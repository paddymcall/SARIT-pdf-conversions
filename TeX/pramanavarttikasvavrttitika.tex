%% require snapshot package to record versions to log files
    \RequirePackage[log]{snapshot}
    \documentclass[article,12pt,a4paper]{memoir}%
    
      %% useful for debugging
      %% \usepackage{syntonly}%
      %%\syntaxonly%
    
	  \usepackage[normalem]{ulem}
	  \usepackage{eulervm}
	  \usepackage{xltxtra}
  \usepackage{polyglossia}
  \PolyglossiaSetup{sanskrit}{
  hyphenmins={2,3},% default is {1,3}
  }
  \setdefaultlanguage{sanskrit}
  % english etc. should also be available, notes and bib
  \setotherlanguages{english,german,italian,french}
  
	\setotherlanguage[numerals=arabic]{tibetan}
      
  \usepackage{fontspec}
  %% redefine some chars (either changed by parsing, or not commonly in font)
  \catcode`⃥=\active \def⃥{\textbackslash}
  \catcode`‿=\active \def‿{\textunderscore}
  \catcode`❴=\active \def❴{\{}
  \catcode`❵=\active \def❵{\}}
  \catcode`〔=\active \def〔{{[}}% translate 〔OPENING TORTOISE SHELL BRACKET
  \catcode`〕=\active \def〕{{]}}% translate 〕CLOSING TORTOISE SHELL BRACKET
  \catcode` =\active \def {\,}
  \catcode`·=\active \def·{\textbullet}
  %% BREAK PERMITTED HERE: \discretionary{-}{}{}\nobreak\hspace{0pt}
  \catcode`‚=\active \def‚{\-}
  \catcode`ꣵ=\active \defꣵ{%
  म्\textsuperscript{cb}%for candrabindu
  }
  %% show a lot of tolerance
  \tolerance=9000
  \def\textJapanese{\fontspec{Kochi Mincho}}
  \def\textChinese{\fontspec{HAN NOM A}}
  \def\textKorean{\fontspec{Baekmuk Gulim} }
  % make sure English font is there
  \newfontfamily\englishfont[Mapping=tex-text]{TeX Gyre Schola}
    % set up a devanagari font
  \newfontfamily\devanagarifont[Script=Devanagari,Mapping=devanagarinumerals,AutoFakeBold=1.5,AutoFakeSlant=0.3]{Chandas}
	\newfontfamily\rmlatinfont[Mapping=tex-text]{TeX Gyre Pagella}
	\newfontfamily\tibetanfont[Script=Tibetan,Scale=1.2]{Tibetan Machine Uni}
  \newcommand\bo\tibetanfont
  
    \defaultfontfeatures{Scale=MatchLowercase,Mapping=tex-text}
	\setmainfont{Chandas}
    \setsansfont{TeX Gyre Bonum}
  
  \setmonofont{DejaVu Sans Mono}
	  %% page layout start: fit to a4 and US letterpaper (example in memoir.pdf)
	  %% page layout start
	  % stocksize (actual size of paper in the printer) is a4 as per class
	  % options;
	  
	  % trimming, i.e., which part should be cut out of the stock (this also
	  % sets \paperheight and \paperwidth):
	  % \settrimmedsize{0.9\stockheight}{0.9\stockwidth}{*}
	  % \settrimmedsize{225mm}{150mm}{*}
	  % % say where you want to trim
	  \setlength{\trimtop}{\stockheight}    % \trimtop = \stockheight
	  \addtolength{\trimtop}{-\paperheight} %           - \paperheight
	  \setlength{\trimedge}{\stockwidth}    % \trimedge = \stockwidth
	  \addtolength{\trimedge}{-\paperwidth} %           - \paperwidth
	  % % this makes trims equal on top and bottom (which means you must cut
	  % % twice). if in doubt, cut on top, so that dust won't settle when book
	  % % is in shelf
	  \settrims{0.5\trimtop}{0.5\trimedge}

	  % figure out which font you're using
	  \setxlvchars
	  \setlxvchars
	  % \typeout{LENGTH: lxvchars: \the\lxvchars}
	  % \typeout{LENGTH: xlvchars: \the\xlvchars}

	  % set the size of the text block next:
	  % this sets \textheight and \textwidth (not the whole page including
	  % headers and footers)
	  \settypeblocksize{230mm}{130mm}{*}

	  % left and right margins:
	  % this way spine and edge margins are the same
	  % \setlrmargins{*}{*}{*}
	  \setlrmargins{*}{*}{1.5}

	  % upper and lower, same logic as before
	  % \setulmargins{*}{*}{*}% upper = lower margin
	  % \uppermargin = \topmargin + \headheight + \headsep
	  %\setulmargins{*}{*}{1.5}% 1.5*upper = lower margin
	  \setulmargins{*}{*}{1.5}% 

	  % header and footer spacings
	  \setheadfoot{2\baselineskip}{2\baselineskip}

	  % \setheaderspaces{ headdrop }{ headsep }{ ratio }
	  \setheaderspaces{*}{*}{1.5}

	  % see memman p. 51 for this solution to widows/orphans 
	  \setlength{\topskip}{1.6\topskip}
	  % fix up layout
	  \checkandfixthelayout
	  %% page layout end
	
	  \sloppybottom
	
	    % numbering depth
	    \maxtocdepth{section}
	    % set up layout of toc
	    \setpnumwidth{4em}
	    \setrmarg{5em}
	    \setsecnumdepth{all}
	    \newenvironment{docImprint}{\vskip 6pt}{\ifvmode\par\fi }
	    \newenvironment{docDate}{}{\ifvmode\par\fi }
	    \newenvironment{docAuthor}{\ifvmode\vskip4pt\fontsize{16pt}{18pt}\selectfont\fi\itshape}{\ifvmode\par\fi }
	    % \newenvironment{docTitle}{\vskip6pt\bfseries\fontsize{18pt}{22pt}\selectfont}{\par }
	    \newcommand{\docTitle}[1]{#1}
	    \newenvironment{titlePart}{ }{ }
	    \newenvironment{byline}{\vskip6pt\itshape\fontsize{16pt}{18pt}\selectfont}{\par }
	    % setup title page; see CTAN /info/latex-samples/TitlePages/, and memoir
	  \newcommand*{\plogo}{\fbox{$\mathcal{SARIT}$}}
	  \newcommand*{\makeCustomTitle}{\begin{english}\begingroup% from example titleTH, T&H Typography
	  \thispagestyle{empty}
	  \raggedleft
	  \vspace*{\baselineskip}
	  
	      % author(s)
	    {\Large Dharmakīrti and Karṇakagomin}\\[0.167\textheight]
	    % maintitle
	    {\Huge Pramāṇavārttikasvavṛttiṭīkā}\\[\baselineskip]
	    {\Large SARIT}\\\vspace*{\baselineskip}\plogo\par
	  \vspace*{3\baselineskip}
	  \endgroup
	  \end{english}}
	  \newcommand{\gap}[1]{}
	  \newcommand{\corr}[1]{($^{x}$#1)}
	  \newcommand{\sic}[1]{($^{!}$#1)}
	  \newcommand{\reg}[1]{#1}
	  \newcommand{\orig}[1]{#1}
	  \newcommand{\abbr}[1]{#1}
	  \newcommand{\expan}[1]{#1}
	  \newcommand{\unclear}[1]{($^{?}$#1)}
	  \newcommand{\add}[1]{($^{+}$#1)}
	  \newcommand{\deletion}[1]{($^{-}$#1)}
	  \newcommand{\quotelemma}[1]{\textcolor{cyan}{#1}}
	  \newcommand{\name}[1]{#1}
	  \newcommand{\persName}[1]{#1}
	  \newcommand{\placeName}[1]{#1}
	  % running latexPackages template
     \usepackage[x11names]{xcolor}
     \definecolor{shadecolor}{gray}{0.95}
     \usepackage{longtable}
     \usepackage{ctable}
     \usepackage{rotating}
     \usepackage{lscape}
     \usepackage{ragged2e}
     
	 \usepackage{titling}
	 \usepackage{marginnote}
	 \renewcommand*{\marginfont}{\color{black}\rmlatinfont\scriptsize}
	 \setlength\marginparwidth{.75in}
	 \usepackage{graphicx}
	 \graphicspath{{images/}}
	 \usepackage{csquotes}
       
	 \def\Gin@extensions{.pdf,.png,.jpg,.mps,.tif}
       
      \usepackage[noend,series={A,B}]{reledmac}
       % simplify what ledmac does with fonts, because it breaks. From the documentation of ledmac:
       % The notes are actually given seven parameters: the page, line, and sub-line num-
       % ber for the start of the lemma; the same three numbers for the end of the lemma;
       % and the font specifier for the lemma. 
       \makeatletter
       \def\select@lemmafont#1|#2|#3|#4|#5|#6|#7|%
       {}
       \makeatother
       \AtEveryPstart{\refstepcounter{parCount}}
       \setlength{\stanzaindentbase}{20pt}
     \setstanzaindents{3,2,2,2,2,2,2,2,2,2,2,2,2,2,2,2,2,2,2,2,2,}
     % \setstanzapenalties{1,5000,10500}
     \lineation{page}
     % \linenummargin{inner}
     \linenumberstyle{arabic}
     \firstlinenum{5}
    \linenumincrement{5}
    \renewcommand*{\numlabfont}{\normalfont\scriptsize\color{black}}
    \addtolength{\skip\Afootins}{1.5mm}
    \Xnotenumfont{\bfseries\footnotesize}
    \sidenotemargin{outer}
    \linenummargin{inner}
    \Xarrangement{twocol}
    \arrangementX{twocol}
    %% biblatex stuff start
	 \usepackage[backend=biber,%
	 citestyle=authoryear,%
	 bibstyle=authoryear,%
	 language=english,%
	 sortlocale=en_US,%
	 ]{biblatex}
	 
		 \addbibresource[location=remote]{https://raw.githubusercontent.com/paddymcall/Stylesheets/HEAD/profiles/sarit/latex/bib/sarit.bib}
	 \renewcommand*{\citesetup}{%
	 \rmlatinfont
	 \biburlsetup
	 \frenchspacing}
	 \renewcommand{\bibfont}{\rmlatinfont}
	 \DeclareFieldFormat{postnote}{:#1}
	 \renewcommand{\postnotedelim}{}
	 %% biblatex stuff end
	 
	 \setcounter{errorcontextlines}{400}
       
	 \usepackage{lscape}
	 \usepackage{minted}
       
	   % pagestyles
	   \pagestyle{ruled}
	   \makeoddhead{ruled}{{Pramāṇavārttikasvavṛttiṭīkā}}{}{          Dharmakīrti and Karṇakagomin}
	   \makeoddfoot{ruled}{{\tiny\rmlatinfont \textit{Compiled: \today}}}{%
	   {\tiny\rmlatinfont \textit{Revision: \href{https://github.com/paddymcall/SARIT-pdf-conversions/commit/a0c8ae0}{a0c8ae0}}}%
	   }{\rmlatinfont\thepage}
	   \makeevenfoot{ruled}{\rmlatinfont\thepage}{%
	   {\tiny\rmlatinfont \textit{Revision: \href{https://github.com/paddymcall/SARIT-pdf-conversions/commit/a0c8ae0}{a0c8ae0}}}%
	   }{{\tiny\rmlatinfont \textit{Compiled: \today}}}
	   
	 
	   \usepackage{perpage}
           \MakePerPage{footnote}
	 
       \usepackage[destlabel=true,% use labels as destination names; ; see dvipdfmx.cfg, option 0x0010, if using xelatex
       pdftitle={Pramāṇavārttikasvavṛttiṭīkā // Dharmakīrti and Karṇakagomin},
       pdfauthor={SARIT: Search and Retrieval of Indic Texts. DFG/NEH Project (NEH-No.
	HG5004113), 2013-2016 },
       unicode=true]{hyperref}
       
       \renewcommand\UrlFont{\rmlatinfont}
       \newcounter{parCount}
       \setcounter{parCount}{0}
       % cleveref should come last; note: also consider zref, this could become more useful than cleveref?
       \usepackage[english]{cleveref}% clashes with eledmac < 1.10.1 standard
       \crefname{parCount}{§}{§§}
     
\begin{document}
    
     \makeCustomTitle
     \let\tabcellsep&
	\frontmatter
	\tableofcontents
	% \listoffigures
	% \listoftables
	\cleardoublepage
         \mainmatter 
	  
	% new div opening: depth here is 0
	
	    
	    \beginnumbering% beginning numbering from div depth=0
	    
	  
\chapter[{प्र‚माण‚व‚र्त्तिक‚स्व‚व्रित्तिटीका}][{प्र‚माण‚व‚र्त्तिक‚स्व‚व्रित्तिटीका}]{प्र‚माण‚व‚र्त्तिक‚स्व‚व्रित्तिटीका}\textsuperscript{\textenglish{1/s}}
	    
	    \stanza[\smallbreak]
	  यो विध्व‚स्त‚स‚म‚स्त‚ब‚न्ध‚न‚ग‚तिस्स‚म्य‚ग् व‚शित्वे स्थितः [,]&\leavevmode\ledsidenote{\textenglish{1b/PSVTa}}‚{\tiny $_{lb}$}‚स‚र्व्व‚ज्ञेय‚विसारिनिर्म्म‚ल‚त‚म‚ज्ञान‚प्र‚ब‚न्धोद‚यः ।&‚{\tiny $_{lb}$}‚स‚त्त्वार्थोद्य‚त‚मान‚स‚श्च सुचिरं श्रीम‚ञ्जुनाथो विभुः [,]&‚{\tiny $_{lb}$}‚त‚न्न‚त्वा ब‚हुशोद्य वार्त्तिक‚ग‚तं किञ्चिद्विव‚क्षाम्य‚हं ॥&‚{\tiny $_{lb}$}‚टीकाकृतः स‚क‚ल एव गुणः स एष ब्रूतेत्र व‚स्तुग‚ह‚नेपि य‚द‚स्म‚दादिः ।&‚{\tiny $_{lb}$}‚\textbf{दिग्नाग}द‚न्त‚मु\textbf{स}\edtext{}{\lemma{मु}\Bfootnote{? श}} लैर्विष‚मेऽप‚नीते ल‚ब्धेषु व‚र्त्म‚सु सुखं क‚ल‚भाः प्र‚यान्ति‚{\tiny $_{१}$}‚ ॥&‚{\tiny $_{lb}$}‚\edtext{\textsuperscript{*}}{\edlabel{pvsvt_1-3}\label{pvsvt_1-3}\lemma{*}\Bfootnote{A parody of Bhavabhūti's Malatimādhavam 1:4. }}यो माम‚व‚ज्ञाय‚ति कोपि गुणाभिमानी जानात्य‚सौ किम‚पि तं प्र‚ति नैष य‚त्नः ।&‚{\tiny $_{lb}$}‚क‚श्चिद् भ‚विष्य‚ति क‚दाचिद‚नेन चार्थी नानाधियाञ्ज‚ग‚ति ज‚न्म‚व‚तां हि नान्तः ॥\&[\smallbreak]
	  
	  
	  ‚{\tiny $_{lb}$}‚

	  
	  \pstart \leavevmode% starting standard par
	य‚द्य‚पि हि शास्त्रार‚म्भे न‚म‚स्कार‚श्लोकोप‚न्यास‚म‚न्त‚रेण काय‚वाङ‚म‚नोभिरिष्ट‚{\tiny $_{lb}$}‚देव‚तान‚म‚स्कार‚क‚र‚णेन पुण्योप‚च‚याद‚विघ्नेन शास्त्र‚स्य प‚रिस‚माप्तिर्भ‚व‚ति ।‚{\tiny $_{lb}$}‚ त‚थापि व्याख्यातृश्रोतॄणां स्तुतिपुर‚स्स‚र‚या प्र‚वृत्त्या पुण्या‚{\tiny $_{२}$}‚तिश‚योत्पादात् पारार्थ्यं‚{\tiny $_{lb}$}‚ स‚दाचारानुपाल‚नं चालोच्य विशिष्ट‚देव‚तापूजाश्लोक‚मुप‚न्य‚स्त‚वानाचार्यः ।‚{\tiny $_{lb}$}‚ \textbf{विधूत‚क‚ल्प‚नेत्या}दि ।
	{\color{gray}{\rmlatinfont\textsuperscript{§~\theparCount}}}
	\pend% ending standard par
      ‚{\tiny $_{lb}$}‚

	  
	  \pstart \leavevmode% starting standard par
	य‚दा \textbf{स‚म‚न्त‚भ‚द्र}श‚ब्दो रूढ्या बोधिस‚त्त्व‚वृत्तो न गृह्य‚ते त‚देयं बुद्ध‚स्य‚{\tiny $_{lb}$}‚ भ‚ग‚व‚तः पूजा । सा च द्विधा स्तोत्र‚तः प्र‚णाम‚त‚श्च । \textbf{न‚मः}श‚ब्देन प्र‚णाम‚तः‚{\tiny $_{lb}$}‚ प‚रिशिष्टैः स्तोत्र‚तः । स्तोत्र‚म‚पि स्वार्थ‚स‚म्प‚त्तितः प‚रार्थ‚स‚म्प‚त्तितः प‚रार्थ‚{\tiny $_{lb}$}‚संप‚{\tiny $_{३}$}‚दुपाय‚त‚श्च त्रिधा । स्वार्थ‚स‚म्प‚न्न‚श्च प‚रार्थंप्र‚ति स‚म‚र्थो भ‚व‚तीवि‚{\tiny $_{lb}$}‚  \leavevmode\ledsidenote{\textenglish{2/s}}प्र‚थ‚मं पूर्व्वार्द्धेन स्वार्थ‚स‚म्प‚दुक्ता । स्वार्थ‚स‚म्प‚च्च काय‚त्र‚य‚ल‚क्ष‚णा त्रिभिर्व्विशे‚{\tiny $_{lb}$}‚षैर्व्विशिष्टोद्भाविता । \textbf{आव‚र‚ण}प्र‚हाण‚विशेषेण । \textbf{गाम्भीर्य}विशेषेण । \textbf{औदार्य}‚{\tiny $_{lb}$}‚विशेषेण च ।
	{\color{gray}{\rmlatinfont\textsuperscript{§~\theparCount}}}
	\pend% ending standard par
      ‚{\tiny $_{lb}$}‚

	  
	  \pstart \leavevmode% starting standard par
	त‚त्रात्मात्मीयाद्याकार‚प्र‚वृत्ता ग्राह्य‚ग्राह‚काकार‚प्र‚वृत्ताश्च त्रैधातुकाश्चित्त‚{\tiny $_{lb}$}‚चैत्ताः \textbf{क‚ल्प‚ना} । सैव \textbf{जाल}म्ब‚न्ध‚नात्म‚क‚त्वात् ।‚{\tiny $_{४}$}‚ त‚द्विधूतं विध्व‚स्तं स‚वास‚नं‚{\tiny $_{lb}$}‚ पुन‚र‚नुत्प‚त्तिध‚र्म‚त्व‚मापादितं यासां \textbf{मूर्त्ती}नान्तास्त‚थोक्ताः । एतेनाव‚र‚ण‚प्र‚हाण‚{\tiny $_{lb}$}‚विशेष उक्तः । गाम्भीर्यौदार्य‚विशेषौ तु ग‚म्भीरोदार‚प‚दाभ्यामुक्तौ । विधूत‚{\tiny $_{lb}$}‚क‚ल्प‚नाजाला \textbf{ग‚म्भीराः} श्राव‚क‚प्र‚त्येक‚बुद्धाद्य‚विष‚य‚त्वात् । \textbf{उदारा}स्स‚क‚ल‚ज्ञेय‚{\tiny $_{lb}$}‚स‚क‚ल‚स‚त्वार्थ‚व्याप‚नात् । \textbf{मूर्त्त‚य}स्त्र‚यः कायाः स्वाभाविक‚साम्भोगिक‚नैर्माणिका‚{\tiny $_{५}$}‚‚{\tiny $_{lb}$}‚ य‚स्य भ‚ग‚व‚तः । असौ विधूत‚क‚ल्प‚नाजाल\textbf{ग‚म्भीरोदार‚मूर्त्तिः} । भ‚द्रं क‚ल्याण‚म‚भ्यु‚{\tiny $_{lb}$}‚द‚य‚निःश्रेय‚स‚ल‚क्ष‚णं । त‚त् स‚म‚न्तान्निर‚व‚शेष‚न्त‚द‚र्थिनां य‚था भ‚व्य‚म्भ‚व‚ति य‚स्मात्‚{\tiny $_{lb}$}‚ स‚काशा\add{द‚सौ \textbf{स‚म‚न्त‚भ‚द्रः} । एतेन प‚रार्थ‚स‚म्प‚दुक्ता । अस्याश्च प‚रार्थ‚स‚म्प‚दः‚{\tiny $_{lb}$}‚ स‚म‚न्त‚स्फुर‚ण‚त्विष}\edtext{}{\edlabel{pvsvt_2-1}\label{pvsvt_2-1}\lemma{काशा}\Bfootnote{Missing portion is found in the margin in a different hand. }} इत्य‚नेन सिद्ध्युपाय उक्तः । \add{त्विषो र‚श्म‚य इह तु य‚था‚{\tiny $_{lb}$}‚व‚स्थित‚व‚स्तुग्र‚ह‚ण}\edtext{\textsuperscript{*}}{\edlabel{pvsvt_2-1b}\label{pvsvt_2-1b}\lemma{*}\Bfootnote{Missing portion is found in the margin in a different hand. }} ‚{\tiny $_{६}$}‚एवं साध‚र्म्येण ध‚र्म‚देश‚ना । अभ्युद‚य‚निःश्रेय‚स‚साध‚न‚{\tiny $_{lb}$}‚विष‚यास्त्विष इव त्विष उच्य‚न्ते । स‚म‚न्तो निर‚व‚शेषः प‚रार्थ‚साध‚नोपायः स्फ‚र्य‚ते‚{\tiny $_{lb}$}‚ व्याप्य‚ते विन‚येभ्यः साक‚ल्येन \add{क‚थ‚नात् याभिस्ताः स‚म‚न्त‚स्फ‚र‚ण्य‚स्त‚त्र‚{\tiny $_{lb}$}‚त‚थाभूतास्त्विषो ध‚र्म‚देश‚ना य‚स्य स स‚म‚न्त‚स्फ‚र‚ण‚त्विट् । न‚मः श‚ब्द‚योगाच्च‚{\tiny $_{lb}$}‚ स‚र्व्व‚त्र च‚तुर्थी ।‚{\tiny $_{lb}$}‚ \leavevmode\ledsidenote{\textenglish{2a/PSVTa}} य‚दा तु रूढिर‚पेक्ष्य‚ते त‚दायं स‚म‚न्त‚भ‚द्र‚श}\edtext{\textsuperscript{*}}{\edlabel{pvsvt_2-1c}\label{pvsvt_2-1c}\lemma{*}\Bfootnote{Missing portion is found in the margin in a different hand. }} ‚{\tiny $_{७}$}‚ब्दो म‚हायाने बोधिस‚त्व‚विशेषे‚{\tiny $_{lb}$}‚ रूढ इति बोधिस‚त्व‚स्येयं पूजा [।] प‚दार्थ‚स्तु पूर्व्व‚व‚द् योज्यः । अय‚न्तु विशेषो‚{\tiny $_{lb}$}‚ विधूत‚क‚ल्प‚नाजाल‚त्वं बोधिस‚त्व‚भूम्याव‚र‚ण‚प्र‚हाण‚तो द्र‚ष्ट‚व्यं । गाम्भीर्यं श्राव‚क‚{\tiny $_{lb}$}‚\add{प्र‚त्येक‚बुद्ध‚पृथ‚ग्ज‚नाविष‚य‚त्व‚तः । औदार्य‚न्तु बोधिस‚त्व‚मा हात्म\edtext{}{\lemma{हात्म}\Bfootnote{? त्म्य}}तः ।‚{\tiny $_{lb}$}‚ काय‚त्र‚य‚म‚प्य‚नुरूपं बोधिस‚त्वानां विद्य‚त ए}\edtext{}{\edlabel{pvsvt_2-1d}\label{pvsvt_2-1d}\lemma{क}\Bfootnote{Missing portion is found in the margin in a different hand. }}व प्र‚क‚र्ष‚ग‚म‚नात्तु बुद्धानां व्य‚व‚स्थाप्य‚त‚{\tiny $_{lb}$}‚ इति ॥
	{\color{gray}{\rmlatinfont\textsuperscript{§~\theparCount}}}
	\pend% ending standard par
      ‚{\tiny $_{lb}$}‚

	  
	  \pstart \leavevmode% starting standard par
	स‚न्त्येव हि स‚न्तोस्य‚{\tiny $_{१}$}‚ वा र्त्ति का ख्य‚स्य शास्त्र‚स्य ग्र‚हीतार‚स्त‚थापि श्रोतृदोष‚{\tiny $_{lb}$}‚बाह‚ल्येन स‚न्त‚म‚प्युप‚कार‚म‚स‚न्त‚मिव कृत्वा सूक्ताभ्यास‚भावित‚चित्त‚त्त्व‚मेव शास्त्रार‚म्भे‚{\tiny $_{lb}$}‚ कार‚ण‚न्द‚र्श‚य‚न् । अयं च म‚हार्थ‚भ्र‚ङ्शे हेतुदोष‚स्त्य‚क्तुं युक्त इत्येत‚च्च व‚क्रोक्त्या‚{\tiny $_{lb}$}‚ क‚थ‚यितुं द्वितीयं श्लोक‚माह ।
	{\color{gray}{\rmlatinfont\textsuperscript{§~\theparCount}}}
	\pend% ending standard par
      ‚{\tiny $_{lb}$}‚‚{\tiny $_{lb}$}‚‚{\tiny $_{lb}$}‚\textsuperscript{\textenglish{3/s}}

	  
	  \pstart \leavevmode% starting standard par
	\textbf{प्रायःप्राकृते}त्यादि । अत्र च‚तुर्व्विधः श्रोतृदोष उद्भावितः । कुप्र‚ज्ञ‚त्व‚म‚ज्ञ‚त्वं‚{\tiny $_{lb}$}‚ अन‚र्थित्वं अमाध्य‚स्थ्य‚ञ्च । \textbf{प्रायः} श‚ब्द ओकारान्तो‚{\tiny $_{२}$}‚ बाहुल्य‚व‚च‚नः । प्रायो ज‚नो‚{\tiny $_{lb}$}‚ भूयान् ज‚नः । \textbf{प्राकृत‚स‚क्तिः प्राकृता}नि ब‚हिःशास्त्राणि त‚त्र \textbf{स‚क्ति}र्य‚स्येति [।]‚{\tiny $_{lb}$}‚ ग‚म‚क‚त्वाद् व्य‚धिक‚र‚णो ब‚हुब्रीहिः । प्राकृता वा स‚क्तिर्य‚स्येति । स‚मानाधिक‚र‚ण एव ।‚{\tiny $_{lb}$}‚ प्राकृत‚विष‚य‚त्वाच्च स‚क्तिः प्राकृता । अनेन कुप्र‚ज्ञ‚त्वं श्रोतृदोष उक्तः ।
	{\color{gray}{\rmlatinfont\textsuperscript{§~\theparCount}}}
	\pend% ending standard par
      ‚{\tiny $_{lb}$}‚

	  
	  \pstart \leavevmode% starting standard par
	अप्र‚तिब‚ला शास्त्र‚ग्र‚ह‚ण‚म्प्र‚त्य‚सशक्ता प्र‚ज्ञा य‚स्य सो\textbf{प्र‚तिब‚ल‚प्र‚ज्ञः प्रायो ज‚न}‚{\tiny $_{lb}$}‚ इति स‚म्ब‚न्धः । अनेनाज्ञ‚त्व‚मुक्तं‚{\tiny $_{३}$}‚ [।] \textbf{सुभाषितैर्नान‚र्थ्येव} किन्तु सुभाषिताभिधायिन‚{\tiny $_{lb}$}‚\textbf{म्विद्वेष्ट्य‚पीर्ष्याम‚लैः प‚रिग‚तः} स‚न् । अन‚र्थी च विद्वेष्टि चेत्य‚र्थः । एतेन य‚था‚{\tiny $_{lb}$}‚क्र‚म‚म‚न‚र्थित्व‚म‚माध्य‚स्थ्यं चोक्तं । अत्रापि प्रायो ज‚न इति स‚म्ब‚न्ध‚नीयं ।
	{\color{gray}{\rmlatinfont\textsuperscript{§~\theparCount}}}
	\pend% ending standard par
      ‚{\tiny $_{lb}$}‚

	  
	  \pstart \leavevmode% starting standard par
	अन्ये तु प्राय‚श्श‚ब्द‚स्या\edtext{}{\lemma{स्या}\Bfootnote{? स}}कारान्तोप्य‚स्ति निपातः [।] स च बाहुल्ये‚{\tiny $_{lb}$}‚नेत्य‚स्मिंस्तृतीयार्थे स्व‚भावाद्व‚र्त्त‚त इति व्याच‚क्ष‚ते । \textbf{ईर्ष्या} प‚र‚स‚म्प‚त्तौ चेत‚सो‚{\tiny $_{lb}$}‚ व्यारोषः । सैव म‚ल‚श्चित्त‚म‚लि‚{\tiny $_{४}$}‚नीक‚र‚णात् । व्य‚क्तिभेदाद् ब‚हुव‚च‚नं । य‚त एव‚न्तेन‚{\tiny $_{lb}$}‚ कार‚णेनाय‚मारिप्सितो वा र्ति का ख्यो ग्र‚न्थः । \textbf{प‚रोप‚कारः} प‚रेषामुप‚कारः । उप‚{\tiny $_{lb}$}‚क्रिय‚तेनेनेति क‚र‚णे घ‚ञ् । प‚रान् वोप‚क‚रोतीति प‚रोप‚कारः क‚र्म‚ण्य‚ण्‚{\tiny $_{lb}$}‚ \href{http://sarit.indology.info/?cref=P\%C4\%81.3.2.1}{पाणिनिः ३।२।१} । प‚रोप‚कार \textbf{इति नः अस्माकं चिन्तापि} नास्ति । क‚थ‚न्त‚र्हि‚{\tiny $_{lb}$}‚ शास्त्र‚र‚च‚नायां प्र‚वृत्तिरित्याह । \textbf{चेत‚श्चिर}मित्यादि । चिर‚न्दीर्घ‚कालं \textbf{सूक्ताभ्यासेन‚{\tiny $_{lb}$}‚ विव‚र्द्धितं व्य}स‚नं‚{\tiny $_{५}$}‚ स‚क्तिस्त‚त्प‚र‚ता सूक्ताभ्यास‚विव‚र्द्धितं व्य‚स‚नं य‚स्य चेत‚स‚स्त‚{\tiny $_{lb}$}‚ त्त‚थोक्तं । इति हेतोर‚त्र वा र्त्ति क र‚च‚नाया\textbf{म‚नुब‚द्ध‚स्पृहं} जाताभिलाषं चेत इति ।‚{\tiny $_{lb}$}‚ एव‚मेके व्याच‚क्ष‚ते ।
	{\color{gray}{\rmlatinfont\textsuperscript{§~\theparCount}}}
	\pend% ending standard par
      ‚{\tiny $_{lb}$}‚

	  
	  \pstart \leavevmode% starting standard par
	अन्ये त्व‚न्य‚था । क‚स्माद‚य‚माचार्य ध र्म्म की र्त्ति र्वार्तिक‚न्यायेन \textbf{प्र‚माण‚स‚मुच्च}य‚{\tiny $_{lb}$}‚व्याख्यां क‚रोति न पुनः स्व‚त‚न्त्र‚मेव शास्त्र‚मित्य‚स्मिन् प्र‚श्नाव‚स‚रे प्राह । \textbf{प्राय}‚{\tiny $_{lb}$}‚ इत्यादि । अस्य श्लोक‚स्यायं स‚मासार्थः । \textbf{चिन्त}या क‚रुण‚या च‚{\tiny $_{६}$}‚ मे प्र मा ण स मु‚{\tiny $_{lb}$}‚च्च य व्याख्यायां \textbf{चेतो} जाताभिलाष‚मिति । चिन्ता क‚रुणा च आचार्य दि ग्ना ग‚{\tiny $_{lb}$}‚र‚चित‚शास्त्र‚स्याल्पोप‚कारित्वेन । अल्पोप‚कारित्व‚ञ्च श्रातृज‚नाप‚राधेन । प‚दार्थ‚{\tiny $_{lb}$}‚स्तूच्य‚ते । \textbf{प्राय} इति बाहुल्येन \textbf{प्राकृत‚स‚क्तिर्ज‚न} इति स‚म्ब‚न्धः । प्राकृत उच्य‚ते‚{\tiny $_{lb}$}‚ ‚{\tiny $_{lb}$}‚ \leavevmode\ledsidenote{\textenglish{4/s}}लोके नीचः । य‚स्य दुष्टोन्व‚यः । एव‚न्ती र्थि क शास्त्राणि प‚र‚प्र‚णीतानि चाचार्य‚नीति‚{\tiny $_{lb}$}‚\leavevmode\ledsidenote{\textenglish{2b/PSVTa}} शास्त्र‚दूष‚णानि । विप‚र्य‚स्त‚ज्ञान‚प्र‚भ‚व‚{\tiny $_{७}$}‚त्वाद् दुष्टान्व‚याद् य‚तः प्राकृतानि । तेषु‚{\tiny $_{lb}$}‚ स‚क्तिर‚नुरागो य‚स्य स त‚थोक्तः । क‚स्मात्पुनः प्राकृत‚स‚क्तिरित्याह । \textbf{अप्र‚तिब‚ल}‚{\tiny $_{lb}$}‚प्र‚ज्ञ इति । अतोसौ दुर्भाषित‚म‚पि सुभाषित‚मिति गृहीत्वा प्राकृते स‚ज्य‚ते । अप्र‚ति‚{\tiny $_{lb}$}‚ब‚ल‚प्र‚ज्ञ‚त्वादेव चाचार्य‚सुभाषितानि स्व‚यं य‚थाव‚द‚व‚बोद्धुम‚क्ष‚मो दोष‚व‚त्त्वेन गृहीत्वा‚{\tiny $_{lb}$}‚ तैराचार्य‚सुभाषितै\textbf{र‚न‚र्थी} । आचार्ये च विद्वेष‚वान् भ‚व‚तीत्याह । \textbf{केव‚ल}मित्या‚{\tiny $_{१}$}‚दि ।‚{\tiny $_{lb}$}‚ न केव‚ल‚म‚न‚र्थी \textbf{सुभाषितै}राचार्यीयै\textbf{र‚पि} तु \textbf{विद्वेष्ट्य‚पीर्ष्याम‚लैः प‚रिग‚तः} स‚न्नाचार्य‚{\tiny $_{lb}$}‚दि ग्ना गं [।] किं भूतं \textbf{सूक्ताभ्यास‚विव‚र्द्धित‚व्य‚स‚नं} । व्य‚व‚हितेनापि स‚म्ब‚न्धो भ‚व‚त्येव ।
	{\color{gray}{\rmlatinfont\textsuperscript{§~\theparCount}}}
	\pend% ending standard par
      ‚{\tiny $_{lb}$}‚

	  
	  \pstart \leavevmode% starting standard par
	\hphantom{.}येन य‚स्याभिस‚म्ब‚न्धो दूर‚स्थ‚स्यापि तेन स इति न्यायात् ।
	{\color{gray}{\rmlatinfont\textsuperscript{§~\theparCount}}}
	\pend% ending standard par
      ‚{\tiny $_{lb}$}‚

	  
	  \pstart \leavevmode% starting standard par
	शोभ‚न‚मुक्तं \textbf{सूक्तं} भ‚ग‚व‚त्प्र‚व‚च‚न‚न्त‚त्रा\textbf{भ्यास}स्त‚त्र \textbf{विव‚र्द्धितं व्य‚स}न‚न्त‚त्रैवात्य‚र्थ‚मा‚{\tiny $_{lb}$}‚स‚क्त‚त्वं य‚स्याचार्य‚दिग्नाग‚स्य स त‚थोक्तः । अनेनाचार्य‚दिग्नाग‚स्यो‚{\tiny $_{२}$}‚प‚चित‚पुण्य‚ज्ञान‚{\tiny $_{lb}$}‚त्व‚माह । उप‚चित‚पुण्य‚ज्ञाना एव हि सूक्ताभ्यास‚विव‚र्द्धित‚व्य‚स‚ना भ‚व‚न्ति । येना‚{\tiny $_{lb}$}‚\textbf{ऽप्र‚तिब‚ल‚प्र‚ज्ञ} आचार्य‚सुभाषितैर‚न‚र्थी प्राकृत‚स‚क्तिश्च \textbf{तेन} कार‚णेनायं प्र मा ण स मु‚{\tiny $_{lb}$}‚च्च यो न \textbf{प‚रोप‚कारः} । उप‚क‚र‚ण‚मुप‚कारो भावे घ‚ञ् \href{http://sarit.indology.info/?cref=P\%C4\%81.3.3.18}{पाणिनिः ३।३।१८} । प‚र‚{\tiny $_{lb}$}‚ उत्कृष्ट उप‚कारो नास्माद् भ‚व‚तीति कृत्वा न प‚रोप‚कारोऽल्प‚स्तूप‚कारोस्त्येव स च‚{\tiny $_{lb}$}‚ प्राय‚श‚ब्देन सूचित एव । \textbf{इति} श‚ब्दो हेतौ‚{\tiny $_{३}$}‚ [।] अस्माद्धेतोर‚स्माकं \textbf{चिन्ता} । म‚हार्थ‚{\tiny $_{lb}$}‚म‚पीदं शास्त्रं न ब‚हूनामुप‚कार‚कं जात‚न्त‚त्क‚थ‚म‚स्यात्य‚र्थं साफ‚ल्यं कुर्यामित्येव‚मा‚{\tiny $_{lb}$}‚कारा । आचार्ये च बोधिस‚त्व‚क‚ल्पे विद्वेषः स्व‚ल्पोप्य‚न‚र्थ‚हेतुर‚तोह‚माचार्य‚नीतेर‚विप‚{\tiny $_{lb}$}‚रीत‚प्र‚काश‚नेनाचार्ये ब‚हुमान‚मुत्पाद्य त‚तोन‚र्थ‚हेतोर्ज‚न‚न्निव‚र्त्त‚यिष्यामीत्येवं दुःख‚वि‚{\tiny $_{lb}$}‚योगेच्छाकारा क‚रुणाप्य‚पिश‚ब्दात् । इत्य\textbf{त्रानुब‚द्ध‚स्पृह‚मि}ति द्वि‚{\tiny $_{४}$}‚तीयेनेति श‚ब्देन‚{\tiny $_{lb}$}‚ चिन्ताक‚रुण‚योर्हेतुत्व‚माह । इत्याभ्यां चिन्ताक‚रुणाभ्यां \textbf{चेत‚श्चिरं} दीर्घ्र‚काल\textbf{म‚त्र}‚{\tiny $_{lb}$}‚ प्र मा ण स मु च्च य व्याख्याभूत प्र मा ण वा र्त्ति क र‚च‚नाया\textbf{म‚नुब‚द्ध‚स्पृहं} स‚न्तानेन‚{\tiny $_{lb}$}‚ प्र‚वृत्तेच्छ‚मिति ॥
	{\color{gray}{\rmlatinfont\textsuperscript{§~\theparCount}}}
	\pend% ending standard par
      ‚{\tiny $_{lb}$}‚

	  
	  \pstart \leavevmode% starting standard par
	य‚दि प्र मा ण स मु च्च य व्याख्यां चिकीर्षुराचार्य‚ध‚र्म‚कीर्त्तिः क‚स्मात् स्वात‚{\tiny $_{lb}$}‚न्त्र्येणानुमानं व्य‚व‚स्थाप‚य‚तीत्याश‚ङ्काम‚प‚न‚य‚न्नाह । \textbf{अर्थान‚र्थेत्}यादि । अर्थो हित‚{\tiny $_{lb}$}‚‚{\tiny $_{lb}$}‚ \leavevmode\ledsidenote{\textenglish{5/s}}म‚हित‚म‚न‚र्थ‚स्त‚{\tiny $_{५}$}‚यो\textbf{र्विवेच‚न}न्त\textbf{स्यानुमानाश्र‚य‚त्वा}द‚नुमान‚माश्र‚यो य‚स्येति विग्र‚हः ।‚{\tiny $_{lb}$}‚ अनुमानेन ह्य‚र्थान‚र्थौ निश्चित्यानुमान‚पृष्ठ‚भाविना प्र‚ब‚न्ध‚प्र‚वृत्तेन ज्ञानेनार्थान‚र्थौ‚{\tiny $_{lb}$}‚ य‚थाक्र‚मं प्राप्ति \add{प‚रिहारार्थ‚म्विभागेन व्य‚व‚स्थाप‚य‚ति । त‚स्माद‚नुमानाश्र‚य‚{\tiny $_{lb}$}‚म‚र्थान‚र्थ‚विवेच‚नं । \textbf{त‚द्विप्र‚तिप‚त्ते}रिति त‚स्मिन्न}\edtext{}{\edlabel{pvsvt_5-1}\label{pvsvt_5-1}\lemma{प्राप्ति}\Bfootnote{In the margin. }}नुमाने विप्र‚तिप‚त्ते\textbf{स्त‚द्व्य‚व‚स्था‚{\tiny $_{lb}$}‚प‚नाय‚हि}त्येव‚मेके व्याच‚क्ष‚ते ।
	{\color{gray}{\rmlatinfont\textsuperscript{§~\theparCount}}}
	\pend% ending standard par
      ‚{\tiny $_{lb}$}‚

	  
	  \pstart \leavevmode% starting standard par
	अत्र त्विदं चिन्त्यं ।
	{\color{gray}{\rmlatinfont\textsuperscript{§~\theparCount}}}
	\pend% ending standard par
      ‚{\tiny $_{lb}$}‚

	  
	  \pstart \leavevmode% starting standard par
	य‚दि ताव‚देव‚म‚व‚धार्य‚ते\textbf{र्थाऽन‚र्थ‚विवे}च‚न‚स्यैवानुमानाश्र‚य‚त्वादिति [।] त‚न्न ।‚{\tiny $_{lb}$}‚ अर्थान‚र्थाभ्याम‚न्य‚स्याप्युपेक्ष‚णीय‚स्य तृतीय‚स्य विष‚य‚स्य य‚द्विवेच‚न‚न्त‚स्याप्य‚नुमाना‚{\tiny $_{lb}$}‚श्र‚य‚त्वात् \add{अर्थान‚र्थ‚विवेच‚न‚स्य चानिय‚त‚त्वात् । प्र‚त्य‚क्षाश्र‚य‚त्व‚म‚निवारित‚मिति ‚{\tiny $_{lb}$}‚कोतिश‚योनुमान‚स्य ख्यापितो येन त‚दादौ व्युत्पाद्य‚ते ।}\edtext{}{\edlabel{pvsvt_5-1b}\label{pvsvt_5-1b}\lemma{त्वात्}\Bfootnote{In the margin. }} अथाप्येव‚म‚व‚धार्य‚ते । ‚{\tiny $_{lb}$}‚ अर्थान‚र्थ‚विवेच‚न‚स्यानुमानाश्र‚{\tiny $_{७}$}‚य‚त्वादेवेति । त‚थाप्य‚युक्त‚म‚व‚धार‚णं । अर्थान‚र्थ‚विवे‚{\tiny $_{lb}$}‚च‚न‚स्य प्र‚त्य‚क्षाश्र‚य‚त्वाद‚पि । त‚था ह्य‚र्थान‚र्थौ विभ‚क्त‚रूपावेव प्र‚त्य‚क्षे प्र‚तिभासेते । \leavevmode\ledsidenote{\textenglish{3a/PSVTa}}‚{\tiny $_{lb}$}‚ त‚च्च प्र‚त्य‚क्ष‚म‚भ्यासातिश‚य‚स‚मासादित‚पाट‚व‚त‚या अप‚सारित‚भ्रान्तिनिमित्तं पाश्चा‚{\tiny $_{lb}$}‚त्य‚म‚र्थान‚र्थ‚विवेच‚न\add{विक‚ल्पं ज‚न‚य‚ति । एत‚देव हि प्र‚त्य‚क्ष‚स्यार्थान‚र्थ‚विवे}\edtext{}{\edlabel{pvsvt_5-1c}\label{pvsvt_5-1c}\lemma{न}\Bfootnote{In the margin. }}‚{\tiny $_{lb}$}‚ च‚नाश्र‚य‚त्वं य‚द‚र्थान‚र्थौ विभागेनानुभूय य‚थानुभ‚व‚न्त‚त्र निश्च‚{\tiny $_{१}$}‚य‚ज‚न‚नं ।
	{\color{gray}{\rmlatinfont\textsuperscript{§~\theparCount}}}
	\pend% ending standard par
      ‚{\tiny $_{lb}$}‚

	  
	  \pstart \leavevmode% starting standard par
	न च श‚क्य‚म्व‚क्तुं पाश्चात्येनैवार्थान‚र्थौ विभ‚क्ताविति । विक‚ल्पेन व‚स्तु‚{\tiny $_{lb}$}‚स्व‚रूप‚स्याग्र‚ह‚णाद् ग्र‚ह‚णे वा विक‚ल्प‚क‚त्व‚हानेर्विक‚ल्पानुब‚द्ध‚स्य प्र‚मातुः स्प‚ष्टार्थ‚{\tiny $_{lb}$}‚प्र‚तिभासित्व‚विरोधात् ।
	{\color{gray}{\rmlatinfont\textsuperscript{§~\theparCount}}}
	\pend% ending standard par
      ‚{\tiny $_{lb}$}‚

	  
	  \pstart \leavevmode% starting standard par
	य‚त्र तु क्व‚चिद् विष‚ये पाट‚वाभावाद् भ्रान्तिनिमित्ताप‚न‚य‚नास‚म‚र्थं प्र‚त्य‚क्षं‚{\tiny $_{lb}$}‚ त‚त्रानुमानान्निश्च‚यः प्रार्थ्य‚ते न स‚र्व‚त्र [।] त‚स्मात् प्र‚त्य‚क्षे स्व‚तः प‚र‚त‚श्च‚{\tiny $_{lb}$}‚ प्रामाण्य‚निश्च‚यः । निश्चाय‚यिष्य‚ते चाय‚म‚र्थो द्वितीय‚{\tiny $_{२}$}‚ प‚रिच्छेद इति नेह‚{\tiny $_{lb}$}‚ प्र‚त‚न्य‚ते ।
	{\color{gray}{\rmlatinfont\textsuperscript{§~\theparCount}}}
	\pend% ending standard par
      ‚{\tiny $_{lb}$}‚

	  
	  \pstart \leavevmode% starting standard par
	अव‚श्यं च प्र‚त्य‚क्ष‚स्याभ्यास‚ब‚लाद‚प‚सारित‚भ्रान्तिनिमित्त‚स्य य‚थानुभ‚व‚न्नि‚{\tiny $_{lb}$}‚श्च‚य‚ज‚न‚नात् स्व‚तोर्थान‚र्थ‚विवेच‚नाश्र‚य‚त्व‚मेष्ट‚व्य‚म‚न्य‚थानुमान‚स्यापि व्य‚व‚स्था न‚{\tiny $_{lb}$}‚ स्याद् धूमादेर्लिङ्ग‚स्यानिश्च‚यात् । धूमादेर‚प्य‚नुमानात् प्र‚तिप‚त्ताव‚न‚व‚स्था स्यात् ।‚{\tiny $_{lb}$}‚ त‚त्रापि लिङ्गान्त‚र‚स्यानुमानान्त‚रेण निश्च‚यादिति ।
	{\color{gray}{\rmlatinfont\textsuperscript{§~\theparCount}}}
	\pend% ending standard par
      ‚{\tiny $_{lb}$}‚

	  
	  \pstart \leavevmode% starting standard par
	य‚दि च प्र‚त्य‚क्ष‚म‚र्थान‚र्थ‚विवेच‚न‚स्यानाश्र‚य‚{\tiny $_{३}$}‚स्त‚दा शास्त्र‚कारेण प्र‚क‚र‚णान्त‚रे‚{\tiny $_{lb}$}‚ य‚दुक्तं [।] हिताहित‚प्राप्तिप‚रिहार‚योर्निय‚मेन स‚म्य‚ग्ज्ञान‚पूर्व‚क‚त्वादित्यादि‚{\tiny $_{lb}$}‚ त‚द् बाध्येत । त‚था न ह्य‚स्याम‚र्थ‚म्प‚रिच्छिद्येत्यादि । पुन‚श्चोक्तं‚{\tiny $_{lb}$}‚ ‚{\tiny $_{lb}$}‚ ‚{\tiny $_{lb}$}‚ \leavevmode\ledsidenote{\textenglish{6/s}}दृष्टेषु स‚म्वित्साम‚र्थ्य‚भाविनं स्म‚र‚णादि त्यादि ।
	{\color{gray}{\rmlatinfont\textsuperscript{§~\theparCount}}}
	\pend% ending standard par
      ‚{\tiny $_{lb}$}‚

	  
	  \pstart \leavevmode% starting standard par
	त‚स्माद‚न्य‚था व्याख्याय‚त इत्य‚प‚रे । आचार्य दि ग्ना ग प्र‚णीतं प्र‚माण‚ल‚क्ष‚णा‚{\tiny $_{lb}$}‚दिक‚म‚र्थो युक्त‚त्वात् । ती र्थि क प्र‚णीतं न युक्त‚त्वाद‚न‚र्थ‚स्त‚{\tiny $_{४}$}‚योर्विवेच‚नं युक्ता‚{\tiny $_{lb}$}‚युक्त‚त्वेन व्य‚व‚स्थाप‚न‚न्त‚स्यानुमानाश्र‚य‚त्वात् । अनुमान‚मेव ह्याश्रित्य ल‚क्ष‚ण‚वा‚{\tiny $_{lb}$}‚क्यानां युक्तायुक्त‚त्वं व्य‚व‚स्थाप्य‚न्नं प्र‚त्य‚क्ष‚न्त‚स्याविचार‚क‚त्वादिति ।
	{\color{gray}{\rmlatinfont\textsuperscript{§~\theparCount}}}
	\pend% ending standard par
      ‚{\tiny $_{lb}$}‚

	  
	  \pstart \leavevmode% starting standard par
	त‚द‚प्य‚युक्तं । य‚तो ल‚क्ष‚ण‚वाक्यानां न स्व‚रूपेण युक्तायुक्त‚त्व‚म‚पि त्व‚र्थ‚द्वा‚{\tiny $_{lb}$}‚रेण [।] स चार्थो य‚थानुमानेन युक्तः प्र‚तीय‚ते त‚था प्र‚त्य‚क्षेणापि । त‚था च‚{\tiny $_{lb}$}‚ व‚क्ष्य‚ति [।]
	{\color{gray}{\rmlatinfont\textsuperscript{§~\theparCount}}}
	\pend% ending standard par
      ‚{\tiny $_{lb}$}‚

	  
	  \pstart \leavevmode% starting standard par
	\hphantom{.}प्र‚त्य‚क्षं क‚ल्प‚नापोढं प्र‚त्य‚क्षेणैव सिध्य‚ति । \href{http://sarit.indology.info/?cref=pv.2.123}{प्र० वा० ३ । १२३}‚{\tiny $_{lb}$}‚ त‚था
	{\color{gray}{\rmlatinfont\textsuperscript{§~\theparCount}}}
	\pend% ending standard par
      ‚{\tiny $_{lb}$}‚
	  \bigskip
	  \begingroup
	
	    
	    \stanza[\smallbreak]
	  {\normalfontlatin\large ``\qquad}प‚क्ष‚ध‚र्म‚त्व‚निश्च‚यः प्र‚त्य‚क्ष‚त इत्यादि ।{\normalfontlatin\large\qquad{}"}\&[\smallbreak]
	  
	  
	  
	  \endgroup
	‚{\tiny $_{lb}$}‚

	  
	  \pstart \leavevmode% starting standard par
	योपि म‚न्य‚ते [।] स‚त्य‚म‚र्थान‚र्थ‚विवेच‚नं प्र‚त्य‚क्षानुमानाभ्यां क्रिय‚त एव‚{\tiny $_{lb}$}‚ [।] केव‚लं य‚द‚र्थान‚र्थ‚विवेच‚न‚स्यानुमानाश्र‚य‚त्व‚मुच्य‚ते त‚त्प्र‚त्य‚क्ष‚विष‚येपि विवाद‚{\tiny $_{lb}$}‚स‚म्भ‚वे स‚ति नानुमानाद‚न्य‚न्निर्ण्ण‚य‚निब‚न्ध‚न‚म‚स्त्य‚तोनुमान‚स्य प्राधान्यात्त‚द्विवेच‚ना‚{\tiny $_{lb}$}‚श्र‚य‚त्व‚मुक्त‚मिति ।
	{\color{gray}{\rmlatinfont\textsuperscript{§~\theparCount}}}
	\pend% ending standard par
      ‚{\tiny $_{lb}$}‚

	  
	  \pstart \leavevmode% starting standard par
	एत‚द‚प्य‚युक्तं । य‚तः प्र‚त्य‚क्ष‚स्य स एव विष‚यो व्य‚व‚स्थाप्य‚ते यो निश्चितो‚{\tiny $_{६}$}‚‚{\tiny $_{lb}$}‚ न च निश्चिते विवादः स‚म्भ‚व‚तीत्य‚युक्त‚मेत‚त् ।
	{\color{gray}{\rmlatinfont\textsuperscript{§~\theparCount}}}
	\pend% ending standard par
      ‚{\tiny $_{lb}$}‚

	  
	  \pstart \leavevmode% starting standard par
	अन्य‚स्त्वाह [।] अर्थान‚र्थ‚विवेच‚न‚म‚नुमानादेव भ‚व‚ति न प्र‚त्य‚क्षात् । य‚तो‚{\tiny $_{lb}$}‚ येर्थान‚र्था अनुभूत‚फ‚ला अनुभूय‚मान‚फ‚ला वा न ते प्र‚वृत्तिविष‚या निष्प‚न्न‚त्वात् फ‚ल‚स्य ।‚{\tiny $_{lb}$}‚ त‚स्माद‚नाग‚तार्थ‚क्रियार्थ‚न्त‚त्स‚म‚र्थेष्व‚र्थान‚र्थेषु प्र‚वृत्तिः । न च त‚त्साम‚र्थ्य‚न्तेषु प्र‚त्य‚{\tiny $_{lb}$}‚\leavevmode\ledsidenote{\textenglish{3b/PSVTa}} क्षेण प्र‚तीय‚ते येन प्र‚वृत्तिविष‚य‚त्वं स्यात् । प्र‚वृत्तिसाध्यार्थ‚क्रियाया भा‚{\tiny $_{७}$}‚वित्वेन‚{\tiny $_{lb}$}‚ त‚त्साम‚र्थ्यकुर्व‚द्रूप‚तास्यापि भावित्वात् । त‚स्मात् पूर्वानुभूतार्थ‚क्रियासाध‚न‚{\tiny $_{lb}$}‚व‚स्तुसाध‚र्म्यात् प्र‚त्य‚क्षेष्व‚पि व‚स्तुष्व‚नाग‚त‚फ‚ल‚योग्य‚तानिश्च‚यो न प्र‚त्य‚क्ष‚त‚{\tiny $_{lb}$}‚स्तेनानुमानादेवार्थान‚र्थ‚विवेच‚न‚मिति ।
	{\color{gray}{\rmlatinfont\textsuperscript{§~\theparCount}}}
	\pend% ending standard par
      ‚{\tiny $_{lb}$}‚

	  
	  \pstart \leavevmode% starting standard par
	त‚द‚प्य‚युक्तं । य‚तो य‚दि सा योग्य‚ता कुर्व‚द्रूप‚ता र्थेषु व‚र्त्त‚मान‚काल‚भाविनी‚{\tiny $_{lb}$}‚ त‚दाभ्यासातिश‚य‚व‚तापि प्र‚त्य‚क्षेण निश्चीयेत । लिङ्ग‚व‚त् । अथानाग‚तैव सा‚{\tiny $_{lb}$}‚ त‚दानुमानेनापिननिश्चीयेतानाग‚तेर्थे‚{\tiny $_{१}$}‚नुमानाभावादिति व‚क्ष्य‚ति ।
	{\color{gray}{\rmlatinfont\textsuperscript{§~\theparCount}}}
	\pend% ending standard par
      ‚{\tiny $_{lb}$}‚

	  
	  \pstart \leavevmode% starting standard par
	\hphantom{.}एतेन य‚दुच्य‚ते प्र‚वृत्तिविष‚य‚व‚स्तुप्राप‚णं प्र‚त्य‚क्षानुमान‚योर‚विशिष्ट‚मि ति‚{\tiny $_{lb}$}‚ त‚द‚पि निर‚स्तं द्र‚ष्ट‚व्यं । अनाग‚तार्थ‚क्रियास‚म‚र्थो हि प्र‚वृत्तिविष‚यो न च त‚त्सा‚{\tiny $_{lb}$}‚म‚र्थ्यं प्र‚त्य‚क्षानुमानाभ्यां निश्चीय‚त इत्युक्तं । प‚रिच्छिन्न‚श्च प्र‚वृत्तिविष‚य इष्य‚ते‚{\tiny $_{lb}$}‚ न च स‚न्तानः प्र‚त्य‚क्षादिक्ष‚णेन प‚रिच्छिन्न‚स्त‚त्क‚थं प्र‚वृत्तिविष‚यः ।
	{\color{gray}{\rmlatinfont\textsuperscript{§~\theparCount}}}
	\pend% ending standard par
      ‚{\tiny $_{lb}$}‚‚{\tiny $_{lb}$}‚\textsuperscript{\textenglish{7/s}}

	  
	  \pstart \leavevmode% starting standard par
	अथैक‚स्मिन् क्ष‚णे प्र‚त्य‚क्षं प्र‚वृत्त‚म‚पि नि‚{\tiny $_{२}$}‚श्च‚य‚व‚शात् क्ष‚ण‚सामान्य‚विष‚य‚त्वे‚{\tiny $_{lb}$}‚नैक‚स‚न्त‚तिविष‚य‚न्त‚द‚युक्तं । प्र‚तिभास‚मानेनैव हि विष‚येण निश्च‚य‚व‚शात् प्र‚त्य‚क्षं‚{\tiny $_{lb}$}‚ सामान्य‚विष‚यं व्य‚व‚स्थाप्य‚ते न प‚र‚मार्थ‚तः । स्व‚ल‚क्ष‚ण‚विष‚य‚त्वात् । य‚था लिंग‚{\tiny $_{lb}$}‚विष‚यं प्र‚त्य‚क्षं । त‚था हि प्र‚तिभास‚मान‚मिदं धूम‚स्व‚ल‚क्ष‚ण‚न्तार्ण्ण‚म्वा पार्ण्ण‚म्वा‚{\tiny $_{lb}$}‚न्य‚द्वा स‚म्भ‚व‚ति [।] त‚त्र च विशेषान‚व‚धार‚णेन क्ष‚ण‚मात्र‚निश्च‚येन च स्व‚ल‚क्ष‚ण‚{\tiny $_{lb}$}‚विष‚य‚म‚पि सामान्य‚{\tiny $_{३}$}‚विष‚यं प्र‚त्य‚क्षं व्य‚व‚स्थाप्य‚ते । प्र‚त्य‚क्ष‚पृष्ठ‚भाविनो निश्च‚य‚स्य‚{\tiny $_{lb}$}‚ प्र‚त्य‚क्ष‚विष‚यानुसारित्वात् । न पुन‚रेक‚क्ष‚ण‚विष‚य‚म्प्र‚त्य‚क्ष‚मेवं व्य‚व‚स्थाप‚यितुं‚{\tiny $_{lb}$}‚ श‚क्य‚ते य‚तो यः प्र‚त्य‚क्षे प्र‚तिभास‚ते क्ष‚णो नासौ निश्चितो नापि पूर्व‚क्ष‚ण‚रूपो प‚र‚क्ष‚ण‚{\tiny $_{lb}$}‚रूपो वा स‚म्भ‚व‚ति भिन्न‚काल‚त्वात् । त‚त् क‚थ‚न्त‚त्र विशेषान‚व‚धार‚णेन क्ष‚ण‚{\tiny $_{lb}$}‚मात्र‚निश्च‚येन च क्ष‚ण‚सामान्य‚विष‚यं प्र‚त्य‚क्षं व्य‚व‚स्थाप्येत । प्र‚त्य‚क्ष‚{\tiny $_{४}$}‚पृष्ठ‚भाविनः स‚{\tiny $_{lb}$}‚ एवाय‚मिति निश्च‚य‚स्य सामान्य‚विष‚य‚त्वेपि न स‚न्तान‚विष‚य‚त्वं । प्र‚तिप‚न्न‚प्र‚तीय‚{\tiny $_{lb}$}‚मान‚योर्विष‚यीक‚र‚णेनानाग‚त‚क्ष‚णानिश्च‚याद् विजातीय‚व्यावृत्त‚रूप‚विष‚य‚त्वाच्चात‚{\tiny $_{lb}$}‚ एव क्ष‚ण‚स्य प्र‚तिभासेप्य‚यं घ‚ट इति ज्ञाप‚न‚निश्च‚यः ।
	{\color{gray}{\rmlatinfont\textsuperscript{§~\theparCount}}}
	\pend% ending standard par
      ‚{\tiny $_{lb}$}‚

	  
	  \pstart \leavevmode% starting standard par
	न‚नु य‚द्य‚र्थ‚क्रियास‚म‚र्थो व‚स्तुस‚न्तानोनप्र‚त्य‚क्षानुमानाभ्यां प‚रिच्छिद्य‚ते [।]‚{\tiny $_{lb}$}‚ क‚थ‚न्त‚र्हि त‚योः प्र‚व‚र्त्त‚क‚त्व‚म् [।] व‚स्तुस्व‚रूप‚मात्र‚प‚रि‚{\tiny $_{५}$}‚च्छेदादिति ब्रूमः । त‚था‚{\tiny $_{lb}$}‚ ह्य‚ग्न्यादिक‚म्व‚स्त्वेक‚त्र प्र‚ब‚न्ध‚वृत्त्या दाहाद्य‚र्थ‚क्रियास‚म‚र्थं प्र‚त्य‚क्षेण निश्चिन्व‚न्न‚न्य‚{\tiny $_{lb}$}‚त्रापि देशादावेत‚दुप‚ल‚भ्य‚मानं प्र‚ब‚न्ध‚वृत्त्यैवेदृग्विधार्थ‚कारीति निश्चाय‚य‚ति ।‚{\tiny $_{lb}$}‚ स एव पुनः कालान्त‚रे फ‚लार्थी त‚त्स्व‚रूप‚म्प‚र्येष‚माणोग्न्यादिकं प्र‚त्य‚क्षानुमाना‚{\tiny $_{lb}$}‚भ्यां दृष्ट्वा त‚देवेद‚मिति निश्च‚य‚पूर्व्व‚कं प्र‚व‚र्त्त‚ते । तेनाय‚म‚र्थ‚स्त‚त्स्व‚रूप‚मात्र‚प‚रिच्छेदे‚{\tiny $_{lb}$}‚ प्र‚माण‚व्यापारो न प्र‚ब‚न्ध‚वृ‚{\tiny $_{६}$}‚त्तिप‚रिच्छेदे । नाप्य‚र्थ‚क्रियासाम‚र्थ्य‚प‚रिच्छेदे पूर्व‚{\tiny $_{lb}$}‚मेवास्य स‚र्व‚स्य प‚रिच्छिन्न‚त्वात् । तेन प्र‚त्य‚क्षानुमान‚योः प्र‚व‚र्त्त‚क‚त्त्वं युक्त‚मेव ।‚{\tiny $_{lb}$}‚ क‚थ‚न्त‚र्ह्य‚र्थान‚र्थ‚विवेच‚न‚स्यानुमानाश्र{य‚त्व‚मुक्त‚म् ।... ... ... ... ... ...} त नैवात्रार्था\edtext{}{\lemma{नैवात्रार्था}\Bfootnote{? त्म}} नाभिप्रेत‚म‚पि तु च‚तुरार्य‚स‚त्यं । य‚तो‚{\tiny $_{lb}$}‚ मुक्त्य‚र्थिनो व‚यं मुक्तिश्च च‚तुरार्य‚स‚त्य‚द‚र्श‚नाद् भ‚व‚तीति भ‚ग‚व‚तोक्तं [।] त‚द्द‚र्श‚न‚ञ्च‚{\tiny $_{७}$}‚‚{\tiny $_{lb}$}‚ भाव‚नाभ्यास‚तो निष्प‚द्य‚ते[।]भाव‚नायां प्र‚वृत्तिश्च च‚तुरार्य‚स‚त्य‚निश्च‚येन[।] \leavevmode\ledsidenote{\textenglish{4a/PSVTa}}‚{\tiny $_{lb}$}‚ त‚न्निश्च‚य‚श्च प‚रोक्ष‚त्वाद‚नुमानादेव भ‚व‚तीत्य‚र्थान‚र्थ‚विवेच‚नाश्र‚य‚त्व‚म‚नुमान‚स्यैव ।
	{\color{gray}{\rmlatinfont\textsuperscript{§~\theparCount}}}
	\pend% ending standard par
      ‚{\tiny $_{lb}$}‚

	  
	  \pstart \leavevmode% starting standard par
	अर्थो निरोध‚मार्गावुपादेय‚त्वाद‚न‚र्थो दुःख‚स‚मुद‚यौ त्याज्य‚त्वात् । य‚द्वार्थः‚{\tiny $_{lb}$}‚ प‚र‚मार्थ‚स‚त्य‚म‚न‚र्थः संवृत्तिस‚त्यं त‚योर्य‚द् विवेच‚नं स्व‚रूपेण व्य‚व‚स्थाप‚न‚न्त‚स्यानु‚{\tiny $_{lb}$}‚मानाश्र‚य‚त्वात् । अर्थान‚र्थ‚विवेच‚न‚कारि च स‚र्वं ज्ञानं‚{\tiny $_{१}$}‚ न स्व‚ल‚क्ष‚णं गृह्णात्य‚पि‚{\tiny $_{lb}$}‚ त्व‚ध्य‚व‚स्य‚तीति भ्रान्त‚मेव [।] तेन य‚दि श‚ब्दादिज्ञान‚म‚र्थान‚र्थ‚विवेच‚नाश्र‚य‚मिष्य‚ते‚{\tiny $_{lb}$}‚ त‚द‚पि भ्रान्त‚त्वाद‚प्र‚माण‚मेव [।] अत एव न त‚स्येह व्युत्पाद्य‚ताप्र‚संगः । अनुमान‚स्य‚{\tiny $_{lb}$}‚ ‚{\tiny $_{lb}$}‚ \leavevmode\ledsidenote{\textenglish{8/s}}तु भ्रान्त‚त्वे स‚त्य‚पि प्र‚तिब‚न्ध‚व‚शात् प्रामाण्यं [।] श‚ब्दादिज्ञान‚स्य त्वेवं‚{\tiny $_{lb}$}‚ प्रामाण्येभ्युप‚ग‚म्य‚मानेऽनुमानेन्त‚र्भावाद‚प‚क्ष‚ध‚र्म‚स्याग‚म‚क‚त्वाद‚र्थान‚र्थ‚विवेच‚नाश्र‚य‚त्व‚{\tiny $_{lb}$}‚म‚नुमान‚स्यैव ।
	{\color{gray}{\rmlatinfont\textsuperscript{§~\theparCount}}}
	\pend% ending standard par
      ‚{\tiny $_{lb}$}‚

	  
	  \pstart \leavevmode% starting standard par
	न‚नु प्र‚तिब‚{\tiny $_{२}$}‚न्ध‚व‚शाद‚नुमान‚स्य प्रामाण्ये नित्यादिविक‚ल्प‚स्यापि प्रामाण्यं‚{\tiny $_{lb}$}‚ स्यात् क्ष‚णिकाद्य‚र्थे प्र‚तिब‚न्धाद् [।] अथाध्य‚व‚सितार्थ‚प्र‚तिब‚न्धेन प्रामाण्यं म‚णिप्र‚{\tiny $_{lb}$}‚भायाम्म‚णिज्ञान‚स्य प्रामाण्यं स्यात् । त‚द‚पि ह्य‚ध्य‚व‚सितेन म‚णिना स‚म्ब‚द्ध‚न्त‚{\tiny $_{lb}$}‚स्मात् स‚त्य‚पि प्र‚तिब‚न्धे य‚द्देशादिस‚म्ब‚न्धित‚या योर्थोध्य‚व‚सित‚स्त‚द्देशादिस‚म्ब‚न्धित‚या‚{\tiny $_{lb}$}‚ स‚न्तानैक‚त्वाध्य‚व‚सायात् त‚म‚र्थं प्राप‚य‚द‚नुमान‚ज्ञानं प्र‚मा‚{\tiny $_{३}$}‚ण‚मेव । न स‚र्वं ज्ञानं ।
	{\color{gray}{\rmlatinfont\textsuperscript{§~\theparCount}}}
	\pend% ending standard par
      ‚{\tiny $_{lb}$}‚

	  
	  \pstart \leavevmode% starting standard par
	\textbf{अर्थान‚र्थ‚विवे}च‚नं चाधिग‚म‚रूप‚माकारोनुमान‚म्प्र‚माण‚त्वात् । य‚द्वा लिङ्ग‚{\tiny $_{lb}$}‚मेवानुमान‚म‚तोर्थान‚र्थ‚विवेच‚न‚स्या\textbf{नुमानाश्र‚य‚त्वं । त‚द्विप्र‚तिप‚त्ते}रिति त‚स्मिन्न‚नुमाने‚{\tiny $_{lb}$}‚ स‚म्मोहात् । \textbf{त‚द्व्य‚व‚स्थाप‚नाय} त‚स्यानुमान‚स्य विप्र‚तिप‚त्त्य‚प‚न‚य‚नेनाव‚स्थाप‚ना\textbf{याह}‚{\tiny $_{lb}$}‚ सू त्र का रः ।
	{\color{gray}{\rmlatinfont\textsuperscript{§~\theparCount}}}
	\pend% ending standard par
      ‚{\tiny $_{lb}$}‚

	  
	  \pstart \leavevmode% starting standard par
	\textbf{प‚क्ष‚ध‚र्म} इत्यादि ।
	{\color{gray}{\rmlatinfont\textsuperscript{§~\theparCount}}}
	\pend% ending standard par
      ‚{\tiny $_{lb}$}‚

	  
	  \pstart \leavevmode% starting standard par
	य‚द्य‚नुमानं व्य‚व‚स्थाप्यं क‚स्मात् प‚क्ष‚ध‚र्म इ‚{\tiny $_{४}$}‚त्यादिना हेतुमेव व्य‚व‚स्थाप‚य‚तीति‚{\tiny $_{lb}$}‚ चेत् ॥ अदोषोयं । हेतुविप्र‚तिप‚त्तिद्वारेणानुमाने विप्र‚तिप‚त्तेस्त‚द्व्युत्प‚त्तिद्वारेणैव‚{\tiny $_{lb}$}‚ त‚स्य व्य‚व‚स्थाप‚नं । अनुमान‚ज्ञानं च त्रिरूप‚लिंगादुत्प‚द्य‚मानं लोक‚प्र‚तीत‚मेवातो‚{\tiny $_{lb}$}‚ विप्र‚तिप‚त्तिः प्र‚तीत्यैव निराक्रिय‚ते ।
	{\color{gray}{\rmlatinfont\textsuperscript{§~\theparCount}}}
	\pend% ending standard par
      ‚{\tiny $_{lb}$}‚

	  
	  \pstart \leavevmode% starting standard par
	य‚द्वानुमान‚श‚ब्देन य‚दा लिङ्ग‚मेवोच्य‚ते त‚दा त‚द्विप्र‚तिप‚त्तेर्हेतुमेव व्य‚व‚स्था‚{\tiny $_{lb}$}‚प‚य‚तीत्य‚दोषः ।
	{\color{gray}{\rmlatinfont\textsuperscript{§~\theparCount}}}
	\pend% ending standard par
      ‚{\tiny $_{lb}$}‚

	  
	  \pstart \leavevmode% starting standard par
	अत्र‚{\tiny $_{५}$}‚ श्लोके लिंग‚स्य ल‚क्ष‚णं संख्यानिय‚मः संख्यानिय‚म‚कार‚ण‚म्विप‚क्ष‚निवृत्ति‚{\tiny $_{lb}$}‚श्चोक्ता । \textbf{प‚क्ष‚ध‚र्म‚स्त‚द‚ङ्शेन व्याप्त} इति ल‚क्ष‚णं ।
	{\color{gray}{\rmlatinfont\textsuperscript{§~\theparCount}}}
	\pend% ending standard par
      ‚{\tiny $_{lb}$}‚

	  
	  \pstart \leavevmode% starting standard par
	त‚स्य \textbf{प‚क्ष}स्याङ्शः साध‚यितुमिष्टो \textbf{ध}र्म‚स्तेन \textbf{व्याप्त} एवेत्य‚व‚धार‚णं [।]‚{\tiny $_{lb}$}‚ \textbf{त्रिधैवे}ति संख्यानिय‚मः । \textbf{अविनाभाव‚निय‚मादि}ति संख्यानिय‚म‚कार‚णं ।‚{\tiny $_{lb}$}‚ त्रिष्वेवाविनाभाव‚स्य निय‚त‚त्वादित्य‚र्थः । \textbf{हेत्वाभासास्त‚तोप‚र} इति विप‚क्ष‚निवृत्तिः ।‚{\tiny $_{६}$}‚‚{\tiny $_{lb}$}‚ त‚स्माद्धेतुत्र‚याद‚न्ये हेत्वाभासाः ।
	{\color{gray}{\rmlatinfont\textsuperscript{§~\theparCount}}}
	\pend% ending standard par
      ‚{\tiny $_{lb}$}‚‚{\tiny $_{lb}$}‚\textsuperscript{\textenglish{9/s}}

	  
	  \pstart \leavevmode% starting standard par
	न‚नु य‚दि त‚दंश‚व्याप्तिर्दृष्टान्त एव गृह्य‚ते । त‚दानुमान‚स्योत्थान‚न्न स्यात् ।‚{\tiny $_{lb}$}‚ साध्य‚ध‚र्मिणि साध्य‚ध‚र्मेण हेतोर्व्याप्त्य‚ग्र‚हात् । त‚दा च प‚क्ष‚ध‚र्मो हेतुरिति व्य‚र्थं‚{\tiny $_{lb}$}‚ ल‚क्ष‚ण‚म‚ग‚म‚क‚त्वात् । स‚र्वोप‚संहारेण व्याप्तिग्र‚ह‚णेपि नानुमान‚स्य प्रामाण्यं‚{\tiny $_{lb}$}‚ स्यात् । व्याप्तिग्राह‚क‚प्र‚माण‚प्र‚तिप‚न्न‚विष‚य‚त्वेन स्मृतिरूप‚त्वात् । प‚क्ष‚ध‚र्मो‚{\tiny $_{lb}$}‚ हेतुरिति‚{\tiny $_{७}$}‚ च न व‚क्त‚व्य‚न्त‚द‚ङ्श‚व्याप्तिव‚च‚नेनैव ग‚त‚त्वात् । त‚द‚ङ्श‚व्याप्तिब‚लेन \leavevmode\ledsidenote{\textenglish{4b/PSVTa}}‚{\tiny $_{lb}$}‚ च प‚क्ष‚ध‚र्म‚स्य ग‚म‚क‚त्व‚न्नं प‚क्षे स‚त्तामात्रेण त‚त्र‚स्थ‚स्य ग‚र्द‚भादेर‚ग‚म‚क‚त्वात् । त‚स्मान्नं‚{\tiny $_{lb}$}‚ प‚क्ष‚ध‚र्मो हेतुरिति पृथ‚ग् ल‚क्ष‚ण‚म्व‚क्त‚व्यं ।
	{\color{gray}{\rmlatinfont\textsuperscript{§~\theparCount}}}
	\pend% ending standard par
      ‚{\tiny $_{lb}$}‚

	  
	  \pstart \leavevmode% starting standard par
	त‚दुक्त‚म् ।
	{\color{gray}{\rmlatinfont\textsuperscript{§~\theparCount}}}
	\pend% ending standard par
      ‚{\tiny $_{lb}$}‚
	  \bigskip
	  \begingroup
	
	    
	    \stanza[\smallbreak]
	  {\normalfontlatin\large ``\qquad}अन्य‚थानुप‚प‚न्न‚त्वं य‚त्र त‚त्र त्र‚येण किं ।&‚{\tiny $_{lb}$}‚नान्य‚थानुप‚प‚न्न‚त्वं य‚त्र त‚त्र त्र‚येण किमिति ।{\normalfontlatin\large\qquad{}"}\&[\smallbreak]
	  
	  
	  
	  \endgroup
	‚{\tiny $_{lb}$}‚

	  
	  \pstart \leavevmode% starting standard par
	किं चानुप‚ल‚ब्धेस्ताव‚न्न प‚क्ष‚ध‚र्म‚त्व‚म‚न्योप‚ल‚ब्धेः पुरुष‚ध‚र्म‚त्वात् । स्व‚भाव‚हेतोश्च‚{\tiny $_{lb}$}‚ ध‚{\tiny $_{१}$}‚र्मिरूप‚त्वात् कार्य‚हेतोर‚पि स्वात‚न्त्र्येण ध‚र्म्य‚न‚पेक्ष‚त्वात् । न च क‚ल्पित‚स्य‚{\tiny $_{lb}$}‚ प‚क्ष‚ध‚र्म‚स्य कार्य‚स्व‚भाव‚हेतुत्वं साध्य‚व्याप्तिश्चेति न प‚क्ष‚ध‚र्मो हेतुरिति व‚क्त‚व्यं ।
	{\color{gray}{\rmlatinfont\textsuperscript{§~\theparCount}}}
	\pend% ending standard par
      ‚{\tiny $_{lb}$}‚

	  
	  \pstart \leavevmode% starting standard par
	त‚था य‚दि हेतुत्वेन त्रित्वं व्याप्त‚न्त‚दा हेतुत्व‚स्यानिय‚त‚त्वाद‚न्य‚त्रापि संयोग्या‚{\tiny $_{lb}$}‚दिषु हेतुत्व‚म‚निवारित‚मेवेति त्रिधैवेत्य‚व‚धार‚णं न युज्य‚ते ।
	{\color{gray}{\rmlatinfont\textsuperscript{§~\theparCount}}}
	\pend% ending standard par
      ‚{\tiny $_{lb}$}‚

	  
	  \pstart \leavevmode% starting standard par
	अथ हेतुत्वं त्रित्वेन व्याप्त‚न्त‚दा त्रित्व‚स्य हेताव‚निय‚त‚त्वात् कार्यादीनाम‚प्य‚{\tiny $_{lb}$}‚हेतु‚{\tiny $_{२}$}‚त्व‚न्त‚त‚श्च कार्यादेरेव हेतुत्व‚मिति न घ‚ट‚ते ।
	{\color{gray}{\rmlatinfont\textsuperscript{§~\theparCount}}}
	\pend% ending standard par
      ‚{\tiny $_{lb}$}‚

	  
	  \pstart \leavevmode% starting standard par
	किं च । य‚द्य‚नुप‚ल‚म्भ‚स्य साध्य‚प्र‚तिब‚न्धो नास्ति त‚दाऽप्र‚तिब‚द्धोपि हेतुर्ग‚म‚क‚{\tiny $_{lb}$}‚ इति \textbf{त्रिधैव स} इति निय‚मो न घ‚ट‚ते [।]
	{\color{gray}{\rmlatinfont\textsuperscript{§~\theparCount}}}
	\pend% ending standard par
      ‚{\tiny $_{lb}$}‚

	  
	  \pstart \leavevmode% starting standard par
	अथ प्र‚तिब‚न्धोस्ति त‚दा कार्य‚स्व‚भाव‚योरेवान्त‚र्भावात् त्रिधैव स हेतुरिति‚{\tiny $_{lb}$}‚ त‚थापि न युज्य‚ते । \textbf{हेत्वाभासास्त‚तोऽप‚र} इति न युक्तं हेत्व‚न्त‚र‚स्यात्य‚न्त‚प‚रोक्ष‚त्वान्न‚{\tiny $_{lb}$}‚ त‚द‚भावः प्र‚त्य‚क्षादिनिश्चित इत्य‚युक्त‚मुक्तं ।
	{\color{gray}{\rmlatinfont\textsuperscript{§~\theparCount}}}
	\pend% ending standard par
      ‚{\tiny $_{lb}$}‚
	    
	    \stanza[\smallbreak]
	  \textbf{प‚क्ष‚ध‚र्म‚स्त‚दंशेन व्याप्तो हेतुस्त्रिधैव सः ।‚{\tiny $_{lb}$}‚ अविनाभाव‚निय‚माद्धेत्वाभासास्त‚तोऽप‚र इति} ॥\&[\smallbreak]
	  
	  
	  ‚{\tiny $_{lb}$}‚

	  
	  \pstart \leavevmode% starting standard par
	अत्रोच्य‚ते । य‚द्य‚पि साध्य‚साध‚न‚योर्व्याप्तिः स‚र्वोप‚संहारेण प्र‚तिप‚न्ना त‚थापि‚{\tiny $_{lb}$}‚ न व्याप्तिग्र‚ह‚ण‚मात्रादिह साध्य‚ध‚र्मिणीदानीं साध्य‚ध‚र्म इति विशेषेण निश्च‚यो‚{\tiny $_{lb}$}‚ भ‚व‚त्य‚नुमानात्तु स्यात् । त‚स्माद् प्र‚तिप‚न्न‚विशिष्ट‚देशादिस‚म्ब‚न्धिसाध्यार्थ‚{\tiny $_{lb}$}‚प्र‚तिपाद‚क‚त्वेन प्र‚माण‚मेवानु‚{\tiny $_{४}$}‚मान‚न्त‚च्च प‚क्ष‚ध‚र्म‚त्वे स‚त्येव भ‚व‚ति नान्य‚था । य‚तो‚{\tiny $_{lb}$}‚ नान्य‚देशादिस्थेन साध्य‚ध‚र्मिणान्य‚देशादिस्थः साध‚न‚ध‚र्मः स‚म्ब‚द्धोऽतो विशिष्ट‚देशा‚{\tiny $_{lb}$}‚द्य‚व‚च्छिन्न‚साध‚नाव‚ग‚तिसाम‚र्थ्यादेव विशिष्ट‚देशाद्य‚व‚च्छिन्न‚साध्य‚प्र‚तीतिरेवा‚{\tiny $_{lb}$}‚नुमानं [।] न तु ध‚म‚मात्राद‚ग्निमात्र‚प्र‚तीतिस्त‚स्या व्याप्तिग्राह‚क‚प्र‚माण‚फ‚ल‚त्वात् ।‚{\tiny $_{lb}$}‚ ‚{\tiny $_{lb}$}‚ \leavevmode\ledsidenote{\textenglish{10/s}}नापि य‚त्र साध‚न‚ध‚र्म‚स्त‚त्र साध्य‚ध‚र्म इत्य‚विशेषेणाव‚ग‚मेपि सा‚{\tiny $_{५}$}‚ध‚न‚स्य‚{\tiny $_{lb}$}‚ प‚क्ष‚ध‚र्म‚त्वं सिध्य‚ति साध्य‚ध‚र्मिध‚र्म‚त‚या विशेषेणाप्र‚तीतेः सामान्येनाभिधानात् ।‚{\tiny $_{lb}$}‚ त‚स्माद् विशिष्ट‚देशाद्य‚व‚च्छिन्न‚साध्य‚प्र‚तिप‚त्त‚ये प‚क्ष‚ध‚र्म‚त्व‚न्द‚र्श‚नीयं ।
	{\color{gray}{\rmlatinfont\textsuperscript{§~\theparCount}}}
	\pend% ending standard par
      ‚{\tiny $_{lb}$}‚

	  
	  \pstart \leavevmode% starting standard par
	\hphantom{.}तेन य‚दुक्तं य‚त्र य‚त्र धूम‚स्त‚त्र त‚त्राग्निरित्य‚नेनैव प‚क्ष‚ध‚र्म‚स्योक्त‚त्वात् प्र‚देश‚{\tiny $_{lb}$}‚विशेषेग्निसिद्ध्य‚र्थ‚न्धूम‚श्चात्रेति न व‚क्त‚व्य‚मुक्तार्थ‚त्वादिति त‚द‚पास्तं ॥
	{\color{gray}{\rmlatinfont\textsuperscript{§~\theparCount}}}
	\pend% ending standard par
      ‚{\tiny $_{lb}$}‚

	  
	  \pstart \leavevmode% starting standard par
	न‚न्वेव‚म‚नुमान‚स्यं प्रामाण्येऽप‚क्ष‚ध‚र्म्म‚म‚प्य‚नु‚{\tiny $_{६}$}‚मानं प्र‚माणं स्याद‚प्र‚तिप‚न्नाधि‚{\tiny $_{lb}$}‚ग‚मात् । य‚थाऽध‚स्तान्न‚दीपूर‚न्दृष्ट्वोप‚रिवृष्ट्य‚नुमानं । त‚था शिशुर‚यं ब्राह्म‚णः‚{\tiny $_{lb}$}‚ मातापित्रोर्ब्राह्म‚ण्यादिति । त‚दुक्तं ।
	{\color{gray}{\rmlatinfont\textsuperscript{§~\theparCount}}}
	\pend% ending standard par
      ‚{\tiny $_{lb}$}‚
	  \bigskip
	  \begingroup
	
	    
	    \stanza[\smallbreak]
	  {\normalfontlatin\large ``\qquad}न‚दीपूरोप्य‚धो देशे दृष्टः \add{स‚न्नुप‚रिस्थितां ।}&‚{\tiny $_{lb}$}‚\add{निय‚म्यो ग‚म‚य‚त्येव}\edtext{\textsuperscript{*}}{\edlabel{pvsvt_10-1}\label{pvsvt_10-1}\lemma{*}\Bfootnote{In the margin. }}वृत्तां वृष्टिं नियामिकां ॥&‚{\tiny $_{lb}$}‚एवं \add{प्र‚त्य‚क्ष‚ध‚र्म‚त्वं ज्येष्ठं हेत्व‚ङ्ग}\edtext{}{\edlabel{pvsvt_10-1b}\label{pvsvt_10-1b}\lemma{एवं}\Bfootnote{In the margin. }}मिष्य‚ते ।&‚{\tiny $_{lb}$}‚त‚त्पूर्वोक्तान्य‚ध‚र्म‚त्व‚द‚र्श‚नाद् व्य‚भिचार्य‚ते ॥&‚{\tiny $_{lb}$}‚\leavevmode\ledsidenote{\textenglish{5a/PSVTa}}पित्रोश्च ब्राह्म‚ण‚त्वेन पुत्र‚ब्राह्म‚ण‚{\tiny $_{७}$}‚तानुमा ।&‚{\tiny $_{lb}$}‚स‚र्व‚लोक‚प्र‚सिद्धा न प‚क्ष‚ध‚र्म‚म‚पेक्ष‚ते ॥&‚{\tiny $_{lb}$}‚क्लेशेन प‚क्ष‚ध‚र्म‚त्वं य‚स्त‚त्रापि प्र‚क‚ल्प‚येत् ।&‚{\tiny $_{lb}$}‚न संग‚च्छेत त‚स्यैत‚ल्ल‚क्ष्येण स‚ह ल‚क्ष‚णं ॥&‚{\tiny $_{lb}$}‚य‚था लोक‚प्र‚सिद्धं च ल‚क्ष‚णैर‚नुग‚म्य‚ते ।&‚{\tiny $_{lb}$}‚ल‚क्ष्य‚स्य ल‚क्ष‚ण‚मेवं स्यात् त‚द‚पूर्व‚न्न साध्य‚त इति ।{\normalfontlatin\large\qquad{}"}\&[\smallbreak]
	  
	  
	  
	  \endgroup
	‚{\tiny $_{lb}$}‚

	  
	  \pstart \leavevmode% starting standard par
	अत्रोच्य‚ते । क‚स्मादुप‚र्येव वृष्ट्य\add{नुमानं नान्य‚त्र [।] पूर‚स्य त‚त्स‚म्ब‚न्धि‚{\tiny $_{lb}$}‚त्वादि}\edtext{}{\edlabel{pvsvt_10-1c}\label{pvsvt_10-1c}\lemma{वृष्ट्य}\Bfootnote{In the margin. }}ति चेत् । य‚द्येवं य‚तोयं न‚दीपूर आयात‚स्त‚त्र‚{\tiny $_{१}$}‚ वृष्ट्य‚नुमान‚न्नान्य‚त्र व्य‚भि‚{\tiny $_{lb}$}‚चारात् । प‚र‚स्य च त‚त्स‚म्ब‚न्धित्व‚निश्च‚ये स‚ति ग‚म‚क‚त्व‚म‚न्य‚थाऽनैकान्तिक‚त्वं‚{\tiny $_{lb}$}‚ स्यात् । त‚था शिशुर‚यं ब्राह्म‚णः मातापित्रोर्ब्राह्म‚ण्यादित्य‚त्रापि य‚स्यैव शिशो‚{\tiny $_{lb}$}‚र्ब्राह्म‚ण्यं साध्य‚न्त‚स्यैव मातापितृब्राह्म‚ण्य‚ल‚क्ष‚णो ध‚र्मः स‚म्ब‚न्धी ग‚म‚को न मातापितृ‚{\tiny $_{lb}$}‚\add{ब्राह्म‚ण्य‚मात्र‚म‚न्य‚स‚म्ब‚न्धिमा}\edtext{}{\edlabel{pvsvt_10-1d}\label{pvsvt_10-1d}\lemma{मातापितृ}\Bfootnote{In the margin. }}तापितृब्राह्म‚ण्य‚स्याग‚म‚क‚त्वात् । तेनास्यापि‚{\tiny $_{lb}$}‚ प‚क्ष‚ध‚र्म‚त्वे स‚ति ग‚म‚क‚त्व‚म‚तो‚{\tiny $_{२}$}‚ न क्लेशेन प‚क्ष‚ध‚र्म‚त्व‚क‚ल्प‚ना [।] य‚द्वा य एवाव्य‚भि‚{\tiny $_{lb}$}‚चारे निमित्तं स एव हेतुर्य‚था धूम‚स्याग्निकार्य‚त्वं ब्राह्म‚ण‚भूत‚मातापितृज‚न्य‚त्वं‚{\tiny $_{lb}$}‚ च शिशोर्ब्राह्म‚ण्य‚निमित्त‚मिति त‚देव हेतुर्युक्तोन्य‚स्य त‚त्क‚ल्प‚ना क्लेशेन स्यादिति ।‚{\tiny $_{lb}$}‚ त‚था न च‚न्द्रोद‚यात् स‚मुद्र‚वृद्ध्य‚नुमानं च‚न्द्रोद‚यात् \add{पूर्वं प‚श्चाद‚पि}\edtext{}{\edlabel{pvsvt_10-1e}\label{pvsvt_10-1e}\lemma{यात्}\Bfootnote{In the margin. }}‚{\tiny $_{lb}$}‚  \leavevmode\ledsidenote{\textenglish{11/s}}त‚द‚नुमान‚प्र‚स‚ङ्गात् । च‚न्द्रोद‚य‚काल एव त‚द‚नुमान‚न्त‚दैव व्याप्तेर्गृहीत‚त्वा‚{\tiny $_{३}$}‚‚{\tiny $_{lb}$}‚दिति चेत् ।
	{\color{gray}{\rmlatinfont\textsuperscript{§~\theparCount}}}
	\pend% ending standard par
      ‚{\tiny $_{lb}$}‚

	  
	  \pstart \leavevmode% starting standard par
	य‚द्येव‚न्त‚त्काल‚स‚म्ब‚न्धित्व‚मेव साध्य‚साध‚न‚योः । त‚दा च स एव कालो ध‚र्मीं‚{\tiny $_{lb}$}‚ त‚त्रैव च साध्यानुमानं च‚न्द्रोद‚य‚श्च त‚त्स‚म्ब‚न्धीति क‚थ‚म‚प‚क्ष‚ध‚र्म‚त्व‚म् [।] अथ‚{\tiny $_{lb}$}‚ कालो नेष्य‚ते न त‚दा त‚र्ह्येत‚द‚नुमान‚म्व्य‚भिचाराद् [।] अथ बौ द्धा नामेत‚द‚नुमान‚{\tiny $_{lb}$}‚न्नास्ति कालाभावात् [।]
	{\color{gray}{\rmlatinfont\textsuperscript{§~\theparCount}}}
	\pend% ending standard par
      ‚{\tiny $_{lb}$}‚

	  
	  \pstart \leavevmode% starting standard par
	त‚द‚युक्तं । पूर्वाह्णादिप्र‚त्य‚य‚विष‚य‚स्य म‚हाभूत‚विशेष‚स्य काल इत्य‚भिधेय‚{\tiny $_{lb}$}‚स्याभ्युप‚ग‚मात् ।‚{\tiny $_{४}$}‚ एव‚म‚न्य‚त्रापि प‚क्ष‚ध‚र्म‚त्वं योज्यं ।
	{\color{gray}{\rmlatinfont\textsuperscript{§~\theparCount}}}
	\pend% ending standard par
      ‚{\tiny $_{lb}$}‚

	  
	  \pstart \leavevmode% starting standard par
	न‚नु भ‚व‚तु प‚क्ष‚ध‚र्म‚त्वे स‚त्य‚नुमान‚स्य प्रामाण्य‚न्त‚थापि प‚क्ष‚ध‚र्म इति पृथ‚ग्‚{\tiny $_{lb}$}‚ ल‚क्ष‚णं न क‚र्त्त‚व्य‚न्त‚द‚ङ्श‚व्याप्त‚व‚च‚नेनैव ग‚त‚त्वात् ।
	{\color{gray}{\rmlatinfont\textsuperscript{§~\theparCount}}}
	\pend% ending standard par
      ‚{\tiny $_{lb}$}‚

	  
	  \pstart \leavevmode% starting standard par
	स‚त्यं [।] किन्त्व‚प‚क्ष‚ध‚र्म‚स्यापि साध्य‚व्याप्त‚स्य हेतुत्व‚निरासार्थं कृतं । म‚हान‚{\tiny $_{lb}$}‚सादिदृष्ट‚धूमादि चोद‚धाव‚ग्न्य‚नुमाने ।
	{\color{gray}{\rmlatinfont\textsuperscript{§~\theparCount}}}
	\pend% ending standard par
      ‚{\tiny $_{lb}$}‚

	  
	  \pstart \leavevmode% starting standard par
	न‚नु व्याप्त‚स्य लिङ्ग‚त्वं न च म‚हान‚सादिग‚तो धूम उद‚धौ साध्येनाग्निना‚{\tiny $_{५}$}‚‚{\tiny $_{lb}$}‚ व्याप्तः [।]
	{\color{gray}{\rmlatinfont\textsuperscript{§~\theparCount}}}
	\pend% ending standard par
      ‚{\tiny $_{lb}$}‚

	  
	  \pstart \leavevmode% starting standard par
	स‚त्यं [।] केव‚लं व्याप्तो हेतुरित्येताव‚न्मात्र‚केण ल‚क्ष‚ण‚व‚च‚नेन य‚त्रैव व्याप्य‚ध‚र्म‚{\tiny $_{lb}$}‚स्त‚त्रैव व्याप‚क‚ध‚र्मानुमानमित्येत‚न्न ल‚भ्य‚ते । त‚त‚श्चान्य‚त्रापि साध्यानुमाना‚{\tiny $_{lb}$}‚श‚ङ्कानिवृत्त्य‚र्थं प‚क्ष‚ध‚र्म‚व‚च‚नं । अनुप‚ल‚ब्धेर‚पि प‚क्ष‚ध‚र्म‚त्व‚म‚स्त्येव । य‚दा ह्य‚न्य‚स्य‚{\tiny $_{lb}$}‚ भूत‚लादेरुप‚ल‚म्भ‚ज‚न‚न‚योग्य‚तैवान्यानुप‚ल‚ब्धिस्त‚दा योग्य‚तान्य‚भूत‚लादिस्व‚भावेति‚{\tiny $_{६}$}‚‚{\tiny $_{lb}$}‚ क‚थ‚न्नानुप‚ल‚ब्धेः प‚क्ष‚ध‚र्म‚त्वं । कृत‚क‚त्वादेर‚प्येवं श‚ब्दादिध‚र्म‚त्वं । पुरुष‚ध‚र्म‚रूपाया‚{\tiny $_{lb}$}‚ अप्य‚नुप‚ल‚ब्धेर‚न्य‚भूत‚लादिकार्य‚त्व‚मेव प‚र‚मार्थ‚त‚स्त‚द्ध‚र्म‚त्व‚न्त‚दाय‚त्त‚त्वात् । धूमा‚{\tiny $_{lb}$}‚देर‚पि कार्य‚स्यैवं प्र‚देशादिध‚र्म‚त्व‚ङ्केव‚ल‚म्विक‚ल्पेन तेषां स‚म्ब‚न्धिस्व‚रूप‚मेव‚{\tiny $_{lb}$}‚ प‚क्ष‚स्यायं ध‚र्म इति व्य‚व‚स्थाप्य‚ते ।
	{\color{gray}{\rmlatinfont\textsuperscript{§~\theparCount}}}
	\pend% ending standard par
      ‚{\tiny $_{lb}$}‚

	  
	  \pstart \leavevmode% starting standard par
	हेतुत्वं च धूमादेर‚विनाभावेन व्याप्त‚म‚ज्ञातावि‚{\tiny $_{७}$}‚नाभाव‚स्याग‚म‚क‚त्वेना हेतु- \leavevmode\ledsidenote{\textenglish{5b/PSVTa}}‚{\tiny $_{lb}$}‚ त्वात् । अविनाभाव‚श्च कार्य‚स्व‚भावाभ्यां व्याप्तः । विधिप्र‚तिषेध‚योश्च साध्य‚त्वे‚{\tiny $_{lb}$}‚ स‚त्य‚र्थान्त‚र‚विधाने कार्य‚हेतोः स्व‚भाव‚हेतोः प्र‚तिषेधे चानुप‚ल‚ब्धेस्तेन हेतु‚{\tiny $_{lb}$}‚स्त्रित्वेन व्याप्तोऽतोसौ त्रिविध एव भ‚व‚ति । न तु त्रिविधो हेतुरेव कार्यादेर‚प्य‚{\tiny $_{lb}$}‚ज्ञात‚स्याहेतुत्वात् । कार‚ण‚व्याप‚कानुल‚ब्द्योर‚पि प्र‚तिब‚न्धादेव ग‚म‚क‚त्वं ।‚{\tiny $_{१}$}‚‚{\tiny $_{lb}$}‚ त‚था हि [।] य‚त एव प्र‚तिब‚न्धात् कार्य‚व्याप्ये कार‚ण‚व्याप‚के ग‚म‚य‚तः । त‚त एव‚{\tiny $_{lb}$}‚ प्र‚तिब‚न्धात् कार‚ण‚व्याप‚कानुप‚ल‚ब्धी कार्य‚व्याप्याभाव‚ङ्ग‚म‚य‚तः । स्व‚भावानुप‚ल‚{\tiny $_{lb}$}‚ब्धाव‚पि य‚दा घ‚टादेरुप‚ल‚म्भ‚ज‚न‚न‚योग्य आत्मा उप‚ल‚ब्धिरुच्य‚तेन्य‚हेतुसाक‚ल्ये‚{\tiny $_{lb}$}‚ चोप‚ल‚म्भाव्य‚भिचारादुप‚ल‚ब्धिः स‚त्ता त‚दान‚योस्तादात्म्य‚न्तेनात्रापि प‚क्षे प्र‚ति‚{\tiny $_{lb}$}‚‚{\tiny $_{lb}$}‚ \leavevmode\ledsidenote{\textenglish{12/s}}ब‚न्ध‚निब‚न्ध‚न‚मेव ग‚म‚क‚त्वं । न चानु‚{\tiny $_{२}$}‚प‚ल‚ब्धेः कार्यादिहेताव‚न्त‚र्भावः स्व‚साध्ये‚{\tiny $_{lb}$}‚ प्र‚तिब‚न्धान‚पेक्ष‚त्वात् । हेतोश्च स‚काशात् साध्य‚प्र‚तीतिस्त‚दाय‚त्त‚त्वे स‚ति स्यात्‚{\tiny $_{lb}$}‚ [।] न च संयोगे स‚ति त‚दाय‚त्त‚त्व‚म‚संयुक्तानां संयोगाभावात् [।] संयुक्तानाम‚पि न‚{\tiny $_{lb}$}‚ संयोग‚त‚स्त‚दाय‚त्त‚ताऽन्य‚त एव संयुक्तानामुत्प‚त्तेः । एवं स‚म‚वायेपि वाच्यं । त‚स्मा‚{\tiny $_{lb}$}‚त्त‚दाय‚त्त‚त्व‚न्तादात्म्य‚त‚दुत्प‚त्तिभ्यान्तेन कार्य‚स्व‚भावानुप‚ल‚म्भ‚ब‚हिर्भूतानां संयो‚{\tiny $_{३}$}‚‚{\tiny $_{lb}$}‚ग्यादीनाम‚हेतुत्व‚न्त‚दाय‚त्त‚त्वाभावेनाविनाभावाभावात् ।
	{\color{gray}{\rmlatinfont\textsuperscript{§~\theparCount}}}
	\pend% ending standard par
      ‚{\tiny $_{lb}$}‚

	  
	  \pstart \leavevmode% starting standard par
	अथ त‚दाय‚त्त‚त्व‚म‚स्ति त‚देव त‚र्हि ग‚म‚क‚त्वे निमित्त‚मिति स हेतुस्त्रिधैवेति सिद्धं ।
	{\color{gray}{\rmlatinfont\textsuperscript{§~\theparCount}}}
	\pend% ending standard par
      ‚{\tiny $_{lb}$}‚

	  
	  \pstart \leavevmode% starting standard par
	न‚नु त‚थापि क‚थ‚म‚नुमान‚स्योत्थान‚न्निर्विक‚ल्प‚क‚प्र‚त्य‚क्षेण ध‚र्म‚ध‚र्मित‚त्स‚म्ब‚न्धा‚{\tiny $_{lb}$}‚ग्र‚हात् स्वात‚न्त्र्येण व‚स्तुद्व‚याधिग‚तेः ।
	{\color{gray}{\rmlatinfont\textsuperscript{§~\theparCount}}}
	\pend% ending standard par
      ‚{\tiny $_{lb}$}‚

	  
	  \pstart \leavevmode% starting standard par
	उच्य‚ते [।] स‚विक‚ल्प‚केनापि धूम‚प्र‚देशादीनां ध‚र्म‚ध‚र्मिभाव‚ग्र‚ह‚ण‚मुप‚र्युप‚रि‚{\tiny $_{lb}$}‚भाव‚प्र‚तिभास‚{\tiny $_{४}$}‚ एव [।] स च निर्विक‚ल्प‚केप्य‚स्ति केव‚लं विक‚ल्पेत‚र‚योर्ज्ञान‚यो‚{\tiny $_{lb}$}‚रुल्लेखानुल्लेख‚कृतो विशेषः [।] त‚च्च निर्विक‚ल्प‚कं ज्ञानं य‚थागृहीत‚विष‚य‚मेवेद‚{\tiny $_{lb}$}‚मिहास्तीदं नास्तीति विधिप्र‚तिषेधं ज‚न‚य‚त् प्र‚माण‚मिष्य‚ते । तेन ध‚र्म‚ध‚र्मिणोः‚{\tiny $_{lb}$}‚ स्व‚रूप‚निश्च‚यः स‚म्ब‚न्ध‚निश्च‚य\add{प्र‚त्य‚क्ष‚कृत}\edtext{}{\edlabel{pvsvt_12-1}\label{pvsvt_12-1}\lemma{य}\Bfootnote{In the margin. }} एव भ‚व‚तीति ॥
	{\color{gray}{\rmlatinfont\textsuperscript{§~\theparCount}}}
	\pend% ending standard par
      ‚{\tiny $_{lb}$}‚

	  
	  \pstart \leavevmode% starting standard par
	प‚क्ष‚ध‚र्म इत्युक्तं [।] \textbf{सूत्रे} प‚क्ष‚ध‚र्म‚श्च ध‚र्म‚ध‚र्मिस‚मुदायः । न च प्र‚तिवादि‚{\tiny $_{५}$}‚नं‚{\tiny $_{lb}$}‚ प्र‚ति स‚मुदाय‚ध‚र्म‚त्वं हेतोः सिद्ध‚न्तेन स स‚र्वो हेतुर‚सिद्धः स्यात् । सिद्धौ वानुमान‚स्य‚{\tiny $_{lb}$}‚ वैय‚र्थ्य‚मित्याह ।
	{\color{gray}{\rmlatinfont\textsuperscript{§~\theparCount}}}
	\pend% ending standard par
      ‚{\tiny $_{lb}$}‚

	  
	  \pstart \leavevmode% starting standard par
	\textbf{प‚क्षो ध‚र्मीति} । अव‚य‚वे स‚मुदायोप‚चारात् । एक‚देश‚त्वं च स‚मुदायोप‚चार‚नि‚{\tiny $_{lb}$}‚मित्त‚न्तेन न दृष्टान्त‚ध‚र्मी प‚क्ष उच्य‚ते ।
	{\color{gray}{\rmlatinfont\textsuperscript{§~\theparCount}}}
	\pend% ending standard par
      ‚{\tiny $_{lb}$}‚

	  
	  \pstart \leavevmode% starting standard par
	ई श्व र से नः प्राह । ध‚र्मिध \add{र्म्मो हेतुरित्येताव‚द् व‚क्त‚व्यं \textbf{प्र}}\edtext{}{\edlabel{pvsvt_12-1b}\label{pvsvt_12-1b}\lemma{र्मिध}\Bfootnote{In the margin. }}\textbf{योज‚नाभा‚{\tiny $_{lb}$}‚वाद‚नुप‚चार इति} । त‚दाह । \textbf{प्र‚योज‚ने}त्यादि । नेत्यादिना प्र‚{\tiny $_{६}$}‚तिषेध‚ति । न प्र‚यो‚{\tiny $_{lb}$}‚ज‚न‚स्याभावः । क‚थं । \textbf{स‚र्व‚ध‚र्मिध‚र्म‚प्र‚तिषेधार्थ‚त्वा}दुप‚चार‚स्येत्य‚पेक्ष्य‚ते । अस‚त्युप‚चारे‚{\tiny $_{lb}$}‚ ध‚र्मिध‚र्म इति निर्देशः कार्यः । त‚था च दृष्टान्त‚ध‚र्मिणोपि ध‚र्मो हेतुः स्यात् । उप‚चारे‚{\tiny $_{lb}$}‚ तु स‚र्व‚स्य ध‚र्मिणो ध‚र्मः प्र‚तिषिद्धो भ‚व‚ति ॥ क‚थ \add{मिति चेदाह । \textbf{त‚देक‚देश‚त्वादि}}\edtext{}{\edlabel{pvsvt_12-1c}\label{pvsvt_12-1c}\lemma{थ}\Bfootnote{In the margin. }} ।‚{\tiny $_{lb}$}‚ \leavevmode\ledsidenote{\textenglish{6a/PSVTa}} त‚था हि स‚मुदाय‚स्याव‚य‚वेषूप‚चारः ।‚{\tiny $_{७}$}‚ त‚देक‚देश‚त्वंनिब‚न्ध‚न‚त्वेन न स‚र्व‚त्रोप‚चारः ।‚{\tiny $_{lb}$}‚ साध्य‚ध‚र्मी च त‚देक‚देश‚त्वात् प‚क्ष‚स्याव‚य‚व‚त्वात् प‚क्षोप‚चार‚योग्य‚स्त‚स्य यो ध‚र्म‚{\tiny $_{lb}$}‚स्त‚त्प्र‚तिप‚त्त्य‚र्थ‚मुप‚चार‚क‚र‚णं । \textbf{त‚था चेत्युप‚चार‚योग्य‚ध‚र्मिप्र‚तिप‚त्तौ चाक्षुष‚त्वादि‚{\tiny $_{lb}$}‚प‚रिहारः} । आदिश‚ब्दात् काक‚स्य का\add{र्ष्ण्यादित्यादि ॥‚{\tiny $_{lb}$}‚ \leavevmode\ledsidenote{\textenglish{13/s}}न‚नु य‚दि चाक्षुष‚त्वादिति}\edtext{}{\edlabel{pvsvt_13-3}\label{pvsvt_13-3}\lemma{का}\Bfootnote{In the margin. }} च‚क्षुर्विज्ञान‚विष‚य‚त्वादिति हेत्व‚र्थ‚स्त‚दायं हेतु‚{\tiny $_{lb}$}‚र‚नैकान्तिक‚त्वात् त‚द‚ङ्श‚व्या‚{\tiny $_{१}$}‚प्तिव‚च‚नेनैव निर‚स्त इति किमुप‚चारेण । अथ‚{\tiny $_{lb}$}‚ च‚क्षुर्विज्ञान‚ज‚न‚क‚त्वादिति हेत्व‚र्थ‚स्त‚दापि त‚ज्ज‚न‚क‚त्वं स‚त्त्व‚मेव [।] त‚च्च श‚ब्दे‚{\tiny $_{lb}$}‚प्य‚स्तीति न दृष्टान्त‚ध‚र्मिध‚र्म एवाय‚मिति क‚श्चित् ॥
	{\color{gray}{\rmlatinfont\textsuperscript{§~\theparCount}}}
	\pend% ending standard par
      ‚{\tiny $_{lb}$}‚

	  
	  \pstart \leavevmode% starting standard par
	त‚द‚युक्तं । च‚क्षुर्विज्ञान‚ज‚न‚क‚त्वं हि स‚त्त्व‚विशेषः स च घ‚टादीनामेव ध‚र्मो न‚{\tiny $_{lb}$}‚ श\add{ब्द‚स्यात‚स्त‚न्निवृत्य‚र्थ‚मुप‚चार}\edtext{}{\edlabel{pvsvt_13-3b}\label{pvsvt_13-3b}\lemma{श}\Bfootnote{In the margin. }}क‚र‚णं । न त्व‚स‚त्य‚प्युप‚चारे व्याप्त‚स्य लिंग‚त्वं‚{\tiny $_{lb}$}‚ [।] न च श‚ब्दानित्य‚त्त्वेन चाक्षुष‚त्वं‚{\tiny $_{२}$}‚ व्याप्त‚न्त‚त्क‚थ‚म‚स्य लिङ्ग‚त्वं ॥
	{\color{gray}{\rmlatinfont\textsuperscript{§~\theparCount}}}
	\pend% ending standard par
      ‚{\tiny $_{lb}$}‚

	  
	  \pstart \leavevmode% starting standard par
	नैष दोषः । अनित्य‚त्व‚मात्रेण ह्य‚स्य व्याप्त‚त्व‚न्त‚च्च श‚ब्देप्य‚स्तीति क‚थ‚म‚{\tiny $_{lb}$}‚हेतुत्वं स्यात् ॥
	{\color{gray}{\rmlatinfont\textsuperscript{§~\theparCount}}}
	\pend% ending standard par
      ‚{\tiny $_{lb}$}‚

	  
	  \pstart \leavevmode% starting standard par
	\textbf{ध‚र्म‚व‚च‚नेने}त्यादि । ध‚र्म‚स्त‚दंशेन व्याप्त इति केव‚लेन \textbf{ध‚र्म‚व‚च‚नेना}पि ध‚र्मि‚{\tiny $_{lb}$}‚प‚र‚त‚न्त्र‚त्वाद् ध‚र्म‚स्याव‚श्य‚म‚सौ ध‚र्मिण‚माक्षिप‚ति [।] तेन \textbf{ध‚र्मि}ण \textbf{आश्र‚य}ण‚माश्र‚{\tiny $_{lb}$}‚\add{यः \textbf{प‚रिग्र}ह‚स्त}\edtext{}{\edlabel{pvsvt_13-3c}\label{pvsvt_13-3c}\lemma{माश्र}\Bfootnote{In the margin. }}स्य \textbf{सिद्धौ} स‚त्यां य‚देत‚द्ध‚र्मिध‚र्म इत्य‚त्र ध‚र्मिग्र‚ह‚ण‚न्त‚स्य साम‚र्थ्यात्‚{\tiny $_{lb}$}‚ प्र‚त्यास‚त्तिरिह‚{\tiny $_{३}$}‚ विव‚क्षितेति ग‚म्य‚ते [।] व्याप्तिविव‚क्षायां ध‚र्मिग्र‚ह‚ण‚म‚न‚र्थ‚कं‚{\tiny $_{lb}$}‚ स्यात् । त‚स्माद् \textbf{ध‚र्मिव‚च‚न‚साम‚र्थ्यात् प्र‚त्यास‚त्तिः} सिद्धा [।] प्र‚त्यास‚त्तिश्च‚{\tiny $_{lb}$}‚ साध्य‚ध‚र्मिण एव त‚त्र प्र‚थ‚मं हेतूप‚द‚र्श‚नात् । त‚या प्र‚त्यास‚त्त्या साध्य‚ध‚र्मिप‚रिग्र‚हो‚{\tiny $_{lb}$}‚ भ‚विष्य‚ति । न प्र‚त्यास‚त्तेः साध्य‚ध‚र्मिप‚रिग्र‚हः कुतोः [।] दृष्टान्त‚ध‚र्मिणोपि न‚{\tiny $_{lb}$}‚ केव‚लं साध्य‚ध‚र्मिणः प्र‚त्यास‚त्तेः । क‚दाचिद् व्याप्तिद‚र्श‚न‚पूर्व‚के प्र‚योगे दृ‚{\tiny $_{४}$}‚ष्टान्त‚{\tiny $_{lb}$}‚ध‚र्मिणोपि प्र‚थ‚मं हेतुस‚द्भावोप‚द‚र्श‚नात् ॥
	{\color{gray}{\rmlatinfont\textsuperscript{§~\theparCount}}}
	\pend% ending standard par
      ‚{\tiny $_{lb}$}‚

	  
	  \pstart \leavevmode% starting standard par
	य‚दि न प्र‚त्यास‚त्तेः साध्य‚ध‚र्मिसिद्धिः पारिशेष्यात्त‚र्हि भ‚विष्य‚ति । य‚त‚स्त\textbf{दं‚{\tiny $_{lb}$}‚श‚व्याप्त्या} हेतुभूत‚या \textbf{दृष्टान्त‚ध‚र्मिणि} ध‚र्म‚स्य \textbf{स‚त्त्व‚सिद्धिः} । न हि दृष्टान्त‚म‚न्त‚रेण‚{\tiny $_{lb}$}‚ हेतोः साध्येन व्याप्तिः प्र‚द‚र्श‚यितुं श‚क्य‚त इति म‚न्य‚ते । त‚तो ध‚र्मिग्र‚ह‚णाद् व्य‚ति‚{\tiny $_{lb}$}‚रिच्य‚मानात् \textbf{साध्य‚ध‚र्मिण एव प‚रिग्र‚हः} । त‚दंशेनेति च त‚च्छ‚ब्देन ध‚र्म‚व‚च‚{\tiny $_{५}$}‚नाक्षिप्तो‚{\tiny $_{lb}$}‚ ध‚र्मी स‚म्ब‚ध्य‚त इति त‚त्स‚म्ब‚न्ध‚नार्थ‚म‚पि ध‚र्मिग्र‚ह‚णं नाश‚ङ्क‚नीयं ॥
	{\color{gray}{\rmlatinfont\textsuperscript{§~\theparCount}}}
	\pend% ending standard par
      ‚{\tiny $_{lb}$}‚

	  
	  \pstart \leavevmode% starting standard par
	य‚त्र प्र‚योज‚नान‚न्त‚रं न स‚म्भ‚व‚ति स पारिशेष्य‚स्य विष‚यो । ध‚र्मिव‚च‚न‚स्य त्व‚न्य‚{\tiny $_{lb}$}‚  ‚{\tiny $_{lb}$}‚ ‚{\tiny $_{lb}$}‚ \leavevmode\ledsidenote{\textenglish{14/s}}द‚पि प्र‚योज‚नं स‚म्भाव्य‚त इति म‚न्य‚मानः सिद्धान्त‚वाद्याह । सिद्धे त‚द‚ङ्श‚व्याप्त्या‚{\tiny $_{lb}$}‚ दृष्टान्त‚ध‚र्मिणि स‚त्त्वे पुन‚र्द्ध‚र्मिणो \textbf{व‚च‚न}न्दृष्टान्त‚ध‚र्मिण एव यो ध‚र्मः स हेतुरिति‚{\tiny $_{lb}$}‚ \textbf{निय‚मार्थ‚माशंक्य‚ते} । त‚त‚श्च चाक्षुष‚त्वा‚{\tiny $_{६}$}‚द‚य एव हेत‚वः स्युर्न कृत‚क‚त्वाद‚य इत्य‚{\tiny $_{lb}$}‚निष्ट‚मेव स्यात् । त‚स्मादुप‚चारः क‚र्त्त‚व्य इति ।
	{\color{gray}{\rmlatinfont\textsuperscript{§~\theparCount}}}
	\pend% ending standard par
      ‚{\tiny $_{lb}$}‚

	  
	  \pstart \leavevmode% starting standard par
	किं पुनः क्व‚चित् त र्क्क शा स्त्रे दृष्टं निय‚मार्थ‚म्व‚च‚न‚मित्य‚त आह । \textbf{स‚जातीय‚{\tiny $_{lb}$}‚ एवे}त्यादि । त‚त्र यः स‚न् स‚जातीये द्वेधा चासंस्त‚द‚त्य‚ये स हेतुरित्य‚त्रा चा र्यी ये‚{\tiny $_{lb}$}‚ हेतुल‚क्ष‚णे । \textbf{स‚जातीय एव स‚त्त्व‚मि}त्य‚व‚धार‚णेन सिद्धेपि \textbf{विजातीयाद्} विप‚क्षाद्धेतो‚{\tiny $_{lb}$}‚\leavevmode\ledsidenote{\textenglish{6b/PSVTa}} \textbf{र्व्य‚तिरेके} य‚देत‚{\tiny $_{७}$}‚द‚संस्त‚द‚त्य‚य इति \textbf{साध्याभावेऽस‚त्त्व‚व‚च‚नं} त‚न्निय‚मार्थ‚मा चा र्ये ण‚{\tiny $_{lb}$}‚ व्याख्यात‚म‚स‚त्येव नास्तिता य‚था स्यान्नान्य‚त्र न विरुद्ध इति । त‚थेहापि ध‚र्मिव‚च‚{\tiny $_{lb}$}‚न‚म्भाव‚निय‚मार्थ‚माशंक्य‚ते ॥
	{\color{gray}{\rmlatinfont\textsuperscript{§~\theparCount}}}
	\pend% ending standard par
      ‚{\tiny $_{lb}$}‚

	  
	  \pstart \leavevmode% starting standard par
	न‚न्व‚प‚क्ष‚ध‚र्म‚स्याहेतुत्वान्न निय‚मार्था श‚ङ्का । य‚तो व्याप्त‚स्य हेतुत्वं न चान्य‚{\tiny $_{lb}$}‚ध‚र्मिस्थेन साध्य‚ध‚र्मेणान्य‚ध‚र्मिस्थः साध‚न‚ध‚र्मो व्याप्त‚स्त‚स्मात्त‚दंश‚व्याप्त‚हेतु‚{\tiny $_{lb}$}‚व‚च‚न‚साम‚र्थ्यादेव साध्य‚ध‚र्मिप‚रिग्र‚हो‚{\tiny $_{१}$}‚ भ‚विष्य‚तीत्य‚त आह । \textbf{साम‚र्थ्यादित्यादि} ।‚{\tiny $_{lb}$}‚ अन‚न्त‚रोदितात् \textbf{साम‚र्थ्याद‚र्थ}स्य साध्य‚ध‚र्मिप‚रिग्र‚ह‚ल‚क्ष‚ण‚स्य भ‚व‚ति प्र‚तीतिः‚{\tiny $_{lb}$}‚ प‚टुधियां श्रोतॄणां किन्त्व‚श‚ब्द‚क‚म‚र्थं स्व‚य‚म‚नुस‚र‚तां \textbf{प्र‚तिप‚त्तिगौर‚वं} स्यात् । अत‚{\tiny $_{lb}$}‚ उप‚चार‚मात्रात् स्व‚य‚म‚श‚ब्द‚कार्थाभ्यूह‚र‚हिताद् ध‚र्मिध‚र्म इत्य‚नेन \textbf{प‚क्ष‚ध‚र्म} इति‚{\tiny $_{lb}$}‚ स‚मान‚निर्देशात् प्र‚तिप‚त्तिगौर‚वं च \textbf{प‚रिहृत}म्भ‚व‚ति । च श‚ब्देनैत‚दाह । ये प‚रो‚{\tiny $_{lb}$}‚प‚दे‚{\tiny $_{२}$}‚श‚माकांक्ष‚न्ति तैर‚य‚म‚र्थो ल‚क्ष‚ण‚व‚च‚नाद् बोद्ध‚व्य इति ।
	{\color{gray}{\rmlatinfont\textsuperscript{§~\theparCount}}}
	\pend% ending standard par
      ‚{\tiny $_{lb}$}‚

	  
	  \pstart \leavevmode% starting standard par
	य‚थाल‚क्ष‚णं प्र‚तीतेर‚प‚क्ष‚ध‚र्मो न हेतुरिति कुत इय‚माश‚ङ्का । त‚त‚स्तेषां ल‚क्ष‚{\tiny $_{lb}$}‚णानुसारिणां निय‚माशंकाप‚रिहारार्थ‚ञ्चोप‚चार‚क‚र‚ण‚मिति ॥ इह व्य‚व‚च्छेद‚फ‚ल‚त्वा‚{\tiny $_{lb}$}‚च्छ‚ब्द‚प्र‚योग‚स्याव‚श्य‚मेवाव‚धार‚यित‚व्यं [।] ष‚ष्ठीस‚मासाच्च प‚क्ष‚ध‚र्म इति नान्य‚{\tiny $_{lb}$}‚स्स‚मास‚स्स‚म्भ‚व‚ति । त‚था च प‚क्ष‚स्यैव ध‚र्म्म इत्येव‚म‚व‚धार‚णात् त‚दंश‚व्या‚{\tiny $_{३}$}‚प्ति‚{\tiny $_{lb}$}‚र्विरुध्य‚त इति विरुद्ध‚ल‚क्ष‚ण‚तामुद्भाव‚य‚न्नाह ।
	{\color{gray}{\rmlatinfont\textsuperscript{§~\theparCount}}}
	\pend% ending standard par
      ‚{\tiny $_{lb}$}‚

	  
	  \pstart \leavevmode% starting standard par
	\textbf{प‚क्ष‚स्य ध‚र्म‚त्वे त}म्प‚क्षं \textbf{विशेष‚ण}म‚न्य‚तो व्य‚व‚च्छेद\textbf{म‚पेक्ष‚त} इति । त‚द्वि‚{\tiny $_{lb}$}‚शेष‚णापेक्ष‚स्य ध‚र्म‚स्यान्य‚त्र प‚क्षीकृताद‚न्य‚स्मिन् स‚प‚क्षेऽन‚नुवृत्तिः । त‚था हि यः‚{\tiny $_{lb}$}‚  \leavevmode\ledsidenote{\textenglish{15/s}}प‚क्षेण विशेष्य‚ते स प‚क्ष‚स्यैव भ‚व‚ति । य‚था देव‚द‚त्त‚स्य पुत्रः [।] त‚तो\textbf{न्य‚त्रान‚नु‚{\tiny $_{lb}$}‚वृत्तेर‚साधार‚ण‚ता} साधार‚ण‚ता न \textbf{स्यात्} । त‚द‚ङ्श‚व्याप्तिविरोध इति या‚{\tiny $_{४}$}‚व‚त् ।‚{\tiny $_{lb}$}‚ साधार‚ण‚त‚या त‚दंश‚व्याप्तिप्र‚तिपाद‚नात् । त‚तो य‚दि प‚क्ष‚ध‚र्मो न त‚दंश‚व्याप्तिर‚थ‚{\tiny $_{lb}$}‚ त‚दंश‚व्याप्तिर्न प‚क्ष‚ध‚र्म इति व्याह‚तं ल‚क्ष‚ण‚मिति ।
	{\color{gray}{\rmlatinfont\textsuperscript{§~\theparCount}}}
	\pend% ending standard par
      ‚{\tiny $_{lb}$}‚

	  
	  \pstart \leavevmode% starting standard par
	न‚नु य‚दि साध्य‚ध‚र्मिणि साध‚न‚ध‚र्म‚स्य साध्य‚व्याप्तिर्न गृहीता त‚दा हेतोर‚नै‚{\tiny $_{lb}$}‚कान्तिक‚त्व‚म‚थ गृहीता किं दृष्टान्ते हेतो \add{र‚न्व‚येन क‚थ‚ञ्च प‚क्ष‚ध}\edtext{}{\edlabel{pvsvt_15-1}\label{pvsvt_15-1}\lemma{हेतो}\Bfootnote{In the margin. }}र्म‚स्य त‚दंश‚{\tiny $_{lb}$}‚व्याप्तिर्विरुध्य‚ते साध्य‚ध‚र्मिण्य‚पि व्याप्तेः प्र‚तिप‚न्न‚त्वात् । स‚र्व‚प‚दार्थ‚स्य‚{\tiny $_{५}$}‚ क्ष‚णिक‚त्वे‚{\tiny $_{lb}$}‚ साध्ये स‚त्त्व‚ल‚क्ष‚ण‚स्य वा हेतोः को दृष्टान्तेन्व‚यः । त‚स्मात् स्व‚साध्य‚प्र‚तिब‚न्धा‚{\tiny $_{lb}$}‚द्धेतुस्तेन व्याप्तः सिध्य‚ति [।] स च विप‚र्य‚ये बाध‚क‚प्र‚माण‚वृत्त्या साध्य‚ध‚र्मिण्य‚पि‚{\tiny $_{lb}$}‚ सिध्य‚तीति न किंचिद‚न्य‚त्रान्व‚यापेक्ष‚या [।] त‚त्क‚थ‚मिद‚माशंकित‚म‚न्य‚त्रान‚न‚{\tiny $_{lb}$}‚वृत्तेर‚साधार‚ण‚तेति ।
	{\color{gray}{\rmlatinfont\textsuperscript{§~\theparCount}}}
	\pend% ending standard par
      ‚{\tiny $_{lb}$}‚

	  
	  \pstart \leavevmode% starting standard par
	\add{स‚त्त्यं य‚द्य‚पि साध्य‚ध‚र्मिणि हेतोः}\edtext{\textsuperscript{*}}{\edlabel{pvsvt_15-1b}\label{pvsvt_15-1b}\lemma{*}\Bfootnote{In the margin. }} साध्य‚व्याप्तिम‚न्त‚रेण नानुमान‚स्यो‚{\tiny $_{lb}$}‚त्थान‚न्त‚थापि दृष्टान्ते साध्य‚साध‚न‚योः‚{\tiny $_{६}$}‚ प्र‚तिब‚न्ध‚ग्राह‚क‚प्र‚माण‚म‚न्त‚रेण न साध्य‚{\tiny $_{lb}$}‚ध‚र्म्मिण्य‚पि व्याप्तिः सिध्य‚तीति त‚द‚र्थ‚मिद‚माशंकितं ।
	{\color{gray}{\rmlatinfont\textsuperscript{§~\theparCount}}}
	\pend% ending standard par
      ‚{\tiny $_{lb}$}‚

	  
	  \pstart \leavevmode% starting standard par
	य‚त्तूच्य‚ते कार्य‚हेत्व‚पेक्ष‚या स्व‚भाव‚हेतुविशेषापेक्ष‚यैत‚दाशंकितं त‚तु क्ष‚णिक‚{\tiny $_{lb}$}‚त्वानुमाने स‚त्त्वापेक्ष‚या । त‚स्य हि विप‚क्ष‚बाध‚क‚प्र‚माण‚वृत्त्यैव ग‚म‚क‚त्वा \add{दिति‚{\tiny $_{lb}$}‚ त‚देत‚दुत्त‚र‚त्र निरूप‚यिष्या}\edtext{}{\edlabel{pvsvt_15-1c}\label{pvsvt_15-1c}\lemma{त्वा}\Bfootnote{In the margin. }}मः । त‚स्मात् पूर्व‚गृहीत‚प्र‚तिब‚न्ध‚साध‚क‚प्र‚माण‚स्मृत‚ये‚{\tiny $_{lb}$}‚ हेतोर‚न्य‚त्र वृत्तिर‚पे‚{\tiny $_{७}$}‚क्ष‚णीया ।
	{\color{gray}{\rmlatinfont\textsuperscript{§~\theparCount}}}
	\pend% ending standard par
      \textsuperscript{\textenglish{7a/PSVTa}}‚{\tiny $_{lb}$}‚

	  
	  \pstart \leavevmode% starting standard par
	एत‚त्प‚रिह‚र‚ति \textbf{नेत्या}दिना [।] \textbf{न} अन्य‚त्रान‚नुवृत्तिः [।] कुतः । अयोगो‚{\tiny $_{lb}$}‚ स‚म्ब‚न्ध‚स्त‚द्व्य‚व‚च्छेदेन \textbf{विशेष‚णात्} प‚क्ष‚स्य । न ह्य‚न्य‚योग‚व्य‚व‚च्छेदेनैव विशेष‚{\tiny $_{lb}$}‚ण‚म्भ‚व‚ति किन्त्व\textbf{योग‚व्य‚व‚च्छेदे}नापि । य‚त्र ध‚र्मिणि ध‚र्म‚स्य स‚द्भावः स‚न्दिह्य‚ते‚{\tiny $_{lb}$}‚ त‚त्रायोग‚व्य‚व‚च्छेद‚स्य न्याय‚प्राप्त‚त्वात् । अत्र दृष्टान्तो \textbf{य‚था चैत्रो ध‚नुर्ध‚र इति}‚{\tiny $_{lb}$}‚ [।] चैत्रे हि ध‚नुर्ध‚र‚त्वं स‚न्दिह्य‚ते किम‚स्ति नास्तीति । त‚त‚श्चैत्रो ध‚नुर्द्ध‚{\tiny $_{१}$}‚र इत्युक्ते‚{\tiny $_{lb}$}‚ प‚क्षान्त‚र‚म‚ध‚नुर्द्ध‚र‚त्वं श्रोतुराकांक्षोप‚स्थापितं निराक‚रोत्य‚योग‚व्य‚व‚च्छेदोत्र न्याय‚{\tiny $_{lb}$}‚प्राप्तः ।
	{\color{gray}{\rmlatinfont\textsuperscript{§~\theparCount}}}
	\pend% ending standard par
      ‚{\tiny $_{lb}$}‚

	  
	  \pstart \leavevmode% starting standard par
	प‚राभिम‚त‚व्य‚व‚च्छेदं निराचिकीर्ष‚न्नाह । \textbf{नान्य‚योग‚व्य‚व‚च्छेदेन} विशेष‚णा‚{\tiny $_{lb}$}‚‚{\tiny $_{lb}$}‚ ‚{\tiny $_{lb}$}‚ \leavevmode\ledsidenote{\textenglish{16/s}}द‚न्य‚त्रान‚नुवृत्तेर‚साधार‚ण‚तेति स‚म्ब‚न्धः । अत्रापि दृष्टान्तो \textbf{य‚था} \add{\textbf{पार्थो ध‚नुर्द्ध‚र‚{\tiny $_{lb}$}‚ इति} सामान्य‚श‚ब्दोप्य‚यं}\edtext{\textsuperscript{*}}{\edlabel{pvsvt_16-1}\label{pvsvt_16-1}\lemma{*}\Bfootnote{In the margin. }} ध‚नुर्द्ध‚र‚श‚ब्दः प्र‚क‚र‚ण‚साम‚र्थ्यादिना प्र‚कृष्ट‚गुण‚वृत्तिरिह‚{\tiny $_{lb}$}‚ पा र्थे हि ध‚नु‚{\tiny $_{२}$}‚र्द्ध‚र‚त्वं सिद्ध‚मेवेति नायोगाश‚ङ्का । तादृश‚न्तु सातिश‚यं किम‚न्य‚त्रा‚{\tiny $_{lb}$}‚प्य‚स्ति नास्तीत्य‚न्य‚योग‚शंकायां श्रोतुर्य‚दा पार्थो ध‚नुर्द्ध‚र इत्युच्य‚ते त‚दा सातिश‚यः‚{\tiny $_{lb}$}‚ पार्थ एव ध‚नुर्द्ध‚रो नान्य इति प्र‚तीय‚ते । तेनात्रान्य‚योग‚व्य‚व‚च्छेदो न्याय‚प्राप्तः ।‚{\tiny $_{lb}$}‚ त‚था किं प‚क्षेस्त्य‚यं ध‚र्मो न \add{वेति संश‚ये प‚क्ष‚ध‚र्म}\edtext{}{\edlabel{pvsvt_16-1b}\label{pvsvt_16-1b}\lemma{न}\Bfootnote{In the margin. }} इत्युक्ते प‚क्ष‚स्य ध‚र्म एव नाध‚र्मः ।‚{\tiny $_{lb}$}‚ ध‚र्म‚श्चाश्रित‚त्वाद् विशेष‚ण‚न्तेनायोगो व्य‚व‚च्छिद्य‚{\tiny $_{३}$}‚ते नान्य‚योगः । त‚दंश‚व्या‚{\tiny $_{lb}$}‚प्त्यान्य‚योग‚स्य प्र‚तिपाद‚नेन दृष्टान्ते स‚न्देहाभावात् । \textbf{आक्षेप्स्याम} इति निर्देक्ष्याम‚{\tiny $_{lb}$}‚श्च‚तुर्थे प‚रिच्छेदे \href{http://sarit.indology.info/?cref=}{४ । १९०} ॥
	{\color{gray}{\rmlatinfont\textsuperscript{§~\theparCount}}}
	\pend% ending standard par
      ‚{\tiny $_{lb}$}‚

	  
	  \pstart \leavevmode% starting standard par
	त‚दंश‚स्त‚द्ध‚र्म इति त‚च्छ‚ब्देन प‚क्षः प‚रामृश्य‚ते न ध‚र्मः ? । ध‚र्म‚स्य ध‚र्मास‚म्भ‚वात् ।‚{\tiny $_{lb}$}‚ त‚स्य प‚क्ष‚स्यांश‚स्त‚स्यैव साध्यो ध‚र्मः ॥ एक‚देशे रूढोङ्श‚श‚ब्दः क‚थं ध‚र्मं प्र‚तिपाद‚{\tiny $_{lb}$}‚य‚तीति चेदाह । \textbf{व‚क्तुर‚भिप्राय‚व‚शा}दिति । न व‚स्तुब‚लेन श‚ब्दानां वाच‚क‚त्वं‚{\tiny $_{४}$}‚ किन्तु‚{\tiny $_{lb}$}‚ व‚क्तुर्विव‚क्षाव‚शान्न \textbf{त‚देक‚देश‚स्त‚दंश} इति प्र‚कृतेन स‚म्ब‚न्धः [।]
	{\color{gray}{\rmlatinfont\textsuperscript{§~\theparCount}}}
	\pend% ending standard par
      ‚{\tiny $_{lb}$}‚

	  
	  \pstart \leavevmode% starting standard par
	किं पुन‚रेव‚मिति चेदाह । \textbf{प‚क्ष‚श‚ब्देन स‚मुदायाव‚च‚नादिति} । य‚दि प‚क्ष‚श‚ब्देन‚{\tiny $_{lb}$}‚ स‚मुदायोभिहितः स्यात् त‚दा ध‚र्म‚ध‚र्मिस‚मुदायात्म‚क‚स्य प‚क्ष‚स्यैक‚देशो ध‚र्मात्म‚कोङ्शो‚{\tiny $_{lb}$}‚ भ‚व‚ति । उप‚च‚रितेन तु प‚क्ष‚श‚ब्देन ध‚र्म्येवाभिधीय‚ते [।] त‚स्य चैकात्म‚क‚स्य कुत‚{\tiny $_{lb}$}‚ एक‚देशः ।
	{\color{gray}{\rmlatinfont\textsuperscript{§~\theparCount}}}
	\pend% ending standard par
      ‚{\tiny $_{lb}$}‚

	  
	  \pstart \leavevmode% starting standard par
	\textbf{व्याप्तं प‚द‚म्व्याप्तिरि}त्यादिना व्याच‚{\tiny $_{५}$}‚ष्टे । त‚स्य प‚क्ष‚ध‚र्म‚स्य स‚तो व्याप्तिर्यों‚{\tiny $_{lb}$}‚ व्याप्नोति य‚श्च व्याप्य‚ते व्याप्य‚व्याप‚क‚ध‚र्म‚त‚या प्र‚तीतेः । य‚दा व्याप‚क‚ध‚र्म‚त‚या‚{\tiny $_{lb}$}‚ विव‚क्ष्य‚ते त‚दा व्याप‚क‚स्य ग‚म्य‚स्य भाव एवेति स‚म्ब‚न्धः । \textbf{त‚त्रे}ति स‚प्त‚म्य‚र्थ‚{\tiny $_{lb}$}‚प्र‚धान‚मेत‚न्नाधारार्थ‚प्र‚धानं ध‚र्माणां ध‚र्मान्त‚र‚त्वाभावात् । तेनाय‚म‚र्थः [।]‚{\tiny $_{lb}$}‚ य‚त्र ध‚र्मिणि व्याप्य‚म‚स्ति \textbf{त‚त्र} स‚र्व‚त्र \textbf{व्याप‚क‚स्य भाव एवे}ति व्याप‚क‚ध‚र्मो व्याप्तिः ।‚{\tiny $_{lb}$}‚ न त्वे‚{\tiny $_{६}$}‚व‚म‚व‚धार्य‚ते व्याप‚क‚स्यैव त‚त्र भाव इति । हेत्व‚भाव‚प्र‚संगात् । अव्याप‚क‚स्यापि‚{\tiny $_{lb}$}‚ मूर्त्त‚त्वादेस्त‚त्र भावात् । \textbf{नापि त‚त्रै}वेत्य‚व‚धार्य‚ते । प्र‚य‚त्नान‚न्त‚रीय‚क‚त्वादेर‚हेतु‚{\tiny $_{lb}$}‚त्वाप‚त्तेः । साधार‚ण‚श्च हेतुः स्यान्नित्य‚त्व‚स्य प्र‚मेयेष्वेव भावात् ॥ य‚दा तु व्याप्य‚{\tiny $_{lb}$}‚‚{\tiny $_{lb}$}‚ ‚{\tiny $_{lb}$}‚ \leavevmode\ledsidenote{\textenglish{17/s}}ध‚र्म‚ता व्याप्तेर्विव‚क्षिता त‚दा य‚त्र ध‚र्मिणि व्याप‚कोस्ति त‚त्रैव \textbf{व्याप्य‚स्य भावो}‚{\tiny $_{lb}$}‚ नान्य‚त्र । अत्रापि व्याप्य‚स्यैव ।‚{\tiny $_{७}$}‚ त‚त्र भाव इत्य‚व‚धार‚णं हेत्व‚भाव‚प्र‚स‚क्तेरेव \leavevmode\ledsidenote{\textenglish{7b/PSVTa}}‚{\tiny $_{lb}$}‚ नाश्रित‚म‚व्याप्य‚स्यापि त‚त्र भावात् । नापि व्याप्य‚स्य त‚त्र भाव एवेत्य‚व‚धार्य‚ते ।‚{\tiny $_{lb}$}‚ स‚प‚क्षैक‚देश‚वृत्तेर‚हेतुत्व‚प्राप्तेः । साधार‚ण‚स्य च हेतुत्वं स्यात् प्र‚मेय‚त्व‚स्य नित्ये‚{\tiny $_{lb}$}‚ष्व‚व‚श्य‚म्भावादिति । व्याप‚क‚स्य त‚त्र भाव इत्य‚नेन चान्व‚य आक्षिप्तो व्याप्य‚स्य‚{\tiny $_{lb}$}‚ वा त‚त्रैव भाव इत्य‚नेन व्य‚तिरेक आक्षिप्तः ।
	{\color{gray}{\rmlatinfont\textsuperscript{§~\theparCount}}}
	\pend% ending standard par
      ‚{\tiny $_{lb}$}‚

	  
	  \pstart \leavevmode% starting standard par
	य‚द्वा व्याप्तेर्व्याप्य‚व्याप‚क‚ध‚र्म्म‚स‚म्व‚र्ण्ण‚नं नि‚{\tiny $_{१}$}‚य‚तानिय‚त‚त्व‚ख्याप‚नार्थं । तेन‚{\tiny $_{lb}$}‚ व्याप्तो \textbf{हेतु}र्भ‚व‚ति न तु व्याप‚कोऽनिय‚त‚त्वात् ।
	{\color{gray}{\rmlatinfont\textsuperscript{§~\theparCount}}}
	\pend% ending standard par
      ‚{\tiny $_{lb}$}‚

	  
	  \pstart \leavevmode% starting standard par
	न‚नु यो ध‚र्मो व्याप्य‚म‚न्त‚रेण भ‚व‚ति स क‚थं व्याप‚को व्याप्यास‚म्ब‚न्धेनाव्या‚{\tiny $_{lb}$}‚प‚क‚त्वात् ।
	{\color{gray}{\rmlatinfont\textsuperscript{§~\theparCount}}}
	\pend% ending standard par
      ‚{\tiny $_{lb}$}‚

	  
	  \pstart \leavevmode% starting standard par
	स‚त्यं । केव‚ल‚न्ध‚र्म‚योः सामान्येन व्याप्य‚व्याप‚क‚भावो निश्चीय‚ते । य‚च्चा‚{\tiny $_{lb}$}‚नित्य‚त्व‚सामान्यं प्र‚य‚त्नान‚न्त‚रीय‚क‚त्व‚व्याप‚कं निश्चित‚न्त‚द‚प्र‚य‚त्नान‚न्त‚रीय‚केपि‚{\tiny $_{lb}$}‚ दृश्य‚त इति व्याप‚कोऽनिय‚त उच्य‚ते । अथ प्र‚य‚{\tiny $_{२}$}‚त्नान‚न्त‚रीय‚क‚स्व‚भाव‚मेवानि‚{\tiny $_{lb}$}‚त्य‚त्वं निश्चेतुम्पार्य‚ते त‚दान‚योः प‚र‚स्प‚रं व्याप्य‚त्व‚मिति व्याप्त एव हेतुर्भ‚व‚ति ।‚{\tiny $_{lb}$}‚ य‚दा च य‚त्र विप्र‚तिप‚त्तिस्त‚देव साध्य‚मित‚र‚त् साध‚न‚मिति न्याय एषः ॥
	{\color{gray}{\rmlatinfont\textsuperscript{§~\theparCount}}}
	\pend% ending standard par
      ‚{\tiny $_{lb}$}‚

	  
	  \pstart \leavevmode% starting standard par
	\hphantom{.}य‚दि त‚र्हि प‚क्ष‚ध‚र्म‚स्त‚दंशेन व्याप्त इत्येताव‚द्धेतुल‚क्ष‚णं त‚तः प‚क्ष‚ध‚र्म‚त्व‚{\tiny $_{lb}$}‚न्त‚द‚ङ्श‚व्याप्तिश्चेति द्विरूपो हेतुः स्याद‚न्य‚त्र च त्रिरूप उक्त‚स्त‚त्क‚थ‚न्न व्याघात‚{\tiny $_{lb}$}‚ इत्याह ।
	{\color{gray}{\rmlatinfont\textsuperscript{§~\theparCount}}}
	\pend% ending standard par
      ‚{\tiny $_{lb}$}‚

	  
	  \pstart \leavevmode% starting standard par
	\textbf{एतेन} त‚दं‚{\tiny $_{३}$}‚श‚व्याप्तिव‚च‚नेना\textbf{न्व‚य‚व्य‚तिरेका}वुक्तौ । अन्व‚य‚व्य‚तिरेक‚{\tiny $_{lb}$}‚रूप‚त्वाद् व्याप्तेरिति भावः । त‚था हि [।] य एव येनान्वितो य‚न्निवृत्तौ च निव‚र्त्त‚ते‚{\tiny $_{lb}$}‚ स एव तेन व्याप्त उच्य‚त इति त‚दात्म‚क‚त्वाद् व्याप्तेर्व्याप्तिव‚च‚नेनान्व‚य‚व्य‚तिरेका‚{\tiny $_{lb}$}‚भिधान‚न्त‚तो व्याप्तिव‚च‚नेन रूप‚द्व‚याभिधानान्न व्याघात इति । तौ च ज्ञाप‚क‚{\tiny $_{lb}$}‚हेत्व‚धिकारान्निश्चितौ । निश्च‚य‚श्च त‚योर्नैकेनेव प्र‚माणेनापि तु‚{\tiny $_{४}$}‚ \textbf{य‚थास्वं} य‚स्य‚{\tiny $_{lb}$}‚ य‚दात्मीयं प्र‚माणं निश्चाय‚क‚न्तेन । य‚स्य च य‚त्प्र‚माण‚न्त‚दुत्त‚र‚त्र व‚क्ष्य‚ते ॥
	{\color{gray}{\rmlatinfont\textsuperscript{§~\theparCount}}}
	\pend% ending standard par
      ‚{\tiny $_{lb}$}‚

	  
	  \pstart \leavevmode% starting standard par
	न‚नु भाव‚रूप‚त्वाल्लिङ्ग‚स्य क‚थं व्य‚तिरेकः [।] साध्याभावेऽभाव‚ल‚क्ष‚णोस्य‚{\tiny $_{lb}$}‚ रूप‚मिति चेत् । न । य एव हि साध्य एव लिङ्ग‚स्य भावः स एव साध्याभावे‚{\tiny $_{lb}$}‚ व्य‚तिरेकः । तेनान्व‚य‚व्य‚तिरेक \add{योर‚पि तादात्म्यं वि}\edtext{}{\edlabel{pvsvt_17-1}\label{pvsvt_17-1}\lemma{तिरेक}\Bfootnote{In the margin. }}क‚ल्प‚क‚ल्पित‚स्तु भेदः ।‚{\tiny $_{lb}$}‚ साध्याभावे लिंग‚स्य निर्वृत्तिध‚र्म‚क‚त्वं व्य‚तिरेक इ‚{\tiny $_{५}$}‚त्य‚प‚रे । य‚त‚श्च य‚त्र य‚त्र साध‚न‚{\tiny $_{lb}$}‚ध‚र्म्म‚स्त‚त्र त‚त्र साध्य‚ध‚र्म इत्येवं रूपो\textbf{न्व‚यः} । तेन य‚दुच्य‚ते कु मा रि ल भ ट्टे न ॥
	{\color{gray}{\rmlatinfont\textsuperscript{§~\theparCount}}}
	\pend% ending standard par
      ‚{\tiny $_{lb}$}‚‚{\tiny $_{lb}$}‚‚{\tiny $_{lb}$}‚\textsuperscript{\textenglish{18/s}}
	  \bigskip
	  \begingroup
	
	    
	    \stanza[\smallbreak]
	  {\normalfontlatin\large ``\qquad}यः स‚वित्रुद‚यो भावी न तेनाद्योद‚योन्वितः ।&‚{\tiny $_{lb}$}‚अथ चाद्योद‚यात् सोपि भ‚विता श्वोनुमीय‚ते ॥&‚{\tiny $_{lb}$}‚व्योम्नि दृष्टं च धूमाग्रं भूमौ ब‚ह्निः प्र‚तीय‚ते ।&‚{\tiny $_{lb}$}‚\add{धूमाग्र‚म‚ग्नेर‚न्वेति न च भूमौ प्र‚ति}\edtext{\textsuperscript{*}}{\edlabel{pvsvt_18-1}\label{pvsvt_18-1}\lemma{*}\Bfootnote{In the margin. }}ष्ठितः ॥&‚{\tiny $_{lb}$}‚एव‚न्न देश‚कालाभ्यां लिङ्गं लिङ्ग्य‚नुग‚च्छ‚ति ।&‚{\tiny $_{lb}$}‚त‚स्मान्नास्यान्व‚यो नाम‚{\tiny $_{६}$}‚ स‚म्ब‚न्धोङ्शः प्र‚तीय‚त इति [।]{\normalfontlatin\large\qquad{}"}\&[\smallbreak]
	  
	  
	  
	  \endgroup
	‚{\tiny $_{lb}$}‚

	  
	  \pstart \leavevmode% starting standard par
	त‚द‚पास्तं । य‚त‚श्च य‚थोप‚व‚र्ण्णितः साध्यान्व‚यो हेतुर्विद्य‚ते । तेनैत‚द‚पि‚{\tiny $_{lb}$}‚ प्र‚त्युक्तं ।
	{\color{gray}{\rmlatinfont\textsuperscript{§~\theparCount}}}
	\pend% ending standard par
      ‚{\tiny $_{lb}$}‚
	  \bigskip
	  \begingroup
	
	    
	    \stanza[\smallbreak]
	  {\normalfontlatin\large ``\qquad}प्र‚त्याख्येयैव‚मेवेह व्याप्तिस‚म्ब‚न्ध‚क‚ल्प‚ना ।&‚{\tiny $_{lb}$}‚यो हि नान्वीय‚ते येन स तेन व्याप्य‚ते कुत इति ।{\normalfontlatin\large\qquad{}"}\&[\smallbreak]
	  
	  
	  
	  \endgroup
	‚{\tiny $_{lb}$}‚

	  
	  \pstart \leavevmode% starting standard par
	अत एवेदृशीम्प‚र‚प‚रिक‚ल्पि\add{तां व्याप्तिं निराक‚र्त्तुमा चा र्यो \textbf{व्याप‚क‚स्य}}\edtext{}{\edlabel{pvsvt_18-1b}\label{pvsvt_18-1b}\lemma{ल्पि}\Bfootnote{In the margin. }} \leavevmode\ledsidenote{\textenglish{8a/PSVTa}} \textbf{त‚त्र भाव} इत्यादिना लौकिकीव्याप्तिन्द‚र्शित‚वान् । स‚म्ब‚न्ध‚ग्राह‚{\tiny $_{७}$}‚कं प्र‚माणं‚{\tiny $_{lb}$}‚ लिङ्ग‚स्य साध्याय‚त्त‚ताग्राह‚कं । य‚च्च त‚दाय‚त्त‚तां गृह्णाति त‚देवान्व‚य‚व्य‚ति‚{\tiny $_{lb}$}‚रेकात्मिकाया व्याप्तेर्ग्राह‚कं । साध्याय‚त्त‚ताया एव व्याप्तिरूप‚त्वात् । त‚स्माद्‚{\tiny $_{lb}$}‚ व्याप्तिग्राह‚कादेव प्र‚माणात् य‚त्र व्याप्य‚स‚म्भ‚व‚स्त‚त्र व्याप‚क‚भावो य‚त्र व्याप‚का‚{\tiny $_{lb}$}‚भाव‚स्त‚त्र व्याप्याभाव इत्य \add{भावेपि निश्च‚यो भ‚व‚ति । तेन य‚दुच्य‚ते}\edtext{}{\edlabel{pvsvt_18-1c}\label{pvsvt_18-1c}\lemma{इत्य}\Bfootnote{In the margin. }} कुमारिल भ ट्टेन ।\edtext{\textsuperscript{*}}{\edlabel{pvsvt_18-2}\label{pvsvt_18-2}\lemma{*}\Bfootnote{\href{http://sarit.indology.info/?cref=\%C5\%9Bv.114}{Ślokavārtika 114}.}}
	{\color{gray}{\rmlatinfont\textsuperscript{§~\theparCount}}}
	\pend% ending standard par
      ‚{\tiny $_{lb}$}‚
	  \bigskip
	  \begingroup
	
	    
	    \stanza[\smallbreak]
	  {\normalfontlatin\large ``\qquad}सामान्य‚विष‚य‚त्वाच्च न प्र‚त्य‚क्षेन्व‚य‚म्भ‚वेत् ।&‚{\tiny $_{lb}$}‚न चानुमीय‚ते पूर्व‚म‚विज्ञा‚{\tiny $_{१}$}‚तान्व‚यान्त‚रात् ॥&‚{\tiny $_{lb}$}‚अथान्व‚येनुमानं स्याद‚न्व‚यान्त‚र‚व‚र्ज्जितं ।&‚{\tiny $_{lb}$}‚सिद्धे त‚द‚न‚पेक्षेस्मिन्न‚न्य‚त्राप्य‚न्व‚येन किं ॥&‚{\tiny $_{lb}$}‚व्य‚तिरेकोपि लिङ्ग‚स्य विप‚क्षान्नैव ल‚भ्य‚ते ।&‚{\tiny $_{lb}$}‚अभावे स न ग‚म्येत कृत‚य‚त्नैर‚बोध‚नात् ॥&‚{\tiny $_{lb}$}‚याव‚त्स‚र्व‚विप‚क्षाणां प‚र्य‚न्तो नाव‚धारितः ।&‚{\tiny $_{lb}$}‚ताव‚द्धेतोर‚वृत्तित्वं क‚स्त‚स्माज्ज्ञातुम‚र्ह‚तीति{\normalfontlatin\large\qquad{}"}\&[\smallbreak]
	  
	  
	  
	  \endgroup
	‚{\tiny $_{lb}$}‚

	  
	  \pstart \leavevmode% starting standard par
	त‚द‚प्य‚पास्तं ।
	{\color{gray}{\rmlatinfont\textsuperscript{§~\theparCount}}}
	\pend% ending standard par
      ‚{\tiny $_{lb}$}‚\textsuperscript{\textenglish{19/s}}

	  
	  \pstart \leavevmode% starting standard par
	\textbf{प‚क्ष} \textbf{\edtext{\textsuperscript{*}}{\edlabel{pvsvt_19-1}\label{pvsvt_19-1}\lemma{*}\Bfootnote{In the margin. }}ध‚र्म‚श्च} किं \textbf{य‚थास्वं प्र‚माणेन निश्चित} उक्तो वेदित‚व्य इति स‚म्ब‚न्धः‚{\tiny $_{lb}$}‚ प‚क्ष‚ध‚र्म्म‚व‚{\tiny $_{२}$}‚च‚नेनैव । एवं च त्रैरूप्य‚मेवोक्तं लिंग‚स्येत्य‚विरोधः ।
	{\color{gray}{\rmlatinfont\textsuperscript{§~\theparCount}}}
	\pend% ending standard par
      ‚{\tiny $_{lb}$}‚

	  
	  \pstart \leavevmode% starting standard par
	\hphantom{.}तेन य‚दुच्य‚ते ऽवि द्ध क र्ण्णेन । स‚त्य‚म‚नुमान‚मिष्य‚त एवास्माभिः प्र‚माणं‚{\tiny $_{lb}$}‚ लोक‚प्र‚तीत‚त्वात् केव‚लं लिंग‚ल‚क्ष‚ण‚म‚युक्त मिति त‚द‚पास्तं । त्रैरूप्य‚स्यापि लिङ्ग‚{\tiny $_{lb}$}‚ल‚क्ष‚ण‚स्य लोक‚प्र‚तीत‚त्वात् धूमादाविव ।
	{\color{gray}{\rmlatinfont\textsuperscript{§~\theparCount}}}
	\pend% ending standard par
      ‚{\tiny $_{lb}$}‚

	  
	  \pstart \leavevmode% starting standard par
	न‚नु क‚थं य‚थास्वं प्र‚मा \add{णेन प‚क्ष‚ध‚र्म‚निश्च‚यः}\edtext{}{\edlabel{pvsvt_19-1b}\label{pvsvt_19-1b}\lemma{मा}\Bfootnote{In the margin. }}सामान्य‚स्य लिङ्ग‚त्वात् [।]‚{\tiny $_{lb}$}‚ त‚स्य च प्र‚त्य‚क्षेण स्व‚ल‚क्ष‚ण‚विष‚य‚त्वेनाग्र‚ह‚णात् । अगृही‚{\tiny $_{३}$}‚त‚स्य चालिंग‚त्वात् ।‚{\tiny $_{lb}$}‚ गृहीत‚स्य च स्व‚ल‚क्ष‚ण‚स्यान‚न्व‚येनालिंग‚त्वात् । नाप्य‚नुमानेन सामान्य‚ग्र‚ह‚ण‚न्त‚{\tiny $_{lb}$}‚ल्लिङ्ग‚स्यापि सामान्य‚रूप‚त्वेन प्र‚त्य‚क्षेणाग्र‚ह‚णाद‚नुमानेन ग्र‚ह‚णेऽन‚व‚स्थाप्र‚स‚ङ्गात् ॥
	{\color{gray}{\rmlatinfont\textsuperscript{§~\theparCount}}}
	\pend% ending standard par
      ‚{\tiny $_{lb}$}‚

	  
	  \pstart \leavevmode% starting standard par
	त‚दाह ।
	{\color{gray}{\rmlatinfont\textsuperscript{§~\theparCount}}}
	\pend% ending standard par
      ‚{\tiny $_{lb}$}‚
	  \bigskip
	  \begingroup
	
	    
	    \stanza[\smallbreak]
	  {\normalfontlatin\large ``\qquad}लिङ्ग‚लिङ्ग्य‚नुमानानामान‚न्त्यादेक‚लिङ्गिनि ।&‚{\tiny $_{lb}$}‚ग‚तिर्युग‚स‚ह‚स्रेपु ब‚हुष्व‚पि न \add{विद्य‚त}\edtext{}{\edlabel{pvsvt_19-1c}\label{pvsvt_19-1c}\lemma{न}\Bfootnote{In the margin. }} इति\edtext{}{\edlabel{pvsvt_19-2}\label{pvsvt_19-2}\lemma{इति}\Bfootnote{Ślokavārtika. 153:3 }} ।{\normalfontlatin\large\qquad{}"}\&[\smallbreak]
	  
	  
	  
	  \endgroup
	‚{\tiny $_{lb}$}‚

	  
	  \pstart \leavevmode% starting standard par
	अथ कार्य‚स्व‚भाव‚विक‚ल्प‚प्र‚तिभासि सामान्यं कार्यादिद‚र्श‚नाश्र‚य‚त‚या त‚द‚ध्य‚व‚{\tiny $_{४}$}‚‚{\tiny $_{lb}$}‚सायाच्च कार्यादिहेतुरित्युच्य‚ते ।
	{\color{gray}{\rmlatinfont\textsuperscript{§~\theparCount}}}
	\pend% ending standard par
      ‚{\tiny $_{lb}$}‚

	  
	  \pstart \leavevmode% starting standard par
	त‚द‚युक्तं [।] त‚स्यापि विक‚ल्पाव्य‚तिरिक्त‚त्वाद् विक‚ल्प‚व‚द् अन्य‚त्र विक‚ल्पा‚{\tiny $_{lb}$}‚न्त‚रेऽन‚नुग‚मात् क‚थं सामान्यं लिंगं । त‚स्माद् विजातीय‚व्यावृत्त‚मेव धूमादे रूपं‚{\tiny $_{lb}$}‚ ज्ञाप‚क‚हेत्व‚धिकारात् प्र‚त्य‚क्ष‚निश्चित‚म्विशेषान‚व‚धार‚णेन सामान्य‚ल‚क्ष‚णं लिंग‚{\tiny $_{lb}$}‚मुच्य‚ते । न तु विजातीय‚व्यावृत्तिर्विक‚ल्पाकारो वाऽव‚स्तुत्वात् । तेनाय‚म‚र्थः ।‚{\tiny $_{lb}$}‚ प्र‚त्य‚क्ष‚पृष्ठ‚भाविना निश्च‚{\tiny $_{५}$}‚येनाधूम‚व्यावृत्त‚रूपाव‚धार‚णेन धूमादिस्व‚ल‚क्ष‚ण‚मिदं‚{\tiny $_{lb}$}‚ प्र‚तिभास‚मानं क‚दाचित्तार्ण्ण‚म्पार्ण्ण‚म‚न्य‚द्वेति विशेषान‚व‚धार‚णेन चानेक‚स्व‚ल‚क्ष‚ण‚{\tiny $_{lb}$}‚रूपं सामान्य‚ल‚क्ष‚णं लिङ्गं प्र‚त्य‚क्ष‚विष‚ये व्य‚व‚स्थाप्य‚ते । य‚था च लिङ्ग‚स्य विशेषा‚{\tiny $_{lb}$}‚न‚व‚धार‚णेन सामान्य‚रूप‚त्व‚न्त‚था साध्य‚स्यापि । त‚दाह ।
	{\color{gray}{\rmlatinfont\textsuperscript{§~\theparCount}}}
	\pend% ending standard par
      ‚{\tiny $_{lb}$}‚
	  \bigskip
	  \begingroup
	
	    
	    \stanza[\smallbreak]
	  {\normalfontlatin\large ``\qquad}अत‚द्रूप‚प‚रावृत्त‚व‚स्तुमात्र‚प्र‚साध‚नात् [।]&‚{\tiny $_{lb}$}‚सामान्य‚विष‚यं प्रोक्तं लिङ्गं भेदाप्र‚तिष्ठितेरिति ।{\normalfontlatin\large\qquad{}"}\&[\smallbreak]
	  
	  
	  
	  \endgroup
	‚{\tiny $_{lb}$}‚

	  
	  \pstart \leavevmode% starting standard par
	त‚थाभूत‚स्य‚{\tiny $_{६}$}‚ च सामान्य‚ल‚क्ष‚ण‚स्य लिङ्ग‚स्य साध्य‚कार्य‚त्वं साध्य‚स्व‚भाव‚त्वं च‚{\tiny $_{lb}$}‚ व‚स्तुत्वाद‚विरुद्धं । त‚च्च लिङ्गं प्र‚त्य‚क्षादिनिश्चित‚मिति स‚र्वं सुस्थं ॥
	{\color{gray}{\rmlatinfont\textsuperscript{§~\theparCount}}}
	\pend% ending standard par
      ‚{\tiny $_{lb}$}‚
	    
	    \stanza[\smallbreak]
	  \textbf{त एत} इत्यादिना त्रिधैव स इत्येत‚द् व्याच‚ष्टे ।\&[\smallbreak]
	  
	  
	  ‚{\tiny $_{lb}$}‚\textsuperscript{\textenglish{20/s}}

	  
	  \pstart \leavevmode% starting standard par
	त एत इति । प‚क्ष‚ध‚र्म‚त्वेन य‚थोक्त‚या च व्याप्त्या युक्ताः \textbf{कार्य‚स्व‚भावानुप‚ल‚ब्ध‚यो‚{\tiny $_{lb}$}‚ ल‚क्ष‚णं} स्व‚भावो येषान्ते त‚थोक्ताः । \textbf{धूमादिति} कार्य‚हेतोराख्यानं । अग्निर‚त्रेति‚{\tiny $_{lb}$}‚ साध्य‚फ‚ल‚स्य । न त्व‚य‚म्प‚क्ष‚प्र‚योगः‚{\tiny $_{७}$}‚ ॥
	{\color{gray}{\rmlatinfont\textsuperscript{§~\theparCount}}}
	\pend% ending standard par
      ‚{\tiny $_{lb}$}‚

	  
	  \pstart \leavevmode% starting standard par
	\leavevmode\ledsidenote{\textenglish{8b/PSVTa}} न‚नु यः प्र‚देशोग्निस‚म्ब‚न्धी सोप्र‚त्य‚क्षः । य‚श्च प्र‚त्य‚क्षो न‚भोभाग‚रूप आलोका‚{\tiny $_{lb}$}‚द्यात्मा धूम‚व‚त्त‚या दृश्य‚मानो न सोग्निमान‚तः क‚थं प्र‚देशे धूम‚स्य प्र‚त्य‚क्ष‚तः सिद्धिस्त‚{\tiny $_{lb}$}‚स्माद् धूम एव ध‚र्मी युक्तः ।
	{\color{gray}{\rmlatinfont\textsuperscript{§~\theparCount}}}
	\pend% ending standard par
      ‚{\tiny $_{lb}$}‚

	  
	  \pstart \leavevmode% starting standard par
	\hphantom{.}साग्निर‚यं धूमो धूम‚त्वादित्येवं साध्य‚साध‚न‚भाव इत्यु द्यो त क र\edtext{}{\edlabel{pvsvt_20-1}\label{pvsvt_20-1}\lemma{र}\Bfootnote{Cf. \href{http://sarit.indology.info/?cref=nv.1.1.5}{Nyāyavārtika 1:1:5 }.}}ः । त‚स्यापि‚{\tiny $_{lb}$}‚ साग्नेर्धूमाव‚य‚व‚स्याप्र‚त्य‚क्ष‚त्वात् । प‚रिदृश्य‚मान‚स्य चोर्द्ध्व‚भाग‚व‚र्तिनोग्निना‚{\tiny $_{lb}$}‚ स‚हावृत्तेः क‚थं धूम‚सामान्य‚स्य साध्य‚ध‚{\tiny $_{१}$}‚र्मिणि प्र‚त्य‚क्ष‚तो निश्च‚यः । धूमाव‚य‚वी‚{\tiny $_{lb}$}‚ प्र‚त्य‚क्ष इति चेत् । न [।] अव‚य‚व‚व्य‚तिरेकेण त‚स्याभावात । लोकाध्य‚व‚सायः त‚{\tiny $_{lb}$}‚स्यैक‚त्वे वा प्र‚देश‚स्यापि ताव‚तः क‚ल्पित‚मेकात्म‚क‚त्वं न वार्य‚ते । प्र‚देश एव च‚{\tiny $_{lb}$}‚ लोकोग्निं प्र‚तिप‚द्य‚ते न धूमे ध‚र्मिणि । तेन य‚द्य‚ग्नेर‚नुमान‚मिष्य‚ते प्र‚देश एव‚{\tiny $_{lb}$}‚ ध‚र्मिण्य‚नुमान‚म‚स्त्वित्येव‚म्प‚र‚मेत‚त् ।
	{\color{gray}{\rmlatinfont\textsuperscript{§~\theparCount}}}
	\pend% ending standard par
      ‚{\tiny $_{lb}$}‚

	  
	  \pstart \leavevmode% starting standard par
	न त्व‚त्र पूर्वोक्तो दोष‚प‚रिहार इत्येके । य‚द्वा दृश्य‚मा‚{\tiny $_{२}$}‚नः प्र‚देशो ध‚र्मी‚{\tiny $_{lb}$}‚ अध‚स्ताद‚ग्निमानित्येताव‚त् साध्य‚ध‚र्मो नाग्निमात्रं । ईदृग्विधेन च साध्य‚ध‚र्मेण‚{\tiny $_{lb}$}‚ पूर्व‚मेव व्याप्तिः प्र‚तिप‚न्ना । धूम‚श्चात्र प्र‚त्य‚क्ष‚सिद्ध इति क‚थं नानुमानं । य‚तु‚{\tiny $_{lb}$}‚ देशाद्य‚पेक्ष‚या कार्य‚हेतोर्ग‚म‚क‚त्व‚म‚त्रोच्य‚ते । त‚द‚स‚ङ्ग‚त‚मेव देशादेर्विशेष‚ण‚स्या‚{\tiny $_{lb}$}‚सिद्ध‚त्वात् । धूम‚मात्र‚द‚र्श‚नादेवास्य साध्य‚स्य सिद्ध‚त्वाच्च । य‚द्वा प्र‚देशेऽग्निं‚{\tiny $_{lb}$}‚ दृष्ट्वा किंशुकादिरूपेण स‚{\tiny $_{३}$}‚न्देहं य‚दा धूम‚द‚र्श‚नान्निव‚र्त्त‚य‚ति । त‚दैत‚दुदाह‚र‚णं द्र‚ष्ट‚व्यं ।‚{\tiny $_{lb}$}‚ त‚दा हि प्र‚त्य‚क्षेण ध‚र्मी साध‚न‚ध‚र्म‚श्च सिद्धो भ‚व‚ति ।
	{\color{gray}{\rmlatinfont\textsuperscript{§~\theparCount}}}
	\pend% ending standard par
      ‚{\tiny $_{lb}$}‚

	  
	  \pstart \leavevmode% starting standard par
	य‚त्तूच्य‚ते [।] प्र‚त्य‚क्षाप्र‚त्य‚क्ष‚रूप एव ध‚र्मिण्य‚नुमान‚मिति । त‚द‚युक्तं ।‚{\tiny $_{lb}$}‚ प्र‚त्य‚क्षाङ्शे य‚द्य‚पि हेतुः सिद्ध‚स्त‚थापि न त‚त्र साध्य‚ध‚र्मानुमानं प्र‚त्य‚क्ष‚बाधित‚त्वात् ।‚{\tiny $_{lb}$}‚ प‚रोक्षाङ्शे तु स्याद‚नुमानं केव‚ल‚न्त‚त्र हेतुर‚सिद्धः । न च प्र‚त्य‚क्षं प्र‚त्य‚क्ष‚रूप‚स्य धंर्मि‚{\tiny $_{४}$}‚णो‚{\tiny $_{lb}$}‚ ध‚र्मः प्र‚त्य‚क्ष‚सिद्धोऽप्र‚त्य‚क्षेङ्शे प्र‚त्य‚क्षाव्यापारात् । य‚दा वा श्र‚व‚ण‚ग्राह्ये श‚ब्देऽनित्य‚{\tiny $_{lb}$}‚त्वानुमान‚न्त‚दा क‚थं ध‚र्मिणः प्र‚त्य‚क्षाप्र‚त्य‚क्ष‚रूप‚तेति य‚त्किञ्चिदेत‚त् ॥
	{\color{gray}{\rmlatinfont\textsuperscript{§~\theparCount}}}
	\pend% ending standard par
      ‚{\tiny $_{lb}$}‚
	    
	    \stanza[\smallbreak]
	  शिङ्श‚पात्वादिति स्व‚भाव‚हेतोरुदाह‚र‚णं ।\&[\smallbreak]
	  
	  
	  ‚{\tiny $_{lb}$}‚‚{\tiny $_{lb}$}‚‚{\tiny $_{lb}$}‚\textsuperscript{\textenglish{21/s}}

	  
	  \pstart \leavevmode% starting standard par
	\hphantom{.}उ म्वे क स्त्व‚त्राह । स्व‚भाव‚हेतोर्ग‚म‚क‚त्वं दूरोत्सारि\add{त‚मेव । भेदाधि‚{\tiny $_{lb}$}‚ष्ठान}\edtext{}{\edlabel{pvsvt_21-1}\label{pvsvt_21-1}\lemma{दूरोत्सारि}\Bfootnote{In the margin. }}त्वाद् ग‚म्य‚ग‚म‚क‚भाव‚स्य । न ह्य‚भिन्ने प्र‚तिब‚न्धो नाम । न शिङ्श‚पा‚{\tiny $_{lb}$}‚ वृक्षात्मिका ।‚{\tiny $_{५}$}‚ त‚तो व्याव‚र्त्त‚मान‚त्वात् । यो हि य‚स्माद् व्याव‚र्त्त‚ते न स त‚दात्मा ।‚{\tiny $_{lb}$}‚ घ‚टादिव प‚टः । व्याव‚र्त्त‚ते च ख‚दिरादिभ्यः शिशंपेत्य‚त‚दात्मिका । त‚दात्म‚त्वे‚{\tiny $_{lb}$}‚ च लिङ्ग‚ग्र‚ह‚ण‚वेलायामेवाव्य‚तिरेकात् साध्य‚स्व‚रूप‚व‚ल्लिङ्गाग्र‚ह‚ण‚व‚त्त्वं‚{\tiny $_{lb}$}‚ साध्य‚स्य गृहीत‚त्वाद‚नुमे\add{य‚त्व‚हानिः । तादात्म्येन च शिं}\edtext{}{\edlabel{pvsvt_21-1b}\label{pvsvt_21-1b}\lemma{नुमे}\Bfootnote{In the margin. }}श‚पात्व‚स्य ग‚म‚क‚त्वे‚{\tiny $_{lb}$}‚ तादात्म्याविशेषाद् वृक्ष‚त्व‚स्य शिंश‚पां प्र‚ति ग‚म‚{\tiny $_{६}$}‚क‚त्व‚प्र‚स‚ङ्गः । अथास्यानिय‚त‚{\tiny $_{lb}$}‚त्वाद‚ग‚म‚क‚त्वं निय‚त‚तैव त‚र्हि ग‚म‚क‚त्वे निमित्तं न तु तादात्म्यं व्य‚भिचारिण्य‚पि‚{\tiny $_{lb}$}‚ वृक्ष‚त्वे तादात्म्य‚स्य द‚र्श‚नात् । अथ त‚दात्म‚नैव वृक्षे नास्ति शिंश‚पा{... ... ... ...}त‚दात्मिका । स‚र्व‚था य‚योरेव{...}त‚या ग‚म्य‚ग‚म‚क\add{भावो‚{\tiny $_{lb}$}‚ नानात्म्ये तु तादात्म्याभावः ।}\edtext{}{\edlabel{pvsvt_21-1c}\label{pvsvt_21-1c}\lemma{क}\Bfootnote{In the margin. }}एव‚म‚शिंश‚पाऽवृक्षापोह‚योर‚पोह‚बुद्ध्योर्वा ग‚म्य‚{\tiny $_{lb}$}‚ग‚म‚क‚भावो निराक‚{\tiny $_{७}$}‚र्त्त‚व्यइति ।
	{\color{gray}{\rmlatinfont\textsuperscript{§~\theparCount}}}
	\pend% ending standard par
      \textsuperscript{\textenglish{9a/PSVTa}}‚{\tiny $_{lb}$}‚

	  
	  \pstart \leavevmode% starting standard par
	त‚द‚युक्तं । शिंश‚पा हि वृक्ष‚विशेष‚स्व‚भावा । वृक्ष‚विशेषोपि शिंश‚पास्व‚भाव‚{\tiny $_{lb}$}‚ एवेत्युभ‚य‚ग‚त‚न्तादात्म्यं । तादात्म्येपि च क‚श्चिद‚वृक्ष‚व्यावृत्ते स्व‚भावे क‚थंचिन्मूढो‚{\tiny $_{lb}$}‚ नाशिंश‚पाव्यावृत्ते [।] तेन शिंश‚पादेर्लिङ्ग‚स्य ग्र‚ह‚णान्नानु\add{मेय‚त्व‚हानिः । य‚त‚श्च‚{\tiny $_{lb}$}‚ न वृक्ष‚मात्र}\edtext{}{\edlabel{pvsvt_21-1d}\label{pvsvt_21-1d}\lemma{णान्नानु}\Bfootnote{In the margin. }}स्व‚भावा शिंश‚पा [।] तेन न वृक्ष‚त्व‚स्य शिंश‚पां प्र‚ति ग‚म‚क‚त्वं ।‚{\tiny $_{lb}$}‚ साध्य‚साध‚नाद्य‚{\tiny $_{१}$}‚भाव‚दोष‚न्तु स्व‚य‚मेव शास्त्र‚कारो निराक‚रिष्य‚तीति‚{\tiny $_{lb}$}‚ य‚त्किञ्चिदेत‚त् ।
	{\color{gray}{\rmlatinfont\textsuperscript{§~\theparCount}}}
	\pend% ending standard par
      ‚{\tiny $_{lb}$}‚

	  
	  \pstart \leavevmode% starting standard par
	\textbf{प्र‚देश} इत्य‚नुप‚ल‚ब्धेः क‚थ‚नं । \textbf{प्र‚देश‚विशेष} इत्युद्दिष्टे देशे । \textbf{क्व‚चि}दिति वादि‚{\tiny $_{lb}$}‚प्र‚तिवादिप्र‚सिद्धे । \textbf{न घ‚ट} इति घ‚टाभाव‚व्य‚व‚हारः साध्यः । \textbf{उप‚ल‚ब्धी}त्यादिना हेतुनि‚{\tiny $_{lb}$}‚र्देशः । उप‚ल‚ब्धे\add{र्ल‚क्ष‚णानि क‚र‚णानि च‚क्षुरादीन्य}\edtext{}{\edlabel{pvsvt_21-1e}\label{pvsvt_21-1e}\lemma{ब्धे}\Bfootnote{In the margin. }}प्र‚तिब‚द्ध‚साम‚र्थ्याद्यु\textbf{प‚ल‚ब्धि‚{\tiny $_{lb}$}‚ल‚क्ष}णानि । तानि प्राप्तः स्वाभास‚ज्ञान‚ज‚न‚न‚यो‚{\tiny $_{२}$}‚ग्यः स्व‚भाव‚विशेषः ।‚{\tiny $_{lb}$}‚ स्व‚ज्ञान‚ज‚न‚न‚साम‚ग्र्य‚न्त‚र्भूतोर्थ इत्य‚र्थः । \textbf{त‚स्यानुप‚ल‚ब्धे}स्त‚थाभूत‚स्यास‚द्व्य‚व‚हार‚{\tiny $_{lb}$}‚सिद्धिः ॥
	{\color{gray}{\rmlatinfont\textsuperscript{§~\theparCount}}}
	\pend% ending standard par
      ‚{\tiny $_{lb}$}‚

	  
	  \pstart \leavevmode% starting standard par
	क‚थं पुन‚र्यो य‚त्र नास्ति स त‚त्रोप‚ल‚ब्धिल‚क्ष‚ण‚प्राप्तो भ‚व‚ति । उप‚ल‚ब्धिल‚क्ष‚ण‚{\tiny $_{lb}$}‚\textbf{प्राप्त‚त्वं} हि ज्ञान‚विष‚य‚त्व‚न्त‚स्मिंश्च स‚ति कुतो नास्तित्वं ।
	{\color{gray}{\rmlatinfont\textsuperscript{§~\theparCount}}}
	\pend% ending standard par
      ‚{\tiny $_{lb}$}‚

	  
	  \pstart \leavevmode% starting standard par
	\textbf{उच्य‚ते} । याव‚त्यां \add{साम‚ग्र्यां स‚त्यां पूर्वं प्र‚ति}\edtext{}{\edlabel{pvsvt_21-1f}\label{pvsvt_21-1f}\lemma{त्यां}\Bfootnote{In the margin. }}प‚न्नो भाव‚स्ताव‚त्यामेव‚{\tiny $_{lb}$}‚ ‚{\tiny $_{lb}$}‚ ‚{\tiny $_{lb}$}‚ \leavevmode\ledsidenote{\textenglish{22/s}}साम‚ग्र्यां स‚त्यां \textbf{य‚दि स्यात्} पूर्व‚काल‚व‚दुप‚ल‚भ्येतेत्येव‚{\tiny $_{३}$}‚मुप‚ल‚ब्धिल‚क्ष‚ण‚प्राप्त‚त्वं‚{\tiny $_{lb}$}‚ बुद्ध्या प‚रामृश्य भाव‚स्याप्र‚तिभास‚नान्नास्तीति निषेधः क्रिय‚ते । न त्व‚दृश्य‚स्य ।‚{\tiny $_{lb}$}‚ प्र‚तिभास‚प‚राम‚र्शोपायाभावात् । स‚र्व‚दाऽप्र‚तिप‚न्न‚त्वात् ॥
	{\color{gray}{\rmlatinfont\textsuperscript{§~\theparCount}}}
	\pend% ending standard par
      ‚{\tiny $_{lb}$}‚

	  
	  \pstart \leavevmode% starting standard par
	न‚नु दृश्य‚स्याभावे स‚ति सैव साम‚ग्री क‚थं प्र‚तिप‚न्नेति चेत् ॥ न । एक‚ज्ञान‚{\tiny $_{lb}$}‚विष‚य‚योर्भाव‚योरेको\add{प‚ल‚म्भादिति}\edtext{}{\edlabel{pvsvt_22-1}\label{pvsvt_22-1}\lemma{योरेको}\Bfootnote{In the margin. }} ब्रूमः । स्व‚त एव च निषिध्य‚मान‚स्यो‚{\tiny $_{lb}$}‚प‚ल‚ब्धिल‚क्ष‚ण‚प्राप्त‚त्व‚न्निश्चीय‚ते । एत‚देवैक‚{\tiny $_{४}$}‚ज्ञान‚ज‚न‚न‚योग्य‚त‚योप‚ल‚ब्धिल‚क्ष‚ण‚{\tiny $_{lb}$}‚प्राप्त‚त्व‚न्द‚र्श‚यितुमाह ।
	{\color{gray}{\rmlatinfont\textsuperscript{§~\theparCount}}}
	\pend% ending standard par
      ‚{\tiny $_{lb}$}‚

	  
	  \pstart \leavevmode% starting standard par
	\textbf{य‚दि स्याद्} घ‚टादिरुप‚ल‚भ्यं स‚त्त्वं य‚स्य स त‚थाभूत एव स्यात् । \textbf{नान्य‚थे}ति न‚{\tiny $_{lb}$}‚ क‚दाचिद‚ग्राह्य‚स्त‚थाभूतोऽव‚श्यं ज्ञान‚न्न व्य‚भिच‚र‚तीति याव‚त् । य‚त एव\textbf{न्तेन} कार\textbf{णेन ।‚{\tiny $_{lb}$}‚ उप‚ल‚ब्धिल‚क्ष‚ण‚प्राप्त}स्येति य‚द्विशेष‚ण‚मुक्त‚न्त‚द‚र्थादु\textbf{प‚ल‚ब्धिल‚क्ष‚ण‚प्राप्त‚स‚त्त्व‚स्ये}त्युक्त‚{\tiny $_{lb}$}‚म्भ‚व‚ति । अत्र त्व‚यं बाह्यार्थः । ल‚क्ष्य‚तेऽ‚{\tiny $_{५}$}‚नेनेति ल‚क्ष‚णं । उप‚ल‚ब्धिरेव‚{\tiny $_{lb}$}‚ ल‚क्ष‚ण‚न्त‚त्प्राप्त‚मुप‚ल‚ब्धिल‚क्ष‚ण‚प्राप्तं ज्ञानेनाव्य‚भिच‚रित‚स‚म्ब‚न्ध‚मित्य‚र्थः ।‚{\tiny $_{lb}$}‚ एवंभूतं स‚त्त्वं य‚स्य त‚त्त‚थोक्तं । त‚थाभूतं हि स‚त्त्वं ज्ञानं-निवृत्त्याव‚श्यं‚{\tiny $_{lb}$}‚ निव‚र्त्त‚त इति भावः ॥ विधिप्र‚तिषेधाभ्यां स‚र्वं साध‚नं व्याप्तं । विधीय‚मान‚श्च‚{\tiny $_{lb}$}‚ साध्यः प्र‚तिब‚न्ध‚द्व‚येन भिद्य‚त इति विधिप्र‚तिषेधौ हेतुत्र‚याय‚त्ताविति द‚र्श‚य‚न्नाह ।‚{\tiny $_{lb}$}‚ \textbf{त‚त्रे}त्यादि ।
	{\color{gray}{\rmlatinfont\textsuperscript{§~\theparCount}}}
	\pend% ending standard par
      ‚{\tiny $_{lb}$}‚

	  
	  \pstart \leavevmode% starting standard par
	\textbf{त‚{\tiny $_{६}$}‚त्र} त्रिषु हेतुषु \textbf{द्वौ} कार्य‚स्व‚भाव‚हेतू \textbf{व‚स्तुसाध‚नौ} विधिसाध‚नौ । \textbf{द्वावेवे}ति‚{\tiny $_{lb}$}‚ चाव‚धार‚णं न तु व‚स्तुसाध‚नावेवेति । आभ्यां साम‚र्थ्याद् व्य‚व‚च्छेद‚स्यापि सिद्धेः ।‚{\tiny $_{lb}$}‚ \textbf{एकः प्र‚तिषेध‚हेतु}रिति । उक्त‚ल‚क्ष‚णोनुप‚ल‚म्भः प्र‚तिषेध‚हेतुरेव । न त्वेक एवेत्य‚व‚धार्य‚ते‚{\tiny $_{lb}$}‚ पूर्व्वाभ्याम‚पि व्य‚व‚च्छेद‚ग‚तेः ॥
	{\color{gray}{\rmlatinfont\textsuperscript{§~\theparCount}}}
	\pend% ending standard par
      ‚{\tiny $_{lb}$}‚

	  
	  \pstart \leavevmode% starting standard par
	\leavevmode\ledsidenote{\textenglish{9b/PSVTa}} क‚श्चिदाह । व्य‚व‚च्छेदः श‚ब्द‚लिङ्गाभ्यां प्र‚साध्य‚ते यावांश्चिद् व्य‚व‚च्छेदः‚{\tiny $_{७}$}‚ स‚{\tiny $_{lb}$}‚ स‚र्वोनुप‚ल‚म्भादेवेत्य‚नुप‚ल‚म्भ एवैको हेतुरि ति ।
	{\color{gray}{\rmlatinfont\textsuperscript{§~\theparCount}}}
	\pend% ending standard par
      ‚{\tiny $_{lb}$}‚

	  
	  \pstart \leavevmode% starting standard par
	त‚द‚युक्तं । य‚तो न ताव‚त् प्र‚माण‚व्यापारापेक्ष‚यैत‚दुच्य‚ते । व‚स्तुन्येव‚{\tiny $_{lb}$}‚ प्र‚माण‚व्यापारात् । त‚दाहात \textbf{एव व}स्तुविष‚यं प्रामाण्यं द्व‚योरिति । नाप्य‚ध्य‚व‚{\tiny $_{lb}$}‚साय‚व‚शादेत‚दुच्य‚ते व‚स्तुन एवाध्य‚व‚सायात् । व्य‚व‚च्छेदेन स‚ह लिंग‚स्य‚{\tiny $_{lb}$}‚ स‚म्ब‚न्धाभावाच्च ।
	{\color{gray}{\rmlatinfont\textsuperscript{§~\theparCount}}}
	\pend% ending standard par
      ‚{\tiny $_{lb}$}‚‚{\tiny $_{lb}$}‚‚{\tiny $_{lb}$}‚\textsuperscript{\textenglish{23/s}}

	  
	  \pstart \leavevmode% starting standard par
	यो पि म‚न्य‚तेऽनुप‚ल‚म्भेऽस‚द्व्य‚व‚हार‚योग्य‚ता साध्य‚ते न प्र‚तिषेधः ।‚{\tiny $_{१}$}‚ योग्य‚ता‚{\tiny $_{lb}$}‚ च स्व‚भाव‚भूतैव । त‚थाग्निम‚ति प्र‚देशे साध्ये अग्निम‚त्ता प्र‚देश‚स्य स्व‚भाव एव‚{\tiny $_{lb}$}‚ साध्यो धूम‚व‚त्त्वादिति च हेतुः । प्र‚देशाभिन्न इति स‚र्वो हेतुः स्व‚भाव‚हेतुरेवेति [।]‚{\tiny $_{lb}$}‚ त‚द‚युक्त‚म्[।]एवं हि ग‚म‚क‚त्त्वे ग‚र्द‚भ‚त्वादेर‚पि ग‚म‚क‚त्वं स्यात्[।] न भ‚व‚त्य‚ग्नि‚{\tiny $_{lb}$}‚कार्य‚त्वाभावाद् ग‚र्द‚भ‚स्येति चेत् । य‚द्येवं धूम‚स्याग्निकार्य‚त्व‚मेव ग‚म‚क‚त्वे‚{\tiny $_{lb}$}‚ निब‚न्ध‚न‚मिति क‚थं न कार्य‚हेतुः ।
	{\color{gray}{\rmlatinfont\textsuperscript{§~\theparCount}}}
	\pend% ending standard par
      ‚{\tiny $_{lb}$}‚

	  
	  \pstart \leavevmode% starting standard par
	अन्य‚{\tiny $_{२}$}‚स्तु म‚न्य‚तेऽनुप‚ल‚म्भ‚स्तु प्र‚देश‚कार्य इत्य‚स‚द्व्य‚व‚हारे साध्ये कार्य‚हेतुरेव ।‚{\tiny $_{lb}$}‚ कृत‚क‚विक‚ल्प‚श्चानित्य‚श‚ब्द‚कार्य इति स‚र्व एव हेतुः कार्य‚हेतुरिति ।
	{\color{gray}{\rmlatinfont\textsuperscript{§~\theparCount}}}
	\pend% ending standard par
      ‚{\tiny $_{lb}$}‚

	  
	  \pstart \leavevmode% starting standard par
	त‚द‚प्य‚युक्तं [।] कृत‚क‚विक‚ल्पो हि कृत‚क‚कार‚ण‚त्वात् त‚स्यैवानुमाप‚कः‚{\tiny $_{lb}$}‚ स्यान्नानित्य‚स्य । कृत‚क‚स्यानित्य‚स्व‚भाव‚त्वाद‚नित्य‚त्वं साध‚य‚तीति चेत् । य‚द्येवं‚{\tiny $_{lb}$}‚ कृत‚क‚त्वादेवानित्य‚त्व‚प्र‚तीतिः स्यान्न कृत‚क‚विक‚ल्पात् । कृ‚{\tiny $_{३}$}‚त‚क‚स्व‚भाव‚त‚यैवा‚{\tiny $_{lb}$}‚नित्य‚त्वे ग‚म‚क‚त्वात् । य‚द्वा कृत‚क‚विक‚ल्प‚श्च स्यान्न वानित्य इत्य‚नैकान्तः ।‚{\tiny $_{lb}$}‚ कृत‚क‚विक‚ल्प‚ज‚न‚न‚साम‚र्थ्यं कृत‚काव्य‚तिरिक्त‚न्त‚च्च न पूर्व‚न्न प‚श्चात्तेनासाव‚नित्यो‚{\tiny $_{lb}$}‚ऽव्य‚तिरिक्त‚न्त‚र्हि साम‚र्थ्य‚म‚नित्य‚त्वं साध‚य‚तीति क‚थं न स्व‚भावो हेतुरिति‚{\tiny $_{lb}$}‚ य‚त्किञ्चिदेत‚त् ॥
	{\color{gray}{\rmlatinfont\textsuperscript{§~\theparCount}}}
	\pend% ending standard par
      ‚{\tiny $_{lb}$}‚

	  
	  \pstart \leavevmode% starting standard par
	किं पुन‚र्द्वावेव व‚स्तुसाध‚नावित्याह । \textbf{स्व‚भावेन प्र‚तिब‚न्धः} साध‚नं कृते ति‚{\tiny $_{lb}$}‚ स‚{\tiny $_{४}$}‚मासः । \textbf{स्व‚भावेन प्र‚तिब‚द्ध‚त्वं} । प्र‚तिब‚द्ध‚स्व‚भाव‚मिति याव‚त् । \textbf{त‚स्मिन् स‚त्य‚र्थो}‚{\tiny $_{lb}$}‚ लिंग\textbf{म‚र्थ}मिति लिङ्गिनं \textbf{न व्य‚भिच‚र‚ति} ।
	{\color{gray}{\rmlatinfont\textsuperscript{§~\theparCount}}}
	\pend% ending standard par
      ‚{\tiny $_{lb}$}‚

	  
	  \pstart \leavevmode% starting standard par
	\textbf{स च} स्व‚भाव‚प्र‚तिब‚न्धः कुत इत्याह । \textbf{त‚दात्म‚त्वा}दिति साध्य‚स्व‚{\tiny $_{lb}$}‚भाव‚त्वात् ॥ \textbf{त‚दात्म‚त्वे} साध्य‚स्व‚भावात्म‚त्वे साध‚न‚स्य । य‚देव साध्य‚न्त‚देव साध‚{\tiny $_{lb}$}‚\add{न‚मिति \textbf{साध्य‚साध‚न‚यो}}\edtext{}{\edlabel{pvsvt_23-1}\label{pvsvt_23-1}\lemma{साध}\Bfootnote{In the margin. }}\textbf{र्भेदाभाव}स्तेन प्र‚तिज्ञार्थैक‚देशो हेतुरिति प‚र‚माश‚ङ्‚{\tiny $_{lb}$}‚क्याह । \textbf{ने}त्यादि । सा‚{\tiny $_{५}$}‚ध्य‚साध‚न‚भूतानां \textbf{ध‚र्म‚भेदानां} व्यावृत्तिभेदेन \textbf{प‚रिक‚ल्प‚नादा}‚{\tiny $_{lb}$}‚रोपाद् [।]
	{\color{gray}{\rmlatinfont\textsuperscript{§~\theparCount}}}
	\pend% ending standard par
      ‚{\tiny $_{lb}$}‚

	  
	  \pstart \leavevmode% starting standard par
	एत‚दुक्त‚म्भ‚व‚ति । ध‚र्म‚भेदः स‚मारोप्य‚ते तेन साध्य‚साध‚न‚भेदः । य‚तो निश्चितो‚{\tiny $_{lb}$}‚ ग‚म‚को निश्चेत‚व्य‚श्च ग‚म्य उच्य‚ते । निश्च‚य‚विष‚य‚श्चारोपित एवेति निश्च‚य‚भेदा‚{\tiny $_{lb}$}‚रोपितो ध‚र्म\add{भेदः । एत‚च्च व‚क्ष्यामोऽन्यापोह‚प्र}\edtext{}{\edlabel{pvsvt_23-1b}\label{pvsvt_23-1b}\lemma{र्म}\Bfootnote{In the margin. }}स्तावे ।\edtext{\textsuperscript{*}}{\edlabel{pvsvt_23-2}\label{pvsvt_23-2}\lemma{*}\Bfootnote{\href{http://sarit.indology.info/?cref=pv.3.163-173}{ Pramāṇavārtika 3:163-73. }}}
	{\color{gray}{\rmlatinfont\textsuperscript{§~\theparCount}}}
	\pend% ending standard par
      ‚{\tiny $_{lb}$}‚\textsuperscript{\textenglish{24/s}}

	  
	  \pstart \leavevmode% starting standard par
	आचार्य दि ग्ना गे नाप्येत‚दुक्त‚मित्याह । \textbf{त‚था चे}त्यादि । \textbf{स‚र्व एवे}ति‚{\tiny $_{६}$}‚‚{\tiny $_{lb}$}‚ य‚त्रापि साध्य‚साध‚न‚योर‚ग्निधूम‚योर्वास्त‚वो भेद‚स्त‚त्रापि स्व‚ल‚क्ष‚णेन व्य‚व‚हारा‚{\tiny $_{lb}$}‚योगाद‚नुमीय‚तेनेने\textbf{त्य‚नुमानं}लिङ्ग‚म\textbf{नुमेयः} साध्य‚ध‚र्मी साध्य‚ध‚र्म‚श्च तेषां \textbf{व्य‚व‚हारो}‚{\tiny $_{lb}$}‚ नानात्व‚प्र‚तिरूपः । \textbf{बुद्ध्यारूढे}न \textbf{ध‚र्म‚ध‚र्मि}णो\add{\textbf{र्भेद}स्ते\textbf{न} बुद्धिप्र‚तिभास‚ग‚तेन}\edtext{}{\edlabel{pvsvt_24-3}\label{pvsvt_24-3}\lemma{णो}\Bfootnote{In the margin. }}‚{\tiny $_{lb}$}‚ भिन्नेन रूपेण भेद‚व्य‚व‚हार इति याव‚त् ॥ य‚दि त‚र्हि \textbf{बुद्धिप‚रिक‚ल्पि}तो ध‚{\tiny $_{७}$}‚र्म‚ध‚र्मि‚{\tiny $_{lb}$}‚\leavevmode\ledsidenote{\textenglish{10a/PSVTa}} व्य‚व‚हार एव‚न्त‚र्हि क‚ल्पिताद्धेतोः साध्य‚सिद्धिः प्राप्ता । त‚त‚श्च हेतुदोषो यावानुच्य‚ते‚{\tiny $_{lb}$}‚ स स‚र्वः स्यात् । त‚दाह भ ट्टः ॥\edtext{\textsuperscript{*}}{\edlabel{pvsvt_24-2}\label{pvsvt_24-2}\lemma{*}\Bfootnote{Ślokavārtika, Nirālamba 171-72. }}
	{\color{gray}{\rmlatinfont\textsuperscript{§~\theparCount}}}
	\pend% ending standard par
      ‚{\tiny $_{lb}$}‚
	  \bigskip
	  \begingroup
	
	    
	    \stanza[\smallbreak]
	  {\normalfontlatin\large ``\qquad}य‚दि वा विद्य‚मानोपि भेदो बुद्धिप्र‚क‚ल्पितः [।]&‚{\tiny $_{lb}$}‚साध्य‚साध‚न‚ध‚र्मादेर्व्य‚व‚हाराय क‚ल्प्य‚ते ॥&‚{\tiny $_{lb}$}‚त‚तो भ‚व‚त्प्र‚युक्तेस्मिन् साध‚नं या\add{व‚दुच्य‚ते ।}&‚{\tiny $_{lb}$}‚\add{स‚र्व‚त्रोत्प‚द्य‚ते बुद्धिरिति}\edtext{\textsuperscript{*}}{\edlabel{pvsvt_24-3b}\label{pvsvt_24-3b}\lemma{*}\Bfootnote{In the margin. }} दूष‚ण‚ता भ‚वेदिति ।{\normalfontlatin\large\qquad{}"}\&[\smallbreak]
	  
	  
	  
	  \endgroup
	‚{\tiny $_{lb}$}‚

	  
	  \pstart \leavevmode% starting standard par
	अत्राह । \textbf{भेद} इत्यादि । एत‚दाहार्थ ए\textbf{व} वार्थं ग‚म‚य‚ति केव‚लं \textbf{ध‚र्म‚{\tiny $_{lb}$}‚ध‚र्मित‚या}ऽयं ध‚र्मोऽयं ध‚र्मीति यो भेदो नानात्व‚म‚य‚मेव \textbf{बुद्ध्याकार‚कृतो} बुद्ध्या प‚रि‚{\tiny $_{lb}$}‚क‚ल्पितो \textbf{नार्थो}पि न लिङ्ग‚म‚पि बुद्ध‚याकार‚कृत‚म्[।]विक‚ल्प‚निर्मितादेव लिङ्गात्‚{\tiny $_{lb}$}‚ क‚स्माद् अर्थ‚प्र‚तिप‚त्तिर्न भ‚व‚तीत्याह । \textbf{विक‚ल्पे}त्यादि । \textbf{विक‚ल्प‚भेदानां} विक‚ल्प‚{\tiny $_{lb}$}‚विशे\add{षाणा\textbf{मिच्छामात्रानुरोधित्वेन} स्व‚त}\edtext{}{\edlabel{pvsvt_24-3c}\label{pvsvt_24-3c}\lemma{विशे}\Bfootnote{In the margin. }}न्त्राणाम\textbf{न‚र्थाश्र‚या}द‚र्थाप्र‚तिब‚द्ध‚त्वे‚{\tiny $_{lb}$}‚ \add{साक्षाद‚नुत्प‚त्ते}\edtext{\textsuperscript{*}}{\edlabel{pvsvt_24-3d}\label{pvsvt_24-3d}\lemma{*}\Bfootnote{In the margin. }}र्नार्थाल‚म्ब‚न‚त्वादित्य‚र्थः । तैर‚र्थी-नाश्र‚यै‚{\tiny $_{२}$}‚र्विक‚ल्पैः क‚ल्पित‚{\tiny $_{lb}$}‚श्चासौ विष‚य‚श्चेति \textbf{त‚त्क‚ल्पित‚विष‚य}स्त‚स्\textbf{मा}देवंभूताद्धेतो\textbf{र‚र्थ‚प्र‚तीताव}भ्युप‚ग‚म्य‚{\tiny $_{lb}$}‚मानायाम\textbf{न‚र्थ‚प्र‚तिल‚म्भ एव स्या}द‚र्थ‚प्र‚तिल‚म्भ एव न स्यात् ॥
	{\color{gray}{\rmlatinfont\textsuperscript{§~\theparCount}}}
	\pend% ending standard par
      ‚{\tiny $_{lb}$}‚

	  
	  \pstart \leavevmode% starting standard par
	द्वितीयं प्र‚तिब‚न्ध‚ल‚क्ष‚ण‚माह । \textbf{कार्य‚स्यापी}त्यादि । \textbf{त‚त्स्व‚भाव}स्येति कार्यं‚{\tiny $_{lb}$}‚स्व‚भाव‚स्य \textbf{त‚दुत्प‚त्तेः} कार‚णादुत्प‚त्तेर् योऽनुमान‚स्य प्रामाण्यं नेच्छ‚ति तं प्र‚त्याह ।‚{\tiny $_{lb}$}‚ \textbf{एताव‚नुमेय‚प्र‚त्य‚यौ} प्र‚माण‚मिति स‚म्ब‚{\tiny $_{३}$}‚न्धः ॥ एताविति स्व‚भाव‚कार्य‚लिङ्गौ ।‚{\tiny $_{lb}$}‚ अनुप‚ल‚ब्धेः स्व‚भाव‚हेताव‚न्त‚र्भावाद् द्वावित्याह । य‚द्वा प्र‚क्रान्तापेक्ष‚या द्वावित्याह ।
	{\color{gray}{\rmlatinfont\textsuperscript{§~\theparCount}}}
	\pend% ending standard par
      ‚{\tiny $_{lb}$}‚\textsuperscript{\textenglish{25/s}}

	  
	  \pstart \leavevmode% starting standard par
	स्व‚भाव‚कार‚ण‚योर‚नुमेय‚योः प्र‚त्य‚यावित्य‚नुमेय‚प्र‚त्य‚यौ \textbf{अत‚त्प्र‚तिभासित्वे}पीत्य‚{\tiny $_{lb}$}‚नुमेय‚स्व‚ल‚क्ष‚णाप्र‚तिभासित्वेपि । अत‚त्प्र‚तिभासित्व‚न्त‚तः \textbf{साक्षाद‚नुत्प‚त्तेः} । क‚थ‚न्त‚र्ह्य‚{\tiny $_{lb}$}‚व्य‚भिचार इत्याह । \textbf{त‚दुत्प‚त्ते}रित्य‚नुमेयाभ्यां स्व‚भाव‚कार‚णाभ्यां पा‚{\tiny $_{४}$}‚र‚म्प‚र्येणोत्प‚त्तेः ।‚{\tiny $_{lb}$}‚ \textbf{त‚द‚व्य‚भिचारिणा}वित्य‚नुमेयांव्य‚भिचारिणौ । \textbf{इति} हेतोः \textbf{प्र‚माणं प्र‚त्य‚क्ष‚व‚त्} ।
	{\color{gray}{\rmlatinfont\textsuperscript{§~\theparCount}}}
	\pend% ending standard par
      ‚{\tiny $_{lb}$}‚

	  
	  \pstart \leavevmode% starting standard par
	\hphantom{.}तेन य‚दुच्य‚ते ऽवि द्ध क र्ण्णे नान‚धिग‚तार्थ‚प‚रिच्छित्तिः प्र‚माण‚म‚तो नानुमान‚{\tiny $_{lb}$}‚म्प्र‚माण‚म‚र्थ‚प‚रिच्छेद‚क‚त्वाभावादि ति त‚द‚पास्तं । य‚तः स‚र्व एव प्रेक्षावान् प्र‚वृत्ति‚{\tiny $_{lb}$}‚कामः प्र‚माण‚म‚न्वेष‚ते प्र‚वृत्तिविष‚यार्थोप‚द‚र्श‚क‚त्वेन प्र‚वृत्तिविष‚य‚श्चार्थोऽर्थ‚क्रिया‚{\tiny $_{lb}$}‚स‚म‚र्थ ए‚{\tiny $_{५}$}‚व । न चानाग‚तं प्र‚वृत्तिसाध्यार्थ‚क्रिया साम‚र्थ्य‚म्व‚स्तुनः प्र‚त्य‚क्ष‚म्प‚रिच्छि‚{\tiny $_{lb}$}‚न‚त्तीत्युक्त‚म‚तः क‚थ‚म‚स्यार्थ‚प‚रिच्छेद‚मात्रात्प्रामाण्यं । त‚स्मात् स्व‚विष‚ये त‚दुत्प‚त्त्या‚{\tiny $_{lb}$}‚ प्र‚त्य‚क्षं य‚न्म‚या पूर्व‚प्र‚तिप‚न्नं प्र‚ब‚न्धेनार्थ‚क्रियाकारि त‚देवेद‚मिति निश्च‚यं कुर्व‚त्‚{\tiny $_{lb}$}‚ प्र‚व‚र्त्त‚क‚त्वात् प्र‚माण‚न्त‚थानुमान‚म‚पि ।
	{\color{gray}{\rmlatinfont\textsuperscript{§~\theparCount}}}
	\pend% ending standard par
      ‚{\tiny $_{lb}$}‚

	  
	  \pstart \leavevmode% starting standard par
	\textbf{प्र‚त्य‚क्ष‚स्यापी}त्यादिना व्याप्तिमाह । \textbf{अर्थाव्य‚भिचार एवेति} । पूर्व‚म‚भि‚{\tiny $_{lb}$}‚म‚तार्थ‚{\tiny $_{६}$}‚कारित्वेन निश्चित‚स्यार्थ‚स्य स‚म्भ‚वे स‚ति भाव एव प्र‚त्य‚क्ष‚स्य प्रामाण्य‚म‚न्य‚था‚{\tiny $_{lb}$}‚ \textbf{त‚द‚भावे} अर्थाभावे \textbf{भाविनः} प्र‚त्य‚क्ष‚स्य \textbf{त‚द्विप्र‚ल‚म्भा}न्निश्चितार्थास‚म्वादात् । एत‚दु‚{\tiny $_{lb}$}‚क्त‚म्भ‚व‚ति । य‚दार्थ‚क्रियास‚म‚र्थ‚म्व‚स्तु प्र‚त्य‚क्षं न प‚रिच्छिन‚त्ति । य‚दि च त‚थाभूत‚{\tiny $_{lb}$}‚म‚पि व‚स्तु व्य‚भिच‚रेत्प्र‚माण‚म‚पि न स्यात् । \textbf{अव्य‚भिचार‚श्चान्य‚स्य कोन्य‚स्त‚दुत्प‚त्ते}‚{\tiny $_{lb}$}‚रित्य‚न्यंस्यार्थान्त‚र‚भूत‚स्य‚{\tiny $_{७}$}‚ योन्येन स‚हाव्य‚भिचारः स त‚दुत्प‚त्तेः कोन्यो नैवान्यः । \leavevmode\ledsidenote{\textenglish{10b/PSVTa}}‚{\tiny $_{lb}$}‚ त‚दुत्प‚त्तिरेवाव्य‚भिचार इत्य‚र्थः ।
	{\color{gray}{\rmlatinfont\textsuperscript{§~\theparCount}}}
	\pend% ending standard par
      ‚{\tiny $_{lb}$}‚

	  
	  \pstart \leavevmode% starting standard par
	क‚स्माद्[।]अ\textbf{नाय‚त्त‚रूपाणा}म‚प्र‚तिब‚द्ध‚स्व‚भावानां \textbf{स‚ह‚भाव‚निय‚माभावाद}व्य‚भि‚{\tiny $_{lb}$}‚चार‚निय‚माभावात् । त‚स्माद‚र्थ‚क्रियाकारित्वेन निश्चिताद‚र्थादुत्प‚त्तिरेव प्र‚त्य‚क्ष‚{\tiny $_{lb}$}‚स्याव्य‚भिचार इति प्रामाण्य‚न्त‚च्चानुमानेप्य‚स्तीति स‚मं द्व‚य‚मिति भावः ॥
	{\color{gray}{\rmlatinfont\textsuperscript{§~\theparCount}}}
	\pend% ending standard par
      ‚{\tiny $_{lb}$}‚

	  
	  \pstart \leavevmode% starting standard par
	\hphantom{.}एतेनैत‚द‚पि निर‚स्तं प्र‚माण‚स्यागौ‚{\tiny $_{१}$}‚ण‚त्वाद‚नुमानाद‚र्थ‚निश्च‚यो दुर्ल‚भ इति ।‚{\tiny $_{lb}$}‚ य‚द्य‚गौण‚त्व‚म‚नुप‚च‚रित‚त्व‚मुच्य‚ते त‚दानुमान‚म‚प्य‚नुप‚च‚रित‚मेवास्ख‚ल‚द्बुद्धिरूप‚त्वात् ॥
	{\color{gray}{\rmlatinfont\textsuperscript{§~\theparCount}}}
	\pend% ending standard par
      ‚{\tiny $_{lb}$}‚

	  
	  \pstart \leavevmode% starting standard par
	अथ ध‚र्म‚ध‚र्मिस‚मुदाय‚स्य साध्य‚त्वे हेतोः प‚क्ष‚ध‚र्म‚त्व‚म‚न्व‚यो वा न स‚म्भ‚व‚ति‚{\tiny $_{lb}$}‚ तेन प‚क्ष‚ध‚र्म‚त्व‚प्र‚सिद्ध्य‚र्थं ध‚र्मिणः साध्य‚त्व‚मुप‚च‚रित‚व्य‚म‚न्व‚य‚सिद्ध्य‚र्थ‚ञ्च‚{\tiny $_{lb}$}‚ ध‚र्म‚स्येत्येव‚मुप‚च‚रित‚विष‚य‚त्वाद‚नुमान‚मुप‚च‚रितं ।
	{\color{gray}{\rmlatinfont\textsuperscript{§~\theparCount}}}
	\pend% ending standard par
      ‚{\tiny $_{lb}$}‚‚{\tiny $_{lb}$}‚\textsuperscript{\textenglish{26/s}}

	  
	  \pstart \leavevmode% starting standard par
	त‚द‚युक्तं‚{\tiny $_{२}$}‚ य‚तो लोके धूम‚मात्र‚म‚ग्निमात्र‚व्याप्तं य‚त्र ध‚र्मिणि दृश्य‚ते त‚त्रैवाग्नि‚{\tiny $_{lb}$}‚प्र‚तीतिर्भ‚व‚तीति क‚स्यात्रोप‚चार एवं च स‚मुदाय‚स्यापि साध्य‚त्वं सिध्य‚ति । य‚दाह ।‚{\tiny $_{lb}$}‚ केव‚ल एव ध‚र्मो ध‚र्मिणि साध्य‚स्त‚थेष्ट‚स‚मुदाय‚स्य सिद्धिः कृता भ‚व‚तीति [।]‚{\tiny $_{lb}$}‚ न चानुमान‚विष‚ये साध्य‚श‚ब्दोप‚चारे स‚त्य‚नुमान‚मुप‚च‚रित‚न्नाम ॥
	{\color{gray}{\rmlatinfont\textsuperscript{§~\theparCount}}}
	\pend% ending standard par
      ‚{\tiny $_{lb}$}‚

	  
	  \pstart \leavevmode% starting standard par
	अथ प्र‚माण‚स्यागौण‚त्वाद‚भ्रान्त‚त्वाद‚नुमान‚स्य तु भ्रान्त‚त्वाद‚प्रा‚{\tiny $_{३}$}‚माण्य‚मित्यु‚{\tiny $_{lb}$}‚च्य‚ते ।
	{\color{gray}{\rmlatinfont\textsuperscript{§~\theparCount}}}
	\pend% ending standard par
      ‚{\tiny $_{lb}$}‚

	  
	  \pstart \leavevmode% starting standard par
	त‚द‚युक्तं [।] भ्रान्त‚स्याप्य‚ग्न्य‚नुमान‚स्य त‚दुत्प‚त्त्या बाह्याग्न्य‚ध्य‚व‚सायेन लोके‚{\tiny $_{lb}$}‚ प्रामाण्य‚द‚र्श‚नात् प्र‚त्य‚क्ष‚व‚त् । अथ प्र‚त्य‚क्ष‚म‚पि प्र‚माण‚न्नेष्य‚ते त‚दा लोक‚प्र‚तीति‚{\tiny $_{lb}$}‚बाधा । प्र‚त्य‚क्षानुमान‚योः प्र‚माण‚योर्लोक‚प्र‚तीत‚त्वात् ।
	{\color{gray}{\rmlatinfont\textsuperscript{§~\theparCount}}}
	\pend% ending standard par
      ‚{\tiny $_{lb}$}‚

	  
	  \pstart \leavevmode% starting standard par
	अथ नैव प्र‚त्य‚क्षानुमान‚योः प्र‚माण‚त्वं प्र‚तिषिध्य‚ते । किन्तु लिङ्ग‚न्त्रिल‚क्ष‚णं‚{\tiny $_{lb}$}‚ च‚तुर्ल‚क्ष‚णं वा न केन‚चित् प्र‚माणेन सिद्ध‚मिति प‚र्य‚{\tiny $_{४}$}‚नुयोगे य‚द्य‚नुमान‚मुच्य‚ते साध‚कं ।‚{\tiny $_{lb}$}‚ पुन‚स्त‚त्रापि सं एव प‚र्य‚नुयोग इत्येवं स‚र्व‚त्र प‚र्य‚नुयोग‚प‚राण्येव सूत्राणि । त‚था च‚{\tiny $_{lb}$}‚ \textbf{सूत्रं} [।]
	{\color{gray}{\rmlatinfont\textsuperscript{§~\theparCount}}}
	\pend% ending standard par
      ‚{\tiny $_{lb}$}‚
	    
	    \stanza[\smallbreak]
	  \textbf{
	  \bigskip
	  \begingroup
	विशेषेनुग‚माभावः सामान्ये सिद्ध‚साध‚न‚मि
	  \endgroup
	ति\edtext{}{\edlabel{pvsvt_26-1}\label{pvsvt_26-1}\lemma{ति}\Bfootnote{Pramāna-Samuccaya. }}} [।]\&[\smallbreak]
	  
	  
	  ‚{\tiny $_{lb}$}‚

	  
	  \pstart \leavevmode% starting standard par
	त‚द‚प्य‚युक्तं । पूर्व‚मेव त्रैरूप्य‚ग्राह‚क‚स्य प्र‚माण‚स्य व्याप्तिव‚चंनेनाक्षिप्त‚त्वात् ।‚{\tiny $_{lb}$}‚ \add{न चाप्र‚माण‚केन प‚रः}\edtext{\textsuperscript{*}}{\edlabel{pvsvt_26-2}\label{pvsvt_26-2}\lemma{*}\Bfootnote{In the margin. }} प‚र्य‚नुयुज्य‚ते वादिप्र‚तिवादिनोर‚सिद्ध‚त्वात् ।
	{\color{gray}{\rmlatinfont\textsuperscript{§~\theparCount}}}
	\pend% ending standard par
      ‚{\tiny $_{lb}$}‚

	  
	  \pstart \leavevmode% starting standard par
	अथ व‚च‚नात्म‚क‚म‚नुमा‚{\tiny $_{५}$}‚नं न व‚क्तुः प्र‚माण‚म‚थ वच‚नेन प‚रं प्र‚तिपाद‚य‚ति‚{\tiny $_{lb}$}‚ त‚थाऽप्र‚माण‚केन प‚र्य‚नुयोगः क्रिय‚त इति ।
	{\color{gray}{\rmlatinfont\textsuperscript{§~\theparCount}}}
	\pend% ending standard par
      ‚{\tiny $_{lb}$}‚

	  
	  \pstart \leavevmode% starting standard par
	त‚द‚प्य‚युक्तं । द्व‚योर‚पि हि व‚च‚नाद‚र्थ‚प्र‚तीतिः प्र‚माण‚भूतैवोत्प‚द्य‚तेऽर्थ‚प‚रिच्छेद‚{\tiny $_{lb}$}‚क‚त्वात् केव‚ल‚म्व‚क्तुर‚धिग‚म‚स्य निष्प‚न्न‚त्वात् प्र‚माणं नो वा तेन पुन‚र‚प्र‚माणं‚{\tiny $_{lb}$}‚ भ‚व‚त्य‚प्रामाण्ये वा द्व‚योर‚प्य‚प्र‚मा \edtext{\textsuperscript{*}}{\edlabel{pvsvt_26-2b}\label{pvsvt_26-2b}\lemma{*}\Bfootnote{२ In the margin.}}ण‚मिति क‚थ‚न्त‚तोर्थ‚प्र‚तीतिः ।
	{\color{gray}{\rmlatinfont\textsuperscript{§~\theparCount}}}
	\pend% ending standard par
      ‚{\tiny $_{lb}$}‚

	  
	  \pstart \leavevmode% starting standard par
	य‚द‚प्युच्य‚ते [।] प‚र‚सिद्धेनानुमानेनानुमा‚{\tiny $_{६}$}‚न‚न्निषिध्य‚तइति \textbf{त‚द‚प्येतेन}‚{\tiny $_{lb}$}‚ निर‚स्त‚मिति ॥ संयोग‚व‚शाद् ग‚म‚क‚त्वे ।
	{\color{gray}{\rmlatinfont\textsuperscript{§~\theparCount}}}
	\pend% ending standard par
      ‚{\tiny $_{lb}$}‚
	  \bigskip
	  \begingroup
	
	    
	    \stanza[\smallbreak]
	  {\normalfontlatin\large ``\qquad}न च केन‚चिद‚ङ्गेन न संयोगी हुताश‚नः ।&‚{\tiny $_{lb}$}‚धूमो वा स‚र्व‚था तेन प्राप्तं धूमात् प्र‚काश‚न‚मिति ।{\normalfontlatin\large\qquad{}"}\&[\smallbreak]
	  
	  
	  
	  \endgroup
	‚{\tiny $_{lb}$}‚

	  
	  \pstart \leavevmode% starting standard par
	यः स‚र्व‚था ग‚म्य‚ग‚म‚क‚भाव‚प्र‚संग आचार्य दि ग्ना गे नाक्षिप्तं प‚रं प्र‚ति‚{\tiny $_{lb}$}‚ त‚दिहापि कार्य\add{हेतौ आश‚ङ्क‚ते \textbf{य‚दीत्या}दिना}\edtext{}{\edlabel{pvsvt_26-2c}\label{pvsvt_26-2c}\lemma{कार्य}\Bfootnote{In the margin. }} साध्या\textbf{दुत्प‚त्तेः} कार‚णात् \textbf{कार्य-}‚{\tiny $_{lb}$}‚ \leavevmode\ledsidenote{\textenglish{27/s}}\textbf{ङ्ग‚म‚क‚ङ्}कार‚ण‚स्येत्य‚ध्याहारः \textbf{स‚र्व‚था ग‚म्य‚{\tiny $_{७}$}‚ग‚म‚क‚भावः प्राप्तः} । अग्नेः सामान्य- \leavevmode\ledsidenote{\textenglish{11a/PSVTa}}‚{\tiny $_{lb}$}‚ ध‚र्म‚व‚द्विशेष‚ध‚र्मा अपि तार्ण‚पार्ण्णाद‚यो ग‚म्याः स्युः । धूस‚स्यापि विशेष‚ध‚र्म‚ब‚द्‚{\tiny $_{lb}$}‚ द्र‚व्य‚त्व‚पार्थिव‚त्वाद‚योपि सामान्य‚ध‚र्मा ग‚म‚का भ‚वेयुः । कुतः । \textbf{स‚र्व‚था ज‚न्य‚ज‚न‚क‚{\tiny $_{lb}$}‚भावा}त् [।] त‚था हि य‚थाग्निर‚ग्नित्व‚द्र‚व्य‚त्व‚स‚त्त्वादिभिः \add{सामान्य‚ध‚र्म्मैर्ज‚न‚कः‚{\tiny $_{lb}$}‚ त‚था तार्ण्ण‚पार्ण्णादिभि}\edtext{}{\edlabel{pvsvt_27-1}\label{pvsvt_27-1}\lemma{त्त्वादिभिः}\Bfootnote{In the margin. }}र्विशेषैर‚पि । य‚था च धूमो धूम‚त्व‚पाण्डुत्वादिभिः‚{\tiny $_{lb}$}‚ स्व‚निय‚तैर्विशेष‚ध‚र्मै‚{\tiny $_{१}$}‚र्युक्तो ज‚न्य‚स्त‚था सामान्य‚ध‚र्मैर‚पि स‚त्त्व‚द्र‚व्य‚त्वादिभिस्त‚त‚श्च‚{\tiny $_{lb}$}‚ य‚थान‚योः कार्य‚कार‚ण‚भाव‚स्त‚थैव ग‚म्य‚ग‚म‚क‚भावः स्यादित्य‚त आह ।
	{\color{gray}{\rmlatinfont\textsuperscript{§~\theparCount}}}
	\pend% ending standard par
      ‚{\tiny $_{lb}$}‚

	  
	  \pstart \leavevmode% starting standard par
	\textbf{ने}त्यादि । न स‚र्व‚था ज‚न्य‚ज‚न‚क‚भाव‚स्त‚त‚श्च कुत‚स्त‚था ग‚म्य‚ग‚म‚क‚भावः स्यात् ।‚{\tiny $_{lb}$}‚ क‚स्मादिति चेत् । त‚द‚भावे तेषान्तार्ण्ण\add{पार्ण्ण‚त्वादीनां विशेष‚ध‚र्म्माणाम‚भावे}\edtext{}{\edlabel{pvsvt_27-1b}\label{pvsvt_27-1b}\lemma{तेषान्तार्ण्ण}\Bfootnote{In the margin. }}‚{\tiny $_{lb}$}‚ भ‚व‚तो धूम‚मात्र‚स्य तेभ्य एव विशेष‚ध‚र्मेभ्यो भ‚व‚तीत्येव‚मात्म‚न‚स्त‚{\tiny $_{२}$}‚दुत्प‚त्तिनिय‚म‚{\tiny $_{lb}$}‚स्याभावात् । त‚था \textbf{त‚द‚भावे}ऽग्न्य‚भावे \textbf{भ‚व‚तो} द्र‚व्य‚त्वादेः सामान्य‚ध‚र्म‚स्याग्नेरेवायं‚{\tiny $_{lb}$}‚ भ‚व‚तीत्येवं--रूप‚स्य \textbf{त‚दुत्प‚त्तिनिय‚म}स्या\textbf{भावात्} । कुतः स‚र्व‚था ज‚न्य‚ज‚न‚क‚भावो‚{\tiny $_{lb}$}‚ य‚तः स‚र्व‚था ग‚म्य‚ग‚म‚क‚भावः स्यात् ॥
	{\color{gray}{\rmlatinfont\textsuperscript{§~\theparCount}}}
	\pend% ending standard par
      ‚{\tiny $_{lb}$}‚

	  
	  \pstart \leavevmode% starting standard par
	य‚त \textbf{एव}न्त‚स्मात् \textbf{कार्यं} धूमादिकं \textbf{स्व‚भावैर्या} \add{\textbf{व‚द्भि}र्धूम‚त्वादिभिः स्व‚ग‚तै}\edtext{\textsuperscript{*}}{\edlabel{pvsvt_27-1c}\label{pvsvt_27-1c}\lemma{*}\Bfootnote{In the margin. }}‚{\tiny $_{lb}$}‚रित्थंभूत‚ल‚क्ष‚णा तृतीया । \textbf{अविनाभावि} । विना न भ‚व‚ति । क्वाविनाभावि [।]‚{\tiny $_{lb}$}‚ \textbf{कार‚{\tiny $_{३}$}‚णे} । कार‚ण‚विष‚ये । य‚द्वा कार‚णे इत्याधार‚स‚प्त‚मी । कार‚ण‚स्थैः स्व‚भावै‚{\tiny $_{lb}$}‚र्याव‚द्भिर‚ग्नित्व‚द्र‚व्य‚त्वादिभिर‚विनाभावि । \textbf{तेषां} कार‚ण‚ग‚तानां सामान्य‚ध‚र्माणां‚{\tiny $_{lb}$}‚ \textbf{हेतुः} कार्यं ग‚म‚क‚मित्य‚र्थः । किङ्कार‚णं [।] \textbf{त‚त्कार्य‚त्व‚निय‚मा}त् । तेषामेव‚{\tiny $_{lb}$}‚ कार‚ण‚ग‚तानां सामान्य‚ध‚र्माणान्त‚त्कार्य‚मित्ये\add{वं रूप‚स्य निय}\edtext{}{\edlabel{pvsvt_27-1d}\label{pvsvt_27-1d}\lemma{मित्ये}\Bfootnote{In the margin. }}म‚स्य स‚द्भावात् ।‚{\tiny $_{lb}$}‚ न हि त‚त्सामान्य‚ध‚र्मात् क‚दाचिद‚पि कार्यं व्य‚भिच‚र‚ति । एव‚न्ताव‚{\tiny $_{४}$}‚त् कार‚ण‚ग‚ताः‚{\tiny $_{lb}$}‚ सामान्य‚ध‚र्मा ग‚म्या इत्याख्यातं ।
	{\color{gray}{\rmlatinfont\textsuperscript{§~\theparCount}}}
	\pend% ending standard par
      ‚{\tiny $_{lb}$}‚

	  
	  \pstart \leavevmode% starting standard par
	कार्य‚ग‚तास्तु विशेष‚ध‚र्मा ग‚म‚का इति द‚र्श‚य‚न्नाह । \textbf{तैरेवे}त्यादि । कार्य‚म‚पि‚{\tiny $_{lb}$}‚ \textbf{तैरेव} ध‚र्मैः स्व‚ग‚तैः कार‚ण‚ग‚तानां ध‚र्माणां ग‚म‚काः । येर्थान्त‚रास‚म्भ‚विनो धूम‚त्व‚पा‚{\tiny $_{lb}$}‚‚{\tiny $_{lb}$}‚ ‚{\tiny $_{lb}$}‚ \leavevmode\ledsidenote{\textenglish{28/s}}ण्डुत्वाद‚यो विशेष‚रूपा\textbf{स्तैः} कार‚ण‚ग‚तैः सामान्य‚ध‚र्मै\textbf{र्विना न भ‚व‚न्ति} । अत्रापि‚{\tiny $_{lb}$}‚ त‚त्कार्य‚त्व‚निय‚मादित्य‚पेक्ष्य‚ते । तेषामेव कार्य‚ग‚तानां विशेष‚{\tiny $_{५}$}‚ध‚र्माणां कार‚ण‚ग‚त‚{\tiny $_{lb}$}‚सामान्य‚ध‚र्मापेक्ष‚या कार्य‚त्व‚निय‚मात् ॥
	{\color{gray}{\rmlatinfont\textsuperscript{§~\theparCount}}}
	\pend% ending standard par
      ‚{\tiny $_{lb}$}‚

	  
	  \pstart \leavevmode% starting standard par
	\textbf{य‚दि} सामान्य‚ध‚र्माणां कार‚ण‚ग‚तानां कार्य‚ग‚तैर्विशेष‚ध‚र्मैरेवाविनाभावाद्‚{\tiny $_{lb}$}‚ ग‚म्य‚ग‚म‚क‚भाव‚स्त‚दां\textbf{शेन ज‚न्य‚ज‚न‚क‚भावः} स्यात् । अग्नेः सामान्य‚ध‚र्मा एव ज‚न‚काः‚{\tiny $_{lb}$}‚ [।] धूम‚स्य च विशेष‚ध‚र्मा एव ज‚न्याः स्युः । स‚र्व‚था च ज‚न्य‚ज‚न‚क‚भावोभिम‚त‚{\tiny $_{lb}$}‚ इत्य‚भ्युप‚ग‚म‚विरोधः ॥ एत‚त्प‚रिह‚र‚ति [।] नांशेन‚{\tiny $_{६}$}‚ ज‚न्य‚ज‚न‚क‚भाव‚प्र‚स‚ङ्गः ।‚{\tiny $_{lb}$}‚ निरंश‚त्वेन व‚स्तुनः स‚र्व‚था ज‚न्य‚ज‚न‚क‚त्वाभ्युप‚ग‚मात् । ग‚म्य‚ग‚म‚क‚भाव‚स्यापि‚{\tiny $_{lb}$}‚ स‚र्व‚थाभिम‚त‚त्वात् । त‚दाह । \textbf{त‚ज्ज‚न्ये}त्यादि । य‚दि हि कार्य‚स्य तैः‚{\tiny $_{lb}$}‚ कार‚ण‚ग‚तैर्विशेष‚ध‚र्मैर्ज‚न्यो यो विशेषः स ग्र‚हीतुं श‚क्य‚ते ज्ञाप‚क‚हेत्व‚धि‚{\tiny $_{lb}$}‚कारात् । त‚दा \textbf{त‚ज्ज‚न्य‚विशेष‚ग्र‚ह}णेऽभिम‚त‚त्वात् कार‚ण‚ग‚त‚विशेष‚ध‚र्माणां‚{\tiny $_{lb}$}‚ \leavevmode\ledsidenote{\textenglish{11b/PSVTa}} ग‚म्य‚त्व‚स्य । त‚था ह्य‚गुरु‚{\tiny $_{७}$}‚धूम‚ग्र‚ह‚णे भ‚व‚त्येव त‚द‚ग्नेर‚नुमानं । त‚था लिङ्ग‚विशेषो‚{\tiny $_{lb}$}‚ लिङ्ग‚मेव विशेषो धूम‚ल‚क्ष‚णः स उपाधिर्विशेष‚णं येषां द्र‚व्य‚त्वादीनान्तेषां‚{\tiny $_{lb}$}‚ \textbf{ग्र‚ह‚णेऽभिम‚त‚त्वाद्} ग‚म‚क‚त्व‚स्य । न हि धूमेन विशेषिता द्र‚व्य‚त्वाद‚योऽग्निं‚{\tiny $_{lb}$}‚ व्य‚भिच‚र‚न्ति ॥
	{\color{gray}{\rmlatinfont\textsuperscript{§~\theparCount}}}
	\pend% ending standard par
      ‚{\tiny $_{lb}$}‚

	  
	  \pstart \leavevmode% starting standard par
	न‚नु धूम एव त‚त्र ग‚म‚को न तु त‚द्विशिष्टा द्र‚व्य‚त्वाद‚यः । य‚था कृत‚क‚त्वे स‚ति‚{\tiny $_{lb}$}‚ प्र‚मेय‚त्वादित्य‚त्र कृत‚क‚त्व‚मेव ग‚म‚कं न प्र‚मेय‚त्वं ।
	{\color{gray}{\rmlatinfont\textsuperscript{§~\theparCount}}}
	\pend% ending standard par
      ‚{\tiny $_{lb}$}‚

	  
	  \pstart \leavevmode% starting standard par
	स‚त्य‚मेत‚द् । अव्य‚भि‚{\tiny $_{१}$}‚चार‚मात्र‚प्र‚द‚र्श‚नार्थ‚न्त्वेव‚म‚भिधान‚मित्येके । अन्य‚स्त्वाह ।‚{\tiny $_{lb}$}‚ न धूम‚स्य व्य‚भिचारादिह सामान्योपादानं किन्त‚र्हि स‚र्वेषां प्र‚तिप‚त्तॄणां दृष्टे‚{\tiny $_{lb}$}‚ व‚स्तुनि सामान्याकारे प्र‚तिप‚त्तिर्भ‚व‚ति प‚श्चाद् विशेषाव‚सायः [।] त‚त्र च य‚दुपात्तं‚{\tiny $_{lb}$}‚ सामान्य‚न्त‚द‚प‚रित्य‚क्त‚मेव । त‚स्मात् प्र‚तिप‚त्तुर‚ध्य‚व‚साय‚व‚शाद् विशेषोप‚हितं‚{\tiny $_{lb}$}‚ सामान्य‚ङ्ग‚म‚क‚म्भ‚व‚ति न विशेष‚स्य व्य‚भिचारादिति ।
	{\color{gray}{\rmlatinfont\textsuperscript{§~\theparCount}}}
	\pend% ending standard par
      ‚{\tiny $_{lb}$}‚

	  
	  \pstart \leavevmode% starting standard par
	यु‚{\tiny $_{२}$}‚क्त‚मेत‚त् । केव‚लं य‚द्येष निय‚मः सामान्य‚प्र‚तिप‚त्तिपुर‚स्स‚रैव विशेष‚प्र‚तिप‚त्तिः‚{\tiny $_{lb}$}‚ [।] क‚थ‚न्त‚र्हि धूम‚मात्र‚स्य द्र‚व्य‚त्वादिर‚हित‚स्य प्र‚तीतिः । पूर्वोक्तं च चोद्य‚न्त‚द‚{\tiny $_{lb}$}‚व‚स्थ‚मेव । त‚स्मादिद‚म‚त्र साधु [।] लिङ्ग‚विशेष‚स्य सामान्य‚विशेष‚ण‚त्वेनैवोपादाना‚{\tiny $_{lb}$}‚ द् अहेतुत्वं हेतुत्वोपादाने हि हेतुत्वं स्यान्नान्य‚था ।
	{\color{gray}{\rmlatinfont\textsuperscript{§~\theparCount}}}
	\pend% ending standard par
      ‚{\tiny $_{lb}$}‚

	  
	  \pstart \leavevmode% starting standard par
	क‚दा त‚र्हि लिङ्ग‚ग‚तानां सामान्य‚ध‚र्माणाम‚ग‚म‚क‚त्व‚मित्याह‚{\tiny $_{३}$}‚ । \textbf{अविशिष्टे}‚{\tiny $_{lb}$}‚त्यादि । \textbf{य‚दा द्र‚व्य‚त्वादीन्य‚विशिष्टानि विव‚क्षितानि त‚दा तेषां व्य‚भिचाराद्}‚{\tiny $_{lb}$}‚ ‚{\tiny $_{lb}$}‚ \leavevmode\ledsidenote{\textenglish{29/s}}\textbf{ग‚म‚क‚त्व‚न्नेष्य‚ते} ।
	{\color{gray}{\rmlatinfont\textsuperscript{§~\theparCount}}}
	\pend% ending standard par
      ‚{\tiny $_{lb}$}‚

	  
	  \pstart \leavevmode% starting standard par
	स्व‚भाव‚हेतुम‚धिकृत्याह ।
	{\color{gray}{\rmlatinfont\textsuperscript{§~\theparCount}}}
	\pend% ending standard par
      ‚{\tiny $_{lb}$}‚

	  
	  \pstart \leavevmode% starting standard par
	\textbf{स्व‚भाव} इत्यादि । \textbf{हेतुरिति व‚र्त्त‚त} इति तेषां हेतुरित्य‚तः । स्व‚भावे साध्ये‚{\tiny $_{lb}$}‚ किम्भूते \textbf{भाव‚मात्रानुरोधिनि} हेतुस‚द्भाव‚मात्रानुरोधिनि \add{\textbf{भावो} हेतुः}\edtext{}{\edlabel{pvsvt_29-1}\label{pvsvt_29-1}\lemma{मात्रानुरोधिनि}\Bfootnote{In the margin. }}स्व‚भावो‚{\tiny $_{lb}$}‚ हेतुः । मात्र‚ग्र‚ह‚ण‚म‚र्थान्त‚रान‚पेक्षास‚न्द‚र्श‚नार्थं ॥ क‚स्मात्त‚न्मात्रा‚{\tiny $_{४}$}‚नुरोधिन्येव स्व‚भावो‚{\tiny $_{lb}$}‚ हेतुरित्याह । \textbf{तादात्म्यं} ह्य‚र्थ‚स्य \textbf{त‚न्मात्रानुरोधिन्येवे}ति । योसाव‚र्थ‚स्य साध‚न‚स्यात्मा‚{\tiny $_{lb}$}‚ त‚द्भाविन्येव । \textbf{नान्याय‚त्ते} । न कार‚णान्त‚र‚प्र‚तिब‚द्धे प‚श्चाद्भाविनि तादात्म्यं ।‚{\tiny $_{lb}$}‚ क‚स्मादिति चेदाह । \textbf{त‚द्भाव} इत्यादि । त‚स्य हेतो\textbf{र्भावि\textbf{नि}}ऽभूत‚स्य कार\add{णान्त‚{\tiny $_{lb}$}‚राय‚त्त‚स्य ध‚र्म‚स्य \textbf{प‚श्चाद्} यो \textbf{भा}}\edtext{}{\edlabel{pvsvt_29-1b}\label{pvsvt_29-1b}\lemma{कार}\Bfootnote{In the margin. }}व‚स्त‚स्य \textbf{निय‚माभावात्} । न हि कार‚णान्त‚र‚{\tiny $_{lb}$}‚प्र‚तिब‚द्धेन प‚श्चाद्भाविनाऽव‚{\tiny $_{५}$}‚श्यं भ‚वित‚व्यं । किङ्कार‚णं [।] \textbf{कार‚णानां कार्य‚{\tiny $_{lb}$}‚व्य‚भिचारा}त् । स‚म्भ‚व‚त्प्र‚तिब‚न्ध‚त्वात् कार‚णानां कुत‚स्तेभ्योऽव‚श्य‚म्भावः कार्य‚स्य ॥
	{\color{gray}{\rmlatinfont\textsuperscript{§~\theparCount}}}
	\pend% ending standard par
      ‚{\tiny $_{lb}$}‚

	  
	  \pstart \leavevmode% starting standard par
	न‚नु च साध्य‚स्व‚भाव‚ता साध‚न‚स्य न केन‚चिदिष्टं त‚त्क‚थ‚मुच्य‚ते \textbf{त‚द्भाव‚मात्रा‚{\tiny $_{lb}$}‚नुरोधिन्येव तादा}त्म्य‚मिति । एव‚म्म‚न्य‚ते व्य‚ति\add{रिक्ताव‚पि कृत‚क‚त्वानित्य‚त्वा‚{\tiny $_{lb}$}‚ख्यौ}\edtext{}{\edlabel{pvsvt_29-1c}\label{pvsvt_29-1c}\lemma{ति}\Bfootnote{In the margin. }} ध‚र्माव‚प्युप‚ग‚च्छ‚द्भिर‚व‚श्य‚म‚भूत्वा भ‚व‚नं भूत्वा चाभ‚व‚न‚म‚भ्युप‚ग‚न्त‚व्य‚{\tiny $_{६}$}‚‚{\tiny $_{lb}$}‚म‚न्य‚थात्मादेरिव कृत‚क‚त्वानित्य‚त्वे प‚टादेर्न स्यातां । त‚स्माद् य‚देवाभूत्वा भ‚व‚न‚{\tiny $_{lb}$}‚म्भाव‚स्य त‚देव कृत‚क‚त्वं य‚देव च भूत्वाऽभ‚व‚न‚म‚न‚व‚स्थायित्व‚न्त‚देवानित्य‚त्व‚म‚स्तु‚{\tiny $_{lb}$}‚ किम‚न्येन सामान्येन क‚ल्पितेनेति ॥ अनुप‚ल{...}इत्याह ।
	{\color{gray}{\rmlatinfont\textsuperscript{§~\theparCount}}}
	\pend% ending standard par
      ‚{\tiny $_{lb}$}‚‚{\tiny $_{lb}$}‚‚{\tiny $_{lb}$}‚\textsuperscript{\textenglish{30/s}}

	  
	  \pstart \leavevmode% starting standard par
	\textbf{अप्र‚वृ} \add{\textbf{त्ति}रित्यादि । केषाम‚प्र‚वृत्तिः \textbf{प्र‚मा}}\edtext{\textsuperscript{*}}{\edlabel{pvsvt_30-1}\label{pvsvt_30-1}\lemma{*}\Bfootnote{In the margin. }}\textbf{णाना}म्[।]व‚हुव‚च‚नं व्य‚क्तिभेदेन‚{\tiny $_{lb}$}‚ प्र‚माणानाम्ब‚हुत्वात् । आग‚मापेक्ष‚{\tiny $_{७}$}‚{... ...}\edtext{}{\edlabel{pvsvt_30-2}\label{pvsvt_30-2}\lemma{मापेक्ष}\Bfootnote{12th leaf is missing. }}
	{\color{gray}{\rmlatinfont\textsuperscript{§~\theparCount}}}
	\pend% ending standard par
      ‚{\tiny $_{lb}$}‚

	  
	  \pstart \leavevmode% starting standard par
	\leavevmode\ledsidenote{\textenglish{13a/PSVTa}} सिध्य‚तीत्युच्य‚त इति ।
	{\color{gray}{\rmlatinfont\textsuperscript{§~\theparCount}}}
	\pend% ending standard par
      ‚{\tiny $_{lb}$}‚

	  
	  \pstart \leavevmode% starting standard par
	एव‚म्म‚न्य‚ते । ज्ञान‚ज्ञेय‚योर्बाधा बोध‚रूप‚त्वेन विशेषाद् । बोध‚रूपं प्र‚त्य‚क्षादिकं‚{\tiny $_{lb}$}‚ प्र‚माणं स्व‚त एव सिध्य‚ति [।] ज्ञेय‚न्तु घ‚टादिकं ज‚ड‚रूप‚त्वात् प्र‚माण‚म‚पेक्ष‚ते ।‚{\tiny $_{lb}$}‚ ज्ञान‚ज्ञेयाभाव‚योस्तु नीरूप‚त्वेन विशेषाभावात् क‚थं ज्ञा\add{नाभाव‚स्य स्व‚तःसिद्धि‚{\tiny $_{lb}$}‚ र्ज्ञेयाभाव‚स्य च}\edtext{}{\edlabel{pvsvt_30-3}\label{pvsvt_30-3}\lemma{ज्ञा}\Bfootnote{३ }}ज्ञानाभावात् सिद्धिरुच्य‚ते । अथ ज्ञानाभावो नान्येन सिध्य‚ति ।‚{\tiny $_{lb}$}‚ त‚था हि‚{\tiny $_{१}$}‚ ज्ञानानां स्व‚स‚म्विदित‚रूप‚त्वेनैक‚ज्ञान‚संस‚र्गित्वाभावात् । केव‚लं य‚दि‚{\tiny $_{lb}$}‚ स्व‚स‚न्ताने ज्ञानं स्याद् उप‚ल‚भ्येतानुप‚ल‚म्भाद‚स‚देव त‚दिति स्व‚त एव ज्ञानाभावः‚{\tiny $_{lb}$}‚ सिद्ध इष्य‚ते ।
	{\color{gray}{\rmlatinfont\textsuperscript{§~\theparCount}}}
	\pend% ending standard par
      ‚{\tiny $_{lb}$}‚

	  
	  \pstart \leavevmode% starting standard par
	\textbf{त‚था स‚त्ताऽभावोपि सिद्धः स्यात्} । त‚त्रापि हि य‚दि स‚त्ता स्यादुप‚ल‚भ्येतानुप‚{\tiny $_{lb}$}‚\add{ल‚म्भान्नास्तीति निश्चीय‚ते त‚त‚श्चा}\edtext{}{\edlabel{pvsvt_30-3b}\label{pvsvt_30-3b}\lemma{भ्येतानुप}\Bfootnote{३ }}\textbf{पार्थिकानुप‚ल‚ब्धि}र‚भाव‚सिद्धौ । विज्ञानं‚{\tiny $_{lb}$}‚ वान्य‚व‚स्तुनीति प‚क्षं दूष‚यितुमा‚{\tiny $_{२}$}‚ह । \textbf{अथे}त्यादि । \textbf{अन्य}स्य घ‚टादिविविक्त‚स्य‚{\tiny $_{lb}$}‚ भूत‚ल‚स्यो\textbf{प‚ल‚ब्ध्या} घ‚टा\textbf{नुप‚ल‚ब्धिसिद्धिरिति प्र‚त्य‚क्ष‚सिद्धानुप‚ल‚ब्धिः} ।
	{\color{gray}{\rmlatinfont\textsuperscript{§~\theparCount}}}
	\pend% ending standard par
      ‚{\tiny $_{lb}$}‚

	  
	  \pstart \leavevmode% starting standard par
	एत‚दुक्त‚म्भ‚व‚ति । घ‚ट‚ग्राह‚क‚त्व‚स्य भूत‚ल‚ग्राह‚क‚स्य चैक‚ज्ञान‚संस‚र्गित्वाद्‚{\tiny $_{lb}$}‚ य‚दा भूत‚ल‚ग्राह‚क‚मेव त‚ज्ज्ञान‚म्भ‚व‚ति । त‚दा घ‚टाग्राह‚क‚त्वाभावं निश्चाय‚य‚तीति‚{\tiny $_{lb}$}‚ प्र‚तीतिप्रत्य‚क्ष‚सिद्धैव घ‚टानुप‚ल‚ब्धिः ॥
	{\color{gray}{\rmlatinfont\textsuperscript{§~\theparCount}}}
	\pend% ending standard par
      ‚{\tiny $_{lb}$}‚\textsuperscript{\textenglish{31/s}}

	  
	  \pstart \leavevmode% starting standard par
	\textbf{त‚थान्य‚स‚त्त‚याऽस‚त्ता किन्न सिध्य‚ति} । त‚थेत्य‚{\tiny $_{३}$}‚नुप‚ल‚ब्धिव‚त् । द्व‚योर‚पि‚{\tiny $_{lb}$}‚ घ‚ट‚प्र‚देश‚योरेक‚ज्ञान‚संस‚र्गित्वादित्य‚भिप्रायः । अन्य‚स्य घ‚ट‚विविक्त‚स्य भूत‚लादेः‚{\tiny $_{lb}$}‚ स‚त्त‚या सिद्ध्या निषेध्य‚स्यार्थ‚स्य स‚त्ता किन्न सिध्य‚ति ॥
	{\color{gray}{\rmlatinfont\textsuperscript{§~\theparCount}}}
	\pend% ending standard par
      ‚{\tiny $_{lb}$}‚

	  
	  \pstart \leavevmode% starting standard par
	न‚नु भाव‚निवृत्तिरूपोऽभावः स क‚थं प्र‚त्य‚क्ष‚सिद्ध इत्युच्य‚ते ।
	{\color{gray}{\rmlatinfont\textsuperscript{§~\theparCount}}}
	\pend% ending standard par
      ‚{\tiny $_{lb}$}‚

	  
	  \pstart \leavevmode% starting standard par
	एव‚म्म‚न्य‚ते । अभावो नाम नास्त्येव केव‚लं मूढ‚स्य भाव‚विष‚य‚मेव प्र‚त्य‚क्ष‚{\tiny $_{lb}$}‚म‚न्या\textbf{भावं} व्य‚व‚हार‚य‚ति । तेन य‚दुक्त‚{\tiny $_{४}$}‚\textbf{म र्था प त्त्या}ऽभावः प्र‚तीय‚त इति त‚द‚युक्तं ।‚{\tiny $_{lb}$}‚ य‚तो न ताव‚द् घ‚टादीनाम‚न्योन्याभावोऽभिन्नः घ‚ट‚विनाशे प‚टाद्युत्प‚त्तिप्र‚स‚ङ्गात् ।‚{\tiny $_{lb}$}‚ प‚टाद्य‚भाव‚स्य विन‚ष्ट‚त्वात् । अथ भिन्नोऽभाव‚स्त‚दा घ‚टादीनां प‚र‚स्प‚रं भेदो न‚{\tiny $_{lb}$}‚ स्यात् । य‚दा हि घ‚टाभाव‚रूपः प‚टो न भ‚व‚ति त‚दा प‚टो घ‚ट एव स्यात् । य‚था वा‚{\tiny $_{lb}$}‚ घ‚ट‚स्य प‚टाभावाद् भिन्न‚त्वाद् घ‚ट‚रूप‚ता त‚था प‚टा‚{\tiny $_{५}$}‚देर‚पि स्यात् । घ‚टाभावाद्‚{\tiny $_{lb}$}‚ भिन्न‚त्वादेव ।
	{\color{gray}{\rmlatinfont\textsuperscript{§~\theparCount}}}
	\pend% ending standard par
      ‚{\tiny $_{lb}$}‚

	  
	  \pstart \leavevmode% starting standard par
	नाप्येषां प‚र‚स्प‚राभिन्नानाम‚भावे न भेदः श‚क्य‚ते क‚र्त्तुं । त‚स्य भिन्नाभिन्न‚{\tiny $_{lb}$}‚भेद‚क‚र‚णेऽकिंचित्क‚र‚त्वात् । न चाभिन्नानाम‚न्योन्याभावः स‚म्भ‚व‚ति । नापि‚{\tiny $_{lb}$}‚ प‚र‚स्प‚र‚भिन्नानाम‚भावेन भेदः क्रिय‚ते स्व‚हेतुभ्य एव भिन्नानामुत्प‚त्तेः । नापि भेद‚{\tiny $_{lb}$}‚व्य‚व‚हारः क्रिय‚ते । य‚तो भावानामात्मीयात्मीय \edtext{}{\lemma{भावानामात्मीयात्मीय}\Bfootnote{?}} रूपेणोत्प‚त्तिरेव स्व‚तो भे‚{\tiny $_{६}$}‚दः‚{\tiny $_{lb}$}‚ [।] स च प्र‚त्य‚क्ष‚प्र‚तिभास‚नादेव भेद‚व्य‚व‚हार‚हेतुः ।
	{\color{gray}{\rmlatinfont\textsuperscript{§~\theparCount}}}
	\pend% ending standard par
      ‚{\tiny $_{lb}$}‚

	  
	  \pstart \leavevmode% starting standard par
	\hphantom{.}तेन य‚दुच्य‚ते व‚स्त्व‚संक‚र‚सिद्धिश्चाभाव‚प्र‚माणाश्रितेति त‚द‚पास्तं । किञ्च‚{\tiny $_{lb}$}‚ [।] भावाभाव‚योर्भेदो नाभाव‚निब‚न्ध‚नोऽन‚व‚स्थाप्र‚संगात् । अथ स्व‚रूपेण भेद‚स्त‚था‚{\tiny $_{lb}$}‚भावानाम‚पि स स्यादिति किम‚भावेन क‚ल्पितेन [।] नापि प्राग‚भावाभावे कार्य‚स्या‚{\tiny $_{lb}$}‚नादित्वं प्र‚स‚ज्य‚ते । हेत्व‚भावेनानुत्प‚त्तेः ।
	{\color{gray}{\rmlatinfont\textsuperscript{§~\theparCount}}}
	\pend% ending standard par
      ‚{\tiny $_{lb}$}‚

	  
	  \pstart \leavevmode% starting standard par
	न‚नु‚{\tiny $_{७}$}‚ \textbf{प्राग‚भावे} स‚ति हेतोः स‚काशादुत्प‚त्तिः स्यान्नास‚ति प्राग‚भावे विद्य‚मान- \leavevmode\ledsidenote{\textenglish{13b/PSVTa}}‚{\tiny $_{lb}$}‚ त्वात् ।
	{\color{gray}{\rmlatinfont\textsuperscript{§~\theparCount}}}
	\pend% ending standard par
      ‚{\tiny $_{lb}$}‚

	  
	  \pstart \leavevmode% starting standard par
	य‚द्येव‚न्न क‚दाच‚नापि कार्योत्प‚त्तिः स्याद् विरोधिनः प्राग‚भाव‚स्य स‚न्निहित‚त्वात् ।‚{\tiny $_{lb}$}‚ न च त‚द्विनाशात् कार्योत्प‚त्तिः प्राग‚भाव‚म‚न्त‚रेण कार्योत्प‚त्त्य‚भ्युप‚ग‚म‚प्र‚स‚ङ्गात् ।‚{\tiny $_{lb}$}‚ नापि कार्योत्प‚त्तिरेव प्राग‚भाव‚विनाश‚स्त‚दुत्प‚त्तेरेव विरोधिस‚न्निधानेनास‚म्भ‚वात् ।‚{\tiny $_{lb}$}‚ कार‚ण‚स‚त्ताकाले प्राग‚भाव‚स्याविनाशात् ।‚{\tiny $_{१}$}‚ कार्योत्प‚त्तिकाले च त‚द्विनाशात्‚{\tiny $_{lb}$}‚ कार‚ण‚विनाश‚व‚त् । त‚स्मादुत्प‚त्तेः पूर्वं कार्य‚स्य न भावो नाप्य‚भावो ध‚र्मोस‚त्त्वात् ।‚{\tiny $_{lb}$}‚ निरंश‚त्वाच्च व‚स्तुनः । किन्तु य‚दोत्प‚द्य‚ते त‚दा स‚त्त्व‚म‚स्यान्य‚दा नास्तीति व्य‚व‚ह्रिय‚ते ।‚{\tiny $_{lb}$}‚ तेनास‚दुत्प‚द्य‚त इत्युच्य‚ते ।
	{\color{gray}{\rmlatinfont\textsuperscript{§~\theparCount}}}
	\pend% ending standard par
      ‚{\tiny $_{lb}$}‚

	  
	  \pstart \leavevmode% starting standard par
	\textbf{प्र ध्वं सा भा व स्य} चास‚त्त्वं स्व‚य‚मेवा\textbf{चार्यो}भिधास्य‚ते । य‚च्च य‚स्मादुत्प‚द्य‚ते‚{\tiny $_{lb}$}‚ ‚{\tiny $_{lb}$}‚ \leavevmode\ledsidenote{\textenglish{32/s}}त‚त्त‚स्य कार्यं कार‚णं चोच्य‚ते । तेषां चैक‚क्ष‚ण‚स्थायि‚{\tiny $_{२}$}‚त्वेनोत्प‚त्तेर्नाशित्वं स्व‚स्व‚{\tiny $_{lb}$}‚रूपेणैवोत्प‚त्तेः प‚र‚स्प‚र‚भिन्न‚ता च सिध्य‚ति । तेन प्राग‚भावाद्य‚भावेपि कार‚णा‚{\tiny $_{lb}$}‚दिविभाग‚तो व्य‚व‚हारो भ‚व‚त्येव । न च प्राग‚भावादीनाम्प‚र‚स्प‚र‚म्भेदः प्र‚तिभास‚ते ।‚{\tiny $_{lb}$}‚ य‚स्माद् घ‚टादेः पूर्व‚म्प‚श्चाद‚न्य‚त्र च निवृत्तिमात्र‚म‚भिन्नं प्र‚तिभास‚ते । य‚दि नाम‚{\tiny $_{lb}$}‚ काल‚भेदः प्र‚तीय‚ते । न हि गोत्व‚म‚नेक‚कालादिस‚म्ब‚न्धित्वेन प्र‚तीय‚मान‚म‚{\tiny $_{३}$}‚नेक‚म्भ‚{\tiny $_{lb}$}‚व‚ति । निवृत्तेर्नीरूप‚त्वाच्च क‚थ‚म‚भाव‚स्य नानात्वं भाव‚निवृत्तिरूप‚त्वाच्चाभाव‚स्य ।‚{\tiny $_{lb}$}‚ केव‚लं यो मूढ उत्प‚त्तेः पूर्वं प‚श्चाद‚न्य‚त्र च कार्य‚स्य भाव‚मिच्छ‚ति तं प्र‚तीद‚मुच्य‚ते‚{\tiny $_{lb}$}‚ [।] कार्य‚स्य पूर्व‚म्प‚श्चाद‚न्य‚त्र चाभाव इति भावारोप‚निषेध‚मात्रं क्रिय‚ते ।
	{\color{gray}{\rmlatinfont\textsuperscript{§~\theparCount}}}
	\pend% ending standard par
      ‚{\tiny $_{lb}$}‚

	  
	  \pstart \leavevmode% starting standard par
	तेन । \textbf{न चाव}स्तुन एते स्युराकारा इत्यादि य‚दुक्त‚न्त‚न्निर‚स्तं ।
	{\color{gray}{\rmlatinfont\textsuperscript{§~\theparCount}}}
	\pend% ending standard par
      ‚{\tiny $_{lb}$}‚

	  
	  \pstart \leavevmode% starting standard par
	इत‚श्चैत‚न्निर‚स्तं द्र‚ष्ट‚व्यं य‚तो न प्राक्प्र‚ध्वंसा‚{\tiny $_{४}$}‚भावाभ्याम्भाव‚स्य क‚श्चित्‚{\tiny $_{lb}$}‚ त‚दुत्प‚त्तिल‚क्ष‚णो विरोध‚ल‚क्ष‚णो वा स‚म्ब‚न्धोऽस‚ह‚भावित्वेनाद्विष्ठ‚त्वात् । अत एव न‚{\tiny $_{lb}$}‚ विशेष‚ण‚विशेष्य‚भावः स‚म्ब‚न्धः । नापि विशेष‚ण‚विशेष्य‚भावोऽस‚ह‚भावित्वादेव ।‚{\tiny $_{lb}$}‚ विशेष‚ण‚विशेष्य‚रूप‚तायाश्च व‚स्तुनोऽभावात् । केव‚ल\add{म‚न्य‚स‚म्ब‚न्ध‚द्वारेणा}\edtext{}{\edlabel{pvsvt_32-1}\label{pvsvt_32-1}\lemma{ल}\Bfootnote{In the margin. }}यं‚{\tiny $_{lb}$}‚ क‚ल्प्य‚ते । द‚ण्ड‚द‚ण्डिनोरिव । य‚दि च विशेष‚ण‚विशेष्य‚भाव‚स‚म्ब‚न्ध‚ब‚ले‚{\tiny $_{५}$}‚न भाव‚स्य‚{\tiny $_{lb}$}‚ प्राग‚भाव इति प्र‚तीतिस्त‚था प्राग‚भावादेर्भाव इत्य‚पि प्र‚तीतिः स्यात् स‚म्ब‚न्ध‚स्या‚{\tiny $_{lb}$}‚विशेषात् । त‚स्मात् प्राग‚भावादेर‚स‚म्ब‚न्धिनो भाव‚स‚म्ब‚न्धित्वेन प्र‚तीतिर्भ्रान्तिरेव ।
	{\color{gray}{\rmlatinfont\textsuperscript{§~\theparCount}}}
	\pend% ending standard par
      ‚{\tiny $_{lb}$}‚

	  
	  \pstart \leavevmode% starting standard par
	न \textbf{चा न्यो न्या भा वो} भावानाम‚स्ति । न हि घ‚ट‚स्य निवृत्तिः प‚ट‚स्य निवृत्ति‚{\tiny $_{lb}$}‚र्भ‚व\add{त्य‚प्र‚तीतेः न च प‚टेऽव‚स्थानात्सा}\edtext{}{\edlabel{pvsvt_32-1b}\label{pvsvt_32-1b}\lemma{व}\Bfootnote{In the margin. }} त‚त्स‚म्ब‚न्धिनी युक्ता । एवं हि प्राग‚भावा‚{\tiny $_{lb}$}‚द्य‚प्य‚न्योन्याभावः स्यात् कार‚णादाव‚{\tiny $_{६}$}‚व‚स्थानात् । त‚स्माद‚न्याभाव एवास्ति नान्यो‚{\tiny $_{lb}$}‚न्याभाव‚स्तेनान्याभावात् प्राग‚भावादीनां न भेद इति क‚थं च‚तुर्विधोऽभाव उच्य‚ते ।
	{\color{gray}{\rmlatinfont\textsuperscript{§~\theparCount}}}
	\pend% ending standard par
      ‚{\tiny $_{lb}$}‚

	  
	  \pstart \leavevmode% starting standard par
	प्र‚त्य‚क्षाभाव‚निराश\edtext{}{\lemma{निराश}\Bfootnote{?}}स‚श्च \textbf{नै रा त्म्य सि द्धा} व‚भिहित इति नेहोच्य‚ते ।
	{\color{gray}{\rmlatinfont\textsuperscript{§~\theparCount}}}
	\pend% ending standard par
      ‚{\tiny $_{lb}$}‚

	  
	  \pstart \leavevmode% starting standard par
	न त्व‚भाव‚स्यास‚त्त्वेनानुभूत‚त्वात् \add{क‚थं प्र‚त्य‚क्षेण निश्च‚यः । नैष दोषो य}\edtext{}{\edlabel{pvsvt_32-1c}\label{pvsvt_32-1c}\lemma{त्वात्}\Bfootnote{In the margin. }}‚{\tiny $_{lb}$}‚\leavevmode\ledsidenote{\textenglish{14a/PSVTa}} स्मादेक‚ज्ञान‚संस‚र्गिणोः प्र‚त्य‚क्षेणैक‚स्य ग्र‚ह‚ण‚मेवान्य‚स्याग्र‚ह‚ण‚{\tiny $_{७}$}‚न्त‚द‚ग्र‚ह‚ण‚मेव च‚{\tiny $_{lb}$}‚ त‚स्याभाव‚ग्र‚ह‚ण‚म्भावे हि त‚स्याग्र‚ह‚णायोगाद् [।] य‚दाहान्य‚हेतुसाक‚ल्ये त‚द‚व्य‚भि‚{\tiny $_{lb}$}‚चाराच्चोप‚ल‚म्भः स‚त्ता । त‚द‚भावोनुप‚ल‚ब्धिर‚स‚त्तान्योप‚ल‚ब्धिश्चानुप‚ल‚ब्धिरिति ।
	{\color{gray}{\rmlatinfont\textsuperscript{§~\theparCount}}}
	\pend% ending standard par
      ‚{\tiny $_{lb}$}‚

	  
	  \pstart \leavevmode% starting standard par
	तेनाय‚म‚र्थः [।] प्र‚त्य‚क्ष‚म‚भाव‚न्निश्चाय‚य‚तीति ताव‚न्न निश्चाय \add{य‚तीत्य‚र्थः । स‚{\tiny $_{lb}$}‚ च दृश्य‚स्य भावानिश्च}\edtext{}{\edlabel{pvsvt_32-1d}\label{pvsvt_32-1d}\lemma{निश्चाय}\Bfootnote{In the margin. }}योऽभाव‚निश्च‚य एव । एवं प्र‚त्य‚क्ष‚पृष्ठ‚भाविनो विक‚ल्प‚स्य‚{\tiny $_{lb}$}‚ प्र‚त्य‚क्ष‚वि‚{\tiny $_{१}$}‚ष‚यानुसारित्वं स‚म‚र्थित‚म्भ‚व‚ति । त‚देव‚मुप‚ल‚ब्ध्य‚भाव‚व्य‚व‚हार‚व‚द्‚{\tiny $_{lb}$}‚ अर्थाभाव‚व्य‚व‚हार‚स्यापि प्र‚त्य‚क्ष‚सिद्ध‚त्वान्न लिङ्गेनासौ साध्य‚ते । एव‚न्ताव‚द‚मूढं‚{\tiny $_{lb}$}‚ ‚{\tiny $_{lb}$}‚ ‚{\tiny $_{lb}$}‚ \leavevmode\ledsidenote{\textenglish{33/s}}प्र‚ति दृश्यानुप‚ल‚म्भो नाभावं व्य‚भिच‚र‚तीत्य‚भाव‚व्य‚व‚हारः प्र‚त्य‚क्ष‚सिद्धः ।
	{\color{gray}{\rmlatinfont\textsuperscript{§~\theparCount}}}
	\pend% ending standard par
      ‚{\tiny $_{lb}$}‚

	  
	  \pstart \leavevmode% starting standard par
	न चाप्य‚भावोनुप‚ल‚ब्धानाम‚पि \add{स‚त्त्वान्नित्यं श‚क्य‚मानानुप}\edtext{}{\edlabel{pvsvt_33-1}\label{pvsvt_33-1}\lemma{पि}\Bfootnote{Roṅ-du-med-pa. }}लंभ‚व्य‚भिचार‚{\tiny $_{lb}$}‚ इति कृत्वा व्य‚व‚ह‚र्त्तुम‚श‚क्य इति व‚क्तुं युक्तं । एवं ह्य‚भाव‚{\tiny $_{२}$}‚स्य निश्चाय‚क‚म‚पि‚{\tiny $_{lb}$}‚ प्र‚त्य‚क्ष‚न्न स्यात् । स‚न्देहान्न चान्य‚न्निश्चाय‚क‚म‚न्य‚द् व्य‚व‚हाराङ्गं युक्तं । त‚स्मात्‚{\tiny $_{lb}$}‚ प्र‚त्य‚क्ष‚निश्चाय‚क‚त्वाद् भाव‚व‚द् दृश्य‚स्याभाव‚म‚पि व्य‚व‚हार‚य‚ति ।
	{\color{gray}{\rmlatinfont\textsuperscript{§~\theparCount}}}
	\pend% ending standard par
      ‚{\tiny $_{lb}$}‚

	  
	  \pstart \leavevmode% starting standard par
	क‚थ‚न्त‚र्ह्य‚स‚द्व्य‚व‚हार‚स्य साध्य‚त्व‚मित्याह । य‚दा पुन‚रित्यादि । एवं-विधे‚{\tiny $_{lb}$}‚ ह्युप‚ल‚म्भ‚योग्यानु\add{प‚ल‚ब्धिरेवास‚तां प‚दा}\edtext{}{\edlabel{pvsvt_33-2}\label{pvsvt_33-2}\lemma{योग्यानु}\Bfootnote{In the margin. }}र्थानाम‚स‚त्ता नान्या । त‚वा सिद्धेपि‚{\tiny $_{lb}$}‚ प्र‚त्य‚क्षेणाभाव‚व्य‚व‚हार‚स्य \textbf{विष‚ये‚{\tiny $_{३}$}‚ । मोहाद् विष‚यिणो} \add{ऽस‚तोऽविद्य‚मान‚स्य}\edtext{\textsuperscript{*}}{\edlabel{pvsvt_33-2b}\label{pvsvt_33-2b}\lemma{*}\Bfootnote{In the margin. }}‚{\tiny $_{lb}$}‚ ज्ञानं । नास्तीत्येव‚माकारं । नास्तीत्येव‚म्भूत‚श्च श‚ब्दः निःश‚ङ्काव‚ग‚म‚नाग‚म‚न‚ल‚क्ष‚णा‚{\tiny $_{lb}$}‚ पुरुष‚स्य प्र‚वृत्तिर्व्य‚व‚हारः । तान‚प्र‚तिप‚द्य‚मानः पुमान् । विष‚य‚प्र‚द‚र्श‚नेनास‚द्व्य‚{\tiny $_{lb}$}‚व‚हार‚विष‚य‚स्य घ‚ट‚विविक्त‚प्र‚देश‚स्योप‚ल‚म्भ‚मान‚स्य प्र‚द‚र्श‚नेन । स‚म‚येऽभाव‚व्य‚व‚हारे‚{\tiny $_{lb}$}‚ प्र‚व‚र्त्त्य‚ते । दृष्टान्त‚माह । य‚थेत्यादि । सास्नादिस‚मुदायात्म‚{\tiny $_{४}$}‚क एव गौः । त‚तो न‚{\tiny $_{lb}$}‚ त‚त्र गोत्वं साध्य‚ते किन्तु गोव्य‚व‚हारः । य‚दायं मूढ‚म‚तिः शाव‚लेये प्र‚व‚र्त्तित‚गोव्य‚{\tiny $_{lb}$}‚व‚हारो बाहुलेये शाव‚लेय‚रूप‚शून्य‚त्वाद् गोव्य‚व‚हारं न प्र‚व‚र्त्त‚य‚ति स निमित्त‚प्र‚द‚र्श‚{\tiny $_{lb}$}‚नेन गोव्य‚व‚हारे प्र‚व‚र्त्त्य‚ते । सास्नादिस‚मुदाय‚निमित्त‚को हि गोव्य‚व‚हारो न‚{\tiny $_{lb}$}‚ शाव‚लेय‚रूप‚निमित्तिकः । बाहुलेयेऽपि त‚न्निमित्त‚म‚स्तीति क‚थ‚म‚सौ न‚{\tiny $_{५}$}‚ प्र‚व‚र्त्त्य‚ते ।‚{\tiny $_{lb}$}‚ त‚द्व‚द् घ‚ट‚विविक्तेपि प्र‚देशेनुप‚ल‚म्भ‚निमित्त‚प्र‚द‚र्श‚नेनास‚द्व्य‚व‚हारे प्र‚व‚र्त्त्य‚ते ।
	{\color{gray}{\rmlatinfont\textsuperscript{§~\theparCount}}}
	\pend% ending standard par
      ‚{\tiny $_{lb}$}‚

	  
	  \pstart \leavevmode% starting standard par
	\textbf{त‚था चे}ति येनैवं व्य‚व‚हारः साध्य‚ते तेन \textbf{दृष्टान्तासिद्धिचोद‚नापि प्र‚तिव्यूढा}‚{\tiny $_{lb}$}‚ प्र‚तिक्षिप्ता । अनुप‚ल‚ब्धेर्लिङ्गाद‚भावे साध्ये येनैव लिङ्गो न साध्य‚ध‚र्मिण्य‚भावः‚{\tiny $_{lb}$}‚साध्य‚स्तेनैव दृष्टान्त‚ध‚र्मिण्य‚पि त‚त्राप्य‚प‚रो दृष्टान्त इत्य‚न‚व‚स्था स्यात् [।] व्य‚व‚{\tiny $_{lb}$}‚ हारे तु साध्ये ना‚{\tiny $_{६}$}‚न‚व‚स्था । प्र‚व‚र्त्तित‚व्य‚व‚हार‚स्यैव पुनः स‚म‚ये प्र‚व‚र्त्त‚नात् ।
	{\color{gray}{\rmlatinfont\textsuperscript{§~\theparCount}}}
	\pend% ending standard par
      ‚{\tiny $_{lb}$}‚

	  
	  \pstart \leavevmode% starting standard par
	न‚नु यो हि विष‚यं प्र‚तिप‚द्य‚ते स विष‚यिण‚म‚पि प्र‚तिप‚द्य‚त इति क‚थं व्य‚व‚हार‚{\tiny $_{lb}$}‚स्यापि साध्य‚त्व‚मित्याह । \textbf{विष‚ये}त्यादि । दृश्य‚न्ते हि लोके त‚थाविधा ये \textbf{विष‚य‚{\tiny $_{lb}$}‚प्र‚तिप‚त्ताव‚प्य‚प्र‚तिप‚न्न‚विष‚यिणः} । य‚था \textbf{सांख्यः स‚त्त्वे} र‚जो नास्तीति प्र‚व‚र्त्तितास‚{\tiny $_{lb}$}‚ \leavevmode\ledsidenote{\textenglish{34/s}}\leavevmode\ledsidenote{\textenglish{14b/PSVTa}} द्व्य‚व‚हारोपि निमित्त‚निश्च‚या‚{\tiny $_{७}$}‚भावान्मृत्पिण्डे न प्र‚व‚र्त्त‚य‚त्य‚नुप‚ल‚म्भ‚निमित्त‚प्र‚द‚र्श‚ने‚{\tiny $_{lb}$}‚ घ‚टाभाव‚व्य‚व‚हारे प्र‚व‚र्त्त्य‚ते ।
	{\color{gray}{\rmlatinfont\textsuperscript{§~\theparCount}}}
	\pend% ending standard par
      ‚{\tiny $_{lb}$}‚

	  
	  \pstart \leavevmode% starting standard par
	एव‚मित्यादिनोप‚संहारः । एव‚मुक्तेन प्र‚कारेणान‚योर‚नुप‚ल‚ब्ध्योर्दृश्यादृश्य‚योः‚{\tiny $_{lb}$}‚ स‚द्व्य‚व‚हार‚प्र‚तिषेध‚फ‚ल‚त्व‚न्तुल्यं ।
	{\color{gray}{\rmlatinfont\textsuperscript{§~\theparCount}}}
	\pend% ending standard par
      ‚{\tiny $_{lb}$}‚

	  
	  \pstart \leavevmode% starting standard par
	क‚थं स्व‚विप‚र्य‚य‚हेत्व‚भाव‚भावाभ्यां । स्व‚श‚ब्देन स‚द्व्य‚व‚हार‚स्य स्व‚रूपं गृह्य‚ते‚{\tiny $_{lb}$}‚ विप‚र्य‚य‚श‚ब्देन स‚द्व्य‚व‚हार‚विरुद्धोस‚द्व्य‚व‚हारो गृह्य‚ते । त‚योर्हे‚{\tiny $_{१}$}‚तू । स्व‚विप‚{\tiny $_{lb}$}‚र्य‚य‚हेतू । त‚त्र स्व‚हेतुरुप‚ल‚ब्धिर्विप‚र्य‚य‚हेतुर्दृश्यानुप‚ल‚ब्धिः । त‚योर‚भाव‚भावौ ।‚{\tiny $_{lb}$}‚ स्व‚विप‚र्य‚य‚हेत्व‚भाव‚भावौ । ताभ्यां ।
	{\color{gray}{\rmlatinfont\textsuperscript{§~\theparCount}}}
	\pend% ending standard par
      ‚{\tiny $_{lb}$}‚

	  
	  \pstart \leavevmode% starting standard par
	एत‚दुक्त‚म्भ‚व‚ति [।] अदृश्यानुप‚ल‚ब्धौ स‚द्व्य‚व‚हार‚निमित्ताया उप‚ल‚ब्धेः‚{\tiny $_{lb}$}‚ प्र‚त्य‚क्षानुमान‚निवृत्ताव‚भावात् स‚द्व्य‚व‚हार‚निवृत्तिः । दृश्यानुप‚ल‚म्भे तु स‚द्व्य‚व‚हार‚{\tiny $_{lb}$}‚विरुद्ध‚स्यास‚द्व्य‚व‚हार‚स्य निमित्त‚स‚द्भावात् प्र‚वृत्तिस्तेन स‚द्व्य‚व‚हार‚स्य निवृत्ति-‚{\tiny $_{२}$}‚‚{\tiny $_{lb}$}‚ रिति स‚द्व्य‚व‚हार‚प्र‚तिषेध‚फ‚ल‚त्व‚न्तुल्यं ।
	{\color{gray}{\rmlatinfont\textsuperscript{§~\theparCount}}}
	\pend% ending standard par
      ‚{\tiny $_{lb}$}‚

	  
	  \pstart \leavevmode% starting standard par
	न‚नूप‚ल‚म्भ‚निवृत्ताव‚प्य‚र्थ‚स्य स‚न्देहात् क‚थं स‚द्व्य‚व‚हारो निव‚र्त्त‚त इत्याह ।‚{\tiny $_{lb}$}‚ एक‚त्रेत्य‚दृश्य‚विष‚यायाम‚नुप‚ल‚ब्धौ स‚त्त्व‚स्य संश‚यात् त‚तो निश्च‚यात्म‚कः स‚त्त्व‚व्य‚व‚{\tiny $_{lb}$}‚हारो निव‚र्त्त‚त एव । स‚न्दिग्ध‚स्तु स‚त्त्व‚व्य‚व‚हारो न निव‚र्त्त‚ते । अन्य‚त्र तु दृश्यानृप‚ल‚ब्धौ‚{\tiny $_{lb}$}‚ विप‚र्य‚यादिति संश‚य‚विप‚र्य‚यो निश्च‚य‚स्त‚स्मात् । अस‚त्त्व‚स्य निश्च‚या‚{\tiny $_{३}$}‚दित्य‚र्थः ।
	{\color{gray}{\rmlatinfont\textsuperscript{§~\theparCount}}}
	\pend% ending standard par
      ‚{\tiny $_{lb}$}‚

	  
	  \pstart \leavevmode% starting standard par
	य‚द्य‚दृश्यानुप‚ल‚ब्धौ संश‚यः क‚थं सा प्र‚माण‚मित्याह । त‚त्राद्येत्यादि । त‚त्र‚{\tiny $_{lb}$}‚ द्व‚योर‚नुप‚ल‚ब्ध्योर्म‚ध्ये आद्या दृश्यानुप‚ल‚ब्धिः प्र‚माण‚मुक्ता स‚द्व्य‚व‚हार‚निषेधे उप‚यो‚{\tiny $_{lb}$}‚गाद् व्यापारात् ।
	{\color{gray}{\rmlatinfont\textsuperscript{§~\theparCount}}}
	\pend% ending standard par
      ‚{\tiny $_{lb}$}‚

	  
	  \pstart \leavevmode% starting standard par
	क्व त‚र्हि त‚स्या अप्रामाण्य‚मित्याह । न त्वित्यादि । व्य‚तिरेक‚स्याभाव‚स्य द‚र्श‚{\tiny $_{lb}$}‚न‚न्निश्च‚यः । आदिग्र‚ह‚णाच्छ‚ब्दो व्य‚व‚हार‚श्च गृह्य‚ते । संश‚याद् य‚तो नाभाव‚{\tiny $_{lb}$}‚निश्च‚य उत्प‚द्य‚ते ।‚{\tiny $_{४}$}‚ त‚स्मान्न प्र‚माणं । द्वितीया त्विति । दृश्य‚विष‚यानुप‚ल‚ब्धिः ।‚{\tiny $_{lb}$}‚ अत्रेति व्य‚तिरेक‚द‚र्श‚नादौ निश्च‚य‚फ‚ल‚त्वान्निश्य‚च एव फ‚ल‚म‚स्या इति कृत्वा । सा‚{\tiny $_{lb}$}‚ च दृश्य‚विष‚यानुप‚ल‚ब्धिः प्र‚योग‚भेदाच्च‚तुर्विधेति स‚म्ब‚न्धः । विरुद्ध‚श्च विरुद्ध‚कायं‚{\tiny $_{lb}$}‚ चेति विरूपैक‚शेषः । सिद्धिरु\add{प‚ल‚ब्धिर्दृश्यात्म‚नो}\edtext{}{\edlabel{pvsvt_34-1}\label{pvsvt_34-1}\lemma{सिद्धिरु}\Bfootnote{In the margin. }}रित्य‚त्रापि स‚म्ब‚न्ध‚नीयं ।
	{\color{gray}{\rmlatinfont\textsuperscript{§~\theparCount}}}
	\pend% ending standard par
      ‚{\tiny $_{lb}$}‚‚{\tiny $_{lb}$}‚‚{\tiny $_{lb}$}‚\textsuperscript{\textenglish{35/s}}
	    
	    \stanza[\smallbreak]
	  एतेन स्व‚भाव‚विरुद्धोप‚ल‚ब्धिर्विरुद्ध‚कार्योप‚ल‚ब्धि‚{\tiny $_{५}$}‚श्च द्वे निर्दिष्टे ।&‚{\tiny $_{lb}$}‚
	  \bigskip
	  \begingroup
	असिद्धिर्हेतुभाव‚योः दृश्यात्म‚नोः
	  \endgroup
	&‚{\tiny $_{lb}$}‚इत्युप‚ल‚ब्धिल‚क्ष‚ण‚प्राप्त‚योः कार‚ण‚स्व‚भाव‚योर‚नुप‚ल‚ब्धिरित्य‚र्थः ।\&[\smallbreak]
	  
	  
	  ‚{\tiny $_{lb}$}‚

	  
	  \pstart \leavevmode% starting standard par
	एतेनापि कार‚णानुप‚ल‚ब्धिः स्व‚भावानुप‚ल‚ब्धिश्च द्वे निर्द्दिष्टे इति‚{\tiny $_{lb}$}‚ च‚तुर्धा भ‚व‚ति । अभावार्थेत्य‚भावोऽभाव‚व्य‚व‚हार‚श्चार्थः प्र‚योज‚नं \add{य‚स्याः सा‚{\tiny $_{lb}$}‚ त‚था ।}\edtext{}{\edlabel{pvsvt_35-2}\label{pvsvt_35-2}\lemma{नं}\Bfootnote{In the margin. }}
	{\color{gray}{\rmlatinfont\textsuperscript{§~\theparCount}}}
	\pend% ending standard par
      ‚{\tiny $_{lb}$}‚

	  
	  \pstart \leavevmode% starting standard par
	न‚नु विरुद्ध‚कार्य‚योः सिद्धिरित्य‚त्रानुप‚ल‚ब्धिरिति न श्रूय‚ते [।] त‚त्क‚थ‚म‚न‚योर‚{\tiny $_{६}$}‚‚{\tiny $_{lb}$}‚नुप‚ल‚ब्धित्व‚मित्य‚त आह । यावान् क‚श्चिदित्यादि । यावान् क‚श्चिदिति व्याप्ता\edtext{}{\lemma{व्याप्ता}\Bfootnote{?}}‚{\tiny $_{lb}$}‚ चैत‚त् क‚थ्येत । न क‚श्चित् प्र‚तिषेध‚व्य‚व‚हारो लिङ्ग‚जोस्ति योनुप‚ल‚ब्धिम‚न्त‚रेण‚{\tiny $_{lb}$}‚ श‚क्य‚ते क‚र्त्तु । न तु प्र‚त्य‚क्ष‚साध्य‚त्व‚म‚भाव‚व्य‚व‚हार‚स्य । निराकृत‚मेत‚च्च‚{\tiny $_{lb}$}‚प्रागेवोक्त‚न्तामेव \add{व्याप्तिं द‚र्श‚यितुमाह । त‚था हीति}\edtext{}{\edlabel{pvsvt_35-2b}\label{pvsvt_35-2b}\lemma{न्तामेव}\Bfootnote{In the margin. }} । स इति प्र‚तिषेधः ।‚{\tiny $_{lb}$}‚ द्विधा क्रियेत व्य‚व‚ह्रियेत क‚स्य‚चिद‚र्थ‚स्य विधिना‚{\tiny $_{७}$}‚ निषेधेन वा [।] क‚स्य‚चिद्विधा- \leavevmode\ledsidenote{\textenglish{15a/PSVTa}}‚{\tiny $_{lb}$}‚ व‚पि क्रिय‚माणे । विरुद्धो वा विधीयेताविरुद्धो वाऽविरुद्ध‚स्य विधौ निषिध्य‚मा‚{\tiny $_{lb}$}‚न‚विधीय‚मान‚योः स‚ह‚भाव‚विरोधाभावाद‚प्र‚तिषेधो निषेध्याभिम‚त‚स्य । विरुद्ध‚स्यापी‚{\tiny $_{lb}$}‚त्यादि । एवं ह्य‚सौ विरुद्धः स्याद् य‚दि त‚त्र स्व‚विरुद्ध‚स्यानुप‚ल\add{धिः । त‚था हीत्या‚{\tiny $_{lb}$}‚दिनैत‚देवा}\edtext{}{\edlabel{pvsvt_35-2C}\label{pvsvt_35-2C}\lemma{ल}\Bfootnote{In the margin. }}ह । अप‚र्य‚न्त‚कार‚ण‚स्येत्य‚क्षीण‚कार‚ण‚स्य‚{\tiny $_{१}$}‚ भ‚व‚तः स‚न्तानेनोत्प‚द्य‚मान‚स्य‚{\tiny $_{lb}$}‚ शीत‚स्प‚र्शादेर‚ग्न्यादिस‚न्निधानात् पूर्व‚मिति द्र‚ष्ट‚व्यं । अन्य‚भावेऽग्न्यादिभावेऽभा‚{\tiny $_{lb}$}‚वाद‚नुत्पादाद् विरोध‚ग‚तिः । न त्व‚भावाद‚हेतुक‚त्वाद् विनाश‚स्य ।
	{\color{gray}{\rmlatinfont\textsuperscript{§~\theparCount}}}
	\pend% ending standard par
      ‚{\tiny $_{lb}$}‚

	  
	  \pstart \leavevmode% starting standard par
	एत‚दुक्त‚म्भ‚व‚ति । पूर्व‚पूर्व‚स्य शीत‚स्प‚र्श‚स्य स्व‚र‚स‚निरोधे स‚त्युत्त‚रोत्त‚र‚स्य‚{\tiny $_{lb}$}‚  \leavevmode\ledsidenote{\textenglish{36/s}}चा\add{ग्नितार‚त‚म्येन शीत‚स्प‚र्श}\edtext{}{\edlabel{pvsvt_36-1}\label{pvsvt_36-1}\lemma{चा}\Bfootnote{In the margin. }}स्याप‚च‚य‚तार‚त‚म्य‚योगिनः क्र‚मेणोत्प‚द्य‚मान‚स्य‚{\tiny $_{lb}$}‚ याव‚त्स‚र्व‚स‚र्वेणानु‚{\tiny $_{२}$}‚त्प‚त्तिरुष्ण‚स्प‚र्श‚ल‚क्ष‚णो भ‚व‚ति । तेन निर्हेतुकेपि विनाशेऽ‚{\tiny $_{lb}$}‚ग्निस‚न्निधानात् पूर्वं प्र‚ब‚न्ध‚प्र‚वृत्त‚स्य शीत‚स्प‚र्श‚स्य स्व‚र‚स‚निरोधेऽन्य‚स्य च प्र‚ब‚न्धे‚{\tiny $_{lb}$}‚नोत्पित्सोर‚ग्निस‚न्निधाने स‚त्य‚नुत्प‚त्तेर‚ग्निशीत‚योर्विरोधाव‚ग‚तिर्लोके न तु प‚र‚मा‚{\tiny $_{lb}$}‚र्थ‚तो विरोधः । अत एव विरोध‚ग‚तिरि\add{त्याह । ‚{\tiny $_{lb}$}‚य‚त्पुन‚रुच्य}\edtext{}{\edlabel{pvsvt_36-1b}\label{pvsvt_36-1b}\lemma{तिरि}\Bfootnote{१ In the margin.}}ते[।]न कार‚ण‚निव‚र्त्त‚न‚म‚न्त‚रेण क‚स्य‚चिद‚ग्न्यादिर्निव‚र्त्त‚को‚{\tiny $_{lb}$}‚ नामेति ।
	{\color{gray}{\rmlatinfont\textsuperscript{§~\theparCount}}}
	\pend% ending standard par
      ‚{\tiny $_{lb}$}‚

	  
	  \pstart \leavevmode% starting standard par
	त‚द‚युक्तं ।‚{\tiny $_{३}$}‚ निर्हेतुक‚त्वाद् विनाश‚स्य क‚थं कार‚ण‚स्य निव‚र्त्त‚कः । अथ स‚हेतु‚{\tiny $_{lb}$}‚क‚विनाश‚म‚भ्युप‚ग‚म्यैव‚मुच्य‚ते त‚दा य‚थासौ कार‚णं निव‚र्त्त‚य‚ति कार्यं किन्न निव‚र्त्त‚{\tiny $_{lb}$}‚य‚ति । य‚दि च कार‚ण‚निव‚र्त्त‚न‚म‚न्त‚रेण न कार्यं निव‚र्त्त‚यितुं श‚क्य‚ते त‚दा त‚त्कार‚ण‚{\tiny $_{lb}$}‚स्यापि क‚थ‚न्निव‚र्त्त‚कं याव‚त्त‚त्कार‚णं न निव‚र्त्त‚य‚ति त‚त्कार‚ण‚स्याप्येव‚मित्य‚न‚व‚स्थ‚या‚{\tiny $_{lb}$}‚ न क‚श्चित् क‚स्य‚चिन्निव‚र्त्त‚कः स्यात् [।] न च स‚न्ताना‚{\tiny $_{४}$}‚पेक्ष‚यैत‚द्व‚क्तुं युज्य‚ते स‚हे‚{\tiny $_{lb}$}‚तुके विनाशे स‚न्तान‚स्यैवाभावादिति य‚त्किञ्चिदेत‚त् । स चेत्य‚न्य‚भावे स‚त्य‚भावो‚{\tiny $_{lb}$}‚नुत्प‚त्तिल‚क्ष‚णो विरोध‚हेतुर‚नुप‚ल‚ब्धेः स‚काशाद् व्य‚व‚ह्रिय‚ते ।
	{\color{gray}{\rmlatinfont\textsuperscript{§~\theparCount}}}
	\pend% ending standard par
      ‚{\tiny $_{lb}$}‚

	  
	  \pstart \leavevmode% starting standard par
	अनेन स‚हान‚व‚स्थाल‚क्ष‚णो विरोधो व्याख्यातः ॥
	{\color{gray}{\rmlatinfont\textsuperscript{§~\theparCount}}}
	\pend% ending standard par
      ‚{\tiny $_{lb}$}‚

	  
	  \pstart \leavevmode% starting standard par
	द्वितीयं विरोध‚न्द‚र्श‚य‚न्नाह । \textbf{अन्योन्योप‚ल‚ब्धी}त्यादि । \textbf{अन्योन्योप‚ल‚ब्धिः}‚{\tiny $_{lb}$}‚ प‚र‚स्प‚र‚प्र‚तिप‚त्तिस्त‚स्याः \textbf{प‚रिहा}रो विवेक‚स्तेन‚{\tiny $_{५}$}‚ \textbf{स्थितं ल‚क्ष‚णं} स्व‚रूपं य‚योस्तौ‚{\tiny $_{lb}$}‚ त‚थोक्तौ । त‚योर्भावोन्योन्योप‚ल‚ब्धिप‚रिहार\textbf{स्थित‚ल‚क्ष‚ण‚ता} । सा वा विरोधः [।]‚{\tiny $_{lb}$}‚ वा श‚ब्दः स‚मुच्च‚ये । \textbf{नित्यानित्य‚त्व‚व‚दि}ति दृष्टान्तः ।
	{\color{gray}{\rmlatinfont\textsuperscript{§~\theparCount}}}
	\pend% ending standard par
      ‚{\tiny $_{lb}$}‚

	  
	  \pstart \leavevmode% starting standard par
	न‚नु प्र‚थ‚म‚विरोधेप्य‚स्त्येव प‚र‚स्प‚र‚प‚रिहारः । द्वितीयेपि स‚हान‚व‚स्थानं ।‚{\tiny $_{lb}$}‚ त‚था हि य‚योरेव ध‚र्म‚योरेक‚त्रान‚व‚स्थान‚न्त‚योरेव द्वितीयो विरोधः । त‚था हि रूप‚{\tiny $_{lb}$}‚र‚स‚योर‚यं नेष्य‚ते ।‚{\tiny $_{६}$}‚ त‚त्क‚स्माद् विरोध‚द्व‚य‚मुक्त‚मिति चेत् [।]
	{\color{gray}{\rmlatinfont\textsuperscript{§~\theparCount}}}
	\pend% ending standard par
      ‚{\tiny $_{lb}$}‚

	  
	  \pstart \leavevmode% starting standard par
	स‚त्यं विष‚य‚विभागार्थ‚न्तूक्तं । पूर्वो विरोधो दृश्य‚व‚स्तुविष‚य एव । द्वितीय‚{\tiny $_{lb}$}‚स्त्व‚व‚स्तुविष‚योप्य‚दृश्य‚विष‚य‚श्चेत्येके । य‚द्वा पूर्व‚कोन्य‚तोन्य‚स्याभाव‚प्र‚तिप‚त्त्यान्येन‚{\tiny $_{lb}$}‚ स‚ह विरोधः । द्वितीय‚स्त्व‚न्येन स‚हैक‚त्वाभावेन स्व‚रूप‚विष‚य इत्य‚न‚योर्विरोध‚{\tiny $_{lb}$}‚\leavevmode\ledsidenote{\textenglish{15b/PSVTa}} योर्म‚हान् भेदः । त‚था हि नित्य‚त्व‚निवृत्तिरूप‚म‚नित्य‚त्व‚म‚{\tiny $_{७}$}‚नित्य‚त्व‚निवृत्तिरूप‚{\tiny $_{lb}$}‚ञ्चानित्य‚त्व‚मित्येव‚म‚न्योन्य‚प‚रिहार एव विरोधः स्व‚रूप‚निष्ठः । न च नील‚{\tiny $_{lb}$}‚‚{\tiny $_{lb}$}‚ ‚{\tiny $_{lb}$}‚ \leavevmode\ledsidenote{\textenglish{37/s}}निवृत्तिरूप‚म्पीत‚म्पीत‚निवृत्तिरूपं च नील‚म‚न‚योर्भाव‚रूप‚त्वान्नीलाभावे पीत‚स्य‚{\tiny $_{lb}$}‚ भाव‚प्र‚संगाच्च[।]त‚स्मान्नान‚योर‚यं विरोधः । नाप्य‚न‚योर‚न्योन्याभावाव्य‚भिचारेणायं‚{\tiny $_{lb}$}‚ विरोधोऽप्र‚तीतेः । अत एवाविरुद्ध‚स्य विधान‚मुच्य‚ते । नील‚स्यापि नील‚निवृत्ति‚{\tiny $_{lb}$}‚रूपेणानीलेना‚{\tiny $_{१}$}‚यं विरोधो न नीलाभाव‚निय‚तेनानीलेन त‚थाभूत‚स्यानील‚व‚स्तुनः‚{\tiny $_{lb}$}‚ पीतादिव्य‚तिरिक्त‚स्याभावात् । क‚थ‚न्त‚र्हि नीलादौ दृश्य‚माने पीतादेस्तादात्म्य‚{\tiny $_{lb}$}‚निषेधः ।
	{\color{gray}{\rmlatinfont\textsuperscript{§~\theparCount}}}
	\pend% ending standard par
      ‚{\tiny $_{lb}$}‚

	  
	  \pstart \leavevmode% starting standard par
	नैष दोषो य‚स्मात् । नील‚स्यैक‚स्योप‚ल‚म्भेन्य‚स्यादृश्य‚स्याप्युप‚ल‚म्भ‚मान‚{\tiny $_{lb}$}‚स्व‚भाव‚त्वे स‚ति त‚थैवोप‚ल‚म्भः स्यादित्येव‚मुप‚ल‚ब्धिल‚क्ष‚ण‚प्राप्त‚त्व‚म्प‚रामृश्य‚{\tiny $_{lb}$}‚ तादात्म्यं स्व‚भावानुप‚ल‚म्भान्निषिध्य‚ते । अ‚{\tiny $_{२}$}‚त्रापि च विरोधे स चानुप‚ल‚ब्धे‚{\tiny $_{lb}$}‚रित्य‚पेक्ष‚णीय‚न्तेनाय‚म‚र्थः[।]स चान्योन्य‚प‚रिहारो विरोधोनुप‚ल‚ब्धेरेव निश्चेत‚{\tiny $_{lb}$}‚व्यः । त‚था हि [।] विरोधिनाम्विरोध एक‚प्र‚तिभासे स‚त्य‚न्याप्र‚तिभास‚न‚मेवोच्य‚ते‚{\tiny $_{lb}$}‚ [।] भाव‚स्य च रूपे प्र‚तिभास‚माने त‚द‚भावो न प्र‚तिभास‚ते [।] त‚स्माद् भावाभा‚{\tiny $_{lb}$}‚व‚योस्तादात्म्येनाप्र‚तिभास‚नाद् विरोधो निश्चीय‚ते । एवं नित्यानित्यादाव‚य‚{\tiny $_{lb}$}‚न्निश्चेत‚व्यः ।
	{\color{gray}{\rmlatinfont\textsuperscript{§~\theparCount}}}
	\pend% ending standard par
      ‚{\tiny $_{lb}$}‚

	  
	  \pstart \leavevmode% starting standard par
	अप्र‚ति‚{\tiny $_{३}$}‚भास‚नं चैक‚प्र‚तिभास‚न‚मेवोच्य‚ते । त‚दाह । \textbf{त‚त्रापी}त्यादि । त‚त्राप्य‚{\tiny $_{lb}$}‚न‚न्त‚रोक्ते विरोधे । \textbf{एकोप‚ल‚ब्ध्या} । एक‚स्य विरोधिन उप‚ल‚ब्ध्या\textbf{न्यानुप‚ल‚ब्धिरेव}‚{\tiny $_{lb}$}‚ निषेध्यानुप‚ल‚ब्धिरे\textbf{वोच्य‚ते । अन्य‚थे}ति य‚द्येकोप‚ल‚ब्ध्यान्यानुप‚ल‚ब्धिर्नोच्य‚ते । त‚दा‚{\tiny $_{lb}$}‚\textbf{ऽनिषिद्धा उप‚ल‚ब्धि}र्य‚स्य निषेध्य‚स्य त‚स्यैकोप‚ल‚ब्धाव‚प्य\textbf{भावासिद्धेः} । त‚त‚श्चो‚{\tiny $_{lb}$}‚प‚ल‚भ्य‚मान‚स्य विरोधित्व‚मेव न स्यात् ।‚{\tiny $_{४}$}‚ त‚स्माद् विरोध‚द्व‚य‚स्याप्य‚नुप‚ल‚ब्धि‚{\tiny $_{lb}$}‚कृत‚त्वाद् विरुद्धोप‚ल‚ब्ध्याद‚यो विधिमुखेन प्र‚युक्ता अप्य‚नुप‚ल‚ब्धिस्व‚भावा‚{\tiny $_{lb}$}‚ भ‚व‚न्ति ॥
	{\color{gray}{\rmlatinfont\textsuperscript{§~\theparCount}}}
	\pend% ending standard par
      ‚{\tiny $_{lb}$}‚

	  
	  \pstart \leavevmode% starting standard par
	न‚नु विरोधिनोर्विरोध‚ल‚क्ष‚ण‚स‚म्ब‚न्ध‚ग्राहिकानुप‚ल‚ब्धिर्दृष्टान्ते । न च स‚म्ब‚न्ध‚{\tiny $_{lb}$}‚ग्राह‚क‚स्य प्र‚माण‚स्य रूपं स‚म्ब‚न्धिनोर्भ‚व‚ति । न ह्य‚ग्निधूम‚योः \add{स‚म्ब‚न्ध‚ग्राह}\edtext{}{\edlabel{pvsvt_37-1}\label{pvsvt_37-1}\lemma{योः}\Bfootnote{In the margin. }}क‚स्य‚{\tiny $_{lb}$}‚ प्र‚त्य‚क्ष‚स्य रूपं धूम‚स्य भ‚व‚ति नाग्नेः । त‚त्क‚थ‚म्विरुद्धोप‚ल‚ब्धिर‚नुप‚ल‚ब्धिर्भ‚व‚त्यु‚{\tiny $_{lb}$}‚प‚ल‚ब्धि‚{\tiny $_{५}$}‚रूप‚त‚या प्र‚तिभास‚नात् । त‚स्माद् दृष्टान्ते गृहीत‚विरोध‚म‚ग्न्यादिक‚म‚न्य‚त्र‚{\tiny $_{lb}$}‚ प्र‚देशे दृष्ट्वा शीताद्य‚भावोनुमीय‚त इति ।
	{\color{gray}{\rmlatinfont\textsuperscript{§~\theparCount}}}
	\pend% ending standard par
      ‚{\tiny $_{lb}$}‚

	  
	  \pstart \leavevmode% starting standard par
	अत्रोच्य‚ते । य‚दि विरोध‚स‚म्ब‚न्ध‚द्वारेण ग‚म्य‚ग‚म‚क‚भावो विरोधिनान्त‚त‚श्चा‚{\tiny $_{lb}$}‚‚{\tiny $_{lb}$}‚ ‚{\tiny $_{lb}$}‚ \leavevmode\ledsidenote{\textenglish{38/s}}ग्निद‚र्श‚नाच्छीत‚स्य प्र‚तीतिः स्यात्त‚योरेव विरोध‚स‚म्ब‚न्धेन \add{स‚मुत्थितत्वात् ।‚{\tiny $_{lb}$}‚ न शीता}\edtext{}{\edlabel{pvsvt_38-1}\label{pvsvt_38-1}\lemma{न्धेन}\Bfootnote{In the margin. }}भाव‚स्यास‚म्ब‚न्धित्वात् । य‚दा चाग्निना शीतो निव‚र्त्त्य‚ते त‚दा क‚थं‚{\tiny $_{lb}$}‚ विरोधः स‚म्ब‚न्धो‚{\tiny $_{६}$}‚ द्विष्ठ‚त्वाभावात् । न च शीताभावेन स‚हाग्नेर्विरोधः स‚हाव‚स्था‚{\tiny $_{lb}$}‚नात् । नाप्य‚ग्नेः शीताभावाव्य‚भिचारित्वात्त‚तः शीताभावाद्य‚नुमानं । अग्निभावेपि‚{\tiny $_{lb}$}‚ शीत‚भाव‚द‚र्श‚नात् । त‚स्माद् य‚था य‚थाग्निस‚द्भावे शीत‚स्यानुप‚ल‚म्भ‚स्त‚था त‚था‚{\tiny $_{lb}$}‚ दृष्टान्ते त‚द‚भाव‚स्य प्र\add{तिप‚न्न‚त्वात् । अन्य‚त्राप्य‚ग्नेः}\edtext{}{\edlabel{pvsvt_38-1b}\label{pvsvt_38-1b}\lemma{प्र}\Bfootnote{१ In the margin.}} शीताभाव‚स्त‚थैवानु‚{\tiny $_{lb}$}‚\leavevmode\ledsidenote{\textenglish{16a/PSVTa}} मीय‚ते । तेनाय‚म‚र्थो‚{\tiny $_{७}$}‚ भ‚व‚ति [।] साध्य‚ध‚र्मिण्य‚प्य‚ग्निस‚द्भावे य‚दा शीत‚स्यानु‚{\tiny $_{lb}$}‚प‚ल‚म्भ‚स्त‚दैवाग्नेर्विरोधित्वं नान्य‚दा । विरोधित्वेन च ग‚म‚क‚त्व‚मित्य‚न्यानुप‚ल‚म्भेन‚{\tiny $_{lb}$}‚ ग‚म‚क‚त्व‚मित्युक्त‚म्भ‚व‚ति [।] अतो विरुद्धोप‚ल‚ब्धिर‚नुप‚ल‚ब्धिनिय‚त‚त्वाद‚नुप‚ल‚ब्धि‚{\tiny $_{lb}$}‚रूप‚त्वं । केव‚ल‚भूत‚लोप‚ल‚ब्धेर्घ‚टानुप\add{ल‚ब्धिरूप‚त्व‚व‚त् । ए}\edtext{}{\edlabel{pvsvt_38-1c}\label{pvsvt_38-1c}\lemma{टानुप}\Bfootnote{१ In the margin.}}वं प‚र‚स्प‚र‚प‚रि‚{\tiny $_{lb}$}‚हारेपि विरुद्धोप‚ल‚ब्धिर‚नुप‚ल‚ब्धिरेव । य‚थार्ह‚तां स्याद्[।]अनित्यो‚{\tiny $_{१}$}‚घ‚टः‚{\tiny $_{lb}$}‚ स्यान्नित्य इत्य‚त्र ह्य‚नित्योप‚ल‚ब्धिरेव नित्यानुप‚ल‚ब्धिर‚तो न नित्यानित्य‚{\tiny $_{lb}$}‚रूप‚म्व‚स्तु ॥
	{\color{gray}{\rmlatinfont\textsuperscript{§~\theparCount}}}
	\pend% ending standard par
      ‚{\tiny $_{lb}$}‚

	  
	  \pstart \leavevmode% starting standard par
	एव‚म‚र्थान्त‚र‚विधाव‚पि निषेध्य‚स्यानुप‚ल‚ब्धिङ्ग‚मिकां प्र‚तिपाद्यार्थान्त‚र‚निषे‚{\tiny $_{lb}$}‚धेपि प्र‚तिपाद‚यितुमाह । \textbf{एक‚स्ये}त्यादि । क‚स्मात् [।] \textbf{निषेध‚स्ये}त्यादि । न चार्था‚{\tiny $_{lb}$}‚न्त‚राभावाद\add{र्थान्त‚र‚निषेधेति}\edtext{}{\edlabel{pvsvt_38-1d}\label{pvsvt_38-1d}\lemma{राभावाद}\Bfootnote{१ In the margin.}}प्र‚स‚ङ्गः [।] य‚त\textbf{स्त‚त्रा}पीत्य‚न्य‚निषेधे साध्ये‚{\tiny $_{lb}$}‚ \textbf{कार्य‚काह‚ण‚यो}र्निषेधो हेतुः स्यात् ।‚{\tiny $_{२}$}‚ \textbf{अनुभ‚य‚स्या}कार्य‚स्याकार‚ण‚स्य \textbf{वा} निषेधो‚{\tiny $_{lb}$}‚ हेतुः स्यात् । \textbf{त‚त्र} तेषु । \textbf{अनुभ‚य‚स्}याकार्य‚कार‚णात्म‚क‚स्य निषेध्येन स‚ह \textbf{प्र‚तिब‚न्धा‚{\tiny $_{lb}$}‚भावात् त‚द‚भावे}ऽप्र‚तिब‚द्ध‚स्याभावे । \textbf{अन्येने}ति प्र‚तिषेध्याभिम‚तेन । न \textbf{भ‚वित‚व्य‚{\tiny $_{lb}$}‚मिति कुत एत}त् । \textbf{कार्यानुप‚ल‚ब्धाव‚पि} कुत‚स्त‚द‚भाव इति स‚म्ब‚न्धः । \textbf{त‚द‚भाव} इति‚{\tiny $_{lb}$}‚ कार‚णाभावः । \textbf{य‚तो नानाकार‚णानि त‚द्व}न्ति कार्य‚व‚न्ति‚{\tiny $_{१}$}‚\textbf{वैक‚ल्य}‚{\tiny $_{३}$}‚प्र‚तिब‚न्ध‚स‚म्भ‚वात् ।‚{\tiny $_{lb}$}‚ एत‚च्च कार‚ण‚मात्र गृहीत्वोक्त‚म‚प्र‚तिब‚द्धासाम‚र्थ्य‚स्य कार‚ण‚स्याभावः कार्याभावाद्‚{\tiny $_{lb}$}‚ ग‚म्य‚त एव[।] य‚तोनुभ‚य‚स्य कार्य‚स्य चानुप‚ल‚ब्धिर्ग‚मिका । \textbf{त‚स्मात् कार‚णानुप‚ल-}‚{\tiny $_{lb}$}‚ ‚{\tiny $_{lb}$}‚ ‚{\tiny $_{lb}$}‚ \leavevmode\ledsidenote{\textenglish{39/s}}\textbf{ब्धिरेवाभावं ग‚म‚य‚ति} । कार्य‚स्येति स‚म्ब‚न्धाद् ग‚म्य‚ते ॥
	{\color{gray}{\rmlatinfont\textsuperscript{§~\theparCount}}}
	\pend% ending standard par
      ‚{\tiny $_{lb}$}‚

	  
	  \pstart \leavevmode% starting standard par
	स्व‚भावानुप‚ल‚ब्धेर‚प्य‚भाव‚हेतुत्वाद‚व‚धार‚ण‚म‚युक्त‚मिति चेदाह । \textbf{स्व‚भावानुप‚ल‚{\tiny $_{lb}$}‚ब्धिस्तु स्व‚य‚म‚स‚त्तैव‚{\tiny $_{४}$}‚} [।] \textbf{नात्रार्थान्त‚र‚स्याभावः साध्य‚ते} । स्व‚भावान्त‚र‚स्य च निषेधे‚{\tiny $_{lb}$}‚ साध्ये कार‚णानुप‚ल‚ब्धिरेवेत्य‚व‚धार‚ण‚म‚तो न व्याघातः ।
	{\color{gray}{\rmlatinfont\textsuperscript{§~\theparCount}}}
	\pend% ending standard par
      ‚{\tiny $_{lb}$}‚

	  
	  \pstart \leavevmode% starting standard par
	किन्त‚र्हि त‚त्र साध्य‚त इत्याह । \textbf{केव‚ल}मित्यादि । \textbf{त‚त्रे}ति स्व‚भावानुप‚ल‚ब्धौ‚{\tiny $_{lb}$}‚ \textbf{विष‚यी} अस‚ज्ज्ञानादिः । \textbf{अस्याम}पीति स्व‚भावानुप‚ल‚ब्धौ \textbf{य‚दा व्याप}को \textbf{यो ध‚र्म}स्त‚{\tiny $_{lb}$}‚स्या\textbf{नुप‚ल‚ब्ध्या व्याप्याभाव‚माह} । य‚था नात्र शिंश‚पा वृक्षाभावादिति । \textbf{त‚दाऽभे‚{\tiny $_{५}$}‚ \edtext{\textsuperscript{*}}{\edlabel{pvsvt_39-1}\label{pvsvt_39-1}\lemma{*}\Bfootnote{Cf. त‚दाऽभावोऽपि}}}‚{\tiny $_{lb}$}‚दोपि व्याप्य‚स्य साध्य‚ते । अपिश‚ब्दाद् व्य‚व‚हारोपि । य‚दा हि स‚मुन्न‚त‚योः प‚र्व‚त‚{\tiny $_{lb}$}‚प्र‚देश‚योरेक‚स्मिन् प्र‚देशे त‚रुव‚नं दृश्य‚मान‚म‚तिग‚ह‚न‚त्वाद‚न‚व‚धारित‚वृक्ष‚विशेष‚म्भ‚व‚ति ।‚{\tiny $_{lb}$}‚ त‚त्राप‚र‚स्मिन् प‚र्व्व‚तोद्देशे शिंश‚पाभावो न निश्चेतुम्पार्य‚ते शिंश‚पाया अदृश्य‚त्वाद्‚{\tiny $_{lb}$}‚ \textbf{वृ}क्ष‚स्तूप‚ल‚ब्धिल‚क्ष‚ण‚प्राप्त इति श‚क्य‚तेऽभाव‚निश्च‚यः क‚र्त्तुन्त‚स्य । त‚दा व्याप‚का‚{\tiny $_{lb}$}‚भावा‚{\tiny $_{६}$}‚द् व्याप्य‚स्याभावः साध्य‚ते ॥
	{\color{gray}{\rmlatinfont\textsuperscript{§~\theparCount}}}
	\pend% ending standard par
      ‚{\tiny $_{lb}$}‚

	  
	  \pstart \leavevmode% starting standard par
	\textbf{इय‚मि}त्यादिनाऽ\textbf{भावार्थे}त्यादिकारिकाभागं व्याच‚ष्टे । \textbf{विरुद्ध}सिद्ध्येति ।‚{\tiny $_{lb}$}‚ स्व‚भाव‚विरुद्धोप‚ल‚ब्ध्या । \textbf{य‚थेत्या}दि । य‚त्र ध‚र्मिणि शीत‚स्प‚र्शः प‚रोक्षः । व‚ह्निश्च‚{\tiny $_{lb}$}‚ दृश्य‚ते [।] दूरात्त‚स्मिन् विष‚येऽयं प्र‚योगः । \textbf{एतेने}ति स्व‚भाव‚विरुद्धोप‚ल‚ब्ध्युदाह‚{\tiny $_{lb}$}‚र‚णेन \textbf{व्याप‚क‚विरुद्ध‚सिद्धिरुक्ता वेदित‚व्या । य‚थे}त्याद्युदाह‚र‚णं । शीत‚{\tiny $_{७}$}‚स्प‚र्श- \leavevmode\ledsidenote{\textenglish{16b/PSVTa}}‚{\tiny $_{lb}$}‚ विशेष एव । हिमानुग‚त‚स्तुषार‚स्त‚स्य व्याप‚कः शीत‚स्त‚स्य विरुद्धोग्निः । त‚त्राग्नि‚{\tiny $_{lb}$}‚र्विरुद्धं शीतं निव‚र्त्त‚य‚न् त‚द्व्याप्य‚न्तुषार‚स्प‚र्श‚म‚पि निव‚र्त्त‚य‚तीत्य‚र्थ‚तः स्व‚भाव‚विरुद्धो‚{\tiny $_{lb}$}‚प‚ल‚ब्धाविय‚म‚न्त‚र्भ‚व‚ति । य‚त्र च व‚ह्निः प्र‚त्य‚क्षः शीत‚स्प‚र्श‚स्तुषार‚स्प‚र्श‚श्चाप्र‚त्य‚क्ष‚{\tiny $_{lb}$}‚स्त‚त्राय‚म‚पि प्र‚योगो द्र‚ष्ट‚व्यः । \textbf{विरुद्ध‚स्य य‚त्कार्य‚न्त‚स्यासिद्ध्या । य‚थे}त्यादि ।‚{\tiny $_{lb}$}‚ \textbf{न शीत‚स्प‚र्शोत्र धूमादिति} शी‚{\tiny $_{१}$}‚त‚विरुद्धोग्निस्त‚स्य कार्यं धूमः सोग्निं स‚न्निधाप‚य‚ति‚{\tiny $_{lb}$}‚ स च शीत‚म‚प‚न‚य‚ति । य‚त्र च शीत‚स्प‚र्शः प‚रोक्षो व‚ह्निर‚पि धूम‚श्च प्र‚त्य‚क्ष‚स्त‚त्रायं‚{\tiny $_{lb}$}‚ प्र‚योगो द्र‚ष्ट‚व्यः । एव‚न्ताव\textbf{द्विरुद्ध‚कार्य‚योः सिद्धि}रित्येत‚द् व्याख्यातं ।
	{\color{gray}{\rmlatinfont\textsuperscript{§~\theparCount}}}
	\pend% ending standard par
      ‚{\tiny $_{lb}$}‚

	  
	  \pstart \leavevmode% starting standard par
	अधुना [।]
	{\color{gray}{\rmlatinfont\textsuperscript{§~\theparCount}}}
	\pend% ending standard par
      ‚{\tiny $_{lb}$}‚‚{\tiny $_{lb}$}‚‚{\tiny $_{lb}$}‚\textsuperscript{\textenglish{40/s}}
	  \bigskip
	  \begingroup
	असिद्धिर्हेतुभाव‚योः ।
	  \endgroup
	‚{\tiny $_{lb}$}‚

	  
	  \pstart \leavevmode% starting standard par
	इत्येत‚दुदाह‚र‚णाख्यानेनाच‚ष्टे [।] \textbf{हेत्व‚सिद्ध्ये}ति कार‚णानुप‚ल‚ध्या । \textbf{य‚थे}‚{\tiny $_{lb}$}‚त्यादि । य‚दा म‚हाह्र‚द‚स्योप‚रि‚{\tiny $_{२}$}‚ वास्पादौ धूमादिरूप‚त‚या स‚न्देहो भ‚व‚ति‚{\tiny $_{lb}$}‚ त‚दाऽयं प्र‚योगो द्र‚ष्ट‚व्यः । त‚त्र हि निष्क‚म्पे म‚हाह्र‚दे य‚द्य‚ग्निः स्यात् प्र‚भास्व‚र‚{\tiny $_{lb}$}‚त‚या प्र‚त्य‚क्ष एव स्याद् [।] अप्र‚त्य‚क्ष‚त्वादेव व‚ह्नेर‚भावात् कार्याभावः‚{\tiny $_{lb}$}‚ साध्य‚ते । \textbf{स्व‚भावा}सिद्ध्येति स्व‚भावानुप‚ल‚ब्ध्या न त‚त्र धूमोऽनुप‚ल‚ब्धेरिति ।‚{\tiny $_{lb}$}‚ उप‚ल‚ब्धिल‚क्ष‚ण‚प्राप्त‚स्येति द्र‚ष्ट‚व्यं । \textbf{एतेने}ति स्व‚भावानुप‚ल‚म्भ‚प्र‚योगेण । \textbf{व्याप‚{\tiny $_{lb}$}‚क}श्चा‚{\tiny $_{३}$}‚सौ \textbf{स्व‚भाव}श्च त‚स्या\textbf{सिद्धि}र‚नुप‚ल‚ब्धि\textbf{रुक्ता । य‚था नात्र शिंश‚पा‚{\tiny $_{lb}$}‚ वृक्षाभावात्} । स्व‚भाव एव वृक्ष‚त्वं शिंश‚पात्व‚स्यातो वृक्ष‚त्वेन शिंश‚पा‚{\tiny $_{lb}$}‚ व्याप्ता ॥
	{\color{gray}{\rmlatinfont\textsuperscript{§~\theparCount}}}
	\pend% ending standard par
      ‚{\tiny $_{lb}$}‚

	  
	  \pstart \leavevmode% starting standard par
	न‚नु च स्व‚भावानुप‚ल‚ब्धावेव दृश्य‚निषेधो न तु विरुद्धोप‚ल‚ब्ध्यादौ [।] त‚था‚{\tiny $_{lb}$}‚ ह्य‚दृश्य‚स्यैव शीत‚स्प‚र्श‚स्य कार्य‚स्य चादृश्य‚स्यैव निषेधः साध्य‚ते [।] दृश्य‚त्वे हि‚{\tiny $_{lb}$}‚ स्व‚भावानुप‚ल‚ब्धिरेव स्यात् । तेन य‚थाऽदृश्य‚स्य शीत‚स्प‚{\tiny $_{४}$}‚र्शादेर्निषेध‚स्त‚था \textbf{पिशा}‚{\tiny $_{lb}$}‚चादेर‚पि स्यादित्य‚त आह । \textbf{स‚र्व‚त्रे}त्यादि ।
	{\color{gray}{\rmlatinfont\textsuperscript{§~\theparCount}}}
	\pend% ending standard par
      ‚{\tiny $_{lb}$}‚

	  
	  \pstart \leavevmode% starting standard par
	एत‚दुक्त‚म्भ‚व‚ति । य‚था स्व‚भावानुप‚ल‚ब्धाव‚न्य‚त्रोप‚ल‚ब्ध‚स्य घ‚टादेः प्र‚देश‚{\tiny $_{lb}$}‚विशेषेऽभावः साध्य‚ते त‚थान्य‚त्र प्र‚तिप‚न्न‚विरोध‚स्य शीत‚स्प‚र्श‚स्य प्र‚तिप‚न्न‚कार्य‚त्व‚स्य‚{\tiny $_{lb}$}‚ च कार्य‚स्य विरुद्धोप‚ल‚ब्ध्यादिना साध्य‚ध‚र्मिण्य‚भावः साध्य‚ते न तु स‚र्व‚दाऽदृश्य‚स्येति ।‚{\tiny $_{lb}$}‚ \textbf{स‚र्व}त्रेति विधिमुखेन‚{\tiny $_{५}$}‚ प्र‚तिषेध‚मुखेन च प्र‚युक्तायाम‚भाव‚साध‚न्यामित्य‚भाव‚श्चाभा‚{\tiny $_{lb}$}‚व‚व्य‚व‚हार‚श्चाभाव‚श‚ब्देनोक्तः ।
	{\color{gray}{\rmlatinfont\textsuperscript{§~\theparCount}}}
	\pend% ending standard par
      ‚{\tiny $_{lb}$}‚

	  
	  \pstart \leavevmode% starting standard par
	\textbf{तेषाम‚पि} येषाम‚भावेनाभावः साध्य‚ते तेषां कार‚णादीनां \textbf{दृश्यात्म‚ना}मेवा\textbf{सिद्धि}‚{\tiny $_{lb}$}‚र‚नुप‚ल‚ब्धिः । \textbf{त‚द्विरुद्धानां च} प्र‚तिषेध्यार्थ‚विरुद्धानाञ्च स्व‚भाव‚विरु\add{द्धादीनां‚{\tiny $_{lb}$}‚ ‚{\tiny $_{lb}$}‚ \leavevmode\ledsidenote{\textenglish{41/s}}\textbf{सिद्धि}रुप‚ल}\edtext{}{\edlabel{pvsvt_41-1}\label{pvsvt_41-1}\lemma{विरु}\Bfootnote{In the margin. }}ब्धिलिङ्ग‚त्वेन \textbf{वेदित‚व्या} । अत्र च \textbf{दृश्यात्म‚नां सिद्धि}रिति न‚{\tiny $_{lb}$}‚ स‚म्ब‚ध्य‚ते उप‚ल‚{\tiny $_{६}$}‚ब्धिव‚च‚नादेव दृश्यात्म‚ताया ल‚ब्ध‚त्वात् । किन्त्वेवं स‚म्ब‚न्धः‚{\tiny $_{lb}$}‚ क‚र्त्त‚व्यः । \textbf{दृश्यात्म‚ना}न्निषेध्याभिम‚तानां ये विरुद्धास्तेषां सिद्धिरिति । किंङ्का‚{\tiny $_{lb}$}‚र‚ण‚म् [।] \textbf{अन्येषा}म‚दृश्यात्म‚नां कार‚णादीना\textbf{म‚भावासिद्धेः} । अदृश्यानां निषे‚{\tiny $_{lb}$}‚ध्याभिम‚तानां स्व‚भाव‚विरुद्धादिति विरो\add{धाद्य‚सिद्धिः ॥‚{\tiny $_{lb}$}‚ \textbf{य‚दी}}\edtext{}{\edlabel{pvsvt_41-1b}\label{pvsvt_41-1b}\lemma{विरो}\Bfootnote{१ In the margin.}}त्यादि प‚रः । शीत\textbf{विरुद्ध}स्याग्नेः \textbf{कार्यं} धूम‚स्यो\textbf{प‚ल‚ब्ध्याप्य‚भाव‚{\tiny $_{lb}$}‚सिद्धिः‚{\tiny $_{७}$}‚} शीत‚स्य । त‚त्कार‚णोप‚ल‚ब्ध्या । \textbf{त‚स्य} शीत‚विरुद्ध‚स्याग्नेर्य‚त् \textbf{कार‚णं} \leavevmode\ledsidenote{\textenglish{17a/PSVTa}}‚{\tiny $_{lb}$}‚ काष्ठादिस्त‚स्यो\textbf{प‚ल‚ब्ध्या किन्न सिध्य‚ति} शीताभावः ।
	{\color{gray}{\rmlatinfont\textsuperscript{§~\theparCount}}}
	\pend% ending standard par
      ‚{\tiny $_{lb}$}‚

	  
	  \pstart \leavevmode% starting standard par
	\textbf{त‚द्विरुद्ध}स्येत्या\textbf{द्याचार्यः} । तेन शीत‚स्प‚र्शेन विरुद्ध‚स्य व‚ह्ने\textbf{र्निमित्तं} काष्ठादि‚{\tiny $_{lb}$}‚\textbf{स्त‚स्य योप‚ल‚ब्धिः प्र‚युज्य‚ते सा व्य‚भिचारिणी} । क‚दा [।] \textbf{निमित्त‚योर्विरुद्ध‚त्वा}‚{\tiny $_{lb}$}‚\add{\textbf{भावे} स‚ति । अ}\edtext{\textsuperscript{*}}{\edlabel{pvsvt_41-1c}\label{pvsvt_41-1c}\lemma{*}\Bfootnote{१ In the margin.}}ग्निशीत‚निमित्त‚योर्विरोधाभावे स‚ति । उदाह‚र‚ण‚माह ।‚{\tiny $_{lb}$}‚ \textbf{य‚थे}त्यादि । अत्र हि काष्ठ‚मा‚{\tiny $_{१}$}‚त्र‚स्य द‚ह‚न‚हेतोः शीत‚निमित्तेन तुषारादिना विरोधा‚{\tiny $_{lb}$}‚भावात् । य‚त्पुन‚र‚प्र‚तिब‚द्ध‚साम‚र्थ्य‚म‚ग्निज‚न‚कं काष्ठ‚न्त‚स्य य‚द्य‚पि शीत‚निमित्तेन‚{\tiny $_{lb}$}‚ विरोध‚स्त‚थापि त‚थाभूत‚स्य काष्ठ‚स्य कार्य‚द‚र्श‚नादेव निश्च‚यात् कार्य‚विरोध एव‚{\tiny $_{lb}$}‚ स्यात् । \textbf{निमित्त‚योः} पुन\textbf{र्विरो}धे \textbf{ग‚मिकै}व च कार‚णानुप‚ल‚ब्धिः । \textbf{य‚था नास्य पुं}सः‚{\tiny $_{lb}$}‚ \textbf{रोम‚ह‚र्षादिविशेषाः} । आदिश‚व्दाद् द‚न्त‚वीणाक‚{\tiny $_{२}$}‚म्पाद‚यः । विशेष‚ग्र‚ह‚णं शीत‚{\tiny $_{lb}$}‚कार्याणां प‚रिहारार्थं । पिशाचादिविकार‚कृता अपि हि ते स‚म्भ‚व‚न्ति । \textbf{स‚न्निहितो‚{\tiny $_{lb}$}‚ द‚ह‚न‚विशेषो} य‚स्य पुरुष‚स्य स त‚था त‚द्भाव‚स्त‚स्मात्[।]अत्रापि विशेष‚ग्र‚ह‚णं य‚था‚{\tiny $_{lb}$}‚भूतो द‚ह‚नो रोम‚ह‚र्षाद्य‚प‚न‚य‚न‚स‚म‚र्थ‚स्त‚थाभूत‚स्य प‚रिग्र‚हार्थं । अत्र हि प‚रिताप‚{\tiny $_{lb}$}‚निमित्त‚स्य द‚ह‚न‚स्य रोम‚ह‚र्षादिनिमित्तेन शीतेन विरोधो‚{\tiny $_{३}$}‚स्ति । त‚स्माद् द‚ह‚नः‚{\tiny $_{lb}$}‚ स्व‚विरुद्धं शीत‚म‚प‚न‚यंस्त‚त्कार्य‚म‚पि रोम‚ह‚र्षादिक‚म‚प‚न‚य‚तीति । शीत‚स्प‚र्श‚स्य त‚त्का‚{\tiny $_{lb}$}‚र्य‚स्य च रोम‚ह‚र्षादेः प‚रोक्ष‚त्वे स‚ति व‚ह्निद‚र्श‚नाच्छीत‚कार‚ण‚निवृत्त्या य‚दा रोम‚{\tiny $_{lb}$}‚ह‚र्षादेर्निवृत्तिः साध्याभिप्रेता त‚दाऽयं प्र‚योगो द्र‚ष्ट‚व्यः ।
	{\color{gray}{\rmlatinfont\textsuperscript{§~\theparCount}}}
	\pend% ending standard par
      ‚{\tiny $_{lb}$}‚

	  
	  \pstart \leavevmode% starting standard par
	\textbf{एतेन} कार‚ण‚विरुद्धोदाह‚र‚णेन \textbf{त‚त्कार्याद}पीति विरुद्ध‚स्य य‚त्कार्य‚न्त‚स्माद‚पि ।‚{\tiny $_{lb}$}‚ ‚{\tiny $_{lb}$}‚ ‚{\tiny $_{lb}$}‚ \leavevmode\ledsidenote{\textenglish{42/s}}\textbf{त‚द्विरुद्ध‚कार्याभा‚{\tiny $_{४}$}‚व‚ग‚तिरुक्तेति} । य‚स्य विरुद्ध‚स्य कार्य‚मुप‚ल‚भ्य‚ते तेन विरुद्धो यो‚{\tiny $_{lb}$}‚ द्वितीयः प्र‚तियोगी स त‚द्विरुद्ध‚स्त‚स्य य‚त् कार्य‚न्त‚स्याभाव‚ग‚तिरुक्ता । \textbf{य‚थे}त्युदाह‚{\tiny $_{lb}$}‚र‚णं । \textbf{रोम‚ह‚र्षादिविशे}षो यः शीत‚कृत‚स्तेन \textbf{युक्तो} यः \textbf{पुरुषः} स य‚स्मिन् प्र\textbf{देशे}‚{\tiny $_{lb}$}‚विद्य‚ते स त‚था । प्र‚देशोपादानं धूमादित्य‚स्य प‚क्ष‚ध‚र्म‚त्व‚प्र‚तिपाद‚नार्थं । पुरुषे हि‚{\tiny $_{lb}$}‚ ध‚र्मिणि न प‚क्ष‚ध‚र्मो हे‚{\tiny $_{५}$}‚तुः स्याद् धूम‚स्य प्र‚देश‚ध‚र्म‚त्वात् । अयं च प्र‚योगः [।]‚{\tiny $_{lb}$}‚ य‚त्र व‚ह्निः शीत‚स्प‚र्शो रोम‚ह‚र्षादिविशेष‚श्च प‚रोक्ष‚स्त‚त्राभाव‚साध‚ने द्र‚ष्ट‚व्यः ।‚{\tiny $_{lb}$}‚ त‚त्र हि शीत‚विरुद्धाग्निकार्य‚स्य धूम‚स्योप‚ल‚ब्ध्या अग्निविरुद्ध‚शीत‚कार्य‚स्य रोम‚ह‚{\tiny $_{lb}$}‚र्षादेर‚भावः साध‚यितुमिष्ट‚स्तेन त‚थाभूत एव साध्यः । धूमो हि द‚ह‚नं स‚न्निधा‚{\tiny $_{lb}$}‚प‚य‚ति स शीत‚न्निव‚र्त‚य‚ति [।] स निव‚र्त्त‚मानः‚{\tiny $_{६}$}‚ स्व‚कार्यं रोम‚ह‚र्षादिक‚न्निव‚र्त्त‚य‚ती‚{\tiny $_{lb}$}‚त्य‚र्थात् कार‚ण‚विरुद्धोप‚ल‚ब्धिरियं ।
	{\color{gray}{\rmlatinfont\textsuperscript{§~\theparCount}}}
	\pend% ending standard par
      ‚{\tiny $_{lb}$}‚

	  
	  \pstart \leavevmode% starting standard par
	\leavevmode\ledsidenote{\textenglish{17b/PSVTa}} \textbf{इय‚न्त‚द्विरुद्धोप‚ल‚ब्धि} \href{http://sarit.indology.info/?cref=}{१ । ३२}रिति स‚म्ब‚न्धः । कार‚ण‚विरुद्धोप‚ल‚ब्धिरित्य‚र्थः ।‚{\tiny $_{lb}$}‚ \textbf{हेत्व‚सिद्ध्यै}व कार‚णानुप‚ल‚ब्ध्यैव \textbf{प्रागेव निर्दिष्टा} । य‚स्माद‚न‚योर‚पि प्र‚योग‚योः‚{\tiny $_{lb}$}‚ कार‚णानुप‚ब्धेरेव कार्याभाव‚ग‚तिस्त‚स्मादियं कार‚णानुप‚ल‚ब्ध्यैवोक्ता ।
	{\color{gray}{\rmlatinfont\textsuperscript{§~\theparCount}}}
	\pend% ending standard par
      ‚{\tiny $_{lb}$}‚

	  
	  \pstart \leavevmode% starting standard par
	\textbf{इतीय}मित्यादि । मौलेन प्र‚भेदेन च‚तुर्विधापि स‚ती अवान्त‚र‚प्र\textbf{योग‚भेदाद्‚{\tiny $_{lb}$}‚ अष्ट‚विधा} भ‚व‚ति । त‚था हि स्व‚भाव‚विरुद्धोप‚ल‚ब्धेर्व्याप‚क‚विरुद्धोप‚ल‚ब्धिः प्र‚भेद‚{\tiny $_{lb}$}‚ उक्तः । स्व‚भावानुप‚ल‚ब्धेर्व्याप‚कानुप‚ल‚ब्धिः । कार‚णानुप‚ल‚ब्धेः कार‚ण‚विरुद्धो‚{\tiny $_{lb}$}‚प‚ल‚ब्धिः कार‚ण‚विरुद्ध‚कार्योप‚ल‚ब्धिश्च । विरुद्ध‚कार्योप‚ल‚ब्धिस्त्वेक‚प्र‚कारेत्य‚ष्ट‚वि‚{\tiny $_{१}$}‚धा‚{\tiny $_{lb}$}‚ भ‚व‚ति ॥
	{\color{gray}{\rmlatinfont\textsuperscript{§~\theparCount}}}
	\pend% ending standard par
      ‚{\tiny $_{lb}$}‚

	  
	  \pstart \leavevmode% starting standard par
	न‚नु चिर‚विन‚ष्टेप्य‚ग्नौ वास‚गृहादौ धूम‚स्य स‚द्भावात् क‚थ‚म्विरुद्ध‚कार्योप‚ल‚ब्धेर्न‚{\tiny $_{lb}$}‚ व्य‚भिचार इत्य‚त आह ।
	{\color{gray}{\rmlatinfont\textsuperscript{§~\theparCount}}}
	\pend% ending standard par
      ‚{\tiny $_{lb}$}‚

	  
	  \pstart \leavevmode% starting standard par
	\textbf{त‚त्रे}त्यादि । \textbf{त‚त्र विरुद्ध‚कार्येपीष्टं देश‚कालाद्य‚पेक्ष‚णं} । नात्र शीतः क‚स्मिंश्चित्‚{\tiny $_{lb}$}‚ काले य‚दाऽग्निर्व‚र्त‚मानीभूत इति कालापेक्ष‚णं । व्योम्नि धूमात् कुत्र‚चिद्देशे नास्ति‚{\tiny $_{lb}$}‚ शीतो य‚त्र स‚न्निहितो व‚ह्निर्य‚तोयं धूम उत्थित इति देशापेक्ष‚णं ।‚{\tiny $_{२}$}‚ आदिंश‚ब्दाद्‚{\tiny $_{lb}$}‚ अव‚स्थाविशेषापेक्ष‚णं [।] योव‚स्थाविशेषो धूम‚स्य स‚न्निहिताग्नेर्दृष्ट‚स्त‚म‚पेक्ष्य‚{\tiny $_{lb}$}‚ व‚र्त्त‚मानेपि काले शीताभावोनुमीय‚ते । अग्निर‚त्र धूमादिति कार्य‚हेताव‚पि देश‚का‚{\tiny $_{lb}$}‚लाद्य‚पेक्ष‚ण‚मिष्टं । अस्यैवार्थ‚स्य स‚मुच्च‚यार्थोऽपिश‚ब्दः ।
	{\color{gray}{\rmlatinfont\textsuperscript{§~\theparCount}}}
	\pend% ending standard par
      ‚{\tiny $_{lb}$}‚

	  
	  \pstart \leavevmode% starting standard par
	न‚नु देश‚कालाद्य‚पेक्षित्व‚स्य कार्य‚हेतुविशेष‚ण‚त्वेऽसिद्धो हेतुः स्याद् ध‚र्मिणो‚{\tiny $_{lb}$}‚ऽभावादिति चेत् [।] न । प्र‚देश ए‚{\tiny $_{३}$}‚व ध‚र्मिणि देश‚कालाद्य‚पेक्षित्वेन ग‚म‚क‚त्वादि‚{\tiny $_{lb}$}‚त्युक्त‚त्वात् ।‚{\tiny $_{lb}$}‚ ‚{\tiny $_{lb}$}‚ \leavevmode\ledsidenote{\textenglish{43/s}}\textbf{अन्य‚थे}ति य‚दि न देशाद्य‚पेक्ष‚ण‚न्त‚दा \textbf{व्य‚भिचारि} विरुद्ध‚कार्यं \textbf{स्यात्} । य‚था \textbf{भ‚स्मा}‚{\tiny $_{lb}$}‚न‚पेक्षित‚देश‚कालं \textbf{अशीत‚साध‚ने} शीताभावे साध्ये व्य‚भिचारि । त‚द्व‚त् एव‚न्ताव‚द्‚{\tiny $_{lb}$}‚ विरुद्धाद्युप‚ल‚ब्धिर‚नुप‚ल‚ब्धिरिति प्र‚तिपादितं ॥
	{\color{gray}{\rmlatinfont\textsuperscript{§~\theparCount}}}
	\pend% ending standard par
      ‚{\tiny $_{lb}$}‚

	  
	  \pstart \leavevmode% starting standard par
	\textbf{य‚स्त‚र्ही}त्यादि प‚रः । \textbf{स‚म‚गुणे}ति स‚न्निहितानुप‚हितेन य‚था क्षितिवीजो‚{\tiny $_{४}$}‚‚{\tiny $_{lb}$}‚द‚कादि\textbf{कार‚ण‚क‚लापं} दृष्ट्वांकुरः कार्योनुमीय‚ते । \textbf{स क‚थं} कार‚णाख्यो हेतु‚{\tiny $_{lb}$}‚\textbf{स्त्रिविधे} स्व‚भाव‚कार्यानुप‚ल‚म्भाख्ये \textbf{हेताव‚न्त‚र्भ‚व‚ति} । न ताव‚द‚नुप‚ल‚ब्धौ विधि‚{\tiny $_{lb}$}‚साध‚न‚त्वात् । कार‚ण‚स्व‚भाव‚त्वान्न कार्य‚हेतौ । अर्थान्त‚रेणार्थान्त‚र‚स्यानुमानान्न‚{\tiny $_{lb}$}‚ स्व‚भाव‚हेतौ ॥
	{\color{gray}{\rmlatinfont\textsuperscript{§~\theparCount}}}
	\pend% ending standard par
      ‚{\tiny $_{lb}$}‚

	  
	  \pstart \leavevmode% starting standard par
	अन्त‚र्भाव‚माह ।
	{\color{gray}{\rmlatinfont\textsuperscript{§~\theparCount}}}
	\pend% ending standard par
      ‚{\tiny $_{lb}$}‚

	  
	  \pstart \leavevmode% starting standard par
	\textbf{हेतु}नेत्यादि । \textbf{स‚म‚ग्रेणेति} याव‚तः कार‚ण‚क‚लापात् कार्य‚मुत्प‚द्य‚मानं दृष्ट‚न्ताव‚ता‚{\tiny $_{५}$}‚‚{\tiny $_{lb}$}‚ नान्त्याव‚स्थाप्राप्तेन त‚त्र लिङ्गिग्र‚ह‚णात् प्रागेव कार्य‚स्य प्र‚त्य‚क्ष‚त्वात् । अप्र‚त्य‚क्ष‚त्वे‚{\tiny $_{lb}$}‚ वाऽन्त्याव‚स्थानिश्च‚यायोगाद‚नुमानं । न च य‚स्तान् निश्चेतुं श‚क्ष्य‚ति त‚स्यानुमान‚{\tiny $_{lb}$}‚म‚न्त्य‚क्ष‚णानाम‚र्वाग्द‚र्श‚नेनानिश्च‚यात् । \textbf{यः कार्योत्पादोनुमीय‚ते} स हेतोः \textbf{स्व‚भावो‚{\tiny $_{lb}$}‚ व‚र्णितः} । कुतो\textbf{र्थान्त‚रान‚पेक्ष‚त्वात्} । तेनाय‚म‚र्थः कार्योत्पाद‚न‚योग्य‚ता\textbf{मात्रा}नुब‚न्धि‚{\tiny $_{६}$}‚\textbf{त्वात्‚{\tiny $_{lb}$}‚ स्व‚भाव‚भूता} ॥
	{\color{gray}{\rmlatinfont\textsuperscript{§~\theparCount}}}
	\pend% ending standard par
      ‚{\tiny $_{lb}$}‚

	  
	  \pstart \leavevmode% starting standard par
	न‚नु य‚दाऽन्त्याव‚स्थापेक्षः कार्योत्पाद‚स्त‚दा क‚थ‚म‚न्यान‚पेक्ष इत्याह । \textbf{असाव}पीति‚{\tiny $_{lb}$}‚ कार्योत्पादः । य‚थास‚न्निहितो यादृशः स‚न्निहितः । कार‚ण‚क‚लापः । तादृशात्‚{\tiny $_{lb}$}‚ स‚न्निहिता\textbf{न्नान्य‚म‚र्थ‚म‚पेक्ष‚त इति तादृग्मात्रानुब‚न्धी स्व‚भा}वः । क‚स्य भावः ।‚{\tiny $_{lb}$}‚ स‚म‚ग्र‚स्य कार‚ण‚क‚लाप‚स्य [।] स‚न्तानापेक्ष‚यैत‚द् उच्य‚ते न क्ष‚णापेक्ष‚या । ज‚न‚कः‚{\tiny $_{७}$}‚‚{\tiny $_{lb}$}‚ कार‚ण‚क‚लाप‚स‚न्तानोन‚पेक्ष इत्य‚र्थः । \leavevmode\ledsidenote{\textenglish{18a/PSVTa}}
	{\color{gray}{\rmlatinfont\textsuperscript{§~\theparCount}}}
	\pend% ending standard par
      ‚{\tiny $_{lb}$}‚

	  
	  \pstart \leavevmode% starting standard par
	य‚द्य‚प्य‚न‚पेक्षः कार्योत्पाद‚स्त‚थाप्य‚र्थान्त‚र‚त्वात् क‚थं स्व‚भाव इत्याह । \textbf{त‚त्रे}त्यादि ।‚{\tiny $_{lb}$}‚ य‚स्मात्त‚त्र स‚म‚ग्रेषु कार‚णेषु स‚म‚ग्रात् कार‚णाल्लिङ्गात् \textbf{कार्योत्प‚त्ति\textbf{स‚म्भ‚व}}‚{\tiny $_{lb}$}‚स्त‚था\textbf{नुमीय‚ते} । स‚म्भ‚व‚त्य‚स्मादिति स‚म्भ‚वः । कार्योत्पाद‚न‚योग्य‚तानुमीय‚त इत्य‚र्थः ।‚{\tiny $_{lb}$}‚ एत‚देव व्य‚न‚क्ति । \textbf{स‚म‚ग्राणां कार्योत्पाद‚न‚योग्य‚तानुमाना}दिति । \textbf{योग्य‚ता‚{\tiny $_{१}$}‚ च}‚{\tiny $_{lb}$}‚ ‚{\tiny $_{lb}$}‚ \leavevmode\ledsidenote{\textenglish{44/s}}\textbf{साम‚ग्रीमात्रानुव‚न्धिनी} कार‚णान्त‚रान‚पेक्ष‚त्वात् ।
	{\color{gray}{\rmlatinfont\textsuperscript{§~\theparCount}}}
	\pend% ending standard par
      ‚{\tiny $_{lb}$}‚

	  
	  \pstart \leavevmode% starting standard par
	य‚दि त‚र्हि कार्योत्पाद‚न‚श‚क्तिस्त‚न्मात्रानुव‚न्धिनी निय‚त‚स्त‚र्हि कार्योत्पाद‚{\tiny $_{lb}$}‚ इति स एव क‚स्मान्नानुमीय‚त इति प‚रः पृच्छ‚ति । किं पुनः साम‚ग्र्याः स‚काशात्‚{\tiny $_{lb}$}‚ कार्य‚मेवानुमीय‚त इत्य‚त्राह । साम‚ग्रीत्यादि । साम‚ग्र्याः फ‚ल‚ञ्च ताः श‚क्त‚य‚श्चेति‚{\tiny $_{lb}$}‚ साम‚ग्रीफ‚ल‚श‚क्त‚यः । त‚था हि पूर्व‚स्मात् स‚म‚ग्रादुत्त‚र‚स्य‚स‚म‚र्थ‚स्य क्ष‚ण‚{\tiny $_{२}$}‚स्योत्प‚त्तिस्त‚स्य‚{\tiny $_{lb}$}‚ चात्मातिश‚यः श‚क्तिरिति साम‚ग्री फ‚लं श‚क्तिस्तासां प‚रिणामः । उत्त‚रोत्त‚र‚प्र‚ब‚न्धे‚{\tiny $_{lb}$}‚नोत्प‚त्तिस्त‚द‚नुव‚न्धिनि त‚द‚पेक्षिणि कार्ये । कार‚णेनानुमात‚व्येऽनैकान्तिक‚ता ।‚{\tiny $_{lb}$}‚ किङ्कार‚णं [।] प्र‚तिब‚न‚धस्यस‚म्भ‚वात् ।
	{\color{gray}{\rmlatinfont\textsuperscript{§~\theparCount}}}
	\pend% ending standard par
      ‚{\tiny $_{lb}$}‚

	  
	  \pstart \leavevmode% starting standard par
	त‚द्व्याच‚ष्टे । न हीत्यादि । स‚म‚ग्राणीत्येव स‚न्निहितानीत्येव कार‚ण‚द्र‚व्याणि‚{\tiny $_{lb}$}‚ स्व‚कार्य ज‚न‚य‚न्ति । किङ्कार‚णं । साम‚ग्रीत्यादि । साम‚ग्र्याः स‚काशा‚{\tiny $_{३}$}‚ज्ज‚न्म‚{\tiny $_{lb}$}‚ यासां श‚क्तीनान्तासामुत्त‚रोत्त‚र‚प‚रिणामः । पूर्व्व‚पूर्व्व‚क्ष‚णादुत्त‚रोत्त‚र‚विशिष्ट‚क्ष‚णो‚{\tiny $_{lb}$}‚त्पादो य‚स्त‚द‚पेक्ष‚त्वात् कार्योत्पाद‚स्य । अत्रान्त‚रे चेति स‚न्तान‚प‚रिणाम‚काले ।‚{\tiny $_{lb}$}‚ प्र‚तिब‚न्ध‚स‚म्भ‚वात् ।
	{\color{gray}{\rmlatinfont\textsuperscript{§~\theparCount}}}
	\pend% ending standard par
      ‚{\tiny $_{lb}$}‚

	  
	  \pstart \leavevmode% starting standard par
	न‚नु योग्य‚ताप्युत्त‚रोत्त‚र‚क्ष‚ण‚प‚रिणाम‚प्र‚तिव‚द्धा त‚त्रापि च प्र‚तिब‚न्ध‚स‚म्भ‚{\tiny $_{lb}$}‚वात् क‚थ‚न्त‚द‚नुमान‚म‚पीत्याह । योग्य‚तायास्त्वित्यादि । द्र‚व्यान्त‚रान‚पेक्ष‚त्वात्‚{\tiny $_{lb}$}‚ स‚न्निहि‚{\tiny $_{४}$}‚त‚कार‚ण‚क‚लाप‚व्य‚तिरेकेण कार‚णान्त‚रान‚पेक्ष‚त्वान्न विरुध्य‚तेऽनुमानं ।
	{\color{gray}{\rmlatinfont\textsuperscript{§~\theparCount}}}
	\pend% ending standard par
      ‚{\tiny $_{lb}$}‚

	  
	  \pstart \leavevmode% starting standard par
	त‚देवानुमान‚माह । उत्त‚रोत्त‚रेत्यादि । पूर्व‚पूर्व‚क्ष‚ण‚मुपादायोत्त‚र‚स‚म‚र्थ‚क्ष‚णोत्पाद‚{\tiny $_{lb}$}‚ उत्त‚रोत्त‚र‚श‚क्तिप‚रिणामः । तेन हेतुभूतेन कार्योत्पाद‚न‚स‚म‚र्थेति साध्य‚निर्देशः ।‚{\tiny $_{lb}$}‚ इयं कार‚ण‚साम‚ग्रीति ध‚र्मी । \textbf{श‚क्तिप‚रिणाम‚प्र‚त्य‚य‚स्यान्य‚स्यापेक्ष‚णीय‚स्याभावा}‚{\tiny $_{lb}$}‚दिति हेतुः‚{\tiny $_{५}$}‚ श‚क्तेः प‚रिणाम‚स्य योऽप‚रः स‚ह‚कारिप्र‚त्य‚य‚स्त‚स्यापेक्ष‚णीय‚स्याभावात् ।
	{\color{gray}{\rmlatinfont\textsuperscript{§~\theparCount}}}
	\pend% ending standard par
      ‚{\tiny $_{lb}$}‚‚{\tiny $_{lb}$}‚\textsuperscript{\textenglish{45/s}}

	  
	  \pstart \leavevmode% starting standard par
	कुत‚स्त‚र्हि श‚क्तेः प्र‚स‚व इत्याह । पूर्वेत्यादि । पूर्व‚स‚जातिः स‚दृशः पूर्वः‚{\tiny $_{lb}$}‚ कार‚ण‚क‚लाप‚स्ताव‚न्मात्रं हेतुर्य‚स्याः श‚क्तिप्र‚सूतेः सा त‚था । त‚द्भाव‚स्त‚स्मात् ।‚{\tiny $_{lb}$}‚ अतः कार‚णात् सा योग्य‚तान‚न्यापेक्षिणीत्युच्य‚ते ॥
	{\color{gray}{\rmlatinfont\textsuperscript{§~\theparCount}}}
	\pend% ending standard par
      ‚{\tiny $_{lb}$}‚

	  
	  \pstart \leavevmode% starting standard par
	न‚नु कार्यं प्र‚ति कार‚ण‚स्य योग्य‚ता य‚दि श‚क्तिरुच्य‚ते त‚दा का‚{\tiny $_{६}$}‚र्य‚व्य‚भिचारे‚{\tiny $_{lb}$}‚ योग्य‚ताया अपि व्य‚भिचार इति क‚थ‚मेत‚द‚नुमानं । अथ योग्य‚तास‚म्भ‚व उच्य‚ते‚{\tiny $_{lb}$}‚ त‚दाय‚म‚र्थः स्यात् कार्यं स्याद्वा न वेति । त‚थापि क‚थ‚म‚स्यानुमानं स‚न्देहादिति ।
	{\color{gray}{\rmlatinfont\textsuperscript{§~\theparCount}}}
	\pend% ending standard par
      ‚{\tiny $_{lb}$}‚

	  
	  \pstart \leavevmode% starting standard par
	अत्रोच्य‚ते । प‚रेण हि क‚थ‚मेत‚द‚नुमान‚न्त्रिविध‚हेतुज‚न्य‚मिति चोद्य‚ते । य‚द्ये‚{\tiny $_{lb}$}‚त‚द‚नुमान‚म्प‚रेण स‚म‚र्थ्य‚ते त‚दा त्रिविध‚लिङ्ग‚ज‚मेवेत्या चा र्ये ण प्र‚तिपाद्य‚ते । न त्वेत‚{\tiny $_{lb}$}‚त्प‚र‚मा‚{\tiny $_{७}$}‚र्थ‚तानुमान‚मित्येव‚म्प‚र‚मेत‚दित्येके । अथ‚वा य‚द्येकान्तेन कार्योत्पाद‚न- \leavevmode\ledsidenote{\textenglish{18b/PSVTa}}‚{\tiny $_{lb}$}‚ योग्य‚तानुमीय‚ते । त‚दा व्य‚भिचाराद‚नुमानं न स्यात् । य‚दा तु क‚दाचित् कार्यं स्यादि‚{\tiny $_{lb}$}‚त्येवंरूपः स‚म्भ‚वोनुमीय‚ते त‚दा क‚थ‚म‚स्य व्य‚भिचारः । तेनाय‚म‚र्थ उत्त‚रोत्त‚र‚प‚रि‚{\tiny $_{lb}$}‚णामे य‚दि प्र‚ब‚न्धाभाव‚स्त‚दा कार्यं स्याद‚न्य‚दा तु नास्तीति । प‚रोक्ते त्वेकान्तेन‚{\tiny $_{lb}$}‚कार्यानुमाने व्य‚भिचार एव ।
	{\color{gray}{\rmlatinfont\textsuperscript{§~\theparCount}}}
	\pend% ending standard par
      ‚{\tiny $_{lb}$}‚

	  
	  \pstart \leavevmode% starting standard par
	अन्ये तु प‚रि‚{\tiny $_{१}$}‚णाम‚व‚त्यां साम‚ग्र्यां प्र‚तिब‚न्ध‚काभावे स‚त्येकान्तेन कार्योत्पाद‚{\tiny $_{lb}$}‚न‚योग्य‚ता भ‚व‚तीति सैवानुमीय‚ते । कार्योत्पाद‚न‚योग्य‚ताप्र‚तीतिश्च कार्य‚म‚पि विशेष‚{\tiny $_{lb}$}‚ण‚त्वेनाक्षिप‚तीति न पृथ‚क् कार्यानुमानं क्रिय‚त इति म‚न्य‚न्ते । केव‚लं साम‚ग्रीमात्रात्‚{\tiny $_{lb}$}‚ कार्यानुमाने व्य‚भिचार उच्य‚ते प्र‚तिब‚न्ध‚काभावः क‚थ‚म्प्र‚तिप‚न्न इति चेत् [।]‚{\tiny $_{lb}$}‚ स‚त्यं । यो हि तं ज्ञातुं श‚क्नोति त‚स्यैत‚द‚नु‚{\tiny $_{२}$}‚मानं यो हि धूम‚स्याग्निज‚न्य‚त्वं ज्ञातुं‚{\tiny $_{lb}$}‚ श‚क्नोति त‚स्य धूमाद‚ग्न्य‚नुमानं नान्य‚स्य त‚द्व‚त् ॥
	{\color{gray}{\rmlatinfont\textsuperscript{§~\theparCount}}}
	\pend% ending standard par
      ‚{\tiny $_{lb}$}‚

	  
	  \pstart \leavevmode% starting standard par
	\textbf{या त‚र्ही}त्यादिना पुन‚र‚पि त्रिधैव स इत्य‚स्य व्याघात‚माह । \textbf{अकार्य‚कार‚ण‚{\tiny $_{lb}$}‚भूतेन} । अनुमेयाद‚र्था\textbf{द‚न्येना}स्व‚भावेन \textbf{र‚सादिना} । आदिश‚ब्दाद् ग‚न्धादि\textbf{ना‚{\tiny $_{lb}$}‚ रूपादिग‚तिः} । अत्राप्यादिश‚ब्दात् स्प‚र्शादिग्र‚ह‚णं । अन्ध‚कारे हि मातुलुङ्गादिर‚{\tiny $_{lb}$}‚स‚मास्वाद्य । च‚म्प‚क‚ग‚न्ध‚माघ्राय । व‚{\tiny $_{३}$}‚ह्नेश्च स्प‚र्श‚म‚नुभूय । तेषां रूप‚सामान्य‚{\tiny $_{lb}$}‚म‚नुमीय‚ते । त‚था व‚ह्निरूपं दृष्ट्वा त‚त्स्प‚र्शः । सा क‚थ‚न्त्रिविधे हेताव‚न्त‚र्भ‚व‚तीति‚{\tiny $_{lb}$}‚ प्र‚कृते ।
	{\color{gray}{\rmlatinfont\textsuperscript{§~\theparCount}}}
	\pend% ending standard par
      ‚{\tiny $_{lb}$}‚‚{\tiny $_{lb}$}‚\textsuperscript{\textenglish{46/s}}

	  
	  \pstart \leavevmode% starting standard par
	न चाप्र‚माण‚मिय‚न्त‚तो लिङ्गान्त‚र‚प्र‚संग इत्याह । \textbf{सापी}त्यादि । सापि ग‚ति‚{\tiny $_{lb}$}‚रिति स‚म्ब‚न्धः । \textbf{रूपादेः} किम्विशिष्ट\textbf{स्यैक‚साम‚ग्र्य‚धीन‚स्}य । र‚स‚स्य ज‚निका या‚{\tiny $_{lb}$}‚ साम‚ग्री त‚स्यामेव साम‚ग्र्यामाय‚त्त‚स्य \textbf{र‚स‚तो} लिङ्गाद्या \textbf{ग‚तिः} [।] सा हे‚{\tiny $_{४}$}‚तुध‚र्मानु‚{\tiny $_{lb}$}‚मानेन । र‚स‚स्य यो \textbf{हेतुः} पूर्व‚म‚पादान‚न्त‚स्य यो \textbf{ध‚र्मों} रूप‚ज‚न‚क‚त्व‚न्त‚स्या\textbf{नुमानेन} ।‚{\tiny $_{lb}$}‚ तेनाय‚म‚र्थो र‚सात् स‚क‚शात् त‚द्धेतोर‚स‚स‚मान‚काल‚भावि रूप‚ज‚न‚क‚त्व‚न्निश्चीय‚ते ।‚{\tiny $_{lb}$}‚ एवं हि त‚स्य र‚स‚स‚मान‚काल‚भावि रूप‚ज‚न‚क‚त्वं निश्चीय‚ते । य‚दि स‚मान‚काल‚{\tiny $_{lb}$}‚भाविनो रूप‚स्यापि निश्च‚यः स्यात्तेनातीतैक‚कालानामेकैव ग‚तिः कार्य‚लिङ्ग‚जा‚{\tiny $_{५}$}‚ ।
	{\color{gray}{\rmlatinfont\textsuperscript{§~\theparCount}}}
	\pend% ending standard par
      ‚{\tiny $_{lb}$}‚

	  
	  \pstart \leavevmode% starting standard par
	न‚न्व‚न‚भिहिताद् र‚स‚हेतोः स‚काशात् प‚श्चात्स‚मान‚काल‚स्य रुप‚स्यानुमानं हेतोः‚{\tiny $_{lb}$}‚ कार्यानुमाने व्य‚भिचारात् । कार्योत्पाद‚न‚योग्य‚तानुमाने च न र‚स‚स‚मान‚काल‚स्य‚{\tiny $_{lb}$}‚ रूप‚स्यानुमानं स्यात् । अनुमितानुमान‚प्र‚तीतेर‚भावाच्च । तेन य‚दुच्य‚ते [।]‚{\tiny $_{lb}$}‚ य‚दातीतानां ग‚तिस्त‚दा कार्य‚श्च त‚ल्लिङ्ग‚श्च [।] त‚स्माज्जातेति कार्य‚लिङ्ग‚जा [।]‚{\tiny $_{lb}$}‚ य‚दा तु स‚मान\textbf{कालानाङ्ग}तिस्त‚दा का‚{\tiny $_{६}$}‚र्यं लिङ्गं य‚स्य हेतुध‚र्मानुमान‚स्य त‚त्कार्य‚{\tiny $_{lb}$}‚लिङ्ग‚न्त‚स्माज्जाता कार्य‚लिङ्ग‚जेति त‚द‚पास्तं । {त‚त्र प‚रे मा...} चाव‚य‚विद्र‚व्ये‚{\tiny $_{lb}$}‚ रूपाद‚यो गुणा व्य‚व‚स्थितास्तेन त‚त्र रूपादेर‚स‚तो ग‚तिर्युक्ता । न \textbf{बौ द्धा नां} रूपा‚{\tiny $_{lb}$}‚दिव्य‚तिरेकेणाव‚य‚विनोऽन‚भ्युप‚ग‚मात् न ध‚र्मिणोऽभावात् । ध‚र्मिण‚म‚न्त‚रेण चानुमाने‚{\tiny $_{lb}$}‚ \leavevmode\ledsidenote{\textenglish{19a/PSVTa}} र‚स‚तो रूपादेः स‚र्व‚त्रानुमान‚प्र‚स‚ङ्गात् ।‚{\tiny $_{७}$}‚ अस‚मुदाय‚श्च साध्यः स्यात् । अप‚क्ष‚ध‚र्म‚श्च‚{\tiny $_{lb}$}‚ हेतुः स्यात् । नापि हेतुध‚र्मानुमानं युज्य‚ते रूपादिकार्य‚त्वेन र‚सादेर‚प्र‚तिप‚न्न‚त्वात् ।‚{\tiny $_{lb}$}‚ न च क्ष‚णिक‚प‚क्ष‚म‚निश्चित्यैव‚मुच्य‚ते । क्ष‚णिक‚त्वे हि रूपादेः स‚मान‚जातीय‚कार्य‚त्वं‚{\tiny $_{lb}$}‚ स्यान्न र‚सादिकार्य‚त्वं प्र‚तिब‚न्ध‚ग्राह‚काभावात् । न क्ष‚णानाम‚नुमानानुमेय‚व्य‚व‚हारः‚{\tiny $_{lb}$}‚ स‚म्ब‚न्धानिश्च‚यात् । स‚न्तानाश्र‚येण त्व‚नुमानादिव्य‚व‚हारे रूपादिस‚न्ता‚{\tiny $_{१}$}‚नानां न‚{\tiny $_{lb}$}‚ प‚र‚स्प‚रं कार्य‚कार‚ण‚भावः प्र‚तिप‚न्न इति क‚थ‚म‚नुमानानुमेय‚व्य‚व‚हार इति ।
	{\color{gray}{\rmlatinfont\textsuperscript{§~\theparCount}}}
	\pend% ending standard par
      ‚{\tiny $_{lb}$}‚

	  
	  \pstart \leavevmode% starting standard par
	एव‚म्म‚न्य‚ते [।] न स‚र्व‚त्र‚र‚सादे रूपाद्य‚नुमान‚म‚पि त्वाम्रादौ ध‚र्मिणि । त‚त्र च‚{\tiny $_{lb}$}‚ रूपादीनां प‚र‚स्प‚राविनिर्भाग‚र्भाग \edtext{}{\lemma{र्भाग}\Bfootnote{?}} निय‚मः प्र‚तीय‚त एव [।] स च प्र‚तिब‚न्ध‚{\tiny $_{lb}$}‚हेतुकोऽन्य‚था घ‚ट‚प‚टादीनाम‚पि प‚र‚स्प‚राविनिर्भाग‚निय‚मः स्यात् । प्र‚तिब‚न्ध‚श्च‚{\tiny $_{lb}$}‚ तेषां न तादात्म्यं‚{\tiny $_{२}$}‚ भेदेन प्र‚तीतेर्[।]नापि त‚दुत्प‚त्तिः स‚मान‚काल‚त्वात् । न चैकार्थ‚{\tiny $_{lb}$}‚स‚म‚वाय‚स्तेषां य‚तो न ताव‚द‚स‚म‚वेतानां स‚म‚वाय‚ब‚लादेकार्थ‚स‚म‚वायः स‚म‚वाय‚स्यैवा‚{\tiny $_{lb}$}‚भावात् । अतिप्र‚स‚ङ्गाच्च । स‚म‚वेतानाम‚पि किं स‚म‚वायेन स्व‚हेतुभ्य एव त‚था‚{\tiny $_{lb}$}‚ निष्प‚त्तेः । त‚था निष्प‚त्त्यैव वाच्याभिधानाद् ग‚म‚क‚त्व‚म्[।]त‚स्मादेक‚साम‚ग्र्य‚धीन‚त्वं‚{\tiny $_{lb}$}‚ प्र‚तिब‚न्धः । निश्चित‚प्र‚तिब‚न्ध‚स्य चान्य‚{\tiny $_{३}$}‚स्मिन् कालेनुमानं । य‚द्य‚पि चात्राव‚य‚वी‚{\tiny $_{lb}$}‚ म विद्य‚ते त‚थाप्याम्रादिप्र‚त्य‚य‚विष‚य‚स्य ध‚र्मित्व‚न्तेन त‚त्र र‚स‚तो रूपादिग‚तिः ।
	{\color{gray}{\rmlatinfont\textsuperscript{§~\theparCount}}}
	\pend% ending standard par
      ‚{\tiny $_{lb}$}‚

	  
	  \pstart \leavevmode% starting standard par
	न‚नु त‚थापि क‚थ‚मेषामेक‚साम‚ग्र्य‚धीन‚त्वं । य‚तो \textbf{यै}\edtext{}{\lemma{तो}\Bfootnote{? येनै}}व स्व‚भावेन रूपं‚{\tiny $_{lb}$}‚ ‚{\tiny $_{lb}$}‚ \leavevmode\ledsidenote{\textenglish{47/s}}रूपं ज‚न‚य‚ति न तेनैव र‚सादिकं ज‚न‚य‚ति तेषां प‚र‚स्प‚राभेद‚प्र‚संगात् । नाप्य‚न्येनान्यं‚{\tiny $_{lb}$}‚ ज‚न‚य‚ति त‚स्य स्व‚भाव‚भेद‚प्र‚संगात् ।
	{\color{gray}{\rmlatinfont\textsuperscript{§~\theparCount}}}
	\pend% ending standard par
      ‚{\tiny $_{lb}$}‚

	  
	  \pstart \leavevmode% starting standard par
	नैष दोषो य‚स्मात् । न त‚त्र रूप‚स्य पूर्व्व‚मेक‚{\tiny $_{४}$}‚कार्य‚ज‚न‚क‚त्वं येनाय‚न्दोषः‚{\tiny $_{lb}$}‚ स्यात् । किन्त्वेक‚काल‚म‚नेक‚कार्य‚ज‚न‚क‚त्व‚मेव । त‚त‚स्त‚स्यैवोत्प‚त्तिद‚र्श‚नात् ।‚{\tiny $_{lb}$}‚ न च कार‚ण‚स्य कार्याभाव एव कार‚ण‚त्वं येनानेक‚मेक‚स्मादुत्प‚द्य‚मान‚मेकं प्र‚स‚ज्येत‚{\tiny $_{lb}$}‚ किन्तु कार्योत्प‚त्तौ प्राग्भाव एव त‚स्य कार‚ण‚त्वं लोके । य‚था चैक‚कार्योत्प‚त्तौ त‚स्य‚{\tiny $_{lb}$}‚ प्राग्भाव‚कार‚ण‚त्व‚न्त‚था क‚लापोत्प‚त्ताव‚पि‚{\tiny $_{५}$}‚ दृष्ट‚त्वात् । य‚द्वा येनैव स्व‚भावेन रूपं‚{\tiny $_{lb}$}‚ रूपं ज‚न‚य‚ति तेनैव र‚सादिक‚म‚पि स्व‚रूप‚भेद‚स्त्वेषामुपादान‚भेद‚कृतो न स‚ह‚कारिकृतः ।‚{\tiny $_{lb}$}‚ त‚था हि वायौ स्प‚र्श‚स‚द्भावेपि रूपाद्य‚भावाद् रूपाद्य‚नुत्प‚त्तिः । अग्नौ च रूपादि‚{\tiny $_{lb}$}‚स‚द्भावेपि र‚साभावाद् र‚सानुत्प‚त्तिः । अप्सु र‚सादिभावेपि ग‚न्धाभावाद् ग‚न्धानु‚{\tiny $_{lb}$}‚त्प‚त्तिस्तेन र‚सादेर्निय‚तं का‚{\tiny $_{६}$}‚र‚णं र‚सादिरेवाव‚सीय‚ते । निय‚तं च कार‚ण‚मुपादान‚{\tiny $_{lb}$}‚कार‚णं स‚मान‚जातीय‚म‚भिन्न‚स‚न्तान‚व‚र्त्ति वा । त‚स्मादुपादान‚कार‚ण‚भेदाद् रूपा‚{\tiny $_{lb}$}‚दीनां स्व‚भाव‚भेदः ।
	{\color{gray}{\rmlatinfont\textsuperscript{§~\theparCount}}}
	\pend% ending standard par
      ‚{\tiny $_{lb}$}‚

	  
	  \pstart \leavevmode% starting standard par
	य‚त्पुन‚रुच्य‚ते । त‚त्रानेक‚श‚क्तीनां स‚मुच्च‚यः । तेन रूप‚मेक‚या श‚क्त्या रूपं‚{\tiny $_{lb}$}‚ ज‚न‚य‚त्य‚न्यान्य‚या र‚सादिक‚मिति ।
	{\color{gray}{\rmlatinfont\textsuperscript{§~\theparCount}}}
	\pend% ending standard par
      ‚{\tiny $_{lb}$}‚

	  
	  \pstart \leavevmode% starting standard par
	त‚द‚युक्त‚म् [।] अनेक‚श‚क्तिद्वारेणाप्ये‚{\tiny $_{७}$}‚क‚स्यानेक‚कार्य‚कार‚णाभ्युप‚ग‚मेऽनेक‚त्व- \leavevmode\ledsidenote{\textenglish{19b/PSVTa}}‚{\tiny $_{lb}$}‚ प्र‚स‚ङ्गोऽनिवारित एव भाव‚स्यानेक‚श‚क्तीनामेवानेक‚स्व‚भाव‚त्वात् । अभिन्न‚त्वाच्च‚{\tiny $_{lb}$}‚ श‚क्तीनां श‚क्तिभेदे रूप‚स्य भेद‚प्र‚स‚ङ्गः । भेदे वा श‚क्तीनां रूप‚स्याकार‚क‚त्व‚{\tiny $_{lb}$}‚प्र‚स‚ङ्गात् । न च श‚क्तियोगात् कार‚क‚त्व‚म‚श‚क्त‚स्य श‚क्तियोगाभावात् । श‚क्त‚स्यापि‚{\tiny $_{lb}$}‚ किं श‚क्तियोगेन स्व‚रूपेणै‚{\tiny $_{१}$}‚व कार‚क‚त्वाच्छ‚क्तेश्च कार‚क‚त्वं न स्याच्छ‚क्तियोगा‚{\tiny $_{lb}$}‚भावात् । अथ श‚क्तित्वान्न सा श‚क्तिम‚पेक्ष‚ते [।] भावोपि त‚र्हि शी क्त‚त्वात्‚{\tiny $_{lb}$}‚ किमिति श‚क्तिम‚पेक्ष‚ते ।
	{\color{gray}{\rmlatinfont\textsuperscript{§~\theparCount}}}
	\pend% ending standard par
      ‚{\tiny $_{lb}$}‚

	  
	  \pstart \leavevmode% starting standard par
	योपि मी मां स को म‚न्य‚ते । भाव‚स्य स्व‚रूपातिश‚य एव \textbf{श क्तिः} सा च भिन्ना‚{\tiny $_{lb}$}‚भिन्ना । य‚तो भावे गृह्य‚माणे श‚क्तिर्न गृह्य‚तेऽतो भावाद् भिन्ना । कार्यान्य‚थानु‚{\tiny $_{lb}$}‚प‚प‚त्त्या तु सा भाव‚स्याभिन्नाऽन्य‚था भाव‚{\tiny $_{२}$}‚स्य कार‚क‚त्व‚न्न स्यात् । त‚दुक्तं [।]
	{\color{gray}{\rmlatinfont\textsuperscript{§~\theparCount}}}
	\pend% ending standard par
      ‚{\tiny $_{lb}$}‚
	  \bigskip
	  \begingroup
	
	    
	    \stanza[\smallbreak]
	  {\normalfontlatin\large ``\qquad}श‚क्त‚यः स‚र्व‚भावानां कार्यार्थाप‚त्तिक‚ल्पिता इति [।]{\normalfontlatin\large\qquad{}"}\&[\smallbreak]
	  
	  
	  
	  \endgroup
	‚{\tiny $_{lb}$}‚

	  
	  \pstart \leavevmode% starting standard par
	सोपि निर‚स्तः । एक‚स्याः श‚क्तेर्भिन्नाभिन्न‚रूप‚त्व‚विरोधात् । किं चार्था‚{\tiny $_{lb}$}‚प‚त्त्या कार्यात् प्राग्भाविन एव भाव‚स्याभिन्ना श‚क्तिः क‚ल्प्य‚तां इति प्राग्भाव‚{\tiny $_{lb}$}‚ एव श‚क्तिः [।] स च प्र‚त्य‚क्ष‚सिद्ध इति क‚थं न श‚क्तिः प्र‚त्य‚क्षा । केव‚लं सा कार्य‚{\tiny $_{lb}$}‚द‚र्श‚नान्निश्चीय‚ते । त‚स्मात् कार्याद् र‚सादेः कार‚ण‚{\tiny $_{३}$}‚ध‚र्मानुमानाद् अस्य कार्य‚हेता‚{\tiny $_{lb}$}‚व‚न्त\textbf{र्भावः । धूमेर‚ध‚न‚विकार‚व‚दि}ति । य‚था धूमाद‚ग्न्यादिसाम‚ग्र्य‚नुमितौ भ‚स्मा‚{\tiny $_{lb}$}‚‚{\tiny $_{lb}$}‚ \leavevmode\ledsidenote{\textenglish{48/s}}ङ्गारादीन्ध‚न‚विकारानुमितिः त‚द्व‚त् ॥
	{\color{gray}{\rmlatinfont\textsuperscript{§~\theparCount}}}
	\pend% ending standard par
      ‚{\tiny $_{lb}$}‚

	  
	  \pstart \leavevmode% starting standard par
	न‚नु च र‚स‚कार्येणानुमितात् कार‚णात् स‚काशात् स‚मान‚कालिनः कार्य‚स्यानु‚{\tiny $_{lb}$}‚मान‚मिदं न तु हेतुध‚र्मानुमान‚मित्य‚त आह । \textbf{त‚त्रे}ति [।] र‚साद् रूपे प्र‚तिप‚त्तौ \textbf{हेतुरेव‚{\tiny $_{lb}$}‚ त‚थाभूत} इति रूप‚ज‚न‚न‚स‚म्ब‚न्धो\textbf{नुमी‚{\tiny $_{४}$}‚य‚ते} [।] य‚स्मिन्न‚नुमीय‚माने कार्यान्त‚र‚म‚पि‚{\tiny $_{lb}$}‚ त‚द्विशेष‚ण‚न्निश्चित‚म्भ‚व‚ति । न त्व‚नुमितात् कार‚णात् प‚श्चात् कार्यान्त‚र‚म‚नुमेयं ।‚{\tiny $_{lb}$}‚ कुत इत्याह । \textbf{हि} य‚स्मात् । प्र‚वृत्ता न प्र‚तिब‚द्धा श‚क्तिर्य‚स्य त‚त् \textbf{प्र‚वृत्त‚श‚क्ति} ।‚{\tiny $_{lb}$}‚ त‚थाभूतं च \textbf{त‚द्रूपोपादान‚कार‚णं} चेति त‚थोक्तं । त‚स्य \textbf{स‚ह‚कारिप्र‚त्य‚यः} स‚न् । र‚स‚हेतू \textbf{र‚सं‚{\tiny $_{lb}$}‚ ज‚न‚य‚ति} । त‚थाभूते च हेताव‚नुमितेर्थाद् रूपानुमानं रू‚{\tiny $_{५}$}‚प‚र‚स‚योरेक‚साम‚ग्र्य‚धीन‚त्वात् ।‚{\tiny $_{lb}$}‚ \textbf{इन्ध‚न‚विकार‚विशेषो} भ‚स्माङ्गारादिः । त‚स्यो\textbf{पादानं} काष्ठं त‚स्य \textbf{स‚ह‚कारिप्र‚त्य‚यो}ग्निः ।‚{\tiny $_{lb}$}‚ त‚स्य य‚था \textbf{धूम‚ज‚न‚न‚न्तेन तुल्य‚न्त‚द्व‚द्} रूप‚र‚स‚योरेक‚साम‚ग्र्य‚धीन‚त्वं ।
	{\color{gray}{\rmlatinfont\textsuperscript{§~\theparCount}}}
	\pend% ending standard par
      ‚{\tiny $_{lb}$}‚

	  
	  \pstart \leavevmode% starting standard par
	\textbf{त‚था ही}त्यादिना स‚म‚र्थ‚य‚ते \textbf{श‚क्तिप्र‚वृत्ते}ति सूत्रं ।\edtext{\textsuperscript{*}}{\edlabel{pvsvt_48-1}\label{pvsvt_48-1}\lemma{*}\Bfootnote{\href{http://sarit.indology.info/?cref=ps}{Pramāṇa-Samuccaya. }}} अस्य व्याख्यानं \textbf{स्व‚कार‚ण‚{\tiny $_{lb}$}‚स्ये}त्यादि । \textbf{स्व‚कार‚ण‚स्येति} र‚स‚स्य य‚त् स्व‚कार‚ण‚न्त‚स्य \textbf{फ‚लोत्पाद‚नं} प्र‚तीतिर‚{\tiny $_{६}$}‚‚{\tiny $_{lb}$}‚सोत्पाद‚नं प्र‚त्या\textbf{भिमुख्येना}नुगुण्येन \textbf{विना न र‚स उत्प‚द्य‚ते । सैवे}ति श‚क्तिप्र‚वृत्तः ।‚{\tiny $_{lb}$}‚ \textbf{अन्य‚कार‚ण}मित्य‚स्य व्याख्यानं \textbf{रूपोपादाने}त्यादि । \textbf{रूप}स्य य \textbf{उपादान‚हेत‚वः} पूर्व‚{\tiny $_{lb}$}‚ल‚क्ष‚ण‚स‚ङ्गृहीता रूप‚प‚र‚माण‚व\textbf{स्तेषां} रूप‚ज‚न‚न‚म्प्र‚त्याभिमुख्यं \textbf{प्र‚वृत्ति}स्त‚स्याः \textbf{कार‚णं} ।‚{\tiny $_{lb}$}‚ \leavevmode\ledsidenote{\textenglish{20b/PSVTa}} र‚स‚हेतुनैव स‚ह‚कारिणा रूपोपादान‚स्य स्व‚कार्यं प्र‚त्याभिमुख्यात् । सापि \textbf{र‚{\tiny $_{७}$}‚सोपा‚{\tiny $_{lb}$}‚दान‚कार‚ण‚प्र‚वृत्ती रूपोपादान‚कार‚ण}स्य रूप‚ज‚न‚न‚म्प्र‚ति या \textbf{प्र‚वृत्तिस्त‚या} स‚ह क‚र्त्तु‚{\tiny $_{lb}$}‚ शीलं य‚स्या \textbf{र‚सोपादान‚कार‚ण‚प्र‚वृत्तेः} सा त‚थोक्ता । य‚त एव\textbf{न्त‚स्माद् य‚था‚{\tiny $_{lb}$}‚भूताद्धेतो}रिति प्र‚वृत्त‚श‚क्तिरूपोपादान‚कार‚ण‚स‚हितात् पूर्व‚क्ष‚ण‚संगृहीताद् र‚सा‚{\tiny $_{lb}$}‚‚{\tiny $_{lb}$}‚ ‚{\tiny $_{lb}$}‚ \leavevmode\ledsidenote{\textenglish{49/s}}ल्लिङ्ग‚त्वेनाभिम‚तो र‚स \textbf{उत्न्न‚स्त‚थाभूत}मेव हेतुम\textbf{नुमाप‚य‚न्} ग‚म‚य‚न् स‚मान‚का‚{\tiny $_{१}$}‚लं‚{\tiny $_{lb}$}‚ \textbf{रूपं} ग‚म‚य‚ति । तेनाय‚म‚र्थो हेतुनिश्च‚य एवेदृशो नान्यः ।
	{\color{gray}{\rmlatinfont\textsuperscript{§~\theparCount}}}
	\pend% ending standard par
      ‚{\tiny $_{lb}$}‚

	  
	  \pstart \leavevmode% starting standard par
	न पुन‚र‚नुमितात् कार‚णात् प‚श्चात् कार्यानुमानं । य‚त‚श्च कार्य‚ज‚न‚क‚त्वेनैव‚{\tiny $_{lb}$}‚ हेतुध‚र्म‚निश्च‚यो नान्य‚था तेनैवाह । \textbf{इति त‚त्रापी}त्यादि । इति अनेन द्वारेण । त‚त्रापि‚{\tiny $_{lb}$}‚ र‚सादे रूपाद्य‚नुमानेऽ\textbf{तीतानामेक‚कालानां च ग‚तिः} । र‚सोपादान‚स‚मान‚काल‚{\tiny $_{lb}$}‚भाविनोऽतीताः । लिङ्ग‚भूत‚र‚स‚स‚ह‚भा‚{\tiny $_{२}$}‚विन एक‚कालास्तेषाङ्ग‚तिः । \textbf{नाऽनाग‚{\tiny $_{lb}$}‚ताना}म्व‚र्त्त‚मानेन लिंगेनानुमानं \textbf{व्य‚भिचारा}त् । अनाग‚तं हि कार‚णान्त‚र‚प्र‚तिब‚द्ध‚{\tiny $_{lb}$}‚न्त‚त्र प्र‚तिब‚न्ध‚वैक‚ल्य‚स‚म्भ‚वान्न भ‚वेद‚पि । य‚च्चाद्योद‚यात् श्वः सूर्योद‚याद्य‚नुमान‚न्न‚{\tiny $_{lb}$}‚ त‚द‚नुमानं नियाम‚क‚लिङ्गाभावात् । अद्य ग‚र्द‚भ‚द‚र्श‚नात् श्वः सूर्योद‚यानुमान‚व‚त् ।‚{\tiny $_{lb}$}‚ \textbf{त‚स्मादिय‚म‚पि} र‚सादे रूपादिग‚तिः \textbf{कार्य‚{\tiny $_{३}$}‚लिङ्ग‚जे}त्य‚नुमानं । य‚त‚श्च साध्याय‚त्त‚{\tiny $_{lb}$}‚त्वेन हेतुर्ग‚म‚क\textbf{स्तेन} कार‚णेन त्रिविधाद्धेतो\textbf{र्नान्यो} हेतुः संयोग्यादि\textbf{र्ग‚म‚को}स्ति । क‚स्मा‚{\tiny $_{lb}$}‚त्तादात्म्य‚त‚दुत्प‚त्तिभ्यां लिङ्गिन्य\textbf{प्र‚तिब‚द्ध‚स्व‚भाव‚स्याविनाभाव‚निय‚माभावात्} ॥
	{\color{gray}{\rmlatinfont\textsuperscript{§~\theparCount}}}
	\pend% ending standard par
      ‚{\tiny $_{lb}$}‚

	  
	  \pstart \leavevmode% starting standard par
	एत‚दुक्त‚म्भ‚व‚ति । न ताव‚द‚श्लिष्टानां संयोगोस्ति । श्लिष्टानाम‚पि किं संयोगेन‚{\tiny $_{lb}$}‚ स्व‚हेतुभ्यः एव श्लिष्टानामुत्प‚त्तेः । त‚स्मान्न संयोग‚व‚शाद् ग‚{\tiny $_{४}$}‚म‚क‚त्व‚म‚प्र‚तिब‚द्ध‚{\tiny $_{lb}$}‚त्वात् । घ‚ट‚घ‚ट‚योरिव । त‚था पृथ‚क्सिद्धानान्न स‚म‚वायः । अपृथ‚क्सिद्धानाम‚पि‚{\tiny $_{lb}$}‚ किं स‚म‚वायेन स्व‚हेतुभ्य एवोप‚र्युप‚रिभावेन निष्प‚त्तेस्तेन न स‚म‚वायेनापि ग‚म‚क‚त्वं‚{\tiny $_{lb}$}‚ [।] साध्याय‚त्त‚त्वाभावान्न च स‚म‚वायोस्तीति व‚क्ष्य‚ति । अत एवैकार्थ‚स‚म‚वायि‚{\tiny $_{lb}$}‚नोर‚भाव इति पूर्व‚मेवोक्तं र‚साद्य‚नुमाने । विरोधी चानुप‚ल‚ब्धाव‚न्त‚र्भावि‚{\tiny $_{५}$}‚तः ।
	{\color{gray}{\rmlatinfont\textsuperscript{§~\theparCount}}}
	\pend% ending standard par
      ‚{\tiny $_{lb}$}‚

	  
	  \pstart \leavevmode% starting standard par
	य‚च्च नै या यि कोक्तं \textbf{पूर्व‚व‚च्छेष‚व‚त्सामान्य‚तो दृष्टं चानुमानं} \href{http://sarit.indology.info/?cref=ns\%C5\%AB.1.1.5}{न्यायसूत्रं १।१।५} । त‚त्र पूर्व‚व‚त् कार‚णात् कार्यानुमानं । त‚च्च व्य‚भिचारीति प्र‚ति‚{\tiny $_{lb}$}‚पादितं । शेष‚व‚द‚नुमानं च कार्यात् कार‚णानुमानं त‚दिष्ट‚मेव शास्त्र‚कृता । सामा‚{\tiny $_{lb}$}‚न्य‚तो दृष्ट‚न्त्व‚नुमानं य‚द‚न्य‚त्र ध‚र्मिणि साध्य‚साध‚न‚योर्व्याप्तिन्दृष्ट्वान्य‚त्रानुमानं ।‚{\tiny $_{lb}$}‚ य‚था देव‚द‚त्त‚स्य देशान्त‚र‚प्राप्तिं ग‚तिपूर्विकां दृष्ट्वा‚{\tiny $_{६}$}‚ऽदित्य‚स्यापि देशान्त‚र‚प्राप्त्या‚{\tiny $_{lb}$}‚ ग‚त्य‚नुमानं । \textbf{एत‚द‚पि कार्य‚लिङ्ग‚ज‚मेव} । देशान्त‚र‚प्राप्तेर्ग‚तिकार्य‚त्वात् । स‚र्व‚मेवा‚{\tiny $_{lb}$}‚नुमानं सामान्य‚तो दृष्ट‚मेवेति स्व‚य‚मेवा चा र्यंस्तृतीये प‚रिच्छेदे व‚क्ष्य‚ति ।
	{\color{gray}{\rmlatinfont\textsuperscript{§~\theparCount}}}
	\pend% ending standard par
      ‚{\tiny $_{lb}$}‚

	  
	  \pstart \leavevmode% starting standard par
	य‚द‚प्यु द्यो त क रे ण\edtext{}{\edlabel{pvsvt_49-1}\label{pvsvt_49-1}\lemma{ण}\Bfootnote{\href{http://sarit.indology.info/?cref=nv.1.5}{ Nyāyavārtika 1:5. }}} सामान्य‚तो दृष्ट‚मुदाहृतं [।] य‚था व‚लाकात‚स्तो‚{\tiny $_{lb}$}‚‚{\tiny $_{lb}$}‚ ‚{\tiny $_{lb}$}‚ \leavevmode\ledsidenote{\textenglish{50/s}}यानुमान‚न्त‚द‚पि कार्य‚लिङ्ग‚ज‚मेवेति प्र मा ण वि नि श्च येऽभिहितं ।
	{\color{gray}{\rmlatinfont\textsuperscript{§~\theparCount}}}
	\pend% ending standard par
      ‚{\tiny $_{lb}$}‚

	  
	  \pstart \leavevmode% starting standard par
	\leavevmode\ledsidenote{\textenglish{20b/PSVTa}} \textbf{त‚स्मा}द्धेतुत्र‚य‚व्य‚तिरेके‚{\tiny $_{७}$}‚ण \textbf{नान्यो हेतुर्ग‚म‚कोस्}तीति स्थितं ॥
	{\color{gray}{\rmlatinfont\textsuperscript{§~\theparCount}}}
	\pend% ending standard par
      ‚{\tiny $_{lb}$}‚

	  
	  \pstart \leavevmode% starting standard par
	\textbf{एते}नेति र‚सादे रूपाद्य‚नुमान‚स्य कार्य‚लिङ्ग‚ज‚त्व‚क‚थ‚नेन । \textbf{पिपीलिकोत्स‚र‚ण}‚{\tiny $_{lb}$}‚न्तासां गृहीताण्डानाम‚न्य‚त्र स‚ञ्च‚र‚णं । \textbf{म‚त्स्या}नामु\textbf{द्व‚र्त्त}न‚म्म‚त्स्य‚विकारः । \textbf{आदि}‚{\tiny $_{lb}$}‚श‚ब्दाद् विद्युद्विकास‚म‚ण्डूक‚रुतादिप‚रिग्र‚हः । \textbf{व‚र्षाद्य‚नुमान‚मि}त्य\textbf{त्रापि} आदिश‚ब्दाद्‚{\tiny $_{lb}$}‚ वाताद्य‚नुमान\textbf{मुक्तं} । य‚था त‚द‚पि कार्य‚लिङ्ग‚ज‚हेतुध‚र्मानुमानात् ।
	{\color{gray}{\rmlatinfont\textsuperscript{§~\theparCount}}}
	\pend% ending standard par
      ‚{\tiny $_{lb}$}‚

	  
	  \pstart \leavevmode% starting standard par
	एत‚देवाह ।‚{\tiny $_{१}$}‚ \textbf{त‚त्रापी}त्याद । त‚त्रापि पिपीलिकोत्स‚र‚णादौ । भूत‚प‚रिणा‚{\tiny $_{lb}$}‚म एव व‚र्ष‚हेतुरिति । एव‚कारो भिन्न‚क्र‚मः । \textbf{व‚र्ष‚हेतुरेव भूत‚प‚रिणामः} । पिपीलि‚{\tiny $_{lb}$}‚कासंक्षोभादेरुत्स‚र‚णादिल‚क्ष‚ण‚स्य हेतुः । त‚स्माद् य‚थाभूताद् भूत‚प‚रिणामाद्‚{\tiny $_{lb}$}‚ व‚र्ष‚हेतुः \textbf{पिपीलिकादी}नाम्विकारो जातः । त‚थाभूत‚स्य हेतोर‚नुमानात् स‚मान‚{\tiny $_{lb}$}‚काल‚व‚र्षाद्य‚नुमानं । अन्य‚त्र तु योग्य‚तानुमी‚{\tiny $_{२}$}‚य‚ते न तु व‚र्ष एव प्र‚तिब‚न्ध‚स‚म्भ‚वेन‚{\tiny $_{lb}$}‚ \textbf{व्य‚भिचा}रात् । त‚स्मात् स‚म‚ग्राद‚पि कार‚ण‚क‚लापान्नास्ति कार्यानुमानं केव‚लं योग्य‚{\tiny $_{lb}$}‚तानुमान‚मेव ॥
	{\color{gray}{\rmlatinfont\textsuperscript{§~\theparCount}}}
	\pend% ending standard par
      ‚{\tiny $_{lb}$}‚

	  
	  \pstart \leavevmode% starting standard par
	ये तु मी मां स का द‚योऽस‚म‚ग्राद‚पि कार‚णात् कार्य‚म‚नुमिम‚ते । तेऽत्य‚न्त‚{\tiny $_{lb}$}‚न्याय‚ब‚हिष्कृता इत्येत‚द् द‚र्श‚य‚न्नाह [।]
	{\color{gray}{\rmlatinfont\textsuperscript{§~\theparCount}}}
	\pend% ending standard par
      ‚{\tiny $_{lb}$}‚

	  
	  \pstart \leavevmode% starting standard par
	\textbf{हेतुना त्व‚स‚म‚ग्रेणेत्यादि} । अस‚म‚ग्रेणेति विक‚लेन । \textbf{शेष‚व‚द‚नु}मान‚म‚नैकान्तिक‚{\tiny $_{lb}$}‚मित्य‚र्थः । कुतो‚{\tiny $_{३}$}‚साम‚र्थ्यात् ॥ अस्य व्याख्यानं \textbf{स‚म‚ग्राण्येव} हीत्यादि । कार्य‚स्य ताव‚त्‚{\tiny $_{lb}$}‚ स‚र्व‚था नानुमानं । \textbf{योग्य‚ताम‚प्य‚नुमाप‚य‚न्ति} स‚म‚ग्राण्येवानुमाप‚य‚न्तीत्येत‚द‚पिश‚ब्देनाह ।‚{\tiny $_{lb}$}‚ \textbf{अस‚म‚ग्र‚स्यैकान्तेनासाम‚र्थ्यादि}त्युत्तरोत्त‚र‚प‚रिणामेनाप्य‚साम‚र्थ्यान्न पाक्षिक‚म‚पि‚{\tiny $_{lb}$}‚ कार्यानुमानं । \textbf{देहाद् रागानुमान‚व‚द्} इत्य‚स्योदाह‚र‚ण‚स्य व्याख्यानं । य‚थेत्यादि ।‚{\tiny $_{lb}$}‚ देह‚श्चेन्द्रियाणि च \textbf{बुद्ध}य‚{\tiny $_{४}$}‚श्चेति द्व‚न्द्वः । रागादिमान‚यं पुरुषो देह‚व‚त्त्वादिन्द्रिय‚व‚त्त्वात्‚{\tiny $_{lb}$}‚ बुद्धि\textbf{म‚त्त्वात्} । एव‚न्देहादिभ्यो \textbf{रागाद्य‚नुमानं} । आदिश‚ब्दाद् द्वेष‚मोहादिप‚रिग्र‚हः ।‚{\tiny $_{lb}$}‚ \textbf{सूत्रे} तु देह‚राग‚ग्र‚ह‚ण‚मुप‚ल‚क्ष‚णं । न हीन्द्रियाद्येव कार‚णं रागादेः ॥ य‚स्मा\textbf{दात्मात्मीया‚{\tiny $_{lb}$}‚भिनिवेश‚पूर्व‚का रागाद‚यः} । आत्म‚न्यात्मीये चाह‚म्म‚मेति योभिनिवेशः स पूर्वः कार‚णं‚{\tiny $_{lb}$}‚ येषां रागा‚{\tiny $_{५}$}‚दीनान्तेन त‚थोक्ताः । \textbf{अयोनिश} इत्याद्य‚स्यैव स‚म‚र्थ‚नं । योनिः प‚दार्था‚{\tiny $_{lb}$}‚नाम‚नित्य‚दुःखानात्मादि । स‚म्य‚ग्द‚र्श‚न‚प्र‚स‚तिहेतुत्वात् । तं शंस‚त्याल‚म्ब‚त इति‚{\tiny $_{lb}$}‚ ‚{\tiny $_{lb}$}‚ \leavevmode\ledsidenote{\textenglish{51/s}}योनिशः । योनिं योनिं म‚न‚स्क‚रोतीति संख्यैक‚व‚च‚नाच्च वीप्साया \edtext{\textsuperscript{*}}{\edlabel{pvsvt_51-1}\label{pvsvt_51-1}\lemma{*}\Bfootnote{\href{http://sarit.indology.info/?cref=P\%C4\%81.5.4.43}{ Pāṇini 5:4:43. }}}मिति‚{\tiny $_{lb}$}‚ श‚स् प्र‚त्य‚यो वा । त‚थाभूत‚श्चासौ म‚न‚स्कार‚श्चेति योनिशोम‚न‚स्कारो नैरात्म्य‚{\tiny $_{lb}$}‚ज्ञानं । त‚द्विरुद्ध‚मात्मादिज्ञान‚म\textbf{योनि‚{\tiny $_{६}$}‚शोम‚न‚स्कार}स्त‚त्\textbf{पूर्व‚क‚त्वात् स‚र्व}रागादि‚{\tiny $_{lb}$}‚\textbf{दोषोत्प‚त्तेः} ॥
	{\color{gray}{\rmlatinfont\textsuperscript{§~\theparCount}}}
	\pend% ending standard par
      ‚{\tiny $_{lb}$}‚

	  
	  \pstart \leavevmode% starting standard par
	न‚नु देहेन्द्रिय‚बुद्ध‚योपि रागादिनां हेत‚व‚स्त‚द्र‚हितेषु रागाद्य‚द‚र्श‚नादित्याह ।‚{\tiny $_{lb}$}‚ \textbf{देहादीनां हेतुत्वेपी}ति \textbf{केव‚लाना}मित्य‚योनिशोम‚न‚स्कार‚र‚हितानां । रागादौ साध्ये ।‚{\tiny $_{lb}$}‚ रागादिर‚हिता उप‚ल‚ख‚ण्डाद‚यो \textbf{विप‚क्ष}स्त‚त्र हेतुर्देहादिम‚त्त्व‚स्य या \textbf{वृत्ति}स्त‚स्या‚{\tiny $_{७}$}‚‚{\tiny $_{lb}$}‚ \textbf{अदृष्टाव‚पि} । विप‚र्य‚ये बाध‚क‚प्र‚माणाभावाच्\textbf{छेष‚व‚द‚नुमा}न‚म‚स्मा\textbf{च्च संश‚यो भ‚व‚ति} \leavevmode\ledsidenote{\textenglish{21a/PSVTa}}‚{\tiny $_{lb}$}‚ न निश्च‚यः ॥ य‚था चैत‚द‚न‚न्त‚रोक्तं न प्र‚माण\textbf{न्त‚था विप‚क्षे} हेतोर\textbf{दृष्टिमात्रेण‚{\tiny $_{lb}$}‚ कार्य‚सामान्य}स्य कार्य‚मात्र‚स्य \textbf{द‚र्श‚नात्} । \textbf{हेतुज्ञान}म्विशिष्ट‚कार‚णानुमानं‚{\tiny $_{lb}$}‚ \textbf{प्र‚माणाभं} प्र‚माणाभास‚म‚नैकान्तिक‚मिति याव‚त् । किमिव [।] \textbf{व‚च‚नाद्रा‚{\tiny $_{lb}$}‚गितादिव‚त्} ॥
	{\color{gray}{\rmlatinfont\textsuperscript{§~\theparCount}}}
	\pend% ending standard par
      ‚{\tiny $_{lb}$}‚

	  
	  \pstart \leavevmode% starting standard par
	न‚नु स‚र्व‚मेव व‚च‚नं‚{\tiny $_{१}$}‚ \textbf{रागादिकार्य}मिति क‚थ‚न्त‚त्कार्य‚सामान्य‚मित्य‚त आह ।‚{\tiny $_{lb}$}‚ \textbf{न ही}त्यादि । ओष्ठ‚योश्\textbf{च‚ल‚नं} स्प‚न्दो \textbf{व‚च‚न}काले च त‚स्याव‚श्यंभावात् । \textbf{आदि}‚{\tiny $_{lb}$}‚श‚ब्दाद् अन्य‚स्यापि व‚च‚न‚काल‚भाविनो मुख‚विकारादेर्ग्र‚ह‚णं । अविशिष्ट‚विव‚क्षामात्रं‚{\tiny $_{lb}$}‚ व‚क्तुकाम‚ता सामान्य‚न्त‚देव \textbf{हेतुर्ये}षां स्प‚न्दादीनान्ते त‚थोक्तास्त‚द्भाव‚स्त‚स्मात् ॥‚{\tiny $_{lb}$}‚ \textbf{सैव} व‚क्तुकाम‚ता \textbf{राग इति चेत्} । आस‚क्तिरूप‚त्वाद् राग‚स्येति म‚{\tiny $_{२}$}‚न्य‚ते ॥ आ चा र्य‚{\tiny $_{lb}$}‚ आह । \textbf{इष्ट‚त्वान्न किंचिद् बाधितं} । व‚क्तुकाम‚ता कार्य‚स्य व‚च‚न‚स्येष्ट‚त्वान्न किञ्चिद्‚{\tiny $_{lb}$}‚ अनिष्टं । न च व‚क्तुकाम‚ता रागः,अपि तु \textbf{नित्य‚सुखात्मात्मीयाकारं य‚द्द‚र्श‚न-}‚{\tiny $_{lb}$}‚ ‚{\tiny $_{lb}$}‚ ‚{\tiny $_{lb}$}‚ \leavevmode\ledsidenote{\textenglish{52/s}}\textbf{म‚भिनिवेश‚स्तेनाक्षिप्तं} ज‚नितं । \textbf{सास्र‚व‚ध‚र्म‚विष‚य‚मि}ति प‚ञ्चोपादान‚स्क‚न्धाल‚म्ब‚नं‚{\tiny $_{lb}$}‚ चेत‚सोभिष्व‚ङ्गं \textbf{राग‚माहु}र्विद्वांसः ॥
	{\color{gray}{\rmlatinfont\textsuperscript{§~\theparCount}}}
	\pend% ending standard par
      ‚{\tiny $_{lb}$}‚

	  
	  \pstart \leavevmode% starting standard par
	स्यान्म‚तं [।] वीत‚रागाभिम‚तानां मैत्रीक‚रुणाद‚य इष्य‚न्ते । ते च स‚त्वा‚{\tiny $_{३}$}‚‚{\tiny $_{lb}$}‚ल‚म्ब‚न‚त्वाद् आत्मादिद‚र्श‚न‚प्र‚वृत्ताः स‚त्त्वानुन‚य‚प्र‚वृत्त्या चाभिष्व‚ङ्ग‚स्व‚भावास्त‚तो‚{\tiny $_{lb}$}‚ वीत‚रागा अपि रागिणः प्र‚स‚क्ता इत्य‚त आह । \textbf{नैवं क‚रुणाद‚य} इति [।] न स‚त्त्वा‚{\tiny $_{lb}$}‚ल‚म्ब‚ना वीत‚रागाणां क‚रुणाद‚यः । \textbf{अन्य‚थापि स‚म्भ‚वात्} । \add{ध‚र्माल‚म्ब‚नानाम‚पि‚{\tiny $_{lb}$}‚ स‚म्भ‚वात्}\edtext{\textsuperscript{*}}{\edlabel{pvsvt_52-1}\label{pvsvt_52-1}\lemma{*}\Bfootnote{In the margin.}}एत‚च्चान‚न्त‚र‚मेव \textbf{निवेद‚यिष्यामः} । अत्र व‚च‚नाद् रागानुमाने क्रिय‚माणे‚{\tiny $_{lb}$}‚ व‚च‚न‚मात्राद‚प्र‚तिप‚त्तिरित्य‚नेन स‚म्ब‚न्धः । य‚स्माद् \textbf{य‚था र‚{\tiny $_{४}$}‚क्तो ब्र‚वीति त‚था‚{\tiny $_{lb}$}‚ विर‚क्तोपी}ति प्र‚क्षीण‚रागोपि । \textbf{न व‚च‚न‚मात्राद्} रागानुमानं किन्त‚र्हि व‚च‚न‚विशेषात् ।‚{\tiny $_{lb}$}‚ यो रागेणैव ज‚न्य‚त इत्याह । \textbf{नापि विशेषा}दिति । किङ्कार‚ण‚म् [।] \textbf{अभिप्राय‚स्य‚{\tiny $_{lb}$}‚ दुर्बोध‚त्वात्} । विर‚क्तो हि र‚क्त‚व‚च्चेष्ट‚ते । र‚क्तोपि विर‚क्त‚व‚दित्य‚भिप्रायो दुर्बोधः ।‚{\tiny $_{lb}$}‚ त‚त‚श्च \textbf{व्य‚व‚हार‚सं\edtext{}{\edlabel{pvsvt_52-1b}\label{pvsvt_52-1b}\lemma{सं}\Bfootnote{१ In the margin.}}क‚रेण स‚र्वेषा}मिति व‚च‚नानां । व‚च‚न‚मात्र‚स्य व‚च‚न‚विशेष‚स्य च‚{\tiny $_{lb}$}‚ रा‚{\tiny $_{५}$}‚गादेर्लिङ्ग‚त्वेनोप‚नीत‚स्य \textbf{व्य‚भिचारात्} । न विशेषाद‚पि रागाद्य‚नुमानं ॥ य‚दुक्त‚{\tiny $_{lb}$}‚न्त‚था विर‚क्तो ब्र‚वीतीति त‚त्रोत्त‚र‚माशंक‚ते । \textbf{प्र‚योज‚ने}त्यादि । व‚च‚नोच्चार‚णं व्य‚व‚{\tiny $_{lb}$}‚ हारः । न हि वीत‚राग‚स्य व‚च‚नोच्चार‚णे फ‚ल‚म‚स्ति त‚था चाव्य‚भिचारो राग‚व‚च‚न‚{\tiny $_{lb}$}‚योरिति भावः । नेत्या चा र्यः । \textbf{न} प्र‚योज‚नाभावः \textbf{प‚रार्थ‚त्वाद्} व्याहार‚स्य ।
	{\color{gray}{\rmlatinfont\textsuperscript{§~\theparCount}}}
	\pend% ending standard par
      ‚{\tiny $_{lb}$}‚

	  
	  \pstart \leavevmode% starting standard par
	\textbf{न युक्त} इत्यादिप‚रः ।‚{\tiny $_{६}$}‚ वीत‚रागो हि प‚रेष्ट‚व्यास‚क्तो न च स‚क्तिम‚न्त‚रेण‚{\tiny $_{lb}$}‚ प‚रार्थ‚प्र‚वृत्तिर‚स्तीति भावः । नैत‚देवं । \textbf{क‚रुण‚यापि प्र‚वॄत्तेः ॥ सैव क‚रुणा राग‚{\tiny $_{lb}$}‚ इति चेत्} । त‚देत\textbf{दिष्टं} क‚रुणा राग इति नाम‚क‚र‚णं ॥ स‚त्त्व‚द‚र्श‚न\textbf{विप‚र्यासायात‚त्वात्}‚{\tiny $_{lb}$}‚ क‚रुणापि रागात्म‚को \textbf{दोष} इति चेदाह । \textbf{अविप‚र्यासे}त्यादि । \textbf{अविप‚र्यासोद्भ‚व‚त्व}‚{\tiny $_{lb}$}‚\leavevmode\ledsidenote{\textenglish{21b/PSVTa}} मेवाह । \textbf{अस‚त्य‚प्यात्म}ग्र‚ह‚ण इत्यादि । \textbf{दुःख‚{\tiny $_{७}$}‚विशेष‚द‚र्श‚न‚मात्रेणे}ति संस्का\textbf{र‚दुःख‚ता}‚{\tiny $_{lb}$}‚‚{\tiny $_{lb}$}‚ ‚{\tiny $_{lb}$}‚ \leavevmode\ledsidenote{\textenglish{53/s}}निरूप‚ण‚मात्रेण । \textbf{अभ्यास‚ब‚लोत्पादिते}ति पूर्व‚पूर्व‚स‚जातीय‚क्ष‚णोत्प‚न्ना भ‚व‚त्येव‚{\tiny $_{lb}$}‚ क‚रुणा । आग‚मेनापि संस्य‚न्द‚य‚न्नाह । \textbf{त‚था ही}त्यादि । आदिश‚ब्दाद् अनाल‚म्ब‚ना‚{\tiny $_{lb}$}‚ गृह्य‚न्ते । \textbf{स‚त्त्वा}ल‚म्ब‚ना पृथ‚ग्ज‚नानां । \textbf{ध‚र्माल‚म्ब‚ना} आर्याणां । अनाल‚म्ब‚ना ग्राह्य‚{\tiny $_{lb}$}‚ग्राह‚काभिनिवेश‚विग‚तानां बुद्ध‚बोधिस‚त्त्वानां । \textbf{मैत्र्याद‚यो} मैत्रीक‚रुणामुदितो‚{\tiny $_{१}$}‚पेक्षा‚{\tiny $_{lb}$}‚ \textbf{इष्य‚न्ते} सिद्धान्ते ।
	{\color{gray}{\rmlatinfont\textsuperscript{§~\theparCount}}}
	\pend% ending standard par
      ‚{\tiny $_{lb}$}‚

	  
	  \pstart \leavevmode% starting standard par
	न‚नु च स‚त्त्वाल‚म्ब‚ना एव पृथ‚ग्ज‚नेनाभ्य‚स्तास्तास्त‚त्क‚थं ध‚र्माद्याल‚म्ब‚ना‚{\tiny $_{lb}$}‚ उच्य‚न्त इत्य‚त आह ।
	{\color{gray}{\rmlatinfont\textsuperscript{§~\theparCount}}}
	\pend% ending standard par
      ‚{\tiny $_{lb}$}‚

	  
	  \pstart \leavevmode% starting standard par
	\textbf{एताश्च} मैत्रीक‚रुणामुदितोपेक्षाः । \textbf{स‚जातीयाभ्यास‚वृत्त‚यः} पूर्व‚पूर्व‚स‚दृश‚क्ष‚ण‚ब‚ले‚{\tiny $_{lb}$}‚नोत्प‚त्तेः । एतावांस्तु विशेषो याव‚त् स‚त्त्व‚म्प‚श्य‚ति ताव‚त्स‚त्त्वाल‚म्ब‚नाः । याव‚द्ध‚र्म‚{\tiny $_{lb}$}‚न्ताव‚द् ध‚र्माल‚म्ब‚नाः । \textbf{न रागापेक्षिण्य} इति नानुश‚येन स‚त्त्वेषु प्र‚व‚र्त्त‚न्त‚{\tiny $_{२}$}‚ इत्य‚र्थः ।
	{\color{gray}{\rmlatinfont\textsuperscript{§~\theparCount}}}
	\pend% ending standard par
      ‚{\tiny $_{lb}$}‚

	  
	  \pstart \leavevmode% starting standard par
	न‚नु दुःख‚विशेष‚द‚र्श‚न‚मात्रेणाभ्यास‚ब‚लोत्पादिनीत्य‚न‚न्त‚र‚मेवाय‚म‚र्थ उक्तः ।
	{\color{gray}{\rmlatinfont\textsuperscript{§~\theparCount}}}
	\pend% ending standard par
      ‚{\tiny $_{lb}$}‚

	  
	  \pstart \leavevmode% starting standard par
	स‚त्यं । क‚रुणामेवाश्रित्याधुना स‚र्वाण्येवेति विशेषः । आत्म‚द‚र्श‚न‚निवृत्ताव‚पि‚{\tiny $_{lb}$}‚ त‚र्हि क‚रुणाव‚द‚भ्यासाद् रागाद‚यः प्र‚व‚र्त्त‚न्त इत्याह । \textbf{नैवं रागाद}य‚स्स‚जातीयाभ्यास‚{\tiny $_{lb}$}‚वृत्त‚यो येनात्म‚दृष्टिनिवृत्ताव‚पि प्र‚व‚र्त्तेर‚न् । आत्मादि\textbf{विप‚र्यासाभावेऽभावात्} ॥
	{\color{gray}{\rmlatinfont\textsuperscript{§~\theparCount}}}
	\pend% ending standard par
      ‚{\tiny $_{lb}$}‚

	  
	  \pstart \leavevmode% starting standard par
	\textbf{कारुणिक‚स्य निष्फ‚{\tiny $_{३}$}‚लः} प‚रार्थ \textbf{आर‚म्भोऽविप‚र्यासादा}त्माद्य‚भिनिवेशाभावेन‚{\tiny $_{lb}$}‚ स्वार्थ‚स्यैवाभावात् । नाय‚न्दोषः \textbf{प‚रार्थ‚स्यैव फ‚ल‚त्वेनेष्ट‚त्वात्} ॥ इष्टो नाम प‚रार्थ‚{\tiny $_{lb}$}‚स्त‚थापि क‚थं फ‚ल‚त्व‚मिति चेत् ॥ आह । \textbf{इच्छे}त्यादि । \textbf{इच्छ‚या} ल‚क्ष्य‚त इतीच्छा‚{\tiny $_{lb}$}‚ल‚क्ष‚ण‚मिच्छाविष‚य‚त्वादित्य‚र्थः । य‚दिष्ट‚न्त‚त्फ‚ल‚मिति याव‚त् ॥
	{\color{gray}{\rmlatinfont\textsuperscript{§~\theparCount}}}
	\pend% ending standard par
      ‚{\tiny $_{lb}$}‚

	  
	  \pstart \leavevmode% starting standard par
	न‚न्व‚ह‚मिति बुद्धिर‚हित‚स्य क‚थं प‚रार्थापि प्र‚वृत्तिरिति चेत् [।] न [।]‚{\tiny $_{lb}$}‚ अक्लिष्टा‚{\tiny $_{४}$}‚ ज्ञान‚स‚द्भावात् प्र‚वृत्तिरित्येके । योग‚ब‚लेन शुद्ध‚लौकिक‚चित्त‚स‚म्मुखी‚{\tiny $_{lb}$}‚क‚र‚णादित्य‚प‚रे । विनेयानां त‚था प्र‚तिभास‚नादित्य‚न्ये ॥
	{\color{gray}{\rmlatinfont\textsuperscript{§~\theparCount}}}
	\pend% ending standard par
      ‚{\tiny $_{lb}$}‚

	  
	  \pstart \leavevmode% starting standard par
	\textbf{स‚र्व‚थे}त्युप‚संहारः । य‚दि व‚च‚नाद् वीत‚राग‚स्य व‚क्तुकाम‚ता साध्य‚ते अथ क‚रुणा‚{\tiny $_{lb}$}‚ तेन स‚राग इत्युच्य‚ते । \textbf{स‚र्व‚थाऽभूतास‚मारोपाद्} वीत‚रागादि\textbf{र्निर्दोषः} । दोष‚स्व‚भाव‚{\tiny $_{lb}$}‚स्य रागादेर‚व‚श्य‚भावात् । \textbf{त‚द‚न्येन} रागादिभ्योन्येन व‚क्तु‚{\tiny $_{५}$}‚काम‚तादिना \textbf{वीत‚राग‚स्य}‚{\tiny $_{lb}$}‚ ‚{\tiny $_{lb}$}‚ \leavevmode\ledsidenote{\textenglish{54/s}}\textbf{दोष‚व‚त्त्व‚साध‚ने न किंचिद् अनिष्टं} ॥
	{\color{gray}{\rmlatinfont\textsuperscript{§~\theparCount}}}
	\pend% ending standard par
      ‚{\tiny $_{lb}$}‚

	  
	  \pstart \leavevmode% starting standard par
	स्यान्म‚तिः [।] व‚क्त‚र्यात्म‚नि रागादिर्दृष्ट‚स्त‚तः स‚प‚क्षे स‚त्त्व‚द‚र्श‚न‚मात्रेण वीत‚{\tiny $_{lb}$}‚रागाभिम‚तेष्व‚प्य‚नुमान‚म्भ‚विष्य‚तीत्याह । \textbf{व‚क्त‚र्यात्म‚नी}त्यादि । \textbf{अन्य‚त्र त‚द‚नुमान}‚{\tiny $_{lb}$}‚ इति वीत‚रागाभिम‚ते रागाद्य‚नुमाने\textbf{ऽतिप्र‚संगः} । व‚क्त‚र्यात्म‚नि यावान् क‚श्चिद् विशेषो‚{\tiny $_{lb}$}‚ दृष्ट‚स्त‚स्य स‚र्व‚स्यान्य‚त्रानुमान‚प्र‚स‚ङ्गः ।‚{\tiny $_{६}$}‚ आत्म‚नि दृष्ट‚स्य श्याम‚गौर‚तादिल‚क्ष‚ण‚स्य‚{\tiny $_{lb}$}‚ विशेष‚स्य \textbf{व्य‚भिचारात् । अन‚न्यानुमान} इति रागाद‚न्य‚स्यात्म‚ग‚त‚स्य विशेष‚स्यानुमाने ।‚{\tiny $_{lb}$}‚ \textbf{इहेति} साध्याभिम‚ते रागादाव‚प्य\textbf{व्य‚भिचार इति को निश्च‚यः} । नैव क‚श्चित् ।‚{\tiny $_{lb}$}‚ प्र‚तिब‚न्धाभावाद् राग‚व‚च‚न‚योः । य‚स्मादात्म‚निद‚र्श‚नान्नानुमान‚न्त‚स्मात् \textbf{क‚र‚ण‚{\tiny $_{lb}$}‚\leavevmode\ledsidenote{\textenglish{22a/PSVTa}} गुण‚व‚क्तुकाम‚ते व‚च‚न‚म‚नुमाप‚येत्} ।‚{\tiny $_{७}$}‚ क‚र‚ण‚गुण‚स्ताल्वादीनाम्पाट‚वादिः स च‚{\tiny $_{lb}$}‚ व‚क्तुकाम‚ता चेति द्व‚न्द्वः । द्वितीयाद्विव‚च‚न‚मेवैत‚त् ॥ \textbf{रागोत्पाद‚न‚योग्य‚तार‚हिते}‚{\tiny $_{lb}$}‚ पाषाणादौ \textbf{व‚च‚नाद‚र्श‚नात्} । सैव व‚च‚नाद‚व‚सीय‚त इत्याह । रागेत्यादि । \textbf{त‚द‚नुमान}‚{\tiny $_{lb}$}‚ इति योग्य‚तानुमाने\textbf{ऽतिप्र‚संगः} । त‚था हि य‚था पाषाणादौ रागोत्पाद‚न‚योग्य‚ताविर‚{\tiny $_{lb}$}‚ह‚स्त‚था स‚र्व‚पुरुष‚ध‚र्मैर‚पि त‚त‚स्त‚थाभूते व‚च‚नाद‚र्श‚{\tiny $_{१}$}‚नाद् य‚दि रागोत्पाद‚न‚योग्य‚ता‚{\tiny $_{lb}$}‚नुमान‚मेवं स‚र्व‚पुरुष‚ध‚र्म‚स्य ॥
	{\color{gray}{\rmlatinfont\textsuperscript{§~\theparCount}}}
	\pend% ending standard par
      ‚{\tiny $_{lb}$}‚

	  
	  \pstart \leavevmode% starting standard par
	अथ म‚तं [।]रागोत्पाद‚न‚योग्य‚ता हि राग‚व‚च‚न‚योरेक‚ङ्कार‚ण‚न्त‚द‚नुमाने च‚{\tiny $_{lb}$}‚ रागोप्य‚र्थ‚तोनुमितो भ‚व‚ति तेन योग्य‚ता एवानुमानं न स‚र्वेषां पुरुष‚ध‚र्माणामि‚{\tiny $_{lb}$}‚त्य‚त्राह [।] \textbf{राग‚स्येत्या}दि । \textbf{राग‚स्यानुप‚योगे} व‚च‚नं प्र‚ति ताल्वादिव्यापारादेव‚{\tiny $_{lb}$}‚ श‚ब्द‚निष्प‚तेः । \textbf{क‚थ‚न्त‚च्छ‚क्तिः} राग‚श‚क्तिर्व‚च‚नं‚{\tiny $_{२}$}‚प्र‚त्यु\textbf{प‚युज्य‚ते । अथ} ताल्वा‚{\tiny $_{lb}$}‚दिव्यापार‚काले राग‚श‚क्तेर‚पि व‚च‚न‚म्प्र‚त्युप‚योग इष्य‚ते [।] त‚त‚श्चाद्य‚व‚र्ण्ण‚निष्प‚{\tiny $_{lb}$}‚त्तिकाले राग‚स्यापि निष्प‚त्तिरेक‚साम‚ग्र्य‚धीन‚त्वात् [।] त‚स्य च राग‚स्य स‚न्तान‚{\tiny $_{lb}$}‚वा\textbf{हित्व}न्त‚था च द्वितीयादिव‚र्ण्ण‚निष्प‚तौ राग एवोप‚युक्त‚स्स्यात् । त‚दाह [।]‚{\tiny $_{lb}$}‚ \textbf{श‚क्त्युप‚योगे हि स एवोप‚युक्तः स्यादि}ति । \textbf{न} च राग‚स्योप‚योगोस्ती\textbf{त्युक्तं}‚{\tiny $_{lb}$}‚ प्राक् ।
	{\color{gray}{\rmlatinfont\textsuperscript{§~\theparCount}}}
	\pend% ending standard par
      ‚{\tiny $_{lb}$}‚

	  
	  \pstart \leavevmode% starting standard par
	अथ‚वात्मात्मीयाभिनिवे‚{\tiny $_{३}$}‚श‚र‚हितानां राग‚स्यानुप‚योगे व‚च‚नंप्र‚ति । क‚थं त‚च्छ‚{\tiny $_{lb}$}‚क्ती राग‚श‚क्तिर्वास‚नाख्यात्मात्मीयाभिनिवेश‚ल‚क्ष‚णा व‚च‚न‚म्प्र‚त्युप‚युज्य‚ते । वीत‚{\tiny $_{lb}$}‚‚{\tiny $_{lb}$}‚ \leavevmode\ledsidenote{\textenglish{55/s}}रागाणामात्माद्य‚भिनिवेश‚स्याभावात् । श‚क्त्युप‚योगे हि व‚च‚नं प्र‚ति । स एव राग‚{\tiny $_{lb}$}‚ उप‚युक्तः स्यादात्माद्य‚भिनिवेश‚व‚तां स‚र्वेषामेव रागित्वात् । त‚च्च राग‚स्य व‚च‚नं‚{\tiny $_{lb}$}‚प्र‚त्युप‚योगित्व‚न्नास्तीत्युक्तं । व‚क्तुकाम‚ता सामान्य‚हेतु‚{\tiny $_{४}$}‚त्वादित्य‚त्रान्त‚रे ॥
	{\color{gray}{\rmlatinfont\textsuperscript{§~\theparCount}}}
	\pend% ending standard par
      ‚{\tiny $_{lb}$}‚

	  
	  \pstart \leavevmode% starting standard par
	\textbf{त‚स्मा}दित्यादि निग‚म‚नं । नान्त‚रीय‚क‚मेवेति कार‚णेना\textbf{विनाभाव्येव} । त‚त्प्र‚{\tiny $_{lb}$}‚तिब‚न्धात् । \textbf{त‚त्र} कार‚णे \textbf{आय‚त्त‚त्वात् । नान्य}द‚प्र‚तिब‚द्धं \textbf{विप‚क्षे} हेतोर\textbf{द‚र्श‚नेपि} ।‚{\tiny $_{lb}$}‚ अर्वाग्द‚र्श‚न‚स्याद‚र्श‚न‚मात्रेणानिश्च‚यात् । \textbf{स‚र्व‚द‚र्शिन‚स्तु द‚र्श‚न‚निवृत्त्या} स्या\textbf{न्निश्च‚{\tiny $_{lb}$}‚य}स्त‚स्य हि स‚र्व‚ज्ञेय‚व्यापि ज्ञानं । एत‚देवाह । \textbf{स‚र्व‚द‚र्श‚न} इत्यादि । किम्पुन‚र‚द‚{\tiny $_{lb}$}‚र्श‚न‚मात्रा‚{\tiny $_{५}$}‚न्नाभाव‚निश्च‚य इत्याह । \textbf{क्व‚चित्त‚थे}त्यादि । तेन प्र‚कारेण \textbf{दृष्टानां}‚{\tiny $_{lb}$}‚ प्राक् । \textbf{पुन‚र्देश‚काल‚संस्कार‚भेदे}नेति भेद‚श‚ब्द‚स्य प्र‚त्येकं स‚म्ब‚न्धः । संस्कारः‚{\tiny $_{lb}$}‚ क्षीराव‚सेकादिः । पूर्व‚दृष्ट‚प्र‚काराद\textbf{न्य‚था} स्यात् । किमिवेत्याह । \textbf{य‚थाम‚ल‚क्य}‚{\tiny $_{lb}$}‚ इति । संस्कार‚भेद‚स्यैत‚दुदाह‚र‚ण‚न्न \textbf{चैव‚म्ब‚हुल}मिति न म‚धुर‚फ‚ला ब‚हुल\textbf{न्दृश्य‚न्ते} ।‚{\tiny $_{lb}$}‚ क‚षाय‚फ‚लानां बाहुल्येन द‚र्श‚नात् । न चेदा‚{\tiny $_{६}$}‚नीं ब‚हुलं म‚धुर‚फ‚लानाम‚द‚र्श‚नात् ।‚{\tiny $_{lb}$}‚ क्व‚चित्स‚म्भ‚विनो म‚धुर‚फ‚ल‚स्य प्र‚तिक्षेपः । त‚था देश‚भेदेन पिण्ड‚ख‚र्ज्जूर‚स्य स‚म्भ‚वः ।‚{\tiny $_{lb}$}‚ काल‚भेदेन पुष्पादेः । आम‚ल‚कीदृष्टान्तेन चैत‚दाह । य‚थाम‚ल‚क्यः क्षीरादिसं‚{\tiny $_{lb}$}‚स्काराद‚न्य‚था भ‚व‚न्ति । त‚था रागादियोग्य‚श्चित्त‚स‚न्तान‚स्त‚त्त्वाभ्यासाभिसं‚{\tiny $_{lb}$}‚स्काराद् योग्यो भ‚व‚ति । अभ्यासाच्च नैरात्म्याल‚म्ब‚न‚मेव विज्ञानं स्प‚{\tiny $_{७}$}‚ष्टाभ‚म्वैरा- \leavevmode\ledsidenote{\textenglish{22b/PSVTa}}‚{\tiny $_{lb}$}‚ राग्य‚मुच्य‚ते । नैरात्म्य‚द‚र्श‚नादात्माद्य‚भिनिवेश‚विग‚मेन रागाद्य‚नुत्प‚त्तेः । स‚र्व‚ध‚र्म‚{\tiny $_{lb}$}‚क्ष‚णिक‚त्वाद्याल‚म्ब‚नं च ज्ञानं स्प‚ष्टाभं \textbf{स‚र्व‚ज्ञ‚त्वं} चोच्य‚ते । स‚र्वाज्ञान‚विग‚मात् ।‚{\tiny $_{lb}$}‚ त‚स्मात् त‚त्त्वाभ्यास‚निमित्ता स्फुटाभ‚त्व‚बृद्धिरेव त‚त्त्व‚साक्षात्क‚र‚णेन प्र‚त्य‚क्ष‚त्व‚{\tiny $_{lb}$}‚कारिणी रागादियोग्य‚स्व‚भाव‚तां ज्ञान‚स्य बाध‚ते । त‚त‚श्चान्य‚थादृष्ट‚म‚पि हेतुब‚ला‚{\tiny $_{lb}$}‚द‚न्य‚था भ‚वेद‚पि [।] तेन य‚दि नाम‚{\tiny $_{१}$}‚ रागार‚हिते क्व‚चिद् व‚च‚नं न दृष्ट‚न्ताव‚ता स‚र्व‚त्र‚{\tiny $_{lb}$}‚ वीत‚रागे व‚च‚नेन न भाव्य‚मिति नास्ति निश्च‚यः । \textbf{त‚त}श्चानिश्चित‚व्य‚तिरेकाद्‚{\tiny $_{lb}$}‚ व‚च‚न‚मात्रान्न राग‚द्य‚नुमानं । य‚त एव‚न्तेन कार‚णे\textbf{नैत‚द्युक्त‚म्व‚क्तुं मादृशो व‚क्ता} ।‚{\tiny $_{lb}$}‚ योह‚मिवायोनिशोम‚न‚स्कारान् । त‚देवाह । \textbf{रागोत्प‚त्ती}त्यादि । कः पुन‚र‚सावित्याह ।‚{\tiny $_{lb}$}‚ \textbf{आत्मे}त्यादि । \textbf{आत्म‚द‚र्श‚नं} स त्का य दृ ष्टिः । नित्य‚सुखादिविप‚र्या‚{\tiny $_{२}$}‚सोऽयोनिशो‚{\tiny $_{lb}$}‚ म‚न‚स्कारः । द्व‚न्द्व‚स‚मास‚श्चायं । आत्म‚द‚र्श‚न‚मेवायोनिशोम‚न‚स्कार इति विशेष‚ण‚{\tiny $_{lb}$}‚‚{\tiny $_{lb}$}‚ \leavevmode\ledsidenote{\textenglish{56/s}}स‚मासोवा । \textbf{त‚दे}ति मादृशो \textbf{व‚क्ते}ति विशेष‚णेप्य\textbf{पार्थ‚को व‚च‚नोदाहारः} । व‚च‚नादित्य‚स्य‚{\tiny $_{lb}$}‚ हेतोरुदाह‚र‚ण‚म‚न‚र्थ‚क‚मित्य‚र्थः । त‚दा हि यो मादृशो रागोत्प‚त्तिप्र‚त्य‚य‚विशेषेण‚{\tiny $_{lb}$}‚ युक्तः स रागी । य‚थाह‚मित्य‚य‚मेव हेतुः स्यात् । य‚त एव‚म‚द‚र्श‚न‚मात्रा‚{\tiny $_{३}$}‚न्नास्ति‚{\tiny $_{lb}$}‚ निश्च‚य‚स्त\textbf{स्माद् विप‚क्षेऽदृष्टि}र्विप‚क्षे हेतोर‚द‚र्श‚न‚म\textbf{हेतु}र्लिङ्ग‚स्य व्य‚तिरेक‚निश्च‚यं‚{\tiny $_{lb}$}‚ प्र‚ति ॥
	{\color{gray}{\rmlatinfont\textsuperscript{§~\theparCount}}}
	\pend% ending standard par
      ‚{\tiny $_{lb}$}‚

	  
	  \pstart \leavevmode% starting standard par
	स्यान्म‚तिः [।] विप‚क्ष‚दृष्ट्या हेतोर्व्य‚भिचारो न च वीत‚रागे व‚च‚नं दृष्ट‚न्त‚{\tiny $_{lb}$}‚स्माद‚द‚र्श‚नात् साध्याभावे व्य‚तिरेकः सिद्ध इत्य‚त आह ।
	{\color{gray}{\rmlatinfont\textsuperscript{§~\theparCount}}}
	\pend% ending standard par
      ‚{\tiny $_{lb}$}‚

	  
	  \pstart \leavevmode% starting standard par
	\textbf{न चाद‚र्श‚न‚मात्रेणे}ति \textbf{विप‚क्षे} हेतोर\textbf{व्य‚भिचारिता} । क‚स्मात् [।] \textbf{म‚म्भाव्य‚{\tiny $_{lb}$}‚व्य‚भिचार‚त्वात्} । स‚म्भाव्यो व्य‚भिचारो य‚स्य स त‚था त‚द्भाव‚स्त‚स्मात्‚{\tiny $_{४}$}‚ । य‚द्य‚पि‚{\tiny $_{lb}$}‚ न दृष्टो विप‚क्षे त‚थापि त‚त्र स‚म्भ‚वो न विरुद्ध इति संम्भाव्य‚ते व्य‚भिचारः ।‚{\tiny $_{lb}$}‚ स्थाल्य‚न्त‚र्ग‚तास्त‚ण्डुलाः \textbf{स्थालीत‚ण्डुलास्तेषां पाक‚व‚त्} ।
	{\color{gray}{\rmlatinfont\textsuperscript{§~\theparCount}}}
	\pend% ending standard par
      ‚{\tiny $_{lb}$}‚

	  
	  \pstart \leavevmode% starting standard par
	एत‚मेव दृष्टान्तं स‚म‚र्थ‚यितुमाह । \textbf{न ही}त्यादि । \textbf{बाहुल्येन} स्थाल्य‚न्त‚र्ग‚तानां‚{\tiny $_{lb}$}‚ \textbf{प‚क्वानान्द‚र्श‚ने}पि न \textbf{स्थाल्य‚न्त‚र्ग‚त‚त्वेन केव‚लेन पाकः सिध्य‚ति} । मात्र‚ग्र‚ह‚ण‚न्तु विशेष‚{\tiny $_{lb}$}‚निरासार्थं । य‚द् व‚क्ष्य‚त्येव‚न्तु स्यादिति । कुतोऽ‚{\tiny $_{५}$}‚सिद्धि\textbf{र्व्य‚भिचार‚स्य द‚र्श‚नात् । एवं}‚{\tiny $_{lb}$}‚स्व‚भावा इति ये प‚क्वा दृष्टास्तैस्तुल्य‚स्व‚भावाः । एतैरेव प‚क्वैः \textbf{स‚मानः}‚{\tiny $_{lb}$}‚ पाक‚हेतुर्येषान्ते प‚क्वा इति । \textbf{अन्य‚था त्वि}त्य‚स‚त्ये-त‚स्मिन् विशेष‚णे । \textbf{शेषो}स्तीति‚{\tiny $_{lb}$}‚ \textbf{शेष‚व‚द}निर्ण्णीतो विष‚योस्तीति याव‚त् । त‚च्च \textbf{व्य‚भिचारि} ॥
	{\color{gray}{\rmlatinfont\textsuperscript{§~\theparCount}}}
	\pend% ending standard par
      ‚{\tiny $_{lb}$}‚

	  
	  \pstart \leavevmode% starting standard par
	किन्नै या यि को क्तं कार्यात् कार‚णानुमान‚रूपं शेष‚व‚द‚नुमान‚मिहाभिप्रेत‚मुता‚{\tiny $_{lb}$}‚‚{\tiny $_{lb}$}‚ \leavevmode\ledsidenote{\textenglish{57/s}}न्य‚देवेति पृच्छ‚ति । \textbf{किं‚{\tiny $_{६}$}‚पुन‚रि}त्यादि । शेष‚व‚त्स्व‚रूप‚माह । \textbf{य‚स्ये}त्यादि । \textbf{य‚स्य} हेतोर‚{\tiny $_{lb}$}‚\textbf{द‚र्श‚न‚मात्रेणा}प्र‚माण‚केन विप‚क्ष\textbf{व्य‚तिरेकः प्र‚साध्य‚ते । त‚स्य} हेतोः \textbf{संश‚य‚हे}तुत्वात्‚{\tiny $_{lb}$}‚ संश‚य‚कारित्वा\textbf{च्छेष‚व‚त्त‚द}नुमान‚मु\textbf{दाहृतं} । किङ्कार‚ण\textbf{न्त‚स्य} हेतोः \textbf{स व्य‚तिरेकोऽनिश्चित‚{\tiny $_{lb}$}‚ इति विप‚क्षेपि वृत्तिराशंक्येत} । किंपुन‚र्न निश्चित इत्याह । \textbf{व्य‚तिरेके}त्यादि ॥‚{\tiny $_{lb}$}‚ अनुप‚ल‚म्भेपि क‚थं संश‚{\tiny $_{७}$}‚य इत्याह । \textbf{न स‚र्वे}त्यादि । दृश्यानु\textbf{प‚ल‚ब्धिरेव न ग‚मिका} । \leavevmode\ledsidenote{\textenglish{23a/PSVTa}}‚{\tiny $_{lb}$}‚ य‚त एव‚न्नाद‚र्श‚न‚मात्राद् व्य‚तिरेक\textbf{स्त‚स्मादेक‚निवृत्त्या} साध्य‚निवृत्त्या\textbf{न्य‚निवृत्तिं}साध‚न‚{\tiny $_{lb}$}‚निवृत्ति\textbf{मिच्छ}ता \textbf{त‚योः} साध्य‚साध‚न‚योः \textbf{क‚श्चित्} स्व‚भावेन \textbf{प्र‚तिब‚न्ध}स्तादात्म्य‚त‚{\tiny $_{lb}$}‚दुत्प‚त्तिल‚क्ष‚णो\textbf{प्येष्ट‚व्यः} । न केव‚ल‚म‚दृष्टिमात्रं । \textbf{अन्य‚था} प्र‚तिब‚न्धानिष्टाव\textbf{ग‚म‚को‚{\tiny $_{lb}$}‚ हेतुः स्यात्} । व्याप्तेर‚निश्चित‚त्वात् ॥
	{\color{gray}{\rmlatinfont\textsuperscript{§~\theparCount}}}
	\pend% ending standard par
      ‚{\tiny $_{lb}$}‚

	  
	  \pstart \leavevmode% starting standard par
	य‚त एव‚न्तेन कार‚णेन \textbf{हे‚{\tiny $_{१}$}‚तोस्त्रिष्व‚पि रूपेषु} प‚क्ष‚ध‚र्मान्व‚य‚व्य‚तिरेकेषु \textbf{निश्च‚यो‚{\tiny $_{lb}$}‚ व‚र्ण्णितः} । आचार्य दि ग्ना गे न प्र मा ण स मु च्च या दिषु । \textbf{असिद्ध‚स्तु} द्व‚योर‚पि‚{\tiny $_{lb}$}‚ साध‚न‚मित्यादिना । क‚स्य निरासेनेत्याह । अ\textbf{सिद्धे}त्यादि । \textbf{आद्या}दित्वात् तृतीयार्थे‚{\tiny $_{lb}$}‚ त‚सि विप‚क्षेणेत्य‚र्थः ।
	{\color{gray}{\rmlatinfont\textsuperscript{§~\theparCount}}}
	\pend% ending standard par
      ‚{\tiny $_{lb}$}‚

	  
	  \pstart \leavevmode% starting standard par
	त‚त्रासिद्ध‚विप‚क्षेण प‚क्ष‚ध‚र्म‚त्व‚निश्च‚यो व‚र्ण्णितः । \textbf{विप‚रोतार्था}विरुद्ध‚स्त‚स्य‚{\tiny $_{lb}$}‚ विप‚क्षेणान्व‚य‚निश्च‚यः । \textbf{व्य‚भिचार्य‚नै}‚{\tiny $_{२}$}‚कान्तिक‚स्त‚स्य विप‚क्षेण व्य‚तिरेक‚निश्च‚यः ॥
	{\color{gray}{\rmlatinfont\textsuperscript{§~\theparCount}}}
	\pend% ending standard par
      ‚{\tiny $_{lb}$}‚

	  
	  \pstart \leavevmode% starting standard par
	अन्व‚य‚व्य‚तिरेक‚निश्च‚यं च व‚र्ण्ण‚य‚ता प्र‚तिब‚न्धोप्य‚र्थ‚तो द‚र्शित एव । य‚स्मान्न‚{\tiny $_{lb}$}‚ \textbf{ह्य‚स‚ति प्र‚तिब‚न्धेऽन्व‚य‚व्य‚तिरेक‚निश्च‚योस्ति [।] तेन} कार‚णेन \textbf{त‚मेव} तादात्म्य‚{\tiny $_{lb}$}‚त‚दुत्प‚त्तिल‚क्ष‚णं प्र‚तिब‚न्ध\textbf{न्द‚र्श‚य}न्नाक्षिप\textbf{न्निश्च‚य‚माह} । य‚स्य दोष‚स्य निराशे\edtext{}{\lemma{निराशे}\Bfootnote{? से}}‚{\tiny $_{lb}$}‚ ‚{\tiny $_{lb}$}‚ \leavevmode\ledsidenote{\textenglish{58/s}}न यो निश्च‚य उक्त‚स्तं व्याच‚ष्टे । \textbf{त‚त्रे}त्यादि । \textbf{विरुद्ध‚त‚त्प‚क्ष्याणा}मिति विरु‚{\tiny $_{३}$}‚द्धानां‚{\tiny $_{lb}$}‚ विरुद्ध‚प‚क्ष्याणां च । विरुद्ध‚प‚क्ष्या येषां स‚त्त्वं विप‚क्षे निश्चितं स‚प‚क्ष‚स‚त्त्वं स‚न्दिग्धं ।‚{\tiny $_{lb}$}‚ स‚न्दिग्धानैकान्तिका एवैते प‚र‚प्र‚सिद्ध्या त्वेव‚म‚भिधानं । \textbf{व्य‚तिरेक‚स्यानिश्च‚ये}नेति‚{\tiny $_{lb}$}‚ प्र‚कृतेन स‚म्ब‚न्धः । अनैकान्तिक‚स्य निरास इति स‚म्ब‚न्धः । तेनाय‚म‚र्थः साधार‚णा‚{\tiny $_{lb}$}‚नैकान्तिक‚स्येति । \textbf{त‚त्प‚क्ष}स्य च । अनैकान्तिक‚प‚क्ष‚स्य च । \textbf{शेष‚व‚दादेः} । आदि‚{\tiny $_{lb}$}‚श‚ब्दात् स‚प‚क्ष‚वि‚{\tiny $_{४}$}‚प‚क्ष‚योस्स‚न्दिग्ध‚स्य निरासः । य‚स्य स‚प‚क्ष‚स‚त्त्वं विप‚क्षे चाद‚{\tiny $_{lb}$}‚र्श‚न‚मात्राद् व्य‚तिरेक‚स्त‚च्छेष‚व‚त् ।
	{\color{gray}{\rmlatinfont\textsuperscript{§~\theparCount}}}
	\pend% ending standard par
      ‚{\tiny $_{lb}$}‚

	  
	  \pstart \leavevmode% starting standard par
	\textbf{प्र‚सिद्ध‚स्तु द्व‚यो}रित्या चा र्य ग्र‚न्थ‚मिदानीं व्याच‚ष्टे । \textbf{द्व‚यो}रित्यादि । द्व‚योरित्य‚{\tiny $_{lb}$}‚\textbf{नेनैक}स्य वादिनः प्र‚तिवादिनो वा यः \textbf{सिद्ध‚स्त‚स्य‚प्र‚तिषे}धः [।] \textbf{प्र‚सिद्ध‚व‚च‚नेन} स‚न्दि‚{\tiny $_{lb}$}‚ग्ध‚योः \textbf{शेष‚व‚द‚साधार‚ण‚योः} प्र‚तिषेध इति स‚म्ब‚न्धः [।] क्व स‚न्दिग्ध‚योरित्याह ।‚{\tiny $_{lb}$}‚ \textbf{स‚प‚क्ष‚वि‚{\tiny $_{५}$}‚प‚क्ष‚योर‚पी}ति \textbf{शेष‚व‚तो}ऽस‚प‚क्ष‚स‚न्देहः । \textbf{असा}धार‚ण‚स्य तु स‚प‚क्ष‚विप‚क्ष‚योः ।‚{\tiny $_{lb}$}‚ त‚स्मान्निश्च‚य‚व‚च‚नादा चा र्ये णा पि प्र‚तिब‚न्ध इष्ट एव ॥
	{\color{gray}{\rmlatinfont\textsuperscript{§~\theparCount}}}
	\pend% ending standard par
      ‚{\tiny $_{lb}$}‚

	  
	  \pstart \leavevmode% starting standard par
	\textbf{अन्य‚थाऽस‚ति प्र‚तिब‚न्धे} साध्य‚साध‚न‚योः । \textbf{विप‚क्षेऽद‚र्श‚न‚मात्रेण व्य‚तिरेके}‚{\tiny $_{lb}$}‚ आचार्येणेष्य‚माणे । \textbf{व्य‚भिचारिविप‚क्षेणा}नैकान्तिक‚प्र‚तिप‚क्षेण \textbf{वैध‚र्म्य‚व‚च‚नं च य‚त्}‚{\tiny $_{lb}$}‚ प्र‚तिज्ञात‚न्त‚द‚पार्थ‚क‚मित्याकूतं ।
	{\color{gray}{\rmlatinfont\textsuperscript{§~\theparCount}}}
	\pend% ending standard par
      ‚{\tiny $_{lb}$}‚

	  
	  \pstart \leavevmode% starting standard par
	क्व पुन‚राचा‚{\tiny $_{६}$}‚र्येणं प्र‚तिज्ञात‚मित्याह । \textbf{य‚दे}त्यादि । न्या य मु खे चैत‚दुक्तं ।‚{\tiny $_{lb}$}‚ \textbf{य‚दुभ‚यं व‚क्त‚व्य}मिति साध‚र्म्यं वैध‚र्म्यं च । क‚स्य प्र‚तिप‚क्षेण किमुक्त‚मित्याह ।‚{\tiny $_{lb}$}‚ \textbf{विरुद्धे}त्यादि । साध‚न‚स्य व‚च‚नं \textbf{विरुद्ध‚प्र‚तिप‚क्षे}ण वैध‚र्म्य‚व‚च‚न\textbf{म‚नैकान्तिक‚प्र‚तिप‚क्षेण} ।‚{\tiny $_{lb}$}‚ य‚द्य‚दृष्टिफ‚ल‚न्त‚च्च । अद‚र्श‚न‚मात्र‚फ‚ल‚न्त‚च्चेति \textbf{वैध‚र्म्य‚व‚च‚नं} ।
	{\color{gray}{\rmlatinfont\textsuperscript{§~\theparCount}}}
	\pend% ending standard par
      ‚{\tiny $_{lb}$}‚‚{\tiny $_{lb}$}‚\textsuperscript{\textenglish{59/s}}

	  
	  \pstart \leavevmode% starting standard par
	एत‚देव व्याच‚ष्टे । \textbf{य‚दी}त्यादि । तेनेति \textbf{वैध‚र्म्य‚व‚च‚ने‚{\tiny $_{७}$}‚न विप‚क्षे} हेतोर\textbf{द‚र्श‚नं \leavevmode\ledsidenote{\textenglish{23b/PSVTa}}‚{\tiny $_{lb}$}‚ ख्याप्य‚ते । त‚दि}त्य‚द‚र्श‚न‚म\textbf{नुक्तेपि} वैध‚र्म्ये \textbf{ग‚म्य‚ते} । द‚र्श‚नाभाव‚ल‚क्ष‚ण‚स्याद‚र्श‚न‚स्य‚{\tiny $_{lb}$}‚ वैध‚र्म्य‚व‚च‚नात् प्राग‚पि सिद्ध‚त्वात् । त‚स्माद‚पार्थ‚क‚म्वैध‚र्म्य‚व‚च‚नं ।
	{\color{gray}{\rmlatinfont\textsuperscript{§~\theparCount}}}
	\pend% ending standard par
      ‚{\tiny $_{lb}$}‚

	  
	  \pstart \leavevmode% starting standard par
	हेतोर्द‚र्श‚न‚भ्रान्तिर्विप‚क्षेस्ति त‚न्निवृत्य‚र्थ‚म्वैध‚र्म्य‚व‚च‚न‚मिति चेदाह । \textbf{न ही}‚{\tiny $_{lb}$}‚त्यादि । \textbf{त‚स्ये}ति प्र‚तिपाद्य‚स्य वैध‚र्म्य‚व‚च‚नात् \textbf{प्राक्} । हेतोर्विप‚क्षे \textbf{द‚र्श‚न‚भ्रान्ति}‚{\tiny $_{lb}$}‚र‚स्ति \textbf{या} वैध‚र्म्य\textbf{व‚च‚नेन निव‚र्त्त्य‚ते} ॥ त‚स्मादेत‚द् य‚था‚{\tiny $_{१}$}‚ स‚प‚क्षे हेतोर्द‚र्श‚ने न भ्रान्तिः‚{\tiny $_{lb}$}‚ किन्तु त‚द्विस्मृत‚मिति साध‚र्म्य‚व‚च‚नेन स्म‚र्य‚ते । त‚था \textbf{स्मृतिर्वाचा} वैध‚र्म्य‚व‚च‚ने‚{\tiny $_{lb}$}‚ना\textbf{द‚र्श‚ने क्रिय‚त इति चेत्} । दृष्टान्त‚मेव विघ‚ट‚यितुमाह । \textbf{द‚र्श‚न}मित्यादि ।‚{\tiny $_{lb}$}‚ \textbf{द‚र्श‚न‚म‚प्र‚तीय‚मान}म‚स्म‚र्य‚माणं \textbf{न} साध्य‚प्र‚तिप‚त्त्य\textbf{ङ्ग‚मिति युक्त‚न्त‚त्रेति} द‚र्श‚ने \textbf{वाचा‚{\tiny $_{lb}$}‚ स्म‚र‚णाधानं} स्मृतिज‚न‚नं । अद‚र्श‚नार्थ‚न्तु न युक्तं । य‚स्माद\textbf{द‚र्श‚न‚न्तु द‚र्श‚नाभावो}‚{\tiny $_{lb}$}‚ हेतो‚{\tiny $_{२}$}‚र्विप‚क्षे । \textbf{स द‚र्श‚नेन बाध्य‚ते । त‚द‚भावे} विप‚क्ष‚द‚र्श‚नाभावे \textbf{सिद्ध एव} द‚र्श‚ना‚{\tiny $_{lb}$}‚भाव \textbf{इत्य‚पार्थ‚क‚न्त‚त्सिद्ध‚ये} द‚र्श‚नाभाव‚सिद्ध‚ये वैध‚र्म्य\textbf{व‚च‚नं} । अय‚म‚भिप्रायो द‚र्श‚{\tiny $_{lb}$}‚न‚म्प‚रोक्ष‚त्वाद् विस्म‚र्येतेति त‚त्स्म‚र‚णार्थ‚म्व‚च‚नं युक्तं । द‚र्श‚नाभाव‚स्तु द‚र्श‚न‚निवृ‚{\tiny $_{lb}$}‚त्तिरूपः स च द‚र्श‚नानुभ‚वाभावादेव प्र‚तिभास‚ते [।] प्र‚तिभास‚मान‚स्य च किं स्म‚र‚{\tiny $_{lb}$}‚णेन । \textbf{त‚त्सि}द्ध‚येऽपार्थ‚क‚म्व‚च‚नं ॥
	{\color{gray}{\rmlatinfont\textsuperscript{§~\theparCount}}}
	\pend% ending standard par
      ‚{\tiny $_{lb}$}‚

	  
	  \pstart \leavevmode% starting standard par
	न वै अनुप‚ल‚भ‚मान‚स्य पुंस‚स्ताव‚तेत्य‚द‚र्श‚न‚मात्रेण विप‚क्षे हेतुर्नास्तीत्येवं‚{\tiny $_{lb}$}‚ निश्च‚यो भ‚व‚ति देशादिविप्र‚कृष्टानाम‚नुप‚ल‚म्भेपि स‚त्त्वात् । त‚द‚र्थ‚न्नास्तीति‚{\tiny $_{lb}$}‚ निश्च‚योत्पाद‚नार्थ‚म्वैध‚र्म्य‚व‚च‚न‚मिति चेत्त‚न्न । य‚स्मान्न च नास्तीति व‚च‚नाद‚{\tiny $_{lb}$}‚प्र‚माण‚कात् त‚त्प्र‚तिक्षिप्य‚माणं नास्त्येव । क‚थ‚न्त‚र्हि नास्तीति ग‚म्य‚त इत्याह ।‚{\tiny $_{lb}$}‚ \textbf{य‚थे}त्यादि ।
	{\color{gray}{\rmlatinfont\textsuperscript{§~\theparCount}}}
	\pend% ending standard par
      ‚{\tiny $_{lb}$}‚

	  
	  \pstart \leavevmode% starting standard par
	एत‚दुक्त‚म्भ‚व‚ति । हेतोः स्व‚साध्ये प्र‚तिब‚न्ध‚ग्राह‚{\tiny $_{४}$}‚क‚मेव प्र‚माणं साध्याय‚त्त‚ता‚{\tiny $_{lb}$}‚ग्राह‚क‚न्त‚च्चेह द‚र्श‚न‚म‚भिप्रेतं य‚च्च साध्य एव हेतोर्द‚र्श‚न‚मिद‚मेव स‚र्व‚त्र विप‚क्षेऽद‚र्श‚नं ।‚{\tiny $_{lb}$}‚ तेन य‚स्य साध्य‚साध‚न‚योः प्र‚तिब‚न्ध‚ग्राह‚कं द‚र्श‚नं प्र‚वृत्त‚म्विस्मृतं च त‚स्यैव पुंसः‚{\tiny $_{lb}$}‚ द‚र्श‚नाद‚र्श‚न‚योः प्र‚तीत‚योः साध‚र्म्य‚वैध‚र्म्य‚व‚च‚नाभ्यां स्म‚र‚णं क्रिय‚ते नान्य‚स्येत्य‚र्थः ।‚{\tiny $_{lb}$}‚ य‚दाह [।] प्र‚माणं दृष्टान्ताभ्यामुप‚द‚र्श्य‚त इति । य‚था य‚दि नास्ति स ‚{\tiny $_{५}$}‚ख्याप्य‚त‚{\tiny $_{lb}$}‚ ‚{\tiny $_{lb}$}‚ \leavevmode\ledsidenote{\textenglish{60/s}}इति । य‚था येन प्र‚कारेण स्व‚साध्य‚प्र‚तिब‚न्धेन विप‚क्षे हेतुर्नास्ति य‚दि स न्याय‚{\tiny $_{lb}$}‚ इति प्र‚तिब‚न्ध‚ग्राह‚कं प्र‚माणं ख्याप्य‚ते स्म‚र्य‚ते त‚दा नास्तीति ग‚म्य‚ते । न तु प्र‚ति‚{\tiny $_{lb}$}‚ब‚न्ध‚म‚न्त‚रेण [।]
	{\color{gray}{\rmlatinfont\textsuperscript{§~\theparCount}}}
	\pend% ending standard par
      ‚{\tiny $_{lb}$}‚

	  
	  \pstart \leavevmode% starting standard par
	\textbf{य‚दी}त्यादि प्र‚थ‚मं कारिकाभाग‚माह । \textbf{य‚द्य‚नुप‚ल‚भ‚मानोनुप‚ल‚म्भ‚मात्रान्नास्तीति}‚{\tiny $_{lb}$}‚ न प्र‚त्येति त‚दा वैध‚र्म्य‚व‚च‚नाद‚प्य‚प्र‚माण‚कान्न प्र‚त्येष्य‚ति । य‚स्मात्त‚द‚पि वैध‚र्म्य‚व‚च‚नं ।‚{\tiny $_{lb}$}‚ त्व‚{\tiny $_{६}$}‚ता \edtext{}{\lemma{ता}\Bfootnote{?}} \textbf{तेना}नुप‚ल‚म्भ‚मेव \textbf{ख्याप‚य}ति नाधिक‚म्विशेषं ।
	{\color{gray}{\rmlatinfont\textsuperscript{§~\theparCount}}}
	\pend% ending standard par
      ‚{\tiny $_{lb}$}‚

	  
	  \pstart \leavevmode% starting standard par
	स्यान्म‚तं । साध्याभाव‚कृतो हेत्व‚भावो वैध‚र्म्य‚व‚च‚नेन ख्याप्य‚ते । त‚तोस्ति‚{\tiny $_{lb}$}‚ विशेष इत्य‚त आह । \textbf{न चैके}त्यादि । \textbf{एकानुप‚ल‚म्भ} इति साध्यानुप‚ल‚म्भः । \textbf{अन्या‚{\tiny $_{lb}$}‚भावं} साध‚नाभावं । अस‚ति प्र‚तिब‚न्ध इति भावः । \textbf{अतिप्र‚संगादि}ति प्र‚तिब‚न्ध‚{\tiny $_{lb}$}‚म‚न्त‚रेण निवृत्तौ गोनिवृत्त्याप्य‚श्व‚स्य निय‚मेन निवृत्तिः स्यात् ।
	{\color{gray}{\rmlatinfont\textsuperscript{§~\theparCount}}}
	\pend% ending standard par
      ‚{\tiny $_{lb}$}‚

	  
	  \pstart \leavevmode% starting standard par
	\leavevmode\ledsidenote{\textenglish{24a/PSVTa}} अथ म‚त‚म् [।] आचार्य‚दि‚{\tiny $_{७}$}‚ ग्ना गे न विप‚क्षे हेतुर्न्नास्ती त्युक्त‚म‚त एव निश्च‚य‚{\tiny $_{lb}$}‚ इत्य‚त आह । \textbf{न चे}त्यादि । \textbf{तेना} चा र्ये ण \textbf{नास्तीति} य‚द् वैध‚र्म्य\textbf{व‚च‚नं} कृतं त\textbf{स्मात्त‚था‚{\tiny $_{lb}$}‚ भ‚व‚ति} व‚स्तुनो नास्तित्व‚मेव भ‚व‚त्य\textbf{तिप्र‚स‚ङ्गात्} । त‚द्व‚च‚न‚स्य हि प्रामाण्ये प्र‚तिज्ञा‚{\tiny $_{lb}$}‚मात्राद‚पि साध्य‚सिद्धिः स्यात् । \textbf{त‚दि}ति त‚स्मात् \textbf{क‚थ‚म्वैध‚र्म्य‚व‚च‚नेनानैकान्तिक‚प‚रि‚{\tiny $_{lb}$}‚हारः} [।] नैव । \textbf{त‚स्माद्} विप‚क्षाद्धेतो\textbf{र्व्यावृत्तिमिच्छ}ता त‚त्र व्यावृत्तौ \textbf{न्यायो व‚क्त‚व्यः} ।‚{\tiny $_{lb}$}‚ सा‚{\tiny $_{१}$}‚ध्य‚साध‚न‚योः प्र‚तिब‚न्ध‚ग्राह‚कं प्र‚माणं \textbf{य‚त} इति न्यायात् । \textbf{अस्ये}ति प्र‚तिपाद्य‚स्य ।‚{\tiny $_{lb}$}‚ साध‚नं \textbf{व्यावृत्तिमिति निश्च‚यो भ‚व‚ति} ॥
	{\color{gray}{\rmlatinfont\textsuperscript{§~\theparCount}}}
	\pend% ending standard par
      ‚{\tiny $_{lb}$}‚

	  
	  \pstart \leavevmode% starting standard par
	\textbf{न‚न्}वित्यादि प‚रः । \textbf{त‚द‚भावे} साध्याभावेऽ\textbf{नुप‚ल‚म्भात्\textbf{अ}सिद्धा व्यावृ}त्तिर्हेतो‚{\tiny $_{lb}$}‚र‚य‚मेव न्याय इति म‚न्य‚ते । \textbf{य‚द्य‚दृष्ट्या} हेतोर्विप‚क्षा\textbf{न्निवृत्तिः स्यात्त}दा \textbf{शेष‚व‚द}नुमानं‚{\tiny $_{lb}$}‚ \textbf{व्य‚भिचारि किं} । नैव व्य‚भिचारि स्यात् [।] कीदृशं पुन‚स्त‚दित्याह । य‚थेत्यादि ।‚{\tiny $_{lb}$}‚ ‚{\tiny $_{lb}$}‚ \leavevmode\ledsidenote{\textenglish{61/s}}\textbf{एतानि फ‚ला‚{\tiny $_{२}$}‚नी}त्युप‚युक्ताद‚न्यानि । अयं च ध‚र्मिनिर्द्देशः । \textbf{प‚क्वान्येवं र‚सानि} चेति‚{\tiny $_{lb}$}‚ साध्य‚ध‚र्मः । एवं-र‚सानि म‚धुराण्य‚म्लानि वा । \textbf{रूपाविशेषादि}ति हेतुः । उप‚{\tiny $_{lb}$}‚युक्त‚स्य फ‚ल‚स्य य‚द्रूपं र‚क्त‚तादि । तेन तुल्य‚त्वात् । \textbf{एक‚शाखाप्र‚भ‚वाद्वे}ति हेतुः ।‚{\tiny $_{lb}$}‚ \textbf{उप‚युक्त‚व‚दि}ति दृष्टान्तः । क‚स्माद‚द‚र्श‚न‚मात्राद् व्य‚तिरेक इष्य‚माणे शेष‚व‚तो‚{\tiny $_{lb}$}‚व्य‚भिचारित्वं स्यादित्याह । \textbf{अत्रापी}त्यादि । अत्र शेष‚व‚द‚{\tiny $_{३}$}‚नुमाने \textbf{विव‚क्षितं}‚{\tiny $_{lb}$}‚ रूप‚विशेषादियुक्त‚मुप‚भुक्ताद‚न्य‚त्फ‚ल‚न्त‚स्या\textbf{शेष‚स्य} प\textbf{क्षीक‚र}णे । \textbf{हेतो} रूपादिशेषा‚{\tiny $_{lb}$}‚दिक‚स्य \textbf{साध्याभावेनुप‚लंभोस्तीति} विप‚क्षाव्यावृत्तिर‚त‚श्चा\textbf{व्य‚भिचारिता} स्यात् ।‚{\tiny $_{lb}$}‚ त‚था हि य‚त् प‚क्षीकृतं फ‚लं य‚च्चोप‚युक्तं स‚प‚क्ष‚त्वेनोपात्त‚न्त‚द्व्य‚तिरेकेण विप‚क्ष‚भूते‚{\tiny $_{lb}$}‚ तृतीये राशौ नास्ति य‚थोक्त‚स्य हेतोर्वृत्तिः ।
	{\color{gray}{\rmlatinfont\textsuperscript{§~\theparCount}}}
	\pend% ending standard par
      ‚{\tiny $_{lb}$}‚

	  
	  \pstart \leavevmode% starting standard par
	\textbf{प्र‚त्य‚क्ष‚बाधे}त्यादि । क‚दाचिदेक‚शाखाप्र‚भ‚व‚स्यापि फ‚{\tiny $_{४}$}‚ल‚स्याप‚क्व‚स्यात‚द्र‚{\tiny $_{lb}$}‚स‚स्य वा प्र‚त्य‚क्षेणानुभ‚व‚स‚म्भ‚वात् \textbf{प्र‚त्य‚क्ष‚बाधाश‚ङ्का एव} शेष‚व‚तो \textbf{व्य‚भि‚{\tiny $_{lb}$}‚चार इत्येके} ई श्व र से न प्र‚भृत‚यः । त‚द‚य‚म‚र्थो न केव‚लाभ्याम‚न्व‚य‚व्य‚तिरेकाभ्यां‚{\tiny $_{lb}$}‚ हेतुर्ग‚म‚कः, अपि त्व‚बाधित‚विष‚य‚त्वे स‚तीति । नेत्यादिना प‚रिह‚र‚ति । \textbf{प‚क्षीकृतो}‚{\tiny $_{lb}$}‚ यो\textbf{विष}यः प‚रोक्ष‚स्त‚त्र प्र‚त्य‚क्ष‚बाधाया \textbf{अभावात्} । न प्र‚वृत्तेन प्र‚त्य‚क्षेण बाधाश‚ङ्का‚{\tiny $_{lb}$}‚ किन्तु क‚दाचिद् ग‚{\tiny $_{५}$}‚न्ध‚प्र‚त्य‚क्ष‚बाधाप‚क्षं इति चेदाह । \textbf{त‚थे}त्यादि । \textbf{त‚था} क‚दाचित्‚{\tiny $_{lb}$}‚ प्र‚त्य‚क्ष‚बाधा भ‚वेदित्याश‚ङ्कायाम‚तिप्र‚संगः । य‚स्मा\textbf{द‚न्य‚त्रा}भिम‚ते हेतौ प्र‚त्य‚क्ष‚{\tiny $_{lb}$}‚बाधाया \textbf{अभाव‚निय‚माभावात्} । न हि स‚म्ब‚न्धाभ्युप‚ग‚मे प‚र‚स्य बाधाश‚ङ्का निव‚र्त्त‚त‚{\tiny $_{lb}$}‚ इति भावः । त‚स्मात् प्र‚तिब‚न्धान‚भ्युप‚ग‚म‚वादिना \textbf{वृत्तं प्र‚माणं बाध‚क‚मे}ष्ट‚व्यं ।‚{\tiny $_{lb}$}‚ \textbf{अवृत्त‚बाध‚ने}ऽप्र‚वृत्तेनैव प्र‚माणेन बाध‚ने \textbf{स‚र्व‚त्रानाश्वा‚{\tiny $_{६}$}‚सः} स‚र्व‚त्र हेतौ न स्यादाश्वासो‚{\tiny $_{lb}$}‚ ग‚म‚क‚त्व‚निश्च‚यः । बाध‚क‚स्य शंक्य‚मान‚त्वात् । नैवं प्र‚तिब‚न्ध‚वादिनः स‚र्व‚त्र हेताव‚{\tiny $_{lb}$}‚नाश्वासः साध्य‚प्र‚तिब‚द्धे हेतौ बाधाशंकाया अप्य‚भावात् । हेतुप्र‚योगात्तु पूर्वं स्याद्‚{\tiny $_{lb}$}‚ बाधाशंका [।] अत एव स‚न्दिग्धे हेतुव‚च‚न‚मुच्य‚ते [।] न च वृत्तं प्र‚माणं शेष‚व‚तो‚{\tiny $_{lb}$}‚ बाध‚क‚म‚स्ति । त‚स्मात् स्थित‚मेत‚द् अद‚र्श‚न‚मात्राद् व्य‚तिरेके शेष‚व‚तो‚{\tiny $_{७}$}‚प्य‚व्य‚भि- \leavevmode\ledsidenote{\textenglish{24b/PSVTa}}‚{\tiny $_{lb}$}‚ ‚{\tiny $_{lb}$}‚ \leavevmode\ledsidenote{\textenglish{62/s}}चारित्वं स्यादिति ॥
	{\color{gray}{\rmlatinfont\textsuperscript{§~\theparCount}}}
	\pend% ending standard par
      ‚{\tiny $_{lb}$}‚

	  
	  \pstart \leavevmode% starting standard par
	न‚नु प्र‚तिब‚न्ध‚ब‚लात् साध्याभावे हेतोर्व्य‚तिरेके स‚ति स‚त्तामात्रेण व्य‚तिरेको‚{\tiny $_{lb}$}‚ ग‚म‚कः स्यादित्य‚त आह । \textbf{व्य‚तिरेक‚स्त्वित्यादि} । हेतोर्यो विप‚क्षाद् \textbf{व्य‚तिरेकः}‚{\tiny $_{lb}$}‚ स \textbf{सिद्ध एव} निश्चित एव \textbf{साध‚नं । इति} हेतोस्त\textbf{थाभाव‚निश्च‚यं} साध्याभावे यो‚{\tiny $_{lb}$}‚ हेत्व‚भाव‚स्त‚न्निश्च‚य\textbf{म‚पेक्ष‚ते} । एत‚दाह [।] नास्माक‚म्भ‚व‚तामिव द‚र्श‚नाभाव‚मात्राद्‚{\tiny $_{lb}$}‚ व्य‚तिरेकः । किन्तु साध्य‚सा‚{\tiny $_{१}$}‚ध‚न‚योः स‚ति प्र‚तिब‚न्धे साध्याभावे हेतोर्विप‚क्षा‚{\tiny $_{lb}$}‚द्व्यावृत्त‚त्वं स्व‚ग‚तो ध‚र्मः । हेतोश्च य‚द्रूप‚न्त‚द‚व‚श्यं स्व‚निश्च‚य‚म‚पेक्ष‚ते ज्ञाप‚क‚त्वादिति ।‚{\tiny $_{lb}$}‚ क‚थं त्व‚न्म‚तेन शेष‚व‚तो व्य‚भिचारित्व‚मिति चेदाह । \textbf{अनुप‚ल‚म्भा}त्त्वित्यादि ।‚{\tiny $_{lb}$}‚ \textbf{क्व‚चि}द्विप‚क्षैक‚देशे हेतो\textbf{र‚भाव‚सिद्धाव‚प्य‚प्र‚तिब‚द्ध‚स्य} हेतोः साध्ये । \textbf{त‚द‚भावे} साध्याभावे‚{\tiny $_{lb}$}‚ \textbf{स‚र्व‚त्र} विप‚क्षेऽ\textbf{भावासिद्धेः} कार‚णात् \textbf{संश‚य}स्त‚तो वि‚{\tiny $_{२}$}‚प‚क्षा\textbf{द‚व्य‚तिरेको} यः स \textbf{एव‚{\tiny $_{lb}$}‚ व्य‚भिचारः शेष‚व‚तः} ॥
	{\color{gray}{\rmlatinfont\textsuperscript{§~\theparCount}}}
	\pend% ending standard par
      ‚{\tiny $_{lb}$}‚

	  
	  \pstart \leavevmode% starting standard par
	\textbf{किंचे}ति दोषान्त‚र‚स‚मुच्च‚यः । य‚द्य‚द‚र्श‚नाद् व्य‚तिरेक‚स्त‚दा \textbf{व्य‚तिरेक्य‚पि हेतुः‚{\tiny $_{lb}$}‚ स्यात्} । कीदृशं \textbf{नेदं निरात्म‚क}मित्यादि । प्र‚स‚ङ्ग‚मुखेन चेद‚मुक्त‚म\textbf{प्राणादिम‚त्त्व‚{\tiny $_{lb}$}‚प्र‚स‚ङ्गात्} । प्राणादिम‚त्वाच्च सात्म‚कं । अयं च हेतुः सात्म‚के क्व‚चिन्न दृष्टो‚{\tiny $_{lb}$}‚निरात्म‚केभ्य‚श्च व्यावृत्त इति साध्य‚निवृत्तौ निवृत्तिध‚र्माव्य‚तिरे‚{\tiny $_{३}$}‚की क‚थ्य‚ते ।‚{\tiny $_{lb}$}‚ त‚था हि \textbf{निरात्म‚के}ष्वात्म‚र‚हितेषु \textbf{घ‚टादिषु दृष्टादृष्टेषु प्राणाद्य‚द‚र्श‚नात्} प्राणापानो‚{\tiny $_{lb}$}‚न्मेष‚निमेषाद‚र्श‚नात् । दृष्टेषु स्व‚भावानुप‚ल‚म्भेनैवाद‚र्श‚नं प्राणादीनामुप‚ल‚ब्धिल‚क्ष‚ण‚{\tiny $_{lb}$}‚प्राप्त‚त्वात् । अदृष्टेष्व‚पि त‚ज्जातीय‚त‚या । तेन नैरात्म्यं प्राणाद्य‚भावेन व्याप्तं ।‚{\tiny $_{lb}$}‚ जीव‚च्छ‚रीरे तु \textbf{त‚न्निवृ}त्त्या प्राणादिम‚त्त्वाभाव‚निवृत्त्या नैरात्म्य‚स्य निवृत्ते\textbf{रात्म‚ग‚तिः}‚{\tiny $_{lb}$}‚ स्या‚{\tiny $_{४}$}‚त् ॥ त‚व त‚र्हि क‚थं संश‚य‚हेतुरिति चेदाह । \textbf{अदृश्ये}त्यादि । अदृश्य‚स्यात्म‚नो‚{\tiny $_{lb}$}‚\textbf{नुप‚ल‚म्भाद्} घ‚टादिस्व\edtext{}{\lemma{टादिस्व}\Bfootnote{? ष्व}}\textbf{भावाप्र‚सिद्धौ घ‚टादी\textbf{ना}न्नैरात्म्यासिद्धेः} कार‚{\tiny $_{lb}$}‚णान्निरात्म‚कात् \textbf{प्राणादेर‚निवृत्तिः} । प्राणाद्य‚भावेन स‚न्दिग्ध‚स्य नैरात्म्य‚स्याप्य‚{\tiny $_{lb}$}‚‚{\tiny $_{lb}$}‚ \leavevmode\ledsidenote{\textenglish{63/s}}सिद्धिरिति याव‚त् ॥
	{\color{gray}{\rmlatinfont\textsuperscript{§~\theparCount}}}
	\pend% ending standard par
      ‚{\tiny $_{lb}$}‚

	  
	  \pstart \leavevmode% starting standard par
	\textbf{बौद्धेन} नैरात्म्य‚म‚भ्युग‚त‚म‚तो\textbf{भ्युप‚ग‚मा}न्निरात्म‚क‚त्वं घ‚टादेः \textbf{सिद्ध‚मिति चेत्} ।‚{\tiny $_{lb}$}‚ य‚दि बौद्धाभ्युप‚ग‚{\tiny $_{५}$}‚मः प्र‚माणं \textbf{क‚थ‚मिदानीमात्म‚सि}द्धिर्जीव‚च्छ‚रीरे । त‚द‚पि बौद्धेन‚{\tiny $_{lb}$}‚ निरात्म‚क‚मिष्टं ॥ जीव‚च्छ‚रीरे नैरात्म्याभ्युप‚ग‚मोऽप्र‚माण‚क‚स्त‚तोस्यात्मा साध्य‚त‚{\tiny $_{lb}$}‚ इति ॥ य‚द्येवं प‚र‚स्यापि जीव‚च्छ‚रीरा\textbf{द‚न्य‚स्या}पि घ‚टादे\textbf{र‚प्र‚माणिका क‚थं नैरात्म्य‚{\tiny $_{lb}$}‚सिद्धिः} । न हि बौद्ध‚स्याभ्युप‚ग‚मः क्व‚चित्प्र‚माणं क्व‚चिन्नेति । किं चाभ्यु\textbf{प‚ग‚मेन‚{\tiny $_{lb}$}‚ के}व‚लेन \textbf{सात्म‚कानात्म‚कौ विभ‚ज्य‚{\tiny $_{६}$}‚} घ‚टाद‚यः प‚रेणास्माभिश्चानात्म‚का अभ्यु‚{\tiny $_{lb}$}‚प‚ग‚ताः । तेनानात्म‚काः । जीव‚च्छ‚रीरं सात्म‚क‚म‚स्माभिर‚भ्युप‚ग‚त‚न्त्व‚या तु निरा‚{\tiny $_{lb}$}‚त्म‚क‚मेव‚म्विभ‚ज्य \textbf{त‚त्र} निरात्म‚केषु प्राणादीनाम\textbf{भा}वे-नात्म‚विष‚ये \textbf{ग‚म‚क‚त्वं क‚थ‚य‚ता}‚{\tiny $_{lb}$}‚ प‚रेणा\textbf{ग‚मिक‚त्व‚मात्म‚नि प्र‚तिप‚न्नं नानुमेय‚त्वं । त‚स्मा}दात्म‚नो घ‚टादाव\textbf{द‚र्श‚नेप्य}‚{\tiny $_{lb}$}‚दृश्य‚स्व‚भाव‚स्या\textbf{त्म‚नो निवृत्त्य‚सिद्धेर्नास्ति कुत‚श्चि}न्नि‚{\tiny $_{७}$}‚रात्म‚कात् प्राणादेर्न्निवृ- \leavevmode\ledsidenote{\textenglish{25a/PSVTa}}‚{\tiny $_{lb}$}‚ त्तिरित्य‚ग‚म‚क‚त्वं । एत‚त्ताव‚न्नैवात्म‚नः कुत‚श्चिन्निवृत्तिः सिद्धा । अभ्युप‚ग‚म्य‚{\tiny $_{lb}$}‚ तूच्य‚ते । त‚न्निवृत्ताव‚प्यात्म‚निवृत्ताव‚पि क्व‚चिदिति दृष्टे घ‚टादौ \textbf{निवृत्ताव‚पि प्राणा‚{\tiny $_{lb}$}‚दीना}म‚प्र‚तिब‚न्धादात्म‚ना स‚ह \textbf{स‚म्ब‚न्धाभावा}त् । \textbf{स‚र्व}त्रादृष्टेष्व‚पि घ‚टादिष्वात्म‚{\tiny $_{lb}$}‚निवृत्त्या प्राणादीनां \textbf{निवृत्त्य‚सिद्धेः} स‚न्दिग्ध‚व्य‚तिरेकित्वाद्न\textbf{ग‚म‚क‚त्वं} ।
	{\color{gray}{\rmlatinfont\textsuperscript{§~\theparCount}}}
	\pend% ending standard par
      ‚{\tiny $_{lb}$}‚

	  
	  \pstart \leavevmode% starting standard par
	अद‚र्श‚न‚मात्राद् व्य‚तिरेकाभ्युप‚ग‚मे स‚त्य‚य‚म‚प‚रो दोष इत्याह । \textbf{यापी}त्यादि ।‚{\tiny $_{lb}$}‚ का पुनः सेत्याह । \textbf{त‚था स‚प‚क्षे स‚न्नि}त्यादि । आ चा र्य स्य चाय‚ङ्ग्र‚न्थः । त‚त्र‚{\tiny $_{lb}$}‚ स‚न्दिह्युक्त‚म्प‚क्ष‚ध‚र्मो वादिप्र‚तिवादिनिश्चितो गृह्य‚ते । तेनोभ‚योर‚न्य‚त‚र‚स्य चासिद्ध‚स्य‚{\tiny $_{lb}$}‚ ग्ध‚स्याश्र‚यासिद्ध‚स्य च व्युदासः । य‚था च प‚क्ष‚ध‚र्म‚निश्च‚येन च‚तुर्विध‚स्यासिद्ध‚स्य‚{\tiny $_{lb}$}‚ व्युदास‚स्त‚था \textbf{स‚प‚क्षे स‚न्न‚{\tiny $_{२}$}‚स‚न्नित्येव‚मा}दिष्व‚प्य‚न्व‚य‚व्य‚तिरेक‚निश्च‚येन निर‚स्त‚{\tiny $_{lb}$}‚म‚सिद्ध‚जात‚म‚न्य‚त‚रासिद्धादीनां स‚प‚क्षादिष्व‚स‚म्भ‚वात् । \textbf{य‚थायोग‚मुदाहार्य‚मित्}याह ।‚{\tiny $_{lb}$}‚ \textbf{सापि न वाच्या} असिद्धियोज‚ना ।
	{\color{gray}{\rmlatinfont\textsuperscript{§~\theparCount}}}
	\pend% ending standard par
      ‚{\tiny $_{lb}$}‚‚{\tiny $_{lb}$}‚\textsuperscript{\textenglish{64/s}}

	  
	  \pstart \leavevmode% starting standard par
	त‚द्व\textbf{याच‚ष्टेऽनुप‚ल‚म्भ एवे}त्यादि । अप्र‚माण‚के\textbf{नुप‚ल‚म्भ एव स‚ति} हेतोर्विप‚क्षे‚{\tiny $_{lb}$}‚ संश‚यात् [।] क‚थ‚मुप‚ल‚म्भे त‚द‚स‚म्भ‚वात् । विप‚क्षे हेतो\textbf{रुप‚ल‚म्भे स‚ति} त‚स्य स‚ङ‚{\tiny $_{lb}$}‚श‚य‚स्या\textbf{भावात्} । त‚स्मा‚{\tiny $_{३}$}‚\textbf{द‚नुप‚ल‚म्भा}द्धेतोः विप‚क्षाद् \textbf{व्य‚तिरेक} इत्य‚र्थात् स‚न्दि‚{\tiny $_{lb}$}‚ग्ध‚व्य‚तिरेको हेतुरिष्ट एव । त‚स्मात् \textbf{संश‚यितोऽनिवार्यः} । संश‚येन विष‚यीकृतः‚{\tiny $_{lb}$}‚ संश‚यितो व्य‚तिरेको न वार्यः स्यात् ।
	{\color{gray}{\rmlatinfont\textsuperscript{§~\theparCount}}}
	\pend% ending standard par
      ‚{\tiny $_{lb}$}‚

	  
	  \pstart \leavevmode% starting standard par
	\textbf{य‚थायोग‚मुदाहार्य}मित्य‚तो \textbf{य‚थायोग‚व‚च‚नात्} संश‚यितो\textbf{ऽनिवारित एवेति चेत} ।‚{\tiny $_{lb}$}‚ नैत‚देवं । त‚द‚न‚न्त‚र‚मेव \textbf{य एव तूभ‚य‚निश्च‚य‚वाचीत्या}दि व‚च‚नात् । य एव श‚ब्द‚{\tiny $_{lb}$}‚ उ‚{\tiny $_{४}$}‚भ‚य‚निश्चित‚स्य त्रैरूप्य‚स्यासिद्ध‚त्वादेर्वाच‚कः स एव साध‚नं दूष‚णं न चान्य‚{\tiny $_{lb}$}‚त‚र‚प्र‚सिद्ध‚स‚न्दिग्ध‚वाची पुनः साध‚नापेक्ष‚णादित्य‚यं ग्र‚न्थ इहोदाह‚र‚णं । अन्य‚त‚र‚स्य‚{\tiny $_{lb}$}‚ वादिनः प्र‚तिवादिनो वा [।] योसिद्धं स‚न्दिग्ध‚म्वा व‚क्ति न स साध‚नं दूष‚णं वा [।]‚{\tiny $_{lb}$}‚ स‚न्दिग्ध‚व्य‚तिरेक‚निश्च‚य‚हेतुरुभ‚योर‚पि विप‚क्षे स‚न्दिग्ध‚स्त‚स्मात्त‚द‚भिधान‚म‚साध‚न‚म् ।‚{\tiny $_{lb}$}‚ य‚स्मादुभ‚य‚नि‚{\tiny $_{५}$}‚श्चित एव हेतुरुक्त‚स्ते\textbf{नानुप‚ल‚म्भेपि} स‚ति निवृत्ति\textbf{संश‚याद्} विप‚क्षा‚{\tiny $_{lb}$}‚द्धेतोर\textbf{निवृत्तिं म‚न्य‚मान‚स्त}स्य स‚न्दिग्ध‚व्य‚तिरेक‚स्य हेतुत्व\textbf{प्र‚तिषेध‚माह} असिद्धि‚{\tiny $_{lb}$}‚योज‚न‚या ॥
	{\color{gray}{\rmlatinfont\textsuperscript{§~\theparCount}}}
	\pend% ending standard par
      ‚{\tiny $_{lb}$}‚

	  
	  \pstart \leavevmode% starting standard par
	\textbf{किञ्च} । य‚द्य‚दृष्ट्या निवृत्तिरिष्य‚ते त‚दा \textbf{विशेष‚स्या}साधार‚ण‚स्य श्राव‚ण‚{\tiny $_{lb}$}‚त्वा\textbf{देर्व्य‚व‚च्छेद‚हेतुता स्यात्} । किं कार‚णं [।] नित्त्यानित्य‚यो\textbf{र‚द‚र्श‚नात्} । त‚द्‚{\tiny $_{lb}$}‚ व्याच‚ष्टे[।]\textbf{श्राव‚ण‚त्व‚स्यापी}त्यादि । त‚था ह्य‚द‚र्श‚{\tiny $_{६}$}‚न‚मात्राद् व्यावृत्तिरिष्टा।अस्ति‚{\tiny $_{lb}$}‚ च \textbf{नित्यानित्य‚योर‚द‚र्श‚नं । श्राव‚ण‚त्व‚स्य त‚द्व्य‚व‚च्छेद‚हेतुता स्यात्} । नित्यानित्य‚{\tiny $_{lb}$}‚प्र‚तिषेध‚हेतुत्व‚म्भ‚वेत् । य‚दि नामाद‚र्श‚नात् त‚तो व्याव‚र्त्तेत श्राव‚ण‚त्व‚न्त‚{\tiny $_{lb}$}‚द्व्य‚व‚च्छेद‚हेतुत्व‚न्तु कुत इत्याह । \textbf{न ही}ति । \textbf{न त‚द्व्यावृत्तेर‚न्य‚द् व्य‚व‚च्छेद‚न‚न्नित्या}‚{\tiny $_{lb}$}‚नित्य‚व्य‚व‚च्छेद‚न‚म्॥त‚वापि क‚थं श्राव‚ण‚त्व‚न्न व्य‚व‚च्छेद‚हेतुरिति चेदाह । \textbf{अव्य‚व‚च्छे-}‚{\tiny $_{lb}$}‚  \leavevmode\ledsidenote{\textenglish{65/s}}\textbf{द‚स्त्वि}‚{\tiny $_{७}$}‚त्यादि । \textbf{कुत‚श्चि}न्नित्याद‚नित्याच्चाद‚र्श‚न‚मात्रेण \textbf{व्यावृत्तेरेवानिश्च‚यात्} । \leavevmode\ledsidenote{\textenglish{25b/PSVTa}}‚{\tiny $_{lb}$}‚ अनिश्च‚य‚श्चान्य‚त‚र‚त्र प्र‚तिब‚न्धानिश्च‚यात् । अव‚श्यं चैत‚देव‚म‚न्य‚था यो हि ध‚र्मो‚{\tiny $_{lb}$}‚ \textbf{य‚त्र नास्तीति निश्चितः} स भ‚व‚न् क्व‚चिद् ध‚र्मिणि \textbf{क‚थ‚न्त‚द‚भावं य‚त्र} नास्तीति‚{\tiny $_{lb}$}‚ निश्चित‚स्त‚स्याभावं \textbf{क‚थं न ग‚म}येत् ॥
	{\color{gray}{\rmlatinfont\textsuperscript{§~\theparCount}}}
	\pend% ending standard par
      ‚{\tiny $_{lb}$}‚

	  
	  \pstart \leavevmode% starting standard par
	नित्यानित्याद् व्यावृत्त‚स्यापि श्राव‚ण‚त्व‚स्योभ‚य‚व्य‚व‚च्छेदे साध्ये \textbf{प्र‚माणान्त‚र‚{\tiny $_{lb}$}‚बाधा चे}च्छंक्येत । \textbf{अ‚{\tiny $_{१}$}‚थापी}त्याद्य‚स्यैव व्याख्यानं । श्राव‚ण‚त्वे\textbf{नोभ‚य‚व्य‚व‚च्छेदे}‚{\tiny $_{lb}$}‚ नित्यानित्य‚व्य‚व‚च्छेदे साध्ये \textbf{प्र‚माणान्त‚रं बाध‚क‚म‚स्ति} । त‚दाह । \textbf{अन्योन्येत्यादि} ।‚{\tiny $_{lb}$}‚ अन्योन्य\textbf{व्य‚व‚च्छेदो रूपं} येषान्ते त‚था । त‚था हि नित्य‚व्य‚व‚च्छेद एवानित्य‚त्व‚न्त‚द्व्य‚{\tiny $_{lb}$}‚व‚च्छेद एव च नित्य‚त्व‚न्त‚था भाव‚व्य‚व‚च्छेद एवाभावोऽभाव‚व्य‚व‚च्छेद एव च भावः ।‚{\tiny $_{lb}$}‚ \textbf{तेषामेक‚स्य व्य‚व‚च्छेदेनान्य‚विधानात्} । द्वितीय‚स्य विरुद्ध‚स्य‚{\tiny $_{२}$}‚विधाना\textbf{द‚प्र‚तिषेधः} ।
	{\color{gray}{\rmlatinfont\textsuperscript{§~\theparCount}}}
	\pend% ending standard par
      ‚{\tiny $_{lb}$}‚

	  
	  \pstart \leavevmode% starting standard par
	एत‚दुक्त‚म्भ‚व‚ति । य‚दा हि श्राव‚ण‚त्वं नित्याद् व्यावृत्त‚मिति त‚द् व्य‚व‚च्छिद्यात् ।‚{\tiny $_{lb}$}‚ त‚देवानित्य‚त्व‚म्विद‚ध्यात् । त‚देव च त‚द‚नित्याद् व्यावृत्त‚मित्य‚नित्यं व्य‚व‚च्छिद्यात् ।‚{\tiny $_{lb}$}‚ नित्यं च विध‚त्त इत्येक‚स्यैक‚दैव विधिप्र‚तिषेधौ स्यातां त‚च्चायुक्त‚मिति \textbf{विधिप्र‚तिषेध‚{\tiny $_{lb}$}‚यो}र्युग‚प‚द्वि\textbf{रोधान्न} क‚स्य‚चिद‚पि प्र‚तिषेधः । प्र‚तिषेध‚विनिवृत्तिल‚क्ष‚णो हि विधिः ।‚{\tiny $_{lb}$}‚ विधिनिवृत्ति‚{\tiny $_{३}$}‚रेव च प्र‚तिषेध‚स्तौ च प‚र‚स्प‚र‚विरुद्धौ युग‚प‚देक‚स्य क‚थं स्याताम्‚{\tiny $_{lb}$}‚ [।] अतो न कुत‚श्चिद‚पि व्यावृत्तिनिश्च‚यः श्राव‚ण‚त्वात् ।
	{\color{gray}{\rmlatinfont\textsuperscript{§~\theparCount}}}
	\pend% ending standard par
      ‚{\tiny $_{lb}$}‚

	  
	  \pstart \leavevmode% starting standard par
	\textbf{नेदानी}मिति सिद्धान्त‚वादी । इदानीमिति बाधास‚म्भ‚वे स‚ति । \textbf{अदृशोऽ}‚{\tiny $_{lb}$}‚द‚र्श‚नाद् विप‚क्षे हेतो\textbf{र्नास्तिता} । त‚स्मा\textbf{देवं स‚त्य‚द‚र्श‚न‚न्न प्र‚माण‚म्बाधास‚म्भ‚वात्} ॥
	{\color{gray}{\rmlatinfont\textsuperscript{§~\theparCount}}}
	\pend% ending standard par
      ‚{\tiny $_{lb}$}‚

	  
	  \pstart \leavevmode% starting standard par
	श्राव‚ण‚त्व एवाप्र‚माणं भ‚व‚तु नान्य‚त्र बाधाऽभावादिति चेदाह । \textbf{त‚थेत्या}दि ।‚{\tiny $_{lb}$}‚ अन्य‚{\tiny $_{४}$}‚त्रापि हेतोर्व्य‚तिरेक‚साध‚न‚स्याद‚र्श‚न‚स्य \textbf{स‚म्भाव्यं प्र‚माणान्त‚र‚बाध‚नं} । कुतः ।‚{\tiny $_{lb}$}‚ \textbf{ल‚क्ष‚णे}त्यादि । हेतोर्विप‚क्षाद‚र्श‚न‚व्यावृत्तिनिब‚न्ध‚न‚मिति य‚ल्ल‚क्ष‚ण‚न्तेन युक्तं श्राव‚{\tiny $_{lb}$}‚‚{\tiny $_{lb}$}‚ \leavevmode\ledsidenote{\textenglish{66/s}}ण‚त्व‚स्य य‚द‚द‚र्श‚न‚न्त‚स्मि\textbf{न्बाधास‚म्भ‚वे} स‚ति \textbf{त‚ल्ल‚क्ष‚ण‚मेव} त‚स्य व्य‚तिरेक‚साध‚न‚{\tiny $_{lb}$}‚स्याद‚र्श‚न‚स्य स‚र्व‚विष‚य‚मेव ल‚क्ष‚णं स्व‚रूपं \textbf{दूषितं स्यादि}ति \textbf{स‚र्व‚त्रा}द‚र्श‚ने व्य‚तिरेक‚{\tiny $_{lb}$}‚साध‚ने‚{\tiny $_{५}$}‚\textbf{नाश्वासः} । न ग‚म‚क‚त्व‚निश्च‚यः ॥
	{\color{gray}{\rmlatinfont\textsuperscript{§~\theparCount}}}
	\pend% ending standard par
      ‚{\tiny $_{lb}$}‚

	  
	  \pstart \leavevmode% starting standard par
	\textbf{य‚द्येव‚म‚नुमान‚विष‚येपि} क्व‚चित् प्र‚त्य\textbf{क्षानुमान‚विरोध‚द‚र्श‚ना}त्[।]त‚था हि नित्यः‚{\tiny $_{lb}$}‚ श‚ब्दः श्राव‚ण‚त्वाच्छ‚ब्द‚त्व‚व‚दिति कृते नित्य‚त्व‚म‚नुमानेन बाध्य‚ते । एव‚म‚श्राव‚णः‚{\tiny $_{lb}$}‚ श‚ब्दः स‚त्त्वाद् घ‚ट‚व‚दिति प्र‚त्य‚क्षेण । त‚त‚श्च स‚र्व‚त्र त‚द‚व‚शिष्ट‚ल‚क्ष‚णेनानुमाने‚{\tiny $_{lb}$}‚प्य‚ना\textbf{श्वास‚प्र‚संग इति चे}त् ।
	{\color{gray}{\rmlatinfont\textsuperscript{§~\theparCount}}}
	\pend% ending standard par
      ‚{\tiny $_{lb}$}‚

	  
	  \pstart \leavevmode% starting standard par
	\textbf{नैत‚देवं । य‚थोक्त} इति कार्य‚स्व‚भा‚{\tiny $_{६}$}‚वानुप‚ल‚म्भ‚जेङ्ग‚जेनुमाने प्र‚त्य‚क्षानुमान‚{\tiny $_{lb}$}‚विरोध‚स्या\textbf{भावा}त् । प्र‚त्य‚क्षादिविरोध\textbf{स‚म्भ‚विन‚श्चात‚ल्ल‚क्ष‚ण‚त्वात्} त‚द‚नुमानाल‚क्ष‚ण‚{\tiny $_{lb}$}‚त्वात् ॥ \textbf{य‚दि} य‚थोक्त‚ल‚क्ष‚णेऽनुमाने नास्ति बाधा त‚दा हेतुल‚क्ष‚ण‚युक्तं प‚र‚स्प‚र‚{\tiny $_{lb}$}‚विरुद्धार्थ‚साध‚कं हेतुद्व‚य‚मेक‚स्मिन् ध‚र्मिण्य‚व‚तीर्ण्ण‚म्विरुद्धाव्य‚भिचार्युक्त‚मा चा र्ये‚{\tiny $_{lb}$}‚ \leavevmode\ledsidenote{\textenglish{26a/PSVTa}} ण त‚स्याव‚च‚{\tiny $_{७}$}‚न‚मिति चेत् । अनेनाभ्युपेत‚हानिमाह ॥ \textbf{अनुमान‚विष‚ये} विरुद्ध‚व्य‚भि‚{\tiny $_{lb}$}‚चार्य\textbf{व‚च‚नादिष्ट‚मेवे}ति कुतोभ्युपेत‚हानं । क्व त‚र्ह्याचार्येणोक्त इत्याह । \textbf{विष‚यं}‚{\tiny $_{lb}$}‚ चेत्यादि । अस्य विरुद्धाव्य‚भिचारिणः ॥ \textbf{किंचे}त्यादि । इह \textbf{वै शे षि} के ण वायोः‚{\tiny $_{lb}$}‚ स‚त्त्व‚साध‚नार्थं \textbf{स्प‚र्श‚श्च} न च दृष्टानामिति सूत्र\edtext{}{\edlabel{pvsvt_66-2}\label{pvsvt_66-2}\lemma{सूत्र}\Bfootnote{\href{http://sarit.indology.info/?cref=vs\%C5\%AB.2.1.9-10}{ Vaiśeṣikasūtra 2:9. }}}मुक्तं । अस्याय‚म‚र्थः [।]‚{\tiny $_{lb}$}‚ यो गुणः स द्र‚व्याश्र‚यी त‚द्य‚था रूपादिः । अपाक‚{\tiny $_{१}$}‚जानुष्णाशीत‚स्प‚र्श‚श्च गुण‚स्त‚{\tiny $_{lb}$}‚स्मात्त‚स्याश्र‚य‚भूतेन द्र‚व्येण भ‚वित‚व्यं । न चायं दृष्टानां पृथिव्यादीनां गुण‚स्तेषां‚{\tiny $_{lb}$}‚ पाक‚जानुष्णाशीत‚स्प‚र्शादिगुण‚त्वात् । त‚तो य‚स्यायं गुणः स वायुर्भ‚विष्य‚तीत्युक्ते‚{\tiny $_{lb}$}‚ वैशेषिकेण । त‚त्राचार्य दि ङ् ना गे नोक्तं [।] य‚द्य‚द‚र्श‚न‚मात्रेण दृष्टेभ्यः प्र‚तिषेधः‚{\tiny $_{lb}$}‚ क्रिय‚ते न च सोपि युक्त इति [।] य‚देत‚दुक्त‚न्त‚द् विरुध्य‚त इति वा र्त्ति‚{\tiny $_{२}$}‚ क का रो‚{\tiny $_{lb}$}‚ द‚र्श‚य‚न्नाह । \textbf{द‚ष्टे}त्यादि । य‚द्य‚दृष्ट्या निवृत्तिः \textbf{स्यात्} त‚थाऽदृष्टेर‚द‚र्श‚नात् कार‚णाद्‚{\tiny $_{lb}$}‚  \leavevmode\ledsidenote{\textenglish{67/s}}अपाक‚ज‚स्यानुष्णाशीत\textbf{स्प‚र्श‚स्य दृष्टाऽयुक्तिः} । दृष्टेषु पृथिव्यादिष्व‚स‚ङ्ग‚तिर्या‚{\tiny $_{lb}$}‚ व‚र्ण्णिता वैशेषिकैर्य‚स्या आचार्येणायुक्त‚त्व‚मुक्तं सा स्याद् \textbf{अविरोधिनी} युक्तैव‚{\tiny $_{lb}$}‚ स्यादित्य‚र्थः ।
	{\color{gray}{\rmlatinfont\textsuperscript{§~\theparCount}}}
	\pend% ending standard par
      ‚{\tiny $_{lb}$}‚

	  
	  \pstart \leavevmode% starting standard par
	त‚द्व्याच‚ष्टे । \textbf{य‚दी}त्यादि । य‚दाहाचार्यः । वायुप्र‚क‚र‚णे \textbf{य‚द्य‚द‚र्श‚न‚मात्रेण‚{\tiny $_{lb}$}‚ दृष्टेभ्यः} पृ‚{\tiny $_{३}$}‚थिव्यादिभ्यः स्प‚र्श‚स्य \textbf{प्र‚तिषेधः क्रि}य‚ते वै शे षि के ण \textbf{न च सोपि युक्त}‚{\tiny $_{lb}$}‚ इति । य‚द्याचार्य‚स्याप्य\textbf{द‚र्श‚न‚मात्रेण} व्य‚तिरेकोभिम‚त‚स्त‚दा \textbf{क‚थ‚म‚युक्तः} स्प‚र्श‚स्य‚{\tiny $_{lb}$}‚ प्र‚तिषेधो युक्त एव स्यात् । किं कार‚ण‚म् [।] अत्रापि हेतो\textbf{र‚नुप‚ल‚म्भाद‚भाव‚सिद्धेर}‚{\tiny $_{lb}$}‚भ्युप‚ग‚मात् ॥
	{\color{gray}{\rmlatinfont\textsuperscript{§~\theparCount}}}
	\pend% ending standard par
      ‚{\tiny $_{lb}$}‚

	  
	  \pstart \leavevmode% starting standard par
	\textbf{न‚न्वि}त्यादि प‚रः । उप‚ल‚ब्धिल‚क्ष‚ण‚प्राप्तः कार‚णाद् अपाक‚ज‚स्यानुष्णाशीत‚{\tiny $_{lb}$}‚\textbf{स्प‚र्श‚स्या}नुप‚ल‚{\tiny $_{४}$}‚भ्य‚मान‚स्य पृथिव्यादिषु \textbf{युक्त एव प्र‚तिषेधः} ।
	{\color{gray}{\rmlatinfont\textsuperscript{§~\theparCount}}}
	\pend% ending standard par
      ‚{\tiny $_{lb}$}‚

	  
	  \pstart \leavevmode% starting standard par
	\textbf{न युक्त} इति सिद्धान्त‚वादी । किं कार‚णं [।] यः पृथिव्यादिर‚नुष्णाशीताद‚न्येन‚{\tiny $_{lb}$}‚ स्प‚र्शेन युक्तः प्र‚त्य‚क्षः स दृश्यः । अन्य‚त्र च देशादाव‚प्र‚त्य‚क्षोपि त‚थाभूत‚स्प‚र्श‚युक्त‚{\tiny $_{lb}$}‚ एव पृथिव्यादिस्त‚त्स्व‚भावः । दृश्य‚श्च त‚त्स्व‚भाव‚श्चेति द्व‚न्द्वः । त‚देव \textbf{विष‚यः} स‚{\tiny $_{lb}$}‚ एव त‚न्मात्रं सुप्सुपेति स‚मासः । त‚स्मिन्न\textbf{प्र‚तिषेधात्} स्प‚र्श‚{\tiny $_{५}$}‚स्य । त‚त्र हि प्र‚तिषेधे‚{\tiny $_{lb}$}‚ स्यादेव \textbf{निश्च‚यः याव‚ता पृथिव्यादिध}र्मिभूतं \textbf{सामान्येना}विशेषेणैव \textbf{गृहीत्वाऽय‚म्वै}‚{\tiny $_{lb}$}‚शे षि कः स्प‚र्श‚स्य \textbf{प्र‚तिषेध‚माह} । य‚त्किंचित् पृथिव्यादि त‚त्स‚र्व‚म‚नुष्णाशीत‚स्प‚र्श‚{\tiny $_{lb}$}‚र‚हित‚मिति । न च पृथिव्यादिमात्रे स्प‚र्श‚स्योप‚ल‚ब्धिल‚क्ष‚ण‚प्राप्तिर‚स्ति । न चैक‚{\tiny $_{lb}$}‚स्प‚र्श‚निय‚मः पृथिव्यादीनां । य‚त‚स्त‚त्र पृथिव्यादिसामान्ये \textbf{तूलोप‚ल‚प‚ल्ल‚वा}‚{\tiny $_{६}$}‚दिषु‚{\tiny $_{lb}$}‚ भेदेषु \textbf{त‚द्भावेपि} पृथिव्यादिभावेपि \textbf{स्प‚र्श‚भेद‚द‚र्श‚ना}त् । त‚था हि तूल‚स्यान्यः स्प‚र्शः‚{\tiny $_{lb}$}‚ श्ल‚क्ष्ण‚त्व‚ल‚क्ष‚णोन्य‚श्चोप‚लादेः क‚र्क‚श‚त्वादिल‚क्ष‚णः । \textbf{अस्या}पीत्य‚नुष्णाशीत‚स्प‚र्श‚स्य‚{\tiny $_{lb}$}‚ \textbf{क्व‚चित् पार्थिव‚द्र‚व्य‚विशेषे स‚म्भ‚वाशंक‚या भ‚वित‚व्य‚मि}ति कृत्वा स‚र्व‚त्र पृथिव्या‚{\tiny $_{lb}$}‚  \leavevmode\ledsidenote{\textenglish{68/s}}\textbf{दाव‚द‚र्श‚न‚मात्रे}णानुष्णाशीत‚स्प‚र्श‚स्या\textbf{युक्तः प्र‚तिषेधः} ।
	{\color{gray}{\rmlatinfont\textsuperscript{§~\theparCount}}}
	\pend% ending standard par
      ‚{\tiny $_{lb}$}‚

	  
	  \pstart \leavevmode% starting standard par
	\leavevmode\ledsidenote{\textenglish{26b/PSVTa}} य‚त्तूच्य‚ते [।] स‚र्व‚त्र पृथिव्यादेस्तुल्य‚{\tiny $_{७}$}‚त्वात्तुल्य एवानुष्णाशीत‚स्प‚र्शो य‚स्तु‚{\tiny $_{lb}$}‚ पृथिव्यादिभावेपि श्ल‚क्ष्ण‚त्वादिभेदः स पृथिव्यादिप‚र‚माणुसंयोग‚स्य निविडानिविड‚{\tiny $_{lb}$}‚कृत‚त्वादिति ॥
	{\color{gray}{\rmlatinfont\textsuperscript{§~\theparCount}}}
	\pend% ending standard par
      ‚{\tiny $_{lb}$}‚

	  
	  \pstart \leavevmode% starting standard par
	त‚द‚युक्त‚म् [।] अनिविडानां हि नैर‚न्त‚र्याभावात् संयोगाभावः । तेन य‚थो‚{\tiny $_{lb}$}‚प‚ल‚प‚र‚माणूनां संयोग‚स्य निविड‚त्व‚न्त‚था तूल‚प‚र‚माणूनामिति क‚थं श्ल‚क्ष्ण‚त्वादिभेदः‚{\tiny $_{lb}$}‚ स्यात् । त‚स्माद् स‚त्य‚पि संयोगे स्व‚रूपेणान्यादृ‚{\tiny $_{१}$}‚शा एवोप‚ल‚प‚र‚माण‚वोन्यादृशाश्च‚{\tiny $_{lb}$}‚ तूल‚प‚र‚माण‚व‚स्त‚त्कृत एव च श्ल‚क्ष्ण‚त्वादिभेदः । न संयोग‚निविडानिविड‚कृतः ।‚{\tiny $_{lb}$}‚ त‚था च य‚था पृथिव्यादिभेदात् स्प‚र्श‚स्य श्ल‚क्ष्ण‚त्वादिभेद‚स्त‚थानुष्णाशीत‚भेदोपि‚{\tiny $_{lb}$}‚ स‚म्भाव्येत ॥
	{\color{gray}{\rmlatinfont\textsuperscript{§~\theparCount}}}
	\pend% ending standard par
      ‚{\tiny $_{lb}$}‚

	  
	  \pstart \leavevmode% starting standard par
	न‚नु माभूद् अदृष्ट‚विष‚ये वाय्व‚नुनं दृष्ट‚विष‚ये त्व‚नुष्णाशीत‚स्प‚र्श‚स्य दृश्य‚स्य‚{\tiny $_{lb}$}‚ प‚थिव्यादिस‚म्ब‚न्धित्वेनानुप‚ल‚भ्य‚मान‚त्वा‚{\tiny $_{२}$}‚त् त‚तो वाय्व‚नुमानं स्यात् ।
	{\color{gray}{\rmlatinfont\textsuperscript{§~\theparCount}}}
	\pend% ending standard par
      ‚{\tiny $_{lb}$}‚

	  
	  \pstart \leavevmode% starting standard par
	एव‚म्म‚न्य‚ते । य‚दि स्प‚र्शादेर्गुण‚रूप‚ता सिद्धा स्यात् त‚तो वायुद्र‚व्यानुमानं‚{\tiny $_{lb}$}‚ स्यात् सैव त्व‚सिद्धा । स्वात‚न्त्र्येण प्र‚तीतेः । स्प‚र्श‚विशेष एव चास्माक‚म्वायुरुच्य‚ते ।‚{\tiny $_{lb}$}‚ आचार्य दि ग्ना गे न तु स्प‚र्श‚व्य‚तिरिक्तं वायुंम‚भ्युप‚ग‚म्य त‚त्र प‚र‚कीय‚म‚नुमान‚म‚युक्त‚{\tiny $_{lb}$}‚मुक्त‚मित्य‚दोषः ।
	{\color{gray}{\rmlatinfont\textsuperscript{§~\theparCount}}}
	\pend% ending standard par
      ‚{\tiny $_{lb}$}‚

	  
	  \pstart \leavevmode% starting standard par
	त‚स्मात् स्थित‚मेत‚द् [।] अन्व‚य‚व्य‚तिरेक‚योर्निश्च‚य‚म‚द‚र्श‚{\tiny $_{३}$}‚न‚मात्राच्च प्र‚ति‚{\tiny $_{lb}$}‚षेधाभावं ब्रुव‚ताचार्येणेष्टः प्र‚तिब‚न्ध इति ॥
	{\color{gray}{\rmlatinfont\textsuperscript{§~\theparCount}}}
	\pend% ending standard par
      ‚{\tiny $_{lb}$}‚

	  
	  \pstart \leavevmode% starting standard par
	क‚स्त‚र्ह्येव‚मुपाल‚ब्ध इति चेदाह । \textbf{एव‚मि}त्यादि । एव‚मित्य‚न‚न्त‚रोक्ताभिरुप‚{\tiny $_{lb}$}‚प‚त्तिभिः । आचार्य‚स्य शिष्य आ चा र्यी यः क‚श्चिदाचार्य‚ग्र‚न्थान‚भिज्ञः \textbf{अनुप‚ल‚म्भाद्‚{\tiny $_{lb}$}‚ अभावं ब्रुवाण उपाल‚ब्धः} ॥
	{\color{gray}{\rmlatinfont\textsuperscript{§~\theparCount}}}
	\pend% ending standard par
      ‚{\tiny $_{lb}$}‚

	  
	  \pstart \leavevmode% starting standard par
	अपि चेत्य‚द‚र्श‚न‚मात्रेणाभावाभ्युप‚ग‚मे प्र‚त्य‚क्ष‚बाधां द‚र्श‚यितुमाह । \textbf{देशा}दि‚{\tiny $_{४}$}‚‚{\tiny $_{lb}$}‚भेदात् । आदिश‚ब्दात् काल‚संस्कार‚भेदात् । \textbf{भिन्ना} नानारूपा \textbf{दृश्य‚न्ते द्र‚व्येष्वेक}‚{\tiny $_{lb}$}‚जातीयेष्व‚पि \textbf{श‚क्त‚यो} र‚स‚वीर्य‚विपाकादिल‚क्ष‚णाः । त‚त्रेत्य‚नेक‚श‚क्तिषु द्र‚व्येष्वेक‚दृष्ट्या‚{\tiny $_{lb}$}‚ एक‚स्व‚भाव‚स्य द्र‚व्य‚स्य क्व‚चिद् द‚र्श‚नान्नान्य‚त्रापि देशादौ \textbf{युक्त‚स्त‚द्भाव‚निश्च‚यः} ।‚{\tiny $_{lb}$}‚ ‚{\tiny $_{lb}$}‚ \leavevmode\ledsidenote{\textenglish{69/s}}य‚थाप‚रिदृष्ट‚द्र‚व्य‚स्व‚भाव‚निश्च‚यः ।
	{\color{gray}{\rmlatinfont\textsuperscript{§~\theparCount}}}
	\pend% ending standard par
      ‚{\tiny $_{lb}$}‚

	  
	  \pstart \leavevmode% starting standard par
	\textbf{य‚दी}त्यादिनैत‚देव व्याच‚ष्टे । \textbf{य‚दि विप‚क्षे} हेतो\textbf{र‚द‚{\tiny $_{५}$}‚र्श‚न‚मात्रेणाप्र‚तिब‚द्ध}स्य‚{\tiny $_{lb}$}‚ स्व‚साध्ये \textbf{त‚द‚व्य‚भिचारः} साध्याव्य‚भिचार इष्य‚ते । त‚दैक‚त्र दृष्ट‚स्य द्र‚व्य‚स्य य‚द्‚{\tiny $_{lb}$}‚ रूप‚मुप‚ल‚ब्ध‚न्त‚त्त‚स्यान्य‚त्रापि द्र‚व्य‚सामान्याद‚नुमेयं स्यात् । अस्ति हि त‚त्रापि त‚त्कालं‚{\tiny $_{lb}$}‚ हेतोर्विप‚क्षेऽद‚र्श‚न‚मात्रं ।
	{\color{gray}{\rmlatinfont\textsuperscript{§~\theparCount}}}
	\pend% ending standard par
      ‚{\tiny $_{lb}$}‚

	  
	  \pstart \leavevmode% starting standard par
	न चेदं युक्तं । य‚स्मात् \textbf{क्व‚चिद् देशे कानिचिद् द्र‚व्याणि क‚थंचिद् दृष्टानि}‚{\tiny $_{lb}$}‚ प्र‚तिनिय‚त‚र‚सादित्वेन । पुन‚र‚न्य‚थे‚{\tiny $_{६}$}‚ति य‚थादृष्टाकार‚वैप‚रीत्येना\textbf{न्य‚त्र} देशे \textbf{दृश्य‚न्ते ।‚{\tiny $_{lb}$}‚ य‚थे}त्यादिना विष‚य‚माह । वीर्य‚न्दोषाप‚न‚य‚न‚श‚क्तिः [।] प‚रिणामो विपाकः । \textbf{विशिष्टा‚{\tiny $_{lb}$}‚ र‚स‚वीर्य‚विपाका} यासामिति विग्र‚हः । \textbf{नान्य}त्रेति क्षेत्र‚विशेषाद‚न्य‚त्र । \textbf{य‚था} देश‚{\tiny $_{lb}$}‚\textbf{विशेषात् त‚था काल‚संस्कार‚भेदाद् विशिष्ट‚र‚स‚वीर्य‚विपा}का भ‚व‚न्ति । संस्कारः‚{\tiny $_{lb}$}‚ क्षीराद्य‚व‚सेकः । \textbf{न च त‚द्दे‚{\tiny $_{७}$}‚शै}रिति प्र‚देशो येषां पुरुषाणान्तै\textbf{स्त‚था}विशिष्ट‚र‚सा- \leavevmode\ledsidenote{\textenglish{27a/PSVTa}}‚{\tiny $_{lb}$}‚ दियुक्ता \textbf{दृष्टा इति} कृत्वा \textbf{स‚र्वा} अत‚द्देशा अपि \textbf{त‚त्त्वेन} तुल्य‚रूपादित्वेन \textbf{त‚था}भूता‚{\tiny $_{lb}$}‚ य‚थागृहीत‚स्व‚भाव‚तुल्याः \textbf{सिध्य‚न्ति} । किं कार‚णं [।] \textbf{गुणान्त‚रा}णां र‚सादिविशेषाणां‚{\tiny $_{lb}$}‚ \textbf{कार‚णान्त‚रापेक्ष‚त्वात् । विशेष‚हेत्व‚भा}वे \textbf{तु स्यादे}क‚स्व‚भाव‚ता\textbf{नुमानं ॥ अद्ष्टः‚{\tiny $_{lb}$}‚ क‚र्त्ता} य‚स्य वैदिक‚स्य त‚त्त‚था । त\textbf{द‚पि पुरुष‚संस्कार‚पूर्व‚कं} पुरुष‚प्र‚य‚त्न‚{\tiny $_{१}$}‚हेतुकं । एत‚च्च‚{\tiny $_{lb}$}‚ साध्य‚फ‚लं । \textbf{वाक्येषु} पौरुषेयापौरुषेय‚त्वेनाभिम‚तेषु \textbf{विशेषाभावाद}नेन हेतुः क‚थितः ।‚{\tiny $_{lb}$}‚ प्र‚योगा पुन‚र्य‚द्व‚स्तु य‚द‚भिन्न‚स्व‚भाव‚न्तत्तत्स‚मान‚हेंतुकं [।] य‚थैको धूमो धूमान्त‚र‚{\tiny $_{lb}$}‚स‚मान‚जातीयः । पौरुषेय‚वाक्याभिन्न‚स्व‚भावानि चापौरुषेयाभिम‚तानि वाक्या‚{\tiny $_{lb}$}‚नीति स्व‚भाव‚हेतुः । न चासिद्धो हेतुः । त‚था हि यैः प्र‚कारैर्विषाद्य‚प‚न‚य‚नादिति‚{\tiny $_{lb}$}‚ वैदिकानां विशेष‚{\tiny $_{२}$}‚ इष्य‚ते । तेषां \textbf{स‚र्व‚प्र‚काराणां पुरुषैः कार‚ण‚द‚र्श‚नात्} । त‚था हि‚{\tiny $_{lb}$}‚ श ब रा दिम‚न्त्राणामिपि विषाद्य‚प‚न‚य‚नाद‚यो दृश्य‚न्ते ॥
	{\color{gray}{\rmlatinfont\textsuperscript{§~\theparCount}}}
	\pend% ending standard par
      ‚{\tiny $_{lb}$}‚

	  
	  \pstart \leavevmode% starting standard par
	य‚दि श‚ब्द‚स्व‚भाव‚साम्याद‚दृष्ट‚क‚र्त्तृक‚स्यापि पौरुषेय‚त्व‚म‚नुमीय‚ते । एव‚न्त‚र्हि‚{\tiny $_{lb}$}‚ स्व‚चित्त‚स‚न्त‚तिद‚र्श‚नात् स‚र्वे चित्त‚ध‚र्मा ज्ञाताः [।] तेन याव‚द् बोध‚रूप‚न्ताव‚त्‚{\tiny $_{lb}$}‚ ‚{\tiny $_{lb}$}‚ \leavevmode\ledsidenote{\textenglish{70/s}}प‚ञ्चेन्द्रियाश्रितं स‚विक‚ल्प‚कं च स‚र्वं च रागादिज‚न‚न‚वास‚नाग‚र्भं य‚थेदानीन्त‚था‚{\tiny $_{lb}$}‚नाग‚ताव‚{\tiny $_{३}$}‚स्थायाम‚पि य‚था च म‚म त‚था प‚रेषाम‚पि[।] तेन चित्त‚त्वेनेन्द्रियाश्रित‚{\tiny $_{lb}$}‚त्वेन व‚च‚नादिहेतुत्वेन च स‚र्व‚स्य पुंस‚श्चित्तं स‚र्व‚दा रागादियुक्त‚न्निय‚त‚व‚स्तुग्राहि‚{\tiny $_{lb}$}‚त्वाद‚स‚र्व‚विष‚यं किन्नानुमीय‚त इत्याशंक्याह । \textbf{नैव}मित्यादि । य‚था वाक्यान्य‚{\tiny $_{lb}$}‚स‚म्भ‚व‚द्विशेष‚हेतूनि \textbf{नैवंम‚स‚म्भ‚व‚द्विशेष‚हेत‚वः पुरुषाः} येन विशेष‚हेत्व‚स‚म्भ‚वेन ।‚{\tiny $_{lb}$}‚ व‚च‚नादेरादिग्र‚ह‚णेनेन्द्रियाश्रित‚त्व‚चित्त‚त्वादि‚{\tiny $_{४}$}‚ग्र‚ह‚णं । \textbf{व‚च‚ना}देरेव \textbf{किंचिन्मा}त्रेणै‚{\tiny $_{lb}$}‚क‚देशेन \textbf{साध‚र्म्या}त्त‚स्माल्लिङ्गात् पुंसां रागादिम‚त्वेन निय‚त‚विष‚य‚त्वेनान्येन वा‚{\tiny $_{lb}$}‚ \textbf{स‚र्वाकारेण साम्य‚म‚नुमीयेत} ।
	{\color{gray}{\rmlatinfont\textsuperscript{§~\theparCount}}}
	\pend% ending standard par
      ‚{\tiny $_{lb}$}‚

	  
	  \pstart \leavevmode% starting standard par
	विशेष‚हेतुस‚म्भ‚व एव कुत इत्याह । \textbf{स‚र्वेष्वेव} चेतोगु\textbf{णेषु विशेष‚स्य द‚र्श‚नात्} ।‚{\tiny $_{lb}$}‚ राग‚प्र‚ज्ञाद‚यो हि स्व‚विष‚य‚ग्र‚ह‚णंप्र‚ति म‚न्द‚म‚न्द‚वृत्त‚योपि कालान्त‚रेण प‚टुत‚रा भ‚व‚न्ति ।‚{\tiny $_{lb}$}‚ स्व‚विष‚य‚श्चैषां स्प‚ष्ट‚त‚रो भ‚व‚तीति वि‚{\tiny $_{५}$}‚शेषो दृश्य‚ते । स पुनः कुतो भ‚व‚तीत्याह ।‚{\tiny $_{lb}$}‚ \textbf{संस्कारो}भ्यास‚स्त‚स्य \textbf{विशेषेण} प्र‚ज्ञादीनां \textbf{विशेष‚प्र‚तीतेः} ॥ भ‚व‚तु प्र‚ज्ञादीनां म‚नोगु‚{\tiny $_{lb}$}‚णानाम‚भ्यासात् प्र‚क‚र्षो दृष्ट‚त्वात् । स‚र्व‚ज्ञाद‚य‚स्तु न दृष्टा इति क‚थ‚न्तेषाम्भाव‚{\tiny $_{lb}$}‚इत्य‚त आह । \textbf{त‚द्व‚दि}त्यादि । \textbf{अन्य‚स्यापि} स‚र्व‚ज्ञ‚त्वादेर्विशेष‚स्य \textbf{स‚म्भ‚वात्} । अभ्या‚{\tiny $_{lb}$}‚स‚व‚शाच्च नैरात्म्याकार‚स्य स‚र्व‚प‚दार्थ‚ग्राहिण‚श्च सामान्य‚विक‚ल्प‚स्य य‚त्स्फुटाभ‚{\tiny $_{६}$}‚‚{\tiny $_{lb}$}‚त्व‚न्त‚देव वैराग्यं सार्व‚ज्ञं च प‚दं । तेन त्रिभिः प्र‚कारैर्वैराग्य‚स‚र्व‚ज्ञ‚त्व‚विशेषास‚म्भ‚{\tiny $_{lb}$}‚वानुमानं स्यात् । य‚दि नैरात्म्य‚विष‚य‚स्य स‚र्व‚विष‚य‚स्य वा विक‚ल्प‚स्यास‚म्भ‚वः ।‚{\tiny $_{lb}$}‚ स‚म्भ‚वे वा य‚दि म‚नोगुणानां न विशेषः स्यात् । स‚ति वा विशेषे विशेष‚हेतुर्य‚दि न ज्ञातः‚{\tiny $_{lb}$}‚ स्यात् । याव‚ता नैरात्म्यादिविष‚य‚स्य विक‚ल्प‚स्य स‚म्भ‚वोस्ति विशेष‚हेतुश्चाभ्यासो‚{\tiny $_{lb}$}‚ \leavevmode\ledsidenote{\textenglish{27b/PSVTa}} विज्ञातः । त‚स्माद् य‚थाऽस‚{\tiny $_{७}$}‚ति प्र‚तिब‚न्ध‚वैक‚ल्ये स‚म‚र्थेयं बीज‚कार‚ण‚साम‚ग्री‚{\tiny $_{lb}$}‚ अङ्कुरोत्पादायेति साम‚र्थ्यानुमानं । त‚द्व‚द‚स‚ति प्र‚तिब‚न्ध‚वैक‚ल्ये चिर‚कालान्नैर‚न्त‚{\tiny $_{lb}$}‚र्य‚व‚त‚श्चाभ्यास‚विशेषाच्च नैरात्म्य‚विष‚य‚स्य स‚र्व‚प‚दार्थ‚विष‚य‚स्य च ज्ञान‚स्य स्फुटाभ‚त्वं‚{\tiny $_{lb}$}‚ स‚म्भ‚व‚तीति वैराग्य‚स‚र्व‚ज्ञ‚त्व‚योः स‚म्भ‚वानुमानं । एव‚न्ताव‚त् स‚म्भ‚वानुमान‚म‚स्माकं ।
	{\color{gray}{\rmlatinfont\textsuperscript{§~\theparCount}}}
	\pend% ending standard par
      ‚{\tiny $_{lb}$}‚

	  
	  \pstart \leavevmode% starting standard par
	\textbf{अस‚म्भ‚वे त्व‚नुमात‚व्ये} भ‚व‚तां न त‚स्य‚{\tiny $_{१}$}‚ध‚र्म‚स्य बाध‚कः क‚श्चिद्धेतुर्विद्य‚ते । त‚द‚{\tiny $_{lb}$}‚\textbf{भावान्}नासंभ‚व‚द्विशेष‚हेत‚वः पुमांस इति स‚म्ब‚न्धः । क‚थं पुन‚र्बाध‚क‚स्याभाव इत्याह ।‚{\tiny $_{lb}$}‚ \textbf{वैराग्य‚स्यादृष्ट‚त्वाद्} [।] उप‚ल‚क्ष‚णं चैत‚त्स‚र्व‚ज्ञ‚स्याप्य‚दृष्ट‚त्वात् । \textbf{अदृष्टेन च वैरा}‚{\tiny $_{lb}$}‚  \leavevmode\ledsidenote{\textenglish{71/s}}ग्येण स‚ह हेतो\textbf{र्बाध्य‚बाध‚क‚भावा}सिद्धेः ॥
	{\color{gray}{\rmlatinfont\textsuperscript{§~\theparCount}}}
	\pend% ending standard par
      ‚{\tiny $_{lb}$}‚

	  
	  \pstart \leavevmode% starting standard par
	स्यान्म‚तं [।] न व‚च‚नादेर्वैराग्यादीनां साक्षाद् अस‚म्भ‚वोनुमीय‚ते येन बाध‚को‚{\tiny $_{lb}$}‚ हेतुर्मृग्य‚ते । किन्तु वीत‚रा‚{\tiny $_{२}$}‚गाभिम‚तेषु रागाद‚योनुमीय‚न्ते [।] तेष्व‚नुमितेष्व‚र्थ‚तो‚{\tiny $_{lb}$}‚ वीत‚रागादिप्र‚तिषिद्ध‚म‚दृष्टेनापि वैराग्येन्\edtext{}{\lemma{वैराग्येन्}\Bfootnote{? ण}} रागित्व‚स्य विरोधादित्य‚त आह ।‚{\tiny $_{lb}$}‚ \textbf{रागे}त्यादि । \textbf{न हि रागादीनाम‚व्य‚भिचारि कार्य}म‚स्ति । आदिग्र‚ह‚णाद् अस‚र्व‚ज्ञ‚प‚रि‚{\tiny $_{lb}$}‚ग्र‚हः । त‚था हि व्य‚व‚हार‚व्युत्प‚त्तिमार‚भ्य य‚था य‚थार्थ‚प‚रिज्ञान‚न्त‚था त‚था‚{\tiny $_{lb}$}‚ त‚द्विष‚य‚म्व‚क्तृत्व‚म्भ‚व‚तीति तेन य‚दि स‚र्व‚विष‚यं क‚{\tiny $_{३}$}‚स्य‚चिद् विज्ञानं स्यात् त‚द्विष‚य‚म‚पि‚{\tiny $_{lb}$}‚ व‚क्तृत्वं केन वार्य‚ते ॥
	{\color{gray}{\rmlatinfont\textsuperscript{§~\theparCount}}}
	\pend% ending standard par
      ‚{\tiny $_{lb}$}‚

	  
	  \pstart \leavevmode% starting standard par
	स्यादेत‚द् [।] य‚दि वैराग्यादिल‚क्ष‚ण‚न्त‚स्य विशेषोस्ति क‚थ‚म‚स्माभिर्नोप‚ल‚भ्य‚त‚{\tiny $_{lb}$}‚ इत्याह । \textbf{स‚म्भ‚वेपि} तेषां वैराग्यादिल‚क्ष‚णानां \textbf{विशेषाणां} प‚र‚स‚न्ताने \textbf{द्र‚ष्टुम‚श‚क्य‚{\tiny $_{lb}$}‚त्वात्} । न च त‚थाभूतानाम‚नुप‚ल‚म्भात् प्र‚तिक्षेप इत्याह । तादृशो च विप्र‚क‚र्षि‚{\tiny $_{lb}$}‚णाम‚प्र‚तिक्षेपार्ह‚त्वात् ॥
	{\color{gray}{\rmlatinfont\textsuperscript{§~\theparCount}}}
	\pend% ending standard par
      ‚{\tiny $_{lb}$}‚

	  
	  \pstart \leavevmode% starting standard par
	स्यादेत‚द् [।] य‚था पुरुषा अप्र‚तिक्षेपार्हा‚{\tiny $_{४}$}‚स्त‚द्व‚द् वाक्यानीत्याह । \textbf{नैव‚{\tiny $_{lb}$}‚मित्}यादि । किं कार‚णं । \textbf{दृश्य‚विशेष‚त्वात्} । त‚था हि वैदिकानां वाक्यानाम्वि‚{\tiny $_{lb}$}‚शेषो दृश्य एवेष्य‚ते । अथ नेष्य‚ते । एव‚म\textbf{दृश्य‚त्वेपि} विशेष‚स्याभ्युप‚ग‚म्य‚माने ।‚{\tiny $_{lb}$}‚ तेनानुप‚ल‚ब्धेन विशेषेणा\textbf{दृष्ट‚विशेषाणां} वैदिक‚वाक्यानां लौकिक‚वाक्येभ्यो \textbf{विजा‚{\tiny $_{lb}$}‚तीय‚त्वोप‚ग‚म‚विरोधात्} ॥
	{\color{gray}{\rmlatinfont\textsuperscript{§~\theparCount}}}
	\pend% ending standard par
      ‚{\tiny $_{lb}$}‚

	  
	  \pstart \leavevmode% starting standard par
	स्यान्म‚तं [।] दृश्या एव विशेषा वैदिकानां दुःश्र‚व‚ण‚त्व‚{\tiny $_{५}$}‚दुर्भ‚ण‚त्वाद‚य‚स्तैः‚{\tiny $_{lb}$}‚ पौरुषेयेभ्यो भिन्नानि भ‚विष्य‚न्तीत्य‚त आह । \textbf{त‚द्विशेषाणामि}त्यादि । \textbf{अन्य‚त्रे}ति‚{\tiny $_{lb}$}‚ पौरुषेयेषु । न केव‚ल‚म‚दृष्ट‚विशेषाणां विजातीय‚त्वोप‚ग‚म‚विरोधाद‚दृश्य‚त्वं विशे‚{\tiny $_{lb}$}‚षाणाम‚युक्त‚मित‚श्च \textbf{प्र‚त्य‚क्षाणाम‚प्र‚त्य‚क्ष‚स्व‚भाव‚विरोधात्} । न ह्येक‚स्य स्व‚भाव‚द्व‚यं‚{\tiny $_{lb}$}‚ विरुद्धं घ‚ट‚ते ॥ विशेषाः प्र‚त्य‚क्षा एव केव‚लं भ्रान्तिनिमित्त‚स‚द्भावात् । वि‚{\tiny $_{६}$}‚षादि‚{\tiny $_{lb}$}‚श‚क्तिव‚न्नाव‚धार्य‚न्त इति चेदाह । \textbf{भ्रान्तिनिमित्ताभावादि}ति । रूप‚साध‚र्म्य‚द‚र्श‚नं हि‚{\tiny $_{lb}$}‚ भ्रान्तिनिमित्तं विषादिषु । नैवं वैदिकेषु ॥ क‚थ‚ङ्ग‚म्य‚त इति चेदाह । \textbf{बाध‚का}भावात्‚{\tiny $_{lb}$}‚ \textbf{भ्रान्त्य‚सिद्धेरि}ति । य‚दि वैदिकानां विशेषे भ्रान्त्यानुप‚ल‚क्ष्य‚माणे पुन‚र्विशेषाव‚ल‚म्बि‚{\tiny $_{lb}$}‚ ‚{\tiny $_{lb}$}‚ \leavevmode\ledsidenote{\textenglish{72/s}}प्र‚माण‚मुत्प‚द्येत भ्रान्तेर्बाध‚क‚न्त‚दा भ्रान्तेस्त‚न्निमित्त‚स्य च भ‚व‚ति निश्च‚य‚स्त‚च्च‚{\tiny $_{lb}$}‚ \leavevmode\ledsidenote{\textenglish{28a/PSVTa}} नास्ति ।‚{\tiny $_{७}$}‚ त‚स्माल्लौकिकैः श‚ब्दैः वैदिकानाम‚विशेषे साध्ये नास्ति साध‚कं प्र‚माणं ॥
	{\color{gray}{\rmlatinfont\textsuperscript{§~\theparCount}}}
	\pend% ending standard par
      ‚{\tiny $_{lb}$}‚

	  
	  \pstart \leavevmode% starting standard par
	पुरुषेषु त‚र्हि किं बाध‚कं येन स‚र्वाकार‚गुण‚साम्य‚साध‚ने दोष इत्याह । \textbf{पुरुषे}‚{\tiny $_{lb}$}‚ष्वित्यादि । प्र‚ज्ञादिविष‚य‚स्यातिश‚य‚स्याभ्यास‚पूर्व‚क‚स्य य‚द् द‚र्श‚न‚न्त‚देव बाध‚कं ।‚{\tiny $_{lb}$}‚ य‚द्य‚स‚म्भ‚व‚द्वैराग्यं पुरुष‚स्य चित्त‚म्भ‚वेत् । नाभ्यासाधेय‚विशेष‚म्भ‚वेद् [।] भ‚व‚ति‚{\tiny $_{lb}$}‚ च [।] त‚तो \textbf{विशेष‚द‚र्श‚न‚स्य बाध‚क‚त्वाद् अ‚{\tiny $_{१}$}‚स‚मा}नं ॥ वेद‚वाक्यानुमाने य‚दुक्तं‚{\tiny $_{lb}$}‚ प्र‚त्य‚क्षाणां श‚ब्दानाम‚प्र‚त्य‚क्ष‚स्व‚भावाभावादिति [।] स्याद‚य‚न्दोषो य‚दि विशेषः स्व‚भा‚{\tiny $_{lb}$}‚व‚भूतः स्यात् । किन्तु प‚र‚भाव‚भूत इत्याह । \textbf{प‚र‚भाव‚भूत‚स्ये}ति । प‚र‚भाव‚ङ्ग‚तः प‚र‚भाव‚{\tiny $_{lb}$}‚भूतः । अन्य‚स्व‚भाव इत्य‚र्थः । \textbf{अत‚द्विशेष‚त्वा}दित्य‚वाक्य‚विशेष‚त्वात् । \textbf{य‚तो नास्ति‚{\tiny $_{lb}$}‚ विशेषो वाक्याना}न्त‚त्त‚स्माद\textbf{भिन्न‚स्व‚भावानां स‚र्वेषां} पौ‚{\tiny $_{२}$}‚रुषेयापौरुषेयाभिम‚तानां‚{\tiny $_{lb}$}‚ \textbf{पुरुष‚क्रि}या । पुरुषैः क‚र‚णं । \textbf{न वा क‚स्य‚चित्} । लौकिस्यापि न पुरुष‚क्रियेत्य‚र्थः ॥
	{\color{gray}{\rmlatinfont\textsuperscript{§~\theparCount}}}
	\pend% ending standard par
      ‚{\tiny $_{lb}$}‚

	  
	  \pstart \leavevmode% starting standard par
	एव‚माचार्यीय‚स्याद‚र्श‚न‚मात्रेण विप‚क्षाद्धेतोर्व्य‚तिरेक‚मिच्छ‚तः ग्र‚न्थ‚विरोधं‚{\tiny $_{lb}$}‚ प्र‚माण‚विरोधं चोक्त्त्वा ती र्थि का नां प‚र‚स्प‚र‚व्याघात‚माह । \textbf{किंचे}त्यादि । मृद‚{\tiny $_{lb}$}‚श्चेत‚ना । एत‚च्च लो का य त द‚र्श‚नं । \textbf{आत्मा} च \textbf{मृच्चेत‚ना} चेति द्व‚न्द्वः । \textbf{आदि}‚{\tiny $_{lb}$}‚श‚{\tiny $_{३}$}‚ब्दात् क्षीरादिषु द्र‚व्यादि । तेषा\textbf{म‚भाव}स्य साध‚नायानुप‚ल‚म्भः प‚रेणोक्तोपि‚{\tiny $_{lb}$}‚ \textbf{य}स्त‚स्या\textbf{भाव‚स्याप्र‚साध‚क} इष्ट आत्मादिवादिभिर‚नुप‚ल‚म्भ‚मात्र‚स्याप्र‚माण‚त्वादिति ।‚{\tiny $_{lb}$}‚ \textbf{स एवानुप‚ल‚म्भ} आत्मादिनिषेधे प्र‚माण‚त्वेनानिष्टः \textbf{किं हेत्व‚भाव‚स्य} हेतोविप‚क्षाद्‚{\tiny $_{lb}$}‚ व्य‚तिरेक‚स्य \textbf{साध‚कः} ॥ हेत्व‚भावे चा\textbf{नुप‚ल‚म्भं} चा\textbf{स्य} वै शे षि का देः \textbf{प्र‚माण‚य‚तः‚{\tiny $_{lb}$}‚ आत्म‚वादो निराल‚म्बो} निरा‚{\tiny $_{४}$}‚श्र‚यः \textbf{स्या}त् । त‚त्रानुप‚ल‚म्भ‚स्याभाव‚साध‚न‚स्य‚{\tiny $_{lb}$}‚ स‚म्भ‚वात् । त‚था हि न \textbf{प्र‚त्य‚क्षेणात्म‚न उप‚ल‚म्भो} नित्य‚प‚रोक्ष‚त्वाभ्युप‚ग‚मात् । अथ‚{\tiny $_{lb}$}‚ स्यादात्म‚नोनुमान‚मेवोप‚ल‚म्भोस्त्येवेत्य‚त आह । \textbf{त‚त्कार्ये}त्यादि । अप्र‚त्य‚क्ष‚त्वा‚{\tiny $_{lb}$}‚देवात्म‚न‚स्त‚त्कार्य‚स्व‚भाव‚रूप‚स्य लिङ्ग‚स्यानिश्च‚यान्नानुमान‚मुप‚ल‚म्भः ।
	{\color{gray}{\rmlatinfont\textsuperscript{§~\theparCount}}}
	\pend% ending standard par
      ‚{\tiny $_{lb}$}‚‚{\tiny $_{lb}$}‚\textsuperscript{\textenglish{73/s}}

	  
	  \pstart \leavevmode% starting standard par
	य‚था नित्य‚प‚रोक्षाणाम‚पीन्द्रियादीनाम‚नुमान‚न्त‚थात्म‚नो भ‚विष्य‚तीति चे‚{\tiny $_{५}$}‚‚{\tiny $_{lb}$}‚ दाह । \textbf{इन्द्रियाणामि}त्यादि । आदिश‚ब्दात् स्मृतिबीजादीनां । \textbf{विज्ञार्न}मेव \textbf{कार्य}‚{\tiny $_{lb}$}‚न्त‚स्य \textbf{कादाचित्क‚त्वा}त् । त‚था हि स‚त्स्व‚पि रूपालोक‚म‚न‚स्कारेषु निमीलित‚लोच‚ना‚{\tiny $_{lb}$}‚द्य‚व‚स्थासु विज्ञान‚स्याभावात् । पुन‚श्चोन्मीलित‚लोच‚नाव‚स्थासु भावात् । विज्ञा‚{\tiny $_{lb}$}‚न‚कार्यं कार‚णान्त‚र‚सापेक्षं सिध्य‚ति त‚तोस्य \textbf{सापेक्ष्य‚सिद्ध्}या इन्द्रियादीना\textbf{म्प्र‚सिद्धि‚{\tiny $_{lb}$}‚रुच्य}ते ।
	{\color{gray}{\rmlatinfont\textsuperscript{§~\theparCount}}}
	\pend% ending standard par
      ‚{\tiny $_{lb}$}‚

	  
	  \pstart \leavevmode% starting standard par
	एत‚दुक्त‚म्भ‚व‚ति । य‚त्सापेक्ष‚{\tiny $_{६}$}‚मिद‚ङ्कादाचित्कं । विज्ञान‚न्त\textbf{त्किम‚प्य‚स्य} विज्ञान‚स्य‚{\tiny $_{lb}$}‚ \textbf{कार‚ण‚म‚स्ती}त्य‚नुमीय‚ते । त‚देव चेन्द्रिय‚मिति व्य‚व‚ह्रिय‚ते । \textbf{न त्वेवंभूत‚मिति} न‚{\tiny $_{lb}$}‚ रूप‚विशेषेण मूर्त्त‚त्वादिना युक्त‚मिन्द्रिय‚म‚नुमीय‚त इत्य‚र्थः ॥ \textbf{एव}मिति य‚था‚{\tiny $_{lb}$}‚ कादाचित्क‚विज्ञान‚कार्यान्य‚थानुप‚प‚त्त्येन्द्रियानुमानं । त‚था सुखादिकार्यं । य‚स्य‚{\tiny $_{lb}$}‚\textbf{सुखादिकार्य}न्त‚त्किम‚प्य‚स्तीत्य‚नुमानेन त‚च्चात्म‚स्व‚रूप‚मिति \textbf{प्र‚साधितं‚{\tiny $_{७}$}‚ क‚ञ्चि- \leavevmode\ledsidenote{\textenglish{28b/PSVTa}}‚{\tiny $_{lb}$}‚ द‚र्थंमा}त्म‚वादिनो \textbf{न पुष्णा}ति । \textbf{येन केन‚चिद}निर्दिष्ट‚विशेषेण कार‚णेन \textbf{कार‚ण‚{\tiny $_{lb}$}‚व‚त्त्वाभ्युप‚ग‚मात्} सुखादीनां । न चैवंभूत आत्मा । नित्य‚क‚र्त्तृभोक्तृत्वादिल‚क्ष‚ण‚{\tiny $_{lb}$}‚त्वेनाभ्युप‚ग‚मात् । य‚त‚श्च य‚थाभ्युप‚ग‚त‚स्यात्म‚नो नास्ति कार्य‚लिङ्गं । \textbf{त‚था च‚{\tiny $_{lb}$}‚ स‚त्य‚नुप‚ल‚म्भ एवात्म‚नः स्या}त् । \textbf{त‚स्मात् त‚मा}त्मान‚न्तेना\textbf{नुप‚ल‚म्भेन} प्र‚त्य‚क्षानुमान‚{\tiny $_{lb}$}‚निवृत्तिल‚क्ष‚णेन \textbf{प्र‚त्याच‚क्षाणः किमि}ति \textbf{प्र‚तिव्यूढः} प्र‚तिक्षिप्त आ‚{\tiny $_{१}$}‚त्म‚वादिना ।‚{\tiny $_{lb}$}‚ अनुप‚ल‚म्भ‚मात्रान्नास्त्य‚स‚त्त्व‚मात्म‚न इति । \textbf{क‚थ}म‚साध‚नं स‚द् विप‚क्षाद्धेतोः प्राणा‚{\tiny $_{lb}$}‚दिम‚त्वादे\textbf{र्व्य‚तिरेकं साध‚येत्} ॥ भूतानामेव श‚क्तिश्चैत‚न्य‚मिष्य‚ते चा र्वा कैः [।]‚{\tiny $_{lb}$}‚ भूत‚स्व‚भावा च मृदित्येवं \textbf{मृदः} ख‚ल्व‚पि \textbf{चैत‚न्य‚म‚नुप‚ल‚भ्य‚मान‚म‚पीच्छ‚न्} लो का‚{\tiny $_{lb}$}‚य ति कः । य‚दाह [।] \textbf{तेभ्यो भूते}भ्य‚श्चैत‚न्य‚म्म‚द‚श‚क्तिव‚द्विज्ञान‚मिति । पुन‚स्त‚त‚{\tiny $_{lb}$}‚ एवाप्र‚माण‚काद‚नुप‚ल‚म्भात् । \textbf{अद‚{\tiny $_{२}$}‚र्श‚नाद् व‚च‚ना}देर‚स‚र्व‚ज्ञ‚त्वादिसाध‚नाय लिंग‚{\tiny $_{lb}$}‚त्वेनोप‚नीत‚स्य विप‚क्षाद् \textbf{व्यावृत्तिमा}ह ॥ \textbf{द‚ध्यादिकं क्षीरादिष्}व‚नुप‚ल‚भ्य‚मान‚म‚पी‚{\tiny $_{lb}$}‚च्छ‚न् ।
	{\color{gray}{\rmlatinfont\textsuperscript{§~\theparCount}}}
	\pend% ending standard par
      ‚{\tiny $_{lb}$}‚‚{\tiny $_{lb}$}‚\textsuperscript{\textenglish{74/s}}

	  
	  \pstart \leavevmode% starting standard par
	\textbf{अप‚र} इति सां ख्यः पुनः स एव । प‚रार्थाश्च‚क्षुराद‚यः स‚ङ्घात‚त्वादित्य‚भिधा‚{\tiny $_{lb}$}‚या\textbf{प‚रार्थेषु} श‚श‚विषाणादिषु \textbf{स‚ङ्घात‚त्व‚स्याद‚र्श‚नाद् व्य‚तिरेक}माह । एव‚न्ताव‚द‚स्य‚{\tiny $_{lb}$}‚ प‚र‚स्प‚र‚व्याघातः । न चाद‚र्श‚न‚मात्रेणास्य हेतोर्व्याप्तिः सिध्य‚ति ।‚{\tiny $_{३}$}‚ \textbf{को ह्य‚त्र‚{\tiny $_{lb}$}‚ निय‚मः स‚ङ्घातैर‚व‚श्यं प‚रार्थैर्भ‚वित‚व्यं} य‚तः संघात‚त्वाच्च‚क्षुरादीनां पारार्थ्य‚सिद्ध्या‚{\tiny $_{lb}$}‚त्मार्थ‚त्वं सां ख्य स्य सिध्येत् ॥ य‚दुक्तं द‚ध्यादिकं क्षीरादिष्व‚प्य‚नुप‚ल‚भ्य‚मान‚म‚पीति [।]‚{\tiny $_{lb}$}‚ त‚न्न । य‚स्माद‚स्त्ये\textbf{वोप‚ल‚म्भो द‚ध्यादीनां क्षीरादि}षु [।] कोसावित्याहा\textbf{नुमा‚{\tiny $_{lb}$}‚न‚मि}ति । अनुमानं चाहा\textbf{श‚क्ताद‚नुत्प‚त्ते}रिति । य‚दि हि क्षीरादौ द‚ध्यादिश‚क्तिर्न‚{\tiny $_{lb}$}‚ स्यात्त‚तो ऽश‚क्तात् क्षीरादेर्द‚ध्या‚{\tiny $_{४}$}‚दि नोत्प‚द्येत । प्र‚योग‚स्तु य‚द्य‚ज्ज‚न‚ने न श‚क्तं‚{\tiny $_{lb}$}‚ न त‚स्य त‚त उत्प‚त्तिर्य‚था शालिबीजाद् य‚वाङ्कुर‚स्य [।] उत्प‚द्य‚ते च द‚ध्यादिः‚{\tiny $_{lb}$}‚ क्षीरादिभ्य‚स्त‚स्माद‚स्ति द‚ध्यादिश‚क्तिः क्षीरादाविति कार्य‚हेतुप्र‚तिरूप‚को वैध‚र्म्य‚{\tiny $_{lb}$}‚प्र‚योगः । श‚क्तेरेव च द‚ध्यादिः कार्य‚कार‚ण‚योर‚भेदादिति म‚न्य‚ते ।
	{\color{gray}{\rmlatinfont\textsuperscript{§~\theparCount}}}
	\pend% ending standard par
      ‚{\tiny $_{lb}$}‚

	  
	  \pstart \leavevmode% starting standard par
	\textbf{अथेत्यादि} सिद्धान्त‚वादी । योसौ द‚ध्यादिको भावः प‚श्चादुप‚ल‚भ्य‚ते \textbf{किं स एव‚{\tiny $_{lb}$}‚ भा‚{\tiny $_{५}$}‚वः श‚क्तिरुतान्य‚देव किञ्चि}द् द‚ध्यादेर‚र्थान्त‚रं । \textbf{त‚थै}वेति निष्प‚न्न‚रूप‚द‚ध्यादिव‚त्‚{\tiny $_{lb}$}‚ क्षीराव‚स्थाया\textbf{मुप‚ल‚भ्येत । विशेषाभावात् । अन्य‚च्चे}दिति । द‚ध्यादिभ्योर्थान्त‚रं‚{\tiny $_{lb}$}‚ चेच्छ‚क्तिः । त‚दा \textbf{क‚थ‚म‚न्य‚भावे}न्य‚स्य श‚क्त्याख्य‚स्य भावे । \textbf{त‚द्द}ध्यादिक‚म‚स्ति ।‚{\tiny $_{lb}$}‚ नैवेत्य‚भिप्रायः । द‚ध्यादिज‚न‚न‚साम‚र्थ्यात् क्षीरादौ द‚ध्यादीत्यु\textbf{प‚चार‚मात्रं स्यात्} ।‚{\tiny $_{lb}$}‚ अनुप‚ल‚म्भ‚म‚प्र‚माणीकृत्य पुन‚स्त‚स्यैव‚{\tiny $_{६}$}‚ प्र‚माणीक‚र‚ण\textbf{म‚य‚म्प‚र‚स्प‚र‚व्याघात एषामात्मा}‚{\tiny $_{lb}$}‚दिवादिनामित्युप‚संहारः ॥
	{\color{gray}{\rmlatinfont\textsuperscript{§~\theparCount}}}
	\pend% ending standard par
      ‚{\tiny $_{lb}$}‚

	  
	  \pstart \leavevmode% starting standard par
	य‚त‚श्चाद‚र्श‚न‚मात्रान्नास्ति व्य‚तिरेक‚स्\textbf{त‚स्मात्त‚न्मात्र‚स‚म्ब‚न्धः} । हेतुस‚त्तामात्र‚{\tiny $_{lb}$}‚स‚म्ब‚द्ध\textbf{स्व‚भावः} साध्य‚त्वेनाभिम‚तः स्व‚य‚न्निव‚र्त्त‚मानो \textbf{भाव‚मेवे} स्व‚भाव‚भूत‚मेव‚{\tiny $_{lb}$}‚ हेतुत्वेनोप‚नीतं \textbf{निव‚र्त्त‚ये}त् । \textbf{वा} श‚ब्दो व‚क्ष्य‚माण‚विक‚ल्पापेक्षी ॥ \textbf{य‚था वृक्षो} निव‚र्त्त‚मानः‚{\tiny $_{lb}$}‚ \leavevmode\ledsidenote{\textenglish{29a/PSVTa}} \textbf{शिंश‚पा}न्नि‚{\tiny $_{७}$}‚व‚र्त्त‚य‚ति । क‚स्मा\textbf{च्छाखादिम‚द्विशेष‚स्यैव त‚था} शिंश‚पेति \textbf{प्र‚सिद्धेः स}‚{\tiny $_{lb}$}‚ ‚{\tiny $_{lb}$}‚ \leavevmode\ledsidenote{\textenglish{75/s}}वृक्ष‚स्\textbf{त‚स्य} शिंश‚पाख्य‚स्य \textbf{स्व‚भावः । स्व‚ञ्च स्व‚भा}वं वृक्षं \textbf{प‚रित्य‚ज्य क‚थं} शिंश‚पाख्यो‚{\tiny $_{lb}$}‚ \textbf{भावो भ‚वेत्} । किङ्कार‚णं [।] \textbf{स्व‚भाव‚स्यै}व वृक्ष‚स्वात्म‚न एव भाव‚त्वाच्छिंश‚पा\textbf{रूप‚{\tiny $_{lb}$}‚त्वा}त् । \textbf{इति} हेतोस्त‚स्यात्म‚भूत‚स्य साध‚न‚स्य शिंश‚पादेः स्व‚भाव\textbf{प्र‚तिब‚न्धादेव} स्व‚भावे‚{\tiny $_{lb}$}‚ साध्याभिम‚ते वृक्षादौ य‚थोक्तेन प्र‚कारेण प्र‚ति‚{\tiny $_{१}$}‚ब‚न्धादेवा\textbf{व्य‚भिचारः ॥ कार‚णं वा}‚{\tiny $_{lb}$}‚ निव‚र्त्त‚मान‚मित्य‚ध्याहारः । \textbf{कार्य‚न्नि}व‚र्त्त‚येदिति प्र‚कृतं । क‚स्माद् [।] \textbf{अव्य‚भिचार‚तः}‚{\tiny $_{lb}$}‚ कार्य‚स्य कार‚णाव्य‚भिचारादित्य‚र्थः ॥ \textbf{कार‚ण‚मित्या}दिना व्याच‚ष्टे ।
	{\color{gray}{\rmlatinfont\textsuperscript{§~\theparCount}}}
	\pend% ending standard par
      ‚{\tiny $_{lb}$}‚

	  
	  \pstart \leavevmode% starting standard par
	\textbf{अन्य}थेति । य‚दि कार‚णे निव‚र्त्त‚माने कार्यं न निव‚र्त्तेत । त‚दा \textbf{त‚त्}कार्याभिम‚तं‚{\tiny $_{lb}$}‚ \textbf{त‚स्य} कार‚ण‚स्य \textbf{कार्य‚मेव न स्यात्} । त‚स्मात् \textbf{कार‚णं निव‚र्त्त‚मानं कार्य‚म}व‚श्यं‚{\tiny $_{lb}$}‚ \textbf{निव‚र्त्त‚य‚ति} । य‚द्य‚पि वास‚गृहादाव‚ग्निकार‚ण‚{\tiny $_{२}$}‚निवृत्ताव‚पि न धूम‚स्य निवृत्तिस्त‚थापि‚{\tiny $_{lb}$}‚ दृष्ट‚कार‚ण‚व्य‚तिरेकेण नान्य‚स्माद‚स्योत्प‚त्तिर‚भिप्रेतेत्य‚र्थः । अत एवाह । \textbf{सिद्ध‚स्त्वि}‚{\tiny $_{lb}$}‚त्यादि । \textbf{सिद्ध‚स्तु कार्य‚कार‚ण‚भावः} कार्य‚स्य \textbf{स्व‚भावं} कार‚णे \textbf{निय‚म‚य‚ति} [।] स‚ति‚{\tiny $_{lb}$}‚ त‚स्मिन्भ‚व‚त्य‚स‚ति न भ‚व‚तीत्येव‚न्त‚द‚व्य‚भिचारिणं क‚रोति । \textbf{उभ‚य‚थे}ति तादात्म्येन‚{\tiny $_{lb}$}‚ त‚दुत्प‚त्त्या वा यः \textbf{स्व‚भाव‚प्र‚तिब‚न्ध}स्त‚स्मा\textbf{देव} । साध्य‚निवृत्त्या हेतो\textbf{र्निवृत्तिः} ।
	{\color{gray}{\rmlatinfont\textsuperscript{§~\theparCount}}}
	\pend% ending standard par
      ‚{\tiny $_{lb}$}‚

	  
	  \pstart \leavevmode% starting standard par
	\textbf{अन्य}थेति य‚दि‚{\tiny $_{३}$}‚ प्र‚तिब‚न्धो नेष्य‚ते । \textbf{एक}स्याप्र‚तिब‚न्ध‚क‚स्य साध्य‚स्य \textbf{निवृत्त्या‚{\tiny $_{lb}$}‚न्य‚निवृत्तिः} । अप्र‚तिब‚द्ध‚स्य साध‚न‚ध‚र्म‚स्य \textbf{निवृतिः क‚थ‚म्भ‚वेत्} [।] नैव । य‚स्मा‚{\tiny $_{lb}$}‚\textbf{न्नाश्व‚वानि}त्य‚श्व‚र‚हित इति कृत्वा \textbf{म‚र्त्त्येन} म‚नुष्येण \textbf{न भाव्यं गोम‚तापि किं ।‚{\tiny $_{lb}$}‚ स‚न्निधानात्त‚थैक‚स्ये}ति स्व‚भावेनास‚म्ब‚द्ध‚स्य हेदोः स‚न्निधानात् \textbf{क‚थ‚म‚न्य‚स्य} साध्य‚स्य‚{\tiny $_{lb}$}‚ \textbf{स‚न्निधि}र्नैव स‚न्निधानं । य‚स्माद्[।]
	{\color{gray}{\rmlatinfont\textsuperscript{§~\theparCount}}}
	\pend% ending standard par
      ‚{\tiny $_{lb}$}‚
	  \bigskip
	  \begingroup
	
	    
	    \stanza[\smallbreak]
	  {\normalfontlatin\large ``\qquad}गोमानित्येव म‚र्त्त्येन भाव्य‚म‚श्व‚व‚ता‚{\tiny $_{४}$}‚पि किं ।{\normalfontlatin\large\qquad{}"}\&[\smallbreak]
	  
	  
	  
	  \endgroup
	‚{\tiny $_{lb}$}‚‚{\tiny $_{lb}$}‚\textsuperscript{\textenglish{76/s}}

	  
	  \pstart \leavevmode% starting standard par
	य‚त एव‚न्त\textbf{स्मात् स्व‚भाव‚प्र‚तिब‚न्धादेव} साध्याभिम‚ते व‚स्तुनि प्र‚तिब‚द्ध‚त्वादेव‚{\tiny $_{lb}$}‚ \textbf{हेतुः स्व‚साध्य‚ङ्ग‚म‚य‚ति} । न तु स‚म्ब‚न्धात् । कार्य‚कार‚ण‚योर‚स‚ह‚भावेन व्याप्य‚{\tiny $_{lb}$}‚व्याप‚क‚योश्चैक‚त्वेन द्विष्ठ‚स‚म्ब‚न्धाभावात् । \textbf{स चेति} स्व‚भाव‚प्र‚तिब‚न्धः । \textbf{त‚द्भा‚{\tiny $_{lb}$}‚व‚ल‚क्ष‚ण} इति साध्य‚स्व‚भाव‚ल‚क्ष‚ण\textbf{स्त‚दुत्प‚त्तिल‚क्ष‚णो वा} ॥ स एव स्व‚भाव‚प्र‚तिब‚न्धो‚{\tiny $_{lb}$}‚\textbf{ऽविनाभावा}ख्यः साध‚र्म्य‚वैध‚र्म्य‚दृष्टा\textbf{न्ताभ्यां‚{\tiny $_{५}$}‚ प्र‚द‚र्श्य‚ते} ॥
	{\color{gray}{\rmlatinfont\textsuperscript{§~\theparCount}}}
	\pend% ending standard par
      ‚{\tiny $_{lb}$}‚

	  
	  \pstart \leavevmode% starting standard par
	एत‚दुक्त‚म्भ‚व‚ति । साध्य‚साध‚न‚योः प्र‚तिब‚न्ध‚ग्राह‚क‚मेव प्र‚माणं व्याप्तिग्राह‚क‚{\tiny $_{lb}$}‚न्तेनैव साध‚न‚स्य साध्याय‚त्त‚ताग्र‚ह‚णात् साध्याभावेऽभावो गृहीत एव केव‚ल‚न्त‚द‚{\tiny $_{lb}$}‚विनाभाव‚ग्राह‚कं प्र‚माणं विस्मृत‚त्वाद् दृष्टान्ताभ्यामुप‚द‚र्श्य‚ते [।] य‚त‚श्च प्र‚माण‚{\tiny $_{lb}$}‚ख्याप‚नादेवाविनाभाव‚स्मृत्या साध्याभावे साध‚नाभावो निश्चितो भ‚व‚ति [।]
	{\color{gray}{\rmlatinfont\textsuperscript{§~\theparCount}}}
	\pend% ending standard par
      ‚{\tiny $_{lb}$}‚

	  
	  \pstart \leavevmode% starting standard par
	\textbf{त‚स्माद् वैध‚र्म्य‚दृष्टान्ते} । त‚द्विष‚ये\textbf{ऽव‚श्यं} निय‚{\tiny $_{६}$}‚मेन इह हेतौ कार्य‚स्व‚भाव‚ल‚क्ष‚ण‚{\tiny $_{lb}$}‚ \textbf{आश्र‚यो} व‚स्तुभूतो ध‚र्मी \textbf{नेष्टः} स्व‚भावानुप‚ल‚म्भे त्विष्ट एव ।
	{\color{gray}{\rmlatinfont\textsuperscript{§~\theparCount}}}
	\pend% ending standard par
      ‚{\tiny $_{lb}$}‚

	  
	  \pstart \leavevmode% starting standard par
	त‚त्र हि विप‚र्य‚येणोप‚ल‚म्भः ख्याप‚नीयः । किङ्कार‚ण‚म् [।] \textbf{आश्र‚यो नेष्ट} इत्याह ।‚{\tiny $_{lb}$}‚ \quotelemma{त‚द‚भावे चे}त्यादि । \quotelemma{त‚द‚भावे} व्याप‚क‚कार‚ण‚योर‚भावे त‚द्व्याप्य‚कार्याख्यं लिङ्गं \quotelemma{नेति}‚{\tiny $_{lb}$}‚ न भ‚व‚तीत्येवं वैध‚र्म्य\quotelemma{व‚च‚नाद्} अप्याश्र‚य‚र‚हिता\quotelemma{त्त‚द्ग‚ते}र्व्य‚तिरेक‚ग‚तेः \textenglish{See →} ॥ २८ ॥
	{\color{gray}{\rmlatinfont\textsuperscript{§~\theparCount}}}
	\pend% ending standard par
      ‚{\tiny $_{lb}$}‚

	  
	  \pstart \leavevmode% starting standard par
	\leavevmode\ledsidenote{\textenglish{29b/PSVTa}} किङ्कार‚णं । \textbf{य‚त} इत्यादि । स्व‚भाव‚हेतौ सा‚{\tiny $_{७}$}‚ध्य‚स्य \textbf{त‚द्भा}वः साध‚न‚व्या‚{\tiny $_{lb}$}‚प‚क‚त्वं । कार्य‚हेतौ साध्य‚स्य \textbf{हेतुभावः} कार‚ण‚त्वं \textbf{ख्याप्य‚ते । त‚द‚वेदिन} इति त‚द्‚{\tiny $_{lb}$}‚भाव‚हेतुभावावेदिनः पुंसः ॥
	{\color{gray}{\rmlatinfont\textsuperscript{§~\theparCount}}}
	\pend% ending standard par
      ‚{\tiny $_{lb}$}‚

	  
	  \pstart \leavevmode% starting standard par
	त‚द्व्याच‚ष्टे । \textbf{दृष्टान्ते ही}त्यादि । \textbf{साध्य‚ध‚र्म‚स्य त‚द्भावः} साध‚न\textbf{स्व‚भाव‚त्वं‚{\tiny $_{lb}$}‚ ख्याप्य‚ते त‚न्मात्रानुब‚न्धेन} । साध‚न‚मात्रानुब‚न्धेन । \textbf{कृत‚क\textbf{त्व}निष्प‚त्तावेव}‚{\tiny $_{lb}$}‚ निष्प‚न्न‚स्यानित्य‚त्व‚स्य कृत‚क‚मात्रानुब‚न्धेन या \textbf{त‚त्स्व‚भाव‚ता} साध‚न‚स्व‚भाव‚ता‚{\tiny $_{lb}$}‚ त‚या । एवंभू‚{\tiny $_{१}$}‚त‚या त‚द्भावः \textbf{ख्याप्य‚ते} । न तु निमित्तान्त‚रात् प‚श्चाद् उत्प‚द्य‚{\tiny $_{lb}$}‚‚{\tiny $_{lb}$}‚ \leavevmode\ledsidenote{\textenglish{77/s}}मानेनानित्य‚त्वेन । \textbf{य‚थैके} विप्र‚तिप‚न्ना इति । त‚न्मात्र‚व‚त्व‚मेव द‚र्श‚य‚न्नाह । \textbf{य}‚{\tiny $_{lb}$}‚ इत्यादि । \textbf{यो} हेतुः \textbf{कृत‚कं-स्व‚भा}वें \textbf{ज‚न‚य‚ति सोऽनित्य‚रूप‚मेव स‚न्तं ज‚न‚य‚तीत्}य‚र्थः ।‚{\tiny $_{lb}$}‚ इति अनेन द्वारेणाविनाभाव‚विष‚यं \textbf{प्र‚माणं दृष्टान्तेन ख्याप्य‚ते} । न तु द‚र्श‚ना‚{\tiny $_{lb}$}‚र्श‚न‚मात्रं । \textbf{अन्य‚थेति} य‚द्येवं प्र‚माणं नोप‚{\tiny $_{२}$}‚द‚र्श्य‚ते । त\textbf{दैक‚ध‚र्म‚स‚द्भावात्} साध‚न‚{\tiny $_{lb}$}‚ध‚र्म‚स‚द्भावात् । \textbf{त‚द‚न्येनापि} साध्य‚ध‚र्मेणापि \textbf{भ‚वित‚व्य‚मिति निय‚माभावा}त् साध‚न‚स्य‚{\tiny $_{lb}$}‚ साध्य\textbf{व्य‚भिचाराशंका स्यात्} ।
	{\color{gray}{\rmlatinfont\textsuperscript{§~\theparCount}}}
	\pend% ending standard par
      ‚{\tiny $_{lb}$}‚

	  
	  \pstart \leavevmode% starting standard par
	य‚दि नाम दृष्टान्तेन प्र‚माण‚मुप‚द‚र्श‚नीय‚न्त‚थापि किं सिद्ध‚मित्याह । \textbf{तेन च‚{\tiny $_{lb}$}‚ प्र‚माणेन} त‚न्\textbf{मात्रानुब‚न्ध} इति साध‚न‚मात्रानुब‚न्धः । क‚थं ख्याप्य‚त इत्याह । \textbf{कृत‚क}‚{\tiny $_{lb}$}‚स्य य‚त्\textbf{कार‚ण}न्त‚स्मा\textbf{देव} कृत‚क\textbf{स्त‚था जातो}‚{\tiny $_{३}$}‚ जातो यो \textbf{न‚श्व‚रः} क्ष‚ण‚स्थितिध‚र्मा ।‚{\tiny $_{lb}$}‚ क्ष‚णिक‚त्वेनैव न‚श्व‚रो न तु कालान्त‚रं स्थित्वेत्य‚र्थः ।
	{\color{gray}{\rmlatinfont\textsuperscript{§~\theparCount}}}
	\pend% ending standard par
      ‚{\tiny $_{lb}$}‚

	  
	  \pstart \leavevmode% starting standard par
	क‚थं पुनः स्व‚हेतोरेव त‚थोत्प‚न्न इत्याह । \textbf{अन्य‚त} इति स्व‚हेतोर‚न्य‚स्माद्विनाश‚{\tiny $_{lb}$}‚हेतोः । त‚स्य कृत‚क‚स्य \textbf{त‚द्भाव‚निषेधाद्} अनित्य‚तास्व‚भाव‚निषेधाद् व‚क्ष्य‚माण‚{\tiny $_{lb}$}‚कात् ॥ \textbf{हेतुभावो} वा । तेन च प्र‚माणेन ख्याप्य‚त इति स‚म्ब‚न्धः । साध्य‚ध‚र्म‚स्य‚{\tiny $_{lb}$}‚ साध‚नंप्र‚ति \textbf{हेतुभावो वा} कार‚ण‚{\tiny $_{४}$}‚त्व‚म्वा ख्याप्य‚ते । त‚स्मिन् \textbf{स‚त्येव} साध‚न‚स्य \textbf{भा}वा‚{\tiny $_{lb}$}‚\textbf{दित्य}नेन प्र‚कारेण प्र‚माणं दृष्टान्तेन \textbf{प्र‚द‚र्श्य‚ते} । क‚स्य पुनः साध्य‚स्य हेतुभावः प्र‚द‚{\tiny $_{lb}$}‚र्श्य‚त इत्याह [।] \textbf{अर्थान्त‚र‚स्य} साध‚नाद् व्य‚तिरिक्त‚स्य । \textbf{त‚था} दृष्टान्तोप‚द‚र्शितेन‚{\tiny $_{lb}$}‚ प्र‚माणेन प्र‚सिद्धे \textbf{त‚द्भाव‚हेतुभावे} । स्व‚भाव‚स्य साध्य‚स्य त‚द्भावे । साध‚न‚स्व‚भाव‚त्वे ।‚{\tiny $_{lb}$}‚ कार‚ण‚स्य हेतुभावे \textbf{प्र‚सिद्धे स‚ति । द‚ह‚नाभावे धूमो} न भ‚व‚ती‚{\tiny $_{५}$}‚ति प्र‚कृतेन स‚म्ब‚न्धः ।‚{\tiny $_{lb}$}‚ \textbf{स} इत्य‚नित्य‚स्व‚भावो व‚ह्निश्च । त‚स्येति कृत‚क‚त्व‚स्य धूम‚स्य च य‚थाक्र‚मं \textbf{स्व‚भावो}‚{\tiny $_{lb}$}‚ हेतुर्वा [।] वा श‚ब्दः स‚मुच्च‚ये । य‚त एवं \textbf{क‚थ}म‚सौ कृत‚को धूमो वा \textbf{स्वं स्व‚भा}व‚म‚नित्यं‚{\tiny $_{lb}$}‚ \textbf{हेतुं चाग्निम‚न्त‚रेण} भ‚वेत् [।] नैव \textbf{भ‚वे}दित्येव‚म‚नुद्दिष्ट‚रूपे विष‚ये व्य‚तिरेके‚{\tiny $_{lb}$}‚  \leavevmode\ledsidenote{\textenglish{78/s}}क‚थ्य‚माने \textbf{आश्र‚य‚म‚न्त‚रेणापि वैध‚र्म्य‚दृष्टान्ते प्र‚सिध्य‚ति व्य‚तिरेकः} ॥
	{\color{gray}{\rmlatinfont\textsuperscript{§~\theparCount}}}
	\pend% ending standard par
      ‚{\tiny $_{lb}$}‚

	  
	  \pstart \leavevmode% starting standard par
	तेन य‚दुच्य‚ते‚{\tiny $_{६}$}‚ भ ट्टो द्यो त क रा भ्यां ।‚{\tiny $_{lb}$}‚ 
	    \pend% close preceding par
	  
	    
	    \stanza[\smallbreak]
	  {\normalfontlatin\large ``\qquad}व्य‚तिरेकोपि लिंग‚स्य विप‚क्षान्नैव ल‚भ्य‚ते ।&‚{\tiny $_{lb}$}‚अभावे स न ग‚म्येत कृत‚य‚त्नैर‚बोध‚नादि ति [।]{\normalfontlatin\large\qquad{}"}\&[\smallbreak]
	  
	  
	  
	    \pstart  \leavevmode% new par for following
	    \hphantom{.}
	  ‚{\tiny $_{lb}$}‚ त‚न्निर‚स्तं ॥
	{\color{gray}{\rmlatinfont\textsuperscript{§~\theparCount}}}
	\pend% ending standard par
      ‚{\tiny $_{lb}$}‚

	  
	  \pstart \leavevmode% starting standard par
	एव‚न्ताव‚त् [।] त‚द्भाव‚हेतुभाव‚ख्याप‚नाय त‚द‚वेदिनः दृष्टान्तो व‚क्त‚व्यः ॥
	{\color{gray}{\rmlatinfont\textsuperscript{§~\theparCount}}}
	\pend% ending standard par
      ‚{\tiny $_{lb}$}‚

	  
	  \pstart \leavevmode% starting standard par
	\textbf{येषां} पुनः पूर्वं \textbf{प्र‚सिद्धावेव त‚द्भाव‚हेतुभावौ} य‚था-स्वं प्र‚माणेन प‚क्ष‚ध‚र्म‚मात्र‚त्वं‚{\tiny $_{lb}$}‚ \leavevmode\ledsidenote{\textenglish{30a/PSVTa}} निश्चित\textbf{न्तेषा}न्त‚द्भाव‚हेतुभावंप्र‚ति \textbf{विदुषां हेतुरेव ॥ य‚द‚र्थ}‚{\tiny $_{७}$}‚मित्य‚न्व‚य‚व्य‚तिरेक‚{\tiny $_{lb}$}‚निश्च‚यार्थं । प्र‚तिपाद्य‚स्य स्व‚य‚मेव \textbf{सोर्थः सिद्ध इति} किन्त‚द्व‚च‚नेन । \textbf{त‚देति} निश्चि‚{\tiny $_{lb}$}‚तान्व‚य‚व्य‚तिरेक‚काले ।
	{\color{gray}{\rmlatinfont\textsuperscript{§~\theparCount}}}
	\pend% ending standard par
      ‚{\tiny $_{lb}$}‚

	  
	  \pstart \leavevmode% starting standard par
	य‚द\textbf{पि} मूढं प्र‚ति दृष्टान्त‚प्र‚द‚र्श‚नं क्रिय‚ते त‚दा \textbf{त‚त्प्र‚द‚र्श‚ने}पि दृष्टान्त‚प्र‚द‚र्श‚नेपि‚{\tiny $_{lb}$}‚ वैध‚र्म्यं । विनाप्याश्र‚येण य‚थोक्त‚विधिना सिध्य‚त्येव व्य‚तिरेकः । त‚तः \textbf{किम्वैध‚र्म्य‚{\tiny $_{lb}$}‚दृष्टान्ताश्र‚येणेति म‚न्य‚मान} आ चा र्य \textbf{आश्र‚यं‚{\tiny $_{१}$}‚ प्र‚तिक्षिप‚ति} न्या य मु खा दौ ।‚{\tiny $_{lb}$}‚ त‚था हि त‚त्रैवं चोदितं य‚दा त‚र्ह्याकाशादिकं नित्य‚न्ताव‚दभ्युपैति प्र‚तिवादी‚{\tiny $_{lb}$}‚ [।] त‚दा क‚थ‚न्नित्यात् कृत‚क‚त्व‚स्य व्य‚तिरेक इति [।] त‚त्रा \textbf{चा र्य} आश्र‚यं‚{\tiny $_{lb}$}‚ प्र‚तिक्षिप‚न्नाह । \textbf{त‚दा स‚न्देह} एव नास्ति त‚द‚भावात्त‚त्रावृत्तेरिति ।
	{\color{gray}{\rmlatinfont\textsuperscript{§~\theparCount}}}
	\pend% ending standard par
      ‚{\tiny $_{lb}$}‚

	  
	  \pstart \leavevmode% starting standard par
	एत‚दुक्त‚म्भ‚व‚ति । गृहीत‚प्र‚तिब‚न्ध‚स्य त‚त्राकाशादौ व्याप‚काभावाद् व्याप्या‚{\tiny $_{lb}$}‚भाव‚सिद्धेः । अनित्याभाव‚श्च नित्य‚स्यास‚{\tiny $_{२}$}‚त्वात् सिद्ध इति याव‚त् ॥ य‚स्माद‚{\tiny $_{lb}$}‚ दृष्टान्ताभ्यां प्र‚तिब‚न्धः क‚थ्य‚ते । \textbf{तेन} कार‚णेन \textbf{ज्ञात‚स‚म्ब‚न्धे} हेतौ स‚ति \textbf{द्व‚योः} साध‚{\tiny $_{lb}$}‚र्म्य‚वैध‚र्म्य‚दृष्टान्त‚यो\textbf{र‚न्य‚त‚रोक्तितः । द्वितीयेपि} ताभ्यामेवान्य‚त‚र‚स्मिन्न‚नुक्तेपि‚{\tiny $_{lb}$}‚ \textbf{स्मृतिः स‚मुप‚जाय‚तेऽर्थाप‚त्त्या} ।
	{\color{gray}{\rmlatinfont\textsuperscript{§~\theparCount}}}
	\pend% ending standard par
      ‚{\tiny $_{lb}$}‚‚{\tiny $_{lb}$}‚\textsuperscript{\textenglish{79/s}}

	  
	  \pstart \leavevmode% starting standard par
	एत‚द‚प्याचार्य‚व‚च‚नेन संस्य‚न्द‚य‚न्नाह । \textbf{य‚दाहे}त्यादि [।] न्या य मु खे चायं‚{\tiny $_{lb}$}‚ ग्र‚न्थः । वाश‚ब्द‚स्त‚त्रैव पूर्व‚विक‚ल्पापेक्षी ।‚{\tiny $_{३}$}‚ \textbf{अन्य‚त‚रे}णेति साध‚र्म्य‚दृष्टान्तेन वैध‚र्म्य‚{\tiny $_{lb}$}‚दृष्टान्तेन वा । \textbf{उभ‚य‚प्र‚द‚र्श‚ना}द् अन्व‚य‚व्य‚तिरेक‚प्र‚द‚र्श‚नात् । \textbf{त‚त्रापि} ग्र‚न्थे । \textbf{दृष्टान्तेन}‚{\tiny $_{lb}$}‚ स्व‚भाव‚हेतौ कार्यंहेतौ च य‚था क्र‚म\textbf{न्त‚द्भाव‚हेतुभाव‚प्र‚द‚र्श‚नं} क्रिय‚त इति \textbf{म‚न्य‚मान}‚{\tiny $_{lb}$}‚ आचा\textbf{र्योर्थाप‚त्त्}या \textbf{एक}स्यान्व‚य‚स्य व्य‚तिरेक‚स्य वा \textbf{व‚च‚नेन} द्वितीय‚स्य य‚थाक्र‚मं‚{\tiny $_{lb}$}‚ व्य‚तिरेक‚स्यान्व‚य‚स्य वा \textbf{सिद्धिमा}ह । एत‚देवा‚{\tiny $_{४}$}‚ह । \textbf{त‚था ही}ति । \textbf{य}त्किञ्चित् \textbf{कृत‚क‚{\tiny $_{lb}$}‚न्त‚द‚नित्य‚मेवेत्युक्ते व्य‚क्त‚म}व‚श्य\textbf{म‚य}म‚नित्य‚त्वाख्यो ध‚र्मो\textbf{स्य} कृत‚क‚स्य \textbf{स्व‚भाव‚स्त‚{\tiny $_{lb}$}‚न्मात्रानुब‚न्धी} कृत‚क‚मात्रानुब‚न्धी \textbf{प्र‚माण‚दृष्ट इति} प्र‚माणेन \textbf{निश्चितः} ।
	{\color{gray}{\rmlatinfont\textsuperscript{§~\theparCount}}}
	\pend% ending standard par
      ‚{\tiny $_{lb}$}‚

	  
	  \pstart \leavevmode% starting standard par
	न‚नु कार्येपि कार‚ण‚म‚व‚श्य‚म्भ‚व‚ति । न च त‚त्त‚स्य स्व‚भाव इत्य‚त आह ।‚{\tiny $_{lb}$}‚ अन‚र्थान्त‚र इति । क‚थ‚न्त‚न्मात्रानुब‚न्धीत्याह । \textbf{त‚द्भाव‚निय}मादिति । कृत‚क‚भावे‚{\tiny $_{५}$}‚‚{\tiny $_{lb}$}‚ऽव‚श्य‚म‚नित्य‚ताभावादित्य‚र्थः ॥
	{\color{gray}{\rmlatinfont\textsuperscript{§~\theparCount}}}
	\pend% ending standard par
      ‚{\tiny $_{lb}$}‚

	  
	  \pstart \leavevmode% starting standard par
	न‚नु न कृत‚क‚मात्रानुब‚न्धी स्व‚भावो नित्य‚त्व‚स्य प्र‚त्य‚क्ष‚निश्चितः क्ष‚णिकोय‚{\tiny $_{lb}$}‚मित्य‚निश्च‚यात्त‚त्क‚थ‚मुच्य‚ते प्र‚माण‚दृष्ट इति । अथ कृत‚को विनाशं प्र‚त्य‚न‚पेक्ष‚त्वा‚{\tiny $_{lb}$}‚त्त‚द्भाव‚निय‚त इत्य‚नुमान‚दृष्टः [।]
	{\color{gray}{\rmlatinfont\textsuperscript{§~\theparCount}}}
	\pend% ending standard par
      ‚{\tiny $_{lb}$}‚

	  
	  \pstart \leavevmode% starting standard par
	त‚द‚युक्तं [।] य‚तो निर्हेतुकेपि विनाशे य‚दैव घ‚टादेर्नाशः प्र‚तीयेत त‚दैवाहेतुकः‚{\tiny $_{lb}$}‚ स्यान्नान्य‚दा । त‚त्क‚थं‚{\tiny $_{६}$}‚ क्ष‚णिक‚त्वं । अथैक‚क्ष‚ण‚स्थायित्वेन घ‚ट‚स्योत्प‚त्तेः पूर्व‚म‚पि‚{\tiny $_{lb}$}‚ नाशः [।]
	{\color{gray}{\rmlatinfont\textsuperscript{§~\theparCount}}}
	\pend% ending standard par
      ‚{\tiny $_{lb}$}‚

	  
	  \pstart \leavevmode% starting standard par
	न‚नु य‚थैक‚क्ष‚ण‚स्थायित्वेनोत्प‚त्तिः स्व‚हेतुभ्य‚स्त‚थानेक‚क्ष‚ण‚स्थायित्वेनाप्युत्प‚त्तिः‚{\tiny $_{lb}$}‚ स्यात् । विचित्र‚श‚क्त‚यो हि साम‚ग्र्यो दृश्य‚न्ते । न च य‚दि विनाशः क्व‚चित् क‚दाचिद्‚{\tiny $_{lb}$}‚ भ‚वेत् [।] त‚त्काल‚द्र‚व्यापेक्ष‚त्वाद् अस्यान‚पेक्ष‚त्व‚हानिः । विनाश‚क‚हेत्व‚न‚पेक्ष‚त्वाद्‚{\tiny $_{lb}$}‚ अन्य‚{\tiny $_{७}$}‚था द्वितीयेपि क्ष‚णे विनाशो न स्यात् त‚त्कालाद्य‚पेक्ष‚त्वात् । अथ क्र‚म‚यौग‚प‚द्याभ्यां \leavevmode\ledsidenote{\textenglish{30b/PSVTa}}‚{\tiny $_{lb}$}‚ साम‚र्थ्य‚ल‚क्ष‚णं स‚त्त्वं व्याप्तं । नित्येषु च क्र‚माक्र‚म‚निवृत्तौ स‚त्त्वं निव‚र्त्त‚मानं‚{\tiny $_{lb}$}‚ क्ष‚णिकेष्वेवाव‚तिष्ठ‚त इति स‚त्त्व‚युक्त‚स्य कृत‚क‚स्य ग‚म‚क‚त्वं ।
	{\color{gray}{\rmlatinfont\textsuperscript{§~\theparCount}}}
	\pend% ending standard par
      ‚{\tiny $_{lb}$}‚

	  
	  \pstart \leavevmode% starting standard par
	त‚द‚प्य‚युक्तं । क्ष‚णिक‚त्वे स‚ति क्र‚माप्र‚तिप‚त्तेर्येन हि ज्ञान‚क्ष‚णेन पूर्व‚क‚म्व‚स्तु‚{\tiny $_{lb}$}‚ प्र‚तिप‚न्नं न तेनोत्त‚रं येनोत्त‚रं न तेन पूर्व‚क‚मिति क‚थं क्र‚म‚प्र‚तीतिः ।‚{\tiny $_{१}$}‚ यो हि पूर्व‚{\tiny $_{lb}$}‚  \leavevmode\ledsidenote{\textenglish{80/s}}व‚स्तुप्र‚तिप‚त्य‚न‚न्त‚र‚म‚प‚र‚स्य ग्राह‚कः स क्र‚म‚ग्राही स्यात् त‚था वा-क्ष‚णिक‚त्व‚म‚स्य‚{\tiny $_{lb}$}‚ स्यात् । य‚स्य च बौ द्ध स्य काल एव नास्ति त‚स्य क‚थं क्र‚म‚ग्र‚हः । भिन्न‚काल‚व‚स्त्व‚{\tiny $_{lb}$}‚ग्र‚हात् । कालाभावे चानेक‚व‚स्तुरूप एव क्र‚मः । त‚था च नित्य‚स्यापि क्र‚म‚क‚र्त्तृत्वं‚{\tiny $_{lb}$}‚ न विरुध्य‚ते । य‚था च नित्य‚स्य क्र‚म‚क‚र्त्तृत्वाद‚नेक‚रूप‚त्व‚न्त‚था क्ष‚ण‚स्यापि स्यात् ॥‚{\tiny $_{lb}$}‚ अथ क्ष‚ण‚व‚द् द्वितीये क्ष‚{\tiny $_{२}$}‚णे नित्य‚स्याप्य‚भावः स्यात् । कार्याभावात् ।
	{\color{gray}{\rmlatinfont\textsuperscript{§~\theparCount}}}
	\pend% ending standard par
      ‚{\tiny $_{lb}$}‚

	  
	  \pstart \leavevmode% starting standard par
	त‚द‚युक्तं कालाभावात् । भ‚व‚तु वा क्र‚म‚ग्र‚ह‚स्त‚थापि क‚थं क्र‚माक्र‚माभ्यां स‚त्त्व‚स्य‚{\tiny $_{lb}$}‚ व्याप्तिः । क्र‚म‚यौग‚प‚द्य‚व्य‚तिरेकेणान्येन प्र‚कारेणार्थ‚क्रियास‚म्भ‚वात् । न च प्र‚का‚{\tiny $_{lb}$}‚रान्त‚र‚स्य दृश्यानुप‚ल‚म्भाद‚भाव‚निश्च‚यः । एवं हि विशिष्ट‚देशादावेवाभाव‚निश्च‚यः‚{\tiny $_{lb}$}‚ स्यान्न स‚र्व‚दा । नाप्य‚दृश्यानुप‚ल‚म्भाद् अभाव‚निश्च‚यः स‚न्देहात्‚{\tiny $_{३}$}‚ । त‚स्मान्नित्येषु‚{\tiny $_{lb}$}‚ क्र‚माक्र‚मायोगेपि स‚त्त्वानिवृत्तेः क‚थं स‚त्त्व‚स्य क्ष‚णिक‚स्व‚भाव‚त्व‚मिति शं क र प्र‚भृत‚यः ।
	{\color{gray}{\rmlatinfont\textsuperscript{§~\theparCount}}}
	\pend% ending standard par
      ‚{\tiny $_{lb}$}‚

	  
	  \pstart \leavevmode% starting standard par
	भ‚व‚तु वा प्र‚कारान्त‚राभावात् क्र‚म‚यौग‚प‚द्याभ्यां स‚त्त्व‚स्य व्याप्तिस्त‚थापि नि‚{\tiny $_{lb}$}‚त्येषु न प्र‚त्य‚क्षादिना क्र‚माक्र‚मायोगः सिद्धो नित्त्यानाम‚तीन्द्रिय‚त्वात् [।] त‚द‚सिद्धौ‚{\tiny $_{lb}$}‚ च न तेषु स‚त्त्व‚निवृत्तिसिद्धिस्त‚द‚सिद्धौ च न स‚त्व‚स्य क्ष‚णिक‚स्व‚भाव‚त्व‚सिद्धिः ।
	{\color{gray}{\rmlatinfont\textsuperscript{§~\theparCount}}}
	\pend% ending standard par
      ‚{\tiny $_{lb}$}‚

	  
	  \pstart \leavevmode% starting standard par
	किञ्च‚{\tiny $_{४}$}‚[।]स‚त्त्वात् क्र‚म‚यौग‚प‚द्यानुमानं स्यात् तेनैव व्याप्त‚त्वान्न तु क्ष‚णिक‚त्वा‚{\tiny $_{lb}$}‚नुमान‚न्त‚त्र क्र‚म‚क‚र्त्तृत्वास‚म्भ‚वादिति ।
	{\color{gray}{\rmlatinfont\textsuperscript{§~\theparCount}}}
	\pend% ending standard par
      ‚{\tiny $_{lb}$}‚

	  
	  \pstart \leavevmode% starting standard par
	अत्रोच्य‚ते । क्र‚म‚यौग‚प‚द्ये प्र‚त्य‚क्ष‚सिद्धे एव । स‚ह‚भावो हि भावानां यौग‚प‚द्यं‚{\tiny $_{lb}$}‚ क्र‚म‚स्तु पूर्वाप‚र‚भावः स च क्र‚मिणाम‚भिन्न‚स्त‚त्प्र‚तिभास‚श्चैक‚प्र‚तिभासः । स त्वेक‚{\tiny $_{lb}$}‚प्र‚तिभासान‚न्त‚र‚म‚प‚र‚स्य प्र‚तिभासः । क्र‚म‚प्र‚तिभासो न त्वेक‚स्यैवातिप्र‚स‚ङ्गात् ।
	{\color{gray}{\rmlatinfont\textsuperscript{§~\theparCount}}}
	\pend% ending standard par
      ‚{\tiny $_{lb}$}‚

	  
	  \pstart \leavevmode% starting standard par
	स‚त्यं ‚{\tiny $_{५}$}‚[।] त‚त्रापि य‚दैक‚स्य प्र‚तिभासो न त‚दाप‚र‚स्य त‚द्भावे हि यौग‚प‚द्य‚{\tiny $_{lb}$}‚प्र‚तिभासः स्यात् । त‚स्मात् क्र‚मिणोः पूर्वोत्त‚राभ्यां ज्ञानाभ्यां ग्र‚हे क्र‚मो गृहीत एव‚{\tiny $_{lb}$}‚ त‚तोऽभेदात् । केव‚लं पूर्वानुभूत‚व‚स्त्वाहित‚संस्कार‚प्र‚बोधेनेद‚म‚स्माद‚न‚न्त‚र‚मित्या‚{\tiny $_{lb}$}‚नुपूर्वीविक‚ल्पोत्प‚त्त्या क्र‚म‚ग्र‚हो व्य‚व‚स्थाप्य‚ते । क्र‚मिणां ग्र‚हेपि क‚थंचिदानुपूर्वी‚{\tiny $_{lb}$}‚ विक‚ल्पानुत्प‚त्तौ क्र‚माग्र‚ह‚व्य‚व‚स्थाप‚नाद‚{\tiny $_{६}$}‚त एव क्र‚मिणामेक‚ग्र‚हेपि न क्र‚म‚ग्र‚ह उच्य‚ते ।
	{\color{gray}{\rmlatinfont\textsuperscript{§~\theparCount}}}
	\pend% ending standard par
      ‚{\tiny $_{lb}$}‚

	  
	  \pstart \leavevmode% starting standard par
	किं च कालाभ्युप‚ग‚म‚वादिनोपि क‚थं क्र‚म‚ग्र‚हः । एक‚काल‚त्वात् स‚र्व‚कार्याणां ।‚{\tiny $_{lb}$}‚ अथ भिन्न‚काल‚कार‚णोपाधिक्र‚मात् कार्य‚क्र‚म‚स्त‚द‚युक्तं काल‚स्यैक‚त्वात । अत एव‚{\tiny $_{lb}$}‚ न नित्यंस्य भावः ।
	{\color{gray}{\rmlatinfont\textsuperscript{§~\theparCount}}}
	\pend% ending standard par
      ‚{\tiny $_{lb}$}‚

	  
	  \pstart \leavevmode% starting standard par
	अथ पूर्वाप‚र‚रूप‚त्वात् क्र‚म‚वान् कालः ।
	{\color{gray}{\rmlatinfont\textsuperscript{§~\theparCount}}}
	\pend% ending standard par
      ‚{\tiny $_{lb}$}‚

	  
	  \pstart \leavevmode% starting standard par
	न‚नु त‚स्यापि क्र‚मो य‚द्य‚प‚र‚कालापेक्ष‚स्त‚दान‚व‚स्था स्यात् । अथ त‚स्य स्व‚रूपेणा‚{\tiny $_{७}$}‚‚{\tiny $_{lb}$}‚ \leavevmode\ledsidenote{\textenglish{31a/PSVTa}} क्र‚म‚स्त‚था स‚हाय‚र‚हितानाम्ब‚हूनां कार्याणाम‚पि क्र‚मः स्यात् । अस्माक‚न्तु पूर्वा‚{\tiny $_{lb}$}‚दिप्र‚त्य‚य‚विष‚यो म‚हाभूत‚विशेषः कालो लोक‚प्र‚तीतोस्त्येव । त‚स्य च भेदात्‚{\tiny $_{lb}$}‚ क्र‚मादिप्र‚तीतिर्युज्य‚त एव । नापि प्र‚कारान्त‚रेण नित्य‚स्य क‚र्त‚त्वं स‚म्भ‚व‚ति । य‚तः‚{\tiny $_{lb}$}‚ ‚{\tiny $_{lb}$}‚ \leavevmode\ledsidenote{\textenglish{81/s}}प्र‚कारान्त‚रेणैक‚दैक‚कार्य‚क‚र‚णेऽनेक‚क‚र‚णे वान्य‚दाऽव‚स्तुत्वं स्यात् कार्याभावात् ।‚{\tiny $_{lb}$}‚ पुनः पुनः का‚{\tiny $_{१}$}‚र्य‚क‚र‚णे च क्र‚म एव न प्र‚कारान्त‚र‚स‚म्भ‚वः । अथ प्र‚कारान्त‚रेण नैक‚दा‚{\tiny $_{lb}$}‚ कार्यं क‚रोति पुनः पुन‚श्च न क‚रोति त‚दास्याव‚स्तुत्वं स्यात् । स‚र्व‚दाऽक‚र्त्तृत्वात् ।
	{\color{gray}{\rmlatinfont\textsuperscript{§~\theparCount}}}
	\pend% ending standard par
      ‚{\tiny $_{lb}$}‚

	  
	  \pstart \leavevmode% starting standard par
	त‚स्मात् क्र‚माक्र‚माभ्यां घ‚टादिर‚र्थ‚क्रियाकारी प्र‚त्य‚क्ष‚सिद्धः स एवाय‚मिति ज्ञानाद‚{\tiny $_{lb}$}‚क्ष‚णिक‚श्च प्र‚तीय‚त एव । त‚स्य च य‚दैक‚कार्य‚क‚र‚णंप्र‚ति साम‚र्थ्य‚न्त‚त्त‚दैव न पूर्वं न‚{\tiny $_{lb}$}‚ प‚श्चात्त‚त्कार्या‚{\tiny $_{२}$}‚भावात् । साम‚र्थ्यं च त‚द‚व्य‚तिरिक्त‚मेव‚मुत्त‚रोत्त‚र‚कार्योत्प‚{\tiny $_{lb}$}‚त्ताव‚पि द्र‚ष्ट‚व्यं । साम‚र्थ्य‚भेदेन च प‚दार्थ‚भेदात् क्ष‚णिक एव क्र‚माक्र‚म‚योर्निय‚मः ।‚{\tiny $_{lb}$}‚ तेन य‚त्र स‚त्त्व‚न्त‚त्र क्र‚माक्र‚म‚प्र‚तीताव‚पि क्ष‚णिक‚त्व‚प्र‚तीतिरेव । य एव क्ष‚णिके क्र‚मा‚{\tiny $_{lb}$}‚क्र‚म‚योर्निय‚मोय‚मेव नित्येषु त‚योर‚योगः । त‚स्माद् य‚देत‚द् घ‚टादौ नित्य‚त्वं प्र‚तीतं‚{\tiny $_{lb}$}‚ त‚त् स‚त्त्व‚विरुद्ध‚मिति‚{\tiny $_{३}$}‚ नित्य‚स्य क्र‚म‚यौग‚प‚द्याभ्याम‚र्थ‚क्रियाविरोधः सिद्ध उच्य‚ते ।‚{\tiny $_{lb}$}‚ य‚था च दृष्टे घ‚टादौ स‚त्त्वं क्ष‚णिक‚त्व‚व्याप्तं त‚थाऽदृष्टेष्व‚प्य‚विशेषादिति‚{\tiny $_{lb}$}‚ व्याप्तिं स‚र्वोप‚संहारेण प्र‚तिप‚द्य य‚था य‚था तेषु स‚त्त्वं निश्चीय‚ते त‚था त‚था क्ष‚णि‚{\tiny $_{lb}$}‚क‚त्वानुमानं । स‚त्त्वानिश्च‚ये तु श‚श‚विषाण‚व‚त्तेष्व‚स‚त्ताशंक‚या क्ष‚णिक‚त्वाप्र‚तीतिः‚{\tiny $_{lb}$}‚ स्यात् [।] न च त‚त्रापि बाध‚क‚प्र‚माणेनैव क्ष‚णि‚{\tiny $_{४}$}‚क‚त्व‚स्य सिद्ध‚त्वाद‚नुमान‚स्य‚{\tiny $_{lb}$}‚ वैय‚र्थ्यं । गृहीत‚व्याप्तिक‚स्य पुंसः स‚त्त्व‚निश्च‚य‚मात्रेणैव साध्यार्थाव‚ग‚तेर्बा‚{\tiny $_{lb}$}‚ध‚कोत्थान‚वैय‚र्थ्यात् । विस्त‚र‚त‚स्त्व‚यं व्याप्तिग्र‚ह‚ण‚प्र‚कारो \textbf{नैरात्म्य‚सिद्धा}व‚भिहित‚{\tiny $_{lb}$}‚ इति त‚त्रैवाव‚धार्यः ।
	{\color{gray}{\rmlatinfont\textsuperscript{§~\theparCount}}}
	\pend% ending standard par
      ‚{\tiny $_{lb}$}‚

	  
	  \pstart \leavevmode% starting standard par
	ये तु स‚त्त्व‚स्य विप‚क्षाद् अभावेन स‚र्व‚त्र क्ष‚णिक‚त्व‚व्याप्तिं प्र‚तिप‚द्य स‚त्त्वात्त‚त्रैव‚{\tiny $_{lb}$}‚ क्ष‚णिक‚त्व‚म‚नुमाप‚य‚न्ति । तेषाम‚नुमानोत्थान‚मे‚{\tiny $_{५}$}‚व न स्यात् । व्याप्तिग्र‚ह‚णादेव‚{\tiny $_{lb}$}‚ प्र‚माणात् स‚र्व‚त्र क्ष‚णिक‚त्व‚स्य सिद्ध‚त्वात् । न च ध‚र्मी सिद्धः स‚र्व‚स्य त्रैलोक्य‚स्य‚{\tiny $_{lb}$}‚ प्र‚त्य‚क्ष‚त्वाद्धेतुश्चासिद्धः । प‚क्षीकृते च स‚र्व‚स्मिन् ध‚र्मिणि बाध‚क‚व‚शाद् य‚दि विप‚क्षा‚{\tiny $_{lb}$}‚भावः सिद्ध‚स्त‚दा साध्य‚स्यापि सिद्ध‚त्वान्नानुमान‚स्योत्थानं स्यात् ।
	{\color{gray}{\rmlatinfont\textsuperscript{§~\theparCount}}}
	\pend% ending standard par
      ‚{\tiny $_{lb}$}‚

	  
	  \pstart \leavevmode% starting standard par
	नान्य‚श्च ध‚र्मी सिद्ध इति क‚थं बाध‚क‚स्य प्र‚वृत्तिरिति य‚त्किञ्चिदेत‚त् [।]‚{\tiny $_{lb}$}‚ त‚स्मात् स्थित‚मेत‚द‚{\tiny $_{६}$}‚स्य स‚त्त्व‚विशिष्ट‚स्य कृत‚क‚त्व‚स्य क्ष‚णिक‚त्वाख्यो ध‚र्मः स्व‚भाव‚{\tiny $_{lb}$}‚स्त‚न्मात्रानुब‚न्धी अस्य वा क्ष‚णिक‚त्व‚स्य स्व‚भाव‚स्त‚न्मात्रानुब‚न्धी कृत‚क‚मात्रानुब‚न्धी‚{\tiny $_{lb}$}‚ प्र‚माण‚दृष्ट इति ।
	{\color{gray}{\rmlatinfont\textsuperscript{§~\theparCount}}}
	\pend% ending standard par
      ‚{\tiny $_{lb}$}‚

	  
	  \pstart \leavevmode% starting standard par
	य‚द्वा स‚त्त्व‚विशेष‚ण‚र‚हित‚स्यापि कृत‚क‚त्वादेः क्ष‚णिक‚त्वे साध्ये नानैकान्ति‚{\tiny $_{lb}$}‚क‚त्वं । य‚त‚स्त‚स्य प्र‚थ‚मे क्ष‚णे य एव स्व‚भावः स एव चेत् द्वितीये क्ष‚णे त‚दाऽभूत्वा‚{\tiny $_{lb}$}‚ भ‚व‚न‚मेव स्यात् प्र‚थ‚म‚{\tiny $_{७}$}‚क्ष‚ण‚व‚त् । त‚त‚श्च क्ष‚णिक‚त्वं । अथ प्र‚थ‚म‚क्ष‚णे कृत‚क‚स्य \leavevmode\ledsidenote{\textenglish{41b/PSVTa}}‚{\tiny $_{lb}$}‚ ज‚न्मैव न स्थितिर्द्वितीये तु क्ष‚णे स्थितिरेव न ज‚न्म । एव‚म‚पि क्ष‚णिक‚त्वं स्यात् । ज‚न्म‚{\tiny $_{lb}$}‚ज‚न्मिनोर‚भेदात् । स्थितिस्थितिम‚तोश्च । न च द्वितीये क्ष‚णे ज‚न्म विना स्थिति‚{\tiny $_{lb}$}‚‚{\tiny $_{lb}$}‚ \leavevmode\ledsidenote{\textenglish{82/s}}र्युक्ता । ज‚न्म चेन्न त‚दा स्थितिस्त‚स्याद्वितीयादिक्ष‚ण‚भावित्वात् । त‚त्राप्येव‚मिति‚{\tiny $_{lb}$}‚ स‚र्व‚त्रोत्प‚त्तिरेव न स्थितिरिति क्ष‚णिक‚त्व‚मेव । उत्प‚त्तिश्च हेतुकृतेति त‚त्रै‚{\tiny $_{१}$}‚व कृत‚क‚त्वं ।‚{\tiny $_{lb}$}‚ न स्थितौ । त‚स्मात् कृत‚क‚त्व‚स्याक्ष‚णिक‚त्व‚विरुद्ध‚त्वान्नानैकान्तिक‚त्वं । विस्त‚रेण चास्य‚{\tiny $_{lb}$}‚ कृत‚क‚त्व‚स्य क्ष‚णिक‚त्व‚व्याप्ति\textbf{र्नै रा त्म्य सि द्धा} व\edtext{}{\edlabel{pvsvt_82-1}\label{pvsvt_82-1}\lemma{व}\Bfootnote{\href{http://sarit.indology.info/?cref=pv.1.224}{ Pramāṇavārtika 1:224. }}}भिहितेति त‚त्रैवाव‚ग‚न्त‚व्या ।
	{\color{gray}{\rmlatinfont\textsuperscript{§~\theparCount}}}
	\pend% ending standard par
      ‚{\tiny $_{lb}$}‚

	  
	  \pstart \leavevmode% starting standard par
	त‚स्मात् स‚त्त्वान‚न्त‚र्भूत‚स्यापि कृत‚क‚त्व‚स्य व्याप्तिः प्र‚माण‚दृष्टा । त‚दाहास्य‚{\tiny $_{lb}$}‚ स्व‚भाव‚स्त‚न्मात्रानुब‚न्धी प्र‚माण‚दृष्ट‚स्त‚द्भाव‚निय‚मात् । कृत‚क‚भावेऽव‚श्य‚म‚नित्य‚{\tiny $_{lb}$}‚ताभाव‚निय‚मादिति ।
	{\color{gray}{\rmlatinfont\textsuperscript{§~\theparCount}}}
	\pend% ending standard par
      ‚{\tiny $_{lb}$}‚

	  
	  \pstart \leavevmode% starting standard par
	\textbf{एवं ज्ञात‚त‚{\tiny $_{२}$}‚द्भाव‚स्}यानित्य‚स्व‚भावं कृत‚कं ज्ञात‚व‚तः पुंसो\textbf{र्थाप‚त्त्या} साध्य‚स्या‚{\tiny $_{lb}$}‚\textbf{नित्य‚त्व‚स्याभावे कृत‚क‚न्न भ‚व‚ती}त्येवंनिश्च‚यो \textbf{भ‚व}तीति । य‚स्मा\textbf{न्न‚हि} स्व‚स्यात्म‚{\tiny $_{lb}$}‚भूत‚स्यानित्य‚त्व‚स्या\textbf{भावे भावो भ}व‚ति । त‚दात्म‚भूतं कृत‚क‚त्व‚म्भ‚व‚ति । किं कार‚ण‚{\tiny $_{lb}$}‚म् [।] \textbf{अभेदात्} साध्य‚साध‚न‚योः । \textbf{अन्य‚थे}ति य‚द्य‚नित्य‚त्वाभावे कृत‚क‚त्व‚म्भ‚वेत् ।‚{\tiny $_{lb}$}‚ त‚दा \textbf{त‚द्भावे} कृत‚क‚त्वाभावेऽव‚श्य‚म‚नित्य‚त्व\textbf{म्भ‚व‚ती‚{\tiny $_{३}$}‚त्येव न स्यात् ॥ त‚थे}ति य‚था‚{\tiny $_{lb}$}‚ साध‚र्म्य‚प्र‚योगे वैध‚र्म्य‚ग‚तिस्त‚था वैध‚र्म्य‚प्र‚योगेन्व‚य‚स्मृतिर्भ‚व‚तीति व‚क्ष्य‚माणेन‚{\tiny $_{lb}$}‚ स‚म्ब‚न्धः । त‚द‚भाव इति नित्य‚त्वाभावे कृत‚क‚त्व‚न्न भ‚व‚त्य‚व‚श्य‚मित्युक्ते । त‚त‚{\tiny $_{lb}$}‚ एव वैध‚र्म्य‚प्र‚योगात् \textbf{त‚द्भाव‚तावेदिनः} । साध‚न‚स्य साध्य‚स्व‚भाव‚तावेदिनः पुंसः ।‚{\tiny $_{lb}$}‚ क‚थ‚न्त‚द्भाव‚तां वेत्तीति चेदाह । \textbf{त‚था ही}त्यादि । \textbf{अय}म‚नित्य‚त्वाख्यो ध‚र्मोस्य‚{\tiny $_{lb}$}‚ कृ‚{\tiny $_{४}$}‚त‚क‚त्व‚स्य \textbf{स्व‚भा}वो येन \textbf{त‚द‚भावे} कृत‚क‚त्व‚न्न \textbf{भ‚व}ति । \textbf{अन्य}थेति [।] अनित्य‚त्वं‚{\tiny $_{lb}$}‚ य‚दि कृत‚क‚स्य स्व‚भावो न भ‚वेत् । त‚दा त‚द‚भावे कृत‚क‚त्व‚न्न भ‚वेदित्य‚स्या\textbf{योगात् ।‚{\tiny $_{lb}$}‚ इति} हेतोस्त‚त्स्व\textbf{भाव‚ताप्र‚तिप‚त्त्या} कृत‚क‚स्यानित्य‚स्व‚भाव‚तोप्र‚तिप‚त्त्या\textbf{न्व‚य‚स्मृति‚{\tiny $_{lb}$}‚र्भ‚व‚ति} ।
	{\color{gray}{\rmlatinfont\textsuperscript{§~\theparCount}}}
	\pend% ending standard par
      ‚{\tiny $_{lb}$}‚

	  
	  \pstart \leavevmode% starting standard par
	एत‚दुक्त‚म्भ‚व‚ति । य एव हेतोः साध्य एव भावः स एव विप‚क्षेऽभाव इत्य‚न्व‚य‚{\tiny $_{lb}$}‚व्य‚तिरेक‚योस्ता‚{\tiny $_{५}$}‚दात्म्य‚म‚न्योन्य‚व्याप्तिश्चातो हेताव‚न्व‚य‚प्र‚तीत्या व्य‚तिरेक‚प्र‚ती‚{\tiny $_{lb}$}‚तिर्व्य‚तिरेक‚प्र‚तीत्या चान्व‚य‚प्र‚तीतिर‚नुमान‚मेव ।
	{\color{gray}{\rmlatinfont\textsuperscript{§~\theparCount}}}
	\pend% ending standard par
      ‚{\tiny $_{lb}$}‚

	  
	  \pstart \leavevmode% starting standard par
	तेन य‚दुच्य‚ते ।
	{\color{gray}{\rmlatinfont\textsuperscript{§~\theparCount}}}
	\pend% ending standard par
      ‚{\tiny $_{lb}$}‚‚{\tiny $_{lb}$}‚‚{\tiny $_{lb}$}‚\textsuperscript{\textenglish{83/s}}

	  
	  \pstart \leavevmode% starting standard par
	\hphantom{.}साम‚र्थ्य‚मिच्छ‚तः की र्त्ते र्न ष्टं द्वित्वाव‚धार‚ण‚मिति\href{http://sarit.indology.info/?cref=ny\%C4\%81ma}{न्याय‚मंज‚री ?}त‚द‚पास्तं ।
	{\color{gray}{\rmlatinfont\textsuperscript{§~\theparCount}}}
	\pend% ending standard par
      ‚{\tiny $_{lb}$}‚

	  
	  \pstart \leavevmode% starting standard par
	य‚द्वा नैवैत‚त् प्र‚माणं केव‚लं संकेत‚व‚शाद् अन्व‚य‚मुखेन व्य‚तिरेक‚मुखे\textbf{ण}\edtext{}{\lemma{मुखे}\Bfootnote{? न}} वा‚{\tiny $_{lb}$}‚ प्र‚युक्त‚मेकं वाक्य‚मुभ‚यं ग‚म‚य‚तीत्य‚दोषः ।
	{\color{gray}{\rmlatinfont\textsuperscript{§~\theparCount}}}
	\pend% ending standard par
      ‚{\tiny $_{lb}$}‚

	  
	  \pstart \leavevmode% starting standard par
	स्व‚भाव‚हेताव‚न्य‚त‚र‚प्र‚{\tiny $_{६}$}‚योगादुभ‚य‚ग‚तिमुक्त्वा कार्य‚हेतावाह । \textbf{त‚थे}त्यादि ।‚{\tiny $_{lb}$}‚ य‚त्रेति स‚र्व‚नाम्ना स‚र्वोप‚संहार\textbf{स्त‚त्राग्नि}र‚व‚श्य\textbf{मित्युक्ते}ऽस्मादेवान्व‚य‚व‚च‚नात्‚{\tiny $_{lb}$}‚ \textbf{कार्य‚न्धूमो द‚ह‚न‚स्ये}त्येव‚न्निश्च‚यो भ‚व‚ति । किङ्कार‚णं[।]येन \textbf{धूमेऽव‚श्य‚म‚ग्नि}र्भ‚व‚ति ।‚{\tiny $_{lb}$}‚ \textbf{अन्य‚था} य‚दि न कार‚ण‚म‚ग्निधूर्म‚स्य त‚दा\textbf{र्थान्त‚र}स्याग्नेस्\textbf{त‚द‚नुब‚न्ध‚निय‚माभावात्} ।‚{\tiny $_{lb}$}‚ धूमे योनुब‚न्धो व्याप‚नं‚{\tiny $_{७}$}‚त‚न्निय‚म‚स्याभावात् । त‚त‚श्च \textbf{स्वात‚न्त्र्य‚म्भाव‚स्य} धूम‚स्व- \leavevmode\ledsidenote{\textenglish{32a/PSVTa}}‚{\tiny $_{lb}$}‚ भाव‚स्य \textbf{स्यात् । अत} इति स्वात‚न्त्र्यात् \textbf{त‚द‚भावेप्य}ग्न्य‚भावेपि धूम‚स्व\textbf{भाव‚स्यावैक‚{\tiny $_{lb}$}‚ल्यान्नाभावः स्या}न्न चैव‚न्त‚स्मात् कार्यो धूम एष्ट‚व्यः । य‚तः \textbf{कार्ये} तु धूमेभ्यु‚{\tiny $_{lb}$}‚प‚ग‚ते\textbf{ऽव‚श्य}न्त‚त्र \textbf{कार‚ण}म‚ग्नि\textbf{र्भ‚व‚ति । इद}मित्य‚स्यैव स‚म‚र्थ‚नं । \textbf{अर्थान्त‚र‚भा}व इति‚{\tiny $_{lb}$}‚ कार्याभिम‚त‚स्य भावे \textbf{स्व‚भावोप‚धानं} स्व‚रूप‚प्र‚त्युप‚स्थानं । \textbf{कार्य‚स्या}‚{\tiny $_{१}$}‚पि \textbf{त‚द्भाव‚{\tiny $_{lb}$}‚ एव} कार‚ण‚भाव एव \textbf{भावः} कार्य‚त्वं । \textbf{त‚च्च} कार‚ण‚भाव एव भावित्व‚म\textbf{स्ति धूमे ।‚{\tiny $_{lb}$}‚ त‚स्मात् कार्यं धूम इत्येव‚म‚न्व‚येन विदित‚त‚त्कार्य‚त्व‚स्}य पुंसो \textbf{द‚ह‚नाभावे धूमो न भ‚व}‚{\tiny $_{lb}$}‚तीत्य‚र्थाद् भ‚व‚ति ।
	{\color{gray}{\rmlatinfont\textsuperscript{§~\theparCount}}}
	\pend% ending standard par
      ‚{\tiny $_{lb}$}‚

	  
	  \pstart \leavevmode% starting standard par
	अधुना वैध‚र्म्येणान्व‚य‚ग‚तिमाह । \textbf{त‚थे}त्यादि । \textbf{अस‚त्य‚ग्नौ धूमो न भ‚व‚तीत्युक्ते}‚{\tiny $_{lb}$}‚ऽस्माद् व्य‚तिरेक‚व‚च‚नाद् विदित‚त‚त्कार्य‚त्व‚स्य \textbf{धूमेऽव‚श्य‚म‚ग्निर्भ‚व‚तीत्येव‚म‚र्था‚{\tiny $_{२}$}‚द्‚{\tiny $_{lb}$}‚ अन्व‚य‚प्र‚तिप‚त्तिर्भ‚व‚ति । अन्य‚था} य‚दि वैध‚र्म्य‚व‚च‚नेनाग्नेः कार्यं धूम इत्येत‚न्न क‚थ्य‚ते‚{\tiny $_{lb}$}‚ \textbf{त‚द‚भावे}ऽग्न्य‚भावे \textbf{किन्न भ‚वेद्} धूमो । भ‚वेदेवेति व्य‚तिरेक‚निश्च‚य एव न स्यात् ।‚{\tiny $_{lb}$}‚ त‚स्मात् स्थित‚मेत‚त् [।] स‚त्य‚र्थान्त‚र‚त्वे य‚द‚भावे य‚द‚व‚श्य‚न्न भ‚व‚ति । त‚त्त‚स्य‚{\tiny $_{lb}$}‚ कार्य‚म‚त‚श्च व्य‚तिरेक‚क‚थ‚नाद‚न्व‚यो ग‚म्य‚त इति ॥
	{\color{gray}{\rmlatinfont\textsuperscript{§~\theparCount}}}
	\pend% ending standard par
      ‚{\tiny $_{lb}$}‚‚{\tiny $_{lb}$}‚\textsuperscript{\textenglish{84/s}}

	  
	  \pstart \leavevmode% starting standard par
	अत्र प‚रो व्य‚भिचार‚माह । \textbf{न‚नु} चेत्यादि । \textbf{नित्यानित्यार्थ‚योः‚{\tiny $_{३}$}‚ का}र्य‚न्नित्या‚{\tiny $_{lb}$}‚नित्यार्थ‚कार्य‚न्त‚द्भाव‚स्त‚त्त्वं । त‚स्या\textbf{भावे}पि । त‚था हि \textbf{श्र‚व‚ण‚ज्ञानं} श‚ब्द‚स्यैव घ‚र्म‚{\tiny $_{lb}$}‚त्वाद‚साधार‚ण‚न्त‚त्र न ज्ञाय‚ते किन्नित्य‚स्य स‚तः श‚ब्द‚स्य कार्यं श्र‚व‚ण‚ज्ञान‚मुतानित्य‚{\tiny $_{lb}$}‚स्येति । त‚त‚श्च न नित्यार्थ‚कार्य‚त्वं श्र‚व‚ण‚ज्ञान‚स्य नाप्य‚नित्यार्थ‚कार्य‚त्व‚न्त‚थापि‚{\tiny $_{lb}$}‚ \textbf{श्र‚व‚ण‚ज्ञान‚न्न भ‚व‚ति} त‚द‚भावेपि नित्यानित्यार्थाभावे । त‚द्व्य‚तिरेके व्य‚तिरिच्य‚त इति‚{\tiny $_{lb}$}‚ याव‚त् । नि‚{\tiny $_{४}$}‚त्यार्थाभावे ताव‚न्न भ‚व‚त्य‚नित्येषु घ‚टादिष्व‚भावात् । अनित्याभावेपि‚{\tiny $_{lb}$}‚ न भ‚व‚ति नित्येष्वाकाशादिष्व‚भावात् । \textbf{न वै न भ‚व}तीति सि द्धा \textbf{न्त‚वादी} । न वै‚{\tiny $_{lb}$}‚ त‚च्छ्राव‚ण‚त्व‚न्नित्यानित्याभावे न भ‚व‚तीत्येवं श‚क्य‚म्विज्ञातुं । य‚दि तु त‚द‚भावेन‚{\tiny $_{lb}$}‚ भ‚व‚तीत्येव‚न्निश्च‚यः स्यात् त‚दा त‚त्कार्य‚त्व‚न्निय‚मेन स्यात् । किन्तु त‚द‚भावे न भ‚व‚तीत्येव‚{\tiny $_{lb}$}‚ नास्ति । किङ्कार‚ण‚{\tiny $_{५}$}‚\textbf{न्त‚योरेव} नित्यानित्य‚त्व‚यो\textbf{स्त‚तः} श्राव‚ण‚त्वात् \textbf{संश‚यात् । अन्य‚था}‚{\tiny $_{lb}$}‚ नित्यानित्ये व‚स्तु\textbf{न्य‚भावेन निश्चिता}च्छ्राव‚ण‚त्वात्क\textbf{थ‚न्त‚द्भाव‚प‚राम‚र्शेन श‚ब्दे संश‚यः‚{\tiny $_{lb}$}‚ स्यात्} । त‚स्मान्नित्यानित्याभ्यां व्यावृत्तिरेव नास्ति श्राव‚ण‚त्व‚स्य ।
	{\color{gray}{\rmlatinfont\textsuperscript{§~\theparCount}}}
	\pend% ending standard par
      ‚{\tiny $_{lb}$}‚

	  
	  \pstart \leavevmode% starting standard par
	क‚थ‚न्त‚र्ह्य‚साधार‚ण‚त्वाच्छ्राव‚ण‚त्वं नित्यानित्य‚योर्नास्तीत्युच्य‚त इत्याह । \textbf{केव‚ल}‚{\tiny $_{lb}$}‚न्त्वित्यादि ।‚{\tiny $_{६}$}‚ नित्यानित्येषु श्राव‚ण‚त्व‚स्य \textbf{भाव‚निश्च‚याभावात्} । श्राव‚ण‚त्वं नित्या‚{\tiny $_{lb}$}‚न‚त्य‚यो\textbf{र्नास्तीत्युच्य‚ते} ।
	{\color{gray}{\rmlatinfont\textsuperscript{§~\theparCount}}}
	\pend% ending standard par
      ‚{\tiny $_{lb}$}‚

	  
	  \pstart \leavevmode% starting standard par
	न‚न्व‚नित्यादिके साध्ये य‚दि श्राव‚ण‚त्वं स‚प‚क्ष‚विप‚क्ष‚योर्दृष्टं स्यात् । स्या‚{\tiny $_{lb}$}‚च्छ‚ब्दे श्राव‚ण‚त्वात् स‚न्देहः । प्र‚मेय‚त्वादिव । न चैत‚त् स‚प‚क्षाविप‚क्ष‚योर्दृष्ट‚म‚तो‚{\tiny $_{lb}$}‚ \leavevmode\ledsidenote{\textenglish{32b/PSVTa}} ऽप्र‚तिप‚त्तिरिति भ ट्टो द्यो त क रौ । अथ श‚ब्द‚व‚स्तु क‚दाचिन्नित्य‚म‚नित्य‚म्वा व‚{\tiny $_{७}$}‚स्तु‚{\tiny $_{lb}$}‚ ध‚र्म‚श्च श्राव‚ण‚त्व‚न्तेनातः स‚न्देह उच्य‚ते ।
	{\color{gray}{\rmlatinfont\textsuperscript{§~\theparCount}}}
	\pend% ending standard par
      ‚{\tiny $_{lb}$}‚

	  
	  \pstart \leavevmode% starting standard par
	त‚द‚युक्तं । एवं हि व‚स्तुध‚र्म‚त्व‚स्यैव स‚न्देह‚हेतुत्वं स्यान्न श्राव‚ण‚त्व‚स्येति ।‚{\tiny $_{lb}$}‚ अत्रोच्य‚ते । य‚दि हि य‚त्र य‚त्र श्राव‚ण‚त्व‚न्त‚त्र त‚त्र नित्यानित्य‚योर‚भाव इति प्र‚तिप‚न्नं‚{\tiny $_{lb}$}‚ स्यात् त‚तो नित्यानित्य‚योर‚प्र‚तिप‚त्तिः स्यात् । न चानित्यादियुक्ते घ‚टादौ श्राव‚ण‚{\tiny $_{lb}$}‚त्व‚स्याभाव इति श‚ब्देप्य‚भाव‚स्तेन श्राव‚ण‚त्वात् त‚त्र स‚न्देह एव ।
	{\color{gray}{\rmlatinfont\textsuperscript{§~\theparCount}}}
	\pend% ending standard par
      ‚{\tiny $_{lb}$}‚

	  
	  \pstart \leavevmode% starting standard par
	न‚नु श्राव‚ण‚त्वं श्र‚व‚ण‚ज्ञानंप्र‚{\tiny $_{१}$}‚ति साम‚र्थ्य‚न्न च नित्य‚स्य साम‚र्थ्य‚म‚स्त्य‚र्थ‚क्रिया‚{\tiny $_{lb}$}‚विरोधात् [।] त‚त्क‚थ‚म‚तः स‚न्देहः [।] अनित्य‚त्व‚स्यैव निश्च‚यादिति ।
	{\color{gray}{\rmlatinfont\textsuperscript{§~\theparCount}}}
	\pend% ending standard par
      ‚{\tiny $_{lb}$}‚

	  
	  \pstart \leavevmode% starting standard par
	एव‚म्म‚न्य‚ते । य‚दि साम‚र्थ्य‚मात्रं हेतुस्त‚दा स‚त्त्व‚मेव त‚दिति न काचित् क्ष‚तिः [।]‚{\tiny $_{lb}$}‚ [।] त‚स्य स‚प‚क्ष‚साधार‚ण‚त्वादेवं प्र‚मेय‚त्वादिष्व‚पि द्र‚ष्ट‚व्यं । अथ श्र‚व‚ण‚ज्ञानंप्र‚ति‚{\tiny $_{lb}$}‚ य‚त्साम‚र्थ्य‚न्त‚द्धेतुस्त‚च्च न क्व‚चिद‚नित्य‚त्व‚व्याप्तं सिद्ध‚मिति क‚थ‚म‚तोऽनित्य‚त्व‚{\tiny $_{lb}$}‚सिद्धिर‚साधार‚ण‚त्वाद‚यंश‚ब्द एव‚{\tiny $_{२}$}‚ त‚द‚नित्य‚त्व‚व्याप्तं सिद्धं । तेनैव हि बाध‚केन‚{\tiny $_{lb}$}‚ ‚{\tiny $_{lb}$}‚ \leavevmode\ledsidenote{\textenglish{85/s}}प्र‚माणेनानित्य‚त्व‚स्य त‚त्र सिद्ध‚त्वाच्छ्राव‚ण‚त्व‚स्य वैय‚र्थ्यं स्यात् । एवं स चासाधार‚{\tiny $_{lb}$}‚ण‚हेतूनाम‚ग‚म‚क‚त्वं बोद्ध‚व्यं ।
	{\color{gray}{\rmlatinfont\textsuperscript{§~\theparCount}}}
	\pend% ending standard par
      ‚{\tiny $_{lb}$}‚

	  
	  \pstart \leavevmode% starting standard par
	त‚स्मात् स्थित‚मेत‚त् [।] कार्य‚हेतौ दृष्टान्ताभ्यां साध्य‚साध‚न‚यो\textbf{र्हेतुफ‚ल‚भावः}‚{\tiny $_{lb}$}‚ क‚थ‚नीयो न तु द‚र्श‚नाद‚र्श‚न‚मात्रं । एवं ह्य‚र्थाप‚त्त्याऽन्य‚त‚रेण द्वितीय‚प्र‚तीतिर्भ‚वेत् ।‚{\tiny $_{lb}$}‚ अन्य‚था न स्यादित्याह । \textbf{य‚दा पुन}रित्या‚{\tiny $_{३}$}‚दि । \textbf{त‚दा य‚त्र धूम‚स्त‚त्राग्निरित्येव‚{\tiny $_{lb}$}‚ न स्या}दित्य‚न्व‚य एव न स्यात् \textbf{प्र‚तिब‚न्धाभावात्} । य‚दा चान्व‚य एव न सिद्ध‚स्त‚दा‚{\tiny $_{lb}$}‚ कुतो\textbf{ग्न्य‚भावे धूमो नास्ती}त्य‚र्थाद् व्\textbf{य‚तिरेक‚सिद्धिः । त‚था वैध‚र्म्ये}णाद‚र्श‚न‚मात्र‚स्य‚{\tiny $_{lb}$}‚ ख्याप‚नात् साध्याभावे हेत्\textbf{व‚भावासिद्धेः} कुत‚स्त‚द्वारेणा\textbf{न्व‚य‚स्मृ}तिः ॥ \textbf{य‚थोक्त} इति‚{\tiny $_{lb}$}‚ तादात्म्य‚त‚दुत्प‚त्तिल‚क्ष‚णः । \textbf{एक‚स‚द्भावे} कार्य‚स्व‚भाव‚लिङ्ग‚स्य स‚द्भावे\textbf{ऽन्य‚प्र‚{\tiny $_{४}$}‚सि‚{\tiny $_{lb}$}‚द्ध्य‚र्थं} कार‚ण‚स्य स्व‚भाव‚स्य च लिङ्गिनः प्र‚सिद्ध्य‚र्थं । \textbf{त‚द‚भावे य‚थोक्त‚प्र‚तिब‚न्धा‚{\tiny $_{lb}$}‚भावे} स‚त्येक‚स‚द्भावेऽन्य‚प्र‚सिद्धे\textbf{र‚स‚म्भ‚वात्} ॥
	{\color{gray}{\rmlatinfont\textsuperscript{§~\theparCount}}}
	\pend% ending standard par
      ‚{\tiny $_{lb}$}‚

	  
	  \pstart \leavevmode% starting standard par
	\textbf{हेतुस्व‚भावा}भाव इति हेत्व‚भावो व्याप‚क‚स्व‚भावाभाव‚श्च । अत इत्य‚न‚न्त‚{\tiny $_{lb}$}‚रोक्तात् कार‚णात् \textbf{क‚स्य‚चि}त् कार्य‚स्य व्याप्य‚स्य \textbf{च प्र‚तिषे}धे । च‚कारात् प्र‚तिषे‚{\tiny $_{lb}$}‚ध‚व्य‚व‚हारे च साध्ये । \textbf{हेतु}र्लिङ्गं । किङ्कार‚णं [।] य‚स्मात् \textbf{तावेव हि} कार‚ण‚व्या‚{\tiny $_{lb}$}‚प‚कौ‚{\tiny $_{५}$}‚\textbf{नि व‚र्त्त‚मानौ स्व‚प्र‚तिब}द्धं कार्यं व्याप्यं च स्व‚भावं \textbf{निव‚र्त्त‚य‚त इति क‚स्य‚चि‚{\tiny $_{lb}$}‚द‚र्थ‚स्य} कार्य‚स्य व्याप्य‚स्य वा \textbf{प्र‚तिषेध‚म‚पि साध‚यितुकामे}न । \textbf{अपि}श‚ब्दाद्‚{\tiny $_{lb}$}‚ व्य‚व‚हार‚म‚पि [।] \textbf{हेतोः} कार‚ण‚स्य व्\textbf{याप‚क‚स्}य च \textbf{स्व‚भाव‚स्य निवृत्तिर्हेतुत्वे‚{\tiny $_{lb}$}‚नाख्ये}या । किङ्कार‚णं । \textbf{अप्र‚तिब‚न्धे हीत्}यादि । न च ताभ्याम‚न्यः प्र‚तिब‚न्धो‚{\tiny $_{lb}$}‚स्तीति भावः ॥ \textbf{युक्तो} न्याय्य \textbf{उप‚ल‚म्भो} य‚स्य स त‚था‚{\tiny $_{६}$}‚दृश्य‚स्येत्य‚र्थः । \textbf{त‚स्य‚{\tiny $_{lb}$}‚ ‚{\tiny $_{lb}$}‚ \leavevmode\ledsidenote{\textenglish{86/s}}चे}ति स्व‚भाव‚स्य्\textbf{आनुप‚ल‚म्भ}नं \textbf{प्र‚तिषेध‚हेतुः} । न चायं प्र‚तिषेध‚स्यैव हेतुः किन्तु‚{\tiny $_{lb}$}‚ \textbf{प्र‚तिषेध‚विष‚यो व्य‚व‚हार‚स्त}स्य \textbf{हेतुरिति} कृत्वा त‚द्धेतुः प्र‚तिषेध‚हेतु\textbf{रित्युक्तः} । किं‚{\tiny $_{lb}$}‚ कार‚णं [।] न प्र‚तिषेध‚हेतुस्\textbf{त‚थाभूतानुप‚ल‚म्भ‚स्}य दृश्यानुप‚ल‚म्भ‚स्य \textbf{स्व‚यं प्र‚तिषेध‚{\tiny $_{lb}$}‚\leavevmode\ledsidenote{\textenglish{33a/PSVTa}} रूप‚त्वात्} । हेतुर्व्याप‚कानुप‚ल‚ब्धिरिति \textbf{कार‚णा}नुप‚ल‚ब्धि\textbf{र्व्याप‚का‚{\tiny $_{७}$}‚नुप‚ल‚ब्धि}श्च । \textbf{उभ‚{\tiny $_{lb}$}‚य‚स्यापी}ति प्र‚तिषेध‚स्य प्र‚तिषेध‚व्य‚व‚हार‚स्य च ॥
	{\color{gray}{\rmlatinfont\textsuperscript{§~\theparCount}}}
	\pend% ending standard par
      ‚{\tiny $_{lb}$}‚

	  
	  \pstart \leavevmode% starting standard par
	\textbf{इति} एव\textbf{मिय‚म}नुप‚ल‚ब्धिः । संक्षिप्य \textbf{त्रिधाप्युक्ता} स‚ती पुन\textbf{र‚नेक‚धो}क्ता । य‚था‚{\tiny $_{lb}$}‚त्रैवाऽष्ट‚धा प्राग् विभ‚क्ता । केन प्र‚कारेणेत्याह । \textbf{त‚त्त‚द्विरुद्धे}त्यादि । त‚च्छ‚{\tiny $_{lb}$}‚ब्देन प्र‚क्रान्तं स्व‚भाव‚कार‚ण‚व्याप‚क‚त्र‚यं गृह्य‚ते । तेन स्व‚भाव‚दित्र‚येण विरुद्ध‚न्त‚{\tiny $_{lb}$}‚ द्विरुद्ध‚न्त्रि‚{\tiny $_{१}$}‚विध‚मेव भ‚व‚ति । स्व‚भाव‚विरुद्ध‚कार‚ण‚विरुद्ध‚व्याप‚क‚विरुद्ध‚भेदाः ।‚{\tiny $_{lb}$}‚ त‚द्विरुद्ध‚मादिर्य‚स्य \textbf{त‚त्त‚द्विरुद्धादि} । आदिश‚ब्देन विरुद्ध‚कार्य‚स्य कार‚ण‚विरुद्ध‚{\tiny $_{lb}$}‚कार्य‚स्य च प‚रिग्र‚हः । त‚च्च त‚द्विरुद्धादी । त‚यो\textbf{र‚ग‚तिग‚ती} । त‚द‚ग‚तिस्त‚द्विरुद्धादि‚{\tiny $_{lb}$}‚ग‚तिश्चेत्य‚र्थः । त‚यो\textbf{र्भे}द‚स्तेन \textbf{प्र‚योग‚स्त‚स्मा}त् प्र‚योग‚भेद‚तोनेक‚धोक्ता । त‚त्र‚{\tiny $_{lb}$}‚ त‚द‚ग‚त्या तिस्रोनुप‚{\tiny $_{२}$}‚ल‚ब्ध‚यः संगृहीताः स्व‚भावानुप‚ल‚ब्धिः कार‚णानुप‚ल‚ब्धिः व्या‚{\tiny $_{lb}$}‚प‚कानुप‚ल‚ब्धिश्च । त‚द्विरुद्ध‚ग‚त्या तिस्र एव । स्व‚भाव‚विरुद्धोप‚ल‚ब्धिः कार‚ण‚{\tiny $_{lb}$}‚विरुद्धोप‚ल‚ब्धिः व्याप‚क‚विरुद्धोप‚ल‚ब्धिश्च । आदिश‚ब्दात् विरुद्ध‚कार्योप‚ल‚ब्धिः‚{\tiny $_{lb}$}‚ कार‚ण‚विरुद्ध‚कार्योप‚ल‚ब्धिश्चेति । एव‚म‚ष्ट‚विध‚स्य प्रागुक्त‚स्यानुप‚ल‚म्भ‚स्य संग्र‚हो‚{\tiny $_{lb}$}‚ भ‚व‚ति ।
	{\color{gray}{\rmlatinfont\textsuperscript{§~\theparCount}}}
	\pend% ending standard par
      ‚{\tiny $_{lb}$}‚

	  
	  \pstart \leavevmode% starting standard par
	\textbf{त्रिविध एव} हीत्यादिना कारि‚{\tiny $_{३}$}‚कार्थ‚माह । \textbf{उप‚ल‚भ्य‚स‚त्त्}व‚स्येत्युप‚ल‚म्भ‚योग्य‚स्य‚{\tiny $_{lb}$}‚ हेतो\textbf{र‚नुप‚ल‚ब्धिरि}ति स‚म्ब‚न्धः । \textbf{व्याप‚क‚स्य स्वात्म‚न‚श्चोप‚ल‚भ्य‚स‚त्त्व‚स्}येति व‚र्त्त‚ते ।‚{\tiny $_{lb}$}‚ \textbf{सोय}न्त्रिविध‚प्र‚तिषेध‚हेतुः \textbf{प्र‚योग‚व‚शेना}नेक‚प्र‚कार उक्त इति स‚म्ब‚न्धः । क‚थं \textbf{प्र‚योग}‚{\tiny $_{lb}$}‚ व‚शेनेत्याह । \textbf{त‚त्त‚द्विरुद्धाद्य‚ग‚तिग‚तिभेद‚प्र‚योग‚त} इति । \textbf{एत‚देव त‚स्याग‚त्येत्या}दिना‚{\tiny $_{lb}$}‚ विभ‚ज्य‚ते ।‚{\tiny $_{४}$}‚ \textbf{त‚स्याग‚त्}येति स्व‚भाव‚कार‚ण‚व्याप‚कानुप‚ल‚ब्ध्या । \textbf{त‚द्विरुद्ध‚ग‚त्ये}ति‚{\tiny $_{lb}$}‚ स्व‚भाव‚कार‚ण‚व्याप‚क‚विरुद्धोप‚ल‚ब्ध्या । \textbf{विरुद्ध‚कार्य‚ग‚त्ये}ति विरुद्ध‚कार्योप‚ल‚ब्ध्या ।‚{\tiny $_{lb}$}‚ ‚{\tiny $_{lb}$}‚ \leavevmode\ledsidenote{\textenglish{87/s}}\textbf{इत्यादिभेद‚प्र‚योगै}रिति कार‚ण‚विरुद्ध‚कार्योप‚ल‚ब्ध्यादिभेद‚प्र‚योगैः । \textbf{य‚थोक्तं प्राग}नु‚{\tiny $_{lb}$}‚ प‚ल ब्धि प्र भे द चि न्ता यां \href{http://sarit.indology.info/?cref=}{१ । ६} ॥
	{\color{gray}{\rmlatinfont\textsuperscript{§~\theparCount}}}
	\pend% ending standard par
      ‚{\tiny $_{lb}$}‚

	  
	  \pstart \leavevmode% starting standard par
	य‚त एवं प्र‚तिब‚न्ध‚व‚शाद् ग‚म‚क‚त्वात्त‚स्मात् । कार्य\textbf{कार‚ण‚भावाद्वा‚{\tiny $_{५}$}‚ नियाम‚कात्}‚{\tiny $_{lb}$}‚ साध्य‚साध‚न‚योर‚व्य‚भिचार‚साध‚कात् \textbf{स्व‚भावा}द्वा तादात्म्य‚ल‚क्ष‚णा\textbf{न्नियाम‚का}त् ।‚{\tiny $_{lb}$}‚ कार्य‚स्य स्व‚भाव‚स्य च लिङ्ग‚स्या\textbf{विनाभावः} साध्य‚ध‚र्मं विना न भाव इत्य‚र्थः ॥ न‚{\tiny $_{lb}$}‚ चासाधार‚ण‚स्य साध्याविनाभावोस्ति स‚न्देह‚हेतुत्वात् । अविनाभावे तु त‚न्निश्चाय‚के‚{\tiny $_{lb}$}‚नैव प्र‚माणेन त‚त्र ध‚र्मिणि साध्य‚स्य सिद्ध‚त्वात् क‚थ‚म‚स्य ग‚म‚क‚त्वं ।
	{\color{gray}{\rmlatinfont\textsuperscript{§~\theparCount}}}
	\pend% ending standard par
      ‚{\tiny $_{lb}$}‚

	  
	  \pstart \leavevmode% starting standard par
	तेन भ ट्टे न य‚{\tiny $_{६}$}‚दुच्य‚ते ॥
	{\color{gray}{\rmlatinfont\textsuperscript{§~\theparCount}}}
	\pend% ending standard par
      ‚{\tiny $_{lb}$}‚
	  \bigskip
	  \begingroup
	
	    
	    \stanza[\smallbreak]
	  {\normalfontlatin\large ``\qquad}अविनाभाव‚श‚ब्दोप्य{...}स‚क‚लार्थ‚भाक् ।&‚{\tiny $_{lb}$}‚नानुमा योग्य‚स‚म्ब‚न्ध‚प्र‚तिप‚त्ति क‚रोति नः ॥&‚{\tiny $_{lb}$}‚य‚दि ताव‚द् विनाभावो न स प‚श्चाद् विशिष्य‚ते ।&‚{\tiny $_{lb}$}‚त‚तोऽसाधार‚णेप्य‚स्ति स इति स्याद‚कार‚णं ॥&‚{\tiny $_{lb}$}‚यो ह्य‚साधार‚णो ध‚र्मः स तेनैवात्म‚सात्कृतः ।&‚{\tiny $_{lb}$}‚विना न भ‚व‚तीत्येव ज्ञातो हेतुः प्र‚स‚ज्य‚त इति [।]{\normalfontlatin\large\qquad{}"}\&[\smallbreak]
	  
	  
	  
	  \endgroup
	‚{\tiny $_{lb}$}‚

	  
	  \pstart \leavevmode% starting standard par
	त‚द‚पास्तं ।‚{\tiny $_{७}$}‚ अविनाभाव एव हि निय‚मः । साध्यं विना न भ‚व‚तीति कृत्वा । \leavevmode\ledsidenote{\textenglish{33b/PSVTa}}
	{\color{gray}{\rmlatinfont\textsuperscript{§~\theparCount}}}
	\pend% ending standard par
      ‚{\tiny $_{lb}$}‚

	  
	  \pstart \leavevmode% starting standard par
	य‚द्येवं किम‚र्थं पुन‚र्निय‚म‚ग्र‚ह‚णं[।]स‚त्त्यं[।]प‚र‚म‚त‚निरासार्थं । स ह्य‚विना‚{\tiny $_{lb}$}‚भाव‚व्य‚तिरेकेणान्यं निय‚म‚मिच्छ‚ति । य‚दाह भ ट्टः ॥‚{\tiny $_{lb}$}‚ 
	    \pend% close preceding par
	  
	    
	    \stanza[\smallbreak]
	  {\normalfontlatin\large ``\qquad}एव‚म‚न्योक्त‚स‚म्ब‚न्ध‚प्र‚त्याख्याने कृते स‚ति ।&‚{\tiny $_{lb}$}‚निय‚मो नाम स‚म्ब‚न्धः स्व‚म‚तेनोच्य‚तेऽधुना ॥&‚{\tiny $_{lb}$}‚कार्य‚कार‚ण‚भावादिस‚म्ब‚न्धानां द्व‚यी ग‚तिः ।&‚{\tiny $_{lb}$}‚निय‚मानिय‚माभ्यां । स्यान्निय‚म‚स्यानुमाङ्ग‚ता ॥&‚{\tiny $_{lb}$}‚स‚र्वेप्य‚निय‚मा ह्येते नानुमोत्प‚त्तिकार‚णं ॥&‚{\tiny $_{lb}$}‚निय‚मात् केव‚लादेव‚न्न किञ्चिन्नानुमीय‚ते ॥&‚{\tiny $_{lb}$}‚त‚स्मान्निय‚म एवैकः स‚म्ब‚न्धोऽत्राव‚धार्य‚ते ।&‚{\tiny $_{lb}$}‚ग‚म‚क‚स्यैव ग‚म्येन स चेष्टः प्राङ् निरूपितः ॥&‚{\tiny $_{lb}$}‚निय‚म‚स्म‚र‚तः स‚म्य‚ग् निय‚म्यैकाङ्ग‚द‚र्श‚नात् ।&‚{\tiny $_{lb}$}‚नियाम‚काङ्ग‚विज्ञान‚म‚नुमान‚न्त‚द‚ङ्गिष्विति [।]{\normalfontlatin\large\qquad{}"}\&[\smallbreak]
	  
	  
	  
	    \pstart  \leavevmode% new par for following
	    \hphantom{.}
	  ‚{\tiny $_{१}$}‚ त‚द‚पा‚{\tiny $_{२}$}‚स्तं ॥
	{\color{gray}{\rmlatinfont\textsuperscript{§~\theparCount}}}
	\pend% ending standard par
      ‚{\tiny $_{lb}$}‚\textsuperscript{\textenglish{88/s}}

	  
	  \pstart \leavevmode% starting standard par
	काय‚कार‚ण‚भावादिस‚म्ब‚न्धाभावे निय‚म एव न स्यात् । अप्र‚तिब‚द्धानां‚{\tiny $_{lb}$}‚ स‚ह‚भाव‚निय‚माभावात् । पाण्डुत्वादिमात्र‚स्य चाग्न्य‚कार्य‚त्वात् । देश‚कालाद्य‚{\tiny $_{lb}$}‚पेक्ष‚या च लिङ्ग‚स्य ग‚म‚क‚त्वान्निय‚म एवेति क‚थं कार्य‚कार‚ण‚भावादिस‚म्ब‚न्धेपि‚{\tiny $_{lb}$}‚ द्व‚यी ग‚तिरुच्य‚ते । अथ स्यात् [।] कार्य‚कार‚ण‚भावादिस‚म्ब‚न्ध‚म‚भ्युप‚ग‚च्छ‚तापि याव‚त्‚{\tiny $_{lb}$}‚ साध‚न‚स्य साध्ये‚{\tiny $_{३}$}‚निय‚मो न निश्चित‚स्ताव‚न्न साध्य‚प्र‚तिप‚त्त्य‚ङ्ग‚त्व‚न्तेन कार्य‚कार‚ण‚{\tiny $_{lb}$}‚भावे स‚त्य‚पि निय‚म एव स‚म्ब‚न्धोभ्युप‚ग‚म्य‚ते । त‚स्यैव प्र‚तिप‚त्त्य‚ङ्ग‚त्वात्त‚द्व‚क्तुं स‚{\tiny $_{lb}$}‚ एवैकोनुमानाङ्गं शेषास्त‚द्व्य‚क्तिहेत‚व इति । त‚द‚युक्तं [।] निय‚मो हि त‚दा‚{\tiny $_{lb}$}‚य‚त्त‚तैव [।] सा च तादात्म्य‚त‚दुत्प‚त्तिस्व‚भावैव [।] तेन तादात्म्य‚त‚दुत्प‚त्तिनिश्च‚य‚{\tiny $_{lb}$}‚ एव निय‚म‚निश्च‚यो न पुन‚र्द‚र्श‚नाद‚{\tiny $_{४}$}‚र्श‚नाभ्यान्निय‚म‚निश्च‚यो व्य‚भिचारात् ।
	{\color{gray}{\rmlatinfont\textsuperscript{§~\theparCount}}}
	\pend% ending standard par
      ‚{\tiny $_{lb}$}‚

	  
	  \pstart \leavevmode% starting standard par
	न‚नु य‚था द‚र्श‚नाद‚र्श‚न‚योर्निय‚म‚निश्च‚यंप्र‚ति व्य‚भिचार‚स्त‚था कार्य‚कार‚ण‚{\tiny $_{lb}$}‚भाव‚निश्च‚येपि स्यादिति ।
	{\color{gray}{\rmlatinfont\textsuperscript{§~\theparCount}}}
	\pend% ending standard par
      ‚{\tiny $_{lb}$}‚

	  
	  \pstart \leavevmode% starting standard par
	त‚द‚युक्तं । विशिष्टाभ्यामेव द‚र्श‚नाद‚र्श‚नाभ्यां कार्य‚कार‚ण‚भाव‚निश्च‚याभ्यु‚{\tiny $_{lb}$}‚प‚ग‚मात् । एत‚च्चात्रैव व‚क्ष्य‚ति ।
	{\color{gray}{\rmlatinfont\textsuperscript{§~\theparCount}}}
	\pend% ending standard par
      ‚{\tiny $_{lb}$}‚

	  
	  \pstart \leavevmode% starting standard par
	\hphantom{.}य‚द‚प्यु म्वे के नोच्य‚ते । श‚त‚शो य‚{\tiny $_{५}$}‚द‚ग्नौ धूम‚द‚र्श‚न‚न्त‚द‚न्य‚थानुप‚प‚त्त्या‚{\tiny $_{lb}$}‚ निय‚तोयं धूमोग्नाविति य‚न्निय‚म‚ज्ञान‚मुत्प‚द्य‚ते । त‚स्यान‚ग्नौ धूम‚द‚र्श‚न‚म्बाध‚कं [।]‚{\tiny $_{lb}$}‚ न च त‚द‚स्तीति धूम‚स्याग्नौ निय‚म इति ।
	{\color{gray}{\rmlatinfont\textsuperscript{§~\theparCount}}}
	\pend% ending standard par
      ‚{\tiny $_{lb}$}‚

	  
	  \pstart \leavevmode% starting standard par
	त‚द‚युक्तं । अग्निकार्य‚त्वाभावे ह्य‚न‚ग्नौ धूमाद‚र्श‚न‚स्यानुप‚ल‚ब्धिमात्र‚त्वेना‚{\tiny $_{lb}$}‚प्र‚माण‚स्याबाध‚क‚त्वाद‚न‚ग्नौ धूम‚स्य शंक्य‚मान‚त्वेन क‚थ‚म‚ग्नौ निय‚मः [।] स्त‚स्मात्‚{\tiny $_{lb}$}‚ स्थित‚मेत‚त् [।]
	{\color{gray}{\rmlatinfont\textsuperscript{§~\theparCount}}}
	\pend% ending standard par
      ‚{\tiny $_{lb}$}‚

	  
	  \pstart \leavevmode% starting standard par
	\hphantom{.}
	    \pend% close preceding par
	  
	    
	    \stanza[\smallbreak]
	  \flagstanza{\tiny\textenglish{...p88a}}{\normalfontlatin\large ``\qquad}कार्य‚का‚{\tiny $_{६}$}‚र‚ण‚भावाद्वा स्व‚भावाद्वा नियाम‚काद् [।]&‚{\tiny $_{lb}$}‚अविनाभाव‚निय‚मः{\normalfontlatin\large\qquad{}"}\&[\smallbreak]
	  
	  
	  
	    \pstart  \leavevmode% new par for following
	    \hphantom{.}
	   [।] कार्य‚कार‚ण‚भावादिनिश्च‚याच्चाविनाभाव‚निय‚म‚नि‚{\tiny $_{lb}$}‚श्च‚यो 
	    \pend% close preceding par
	  
	    
	    \stanza[\smallbreak]
	  \flagstanza{\tiny\textenglish{...p88b}}{\normalfontlatin\large ``\qquad}ऽद‚र्श‚नान्न न द‚र्श‚नात्{\normalfontlatin\large\qquad{}"}\&[\smallbreak]
	  
	  
	  
	    \pstart  \leavevmode% new par for following
	    \hphantom{.}
	   [।] साध्याभावे हेतोर‚द‚र्श‚न‚मात्रान्नाविनाभाव‚निय‚{\tiny $_{lb}$}‚म‚निश्च‚यः । न द‚र्श‚नात् । नापि साध्य‚साध‚न‚योः स‚ह‚भाव‚द‚र्श‚नात् ।
	{\color{gray}{\rmlatinfont\textsuperscript{§~\theparCount}}}
	\pend% ending standard par
      ‚{\tiny $_{lb}$}‚

	  
	  \pstart \leavevmode% starting standard par
	\leavevmode\ledsidenote{\textenglish{34a/PSVTa}} त‚स्मात्त‚दुत्प‚त्त्यैवार्थान्त‚र‚स्यार्था‚{\tiny $_{७}$}‚न्त‚रेणाविनाभावः ।
	{\color{gray}{\rmlatinfont\textsuperscript{§~\theparCount}}}
	\pend% ending standard par
      ‚{\tiny $_{lb}$}‚

	  
	  \pstart \leavevmode% starting standard par
	\textbf{अन्य‚थे}त्यास‚त्यान्त‚दुत्प‚त्तौ \textbf{प‚रैः} साध्याभिम‚तैः प‚र‚स्यानात्माभूत‚स्य लिंग‚स्य‚{\tiny $_{lb}$}‚ \textbf{कोऽव‚श्य‚म्भाव‚निय}मः । \textbf{अन‚र्थान्त‚रे} तु लिङ्गे त‚न्मात्रानुब‚न्धित्वं साध्य‚ध‚र्म‚स्ये‚{\tiny $_{lb}$}‚ष्ट‚व्य‚म‚न्य‚था कृत‚क‚त्व‚स्य य\textbf{न्निमित्त}न्त‚स्मा\textbf{द‚र्थान्त}र‚मुद्ग‚रादि\textbf{निमित्तं} य‚स्या नित्य‚{\tiny $_{lb}$}‚‚{\tiny $_{lb}$}‚ \leavevmode\ledsidenote{\textenglish{89/s}}त्व‚स्येष्य‚ते । त‚स्मिन् वा \textbf{ध‚र्मे}ऽव‚श्य‚म्भाव‚निय‚मः कः । किमिव \textbf{वास‚सि राग‚व‚त्} ।‚{\tiny $_{lb}$}‚ निष्प‚न्ने वास‚सि कुसुम्भादि‚{\tiny $_{१}$}‚निमित्तो यो रागः प‚श्चाद्भावी ॥
	{\color{gray}{\rmlatinfont\textsuperscript{§~\theparCount}}}
	\pend% ending standard par
      ‚{\tiny $_{lb}$}‚

	  
	  \pstart \leavevmode% starting standard par
	त‚त्र य‚था नाव‚श्य‚म्भाव‚निय‚म‚स्त‚द्व‚द‚नित्य‚त्व‚स्यार्थान्त‚र‚हेतुत्व इष्य‚माणे न‚{\tiny $_{lb}$}‚ केव‚ल‚म‚य‚न्दोषोऽय‚म‚प‚रो दोष इत्याह । \textbf{अपि} चेत्यादि । \textbf{अर्थान्त‚र‚निमित्तो} नित्य‚{\tiny $_{lb}$}‚त्वाख्यो ध\textbf{र्मः स्याद‚न्य एव} [।] त‚स्मात् स्व‚भाव‚भूतात् कृत‚कादेस्त‚थाहि साध्य‚{\tiny $_{lb}$}‚ध‚र्म‚स्यार्थान्त‚र‚निमित्त‚त्वाभ्युप‚ग‚मे द्व‚य‚मिष्टं साध‚न‚निष्प‚त्ताव‚निष्प‚त्तिर्भि‚{\tiny $_{२}$}‚न्न‚हेतु‚{\tiny $_{lb}$}‚क‚त्वं च । एत‚च्च नान्त‚रेण स्व‚भाव‚भेदं घ‚ट‚ते । य‚स्मा\textbf{न्न हि त‚स्मि}न् साध‚न‚स्व‚भावे‚{\tiny $_{lb}$}‚ \textbf{निष्प‚न्नेप्य‚निष्प‚न्नो भिन्न‚हेतुको वा} साध्य‚ध‚र्म‚स्\textbf{त‚त्स्व‚भावो युक्}तः । पूर्व‚निष्प‚न्न‚स्य‚{\tiny $_{lb}$}‚ भिन्न‚हेतुक‚स्य च लिङ्ग‚स्य स्व‚भावो युक्तो य‚स्माद\textbf{य‚मेव} ख‚लु लोक‚प्र‚तीतो \textbf{भेदो‚{\tiny $_{lb}$}‚ भावानां} यो विरुद्ध‚ध‚र्माध्यासो विरुद्ध‚ध‚र्म‚योगः । निष्प‚त्त्य‚निष्प‚त्ती चात्र विरुद्धौ‚{\tiny $_{lb}$}‚ ध‚र्मौ । त‚थाय‚मेव‚{\tiny $_{३}$}‚ भेद‚हेतुर्भेद‚स्य ज‚न‚को यः कार‚ण‚भेदः । साम‚ग्रीभेद‚श्चात्र \textbf{कार‚ण‚{\tiny $_{lb}$}‚भेदो} द्र‚ष्ट‚व्यः । एतेन भेद‚स्व‚रूप‚भेद‚कार‚ण‚ञ्चोक्तं ॥
	{\color{gray}{\rmlatinfont\textsuperscript{§~\theparCount}}}
	\pend% ending standard par
      ‚{\tiny $_{lb}$}‚

	  
	  \pstart \leavevmode% starting standard par
	भेद‚प्र‚तिभास‚स्तु भेद‚ग्राह‚कः ॥ \textbf{तौ चेद् बिरुद्ध‚ध‚र्माध्यास‚कार‚ण‚भे}दौ न भेद‚का‚{\tiny $_{lb}$}‚वि\textbf{श्ये}\edtext{}{\lemma{वि}\Bfootnote{? ष्ये}}ते । \textbf{त‚दा न क‚स्य}चिद्व‚स्तुनः कु\textbf{त‚श्चिद‚र्था}द् \textbf{भेद इत्येक‚न्द्र‚व्य‚म्विश्वं}‚{\tiny $_{lb}$}‚ स‚म‚स्त‚ञ्ज‚ग‚त् \textbf{स्यात्} ॥ त्रैगुण्य‚स्याविशेषादैक्यं स‚र्व‚स्येष्ट‚मेवेति चेदाह । \textbf{त‚त‚श्चे}‚{\tiny $_{४}$}‚त्ये‚{\tiny $_{lb}$}‚क‚त्वात् \textbf{स‚होत्प‚त्तिविनाशौ} । एक‚स्योत्पादे स‚र्व‚स्योत्पादो विनाशे च विनाशः‚{\tiny $_{lb}$}‚ स्यादित्य‚र्थः । \textbf{स‚र्व‚स्य च स‚र्व}त्र कार्य \textbf{उप‚योगः} कार‚ण‚त्वं \textbf{स्याद्} । स‚होत्प‚त्याद्य‚न‚{\tiny $_{lb}$}‚भ्युप‚ग‚मे । स‚र्व‚म्व‚स्त्\textbf{वेक‚मित्येव न स्या}त् । अथोप‚योगादिभेदेन प‚र‚स्प‚र‚भिन्नात्म‚{\tiny $_{lb}$}‚तेष्य‚ते भेदा \textbf{\textbf{नां} नामान्त‚र‚म्वा} स्यात् । ब‚हूनामेक‚मिति संज्ञा कृता स्यात् ।‚{\tiny $_{lb}$}‚ किङ्कार‚ण‚म [।] \textbf{अर्थं} प‚र‚स्प‚र‚भिन्न‚{\tiny $_{५}$}‚\textbf{म‚भ्युप‚ग‚म्य त‚थाभिधानात्} । एक‚मित्य‚भि‚{\tiny $_{lb}$}‚धानात् ॥
	{\color{gray}{\rmlatinfont\textsuperscript{§~\theparCount}}}
	\pend% ending standard par
      ‚{\tiny $_{lb}$}‚\textsuperscript{\textenglish{90/s}}

	  
	  \pstart \leavevmode% starting standard par
	अथ स्यात् प्राक्प्र‚ध्वंसाभावान्त‚र्व‚र्तिस‚त्तास‚म्ब‚न्धोऽनित्य‚ता । सा च कृत‚क‚{\tiny $_{lb}$}‚निष्प‚त्तिकाले निष्प‚न्नैव केव‚लं प्र‚ध्वंसेन्नोत्त‚र‚काल‚म‚भिव्य‚ज्य‚त इति [।]
	{\color{gray}{\rmlatinfont\textsuperscript{§~\theparCount}}}
	\pend% ending standard par
      ‚{\tiny $_{lb}$}‚

	  
	  \pstart \leavevmode% starting standard par
	त‚द‚प्य‚युक्तं । य‚तो याव‚त् प्र‚ध्वंसो नोत्प‚द्य‚ते ताव‚त् क‚थ‚म‚न्त‚राल‚व‚र्त्त्य‚नित्य‚ता ।‚{\tiny $_{lb}$}‚ प्र‚ध्वंसोत्प‚त्ताव‚पि क‚थ‚म‚न्त‚राल‚व‚र्त्तित्व‚म‚स्याः [।] कृत‚क‚स्व‚भाव‚त्व‚म्वा । भाव‚{\tiny $_{६}$}‚‚{\tiny $_{lb}$}‚स्यैवाभावात् ॥
	{\color{gray}{\rmlatinfont\textsuperscript{§~\theparCount}}}
	\pend% ending standard par
      ‚{\tiny $_{lb}$}‚

	  
	  \pstart \leavevmode% starting standard par
	\textbf{न‚न्वि}त्यादि प‚रः । \textbf{अन‚र्थान्त‚र‚हेतुत्वेपि} विनाश‚कार‚णान‚पेक्ष‚त्वेपि त्व‚न्म‚तेनानि‚{\tiny $_{lb}$}‚त्य‚तायाः । \textbf{भाव‚कालेऽनित्य‚तानिष्प‚त्तेः} । भाव‚स्य स‚त्ताकाले त‚स्या अनित्य‚ताया‚{\tiny $_{lb}$}‚ अनिष्प‚त्ते र्भावादुत्त‚र‚काल‚म‚नित्य‚ता भ‚व‚तीति म‚न्य‚ते । तुल्याऽत‚त्स्व‚भावा ।‚{\tiny $_{lb}$}‚ य‚थार्थान्त‚र‚हेतुत्वेपि निष्प‚त्तिः स्यात् [।] त‚योर्नानात्व‚न्त‚र्थाऽन‚र्थान्त‚र‚हेतुत्वेपीति‚{\tiny $_{७}$}‚‚{\tiny $_{lb}$}‚ \leavevmode\ledsidenote{\textenglish{34b/PSVTa}} \textbf{तुल्याऽत‚त्स्व‚भाव‚ता} ।
	{\color{gray}{\rmlatinfont\textsuperscript{§~\theparCount}}}
	\pend% ending standard par
      ‚{\tiny $_{lb}$}‚

	  
	  \pstart \leavevmode% starting standard par
	\textbf{नेत्या}दिना प्र‚तिविध‚त्ते । \textbf{अपूर्व‚स्व‚भाव‚लाभो निष्प‚त्तिरुच्य‚तेऽनित्य‚ता} च‚{\tiny $_{lb}$}‚ भाव‚निवृत्तिरूपा । त‚तो निष्प‚त्तेरेवाभावात् क‚थं विरुद्ध‚ध‚र्म‚संर्गः । य‚दि त‚र्हि नानि‚{\tiny $_{lb}$}‚त्य‚ता व‚स्तु स‚ती क‚थं साध्य‚साध‚न‚योस्तादात्म्य‚ल‚क्ष‚णः स‚म्ब‚न्ध इत्याह । \textbf{स एव‚{\tiny $_{lb}$}‚ हि भाव} इति । क्ष‚णे \textbf{स्थि}तिर्या \textbf{सैव ध‚र्मो} य‚स्येति । निवृत्तिध‚र्मा स्व‚भाव \textbf{एवा‚{\tiny $_{lb}$}‚नित्य‚तो}च्य‚ते स एव साध्यः ।‚{\tiny $_{१}$}‚ तेन तादात्म्यं हेतुसाध्य‚योर्व्य‚तिरिक्तार्थ‚त्व‚नित्य‚ता‚{\tiny $_{lb}$}‚ नीरूपा [।] तेन भाव‚स्यानित्य‚ता भ‚व‚तीतीत्येव‚मादिभिर्वाक्यैर्भाव‚स्य न किंचिद्रुपं‚{\tiny $_{lb}$}‚ विधीय‚तेऽपि तु दृष्टं रूपं नास्तीत्य‚य‚म‚र्थोभिधीय‚ते ध‚र्मान्त‚राभिधाने भाव‚निवृत्त्य‚{\tiny $_{lb}$}‚प्र‚तिपाद‚न‚प्र‚स‚ङ्गात् ।
	{\color{gray}{\rmlatinfont\textsuperscript{§~\theparCount}}}
	\pend% ending standard par
      ‚{\tiny $_{lb}$}‚

	  
	  \pstart \leavevmode% starting standard par
	य‚दि भाव एवानित्य‚ता क‚थ‚न्त‚र्हि श‚ब्द‚स्य ध‚र्मिणो नित्य‚ता ध‚र्म इति व‚च‚न‚भेद‚{\tiny $_{lb}$}‚ इत्य‚त आह । \textbf{व‚च‚न‚भेदे}पीत्यादि । \textbf{ध‚र्म‚ध‚र्मि‚{\tiny $_{२}$}‚त‚या} यो \textbf{व‚च‚न‚भेदो वाच‚कान्य‚त्व}‚{\tiny $_{lb}$}‚न्त‚त्र्\textbf{आपि निमित्त‚मुत्त‚र‚त्र व‚क्ष्या}मः ॥
	{\color{gray}{\rmlatinfont\textsuperscript{§~\theparCount}}}
	\pend% ending standard par
      ‚{\tiny $_{lb}$}‚

	  
	  \pstart \leavevmode% starting standard par
	एतेन य‚द‚प्युच्य‚ते ऽध्य य ना वि द्ध क ण् र्णो द्यो त क रा दि भिः । य‚दि‚{\tiny $_{lb}$}‚ तुलान्त‚योर्नामोन्नाम‚व‚त्कार्येत्प‚त्तिकाल एव कार‚ण‚विनाशः । य‚दि\edtext{}{\lemma{दि}\Bfootnote{? त‚दा}} कार्य‚का‚{\tiny $_{lb}$}‚र‚ण‚भावो न स्याद् य‚तः कार‚ण‚स्य विनाशः कार‚णोत्पादः ।एवं भाव एव नाश इति‚{\tiny $_{lb}$}‚ व‚च‚नादेव‚ञ्च कार‚णेन स‚ह कार्य‚मुत्प‚न्न‚मिति प्राप्तं । य‚दि च भा‚{\tiny $_{३}$}‚व एव नाशः‚{\tiny $_{lb}$}‚ प्र‚थ‚मेपि क्ष‚णे भाव‚स्य न स‚त्ता स्यात् । विनाशाद् भाव‚निवृत्तिश्च विनाशो लोक‚{\tiny $_{lb}$}‚‚{\tiny $_{lb}$}‚ \leavevmode\ledsidenote{\textenglish{91/s}}प्र‚तीतो न भाव एव । स‚र्व‚कालं च नाश‚स‚द्भावाद् भाव‚स्य स‚त्वं स्यात् । अथ कार‚{\tiny $_{lb}$}‚णोत्पादात् कार‚ण‚विनाशो भिन्न‚स्त‚दा कृत‚क‚स्व‚भाव‚त्व‚म‚नित्य‚त्व‚स्य न स्यात् ।‚{\tiny $_{lb}$}‚ व्य‚तिरिक्ते च नाशे जाते त‚स्य क्ष‚ण‚स्य न निवृत्तिरिति क‚थं क्ष‚णिक‚त्व‚मिति [।]
	{\color{gray}{\rmlatinfont\textsuperscript{§~\theparCount}}}
	\pend% ending standard par
      ‚{\tiny $_{lb}$}‚

	  
	  \pstart \leavevmode% starting standard par
	त‚द‚पास्तं [।] द्विविधो हि विनाश इष्य‚ते भा‚{\tiny $_{४}$}‚व‚निवृत्तिरूपो भाव‚श्च [।]‚{\tiny $_{lb}$}‚ तेनोत्प‚न्नो भावः कार्य‚ङ्क‚रोति कार्य‚काले च कार‚ण‚निवृत्तिरुपो विनाशो लोक‚प्र‚तीत‚{\tiny $_{lb}$}‚ एव [।] नाय‚म्भाव‚स्व‚भाव इष्य‚ते [।] नापि कार‚णोत्पादाद् अभिन्नो भिन्नो वा‚{\tiny $_{lb}$}‚ नीरूप‚त्वात् केव‚ल‚म‚स्य भेदाभेद‚प्र‚तिषेध एव क्रिय‚ते । त‚था च व‚क्ष्य‚ति [।]
	{\color{gray}{\rmlatinfont\textsuperscript{§~\theparCount}}}
	\pend% ending standard par
      ‚{\tiny $_{lb}$}‚

	  
	  \pstart \leavevmode% starting standard par
	\hphantom{.}भावे ह्येष विक‚ल्पः स्याद् विधेर्व‚स्त्व‚नुरोध‚त \href{http://sarit.indology.info/?cref=pv.3.278}{१ । २८१} इति ।
	{\color{gray}{\rmlatinfont\textsuperscript{§~\theparCount}}}
	\pend% ending standard par
      ‚{\tiny $_{lb}$}‚

	  
	  \pstart \leavevmode% starting standard par
	तेन व्य‚तिरिक्ते नाशे जाते क्ष‚ण‚स्य न निवृत्ति‚{\tiny $_{५}$}‚रित्य‚पास्तं । य‚त‚श्च द्वितीय‚{\tiny $_{lb}$}‚क्ष‚णोत्प‚त्तिकाल एव प्र‚थ‚म‚क्ष‚णे निवृत्तिस्तेनैक‚क्ष‚ण‚स्थायी भावो विनाश‚श‚ब्देनो‚{\tiny $_{lb}$}‚च्य‚तेऽयं च विनाशो भाव‚रूप‚त्वात्साध‚न‚स्व‚भाव एव । कार्योत्प‚त्तिकाले च निव‚र्त्तेत‚{\tiny $_{lb}$}‚ इति कार्य‚भिन्न‚काल‚भावी न चास्य स‚र्व‚काल‚म्भावो भाव‚स्यास‚त्वात् ।
	{\color{gray}{\rmlatinfont\textsuperscript{§~\theparCount}}}
	\pend% ending standard par
      ‚{\tiny $_{lb}$}‚

	  
	  \pstart \leavevmode% starting standard par
	य‚द्वा विन‚श्व‚रोऽयं विनाशोऽस्येति द्वाभ्यां ध‚र्म‚ध‚र्मिवाच‚काभ्याम‚विनाशिव्या‚{\tiny $_{lb}$}‚वृ‚{\tiny $_{६}$}‚त्त‚स्यैवैक‚स्य भाव‚स्य भेदान्त‚र‚प्र‚तिक्षेपाप्र‚तिक्षेपाभ्याम‚भिधानाद् भाव एव‚{\tiny $_{lb}$}‚ नाश उच्य‚ते इति स‚र्वं सुस्थं ।
	{\color{gray}{\rmlatinfont\textsuperscript{§~\theparCount}}}
	\pend% ending standard par
      ‚{\tiny $_{lb}$}‚

	  
	  \pstart \leavevmode% starting standard par
	य‚दि त‚र्हि भाव एवानित्य‚ता त‚दा भाव‚प्र‚त्य‚क्षीक‚र‚णे सापि प्र‚त्य‚क्षैवेति क‚स्मान्न‚{\tiny $_{lb}$}‚ त‚थैव निश्चीय‚त इत्य‚त आह । \textbf{तामि}त्यादि । \textbf{क्ष‚ण‚स्थितिध‚र्म‚तां स्व‚भाव}म‚नित्य‚ताख्यं‚{\tiny $_{lb}$}‚ \textbf{प‚श्य‚न्न}पि प्र‚त्य‚क्षीकुर्वाणोपि \textbf{न व्य‚व‚स्य}ति । न निश्चिनोतीति स‚म्ब‚{\tiny $_{७}$}‚न्धः ।
	{\color{gray}{\rmlatinfont\textsuperscript{§~\theparCount}}}
	\pend% ending standard par
      \textsuperscript{\textenglish{35a/PSVTa}}‚{\tiny $_{lb}$}‚

	  
	  \pstart \leavevmode% starting standard par
	क‚स्मात् क्ष‚ण‚स्थितिध‚र्म‚तास्व‚भाव इत्याह । \textbf{स्व‚हेतोरेव} स‚काशा\textbf{त्त‚था क्ष‚ण‚{\tiny $_{lb}$}‚स्थित}ध‚र्म‚त‚यो\textbf{त्प‚त्तेः} । किं पुनः प‚श्य‚न्न‚पि न व्य‚व‚स्य‚तीत्याह । \textbf{म‚न्द‚बुद्धि}रिति ।‚{\tiny $_{lb}$}‚ अनादिसंसाराभ्य‚स्त‚या नित्यादिरूपाविद्यावास‚न‚या म‚न्दा बुद्धिर्य‚स्य स त‚थाऽन्य‚था‚{\tiny $_{lb}$}‚ दृष्टे व‚स्तुनि स‚र्वात्म‚नां किमिति न निश्च‚यः स्याद् [।] अनेन तु योगिनां स‚त्य‚पि‚{\tiny $_{lb}$}‚ स‚दृश‚द‚र्श‚ने म‚न्द‚बुद्धित्वाभावात् क्ष‚णिक‚त्व‚नि‚{\tiny $_{१}$}‚श्च‚यो भ‚व‚तीत्युक्त‚म्भ‚व‚ति ।
	{\color{gray}{\rmlatinfont\textsuperscript{§~\theparCount}}}
	\pend% ending standard par
      ‚{\tiny $_{lb}$}‚

	  
	  \pstart \leavevmode% starting standard par
	य‚दि त‚र्ह्य‚विद्य‚या नानित्य‚त्वाध्य‚व‚सायो विन‚श्य‚त्य‚पि भावे माभूद‚नित्य‚ता‚{\tiny $_{lb}$}‚ध्य‚व‚साय इत्याशंक्य बाह्म‚म‚पि भ्रान्तिबीज‚माह । \textbf{स‚त्तोप‚ल‚म्भे}नेत्यादि । यः स‚त्ताया‚{\tiny $_{lb}$}‚ एवोप‚ल‚म्भो नाभाव‚स्य तेन स‚त्तोप‚ल‚म्भेन ।
	{\color{gray}{\rmlatinfont\textsuperscript{§~\theparCount}}}
	\pend% ending standard par
      ‚{\tiny $_{lb}$}‚

	  
	  \pstart \leavevmode% starting standard par
	एत‚दुक्त‚म्भ‚व‚ति [।] उत्त‚र‚क्ष‚णोत्पाद‚काल एव पूर्व‚क्ष‚ण‚विनाशात् पूर्वोत्त‚र‚योः‚{\tiny $_{lb}$}‚ क्ष‚ण‚योर‚भावेनाव्य‚व‚धानान्नैर‚न्त‚र्येणा‚{\tiny $_{२}$}‚न्य‚त्वाग्र‚हात् \textbf{स‚र्व‚दा} द्वितीयादिक्ष‚णेष्व‚पि‚{\tiny $_{lb}$}‚ ‚{\tiny $_{lb}$}‚ \leavevmode\ledsidenote{\textenglish{92/s}}स‚त्ताया एवोप‚ल‚म्भेन \textbf{त‚थाभावः} पूर्व‚दृष्ट‚स्य भावः स‚द्भाव‚स्त‚स्य या \textbf{श‚ङ्का} क‚दाचित्स‚{\tiny $_{lb}$}‚ एवाय‚मित्येवंरूपा भूता भ्रान्त‚स्यापि स एवाय‚मिति द‚र्श‚नाच्छ‚केत्याह । त‚या \textbf{विप्र}‚{\tiny $_{lb}$}‚ल‚ब्धो वंचितो \textbf{न व्य‚व‚स्य‚ति} ।
	{\color{gray}{\rmlatinfont\textsuperscript{§~\theparCount}}}
	\pend% ending standard par
      ‚{\tiny $_{lb}$}‚

	  
	  \pstart \leavevmode% starting standard par
	अग्निधूम‚योर‚पि त‚र्हि कार्य‚कार‚ण‚भाव‚निश्च‚यो न स्याद‚भावाव्य‚व‚धानेनान्य‚{\tiny $_{lb}$}‚त्वाग्र‚हादित्याह । \textbf{स‚दृशाप‚रोत्प‚त्ते}रि‚{\tiny $_{३}$}‚त्यादि । दृष्टं च स‚दृशाप‚र‚द‚र्श‚नं शुक्तिकादौ‚{\tiny $_{lb}$}‚ स‚त्य‚पि भेद‚भ्रान्तिनिमित्तं । एत‚च्च \textbf{नै रा त्म्य सि द्धौ\edtext{}{\edlabel{pvsvt_92-1}\label{pvsvt_92-1}\lemma{द्धौ}\Bfootnote{\href{http://sarit.indology.info/?cref=pv.1.225}{ Pramāṇavārtika 1: 225. }}}} विभ‚क्त‚मिति त‚त्रैवाब‚धार्य ।
	{\color{gray}{\rmlatinfont\textsuperscript{§~\theparCount}}}
	\pend% ending standard par
      ‚{\tiny $_{lb}$}‚

	  
	  \pstart \leavevmode% starting standard par
	तेन स‚र्वात्म‚ना पूर्व‚क्ष‚ण‚स‚दृश‚स्याप‚र‚स्योत्प‚त्तिस्त‚या विप्र‚ल‚ब्धो न पूर्व‚क्ष‚णा‚{\tiny $_{lb}$}‚दुत्त‚र‚क्ष‚ण‚म‚न्य‚त्वेनाध्य‚व‚स्य‚त्य‚पि तु स एवाय‚मित्य‚त एव न पूर्व‚क्ष‚ण‚स्य विनाश‚{\tiny $_{lb}$}‚प्र‚तीतिरुत्त‚र‚स्य चोत्प‚त्तिप्र‚तीतिः । अग्निधूम‚योस्त्वेकान्तेन विस‚{\tiny $_{४}$}‚दृश‚त्वान्नै‚{\tiny $_{lb}$}‚र‚न्त‚र्ये स‚त्य‚प्य‚न्य‚त्व‚ग्र‚हाद् भ‚व‚ति कार्य‚कार‚ण‚भाव‚निश्च‚यः । वा श‚ब्द‚स्त्व‚न‚व‚कॢ‚{\tiny $_{lb}$}‚प्तिसूच‚नार्थः ।
	{\color{gray}{\rmlatinfont\textsuperscript{§~\theparCount}}}
	\pend% ending standard par
      ‚{\tiny $_{lb}$}‚

	  
	  \pstart \leavevmode% starting standard par
	तेनाय‚म‚र्थो य‚दि स‚त्तोप‚ल‚म्भे व्य‚भिचारः स‚दृशाप‚रोत्प‚त्त्या वा \textbf{विप्र‚ल‚म्भः}‚{\tiny $_{lb}$}‚ स‚र्व‚दास्त्येव विप्र‚ल‚म्भ इत्येवं प‚रः ।
	{\color{gray}{\rmlatinfont\textsuperscript{§~\theparCount}}}
	\pend% ending standard par
      ‚{\tiny $_{lb}$}‚

	  
	  \pstart \leavevmode% starting standard par
	अथ‚वा किं पुनः प‚श्य‚न्न‚पि न व्य‚व‚स्य‚तीत्याह । स‚त्तोप‚ल‚म्भेन । पूर्वं यः‚{\tiny $_{lb}$}‚ स‚त्तोप‚ल‚म्भेन प्र‚तीय‚माने त‚द्भाव‚श‚ङ्का पूर्व‚दृष्ट‚{\tiny $_{५}$}‚भावारोप‚स्तेन विप्र‚ल‚ब्धः न व्य‚व‚{\tiny $_{lb}$}‚स्य‚ति । एव‚न्त‚र्ह्यादिक्ष‚ण‚द‚र्श‚न एवाध्य‚व‚सायः स्यात् । पूर्वं स‚त्तोप‚ल‚म्भाभावादित्या‚{\tiny $_{lb}$}‚शंक्याह । \textbf{स‚दृशाप‚रोत्प‚त्तिविप्र‚ल‚ब्धो} वेति । वा श‚ब्द‚श्चार्थे । प्र‚थ‚म‚क्ष‚ण‚स‚दृश‚स्य‚{\tiny $_{lb}$}‚ द्वितीय‚क्ष‚ण‚स्योत्प‚त्त्या च विप्र‚ल‚ब्धो न व्य‚व‚स्य‚ति । योगिनाम‚पि त‚र्हि \textbf{निश्च‚यो}‚{\tiny $_{lb}$}‚ न स्यादित्याह । \textbf{म‚न्द‚बुद्धिरि}ति । तेन बाह्याध्यात्मिक‚विप्र‚ल‚म्भ‚निमित्त‚स‚द्भा‚{\tiny $_{lb}$}‚‚{\tiny $_{६}$}‚वात् पृथ‚ग्ज‚नानां निश्च‚यः । योगिनान्तु स‚त्य‚पि स‚दृश‚द‚र्श‚ने प‚टुबुद्धित्वान्निश्च‚यो‚{\tiny $_{lb}$}‚ भ‚व‚त्येव ।
	{\color{gray}{\rmlatinfont\textsuperscript{§~\theparCount}}}
	\pend% ending standard par
      ‚{\tiny $_{lb}$}‚

	  
	  \pstart \leavevmode% starting standard par
	त‚स्मात् स्थित‚मेत‚त् [।] क्ष‚ण‚स्थितिध‚र्म‚तां प‚श्य‚न्न‚पि स‚दृशाप‚रोत्प‚त्त्या‚{\tiny $_{lb}$}‚ विप्र‚ल‚ब्धो न व्य‚व‚स्य‚तीति ॥
	{\color{gray}{\rmlatinfont\textsuperscript{§~\theparCount}}}
	\pend% ending standard par
      ‚{\tiny $_{lb}$}‚

	  
	  \pstart \leavevmode% starting standard par
	न‚नु भाव‚स्य क्ष‚णिक‚त्वे स‚ति पूर्वोत्त‚र‚क्ष‚णानां विभागेन प्र‚तिभासः स्यात् ।‚{\tiny $_{lb}$}‚ अप्र‚तिभास‚नाच्च क‚थ‚म्प‚श्य‚न्न‚पि न व्य‚व‚स्य‚तीत्युच्य‚ते । अथ नीलाद्य‚व्य‚तिरिक्त‚{\tiny $_{lb}$}‚\leavevmode\ledsidenote{\textenglish{35b/PSVTa}} त्वात् क्ष‚{\tiny $_{७}$}‚णिंक‚त्व‚स्य नील‚ग्र‚हे ग्र‚हः ।
	{\color{gray}{\rmlatinfont\textsuperscript{§~\theparCount}}}
	\pend% ending standard par
      ‚{\tiny $_{lb}$}‚

	  
	  \pstart \leavevmode% starting standard par
	युक्त‚मेत‚त् । किन्त्विद‚म‚त्र निरूप्य‚ते [।] किमिदं नील‚म‚क्ष‚णिक‚मुत क्ष‚ण‚{\tiny $_{lb}$}‚रूप‚म‚थ स‚न्तानः । त‚त्र य‚द्य‚क्ष‚णिक‚न्त‚दा नील‚प्र‚तिभासे क‚थं क्ष‚णिक‚त्व‚प्र‚तिभासः ।‚{\tiny $_{lb}$}‚ अथ क्ष‚ण‚रूपं । न‚न्विद‚मेवासिद्ध‚मिति क‚थं नील‚प्र‚तिभासे क्ष‚ण‚प्र‚तिभास उच्य‚ते ।‚{\tiny $_{lb}$}‚ ‚{\tiny $_{lb}$}‚ ‚{\tiny $_{lb}$}‚ \leavevmode\ledsidenote{\textenglish{93/s}}क्ष‚णाप्र‚तिभास‚नाच्च न स‚न्तान‚रूप‚स्य नील‚स्य प्र‚तिभासः । अथ नील‚मात्र‚प्र‚ति‚{\tiny $_{lb}$}‚भासे स‚ति क्ष‚{\tiny $_{१}$}‚णिक‚त्व‚प्र‚तिभासः । त‚द‚युक्तं । प्र‚तिभासाप्र‚तिभासाभ्यां हि प्र‚त्य‚{\tiny $_{lb}$}‚क्ष‚स्य ग्र‚ह‚णाग्र‚ह‚णे । नान्य‚था । त‚दाह । त‚द् य‚द‚पि गृह्णाति त‚त्प्र‚तिभासेनेति ।‚{\tiny $_{lb}$}‚ न च क्ष‚णानां प्र‚तिभास इत्युक्तं ।
	{\color{gray}{\rmlatinfont\textsuperscript{§~\theparCount}}}
	\pend% ending standard par
      ‚{\tiny $_{lb}$}‚

	  
	  \pstart \leavevmode% starting standard par
	य‚दि च नीलाद्य‚व्य‚तिरिक्तं क्ष‚णिक‚त्व‚न्त‚दा नील‚निश्च‚ये क्ष‚णिक‚त्व‚स्य निश्चित‚{\tiny $_{lb}$}‚त्वाद‚नुमान‚स्य वैय‚र्थ्यं स्यात् । य‚स्त्वाह । एक‚ज्ञान‚विष‚य‚त्व‚मेव क्ष‚णिक‚त्वं‚{\tiny $_{lb}$}‚ पूर्वोत्त‚र‚ज्ञान‚विष‚य‚{\tiny $_{२}$}‚त्व‚व्यावृत्त‚स्यैव चेदानीन्त‚न‚ज्ञान‚विष‚य‚त्व‚स्य प्र‚तिभास‚नात् [।]‚{\tiny $_{lb}$}‚ पूर्वाप‚र‚क्ष‚ण‚विल‚क्ष‚ण एव क्ष‚णः प्र‚त्य‚क्षेणानुभूत‚निश्चितोनुमानेन तु प्र‚त्य‚क्ष‚वृत्त‚मेव‚{\tiny $_{lb}$}‚ प‚रामृश्य‚त इति नानुमान‚स्य वैय‚र्थ्य‚मिति ।
	{\color{gray}{\rmlatinfont\textsuperscript{§~\theparCount}}}
	\pend% ending standard par
      ‚{\tiny $_{lb}$}‚

	  
	  \pstart \leavevmode% starting standard par
	त‚द‚युक्तं । य‚तो य‚द्येक‚ज्ञान‚स्याक्ष‚णिक‚त्व‚न्त‚दार्थ‚स्याप्य‚क्ष‚णिक‚त्वं स्यात् । अथ‚{\tiny $_{lb}$}‚ त‚स्य क्ष‚णिक‚त्व‚न्त‚त्कुतोऽव‚ग‚तं । त‚स्याप्येक‚ज्ञान‚विष‚य‚त्वादिति चे‚{\tiny $_{३}$}‚द‚न‚व‚स्थ‚यैव‚{\tiny $_{lb}$}‚ प्र‚तिप‚त्तिः क्ष‚णिक‚त्व‚स्य । अथ ज्ञान‚क्ष‚ण‚स्य प्र‚तिभासोऽभ्युप‚ग‚म्य‚तेसाव‚र्थ‚क्ष‚ण‚स्य‚{\tiny $_{lb}$}‚ किन्नाभ्युप‚ग‚म्य‚ते । अन्य‚थैक‚स्यापि क्ष‚ण‚स्यानेक‚ज्ञान‚विष‚त्वाद‚नेक‚त्वं स्यात् ।
	{\color{gray}{\rmlatinfont\textsuperscript{§~\theparCount}}}
	\pend% ending standard par
      ‚{\tiny $_{lb}$}‚

	  
	  \pstart \leavevmode% starting standard par
	\hphantom{.}भ ट्ट वा सु दे व स्त्वा ह । पूर्वोत्त‚र‚क्ष‚णानां विनाशेनाप्र‚तिभास‚न‚मेवाक्ष‚णि‚{\tiny $_{lb}$}‚क‚त्व‚प्र‚तिभास‚न‚म‚तोक्ष‚णिक‚त्व‚ग्राह‚क‚मेव स‚र्व‚म्प्र‚त्य‚क्षं केव‚लं क्ष‚णिक‚त्वा‚{\tiny $_{४}$}‚नुमानेन‚{\tiny $_{lb}$}‚ भ्रान्तं साध्य‚त इति [।]
	{\color{gray}{\rmlatinfont\textsuperscript{§~\theparCount}}}
	\pend% ending standard par
      ‚{\tiny $_{lb}$}‚

	  
	  \pstart \leavevmode% starting standard par
	एत‚द‚प्य‚युक्तं । प्र‚तिज्ञायाः प्र‚त्य‚क्ष‚बाधित‚त्वेनानुमान‚स्योत्थानाभावात् ।‚{\tiny $_{lb}$}‚ निर्विक‚ल्प‚क‚स्यापि प्र‚त्य‚क्ष‚स्य भ्रान्त‚त्वे स‚म्ब‚न्ध‚ग्र‚ह‚णाच्च । प‚श्य‚न्न‚पीति ग्र‚न्थ‚{\tiny $_{lb}$}‚विरोध‚श्च [।] त‚स्माद‚युक्त‚मुक्तं । प‚श्य‚न्न‚पि न व्य‚व‚स्य‚तीति ।
	{\color{gray}{\rmlatinfont\textsuperscript{§~\theparCount}}}
	\pend% ending standard par
      ‚{\tiny $_{lb}$}‚

	  
	  \pstart \leavevmode% starting standard par
	अत्रोच्य‚ते । य‚था ह्य‚र्थ‚क्ष‚णानां पौर्वाप‚र्य‚न्त‚था ज्ञान‚क्ष‚णानाम‚पि तेन पूर्व‚केण‚{\tiny $_{lb}$}‚ ज्ञान‚क्ष‚णेन पूर्व‚क एवार्थ‚क्ष‚णो गृह्य‚ते‚{\tiny $_{५}$}‚ नोत्त‚रः । उत्त‚रेणाप्युत्त‚र एव न पूर्व इति [।]‚{\tiny $_{lb}$}‚ एक‚स्मिन् ज्ञाने त‚योर‚प्र‚तिभास‚नात् क‚थ‚म्पूर्व‚स्माद‚य‚म‚न्य इति विभागेन प्र‚तिभासः‚{\tiny $_{lb}$}‚ स्यादिति चोद्य‚ते । स्व‚रूप‚प्र‚तिभास एव च भाव‚स्यान्य‚स्माद् विवेक‚प्र‚तिभासः‚{\tiny $_{lb}$}‚ सुमेरुभिन्न‚प्र‚तिभास‚व‚त् । स च क्ष‚ण‚स्याप्य‚स्त्ये वेति क‚थं न विवेक‚प्र‚तिभासः ।‚{\tiny $_{lb}$}‚ दृष्टो दृश्य‚त इति प्र‚तीतेश्च । अन्यो हि दृष्टः स्व‚भा‚{\tiny $_{६}$}‚वोन्य‚श्च दृश्य‚मानः । त‚था हि‚{\tiny $_{lb}$}‚ प्र‚थ‚म‚द‚र्शी दृश्य‚मान‚मेव स्व‚भाव‚म्भाव‚स्य प‚श्य‚ति न तु दृष्ट‚मित्य‚न‚योर्भेद एव ।‚{\tiny $_{lb}$}‚ केव‚ल‚मेकान्त‚स‚दृश‚योः पूर्वाप‚र‚क्ष‚ण‚योर‚भावेनाव्य‚व‚धानाद् घ‚ट‚प‚टादिव‚द् विभाग‚{\tiny $_{lb}$}‚प्र‚तिप‚त्तिर्न भ‚व‚ति । नापि विभागेनाप्र‚तिभासाद‚भेदोपि [।] न हि शुक्ति‚{\tiny $_{lb}$}‚कायान्त‚देवेद‚म‚स्म‚दीयं र‚ज‚त‚मिति प्र‚व‚र्त्त‚मान‚स्य शुक्तिकार‚ज‚त‚यो‚{\tiny $_{७}$}‚र्विवेक‚प्र‚तिभा- \leavevmode\ledsidenote{\textenglish{36a/PSVTa}}‚{\tiny $_{lb}$}‚ साभावाद‚भेदोपि । त‚स्माद् य‚थात्र निर्विक‚ल्प‚के ज्ञाने शुक्तिकायाः स्व‚रूप‚प्र‚तिभास‚{\tiny $_{lb}$}‚ एवान्य‚स्माद विवेक‚प्र‚तिभासः । त‚थैक‚स्यापि क्ष‚ण‚स्य स्यात [।] केव‚लं पूर्वः क्ष‚णः‚{\tiny $_{lb}$}‚ ‚{\tiny $_{lb}$}‚ \leavevmode\ledsidenote{\textenglish{94/s}}क‚स्मान्न विभागेन स्म‚र्य‚त इति य‚दि प‚रं चोद्यं स्यात्त‚त्र चोक्त‚मेव स‚दृशाप‚रोत्प‚त्ति‚{\tiny $_{lb}$}‚विप्र‚ल‚ब्धैर्न स्म‚र्य‚त इति ।
	{\color{gray}{\rmlatinfont\textsuperscript{§~\theparCount}}}
	\pend% ending standard par
      ‚{\tiny $_{lb}$}‚

	  
	  \pstart \leavevmode% starting standard par
	न‚नु त‚थापि क्ष‚णो न प्र‚तिभास‚ते । एकाण्व‚त्य‚य‚काल‚त्वेनैव भाव‚स्याप्र‚{\tiny $_{१}$}‚तीतेः ।‚{\tiny $_{lb}$}‚ न तु य‚द्येक‚स्मिन् क्ष‚णेस्याप्र‚तिभासः क‚थ‚म‚क्ष‚णिक‚स्य प्र‚तिभासः प्र‚तिक्ष‚ण‚म‚प्र‚ति‚{\tiny $_{lb}$}‚भास‚नात् । उत्प‚द्य‚मान‚स्य च भाव‚स्य पूर्वाप‚र‚रूप‚विविक्त‚स्य प्र‚त्य‚क्षेण ग्र‚ह‚णात्‚{\tiny $_{lb}$}‚ क‚थं क्ष‚णिक‚त्व‚ग्र‚हः । नाप्य‚क्ष‚णिकः प्र‚तीय‚ते पूर्वाप‚र‚काल‚योर‚प्र‚तिभासादेव त‚त्स‚{\tiny $_{lb}$}‚म्ब‚न्धित‚येदानीं प्र‚त्य‚क्षेऽप्र‚तिभास‚नात् पूर्व‚काल‚स‚म्ब‚न्धिस्व‚भाव‚स्येदानीम‚प्र‚तिभास‚{\tiny $_{lb}$}‚ ए‚{\tiny $_{२}$}‚व विनाशोऽन्य‚स्व‚भाव‚स्य प्र‚तिभास एवोत्पाद इति क‚थ‚मुच्य‚ते पूर्वोत्त‚र‚क्ष‚णानां‚{\tiny $_{lb}$}‚ विनाशोत्पादाप्र‚तिभास‚नाद् अक्ष‚णिक इति । नाप्य‚नेक‚क्ष‚ण‚रूप इदानीन्त‚नः कालो‚{\tiny $_{lb}$}‚नेक‚क्ष‚ण‚स‚म्भ‚वे गृहीतादिरूप‚ताऽस्य स्यात् न व्य‚र्थ‚ता ।
	{\color{gray}{\rmlatinfont\textsuperscript{§~\theparCount}}}
	\pend% ending standard par
      ‚{\tiny $_{lb}$}‚

	  
	  \pstart \leavevmode% starting standard par
	न हि प्र‚त्य‚क्ष‚भाविना निश्च‚येनेदानीमेवेद‚म‚स्तीति निश्चीय‚ते [।] किन्त‚र्ही‚{\tiny $_{lb}$}‚दानीम‚स्तीति । अनुमानेन त्विदानीमेवा‚{\tiny $_{४}$}‚स्तीति साध्य‚ते ।
	{\color{gray}{\rmlatinfont\textsuperscript{§~\theparCount}}}
	\pend% ending standard par
      ‚{\tiny $_{lb}$}‚

	  
	  \pstart \leavevmode% starting standard par
	त‚स्मात् स्थित‚मेत‚त् [।] प‚श्य‚न्न‚पि न व्य‚व‚स्य‚तीत्यादि ।
	{\color{gray}{\rmlatinfont\textsuperscript{§~\theparCount}}}
	\pend% ending standard par
      ‚{\tiny $_{lb}$}‚

	  
	  \pstart \leavevmode% starting standard par
	न‚नु य‚दि नित्यं स‚दृश इति प्र‚त्य‚क्षेण निश्च‚यः स्यात्स एवाय‚मिति बुद्धि‚{\tiny $_{lb}$}‚र्भ्रान्तिर्याव‚ता स‚र्व‚दा स एवाय‚मिति प्र‚तीतिर्दृढ‚रूपोत्प‚द्य‚त इति क‚थं भ्रान्तिस्त‚{\tiny $_{lb}$}‚दाह भ‚ट्टः\edtext{}{\edlabel{pvsvt_94-1}\label{pvsvt_94-1}\lemma{ट्टः}\Bfootnote{\href{http://sarit.indology.info/?cref=\%C5\%9Bv}{Ślokavārtika:  श‚ब्द‚नित्त्य‚ताऽधिक‚र‚णे ३७३, ३७४} त‚दाऽस्तित्वाऽधि‚{\tiny $_{lb}$}‚क‚त्वाच्च साधितं ।}} ॥
	{\color{gray}{\rmlatinfont\textsuperscript{§~\theparCount}}}
	\pend% ending standard par
      ‚{\tiny $_{lb}$}‚
	  \bigskip
	  \begingroup
	
	    
	    \stanza[\smallbreak]
	  {\normalfontlatin\large ``\qquad}नित्यं स‚दृश एवेति य‚त्र रूढा म‚तिर्भ‚वेत् ।&‚{\tiny $_{lb}$}‚स इति प्र‚त्य‚भिज्ञानं भ्रान्तिस्त‚त्राव‚क‚ल्प‚ते ॥&‚{\tiny $_{lb}$}‚इह नित्यं स एवेति विज्ञा‚{\tiny $_{५}$}‚नं जाय‚ते दृढं ।&‚{\tiny $_{lb}$}‚त‚द‚स्तित्वातिरेकाच्च प्रामाण्य‚न्त‚स्य युज्य‚ते ॥&‚{\tiny $_{lb}$}‚देश‚कालादिभेदेन त‚त्रास्त्य‚व‚स‚रो मितः ।&‚{\tiny $_{lb}$}‚इदानीन्त‚न‚म‚स्तित्वं न हि पूर्व‚धियो ग‚तं ॥{\normalfontlatin\large\qquad{}"}\&[\smallbreak]
	  
	  
	  
	  \endgroup
	‚{\tiny $_{lb}$}‚

	  
	  \pstart \leavevmode% starting standard par
	न च स‚र्व‚दानुमानेन स‚दृश‚निश्च‚यः । प्र‚त्य‚क्ष‚प्र‚त्य‚भिज्ञाबाधित‚त्वात् प्र‚त्य‚भि‚{\tiny $_{lb}$}‚ज्ञाया अनुमानोत्थानाभावात् । न च प्र‚त्य‚भिज्ञाया अप्रामाण्य‚मित‚रेत‚राश्र‚य‚दोषात् ।‚{\tiny $_{lb}$}‚ य‚तो न याव‚द‚प्रामाण्य‚म‚स्यास्ता‚{\tiny $_{६}$}‚व‚न्नानुमान‚स्योत्थानं याव‚च्च नानुमानोत्थान‚न्ता‚{\tiny $_{lb}$}‚व‚न्नास्या अप्रामाण्य‚म‚न्योन्याश्र‚य‚दोष इति ।
	{\color{gray}{\rmlatinfont\textsuperscript{§~\theparCount}}}
	\pend% ending standard par
      ‚{\tiny $_{lb}$}‚

	  
	  \pstart \leavevmode% starting standard par
	अत्रोच्य‚ते । स इत्य‚नेन पूर्व‚काल‚स‚म्ब‚न्धी स्व‚भावो विष‚यीक्रिय‚ते । अय‚{\tiny $_{lb}$}‚मित्य‚नेन च व‚र्त्त‚मान‚काल‚स‚म्ब‚न्धी । अन‚योश्च भेदो न च क‚थंचिद‚भेदो व‚र्त्त‚मान‚{\tiny $_{lb}$}‚‚{\tiny $_{lb}$}‚ ‚{\tiny $_{lb}$}‚ \leavevmode\ledsidenote{\textenglish{95/s}}काल‚भाविरूपैक‚स्व‚भाव‚त्वाद् व‚स्तुनः । त‚स्माद् भेद एव प्र‚त्य‚भिज्ञाने स‚ति भास‚त‚{\tiny $_{lb}$}‚ इति क‚थ‚{\tiny $_{७}$}‚म‚नेन क्ष‚णिक‚त्वानुमान‚बाधा ।
	{\color{gray}{\rmlatinfont\textsuperscript{§~\theparCount}}}
	\pend% ending standard par
      \textsuperscript{\textenglish{36b/PSVTa}}‚{\tiny $_{lb}$}‚

	  
	  \pstart \leavevmode% starting standard par
	य‚द्वा व‚स्तुनः पूर्व‚काल‚स‚म्ब‚न्धित्व‚मिदानीम‚स‚देव । पूर्व‚कालाभावात् । स‚त्त्वे‚{\tiny $_{lb}$}‚ वास्य व‚र्त्त‚मान‚काल‚स‚म्ब‚न्धित्व‚मेव स्यान्न पूर्व‚काल‚स‚म्ब‚न्धित्वं विरोधादित्युक्तं ।‚{\tiny $_{lb}$}‚ त‚स्मात् पूर्व‚काल‚स‚म्ब‚न्धित्व‚स्यास‚तो ग्राह‚कः स इति ज्ञानांशो भ्रान्तोऽन्य‚था व‚स्तुनः‚{\tiny $_{lb}$}‚ स्प‚ष्ट‚बालाद्य‚व‚स्था ग्राह‚कः स्यान्न च भ‚व‚ति । त‚स्माद् भ्रान्तात् पूर्व‚दृष्ट‚रूपा‚{\tiny $_{१}$}‚‚{\tiny $_{lb}$}‚रोपेण स एवाय‚मिति ज्ञानात् क‚थ‚म‚नुमान‚बाधा । य‚त्र चाक्ष‚व्यापार‚स्त‚द्ग्राह‚कं‚{\tiny $_{lb}$}‚ प्र‚त्य‚क्ष‚मुत्प‚द्य‚ते । न च पूर्व‚कालास्तित्वेऽधुना क्व व्यापारोऽस‚न्निहित‚त्वात् । नापि‚{\tiny $_{lb}$}‚ त‚द‚भेदेन त‚त्क‚थं पूर्वोत्त‚र‚कालास्तित्व‚योरैक्य‚ग्राह‚कं ज्ञानं प्र‚त्य‚क्षं स्यात् । य‚दा च‚{\tiny $_{lb}$}‚ बालाद्य‚व‚स्थायां दृष्टः वृद्धाव‚स्थायां प्र‚त्य‚भिज्ञाय‚ते । न त‚त्र वा प्र‚त्य‚भिज्ञाने‚{\tiny $_{lb}$}‚ नित्य‚त्वं प्र‚तिभास‚ते । केव‚{\tiny $_{२}$}‚लं स एवाय‚मिति त‚त्त्व‚म‚ध्य‚व‚सीय‚ते । न च त‚त्त्व‚म्बा‚{\tiny $_{lb}$}‚ल‚वृद्धाव‚स्थ‚योर्भेदात् । नापि त‚त्त्व‚ग्र‚ह‚णान्य‚थानुप‚प‚त्त्या नित्य‚त्वादिक‚ल्प‚ना ।‚{\tiny $_{lb}$}‚ स‚दृशाप‚रोत्प‚त्तिभ्रान्तिनिमित्ताद‚प्युत्प‚त्तिस‚म्भ‚वात् । लून‚पुन‚र्जातेष्विव केशेषु ।‚{\tiny $_{lb}$}‚ न चात्र केश‚त्व‚सामान्याद् भिन्नात् प्र‚त्य‚भिज्ञानं । त‚दिहेति बुद्ध्युत्पाद‚प्र‚स‚ङ्गात् ।‚{\tiny $_{lb}$}‚ नाप्य‚भिन्नाद‚न्य‚त्रानुग‚मे वाऽनुग‚त‚व्या‚{\tiny $_{३}$}‚वृत्त‚रूप‚योः । प‚र‚म्प‚रासंश्लेषेणैकान्त‚{\tiny $_{lb}$}‚भिन्न‚त्वात् । त‚त‚श्च त एवामी केशा इति प्र‚त्य‚भिज्ञा स‚दृशाप‚रोत्प‚त्तिनिमित्तैवात‚{\tiny $_{lb}$}‚ एव भ्रान्तिः । त‚था घ‚टादिष्व‚स्याभ्रान्त‚त्वाशंक‚यानुमान‚स्योत्थानं युज्य‚त एव ।‚{\tiny $_{lb}$}‚ नापीत‚रेत‚राश्र‚य‚त्वं य‚तो नानुमानेन प्र‚त्य‚भिज्ञाया अप्रामाण्यं क्रिय‚तेऽपि तु‚{\tiny $_{lb}$}‚ ज्ञाय‚ते स्व‚हेतुत एवाप्र‚माण‚रूपाया निष्प‚त्तेः । दृष्टो दृश्य‚त‚{\tiny $_{४}$}‚ इति ग्र‚हात् ।‚{\tiny $_{lb}$}‚ दृष्ट‚त्वं हि न पूर्व‚म्भाव‚स्यास्ति । त‚दा दृश्य‚मानैक‚रूप‚त्वात् । नाप्य‚धुनाऽत एव ।‚{\tiny $_{lb}$}‚ नापि पूर्व‚काल‚स‚म्ब‚न्धित्वं दृश्य‚मान‚स्येदानीं पूर्व‚म‚भावात् । त‚स्माद् दृष्ट‚त्व‚{\tiny $_{lb}$}‚मारोप्य ग्राहिका प्र‚त्य‚भिज्ञा भ्रान्त‚त्वाद‚प्र‚माणैवोत्प‚द्य‚ते । नाप्य‚स्या अप्रामाण्य‚नि‚{\tiny $_{lb}$}‚मित्त‚म‚नुमान‚स्योत्थान‚म‚पि तु साध्य‚प्र‚तिब‚न्ध‚निमित्त‚म‚तः प्र‚त्य‚भिज्ञाया बाध‚कं ।‚{\tiny $_{lb}$}‚ विस्त‚र‚स्त्व‚{\tiny $_{५}$}‚यं प्र‚त्य‚भिज्ञाभ‚ङ्ग‚विचारो नै रा त्म्य सि द्धौ कृत इति त‚त्रैवाव‚धार्यः ॥
	{\color{gray}{\rmlatinfont\textsuperscript{§~\theparCount}}}
	\pend% ending standard par
      ‚{\tiny $_{lb}$}‚

	  
	  \pstart \leavevmode% starting standard par
	त‚स्मात् स्थित‚मेत‚त् [।] तां पुनः क्ष‚ण‚स्थितिध‚र्म‚तां स्व‚भाव‚म्प‚श्य‚न्न‚पि‚{\tiny $_{lb}$}‚ स‚दृशाप‚रोत्प‚त्तिविप्र‚ल‚ब्धः पूर्व‚क्ष‚ण‚विनाशाविनिश्च‚यान्न व्य‚व‚स्य‚तीति ॥
	{\color{gray}{\rmlatinfont\textsuperscript{§~\theparCount}}}
	\pend% ending standard par
      ‚{\tiny $_{lb}$}‚

	  
	  \pstart \leavevmode% starting standard par
	क‚थं पुन‚र्ग‚म्य‚ते स‚दृशाप‚रोत्प‚त्त्या विप्र‚ल‚ब्धो न व्य‚व‚स्य‚तीत्य‚त आह । \textbf{अन्त्ये}‚{\tiny $_{lb}$}‚त्यादि । स‚दृश‚क्ष‚णान्त‚राप्र‚तिस‚न्धायी क्ष‚णो\textbf{ऽन्त्य‚क्ष}ण‚स्त‚{\tiny $_{६}$}‚\textbf{द्द‚र्शिनां} न‚ष्ट इति \textbf{निश्च‚{\tiny $_{lb}$}‚‚{\tiny $_{lb}$}‚ \leavevmode\ledsidenote{\textenglish{96/s}}यात्} । अव‚ग‚म्य‚ते प्राग‚प्य‚य‚म्प्र‚तिक्ष‚ण‚मेव न‚श्य‚ति केव‚लं य‚थोक्तादेव विप्र‚ल‚म्भ‚{\tiny $_{lb}$}‚हेतोर्न निश्चीय‚ते ॥
	{\color{gray}{\rmlatinfont\textsuperscript{§~\theparCount}}}
	\pend% ending standard par
      ‚{\tiny $_{lb}$}‚

	  
	  \pstart \leavevmode% starting standard par
	न‚न्व‚न्त्य‚क्ष‚ण‚द‚र्शिनोपि क‚थ‚न्न‚श्व‚र‚त्व‚निश्च‚यो याव‚ता त‚दापि स‚त्तोप‚ल‚म्भो‚{\tiny $_{lb}$}‚ऽस्ति । न हि स‚त्तायामेवोप‚ल‚भ‚मान‚स्त‚द‚भाव‚म‚वैतीत्याह । \textbf{प‚श्चा}दित्यादि । न‚{\tiny $_{lb}$}‚ ब्रूमोन्त्य‚क्ष‚ण‚द‚र्श‚न‚मात्रान्निश्च‚यो भ‚व‚तीति किन्त्व‚न्त्यं क्ष‚णं दृष्ट्वा प‚श्चाद् वि‚{\tiny $_{७}$}‚‚{\tiny $_{lb}$}‚ \leavevmode\ledsidenote{\textenglish{38a/PSVTa}} \edtext{\textsuperscript{*}}{\edlabel{pvsvt_96-1}\label{pvsvt_96-1}\lemma{*}\Bfootnote{37th leaf is missing. }}भ‚व‚ति [।] त‚दा नोप‚ल‚भ्य‚ते \textbf{त‚त्त‚स्य कार्यं} । न चाग्निकाष्टादिस‚न्निधाने भ‚व‚तो‚{\tiny $_{lb}$}‚ धूम‚स्याप‚नीतेऽश्वादौ अनुप‚ल‚म्भोस्ति । अग्न्यादौ त्व‚नीते भ‚व‚त्य‚नुप‚ल‚म्भः । एव‚{\tiny $_{lb}$}‚म्प‚र‚स्प‚र‚स‚हितौ प्र‚त्य‚क्षानुप‚ल‚म्भाव‚भिम‚तेष्वेव कार‚णेष्व‚स‚न्दिग्धं कार‚ण‚त्वं‚{\tiny $_{lb}$}‚ साध्य‚त इति । \textbf{त‚च्चे}ति य‚थोक्तं कार्य‚ल‚क्ष‚णं \textbf{धूमेस्ति} त‚स्माद‚ग्नेरेव धूमो भ‚व‚ति ।‚{\tiny $_{lb}$}‚ स‚र्व‚कालं चाग्निस‚न्निधाने भ‚{\tiny $_{१}$}‚व‚तो धूम‚स्यान‚ग्निज‚न्य‚त्वं क‚दाचित्स‚द‚स‚तोर‚ज‚न्य‚त्वा‚{\tiny $_{lb}$}‚‚{\tiny $_{lb}$}‚ ‚{\tiny $_{lb}$}‚ \leavevmode\ledsidenote{\textenglish{97/s}}द‚हेतुत्वाद‚दृश्य‚हेतुत्वाद्वोच्य‚ते । त‚त्र न ताव‚त् प्र‚थ‚मः प‚क्ष इति तृ ती ये प रि च्छे दे‚{\tiny $_{lb}$}‚ अश‚क्तं स‚र्व‚मिति चेदि त्य‚त्रान्त‚रे \href{http://sarit.indology.info/?cref=pv.2.4}{प्र० वा० ३।४} व‚क्ष्यामः । नाप्य‚हेतुत्व‚मिति त‚त्रैव‚{\tiny $_{lb}$}‚ व‚क्ष्य‚ति । नाप्य‚दृश्य‚हेतुत्वं धूम‚स्य । अग्न्यादिसाम‚ग्र्य‚न्व‚य‚व्य‚तिरेकानुविधानात् ॥
	{\color{gray}{\rmlatinfont\textsuperscript{§~\theparCount}}}
	\pend% ending standard par
      ‚{\tiny $_{lb}$}‚

	  
	  \pstart \leavevmode% starting standard par
	अथ स्याद् [।] अदृश्य‚स्यायं स्व‚भावो य‚द‚ग्न्यादिस‚न्निधा‚{\tiny $_{२}$}‚न एव धूमं क‚र्पूरा‚{\tiny $_{lb}$}‚दिदाह‚काले सुग‚न्धादियुक्तं च क‚रोति नान्य‚देति । त‚त्किम‚ग्निम‚न्त‚रेण क‚दा‚{\tiny $_{lb}$}‚ चिद् धूमोत्प‚त्तिर्द‚ष्टा येनैव‚मुच्य‚ते । नेति चेत् । त‚त्क‚थ‚न्नाग्निकार्यो धूम‚स्त‚द्भावे‚{\tiny $_{lb}$}‚ भावात् । धूमोत्प‚त्तिकाले चाग्निः स‚र्व‚दा प्र‚तीय‚मानोपि काक‚तालीय‚न्यायेनाव‚{\tiny $_{lb}$}‚स्थित इत्य‚लौकिकोयं व्य‚प‚देशः ।
	{\color{gray}{\rmlatinfont\textsuperscript{§~\theparCount}}}
	\pend% ending standard par
      ‚{\tiny $_{lb}$}‚

	  
	  \pstart \leavevmode% starting standard par
	अथ‚वा स एवादृश्य‚स्य स्व‚भावो य‚द्य‚ग्निना नोप‚{\tiny $_{३}$}‚क्रिय‚ते त‚त्किम‚ग्न्यादिस‚{\tiny $_{lb}$}‚न्निधान एव धूमं क‚रोति न पूर्व‚न्न प‚श्चात् । त‚स्माद‚ग्न्यादिस‚न्निधान एवा‚{\tiny $_{lb}$}‚स्व‚भावो धूम‚ज‚न‚को भ‚व‚ति नान्य‚देति त‚त्रापि पार‚म्प‚र्येण धूम‚स्याग्निज‚न्य‚त्व‚मेव‚{\tiny $_{lb}$}‚ स्यात् ।
	{\color{gray}{\rmlatinfont\textsuperscript{§~\theparCount}}}
	\pend% ending standard par
      ‚{\tiny $_{lb}$}‚

	  
	  \pstart \leavevmode% starting standard par
	किञ्च । य‚था देश‚कालादिक‚म‚न्त‚रेण धूम‚स्यानुत्प‚त्तेस्त‚द‚पेक्षा प्र‚तीय‚ते त‚था‚{\tiny $_{lb}$}‚ स‚र्व‚दाग्निम‚न्त‚रेणानुत्प‚त्तिद‚र्श‚नाद‚ग्न्य‚पेक्षाऽस्य केन वा‚{\tiny $_{४}$}‚र्येत । त‚द‚पेक्षा च‚{\tiny $_{lb}$}‚ त‚त्कार्य‚तैव । य‚था चादृश्य‚भाव एव धूम‚स्य भावात्त‚ज्ज‚न्य‚त्व‚मिष्य‚ते त‚था स‚र्व‚{\tiny $_{lb}$}‚काल‚म‚ग्निभावे भाव‚द‚र्श‚नाद‚ग्निज‚न्य‚त्वं किन्नेष्य‚ते । याव‚तां च स‚न्निधान एवो‚{\tiny $_{lb}$}‚त्प‚द्य‚मानो भावो दृश्य‚ते ताव‚तामेव हेतुत्वं स‚र्वेषां प्राग्भाव‚स्य तुल्य‚त्वात् । त‚था‚{\tiny $_{lb}$}‚ चाग्न्य‚दृश्यादिसाम‚ग्रीज‚न्य‚त्वं धूम‚स्येति कुतोग्निव्य‚भिचारः ॥
	{\color{gray}{\rmlatinfont\textsuperscript{§~\theparCount}}}
	\pend% ending standard par
      ‚{\tiny $_{lb}$}‚

	  
	  \pstart \leavevmode% starting standard par
	अन्य‚स्त्वाह । भ‚{\tiny $_{५}$}‚व‚त्व‚ग्निधूम‚योः कार्य‚कार‚ण‚भाव‚त‚स्त‚थापि न त‚योरेकेन‚{\tiny $_{lb}$}‚ ज्ञानेन ग्र‚ह‚ण‚म्भिन्न‚काल‚त्वात् । नाप्य‚नेन पूर्व‚केण हि निर्विक‚ल्प‚केन पूर्व‚क‚म्व‚स्तु‚{\tiny $_{lb}$}‚ मात्रं गृहीतं न तु कार‚ण‚रूपं कार्य‚स्य भावित्वेनाप्र‚त्य‚क्ष‚त्वात् । उत्त‚रेणाप्युत्त‚र‚म्व‚स्तु‚{\tiny $_{lb}$}‚मात्रं गृह्य‚ते न तु कार्य‚रूपं कार‚ण‚स्यातीत‚त्वेनाग्र‚हात् । नापि स‚विक‚ल्प‚केन‚{\tiny $_{lb}$}‚ त‚त्राप्य‚स्य चोद्य‚स्य तुल्य‚{\tiny $_{६}$}‚त्वात् । तेनेद‚म‚स्मादुत्प‚न्न‚मिति न केन‚चिद् गृहीत‚{\tiny $_{lb}$}‚म‚त एव न स्म‚र‚णेनापि गृह्य‚तेनुभ‚वाभावादिति ।
	{\color{gray}{\rmlatinfont\textsuperscript{§~\theparCount}}}
	\pend% ending standard par
      ‚{\tiny $_{lb}$}‚

	  
	  \pstart \leavevmode% starting standard par
	अत्रोच्य‚ते । कार्य‚स्य ताव‚द‚नुत्प‚न्नाव‚स्थायाम‚स‚त्त्वादेव न कार‚ण‚स‚म्ब‚न्धित्वं‚{\tiny $_{lb}$}‚ निष्प‚न्नाव‚स्थायाम‚प्येवं । निर‚पेक्ष‚त्वात् [।] त‚था कार‚ण‚म‚पि कार्य‚निष्प‚त्त्य‚निष्प‚त्त्य‚{\tiny $_{lb}$}‚व‚स्थायां कार्यास‚म्ब‚न्ध्येव । नाप्य‚न‚योः कार्य‚कार‚ण‚भावः स‚म्ब‚न्धो भिन्न‚{\tiny $_{७}$}‚काल- \leavevmode\ledsidenote{\textenglish{38b/PSVTa}}‚{\tiny $_{lb}$}‚ त्वात् । केव‚ल‚म‚स्येदं कार्यं कार‚णं चेति क‚ल्पितोयं व्य‚प‚देशः । तेन हेतोः स‚का‚{\tiny $_{lb}$}‚शात् स्व‚रूप‚लाभ एव कार्य‚त्वं । कार‚ण‚स्यापि कार्यं प्र‚ति प्राग्भाव एव कार‚ण‚त्वं‚{\tiny $_{lb}$}‚ स चात्म‚लाभः प्राग्भाव‚श्च भाव‚स्याभिन्न‚त्वात् प्र‚त्य‚क्ष‚गृहीत एव चेति क‚थं न प्र‚त्य‚क्ष‚{\tiny $_{lb}$}‚‚{\tiny $_{lb}$}‚ \leavevmode\ledsidenote{\textenglish{98/s}}ग्राह्यः कार्य‚कार‚ण‚भावः केव‚लं कार्य‚द‚र्श‚ने स‚तीद‚म‚स्य कार्य‚कार‚णं चेति व्य‚व‚ह्रिय‚ते ।‚{\tiny $_{lb}$}‚ य‚तो नाकार्य‚कार‚ण‚योः का‚{\tiny $_{१}$}‚र्य‚कार‚ण भावः स‚म्भ‚व‚ति । नापि कार्य‚कार‚ण‚भाव‚यो‚{\tiny $_{lb}$}‚गात्त‚योः कार्य‚कार‚ण‚ताऽभिन्ना क‚र्त्तुं श‚क्य‚ते विरोधात् । नापि भिन्ना त‚योः‚{\tiny $_{lb}$}‚ स्व‚रूपेणाकार्य‚कार‚ण‚ता प्र‚स‚ङ्गात् । स्व‚रूपेण कार्य‚कार‚ण‚योर‚पि किं कार्य कार‚ण‚{\tiny $_{lb}$}‚भावेनार्थान्त‚रेण क‚ल्पितेन स्व‚रूपेणैव कार्य‚कार‚ण‚रूप‚त्वात् [।] त‚स्मात् पूर्वोत्त‚र‚भाव‚{\tiny $_{lb}$}‚ एव त‚योः कार्य‚कार‚ण‚भावः । तेन पूर्व‚के व‚स्तुनि गृह्य‚{\tiny $_{२}$}‚माणे कार्यंप्र‚त्यान‚न्त‚र्य‚कार‚{\tiny $_{lb}$}‚णात्म‚कं गृहीत‚मेव । उत्त‚रेण च ज्ञानेनोत्त‚र‚म्व‚स्तु कार‚णान‚न्त‚रं गृह्य‚माणं कार्या‚{\tiny $_{lb}$}‚त्म‚क‚मेव गृह्य‚ते [।] त‚दान‚न्त‚र्य‚स्य त‚द‚भिन्न‚स्व‚भाव‚त्वात् । अत एवास्माद‚न‚न्त‚र‚{\tiny $_{lb}$}‚मिद‚म्भ‚व‚तीति स्म‚र‚ण‚म‚पि भ‚व‚त्यान‚न्त‚र्य‚स्यानुभूत‚त्वादिति ।
	{\color{gray}{\rmlatinfont\textsuperscript{§~\theparCount}}}
	\pend% ending standard par
      ‚{\tiny $_{lb}$}‚

	  
	  \pstart \leavevmode% starting standard par
	\hphantom{.}अ वि द्ध क र्ण्ण स्त्वाह । अविनाभावित्वं एकं दृष्ट्वा द्वितीयादिद‚र्श‚ने स‚ति‚{\tiny $_{lb}$}‚ सिध्य‚ति [।] न च क्ष‚णिक‚वादिनो द्र‚ष्टुर‚{\tiny $_{३}$}‚र?व‚स्थान‚म‚स्ति । न चान्येनानुभूते‚{\tiny $_{lb}$}‚र्थेन्य‚स्याविनाभावित्व‚स्म‚र‚ण‚म‚स्त्य‚तिप्र‚स‚ङ्गादिति [।]
	{\color{gray}{\rmlatinfont\textsuperscript{§~\theparCount}}}
	\pend% ending standard par
      ‚{\tiny $_{lb}$}‚

	  
	  \pstart \leavevmode% starting standard par
	त‚द‚युक्तं । प्र‚थ‚मादेर‚र्थ‚क्ष‚ण‚स्य प्र‚थ‚मादिज्ञान‚क्ष‚णेन ग्र‚ह‚णादेक‚स‚न्त‚तिप‚तितानां‚{\tiny $_{lb}$}‚ कार्य‚कार‚ण‚भावेन स्म‚र‚ण‚स‚म्भ‚वाच्च । य‚था च क्ष‚णिक‚प‚क्षे कार्य‚कार‚ण‚भाव‚{\tiny $_{lb}$}‚स्त‚थोक्त\edtext{}{\edlabel{pvsvt_98-1}\label{pvsvt_98-1}\lemma{थोक्त}\Bfootnote{\href{http://sarit.indology.info/?cref=pv.1.33}{ Pramāṅavārtika 1: 33. }}}मेव व‚क्ष्य‚तिच ।
	{\color{gray}{\rmlatinfont\textsuperscript{§~\theparCount}}}
	\pend% ending standard par
      ‚{\tiny $_{lb}$}‚

	  
	  \pstart \leavevmode% starting standard par
	न‚न्वेव‚म‚पि क्ष‚णानाम‚निश्च‚येन क‚थं कार्य‚कार‚ण‚भाव‚निश्च‚यो न च स‚न्तानेन‚{\tiny $_{lb}$}‚ त‚{\tiny $_{४}$}‚न्निश्च‚य‚स्त‚स्य स‚न्तानिभ्यो भिन्न‚स्याभावात् केव‚लं स‚न्तानिन एव पूर्वाप‚र‚काल‚{\tiny $_{lb}$}‚भाविनः [।] त‚त्र च य‚दैकः क्ष‚णो न त‚दान्य इति एक‚क्ष‚णाव‚भास एवेति क‚थं‚{\tiny $_{lb}$}‚ स‚न्तानाव‚भासः [।] त‚द‚भावात् क‚थं कार्य‚कार‚ण‚भाव इति [।]
	{\color{gray}{\rmlatinfont\textsuperscript{§~\theparCount}}}
	\pend% ending standard par
      ‚{\tiny $_{lb}$}‚

	  
	  \pstart \leavevmode% starting standard par
	त‚द‚युक्त‚म् [।] एक‚प‚र‚माण्वात्म‚क‚स्य व‚स्तुनो भावात् स्थूलात्म‚नां स‚न्तानिनां‚{\tiny $_{lb}$}‚ नैर‚न्त‚र्य‚प्र‚तिभास एव स‚न्तान‚प्र‚तिभास‚स्त‚त्र च क्ष‚ण‚विवेकान‚व‚धार‚णेन सादृ‚{\tiny $_{५}$}‚श्येन‚{\tiny $_{lb}$}‚ चैक‚त्वाध्य‚व‚सायादेक‚स‚न्त‚तिव‚र्तिनां क्ष‚णानां न कार्य‚कार‚ण‚भाव‚निश्च‚यः । भिन्न‚{\tiny $_{lb}$}‚स‚न्तान‚व‚र्तिनां तु स‚न्तान‚प्र‚वृत्त्या विजातीय‚त्वाद् भ‚व‚ति त‚न्निश्च‚य‚स्तेनाग्निस‚न्तान‚{\tiny $_{lb}$}‚पूर्व‚क‚स्य धूम‚स‚न्तान‚स्य प्र‚तीतेर‚ग्निधूम‚स‚न्तान‚योः कार्य‚कार‚ण‚भाव‚निश्च‚य उच्य‚त‚{\tiny $_{lb}$}‚ इति य‚त्किञ्चिदेत‚त् ।
	{\color{gray}{\rmlatinfont\textsuperscript{§~\theparCount}}}
	\pend% ending standard par
      ‚{\tiny $_{lb}$}‚

	  
	  \pstart \leavevmode% starting standard par
	\hphantom{.}अ ध्य य न स्त्वाह । स्व‚ल‚क्ष‚ण‚योः कार्य‚कार‚ण‚भाव‚ग्र‚ह‚णे स‚ति क‚थं सा‚{\tiny $_{६}$}‚मा‚{\tiny $_{lb}$}‚न्य‚योर्ग‚म्य‚ग‚म‚क‚भावो भिन्न‚त्वादिति ।
	{\color{gray}{\rmlatinfont\textsuperscript{§~\theparCount}}}
	\pend% ending standard par
      ‚{\tiny $_{lb}$}‚

	  
	  \pstart \leavevmode% starting standard par
	त‚द‚प्य‚युक्त‚म् [।] अनेक‚स्व‚ल‚क्ष‚णात्म‚क‚स्य सामान्य‚स्याभ्युप‚ग‚मात् । त‚दुक्त‚म् [।]
	{\color{gray}{\rmlatinfont\textsuperscript{§~\theparCount}}}
	\pend% ending standard par
      ‚{\tiny $_{lb}$}‚‚{\tiny $_{lb}$}‚‚{\tiny $_{lb}$}‚\textsuperscript{\textenglish{99/s}}
	  \bigskip
	  \begingroup
	
	    
	    \stanza[\smallbreak]
	  {\normalfontlatin\large ``\qquad}अत‚द्रूप‚प‚रावृत्त‚व‚स्तुमात्र‚प्र‚साध‚नात् ।&‚{\tiny $_{lb}$}‚सामान्य‚विष‚यं प्रोक्तं लिङ्गं भेदाप्र‚तिष्ठितेरिति ॥ \href{http://sarit.indology.info/?cref=pva.144}{प्र० स०}{\normalfontlatin\large\qquad{}"}\&[\smallbreak]
	  
	  
	  
	  \endgroup
	‚{\tiny $_{lb}$}‚

	  
	  \pstart \leavevmode% starting standard par
	तेन स्व‚ल‚क्ष‚णानां स‚म्ब‚न्ध‚ग्र‚ह एव सामान्यानां स‚म्ब‚न्ध‚ग्र‚हो न त्व‚न्यः ।
	{\color{gray}{\rmlatinfont\textsuperscript{§~\theparCount}}}
	\pend% ending standard par
      ‚{\tiny $_{lb}$}‚

	  
	  \pstart \leavevmode% starting standard par
	न‚न्व‚नुमान‚वादिना बौ द्धे न याव‚न्ति स्व‚ल‚क्ष‚णानि त्रैलोक्ये । तानि स‚र्वा‚{\tiny $_{७}$}‚ण्य‚ग्नि- \leavevmode\ledsidenote{\textenglish{39a/PSVTa}}‚{\tiny $_{lb}$}‚ व्याप्तानि गृहीत‚व्यान्येक‚स्याप्य‚ग्र‚हीते नैवानैकान्तिको हेतुः स्यात् [।] न चैवं प्र‚त्य‚क्षं‚{\tiny $_{lb}$}‚ क‚र्त्तुं श‚क्नोति स‚न्निहित‚विष‚य‚त्वात् । न वान्येषां स्व‚ल‚क्ष‚णानाम‚नुमान‚तः साध्य‚ध‚र्मेण‚{\tiny $_{lb}$}‚ व्याप्तिग्र‚ह‚ण‚म‚न‚व‚स्थाप्र‚स‚ङ्गादिति ।
	{\color{gray}{\rmlatinfont\textsuperscript{§~\theparCount}}}
	\pend% ending standard par
      ‚{\tiny $_{lb}$}‚

	  
	  \pstart \leavevmode% starting standard par
	त‚द‚युक्तं य‚तः [।] प्र‚त्य‚क्ष‚म‚ग्निभेद‚स‚न्निधान एव धूम‚भेदात् प्र‚तिप‚त् ।‚{\tiny $_{lb}$}‚ एष्व‚धूम‚व्यावृत्तं रूपं धूम‚म‚न‚ग्निव्यावृत्ताग्निमात्र‚कार्य‚मेवेत्य‚व‚धार‚य‚ति [।]‚{\tiny $_{lb}$}‚ य‚थात्र त‚थान्य‚त्रापि देशा‚{\tiny $_{१}$}‚ दावेत‚द् रूप‚म‚ग्निज‚न्य‚मेवेति चाव‚धार‚य‚त्य‚न्य‚थात्राग्नि‚{\tiny $_{lb}$}‚स‚म्ब‚न्धित‚या न प्र‚त्य‚क्षेण गृह्य‚ते । एव‚म‚ग्न्य‚न‚ग्निकार्य‚त्वेस्योभ‚य‚स‚म्ब‚न्धित‚या‚{\tiny $_{lb}$}‚ प्र‚तीतिः स्यान्नाग्निस‚म्ब‚त्धित‚यैव । प्र‚तीय‚ते च[।]त‚स्माद‚न्य‚त्राप्येत‚द्रूप‚म‚ग्नेरेव भ‚व‚{\tiny $_{lb}$}‚तीति निश्च‚यात् कुतो धूम‚स्याग्निव्य‚भिचारः । य‚श्च त‚द्रूप‚म्वाष्पादिविल‚क्ष‚ण‚{\tiny $_{lb}$}‚म‚व‚धार‚यितुं श‚क्नोति त‚स्यैवैत‚द‚नुमानं नान्य‚स्य । सामान्य‚व्याप्ति‚{\tiny $_{२}$}‚ग्र‚ह‚ण‚वादि‚{\tiny $_{lb}$}‚नापि गोपाल‚घ‚टिकादाव‚ग्निम‚न्त‚रेण धूम‚सामान्य‚द‚र्श‚नाव्य‚भिचार‚शंक‚याग्नि‚{\tiny $_{lb}$}‚निय‚त‚धूम‚सामान्याव‚धार‚णेनैव त‚द‚नुमान‚म् [।] अग्निनिय‚त‚धूम‚सामान्याव‚धार‚ण‚{\tiny $_{lb}$}‚ञ्चाग्निसंम्ब‚द्ध‚धूमाव्य‚क्त्य‚व‚धार‚ण‚पुर‚स्स‚र‚मेव । न च स‚र्व‚त्र देशादावाग्निस‚म्ब‚{\tiny $_{lb}$}‚द्ध‚धूम‚व्य‚क्तिविशिष्ट‚स्य धूम‚सामान्य‚स्य ग्र‚ह‚ण‚ङ्केन‚चित् प्र‚माणेन स‚म्भ‚व‚ति ।‚{\tiny $_{lb}$}‚ नापि म‚हान‚सादाव‚{\tiny $_{३}$}‚ग्निस‚म्ब‚द्ध‚धूम‚व्य‚क्तिविशिष्टं धूम‚सामान्यं प्र‚तिप‚न्न‚म‚न्य‚त्रा‚{\tiny $_{lb}$}‚नुयायिव्य‚क्तेर‚न‚न्व‚यात् । य‚च्च धूम‚सामान्य‚म‚नुयायि त‚न्नाग्न्य‚व्य‚भिचारि ।‚{\tiny $_{lb}$}‚ त‚स्मात् सामान्य‚व्याप्तिग्र‚ह‚ण‚वादिनोपि क‚थं विशिष्टं धूम‚सामान्यं स‚र्व‚त्राग्निना‚{\tiny $_{lb}$}‚ व्याप्तं प्र‚तिप‚न्न‚मिति तुल्यं चोद्यं ।
	{\color{gray}{\rmlatinfont\textsuperscript{§~\theparCount}}}
	\pend% ending standard par
      ‚{\tiny $_{lb}$}‚

	  
	  \pstart \leavevmode% starting standard par
	अथ धूम‚स्यान्य‚त्राग्निज‚न्य‚त्वे न किंचिद् बाध‚क‚म‚स्ति त‚देवेद‚मिति च प्र‚तीतेस्त‚{\tiny $_{lb}$}‚त्सामान्य‚म्प्र‚{\tiny $_{४}$}‚तिप‚न्न‚मिष्य‚तेऽस्माक‚म‚पि त‚देवेद‚मिति प्र‚त्य‚य‚स्योत्प‚त्तेस्त‚त्प्र‚तिप‚न्न‚मिष्य‚त‚{\tiny $_{lb}$}‚ इत्याव‚योः को भेद इति [।]
	{\color{gray}{\rmlatinfont\textsuperscript{§~\theparCount}}}
	\pend% ending standard par
      ‚{\tiny $_{lb}$}‚

	  
	  \pstart \leavevmode% starting standard par
	य‚त्किञ्चिदेत‚त् । एत‚मेवार्थ‚न्द‚र्श‚य‚न्नाह । \textbf{स} धूमो भ‚वँस्त\textbf{द‚भावे}ऽग्न्य‚भावे‚{\tiny $_{lb}$}‚ \textbf{हेतुम‚त्ताम्बिलंघ‚येद}हेतुकः स्यात् । य‚द्वा स धूमोग्निस‚म्ब‚न्धित‚या प्र‚तीत‚स्त\textbf{द‚भावे}‚{\tiny $_{lb}$}‚ग्न्य‚भावे \textbf{भ‚व‚न् हेतुम‚त्ता}म‚ग्निस‚म्ब‚न्धित‚या न प्र‚त्य‚क्षेण प्र‚तीयेत । प्र‚तीय‚ते च ।‚{\tiny $_{५}$}‚‚{\tiny $_{lb}$}‚ त‚स्मात् \textbf{स‚कृद‚पि} न केव‚लं भूय‚स्त\textbf{थाद‚र्श‚नादि}त्य‚न‚न्त‚रोक्तात् प्र‚त्य‚क्षानुप‚ल‚म्भात ॥‚{\tiny $_{lb}$}‚ ‚{\tiny $_{lb}$}‚ \leavevmode\ledsidenote{\textenglish{100/s}}किङ्कार‚ण‚म् [।] \textbf{अकार‚णा}द‚ग्नेः \textbf{स‚कृद‚पि} न केव‚लं भूयोऽ\textbf{भावात्} । न हि बालुकाभ्यः‚{\tiny $_{lb}$}‚ स‚कृद‚पि तैल‚म्भ‚व‚ति । \textbf{कार्य‚स्ये}त्यादिना कारिकार्थ‚माह । \textbf{न हीत्या}द्य‚स्यैव स‚म‚र्थ‚नं ।
	{\color{gray}{\rmlatinfont\textsuperscript{§~\theparCount}}}
	\pend% ending standard par
      ‚{\tiny $_{lb}$}‚

	  
	  \pstart \leavevmode% starting standard par
	एतेन व्याप्तिः क‚थिता भ‚व‚ति च धूमोग्निम‚न्त‚रेण व्य‚भिचार‚वादिनः । अनेन‚{\tiny $_{lb}$}‚ च प‚क्ष‚ध‚र्मः क‚थितः । \textbf{त‚{\tiny $_{६}$}‚दि}ति त‚स्मा\textbf{द‚ग्निम‚न्त‚रेण भावान्न त‚द्धेतु}र्नाग्निहेतुस्त‚था‚{\tiny $_{lb}$}‚ चाहेतुः \textbf{स्या}दिति भावः ॥
	{\color{gray}{\rmlatinfont\textsuperscript{§~\theparCount}}}
	\pend% ending standard par
      ‚{\tiny $_{lb}$}‚

	  
	  \pstart \leavevmode% starting standard par
	\textbf{अन्य‚हेतुक‚त्वादि} व‚ह्नेर्य‚द‚न्य‚च्छ‚क्र‚मूर्द्धादि । त‚द्धेतुक‚त्वाद् धूम‚स्य \textbf{नाहेतुक‚त्वं‚{\tiny $_{lb}$}‚ इति चेत्} ।
	{\color{gray}{\rmlatinfont\textsuperscript{§~\theparCount}}}
	\pend% ending standard par
      ‚{\tiny $_{lb}$}‚

	  
	  \pstart \leavevmode% starting standard par
	\textbf{नैत‚देव‚न्त‚त्राप्य}र्थान्त‚रे हेतौ क‚ल्प्य‚माने \textbf{तुल्य‚त्वात्} । त‚था हि [।] \textbf{त‚द‚भावे}प्य‚{\tiny $_{lb}$}‚न्य‚कार‚णाभावेपि पुन‚र\textbf{ग्नौ भ‚व‚ती}ति \textbf{त‚द‚प्य‚न्य‚त्}कार‚णं न हेतुः \textbf{स्या}त् । अपि च योसौ‚{\tiny $_{lb}$}‚ \leavevmode\ledsidenote{\textenglish{39b/PSVTa}} व‚ह्निर्य‚च्च त‚तो‚{\tiny $_{७}$}‚न्य‚त्कार‚ण‚न्त‚त्किं धूम‚ज‚न‚न‚स्व‚भाव‚माहोस्विन्न । य‚द्य‚ज‚न‚न‚{\tiny $_{lb}$}‚स्व‚भाव‚न्त‚दा क‚थ‚न्त‚तोग्नेर‚न्य‚तो वा \textbf{श‚क्र‚मूर्द्धा}देर\textbf{त‚ज्ज‚न‚न‚स्व‚भाव‚त्वात्} अधूम‚ज‚न‚{\tiny $_{lb}$}‚न‚स्व‚भावाद्धूमो भ‚व‚न्नैव \textbf{भ‚वेत्} । किं कार‚ण‚म्[।]\textbf{अत‚त्स्व‚भाव‚स्य} स्व‚य‚म‚धूम‚ज‚न‚न‚{\tiny $_{lb}$}‚स्व‚भाव‚स्या\textbf{ज‚न‚नात्त‚स्य} धूम‚स्याहेतु\textbf{ता स्यात्} ।
	{\color{gray}{\rmlatinfont\textsuperscript{§~\theparCount}}}
	\pend% ending standard par
      ‚{\tiny $_{lb}$}‚

	  
	  \pstart \leavevmode% starting standard par
	अथ धूम‚ज‚न‚न‚स्व‚भावोन्य‚स्त‚दा द्व‚योर‚पि व‚ह्नित्वं धूम‚ज‚न‚न‚स्व‚भाव‚ल‚क्ष‚ण‚{\tiny $_{lb}$}‚त्वाद् व‚ह्नेः । एत‚{\tiny $_{१}$}‚च्चोत्त‚र‚त्राभिधास्य‚ते ॥
	{\color{gray}{\rmlatinfont\textsuperscript{§~\theparCount}}}
	\pend% ending standard par
      ‚{\tiny $_{lb}$}‚

	  
	  \pstart \leavevmode% starting standard par
	\textbf{न वै स एवे}त्यादि व्य‚भिचार‚वादी । अथ‚वाग्निज‚नितो धूमः स एवान्य‚तो‚{\tiny $_{lb}$}‚ भ‚व‚तीत्येवं नोच्य‚ते । य‚दि स एवान्य‚तः स्याद् भ‚वेद‚हेतुत्व\textbf{न्तादृश‚स्य} व‚ह्निज‚नित‚{\tiny $_{lb}$}‚ध‚र्म‚स्व‚भाव‚तुल्य‚स्यान्य‚तो \textbf{भावात्} ।
	{\color{gray}{\rmlatinfont\textsuperscript{§~\theparCount}}}
	\pend% ending standard par
      ‚{\tiny $_{lb}$}‚

	  
	  \pstart \leavevmode% starting standard par
	\textbf{अन्या}दृशादित्यादि सि द्धा न्त वा दी । त‚त्किं कार्य‚स‚दृशं कार‚ण‚मिष्य‚ते‚{\tiny $_{lb}$}‚ येनैव‚मुच्य‚तेन्यादृशाद् \textbf{भ‚व‚न् क‚थ‚न्तादृश इति} । नान्यार्थ‚त्वात् । यो हि धूम‚{\tiny $_{२}$}‚‚{\tiny $_{lb}$}‚ज‚न‚को व‚ह्निर्दृष्ट‚स्त‚तो विस‚दृशाद् भ‚व‚न् धूमः क‚थ‚न्तादृशो भ‚व‚ति व‚ह्निज‚नित‚{\tiny $_{lb}$}‚ध‚र्म‚तुल्य‚स्व‚भावो भ‚व‚ति ।
	{\color{gray}{\rmlatinfont\textsuperscript{§~\theparCount}}}
	\pend% ending standard par
      ‚{\tiny $_{lb}$}‚

	  
	  \pstart \leavevmode% starting standard par
	एत‚दुक्त‚म्भ‚व‚ति । य‚था धूम‚भेदानान्तार्ण्ण‚प‚र्ण्णादीनां प‚र‚स्प‚रापेक्ष‚या तादृश‚त्व‚{\tiny $_{lb}$}‚न्त‚थाग्निभेदानाम‚पि तार्ण्ण‚पार्ण्णादीनां धूम‚ज‚न‚कानान्ताद‚श‚त्वं प‚र‚स्प‚रापेक्ष‚यैव ।‚{\tiny $_{lb}$}‚ ‚{\tiny $_{lb}$}‚ \leavevmode\ledsidenote{\textenglish{101/s}}तेन यादृशो धूम‚भेद एक‚स्माद‚ग्निभेदादुत्प‚द्य‚मानो दृष्ट‚स्तादृश‚स्य धूम‚भेद‚स्य ता‚{\tiny $_{३}$}‚दृ‚{\tiny $_{lb}$}‚शादेवाग्निभेदादुत्प‚त्तिः । य‚स्त्व‚न‚ग्नेरुत्प‚न्नः सोन्यादृश एव । वाष्पादिव‚त् ।
	{\color{gray}{\rmlatinfont\textsuperscript{§~\theparCount}}}
	\pend% ending standard par
      ‚{\tiny $_{lb}$}‚

	  
	  \pstart \leavevmode% starting standard par
	न‚न्व‚ग्निज‚न्येन धूम‚क्ष‚णेन तादृशो धूम‚क्ष‚ण‚ज‚न्यो धूम‚क्ष‚ण‚स्तेनान्यादृशाद‚पि‚{\tiny $_{lb}$}‚ तादृशो भ‚व‚तीति चेत् [।] न । अग्निज‚न्य‚स्य हि धुम‚क्ष‚ण‚स्याग्निज‚न्य एवान्यो‚{\tiny $_{lb}$}‚ धूम‚क्ष‚ण‚स्तादृशो भ‚व‚ति नान्यः । न हि वाष्पादीनान्तादृश‚त्वाध्य‚व‚सायेपि तादृश‚{\tiny $_{lb}$}‚त्व‚म्भ‚व‚ति । धूम‚क्ष‚ण‚ज‚न्य‚स्यापि धू‚{\tiny $_{४}$}‚म‚क्ष‚ण‚स्यान्यो धूम‚क्ष‚ण‚ज‚न्य एव धूम‚क्ष‚ण‚स्तादृशो‚{\tiny $_{lb}$}‚ भ‚व‚ति नान्यः । त‚स्मात्तादृशादेव तादृशोत्प‚त्तिरिति कुतो व्य‚भिचारः । तेन ।
	{\color{gray}{\rmlatinfont\textsuperscript{§~\theparCount}}}
	\pend% ending standard par
      ‚{\tiny $_{lb}$}‚
	  \bigskip
	  \begingroup
	
	    
	    \stanza[\smallbreak]
	  {\normalfontlatin\large ``\qquad}क्ष‚णिक‚त्वे क‚थ‚म्भावाः क्व‚चिदाय‚त्त‚वृत्त‚यः ।&‚{\tiny $_{lb}$}‚प्र‚सिद्ध‚कार‚णाभावे येषाम्भाव‚स्त‚तोन्य‚तः ॥&‚{\tiny $_{lb}$}‚त‚त‚श्चान‚ग्नितो धूमाद् य‚था धूम‚स्य स‚म्भ‚वः ।&‚{\tiny $_{lb}$}‚श‚क्र‚मूर्ध्न‚स्त‚था त‚स्य केन वार्येत स‚म्भ‚व इति ॥{\normalfontlatin\large\qquad{}"}\&[\smallbreak]
	  
	  
	  
	  \endgroup
	‚{\tiny $_{lb}$}‚

	  
	  \pstart \leavevmode% starting standard par
	निर‚स्तं । प्र‚थ‚म‚स्य ह्य‚ग्निज‚न्य‚स्य धू‚{\tiny $_{५}$}‚म‚क्ष‚ण‚स्याप‚रोग्निज‚न्य एव धूम‚क्ष‚ण‚स्तादृशो‚{\tiny $_{lb}$}‚ धूम‚क्ष‚ण‚ज‚न्य‚स्य धूम‚क्ष‚ण‚स्य द्वितीय‚स्यान्यो धूम‚क्ष‚ण‚ज‚न्य एव द्वितीयो धूम‚क्ष‚ण‚स्तादृशो‚{\tiny $_{lb}$}‚ भ‚व‚ति [।] त‚था तृतीयादिक्ष‚णेष्व‚पीति क्ष‚णापेक्ष‚याप्य‚व्य‚भिचार‚स्तादृश‚स्य ।‚{\tiny $_{lb}$}‚ एत‚मेवाह । \textbf{तादृशाद्धि भ‚वँस्तादृशः स्यात् । अन्यादृशाद‚प्य}व‚ह्निस‚दृशाद‚पि \textbf{य‚दि‚{\tiny $_{lb}$}‚ तादृशो भ‚वेद्} व‚ह्निज‚नित‚धूम‚तुल्य‚स्व‚भावो भ‚वेत् ।‚{\tiny $_{६}$}‚ \textbf{त‚च्छ‚क्तिनिय‚माभावात्} ।‚{\tiny $_{lb}$}‚ स‚दृशास‚दृश‚योः कार‚ण‚योर्या श‚क्तिस्त‚स्या यो निय‚मः स‚दृशी स‚दृश‚मेव ज‚न‚य‚त्य‚स‚{\tiny $_{lb}$}‚दृशी विल‚क्ष‚ण‚मिति त‚स्याभावात् कार‚णान्न \textbf{हेतुभेदः} कार्य‚स्य \textbf{भेद‚क इति} कृत्वाऽ‚{\tiny $_{lb}$}‚\textbf{कार‚ण‚म्विश्व‚स्य वैश्व‚रूप्यं} स्यात् । य‚तः कुत‚श्चित्कार‚णादुत्प‚त्तेर्नाहेतुक‚त्व‚मिति‚{\tiny $_{lb}$}‚ चेदाह । \textbf{स‚र्वं वा स‚र्व‚स्मा}दित्यादि । अश‚क्ताद‚पि चेदुत्प‚त्तिः \textbf{स‚र्व स‚र्व‚स्मा‚{\tiny $_{७}$}‚ज्जायेत} [।] \leavevmode\ledsidenote{\textenglish{40a/PSVTa}}‚{\tiny $_{lb}$}‚ न चैवं । \textbf{त‚स्मा}दित्यादि । त‚न्न धूम इति । त‚दिति त‚स्माद् । य‚त उत्प‚द्य‚मानो \textbf{धूमो‚{\tiny $_{lb}$}‚ दृष्टः} स \textbf{दृष्टाकारो} व‚ह्निः । त‚द्वि\textbf{जातीया}द‚व‚ह्नेरित्य‚र्थः ॥
	{\color{gray}{\rmlatinfont\textsuperscript{§~\theparCount}}}
	\pend% ending standard par
      ‚{\tiny $_{lb}$}‚\textsuperscript{\textenglish{102/s}}

	  
	  \pstart \leavevmode% starting standard par
	\textbf{त‚था चे}त्य‚हेतुक‚त्वे स‚ति \textbf{नित्यं स‚त्त्व‚म‚स‚त्त्व‚म्वा} धूम‚स्य स्यात् । किङ्कार‚ण‚म् [।]‚{\tiny $_{lb}$}‚ \textbf{अहेतो}र्भाव‚स्य स्व‚निष्प‚त्ताव\textbf{न्यान‚पेक्ष‚णा}त् । कार‚णान्त‚रान‚पेक्ष‚णात् । कार‚णा‚{\tiny $_{lb}$}‚न्त‚रान‚पेक्ष‚त्वेपि कादाचित्कं स्व‚भाव‚तो भ‚विष्य‚तीति चेदाह । \textbf{अपेक्षातो‚{\tiny $_{१}$}‚} हीति ।
	{\color{gray}{\rmlatinfont\textsuperscript{§~\theparCount}}}
	\pend% ending standard par
      ‚{\tiny $_{lb}$}‚

	  
	  \pstart \leavevmode% starting standard par
	एत‚दुक्त‚म्भ‚व‚ति [।] अनिष्प‚न्न‚स्यास‚त्त्वादेव क‚थं स्व‚भाव‚तः \textbf{कादाचित्क‚त्वं} ।‚{\tiny $_{lb}$}‚ निष्प‚न्न‚स्य त्व‚स्ति स्व‚भावः केव‚लं सैव निष्प‚त्तिः क‚थं क्व‚चिद् भ‚व‚तीति‚{\tiny $_{lb}$}‚ चोद्य‚ते । \textbf{स ही}त्यादिमैत‚देव व्याच‚ष्टे । \textbf{न क‚दाचिन्न भ‚वे}त् स‚र्व‚काल‚म्भ‚वेत् ।‚{\tiny $_{lb}$}‚ किङ्कार‚ण\textbf{न्त‚द्भा}वे धूम‚स्व‚भाव‚स्य भावे । कार‚णान‚पेक्ष‚त्वेन \textbf{वैक‚ल्याभावात् ।‚{\tiny $_{lb}$}‚ इष्ट‚काल}व‚त् । \textbf{त‚दापि चे}ति दृष्ट‚कालेपि धूमो न भ‚वेत् । धूमा\textbf{भा‚{\tiny $_{२}$}‚व‚काला‚{\tiny $_{lb}$}‚विशेषात्} ।
	{\color{gray}{\rmlatinfont\textsuperscript{§~\theparCount}}}
	\pend% ending standard par
      ‚{\tiny $_{lb}$}‚

	  
	  \pstart \leavevmode% starting standard par
	प‚श्चाद‚र्द्ध‚म्विभ‚ज‚न्नाह । \textbf{अपेक्ष‚या ही}ति । योग्य‚देश‚कालापेक्ष‚या । य‚स्मात्‚{\tiny $_{lb}$}‚ कार्य‚स्य यौ \textbf{भावाभाव‚कालौ त‚यो}र्य‚थाक्र‚म\textbf{न्त‚द्भा}व‚स्य कार्योत्पाद‚स्य ये \textbf{योग्य‚ता‚{\tiny $_{lb}$}‚योग्य‚ते} ताभ्यां \textbf{योगात्} । काल‚ग्र‚ह‚ण‚मुप‚ल‚क्ष‚ण‚प‚र‚मेवं देश‚द्र‚व्य‚योर‚पि वाच्यं । अथैवं‚{\tiny $_{lb}$}‚ नेष्य‚ते । त‚दा तुल्ये योग्य‚तायोग्य‚ते य‚योः कार्य‚भावाभाव‚व‚तो\textbf{र्देश‚काल‚यो}स्त‚यो‚{\tiny $_{३}$}‚‚{\tiny $_{lb}$}‚\textbf{स्त‚द्व‚त्तेत‚र‚योर्निय‚मायोगात्} । कार्य‚काल‚स्यैव त‚द्व‚त्ता । कार्य‚व‚त्ता । त‚द‚न्य‚स्येत‚रा ।‚{\tiny $_{lb}$}‚ अकार्य‚व‚त्तेत्य‚स्य निय‚म‚स्यायोगात् । द्वाव‚पि तौ कार्य‚भावाभाव‚कालौ कार्य‚व‚न्तौ‚{\tiny $_{lb}$}‚ स्यातां योग्य‚तासादृश्यात् । न वा । तुल्य‚त्वाद‚योग्य‚तायाः । त‚स्मात् त‚द्भाव‚{\tiny $_{lb}$}‚काल‚स्यैव योग्य‚ता । तां वापे\textbf{क्ष‚माण\textbf{ा}भावाः कादाचित्का भ‚व‚न्ति} ॥
	{\color{gray}{\rmlatinfont\textsuperscript{§~\theparCount}}}
	\pend% ending standard par
      ‚{\tiny $_{lb}$}‚

	  
	  \pstart \leavevmode% starting standard par
	भ‚व‚तु नामेष्ट‚स्य देश‚कालादेर्योग्य‚ता । न ताव‚{\tiny $_{४}$}‚ता हेतुभाव इत्य‚त आह ।‚{\tiny $_{lb}$}‚ \textbf{सा चे}त्यादि । य‚त एव‚म‚हेतुत्वे नित्यं स‚त्त्वास‚त्त्वं स्यान्न च भ‚व‚ति ।
	{\color{gray}{\rmlatinfont\textsuperscript{§~\theparCount}}}
	\pend% ending standard par
      ‚{\tiny $_{lb}$}‚

	  
	  \pstart \leavevmode% starting standard par
	\textbf{त‚स्मादि}त्यादि । य‚त्\textbf{प‚रिहारेण} प्र‚व‚र्त्त‚ते त‚द‚न‚पेक्षः । य‚त्र च व‚र्त्त‚ते त‚त्सापेक्षः ।‚{\tiny $_{lb}$}‚ य‚दि नाम क्व‚चिद् देशादौ वृत्त‚स्त‚थापि क‚थ\textbf{न्त‚त्सापेक्ष इति} चेदाह । \textbf{त‚था} हीति ।‚{\tiny $_{lb}$}‚ \textbf{त‚था वृत्ति}रित्येक‚प‚रिहारेणान्य‚त्र वृत्तिः ॥ देशादिक‚म\textbf{पेक्ष‚त एव} भावः[।]तेन तु‚{\tiny $_{lb}$}‚ देशादिना‚{\tiny $_{५}$}‚ न त‚स्योप‚कारः क्रिय‚त इति चेदाह । \textbf{त‚त्कृतोप‚कारे}त्यादि । त‚न्निय‚{\tiny $_{lb}$}‚‚{\tiny $_{lb}$}‚ \leavevmode\ledsidenote{\textenglish{103/s}}\textbf{य‚मायोगादिति} । त‚स्मिन्नैव देशादौ तेन न भाव्य‚मिति निय‚मायोगात् [।] \textbf{त‚दि}ति‚{\tiny $_{lb}$}‚ त‚स्माद् । देश‚काल‚ग्र‚ह‚ण‚मुप‚ल‚क्ष‚णं द्र‚व्य‚स्यापि प‚रिग्र‚हः । \textbf{य‚त्रे}ति देशादौ \textbf{दृष्टः स‚कृ}दि‚{\tiny $_{lb}$}‚ति । य‚थोक्तेन प्र‚त्य‚क्षेण येषां \textbf{स‚न्निधा}ने दृष्ट‚स्तेषामेवान्य‚त\textbf{र‚वैक‚ल्ये च} पुन‚र्न \textbf{दृष्टः ।‚{\tiny $_{lb}$}‚ अन्य}थेति य‚दि त‚ज्ज‚न्योस्य स्व‚{\tiny $_{६}$}‚भावो न स्यात् । \textbf{स} इति धूम‚स्त‚त्प्र‚तिनिय‚तोग्न्या‚{\tiny $_{lb}$}‚दिक‚साम‚ग्री-हि-य‚तः । अग्नेर\textbf{न्य‚त्र क‚थ‚म्भ‚वेन्नै}व भ‚वेत् । \textbf{भ‚व‚न् वा न धूमः स्यात्} ।‚{\tiny $_{lb}$}‚ य‚स्मात् \textbf{त‚ज्ज‚नितो} ह्य‚ग्निज‚नितो हि \textbf{स्व‚भाव‚विशेषो धूम} इति । \textbf{त‚था हेतुर‚पि}‚{\tiny $_{lb}$}‚ व‚ह्निस्त\textbf{थाभूत‚कार्य‚ज‚न‚न‚स्व‚भावो} धूम‚ज‚न‚न‚योग्य‚तास्व‚भावो । धूम‚र‚हिताव‚स्थाया‚{\tiny $_{lb}$}‚म‚प्य‚स्त्येव योग्य‚ता कार‚ण‚भूतेति । तेन नाऽव्यापि ल‚क्ष‚णं ।‚{\tiny $_{७}$}‚ एव‚म‚ग्निधूम‚योः \leavevmode\ledsidenote{\textenglish{40b/PSVTa}}‚{\tiny $_{lb}$}‚ प‚र‚स्प‚रापेक्ष‚या निय‚त‚स्व‚भाव‚त्वे प्र‚त्य‚क्ष‚व्य‚व‚स्थापिते ।
	{\color{gray}{\rmlatinfont\textsuperscript{§~\theparCount}}}
	\pend% ending standard par
      ‚{\tiny $_{lb}$}‚

	  
	  \pstart \leavevmode% starting standard par
	\textbf{य‚दि त‚स्य} धूम‚स्याग्ने\textbf{र‚न्य‚तोपि भाव} इष्य‚ते \textbf{त‚दा न स} धूम‚ज‚न‚नः \textbf{स्व‚भाव‚स्त‚स्या}‚{\tiny $_{lb}$}‚व‚ह्नेः । त‚था ह्य‚न‚ग्नेर्य‚दा धूम‚स्योत्प‚त्तिस्त‚दान‚ग्नेरेव धूम‚ज‚न‚नः स्व‚भावो जातः ।‚{\tiny $_{lb}$}‚ य‚श्चान‚ग्नेः स्व‚भावः स क‚थ‚म‚ग्नेः स्यात् । त‚त‚श्चाधूम‚ज‚न‚न‚स्व‚भाव‚त्वाद‚ग्नेः‚{\tiny $_{lb}$}‚ \textbf{स‚कृद‚पि न धूमं ज‚न‚येत्} । धूम‚स्यापि‚{\tiny $_{१}$}‚ धूम‚स्व‚भाव‚ता न स्यादित्याह । \textbf{न‚{\tiny $_{lb}$}‚ चे}त्यादि । अग्नेर‚न्य‚तो भ‚व‚न्न वा स \textbf{धूमः} [।] किं कार‚ण‚म् [।] \textbf{अधूम‚ज‚न‚न‚स्व‚{\tiny $_{lb}$}‚भावा}द‚न‚ग्ने\textbf{र्भावा}दुत्प‚त्तेः । \textbf{त‚त्स्व‚भाव‚त्वे चा}न‚ग्नेर‚पि धूम‚ज‚न‚न‚स्व‚भाव‚त्वे चाभ्यु‚{\tiny $_{lb}$}‚प‚ग‚म्य‚माने । \textbf{स एवा}ग्निर्धूम‚ज‚न‚क‚रूप‚त्वाद‚स्य इत्य‚नेन द्वारेणा\textbf{व्य‚भिचा}रो‚{\tiny $_{lb}$}‚ धूम‚स्य ॥
	{\color{gray}{\rmlatinfont\textsuperscript{§~\theparCount}}}
	\pend% ending standard par
      ‚{\tiny $_{lb}$}‚

	  
	  \pstart \leavevmode% starting standard par
	सुख‚ग्र‚ह‚णार्थं \textbf{अग्निस्व‚भाव} इत्यादि श्लोक‚द्व‚य‚माह ।
	{\color{gray}{\rmlatinfont\textsuperscript{§~\theparCount}}}
	\pend% ending standard par
      ‚{\tiny $_{lb}$}‚\textsuperscript{\textenglish{104/s}}

	  
	  \pstart \leavevmode% starting standard par
	\textbf{अग्निस्व‚भाव} इति धूम‚ज‚न‚न‚स्व‚भा‚{\tiny $_{२}$}‚वो \textbf{य‚दी}त्य‚र्थः । \textbf{अग्निरेव स श‚क्र‚मूर्द्धा}‚{\tiny $_{lb}$}‚ धूम‚ज‚न‚न‚स्व‚भाव‚त्वात् । \textbf{अथान‚ग्निस्व‚भावोसौ} श‚क्र‚मूर्द्धा । \textbf{त‚त्रे}ति श‚क्र‚मूर्द्ध‚नि ।‚{\tiny $_{lb}$}‚ क‚स्मान्न भ‚वेदित्याह । \textbf{धूमे}त्यादि । हि य‚स्मात् । \textbf{धूम‚हेतुस्व‚भावो} य‚स्येति विग्र‚हः ।‚{\tiny $_{lb}$}‚ कुत एत‚त् त‚च्छ\textbf{क्तिभेद‚वा}न् । त‚या धूम‚ज‚निक‚या श‚क्त्या क‚र‚ण‚भूत‚या व‚स्त्व‚न्त‚रात्‚{\tiny $_{lb}$}‚ ख‚द्योतादेर्भेद‚वान् विस‚दृशः । \textbf{अधूम‚हेतो}रित्य‚व‚ह्निस्व‚भावाद् \textbf{धू‚{\tiny $_{३}$}‚म‚स्य भावे}‚{\tiny $_{lb}$}‚ उत्पादेऽभ्युप‚ग‚म्य‚माने \textbf{स} धूमः \textbf{स्याद‚हेतुकः} । य‚थोक्तं प्राक् ॥
	{\color{gray}{\rmlatinfont\textsuperscript{§~\theparCount}}}
	\pend% ending standard par
      ‚{\tiny $_{lb}$}‚

	  
	  \pstart \leavevmode% starting standard par
	\textbf{क‚थ‚मि}त्यादि प‚रः । \textbf{इदानी}मित्येक‚स्य धूमादेर्विजातीयादुत्प‚त्त्य‚न‚भ्युप‚ग‚मे ।‚{\tiny $_{lb}$}‚ \textbf{क‚थ‚म्भिन्नात्} प‚र‚स्प‚र‚विजातीयात् । \textbf{स‚ह‚कारिणः} स‚काशादेक‚स्य \textbf{कार्य‚स्योत्प‚त्तिः} ।‚{\tiny $_{lb}$}‚ क‚थ‚मित्याह । \textbf{य‚थेत्या}दि । आदिश‚ब्दाद् आलोक‚म‚न‚स्काराद‚यः । एव‚ञ्च स‚ति च‚क्षुः‚{\tiny $_{lb}$}‚स्व‚भावाद‚प्युत्प‚द्य‚ते \textbf{विज्ञा}न‚म‚{\tiny $_{४}$}‚च‚क्षुःस्व‚भावाद‚पि रूप‚म‚न‚स्कारादेर्न चेद‚म‚हेतुकं ।‚{\tiny $_{lb}$}‚ एवं धूमोप्य‚ग्नेरुत्प‚द्य‚ताम‚न‚ग्नेश्च श‚क्र‚मूर्ध्नः । न चाहेतुको भ‚विष्य‚तीति चोद‚को‚{\tiny $_{lb}$}‚ म‚न्य‚ते ।
	{\color{gray}{\rmlatinfont\textsuperscript{§~\theparCount}}}
	\pend% ending standard par
      ‚{\tiny $_{lb}$}‚

	  
	  \pstart \leavevmode% starting standard par
	\textbf{न वै किंचि}दित्यादिना प्र‚तिविध‚त्ते । च‚क्षुरादिषु \textbf{त‚त्स्व‚भावं} ज‚न‚क‚स्व‚भावं‚{\tiny $_{lb}$}‚ स‚दे\textbf{कैकं} प‚र‚स्प‚रान‚पेक्षं न वै \textbf{ज‚न‚कं} [।] य‚दि हि स्यात् त‚दा प्र‚त्येकं कार‚ण‚व्य‚भिचा‚{\tiny $_{lb}$}‚राद‚हेतुकं स्यात् । \textbf{किन्तु साम‚ग्री ज‚निका । त‚त्स्व‚भावा} ज‚न‚{\tiny $_{५}$}‚क‚स्व‚भावा । साम‚ग्री‚{\tiny $_{lb}$}‚ ज‚निकेत्येताव‚तैव त‚त्स्व‚भाव‚त्वं ल‚ब्ध‚म‚त‚त्स्व‚भाव‚स्याज‚न‚क‚त्वात् त‚त्किं स्व‚भावेति‚{\tiny $_{lb}$}‚ पृथ‚गुच्य‚ते ॥ स‚त्यं किन्त्व‚व‚धार‚णार्थ‚मुक्तं । साम‚ग्र्य‚व‚स्थायामेव त‚त्स्व‚भाव‚ता । न‚{\tiny $_{lb}$}‚ पूर्व‚न्न प‚श्चान्न पृथ‚गिति । स्व‚हेतुसाम‚र्थ्य‚निय‚त‚स‚न्निधीत्येक‚स्मिन् कार्ये स‚म‚स्ताव‚न्ये‚{\tiny $_{lb}$}‚ कार‚णानि हेतुरिति स‚मुदायार्थः ।
	{\color{gray}{\rmlatinfont\textsuperscript{§~\theparCount}}}
	\pend% ending standard par
      ‚{\tiny $_{lb}$}‚

	  
	  \pstart \leavevmode% starting standard par
	केचित्तु वा श‚ब्दं प‚ठ‚न्ति [।] साम‚ग्री ज‚निका त‚त्स्व‚भावा‚{\tiny $_{६}$}‚ वेति । अत्र तु‚{\tiny $_{lb}$}‚ वाश‚ब्द‚स्य न किंचित् प्र‚योज‚न‚मित्य प पाठ एवायं । य‚दि साम‚ग्रीत्युक्त्वा‚{\tiny $_{lb}$}‚ साम‚ग्र्य‚न्त‚राद‚पि च‚क्षुर्विज्ञानं स्यात्त‚दा भिन्नादुत्पादेर‚हेतुक‚त्व‚म्भ‚वेत् । सैव साम‚{\tiny $_{lb}$}‚ग्र्\textbf{य‚नुमीय‚ते} । कार्येण न प्र‚त्येकं कार‚णाद‚न्य‚तो नास्ति व्य‚भिचार इति भावः ।
	{\color{gray}{\rmlatinfont\textsuperscript{§~\theparCount}}}
	\pend% ending standard par
      ‚{\tiny $_{lb}$}‚

	  
	  \pstart \leavevmode% starting standard par
	स्यादेत‚द अग्न्यादिसाम‚ग्र‚या आद्य एव धूम‚क्ष‚णो ज‚नितो न च त‚स्य लिङ्ग‚{\tiny $_{lb}$}‚‚{\tiny $_{lb}$}‚ \leavevmode\ledsidenote{\textenglish{105/s}}त्व‚न्त‚स्यानिश्च‚यात् । य‚श्च धूम‚प्र‚तिब‚न्धो गृह्य‚{\tiny $_{७}$}‚ते स पूर्व‚धूम‚हेतुरेव [।] \leavevmode\ledsidenote{\textenglish{41a/PSVTa}}‚{\tiny $_{lb}$}‚ क‚थ‚न्त‚तोग्न्यादिसाम‚ग्र्या अनुमान‚मित्य‚त आह । \textbf{सैव चे}त्यादि \textbf{कार्य}प्र‚ब‚न्ध‚स्य‚{\tiny $_{lb}$}‚ \textbf{स्व‚भाव‚स्थितेः । आश्र‚य} आद्यं कार‚णं । साम‚ग्रीम‚न्त‚रेण धूम‚स‚न्तान‚स्यैवाभावात् ।‚{\tiny $_{lb}$}‚ त‚तो धूम‚स‚न्तान‚म‚ग्निकार्य‚त्वेनैकीकृत्य द‚ह‚नादिस‚न्त‚तेः कार‚ण‚भूताया अनुमानं ।‚{\tiny $_{lb}$}‚ न हि क्ष‚ण‚विभागेनार्वाग्द‚र्श‚न‚स्य व्य‚व‚हारः स‚म्भ‚व‚तीति ।
	{\color{gray}{\rmlatinfont\textsuperscript{§~\theparCount}}}
	\pend% ending standard par
      ‚{\tiny $_{lb}$}‚

	  
	  \pstart \leavevmode% starting standard par
	स्यादेत‚द् [।] अतीव एव व‚{\tiny $_{१}$}‚ह्निर‚नुमीय‚ते । न च तेनार्थ‚क्रियार्थिनः किंचित् प्र‚{\tiny $_{lb}$}‚योज‚न‚मित्येत‚द‚पि चोद्य‚मानेनैव प‚रिहृतं । त‚दा चाय‚म‚र्थः [।] सैव च साम‚ग्री‚{\tiny $_{lb}$}‚ प्र‚ब‚न्धेन प्र‚व‚र्त्त‚माना कार्य‚स्व‚भाव‚स्थितेः कार्य‚प्र‚ब‚न्ध‚वृत्तेर्हेतुः पूर्व‚पूर्व‚म‚ग्न्यादिक्ष‚णं‚{\tiny $_{lb}$}‚ प्र‚तीत्योत्त‚रोत्त‚र‚स्य धूम‚क्ष‚ण‚स्योत्प‚त्तेः । न ह्य‚यं निय‚मो य‚देक एव धूम‚क्ष‚णोग्निना‚{\tiny $_{lb}$}‚ ज‚न्यो नाप‚र इति । याव‚दिन्ध‚न‚स्य न स‚र्व‚था भ‚स्मीभ‚व‚न‚न्ता‚{\tiny $_{२}$}‚व‚द् धूम‚क्ष‚णानामु‚{\tiny $_{lb}$}‚त्प‚त्तिर‚विरुद्धा । त‚स्मादुत्त‚रोत्त‚र‚द‚ह‚न‚क्ष‚ण‚ज‚न‚न‚म‚र्थेन‚प्र‚ब‚न्ध‚प्र‚वृत्तेनाग्निना य‚था‚{\tiny $_{lb}$}‚भूतो धूम‚प्र‚ब‚न्धो ज‚नितः प्र‚त्य‚क्षाव‚धारित‚स्त‚थाभूत‚म‚न्य‚त्राव‚धार्यार्थ‚क्रियास‚म‚र्थ‚{\tiny $_{lb}$}‚द‚ह‚न‚स‚न्तान‚स्यानुमान‚न्त‚द‚र्थिनो न विरुद्ध‚म् [।] अन्य एव च स्व‚भावो विच्छिन्न‚{\tiny $_{lb}$}‚द‚ह‚न‚स‚न्तान‚स्य धूम‚स्यान्य एव विच्छिन्न‚द‚ह‚न‚स‚न्त‚तेर्वास‚गृहादिस्थ‚स्य । स्फुट‚श्च‚{\tiny $_{lb}$}‚ त‚यो‚{\tiny $_{३}$}‚र्भेद‚म‚व‚धार‚य‚ति लोक इति कुतो व्य‚भिचारः ॥
	{\color{gray}{\rmlatinfont\textsuperscript{§~\theparCount}}}
	\pend% ending standard par
      ‚{\tiny $_{lb}$}‚

	  
	  \pstart \leavevmode% starting standard par
	अथ‚वाऽन्य‚था व्याख्याय‚ते ।
	{\color{gray}{\rmlatinfont\textsuperscript{§~\theparCount}}}
	\pend% ending standard par
      ‚{\tiny $_{lb}$}‚

	  
	  \pstart \leavevmode% starting standard par
	न‚नु साम‚ग्री स‚म‚ग्रेभ्योन्या । त‚स्याश्चैक‚त्वाद् एक‚मेव कार्य‚मुत्प‚द्य‚तेऽन्य‚था‚{\tiny $_{lb}$}‚ स‚म‚ग्राणां प्र‚त्येकं साम‚र्थ्यात् कार्य‚ब‚हुत्व‚म्प‚र्यायेण चैक‚कार्य‚ज‚न‚क‚त्वं स्यान्न च भ‚व‚{\tiny $_{lb}$}‚ति । त‚स्मात् क‚थ‚मेक‚ज‚न‚क‚त्व‚मित्याशंक्याह ।
	{\color{gray}{\rmlatinfont\textsuperscript{§~\theparCount}}}
	\pend% ending standard par
      ‚{\tiny $_{lb}$}‚

	  
	  \pstart \leavevmode% starting standard par
	\textbf{सैव चे}त्यादि । सैव च \textbf{साम‚ग्री}ति स‚म‚ग्रा एव साम‚ग्री श‚ब्देनोच्य‚न्ते ।‚{\tiny $_{४}$}‚ किं‚{\tiny $_{lb}$}‚ कार‚णं । \textbf{स्व‚भाव‚स्थित्या} स्व‚रूप‚स‚न्निधानेन \textbf{कार्य‚स्याश्र‚यो} भ‚व‚ति य‚तः ।
	{\color{gray}{\rmlatinfont\textsuperscript{§~\theparCount}}}
	\pend% ending standard par
      ‚{\tiny $_{lb}$}‚

	  
	  \pstart \leavevmode% starting standard par
	एव‚म्म‚न्य‚ते । य‚दि साम‚ग्र्या एव कार्योत्प‚त्तिस्त‚दा स‚म‚ग्राणाम‚कार‚क‚त्वं स्यात् ।‚{\tiny $_{lb}$}‚ त‚था च प्र‚तीतिबाधा । स‚म‚ग्राणाम‚पि कार‚क‚त्वे सुत‚रां कार्य‚ब‚हुत्वं स्यात् । प‚र्यायेण‚{\tiny $_{lb}$}‚ चैक‚कार्य‚ज‚न‚क‚त्वं स्यात् । न च साम‚ग्रीब‚लात्तेषामेक‚रूप‚ता । साम‚ग्र्या एवा‚{\tiny $_{lb}$}‚भाव‚प्र‚स‚ङ्गात् । त‚स्मात् ते स‚{\tiny $_{५}$}‚म‚ग्राः स्व‚हेतुभ्य एवैक‚कार्य‚क‚र‚णे निय‚ता उत्प‚न्नाः‚{\tiny $_{lb}$}‚ साम‚ग्रीश‚ब्देनोच्य‚न्ते । न च ब‚हूनामेक‚कार्य‚क‚र‚णे बाध‚क‚म‚स्ति । न चास्माक‚ङ्का‚{\tiny $_{lb}$}‚र‚ण‚मेव कार्यो भ‚व‚तीति म‚तं येनानेक‚स्यैक‚कार्य‚त्व‚म्विरुध्येत । य‚था चैक‚कार्यं‚{\tiny $_{lb}$}‚ त्येप्र‚क‚स्य प्राग्भाव एव कार‚ण‚त्व‚न्दृष्ट‚त्वात् त‚थानेक‚स्यापि [।] य‚था वा त एव‚{\tiny $_{lb}$}‚ स‚म‚ग्राः संयोग‚ल‚क्ष‚णामेकां साम‚ग्रीञ्ज‚न‚य‚न्ति । त‚थै‚{\tiny $_{६}$}‚क‚म‚पि कार्यं किन्न कुर्व‚न्तीति[।]
	{\color{gray}{\rmlatinfont\textsuperscript{§~\theparCount}}}
	\pend% ending standard par
      ‚{\tiny $_{lb}$}‚‚{\tiny $_{lb}$}‚\textsuperscript{\textenglish{106/s}}

	  
	  \pstart \leavevmode% starting standard par
	य‚त्किञ्चिदेत‚त् । य‚त‚श्च साम‚ग्रीज‚निका\textbf{ऽत एव स‚ह‚कारिणाम‚प‚र्यायेण ज‚न‚नं} ।‚{\tiny $_{lb}$}‚ प‚रिपाट्या ज‚न‚नं नास्तीत्य‚र्थः ।
	{\color{gray}{\rmlatinfont\textsuperscript{§~\theparCount}}}
	\pend% ending standard par
      ‚{\tiny $_{lb}$}‚

	  
	  \pstart \leavevmode% starting standard par
	\textbf{य‚द‚पी}त्यादिना \textbf{विजातीया}दुत्प‚त्तिमाशंक्य प‚रिह‚र‚ति । नान्यादृशात्तादृश न्त‚{\tiny $_{lb}$}‚स्यो\textbf{त्प‚त्ति}रिति य‚दुक्त‚न्त‚देवात्र द‚र्श‚य‚तीत्य‚र्थः । आदिश‚ब्दाद् गोशृङ्गाच्छ‚रो‚{\tiny $_{lb}$}‚ \leavevmode\ledsidenote{\textenglish{41b/PSVTa}} गोम‚याद् वृश्चिकः [।] \textbf{त‚त्रापी}‚{\tiny $_{७}$}‚ति विजातीयात् कार्योत्प‚त्ताव‚पि । \textbf{त‚थाभिधानेपी}ति‚{\tiny $_{lb}$}‚ विजातीयोत्प‚न्न‚स्य शालूकादेः शालूकादिरित्य‚भिधानेपि \textbf{स्व‚वीजा}च्छालूकादि‚{\tiny $_{lb}$}‚ल‚क्ष‚णात् \textbf{प्र‚भ‚व} उत्प‚त्तिर्य\textbf{स्य} त‚स्मात् स्व‚वीज‚प्र‚भ‚वाच्छालूकादेः स‚क‚शाद\textbf{स्त्येव‚{\tiny $_{lb}$}‚ स्व‚भाव‚भेदः} । किङ्कार‚णं [।] \textbf{हेतुस्व‚भाव‚भेदात्} । हेतोः साम‚ग्रीद्व‚य‚ल‚क्ष‚ण‚स्य‚{\tiny $_{lb}$}‚ स्व‚भाव‚भेदात् । \textbf{बीजा}त् \textbf{क‚न्दा}च्चोद्\textbf{भ‚वो} य‚{\tiny $_{१}$}‚स्याः क‚द‚ल्याः सा त‚था । सा च‚{\tiny $_{lb}$}‚ हेतुत्व‚भेदात् प‚र‚स्प‚र‚भिन्ना ।
	{\color{gray}{\rmlatinfont\textsuperscript{§~\theparCount}}}
	\pend% ending standard par
      ‚{\tiny $_{lb}$}‚

	  
	  \pstart \leavevmode% starting standard par
	न चाय‚म्भेदः साध‚नीय इत्याह । \textbf{स्फुट}मित्यादि । विवेच‚य‚ति भेदे वा भाव‚स्येति ॥‚{\tiny $_{lb}$}‚ \textbf{सुविवेचिताकारं} भ्रान्तिहेतुभ्यः स‚दृशाकारेभ्यो विभागेन निश्चिताकारं ॥
	{\color{gray}{\rmlatinfont\textsuperscript{§~\theparCount}}}
	\pend% ending standard par
      ‚{\tiny $_{lb}$}‚

	  
	  \pstart \leavevmode% starting standard par
	त‚द्भाव एव भावोऽन्व‚यः । त‚द‚भावे चाभाव एव व्य‚तिरेकः । अन्व‚यो व्य‚{\tiny $_{lb}$}‚तिरेक‚श्चा\textbf{न्व‚य‚व्य‚तिरेक‚न्त‚स्मा}त् । यः कार्य‚स्व‚भावो \textbf{य‚स्या}नु‚{\tiny $_{२}$}‚व‚र्त्त‚नीय‚स्य कार‚ण‚स्या‚{\tiny $_{lb}$}‚नुव\textbf{र्त्त‚को दृष्ट‚स्त‚स्या}नुव‚र्त्त‚क‚स्य \textbf{स्व‚भाव‚स्त‚द्धेतुः} सोनुव‚र्त्त‚नीयः कार‚णात्-सा हेतु‚{\tiny $_{lb}$}‚र्य‚स्येति विग्र‚हः । य‚दा तु कार‚णापेक्ष‚योच्य‚ते त‚दा स्व‚भाव‚स्त‚स्यानुव‚र्त्त‚नीय‚स्याग्न्या‚{\tiny $_{lb}$}‚देस्त‚द्धेतुस्त‚स्य कार्याभिम‚त‚स्य धूमादेर्हेतुः । य‚त‚श्चैवं प‚र‚म्प‚रापेक्ष‚या कार्य‚कार‚ण‚यो‚{\tiny $_{lb}$}‚ ः स्व‚भाव‚निय‚मः । \textbf{अतः} कार‚णाद् \textbf{भिन्ना}द् विजातीयान्न \textbf{स‚म्भ‚{\tiny $_{३}$}‚वः} ।
	{\color{gray}{\rmlatinfont\textsuperscript{§~\theparCount}}}
	\pend% ending standard par
      ‚{\tiny $_{lb}$}‚

	  
	  \pstart \leavevmode% starting standard par
	\textbf{त‚स्मात् स‚कृद‚पि द‚र्श‚नाद‚र्श‚नाभ्यामि}ति य‚थोक्ताभ्यां । प्र‚त्य‚क्षानुप‚ल‚म्भाभ्यां‚{\tiny $_{lb}$}‚ \textbf{कार्य‚का}र‚ण\textbf{भाव‚सिद्धेः} कार‚णाद् भ‚व‚ति [।] \textbf{त‚तः} कार्य‚कार‚ण‚भाव‚सिद्धितः \textbf{त‚त्प्र‚तिप}‚{\tiny $_{lb}$}‚त्तिर‚न्व‚यं व्य‚तिरेक‚प्र‚तिप‚त्ति\textbf{र्नान्य‚थेति} कार्य‚कार‚ण‚भाव‚निश्च‚य‚न्त्य‚क्त्वा केव‚लाभ्यान्द‚{\tiny $_{lb}$}‚‚{\tiny $_{lb}$}‚ \leavevmode\ledsidenote{\textenglish{107/s}}र्श‚नाद‚र्श‚नाभ्यां नान्व‚य‚व्य‚तिरेक‚योः प्र‚तिप‚त्तिः । किङ्कार‚णं [।] \textbf{निःशेषिद‚र्श‚नाद‚र्श‚{\tiny $_{lb}$}‚ना‚{\tiny $_{४}$}‚य‚त्त‚त्वाद्} द‚र्श‚न‚मात्र‚प्र‚तिब‚द्धाया अन्व‚य‚व्य‚तिरेक‚प्र‚तिप‚त्तेः । एवं हि द‚र्श‚न‚म‚न्व‚यं‚{\tiny $_{lb}$}‚ साध‚य‚ति य‚दि निःशेषे स‚प‚क्षे हेतोर्द‚र्श‚नं स्यात् । एव‚म‚द‚र्श‚न‚म‚पि व्य‚तिरेकं साध‚येत् ।‚{\tiny $_{lb}$}‚ य‚दि \textbf{निःशेषे} साध्य‚व्य‚तिरेके हेतोर‚द‚र्श‚नं स्यात् ॥ य‚थैक‚त्र धूम‚व्य‚क्तौ स‚कृद‚पि‚{\tiny $_{lb}$}‚ कार्य‚त्व‚सिद्ध्या स‚र्व‚त्र त‚थाभावः [।] एवं क्व‚चित् स‚प‚क्षास‚प‚क्ष‚योर्द‚र्श‚नाद‚र्श‚{\tiny $_{५}$}‚नात्‚{\tiny $_{lb}$}‚ स‚र्व‚त्रान्व‚य‚व्य‚तिरेक‚निश्च‚यो भ‚विष्य‚तीत्य‚पि मिथ्या [।] य‚स्मात् \textbf{क्व‚चिद्} अमूर्त‚त्वे‚{\tiny $_{lb}$}‚ आकाशादिग‚ते \textbf{नित्य‚त्व‚स्य द‚र्श‚ने} । प‚र‚प्र‚सिद्ध्या चैत‚दुच्य‚ते । \textbf{अन्य‚त्र} सुखादौ ।‚{\tiny $_{lb}$}‚ \textbf{अन्य‚थेत्य}नित्य‚त्वे हेतोर‚मूर्त्त‚त्व‚स्य \textbf{दृष्टेः} कार‚णादेक‚त्र द‚र्श‚नं न स‚र्व‚त्र त‚थाभाव‚{\tiny $_{lb}$}‚साध‚न‚म‚तो द‚र्श‚न‚म‚न्व‚ये व्य‚भिचारि । त‚था \textbf{क्व‚चिद्} घ‚टादौ \textbf{नित्य‚त्वाभावेप्य‚दृष्ट}‚{\tiny $_{lb}$}‚स्यामू‚{\tiny $_{६}$}‚र्त्त‚त्व‚स्य पुन‚र्नित्य‚त्वाभाव एव सुखादौ \textbf{दृष्टे}र‚तो विप‚क्षैक‚देशाद‚र्श‚न‚म‚साध‚नं‚{\tiny $_{lb}$}‚ व्य‚तिरेक‚निश्च‚ये त‚स्माद्धेतुफ‚ल‚भाव‚निश्च‚यादेवान्व‚य‚व्य‚तिरेक‚योर्निश्च‚यः ॥
	{\color{gray}{\rmlatinfont\textsuperscript{§~\theparCount}}}
	\pend% ending standard par
      ‚{\tiny $_{lb}$}‚

	  
	  \pstart \leavevmode% starting standard par
	\textbf{स्व‚भाव} इत्यादि प‚रः । इदानीमिति कार्य‚हेताव‚विनाभावे साधिते संप्र‚ति‚{\tiny $_{lb}$}‚ \textbf{स्व‚भाव}हेतौ \textbf{क‚र्थ} साध्येना\textbf{विनाभावः} ।
	{\color{gray}{\rmlatinfont\textsuperscript{§~\theparCount}}}
	\pend% ending standard par
      ‚{\tiny $_{lb}$}‚

	  
	  \pstart \leavevmode% starting standard par
	न‚नु \textbf{स्व‚भावे} भावोपि \textbf{भाव‚मात्रानुरोधिनि} हेतुरि‚{\tiny $_{७}$}‚त्यादिना प्रागेव स्व‚भाव- \leavevmode\ledsidenote{\textenglish{42a/PSVTa}}‚{\tiny $_{lb}$}‚ हेताव‚विनाभावः साधितः ।
	{\color{gray}{\rmlatinfont\textsuperscript{§~\theparCount}}}
	\pend% ending standard par
      ‚{\tiny $_{lb}$}‚

	  
	  \pstart \leavevmode% starting standard par
	स‚त्यं । एव‚न्तु म‚न्य‚ते [।] य‚थाभूते स्व‚भावे त‚न्मात्र‚भाविन्य‚विनाभावो‚{\tiny $_{lb}$}‚ व‚र्ण्णित‚स्त‚त्र प्र‚तिज्ञार्थैक‚देश‚ता प्राप्नोतीति । सिद्धान्त‚वाद्य‚प्य‚न‚न्त‚रेणान्यापोह‚{\tiny $_{lb}$}‚प्र‚साध‚नेन प्र‚तिज्ञार्थैक‚देश‚ताप‚रिहार‚म्म‚न्य‚मानः त‚न्मात्रानुरोधिन्येवाविनाभावं‚{\tiny $_{lb}$}‚ पूर्वोक्त‚म‚नुव‚द‚ति । \textbf{स्व‚भावेप्य‚विनाभा‚{\tiny $_{१}$}‚व} इत्यादि । स्व‚भावेपि स्व‚भाव‚हेताव‚{\tiny $_{lb}$}‚प्य‚विनाभावः [।] क‚स्मिन् साध्ये [।] भाव‚मात्रानुरोधिनि । \textbf{यो ही}त्याद्य‚स्यैव‚{\tiny $_{lb}$}‚ व्याख्यानं । \textbf{अविनाभावो भाव‚स्ये}ति कृत‚क‚त्वादेः । य‚स्मात् \textbf{त‚द‚भावे भाव‚मात्रा‚{\tiny $_{lb}$}‚नुरोधि}साध्य‚ध‚र्मा\textbf{भावे भाव‚स्}य हेतुत्वेनोपात्त‚स्या\textbf{भावः स्यात्ः} । किं कार‚णं [।]‚{\tiny $_{lb}$}‚  \leavevmode\ledsidenote{\textenglish{108/s}}साध्य‚साध‚न‚यो\textbf{र‚भेद‚तः} । अभेद‚मेव य \textbf{एवे}त्यादिना व्याच‚ष्टे । \textbf{य एवा}नित्यादिको‚{\tiny $_{lb}$}‚ \textbf{भावः} कृत‚क\textbf{भाव‚मात्रानु‚{\tiny $_{२}$}‚रोधी-स्व‚भाव इत्युच्य}ते \textbf{स एव स्व}य‚म‚न्य‚निमित्तान‚पेक्ष‚{\tiny $_{lb}$}‚त‚या \textbf{व‚स्तुतः} प‚र‚मार्थ‚तो \textbf{भावः} कृत‚क‚त्वं \textbf{स च} भाव \textbf{आत्मानं} स्व‚भाव‚भूत‚म‚नित्य‚त्व‚{\tiny $_{lb}$}‚\textbf{म्प‚रित्य‚ज्य क‚थ‚म्भ‚वेत्} ।
	{\color{gray}{\rmlatinfont\textsuperscript{§~\theparCount}}}
	\pend% ending standard par
      ‚{\tiny $_{lb}$}‚

	  
	  \pstart \leavevmode% starting standard par
	\textbf{य एव त‚र्ही}ति प‚रः । प‚क्ष‚निर्देशः प्र‚तिज्ञा\edtext{\textsuperscript{*}}{\edlabel{pvsvt_108-1}\label{pvsvt_108-1}\lemma{*}\Bfootnote{\href{http://sarit.indology.info/?cref=ns\%C5\%AB.1.1.33}{ Nyāyasūtra. 1: 1: 33. }}} । त‚स्या \textbf{अर्थो} ध‚र्म‚ध‚र्मिस‚मुदाय‚{\tiny $_{lb}$}‚स्त‚स्यै\textbf{क‚देशः} साध्य‚ध‚र्मात्म‚को \textbf{हेतुः स्यात्} । त‚था हि याव‚दुक्त‚म‚नित्यः श‚ब्दोऽनित्य‚{\tiny $_{lb}$}‚त्वादिति ताव‚द् अनित्यः कृत‚क‚त्वादिति त‚{\tiny $_{३}$}‚था चासिद्धो हेतुरिति भावः ।
	{\color{gray}{\rmlatinfont\textsuperscript{§~\theparCount}}}
	\pend% ending standard par
      ‚{\tiny $_{lb}$}‚

	  
	  \pstart \leavevmode% starting standard par
	नैष दोष इति सि द्धा न्त वा दी । \textbf{य‚स्मात् स‚र्वे भावा} इत्यादि । अत्र प्र‚थ‚म‚या‚{\tiny $_{lb}$}‚ कारिक‚या ध‚र्म‚क‚ल्प‚नाबीजं । द्वितीय‚या ध‚र्म‚क‚ल्प‚ना । तृतीय‚या प्र‚तिज्ञार्थैक‚देश‚ता‚{\tiny $_{lb}$}‚प‚रिहार‚श्च क‚थ्य‚ते इति स‚मुदायार्थः । \textbf{स‚र्वे भावाः स्व‚भाव‚प‚र‚भावाभ्यां व्यावृत्ति}‚{\tiny $_{lb}$}‚म्भ‚ज‚न्त इति घिनुण् । स‚र्व‚भावाः \textbf{स्व‚भावेन} स्व‚रूपेण न प‚र‚रूपेण स‚जाती‚{\tiny $_{४}$}‚याद्‚{\tiny $_{lb}$}‚ विजातीयाच्च व्यावृत्ताः । \textbf{स्व‚स्व‚भाव‚व्य‚व‚स्थितेः} । स्व‚स्मिन् स्व‚भावेऽव‚स्थानात् ।
	{\color{gray}{\rmlatinfont\textsuperscript{§~\theparCount}}}
	\pend% ending standard par
      ‚{\tiny $_{lb}$}‚

	  
	  \pstart \leavevmode% starting standard par
	न‚न्व‚श्वादिभ्यो गौर्जात्या भिन्नः । विषाणी गौर्द्र‚व्येण गोव्य‚क्त्य‚न्त‚राद् भिन्नः ।‚{\tiny $_{lb}$}‚ शुक्लो गौर्गुणेन विषाणिनो गोव्य‚क्त्य‚न्त‚राद् भिन्न इति । एव‚मादिप‚र‚माण्व‚न्तो‚{\tiny $_{lb}$}‚ भेदो जात्यादिविशेष‚ण‚कृतः स‚र्व‚भावानां न स्व‚भा‚{\tiny $_{५}$}‚वेनेति ।
	{\color{gray}{\rmlatinfont\textsuperscript{§~\theparCount}}}
	\pend% ending standard par
      ‚{\tiny $_{lb}$}‚

	  
	  \pstart \leavevmode% starting standard par
	अत्रोच्य‚ते [।] न जात्यादिना ताव‚द् भावानाम‚भिन्नानाम्भेदः क्रिय‚ते । भिन्ना‚{\tiny $_{lb}$}‚भिन्न‚भेद‚क‚र‚णे त‚त्र त‚स्याकिञ्चित्क‚र‚त्वात् । नापि भिन्नानाम्वैय‚र्थ्यात् । नाप्ये‚{\tiny $_{lb}$}‚षाम्भेद‚व्य‚व‚हारः क्रिय‚ते स्व‚रूप‚भिन्नानाम्प्र‚त्य‚क्षेऽव‚भासादेव भेद‚व्य‚व‚हार‚सिद्धेः ।
	{\color{gray}{\rmlatinfont\textsuperscript{§~\theparCount}}}
	\pend% ending standard par
      ‚{\tiny $_{lb}$}‚

	  
	  \pstart \leavevmode% starting standard par
	किञ्च । जात्यादीनाम‚न्योन्य‚न्त‚द्व‚त‚श्च स‚काशाद् भेदो नान्य‚तो जात्यादे‚{\tiny $_{lb}$}‚‚{\tiny $_{lb}$}‚ ‚{\tiny $_{lb}$}‚ \leavevmode\ledsidenote{\textenglish{109/s}}र‚न‚व‚स्थाप्र‚स‚ङ्गात् । स्व‚रूपेण च भे‚{\tiny $_{६}$}‚दे भावानामेवासौ किन्नाभ्युप‚ग‚म्य‚ते किं‚{\tiny $_{lb}$}‚ जात्यादिक‚ल्प‚न‚या ।
	{\color{gray}{\rmlatinfont\textsuperscript{§~\theparCount}}}
	\pend% ending standard par
      ‚{\tiny $_{lb}$}‚

	  
	  \pstart \leavevmode% starting standard par
	योपि दि ग म्ब रो म‚न्य‚ते [।] स‚र्वात्म‚क‚मेकं स्याद‚न्यापोह‚व्य‚तिक्र‚मे ।‚{\tiny $_{lb}$}‚ त‚स्माद् भेद एवान्य‚था न स्याद् अन्योन्याभावो भावानां य‚दि न भ‚वेदिति ।
	{\color{gray}{\rmlatinfont\textsuperscript{§~\theparCount}}}
	\pend% ending standard par
      ‚{\tiny $_{lb}$}‚

	  
	  \pstart \leavevmode% starting standard par
	सोप्य‚नेन निर‚स्तः । अभावेन भाव‚भेद‚स्य क‚र्त्तुम‚श‚क्य‚त्वात् । नाप्य‚भिन्नानां‚{\tiny $_{lb}$}‚ हेतुतो निष्प‚न्नानाम‚न्योन्याभावः स‚म्भ‚व‚ति । भिन्नाश्चेत् निष्प‚न्नाः क‚थ‚म‚न्योन्या‚{\tiny $_{lb}$}‚‚{\tiny $_{७}$}‚भावः स‚म्भ‚व‚ति । भिन्नाश्चेन् निष्प‚न्नाः क‚थ‚म‚न्योन्याभाव‚क‚ल्प‚नेत्युक्तं । \leavevmode\ledsidenote{\textenglish{42b/PSVTa}}
	{\color{gray}{\rmlatinfont\textsuperscript{§~\theparCount}}}
	\pend% ending standard par
      ‚{\tiny $_{lb}$}‚

	  
	  \pstart \leavevmode% starting standard par
	न‚नु स‚र्वे भावा भिन्ना इति य‚द्येत‚द‚नुमान‚वृत्त‚न्त‚दाश्र‚यासिद्धो हेतुः स‚र्व‚भावानां‚{\tiny $_{lb}$}‚ प्र‚त्य‚क्षाविष‚य‚त्वाद् [।] अत एव नैत‚त् प्र‚त्य‚क्ष‚वृत्तं प‚रामृश्य‚ते ।
	{\color{gray}{\rmlatinfont\textsuperscript{§~\theparCount}}}
	\pend% ending standard par
      ‚{\tiny $_{lb}$}‚

	  
	  \pstart \leavevmode% starting standard par
	अन्य‚स्त्वाह । य‚द्य‚पि भावाः स्व‚भावेन भिन्नास्तेषान्तु जात्याद‚यो ध‚र्मास्स‚{\tiny $_{lb}$}‚न्त्येव प्र‚तीय‚मान‚त्वात् । त‚थापि निर्विक‚ल्प‚क‚न्तु विज्ञान‚ङ्ग‚वादिषु स‚त्तामात्रं‚{\tiny $_{१}$}‚‚{\tiny $_{lb}$}‚ गृह्वाति न भेदं । अन्य‚स्माद् विशेष‚ग्र‚ह‚ण‚मेव हि भेद‚ग्र‚ह‚ण‚म्विशेष‚श्च नाविक‚ल्प्य‚{\tiny $_{lb}$}‚ गृह्य‚त इति स‚विक‚ल्प‚क‚स्य विष‚यो न निर्विक‚ल्प‚स्य । त‚दुक्तं ॥
	{\color{gray}{\rmlatinfont\textsuperscript{§~\theparCount}}}
	\pend% ending standard par
      ‚{\tiny $_{lb}$}‚
	  \bigskip
	  \begingroup
	
	    
	    \stanza[\smallbreak]
	  {\normalfontlatin\large ``\qquad}विशेषास्तु प्र‚तीय‚न्ते स‚विक‚ल्प‚क‚बुद्धिभिः ।&‚{\tiny $_{lb}$}‚ते च केचित् प्र‚तिद्र‚व्यं केचिद् व‚हुषु संस्थिताः ॥&‚{\tiny $_{lb}$}‚तान‚क‚ल्प‚य‚दुत्प‚न्नं व्यावृत्तानुग‚मात्म‚ना ।&‚{\tiny $_{lb}$}‚ग‚वाश्वे चोप‚जात‚न्तु प्र‚त्य‚क्ष‚न्न विशिष्य‚त इति ।{\normalfontlatin\large\qquad{}"}\&[\smallbreak]
	  
	  
	  
	  \endgroup
	‚{\tiny $_{lb}$}‚

	  
	  \pstart \leavevmode% starting standard par
	त‚स्मान्न निर्विक‚ल्प‚के प्र‚त्य‚{\tiny $_{२}$}‚क्ष‚भेदाव‚भास इति ।
	{\color{gray}{\rmlatinfont\textsuperscript{§~\theparCount}}}
	\pend% ending standard par
      ‚{\tiny $_{lb}$}‚

	  
	  \pstart \leavevmode% starting standard par
	योप्याह [।] स‚विक‚ल्प‚केनापि भेदो न गृह्य‚तेऽन्योन्याभाव‚ग्र‚ह‚ण‚निमित्त‚को‚{\tiny $_{lb}$}‚ हि भावानां भेद‚ग्र‚होन्योन्याभाव‚श्च भेदो न चाभावः । प्र‚त्य‚क्ष‚ग्राह्यो । न हि ग‚व्य‚{\tiny $_{lb}$}‚श्वोस्तीति प्र‚त्य‚क्षं प‚रिच्छिन‚त्त्य‚तः स‚त्तामात्र‚स्यैव ग्राह‚कं प्र‚त्य‚क्ष‚मिति ।
	{\color{gray}{\rmlatinfont\textsuperscript{§~\theparCount}}}
	\pend% ending standard par
      ‚{\tiny $_{lb}$}‚

	  
	  \pstart \leavevmode% starting standard par
	त‚दुक्तं म ण्ड ने न ॥
	{\color{gray}{\rmlatinfont\textsuperscript{§~\theparCount}}}
	\pend% ending standard par
      ‚{\tiny $_{lb}$}‚
	  \bigskip
	  \begingroup
	
	    
	    \stanza[\smallbreak]
	  {\normalfontlatin\large ``\qquad}आहुर्विधातृ प्र‚त्य‚क्षं न निषेद्ध्य विप‚श्चितः ।&‚{\tiny $_{lb}$}‚नैक‚त्व आग‚म‚स्तेन प्र‚त्य‚क्षेण विरुध्य‚त इ‚{\tiny $_{३}$}‚ति ।{\normalfontlatin\large\qquad{}"}\&[\smallbreak]
	  
	  
	  
	  \endgroup
	‚{\tiny $_{lb}$}‚

	  
	  \pstart \leavevmode% starting standard par
	त‚देत‚दुभ‚य‚म‚प्य‚युक्तं । स‚त्तामात्र‚स्याप्र‚तिभास‚नात् । ग‚वाश्वादीनां स्व‚स्व‚{\tiny $_{lb}$}‚रूपेणैव प्र‚तिभास‚नात् । त‚दुक्तं ॥
	{\color{gray}{\rmlatinfont\textsuperscript{§~\theparCount}}}
	\pend% ending standard par
      ‚{\tiny $_{lb}$}‚
	  \bigskip
	  \begingroup
	
	    
	    \stanza[\smallbreak]
	  {\normalfontlatin\large ``\qquad}त‚त्त्व‚युक्त‚म्प्र‚तिद्र‚व्यं भिन्न‚रूपोप‚ल‚म्भ‚नात् ।&‚{\tiny $_{lb}$}‚न ह्याख्यातुम‚श‚क्य‚त्वाद् भेदो नास्तीति ग‚म्य‚त इति ।{\normalfontlatin\large\qquad{}"}\&[\smallbreak]
	  
	  
	  
	  \endgroup
	‚{\tiny $_{lb}$}‚

	  
	  \pstart \leavevmode% starting standard par
	योप्याह [।] भावानाम्भेद एव नास्ति । त‚था हि गोर‚श्वानुत्पादे यादृशं‚{\tiny $_{lb}$}‚ स्व‚रूप‚म‚श्वोत्पादेपि तादृश‚मेव नाप‚र‚म‚धिकं किञ्चिज्जात‚मिति क‚थ‚म‚तो‚{\tiny $_{४}$}‚ भेदः ।‚{\tiny $_{lb}$}‚ ‚{\tiny $_{lb}$}‚ \leavevmode\ledsidenote{\textenglish{110/s}}भ‚व‚तु वा भेदो नासौ प्र‚त्य‚क्ष‚ग्राह्योऽस्माद‚य‚म्भिन्न इति । एवंरूप‚स्य व्यापार‚स्य‚{\tiny $_{lb}$}‚ प्र‚त्य‚क्षेऽभावात् । य‚दाह । न हीद‚मिय‚तो व्यापारान् क‚र्त्तुं स‚म‚र्थ‚मिति [।]
	{\color{gray}{\rmlatinfont\textsuperscript{§~\theparCount}}}
	\pend% ending standard par
      ‚{\tiny $_{lb}$}‚

	  
	  \pstart \leavevmode% starting standard par
	सोपि निर‚स्तः । ग‚वाश्वादीनां स्व‚स्व‚रूपेणोत्प‚त्तिरेव भेदः । ते च स्व‚स्व‚रूपेण‚{\tiny $_{lb}$}‚ प्र‚त्य‚क्षेव‚भास‚न्ते । त‚थाव‚भास‚श्च लोके भेदाव‚भास इति य‚त्किञ्चिदेत‚त् । त‚स्मात्‚{\tiny $_{lb}$}‚ पुरोव‚स्थितेषु स्व‚स्व‚भाव‚व्य‚व‚स्थि‚{\tiny $_{५}$}‚तेरित्य‚स्य हेतोः [।] प्र‚त्य‚क्षेण भेदं प्र‚तिप‚द्य‚{\tiny $_{lb}$}‚मानः स‚र्वोप‚संहारेण प्र‚तिप‚द्य‚तेऽतः स‚र्व‚भावा व्यावृत्तिभागिन इति व्याप्तिग्र‚ह‚ण‚{\tiny $_{lb}$}‚प्र‚माण‚फ‚ल‚मिति । य‚त‚श्च जात्याद‚योर्थान्त‚र‚भूता न स‚न्तीति प्र‚तिपाद‚यिष्य‚ते‚{\tiny $_{lb}$}‚ [।] अतः पार‚मार्थिको ध‚र्म‚ध‚र्मिभावो नास्तीत्युक्त‚म्भ‚व‚ति ।
	{\color{gray}{\rmlatinfont\textsuperscript{§~\theparCount}}}
	\pend% ending standard par
      ‚{\tiny $_{lb}$}‚

	  
	  \pstart \leavevmode% starting standard par
	न‚नु सामान्य‚योगात् स‚जातीया उच्य‚न्ते । य‚दि च सामान्य‚न्नास्ति क‚थं स‚जा‚{\tiny $_{lb}$}‚तीयाद् व्यावृत्तिरित्यु‚{\tiny $_{६}$}‚च्य‚ते ।
	{\color{gray}{\rmlatinfont\textsuperscript{§~\theparCount}}}
	\pend% ending standard par
      ‚{\tiny $_{lb}$}‚

	  
	  \pstart \leavevmode% starting standard par
	नैत‚द‚स्ति । न स‚मानानामुत्प‚न्नानाम्भावानां सामान्य‚योगात् स‚मान‚रूप‚ता‚{\tiny $_{lb}$}‚ स्व‚हेतुभ्य एव त‚थानिष्प‚न्न‚त्वात् ।
	{\color{gray}{\rmlatinfont\textsuperscript{§~\theparCount}}}
	\pend% ending standard par
      ‚{\tiny $_{lb}$}‚

	  
	  \pstart \leavevmode% starting standard par
	\hphantom{.}तेन य‚दु द्यो त क रे णो च्य‚ते । न ग‚वि गोत्वं येन गोत्व‚योगात् प्राग् गौरेवासा‚{\tiny $_{lb}$}‚विति व्य‚र्थं गोत्वं स्याद् [।] अपि तु य‚दैव व‚स्तु त‚दैव गोत्वेनाभिस‚म्ब‚ध्य‚ते ।‚{\tiny $_{lb}$}‚ गोत्व‚योगात्तु प्राग् व‚स्तु न विद्य‚ते । न चाविद्य‚मान‚ङ् गौरिति वाऽगौरिति वा श‚क्यं‚{\tiny $_{lb}$}‚ \leavevmode\ledsidenote{\textenglish{43a/PSVTa}} व्य‚{\tiny $_{७}$}‚प‚देष्टुमिति [।]
	{\color{gray}{\rmlatinfont\textsuperscript{§~\theparCount}}}
	\pend% ending standard par
      ‚{\tiny $_{lb}$}‚

	  
	  \pstart \leavevmode% starting standard par
	त‚न्निर‚स्तं । य‚दैव व‚स्तु त‚दैव त‚स्य गोरूप‚त‚या निष्प‚न्न‚त्वात् किं गोत्व‚योगेन ।‚{\tiny $_{lb}$}‚ नाप्य‚स‚मानानां सामान्य‚योगात् स‚मान‚रूप‚ता । तेषां सामान्य‚स्यैवाभावात् । स‚मा‚{\tiny $_{lb}$}‚नानां च भावः सामान्य‚मित्य‚भ्युप‚ग‚म्य‚ते । सामान्याच्च स‚मान‚रूप‚त्वे भावानाम‚{\tiny $_{lb}$}‚भ्युप‚ग‚म्य‚माने याव‚न्न सामान्य‚योग‚स्ताव‚न्न स‚माना भावाः । याव‚च्च न स‚माना‚{\tiny $_{lb}$}‚स्ताव‚न्न सामान्य‚{\tiny $_{१}$}‚योग इत्य‚न्योन्याश्र‚य‚त्वं स्यात् । त‚स्मात् स्व‚हेतुभ्य एव स‚माना‚{\tiny $_{lb}$}‚ उत्प‚न्नाः [।] तेन ।
	{\color{gray}{\rmlatinfont\textsuperscript{§~\theparCount}}}
	\pend% ending standard par
      ‚{\tiny $_{lb}$}‚
	  \bigskip
	  \begingroup
	
	    
	    \stanza[\smallbreak]
	  {\normalfontlatin\large ``\qquad}शाव‚लेयाच्च भिन्न‚त्वं बाहुलेयाश्व‚योस्स‚मं ।&‚{\tiny $_{lb}$}‚सामान्यं नान्य‚दिष्टं चेत् क्वागोपोहः प्र‚क‚ल्प्य‚तामिति [।]{\normalfontlatin\large\qquad{}"}\&[\smallbreak]
	  
	  
	  
	  \endgroup
	‚{\tiny $_{lb}$}‚

	  
	  \pstart \leavevmode% starting standard par
	निर‚स्तं । सामान्येष्वेवागोपोह‚प्र‚क‚ल्प‚नात् । अस‚मानानां चापोह्यात्म‚त‚या‚{\tiny $_{lb}$}‚ प्र‚क‚ल्प‚नात् ।
	{\color{gray}{\rmlatinfont\textsuperscript{§~\theparCount}}}
	\pend% ending standard par
      ‚{\tiny $_{lb}$}‚

	  
	  \pstart \leavevmode% starting standard par
	य‚द‚प्युच्च‚ते [।] स‚माना इति प्र‚तिभासादेव निर्विक‚ल्प‚के ज्ञाने सामान्य‚प्र‚ति‚{\tiny $_{lb}$}‚भासोन्य‚{\tiny $_{२}$}‚था बाहुलेयाश्व‚व‚त् । शाव‚लेय‚बाहुलेय‚योर‚पि वैल‚क्ष‚ण्य‚प्र‚तीतिः स्यात्‚{\tiny $_{lb}$}‚ स‚र्वात्म‚ना भेदाद् [।] भ‚व‚ति च स‚माना इति प्र‚तीतिस्त‚स्माद‚स्यैव सामान्य‚मिति ।
	{\color{gray}{\rmlatinfont\textsuperscript{§~\theparCount}}}
	\pend% ending standard par
      ‚{\tiny $_{lb}$}‚

	  
	  \pstart \leavevmode% starting standard par
	त‚दुक्तं [।] निर्विक‚ल्प‚क‚बोधेन द्व्यात्म‚क‚स्य व‚स्तुनो ग्र‚ह‚ण‚मिति । य‚द्वा‚{\tiny $_{lb}$}‚ स‚विक‚ल्प‚के न चेत् सामान्यं गृह्य‚ते । निर्विक‚ल्प‚केनापि गृहीत‚मेव । स‚विक‚ल्पा‚{\tiny $_{lb}$}‚‚{\tiny $_{lb}$}‚ \leavevmode\ledsidenote{\textenglish{111/s}}भिन्न‚प्र‚तिभास‚त्वात् । त‚था हि [।] य एव शाव‚लेयाद‚यो‚{\tiny $_{३}$}‚ गौरिति ज्ञानेन गृह्य‚न्ते ।‚{\tiny $_{lb}$}‚ त एव निर्विक‚ल्प‚के ज्ञाने प्र‚तिभास‚न्ते केव‚ल‚मेक‚न्तान‚विक‚ल्प्य गृह्णात्य‚न्य‚द् विक‚ल्प्येति‚{\tiny $_{lb}$}‚ त‚योः स्व‚रूप‚भेदो न प्र‚तिभास‚भेदः । त‚स्मान्निर्विक‚ल्प‚केपि ज्ञाने सामान्यं प्र‚तिभास‚त‚{\tiny $_{lb}$}‚ इति [।]
	{\color{gray}{\rmlatinfont\textsuperscript{§~\theparCount}}}
	\pend% ending standard par
      ‚{\tiny $_{lb}$}‚

	  
	  \pstart \leavevmode% starting standard par
	त‚द‚पि निर‚स्तं । स्व‚हेतुभ्य एव केषाञ्चित् स‚मानानां प्र‚तिभास‚नात् । सामा‚{\tiny $_{lb}$}‚न्य‚स्य च व्य‚क्तिप‚र‚त‚न्त्रं स्व‚रूपं न च निर्विक‚ल्प‚कं ज्ञानं पार‚त‚न्त्र्य‚{\tiny $_{४}$}‚म्व‚स्तुनो गृह्णाति ।‚{\tiny $_{lb}$}‚ स्वात‚न्त्र्येण व‚स्तुग्राहित्वात् त‚त्क‚थं सामान्य‚ग्राह‚क‚मुच्य‚ते । अनुग‚त‚स्य च‚{\tiny $_{lb}$}‚ रूप‚स्य प्र‚त्येक‚व‚द् युग‚प‚द् ब‚हुष्व‚प्र‚तिभास‚नात् । अत एव विक‚ल्पः स‚मानेष्वेकान्त‚{\tiny $_{lb}$}‚भिन्नेषु निर्विक‚ल्प‚क‚प्र‚त्य‚क्ष‚बाधित‚म‚नुग‚ताकारं गृह्ण‚न् भ्रान्तो भ‚व‚ति । निर्वि‚{\tiny $_{lb}$}‚क‚ल्प‚क‚गृहीत‚सामान्य‚ग्राही वाऽप्र‚माणं स्याद् गृहीत‚ग्राहित्वात् । अथ विक‚ल्प्य‚{\tiny $_{lb}$}‚ ग्राहित्वाद् अगृ‚{\tiny $_{५}$}‚हीत‚ग्राहित्वं स्मृत्यादेर‚पित‚र्हि स्म‚र्य‚माण‚विष‚य‚ता गृहीत‚ग्राहित्व‚न्न‚{\tiny $_{lb}$}‚ स्यात् । अथ जात्यादिविशिष्ट‚व‚स्तुग्राहित‚या विक‚ल्प‚स्यागृहीत‚ग्राहित्व‚मेव‚म‚पि‚{\tiny $_{lb}$}‚ य‚दि जात्यादिविशिष्ट‚त्व‚म्व‚स्तुनः पार‚मार्थिको ध‚र्म‚स्त‚दा निर्विक‚ल्प‚केनापि गृही‚{\tiny $_{lb}$}‚त‚मेवेति क‚थ‚म‚गृहीत‚ग्राहित्व‚म् [।] अथ क‚ल्पिक‚स्त‚दा त‚द्ग्राह‚क‚स्य क‚थं प्रामाण्य‚{\tiny $_{lb}$}‚मारोपितार्थ‚त्वात् । त‚स्मान्ना‚{\tiny $_{६}$}‚स्त्येव जात्यादिरिति स्थितं ।
	{\color{gray}{\rmlatinfont\textsuperscript{§~\theparCount}}}
	\pend% ending standard par
      ‚{\tiny $_{lb}$}‚

	  
	  \pstart \leavevmode% starting standard par
	क‚थ‚न्त‚र्हि भावा व्यावृत्तिभागिन इत्युच्य‚न्ते [।] क‚ल्पित‚ध‚र्म‚द्वारेणायं व्य‚प‚देश‚{\tiny $_{lb}$}‚ इत्य‚दोषः । अतः [।]
	{\color{gray}{\rmlatinfont\textsuperscript{§~\theparCount}}}
	\pend% ending standard par
      ‚{\tiny $_{lb}$}‚
	  \bigskip
	  \begingroup
	
	    
	    \stanza[\smallbreak]
	  {\normalfontlatin\large ``\qquad}अगोनिवृत्तिः सामान्यं वाच्यं यैः प‚रिक‚ल्प्य‚ते ।&‚{\tiny $_{lb}$}‚गोत्व‚व‚स्त्वेव तैरुक्त‚म‚गोपोह‚गिरा स्फुट‚मिति [।]\edtext{}{\edlabel{pvsvt_111-1}\label{pvsvt_111-1}\lemma{मिति}\Bfootnote{\href{http://sarit.indology.info/?cref=\%C5\%9Bv-apohav\%C4\%81da}{ Ślokavārtika. Apohavāda. }}}{\normalfontlatin\large\qquad{}"}\&[\smallbreak]
	  
	  
	  
	  \endgroup
	‚{\tiny $_{lb}$}‚

	  
	  \pstart \leavevmode% starting standard par
	निर‚स्तं । पार‚मार्थिक‚स्य गोत्व‚स्य निषेधः क्रिय‚ते न तु क‚ल्पित‚स्येति व‚क्ष्य‚ति ।
	{\color{gray}{\rmlatinfont\textsuperscript{§~\theparCount}}}
	\pend% ending standard par
      ‚{\tiny $_{lb}$}‚

	  
	  \pstart \leavevmode% starting standard par
	य‚दि प‚र‚मार्थिको ध‚र्म‚ध‚र्मिभावो नास्ति क‚थ‚न्त‚{\tiny $_{७}$}‚र्हि कृत‚कः श‚ब्दो नित्य इति \leavevmode\ledsidenote{\textenglish{43b/PSVTa}}‚{\tiny $_{lb}$}‚ बुद्धीनाम्भेदः श‚ब्दैक‚स्व‚रूप‚विष‚य‚त्वात् ।
	{\color{gray}{\rmlatinfont\textsuperscript{§~\theparCount}}}
	\pend% ending standard par
      ‚{\tiny $_{lb}$}‚
	  \bigskip
	  \begingroup
	
	    
	    \stanza[\smallbreak]
	  {\normalfontlatin\large ``\qquad}अथ निर्विष‚या एता वास‚नाबीज‚मात्र‚तः ।&‚{\tiny $_{lb}$}‚प्र‚तिप‚त्तिः प्र‚वृत्तिश्च बाह्येर्थेषु क‚थ‚म्भ‚वेत् ।{\normalfontlatin\large\qquad{}"}\&[\smallbreak]
	  
	  
	  
	  \endgroup
	‚{\tiny $_{lb}$}‚

	  
	  \pstart \leavevmode% starting standard par
	अथ बाह्याध्य‚व‚सायात् प्र‚वृत्तिरेव‚म‚पि कृत‚क‚त्व‚स्य योव‚सायः स एवानित्य‚{\tiny $_{lb}$}‚स्याभेदादिति क‚थ‚म्बुद्धिभेदः । कृत‚कानित्य‚योर‚भेदादेव त‚द‚नुभ‚वाहित‚वास‚नाभे‚{\tiny $_{lb}$}‚द‚स्याभावान्न त‚त्त्व‚तो बुद्धिभेद इत्य‚त्राह ।
	{\color{gray}{\rmlatinfont\textsuperscript{§~\theparCount}}}
	\pend% ending standard par
      ‚{\tiny $_{lb}$}‚

	  
	  \pstart \leavevmode% starting standard par
	य‚स्मादित्या‚{\tiny $_{१}$}‚दि । य‚स्मात् स‚र्व‚स्मात् स‚र्व‚भावा व्यावृत्तास्त\textbf{स्माद् य‚तो य‚तो}‚{\tiny $_{lb}$}‚ नित्याकृत‚कादेः श‚ब्दादीना\textbf{म‚र्थानां व्यावृत्तिस्त‚न्निब‚न्ध‚नाः} । व्यावृत्त्याव‚धिव्या‚{\tiny $_{lb}$}‚‚{\tiny $_{lb}$}‚ ‚{\tiny $_{lb}$}‚ \leavevmode\ledsidenote{\textenglish{112/s}}वृत्तिनिब‚न्ध‚ना \textbf{ध‚र्म‚भेदा} अनित्य‚कृत‚काद‚यः \textbf{क‚ल्प्य‚न्ते} विक‚ल्पैरारोप्य‚न्ते । किम्वि‚{\tiny $_{lb}$}‚शिष्टास्\textbf{त‚द्विशेषाव‚गाहिनः । त‚स्य} स्व‚ल‚क्ष‚ण‚स्य ये \textbf{विशेषा} अकृत‚कादिव्यावृत्ति‚{\tiny $_{lb}$}‚रूप‚ल‚क्ष‚णास्त‚द\textbf{व‚गाहिनः} । त‚द‚व‚गाह‚न‚शीला‚{\tiny $_{२}$}‚स्त‚द‚भेदाव‚भास‚न‚शीला इत्य‚र्थः ।
	{\color{gray}{\rmlatinfont\textsuperscript{§~\theparCount}}}
	\pend% ending standard par
      ‚{\tiny $_{lb}$}‚

	  
	  \pstart \leavevmode% starting standard par
	एत‚दुक्त‚म्भ‚व‚ति[।]अकृत‚को न भ‚व‚तीत्य‚नेन द्वारेण प्र‚बोधिताया एव कृत‚क‚{\tiny $_{lb}$}‚विक‚ल्प‚वास‚नाया एषा प्र‚कृतिर्य‚त‚स्त‚तो विक‚ल्प उत्प‚द्य‚मानः कृत‚क इति स्वाकारा‚{\tiny $_{lb}$}‚भिन्नः स कृत‚क‚व्यावृत्त‚मेव श‚ब्द‚स्व‚ल‚क्ष‚णं प्र‚तिप‚द्य‚ते न त्व‚नित्य इति [।] त‚था नित्यो‚{\tiny $_{lb}$}‚ न भ‚व‚तीत्य‚नेनापि द्वारेण प्र‚बोधिताया एवानित्य‚वास‚नायाः साम‚{\tiny $_{३}$}‚र्थ्यं य‚त्त‚तो विक‚ल्प‚{\tiny $_{lb}$}‚ उत्प‚द्य‚मानोऽनित्य इति स्वाकाराभिन्नं नित्य‚व्यावृत्त‚मेव श‚ब्द‚स्व‚ल‚क्ष‚णं प्र‚तिप‚द्य‚ते‚{\tiny $_{lb}$}‚ न तु कृत‚क इति [।] तेन बुद्धिभेदो भ‚व‚ति । य‚त‚श्च बाध्य‚मान‚त्वाद् विक‚ल्प‚प्र‚ति‚{\tiny $_{lb}$}‚भास्य‚र्थो बाह्यो न भ‚व‚त्य‚तो वास‚नाव‚शादेव बाह्याव‚भासो विक‚ल्प‚स्तेन विक‚ल्प‚स्य‚{\tiny $_{lb}$}‚ बाह्य‚रूप एव प्र‚तिभास‚मानोर्थः स्वाकार उच्य‚ते । न तु स्वाकारे बाह्यारोपः‚{\tiny $_{४}$}‚‚{\tiny $_{lb}$}‚ सादृश्यात् । य‚दाह । नाम‚निमित्त‚प्र‚क‚र‚णे\edtext{}{\edlabel{pvsvt_112-1}\label{pvsvt_112-1}\lemma{णे}\Bfootnote{\href{http://sarit.indology.info/?cref=pv.3.12}{ Pramāṇavārtika 3: 12. }}} [।]
	{\color{gray}{\rmlatinfont\textsuperscript{§~\theparCount}}}
	\pend% ending standard par
      ‚{\tiny $_{lb}$}‚
	    
	    \stanza[\smallbreak]
	  सारूप्याद् भ्रान्तितो वृत्तिर‚र्थे चेत् स्यान्न स‚र्व‚दा ।&‚{\tiny $_{lb}$}‚देश‚भ्रान्तिश्च न ज्ञाने तुल्य‚मुत्प‚त्तितो धियः ।\&[\smallbreak]
	  
	  
	  ‚{\tiny $_{lb}$}‚

	  
	  \pstart \leavevmode% starting standard par
	त‚थाविधाया बाह्यार्थ‚प्र‚तिभासाया इति ।
	{\color{gray}{\rmlatinfont\textsuperscript{§~\theparCount}}}
	\pend% ending standard par
      ‚{\tiny $_{lb}$}‚

	  
	  \pstart \leavevmode% starting standard par
	एतेन य‚दुच्य‚ते [।] य‚द्वाह्यात्य‚न्त‚विस‚दृश‚स्य स्वाकार‚स्य ताद्रूप्य‚ग्र‚ह‚ण‚न्त‚द‚{\tiny $_{lb}$}‚न्य‚निवृत्तिकृत‚सादृश्य‚प‚रं । य‚था घ‚ट‚विस‚दृशोपि प‚टो वृक्षाभाव‚विशिष्टोऽव‚धार्य‚माणो‚{\tiny $_{lb}$}‚ य‚{\tiny $_{५}$}‚म‚प्य‚वृक्ष इति घ‚ट‚स‚दृशोव‚धार्य‚ते । वृक्ष‚व्यावृत्तेर्घ‚ट‚प‚ट‚योस्तुल्य‚त्वात् । त‚था‚{\tiny $_{lb}$}‚ विक‚ल्प‚विष‚योऽत्य‚न्तं वामे विस‚दृशोपि वामेऽव‚धार्य‚तेऽन्य‚निवृत्तिकृत‚सारूप्य‚{\tiny $_{lb}$}‚ ग्र‚ह‚णादिति [।]
	{\color{gray}{\rmlatinfont\textsuperscript{§~\theparCount}}}
	\pend% ending standard par
      ‚{\tiny $_{lb}$}‚

	  
	  \pstart \leavevmode% starting standard par
	त‚द‚पास्तं । स्व‚हेतुत एव बाह्याभासाया विक‚ल्प‚बुद्धेरुत्प‚त्तेः । न चास‚दृशानाम‚{\tiny $_{lb}$}‚न्य‚निवृत्त्या सादृश्यं क्रिय‚ते [।] त‚त्र त‚स्या भेदाभेद‚सादृश्य‚क‚र‚णेऽकिञ्चित्क‚र‚त्वात् ।‚{\tiny $_{lb}$}‚ न चा‚{\tiny $_{६}$}‚न्य‚निवृत्तिः स‚दृशी विद्य‚ते । स‚दृशानाम‚पि क‚थ‚म‚न्य‚निवृत्त्या सादृश्यं क्रिय‚ते‚{\tiny $_{lb}$}‚ स्व‚रूपेणैव सादृश्यात् । नापि पूर्वं स्वाकार‚प्र‚तिभासः प‚श्चात् त‚त्रान्य‚निवृत्तिकृत‚{\tiny $_{lb}$}‚सादृश्य‚ग्र‚ह‚ण‚प‚रो बाह्यारोपः प्र‚तिभास‚ते । न हि म‚रीचिकायां पूर्वं स्व‚रूपाप्र‚तिभासे‚{\tiny $_{lb}$}‚ ‚{\tiny $_{lb}$}‚ ‚{\tiny $_{lb}$}‚ \leavevmode\ledsidenote{\textenglish{113/s}}स‚ति सादृश्य‚ग्र‚ह‚ण‚हेतुको ज‚लारोपः स‚म्भ‚व‚ति [।] बाह्यारोपाच्च पूर्वं न स्वाका‚{\tiny $_{७}$}‚रो \leavevmode\ledsidenote{\textenglish{44a/PSVTa}}‚{\tiny $_{lb}$}‚ ग्राह्यारोपोस्ति । विक‚ल्प‚स्यैवाभावात् । भावे बोधैक‚रूप एवासाविति न बाह्य‚स‚दृशः ।‚{\tiny $_{lb}$}‚ त‚त‚श्चान्य‚निवृत्तिकृत‚सादृश्याभावात् क‚थ‚म्बाह्य‚रूपः प्र‚तीय‚ते । अनुमान‚स्य तु‚{\tiny $_{lb}$}‚ नान्य‚निवृत्तिग्र‚ह‚ण‚पूर्विका प्र‚वृत्तिर्लिङ्ग‚स्य त‚या स‚ह स‚म्ब‚न्धात् सिद्धेः । य‚दा च‚{\tiny $_{lb}$}‚ त‚द्विशेषाव‚गाहित्व‚म्विक‚ल्पानां प्र‚तीय‚ते त‚दा विधिरूपेणैव प्र‚वृत्तिर्व‚स्तुस्व‚रूप‚स्य‚{\tiny $_{lb}$}‚ विधोय‚मान‚त्वेनाध्य‚व‚{\tiny $_{१}$}‚सायाद‚र्थाद‚न्य‚निवृत्तिः प्र‚तीय‚ते । तेन विधिरेव श‚ब्दार्थोस्मा‚{\tiny $_{lb}$}‚क‚म‚पि । बाह्य‚त‚याऽरोपित‚स्य च विधिप्र‚तिषेधाभ्यां स‚म्ब‚न्धः ।
	{\color{gray}{\rmlatinfont\textsuperscript{§~\theparCount}}}
	\pend% ending standard par
      ‚{\tiny $_{lb}$}‚

	  
	  \pstart \leavevmode% starting standard par
	न‚नु निय‚त‚रूप‚ग्राही विक‚ल्पः प्र‚तिभास‚ते । तेनेत‚र‚रूप‚शून्य‚मेव विक‚ल्प‚य‚न्नि‚{\tiny $_{lb}$}‚य‚त‚म‚र्थं विक‚ल्प‚य‚ति । त‚स्माद्य‚न्निय‚त‚रूपाव‚धार‚ण‚न्त‚द‚न्य‚निवृत्तिविष‚य‚न्त‚त्क‚थ‚मुच्य‚ते‚{\tiny $_{lb}$}‚ विधिरेव श‚ब्दार्थ इति ।
	{\color{gray}{\rmlatinfont\textsuperscript{§~\theparCount}}}
	\pend% ending standard par
      ‚{\tiny $_{lb}$}‚

	  
	  \pstart \leavevmode% starting standard par
	त‚द‚युक्तं । अन्य‚निवृत्तिम‚हं विक‚ल्प‚{\tiny $_{२}$}‚यामीत्य‚प्र‚तीतेः । न च स विक‚ल्प‚विष‚यो यो‚{\tiny $_{lb}$}‚ न विक‚ल्प्य प्र‚तिभास‚ते । य‚दि चेत‚र‚रूप‚निवृत्तिमेव विक‚ल्प‚य‚न्निय‚त‚म‚र्थ‚म्विक‚ल्प‚य‚ति‚{\tiny $_{lb}$}‚ त‚देत‚रेत‚राश्र‚य‚दोषः स्यात् । इत‚र‚रूप‚स्यापि निय‚त‚रूप‚त्वेनान्य‚निवृत्तिद्वारेण‚{\tiny $_{lb}$}‚ प्र‚तिप‚त्तिप्र‚स‚ङ्गात् । अथ विधिरूपेणेत‚र‚रूपं विक‚ल्प‚य‚ति न त‚र्हि त‚द‚न्य‚निवृत्त्य‚{\tiny $_{lb}$}‚व‚धार‚ण‚पूर्व‚क‚न्निय‚त‚रूपाव‚धार‚णं ।
	{\color{gray}{\rmlatinfont\textsuperscript{§~\theparCount}}}
	\pend% ending standard par
      ‚{\tiny $_{lb}$}‚

	  
	  \pstart \leavevmode% starting standard par
	न च य‚{\tiny $_{३}$}‚था सामान्य‚व‚न्त‚म‚र्थ‚म्प‚श्यामीति नास्ति जातित‚द्व‚तोर्विशेष्य‚विशेष‚ण‚भावे‚{\tiny $_{lb}$}‚ प्र‚तिभासः [।] अथ च विशिष्ट‚प्र‚त्य‚यानुरोधात् सामान्य‚विशिष्ट‚व्य‚क्तिद‚र्श‚न‚न्नै‚{\tiny $_{lb}$}‚या यि का द‚यः क‚ल्पित‚व‚न्त‚स्त‚थाऽन्ये\edtext{}{\lemma{थाऽन्ये}\Bfootnote{? न्यैर्}} निवृत्तिम‚ह‚म्विक‚ल्प‚यामीत्य‚प्र‚तीताव‚पि‚{\tiny $_{lb}$}‚ बाह्य‚स‚दृशारोपान्य‚थानुप‚प‚त्त्यान्य‚निवृत्तिप‚रं विक‚ल्पाकारे बाह्य‚रूप‚मारोप्य‚त इति‚{\tiny $_{lb}$}‚ युक्तं । सादृश्य‚म‚न्त‚रेण‚{\tiny $_{४}$}‚ वास‚नाब‚लादेवाध्य‚व‚सित‚बाह्य‚रूप‚स्य विक‚ल्प‚स्योत्प‚त्तेः ।‚{\tiny $_{lb}$}‚ य‚दाह । त‚द‚नुभ‚वाहित‚वास‚नाप्र‚भ‚व‚प्र‚कृतेर‚ध्य‚व‚सित‚त‚द्भाव‚स्व‚रूपं । त‚था विक‚ल्प‚{\tiny $_{lb}$}‚वास‚नायाश्च तादृशी प्र‚कृतिर्य‚देव‚मेषा प्र‚तिभातीति ।
	{\color{gray}{\rmlatinfont\textsuperscript{§~\theparCount}}}
	\pend% ending standard par
      ‚{\tiny $_{lb}$}‚

	  
	  \pstart \leavevmode% starting standard par
	नापि बाह्य‚रूपारोप‚क‚स्य ज्ञान‚स्यान्य‚वित्तिकृत‚सारूप्य‚निमित्त‚त्वेप्य‚न्य‚निवृत्ति‚{\tiny $_{lb}$}‚विष‚य‚त्वं । न हि म‚रीचिकायां ज‚ल‚ज्ञान‚स्य सादृश्य‚विष‚य‚त्व‚मारोपि‚{\tiny $_{५}$}‚त‚ज‚ल‚विष‚य‚{\tiny $_{lb}$}‚त्वात् । न च निय‚तं रूपं भावानाम‚न्य‚निवृत्त्या क्रिय‚ते । त‚स्या अव‚स्तुत्वेनाकार‚{\tiny $_{lb}$}‚क‚त्वात् । स्व‚हेतुभ्य एव च त‚तो\edtext{}{\lemma{तो}\Bfootnote{? त‚त उ}} त्प‚त्ते ।
	{\color{gray}{\rmlatinfont\textsuperscript{§~\theparCount}}}
	\pend% ending standard par
      ‚{\tiny $_{lb}$}‚

	  
	  \pstart \leavevmode% starting standard par
	नापि निय‚त‚रूपाव‚धार‚ण‚न्त‚द‚न्य‚निवृत्तिविष‚यं निय‚त‚रूप‚विष‚य‚त्वाद‚स्य [।]‚{\tiny $_{lb}$}‚ अत एव न त‚द‚न्य‚निवृत्तिपुर‚स्स‚र‚मेव धार्य‚ते प्र‚त्य‚क्षेणैव ।
	{\color{gray}{\rmlatinfont\textsuperscript{§~\theparCount}}}
	\pend% ending standard par
      ‚{\tiny $_{lb}$}‚

	  
	  \pstart \leavevmode% starting standard par
	न च त‚द‚न्य‚निवृत्तिर‚र्थान्त‚र‚भूता युज्य‚त‚{\tiny $_{६}$}‚ इति व‚क्ष्य‚ति । नापि सा त‚त्त्वा‚{\tiny $_{lb}$}‚न्य‚त्वाभ्याम‚वाच्या युज्य‚ते[।] एवं ह्य‚भाव एवास्याः स्याच्छ‚श‚विषाण‚व‚त् । न च‚{\tiny $_{lb}$}‚ स्यात् प्र‚त्य‚क्ष‚ग‚म्या नीरूप‚त्वात् । नाप्य‚नुमान‚ग‚म्या । स‚म्ब‚न्धाभावेन लिङ्ग‚स्यासिद्धेः ।‚{\tiny $_{lb}$}‚ ‚{\tiny $_{lb}$}‚ \leavevmode\ledsidenote{\textenglish{114/s}}नापि निय‚त‚रूपान्य‚थानुप‚प‚त्त्या त‚त्क‚ल्प‚ना । अनिय‚त‚रूपाणाम‚न्य‚निवृत्तेर‚भावात् ।‚{\tiny $_{lb}$}‚ \leavevmode\ledsidenote{\textenglish{44b/PSVTa}} निय‚त‚रूपाणां च क‚थ‚म‚न्य‚निवृत्त्या निय‚त‚रूप‚त्वं स्व‚हेतुभ्य एव निय‚त‚रू‚{\tiny $_{७}$}‚पाणामुत्प‚त्तेः ।‚{\tiny $_{lb}$}‚ त‚स्मान्निय‚त‚रूपाव‚धार‚ण‚पुर‚स्स‚रैवान्य‚निवृत्तिः प्र‚तीय‚ते । त‚था हि घ‚ट इत्युक्तेऽघ‚टो‚{\tiny $_{lb}$}‚ न भ‚व‚तीति साम‚र्थ्यात् प्र‚तीय‚तेऽतो विक‚ल्प‚क‚ल्पितैवेषा न प‚र‚मार्थ‚तोस्ति ।
	{\color{gray}{\rmlatinfont\textsuperscript{§~\theparCount}}}
	\pend% ending standard par
      ‚{\tiny $_{lb}$}‚

	  
	  \pstart \leavevmode% starting standard par
	त‚स्मात् स्थित‚मेत‚च्छ‚द्ब‚लिङ्गाभ्याम्विधिरूपेण व‚स्तु प्र‚तिपाद्य‚तेर्थाद‚न्य‚निषेघः ।‚{\tiny $_{lb}$}‚ तंथा च व‚क्ष्य‚ति त‚द्ग‚तावेव श‚ब्देभ्यो ग‚म्य‚तेऽन्य‚निव‚र्त्त‚न‚मित्यादि \href{http://sarit.indology.info/?cref=pv.3.125}{१ । १२८} ।
	{\color{gray}{\rmlatinfont\textsuperscript{§~\theparCount}}}
	\pend% ending standard par
      ‚{\tiny $_{lb}$}‚

	  
	  \pstart \leavevmode% starting standard par
	तेन य‚दुच्य‚ते कु मा रि ले न ॥
	{\color{gray}{\rmlatinfont\textsuperscript{§~\theparCount}}}
	\pend% ending standard par
      ‚{\tiny $_{lb}$}‚
	  \bigskip
	  \begingroup
	
	    
	    \stanza[\smallbreak]
	  {\normalfontlatin\large ``\qquad}न त्व‚न्यापो‚{\tiny $_{१}$}‚ह‚कृच्छ‚ब्दो युष्म‚त्प‚क्षेनुव‚र्ण्णितः ।&‚{\tiny $_{lb}$}‚निषेध‚मात्र‚न्नैवेह प्र‚तिभासेव ग‚म्य‚ते ॥&‚{\tiny $_{lb}$}‚किन्तु गौर्ग‚व‚यो ह‚स्ती वृक्ष इत्यादि श‚ब्द‚तः ।&‚{\tiny $_{lb}$}‚विधिरूपाव‚सायेन म‚तिः शाब्दी प्र‚व‚र्त्त‚ते ॥&‚{\tiny $_{lb}$}‚त‚स्माद्येष्वेव श‚ब्देषु न‚ञ्योग‚स्तेषु केव‚लं ।&‚{\tiny $_{lb}$}‚भ‚वेद‚न्य‚निवृत्त्य‚ङ्गः स्वात्मेवान्य‚त्र ग‚म्य‚त\edtext{}{\edlabel{pvsvt_114-1}\label{pvsvt_114-1}\lemma{त}\Bfootnote{\href{http://sarit.indology.info/?cref=\%C5\%9Bv}{ Ślokavārtika. }}} इति [।]{\normalfontlatin\large\qquad{}"}\&[\smallbreak]
	  
	  
	  
	  \endgroup
	‚{\tiny $_{lb}$}‚

	  
	  \pstart \leavevmode% starting standard par
	एत‚त्सिद्धं साध्य‚ते । विधिरूप‚स्यापि श‚ब्दार्थ‚स्येष्ट‚त्वात् [।] क‚थ‚न्त‚र्हि प‚र‚म‚ताद्‚{\tiny $_{lb}$}‚ बौ द्ध म‚त‚स्य भेदः [।] क‚थं वा श‚ब्द‚लिंग‚यो‚{\tiny $_{२}$}‚र‚पोहो विष‚य उच्य‚ते ॥
	{\color{gray}{\rmlatinfont\textsuperscript{§~\theparCount}}}
	\pend% ending standard par
      ‚{\tiny $_{lb}$}‚

	  
	  \pstart \leavevmode% starting standard par
	न‚न्व‚स्त्येव म‚हान् भेदः प‚रैः पार‚मार्थिकार्थ‚विष‚य‚त्वेनेष्ट‚स्य विक‚ल्प‚स्य बौ द्धैः‚{\tiny $_{lb}$}‚ क‚ल्पित‚विष‚य‚त्वेनेष्ट‚त्वात् । क‚ल्पित‚श्चाकारोऽपोहाश्रित‚त्वाद‚पोह उच्य‚ते । अपो‚{\tiny $_{lb}$}‚ह्य‚तेऽनेनेति वा । अन्य‚निवृत्तिमात्रं त्व‚र्थादाक्षिप्त‚म‚पोह‚न‚म‚पोह इत्युच्य‚ते [।]‚{\tiny $_{lb}$}‚ स्व‚ल‚क्ष‚णं त्व‚पोह्य‚तेस्मिन्नित्य‚पोह उच्य‚ते ।
	{\color{gray}{\rmlatinfont\textsuperscript{§~\theparCount}}}
	\pend% ending standard par
      ‚{\tiny $_{lb}$}‚

	  
	  \pstart \leavevmode% starting standard par
	त‚स्माद‚न्यान्य‚व्याव‚र्त्त्य‚व‚स्तुव्य‚पेक्ष‚या ध‚र्माः क‚ल्पित‚भे‚{\tiny $_{३}$}‚दा विक‚ल्पैर्विष‚यी‚{\tiny $_{lb}$}‚क्रिय‚न्ते [।] अतो भिन्न‚विष‚या विक‚ल्पास्त‚त्स‚मान‚विष‚याश्च श‚ब्दा अप्य‚प‚र्याया‚{\tiny $_{lb}$}‚ इति द‚र्श‚यितुमाह । \textbf{त‚स्मा}दित्यादि । य‚त‚श्चैवं ध‚र्म‚भेदाः क‚ल्प्य‚न्ते \textbf{त‚स्माद्यः} स्व‚ल‚क्ष‚{\tiny $_{lb}$}‚ण‚वि\textbf{शेषो} व्याव‚र्त्त‚नीय‚नित्य‚व्य‚पेक्ष‚या व्य‚व‚स्थापितोऽनित्य‚ल‚क्ष‚णः । \textbf{येन ध‚र्मेण}‚{\tiny $_{lb}$}‚ येन श‚ब्देन । य‚थाऽनित्य‚श‚ब्देन । श‚ब्दोपि ध‚र्म‚वाच‚क‚त्वाद् ध‚र्म उच्य‚ते । न‚{\tiny $_{lb}$}‚ स \textbf{श‚क्य‚स्त‚तोन्येन} । अनि‚{\tiny $_{४}$}‚त्य‚श‚ब्दाद‚न्येन कृत‚कादिश‚ब्देन । व्याव‚र्त्त‚नीयान्त‚र‚{\tiny $_{lb}$}‚व‚स्त्व‚धिकेन प्र‚त्येतुं । \textbf{तेन भिन्ना व्य‚व‚स्थितिः} । तेन कार‚णेन विक‚ल्पानां नैक‚{\tiny $_{lb}$}‚‚{\tiny $_{lb}$}‚ ‚{\tiny $_{lb}$}‚ \leavevmode\ledsidenote{\textenglish{115/s}}विष‚य‚त्वं । श‚ब्दानां च न प‚र्याय‚त्वं ।
	{\color{gray}{\rmlatinfont\textsuperscript{§~\theparCount}}}
	\pend% ending standard par
      ‚{\tiny $_{lb}$}‚

	  
	  \pstart \leavevmode% starting standard par
	तेन य‚दुच्य‚ते [।] य‚योस्तादात्म्य‚न्न त‚योर्ग‚म्य‚ग‚म‚क‚भावो य‚योश्च विक‚ल्पित‚{\tiny $_{lb}$}‚रूप‚योर्ग‚म्य‚ग‚म‚क‚भावो न त‚योः स‚म्ब‚न्ध इति [।]
	{\color{gray}{\rmlatinfont\textsuperscript{§~\theparCount}}}
	\pend% ending standard par
      ‚{\tiny $_{lb}$}‚

	  
	  \pstart \leavevmode% starting standard par
	त‚द‚पास्तं । अकृत‚क‚व्यावृत्त‚स्यैव स्व‚ल‚क्ष‚ण‚स्य ज्ञाप‚क‚हेत्व‚धिकारात् । कृत‚क‚{\tiny $_{५}$}‚‚{\tiny $_{lb}$}‚ इति ज्ञात‚स्य ग‚म‚क‚त्वात् त‚स्य च नित्य‚व्यावृत्त‚व‚स्तुरूप‚त्वात् तादात्म्यं ।
	{\color{gray}{\rmlatinfont\textsuperscript{§~\theparCount}}}
	\pend% ending standard par
      ‚{\tiny $_{lb}$}‚

	  
	  \pstart \leavevmode% starting standard par
	\textbf{स‚र्व एव} हीत्यादिनाऽद्वै त वा दं निराकुर्व‚न् कारिकार्थ‚माह । स्व‚रूपे स्वात्म‚नि‚{\tiny $_{lb}$}‚ स्थितिर्येषान्ते त‚था । \textbf{स‚र्व एव हि भावाः स्व‚रूप‚स्थित‚यो नात्मानं प‚रेण मिश्र‚य‚न्ति} ।‚{\tiny $_{lb}$}‚ एकीकुर्व‚न्ति । किङ्कार‚णं [।] \textbf{त‚स्य} मिश्रीक्रिय‚माण‚स्य प‚र‚स्या\textbf{प‚र‚त्व‚प्र‚स‚ङ्गात्} ।‚{\tiny $_{lb}$}‚ आत्म‚ताप‚त्तेः ।
	{\color{gray}{\rmlatinfont\textsuperscript{§~\theparCount}}}
	\pend% ending standard par
      ‚{\tiny $_{lb}$}‚

	  
	  \pstart \leavevmode% starting standard par
	स्यान्म‚तं [।]
	{\color{gray}{\rmlatinfont\textsuperscript{§~\theparCount}}}
	\pend% ending standard par
      ‚{\tiny $_{lb}$}‚
	  \bigskip
	  \begingroup
	
	    
	    \stanza[\smallbreak]
	  {\normalfontlatin\large ``\qquad}स‚र्व‚व‚स्तुषु बुद्धिश्च व्यावृत्तानुग‚मा‚{\tiny $_{६}$}‚त्मिका ।&‚{\tiny $_{lb}$}‚जाय‚ते द्व्यात्म‚क‚त्वेन विना सा च न युज्य‚ते ॥{\normalfontlatin\large\qquad{}"}\&[\smallbreak]
	  
	  
	  
	  \endgroup
	‚{\tiny $_{lb}$}‚

	  
	  \pstart \leavevmode% starting standard par
	अतः सामान्यात्म‚का विशेषा विशेषात्म‚क‚ञ्च सामान्य‚मित्युभ‚य‚रूप‚म्व‚स्त्वि‚{\tiny $_{lb}$}‚त्य‚त्राप्येक‚स्य वा रूप‚स्य भिन्नेभ्योऽभेदो भिन्न‚स्य चैक‚स्माद‚भेदः [।] त‚त्र प्र‚थ‚मं प‚क्षं‚{\tiny $_{lb}$}‚ निराक‚र्त्तुमाह । \textbf{तेषामिति} भावाना\textbf{म‚भिन्न}मित्ये\textbf{कात्म‚भू}त‚मित्य‚व्य‚तिरिक्तं \textbf{य‚द्रूपं}‚{\tiny $_{lb}$}‚ स्व‚भावो \textbf{न त‚त्तेषा}म्भावानामिति श‚क्य‚म्व‚क्तुं । क‚स्मात्त\textbf{दानीन्त}स्येत्य‚पे‚{\tiny $_{७}$}‚क्ष्य‚ते । \leavevmode\ledsidenote{\textenglish{45a/PSVTa}}‚{\tiny $_{lb}$}‚ त‚स्याभिन्न‚स्य रूप‚स्य \textbf{तेषाम‚भावाद}भेदादेव ।
	{\color{gray}{\rmlatinfont\textsuperscript{§~\theparCount}}}
	\pend% ending standard par
      ‚{\tiny $_{lb}$}‚

	  
	  \pstart \leavevmode% starting standard par
	अथ पुन‚रेक‚स्माद् भिन्न‚स्याभेद‚स्त‚त्राप्याह । \textbf{त‚देव हि स्याद‚भिन्न‚स्य भावात्} ।‚{\tiny $_{lb}$}‚ एकं चेद्रूपं प्र‚तिप‚न्नाभाव‚स्त‚देवाभिन्नं रूप‚न्तेषां स्यान्न भिन्नं । कुत एत‚त् । त‚स्यै‚{\tiny $_{lb}$}‚वाभिन्न‚स्य रूप‚स्य भावात् ।
	{\color{gray}{\rmlatinfont\textsuperscript{§~\theparCount}}}
	\pend% ending standard par
      ‚{\tiny $_{lb}$}‚

	  
	  \pstart \leavevmode% starting standard par
	अथ स्यात् [।] तेषाम्भेदोपीष्य‚त एवेत्य‚त्राह । त‚स्माद‚भिन्नात्म‚नोर्थान्त‚र‚स्य‚{\tiny $_{lb}$}‚ \textbf{भिन्न‚स्य} नानारूप‚स्या\textbf{भावात्} ।
	{\color{gray}{\rmlatinfont\textsuperscript{§~\theparCount}}}
	\pend% ending standard par
      ‚{\tiny $_{lb}$}‚

	  
	  \pstart \leavevmode% starting standard par
	अथ स्याद् [।] विशेष‚स्य यो भेद‚{\tiny $_{१}$}‚स्स एव सामान्य‚स्याभेदाद् । य‚दाह [।]‚{\tiny $_{lb}$}‚ सामान्य‚स्य तु यो भेदं ब्रूते त‚स्य विशेष‚तो द‚र्श‚यित्वाभ्युपेत‚व्य इत्य‚त्राप्याह । \textbf{तंस्यैव‚{\tiny $_{lb}$}‚ च पुन‚र्भेद‚विरोधा}त् । त‚स्यैवैक‚स्यानेक‚त्वायोगात् ।
	{\color{gray}{\rmlatinfont\textsuperscript{§~\theparCount}}}
	\pend% ending standard par
      ‚{\tiny $_{lb}$}‚

	  
	  \pstart \leavevmode% starting standard par
	अथ स्याद् [।] विशेष‚द्वारेण सामान्य‚स्य भेदो न स्व‚रूप‚स्त‚दानुग‚त‚व्यावृत्त‚{\tiny $_{lb}$}‚‚{\tiny $_{lb}$}‚ \leavevmode\ledsidenote{\textenglish{116/s}}रूप‚योः प‚र‚स्प‚रासंश्लेषादेकान्तेन भेदः स्यात् । त‚दाह [।] त‚च्चैक‚रूपं \textbf{स्वात्म‚नि}‚{\tiny $_{lb}$}‚ स्व‚स्व‚भावे \textbf{व्य‚व‚स्थित‚म‚मिश्र‚मेव}‚{\tiny $_{२}$}‚ व्य‚क्तिरूपेण । अमिश्रे च मिश्र‚रूप‚त‚याप्र‚तीते‚{\tiny $_{lb}$}‚र्मिथ्यात्व‚मेव ।
	{\color{gray}{\rmlatinfont\textsuperscript{§~\theparCount}}}
	\pend% ending standard par
      ‚{\tiny $_{lb}$}‚

	  
	  \pstart \leavevmode% starting standard par
	अथ क‚थंचित् सामान्य‚स्य व्य‚क्त्य‚भिन्न‚त्वान्नैत‚न्मिथ्यात्वं । त‚दाह ।
	{\color{gray}{\rmlatinfont\textsuperscript{§~\theparCount}}}
	\pend% ending standard par
      ‚{\tiny $_{lb}$}‚
	  \bigskip
	  \begingroup
	
	    
	    \stanza[\smallbreak]
	  {\normalfontlatin\large ``\qquad}नैत‚द‚श्वादिबुद्धीनाम‚ध्यारोपाद्य‚स‚म्भ‚वात् ।&‚{\tiny $_{lb}$}‚स्थितं नैव हि जात्यादेर्भिन्न‚त्वं व्य‚क्तितो हि न इति ।{\normalfontlatin\large\qquad{}"}\&[\smallbreak]
	  
	  
	  
	  \endgroup
	‚{\tiny $_{lb}$}‚

	  
	  \pstart \leavevmode% starting standard par
	त‚द‚युक्त‚म् [।] एक‚त्वेन येनैव रूपेण भिन्न‚न्तेनैवास्याभेदो विरोधान्न चैक‚{\tiny $_{lb}$}‚स्मिन् प्र‚माणे भेदाभेदं प्र‚तिभास‚ते । एकेन च भेद‚ग्र‚ह‚णे स‚ति य‚द्य‚{\tiny $_{३}$}‚न्येनाभेदो गृह्य‚ते‚{\tiny $_{lb}$}‚ क‚थ‚न्त‚द्ग्राह‚कं प्र‚माणं भ्रान्तं न स्याद् [।] अन्य‚था ग्र‚ह‚णात्त‚स्य चैक‚रूप‚त्वात् ।
	{\color{gray}{\rmlatinfont\textsuperscript{§~\theparCount}}}
	\pend% ending standard par
      ‚{\tiny $_{lb}$}‚

	  
	  \pstart \leavevmode% starting standard par
	य‚द्वाऽमिश्र‚णादेवैक‚स्य रूप‚स्य सामान्य‚रूप‚ता न स्याद‚र्थान्त‚र‚त्वाद् घ‚ट‚व‚त् ।
	{\color{gray}{\rmlatinfont\textsuperscript{§~\theparCount}}}
	\pend% ending standard par
      ‚{\tiny $_{lb}$}‚

	  
	  \pstart \leavevmode% starting standard par
	उ द्यो त क रस्त्वाह । ग‚वादिष्व‚नुवृत्तिप्र‚त्य‚यः पिण्डादिव्य‚तिरिक्त‚नि‚{\tiny $_{lb}$}‚मित्त‚भावी । विशेष‚त्त्वान्नीलादिप्र‚त्य‚य‚व‚त् । त‚च्च निमित्तं स‚मान‚व्य‚क्तिक‚र‚णात‚{\tiny $_{lb}$}‚ सामान्य‚मित्युच्य‚ते\edtext{\textsuperscript{*}}{\edlabel{pvsvt_116-1}\label{pvsvt_116-1}\lemma{*}\Bfootnote{\href{http://sarit.indology.info/?cref=nv.2.2.65}{ Cf. Nyāyavārtika 2: 2: 65. }}} [।]
	{\color{gray}{\rmlatinfont\textsuperscript{§~\theparCount}}}
	\pend% ending standard par
      ‚{\tiny $_{lb}$}‚

	  
	  \pstart \leavevmode% starting standard par
	अत्रापि न ताव‚त्स‚मानाना‚{\tiny $_{४}$}‚म‚र्थान्त‚रेण स‚मान‚रूप‚ता क्रिय‚ते त‚थैव निष्प‚न्न‚त्वा‚{\tiny $_{lb}$}‚द प्य‚स‚मानानामिति द‚र्श‚य‚न्नाह । \textbf{अर्थान्त‚र‚मिति स‚त्ता} गोत्वादिकं \textbf{न तेषां}‚{\tiny $_{lb}$}‚ व्य‚क्तिभेदानां \textbf{सामान्य‚म‚त‚द्रूप‚त्वात्} । तेषां भेदानाम‚स‚मान‚रूप‚त्वात् । स‚मानानां च‚{\tiny $_{lb}$}‚ भावः सामान्य‚मिष्य‚ते ।
	{\color{gray}{\rmlatinfont\textsuperscript{§~\theparCount}}}
	\pend% ending standard par
      ‚{\tiny $_{lb}$}‚

	  
	  \pstart \leavevmode% starting standard par
	अथ‚वाऽत‚द्रूप‚त्वाद् व्य‚क्तिभ्योर्थान्त‚र‚त्वादेक‚त्व‚व‚त् । अथार्थान्त‚र‚म‚पि \textbf{ब‚हुषु}‚{\tiny $_{lb}$}‚ स‚म‚वेत‚मिति य‚दि त‚त्तेषां सामा‚{\tiny $_{५}$}‚न्यं । त‚दा \textbf{द्वित्वादिकार्य‚द्र‚व्येष्व‚पि प्र‚संगो} द्वित्व‚{\tiny $_{lb}$}‚म‚पि ह्य‚नेक‚द्र‚व्य‚स‚म‚वेत‚म् [।] आदिग्र‚ह‚णाद् ब‚हुत्वादिः । त‚था संयोगोनेक‚द्र‚व्य‚स‚म‚वेतः ।‚{\tiny $_{lb}$}‚ कार्य‚द्र‚व्यं चाव‚य‚विसंज्ञित‚मार‚म्भ‚क‚द्र‚व्येषु स‚म‚वेत‚म‚तो द्वित्वादिषु सामान्य‚रूप‚ता‚{\tiny $_{lb}$}‚प्र‚संगः । य‚स्मिन् नार्थान्त‚रे स‚ति स‚माना भेदा भ‚व‚न्ति त‚देव सामान्यं न‚{\tiny $_{lb}$}‚ स‚र्व‚मित्य‚त्राह । \textbf{न हि} । य‚स्मात् \textbf{स‚म्ब‚न्धिनान्येना}र्थान्त‚{\tiny $_{६}$}‚रेणैक‚त्व‚ल‚क्ष‚णेना\textbf{न्येऽ‚{\tiny $_{lb}$}‚स‚माना} भिन्ना न स‚माना नैकीक्रिय‚न्ते भिन्नाभिन्न‚स‚मान‚भाव‚क‚र‚णे त‚द‚नु‚{\tiny $_{lb}$}‚प‚योगात् । भिन्न‚देशादीनां प्र‚तिभास‚नाच्च क‚थं स‚माना एव भ‚व‚न्ति । केव‚लं‚{\tiny $_{lb}$}‚ \textbf{त‚द्व‚न्त} एक ध‚र्म‚व‚न्तः \textbf{स्यु}र्भेदाः । \textbf{भूतानि} ग्र‚ह‚न‚क्ष‚त्राणि तेषां \textbf{क‚ण्ठे} दीर्घा \textbf{गुणो}‚{\tiny $_{lb}$}‚‚{\tiny $_{lb}$}‚ ‚{\tiny $_{lb}$}‚ \leavevmode\ledsidenote{\textenglish{117/s}}र्च्च‚नार्थं निब‚ध्य‚ते । तेनैकेन क‚ण्ठे गुणेन य‚था भूतानि त‚द्व‚न्ति न त्वेकीभ‚व‚न्ति‚{\tiny $_{lb}$}‚ त‚द्व‚द्व्य‚क्त‚योपि ।
	{\color{gray}{\rmlatinfont\textsuperscript{§~\theparCount}}}
	\pend% ending standard par
      ‚{\tiny $_{lb}$}‚

	  
	  \pstart \leavevmode% starting standard par
	न‚नु गु‚{\tiny $_{७}$}‚ण‚स्य मूर्त्त‚त्वाद् त‚द्व‚त्ता प्र‚तीतिर्युक्ता । न व्य‚क्तिषु सामान्य‚स्यामूर्त्त- \leavevmode\ledsidenote{\textenglish{45b/PSVTa}}‚{\tiny $_{lb}$}‚ त्वादिति चेत् [।] न । त‚त्स‚म‚वेत‚त्व‚स्येष्ट‚त्वाद‚र्थान्त‚र‚भाव‚स्य च । य‚त‚श्चै‚{\tiny $_{lb}$}‚क‚स‚म्ब‚न्धेपि न स‚माना व्य‚क्त‚य‚स्त‚त एव \textbf{नाभिन्न‚प्र‚त्य‚य‚विष‚याः} एकाकार‚ज्ञान‚{\tiny $_{lb}$}‚स्याभ्रान्त‚स्य न विष‚याः । \textbf{भूत‚व‚त्} । य‚था भूतान्य‚गुण‚स्व‚भावानि नैक‚गुणाकार‚{\tiny $_{lb}$}‚प्र‚त्य‚य‚विष‚यः । त‚द्व‚त् । एव‚न्ताव‚द‚नेक‚स‚म्ब‚न्धेप्य‚र्थान्त‚रं न तेषां सामान्यं‚{\tiny $_{१}$}‚‚{\tiny $_{lb}$}‚द्वित्वादिष प्र‚स‚ङ्ग‚दिति स्थितं [।] प्र‚तीय‚न्ते च स‚माना इति [।] त‚स्मात् \textbf{त‚दात्मा}न‚{\tiny $_{lb}$}‚\textbf{मेव} हि त‚योर्भेद‚योरात्मान‚मेव । \textbf{एक‚रूप‚मिदं द्व्य}मित्येकांशेन स्व‚ग‚तेन \textbf{संसृज‚न्ती‚{\tiny $_{lb}$}‚ बुद्धिः सामान्य‚विष‚या प्र‚तिभास}ते । \textbf{नैक‚स‚म्ब‚न्धिनावित्ये}केन सामान्येन स‚म्ब‚{\tiny $_{lb}$}‚न्धिनावेताविति नैव‚म्बुद्धिः प्र‚तिभास‚ते येन सामान्य‚म‚र्थान्त‚र‚म्प्र‚माण‚सिद्धं स्यात् ।‚{\tiny $_{lb}$}‚ \textbf{भूत}व‚दिति वैध‚र्म्य‚दृष्टान्तः ।‚{\tiny $_{२}$}‚ य‚था बुद्धिर्भूतान्येकेन गु‚{\tiny $_{३}$}‚णेन स‚म्ब‚द्धानि गृह्णाति‚{\tiny $_{lb}$}‚ नैव‚मित्य‚र्थः ।
	{\color{gray}{\rmlatinfont\textsuperscript{§~\theparCount}}}
	\pend% ending standard par
      ‚{\tiny $_{lb}$}‚

	  
	  \pstart \leavevmode% starting standard par
	य‚द्वा य‚था भूतान्येक‚स‚म्ब‚न्धीनि त‚था भिन्नावेक‚स‚म्ब‚न्धिनाविति सामान्य‚{\tiny $_{lb}$}‚विष‚या बुद्धिः प्र‚तिभास‚त इति नैवं ।
	{\color{gray}{\rmlatinfont\textsuperscript{§~\theparCount}}}
	\pend% ending standard par
      ‚{\tiny $_{lb}$}‚

	  
	  \pstart \leavevmode% starting standard par
	अथ स्यात् [।] सामान्यं हि व्य‚क्तीनाम्विशेष‚णं । विशेष‚णं च विशेषे स्वानु‚{\tiny $_{lb}$}‚र‚क्तां बुद्धिं ज‚न‚य‚त्य‚तो नास्य व्य‚क्तिभ्योर्थान्त‚र‚भावेन प्र‚तिभासः ।
	{\color{gray}{\rmlatinfont\textsuperscript{§~\theparCount}}}
	\pend% ending standard par
      ‚{\tiny $_{lb}$}‚

	  
	  \pstart \leavevmode% starting standard par
	न‚नु विशेष‚ण‚त्वेपि त‚स्य न विशेष्येण स‚{\tiny $_{३}$}‚हैक्यं द‚ण्ड‚स्येव द‚ण्डिना त‚त्क‚थ‚म‚भेद‚{\tiny $_{lb}$}‚प्र‚तिभासः । अभेदांशेनाप्येक‚त्वान्न विशेष‚ण‚विशेष्य‚भावः । त‚स्मात् सामान्य‚स्य‚{\tiny $_{lb}$}‚ व्य‚क्त्य‚भेद‚प्र‚तीतिर्भ्रान्तिरेव ।
	{\color{gray}{\rmlatinfont\textsuperscript{§~\theparCount}}}
	\pend% ending standard par
      ‚{\tiny $_{lb}$}‚

	  
	  \pstart \leavevmode% starting standard par
	अथ स‚र्व‚दैवं प्र‚तीतेर‚भ्रान्तिः । त‚दाह ।
	{\color{gray}{\rmlatinfont\textsuperscript{§~\theparCount}}}
	\pend% ending standard par
      ‚{\tiny $_{lb}$}‚
	  \bigskip
	  \begingroup
	
	    
	    \stanza[\smallbreak]
	  {\normalfontlatin\large ``\qquad}यो ह्य‚न्य‚रूप‚संवेद्यः स‚म्वेद्येतान्य‚था पुनः ।&‚{\tiny $_{lb}$}‚स मिथ्या न तु तेनैव यो नित्य‚म‚व‚ग‚म्य‚त इति [।]{\normalfontlatin\large\qquad{}"}\&[\smallbreak]
	  
	  
	  
	  \endgroup
	‚{\tiny $_{lb}$}‚

	  
	  \pstart \leavevmode% starting standard par
	त‚द‚युक्तं । त‚स्य व्य‚क्त्य‚भिन्न‚त्व‚मेव स्यात् । गोर्गोत्व‚मिति प्र‚तीतिभे‚{\tiny $_{४}$}‚दा‚{\tiny $_{lb}$}‚भ्युप‚ग‚माच्च । न च लाक्षास्फ‚टिक‚योरिव जातित‚द्व‚तोः संस‚र्गाव‚ग‚तिर‚भ्रान्तिः ।‚{\tiny $_{lb}$}‚ अलाक्षारूप‚स्य स्फ‚टिक‚स्य लाक्षारूपेण ग‚तेर्भ्रान्त‚त्वात् । एवं जातितंद्व‚तोरेक‚त्व‚{\tiny $_{lb}$}‚  \leavevmode\ledsidenote{\textenglish{118/s}}ग्र‚हो भ्रान्तिः । सामान्यं केव‚लं प‚श्य‚त्येव बुद्धिः । त‚स्यास्तु \textbf{त‚र्द्द‚र्शिन्याः} स‚म‚वाय‚स्य‚{\tiny $_{lb}$}‚ सूक्ष्म‚त्वात् \textbf{सा भ्रान्ति}र्य‚देत‚द् व्य‚क्तीनां सामान्याभेदेन ग्र‚ह‚ण‚मिति चेत् । \textbf{त‚द्द‚र्शिनी‚{\tiny $_{lb}$}‚ति कुतः} [।] पार‚मार्थिक‚{\tiny $_{५}$}‚सामान्य‚द‚र्शिनी सा बुद्धिरिति कुतो निश्च‚यः स‚र्व‚दास्या‚{\tiny $_{lb}$}‚ व्य‚क्त्य‚भेद‚विष‚य‚त्वात् । \textbf{नास्या बीज‚म‚स्तीति निर्बीजा} त‚था चासौ \textbf{भ्रान्ति}श्च‚{\tiny $_{lb}$}‚ त‚स्या \textbf{अयोगा}त् । न हि भ्रान्तिरुत्प‚द्य‚माना निर्निमित्ता घ‚ट‚ते । ज‚लादिभ्रान्ति‚{\tiny $_{lb}$}‚व‚त् । भ्रान्तिश्चेय‚म‚संसृष्टान‚पि भावान् संसृज‚न्ती बुद्धिर‚तोस्या निमित्तेन भ‚वित‚व्यं‚{\tiny $_{lb}$}‚ [।] य‚त्त‚न्निमित्त‚न्त‚त् सामान्य‚मित्य‚त्राह । \textbf{त एव} भेदास्त‚त्सामान्य‚ज्ञा‚{\tiny $_{६}$}‚नाद्येक‚ङ्कार्यं‚{\tiny $_{lb}$}‚ येषान्ते \textbf{त‚देक‚कार्या बीजं भ्रान्तेः} । एत‚च्च प्र‚तिपाद‚यिष्य‚ते ।
	{\color{gray}{\rmlatinfont\textsuperscript{§~\theparCount}}}
	\pend% ending standard par
      ‚{\tiny $_{lb}$}‚

	  
	  \pstart \leavevmode% starting standard par
	न च सामान्यं दृष्टात‚दारोपेण व्य‚क्तिष्वेकाकारा भ्रान्तिर्युज्य‚ते । त‚था हि‚{\tiny $_{lb}$}‚ [।] सादृश्यं भ्रान्तिकार‚ण‚म‚त्य‚न्त‚विल‚क्ष‚णं च सामान्यं व्य‚क्तिभ्य‚स्त‚स्याव‚र्ण्ण‚संस्था‚{\tiny $_{lb}$}‚नाकार‚त्वाद् व्य‚क्तीनां च व‚र्ण्ण‚संस्थानाकार‚त्वात् ।
	{\color{gray}{\rmlatinfont\textsuperscript{§~\theparCount}}}
	\pend% ending standard par
      ‚{\tiny $_{lb}$}‚

	  
	  \pstart \leavevmode% starting standard par
	अथ स्यात् [।] न सामान्य‚स्य सादृश्य‚निमित्तो भेदेष्वारोपोपि त्वेकान्तेन‚{\tiny $_{lb}$}‚ \leavevmode\ledsidenote{\textenglish{46a/PSVTa}} भिन्नेषु स एवाय‚मित्ये‚{\tiny $_{७}$}‚क‚त्वाव‚साय‚विभ्र‚मो नैक‚रूप‚म‚न्त‚रेणेति ब्रूमः । य‚द्येव‚मेकेन‚{\tiny $_{lb}$}‚ स‚म्ब‚न्धिन इत्येव कृत्वा विनापि भ्रान्तिनिमित्तेन य‚द्येकाकार‚भ्रान्तिविष‚या‚{\tiny $_{lb}$}‚ भ‚व‚न्ति । त‚दा \textbf{संख्या च संयोग}श्च \textbf{कार्य‚द्र‚व्यं} चादिश‚ब्दाद् विभागादि चैक\textbf{म्व‚स्तु}‚{\tiny $_{lb}$}‚ विद्य‚ते येषान्ते त\textbf{द्व‚न्त}स्तेषु \textbf{भूते}षु चैक‚गुणेन युक्तेषु स्यादेकाकारा भ्रान्तिः [।] न‚{\tiny $_{lb}$}‚ च भ‚व‚ति [।] अतो व्य‚क्तीनाम‚पि नैक‚त्व‚निमित्ता भ्रान्तिरिति । \textbf{त‚दि}ति त‚{\tiny $_{१}$}‚स्माद्‚{\tiny $_{lb}$}‚ य‚था व्य‚क्तिभ्यो भेदेनेष्टं सामान्यं \textbf{त‚था सामा}न्य‚बुद्धौ \textbf{निवेशाभावा}त् प्र‚तिभासाभा‚{\tiny $_{lb}$}‚वान्\textbf{न सामान्य‚म‚न्}य‚त् । \textbf{स‚ति वा} सामान्ये \textbf{त‚स्या}पि सामान्य‚स्य \textbf{स्वात्म‚नि}‚{\tiny $_{lb}$}‚ स्व‚स्मिन् स्व‚भावेऽ\textbf{व‚स्थानाद‚मिश्र‚ण‚म‚न्ये}न व्य‚क्तिरूपेण । व्य‚क्तिष्व‚न‚व‚स्थानान्न‚{\tiny $_{lb}$}‚ ध‚र्म‚रूप‚त्व‚न्त‚स्येत्य‚र्थः । \textbf{त‚स्मादि}ति स्व‚स्व‚भाव‚व्य‚व‚स्थानाद् \textbf{इमे भावा} घ‚टाद‚यः‚{\tiny $_{lb}$}‚ \textbf{स‚जातीयाभिम‚ता}त् तुल्याकार‚त्वेना‚{\tiny $_{२}$}‚भिम‚ताद् \textbf{अन्य‚स्माच्चे}ति विजातीयादि‚{\tiny $_{lb}$}‚\textbf{व्य‚तिरिक्}ताः पृथ‚ग्भूताः \textbf{स्व‚भावेन} प्र‚कृत्यै\textbf{क‚रूप‚त्वात् । स्व‚स्व‚भाव‚व्य‚व‚स्थि}तेरिति‚{\tiny $_{lb}$}‚ याव‚त् ॥
	{\color{gray}{\rmlatinfont\textsuperscript{§~\theparCount}}}
	\pend% ending standard par
      ‚{\tiny $_{lb}$}‚

	  
	  \pstart \leavevmode% starting standard par
	द्वितीय‚कारिकार्थ‚माह । \textbf{य‚तो य‚तो भिन्ना}स्ते भावास्त‚स्माद् भेद‚स्त‚द्भेदः ।‚{\tiny $_{lb}$}‚ त‚स्माद् भिन्नाः स्व‚भावास्त‚स्य \textbf{प्र‚त्याय‚ना}य प्र‚तिभिन्न‚स्व‚भावं \textbf{कृत‚स‚न्निवे}शैः \textbf{श‚ब्दैः}‚{\tiny $_{lb}$}‚ ‚{\tiny $_{lb}$}‚ \leavevmode\ledsidenote{\textenglish{119/s}}कृत‚संकेतैः श‚ब्दैः क‚र‚ण‚भूतैर\textbf{नेक‚ध‚र्मा}णः कृत‚क‚त्वादि\textbf{ध‚र्म्म}व‚{\tiny $_{३}$}‚न्तः \textbf{प्र‚तीय‚न्ते} स्व‚भावा‚{\tiny $_{lb}$}‚ भेदेपि । न ह्येक‚स्य स्व‚ल‚क्ष‚ण‚स्य स्व‚भाव‚नानात्व‚म‚स्ति । क‚थ‚न्त‚र्ह्य‚नेक‚ध‚र्म‚त्व‚मित्याह ।‚{\tiny $_{lb}$}‚ \textbf{त‚त‚स्त‚तो भेद‚मुपादा}येति [।] य‚तो य‚तो व्यावृत्तास्त‚त‚स्त‚तो भेद‚मुपादायाश्रित्य‚{\tiny $_{lb}$}‚ तेपि श‚ब्दा ये त‚स्य त‚स्य भेद‚स्य भिन्न‚स्व‚भाव‚स्य ख्याप‚नाय \textbf{कृत‚संकेता}स्तेष्व‚स्मिन्‚{\tiny $_{lb}$}‚ भिन्ने स्व‚भावे वाच‚क‚त्वेन निय‚ताः [।] त‚तो नैकः श‚ब्दः स‚र्वान् भिन्न‚स्व‚{\tiny $_{४}$}‚भावानाक्षि‚{\tiny $_{lb}$}‚प‚त्य‚त‚स्\textbf{तेपि श‚ब्दाः} प्र‚त्येक‚न्तैः \textbf{स‚र्व‚भेदानाक्षेपेपि । एक‚भेद‚चोद‚ना}देकैक‚स्य भिन्न‚स्य‚{\tiny $_{lb}$}‚ स्व‚भाव‚स्य चोद‚नात्त\textbf{त्स्व‚ल‚क्ष‚ण‚निष्ठा एव भ‚व‚न्ति} । य‚स्य स्व‚ल‚क्ष‚ण‚स्य स स्व‚भावो‚{\tiny $_{lb}$}‚न्य‚स्माद् भिन्नो यः श‚ब्देनाध्यंव‚सीय‚ते । त‚च्च त‚त्स्व‚ल‚क्ष‚णं च । त‚न्निष्ठा एव‚{\tiny $_{lb}$}‚ त‚द्विष‚या एव भ‚व‚न्ति ।
	{\color{gray}{\rmlatinfont\textsuperscript{§~\theparCount}}}
	\pend% ending standard par
      ‚{\tiny $_{lb}$}‚

	  
	  \pstart \leavevmode% starting standard par
	न‚नु बाह्य‚त‚याध्य‚व‚सितोर्थः श‚ब्द‚प्र‚तिभासी साधा‚{\tiny $_{५}$}‚र‚णोन्य एवान्य‚च्च साधार‚णं‚{\tiny $_{lb}$}‚ स्व‚ल‚क्ष‚ण‚न्त‚त्क‚थ‚न्त‚न्निष्ठा इत्य‚त्राह । \textbf{त‚देक‚स्माद‚पि} य‚तो य‚तो व्यावृत्तोर्थः श‚ब्दै‚{\tiny $_{lb}$}‚र्विष‚यीक्रिय‚ते त‚स्मात्त‚स्माद‚त‚त्कार‚णाद‚त‚त्कार्याच्चैक‚स्माद‚पि \textbf{त‚स्य} स्व‚ल‚क्ष‚ण‚स्या‚{\tiny $_{lb}$}‚न‚क‚व्यावृत्त‚स्य \textbf{भेदोस्तीति} कृत्वा त‚द्विष‚या उच्य‚न्ते न तु त‚द्विष‚या एव । य‚द्वा त‚स्य‚{\tiny $_{lb}$}‚ व्यावृत्त्याश्र‚य‚स्य ध‚र्म‚स्य य‚त् स्व‚ल‚क्ष‚ण‚न्त‚न्निष्ठा एव त‚{\tiny $_{६}$}‚त्प्राप्तिप‚र्य‚व‚साना एव‚{\tiny $_{lb}$}‚ भ‚व‚न्ति । किं कार‚णं [।] त‚देक‚स्माद‚पि त‚स्य भेदोस्तीति व्याख्यात‚मेव ।
	{\color{gray}{\rmlatinfont\textsuperscript{§~\theparCount}}}
	\pend% ending standard par
      ‚{\tiny $_{lb}$}‚

	  
	  \pstart \leavevmode% starting standard par
	उप‚संह‚र‚न्नाह । \textbf{त‚स्मा}दिति । य‚स्मात् स‚जातीय‚विजातीयाद् व्यावृत्तिस्त‚{\tiny $_{lb}$}‚\textbf{स्मादेक‚स्य भाव‚स्य याव‚न्ति प‚र‚रूपाणि ताव‚त्य‚स्त‚द‚पेक्ष}या । प‚र‚रूपापेक्ष‚या । प‚र‚रू‚{\tiny $_{lb}$}‚पेभ्यो व्यावृत्त‚यः क‚{\tiny $_{७}$}‚ल्पिता ध‚र्म‚भेदाः । व्याव‚र्त्त‚ते विजातीय‚मेभिरिति कृत्वा [।] \leavevmode\ledsidenote{\textenglish{46b/PSVTa}}‚{\tiny $_{lb}$}‚ किं कार‚णं [।] त‚स्मिन् व्याव‚र्त्त्येऽव‚धिभूते ध‚र्मिण्य‚स‚म्भ‚वि कार्यं कार‚णं च य‚स्य‚{\tiny $_{lb}$}‚ विव‚क्षित‚स्य ध‚र्मिणः \textbf{स त‚द‚स‚म्भ‚विकार्य‚कार‚ण‚स्त‚स्य त‚द्भेदात्} । त‚स्माद‚त‚त्कार्या‚{\tiny $_{lb}$}‚द‚त‚त्कार‚णाच्च भेदाद् व्यावृत्त‚त्वात् । \textbf{याव‚त्य‚श्च व्यावृत्त‚य‚स्ताव‚त्यः श्रुत}यो निवे‚{\tiny $_{lb}$}‚  \leavevmode\ledsidenote{\textenglish{120/s}}शिता \textbf{अत‚त्कार्य‚कार‚ण‚प‚रिहारेण व्य‚व‚हारार्थाः} । त‚द्विव‚क्षितं कार्यं कार‚णं‚{\tiny $_{१}$}‚ च य‚स्य‚{\tiny $_{lb}$}‚ स त‚त्कार्य‚कार‚णः । \textbf{य‚था} श्रोत्र‚विज्ञान‚कार्यः श‚ब्दः प्र‚य‚त्न‚कार‚ण‚श्च स त‚था यो न‚{\tiny $_{lb}$}‚ भ‚व‚ति सोऽत‚त्कार्य‚कार‚ण‚स्त‚स्य प‚रिहारेण व्य‚व‚हारार्थाः प्र‚वृत्तिनिवृत्तिल‚क्ष‚णो व्य‚व‚{\tiny $_{lb}$}‚हारोर्थः फ‚लं यासामिति विग्र‚हः ।
	{\color{gray}{\rmlatinfont\textsuperscript{§~\theparCount}}}
	\pend% ending standard par
      ‚{\tiny $_{lb}$}‚

	  
	  \pstart \leavevmode% starting standard par
	विष‚य‚माह । प्र‚य‚त्नः कार‚णं य‚स्य स प्र‚य‚त्नान‚न्त‚रीय‚क उच्य‚ते । तेनाऽत‚त्का‚{\tiny $_{lb}$}‚र‚ण‚स्याऽप्र‚य‚त्न‚कार‚ण‚स्य विद्युदादेः प‚रिहारार्थः \textbf{प्र‚{\tiny $_{२}$}‚य‚त्नान‚न्त‚रीय}क‚ध्व‚निः । श्रोत्र‚{\tiny $_{lb}$}‚ज्ञानं कार्यं य‚स्य \textbf{त‚छ्राव‚ण}न्तेन \textbf{श्राव‚ण}ध्व‚नि\textbf{र‚त‚त्कार्य‚प‚रिहारार्थः} । स‚र्व‚व्य‚व‚हाराणां‚{\tiny $_{lb}$}‚ चार्थ‚क्रियानिमित्त‚त्वात् । हेतुफ‚ल‚भाव‚ल‚क्ष‚ण‚त्वाच्चार्थ‚क्रियाया घ‚टादिश‚ब्देष्व‚पि‚{\tiny $_{lb}$}‚ त‚द‚त‚त्कार्य‚कार‚ण‚प‚रिहार्थ‚त्वं योज्यं ॥
	{\color{gray}{\rmlatinfont\textsuperscript{§~\theparCount}}}
	\pend% ending standard par
      ‚{\tiny $_{lb}$}‚

	  
	  \pstart \leavevmode% starting standard par
	\textbf{त‚स्मादि}त्यादिना तृतीय‚श्लोकं व्याच‚ष्टे । य‚स्माद् भिन्ना व्यावृत्तिभेदाः‚{\tiny $_{lb}$}‚ श्रुत‚य‚श्च य‚थास्व‚म्भेदे निय‚तास्त‚स्मा‚{\tiny $_{३}$}‚देक‚स्य \textbf{ध‚र्मिणः स्व‚भावाऽभेदेपि । ध‚र्मिणे}‚{\tiny $_{lb}$}‚त्य‚स्य विव‚र‚णं नाम्नेति श‚ब्देनेत्य‚र्थः । \textbf{यो विशेष} इत्य‚स्यार्थो \textbf{भेदः} कृत‚कादिल‚क्ष‚णो‚{\tiny $_{lb}$}‚ ध‚र्मः क‚ल्पितः \textbf{प्र‚तीय‚ते । न स श‚क्य‚स्त‚तोन्येन} । त‚स्माद् विव‚क्षितात् कृत‚कादिश‚{\tiny $_{lb}$}‚ब्दाद‚न्येनानित्यादिश‚ब्देन प्र‚त्याय‚यितुन्त\textbf{स्मान्न प्र‚तिज्ञार्थेक‚देशो} हेतुर‚नित्य‚कृत‚क‚{\tiny $_{lb}$}‚श‚ब्द‚योर्भिन्नार्थ‚त्वात् ॥
	{\color{gray}{\rmlatinfont\textsuperscript{§~\theparCount}}}
	\pend% ending standard par
      ‚{\tiny $_{lb}$}‚

	  
	  \pstart \leavevmode% starting standard par
	\textbf{क‚थ‚मि}त्यादि प‚रः ।‚{\tiny $_{४}$}‚ श‚ब्द‚श्च लिङ्ग‚श्च ताभ्यां । य‚द्य‚पीह लिङ्गं प्र‚कृत‚न्त‚{\tiny $_{lb}$}‚थापि श‚ब्द‚स्योपादानं लिङ्ग‚व‚त् निर्विष‚य‚त्व‚ख्याप‚नार्थ‚न्तेन वे द स्य प्रामाण्यं‚{\tiny $_{lb}$}‚ निराकृत‚म्भ‚व‚ति ।
	{\color{gray}{\rmlatinfont\textsuperscript{§~\theparCount}}}
	\pend% ending standard par
      ‚{\tiny $_{lb}$}‚

	  
	  \pstart \leavevmode% starting standard par
	न‚नु विधिरूपेण श‚ब्द‚लिंगे अर्थः प्र‚तिपाद्य‚त इत्युक्त‚न्त‚त्क‚थ‚मिद‚माशंकितं‚{\tiny $_{lb}$}‚ व्य‚व‚च्छेदः प्र‚तिपाद्य‚त इति क‚थं ग‚म्य‚त इति ।
	{\color{gray}{\rmlatinfont\textsuperscript{§~\theparCount}}}
	\pend% ending standard par
      ‚{\tiny $_{lb}$}‚

	  
	  \pstart \leavevmode% starting standard par
	नैत‚द‚स्ति । व्य‚व‚च्छिद्य‚तेनेनेति व्य‚व‚च्छेदो बाह्य‚रूप‚त‚यारोपित ए‚{\tiny $_{५}$}‚वाकार‚{\tiny $_{lb}$}‚ उच्य‚ते । तेन स एव श‚ब्द‚लिङ्गाभ्यां विधिना प्र‚तिपाद्य‚ते न व‚स्तुरूप‚मिति कुतो‚{\tiny $_{lb}$}‚ ग‚म्य‚ते । य‚द्वा व्य‚व‚च्छिद्य‚तेस्मिन्निति व्य‚व‚च्छेदः । स्व‚ल‚क्ष‚ण‚मुच्य‚ते । \textbf{स एव‚{\tiny $_{lb}$}‚  \leavevmode\ledsidenote{\textenglish{121/s}}श‚ब्द‚लिंगाभ्यां विधिना} विधिरूपेण \textbf{प्र‚तिपाद्य}तेऽध्य‚व‚सीय‚ते \textbf{न} पुन\textbf{र्व‚स्तुनो रूपं}‚{\tiny $_{lb}$}‚ पार‚मार्थिक‚ध‚र्म‚ध‚र्मिभाव‚ल‚क्ष‚णं प्र‚तिपाद्य‚त \textbf{इति} कुतो \textbf{ग‚म्य}ते ।
	{\color{gray}{\rmlatinfont\textsuperscript{§~\theparCount}}}
	\pend% ending standard par
      ‚{\tiny $_{lb}$}‚

	  
	  \pstart \leavevmode% starting standard par
	\textbf{प्र‚माणान्त‚र}स्येत्यादि प्र‚{\tiny $_{६}$}‚तिव‚च‚नं । तेनाय‚म‚र्थो य‚दि ताव‚त् पार‚मार्थिको‚{\tiny $_{lb}$}‚ ध‚र्म‚ध‚र्मिभावः स‚र्वात्म‚नाऽभिन्न‚स्त‚दैकेन प्र‚माणेन श‚ब्देन वाधिग‚तेर्थे स‚र्वात्म‚ना‚{\tiny $_{lb}$}‚ प‚रिच्छेदाद‚न्य‚स्य \textbf{प्र‚माणान्त‚र‚स्य} श‚ब्दान्त‚र‚स्य वा प्र‚वृत्तिः स्याद् गृहीत‚ग्राहित्वेना‚{\tiny $_{lb}$}‚प्रामाण्यात् । भ‚व‚ति च श‚ब्दान्त‚रादेः \textbf{प्र‚वृत्तिर‚तः} क‚ल्पित एव ध‚र्म‚ध‚र्मिभावः ।
	{\color{gray}{\rmlatinfont\textsuperscript{§~\theparCount}}}
	\pend% ending standard par
      ‚{\tiny $_{lb}$}‚

	  
	  \pstart \leavevmode% starting standard par
	य‚द्य‚पि पूर्व‚सामान्य‚निराक‚र‚णा‚{\tiny $_{७}$}‚देव क‚ल्पितो ध‚र्म‚ध‚र्मिभावः प्र‚साधित‚स्त‚थापि \leavevmode\ledsidenote{\textenglish{47a/PSVTa}}‚{\tiny $_{lb}$}‚ प्र‚कारान्त‚रेणानेनापि प्र‚साध्य‚त इत्य‚दोषः । एत‚मेव प्र‚तिपाद‚य‚न्नाह । \textbf{त‚था ही}त्यादि ।‚{\tiny $_{lb}$}‚ \textbf{एक‚स्ये}ति । ध‚र्म‚ध‚र्मिविभाग‚र‚हित‚त‚त्वान्निरंश‚स्या\textbf{र्थ‚स्व‚भाव‚स्}यार्थात्म‚नः \textbf{स्व‚यं}स्व‚रूपेण‚{\tiny $_{lb}$}‚ \textbf{प्र‚त्य‚क्ष‚स्य स‚तः कोन्यो न दृष्टो भागः स्यात्} प्र‚त्य‚क्ष‚दृष्टात् स्व‚भावात् कोन्यः‚{\tiny $_{lb}$}‚ स्व‚भावो न दृष्टः स्याद् \textbf{यः प्र‚माणैर}नुमान‚सं‚{\tiny $_{१}$}‚ज्ञ‚कैः \textbf{प‚रीक्ष्}य‚ते [।] व्य‚क्तिभेदाद्‚{\tiny $_{lb}$}‚ ब‚हुव‚च‚नं । स‚र्व एव दृष्टो निरंश‚त्वाद् भाव‚स्य त‚स्मान्न प्र‚माणान्त‚र‚प‚रीक्ष्यः स्यादिति ।‚{\tiny $_{lb}$}‚ \textbf{एक} इत्यादि विव‚र‚णं । \textbf{एको ह्य‚र्थात्मा} निरंशः \textbf{स ताव‚त् प्र‚त्य‚क्षो}भ्युप‚ग‚न्त‚व्यः‚{\tiny $_{lb}$}‚ य‚त्रानुमानं प्र‚व‚र्त्त‚ते । किं कार‚णं [।] \textbf{प्र‚माणे}न प्राग\textbf{सिद्धे ध‚र्मिणि साध‚नास‚म्भ‚वात्} ।‚{\tiny $_{lb}$}‚ आश्र‚यासिद्ध‚त्वेन लिंग‚स्यास‚म्भ‚वात् त‚स्मात् प्र‚सिद्धेन ध‚र्मिणा भा‚{\tiny $_{२}$}‚व्यं ॥
	{\color{gray}{\rmlatinfont\textsuperscript{§~\theparCount}}}
	\pend% ending standard par
      ‚{\tiny $_{lb}$}‚

	  
	  \pstart \leavevmode% starting standard par
	एत‚दुक्त‚म्भ‚व‚ति । लिङ्ग‚स्यैव प्र‚वृत्तिर्न स्यात् किं पुन‚र्लिंगिनिश्चिता प्र‚माणा‚{\tiny $_{lb}$}‚ न्त‚र‚स्य प्र‚वृत्तिर्भ‚विष्य‚ति । \textbf{य‚थाऽनित्ये साध्ये श‚ब्दः} प्र‚त्य‚क्ष‚सिद्ध‚स्\textbf{त‚स्य प्र‚त्य‚क्षेणैव}‚{\tiny $_{lb}$}‚ प्र‚माणेन स्व‚रूप‚सिद्धेः कार‚णात् \textbf{स‚र्वाकार‚सिद्धिः} । क‚स्मात् \textbf{त‚द‚न्य‚स्यासिद्ध‚स्य}‚{\tiny $_{lb}$}‚ श‚ब्द‚स्व‚भावाद‚न्य‚स्य स्व‚भाव‚स्यासिद्ध‚स्या\textbf{भावात् । भावे वा}ऽसिद्ध‚स्य स्व‚भाव‚स्य‚{\tiny $_{lb}$}‚ \textbf{अत‚त्स्व‚भाव‚त्व‚म}श‚ब्द‚स्व‚भाव‚त्वं ।‚{\tiny $_{३}$}‚ सिद्धासिद्ध‚योरेक‚स्व‚भाव‚त्व‚विरोधात् ।
	{\color{gray}{\rmlatinfont\textsuperscript{§~\theparCount}}}
	\pend% ending standard par
      ‚{\tiny $_{lb}$}‚

	  
	  \pstart \leavevmode% starting standard par
	त‚देवाह । \textbf{न ही}त्यादि । अल‚ब्ध‚ध‚र्मानुवृत्तिर्योगः । ल‚ब्ध‚ध‚र्मानुवृत्तिः क्षेमः ।‚{\tiny $_{lb}$}‚ एको योगः क्षेम‚श्च य‚स्य स त‚था । तुल्य‚ध‚र्मेति याव‚त् । \textbf{यः} स्व‚भावो \textbf{येन} स‚दैक‚{\tiny $_{lb}$}‚\textbf{क‚योग‚क्षेमी न भ‚व‚ति स} भिन्न‚योग‚क्षेमः । त‚त्स्व‚भावो य‚तो भिन्न‚योग‚क्षेम‚स्\textbf{त‚स्य‚{\tiny $_{lb}$}‚ ‚{\tiny $_{lb}$}‚ \leavevmode\ledsidenote{\textenglish{122/s}}स्व‚भावो न युक्}तः । \textbf{त‚न्मात्र‚निब‚न्ध‚न‚त्वाद् भेद‚व्य‚व‚हार‚स्य} नानात्वं व्य‚व‚स्थायाः ।
	{\color{gray}{\rmlatinfont\textsuperscript{§~\theparCount}}}
	\pend% ending standard par
      ‚{\tiny $_{lb}$}‚

	  
	  \pstart \leavevmode% starting standard par
	\textbf{अन्य}थेति य‚द्येत‚द् भेद‚कार‚णं नेष्य‚ते त‚दाभेद‚व्य‚व‚हार‚स्या\textbf{भाव‚प्र‚स‚ङ्गादित्युक्तं} ।‚{\tiny $_{lb}$}‚ एष हि भेदाभेद‚हेतुर्वेत्य‚त्र प्र‚स्तावे । य‚त एव‚न्त\textbf{स्मात् प्र‚त्य‚क्षे ध‚र्मि}णि श‚ब्दादौ ।‚{\tiny $_{lb}$}‚ \textbf{त‚त्स्व‚भाव‚साक‚ल्य‚प‚रिच्छेदा}त् । ध‚र्मिस्व‚भाव‚स्य साक‚ल्येनाव‚ग‚मात् \textbf{त‚त्र} ध‚र्मिणि ।‚{\tiny $_{lb}$}‚ \textbf{अन‚व‚काशा} निर्विष‚यी \textbf{प्र‚माणान्त‚र‚स्या}नुमान‚स्य \textbf{वृत्तिः} ॥
	{\color{gray}{\rmlatinfont\textsuperscript{§~\theparCount}}}
	\pend% ending standard par
      ‚{\tiny $_{lb}$}‚

	  
	  \pstart \leavevmode% starting standard par
	त‚वापि तुल्यो दोष इति चेदाह‚{\tiny $_{५}$}‚ [।] नो \textbf{चेत्या}दि । \textbf{भ्रान्तिनिमित्ते}न स‚दृ‚{\tiny $_{lb}$}‚शाप‚रोत्प‚त्त्यादिना कार‚ण‚भूतेन विक‚ल्प‚बुद्ध्या \textbf{संयोज्येत} स‚मारोप्येत \textbf{गुणान्त‚रं ।‚{\tiny $_{lb}$}‚ स्थिर}त्वादि । वा श‚ब्द इवार्थे । \textbf{शुक्ताविव र‚ज‚ताका}रः संयोज्येत । क‚थं । र‚ज‚त‚{\tiny $_{lb}$}‚रूपेण शुक्तिका\textbf{रूप‚स्य} य‚त्\textbf{साध‚र्म्य} चैक‚चिक्यादि । त‚स्य \textbf{द‚र्श‚ना}त् । अन‚व‚धारि‚{\tiny $_{lb}$}‚त‚विशेषं शुक्तिकारूप‚मेव स‚दृश‚प्र‚त्य‚य‚निब‚न्ध‚न‚त्वाद् रूप‚साध‚र्म्य‚मुक्तं‚{\tiny $_{६}$}‚ न तु साध‚र्म्यं‚{\tiny $_{lb}$}‚ नाम द्व‚योः साधार‚ण‚म‚स्ति ।
	{\color{gray}{\rmlatinfont\textsuperscript{§~\theparCount}}}
	\pend% ending standard par
      ‚{\tiny $_{lb}$}‚

	  
	  \pstart \leavevmode% starting standard par
	एत‚दुक्त‚म्भ‚व‚ति । \textbf{य‚दि भ्रान्तिनिमित्तेन गुणान्त‚र‚न्न संयोज्येत} भ‚वेत् म‚मापि‚{\tiny $_{lb}$}‚ दोषः किन्तु स‚मारोप्येत । त‚तः स‚मारोप‚व्य‚व‚च्छेदार्थ‚म्प्र‚माणान्त‚रं प्र‚व‚र्त्त‚ते । तेना‚{\tiny $_{lb}$}‚य‚म‚र्थानुभूत‚निश्चिते प्र‚माणान्त‚र‚स्य गृहीत‚ग्राहित्वेनाप्र‚वृत्तिः स्यात् । स‚मारोपेण‚{\tiny $_{lb}$}‚ \leavevmode\ledsidenote{\textenglish{47b/PSVTa}} त्व‚नुभूतानिश्चिते त‚त्स‚मारोप‚व्य‚व‚च्छेदा‚{\tiny $_{७}$}‚र्थं प्र‚माणान्त‚र‚म‚नुमानं प्र‚व‚र्त्त‚त इत्येके ।
	{\color{gray}{\rmlatinfont\textsuperscript{§~\theparCount}}}
	\pend% ending standard par
      ‚{\tiny $_{lb}$}‚

	  
	  \pstart \leavevmode% starting standard par
	त‚द‚युक्तं । लिङ्ग‚स्य व्य‚व‚च्छेदेने स‚ह स‚म्ब‚न्धासिद्धेः । नापि याव‚द्धेतुना पूर्वं‚{\tiny $_{lb}$}‚ प‚क्ष‚व्य‚व‚च्छेदो न क्रिय‚ते ताव‚त् स‚न्दिग्धा साध्य‚प्र‚तीतिः स्याद्धेतोः साध्येन स‚म्ब‚न्धात् ।‚{\tiny $_{lb}$}‚ य‚त्र च न साध्यं न त‚त्र त‚द्विप‚क्षो विरोधात् । त‚स्माल्लिङ्गं स्व‚व्याप‚कं विधिरूपेण‚{\tiny $_{lb}$}‚ निश्चिन्व‚द‚र्थाद‚न्य‚स‚मारोपं निषेध‚ति ॥
	{\color{gray}{\rmlatinfont\textsuperscript{§~\theparCount}}}
	\pend% ending standard par
      ‚{\tiny $_{lb}$}‚

	  
	  \pstart \leavevmode% starting standard par
	न‚न्वेव‚म‚पि क‚थ‚न्त‚न्निषेधः क्ष‚णिक‚त्वानुमाने‚{\tiny $_{१}$}‚पि प्र‚वृत्तेऽक्ष‚णिकारोपात् ।
	{\color{gray}{\rmlatinfont\textsuperscript{§~\theparCount}}}
	\pend% ending standard par
      ‚{\tiny $_{lb}$}‚

	  
	  \pstart \leavevmode% starting standard par
	स‚त्यं [।] केव‚ल‚म‚नुमानेनाक्ष‚णिकार्थ‚निषेधे कृते नाय‚म‚क्ष‚णिकाकारः प्र‚त्य‚य‚{\tiny $_{lb}$}‚स्त‚ज्ज‚न्योपि तु स‚दृशाप‚रोत्प‚त्तिज‚न्य‚त्वेनारोपित इति प्र‚तिपाद्य‚ते । त‚स्मात्‚{\tiny $_{lb}$}‚ त‚स्यैवानुभूतानिश्चित‚स्य क्ष‚णिक‚त्वादेर्निश्च‚यानुमानं प्र‚व‚र्त्त‚ते । य‚त्रापि विप‚रीत‚{\tiny $_{lb}$}‚ स‚मारोपो नास्त्य‚भ्यासादेरंभावाच्च नानुभ‚वो निश्च‚य‚ज‚न‚न‚स‚म‚र्थो न त‚त्रापि‚{\tiny $_{lb}$}‚ निश्च‚या‚{\tiny $_{२}$}‚र्था प्र‚माणान्त‚र‚वृत्तिरेकांश‚निश्च‚येन स‚र्वात्म‚ना निश्चित‚त्वात् । न च‚{\tiny $_{lb}$}‚ त‚द्व‚स्तु प्र‚तिभास‚ते य‚स्य न क‚श्चिद‚पि स्व‚भावो निश्चितः । त‚स्मात् त‚त्राप्ये‚{\tiny $_{lb}$}‚‚{\tiny $_{lb}$}‚ \leavevmode\ledsidenote{\textenglish{123/s}}क‚प्र‚माण‚प्र‚वृत्ताव‚प‚र‚स्याप्र‚वृत्तिः स्यादेवेति ।
	{\color{gray}{\rmlatinfont\textsuperscript{§~\theparCount}}}
	\pend% ending standard par
      ‚{\tiny $_{lb}$}‚

	  
	  \pstart \leavevmode% starting standard par
	\textbf{य‚दी}त्यादिना विव‚र‚णं । \textbf{दृष्टं स‚र्व‚न्त‚त्त्वं} स्व‚रूपं य‚स्येति विग्र‚हः । त‚था‚{\tiny $_{lb}$}‚ \textbf{निश्च‚य‚प्र‚तिरोधिना} य‚थादृष्ट‚निश्च‚य‚विब‚न्ध‚केन \textbf{भ्रान्तिनिमित्तेन} स‚दृशाप‚रो‚{\tiny $_{lb}$}‚त्प‚त्तिल‚क्ष‚णेन । \textbf{गुणा‚{\tiny $_{३}$}‚न्त‚र}स्थिर‚त्वादि । \textbf{न संयोज्येत} । नारोप्येत । य‚था शुक्तौ‚{\tiny $_{lb}$}‚ विशिष्ट‚रूपेण गृहीतायाम‚पि शुक्ल‚साध‚र्म्याद् र‚ज‚ताकारः संयोज्येत । त‚दा‚{\tiny $_{lb}$}‚ स्याद‚न‚व‚काशा प्र‚माणान्त‚र‚वृत्तिः [।]
	{\color{gray}{\rmlatinfont\textsuperscript{§~\theparCount}}}
	\pend% ending standard par
      ‚{\tiny $_{lb}$}‚

	  
	  \pstart \leavevmode% starting standard par
	स्यान्म‚तं [।] न शुक्तौ विशिष्ट‚रूप‚ग्र‚हे र‚ज‚त‚स‚मारोपः किन्तु य‚द्र‚ज‚त‚रूप‚सामा‚{\tiny $_{lb}$}‚न्य‚न्त‚द्दृष्टेरेव‚म‚ध्य‚व‚साय इत्य‚त्राह । \textbf{न ही}ति । न हि \textbf{शुक्तौ द्वे रूपे स‚म्भ‚व‚तः} ।‚{\tiny $_{lb}$}‚ एकं र‚ज‚तेन \textbf{स‚मानं} रूपं अप‚{\tiny $_{४}$}‚र‚म्वि\textbf{शिष्ट}म‚व‚धार‚णं \textbf{च} । किं कार‚णं । \textbf{त‚था} सामान्य‚{\tiny $_{lb}$}‚विशेष‚रूपेण श‚व‚लाभासायाः प्र‚तिप‚त्तेः \textbf{स‚र्व‚दा प्र‚स‚ङ्गात् । अप्र‚तिप‚त्तौ वा विवेकेने}ति ।‚{\tiny $_{lb}$}‚ इदं सामान्य‚म‚यं विशेष इत्येव‚म्विवेकेन विभागेनाप्र‚तिप‚त्तौ \textbf{द्वित्व‚विक‚ल्पायोगात्} ।‚{\tiny $_{lb}$}‚ शुक्तौ द्वे रूपे इति क‚ल्प‚नाया अयोगात् । प्र‚तिभास‚भेद‚म‚न्त‚रेण द्वित्व‚क‚ल्प‚नाया‚{\tiny $_{lb}$}‚\textbf{म‚तिप्र‚स‚ङ्गात्} । अन्य‚त्राप्येक‚{\tiny $_{५}$}‚त्वाभिम‚ते द्वित्व‚क‚ल्प‚ना स्यात् । नेदं र‚ज‚त‚मिति‚{\tiny $_{lb}$}‚ बाध‚क‚स्यानुत्पाद‚प्र‚संगाच्च । त‚स्माच्छुक्तौ रूप‚द्व‚यायोगात ।
	{\color{gray}{\rmlatinfont\textsuperscript{§~\theparCount}}}
	\pend% ending standard par
      ‚{\tiny $_{lb}$}‚

	  
	  \pstart \leavevmode% starting standard par
	\textbf{प‚श्य‚न् शुक्तिरूपं} पुरुषो \textbf{विशिष्ट‚मेव} स्व‚ल‚क्ष‚ण‚मेव \textbf{प‚श्य‚ति} न सामान्यं ।‚{\tiny $_{lb}$}‚ अभ्यासाद‚यो \textbf{निश्च‚य‚प्र‚त्य‚या}स्तेषाम्वैफ‚ल्यात् त्व‚निश्चिन्व‚न् द्विविशिष्टं‚{\tiny $_{lb}$}‚ शुक्तिकारूप\textbf{न्त‚त्सामान्यं} र‚ज‚त‚रूप‚सामान्य‚म्\textbf{प‚श्यामीति म‚न्य‚ते} । त‚त इति र‚ज‚त‚{\tiny $_{lb}$}‚रूप‚सामान्य‚ग्र‚ह‚णाभिमाना‚{\tiny $_{६}$}‚ द‚स्य द्र‚ष्टुः शुक्तौ र‚ज‚त‚स‚मारोपः । य‚द्वा शुक्ति‚{\tiny $_{lb}$}‚कार‚ज‚त‚योर्य‚त्सामान्यं तुल्यं रूप‚न्त‚त्प‚श्यामीति म‚न्य‚ते [।] \textbf{त‚तो} रूप‚द‚र्श‚ना\textbf{द‚स्य} द्र‚ष्टू‚{\tiny $_{lb}$}‚ \textbf{र‚ज‚त‚स‚मारोपः} । त‚था हि यादृश‚मेव म‚या र‚ज‚त‚स्य रूपं प्र‚तिप‚न्न‚न्तादृश‚मेवास्यापि‚{\tiny $_{lb}$}‚ रूप‚न्त‚स्मात्त‚देवेदं र‚ज‚त‚मिति प्र‚तीतिः । य‚था शुक्तौ र‚ज‚त‚स‚मारोप‚स्त‚था दृष्टे‚{\tiny $_{lb}$}‚ श‚ब्दादौ ध‚र्मिणि \textbf{स‚दृशाप‚राप‚रोत्प‚त्त्या} स‚{\tiny $_{७}$}‚दृश‚स्य द्वितीय‚स्य क्ष‚ण‚स्योत्प‚त्त्या भ्रान्ति- \leavevmode\ledsidenote{\textenglish{48a/PSVTa}}‚{\tiny $_{lb}$}‚ निमित्तेन पूर्वोत्त‚र‚क्ष‚ण‚यो\textbf{र‚ल‚क्षित‚नानात्व}स्य पुंस‚स्\textbf{त‚द्भाव‚स‚मारोपा}त् स‚त्तास‚मारो‚{\tiny $_{lb}$}‚  \leavevmode\ledsidenote{\textenglish{124/s}}पात् \textbf{स्थितिभ्रान्तिः} । त‚स्माद् \textbf{याव‚न्तोस्य} श‚ब्दादेः क्ष‚णिकानात्मादिस्व‚भाव‚स्य‚{\tiny $_{lb}$}‚ \textbf{प‚र‚भा}वा नित्याद‚य‚स्\textbf{ताव‚न्त एव य‚थास्व‚निमित्त‚भाविनः} य‚स्य य‚द‚नुरूपं निमित्त‚न्त‚{\tiny $_{lb}$}‚द्भाविनः \textbf{स‚मारोपा‚{\tiny $_{१}$}‚ इति । त‚द्व्य‚व‚च्छेद‚कानि} त‚स्य स‚मारोप‚स्यार्थाद् व्य‚व‚च्छेद‚{\tiny $_{lb}$}‚कानि \textbf{भ‚व‚न्ती}त्युत्प‚द्य‚मानानि \textbf{प्र‚माणा}न्य‚नुमान‚संज्ञितानि । व्य‚क्तिभेदाद् ब‚हुव‚च‚नं ।‚{\tiny $_{lb}$}‚ \textbf{स‚फ‚लानि स्युः} सार्थ‚कानि स्युः । \textbf{तेषान्तु व्य‚व‚च्छेद‚फ‚लानां} स‚मारोप‚प्र‚तिषेध‚फ‚लानां‚{\tiny $_{lb}$}‚ प्र‚माणानां । \textbf{नाप्र‚तीत‚स्य व‚स्त्व‚ङ‚श‚स्य} व‚स्तुभाग‚स्य \textbf{प्र‚त्याय‚ने} प्र‚काश‚ने \textbf{प्र‚वृत्तिः} ।‚{\tiny $_{lb}$}‚ त‚स्य व‚स्त्वंश‚स्य \textbf{निरंशे} ध‚र्मिणि दृश्य‚{\tiny $_{२}$}‚माने दृष्ट‚त्वात् । किं कार‚ण‚म् [।] \textbf{अनंश‚स्ये‚{\tiny $_{lb}$}‚क‚देशेन द‚र्श‚नायोगात्} ॥
	{\color{gray}{\rmlatinfont\textsuperscript{§~\theparCount}}}
	\pend% ending standard par
      ‚{\tiny $_{lb}$}‚

	  
	  \pstart \leavevmode% starting standard par
	य‚त एव‚न्त\textbf{स्मात्} प्र‚त्य‚क्षेण \textbf{दृष्ट‚स्य भाव‚स्य दृष्ट एवाखिलो गुणः}‚{\tiny $_{lb}$}‚ स‚म‚स्तः स्व‚भाव‚स्त‚थापि \textbf{भ्रान्त्तेः} स‚द्भावा\textbf{न्निश्चीय‚ते नेति} कृत्वा त‚न्निश्च‚यार्थं‚{\tiny $_{lb}$}‚ \textbf{साध‚न}म‚नुमानं विधिरूपेणैव \textbf{प्र‚व‚र्त्त‚त} इति स्प‚ष्ट‚मेवोक्तं ।
	{\color{gray}{\rmlatinfont\textsuperscript{§~\theparCount}}}
	\pend% ending standard par
      ‚{\tiny $_{lb}$}‚

	  
	  \pstart \leavevmode% starting standard par
	\textbf{त‚स्मान्नादृष्ट‚ग्र‚ह‚णार्था दृष्टे} प्र‚त्य‚क्षेण \textbf{प्र‚माणान्त‚र‚स्य} [।] किन्त‚र्हि दृष्ट‚{\tiny $_{lb}$}‚निश्च‚या‚{\tiny $_{३}$}‚ चैव \textbf{प्र‚वृत्ति}रित्य‚नेनापि विध्य‚र्थः । स्प‚ष्ट एवोक्तोन्य‚निषेध‚स्त्व‚र्थात् ॥
	{\color{gray}{\rmlatinfont\textsuperscript{§~\theparCount}}}
	\pend% ending standard par
      ‚{\tiny $_{lb}$}‚

	  
	  \pstart \leavevmode% starting standard par
	\textbf{व‚स्तुग्र‚ह} इत्यादि । च‚कारः प्र‚त्य‚क्षापेक्ष‚या स‚मुच्च‚यार्थः । अनुमानेन च‚{\tiny $_{lb}$}‚ \textbf{व‚स्तुग्र‚हे}ङ्गीक्रिय‚माणे । श‚ब्दादिध‚र्मिणो \textbf{ध‚र्म‚स्यैक‚स्य} कृत‚क‚त्वादिल‚क्ष‚ण‚स्य \textbf{निश्च}‚{\tiny $_{lb}$}‚येन निरंश‚त्वाद् ध‚र्मिणः \textbf{स‚र्व‚ध‚र्म}निश्च‚य‚स्त‚दा नित्य‚त्वाद्य‚नुमानान्त‚र‚वैफ‚ल्य‚{\tiny $_{lb}$}‚मित्याकूतं ।
	{\color{gray}{\rmlatinfont\textsuperscript{§~\theparCount}}}
	\pend% ending standard par
      ‚{\tiny $_{lb}$}‚\textsuperscript{\textenglish{125/s}}

	  
	  \pstart \leavevmode% starting standard par
	अपोहो बाह्य‚त‚या आरोपित आकारो‚{\tiny $_{४}$}‚पोह्य‚तेऽनेनेति कृत्वा । त‚स्मिन्न‚नु‚{\tiny $_{lb}$}‚मानेन प्र‚तिपाद्येङ्गीक्रिय‚माणे य‚द्वापोह्य‚तेस्मिन्नित्य‚पोहः स्व‚ल‚क्ष‚णं [।] त‚स्मिन्‚{\tiny $_{lb}$}‚ स्वाकाराभिन्न‚त‚याऽनुमान‚ग्राह्येऽभ्युप‚ग‚म्य‚माने \textbf{नाय‚म}न‚न्त‚रोक्तो \textbf{दोषः प्र‚स‚ज्य‚ते} ।
	{\color{gray}{\rmlatinfont\textsuperscript{§~\theparCount}}}
	\pend% ending standard par
      ‚{\tiny $_{lb}$}‚

	  
	  \pstart \leavevmode% starting standard par
	\textbf{न केव‚ल‚मि}त्यादि विव‚र‚णं । \textbf{क्व‚चित्} प्र‚त्य‚क्ष‚दृष्टे ध‚र्मिणीति स‚म्ब‚न्धः ।‚{\tiny $_{lb}$}‚ क‚स्मिँश्चित् \textbf{प्र‚त्य‚क्ष‚दृष्टे} ध‚र्मिणीत्य‚र्थः । न स‚र्वो ध‚र्मी प्र‚त्य‚क्षो भ‚व‚तीति क्व‚चिद्‚{\tiny $_{lb}$}‚ ग्र‚{\tiny $_{५}$}‚ह‚णं । अन‚न्त‚रोक्तेन न्यायेन \textbf{प्र‚माणान्त‚रावृत्तिः} ।
	{\color{gray}{\rmlatinfont\textsuperscript{§~\theparCount}}}
	\pend% ending standard par
      ‚{\tiny $_{lb}$}‚

	  
	  \pstart \leavevmode% starting standard par
	\textbf{त‚दानुमान‚म‚पि व‚स्तु} श‚ब्दादिकं \textbf{विधिना प्र‚त्याय‚तीष्य‚ते} व‚स्त्व‚ध्य‚व‚सायेनार्थान्न‚{\tiny $_{lb}$}‚ स‚मारोप\textbf{व्य‚व‚च्छेद‚कृत् । त‚दे}ति विधिरूपेण व‚स्तुस्व‚रूप‚ग्र‚ह‚णे । \textbf{एक‚ध‚र्म‚निश्च‚ये} स‚ति‚{\tiny $_{lb}$}‚ \textbf{त‚द‚व्य‚तिरेका}न्निश्चित‚ध‚र्माव्य‚तिरेकात् \textbf{स‚र्व‚ध‚र्म‚निश्च‚य इति प्र‚माणान्त‚रावृत्ति}‚{\tiny $_{lb}$}‚र‚नुमानान्त‚राप्र‚वृत्तिः । स‚त्य‚प्य‚व्य‚तिरेके । न स‚र्व‚{\tiny $_{६}$}‚ध‚र्म‚निश्च‚य इति चेदाह ।‚{\tiny $_{lb}$}‚ \textbf{न ही}त्यादि । \textbf{त‚स्मि}न्निति प्र‚थ‚मानुमान\textbf{निश्चिते} ध‚र्मे । \textbf{त‚दात्मे}ति । निश्चित‚{\tiny $_{lb}$}‚ध‚र्मात्मा स‚न्न\textbf{निश्चितो} न हि युक्तः ।
	{\color{gray}{\rmlatinfont\textsuperscript{§~\theparCount}}}
	\pend% ending standard par
      ‚{\tiny $_{lb}$}‚

	  
	  \pstart \leavevmode% starting standard par
	अपोहे नाय‚न्दोषः प्र‚स‚ज्य‚त इत्येत‚द्विवृण्व‚न्नाह । \textbf{य‚दा पुन}रित्यादि । य‚दा‚{\tiny $_{lb}$}‚ पुन\textbf{र‚नुमानेन} व‚स्त्व‚ध्य‚व‚सायं कुर्व‚ताऽर्थात् \textbf{स‚मारोप‚व्य‚व‚च्छेदः क्रिय‚ते} न तु साक्षा‚{\tiny $_{lb}$}‚ज्ज्ञाप‚क‚त्वाद‚स्य । तेनैत‚न्निर‚स्तं स‚मारोप‚व्य‚व‚च्छेद‚{\tiny $_{७}$}‚ एव व्य‚व‚हाराद‚न्यार्थ- \leavevmode\ledsidenote{\textenglish{48b/PSVTa}}‚{\tiny $_{lb}$}‚ प्र‚वृत्तिर्न स्याद‚नुमानेनानिश्चित‚त्वादिति । \textbf{त‚दैकेन} कृत‚कानुमानेनैक‚स्याकृत‚क‚{\tiny $_{lb}$}‚\textbf{स‚मारोप‚स्य व्य‚व‚च्छेदाद‚न्य‚स्य} नित्य‚स‚मारोप‚स्य \textbf{व्य‚व‚च्छेदः कृतो न भ‚व‚ती}ति कृत्वा ।‚{\tiny $_{lb}$}‚ \textbf{त‚द‚र्थ}म‚न्य‚स‚मारोप‚व्य‚व‚च्छेदार्थ\textbf{म‚न्य}दित्य‚नित्य‚त्वाद्य‚नुमानं \textbf{प्र‚व‚र्त्त‚ते} । त‚स्मान्न‚{\tiny $_{lb}$}‚ विक‚ल्पानां स्व‚रूपेण बाह्यो ग्राह्योऽपि तु स्वाकारेण स‚हैकीकृत एव बाह्यो विष‚यः‚{\tiny $_{lb}$}‚ [।] स चास‚{\tiny $_{१}$}‚त्योऽपोह्य‚तेऽन्य‚द‚नेनेति अपोह उच्य‚ते ॥
	{\color{gray}{\rmlatinfont\textsuperscript{§~\theparCount}}}
	\pend% ending standard par
      ‚{\tiny $_{lb}$}‚

	  
	  \pstart \leavevmode% starting standard par
	स्थिते चैव‚म्प‚रोन‚व‚ग‚ताभिप्रायः प्राह । \textbf{न‚न्वि}त्यादि । \textbf{पूर्व‚न्त}द्विप‚रीताकार‚{\tiny $_{lb}$}‚स‚मारोपी विप‚र्यासः प्र‚व‚र्त्त‚ते । त‚त‚स्त\textbf{द्विप‚र्यास‚पूर्व‚कोप्र‚तीत‚स्या}र्थ‚स्य लिङ्गान्नि‚{\tiny $_{lb}$}‚\textbf{श्च‚यो भ‚व‚तीति नाय‚न्निय‚मो} विप‚र्यास‚र‚हितेप्य‚नुमान‚स‚म्भ‚वात् । त‚दाह । \textbf{य‚थे}‚{\tiny $_{lb}$}‚‚{\tiny $_{lb}$}‚ \leavevmode\ledsidenote{\textenglish{126/s}}त्यादि । \textbf{अक‚स्मादि}त्य‚त‚र्क्कितोप‚स्थितात् । स‚ह‚सैव क्व‚चित्प्र‚देशे धूमा‚{\tiny $_{२}$}‚द\textbf{ग्निप्र‚ति‚{\tiny $_{lb}$}‚प‚त्तिः । न हि त‚त्रे}त्य‚क‚स्माद‚ग्निप्र‚तिप‚त्ता\textbf{व‚न‚ग्निस‚मारोपः स‚म्भाव्य‚ते । त‚दि}ति‚{\tiny $_{lb}$}‚ त‚स्मान्न \textbf{स‚र्व‚त्र व्य‚व‚च्छेदः क्रिय‚ते}ऽर्था\textbf{दित्य}नुमानेन ।
	{\color{gray}{\rmlatinfont\textsuperscript{§~\theparCount}}}
	\pend% ending standard par
      ‚{\tiny $_{lb}$}‚

	  
	  \pstart \leavevmode% starting standard par
	\textbf{उक्त}मित्या चा र्यः । \textbf{अत्र} हि व‚स्तुस्व‚रूप‚ग्राह‚क‚त्वेनानुमान‚स्य प्र‚वृत्तावुक्त‚{\tiny $_{lb}$}‚मुत्त‚रं । प्र‚त्य‚क्षेण \textbf{ध‚र्मिप्र‚तिप‚त्तौ} त‚द्ध‚र्माणां ध‚र्मिणः स‚काशाद\textbf{भेदात् स‚र्व‚प्र‚पित‚त्तेर}‚{\tiny $_{lb}$}‚नुमान‚वैफ‚ल्य‚मिति । ध‚र्मिणः स‚काशाद् ध‚{\tiny $_{३}$}‚र्म‚स्य भेदे वाभ्युप‚ग‚म्य‚माने ध‚र्मिणा स‚ह‚{\tiny $_{lb}$}‚ न तादात्म्य‚ल‚क्ष‚णः स‚म्ब‚न्धो भेदाभ्युप‚ग‚मात् । न चानित्य‚त्वादीनां ध‚र्मिस्व‚रूपा‚{\tiny $_{lb}$}‚दुत्प‚त्तिर्न चानित्य‚त्वादिभ्यो ध‚र्मिण उत्प‚त्तिः । \textbf{त‚त‚श्चास‚म्ब‚द्ध‚स्य} ध‚र्म‚स्य त‚त्र‚{\tiny $_{lb}$}‚ र्मिण्य\textbf{प्र‚तिप‚त्तिरि}त्येत\textbf{द‚प्युक्}तं । अहेतुफ‚ल‚भूत‚स्य त‚त्रानुमानास‚म्भ‚वादित्य‚त्रान्त‚रे ।‚{\tiny $_{lb}$}‚ य‚त एव\textbf{न्त‚स्मात् त‚त्रा}प्य‚क‚स्माद् धूम‚द‚र्श‚नाद‚ग्निप्र‚तिप‚त्तौ । \textbf{त‚द्द‚{\tiny $_{४}$}‚र्शिनः} प्र‚देश‚द‚र्शिनः‚{\tiny $_{lb}$}‚ पंस‚स्\textbf{त‚त्स्व‚भावानिश्च‚यो}ग्निम‚त्त्व‚स्व‚भावानिश्च‚योस्त्येव । स चानिय‚तः \textbf{कृतो‚{\tiny $_{lb}$}‚ विप‚र्यासादेवा}न‚ग्निम‚ता प्र‚देशेन तुल्य‚त्व‚ग्र‚ह‚णादेव ।
	{\color{gray}{\rmlatinfont\textsuperscript{§~\theparCount}}}
	\pend% ending standard par
      ‚{\tiny $_{lb}$}‚

	  
	  \pstart \leavevmode% starting standard par
	एत‚देवाह [।] \textbf{स} चेत्यादि ।
	{\color{gray}{\rmlatinfont\textsuperscript{§~\theparCount}}}
	\pend% ending standard par
      ‚{\tiny $_{lb}$}‚

	  
	  \pstart \leavevmode% starting standard par
	य‚द्वा त‚त्स्व‚भावानिश्च‚यः कुतः [।] किन्तु निश्च‚य एव स्याद‚विप‚र्यासात् ।‚{\tiny $_{lb}$}‚ भ‚व‚ति चानिश्च‚यः [।] त‚स्मान्न ध‚र्मिप्र‚तिप‚त्तिर्व‚स्तुस्व‚रूप‚ग्राहिणीत्य‚पोह‚विष‚या ।
	{\color{gray}{\rmlatinfont\textsuperscript{§~\theparCount}}}
	\pend% ending standard par
      ‚{\tiny $_{lb}$}‚

	  
	  \pstart \leavevmode% starting standard par
	किञ्च । \textbf{स चे}त्यादि । \textbf{स चे}‚{\tiny $_{५}$}‚ति पुरुष\textbf{स्त‚द्विविक्ते}नेति व‚ह्निविवेकेन \textbf{तं प्र‚देश}‚{\tiny $_{lb}$}‚म‚ग्निम‚न्त‚म‚पि \textbf{त‚द्विविक्तेन रूपेण निश्चिन्व‚न् बुद्ध्या} । किम्भूत‚या । \textbf{अग्निस‚त्ता‚{\tiny $_{lb}$}‚भाव‚नाविमुक्त‚या} । अग्निस‚त्तास‚म्भाव‚नार‚हित‚या [।] \textbf{क‚थ‚म‚विप‚र्य‚स्तो} नाम‚विप‚र्य‚स्त‚{\tiny $_{lb}$}‚ एवात‚स्मिँस्त‚द्ग्र‚हात् [।] त‚था हि [।] याव‚त्त‚त्र प्र‚देशे धूम‚न्न प‚श्य‚ति ताव‚{\tiny $_{lb}$}‚द‚न्येनाग्निर‚हितेन प्र‚देशेन स‚दृश‚न्त‚म‚प्य‚ध्य‚व‚स्य‚ति । य‚दि नाम क्व‚चिद‚नु‚{\tiny $_{६}$}‚भ‚व‚{\tiny $_{lb}$}‚ योगे स‚ति य‚दीह निश्च‚य‚स्त‚थापि त‚त्र संश‚येन भाव्यं संश‚य‚श्चोभ‚यांशाव‚ल‚म्बी ।‚{\tiny $_{lb}$}‚ स च प‚क्षे त‚द्विप‚रीतं संस्पृश‚त्येवातः संश‚य‚व्युदासेप्य‚न्य‚व्य‚व‚च्छेदः कुतो भ‚व‚त्येव‚{\tiny $_{lb}$}‚ लिङ्गेन । अव‚श्यं च लिङ्गादेवानुमेयं प्र‚तिप‚द्य‚मान‚स्तंत्रानुमेये विप‚र्यास‚संश‚याभ्यां‚{\tiny $_{lb}$}‚ युक्तो भ‚व‚ति । य‚स्मात् \textbf{त‚दाकार‚स‚मारोप‚संश‚य‚र‚हित‚श्च} । अन‚ग्न्याकार‚स‚मारोपे‚{\tiny $_{७}$}‚ण‚{\tiny $_{lb}$}‚ ‚{\tiny $_{lb}$}‚ \leavevmode\ledsidenote{\textenglish{127/s}}संश‚येन च र‚हित‚श्च पुरुष\textbf{स्त‚त्प्र‚तिप‚त्तौ} अग्निप्र‚तिप‚त्तौ \textbf{न लिङ्ग‚म‚नुस‚रेत्} । न प‚क्ष‚ध‚र्मं \leavevmode\ledsidenote{\textenglish{49a/PSVTa}}‚{\tiny $_{lb}$}‚ स‚माश्र‚येत् । त‚था \textbf{न त‚स्य} लिङ्ग‚स्या\textbf{न्व‚य‚व्य‚तिरेक‚योराद्रियेत} । याव‚तानुस‚रे‚{\tiny $_{lb}$}‚दाद्रिय‚ते च । त‚स्मात्त‚दाकार‚स‚मारोप‚संश‚य‚वान् प्र‚तिप‚त्ता लिङ्ग‚ब‚लेन स‚मारोप‚म‚प‚न‚{\tiny $_{lb}$}‚य‚ति क्व‚चित्संश‚य‚म‚तः स‚र्व‚त्र स्व‚व्याप‚क‚प्र‚तिप‚त्तिद्वारेण स‚मारोप‚व्य‚व‚च्छेदः क्रिय‚ते ।
	{\color{gray}{\rmlatinfont\textsuperscript{§~\theparCount}}}
	\pend% ending standard par
      ‚{\tiny $_{lb}$}‚

	  
	  \pstart \leavevmode% starting standard par
	\textbf{त‚स्मा‚{\tiny $_{१}$}‚द‚पोह‚विष‚य}म‚न्य‚व्य‚व‚च्छेद‚विष‚यं । इत्युक्तेन प्र‚कारेण न तु साक्षात् ।‚{\tiny $_{lb}$}‚ \textbf{अन्य‚थे}ति य‚दि स‚मारोप‚म‚न्त‚रेणानुमान‚स्य प्र‚तिप‚त्तिरिष्य‚ते । त‚दा केन‚चित् प्र‚माणेन‚{\tiny $_{lb}$}‚ \textbf{ध‚र्मिणः सिद्धौ किम‚सिद्ध‚न्त}स्य ध‚र्मिणो रूप\textbf{म‚तः} सिद्धात् स्व‚भावात् \textbf{प‚र‚म}न्य‚द‚स्ति‚{\tiny $_{lb}$}‚ य‚स्य प्र‚त्याय‚नाय लिङ्गं प्र‚व‚र्त्तेत ॥
	{\color{gray}{\rmlatinfont\textsuperscript{§~\theparCount}}}
	\pend% ending standard par
      ‚{\tiny $_{lb}$}‚

	  
	  \pstart \leavevmode% starting standard par
	न‚नु संश‚य‚विप‚र्य‚योत्पादे स‚ति भागः स्यात् । य‚दाह [।] न चेमाः क‚ल्प‚{\tiny $_{१}$}‚ना‚{\tiny $_{lb}$}‚ अप्र‚तिस‚म्विदिता एवोद‚य‚न्ते व्य‚य‚न्ते चेति । नापि त‚त्प्र‚तिप‚त्तौ लिङ्गानुस‚र‚णेन‚{\tiny $_{lb}$}‚ त‚दाकार‚स‚मारोप‚संश‚यः श‚क्य‚ते क‚ल्प‚यितुन्निश्च‚यानुत्प‚त्तिप‚क्षेपि [।] साध्य‚निश्च‚{\tiny $_{lb}$}‚यार्थं लिङ्गानुस‚र‚ण‚स्य स‚म्भ‚वात् ।
	{\color{gray}{\rmlatinfont\textsuperscript{§~\theparCount}}}
	\pend% ending standard par
      ‚{\tiny $_{lb}$}‚
	  \bigskip
	  \begingroup
	
	    
	    \stanza[\smallbreak]
	  {\normalfontlatin\large ``\qquad}अन्य‚था ध‚र्मिणः सिद्धाव‚सिद्धं किम‚तः प‚र‚म् [।]{\normalfontlatin\large\qquad{}"}\&[\smallbreak]
	  
	  
	  
	  \endgroup
	‚{\tiny $_{lb}$}‚

	  
	  \pstart \leavevmode% starting standard par
	इत्य‚नेनापि ध‚र्मिविष‚य‚स्य ज्ञान‚स्य क‚ल्पित‚विष‚य‚त्वं स्यात् । न तु ध‚र्मिणि‚{\tiny $_{lb}$}‚ स‚मारोपः सिध्य‚तीति क‚थ‚{\tiny $_{३}$}‚न्त‚द‚नुमान‚स्यार्थात् स‚मारोप‚व्य‚व‚च्छेद इति ।
	{\color{gray}{\rmlatinfont\textsuperscript{§~\theparCount}}}
	\pend% ending standard par
      ‚{\tiny $_{lb}$}‚

	  
	  \pstart \leavevmode% starting standard par
	स‚त्त्यं [।] किन्त्व‚य‚म‚भिप्रायः [।] न याव‚त् प‚क्ष‚ध‚र्म‚त्व‚निश्च‚यो न ताव‚द‚नुमान‚{\tiny $_{lb}$}‚स्योत्थान‚म‚स्ति हेतोर‚सिद्ध‚त्वात् । अग्निम‚ति च प्र‚देशे प्र‚देश‚मात्र‚स‚म्ब‚न्धित‚या‚{\tiny $_{lb}$}‚ यो धूम‚स्य निश्च‚योऽय‚मेव विप‚र्यासोऽन्य‚थाव‚धार‚णात् । न ह्य‚त्राग्निर्नास्तीत्येवं‚{\tiny $_{lb}$}‚ रूप एव विप‚र्यासः साग्नेः प्र‚देशाद‚य‚म‚न्य इत्येवं रूप‚स्यापि‚{\tiny $_{४}$}‚\textbf{र्षि}\edtext{}{\lemma{स्यापि}\Bfootnote{?}}प्र‚त्य‚य‚स्य विप‚र्या‚{\tiny $_{lb}$}‚सात्तेनात्राप्य‚नुमाने साग्निर‚यं प्र‚देशो न प्र‚देश‚मात्र‚मिति स‚मारोप‚निषेधोस्त्येव ।
	{\color{gray}{\rmlatinfont\textsuperscript{§~\theparCount}}}
	\pend% ending standard par
      ‚{\tiny $_{lb}$}‚

	  
	  \pstart \leavevmode% starting standard par
	भ‚व‚तु ताव‚त् स‚मारोपे त‚द्व्य‚व‚च्छेदाय स्व‚विष‚ये प्र‚व‚र्त्त‚मानं लिङ्ग‚म‚न्यापोह‚कृत् ।‚{\tiny $_{lb}$}‚ य‚त्पुन‚रिदं प्र‚त्य‚क्ष‚पृष्ठ‚भावि विक‚ल्प‚विज्ञान‚न्त‚द‚स‚ति स‚मारोपे भ‚व‚त् क‚थ‚म‚न्यापोह‚{\tiny $_{lb}$}‚कृत् । अव‚श्यं च त‚द‚न्यापोह‚विष‚य‚मेष्ट‚व्यं सामान्य‚विष‚य‚त्वाद‚न्यापो‚{\tiny $_{५}$}‚ह‚ल‚क्ष‚ण‚त्वाच्च‚{\tiny $_{lb}$}‚ सामान्य‚स्य भ‚व‚तान्द‚र्श‚नेनेति [।]
	{\color{gray}{\rmlatinfont\textsuperscript{§~\theparCount}}}
	\pend% ending standard par
      ‚{\tiny $_{lb}$}‚‚{\tiny $_{lb}$}‚\textsuperscript{\textenglish{128/s}}

	  
	  \pstart \leavevmode% starting standard par
	अत आह । \textbf{क्व‚चिदि}त्यादि । क‚स्मिँश्चिद्रूपादौ \textbf{दृष्टे}पि प्र‚त्य‚क्षेण । त‚त्पृष्ठ‚{\tiny $_{lb}$}‚भावि य‚ज्\textbf{ज्ञानं सामान्यार्थं सामान्य‚विष‚य}म‚त एव विक‚ल्प‚कं । \textbf{अस‚मारोपितः}‚{\tiny $_{lb}$}‚ अन्या\textbf{ङ्शः} प्र‚तियोग्याकारो य‚स्मिन् विष‚ये स त‚था त‚त्र प्र‚व‚र्त्त‚मान‚न्त‚द‚पि \textbf{त‚न्मात्रा‚{\tiny $_{lb}$}‚पोह‚गोच‚रं} । तेनाय‚म‚र्थो भ‚व‚ति [।] स‚मारोप‚र‚हितं स्व‚ल‚क्ष‚{\tiny $_{६}$}‚णं स्वाकार‚भेदेन‚{\tiny $_{lb}$}‚ गृह्ण‚न् विक‚ल्प‚कं ज्ञान‚म्भ्रान्त‚त्वात् त‚त्स‚मारोप‚र‚हित‚बाह्याध्य‚वासाय‚क‚मेव न तु‚{\tiny $_{lb}$}‚ बाह्य‚स्व‚रूप‚ग्राह‚क‚म् [।] अत‚स्त‚न्मात्र‚मेव निय‚त‚बाह्याव‚साय एवान्य‚स्य स‚मारोप‚{\tiny $_{lb}$}‚स्यापोह‚गोच‚र‚म्विक‚ल्प‚कं ज्ञानं ।
	{\color{gray}{\rmlatinfont\textsuperscript{§~\theparCount}}}
	\pend% ending standard par
      ‚{\tiny $_{lb}$}‚

	  
	  \pstart \leavevmode% starting standard par
	य‚द्वा अस‚मारोपित‚श्चासाव‚न्यांश‚श्च त‚स्मिन् स‚ति विक‚ल्प‚कं ज्ञानं प्र‚व‚र्त्त‚मान‚{\tiny $_{lb}$}‚\leavevmode\ledsidenote{\textenglish{49b/PSVTa}} न्त‚न्मात्रापोह‚गोच‚रं । योसाव‚स‚मारोपितोन्याङ्श‚स्त‚न्मात्र‚व्य‚व‚{\tiny $_{७}$}‚च्छेद‚विष‚य‚म्भ‚व‚ति ।
	{\color{gray}{\rmlatinfont\textsuperscript{§~\theparCount}}}
	\pend% ending standard par
      ‚{\tiny $_{lb}$}‚

	  
	  \pstart \leavevmode% starting standard par
	एत‚दुक्त‚म्भ‚व‚ति । य‚त्रापि स‚मारोपः प्र‚वृत्तो न त‚त्रापि स‚मारोप‚निषेधः श‚ब्द‚{\tiny $_{lb}$}‚लिङ्गाभ्यां प्र‚तिपाद्य‚ते स‚म्ब‚न्धाभावाद‚त एवायं न क्रिय‚तेऽहेतुत्वाच्च नाश‚स्य । केव‚लं‚{\tiny $_{lb}$}‚ पूर्व‚क‚स्य स‚मारोप‚स्य स्व‚र‚स‚निरोधात् । श‚ब्द‚लिङ्गाभ्याम‚नित्यादिनिश्च‚ये स‚त्य‚न्य‚स्य‚{\tiny $_{lb}$}‚ स‚मारोप‚स्यानुत्पादे स‚ति स‚मारोप‚निषेधः कृतो भ‚व‚ति । त‚था प्र‚त्य‚क्ष‚दृष्टेप्य‚न्य‚{\tiny $_{lb}$}‚विक‚{\tiny $_{१}$}‚ल्प‚स्य स‚मारोप‚व्य‚व‚च्छेदः केन वार्य‚ते । तेन न पूर्व‚त्रान्यादृश एव स‚मारोप‚{\tiny $_{lb}$}‚व्य‚व‚च्छेद उक्तोऽधुनान्यादृश एवोच्य‚त इति भिन्न‚वाक्य‚ता ।
	{\color{gray}{\rmlatinfont\textsuperscript{§~\theparCount}}}
	\pend% ending standard par
      ‚{\tiny $_{lb}$}‚

	  
	  \pstart \leavevmode% starting standard par
	तेनेदं च निर‚स्तं [।]
	{\color{gray}{\rmlatinfont\textsuperscript{§~\theparCount}}}
	\pend% ending standard par
      ‚{\tiny $_{lb}$}‚
	  \bigskip
	  \begingroup
	
	    
	    \stanza[\smallbreak]
	  {\normalfontlatin\large ``\qquad}प्राग‚गौरिति विज्ञानं गोश‚ब्द‚श्राविणो भ‚वेद् [।]&‚{\tiny $_{lb}$}‚येनागोव्य‚व‚च्छेदाय प्र‚वृत्तो गौरिति ध्व‚निरि\edtext{}{\edlabel{pvsvt_128-1}\label{pvsvt_128-1}\lemma{निरि}\Bfootnote{\href{http://sarit.indology.info/?cref=\%C5\%9Bv-apohav\%C4\%81da.92}{ Ślokavārtika, Apohavāda. 92. }}}ति ।{\normalfontlatin\large\qquad{}"}\&[\smallbreak]
	  
	  
	  
	  \endgroup
	‚{\tiny $_{lb}$}‚

	  
	  \pstart \leavevmode% starting standard par
	य‚दा श‚ब्द‚लिंग‚योः स्व‚विष‚ये निश्च‚य‚ज‚न‚नेनान्य‚निषेधे व्यापारः क‚ल्प्य‚ते ।‚{\tiny $_{lb}$}‚ त‚दा विधिरूपेणैव प्र‚वृत्ति‚{\tiny $_{२}$}‚रिति सिद्धं । तेन [।]‚{\tiny $_{lb}$}‚ 
	    \pend% close preceding par
	  
	    
	    \stanza[\smallbreak]
	  {\normalfontlatin\large ``\qquad}य‚दि गौरित्य‚यं श‚ब्दः स‚म‚र्थोन्य‚निव‚र्त्त‚ने ।&‚{\tiny $_{lb}$}‚ज‚न‚को ग‚वि गोबुद्धेर्मृग्य‚ताम‚प‚रो ध्व‚निरिति ॥{\normalfontlatin\large\qquad{}"}\&[\smallbreak]
	  
	  
	  
	    \pstart  \leavevmode% new par for following
	    \hphantom{.}
	  ‚{\tiny $_{lb}$}‚ निर‚स्तं ।
	{\color{gray}{\rmlatinfont\textsuperscript{§~\theparCount}}}
	\pend% ending standard par
      ‚{\tiny $_{lb}$}‚

	  
	  \pstart \leavevmode% starting standard par
	त‚था य‚द‚प्युच्य‚ते ।
	{\color{gray}{\rmlatinfont\textsuperscript{§~\theparCount}}}
	\pend% ending standard par
      ‚{\tiny $_{lb}$}‚
	  \bigskip
	  \begingroup
	
	    
	    \stanza[\smallbreak]
	  {\normalfontlatin\large ``\qquad}य‚द्य‚प्पोह विनिर्मुक्ते न वृत्तिः श‚ब्द‚लिंग‚योः ।&‚{\tiny $_{lb}$}‚युक्ता त‚थापि बोध‚स्तु ज्ञातुर्व‚स्त्व‚व‚ल‚म्ब‚त \edtext{}{\edlabel{pvsvt_128-1b}\label{pvsvt_128-1b}\lemma{त}\Bfootnote{\href{http://sarit.indology.info/?cref=\%C5\%9Bv-apohav\%C4\%81da.92}{Ślokavārtika, Apohavāda. 92.}}}इति ॥{\normalfontlatin\large\qquad{}"}\&[\smallbreak]
	  
	  
	  
	  \endgroup
	‚{\tiny $_{lb}$}‚‚{\tiny $_{lb}$}‚‚{\tiny $_{lb}$}‚\textsuperscript{\textenglish{129/s}}

	  
	  \pstart \leavevmode% starting standard par
	त‚दिष्ट‚मेवास्माकं । श‚ब्द‚लिंग‚प्र‚तिपादित‚स्य चार्थ‚स्यान्य‚निषेधे व्यापारो न‚{\tiny $_{lb}$}‚ श‚ब्द‚लिङ्ग‚योः ।
	{\color{gray}{\rmlatinfont\textsuperscript{§~\theparCount}}}
	\pend% ending standard par
      ‚{\tiny $_{lb}$}‚

	  
	  \pstart \leavevmode% starting standard par
	तेन य‚दुच्य‚ते कु मा ‚{\tiny $_{३}$}‚ रि ले न [।] निषेध‚स्य निरूप‚त्वाद् भेदाभावाच्च न‚{\tiny $_{lb}$}‚ लिङ्ग‚लिङ्गिभावः नापि श‚ब्द‚वाच्य‚त्वं ।
	{\color{gray}{\rmlatinfont\textsuperscript{§~\theparCount}}}
	\pend% ending standard par
      ‚{\tiny $_{lb}$}‚
	  \bigskip
	  \begingroup
	
	    
	    \stanza[\smallbreak]
	  {\normalfontlatin\large ``\qquad}न ग‚म्य‚ग‚म‚क‚त्वं स्याद‚व‚स्तुत्वाद‚पोह‚योः ।&‚{\tiny $_{lb}$}‚भ‚व‚त्प‚क्षे य‚था लोके ख‚पुष्प‚श‚श‚शृङ्ग‚योः ।&‚{\tiny $_{lb}$}‚निषेध‚मात्र‚रूपं च श‚ब्दार्थो य‚दि क‚ल्प्य‚ते ।&‚{\tiny $_{lb}$}‚अभाव‚श‚ब्द‚वाच्या स्याच्छून्य‚तान्य‚प्र‚कारिका । \href{http://sarit.indology.info/?cref=\%C5\%9Bv-apohav\%C4\%81da.36}{३६}{\normalfontlatin\large\qquad{}"}\&[\smallbreak]
	  
	  
	  ‚{\tiny $_{lb}$}‚ 
	    
	    \stanza[\smallbreak]
	  {\normalfontlatin\large ``\qquad}भिन्न‚सामान्य‚व‚च‚ना विशेष‚व‚च‚नाश्र‚ये ।&‚{\tiny $_{lb}$}‚स‚र्वे भ‚वेयुः प‚र्याया य‚द्य‚पोह‚{\tiny $_{४}$}‚स्य वाच्य‚ता ॥ \href{http://sarit.indology.info/?cref=\%C5\%9Bv-apohav\%C4\%81da.42}{४२}\edtext{\textsuperscript{*}}{\edlabel{pvsvt_129-2}\label{pvsvt_129-2}\lemma{*}\Bfootnote{\href{http://sarit.indology.info/?cref=\%C5\%9Bv-apohav\%C4\%81da}{ Ślokavārtika. Apohavāda }}}{\normalfontlatin\large\qquad{}"}\&[\smallbreak]
	  
	  
	  
	  \endgroup
	‚{\tiny $_{lb}$}‚

	  
	  \pstart \leavevmode% starting standard par
	त‚था य‚दि चापोह्य‚भेदेनापोह‚स्य भेद‚स्त‚दौप‚चारिकः स्यात् । य‚स्य चापोह‚स्य‚{\tiny $_{lb}$}‚ नीरूप‚त्वे ध‚र्मिभेदेन न भेदः । क‚थ‚न्त‚स्य ब‚हिर्भूतैर‚पोहैर्भेदः क्रिय‚ते । त‚दाह\edtext{}{\edlabel{pvsvt_129-2b}\label{pvsvt_129-2b}\lemma{दाह}\Bfootnote{\href{http://sarit.indology.info/?cref=\%C5\%9Bv-apohav\%C4\%81da}{Ślokavārtika, Apohavāda.}}} ।
	{\color{gray}{\rmlatinfont\textsuperscript{§~\theparCount}}}
	\pend% ending standard par
      ‚{\tiny $_{lb}$}‚
	  \bigskip
	  \begingroup
	
	    
	    \stanza[\smallbreak]
	  {\normalfontlatin\large ``\qquad}न‚नु चापोह‚भेदेन भेदोपोह‚स्य सेत्स्य‚ति ।&‚{\tiny $_{lb}$}‚न विशेषः स्व‚त‚स्त‚स्य प‚र‚त‚श्चौप‚चारिकः ॥ \href{http://sarit.indology.info/?cref=\%C5\%9Bv-apohav\%C4\%81da.47}{४७}{\normalfontlatin\large\qquad{}"}\&[\smallbreak]
	  
	  
	  ‚{\tiny $_{lb}$}‚ 
	    
	    \stanza[\smallbreak]
	  {\normalfontlatin\large ``\qquad}संस‚र्गिणोपि चाधारा य‚न्न भिन्द‚न्ति भाव‚तः ।&‚{\tiny $_{lb}$}‚अपोह्यैः स व‚हिः‚{\tiny $_{५}$}‚संस्थैर्भिद्येतेत्य‚तिक‚ल्प‚नेति \href{http://sarit.indology.info/?cref=\%C5\%9Bv-apohav\%C4\%81da.52}{॥ ५२ ॥}{\normalfontlatin\large\qquad{}"}\&[\smallbreak]
	  
	  
	  
	  \endgroup
	‚{\tiny $_{lb}$}‚
	    
	    \stanza[\smallbreak]
	  निर‚स्तं । व्य‚व‚च्छेद‚मात्र‚स्य श‚ब्दाद्य‚विष‚य‚त्वात् ।\&[\smallbreak]
	  
	  
	  ‚{\tiny $_{lb}$}‚

	  
	  \pstart \leavevmode% starting standard par
	\textbf{य‚दि}त्यादिना श्लोकं व्याच‚ष्टे । आदिश‚ब्दाच्छ‚ब्दादिप‚रिग्र‚हः । नास्य लिंग‚{\tiny $_{lb}$}‚म‚स्तीत्यि\textbf{लिङ्ग‚रूपादिक‚मेत‚दिति निश्च‚य‚ज्ञानं । अस‚ति स‚मारोपे} भ‚व‚ति । \textbf{न हि}‚{\tiny $_{lb}$}‚ प्र‚त्य‚क्ष\textbf{दृष्टे रूपादौ} त‚दानीम्विप‚रीताकार‚स‚मारोपोस्ति । \textbf{त‚त्क‚थं व्य‚व‚च्छेद‚विष‚य‚{\tiny $_{lb}$}‚म्भ‚व‚ति} ।
	{\color{gray}{\rmlatinfont\textsuperscript{§~\theparCount}}}
	\pend% ending standard par
      ‚{\tiny $_{lb}$}‚

	  
	  \pstart \leavevmode% starting standard par
	इय‚ता श्लोक‚{\tiny $_{६}$}‚स्य पूर्वार्द्धो व्याख्यातः । उत्त‚रार्द्धं व्याख्यातुमाह । \textbf{स‚मा‚{\tiny $_{lb}$}‚रोप‚विष‚ये त‚स्य} निश्च‚य‚ज्ञान‚स्या\textbf{भावात्} । त‚द्व्य‚व‚च्छेद‚विष‚य‚म्भ‚व‚तीति प्र‚कृतेन‚{\tiny $_{lb}$}‚ स‚म्ब‚न्धः । एत‚देवाह । \textbf{य‚त्र} भेदेस्य पुंसः \textbf{स‚मारोपो न त‚त्र} भेदे स‚मारोप‚विष‚ये‚{\tiny $_{lb}$}‚ \textbf{निश्च‚यो भ‚व‚त्य}स्थिरो निरात्म‚क इति वा ॥
	{\color{gray}{\rmlatinfont\textsuperscript{§~\theparCount}}}
	\pend% ending standard par
      ‚{\tiny $_{lb}$}‚\textsuperscript{\textenglish{130/s}}

	  
	  \pstart \leavevmode% starting standard par
	\leavevmode\ledsidenote{\textenglish{50a/PSVTa}} किं कार‚णं [।] \textbf{निश्च‚यारोप‚म‚न‚सोर्बाध्य‚बाध‚क‚भावः} । निश्च‚य‚ज्ञा‚{\tiny $_{७}$}‚न‚स्य‚{\tiny $_{lb}$}‚ त‚द्विप‚रीत‚स‚मारोप‚ज्ञान‚स्य च बाध्य‚बाध‚क\textbf{भाव‚तः} । बाध्य‚बाध‚क‚भाव‚मेव साध‚य‚न्नाह ।‚{\tiny $_{lb}$}‚ \textbf{न ही}ति \textbf{स‚र्व‚तः} स‚जातीयाद्विजातीयाच्च \textbf{भिन्नो दृष्टोपि भाव‚स्त‚थैवे}ति य‚थादृष्टेन‚{\tiny $_{lb}$}‚ स‚र्वेणाकारेण \textbf{प्र‚त्य‚भिज्ञाय‚ते} । निश्चीय‚ते न हीति स‚म्ब‚न्धः । किं कार‚णं [।]‚{\tiny $_{lb}$}‚ \textbf{क्व‚चिद् भेदे} क्ष‚णिक‚त्वादिके \textbf{व्य‚व‚धान‚स‚म्भ‚वा}त् । भ्रान्तिनिमित्त‚ग‚तः \textbf{य‚था शुक्तेः}‚{\tiny $_{lb}$}‚ स‚र्व‚तो व्यावृत्ताया द‚र्श‚नेपि शु‚{\tiny $_{१}$}‚क्तिकादित्वे र‚ज‚त‚साध‚र्म्य‚स्य भ्रान्तिनिमित्त‚स्य‚{\tiny $_{lb}$}‚ स‚म्भ‚वान्न निश्च‚यः । \textbf{य‚त्र} त्वाकारे \textbf{भ्रान्तिनिमि}त्तं नास्ति \textbf{त‚त्रैवास्य प्र‚तिप‚त्तु}र‚नुभ‚वो‚{\tiny $_{lb}$}‚ त्त‚र‚काल‚भावी \textbf{स्मार्त्तो निश्च‚यो भ‚व‚ति} । त\textbf{द्द‚र्श‚नाविशेषे}पि स‚र्व‚स्वाकारेषु प्र‚त्य‚क्ष‚{\tiny $_{lb}$}‚स्याविशेषेपि स्मार्त्त इति स्मृतिरूपः ।
	{\color{gray}{\rmlatinfont\textsuperscript{§~\theparCount}}}
	\pend% ending standard par
      ‚{\tiny $_{lb}$}‚

	  
	  \pstart \leavevmode% starting standard par
	न‚नु त‚दित्युल्लेखेनानुत्प‚त्तेः क‚थं स्मृतिरूपः [।]
	{\color{gray}{\rmlatinfont\textsuperscript{§~\theparCount}}}
	\pend% ending standard par
      ‚{\tiny $_{lb}$}‚

	  
	  \pstart \leavevmode% starting standard par
	स‚त्यं [।] निर्विक‚ल्प‚क‚विष‚य‚स्य स्वाका‚{\tiny $_{२}$}‚रेणैकीकृत्य विष‚यीक‚र‚णात् स्मृति‚{\tiny $_{lb}$}‚रूप उच्य‚ते । य‚त‚श्च प्र‚त्य‚क्षाविशेषेपि स‚मारोप‚र‚हित एव विष‚ये निश्च‚यो भ‚व‚ति‚{\tiny $_{lb}$}‚त‚स्मात् स‚मारोप‚निश्च‚य‚योर्बाध्य‚बाध‚क‚भावो ग‚म्य‚ते । त‚तो \textbf{बाध्य‚बाध‚क‚भावात्}‚{\tiny $_{lb}$}‚ कार‚णात् \textbf{स‚मारोप‚विवेके} स‚मारोप‚विर‚ह‚निश्च‚य‚स्या\textbf{स्य प्र‚वृत्तिरिति ग‚म्य‚ते} ।
	{\color{gray}{\rmlatinfont\textsuperscript{§~\theparCount}}}
	\pend% ending standard par
      ‚{\tiny $_{lb}$}‚

	  
	  \pstart \leavevmode% starting standard par
	भ‚व‚तु नाम स‚मारोप‚विवेके प्र‚वृत्तिस्त‚थापि नान्या‚{\tiny $_{३}$}‚पोह‚विष‚य‚त्व‚म्विधिरूपेण‚{\tiny $_{lb}$}‚ प्र‚वृत्तेरित्याह । \textbf{त‚द्विवेक एवान्यापोहः} स‚मारोप‚विवेक एव चान्यापोहः । \textbf{त‚स्मात्}‚{\tiny $_{lb}$}‚ स‚मारोप‚र‚हिते वृत्तिव‚शात् \textbf{त‚द‚पी}ति न केव‚ल‚म‚नित्यः श‚ब्द इति निश्च‚य‚ज्ञानं‚{\tiny $_{lb}$}‚ पूर्वोक्तेन न्यायेन \textbf{त‚न्मात्रापोह‚गोच‚र‚न्त‚द‚पि} प्र‚त्य‚क्ष‚पृष्ठ‚भाविनिश्चिय‚ज्ञान‚म‚पि‚{\tiny $_{lb}$}‚ \textbf{त‚न्मात्रापोह‚गोच‚रं न व‚स्तुस्व‚भाव‚निश्चाय‚कं स्व‚रूपे‚{\tiny $_{४}$}‚ण} [।] किङ्कार‚णं [।]‚{\tiny $_{lb}$}‚ \textbf{त‚था हि} क‚स्य‚चिदाकार‚स्य रूप‚त्वादे\textbf{र्निश्च‚ये}प्य‚न्य‚स्य क्ष‚णिक‚त्वाद्याकार‚स्या\textbf{प्र‚तिप‚त्ति‚{\tiny $_{lb}$}‚द‚र्श}नात् ।
	{\color{gray}{\rmlatinfont\textsuperscript{§~\theparCount}}}
	\pend% ending standard par
      ‚{\tiny $_{lb}$}‚‚{\tiny $_{lb}$}‚\textsuperscript{\textenglish{131/s}}

	  
	  \pstart \leavevmode% starting standard par
	य‚दि तु प्र‚त्य‚क्ष‚पृष्ठ‚भाविना निष्च‚येन व‚स्तुस्व‚भाव‚स्य निश्च‚यः क्रिय‚ते त‚दा‚{\tiny $_{lb}$}‚ \textbf{त‚त्स्व‚भाव‚निश्च‚ये च} निरंश‚त्वाद् व‚स्तुन‚स्त\textbf{स्यायो}गाद‚न्य‚स्याकारान्त‚र‚स्यानिश्च\textbf{या‚{\tiny $_{lb}$}‚योगात्} ॥
	{\color{gray}{\rmlatinfont\textsuperscript{§~\theparCount}}}
	\pend% ending standard par
      ‚{\tiny $_{lb}$}‚

	  
	  \pstart \leavevmode% starting standard par
	य‚त‚श्च व‚स्त्व‚ध्य‚व‚सायेनैव निश्च‚य‚स्य प्र‚वृत्तिः श‚ब्द‚{\tiny $_{५}$}‚स्य वा न व‚स्तु‚{\tiny $_{lb}$}‚स्व‚रूप‚ग्राह‚क‚त्वेन । त‚स्माद् \textbf{याव}न्त एक‚स्\textbf{यांश‚स‚मारोपा} रूपान्त‚र‚स‚मारोपाः प्र‚वृत्ता‚{\tiny $_{lb}$}‚ अप्र‚वृत्ताश्च \textbf{त‚न्निरासे} स‚मारोप‚निरासार्थ\textbf{न्ताव‚न्त एव निश्च‚याः श‚ब्दा}श्च ताव‚{\tiny $_{lb}$}‚न्त एव स्व‚विष‚ये प्र‚व‚र्त्त‚न्ते । \textbf{तेन} कार‚णेन \textbf{ते} निश्च‚याः श‚ब्दाश्च \textbf{भिन्न‚गोच‚रा}‚{\tiny $_{lb}$}‚ भिन्न‚विष‚याः । स्व‚स्व‚हेतुतोध्य‚व‚सित‚स्व‚स्वाकाराभिन्न‚बाह्य‚विष‚य‚त्वात्तेनैत‚द् [।]
	{\color{gray}{\rmlatinfont\textsuperscript{§~\theparCount}}}
	\pend% ending standard par
      ‚{\tiny $_{lb}$}‚
	  \bigskip
	  \begingroup
	
	    
	    \stanza[\smallbreak]
	  {\normalfontlatin\large ``\qquad}बुद्ध्यारोपित‚बुद्धिस्थो ना‚{\tiny $_{६}$}‚र्थ‚बुद्ध्य‚न्त‚रानुगः ।&‚{\tiny $_{lb}$}‚नाभिप्रेतार्थ‚कारी च सोपि वाच्यो न त‚त्त्व‚तः ।&‚{\tiny $_{lb}$}‚प्र‚तिभापि च श‚ब्दार्थो बाह्यार्थ‚विष‚या य‚दि ।&‚{\tiny $_{lb}$}‚एकात्म‚निय‚ते बाह्ये विचित्राः प्र‚तिभाः क‚थं ।&‚{\tiny $_{lb}$}‚अथ निर्विष‚या एता वास‚नामात्र‚भाव‚तः ।&‚{\tiny $_{lb}$}‚प्र‚तिप‚त्तिः प्र‚वृत्तिश्च बाह्यार्थेषु क‚थ‚म्भ‚वेत् ।&‚{\tiny $_{lb}$}‚स्वाँशे बाह्याधिमोक्षेण प्र‚वृत्तिश्चेत्स‚दा म‚ता ।&‚{\tiny $_{lb}$}‚श‚ब्दार्थोऽतात्त्विकः प्राप्त‚स्त‚था भ्रान्त्या प्र‚व‚र्त्त‚नादिति‚{\tiny $_{७}$}‚ [।]\edtext{\textsuperscript{*}}{\edlabel{pvsvt_131-1}\label{pvsvt_131-1}\lemma{*}\Bfootnote{\href{http://sarit.indology.info/?cref=\%C5\%9Bv}{ Ślokavārtika. }}}{\normalfontlatin\large\qquad{}"}\&[\smallbreak]
	  
	  
	  
	  \endgroup
	\textsuperscript{\textenglish{50b/PSVTa}}‚{\tiny $_{lb}$}‚

	  
	  \pstart \leavevmode% starting standard par
	निर‚स्तं । क‚ल्पित‚विष‚य‚त्वेनेष्ट‚त्वाद् विक‚ल्प‚स्य ॥
	{\color{gray}{\rmlatinfont\textsuperscript{§~\theparCount}}}
	\pend% ending standard par
      ‚{\tiny $_{lb}$}‚

	  
	  \pstart \leavevmode% starting standard par
	\textbf{अन्य‚थे}ति बुद्धिश‚ब्दाभ्याम्व‚स्तुस्व‚रूप‚ग्र‚ह‚णे । \textbf{एकेन श‚ब्देन व्याप्ते} स‚र्वा‚{\tiny $_{lb}$}‚कारेण \textbf{विष‚यीकृ}ते । \textbf{एक‚त्रैक}स्मिन् \textbf{बुद्ध्या} वा निश्च‚यात्मिक‚या व्याप्तेनान्य‚{\tiny $_{lb}$}‚विष‚यः । \textbf{अन्य}श्चासावाकारो \textbf{विष}य‚श्चेत्य‚न्य‚विष‚यः । त‚स्य व‚स्तुनो नाप‚र आकारो‚{\tiny $_{lb}$}‚ विष‚य‚भूतो विद्य‚ते प्र‚त्याय्यः । अथ‚वा त‚द्व‚स्तुप्र‚त्याय‚क‚स्यान्य‚स्य श‚ब्द‚स्य ज्ञान‚स्य‚{\tiny $_{lb}$}‚ ‚{\tiny $_{lb}$}‚ ‚{\tiny $_{lb}$}‚ \leavevmode\ledsidenote{\textenglish{132/s}}वा न विष‚यः ।‚{\tiny $_{१}$}‚ \textbf{इति} हेतोः श‚ब्दानां प्र‚तीते विष‚ये प‚श्चात् प्र‚व‚र्त्त‚मानानां \textbf{प‚र्याय‚ता‚{\tiny $_{lb}$}‚ स्या}त् । वृक्ष‚पाद‚पादिश‚ब्द‚व‚त् । । म‚धुरो र‚सः स्निग्धो गुरुः शीत इत्येव‚मादि‚{\tiny $_{lb}$}‚भिन्न‚विष‚यानुपातिन्याश्च बुद्धेः प्र‚वृत्तिर्न स्यादित्येक‚विष‚य‚त्व‚प्र‚संगः ॥
	{\color{gray}{\rmlatinfont\textsuperscript{§~\theparCount}}}
	\pend% ending standard par
      ‚{\tiny $_{lb}$}‚

	  
	  \pstart \leavevmode% starting standard par
	भिन्नं ध‚र्म‚ध‚र्मिभावं पार‚मार्थिक‚न्दूष‚यितुमुप‚न्य‚स्य‚न्नाह । \textbf{य‚स्यापी}त्यादि ।‚{\tiny $_{lb}$}‚ य‚स्यापि वै शे षि क स्य \textbf{प‚र‚स्प‚र‚माश्र‚याच्च} भिन्न‚त्वा\textbf{न्नाना उपा}ध‚यो विशे‚{\tiny $_{२}$}‚ष‚णानि‚{\tiny $_{lb}$}‚ द्र‚व्य‚त्वाद‚यो य‚स्यार्थ‚स्य घ‚टादेः स नानोपाधिस्त‚स्य त‚त एवोपाधिभेदाद् \textbf{भेदि‚{\tiny $_{lb}$}‚नोर्थ‚स्य} विधिनैव बुद्धि\textbf{ग्राहिका} निश्च‚यात्मिका \textbf{धीः} सा च प्र‚त्युपाधि भिन्ना ॥‚{\tiny $_{lb}$}‚ धिय‚श्च विष‚य‚भेद‚न्द‚र्श‚य‚ता श‚ब्दानाम‚प्य‚र्थ‚तो द‚र्शित एव ।
	{\color{gray}{\rmlatinfont\textsuperscript{§~\theparCount}}}
	\pend% ending standard par
      ‚{\tiny $_{lb}$}‚

	  
	  \pstart \leavevmode% starting standard par
	त‚द्व्याच‚ष्टे \textbf{योपी}त्यादिना [।] \textbf{उपाध‚यो} द्र‚व्य‚त्वाद‚यः \textbf{प‚र‚स्प‚र‚म}न्योन्य\textbf{म्भिन्ना‚{\tiny $_{lb}$}‚ आश्र‚याच्चे}त्युपाधिम‚तो भिन्नाः । \textbf{त‚न्निब‚न्ध‚ना} भिन्नोपाधिनिब‚न्ध‚नाः‚{\tiny $_{३}$}‚ श्रुति‚{\tiny $_{lb}$}‚ग्र‚ह‚ण‚मुप‚ल‚क्ष‚ण‚मेवं बुद्ध‚योपि । \textbf{त‚दाधारे} स्थित्युपाधीनामाधारेषु । \textbf{त‚त्रैव} चेत्युपा‚{\tiny $_{lb}$}‚धिष्वेव । \textbf{व‚र्त्त‚न्ते} वाच‚क‚त‚या प्र‚व‚र्त्त‚न्ते । \textbf{त‚दि}ति त‚स्माद् । \textbf{अय}मिति श‚ब्द‚ज्ञाना‚{\tiny $_{lb}$}‚न्त‚राणां प‚र्याय‚ताल‚क्ष‚णोऽ\textbf{प्र‚संगः} ।
	{\color{gray}{\rmlatinfont\textsuperscript{§~\theparCount}}}
	\pend% ending standard par
      ‚{\tiny $_{lb}$}‚

	  
	  \pstart \leavevmode% starting standard par
	उत्त‚र‚माह । \textbf{त‚स्यापी}त्यादि । \textbf{नाना}प्र‚काराणामु\textbf{पाधीनामुप‚कार‚स्याङ्गं} कार‚णं‚{\tiny $_{lb}$}‚ याः श‚क्त‚यः । ताभ्योऽ\textbf{भिन्नात्म}न उपाधिम‚त \textbf{एकेन} निश्च‚य‚ज्ञानेन \textbf{ग्र‚हे} निश्च‚ये‚{\tiny $_{४}$}‚‚{\tiny $_{lb}$}‚ \textbf{स‚र्वात्म‚ना} कृते स‚ति । \textbf{उप‚कार्य}स्योपाधिक‚लाप‚स्य \textbf{को भेदः} क उपाधिविशेषः‚{\tiny $_{lb}$}‚ \textbf{स्याद‚नि}श्चितः [।] स‚र्व एव निश्चितः स्यात् ।
	{\color{gray}{\rmlatinfont\textsuperscript{§~\theparCount}}}
	\pend% ending standard par
      ‚{\tiny $_{lb}$}‚

	  
	  \pstart \leavevmode% starting standard par
	न‚न्व‚ग्निधूम‚योः स‚त्य‚पि स‚म्ब‚न्धे नाग्निनिश्च‚ये धूम‚स्य निश्च‚यो दृश्य‚ते त‚था‚{\tiny $_{lb}$}‚ ध‚र्मिनिश्च‚ये ध‚र्मानिश्च‚यो भ‚विष्य‚ति ।
	{\color{gray}{\rmlatinfont\textsuperscript{§~\theparCount}}}
	\pend% ending standard par
      ‚{\tiny $_{lb}$}‚

	  
	  \pstart \leavevmode% starting standard par
	न‚न्विद‚मेवाद‚र्श‚न‚न्न स्यात् । निश्च‚य‚प्र‚त्य‚येन स‚र्वात्म‚नाऽग्निस्व‚रूप‚ग्र‚हे स‚ति‚{\tiny $_{lb}$}‚  \leavevmode\ledsidenote{\textenglish{133/s}}धूमादिकार‚ण‚त्वेनैव निश्च‚यात् । धूमा‚{\tiny $_{५}$}‚दीनाम‚पि निश्चित‚त्वात् । त‚स्मान्न निश्च‚{\tiny $_{lb}$}‚येन त‚त्स्व‚रूप‚ग्र‚ह‚णं । निर्विक‚ल्प‚केनापि त‚र्हि स‚र्वात्म‚ना ग्र‚हो न स्याद् धूमाप्र‚ति‚{\tiny $_{lb}$}‚भासादेव त‚त्कार्य‚त्वाग्र‚हात् ।
	{\color{gray}{\rmlatinfont\textsuperscript{§~\theparCount}}}
	\pend% ending standard par
      ‚{\tiny $_{lb}$}‚

	  
	  \pstart \leavevmode% starting standard par
	नैत‚द‚स्ति [।] अग्नेर्हि धूम‚ज‚न‚नंप्र‚ति कार‚ण‚त्वं पूर्व‚भाव एवोच्य‚ते । स च‚{\tiny $_{lb}$}‚ प्र‚त्य‚क्षे प्र‚तिभास‚त इति क‚थं नाग्नेस्त‚त्कार्य‚त्व‚ग्र‚हः । तेनाय‚म‚र्थः [।] नाग्निधू‚{\tiny $_{lb}$}‚म‚योः प‚र‚मार्थ‚तः प‚र‚स्प‚रापेक्षिता विद्य‚ते । निष्प‚न्ना‚{\tiny $_{६}$}‚निष्प‚न्नाव‚स्थायां स‚म्ब‚न्धा‚{\tiny $_{lb}$}‚भावात् [।] केव‚ल‚म‚ग्नौ स‚त्येव धूमो भ‚व‚तीति तौ कार्य‚कार‚णे उच्येते । ध‚र्म‚ध‚र्मि‚{\tiny $_{lb}$}‚णोस्तु प‚र‚स्प‚रापेक्षित्वाद्ध‚र्मिणः स‚र्वात्म‚ना निश्च‚ये स‚र्व‚ध‚र्माणां साक्षान्निश्च‚यः‚{\tiny $_{lb}$}‚ स्यान्नार्थादाक्षेपः । य‚दा तु बाह्याध्य‚व‚साय‚को विक‚ल्पो भ‚व‚ति न तु ग्राह‚क‚स्त‚दाध्य‚{\tiny $_{lb}$}‚व‚सित‚स्यार्थ‚स्यान्य‚व्यावृत्तिस‚म्भ‚वेनार्थाद‚न्य‚ध‚र्माक्षेपो युज्य‚ते [।] प‚र‚स्प‚रापेक्ष‚त्वे‚{\tiny $_{lb}$}‚ च‚{\tiny $_{७}$}‚ याव‚न्न ध‚र्म‚निश्च‚यो न ताव‚द्ध‚र्मिनिश्च‚यो [।] याव‚च्च न ध‚र्मिनिश्च‚य‚स्ताव‚न्न \leavevmode\ledsidenote{\textenglish{51a/PSVTa}}‚{\tiny $_{lb}$}‚ ध‚र्म‚निश्च‚य इत्य‚न्योन्याश्र‚य‚त्व‚ञ्च स्यात् ।
	{\color{gray}{\rmlatinfont\textsuperscript{§~\theparCount}}}
	\pend% ending standard par
      ‚{\tiny $_{lb}$}‚

	  
	  \pstart \leavevmode% starting standard par
	न‚नु चोपाध्युपाधिम‚द्भाव आश्र‚याश्र‚यिभाव एवोच्य‚ते । स च स‚मान‚काल‚{\tiny $_{lb}$}‚भाविनोरेव [।] न च त‚योरुप‚कार्योप‚कार‚क‚भावो भिन्न‚काल‚त्वाद‚स्य [।] त‚त्क‚थ‚{\tiny $_{lb}$}‚मुच्य‚ते । एकोपाधिविशिष्ट‚ग्र‚हे स‚र्व‚ग्र‚ह इति ।
	{\color{gray}{\rmlatinfont\textsuperscript{§~\theparCount}}}
	\pend% ending standard par
      ‚{\tiny $_{lb}$}‚

	  
	  \pstart \leavevmode% starting standard par
	स‚त्त्यं [।] किन्तु प‚रैर‚न्य एव ज‚न्य‚ज‚न‚क‚भावोन्य‚श्चोप‚का‚{\tiny $_{१}$}‚र्योप‚कार‚क‚भाव‚{\tiny $_{lb}$}‚ इष्य‚ते । त‚था हि [।] व‚द‚र‚द्र‚व्यं स्व‚हेतुज‚न्य‚म‚पि कुण्डेनोप‚क्रिय‚तेऽत एवं स‚मान‚{\tiny $_{lb}$}‚काल‚भाविनोर‚य‚मुप‚कार्योप‚कार‚क‚भाव इष्य‚त इति त‚द‚भिप्रायादिद‚मुक्तं । य‚द्वा‚{\tiny $_{lb}$}‚ ध‚र्मो निश्चीय‚मानः ध‚र्म्याश्रित‚त्वादेवाश्र‚य‚स्य प्र‚तीतिमाक्षिप‚ति स चाश्रितानां‚{\tiny $_{lb}$}‚ ध‚र्माणामिति स‚र्व‚निश्च‚यः । त‚स्माच्छ‚ब्द‚प्र‚माणान्त‚र‚वृत्तेः क‚ल्पित एव ध‚र्म‚ध‚र्मि‚{\tiny $_{lb}$}‚भावः‚{\tiny $_{२}$}‚ [।] न चास्मिन् प‚क्षे ध‚र्म‚भेदाभेद‚क‚ल्प‚नायां प्र‚माणान्त‚र‚वैय‚र्थ्य‚म‚व‚स्तुत्वेन‚{\tiny $_{lb}$}‚ तेषां भेदाभेद‚स्य प‚र‚मार्थ‚तोऽभावात् । प्र‚माणान्त‚रैरेव च ध‚र्मान्त‚राणां क‚ल्प‚नीय‚{\tiny $_{lb}$}‚त्वात् त‚द‚भावे क‚थं ध‚र्म‚भेदाभेद‚क‚ल्प‚नेति य‚त्किञ्चिदेत‚त् ॥
	{\color{gray}{\rmlatinfont\textsuperscript{§~\theparCount}}}
	\pend% ending standard par
      ‚{\tiny $_{lb}$}‚

	  
	  \pstart \leavevmode% starting standard par
	\textbf{य‚द्य‚पी}त्यादिना कारिकार्थं व्याच‚ष्टे । \textbf{श‚ब्दान्त‚रा}णां ज्ञानान्त‚राणा\textbf{म‚र्थ} उपा‚{\tiny $_{lb}$}‚धिम‚ति प्र‚तिप‚त्तौ प्र‚ति\textbf{निमित्तं भिन्ना एवोपाध‚यो} य‚द्य‚प्य‚भ्यु‚{\tiny $_{३}$}‚प‚ग‚म्य‚न्ते [।] \textbf{स तु‚{\tiny $_{lb}$}‚ त‚द्वानुपाधिमान‚र्थः श‚ब्द‚ज्ञानैरुप‚लीय‚ते} विष‚यीक्रिय‚ते । \textbf{त‚स्य} त‚द्व‚तो \textbf{नानोपाधीनामु‚{\tiny $_{lb}$}‚प‚काराश्र‚या याः श‚क्त‚य‚स्त‚त्स्व‚भाव‚स्य स्वात्म}नि स्व‚रूपे \textbf{भेदो नास्ति} । त‚त‚श्चैकोपा‚{\tiny $_{lb}$}‚‚{\tiny $_{lb}$}‚ \leavevmode\ledsidenote{\textenglish{134/s}}धिद्वारेणापि \textbf{ग्र‚हे स‚र्वात्म‚ना} ग्र‚ह‚ण‚न्त‚स्मिन्स‚ति \textbf{क एवोपाधिभेद}स्त‚स्या\textbf{निश्चितः}‚{\tiny $_{lb}$}‚ किन्तु स‚र्व एव निश्चितः स्यात् । किं कार‚णं [।] \textbf{स‚र्वोपाध्युप‚कार‚क‚त्वे‚{\tiny $_{४}$}‚न‚{\tiny $_{lb}$}‚ ग्र‚ह‚णात्} ।
	{\color{gray}{\rmlatinfont\textsuperscript{§~\theparCount}}}
	\pend% ending standard par
      ‚{\tiny $_{lb}$}‚

	  
	  \pstart \leavevmode% starting standard par
	एकोपाधिद्वारेणोपाधिम‚तः स्व‚रूप‚मेव ग्र‚हीतं न तूपाध्युप‚कार‚क‚त्व‚मिति‚{\tiny $_{lb}$}‚ चेदाह । \textbf{न ही}त्यादि । न ह्युपाध्युप‚कार‚क‚त्व‚म‚न्य‚देवागृहीत‚मित्य‚नेन स‚म्ब‚न्धः ।‚{\tiny $_{lb}$}‚ \textbf{त‚स्ये}त्युपाधिम‚तः \textbf{स्वेन रूपेण गृह्य‚माण‚स्य} स्व‚रूपा\textbf{दुप‚कार‚क‚त्व‚स्याभेदात्} । य‚त‚{\tiny $_{lb}$}‚ एव‚म‚तः कार‚णा\textbf{द‚स्ये}त्युपाधिम‚तः । \textbf{य‚देव स्व‚भावेन} स्व‚रूपेण \textbf{ग्र‚ह‚{\tiny $_{५}$}‚ण‚न्त‚देवोप‚{\tiny $_{lb}$}‚कार‚क‚त्वेनापि} ग्र‚ह‚ण‚मिति ।
	{\color{gray}{\rmlatinfont\textsuperscript{§~\theparCount}}}
	\pend% ending standard par
      ‚{\tiny $_{lb}$}‚

	  
	  \pstart \leavevmode% starting standard par
	भ‚व‚त्वेकोपाधिद्वारेणोपाधिम‚तः स‚र्वोपाध्युप‚कार‚क‚त्व‚स्य स्व‚भाव‚भूत‚स्य ग्र‚हः ।‚{\tiny $_{lb}$}‚ उपाधीनान्तु त‚स्माद् व्य‚तिरिक्तानां क‚थं ग्र‚ह‚ण‚मित्य‚त आह । \textbf{त‚योरि}त्यादि ।‚{\tiny $_{lb}$}‚ उपाधिक‚लाप‚स्योपाधिम‚त‚श्च उप‚कार्योप‚कार‚क‚भूत‚यो\textbf{रात्म‚नि} स्व‚भावेन्योन्य\textbf{स‚म्ब‚न्धो}‚{\tiny $_{lb}$}‚ स‚म्ब‚न्धादुप‚कार्योप‚कार‚क‚स‚म्ब‚न्ध‚स्यात्म‚{\tiny $_{६}$}‚भूत‚त्वादिति याव‚त् । त‚त‚श्चोप‚कार‚क‚{\tiny $_{lb}$}‚स्व‚भाव\textbf{स्यैक‚स्य ज्ञाने स‚ति} स‚म्ब‚न्धाद् \textbf{द्व‚य‚ग्र‚हः} । उपाध्युपाधिम‚तोर्ग्र‚हः ।
	{\color{gray}{\rmlatinfont\textsuperscript{§~\theparCount}}}
	\pend% ending standard par
      ‚{\tiny $_{lb}$}‚

	  
	  \pstart \leavevmode% starting standard par
	स्त‚द्व्याच‚ष्टे । \textbf{आत्म‚भूत‚स्ये}त्यादि । त‚था ह्युपाधिम‚ति गृहीते त‚स्यात्म‚{\tiny $_{lb}$}‚भूत उप‚कार‚क‚भाव‚स्ताव‚द् गृहीत‚स्त‚स्मिन् गृहीते उपाधीनाम‚प्यु\textbf{कार्य}भाव आत्म‚{\tiny $_{lb}$}‚\leavevmode\ledsidenote{\textenglish{51b/PSVTa}} भूतो गृहीत‚स्त‚द्ग्र‚ह‚ण‚नान्त‚रीय‚क‚त्वा\textbf{दुप‚कार‚क‚भाव}ग्र‚ह‚ण‚स्य ‚{\tiny $_{७}$}‚[।] \textbf{अतः} कार‚णा‚{\tiny $_{lb}$}‚\textbf{देक‚ज्ञाने द्व‚यो}र‚प्युपाध्युपाधिम‚तो\textbf{र्ग्र‚ह‚ण}मिति कृत्वा \textbf{एकोपाधिविशिष्टेपि} त‚स्मिन्नुपा‚{\tiny $_{lb}$}‚धिम‚ति \textbf{गृह्य‚माणे स‚र्वोपाधीनां ग्र‚ह‚णं । त‚द्ग्र‚ह‚ण‚नान्त‚रीय}क‚त्वादित्युपाधिग्र‚ह‚ण‚{\tiny $_{lb}$}‚नान्त‚रीय‚क‚त्वात् । अन्य‚थेत्युपाधीनाम‚ग्र‚हे \textbf{त‚थापि न गृह्येत} । उपाधीनामुप‚{\tiny $_{lb}$}‚कार‚क उपाधिमानित्येव‚म‚पि न गृह्येत ।
	{\color{gray}{\rmlatinfont\textsuperscript{§~\theparCount}}}
	\pend% ending standard par
      ‚{\tiny $_{lb}$}‚

	  
	  \pstart \leavevmode% starting standard par
	य एव त‚दानीं ज्ञान‚श‚ब्द‚प्र‚वृत्तिनिमित्त‚{\tiny $_{१}$}‚मुपाधिस्तं प्र‚त्येवोप‚कार‚क‚त्व‚मुपाधिम‚तो‚{\tiny $_{lb}$}‚ गृहीतं न तूपाध्य‚न्त‚रोप‚कार‚क‚त्व‚मिति चेदाह । \textbf{न ह्य‚न्य} एवेत्यादि । \textbf{अन्योप‚कार‚क}‚{\tiny $_{lb}$}‚ इत्य‚न्य‚स्योपाधेरुप‚कार‚कः स्व‚भावो \textbf{यो न गृहीतः} किन्तु स‚र्व एव गृहीतो निरंश‚त्वाद‚{\tiny $_{lb}$}‚ व‚स्तुनः ।
	{\color{gray}{\rmlatinfont\textsuperscript{§~\theparCount}}}
	\pend% ending standard par
      ‚{\tiny $_{lb}$}‚‚{\tiny $_{lb}$}‚\textsuperscript{\textenglish{135/s}}

	  
	  \pstart \leavevmode% starting standard par
	स्यान्म‚त‚म् [।] उपाधिम‚तोन‚पेक्षित‚स‚म्ब‚न्धिन उप‚कार‚क इत्येव ग्र‚ह‚णं न‚{\tiny $_{lb}$}‚ त्व‚स्योप‚कार‚क इति त‚तो नोपाधीनां ग्र‚ह‚ण‚मित्य‚त आ‚{\tiny $_{२}$}‚ह । \textbf{न चापी}त्यादि । \textbf{त‚था‚{\tiny $_{lb}$}‚ गृहीत} इत्युप‚कार‚क इत्येवं गृहीते \textbf{उप‚कार‚कार्य‚स्यो}पाधेर\textbf{ग्र‚ह‚णं} । किं कार‚णं [।]‚{\tiny $_{lb}$}‚ \textbf{त‚स्या}प्युकार‚क‚स्य उप‚कार‚क इत्ये\textbf{व‚म‚ग्र‚ह‚ण‚प्र‚संगात्} । एत‚त् क‚थ‚य‚ति [।] स‚म्ब‚न्धि‚{\tiny $_{lb}$}‚त्वादुप‚कार‚क‚स्यान्त‚रेण द्वितीय‚स‚म्ब‚न्धिग्र‚ह‚ण‚मुप‚कार‚क इत्य‚पि ग्र‚ह‚णं नास्तीति ।‚{\tiny $_{lb}$}‚ त‚थाभूत‚ञ्च दृष्टान्त‚माह । \textbf{स्व‚स्वामित्व‚व}दिति । न हि स्व‚ग्र‚ह‚ण‚म‚न्त‚रेणा‚{\tiny $_{३}$}‚स्ति‚{\tiny $_{lb}$}‚ स्वामित्व‚स्य ग्र‚ह‚णं । य‚त एव‚मेकोपाधिद्वारेण प्र‚वृत्तेप्येक‚स्मिन् ज्ञाने श‚ब्दे च‚{\tiny $_{lb}$}‚ स‚र्वोपाधीनां ग्र‚ह‚ण‚मुपाधिम‚त‚श्च स‚र्वात्म‚ना । \textbf{त‚स्माद‚र्थान्त‚रोपाधिवादेपि}‚{\tiny $_{lb}$}‚ अर्थान्त‚र‚भूता उपाध‚य इत्येवंवादेपि \textbf{स‚मानः प्र‚स‚ङ्गः} । श‚ब्द‚ज्ञानान्त‚राणां प‚र्याय‚ता‚{\tiny $_{lb}$}‚ प्राप्नोतीति ।
	{\color{gray}{\rmlatinfont\textsuperscript{§~\theparCount}}}
	\pend% ending standard par
      ‚{\tiny $_{lb}$}‚

	  
	  \pstart \leavevmode% starting standard par
	\textbf{अथा}पीत्यादिना प‚र‚माश‚ङ्क‚ते । \textbf{याभिः} श‚क्तिभिः श‚क्तिमा\textbf{नुपाधीनुप‚क‚रोति‚{\tiny $_{lb}$}‚ ताः‚{\tiny $_{४}$}‚ श‚क्त‚यः} श‚क्तिम‚तः स‚काशाद् \textbf{भिन्नाः । त‚त} इति भेदात् । \textbf{नायं प्र‚संग} इति ।‚{\tiny $_{lb}$}‚ एकोपाध्युप‚कार‚क‚श‚क्त्य‚भेद‚ग्र‚हे स‚र्व‚श‚क्तीनां ग्र‚ह‚णं । त‚द्ग्र‚ह‚णाच्च स‚र्वोपाधीनामित्य‚यं‚{\tiny $_{lb}$}‚ प्र‚स‚ङ्गो नास्ति ।
	{\color{gray}{\rmlatinfont\textsuperscript{§~\theparCount}}}
	\pend% ending standard par
      ‚{\tiny $_{lb}$}‚

	  
	  \pstart \leavevmode% starting standard par
	उत्त‚र‚माह । \textbf{ध‚र्मोप‚कारे}त्यादि । ध‚र्मा उपाध‚य‚स्तेषामुप‚कार‚स्य याः \textbf{श‚क्त‚य‚स्तासां}‚{\tiny $_{lb}$}‚ श‚क्तिम‚तः स‚काशाद् \textbf{भेदे}ऽभ्युप‚ग‚म्य‚माने \textbf{ताः} श‚क्त‚य\textbf{स्त‚स्य} श‚क्तिम‚तः किम्भ‚व‚{\tiny $_{५}$}‚न्ति ।‚{\tiny $_{lb}$}‚ न किंचिद् भ‚व‚न्ति न त‚त्स‚म्ब‚न्धिन्य इत्य‚र्थः । य‚दा \textbf{य‚दि नोप‚कार‚स्त‚तः} । श‚क्तिम‚{\tiny $_{lb}$}‚तं\textbf{स्तासां} श‚क्तीनां । अथ स‚म्ब‚न्ध‚सिद्ध्य‚र्थ‚मुप‚कार इष्य‚ते । त‚दा श‚क्त्युप‚कारिण्यो‚{\tiny $_{lb}$}‚ प‚रा व्य‚तिरिक्ताः श‚क्त‚योङ्गीक‚र्त्त‚व्याः । याभिः श‚क्तीरुप‚क‚रोति [।] तासां च‚{\tiny $_{lb}$}‚ स‚म्ब‚न्ध‚त्व‚सिद्ध्य‚र्थ‚मुप‚कारः क‚ल्प‚नीयः । त‚त्राप‚रा श‚क्तिक‚ल्प‚नेति \textbf{त‚था स्याद‚न‚{\tiny $_{lb}$}‚व‚स्थितिः} ।
	{\color{gray}{\rmlatinfont\textsuperscript{§~\theparCount}}}
	\pend% ending standard par
      ‚{\tiny $_{lb}$}‚\textsuperscript{\textenglish{136/s}}

	  
	  \pstart \leavevmode% starting standard par
	त‚द्व्याच‚ष्टे [।] \textbf{य‚दी}‚{\tiny $_{६}$}‚त्यादि । उपाधिमुपाधिम्प्र‚ति \textbf{प्र‚त्युपाधि । उप‚कार‚क‚त्वा}नि‚{\tiny $_{lb}$}‚ श‚क्त‚य\textbf{स्त‚स्यो}पाधिम‚तः \textbf{न स्वात्म‚भूतानि} । न स्व‚भाव‚भूतानि । किन्तु व्य‚तिरिक्तानि ।‚{\tiny $_{lb}$}‚ नापि त‚त उपाधिम‚तः \textbf{उप‚कार‚म‚नुभ‚व}न्ति । आत्म‚सात्कुर्व‚न्ति । \textbf{किन्त‚स्य ता उच्य}न्ते ।‚{\tiny $_{lb}$}‚ उपाधिम‚तः श‚क्त‚य इति क‚स्मादुच्य‚न्ते । स‚म्ब‚न्धाभावात् । स‚म्ब‚न्ध‚सिद्ध्य‚र्थ‚मुपाधि‚{\tiny $_{lb}$}‚\leavevmode\ledsidenote{\textenglish{52a/PSVTa}} म‚तः स‚काशात् तासामुपाध्युप‚कारिणी‚{\tiny $_{७}$}‚नां श‚क्तीनामुप‚कारेवाङ्गीक्रिय‚माणे याभिः‚{\tiny $_{lb}$}‚ श‚क्तिभिरुपाध्युप‚कारिणीः श‚क्तीरुप‚क‚रोत्य‚य‚मुपाधिमान् । \textbf{य‚दि} तास्त‚स्यात्म‚भूता‚{\tiny $_{lb}$}‚ इष्य‚न्ते त‚दा \textbf{स्वात्म‚भूताभिः श‚क्तिभिर‚य‚मु}पाधिमान् \textbf{एक} इत्य‚नंशः । उपाध्युप‚का‚{\tiny $_{lb}$}‚रिणीः श\textbf{क्तीरुप‚कुर्व}न् ।
	{\color{gray}{\rmlatinfont\textsuperscript{§~\theparCount}}}
	\pend% ending standard par
      ‚{\tiny $_{lb}$}‚

	  
	  \pstart \leavevmode% starting standard par
	\textbf{त‚था ही}त्यादिना \textbf{स‚र्वात्म‚ना ग्र‚ह}णं साध‚य‚ति । एकोपाधि\textbf{ग्र‚ह‚णे त‚दुप‚का‚{\tiny $_{lb}$}‚रिण्यो} उपाध्युप‚कारिण्यो व्य‚तिरिक्तायाः \textbf{श‚क्तेर्ग्र‚ह‚णं । त‚द्ग्र‚ह‚ण} इति ।‚{\tiny $_{१}$}‚‚{\tiny $_{lb}$}‚ उपाध्युप‚कारिश‚क्तिग्र‚हे । \textbf{त‚दुप‚कारी} उपाध्युप‚कारिश‚क्त्युप‚कारी । किंभूतः‚{\tiny $_{lb}$}‚ \textbf{स्वात्म‚भूत‚स‚क‚ल‚श‚क्त्य‚प‚कारः} स्वात्म‚भूताः स‚क‚ला उपाध्युप‚कारिणीनां श‚क्तीना‚{\tiny $_{lb}$}‚मुप‚काराः श‚क्त‚यो य‚स्य स त‚थाभूतो \textbf{भावो गृहीतः स‚र्वा} उपाध्युप‚कारिकाः‚{\tiny $_{lb}$}‚ श\textbf{क्तीर्ग्राह‚य‚ति । ताश्चे}माः श‚क्त‚यो गृहीताः स्वोप‚कार्या\textbf{नुपाधीन् ग्राह‚य‚न्तीति‚{\tiny $_{lb}$}‚ त‚द‚व‚स्थः प्र‚संगः} को भेदः स्याद‚निश्चित इति य उक्तः ॥
	{\color{gray}{\rmlatinfont\textsuperscript{§~\theparCount}}}
	\pend% ending standard par
      ‚{\tiny $_{lb}$}‚

	  
	  \pstart \leavevmode% starting standard par
	\textbf{अथ} माभूदेष दोष इति \textbf{ता अपि श‚क्त्युप‚कारिण्यः श‚क्त‚यो भिन्ना एवो}पाधि‚{\tiny $_{lb}$}‚म‚तो \textbf{भावा}दिष्य‚न्ते । त‚दा त‚दुप‚कारिण्योपि श‚क्त‚यो व्य‚तिरिक्ताः क‚ल्प‚नीयास्त‚था‚{\tiny $_{lb}$}‚ त‚दुप‚कारिण्य इत्येव‚म‚न‚व‚स्थानात् । \textbf{उपाधीनां} त\textbf{च्छ‚क्तीनां} च । \textbf{उपाध्युप‚कार‚श‚{\tiny $_{lb}$}‚क्तीनां चाप‚राप‚रास्वेव श‚क्तिष्व‚प‚र्य‚व‚सानेनानिष्ठ‚या} य‚द् घ‚ट‚न‚न्त‚स्योद्घ‚ट‚नात्‚{\tiny $_{lb}$}‚ स‚म्ब‚न्ध‚नात् । त‚था ह्युपाध‚यो व्य‚तिरिक्तासु श‚क्तिषु स‚म्ब‚द्धास्ता अपि व्य‚ति‚{\tiny $_{३}$}‚‚{\tiny $_{lb}$}‚रिक्तास्वेव । एव‚मुत्त‚रोत्त‚रा श‚क्तिः पूर्व‚पूर्वासु श‚क्तिषु व्य‚तिरिक्तास्वेवान‚व‚स्था‚{\tiny $_{lb}$}‚नेन स‚म्ब‚द्धा । न तूपाधिम‚ति । त‚त‚श्च \textbf{स एक} उपाधिमान् । \textbf{ताभि}रुपाध्युप‚कारि‚{\tiny $_{lb}$}‚  ‚{\tiny $_{lb}$}‚ ‚{\tiny $_{lb}$}‚ \leavevmode\ledsidenote{\textenglish{137/s}}काभिः श‚क्तिभिस्स‚ह \textbf{क‚दाचिद‚प्य‚गृहीत‚स्त‚दुप‚कारात्मा} । श‚क्त्युप‚कारात्मा ।‚{\tiny $_{lb}$}‚ उपाध्युप‚कारिकाणां श‚क्तीनां याः श‚क्त‚य‚स्त‚दात्मेति याव‚त् । श‚क्तीनान्त‚तो व्य‚ति‚{\tiny $_{lb}$}‚रेकात् । \textbf{त‚द्व‚त्त्वे}न उपाध्युप‚कार‚श‚क्तिम‚त्त्वेन । \textbf{त‚द‚ग्र‚{\tiny $_{४}$}‚हा}दुपाधिम‚त्त्वेनाप्य‚तो व्य‚र्थै‚{\tiny $_{lb}$}‚वोपाधिक‚ल्प‚नेति भावः ।
	{\color{gray}{\rmlatinfont\textsuperscript{§~\theparCount}}}
	\pend% ending standard par
      ‚{\tiny $_{lb}$}‚

	  
	  \pstart \leavevmode% starting standard par
	एव‚न्ताव‚द् य‚दोपाधिम‚ति ज्ञान‚श‚ब्द‚योर्वृत्तिस्त‚दोक्तो दोषः । य‚दोपाधिष्वेव‚{\tiny $_{lb}$}‚ त‚दान्यो दोषो व‚क्त‚व्यः । त‚द‚भिधानायोपाधिप‚क्ष‚मुप‚न्य‚स्य‚ति । \textbf{य‚दि पुन}रित्यादि ।‚{\tiny $_{lb}$}‚ \textbf{केव‚ला}नित्याश्र‚य‚र‚हितान् उ\textbf{पाधी}न् विशेष‚ण‚भेदान् \textbf{श‚ब्द‚ज्ञानान्युप‚लीयेर}न् ।‚{\tiny $_{lb}$}‚ प्र‚त्याय‚क‚त्वेन स‚माश्र‚येयुः । त‚स्योपाधिम‚तः श‚ब्द‚ज्ञानैर‚स‚{\tiny $_{५}$}‚मावेशाद‚विष‚यीक‚र‚णात् ।‚{\tiny $_{lb}$}‚ \textbf{त‚त्प्र‚ति}प‚त्तिमुखेनोपाधिम‚त्प्र‚तिप‚त्तिमुखेन । एकेनापि श‚ब्द‚ज्ञानेन स‚र्व‚स्योपाधेः‚{\tiny $_{lb}$}‚ प्र‚तिप‚त्तिः [।]
	{\color{gray}{\rmlatinfont\textsuperscript{§~\theparCount}}}
	\pend% ending standard par
      ‚{\tiny $_{lb}$}‚

	  
	  \pstart \leavevmode% starting standard par
	अत्रापि दोष‚माह । \textbf{त‚दापी}त्यादि । त‚स्योपाधिम‚तः । \textbf{अनाक्षेपा}दित्य‚प्र‚ति‚{\tiny $_{lb}$}‚पाद‚नात् । \textbf{त‚त्रे}त्युपाधिम‚ति [।] क‚थं व्य‚र्थ इत्याह । \textbf{अर्थ‚क्रिये}त्यादि । अर्थ‚क्रियां‚{\tiny $_{lb}$}‚ पुरोधाय प्र‚वृत्तेर‚र्थ‚क्रिया \textbf{आश्र‚य} आल‚म्ब‚नं य‚स्य व्य‚व‚हार‚स्य स त‚था । \textbf{स‚र्वो} यावान्‚{\tiny $_{lb}$}‚ क‚श्चित् प्रेक्षा‚{\tiny $_{६}$}‚पूर्व‚कारिणां \textbf{व्य‚व‚हारो} हिताहित‚विष‚यः । स च द्वाभ्यां प्र‚काराभ्यां‚{\tiny $_{lb}$}‚ \textbf{विधिप्र‚तिषेधाभ्यां} तृतीय‚प्र‚काराभावात् । इत्थंभूत‚ल‚क्ष‚णा \href{http://sarit.indology.info/?cref=P\%C4\%81.2.3.21}{पाणिनिः} चेयं‚{\tiny $_{lb}$}‚ तृतीया । \textbf{उपाध्यो}\edtext{\textsuperscript{*}}{\lemma{*}\Bfootnote{? ध‚य}}श्च गोत्वाद‚य\textbf{स्त‚त्रा}र्थ‚क्रियायाम\textbf{स‚म‚र्थाः । स‚म‚र्थ}श्च‚{\tiny $_{lb}$}‚ व्य‚क्तिभेदः श‚ब्देनै\textbf{वोच्य‚त इति किम}फ‚लैः \textbf{श‚ब्द‚प्र‚योगैः} । य‚त‚श्चैव‚म‚र्थ‚क्रियास‚म‚र्थो‚{\tiny $_{lb}$}‚ व्य‚क्तिभेदो न चोच्य‚ते श‚ब्दैः । \textbf{त‚त‚श्}चाभिम‚ता उ\textbf{पाध‚यो} गोत्वाद‚य \textbf{उपाध‚यो} न‚{\tiny $_{७}$}‚ \leavevmode\ledsidenote{\textenglish{32b/PSVTa}}‚{\tiny $_{lb}$}‚ स्युर्न विशेष‚णानि स्युः । य‚स्मात् \textbf{क्व‚चि}दुपाधिम‚त्युपाधिद्वारेण श‚ब्द‚स्य ज्ञान‚स्य वा‚{\tiny $_{lb}$}‚ \textbf{प्र‚वृत्तौ} स‚त्यां । \textbf{क‚स्य}चिदुपाधिम‚तः । \textbf{प्र‚धान‚स्ये}ति विशेष्य‚स्या\textbf{ङ्गाभावाद्} विशे‚{\tiny $_{lb}$}‚ष‚ण‚भावात् \textbf{त‚द‚पेक्ष‚या} प्र‚धानापेक्ष‚या । \textbf{त‚थोच्य}न्ते । उपाध‚य इत्युच्य‚न्ते । इयं‚{\tiny $_{lb}$}‚ न्यायोपाधिव्य‚व‚स्था । य‚दा तूपाध‚य एव श‚ब्देनोच्य‚न्ते । त‚दा त‚स्योपाधिम‚तः‚{\tiny $_{lb}$}‚  ‚{\tiny $_{lb}$}‚ ‚{\tiny $_{lb}$}‚ \leavevmode\ledsidenote{\textenglish{138/s}}\textbf{श‚ब्देनाऽनाक्षेपाद}प्र‚तिपाद‚नान्न \textbf{ते} उपाध‚यः \textbf{क‚स्य‚चि}त् प्र‚धा‚{\tiny $_{१}$}‚न‚स्या\textbf{ङ्ग‚भूता इति‚{\tiny $_{lb}$}‚ किमुपाध‚यो} नैवेति याव‚त् । \textbf{य‚द्युपा}धिमात्रं चोद्य‚ते त‚थापि श‚ब्दै\textbf{र्ल‚क्षिता} ये उपाध‚{\tiny $_{lb}$}‚य‚स्तैरुपाधिम‚तो \textbf{ल‚क्ष‚णात्} प‚रिच्छेदा\textbf{द‚दोषः} । श‚ब्द‚प्र‚योग‚वैय‚र्थ्य‚दोषो \textbf{नेति चेत्} ।‚{\tiny $_{lb}$}‚ स \textbf{स‚मानः} स‚र्वोपाधिग्र‚ह‚ण\textbf{प्र‚स‚ङ्गः} । त‚मेवाह । \textbf{स ताव‚दि}त्यादि । स इत्युपाधि‚{\tiny $_{lb}$}‚मान् । \textbf{नान्त‚रीय‚क‚त‚ये}त्युपाध्युपाधिम‚तोर‚व्य‚भिचारेण \textbf{उप‚ल‚क्ष्य‚माण एकेनाप्युपा‚{\tiny $_{lb}$}‚धि}‚{\tiny $_{२}$}‚ना निर‚ङ्श‚त्वात् \textbf{स‚र्वात्म‚नोप‚ल‚क्षित इति त‚द‚व‚स्थः} स‚र्वोपाधिग्र‚ह\textbf{प्र‚संगः} ।
	{\color{gray}{\rmlatinfont\textsuperscript{§~\theparCount}}}
	\pend% ending standard par
      ‚{\tiny $_{lb}$}‚

	  
	  \pstart \leavevmode% starting standard par
	स्यान्म‚तं [।] य‚त्र श‚ब्देन साक्षादुपाधिम‚त‚श्चोद‚न‚न्त‚त्रायं प्र‚संगः । न तु य‚त्रा‚{\tiny $_{lb}$}‚र्थ‚व‚शादित्य‚त आह । \textbf{को ह्य‚त्र विशेष} इति । \textbf{श‚ब्दा} वा \textbf{एन‚मु}पाधिम‚न्तं साक्षात्‚{\tiny $_{lb}$}‚ प्र‚तिपाद‚येयुः । \textbf{त‚ल्ल‚क्षि}ता वा श‚ब्द‚ल‚क्षिता वोपाध‚य उपाधिम‚न्तं \textbf{ल‚क्ष‚ये}युरिति को‚{\tiny $_{lb}$}‚ विशेषो न क‚श्चित् [।] त‚था हि [।] \textbf{स ता}व‚दुपाधि‚{\tiny $_{३}$}‚मान् \textbf{त‚दा}नीमुपाधिब‚लेन‚{\tiny $_{lb}$}‚ ल‚क्ष‚ण‚काले निश्चीय‚ते । \textbf{स‚र्वोप‚कार‚कः} स‚र्वेषामुपाधीनामुप‚कार‚क \textbf{इति} । त‚था च‚{\tiny $_{lb}$}‚ पूर्व‚व‚त् स‚र्वोपाधिग्र‚ह‚ण‚प्र‚संगोऽतो ल‚क्षित‚ल‚क्ष‚णादिति य‚दुक्त‚मे\textbf{त‚न्न किञ्चित्}‚{\tiny $_{lb}$}‚ पूर्वोक्त‚दोष‚दुष्ट‚त्वात् । य‚स्मादुपाध्युप‚कारिकाणां श‚क्तीनाम्व्य‚तिरेकेऽन‚व‚स्था‚{\tiny $_{lb}$}‚ स्याद‚तो न व्य‚तिरिक्ताः श‚क्त‚यः ॥
	{\color{gray}{\rmlatinfont\textsuperscript{§~\theparCount}}}
	\pend% ending standard par
      ‚{\tiny $_{lb}$}‚

	  
	  \pstart \leavevmode% starting standard par
	\textbf{त‚स्मादेक‚स्यो}पाधेरु\textbf{प‚कार}के त‚स्मिन्नुपाधिम‚{\tiny $_{४}$}‚ति \textbf{ग्राह्ये}भ्युप‚ग‚म्य‚माने उपाध्य‚{\tiny $_{lb}$}‚न्त‚राणा\textbf{मुप‚कार‚काः} श‚क्तिभेदाः । \textbf{त‚त} एकोपाध्युप‚कार‚क‚स्व‚भावा\textbf{द‚प‚रे}ऽन्ये \textbf{न} भ‚व‚न्ति‚{\tiny $_{lb}$}‚  ‚{\tiny $_{lb}$}‚ ‚{\tiny $_{lb}$}‚ \leavevmode\ledsidenote{\textenglish{139/s}}\textbf{ये} दृष्टे \textbf{त‚स्मि}न्नुपाधिम‚त्य\textbf{दृष्टा} भ‚व‚न्ति । किन्त्व‚न‚न्ये । अतः कार‚णात्त‚द्ग्राह्ये ।‚{\tiny $_{lb}$}‚ त‚स्योपाधिम‚तो ग्र‚हे \textbf{स‚क}लोपाध्युप‚कार‚क स्व‚भाव‚स्य \textbf{ग्र‚हः} ।
	{\color{gray}{\rmlatinfont\textsuperscript{§~\theparCount}}}
	\pend% ending standard par
      ‚{\tiny $_{lb}$}‚

	  
	  \pstart \leavevmode% starting standard par
	योपि भ ट्टो म‚न्य‚ते [।] भिन्ना भिन्ना एव ध‚र्मास्तेनैक‚ध‚र्मेण ध‚र्मिण्य‚व‚धार्य‚माणे‚{\tiny $_{lb}$}‚न स‚{\tiny $_{५}$}‚र्व‚ध‚र्माव‚धार‚णं भेदात् । त‚दाह ।‚{\tiny $_{lb}$}‚ 
	    \pend% close preceding par
	  
	    
	    \stanza[\smallbreak]
	  {\normalfontlatin\large ``\qquad}आविर्भाव‚तिरोभाव‚ध‚र्म‚केष्व‚नुयायि य‚त् ।&‚{\tiny $_{lb}$}‚त‚द्ध‚र्मि य‚त्र वा ज्ञानं प्राग्ध‚र्म‚ग्र‚ह‚णाद् भ‚वेत् ॥ \href{http://sarit.indology.info/?cref=\%C5\%9Bv-pratyak\%E1\%B9\%A3a}{१५२}&‚{\tiny $_{lb}$}‚अन‚न्त‚ध‚र्म‚के ध‚र्मिण्येक‚ध‚र्माव‚धार‚णे ।&‚{\tiny $_{lb}$}‚श‚ब्दोभ्युपाय‚मात्रं स्यान्न तु स‚र्वाव‚धार‚ण \href{http://sarit.indology.info/?cref=\%C5\%9Bv-pratyak\%E1\%B9\%A3a.178}{१७८} इति ।\edtext{\textsuperscript{*}}{\edlabel{pvsvt_139-3}\label{pvsvt_139-3}\lemma{*}\Bfootnote{\href{http://sarit.indology.info/?cref=\%C5\%9Bv-pratyak\%E1\%B9\%A3a}{ Ślokavārtika  प्र‚त्य‚क्ष‚प‚रि०}}}{\normalfontlatin\large\qquad{}"}\&[\smallbreak]
	  
	  
	  
	    \pstart  \leavevmode% new par for following
	    \hphantom{.}
	  ‚{\tiny $_{lb}$}‚ सोप्युभ‚य‚प‚क्ष‚भाविदोष‚प्र‚संगादेव निर‚स्तः ॥
	{\color{gray}{\rmlatinfont\textsuperscript{§~\theparCount}}}
	\pend% ending standard par
      ‚{\tiny $_{lb}$}‚

	  
	  \pstart \leavevmode% starting standard par
	\textbf{य‚दी}त्यादिना प‚राभिप्राय‚माशंक‚ते । एकेन निश्च‚य‚ज्ञानेन स‚र्वात्म‚ना \textbf{गृहीतेपि}‚{\tiny $_{६}$}‚‚{\tiny $_{lb}$}‚ व‚स्तुनि \textbf{भ्रान्तिनिवृत्त्य‚र्थं । अन्य‚दि}ति प्र‚माणान्त‚रं ।
	{\color{gray}{\rmlatinfont\textsuperscript{§~\theparCount}}}
	\pend% ending standard par
      ‚{\tiny $_{lb}$}‚

	  
	  \pstart \leavevmode% starting standard par
	\textbf{स्यादेत}दित्यादिनैत‚देव व्याच‚ष्टे । \textbf{निर्भाग‚स्य} निरंश‚स्य व‚स्तुनो \textbf{ग्र‚ह‚णे} स‚ति‚{\tiny $_{lb}$}‚ \textbf{कोन्यो} भागः \textbf{त‚दा} निर्भाग‚व‚स्तुग्र‚ह‚ण‚काले । \textbf{न गृही}तो नाम स‚र्व एव गृहीतः [।]‚{\tiny $_{lb}$}‚ \textbf{स तु} गृहीतोपि \textbf{भ्रान्त्या नाव‚धार्य‚त इति प्र‚माणान्त‚रं प्र‚व‚र्त्त}ते ।
	{\color{gray}{\rmlatinfont\textsuperscript{§~\theparCount}}}
	\pend% ending standard par
      ‚{\tiny $_{lb}$}‚

	  
	  \pstart \leavevmode% starting standard par
	\textbf{य‚द्येव‚मि}त्यादिना सिद्धान्त‚वादी । य‚त्त‚द् \textbf{भ्रान्ति}निवृत्त्य‚र्थ‚मुत्त‚र‚म्प्र‚माण‚{\tiny $_{७}$}‚\textbf{मिष्य‚ते} \leavevmode\ledsidenote{\textenglish{53a/PSVTa}}‚{\tiny $_{lb}$}‚ \textbf{त‚द्व्य‚व‚च्छेद‚विष‚य}म‚न्यापोह‚विष‚यं \textbf{सिद्धं} पूर्वोक्तेन न्यायेन । \textbf{त‚द्व‚दु}त्त‚र‚प्र‚माण‚व‚त् ।‚{\tiny $_{lb}$}‚ \textbf{त‚त} उत्त‚र‚काल‚भावि प्र‚माणाद\textbf{प‚र‚म}पि पूर्व‚काल‚भाविनिश्च‚य‚ज्ञान‚न्त‚द‚पि व्य‚व‚च्छेद‚{\tiny $_{lb}$}‚विष‚यं । किं कार‚ण‚म् [।] \textbf{अस‚मारोप‚विष‚ये वृत्तेः} ।
	{\color{gray}{\rmlatinfont\textsuperscript{§~\theparCount}}}
	\pend% ending standard par
      ‚{\tiny $_{lb}$}‚‚{\tiny $_{lb}$}‚‚{\tiny $_{lb}$}‚\textsuperscript{\textenglish{140/s}}

	  
	  \pstart \leavevmode% starting standard par
	\textbf{त‚त्त‚र्ही}त्यादिना श्लोकं व्याच‚ष्टे । \textbf{अन्य}स्याकार‚स्य यः \textbf{स‚मारोप}स्त\textbf{द्व्य‚व‚च्छे}द‚{\tiny $_{lb}$}‚\textbf{फ‚ल‚मिति} कृत्वा \textbf{सिद्ध‚म‚न्यापोह‚विष‚य}मुत्पित्सुस‚{\tiny $_{१}$}‚मारोप‚निषेध‚द्वारेण । \textbf{त‚द्व‚द‚न्य‚द‚पि}‚{\tiny $_{lb}$}‚ पूर्व‚म‚पि निश्च‚य‚ज्ञान‚म‚न्यापोह‚विष‚यं [।] किं कार‚ण‚म् [।] \textbf{अविद्य‚मान‚स‚मारोपे‚{\tiny $_{lb}$}‚ विष‚ये वृ}त्तेः । एत‚देवाह । \textbf{य‚त्राकारेस्य प्र‚तिप‚त्तेः । इति} हेतोः \textbf{स‚मारोपाभावे‚{\tiny $_{lb}$}‚ व‚र्त्त‚मानः} पौर‚स्त्यो निश्च‚यो\textbf{न्यापोह‚विष‚यः सिद्धः} ॥
	{\color{gray}{\rmlatinfont\textsuperscript{§~\theparCount}}}
	\pend% ending standard par
      ‚{\tiny $_{lb}$}‚

	  
	  \pstart \leavevmode% starting standard par
	किञ्च [।] निश्च‚य‚गृहीतेप्य‚र्थे भ्रान्तिनिवृत्त्य‚र्थं प्र‚माणान्त‚र‚मिच्छ‚ता‚{\tiny $_{lb}$}‚ निश्च‚य‚विष‚य‚श्च न च निश्चित इ‚{\tiny $_{२}$}‚त्य‚भ्युप‚ग‚तं स्याद् [।] अन्य‚था भ्रान्तेर‚यो‚{\tiny $_{lb}$}‚गात् [।] त‚च्चायुक्त‚मित्याह । \textbf{अपि चे}त्यादि । \textbf{य‚द्रूपं निश्च‚यैर्न निश्चीय‚ते त‚द्रूप‚न्ते}षां‚{\tiny $_{lb}$}‚ निश्च‚यानां \textbf{विष‚यः क‚थ‚न्नै}व [।] किं कार‚णं । य‚स्मा\textbf{दिय‚मेव निश्च‚यानां‚{\tiny $_{lb}$}‚ स्वार्थ‚प्र‚तिप‚त्तिर्य‚त्त}स्यार्थ‚स्य निश्च‚य‚नं । \textbf{त‚च्चे}न्निश्च‚य‚विष‚याभिम‚त‚मा\textbf{कारान्त}र‚{\tiny $_{lb}$}‚\textbf{व‚द‚निश्चि}तं । य‚न्निश्च‚य‚विष‚य‚त्वेनान‚भिम‚त‚न्त‚द्व‚त्तैर्निश्च‚ये गृहीतं ॥
	{\color{gray}{\rmlatinfont\textsuperscript{§~\theparCount}}}
	\pend% ending standard par
      ‚{\tiny $_{lb}$}‚

	  
	  \pstart \leavevmode% starting standard par
	\textbf{क‚थ‚मित्या}‚{\tiny $_{३}$}‚दि प‚रः [।] प्र‚त्य‚क्ष‚गृहीते स‚मारोप‚व्य‚व‚च्छेदार्थं प्र‚माणान्त‚र‚{\tiny $_{lb}$}‚मिच्छ‚ताऽन्यापोह‚वादिनाप्य‚निश्चीय‚मान आकारः प्र‚त्य‚क्ष‚गृहीत‚त्वेनेष्टो य‚दि वा‚{\tiny $_{lb}$}‚ निश्च‚य‚व‚शादेव ग्र‚ह‚णं । \textbf{क‚थ‚मिदानीम‚निश्चीय‚मानं} रूपं \textbf{प्र‚त्य‚क्षेणापि गृहीत}मिति‚{\tiny $_{lb}$}‚ तुल्यः प्र‚संगः ।
	{\color{gray}{\rmlatinfont\textsuperscript{§~\theparCount}}}
	\pend% ending standard par
      ‚{\tiny $_{lb}$}‚

	  
	  \pstart \leavevmode% starting standard par
	\textbf{ने}त्यादिना प‚रिह‚र‚ति । क‚ल्प‚नाविविक्त‚त्वादित्य‚भिप्रायः । \textbf{त‚दि}ति प्र‚त्य‚क्षं‚{\tiny $_{lb}$}‚ [।] \textbf{य‚म‚पि} नीला‚{\tiny $_{४}$}‚द्याकार‚ङ्गृ\textbf{ह्णाती}त्युच्य‚ते [।] \textbf{त‚द्} ग्र‚ह‚णं \textbf{न निश्च‚येन [।]‚{\tiny $_{lb}$}‚ किन्त‚र्हि [।] प्र‚तिभासेन} । निर‚ङ्श‚स्य व‚स्तुनः स‚र्व‚था प्र‚तिभास‚न‚मिति स‚र्व‚था‚{\tiny $_{lb}$}‚  ‚{\tiny $_{lb}$}‚ ‚{\tiny $_{lb}$}‚ \leavevmode\ledsidenote{\textenglish{141/s}}ग्र‚ह‚णं [।] \textbf{त‚दि}ति त‚स्मा\textbf{न्न निश्च‚यानिश्च‚य‚व‚शाद्} य‚थाक्र‚मं \textbf{प्र‚त्य‚क्ष‚स्य ग्र‚ह‚णाग्र}ह‚णे‚{\tiny $_{lb}$}‚ किंतु प्र‚तिभास‚नाप्र‚तिभास‚न‚व‚शात् । त‚स्माद‚निश्च‚ये स‚ति प्र‚तिभास‚न‚मात्रेण‚{\tiny $_{lb}$}‚ प्र‚त्य‚क्ष‚गृहीत‚व्य‚व‚स्थाप‚न‚न्न विरुध्य‚ते । \textbf{नैवं निश्च‚या‚{\tiny $_{५}$}‚नां} प्र‚त्य‚क्ष‚व‚द‚निश्चित‚स्या‚{\tiny $_{lb}$}‚प्याकार‚स्य प्र‚तिभास‚न‚मात्रेण ग्र‚ह‚ण‚म‚प्र‚तिभास‚मात्रेणाग्र‚ह‚ण‚मिति स‚म्ब‚न्धः ।‚{\tiny $_{lb}$}‚ क‚स्मादिति चेदाह । \textbf{किञ्चिदि}त्यादि । य‚था पुरुषं दृष्ट्वा पुरुष‚त्व\textbf{न्निश्चिन्व}तो\textbf{प्य‚न्य‚त्र}‚{\tiny $_{lb}$}‚ त‚स्क‚रादाव\textbf{निश्च‚येन प्र‚वृत्तिभेदाद्} व्य‚व‚हार‚भेदात् । त‚था हि पुरुष‚त्व‚निश्च‚येन‚{\tiny $_{lb}$}‚ पुरुषोनुरूपो विश्वासादिव्य‚व‚हारो दृश्य‚ते । चौर‚त्वानि‚{\tiny $_{६}$}‚श्च‚याच्च त‚द‚नुरूपो‚{\tiny $_{lb}$}‚ भ‚यादिव्य‚व‚हारो न दृश्य‚ते । त‚त‚श्च य‚न्निश्च‚यानुरूपः प्र‚वृत्तिभेद‚स्त‚स्य निश्च‚येन‚{\tiny $_{lb}$}‚ ग्र‚ह‚णं । य‚द‚निश्च‚यानुरूप‚श्चाप्र‚वृत्तिभेद‚स्त‚स्याग्र‚ह‚ण‚मिति । \textbf{य‚त एव‚न्त‚स्मा}दित्यादि ।‚{\tiny $_{lb}$}‚ \textbf{अस्ये}त्याकार‚स्य [।] \textbf{अन्य‚थे}ति य‚दि निश्च‚य‚व‚शात्त‚स्य ग्र‚ह‚णं न व्य‚व‚स्थाप्य‚ते ।‚{\tiny $_{lb}$}‚ \textbf{त‚दैका}कारेपि । निश्चित‚त्वेनाभिम‚तेप्याकारे । \textbf{त‚दि}ति निश्च‚येन ग्र‚ह‚णं‚{\tiny $_{७}$}‚ \textbf{न स्या}त् ।
	{\color{gray}{\rmlatinfont\textsuperscript{§~\theparCount}}}
	\pend% ending standard par
      \textsuperscript{\textenglish{53b/PSVTa}}‚{\tiny $_{lb}$}‚

	  
	  \pstart \leavevmode% starting standard par
	\textbf{किं पुनः कार‚ण}मिति प‚रः । \textbf{स‚र्व‚तो भिन्न} इति स‚जातीय‚विजातीयाद्व्यावृत्तेः ।‚{\tiny $_{lb}$}‚ \textbf{त‚थैवे}ति य‚थानुभ‚वं स‚र्वेष्वेव भेदेषु \textbf{न स्मार्त्तो निश्च‚यो भ‚व‚ति} । य‚तो भेदान्त‚रेन्या‚{\tiny $_{lb}$}‚कार‚व्य‚व‚च्छेदार्थ‚म‚न्यापोह‚वादिना प्र‚माणान्त‚र‚वृत्तिरिष्य‚ते ।
	{\color{gray}{\rmlatinfont\textsuperscript{§~\theparCount}}}
	\pend% ending standard par
      ‚{\tiny $_{lb}$}‚

	  
	  \pstart \leavevmode% starting standard par
	\textbf{स‚ह‚कारिवैक‚ल्या}दिति सिद्धान्त‚वादी । न ह्य‚नुभ‚व‚मात्र‚निश्च‚य‚हेतुः किन्त्व‚{\tiny $_{lb}$}‚भ्यासाद‚योपि स‚ह‚कारिणः [।] ते य‚त्रैव स‚न्ति त‚त्रैवाकारे‚{\tiny $_{१}$}‚ निश्च‚यो नान्य‚त्र ।
	{\color{gray}{\rmlatinfont\textsuperscript{§~\theparCount}}}
	\pend% ending standard par
      ‚{\tiny $_{lb}$}‚

	  
	  \pstart \leavevmode% starting standard par
	न‚नु क्ष‚णिकाकारेपि स‚र्व‚दा द‚र्श‚नाद‚भ्यासोस्त्येवेति निश्च‚यः स्यात् । नानु‚{\tiny $_{lb}$}‚भूत‚निश्चित‚विष‚योत्राभ्यासोभिप्रेतो न च क्ष‚णिकं भ्रान्तिनिमित्त‚स‚म्भ‚वाद‚नु‚{\tiny $_{lb}$}‚भूत‚निश्चित‚मिति क‚थ‚न्त‚त्राभ्यासः । त‚स्मात् स्थित‚मेत‚त् य‚त्रैवाकारेऽभ्यास‚स्त‚त्रैव‚{\tiny $_{lb}$}‚ निश्च‚य इति ॥
	{\color{gray}{\rmlatinfont\textsuperscript{§~\theparCount}}}
	\pend% ending standard par
      ‚{\tiny $_{lb}$}‚

	  
	  \pstart \leavevmode% starting standard par
	त‚देवाह । \textbf{त‚त‚श्चे}त्यादि । \textbf{विशेषे} स‚र्व‚तो व्यावृत्ते नीलादिल‚क्ष‚णे । \textbf{अङ्श‚{\tiny $_{lb}$}‚विव‚र्जिते} निर्विभा‚{\tiny $_{२}$}‚गे स‚र्वात्म‚ना \textbf{प्र‚त्य‚क्षेण गृहीतेपि} स‚ति \textbf{य}स्य \textbf{विशेष‚स्यार्व‚सा}ये‚{\tiny $_{lb}$}‚ निश्च‚ये\textbf{स्ति} स‚ह‚कारि\textbf{प्र‚त्य‚यः स प्र‚तीय‚ते} निश्चीय‚ते ।
	{\color{gray}{\rmlatinfont\textsuperscript{§~\theparCount}}}
	\pend% ending standard par
      ‚{\tiny $_{lb}$}‚‚{\tiny $_{lb}$}‚\textsuperscript{\textenglish{142/s}}

	  
	  \pstart \leavevmode% starting standard par
	\textbf{य‚द्य‚पीत्या}दिना व्याच‚ष्टे । \textbf{स‚र्व‚भेदेषु} क्ष‚णिक‚त्वादिषु \textbf{ताव‚तेत्य‚नुभाव‚मात्रेण} ।‚{\tiny $_{lb}$}‚ निश्च‚योत्पाद‚नंप्र‚त्य‚नुभ‚व‚ज्ञान‚स्य \textbf{कार‚ण‚न्त‚रापेक्ष‚त्वात्} । तंदेवाह \textbf{अनुभ‚वो‚{\tiny $_{lb}$}‚ हीत्}यादि । \textbf{य‚थाविक‚ल्पाभ्यास‚मि}ति य‚स्य यादृशो विक‚ल्पाभ्या‚{\tiny $_{३}$}‚स‚स्तेन स‚ह‚का‚{\tiny $_{lb}$}‚रिणा \textbf{ज‚न‚य‚ती}त्य‚र्थः । उदाह‚र‚ण‚माह । \textbf{य‚थेत्या}दि । मृत‚स्त्री\textbf{रूप‚द‚र्श‚नाविशेषेपि}‚{\tiny $_{lb}$}‚ प‚रिव्राट्कामुक‚शुनां य‚थाक्र‚मं \textbf{कुण‚प‚कामिनीभ‚क्ष्य‚विक‚ल्पा} य‚थाविक‚ल्पाभ्यास‚{\tiny $_{lb}$}‚ञ्जाय‚न्ते । न च विक‚ल्पाभ्यास एव स‚ह‚कारी । किन्त्व‚न्योप्य‚स्तीत्याह । \textbf{त‚त्रे}‚{\tiny $_{lb}$}‚त्यादि । त‚त्रापि रूप‚द‚र्श‚नाविशेषेपि । \textbf{बुद्धेः पाट}व‚न्तीक्ष्ण‚ता । य‚था योगिनां बुद्धि‚{\tiny $_{lb}$}‚पाट‚{\tiny $_{४}$}‚वाद् द‚र्श‚न‚मात्रेण क्ष‚णिक‚त्वादिनिश्च‚यः । आदिश‚ब्दाद‚र्थित्व‚साम‚र्थ्यादिप‚रि‚{\tiny $_{lb}$}‚ग्र‚हः । \textbf{इत्याद‚य} इत्येव‚माद‚यः । \textbf{अनुभ‚वात्} प्र‚त्य‚क्षादुपादान‚कार‚णात्स‚काशाद्‚{\tiny $_{lb}$}‚ \textbf{भेद‚निश्च‚य‚स्योत्प‚त्तौ स‚ह‚कारिणः} । य‚दा त‚र्हि ब‚हुषु निश्च‚येषु य‚थोक्तानि कार‚णानि‚{\tiny $_{lb}$}‚ न भ‚व‚न्ति त‚दा तेषां निश्च‚यानां क‚थं क्र‚म‚भाव इत्याह । \textbf{तेषामेव} चेत्यादि [।]‚{\tiny $_{lb}$}‚ तेषामिति निश्च‚य‚कार‚णानां‚{\tiny $_{६}$}‚ \textbf{प्र‚त्यास}त्तितार‚त‚म्य‚भेदात् । य‚स्य निश्च‚य‚स्य प्र‚त्या‚{\tiny $_{lb}$}‚स‚न्न‚त‚म‚न्निश्च‚य‚कार‚ण‚न्त‚त्ताव‚दादावुत्प‚द्य‚ते । \textbf{आदि}श‚ब्दाद‚धिमात्र‚तार‚त‚म्य‚स्य भेदा‚{\tiny $_{lb}$}‚ न्निश्च‚यानां \textbf{पौर्वाप‚र्यं । य‚थे}त्यादिनोदाह‚र‚ण‚माह । पितेव य‚दोपाध्यायो भ‚व‚ति । त‚दै‚{\tiny $_{lb}$}‚क‚स्य पुरुष‚स्य ज\textbf{न‚क‚त्वाध्याप‚क‚त्वाविशेषे}पि । \textbf{पित‚र‚मायान्तं दृष्ट्वा पिता मे आग‚च्छ‚{\tiny $_{lb}$}‚\leavevmode\ledsidenote{\textenglish{54a/PSVTa}} तीति निश्चिनोति‚{\tiny $_{७}$}‚ नोपाध्याय} इति । पितृत्व‚निश्च‚ये कार‚ण‚स्य प्र‚त्यास‚न्न‚ताम‚त्वात् ॥
	{\color{gray}{\rmlatinfont\textsuperscript{§~\theparCount}}}
	\pend% ending standard par
      ‚{\tiny $_{lb}$}‚

	  
	  \pstart \leavevmode% starting standard par
	न‚नु स‚त्य‚पि क्ष‚णिक‚त्व‚नैरात्म्य‚विक‚ल्पाभ्यासे स‚ह‚कारिणी त‚त्त्वादार्शिनां नं‚{\tiny $_{lb}$}‚ प्र‚त्य‚क्षात् क्ष‚णिक‚त्वादिनिश्च‚यो भ‚व‚तीत्य‚त आह । \textbf{सोपी}त्यादि । \textbf{अस‚ति भ्रान्तिकार‚णे‚{\tiny $_{lb}$}‚ भ‚व‚ति} न तु निश्च‚य‚प्र‚त्य‚य‚मात्रात् । य‚त एव‚म‚स‚ति भ्रान्तिकार‚णे स‚ह‚कारिप्र‚त्य‚साक‚ल्ये‚{\tiny $_{lb}$}‚ च स‚ति प्र‚त्य‚क्षान्निश्च‚य उत्प‚द्य‚{\tiny $_{१}$}‚ते न केव‚लात् । \textbf{त‚स्मान्नानुभूत} इत्यादि ।
	{\color{gray}{\rmlatinfont\textsuperscript{§~\theparCount}}}
	\pend% ending standard par
      ‚{\tiny $_{lb}$}‚‚{\tiny $_{lb}$}‚‚{\tiny $_{lb}$}‚\textsuperscript{\textenglish{143/s}}

	  
	  \pstart \leavevmode% starting standard par
	त‚त‚श्च स्थित‚मेत‚द् [।] अन्य‚व्य‚व‚च्छेदः श‚ब्द‚लिङ्गाभ्यां प्र‚तिपाद्य‚त इति ।
	{\color{gray}{\rmlatinfont\textsuperscript{§~\theparCount}}}
	\pend% ending standard par
      ‚{\tiny $_{lb}$}‚

	  
	  \pstart \leavevmode% starting standard par
	न‚नु व्य‚व‚च्छेदोपि य‚दि प‚दार्थाद‚भिन्न‚स्त‚दैकेन प्र‚माणेन श‚ब्देन वास्य विष‚यी‚{\tiny $_{lb}$}‚क‚र‚णेन्य‚स्य वैय‚र्थ्यं स्यात् स‚र्वात्म‚ना निश्चित‚त्वात् । अथ भिन्न‚स्त‚दापि त‚स्याश्रित‚{\tiny $_{lb}$}‚त्वादेक‚व्य‚व‚च्छेदोपाधिके प‚दार्थे प्र‚माणेनैकेन निश्चीय‚माने पूर्वोक्तेन न्या‚{\tiny $_{२}$}‚येन‚{\tiny $_{lb}$}‚ स‚र्वेषां व्य‚व‚च्छेदानां निश्चित‚त्वाद‚न्येषां प्र‚माणादीनाम‚प्र‚वृत्तिः स्याद् [।] अतः‚{\tiny $_{lb}$}‚ स‚मानः प्र‚संग इति ।
	{\color{gray}{\rmlatinfont\textsuperscript{§~\theparCount}}}
	\pend% ending standard par
      ‚{\tiny $_{lb}$}‚

	  
	  \pstart \leavevmode% starting standard par
	त‚न्न । य‚तो न भावानाम‚न्योन्य‚व्य‚व‚च्छेदोऽभिन्नो भिन्नो वाऽस्ति । केव‚लं‚{\tiny $_{lb}$}‚ स्व‚हेतुभ्य एव भिन्नाः स‚मुत्प‚न्ना इत्युक्त‚म्व‚क्ष्य‚ति च ॥
	{\color{gray}{\rmlatinfont\textsuperscript{§~\theparCount}}}
	\pend% ending standard par
      ‚{\tiny $_{lb}$}‚

	  
	  \pstart \leavevmode% starting standard par
	क‚थ‚न्त‚र्ह्य‚न्य‚व्यावृत्तिरित्यादि व्य‚प‚देशो बुद्धिश्च प्र‚व‚र्त्त‚त इत्य‚त्राह । \textbf{त‚त्रा‚{\tiny $_{lb}$}‚पी}त्यादि । \textbf{त‚त्रा}पि च‚न्यापोहे श‚ब्दार्थे ।‚{\tiny $_{३}$}‚ \textbf{अन्य}स्माद् \textbf{व्यावृत्तिर‚न्य}स्माद् \textbf{व्यावृ}त्तो‚{\tiny $_{lb}$}‚य\textbf{मित्य‚पि} [।] ये \textbf{श‚ब्दा} ध‚र्म‚ध‚र्मिव‚च‚नाः \textbf{निश्च‚या}श्चोभ‚य‚विष‚यास्ते \textbf{संकेत‚म‚नुरुन्ध‚ते} ।‚{\tiny $_{lb}$}‚ संकेतानुविधानेनैषां ध‚र्म‚ध‚र्मिविष‚य‚विभागः क‚ल्पितः प‚र‚मार्थ‚त‚स्तु व्यावृत्तिरेव‚{\tiny $_{lb}$}‚ नास्तीत्य‚र्थः ।
	{\color{gray}{\rmlatinfont\textsuperscript{§~\theparCount}}}
	\pend% ending standard par
      ‚{\tiny $_{lb}$}‚

	  
	  \pstart \leavevmode% starting standard par
	त‚द्व्याच‚ष्टे । \textbf{त‚त्रान्यापोह} इत्यादिना गोर‚श्वाद् \textbf{व्यावृत्तिर‚न्या} ध‚र्म‚भूता \textbf{अन्य}‚{\tiny $_{lb}$}‚ एवाश्वाद् \textbf{व्यावृत्तो ध‚र्मी} । व्यावृत्त्या‚{\tiny $_{४}$}‚ विशिष्टो गौरित्येत\textbf{न्नास्ति} । किन्तु यैव‚{\tiny $_{lb}$}‚ व्यावृत्तिः स एव व्यावृत्त इति व‚क्ष्य‚ति ।
	{\color{gray}{\rmlatinfont\textsuperscript{§~\theparCount}}}
	\pend% ending standard par
      ‚{\tiny $_{lb}$}‚

	  
	  \pstart \leavevmode% starting standard par
	य‚दि चाश्वाद् व्यावृत्तिर‚न‚श्व‚ता गोद्र‚व्य‚स्यान्या स्यात् त‚दाश्व‚व्यावृत्तेर‚पि‚{\tiny $_{lb}$}‚ गोद्र‚व्येण निव‚र्त्तित‚व्य‚म्भेदात् । त‚त‚श्च \textbf{त‚द्व्यावृत्तेर‚न}श्व‚तायाः स‚काशा\textbf{न्निव‚र्त्त‚मा}न‚स्य‚{\tiny $_{lb}$}‚ गोस्त‚द्\textbf{भाव‚प्र‚स‚ङ्गात्} । अश्व‚भाव‚प्र‚स‚ङ्गाद‚श्व‚व‚त् । एवं ह्य‚श्व‚व्यावृत्तेर‚न‚श्व‚त्व‚{\tiny $_{lb}$}‚ल‚क्ष‚णाया‚{\tiny $_{५}$}‚ गौर्व्यावृत्तो भ‚व‚ति य‚द्य‚स्याश्व‚त्वं स्यात् । \textbf{त‚था} च गोर‚श्व‚भावाप‚त्तेः ।‚{\tiny $_{lb}$}‚ अश्वाद् गो\textbf{र्व्यावृत्ति}स्त‚स्या \textbf{अभा}वः । ग‚वाश्व‚योरेक‚त्वात् ।
	{\color{gray}{\rmlatinfont\textsuperscript{§~\theparCount}}}
	\pend% ending standard par
      ‚{\tiny $_{lb}$}‚

	  
	  \pstart \leavevmode% starting standard par
	तेन य‚दुक्त‚ञ्जैन‚जैमिनीयैः [।]
	{\color{gray}{\rmlatinfont\textsuperscript{§~\theparCount}}}
	\pend% ending standard par
      ‚{\tiny $_{lb}$}‚
	  \bigskip
	  \begingroup
	
	    
	    \stanza[\smallbreak]
	  {\normalfontlatin\large ``\qquad}स‚र्वात्म‚क‚मेकं स्याद‚न्यापोह‚व्य‚तिक्र‚म इति ।\edtext{\textsuperscript{*}}{\edlabel{pvsvt_143-1}\label{pvsvt_143-1}\lemma{*}\Bfootnote{\href{http://sarit.indology.info/?cref=\%C5\%9Bv}{श्लो० वा०}}}{\normalfontlatin\large\qquad{}"}\&[\smallbreak]
	  
	  
	  
	  \endgroup
	‚{\tiny $_{lb}$}‚

	  
	  \pstart \leavevmode% starting standard par
	तान्प्र‚तीद‚मुक्तं । य‚द्य‚न्य‚व्यावृत्तिर‚र्थान्त‚रं स्याद् ग‚वाश्वादीनामेक‚त्वं स्यादिति ।
	{\color{gray}{\rmlatinfont\textsuperscript{§~\theparCount}}}
	\pend% ending standard par
      ‚{\tiny $_{lb}$}‚‚{\tiny $_{lb}$}‚‚{\tiny $_{lb}$}‚\textsuperscript{\textenglish{144/s}}

	  
	  \pstart \leavevmode% starting standard par
	नापि गोर‚भिन्नाश्व‚व्यावॄत्तिर‚श्व‚व्यावृत्तौ गौरि‚{\tiny $_{६}$}‚त्य‚प्र‚तीतिप्र‚संगात् । गोविनाशे‚{\tiny $_{lb}$}‚ चाश्व‚स्योत्प‚त्तिप्र‚स‚ङ्गाद‚श्व‚निवृत्तेर्विन‚ष्ट‚त्वात् । त‚स्मान्नास्त्येव व्यावृत्तिः ।
	{\color{gray}{\rmlatinfont\textsuperscript{§~\theparCount}}}
	\pend% ending standard par
      ‚{\tiny $_{lb}$}‚

	  
	  \pstart \leavevmode% starting standard par
	\hphantom{.}तेन य‚दुच्य‚ते भ ट्टो द्यो त क रा भ्यां । योय‚म‚गोपोहः स किं ग‚वि भिन्नेऽर्थो‚{\tiny $_{lb}$}‚ भिन्नः । य‚दि भिन्नः किमाश्रितोऽथानाश्रितः [।] य‚द्याश्रित‚स्त‚दाश्रित‚त्वाद् गुण‚{\tiny $_{lb}$}‚ \leavevmode\ledsidenote{\textenglish{54b/PSVTa}} इति गोश‚ब्देन त‚दा गुण्य‚भिधीय‚ते न गोद्र‚व्य‚मिति । गौस्ति‚{\tiny $_{७}$}‚ष्ठ‚तीति सामानाधिक‚र‚ण्यं‚{\tiny $_{lb}$}‚ स्यात् । अथानाश्रितः केनार्थेन ष‚ष्ठ्य‚र्थः । गोर‚पोह इति । अथाभिन्नो गौरेव‚{\tiny $_{lb}$}‚ स्यादिति न किञ्चिद‚निष्ठं । अयं चापोहः प्र‚तिव‚स्तु य‚द्येकोनेक‚स‚म्ब‚न्धी च त‚देव‚{\tiny $_{lb}$}‚ गोत्व‚मि ति [।]
	{\color{gray}{\rmlatinfont\textsuperscript{§~\theparCount}}}
	\pend% ending standard par
      ‚{\tiny $_{lb}$}‚

	  
	  \pstart \leavevmode% starting standard par
	त‚न्निर‚स्तं । अन्य‚व्यावृत्तेरेवाभावात् केव‚लं स्व‚हेतुतः स्व‚कीयेन रूपेणोत्प‚न्नो‚{\tiny $_{lb}$}‚ भावोन्य‚स्माद् व्यावृत्त‚स्त‚स्य चान्य‚स्माद् व्यावृत्तिः क‚ल्प्य‚ते । य‚त‚श्च न प‚र‚मा‚{\tiny $_{lb}$}‚‚{\tiny $_{१}$}‚र्थ‚तो व्यावृत्तिर‚स्ति । \textbf{त‚स्माद् यैव व्यावृत्तिः स एव व्यावृत्तः} । द्वाभ्यामैक‚स्यैव‚{\tiny $_{lb}$}‚ विष‚यीक‚र‚णात् । त‚स्यैव चान्य‚व्यावृत्त‚स्य लिङ्ग‚त्वं लिङ्गित्वं स‚म्ब‚न्धो विक‚ल्प‚विष‚य‚{\tiny $_{lb}$}‚त्व‚ञ्च [।] विक‚ल्पो ह्य‚न्य‚व्यावॄत्तं स्वाकाराभिन्न‚म‚ध्य‚स्य पुरुष‚न्त‚त्र प्र‚व‚र्त्त‚य‚तीत्य‚{\tiny $_{lb}$}‚त्य‚र्थ‚कारित्वाद् [।] अतः स एव बाह्यः श‚ब्दार्थोन्य‚व्यावृत्तः ।
	{\color{gray}{\rmlatinfont\textsuperscript{§~\theparCount}}}
	\pend% ending standard par
      ‚{\tiny $_{lb}$}‚

	  
	  \pstart \leavevmode% starting standard par
	य‚द‚प्युच्य‚ते कु मा रि ले न [।] क‚दाचिदेक‚स्मादेव भाव‚स्यापो‚{\tiny $_{२}$}‚हः स्यात् ।‚{\tiny $_{lb}$}‚ स‚र्वास्माद्वा । य‚द्येक‚स्मादेव त‚दा य‚थाश्वापोह‚द्वारेण गोद्र‚व्य‚स्य गौरित्य‚भिधान‚{\tiny $_{lb}$}‚न‚न्त‚था सिंहादेर‚पि स्याद् अश्वापोह‚स्य गोश‚ब्द‚प्र‚वृत्तिनिमित्त‚स्य भावात् । त‚दाह ।
	{\color{gray}{\rmlatinfont\textsuperscript{§~\theparCount}}}
	\pend% ending standard par
      ‚{\tiny $_{lb}$}‚
	  \bigskip
	  \begingroup
	
	    
	    \stanza[\smallbreak]
	  {\normalfontlatin\large ``\qquad}त‚तोश्वापोह‚रूप‚त्वात् सिंहादिः स‚र्व एव ते ।&‚{\tiny $_{lb}$}‚त‚न्निमित्त‚म‚गोपोहं विभ्र‚दुच्येत गौरिती ति ।\edtext{\textsuperscript{*}}{\edlabel{pvsvt_144-1}\label{pvsvt_144-1}\lemma{*}\Bfootnote{\href{http://sarit.indology.info/?cref=\%C5\%9Bv}{ Ślokavārtika. 57 }}}{\normalfontlatin\large\qquad{}"}\&[\smallbreak]
	  
	  
	  
	  \endgroup
	‚{\tiny $_{lb}$}‚

	  
	  \pstart \leavevmode% starting standard par
	अथ स‚र्व‚स्माद‚पोहो गोद्र‚व्य‚स्य । त‚त्रापि य‚दि प्र‚त्येक‚म‚पोह्यं अश्वाद‚य‚स्त‚दा‚{\tiny $_{३}$}‚‚{\tiny $_{lb}$}‚पोह्यानामान‚न्त्याद‚पोह एव न सिध्येत् । अपोह्यानां च भिन्न‚त्वाद‚पोह‚भेदः प्र‚स‚{\tiny $_{lb}$}‚ज्य‚ते । त‚था चैक‚स्मिन्न‚पि पिण्डे जातिब‚हुत्वाज्जात्य‚न्त‚र‚बुद्धिः स्यात् । जात्य‚न्त‚रे‚{\tiny $_{lb}$}‚ष्विवाश्वादिषु ।
	{\color{gray}{\rmlatinfont\textsuperscript{§~\theparCount}}}
	\pend% ending standard par
      ‚{\tiny $_{lb}$}‚
	  \bigskip
	  \begingroup
	
	    
	    \stanza[\smallbreak]
	  {\normalfontlatin\large ``\qquad}त‚तो गौरिति सामान्यं वाच्य‚मेकं न सिध्य‚ति ।\edtext{\textsuperscript{*}}{\edlabel{pvsvt_144-2}\label{pvsvt_144-2}\lemma{*}\Bfootnote{\href{http://sarit.indology.info/?cref=\%C5\%9Bv}{ Ibid, 60 }}}{\normalfontlatin\large\qquad{}"}\&[\smallbreak]
	  
	  
	  
	  \endgroup
	‚{\tiny $_{lb}$}‚

	  
	  \pstart \leavevmode% starting standard par
	नापि ते स‚मुदाय‚रूपेण स‚र्वेऽपोह्याः स‚म्भ‚व‚न्ति । स‚मुदायो ह्येक‚देश‚त्वेन वा‚{\tiny $_{lb}$}‚ स्यान्न चापोह्यानामेक‚देशादित्वं स‚{\tiny $_{४}$}‚म्भ‚व‚ति । नापि तेषां स‚मुदायो व्य‚तिरिक्तो‚{\tiny $_{lb}$}‚ऽस्त्य‚व्य‚तिरेके चान‚न्त्यं त‚द‚व‚स्थं । न चापि सामान्य‚रूपेण तेऽपोह्याः सामान्य‚{\tiny $_{lb}$}‚‚{\tiny $_{lb}$}‚ ‚{\tiny $_{lb}$}‚ ‚{\tiny $_{lb}$}‚ \leavevmode\ledsidenote{\textenglish{145/s}}स्याव‚स्तुत्वात् । अपोह्य‚त्वे च व‚स्तुत्वं स्यादिति ।
	{\color{gray}{\rmlatinfont\textsuperscript{§~\theparCount}}}
	\pend% ending standard par
      ‚{\tiny $_{lb}$}‚

	  
	  \pstart \leavevmode% starting standard par
	त‚द‚युक्तं य‚तः [।] स‚र्व‚भावानां स्वेनैव स्वेनैव रूपेणोत्प‚द्य‚मानानां स‚र्व‚स्मा‚{\tiny $_{lb}$}‚द‚पोहः स्व‚हेतुभ्यः सिद्ध एव ।
	{\color{gray}{\rmlatinfont\textsuperscript{§~\theparCount}}}
	\pend% ending standard par
      ‚{\tiny $_{lb}$}‚

	  
	  \pstart \leavevmode% starting standard par
	अथ क‚थ‚म‚सौ ज्ञाय‚त इति चोद्य‚ते । त‚त्किङ्गौर‚तीतानाग‚{\tiny $_{५}$}‚त‚व‚र्त्त‚मानाऽर्श्वा‚{\tiny $_{lb}$}‚दिस्व‚भावः प्र‚त्य‚क्षे प्र‚तिभास‚ते । नेति चेत् । क‚थं न त‚त्र स‚र्वापोहः प्र‚त्य‚क्ष‚सिद्धः ।‚{\tiny $_{lb}$}‚ न हि प्र‚माणं ह‚स्ताभ्याङ्गृहीत्वान्य‚द‚पोह‚त्य‚पि तु निय‚त‚रूपार्थ‚प्र‚काश‚न‚मेवास्यान्या‚{\tiny $_{lb}$}‚पोहं । त‚स्मान्निय‚त‚रूपार्थ‚प्र‚तिभास एव प्र‚त्य‚क्ष‚स्य स‚र्व‚स्माद‚पोह‚ग्र‚हः । त‚च्च‚{\tiny $_{lb}$}‚ स्व‚विष‚य‚न्निश्चाय‚य‚द् य‚देवं न भ‚व‚ति त‚त्स‚र्व‚म‚न्य‚त्वेन नि‚{\tiny $_{६}$}‚श्चाय‚य‚त्य‚तो युग‚प‚त्स‚र्व‚{\tiny $_{lb}$}‚स्यान्य‚स्य सामान्येनाविशेषेण निषेधः क्रिय‚ते । सामान्य‚स्यानिर्द्धारित‚विशेष‚{\tiny $_{lb}$}‚रूप‚त्वात् । त‚दुक्त‚म् [।]
	{\color{gray}{\rmlatinfont\textsuperscript{§~\theparCount}}}
	\pend% ending standard par
      ‚{\tiny $_{lb}$}‚
	  \bigskip
	  \begingroup
	
	    
	    \stanza[\smallbreak]
	  {\normalfontlatin\large ``\qquad}अत‚द्रूप‚प‚रावृत्त‚व‚स्तुमात्र‚प्र‚साध‚नात् [।]&‚{\tiny $_{lb}$}‚सामान्य‚विष‚यं प्रोक्तं लिङ्ग‚भेदाप्र‚तिष्ठितेरिति \href{http://sarit.indology.info/?cref=pva.2.144}{। प्र० स०}{\normalfontlatin\large\qquad{}"}\&[\smallbreak]
	  
	  
	  
	  \endgroup
	‚{\tiny $_{lb}$}‚

	  
	  \pstart \leavevmode% starting standard par
	तेनापोह्य‚स्य क‚स्य‚चिद् व‚स्तुत्व‚मिष्य‚त एव ।
	{\color{gray}{\rmlatinfont\textsuperscript{§~\theparCount}}}
	\pend% ending standard par
      ‚{\tiny $_{lb}$}‚

	  
	  \pstart \leavevmode% starting standard par
	न चापोह्य‚त्वाद् व‚स्तुत्व‚मित्य‚त्र किञ्चिद् प्र‚माण‚म‚स्त्य‚भाव‚स्याप्य‚पोह्य‚त्वान्न‚{\tiny $_{lb}$}‚ चास्य‚{\tiny $_{७}$}‚ व‚स्तुत्व‚मित्युक्तं । त‚स्माद् युग‚प‚त् स‚र्वापोह‚ल‚क्ष‚णेनागोपोहेनैक‚स्मिन्न‚पि \leavevmode\ledsidenote{\textenglish{55a/PSVTa}}‚{\tiny $_{lb}$}‚ पिण्डे गोत्वं । प्र‚त्येकाश्वाद्य‚पोहेनान‚श्व‚त्वासिंह‚त्वाम‚हिष‚त्वाद‚यो जातिभेदाः क‚ल्पिता‚{\tiny $_{lb}$}‚स्त‚द्द्वारेण च त‚द‚भिधाय‚काः प्र‚व‚र्त्त‚न्त इति य‚त्किञ्चिदेत‚त् ।
	{\color{gray}{\rmlatinfont\textsuperscript{§~\theparCount}}}
	\pend% ending standard par
      ‚{\tiny $_{lb}$}‚

	  
	  \pstart \leavevmode% starting standard par
	य‚दि व्य‚वृत्तिव्यावृत्ताऽभिधेयार्थ‚स्य न भेदः । क‚थं व्यावृत्तिव्यावृत्त इति‚{\tiny $_{lb}$}‚ श‚ब्द‚ज्ञान‚भेदः । त‚था हि व्यावृत्तिरित्य‚न्यः श‚ब्दो व्यावृत्त इत्य‚न्य एव श‚{\tiny $_{१}$}‚ब्दः ।‚{\tiny $_{lb}$}‚ त‚था ज्ञान‚भेदोपि [।] व्यावृत्तिरित्युक्ते ध‚र्म‚मात्र‚म्प्र‚तीय‚ते । व्यावृत्त इति ध‚र्मीति ।‚{\tiny $_{lb}$}‚ त‚त आह[।] \textbf{श‚ब्दे}त्यादि । श‚ब्दाद् ध‚र्म‚ध‚र्मिवाचिनो या प्र‚तीतिः सा श‚ब्द‚प्र‚तिप‚त्तिः ।‚{\tiny $_{lb}$}‚ श‚ब्द‚श्च \textbf{श‚ब्द‚प्र‚तिप‚त्ति}श्चेति विरूपैक‚शेषः । श‚ब्द\textbf{भेदः} श‚ब्दाच्च या प्र‚तिप‚त्तिस्त‚स्या‚{\tiny $_{lb}$}‚भेद इत्य‚र्थः संज्ञासंज्ञिस‚म्ब‚न्धिक‚र‚णं \textbf{संके}त‚स्त‚स्य \textbf{भेदा}त् । संकेत‚भेदं चान‚न्त‚र‚मेव‚{\tiny $_{lb}$}‚ \href{http://sarit.indology.info/?cref=}{१ । ६३} \textbf{भेदान्त‚र‚प्र‚तिक्षे}‚{\tiny $_{२}$}‚पेत्यादिना प्र‚तिपाद‚यिष्य‚ते । \textbf{न वाच्य‚भेदोस्ति} ध‚र्म‚ध‚र्मि‚{\tiny $_{lb}$}‚श‚ब्द‚योर्व‚स्तुत इत्य‚ध्याहारः ॥
	{\color{gray}{\rmlatinfont\textsuperscript{§~\theparCount}}}
	\pend% ending standard par
      ‚{\tiny $_{lb}$}‚

	  
	  \pstart \leavevmode% starting standard par
	\textbf{न‚नु चेत्यादि} प‚रः । किं पुन‚र्वाच्याविशेषे \textbf{संकेत‚भेदो न युक्त} इति चेदाह ।‚{\tiny $_{lb}$}‚ \textbf{द्व‚योरि}त्यादि । क‚र्त्त‚रि चेयं ष‚ष्ठी । क‚र्त्तृ क‚र्म‚णोः कृतीति  उभ‚य‚प्राप्तौ क‚र्म‚णीति‚{\tiny $_{lb}$}‚ ‚{\tiny $_{lb}$}‚ \leavevmode\ledsidenote{\textenglish{146/s}}निय‚म‚स्य शेषे विभाषेति\edtext{}{\edlabel{pvsvt_146-1}\label{pvsvt_146-1}\lemma{विभाषेति}\Bfootnote{\href{http://sarit.indology.info/?cref=P\%C4\%81.2.3.65-66}{ Pāṇini. 2: 3: 65. }}} विक‚ल्प‚नात् । द्वाभ्यां ध‚र्म‚ध‚र्मिश‚ब्दाभ्या\textbf{मेक}स्यार्थ‚स्या‚{\tiny $_{lb}$}‚\textbf{भिधाना}दित्य‚र्थः । एकं चेद् द्वाभ्याम‚भिधेय‚न्त‚{\tiny $_{३}$}‚तो व्य‚र्थः संकेतः । \textbf{त‚था चे}ति ध‚र्म‚{\tiny $_{lb}$}‚ध‚र्मिणोर‚भेदे \textbf{व्य‚तिरेकिण्या} इति व्य‚तिरेकाभिधायिन्यां गोर्गोत्व‚मिति ष‚ष्ठ्याः ।‚{\tiny $_{lb}$}‚ \textbf{त‚स्या} इति व्य‚तिरेक‚विभ‚क्ते\textbf{र्भेदाश्र‚य‚त्वाद् व‚स्तुभेद‚माश्रि}त्य प्र‚वृत्तेः । य‚था देव‚द‚त्त‚स्य‚{\tiny $_{lb}$}‚ क‚म‚ण्ड‚लुरिति । एवं संकेताभावे व्य‚तिरेक‚विभ‚क्त्य‚भावे च चोदिते ।
	{\color{gray}{\rmlatinfont\textsuperscript{§~\theparCount}}}
	\pend% ending standard par
      ‚{\tiny $_{lb}$}‚

	  
	  \pstart \leavevmode% starting standard par
	विभ‚क्त्य‚भाव‚दोष‚न्ताव‚त्प‚रिह‚र‚न्नाह । \textbf{द्व‚यो}रित्यादि । ध‚र्म‚ध‚र्मिवाचिनोः श‚ब्द‚{\tiny $_{४}$}‚‚{\tiny $_{lb}$}‚यो\textbf{रेक}स्यार्थ‚स्या\textbf{भिधानेपि विभ‚क्तिर्व्य‚तिरेकिणी} । व्य‚तिरेक‚स्य वाचिका ष‚ष्ठी । इव‚{\tiny $_{lb}$}‚ श‚ब्दो भिन्न‚क्र‚मः । \textbf{भिन्न‚मिवार्थ‚म‚न्वेति} द‚र्श‚य‚ति । \textbf{वाच्ये} संकेत‚भेद‚कृतेन \textbf{ले}शेन‚{\tiny $_{lb}$}‚ मात्र‚या यो \textbf{विशेष‚स्त}तः कार‚णान्न तु प‚र‚मार्थ‚तो व‚स्तुभेदात् ।
	{\color{gray}{\rmlatinfont\textsuperscript{§~\theparCount}}}
	\pend% ending standard par
      ‚{\tiny $_{lb}$}‚

	  
	  \pstart \leavevmode% starting standard par
	य‚द्व्याच‚ष्टे । \textbf{न वै श‚ब्दानामि}त्यादिना । \textbf{विष‚य‚स्व‚भावाय‚त्ते}ति बाह्य‚स्व‚ल‚क्ष‚{\tiny $_{lb}$}‚णाय‚त्ता [।] किं कार‚ण‚म् [।] \textbf{इच्छातः} पुरुषे‚{\tiny $_{५}$}‚च्छाव‚शाद\textbf{भावेष्व‚पि} वृत्त्य‚भाव\textbf{प्र‚स‚{\tiny $_{lb}$}‚ङ्गात् । त} इति । इच्छाप्र‚तिब‚द्ध‚वृत्त‚यः श‚ब्दा \textbf{य‚था} येन प्र‚कारेण भेद‚प्र‚तिपाद‚नेन‚{\tiny $_{lb}$}‚ व्य‚तिरिक्ते य‚था राज्ञः पुरुष इति । अव्य‚तिरिक्ते य‚थात्मैव ह्यात्म‚नो द्र‚ष्टेति । \textbf{त‚था‚{\tiny $_{lb}$}‚ नियुक्ता} इत्य‚भिन्नेप्य‚र्थे भेद‚मिवोपादाय प्र‚युक्ता\textbf{स्त‚म‚र्थ‚म}प्र‚तिब‚न्धेन \textbf{भिन्न‚मिव प्र‚का‚{\tiny $_{lb}$}‚श‚य}न्ति । व‚स्तुतः स्व‚ल‚क्ष‚ण‚स्याभेदेपि य‚त एव‚न्तेन कार‚णेन । \textbf{गौ‚{\tiny $_{६}$}‚रिति} ध‚र्मि‚{\tiny $_{lb}$}‚वाचिन‚माह । \textbf{गोत्व}मिति ध‚र्म‚वाचिनं । आभ्या\textbf{मेकाभिधाने}प्य‚गोव्यावृत्त‚स्य‚{\tiny $_{lb}$}‚ गोर‚भिधानेपि \textbf{क‚स्य‚चिद्} विष‚य‚स्य \textbf{प्र‚त्याय‚नार्थ}मिति । अगोव्यावृत्तिनिमित्त‚स्य‚{\tiny $_{lb}$}‚ गोत्व‚स्य प्र‚काश‚नार्थं । अगोव्यावृत्तिमात्रं गोत्व‚श‚ब्देन प्र‚तिपाद्य‚मित्येवंकृते संकेते‚{\tiny $_{lb}$}‚ भेदे । व्य‚तिरिक्तार्था न विभ‚क्तिर‚स्य गोत्व‚मिति भ‚व‚ति ष‚ष्ठी । व्य‚तिरिक्तोऽर्थोस्या‚{\tiny $_{lb}$}‚ \leavevmode\ledsidenote{\textenglish{55b/PSVTa}} इति विग्र‚हः । ध‚र्मिण‚स्स‚{\tiny $_{७}$}‚काशाद् ध‚र्म‚म\textbf{र्थान्त‚र‚मिवाद‚र्श‚य‚न्ती प्र‚तिभाति । अन‚{\tiny $_{lb}$}‚‚{\tiny $_{lb}$}‚ ‚{\tiny $_{lb}$}‚ \leavevmode\ledsidenote{\textenglish{147/s}}र्थान्त‚रे}पीत्य‚व्य‚तिरिक्तेपि ध‚र्मे [।] किं कार‚णं [।] \textbf{त‚था प्र‚योग‚द‚र्श‚नाभ्यासात्} ।‚{\tiny $_{lb}$}‚ व‚स्तुभेदे स‚ति ष‚ष्ठ्याः प्र‚योग‚द‚र्श‚नाभ्यासाद् देव‚द‚त्त‚स्य क‚म‚ण्ड‚लुरित्यादौ ।
	{\color{gray}{\rmlatinfont\textsuperscript{§~\theparCount}}}
	\pend% ending standard par
      ‚{\tiny $_{lb}$}‚

	  
	  \pstart \leavevmode% starting standard par
	एत‚दुक्त‚म्भ‚व‚ति । व‚स्तुभिन्न‚म्भ‚व‚तु मा वा भूत् स‚र्व‚था व्य‚तिरेक‚विभ‚क्ति‚{\tiny $_{lb}$}‚रिच्छामात्रानुरोधिनी केव‚लं प्र‚योग‚द‚र्श‚नाभ्यासाच्छ‚ब्दार्थ‚म्भिन्न‚मिव द‚र्श‚य‚तीति ।‚{\tiny $_{lb}$}‚ \textbf{ताव‚ते}ति विभ‚क्तिप्र‚यो‚{\tiny $_{१}$}‚ग‚मात्रात् । \textbf{स‚र्व‚त्रे}त्य‚र्थाभेदेपि । गौर्गोत्व‚मित्यादौ न ध‚र्म‚{\tiny $_{lb}$}‚ध‚र्मिणोः प‚र‚मार्थ‚तो \textbf{भेदः} । त‚स्मा\textbf{द‚न्य‚त्रा}प्य‚र्थाभेदेपि \textbf{पुरुषेच्छाव‚शात् प्र‚वृत्त‚स्य}‚{\tiny $_{lb}$}‚ व्य‚तिरेकाभिधायिनः श‚ब्द‚स्य \textbf{प्र‚तिब‚न्धाभावात्} । दृष्टा च पुरुषेच्छाव‚शाच्छ‚{\tiny $_{lb}$}‚ब्दानां प्र‚वृत्तिर‚स‚त्य‚पि त‚थाभूते बाह्ये वाच्य इत्याह । \textbf{य‚थेत्या}दि । \textbf{एक‚म्}व‚स्तु‚{\tiny $_{lb}$}‚ \textbf{क्व‚चि}त् प्र‚क‚र‚णे \textbf{एक‚व‚च‚ने}न \textbf{ख्याप्य}ते । य‚था त्व‚मिति । \textbf{त‚द‚विशेषे‚{\tiny $_{२}$}‚पि} एक‚त्वा‚{\tiny $_{lb}$}‚विशेषेपि त‚देव व‚स्तु \textbf{ब‚हुव‚च‚नेन} यूय‚मिति । अत‚श्चैक‚स्मिन्न‚पि ब‚हुव‚च‚न‚द‚र्श‚नान्न‚{\tiny $_{lb}$}‚ य‚थाव‚स्तु श‚ब्दानाम्प्र‚वृत्तिरिति ग‚म्य‚ते । युष्म‚दि गुरावेकेषा\edtext{}{\edlabel{pvsvt_147-1}\label{pvsvt_147-1}\lemma{गुरावेकेषा}\Bfootnote{\href{http://sarit.indology.info/?cref=P\%C4\%81.1.2.58}{ Pāṇini 1: 2: 58. }}}मित्य‚तिदेश‚वाक्याद्‚{\tiny $_{lb}$}‚ एक‚स्मिन्न‚पि ब‚हुव‚च‚न‚मिति चेत् । चिन्त्य‚मेत‚त् । किम‚तिदेश‚वाक्येंनैक‚स्य ब‚हुत्वं‚{\tiny $_{lb}$}‚ क्रिय‚ते किम्वा ब‚हुव‚च‚न‚मात्र‚म‚प्राप्तं विधीय‚त इति [।] न ताव‚दाद्यः प‚क्षो व‚च‚न‚{\tiny $_{lb}$}‚मात्रेण‚{\tiny $_{३}$}‚ व‚स्तूनाम्विधानास‚म्भ‚वात् । द्वितीयेपि प‚क्षे सिद्धैवेच्छामात्रेण श‚ब्दानां‚{\tiny $_{lb}$}‚ प्र‚वृत्तिरिति । एव‚न्ताव‚द्विभ‚क्त्य‚भाव‚दोषः प‚रिहृतः [।]
	{\color{gray}{\rmlatinfont\textsuperscript{§~\theparCount}}}
	\pend% ending standard par
      ‚{\tiny $_{lb}$}‚

	  
	  \pstart \leavevmode% starting standard par
	संकेताभाव‚दोष‚न्तु प‚रिह‚र्त्तुन्त‚मेवोप‚न्य‚स्य‚ति [।] \textbf{प्र‚योज‚नाभावा}त् त्वित्यादि ।‚{\tiny $_{lb}$}‚ ध‚र्मिध‚र्म‚श‚ब्दाभ्यामेक‚स्याभिधानात् प्र‚योज‚नाभावः । \textbf{त‚द‚पि} प्र‚योज‚न‚म\textbf{स्त्ये}व ।‚{\tiny $_{lb}$}‚ गोत्वापेक्ष‚या \textbf{भेदान्त}राणि द्र‚व्य‚त्व‚पार्थिव‚त्वादीनि । \textbf{तेषां} प्र‚ति‚{\tiny $_{४}$}‚क्षेपोऽस्वीकारः ।‚{\tiny $_{lb}$}‚ तौ \textbf{प्र‚तिक्षेपाप्र‚तिक्षेपौ} य‚थाक्र‚म‚न्त‚यो\textbf{र्द्व‚यो}द्ध‚र्म‚ध‚र्मिवाचिनोः श‚ब्द‚योर्यः \textbf{संकेत‚भेद}‚{\tiny $_{lb}$}‚स्त‚स्य [।] किम्विशिष्ट‚स्य \textbf{ज्ञातृवाञ्छानुरोधिनः} प्र‚तिप‚त्त्रिच्छानुविधायिनः प‚दं‚{\tiny $_{lb}$}‚ प्र‚योज‚नं ॥
	{\color{gray}{\rmlatinfont\textsuperscript{§~\theparCount}}}
	\pend% ending standard par
      ‚{\tiny $_{lb}$}‚

	  
	  \pstart \leavevmode% starting standard par
	एत‚दुक्त‚म्भ‚व‚ति । य‚दान्य‚व्याव‚त्त‚रूप‚निराकाङ्क्षः प्र‚तिप‚त्ताश्वादेवैक‚स्माद्‚{\tiny $_{lb}$}‚ ‚{\tiny $_{lb}$}‚ ‚{\tiny $_{lb}$}‚ \leavevmode\ledsidenote{\textenglish{148/s}}व्यावृत्तं गोपिण्डं जिज्ञास‚ते त‚दा याव‚द‚श्वाद् व्या‚{\tiny $_{५}$}‚वृत्तिम‚र्थान्त‚र‚भूतामारोप्य त‚थैव‚{\tiny $_{lb}$}‚ संकेत‚पूर्व‚कं लौकिकेन ध‚र्म‚वाच‚केन श‚ब्देन न क‚थ‚य‚ति [।] ताव‚न्न प‚र‚स्य जिज्ञा‚{\tiny $_{lb}$}‚सितोर्थः प्र‚तिपाद‚यितुं श‚क्य‚ते [।] अत‚स्तं प्र‚त्य‚न‚श्व‚त्व‚म‚स्येत्युच्य‚ते । एवं हि ध‚र्मि‚{\tiny $_{lb}$}‚णोऽप्राधान्याद‚न्य‚व्यावृत्त‚रूपानाक्षेपः कृतो भ‚व‚त्य‚श्वादेवैक‚स्माद् व्यावृत्तिश्च ।‚{\tiny $_{lb}$}‚ न त‚दैव‚मुच्य‚तेऽन‚श्व इत्य‚नेन ह्य‚न्य‚व्यावृत्त‚स्यापि रूप‚स्याक्षेपः‚{\tiny $_{६}$}‚ कृतः स्यात् [।]‚{\tiny $_{lb}$}‚ न चैव‚म्प‚रेण जिज्ञासित‚म‚जिज्ञासितं च क‚थ‚य‚न् क‚थ‚न्नोन्म‚त्तः स्यात् ।
	{\color{gray}{\rmlatinfont\textsuperscript{§~\theparCount}}}
	\pend% ending standard par
      ‚{\tiny $_{lb}$}‚

	  
	  \pstart \leavevmode% starting standard par
	य‚दा पुन‚र‚न्य‚व्यावृत्त‚रूप‚साकांक्षेऽश्वाद् व्यावृत्तं गोपिण्डं जिज्ञास‚ते । त‚दापि‚{\tiny $_{lb}$}‚ याव‚द‚श्व‚व्यावृत्तिविशिष्टं पिण्डं ध‚र्मिस्व‚भाव‚त‚याऽरोप्य त‚थैव संकेत‚पूर्व‚कं लौकि‚{\tiny $_{lb}$}‚केन ध‚र्मिवाच‚केन न क‚थ‚य‚ति ताव‚न्न प‚र‚स्य जिज्ञासितोर्थः प्र‚तिपाद‚यितुं श‚क्य‚तेऽ‚{\tiny $_{lb}$}‚\leavevmode\ledsidenote{\textenglish{56a/PSVTa}} त‚स्तं प्र‚त्य‚न‚श्वो‚{\tiny $_{७}$}‚य‚मित्युच्य‚ते । एवं हि ध‚र्मिणः प्राधान्याद‚न्य‚व्यावृत्त‚रूपाक्षेपः‚{\tiny $_{lb}$}‚ कृतो भ‚व‚त्य‚श्व‚व्यावृत्त‚श्च गोपिण्डः क‚थितो भ‚व‚ति । न त‚दैवं ख्याप्य‚तेऽन‚श्व‚त्व‚{\tiny $_{lb}$}‚म‚स्येति प‚र‚जिज्ञासितान्य‚व्यावृत्त‚रूपानाक्षेप‚प्र‚स‚ङ्गात् । अजिज्ञासितं चार्थं क‚थ‚{\tiny $_{lb}$}‚य‚न् क‚थ‚न्नोन्म‚त्तः । स‚र्व‚श्च शाब्दो व्य‚व‚हारः संकेत‚पूर्व‚कः संकेत‚श्च विक‚ल्प‚क‚ल्पि‚{\tiny $_{lb}$}‚तार्थ‚पूर्व‚क एवेति विक‚ल्पैर‚{\tiny $_{१}$}‚प्य‚नेनैव द्वारेण ध‚र्म‚ध‚र्मिभाव‚प्र‚तीतिर्युक्ता ।
	{\color{gray}{\rmlatinfont\textsuperscript{§~\theparCount}}}
	\pend% ending standard par
      ‚{\tiny $_{lb}$}‚

	  
	  \pstart \leavevmode% starting standard par
	तेन य‚दुच्य‚ते [।] भ‚व‚तु ध‚र्म‚ध‚र्मिवाच‚कानां भेदान्त‚र‚प्र‚तिक्षेपाप्र‚तिक्षेपार्थ‚{\tiny $_{lb}$}‚प्र‚वृत्तिः । ध‚र्मिध‚र्म‚विक‚ल्पानान्तु क‚थं प्र‚तिप‚त्तिरित्य‚पास्तं ।
	{\color{gray}{\rmlatinfont\textsuperscript{§~\theparCount}}}
	\pend% ending standard par
      ‚{\tiny $_{lb}$}‚

	  
	  \pstart \leavevmode% starting standard par
	एत‚देव वृत्त्या स्प‚ष्ट‚य‚न्नाह । \textbf{य‚दाय}मित्यादि । \textbf{प्र‚तिप‚त्ते}ति श्रोता । त‚स्माद्‚{\tiny $_{lb}$}‚ अश्वाद्योऽ\textbf{न्यो} म‚हिषादिस्त‚स्माद् \textbf{व्य‚व‚च्छे}दो म‚हिषादिव्यावृत्तः स्व‚भाव‚स्त‚स्य \textbf{भावा‚{\tiny $_{lb}$}‚न‚पेक्षः} स‚{\tiny $_{२}$}‚त्तान‚पेक्षः । \textbf{पिण्ड‚विशे}षे ग‚वि । \textbf{अश्व‚व्य‚व‚च्छेद‚मात्रं जिज्ञास‚ते} । किम‚{\tiny $_{lb}$}‚स्याश्वाद् व्यावृत्तं रूप‚म‚स्तीति । \textbf{त‚थाभूत‚ज्ञाप‚नार्थ}मिति य‚था प्र‚तिप‚त्त्रा ज्ञातुमिष्ट‚{\tiny $_{lb}$}‚न्त‚द‚नुरोधेन त‚थाभूत‚स्याश्वाद् भेद‚मात्र‚स्य ज्ञाप‚नार्थ‚न्त\textbf{थाकृत‚संके}तेनेत्य‚श्व‚व्य‚व‚च्छे‚{\tiny $_{lb}$}‚द‚मात्रे प्र‚तिक्षिप्त‚भेदान्त‚रे कृत‚संकेतेनान‚श्व‚त्वं \textbf{श‚ब्देन प्र‚बोध्य‚ते} प्र‚काश्य‚तेऽन‚श्व‚त्व‚{\tiny $_{lb}$}‚म‚स्य पिण्ड‚स्यास्तीति । अश्वा‚{\tiny $_{३}$}‚द्यो व्य‚व‚च्छेद‚स्त‚द‚पेक्ष‚या म‚हिषादिभ्यो व्यावृ‚{\tiny $_{lb}$}‚त्त‚योर्\textbf{व्य‚व‚च्छेदान्त}राणि । तेष्व‚नि\textbf{राकां}क्षः प्र‚तिप‚त्ता । \textbf{त‚मि}ति पिण्डं ।‚{\tiny $_{lb}$}‚ अप्र‚तिक्षिप्त‚भेदान्त‚रेणाश्व‚व्य‚व‚च्छेदेन युक्त‚ङ् गोद्र‚व्यं ज्ञातुमिच्छ‚तीति याव‚त् ।‚{\tiny $_{lb}$}‚ अप‚रित्य‚क्तानि म‚हिषादिव्य‚व‚च्छेदान्त‚राणि येन । त‚स्मिन्न\textbf{प‚रित्य‚क्त‚भेदान्त‚रे} ।‚{\tiny $_{lb}$}‚ त‚त्रैव वाश्व‚व्य‚व‚च्छेदे । ध‚र्मिवाचिनं श‚ब्दं \textbf{प्र‚युञ्ज‚ते} व‚क्तारोऽ\textbf{न‚श्वोय‚मि‚{\tiny $_{४}$}‚ति} ।‚{\tiny $_{lb}$}‚ क‚थं प्र‚युञ्ज‚त इत्याह । \textbf{त‚था प्र‚काश‚नाये}ति । अप्र‚तिक्षिप्त‚भेदान्त‚र‚स्याश्व‚व्य‚व‚{\tiny $_{lb}$}‚‚{\tiny $_{lb}$}‚ \leavevmode\ledsidenote{\textenglish{149/s}}व‚च्छेद‚स्य प्र‚काश‚नाय । \textbf{अप्र‚तिक्षिप्त‚भेदान्त}र‚मेवाश्व‚व्य‚व‚च्छेद‚न्त‚था प्र‚काश‚नाये‚{\tiny $_{lb}$}‚त्य‚न्ये प‚ठ‚न्ति । त‚दाप्य‚य‚म‚र्थः । अप्र‚तिक्षिप्त‚भेदान्त‚र‚न्त‚मेवाश्व‚व्य‚व‚च्छेद‚म‚श्व‚{\tiny $_{lb}$}‚व्यावृत्तिरूपं प्र‚युञ्ज‚ते अभिद‚ध‚त्य‚न‚श्वोय‚मित्य‚नेन ध‚र्मिव‚च‚नेन श‚ब्देन । किम‚{\tiny $_{lb}$}‚र्थ‚म्[।]त‚थाप्र‚काश‚{\tiny $_{५}$}‚नायाप्र‚तिक्षिप्त‚भेदान्त‚र‚स्य प्र‚काश‚नायेति । येनैव ध‚र्म‚वाची‚{\tiny $_{lb}$}‚ श‚ब्दः प्र‚तिक्षिप्त‚भेदान्त‚रः । \textbf{अत एव पूर्व}त्रेति ध‚र्म‚वाचिनि श‚ब्दे \textbf{प्र‚तिक्षिप्त‚म्भे‚{\tiny $_{lb}$}‚दान्त‚रं} येनेति । सामान्येनान्य‚प‚दार्थं कृत्वा भाव‚प्र‚त्य‚यः क‚र्त्त‚व्यः । प‚श्चा\textbf{च्छ‚ब्द‚{\tiny $_{lb}$}‚वृत्तेरित्}य‚नेन स‚म्ब‚न्धः । अन्य‚था प्र‚तिक्षिप्तं भेदान्त‚र‚त्वादिति स्यात् । एव‚म‚न्य‚{\tiny $_{lb}$}‚त्राप्येवंजातीयेषु श‚ब्देषु व्युत्प‚त्तिर्द्र‚{\tiny $_{६}$}‚ष्ट‚व्या ।
	{\color{gray}{\rmlatinfont\textsuperscript{§~\theparCount}}}
	\pend% ending standard par
      ‚{\tiny $_{lb}$}‚

	  
	  \pstart \leavevmode% starting standard par
	भिन्न‚निमित्त‚योः श‚ब्द‚योरेक‚स्मिन्न‚धिक‚र‚णे वृत्तिः \textbf{सामानाधिक‚र‚ण्यं । विशेष्}य‚{\tiny $_{lb}$}‚\textbf{विशेष‚ण‚भावो} व्य‚व‚च्छेद्य‚व्य‚व‚च्छेद‚क‚भावः । उदाह‚र‚ण‚ङ् \textbf{गोत्व‚म‚स्य शुक्ल‚मिति} ।‚{\tiny $_{lb}$}‚ गुण‚श‚ब्द‚स्याभिधेय‚व‚ल्लिङ्ग‚व‚त्त्वेन न‚पुंस‚क‚त्वं [।] शुक्ल इत्य‚न्ये प‚ठ‚न्ति । एवं‚{\tiny $_{lb}$}‚ चाच‚क्ष‚ते [।] गुण‚श‚ब्दो हि प्र‚तिक्षिप्त‚भेदान्त‚रेण गुण‚मात्रे व‚र्त्त‚मान उपात्तो‚{\tiny $_{lb}$}‚ गु‚{\tiny $_{७}$}‚ण‚मात्र‚वृत्तीनां शुक्लादिश‚ब्दानां पुल्लिङ्ग‚त्वं । त‚द्व‚ति तु व‚र्त्त‚मानानाम‚भिधेय- \leavevmode\ledsidenote{\textenglish{56b/PSVTa}}‚{\tiny $_{lb}$}‚ व‚ल्लिङ्ग‚ता । एवं चान‚योर्द्ध‚र्म‚मात्र‚वृत्त्योर्न सामानाधिक‚र‚ण्यं नापि विशेष‚ण‚{\tiny $_{lb}$}‚विशेष्य‚भाव इति ।
	{\color{gray}{\rmlatinfont\textsuperscript{§~\theparCount}}}
	\pend% ending standard par
      ‚{\tiny $_{lb}$}‚

	  
	  \pstart \leavevmode% starting standard par
	क‚स्मान्न सामानाधिक‚र‚ण्य‚मित्याह । \textbf{त‚न्मात्रे}त्यादि । एत‚त्क‚थ‚य‚ति बुद्धि‚{\tiny $_{lb}$}‚प्र‚तिभासिन्येवार्थ‚सामानाधिक‚र‚ण्यादि । न बाह्ये स्व‚ल‚क्ष‚णे त‚स्यावाच्य‚त्वात्‚{\tiny $_{lb}$}‚ [।] केव‚ल‚म‚ध्य‚व‚सा‚{\tiny $_{१}$}‚याद् बाह्येप्युच्य‚ते । य‚दि च ध‚र्म‚द्व‚य‚युक्तैक‚ध‚र्मिप्र‚तिभा‚{\tiny $_{lb}$}‚सिनी श‚ब्द‚द्व‚य‚ज‚निता बुद्धिरेकार्थोत्प‚द्येत भ‚वेत्सामानाधिक‚र‚ण्यं । इह तु \textbf{त‚न्मात्र‚{\tiny $_{lb}$}‚विशेषेण} प्र‚तिक्षिप्त‚भेदान्त‚रेण गोत्व‚मात्र‚विशेष‚णोप‚र‚क्ताया \textbf{बुद्धेस्त‚दाश्र‚य‚भूताया}‚{\tiny $_{lb}$}‚ इति विशेष‚ण‚विशेष्य‚भावः सामानाधिक‚र‚ण्याश्र‚य‚भूताया \textbf{एक‚त्वे}न ध‚र्म्य‚भेदेना\textbf{प्र‚ति‚{\tiny $_{lb}$}‚भास‚नात्} । गो‚{\tiny $_{२}$}‚त्व‚शुक्ल‚त्वाभ्यां युक्त‚मेक‚न्ध‚र्मिणं गृहीत्वा बुद्धेर‚प्र‚तिभास‚नादि‚{\tiny $_{lb}$}‚त्य‚र्थः ।
	{\color{gray}{\rmlatinfont\textsuperscript{§~\theparCount}}}
	\pend% ending standard par
      ‚{\tiny $_{lb}$}‚

	  
	  \pstart \leavevmode% starting standard par
	य‚द्वा त‚दाश्र‚य‚भूताया इति त‚देव गोत्व‚माश्र‚य‚भूतं य‚स्यास्त‚स्या बुद्धेस्त‚न्मात्र‚{\tiny $_{lb}$}‚विशेषेण प्र‚तिक्षिप्त‚ध‚र्मान्त‚रेण गोत्व‚मात्रेण विषे\edtext{}{\lemma{विषे}\Bfootnote{? शे}}षेण स‚ह गोपिण्ड‚स्यैक‚त्वे‚{\tiny $_{lb}$}‚नाप्र‚तिभास‚नात् । त‚था ह्य‚स्य गोत्व‚मिति प्र‚योगे निष्कृष्ट‚रूपं ध‚र्मं प्र‚तिय‚ती‚{\tiny $_{lb}$}‚  ‚{\tiny $_{lb}$}‚ ‚{\tiny $_{lb}$}‚ \leavevmode\ledsidenote{\textenglish{150/s}}बुद्धिरुत्प‚द्य‚ते । त‚तो न सामा‚{\tiny $_{३}$}‚नाधिक‚र‚ण्य‚मिति । ध‚र्मान्त‚र‚प्र‚तिक्षेपादेव त‚द‚न्येषु‚{\tiny $_{lb}$}‚ भेदेषु \textbf{निराकांक्ष‚त्वा}च्च बुद्धेर्न विशेष‚ण‚विशेष्य‚भावः ।
	{\color{gray}{\rmlatinfont\textsuperscript{§~\theparCount}}}
	\pend% ending standard par
      ‚{\tiny $_{lb}$}‚

	  
	  \pstart \leavevmode% starting standard par
	\textbf{द्वितीये तु} ध‚र्मिवाचिश‚ब्द‚प‚क्षे \textbf{भ‚व‚ति} सामानाधिक‚र‚ण्य‚म्विशेष‚ण‚विशेष्य‚भावो‚{\tiny $_{lb}$}‚ वा शुक्लो गौरिति । सामानाधिक‚र‚ण्ये कार‚ण‚माह । \textbf{त‚थे}त्यादि । \textbf{त‚था संकेतानुसा‚{\tiny $_{lb}$}‚रेणे}त्य‚प्र‚तिक्षिप्त‚भेदान्त‚रे व‚स्तुनि ध‚र्मिश‚ब्द‚स्य संके‚{\tiny $_{४}$}‚तानुसारेण हेतुना । एक‚स्मिन्‚{\tiny $_{lb}$}‚ ध‚र्मिणि योज‚नं संहारः । व्य‚व‚च्छेद‚हेतुका ध‚र्मा व्य‚व‚च्छेद‚ध‚र्माः संहृताश्च ते \textbf{स‚क‚{\tiny $_{lb}$}‚ल‚व्य‚व‚च्छेद‚ध‚र्माश्चे}ति क‚र्म‚धार‚यः । \textbf{तै}र्ध‚र्मैः क‚र‚ण‚भूतै\textbf{र्विभाग‚व‚तः} । विभ‚क्तानेक‚{\tiny $_{lb}$}‚ध‚र्म‚व‚तो ध‚र्मि\textbf{ण एक‚स्येव} श‚ब्द\textbf{स‚न्द‚र्श‚नेन} प्र‚द‚र्श‚नेन बुद्धेः \textbf{प्र‚तिभास‚नात्} । अनेक‚ध‚र्म‚{\tiny $_{lb}$}‚व‚न्त‚न्ध‚र्मिण‚मेक‚मिव स‚न्द‚र्श‚य‚न्ती बुद्धिः‚{\tiny $_{५}$}‚ प्र‚तिभास‚त इति याव‚त् । न तु बुद्धि‚{\tiny $_{lb}$}‚प्र‚तिभास‚स‚न्द‚र्शितो ध‚र्मी व‚स्तुत एकः [।] विक‚ल्प‚निर्मित‚स्य ध‚र्म‚ध‚र्मिविभाग‚{\tiny $_{lb}$}‚स्यालीक‚त्वात् । \textbf{एक‚स्यैवे}त्य‚पि प‚ठ‚न्ति । त‚त्रापि प्र‚तिप‚त्त्र‚ध्य‚व‚साय‚व‚शादेव युक्त‚{\tiny $_{lb}$}‚मिति बोद्ध‚व्यं । त‚तः सिद्धं सामानाधिक‚र‚ण्यं [।] य‚त‚श्च भेदान्त‚राप्र‚तिक्षेपेण‚{\tiny $_{lb}$}‚ ध‚र्म्मिश‚ब्दः प्र‚वृत्त‚स्त‚त एव त‚ज्ज‚निताया बुद्धेर‚प्र‚तिक्षिप्त‚{\tiny $_{६}$}‚भेदा\textbf{न्त‚रापेक्ष‚त्वाद्}‚{\tiny $_{lb}$}‚ भ‚व‚ति विशेष‚ण‚विशेष्य‚भावो य‚द् गौः शुक्लो नीलो वेति । च‚श‚ब्द‚श्च पूर्व‚व‚द‚तीत‚{\tiny $_{lb}$}‚हेत्व‚पेक्षः ॥
	{\color{gray}{\rmlatinfont\textsuperscript{§~\theparCount}}}
	\pend% ending standard par
      ‚{\tiny $_{lb}$}‚

	  
	  \pstart \leavevmode% starting standard par
	इद‚मेव व्याप‚कं स‚र्व‚व्य‚व‚हार‚स्य नाप‚र‚स्प‚र‚प‚रिक‚ल्पित सामान्य‚गुणादिकान्त‚स्य‚{\tiny $_{lb}$}‚ प्र‚माण‚बाधित‚त्वादित्याह । \textbf{भेदोय‚मेवे}त्यादि । \textbf{द्र‚व्य‚भावाभिधायिनोः श‚ब्द‚योर‚{\tiny $_{lb}$}‚\leavevmode\ledsidenote{\textenglish{57a/PSVTa}} य‚मेव भेदो} ध‚र्मान्त‚र‚प्र‚{\tiny $_{७}$}‚तिक्षेपाप्र‚तिक्षेप‚ल‚क्ष‚णः । स‚र्व‚त्रेति । सामान्य‚सामान्य‚व‚ति ।‚{\tiny $_{lb}$}‚ गुण‚गुण‚व‚ति । क्रियाक्रियाव‚ति । स‚र्व‚स्मिन् विष‚ये ध‚र्मिव‚च‚नो द्र‚व्याभिधार्या ।‚{\tiny $_{lb}$}‚ ध‚र्म‚व‚च‚नो भावाभिधायी । य‚त एव\textbf{न्तेन} कार‚णेन न \textbf{त‚यो}र्द्र‚व्य‚भाव\textbf{श‚ब्द‚योर्वाच्ये‚{\tiny $_{lb}$}‚ विशेषः} प‚र‚मार्थ‚तः \textbf{क‚श्च‚ना}स्ति ।
	{\color{gray}{\rmlatinfont\textsuperscript{§~\theparCount}}}
	\pend% ending standard par
      ‚{\tiny $_{lb}$}‚

	  
	  \pstart \leavevmode% starting standard par
	त‚द्व्याच‚ष्टे । \textbf{त‚स्मा}दित्यादि । \textbf{निश्च‚य‚प्र‚त्य‚य‚विष‚य‚त्वेन} क‚र‚णेन । \textbf{न क‚श्चि‚{\tiny $_{lb}$}‚द्विशेषः} । त‚था हि‚{\tiny $_{१}$}‚ य‚था गोत्व‚मित्युक्ते त‚त्रैवागोव्य‚व‚च्छेदे निश्च‚य‚स्त‚था गौरि‚{\tiny $_{lb}$}‚त्युक्ते य‚द्य‚प्य‚प्र‚तिक्षिप्त‚भेदान्त‚र‚स्यागोव्य‚व‚च्छिन्न‚स्याभिधान‚न्त‚थाप्य‚गोव्य‚व‚च्छे‚{\tiny $_{lb}$}‚‚{\tiny $_{lb}$}‚ \leavevmode\ledsidenote{\textenglish{151/s}}द‚मात्रे निश्च‚योन्येषान्तु भेदानाम‚प्र‚तिक्षेप‚मात्रं । स एव च श‚ब्दार्थो य‚त्र शाब्दो‚{\tiny $_{lb}$}‚ निश्च‚यो भ‚व‚तीति नास्ति भाव‚द्र‚व्याभिधायिनोः श‚ब्द‚योर्वाच्ये विशेषो भेदान्त‚र‚{\tiny $_{lb}$}‚प्र‚तिक्षेपाप्र‚तिक्षेप‚मात्र‚न्तु भिद्य‚ते । त‚देवा‚{\tiny $_{२}$}‚ह । \textbf{एक‚स्त‚मे}वेत्यादि । एक इति‚{\tiny $_{lb}$}‚ ध‚र्म‚श‚ब्द‚स्त‚मित्य‚गोव्य‚व‚च्छिन्नं । \textbf{प्र‚तिक्षिप्तं भेदान्त‚रं} येन ध‚र्म‚श‚ब्देन स त‚थोक्तः ।‚{\tiny $_{lb}$}‚ \textbf{अन्य} इति ध‚र्मिश‚ब्दोऽ\textbf{प्र‚तिक्षेपेण} त‚मेव पिण्डं सामानाधिक‚र‚ण्येन \textbf{ग‚म‚य‚तीति} नास्ति‚{\tiny $_{lb}$}‚ द्र‚व्य‚निश्च‚य‚म्प्र‚ति \textbf{भेदः} प्र‚तिक्षेपाप्र‚तिक्षेप‚मात्र‚न्तु भिद्य‚ते । एवं ग‚म‚न‚न्देव‚द‚त्त‚स्य‚{\tiny $_{lb}$}‚ ग‚च्छ‚ति देव‚द‚त्त इति न क‚श्च‚न भेद इत्य‚न्य‚त्राप्येवं‚{\tiny $_{३}$}‚ योज्यं ॥
	{\color{gray}{\rmlatinfont\textsuperscript{§~\theparCount}}}
	\pend% ending standard par
      ‚{\tiny $_{lb}$}‚

	  
	  \pstart \leavevmode% starting standard par
	यापि जातिगुण‚क्रियास‚म्ब‚न्ध‚भेदेन च‚तुष्ठ‚यी श‚ब्दानां वृत्तिः साप्य‚नेनैव‚{\tiny $_{lb}$}‚ व‚स्तुग‚त‚ध‚र्म‚भेदेन संगृहीतेत्याह । \textbf{जिज्ञाप‚यिषु}रित्यादि । ज्ञाप‚यितुमिच्छुर\textbf{र्थ‚न्त}म्भे‚{\tiny $_{lb}$}‚दान्त‚र‚प्र‚तिक्षेपाप्र‚तिक्षेप‚ल‚क्ष‚णं । \textbf{त‚द्धितेन} त‚द्धित‚प्र‚त्य‚यान्तेन । \textbf{कृतापि वा} । कृत्सं‚{\tiny $_{lb}$}‚ज्ञ‚क‚प्र‚त्य‚यान्तेन वा । \textbf{अन्येन वा} कृत्त‚द्धित‚व्य‚तिरिक्तेन तिङ्न्तेनाऽव्युत्प‚{\tiny $_{४}$}‚न्नेन वा‚{\tiny $_{lb}$}‚ श‚ब्देन शुक्लादिना \textbf{य‚दि ब्रूयात् । त‚तो} भेदान्त‚र‚प्र‚तिक्षेपाप्र‚तिक्षेप‚ल‚क्ष‚णाद् विशेषा‚{\tiny $_{lb}$}‚\textbf{द‚प‚रो भेदो नास्ति} ।
	{\color{gray}{\rmlatinfont\textsuperscript{§~\theparCount}}}
	\pend% ending standard par
      ‚{\tiny $_{lb}$}‚

	  
	  \pstart \leavevmode% starting standard par
	त‚द्व्याच‚ष्टे । \textbf{एताव‚न्त‚मि}त्यादिना । एताव‚न्त‚मिति प्र‚तिक्षिप्त‚भेदान्त‚र‚ल‚क्ष‚णं ।‚{\tiny $_{lb}$}‚ कृतापि वा । \textbf{द‚र्श‚ये}दिति स‚म्ब‚न्धः ।
	{\color{gray}{\rmlatinfont\textsuperscript{§~\theparCount}}}
	\pend% ending standard par
      ‚{\tiny $_{lb}$}‚

	  
	  \pstart \leavevmode% starting standard par
	य‚दा वाधिश्र‚य‚णादिक्रियायां क‚र्त्तृस्थायाम्प‚चिर्व‚र्त्त‚ते । त‚त्रैव च घ‚ञ् प्र‚त्य‚य‚{\tiny $_{lb}$}‚स्त‚दा पाच‚क‚त्व‚श‚ब्देन क्रियाकार‚क‚योः स‚{\tiny $_{५}$}‚म्ब‚न्धः स‚म‚वायोभिधीय‚त इति पाच‚{\tiny $_{lb}$}‚क‚त्व‚श‚ब्देन स‚मानार्थः पाक‚श‚ब्दः । द्वाव‚प्येतौ प्र‚तिक्षिप्त‚भेदान्त‚र‚म‚पाच‚क‚व्य‚व‚{\tiny $_{lb}$}‚च्छिन्न‚म‚र्थं प्र‚तिपाद‚य‚तः । य‚दा तु क‚र्म‚स्थैव क्रिया विकॢत्तिः प‚चेर‚र्थ‚स्त‚दा‚{\tiny $_{lb}$}‚ पाच‚क‚त्व‚श‚ब्द‚स्य क‚थं स‚म्ब‚न्धाभिधायित्वं । पाच‚क‚त्व‚पाक‚योर्भिन्नार्थ‚त्वात् ॥‚{\tiny $_{lb}$}‚ अन्येन वा कृत्त‚द्धित‚व्य‚तिरिक्तेन तिङादिना । \textbf{त‚थाभूत‚ज्ञाप‚नाय} । प्र‚तिक्षिप्त‚{\tiny $_{lb}$}‚  ‚{\tiny $_{lb}$}‚ ‚{\tiny $_{lb}$}‚ \leavevmode\ledsidenote{\textenglish{152/s}}भेदान्त‚र‚ज्ञाप‚नाय स्व‚{\tiny $_{६}$}‚यं कृतेन स‚म‚येन द‚र्श‚येदिति स‚म्ब‚न्धः । य‚था देव‚द‚त्तेन श‚य्य‚ते‚{\tiny $_{lb}$}‚ प‚ट‚स्य शुक्ल‚त्व‚मिति । अत्रापि शाय‚क‚श‚ब्द‚स्य य एवार्थः स एव प्र‚तिक्षिप्त‚भेदान्त‚रः ।‚{\tiny $_{lb}$}‚ श‚य्य‚त इत्य‚स्यापि । त‚था शुक्लः प‚ट इति य एवाशुक्ल‚व्य‚व‚च्छिन्नो प्र‚तिक्षिप्त‚भे‚{\tiny $_{lb}$}‚दान्त‚रोर्थः स एव प्र‚तिक्षिप्त‚भेदान्त‚रः प‚ट‚स्य शुक्ल इत्य‚स्यापि । \textbf{त‚थाभिधान‚{\tiny $_{lb}$}‚\leavevmode\ledsidenote{\textenglish{57b/PSVTa}} मात्रे}णेति प्र‚तिक्षिप्ताप्र‚{\tiny $_{७}$}‚तिक्षिप्त‚भेदान्त‚र‚स्यैक‚ध‚र्मिग‚त‚स्य व्य‚व‚च्छेद‚स्याभिधान‚{\tiny $_{lb}$}‚मात्रेण त‚देव व‚स्त्व\textbf{र्थान्त‚र‚मेव} प‚र‚मार्थ‚तो ध‚र्म‚ध‚र्मिरूपेण विभ‚क्त‚मेव । न पुन‚{\tiny $_{lb}$}‚र्भ‚व‚तीति स‚म्ब‚न्धः । किङ्कार‚णं [।] त‚थाभूत‚स्यैव प्र‚तिक्षिप्ताप्र‚तिक्षिप्त‚भेदान्त‚र‚{\tiny $_{lb}$}‚स्यैवैक‚स्य ज्ञाप‚नाय ध‚र्म‚ध‚र्मिश‚ब्द‚स्य कृत‚संकेत‚त्वात् ।
	{\color{gray}{\rmlatinfont\textsuperscript{§~\theparCount}}}
	\pend% ending standard par
      ‚{\tiny $_{lb}$}‚

	  
	  \pstart \leavevmode% starting standard par
	य‚द्वा \textbf{त‚थाभिधान‚मात्रेणे}ति । अर्थान्त‚र‚भूत‚ध‚र्माभिधान‚मात्रेण त‚द्ध‚र्म‚स्व‚{\tiny $_{१}$}‚‚{\tiny $_{lb}$}‚रूप‚म्प‚र‚मार्थ‚तो\textbf{र्थान्त‚र‚मेव भ‚व‚ति । त‚थाभूत‚स्}यैव भेदान्त‚र‚निर‚पेक्ष‚स्यैव त‚स्यैक‚{\tiny $_{lb}$}‚व्यावृत्त‚स्य \textbf{ज्ञाप‚नाय} ध‚र्म\textbf{श‚ब्द‚स्य कृत‚संकेत‚त्वात्} ॥
	{\color{gray}{\rmlatinfont\textsuperscript{§~\theparCount}}}
	\pend% ending standard par
      ‚{\tiny $_{lb}$}‚

	  
	  \pstart \leavevmode% starting standard par
	\textbf{न‚नु चे}त्यादि प‚रः । \textbf{स‚म्ब‚न्ध} उच्य‚त इति पाक‚क्रियायाः पाच‚क‚स्य च क‚र्त्तुः‚{\tiny $_{lb}$}‚ स‚म्ब‚न्धः स‚म‚वाय‚ल‚क्ष‚णः । त‚था हि कृद‚न्ताद् भाव‚प्र‚त्य‚यः स‚म्ब‚न्ध‚स्याभिधाय‚को‚{\tiny $_{lb}$}‚ दृष्टो य‚थाह । स‚मास‚कृत्त‚द्धितेषु स‚म्ब‚न्धाभिधान‚मि‚{\tiny $_{२}$}‚ति । कृद‚न्त‚श्च पाच‚क‚{\tiny $_{lb}$}‚श‚ब्दः [।] न पाक एव क्रियात्म‚कः पाच‚क‚त्व‚श‚ब्देनोच्य‚ते ।
	{\color{gray}{\rmlatinfont\textsuperscript{§~\theparCount}}}
	\pend% ending standard par
      ‚{\tiny $_{lb}$}‚

	  
	  \pstart \leavevmode% starting standard par
	एत‚दुक्त‚म्भ‚व‚ति [।] अन्यैव क‚र्त्तृव्य‚तिरिक्ता क्रियान्य‚श्च त‚योश्च स‚म्ब‚न्धो‚{\tiny $_{lb}$}‚न्य एव । त‚त‚श्च क‚र्त्तृस्थ‚क्रियाभिधाने स‚त्य‚पि प‚च‚तेर्न पाक‚पाच‚क‚त्व‚श‚ब्द‚यो‚{\tiny $_{lb}$}‚स्तुल्योर्थ इति । \textbf{न वै पाकेने}त्यादिना प‚रिह‚र‚ति । \textbf{पाकेन} क‚र्तृस्थे न व‚स्तुभूतेन‚{\tiny $_{lb}$}‚ व्यापारेण युक्तो\textbf{न्य एव} पाक‚क्रिया व्य‚तिरिक्तः \textbf{पा‚{\tiny $_{३}$}‚च‚को} नाम क‚र्त्ता\textbf{भिधीय‚ते}‚{\tiny $_{lb}$}‚ पाच‚क‚श‚ब्देन [।] यादृशो व‚र्ण्ण्य‚ते वै शे षि का दिभिः क्रिया व्य‚तिरिक्तः स्व‚त‚न्त्रः‚{\tiny $_{lb}$}‚ क‚र्ता य‚त्र क्रियाकार‚क‚स‚म्ब‚न्धो व‚स्तुभूतः स्यात् । त‚स्य स्व‚त‚न्त्र‚स्य क‚र्त्तुः क्रिया‚{\tiny $_{lb}$}‚व्य‚तिरिक्त‚स्य निषेत्स्य‚मान‚त्वात् । न चेद् व्य‚तिरिक्ता क्रिया क‚र्त्ता वा कुत‚स्स‚{\tiny $_{lb}$}‚म्ब‚न्धः य‚स्य भाव‚प्र‚त्य‚येनाभिधान‚मिति भावः । \textbf{य‚त्पुन‚र‚स्ये}ति पाच‚क‚श‚ब्द‚स्या‚{\tiny $_{४}$}‚‚{\tiny $_{lb}$}‚स्या\textbf{भिधे}य‚म‚पाच‚के व्य‚व‚च्छिन्न‚म‚प्र‚तिक्षिप्त‚भेदान्त‚रं व‚स्तुमात्र‚न्त‚देव पाच‚क‚{\tiny $_{lb}$}‚  ‚{\tiny $_{lb}$}‚ ‚{\tiny $_{lb}$}‚ \leavevmode\ledsidenote{\textenglish{153/s}}श‚ब्दा\textbf{भिधे}यं पा\textbf{च‚क‚त्वेनाप्य‚भि}धीय‚त इत्य‚ध्याहार्यः । त‚स्यैव प्र‚तिक्षिप्त‚भेदान्त‚र‚{\tiny $_{lb}$}‚स्याभिधानात् । न तु स‚म्ब‚न्धोभिधीय‚ते । त‚स्यास‚त्त्वात् । त‚देव य‚थोक्तं पाच‚क‚{\tiny $_{lb}$}‚श‚ब्दाभिधेयं पाक‚श‚ब्देनापीत्य‚पिश‚ब्दात् ।
	{\color{gray}{\rmlatinfont\textsuperscript{§~\theparCount}}}
	\pend% ending standard par
      ‚{\tiny $_{lb}$}‚

	  
	  \pstart \leavevmode% starting standard par
	अप‚रं व्याख्यानं । \textbf{न वै पाकेने}ति पाक‚श‚ब्देनान्यापोह‚वादिप‚{\tiny $_{५}$}‚क्षे \textbf{अन्य} एव‚{\tiny $_{lb}$}‚ व्य‚तिरिक्तः क्रियाश्र‚य‚भूतः \textbf{पाच‚को}भिधीय‚ते । यादृशो व‚र्ण्ण्य‚ते प‚रेण [।] यः क्रिया‚{\tiny $_{lb}$}‚कार‚क‚स‚म्ब‚न्ध‚स्याश्र‚यः स्यात् । त‚स्यासिद्ध‚त्वात्[।]किन्त्व‚पाच‚क‚व्यावृत्तिर्भेदान्त‚र‚{\tiny $_{lb}$}‚प्र‚तिक्षेपेणाभिधीय‚ते । त‚देवाह । \textbf{य‚त्पुन‚रि}त्यादि । य‚दित्य‚पाच‚क‚व्यावृत्तिल‚क्ष‚णं‚{\tiny $_{lb}$}‚ प्र‚तिक्षिप्त‚भेदान्त‚र‚म‚भिधेयं । \textbf{अस्ये}ति पाक‚श‚ब्द‚स्यान‚न्त‚र‚मेव द‚र्शितं त‚देव पाक‚{\tiny $_{६}$}‚‚{\tiny $_{lb}$}‚श‚ब्दाभिधेय‚म्पा\textbf{च‚क}त्व‚श‚ब्देनाप्य\textbf{भिधीय‚ते} [।] न स‚म्ब‚न्धः । त‚स्यासिद्ध‚त्वात् ।‚{\tiny $_{lb}$}‚ \textbf{अप्र‚तिष्ठि}तैर‚व‚स्तुब‚लायातैर‚त एव \textbf{मिथ्याविक‚ल्पाः} ॥
	{\color{gray}{\rmlatinfont\textsuperscript{§~\theparCount}}}
	\pend% ending standard par
      ‚{\tiny $_{lb}$}‚

	  
	  \pstart \leavevmode% starting standard par
	क‚थ‚म्पुन‚र्ग‚म्य‚ते क्रिया व्य‚तिरिक्ता नास्ति त‚त्स‚म‚वायो वेत्य‚त आह । \textbf{य‚थे}‚{\tiny $_{lb}$}‚त्यादि । \textbf{त‚त्स‚म‚वायो} वेति क्रियाकार‚क‚स‚म‚वायः । य‚त‚श्च व्यावृत्तिव्यावृत्तिम‚तो‚{\tiny $_{lb}$}‚र‚भेद‚स्तेन कार‚णेना\textbf{न्यापो}ह\textbf{विष‚यो} जातिमान् श‚ब्दै‚{\tiny $_{७}$}‚र‚भिधीय‚त इति [।] त‚द्व- \leavevmode\ledsidenote{\textenglish{58a/PSVTa}}‚{\tiny $_{lb}$}‚ त्प‚क्ष‚स्त‚त्र यो दोषः सोन्यापोहेपि स्यादिति \textbf{त‚द्व‚त्प‚क्षोप‚व‚र्ण्ण्ण‚नं प्र‚त्याख्या}तं ।‚{\tiny $_{lb}$}‚ य‚स्मात् \textbf{पृथ‚क‚त्वे हि जातित‚द्व‚तो}र‚भ्युप‚ग‚म्य‚माने \textbf{स्यात्} त‚द्व‚त्प‚क्षोदितो \textbf{दोषः} ।
	{\color{gray}{\rmlatinfont\textsuperscript{§~\theparCount}}}
	\pend% ending standard par
      ‚{\tiny $_{lb}$}‚

	  
	  \pstart \leavevmode% starting standard par
	\textbf{त‚द्विशिष्ट}स्येत्य‚न्यापोह‚विशिष्ट‚स्\textbf{यार्थ‚स्य श‚ब्दैर}भिधानात् \textbf{त‚द्व‚त्प‚क्षोदित} इति‚{\tiny $_{lb}$}‚ त‚द्व‚त्प‚क्षे य उक्तः । य‚था किल सामान्य‚म‚भिधाय त‚द्व‚ति व‚र्त्त‚मानः श‚ब्दोऽस्व‚त‚न्त्रः‚{\tiny $_{lb}$}‚ स्यात्त‚{\tiny $_{१}$}‚त‚श्च श‚ब्द‚प्र‚वृत्तिनिमित्त‚भूतेन सामान्येन व‚शीकृत‚स्य श‚ब्द‚स्य व्य‚क्तिग‚त‚{\tiny $_{lb}$}‚प‚र‚स्प‚र‚भेदानाक्षेपात्तैः सामानाधिक‚र‚ण्यं न स्यात् । उप‚च‚रिता च त‚द्व‚ति श‚ब्द‚प्र‚{\tiny $_{lb}$}‚वृत्तिरित्यादिको दोष इत्येवं व्यावृत्तिम‚भिधाय त‚द्व‚ति व‚र्त्त‚मानोस्व‚त‚न्त्रो ध्व‚नि‚{\tiny $_{lb}$}‚रिति \textbf{स‚र्वः प्र‚संगः स्यात् । त‚द‚पि} त‚द्व‚त्प‚क्षोप‚व‚र्ण्ण‚नं । \textbf{अनेने}ति व्यावृत्तिव्यावृत्ति‚{\tiny $_{lb}$}‚म‚तोर‚न‚न्य‚त्वेन‚{\tiny $_{२}$}‚ \textbf{प्र‚तिव्यूढं} प्र‚त्याख्यातं । य‚स्मात् \textbf{त‚त्र हि} त‚द्व‚त्प‚क्षे । \textbf{अर्थान्त‚र‚मु‚{\tiny $_{lb}$}‚पादा}येति व‚स्तुभूतं सामान्य‚मुपादायान्य‚त्रार्थान्त‚रे त‚द्व‚ति । साक्षात् सामान्य‚व‚तो‚{\tiny $_{lb}$}‚ऽन‚भिधानाद‚स्वात‚न्त्र्यं । आदिश‚ब्दाद‚स‚मानाधिक‚र‚ण्योप‚चार‚दोष‚प‚रिग्र‚हः ।
	{\color{gray}{\rmlatinfont\textsuperscript{§~\theparCount}}}
	\pend% ending standard par
      ‚{\tiny $_{lb}$}‚

	  
	  \pstart \leavevmode% starting standard par
	अन्यापोह‚प‚क्षे तु व्यावृत्तिव्यावृत्तिम‚तोरैक्यान्नार्थान्त‚र‚मुपादायार्थान्त‚रे श‚ब्द‚प्र‚{\tiny $_{lb}$}‚‚{\tiny $_{lb}$}‚ \leavevmode\ledsidenote{\textenglish{154/s}}वृत्तिस्त‚तो नास्त्य‚स्वात\textbf{न्त्रा}\edtext{}{\lemma{स्वात}\Bfootnote{? न्त्र्या}} दिदोष इत्याह ।‚{\tiny $_{३}$}‚ \textbf{न चे}त्यादि । \textbf{अन्य‚स्माद्}‚{\tiny $_{lb}$}‚ व‚स्तुनोर्या व्या\textbf{वृत्तिः} सा \textbf{व्यावृत्तान्नान्या । द्व‚यो}र्ध‚र्म‚ध‚र्मिवाचिनोः श‚ब्द‚यो\textbf{रेक‚स्य}‚{\tiny $_{lb}$}‚ व्यावृत्तिभेद‚स्या\textbf{भिधानादित्युक्त}म‚न‚न्त‚र‚मेव ।
	{\color{gray}{\rmlatinfont\textsuperscript{§~\theparCount}}}
	\pend% ending standard par
      ‚{\tiny $_{lb}$}‚

	  
	  \pstart \leavevmode% starting standard par
	\textbf{क‚थ}मित्यादि प‚रः । \textbf{इदानी}मिति व्यावृत्तित‚द्व‚तोरैक्ये । \textbf{एक‚स्य व्यावृत्त‚स्य}‚{\tiny $_{lb}$}‚ स्व‚ल‚क्ष‚ण‚स्या\textbf{न‚नुग‚मा}त् । अर्थान्त‚रासंस‚र्गात् । क‚थ‚न्त‚स्य स्व‚ल‚क्ष‚ण‚स्यात्म‚भूता‚{\tiny $_{lb}$}‚ \textbf{व्यावृत्तिः} स्व‚ल‚क्ष‚ण‚व‚द‚न‚न्व‚यिनी \textbf{सामान्यं} स्यात् ।‚{\tiny $_{४}$}‚ नैव । दृष्टा च सामान्यं ।
	{\color{gray}{\rmlatinfont\textsuperscript{§~\theparCount}}}
	\pend% ending standard par
      ‚{\tiny $_{lb}$}‚

	  
	  \pstart \leavevmode% starting standard par
	\textbf{त‚द्बुद्धा}वित्यादिना सिद्धान्त‚वादी । सामान्य‚बुद्धौ विक‚ल्पिकायां \textbf{त‚थै}काकारेण‚{\tiny $_{lb}$}‚ \textbf{प्र‚तिभास‚नादे}काकार एव व्याव‚र्त्य‚तेनेनेति व्यावृत्तिः । सामान्य‚मुच्य‚ते । एत‚दाह‚{\tiny $_{lb}$}‚ [।] न व्यावृत्तेषु स्व‚ल‚क्ष‚णेष्वात्म‚भूता व्यावृत्तिरेका सामान्यं केव‚लं व्यावृत्त‚{\tiny $_{lb}$}‚स्व‚ल‚क्ष‚णानुभ‚वोत्त‚र‚काल‚भावी विक‚ल्पः प्र‚कृत्या । एक‚कार्येषु भावेष्वेक‚माकार‚{\tiny $_{lb}$}‚माद‚र्श‚य‚न्निवो‚{\tiny $_{५}$}‚त्प‚द्य‚ते । त‚द्विक‚ल्प‚व‚शात् सामान्य‚मास्थीय‚ते [।] निःसामान्येष्व‚{\tiny $_{lb}$}‚प्य‚नेन च साक्षाच्छ‚ब्दादिविष‚यो द‚र्शितः ।
	{\color{gray}{\rmlatinfont\textsuperscript{§~\theparCount}}}
	\pend% ending standard par
      ‚{\tiny $_{lb}$}‚

	  
	  \pstart \leavevmode% starting standard par
	एत‚देव स्फुट‚य‚न्नाह । \textbf{न वै किंचि}दित्यादि । व‚स्तुभूत‚मित्य‚भिप्रायः । क‚थ‚न्त‚र्हि‚{\tiny $_{lb}$}‚ सामान्य‚सामानाधिक‚र‚ण्यादिव्य‚व‚हार इत्याह । \textbf{श‚ब्दे}त्यादि । \textbf{श‚ब्द आश्र‚यः} स‚ह‚का‚{\tiny $_{lb}$}‚रिकार‚ण‚त्वेन य‚स्याः सा विक‚ल्पिका \textbf{बुद्धिर‚नादिवास‚नासाम‚र्थ्यात् । ध‚र्मान‚संसृ‚{\tiny $_{६}$}‚‚{\tiny $_{lb}$}‚ष्टान‚पि संसृज‚न्ती} एकाकारानिव कुर्वाणा \textbf{जाय‚ते} । त‚स्या बुद्धेरेका\textbf{कार‚प्र‚तिभास‚{\tiny $_{lb}$}‚व‚शेन सामान्यं} । ध‚र्म‚द्व‚य‚युक्तैक‚ध‚र्मिप्र‚तिभास‚व‚शेन \textbf{सामानाधिक‚र‚ण्यं च व्य‚व‚{\tiny $_{lb}$}‚स्थाप्य‚ते} । अयं च सामान्यादिव्य‚व‚हारोऽस‚द्व्यापि व्य‚व‚स्थाप्य‚ते । क‚थ‚म‚स‚द‚र्थ‚{\tiny $_{lb}$}‚\leavevmode\ledsidenote{\textenglish{58b/PSVTa}} इत्याह । \textbf{अर्थानामित्या}दि । स्व‚ल‚क्ष‚णांनां संस‚र्गाभावात् सामान्य‚व्य‚व‚हारोऽस‚द‚र्थः ।‚{\tiny $_{७}$}‚‚{\tiny $_{lb}$}‚ \textbf{एक‚स्य च} स्व‚ल‚क्ष‚ण‚स्य \textbf{भेदाभावात्} सामानाधिक‚र‚ण्य‚व्य‚व‚हारोस‚द‚र्थः ।
	{\color{gray}{\rmlatinfont\textsuperscript{§~\theparCount}}}
	\pend% ending standard par
      ‚{\tiny $_{lb}$}‚

	  
	  \pstart \leavevmode% starting standard par
	न‚नु विरूप‚त‚याऽयं स‚र्व‚व्य‚व‚हारः प्र‚वृत्त इति क‚थ‚म‚न्यापोह‚विष‚य इत्य‚त \textbf{आह ।‚{\tiny $_{lb}$}‚ त‚स्य स‚र्व‚स्य} सामान्यादिव्य‚व‚हार‚स्यार्थाः \textbf{स‚माश्र}य इत्य‚नेन स‚म्ब‚न्धः । \textbf{त‚त्कार्य‚न्त‚च्च‚{\tiny $_{lb}$}‚  ‚{\tiny $_{lb}$}‚ ‚{\tiny $_{lb}$}‚ \leavevmode\ledsidenote{\textenglish{155/s}}कार‚ण}म‚नुरूपं येषान्तेषाम्भाव\textbf{स्त‚या} । क‚र‚ण‚भूत‚या । \textbf{अन्येभ्य} इत्य‚त‚त्कार्य‚कार‚णेभ्यो‚{\tiny $_{lb}$}‚ \textbf{भिद्य‚माना अर्थाः} स‚र्व‚स्य सामान्या‚{\tiny $_{१}$}‚दिव्य‚व‚हार‚स्या\textbf{श्र‚यो भ‚व}न्त्य‚तः कार‚णाद् अन्या‚{\tiny $_{lb}$}‚पोह‚विष‚य उक्तः [।] न त्व‚न्यापोह‚स्त‚त्र प्र‚तिभास‚ते बाह्य‚स्यैवैकाकार‚स्य विधि‚{\tiny $_{lb}$}‚रूप‚त‚या प्र‚तिभास‚नात् । य‚स्माच्च निश्च‚य‚प्र‚युक्तः पुरुष‚म‚निष्ट‚प‚रिहारेणा\textbf{निष्टाद्‚{\tiny $_{lb}$}‚ व्यावृत्ते} स्व‚ल‚क्ष‚णे \textbf{प्र‚व‚र्त्त}य‚त्य\textbf{तोपि} कार‚णाद् \textbf{अन्यापोह‚विष‚य उक्तो} न तु प्र‚तिभा‚{\tiny $_{lb}$}‚सापेक्षो विधेः प्र‚तिभास‚नात् ।
	{\color{gray}{\rmlatinfont\textsuperscript{§~\theparCount}}}
	\pend% ending standard par
      ‚{\tiny $_{lb}$}‚

	  
	  \pstart \leavevmode% starting standard par
	तेन य‚दुच्य‚ते भ ट्टो द्यो त का राभ्यां [।]\edtext{}{\edlabel{pvsvt_155-1}\label{pvsvt_155-1}\lemma{राभ्यां}\Bfootnote{\href{http://sarit.indology.info/?cref=\%C5\%9Bv-apoha}{ Cf. Ślokavārtika. Apoha} \href{http://sarit.indology.info/?cref=nv.2.2.71}{ Nyāyavārtika 2: 2: 71. }}} गोश‚ब्द‚स्यार्थः‚{\tiny $_{२}$}‚ किम्भावोथाभावः ।‚{\tiny $_{lb}$}‚ य‚दि भावो किं गौर‚थागौः । य‚दि गौर्नास्ति विवादः । अथागौर्गोश‚ब्द‚स्यागौर‚र्थ‚{\tiny $_{lb}$}‚ इति अतिश‚ब्द‚कौश‚लं । अथाभाव‚स्त‚द‚युक्तं । न हि गोश‚ब्द‚श्र‚व‚णाद‚भावे प्रेष्य‚{\tiny $_{lb}$}‚संप्र‚तिप‚त्तिः । श‚ब्दार्थ‚श्च प्र‚तिप‚त्त्या प्र‚तीय‚ते न च गोश‚ब्दाद‚भावं क‚श्चित्प्र‚तिप‚द्य‚ते‚{\tiny $_{lb}$}‚ त‚थाऽगौर्न भ‚व‚तीत्य‚य‚म‚पोहः किं गोविष‚योथाऽगोविष‚यः । य‚दि गोविष‚यः क‚थं‚{\tiny $_{३}$}‚‚{\tiny $_{lb}$}‚ गोर्ग‚व्येवाभावः । अथागोविष‚यः [।] क‚थ‚म‚न्य‚विष‚याद् अपोहाद् अन्य‚त्र प्र‚तिप‚त्तिः ।‚{\tiny $_{lb}$}‚ न हि ख‚दिरेच्छिद्य‚माने प‚लाशेच्छिदा भ‚व‚ति । अथा गौर्ग‚वि प्र‚तिषेधोऽगौर्न‚{\tiny $_{lb}$}‚ भ‚व‚तीति । केन गोर‚गोत्वं प्र‚स‚क्तं य‚त्प्र‚तिषिध्य‚त इति [।]
	{\color{gray}{\rmlatinfont\textsuperscript{§~\theparCount}}}
	\pend% ending standard par
      ‚{\tiny $_{lb}$}‚

	  
	  \pstart \leavevmode% starting standard par
	अपास्तं [।] गोविष‚य‚त्वाद् गोश‚ब्द‚स्य केव‚लं क‚ल्पित‚विष‚य‚त्वाद् विवादः ।‚{\tiny $_{lb}$}‚ य‚था वा गोप्र‚तिषेधोऽस‚त्य‚पि स‚मारोपे त‚थोक्तं प्रागिति य‚त्किञ्चिदेत‚त् ।
	{\color{gray}{\rmlatinfont\textsuperscript{§~\theparCount}}}
	\pend% ending standard par
      ‚{\tiny $_{lb}$}‚
	    
	    \stanza[\smallbreak]
	  त‚स्मात् स्थित‚मेत‚द्विधिरेव श‚ब्दार्थ इति ॥\&[\smallbreak]
	  
	  
	  ‚{\tiny $_{lb}$}‚

	  
	  \pstart \leavevmode% starting standard par
	एत‚देव द‚र्श‚य‚न्नाह । \textbf{त‚त्रे}त्यादि । \textbf{अन‚पेक्षि}तं स्व‚रूपेण \textbf{बाह्य‚त‚त्त्}वं येन विक‚ल्प‚{\tiny $_{lb}$}‚बुद्धिप्र‚तिभासिना ध‚र्मिणा स त‚थोक्तः । \textbf{बुद्धिप्र‚तिभास‚व‚शादेकोनेक‚व्यावृत्त} इति ।‚{\tiny $_{lb}$}‚ अनेक‚स्माद् व्यावृत्त‚स्यैक‚स्य ध‚र्मिणः स‚न्द‚र्श‚नेन बुद्धेः प्र‚तिभास‚नात् । त‚द्व‚शेनैको‚{\tiny $_{lb}$}‚ ध‚र्मी अनेक‚व्यावृत्तो व्य‚व‚स्थाप्य‚ते । य‚श्चानेक‚स्माद् व्यावृत्त‚स्त‚स्मात् त‚त्र व्यावृ‚{\tiny $_{५}$}‚‚{\tiny $_{lb}$}‚त्त‚यो ध‚र्म‚भेदाः क‚ल्प्य‚न्त इति भावः । स एव भूतो ध‚र्मी श‚ब्दैर्विष‚यीक्रिय‚ते । त‚था‚{\tiny $_{lb}$}‚भूत‚विक‚ल्प‚प्र‚तिभास‚ज‚न‚नाय व‚क्तृभिः श‚ब्द‚स्योच्चार‚णात् । य‚त‚श्चान्य‚व्यावृत्तो‚{\tiny $_{lb}$}‚ विक‚ल्प‚प्र‚तिभासः श‚ब्दैर्विष‚यीक्रिय‚ते । त‚तो विधिविष‚य‚त्वं सिद्ध‚मिति भावः ।
	{\color{gray}{\rmlatinfont\textsuperscript{§~\theparCount}}}
	\pend% ending standard par
      ‚{\tiny $_{lb}$}‚‚{\tiny $_{lb}$}‚‚{\tiny $_{lb}$}‚\textsuperscript{\textenglish{156/s}}

	  
	  \pstart \leavevmode% starting standard par
	न केव‚लं श‚ब्दैर्विक‚ल्पैर‚पि विष‚यीक्रिय‚त इत्याह । \textbf{त‚द‚नुभ‚वे}त्यादि । त‚स्य‚{\tiny $_{lb}$}‚ त‚स्य स्व‚ल‚क्ष‚ण‚स्यानुभ‚व‚स्त\textbf{द‚{\tiny $_{६}$}‚नुभ‚व‚स्तेनाहिता वास‚ना} श‚क्तिस्त‚स्याः प्र‚बोधः‚{\tiny $_{lb}$}‚ कार्योत्पादानुग‚ण्य‚न्त‚तो ज‚न्म येषां विक‚ल्पानान्तै\textbf{र्विष‚यीक्रिय}त इति स‚म्ब‚न्धः ।‚{\tiny $_{lb}$}‚ किं विशिष्टै\textbf{र‚ध्य‚व‚सित‚त‚द्भावा}र्थैः । अध्य‚व‚सित‚स्त‚द्भावो बाह्य‚भावो य‚स्मिन्‚{\tiny $_{lb}$}‚ विक‚ल्प‚प्र‚तिभासे सोध्य‚स‚वित‚त‚द्भाव एवं भूतोर्थो विष‚यो येषाम्विक‚ल्पानान्ते‚{\tiny $_{lb}$}‚ त‚था । दृश्य‚विक‚ल्पावेकीकृत्य प्र‚वृत्तेरिति याव‚त् ।
	{\color{gray}{\rmlatinfont\textsuperscript{§~\theparCount}}}
	\pend% ending standard par
      ‚{\tiny $_{lb}$}‚

	  
	  \pstart \leavevmode% starting standard par
	\leavevmode\ledsidenote{\textenglish{59a/PSVTa}} अध्य‚व‚सित‚{\tiny $_{७}$}‚त‚द्भावार्थ इति पाठान्त‚र‚न्त‚दाध्य‚व‚सित‚त‚द्भाव‚श्चासाव‚र्थ‚{\tiny $_{lb}$}‚श्चेति क‚र्म‚धार‚यः । एकोप्य‚नेक‚व्यावृत्तोध्य‚व‚सित‚त‚द्भावार्थ इति \textbf{स‚म्ब‚न्धः} ।‚{\tiny $_{lb}$}‚ त‚स्माद् बुद्धिप्र‚तिभास‚व‚शात् । सामान्यादिव्य‚व‚हारः । \textbf{त‚त्रैव च} बुद्धिप्र‚तिभासे‚{\tiny $_{lb}$}‚ऽय‚मिति भेदान्त‚र‚प्र‚तिक्षेपाप्र‚तिक्षेप‚ल‚क्ष‚णे । व्य‚व‚ह्रिय‚त इति व्य‚व‚हारः क‚र्म‚{\tiny $_{lb}$}‚साध‚नः । \textbf{ध‚र्म‚ध‚र्मिणा}वेव \textbf{व्य‚व‚हारः} ।
	{\color{gray}{\rmlatinfont\textsuperscript{§~\theparCount}}}
	\pend% ending standard par
      ‚{\tiny $_{lb}$}‚

	  
	  \pstart \leavevmode% starting standard par
	एत‚दुक्त‚म्भ‚व‚ति [।] बुद्धिप्र‚{\tiny $_{१}$}‚तिभासे यौ ध‚र्म‚ध‚र्मिणो व्य‚व‚स्थाप्येते । तौ‚{\tiny $_{lb}$}‚ \textbf{प‚र‚स्प‚र‚न्त‚त्त्वान्य‚त्वाभ्याम‚वाच्यादिति प्र‚त‚न्य‚ते} [।] प‚र‚मार्थ‚त‚स्तेन पार‚मार्थिक‚{\tiny $_{lb}$}‚ध‚र्म‚ध‚र्मित‚त्त्वान्य‚त्व‚प‚क्षे यो दोषः प्र‚माणान्त‚रादिवैय‚र्थ्यं स्वात‚न्त्र्यादिल‚क्ष‚ण उक्तः‚{\tiny $_{lb}$}‚ स इह न भ‚व‚तीत्युक्त‚म्भ‚व‚ति ।
	{\color{gray}{\rmlatinfont\textsuperscript{§~\theparCount}}}
	\pend% ending standard par
      ‚{\tiny $_{lb}$}‚

	  
	  \pstart \leavevmode% starting standard par
	त‚त्त्वान्य‚त्व‚प‚क्ष‚योर्दोषान्त‚र‚म‚प्याह । \textbf{न हीत्या}दि । \textbf{ध‚र्मिणः} स‚काशान्ना\textbf{न्यो‚{\tiny $_{lb}$}‚ध‚र्मः} । किङ्कार‚ण‚म् [।] \textbf{अन‚र्थान्त‚राभिधा‚{\tiny $_{२}$}‚नात्} । ध‚र्म‚ध‚र्मिश‚ब्दाभ्यामेक‚स्मादेव‚{\tiny $_{lb}$}‚ व्य‚व‚च्छिन्न‚स्याभिधानात् । \textbf{नापि} य एव ध‚र्मी \textbf{स एव} ध‚र्मः । क‚स्मात् । \textbf{त‚द्वाचि}ना‚{\tiny $_{lb}$}‚\textbf{मिव} । ध‚र्मिवाचिनामिव श‚ब्दानां । \textbf{ध‚र्म‚वाचिनाम‚पि व्य‚व‚च्छेदान्त‚राक्षेप‚प्र‚सं‚{\tiny $_{lb}$}‚गा}त् । \textbf{त‚था चेष्टाप्र‚त्याय‚नात्} । ध‚र्म‚श‚ब्देनेष्ट‚स्य प्र‚तिक्षिप्त‚भेदान्त‚र‚स्य भेद‚स्या‚{\tiny $_{lb}$}‚प्र‚त्याय‚नात् । \textbf{संकेत‚भेदाक‚र‚णं} । प्र‚तिक्षिप्त‚भेदान्त‚रं व्य‚व‚च्छेदं प्र‚त्याय‚{\tiny $_{३}$}‚य‚ति‚{\tiny $_{lb}$}‚ ध‚र्म‚श‚ब्द इत्य‚स्य संकेत‚भेद‚स्याक‚र‚णं । एत‚द‚न‚न्त‚रोक्त‚न्त‚त्त्वान्य‚त्त्वाभ्या\textbf{म‚वाच्य‚त्वं‚{\tiny $_{lb}$}‚ ध‚र्म‚ध‚र्मिणोः श‚ब्दा}र्थे बुद्धिप्र‚तिभासिन्य‚र्थे उक्तं । \textbf{व‚स्तुनी}ति बाह्य‚स्व‚ल‚क्ष‚णे ।‚{\tiny $_{lb}$}‚ \textbf{अविद्य‚मान‚त्वादे}व त‚त्त्वान्य‚त्त्वाभ्या\textbf{म‚वाच्यं} य‚थाप्र‚तिभास‚न्तु श‚ब्दादिविष‚यो व्य‚व‚{\tiny $_{lb}$}‚स्थाप्य‚ते ।
	{\color{gray}{\rmlatinfont\textsuperscript{§~\theparCount}}}
	\pend% ending standard par
      ‚{\tiny $_{lb}$}‚‚{\tiny $_{lb}$}‚\textsuperscript{\textenglish{157/s}}

	  
	  \pstart \leavevmode% starting standard par
	\textbf{न‚नु चे}त्यादि प‚रः । दृष्टा प्र‚योगेषूप‚ल‚ब्धा । गोर्गोत्व‚मिति \textbf{ष‚ष्ठी । आदि}‚{\tiny $_{lb}$}‚श‚ब्दाद् ग‚वि व्य‚व‚स्थि‚{\tiny $_{४}$}‚तं गोत्वं । गोत्वेन निमित्तेन ग‚वि गोश‚ब्दो व‚र्त्त‚त इत्यादि‚{\tiny $_{lb}$}‚ विभ‚क्तिप‚रिग्र‚हः । गोत्व‚द्र‚व्य‚त्वादीनां च ध‚र्माणां ब‚हुत्वात् । \textbf{त‚त्र ब‚हुषु ध‚र्मेषु} ।‚{\tiny $_{lb}$}‚ गोत्व‚द्र‚व्य‚त्व‚पार्थिव‚त्वानीति \textbf{दृष्टो} यो \textbf{व‚च‚न‚भेदः स} न स्याद् [।] \textbf{ध‚र्म‚ध‚र्मिणोर‚{\tiny $_{lb}$}‚भेदे} पार‚मार्थिक‚भेदाभावे ध‚र्माणां च प‚र‚स्प‚र‚र‚म्[।]
	{\color{gray}{\rmlatinfont\textsuperscript{§~\theparCount}}}
	\pend% ending standard par
      ‚{\tiny $_{lb}$}‚

	  
	  \pstart \leavevmode% starting standard par
	\textbf{उक्त‚म}त्रेति सिद्धान्त‚वादी । \textbf{न वै श ब्दानां} काचिद् विष‚य‚स्व‚भावाय‚त्ता‚{\tiny $_{lb}$}‚ वृत्तिरित्यादिनोक्त‚त्वात्‚{\tiny $_{५}$}‚ ॥
	{\color{gray}{\rmlatinfont\textsuperscript{§~\theparCount}}}
	\pend% ending standard par
      ‚{\tiny $_{lb}$}‚

	  
	  \pstart \leavevmode% starting standard par
	भूय‚श्चाधिकार्थ‚विधानेन प्र‚तिपाद‚यितुमाह । \textbf{अपि चेत्यादि । येषां} वादिनां‚{\tiny $_{lb}$}‚ \textbf{व‚स्तुव‚शा वाचो} व‚स्त्वाय‚त्ताः । \textbf{न विव‚क्षाप‚राश्र‚याः} । विव‚क्षैव प‚रः प्र‚धान‚{\tiny $_{lb}$}‚माश्र‚यो यासां वाचान्ता विव‚क्षाप‚राश्र‚याः ष‚ष्ठी न स्याद् व‚च‚न‚भेदाद‚य‚श्च‚{\tiny $_{lb}$}‚ न स्युरित्येवं ष‚ष्ठीव‚च‚न‚भेदादिषु चोद्यं \textbf{ष‚ष्ठीव‚च‚न‚भेदोदि चोद्यं} । आदिश‚ब्दात् ।‚{\tiny $_{lb}$}‚ गोर्भावो गोत्व‚मित्यादि । त‚{\tiny $_{६}$}‚द्धित‚प्र‚त्य‚याभाव‚चोद्यं । \textbf{तान्} व‚स्तुवादिनः \textbf{प्र‚ति ।‚{\tiny $_{lb}$}‚ युक्तिम‚त् । एते} श‚ब्दाः ष‚ष्ठ्याद‚यः \textbf{क्व‚चि}दिति व‚स्त्व‚भेदेपि \textbf{प्र‚णिनीषिताः} प्र‚णेतु‚{\tiny $_{lb}$}‚मिष्टाः । \textbf{व‚स्तुप्र‚तिब‚न्धात्} । व‚स्त्वाय‚त्त‚त्वात् । \textbf{धूमादिव‚त् । न ह्य}ग्निप्र‚तिब‚द्धो धूमो‚{\tiny $_{lb}$}‚ व‚ह्निप्र‚त्याय‚न‚स‚म‚र्थ‚स्त‚द्वैप‚रीत्येन ज‚ल‚प्र‚त्याय‚ने \textbf{नियोक्तुं पार्य‚ते । त‚दा} व‚स्तुप्र‚तिब‚द्ध‚त्वे‚{\tiny $_{lb}$}‚ श‚ब्दाना\textbf{म‚य‚मुपाल‚म्भः स्या}द‚स‚ति‚{\tiny $_{७}$}‚व्य‚तिरेके \textbf{क‚थं ष‚ष्ठ्याद‚य इति} ॥
	{\color{gray}{\rmlatinfont\textsuperscript{§~\theparCount}}}
	\pend% ending standard par
      \textsuperscript{\textenglish{59b/PSVTa}}‚{\tiny $_{lb}$}‚

	  
	  \pstart \leavevmode% starting standard par
	एत‚देव नास्तीत्याह । \textbf{य‚दा पुन‚रि}त्यादि । \textbf{य‚द्} व‚चो \textbf{य‚था} येन प्र‚कारेण भेद‚स्या‚{\tiny $_{lb}$}‚भेद‚स्य वा प्र‚तिपाद‚नाय । किं विशिष्ट‚म\textbf{न‚पेक्षित‚बाह्यार्थं वाच‚क‚त्वेन} रूपेण \textbf{व‚क्तृ‚{\tiny $_{lb}$}‚‚{\tiny $_{lb}$}‚ \leavevmode\ledsidenote{\textenglish{158/s}}भिर्विनिय‚म्य‚ते । त‚त्त}थेति त‚द्व‚च‚नं य‚थायोगं \textbf{वाच}कं ।
	{\color{gray}{\rmlatinfont\textsuperscript{§~\theparCount}}}
	\pend% ending standard par
      ‚{\tiny $_{lb}$}‚

	  
	  \pstart \leavevmode% starting standard par
	त‚द्व्याच‚ष्टे । [।] \textbf{न ही}त्यादि । \textbf{व्य‚तिरेके} व‚स्तुभेदे स‚ति \textbf{ष‚ष्ठी}विभ‚क्ति\textbf{र्बाहुल्ये}‚{\tiny $_{lb}$}‚ ज‚साद‚यो ब‚हुव‚च‚न‚प्र‚त्य‚{\tiny $_{१}$}‚या भ‚व‚न्तीति वै या क र णा नां व्य‚व‚स्थान\textbf{मेत‚द‚पि‚{\tiny $_{lb}$}‚ पुरुषाभिप्राय‚निर‚पेक्ष‚म्व‚स्तुस‚न्निधिमात्रेण न स्व‚यं प्र‚वृत्तं} । संकेत‚ब‚लेनैव प्र‚वृत्त‚मिति‚{\tiny $_{lb}$}‚ याव‚त् । एत‚देवाह । \textbf{ते तु त‚त्रे}त्यादि । ते तु वैयाक‚र‚णाद‚य‚स्त‚त्र व्य‚तिरेके बाहुल्ये‚{\tiny $_{lb}$}‚ च \textbf{त‚थे}ति । ष‚ष्ठी ब‚हुव‚च‚नं च य‚थासंकेतं \textbf{प्र‚युञ्ज‚त इति} कृत्वा \textbf{त‚तः} ष‚ष्ठ्यादेस्\textbf{त‚था‚{\tiny $_{lb}$}‚ प्र‚तीति}र्भ‚व‚ति । व्य‚तिरेकादिप्र‚तीतिर‚{\tiny $_{२}$}‚न्येषाम‚पि भ‚व‚ति । त‚थाभूत‚व्य‚व‚हारोप‚ल‚म्भात् ।‚{\tiny $_{lb}$}‚ न तु ताव‚ता व‚स्तुब‚लेन व्य‚तिरेक‚बाहुल्ये च ष‚ष्ठ्यादीनां निय‚मः । \textbf{एव‚म‚न्य‚त्रापीति} ।‚{\tiny $_{lb}$}‚ ध‚र्म‚ध‚र्मिणोर‚व्य‚तिरेकेपि एक‚त्वे च व‚स्तुनः कृत क‚त्वानित्य‚त्वादीनां \textbf{क‚थंचि}दिति‚{\tiny $_{lb}$}‚ भेदान्त‚र‚प्र‚तिक्षेपाप्र‚तिक्षेप‚ल‚क्ष‚णं ध‚र्म‚ध‚र्मिणोर्भेद‚मुपादाय । कृत‚क‚त्वादिषु व्यावृत्ति‚{\tiny $_{lb}$}‚भेदोप‚ल‚क्षित‚नानात्व‚मुपा‚{\tiny $_{३}$}‚दाय । य‚थाक्र‚मं ष‚ष्ठी ब‚हुव‚च‚नाद‚य‚स्ते प्र‚योक्तृभिः‚{\tiny $_{lb}$}‚ \textbf{प्र‚युक्तास्त‚थैव} य‚थायोगं \textbf{प्र‚तीतिहेत‚वो भ‚व}न्ति । \textbf{त‚त्रै}व‚मिच्छामात्र‚निब‚न्ध‚न‚त्वे श‚ब्दानां‚{\tiny $_{lb}$}‚ स्थिते स‚ति । \textbf{पुरुषाय‚त्त‚वृत्तीना}न्त‚दिच्छाव‚शेन प्र‚वृत्ते\textbf{र‚व‚स्तुस‚न्द‚र्शिनां श‚ब्दे}भ्यः‚{\tiny $_{lb}$}‚ स्व‚ल‚क्ष‚ण‚स्याप्र‚तिभास‚नात् । \textbf{य‚थाभ्यासं} य‚स्य य‚था संकेताभ्यास‚स्त‚था \textbf{विक‚ल्प‚{\tiny $_{lb}$}‚प्र‚बोधो} विक‚ल्पोद‚य‚स्त‚{\tiny $_{४}$}‚स्य \textbf{हेतू}नां । संकेतानुरूप‚स्य श्रोतृस‚न्ताने विक‚ल्प‚स्य कार‚णा‚{\tiny $_{lb}$}‚नामित्य‚र्थः । एवं \textbf{भूतानां श‚ब्दानां वाच्ये}ष्व‚र्थेषु येय\textbf{म्प्र‚वृत्तिचि}न्ता व्य‚तिरेके ष‚ष्ठ्याद‚य‚{\tiny $_{lb}$}‚ इत्यादिका । नै या यि का दीना\textbf{न्त‚द्व‚शादि}ति श‚ब्द‚व‚शाद् \textbf{व‚स्तुव्य‚व‚स्थानं} । व्य‚ति‚{\tiny $_{lb}$}‚रिक्त‚स्य व‚स्तुनोङ्गीक‚र‚णं । गोर्गोत्व‚मिति य‚स्मात् ष‚ष्ठी त‚स्मात् सामान्यं व्य‚ति‚{\tiny $_{lb}$}‚रिक्त‚मित्यादि । \textbf{जाड्य‚ख्याप‚नं}‚{\tiny $_{५}$}‚ श‚ब्दार्थ‚व्य‚व‚स्थाऽन‚भिज्ञ‚त्व‚ख्याप‚न‚मेव \textbf{केव‚लं} ।
	{\color{gray}{\rmlatinfont\textsuperscript{§~\theparCount}}}
	\pend% ending standard par
      ‚{\tiny $_{lb}$}‚

	  
	  \pstart \leavevmode% starting standard par
	\textbf{त‚थे}त्यादि प‚रः । \textbf{त‚थाकृत‚व्य‚व‚स्था ध‚र्म‚ध‚र्म्यादिष्वि}ति ध‚र्मे ध‚र्मिणि च भेदा‚{\tiny $_{lb}$}‚न्त‚र‚प्र‚तिक्षेपाप्र‚तिक्षेपाभ्यां ध‚र्म‚ध‚र्मिश‚ब्दाः कृत‚व्य‚व‚स्थाः । आदिश‚ब्दाद् द्र‚व्य‚त्व‚{\tiny $_{lb}$}‚पार्थिव‚त्वानीत्यादिब‚हुव‚च‚न‚श‚ब्दा \textbf{व्यावृत्तिभेदेन} कृत‚व्य‚व‚स्थाः । \textbf{न पुन‚र्वास्त‚वा‚{\tiny $_{lb}$}‚देव} ध‚र्म‚ध‚र्मिणोर्व्य‚तिरेकात् ष‚ष्ठीव‚स्तुभेदा‚{\tiny $_{६}$}‚द् द्र‚व्य‚त्वादीनां ध‚र्माणां प‚र‚मार्थ‚त‚{\tiny $_{lb}$}‚ एव भेदाद्व‚हुव‚च‚न‚मिति । \textbf{कुत एत}त् [।] \textbf{त‚थे}त्यादि प्र‚तिव‚च‚नं । \textbf{त‚था व्य‚व‚हारा‚{\tiny $_{lb}$}‚‚{\tiny $_{lb}$}‚ \leavevmode\ledsidenote{\textenglish{159/s}}योगादि}ति । व्य‚व‚हार‚विष‚य‚योर्ध‚र्म‚ध‚र्मिणोर्वास्त‚वे व्य‚तिरेके । ध‚र्माणां च प‚र‚स्प‚रं‚{\tiny $_{lb}$}‚ प‚र‚प‚र‚मार्थ‚तो भेदे सामान्यादिव्य‚व‚हारायोगात् ।
	{\color{gray}{\rmlatinfont\textsuperscript{§~\theparCount}}}
	\pend% ending standard par
      ‚{\tiny $_{lb}$}‚

	  
	  \pstart \leavevmode% starting standard par
	एत‚देव‚ग्र‚ह‚ण‚क‚वाक्यं \textbf{न ही}त्यादिना व्याच‚ष्टे । व्य‚व‚हार‚विष‚य‚योर्ध\textbf{र्म‚ध‚र्मिणो}‚{\tiny $_{lb}$}‚र्व‚स्तुत्वे प‚र‚स्प‚रं‚{\tiny $_{७}$}‚ त‚त्त्व‚म‚न्य‚त्वं वाभ्युप‚ग‚न्त‚व्यं व‚स्तुनः प्र‚कारान्त‚राभावादिति \leavevmode\ledsidenote{\textenglish{60a/PSVTa}}‚{\tiny $_{lb}$}‚ द्व‚य‚मुप‚न्य‚स्तं । \textbf{भेदे त‚त्त्व‚रूप‚त्वे चे}ति प‚क्ष‚द्व‚येपि दोषोद्भाव‚नार्थ‚म‚न्य‚था प‚रेण व्य‚ति‚{\tiny $_{lb}$}‚रेक‚व‚स्तुभेदादिति भेद‚प‚क्षेऽव‚ल‚म्बिते त‚त्त्व‚प‚क्षोप‚न्यासो न प्र‚क‚र‚णानुरूपः स्यात् ।‚{\tiny $_{lb}$}‚ त‚त्स‚म्ब‚द्ध इति सामान्य‚त‚द्व‚तोः स‚म्ब‚न्धः । \textbf{श‚ब्दानाम्वा य‚थाव‚स्तु प्र‚वृत्ता}व‚भ्युप‚{\tiny $_{lb}$}‚ग‚म्य‚मानायां \textbf{सामान्याद‚यो यु‚{\tiny $_{१}$}‚ज्य‚न्ते । एत‚च्चान्त‚र‚मेव व‚क्ष्यामः । व‚स्तुकृ}त‚{\tiny $_{lb}$}‚मिति व‚स्तूनामेकानेक‚त्वादिकं \textbf{श‚ब्द‚प्र‚वृत्तिभेद}मेकंव‚च‚न‚ब‚हुव‚च‚नादीनां प्र‚वृत्ति‚{\tiny $_{lb}$}‚भेदं । दाराः श‚ब्दो नित्य‚ब‚हुव‚च‚नान्तः पुल्लिङ्ग‚श्चेष्य‚ते । य‚त्र य‚दैक‚स्त्रीविष‚यो‚{\tiny $_{lb}$}‚ \textbf{दारा} इति श‚ब्द‚स्त‚दा भेद‚व्य‚व‚स्थितेः । \textbf{ष‚ण्ण‚ग‚रीति} ब‚हूनान्न‚ग‚राणामेक‚व‚च‚ने\textbf{ना‚{\tiny $_{lb}$}‚भिधानाद‚भेद‚व्य‚व‚स्थितेः} किन्निब‚न्ध‚नं [।] बाह्यं‚{\tiny $_{२}$}‚ नैव किंचित् । \textbf{आदि}श‚ब्दात्मिक‚ता‚{\tiny $_{lb}$}‚ प्रासाद‚मालेत्यादौ भेदाभेद‚व्य‚व‚स्थितेः । \textbf{ख‚स्य स्व‚भावः ख‚त्वं चेति} ख‚स्य स्व‚भाव‚{\tiny $_{lb}$}‚ \textbf{इत्य‚त्र} व्य‚तिरेक‚ष‚ष्ठ्याः \textbf{किन्निब‚न्ध}नं । अथ त‚त्त्व‚मित्य‚नेनोक्तेन किं य‚दि ख‚स्य‚{\tiny $_{lb}$}‚ स्व‚भाव इत्य‚स्य वाक्य‚स्य ख‚त्व‚मितीय‚न्त‚द्धित‚वृत्तिर्भ‚व‚तीत्येत्क‚थ्य‚ते त‚न्नास्ति । न‚{\tiny $_{lb}$}‚ हि स्व‚भाव इत्य‚स्मिन्न‚र्थे भाव‚प्र‚त्य‚यः किन्त‚र्हि भावार्थ‚{\tiny $_{३}$}‚ । न च त‚द्धित‚वृत्तिप्र‚द‚र्श‚नेन‚{\tiny $_{lb}$}‚ किञ्चित् प्र‚योज‚न‚म‚स्त्य‚न्य‚त‚रेण व्य‚तिरेक‚प्र‚द‚र्श‚नात् ।
	{\color{gray}{\rmlatinfont\textsuperscript{§~\theparCount}}}
	\pend% ending standard par
      ‚{\tiny $_{lb}$}‚

	  
	  \pstart \leavevmode% starting standard par
	अत्रैके वृत्तिवाक्याभ्यां स‚र्वो व्य‚व‚हारो व्याप्त इति त‚द्व्याप्तिप्र‚द‚र्श‚नार्थं द्व‚य‚{\tiny $_{lb}$}‚मुक्त‚मिति ।
	{\color{gray}{\rmlatinfont\textsuperscript{§~\theparCount}}}
	\pend% ending standard par
      ‚{\tiny $_{lb}$}‚

	  
	  \pstart \leavevmode% starting standard par
	अन्येऽन्य‚था व्याच‚क्ष‚ते । ख‚स्य स्व‚भाव इति व्य‚तिरेके किन्निब‚न्ध‚नं । त‚था‚{\tiny $_{lb}$}‚ ख‚त्व‚मिति व्य‚तिरेकाभिधायिनो भाव‚प्र‚त्य‚य‚स्य किन्निब‚न्ध‚न‚मिति ।
	{\color{gray}{\rmlatinfont\textsuperscript{§~\theparCount}}}
	\pend% ending standard par
      ‚{\tiny $_{lb}$}‚‚{\tiny $_{lb}$}‚‚{\tiny $_{lb}$}‚\textsuperscript{\textenglish{160/s}}

	  
	  \pstart \leavevmode% starting standard par
	\textbf{य‚देत्यादिना} व्याच‚ष्टे‚{\tiny $_{४}$}‚ । य‚दा य‚स्मिन् काले । य‚च्छ‚ब्द‚म‚न्ये प‚ठ‚न्ति य‚स्मा‚{\tiny $_{lb}$}‚दित्य‚र्थः । \textbf{येनैव‚म्भ‚व‚तोति} दारा इत्यादि ब‚हुव‚च‚न‚म्भ‚व‚ति । एक‚त्वादेक‚व‚च‚न‚{\tiny $_{lb}$}‚मेव प्राप्नोतीति भावः । \textbf{एक‚स्या अपि स्त्रियः सिक‚ता}नां च ब‚ह्व्यः श‚क्त‚य‚स्त‚तः‚{\tiny $_{lb}$}‚ \textbf{श‚क्तिभेदो} ब‚हुव‚च‚न‚कार‚ण‚मिति । \textbf{स‚र्व‚त्रे}ति य‚त्राप्येक‚व‚च‚न‚मिष्ट‚म्वृक्ष इत्यादौ ।‚{\tiny $_{lb}$}‚ एक‚श‚क्तेर‚र्थ‚स्याभावात् । स‚र्व‚स्य ना‚{\tiny $_{५}$}‚\textbf{ना श‚क्तित्वात् । एवं} स‚त्ये\textbf{क‚स्मिन्नेक‚व‚च‚न}\edtext{}{\edlabel{pvsvt_160-5}\label{pvsvt_160-5}\lemma{त्ये}\Bfootnote{\href{http://sarit.indology.info/?cref=P\%C4\%81.1.4.22}{ Pāṇini 1: 4: 22. }}}‚{\tiny $_{lb}$}‚मित्य‚यं \textbf{य‚त्न‚श्च व्य‚र्थः स्या}त् । स‚त्य‚पि श‚क्तिभेदे \textbf{व‚स्त्व‚भेदात्} । श‚क्त्या‚{\tiny $_{lb}$}‚श्र‚य‚स्याभेदात् । अन्य‚त्रैक‚व‚च‚न‚विष‚येर्थे वृक्षः प‚ट इत्यादावे\textbf{क‚व‚च‚न‚मिति चे}त् ।‚{\tiny $_{lb}$}‚ \textbf{इहापि} दारादावेक‚व‚च‚न‚मेक‚स्याः स्त्रिया व‚स्त्व‚भेदात् । य‚त एवं न व‚स्त्व‚श‚क्त्या‚{\tiny $_{lb}$}‚श्र‚यो वा श‚ब्द‚प्र‚वृत्तिभेदः । \textbf{त‚स्माद}यं श‚{\tiny $_{६}$}‚ब्द‚प्र‚वृत्ति\textbf{निय}मो \textbf{निर्व‚स्तुको} बाह्य‚व‚स्त्व‚{\tiny $_{lb}$}‚नाश्र‚यः \textbf{क्रिय‚माणः} पुरुषे\textbf{च्छायाः स्वात‚न्त्र्}यं \textbf{श‚ब्द‚प्र‚योगे ख्याप‚य‚ति} ।
	{\color{gray}{\rmlatinfont\textsuperscript{§~\theparCount}}}
	\pend% ending standard par
      ‚{\tiny $_{lb}$}‚

	  
	  \pstart \leavevmode% starting standard par
	ष‚ण्णां न‚ग‚राणां स‚माहारः क्रियात्म‚को गुणात्म‚को वा । एकोस्ति त‚त एक‚{\tiny $_{lb}$}‚व‚च‚न‚मिति चेदाह । \textbf{न हि न‚ग‚राण्येव किंचिदिति} न‚ग‚राव‚य‚विद्र‚व्य‚स्यान‚भ्युप‚ग‚मात्‚{\tiny $_{lb}$}‚ \leavevmode\ledsidenote{\textenglish{60b/PSVTa}} \textbf{कुत‚स्ते}षान्न‚ग‚राणां \textbf{स‚माहारः} क्रियात्म‚को गुणात्म‚को वा‚{\tiny $_{७}$}‚ य‚त एव‚म‚भिधीयेत क्रिया‚{\tiny $_{lb}$}‚गुण‚योर्द्र‚व्याश्रित‚त्वात् । किं पुन‚र्द्र‚व्य‚मित्याह । \textbf{प्रासा}देत्यादि । गृहादिस‚मुदायो‚{\tiny $_{lb}$}‚ न‚ग‚रं । \textbf{विजातीया}नां च प्रासादीदानां द्र‚व्या\textbf{र‚म्भा}न‚भ्युप‚ग‚मात् कुत‚स्त‚त्स‚मुदायः‚{\tiny $_{lb}$}‚ प्रासादादिस‚मुदायो न‚ग‚रं द्र‚व्यं स्यात् । याव‚ता प्रासाद‚तोर‚ण‚पुरुषादीनां स‚मुदायो‚{\tiny $_{lb}$}‚ न‚ग‚र‚मिष्य‚ते । \textbf{तेषां} प्रासादादीनां स‚म‚स्ताना\textbf{म‚संयोगाच्च} कार‚णान्न‚{\tiny $_{१}$}‚ग‚र\textbf{न्द्र‚व्यं} ।‚{\tiny $_{lb}$}‚ संयोग‚स‚हायानां द्र‚व्याणां द्र‚व्यार‚म्भ‚क‚त्व‚मिष्य‚ते । \textbf{न} प्रासाद‚पुरुष‚कुड्यादीनां‚{\tiny $_{lb}$}‚ विश्लिष्टानां \textbf{संयोगोस्ति} । येन प्रासादादिज‚न्यं न‚ग‚र‚न्द्र‚व्यं स्यात् ।
	{\color{gray}{\rmlatinfont\textsuperscript{§~\theparCount}}}
	\pend% ending standard par
      ‚{\tiny $_{lb}$}‚

	  
	  \pstart \leavevmode% starting standard par
	स्यान्म‚तं [।] य‚द्य‚पि साक‚ल्येन प्रासादादीनां नास्ति संयोग‚स्त‚थापि येषां‚{\tiny $_{lb}$}‚ ताव‚त् प्रासाद‚पुरुषादीनां प‚र‚स्प‚रं संयोग‚स्त‚त्संयोगात्म‚कं न‚ग‚र‚म्भ‚विष्य‚त्येव‚म‚पि‚{\tiny $_{lb}$}‚  ‚{\tiny $_{lb}$}‚ ‚{\tiny $_{lb}$}‚ ‚{\tiny $_{lb}$}‚ \leavevmode\ledsidenote{\textenglish{161/s}}व‚स्तुत्वं न‚ग‚र‚स्यासिद्धं संयो‚{\tiny $_{२}$}‚ग‚स्य गुण‚प‚दार्थ‚त्वादित्य‚त आह । \textbf{न संयोग} इत्यादि ।‚{\tiny $_{lb}$}‚ न संयोग‚स्व‚भाव‚न्न‚ग‚रं । त‚था काष्ठेष्ट‚कादीनाम्विजातीयानां कार्य‚द्र‚व्यानार‚म्भात्‚{\tiny $_{lb}$}‚ प्रासादोपि न द्र‚व्यात्म‚कः किन्तु \textbf{संयोग‚स्व‚भाव} इष्य‚ते [।] संयोग‚श्च गुणो निर्गुणाश्च‚{\tiny $_{lb}$}‚ गुणा इति कुतः प्रासाद‚स्य संयोगो येन त‚त्संयोगात्म‚कं न‚ग‚रं स्यात् ।
	{\color{gray}{\rmlatinfont\textsuperscript{§~\theparCount}}}
	\pend% ending standard par
      ‚{\tiny $_{lb}$}‚

	  
	  \pstart \leavevmode% starting standard par
	एत‚देवाह । \textbf{प्रासाद‚स्}येत्यादि । \textbf{प‚रे}णे‚{\tiny $_{३}$}‚त्य‚र्थान्त‚रेणा\textbf{संयोगाच्}च न संयोगो‚{\tiny $_{lb}$}‚ न‚ग‚रं । \textbf{च}कारेणान‚न्त‚र‚निर्दिष्टात् प्रासादादीनां विश्लिष्टानाम‚संयोगाच्च न‚{\tiny $_{lb}$}‚ संयोगो न‚ग‚र‚मित्येत‚त् स‚मुच्चीय‚ते । त‚देवं प्रासादादीनामुभ‚य‚था संयोगाभावेन ।‚{\tiny $_{lb}$}‚ न‚ग‚र‚स्य संयोग‚स्व‚भाव‚ता निर‚स्ता ।
	{\color{gray}{\rmlatinfont\textsuperscript{§~\theparCount}}}
	\pend% ending standard par
      ‚{\tiny $_{lb}$}‚

	  
	  \pstart \leavevmode% starting standard par
	प्रासादादीनां या संख्या त‚दात्म‚कं न‚ग‚र‚म्भ‚विष्य‚तीति चेदाह । \textbf{त‚त एव‚{\tiny $_{lb}$}‚ संख्याभाव} इति । य‚स्मात् सं‚{\tiny $_{४}$}‚योगात्म‚क‚प्रासाद‚स्त‚त एव कार‚णात् प्रासाद‚स्य‚{\tiny $_{lb}$}‚ संख्याया अभावो निर्गुण‚त्वाद् गुणानां । संख्यापि हि गुण‚स्व‚भावा । स चासौ‚{\tiny $_{lb}$}‚ संयोग‚श्च त‚त्संयोगः । प्रासादात्म‚कः संयोग इत्य‚र्थः । \textbf{त‚त्संयोगेन पुरुषैश्च} विशिष्टा‚{\tiny $_{lb}$}‚ या \textbf{स‚त्ता} सा \textbf{न‚ग‚र‚मिति चे}त् । \textbf{किम‚स्याः स‚त्ताया एक‚त्वा}न्नित्य‚त्वाच्च \textbf{निर}ति‚{\tiny $_{lb}$}‚\textbf{श‚याया विशेष‚णं} । न हि प्रासाद‚पुरुषाद‚य‚{\tiny $_{५}$}‚स्स‚त्तां विशिंष‚न्ति । अनाधेयातिश‚य‚{\tiny $_{lb}$}‚त्वात् । त‚स्मात् स‚त्ता निर्विशेष‚णा । त‚स्या न‚ग‚र‚त्वे स‚र्व‚त्र न‚ग‚र‚त्वं स्यादित्य‚भि‚{\tiny $_{lb}$}‚प्रायः । \textbf{स‚त्तायाश्चैक‚त्वा}दिति । द्र‚व्य‚गुण‚क‚र्म‚स्वेकैव स‚त्ता व्यापिनी । \textbf{न‚ग‚र‚ब‚हुत्वेपि}‚{\tiny $_{lb}$}‚ न‚ग‚र‚व्य‚व‚स्थाश्र‚याणां प्रासादादिस‚मुदायानां ब‚हुत्वेपीत्य‚र्थः । अन्य‚था स‚त्तात्म‚के‚{\tiny $_{lb}$}‚ न‚ग‚रे प्र‚कृते न‚ग‚र‚ब‚हुत्वं क‚थं स्यात् । \textbf{द्व‚य‚{\tiny $_{६}$}‚स्}येति प्रासाद‚पुरुषादेः स‚त्तायाश्च‚{\tiny $_{lb}$}‚ या \textbf{प‚र‚स्प‚र‚स‚हित‚ता} सा \textbf{न‚ग‚र‚मिति चे}त् । एवं हि स‚ति न स‚र्व‚त्र न‚ग‚र‚बुद्धिः ।‚{\tiny $_{lb}$}‚ प्रासादादीनां स‚र्व‚त्राभावात् । प्रासादादिब‚हुत्वाद् ब‚हुव‚च‚नं च सिद्ध‚मिति‚{\tiny $_{lb}$}‚ प‚रो म‚न्य‚ते ।
	{\color{gray}{\rmlatinfont\textsuperscript{§~\theparCount}}}
	\pend% ending standard par
      ‚{\tiny $_{lb}$}‚

	  
	  \pstart \leavevmode% starting standard par
	उत्त‚र‚माह । \textbf{अनुप‚का}र्येत्यादि । \textbf{अनुप‚कार्योप‚कार‚क‚योः} स‚त्ताप्रासाद‚योः‚{\tiny $_{lb}$}‚ \textbf{क‚स्स‚हायीभावः} [।] त‚था हि द्विविधः स‚{\tiny $_{७}$}‚ह‚कारार्थः प‚र‚स्प‚रातिश‚याधानेन स‚न्ताने \leavevmode\ledsidenote{\textenglish{61a/PSVTa}}‚{\tiny $_{lb}$}‚ विशिष्ट‚क्ष‚णोत्पाद‚न‚ल‚क्ष‚णः । पूर्व‚स्व‚हेतोरेव स‚म‚र्थानामुत्प‚न्नानामेक‚कार्य‚क्रिया‚{\tiny $_{lb}$}‚  ‚{\tiny $_{lb}$}‚ ‚{\tiny $_{lb}$}‚ \leavevmode\ledsidenote{\textenglish{162/s}}ल‚क्ष‚ण‚श्च । न ताव‚त्पूर्वः स‚त्ताया अनाधेयातिश‚य‚त्वात् । नापि द्वितीयो य‚स्माद् य‚था‚{\tiny $_{lb}$}‚ स‚त्ता केव‚ला न‚ग‚र‚बुद्धिज‚न‚नं प्र‚त्य‚स‚म‚र्था त‚था प्रासादादिस‚हितापि साम‚र्थ्ये वा‚{\tiny $_{lb}$}‚ केव‚लापि ज‚न‚येत् । य‚दि च द्व‚य‚स्य प‚र‚स्प‚र‚{\tiny $_{१}$}‚स‚हित‚ता न‚ग‚रं । त‚दैक‚म‚पि न‚ग‚र‚म‚नेका‚{\tiny $_{lb}$}‚त्म‚कं प्रासादाद्यात्म‚क‚त्वात् । त‚तः \textbf{पुरुष‚संयोग‚स‚त्तानाम्ब‚हुत्वा}त् । \textbf{न‚ग‚र‚मित्येक‚व‚च‚नं‚{\tiny $_{lb}$}‚ स्या}त् । संयोग‚श‚ब्देन प्रासादात्म‚कः संयोग उक्तः । \textbf{त‚थाभूताना}मिति प‚र‚स्प‚र‚स‚हितानां‚{\tiny $_{lb}$}‚ पुरुष‚संयोग‚स‚त्तानां \textbf{क्व‚चिद}र्थ इति न‚ग‚र‚मिति विज्ञाने । श‚ब्दे च निष्पाद्ये ।‚{\tiny $_{lb}$}‚ \textbf{अभिन्नै}का \textbf{श‚क्तिर‚स्ति} ।‚{\tiny $_{२}$}‚ \textbf{से}त्य‚भिन्ना श‚क्ति\textbf{र्निमित्त}मेक‚व‚च‚न‚स्\textbf{येति चेत् । न} । किं‚{\tiny $_{lb}$}‚ कार‚णं [।] \textbf{श‚क्तेर्व‚स्तु स्व‚रूपाव्य‚तिरेकात्} । पुरुषादिभ्यो व‚स्तुरूपेभ्योऽव्य‚तिरेकात् ।‚{\tiny $_{lb}$}‚ त‚द्व‚देवानेक‚त्व‚मिति कुत‚स्त‚दाश्र‚य‚मेक‚व‚च‚नं । व‚स्तुस्व‚रूपाद् व्य‚तिरेके वा श‚क्ते‚{\tiny $_{lb}$}‚र‚भ्युप‚म्य‚म‚गाने पुरुष‚संयोग‚स‚त्ताभिर‚नुप‚कार्य‚स्य श‚क्तिरूप‚स्य पुरुषादिपार‚त‚न्त्र्य‚न्न‚{\tiny $_{lb}$}‚ स्यात् । त‚त‚श्च पुरुषादी‚{\tiny $_{३}$}‚नां श‚क्तिरिति स‚म्ब‚न्धो न स्यादिति भावः ।
	{\color{gray}{\rmlatinfont\textsuperscript{§~\theparCount}}}
	\pend% ending standard par
      ‚{\tiny $_{lb}$}‚

	  
	  \pstart \leavevmode% starting standard par
	\textbf{अथ व्य‚तिरिक्ताया अपि} श‚क्तेः पुरुषादिपार‚त‚न्त्र्य‚सिद्ध्य‚र्थं पुरुषादिकृत \textbf{उप‚कार}‚{\tiny $_{lb}$}‚ इष्य‚ते । \textbf{त‚दा श‚क्तेरुप‚कारे} वा पुरुषादिकृते इष्य‚माणे । य‚या श‚क्त्या पुरुषाद‚यः‚{\tiny $_{lb}$}‚ प्र‚थ‚मं श‚क्तिमुप‚कुर्व‚ते । त‚स्याः प्र‚थ‚म‚श‚क्\textbf{त्युप‚कारिण्या} अपि श‚क्ते \textbf{श‚क्तेर्व्य‚तिरेके‚{\tiny $_{lb}$}‚न‚व‚स्था} स्याद\textbf{व्य‚तिरेके वा । आद्यायाम‚{\tiny $_{४}$}‚प्ये}क‚व‚च‚न‚निब‚न्ध‚न‚त्वेनेष्टायां श‚क्ताव‚{\tiny $_{lb}$}‚ व्य‚तिरेक\textbf{प्र‚संगः} । अव्य‚तिरेके च व‚स्तुव‚देव बाहुल्य‚मिति \textbf{त‚द‚व‚स्थो ब‚हुषु ब‚हुव‚च‚न}\edtext{}{\edlabel{pvsvt_162-4}\label{pvsvt_162-4}\lemma{मिति}\Bfootnote{\href{http://sarit.indology.info/?cref=P\%C4\%81.1.4.21}{ Pāṇini 1: 4: 21. }}}‚{\tiny $_{lb}$}‚ प्र‚संग इति \textbf{य‚त्किञ्चिदेत}त्[।]श‚क्तिप‚रिक‚ल्प‚ने \textbf{ख‚स्य स्व‚भाव} इति \textbf{व्य‚तिरेकाश्र‚या‚{\tiny $_{lb}$}‚ ष‚ष्ठी} न स्यात् । ष‚ष्ठीकार‚ण‚त्वाद् भाव‚प्र‚त्य‚योप्युप‚चारात् ष‚ष्ठीश‚ब्देनोक्तः ।‚{\tiny $_{lb}$}‚ तेनाय‚म‚प‚रोर्थः \textbf{ख‚त्व}मिति व्य‚तिरेका‚{\tiny $_{५}$}‚श्र‚या त‚द्धितोत्प‚त्ति\textbf{र्न स्यादिति} ख‚स्य स्व‚भावः‚{\tiny $_{lb}$}‚ ख‚त्व‚मित्य‚न‚या व्युत्प‚त्त्या भाव‚प्र‚त्य‚य‚स्योत्प‚त्तेर्व्य‚तिरेकाश्र‚य‚त्वं ।
	{\color{gray}{\rmlatinfont\textsuperscript{§~\theparCount}}}
	\pend% ending standard par
      ‚{\tiny $_{lb}$}‚

	  
	  \pstart \leavevmode% starting standard par
	अथ‚वा य‚थायोगं स‚म्ब‚न्धो ग्र‚न्थ‚च्छेद‚श्च कार्यः [।] \textbf{ख‚त्व‚मिति व्य‚तिरेकाश्र‚या‚{\tiny $_{lb}$}‚  ‚{\tiny $_{lb}$}‚ ‚{\tiny $_{lb}$}‚ ‚{\tiny $_{lb}$}‚ \leavevmode\ledsidenote{\textenglish{163/s}}न स्या}त् त‚द्धितोत्प‚त्तिरित्य‚ध्याहारः । ख‚स्य स्व‚भाव इति ष‚ष्ठी न स्यादिति । न‚{\tiny $_{lb}$}‚ हि ख‚श‚ब्द‚वाच्याद् अर्थाद‚न्यः स्व‚भावोस्ति भावो वा । यो य‚थाक्र‚{\tiny $_{६}$}‚मं व्य‚तिरेक‚{\tiny $_{lb}$}‚ष‚ष्ठ्यास्त‚द्धित‚स्य वा निब‚न्ध‚नं स्यात् । ख‚त्वं नाम सामान्य‚म‚स्ति त‚द्व्य‚तिरेक‚{\tiny $_{lb}$}‚निब‚न्ध‚न‚मिति चेदाह । \textbf{न त‚त्रे}त्यादि । एकात्म‚क‚त्वात् ख‚स्य नास्मिन् ख‚त्व\textbf{सामान्यं}‚{\tiny $_{lb}$}‚ [।] य‚द्य‚पि स‚त्त्वं द्र‚व्य‚त्वं चाकाशेस्ति । त‚थापि न त‚त् ख‚स्य स्व‚भावः । घ‚टादि‚{\tiny $_{lb}$}‚साधार‚ण‚त्वात् । नापि स‚त्त्व‚द्र‚व्य‚त्वे ख‚त्व‚मित्य‚त्र भाव‚प्र‚त्य‚य‚स्य निब‚न्ध‚नं । त‚योः‚{\tiny $_{lb}$}‚ श \edtext{}{\lemma{श}\Bfootnote{?}} श‚ब्द‚{\tiny $_{७}$}‚प्र‚त्य‚याकार‚ण‚त्वात् । स्वानुरूप‚ज्ञानाभिधान‚निब‚न्ध‚न‚स्व‚भाव‚प्र‚त्य‚य‚स्य \leavevmode\ledsidenote{\textenglish{61b/PSVTa}}‚{\tiny $_{lb}$}‚ कार‚ण‚मिष्टं । \textbf{नापि विभुत्वाद‚यो गुणा} इति । आदिश‚ब्दादेक‚त्व‚प‚र‚त्वादिप‚रिग्र‚हः ।‚{\tiny $_{lb}$}‚ त‚थोच्य‚त इति ख‚स्य स्व‚भाव इति । द्र‚व्या\textbf{द‚र्थान्त‚र‚स्य} विभुत्वादेर्गुण‚स्य \textbf{त‚त्स्व‚भाव‚{\tiny $_{lb}$}‚त्वायोगाद्} आकाश‚स्व‚भाव‚त्वायोगात् । न ह्य‚र्थान्त‚र‚म‚र्थान्त‚र‚स्य स्व‚भावो युक्तः ।‚{\tiny $_{lb}$}‚ य‚दि च विभुत्वाद‚य आकाश‚{\tiny $_{१}$}‚स्व‚भावाः । त‚दा \textbf{तेषां च} विभुत्वादीनामाकाश‚स्व‚भाव‚त्वे‚{\tiny $_{lb}$}‚ \textbf{निःस्व‚भाव‚त्व‚प्र‚संगात्} । त‚था हि य‚स्तेषां गुण‚स्व‚भावः स‚त्याकाश‚मेव जातं न‚{\tiny $_{lb}$}‚ \textbf{चा}प‚र‚स्व‚भावोस्तीति निःस्व‚भाव‚ता स्यात् । तेषां च निःस्व‚भाव‚त्वे आकाश‚स्य‚{\tiny $_{lb}$}‚ व्य‚तिरिक्तः स्व‚भावो न स्यादिति भावः । विभुत्वादे\textbf{र‚प्य‚र्थान्त‚र‚स्व‚भाव‚त्व‚मिति‚{\tiny $_{lb}$}‚ चे}दाह । \textbf{त‚स्यापी}त्यादि । \textbf{त‚स्ये}ति विभुत्वादे\textbf{र‚थान्त‚रं स्व‚भा}‚{\tiny $_{२}$}‚वोऽस्येति विग्र‚हः ।‚{\tiny $_{lb}$}‚ \textbf{अतिप्र‚संग} इति य‚त्त‚द‚र्थान्त‚रं विभ‚त्वादेः स्व‚भाव‚त्वेनेष्ट‚न्त‚स्याप्य‚र्थान्त‚र‚स्व‚{\tiny $_{lb}$}‚भाव‚त्वेन भाव्यं । अन्य‚था त‚स्यापि विभुत्वाव्य‚तिरेकात् त‚द्व‚देव निःस्व‚भाव‚ता‚{\tiny $_{lb}$}‚ स्यात् । \textbf{त‚था} चाप‚राप‚र‚स्व‚भाव‚प‚रिमार्ग‚णेनान‚व‚स्थानादेक‚स्यापि प्र‚तिष्ठित‚स्व‚{\tiny $_{lb}$}‚भाव‚स्याभावात् । आद्य‚स्याकाश‚स्व‚भाव‚स्या\textbf{प्र‚तिप‚त्ति}स्त‚त‚श्च स एव व्य‚तिरेकाभा‚{\tiny $_{३}$}‚व‚{\tiny $_{lb}$}‚ इत्य‚भिप्रायः ।
	{\color{gray}{\rmlatinfont\textsuperscript{§~\theparCount}}}
	\pend% ending standard par
      ‚{\tiny $_{lb}$}‚

	  
	  \pstart \leavevmode% starting standard par
	\textbf{एवं} द्र‚व्य‚गुण‚क‚र्म‚सामान्य‚विशेष‚स‚म‚वायानां\textbf{\edtext{\textsuperscript{*}}{\edlabel{pvsvt_163-1}\label{pvsvt_163-1}\lemma{*}\Bfootnote{\href{http://sarit.indology.info/?cref=vs\%C5\%AB.1.4}{ Vaiśeṣikasūtra 1:4. }}}ष‚ण्णां प‚दार्थानां व‚र्गः । आदि}‚{\tiny $_{lb}$}‚श‚ब्दात् प्रासाद‚मालेत्याद‚यो \textbf{वाच्याः} ।
	{\color{gray}{\rmlatinfont\textsuperscript{§~\theparCount}}}
	\pend% ending standard par
      ‚{\tiny $_{lb}$}‚

	  
	  \pstart \leavevmode% starting standard par
	क‚थ‚म‚स‚ति व्य‚तिरेके ष‚ष्ठीति । \textbf{न हि त‚त्र} ष‚ट्प‚दार्थेषु \textbf{सामान्यं स‚म्भ‚व‚ति} ।‚{\tiny $_{lb}$}‚ ‚{\tiny $_{lb}$}‚ ‚{\tiny $_{lb}$}‚ \leavevmode\ledsidenote{\textenglish{164/s}}य‚द्व‚र्ग‚श‚ब्देनोच्य‚ते । द्र‚व्य‚गुण‚क‚र्म‚स्वेव सामान्याभ्युप‚ग‚मात् । त‚था \textbf{संख्या संयोगो‚{\tiny $_{lb}$}‚ वा} न \textbf{स‚म्भ‚व‚ति} त‚योर्गुण‚प‚दार्थ‚त्वेन द्र‚{\tiny $_{४}$}‚व्य एव भावात् ।
	{\color{gray}{\rmlatinfont\textsuperscript{§~\theparCount}}}
	\pend% ending standard par
      ‚{\tiny $_{lb}$}‚

	  
	  \pstart \leavevmode% starting standard par
	\textbf{क‚थ‚मि}त्यादि प‚रः । \textbf{इदानी}मित्य‚र्थान्त‚र‚स्व‚भावान‚भ्युप‚ग‚मे ख‚श‚ब्द‚वाच्य‚स्य‚{\tiny $_{lb}$}‚ भाव‚श‚ब्द‚वाच्य‚स्य चार्थ‚स्यास‚त्य‚निश्च‚ये क‚थं \textbf{स्व‚भाव इति} भेदेन निर्देशः । \textbf{न पुनः‚{\tiny $_{lb}$}‚ ख‚मित्येव} । भेद‚निब‚न्धाभावाद‚भेदेनैव निर्देशो न्याय्य इत्य‚र्थः ।
	{\color{gray}{\rmlatinfont\textsuperscript{§~\theparCount}}}
	\pend% ending standard par
      ‚{\tiny $_{lb}$}‚

	  
	  \pstart \leavevmode% starting standard par
	\textbf{ख‚स्ये}त्यादिना प‚रिह‚र‚ति । ख‚स्येति ख‚श‚ब्द‚वाच्य‚स्यार्थ‚स्या\textbf{र्थान्त‚र}स्येति पृथि‚{\tiny $_{lb}$}‚‚{\tiny $_{५}$}‚व्यादेर्य\textbf{त्साधार‚णं रूप‚म}नुपात्त‚विशेषान्त‚र‚स्या\textbf{प‚राम‚र्शेन} । त्यागेन । ख‚श‚ब्द‚{\tiny $_{lb}$}‚\textbf{प्र‚वृत्तिनिब‚न्ध‚नं रूप}म‚न्य‚द्र‚व्यासाधार‚णं \textbf{त‚था जिज्ञासाया} अत्रार्थान्त‚रासंस‚र्गि{...}‚{\tiny $_{lb}$}‚ \textbf{{...}मेव‚मुच्य‚ते} ख‚स्य स्व‚भाव इति । त‚था ख‚त्व‚मिति । य‚था ग‚म्येत ख‚स्यायं‚{\tiny $_{lb}$}‚ स्व‚भावो नान्य‚स्येति ।
	{\color{gray}{\rmlatinfont\textsuperscript{§~\theparCount}}}
	\pend% ending standard par
      ‚{\tiny $_{lb}$}‚

	  
	  \pstart \leavevmode% starting standard par
	अनेन भेदान्त‚र‚प्र‚तिक्षेपेणैक\textbf{व्यावृत्त‚रूपान‚भिधाना}द‚न्य‚देव व्य‚तिरेकाभिधान‚नि‚{\tiny $_{lb}$}‚मित्त‚{\tiny $_{६}$}‚मुक्तं । तेन घ‚ट‚स्याभाव इत्यादिव्य‚प‚देशः सिद्धो भ‚व‚ति ।
	{\color{gray}{\rmlatinfont\textsuperscript{§~\theparCount}}}
	\pend% ending standard par
      ‚{\tiny $_{lb}$}‚

	  
	  \pstart \leavevmode% starting standard par
	अन्ये पुन‚राहुः । ख‚स्य स्व‚भावं प्र‚तिपाद‚य‚न् स्व‚भाव‚श‚ब्दो स्व‚भाव‚व्यावृत्ति‚{\tiny $_{lb}$}‚मात्रेण प्र‚तिपाद‚य‚ति न तु स्व‚भावान्त‚राप्र‚तिक्षेपेण । ख‚श‚ब्द‚स्तु त‚मेव रूपान्त‚रा‚{\tiny $_{lb}$}‚प्र‚तिक्षेपेण । त‚तः स्व‚भावान्त‚र‚प्र‚तिक्षेपाप्र‚तिक्षेप‚ल‚क्ष‚णेन भेद‚लेशेन ख‚स्यायं स्व‚भावः‚{\tiny $_{lb}$}‚ \leavevmode\ledsidenote{\textenglish{62a/PSVTa}} खत्वमि‚{\tiny $_{७}$}‚ति भेदेन निर्द्दिश्य‚त इति त‚देत‚न्नातिश्लिष्टं भेदान्त‚र‚प्र‚तिक्षेपाप्र‚तिप‚त्तेः ।
	{\color{gray}{\rmlatinfont\textsuperscript{§~\theparCount}}}
	\pend% ending standard par
      ‚{\tiny $_{lb}$}‚

	  
	  \pstart \leavevmode% starting standard par
	\textbf{अर्थान्त‚र‚साधार‚ण‚रूपाप‚राम‚र्शेन ख‚श‚ब्द‚प्र‚वृत्तिनिब‚न्ध}नं रूप‚मे\textbf{व‚मुच्य‚त} इति‚{\tiny $_{lb}$}‚ ब्रुव‚ता स्व‚ल‚क्ष‚ण‚मेव वाच्य‚मुक्त‚मिति म‚त्वा प‚रो ब्रूते इत्यादि । न \textbf{तु स‚र्व} इत्यादि ।
	{\color{gray}{\rmlatinfont\textsuperscript{§~\theparCount}}}
	\pend% ending standard par
      ‚{\tiny $_{lb}$}‚

	  
	  \pstart \leavevmode% starting standard par
	\textbf{ने}त्यादि सिद्धान्त‚वादी । य‚त् \textbf{स‚र्व}स्माद् \textbf{व्यावृ}त्तं स्व‚ल‚क्ष‚णात्म‚क\textbf{न्त‚देव रूपं}‚{\tiny $_{lb}$}‚ श‚ब्दोत्थायां \textbf{बुद्धौ} श‚ब्दैः स\textbf{म‚र्प्य‚ते । ने}ति स‚म्ब‚न्धः । क‚स्मा‚{\tiny $_{१}$}‚त् त‚स्य स्व‚ल‚क्ष‚ण‚स्येन्द्रिय‚{\tiny $_{lb}$}‚बुद्धाविव । शाब्दे विज्ञाने प्र‚त्य‚व‚भास‚ने स‚त्य\textbf{म‚तीन्द्रिय‚त्व‚प्र‚स‚ङ्गा}त् । प्र‚त्य‚क्ष‚त्व‚प्र‚स‚{\tiny $_{lb}$}‚ङ्गात् । किन्त‚र्हि श‚ब्देन क्रिय‚त इति चेदाह । \textbf{केव‚ल}मित्यादि । \textbf{अय}मिति प्र‚तिपाद‚कः ।‚{\tiny $_{lb}$}‚ दृश्य‚विक‚ल्प‚योरेक‚त्वाध्य‚व‚साय‚विप्र‚ल‚ब्ध‚स्त‚थाभूत‚म‚साधार‚ण‚म‚र्थं प्र‚त्याय‚यिष्यामीत्ये‚{\tiny $_{lb}$}‚व\textbf{म‚भिप्रा}यः \textbf{श‚ब्देन} क‚र‚ण‚भूतेन \textbf{श्रोत}रि यो \textbf{विक‚ल्प}स्त‚स्य \textbf{प्र‚तिविम्ब}बाह्य‚{\tiny $_{२}$}‚त‚याऽ\textbf{ध्य‚{\tiny $_{lb}$}‚‚{\tiny $_{lb}$}‚ \leavevmode\ledsidenote{\textenglish{165/s}}स्त}माकार‚म\textbf{र्प‚य‚ति} [।] किं भूत\textbf{म‚संसृष्ट‚त‚त्स्व‚भा}वं । अगृहीत‚व‚स्तुरूपं । आचार्य‚{\tiny $_{lb}$}‚दि ङ् ना ग स्याप्येत‚द‚भिम‚त‚मित्याह । \textbf{य‚दाहे}त्यादि । अदृष्टार्थे स्व‚र्गादिश‚ब्दे‚{\tiny $_{lb}$}‚ उच्च‚रि\textbf{तेर्थ‚विक‚ल्प‚मात्रं} श्रोतुर्भ‚व‚त्य‚ध्य‚व‚सित‚बाह्यार्थ‚स्व‚भावो विक‚ल्पो भ‚व‚ति न‚{\tiny $_{lb}$}‚ तु बाह्य‚स्व‚रूप‚ग्राह‚कं ।
	{\color{gray}{\rmlatinfont\textsuperscript{§~\theparCount}}}
	\pend% ending standard par
      ‚{\tiny $_{lb}$}‚

	  
	  \pstart \leavevmode% starting standard par
	अनेन चा चा र्ये णा पि विध्य‚र्थः श‚ब्दार्थोऽभिप्रेत इति द‚र्श‚य‚ति । \textbf{नैवं} विक‚ल्प‚{\tiny $_{lb}$}‚प्र‚ति‚{\tiny $_{३}$}‚विम्बे श‚ब्देन श्रोत‚रि ज‚निते \textbf{प्र‚तिपाद्य‚प्र‚तिपाद‚काभ्यां} य‚थासंख्यं \textbf{स्व‚ल‚क्ष}णं‚{\tiny $_{lb}$}‚ \textbf{प्र‚तिप‚न्नं प्र‚तिपादितं वा भ‚व‚ति} ।
	{\color{gray}{\rmlatinfont\textsuperscript{§~\theparCount}}}
	\pend% ending standard par
      ‚{\tiny $_{lb}$}‚

	  
	  \pstart \leavevmode% starting standard par
	य‚दि हि श‚ब्देन स्व‚ल‚क्ष‚णं प्र‚तिपाद्य‚ते । त‚दा \textbf{स्व‚र्गा}दिश‚ब्द\textbf{श्र‚व‚णेपि । त‚द‚नु‚{\tiny $_{lb}$}‚भ‚विनामिव} । स्व‚र्गादिप्र‚त्य‚क्ष‚वेदिनामिव \textbf{प्र‚तिभासाभेदः} स्यात् । श्रोतुर‚पि‚{\tiny $_{lb}$}‚ स्व‚र्गादिस्व‚ल‚क्ष‚णाकार‚प्र‚तिप‚त्तिः स्यात् । य‚त‚श्च न श‚ब्दात् स्व‚ल‚क्ष‚ण‚प्र‚तिप‚त्ति\textbf{स्त‚{\tiny $_{४}$}‚‚{\tiny $_{lb}$}‚स्माद‚यं} श्रोता श‚ब्दाद\textbf{प्र‚तिप‚द्य‚मानोपि भाव‚स्व‚भाव‚स्त‚थाभूत ए}वासंसृष्ट‚व‚स्तुस्व‚भाव‚{\tiny $_{lb}$}‚ एव \textbf{विक‚ल्प‚प्र‚तिबिम्बे त‚द‚ध्य‚व‚सायी} स्व‚ल‚क्ष‚णाध्य‚व‚सायी स्व‚ल‚क्ष‚ण‚मेव म‚या प्र‚ति‚{\tiny $_{lb}$}‚प‚न्न‚मिति \textbf{स‚न्तुष्य‚ति} [।] किं कार‚णं [।] \textbf{त‚था भूत‚त्वादेव} । स्व‚ल‚क्ष‚ण‚स्याग्र‚हेप्य‚{\tiny $_{lb}$}‚ध्य‚व‚सित‚स्व‚ल‚क्ष‚ण‚रूप‚त्वादेव श‚ब्दार्थ‚प्र‚तिप‚त्तेः । य‚त‚श्च स्व‚ल‚क्ष‚णाध्य‚व‚सायेन‚{\tiny $_{lb}$}‚ \textbf{श‚ब्दार्थ‚स्य प्र‚तीतिस्ते‚{\tiny $_{५}$}‚नैत‚देव‚मुच्य‚ते} श‚ब्दः \textbf{स्व‚रूप‚माहे}ति । अर्थान्त‚र‚साधार‚ण‚{\tiny $_{lb}$}‚रूपाप‚राम‚र्शेन ख‚श‚ब्द‚प्र‚वॄत्तिनिब‚न्ध‚म‚साधार‚णं रूप‚मुच्य‚त इति व‚च‚नात् । \textbf{न पुनः‚{\tiny $_{lb}$}‚ स्व‚रूप‚प्र‚तिभास‚स्यैव विज्ञान‚स्य ज‚न‚नात्} ।
	{\color{gray}{\rmlatinfont\textsuperscript{§~\theparCount}}}
	\pend% ending standard par
      ‚{\tiny $_{lb}$}‚

	  
	  \pstart \leavevmode% starting standard par
	त‚स्मात् स्थित‚मेत‚द् विध्य‚र्थः श‚ब्दार्थोलीक‚त्वात् प‚र‚म‚ताद् भेद इति ।
	{\color{gray}{\rmlatinfont\textsuperscript{§~\theparCount}}}
	\pend% ending standard par
      ‚{\tiny $_{lb}$}‚

	  
	  \pstart \leavevmode% starting standard par
	\textbf{क‚थ‚मि}त्यादि प‚रः । \textbf{एकान्त‚व्यावृत्तं} रूपं येषामिति विग्र‚हः । \textbf{तेषाम्भावाना‚{\tiny $_{lb}$}‚म‚संस‚र्गाद‚{\tiny $_{६}$}‚न्य‚स्य चे}ति भावेभ्यो व्य‚तिरिक्त‚स्य सामान्य‚स्य ।
	{\color{gray}{\rmlatinfont\textsuperscript{§~\theparCount}}}
	\pend% ending standard par
      ‚{\tiny $_{lb}$}‚

	  
	  \pstart \leavevmode% starting standard par
	\textbf{उक्त}मिति सिद्धान्त‚वादी । कीदृश‚मुक्त‚मित्याह । \textbf{असंसृष्टानां} प‚र‚स्प‚र‚व्यावृत्ता‚{\tiny $_{lb}$}‚‚{\tiny $_{lb}$}‚ \leavevmode\ledsidenote{\textenglish{166/s}}नामे\textbf{केन} विजातीयेना\textbf{संस‚र्गः} । य‚था गोव्य‚क्तीनामेकेनागोस्व‚भावेनासंस‚र्गः \textbf{स‚{\tiny $_{lb}$}‚ एवा}संस‚र्गः । \textbf{त‚द्व्य‚तिरेकिणान्}त‚स्माद‚गोस्व‚भावाद् व्यावृत्तानां गोभेदानां \textbf{स‚मान‚ता}‚{\tiny $_{lb}$}‚ \leavevmode\ledsidenote{\textenglish{62b/PSVTa}} गोत्वं । एव‚म‚न्य‚द‚पि सामान्यं बोद्ध‚व्यं ।
	{\color{gray}{\rmlatinfont\textsuperscript{§~\theparCount}}}
	\pend% ending standard par
      ‚{\tiny $_{lb}$}‚

	  
	  \pstart \leavevmode% starting standard par
	एत‚दुक्त‚म्भ‚व‚ति । स‚मानामिति क‚र्त्त‚रि ष‚ष्ठी तेन स‚मानानाम्भावः सामान्य‚{\tiny $_{lb}$}‚मिति विजातीय‚व्यावृत्ताः स‚मानाकारोत्प‚न्ना भावाः सामान्य‚मित्य‚र्थः । अस्यैव‚{\tiny $_{lb}$}‚ च सामान्य‚स्य लिङ्ग‚त्वं लिङ्गित्व‚न्त‚त्स‚म्ब‚न्ध‚श्च व‚स्तुत्वात् । अर्थ‚क्रियाकारि‚{\tiny $_{lb}$}‚त्वाच्च प्र‚वृत्तिविष‚य‚त्वं विक‚ल्प‚विष‚य‚त्वं च य‚थाऽध्य‚व‚सायं न तु ज्ञानाकार‚स्य ।‚{\tiny $_{lb}$}‚ एत‚द्विप‚रीत‚त्वात् ॥
	{\color{gray}{\rmlatinfont\textsuperscript{§~\theparCount}}}
	\pend% ending standard par
      ‚{\tiny $_{lb}$}‚

	  
	  \pstart \leavevmode% starting standard par
	त‚देवं स‚मारोप‚प‚क्षे प‚रोक्षं दूष‚ण‚म्प‚रिहृत्यान्य‚व्यावृत्तिप‚क्षे प‚रोक्षं दूष‚ण‚म‚न्य‚{\tiny $_{lb}$}‚व्यावृत्त्य‚न‚भ्युप‚ग‚मादेव निर‚स्यान्य‚व्यावृत्ता एव भावा एक‚त्वेनाध्य‚व‚सीय‚मानाः‚{\tiny $_{lb}$}‚ सामान्य‚मिति च प्र‚तिपाद्य बुद्ध्याकारेपि सामान्ये प‚रोक्तं दूष‚ण‚म‚प‚नेतुमाह । \textbf{अपि‚{\tiny $_{lb}$}‚ चे}त्यादि । \textbf{भावानाश्रित्य भेदिन} इति व्यावृत्तानि स्व‚ल‚क्ष‚ण‚म्वाश्रित्य धीरे\textbf{कार्थ‚{\tiny $_{lb}$}‚प्र‚तिभासि‚{\tiny $_{२}$}‚न्यु}त्प‚द्य‚ते । \textbf{य‚या स्व‚रू}पेण स्वाकारेणैकेन रूपेण । \textbf{प‚र‚रूपं} । प‚र‚स्प‚र‚व्यावृत्तं‚{\tiny $_{lb}$}‚ व‚ल‚क्ष‚णं \textbf{स‚म्व्रिय‚ते} प्र‚च्छाद्य‚ते । दिति । धि । किम्विशिष्ट‚या या [।] \textbf{संवृत्या} ।‚{\tiny $_{lb}$}‚ संव्रिय‚तेऽन‚या स्व‚ल‚क्ष‚ण‚मिति कृत्वा । \textbf{स‚म्वृत‚नानार्था\textbf{ः}} स्थ‚गित‚नानार्थाः‚{\tiny $_{lb}$}‚ \textbf{स्व‚य‚म्भेदिनो}पि \textbf{केन‚चिद् रूपेण} विजातीय‚व्यावृत्त्युप‚क‚ल्पितेन गोत्वादिरूपेणा‚{\tiny $_{lb}$}‚\textbf{भेदिन इवाभान्ति} संसृष्ट इव । तेन [।]‚{\tiny $_{lb}$}‚ 
	    \pend% close preceding par
	  
	    
	    \stanza[\smallbreak]
	  {\normalfontlatin\large ``\qquad}बुद्ध्या‚{\tiny $_{३}$}‚कार‚श्च बुद्धिस्थो नार्थ‚बुद्ध्य‚न्त‚रानुगः ।&‚{\tiny $_{lb}$}‚नाभिप्रेतार्थ‚कारी च सोपि वाच्यो न त‚त्व‚त{\normalfontlatin\large\qquad{}"}\&[\smallbreak]
	  
	  
	  
	    \pstart  \leavevmode% new par for following
	    \hphantom{.}
	   इत्यादि निर‚स्तं ॥
	{\color{gray}{\rmlatinfont\textsuperscript{§~\theparCount}}}
	\pend% ending standard par
      ‚{\tiny $_{lb}$}‚

	  
	  \pstart \leavevmode% starting standard par
	त‚स्माद् बुद्धे\textbf{र‚भिप्राय‚व}शात् । एकाकाराध्यारोप‚व‚शात् \textbf{सामान्यंस} केस्न‚{\tiny $_{lb}$}‚\textbf{प्र‚कीर्तितं} । बुद्ध्यारोपित एवाकारः सामान्य‚मुक्त‚मिति याव‚त् । \textbf{य‚था} त‚या \textbf{संक‚{\tiny $_{lb}$}‚‚{\tiny $_{lb}$}‚ \leavevmode\ledsidenote{\textenglish{167/s}}ल्पित‚मा}रोपितं \textbf{त‚था त‚त्} सामान्य\textbf{म‚स‚त् प‚र‚मार्थेन} ।
	{\color{gray}{\rmlatinfont\textsuperscript{§~\theparCount}}}
	\pend% ending standard par
      ‚{\tiny $_{lb}$}‚

	  
	  \pstart \leavevmode% starting standard par
	न‚नु य‚द्य‚भिन्नः प्र‚तिभासोऽभ्युप‚ग‚म्य‚{\tiny $_{४}$}‚ते क‚थं सामान्य‚म‚स‚दित्युच्य‚ते । न [।]‚{\tiny $_{lb}$}‚ व्य‚क्तिभ्यो भिन्न‚स्याभिम‚त‚स्याभेदेन प्र‚तिभास‚नात् । स‚र्वात्म‚ना चाभेदे व्य‚क्तिव‚द‚{\tiny $_{lb}$}‚न‚न्व‚याद‚नुग‚त‚स्य प्र‚तीतिर्भ्रान्तिरेव । भेदाभेदे च त‚थैवोभ‚य‚रूप‚त‚या प्र‚तिभासः स्यात्‚{\tiny $_{lb}$}‚ [।] न च य‚त्र प्र‚त्य‚ये य‚दैव गौरिति प्र‚तिभासः त‚दैव त‚त्र गोत्व‚म‚स्येति प्र‚तिभासः ।
	{\color{gray}{\rmlatinfont\textsuperscript{§~\theparCount}}}
	\pend% ending standard par
      ‚{\tiny $_{lb}$}‚

	  
	  \pstart \leavevmode% starting standard par
	अथ स्याद् [।] य‚द‚न‚नुग‚मे य‚स्यानुग‚म‚स्त‚त्त‚तो भिन्नं य‚{\tiny $_{५}$}‚था नीलाद‚नीलं ।‚{\tiny $_{lb}$}‚ शाव‚लेयान‚नुग‚मे च गोत्व‚स्यानुग‚मो व्य‚क्त्य‚न्त‚रे [।] त‚स्मात् त‚तो भिन्नं शाव‚लेया‚{\tiny $_{lb}$}‚भिन्न‚गोत्व‚ग्राह‚क‚न्त‚र्हि प्र‚त्य‚क्षं भ्रान्तं स्याद् भिन्न‚स्यान्य‚प्र‚माण‚गृहीत‚स्याभेदेन‚{\tiny $_{lb}$}‚ ग्र‚ह‚णात् । न चैक‚स्य येनैव रूपेण भेद‚स्तेनैवाभेदो विरोधात् । नापि व‚स्तूनां विशेष‚{\tiny $_{lb}$}‚रूप‚त‚या भेदः । सामान्य‚रूप‚त‚याऽभेदः । एवं हि विशेषाणामेक‚देशा‚{\tiny $_{६}$}‚दित्वं स्यान्न‚{\tiny $_{lb}$}‚ च स्यात् । सामान्य‚विशेष‚योश्च प्र‚तिभास‚भेदो न स्यात् स‚र्वात्म‚नाऽभिन्न‚त्वाद् ।‚{\tiny $_{lb}$}‚ भेदे वानुग‚त‚व्यावृत्त‚रूप‚योः प‚र‚स्प‚रासंश्लेषेणैकान्त‚भिन्न‚त्व‚प्र‚स‚ङ्गात् ।
	{\color{gray}{\rmlatinfont\textsuperscript{§~\theparCount}}}
	\pend% ending standard par
      ‚{\tiny $_{lb}$}‚

	  
	  \pstart \leavevmode% starting standard par
	य‚दि च प्र‚तिव्य‚क्ति त‚द‚भिन्न‚न्त‚दैकं सामान्य‚म‚स्तीति कुतः । न च त‚त्र प्र‚थ‚म‚{\tiny $_{lb}$}‚व्य‚क्तिद‚र्श‚नेऽगृहीत‚म‚पि द्वितीयादिव्य‚क्तौ स एवाय‚मिति प्र‚तीतेरेकं‚{\tiny $_{७}$}‚ सामान्य- \leavevmode\ledsidenote{\textenglish{63a/PSVTa}}‚{\tiny $_{lb}$}‚ म‚स्तीति युक्तं [।] स इति स्म‚र‚णांश‚स्यानुभूत‚सामान्य‚विष‚याभावेनोत्प‚त्त्य‚स‚म्भ‚{\tiny $_{lb}$}‚वात् । त‚त‚श्च य‚था प्र‚थ‚म‚व्य‚क्तौ न गृहीत‚न्त‚था द्वितीयादिव्य‚क्ताव‚पि प्र‚त्येकं गृहीत‚{\tiny $_{lb}$}‚मिति क‚थं स एवाय‚मिति ज्ञान‚स्योत्प‚त्तिः स्यात् । न च स इति स्म‚र‚णान्य‚थानु‚{\tiny $_{lb}$}‚प‚प‚त्त्या प्र‚थ‚म‚व्य‚क्तौ निर्विक‚ल्प‚केनान्येन वा ज्ञानेन सामान्य‚ग्र‚ह‚ण‚क‚ल्प‚ना युक्ताऽ‚{\tiny $_{lb}$}‚नुग‚त‚{\tiny $_{१}$}‚रूप‚प्र‚तिभासाभावात् । नापि ध‚र्म‚रूप‚त‚या त‚त्रास्य ग्र‚ह‚णं युक्त‚म‚नुग‚तैक‚{\tiny $_{lb}$}‚रूप‚त्वात् सामान्य‚स्य । नापि पूर्वाप‚र‚व्य‚क्तिस‚म्ब‚न्धित‚यास्य प्र‚त्य‚भिज्ञाप्र‚त्य‚क्षेण‚{\tiny $_{lb}$}‚ ग्र‚ह‚ण‚मिति श‚क्य‚ते व‚क्तुं । इद‚मेवेत्येवं प्र‚त्य‚क्ष‚स्योत्प‚त्तिप्र‚संगात् । पूर्व‚व्य‚क्तेर‚स‚{\tiny $_{lb}$}‚न्निहित‚त्वाच्च स‚न्निहितार्थ‚निश्च‚य‚ल‚क्ष‚णं च प्र‚त्य‚क्ष‚मिष्य‚ते ।
	{\color{gray}{\rmlatinfont\textsuperscript{§~\theparCount}}}
	\pend% ending standard par
      ‚{\tiny $_{lb}$}‚

	  
	  \pstart \leavevmode% starting standard par
	त‚स्माद‚गोव्यावृत्तैक‚विशेषानुभ‚व‚पूर्व‚क‚{\tiny $_{२}$}‚म‚न्य‚स्मिन्न‚गोव्यावृत्तेऽनुभूय‚माने स‚ति स‚{\tiny $_{lb}$}‚ एवाय‚मित्येकाकार‚स्य भ्रान्त‚स्य प्र‚त्य‚य‚स्य वास‚नासाम‚र्थ्येनोत्प‚त्तेस्त‚देवास्य निमित्तं‚{\tiny $_{lb}$}‚ क‚ल्प‚यितुं युक्त‚न्नान्य‚द् [।] भ्रान्त‚त्वाच्च बुद्धेर‚यं सामान्याकारो व्य‚व‚स्थाप्य‚ते‚{\tiny $_{lb}$}‚ न वाह्य‚स्य [।] य‚थाप्र‚तिभासं तु बाह्यानामेवार्थानां सामान्य‚मुच्य‚ते ।
	{\color{gray}{\rmlatinfont\textsuperscript{§~\theparCount}}}
	\pend% ending standard par
      ‚{\tiny $_{lb}$}‚

	  
	  \pstart \leavevmode% starting standard par
	\textbf{बुद्धिरि}त्यादिना कारिकार्थं व्याच‚ष्टे । बुद्धि\textbf{र्विक‚ल्पिके}त्य‚नेन स‚म्ब‚न्धः ।‚{\tiny $_{lb}$}‚ तेभ्यः‚{\tiny $_{३}$}‚ स‚जातीयेभ्योन्य‚स्त‚स्माद् व्य‚तिरेकः स येषाम‚स्ति । ते \textbf{त‚द‚न्य‚व्य‚तिरेकिणः} ।‚{\tiny $_{lb}$}‚ विजातीय‚व्यावृत्तानित्य‚र्थः । \textbf{स्व‚वास‚नाप्र‚कृतिं} विक‚ल्प\textbf{वास‚नास्व‚भाव}म्भिन्नेष्व‚भि‚{\tiny $_{lb}$}‚‚{\tiny $_{lb}$}‚ \leavevmode\ledsidenote{\textenglish{168/s}}न्न‚प्र‚त्य‚य‚ज‚न‚न‚साम‚र्थ्य‚न्त‚द‚नु\textbf{विद‚ध‚ती} । अनुकुर्व‚ती । एवं हि त‚या स्व‚कार‚ण‚म‚नुकृत‚{\tiny $_{lb}$}‚म्भ‚व‚ति [।] य‚दि भिन्नेष्व‚प्येकाकारोत्प‚द्य‚ते । \textbf{एषा}मिति प‚दार्थानां । \textbf{अध्य‚स्ये}ति‚{\tiny $_{lb}$}‚ प‚दार्थेष्वारोप्य‚{\tiny $_{४}$}‚ भावानामेवैकं रूप‚मिति । \textbf{तान्} भावान् \textbf{सृज‚न्ती । अभिन्नानिव}‚{\tiny $_{lb}$}‚ कुर्वाणा \textbf{स‚न्द‚र्श‚य‚ति} । एक‚रूपानिव द‚र्श‚य‚तीति याव‚द् ।
	{\color{gray}{\rmlatinfont\textsuperscript{§~\theparCount}}}
	\pend% ending standard par
      ‚{\tiny $_{lb}$}‚

	  
	  \pstart \leavevmode% starting standard par
	न‚नु विक‚ल्प‚स्यानुभूतार्थाहित‚वास‚नाब‚लोत्प‚त्ताविष्य‚माणायां क‚थं सामान्या‚{\tiny $_{lb}$}‚कार‚स्य विक‚ल्प‚स्योत्प‚त्तिः सामान्य‚स्यान‚नुभूत‚त्वादित्य‚त आह । \textbf{सा चे}त्यादि ।‚{\tiny $_{lb}$}‚ एकं स‚दृशं साध्यं साध‚न‚ञ्च येषाम्भावा\textbf{नां} । य‚{\tiny $_{५}$}‚था घ‚टादीनामेक‚मुद‚कादिधार‚णा‚{\tiny $_{lb}$}‚दि साध्यं । साध‚नं च मृत्पिण्डादि । ते \textbf{एक‚साध्य‚साध‚ना}स्त‚द्भाव\textbf{स्त‚या} । क‚र‚ण‚{\tiny $_{lb}$}‚भूत‚या । \textbf{अन्ये}भ्योत‚त्साध्य‚साध‚नेभ्यो \textbf{विवेकिनां भावाना} सा प्र‚कृतिः स्व‚भावः‚{\tiny $_{lb}$}‚ भिन्नानाम‚पि प्र‚कृत्या एकाकार‚विक‚ल्प‚ज‚न‚न‚ल‚क्ष‚णः । त‚थाभूत‚स्य \textbf{विक‚ल्प‚स्य}‚{\tiny $_{lb}$}‚ हेतुर्या \textbf{वास}ना । त‚स्याश्च सा तादृशी \textbf{प्र‚कृति}र्य‚देवं स्व‚रूपं प‚र‚{\tiny $_{६}$}‚त्रारोप‚य‚न्ती । एषा‚{\tiny $_{lb}$}‚ बुद्धिः प्र‚तिभाति । व्य‚क्त्य‚भिन्न‚सामान्य‚ग्राहिणी प्र‚तिभास‚त इति याव‚त् ।
	{\color{gray}{\rmlatinfont\textsuperscript{§~\theparCount}}}
	\pend% ending standard par
      ‚{\tiny $_{lb}$}‚

	  
	  \pstart \leavevmode% starting standard par
	तेन य‚दुच्य‚ते ।
	{\color{gray}{\rmlatinfont\textsuperscript{§~\theparCount}}}
	\pend% ending standard par
      ‚{\tiny $_{lb}$}‚
	  \bigskip
	  \begingroup
	
	    
	    \stanza[\smallbreak]
	  {\normalfontlatin\large ``\qquad}स्मार्त्त‚मेत‚द‚भेदेन विज्ञान‚मिति यो व‚देत् ।&‚{\tiny $_{lb}$}‚नून‚म्ब‚न्ध्यासुतेप्य‚स्ति त‚स्य स्म‚र‚ण‚श‚क्य‚ता ॥\edtext{}{\edlabel{pvsvt_168-3}\label{pvsvt_168-3}\lemma{ता}\Bfootnote{\href{http://sarit.indology.info/?cref=\%C5\%9Bv-anum\%C4\%81na.160}{ Ślokavārtika,  अनुमान‚प‚रि० १६०.}}}{\normalfontlatin\large\qquad{}"}\&[\smallbreak]
	  
	  
	  
	  \endgroup
	‚{\tiny $_{lb}$}‚

	  
	  \pstart \leavevmode% starting standard par
	इति निर‚स्तं । भिन्नानामेवानुभूतानामेक‚कार्य‚क‚र्त्तृत्वेन स्व‚विष‚याभिन्नाकार‚{\tiny $_{lb}$}‚\leavevmode\ledsidenote{\textenglish{63b/PSVTa}} प्र‚त्य‚य‚ज‚न‚ने साम‚र्थ्या‚{\tiny $_{७}$}‚द‚नुभूत‚स्प‚ष्टाकार‚स्यार्थ‚स्य स्व‚विष‚योस्प‚ष्ट‚स्म‚र‚ण‚ज‚न‚न‚साम‚{\tiny $_{lb}$}‚र्थ्य‚व‚त् । \textbf{त‚दुद्भ}वेति व‚र्ण्ण‚विक‚ल्प‚वास‚नाया विवेकिभ्यः स्व‚भावेभ्यो य‚थासंख्यं‚{\tiny $_{lb}$}‚ साक्षात् पार‚म्प‚र्येण चोद्भ‚वो य‚स्याः सा त‚था । \textbf{सा चेय}मिति बुद्धिः \textbf{संवृति}रित्युच्य‚ते ।‚{\tiny $_{lb}$}‚ स‚म्व्रिय‚तेऽन‚या बुद्ध्या \textbf{स्व‚रूपेण} स्व‚प्र‚तिभासेन \textbf{प‚र‚रू}प‚म्व‚स्तुरूप‚मिति क‚र‚ण‚{\tiny $_{१}$}‚साध‚नं‚{\tiny $_{lb}$}‚ क्तिनं विधाय । \textbf{ते च} भावा\textbf{स्त‚ये}ति बुद्ध्या । \textbf{संवृत‚भेदाः} प्र‚च्छादित‚नानात्वाः ।‚{\tiny $_{lb}$}‚ \textbf{केन‚चिद् रूपेणे}ति विक‚ल्प‚बुद्ध्यारोपितेन । \textbf{प्र‚तिभान्ति} विक‚ल्प‚बुद्धौ ।
	{\color{gray}{\rmlatinfont\textsuperscript{§~\theparCount}}}
	\pend% ending standard par
      ‚{\tiny $_{lb}$}‚‚{\tiny $_{lb}$}‚‚{\tiny $_{lb}$}‚‚{\tiny $_{lb}$}‚\textsuperscript{\textenglish{169/s}}

	  
	  \pstart \leavevmode% starting standard par
	\textbf{त‚देषा}मित्यादि । \textbf{बुद्धिप्र‚तिभासं} विक‚ल्प‚बुद्ध्याकारं । \textbf{अनुरुन्धानैः} पुरुषैस्त‚द्‚{\tiny $_{lb}$}‚\textbf{बुद्ध्युप‚स्थापित}मेकं \textbf{रूपं सामान्य‚मुच्य‚त} इत्य‚नेन स‚म्ब‚न्धः । केषां सामान्य‚मुच्य‚ते ।‚{\tiny $_{lb}$}‚ \textbf{बुद्धिप‚रिव‚{\tiny $_{२}$}‚र्त्तिनामेव । एषां} बुद्धिप‚रिव‚र्त्तिनामिति स‚म्ब‚न्धः । स्व‚ल‚क्ष‚णान्य‚नुभूय‚{\tiny $_{lb}$}‚ य‚थानुभ‚व‚म‚सौ शुक्लो घ‚टः कृष्णोन्यो वेत्येवं विक‚ल्प‚बुद्धिष्व‚स्प‚ष्टाः घ‚टाकारास्ते‚{\tiny $_{lb}$}‚ बुद्धिप‚रिव‚र्त्तिनो भावा ये \textbf{तेषा}मेव सामान्यं स‚म्ब‚न्धि । न स्व‚ल‚क्ष‚णानां सामान्य‚बुद्धाव‚{\tiny $_{lb}$}‚प्र‚तिभास‚नाद‚स‚म्ब‚न्धित्वाच्च । बुद्धिप‚रिव‚र्त्तिनामेव विशेष‚णं \textbf{ब‚हिरिव प‚रिस्फुर‚ता‚{\tiny $_{lb}$}‚मि}‚{\tiny $_{३}$}‚ति ब‚हिरिव प्र‚तिभास‚मानानामित्य‚र्थः । ब‚हिःस्फुर‚णे च कार‚ण‚माह । \textbf{आकार‚{\tiny $_{lb}$}‚विशेष‚प‚रिग्र‚हा}दिति [।]
	{\color{gray}{\rmlatinfont\textsuperscript{§~\theparCount}}}
	\pend% ending standard par
      ‚{\tiny $_{lb}$}‚

	  
	  \pstart \leavevmode% starting standard par
	बाह्यार्थ‚विक‚ल्प‚संस्थान‚स्य स्प‚ष्ट‚स्यानुकाराद् बाह्याध्य‚व‚साय इत्येके ।
	{\color{gray}{\rmlatinfont\textsuperscript{§~\theparCount}}}
	\pend% ending standard par
      ‚{\tiny $_{lb}$}‚

	  
	  \pstart \leavevmode% starting standard par
	एत‚च्चायुक्तं । न हि सादृश्य‚निमित्तो बाह्य‚त्वारोप इति निवेदित‚मेत‚त् ।‚{\tiny $_{lb}$}‚ त‚स्मादाकार‚विशेषो घ‚टाद्याकार‚स्त‚स्य प‚रिग्र‚होनुभ‚व‚स्त‚स्माद् ब‚हिरिव‚{\tiny $_{lb}$}‚ प‚रिस्फु‚{\tiny $_{४}$}‚र‚णं ।
	{\color{gray}{\rmlatinfont\textsuperscript{§~\theparCount}}}
	\pend% ending standard par
      ‚{\tiny $_{lb}$}‚

	  
	  \pstart \leavevmode% starting standard par
	एत‚दुक्त‚म्भ‚व‚ति । घ‚टाद्य‚नुभ‚वाहित‚वास‚नासाम‚र्थ्येन विक‚ल्प उत्प‚द्य‚मानः स्वाकारं‚{\tiny $_{lb}$}‚ बाह्य‚घ‚टाद्य‚भेदेनाध्य‚व‚स्य‚ति न तु गृह्णाति [।] तेन श‚ब्द‚विहितेर्थे क्व‚चित् संश‚यो‚{\tiny $_{lb}$}‚ भ‚व‚त्य‚ग्र‚ह‚णात् । त‚त्र तु विक‚ल्पः स्व‚हेतुत एव बाह्याभिन्नं स्वाकार‚म ध्य व‚स्य‚ति ।‚{\tiny $_{lb}$}‚ न तु सादृश्यात् स‚दृश‚स्यार्थ‚स्याभावाद‚ग्र‚ह‚णाच्च ।
	{\color{gray}{\rmlatinfont\textsuperscript{§~\theparCount}}}
	\pend% ending standard par
      ‚{\tiny $_{lb}$}‚

	  
	  \pstart \leavevmode% starting standard par
	न‚नु बुद्धिप‚रिव‚र्त्तिनाम‚पि स्व‚रूपे‚{\tiny $_{५}$}‚ व्य‚तिरेकेण कोन्य एव आकारः प्र‚तिभास‚ते‚{\tiny $_{lb}$}‚ य‚त्सामान्यं स्यात् ।
	{\color{gray}{\rmlatinfont\textsuperscript{§~\theparCount}}}
	\pend% ending standard par
      ‚{\tiny $_{lb}$}‚

	  
	  \pstart \leavevmode% starting standard par
	स‚त्यं । त‚त्र ताव‚त् केचिदाहुरेक‚ज्ञानाव्य‚तिरेकादेक‚त्व‚न्तेषामिति । अन्ये‚{\tiny $_{lb}$}‚ त्वाहुः प्र‚त्येक‚न्तेषां स्व‚व्य‚क्त्य‚पेक्ष‚या सामान्य‚रूप‚तेति [।] त‚देत‚दुभ‚य‚म‚प्य‚युक्त‚{\tiny $_{lb}$}‚मेक‚प्र‚तिभासाभावात् । केव‚ल‚मेक‚रूप‚त‚या तेषाम‚ध्य‚व‚सायात् प‚रैः सामान्य‚मिष्ट‚{\tiny $_{lb}$}‚मिति त‚द‚भिप्रायादेव‚{\tiny $_{६}$}‚मुच्य‚ते । आ चा र्य स्य तु स‚माना इति प्र‚तिभासोभिप्रेतो‚{\tiny $_{lb}$}‚ नैक इति त‚था च व‚क्ष्य‚ति ।
	{\color{gray}{\rmlatinfont\textsuperscript{§~\theparCount}}}
	\pend% ending standard par
      ‚{\tiny $_{lb}$}‚

	  
	  \pstart \leavevmode% starting standard par
	अथ‚वा[।] अस्तु प्र‚तिभासो धियाम्भिन्नः स‚माना इति त‚द्ग्र‚ह‚णादि‚{\tiny $_{lb}$}‚\href{http://sarit.indology.info/?cref=pv.3.106}{१ । १०९} ति । य‚दि बुद्धिप‚रिव‚र्त्तिनामेव विक‚ल्प‚बुद्धिस‚न्द‚र्शितं रूपं सामान्य‚न्त‚स्य‚{\tiny $_{lb}$}‚ च विधिरूप‚त्वात् \textbf{क‚थ‚मिदानीम‚न्यापोहः सामान्य}मित्युच्य‚ते ।
	{\color{gray}{\rmlatinfont\textsuperscript{§~\theparCount}}}
	\pend% ending standard par
      ‚{\tiny $_{lb}$}‚‚{\tiny $_{lb}$}‚\textsuperscript{\textenglish{170/s}}

	  
	  \pstart \leavevmode% starting standard par
	\leavevmode\ledsidenote{\textenglish{64a/PSVTa}} \textbf{स एवे}ति सिद्धान्त‚वादी । स एव विक‚ल्प‚बुद्धि‚{\tiny $_{७}$}‚व्य‚व‚स्थापितः प्र‚तिभास‚मा‚{\tiny $_{lb}$}‚नो\textbf{ऽन्यापोह} उच्य‚ते । अन्य‚विविक्त‚प‚दार्थ‚द‚र्श‚न‚द्वारायात‚त्वात् ।
	{\color{gray}{\rmlatinfont\textsuperscript{§~\theparCount}}}
	\pend% ending standard par
      ‚{\tiny $_{lb}$}‚

	  
	  \pstart \leavevmode% starting standard par
	य‚दि विक‚ल्पाकारः सामान्यं क‚थं बाह्यानां स‚मान‚रूप‚त‚या प्र‚तीतिः । य‚दाह [।]
	{\color{gray}{\rmlatinfont\textsuperscript{§~\theparCount}}}
	\pend% ending standard par
      ‚{\tiny $_{lb}$}‚
	  \bigskip
	  \begingroup
	
	    
	    \stanza[\smallbreak]
	  {\normalfontlatin\large ``\qquad}अथ निर्विष‚या एता वास‚नामात्र‚भाव‚तः ।&‚{\tiny $_{lb}$}‚प्र‚तिप‚त्तिः प्र‚वृत्तिश्च बाह्यार्थेषु क‚थ‚म्भ‚वेद् [।]{\normalfontlatin\large\qquad{}"}\&[\smallbreak]
	  
	  
	  
	  \endgroup
	‚{\tiny $_{lb}$}‚

	  
	  \pstart \leavevmode% starting standard par
	इत्य‚त आह । \textbf{त‚मे}वेत्यादि । त‚मिति विक‚ल्प‚बुद्धिप्र‚तिभासं \textbf{गृह्ण‚ती} सा विक‚{\tiny $_{lb}$}‚ल्पिका बु‚{\tiny $_{१}$}‚द्धि\textbf{र्व‚स्तुग्राहिणीव प्र‚तिभाति} । क‚स्माद् \textbf{विक‚ल्पानां प्र‚कृतिविभ्र‚मात्}‚{\tiny $_{lb}$}‚ स्व‚भावेनैव स्वाकाराभेदेनार्थ‚ग्र‚ह‚ण‚विभ्र‚मात् ।
	{\color{gray}{\rmlatinfont\textsuperscript{§~\theparCount}}}
	\pend% ending standard par
      ‚{\tiny $_{lb}$}‚

	  
	  \pstart \leavevmode% starting standard par
	क‚थ‚न्त‚र्ह्य‚पोह‚विष‚येत्युच्य‚त इत्याह । \textbf{सा हि} विक‚ल्पिका बुद्धिर‚ध्य‚व‚साय‚{\tiny $_{lb}$}‚व‚शात् \textbf{त‚द‚न्य‚विवेकिषु भावेषु} स्व‚ल‚क्ष‚णेषु \textbf{भ‚व‚न्ती विवेक‚विष‚येति ग‚म्य}ते । कार्य‚तो‚{\tiny $_{lb}$}‚ न तु विवेक‚स्व‚भाव‚विष‚यीक‚र‚णात् ।
	{\color{gray}{\rmlatinfont\textsuperscript{§~\theparCount}}}
	\pend% ending standard par
      ‚{\tiny $_{lb}$}‚

	  
	  \pstart \leavevmode% starting standard par
	प‚र‚स्त्व‚विदिताभि‚{\tiny $_{२}$}‚प्रायः प्राह । \textbf{न‚न्}वित्यादि । सामान्याद् बा\textbf{ह्याः} स्व‚ल‚क्ष‚णा‚{\tiny $_{lb}$}‚नीत्य‚र्थः । एवं हि बाह्याध्यात्मिकानां संग्र‚हः कृतो भ‚व‚ति । \textbf{विवेकिनः} प‚र‚स्प‚र‚{\tiny $_{lb}$}‚विल‚क्ष‚णा [।] \textbf{न च तेष्वि}ति स्व‚ल‚क्षेणेषु [।] \textbf{क‚थ‚न्तेषु विक‚ल्प‚बुद्धिर्भ‚व}तीत्युच्य‚ते ।‚{\tiny $_{lb}$}‚ \textbf{व्याख्यातार} इत्यादिना प‚रिह‚र‚ति । ते हि य‚थाव‚स्थित‚म्व‚स्तु व्य‚व‚स्थाप‚य‚न्त एव\textbf{म्वि‚{\tiny $_{lb}$}‚वेच‚य‚न्ति} । अन्यो विक‚ल्प‚बुद्धिप्र‚तिभासो न्य‚त्स्व‚{\tiny $_{३}$}‚ल‚क्ष‚ण‚मिति । \textbf{न व्य‚व‚ह‚र्त्तार}‚{\tiny $_{lb}$}‚ एवं विवेच‚य‚न्ति । \textbf{ते तु} व्य‚व‚ह‚र्त्तारः \textbf{स्वाल‚म्ब‚न‚मेवे}ति विक‚ल्प‚प्र‚तिभास‚मे\textbf{वार्थ‚{\tiny $_{lb}$}‚क्रियायोग्यं} बाह्य‚स्व‚ल‚क्ष‚ण‚रूप\textbf{म्म‚न्य‚मानाः} । एत‚देव स्प‚ष्ट‚य‚ति [।] दृश्योर्थः‚{\tiny $_{lb}$}‚ स्व‚ल‚क्ष‚ण\textbf{म्विक‚ल्प्योर्थः} सामान्य‚प्र‚तिभास‚स्ता\textbf{वेकीकृत्य} स्व‚ल‚क्ष‚ण‚मेवेद‚म्विक‚ल्प‚{\tiny $_{lb}$}‚बुद्ध्या विष‚यीक्रिय‚ते श‚ब्देन चोद्य‚त इत्येव‚म‚धिमुच्यार्थ‚क्रि‚{\tiny $_{४}$}‚याकारिण्य‚र्थे \textbf{प्र‚व‚र्त्त‚न्ते ।‚{\tiny $_{lb}$}‚ त‚द‚भिप्राय‚व‚शा}द् व्य‚व‚ह‚र्त्तृणाम‚भिप्राय‚व‚शा\textbf{देव‚मुच्य‚ते} विवेकिषु भावेषु विक‚ल्प‚{\tiny $_{lb}$}‚बुद्धिर्भ‚व‚तीति दृश्य‚विक‚ल्प्यावेकीकृत्य प्र‚वृत्तेरिति व‚द‚ता [।] न स्वाकारे बाह्या‚{\tiny $_{lb}$}‚रोप इत्युक्त‚म्भ‚व‚त्य‚न्य‚था स्वाकार एव प्र‚वृत्तिप्र‚स‚ङ्गात् । म‚रीचिकायां ज‚लारो‚{\tiny $_{lb}$}‚पादिव । नापि बाह्ये स्वाकारारोपः । आरोप्य‚माण‚फ‚लार्थित्वेनैव प्र‚{\tiny $_{५}$}‚वृत्तिप्र‚संगात् ।‚{\tiny $_{lb}$}‚ ज‚लार्थिन इव ज‚ल‚भ्रान्तौ ।
	{\color{gray}{\rmlatinfont\textsuperscript{§~\theparCount}}}
	\pend% ending standard par
      ‚{\tiny $_{lb}$}‚‚{\tiny $_{lb}$}‚\textsuperscript{\textenglish{171/s}}

	  
	  \pstart \leavevmode% starting standard par
	न‚नु दृश्य‚विल्प‚योरेकीक‚र‚णं किमुच्य‚ते । य‚दि दृश्य‚स्य विक‚ल्प्याद‚भेदः बाह्येर्थे‚{\tiny $_{lb}$}‚ प्र‚वृत्तिर्न स्यात् । विक‚ल्प्य‚स्य दृश्याद‚भेदः स्व‚ल‚क्ष‚णं श‚ब्दार्थः स्यात् । न च‚{\tiny $_{lb}$}‚ दृश्य‚विक‚ल्प्य‚योरेकीक‚र‚णं प्र‚त्य‚क्षेण त‚स्य विक‚ल्प्याविष‚य‚त्वात् [।] नापि विक‚ल्पेन‚{\tiny $_{lb}$}‚ त‚स्य दृश्याविष‚य‚त्वात् । अतीतादौ च दृश्याभा‚{\tiny $_{६}$}‚वात् क‚थ‚न्त‚योरेकीक‚र‚णं ।
	{\color{gray}{\rmlatinfont\textsuperscript{§~\theparCount}}}
	\pend% ending standard par
      ‚{\tiny $_{lb}$}‚

	  
	  \pstart \leavevmode% starting standard par
	अत्रोच्य‚ते । अर्थानुभ‚वे स‚ति त‚त्संस्कार‚प्र‚बोधेन त‚दाकार उत्प‚द्य‚मानो विक‚ल्पः‚{\tiny $_{lb}$}‚ स्वाकार‚म्बाह्याभिन्न‚म‚ध्य‚व‚स्य‚ति न त्व‚भिन्नं क‚रोति । तेन विक‚ल्प‚विष‚य‚स्य दृश्या‚{\tiny $_{lb}$}‚त्म‚त‚याध्य‚व‚सायाद् दृश्य‚विक‚ल्प‚योरेकीक‚र‚ण‚मुच्य‚ते [।] द‚र्श‚नार्हो दृश्यः श‚क्य‚ते‚{\tiny $_{lb}$}‚ \textbf{वा} द्र‚ष्टुमिति विक‚ल्प‚क‚द‚र्श‚न‚स्यापि‚{\tiny $_{७}$}‚ यो विष‚यः स दृश्य‚स्तेनानाग‚त‚स्याप्य‚र्थ‚स्या- \leavevmode\ledsidenote{\textenglish{64b/PSVTa}}‚{\tiny $_{lb}$}‚ ग‚मोप‚द‚र्शित‚स्य दृश्य‚त्वं सिद्ध‚न्त‚थाऽभाव‚स्यापि । अत एवाभावः प्र‚त्य‚यः स्वाकार‚{\tiny $_{lb}$}‚म्भाव‚रूप‚म‚पि प‚दार्थाभावाव्य‚तिरेकेणाध्य‚स्य प्र‚त्येतीत्य‚भाव‚विष‚य उच्य‚ते न त्व‚भावं‚{\tiny $_{lb}$}‚ गृह्णाति [।] स‚र्व‚विक‚ल्पानां निर्विष‚य‚त्वात् ।
	{\color{gray}{\rmlatinfont\textsuperscript{§~\theparCount}}}
	\pend% ending standard par
      ‚{\tiny $_{lb}$}‚

	  
	  \pstart \leavevmode% starting standard par
	\textbf{त‚त्कारित‚येत्यादि} । न केव‚ल‚न्त‚द‚भिप्राय‚व‚शाद् विक‚ल्प‚बुद्धिः स्व‚ल‚क्ष‚णेषु‚{\tiny $_{lb}$}‚ विवेकेषु भ‚व‚तीत्युच्य‚ते । \textbf{त‚था त‚त्कारित‚या} क‚र‚णेना\textbf{त‚त्कारिभ्यो भिन्नान}र्थान्‚{\tiny $_{lb}$}‚ \textbf{श‚ब्देन} व‚क्तारः \textbf{प्र‚तिपाद‚य‚न्तीत्युच्य‚ते} । व्याख्यातारोपि दृश्य‚विक‚ल्प‚योरैक्यं किमिति‚{\tiny $_{lb}$}‚ न प्र‚तिप‚द्य‚न्त इति चेदाह । \textbf{प्र‚तिभास‚भेदादिभ्य} इति दृश्य‚स्य हि स्प‚ष्टः प्र‚तिभासो‚{\tiny $_{lb}$}‚ न विक‚ल्प‚स्य । विक‚ल्पानुब‚द्ध‚स्य स्प‚ष्ट‚त्वायोगात् । आदिश‚ब्दान्निरुद्धेपि दृश्ये‚{\tiny $_{lb}$}‚ विक‚ल्प‚स्यानिरोधात् ।‚{\tiny $_{२}$}‚ अर्थ‚क्रियायाः क‚र‚णाद‚क‚र‚णाच्च । \textbf{त‚त्त्व‚चिन्त‚का} न्याया‚{\tiny $_{lb}$}‚नुसारिणः व्याख्यातारः [।] शेष‚ष‚ष्ठी चेय‚न्त‚त्त्व‚स्य चिन्त‚का इति । त‚त्त्व‚स्य ते‚{\tiny $_{lb}$}‚ चिन्त‚का नात‚त्त्व‚स्येति स‚म्ब‚न्ध्य‚न्त‚र‚व्य‚व‚च्छेद‚मात्रापेक्षायां क्रियाकार‚क‚भाव‚स्या‚{\tiny $_{lb}$}‚विव‚क्षित‚त्वात् । [।] चिन्त‚य‚तेर्वा प‚चाद्य‚ज‚न्त‚स्य त‚त्त्व‚श‚ब्देन ष‚ष्ठीस‚मासं कृत्वा‚{\tiny $_{lb}$}‚ स्वार्थिकः क‚न्प्र‚त्य‚यः कृतः । त‚तोत्र \textbf{तृज‚काभ्या}\edtext{}{\edlabel{pvsvt_171-1}\label{pvsvt_171-1}\lemma{तोत्र}\Bfootnote{\href{http://sarit.indology.info/?cref=P\%C4\%81.2.2.15}{ Pāṇini. 2: 2: 15. }}}मित्यादिना ष‚ष्ठी‚{\tiny $_{३}$}‚स‚मास‚प्र‚तिषेधो‚{\tiny $_{lb}$}‚ नाशंक्यः । \textbf{नाभेद‚म‚नुस‚न्ध‚त्ते} । दृश्य‚विक‚ल्प‚योरिति प्र‚कृते ।
	{\color{gray}{\rmlatinfont\textsuperscript{§~\theparCount}}}
	\pend% ending standard par
      ‚{\tiny $_{lb}$}‚

	  
	  \pstart \leavevmode% starting standard par
	\textbf{य‚दीत्या}दि चोद‚कः । \textbf{प्र‚तिप‚त्त्र‚भिप्रायोनुविधीय‚ते} बाह्येषु सामान्य‚व्य‚व‚स्थानं ।‚{\tiny $_{lb}$}‚ \textbf{त‚दाऽन्यापोहोपि सामान्यं मा भूत} । किं कार‚णं [।] \textbf{न ह्येव}मिति । अन्यापोहः‚{\tiny $_{lb}$}‚ श‚ब्देन चोद्य‚त इत्येवं व्य‚व‚ह‚र्त्तॄणां नास्ति \textbf{प्र‚तिप‚त्तिः} । आ चा र्यो पि तुल्य‚तां‚{\tiny $_{lb}$}‚ ‚{\tiny $_{lb}$}‚ ‚{\tiny $_{lb}$}‚ \leavevmode\ledsidenote{\textenglish{172/s}}ख्याप‚य‚न्नाह । \textbf{न वै केव‚{\tiny $_{४}$}‚ल‚मि}त्यादि । \textbf{एव‚म‚प्र‚तिप‚त्ति}रित्य‚न्यापोह‚रूपेण ।‚{\tiny $_{lb}$}‚ त्व‚या य‚था \textbf{व्य‚क्तिव्य‚तिरिक्तादिभिराकारै}रिष्टं सामान्य‚न्त‚था\textbf{पि नैव प्र‚तिप‚त्तिः} ।‚{\tiny $_{lb}$}‚ आश्र‚याद् व्य‚तिरिक्तं वै शे षि का दीनाम‚व्य‚तिरिक्तं सां ख्या दी ना मादिश‚ब्दात्‚{\tiny $_{lb}$}‚ प्र‚त्येक‚प‚रिस‚माप्त‚त्वादिप‚रिग्र‚हः । त‚त्र भेद‚स्ताव‚त्सामान्य‚स्य न प्र‚तिभास‚त एव ।‚{\tiny $_{lb}$}‚ इह सामान्य‚मिति बुद्ध्य‚नुत्प‚त्तेः‚{\tiny $_{५}$}‚ अभेदेपि व्य‚क्तीनामेव प्र‚तिभासः स्यान्न‚{\tiny $_{lb}$}‚ सामान्य‚स्य । भेदाभेदे चोभ‚य‚प‚क्ष‚भावी दोषः स्यात् । नापि त‚त्प्र‚त्येक‚प‚रिस‚माप्तं‚{\tiny $_{lb}$}‚ युज्य‚ते । य‚त एक‚व्य‚क्तावेक‚स‚म्ब‚न्ध्येव त‚त् स्यान्नानेक‚स‚म्ब‚न्धि । न चैवं सामान्य‚{\tiny $_{lb}$}‚म‚नेक‚स‚म्ब‚न्धित्वाद‚स्य स‚मानानाम्भावः सामान्य‚मिति व‚च‚नात् ।
	{\color{gray}{\rmlatinfont\textsuperscript{§~\theparCount}}}
	\pend% ending standard par
      ‚{\tiny $_{lb}$}‚

	  
	  \pstart \leavevmode% starting standard par
	अथ व्य‚क्त्य‚न्त‚रेषु त‚स्य प्र‚त्य‚भिज्ञानात् त‚त्स‚म्ब‚न्धित्व‚मेव‚म‚पि भू‚{\tiny $_{६}$}‚त‚गुण‚{\tiny $_{lb}$}‚व‚देक‚म‚नेक‚स‚म्ब‚न्धि स्यान्न प्र‚त्येक‚प‚रिस‚माप्तं । एकैक‚स्यां व्य‚क्ताव‚नेक‚स‚म्ब‚न्धि‚{\tiny $_{lb}$}‚त्वेनाप्र‚तीतेः । प्र‚तीतौ वा त‚त्रानेक‚व्य‚क्तिप्र‚तिभासः स्यात् । अथ व्य‚क्त्य‚न्त‚रा‚{\tiny $_{lb}$}‚लेऽप्र‚तिभास‚नात् प्र‚त्येक‚प‚रिस‚माप्त‚न्त‚दुच्य‚ते । त‚त्किमेताव‚ता त‚स्यानेक‚स‚म्ब‚न्धित्व‚{\tiny $_{lb}$}‚मेक‚त्र सिध्य‚ति । न च सामान्ये प्र‚त्य‚भिज्ञानं युज्य‚त इत्य‚प्युक्तं । त‚स्मान्न त‚स्य‚{\tiny $_{lb}$}‚ \leavevmode\ledsidenote{\textenglish{65a/PSVTa}} प्र‚{\tiny $_{७}$}‚त्येक‚प‚रिस‚माप्तिर्नापि नित्य‚त्व‚मेक‚त्वं व्यापित्वं च प्र‚तिभास‚त इति स्थितं ।
	{\color{gray}{\rmlatinfont\textsuperscript{§~\theparCount}}}
	\pend% ending standard par
      ‚{\tiny $_{lb}$}‚

	  
	  \pstart \leavevmode% starting standard par
	य‚दि व‚स्तुभूत‚स्य सामान्य‚स्यापोह‚स्य च तुल्यः प्र‚तिप‚त्त्य‚भावो व्य‚व‚हारे । क‚स्त‚{\tiny $_{lb}$}‚र्ह्य‚स्या \textbf{आश्र‚य} इत्य‚त आह । \textbf{केव‚ल}मित्यादि । स्व‚ल‚क्ष‚णानां विजातीय‚र‚हित‚त्व‚{\tiny $_{lb}$}‚\textbf{म‚न्यापोहः} सोऽ\textbf{भिन्नाकाराया} बुद्धेर्निमित्त‚त्वे\textbf{नोच्य‚ते} । किं कार‚णं [।] \textbf{त‚स्य} विजाती‚{\tiny $_{lb}$}‚य‚विर‚ह‚ल‚क्ष‚ण‚स्या‚{\tiny $_{१}$}‚न्यापोह‚स्य भिन्नेष्व‚पि स‚र्व‚त्र \textbf{व‚स्तुषु भावात्} त‚थाभूत‚स्य चान्या‚{\tiny $_{lb}$}‚पोह‚स्य सामान्य‚बुद्धिहेतुत्व‚म्प्र‚त्य\textbf{विरोधात्} । त‚था हि य‚थैक‚म्वृक्ष‚म‚वृक्षाद् व्यावृत्तं‚{\tiny $_{lb}$}‚ प‚श्य‚त्येव‚म‚न्य‚म‚प्य‚त‚स्त‚त्रैकाकारा बुद्धिरुत्प‚द्य‚ते । न चात्र बाध‚कं प्र‚माण‚म‚स्ति ।
	{\color{gray}{\rmlatinfont\textsuperscript{§~\theparCount}}}
	\pend% ending standard par
      ‚{\tiny $_{lb}$}‚

	  
	  \pstart \leavevmode% starting standard par
	तृतीय‚कार‚ण‚माह । \textbf{व्य‚व‚हार‚स्य चे}त्यादि । \textbf{त‚था द‚र्श‚ना}दिति । अन्यापोह‚{\tiny $_{lb}$}‚निब‚न्ध‚न‚त्वेन द‚र्श‚नात् । एत‚च्च त‚द‚{\tiny $_{२}$}‚न्य‚प‚रिहारेण प्र‚व‚र्त्त‚ते\edtext{}{\lemma{ते}\Bfootnote{? त इ}} ति च ध्व‚निरु‚{\tiny $_{lb}$}‚च्य‚त इत्यादिना प्र‚तिपाद‚यिष्य‚ते । \textbf{य‚थे}त्येकाकारा \textbf{इयं} विक‚ल्प\textbf{बुद्धिः प्र‚तिभाति} [।]‚{\tiny $_{lb}$}‚  ‚{\tiny $_{lb}$}‚ ‚{\tiny $_{lb}$}‚ \leavevmode\ledsidenote{\textenglish{173/s}}त‚था बाह्य‚न्त‚त्सामान्य‚न्नास्तीति वाक्यार्थः । त\textbf{तो न व‚स्तुभूतं सामान्यं} विक‚ल्प‚{\tiny $_{lb}$}‚द‚र्श‚नाश्र‚य इत्य‚भिप्रायः ॥
	{\color{gray}{\rmlatinfont\textsuperscript{§~\theparCount}}}
	\pend% ending standard par
      ‚{\tiny $_{lb}$}‚

	  
	  \pstart \leavevmode% starting standard par
	सामान्याभावे च कार‚ण‚माह । \textbf{य‚स्मा}दित्यादि । \textbf{व्य‚क्त‚यः} स्व‚ल‚क्ष‚णानि ।‚{\tiny $_{lb}$}‚ \textbf{नानुय‚न्ति} न मिश्रीभ‚व‚न्ति । एतेनाव्य‚{\tiny $_{३}$}‚तिरिक्त‚सामान्याभ्युप‚ग‚मो निर‚स्तः ।
	{\color{gray}{\rmlatinfont\textsuperscript{§~\theparCount}}}
	\pend% ending standard par
      ‚{\tiny $_{lb}$}‚

	  
	  \pstart \leavevmode% starting standard par
	व्य‚तिरिक्त‚निराक‚र‚णार्थ‚माह । \textbf{अन्य‚द‚नुयायी}ति । अनुयायि य‚दिष्टं भेदे‚{\tiny $_{lb}$}‚भ्योन्य‚त् सामान्य‚रूप‚न्त‚त्प्र‚त्य‚क्ष‚बुद्धौ \textbf{न भास‚ते} ।
	{\color{gray}{\rmlatinfont\textsuperscript{§~\theparCount}}}
	\pend% ending standard par
      ‚{\tiny $_{lb}$}‚

	  
	  \pstart \leavevmode% starting standard par
	न‚नु विक‚ल्प‚प्र‚तिभासेपि सामान्ये भेदाभेद‚प‚क्ष‚योर‚य‚म‚न‚न्व‚यादिदोष‚स्तुल्य‚{\tiny $_{lb}$}‚ एवेति क‚थं सामान्य‚मिष्य‚ते । स‚मानाकार‚स्ताव‚त् प्र‚तिभास‚त एव । याव‚द‚सौ‚{\tiny $_{lb}$}‚ न बुध्य‚ते भेदाभेद‚र‚हित इति ताव‚द‚स्याल‚{\tiny $_{४}$}‚ङ्कार एवाव‚स्तुत्व‚प्र‚तिपाद‚नाद् [।]‚{\tiny $_{lb}$}‚ अत एव य‚थाप्र‚तिभासं त्व‚साव‚स्तीत्युच्य‚ते ।
	{\color{gray}{\rmlatinfont\textsuperscript{§~\theparCount}}}
	\pend% ending standard par
      ‚{\tiny $_{lb}$}‚

	  
	  \pstart \leavevmode% starting standard par
	\textbf{अन्वाविश‚न्ति} मिश्रीभ‚व‚न्ति । य‚दि \textbf{प‚र‚स्प‚र}म‚न्वावेशः स्यात्त‚दैक‚रूपाप‚त्ते\textbf{र्भेदा‚{\tiny $_{lb}$}‚भावः} । तेन कार‚णेन \textbf{सामान्य‚स्यैवाभाव‚प्र‚स‚ङ्गात्} । तेन भेदानां स‚मानानामेका‚{\tiny $_{lb}$}‚कार‚प्र‚त्य‚य‚निब‚न्ध‚न‚त्वं ध‚र्मः सामान्यं । त‚द्भेदाभावेन भ‚वेदित्य‚र्थः । \textbf{अन्य‚च्चे}ति‚{\tiny $_{lb}$}‚ व्य‚क्तिभ्यः अन्य‚{\tiny $_{५}$}‚दित्य‚स्य विव‚र‚णं \textbf{व्य‚तिरिक्त}मिति । \textbf{त‚था} तेन रूपेण सामान्य\textbf{बुद्धौ}‚{\tiny $_{lb}$}‚[।] अनेनोप‚ल‚ब्धिल‚क्ष‚ण‚प्राप्त‚स्य सामान्य‚स्याभाव‚व्य‚व‚हारे साध्ये स्व‚भावानुप‚ल‚म्भ‚{\tiny $_{lb}$}‚ उक्तः । न चानुप‚ल‚ब्धिल‚क्ष‚ण‚प्राप्तं सामान्य‚म‚भ्युपेयं । य‚स्माद\textbf{प्र‚तिभास}मानं \textbf{च}‚{\tiny $_{lb}$}‚ सामान्यं \textbf{क‚थ‚मात्म‚ना} स्वेन सामान्य‚रूपेण । \textbf{अन्य}मिति सामान्य‚व‚न्तं \textbf{ग्राह‚ये}त् ।‚{\tiny $_{lb}$}‚ \textbf{व्य‚प‚देश‚येद्वा} । स्वेन‚{\tiny $_{६}$}‚ रूपेण । क‚थ‚म‚न्य‚मिति प्र‚कृतेन स‚म्ब‚न्धः । सामान्य‚ब‚लाद्‚{\tiny $_{lb}$}‚ व्य‚क्तिष्व‚भिन्नाभिधान‚ज्ञान‚वृत्तिरिष्टा । त‚स्मात् सामान्य‚मुप‚ल‚ब्धिल‚क्ष‚ण‚प्राप्त‚{\tiny $_{lb}$}‚मेवेष्ट‚मिति स‚मुदायार्थः । एव‚न्ताव‚द् ग्राह्य‚ल‚क्ष‚ण‚प्राप्त‚स्यानुप‚ल‚म्भान्नास्त्येक‚म‚ने‚{\tiny $_{lb}$}‚क‚स‚म्ब‚न्धि [।]
	{\color{gray}{\rmlatinfont\textsuperscript{§~\theparCount}}}
	\pend% ending standard par
      ‚{\tiny $_{lb}$}‚

	  
	  \pstart \leavevmode% starting standard par
	भ‚व‚तु नामैक‚म‚नेक‚स‚म्ब‚द्ध‚न्त‚थापि सामान्य‚रूप‚ता न युक्तेत्याह । \textbf{न चे}त्यादि ।‚{\tiny $_{lb}$}‚ \textbf{तैरि}ति व्य‚क्तिभेदैः \textbf{उक्त}‚{\tiny $_{७}$}‚मिति द्वित्वादि । संयोग‚कार्य‚द्र‚व्येष्व‚पि सामान्य‚स्व‚भाव‚त्वं \leavevmode\ledsidenote{\textenglish{65b/PSVTa}}‚{\tiny $_{lb}$}‚ प्राप्नोतीत्य‚प्र‚संग‚स्योक्त‚त्वात् । एक‚म‚नेक‚स‚म्ब‚द्ध‚मित्येव कृत्वा न सामान्यं किन्त्व‚{\tiny $_{lb}$}‚‚{\tiny $_{lb}$}‚ \leavevmode\ledsidenote{\textenglish{174/s}}\textbf{भिन्नाभिधान‚प्र‚त्य‚य‚निमित्त‚मेक‚सामान्यं न स‚र्व} द्वित्वाद्य‚पि । त‚स्य य‚थोक्त‚श‚ब्द‚{\tiny $_{lb}$}‚ज्ञानानिमित्त‚त्वादिति चेत् । \textbf{क‚थ‚म‚न्य‚तः} सामान्या\textbf{द‚न्य}त्र व्य‚क्तिभेदे सामान्येनैक‚रूप‚{\tiny $_{lb}$}‚ताम‚नापादिते । \textbf{प्र‚त्य‚य‚वृ‚{\tiny $_{१}$}‚त्तिरे}काकार‚ज्ञान‚वृत्तिः ।
	{\color{gray}{\rmlatinfont\textsuperscript{§~\theparCount}}}
	\pend% ending standard par
      ‚{\tiny $_{lb}$}‚

	  
	  \pstart \leavevmode% starting standard par
	\textbf{त‚त्स‚म्ब‚न्धादि}ति प‚रः । ताभिर्व्य‚क्तिभिः स‚म्ब‚न्धाद‚न्य‚तोपि सामान्याद‚न्य‚त्र‚{\tiny $_{lb}$}‚ प्र‚त्य‚य‚वृत्तिः ।
	{\color{gray}{\rmlatinfont\textsuperscript{§~\theparCount}}}
	\pend% ending standard par
      ‚{\tiny $_{lb}$}‚

	  
	  \pstart \leavevmode% starting standard par
	\textbf{संख्येत्या} चा र्यः । भाव‚ल‚क्ष‚णा चेयं स‚प्त‚मी । संख्यायां स‚त्यां \textbf{कार्य‚द्र‚व्येऽव‚य}‚{\tiny $_{lb}$}‚विनि \textbf{स‚ति । आदि}श‚ब्दात् संयोगादिषु स‚न्न‚प्र‚त्य‚य‚वृत्तेस्तेपि सामान्यं प्राप्नुव‚न्तीति‚{\tiny $_{lb}$}‚ स‚मुदायार्थः । \textbf{असामान्यात्म‚क‚त्वात्} संख्यादीनान्त‚{\tiny $_{२}$}‚द्व‚लेन द्र‚व्ये नैकाकार‚प्र‚त्य‚य‚{\tiny $_{lb}$}‚वृत्तिरिति \textbf{चेत्} ।
	{\color{gray}{\rmlatinfont\textsuperscript{§~\theparCount}}}
	\pend% ending standard par
      ‚{\tiny $_{lb}$}‚

	  
	  \pstart \leavevmode% starting standard par
	न‚नु स एवायं सामान्यात्मा विचार्य‚ते [।] \textbf{कोयं सामान्यात्मे}ति । संख्यादिभ्यो‚{\tiny $_{lb}$}‚ विवेकेन सामान्य‚ल‚क्ष‚ण‚स्यैवाप्र‚तीत‚त्वात् । नैत‚द् व्य‚क्त‚मिति याव‚त् । त‚देवाह ।‚{\tiny $_{lb}$}‚ \textbf{त‚त्रे}त्यादि । \textbf{त‚त्र} कोयं सामान्यात्मेति पृष्टे त्व‚योक्त‚मेक‚स्यानेकेन \textbf{स‚ति स‚म्ब‚न्धे}‚{\tiny $_{lb}$}‚ व्य‚क्तिष्व‚भिन्न\textbf{प्र‚त्य‚य‚वृत्तिः । त‚त} इति प्र‚{\tiny $_{३}$}‚त्य‚य‚वृत्तेः कार‚णाद‚नेक‚स‚म्ब‚द्ध‚मेकं‚{\tiny $_{lb}$}‚ \textbf{सामान्य‚मि}ति सामान्य‚ल‚क्ष‚णं ।
	{\color{gray}{\rmlatinfont\textsuperscript{§~\theparCount}}}
	\pend% ending standard par
      ‚{\tiny $_{lb}$}‚

	  
	  \pstart \leavevmode% starting standard par
	अत्र सामान्य‚ल‚क्ष‚णेऽस्माभिरुच्य‚ते । \textbf{अनेक‚स‚म्ब‚न्धि}नो विद्य‚न्ते येषां कार्य‚{\tiny $_{lb}$}‚द्र‚व्यादीनां । आदिश‚ब्दात् द्वित्वादिप‚रिग्र‚हः । तेभ्योपि त‚दाश्र‚य‚द्र‚व्येष्वेकाकार‚{\tiny $_{lb}$}‚प्र‚त्य‚योत्प‚त्तिः स्यात् । किङ्कार‚ण‚म् [।] \textbf{निमित्त‚स‚म्भ‚वात्} । त‚था ह्य‚नेक‚स‚म्ब‚न्धा‚{\tiny $_{lb}$}‚देव निमित्ता‚{\tiny $_{४}$}‚त् सामान्यादेक‚प्र‚त्य‚योत्प‚त्तिरिष्य‚ते[।]अस्ति चानेक‚स‚म्ब‚न्धित्व‚मेक‚{\tiny $_{lb}$}‚प्र‚त्य‚य‚निमित्त‚न्द्र‚व्यादिष्व‚पि । \textbf{त‚त}श्चेत्येक‚प्र‚त्य‚य‚प्र‚वृत्तेः संख्यादीनां \textbf{सामान्यात्म‚ता ।‚{\tiny $_{lb}$}‚ अन्य}थेति । य‚थोक्त‚सामान्य‚ल‚क्ष‚ण‚योगेपि संख्यादिषु सामान्यात्म‚ता य‚दि नेष्य‚ते ।‚{\tiny $_{lb}$}‚ \textbf{अन्य‚त्रापि} सामान्याभिम‚ते \textbf{मा भूत्} । किं कार‚णं । \textbf{विशेषाभावात् । त‚था च}‚{\tiny $_{lb}$}‚ द्र‚व्यादी‚{\tiny $_{५}$}‚नाम‚पि सामान्य‚रूप‚ताप‚त्तौ \textbf{द्र‚व्य‚गुणादीनां रूप‚संक‚रः । बुद्धेरेव प्र‚तिभास}‚{\tiny $_{lb}$}‚  ‚{\tiny $_{lb}$}‚ ‚{\tiny $_{lb}$}‚ \leavevmode\ledsidenote{\textenglish{175/s}}इति विक‚ल्प‚बुद्धेरेकाकार‚प्र‚तिभासः \textbf{सामान्य}मिति स‚म्ब‚न्धः । स च ज्ञान‚रूप‚त्वात् ।‚{\tiny $_{lb}$}‚ ज्ञान‚व‚त् स‚न्नेव । त‚दुक्तं [।]
	{\color{gray}{\rmlatinfont\textsuperscript{§~\theparCount}}}
	\pend% ending standard par
      ‚{\tiny $_{lb}$}‚
	  \bigskip
	  \begingroup
	
	    
	    \stanza[\smallbreak]
	  {\normalfontlatin\large ``\qquad}सामान्यं व‚स्तु रूपं हि बुद्ध्याकारो भ‚विष्य‚ति ।&‚{\tiny $_{lb}$}‚व‚स्तुरूपा च सा बुद्धिः श‚ब्दार्थेषूप‚जाय‚ते ।&‚{\tiny $_{lb}$}‚तेन व‚स्त्वेव क‚ल्प्येत वाच्यं बुद्ध्य‚न‚पोह‚क‚मिति ।{\normalfontlatin\large\qquad{}"}\&[\smallbreak]
	  
	  
	  
	  \endgroup
	‚{\tiny $_{lb}$}‚

	  
	  \pstart \leavevmode% starting standard par
	\textbf{त‚न्ने}त्यादिना प्र‚तिषेध‚ति । य‚त्त‚ज्ज्ञान‚रूपं सामान्य‚मिष्य‚ते । त‚ज्ज्ञा\textbf{नाद‚व्य‚तिरिक्तं}‚{\tiny $_{lb}$}‚ ज्ञान‚स्व‚ल‚क्ष‚ण‚व‚त् \textbf{क‚थ‚म‚थ‚न्त‚र}म्बाह्यं \textbf{ब्र‚जेत्} । न तेषां सामान्य‚मिति याव‚त् [।]
	{\color{gray}{\rmlatinfont\textsuperscript{§~\theparCount}}}
	\pend% ending standard par
      ‚{\tiny $_{lb}$}‚

	  
	  \pstart \leavevmode% starting standard par
	त‚द् व्याच‚ष्टे [।] \textbf{ज्ञान‚स्ये}त्यादि । \textbf{त‚स्य} ज्ञान‚रूप‚स्य \textbf{तेष्व‚र्थेष्व‚भावात्} ।
	{\color{gray}{\rmlatinfont\textsuperscript{§~\theparCount}}}
	\pend% ending standard par
      ‚{\tiny $_{lb}$}‚

	  
	  \pstart \leavevmode% starting standard par
	स‚त्यं [।] न ज्ञान‚रूप‚स्य व्य‚क्तिष्व‚न्व‚यः किन्तु त‚स्मिन् बुद्धिप्र‚तिभासे \textbf{त‚द्‚{\tiny $_{lb}$}‚भावाध्य‚व‚साया}त् । बाह्य‚भा‚{\tiny $_{७}$}‚वाध्य‚व‚सायात् । \textbf{त‚था भ्रान्त्या} स‚मान\textbf{व्य‚व‚हार} \leavevmode\ledsidenote{\textenglish{66a/PSVTa}}‚{\tiny $_{lb}$}‚ इति चेत् ।
	{\color{gray}{\rmlatinfont\textsuperscript{§~\theparCount}}}
	\pend% ending standard par
      ‚{\tiny $_{lb}$}‚

	  
	  \pstart \leavevmode% starting standard par
	एत‚च्चेष्ट‚मेव सि द्धा न्त वा दि नः । केव‚लं प्र‚कृत्यैक‚कार्याः व्य‚क्त‚योऽत‚त्कार्याद्‚{\tiny $_{lb}$}‚ व्यावृत्ताः । त‚थाभूताया विक‚ल्प‚बुद्धेर्निमित्त‚मित्य‚न्यापोहाश्र‚या सा बुद्धिरित्य‚{\tiny $_{lb}$}‚भिम‚तं शास्त्र‚कार‚स्य ।
	{\color{gray}{\rmlatinfont\textsuperscript{§~\theparCount}}}
	\pend% ending standard par
      ‚{\tiny $_{lb}$}‚

	  
	  \pstart \leavevmode% starting standard par
	प‚र‚स्त्वेवंभूतं निमित्तं नेच्छ‚ति । अत एव सिद्धान्त‚वादी निमित्त‚मेव प‚र्य‚नु‚{\tiny $_{lb}$}‚युंक्ते‚{\tiny $_{१}$}‚[।] \textbf{त‚त्रे}त्यादि । \textbf{त‚त्र} व्य‚क्तिभेदेष्व‚व‚स्तुभूतेषु व्य‚क्तीनां च प्र‚कृत्या विजा‚{\tiny $_{lb}$}‚तीय‚व्यावृत्तानामेक‚कार्य‚त्वानिच्छ‚त‚चो\edtext{}{\lemma{चो}\Bfootnote{? तो}}ज्ञानो\textbf{त्प‚त्तेः किन्निब‚न्ध‚नं} । नैव‚{\tiny $_{lb}$}‚ किञ्चित् । त‚था हि प‚रो विजातीय‚व्यावृत्तानां भेदानामेक‚प्र‚त्य‚य‚हेतुत्व‚न्नेच्छ‚ति ।‚{\tiny $_{lb}$}‚ न चास्ति व‚स्तुभूतं सामान्यं । \textbf{अनाश्र‚य‚स्ये}त्य‚निमित्तिस्य सामान्य‚ज्ञान‚स्यो\textbf{त्प‚त्तौ‚{\tiny $_{lb}$}‚ स‚र्व‚त्र स्या}दिति गौरित्येकाकारः प्र‚त्य‚यो वृक्षेष्व‚पि‚{\tiny $_{२}$}‚ स्यात् ।
	{\color{gray}{\rmlatinfont\textsuperscript{§~\theparCount}}}
	\pend% ending standard par
      ‚{\tiny $_{lb}$}‚

	  
	  \pstart \leavevmode% starting standard par
	\textbf{ज्ञानाद‚व्य‚तिरिक्त}मित्यादेर‚प‚र‚म‚र्थ‚माह । \textbf{अथ‚वे}त्यादि । पूर्व‚म‚र्थान्त‚र‚श‚ब्देन‚{\tiny $_{lb}$}‚  ‚{\tiny $_{lb}$}‚ ‚{\tiny $_{lb}$}‚ \leavevmode\ledsidenote{\textenglish{176/s}}बाह्य‚मुक्त‚म‚धुना ज्ञानान्त‚रं निर्दिश्य‚ते । अत एवाह [।] \textbf{क‚थ‚म‚न्य‚स्य पुन‚र्ज्ञान‚स्येति} ।‚{\tiny $_{lb}$}‚ किम्विशिष्ट‚स्य ज्ञान‚स्य [।] \textbf{व्य‚क्त्य‚न्त‚र‚भावि}नः । एक‚स्यां गोव्य‚क्तौ य‚द्विक‚ल्प‚{\tiny $_{lb}$}‚विज्ञान‚न्त‚तोन्य‚त्र गोव्य‚क्त्य‚न्त‚रेण स‚मुत्प‚न्न‚स्य विक‚ल्प‚ज्ञान‚स्येत्य‚र्थः । त‚था ह्य‚नेक‚{\tiny $_{lb}$}‚ज्ञान‚व्याप‚नाद्वा सामान्य‚म्भ‚वेत् । बाह्य‚व्य‚क्तिव्या‚{\tiny $_{३}$}‚प‚नाद्वा । ज्ञान‚प्र‚तिभास‚स्य तु‚{\tiny $_{lb}$}‚ द्व‚य‚म‚प्य‚स‚त् । त‚दाह । \textbf{त‚त‚श्चे}त्यादि । \textbf{व्य‚क्त्य‚न्त‚र}मिति बाह्यं ॥
	{\color{gray}{\rmlatinfont\textsuperscript{§~\theparCount}}}
	\pend% ending standard par
      ‚{\tiny $_{lb}$}‚

	  
	  \pstart \leavevmode% starting standard par
	\textbf{त‚स्मा}दित्युप‚संहारः ।
	{\color{gray}{\rmlatinfont\textsuperscript{§~\theparCount}}}
	\pend% ending standard par
      ‚{\tiny $_{lb}$}‚

	  
	  \pstart \leavevmode% starting standard par
	\textbf{न ही}त्यादिना व्याच‚ष्टे । \textbf{केन‚चि}दिति सामान्य‚रूपेण । \textbf{त‚थैषां ग्र‚ह‚ण}मित्य‚र्थानां‚{\tiny $_{lb}$}‚ स‚माना इति ग्र‚ह‚णं ॥
	{\color{gray}{\rmlatinfont\textsuperscript{§~\theparCount}}}
	\pend% ending standard par
      ‚{\tiny $_{lb}$}‚

	  
	  \pstart \leavevmode% starting standard par
	न‚नु सि द्धा न्त वा दि ना प्य‚स्य विक‚ल्प‚स्य निब‚न्ध‚नं वाच्य‚म‚नाश्र‚य‚स्योत्प‚त्तौ‚{\tiny $_{lb}$}‚ स‚र्व‚त्र प्र‚संगादित्याह । \textbf{इत‚रेत‚र‚भेद} इत्यादि । \textbf{संज्ञा} संकेत‚क्रिया \textbf{य‚द‚{\tiny $_{४}$}‚र्थिका} [।]‚{\tiny $_{lb}$}‚ य इत‚रेत‚र‚भेदः । अर्थः फ‚लं प्र‚त्याय्य‚त्वेन य‚स्य इति कृत्वा स एव‚म्भूत इत‚रेत‚र‚भेदो‚{\tiny $_{lb}$}‚ भावानाम‚न्योन्य‚व्यावृत्तिल‚क्ष‚णोस्य मिथ्याविक‚ल्प‚स्य \textbf{बीजं संज्ञा} संकेत‚क्रिया ।‚{\tiny $_{lb}$}‚ \textbf{य‚द‚र्थिका} । य‚स्येत‚रेत‚र‚भेद‚स्य प्र‚त्याय‚न‚फ‚ला ।
	{\color{gray}{\rmlatinfont\textsuperscript{§~\theparCount}}}
	\pend% ending standard par
      ‚{\tiny $_{lb}$}‚

	  
	  \pstart \leavevmode% starting standard par
	\textbf{य‚स्ये}त्यादिना व्याच‚ष्टे । य‚स्ये\textbf{त‚रेत‚र‚भेद}स्य \textbf{प्र‚त्याय‚नार्थं संकेतः क्रिय‚ते ।‚{\tiny $_{lb}$}‚ अत‚त्साध्येभ्य} इत्य‚त‚त्कार्येभ्यो \textbf{भिन्नासाध्यान्भावा}न्भेदेन‚{\tiny $_{५}$}‚ \textbf{ज्ञात्वा} त\textbf{त्प‚रिहारेणे}‚{\tiny $_{lb}$}‚त्येत‚त् कार्य‚प‚रिहारेण त‚त्कार्येषु \textbf{प्र‚व‚र्त्ते}तेति कृत्वा \textbf{संकेतः क्रिय‚ते । सोयं} य‚थोक्त‚{\tiny $_{lb}$}‚ इत‚रेत‚र‚भेद\textbf{स्त‚स्यैकात्म‚ताप्र‚तिभा}सिन एकाकार‚स्य मि\textbf{थ्याविक‚ल्प}स्य \textbf{बीजं} कार‚णं ।‚{\tiny $_{lb}$}‚ \textbf{त‚मेव गृह्ण}न्निति भेदं भिन्न‚मित्य‚र्थः । एत‚च्चाध्य‚व‚साय‚व‚शादुच्य‚ते । न पुन‚र्विक‚ल्प‚स्य‚{\tiny $_{lb}$}‚ व‚स्तुगृह‚ण‚म‚स्ति[।]\textbf{एष विक‚ल्प} इति सामान्याकारो विक‚ल्पः \textbf{स्व‚वास‚ना‚{\tiny $_{६}$}‚प्र‚कृते}रिति‚{\tiny $_{lb}$}‚ विक‚ल्प‚वास‚नास्व‚भावात् । \textbf{ए}व‚मित्येकाकार‚त‚या \textbf{प्र‚तिभा}ति । \href{http://sarit.indology.info/?cref=pv.3.71}{। ७४ ॥}
	{\color{gray}{\rmlatinfont\textsuperscript{§~\theparCount}}}
	\pend% ending standard par
      ‚{\tiny $_{lb}$}‚

	  
	  \pstart \leavevmode% starting standard par
	\textbf{क‚थं पुन‚र्भिन्नानां} स्व‚ल‚क्ष‚णाना\textbf{म‚भिन्नं कार्य}मेकाकार‚विक‚ल्पात्म‚कं ।‚{\tiny $_{lb}$}‚ ‚{\tiny $_{lb}$}‚ \leavevmode\ledsidenote{\textenglish{177/s}}\textbf{ये}नेत्येक‚कार्य‚त्वेन । \textbf{त‚द‚न्येभ्यो}ऽत‚त्कार्य‚भ्यो \textbf{भेदा}द्धेतोर्व्य‚क्तीना\textbf{म‚भेद इत्युच्ते} ।‚{\tiny $_{lb}$}‚ एकासंस‚र्ग‚स्त‚द्व्य‚तिरेकिणां स‚मान‚तेति व‚च‚नात् ।
	{\color{gray}{\rmlatinfont\textsuperscript{§~\theparCount}}}
	\pend% ending standard par
      ‚{\tiny $_{lb}$}‚

	  
	  \pstart \leavevmode% starting standard par
	\textbf{प्र‚कृति}रित्यादिना प‚रिह‚र‚ति । प्र‚कृतिः स्व‚भाव एकाकारं‚{\tiny $_{७}$}‚ प्र‚त्य‚भिज्ञान- \leavevmode\ledsidenote{\textenglish{66b/PSVTa}}‚{\tiny $_{lb}$}‚ \textbf{मेक‚प्र‚त्य‚व‚म‚र्शः} । अनुभ‚व‚ज्ञान‚म‚र्थ‚ज्ञानं । \textbf{एक‚प्र‚त्य‚व‚म‚र्श‚श्चार्थ‚ज्ञा}नं चेति द्व‚न्द्वः ।‚{\tiny $_{lb}$}‚ पूर्व‚निपात‚ल‚क्ष‚ण‚स्य व्य‚भिचारित्वात् । अल्पाज्त‚रत्वे\edtext{}{\lemma{त्वे}\Bfootnote{\href{http://sarit.indology.info/?cref=P\%C4\%81.2.2.34}{ Pāṇini 2: 2: 34. }}}प्य‚र्थ‚ज्ञान‚श‚ब्द‚स्य न पूर्व‚{\tiny $_{lb}$}‚निपातः कृतः । ते \textbf{आदी} य‚स्येति विग्र‚हः । आदिश‚ब्दाद् द‚ह‚न‚गृहादिकार्य‚ग्र‚ह‚णं ।‚{\tiny $_{lb}$}‚ एक‚प्र‚त्य‚व‚म‚र्शादिरेवैकोर्थ इति क‚र्म‚धार‚यः । \textbf{त‚स्य साध‚ने} सिद्धौ \textbf{भेदेपि} नानात्वेपि‚{\tiny $_{lb}$}‚ \textbf{निय‚ताः}‚{\tiny $_{१}$}‚ केचित् । \textbf{स्व‚भावेन} प्र‚कृत्या । \textbf{इन्द्रिया}दिव‚त् । अथ‚वैकान्तेन \textbf{भेदे}पि स्व‚हे‚{\tiny $_{lb}$}‚तुभ्यः \textbf{केचि}त् स‚माना उत्प‚न्नाः केचिद‚स‚माना इत्येत‚च्चोक्त‚न्त‚त्र ये स‚माना उत्प‚न्नास्ते‚{\tiny $_{lb}$}‚ तेन स्व‚भावेनैकाकारं प्र‚त्य‚य‚ञ्ज‚न‚य‚न्ति विनापि \textbf{सामान्येनेन्द्रियादिव}त् । त‚त्रैक‚प्र‚त्य‚{\tiny $_{lb}$}‚व‚म‚र्श‚ज्ञान‚साध‚ने \textbf{निय‚ता} इत्येत‚द् दार्ष्टान्तिक‚त्वेनोप‚न्य‚स्त‚म‚र्थ‚ज्ञानाद्येकार्थ‚साध‚न‚{\tiny $_{lb}$}‚ इत्येत‚त्तु दृष्टान्त‚त्वेनोभ‚य‚सि‚{\tiny $_{२}$}‚द्ध‚त्वात् ।
	{\color{gray}{\rmlatinfont\textsuperscript{§~\theparCount}}}
	\pend% ending standard par
      ‚{\tiny $_{lb}$}‚

	  
	  \pstart \leavevmode% starting standard par
	अत एवादौ विभ‚ज्य‚ते । \textbf{य‚थेन्द्रिये}त्यादि । \textbf{य‚थेन्द्रियालोक‚म‚न‚स्कारा रूप‚वि‚{\tiny $_{lb}$}‚ज्ञान‚मेकं ज‚न‚य‚न्तीति} स‚म्ब‚न्धः । एत‚द्व‚स्तुब‚ल‚सिद्ध‚मुदाह‚र‚ति । \textbf{आत्मे}त्यादि प‚र‚{\tiny $_{lb}$}‚सिद्धान्ताश्र‚येण । नित्य‚म‚णु म‚नः शीघ्रं चेत्य‚णुस्व‚रूप‚म्म‚नः । \textbf{त‚त्स‚न्निक}र्षा‚{\tiny $_{lb}$}‚इत्या\textbf{त्मेन्द्रिय‚म‚नोर्थ‚स‚न्निक‚र्षाः} । आत्मा म‚न‚सा संयुज्य‚ते म‚न इन्द्रियेणेन्द्रिय‚म‚{\tiny $_{lb}$}‚र्थेनेति व‚च‚नात्\edtext{}{\edlabel{pvsvt_177-3}\label{pvsvt_177-3}\lemma{नात्}\Bfootnote{\href{http://sarit.indology.info/?cref=nbh.1.1.4}{ Nyāya-bhāṣya 1: 1: 4. }}} । आत्मे‚{\tiny $_{३}$}‚न्द्रिय‚म‚नोर्थाश्च त‚त्स‚न्निक‚र्षाश्चेति द्व‚न्द्वः । आत्मे‚{\tiny $_{lb}$}‚न्द्रिय‚म‚नोर्थाः । य‚थास्वं स‚न्निक‚र्ष‚स‚हाया विज्ञानं ज‚न‚य‚न्तीति प‚राभ्युप‚ग‚मः ।‚{\tiny $_{lb}$}‚ \textbf{अस‚त्य‚पि} त\textbf{द्भाव‚नि}य‚त इत्येक‚कार्य‚त्व‚निय‚ते । न हि च‚क्षुरादीनां च‚क्षुर्विज्ञान‚ज‚{\tiny $_{lb}$}‚न‚क‚त्वं नाम सामान्य‚म्प‚रेणेष्टं ।
	{\color{gray}{\rmlatinfont\textsuperscript{§~\theparCount}}}
	\pend% ending standard par
      ‚{\tiny $_{lb}$}‚‚{\tiny $_{lb}$}‚‚{\tiny $_{lb}$}‚‚{\tiny $_{lb}$}‚\textsuperscript{\textenglish{178/s}}

	  
	  \pstart \leavevmode% starting standard par
	अधुना दार्ष्टान्तिक‚म्व्याच‚ष्टे । \textbf{शिंश‚पाद‚य} इति शिंश‚पाख‚दिर‚न्य‚ग्रोधाद‚यः‚{\tiny $_{lb}$}‚ \textbf{प‚र‚स्प‚रान‚न्व‚ये}पि । वृ‚{\tiny $_{४}$}‚क्ष‚त्व‚सामान्य‚विर‚हेपि वृक्ष इत्\textbf{येकाकारं प्र‚त्य‚भिज्ञानं ज‚न‚{\tiny $_{lb}$}‚य‚न्ति} । प्र‚त्य‚भिज्ञाना\textbf{द‚न्याम्वा द‚ह‚न‚गृहादि}कां \textbf{काष्ठ‚साध्याम‚र्थ‚क्रि}यां शिंश‚पाद‚यो‚{\tiny $_{lb}$}‚ \textbf{ज‚न‚य‚न्ती}ति प्र‚कृतं । \textbf{य‚थाप्र‚त्य‚य}मिति याव‚द् [।] अग्निस‚ह‚कारिप्र‚त्य‚य‚लाभ‚स्त‚दा‚{\tiny $_{lb}$}‚ द‚ह‚नं ज‚न‚य‚न्ति । गृहानुकूल‚प्र‚त्य‚य‚संपाते गृहं । आदिश‚ब्दाद् र‚थादिकार्य‚प‚रिग्र‚हः ।‚{\tiny $_{lb}$}‚ \textbf{न तु भेदाविशेषेपि ज‚लाद‚यः}‚{\tiny $_{५}$}‚ काष्ठ‚साध्यार्थ‚क्रियास‚म‚र्थाः प्र‚कृत्या तेषाम‚त‚त्कार्य‚{\tiny $_{lb}$}‚त्वात् । अत्रापि दृष्टान्त‚माह । \textbf{श्रोत्रादिव‚द् रूप‚ज्ञान} इति । य‚था श्रोत्र‚श‚ब्दाद‚यो‚{\tiny $_{lb}$}‚ रूप‚विज्ञाने क‚र्त्त‚व्ये न स‚म‚र्थाः । आदिश‚ब्दाद् र‚सादिविज्ञाने ॥
	{\color{gray}{\rmlatinfont\textsuperscript{§~\theparCount}}}
	\pend% ending standard par
      ‚{\tiny $_{lb}$}‚

	  
	  \pstart \leavevmode% starting standard par
	स्यादेत‚द् [।] ब‚हूनां प्र‚त्येक‚मेक‚कार्य‚क‚र्त्तृत्वं सामान्य‚म‚न्त‚रेण न सिध्य‚तीत्य‚त‚{\tiny $_{lb}$}‚ आह । \textbf{ज्व‚रा}दित्यादि । \textbf{स‚हे}ति व्य‚क्त्य‚न्त‚र‚स‚हिताः । \textbf{प्र‚त्येक}मित्येकैक‚रूपा‚{\tiny $_{६}$}‚‚{\tiny $_{lb}$}‚ ज्व‚रादिश‚म‚ने एक‚स्मिन् कार्ये \textbf{दृष्टा य‚थौष‚ध‚यः । वा} श‚ब्दः पूर्व‚दृष्टान्तापेक्ष‚या ।‚{\tiny $_{lb}$}‚ \textbf{न चाप‚रा} द‚धित्र‚पु\textbf{सा}\edtext{}{\lemma{पु}\Bfootnote{? षा}}द‚यः ।
	{\color{gray}{\rmlatinfont\textsuperscript{§~\theparCount}}}
	\pend% ending standard par
      ‚{\tiny $_{lb}$}‚

	  
	  \pstart \leavevmode% starting standard par
	\textbf{य‚थे}त्यादिना व्याच‚ष्टे । \textbf{न त‚त्र} ज्व‚रादिश‚म‚ने क‚र्त्त‚व्ये \textbf{सामान्य}मोष‚धित्वं‚{\tiny $_{lb}$}‚ नामा\textbf{पेक्ष‚न्ते}? । किङ्कार‚णं [।] \textbf{भेदेपि त‚त्प्र‚कृतिक‚त्वा}त् । ज्व‚रादिश‚म‚न‚कार्य‚{\tiny $_{lb}$}‚स्व‚भाव‚त्वात् । य‚दि भेदानाम‚साम‚र्थ्यं स्यात् । भ‚वेत्सामान्यापेक्षा । \textbf{न त‚द‚वि‚{\tiny $_{lb}$}‚\leavevmode\ledsidenote{\textenglish{67a/PSVTa}} शेषेपि}‚{\tiny $_{७}$}‚ भेदाविशेषेपि \textbf{द‚धित्र‚पुसाद‚यः} । द‚ध्येव म‚न्द‚जात‚न्द‚धित्र‚पुसं । द‚धि च‚{\tiny $_{lb}$}‚ त्र‚पुस‚श्चेति द्व‚न्द्व\textbf{म‚न्ये} व्याच‚क्ष‚ते ॥
	{\color{gray}{\rmlatinfont\textsuperscript{§~\theparCount}}}
	\pend% ending standard par
      ‚{\tiny $_{lb}$}‚

	  
	  \pstart \leavevmode% starting standard par
	तासु गुडूचीव्य‚क्त्यादिषु \textbf{त‚थाभूतास्वे}क‚कार्य‚कारिणीषु । \textbf{किंचिदि}ति व्य‚ति‚{\tiny $_{lb}$}‚रिक्त‚म‚व्य‚तिरिक्तं च । त‚दुक्तं [।]
	{\color{gray}{\rmlatinfont\textsuperscript{§~\theparCount}}}
	\pend% ending standard par
      ‚{\tiny $_{lb}$}‚‚{\tiny $_{lb}$}‚‚{\tiny $_{lb}$}‚\textsuperscript{\textenglish{179/s}}
	  \bigskip
	  \begingroup
	
	    
	    \stanza[\smallbreak]
	  {\normalfontlatin\large ``\qquad}निर्व‚र्त्त्य‚मानं य‚त्क‚र्म जातिस्त‚त्रापि साध‚नं ।&‚{\tiny $_{lb}$}‚स्वाश्र‚य‚स्याभिनिष्प‚त्त्यै सा क्रियायाः प्र‚योजिकेति ।{\normalfontlatin\large\qquad{}"}\&[\smallbreak]
	  
	  
	  
	  \endgroup
	‚{\tiny $_{lb}$}‚

	  
	  \pstart \leavevmode% starting standard par
	\textbf{त‚त एव} सामान्यात् \textbf{त‚देकं} ज्व‚रादिश‚म‚{\tiny $_{१}$}‚न‚ल‚क्ष‚णं \textbf{कार्य‚न्}त‚त‚श्चासिद्धो दृष्टान्त‚{\tiny $_{lb}$}‚ इति भावः ।
	{\color{gray}{\rmlatinfont\textsuperscript{§~\theparCount}}}
	\pend% ending standard par
      ‚{\tiny $_{lb}$}‚

	  
	  \pstart \leavevmode% starting standard par
	\textbf{त‚द‚युक्त‚मि}ति सि द्धा न्त वा दी । \textbf{अविशेषात् सामान्य‚स्}येति । एक‚त्वान्नित्य‚{\tiny $_{lb}$}‚त्वाच्च अविशिष्टं \textbf{सामान्य‚न्\textbf{न}त‚त्कार्य‚कृदि}ति श‚म‚न‚कार्य‚कृत् । अन्य‚था सामान्य‚{\tiny $_{lb}$}‚स्याविशेषा\textbf{त्तासां} गूडूच्यादिव्य‚क्तीनां \textbf{क्षेत्रादिभेदे}पि त‚स्यापि ज्व‚रादिश‚म‚न‚कार्य‚{\tiny $_{lb}$}‚\textbf{स्याविशेष‚प्र‚संग‚तः । चिर‚शीघ्रे}त्यादि । विशिष्ट‚क्षेत्रो‚{\tiny $_{२}$}‚त्प‚न्नानां शीघ्र‚प्र‚श‚म‚नं‚{\tiny $_{lb}$}‚ विप‚रीतानां चिर‚प्र‚श‚म‚नं । \textbf{आदि}श‚ब्दाच्चिर‚त‚र‚शीघ्र‚त‚रादिप‚रिग्र‚हः । क्षेत्र‚{\tiny $_{lb}$}‚संस्कारादिभिन्नानां गूडूच्यादीनामुप‚योगाद्देहे आरोग्यादिल‚क्ष‚ण‚स्य \textbf{गुण‚स्य तार‚त‚म्यं‚{\tiny $_{lb}$}‚ च न स्यात्} । सामान्य‚स्यैक्यात् । \textbf{अथ} क्षेत्रादिभेदेन \textbf{सामान्य‚स्य विशेष} इष्य‚ते ।‚{\tiny $_{lb}$}‚ त‚दा विशेषे वा सामान्य‚स्येष्य‚माणे \textbf{स्व‚भाव‚भेदः} स्याद् विशेष‚ल‚क्ष‚{\tiny $_{३}$}‚ण‚त्वाद् भेद‚स्य ।‚{\tiny $_{lb}$}‚ \textbf{त‚त}श्च \textbf{स्व‚रूप‚हानं} । सामान्य‚स्व‚रूप‚मेकं हीय‚ते । \textbf{ध्रौव्याच्च} कार‚णात् \textbf{सामान्य}स्य‚{\tiny $_{lb}$}‚ व्य‚क्तिभ्यो\textbf{नुप‚कार‚तो} न सामान्यं कार्य‚कृदिति व‚र्त्त‚ते । \textbf{य‚दि हि सामान्य‚मुप‚कुर्यात्}‚{\tiny $_{lb}$}‚ त‚दा नित्य‚त्वात् स‚ह‚कारिभि\textbf{र‚नाधेय‚विशे}ष‚स्या\textbf{न्यान‚पेक्ष‚ण‚त्वात्} स‚ह‚कार्य‚न‚पेक्ष‚णात्‚{\tiny $_{lb}$}‚ त‚त् सामान्यं \textbf{स्व‚कार्य स‚कृज्ज‚न‚येत्} । अथ न ज‚न‚येत् त‚दा \textbf{त‚ज्ज‚न‚न‚स्व‚भावं न‚{\tiny $_{lb}$}‚ भ‚{\tiny $_{४}$}‚व}ति । अज‚न‚नाव‚स्थाया अविशेषात् कार्य‚कालेपि न ज‚न‚येदिति याव‚त् ।‚{\tiny $_{lb}$}‚ व्य‚क्तीनां त्व‚नित्यानां कार्य‚कृत्त्वे नायं दोष इत्याह । \textbf{व्य‚क्त‚य‚स्}त्वित्यादि [।]‚{\tiny $_{lb}$}‚ \textbf{संस्कारो} ज‚लाव‚सेकादि । \textbf{विशिष्टा उत्प}त्तिर्यांसान्तास्त‚था । विशेषोस्यास्तीति‚{\tiny $_{lb}$}‚  ‚{\tiny $_{lb}$}‚ ‚{\tiny $_{lb}$}‚ \leavevmode\ledsidenote{\textenglish{180/s}}\textbf{विशेष‚व‚त् कार्यं} ज्व‚रादिश‚म‚न‚ल‚क्ष‚णं । न च तासु व्य‚क्तिषु य‚च्छीघ्र‚कारित्वा‚{\tiny $_{lb}$}‚दिल‚क्ष‚ण‚म‚वान्त‚र‚सामान्य‚म‚व‚स्थि‚{\tiny $_{५}$}‚न्त‚देव विशेष‚व‚त् कार्य‚कारीति युक्त‚म्व‚क्तुं ।‚{\tiny $_{lb}$}‚ ओष‚ध्य‚नुप‚योगेपि पुंसः त‚त्कार्योद‚य‚प्र‚स‚ङ्गात् । \textbf{त‚द्व‚दि}ति विशिष्ट‚व्य‚क्तिव‚त् ।‚{\tiny $_{lb}$}‚ \textbf{केचि}दिति स‚जातीया \textbf{एक‚प्र‚त्य‚भिज्ञाना}दिकं । \textbf{आदि}श‚ब्दाद् एकोद‚काद्याह‚र‚णादि ।‚{\tiny $_{lb}$}‚ \textbf{त‚द‚कारिभ्य} इति । प्र‚त्य‚भिज्ञानाद्य‚कारिभ्यो \textbf{भेदाद}भिन्ना इत्युच्य‚न्ते । न‚{\tiny $_{lb}$}‚ त्वेक‚सामान्य‚योगात् ।
	{\color{gray}{\rmlatinfont\textsuperscript{§~\theparCount}}}
	\pend% ending standard par
      ‚{\tiny $_{lb}$}‚

	  
	  \pstart \leavevmode% starting standard par
	कार्य‚द्वारेणाभेदं प्र‚तिपाद्य का‚{\tiny $_{६}$}‚र‚ण‚द्वारेणाह । \textbf{एकेन वे}त्यादि । य‚था प्र‚य‚त्नेन‚{\tiny $_{lb}$}‚ घ‚ट‚भेदा \textbf{अत‚ज्ज‚न्येभ्य} इत्य‚प्र‚य‚त्न‚ज‚न्येभ्यो \textbf{भेदाद‚भिन्ना} इत्युच्य‚न्ते । य‚द्य‚पि प्र‚ति‚{\tiny $_{lb}$}‚घ‚टं प्र‚य‚त्न‚स्य भेद‚स्त‚थाप्येक‚प्र‚त्य‚भिज्ञान‚हेतुत्वेन त‚स्याप्येक‚त्वं । एत‚च्चोत्त‚र‚त्र‚{\tiny $_{lb}$}‚ निश्चाय‚यिष्य‚ते ।
	{\color{gray}{\rmlatinfont\textsuperscript{§~\theparCount}}}
	\pend% ending standard par
      ‚{\tiny $_{lb}$}‚

	  
	  \pstart \leavevmode% starting standard par
	\textbf{किम्पुन‚रि}त्यादि प‚रः । \textbf{भेदो} व्यावृत्ति\textbf{र्ल‚क्ष‚णं} निमित्तं य‚स्य \textbf{तेन} सामान्येना‚{\tiny $_{lb}$}‚\leavevmode\ledsidenote{\textenglish{67b/PSVTa}} त‚त्कार्ये‚{\tiny $_{७}$}‚भ्योऽत‚त्कार‚णेभ्य‚श्च व्यावृत्तं \textbf{स्व‚ल‚क्ष‚णं स‚मान‚मिति प्र‚त्ये}यं । \textbf{अन्य‚थान्य‚देवे}ति‚{\tiny $_{lb}$}‚ स्व‚ल‚क्ष‚णाद‚न्य‚द् विक‚ल्प‚बुद्धिप‚रिव‚र्त्तिरूप‚म‚न‚र्थ‚क्रियाकारि । \textbf{तेन} भेद‚ल‚क्ष‚णेन‚{\tiny $_{lb}$}‚ सामान्येन स‚मान‚मिति प्र‚त्येयं ।
	{\color{gray}{\rmlatinfont\textsuperscript{§~\theparCount}}}
	\pend% ending standard par
      ‚{\tiny $_{lb}$}‚

	  
	  \pstart \leavevmode% starting standard par
	\textbf{न‚नु} त‚देषां बुद्धिप्र‚तिभास‚म‚नुरुन्धानैर्बुद्धिविप‚रिव‚र्त्तिनामेव भावानामाकार‚{\tiny $_{lb}$}‚विशेष‚प‚रिग्र‚हाद् ब‚हिरिव प‚रिस्फुर‚तां सामान्य‚मुच्य‚त इति पूर्व‚मु‚{\tiny $_{१}$}‚क्त‚त्वात् स‚न्दे‚{\tiny $_{lb}$}‚हानुवृत्तिरेव ।
	{\color{gray}{\rmlatinfont\textsuperscript{§~\theparCount}}}
	\pend% ending standard par
      ‚{\tiny $_{lb}$}‚

	  
	  \pstart \leavevmode% starting standard par
	स‚त्यं । किन्त्व‚धिक‚स्य दोष‚स्य विधानार्थं उप‚न्यासः । दोष‚विज्ञानार्थ‚माह ।‚{\tiny $_{lb}$}‚ \textbf{किंचात} इति । इत‚रो \textbf{य‚दी}त्यादिना प्र‚श्नाभिप्राय‚माह । \textbf{य‚दि स्व‚ल}क्ष‚णं प्र‚त्येयं‚{\tiny $_{lb}$}‚ \textbf{क‚थं विक‚ल्प‚विष‚यः} । त‚था हि विक‚ल्प‚बुद्ध्य‚भिप्राय‚व‚शाद् भेद‚ल‚क्ष‚णं सामान्यं‚{\tiny $_{lb}$}‚ व्य‚व‚स्थाप्य‚ते त‚स्या अभिप्राय‚व‚शात् । \textbf{सामान्यं स‚त्प्र‚कीर्ति}त‚मिति व‚च‚{\tiny $_{२}$}‚नात् ।‚{\tiny $_{lb}$}‚ त‚थाभूतेन चेत् सामान्येन स्व‚ल‚क्ष‚णं स‚मानं प्र‚त्येयं । त‚दा विक‚ल्प‚स्य विष‚यः स्यात् ।‚{\tiny $_{lb}$}‚ न चैत‚द् युक्त‚म‚थान्य‚देव बुद्धिप‚रिव‚र्त्ति रूपं स‚मान‚मिति प्र‚त्येयं । अत्रापि दोष‚माह ।‚{\tiny $_{lb}$}‚ \textbf{अन्य‚तो वा क‚थ‚म‚र्थ‚क्रि}या [।] न हि बुद्धिप्र‚तिभासिरूपाद् अर्थ‚क्रिया स‚म्भ‚व‚ति ।‚{\tiny $_{lb}$}‚ ‚{\tiny $_{lb}$}‚ \leavevmode\ledsidenote{\textenglish{181/s}}त‚त‚श्चात‚त्कारिभ्यो भेदाद‚भिन्ना इत्युच्य‚न्त इति कार्य‚द्वारेण सामान्य‚व्य‚व‚स्था न‚{\tiny $_{३}$}‚‚{\tiny $_{lb}$}‚ घ‚ट‚ते । य‚त‚श्च बुद्धिप‚रिव‚र्त्ति रूपं स‚मान‚न्त‚तः \textbf{स्व‚ल‚क्ष‚णे चानित्य‚त्वादि}सामान्य‚{\tiny $_{lb}$}‚स्या\textbf{प्र‚तीतेर‚ताद्रूप्य‚म}नित्यादिरूप‚त्वं स्व‚ल‚क्ष‚ण‚स्य न भ‚वेत् । स्व‚ल‚क्ष‚णे चानित्य‚{\tiny $_{lb}$}‚त्वादीनाम‚प्र‚तीते\textbf{स्तेषां चा}नित्य‚त्वादीना\textbf{म‚व‚स्तुध‚र्म‚ता} ।
	{\color{gray}{\rmlatinfont\textsuperscript{§~\theparCount}}}
	\pend% ending standard par
      ‚{\tiny $_{lb}$}‚

	  
	  \pstart \leavevmode% starting standard par
	\textbf{ने}त्यादिना प‚रिह‚र‚ति । बुद्धिप्र‚तिभासिन्येव रूपे सामान्यादिव्य‚व‚हार इत्य‚यं‚{\tiny $_{lb}$}‚ प‚क्षो गृहीतः । त‚दाह । \textbf{ज्ञान‚प्र‚तिभासि‚{\tiny $_{४}$}‚न्य‚र्थ} इत्यादि ।
	{\color{gray}{\rmlatinfont\textsuperscript{§~\theparCount}}}
	\pend% ending standard par
      ‚{\tiny $_{lb}$}‚

	  
	  \pstart \leavevmode% starting standard par
	एत‚च्च ग्र‚ह‚ण‚क‚वाक्यं । अस्यैव व्याख्यानं । \textbf{य‚देत‚ज्ज्ञानं विक‚ल्प‚क}मित्य‚नेन‚{\tiny $_{lb}$}‚ स‚म्ब‚न्धः । \textbf{अत‚द्विष‚य‚म}पि व‚स्तुस्व‚भाव‚विष‚य‚म‚पि \textbf{त‚द्विष‚य‚मिव} स्व‚ल‚क्ष‚ण‚विष‚{\tiny $_{lb}$}‚य‚मिव । \textbf{अध्य‚व‚सित‚त‚द्भाव‚मारोपित‚बाह्य}भावं स्व‚रूपं य‚स्य त‚त्त‚था । य‚त‚श्चा‚{\tiny $_{lb}$}‚ध्य‚व‚सित‚त‚द्भाव‚म‚तः स्व‚ल‚क्ष‚ण‚विष‚य‚मेवेति म‚न्य‚ते । \textbf{अध्य‚व‚सित‚त‚द्भाव‚स्व‚रू}प‚त्व‚{\tiny $_{lb}$}‚मे‚{\tiny $_{५}$}‚व क‚थ‚मिति चेदाह । \textbf{त‚द‚नुभ‚वाहित‚वास‚नाप्र‚भ‚व‚प्र‚कृते}रिति[।]त‚स्य स्व‚ल‚क्ष‚ण‚स्य‚{\tiny $_{lb}$}‚ योनुभ‚व‚स्तेनाहिता वास‚ना त‚तः प्र‚भ‚व उत्पाद‚स्त‚स्य सामान्य‚स्य सा प्र‚कृतिः स्व‚भावो‚{\tiny $_{lb}$}‚ येनाध्य‚व‚सित‚भाव‚स्व‚रूप‚म्भ‚व‚तीत्य‚र्थः । \textbf{अभिन्न‚का}र्या ये \textbf{प‚दा}र्था घ‚टाद‚यः । एका‚{\tiny $_{lb}$}‚कार‚प्र‚त्य‚य‚ज्ञान‚हेत‚व\textbf{स्तेभ्यः} प‚र‚म्प‚र‚या \textbf{प्र‚सू}तेर\textbf{भिन्नार्थ‚ग्राही}व प्र‚तिभाति । न तु‚{\tiny $_{lb}$}‚ सामान्य‚व‚स्तु‚{\tiny $_{६}$}‚भूतं किंचिद् व्य‚तिरिक्त‚म‚व्य‚तिरिक्त‚म्वाऽस्ति य‚त् त‚द् गृह्णीयात् ।‚{\tiny $_{lb}$}‚ \textbf{प‚र‚मार्थ‚त‚स्तु त‚द‚न्य‚भेद‚स‚माका}रं तेभ्यः स‚जातीयाभिम‚तेभ्योऽन्ये विजातीयास्तेभ्यो‚{\tiny $_{lb}$}‚ भेदः भिन्नः स्व‚भावः स एव प‚र‚मार्थेन स‚मान आकारो य‚स्येति विग्र‚हः । त‚त एव‚{\tiny $_{lb}$}‚ त‚स्योत्प‚त्तेस्त‚न्निव‚र्त्त‚न‚त्वाच्च \textbf{त‚द‚न्य‚भेद}स्त‚स्य \textbf{स‚मान आका}र इत्युच्य‚ते ।
	{\color{gray}{\rmlatinfont\textsuperscript{§~\theparCount}}}
	\pend% ending standard par
      ‚{\tiny $_{lb}$}‚

	  
	  \pstart \leavevmode% starting standard par
	\textbf{त‚त्रा}न‚न्त‚रोक्ते ज्ञा‚{\tiny $_{७}$}‚ने \textbf{एक इवे}ति स‚र्व‚व्य‚क्त्य‚नुग‚त इव । \textbf{त‚त्कारीवे}त्य‚र्थ- \leavevmode\ledsidenote{\textenglish{68a/PSVTa}}‚{\tiny $_{lb}$}‚ क्रियाकारीव । किं पुन‚स्त‚था प्र‚तिभातीति चेदाह । \textbf{व्य‚व‚हारिणा}मित्यादि ।‚{\tiny $_{lb}$}‚ \textbf{त‚थाध्य‚व‚सा}येति विक‚ल्पांश‚मेव \textbf{बाह्य‚त्वेनैक‚त्वेनार्थ‚क्रियाकारित्वे}नाध्य‚व‚साय‚{\tiny $_{lb}$}‚  ‚{\tiny $_{lb}$}‚ ‚{\tiny $_{lb}$}‚ \leavevmode\ledsidenote{\textenglish{182/s}}व्य‚व‚हारिणां \textbf{प्र‚वृत्तेः । व्य‚व‚हारिभि}रित्य‚न्ये प‚ठ‚न्ति । व्य‚व‚हारिभिरित्य‚ध्य‚व‚साये‚{\tiny $_{lb}$}‚त्य‚नेन पूर्व‚स‚म्ब‚न्धात् तृतीयैव कृता । न तु ष‚ष्ठी । \textbf{न लोकाव्य‚य‚निष्ठे}ति‚{\tiny $_{१}$}‚ \edtext{\textsuperscript{*}}{\edlabel{pvsvt_182-1}\label{pvsvt_182-1}\lemma{*}\Bfootnote{\href{http://sarit.indology.info/?cref=p\%C4\%81.2.3.69}{ Pāṇini 2:3:69. }}}‚{\tiny $_{lb}$}‚ष‚ष्ठी प्र‚तिषेधात् । \textbf{अन्य‚थेति} य‚दि विक‚ल्पांशे बाह्याध्य‚व‚सायो न भ‚वेत्त‚दा त‚था‚{\tiny $_{lb}$}‚भूते विक‚ल्पे जातेप्य‚र्थ‚क्रियाकारिणी \textbf{प्र‚वृत्तिर्न स्यात्} । त‚द‚पि विक‚ल्प‚प्र‚तिबिम्ब‚कं‚{\tiny $_{lb}$}‚ व्य‚व‚ह‚र्त्तृपुरुषाध्य‚व‚साय‚व‚शा\textbf{द‚र्थ‚क्रियाकारित}या \textbf{प्र‚तिभा}स‚ते । त‚त‚श्च \textbf{त‚द‚त‚त्का‚{\tiny $_{lb}$}‚रिभ्यो भिन्न‚मिव} । विक‚ल्प‚प्र‚तिविम्ब‚क‚मेव त‚त्त्वं क‚स्मान्नेति चेदाह । \textbf{न चे}‚{\tiny $_{lb}$}‚त्यादि । त‚द्विक‚ल्प‚प्र‚{\tiny $_{२}$}‚तिविम्ब‚क‚न्न \textbf{त‚त्त्वं} न व‚स्तु । किङ्कार‚णं [।] अन‚र्थ‚क्रिया‚{\tiny $_{lb}$}‚कारित्वेन प\textbf{रीक्षा}या व्य‚भिचार‚स्या\textbf{न‚ङ्ग‚त्वा}त् । एत‚च्चान‚न्त‚र‚मेव प्र‚तिपाद‚यि‚{\tiny $_{lb}$}‚ष्यामः ।
	{\color{gray}{\rmlatinfont\textsuperscript{§~\theparCount}}}
	\pend% ending standard par
      ‚{\tiny $_{lb}$}‚

	  
	  \pstart \leavevmode% starting standard par
	त‚त्र ये स्व‚ल‚क्षणद्वारा याता अर्थाकारा विक‚ल्प‚बुद्धौ प्र‚तिभान्ति \textbf{तेऽर्था}‚{\tiny $_{lb}$}‚ विक‚ल्प‚बुद्धिप्र‚तिभासिन‚स्\textbf{तेन} भेद‚ल‚क्ष‚णेन सामान्येन \textbf{स‚मा}ना \textbf{इति गृह्य‚न्ते । कुत‚श्चिद्‚{\tiny $_{lb}$}‚ व्यावृत्ता}इति विजातीय‚व्यावृत्त्या । त‚था हि [।] विक‚{\tiny $_{३}$}‚ल्प‚प्र‚तिभासिनोपि वृक्ष‚{\tiny $_{lb}$}‚भेदा अध्य‚व‚सित‚बाह्य‚रूप‚त्वाद् अवृक्षेभ्यो व्यावृत्ता इव भास‚न्ते । त‚थान्येति ।‚{\tiny $_{lb}$}‚ \textbf{न स्व‚ल‚क्ष‚ण}न्तेन स‚मान‚मिति गृह्य‚त इति लिङ्ग‚व‚च‚नविप‚रिणामेन स‚म्ब‚न्धः ।‚{\tiny $_{lb}$}‚ किङ्कार‚णं [।] \textbf{त‚त्र} सामान्य‚प्र‚तिभासिनि विक‚ल्पे स्व‚ल‚क्ष‚णा\textbf{प्र‚तिभास‚नात्} ।
	{\color{gray}{\rmlatinfont\textsuperscript{§~\theparCount}}}
	\pend% ending standard par
      ‚{\tiny $_{lb}$}‚

	  
	  \pstart \leavevmode% starting standard par
	एव‚न्ताव‚द् बुद्धिप्र‚तिभासिन्य‚र्थे सामान्य‚व्य‚व‚हार उक्तः । संप्र‚ति सामाना‚{\tiny $_{lb}$}‚धिक‚र‚ण्य‚व्य‚व‚हा‚{\tiny $_{४}$}‚र‚माह । \textbf{त एवे}त्यादि । \textbf{त एव} विक‚ल्प‚प्र‚तिभासिनोर्थाः । \textbf{कुत‚श्चिद्‚{\tiny $_{lb}$}‚ व्यावृत्ता} इव स‚न्तो य‚थानुत्प‚लाद् व्यावृत्ता उत्प‚ल‚भेदास्त एव पुन\textbf{र‚न्य‚तोप्य}नीलाद्‚{\tiny $_{lb}$}‚ \textbf{व्यावृत्तिम‚न्तः प्र‚तिभान्ति} [।] त‚त‚श्च व्यावृत्तिद्व‚यानुग‚त‚स्यैक‚स्यैव ध‚र्मिणः प्र‚ति‚{\tiny $_{lb}$}‚भास‚नात् सामानाधिक‚र‚ण्यं ।
	{\color{gray}{\rmlatinfont\textsuperscript{§~\theparCount}}}
	\pend% ending standard par
      ‚{\tiny $_{lb}$}‚

	  
	  \pstart \leavevmode% starting standard par
	अयं चान‚न्त‚रानुक्रान्तो बुद्धिप्र‚तिभासिष्व‚र्थेषु सामान्य‚सामानाधिक‚र‚ण्य‚{\tiny $_{lb}$}‚व्य‚व‚हा‚{\tiny $_{५}$}‚रो मिथ्यार्थ एव क्रिय‚ते । किं कार‚णं [।] \textbf{स्व‚य‚म‚स‚ताम}पि विक‚ल्पाकारा‚{\tiny $_{lb}$}‚णान्त‚था एकाकारानुग‚त‚त्वेन । व्यावृत्तिद्व‚यानुग‚तेन ध‚र्म्म‚स्व‚रूपेण । विक‚ल्प\textbf{बुद्ध्योप‚{\tiny $_{lb}$}‚‚{\tiny $_{lb}$}‚ ‚{\tiny $_{lb}$}‚ \leavevmode\ledsidenote{\textenglish{183/s}}द‚र्श‚नात्} । एकाकारेण प्र‚तिभास‚नात् \textbf{सामा}न्य‚व्य‚व‚हारः । अनेकाकारेण‚{\tiny $_{lb}$}‚ चैक‚स्य प्र‚तिभास‚नात् \textbf{सामानाधिक‚र‚ण्य‚व्य‚व‚हारः} ।
	{\color{gray}{\rmlatinfont\textsuperscript{§~\theparCount}}}
	\pend% ending standard par
      ‚{\tiny $_{lb}$}‚

	  
	  \pstart \leavevmode% starting standard par
	य‚दि मिथ्यार्थ एव स‚र्वो विक‚ल्पः क‚स्मात् कृत‚क‚त्वादि‚{\tiny $_{६}$}‚द्वारायाता अनित्या‚{\tiny $_{lb}$}‚नात्मादिविक‚ल्पाः प्र‚माणं नित्या विक‚ल्पास्तु नेत्य‚त आह । \textbf{स‚र्व‚श्चाय}मित्यादि ।‚{\tiny $_{lb}$}‚ \textbf{स‚र्वो विप्ल‚व} इति स‚म्ब‚न्धः । \textbf{विप्ल‚वो} भ्रान्तिः । \textbf{अय}मिति सामान्यादिरूपः ।‚{\tiny $_{lb}$}‚ \textbf{स्व‚ल‚क्ष‚णानामेव} य‚द्द‚र्श‚न‚न्तेनाहिता या वास‚ना त‚त्कृतः । प‚र‚म्प‚र‚या स‚र्व‚विक‚ल्पा‚{\tiny $_{lb}$}‚नाम्व‚स्तुद‚र्श‚न‚द्वारायात‚त्वात् । त‚था हि नित्यादिविक‚ल्पा अपि‚{\tiny $_{७}$}‚ व‚स्तुद‚र्श‚नेनैवो- \leavevmode\ledsidenote{\textenglish{68b/PSVTa}}‚{\tiny $_{lb}$}‚ त्प‚न्नाः स‚दृशाप‚राप‚रोत्प‚त्तिद‚र्श‚नायात‚त्वात् । त‚त्र तुल्ये स‚र्व‚विक‚ल्पानाम्व‚{\tiny $_{lb}$}‚स्तुद‚र्श‚न‚द्वारायात‚त्वे । \textbf{त‚त्प्र‚तिब‚द्ध‚ज‚न्म‚नाम}नित्यादि\textbf{विक‚ल्पानाम‚त‚त्प्र‚तिभासित्वेपि}‚{\tiny $_{lb}$}‚ स्व‚ल‚क्ष‚णाप्र‚तिभासित्वेपि \textbf{व‚स्तुन्य}विस‚म्वादः । अध्य‚स्त‚स्यानित्यादिरूप‚स्य व‚स्तुनि‚{\tiny $_{lb}$}‚ विद्य‚मान‚त्वात् केव‚लं स्व‚ल‚क्ष‚ण‚रूपेण न प्र‚तिभास‚त इति विक‚ल्पो‚{\tiny $_{१}$}‚ विभ्र‚म‚{\tiny $_{lb}$}‚ उच्य‚ते । \textbf{म‚णिप्र‚भायामिव म‚णिभ्रान्ते}र्म‚णिस्व‚रूपाग्र‚हेप्य‚विस‚म्वादो म‚णिप्र‚भाया‚{\tiny $_{lb}$}‚ म‚णौ प्र‚तिब‚द्ध‚त्वात् । प्र‚भाश्र‚येण च म‚णिभ्रान्तेरुत्प‚त्तेः ।
	{\color{gray}{\rmlatinfont\textsuperscript{§~\theparCount}}}
	\pend% ending standard par
      ‚{\tiny $_{lb}$}‚

	  
	  \pstart \leavevmode% starting standard par
	न त्वेव‚न्नित्यादिविक‚ल्पास्तेषाम्व‚स्तुद‚र्श‚न‚द्वारायात‚त्वेपि व‚स्तुन्य‚विद्य‚मान‚{\tiny $_{lb}$}‚स्यैवाकार‚स्य स‚मारोपात् । त‚दाह । \textbf{नान्येषा}मित्यादि । अन्येषां नित्यादि‚{\tiny $_{lb}$}‚विक‚ल्पानां व‚स्तुनि स‚म्वाद इत्य‚ने‚{\tiny $_{२}$}‚न स‚म्ब‚न्धः । \textbf{त‚द्भेद‚प्र‚भ‚वे स‚त्य‚पी}ति । अर्था‚{\tiny $_{lb}$}‚भेदाद् उत्पादेपि स‚तीत्य‚र्थः । \textbf{य‚था दृष्टो} यो \textbf{विशेषः} क्ष‚णिक‚त्वादिल‚क्ष‚ण‚स्त‚स्या‚{\tiny $_{lb}$}‚\textbf{नुस‚र‚णं} निश्च‚यं \textbf{प‚रित्य‚ज्य किञ्चित्सामान्य}मिति व्य‚तिरिक्त‚स्याव्य‚तिरिक्त‚स्य‚{\tiny $_{lb}$}‚ वा सामान्य‚स्य \textbf{ग्र‚ह‚णेन} विशेषात्त‚स्य स्थिर‚त्वादेः \textbf{स‚मारोपात् । दीप‚प्र‚भायामिव}‚{\tiny $_{३}$}‚‚{\tiny $_{lb}$}‚ भासुर‚त्वादिसाम्यात् प्र‚वृत्ताया \textbf{म‚णिबुद्धेर्न} म‚णिव‚स्तुस‚म्वादः । पार‚म्प‚र्येणाप्य‚{\tiny $_{lb}$}‚ध्य‚व‚सिते म‚णाव‚प्र‚तिब‚द्ध‚त्वात् । य‚त‚श्च मिथ्यार्था एव विक‚ल्पास्\textbf{तेन न विक‚ल्प‚{\tiny $_{lb}$}‚विष‚येष्व‚र्थेष्व‚र्थ‚क्रियाकारित्वं} [।] त‚त‚श्च य‚दुक्त‚म‚न्य‚तो वा क‚थ‚म‚र्थ‚क्रियेति त‚त्सिद्धं‚{\tiny $_{lb}$}‚ साध्य‚ते ।
	{\color{gray}{\rmlatinfont\textsuperscript{§~\theparCount}}}
	\pend% ending standard par
      ‚{\tiny $_{lb}$}‚

	  
	  \pstart \leavevmode% starting standard par
	क‚थ‚न्त‚र्ह्य‚त‚त्कारिव्य‚व‚च्छेद‚ल‚क्ष‚णं सामान्यं विक‚ल्प‚विष‚येष्व‚र्थेषु व्य‚व‚स्थाप्य‚त‚{\tiny $_{४}$}‚‚{\tiny $_{lb}$}‚ इति चेत् । न । ब‚हिरिव प‚रिस्फुर‚तामेकार्थ‚क्रियाकारित‚या त‚द‚कारिभ्यो भिन्ना‚{\tiny $_{lb}$}‚‚{\tiny $_{lb}$}‚ \leavevmode\ledsidenote{\textenglish{184/s}}नामिव प्र‚तिभास‚नात् । य‚च्चोक्तं [।] स्व‚ल‚क्ष‚णे चानित्य‚त्वाद्य‚प्र‚तीतेर‚ताद्रूप्य‚{\tiny $_{lb}$}‚मिति त‚त्प‚रिहारार्थ‚माह । \textbf{नापीत्यादि । च‚लाद् व‚स्तुनो} य‚स्मा\textbf{न्नानित्य‚त्व‚न्नाम‚{\tiny $_{lb}$}‚ किञ्च‚द‚स्ति} । येनास‚म्ब‚द्धात् स्व‚ल‚क्ष‚ण‚स्यानित्य‚त्वेनायोगः स्यात् । किन्तु च‚ल‚मेव‚{\tiny $_{lb}$}‚ व‚स्तु नित्यं स्व‚{\tiny $_{५}$}‚ल‚क्ष‚ण‚स्यैवानित्य‚रूप‚त्वादेव‚म‚नात्माद्य‚पि द्र‚ष्ट‚व्यं । तेन प्र‚त्य‚क्षेण‚{\tiny $_{lb}$}‚ स्व‚ल‚क्ष‚णे गृह्य‚माणेऽनित्य‚त्वं गृहीत‚मेव केव‚लं भ्रान्तिनिमित्त‚स‚द्भावाद‚निश्चित‚म्‚{\tiny $_{lb}$}‚ [।] अत‚स्त‚न्निश्च‚य‚मात्रेऽनुमान‚व्यापार‚स्तेन त‚न्निश्च‚य एव स्व‚ल‚क्ष‚णेऽनित्य‚त्व‚{\tiny $_{lb}$}‚प्र‚तीतिरिति सिद्धं ।
	{\color{gray}{\rmlatinfont\textsuperscript{§~\theparCount}}}
	\pend% ending standard par
      ‚{\tiny $_{lb}$}‚

	  
	  \pstart \leavevmode% starting standard par
	य‚दि स्व‚ल‚क्ष‚ण‚मेवानित्यं क‚थ‚म‚नित्योय‚म‚र्थोऽनित्य‚त्व‚म‚स्येति वा ध‚र्मिध‚र्म‚{\tiny $_{lb}$}‚रूप‚त‚या‚{\tiny $_{६}$}‚ प्र‚तीतिरित्य‚त आह । \textbf{क्ष‚णे}त्यादि । स्व‚ल‚क्ष‚ण‚स्य \textbf{त‚थाभूत‚स्ये}ति च‚ल‚{\tiny $_{lb}$}‚रूप‚स्य क्ष‚णाप्र‚त्युप‚स्थान‚त‚या । एक\textbf{क्ष‚ण‚स्थायित्वेन ग्र‚ह‚णाद्} उत्त‚र‚काल‚म‚न्त्य‚क्ष‚ण‚{\tiny $_{lb}$}‚द‚र्शिनामेत‚देव‚म्भ‚व‚त्य‚नित्योय‚मित्यादि । भेदान्त‚राप्र‚तिक्षेप‚विव‚क्षाया\textbf{म‚नित्योय‚मिति}‚{\tiny $_{lb}$}‚ \leavevmode\ledsidenote{\textenglish{69a/PSVTa}} भेदान्त‚र‚प्र‚तिक्षेप‚विव‚क्षाया\textbf{म‚नित्य‚त्व‚म‚स्येत्ये}वं ध‚र्मिध‚र्म‚भाव‚{\tiny $_{७}$}‚प्र‚तीतिर्भ‚व‚ति ।
	{\color{gray}{\rmlatinfont\textsuperscript{§~\theparCount}}}
	\pend% ending standard par
      ‚{\tiny $_{lb}$}‚

	  
	  \pstart \leavevmode% starting standard par
	विक‚ल्प‚क‚ल्पित‚त्वात् क‚थं बाह्ये ध‚र्म‚ध‚र्मिभाव इत्य‚त आह । \textbf{त‚द्ध‚र्म‚ता}मित्यादि ।‚{\tiny $_{lb}$}‚ त‚द्ध‚र्म‚तां स्व‚ल‚क्ष‚ण‚ध‚र्म‚तामे\textbf{वाव‚त‚र‚न्तः} स्व‚ल‚क्ष‚ण‚म‚ध्य‚व‚स्य‚न्तो \textbf{विक‚ल्पा} इत्य‚र्थः ।‚{\tiny $_{lb}$}‚ व्यावृत्तिभेदे कृत‚संकेत‚श‚ब्दानुसारेण \textbf{नाना}रूपा \textbf{एक}रूपाश्च \textbf{ध‚र्मा}स्ते च \textbf{व्य‚तिरे}‚{\tiny $_{lb}$}‚काश्चेति द्व‚न्द्वः । नानाध‚र्मान् अनित्य‚कृत‚क‚त्वादीन् । एकं ध‚र्मं ब‚हूनां घ‚टादीना‚{\tiny $_{lb}$}‚म‚नित्य‚त्वं‚{\tiny $_{१}$}‚ व्य‚तिरेक‚श्च घ‚टादीनाम‚नित्य‚त्व‚मिति \textbf{द‚र्श‚य}न्ति । व‚स्तुनीत्य‚ध्या‚{\tiny $_{lb}$}‚हारः । न च विक‚ल्प‚व्य‚व‚स्थांपित‚न्नानैक‚ध‚र्मादिक‚न्त‚त्त्व‚म्विक‚ल्प‚स्याव‚स्तुग्राहि‚{\tiny $_{lb}$}‚त्वाद् [।] अत एवाह [।] \textbf{द‚र्श‚य‚न्ती}ति ।
	{\color{gray}{\rmlatinfont\textsuperscript{§~\theparCount}}}
	\pend% ending standard par
      ‚{\tiny $_{lb}$}‚

	  
	  \pstart \leavevmode% starting standard par
	अव‚स्तुग्राहित्वात्त‚र्हि \textbf{ते निराश्र‚याः} प्राप्नुव‚न्तीति चेदाह । \textbf{न च त} इति । न‚{\tiny $_{lb}$}‚ इति नानाध‚र्मादिद‚र्श‚का विक‚ल्प‚व‚स्तु\textbf{भे}द‚स्यानित्यादिरूप‚स्य स्व‚ल‚क्ष‚ण‚स्य य‚द्‚{\tiny $_{lb}$}‚ \textbf{द‚र्श}न‚म‚नुभ‚व‚{\tiny $_{२}$}‚स्त‚दा\textbf{श्र‚य‚त्वाद्} विक‚ल्पानां । त‚था हि प‚र‚मार्थ‚तोऽनित्यादिरूपं‚{\tiny $_{lb}$}‚ स्व‚ल‚क्ष‚णं दृष्ट्वा द‚र्श‚न‚साम‚र्थ्य‚भाविनो विक‚ल्पा दृष्टाकाराध्य‚व‚सायेन प्र‚व‚र्त्त‚न्ते ।‚{\tiny $_{lb}$}‚ य‚त‚श्च य‚थादृष्ट‚स्यैवाभिल‚प‚नेन प्र‚व‚र्त्त‚न्ते विक‚ल्पा अनित्याकारा नार्थान्त‚र‚न्नित्य‚{\tiny $_{lb}$}‚त्वादिविक‚ल्प‚व‚द‚नुस‚र‚न्ति । त‚तो य‚दुक्तं [।] तेषां चाव‚स्तु ध‚र्म‚तेति प‚रिहृत‚म्भ‚{\tiny $_{lb}$}‚व‚तीत्याह । \textbf{ने}ति । ते‚{\tiny $_{३}$}‚षाम‚नित्य‚त्वादीनां \textbf{नाव‚स्तुध‚र्म‚ता} । किं कार‚णं । \textbf{त‚त्स्व‚भा‚{\tiny $_{lb}$}‚व‚स्यैव त‚था}ऽनित्यादिध‚र्म‚त‚या ख्यातेः \textbf{प्र‚तिभास‚ना}द‚ध्य‚व‚सायादिति याव‚त् । य‚दि‚{\tiny $_{lb}$}‚ ‚{\tiny $_{lb}$}‚ \leavevmode\ledsidenote{\textenglish{185/s}}व‚स्तुध‚र्म एवानित्य‚त्वादिकं ख्याति । क‚स्त‚र्हि विक‚ल्प‚कृतो विभ्र‚म इत्याह ।‚{\tiny $_{lb}$}‚ \textbf{व‚स्तुन‚स्त्वि}त्यादि । एक‚स्य व‚स्तुनो \textbf{नाना}रूपेण ग्र‚हः । ब‚हूनां \textbf{चैक}त्वेन । ध‚र्म‚ध‚{\tiny $_{lb}$}‚र्मिणोश्च \textbf{व्य‚तिरेकेण‚{\tiny $_{४}$}‚ ग्र‚हो विभ्र‚मो} भ्रान्त इत्य‚र्थः । किं पुनः कार‚ण‚मेक‚त्वादिग्र‚हो‚{\tiny $_{lb}$}‚ विभ्र‚म इत्याह । \textbf{त‚स्यैकाने}केत्यादि । एक‚कार्य‚कारिणो घ‚टादिभेद‚स्यैकोद‚काद्याह‚{\tiny $_{lb}$}‚र‚णादिकार्य‚कारिणः । \textbf{त‚था भाव‚जिज्ञासा}सु । एक‚कार्य‚क‚र्त्तृत्व‚जिज्ञासासु । \textbf{त‚थाभा‚{\tiny $_{lb}$}‚व‚ख्याप‚ना}यैक‚कार्य‚कारित्व‚ख्याप‚नाय । त‚थाकृत‚स्थितित्वात् । घ‚टादिना‚{\tiny $_{५}$}‚ एक‚रूपेण‚{\tiny $_{lb}$}‚ व्य‚व‚हार‚लाघ‚वार्थ‚म्व्य‚व‚स्थापित‚त्वात् । त‚थ‚क‚स्याप्य‚नेकार्य‚कारिणः । य‚था घ‚ट‚स्य‚{\tiny $_{lb}$}‚ च‚क्षुर्विज्ञानोद‚क‚धार‚ण‚कादाचित्क‚ज्ञानादिकार्य‚कारिण‚स्त‚थाभाव‚जिज्ञासासु त‚थाभा‚{\tiny $_{lb}$}‚व‚ख्याप‚नाय । अनेक‚कार्य‚त्व‚ख्याप‚नाय त‚थाकृत‚स्थित‚त्वात् । चाक्षुष‚पार्थिवानित्या‚{\tiny $_{lb}$}‚दिरूपेण \textbf{व्य‚व‚स्थापित‚त्वा}त्‚{\tiny $_{६}$}‚ । एव‚म‚न्य‚त्रापि य‚थायोगं वाच्यं । \textbf{न व‚स्तुभेदाद्}‚{\tiny $_{lb}$}‚ एक‚स्मिन् प‚दार्थेऽनेक‚ध‚र्म‚व्य‚व‚स्थाप‚नं । किङ्कार‚णं [।] \textbf{त‚स्यैवैक}स्य व‚स्तुनोऽ\textbf{नेक‚{\tiny $_{lb}$}‚त्वायोगा}त् [।] त‚त‚श्चैक‚स्यानेक‚त्व‚ग्र‚हो विभ्र‚म इत्याख्यातं । त‚थानेक‚स्याप्येक‚{\tiny $_{lb}$}‚त्व‚व्य‚व‚स्थाप‚न‚न्त‚द‚कार्य‚व्यावृत्तिद्वारेणैव न व‚स्त्व‚भेदादित्याह । \textbf{अनेक‚स्य चैक‚त्वा‚{\tiny $_{lb}$}‚योगा}दिति‚{\tiny $_{७}$}‚ । त‚था चानेक‚स्यैक‚त्व‚ग्र‚हो विभ्र‚मः । एवं ध‚र्म‚ध‚र्मिणोर्व्य‚तिरेक‚ग्र‚होपि \leavevmode\ledsidenote{\textenglish{69b/PSVTa}}‚{\tiny $_{lb}$}‚ भ्रान्त एव । किङ्कार‚णं [।] \textbf{व्य‚तिरिक्त}स्य च सामान्य‚स्य प्रागेव \textbf{निषेधात् ।‚{\tiny $_{lb}$}‚ व्य‚क्त‚यो नानुय‚न्त्य‚न्य‚दित्यादिना} ।
	{\color{gray}{\rmlatinfont\textsuperscript{§~\theparCount}}}
	\pend% ending standard par
      ‚{\tiny $_{lb}$}‚

	  
	  \pstart \leavevmode% starting standard par
	ध‚र्मिणः स‚काशाद् व्य‚तिरिक्ता एव ध‚र्मास्त‚त‚श्च य‚थाव‚स्तु श‚ब्दार्थो‚{\tiny $_{lb}$}‚ भ‚विष्य‚तीत्याह । \textbf{तेषा}मित्यादि । \textbf{तेषा}म‚नित्य‚त्वादीनां ध‚र्माणां \textbf{प्र‚कृतेः} स्व‚भाव‚स्य‚{\tiny $_{lb}$}‚ \textbf{भेदात्} कार‚णाद् \textbf{य‚थाव‚स्तु श‚ब्दा‚{\tiny $_{१}$}‚र्थाभ्युप‚ग‚मे} नीलोत्प‚ल‚म‚नित्यः श‚ब्द इत्यादि‚{\tiny $_{lb}$}‚ \textbf{सामानाधिक‚र‚ण्यायोगात्} । नीलादिगुणानामुत्प‚लादिजातीनाञ्च प‚र‚स्प‚रं स्व‚भाव‚{\tiny $_{lb}$}‚भेदात् । त‚द्वाचिनां श‚ब्दानामेक‚स्मिन्न‚धिक‚र‚णे वृत्तिर्नास्तीति सामानाधिक‚र‚ण्या‚{\tiny $_{lb}$}‚योगः । त‚स्मान्न य‚थाव‚स्तु श‚ब्दार्थ‚व्य‚व‚स्थेति भावः ।
	{\color{gray}{\rmlatinfont\textsuperscript{§~\theparCount}}}
	\pend% ending standard par
      ‚{\tiny $_{lb}$}‚‚{\tiny $_{lb}$}‚‚{\tiny $_{lb}$}‚\textsuperscript{\textenglish{186/s}}

	  
	  \pstart \leavevmode% starting standard par
	\textbf{त‚दुपाधे}रित्यादिना प‚राभिप्राय‚माशंक‚ते । ते नीलाद‚यो ध‚र्मा उपाधिर्य‚स्य‚{\tiny $_{lb}$}‚ द्र‚व्य‚{\tiny $_{२}$}‚स्य त‚स्यै\textbf{क‚स्य} द्वाभ्यां गुण‚जातिभ्याम\textbf{भिधाना}द् \textbf{अदोषः} । सामानाधिक‚र‚ण्या‚{\tiny $_{lb}$}‚भाव‚दोषो नास्ति । उत्त‚र‚माह [।] \textbf{अनुप‚कारिणी}त्यादि । उपाधित्वेनाभिम‚तानां‚{\tiny $_{lb}$}‚ नीलादीनां ध‚र्माणाम‚नुप‚कार‚क‚न्द्र‚व्यं । न हि नील‚गुण‚स्योत्प‚ल‚त्व‚जातेर्वा द्र‚व्येणोप‚कारः‚{\tiny $_{lb}$}‚ क‚श्चिद् क्रिय‚ते [।] त‚त‚श्चा\textbf{नुप‚कारिणि} द्र‚व्ये त‚द‚नाधेय‚वृत्तीनां नीलादीनां‚{\tiny $_{lb}$}‚ \textbf{पार‚त‚न्त्र्यायोगा}त् । नीलाद‚यो‚{\tiny $_{३}$}‚ ध‚र्मा अनुपाधिः । पार‚त‚न्त्र्याभावात् । अन्य‚थाऽस्याय‚{\tiny $_{lb}$}‚मुपाधिरित्येव न स्यात् । अथेष्य‚ते द्र‚व्य‚विष‚यं पार‚त‚न्त्र्यं ध‚र्माणान्त‚दा \textbf{पार‚त‚न्त्र्येपि}‚{\tiny $_{lb}$}‚ द्र‚व्य\textbf{ज‚न्य}त्व‚म‚ङ्गीक‚र्त्त‚व्य‚म‚न्य‚था द्र‚व्य‚पार‚त‚न्त्र्यायोगात् । \textbf{ज‚न‚कं} च क्ष‚णिक‚मेष्ट‚व्य‚म‚{\tiny $_{lb}$}‚क्ष‚णिक‚स्यार्थ‚क्रियायोगात् । त‚त‚श्च कार्याभिम‚तानामुपाधीनां कार‚णाभिम‚त‚स्य च‚{\tiny $_{lb}$}‚ द्र‚व्य‚स्य \textbf{स‚हान‚व‚स्थितेः}‚{\tiny $_{४}$}‚ कार‚णात् । \textbf{द्व‚यो}र्विशेष‚ण‚विशेष्य‚योर्व‚स्तुरूप‚योर्युग‚प\textbf{द‚न‚भिधानं}‚{\tiny $_{lb}$}‚ [।] कार‚णाभिम‚त‚स्य विशेष‚स्य त‚दानीं निरोधात् । \textbf{एक‚स्ये}ति विशिष्ट‚स्याप्युपा‚{\tiny $_{lb}$}‚धिम‚तः । \textbf{अध्याहार} उप‚द‚र्श‚न‚न्त‚दा \textbf{न व‚स्तुविष‚यः श‚ब्दार्थः स्यात्} । बुद्ध्यारोपित‚स्यैव‚{\tiny $_{lb}$}‚ विशेष‚स्य श‚ब्देनाभिधानात् ।
	{\color{gray}{\rmlatinfont\textsuperscript{§~\theparCount}}}
	\pend% ending standard par
      ‚{\tiny $_{lb}$}‚

	  
	  \pstart \leavevmode% starting standard par
	स्यान्म‚तं [।] य‚द्विन‚ष्टं विशेष्य‚न्त‚द्विष‚य‚स्य श‚ब्द‚स्य भ‚व‚तु बुद्धिप्र‚ति‚{\tiny $_{५}$}‚भास‚{\tiny $_{lb}$}‚विष‚य‚त्वं [।] यः पुनः स‚न्नेवोपाधिस्त‚द्वाचिनः श‚ब्द‚स्य व‚स्तुविष‚य‚त्व‚मेवास्त्विति‚{\tiny $_{lb}$}‚ चेदाह । \textbf{बुद्धिप्र‚तिभासे}त्यादि । बुद्धिप्र‚तिभासो विष‚यो य‚स्याभिधान‚स्य त‚त्त‚था‚{\tiny $_{lb}$}‚ [।] त‚द्भाव‚स्त‚स्मिन् स‚ति \textbf{स‚र्वं} विशेष‚ण‚विष‚याभिम‚त‚म‚प्य‚भिधान‚न्त\textbf{थैव}‚{\tiny $_{lb}$}‚ विक‚ल्प‚बुद्धिप्र‚तिभास‚विष‚य‚मेवास्तु । किं कार‚णं [।] \textbf{त‚था भिन्नोपाधिम‚तो}‚{\tiny $_{lb}$}‚ नानाविशेष‚ण‚व‚त \textbf{एक‚{\tiny $_{६}$}‚स्य ग्र‚ह‚णे} बुद्धाव‚भा\textbf{स‚नात्} । त‚था ह्युपाधिम‚तो विन‚ष्ट‚{\tiny $_{lb}$}‚स्याध्याहारिका विक‚ल्प‚बुद्धिर‚ङ्गीक‚र्त्त‚व्या [।] त‚दा चोपाधिम‚तोऽभावे उपाधे‚{\tiny $_{lb}$}‚र‚प्य‚भावः पार‚त‚न्त्र्याभावात् [।] \textbf{भिन्नोपाधिम‚त एक‚स्याप्र‚तिभा}स‚ने कुतः‚{\tiny $_{lb}$}‚ सामानाधिक‚र‚ण्यं ।
	{\color{gray}{\rmlatinfont\textsuperscript{§~\theparCount}}}
	\pend% ending standard par
      ‚{\tiny $_{lb}$}‚

	  
	  \pstart \leavevmode% starting standard par
	य‚दा तु विशेष‚ण‚विशेष्य‚योर्द्व‚योर‚पि विक‚ल्प‚बुद्धिप्र‚तिभासित्व‚मिष्ट‚न्त‚दा‚{\tiny $_{lb}$}‚ \leavevmode\ledsidenote{\textenglish{70a/PSVTa}} क‚ल्पित ध‚र्म‚द्व‚य‚गृहीतैक‚ध‚{\tiny $_{७}$}‚र्मिप्र‚तिभासिन्येकैव बुद्धिर्जाय‚त इत्य‚विरुद्धं सामाना‚{\tiny $_{lb}$}‚धिक‚र‚ण्यं ।
	{\color{gray}{\rmlatinfont\textsuperscript{§~\theparCount}}}
	\pend% ending standard par
      ‚{\tiny $_{lb}$}‚

	  
	  \pstart \leavevmode% starting standard par
	\textbf{उप‚कार्येत्या}दिना प‚राभिप्राय‚माशंक‚ते । \textbf{अदोषो} योय‚मेक‚स्य बुद्ध्याध्याहार‚{\tiny $_{lb}$}‚ इत्यादिनोक्तः । \textbf{उपाधिम‚ता स‚म‚काल}स्य निष्प‚न्न‚रूप‚स्योपाधेः \textbf{पार‚त‚न्त्र्याभावा‚{\tiny $_{lb}$}‚द‚नुपाधित्वं} । प्राक् पार‚त‚न्त्र्य‚न्त‚देवोपाधित्व‚मिति चेदाह । \textbf{ने}त्यादि । न ह्य‚{\tiny $_{lb}$}‚‚{\tiny $_{lb}$}‚ \leavevmode\ledsidenote{\textenglish{187/s}}निष्प‚न्न‚स्य श‚श‚विषा‚{\tiny $_{१}$}‚ण‚तुल्य‚स्य पार‚त‚न्त्र्य‚मुपाधित्व‚म्वा [।] स‚र्व‚था स‚तोऽस‚त‚{\tiny $_{lb}$}‚श्चास‚त्पार‚त‚न्त्र्य‚मिति हेतोः क‚ल्प‚नारोपित‚मुपाधीनाम्पार‚त‚न्त्र्यं कृत्वा पार‚त‚न्त्र्य‚{\tiny $_{lb}$}‚व्य‚व‚हारे । \textbf{स‚र्व‚थे}ति स‚र्वेण प्र‚कारेण य‚था पार‚त‚न्त्र्य\textbf{व्य‚व‚हारे} बुद्धिर‚नुविधीय‚ते‚{\tiny $_{lb}$}‚ त‚था विशेष‚ण‚विशेष्य‚व्य‚व‚हारे \textbf{सैवा}रोपिका \textbf{बुद्धिः किन्नानुविधी}य‚ते । त‚द‚नुविधानं‚{\tiny $_{lb}$}‚ हि न्याय्यं बुद्धिस‚न्द‚र्शितार्थ‚प्र‚तिभास‚{\tiny $_{२}$}‚म‚नाश्रित्य व्य‚व‚ह‚र्त्तुम‚श‚क्य‚त्वात् । \textbf{व‚स्तुन‚{\tiny $_{lb}$}‚ एकेन श‚ब्देन प्र‚माणेन च विष‚यीक‚र‚णे व‚स्तुब‚ला}द्व‚स्तुस्व‚भाव‚भूता\textbf{शेष‚ध‚र्माक्षेपात्‚{\tiny $_{lb}$}‚ त‚द‚न्य‚स्य} श‚ब्दादेर्वै\textbf{य‚र्थ्यं च} य‚त्प्रागुक्त‚मेक‚स्यार्थ‚स्व‚भाव‚स्य प्र‚त्य‚क्ष‚स्येत्यादि ।
	{\color{gray}{\rmlatinfont\textsuperscript{§~\theparCount}}}
	\pend% ending standard par
      ‚{\tiny $_{lb}$}‚

	  
	  \pstart \leavevmode% starting standard par
	व स्तु वा दि प‚क्षे त‚द्बुद्धिप्र‚तिभासानुरोधेन स्यात् । किं कार‚णं [।] \textbf{बुद्धिप्र‚ति‚{\tiny $_{lb}$}‚भास‚स्य निर्व‚स्तुक‚त्वाद् व‚स्तुसाम‚र्थ्य‚भाविना}न्दोषाणाम\textbf{प्र‚स‚{\tiny $_{३}$}‚ङ्गः} । ब‚हुव‚च‚नेनैत‚दाह ।‚{\tiny $_{lb}$}‚ न केव‚लं त‚द‚न्य‚वैय‚र्थ्य‚दोष‚स्याप्र‚स‚ङ्ग‚स्त‚था व्य‚तिरिक्त‚स्य सामान्य‚स्याभाव उपाधीनां‚{\tiny $_{lb}$}‚ च पार‚त‚न्त्र्यायोगात् ।
	{\color{gray}{\rmlatinfont\textsuperscript{§~\theparCount}}}
	\pend% ending standard par
      ‚{\tiny $_{lb}$}‚

	  
	  \pstart \leavevmode% starting standard par
	विशेष‚ण‚विशेष्य‚त्वाभावः भिन्नोपाधिम‚त एक‚स्य बुद्धाव‚प्र‚तिभास‚नात् [।]‚{\tiny $_{lb}$}‚ सामानाधिक‚र‚ण्याभाव‚श्च य उक्त‚स्तेषाम‚प्य‚त्र स‚म्भ‚वो नास्तीति त‚देवाह । \textbf{त‚द‚भि‚{\tiny $_{lb}$}‚न्न‚मि}त्यादि । त‚था हि त‚द्बुद्धिप्र‚तिभासि रू‚{\tiny $_{४}$}‚प‚म‚भिन्न‚म्प्र‚तिभाति । त‚स्मात् सामान्यं‚{\tiny $_{lb}$}‚ य‚था प्र‚तीतिर्न विरुध्य‚त इति व‚च‚न‚प‚रिणामं कृत्वा व‚क्ष्य‚माणेन स‚म्ब‚न्धः । नील‚{\tiny $_{lb}$}‚मित्युक्तेऽनील‚व्यावृत्त्या नील‚त्व‚स्यै\textbf{क‚स्याकार‚स्य विष‚यीक‚र‚णेपि} विक‚ल्प‚बुद्ध्या‚{\tiny $_{lb}$}‚ त‚त्रैव नीलाकारे संश‚य‚व्यावृत्ति\textbf{र्नाकारान्त‚रे} [।] त‚त‚स्त‚द्बुद्धिप्र‚तिभासि रूप‚म‚नि‚{\tiny $_{lb}$}‚श्चिताद्याकार‚म‚नुत्प‚ल‚व्यावृत्तोत्प‚लाकार‚स्या\textbf{निश्च‚या}त्‚{\tiny $_{५}$}‚ । य‚स्मिन्नाकारे निश्च‚यो‚{\tiny $_{lb}$}‚ नोत्प‚न्न‚स्त\textbf{दाकारान्त}र‚न्त‚त्र \textbf{साकांक्ष}योत्प‚ल‚श‚ब्द‚प्र‚योगाद् उत्प‚न्न‚या \textbf{बुद्ध्या ग्राह्यं}‚{\tiny $_{lb}$}‚ प्र‚तिभातीति स‚म्ब‚न्धः ।
	{\color{gray}{\rmlatinfont\textsuperscript{§~\theparCount}}}
	\pend% ending standard par
      ‚{\tiny $_{lb}$}‚

	  
	  \pstart \leavevmode% starting standard par
	एतेन विशेष‚ण‚विशेष्य‚भाव‚स्य निमित्त‚मुक्तं [।] सामानाधिक‚र‚ण‚स्याह ।‚{\tiny $_{lb}$}‚ \textbf{भिन्ने}त्यादि । \textbf{भिन्न‚स्य श‚ब्दार्थ‚स्य} नीलोत्प‚ल‚ल‚क्ष‚ण‚स्योप‚संहारे \textbf{प्र‚तिपाद‚नेपि} ।‚{\tiny $_{lb}$}‚ ‚{\tiny $_{lb}$}‚ \leavevmode\ledsidenote{\textenglish{188/s}}ध‚र्म‚द्व‚योप‚गृहीत\textbf{म‚भिन्न‚मे}क‚ध‚र्मित‚या बुद्धिप्र‚तिभास‚रू‚{\tiny $_{६}$}‚पं विक‚ल्प\textbf{बुद्धौ प्र‚तिभातीति}‚{\tiny $_{lb}$}‚ कृत्वा \textbf{सामान्या}दीनि \textbf{य‚थाप्र‚तीति न विरुध्य‚न्ते} ।
	{\color{gray}{\rmlatinfont\textsuperscript{§~\theparCount}}}
	\pend% ending standard par
      ‚{\tiny $_{lb}$}‚

	  
	  \pstart \leavevmode% starting standard par
	तेन य‚दुच्य‚ते भ ट्टे न [।] अन्य‚निवृत्तिमात्र‚म‚पोहं गृहीत्वा अनीलादि‚{\tiny $_{lb}$}‚व्यावृत्ताव‚नुत्प‚लादिव्यावृत्तेर‚भावाद‚नुत्प‚लादिव्यावृत्तौ चानीलादिव्यावृत्तेर‚भावाच्च‚{\tiny $_{lb}$}‚ न विशेष‚ण‚विशेष्य‚भाव‚स्त‚योर्ध‚र्म‚योस्त‚द‚भावाच्च न त‚दाभिधाय‚क‚योः श‚ब्द‚योः‚{\tiny $_{lb}$}‚ \leavevmode\ledsidenote{\textenglish{70b/PSVTa}} श‚ब्द‚योर‚पि विशेष‚ण‚विशेष्य‚{\tiny $_{७}$}‚भावः । नापि सामानाधिक‚र‚ण्यं श‚ब्द‚वाच्य‚योर्भिन्न‚{\tiny $_{lb}$}‚त्वात् [।] नापि य‚त्रार्थेऽपोह‚योर्भाव‚स्त‚द्द्वार‚कं सामानाधिक‚र‚ण्य‚म‚व‚स्तुत्वेनापोह‚{\tiny $_{lb}$}‚स्याधेय‚त्वाभावात् [।] न च स्व‚ल‚क्ष‚णं श‚ब्द‚विष‚योन्य‚च्चाधिक‚र‚णं नेष्य‚ते । न च‚{\tiny $_{lb}$}‚ स्व‚ल‚क्ष‚णेपोह‚योर्भावेपि श‚ब्द‚योः सामानाधिक‚र‚ण्य‚मेक‚विष‚य‚त्व‚स्याप्र‚तीतेरित्य‚पास्तं ।
	{\color{gray}{\rmlatinfont\textsuperscript{§~\theparCount}}}
	\pend% ending standard par
      ‚{\tiny $_{lb}$}‚

	  
	  \pstart \leavevmode% starting standard par
	बाह्य‚भिन्न‚स्य स्वाकार‚स्य श‚ब्दादिविष‚य‚त्वेनेष्ट‚त्वा‚{\tiny $_{१}$}‚त् तेन नीलोत्प‚लादिश‚ब्देषु‚{\tiny $_{lb}$}‚ श‚ब्दार्थाभिधायित्व‚मिष्य‚त एवेति स‚र्वं सुस्थं ।
	{\color{gray}{\rmlatinfont\textsuperscript{§~\theparCount}}}
	\pend% ending standard par
      ‚{\tiny $_{lb}$}‚

	  
	  \pstart \leavevmode% starting standard par
	\textbf{ध‚र्म‚ध‚र्मिभेदोप्य‚स्य} बुद्धिप्र‚तिभास‚स्य य‚था प्र‚तीतिर्न विरुध्य‚त इति व‚च‚न‚{\tiny $_{lb}$}‚प‚रिणामेन स‚म्ब‚न्धः । त‚मेवा\textbf{नेकेत्या}दिनाह । \textbf{अनेक‚स्माद‚र्थाद्} बुद्धिप्र‚तिभास‚स्या‚{\tiny $_{lb}$}‚लीक‚त्वात् । कुतोनेकार्थ‚भेदः केव‚लं पुरुषाध्य‚व‚साय‚व‚शादेव‚मुच्य‚ते । \textbf{त}स्य बुद्धि‚{\tiny $_{lb}$}‚प्र‚तिभास‚स्\textbf{यैक‚स्मा‚{\tiny $_{२}$}‚द}र्थाद् यो \textbf{भेद}स्त‚स्य \textbf{विधिप्र‚तिषेध‚जिज्ञासायां} । किम‚नित्यः‚{\tiny $_{lb}$}‚ श‚ब्दो भ‚व‚ति चाक्षुषो न भ‚व‚तीति \textbf{त‚देव} बुद्धिप्र‚तिभास‚भूत\textbf{म्व‚स्तु} प्र‚द‚र्श्य‚त इति‚{\tiny $_{lb}$}‚ स‚म्ब‚न्धः [।] केन प्र‚कारेणेत्याह । \textbf{प्र‚तिक्षि}प्तेत्यादि ।
	{\color{gray}{\rmlatinfont\textsuperscript{§~\theparCount}}}
	\pend% ending standard par
      ‚{\tiny $_{lb}$}‚

	  
	  \pstart \leavevmode% starting standard par
	य‚द्वा ध‚र्म‚ध‚र्मिभेदोप्य‚स्य व‚स्तुनो न विरुध्य‚त इति स‚म्ब‚न्धः । कुत इत्याह ।‚{\tiny $_{lb}$}‚ अनेक‚स्माद‚र्थाद् बाह्य‚स्य भेद‚स‚म्भ‚वे स‚ति त‚स्यैक‚स्माद् यो भेद‚स्त‚स्य वि‚{\tiny $_{३}$}‚धि‚{\tiny $_{lb}$}‚प्र‚तिषेध‚जिज्ञासायां किम‚नित्यः श‚ब्दो भ‚व‚ति चाक्षुषो न भ‚व‚तीति त‚देव‚{\tiny $_{lb}$}‚ बाह्य‚म्व‚स्तु प्र‚द‚र्श्य‚त इति स‚म्ब‚न्धः । केन प्र‚द‚र्श्य‚त इत्याह । \textbf{प्र‚तिक्षिप्ते}त्यादि ।‚{\tiny $_{lb}$}‚ \textbf{ध‚र्म‚श‚ब्देन संचो}द्य व्य\textbf{तिरिक्तं ध‚र्म‚मिव व्य‚व‚स्था}प्येति स‚म्ब‚न्धः । त‚था ह्य‚नि‚{\tiny $_{lb}$}‚त्य‚त्व‚न्न चाक्षुष‚त्व‚मिति ध‚र्म‚श‚ब्देन चोद‚ने कृते व्य‚तिरिक्त इवानित्य‚त्वादिको ध‚र्मो‚{\tiny $_{lb}$}‚ व्य‚व‚स्थापितो भ‚व‚{\tiny $_{४}$}‚ति । किङ्कार‚ण‚मित्याह । त\textbf{था बुद्धेः प्र‚तिभास‚ना}त् । ध‚र्म‚{\tiny $_{lb}$}‚  ‚{\tiny $_{lb}$}‚ ‚{\tiny $_{lb}$}‚ \leavevmode\ledsidenote{\textenglish{189/s}}श‚ब्देन चोद‚ने व्य‚तिरिक्त‚स्यैव ध‚र्म‚स्य ग्राहिण्या बुद्धेः प्र‚तिभास‚नात् । \textbf{अविशेषेणे}ति‚{\tiny $_{lb}$}‚ स‚र्व‚भेदाप्र‚तिक्षेपेण्\textbf{आप‚र‚म‚स्य} बाह्य‚स्याप्र‚तिक्षिप्त‚भेदान्त‚रं \textbf{स्व‚भा}वं \textbf{ध‚र्मित‚या व्य‚व‚{\tiny $_{lb}$}‚स्थाप्य} श‚ब्देन प्र\textbf{द‚र्श्य‚ते} ।
	{\color{gray}{\rmlatinfont\textsuperscript{§~\theparCount}}}
	\pend% ending standard par
      ‚{\tiny $_{lb}$}‚

	  
	  \pstart \leavevmode% starting standard par
	एत‚दुक्त‚म्भ‚व‚ति । \textbf{ध‚र्म‚श‚ब्देन संचोद्य व्य‚तिरिक्तं ध‚र्म‚मि}व प्र‚द‚र्श्य‚{\tiny $_{५}$}‚ पु\textbf{न‚र्द्ध}‚{\tiny $_{lb}$}‚मिश‚ब्देन संचोद्या\textbf{प‚रं स्व‚भावं ध‚र्मित}या \textbf{व्य‚व‚स्थाप्य} त‚देव बाह्यं व‚स्तु प्र‚द‚र्श्य‚ते ।‚{\tiny $_{lb}$}‚ अनित्य‚त्वं श‚ब्द‚स्य न चाक्षुष‚त्व‚म् [।] अनित्यो न चाक्षुषः श‚ब्द इति । भाव‚भावा‚{\tiny $_{lb}$}‚ऽङ्शेनेति । भेदान्त‚र‚प्र‚तिक्षेपाप्र‚तिक्षेपेण \textbf{ध‚र्म‚ध‚र्मिणोर्भेदाद् भेद‚व‚तीव बुद्धि}र्विक‚ल्पिका‚{\tiny $_{lb}$}‚ \textbf{प्र‚तिभाति} भिन्नाकारेव । न तु व‚स्तुनो भेदः । \textbf{न व‚स्तुभेदाद्} भेद‚{\tiny $_{६}$}‚व‚ती बुद्धिः [।]‚{\tiny $_{lb}$}‚ कुतः [।] \textbf{य‚थोक्त‚दोषाद्} अनुप‚कारिणि ध‚र्मिणि ध‚र्माणां पार‚त‚न्त्र्यायोगात् ।
	{\color{gray}{\rmlatinfont\textsuperscript{§~\theparCount}}}
	\pend% ending standard par
      ‚{\tiny $_{lb}$}‚

	  
	  \pstart \leavevmode% starting standard par
	त‚था साध्य‚म‚नित्य‚त्वं साध‚नं कृत‚क‚त्व‚मिति [।] साध्य‚साध‚न‚भेद‚श्चेत्य‚त आह ।‚{\tiny $_{lb}$}‚ \textbf{त‚थाभूते}त्यादि । \textbf{त‚थाभूताना}म्प्र‚तिक्षिप्त‚भेदान्त‚राणां \textbf{ध‚र्म‚भेदानाम्बाहुल्य}चोद‚न‚या‚{\tiny $_{lb}$}‚ \textbf{व‚च‚न‚भेदः साध्य‚साध‚न‚भेद‚श्च} क्रिय‚त इति स‚म्ब‚{\tiny $_{७}$}‚न्धः ।
	{\color{gray}{\rmlatinfont\textsuperscript{§~\theparCount}}}
	\pend% ending standard par
      \textsuperscript{\textenglish{71a/PSVTa}}‚{\tiny $_{lb}$}‚

	  
	  \pstart \leavevmode% starting standard par
	\textbf{त‚स्य} त‚त‚स्त‚तो व्यावृत्त‚स्य व‚स्तुनः स्व‚भावः \textbf{स‚माश्र‚यो} येषान्तै\textbf{र्द्ध‚र्म‚प्र‚तिभास‚भेदै}‚{\tiny $_{lb}$}‚र्विक‚ल्प‚बुद्धिप्र‚तिविम्बैर्द्ध‚र्मात्म‚कैः प्र‚तिभास‚भेदैरित्य‚र्थः । किम‚र्थं क्रिय‚त इत्याह ।‚{\tiny $_{lb}$}‚ \textbf{त‚त्स्व‚भा}वेत्यादि । त‚स्यैव व्यावृत्त‚स्य व‚स्तु\textbf{स्व‚भाव}स्य \textbf{प्र‚तिप‚त्त‚ये} प्राप्त‚ये वा ॥
	{\color{gray}{\rmlatinfont\textsuperscript{§~\theparCount}}}
	\pend% ending standard par
      ‚{\tiny $_{lb}$}‚

	  
	  \pstart \leavevmode% starting standard par
	प्र‚णालिक‚या तेषाम्व‚स्तुप्र‚तिब‚न्धात् प्र‚कृत‚स्यैवार्थ‚स्य सुख‚ग्र‚ह‚णार्थं संग्र‚ह‚श्लोक‚{\tiny $_{lb}$}‚माह । \textbf{त‚त्स्व‚भावे}त्यादि‚{\tiny $_{१}$}‚ । त‚स्य व‚स्तुस्व‚भाव‚स्यानुभ‚वादूर्द्ध्वं या धीः प्र‚ज्ञाय‚ते \textbf{विक‚{\tiny $_{lb}$}‚ल्पिका} । अपिश‚ब्दो भिन्न‚क्र‚मः । अ\textbf{न‚र्थिका}पि \textbf{त‚द‚र्थेव} । व‚स्तुविष‚येव । अध्य‚व‚सा‚{\tiny $_{lb}$}‚य‚व‚शाद‚त‚त्कार्येभ्यो यो \textbf{भेद}स्त\textbf{न्निष्ठा} । त‚द‚नुभ‚व‚ब‚लोत्प‚त्तेर्व्यावृत्त‚स्य च व‚स्तुनः‚{\tiny $_{lb}$}‚ स‚म्वादात्त‚न्निष्ठेत्युच्य‚ते ॥
	{\color{gray}{\rmlatinfont\textsuperscript{§~\theparCount}}}
	\pend% ending standard par
      ‚{\tiny $_{lb}$}‚

	  
	  \pstart \leavevmode% starting standard par
	\textbf{त‚स्या}मित्थं भूतायाम्बुद्धौ योर्थाकारः दृश्य‚विक‚ल्प‚योरेकीक‚र‚णाद् \textbf{बाह्य‚{\tiny $_{lb}$}‚मिव} । स‚जातीयासु‚{\tiny $_{२}$}‚ व्य‚क्तिषु स‚म‚म्प्र‚तिभास‚मान‚मे\textbf{क‚मिवान्य‚तो} विजातीयाद् व्या‚{\tiny $_{lb}$}‚वृत्त‚मि\textbf{वाभाति} । त‚त्तु \textbf{व्यावृत्त‚मे}व बुद्धिरूप‚स्यालीक‚त्वात् । त‚द्बुद्धिरूपं \textbf{निस्त‚त्त्वं}‚{\tiny $_{lb}$}‚ ‚{\tiny $_{lb}$}‚ \leavevmode\ledsidenote{\textenglish{190/s}}निःस्व‚भावं । \textbf{किङ्कार‚णं [।] प‚रीक्षान‚ङ्ग‚भाव‚तः} । अर्थ‚क्रियास‚म‚र्थ‚मेव प‚रीक्षा‚{\tiny $_{lb}$}‚ङ्ग‚म‚तः \textbf{प‚रीक्षान‚ङ्ग‚भा}वेनार्थ‚क्रियां प्र‚त्य‚स‚म‚र्थ‚त्वादित्युक्त‚म्भ‚व‚ति । य‚त‚श्च बुद्धि‚{\tiny $_{lb}$}‚प्र‚तिभासि रूप‚न्निस्त‚त्त्व‚म‚त‚स्त‚द्विष‚{\tiny $_{३}$}‚यो व्य‚व‚हारो मिथ्यार्थ एव प्र‚व‚र्त्त‚ते‚{\tiny $_{lb}$}‚ इत्याह ॥
	{\color{gray}{\rmlatinfont\textsuperscript{§~\theparCount}}}
	\pend% ending standard par
      ‚{\tiny $_{lb}$}‚

	  
	  \pstart \leavevmode% starting standard par
	\textbf{अर्था} इत्यादि । \textbf{ज्ञान‚विशिष्}टा इति विक‚ल्प‚बुद्ध्यारूढास्ते ज्ञान‚विशिष्टा‚{\tiny $_{lb}$}‚स्स‚न्तः य‚तो विजातीयाद् \textbf{व्यावृत्तिरूपिणो} व्यावृत्तिरूप‚व‚न्तः । य‚थानुत्प‚लाद्‚{\tiny $_{lb}$}‚ व्यावृत्तिरूपिण उत्प‚लार्थाः । \textbf{तेने}त्य‚न्त‚तो व्यावृत्तिरूपेणोत्प‚ल‚त्वेनाभिन्ना इवाभान्ति‚{\tiny $_{lb}$}‚ न प‚र‚मार्थ‚तो बुद्धिरूप‚स्यालीक‚त्वात् । एतेन सा‚{\tiny $_{४}$}‚मान्य‚व्य‚व‚हार‚स्य निमित्त‚मुक्तं ।‚{\tiny $_{lb}$}‚ \textbf{व्यावृत्ताः पुन‚र‚न्य‚त‚स्त एवे}ति [।] त एव ज्ञान‚विशिष्टा अर्था अन्य‚तो व्यावृत्तिरूपिणः‚{\tiny $_{lb}$}‚ स‚न्तः पुन‚र‚न्य‚तः स‚जातीयाद‚पि व्यावृत्ता भान्ति । य‚था त एव नील‚भेदा‚{\tiny $_{lb}$}‚ अनीलात् । अत‚श्च व्यावृत्तिद्व‚योप‚गृहीत‚स्यैक‚स्यार्थ‚स्य भास‚नात् ॥
	{\color{gray}{\rmlatinfont\textsuperscript{§~\theparCount}}}
	\pend% ending standard par
      ‚{\tiny $_{lb}$}‚

	  
	  \pstart \leavevmode% starting standard par
	सामानाधिक‚र‚ण्य‚बीज‚मुक्तं । त‚देवाह । \textbf{तेषा}मित्यादि । ते‚{\tiny $_{५}$}‚षामिति बुद्धि‚{\tiny $_{lb}$}‚प्र‚तिभासिनाम‚र्थानां \textbf{व्य‚व‚हारः प्र‚त‚न्य‚त} इति स‚म्ब‚न्धः । तेषामिति \textbf{व्य‚व‚हा}रापेक्षा‚{\tiny $_{lb}$}‚ क‚र्म‚णि ष‚ष्ठी । किंविशिष्टो \textbf{मिथ्या}र्थः । कैः क‚र‚ण‚भूतैरित्याह । \textbf{सामान्ये}त्यादि ।‚{\tiny $_{lb}$}‚ स‚मानाधार‚त्वं स\textbf{मानाधा}रः । भाव‚प्र‚धान‚त्वान्निर्देश‚स्य । सामान्य\textbf{विष‚यैः} सामा‚{\tiny $_{lb}$}‚नाधिक‚र‚ण्य‚विष‚यैश्च ज्ञानाभिधानैः \textbf{सामान्य‚गोच‚रैः} सामा‚{\tiny $_{६}$}‚न्य‚व्य‚व‚हारः प्र‚त‚न्य‚ते ।‚{\tiny $_{lb}$}‚ इत‚रैः सामानाधिक‚र‚ण्य‚व्य‚व‚हारः ॥
	{\color{gray}{\rmlatinfont\textsuperscript{§~\theparCount}}}
	\pend% ending standard par
      ‚{\tiny $_{lb}$}‚

	  
	  \pstart \leavevmode% starting standard par
	य‚द्य‚पि श‚ब्द‚ज्ञानात्म‚क एवेह व्य‚व‚हार‚स्त‚थापि स‚व्यापार‚तामुपा\textbf{दाय ज्ञा}न‚श‚ब्द‚{\tiny $_{lb}$}‚योः क‚र‚ण‚रूप‚ता । त‚योरेवार्थ‚प्र‚काश‚ल‚क्ष‚णा क्रिया व्य‚व‚हार‚त्वेन विव‚क्षितेत्य‚दोषः ।‚{\tiny $_{lb}$}‚ य‚था ज्ञान‚स्य प्र‚माण‚त्वं फ‚ल‚त्वं वेति । \textbf{स च स‚र्वः} ज्ञानाभिधान‚ल‚क्ष\textbf{णो} व्य‚व‚हारः‚{\tiny $_{lb}$}‚ \leavevmode\ledsidenote{\textenglish{71b/PSVTa}} \textbf{प‚दा‚{\tiny $_{७}$}‚र्थानां} स्व‚ल‚क्ष‚णानां यो\textbf{न्योन्याभा}वः प‚र‚स्प‚र‚व्य‚च्छेद‚स्त‚त्\textbf{संश्र‚यः} । व्यावृत्त‚प‚दार्था‚{\tiny $_{lb}$}‚नुभ‚व‚द्वारेणोत्प‚त्तेः । \textbf{तेने}त्य‚न्योन्याभाव‚संश्र‚य‚त्वेन स व्य‚व‚हारो\textbf{न्यापोह‚विष‚य} उच्य‚ते‚{\tiny $_{lb}$}‚ न त्व‚न्य‚व्यावृत्तिविष‚य‚त्वात् । विधिविष‚य‚त्वाद‚स्य व्य‚व‚हार‚स्य ॥
	{\color{gray}{\rmlatinfont\textsuperscript{§~\theparCount}}}
	\pend% ending standard par
      ‚{\tiny $_{lb}$}‚‚{\tiny $_{lb}$}‚\textsuperscript{\textenglish{191/s}}

	  
	  \pstart \leavevmode% starting standard par
	\textbf{व‚स्तुलाभ}स्य च व‚स्तुप्राप्तेश्\textbf{चाश्र‚यो} न भ‚व‚ति व्य‚व‚हारः । न च स‚र्वः किन्तु‚{\tiny $_{lb}$}‚ \textbf{य‚त्र} व्य‚व‚हारेऽ\textbf{स्ति} पार‚म्प‚र्येण त‚था‚{\tiny $_{१}$}‚भूत\textbf{व‚स्तुस‚म्ब‚न्धः} । उदाह‚र‚ण‚माह । \textbf{य‚थो‚{\tiny $_{lb}$}‚क्तानुमितौ य‚थे}ति । \textbf{य‚थोक्तानुमि}तिः पूर्वोक्तानुमान‚विक‚ल्पः । \textbf{नान्य}त्र स्थिरादि‚{\tiny $_{lb}$}‚विक‚ल्पे त‚त्र पार‚म्प‚र्येणापि व‚स्तुस‚म्ब‚न्धाभावात् । व‚स्तुनोऽस्थिरादिरूप‚त्वात् ।‚{\tiny $_{lb}$}‚ अनुमान‚विक‚ल्प‚स्येत‚र‚स्य च स्व‚प्र‚तिभासेन‚र्थेऽर्थाध्य‚व‚साय‚द् \textbf{भ्रान्तिसाम्येपि । दी}प‚{\tiny $_{lb}$}‚\textbf{तेजो म‚णौ य‚थे}ति । य‚था म‚णितेज‚सि म‚{\tiny $_{२}$}‚णिबुद्धिर्भ्रान्ता । त‚था दीप‚तेज‚स्य‚पि‚{\tiny $_{lb}$}‚तुल्येपि भ्रान्त‚त्वे म‚णिप्र‚भा म‚णित्वेन गृहीता म‚णाव‚धिग‚न्त‚व्ये स‚म्वादिका । न‚{\tiny $_{lb}$}‚ तु दीप‚तेजः । त‚द्व‚त् स्थिरादिविक‚ल्पो स‚म्वाद‚क इत्य‚र्थः ॥
	{\color{gray}{\rmlatinfont\textsuperscript{§~\theparCount}}}
	\pend% ending standard par
      ‚{\tiny $_{lb}$}‚

	  
	  \pstart \leavevmode% starting standard par
	य‚दि ज्ञान‚निविष्टानामेवार्थानां सामान्यादिव्य‚व‚हारः बाह्येष्व‚र्थेषु त‚र्हि सामा‚{\tiny $_{lb}$}‚न्यादिव्य‚व‚हाराभावात् प्र‚वृत्तिर्न स्यादित्य‚त आह । \textbf{त‚त्रे}त्यादि । \textbf{त‚त्र} श‚ब्दो वाक्यो‚{\tiny $_{lb}$}‚प‚न्या‚{\tiny $_{३}$}‚से । निर्द्धार‚णे वा । त‚त्र व्य‚क्तिष्वेक‚कार्या व्य‚क्त‚यो निर्द्धार्य‚न्ते । \textbf{अनेको‚{\tiny $_{lb}$}‚प्येक‚कार्यो} य‚था घ‚ट‚भेदा एवोद‚काह‚र‚णादिकार्यास्त‚द‚कार्याय त‚थाभूत‚कार्य‚नुकुर्व‚ते ।‚{\tiny $_{lb}$}‚ तेभ्योन्य‚ताव्यावृत्तिः सा आश्र‚यो येषां \textbf{ज्ञानाभिधाना}त्तैर‚नेकोपि प‚दार्थ \textbf{एक‚त्वेन‚{\tiny $_{lb}$}‚ व्य‚व‚हार‚म्प्र‚तार्य}ते । सामान्य‚व्य‚व‚हारं प्राप्य‚त इति याव‚त् ॥
	{\color{gray}{\rmlatinfont\textsuperscript{§~\theparCount}}}
	\pend% ending standard par
      ‚{\tiny $_{lb}$}‚

	  
	  \pstart \leavevmode% starting standard par
	त‚थेत्य‚न‚न्त‚र‚सा‚{\tiny $_{४}$}‚मान्य‚व्य‚पेक्ष‚या । \textbf{एकोप्य‚नेक‚कार्य}कृत् । य‚था घ‚ट‚श्च च‚क्षुर्वि‚{\tiny $_{lb}$}‚ज्ञानोद‚काह‚र‚णादिकार्य‚कृत् । \textbf{त‚द्भाव‚दीप‚ने} । अनेक‚कार्य‚क‚र्त्तृत्व‚प्र‚काश‚ने । \textbf{अत‚{\tiny $_{lb}$}‚त्कार्ये}भ्यो \textbf{भेदेन} हेतुना । \textbf{नानाध‚र्मा} घ‚ट‚श्चाक्षुषः पार्थिव इत्यादि । तेन सामाना‚{\tiny $_{lb}$}‚धिक‚र‚ण्य‚विशेष‚ण‚विशेष्य‚भाव‚व्य‚व‚हार‚श्च बाह्येष्वेव द‚र्शितः ॥
	{\color{gray}{\rmlatinfont\textsuperscript{§~\theparCount}}}
	\pend% ending standard par
      ‚{\tiny $_{lb}$}‚

	  
	  \pstart \leavevmode% starting standard par
	य‚दि बाह्येषु सामान्यादि‚{\tiny $_{५}$}‚व्य‚व‚हारः पार‚मार्थिक‚स्त‚र्हि प्राप्त इत्याह । \textbf{य‚था‚{\tiny $_{lb}$}‚प्र‚तीतिक‚थितः} इति । विक‚ल्प‚बुद्ध्य‚नुरोधेन \textbf{श‚ब्दार्थः} सामान्य‚ल‚क्ष‚णः प‚र‚मार्थ‚{\tiny $_{lb}$}‚\textbf{तोसाव‚स‚न्न}पि य‚थाप्र‚तीतिक‚थितः । \textbf{सामानाधिक‚र‚ण्यं} च य‚थाप्र‚तीतिक‚थितं ।‚{\tiny $_{lb}$}‚ य‚स्माद् \textbf{व‚स्तुन्य‚स्य} श‚ब्दार्थ‚स्य सामानाधिक‚र‚ण्य‚स्य च \textbf{न स‚म्भ‚वः} ॥
	{\color{gray}{\rmlatinfont\textsuperscript{§~\theparCount}}}
	\pend% ending standard par
      ‚{\tiny $_{lb}$}‚‚{\tiny $_{lb}$}‚\textsuperscript{\textenglish{192/s}}

	  
	  \pstart \leavevmode% starting standard par
	ध‚र्म‚ध‚र्मिव्य‚व‚हार‚श्च ज्ञान‚प्र‚तिभासि‚{\tiny $_{६}$}‚न्य‚र्थ इति य‚त् प्र‚कृत‚न्त‚त्संग्र‚हार्थ‚माह ।‚{\tiny $_{lb}$}‚ \textbf{ध‚र्मे}त्यादि । अयं \textbf{ध‚र्मो}ऽयं \textbf{ध‚र्मी}ति \textbf{व्य‚व‚स्थान}न्त‚योश्च \textbf{भेदो}ऽस्यायं ध‚र्मः कृत‚क‚त्वा‚{\tiny $_{lb}$}‚दिल‚क्ष‚णः । अ\textbf{भेदः} कृत‚कोय‚मिति \textbf{यादृशः} । अयं च विक‚ल्पारोपित‚त्वात् । \textbf{अस‚मी‚{\tiny $_{lb}$}‚क्षित‚त‚त्त्वार्थः} । अन‚पेक्षित‚स्त‚त्त्वार्थो व‚स्त्व‚र्थो येनेति विग्र‚हः । \textbf{य‚था लोके}‚{\tiny $_{lb}$}‚ \leavevmode\ledsidenote{\textenglish{72a/PSVTa}} बुद्ध्यारूढोप्य‚ध्य‚व‚सित‚त‚द्भाव‚त‚या \textbf{प्र‚तीय‚ते}‚{\tiny $_{७}$}‚ ॥
	{\color{gray}{\rmlatinfont\textsuperscript{§~\theparCount}}}
	\pend% ending standard par
      ‚{\tiny $_{lb}$}‚

	  
	  \pstart \leavevmode% starting standard par
	\textbf{तं ध‚र्मादि}विभाग\textbf{न्त‚थै}वेति य‚थालोक‚प्र‚तीतिः \textbf{साध्य‚साध‚न‚संस्थितिर्विद्व‚द्भि‚{\tiny $_{lb}$}‚र‚व‚क‚ल्प्य‚त} इति स‚म्ब‚न्धः । किम‚र्थं । \textbf{प‚र‚मार्थाव‚ताराय} । अनित्यादिव‚स्तुस्व‚भाव‚{\tiny $_{lb}$}‚स्याव‚गाह‚नाय ॥
	{\color{gray}{\rmlatinfont\textsuperscript{§~\theparCount}}}
	\pend% ending standard par
      ‚{\tiny $_{lb}$}‚

	  
	  \pstart \leavevmode% starting standard par
	किं पुनः कार‚ण‚मेक‚त्वादिव्य‚व‚हार‚स्य व‚स्तुन्य‚स‚म्भ‚व इत्याह । \textbf{संसृज्य‚न्त}‚{\tiny $_{lb}$}‚ इत्यादि । \textbf{पार‚मार्थिका अर्थाः स्व‚तो न संसृज्य‚न्ते} । त‚तो न सामान्य‚व्य‚व‚हारो व‚स्तुनि ।‚{\tiny $_{lb}$}‚ \textbf{न भि‚{\tiny $_{१}$}‚द्य‚न्ते} कृत‚क‚त्वादिध‚र्मैः प्र‚त्येकं त‚स्य व‚स्तुनोनेक‚त्वायोगात् । अत‚श्च \textbf{तेष्व}र्थेषु‚{\tiny $_{lb}$}‚ ब‚हुषु रूप\textbf{मेक}मेकंस्मिन्न‚र्थे\textbf{ऽनेकं रूपं} य‚द‚ध्य‚व‚सीय‚ते त‚द्बुद्धेर्विक‚ल्पिकाया \textbf{उप‚प्ल‚वो}‚{\tiny $_{lb}$}‚ भ्रान्तिः । य‚त‚श्च न संसृज्य‚न्ते न भिद्य‚न्ते पार‚मार्थिका अर्थाः ॥
	{\color{gray}{\rmlatinfont\textsuperscript{§~\theparCount}}}
	\pend% ending standard par
      ‚{\tiny $_{lb}$}‚

	  
	  \pstart \leavevmode% starting standard par
	त‚तः कार‚णात् \textbf{सामान्य}मिदं ब‚हूनां । त‚था ध‚र्माणां ध‚र्मिणां च \textbf{भेद इत्य‚पि}‚{\tiny $_{lb}$}‚ योय\textbf{म्भेदो} नानात्व‚व्य‚व‚हारः स \textbf{बौ‚{\tiny $_{२}$}‚द्धेर्थे} । न बाह्ये स्व‚ल‚क्ष‚णे । बुद्धिप्र‚तिभास‚स्या‚{\tiny $_{lb}$}‚लीक‚त्वात् [।] क‚थ‚न्त‚र्हि त‚त्र स्व‚ल‚क्ष‚णे कृत‚क‚त्वादिध‚र्म‚भेद इति चेदाह । \textbf{त‚स्यैव‚{\tiny $_{lb}$}‚ चे}त्यादि । \textbf{त‚स्यैव} स्व‚ल‚क्ष‚ण‚स्या\textbf{न्य‚व्यावृत्त्या ध‚र्म‚भेदः प्र‚क‚ल्प्य‚ते} ॥
	{\color{gray}{\rmlatinfont\textsuperscript{§~\theparCount}}}
	\pend% ending standard par
      ‚{\tiny $_{lb}$}‚

	  
	  \pstart \leavevmode% starting standard par
	क‚स्मात् क‚ल्पित‚ध‚र्म‚भेद‚द्वारेण ग‚म्य‚ग‚म‚क‚भावो न व‚स्तुद‚र्श‚न‚मात्रेणेत्य‚त आह ।‚{\tiny $_{lb}$}‚ \textbf{साध्ये}त्यादि । इदं \textbf{साध्य}मिदं \textbf{साध‚न}मित्य‚स्मिन् \textbf{संक‚ल्पे व‚स्तुद‚{\tiny $_{३}$}‚र्श‚न‚हानितः} । य‚द्वा‚{\tiny $_{lb}$}‚ प्र‚त्य‚क्ष‚व‚द‚नुमानेन सामान्य‚विशिष्टं स्व‚ल‚क्ष‚ण‚मेव क‚स्मान्न गृह्य‚त इत्य‚त आह ।‚{\tiny $_{lb}$}‚ ‚{\tiny $_{lb}$}‚ \leavevmode\ledsidenote{\textenglish{193/s}}\textbf{साध्येत्यादि} । साध्य‚साध‚न‚संक‚ल्पे व‚स्तुद‚र्श‚न‚हानितः । कुतः स्व‚ल‚क्ष‚ण‚स्य सामान्य‚{\tiny $_{lb}$}‚विशिष्ट‚स्य ग्र‚ह‚णं । \textbf{भेदः सामान्य‚संसृष्टो ग्राह्य} इत्याचार्य दि ग्ना ग प्र‚भृतिभिः‚{\tiny $_{lb}$}‚ सामान्य‚संसृष्ट‚स्य स्व‚ल‚क्ष‚ण‚स्य ग्र‚ह‚णं प्र‚तिज्ञात‚मित्या‚{\tiny $_{४}$}‚श‚ङ्काम‚प‚न‚य‚न्नाह । \textbf{भेद}‚{\tiny $_{lb}$}‚ इत्यादि । \textbf{भेदः सामान्य‚संसृष्टः प्र‚तीय‚त} इत्\textbf{य‚त्रा}पि व‚च‚ने ग्राह्यं न \textbf{स्व‚ल‚क्ष‚ण‚मे}व‚{\tiny $_{lb}$}‚ निर्दिष्ट‚मिति नैव‚म्बोद्ध‚व्य‚मित्य‚र्थः । किन्तु बाह्या एव भेदास्तेनान्यापोह‚ल‚क्ष‚णेन‚{\tiny $_{lb}$}‚ सामान्येन संसृष्टा अध्य‚व‚सीय‚न्ते न तु गृह्य‚न्त इति त‚त्रापि बोद्ध‚व्यं ।
	{\color{gray}{\rmlatinfont\textsuperscript{§~\theparCount}}}
	\pend% ending standard par
      ‚{\tiny $_{lb}$}‚

	  
	  \pstart \leavevmode% starting standard par
	अन्ये तु \textbf{भेदः सामान्य‚संसृष्टो ग्राह्य} इति पुल्लिङ्गे‚{\tiny $_{५}$}‚न प‚ठ‚न्ति । त‚त्राय‚म‚र्थो‚{\tiny $_{lb}$}‚ भेदः । \textbf{सामान्य‚संसृष्टो} ग्राह्य इत्य‚त्रापि व‚च‚ने । न स्व‚ल‚क्ष‚णं बोद्ध‚व्यं ॥
	{\color{gray}{\rmlatinfont\textsuperscript{§~\theparCount}}}
	\pend% ending standard par
      ‚{\tiny $_{lb}$}‚

	  
	  \pstart \leavevmode% starting standard par
	किम्पुनः कार‚ण‚न्त‚त्रैव बोद्ध‚व्य‚मिति चेदाह । \textbf{स‚माने}त्यादि । अनेक‚स्मिन्ने‚{\tiny $_{lb}$}‚काकारः \textbf{स‚माना}कारः । एक‚स्मिन्न‚नेक‚ध‚र्म‚त्व\textbf{म्भिन्ना}कारः । \textbf{आदि}श‚ब्दाद् ध‚र्म‚{\tiny $_{lb}$}‚ध‚र्म्याकार‚प‚रिग्र‚हः । \textbf{न त‚त्} स्व‚ल‚क्ष‚णं \textbf{ग्राह्यं क‚थंच‚न} । किं कार‚णं [।] त‚त्रैक‚स्मिन्‚{\tiny $_{lb}$}‚ स्व‚ल‚क्ष‚णे‚{\tiny $_{६}$}‚ कृत‚क‚त्वानित्य‚त्वादिरूपेण \textbf{ब‚हुभेदा}नान्ध‚र्माणां किम्विशिष्टाना\textbf{म्भेदा‚{\tiny $_{lb}$}‚ना}म्व‚स्तुरूपाणा\textbf{न्त‚त्रैक‚स्मिन्} स्व‚ल‚क्ष‚णेऽ\textbf{योगात्} । न ह्येक‚स्य व‚स्तुरूपाणि ब‚हूनि‚{\tiny $_{lb}$}‚ युज्य‚न्ते निर‚ङ्श‚त्वात् स्व‚ल‚क्ष‚ण‚स्य ॥
	{\color{gray}{\rmlatinfont\textsuperscript{§~\theparCount}}}
	\pend% ending standard par
      ‚{\tiny $_{lb}$}‚

	  
	  \pstart \leavevmode% starting standard par
	उप‚संह‚र‚न्नाह । \textbf{त‚द्रूप}मित्यादि । \textbf{त‚त्त}स्माद् \textbf{रूपं} स्व‚ल‚क्ष‚णं \textbf{स‚र्व‚तो भिन्न}‚{\tiny $_{lb}$}‚म‚साधार‚ण\textbf{न्त‚था} तेनासाधार‚णेन रूपेण \textbf{त}स्य स्व‚ल‚क्ष‚ण‚स्य न‚{\tiny $_{७}$}‚ \textbf{प्र‚तिपादिका श्रुतिः} \leavevmode\ledsidenote{\textenglish{72b/PSVTa}}‚{\tiny $_{lb}$}‚ श‚ब्दो \textbf{ना}स्ति । \textbf{क‚ल्प‚ना वास्ति} । नेति प्र‚कृतं । असाधार‚णेन स्व‚रूपेण स्व‚ल‚क्ष‚ण‚स्य‚{\tiny $_{lb}$}‚ ग्राह‚को नास्तीत्य‚र्थः । किं कार‚णं [।] \textbf{सामान्येनैव} श‚ब्द‚स्य क‚ल्प‚नायाश्च‚{\tiny $_{lb}$}‚ \textbf{वृत्तितः} ॥
	{\color{gray}{\rmlatinfont\textsuperscript{§~\theparCount}}}
	\pend% ending standard par
      ‚{\tiny $_{lb}$}‚

	  
	  \pstart \leavevmode% starting standard par
	त‚त्प्र‚तिपादिका न श्रुतिर‚स्तीति ब्रुव‚ता स्व‚ल‚क्ष‚णे श‚ब्दा न नियुज्य‚न्त इत्युक्त‚{\tiny $_{lb}$}‚म‚त‚श्चोद‚य‚ति ।
	{\color{gray}{\rmlatinfont\textsuperscript{§~\theparCount}}}
	\pend% ending standard par
      ‚{\tiny $_{lb}$}‚

	  
	  \pstart \leavevmode% starting standard par
	\textbf{किम्पुन}रित्यादि । संकेतेन विष‚यीकृताः \textbf{संकेतिनः} । त‚मा\textbf{हुः} श‚ब्दाः। व्य‚{\tiny $_{१}$}‚व‚{\tiny $_{lb}$}‚\textbf{हाराय} संकेतः \textbf{स्मृतः । त‚दा} व्य‚व‚हार‚काले त\textbf{त्स्व‚ल‚क्ष‚ण‚न्नास्ति} य‚त्र संकेतः कृतः ।‚{\tiny $_{lb}$}‚ ‚{\tiny $_{lb}$}‚ \leavevmode\ledsidenote{\textenglish{194/s}}एक‚स्यापि स्व‚ल‚क्ष‚ण‚स्य क्ष‚णिक‚त्वात् कालान्त‚रे तेनैव रूपेणानुग‚मो नास्त्य‚क्ष‚णिक‚त्वे‚{\tiny $_{lb}$}‚ वा संकेतः ज्ञानाभावादेव त‚द्विष‚य‚त्व‚स्य कालान्त‚रेनुग‚मो नास्ति किमुत देश‚काल‚{\tiny $_{lb}$}‚भिन्नेषु स्व‚ल‚क्ष‚णेषु । \textbf{तेन} कार‚णेन \textbf{त‚त्र} स्व‚ल‚क्ष‚णे \textbf{संकेतो न} क्रिय‚त इत्य‚ध्या‚{\tiny $_{२}$}‚हारः ।
	{\color{gray}{\rmlatinfont\textsuperscript{§~\theparCount}}}
	\pend% ending standard par
      ‚{\tiny $_{lb}$}‚

	  
	  \pstart \leavevmode% starting standard par
	\textbf{न ही}त्यादिना व्याच‚ष्टे ।
	{\color{gray}{\rmlatinfont\textsuperscript{§~\theparCount}}}
	\pend% ending standard par
      ‚{\tiny $_{lb}$}‚

	  
	  \pstart \leavevmode% starting standard par
	\textbf{अपि नामे}ति क‚थ‚न्नु नाम । \textbf{प्रा}गिति संकेत‚काल‚कृत‚स‚म्ब‚न्ध‚स्य श‚ब्द‚स्येति‚{\tiny $_{lb}$}‚ स‚म्ब‚न्धः । एक‚त्रैक‚स्सिम‚न् \textbf{स्व‚ल‚क्ष}णे \textbf{प‚श्चा}दिति व्य‚व‚हार‚काले । किङ्कार‚णं [।]‚{\tiny $_{lb}$}‚ \textbf{न युक्त}मित्याह । \textbf{त‚स्ये}त्यादि । \textbf{त‚स्ये}ति संकेत‚काल‚दृष्ट‚स्य व्य‚व‚हाराव‚स्थाना‚{\tiny $_{lb}$}‚दिषु दे\textbf{श‚काल‚भेदेष्व‚नास्क‚न्द‚ना}त् । अनुग‚मात् । न ह्येक‚त्र दृष्टो भेदोन्य‚{\tiny $_{३}$}‚त्र‚{\tiny $_{lb}$}‚ स‚म्भ‚व‚ति ॥
	{\color{gray}{\rmlatinfont\textsuperscript{§~\theparCount}}}
	\pend% ending standard par
      ‚{\tiny $_{lb}$}‚

	  
	  \pstart \leavevmode% starting standard par
	\textbf{व्य‚तिरिक्त}मिति वै शे षि क द‚र्श‚ने\textbf{नाव्य‚तिरिक्तं} सां ख्य द‚र्श‚नेन । स‚मान‚{\tiny $_{lb}$}‚जातीय‚व्य‚क्ति\textbf{व्याप‚नाद्} व्यापि सामान्यं । \textbf{त‚त्त}स्मा\textbf{न्न व्य‚व‚हार‚कालाभाव‚दोषः} ।‚{\tiny $_{lb}$}‚ व्य‚व‚हार‚काले श‚ब्दार्थ‚स्याभाव‚दोषो नास्ति । सामान्य‚स्य श‚ब्दार्थ‚त्वात्त‚स्यैवैक‚त्वेन‚{\tiny $_{lb}$}‚ संकेत‚व्य‚व‚हार‚काल‚योर्विद्य‚मान‚त्वात् ।
	{\color{gray}{\rmlatinfont\textsuperscript{§~\theparCount}}}
	\pend% ending standard par
      ‚{\tiny $_{lb}$}‚

	  
	  \pstart \leavevmode% starting standard par
	क‚थं नामेत्य‚स्मिन्न‚र्थे \textbf{अपि}श‚ब्दः‚{\tiny $_{४}$}‚ [।] व्य‚व‚हार‚काले श‚ब्दादुच्च‚रिता\textbf{द‚र्थ‚{\tiny $_{lb}$}‚क्रियाक्ष‚मा}न् अर्थान् \textbf{विज्ञाय त‚त्साध‚नाया}र्थ‚क्रियासाध‚नाय क‚थ‚न्नाम प्र‚व‚र्त्त‚त‚{\tiny $_{lb}$}‚ \textbf{पुमा}नित्य‚नेनाभिप्रायेणा\textbf{र्थेषु संयोज्य‚न्तेऽभिधाय‚काः} श‚ब्दाः ।
	{\color{gray}{\rmlatinfont\textsuperscript{§~\theparCount}}}
	\pend% ending standard par
      ‚{\tiny $_{lb}$}‚

	  
	  \pstart \leavevmode% starting standard par
	\textbf{न ख}ल्वित्यादिना व्याच‚ष्टे । फ‚ल‚निर‚पेक्षं क्व‚चित् तात्प‚र्य \textbf{व्य‚स‚नं । य‚द‚यं}‚{\tiny $_{lb}$}‚ लोको सं\textbf{केत‚य‚न्} संकेत‚म‚कुर्वाणः संकेतितेर्थे श\textbf{ब्दान् प्र‚{\tiny $_{५}$}‚युञ्जानो} वा । \textbf{स‚र्व एवेति}‚{\tiny $_{lb}$}‚ ‚{\tiny $_{lb}$}‚ \leavevmode\ledsidenote{\textenglish{195/s}}शाब्दोन्यो वा\textbf{व‚धेयो} ग्र‚ह‚णार्ह \textbf{आर‚म्भो} व्य‚व‚हारः \textbf{फ‚लार्थः} । न तु निष्फ‚लः [।] कि‚{\tiny $_{lb}$}‚ङ्कार‚ण [।] \textbf{निष्फ‚लार‚म्भ‚स्य} प्रेक्षापूर्व‚कारिभि\textbf{रुपेक्ष‚णीय‚त्वा}द‚ग्राह्य‚त्वात् । त‚दिति‚{\tiny $_{lb}$}‚ त‚स्मात् । \textbf{अयं} प्र‚तिप‚त्ता \textbf{क्व‚चिद}भिम‚तेर्थे \textbf{नियुञ्जा}नः संकेत‚य‚न् \textbf{फ‚ल‚मेवेहितुं‚{\tiny $_{lb}$}‚ युक्त} इति प्र‚योज‚न‚मेवापेक्षितुम‚र्ह‚तीति याव‚त्‚{\tiny $_{६}$}‚ । \textbf{त‚च्चे}ति फ‚ल‚मिष्ट‚स्याप्तिल‚क्ष‚ण‚{\tiny $_{lb}$}‚म‚निष्ट‚स्य च त्याग‚ल‚क्ष‚ण‚मिति य‚थायोगं स‚म्ब‚न्धः । येने\textbf{ष्टानिष्ट‚प्राप्तिप‚रिहार}‚{\tiny $_{lb}$}‚रूप एव पुरुषार्थोभिप्रेत\textbf{स्तेनायं} पुरुष‚स्त‚यो\textbf{रिष्ट‚योः} फ‚ल‚योः \textbf{साध‚न‚म‚साध‚नं} चार्थं‚{\tiny $_{lb}$}‚ ज्ञात्वा \textbf{त‚त्रे}ष्ट‚साध‚ने \textbf{प्र‚वृत्ति}म‚निष्ट‚साध‚ने च \textbf{निवृत्तिं कुर्यां कार‚येय‚म्वा} प‚रानित्य‚ने‚{\tiny $_{lb}$}‚नाभिप्रायेण \textbf{श‚ब्दान्नियुञ्जी}त प्र‚योक्ता‚{\tiny $_{७}$}‚ श्रोतापि \textbf{नियोगे वाद्रिये}त । \leavevmode\ledsidenote{\textenglish{73a/PSVTa}}
	{\color{gray}{\rmlatinfont\textsuperscript{§~\theparCount}}}
	\pend% ending standard par
      ‚{\tiny $_{lb}$}‚

	  
	  \pstart \leavevmode% starting standard par
	युक्त‚न्ताव‚त् प‚रं व्य‚व‚हार‚येय‚मिति श‚ब्द‚नियोगः । श‚ब्द‚नियोग‚स्य प‚राङ्ग‚त्वात् ।‚{\tiny $_{lb}$}‚ स्व‚य‚न्तु प्र‚वृत्तिनिवृत्तिकार‚णे कः श‚ब्द‚स्योप‚योगः ।
	{\color{gray}{\rmlatinfont\textsuperscript{§~\theparCount}}}
	\pend% ending standard par
      ‚{\tiny $_{lb}$}‚

	  
	  \pstart \leavevmode% starting standard par
	स‚त्यं [।] केव‚लं श‚ब्द‚प्र‚योगाभ्यासात् स्व‚य‚म‚पि प्र‚तिप‚द्य‚मानः क‚दाचिदेवं‚{\tiny $_{lb}$}‚ प्र‚तिप‚द्य‚त इत्युप‚न्यासः कृतः । \textbf{अन्य‚थोपेक्ष‚णीय‚त्वादि}ति फ‚ल‚म‚न्त‚रेण श‚ब्द‚नियोग‚{\tiny $_{lb}$}‚स्योपेक्ष‚णीय‚त्वात् । \textbf{त‚त्रै}वं व्य‚व‚स्थिते न्याये \textbf{जाति‚{\tiny $_{१}$}‚र‚न‚र्थ‚क्रियायोग्या}ऽतो न श‚ब्द‚{\tiny $_{lb}$}‚विष‚या ।
	{\color{gray}{\rmlatinfont\textsuperscript{§~\theparCount}}}
	\pend% ending standard par
      ‚{\tiny $_{lb}$}‚

	  
	  \pstart \leavevmode% starting standard par
	त‚द्व्याच‚ष्टे \textbf{न ही}त्यादि । \textbf{न जातिर्वाह‚दोहादिकं} क‚र्त्तुं स‚म‚र्था [।] त‚त‚श्च‚{\tiny $_{lb}$}‚ वाह‚दोहाद्य‚र्थिनो जातिचोद‚ना निष्फ‚लेति न त‚द‚र्थः \textbf{श‚ब्द‚प्र‚योगः} ।
	{\color{gray}{\rmlatinfont\textsuperscript{§~\theparCount}}}
	\pend% ending standard par
      ‚{\tiny $_{lb}$}‚

	  
	  \pstart \leavevmode% starting standard par
	यापि स्व‚प्र‚तिप‚त्तिल‚क्ष‚णार्थ‚क्रिया जातेरुप‚व‚र्ण्ण्य‚ते । न त‚द‚र्थ‚म्पुरुषः प्र‚व‚र्त्त‚ते‚{\tiny $_{lb}$}‚ श‚ब्द‚प्र‚योगादेव त‚स्याः सिद्ध‚त्वात् । जातिमात्र‚प्र‚तिप‚त्त्य‚र्थं श‚ब्द‚प्र‚योगो भ‚विष्य‚{\tiny $_{lb}$}‚तीति चेद‚त आह । \textbf{न वे}त्यादि । \textbf{तादृश}मिति वाह‚दोहादि\textbf{प्र‚क‚र‚णं} निष्फ‚ल‚स्य‚{\tiny $_{lb}$}‚ श‚ब्द‚प्र‚योग‚स्योपेक्ष‚णीय‚त्वादित्युक्त‚त्वात् । जातौ च वाच्यायां स‚त्यां । गामान‚ये‚{\tiny $_{lb}$}‚त्य‚त्र वाक्येन वाक्यार्थ‚प्र‚तीतिः स्यात् । गोत्व‚स्य क्रियात्वेन्व‚याभावात् । नापि ल‚क्षि‚{\tiny $_{lb}$}‚त‚ल‚क्ष‚ण‚या वाक्यार्थ‚प्र‚तीतिर‚प्र‚तीतेर्न हि प‚देभ्य‚स्ताव‚त् सामान्यानां प्र‚तीतिः [।]‚{\tiny $_{lb}$}‚ पुन‚स्तेभ्यो विशेषाणां विशेषेभ्य‚श्चान्व‚य‚स्येत्येवं विल‚म्बित‚रूपा वाक्यार्थ‚{\tiny $_{३}$}‚प्र‚तीतिः ।
	{\color{gray}{\rmlatinfont\textsuperscript{§~\theparCount}}}
	\pend% ending standard par
      ‚{\tiny $_{lb}$}‚

	  
	  \pstart \leavevmode% starting standard par
	न‚न्व‚पोहेपि वाच्ये क‚थं बाह्यार्थ‚प्र‚तीतिर्नीरूप‚त्वाद‚पोह‚स्य [।] न च ज्ञानांशे‚{\tiny $_{lb}$}‚ ‚{\tiny $_{lb}$}‚ \leavevmode\ledsidenote{\textenglish{196/s}}श‚ब्द‚निवेशो युक्तोऽन‚र्थ‚क्रियाकारित्वात् [।]
	{\color{gray}{\rmlatinfont\textsuperscript{§~\theparCount}}}
	\pend% ending standard par
      ‚{\tiny $_{lb}$}‚

	  
	  \pstart \leavevmode% starting standard par
	स‚त्त्यं [।] केव‚ल‚म‚र्थ‚क्रियाकारित्वेनैव प्र‚तिभास‚नात्त‚त्र श‚ब्द‚निवेशो युक्त‚{\tiny $_{lb}$}‚ इति प्र‚तिपाद‚यिष्य‚ते ।
	{\color{gray}{\rmlatinfont\textsuperscript{§~\theparCount}}}
	\pend% ending standard par
      ‚{\tiny $_{lb}$}‚

	  
	  \pstart \leavevmode% starting standard par
	न त्वेव‚म‚पि त‚स्य ज्ञानाङ्श‚स्य स्व‚ल‚क्ष‚ण‚त्वात् क‚थं श‚ब्द‚वाच्य‚त्वं ।
	{\color{gray}{\rmlatinfont\textsuperscript{§~\theparCount}}}
	\pend% ending standard par
      ‚{\tiny $_{lb}$}‚

	  
	  \pstart \leavevmode% starting standard par
	अत्रोच्य‚ते । बाह्याभिन्न‚स्ताव‚त् स्वाङ्शो विक‚ल्पे प्र‚तिभास‚त एव [।] न‚{\tiny $_{lb}$}‚ ताव‚द‚स्य विक‚ल्प‚{\tiny $_{४}$}‚ग्राह्य‚त्वात् प्र‚तिभासः स‚र्वात्म‚ना निश्च‚य‚प्र‚स‚ङ्गाद‚न‚भ्युप‚ग‚माच्च ।
	{\color{gray}{\rmlatinfont\textsuperscript{§~\theparCount}}}
	\pend% ending standard par
      ‚{\tiny $_{lb}$}‚

	  
	  \pstart \leavevmode% starting standard par
	नापि विक‚ल्पेन बाह्यात्म‚त‚याध्य‚व‚साय एवास्य ग्र‚ह‚णं य‚थाव‚स्थितेन स्व‚रू‚{\tiny $_{lb}$}‚पेणाग्र‚ह‚णाद‚ग्र‚ह‚णे च क‚थ‚न्त‚त्र प्र‚तिभासः । ज्ञान‚स्व‚ल‚क्ष‚ण‚त्वे तु स्वांश‚स्य स‚म्वित्स्व‚{\tiny $_{lb}$}‚भाव‚त्वात् प्र‚तिभासो युक्तः । तेनाविद्यारूप‚स्य स्वांश‚स्य विक‚ल्प‚स्य च य‚दि ज्ञान‚{\tiny $_{lb}$}‚स्व‚ल‚क्ष‚ण‚त्वं नेष्य‚ते त‚दा प्र‚तिभास एव न स्यादे‚{\tiny $_{५}$}‚व‚म‚ज्ञान‚रूपेण च विक‚ल्पेन क‚थं‚{\tiny $_{lb}$}‚ स्वाङ्श‚स्य प‚रिच्छेदोस्य ज्ञान‚ध‚र्म‚त्वात् । त‚स्माज्ज्ञान‚स्व‚ल‚क्ष‚ण‚त्वादेव स्वांश‚स्य‚{\tiny $_{lb}$}‚ विक‚ल्पे प्र‚तिभासः स बाह्याभिन्नो विक‚ल्प‚विष‚यो व्य‚व‚स्थाप्य‚ते । त‚स्य स‚म्विदि‚{\tiny $_{lb}$}‚त‚रूप‚स्यैव बाह्याभेदेन विक‚ल्पेनाध्य‚व‚सीय‚मान‚त्वाद‚त एव विक‚ल्पः सामान्य‚विष‚य‚{\tiny $_{lb}$}‚ उच्य‚ते न स्व‚ल‚क्ष‚ण‚विष‚योऽर्थ‚स्वांश‚योरेक‚स्यापि स्व‚रूपेणाग्र‚ह‚णात्‚{\tiny $_{६}$}‚ । तेन स्वांश‚स्य‚{\tiny $_{lb}$}‚ ज्ञान‚स्व‚ल‚क्ष‚ण‚स्यापि बाह्यात्म‚त‚याध्य‚स्त‚स्य सामान्य‚रूप‚त्वं । त‚था च व‚क्ष्य‚ति ।
	{\color{gray}{\rmlatinfont\textsuperscript{§~\theparCount}}}
	\pend% ending standard par
      ‚{\tiny $_{lb}$}‚
	  \bigskip
	  \begingroup
	
	    
	    \stanza[\smallbreak]
	  {\normalfontlatin\large ``\qquad}ज्ञान‚रूप‚त‚यार्थ‚त्वे सामान्ये चेत् प्र‚स‚ज्य‚ते ।&‚{\tiny $_{lb}$}‚त‚थेष्ट‚त्वाद‚पोह्य‚र्थ‚रूप‚त्वेन स‚मान‚तेति । \href{http://sarit.indology.info/?cref=pv.2.9}{प्र० वा० ३ । ९}{\normalfontlatin\large\qquad{}"}\&[\smallbreak]
	  
	  
	  
	  \endgroup
	‚{\tiny $_{lb}$}‚

	  
	  \pstart \leavevmode% starting standard par
	त‚स्य च श‚ब्द‚वाच्य‚त्वं युक्त‚मेव ।
	{\color{gray}{\rmlatinfont\textsuperscript{§~\theparCount}}}
	\pend% ending standard par
      ‚{\tiny $_{lb}$}‚

	  
	  \pstart \leavevmode% starting standard par
	\textbf{ल‚क्षित‚ल‚क्ष‚णेत्या}दि प‚रः । स‚त्यं न सामान्य‚म‚र्थ‚क्रियाकारि किन्तु व्य‚क्तिरेव‚{\tiny $_{lb}$}‚ \leavevmode\ledsidenote{\textenglish{73b/PSVTa}} केव‚लं व्य\textbf{क्तेर‚श‚क्य‚चोद‚न‚त्वा}त् कार‚णात् सामान्ये नियुक्तः श‚ब्दः‚{\tiny $_{७}$}‚ सामान्यं‚{\tiny $_{lb}$}‚ ल‚क्ष‚य‚ति [।] तेन सामान्येन श‚ब्द‚ल‚क्षितेन स‚म्ब‚न्धाद् व्य‚क्तिर‚पि ल‚क्ष्य‚त इति ।
	{\color{gray}{\rmlatinfont\textsuperscript{§~\theparCount}}}
	\pend% ending standard par
      ‚{\tiny $_{lb}$}‚

	  
	  \pstart \leavevmode% starting standard par
	त‚देत‚द‚प्र‚तीतिकं । न हि गोश‚ब्दादुच्च‚रिताद् गोत्वं प्र‚तीय‚ते गौर‚पि तु गौरे‚{\tiny $_{lb}$}‚वाव‚सीय‚ते । न नामैव‚न्त‚थाप्युच्य‚ते । \textbf{अश‚ब्द‚चोदि}तेत्यादि ।
	{\color{gray}{\rmlatinfont\textsuperscript{§~\theparCount}}}
	\pend% ending standard par
      ‚{\tiny $_{lb}$}‚

	  
	  \pstart \leavevmode% starting standard par
	य‚दि नाम जातित‚द्व‚तोस्स‚म्ब‚न्ध‚स्त‚था\textbf{प्य‚श‚ब्द‚चोदिते} व्य‚क्तिविशेषे \textbf{क‚थं‚{\tiny $_{lb}$}‚ प्र‚व‚र्त्त‚ते} [।] नैव । द‚ण्ड‚द‚ण्डिनोस्स‚त्य‚पि स‚म्ब‚न्धे \textbf{न हि क‚श्चि}त् प्रे‚{\tiny $_{२}$}‚क्षापूर्व‚कारी‚{\tiny $_{lb}$}‚ \textbf{द‚ण्ड‚ञ्छिन्धीत्युक्ते द‚ण्डिन‚ञ्छिन‚त्ति} । अश‚ब्द‚चोदित‚त्वात्त‚था जातौ चोदितायां‚{\tiny $_{lb}$}‚ \textbf{व्य‚क्तौ प्र‚वृत्तिर्न} युक्तेत्य‚र्थः ।
	{\color{gray}{\rmlatinfont\textsuperscript{§~\theparCount}}}
	\pend% ending standard par
      ‚{\tiny $_{lb}$}‚

	  
	  \pstart \leavevmode% starting standard par
	जातौ वाह‚दोहादीनाम‚स‚म्भ‚वाद‚श‚ब्द‚चोदितायाम‚पि व्य‚क्तौ प्र‚वृत्तिर्भ‚{\tiny $_{lb}$}‚‚{\tiny $_{lb}$}‚ \leavevmode\ledsidenote{\textenglish{197/s}}विष्य‚तीति चेदाह । \textbf{नापी}त्यादि । केव‚ल‚म‚र्थान्त‚र‚स‚म्भ‚वि कार्य‚म‚स‚म्भ‚विन्य‚र्थे‚{\tiny $_{lb}$}‚ चोद‚य‚न् व‚क्ता प्र‚तिपाद्य‚स्या\textbf{स‚म्ब‚द्ध‚प्र‚लापी स्यात्} । न पुन‚स्त‚तोस‚म‚र्थाच्चोदितात्‚{\tiny $_{lb}$}‚ स‚म‚र्थे‚{\tiny $_{२}$}‚ प्र‚वृत्ति\textbf{र्ब‚लीव‚र्द‚दोह‚चोद‚नाव‚त्} [।] न हि केन‚चिद् ब‚लीव‚र्द‚न्दोग्धीत्युक्ते‚{\tiny $_{lb}$}‚ त‚त्र दोहा\textbf{स‚म्भ‚वा}त् । ब‚लीव‚र्दा\textbf{द‚न्य‚त्र} स‚म्भ‚वायां ग‚वि दोग्धुं \textbf{प्र‚व‚र्त्त‚ते} श्रोता [।]‚{\tiny $_{lb}$}‚ केव‚ल‚न्त\textbf{स्यैव}म्भ‚व‚त्य \textbf{\textbf{सं}ब‚द्ध‚प्र‚लाप्य‚यं} व‚क्तेति ।
	{\color{gray}{\rmlatinfont\textsuperscript{§~\theparCount}}}
	\pend% ending standard par
      ‚{\tiny $_{lb}$}‚

	  
	  \pstart \leavevmode% starting standard par
	स्यान्म‚तं [।] व‚लीव‚र्द‚चोद‚ने स‚म्ब‚न्धाभावात् मा भूत् स्त्रीग‚व्यां प्र‚वृत्तिः । जातौ‚{\tiny $_{lb}$}‚ तु चोदितायां स‚म्ब‚न्धात् त‚द्व्य‚क्तौ प्र‚वृत्तिर्भ‚विष्य‚तीत्याह । \textbf{न चे}त्यादि । \textbf{अर्था‚{\tiny $_{lb}$}‚न्त}र‚स्य‚{\tiny $_{३}$}‚ सामान्य‚स्य स‚म्ब‚न्ध‚स्यापि \textbf{चोद‚नेनार्थान्त\textbf{र}स्येति व्य‚क्तेः} ।
	{\color{gray}{\rmlatinfont\textsuperscript{§~\theparCount}}}
	\pend% ending standard par
      ‚{\tiny $_{lb}$}‚

	  
	  \pstart \leavevmode% starting standard par
	न‚न्व‚श‚ब्द‚चोदिते स‚त्य‚पि स‚म्ब‚न्ध इत्यादिनोक्त एवाय‚म‚र्थः ।
	{\color{gray}{\rmlatinfont\textsuperscript{§~\theparCount}}}
	\pend% ending standard par
      ‚{\tiny $_{lb}$}‚

	  
	  \pstart \leavevmode% starting standard par
	स‚त्य‚म् [।] अधिक‚विधानार्थ‚न्तु पुनः प्र‚स्तावः । त‚देव पूर्व‚प‚क्ष‚य‚ति । \textbf{अनिय‚{\tiny $_{lb}$}‚ते}त्यादि । \textbf{अनिय‚तः स‚म्ब‚न्}धो य‚स्य द‚ण्ड‚स्य स त‚थोक्तः । त‚था हि द‚ण्डिन‚म‚न्त‚रेणापि‚{\tiny $_{lb}$}‚ द‚ण्डे विद्य‚ते त‚द्भाव‚स्त‚स्मात् । \textbf{त‚त्रे}ति द‚ण्डिनि । \textbf{नेति चे}दिति द‚ण्डे चो‚{\tiny $_{४}$}‚दिते \textbf{प्र‚वृत्ति}र्न‚{\tiny $_{lb}$}‚ भ‚व‚तीत्य‚र्थः । जातौ तु चोदितायां निय‚त‚स‚म्ब‚न्धाद् व्य‚क्तौ प्र‚तीतिर्भ‚व‚तीति भावः ।
	{\color{gray}{\rmlatinfont\textsuperscript{§~\theparCount}}}
	\pend% ending standard par
      ‚{\tiny $_{lb}$}‚

	  
	  \pstart \leavevmode% starting standard par
	\textbf{त‚दित्या}दि सि द्धा न्त वा दी । \textbf{त‚दि}त्य‚निय‚त‚स‚म्ब‚द्ध‚त्वं \textbf{तुल्यं जाताव‚पि} ।
	{\color{gray}{\rmlatinfont\textsuperscript{§~\theparCount}}}
	\pend% ending standard par
      ‚{\tiny $_{lb}$}‚

	  
	  \pstart \leavevmode% starting standard par
	क‚थ‚मिति चेदाह । \textbf{व्य‚क्तीना}मित्यादि । स्यादेत‚द् [।] य‚था भ्रात्रादिश‚ब्दाः‚{\tiny $_{lb}$}‚ स्वार्थ‚म‚भिद‚धाना द्वितीय‚माक्षिप‚न्ति त‚था जातिश‚ब्दा इत्य‚त आह । \textbf{भ्रात्रादिश‚ब्दा‚{\tiny $_{lb}$}‚स्त्वि}त्यादि । आदिश‚ब्दात् पुत्रादि‚{\tiny $_{५}$}‚श‚ब्दाः । \textbf{स‚म्ब‚न्धिश‚ब्द}वाच्य\textbf{त्वात्} स‚म्ब‚न्ध्य‚{\tiny $_{lb}$}‚न्त‚रापेक्षैव तेषां व्य‚व‚स्थापित‚त्वादिति याव‚त् । \textbf{आक्षिपेयुः प‚र‚मिति} द्वितीयं‚{\tiny $_{lb}$}‚ भ्रात्रादिकं । आक्षेप‚श्च द्वितीय‚स‚म्ब‚न्धाकार‚विक‚ल्प‚ज‚न‚नं । न तूप‚स्थाप‚न‚मेव‚{\tiny $_{lb}$}‚ विन‚ष्टेपि स‚म्ब‚न्धिनि विक‚ल्पोत्प‚त्तेः । \textbf{न त‚थे}ति वैध‚र्म्य‚क‚थ‚नं । य‚दि स‚म्ब‚न्धिवाचिन्यः‚{\tiny $_{lb}$}‚ स्युस्त‚दाय‚न्दोषः स्यादित्याह । \textbf{अपेते}त्यादि । \textbf{अपे}ता विन‚{\tiny $_{६}$}‚ष्टा \textbf{व्य‚क्त}य‚स्स‚म्ब‚न्धित्वेन‚{\tiny $_{lb}$}‚ यासां पा ण्ड वा दि \textbf{जाती}नान्तासाम‚पि \textbf{त‚च्छ्रुतिभ्यो} जातिवाच‚केभ्यः श‚ब्देभ्यो‚{\tiny $_{lb}$}‚ \textbf{नित्य‚म}पेत‚व्य‚क्तिस‚म्ब‚न्धित्वेनाप्य\textbf{नुग‚म‚प्र‚संगः} । एवं ह्य‚पेत‚व्य‚क्तिस‚म्ब‚न्धित्वेन‚{\tiny $_{lb}$}‚ तासाम‚नुग‚मो य‚द्य‚पेतानाम‚पि व्य‚क्तीनाम‚नुग‚मः स्यात् । य‚द्वा जातीनां स‚म्ब‚न्धिभ्यो‚{\tiny $_{lb}$}‚ या अपेता व्य‚क्त‚य‚स्तासां त‚च्छ‚ब्देभ्यो नित्य‚म\textbf{नुग‚म‚न‚प्र‚स‚ङ्गात्} ।
	{\color{gray}{\rmlatinfont\textsuperscript{§~\theparCount}}}
	\pend% ending standard par
      ‚{\tiny $_{lb}$}‚‚{\tiny $_{lb}$}‚\textsuperscript{\textenglish{198/s}}

	  
	  \pstart \leavevmode% starting standard par
	\leavevmode\ledsidenote{\textenglish{74a/PSVTa}} \textbf{स‚र्व‚दे}त्या‚{\tiny $_{७}$}‚दि प‚रः । \textbf{स‚र्व‚दे}ति व्य‚क्त्य‚पायान‚षाय‚काल‚यो\textbf{स्त‚त्स‚म्ब‚न्ध‚योग्य‚ता‚{\tiny $_{lb}$}‚प्र‚तीतेरिष्ट‚मेव} व्य‚क्त्य‚नुग‚म‚न\textbf{मिति चेत्} । एत‚त्क‚थ‚य‚ति [।] य‚था भ्रात्रादिश‚ब्दाः‚{\tiny $_{lb}$}‚ स्वार्थ‚म‚भिद‚धानाः स‚म्ब‚न्धिन‚म‚विशेष‚णात् क्षिप‚न्ति त‚था जातिश‚ब्दा अपीति ।
	{\color{gray}{\rmlatinfont\textsuperscript{§~\theparCount}}}
	\pend% ending standard par
      ‚{\tiny $_{lb}$}‚

	  
	  \pstart \leavevmode% starting standard par
	उत्त‚र‚माह । \textbf{स‚र्व‚दे}ति । \textbf{स‚र्व‚दे}ति व्य‚क्त्य‚पायाऽन‚पाय‚काल‚यो\textbf{र्गोश‚ब्दाद‚प्र‚वृत्ति}‚{\tiny $_{lb}$}‚र्वाह‚दोदादियोग्ये व्य‚क्तिविशेषे । किङ्कार‚णंमिति‚{\tiny $_{१}$}‚ चेदाह । \textbf{स‚हिते}त्यादि ।‚{\tiny $_{lb}$}‚ स‚प्त‚मीद्विव‚च‚न‚मेत‚त् । जातेर्व्य‚क्ति\textbf{स‚हितास‚हिताव‚स्थ‚योर्विशेषेणा}र्थ‚क्रियाक्ष‚म‚स्य‚{\tiny $_{lb}$}‚ विशेष‚स्या\textbf{नाक्षेपा}त् ।‚{\tiny $_{२}$}‚ एत‚दुक्त‚म्भ‚व‚ति । य‚थाऽतीतानाग‚त‚व्य‚क्तेः श‚ब्दार्थ‚रूप‚त‚या‚{\tiny $_{lb}$}‚ जातिश‚ब्देनाक्षेप‚स्त‚था व‚र्त्त‚मानाया व्य‚क्तेरुभ‚य‚त्र श‚ब्दार्थ‚रूप‚त‚या प्र‚तिभास‚न‚स्या‚{\tiny $_{lb}$}‚विशेष‚णादिति ।
	{\color{gray}{\rmlatinfont\textsuperscript{§~\theparCount}}}
	\pend% ending standard par
      ‚{\tiny $_{lb}$}‚

	  
	  \pstart \leavevmode% starting standard par
	व्य\textbf{क्तिस‚म्ब‚न्धिन्या जातेश्चोद‚ना}द् व्य‚क्तौ प्र‚तीतिर्न प्राप्नोतीत्य‚य‚{\tiny $_{२}$}‚\textbf{म‚दोष‚{\tiny $_{lb}$}‚ इति चेत्} ॥
	{\color{gray}{\rmlatinfont\textsuperscript{§~\theparCount}}}
	\pend% ending standard par
      ‚{\tiny $_{lb}$}‚

	  
	  \pstart \leavevmode% starting standard par
	भ‚व‚त्वेवं किन्तु \textbf{सापि} व्य‚क्ति\textbf{स्त‚द्विशेष‚ण‚त्वेन} जातिविशेष‚ण‚त्वेन जाति‚{\tiny $_{lb}$}‚चोद‚नायामा\textbf{क्षिप्तैवेति} न जातिः केव‚लाभिधेया । किन्तु \textbf{त‚द्वान‚भिधेयः स्या}दिति‚{\tiny $_{lb}$}‚ प‚क्षान्त‚र‚प‚रिग्र‚हः स्यात् । त‚त्र चान‚न्त‚र‚मेव दोष‚म्व‚क्ष्याम इत्याकूतं ॥
	{\color{gray}{\rmlatinfont\textsuperscript{§~\theparCount}}}
	\pend% ending standard par
      ‚{\tiny $_{lb}$}‚

	  
	  \pstart \leavevmode% starting standard par
	\textbf{जातित‚द्व‚तोः} स‚म्ब‚न्ध‚म‚भ्युप‚ग‚म्यैत‚दुक्तं [।] \textbf{स‚म्ब‚न्ध एव} तु नास्तीत्याह ।‚{\tiny $_{lb}$}‚ \textbf{न चे}त्यादि । किं कार‚ण‚म् [।] \textbf{अन्योन्य}‚{\tiny $_{३}$}‚म्प‚र‚स्प‚र‚म\textbf{ज‚न्य‚ज‚न‚क‚त्वेनानुप‚कारात् ।‚{\tiny $_{lb}$}‚ त‚त} इति स‚म्ब‚न्धाभावाज्जातिचोद‚न‚या व्य‚क्तेर्ल\textbf{क्ष‚ण‚म‚प्युक्तं । फ‚लाभावा}दित्य‚र्थ‚{\tiny $_{lb}$}‚क्रियाया अभावात् ।
	{\color{gray}{\rmlatinfont\textsuperscript{§~\theparCount}}}
	\pend% ending standard par
      ‚{\tiny $_{lb}$}‚

	  
	  \pstart \leavevmode% starting standard par
	व्य‚क्त्य‚भिन्न‚सामान्य‚वादिनोपि प्र‚त्य‚क्ष‚व‚च्छाब्दे ज्ञाने व्य‚क्तिप्र‚तिभासः स्यात् ।‚{\tiny $_{lb}$}‚ भेदांशेन तु त‚स्यापि ल‚क्ष‚ण‚म‚युक्तं । \textbf{एव}मित्यादिना प‚क्षान्त‚र‚माश‚ङ्क‚ते । \textbf{त‚द्वानि}ति‚{\tiny $_{lb}$}‚ जातिमान् । \textbf{अल‚मि}ति स‚म‚{\tiny $_{४}$}‚र्थः । \textbf{त‚त्रे}ति त‚द्व‚ति । \textbf{स चे}ति सि द्धा न्त वा दी ।
	{\color{gray}{\rmlatinfont\textsuperscript{§~\theparCount}}}
	\pend% ending standard par
      ‚{\tiny $_{lb}$}‚

	  
	  \pstart \leavevmode% starting standard par
	अस्यैव व्याख्यानं \textbf{स च श‚ब्द} इत्यादि । \textbf{त‚त्रे}ति व्य‚क्तौ \textbf{किम‚न्येन} सामान्येन‚{\tiny $_{lb}$}‚ ‚{\tiny $_{lb}$}‚ \leavevmode\ledsidenote{\textenglish{199/s}}\textbf{व्य‚व‚धिना} व्य‚व‚धाय‚केन क‚ल्पितेन ।
	{\color{gray}{\rmlatinfont\textsuperscript{§~\theparCount}}}
	\pend% ending standard par
      ‚{\tiny $_{lb}$}‚
	    
	    \stanza[\smallbreak]
	  
	  \bigskip
	  \begingroup
	आन‚न्त्याच्चे
	  \endgroup
	ति प‚रः ।&‚{\tiny $_{lb}$}‚
	  \bigskip
	  \begingroup
	इद‚मान‚न्त्यं स‚म
	  \endgroup
	मित्युत्त‚रं ।\&[\smallbreak]
	  
	  
	  ‚{\tiny $_{lb}$}‚

	  
	  \pstart \leavevmode% starting standard par
	एत‚देव व्याच‚ष्टे । \textbf{स्यादेत}दित्यादिना । \textbf{त‚द्व‚त्य}पीति जातिम‚त्य‚पि । य‚स्मा‚{\tiny $_{lb}$}‚\textbf{ज्जात्या}दिवि\textbf{शिष्टाः} स‚त्यो व्\textbf{य‚क्त‚य एव} व\textbf{क्त‚व्}या \textbf{इति} हेतोर\textbf{व‚{\tiny $_{५}$}‚श्य‚न्त‚त्रे}ति व्य‚क्तिषु‚{\tiny $_{lb}$}‚ श‚ब्द‚स्य \textbf{स‚म्ब‚न्धः क‚र‚णीयः} [।] क‚स्माद् [।] \textbf{अकृत‚स‚म्ब‚न्ध‚स्यान‚भिधाना}त् ।‚{\tiny $_{lb}$}‚ क‚र्त्त‚रि ष‚ष्ठी । कृतः स‚म्ब‚न्धो य‚स्य श‚ब्द‚स्य । तेनान‚भिधानादित्य‚र्थः । क‚र्म‚णि‚{\tiny $_{lb}$}‚ वा ष‚ष्ठी । अकृत‚स‚म्ब‚न्ध‚स्य वार्थ‚स्य श‚ब्देनान‚भिधानात् । \textbf{स चे}ति स‚म्ब‚न्धः ।‚{\tiny $_{lb}$}‚ त‚द्व‚ता स‚ह \textbf{न} श\textbf{क्यं} क‚र्त्तुमान‚न्त्यात् । त‚स्माद‚युक्तोय‚म्प‚क्षः ।
	{\color{gray}{\rmlatinfont\textsuperscript{§~\theparCount}}}
	\pend% ending standard par
      ‚{\tiny $_{lb}$}‚

	  
	  \pstart \leavevmode% starting standard par
	\textbf{त‚त्स‚म्ब‚न्धिनि} । व्य‚क्तिस‚म्ब‚न्धिनि सामान्ये‚{\tiny $_{६}$}‚ श‚ब्द‚स्य स‚म्ब‚न्ध\textbf{क‚र‚णा}द्धेतो‚{\tiny $_{lb}$}‚स्त\textbf{त्रापि} जातिस‚म्ब‚न्धिभ्यां व्य‚क्तौ स‚म्ब‚न्धः \textbf{कृत एवेति चे}त् ।
	{\color{gray}{\rmlatinfont\textsuperscript{§~\theparCount}}}
	\pend% ending standard par
      ‚{\tiny $_{lb}$}‚

	  
	  \pstart \leavevmode% starting standard par
	\textbf{उक्त‚म‚त्रो}त्त‚रं [।] सामान्य‚स्य \textbf{स‚त्य‚पि स‚म्ब‚न्धे एक‚त्र} जातौ \textbf{कृतात्} संकेता‚{\tiny $_{lb}$}‚\textbf{द‚न्य‚त्र} व्य‚क्ता\textbf{व‚प्र‚तीति}र्न च जातित‚द्व‚तोः स‚म्ब‚न्धोस्तीत्येत‚च्चोक्तं । न हि स‚त्य‚पि‚{\tiny $_{lb}$}‚ स‚म्ब‚न्धे द‚ण्ड‚श‚ब्दाद् द‚ण्डिनि प्र‚तिप‚त्तिः [।] त‚था \textbf{न च} जातित‚द्व‚तोः क‚श्चित्\textbf{स‚म्ब‚न्धो‚{\tiny $_{lb}$}‚स्तीति} स‚म्प्र‚त्युक्त‚त्वात् ॥
	{\color{gray}{\rmlatinfont\textsuperscript{§~\theparCount}}}
	\pend% ending standard par
      ‚{\tiny $_{lb}$}‚

	  
	  \pstart \leavevmode% starting standard par
	एव‚न्ताव‚त् \textbf{स‚र्व‚भावा} इत्यादिना वा र्त्ति क का रः स्वाभिम‚तं विधिश‚ब्द‚लिंग- \leavevmode\ledsidenote{\textenglish{74b/PSVTa}}‚{\tiny $_{lb}$}‚ विष‚य‚माख्याय संप्र‚ति येनाभिप्रायेणाचार्य दि ग्ना गे न भेद‚ल‚क्ष‚णं सामान्य‚मुक्त‚न्त‚{\tiny $_{lb}$}‚‚{\tiny $_{lb}$}‚ \leavevmode\ledsidenote{\textenglish{200/s}}द्द‚र्श‚यितुं पृच्छ‚ति । \textbf{अपि चे}त्यादि । एव‚म्म‚न्य‚ते । य‚था गोश‚ब्दाद‚प्र‚तीय‚माने‚{\tiny $_{lb}$}‚ गोत्वे गोश‚ब्दः संकेत्य‚ते त‚था । \textbf{त‚त्कारिणां} विव‚क्षितार्थ‚क्रियाकारिणाम\textbf{त‚त्कारि}भ्यो‚{\tiny $_{lb}$}‚ ये विव‚क्षिता‚{\tiny $_{१}$}‚र्थ‚क्रियाकारिणो न भ‚व‚न्ति तेभ्यो यो \textbf{भेद}स्तेन \textbf{सामा}न्यं स‚र्वेषान्त‚त्का‚{\tiny $_{lb}$}‚रिणाम‚त‚त्कारिभ्यो भिन्न‚त्वाद‚त‚स्त‚स्मि\textbf{न्भेद‚साम्ये} अन्यापोह‚ल‚क्ष‚णे \textbf{किन्न कृतः} क‚स्मात्‚{\tiny $_{lb}$}‚ संकेतो न कृत इति पृच्छ‚ति प‚रं । एत‚देवाह । \textbf{याम‚र्थ‚क्रिया}मित्यादि । दाह‚पाकादि‚{\tiny $_{lb}$}‚ल‚क्ष‚ण‚स्यार्थ‚स्य क्रियां निष्प‚त्ति\textbf{म‚धिकृत्या}भिप्रेत्यायं पुरुषोर्थेष्व‚भिप्रेतार्थ‚क्रियाकारिष‚{\tiny $_{lb}$}‚ \textbf{श‚ब्दान्नि‚{\tiny $_{२}$}‚युङ्क्ते} प्र‚युङ्क्ते । \textbf{त‚त्कारिणा}म‚भिप्रेतार्थ‚क्रियाकारिणा\textbf{म‚न्येभ्यो}ऽत‚त्कारिभ्यो‚{\tiny $_{lb}$}‚ \textbf{भेदा}त् कार‚णात् त‚त्कारिणः स‚र्व‚विजातीय‚व्यावृत्ता अभिन्ना भ‚व‚न्ति । त‚त‚{\tiny $_{lb}$}‚ \textbf{एषा}म‚र्थाना\textbf{न्त‚त्रैवाभेद इति} अन्य‚व्यावृत्तिल‚क्ष‚णे \textbf{किन्न श‚ब्दः प्र‚युज्य‚ते} । व्यावृत्ति‚{\tiny $_{lb}$}‚विशिष्ट‚स्यापि संकेत‚व‚शात् प्र‚तीतिः स्यादिति प्र‚श्नाभिप्रायः ।
	{\color{gray}{\rmlatinfont\textsuperscript{§~\theparCount}}}
	\pend% ending standard par
      ‚{\tiny $_{lb}$}‚

	  
	  \pstart \leavevmode% starting standard par
	\textbf{त‚द्व‚दित्}यादि । जातिम‚त्प‚क्षे यो दोष आचार्य दि ग्ना गे नो क्त‚{\tiny $_{३}$}‚ स्त‚द्व‚तो‚{\tiny $_{lb}$}‚ नास्व‚त‚न्त्र‚त्वादित्यादिना य‚स्त‚द्व‚द्\textbf{दोष}स्त\textbf{स्य साम्या}त्त‚स्य दोष‚स्याव‚ताराद् भेदेन्य‚{\tiny $_{lb}$}‚व्यावृत्तिल‚क्ष‚णे श‚ब्दो न नियुज्य‚ते । \textbf{अस्त्व‚य‚न्दोष} इत्य‚भ्युप‚ग‚च्छ‚ति । नैवाय‚न्दो‚{\tiny $_{lb}$}‚षोस्तीति प्र‚तिपादित‚म‚भ्युग‚म्य त्वेव‚मुक्तं । \textbf{जातिर‚ल‚म्प‚रा} । जातिस्त्व‚न्या न‚{\tiny $_{lb}$}‚ युक्तेत्य‚र्थः ।
	{\color{gray}{\rmlatinfont\textsuperscript{§~\theparCount}}}
	\pend% ending standard par
      ‚{\tiny $_{lb}$}‚

	  
	  \pstart \leavevmode% starting standard par
	\textbf{स्यादि}त्यादिना व्याच‚ष्टे । \textbf{अन्य‚स्माद्} अत‚त्कारिणो \textbf{व्यावृत्तेपि} व‚स्तुनि \textbf{श‚ब्दा‚{\tiny $_{lb}$}‚र्थे}भ्यु‚{\tiny $_{४}$}‚प‚ग‚म्य‚माने । \textbf{त‚द्व‚त} इति व्यावृत्तिम‚तः । न \textbf{त}द्व\textbf{त्प‚क्षाद}स्य व्यावृत्तिम‚त्प‚क्ष‚स्य‚{\tiny $_{lb}$}‚ \textbf{विशेषः} । एत‚देवाह । \textbf{को ही}त्यादि । त्व‚या व्यावृत्तिरित्युक्तं प‚रेण जातिरित्यादि ।‚{\tiny $_{lb}$}‚ \textbf{अत्र} वाच्ये \textbf{को विशे}षः [।] नैव क‚श्चिद‚न्य‚त्र श‚ब्द‚भेदात् ॥
	{\color{gray}{\rmlatinfont\textsuperscript{§~\theparCount}}}
	\pend% ending standard par
      ‚{\tiny $_{lb}$}‚

	  
	  \pstart \leavevmode% starting standard par
	\textbf{अस्तु ना}मेति सि द्धा न्त वा दी । \textbf{जातिर}न्येति व‚स्तुभूता । किम्पुन‚स्तुल्ये दोषे‚{\tiny $_{lb}$}‚ व्यावृत्तिर‚ङ्गीक्रिय‚ते न व‚स्तुभूता जातिरित्याह । \textbf{जाति‚{\tiny $_{५}$}‚म‚पि ही}त्यादि । \textbf{त‚द‚भाव}‚{\tiny $_{lb}$}‚ ‚{\tiny $_{lb}$}‚ \leavevmode\ledsidenote{\textenglish{201/s}}इति व्यावृत्त्य‚भावे \textbf{अस्या} अपीति व‚स्तुभूताया जातेः । भावानां भेदाभावे‚{\tiny $_{lb}$}‚ स‚त्य‚नेकार्थ‚स‚म‚वेत‚रूपाया जातेर‚भावात् । न च जात्याभेदः क्रिय‚त इत्युक्तं ।‚{\tiny $_{lb}$}‚ त‚स्माद‚व‚श्य‚म्भेदोभ्युप‚ग‚न्त‚व्यः [।] स \textbf{चैक‚स्मा}द‚त‚त्कार्या[त् ?] \textbf{भेद}स्त\textbf{द‚न्येषा}न्त‚{\tiny $_{lb}$}‚स्माद‚त‚त्कार्याद‚न्येषान्त‚त्कार्याणा\textbf{म‚भेद}स्त\textbf{द्विशिष्टे}ष्व‚भेद‚विशिष्टेषु \textbf{प्र‚ति‚{\tiny $_{६}$}‚प‚त्तिर‚स्तु} ।‚{\tiny $_{lb}$}‚ प्र‚तिप‚त्त्याल‚म्ब‚न‚त्वात् प्र‚तिप‚त्तिरित्युक्तः । न पुनः स एव प्र‚तिप‚त्तिः । क‚र‚ण‚साध‚नो‚{\tiny $_{lb}$}‚ वा प्र‚तिप‚त्तिश‚ब्दः प्र‚तिप‚द्य‚न्तेऽन‚या व्यावृत्त्या क‚र‚ण‚भूत‚या भावानिति कृत्वा ।
	{\color{gray}{\rmlatinfont\textsuperscript{§~\theparCount}}}
	\pend% ending standard par
      ‚{\tiny $_{lb}$}‚

	  
	  \pstart \leavevmode% starting standard par
	\textbf{स‚र्व‚थे}ति । य‚दि व्यावृत्तिविशिष्टो जातिविशिष्टो वार्थो वाच्य‚स्त‚द्व‚त्प‚क्षोदितो‚{\tiny $_{lb}$}‚ य‚थान‚न्त्यादि\textbf{दोष}स्त‚त्\textbf{प‚रिहार‚स्य क‚र्त्तुम‚श‚क्य‚त्वात्} । तुल्य‚श्चेद्दोषो जातिरेव‚{\tiny $_{७}$}‚‚{\tiny $_{lb}$}‚ क‚स्मान्नाभ्युप‚ग‚म्य‚त इति चेदाह । \textbf{अर्थान्त}रेत्यादि । \textbf{अर्थान्त‚र}म्व‚स्तुभूता जातिः । \leavevmode\ledsidenote{\textenglish{75a/PSVTa}}‚{\tiny $_{lb}$}‚ भिन्नेष्व‚भिन्न‚प्र‚त्य‚य‚ज‚न‚नं जातेः प्र‚योज‚न‚मिति चेदाह । \textbf{त‚द‚र्थ}स्येति जातिसाध्य‚स्य ।‚{\tiny $_{lb}$}‚ \textbf{अन्येने}त्य‚त‚त्कार्य‚व्यावृत्तिल‚क्ष‚णेनाभेदेन [।] जात्यापि हि सोर्थः साध्य‚त इति क‚स्मा‚{\tiny $_{lb}$}‚ज्जातित्यागे व्यावृत्त्य‚भ्युप‚ग‚म इत्य‚त आह । त‚दित्यादि । \textbf{त‚द‚भ्युप‚ग‚म‚स्ये}ति व्या‚{\tiny $_{lb}$}‚वृत्त्य‚भ्युप‚ग‚म‚स्य [।] त‚{\tiny $_{१}$}‚द‚न‚भ्युप‚ग‚मे हि जातिक‚ल्प‚नैव न स्यादित्युक्तं ॥
	{\color{gray}{\rmlatinfont\textsuperscript{§~\theparCount}}}
	\pend% ending standard par
      ‚{\tiny $_{lb}$}‚

	  
	  \pstart \leavevmode% starting standard par
	अधुना श‚ब्देनाव‚श्यं व्यावृत्तिश्चोद‚नीयेत्य‚त आह । \textbf{अपि चे}त्यादि । विव‚क्षि‚{\tiny $_{lb}$}‚ताद‚र्था\textbf{द‚न्य}स्त‚स्य \textbf{प‚रिहारेण} श्रोता \textbf{प्र‚व‚र्तेतेति} कृत्वा \textbf{ध्व‚निरुच्य‚ते} प्र‚तिपाद‚केन ।‚{\tiny $_{lb}$}‚ च‚कार एव‚श‚ब्द‚स्यार्थे भिन्न‚क्र‚म‚श्च त‚द‚न्य‚प‚रिहारेणेत्य‚स्यान‚न्त‚रं द्र‚ष्ट‚व्यं । \textbf{तेने}ति‚{\tiny $_{lb}$}‚ ध्व‚निना । \textbf{तेभ्य} इत्य‚न‚भिम‚तेभ्य‚स्त‚स्याभिम‚त‚स्यार्थ‚स्या‚{\tiny $_{२}$}‚ \textbf{व्य‚व‚च्छेदे}ऽव्य‚व‚च्छेदेनाभि‚{\tiny $_{lb}$}‚धीय‚मान इत्य‚र्थः [।] \textbf{क‚थं} श्रोता \textbf{प्र‚व‚र्त्ते}तेति ।
	{\color{gray}{\rmlatinfont\textsuperscript{§~\theparCount}}}
	\pend% ending standard par
      ‚{\tiny $_{lb}$}‚

	  
	  \pstart \leavevmode% starting standard par
	\textbf{श‚ब्द}मित्यादिनैत‚देव व्याच‚ष्टे । \textbf{एष} व‚क्तार्थेष्व‚भिम‚तार्थ‚क्रियाकारि\textbf{ष्व‚निष्ट‚{\tiny $_{lb}$}‚प‚रिहारे}णान‚भिम‚तार्थ‚व्य‚व‚च्छेदेन \textbf{प्र‚व‚र्त्तेत} क‚थं नाम श्रोतेत्य‚नेना\textbf{भिप्रायेण‚{\tiny $_{lb}$}‚ प्र‚युङ्क्ते} ।
	{\color{gray}{\rmlatinfont\textsuperscript{§~\theparCount}}}
	\pend% ending standard par
      ‚{\tiny $_{lb}$}‚

	  
	  \pstart \leavevmode% starting standard par
	य‚दि श‚ब्देनान्य‚व्य‚व‚च्छेदो न चोद्येत त‚दा \textbf{त‚त्र} प्र‚त्याय्याभिम‚तेऽ\textbf{न्य‚त्र चे}त्य‚न‚भिम‚ते‚{\tiny $_{lb}$}‚ ‚{\tiny $_{lb}$}‚ \leavevmode\ledsidenote{\textenglish{202/s}}प्र‚वृत्तिर‚नुज्ञा‚{\tiny $_{३}$}‚ता भ‚व‚ति । त‚स्यां च स‚त्यान्त‚स्याभिम‚त‚स्यार्थ‚स्य य\textbf{न्नाम अभिधान}‚{\tiny $_{lb}$}‚न्त‚स्य \textbf{ग्र‚ह‚ण‚वैय‚र्थ्य‚प्र‚स‚ङ्गात्} । त‚था ह्यान‚येत्युक्ते व‚स्तुमात्र‚माक्षिप्त‚न्त‚त्रान‚भिम‚त‚{\tiny $_{lb}$}‚व्य‚व‚च्छेदायाग्निश‚ब्दः प्र‚युज्य‚तेऽग्निमान‚येति । य‚दि तु त‚स्मिन्न‚पि प्र‚युक्तेऽन‚ग्न्यान‚य‚नं‚{\tiny $_{lb}$}‚ न व्य‚व‚च्छिन्न‚न्त‚दाग्निश‚ब्देनोच्चारितेन न किंचित् प्र‚योज‚नं । आन‚येत्य‚नेनाप्य‚{\tiny $_{lb}$}‚नान‚य‚न‚स्य‚{\tiny $_{४}$}‚ प्र‚तिक्षेपादान‚य‚न‚म‚नान‚य‚नं चानुज्ञातं स्याद‚तः प्र‚वृत्तिनिवृत्त्यानुज्ञायां च‚{\tiny $_{lb}$}‚ स‚त्यान्त\textbf{न्नाम‚ग्र‚ह‚ण‚वैय‚र्थ्य‚प्र‚संगादि}ति पूर्वेणैव स‚म्ब‚न्धः । एवं एक‚स्याभिम‚त‚स्या‚{\tiny $_{lb}$}‚ग्न्यादे\textbf{रेक}स्य चान‚य‚नादिल‚क्ष‚ण‚स्यानुष्ठान‚स्य या \textbf{चोद‚ना} त‚स्या \textbf{अनाद‚राद‚व‚च‚न‚मे}व‚{\tiny $_{lb}$}‚ श‚ब्दानां \textbf{स्यात् । अन्य‚व्यावृत्त्य‚न‚भिधाने} स‚ति ।
	{\color{gray}{\rmlatinfont\textsuperscript{§~\theparCount}}}
	\pend% ending standard par
      ‚{\tiny $_{lb}$}‚

	  
	  \pstart \leavevmode% starting standard par
	अथ‚वा \textbf{प्र‚वृत्तिनिवृत्त्य‚नु‚{\tiny $_{५}$}‚ज्ञाया}मिति व‚क्ष्य‚माणेन स‚म्ब‚न्ध‚नीयं । य‚थोक्त‚वि‚{\tiny $_{lb}$}‚धिना प्र‚वृत्तिनिवृत्त्य‚नुज्ञायां \textbf{चैक‚चोद‚नाऽनाद‚रा}त् । एक‚स्य प्र‚वृत्तिल‚क्ष‚ण‚स्य निवृ‚{\tiny $_{lb}$}‚त्तिल‚क्ष‚ण‚स्य वा व्यापार‚स्य चोद‚नाऽनाद‚रादिति [।] शेषं पूर्व‚व‚द् व्याख्येयं ।
	{\color{gray}{\rmlatinfont\textsuperscript{§~\theparCount}}}
	\pend% ending standard par
      ‚{\tiny $_{lb}$}‚

	  
	  \pstart \leavevmode% starting standard par
	एवं च श‚ब्द‚व्य‚व‚हारोच्छेदः स्यान्न चैव‚न्त\textbf{स्माद‚व‚श्य}मित्यादि । \textbf{स चे}त्य‚न्य‚{\tiny $_{lb}$}‚व्य‚व‚च्छेदः । \textbf{त‚द‚न्येष्}विति त‚स्माद‚त‚त्कार्याद‚न्ये‚{\tiny $_{६}$}‚ष्वेक‚कार्येष्व\textbf{भिन्नः} । स‚र्वेषाम‚त‚{\tiny $_{lb}$}‚त्कार्याद् व्यावृत्त‚त्वात् । इति कृत्वानेकार्थ‚व्यावृत्तित्वं \textbf{जातिध‚र्मोप्य‚स्ति} व्य‚व‚च्छे‚{\tiny $_{lb}$}‚द‚स्य । \textbf{त‚मि}ति व्य‚व‚च्छेद‚ङ‚किम्विशिष्टं \textbf{निय‚त}म‚भ्युप‚ग‚मो य‚स्य स त‚था । त‚द‚न‚भ्यु‚{\tiny $_{lb}$}‚प‚ग‚मे जातेर‚भाव‚प्र‚स‚ङ्गात् । \textbf{निय‚तं चोद‚न‚म}भिधानं य‚स्य त‚त्त‚था । त‚द‚चोद‚ने श‚ब्द‚{\tiny $_{lb}$}‚\leavevmode\ledsidenote{\textenglish{75b/PSVTa}} प्र‚योग‚स्य नैष्फ‚ल्यं स्यात् । व्य‚क्तिषु बुद्धिश‚ब्द‚योर‚{\tiny $_{७}$}‚नुग‚म‚ल‚क्ष‚णो \textbf{जात्य}र्थ‚स्त‚स्य \textbf{प्र‚सा‚{\tiny $_{lb}$}‚\textbf{ध}नं} प्र‚साध्य‚तेऽनेनेति कृत्वा । एवंभूतं व्य‚व‚च्छेदं \textbf{प‚रित्य‚ज्यार्थान्त}र‚स्य सामान्य‚स्य‚{\tiny $_{lb}$}‚ \textbf{क‚ल्प‚नं केव‚ल‚म‚न‚र्थ‚निर्ब‚न्ध} एवाऽव‚स्त्व‚भिनिवेश एव केव‚लं नान्य‚त् किञ्चित कार‚ण‚{\tiny $_{lb}$}‚म‚स्तीत्य‚र्थः । किङ्कार‚णं । [।] य‚था क‚ल्प‚नं नित्य‚व्यापिताद्य‚कारैर‚स्य सामान्य‚स्या‚{\tiny $_{lb}$}‚\textbf{योगा}दित्य‚न्य‚व्यावृत्त्य‚भिधानेऽय‚म‚भिप्राय आचार्य दि ग्ना ग स्य ॥
	{\color{gray}{\rmlatinfont\textsuperscript{§~\theparCount}}}
	\pend% ending standard par
      ‚{\tiny $_{lb}$}‚

	  
	  \pstart \leavevmode% starting standard par
	\textbf{नेत्या}दि प‚रः । \textbf{न वै न क्रिय‚त एव} श‚ब्देन \textbf{व्य‚व‚च्छेदः} किन्तु क्रिय‚त एव ।‚{\tiny $_{lb}$}‚ किम‚र्थ‚न्त‚र्हि जातिरुक्तेत्य‚त आह । \textbf{प्र‚वृत्ती}त्यादि अर्थ‚क्रियार्थिनो हि या \textbf{प्र‚वृत्ति}‚{\tiny $_{lb}$}‚स्त‚स्या \textbf{विष}यो जातिः । तं \textbf{क‚थ‚य‚द्भि}र‚स्माभिर्जातिरुक्ता ।
	{\color{gray}{\rmlatinfont\textsuperscript{§~\theparCount}}}
	\pend% ending standard par
      ‚{\tiny $_{lb}$}‚

	  
	  \pstart \leavevmode% starting standard par
	\textbf{व्य‚व‚च्छेदे}त्यादि सि द्धा न्त वा दी । अस्य श‚ब्द‚स्याभिधेयो \textbf{व्य‚व‚च्छेदोस्ति चेत्} ।‚{\tiny $_{lb}$}‚ ‚{\tiny $_{lb}$}‚ \leavevmode\ledsidenote{\textenglish{203/s}}अस्य वा जातिम‚तो व्य‚व‚च्छेदः श‚ब्द‚वाच्योस्ति चेत् । \textbf{न‚न्वेताव‚द}न्य‚{\tiny $_{२}$}‚व्य‚व‚च्छेदे‚{\tiny $_{lb}$}‚नेष्ट‚प्र‚व‚र्त्त‚नं \textbf{प्र‚योज‚नं श‚ब्दानामि}ति कृत्वा । \textbf{त‚त्रे}ति व्य‚व‚च्छेदेनेष्ट‚प्र‚व‚र्त्त‚ने क‚र्त्त‚व्ये ।‚{\tiny $_{lb}$}‚ त‚त्र वा प्र‚वृत्तिविष‚ये \textbf{किं सामान्येनाप‚रेण वः} प्र‚योज‚नं ॥
	{\color{gray}{\rmlatinfont\textsuperscript{§~\theparCount}}}
	\pend% ending standard par
      ‚{\tiny $_{lb}$}‚

	  
	  \pstart \leavevmode% starting standard par
	\textbf{न‚न्वि}त्यादि प‚रः । \textbf{उक्त}मित्या चा र्यः । \textbf{त‚था ही}त्य‚युक्त‚त्व‚प्र‚तिपाद‚नं । सेति‚{\tiny $_{lb}$}‚ जातिः । अर्थ‚क्रियां प्र‚त्य‚साम‚र्थ्या\textbf{न्न प्र‚वृत्तियोग्या} जातिः । नापि गोश‚ब्दाज्जातिः‚{\tiny $_{lb}$}‚ प्र‚तीय‚ते । \textbf{निवेदित‚मेत}‚{\tiny $_{३}$}‚दिति त\textbf{त्रान‚र्थ‚क्रियायोग्या जातिरि}\href{http://sarit.indology.info/?cref=pv-pandey.3.94}{ प्र० स०}त्यादिना ।‚{\tiny $_{lb}$}‚ जातिद्वारेण च द्र‚व्येऽर्थ‚क्रियास‚म‚र्थे पुरुष‚स्य प्र‚वृत्तिर्भ‚विष्य‚तीति चेदाह ।‚{\tiny $_{lb}$}‚ \textbf{त‚द्द्वारे}णेत्यादि । अश‚ब्द‚चोदिते स‚त्\textbf{य‚पि} स‚म्ब‚न्धे क‚थं प्र‚व‚र्त्तेतेत्यादिना ।‚{\tiny $_{lb}$}‚ जातिद्वारेण त‚द्वानेव चोद्य‚त इति चेदाह । \textbf{त‚द्व‚च्चोद‚न} इत्यादि । व्य‚व‚धान‚{\tiny $_{lb}$}‚मुक्त‚मिति लिङ्ग‚प‚रिणामेन स‚म्ब‚न्धः । सामान्येन त‚त्र व्य‚व‚धान‚मि‚{\tiny $_{४}$}‚त्युक्तं ।‚{\tiny $_{lb}$}‚ स च साक्षान्न योज्य‚ते क‚स्मात् [।] किन्त‚त्रान्येन व्य‚व‚धानेत्यादिना ।
	{\color{gray}{\rmlatinfont\textsuperscript{§~\theparCount}}}
	\pend% ending standard par
      ‚{\tiny $_{lb}$}‚

	  
	  \pstart \leavevmode% starting standard par
	स्यान्म‚तं [।] न जातिः केव‚ला व्य‚क्तिर्वा श‚ब्दाश्र‚याः प्र‚वृत्तेराश्र‚यः केव‚लाया‚{\tiny $_{lb}$}‚ जात‚रेर्थ‚क्रियायाम‚साम‚र्थ्यं । व्य‚क्तेश्च केव‚लाया अश‚क्य‚चोद‚न‚त्वात् । त‚स्माज्जाति‚{\tiny $_{lb}$}‚त‚द्व‚न्तौ स‚हितौ प्र‚वृत्तिविष‚य‚स्त‚योरेव स‚म‚स्त‚योः श‚ब्दार्थ‚त्व‚मित्य‚त आह । \textbf{जाति‚{\tiny $_{lb}$}‚त‚द्व‚तो}रित्यादि । \textbf{प्र‚वृत्ति‚{\tiny $_{५}$}‚विष}य‚त्व इति श‚ब्दाश्र‚यायाः प्र‚वृत्तेर्विष‚य‚त्वेभ्युप‚ग‚म्य‚{\tiny $_{lb}$}‚माने । \textbf{व्यावृत्तित‚द्व‚न्तौ किन्नेष्ये}ते प्र‚वृत्तिविष‚य‚त्वेनेत्य‚ध्याहारः । प्र‚माण‚सिद्धौ हि‚{\tiny $_{lb}$}‚ व्यावृत्तित‚द्व‚न्ताविति भावः । \textbf{व्यावृत्तेः} श‚ब्द‚भूतायाः बुद्धिप‚रिक‚ल्पित‚त्वा\textbf{द‚व‚स्तुत्व}म‚तो‚{\tiny $_{lb}$}‚ वाह‚दोहाद्य‚र्थ‚क्रियांप्र‚त्\textbf{य‚साध‚न‚त्वा}न्न प्र‚वृत्तिविष‚य‚त्व‚मिति चेत् । \textbf{त‚देत}द‚साध‚न‚त्वं‚{\tiny $_{lb}$}‚ \textbf{जातेस्तुल्यं} सापि‚{\tiny $_{६}$}‚ वाह‚दोहादाव‚स‚म‚र्था ।
	{\color{gray}{\rmlatinfont\textsuperscript{§~\theparCount}}}
	\pend% ending standard par
      ‚{\tiny $_{lb}$}‚

	  
	  \pstart \leavevmode% starting standard par
	\textbf{त‚द्व}त इति जातिम‚तोर्थ‚क्रिया\textbf{साध‚ना}त् प्र‚वृत्त्य‚भाव‚ल‚क्ष‚णो \textbf{दोषो न भ‚व‚तीति‚{\tiny $_{lb}$}‚ चेत्} ।
	{\color{gray}{\rmlatinfont\textsuperscript{§~\theparCount}}}
	\pend% ending standard par
      ‚{\tiny $_{lb}$}‚‚{\tiny $_{lb}$}‚‚{\tiny $_{lb}$}‚\textsuperscript{\textenglish{204/s}}

	  
	  \pstart \leavevmode% starting standard par
	\textbf{तुल्य‚मित्यादि} सि द्धा न्त वा दी । \textbf{त‚दि}ति अर्थ‚क्रियासाध‚न‚त्वं ।
	{\color{gray}{\rmlatinfont\textsuperscript{§~\theparCount}}}
	\pend% ending standard par
      ‚{\tiny $_{lb}$}‚

	  
	  \pstart \leavevmode% starting standard par
	एव‚न्ताव‚त्प्र‚तिब‚न्ध‚क‚न्यायेनाविद्य‚मानाया अपि व्यावृत्तेः स‚द्भाव‚म‚भ्युप‚ग‚म्य‚{\tiny $_{lb}$}‚ श‚ब्दार्थ‚त्व‚मुक्त‚माचार्य दि ग् ना गेनेति व्याख्याय पुन‚र्विधिमेव श‚ब्दार्थ‚माह । \textbf{अव‚{\tiny $_{lb}$}‚\leavevmode\ledsidenote{\textenglish{76a/PSVTa}} स्तुग्रा‚{\tiny $_{७}$}‚ही} चेत्यादि । य‚द्य‚व‚स्तुविष‚यः क‚थ‚म‚र्थ‚क्रियार्थिनं पुरुषं प्र‚व‚र्त्त‚य‚तीति चेदाह ।‚{\tiny $_{lb}$}‚ \textbf{स विभ्र‚मे}त्यादि । स इति शाब्द‚प्र‚त्य‚यः । \textbf{विभ्र‚म‚व‚शा}त् पुरुषं प्र‚व‚र्त्त‚य‚ति । विभ्र‚म एव‚{\tiny $_{lb}$}‚ क‚थ‚मित्याह[।]\textbf{अकार‚केपि} स्व‚प्र‚तिभासेर्थ‚क्रियायोग्य‚त्वात् \textbf{कार‚को} बाह्योर्थ\textbf{स्त‚द‚ध्य‚{\tiny $_{lb}$}‚व‚सा}यी य‚तः । क‚थ‚न्त‚र्ह्य‚नुमानादेर्व‚स्तुस‚म्वाद इत्य‚त आह । \textbf{व‚स्तुस‚म्वाद} इत्यादि ।‚{\tiny $_{lb}$}‚ त‚स्मिन् साध्ये \textbf{प्र‚तिब‚{\tiny $_{१}$}‚न्धे स‚ति}[।]प्र‚तिब‚न्ध एव कुतः । \textbf{व‚स्तूत्प‚त्त्या} साध्य‚व‚स्तू‚{\tiny $_{lb}$}‚त्प‚त्त्या हेतुभूत‚या \textbf{अन्य‚थे}त्य‚स‚ति प्र‚तिब‚न्धे । \textbf{नै}वास्ति स‚म्वादः श‚ब्दादेः प्र‚त्य‚य‚स्य ।‚{\tiny $_{lb}$}‚ \textbf{व‚स्तूत्प‚त्तेर‚भ्रान्तिरिति चे}त् । स्यादेत‚द् य‚दि व‚स्तुन‚श्चोत्प‚द्य‚ते व‚स्तुस‚म्वादि शाब्दा‚{\tiny $_{lb}$}‚दिज्ञान‚मिति व्याप‚क‚विरुद्धोप‚ल‚ब्धिम्म‚न्य‚ते । \textbf{नैत‚देवं} । स‚त्य‚पि व‚स्तूत्प‚त्ता\textbf{व‚त‚त्प्र‚ति‚{\tiny $_{lb}$}‚भासिनो} व‚स्तुरूपाप्र‚तिभासिन\textbf{स्त‚द‚ध्य‚व‚{\tiny $_{२}$}‚साया}द् व‚स्त्व‚ध्य‚व‚सायाद् भ्रान्तित्वं । त‚तो‚{\tiny $_{lb}$}‚ व‚स्तुरूपोत्प‚त्तिभ्रान्त्योर्विरोधाभावात् स‚न्दिग्ध‚व्य‚तिरेकिता हेतोरिति भावः ।‚{\tiny $_{lb}$}‚ म‚णि\textbf{प्र‚भाया}म्म‚णिभ्रान्तिर्म‚णिं स‚म्वाद‚य‚त्येव । व्य‚भिचार‚मेव स‚म‚र्थ‚य‚न्नाह ।‚{\tiny $_{lb}$}‚ \textbf{वित‚थे}त्यादि । एत‚दाह [।] \textbf{य‚दि भ्रान्तेः} स\textbf{म्वा}द‚स्य च विरोधः स्यात् त‚दा भ्रान्तेर‚{\tiny $_{lb}$}‚व‚स्तुस‚म्वाद‚नं साध्यं प्र‚त्य‚व्य‚भिचारः स्यात् । स च नास्ति ।‚{\tiny $_{४}$}‚ य‚स्माद् \textbf{वित‚थ‚{\tiny $_{lb}$}‚प्र‚तिभासो भ्रान्तिल‚क्ष}णं न विस‚म्वाद‚ने । \textbf{त‚न्नान्त‚रीय‚क}त‚येति व‚स्तुनान्त‚रीय‚क‚त‚या‚{\tiny $_{lb}$}‚ त‚त उत्प‚त्तेरिति याव‚त् । अयं \textbf{स‚म्वादो न प्र‚तिभासापे}क्षी । न व‚स्तुग‚त‚म्प्र‚ति‚{\tiny $_{lb}$}‚भास‚म‚पेक्ष‚ते । व‚स्तुप्र‚तिब‚द्ध‚त्वेनैवात‚त्प्र‚तिभास‚स्यापि स‚म्वादात् ।
	{\color{gray}{\rmlatinfont\textsuperscript{§~\theparCount}}}
	\pend% ending standard par
      ‚{\tiny $_{lb}$}‚

	  
	  \pstart \leavevmode% starting standard par
	त‚स्मात् स्थित‚मेत‚द् वित‚थ‚प्र‚तिभास्य‚पि शाब्दः प्र‚त्य‚यः स‚ति व‚स्तुप्र‚तिब‚न्धे‚{\tiny $_{lb}$}‚ त‚स्य स‚म्वा‚{\tiny $_{५}$}‚द‚क इति ।
	{\color{gray}{\rmlatinfont\textsuperscript{§~\theparCount}}}
	\pend% ending standard par
      ‚{\tiny $_{lb}$}‚‚{\tiny $_{lb}$}‚‚{\tiny $_{lb}$}‚\textsuperscript{\textenglish{205/s}}

	  
	  \pstart \leavevmode% starting standard par
	अथ पुन‚र्य‚थाव‚स्त्वेव शाब्दः प्र‚त्य‚य इष्य‚ते । त‚दा \textbf{व‚स्तुनि} बाह्ये । \textbf{य‚थाभा}वं य‚था‚{\tiny $_{lb}$}‚व‚स्तु \textbf{अर्पित‚चेत‚सः} आरोपित‚ज्ञान‚स्य श‚ब्द‚ब‚लाद्य‚न्य‚था व‚स्तूत्प‚न्न‚ज्ञान‚स्येति याव‚त् ।‚{\tiny $_{lb}$}‚ एवं \textbf{प्र‚वृत्ता}व‚भ्युप‚ग‚म्य‚मानायान्त‚स्य शाब्द‚स्य ज्ञान‚स्य सामान्यं \textbf{ग्राह्य}मेष्ट‚व्यं स्व‚ल‚क्ष‚णे‚{\tiny $_{lb}$}‚ श‚ब्देन चोद‚नाभावात् [।] त‚स्य च ग्राह्य‚स्य \textbf{सामान्य‚स्यान‚र्थ‚क्रियायोग्य‚त्वा}द्धेतो\textbf{र‚प्र‚{\tiny $_{६}$}‚‚{\tiny $_{lb}$}‚वृत्तिस्त‚स्मि}न् विक‚ल्प‚विज्ञान‚विष‚ये सामान्ये । \textbf{अन्य‚त्रे}ति सामान्याद‚न्य‚त्र व्य‚क्ताव‚{\tiny $_{lb}$}‚श‚ब्द‚चोदितायाम‚पि प्र‚वृत्ता\textbf{व‚तिप्र‚स‚ङ्गः} । गोश‚ब्दाद‚श्व‚व्य‚क्ताव‚पि प्र‚वृत्तिः स्याद्‚{\tiny $_{lb}$}‚ गोत्व‚सामान्य‚स्याश्व‚व्य‚क्तेश्च स‚म्ब‚न्धाभावान्नैव‚मिति चेन्नाश‚ब्द‚चोदिते स‚त्य‚पि‚{\tiny $_{lb}$}‚ स‚म्ब‚न्धे प्र‚वृत्त्य‚योगादित्याद्युक्तं ।
	{\color{gray}{\rmlatinfont\textsuperscript{§~\theparCount}}}
	\pend% ending standard par
      ‚{\tiny $_{lb}$}‚

	  
	  \pstart \leavevmode% starting standard par
	अथ न केव‚ला जातिः श‚ब्देन चोदितेति किन्तु त‚द्वानिति [।] त‚दा \textbf{त‚दृ‚{\tiny $_{७}$}‚- \leavevmode\ledsidenote{\textenglish{76b/PSVTa}}‚{\tiny $_{lb}$}‚ द्ग्र‚ह‚णे चा}भ्युप‚ग‚म्य‚माने \textbf{सामान्य‚वैय‚र्थ्याद‚यः प्रोक्ताः} । व्य‚क्तिष्वेव साक्षाच्छ‚ब्दो‚{\tiny $_{lb}$}‚ नियुज्य‚तां किं सामान्येनेति सामान्य‚वैय‚र्थ्य‚मुक्तं । आदिश‚ब्दाद् आन‚न्त्यादिदोष‚{\tiny $_{lb}$}‚प‚रिग्र‚हः । जातिरेव श‚ब्देन चोद्य‚ते । सा तु जातिर्व्य‚क्तिस‚म‚वेत‚त्वान्न श‚क्य‚ते‚{\tiny $_{lb}$}‚ केव‚ला गृहीतुम‚तो व्य‚क्तिरूपेणैकीभूता गृह्य‚ते [।] त‚देवाश‚ङ्क‚ते । \textbf{जाती}त्यादि ।‚{\tiny $_{lb}$}‚ \textbf{श्लिष्टाभा}सेति स्व‚सामान्य‚ल‚क्ष‚णा‚{\tiny $_{१}$}‚भ्यां स‚म्भिन्नाभासा \textbf{बुद्धिर}र्थ‚क्रियाकारिण्यां‚{\tiny $_{lb}$}‚ व्य‚क्तौ \textbf{प्र‚व‚र्त्त‚य‚तीति चे}त् । त‚दा \textbf{न जातिर्न त‚द्वा}न् । स्वेन रूपेण गृह्य‚त इत्य‚ध्याहारः ।‚{\tiny $_{lb}$}‚ किङ्कार‚ण‚म् [।] \textbf{एक‚स्यापि} सामान्य‚स्य त‚द्व‚तो वा या \textbf{स्व‚भाव‚स्थि}तिर‚संसृष्टं रूपं‚{\tiny $_{lb}$}‚ त‚स्य श्लिष्टाभास‚या भ्रान्त‚या बुद्ध्या । \textbf{अग्र‚ह‚णा}त् । त‚त‚श्च भ्रान्ताया बुद्धेः‚{\tiny $_{lb}$}‚ प्र‚वृत्त्य‚भ्युप‚म‚गात् \textbf{प‚र‚वाद एवा}न्यापोह‚वादिद‚र्श‚न‚मेव ॥
	{\color{gray}{\rmlatinfont\textsuperscript{§~\theparCount}}}
	\pend% ending standard par
      ‚{\tiny $_{lb}$}‚

	  
	  \pstart \leavevmode% starting standard par
	\textbf{एव‚मि}‚{\tiny $_{२}$}‚त्यादि प‚रः । \textbf{अन्व‚यिन} इत्य‚नेक‚व्य‚क्तिग‚म्य‚स्य सामान्य‚स्य ।
	{\color{gray}{\rmlatinfont\textsuperscript{§~\theparCount}}}
	\pend% ending standard par
      ‚{\tiny $_{lb}$}‚

	  
	  \pstart \leavevmode% starting standard par
	\textbf{नैष दोष} इति सि द्धा न्त वा दी । आदिश‚ब्दादुद‚काह‚र‚णान्तान्ताम्\textbf{भेदेपि} प‚र‚स्प‚र‚{\tiny $_{lb}$}‚व्यावृत्त‚त्वेपि व‚स्तुध‚र्म‚त‚या \textbf{तां तां ज्ञानादिकां} स‚दृशीम\textbf{र्थ‚क्रियां कुर्व‚तो दृष्ट्}वा त‚द‚न्ये‚{\tiny $_{lb}$}‚भ्योत‚त्कार्येभ्यो यो \textbf{विश्लेषो} विच्छिन्नः स्व‚भावः \textbf{स विष}यो येषां \textbf{ध्व‚नी}नान्तैर्ध्व‚नि‚{\tiny $_{lb}$}‚‚{\tiny $_{lb}$}‚ \leavevmode\ledsidenote{\textenglish{206/s}}भिर‚न्त‚र्ज‚ल्प‚रूपैः \textbf{स‚ह संयोज्यार्था}न् स ‚{\tiny $_{३}$}‚एवाय‚मिति कुर्याद‚पि पुमान् ।‚{\tiny $_{lb}$}‚ \textbf{अपि} श‚ब्दो भिन्न‚क्र‚मोन्य‚द‚र्श‚नेपीत्य‚र्थः । पूर्व‚दृष्टाद‚र्थाद‚न्य‚स्य विल‚क्ष‚ण‚स्य द‚र्श‚नेपि‚{\tiny $_{lb}$}‚ स‚दृशार्थ‚क्रियाकारित्वेन विप्र‚ल‚म्भादेक‚त्व‚मारोप्य \textbf{प्र‚त्य‚भिज्ञानं} कुर्यादिति स‚मुदायार्थः ।
	{\color{gray}{\rmlatinfont\textsuperscript{§~\theparCount}}}
	\pend% ending standard par
      ‚{\tiny $_{lb}$}‚

	  
	  \pstart \leavevmode% starting standard par
	क‚थं पुन‚र्भिन्ना अभिन्नाम‚र्थ‚क्रियां कुर्व‚न्तीत्य‚त आह । \textbf{उक्त‚मेत}दित्यादि ।
	{\color{gray}{\rmlatinfont\textsuperscript{§~\theparCount}}}
	\pend% ending standard par
      ‚{\tiny $_{lb}$}‚

	  
	  \pstart \leavevmode% starting standard par
	\hphantom{.}एक‚प्र‚त्य‚व‚म‚र्शार्थ‚ज्ञानाद्येकार्थ‚साध‚ने \href{http://sarit.indology.info/?cref=pv.3.72}{१ । ७५}
	{\color{gray}{\rmlatinfont\textsuperscript{§~\theparCount}}}
	\pend% ending standard par
      ‚{\tiny $_{lb}$}‚

	  
	  \pstart \leavevmode% starting standard par
	इत्यादिना‚{\tiny $_{४}$}‚ प्रागुक्त‚त्वात् । \textbf{एका}मिति स‚दृशीन्तेष्विति भिन्नेष्व‚न्येषु प\textbf{श्य‚तः}‚{\tiny $_{lb}$}‚ पुंसः । अन्येभ्य इत्य‚त‚त्कांरिभ्यो व्याव‚र्त्त‚माना भावा \textbf{व‚स्तुध‚र्म‚त‚यैव} व‚स्तुस्व‚भावे‚{\tiny $_{lb}$}‚नैव । \textbf{त‚दे}वेद‚मित्येव‚माकारं मिथ्या\textbf{प्र‚त्य‚यं ज‚न‚य‚न्ती}ति स‚म्ब‚न्धः । किम्विशिष्ट‚{\tiny $_{lb}$}‚मित्याह । त‚दित्यादि । तेभ्योऽत‚त्कार्येभ्यो या \textbf{व्यावृत्ति}रेकार्थ‚क्रियाकारिणाम‚{\tiny $_{lb}$}‚र्थानां सा \textbf{विष}यो य‚स्य \textbf{ध्व}नेस्तेन \textbf{संसृष्टं} संस‚{\tiny $_{५}$}‚क्तं \textbf{साभिलाप}मिति याव‚त् ।
	{\color{gray}{\rmlatinfont\textsuperscript{§~\theparCount}}}
	\pend% ending standard par
      ‚{\tiny $_{lb}$}‚

	  
	  \pstart \leavevmode% starting standard par
	य‚दि व‚स्तुध‚र्म्म‚त‚या \textbf{ज‚न‚य‚न्ति} किन्न स‚र्व‚देत्याह । \textbf{स्वानुभ‚वे}त्यादि । तेषा‚{\tiny $_{lb}$}‚म्भावानां यः स्वोनुभ‚वः पूर्व‚मुत्प‚न्न‚स्तेन या प्र‚त्य‚भिज्ञानोत्प‚त्त‚ये \textbf{वास‚ना} श‚क्ति‚{\tiny $_{lb}$}‚ल‚क्ष‚णादिता । त‚स्याः \textbf{प्र‚बोधः} कार्योत्पाद‚नंप्र‚त्याभिमुख्य‚न्त‚स्याश्च प्र‚बोधः पुन‚स्त‚{\tiny $_{lb}$}‚ज्जातीय‚प‚दार्थानुभ‚वात् । एवंल‚क्ष‚ण‚श्च प्र‚बोधो न स‚र्व‚काल‚म‚तो न स‚दा प्र‚त्य‚भि‚{\tiny $_{६}$}‚‚{\tiny $_{lb}$}‚ज्ञान‚स‚म्भ‚व इति । \textbf{संसृष्ट‚भेद}मिति पूर्व‚प‚श्चाद्दृष्ट‚योर्व्य‚क्त्योर्भेदः संसृष्ट एकीकृतो‚{\tiny $_{lb}$}‚ येन स त‚था । \textbf{अन्य‚थे}ति य‚द्येक‚कार्य‚त्वेन सादृश्येनैक‚त्व‚मारोप्य भिन्नेष्व‚पि भ्रान्तं‚{\tiny $_{lb}$}‚ प्र‚त्य‚भिज्ञानं नेष्य‚ते [।] अपि त्वेक‚सामान्य‚योगात् त‚दा \textbf{न भेद‚संस‚र्ग‚व‚ती} । भेदानां‚{\tiny $_{lb}$}‚ \leavevmode\ledsidenote{\textenglish{77a/PSVTa}} संस‚र्ग एक‚रूप‚तापाद‚न‚न्त‚द्व‚ती बुद्धिर्न स्यात् । ब‚हुष्वेक‚रूपा \textbf{बुद्धिर्न} स्यादित्य‚र्थः ।
	{\color{gray}{\rmlatinfont\textsuperscript{§~\theparCount}}}
	\pend% ending standard par
      ‚{\tiny $_{lb}$}‚

	  
	  \pstart \leavevmode% starting standard par
	न \textbf{ह्येकेन द‚ण्डेन युक्ता} अपि द‚ण्डिन एक‚त्वेन गृह्य‚न्ते । त‚देवाह [।] \textbf{य‚था‚{\tiny $_{lb}$}‚ द‚ण्डिष्वि}ति । न हीत्येत‚देव व्य‚न‚क्ति । \textbf{त‚त्रे}ति द‚ण्डिषु । \textbf{अन्य}त्रेति एक‚स्माद्‚{\tiny $_{lb}$}‚ द‚ण्डिनोन्य‚स्मिन् द‚ण्डिनि । त‚द्द‚ण्ड‚द्र‚व्यं य‚देक‚द‚ण्डिनि दृष्ट‚न्त‚दिह द्वितीये द‚ण्डिनीत्येवं‚{\tiny $_{lb}$}‚ स्यात् । न तु त‚द्द्वारेण \textbf{स ए}वाय‚न्द‚ण्डीति । य‚द्वा य‚था ब‚हुष्वेक‚द‚ण्ड‚योगात् ।‚{\tiny $_{lb}$}‚ प्र‚त्येक‚म‚य‚म‚पि द‚ण्ड‚स्त‚था स एवाय‚न्द‚{\tiny $_{१}$}‚ण्ड इति न भ‚व‚ति \textbf{प्र‚ती}तिस्त‚द्व‚त् । व्य‚क्ती‚{\tiny $_{lb}$}‚‚{\tiny $_{lb}$}‚ \leavevmode\ledsidenote{\textenglish{207/s}}नाम‚प्येक‚सामान्य‚योगान्न स एवाय‚मिति प्र‚तीतिः स्या\textbf{द‚पि तु त‚दि}हेति ।
	{\color{gray}{\rmlatinfont\textsuperscript{§~\theparCount}}}
	\pend% ending standard par
      ‚{\tiny $_{lb}$}‚

	  
	  \pstart \leavevmode% starting standard par
	भ‚व‚त्वेव‚मिति चेदाह । \textbf{नैव‚मि}त्यादि । \textbf{त‚दि}त्यादिनोप‚संहारः । \textbf{एक}मिति‚{\tiny $_{lb}$}‚ स‚मान\textbf{म‚नेक}त्र व्य‚क्तिषु \textbf{प‚श्य‚तोऽ}पि पुंसो \textbf{भेद‚संस‚र्ग‚व}त् । भेदानां संस‚र्ग एकाकार‚ता‚{\tiny $_{lb}$}‚ त‚द्व‚ज्ज्ञानं \textbf{न युक्तं} । अन्यापोह‚वादिन‚स्त्व‚य‚म‚दो‚{\tiny $_{२}$}‚ष इत्याह । \textbf{विभ्र}मेत्यादि ।‚{\tiny $_{lb}$}‚ भ्रान्तिसाम‚र्थ्यादित्य‚र्थः । \textbf{त‚थे}त्येक‚रूप‚त‚या व‚स्तुभूत‚मेकं सामान्यं प्र‚त्य‚भिज्ञान‚स्य‚{\tiny $_{lb}$}‚ \textbf{निमित्त}न्त‚स्या\textbf{भावा}द् \textbf{विभ्र‚मो न युक्त‚मिति चे}त् । \textbf{त एवे}ति व्यावृत्ता \textbf{भावास्त}स्य‚{\tiny $_{lb}$}‚ ज्ञानादे\textbf{रेक}स्यार्थ‚स्य \textbf{कारि}णः क‚र‚ण‚शीलाः । \textbf{अनुभ}व एव \textbf{द्वारं} हेतुस्तेन \textbf{प्र‚कृत्या}‚{\tiny $_{lb}$}‚ स्व‚भावेन \textbf{विभ्र‚म‚फ‚लाया} भ्रान्तिफ‚लाया \textbf{हेतुत्वान्निमित्तं} ।
	{\color{gray}{\rmlatinfont\textsuperscript{§~\theparCount}}}
	\pend% ending standard par
      ‚{\tiny $_{lb}$}‚

	  
	  \pstart \leavevmode% starting standard par
	न‚नु \textbf{म‚रीचि‚{\tiny $_{३}$}‚कादि}षु \textbf{ज‚लादिभ्रांतेः} सादृश्य‚म‚न्त‚रेणोत्प‚त्ताव‚तिप्र‚संगः ।‚{\tiny $_{lb}$}‚ सादृश्यं चेदिष्य‚ते सामान्य‚म‚पि क‚स्मान्नेष्य‚त इत्याह । \textbf{म‚रीचिकादिष्वि}त्यादि ।‚{\tiny $_{lb}$}‚ प्र‚थ‚मेनादिश‚ब्देन र‚ज्वादिप‚रिग्र‚हः । द्वितीयेन स‚र्पादिभ्रान्तेः । \textbf{तावेवेति} ज‚ल‚{\tiny $_{lb}$}‚म‚रीचिका\textbf{रूपौ} भावो अ\textbf{भिन्नाका}र‚स्य त‚देवेदं ज‚ल‚मित्येवं रूप‚म्\textbf{प‚राम‚र्श‚प्र‚त्य}य‚स्य‚{\tiny $_{lb}$}‚ \textbf{निमित्त}भूतो यो‚{\tiny $_{४}$}‚\textbf{नुभ‚व‚स्त‚स्य ज‚न‚कौ कार‚णं भिन्नाव}पि ।
	{\color{gray}{\rmlatinfont\textsuperscript{§~\theparCount}}}
	\pend% ending standard par
      ‚{\tiny $_{lb}$}‚

	  
	  \pstart \leavevmode% starting standard par
	एत‚दुक्त‚म्भ‚व‚त्य‚स‚दृशानान्ताव‚न्न सादृश्य‚म‚स्ति । स‚दृशानाम‚पि स‚दृश‚मेव‚{\tiny $_{lb}$}‚ स्व‚रूपं भ्रान्तिनिमित्तं । न तु सादृश्यं । त‚था हि ज‚लानुभ‚व‚ज्ञान‚न्तावत् ज‚लाकार‚{\tiny $_{lb}$}‚प‚राम‚र्श‚वास‚नामाध‚त्ते । सा च वास‚ना य‚था पुन‚र्ज‚ल‚स्व‚ल‚क्ष‚णानुभ‚वेन प्र‚बोध्य‚ते ।‚{\tiny $_{lb}$}‚ त‚था म‚रीचिकाख्य‚प‚दार्थानुभ‚वेनापि‚{\tiny $_{५}$}‚ प्र‚कृत्या । त‚स्य त‚त्स्व‚भाव‚त्वात् । त‚तो य‚था‚{\tiny $_{lb}$}‚ ज‚लानुभ‚वाज्ज‚लाकार‚प‚राम‚र्श‚प्र‚त्य‚य उच्य‚ते । त‚था म‚रीचिकानुभ‚वाद‚पि । अत‚श्च‚{\tiny $_{lb}$}‚ तौ ज‚ल‚म‚रीचिकाख्यौ भावानुभ‚व‚द्वारेण ज‚ल‚भ्रान्तेर्निमित्तं भ‚व‚तः । न चातिप्र‚स‚ङ्गः ।‚{\tiny $_{lb}$}‚ तुल्येप्य‚ज‚ल‚रूप‚त्वे म‚रीचिकास्व‚रूप‚स्य स्व‚हेतुभ्य एव सादृश्योत्प‚न्न‚त्वान्न तु‚{\tiny $_{lb}$}‚ सादृश्य‚योगात् स‚दृशो भ‚व‚{\tiny $_{६}$}‚तीति सामान्य‚प्र‚स्तावे न्याय‚स्योक्त‚त्वात् ।
	{\color{gray}{\rmlatinfont\textsuperscript{§~\theparCount}}}
	\pend% ending standard par
      ‚{\tiny $_{lb}$}‚

	  
	  \pstart \leavevmode% starting standard par
	एव‚न्ताव‚द् उ द्यो त क रा दिम‚तं निराकृत्य मी मां स क म‚तं निराक‚र्त्तुमाह ।‚{\tiny $_{lb}$}‚ \textbf{न ही}त्यादि । त‚था हि ज‚ल‚ज्ञाने द्व‚यं प्र‚तिभास‚ते ज‚ल‚सामान्य‚न्त‚स्य च देशादिस‚{\tiny $_{lb}$}‚  ‚{\tiny $_{lb}$}‚ ‚{\tiny $_{lb}$}‚ \leavevmode\ledsidenote{\textenglish{208/s}}म्ब‚न्धित्व‚न्त‚तो नेति बाध‚के प्र‚त्य‚ये न देशादिस‚म्ब‚न्धित्वं बाध्य‚ते न ज‚ल‚सामान्य‚म‚तो‚{\tiny $_{lb}$}‚ \leavevmode\ledsidenote{\textenglish{77b/PSVTa}} ज‚ल‚ज्ञानं सामान्याल‚म्ब‚न‚मेवेत्य‚त उच्य‚ते‚{\tiny $_{७}$}‚ । \textbf{न त‚त्र म‚रीचिकासु अन्य‚दे}वेति भिन्नं‚{\tiny $_{lb}$}‚ \textbf{किंचि}त् \textbf{सामान्यं} ज‚ल‚सामान्य\textbf{न्त‚थे}ति ज‚ल‚रूपेण । \textbf{स‚त्त्वे वा} ज‚ल‚सामान्य‚स्याभ्यु‚{\tiny $_{lb}$}‚प‚ग‚म्य‚माने व‚स्तुभूत‚सामान्य‚ग्राहित्वेन \textbf{स‚द‚र्थ‚ग्राहिणी बुद्धिः} ।
	{\color{gray}{\rmlatinfont\textsuperscript{§~\theparCount}}}
	\pend% ending standard par
      ‚{\tiny $_{lb}$}‚

	  
	  \pstart \leavevmode% starting standard par
	अथ स्याद्[।]अन्य‚देशाद्य‚व‚स्थित‚ज‚ल‚सामान्याल‚म्बिकैव ज‚ल‚बुद्धिर्न भ्रान्तिस्त‚{\tiny $_{lb}$}‚दुक्तं [।]
	{\color{gray}{\rmlatinfont\textsuperscript{§~\theparCount}}}
	\pend% ending standard par
      ‚{\tiny $_{lb}$}‚
	  \bigskip
	  \begingroup
	
	    
	    \stanza[\smallbreak]
	  {\normalfontlatin\large ``\qquad}स‚र्व‚त्राल‚म्ब‚नं बाह्यं देश‚कालान्य‚थात्म‚क‚मिति ।\edtext{\textsuperscript{*}}{\edlabel{pvsvt_208-1}\label{pvsvt_208-1}\lemma{*}\Bfootnote{\href{http://sarit.indology.info/?cref=\%C5\%9Bv-niralambana.108}{ Ślokavārtika.  निराल‚म्व‚न० १०८}}}{\normalfontlatin\large\qquad{}"}\&[\smallbreak]
	  
	  
	  
	  \endgroup
	‚{\tiny $_{lb}$}‚

	  
	  \pstart \leavevmode% starting standard par
	क‚थ‚न्त‚र्हि म‚रीचिकायां‚{\tiny $_{१}$}‚ ज‚ल‚प्र‚तीतिर्भ्रान्तिर‚न्य‚देशाद्य‚व‚स्थित‚स्य ज‚ल‚सामान्य‚स्य‚{\tiny $_{lb}$}‚ त‚त्र स‚मारोपादिति चेत् । एत‚देवाह । \textbf{अभूते}त्यादि । म‚रीचिकास्व‚विद्य‚मान‚स्य‚{\tiny $_{lb}$}‚ ज‚ला\textbf{कार}स्य स\textbf{मारोपा}द् \textbf{भ्रान्तिः} ।
	{\color{gray}{\rmlatinfont\textsuperscript{§~\theparCount}}}
	\pend% ending standard par
      ‚{\tiny $_{lb}$}‚

	  
	  \pstart \leavevmode% starting standard par
	\textbf{नेत्या}दि सि द्धा न्त वा दी । \textbf{त‚त्}सामान्य‚ग्राहिणी अन्य‚देशाव‚स्थित‚ज‚ल\textbf{सामान्य‚{\tiny $_{lb}$}‚ग्राहिणी सा} ज‚ल‚बुद्धिर्\textbf{न भ‚व‚ति} । क‚स्मादित्याह । \textbf{य‚मेव चे}त्यादि । \textbf{य‚मेवाका}र‚{\tiny $_{lb}$}‚मित्य‚भूतं‚{\tiny $_{२}$}‚ज‚लाकार\textbf{मियं} ज‚ल‚बुद्धिस्त‚त्र म‚रीचिकास्वा\textbf{रोप‚य‚ति} । आरोप्य‚माण‚{\tiny $_{lb}$}‚स्यापि क‚स्माद् विष‚य‚त्व‚मित्य‚त आह । \textbf{अविष‚यीकृ}त‚स्येत्यादि । एत‚दाह [।]‚{\tiny $_{lb}$}‚ विक‚ल्पोत्प‚त्तिकाले य‚त्सामान्यं न विष‚यीकृत‚न्तेन न त‚स्य स‚मारोपः । आकारा‚{\tiny $_{lb}$}‚\textbf{न्त‚र‚व}त् । न ह्याकारान्त‚र‚म‚ग्न्यादि त‚त्र स‚मारोप्य‚ते त‚स्य त‚दानीम‚विष‚य‚त्वात् ।‚{\tiny $_{lb}$}‚ य‚श्चाकारो बाह्याभेदेनारोप्य‚{\tiny $_{३}$}‚ते त‚स्यैव विष‚य‚त्वेन प्र‚तिभास‚नं । \textbf{स चे}त्यारोप्य‚माणो‚{\tiny $_{lb}$}‚ ज‚लाकार‚स्\textbf{त‚त्र} म‚रीचिकासु \textbf{नास्}ति बाध्य‚मान‚त्वा\textbf{द‚तो सामान्}यं ज‚ल‚ज्ञान‚न्न विद्य‚ते‚{\tiny $_{lb}$}‚ सामान्य‚म‚स्येति कृत्वा । \textbf{त‚था} म‚रीचिकाव‚त् स‚त्त्य‚ज‚लेष्व‚पि ज‚लाकाराध्यारोप‚कं‚{\tiny $_{lb}$}‚ ज‚ल‚ज्ञान‚म‚सामान्यं ।
	{\color{gray}{\rmlatinfont\textsuperscript{§~\theparCount}}}
	\pend% ending standard par
      ‚{\tiny $_{lb}$}‚

	  
	  \pstart \leavevmode% starting standard par
	\textbf{स‚तीत्}यादि प‚रः । अन्य‚देव ज‚ल\textbf{सामा}न्यं स‚ति त‚स्य \textbf{ग्र‚हे त‚दारोपो} ज‚लारोपः‚{\tiny $_{lb}$}‚ \textbf{नान्य‚था}‚{\tiny $_{४}$}‚[।] य‚दि ज‚ल‚सामान्य‚म‚न्त‚रेण स‚त्य‚ज‚ले ज‚लारोपः स्यात् त‚दा\textbf{ऽतिप्र‚संगः} ।‚{\tiny $_{lb}$}‚ अग्न्यादाव‚पि ज‚लारोपः स्यात् ।
	{\color{gray}{\rmlatinfont\textsuperscript{§~\theparCount}}}
	\pend% ending standard par
      ‚{\tiny $_{lb}$}‚‚{\tiny $_{lb}$}‚‚{\tiny $_{lb}$}‚\textsuperscript{\textenglish{209/s}}

	  
	  \pstart \leavevmode% starting standard par
	\textbf{स‚ती}ति सि द्धा न्त वा दी । \textbf{एकं कार्यं} पानाव‚गाह‚नादि त\textbf{त्क‚र‚ण‚शीलानां} स‚त्त्य‚{\tiny $_{lb}$}‚ज‚लानां \textbf{ग्र‚हे स‚ति किन्नेष्य}ते ज‚लारोपः । सामान्य‚म‚न्त‚रेण भिन्नानामेक‚कार्य‚क‚र‚ण‚{\tiny $_{lb}$}‚श‚क्तिरेव नास्तीति चेदाह । \textbf{अव‚श्यं चे}त्यादि । प्र‚तिपादि‚{\tiny $_{५}$}‚तं चैत‚द् [।]
	{\color{gray}{\rmlatinfont\textsuperscript{§~\theparCount}}}
	\pend% ending standard par
      ‚{\tiny $_{lb}$}‚

	  
	  \pstart \leavevmode% starting standard par
	\hphantom{.}एक‚प्र‚त्य‚व‚म‚र्शार्थ‚ज्ञानाद्येकार्थ‚साध‚न \href{http://sarit.indology.info/?cref=pv.3.72}{३ । ७२}
	{\color{gray}{\rmlatinfont\textsuperscript{§~\theparCount}}}
	\pend% ending standard par
      ‚{\tiny $_{lb}$}‚

	  
	  \pstart \leavevmode% starting standard par
	इत्य‚त्रान्त‚रे ।
	{\color{gray}{\rmlatinfont\textsuperscript{§~\theparCount}}}
	\pend% ending standard par
      ‚{\tiny $_{lb}$}‚

	  
	  \pstart \leavevmode% starting standard par
	अन्ये तु \textbf{न हि त‚त्रान्य}देव किंचित् सामान्य‚म‚स्तीत्यादिकं ग्र‚न्थं सामान्य‚श‚ब्दं‚{\tiny $_{lb}$}‚ सादृश्यार्थं कृत्वा व्याच‚क्ष‚ते । त‚त्तेषां व्याख्यान‚न्नातिश्लिष्टं य‚त्त‚था प्र‚तीय‚त इत्या‚{\tiny $_{lb}$}‚देर‚वाच‚क‚त्वात् । न हि सादृश्यं ज‚ल‚रूप‚त‚या प्र‚तीय‚त इति ।
	{\color{gray}{\rmlatinfont\textsuperscript{§~\theparCount}}}
	\pend% ending standard par
      ‚{\tiny $_{lb}$}‚

	  
	  \pstart \leavevmode% starting standard par
	\textbf{त‚त} इत्येक‚कार‚ण‚श‚क्तेः । \textbf{त एवे}ति य‚थोक्त‚{\tiny $_{६}$}‚श‚क्तिषु युक्ता व्य‚क्त‚य एव‚{\tiny $_{lb}$}‚ न तु सामान्यं । \textbf{अन्येभ्य} इत्य‚त‚त्कार्येभ्यः । \textbf{तादृश}मित्येकाकारं । \textbf{य‚थाभाव}मिति‚{\tiny $_{lb}$}‚ प‚दार्थान‚तिवृत्ताव‚न्व‚यीभावः । य‚था स्व‚ल‚क्ष‚णं सामान्य‚विर‚हि स्थितं ।‚{\tiny $_{lb}$}‚ त‚द‚तिवृत्त्या \textbf{किन्न प्र‚त्येति विक‚ल्}प‚प्र‚त्य‚यः [।] क‚स्मात्त‚त्राभूतं सामान्य‚मारोप‚{\tiny $_{lb}$}‚य‚ति । \textbf{य‚थाभाव}मित्य‚स्यैवार्थो\textbf{संसृष्टे}त्यादि । असंसृष्ट एक‚रूप‚ताम‚ना‚{\tiny $_{७}$}‚- \leavevmode\ledsidenote{\textenglish{78a/PSVTa}}‚{\tiny $_{lb}$}‚ प‚न्नः । \textbf{भेदः} प‚र‚स्प‚रं विशेषो य‚स्य भाव‚स्य स त‚था । अश‚क्ति‚{\tiny $_{lb}$}‚रेषा । य‚थाव‚स्थित‚ग्र‚ह‚णंप्र‚ति \textbf{विक‚ल्पा}नां । क‚स्मात् \textbf{अविद्याप्र‚भ‚वा}त् । भूता‚{\tiny $_{lb}$}‚र्थ‚ग्र‚ह‚णं विद्या । त‚द्विरोधाद् विक‚ल्प एवाविद्या । प्र‚भाव एव प्र‚भ‚व‚श‚ब्देनोक्तः ।‚{\tiny $_{lb}$}‚ विक‚ल्प‚साम‚र्थ्यादित्य‚र्थः । य‚थास्थित‚व‚स्त्व‚ग्र‚ह‚णं हि विक‚ल्प‚स्य स्व‚भावः प्र‚कृत्या‚{\tiny $_{lb}$}‚ भ्रान्त‚त्वात् त‚स्य‚{\tiny $_{१}$}‚ व‚स्तुभूतं सामान्यं विनाश एव विक‚ल्प‚स्य विभ्र‚मो न युक्त इति‚{\tiny $_{lb}$}‚ चेदाह । \textbf{न चे}त्यादि । विक‚ल्प‚स्व‚रूप‚मेवात्रान्त‚रो विप्ल‚व उक्तः । विक‚ल्प‚स्यैव‚{\tiny $_{lb}$}‚ त‚त्स्व‚रूपं येनासौ \textbf{बाह्य}साध‚र्म्य\textbf{म‚न‚पेक्ष्य विभ्र‚मो भ‚व‚ती}त्य‚र्थः । केशाद्याकारा‚{\tiny $_{lb}$}‚ भ्रान्तिः \textbf{केशादिविभ्र‚मः} [।] स य‚था बाह्यार्थान‚पेक्षः स‚न्न‚भूताकारोप‚ग्र‚ह‚ण‚मा\textbf{न्त‚र‚{\tiny $_{lb}$}‚म्विप्ल‚व}माश्रित्य भ्रान्तो भ‚{\tiny $_{२}$}‚व‚ति । त‚द्व‚द् विक‚ल्पोप्य‚भूत‚सामान्याकार‚ग्र‚ह‚{\tiny $_{lb}$}‚णादित्य‚य‚म‚त्रार्थोभिप्रेतः । भ्रान्तिबीज‚मान्त‚रो विप्ल‚व‚स्त‚स्मादुत्प‚त्तेरित्य‚य‚{\tiny $_{lb}$}‚  ‚{\tiny $_{lb}$}‚ ‚{\tiny $_{lb}$}‚ \leavevmode\ledsidenote{\textenglish{210/s}}न्ताव‚द‚र्थोत्र नाभिप्रेतः । एत‚च्चोत्त‚र‚त्र व्य‚क्तीक‚रिष्य‚ते ।
	{\color{gray}{\rmlatinfont\textsuperscript{§~\theparCount}}}
	\pend% ending standard par
      ‚{\tiny $_{lb}$}‚

	  
	  \pstart \leavevmode% starting standard par
	चोद‚क‚स्त्व‚विद्याप्र‚भ‚वादित्य‚त्राविद्याश‚ब्देनाप्र‚हीणाव‚र‚ण‚स‚न्त‚तौ द्व‚य‚निर्भास‚{\tiny $_{lb}$}‚बीज‚मेवोक्तं । त‚त‚श्चोद्भ‚व उत्प‚त्तिस्त‚था आन्त‚रोपि विप्ल‚व‚स्त‚दैव बीज‚मे‚{\tiny $_{३}$}‚वं‚{\tiny $_{lb}$}‚भूतं चाविद्योद्भ‚व‚त्वं स‚र्व‚विज्ञानानाम‚स्तीत्य‚त आह । \textbf{अविद्योद्भ‚वाद् विप्ल‚व‚त्व}‚{\tiny $_{lb}$}‚ इत्यादि । अविद्याया उद्भ‚वादुत्पादाद् विप्ल‚व‚त्वे भ्रान्त‚त्वे \textbf{च‚क्षु}र्विज्ञाना\textbf{दिष्व‚पि}‚{\tiny $_{lb}$}‚ विप्ल‚व\textbf{प्र‚संगः । ने}त्यादिना स्वाभिप्राय‚माह । \textbf{त‚स्या} इत्य‚विद्यायाः सामान्याकारारो‚{\tiny $_{lb}$}‚प‚कं ज्ञान‚म्\textbf{विक‚ल्प}स्त\textbf{ल्ल‚क्ष‚ण‚त्वा}त् । त‚दाह । \textbf{विक‚ल्प एव ही}त्यादि । \textbf{सेत्}य‚विद्या ।‚{\tiny $_{lb}$}‚ \textbf{स्व‚भावे}‚{\tiny $_{४}$}‚नेति प्र‚कृत्या । \textbf{नैव}मित्यादिना प्र‚संगं प‚रिह‚र‚ति । तेषां स्व‚ल‚क्ष‚णाकार‚{\tiny $_{lb}$}‚त्वेनाविक‚ल्प‚क‚त्वात् । त‚स्मान्न तानि विक‚ल्प‚व‚त् स्व‚भावेन विप‚र्य‚स्तानि \textbf{इन्द्रि}‚{\tiny $_{lb}$}‚यादिविकारेण तु केषांचिद् भ‚व‚ति भ्रान्त‚ता ।
	{\color{gray}{\rmlatinfont\textsuperscript{§~\theparCount}}}
	\pend% ending standard par
      ‚{\tiny $_{lb}$}‚

	  
	  \pstart \leavevmode% starting standard par
	बाह्यार्थ‚न‚येनोक्ताऽधुनान्त‚र्ज्ञेय‚न‚येनाह । \textbf{न चे}त्यादि । एत‚दाह । भ‚व‚तु‚{\tiny $_{lb}$}‚नाम यादृश‚श्चोद्याकारेणाविद्याश‚ब्द‚स्यार्थः क‚ल्पित‚स्त‚थाप्य‚ति‚{\tiny $_{५}$}‚प्र‚स‚ङ्ग‚दोषो नास्ती‚{\tiny $_{lb}$}‚ष्ट‚त्वादिति । \textbf{न वा तेष्व‚पि} च‚क्षुरादिज्ञानेष्\textbf{वेष} भ्रान्त‚त्व\textbf{दोष}स्तेषाम‚पि विप्लु‚{\tiny $_{lb}$}‚त‚त्वात् । त‚देवाह । \textbf{अद्व‚याना}मित्यादि । च‚क्षुरादिविज्ञानानामात्म‚स‚म्वेद‚न‚मेवा‚{\tiny $_{lb}$}‚द्व‚य‚न्नात्र द्व‚य‚म‚स्तीति कृत्वा । त‚था हि विज्ञान‚स‚मान‚काल‚म्विच्छिन्न‚प्र‚तिभासि ।‚{\tiny $_{lb}$}‚ ग्राह्य‚त्वेनाभिम‚तं नीलादि । एकानेक‚विचाराक्ष‚म‚त‚या न प‚र‚मा‚{\tiny $_{६}$}‚र्थंस‚त् । त‚द‚पे‚{\tiny $_{lb}$}‚क्ष‚या च य‚द्विज्ञान‚स्य ग्राह‚क‚त्वं क‚र्त्तृरूप‚न्त‚द‚प्य‚स‚त् । न तु स‚म्वेद‚न‚न्त‚स्य‚{\tiny $_{lb}$}‚ प्र‚त्य‚क्ष‚त्वात् । भ्रान्त‚ग्राह‚काकाराव्य‚तिरिक्त‚त्वात्स्व‚स‚म्वित्तेर‚पि भ्रान्त‚त्व‚मिति‚{\tiny $_{lb}$}‚ चेन्न त‚स्याः स्व‚रूपेणास‚त्त्वे प्र‚तिभास एव न स्याच्छ‚श‚विषाण‚व‚त् । भ्रान्तेर‚पि च‚{\tiny $_{lb}$}‚ स्व‚रूपेण स‚त्त्य‚त्व‚म‚न्य‚था भ्रान्तित्वायोगात् । स्व‚रूप‚विज्ञानैक‚रूपं ज्ञान‚ञ्च स्व‚स‚{\tiny $_{lb}$}‚\leavevmode\ledsidenote{\textenglish{78b/PSVTa}} म्विद्रूप‚{\tiny $_{७}$}‚मेवेति क‚थ‚न्न \add{संवित्तेः स‚त्त्य‚त्वं । य‚द्वा द्व‚य‚प्र‚तिभासो भ्रान्तिर्भ्रान्तिश्च‚{\tiny $_{lb}$}‚ त‚त्त्वाधिष्ठाना । द्विच‚न्द्रादिभ्रान्तिव‚त् । त‚त्त्वं च द्व‚य‚विप‚रीत‚म‚द्व‚य‚न्त‚च्च स्व‚स‚म्वि‚{\tiny $_{lb}$}‚द्रूप‚मेव [।] न तु द्व‚याभाव‚तास्यास‚त्त्वादिति क‚थं न सा स‚म्वित्तेः{...}त‚देवं‚{\tiny $_{lb}$}‚ य‚थोक्त‚चोद्य{... ... ... ...}यानि च{.........}स्याद्‚{\tiny $_{lb}$}‚ ग्राह्य‚ग्राह‚क‚रूपेण प्र‚तिभास‚नात् तान्य‚पि भ्रान्तानीति}\edtext{}{\edlabel{pvsvt_210-2}\label{pvsvt_210-2}\lemma{न्न}\Bfootnote{In the margin, illigible. }} व‚क्ष्यामः तृतीये प‚रिच्छेदे‚{\tiny $_{lb}$}‚  ‚{\tiny $_{lb}$}‚ ‚{\tiny $_{lb}$}‚ ‚{\tiny $_{lb}$}‚ \leavevmode\ledsidenote{\textenglish{211/s}}[।] अत एव द्व‚य‚निर्भास‚व‚तां स्व‚संवित्तेः प्र‚त्य‚क्ष‚त्वेपि न त‚त्त्व‚द‚र्शित्वं व्य‚व‚स्थाप्य‚ते‚{\tiny $_{lb}$}‚ प्र‚माणाप्र‚माण‚विभागः । क‚थ‚मिति चेदाह । \textbf{स‚र्वेषामि}त्यादि । \textbf{विप्ल}वो भ्रान्त‚{\tiny $_{lb}$}‚त्वं । \textbf{त‚दाभासः} । प्र‚माणाभासः । त‚यो\textbf{र्व्य‚व‚स्था}विभागः । आश्र‚यो भ्रान्तिबीज‚{\tiny $_{lb}$}‚\textbf{माल‚य}विज्ञान‚न्त‚स्य प‚रावृत्तिराव‚र‚ण‚विग‚मः । आङ् म‚र्यादायाम् [।] \textbf{आश्र}य‚{\tiny $_{lb}$}‚प‚रावृत्तेः स‚र्व‚{\tiny $_{१}$}‚दा\textbf{र्थ‚क्रियायोग्याऽभिम‚त‚स‚म्वाद‚ना}त् प्रामाण्य‚व्य‚व‚स्थेति स‚म्ब‚न्धः ।‚{\tiny $_{lb}$}‚ \textbf{अर्थ‚क्रिया} दाह‚पाकादिनिर्भास‚विज्ञ‚प्तिल‚क्ष‚णा । त‚स्यां \textbf{योग्}यं च त‚द\textbf{भिम}तं पुरु‚{\tiny $_{lb}$}‚ष‚स्येष्ट‚त्वात् । त‚स्य \textbf{स‚म्वाद‚ना}दिति विग्र‚हः । अभिम‚त‚स्येव स‚म्वाद‚नादित्य‚व‚{\tiny $_{lb}$}‚धार‚णं न पुन‚स्स‚म्वाद‚नादेवेति प्र‚माणाद‚पि क‚दाचित् प्र‚त्य‚य‚वैक‚ल्येन स‚म्वादास‚{\tiny $_{lb}$}‚म्भ‚वात् । अर्थ‚क्रियायोग्याभिम‚त‚स‚म्वाद‚{\tiny $_{२}$}‚नादित्युप‚ल‚क्ष‚णं [।] त‚थाभिम‚तास‚म्वा‚{\tiny $_{lb}$}‚द‚नादित्य‚पि द्र‚ष्ट‚व्यं ।
	{\color{gray}{\rmlatinfont\textsuperscript{§~\theparCount}}}
	\pend% ending standard par
      ‚{\tiny $_{lb}$}‚

	  
	  \pstart \leavevmode% starting standard par
	एवं हि प्र‚माणाभास‚व्य‚व‚स्थायाः कार‚ण‚मुक्त‚म्भ‚वेत् । विज्ञान‚वादे बाह्या‚{\tiny $_{lb}$}‚भावात् क‚थ‚म‚र्थ‚क्रियायोग्याभिम‚त‚स‚म्वाद‚नं । नाय‚न्दोषोग्निर्ज‚ल‚निर्भ्रास‚स्यैव‚{\tiny $_{lb}$}‚ ज्ञान‚स्य दाह‚पाकादिनिर्भास‚ज्ञानोत्पाद‚न‚स‚म‚र्थ‚स्य योग्य‚श‚ब्देनाभिधानात् । एव‚{\tiny $_{lb}$}‚न्ताव‚च्च‚क्षुरादिविज्ञान‚स्य धूमादिलिङ्ग‚ज‚न्य‚स्य चाग्न्यादि‚{\tiny $_{३}$}‚निर्भासिनः प्र‚माण‚{\tiny $_{lb}$}‚व्य‚व‚स्थोक्ता । कृत‚कादिलिङ्ग‚ज‚न्य‚स्य त्व‚नात्मादिज्ञान‚स्याह । \textbf{मिथ्ये}त्यादि ।‚{\tiny $_{lb}$}‚ सामान्याकारारोप‚प्र‚वृत्त‚त्वाद‚नात्मादिज्ञान‚स्य \textbf{मिथ्यात्वं} । त‚थापि \textbf{प्र‚श‚मानुकू}ल‚{\tiny $_{lb}$}‚\textbf{त्वात्} प्रामाण्यं प्र‚श‚मो रागादिप्र‚हाणं । अनात्मादिसामान्याकारेण व‚स्तु गृहीत्वा‚{\tiny $_{lb}$}‚ भाव‚य‚ताम्भाव‚नानिष्प‚त्ताव‚नात्मादिस्व‚ल‚क्ष‚ण‚प्र‚त्य‚क्षीकारेण रागादिप्र‚हाणात् । क‚स्य‚{\tiny $_{lb}$}‚ पु‚{\tiny $_{४}$}‚न‚र्मिथ्यात्वेपि प्र‚श‚मानुकूल‚त्वं दृष्ट‚मित्य‚त आह । \textbf{मातृसंज्ञे}त्यादि । अमा‚{\tiny $_{lb}$}‚त‚रि मातृसंज्ञा मिथ्यापि स‚ती । रागानुत्प‚त्तिकार‚णं । इय‚ता च साध‚र्म्येणायं‚{\tiny $_{lb}$}‚ दृष्टान्तः । न तु मातृसंज्ञादिकं प्र‚माणं । आदिश‚ब्दाद् भ‚गिन्यादिसंज्ञाप‚रिग्र‚हः ।‚{\tiny $_{lb}$}‚ ष‚ष्ठ्य‚र्थे चाय‚म्व‚तिः ।
	{\color{gray}{\rmlatinfont\textsuperscript{§~\theparCount}}}
	\pend% ending standard par
      ‚{\tiny $_{lb}$}‚

	  
	  \pstart \leavevmode% starting standard par
	बाह्यार्थ‚द‚र्श‚नेपि व‚स्तुभूत‚सामान्याभावाद् य‚दि स‚र्वो विक‚ल्पो भ्रान्त‚स्त‚त्र‚{\tiny $_{lb}$}‚ य‚था ज‚ल‚सामान्य‚र‚हित्‚{\tiny $_{५}$}‚आत् \textbf{\textbf{म}रीचिस्व}ल‚क्ष‚णादुत्प‚न्नो \textbf{ज‚ल‚विक‚ल्पो} भ्रान्त‚{\tiny $_{lb}$}‚  ‚{\tiny $_{lb}$}‚ ‚{\tiny $_{lb}$}‚ \leavevmode\ledsidenote{\textenglish{212/s}}स्त‚था स‚त्त्य‚ज‚लाद‚पि ज‚ल‚त्व‚शून्याज्ञातो ज‚ल‚विक‚ल्प‚स्त‚स्याप्य‚त‚स्मिंस्त‚द्ग्र‚ह‚{\tiny $_{lb}$}‚प्र‚वृत्त‚त्वाद् भ्रान्त‚त्वं । त‚त्कुत एत‚देक‚स्यार्थ‚स‚म्वादो प‚र‚स्य नेत्याह । \textbf{म‚रीचि}‚{\tiny $_{lb}$}‚केत्यादि । \textbf{अन्य‚स्य चे}ति स‚त्त्य‚ज‚ले ज‚ल‚ज्ञान‚स्य ज‚ल‚त्व‚सामान्य‚स्याभावान्म‚रीच‚यो‚{\tiny $_{lb}$}‚ ज‚लं च \textbf{भिन्नो भाव‚स्}त‚त उ\textbf{त्प‚त्ति}स्त‚स्या \textbf{अविशेषेपी}ति स‚म्ब‚न्धः । \textbf{वि‚{\tiny $_{६}$}‚भ्र‚स्म‚य‚{\tiny $_{lb}$}‚ चाविशेषे}पीति ज‚ल‚र‚हिते म‚रीचिद्र‚व्ये य‚था ज‚ल‚सामान्याध्यारोपाज्ज‚ल‚विक‚ल्पो‚{\tiny $_{lb}$}‚ विभ्र‚म‚स्त‚था स‚त्त्य‚ज‚लेपि त‚स्याप्य‚त‚स्मिँस्त‚द्ग्र‚ह‚रूप‚त्वाद‚न\textbf{भिप्रेतार्थ‚क्रिया‚{\tiny $_{lb}$}‚ पाना}दिः । त‚स्यां \textbf{योग्यं} ज‚ल‚स्य स्व‚ल‚क्ष‚ण\textbf{म‚योग्}यं म‚रीचिकानां । त‚त \textbf{उत्प‚त्ते}र्हेतोः‚{\tiny $_{lb}$}‚ \leavevmode\ledsidenote{\textenglish{79a/PSVTa}} स‚त्त्य‚ज‚ले म‚रीचिकासु च ज‚ल‚विक‚ल्प‚स्य । \textbf{य‚थाक्र‚मं स‚म्वादेत}रौ‚{\tiny $_{७}$}‚ इत‚र इत्य‚स‚म्वादः ।‚{\tiny $_{lb}$}‚ \textbf{अयो}ग्य‚म‚रीचिकास्व‚ल‚क्ष\textbf{णात्} । न ह्य‚ज‚ल‚रूपं ज‚लाकार‚स्य योग्य‚मिति म‚न्येत ।
	{\color{gray}{\rmlatinfont\textsuperscript{§~\theparCount}}}
	\pend% ending standard par
      ‚{\tiny $_{lb}$}‚

	  
	  \pstart \leavevmode% starting standard par
	\textbf{विक‚ल्पे}त्यादिना प‚रिहारः । \textbf{अर्थ‚प्र‚तिब‚द्धो}र्थाकारानुविधानेनोत्प‚त्तिः ।‚{\tiny $_{lb}$}‚ त‚देव व्याच‚ष्टे । \textbf{न ही}त्यादि । \textbf{य‚थार्थ}मिति प‚दार्थान‚तिवृत्ताव‚व्य‚यीभावः । एव‚{\tiny $_{lb}$}‚कार‚श्च भिन्न‚क्र‚मः । नैव हि प‚दार्थानुरूपं ग्राह‚क‚मुत्प‚द्य‚त इत्य‚र्थः ॥
	{\color{gray}{\rmlatinfont\textsuperscript{§~\theparCount}}}
	\pend% ending standard par
      ‚{\tiny $_{lb}$}‚

	  
	  \pstart \leavevmode% starting standard par
	क‚थ‚न्त‚र्हि ज‚ल‚ज्ञान‚{\tiny $_{१}$}‚स्य म‚रीचिकाभ्य उत्प‚त्तिरुक्तेति चेदाह । \textbf{स‚ती}त्यादि ।‚{\tiny $_{lb}$}‚ म‚रीचिकासु च‚क्षुर्विज्ञानादौ भ्रान्त‚मुप‚जाय‚ते त‚स्मिन् स‚त्य‚नुभूताकाराध्यारोपिणी‚{\tiny $_{lb}$}‚ \textbf{ज‚ल‚भ्रान्ति}रिति पार‚म्प‚र्येण \textbf{त‚दुद्भ‚वा} म‚रीचिकोद्भ‚वेत्युच्य‚ते । \textbf{य‚थास्व‚भा}व‚मिति‚{\tiny $_{lb}$}‚ पूर्व‚व‚द‚व्य‚यीभावः । ज‚ल‚भ्रान्तिज‚न‚नास‚म‚र्थं घ‚टा\textbf{द्य‚ज‚ल}मित्युक्तं । त‚तो \textbf{विवेकि}ना‚{\tiny $_{lb}$}‚ ज‚ल‚भ्रान्तिज‚न‚न‚स\textbf{म‚र्थे}नेति याव‚{\tiny $_{२}$}‚त् । एवंभूतेन म‚रीचिकाख्येनार्थेन । \textbf{न य‚था‚{\tiny $_{lb}$}‚स्व‚भावं ज‚न‚ना}त् त‚दुद्भ‚वेत्युच्य‚ते इति स‚म्ब‚न्धः । य‚था स्व‚भाव‚मित्य‚स्यैवार्थः [।]‚{\tiny $_{lb}$}‚ \textbf{स्व‚भावानुका}रेत्यादि । स्व‚भाव‚म‚नुक‚रोतीति स्व‚भावानुकारः । स्व‚ल‚क्ष‚णानुरूपं‚{\tiny $_{lb}$}‚ प्र‚तिविम्ब‚क‚न्त‚स्\textbf{यार्प‚णेन} ज्ञाने स‚मारोप‚णेन । सा पुनः केन साक्षाज्ज‚न्य‚त‚{\tiny $_{lb}$}‚ इत्याह । \textbf{सा त्वि}त्यादि । \textbf{से}ति ज‚ल‚भ्रान्तिर्ज‚लात् म‚रीचिकाया यो \textbf{वि‚{\tiny $_{३}$}‚शे}ष‚स्त\textbf{स्य‚{\tiny $_{lb}$}‚ ल‚क्ष}ण‚म्भेदेनाव‚धार‚ण‚न्त‚त्रा\textbf{पाट‚वा}द्धेतोः । \textbf{स्व‚वास‚ना} । ज‚ल‚भ्रान्तिबीज‚न्त‚स्याः‚{\tiny $_{lb}$}‚ \textbf{प्र‚बो}ध आनुगुण्यं तेन \textbf{ज‚न्य‚ते} । किम्विशिष्टेन \textbf{प्र‚त्य‚यापेक्षिणा} प्र‚त्य‚यो म‚रीचिका‚{\tiny $_{lb}$}‚द‚र्श‚नं ज‚ल‚साध‚र्म्य‚स्म‚र‚णं च । \textbf{त‚स्मा}दित्युप‚संहारः । विजातीयाद् \textbf{भिन्नो भावः}‚{\tiny $_{lb}$}‚ ‚{\tiny $_{lb}$}‚ \leavevmode\ledsidenote{\textenglish{213/s}}स्व‚ल‚क्ष‚ण‚मात्र‚न्त‚तो \textbf{ज‚न्}म य‚स्य विक‚ल्प‚विभ्र‚म‚स्य स त‚था विक‚ल्प एव विभ्र‚म‚{\tiny $_{lb}$}‚ इति विग्र‚{\tiny $_{४}$}‚हः । \textbf{न व्य‚तिरिक्त‚स्य सामान्य‚स्}य \textbf{द‚र्श‚नात् । इहे}ति प्र‚तीतिप्र‚स‚ङ्गात् ।‚{\tiny $_{lb}$}‚ नाव्य‚तिरिक्त‚स्य सामान्य‚स्य द‚र्श‚नात् । \textbf{प्र‚त्य‚भिज्ञा}न‚मिति प्र‚कृतेन स‚म्ब‚न्धः ।‚{\tiny $_{lb}$}‚ \textbf{व्य‚क्तिव‚द‚न}न्व‚यादिति । व्य‚क्त्यात्म‚के त‚द्व‚देव त‚स्यान‚न्व‚यात् सामान्य‚रूप‚मेव‚{\tiny $_{lb}$}‚ नास्तीति ॥
	{\color{gray}{\rmlatinfont\textsuperscript{§~\theparCount}}}
	\pend% ending standard par
      ‚{\tiny $_{lb}$}‚

	  
	  \pstart \leavevmode% starting standard par
	\textbf{प‚र‚स्ये}ति सामान्य‚वादिनः । सेति सामान्याकारा । \textbf{केव‚ला}दिति व्य‚क्ति‚{\tiny $_{lb}$}‚निर‚पेक्ष‚त्वात् । \href{http://sarit.indology.info/?cref=pv.3.98}{। १०१ ॥}
	{\color{gray}{\rmlatinfont\textsuperscript{§~\theparCount}}}
	\pend% ending standard par
      ‚{\tiny $_{lb}$}‚

	  
	  \pstart \leavevmode% starting standard par
	\textbf{न ही}त्यादि विव‚{\tiny $_{५}$}‚र‚णं ।
	{\color{gray}{\rmlatinfont\textsuperscript{§~\theparCount}}}
	\pend% ending standard par
      ‚{\tiny $_{lb}$}‚

	  
	  \pstart \leavevmode% starting standard par
	क‚स्मान्नाहेतीत्याह । \textbf{नित्य}मित्यादि । \textbf{त‚न्मात्र‚विज्ञान} इति सामान्य‚मात्र‚ग्र‚हे ।
	{\color{gray}{\rmlatinfont\textsuperscript{§~\theparCount}}}
	\pend% ending standard par
      ‚{\tiny $_{lb}$}‚

	  
	  \pstart \leavevmode% starting standard par
	\textbf{य‚दी}त्यादिना व्याच‚ष्टे । \textbf{अन‚ये}ति सामान्याकार‚या । \textbf{अनेन ज्ञानेने}ति सामा‚{\tiny $_{lb}$}‚न्याल‚म्बिना ।
	{\color{gray}{\rmlatinfont\textsuperscript{§~\theparCount}}}
	\pend% ending standard par
      ‚{\tiny $_{lb}$}‚

	  
	  \pstart \leavevmode% starting standard par
	\textbf{त‚दे}ति व्य‚क्तेर‚ग्र‚हे । \textbf{स‚म्ब}द्ध‚स्य सामान्य‚युक्त‚स्य \textbf{त‚द्व}तः सामान्य‚व‚तः ॥
	{\color{gray}{\rmlatinfont\textsuperscript{§~\theparCount}}}
	\pend% ending standard par
      ‚{\tiny $_{lb}$}‚

	  
	  \pstart \leavevmode% starting standard par
	\textbf{त‚द्व‚त्ता} सामान्य‚व‚त्ता । य‚दि सामान्य‚न्त‚दाश्र‚य‚श्च तेन ज्ञानेन गृह्येत त‚दो‚{\tiny $_{lb}$}‚भ‚य‚ग्र‚ह‚ण‚{\tiny $_{६}$}‚पूर्व‚क‚स्त\textbf{द्व‚त्तानिश्च‚यो} न भ‚वेत् । \textbf{त‚तः} सामान्य‚ग्र‚हाद् \textbf{व्य‚व‚हारो} व्य‚क्तौ‚{\tiny $_{lb}$}‚ प्र‚वृत्तिः \textbf{क‚थ‚न्}नैव ।
	{\color{gray}{\rmlatinfont\textsuperscript{§~\theparCount}}}
	\pend% ending standard par
      ‚{\tiny $_{lb}$}‚

	  
	  \pstart \leavevmode% starting standard par
	\textbf{य‚दे}त्यादिना व्याच‚ष्टे । \textbf{न तावि}ति सामान्य‚त‚द्व‚न्तौ । \textbf{इद‚म‚स्य} भेद‚स्य \textbf{सामा‚{\tiny $_{lb}$}‚न्यं । त‚था चे}ति सामान्य‚त‚द्व‚तोः स‚म्ब‚न्धाप्र‚तिप‚त्तौ । \textbf{त‚त्प्र‚तिप‚त्त्या} सामान्य‚प्र‚ति‚{\tiny $_{lb}$}‚प‚त्त्या । \textbf{त‚द्व‚ति} सामान्य‚व‚ति । \textbf{अर्थान्त‚र‚व}दिति न ह्य‚श्व‚प्र‚तिप‚त्तिकाले त‚द्रूपेणा‚{\tiny $_{lb}$}‚‚{\tiny $_{lb}$}‚ \leavevmode\ledsidenote{\textenglish{214/s}}\leavevmode\ledsidenote{\textenglish{79b/PSVTa}} गृहीते गोद्र‚व्ये‚{\tiny $_{७}$}‚ऽश्व‚प्र‚तिप‚त्त्या प्र‚तिप‚त्तिर‚स्ति । एक‚व‚स्तुस‚हाया इति सामान्य‚{\tiny $_{lb}$}‚स‚हायाः ।
	{\color{gray}{\rmlatinfont\textsuperscript{§~\theparCount}}}
	\pend% ending standard par
      ‚{\tiny $_{lb}$}‚

	  
	  \pstart \leavevmode% starting standard par
	\textbf{स्यादेत}दित्येत‚स्यैव व्याख्यानं । \textbf{त‚स्ये}ति ज्ञान‚स्य । \textbf{एकं स‚ह‚का}र्य‚स्तीति‚{\tiny $_{lb}$}‚ सामान्यं स‚ह‚कारि भ‚व‚तीत्य‚र्थः । \textbf{त‚दा त‚त्स‚हि}ताः सामान्य‚स‚हिताः । एवं च सामा‚{\tiny $_{lb}$}‚न्य‚त‚द्व‚तोर्द्व‚यो\textbf{र्ग्र‚ह‚णात्} सामान्य‚प्र‚तिप‚त्त्या त‚द्व‚ति प्र‚तिप‚त्तिः सिद्धेति भावः ॥
	{\color{gray}{\rmlatinfont\textsuperscript{§~\theparCount}}}
	\pend% ending standard par
      ‚{\tiny $_{lb}$}‚

	  
	  \pstart \leavevmode% starting standard par
	\textbf{त‚दि}त्या चा र्यः । \textbf{एक‚म्व‚स्तु} सामान्य\textbf{न्तासां} व्य‚क्तीनां \textbf{नानात्वं स‚म‚पोह}त्य‚{\tiny $_{lb}$}‚प‚न‚य‚ति । किं पुन‚स्तासां नानात्वापोह इष्य‚त इत्याह । \textbf{नानात्वाच्}चेत्यादि । ह्य‚र्थः‚{\tiny $_{lb}$}‚ च श‚ब्दः । \textbf{ता}स्विति व्य‚क्तिषु । \textbf{कि}मित्यादिना व्याच‚ष्टे । \textbf{तेनैकेन} सामान्येन ।‚{\tiny $_{lb}$}‚ भेदेषु य\textbf{न्नानात्व‚न्त‚देक‚विज्ञानाकार‚ण}त्वे भेदानां \textbf{कार‚ण‚मुच्य‚ते} । नानात्वाद्व्य‚{\tiny $_{lb}$}‚क्त‚यो नैकं विज्ञानं ज‚न‚य‚न्तीति । एक‚सामान्य‚स‚म्ब‚न्धेपि य‚दि भेदानान्नानात्वा‚{\tiny $_{lb}$}‚द‚प्र‚च्युतिर्न तेष्\textbf{वेका}का‚{\tiny $_{२}$}‚रं \textbf{विज्ञा}न‚मिति पूर्व‚व‚द्व्य‚क्तीनाम‚ग्र‚ह‚णं ॥
	{\color{gray}{\rmlatinfont\textsuperscript{§~\theparCount}}}
	\pend% ending standard par
      ‚{\tiny $_{lb}$}‚

	  
	  \pstart \leavevmode% starting standard par
	\textbf{अनेक‚म‚पि} व्य‚क्तिरूपं । \textbf{एक}मिति सामान्यं । \textbf{नेत्}यादिना व्याच‚ष्टे । य‚दि‚{\tiny $_{lb}$}‚ \textbf{भेदाज‚न‚न‚विरोधी} स्यात् त‚दा स‚त्य‚पि सामान्ये \textbf{भेदान्न ज‚न‚यंत्येवैकं विज्ञानं} ।
	{\color{gray}{\rmlatinfont\textsuperscript{§~\theparCount}}}
	\pend% ending standard par
      ‚{\tiny $_{lb}$}‚

	  
	  \pstart \leavevmode% starting standard par
	\textbf{ताभि}रिति व्य‚क्तिभिः [।] किं पुनः स‚म‚स्ताभिरेव \textbf{विना । ने}त्याह । \textbf{प्र‚त्येक‚{\tiny $_{lb}$}‚मि}ति । त‚था हि शाव‚लेयाभावे बाहुलेये गोबुद्धिस्त‚था त‚द‚भावेन्य‚त्राप्येव‚{\tiny $_{lb}$}‚म्प्र‚त्येकं स‚र्वा‚{\tiny $_{३}$}‚सां व्य‚क्तीनाम‚भावेपि । \textbf{तेनैके}न सामान्येन \textbf{क्रिय‚माणां} घियं‚{\tiny $_{lb}$}‚  ‚{\tiny $_{lb}$}‚ ‚{\tiny $_{lb}$}‚ \leavevmode\ledsidenote{\textenglish{215/s}}प्र‚त्य‚भिज्ञानात्मिकां प्र‚ति \textbf{साम‚र्थ्य‚न्तासां} व्य‚क्तीनां नास्ति । इति हेतोर\textbf{ग्र‚हो धिया}‚{\tiny $_{lb}$}‚ सामान्य‚ज्ञानेन \textbf{तासां} व्य‚क्तीनां ।
	{\color{gray}{\rmlatinfont\textsuperscript{§~\theparCount}}}
	\pend% ending standard par
      ‚{\tiny $_{lb}$}‚

	  
	  \pstart \leavevmode% starting standard par
	त‚देवाह [।] \textbf{क‚थ}मित्यादि । \textbf{त‚त्र} ज्ञाने सामान्याकारे । \textbf{त‚द्भावादि}ति‚{\tiny $_{lb}$}‚ सामान्य‚प्र‚त्य‚य‚स्य भावात् । एतेन व्य‚तिरेकाभाव उक्तः । प्र‚योग‚स्तु यो‚{\tiny $_{lb}$}‚ य‚द‚भावेपि भ‚व‚ति न त‚त्त‚न्निमि‚{\tiny $_{४}$}‚त्तं य‚था शालिबीजाभावेपि भ‚व‚न् य‚वाङ्‚{\tiny $_{lb}$}‚कुरः । भ‚व‚ति च प्र‚त्येकं शाव‚लेयाद्य‚भावेपि बाहुलेयादौ गोबुद्धिरिति व्याप‚क‚{\tiny $_{lb}$}‚विरुद्धः । त‚न्निमित्त‚तायास्त‚द‚भावेन व्याप्त‚त्वात् । सामान्य‚स्य तु त‚त्र श‚क्ति‚{\tiny $_{lb}$}‚रित्याह । \textbf{अस‚ती}त्यादि । अस‚ति \textbf{सामान्ये} सामान्य‚बुद्धे\textbf{र‚भावात्} । अनेन व्य‚तिरेक‚{\tiny $_{lb}$}‚ उक्तः । \textbf{इत‚र‚था चे}ति स‚ति सामान्ये सामा‚{\tiny $_{५}$}‚न्य‚बुद्धे\textbf{र्भावा}त् । अनेनान्व‚य‚{\tiny $_{lb}$}‚ उक्तः ॥
	{\color{gray}{\rmlatinfont\textsuperscript{§~\theparCount}}}
	\pend% ending standard par
      ‚{\tiny $_{lb}$}‚

	  
	  \pstart \leavevmode% starting standard par
	य‚दि सामान्य‚स‚हितानामेव व्य‚क्तीनां सामान्य‚बुद्धिं प्र‚ति साम‚र्थ्य‚मिष्य‚ते ।‚{\tiny $_{lb}$}‚ त‚दा त‚थोक्त‚न्यायेन सामान्य‚स्यैव श‚क्तिर्न व्य‚क्तीनामिति ।
	{\color{gray}{\rmlatinfont\textsuperscript{§~\theparCount}}}
	\pend% ending standard par
      ‚{\tiny $_{lb}$}‚

	  
	  \pstart \leavevmode% starting standard par
	\textbf{नैष दोष} इत्याह प‚रः । \textbf{एकापायेपी}त्येकैक‚स्यापायेपीत्य‚र्थः । त‚था हि य‚था‚{\tiny $_{lb}$}‚ नीलादिस‚मुदायाल‚म्ब‚नं \textbf{च‚क्षुर्विज्ञान}मुत्प‚द्य‚ते । त‚त्र चैकैक‚स्मिन् नीलादाव‚प‚नीते‚{\tiny $_{lb}$}‚ भ‚व‚{\tiny $_{६}$}‚त्येव प‚रिशिष्टे च व‚र्ण्ण‚स\textbf{स‚मू}हे च‚क्षुर्बुद्धि\textbf{र्न} चेय‚ता नीलादीना\textbf{म‚साम‚र्थ्य}‚{\tiny $_{lb}$}‚ स‚मूहे किन्तु साम‚र्थ्य‚मेव । प्र‚त्येकं नीलादीनां स‚मूह‚ज्ञाने । \textbf{त‚थेहापि} व्य‚क्तिष्\textbf{वेकै‚{\tiny $_{lb}$}‚कापायेपि} भ‚व‚ति सामान्य‚विज्ञान‚मिति । नेय‚ता प्र‚त्येकं \textbf{स‚र्व‚दा} व्य‚क्तीना\textbf{म‚सा‚{\tiny $_{lb}$}‚म‚र्थ्य}मिति स‚म्ब‚न्धः । त‚त‚श्च यो य‚द‚भावेपि भ‚व‚तीत्यादि प्र‚योगेऽनेकान्त इति ।
	{\color{gray}{\rmlatinfont\textsuperscript{§~\theparCount}}}
	\pend% ending standard par
      ‚{\tiny $_{lb}$}‚

	  
	  \pstart \leavevmode% starting standard par
	\textbf{विष‚म} इ त्या चा र्यः‚{\tiny $_{७}$}‚ । उप‚न्य‚स्य‚त इत्युप\textbf{न्या}सो नीलादिदृष्टान्त‚स्त‚स्य प्र‚क्रा- \leavevmode\ledsidenote{\textenglish{80a/PSVTa}}‚{\tiny $_{lb}$}‚ न्तेन साम्य‚न्नास्तीत्य‚र्थः । नैवं व्य‚क्तेः क‚थंच‚नेति । न प्र‚त्येकं स‚म‚स्तानां व्य‚क्तीनां‚{\tiny $_{lb}$}‚ साम‚र्थ्य‚मित्य‚र्थः ।
	{\color{gray}{\rmlatinfont\textsuperscript{§~\theparCount}}}
	\pend% ending standard par
      ‚{\tiny $_{lb}$}‚‚{\tiny $_{lb}$}‚\textsuperscript{\textenglish{216/s}}

	  
	  \pstart \leavevmode% starting standard par
	त‚द्व्याच‚ष्टे । \textbf{नीलादीनामि}त्यादि । \textbf{प्र‚त्येक‚म}पि \textbf{साम‚र्थ्य दृष्ट‚मिति} ।‚{\tiny $_{lb}$}‚ नीलाद‚यो हि य‚था स्वेन स्वेन रूपेण भिन्नास्त‚द्व‚च्च‚क्षुर्विज्ञानान्य‚पि ।‚{\tiny $_{lb}$}‚ स्वाकार‚भेदात् । त‚त्र नील‚स‚हिते‚{\tiny $_{१}$}‚न स‚मूहेन य‚ज्ज‚न्य‚ते च‚क्षुर्विज्ञान‚न्न त‚त्त‚द्विक‚{\tiny $_{lb}$}‚लेन य‚च्च त‚द्विक‚लेन न ज‚न्य‚ते त‚द‚न्य‚देव । त‚स्मात् स‚मूहाकारोप‚र‚क्त‚स्य विज्ञान‚{\tiny $_{lb}$}‚स्य \textbf{प्र‚त्ये}क‚न्न नीलादीन्प्र‚त्य‚न्व‚य‚व्य‚तिरेकानुविधानाद् ग‚म्य‚ते तेषां प्र‚त्येकं साम‚{\tiny $_{lb}$}‚र्थ्य‚मिति \textbf{स‚मूहेपि श‚क्तिर‚विरुद्धा । त‚थे}ति नीलादिव‚त् । \textbf{न क‚दाचि}दिति प्र‚त्येकं‚{\tiny $_{lb}$}‚ संह‚ता वा ।
	{\color{gray}{\rmlatinfont\textsuperscript{§~\theparCount}}}
	\pend% ending standard par
      ‚{\tiny $_{lb}$}‚

	  
	  \pstart \leavevmode% starting standard par
	एत‚दाह । य‚दि शाव‚लेय‚स‚हित‚सामान्य‚ज‚न्यं गोज्ञान‚म‚न्य‚द‚{\tiny $_{२}$}‚न्य‚च्च बाहु‚{\tiny $_{lb}$}‚लेय‚स‚हाय‚ज‚न्यं स्यात्[।]त‚दा व्य‚क्तीनां प्र‚त्येकं स्वाश्र‚य‚द्वार‚भाविज्ञाने श‚क्तिर्ग‚म्येत ।‚{\tiny $_{lb}$}‚ किन्त्वेक‚मेव स‚र्वासु व्य‚क्तिषु प्र‚त्य‚भिज्ञान‚न्त‚स्य स‚र्व‚त्रैकाकार‚त्वात् । प्र‚त्येकं व्य‚क्तीनां‚{\tiny $_{lb}$}‚ चाभावेपि सामान्यादेवास्योत्प‚त्तेः । \textbf{त‚स्माद‚स‚म‚र्था एव व्य‚क्त‚य‚स्त‚त्र} सामान्य‚{\tiny $_{lb}$}‚ज्ञाने । \textbf{इति} हेतोस्\textbf{तेन} सामान्य‚ज्ञानेन \textbf{न गृह्येर‚न्} ॥
	{\color{gray}{\rmlatinfont\textsuperscript{§~\theparCount}}}
	\pend% ending standard par
      ‚{\tiny $_{lb}$}‚

	  
	  \pstart \leavevmode% starting standard par
	\textbf{तासाम्}व्य‚क्तीनाम्म‚ध्येऽ\textbf{न्य}त\textbf{मापेक्ष}मि‚{\tiny $_{३}$}‚ति कांचिद्व्य‚क्तिम‚पेक्ष्येत्य‚र्थः । \textbf{त‚दि}ति‚{\tiny $_{lb}$}‚ सामान्यं \textbf{केव‚लं} व्य‚क्तिनिर‚पेक्षं ।
	{\color{gray}{\rmlatinfont\textsuperscript{§~\theparCount}}}
	\pend% ending standard par
      ‚{\tiny $_{lb}$}‚

	  
	  \pstart \leavevmode% starting standard par
	\textbf{अथे}त्यादिना व्याच‚ष्टे । \textbf{कुविन्द}स्त‚न्तुवायः । ब‚हूनां वेमानाम्म‚ध्ये \textbf{प्र‚त्येकं‚{\tiny $_{lb}$}‚ वेमाभावे}प्येकेनान्य‚त‚मेन \textbf{प‚टं क}रोतीति [।] य‚द्य‚पि स‚र्वेषां व्य‚भिचार‚स्त‚थापि \textbf{न‚{\tiny $_{lb}$}‚ त‚त} एव \textbf{कुविन्दादे}व वेम‚र‚हितात् \textbf{प‚टोत्प‚त्तिः} श‚क्या व‚क्तुं । य‚स्मान्न वेम‚र‚हितः‚{\tiny $_{lb}$}‚ कुविन्दः प‚टं क‚रोति‚{\tiny $_{४}$}‚ केव‚ल‚स्य प‚ट‚क‚र‚णाश‚क्तेः । \textbf{त‚था च न सामान्यं केव‚ल‚न्त‚{\tiny $_{lb}$}‚द्धे}तुर्विज्ञान‚हेतुः ।
	{\color{gray}{\rmlatinfont\textsuperscript{§~\theparCount}}}
	\pend% ending standard par
      ‚{\tiny $_{lb}$}‚

	  
	  \pstart \leavevmode% starting standard par
	\textbf{एव‚मि} त्या चा र्यः । \textbf{तासा}म‚न्य‚त‚मापेक्षं सामान्यं श‚क्त‚मिति ब्रुव‚ता व्य‚क्त्यु‚{\tiny $_{lb}$}‚प‚कार्यं सामान्य‚मिष्ट‚म‚नुप‚कारिण्य‚पेक्षायोगात् ॥
	{\color{gray}{\rmlatinfont\textsuperscript{§~\theparCount}}}
	\pend% ending standard par
      ‚{\tiny $_{lb}$}‚‚{\tiny $_{lb}$}‚\textsuperscript{\textenglish{217/s}}

	  
	  \pstart \leavevmode% starting standard par
	एवं चे\textbf{त्त‚देकं} सामान्य\textbf{मुप‚कुर्युस्ता} व्य‚क्त‚यः \textbf{क‚थ‚मेकां धियं च न} । एव‚श‚ब्दार्थे‚{\tiny $_{lb}$}‚ च‚श‚ब्दः । एकां प्र‚त्य‚भिज्ञानात्मिकान्धिय‚मेव क‚थं नोप‚{\tiny $_{५}$}‚कुर्य‚स्तामेवोप‚कुर्युरिति‚{\tiny $_{lb}$}‚ याव‚त् ।
	{\color{gray}{\rmlatinfont\textsuperscript{§~\theparCount}}}
	\pend% ending standard par
      ‚{\tiny $_{lb}$}‚

	  
	  \pstart \leavevmode% starting standard par
	\textbf{भिन्नेत्या}दिना व्याच‚ष्टे । भिन्नानां विल‚क्ष‚णानामेकार्थोप‚क्रिया । प्र‚त्य‚{\tiny $_{lb}$}‚भिज्ञाद्येकार्थ‚क्रियाविरोधिन्य‚स‚ति सामान्य इति \textbf{स‚र्वोयं} सामान्य‚सिद्ध्य‚र्थं \textbf{आर‚म्भः ।‚{\tiny $_{lb}$}‚ आसा}मिति व्य‚क्तीनां वि\textbf{ज्ञाने}न प्र‚त्य‚भिज्ञानाख्येनाप‚राधः कृतो य‚त् त‚द्विज्ञान‚न्न‚{\tiny $_{lb}$}‚ कुर्व‚न्ति [।] न क‚श्चित्कृत‚स्त‚स्मात्त‚देव कुर्व‚न्तीति भावः । त‚था च \textbf{किम‚{\tiny $_{६}$}‚त्रान्त‚{\tiny $_{lb}$}‚र्ग‚डुना} । घ‚टाम‚स्त‚क‚योर‚न्त‚राल‚व‚र्त्ती मांस‚पिण्डोन्त‚र्ग‚डुस्तेन तुल्य‚स्त‚थोक्तः ।‚{\tiny $_{lb}$}‚ त‚द्व‚न्निष्फ‚लेनेत्य‚र्थः ॥
	{\color{gray}{\rmlatinfont\textsuperscript{§~\theparCount}}}
	\pend% ending standard par
      ‚{\tiny $_{lb}$}‚

	  
	  \pstart \leavevmode% starting standard par
	स्यान्म‚तं [।] \textbf{भिन्नानामेक‚सामान्योप‚कार‚श‚क्ति}र‚स्त्य‚त‚स्ते सामान्य‚मेव साक्षा‚{\tiny $_{lb}$}‚दुप‚कुर्व‚ते न तु विज्ञान‚मित्य‚त आह । \textbf{य‚थे}त्यादि । \textbf{एव‚न्त‚दे}क‚मिति सामान्य‚ज‚न्यं‚{\tiny $_{lb}$}‚ य‚द्वि\textbf{ज्ञान}न्त‚देव कुर्व‚न्तु । किं सामान्योप‚कारेण निष्फ‚लेन । एव‚म्म‚{\tiny $_{७}$}‚न्य‚ते [।] विज्ञाने \leavevmode\ledsidenote{\textenglish{80b/PSVTa}}‚{\tiny $_{lb}$}‚ व्य‚क्त‚यो न स्वाकारोप‚धानेन व्याप्रिय‚न्ते [।] सामान्य‚ज्ञाने स्व‚ल‚क्ष‚ण‚स्याप्र‚तिभा‚{\tiny $_{lb}$}‚स‚नात् । किंत्वाधिप‚त्य‚मात्रेण [।] त‚च्च य‚था सामान्ये त‚था त‚द्विज्ञानेपि तुल्य‚मिति ।
	{\color{gray}{\rmlatinfont\textsuperscript{§~\theparCount}}}
	\pend% ending standard par
      ‚{\tiny $_{lb}$}‚

	  
	  \pstart \leavevmode% starting standard par
	किं च य‚दि व्य‚क्त‚यः सामान्य‚मुप‚कुर्व‚ते त‚दा \textbf{तासां} व्य‚क्तीनां \textbf{कार्य‚श्चासौ}‚{\tiny $_{lb}$}‚ सामान्यात्मा प्राप्तः । य‚स्माज्\textbf{ज‚न‚न‚मेवोप‚क्रिया} ॥
	{\color{gray}{\rmlatinfont\textsuperscript{§~\theparCount}}}
	\pend% ending standard par
      ‚{\tiny $_{lb}$}‚

	  
	  \pstart \leavevmode% starting standard par
	त‚देव स्फुट‚य‚न्नाह । \textbf{न ही}त्यादि । अतिश‚यो विशेष‚स्त‚द‚भावाद‚न‚{\tiny $_{१}$}‚तिश‚य‚{\tiny $_{lb}$}‚मात्मानं स्व‚भाव‚म‚स्य । सामान्य‚स्योप‚कार‚कास‚न्निधेः पूर्व‚व‚दुप‚कार‚क‚स‚न्निधाने‚{\tiny $_{lb}$}‚  ‚{\tiny $_{lb}$}‚ ‚{\tiny $_{lb}$}‚ \leavevmode\ledsidenote{\textenglish{218/s}}पि विभ्र‚तः क\textbf{श्चिदुप‚कार‚को} न हीति स‚म्ब‚न्धः । \textbf{अतिप्र‚स‚ङ्गात्} । एवं हि स‚र्वः‚{\tiny $_{lb}$}‚ स‚र्व‚स्योप‚कार‚कः स्यात् । त‚स्मादुप‚कार‚केणैवोप‚कार्य‚स्यातिश‚यो ज‚न्य‚त इत्य‚{\tiny $_{lb}$}‚भ्युपेयं । स चातिश‚य उप‚कार्य‚स्यात्म‚भूत इति ज‚न‚न‚मेवोप‚क्रिया । सामान्याद‚{\tiny $_{lb}$}‚र्थान्त‚र‚भूत एवातिश‚यो‚{\tiny $_{२}$}‚ व्य‚क्तिभिर्नान्य‚त इत्याह । \textbf{अर्थान्त‚र} इत्यादि । त‚स्यो‚{\tiny $_{lb}$}‚प‚कार्य‚स्य सामान्य‚स्य \textbf{किन्तेना}र्थान्त‚रेणोप‚कारेण \textbf{क्रिय}ते । त‚स्य चोप‚कार‚स्या‚{\tiny $_{lb}$}‚र्थान्त‚र‚स्य किन्तेन सामान्येन येन त‚स्य सामान्य‚स्यासावुप‚कार‚स‚म्ब‚न्धी स्यात् ।‚{\tiny $_{lb}$}‚ सामान्य‚स्य स‚म्ब‚न्धिन उप‚कार‚स्य क‚र‚णाद् व्य‚क्त‚योप्युप‚कारिण्यः स्युः । उप‚कार‚स्य‚{\tiny $_{lb}$}‚ सामान्य‚माश्र‚य‚स्त‚त आश्र‚याश्र‚यिभाव‚ल‚क्ष‚णः स‚{\tiny $_{३}$}‚म्ब‚न्ध‚स्त‚योरित्य‚त आह । \textbf{त‚स्ये}‚{\tiny $_{lb}$}‚त्यादि । त‚स्योप‚कार‚स्य \textbf{त‚दाश्र‚य‚त्वे} । त‚त्सामान्य‚माश्र‚यो य‚स्येति कृत्वा ।‚{\tiny $_{lb}$}‚ त‚स्य वा सामान्य‚स्य त‚दाश्र‚य‚त्वे । त‚स्योप‚कार‚स्याश्र‚य इति कृत्वा । उप‚क‚रोतीत्यु‚{\tiny $_{lb}$}‚प‚कारी त‚द‚भावाद‚नुप‚कारि सामान्य‚मुप‚कारः सामान्य‚कृतोऽस्त्य‚स्योप‚कार‚स्येत्यु‚{\tiny $_{lb}$}‚प‚कारी । त‚त्प्र‚तिषेधाद‚नुप‚कारी । अनुप‚कार्य इत्य‚र्थः । अर्थ‚द्व‚यं चैत‚त्त‚{\tiny $_{४}$}‚न्त्रेणोपात्त‚म्‚{\tiny $_{lb}$}‚ [।] तेनाय‚म‚र्थः [।] अनुप‚कार‚क‚स्य सामान्य‚स्यानुप‚कार्य‚स्य चोप‚कार‚स्य य‚थाक्र‚मं \textbf{क‚{\tiny $_{lb}$}‚ आश्र‚याश्र‚यिभाव} इति । \textbf{अतिप्र‚स‚ङ्गो वे}ति । अनुप‚कारिण आश्र‚याश्र‚यिभावे स‚र्व‚त्र‚{\tiny $_{lb}$}‚ त‚त्प्र‚स‚ङ्गात् । \textbf{उप‚कारे वा} सामान्य‚कृते उप‚कार‚स्याभ्युप‚ग‚म्य‚माने । \textbf{त‚त्रैव} सामान्ये‚{\tiny $_{lb}$}‚ त‚स्योप‚कार‚स्य \textbf{प्र‚तिब‚न्ध इति किम‚न्यो} व्य‚क्तिभेद\textbf{स्त‚स्यो}प‚कार‚स्य \textbf{क‚र‚णा}त् \textbf{त‚स्}य‚{\tiny $_{lb}$}‚ सामा‚{\tiny $_{५}$}‚न्य\textbf{स्योप‚कारी} नैवेति चेत् । \textbf{त‚द‚पेक्ष}स्येति व्य‚क्त्य‚पेक्ष‚स्या\textbf{श्र‚य}स्येति सामा‚{\tiny $_{lb}$}‚न्य‚स्य । उप‚कारं प्र‚त्याश्र‚य‚त्वात् । \textbf{त‚दुप‚योगे} त‚स्मिन्नुप‚कारे । उप‚योगे क‚ल्प्य‚माने‚{\tiny $_{lb}$}‚ नित्य‚त्वाद‚नुप\textbf{कार्य‚त्वे} सामान्य‚स्य \textbf{केयं} व्य‚क्तिं प्र\textbf{त्य‚पेक्षा नाम} । नैव । क‚स्त‚र्ह्य‚पे‚{\tiny $_{lb}$}‚क्ष‚त इत्य‚त आह । \textbf{त‚दुत्प‚त्ती}त्यादि । त‚स्माद‚पेक्ष‚णीयादुत्प‚त्तिः सा ध‚र्मः स्व‚भावो‚{\tiny $_{lb}$}‚ य‚स्य स \textbf{त‚दुत्प‚त्तिध‚{\tiny $_{६}$}‚र्म्म}भावः । स्व‚भाव‚स्य \textbf{प्र‚तिव‚न्धादा}य‚त्त‚त्वा\textbf{द‚पेक्ष‚ते नाम}‚{\tiny $_{lb}$}‚ उप‚कारिणं [।] नाम‚श‚ब्दः प्र‚सिद्धेर्द्योत‚कः । अनुत्प‚त्तिध‚र्म‚क‚म‚पि सामान्य‚म‚पेक्ष‚त इति‚{\tiny $_{lb}$}‚ चेदाह । \textbf{अनाधेयेत्या}दि । य‚त्तैरुप‚कार‚कैर‚नाधेयोनुत्पाद्य आत्मा\textbf{तिश‚यो} य‚स्य स‚{\tiny $_{lb}$}‚ त‚था । एवंभूत आत्मीय‚स्येति पुन‚र्ब‚हुब्रीहिरेव‚म्भूत\textbf{त्वे} \edtext{\textsuperscript{*}}{\lemma{*}\Bfootnote{?}} सामान्य‚प‚दार्थः‚{\tiny $_{७}$}‚‚{\tiny $_{lb}$}‚  ‚{\tiny $_{lb}$}‚ ‚{\tiny $_{lb}$}‚ \leavevmode\ledsidenote{\textenglish{219/s}}\textbf{अपेक्ष‚ते च प‚रानिति व्याह‚त‚मेत‚त्} । व्याप‚क‚विरुद्धोप‚ल‚ब्ध्या । त‚था ह्य‚पेक्षाधेया- \leavevmode\ledsidenote{\textenglish{81a/PSVTa}}‚{\tiny $_{lb}$}‚ तिश‚य‚त्वेन व्याप्ता । त‚द्विरुद्ध‚म‚नाधेयातिश‚त्व‚मिति ।
	{\color{gray}{\rmlatinfont\textsuperscript{§~\theparCount}}}
	\pend% ending standard par
      ‚{\tiny $_{lb}$}‚

	  
	  \pstart \leavevmode% starting standard par
	\textbf{यः क‚श्चि}द् भावः प्र‚तिब‚न्धः क‚स्य‚चिद् व‚स्तुनः \textbf{क्व‚चि}दाश्र‚ये \textbf{स स‚र्वो ज‚न्य‚तायां}‚{\tiny $_{lb}$}‚ कार्य‚तायामे\textbf{वोद्भ‚व‚ति} । आश्र‚येणाश्रित‚स्यानात्म‚भूत एवोप‚कारः क्रिय‚त इति चेदाह ।‚{\tiny $_{lb}$}‚ \textbf{प‚र‚भावेत्या}दि । \textbf{त‚द‚नुप‚कारात्त}स्योप‚कार्य‚स्या‚{\tiny $_{१}$}‚नुप‚कारात् । न ह्य‚न्य‚स्मिन्नुप‚कृतेन्य उप‚{\tiny $_{lb}$}‚कृतो नाम । त‚स्य त‚दाश्र‚य‚त्वेऽनुप‚कारिणः कोय‚माश्र‚याश्र‚यिभाव इति स‚र्व‚म्वाच्यं ।
	{\color{gray}{\rmlatinfont\textsuperscript{§~\theparCount}}}
	\pend% ending standard par
      ‚{\tiny $_{lb}$}‚

	  
	  \pstart \leavevmode% starting standard par
	न च पौन‚रुक्त्य‚दोषः । पूर्वं सामान्य‚त‚द्व‚तोरुप‚कार्योप‚कार‚क‚भाव‚द्वारे‚{\tiny $_{lb}$}‚णोक्त‚म‚धुना स‚र्व‚विष‚यं वास्त‚वं स‚म्ब‚न्ध‚मुपादायेति । त‚स्माद‚र्थान्त‚र‚क‚र‚णादाश्र‚{\tiny $_{lb}$}‚याभिम‚तोकिंचित्क‚रः [।] त‚थाभूतोप्युप‚कार‚क इति चेदाह‚{\tiny $_{२}$}‚ । \textbf{अकिञ्चित्क‚{\tiny $_{lb}$}‚र}स्येत्यादि । य‚त एव\textbf{न्त‚स्माद् विज्ञान‚ज‚न‚न} इत्य‚न्व‚यिविज्ञान‚ज‚न‚ने । \textbf{व्य‚क्त}‚{\tiny $_{lb}$}‚म‚व‚श्य\textbf{म‚स्य} सामान्य‚स्य । \textbf{त‚त्कार्य}ता व्य‚क्तिकार्य‚ता [।]
	{\color{gray}{\rmlatinfont\textsuperscript{§~\theparCount}}}
	\pend% ending standard par
      ‚{\tiny $_{lb}$}‚

	  
	  \pstart \leavevmode% starting standard par
	य‚थोक्त‚दोष‚भ‚यात् \textbf{केव‚ल}स्य व्य‚क्त्य‚न‚पेक्ष‚स्यान्व‚यिविज्ञान‚ज‚न‚नं प्र‚ति \textbf{साम}र्थ्ये‚{\tiny $_{lb}$}‚भ्युप‚ग‚म्य‚माने \textbf{व्य‚क्तीनां क्व‚चिद}पि काले । \textbf{अत्रे}त्य‚न्व‚यिविज्ञाने \textbf{साम‚र्थ्यासि}द्धेः‚{\tiny $_{lb}$}‚कार‚णाद\textbf{ग्राह्य‚त्वं व्य‚क्तीनां} । अकार‚ण‚स्य विष‚य‚त्वायो‚{\tiny $_{३}$}‚गात् । \textbf{विज्ञाने प्र‚तिभा‚{\tiny $_{lb}$}‚स‚ना}दिति । न ह्य‚स‚म‚र्थ‚स्य श‚श‚विषाणादेर‚न्व‚यिविज्ञाने प्र‚तिभास‚न‚म‚स्ति ।‚{\tiny $_{lb}$}‚ \textbf{व्य‚क्त‚य‚स्तु} प्र‚तिभास‚न्ते [।] त‚स्मात् सामान्य‚व‚त्ता अपि \textbf{स‚म‚र्था} इति । असिद्धः‚{\tiny $_{lb}$}‚ सामान्य‚विज्ञाने उप‚कारो यासां व्य‚क्तीनान्ता \textbf{असिद्धोप‚का}रास्ता\textbf{सां क‚थं} सामान्य‚{\tiny $_{lb}$}‚विज्ञाने \textbf{प्र‚तिभा}सः [।] नैव । उप‚कार एव क‚थ‚म‚सिद्ध \textbf{इति चेदा}ह । \textbf{स एवे}त्यादि ।‚{\tiny $_{lb}$}‚ \textbf{स} इ‚{\tiny $_{४}$}‚त्युप‚कारः \textbf{सामान्याभ्युप‚ग‚मे} हि त‚स्यैव त‚त्र साम‚र्थ्य‚न्न व्य‚क्तीनां । य‚थोक्तं ।
	{\color{gray}{\rmlatinfont\textsuperscript{§~\theparCount}}}
	\pend% ending standard par
      ‚{\tiny $_{lb}$}‚

	  
	  \pstart \leavevmode% starting standard par
	\hphantom{.}ताभिर्विनापि प्र‚त्येकं क्रिय‚माणां धियं प्र‚ती त्यादि \href{http://sarit.indology.info/?cref=pv.3.102}{१ । १०५} ।
	{\color{gray}{\rmlatinfont\textsuperscript{§~\theparCount}}}
	\pend% ending standard par
      ‚{\tiny $_{lb}$}‚

	  
	  \pstart \leavevmode% starting standard par
	माभूद् व्य‚क्तीनामुप‚कारः सामान्य‚विज्ञाने प्र‚तिभास‚स्तु क‚स्मान्न भ‚व‚तीत्याह ।‚{\tiny $_{lb}$}‚ \textbf{य‚स्मा}दित्यादि । \textbf{अतिप्र‚सं}गादिति । \textbf{अनुप‚कार‚क‚स्य विष‚य‚त्वे} स‚र्व‚स्य विज्ञान‚स्य‚{\tiny $_{lb}$}‚  ‚{\tiny $_{lb}$}‚ ‚{\tiny $_{lb}$}‚ \leavevmode\ledsidenote{\textenglish{220/s}}स‚र्वो विष‚यः स्यात् । माभूद् विष‚यः प्र‚ति‚{\tiny $_{५}$}‚भासस्तुक‚स्मान्नेति \textbf{चेदाह । नावि‚{\tiny $_{lb}$}‚ष‚य‚स्ये}त्यादि । \textbf{अनुप‚कार‚क‚स्ये}त्यादि । अनुप‚कार‚स्येति पाठान्त‚रं । त‚त्र न विद्य‚ते‚{\tiny $_{lb}$}‚ विज्ञान‚कार्य‚स्योप‚कारो य‚स्माद‚र्थात् सोनुप‚कार इति व्याख्येयं । त‚स्या\textbf{विष‚य‚त्वेऽ‚{\tiny $_{lb}$}‚तीतानाग‚ता}दीनां । आदिश‚ब्दात् प्र‚धानेश्व‚रादीनां य‚थाग‚म‚कं क‚ल्पितानां \textbf{ग्र‚ह‚णं ।‚{\tiny $_{lb}$}‚ अस‚ता}मित्य‚सामान्य‚हेतुः । \textbf{भ‚व‚न्त्वित्}यादिना सि‚{\tiny $_{६}$}‚द्ध‚साध्य‚तामाह । \textbf{त‚द्विष‚याणी}‚{\tiny $_{lb}$}‚त्य‚तीत‚विष‚याणि [।] \textbf{निर्विष‚य‚त्वे} क‚थ‚न्तेष्व‚न्त‚र्भ‚विष्य‚ति चेत्येव‚माद्य‚र्थानु‚{\tiny $_{lb}$}‚कारी प्र‚तिभास इत्य‚त आह । \textbf{निर्विष‚य‚त्वे}पीत्यादि । \textbf{त‚द‚नुप‚कारी}ति योसाव‚र्थो‚{\tiny $_{lb}$}‚नुभूतोऽतीत‚श्च त‚द‚नुपकारी । अस्प‚ष्टेन रूपेणातीत‚स्यैवार्थ‚स्यानुकारान्नातीता‚{\tiny $_{lb}$}‚\leavevmode\ledsidenote{\textenglish{81b/PSVTa}} दिक‚न्नाम किञ्चिद‚स्ति य‚स्य रूप‚म‚नुकुर्यात् । स च प्र‚तिभासो‚{\tiny $_{७}$}‚ \textbf{विज्ञान‚स्यात्म‚{\tiny $_{lb}$}‚भू}त \textbf{एवा}स्प‚ष्ट‚रूप‚स्य ब‚हिर‚विद्य‚मान‚त्वात् । त‚द्रूपानुकारित्वे कार‚ण‚माह । \textbf{त‚द्रू‚{\tiny $_{lb}$}‚पानुभ‚वे}त्यादि । य‚द्रूपो व‚र्त्त‚मानार्थानुभ‚वो जात‚स्तेन या \textbf{वास‚ना आहिता} त‚त \textbf{उत्प‚{\tiny $_{lb}$}‚द्य‚मानं} ज्ञान‚म‚नुभूतार्थाकारेणोत्प‚द्य‚त इत्य‚र्थः । युक्त‚म‚तीते त‚द्रूपार्थानुभ‚वोत्प‚त्तिर्व‚र्त‚{\tiny $_{lb}$}‚ मानाव‚स्थायाम‚र्थ‚स्यानुभूत‚त्वाद् अनाग‚तादौ क‚थं । न हि त‚त्रानुभ‚वो‚{\tiny $_{१}$}‚स्ति । त‚त्रा‚{\tiny $_{lb}$}‚प्येव‚म्भूतोर्थो भ‚विष्य‚तीत्येवंभूताच्छ‚ब्दाद् योभिलाप‚संसृष्टो विक‚ल्पः स एव स्व‚सं‚{\tiny $_{lb}$}‚विदित‚त्वात् त‚द्रूपोनुभ‚व‚स्तेनाहित‚वास‚नोत्प‚त्तेर‚दोषः ।
	{\color{gray}{\rmlatinfont\textsuperscript{§~\theparCount}}}
	\pend% ending standard par
      ‚{\tiny $_{lb}$}‚

	  
	  \pstart \leavevmode% starting standard par
	एवं प्र‚धानादिविक‚ल्पेष्व‚पि य‚थाग‚मं श‚ब्दार्थाकार‚विक‚ल्पानुभ‚व‚वास‚नो‚{\tiny $_{lb}$}‚त्प‚त्तिर्व्याख्येया । एव‚म‚तीतादीनाम‚विष‚य‚त्वे प्र‚स‚ङ्गाद‚नुप‚कार‚को विष‚य इति य‚दु‚{\tiny $_{lb}$}‚क्त‚न्त‚दैकान्तिक‚मेव । त‚था व्य‚क्ती‚{\tiny $_{२}$}‚नाम‚सिद्धोप‚काराणाम‚विष‚य‚त्वान्नास्ति‚{\tiny $_{lb}$}‚ सामान्य‚ज्ञाने प्र‚तिभासः । त‚त‚श्च य‚दुक्तं [।] स‚म‚र्था व्य‚क्त‚यो विज्ञाने प्र‚तिभास‚ना‚{\tiny $_{lb}$}‚दिति [।] त‚स्यासिद्ध‚त्व‚मुक्तं ।
	{\color{gray}{\rmlatinfont\textsuperscript{§~\theparCount}}}
	\pend% ending standard par
      ‚{\tiny $_{lb}$}‚

	  
	  \pstart \leavevmode% starting standard par
	अधुना प्र‚तिभास‚म‚ङ्गीकृत्यानैकान्तिक‚त्व‚माह । \textbf{भावाभावेत्}यादि । अर्थ‚भावे‚{\tiny $_{lb}$}‚ भाव‚स्त‚द‚भावे चाभावो विज्ञान‚स्य \textbf{भावाभावानुविधानं} । त\textbf{स्मा}द्धेतोर‚र्थ‚स्य‚{\tiny $_{lb}$}‚ विज्ञाने \textbf{साम}र्थ्यं ग‚म्य‚ते । \textbf{न तु‚{\tiny $_{३}$}‚ प्र‚तिभास‚नात्} । य‚स्माद् व‚स्तुस्थित्या सामान्य‚{\tiny $_{lb}$}‚स्यास‚त्य‚पि प्र‚तिभास‚ने विक‚ल्प‚विज्ञाने साम‚र्थ्याभ्युप‚ग‚मात् । त‚देवाह [।] \textbf{अप्र‚ति‚{\tiny $_{lb}$}‚भासिनो}पीत्यादि । \textbf{न हि व्य‚क्तिव्य‚तिरेकेण सामान्यं प्र‚तिभास‚ते} । य‚द‚पि सामान्य‚{\tiny $_{lb}$}‚‚{\tiny $_{lb}$}‚ \leavevmode\ledsidenote{\textenglish{221/s}}प्र‚तिभासि विक‚ल्प‚विज्ञान‚न्त‚द‚पि व‚र्ण्ण‚संस्थानाद्याकार‚मेव । न हि त‚त्रापि व‚र्ण्णा‚{\tiny $_{lb}$}‚द्याकार‚विविक्तोन्यः सामान्याकारो ल‚क्ष्य‚ते ।‚{\tiny $_{४}$}‚ न च व‚र्ण्ण‚संस्थानाद्यात्म‚कं सामा‚{\tiny $_{lb}$}‚न्यं । त‚स्मान्न व‚स्तुस्थित्या सामान्यं प्र‚तिभास‚ते [।] त‚थापि त‚स्याप्र‚तिभासिनो‚{\tiny $_{lb}$}‚ भ‚व‚न्म‚तेन भावात् । विज्ञाने साम‚र्थ्यात् । भाव्य‚ते ज‚न्य‚ते कार्य‚म‚नेनेति भावः‚{\tiny $_{lb}$}‚ साम‚र्थ्य‚मुच्य‚ते । स‚त्य‚पि च प्र‚तिभास‚ने नास्ति साम‚र्थ्य‚मिति द‚र्श‚य‚न्नाह । \textbf{प्र‚ति‚{\tiny $_{lb}$}‚भासिना}मित्यादि । विप्ल‚व‚न्ते व‚स्तुत्वाद‚प‚ग‚च्छ‚न्तीति \textbf{विप्ल‚वाः केशाद‚य} एव‚{\tiny $_{५}$}‚ [।]‚{\tiny $_{lb}$}‚ विप्ल‚वा इति विग्र‚हः । ते हि तैमिरिकादिद‚र्श‚ने प्र‚तिभास‚न्ते त‚थापि ते\textbf{षाम‚भावात्} ।‚{\tiny $_{lb}$}‚ पूर्व‚व‚द् भाव‚श‚ब्द‚व्युत्प‚त्तेर‚साम‚र्थ्य‚म‚भाव‚श‚ब्देनोच्य‚ते । विज्ञान‚ज‚न‚नं प्र‚त्य‚साम‚र्थ्य‚{\tiny $_{lb}$}‚मित्य‚र्थः । त‚देव साम‚र्थ्येपि प्र‚तिभास‚द‚र्श‚नात् प्र‚तिभास‚नात् साम‚र्थ्य‚मित्य‚स्याने‚{\tiny $_{lb}$}‚कान्त उक्तो भ‚व‚ति ।
	{\color{gray}{\rmlatinfont\textsuperscript{§~\theparCount}}}
	\pend% ending standard par
      ‚{\tiny $_{lb}$}‚
	  \bigskip
	  \begingroup
	
	    
	    \stanza[\smallbreak]
	  {\normalfontlatin\large ``\qquad}त‚दैक‚मुप‚कुर्युस्ताः क‚थ‚मेकान्धियं च नेति \href{http://sarit.indology.info/?cref=pv.3.105}{१ । १०७}{\normalfontlatin\large\qquad{}"}\&[\smallbreak]
	  
	  
	  
	  \endgroup
	‚{\tiny $_{lb}$}‚

	  
	  \pstart \leavevmode% starting standard par
	य‚दुक्त‚न्त‚त्र‚{\tiny $_{६}$}‚ प‚र‚स्योत्त‚र‚माशंक‚ते । \textbf{अभिन्ने}त्यादिना । \textbf{अभिन्न‚प्र‚तिभासा}‚{\tiny $_{lb}$}‚ एकाकारा । \textbf{भिन्नेषु} स्व‚ल‚क्ष‚णेषु । तेषां स्व‚ल‚क्ष‚णानामाकार‚स्त‚दाकारः सोर्पित‚{\tiny $_{lb}$}‚ आहितो य‚स्याम्बुद्धौ सा त‚था [।] एवंभूता चासा\textbf{व‚भिन्न‚प्र‚तिभासिनी च न हि‚{\tiny $_{lb}$}‚ स्याद्} य‚दि विल‚क्ष‚णेभ्य एवोत्प‚द्येत । भ‚व‚ति चाभिन्न‚प्र‚तिभासिनी । त‚स्मान्न‚{\tiny $_{lb}$}‚ विल‚क्ष‚णेभ्य एवोत्प‚द्य‚ते किन्तु सामा‚{\tiny $_{७}$}‚न्य‚प‚दार्थादिति ।
	{\color{gray}{\rmlatinfont\textsuperscript{§~\theparCount}}}
	\pend% ending standard par
      \textsuperscript{\textenglish{82a/PSVTa}}‚{\tiny $_{lb}$}‚

	  
	  \pstart \leavevmode% starting standard par
	\textbf{ने}त्यादिना प्र‚तिविध‚त्ते । एत‚त्क‚थ‚य‚ति [।] य‚दि स्व‚ल‚क्ष‚णानि स्वाकारा‚{\tiny $_{lb}$}‚ वा \edtext{}{\lemma{वा}\Bfootnote{?}} तेन सामान्याकाराणां बुद्धीनां ज‚न‚कानीष्य‚न्ते त‚देत‚द् युज्य‚ते । \textbf{भिन्नेषु}‚{\tiny $_{lb}$}‚ क‚थ‚म‚भिन्न‚प्र‚तिभासा बुद्धिरिति । त‚च्च नास्ति य‚तो \textbf{न सामान्य‚प्र‚तिभासिनीषु}‚{\tiny $_{lb}$}‚ बुद्धिषु \textbf{स्व‚ल‚क्ष‚ण‚प्र‚तिभासः} । य‚स्मात् \textbf{सामान्य‚ग्राहिणीषु} बुद्धिषु स्व‚ल‚क्ष‚ण‚प्र‚तिभा‚{\tiny $_{lb}$}‚साभ्युप‚ग‚मे तिस्रः क‚ल्प‚{\tiny $_{१}$}‚नाः । १येन रूपेण च‚क्षुरादिबुद्धिषु व्य‚क्त‚यो भास‚न्ते‚{\tiny $_{lb}$}‚ तेनैव सामान्य‚बुद्धिष्व‚पि । २ सामान्य‚बुद्धौ वा य‚द्रूप‚माभाति त‚देव स्व‚ल‚क्ष‚णानां ।‚{\tiny $_{lb}$}‚ ३ रूप‚द्व‚यं वा एक‚स्य भेद‚स्याभ्युप‚ग‚न्त‚व्यं । येन केन च‚क्षुरादिबुद्धिषु भास‚ते‚{\tiny $_{lb}$}‚ऽन्येन विक‚ल्प‚बुद्धिष्विति ।
	{\color{gray}{\rmlatinfont\textsuperscript{§~\theparCount}}}
	\pend% ending standard par
      ‚{\tiny $_{lb}$}‚

	  
	  \pstart \leavevmode% starting standard par
	१ त‚त्राद्ये प‚क्षे च‚क्षुरादिबुद्धिव‚त् स्व‚ल‚क्ष‚णाभावे सामान्युबुद्धीनाम‚प्य‚{\tiny $_{lb}$}‚भावः स्यान्न चैवं । त\textbf{द‚भावेपि} त‚स्य‚{\tiny $_{२}$}‚ स्व‚ल‚क्ष‚ण‚स्याभावेपि \textbf{तासां} सामान्य‚बुद्धीनां‚{\tiny $_{lb}$}‚ ‚{\tiny $_{lb}$}‚ \leavevmode\ledsidenote{\textenglish{222/s}}\textbf{भावा}त् । २ द्वितीय‚स्य प‚क्ष‚स्याभाव‚माह । \textbf{आकारा}न्त‚रेत्यादि । सामान्या‚{\tiny $_{lb}$}‚कारा\textbf{दाकारान्त‚रे}णासाधार‚णेन \textbf{स्व‚ज्ञाने} च‚क्षुरादिज्ञाने प्र‚तिभास‚नान्न सामान्या‚{\tiny $_{lb}$}‚कार एव रूपं व्य‚क्तीनां । ३ तृतीय‚म्प‚क्षं निराक‚र्त्तुमाह । \textbf{अनेकाकारायोगादिति} ।‚{\tiny $_{lb}$}‚ एक‚स्यानेक‚त्व‚म‚युक्त‚मेकानेक‚त्व‚योर्विरोधात् । \textbf{अतिप्र‚स‚ङ्गा‚{\tiny $_{३}$}‚च्चे}त्येक‚स्यानेक‚त्व‚क‚ल्प‚{\tiny $_{lb}$}‚नायां न क्व‚चिदेक‚त्वं स्यादित्य‚र्थः । स्व‚ल‚क्ष‚णं च सामान्य‚बुद्धौ न प्र‚तिभास‚ते ।
	{\color{gray}{\rmlatinfont\textsuperscript{§~\theparCount}}}
	\pend% ending standard par
      ‚{\tiny $_{lb}$}‚

	  
	  \pstart \leavevmode% starting standard par
	य‚त एव\textbf{न्त‚स्मान्नेयं} सामान्याकारा बुद्धिः । \textbf{भिन्नार्थ‚ग्राहिणी} [।] आहित‚स्व‚ल‚{\tiny $_{lb}$}‚क्ष‚णाकारा स‚त्\textbf{य‚भिन्नाकारा भाति । त‚दुद्भ‚वा} भिन्न‚प‚दार्थोद्भ‚वा । किन्तु स्व‚ल‚{\tiny $_{४}$}‚‚{\tiny $_{lb}$}‚ क्ष‚ण‚ग्राहिणोनुभ‚वेनाहितां वास‚नामाश्रित्य प्र‚कृत्या भ्रान्तैवेय‚मुत्प‚द्य‚ते । पार‚म्प‚र्ये‚{\tiny $_{४}$}‚‚{\tiny $_{lb}$}‚च व्य‚क्त‚य‚स्त‚स्याः कार‚णं क‚थ्य‚न्ते ।
	{\color{gray}{\rmlatinfont\textsuperscript{§~\theparCount}}}
	\pend% ending standard par
      ‚{\tiny $_{lb}$}‚

	  
	  \pstart \leavevmode% starting standard par
	य‚दि सामान्य‚बुद्धिर्न स्व‚ल‚क्ष‚ण‚प्र‚तिभासिनी क‚थं स्व‚ल‚क्ष‚णे लोकं प्र‚व‚र्त्त‚य‚{\tiny $_{lb}$}‚तीति चेदाह । \textbf{अत‚त्प्र‚तिभासिन्य‚पी}त्यादि । अस्व‚ल‚क्ष‚ण‚प्र‚तिभासिन्य‚पि स्व‚प्र‚ति‚{\tiny $_{lb}$}‚भासेऽन‚र्थेऽअर्था\textbf{ध्य‚व‚साय‚विभ्र‚मा}द्धेतो\textbf{र्व्य‚व‚हार‚य‚ति लोकं} दृश्य‚विक‚ल्प्यावेकीकृत्य‚{\tiny $_{lb}$}‚ प्र‚व‚र्त्त‚य‚तीति याव‚त् । य‚दि सामान्य‚बुद्धिः सामान्याकारा स एव पार‚मार्थिक‚न्त‚र्हि‚{\tiny $_{lb}$}‚ सामान्य‚म्भ‚विष्य‚तीत्याह । \textbf{स तु} त‚स्यामित्यादि । \textbf{स प्र‚तिभास‚मानः} सामान्या\textbf{कारो‚{\tiny $_{lb}$}‚ नार्थेष्व‚स्ति} । त‚स्य व्य‚तिरिक्त‚स्य व्य‚तिरेकेणानुप‚ल‚म्भ‚नात् । अव्य‚तिरिक्त‚स्य च‚{\tiny $_{lb}$}‚ व्य‚क्तिव‚द‚न‚न्व‚यात् ।
	{\color{gray}{\rmlatinfont\textsuperscript{§~\theparCount}}}
	\pend% ending standard par
      ‚{\tiny $_{lb}$}‚

	  
	  \pstart \leavevmode% starting standard par
	क‚थ‚न्त‚र्हि व्य‚क्तिष्व‚भिन्नाकार‚प्र‚तिभास इत्याह । \textbf{अन्य‚त्र भेदाद‚भेदिन} इति ।‚{\tiny $_{lb}$}‚ भेदोन्यापोहः स एव प्र‚तिव्य‚क्त्य‚भेदी । त‚था हि य‚थैका गोव्य‚क्तिर‚गोव्यावृत्ता त‚{\tiny $_{lb}$}‚थान्यापि‚{\tiny $_{६}$}‚ । त‚द‚नेन प्र‚कारेण स्व‚ल‚क्ष‚णान्येव विजातीय‚व्यावृत्तान्य‚भेदीनि भेद‚{\tiny $_{lb}$}‚इत्युच्य‚न्ते । अन्य‚त्र श‚ब्द‚श्चाय‚म्विभ‚क्त्य‚न्त‚प्र‚तिरूप‚को निपातः । अन्य‚श‚ब्द‚{\tiny $_{lb}$}‚स‚मानार्थः । न त्व‚य‚न्त्र‚ल्प्र‚त्य‚यान्तः स‚प्त‚म्य‚र्थ‚स्याविव‚क्षित‚त्वात् । तेनाय‚म‚र्थो‚{\tiny $_{lb}$}‚ य‚थोक्तेन प्र‚कारेण स्व‚ल‚क्ष‚णात्म‚काद् भेदाद‚भेदिनोन्यः प्र‚तिभास‚मान आकारो‚{\tiny $_{lb}$}‚\leavevmode\ledsidenote{\textenglish{82b/PSVTa}} र्थेषु नास्ति किन्तु स्व‚ल‚क्ष‚णात्म‚क एव भेदो‚{\tiny $_{७}$}‚ विजातीय‚व्यावृत्तेर‚भेदी स‚र्व‚त्र विद्य‚तेऽ‚{\tiny $_{lb}$}‚भेदाध्य‚व‚सायात् । अभेदाध्य‚व‚साय‚स्य च स एव भेदः पार‚म्प‚र्येण निमित्तं ।
	{\color{gray}{\rmlatinfont\textsuperscript{§~\theparCount}}}
	\pend% ending standard par
      ‚{\tiny $_{lb}$}‚

	  
	  \pstart \leavevmode% starting standard par
	न‚नु बुद्धाव‚भिन्नाकारः प्र‚तिभास‚ते क‚थ‚म‚र्थेषु नास्तीत्युच्य‚त इत्याह । \textbf{स चा‚{\tiny $_{lb}$}‚रूप} इति [।] ह्य‚र्थे च श‚ब्दः । स हि विक‚ल्प‚प्र‚तिभ‚स्याकारो निःस्व‚भाव‚स्त‚त्त्वा‚{\tiny $_{lb}$}‚‚{\tiny $_{lb}$}‚ \leavevmode\ledsidenote{\textenglish{223/s}}न्य‚त्वेन प‚र‚मार्थ‚तो व्य‚व‚स्थाप‚यितुम‚श‚क्य‚त्वादिति संप्र‚त्येवोक्त‚त्वात् ।‚{\tiny $_{१}$}‚ \textbf{त‚मेवा‚{\tiny $_{lb}$}‚कार‚ङ्गृही}त‚बुद्धिस्त‚दाकारोत्प‚त्तिरेवास्याः ग्र‚ह‚ण\textbf{न्त‚थे}त्य‚रूप‚स्याकार‚स्य ग्र‚ह‚णाद्‚{\tiny $_{lb}$}‚ \textbf{विप्ल‚व‚ते} भ्रान्ता भ‚व‚ती\textbf{त्युक्त‚म्प्राक्} । अश‚क्तिरेषा विक‚ल्पानाम‚विद्याप्र‚भ‚व‚त्वा‚{\tiny $_{lb}$}‚दिनोक्त‚त्वात् ।
	{\color{gray}{\rmlatinfont\textsuperscript{§~\theparCount}}}
	\pend% ending standard par
      ‚{\tiny $_{lb}$}‚

	  
	  \pstart \leavevmode% starting standard par
	दोष‚स्य प‚रिहार‚मुक्त्वाऽधुना तुल्य‚दोष‚तामापाद‚य‚न्नाह । \textbf{अपि} चेत्यादि । \textbf{न‚{\tiny $_{lb}$}‚ तुल्य‚मि}ति प‚रः । \textbf{त‚त्रेति भिन्नासु} व्य‚क्तिषु त‚त एव सामान्यात्तुल्याकाराद् बुद्धि‚{\tiny $_{lb}$}‚रिति \textbf{न तु‚{\tiny $_{२}$}‚ल्यं चोद्यं} ।
	{\color{gray}{\rmlatinfont\textsuperscript{§~\theparCount}}}
	\pend% ending standard par
      ‚{\tiny $_{lb}$}‚

	  
	  \pstart \leavevmode% starting standard par
	\textbf{न‚न्वि}त्यादि सि द्धा न्त वा दी । \textbf{त‚त्र} व्य‚क्तिषु त‚स्य सामान्य‚स्य \textbf{स‚तोष्या}‚{\tiny $_{lb}$}‚ \edtext{\textsuperscript{*}}{\lemma{*}\Bfootnote{? प्या}} \textbf{भास} आकारो \textbf{न} ल‚क्ष्य‚ते । य‚द्वा त‚त्रेति विक‚ल्पिकाविक‚ल्पिकायां बुद्धौ ।‚{\tiny $_{lb}$}‚ स्यादेत‚द् [।] विक‚ल्पिकायान्त‚स्याभासोस्तीत्याह । \textbf{सा हीत्या}दि । \textbf{सा हि}‚{\tiny $_{lb}$}‚ विक‚ल्पिका बुद्धि\textbf{र्व‚र्ण्ण‚संस्थान‚व‚ती विभाव्य}तेऽनुभूय‚ते । व‚र्ण्णाद्याकार‚मेव सामान्य‚{\tiny $_{lb}$}‚मिति चेदाह । \textbf{न चे}त्यादि । ई‚{\tiny $_{३}$}‚दृश‚मिति व‚र्ण्ण‚संस्थानाकारं गुण‚त्वाद् व‚र्ण्ण‚{\tiny $_{lb}$}‚संस्थानादेः सामान्य‚स्य च गुण‚व्य‚तिरेकात् । \textbf{न च त‚त} इति व‚र्ण्ण‚संस्थानादेः ।
	{\color{gray}{\rmlatinfont\textsuperscript{§~\theparCount}}}
	\pend% ending standard par
      ‚{\tiny $_{lb}$}‚

	  
	  \pstart \leavevmode% starting standard par
	एवं ताव‚द् भिन्नं सामान्यं निराकृत्याभिन्नं निराचिकीर्ष‚न्नाह । \textbf{आकृती}‚{\tiny $_{lb}$}‚त्यादि । स्व‚ल‚क्ष‚णानामात्म‚भूत‚मेव सादृश्य‚माकृतिस्त‚देव सामान्य‚न्त‚स्य वादः‚{\tiny $_{lb}$}‚ स य‚स्यास्ति । त‚द्वा व‚दितुं शीलं य‚स्य सां ख्य स्य \textbf{स त‚था । विशेष‚व‚{\tiny $_{४}$}‚त्} स्व‚ल‚क्ष‚{\tiny $_{lb}$}‚ण‚व‚त् \textbf{त‚स्य} सामान्य‚स्य स्व‚ल‚क्ष‚णा\textbf{द‚व्य‚तिरेका}द्धेतो\textbf{र‚र्था}न्त‚रे द्वितीयादिव्य‚क्तिष्\textbf{व‚{\tiny $_{lb}$}‚वृत्तिः । इति} हेतोस्त‚द‚पि सामान्यं स्व‚ल‚क्ष‚ण‚मेव जात‚न्त‚तो भेदाद्धेतोर्विद्य‚मान‚स्य‚{\tiny $_{lb}$}‚ \textbf{नाभिन्नः प्र‚तिभासो युज्य‚ते} व्य‚क्तिष्वित्य‚ध्याहारः ।
	{\color{gray}{\rmlatinfont\textsuperscript{§~\theparCount}}}
	\pend% ending standard par
      ‚{\tiny $_{lb}$}‚

	  
	  \pstart \leavevmode% starting standard par
	त‚देव‚मु द्यो त क रा द्य‚भिहित‚म‚भिन्न‚प्र‚तिभास‚म‚भ्युप‚ग‚म्य व्य‚तिरिक्त‚स्याव्य‚ति‚{\tiny $_{lb}$}‚रिक्त‚स्य च सामान्य‚स्यायो‚{\tiny $_{५}$}‚गाद् भ्रान्तिरेवायं व्य‚क्तिष्वेकाकार‚प्र‚तिभास‚{\tiny $_{lb}$}‚ इत्युक्तं ।
	{\color{gray}{\rmlatinfont\textsuperscript{§~\theparCount}}}
	\pend% ending standard par
      ‚{\tiny $_{lb}$}‚‚{\tiny $_{lb}$}‚‚{\tiny $_{lb}$}‚\textsuperscript{\textenglish{224/s}}

	  
	  \pstart \leavevmode% starting standard par
	अधुनास्त्येक‚प्र‚तिभासो व्य‚क्तिष्वित्याह । \textbf{अथ‚वास्त्वि}त्यादि । य‚दुक्त‚म्‚{\tiny $_{lb}$}‚ [।] \textbf{अभिन्न‚प्र‚तिभासा धीर्न भिन्ने}ष्वित्य‚त‚द‚स्तु । इष्ट‚मेवैत‚दित्य‚र्थः । य‚तः \textbf{प्र‚ति‚{\tiny $_{lb}$}‚भासो धियां} सामान्य‚बुद्धीनां \textbf{भिन्नः} । किङ्कार‚णं [।] \textbf{स‚माना इति ता}सां‚{\tiny $_{lb}$}‚ व्य‚क्तीनां \textbf{ग्र‚ह‚णात्} ।
	{\color{gray}{\rmlatinfont\textsuperscript{§~\theparCount}}}
	\pend% ending standard par
      ‚{\tiny $_{lb}$}‚

	  
	  \pstart \leavevmode% starting standard par
	\textbf{नैवेत्या}दिना व्याच‚ष्टे । \textbf{तास्वि}ति सामान्य‚बुद्धिषु‚{\tiny $_{६}$}‚ \textbf{अभिन्न} इत्येकः \textbf{प्र‚तिभा}‚{\tiny $_{lb}$}‚सोस्ति [।] किं कार‚णं [।] \textbf{तासां} व्य‚क्तीनां \textbf{स‚माना इति ग्र‚ह‚णात्} ।
	{\color{gray}{\rmlatinfont\textsuperscript{§~\theparCount}}}
	\pend% ending standard par
      ‚{\tiny $_{lb}$}‚

	  
	  \pstart \leavevmode% starting standard par
	न‚नु \textbf{स‚माना} इति ग्र‚हे स‚त्ये\textbf{क‚प्र‚तिभा}सः क‚स्मा\textbf{न्न युज्य}त एवेत्य‚त आह । \textbf{न ही}‚{\tiny $_{lb}$}‚त्यादि । \textbf{किन्त‚र्हि त‚देवे}ति य‚त्पूर्व‚दृष्ट‚न्त‚दे\textbf{वेद‚न्}दृश्य‚त इत्येवं स्यान्न तु पूर्वेणेदं स‚मान‚{\tiny $_{lb}$}‚मिति भेदाधिष्ठान‚त्वात् स‚मान‚व्य‚व‚हार‚स्य । द्व‚य‚स्येत्यादि । \textbf{य‚दि} सामान्य‚मेव‚{\tiny $_{lb}$}‚ \leavevmode\ledsidenote{\textenglish{83a/PSVTa}} केव‚ल‚न्ता‚{\tiny $_{७}$}‚भिर्बुद्धिभिर्गृह्य‚ते त‚दा भ‚वेद‚यं दोषः । किन्तु सामान्यं विशेष‚श्च द्व‚य‚म‚पि‚{\tiny $_{lb}$}‚ सामान्य‚बुद्ध्या गृह्य‚ते । त‚तो द्व‚य‚स्य ग्र‚ह‚णाद‚दोषः । य‚त‚स्तेनैव सामान्येन युक्ता‚{\tiny $_{lb}$}‚ विशेषाः स‚माना इति गृह्य‚न्त इति ।
	{\color{gray}{\rmlatinfont\textsuperscript{§~\theparCount}}}
	\pend% ending standard par
      ‚{\tiny $_{lb}$}‚

	  
	  \pstart \leavevmode% starting standard par
	\textbf{त‚थापी}त्या चा र्यः । द्व‚य‚स्य ग्र‚ह‚णेपि क‚ल्प्य‚माने त\textbf{दिहेति स्यात्} । त‚त् सामान्यं‚{\tiny $_{lb}$}‚ य‚त्पूर्वं व्य‚क्तौ दृष्ट‚न्त‚दिह व्य‚क्त्य‚न्त‚रे दृश्य‚त इत्येवं स्यान्न तु स‚मान इति व्य‚क्तिभ्य‚{\tiny $_{lb}$}‚ ए‚{\tiny $_{१}$}‚कान्त‚भिन्न‚त्वात् सामान्य‚स्य । \textbf{त‚देवेत्य}र्थान्त‚र‚भूतं गोत्वादिकं \textbf{तासां} व्य‚क्तीनां‚{\tiny $_{lb}$}‚ साम्यं येन \textbf{तास्स‚माना इति चेत् । अन्यः} सामान्यात् \textbf{साऽन्य‚स्य} स्व‚ल‚क्ष‚ण‚स्य \textbf{क‚थं}‚{\tiny $_{lb}$}‚ केन प्र‚कारेण \textbf{साम्}यं । न केन‚चिदित्य‚भिप्रायः । त‚था हि व्य‚क्तिरूपानुकारात्‚{\tiny $_{lb}$}‚ सामान्यं व्य‚क्तीनां साम्यं क‚ल्प्येत [।] त‚च्च नास्ति व्य‚क्तिभ्योऽत्य‚न्तंविल‚क्ष‚ण‚{\tiny $_{lb}$}‚त्वात् सामान्य‚स्य । नाप्य‚नेनान्ये स‚माना येन त‚त्साम्यं स्यात्‚{\tiny $_{२}$}‚ । न हि व्य‚क्ति‚{\tiny $_{lb}$}‚रूपानुकारादिना साम्यं किन्तु त‚त्स\textbf{म्ब‚न्धात्} । व्य‚क्तिभिः स‚म्ब‚न्धात् । सामान्यं‚{\tiny $_{lb}$}‚ व्य‚क्तीनां सामान्य\textbf{मिति चेन्नै}त‚देवं । किङ्कार‚ण‚म् [।] अ\textbf{प्र‚तिब‚द्ध}स्य व्य‚क्ताव‚{\tiny $_{lb}$}‚‚{\tiny $_{lb}$}‚ \leavevmode\ledsidenote{\textenglish{225/s}}नाय‚त्त‚स्य ताभिर‚नुप‚कृत‚स्येत्य‚र्थः । स‚म‚वाय‚ल‚क्ष‚णेन स‚म्ब‚न्धेन स‚म्ब‚न्धात् सामान्य‚स्य‚{\tiny $_{lb}$}‚ \textbf{साम्ये}भ्युप‚ग‚म्य‚माने\textbf{ऽतिप्र‚सं}गः संख्यासंयोग‚कार्य‚द्र‚व्याणाम‚पि सामान्य‚रूप‚ता स्यात् ।‚{\tiny $_{३}$}‚‚{\tiny $_{lb}$}‚ त‚त‚श्च त‚त्स‚म्ब‚न्धात् संख्येयादिषु स‚मान‚प्र‚तिभासः स्यात् ।
	{\color{gray}{\rmlatinfont\textsuperscript{§~\theparCount}}}
	\pend% ending standard par
      ‚{\tiny $_{lb}$}‚

	  
	  \pstart \leavevmode% starting standard par
	\textbf{क‚थ‚मि}त्यादि प‚रः । स‚माना इति ग्र‚ह‚णाद् व्य‚क्तिषु प्र‚तिभास‚भेदः साध्य‚ते‚{\tiny $_{lb}$}‚ [।] त‚च्चैत‚द् \textbf{विरुद्धं} । य‚स्मात् \textbf{स‚माश्चेता} व्य‚क्त‚यः क\textbf{थ‚म्भिन्न‚धीग्रा}ह्या एवेति‚{\tiny $_{lb}$}‚ साव‚धार‚णं । अभिन्न‚धीग्राह्या अपि प्राप्नुव‚न्तीत्य‚र्थः । त‚था हि य‚त्र किञ्चित्‚{\tiny $_{lb}$}‚ सामान्यं क‚श्चिच्च विशेष‚स्त‚त्र स‚माना इति ग्र‚ह‚णं युक्त‚म‚न्य‚था‚{\tiny $_{४}$}‚ घ‚ट‚प‚टादिव‚{\tiny $_{lb}$}‚द‚भेद‚प्र‚तिभासः स्यान्न स‚माना इति ।
	{\color{gray}{\rmlatinfont\textsuperscript{§~\theparCount}}}
	\pend% ending standard par
      ‚{\tiny $_{lb}$}‚

	  
	  \pstart \leavevmode% starting standard par
	त‚द्व्याच‚ष्टे । \textbf{न‚न्वि}त्यादि । \textbf{आसु व्य‚क्ति}ष्वित्य‚नेन विशेष‚रूप‚माह । \textbf{अभिन्नः‚{\tiny $_{lb}$}‚ प्र‚तिभा}सः सिद्ध इत्य‚ध्याहार्यः ।
	{\color{gray}{\rmlatinfont\textsuperscript{§~\theparCount}}}
	\pend% ending standard par
      ‚{\tiny $_{lb}$}‚

	  
	  \pstart \leavevmode% starting standard par
	अनेन च सामान्य‚स्य रूप‚मुक्तं ।
	{\color{gray}{\rmlatinfont\textsuperscript{§~\theparCount}}}
	\pend% ending standard par
      ‚{\tiny $_{lb}$}‚

	  
	  \pstart \leavevmode% starting standard par
	त‚देवं साधार‚णाऽसाधार‚ण‚रूप‚ग्र‚ह‚णाद् व्य‚क्त‚यः स‚माना गृह्य‚न्त इति स‚मु‚{\tiny $_{lb}$}‚दायार्थः । त‚त्किं स‚मानेष्वेकानेक‚प्र‚तिभासो विद्य‚ते येनैव‚मुच्य‚ते ।‚{\tiny $_{५}$}‚ य‚दि स्यादि‚{\tiny $_{lb}$}‚हेति बुद्धिः स्यादित्युक्तं । सामान्यात्म‚क‚त्वाद् विशेषाणां \textbf{स‚माना इति प्र‚तिभास‚{\tiny $_{lb}$}‚ इति चेत्} । न‚न्व‚नुग‚त‚प्र‚तिभासाभावे सामान्य‚म‚स्तीति कुतः । न च स‚मान‚रूपान्य‚{\tiny $_{lb}$}‚थानुप‚प‚त्त्या सामान्य‚क‚ल्प‚ना युक्ता । स्व‚हेतुभ्य एवं केषांचित् स‚मानामेवोत्प‚त्तेः‚{\tiny $_{lb}$}‚ केषांचिद‚स‚मानां । न च सामान्यात् तेषां स‚मान‚रूप‚ता युज्य‚त इत्युक्तं ।
	{\color{gray}{\rmlatinfont\textsuperscript{§~\theparCount}}}
	\pend% ending standard par
      ‚{\tiny $_{lb}$}‚

	  
	  \pstart \leavevmode% starting standard par
	तेन य‚दुच्य‚ते भ ट्टे न ॥
	{\color{gray}{\rmlatinfont\textsuperscript{§~\theparCount}}}
	\pend% ending standard par
      ‚{\tiny $_{lb}$}‚
	  \bigskip
	  \begingroup
	
	    
	    \stanza[\smallbreak]
	  {\normalfontlatin\large ``\qquad}न चाप्र‚सिद्ध‚सारूप्यान‚पोह‚विष‚यात्म‚ना ।&‚{\tiny $_{lb}$}‚श‚क्तः क‚श्चिद‚पि ज्ञातुं ग‚वादीन‚विशेष‚तः ॥&‚{\tiny $_{lb}$}‚अथास‚त्य‚पि सारूप्ये स्याद‚पोह‚स्य क‚ल्प‚ना ।&‚{\tiny $_{lb}$}‚ग‚वाश्व‚योर‚यं क‚स्माद‚गोपोहो न क‚ल्प्य‚ते ॥&‚{\tiny $_{lb}$}‚शाव‚लेयाच्च भिन्न‚त्वं बाहुलेयाश्व‚योस्स‚मं ।&‚{\tiny $_{lb}$}‚सामान्यं नान्य‚दिष्टं च क्वागोपोह‚प्र‚व‚र्त्त‚नं ॥&‚{\tiny $_{lb}$}‚अपोह्यान‚पि चाश्वादीनेक‚ध‚र्मान्व‚यादृ‚{\tiny $_{७}$}‚ते ।&‚{\tiny $_{lb}$}‚‚{\tiny $_{lb}$}‚\leavevmode\ledsidenote{\textenglish{226/s}}\leavevmode\ledsidenote{\textenglish{83b/PSVTa}}न निरूप‚यितुं श‚क्य‚स्त‚द‚पोहो न सिध्य‚ती‚{\tiny $_{२}$}‚ति।\edtext{}{\edlabel{pvsvt_226-1}\label{pvsvt_226-1}\lemma{ति}\Bfootnote{\href{http://sarit.indology.info/?cref=\%C5\%9Bv}{ Ślokavārtika. }}}{\normalfontlatin\large\qquad{}"}\&[\smallbreak]
	  
	  
	  
	  \endgroup
	‚{\tiny $_{lb}$}‚

	  
	  \pstart \leavevmode% starting standard par
	अपास्तं । सामान्य‚म‚न्त‚रेणापि स्व‚हेतुभ्य एव ग‚वादीनां स‚मानामुत्प‚त्तेर्य‚दि‚{\tiny $_{lb}$}‚ नाम सारूप्य‚म‚र्थान्त‚र‚भूतं नेष्य‚ते स‚रूपास्त्विष्य‚न्त एव ते । एवंरूपाश्च येन भ‚व‚{\tiny $_{lb}$}‚न्त्य‚श्वाद‚य‚स्ते स‚र्वेऽविशेषेणागोरूप‚त‚या निषिध्य‚न्त इति न काचित् क्ष‚तिः ।
	{\color{gray}{\rmlatinfont\textsuperscript{§~\theparCount}}}
	\pend% ending standard par
      ‚{\tiny $_{lb}$}‚

	  
	  \pstart \leavevmode% starting standard par
	\hphantom{.}आ चा र्य‚स्तु न नाम स्व‚हेतुभ्यः स‚माना उत्प‚न्नास्त‚थापि न सा‚{\tiny $_{१}$}‚मान्य‚ब‚लात्‚{\tiny $_{lb}$}‚ स‚माना इति प्र‚तीतिरि ति द‚र्श‚य‚न्नाह । \textbf{नेत्}यादि । \textbf{त‚द्द‚र्श‚न} इति व्य‚क्तिग्राहिणि‚{\tiny $_{lb}$}‚ ज्ञाने । सामान्य‚ज्ञाने वा । \textbf{भिन्नाभिन्न‚यो}रिति विशेष‚सामान्य‚योः । \textbf{एक‚कार्य‚तासादृश्य}‚{\tiny $_{lb}$}‚मिति । एक‚कार्य‚तैव सादृश्यं साम्यं \textbf{तेनैव} स‚माना व्य‚क्त‚यः \textbf{प्र‚ती}य‚न्त इत्य‚र्थः । न‚{\tiny $_{lb}$}‚ तु पार‚मार्थिकेन सामान्येन । तेनैक‚कार्य‚ता सादृश्यं येषान्त एवापोह‚{\tiny $_{२}$}‚विष‚या येषां‚{\tiny $_{lb}$}‚ त्वेक‚कार्य‚ता नास्ति तेऽपोह्या इति सिद्धं । य‚था चैकान्त‚भिन्ना अप्येक‚कार्य कुर्व‚न्ति‚{\tiny $_{lb}$}‚ त‚थोक्तं प्राक् ।
	{\color{gray}{\rmlatinfont\textsuperscript{§~\theparCount}}}
	\pend% ending standard par
      ‚{\tiny $_{lb}$}‚

	  
	  \pstart \leavevmode% starting standard par
	क‚स्मान्न सामान्येन स‚माना प्र‚तीय‚न्त इत्याह । \textbf{न ही}त्यादि । \textbf{अर्थ‚ज्ञान इत्य}‚{\tiny $_{lb}$}‚नुभ‚व‚ज्ञाने \textbf{द्वावाकारौ} भिन्नौ । \textbf{अर्थ‚द्व‚य‚क‚ल्प‚नेन} स‚मानास‚मान‚क‚ल्प‚नेन । \textbf{क‚ल्प‚ना‚{\tiny $_{lb}$}‚विष‚य‚ता}मित्येक‚त्वारोप‚विक‚ल्प‚विष‚य‚तां । \textbf{त‚थे}ति स‚मान‚{\tiny $_{३}$}‚रूप‚त‚या । \textbf{अन‚या}‚{\tiny $_{lb}$}‚ विक‚ल्प‚बुद्ध्या ।
	{\color{gray}{\rmlatinfont\textsuperscript{§~\theparCount}}}
	\pend% ending standard par
      ‚{\tiny $_{lb}$}‚

	  
	  \pstart \leavevmode% starting standard par
	\textbf{न‚न्वि}त्यादि प‚रः ।
	{\color{gray}{\rmlatinfont\textsuperscript{§~\theparCount}}}
	\pend% ending standard par
      ‚{\tiny $_{lb}$}‚

	  
	  \pstart \leavevmode% starting standard par
	\textbf{तासां} व्य‚क्तीनां \textbf{धीः कार्यं सा च विभिद्य‚त} इत्येतावान् कारिकाभागः ।‚{\tiny $_{lb}$}‚ \textbf{प्र‚तिभाव}मित्येत‚द‚पेक्ष्य प‚ठितः । एव‚म्भावं प्र‚ति । \textbf{त‚द्व‚दि}त्यादि विव‚र‚णं ।‚{\tiny $_{lb}$}‚ \textbf{त‚त्प्र‚तिभासिनोपि} व्य‚क्तिप्र‚तिभासिनोपि \textbf{ज्ञान‚स्य । त‚द्व‚द्व्य}क्ति\textbf{भेदात् । क‚थ‚मेक‚{\tiny $_{lb}$}‚कार्या व्य‚क्त‚यो} नैव । \href{http://sarit.indology.info/?cref=pv.3.107}{। ११० ॥}
	{\color{gray}{\rmlatinfont\textsuperscript{§~\theparCount}}}
	\pend% ending standard par
      ‚{\tiny $_{lb}$}‚

	  
	  \pstart \leavevmode% starting standard par
	स्यादेत‚त् [।] नानुभ‚व‚ज्ञानेनैक‚कार्याः व्य‚{\tiny $_{४}$}‚क्त‚यः किन्तु य‚त्त‚द्विक‚ल्प‚क‚मेक‚{\tiny $_{lb}$}‚‚{\tiny $_{lb}$}‚ ‚{\tiny $_{lb}$}‚ \leavevmode\ledsidenote{\textenglish{227/s}}रूपाध्यारोपेण त‚द‚पेक्ष‚येत्य‚त आह । \textbf{त‚द्धीत्}यादि । त‚द्ध्य‚नुभ‚व‚ज्ञान\textbf{न्तासां} व्य‚क्तीनां‚{\tiny $_{lb}$}‚ \textbf{कार्यं} न विक‚ल्प‚विज्ञान‚न्त‚स्य व्य‚क्त्य‚भावेपि भावात् । त‚स्यैव च विक‚ल्प‚स्य स‚माना‚{\tiny $_{lb}$}‚ इत्येव‚मुत्प‚द्य‚मान‚स्य व्य‚क्तिषु किं साम्यं । प‚टादीनां शीताप‚न‚य‚न‚मित्येव‚मादिप‚रि‚{\tiny $_{lb}$}‚ग्र‚हः । \textbf{प्र‚तिद्र}व्य‚मित्य‚व्य‚यीभावात् ष‚ष्ठ्या अम्भावः ।‚{\tiny $_{६}$}‚ द्र‚व्य‚स्य द्र‚व्य‚स्य यो भेद‚स्त‚{\tiny $_{lb}$}‚स्माद् \textbf{भेदात्} कार‚ण‚भेदाद् \textbf{भिद्य‚त} इति याव‚त् ।
	{\color{gray}{\rmlatinfont\textsuperscript{§~\theparCount}}}
	\pend% ending standard par
      ‚{\tiny $_{lb}$}‚

	  
	  \pstart \leavevmode% starting standard par
	\textbf{नेत्या चा र्यः । एक‚प्र‚त्य‚व‚म}र्श‚स्येति स्व‚विष‚य‚स्यैकाकार‚प्र‚त्य‚य‚स्य \textbf{हेतुत्वाद्}‚{\tiny $_{lb}$}‚ धीर्निर्विक‚ल्पिका स‚विक‚ल्पिका वाऽ\textbf{भेदिनी} भाति । \textbf{एक‚धीहेतुभावेने}त्य‚ध्य‚व‚सितैक‚{\tiny $_{lb}$}‚रूपाया बुद्धेर्हेतुत्वेन \textbf{व्य‚क्तीनाम‚भिन्न‚ता} भाति ।
	{\color{gray}{\rmlatinfont\textsuperscript{§~\theparCount}}}
	\pend% ending standard par
      ‚{\tiny $_{lb}$}‚

	  
	  \pstart \leavevmode% starting standard par
	एत‚दुक्त‚म्भ‚व‚ति । प्र‚त्येकं य‚द्य‚पि व्य‚क्तिस्व‚रूप‚{\tiny $_{७}$}‚ग्राहिण्यो धियो भिन्नास्त- \leavevmode\ledsidenote{\textenglish{84a/PSVTa}}‚{\tiny $_{lb}$}‚ थापि प्र‚त्य‚भिज्ञ‚या तासामेक‚त्व‚म‚ध्य‚व‚सीय‚ते । अनेन चैक‚कार्य‚तासादृश्येन व्य‚क्ती‚{\tiny $_{lb}$}‚नामेक‚त्वं । न त्वेक‚प‚राम‚र्श‚हेतुत्वेनानुभ‚व‚ज्ञानानामेक‚त्व‚मुप‚च‚र्य‚ते । नापि त‚था‚{\tiny $_{lb}$}‚भूतानुभ‚व‚ज्ञान‚हेतुत्वेन व्य‚क्तीनामेक‚त्व‚मुप‚च‚र्य‚ते । स्ख‚लित‚प्र‚त्य‚य‚विष‚य‚त्वा‚{\tiny $_{lb}$}‚भावादुप‚च‚रितोप‚चाराभावाच्च । वृत्त्य‚र्थानुरूप‚श्च कारिकार्थो न व्याख्यातः‚{\tiny $_{lb}$}‚ स्या‚{\tiny $_{१}$}‚त् । \href{http://sarit.indology.info/?cref=pv.3.108}{। १११ ॥}
	{\color{gray}{\rmlatinfont\textsuperscript{§~\theparCount}}}
	\pend% ending standard par
      ‚{\tiny $_{lb}$}‚

	  
	  \pstart \leavevmode% starting standard par
	न‚नु य‚द्येक‚कार्य‚तासादृश्येन व्य‚क्तीनामेक‚त्वाध्य‚व‚सायः क‚थ‚न्त‚र्हि बुद्धीना‚{\tiny $_{lb}$}‚मेक‚त्वाध्य‚व‚सायः । एक‚कार्य‚त्वाभावात् । अथ ताः स्व‚भाव‚त एक‚त्वाव‚सायं ज‚न‚य‚न्ति‚{\tiny $_{lb}$}‚ व्य‚क्त‚योप्येव‚म्भ‚विष्य‚न्तीति किमेक‚कार्य‚तासादृश्येन ।
	{\color{gray}{\rmlatinfont\textsuperscript{§~\theparCount}}}
	\pend% ending standard par
      ‚{\tiny $_{lb}$}‚

	  
	  \pstart \leavevmode% starting standard par
	स‚त्त्य‚म् [।] आचार्य दि ग्ना गा भिप्रायेणैव‚मुक्त‚मित्य‚दोषः । त‚स्मादेक‚त्वा‚{\tiny $_{lb}$}‚ध्य‚व‚सादेक‚त्व‚मिष्य‚ते । न प‚र‚मार्थ‚त इत्य‚त आह‚{\tiny $_{२}$}‚ । \textbf{निवेदि}त‚मित्यादि । \textbf{स‚र्व‚भावाः‚{\tiny $_{lb}$}‚ स्व‚भावेन व्यावृत्तिभागिन} \href{http://sarit.indology.info/?cref=pv.3.39}{प्र० वा० १ । ४२} इत्य‚त्रोक्त‚त्वात् । \textbf{त‚त्रे}त्य‚संस‚र्गिषु‚{\tiny $_{lb}$}‚ भेदेषु । \textbf{एकाकारा बुद्धि}र‚त‚स्मिंस्त‚द्ग्र‚हाद् \textbf{भ्रान्तिरेव} ।
	{\color{gray}{\rmlatinfont\textsuperscript{§~\theparCount}}}
	\pend% ending standard par
      ‚{\tiny $_{lb}$}‚

	  
	  \pstart \leavevmode% starting standard par
	अथ स्यात् [।] सामान्य‚म‚न्त‚रेण भ्रान्त‚रेवायोग इत्य‚त आह । \textbf{तान्त्वि}त्यादि ।‚{\tiny $_{lb}$}‚ ताम्भ्रान्तिम्भेदिनः \textbf{प‚दार्था व्यावृत्तानि} स्व‚ल‚क्ष‚णानि \textbf{क्र‚मेण} ज‚न‚य‚न्ति न साक्षात् ।‚{\tiny $_{lb}$}‚ ‚{\tiny $_{lb}$}‚ \leavevmode\ledsidenote{\textenglish{228/s}}\textbf{स्व‚भाव} त इति प्र‚कृत्या । च‚कारो निवेदित‚मित्य‚स्या‚{\tiny $_{३}$}‚नुक‚र्ष‚णार्थः । \textbf{एत‚द‚पि} त‚त्रैव‚{\tiny $_{lb}$}‚ प्र‚स्तावे \textbf{निवेदितं । क्र‚मेणे}ति य‚दुक्त‚न्त‚स्य व्याख्यानं \textbf{विक‚ल्प‚हेत‚वो भ‚व‚न्त} इति ।‚{\tiny $_{lb}$}‚ विक‚ल्प‚कार‚ण‚त्वाद‚नुभ‚व‚ज्ञान‚म्विक‚ल्पः । विक‚ल्प‚हेतोर‚नुभ‚व‚ज्ञान‚स्य हेत‚वो भ‚व‚न्त‚{\tiny $_{lb}$}‚ इत्य‚र्थः । व्य‚क्त‚योनुभ‚व‚ज्ञानं ज‚न‚य‚न्ति त‚च्चैकाकारां भ्रान्तिमित्य‚यं क्र‚मार्थः ।
	{\color{gray}{\rmlatinfont\textsuperscript{§~\theparCount}}}
	\pend% ending standard par
      ‚{\tiny $_{lb}$}‚

	  
	  \pstart \leavevmode% starting standard par
	\textbf{स त्वेषा}मित्यादिना का रि का र्थ‚माह । स‚र्वेषाम्भावानाम्भेदोन्या‚{\tiny $_{४}$}‚पोहः । किं‚{\tiny $_{lb}$}‚ स्व‚भावो\textbf{ऽत‚त्कारिविवेकः} स एषा\textbf{म‚भिन्न} इत्युच्य‚ते । क‚स्माद् [।] \textbf{ज्ञानादे}र‚र्थ‚{\tiny $_{lb}$}‚स्येन्द्रिय‚स्योद‚काह‚र‚णादेश्च \textbf{क‚स्य‚चि}दित्यात्मानुरूप‚स्\textbf{यैक‚स्य क‚र‚णात्} ।
	{\color{gray}{\rmlatinfont\textsuperscript{§~\theparCount}}}
	\pend% ending standard par
      ‚{\tiny $_{lb}$}‚

	  
	  \pstart \leavevmode% starting standard par
	य‚द्वा [।] न‚नु बुद्धेरेवाय‚म‚भिन्नाकारः क‚थं व्य‚क्तीनामित्य‚त आह । \textbf{स}‚{\tiny $_{lb}$}‚ त्वाकार \textbf{एषां} बाह्यानाम\textbf{भिन्न}स्त‚थैव \textbf{प्र‚तीतेः} [।] प्र‚तीतिरेव कुतः । ज्ञानादेः‚{\tiny $_{lb}$}‚ क‚स्य‚चिदेक‚स्य क‚र‚णात् [।] स चाका‚{\tiny $_{५}$}‚रो भेदोन्यापोह इत्युच्य‚ते न सामान्यं ।‚{\tiny $_{lb}$}‚ किङ्कार‚ण‚म् [।] \textbf{अत‚त्कारिस्व‚भाव‚विवेको} य‚तः ।
	{\color{gray}{\rmlatinfont\textsuperscript{§~\theparCount}}}
	\pend% ending standard par
      ‚{\tiny $_{lb}$}‚

	  
	  \pstart \leavevmode% starting standard par
	एत‚दुक्त‚म्भ‚व‚ति । प्र‚तिव्य‚क्तिर्गौर्गौरिति प्र‚त्य‚येनात‚त्कारिस्व‚भाव‚विवेको‚{\tiny $_{lb}$}‚त‚त्कारिस्व‚भाव‚विविक्त एव स्व‚भावो विष‚यीक्रिय‚ते । न त्व‚र्थान्त‚र‚भूतं सामान्य‚न्तेन‚{\tiny $_{lb}$}‚ भेद इत्युच्य‚ते । एत‚त्प‚श्चार्द्ध‚स्य व्याख्यानं ॥ \href{http://sarit.indology.info/?cref=pv.3.108}{१११ ॥}
	{\color{gray}{\rmlatinfont\textsuperscript{§~\theparCount}}}
	\pend% ending standard par
      ‚{\tiny $_{lb}$}‚

	  
	  \pstart \leavevmode% starting standard par
	न‚नु क‚थं ज्ञानादेरेक‚त्व‚मित्याह । \textbf{त‚द‚पी}त्यादि ।‚{\tiny $_{६}$}‚ पूर्वार्द्ध‚स्यैत‚द् व्याख्यानं ।‚{\tiny $_{lb}$}‚ \textbf{त‚द‚पि} ज्ञानादिकार्य\textbf{म‚भिन्नं ख्यातीति} स‚म्ब‚न्धः । \textbf{प्र‚कृत्ये}ति स्व‚भावेन ।‚{\tiny $_{lb}$}‚ \textbf{अभेदाव‚स्क‚न्दिन} इत्य‚भेदाध्य‚व‚सायिनः । \textbf{त‚थाभूतेत्}यादिना व्य‚क्तीनाम‚पि‚{\tiny $_{lb}$}‚ विक‚ल्पं प्र‚ति पार‚म्प‚र्येण कार‚ण‚त्व‚माह । \textbf{अभेदाव‚सायो} विद्य‚ते य‚स्मिंस्त‚स्य ज्ञानादे‚{\tiny $_{lb}$}‚\leavevmode\ledsidenote{\textenglish{84b/PSVTa}} रित्य‚नुभ‚व‚ज्ञान‚स्योद‚काद्याह‚र‚णादेश्चार्थ‚स्य \textbf{हेतुत्वात्} कार‚णा‚{\tiny $_{७}$}‚त् क्र‚मेण व्य‚क्त‚{\tiny $_{lb}$}‚योप्येकं प्र‚त्य‚यं ज‚न‚य‚न्तीति स‚म्ब‚न्धः । किम्विशिष्ट‚मित्याह । \textbf{संसृष्टाकार‚मि}त्या‚{\tiny $_{lb}$}‚दि । संसृष्टो व्य‚क्तिष्वारोपित एक आकारो येन स त‚था । \textbf{स्व‚भाव‚भेदो}न्य‚{\tiny $_{lb}$}‚व्यावृत्तं रूपं स एव \textbf{प‚र‚मा}र्थोनु\textbf{प‚च‚रितो}ऽस्येति विग्र‚हः । पार‚म्प‚र्येण व्यावृत्त‚स्व‚ल‚क्ष‚ण‚{\tiny $_{lb}$}‚द्वारायात‚त्वाद् विक‚ल्प‚स्य ।
	{\color{gray}{\rmlatinfont\textsuperscript{§~\theparCount}}}
	\pend% ending standard par
      ‚{\tiny $_{lb}$}‚

	  
	  \pstart \leavevmode% starting standard par
	न‚नु च तान्तु भेदिनः प‚दार्था इत्यादिनाऽय‚म‚{\tiny $_{१}$}‚र्थोऽन‚न्त‚र‚मेवोक्तः ।
	{\color{gray}{\rmlatinfont\textsuperscript{§~\theparCount}}}
	\pend% ending standard par
      ‚{\tiny $_{lb}$}‚‚{\tiny $_{lb}$}‚\textsuperscript{\textenglish{229/s}}

	  
	  \pstart \leavevmode% starting standard par
	स‚त्त्यं [।] किन्तु क्र‚मेण विक‚ल्प‚हेत‚वो भ‚व‚न्त इत्याद्य‚स्यैवार्थोऽनेन स्फुटीकृतः ।‚{\tiny $_{lb}$}‚ एक‚कार्य‚भेद‚व‚स्तुभूतं सामान्य‚म्भ‚विष्य‚तीत्य‚त आह । \textbf{सा चे}त्यादि । सा चैक‚कार्य‚ता ।‚{\tiny $_{lb}$}‚ \textbf{अत‚त्का}र्येभ्यो \textbf{विश्लेषो} व्यावृत्त एव स्व‚भावो न व‚स्तुभूतं सामान्यं । किं कार‚ण‚न्त‚{\tiny $_{lb}$}‚\textbf{द‚न्य‚स्य} स्व‚ल‚क्ष‚णाद‚न्य‚स्या\textbf{नुव‚र्त्तिनो}न्व‚यिनो व‚स्तुनो व्य‚क्तिव्य‚तिरेकेणा‚{\tiny $_{२}$}‚दृष्टेः‚{\tiny $_{lb}$}‚ \textbf{प्र‚तिषेधाच्च} पूर्वोक्तात् ।
	{\color{gray}{\rmlatinfont\textsuperscript{§~\theparCount}}}
	\pend% ending standard par
      ‚{\tiny $_{lb}$}‚

	  
	  \pstart \leavevmode% starting standard par
	\textbf{न ही}त्यादिना व्याच‚ष्टे । \textbf{दृश्य‚मु}प‚ल‚ब्धिल‚क्ष‚ण‚प्राप्तं \textbf{विभागेन} व्य‚क्तिभ्यो‚{\tiny $_{lb}$}‚ भेदेन । \textbf{स‚ति वा} सामान्ये नित्य‚त्वाद‚नाधेयातिष\edtext{}{\lemma{नाधेयातिष}\Bfootnote{? श}}य‚त्वेन \textbf{क्व‚चिद्} व्य‚क्त्य‚न्त‚रे‚{\tiny $_{lb}$}‚ \textbf{अनाश्रितं क‚थं ज्ञान‚हेतुः} । नैव । आश्र‚य‚व्यंग्य‚स्य त‚स्य \textbf{ज्ञान‚हेतु}त्व‚मिष्ट‚मित्य‚{\tiny $_{lb}$}‚भिप्रायः । एत‚द‚प्युक्त‚मित्य‚नेन स‚म्ब‚न्ध‚नीयं । अस्यापि प्रागुक्त‚त्वात् ।
	{\color{gray}{\rmlatinfont\textsuperscript{§~\theparCount}}}
	\pend% ending standard par
      ‚{\tiny $_{lb}$}‚

	  
	  \pstart \leavevmode% starting standard par
	य‚त‚{\tiny $_{३}$}‚ एव‚म्भूतं सामान्यं नास्ति । त‚स्मात् \textbf{संकेतोपी}त्यादि । \textbf{त‚दि}त्य‚न्यापोहः‚{\tiny $_{lb}$}‚ स‚म्ब‚ध्य‚ते । त‚स्यास्यापोह‚स्य विक‚ल्पेन स्वाकाराभेदेनाध्य‚स्त‚स्य \textbf{वित्} ज्ञानं ।‚{\tiny $_{lb}$}‚ त‚त्पूर्व‚को लाभ‚श्च । त‚द्वित् । सैवार्थः फ‚ल‚मिति विशेष‚ण‚स‚मासः । स म‚य‚{\tiny $_{lb}$}‚स्यास्ति संकेत‚स्येति म‚त्व‚र्थीय‚ष्ठ‚न् । विक‚ल्पाध्य‚व‚सित‚बाह्यार्थ‚प्र‚तिप‚त्त्य‚र्थ‚मिति‚{\tiny $_{lb}$}‚ स‚मुदायार्थः । स‚र्व‚श्चार‚म्भः फ‚लार्थ इ‚{\tiny $_{४}$}‚ति विदिर्लाभार्थोप्याक्षिप्त एवान्य‚था‚{\tiny $_{lb}$}‚ संकेत‚क‚र‚ण‚स्य वैय‚र्थ्यात् ।
	{\color{gray}{\rmlatinfont\textsuperscript{§~\theparCount}}}
	\pend% ending standard par
      ‚{\tiny $_{lb}$}‚

	  
	  \pstart \leavevmode% starting standard par
	न‚नु त‚द्विद‚र्थो य‚स्येति ब‚हुब्रीहिणा भ‚वित‚व्यं लाघ‚वात् । ब‚हुब्रीहिणोक्त‚त्वान्म‚{\tiny $_{lb}$}‚त्व‚र्थ‚स्य त‚द्विभागानुत्प‚त्तिर्लाघ‚व‚त्वं । त‚था च भाष्य\edtext{}{\lemma{भाष्य}\Bfootnote{\href{http://sarit.indology.info/?cref=vk-mbh}{ Vyākaraṇa-mahābhāṣya. }}} उक्तं क‚र्म‚धार‚याद्‚{\tiny $_{lb}$}‚ ब‚हुव्रीहिर्भ‚व‚तीत्यादि ।
	{\color{gray}{\rmlatinfont\textsuperscript{§~\theparCount}}}
	\pend% ending standard par
      ‚{\tiny $_{lb}$}‚

	  
	  \pstart \leavevmode% starting standard par
	नैष दोषः । इद‚म‚पि त‚त्रोक्तं\edtext{}{\edlabel{pvsvt_229-2}\label{pvsvt_229-2}\lemma{त्रोक्तं}\Bfootnote{\begin{english}\href{http://sarit.indology.info/?cref=vk-mbh}{Ibid.}\end{english}}} क्व‚चित्क‚र्म‚धार‚य एव स‚र्व‚साध‚नाद्य‚र्थ इति ।‚{\tiny $_{lb}$}‚ आकृतिग‚ण‚त्वाच्च स‚{\tiny $_{५}$}‚र्व‚साध‚नादेस्त‚त्राय‚न्त‚द्विद‚र्थिक‚श‚ब्दो द्र‚ष्ट‚व्यः । \href{http://sarit.indology.info/?cref=pv.3.109}{। ११२ ॥}
	{\color{gray}{\rmlatinfont\textsuperscript{§~\theparCount}}}
	\pend% ending standard par
      ‚{\tiny $_{lb}$}‚\textsuperscript{\textenglish{230/s}}

	  
	  \pstart \leavevmode% starting standard par
	\textbf{योय‚मि}त्यादि विव‚र‚णं । \textbf{अन्योन्यं विवेको}न्य‚व्यावृत्तः स्व‚भावो \textbf{भावानान्त‚{\tiny $_{lb}$}‚ त्प्र‚तीत}ये त‚न्निश्च‚यार्थं स्व‚प्र‚तिभासेऽध्य‚व‚सित‚बाह्य‚रूपे \textbf{संकेतोपि क्रिय‚माणः शोभेत}‚{\tiny $_{lb}$}‚ युक्तियुक्त‚त्वात् [।] किम‚र्थं क्रिय‚त इत्य‚त आह । \textbf{अत‚त्कारी}त्यादि । विव‚क्षितार्थ‚{\tiny $_{lb}$}‚क्रियाकारिणो ये न भ‚व‚न्ति तेषां \textbf{विवेके}‚{\tiny $_{६}$}‚न प‚रिहारेण \textbf{प्र‚वृत्त्य‚र्थ‚त‚या} । अत‚त्कारिषु‚{\tiny $_{lb}$}‚प्र‚वृत्तिर्माभूदित्य‚र्थः ।
	{\color{gray}{\rmlatinfont\textsuperscript{§~\theparCount}}}
	\pend% ending standard par
      ‚{\tiny $_{lb}$}‚

	  
	  \pstart \leavevmode% starting standard par
	अमुमेवार्थं व्य‚तिरेक‚मुखे\textbf{ण}\edtext{}{\lemma{मुखे}\Bfootnote{? न}} द्र‚ढ‚य‚न्नाह । \textbf{य‚दीत्}यादि । त‚स्यान्य‚{\tiny $_{lb}$}‚प‚रिहार‚स्य \textbf{प्र‚तीत्य‚र्थो} य‚दि \textbf{न संके}त‚स्त‚त्कारिणी त‚द‚न्य‚प‚रिहार‚स्य व्य‚व‚हार‚काले‚{\tiny $_{lb}$}‚प्य‚संस्प‚र्शात् । य‚द्वा य‚दि न त‚त्प्र‚तीत्य‚र्थः संकेत इति विजातीय‚व्यावृत्त‚स्व‚भाव‚{\tiny $_{lb}$}‚\leavevmode\ledsidenote{\textenglish{85a/PSVTa}} प्र‚तीत्य‚र्थः संकेतः [।] \textbf{त‚दा} त‚स्यान्य‚{\tiny $_{७}$}‚व्यावृत्त‚स्य स्व‚भाव‚स्य \textbf{व्य‚व‚हार‚कालेपि} न केव‚लं‚{\tiny $_{lb}$}‚ संकेत‚कालेऽ\textbf{संस्प‚र्शा}च्छ‚ब्देनाविष‚यीक‚र‚णा\textbf{न्नान्य‚प‚रिहारेण प्र‚व‚र्त्तेत} ।
	{\color{gray}{\rmlatinfont\textsuperscript{§~\theparCount}}}
	\pend% ending standard par
      ‚{\tiny $_{lb}$}‚

	  
	  \pstart \leavevmode% starting standard par
	एत‚दुक्त‚म्भ‚व‚ति । य‚दा विधिरूपेणान्य‚व्यावृतोर्थो विष‚यीकृत‚स्त‚दान्य‚व्य‚व‚{\tiny $_{lb}$}‚च्छेदः प्र‚तीयेत । एत‚देवाह । \textbf{न ही}त्यादि । \textbf{विवेक} इति विविक्तः स्व‚भावः ।‚{\tiny $_{lb}$}‚ \textbf{तेषान्त}त्कारिणा\textbf{न्तेभ्य} इत्य‚त‚त्कार्येभ्यः । य‚दि हि त‚स्य विविक्त‚{\tiny $_{१}$}‚स्य स्व‚भाव‚स्य‚{\tiny $_{lb}$}‚ प्र‚तीत‚ये संकेतः कृतः स्यादेवं व्य‚व‚हारेपि \textbf{श‚ब्देन चोद्येत} [।] त‚था चान्य‚प‚रिहारेण‚{\tiny $_{lb}$}‚ प्र‚व‚र्त्तेतेति संकेतोपि त‚द्विद‚र्थिक एव युक्तः ।
	{\color{gray}{\rmlatinfont\textsuperscript{§~\theparCount}}}
	\pend% ending standard par
      ‚{\tiny $_{lb}$}‚

	  
	  \pstart \leavevmode% starting standard par
	न‚नु च श‚ब्द‚ज‚निता बुद्धिः स्वाकार‚मेव बाह्य‚त‚याध्य‚स्य ग्र‚ह‚णाद‚लीका ।‚{\tiny $_{lb}$}‚ त‚त‚श्च त‚ज्ज‚न‚क‚स्य श‚ब्द‚स्य क‚थ‚म्व‚स्तुस‚म्वादः क‚थं चान्यापोह‚विष‚य‚त्व‚मित्य‚त‚{\tiny $_{lb}$}‚ आह ।
	{\color{gray}{\rmlatinfont\textsuperscript{§~\theparCount}}}
	\pend% ending standard par
      ‚{\tiny $_{lb}$}‚

	  
	  \pstart \leavevmode% starting standard par
	\textbf{सा चेत्या}दि । अन्यापोह‚प्र‚ति‚{\tiny $_{२}$}‚प‚त्त्य‚र्थं या संकेतिका । \textbf{सा च श्रुतिः । धियं‚{\tiny $_{lb}$}‚ ज‚न‚य‚न्त्}य‚पीति स‚म्ब‚न्धः । किम्विशिष्टाम् [।] \textbf{अकार्य‚कृति} । अवाह्य‚रूपे स्वाकारे ।‚{\tiny $_{lb}$}‚ \textbf{त‚त्कारि तुल्य‚रूपे}णार्थ‚क्रियाकारि बाह्यैक‚रूपे\textbf{णाव‚भा}सोध्य‚व‚साय इति तृतीया‚{\tiny $_{lb}$}‚स‚मासः स य‚स्या विद्य‚त इति प‚श्चात् म‚त्व‚र्थीयः ।
	{\color{gray}{\rmlatinfont\textsuperscript{§~\theparCount}}}
	\pend% ending standard par
      ‚{\tiny $_{lb}$}‚‚{\tiny $_{lb}$}‚\textsuperscript{\textenglish{231/s}}

	  
	  \pstart \leavevmode% starting standard par
	एत‚दुक्त‚म्भ‚व‚ति । स्वाकार‚म‚बाह्यं बाह्य‚मिवाध्य‚व‚स्य‚न्तीमिति याव‚त् ।
	{\color{gray}{\rmlatinfont\textsuperscript{§~\theparCount}}}
	\pend% ending standard par
      ‚{\tiny $_{lb}$}‚

	  
	  \pstart \leavevmode% starting standard par
	एतेन प्र‚वृ‚{\tiny $_{३}$}‚त्त्यर्थत्वं श्रुतेराख्यातं ।
	{\color{gray}{\rmlatinfont\textsuperscript{§~\theparCount}}}
	\pend% ending standard par
      ‚{\tiny $_{lb}$}‚

	  
	  \pstart \leavevmode% starting standard par
	व‚स्तुभूत‚सामान्य‚म‚न्त‚रेण कुत‚स्त‚स्या उत्प‚त्तिरिति चेदाह । \textbf{व‚स्त्वित्या}दि ।‚{\tiny $_{lb}$}‚ \textbf{व‚स्तूनाम्पृथ‚ग्भाव} इत‚रेत‚र‚भेद‚स्त\textbf{न्मात्रं बीज}ङ्कार‚णं पार‚म्प‚र्येण य‚स्याः सा त‚थो‚{\tiny $_{lb}$}‚क्ता । य‚त‚श्चानुग‚तं रूपं व्यावृत्तं चैकीकृत्य गृह्णात्य‚तोन\textbf{र्थिकां} ज‚न‚य‚न्त्य‚पि‚{\tiny $_{lb}$}‚ श्रुतिर‚र्थे न विस‚म्वादिका । क‚स्माद् [।] \textbf{अत‚त्कारिप‚रिहाराङ्ग‚भाव‚तः} । विजा‚{\tiny $_{lb}$}‚ती‚{\tiny $_{४}$}‚याव्य‚व‚च्छेद‚हेतुभाव‚तः ।
	{\color{gray}{\rmlatinfont\textsuperscript{§~\theparCount}}}
	\pend% ending standard par
      ‚{\tiny $_{lb}$}‚

	  
	  \pstart \leavevmode% starting standard par
	एत‚दुक्त‚म्भ‚व‚ति । य‚द्य‚न्य‚व्यावृत्त‚व‚स्त्व‚ध्य‚व‚सायिनीं बुद्धिं ज‚न‚येच्छ्रुतिस्त‚दा‚{\tiny $_{lb}$}‚ त‚द‚न्य‚व्यावृत्त एव स्व‚ल‚क्ष‚णे पुरुषं प्र‚व‚र्त्त‚य‚तीति स‚म्वादिका स्यात् । संकेत‚काले च‚{\tiny $_{lb}$}‚ श्रुतेरित‚रेत‚र‚भिन्न एव स्व‚भाव आश्र‚य‚स्त‚त्रास्याः संकेतित‚त्वात् । त‚देव‚म्पार‚म्प‚{\tiny $_{lb}$}‚र्येण \textbf{व‚स्तुभेदाश्र‚याच्च} कार‚णा\textbf{द‚र्थे । न विस‚म्वादिका म‚ता} । व्य‚व‚हा‚{\tiny $_{५}$}‚र‚काले‚{\tiny $_{lb}$}‚प्य‚न्य‚व्यावृत्त‚स्यैव व‚स्तुनः प्राप‚णात् ।
	{\color{gray}{\rmlatinfont\textsuperscript{§~\theparCount}}}
	\pend% ending standard par
      ‚{\tiny $_{lb}$}‚

	  
	  \pstart \leavevmode% starting standard par
	न‚नु विधिरूपेण व‚स्त्व‚ध्य‚व‚सायात्क‚थं श्रुतेर‚न्यापोह‚विष‚य‚त्व‚मित्य‚त आह ।‚{\tiny $_{lb}$}‚ \textbf{त‚त} इत्यादि । य‚त‚श्चा\textbf{त‚त्कारिप‚रिहारांग‚भाव}तः श्रुतेर्व‚स्तुभेदाश्र‚य‚त्वं च \textbf{त‚तः}‚{\tiny $_{lb}$}‚ कार‚णाद\textbf{न्यापोह‚विष‚या} । एत‚देव द्व‚य‚माह । \textbf{त‚त्क‚र्त्राश्रित‚भाव‚त} इति । त‚स्मिन्न‚पोहे‚{\tiny $_{lb}$}‚ क‚र्त्तृभाव‚तः । आश्रित‚भाव‚श्च‚{\tiny $_{६}$}‚ स्वार्थाभिधान‚द्वारेणार्थाद‚त‚त्कारिप‚रिहाराङ्ग‚भाव‚{\tiny $_{lb}$}‚त‚स्त‚स्मिन्न‚पोहे क‚र्त्तृभावः श्रुतेः । व्य‚व‚हार‚काले संकेत‚काले च व‚स्तुभेदाश्र‚य‚द्वारेण‚{\tiny $_{lb}$}‚ प्र‚वृत्तेस्त‚स्मिन्न‚पोहे श्रुतेराश्रित‚भावः ।
	{\color{gray}{\rmlatinfont\textsuperscript{§~\theparCount}}}
	\pend% ending standard par
      ‚{\tiny $_{lb}$}‚

	  
	  \pstart \leavevmode% starting standard par
	\textbf{एके}त्यादिना कारिकार्थ‚माह । \textbf{त‚मित्येक‚माकारं} स्व‚प्र‚तिभासिन\textbf{मारोप्यार्थे}‚{\tiny $_{lb}$}‚ष्व‚ध्य‚स्यो\textbf{त्प‚द्य‚मानां} । स च स्व‚प्र‚तिभासो‚{\tiny $_{७}$}‚ \textbf{मिथ्याव‚भासित्वाद‚कार्य‚कारी} । त‚मे- \leavevmode\ledsidenote{\textenglish{85b/PSVTa}}‚{\tiny $_{lb}$}‚ वंभूत\textbf{म‚पि कार्य‚कारिण‚मिवाध्य‚व‚स्य‚न्ती} । अनेना\textbf{कार्य‚कृती}त्यादि व्याख्यातं ।‚{\tiny $_{lb}$}‚ \textbf{व‚स्तुपृथ‚ग्भाव‚मात्र}म‚न्य‚व्य‚व‚च्छेद‚मात्रं य‚स्या बुद्धेरिति विग्र‚हः । \textbf{मिथ्या}बुद्धिं \textbf{ज‚न‚{\tiny $_{lb}$}‚य‚न्त्य‚पि} श्रुतिः [।] क‚स्मान्मिथ्याबुद्धिरित्य‚त आह । \textbf{स‚मानाध्य‚व‚साया}मेकाकारा‚{\tiny $_{lb}$}‚‚{\tiny $_{lb}$}‚ \leavevmode\ledsidenote{\textenglish{232/s}}ध्य‚व‚सायां । \textbf{त‚द‚न्य‚प‚रिहाराङ्ग‚भावात्} । विव‚क्षिताद‚र्था‚{\tiny $_{१}$}‚द‚न्य‚स्य प‚रिहारांग‚भाव‚त‚या‚{\tiny $_{lb}$}‚ व्य‚व‚च्छेदाश्र‚य‚भाव‚तः । \textbf{प‚र‚मार्थ‚तो} व‚स्तुतः । \textbf{त‚द्व्य‚तिरेकिष्व}त‚त्कार्य‚व्य‚तिरेकिषु‚{\tiny $_{lb}$}‚ \textbf{न विस‚म्वादिके}त्युच्य‚ते । त‚था ह्य‚नित्य‚कृत‚कादिश्रुत‚यो य‚थाभूत‚स्य नित्यादि‚{\tiny $_{lb}$}‚व्यावृत्त‚स्य व‚स्तुनो व्यावृत्तिमुपादाय संकेतित‚त्वाद् व्य‚व‚हारेपि त‚था भूत‚स्या‚{\tiny $_{lb}$}‚न्य‚प‚रिहाराङ्ग‚भावेन प्राप्तिहेत‚वो भ‚व‚न्ति । तेन य‚थार्थ‚द‚र्श‚ना‚{\tiny $_{२}$}‚न्यायात‚त्वा‚{\tiny $_{lb}$}‚च्छ्रुतेर‚पि स‚म्वादः ।
	{\color{gray}{\rmlatinfont\textsuperscript{§~\theparCount}}}
	\pend% ending standard par
      ‚{\tiny $_{lb}$}‚

	  
	  \pstart \leavevmode% starting standard par
	य‚द्वा य‚थाभूते व्य‚व‚च्छेदे सा श्रुतिः संकेतिता त‚स्य प‚र‚मार्थ‚तोस्ति व‚स्तुषु‚{\tiny $_{lb}$}‚ स‚द्भाव इतीय‚ता लेशेनाविस‚म्वादित्वं न तु धूमादिव‚च्छ‚ब्दानामाव‚श्य‚को व‚स्तुनि‚{\tiny $_{lb}$}‚ प्र‚तिब‚न्ध‚स्तेषामिच्छामात्र‚प्र‚तिब‚द्ध‚त्वात् ।
	{\color{gray}{\rmlatinfont\textsuperscript{§~\theparCount}}}
	\pend% ending standard par
      ‚{\tiny $_{lb}$}‚

	  
	  \pstart \leavevmode% starting standard par
	अविस‚म्वादित्व‚मेव स‚म‚र्थ‚य‚न्नाह । \textbf{त‚था ही}त्यादि । \textbf{तेषु} भावेषु \textbf{स व्य‚तिरेको}‚{\tiny $_{lb}$}‚ व्य‚व‚च्छेदो \textbf{भूतः} स‚त्यः‚{\tiny $_{३}$}‚ [।] कुतः [।] \textbf{स‚र्व‚था} सामान्याभ्युप‚ग‚मेप्य‚व‚श्यं व्य‚व‚च्छे‚{\tiny $_{lb}$}‚द‚स्या\textbf{भ्युप‚ग‚म‚नीय‚त्वा}द‚न्य‚था व्य‚व‚च्छिन्न‚स्व‚भावाभावे सामान्य‚स्यैवाभाव‚प्र‚स‚ङ्गा‚{\tiny $_{lb}$}‚दित्युक्तं । सामान्योक्तापि त‚त्र भूत इति चेदाह । \textbf{नैक} इत्यादि । \textbf{एकः} सामा‚{\tiny $_{lb}$}‚न्य‚प‚दार्थो व्य‚क्तेर्व्य‚तिरिक्तो वै शे षि का दीनाम‚व्य‚तिरिक्तः सां ख्या नां न तेषु भूत‚{\tiny $_{lb}$}‚ इति स‚म्ब‚न्धः । क‚स्मात् [।] \textbf{स‚र्व‚{\tiny $_{४}$}‚था व्य‚तिरिक्त‚स्याव्य‚तिरिक्त‚स्य} च प्र‚माण‚{\tiny $_{lb}$}‚बाधित‚त्वेना\textbf{योगाद}स‚म्भ‚वात् । बाध‚क‚म्प्र‚माणं प्रागुक्तं व‚क्ष्य‚ते च ।
	{\color{gray}{\rmlatinfont\textsuperscript{§~\theparCount}}}
	\pend% ending standard par
      ‚{\tiny $_{lb}$}‚

	  
	  \pstart \leavevmode% starting standard par
	य‚दि पुन‚र्य‚था प्र‚तिभास‚म‚पि सामान्यं श‚ब्देन चोद्येत त‚दा त‚स्य सामान्य‚स्य‚{\tiny $_{lb}$}‚ व‚स्तुन्य‚विद्य‚मान‚स्य स‚मावेश‚ने श‚ब्देन विष‚यीक‚र‚णे । \textbf{त‚स्य} वा सामान्य‚स्य प्र‚माण‚{\tiny $_{lb}$}‚बाधित‚स्य \textbf{व‚स्तुनि} बाह्ये \textbf{निवेश‚ने}ऽभ्युप‚ग‚म्य‚माने‚{\tiny $_{५}$}‚ \textbf{दूरोत्सृष्ट‚मेवा}त्य‚न्त‚विप्र‚कृष्ट‚{\tiny $_{lb}$}‚मेव \textbf{व‚स्तु स्यात्} । कुतः [।] \textbf{श‚ब्द‚ज्ञानाभ्यां} श‚ब्दात् त‚दुक्ताच्च ज्ञानादित्य‚र्थः । य‚द्वा‚{\tiny $_{lb}$}‚ श‚ब्द‚ज्ञानाभ्यां दूर‚मुत्सृष्टं त्य‚क्तं स्यात् । कुतः [।] \textbf{त‚द्विष‚याभिम‚त‚स्य} श‚ब्दादि‚{\tiny $_{lb}$}‚विष‚याभिम‚त‚स्य । \textbf{त‚स्ये}ति सामान्य‚स्य व‚स्तुस्व‚भावात् । \textbf{अन्य‚स्य} सामान्य‚व्य‚तिरिक्त‚{\tiny $_{lb}$}‚स्य \textbf{व‚स्तुध‚र्म}स्य व‚स्तुस्व‚भाव‚स्य स्व‚ल‚क्ष‚ण‚स्या\textbf{संस्प‚{\tiny $_{६}$}‚र्शा}द‚ग्र‚ह‚णात् । य‚त‚श्च स्व‚ल‚क्ष‚ण‚न्न‚{\tiny $_{lb}$}‚ गृह्णात्य‚थ च स्वाकाराभिन्न‚म‚ध्य‚व‚स्य‚ति । \textbf{त‚त एव सा श्रुतिर‚न्यापोह‚विष‚येत्युच्य‚ते} ।
	{\color{gray}{\rmlatinfont\textsuperscript{§~\theparCount}}}
	\pend% ending standard par
      ‚{\tiny $_{lb}$}‚‚{\tiny $_{lb}$}‚\textsuperscript{\textenglish{233/s}}

	  
	  \pstart \leavevmode% starting standard par
	न‚नु विधिरूपेण बाह्य‚स्यैवाध्य‚व‚सायात् क‚थ‚म‚न्यापोह‚विष‚येत्युच्य‚त इत्याह ।‚{\tiny $_{lb}$}‚ \textbf{अन्ये}त्यादि । विजातीय\textbf{व्यावृत्तेष्व‚र्थेषु व्यावृत्तिभे}द‚म्विजातीय‚व्य‚च्छिन्न‚स्व‚भाव‚{\tiny $_{lb}$}‚\textbf{म‚विशेषे‚{\tiny $_{७}$}‚णोपादाय} विजातीय‚व्यावृत्त‚मात्रं रूप‚माश्रित्य स‚जातीय‚व्य‚क्तिषु श‚ब्द‚{\tiny $_{lb}$}‚स्य \textbf{निवेश‚नात्} संकेत‚क‚र‚णादित्य‚र्थः । अनेनान्यापोहाश्रित‚त्वं श्रुतेराख्यातं । \textbf{अन्य- \leavevmode\ledsidenote{\textenglish{86a/PSVTa}}‚{\tiny $_{lb}$}‚ प‚रिहारेण प्र‚व‚र्त्त‚ना}दित्य‚न्यापोहं प्र‚ति क‚र्त्तृभावः श्रुतेरुक्तः । \href{http://sarit.indology.info/?cref=pv.3.112}{। ११५ ॥}
	{\color{gray}{\rmlatinfont\textsuperscript{§~\theparCount}}}
	\pend% ending standard par
      ‚{\tiny $_{lb}$}‚

	  
	  \pstart \leavevmode% starting standard par
	\textbf{अवृक्षेत्या}दिना प‚र‚स्य चोद्य‚माशंक‚ते [।] अन्यापोह‚वादिनः किल न विधिरू‚{\tiny $_{lb}$}‚पेण वृक्षार्थ‚स्य ग्र‚ह‚णं नाप्य‚{\tiny $_{१}$}‚वृक्षार्थ‚स्य । किन्त्व‚न्योन्य‚व्य‚व‚च्छेदेन [।] त‚त्र [।] अवृ‚{\tiny $_{lb}$}‚\textbf{क्ष‚व्य‚तिरेकेण वृक्षार्थ‚ग्र‚ह‚णे} वृक्ष‚श‚ब्द‚स्य योर्थ‚स्त‚स्य ग्र‚ह‚णेऽभ्युप‚ग‚म्य‚माने । द्व‚यं‚{\tiny $_{lb}$}‚ वृक्षावृक्ष‚ग्र‚ह‚ण\textbf{म‚न्योन्याश्र‚यं} । त‚था ह्य‚वृक्षार्थ‚व्य‚व‚च्छेदेन वृक्षार्थ‚ग्र‚ह‚णे स‚त्य‚वृक्ष‚{\tiny $_{lb}$}‚ग्र‚ह‚ण‚पूर्व‚कं वृक्ष‚ग्र‚ह‚ण‚मंगीकृतं । अगृहीत‚स्यावृक्ष‚स्य व्य‚व‚च्छेतुम‚श‚क्य‚त्वात् । अवृ‚{\tiny $_{lb}$}‚क्ष‚स्यापि ग्र‚ह‚णं वृक्षार्थ‚व्य‚व‚च्छेदेनेति त‚त्रापि वृक्ष‚ग्र‚ह‚{\tiny $_{२}$}‚ण‚पूर्व‚क‚म‚वृक्ष‚ग्र‚ह‚ण‚माप‚तितं ।‚{\tiny $_{lb}$}‚ वृक्ष‚म‚गृहीत्वा त‚द्व्य‚व‚च्छेदेनावृक्षार्थ‚स्य व्य‚स्थाप‚यितुम‚श‚क्य‚त्वात् । एवं वृक्षावृक्ष‚{\tiny $_{lb}$}‚योर्म‚ध्ये \textbf{एक}स्य वृक्ष‚स्यावृक्ष‚स्य वा \textbf{ग्र‚हाभावे द्व‚याग्र‚हः} ।
	{\color{gray}{\rmlatinfont\textsuperscript{§~\theparCount}}}
	\pend% ending standard par
      ‚{\tiny $_{lb}$}‚

	  
	  \pstart \leavevmode% starting standard par
	य‚त‚श्च द्व‚योर‚प्य‚ग्र‚ह\textbf{स्त‚स्मात्} कार‚णाद‚न्य‚व्य‚व‚च्छिन्नेर्थे श‚ब्द‚स्य यः \textbf{संकेत}‚{\tiny $_{lb}$}‚ उक्त‚स्त‚स्या\textbf{स‚म्भ‚व इति केचिदाच‚क्ष‚ते} । त‚था चाहो द्यो त क रः । स याव‚च्चा‚{\tiny $_{lb}$}‚गान्न प्र‚तिप‚द्य‚{\tiny $_{३}$}‚ते ताव‚द‚ग‚वि प्र‚तिप‚त्तिर्न युक्ता । याव‚च्च गान्न प्र‚तिप‚द्य‚ते ताव‚द्‚{\tiny $_{lb}$}‚ग‚वीत्युभ‚य‚प्र‚तिप‚त्त्य‚भाव इति ।
	{\color{gray}{\rmlatinfont\textsuperscript{§~\theparCount}}}
	\pend% ending standard par
      ‚{\tiny $_{lb}$}‚

	  
	  \pstart \leavevmode% starting standard par
	एत‚मेवार्थ म्भ ट्टो प्याह ।
	{\color{gray}{\rmlatinfont\textsuperscript{§~\theparCount}}}
	\pend% ending standard par
      ‚{\tiny $_{lb}$}‚
	  \bigskip
	  \begingroup
	
	    
	    \stanza[\smallbreak]
	  {\normalfontlatin\large ``\qquad}सिद्ध‚श्चागौर‚पोह्येत गोनिषेधात्म‚क‚श्च सः ।&‚{\tiny $_{lb}$}‚त‚त्र गौरेव व‚क्त‚व्योन‚न्यो यः प्र‚तिषिध्य‚ते ।&‚{\tiny $_{lb}$}‚स चेद‚गोनिवृत्त्यात्मा भ‚वेद‚न्योन्य‚संश्र‚यः ।&‚{\tiny $_{lb}$}‚सिद्ध‚श्चेद् गौर‚पोह्यार्थं वृथापोह‚प्र‚व‚र्त्त‚नं ।&‚{\tiny $_{lb}$}‚ग‚व्य‚सिद्धे त्व‚गौर्नास्ति त‚द‚भा‚{\tiny $_{४}$}‚वेपि गौः कुत\edtext{}{\edlabel{pvsvt_233-1}\label{pvsvt_233-1}\lemma{कुत}\Bfootnote{प्र‚क‚ल्प‚नाम also}} इति ।{\normalfontlatin\large\qquad{}"}\&[\smallbreak]
	  
	  
	  
	  \endgroup
	‚{\tiny $_{lb}$}‚

	  
	  \pstart \leavevmode% starting standard par
	त‚स्माद् व‚स्तुभूतं सामान्य‚मेष्ट‚व्यं । त‚त्र विधिरूपेणैव संकेत इति ते म‚न्य‚न्ते ।
	{\color{gray}{\rmlatinfont\textsuperscript{§~\theparCount}}}
	\pend% ending standard par
      ‚{\tiny $_{lb}$}‚‚{\tiny $_{lb}$}‚‚{\tiny $_{lb}$}‚\textsuperscript{\textenglish{234/s}}

	  
	  \pstart \leavevmode% starting standard par
	\textbf{य‚दी}त्यादिना व्याच‚ष्टे । \textbf{त‚स्ये}त्य‚वृक्ष‚भेद‚ल‚क्ष‚ण‚स्य वृक्ष‚स्य । एव‚न्ताव‚न्न वृक्ष‚स्य‚{\tiny $_{lb}$}‚ ग्र‚ह‚णं नाप्य‚वृक्ष‚स्येत्याह‚{\tiny $_{१}$}‚ [।] \textbf{अविज्ञा}तेत्यादि । \textbf{अविज्ञातो वृक्षो} य‚स्य पुंस‚स्तेन‚{\tiny $_{lb}$}‚ \textbf{त‚द्व्य‚व‚च्छेद‚रूप‚स्ये}ति वृक्ष‚व्य‚व‚च्छेद‚रूप‚स्य \textbf{बुद्धाव‚नारूढेर्थ} इति [।] य‚दि वृक्षावृक्षौ‚{\tiny $_{lb}$}‚ बुद्धावारूढौ स्यातां त‚दाऽवृक्ष‚प‚रिहारेण वृक्षे संकेतः स्यात् । अनारूढे च वृक्षेऽवृक्षे‚{\tiny $_{lb}$}‚ चार्थे क‚थं संकेतः [।]
	{\color{gray}{\rmlatinfont\textsuperscript{§~\theparCount}}}
	\pend% ending standard par
      ‚{\tiny $_{lb}$}‚

	  
	  \pstart \leavevmode% starting standard par
	\textbf{तेषा}मित्यादिना प‚र‚स्याप्य‚य‚न्दोष‚स्तुल्य इत्याह । त‚त‚श्च य‚स्त‚स्य प‚रिहारो‚{\tiny $_{lb}$}‚ म‚मापि स एवेति भावः । \href{http://sarit.indology.info/?cref=pv.3.113}{। ११६ ॥}
	{\color{gray}{\rmlatinfont\textsuperscript{§~\theparCount}}}
	\pend% ending standard par
      ‚{\tiny $_{lb}$}‚

	  
	  \pstart \leavevmode% starting standard par
	\textbf{य ए}क‚मित्यादिना व्याच‚ष्टे । अन्य‚व्य‚व‚च्छेदेन संकेते क्रिय‚माणे ये वादिन‚{\tiny $_{lb}$}‚ \textbf{एक‚म्व‚स्तु सामान्य‚म‚भ्युग‚म्येत‚रेत‚राश्र‚य‚दोषं‚{\tiny $_{६}$}‚ चोद‚य‚न्}ति । \textbf{तेषान्त‚त्रापि} व‚स्तु‚{\tiny $_{lb}$}‚भूते सामान्ये वृक्ष‚त्व‚ल‚क्ष‚णे \textbf{संकेते क्रिय‚माणे} द्व‚यी क‚ल्प‚ना अवृक्षा \textbf{व्य‚व‚च्छिन्ना‚{\tiny $_{lb}$}‚ न वेति} । \href{http://sarit.indology.info/?cref=pv.3.114}{११७ ॥}
	{\color{gray}{\rmlatinfont\textsuperscript{§~\theparCount}}}
	\pend% ending standard par
      ‚{\tiny $_{lb}$}‚

	  
	  \pstart \leavevmode% starting standard par
	य‚दि \textbf{व्य‚व‚च्छि}न्नास्त‚दा ते \textbf{ज्ञाता} अङ्गीक‚र्त्त‚व्या । अज्ञातानां व्य‚व‚च्छेदाभावात् ।‚{\tiny $_{lb}$}‚ त‚च्च ज्ञान‚न्तेषु न युज्य‚ते । त‚दाह । \textbf{क‚थ‚मि}त्यादि । \textbf{क‚थं} ज्ञाता \textbf{वृक्षार्थ‚ग्र‚ह‚णाद् ऋते} ।‚{\tiny $_{lb}$}‚ \leavevmode\ledsidenote{\textenglish{86b/PSVTa}} इत्येतावान् \textbf{कारिका}भागः । \textbf{प्राक्}छ‚ब्द‚स्तु मिश्र‚क‚व्याख्यानेनोपात्तः । वृ‚{\tiny $_{७}$}‚क्षार्थ‚ग्र‚ह‚{\tiny $_{lb}$}‚ण‚म्विना प्राग् वृक्षार्थ‚ग्र‚ह‚णाद‚वृक्षाः । \textbf{क‚थं ज्ञाता} इत्य‚र्थः । ये तु प्राक्श‚ब्दं‚{\tiny $_{lb}$}‚ का रि का यां प‚ठ‚न्ति तैर‚र्थ‚श‚ब्दो न प‚ठित‚व्यः । न ह्य‚वृक्ष‚निश्च‚य‚काले वृक्षार्थ‚{\tiny $_{lb}$}‚ग्र‚ह‚ण‚म‚स्त्य‚वृक्ष‚ग्र‚ह‚ण‚पूर्व‚क‚त्वात् वृक्ष‚ग्र‚ह‚ण‚स्य [।] न च वृक्ष‚निश्च‚य‚म‚न्त‚रेण वृक्षा‚{\tiny $_{lb}$}‚र्थ‚ग्र‚हो युक्त‚स्त‚स्यापि वृक्ष‚ग्र‚ह‚ण‚पूर्व‚क‚त्वात् । एत‚देवाह । \textbf{न ही}त्यादि । \textbf{त‚दे}ति संके‚{\tiny $_{lb}$}‚त‚काले \textbf{प्र‚तिप‚त्ता} । य‚स्मै संके‚{\tiny $_{१}$}‚तः क्रिय‚ते । क‚स्मा\textbf{न्न वेत्ती}त्याह । \textbf{त‚दि}त्यादि ।‚{\tiny $_{lb}$}‚ त‚स्य वृक्षावृक्ष‚स्य \textbf{ज्ञानायैव त‚द‚र्थित‚या} संकेतार्थित‚यो\textbf{प‚ग‚मादु}प‚स्थित‚त्वात् । क‚थं‚{\tiny $_{lb}$}‚ नाम संकेतोत्त‚र‚कालं वृक्षावृक्षौ ज्ञास्यामीति । य‚द्वा त‚द‚र्थित‚योप‚ग‚मादिति संकेता‚{\tiny $_{lb}$}‚र्थित‚या प्र‚वृतेः । अतो नास्ति संकेत‚काले वृक्षावृक्ष‚ज्ञानं प्र‚तिप‚त्तुः । \textbf{स च} वृक्षावृक्ष‚{\tiny $_{lb}$}‚‚{\tiny $_{lb}$}‚ \leavevmode\ledsidenote{\textenglish{235/s}}म‚जानानः \textbf{क‚थ‚म‚वृक्ष‚व्य‚व‚च्छेदं प्र‚तिप‚द्ये‚{\tiny $_{२}$}‚त संके}ते । नैव प्र‚तिप‚द्येत । अवृक्ष‚व्य‚व‚च्छे‚{\tiny $_{lb}$}‚\textbf{दाप्र‚तिप‚त्तौ} च स‚त्या\textbf{म‚प‚रिहृतो} न व्य‚व‚च्छिन्न‚स्त‚द‚न्य‚स्त‚स्माद् वृक्षाद‚न्यो य‚स्मिन्‚{\tiny $_{lb}$}‚ वृक्षार्थे सो \textbf{प‚रिहृ}त‚त‚द‚न्य‚स्त\textbf{स्मिन्निवेश}स्संकेतः । स‚प्त‚मीति योग‚विभागात्‚{\tiny $_{lb}$}‚ स‚मासः । स य‚स्मिन् श‚ब्देस्तीति म‚त्व‚र्थीय इति । अप‚रिहृत‚त‚द‚न्य‚न्निर्देष्टुं‚{\tiny $_{lb}$}‚ शील‚म‚स्येति णिनिर्वा । अवृक्षाद‚व्य‚व‚च्छिन्नेर्थे संप्र‚मुग्ध‚रूपे संकेतितादिति‚{\tiny $_{lb}$}‚ या‚{\tiny $_{३}$}‚व‚त् ।
	{\color{gray}{\rmlatinfont\textsuperscript{§~\theparCount}}}
	\pend% ending standard par
      ‚{\tiny $_{lb}$}‚

	  
	  \pstart \leavevmode% starting standard par
	अप‚रः प्र‚कारः । वृक्षाद‚न्य‚स्त‚द‚न्य‚स्त‚स्मिन्निवेशः संकेतः । अप‚रिहृत‚श्चासौ‚{\tiny $_{lb}$}‚ त‚द‚न्य‚निवेश‚श्चेति क‚र्म‚धार‚यः । स य‚स्यास्ति वृक्ष‚श‚ब्द‚स्येति पूर्व‚व‚त् । अवृक्षाद‚व्य‚व‚{\tiny $_{lb}$}‚च्छिन्न‚त्वाद् वृक्षार्थ‚स्य । त‚त्र संकेत्य‚मान‚स्य श‚ब्द‚स्यावृक्षेपि निवेशः प्र‚स‚क्तः ।‚{\tiny $_{lb}$}‚ वृक्ष‚भेदेष्विवेति स‚मुदायार्थः । \href{http://sarit.indology.info/?cref=pv.3.115}{। ११८ ॥}
	{\color{gray}{\rmlatinfont\textsuperscript{§~\theparCount}}}
	\pend% ending standard par
      ‚{\tiny $_{lb}$}‚

	  
	  \pstart \leavevmode% starting standard par
	त‚स्मादेवंभूताद् वृक्ष‚श‚ब्दाद् \textbf{व्य‚व‚हारिणां} पुंसां व्य‚व‚हार‚का‚{\tiny $_{४}$}‚ले \textbf{त‚त्प‚रिहारे}णा‚{\tiny $_{lb}$}‚वृक्ष‚प‚रिहारेण निय‚ते शाखादिम‚ति \textbf{प्र‚वृत्तिर्न स्यात्} । किन्त्व‚विशेषेण वृक्षावृक्ष‚योः‚{\tiny $_{lb}$}‚ प्र‚वृत्तिर्भ‚वेत् । किम्व‚त् । \textbf{वृक्ष‚भेद‚व‚त्} । न हि वृक्ष‚श‚ब्दात् प्र‚क‚र‚णादिर‚हिताद्‚{\tiny $_{lb}$}‚ वृक्ष‚विशेषे ख‚दिरादौ त‚द‚न्य‚वृक्ष‚प‚रिहारेण प्र‚वृत्तिर्भ‚व‚ति । \textbf{संकेत‚का}ले तेषा\textbf{म‚व्य‚{\tiny $_{lb}$}‚व‚च्छे}दात् ।
	{\color{gray}{\rmlatinfont\textsuperscript{§~\theparCount}}}
	\pend% ending standard par
      ‚{\tiny $_{lb}$}‚

	  
	  \pstart \leavevmode% starting standard par
	\textbf{न ही}त्यादिना व्याच‚ष्टे । \textbf{संकेत}कालः संकेत‚श‚ब्देनोक्तः‚{\tiny $_{५}$}‚ \textbf{प‚रार्थ‚व्य‚व‚च्छेदेनेति}‚{\tiny $_{lb}$}‚ प‚र‚स्माद‚वृक्षाद् वृक्षाद् वृक्षार्थ‚स्याव्य‚व‚च्छेदेन । द्वितीये तु व्याख्याने [।]‚{\tiny $_{lb}$}‚ प‚र‚स्मिन्न‚वृक्षे वृक्ष‚श‚ब्द‚स्य संकेताव्य‚व‚च्छेदेनेति व्याख्येयं । \textbf{निवेशिता}दिति‚{\tiny $_{lb}$}‚ संकेतितात् । \textbf{त‚त्प‚रिहारेणा}वृक्ष‚प‚रिहारेण \textbf{वृक्ष‚व}दित्य‚स्य व्याख्यानं \textbf{शिंश‚पादि‚{\tiny $_{lb}$}‚भेद‚व‚दि}ति । शिंश‚पाद‚य एव भेदा इति विशेष‚ण‚स‚मासः व‚तिः स‚प्त‚म्य‚र्थे । सूत्रे‚{\tiny $_{lb}$}‚ तु वृक्ष‚भेदा इति ष‚ष्ठीस‚मासः । शिंश‚पाद‚यो हि भेदा वृक्ष‚स्य भेदा भ‚व‚न्तीत्य‚नुरूपैव‚{\tiny $_{lb}$}‚ वृत्तिः तेषु च य‚था वृक्ष‚श‚ब्दान्न प‚र‚स्प‚र‚व्य‚व‚च्छेदेन प्र‚वृत्तिस्त‚था सूत्र‚विभाग‚{\tiny $_{lb}$}‚ एव व्याख्यातं ।
	{\color{gray}{\rmlatinfont\textsuperscript{§~\theparCount}}}
	\pend% ending standard par
      ‚{\tiny $_{lb}$}‚

	  
	  \pstart \leavevmode% starting standard par
	\textbf{अथे}त्यादिना प‚राभिप्राय‚माशंक‚ते । \textbf{अविधायानिषिध्यान्य‚दि}ति प्र‚तिषेध‚द्व‚यं‚{\tiny $_{lb}$}‚ ‚{\tiny $_{lb}$}‚ \leavevmode\ledsidenote{\textenglish{236/s}}केचित् प‚ठ‚न्ति । संकेते विष‚य‚म‚भिधायातोन्य‚च्चानिषिध्येति । त‚च्चायुक्त‚मिव‚{\tiny $_{lb}$}‚ दृश्य‚ते विधिप्र‚तिषेधौ मुक्त्वा श‚ब्द‚प्र‚वृत्त्य‚स‚म्भ‚वात् । एक‚स्य हि प्र‚द‚र्श‚न‚म‚भिद‚ध‚ता‚{\tiny $_{lb}$}‚ विधेर‚ङ्गीकृतः । त‚त‚श्चाविधाय प्र‚द‚र्श्येतिप‚द‚द्व‚यं व्याह‚तं स्यात् । त‚स्माद‚विधायेत्य‚त्रैव‚{\tiny $_{lb}$}‚ न‚ञ् द्र‚ष्ट‚व्यः । \textbf{अविधाय निषिध्यान्य‚दिति} पाठः । निषिध्यान्य‚त् पूर्व‚न्त‚द्व्य‚व‚च्छेदेना‚{\tiny $_{lb}$}‚प‚रं संकेत‚वि‚{\tiny $_{१}$}‚ष‚य‚म‚विधाय । प्र‚तिषेध‚पूर्व‚कं विधिम‚कृत्वा विधिमात्र‚मेव केव‚लं‚{\tiny $_{lb}$}‚ कृत्वेत्य‚र्थः अत । एव \textbf{प्र‚द‚र्श्यैक}मिति वृत्ताव‚पि न क‚स्य‚चिद् व्य‚व‚च्छेदेन किंचिद्‚{\tiny $_{lb}$}‚ विधीय‚त इत्य‚न्य‚निषेध‚पूर्व‚क‚मेव विधानं प्र‚तिषेध‚ति । केव‚ल‚स्तु विधिर‚गीकृत‚{\tiny $_{lb}$}‚ एव । \textbf{एक}मिति सामान्यं । एतेन सामान्ये संकेत‚क‚र‚णात् स‚र्व‚व्य‚क्तिषु कृत इत्या‚{\tiny $_{lb}$}‚च‚ष्टे \href{http://sarit.indology.info/?cref=pv.3.116}{। ११९ ॥}
	{\color{gray}{\rmlatinfont\textsuperscript{§~\theparCount}}}
	\pend% ending standard par
      ‚{\tiny $_{lb}$}‚

	  
	  \pstart \leavevmode% starting standard par
	\textbf{वृक्षोय‚मि}ति \textbf{सं‚{\tiny $_{२}$}‚केत}स्व‚रूप‚न्द‚र्श‚य‚ति । य‚द्व‚स्तु प्र‚द‚र्श्य संकेतः \textbf{कृत‚स्त‚त् प्र‚तिप‚द्य‚ते‚{\tiny $_{lb}$}‚ व्य‚व‚हारेपि [।] तेन} कार‚णेनाय‚म‚न‚न्त‚रोक्तो \textbf{व‚स्तुसामान्य‚वादिनोऽदोषः । त‚था‚{\tiny $_{lb}$}‚ दृष्ट‚मेवार्थ}मिति सामान्यं य‚त्र \textbf{संकेतः कृतः । त‚त्स‚म्ब‚न्धिनं} वेति सामान्य‚स‚म्ब‚न्धिन‚{\tiny $_{lb}$}‚माश्र‚यं । \textbf{त‚त्रापी}ति विधिना केव‚लेनापि संकेते क्रिय‚माणे द्वौ विक‚ल्पौ वृक्षोय‚मिति‚{\tiny $_{lb}$}‚ संकेतं कुर्वाणः \textbf{त‚रुर‚{\tiny $_{३}$}‚य‚म}पीति विद‚धीत । त‚रुर‚य‚मेवेति वा । आद्ये प‚क्षे त‚रुत्व‚{\tiny $_{lb}$}‚म‚न्य‚स्याप्य‚निषिद्ध‚मिति व्य‚व‚हारे निय‚मेन प्र‚वृत्तिर्न स्यादिति स एव प्र‚स‚ङ्गः । अथ‚{\tiny $_{lb}$}‚ त‚रुर‚य‚मेवेति त‚दा स एवात‚रुर‚व्य‚व‚च्छेदोङ्गीकृतः । त‚त‚श्च संकेत‚काले प्र‚तिप‚द्य‚मानेन‚{\tiny $_{lb}$}‚ क‚थं वृक्षावृक्षौ ज्ञाताविति त‚द‚व‚स्थः प्र‚स‚ङ्गः । त‚दाह [।] \textbf{प्र‚संगो न निव‚र्त्त‚त इति} ।
	{\color{gray}{\rmlatinfont\textsuperscript{§~\theparCount}}}
	\pend% ending standard par
      ‚{\tiny $_{lb}$}‚

	  
	  \pstart \leavevmode% starting standard par
	\textbf{एक}मित्यादि‚{\tiny $_{४}$}‚ना व्याच‚ष्टे । \textbf{अय‚म‚पि वृक्षोऽय‚मेव वृक्ष इति । ग‚तिमिति}‚{\tiny $_{lb}$}‚ प्र‚कारं । \textbf{त‚योश्चे}ति द्व‚योर‚पि प्र‚कार‚योः । \textbf{न दोष} इत्यादि प‚रः । \textbf{दृष्टोप‚टाकारो}‚{\tiny $_{lb}$}‚नुभूत\textbf{स्त‚द्विप‚रीत}स्य त‚तो विल‚क्ष‚ण‚स्य \textbf{सुज्ञान‚त्वात्} ।
	{\color{gray}{\rmlatinfont\textsuperscript{§~\theparCount}}}
	\pend% ending standard par
      ‚{\tiny $_{lb}$}‚‚{\tiny $_{lb}$}‚\textsuperscript{\textenglish{237/s}}

	  
	  \pstart \leavevmode% starting standard par
	एत‚देव ग्र‚ह‚ण‚क‚वाक्य\textbf{मेकं ही}त्यादिना व्याच‚ष्टे । एकं हि \textbf{किंचि}त् सामान्यं‚{\tiny $_{lb}$}‚ वृक्ष‚त्वादिक\textbf{म्प‚श्य‚तोन्य‚त्र} त‚त्सामान्य‚र‚हिते विल‚क्ष‚णे व‚स्तुनि‚{\tiny $_{५}$}‚ \textbf{त‚दाकार‚विवेकिनीं}‚{\tiny $_{lb}$}‚ य‚था प‚रिदृष्टाकार‚विल‚क्ष‚णाकारा\textbf{म्बुद्धिम‚नुभ‚व‚तः} पुंसो \textbf{य‚थानुभ‚व‚न्त‚तो} य‚था प‚रि‚{\tiny $_{lb}$}‚दृष्टा\textbf{द‚न्य}दित्येवंरूपो \textbf{वैध‚र्म्य‚निश्च‚यो} वैल‚क्ष‚ण्य‚निश्च‚यः ।
	{\color{gray}{\rmlatinfont\textsuperscript{§~\theparCount}}}
	\pend% ending standard par
      ‚{\tiny $_{lb}$}‚

	  
	  \pstart \leavevmode% starting standard par
	एतेन वैध‚र्म्य‚निश्च‚य‚स्य स्व‚भाव उक्तः । \href{http://sarit.indology.info/?cref=pv.3.117}{। १२० ॥}
	{\color{gray}{\rmlatinfont\textsuperscript{§~\theparCount}}}
	\pend% ending standard par
      ‚{\tiny $_{lb}$}‚

	  
	  \pstart \leavevmode% starting standard par
	\textbf{त‚द्विवेच‚न} इति व्यापारः पूर्व‚दृष्टादृष्टार्थ‚विवेच‚नः पृथ‚ग्भाव‚स्य व्य‚व‚स्था‚{\tiny $_{lb}$}‚प‚य‚ति । \textbf{त‚तो}न्य‚दित्य‚नेनैव त‚द्वि‚{\tiny $_{६}$}‚वेच‚ने सिद्धे य‚त्पुन‚स्त‚द्विवेच‚न‚ग्र‚ह‚ण‚न्त‚त्स्प‚ष्टार्थं ।‚{\tiny $_{lb}$}‚ त‚त्रैत‚स्मिन् क्र‚मे स‚ति य‚थानुभ‚व‚म्वैध‚र्म्य‚निश्च‚य‚वान् \textbf{स} प्र‚तिप‚त्ता । \textbf{यं} शाखादि‚{\tiny $_{lb}$}‚म‚न्त‚म‚र्थ‚म्विशिष्ट‚सामान्य‚व‚न्त‚माकारान्त‚राद् \textbf{विवेच}य‚ति । तं पुरोधाया\textbf{य‚मेव‚{\tiny $_{lb}$}‚ वृक्ष इति} प्र‚द‚र्श्य \textbf{व्युत्पादि}तः संकेतं ग्राहितो । \textbf{य‚त्रैव तं} संकेतानुरूपं सामान्यात्मान‚न्न‚{\tiny $_{lb}$}‚ \textbf{प‚श्य‚ति । त‚मेवावृक्षं स्व‚{\tiny $_{७}$}‚य‚मेव} श‚ब्द‚व्यापार‚म्विना \textbf{प्र‚तिप}द्य‚ते । त‚देव‚माकारान्त- \leavevmode\ledsidenote{\textenglish{87b/PSVTa}}‚{\tiny $_{lb}$}‚ रात् स्व‚य‚मेव विवेकेनाव‚धारितं सामान्यात्मान‚मुपादाय संकेते कृते स‚र्वासु स‚जातीय‚{\tiny $_{lb}$}‚व्य‚क्तिषु कृतो भ‚व‚ति । सामान्य‚स्य स‚र्व‚त्रान्व‚यात् । \textbf{अय‚मे}वेति चाव‚धार‚णात् संकेते‚{\tiny $_{lb}$}‚ कृते दृष्ट‚विप‚रीत‚स्य सुज्ञान‚त्वात् । त‚तोन्य‚त्रावृक्ष इति निश्च‚यो भ‚व‚तीति न‚{\tiny $_{lb}$}‚ य‚थोक्त‚दोषः ।
	{\color{gray}{\rmlatinfont\textsuperscript{§~\theparCount}}}
	\pend% ending standard par
      ‚{\tiny $_{lb}$}‚

	  
	  \pstart \leavevmode% starting standard par
	अन्यापोह‚वादिनोप्ये‚{\tiny $_{१}$}‚व‚मिति चेदाह । \textbf{नेद}मित्यादि । \textbf{एक‚त्र} संकेत‚काले \textbf{दृष्ट}‚{\tiny $_{lb}$}‚स्यासाधार‚ण‚स्य \textbf{रूप‚स्य क्व‚चिद्} व्य‚क्त्य‚न्त‚रेऽ\textbf{न‚न्व‚याद‚न}नुग‚मात् । त‚त‚श्च संकेत‚{\tiny $_{lb}$}‚काले यो वृक्ष इत्येव गृहीतो भेद‚स्त‚स्यान्य‚त्र द‚र्श‚न‚न्नास्ति । त‚त्र संकेत‚काले दृष्टे‚{\tiny $_{lb}$}‚ प‚श्चाद् \textbf{दृश्य‚माने} च स्व‚ल‚क्ष‚णे य‚द् भिन्न‚प्र‚तिभासि द‚र्श‚न‚मुत्प‚न्न‚न्तेन हेतुना । वृक्षा‚{\tiny $_{lb}$}‚वृक्ष‚योः \textbf{प्र‚तिप‚त्तौ} क्रिय‚माणायां व्य‚क्त्य‚न्त‚रे‚{\tiny $_{२}$}‚प्य\textbf{न्य‚स्मिन्} वृक्ष‚भेदे\textbf{पि न स्यात्त‚था} वृक्ष‚{\tiny $_{lb}$}‚ इति \textbf{प्र‚तीतिः} । त‚था हि यो वृक्ष‚भेदः संकेत‚काले दृष्ट‚स्त‚स्माद् घ‚टाद‚यो‚{\tiny $_{lb}$}‚ विल‚क्ष‚णास्त‚थान्योपि वृक्ष‚भेदः । त‚त्र य‚था घ‚टादिषु वृक्ष इति प्र‚तिप‚त्तिर्न भ‚व‚ति‚{\tiny $_{lb}$}‚ त‚था वृक्ष‚भेदेपि न स्यात् किन्त्व‚वृक्ष इत्येव प्र‚तिप‚त्तिर्भ‚वेदित्य‚र्थः ।
	{\color{gray}{\rmlatinfont\textsuperscript{§~\theparCount}}}
	\pend% ending standard par
      ‚{\tiny $_{lb}$}‚

	  
	  \pstart \leavevmode% starting standard par
	\textbf{एव‚न्त‚र्ही}त्या चा र्यः त\textbf{त्रापी}ति विक‚ल्पाकारोपि सामान्ये संके‚{\tiny $_{३}$}‚ते क्रिय‚माणे‚{\tiny $_{lb}$}‚ ‚{\tiny $_{lb}$}‚ \leavevmode\ledsidenote{\textenglish{238/s}}तु\textbf{ल्य‚मे}त‚दितीत‚रेत‚राश्र‚य‚प्र‚तिविधानं ।
	{\color{gray}{\rmlatinfont\textsuperscript{§~\theparCount}}}
	\pend% ending standard par
      ‚{\tiny $_{lb}$}‚

	  
	  \pstart \leavevmode% starting standard par
	एत‚दुक्त‚म्भ‚व‚ति । स‚र्व‚भावाः स्व‚हेतुतो भिन्ना इति पूर्व‚मेव प्र‚तिपादितं । तेन‚{\tiny $_{lb}$}‚ वृक्षा अवृक्षाश्च भिन्ना एव निर्विक‚ल्प‚के ज्ञाने प्र‚तिभास‚न्ते [।] वृक्षेषु च विधि‚{\tiny $_{lb}$}‚रूपेणैव वृक्ष‚विक‚ल्प उच्य‚ते । त‚थाऽवृक्षेषु वृक्ष‚निषेधेनावृक्ष‚विक‚ल्प उत्प‚द्य‚त इति‚{\tiny $_{lb}$}‚ कुत इत‚रेत‚राश्र‚य‚त्वं ।
	{\color{gray}{\rmlatinfont\textsuperscript{§~\theparCount}}}
	\pend% ending standard par
      ‚{\tiny $_{lb}$}‚

	  
	  \pstart \leavevmode% starting standard par
	न‚नु य‚{\tiny $_{४}$}‚द्य‚पि विधिरूपेण वृक्ष‚विक‚ल्प‚स्य प्र‚तिप‚त्तिस्त‚थाप्य‚वृक्षादिव्यावृत्तिद्वारे‚{\tiny $_{lb}$}‚णोत्प‚द्य‚मान‚त्वाद‚वृक्षादिप्र‚तिप‚त्त्य‚पेक्ष‚त्व‚न्त‚त‚श्च स एवेत‚रेत‚राश्र‚य‚दोषः ।
	{\color{gray}{\rmlatinfont\textsuperscript{§~\theparCount}}}
	\pend% ending standard par
      ‚{\tiny $_{lb}$}‚

	  
	  \pstart \leavevmode% starting standard par
	नैत‚द‚स्ति । य‚तोऽवृक्षादिव्यावृत्तिर्वृक्षादिस्व‚रूप‚मेव त‚द‚नुभ‚व‚द्वारेणैव वृक्षादि‚{\tiny $_{lb}$}‚विक‚ल्प उत्प‚द्य‚ते न त्व‚वृक्षादिप्र‚तिप‚त्त्य‚पेक्ष इति कुत इत‚रेत‚राश्र‚य‚त्वं । त‚त्र‚{\tiny $_{५}$}‚ वृक्ष‚{\tiny $_{lb}$}‚विक‚ल्पे प्र‚त्येकं शिंश‚पाद्य‚भेदेन वृक्षाकारोऽभिन्नः प्र‚तिभास‚ते । स च संकेतात् पूर्वं‚{\tiny $_{lb}$}‚ स्व‚स‚म्वेद‚न‚प्र‚त्य‚क्ष‚सिद्धः ज्ञान‚रूप‚त्वाद‚त‚स्त‚त्रैव श‚ब्दः संकेत्य‚ते ।
	{\color{gray}{\rmlatinfont\textsuperscript{§~\theparCount}}}
	\pend% ending standard par
      ‚{\tiny $_{lb}$}‚

	  
	  \pstart \leavevmode% starting standard par
	तेन य‚दुच्य‚ते भ ट्टे न ॥
	{\color{gray}{\rmlatinfont\textsuperscript{§~\theparCount}}}
	\pend% ending standard par
      ‚{\tiny $_{lb}$}‚
	  \bigskip
	  \begingroup
	
	    
	    \stanza[\smallbreak]
	  {\normalfontlatin\large ``\qquad}संकेतात् पूर्व‚मिन्द्रियैर‚न्यापोहो न ग‚म्य‚ते ।&‚{\tiny $_{lb}$}‚नान्य‚त्र श‚ब्द‚संकेतः किन्दृष्ट्वा स प्र‚युज्य‚तां ॥&‚{\tiny $_{lb}$}‚अन्व‚येन विमुक्त‚त्वान्नानुमाप्य‚त्र विद्य‚ते ।&‚{\tiny $_{lb}$}‚स‚म्ब‚न्धानुभ‚वो‚{\tiny $_{६}$}‚प्य‚स्य तेन नैवोप‚द्य‚त इति\edtext{}{\edlabel{pvsvt_238-1}\label{pvsvt_238-1}\lemma{इति}\Bfootnote{\href{http://sarit.indology.info/?cref=\%C5\%9Bv}{ Ślokavārtika. }}}{\normalfontlatin\large\qquad{}"}\&[\smallbreak]
	  
	  
	  
	  \endgroup
	‚{\tiny $_{lb}$}‚

	  
	  \pstart \leavevmode% starting standard par
	निर‚स्तं ।
	{\color{gray}{\rmlatinfont\textsuperscript{§~\theparCount}}}
	\pend% ending standard par
      ‚{\tiny $_{lb}$}‚

	  
	  \pstart \leavevmode% starting standard par
	क‚स्मादित‚रेत‚राश्र‚य‚प्र‚तिविधान‚न्तुल्य‚मित्याह । \textbf{य‚स्मा}दित्यादि । स‚जातीय‚{\tiny $_{lb}$}‚व्य‚क्तिष्वेकाकार‚म्प्र‚त्य‚भिज्ञान\textbf{मेक‚प्र‚त्य‚व‚म‚र्श}स्त‚थाऽ\textbf{ख्या} संज्ञा य‚स्य \textbf{ज्ञा}न‚स्य त‚त्त‚था ।‚{\tiny $_{lb}$}‚ अनेन भिन्नास्व‚पि व्य‚क्तिष्वेकाकारं प्र‚त्य‚भिज्ञान‚मेक‚त्व‚मारोप‚य‚तीत्युक्तं । त‚त‚श्च‚{\tiny $_{lb}$}‚ \leavevmode\ledsidenote{\textenglish{88a/PSVTa}} विक‚ल्प‚विज्ञानारोपितैक‚त्वासु व्य‚क्तिषु‚{\tiny $_{७}$}‚ य‚त्र क्व‚चित् संकेतः कृतः स‚र्व‚त्र कृतो भ‚व‚{\tiny $_{lb}$}‚तीत्य‚स्य बीज‚माख्यातं । \textbf{एक‚त्र ही}त्य‚नेन विजातीय‚प‚दार्थ‚प‚राम‚र्श‚शून्याकार‚न्तेन‚{\tiny $_{lb}$}‚ प‚राम‚र्श‚स्य प्र‚तिनिय‚ताकार‚त्व‚माह । विजातीय‚प‚दार्थाकार‚व्यावृत्त्या स‚जातीयेषु‚{\tiny $_{lb}$}‚ स‚र्वेषु य‚देक‚प्र‚त्य‚व‚म‚र्श‚ज्ञान\textbf{न्त‚त्र स्थित} इत्य‚र्थः । एतेनापि स‚जातीयाऽस‚जातीया‚{\tiny $_{lb}$}‚व‚स्तुविभाग‚बीज‚मुक्तं । अत एवाह । त‚दित्य‚नेनैकः‚{\tiny $_{१}$}‚ प‚राम‚र्शो गृह्य‚तेऽत‚च्छ‚ब्देन‚{\tiny $_{lb}$}‚ त‚द्विप‚रीतः । स चास‚श्च त‚द‚तौ । त‚योर्हेत‚व‚स्\textbf{त‚द‚त‚द्धेत‚वः} । तान् \textbf{विभ‚ज}ते । एक‚{\tiny $_{lb}$}‚‚{\tiny $_{lb}$}‚ ‚{\tiny $_{lb}$}‚ \leavevmode\ledsidenote{\textenglish{239/s}}शाखादिम‚दाकार‚प‚राम‚र्श‚हेतून् त‚द्विप‚रीतांश्च पृथ‚क् क‚रोति \textbf{स्व‚य‚मे}व संकेतात् [।]‚{\tiny $_{lb}$}‚ \textbf{प्राग‚पि} निवेदित‚मेत‚द् [।] एक‚प्र‚त्य‚व‚म‚र्शार्थ‚ज्ञानाद्येकार्थ‚साध‚नं \href{http://sarit.indology.info/?cref=pv.3.72}{१ । ७५}‚{\tiny $_{lb}$}‚ इत्य‚त्रान्त‚रे । व्य‚तिरिक्त‚स्याव्य‚तिरिक्त‚स्य च सामान्य‚स्य निषेधान्निषेत्स्य‚मान‚{\tiny $_{lb}$}‚त्वाच्च ।‚{\tiny $_{२}$}‚ \textbf{भावाः प्र‚कृतिभेदिनः} स्व‚भावेनैव विल‚क्ष‚णाः [।] \textbf{ज्ञानादिक}मित्यादि‚{\tiny $_{lb}$}‚श‚ब्दाद् उद‚काद्याह‚र‚णादिकं \textbf{केचिदे}व \textbf{कुर्व‚न्ति नान्ये} \href{http://sarit.indology.info/?cref=pv.3.118}{॥ १२१ ॥}
	{\color{gray}{\rmlatinfont\textsuperscript{§~\theparCount}}}
	\pend% ending standard par
      ‚{\tiny $_{lb}$}‚

	  
	  \pstart \leavevmode% starting standard par
	प्र‚कृत्या त‚द‚त‚ज्ज‚न‚न‚स्व‚भाव‚त्वात्तेषां । \textbf{तान्}भावा\textbf{न‚यं} प्र‚तिप‚त्ता \textbf{स्व‚य‚मेव} श‚ब्द‚{\tiny $_{lb}$}‚व्यापारं विना \textbf{विभ‚ज्य} विभागं कृत्वा \textbf{त‚द्धेतून‚त‚द्धेतूँश्च प्र‚त्येति} । तेन कुत इत‚रेत‚रा‚{\tiny $_{lb}$}‚श्र‚य‚त्व‚दोषः । यो हि त‚द्धेतून‚त‚द्धेतूँश्च भावात्स्व‚{\tiny $_{३}$}‚य‚मेव प्र‚तिप‚द्य‚ते । त‚स्य प्र‚ति‚{\tiny $_{lb}$}‚प‚त्तुस्त‚द्बुद्धिप‚रिव‚र्तिन इत्यादि क‚र्म‚प‚दं प्र‚तिप‚द्येतेत्येत‚त् क्रियाप‚दापेक्षं । अत‚द्धेतु‚{\tiny $_{lb}$}‚भ्य‚स्त‚द्धेतून् विभ‚ज्य स्थाप‚य‚ति या \textbf{बुद्धिः} सा [।] \textbf{त‚द्बुद्धिस्त‚त्प‚रिव}र्त्तिन‚स्त‚दा‚{\tiny $_{lb}$}‚रूढान् । विक‚ल्पिकाया \textbf{धियो हेतुत}या \textbf{भातो} भास‚मानात् ।
	{\color{gray}{\rmlatinfont\textsuperscript{§~\theparCount}}}
	\pend% ending standard par
      ‚{\tiny $_{lb}$}‚

	  
	  \pstart \leavevmode% starting standard par
	इव श‚ब्द‚स्य व‚क्ष्य‚माण‚स्य स‚म्ब‚न्धाद्धेतुत‚येवेति द्र‚ष्ट‚व्यं न तु ते विक‚ल्प‚प्र‚ति‚{\tiny $_{lb}$}‚भासि‚{\tiny $_{४}$}‚नो हेत‚वंस्तेषां ब‚हिर‚स‚त्वात् । केव‚ल‚म्प्र‚तिप‚त्तुस्त‚थाध्य‚व‚सायादेव‚मुच्य‚ते ।‚{\tiny $_{lb}$}‚ \textbf{अहेतुरूप‚विक‚लान् भा}त इत्य‚त्राभिस‚म्ब‚न्धः । \textbf{इव} श‚ब्द‚योग‚श्च पूर्व‚व‚त् । एका‚{\tiny $_{lb}$}‚कार‚प‚राम‚र्श‚बुद्धेर्ये न हेत‚व‚स्तेषां रूपेण विक‚लानिव । दृश्य‚विक‚ल्प‚योरेकीक‚र‚णाद्‚{\tiny $_{lb}$}‚ बाह्येन स\textbf{हैक‚रूपानि}व भात इत्य‚त्राभिस‚म्ब‚न्धः । \textbf{स्व}यं संकेतादुत्त‚र‚काल‚म‚पि‚{\tiny $_{५}$}‚ ।‚{\tiny $_{lb}$}‚ अत‚त्कारिभ्यो \textbf{भेदेन} तान् भावान् \textbf{प्र‚तिप‚द्येतेति} कृत्वा । \textbf{उक्तिः} श‚ब्दो \textbf{भेदे} विजातीय‚{\tiny $_{lb}$}‚व्यावृत्ते स्व‚भावे विक‚ल्पेन स्वाकाराभेदेनाध्य‚स्ते \textbf{नियुज्य}ते संकेत्य‚ते । \textbf{त‚म्}भेदं‚{\tiny $_{lb}$}‚ य‚थोक्तं । \textbf{त‚स्याः} श्रुतेः स‚काशाद् व्य‚व‚हारे \textbf{प्र‚तिय‚ती} प्र‚तिप‚द्य‚माना प‚रि\textbf{धी}र्भ्रान्त्या‚{\tiny $_{lb}$}‚ \textbf{एक‚म्व‚स्त्विवेक्ष‚ते} । स‚जातीय‚व्य‚क्तिषु त‚म्विजातीय‚व्यावृत्तं स्व‚भावं स्वाकाराभेदेन‚{\tiny $_{lb}$}‚ प्र‚तिय‚ती‚{\tiny $_{६}$}‚ धीरेक‚मिव व‚स्तु प्रेक्ष‚त इत्य‚र्थः ।
	{\color{gray}{\rmlatinfont\textsuperscript{§~\theparCount}}}
	\pend% ending standard par
      ‚{\tiny $_{lb}$}‚

	  
	  \pstart \leavevmode% starting standard par
	त‚तः स‚र्वेष्वेक‚प‚राम‚र्श‚हेतुषु वृक्ष इति प्र‚तिप‚त्तिर‚तो य‚दुक्तं द‚र्श‚नेन प्र‚तिप‚त्तौ‚{\tiny $_{lb}$}‚ व्य‚क्त्य‚न्त‚रेपि न स्यादिति त‚द‚पास्तं ।
	{\color{gray}{\rmlatinfont\textsuperscript{§~\theparCount}}}
	\pend% ending standard par
      ‚{\tiny $_{lb}$}‚

	  
	  \pstart \leavevmode% starting standard par
	य‚च्चाप्युक्त‚म्भ ट्टे न ॥
	{\color{gray}{\rmlatinfont\textsuperscript{§~\theparCount}}}
	\pend% ending standard par
      ‚{\tiny $_{lb}$}‚‚{\tiny $_{lb}$}‚\textsuperscript{\textenglish{240/s}}
	  \bigskip
	  \begingroup
	
	    
	    \stanza[\smallbreak]
	  {\normalfontlatin\large ``\qquad}गोश‚ब्दान‚भिधेय‚त्व‚म‚श्वादीनां हि ते क‚थं ।&‚{\tiny $_{lb}$}‚न दृष्ट‚स्त‚त्र गोश‚ब्दः संकेत‚स‚म‚ये य‚दि ।&‚{\tiny $_{lb}$}‚एक‚स्मात्त‚र्हि ते पिण्डाद्य‚द‚न्य‚त् स‚र्व‚म‚व त‚त् ।&‚{\tiny $_{lb}$}‚\leavevmode\ledsidenote{\textenglish{88b/PSVTa}}भ‚वेद‚पो‚{\tiny $_{७}$}‚ह्य‚मित्येवं न‚हि सामान्य‚वाच्य‚तेति [।]\edtext{}{\edlabel{pvsvt_240-1}\label{pvsvt_240-1}\lemma{तेति}\Bfootnote{\href{http://sarit.indology.info/?cref=\%C5\%9Bv}{ Ślokavārtika. }}}{\normalfontlatin\large\qquad{}"}\&[\smallbreak]
	  
	  
	  
	  \endgroup
	‚{\tiny $_{lb}$}‚

	  
	  \pstart \leavevmode% starting standard par
	त‚द‚पि निर‚स्तं । एक‚व्य‚क्तौ गोश‚ब्द‚स्य संकेते विष‚य‚स्य व्य‚क्त्यंत‚रेनुग‚मात् स एवायं‚{\tiny $_{lb}$}‚ गौरिति प्र‚तीतेरिति । त‚थापि [।]
	{\color{gray}{\rmlatinfont\textsuperscript{§~\theparCount}}}
	\pend% ending standard par
      ‚{\tiny $_{lb}$}‚

	  
	  \pstart \leavevmode% starting standard par
	\textbf{तेषा}मित्यादिना का रि कार्थ‚माह । \textbf{तेषां} विजातीय‚व‚स्तुविवेकिनाम‚र्थानां‚{\tiny $_{lb}$}‚ \textbf{प्र‚कृत्या} त‚थाभूत‚विक‚ल्प‚कार‚णानाम‚न्व‚यादिति स‚म्ब‚न्धः । \textbf{प्र‚कृत्या} स्व‚भावेन न‚{\tiny $_{lb}$}‚ पुन‚रेक‚सामान्य‚योगात् । स‚र्वे‚{\tiny $_{१}$}‚ त‚र्हि प‚राम‚र्शं क‚स्मान्न ज‚न‚य‚न्तीत्याह । \textbf{प्र‚त्य‚य‚व‚{\tiny $_{lb}$}‚शा}दिति । अनुभ‚व‚ज्ञानं प्र‚त्य‚य‚स्त‚द्द्वारेण तेषां विक‚ल्प‚ज‚न‚नात् । \textbf{त‚थाभू}त‚स्यैक‚प्र‚त्य‚{\tiny $_{lb}$}‚व‚म‚र्शात्म‚क‚स्य \textbf{विक‚ल्प‚स्य कार‚णानाम‚न्व‚यात्} स‚द्भावाद् [।] य‚थैको वृक्ष‚भेदः‚{\tiny $_{lb}$}‚ प्र‚कृत्या त‚थाभूत‚विक‚ल्प‚हेतुभूत‚स्त‚था द्वितीयादिर‚पीत्य‚नेनाकारेणान्व‚यो न पुन‚रेक‚{\tiny $_{lb}$}‚म्व‚स्तु सामान्यात्म‚क‚म‚स्ति । त‚स्माद‚न्व‚{\tiny $_{२}$}‚याद्धेतोरेक‚कार्य‚व‚त्त्वेनैकाध्य‚व‚साय‚योग्यानिति‚{\tiny $_{lb}$}‚ वाक्य‚शेषः ।
	{\color{gray}{\rmlatinfont\textsuperscript{§~\theparCount}}}
	\pend% ending standard par
      ‚{\tiny $_{lb}$}‚

	  
	  \pstart \leavevmode% starting standard par
	\textbf{त‚द्द्र‚ष्टु}रिति व्य‚क्तिष्वेक‚कार्य‚क‚र‚ण‚स्य द्र‚ष्टु\textbf{र्बुद्धौ विप‚रिव‚र्त्त‚मा}नानारूढान् ।‚{\tiny $_{lb}$}‚ त‚स्माद् द्र‚ष्टुरिति भ‚वित‚व्यं । क‚र्त‚रि चे ति ष‚ष्ठीस‚मास‚प्र‚तिषेधादिति चेन्न ।‚{\tiny $_{lb}$}‚ शेष‚ष‚ष्ठ्या विव‚क्षित‚त्वात् । द्र‚ष्टृश‚ब्द‚स्य चातृन्प्र‚त्य‚यान्त‚त्वात् । त‚त्र तृन्निति ष‚ष्ठी‚{\tiny $_{lb}$}‚ प्र‚तिषेधात् । त‚च्छ‚ब्द‚स्य द्वितीया‚{\tiny $_{३}$}‚न्त‚स्य \textbf{साध‚नं कृते}ति स‚मासः । अन्ये तु \textbf{त‚त्प्र‚योज‚क}‚{\tiny $_{lb}$}‚ इत्यादिनिर्देशात् प्र‚तिषेध‚सूत्र‚स्यानित्य‚त्वं ज्ञाप‚य‚न्ति । एव‚म‚न्येष्व‚पि निर्देशेष्वेवं‚{\tiny $_{lb}$}‚जातीयेष्वेवंरूपाः प‚रिहारा व‚क्त‚व्याः । \textbf{त‚ज्ज्ञान‚हेतुत‚या} त‚स्य विक‚ल्प‚ज्ञान‚स्य‚{\tiny $_{lb}$}‚ हेतुत‚या । \textbf{त‚द‚न्य‚व्यावृत्त्या चे}त्येकाकार‚प्र‚त्य‚भिज्ञान‚हेतुभ्यो येऽन्ये त‚थाभूत‚वि‚{\tiny $_{lb}$}‚क‚ल्पाऽहेत‚वः । तेभ्यो व्या‚{\tiny $_{४}$}‚वृत्त्या च । अ\textbf{त‚थाभूतान}पि न हि ते विक‚ल्पारूढास्त‚द्धे‚{\tiny $_{lb}$}‚त‚वो ब‚हिर‚विद्य‚मान‚त्वात् । अत एवाहेतुरूप‚विक‚ल्प‚त्व‚म‚प्य‚स‚त्तेषाम‚व‚स्तुस‚त्त्वात् ।‚{\tiny $_{lb}$}‚ त\textbf{थाध्य‚व‚सितान्} । त‚ज्ज्ञान‚हेतुत‚या त‚द‚न्य‚व्यावृत्त्या चारोपितान् । अनेन \textbf{भातो‚{\tiny $_{lb}$}‚हेतुत‚या धियः} । अहेतुरूप‚विक‚लानिवेति व्याख्यातं ।
	{\color{gray}{\rmlatinfont\textsuperscript{§~\theparCount}}}
	\pend% ending standard par
      ‚{\tiny $_{lb}$}‚‚{\tiny $_{lb}$}‚‚{\tiny $_{lb}$}‚\textsuperscript{\textenglish{241/s}}

	  
	  \pstart \leavevmode% starting standard par
	\textbf{अविभ‚क्त‚बाह्याध्यात्मिक‚भेदा}नित्य‚नेनैक‚रू‚{\tiny $_{५}$}‚पानित्येत‚द् व्याच‚ष्टे । अविभ‚क्तो‚{\tiny $_{lb}$}‚ बाह्याध्यात्मिक‚भेदो येष्विति विग्र‚हः । दृश्य‚विक‚ल्प‚योरेकीक‚र‚णेन गृहीतानित्य‚र्थः ।‚{\tiny $_{lb}$}‚ य‚स्मै संकेतः क्रिय‚ते स \textbf{प्र‚तिप‚त्ता । प्र‚तिप‚त्तिम‚नुसृत्}य । संकेत‚काले यादृशी त‚स्य‚{\tiny $_{lb}$}‚ प्र‚तिप‚त्तिः । अहेतुरूप‚विक‚ला एक‚कार्या भावा एक‚रूपा येष्व‚यं वृक्ष‚श‚ब्दः संकेतित‚स्त‚{\tiny $_{lb}$}‚ एवामी [।] त‚स्माद् वृक्षा इत्येव‚माकारा ।‚{\tiny $_{६}$}‚ ताम‚नुसृत्य । तां स्मृत्वा । विक‚ल्प‚{\tiny $_{lb}$}‚विज्ञाने स्थित‚स्स‚न् । तान् य‚थोक्तान् भावान् \textbf{त‚द्विज्ञान‚हेतून}त‚द्विप‚रीतेभ्यो भेदेन ।‚{\tiny $_{lb}$}‚ \textbf{एते वृक्षा इति} व्य‚व‚हार‚कालेपि वृक्ष‚श‚ब्द‚श्र‚व‚णात् क‚थ‚न्नाम प्र\textbf{तिप‚द्येतेत्य‚नेनाभि‚{\tiny $_{lb}$}‚प्रायेणा}क्त\textbf{म‚त‚द्धेतुभ्यो भेद}व्य‚व‚च्छिन्ने स्व‚भावे \textbf{विक}ल्पेन विष‚यीकृते नि\textbf{युङ्क्ते}‚{\tiny $_{lb}$}‚ संकेत‚स्य क‚र्त्ता । \href{http://sarit.indology.info/?cref=pv.3.120}{। १२३ ॥}
	{\color{gray}{\rmlatinfont\textsuperscript{§~\theparCount}}}
	\pend% ending standard par
      ‚{\tiny $_{lb}$}‚

	  
	  \pstart \leavevmode% starting standard par
	न‚नु व्यावृत्त‚स्य स्व‚ल‚क्ष‚ण‚स्य व्य‚व‚हा‚{\tiny $_{७}$}‚र‚कालेनुग‚मो नास्ति [।] नापि विक‚ल्प- \leavevmode\ledsidenote{\textenglish{89a/PSVTa}}‚{\tiny $_{lb}$}‚ प्र‚तिभासिनः सामान्याकार‚स्य स्व‚ज्ञानाभिन्न‚त्वाद् विक‚ल्पान्त‚रेन्व‚योस्ति । नापि‚{\tiny $_{lb}$}‚ व‚क्तृस‚म्ब‚न्धिन‚स्त‚स्य श्रोतुः श्रोतृस‚म्ब‚न्धिनो वा व‚क्तुः प्र‚तीतिर‚न्य‚चेतोध‚र्म‚त्वेना‚{\tiny $_{lb}$}‚तीन्द्रिय‚त्वात् । न चाप्र‚तिप‚न्ने स‚मं प्र‚तिपाद्य‚प्र‚तिपाद‚काभ्यां संकेतः स‚म्भ‚व‚तीत्याह ।‚{\tiny $_{lb}$}‚ \textbf{स्व‚प‚रे}त्यादि । स्व‚स्य प्र‚तिपाद‚क‚स्य प‚र‚स्य च प्र‚ति‚{\tiny $_{१}$}‚पाद्य‚स्य \textbf{विक‚ल्पे}ष्वेक‚कार्य‚क‚र‚ण‚{\tiny $_{lb}$}‚ल‚क्ष‚णेन भ्रान्तिनिमित्ते\textbf{नैक‚प्र‚तिभासान्} भावान् संकेत‚विष‚यान्\textbf{आद‚र्श्य} ।
	{\color{gray}{\rmlatinfont\textsuperscript{§~\theparCount}}}
	\pend% ending standard par
      ‚{\tiny $_{lb}$}‚

	  
	  \pstart \leavevmode% starting standard par
	एत‚दुक्त‚म्भ‚व‚ति । य‚थैक‚स्तैमिरिको द्विच‚न्द्र‚न्दृष्ट्वान्य‚तैमिरिकायोप‚दिश‚न्‚{\tiny $_{lb}$}‚ स्व‚दृष्ट‚मेवोप‚दिश‚ति न प‚र‚दृष्ट‚म‚प्र‚त्य‚क्ष‚त्वात् । अथ च त‚स्यैव‚म्भ‚व‚त्य‚य‚मेव म‚या‚{\tiny $_{lb}$}‚ प‚र‚स्मै प्र‚तिपादित इति । प‚रोपि च स्व‚स‚न्तान‚भाविन‚मेव द्विच‚न्द्रा‚{\tiny $_{२}$}‚कार‚म्प्र‚तिय‚न्‚{\tiny $_{lb}$}‚ य एव प्र‚तिपाद‚केन म‚म प्र‚तिपादित‚स्स एव म‚या प्र‚तिप‚न्न इति म‚न्य‚ते । त‚द्व‚त् प्र‚ति‚{\tiny $_{lb}$}‚पाद्य‚प्र‚तिपाद‚क‚योर्बुद्ध्याकार‚स्याध्य‚व‚सित‚बाह्य‚रूप‚स्य भेदेप्येक‚त्वाध्य‚व‚सायात् संकेत‚{\tiny $_{lb}$}‚क‚र‚णं व्य‚व‚हार‚काले च त‚स्यैव प्र‚तीतिरेक‚त्वाध्य‚व‚सात् । त‚मित्य‚न्य‚व्य‚व‚च्छिन्नं‚{\tiny $_{lb}$}‚ स्व‚भाव स्वाकारेणाभिन्न‚म\textbf{ध्य‚स्य प्र‚तिप‚द्य‚माना‚{\tiny $_{३}$}‚ बुद्धिः । त‚स्या} इति श्रुतेः । \textbf{एक‚व‚स्तु‚{\tiny $_{lb}$}‚ग्राहिणीव प्र‚तिभाति} ।
	{\color{gray}{\rmlatinfont\textsuperscript{§~\theparCount}}}
	\pend% ending standard par
      ‚{\tiny $_{lb}$}‚

	  
	  \pstart \leavevmode% starting standard par
	तेन य‚दुच्य‚ते भ ट्टे न ॥
	{\color{gray}{\rmlatinfont\textsuperscript{§~\theparCount}}}
	\pend% ending standard par
      ‚{\tiny $_{lb}$}‚
	  \bigskip
	  \begingroup
	
	    
	    \stanza[\smallbreak]
	  {\normalfontlatin\large ``\qquad}न चान्व‚य‚विनिर्मुक्ता प्र‚वृत्तिः श‚ब्द‚लिङ्ग‚योः ।&‚{\tiny $_{lb}$}‚‚{\tiny $_{lb}$}‚‚{\tiny $_{lb}$}‚\leavevmode\ledsidenote{\textenglish{242/s}}ताभ्यां न विनापोहे धीर्न चासाधार‚णेन्व‚यः ।&‚{\tiny $_{lb}$}‚अपोह‚श्चाप्य‚प्र‚सिद्धोऽव्य‚भिचारः क्व क‚थ्य‚तां ।&‚{\tiny $_{lb}$}‚त‚स्मिन्न‚विद्य‚माने च न त‚योः स्यात्प्र‚माण‚तेति [।]\edtext{}{\edlabel{pvsvt_242-1}\label{pvsvt_242-1}\lemma{तेति}\Bfootnote{\href{http://sarit.indology.info/?cref=\%C5\%9Bv}{ Ślokavārtika. }}}{\normalfontlatin\large\qquad{}"}\&[\smallbreak]
	  
	  
	  
	  \endgroup
	‚{\tiny $_{lb}$}‚

	  
	  \pstart \leavevmode% starting standard par
	अपास्तं । य‚त एक‚स्मिन्न‚न्य‚व्यावृत्ते स्व‚ल‚क्ष‚णे श‚ब्द‚लिङ्गाभ्यां स‚{\tiny $_{४}$}‚म्ब‚न्ध‚म्प्र‚ति‚{\tiny $_{lb}$}‚प‚द्य‚मानोप्य‚न्य‚त्राप्येवंरूपेषु स‚म्ब‚न्धं प्र‚तिप‚द्य‚त एवैक‚त्वाध्य‚व‚सायादिति कुतोन्व‚य‚र‚{\tiny $_{lb}$}‚हित‚त्वादिदोष इति । व‚स्तुभूत‚न्तु सामान्य‚माश्रित्य वृक्षावृक्ष‚विभागो न घ‚ट‚ते ।‚{\tiny $_{lb}$}‚ त‚दाह [।] \textbf{न पुन}रित्यादि । \textbf{एक‚म्व‚स्तु} सामान्यं दृश्य‚मुप‚ल‚ब्धिल‚क्ष‚ण‚प्राप्त‚न्त‚त्र‚{\tiny $_{lb}$}‚ स्व‚ल‚क्ष‚णेष्\textbf{व‚स्ति} । य‚था वृक्ष‚भेदेभ्यो घ‚टाद‚यो भिन्नास्त‚था घ‚वाद‚योपि‚{\tiny $_{५}$}‚ [।]‚{\tiny $_{lb}$}‚ प‚र‚स्प‚र‚न्तेषाम्\textbf{भिन्नानान्द‚र्श‚नेपि य‚स्य} सामान्य‚स्य \textbf{द‚र्श‚नाद‚र्श‚नाभ्यां वृक्षावृक्ष‚विभागं‚{\tiny $_{lb}$}‚ कुर्वी}त य‚त्रेदं सामान्यं दृश्य‚ते स वृक्षो य‚त्र न दृश्य‚ते सोऽवृक्ष इति ।
	{\color{gray}{\rmlatinfont\textsuperscript{§~\theparCount}}}
	\pend% ending standard par
      ‚{\tiny $_{lb}$}‚

	  
	  \pstart \leavevmode% starting standard par
	क‚स्मान्नास्तीत्याह । \textbf{त‚स्ये}त्यादि । सामान्य‚स्य व्\textbf{य‚तिरिक्त‚स्य शाखादिप्र‚ति‚{\tiny $_{lb}$}‚भासा}द् विभागेनाग्र‚ह‚णात् । न हि वृक्षादिषु द्वौ प्र‚तिभासावुप‚ल‚भ्येते । एकः शाखा‚{\tiny $_{६}$}‚‚{\tiny $_{lb}$}‚द्याकारोऽप‚र‚श्चाशाखाद्याकारः । न च शाखाद्याकार एव सामान्यं प्र‚तिभास‚त इति‚{\tiny $_{lb}$}‚ श‚क्य‚म्व‚क्तुन्त‚स्य शाखाद्याकार‚त्वात् । \textbf{द‚ण्ड‚व‚द्द‚ण्डिनीति} वैध‚र्म्य‚दृष्टान्तः । य‚था‚{\tiny $_{lb}$}‚ \textbf{द‚ण्डिनि द‚ण्ड‚स्य} भेदेन ग्र‚ह‚णं । नैवं सामान्य‚स्य । \textbf{अप‚र‚स्मा}च्छाखादिम‚तः \textbf{प्र‚विभा‚{\tiny $_{lb}$}‚\leavevmode\ledsidenote{\textenglish{89b/PSVTa}} गेनागृहीत‚स्य} च सामान्य‚स्य व्य‚क्तिष्\textbf{व‚नुप‚ल‚क्ष‚णात्} । स्व‚रूपेण प‚र‚स्योप‚ल‚{\tiny $_{७}$}‚म्भ‚न‚{\tiny $_{lb}$}‚मुप‚ल‚क्ष‚ण‚न्त‚च्च स्व‚य‚म‚गृहीत‚स्य क‚थ‚म्भ‚वेत् । अन‚र्थान्त‚र‚सामान्य‚वादिन\textbf{स्त्वाकृतेरे‚{\tiny $_{lb}$}‚क‚त्र} व्य‚क्तौ \textbf{दृष्टायाः} स्व‚ल‚क्ष‚णाद‚व्य‚तिरेकात्त‚द्व‚देवा\textbf{न्य‚त्र} व्य‚क्त्य‚न्त‚रे \textbf{द्र‚ष्टुम‚श‚क्य‚{\tiny $_{lb}$}‚त्वात्} । त‚त‚श्च त\textbf{द‚त‚द्व‚तो}रिति संकेत‚काल‚प‚रिदृष्टैक‚वृक्षाकृतिर्य‚स्यास्ति स त‚द्वान् ।‚{\tiny $_{lb}$}‚ प‚श्चाद् व्य‚व‚हार‚काले दृश्य‚मानो वृक्ष‚भेदः पूर्व‚दृष्ट‚वृक्षाकृतिर‚हितोऽत‚द्वान् । त‚यो‚{\tiny $_{lb}$}‚र्य‚था‚{\tiny $_{१}$}‚क्र‚मं \textbf{वृक्षावृक्ष‚त्वे} न्याय‚प्राप्त‚त्वे स‚ति \textbf{व्य‚क्तिरेकै}वेति संकेत‚काले दृष्टैव \textbf{वृक्षः‚{\tiny $_{lb}$}‚ स्या}न्न तु व्य‚व‚हार‚काले दृश्य‚माना । संकेत‚काले दृष्टाया आकृतेर‚न्य‚त्राद‚र्श‚नात् ।
	{\color{gray}{\rmlatinfont\textsuperscript{§~\theparCount}}}
	\pend% ending standard par
      ‚{\tiny $_{lb}$}‚

	  
	  \pstart \leavevmode% starting standard par
	अन्यापोहे श‚ब्दार्थ‚प‚रैर‚व्यापित्वं चोदितं त‚त्प‚रिजिहीर्ष‚वान् पूर्व‚प‚क्ष‚दिग्मात्र‚{\tiny $_{lb}$}‚‚{\tiny $_{lb}$}‚ ‚{\tiny $_{lb}$}‚ \leavevmode\ledsidenote{\textenglish{243/s}}न्ताव‚त् क‚रोति । \textbf{भ‚व‚तु ना}मेत्यादि । क‚स्मात्त‚त्रार्थान्त‚र‚व्य‚व‚च्छेदो नास्तीत्याह ।‚{\tiny $_{lb}$}‚ \textbf{न ह्य‚ज्ञेय‚मि}त्यादि । \textbf{य‚त} इ‚{\tiny $_{२}$}‚त्य‚ज्ञेयात् । अज्ञेयं क‚स्मान्नास्तीति चेदाह । \textbf{त‚त}‚{\tiny $_{lb}$}‚ इत्यादि । अज्ञेयाद् विज्ञेय‚स्य \textbf{भेदेन विष‚यीक‚र‚ण‚म}ङ्गीक‚र्त्त‚व्य‚म‚न्य‚व्य‚व‚च्छेद‚वादिना‚{\tiny $_{lb}$}‚ऽन्य‚था क‚थ‚म‚ज्ञेयात् ज्ञेय‚स्य व्य‚व‚च्छेदः । \textbf{त‚त}श्चाज्ञेयात् \textbf{ज्ञेय‚स्य} भेदेन विष‚यीक‚र‚णे‚{\tiny $_{lb}$}‚ स‚त्येव त‚स्याज्ञेयाभिम‚त‚स्य ज्ञेय‚त्वात् । न ह्य‚विष‚यीकृताद् व्य‚व‚च्छेदः श‚क्यो द‚र्श‚यितुं ।‚{\tiny $_{lb}$}‚ आदिश‚ब्दात् स‚{\tiny $_{३}$}‚र्व‚स‚मुदाय‚द्व्यादिश‚ब्दानां ग्र‚ह‚णं ।
	{\color{gray}{\rmlatinfont\textsuperscript{§~\theparCount}}}
	\pend% ending standard par
      ‚{\tiny $_{lb}$}‚

	  
	  \pstart \leavevmode% starting standard par
	त‚दुक्त‚म्भ ट्टो द्यो त क रा भ्यां । अन्यापोह‚श्च श‚ब्दार्थ इत्य‚युक्त‚म‚व्याप‚क‚त्वात् ।‚{\tiny $_{lb}$}‚ य‚त्र द्वैराश्य‚म्भ‚व‚ति त‚त्रेत‚र‚प्र‚तिषेधादित‚रः प्र‚तीय‚ते य‚था गौरिति प‚देऽगोः प्र‚तिषेधेन‚{\tiny $_{lb}$}‚ गौः प्र‚तीय‚ते ।
	{\color{gray}{\rmlatinfont\textsuperscript{§~\theparCount}}}
	\pend% ending standard par
      ‚{\tiny $_{lb}$}‚

	  
	  \pstart \leavevmode% starting standard par
	न पुनः स‚र्व‚प‚द एत‚द‚स्ति । न ह्य‚स‚र्व‚न्नाम किंचिद‚स्ति य‚त्स‚र्व‚श‚ब्देन विनिव‚र्त्त्येत ।‚{\tiny $_{lb}$}‚ अथ म‚न्य‚से एकाद्य‚स‚र्वं‚{\tiny $_{४}$}‚ त‚त् स‚र्व‚श‚ब्देन निव‚र्त्त्य‚ते । त‚त्र स्वार्थाप‚वाद‚दोष‚प्र‚स‚{\tiny $_{lb}$}‚ङ्गात् । एवं हि स‚त्येकादिव्युदासेन प्र‚व‚र्त्त‚मानः श‚ब्दः । अङ्ग‚प्र‚तिषेधाद‚ङ्ग‚व्य‚ति‚{\tiny $_{lb}$}‚रिक्त‚स्य चाङ्गिनोन‚भ्युप‚ग‚माद‚न‚र्थ‚कः स्यात् । एवं स‚र्व‚स‚मुदाय‚श‚ब्दा एक‚देश‚{\tiny $_{lb}$}‚प्र‚तिषेध‚रूपेण प्र‚व‚र्त्त‚मानाः स‚मुदायिव्य‚तिरिक्त‚स‚मुदायान‚भ्युप‚ग‚माद‚न‚र्थ‚काः प्राप्नु‚{\tiny $_{lb}$}‚व‚न्ति । द्व्या‚{\tiny $_{५}$}‚दिश‚ब्दानां च स‚मुच्च‚य‚विष‚य‚त्वादेकादिप्र‚तिषेधे प्र‚तिषिध्य‚माना‚{\tiny $_{lb}$}‚नाम‚स‚मुच्च‚य‚त्वात् । द्व्यादिश‚ब्दानाम‚न‚र्थ‚क‚त्व‚मिति ।
	{\color{gray}{\rmlatinfont\textsuperscript{§~\theparCount}}}
	\pend% ending standard par
      ‚{\tiny $_{lb}$}‚

	  
	  \pstart \leavevmode% starting standard par
	\textbf{नैष दोष} इत्याचार्यः । य‚स्मादाकांक्षाव‚तीं \textbf{बुद्धिं कुत‚श्चिन्निव‚र्त्त्य} त‚स्या \textbf{बुद्धेः‚{\tiny $_{lb}$}‚ क्व‚चि}द्विनिय‚तेऽर्थे \textbf{निवेश‚नाया}काङ्क्षाव‚तः पुंसः क‚श्चि\textbf{च्छ‚ब्दः प्र‚युज्य}ते प्र‚तिपाद‚{\tiny $_{lb}$}‚यित्रा । \textbf{क्व‚चिन्निवेश‚ना}येत्य‚नेनान्व‚य उक्तः ।‚{\tiny $_{६}$}‚ कुत‚श्चिद्विनिव‚र्त्त्येत्य‚नेन व्य‚तिरेकः ।
	{\color{gray}{\rmlatinfont\textsuperscript{§~\theparCount}}}
	\pend% ending standard par
      ‚{\tiny $_{lb}$}‚

	  
	  \pstart \leavevmode% starting standard par
	किङ्कार‚णं कुत‚श्चिन्निव‚र्त्त्य क्व‚चिन्निवेश्य‚ते श‚ब्द इत्य‚त आह । \textbf{त‚द‚र्थ}स्येत्यादि ।‚{\tiny $_{lb}$}‚ श‚ब्दार्थ‚स्या\textbf{व‚धार‚णात् । अन्य‚था} य‚दि तेन श‚ब्देन न क‚श्चिद‚र्थो व्य‚व‚च्छिद्येत \textbf{व्य‚र्थः}‚{\tiny $_{lb}$}‚ श‚ब्द\textbf{प्र‚योगः स्यात्} । य‚त एव\textbf{न्त‚स्माज्ज्ज्ञेयादिप‚दे}ष्वित्यादिश‚ब्दात् स‚र्व‚विश्वादिप‚{\tiny $_{lb}$}‚  ‚{\tiny $_{lb}$}‚ ‚{\tiny $_{lb}$}‚ \leavevmode\ledsidenote{\textenglish{244/s}}\leavevmode\ledsidenote{\textenglish{90a/PSVTa}} देषु । किम्विशिष्टेषु [।] \textbf{व्य‚व‚हारोप‚नीते}षु । विधिप्र‚{\tiny $_{७}$}‚तिषेध‚ल‚क्ष‚णः शाब्दो व्य‚व‚{\tiny $_{lb}$}‚हार‚स्त‚द‚र्थ‚मुप‚नीतेषु । लौकिक‚प्र‚योग‚स्थेष्विति याव‚त् । तेषु व्य‚व‚हाराङ्गेषु य‚था‚{\tiny $_{lb}$}‚क‚थंचिद् \textbf{व्य‚व‚च्छेद्योस्ति क‚श्}चित् ।
	{\color{gray}{\rmlatinfont\textsuperscript{§~\theparCount}}}
	\pend% ending standard par
      ‚{\tiny $_{lb}$}‚

	  
	  \pstart \leavevmode% starting standard par
	एत‚दुक्त‚म्भ‚व‚ति । य‚त्प‚र‚श्च श‚ब्दः स श‚ब्दार्थ इति विधाय‚क‚स्यापि वाक्य‚स्य‚{\tiny $_{lb}$}‚ व्य‚व‚च्छेद‚प‚र‚त्वाद् व्य‚व‚च्छेदोपि श‚ब्दार्थ उच्य‚ते इति न काचित् क्ष‚तिः ।
	{\color{gray}{\rmlatinfont\textsuperscript{§~\theparCount}}}
	\pend% ending standard par
      ‚{\tiny $_{lb}$}‚

	  
	  \pstart \leavevmode% starting standard par
	\textbf{श‚ब्दं हीत्या}दिना व्याच‚ष्टे । \textbf{स‚र्वः} पुमान् लौकिकः प‚रीक्ष‚{\tiny $_{१}$}‚को वा क‚स्मान्नाति‚{\tiny $_{lb}$}‚व‚र्त्त‚त इत्याह । \textbf{त‚स्ये}त्यादि । \textbf{त‚स्य} श‚ब्द‚स्य \textbf{प्र‚वृत्तिनिवृत्तिफ‚ल‚त्वादिति} कुत‚श्चिन्नि‚{\tiny $_{lb}$}‚व‚र्त्त्य क्व‚चित् प्र‚वृत्त्य‚र्थ‚त्वादित्य‚र्थः । इत‚र‚था श‚ब्द‚प्र‚योगो विफ‚लः स्यात् । त‚दाह ।‚{\tiny $_{lb}$}‚ \textbf{य‚दी}त्यादि । \textbf{अय}म्व‚क्ता \textbf{क‚स्य‚चित्} प्र‚तिपाद्य‚स्य \textbf{कुत‚श्}चिद‚न‚भिम‚ता\textbf{न्न निव‚र्त्त‚येद्‚{\tiny $_{lb}$}‚ बुद्धिम‚निव‚र्त्त्या}भिम‚ते च क्व‚चिन्न प्र‚व‚र्त्त‚येत् त‚दा \textbf{य‚थाभूतानुज्ञाना}दिति श‚ब्द‚प्र‚{\tiny $_{२}$}‚योगात्‚{\tiny $_{lb}$}‚ पूर्वं प्र‚तिप‚त्तुर्य‚था संप्र‚मुग्ध‚रूपोर्थ‚स्त‚थाभूत‚स्य श‚ब्देनान‚नुज्ञानात् । य‚थाक‚थंचिद्‚{\tiny $_{lb}$}‚ यादृश‚स्य तादृश‚स्यानुज्ञानादित्य‚र्थः । \textbf{स‚र्व‚व्य‚व‚हारेषु न किंचि}द्व‚च‚नं व्\textbf{य‚व‚ह‚रेदु}च्चा‚{\tiny $_{lb}$}‚र‚येत् । किङ्कार‚णं [।] श‚ब्द\textbf{व्य‚व‚हार‚स्याव‚धार‚ण‚नान्त‚रीय‚क‚त्वा}त् ।
	{\color{gray}{\rmlatinfont\textsuperscript{§~\theparCount}}}
	\pend% ending standard par
      ‚{\tiny $_{lb}$}‚

	  
	  \pstart \leavevmode% starting standard par
	एत‚देव साध‚य‚न्नाह । \textbf{य‚थे}त्यादि । अत्र ह्यु\textbf{द‚क‚मान}ये त्युक्ते श्रोतुः क‚र‚ण‚{\tiny $_{१}$}‚‚{\tiny $_{lb}$}‚विशेषेऽ\textbf{व‚श्य‚माकां‚{\tiny $_{३}$}‚क्षा} भ‚व‚ति त‚त्र च निय‚मार्थं \textbf{घ‚टेने}त्युच्य‚ते । सोयं घ‚टो नेति‚{\tiny $_{lb}$}‚ श‚ब्दः स्वार्थाभिधान‚पुर‚स्स‚र‚मेव क‚र‚णान्त‚र‚व्य‚व‚च्छेदाक्षेपात् फ‚ल‚वान् भ‚व‚त्य‚न्य‚था स‚{\tiny $_{lb}$}‚त्य‚पि घ‚ट‚श‚ब्द‚प्र‚योगे य‚दि \textbf{नाञ्ज‚लिना} त\textbf{थान्येना}पि क‚र‚णेनोद‚कान‚य‚नं \textbf{य‚थाक‚थंचिदि}‚{\tiny $_{lb}$}‚ति । अल्प‚प्र‚माणं ब‚हुप्र‚माणं वा ज‚लान‚य‚न\textbf{म‚भिप्रे}त‚मित्य‚र्थः । उद‚क‚श‚ब्दोपि क‚र्मान्त‚{\tiny $_{४}$}‚‚{\tiny $_{lb}$}‚र‚व्य‚व‚च्छेदेन य‚दि विशिष्टे क‚र्म‚णि न प्र‚व‚र्त्त‚क‚स्त‚दा त‚स्यापि प्र‚योगोन‚र्थ‚क इत्याह ।‚{\tiny $_{lb}$}‚ \textbf{त‚थे}त्यादि । \textbf{आन‚ये}त्येव \textbf{केव‚ल‚म्व‚च‚नं स्यात्} [।] किंभूत‚म\textbf{नाक्षिप्त‚क‚र‚ण‚क‚र्म‚कं} ।‚{\tiny $_{lb}$}‚ अनाक्षिप्त‚विशेष‚णानाश्रितं क‚र‚णं घ‚टाख्यं क‚र्म चोद‚नाख्यं य‚स्मिन्नान‚येत्येताव‚ति व‚च‚{\tiny $_{lb}$}‚ने त‚त्त‚थोक्तं । त‚थान‚येत्य‚स्मिन् व्यापारेभिमुखीभूतः पुमान् । आन‚येत्य‚ने‚{\tiny $_{५}$}‚न य‚द्या‚{\tiny $_{lb}$}‚न‚य‚नाद‚न्य‚स्माद् व्यापारान्न व्य‚व‚च्छिद्येत त‚दाऽन‚येत्य‚पि न वाच्यं स्यात् ।
	{\color{gray}{\rmlatinfont\textsuperscript{§~\theparCount}}}
	\pend% ending standard par
      ‚{\tiny $_{lb}$}‚‚{\tiny $_{lb}$}‚\textsuperscript{\textenglish{245/s}}

	  
	  \pstart \leavevmode% starting standard par
	एत‚देवाह । \textbf{एव‚मान‚य‚न}मित्यादि । \textbf{अन्य‚द्वा किञ्चिद‚नुष्ठान‚मि}ति । आन‚य‚ना‚{\tiny $_{lb}$}‚द‚न्य‚त् । किम् [।] भोज‚नाद्य‚नुष्ठानं । \textbf{अन‚नुष्ठानं} चेति व्यापाराक‚र‚ण‚म‚नान‚य‚नं च‚{\tiny $_{lb}$}‚ \textbf{य‚द्य‚भिम‚तं स्यात‚दा} क्रियाप\textbf{द‚मान‚येत्य‚पि न ब्रूयात्} । न‚य‚न‚म‚न्य‚द्वेति क्व‚चित् पुस्त‚के‚{\tiny $_{lb}$}‚ पा‚{\tiny $_{६}$}‚ठः \textbf{स त्त्व}\edtext{}{\lemma{ठः}\Bfootnote{? त्व}} युक्तः । आन‚य‚न‚श‚ब्द‚स्य प्र‚क्रान्त‚त्वात् । त‚स्माद् व्य‚व‚हारो‚{\tiny $_{lb}$}‚प‚नीतानां घ‚टादिश‚ब्दानाम‚स्ति \textbf{व्य‚व‚च्छेद्यो} य‚था तेषा\textbf{न्त‚था व्य‚व‚हारोप‚नीतानां‚{\tiny $_{lb}$}‚ ज्ञेयादिश‚ब्दानां} केन‚चिद् व्य‚व‚च्छेद्येनाज्ञेयादिना ।
	{\color{gray}{\rmlatinfont\textsuperscript{§~\theparCount}}}
	\pend% ending standard par
      ‚{\tiny $_{lb}$}‚

	  
	  \pstart \leavevmode% starting standard par
	\textbf{अन‚न्याश‚ङ्कायामि}त्य‚ज्ञेय‚त्वादेराश‚ङ्काऽन्याश‚ङ्का । त‚द‚भावोन‚न्या शंका ।‚{\tiny $_{lb}$}‚ अस‚त्याम‚ज्ञेय‚त्वाद्याश‚ङ्कायामित्य‚र्थः । त‚था ह्य‚नित्यादि‚{\tiny $_{७}$}‚रूपेणाज्ञेयः श‚ब्द इत्या- \leavevmode\ledsidenote{\textenglish{90b/PSVTa}}‚{\tiny $_{lb}$}‚ शंकायामिदं प्र‚युज्य‚तेऽनित्यादिनाकारेण ज्ञेय इति । त‚त्रानित्याद्याकारेण य‚द‚ज्ञे‚{\tiny $_{lb}$}‚य‚त्व‚माशंकितं त‚देव व्य‚व‚च्छिद्य‚ते । एवं ज्ञेयास्स‚र्व‚प‚दार्थास्स‚र्वज्ञज्ञान‚स्येत्य‚त्रापि‚{\tiny $_{lb}$}‚ स र्व ज्ञ ज्ञानापेक्ष‚या य‚द‚ज्ञेय‚त्व‚माशंकित‚न्त‚देव व्य‚व‚च्छेद्यं । त‚था क‚श्चिदाह । निरु‚{\tiny $_{lb}$}‚पाख्यानाम‚भावात्त‚त्र ज्ञान‚स्य वृत्तिर्नास्ति त‚स्माद‚ज्ञेयास्त इति ।‚{\tiny $_{१}$}‚ अत्राप्य‚ज्ञेय‚त्व‚{\tiny $_{lb}$}‚मारोपित‚न्त‚देव व्य‚व‚च्छेद्यं । स‚र्वाभावो न भ‚व‚तीत्येव‚म‚भाव‚स्यापि विष‚यीक‚र‚णात् ।‚{\tiny $_{lb}$}‚ एव‚म‚न्य‚त्रापि ज्ञेय‚श‚ब्द‚प्र‚योगे वाच्यं । त‚था प्र‚मेय‚श‚ब्दे । त‚था \textbf{क्ष‚णिकास्स‚र्वे संस्कारा}‚{\tiny $_{lb}$}‚ इत्य‚त्रापि स‚र्व‚स्य दीपादेरेव क्ष‚णिक‚त्वं कैश्चित् क‚ल्पित‚न्न स‚र्व‚स्य [।] त‚द्व्य‚व‚च्छेदेन‚{\tiny $_{lb}$}‚ स‚र्व‚संस्काराणाम‚नित्य‚त्वं । एवं क‚श्चिदाग‚तः किम्वा स‚{\tiny $_{२}$}‚र्व एवेत्याश‚ङ्कायां स‚र्वो‚{\tiny $_{lb}$}‚ ग्राम आग‚तः । इति क‚स्य‚चिदेव य‚दाग‚म‚न‚माशंकित‚न्त‚देव व्य‚व‚च्छेद्यं । त‚था स‚मुदाया‚{\tiny $_{lb}$}‚ल‚म्ब‚नाः प‚ञ्च विज्ञान‚काया इति चैक‚देशाल‚म्ब‚ना इत्येक‚देशाल‚म्ब‚न‚त्वं निषिध्य‚ते ।‚{\tiny $_{lb}$}‚ एव‚म‚न्येष्व‚पि द्व्यादिश‚ब्देषु व्य‚व‚हारोप‚नीतेषु प्र‚क‚र‚ण‚व‚शाद् य‚थायोगं व्य‚व‚च्छेदो‚{\tiny $_{lb}$}‚ व‚क्त‚व्यः ।
	{\color{gray}{\rmlatinfont\textsuperscript{§~\theparCount}}}
	\pend% ending standard par
      ‚{\tiny $_{lb}$}‚

	  
	  \pstart \leavevmode% starting standard par
	अय‚म‚त्र स‚मुदायार्थः [।] न व‚स्तुभूतं प्र‚ति‚{\tiny $_{३}$}‚योगिन‚म्भिन्न‚बुद्धिग्राह्यं राशिद्व‚ये‚{\tiny $_{lb}$}‚ऽव‚स्थाप्याऽन्यापोहः श‚ब्देन चोद्य‚त इत्युच्य‚ते । किन्तु यः श्रोत्रा त‚थाभूतेप्य‚त‚थाभूत‚{\tiny $_{lb}$}‚ आकार आरोप्य‚ते सोपि व्य‚व‚च्छेद्य एव श‚ब्देनेति ।
	{\color{gray}{\rmlatinfont\textsuperscript{§~\theparCount}}}
	\pend% ending standard par
      ‚{\tiny $_{lb}$}‚

	  
	  \pstart \leavevmode% starting standard par
	एत‚देव स्फुट‚य‚न्नाह । \textbf{त‚त्र हीत्या}दि । \textbf{य‚देव मूढ‚म}तेः प्र‚तिपाद्य‚स्या\textbf{शंकास्था}न‚{\tiny $_{lb}$}‚माशंकाविष‚यः । \textbf{त‚दे}व ज्ञेयादिश‚ब्दानां \textbf{निव‚र्त्त्यं} । श्रोत्रा नैव क‚श्चिदाश‚ङ्कि‚{\tiny $_{४}$}‚त‚{\tiny $_{lb}$}‚ इति चेदाह । \textbf{अनाश‚ङ्क‚मानो वे}त्यादि । य‚द्य‚सौ न किञ्चिंदाश‚ङ्क‚ते । य‚थाभूत‚{\tiny $_{lb}$}‚‚{\tiny $_{lb}$}‚ \leavevmode\ledsidenote{\textenglish{246/s}}निश्च‚य‚वान् त‚दा प‚र‚स्माद् व‚क्तुः \textbf{किमुप‚देश‚म‚पेक्ष}ते । नैवेत्य‚भिप्रायः । आकांक्षाप‚{\tiny $_{lb}$}‚न‚य‚नं \textbf{श्रोतृसंस्का}र‚स्त‚द् य‚त्र व‚च‚ने \textbf{ना}स्ति त‚दा श्रोतृसंस्कारं त‚था\textbf{भूतं च व‚च‚नं‚{\tiny $_{lb}$}‚ कुर्वाणो व‚क्ता क‚थं नोन्म‚तः स्या}त् । \textbf{त‚स्}माद् व‚क्ता श्रोतुराकांक्षाव‚तः संस्कार‚{\tiny $_{lb}$}‚मेवाधित्स‚{\tiny $_{५}$}‚मानः श‚ब्दं प्र‚युंक्ते । किं कार‚णं [।] \textbf{त‚त्संस्कारायै}व श्रोतृसंस्कारायैव‚{\tiny $_{lb}$}‚ \textbf{श‚ब्दानां कृत‚संकेत}त्वात् ।
	{\color{gray}{\rmlatinfont\textsuperscript{§~\theparCount}}}
	\pend% ending standard par
      ‚{\tiny $_{lb}$}‚

	  
	  \pstart \leavevmode% starting standard par
	भ‚व‚तु नाम वाक्य‚स्थानां व्य‚व‚हारार्थ‚मुप‚नीतानां ज्ञेयादिश‚ब्दानां य‚थोक्तं‚{\tiny $_{lb}$}‚ व्य‚व‚च्छेद्यं । ये त्व‚व्य‚व‚हारोप‚नीताः केव‚ला एव ज्ञेयादिश‚ब्दास्तेषु क‚थं । न हि‚{\tiny $_{lb}$}‚ त‚त्र प्र‚तिप‚त्तुराश‚ङ्कास्थान‚म‚स्तीत्य‚त आह । \textbf{अव्य‚व‚हारोप‚नीताश्चे}त्यादि । वा‚{\tiny $_{६}$}‚‚{\tiny $_{lb}$}‚ क्येष्व‚न‚न्त‚र्भूतो वाच‚कः श‚ब्दो नास्तीत्य‚र्थः । य‚तो व‚क्ता फ‚लार्थी प्र‚थ‚म‚न्ताव‚{\tiny $_{lb}$}‚दिम‚म‚र्थ‚म्विशिष्ट‚क्रियास‚म्ब‚द्ध‚म‚नेन श‚ब्देनास्मै प्र‚तिपाद‚यामीत्य‚भिप्रायेण देव‚{\tiny $_{lb}$}‚द‚त्त गामान‚ये त्येवं प्र‚युंक्ते । तेन क्रियान्वितानामेव प‚दार्थानाम‚भिधानं । न त्व‚{\tiny $_{lb}$}‚\leavevmode\ledsidenote{\textenglish{91a/PSVTa}} भिहितानाम्प‚दार्थानाम्प‚श्चाद‚न्व‚यः । गामित्यादौ क‚र्मादिविभ‚क्तेर‚नुत्पाद‚प्र‚{\tiny $_{७}$}‚सं‚{\tiny $_{lb}$}‚गात् । त‚स्माद् वाक्य‚स्थानामेव प्र‚योगः । त‚देवाह । \textbf{वाक्य‚ग‚त‚स्येत्या}दि । त‚स्यै\textbf{वार्थ}‚{\tiny $_{lb}$}‚प्र‚तिपाद‚क‚त्वादिति भावः ।
	{\color{gray}{\rmlatinfont\textsuperscript{§~\theparCount}}}
	\pend% ending standard par
      ‚{\tiny $_{lb}$}‚

	  
	  \pstart \leavevmode% starting standard par
	ये तु वै या क र णैः स‚र्व‚विश्वेत्यादिग‚णेषु प‚ठ्य‚न्ते । प्र‚कृतिप्र‚त्य‚य‚विभागेन वा‚{\tiny $_{lb}$}‚ संस्क्रिय‚न्ते । त‚था नि रु क्त कारैः [।] तेपि रेखाग‚व‚य‚स्थानीया वाक्य‚स्थानामेव प्र‚ति‚{\tiny $_{lb}$}‚प‚त्त्युपाया द्र‚ष्ट‚व्या न तु तेषां लौकिकः क‚श्चिद‚र्थोस्ति । त‚स्मात् वाक्य‚स्थानामेव‚{\tiny $_{lb}$}‚ प‚दानाम‚र्थ‚व‚त्ता । त‚त्रैव चाव‚स्थितानाम‚र्थ‚चिन्ता क्रिय‚ते । त‚दाह । \textbf{क्व पुन‚रि}त्यादि ।‚{\tiny $_{lb}$}‚ \textbf{एत} इति ये वाक्य‚स्थाः \textbf{प्र‚योग‚विष‚य‚चिन्तायां} प्र‚व‚र्त्त‚मानाया\textbf{म‚न्यापो}हः श‚ब्दार्थ‚{\tiny $_{lb}$}‚ \textbf{उच्य‚ते} । अन्योऽपोह्य‚तेऽनेनेति कृत्वा । ये त्व‚प्र‚योग‚स्था ज्ञेयादिश‚ब्दास्तेषाम‚र्था‚{\tiny $_{lb}$}‚स‚म्भ‚वाच्चिन्तैव नास्तीत्याह । \textbf{अनिर्दिष्ट‚प्र‚यो}ग‚मित्यादि । निर्दिष्ट उपात्त‚स्त‚था‚{\tiny $_{२}$}‚‚{\tiny $_{lb}$}‚ चासौ प्र‚योग‚श्चेति क‚र्म‚धार‚यः । प‚श्चान्न‚ञा स‚हाभावार्थेऽव्य‚यं विभ‚क्तीत्यादिना‚{\tiny $_{lb}$}‚ऽव्य‚यीभावः । त‚त‚श्च स‚प्त‚म्या\textbf{स्तृतीया स‚प्त‚म्योर्ब‚हुल\edtext{\textsuperscript{*}}{\edlabel{pvsvt_246-1}\label{pvsvt_246-1}\lemma{*}\Bfootnote{\href{http://sarit.indology.info/?cref=P\%C4\%81.2.4.84}{ Pāṇini. }}}मि}त्य‚म्भावः ।‚{\tiny $_{२}$}‚ उपात्त‚प्र‚यो‚{\tiny $_{lb}$}‚गाभावे स‚ति वाक्येनुप‚नीत‚स्य केव‚ल‚स्य \textbf{ज्ञेय‚श‚ब्द‚स्य कोर्थ} इति \textbf{प्र‚श्न} इत्य‚र्थः ।‚{\tiny $_{lb}$}‚ ‚{\tiny $_{lb}$}‚ ‚{\tiny $_{lb}$}‚ \leavevmode\ledsidenote{\textenglish{247/s}}क्रियाविशेष‚ण‚मेत‚दित्य‚प‚रे । प्र‚श्न‚क्रिया हि विशेष्या । क्रियाविशेष‚णानाञ्च क‚र्म‚{\tiny $_{lb}$}‚त्व‚मिति ।‚{\tiny $_{३}$}‚ अत्र च य‚दि क‚र्म्म‚धार‚य‚स‚मास‚स्त‚स्य स्व‚प‚दार्थ‚वृत्तित्वात्क‚थ‚न्तेन‚{\tiny $_{lb}$}‚ प्र‚श्न‚क्रियाविशेषितेति व‚क्त‚व्यं ।
	{\color{gray}{\rmlatinfont\textsuperscript{§~\theparCount}}}
	\pend% ending standard par
      ‚{\tiny $_{lb}$}‚

	  
	  \pstart \leavevmode% starting standard par
	अथानिर्दिष्टः प्र‚योगो य‚स्मिन् ज्ञेय‚श‚ब्द इति ब‚हुब्रीहिस्त‚दापि श‚ब्दो विशेषितो‚{\tiny $_{lb}$}‚ न क्रिया । य‚दा त्वेव‚म्विग्र‚होऽनिर्दिष्टः प्र‚योगो य‚स्मिन् प्र‚श्न इति त‚दा भ‚व‚ति‚{\tiny $_{lb}$}‚ क्रियाविशेषात् त‚दापि प्र‚श्न‚श‚ब्द‚सामानाधिक‚र‚ण्यात् स‚प्त‚म्येव युक्ता‚{\tiny $_{४}$}‚ऽनिर्दिष्ट‚{\tiny $_{lb}$}‚प्र‚योगे प्र‚श्न इति । किङ्कार‚णं केव‚ल‚स्य ज्ञेय‚श‚ब्द‚स्यार्थो नेति चेदाह । \textbf{त‚त} इति‚{\tiny $_{lb}$}‚ [।] त‚तो ज्ञेय‚श‚ब्दात् \textbf{क्व‚चि}द् \textbf{\textbf{अपि न} व‚स्तुप्र‚तिप‚त्तेः} । विधिप्र‚तिषेध‚फ‚ले व्य‚व‚हारे‚{\tiny $_{lb}$}‚ च केव‚ल‚स्य ज्ञेय‚श‚ब्द‚स्य प्र‚योगाभावात् कुतोर्थ‚प्र‚तिप‚त्तिः । य‚दादिश‚ब्दोऽनित्यादि‚{\tiny $_{lb}$}‚रूपेण किं ज्ञेयो भ‚व‚त्य‚थाज्ञेय इत्येवं प्र‚क्रान्ते ज्ञेय इति केव‚लः प्र‚युज्य‚ते । त‚दापि‚{\tiny $_{lb}$}‚ यार्थ‚प्र‚ति‚{\tiny $_{५}$}‚प‚त्तिः सा प्र‚कृतं श‚ब्दाद‚प‚द‚म‚पेक्ष्य भ‚व‚न्ती वाक्यादेव जाय‚ते । प‚दान्त‚र‚{\tiny $_{lb}$}‚स‚हित‚स्य प‚द‚स्य वाक्य‚त्वात् । त‚स्मान्नास्ति प‚दान्त‚र‚निर‚पेक्षात् प‚दार्थ‚प्र‚तिप‚त्तिः ।‚{\tiny $_{lb}$}‚ य‚था ज्ञेयादिप‚दानां केव‚लानां न किंचिद्वाच्यं \textbf{त‚था घ‚टादिश‚ब्दानाम‚पि केव‚लानां} ।
	{\color{gray}{\rmlatinfont\textsuperscript{§~\theparCount}}}
	\pend% ending standard par
      ‚{\tiny $_{lb}$}‚

	  
	  \pstart \leavevmode% starting standard par
	न‚नु च किं घ‚टेनोद‚क‚मान‚याम्य‚थाञ्ज‚लिनेति प्र‚स्तावे । घ‚टेनेति प्र‚युंक्ते ।‚{\tiny $_{lb}$}‚ त‚त्र च यः प्र‚क‚र‚णं‚{\tiny $_{६}$}‚ न ज्ञात‚वान् त‚स्यापि प्र‚तिप‚त्तुर्घ‚टेनेति केव‚ल‚श‚ब्द‚श्र‚व‚णाद् घ‚टा‚{\tiny $_{lb}$}‚कारा प्र‚तिप‚त्तिरुत्प‚द्य‚त एवेति क‚थ‚मुच्य‚ते केव‚लाच्छ‚ब्दात् न प्र‚तिप‚त्तिरित्याह ।‚{\tiny $_{lb}$}‚ \textbf{यापी}त्यादि । अप‚रिस‚माप्तः स जिज्ञासितोर्थो य‚स्यां प्र‚तिप‚त्तौ सा\textbf{ऽप‚रिस‚माप्त‚{\tiny $_{lb}$}‚त‚द‚र्था} । क‚थ‚म‚प‚रिस‚माप्त‚त‚द‚र्थ‚तेत्याह । \textbf{दृष्ट‚प्र‚योगानुसारेणे}ति । याव‚त्सु न‚य‚ना‚{\tiny $_{lb}$}‚न‚य‚नादिक्रियाचोद‚ना‚{\tiny $_{१}$}‚प्र‚वृत्तेषु । तेन घ‚ट‚श‚ब्द‚स्य \textbf{प्र‚योगो दृष्ट‚स्त‚द‚नुसारेण} ताव‚त्सु‚{\tiny $_{lb}$}‚ पूर्व‚वाक्येष्वाकांक्षाव‚ती प्र‚तीतिर्भ‚व‚ति किम‚य‚म‚र्थो विव‚क्षितः किम्वाय‚मित्येवं‚{\tiny $_{lb}$}‚ \textbf{साकांक्ष‚त्वादु}प‚प्ल‚व‚मानं रूप‚त्वेनास‚माप्तार्था विप्ल‚व‚भ्रान्तिरेव । एत‚त्क‚थ‚य‚ति [।]‚{\tiny $_{lb}$}‚ नैव केव‚ल‚श‚ब्द‚मात्र‚श्र‚व‚णाद‚र्थ‚प्र‚तिप‚त्तिर‚स्ति किन्तु वाक्येषूप‚ल‚ब्ध‚स्यार्थ‚व‚तः प‚द‚स्य‚{\tiny $_{lb}$}‚ सादृश्येनो‚{\tiny $_{२}$}‚प‚हृत‚बुद्धेः केव‚ल‚श‚ब्द‚श्र‚व‚णाद‚र्थ‚प्र‚तिप‚त्त्य‚भिमान इति । य‚था दृष्ट‚प्र‚यो‚{\tiny $_{lb}$}‚गानुसारेण केव‚ल\textbf{घ‚टादिप‚द‚श्र‚व‚णाद‚र्थ}प्र‚तिप‚त्ति\textbf{र्विप्ल‚व‚स्तादृशो ज्ञेयादिश‚ब्देष्व‚पि [।]‚{\tiny $_{lb}$}‚ य‚थाद‚र्श‚नं} । य‚थाप्र‚योगोप‚ल‚म्भं । याव‚त्सु वाक्येषु ज्ञेय‚श‚ब्दः प्र‚युज्य‚मानो दृष्ट‚{\tiny $_{lb}$}‚स्त‚द‚नुसारेण केव‚ल‚ज्ञेय‚श‚ब्द‚श्र‚व‚णाद\textbf{स्त्येवा}र्थ‚प्र‚तिप‚त्तिविप्ल‚वः । अनेन स‚{\tiny $_{३}$}‚र्व‚था घ‚टा‚{\tiny $_{lb}$}‚दिश‚ब्दैर्ज्ञेयादिश‚ब्दानान्तुल्य‚तामाह ।
	{\color{gray}{\rmlatinfont\textsuperscript{§~\theparCount}}}
	\pend% ending standard par
      ‚{\tiny $_{lb}$}‚

	  
	  \pstart \leavevmode% starting standard par
	य‚च्चाप्युक्त‚म् [।] एकादिव्युदासेन प्र‚व‚र्त्त‚मानः स‚र्व‚श‚ब्दाङ्गे प्र‚तिषेधाद‚ङ्ग‚व्य‚{\tiny $_{lb}$}‚‚{\tiny $_{lb}$}‚ \leavevmode\ledsidenote{\textenglish{248/s}}तिरिक्त‚स्य चाङ्गिनोन‚भ्युप‚ग‚माद‚न‚र्थ‚कः स्यादि ति ।
	{\color{gray}{\rmlatinfont\textsuperscript{§~\theparCount}}}
	\pend% ending standard par
      ‚{\tiny $_{lb}$}‚

	  
	  \pstart \leavevmode% starting standard par
	त‚द‚युक्तं । य‚तोन्य एवैकादिबुद्धिविष‚याभावा अन्ये च स‚मुदायादिबुद्धिविष‚याः‚{\tiny $_{lb}$}‚ प्र‚तिभास‚न्ते । ये च विशिष्टाव‚स्थाः स‚मुदायादिबुद्धिविष‚यास्त‚{\tiny $_{४}$}‚ एवाङ्गिन उच्य‚{\tiny $_{lb}$}‚न्तेन्य‚स्याङ्गिनो निषेधात् । यादृग्भूताश्च ते प‚रेण स‚मुदायादिध‚र्मार‚म्भ‚का‚{\tiny $_{lb}$}‚ इष्य‚न्ते तादृग्भूता एवास्माभिः स‚मुदाय‚बुद्धिज‚न‚क‚त्वेन त‚दाल‚म्ब‚ना इष्य‚न्ते विरोधा‚{\tiny $_{lb}$}‚भावात् । तेन स‚र्व‚स‚मुदाय‚द्वित्वादिश‚ब्दानामेकादिनिषेधो घ‚ट‚त एव ।
	{\color{gray}{\rmlatinfont\textsuperscript{§~\theparCount}}}
	\pend% ending standard par
      ‚{\tiny $_{lb}$}‚

	  
	  \pstart \leavevmode% starting standard par
	\textbf{त‚स्मा}दित्यादि । य‚तः स‚र्वं वाक्यं साव‚धार‚ण‚म्वाक्य‚स्थानामेव प‚दाना‚{\tiny $_{५}$}‚म‚र्थ‚{\tiny $_{lb}$}‚व‚त्ता । \textbf{त‚स्मा}त् \textbf{स‚र्व‚श‚ब्द‚प्र‚यो}ग इत्युप‚संहारः । \textbf{त‚त्साफ‚ल्यात्} त‚स्य श‚ब्द‚प्र‚योग‚स्य‚{\tiny $_{lb}$}‚ साफ‚ल्यात् । एवं स‚र्व‚श‚ब्दानां य‚थोक्त‚विधिनाऽन्यापोहे वाच्ये ।
	{\color{gray}{\rmlatinfont\textsuperscript{§~\theparCount}}}
	\pend% ending standard par
      ‚{\tiny $_{lb}$}‚

	  
	  \pstart \leavevmode% starting standard par
	य‚दुक्त‚म्भ‚{\tiny $_{१}$}‚ ट्टो द्यो त क रा भ्यां ।
	{\color{gray}{\rmlatinfont\textsuperscript{§~\theparCount}}}
	\pend% ending standard par
      ‚{\tiny $_{lb}$}‚
	  \bigskip
	  \begingroup
	
	    
	    \stanza[\smallbreak]
	  {\normalfontlatin\large ``\qquad}अन्यापोह‚श्च किम्वाच्यः किम्वाऽवाच्योय‚मिष्य‚ते ।&‚{\tiny $_{lb}$}‚वाच्योपि विधिरूपेण य‚दि वान्य‚निषेध‚तः ॥&‚{\tiny $_{lb}$}‚\leavevmode\ledsidenote{\textenglish{92a/PSVTa}}विध्यात्म‚नास्य वाच्य‚त्वे त्याज्य‚मेकान्त‚द‚र्श‚नं ।&‚{\tiny $_{lb}$}‚स‚र्व‚त्रान्य‚निषेधोयं श‚ब्दार्थ इति व‚र्ण्णितं ॥&‚{\tiny $_{lb}$}‚अन‚पोह‚व्युदासेन य‚द्य‚पोहोभिधीय‚ते ।&‚{\tiny $_{lb}$}‚त‚त्र त‚त्रैव‚मिच्छायाम‚न‚व‚स्था भ‚वेत्त‚व ॥&‚{\tiny $_{lb}$}‚अथाप्य‚वाच्य एवाय‚म‚न्यापोह‚स्त्व‚येष्य‚ते ।&‚{\tiny $_{lb}$}‚तेनान्यापोह‚कृच्छ‚ब्द इति बाध्येत ते व‚चः ॥&‚{\tiny $_{lb}$}‚य‚स्माद् येष्वेव श‚ब्देषु न‚ञ् योग‚स्तेषु केव‚लः ।&‚{\tiny $_{lb}$}‚भ‚वेद‚न्य‚निवृत्त्यंश‚स्स्वात्मैवान्य‚त्र ग‚म्य‚त इति ॥{\normalfontlatin\large\qquad{}"}\&[\smallbreak]
	  
	  
	  
	  \endgroup
	‚{\tiny $_{lb}$}‚

	  
	  \pstart \leavevmode% starting standard par
	त‚द‚युक्तं ।‚{\tiny $_{१}$}‚ विधेः श‚ब्दार्थ‚स्यार्थाद‚न्य‚निषेध‚स्याभ्युप‚ग‚मात् ।
	{\color{gray}{\rmlatinfont\textsuperscript{§~\theparCount}}}
	\pend% ending standard par
      ‚{\tiny $_{lb}$}‚

	  
	  \pstart \leavevmode% starting standard par
	य‚दि त‚र्हि विधिरेव श‚ब्दार्थोर्थाद‚न्य‚निषेधः [।] क‚थ‚न्त‚ह्याचार्य दि ङ् ना गे न‚{\tiny $_{lb}$}‚ श‚ब्दोर्थान्त‚र‚व्यावृत्तिविशिष्टानेव भावानाहेत्याद्युक्तं [।]
	{\color{gray}{\rmlatinfont\textsuperscript{§~\theparCount}}}
	\pend% ending standard par
      ‚{\tiny $_{lb}$}‚

	  
	  \pstart \leavevmode% starting standard par
	न विरुध्य‚त इत्य‚त आह । \textbf{निवेश‚नं चे}त्यादि । अनेन चैत‚द्द‚र्श‚य‚ति [।] संके‚{\tiny $_{lb}$}‚तेपि ताव‚द् विधिरूपेण श‚ब्दः प्र‚व‚र्त्त‚ते किं पुन‚र्व्य‚व‚हार इति । \textbf{यो} वृक्षा‚{\tiny $_{२}$}‚र्थो \textbf{य‚स्मा}‚{\tiny $_{lb}$}‚\edtext{}{\lemma{र्थो}\Bfootnote{\href{http://sarit.indology.info/?cref=\%C5\%9Bv}{ Ślokavārtika. }}}‚{\tiny $_{lb}$}‚ ‚{\tiny $_{lb}$}‚ \leavevmode\ledsidenote{\textenglish{249/s}}द‚वृक्षाद् घ‚टादे\textbf{र्भिद्य}ते \textbf{विनिव‚र्त्त्येत} स वृक्ष‚न्त्य‚क्त्वेत्य‚र्थः । \textbf{निवेश‚नं} संकेत‚क‚र‚णं‚{\tiny $_{lb}$}‚ व‚क्ष‚श‚ब्द‚स्य द्र‚ष्ट‚व्यं । \textbf{त‚म्विनिवृत्त्}येत्य‚नेनावृक्षे वृक्ष‚श‚ब्दो न संकेत्य‚त इत्युक्त‚म्भ‚{\tiny $_{lb}$}‚व‚ति । क्व पुन‚स्त‚न्निवेश‚न‚मित्याह । \textbf{भेदे भिद्य‚मानानां} वृक्षाणां य‚स्त‚द्भेद‚स्त‚स्मा‚{\tiny $_{lb}$}‚द‚वृक्षाद् भेदः । अवृक्षाद् भिन्नः स्व‚भावः । य‚थैव ह्येको वृक्ष‚विशेष‚स्त‚{\tiny $_{३}$}‚स्माद् वृक्षाद्‚{\tiny $_{lb}$}‚ भिन्न‚स्त‚था स‚र्वे वृक्ष‚भेदाः ।
	{\color{gray}{\rmlatinfont\textsuperscript{§~\theparCount}}}
	\pend% ending standard par
      ‚{\tiny $_{lb}$}‚

	  
	  \pstart \leavevmode% starting standard par
	त‚त‚स्तेष्व‚सौ त‚द्भेदो विक‚ल्प‚बुद्ध्या स‚र्वेष्वेक‚त्वेनारोप्य‚त इति स‚मानाकार‚{\tiny $_{lb}$}‚भासी भ‚व‚ति । त‚स्मिन् \textbf{स‚मानाकार‚भासिनि} त‚द्भेदे भिन्न‚स्व‚भावे निवेश‚नं श‚ब्द‚स्य ।‚{\tiny $_{lb}$}‚ स \textbf{चाय‚मिति} च‚श‚ब्दोव‚धार‚णार्थः । स एवाय‚म्विक‚ल्प‚प्र‚तिभास्याकारो बाह्या‚{\tiny $_{lb}$}‚भिन्नः ।
	{\color{gray}{\rmlatinfont\textsuperscript{§~\theparCount}}}
	\pend% ending standard par
      ‚{\tiny $_{lb}$}‚

	  
	  \pstart \leavevmode% starting standard par
	य‚द्वा स एव त‚द्भेदोऽवृक्षाद् भिन्न‚स्व‚भाव‚ल‚क्ष‚{\tiny $_{४}$}‚णः प्रोक्त आचार्य दि ङ् ना गे न ।‚{\tiny $_{lb}$}‚ क‚थं \textbf{प्रोक्त} इत्याह । \textbf{अन्य‚व्यावृत्त्या ग‚म्य‚ते त‚स्य व‚स्तुनः क‚श्चिद् भा}ग इत्य‚नेन ।‚{\tiny $_{lb}$}‚ अस्य चार्थ‚म्वृत्तौ व्य‚क्तीक‚रिष्यामः ।
	{\color{gray}{\rmlatinfont\textsuperscript{§~\theparCount}}}
	\pend% ending standard par
      ‚{\tiny $_{lb}$}‚

	  
	  \pstart \leavevmode% starting standard par
	न‚न्व‚न्य‚व्यावृत्त्या विशिष्टो व‚स्तुभागः प‚र‚मार्थ‚त एव क‚स्मान्न गृह्य‚त इत्याह ।‚{\tiny $_{lb}$}‚ \textbf{रूपं स्व‚भावो नास्यापि} भेद‚स्य \textbf{किञ्च‚न} निवृत्तिरूप‚स्य भेद‚स्यास‚त्त्वात् । विक‚ल्प‚{\tiny $_{lb}$}‚प्र‚तिभासिन‚श्च बुद्धिवि‚{\tiny $_{५}$}‚भ्र‚मात् ।
	{\color{gray}{\rmlatinfont\textsuperscript{§~\theparCount}}}
	\pend% ending standard par
      ‚{\tiny $_{lb}$}‚

	  
	  \pstart \leavevmode% starting standard par
	य‚दि भेद‚स्य न रूपं किञ्च‚न क‚थ‚न्त‚र्हि श‚ब्दोन्य‚व्यावृत्तिविशिष्टानेव भावाना‚{\tiny $_{lb}$}‚हेत्युच्य‚त इत्य‚त आह । \textbf{त‚द्ग‚तावि}त्यादि । त‚स्य य‚थोक्त‚स्य भेद‚स्य विजातीय‚{\tiny $_{lb}$}‚व्यावृत्त‚स्य स्व‚भाव‚स्य विधिरूपेण ग‚तावेव \textbf{श‚ब्देभ्यो ग‚म्य‚तेन्य‚निव‚र्त्त‚नं} । \href{http://sarit.indology.info/?cref=pv.3.124}{। १२७ ।}
	{\color{gray}{\rmlatinfont\textsuperscript{§~\theparCount}}}
	\pend% ending standard par
      ‚{\tiny $_{lb}$}‚

	  
	  \pstart \leavevmode% starting standard par
	त‚था हि वृक्ष इत्युक्तेऽर्थाद‚वृक्ष‚निव‚र्त्त‚नं प्र‚तीय‚ते । एताव‚न्मात्रेण चान्य‚व्यावृ‚{\tiny $_{६}$}‚‚{\tiny $_{lb}$}‚त्तिविशिष्ट‚त्व‚मुक्तं । न तु प‚र‚मार्थ‚तो विशेष‚ण‚विशेष्य‚भावः । त‚दाह । \textbf{नेत्या}दि ।‚{\tiny $_{lb}$}‚ \textbf{त‚त्रे}त्य‚न्यापोहे श‚ब्दार्थे आ चा र्य ग्र‚न्थे वा । \textbf{क‚श्चि}त् \textbf{प‚र} इत्य‚न्य‚स्माद् व्यावृत्तोर्थः ।‚{\tiny $_{lb}$}‚ \textbf{केन‚चि}द‚न्य‚व्यावृत्तिल‚क्ष‚णेन \textbf{विशिष्टो न ग‚म्य‚त} इति ।
	{\color{gray}{\rmlatinfont\textsuperscript{§~\theparCount}}}
	\pend% ending standard par
      ‚{\tiny $_{lb}$}‚

	  
	  \pstart \leavevmode% starting standard par
	तेन य‚दुच्य‚ते भ ट्टे न ।
	{\color{gray}{\rmlatinfont\textsuperscript{§~\theparCount}}}
	\pend% ending standard par
      ‚{\tiny $_{lb}$}‚
	  \bigskip
	  \begingroup
	
	    
	    \stanza[\smallbreak]
	  {\normalfontlatin\large ``\qquad}न चासाधार‚ण‚म्व‚स्तु ग‚म्य‚तेऽपोह‚व‚त्त‚या ।&‚{\tiny $_{lb}$}‚क‚थं वा प‚रिक‚ल्प्येत स‚म्ब‚न्धो‚{\tiny $_{७}$}‚ व‚स्त्व‚व‚स्तुनोः ॥&\leavevmode\ledsidenote{\textenglish{92b/PSVTa}}‚{\tiny $_{lb}$}‚स्व‚रूप‚स‚त्त्व‚मात्रेण न स्यात् किञ्चिद्विशेष‚णं ।&‚{\tiny $_{lb}$}‚‚{\tiny $_{lb}$}‚\leavevmode\ledsidenote{\textenglish{250/s}}स्व‚बुद्ध्या र‚ज्य‚ते येन विशेष्य‚न्त‚द्विशेष‚णं ॥&‚{\tiny $_{lb}$}‚न चाप्य‚श्वादिश‚ब्देभ्यो जाय‚तेऽपोह‚बोध‚नं ।&‚{\tiny $_{lb}$}‚विशेष्य‚बुद्धिरिष्टेह न चाज्ञात‚विशेष‚णा ॥&‚{\tiny $_{lb}$}‚न चान्य‚रूप‚म‚न्यादृक् कुर्याज्ज्ञान‚विशेष‚णं ।&‚{\tiny $_{lb}$}‚क‚थं चान्यादृशे ज्ञाने त‚दुच्येत विशेष‚णं ।&‚{\tiny $_{lb}$}‚अभाव‚ग‚म्य‚रूपे च न विशेष्येस्ति व‚स्तुता ।&‚{\tiny $_{lb}$}‚विशेषित‚म‚पो‚{\tiny $_{१}$}‚हेन व‚स्तु वाच्यं न तेस्त्य‚त ॥\edtext{}{\edlabel{pvsvt_250-1}\label{pvsvt_250-1}\lemma{त}\Bfootnote{\href{http://sarit.indology.info/?cref=\%C5\%9Bv}{ Ślokavārtika. }}}{\normalfontlatin\large\qquad{}"}\&[\smallbreak]
	  
	  
	  
	  \endgroup
	‚{\tiny $_{lb}$}‚

	  
	  \pstart \leavevmode% starting standard par
	इत्य‚पास्तं ।
	{\color{gray}{\rmlatinfont\textsuperscript{§~\theparCount}}}
	\pend% ending standard par
      ‚{\tiny $_{lb}$}‚

	  
	  \pstart \leavevmode% starting standard par
	न‚न्वेक‚स्य श‚ब्द‚स्य क‚थं विधिप्र‚तिषेध‚ल‚क्ष‚णं व्यापार‚द्व‚य‚म् [।] आह ।
	{\color{gray}{\rmlatinfont\textsuperscript{§~\theparCount}}}
	\pend% ending standard par
      ‚{\tiny $_{lb}$}‚

	  
	  \pstart \leavevmode% starting standard par
	\textbf{न चापि श‚ब्दो द्व‚य‚कृ}त् । स्वार्थाभिधान‚म‚न्य‚व्याव‚र्त्त‚नं च द्व‚यं क‚रोति [।]‚{\tiny $_{lb}$}‚ किङ्कार‚ण‚म् [।] \textbf{अन्योन्याभाव इति} । इतिश‚ब्दो हेतौ । य‚स्माद‚वृक्ष‚भेदाभावो‚{\tiny $_{lb}$}‚ वृक्षार्थ‚स्त‚द‚भाव‚श्चावृक्षार्थ इत‚रेत‚राभाव‚त्वेन । त‚स्माद् वृक्ष‚श‚ब्दाद् वृक्षार्थ‚प्र‚ति‚{\tiny $_{lb}$}‚प‚त्त्यैवार्था‚{\tiny $_{२}$}‚द् अवृक्ष‚निवृत्तिप्र‚तिप‚त्तिर‚पि भ‚व‚तीति न द्वौ व्यापारौ साक्षाच्छ‚ब्द‚{\tiny $_{lb}$}‚स्य । तेन ।‚{\tiny $_{lb}$}‚ 
	    \pend% close preceding par
	  
	    
	    \stanza[\smallbreak]
	  {\normalfontlatin\large ``\qquad}य‚दि गौरित्य‚यं श‚ब्दः स‚म‚र्थोन्य‚निव‚र्त्त‚ने ।&‚{\tiny $_{lb}$}‚ज‚न‚को ग‚वि गोबुद्धेर्मृग्य‚ताम‚प‚रो ध्व‚निः ॥{\normalfontlatin\large\qquad{}"}\&[\smallbreak]
	  
	  
	  ‚{\tiny $_{lb}$}‚ 
	    
	    \stanza[\smallbreak]
	  {\normalfontlatin\large ``\qquad}न च ज्ञान‚फ‚लाः श‚ब्दा न चैक‚स्य फ‚ल‚द्व‚यं ।&‚{\tiny $_{lb}$}‚अप‚वाद‚विधिज्ञानं फ‚ल‚मेक‚स्य वः क‚थ‚म् [।]{\normalfontlatin\large\qquad{}"}\&[\smallbreak]
	  
	  
	  
	    \pstart  \leavevmode% new par for following
	    \hphantom{.}
	  ‚{\tiny $_{lb}$}‚ इति निर‚स्तं ।
	{\color{gray}{\rmlatinfont\textsuperscript{§~\theparCount}}}
	\pend% ending standard par
      ‚{\tiny $_{lb}$}‚

	  
	  \pstart \leavevmode% starting standard par
	य‚दि श‚ब्द‚वाच्यो भेद‚स्स‚र्व‚त्रानुयायी त‚देव त‚र्हि पार‚मार्थिकं सामा‚{\tiny $_{३}$}‚न्य‚म्भ‚विष्य‚{\tiny $_{lb}$}‚तीत्य‚त आह । \textbf{असावि}ति श‚ब्द‚विष‚योनुयायी भेदः \textbf{अरूपो} निःस्व‚भावः । त‚स्मिन्न‚{\tiny $_{lb}$}‚रूपे दृश्य‚विक‚ल्प‚योरेकीकृत्य व‚क्तृश्रोत्रोर्य\textbf{द्रूप‚व‚त्त्वेन द‚र्श‚न‚न्त‚द्बुद्धिविप्ल‚वो} भ्रान्ति‚{\tiny $_{lb}$}‚रित्य‚र्थः ।
	{\color{gray}{\rmlatinfont\textsuperscript{§~\theparCount}}}
	\pend% ending standard par
      ‚{\tiny $_{lb}$}‚

	  
	  \pstart \leavevmode% starting standard par
	\textbf{निवेश्य‚मान} इत्यादिना व्याच‚ष्टे । य‚स्माद् भिद्य‚ते \textbf{वृक्षादिकोर्थ‚स्त‚न्निव‚र्त्त्य}‚{\tiny $_{lb}$}‚ ‚{\tiny $_{lb}$}‚ ‚{\tiny $_{lb}$}‚ \leavevmode\ledsidenote{\textenglish{251/s}}त‚त्त्य‚क्त्वा निवेश्य‚त इति स‚म्ब‚न्धः । कुत्रेत्याह । \textbf{भिद्य‚मानाना}मित्यादि । एत‚च्च‚{\tiny $_{lb}$}‚ का रि का व्याख्याने विभ‚क्तार्थ । \textbf{आक्षिप्ता} त‚द‚न्य‚व्या\textbf{वृत्ति}र्येन श‚ब्देनेति‚{\tiny $_{lb}$}‚ विग्र‚हः । स एव चायं \textbf{स‚मान‚रूप‚प्र‚तिभासी भेदो निर्द्दिष्ट} आचार्य दि ङ् ना गे न ।‚{\tiny $_{lb}$}‚ क‚थ‚मित्याह । \textbf{अर्थान्त‚र‚व्यावृत्त्या त‚स्य व‚स्तुनः क‚श्चिद् भागो ग‚म्य‚त इति । त‚था‚{\tiny $_{lb}$}‚ श‚ब्दोर्थान्त‚र‚निवृत्तिविशिष्टानेव भावानाहेत्या}दिना । आदिग्र‚ह‚णाच्छ‚ब्दार्थान्त‚रा‚{\tiny $_{lb}$}‚पोहं कुर्व‚ती श्रुतिर‚भिध‚{\tiny $_{५}$}‚त्त इत्यादि प‚रिग्र‚हः ।
	{\color{gray}{\rmlatinfont\textsuperscript{§~\theparCount}}}
	\pend% ending standard par
      ‚{\tiny $_{lb}$}‚

	  
	  \pstart \leavevmode% starting standard par
	\textbf{स ही}त्याद्य‚स्यैव स‚म‚र्थ‚नं । \textbf{स हि} वृक्ष‚श‚ब्द‚स्त\textbf{म्भेद}म‚वृक्ष‚व्यावृत्तं स्व‚भावं \textbf{क‚थ}‚{\tiny $_{lb}$}‚य‚न्नार्थान्त‚र‚स्यावृक्षार्थ‚स्य \textbf{व्य‚व‚च्छेद‚माक्षिप‚न्नेव व‚र्त्त‚ते} । किं कार‚ण‚म् [।] \textbf{एक‚ग‚त‚{\tiny $_{lb}$}‚भेद}स्येत्यादि । य‚था हि वृक्ष‚भेदा एव ख‚दिराद‚यः स्व‚भावेनैवावृक्षेभ्यो भिन्ना एव‚म‚{\tiny $_{lb}$}‚वृक्षा अपि वृक्षेभ्यः भेद‚स्य द्विष्ठ‚त्वात् । त‚त्र वृक्ष‚श‚ब्देनैक‚ग‚त‚स्य वृक्षा‚{\tiny $_{६}$}‚र्थ‚ग‚त‚स्य‚{\tiny $_{lb}$}‚ \textbf{भेद‚स्य} भिन्न‚स्य स्व‚भाव‚स्यैकाकार‚प्र‚तिभासिनो या \textbf{चोद‚ना त‚स्यास्त‚द‚न्य‚व्यावृत्ति‚{\tiny $_{lb}$}‚ नान्त‚रीय‚क‚त्वात्} । त‚स्माद् वृक्षार्थाद‚न्य‚स्यावृक्ष‚स्य या व्यावृत्तिस्त‚न्नान्त‚रीय‚क‚{\tiny $_{lb}$}‚त्वात् । एवं ह्य‚वृक्षाद् व्यावृत्त‚रूपो वृक्षार्थोऽभिहितः स्याद् य‚द्य‚वृक्षार्थ‚स्य‚{\tiny $_{lb}$}‚ त‚त्र निवृत्तिर्ग‚म्येत । \textbf{स एवा}न्य‚स्माद् भिद्य‚मान‚स्य विक‚ल्प‚बुद्धिप्र‚तिभासी \textbf{भेदो}‚{\tiny $_{lb}$}‚ भि‚{\tiny $_{७}$}‚न्नः स्व‚भावः । \textbf{त‚द्व्यावृत्त्}याऽर्थान्त‚र‚व्यावृत्त्या य‚थोक्त‚न्यायेनार्थाद् ग‚म्य‚मा- \leavevmode\ledsidenote{\textenglish{93a/PSVTa}}‚{\tiny $_{lb}$}‚ न‚या । \textbf{ग‚तो च} बुद्धो \textbf{भा}गो व‚स्तुन इत्युक्त इत्य‚ध्याहारः । व्यावृत्त‚व‚स्तुद‚र्श‚न‚{\tiny $_{lb}$}‚द्वारायात‚त्वाद् व‚स्तुरूप‚त्वेनाध्य‚व‚सायाच्च व‚स्तुभागो ग‚त इत्युच्य‚त इत्य‚भिप्रायः ।
	{\color{gray}{\rmlatinfont\textsuperscript{§~\theparCount}}}
	\pend% ending standard par
      ‚{\tiny $_{lb}$}‚

	  
	  \pstart \leavevmode% starting standard par
	एत‚दुक्त‚म्भ‚व‚ति [।] अत‚त्प‚राम‚र्श‚ज‚न‚नेभ्यो व्यावृत्त‚रूप‚न्त‚त्प‚राम‚र्श‚ज‚न‚नेष्वा‚{\tiny $_{lb}$}‚रोपितैक‚त्वं विक‚ल्प‚बुद्धिप्र‚{\tiny $_{१}$}‚तिभास‚न‚मेवाकार‚म‚विभ‚क्त‚बाह्याध्यात्मिक‚भेदं श‚ब्दः‚{\tiny $_{lb}$}‚ प्र‚तिपाद‚य‚ति शाब्दे ज्ञाने त‚स्यैव प्र‚तिभास‚नात् । त‚ञ्च प्र‚तिपाद‚य‚न्न‚न्य‚व्यावृत्ति‚{\tiny $_{lb}$}‚म‚र्थादाक्षिप‚ति [।] अतोनेनाभिप्रायेणोक्त‚मा चा र्ये णान्य‚व्यावृत्त्या ग‚म्य‚ते त‚स्य‚{\tiny $_{lb}$}‚ व‚स्तुनः क‚श्चिद् भाग इति । न पुन‚र‚न्य‚व्यावृत्त्या श‚ब्दार्थ‚भूत‚या विशेष‚ण‚रूप‚या‚{\tiny $_{lb}$}‚ बाह्य‚स्य व‚स्तुनः क‚श्चिद् भागो गृह्य‚तेऽन्य‚व्या‚{\tiny $_{२}$}‚वृत्तेरेवाभावादिति । एव‚न्ताव‚द‚न्य‚{\tiny $_{lb}$}‚व्यावृत्त्या ग‚म्य‚ते त‚स्य व‚स्तुनः क‚श्चिद् भाग इत्येत‚त्स‚म‚र्थितं ।
	{\color{gray}{\rmlatinfont\textsuperscript{§~\theparCount}}}
	\pend% ending standard par
      ‚{\tiny $_{lb}$}‚

	  
	  \pstart \leavevmode% starting standard par
	अधुना श‚ब्दार्थान्त‚र‚निवृत्तिविशिष्टानेव भावानाहेत्येत‚त्स‚म‚र्थ‚य‚न्नाह । \textbf{त‚द्‚{\tiny $_{lb}$}‚ग‚ते}रित्यादि । \textbf{त‚द्ग‚ते}र‚न्य‚निवृत्तिग‚तेस्\textbf{त‚दुपाधित्वाद}स्ति भेदोपाधित्वात् [।] स एव‚{\tiny $_{lb}$}‚ ‚{\tiny $_{lb}$}‚ \leavevmode\ledsidenote{\textenglish{252/s}}\textbf{भेद‚स्त‚द्विशिष्टो ग‚त इत्युच्य}ते । आ चा र्ये णेति ।
	{\color{gray}{\rmlatinfont\textsuperscript{§~\theparCount}}}
	\pend% ending standard par
      ‚{\tiny $_{lb}$}‚

	  
	  \pstart \leavevmode% starting standard par
	य‚द्वा [।] त‚स्य य‚थोक्त‚स्य व‚स्तुभेद‚स्य श‚ब्दाद् \textbf{ग‚तेः} प्र‚तिप‚त्तेः साव‚धार‚ण‚त्वेन‚{\tiny $_{lb}$}‚ \textbf{त‚दुपा‚{\tiny $_{३}$}‚धित्वात्} । सा त‚द‚न्य‚निवृत्तिरुपाधिर‚र्थाक्षेपाद् य‚स्यास्त‚द्ग‚तेः सा त‚थोक्ता ।‚{\tiny $_{lb}$}‚ त‚द‚न्य‚निवृत्तिनान्त‚रीय‚क‚त्व‚मेव त‚दुपाधित्वं ।
	{\color{gray}{\rmlatinfont\textsuperscript{§~\theparCount}}}
	\pend% ending standard par
      ‚{\tiny $_{lb}$}‚

	  
	  \pstart \leavevmode% starting standard par
	एत‚दुक्त‚म्भ‚व‚ति । यार्थान्त‚र‚व्यावृत्तिर‚र्थाद् ग‚म्य‚ते त‚न्नान्त‚रीय‚क‚त्वात्स एव‚{\tiny $_{lb}$}‚ भेदः श‚ब्दाद् ग‚म्य‚मानोन्य‚व्यावृत्त्या विशिष्टो ग‚त इत्युच्य‚त आ चा र्ये णेति ।
	{\color{gray}{\rmlatinfont\textsuperscript{§~\theparCount}}}
	\pend% ending standard par
      ‚{\tiny $_{lb}$}‚

	  
	  \pstart \leavevmode% starting standard par
	न‚न्व\textbf{र्थान्त‚र‚व्यावृत्तिः} श‚ब्द‚प्र‚वृत्तिनिमित्त‚भूता । \textbf{य‚या विशिष्टा} बाह्या अर्था‚{\tiny $_{४}$}‚‚{\tiny $_{lb}$}‚ ग‚वादि\textbf{श‚ब्दैश्चोद्य‚न्ते} अप्र‚तीतेः । \textbf{द‚ण्डिव}दिति वैध‚र्म्य‚दृष्टान्तः । य‚था दंण्ड‚द्वारेण‚{\tiny $_{lb}$}‚ त‚द्वान् द‚ण्डीत्युच्य‚ते । नैवं व्यावृत्तिद्वारेण व्यावृत्तिमानिति । क‚स्माद् व्यावृत्ति‚{\tiny $_{lb}$}‚र‚र्थान्त‚र‚भूता नेत्याह । \textbf{द्व‚योर्ही}त्यादि । य‚स्माद् द्व‚योर्वृक्षावृक्ष‚योः प‚र‚स्प‚र\textbf{म्भिद्य‚{\tiny $_{lb}$}‚मान‚यो}र्यो \textbf{भेद}स्त‚स्यो\textbf{भ‚य‚ग‚त‚त्वा}त् । वृक्षावृक्ष‚ग‚त‚त्वाद् वृक्षावृक्ष‚भिन्न‚स्व‚भाव‚त्वादि‚{\tiny $_{lb}$}‚त्य‚र्थः । तेनै\textbf{क‚ग‚{\tiny $_{५}$}‚त‚भेदाभिधानेपि} । अवृक्षापेक्ष‚या वृक्ष‚ग‚तो यो भेदः । भिन्नः‚{\tiny $_{lb}$}‚ स्व‚भाव‚स्त‚स्य श‚ब्देनाभिधानेपि \textbf{नान्त‚रीय‚क‚स्त‚द‚न्याक्षेपो भ‚व‚ति} । अवृक्षापेक्ष‚या‚{\tiny $_{lb}$}‚ वृक्ष‚ग‚त‚स्य भेद‚स्यावृक्ष‚निवृत्तिल‚क्ष‚ण‚स्य वृक्षाप‚क्ष‚याप्य‚वृक्ष‚ग‚त‚स्य वृक्ष‚निवृत्तिल‚क्ष‚ण‚{\tiny $_{lb}$}‚स्याक्षेपो भ‚व‚ति । \textbf{इति}श‚ब्दो हेतौ । अस्माद्धेतोस्त‚दुपाधित्वात् त‚द्विशिष्टो ग‚त‚{\tiny $_{lb}$}‚ इत्युक्त‚मित्य‚ध्याहा‚{\tiny $_{६}$}‚रः । \textbf{न त‚यो}रित्येक‚ग‚त‚स्य भेद‚स्य त‚द‚न्य‚व्यावृतेश्च न‚{\tiny $_{lb}$}‚ \textbf{विशेष‚ण‚विशेष्य‚भावः} ।
	{\color{gray}{\rmlatinfont\textsuperscript{§~\theparCount}}}
	\pend% ending standard par
      ‚{\tiny $_{lb}$}‚

	  
	  \pstart \leavevmode% starting standard par
	कः पुन‚र‚य‚मेक‚ग‚तो भेदः का च त‚द‚न्य‚व्यावृत्तिर्येनान्य‚व्यावृत्तिनान्त‚रीय‚क‚{\tiny $_{lb}$}‚स्यैक‚ग‚त‚भेद‚स्य श‚ब्दात् प्र‚त्य‚यो भ‚व‚तीति चेत् ।
	{\color{gray}{\rmlatinfont\textsuperscript{§~\theparCount}}}
	\pend% ending standard par
      ‚{\tiny $_{lb}$}‚

	  
	  \pstart \leavevmode% starting standard par
	उच्य‚ते । वृक्ष‚श‚ब्द‚वाच्य‚स्ताव‚द् विक‚ल्प‚बुद्धिप्र‚तिभासी शाखादिम‚दाकारः स‚र्व‚{\tiny $_{lb}$}‚\leavevmode\ledsidenote{\textenglish{93b/PSVTa}} वृक्षेष्व‚भिन्न‚रूप इवात‚द्रूपेभ्यो भिन्न इव भा‚{\tiny $_{७}$}‚स‚मान एव ग‚तो भेदो भिन्नः स्व‚भाव‚{\tiny $_{lb}$}‚ इत्य‚र्थः । एवं घ‚टादिश‚ब्द‚वाच्योप्येक‚ग‚तो भेदो द्र‚ष्ट‚व्यः । त‚म्भेदं च प्र‚तिपाद‚य‚न्‚{\tiny $_{lb}$}‚ श‚ब्दो विजातीय‚निवृत्तिं प्र‚स‚ज्य‚प्र‚तिषेध‚ल‚क्ष‚णाम‚र्थाद् ग‚म‚य‚ति सा त‚द‚न्य‚व्यावृत्तिः ।
	{\color{gray}{\rmlatinfont\textsuperscript{§~\theparCount}}}
	\pend% ending standard par
      ‚{\tiny $_{lb}$}‚

	  
	  \pstart \leavevmode% starting standard par
	न‚नु य‚द्य‚न्य‚व्यावृत्तिविशिष्टो व‚स्तुभागो न ग‚म्य‚ते किम‚र्थ‚न्त‚र्ह्य‚न्य‚व्यावृत्ति‚{\tiny $_{lb}$}‚विशिष्ट इत्याद्युक्त‚मित्याह । \textbf{एक‚भेदाभिधान} इत्यादि । य‚श्चायं य‚था विभ‚क्त‚{\tiny $_{lb}$}‚ एक‚ग‚{\tiny $_{१}$}‚तो भेदः स एव श‚ब्देन चोद्य‚ते [।] त‚त्प्र‚तीतिरेवान्व‚य‚ग‚तिः [।] या त्व‚र्थाद्‚{\tiny $_{lb}$}‚ ‚{\tiny $_{lb}$}‚ \leavevmode\ledsidenote{\textenglish{253/s}}\textbf{अन्य‚व्यावृत्तिग‚तिः} सा व्य‚तिरेक‚ग‚तिः [।] एव‚म‚न्व‚य‚व्य‚तिरेकाभ्यां श‚ब्दोर्थ‚वान्‚{\tiny $_{lb}$}‚ भ‚व‚ति [।] \textbf{तेनान्व‚य‚व्य‚तिरेक‚चोद‚न‚या}न्व‚य‚स्य साक्षाद‚र्थात्तु व्य‚तिरेक‚स्य चोद‚ना‚{\tiny $_{lb}$}‚ द्र‚ष्ट‚व्या । त‚या चोद‚न‚या \textbf{व्य‚व‚हाराङ्ग‚तां श‚ब्दानां द‚र्श‚य‚न्ना}चार्य दि ङ् ना ग आह‚{\tiny $_{lb}$}‚ \textbf{त‚द्व्यावृत्त्या ग‚म्य‚ते व‚स्तुभाग} इति । \textbf{त‚था त‚द्विशिष्टो वे}त्याह ।‚{\tiny $_{२}$}‚ व्यावृत्तिश‚ब्देन‚{\tiny $_{lb}$}‚ \textbf{व्य‚तिरेक उक्तः} । व‚स्तुभाग‚श‚ब्देनान्व‚यः । अनेनैत‚द‚पि व्याख्यातं श‚ब्दान्त‚रापोहं‚{\tiny $_{lb}$}‚ कुर्व‚न्ती श्रुतिः स्वार्थ‚म‚भिध‚त्त इत्य‚र्थः । त‚त्र योसावेक‚ग‚तो भेदो विक‚ल्प‚बुद्धिप्र‚तिभासी‚{\tiny $_{lb}$}‚ व्याख्यातः स एव स्वार्थ‚स्त‚त्रार्थान्त‚र‚व्यावृत्तिर‚र्थान्त‚रापोहः प्र‚स‚ज्य‚प्र‚तिषेध‚ल‚क्ष‚ण‚स्तं‚{\tiny $_{lb}$}‚ कुर्व‚तीत्य‚र्थाद् ग‚म‚य‚न्तीत्य‚र्थः ।
	{\color{gray}{\rmlatinfont\textsuperscript{§~\theparCount}}}
	\pend% ending standard par
      ‚{\tiny $_{lb}$}‚

	  
	  \pstart \leavevmode% starting standard par
	य‚दि चान्य‚निवृत्तिपुर‚स्स‚रैव वृक्षादिश‚ब्द‚प्र‚वृत्ति‚{\tiny $_{३}$}‚स्त‚दान्व‚य‚व्य‚तिरेक‚चोद‚न‚येत्या‚{\tiny $_{lb}$}‚दिव्याख्यानं व्य‚र्थं स्यात् ।
	{\color{gray}{\rmlatinfont\textsuperscript{§~\theparCount}}}
	\pend% ending standard par
      ‚{\tiny $_{lb}$}‚

	  
	  \pstart \leavevmode% starting standard par
	त‚स्माद्विधिरेव श‚ब्दार्थः ।
	{\color{gray}{\rmlatinfont\textsuperscript{§~\theparCount}}}
	\pend% ending standard par
      ‚{\tiny $_{lb}$}‚

	  
	  \pstart \leavevmode% starting standard par
	य‚त एवैक‚भेदाभिधानेऽर्थाद‚न्य‚व्यावृत्तिग‚ति\textbf{र‚त एवे}त्यादि । \textbf{स्वार्थ}स्य \textbf{भेद‚रूप‚{\tiny $_{lb}$}‚त्वादि}ति । यो वृक्ष‚श‚ब्द‚स्यार्थः शाखादिम‚दाकारः । विक‚ल्प‚प्र‚तिभासी स भेद‚रूपः ।‚{\tiny $_{lb}$}‚ भिन्न‚स्व‚भावोऽभेदाकार‚व्यावृत्त‚त्वात् स्व‚यं । अतो भेद‚रूप‚स्य \textbf{स्वार्थ}स्या\textbf{भिधाना‚{\tiny $_{lb}$}‚देवार्थाद‚न्य‚व्यावृत्ति‚{\tiny $_{४}$}‚ग‚ति}रेवं ह्य‚वृक्षाद् भेद‚रूप‚स्यैव वृक्षार्थ‚स्य ग‚तिर्भ‚वेत् ।
	{\color{gray}{\rmlatinfont\textsuperscript{§~\theparCount}}}
	\pend% ending standard par
      ‚{\tiny $_{lb}$}‚

	  
	  \pstart \leavevmode% starting standard par
	य‚द्य‚र्थात् त‚त्र वृक्ष‚निवृत्तिर्भ‚व‚तीति । \textbf{त‚स्मात् स्वार्थाभिधा}न‚मेव श‚ब्द‚स्य‚{\tiny $_{lb}$}‚ व्यापारो\textbf{न्य‚व्याव‚र्त्त‚न}न्त्व‚र्था\textbf{दिति न श‚ब्द‚स्य} व्यापार‚द्व‚यं ।
	{\color{gray}{\rmlatinfont\textsuperscript{§~\theparCount}}}
	\pend% ending standard par
      ‚{\tiny $_{lb}$}‚

	  
	  \pstart \leavevmode% starting standard par
	न‚नु विधाय‚केन वाक्येनान्व‚य‚मात्र‚म्प्र‚तिपाद्य‚ते नान्य‚स्य विधानं प्र‚तिषेधो वा ।‚{\tiny $_{lb}$}‚ निषेध‚केनापि निषेध‚मात्र‚मेव केव‚लं प्र‚तीय‚ते नान्य‚स्य विधानं प्र‚तिषेधो वेत्य‚त आ‚{\tiny $_{५}$}‚ह ।
	{\color{gray}{\rmlatinfont\textsuperscript{§~\theparCount}}}
	\pend% ending standard par
      ‚{\tiny $_{lb}$}‚

	  
	  \pstart \leavevmode% starting standard par
	\textbf{न ह्य‚न्व}य इत्यादि । प्र‚तिनिय‚त‚स्यार्थ‚स्य विधान‚म‚न्व‚यो य‚था वृक्षंच्छिन्धीति ।‚{\tiny $_{lb}$}‚ त‚त्रावृक्ष‚स्यार्थान्निवृत्तिर्व्य‚तिरेकः । स य‚त्र न विद्य‚ते सो\textbf{व्य‚तिरे}क एवं भूतो\textbf{न्व‚यो‚{\tiny $_{lb}$}‚ न मे स्ति} । किन्तु स‚र्व एव स व्य‚तिरेकः ।
	{\color{gray}{\rmlatinfont\textsuperscript{§~\theparCount}}}
	\pend% ending standard par
      ‚{\tiny $_{lb}$}‚

	  
	  \pstart \leavevmode% starting standard par
	तेन य‚दुच्य‚ते भ ट्टो द्यो त क रा भ्यां [।]
	{\color{gray}{\rmlatinfont\textsuperscript{§~\theparCount}}}
	\pend% ending standard par
      ‚{\tiny $_{lb}$}‚
	  \bigskip
	  \begingroup
	
	    
	    \stanza[\smallbreak]
	  {\normalfontlatin\large ``\qquad}विधिरूप‚श्च श‚ब्दार्थो येन नाभ्युप‚ग‚म्य‚ते ।&‚{\tiny $_{lb}$}‚न भ‚वेद् व्य‚तिरेकोपि त‚स्य त‚त्पूर्व‚को ह्य‚साविति\edtext{}{\edlabel{pvsvt_253-1}\label{pvsvt_253-1}\lemma{साविति}\Bfootnote{\href{http://sarit.indology.info/?cref=\%C5\%9Bv}{ Ślokavārtika. }}}{\normalfontlatin\large\qquad{}"}\&[\smallbreak]
	  
	  
	  
	  \endgroup
	‚{\tiny $_{lb}$}‚‚{\tiny $_{lb}$}‚‚{\tiny $_{lb}$}‚\textsuperscript{\textenglish{254/s}}

	  
	  \pstart \leavevmode% starting standard par
	निर‚स्त‚मिष्ट‚त्वात् । अ‚{\tiny $_{६}$}‚न‚न्व‚यो वान्व‚य‚र‚हितो वा व्य‚तिरेको न हीति स‚म्ब‚न्धः ।‚{\tiny $_{lb}$}‚ त‚त्र विशेष‚स्य प्र‚तिषेधो व्य‚तिरेकः । त‚त्रार्थाच्छेष‚ग‚तिर‚न्व‚यः । \textbf{एकान्}व‚य इत्यादि‚{\tiny $_{lb}$}‚नैत‚देव स‚म‚र्थ्य‚ते । \textbf{एकान्व‚य}स्येत्येक‚स्य विधान‚स्य \textbf{प‚रिहार्याभा}वे व्य‚व‚च्छेद्याभावे‚{\tiny $_{lb}$}‚ निष्फ‚ल‚चोद‚न‚त्वात् ।
	{\color{gray}{\rmlatinfont\textsuperscript{§~\theparCount}}}
	\pend% ending standard par
      ‚{\tiny $_{lb}$}‚

	  
	  \pstart \leavevmode% starting standard par
	य‚दि वृक्ष‚ञ्छिन्धीत्युक्तेर्थाद‚वृक्ष‚स्य न त‚त्र व्य‚व‚च्छेद‚स्त‚दा वृक्ष‚श‚ब्द‚प्र‚योगो‚{\tiny $_{lb}$}‚ \leavevmode\ledsidenote{\textenglish{94a/PSVTa}} निष्फ‚लः स्यात् ।‚{\tiny $_{७}$}‚ य‚द्वा किम्वृक्ष‚ञ्छिन‚द्मि उतान्य‚मिति श्रोतुर्जिज्ञासायां स‚त्यां‚{\tiny $_{lb}$}‚ वृक्ष‚ञ्छिन्धीत्युक्तेन्य‚निषेधः प्र‚तीय‚त एवान्य‚था प‚रिहार्याभावे निष्फ‚ल‚म‚भिधानं‚{\tiny $_{lb}$}‚ स्यात् । त‚था य‚दाप्याकांक्षार‚हितः श्रोता वृक्ष‚ञ्छिन्धीति श‚ब्देन चोद्य‚ते त‚दापि वृक्ष‚{\tiny $_{lb}$}‚ एव प्र‚व‚र्त्त‚ते नान्य‚त्र [।] व‚क्ता हि वृक्ष एवायं प्र‚व‚र्त्त‚ते नान्य‚त्रेत्य‚नेनाभिप्रायेण‚{\tiny $_{lb}$}‚ श‚ब्दं प्र‚युङ्क्ते । विव‚क्षानुग‚म‚नं च ध्व‚नेः । त‚स्मान्ना‚{\tiny $_{१}$}‚स्ति व्य‚तिरेक‚र‚हितोन्व‚य‚स्त\textbf{थैक‚{\tiny $_{lb}$}‚प‚रिहार‚स्ये}त्येक‚प्र‚तिषेध‚स्य प्र‚तिषेध‚स्यैवैक‚स्येत्य‚र्थः । \textbf{क्व‚चिदिति} प्र‚तिषेधेन विष‚यीकृते‚{\tiny $_{lb}$}‚ व‚स्तुनि \textbf{स्थित्य‚भावे निष्फ‚ल‚चोद‚न‚त्वात्} । त‚था हि सुराविष‚य‚स्य पान‚स्य प्र‚तिषेधे ।‚{\tiny $_{lb}$}‚ य‚दि सुराया अन्य‚त्रापि पान‚स्य नाव‚स्थानं स्यात् । त‚दा स‚र्व‚स्मिन् विष‚ये निषिद्ध‚स्य‚{\tiny $_{lb}$}‚ पान‚स्य विशेषे प्र‚तिषेधोन‚र्थ‚कः । त‚त्र सुराऽपेयेति‚{\tiny $_{२}$}‚ सुरापान‚मात्र‚स्य प्र‚तिषेधे च‚रि‚{\tiny $_{lb}$}‚तार्थ‚त्वाद् वाक्य‚स्य य‚द्य‚प्युद‚कादिपानं श‚ब्देन न विधीय‚ते । त‚थापि सुराया अन्य‚त्र‚{\tiny $_{lb}$}‚ पान‚स्याव‚स्थान‚न्न निवार्य‚त इत्य‚र्थाद‚न्व‚य‚ग‚तिर‚न्य‚था सुराग्र‚ह‚ण‚म‚न‚र्थ‚कं स्यात् । य‚द्वा‚{\tiny $_{lb}$}‚ किमुद‚कादिव‚त् सुरा पात‚व्या किम्वा नेति प्र‚श्ने सुरा न पात‚व्येत्युक्ते सुरैवेति‚{\tiny $_{lb}$}‚ प्र‚तीतेर्नोद‚कादिपान‚विधानं प्र‚कृत‚न्निषेध्य‚ते [।] तेन स‚र्व‚त्र विधि‚{\tiny $_{३}$}‚प्र‚तिषेध‚रूप‚{\tiny $_{lb}$}‚स्यैव श‚ब्दार्थ‚त्वं ।
	{\color{gray}{\rmlatinfont\textsuperscript{§~\theparCount}}}
	\pend% ending standard par
      ‚{\tiny $_{lb}$}‚

	  
	  \pstart \leavevmode% starting standard par
	क‚स्त‚र्हि विधिप्र‚तिषेध‚प‚र्युदास‚वाक्यानाम्भेदः । म‚हान्भेदः । विधाय‚कं हि‚{\tiny $_{lb}$}‚ वाक्य‚म्विधिं प्राधान्येनाभिधायान्य‚निषेध‚क‚म‚र्थात् । निषेध‚कं च निषेधं प्राधान्ये‚{\tiny $_{lb}$}‚नाभिधायार्थाद‚न्य‚विधान‚माह । प‚र्युदास‚प्र‚तिपाद‚क‚न्तु वाक्यं प्र‚तिषेध‚पूर्व‚क‚म‚न्य‚विधानं‚{\tiny $_{lb}$}‚ प्राधान्येनाहेत्य‚स्त्येव विशेष इति ।
	{\color{gray}{\rmlatinfont\textsuperscript{§~\theparCount}}}
	\pend% ending standard par
      ‚{\tiny $_{lb}$}‚

	  
	  \pstart \leavevmode% starting standard par
	न‚नु य‚द्य‚न्य‚निवृत्तिर‚{\tiny $_{४}$}‚र्थात् प्र‚तीय‚ते सैव त‚र्हि पार‚मार्थिकी भावानाम‚स्ति‚{\tiny $_{lb}$}‚ य‚या विशिष्टा गृह्य‚न्त इत्येवं क‚स्मान्नेष्य‚ते [।] किं पुन‚रेव‚न्त‚द्ग‚तेस्त‚दुपाधित्वात्‚{\tiny $_{lb}$}‚ त‚द्विशिष्टो ग‚त इति व्याख्याय‚त इत्य‚त आह । \textbf{स चाय}मित्यादि [।] यो \textbf{भेदो}‚{\tiny $_{lb}$}‚ व्यावृत्तिल‚क्ष‚ण आचार्य दि ङ् ना गे न विशेष‚ण‚त्वेनाभिम‚तः स चाय\textbf{म‚रूपो} निःस्व‚{\tiny $_{lb}$}‚‚{\tiny $_{lb}$}‚ \leavevmode\ledsidenote{\textenglish{255/s}}भावः । नास्य व्य‚तिरिक्त‚म‚व्य‚तिरिक्त‚म्वा रूप‚म‚स्तीति कृत्वा । क‚थ‚{\tiny $_{५}$}‚न्त‚र्हि भावानां‚{\tiny $_{lb}$}‚ विजातीयाद् भेद इति प्र‚तीतिरिति चेदाह । \textbf{रूप‚व‚त्त्वेन} भाव‚स‚म्ब‚न्धित्वेन त‚स्य‚{\tiny $_{lb}$}‚ भेद‚स्य \textbf{केव‚ल‚न्द‚र्श‚न}म्प्र‚तीति\textbf{र्विप्ल‚वो} भ्रान्तिः [।] केव‚ल‚मिति त‚थाभूत‚बाह्य‚निर‚{\tiny $_{lb}$}‚पेक्षं [।] का रि का या म‚प्येवं व्याख्यानं द्र‚ष्ट‚व्यं ।
	{\color{gray}{\rmlatinfont\textsuperscript{§~\theparCount}}}
	\pend% ending standard par
      ‚{\tiny $_{lb}$}‚

	  
	  \pstart \leavevmode% starting standard par
	न‚नु य‚दि रूप‚व‚त्त्वेन द‚र्श‚नं क‚थं बुद्धिविप्ल‚व इत्य‚त आह । \textbf{तेनै}वेत्यादि ।‚{\tiny $_{lb}$}‚ रूप‚व‚त्त्वेन भाव‚स‚म्ब‚न्धित्वेन य‚द्द‚र्श‚न‚म्भेद‚स्य तेनैव्\textbf{आप‚{\tiny $_{६}$}‚र‚मार्थो} न व‚स्तुभूतो‚{\tiny $_{lb}$}‚ऽस‚त्त्वात् । \textbf{असा}विति भेदः प्र‚स‚ज्य‚रूपः । \textbf{अन्य‚थे}त्येव‚म‚निष्य‚माणे । \textbf{न हि व‚स्तुनः}‚{\tiny $_{lb}$}‚ स्व‚ल‚क्ष‚ण‚स्य स‚म्ब‚न्धिनी \textbf{व्यावृत्तिर्व‚स्तु भ‚व‚ति} । किं कार‚णं [।] \textbf{भेदोस्यास्मादि‚{\tiny $_{lb}$}‚तीर‚णात्} ।
	{\color{gray}{\rmlatinfont\textsuperscript{§~\theparCount}}}
	\pend% ending standard par
      ‚{\tiny $_{lb}$}‚

	  
	  \pstart \leavevmode% starting standard par
	एत‚दुक्त‚म्भ‚व‚ति । य‚दि हि सा व‚स्तुभूता स्यात् त‚दा वृक्षेभ्योऽवृक्ष‚व्यावृत्तिर‚{\tiny $_{lb}$}‚भिन्ना भिन्ना वा स्यात् । य‚द्य‚भिन्ना । अस्माद‚वृक्षाद् वृक्ष‚स्य भेद इति व्य‚तिरेक‚{\tiny $_{७}$}‚- \leavevmode\ledsidenote{\textenglish{94b/PSVTa}}‚{\tiny $_{lb}$}‚ प्र‚तीतिर्न स्यात् । प‚लाशाच्चैक‚स्माद‚वृक्ष‚व्यावृत्तेर‚भिन्न‚त्वात् । ध‚वादीनाम‚वृक्ष‚{\tiny $_{lb}$}‚व्यावृत्तिर्न स्यात् । प‚लाश‚व‚त् त‚त्स्व‚भावाया व्यावृत्तेस्तेष्व‚न‚नुग‚मात् । अथ‚{\tiny $_{lb}$}‚ प‚लाशाद् भिन्ना सा । त‚त्राप्य‚वृक्ष‚व्यावृत्तेः स‚काशात् प‚लाश‚स्य व्यावृत्तिः स्याद‚वृक्ष‚{\tiny $_{lb}$}‚व्यावृत्तेश्च व्याव‚र्त्त‚मान‚स्य प‚लाश‚स्यावृक्ष‚रूप‚ता स्याद‚वृक्ष‚व‚त् । त‚त‚श्चास्माद‚{\tiny $_{lb}$}‚वृक्षाद‚स्य वृक्ष‚स्य भेद इति प्र‚तीतिर्न स्यादिष्य‚ते‚{\tiny $_{१}$}‚ च । त‚स्मात्त‚द्व्यावृत्तिर्न व‚स्तु ।‚{\tiny $_{lb}$}‚ \textbf{भेदोस्यास्मादितीर‚णा}दुच्चार‚णादित्य‚र्थः । उप‚ल‚क्ष‚णं चैत‚त् प्र‚तीतेर‚पि ग्र‚ह‚णं ।
	{\color{gray}{\rmlatinfont\textsuperscript{§~\theparCount}}}
	\pend% ending standard par
      ‚{\tiny $_{lb}$}‚

	  
	  \pstart \leavevmode% starting standard par
	\textbf{रूपं हीत्या}दिना व्याच‚ष्टे । हि य‚स्मात् । रूपं किमुच्य‚ते \textbf{प‚र‚मा}र्थः । व‚स्तु‚{\tiny $_{lb}$}‚स्व‚भावः । भेदोन्य‚व्यावृत्ति\textbf{र्य‚दि रूपं स्याद्} य‚दि स्व‚भावो भ‚वेत् । त‚दित्य‚वृक्षाद्‚{\tiny $_{lb}$}‚ व्यावृत्तिरूपं । \textbf{त‚द्रूप‚मि}ति य‚त्त‚द्व्यावृत्तं प‚लाश‚स्व‚ल‚क्ष‚णं त‚दात्म‚कं । \textbf{अत‚द्रूपं} वेति‚{\tiny $_{lb}$}‚ त‚तोन्य‚दित्य‚{\tiny $_{२}$}‚र्थः । व‚स्तुन‚स्त‚त्त्वान्य‚त्त्वान‚तिक्र‚मादित्य‚र्थः ।
	{\color{gray}{\rmlatinfont\textsuperscript{§~\theparCount}}}
	\pend% ending standard par
      ‚{\tiny $_{lb}$}‚

	  
	  \pstart \leavevmode% starting standard par
	\textbf{ताद्रूप्य} इत्यादिना प्र‚थ‚म‚प‚क्षे दोष‚माह । \textbf{ताद्रूप्ये} प‚लाशाद‚न्य‚त्त्वे य‚या व्यावृत्त्या‚{\tiny $_{lb}$}‚ सामान्य‚भूत‚या स‚र्वे वृक्ष‚भेदाः व्यावृत्ता दृष्टास्त‚स्यास्ताद्रूप्ये प‚लाशाद‚न‚न्य‚त्वेभ्युप‚{\tiny $_{lb}$}‚ग‚म्य‚माने त‚देव प‚लाश‚स्व‚ल‚क्ष‚ण‚मेव व्यावृत्तिरिति कृत्वा प‚लाश एवावृक्षाद् भिन्नः‚{\tiny $_{lb}$}‚ प्राप्नोति \textbf{नान्यो} ध‚वादिः । \textbf{त‚त} इत्य‚वृक्षाद् \textbf{भिद्येत} । किङ्कार‚ण‚मित्याह । \textbf{न ही}‚{\tiny $_{lb}$}‚त्यादि । अवृक्षाद्धि व्यावृत्तिः प‚लाश‚स्यैव रूपं । \textbf{न च त‚स्य रूप‚{\tiny $_{३}$}‚म‚न्य}स्य ध‚वादेः‚{\tiny $_{lb}$}‚ ‚{\tiny $_{lb}$}‚ \leavevmode\ledsidenote{\textenglish{256/s}}\textbf{स्यात्} ।
	{\color{gray}{\rmlatinfont\textsuperscript{§~\theparCount}}}
	\pend% ending standard par
      ‚{\tiny $_{lb}$}‚

	  
	  \pstart \leavevmode% starting standard par
	\textbf{न त‚दे}वेत्यादिना द्वितीय‚प‚क्षोप‚न्यासः । \textbf{न त‚देव} प‚लाश‚स्व‚ल‚क्ष‚ण‚मेव \textbf{त‚स्य‚{\tiny $_{lb}$}‚ भेद‚स्य रूपं [।] रूपं च} स्व‚भाव‚श्चासौ \textbf{भेद इष्य‚ते । त‚तोन्य‚देव} प‚लाशाद् व्यावृत्ति‚{\tiny $_{lb}$}‚\textbf{रूपं स्यात्} । य‚द्वा रूपं चान्य‚देव भेद‚स्य स्यात् । \textbf{त‚त‚श्चा}न्य‚त्त्वात् कार‚णात् ।‚{\tiny $_{lb}$}‚ \textbf{भावः} प‚लाशात्म‚क‚स्\textbf{त‚स्माद}वृक्ष‚व्यावृत्तिरूपाद् \textbf{व्याव‚र्त्ते}त । \textbf{त‚तः} कार‚णात् ।‚{\tiny $_{lb}$}‚ \textbf{ऊष्मा‚{\tiny $_{४}$}‚दि}त्य‚वृक्षात् \textbf{त‚स्य} प‚लाश‚स्य \textbf{भेद इति न स्यात्} । य‚स्माद् \textbf{य‚त्} ख‚लु व‚स्तु ।‚{\tiny $_{lb}$}‚ \textbf{य‚तो भेदो} य‚द्भेद‚स्त‚स्माद् \textbf{व्याव‚र्त्त‚ते । त‚त्त‚देव भ‚व‚ति} । अवृक्ष‚व्यावृत्तेर्भ‚व‚ति ।‚{\tiny $_{lb}$}‚ अवृक्ष‚व्यावृत्तेर‚वृक्ष‚निवृत्तिरूपाया निव‚र्त्त‚मान‚म्प‚लाश‚स्व‚ल‚क्ष‚ण‚म‚वृक्ष‚मिव स्याद‚वृक्ष‚{\tiny $_{lb}$}‚व‚त् । मा वा भूद् वृक्षावृक्ष‚योरेक‚त्व‚न्त‚थापि \textbf{सोस्य भेद इति च न स्यात्} ।
	{\color{gray}{\rmlatinfont\textsuperscript{§~\theparCount}}}
	\pend% ending standard par
      ‚{\tiny $_{lb}$}‚

	  
	  \pstart \leavevmode% starting standard par
	य‚दि चाय‚म‚वृक्षाद् भेदः प‚लाशाद‚न्यः स्यात्त‚द‚य‚म‚{\tiny $_{५}$}‚वृक्षाद् \textbf{भेदो}स्य प‚लाश‚स्य‚{\tiny $_{lb}$}‚ \textbf{स‚म्ब‚न्धीति न स्यात् । न ही}त्याद्य‚स्येव स‚म‚र्थ‚नं । \textbf{अन्यः} स्व‚त‚न्त्रोन्य‚स्य \textbf{स‚म्ब‚न्धी‚{\tiny $_{lb}$}‚ भेदो} ध‚र्म‚रूपो \textbf{भ‚व‚ति । स‚ति वे}ति \textbf{स‚म्ब‚न्धित्}वेभ्युप‚ग‚म्य‚माने वा । \textbf{स‚म्ब‚न्धः कार्य‚{\tiny $_{lb}$}‚कार‚ण‚भावो}ङ्गीक‚र्त्त‚व्यः । भिन्न‚योः कार्य‚कार‚ण‚म‚न्त‚रेण स‚म्ब‚न्धायोगात् । त‚त्र‚{\tiny $_{lb}$}‚ व्यावृत्तिमान् कार‚णं व्यावृत्तिः कार्यं । \textbf{इति} हेतो \textbf{रूपं} व‚स्त्व‚न्त‚र‚मेव \textbf{त‚{\tiny $_{६}$}‚ज्ज‚नित}न्तेन‚{\tiny $_{lb}$}‚ व्यावृत्तिम‚ता स्व‚ल‚क्ष‚णेन ज‚नितं \textbf{भेद इति । पादा}\edtext{}{\lemma{नितं}\Bfootnote{? प‚दा}}र्थान्त‚र‚स्य नाम कृत‚{\tiny $_{lb}$}‚न्त‚स्मात् कार्य‚त्वेनाभिम‚ताद् भेदाख्यात् प‚दार्थाद् \textbf{अविशेषाद}न्य‚त्रापि \textbf{कार‚णानां} यानि‚{\tiny $_{lb}$}‚ \textbf{कार्याणि} तानि \textbf{स‚र्वाणि व्यावृत्त‚यो} भेदाः \textbf{स्युः} । न चैवं । त‚दान्य‚त्रापि भेदाभिम‚ते‚{\tiny $_{lb}$}‚ भेद इति व्य‚प‚देशो मा भूत् ।
	{\color{gray}{\rmlatinfont\textsuperscript{§~\theparCount}}}
	\pend% ending standard par
      ‚{\tiny $_{lb}$}‚

	  
	  \pstart \leavevmode% starting standard par
	\leavevmode\ledsidenote{\textenglish{95a/PSVTa}} अथ कार्यं भेद इति नोच्य‚ते किन्तु भेदः कार्य इत्युच्य‚ते । त‚दाप्याह । \textbf{अर्थान्त‚{\tiny $_{lb}$}‚र‚त्वे च} भेद‚स्याभ्युप‚ग‚म्य‚माने । \textbf{त‚तोपि} व्यावृत्तिम‚तो\textbf{प्य‚स्य} भेद‚स्य \textbf{भेदोस्ति} ।‚{\tiny $_{lb}$}‚ अन्य‚था भेद‚स्यार्थान्त‚र‚त्व‚मेव न स्यात् । \textbf{त‚त}श्च प‚लाश‚स्यावृक्षाद् यो भेद‚स्त‚स्य‚{\tiny $_{lb}$}‚ भेद‚स्य प‚लाशाद् भेद इति कृत्वा \textbf{स भेदो भेदोपाधिः} स प‚लाशाद् \textbf{भेदः} ।‚{\tiny $_{lb}$}‚ उपाधिविशेष‚णं य‚स्य भेद‚स्य स \textbf{भेदोपाधि}स्त‚द्भाव‚स्त\textbf{त्त्व‚न्}त\textbf{स्माद}साव‚वृक्षाद्‚{\tiny $_{lb}$}‚ भेदः प‚लाश‚स्य \textbf{न भेदः स्या‚{\tiny $_{१}$}‚त् । द्र‚व्यान्त‚र‚व}त् । य‚था हि द्र‚व्यान्त‚रं घ‚टादिक‚{\tiny $_{lb}$}‚‚{\tiny $_{lb}$}‚ \leavevmode\ledsidenote{\textenglish{257/s}}म‚प्य‚घ‚टापेक्ष‚या यो \textbf{भेद}स्त\textbf{दुपा}धित्वान्न प‚लाश‚स्य भेद‚स्त‚द्व‚त् । स्व‚त‚न्त्र‚त्वादित्य‚भि‚{\tiny $_{lb}$}‚प्रायः ।
	{\color{gray}{\rmlatinfont\textsuperscript{§~\theparCount}}}
	\pend% ending standard par
      ‚{\tiny $_{lb}$}‚

	  
	  \pstart \leavevmode% starting standard par
	न‚नु य‚दि नामं भेदोपाधिर्भेद‚स्त‚थापि किमिति प‚लाश‚स्य भेदो न भ‚व‚तीत्य‚त‚{\tiny $_{lb}$}‚ आह । \textbf{न ही}त्यादि । \textbf{अय‚म‚त इति विशेष‚निर्देशात्} । अय‚म‚वृक्षाद् भेदः । \textbf{अत} इत्य‚{\tiny $_{lb}$}‚वृक्षात् । अस्येत्येत‚द‚पेक्ष‚णीयं । अस्य‚{\tiny $_{२}$}‚ प‚लाश‚स्यायं भेदोऽस्माद‚वृक्षादित्येवं विशे‚{\tiny $_{lb}$}‚ष‚निर्देशात् पार‚त‚न्त्र्येण निर्देशात् प‚लाश‚स्य स‚म्ब‚न्धी भेदो ध‚र्मः सिध्य‚ति । पार‚{\tiny $_{lb}$}‚त‚न्त्र्येण च निर्देशो भेद‚स्याभेदे स‚ति सिध्य‚ति भेदान्त‚र‚प्र‚तिक्षेपेण । न त्व‚र्थान्त‚र‚त्वे‚{\tiny $_{lb}$}‚ भेद‚स्य । अर्थान्त‚र‚त्वे हि भेदोपाधित्वाद् घ‚ट‚व‚न्न प‚लाश‚स्य भेदः स्यात् । त‚त‚श्चास्मा‚{\tiny $_{lb}$}‚द‚स्य भेद इति विशेष‚निर्देशो न स्यात् । त‚देवं‚{\tiny $_{३}$}‚ व्य‚तिरेक‚प‚क्षेऽवृक्षाद् भेद‚स्यापि‚{\tiny $_{lb}$}‚ प‚लाशाद् यो भेदः सोपि रूप‚वानिति त‚स्यापि स्वाश्र‚याद् भेदेन भ‚वित‚व्यं । त‚था‚{\tiny $_{lb}$}‚ च स‚र्व‚भेदानां भिन्न‚स्व‚भाव‚त‚या भेदोपाधित्वेन स्व‚य‚न्न रूप‚भेद‚तेति न क‚श्चिद् भेदः‚{\tiny $_{lb}$}‚ स्यात् ।
	{\color{gray}{\rmlatinfont\textsuperscript{§~\theparCount}}}
	\pend% ending standard par
      ‚{\tiny $_{lb}$}‚

	  
	  \pstart \leavevmode% starting standard par
	एत‚देवाह । \textbf{त‚त‚श्चे}त्यादि । \textbf{उपाध्य‚भाव} इति व्यावृत्तिल‚क्ष‚ण‚स्य ध‚र्म‚भूत‚स्यो‚{\tiny $_{lb}$}‚पाधेर‚भावे स‚र्व‚स्य स्व‚भावान्त‚र‚त्वेन ध‚र्मित्वाद् \textbf{भेद‚स्यै‚{\tiny $_{४}$}‚वाभावः स्यात्} त‚स्य ध‚र्मि‚{\tiny $_{lb}$}‚रूप‚त्वात् ।
	{\color{gray}{\rmlatinfont\textsuperscript{§~\theparCount}}}
	\pend% ending standard par
      ‚{\tiny $_{lb}$}‚

	  
	  \pstart \leavevmode% starting standard par
	योपि म‚न्य‚ते [।] य‚दि रूप‚व‚ती व्यावृत्तिः स्यात् स्यात् त‚त्त्वान्य‚त्व‚प‚क्ष‚भावी‚{\tiny $_{lb}$}‚ दोषो याव‚ता नीरूपा सास्ति त‚या च भावा विशिष्टा गृह्य‚न्त इति ।
	{\color{gray}{\rmlatinfont\textsuperscript{§~\theparCount}}}
	\pend% ending standard par
      ‚{\tiny $_{lb}$}‚

	  
	  \pstart \leavevmode% starting standard par
	त‚द‚युक्तं ।‚{\tiny $_{lb}$}‚ 
	    \pend% close preceding par
	  
	    
	    \stanza[\smallbreak]
	  {\normalfontlatin\large ``\qquad}त‚द्ग‚तावेव श‚ब्देभ्यो ग‚म्य‚तेऽन्य‚निव‚र्त्त‚नं ।{\normalfontlatin\large\qquad{}"}\&[\smallbreak]
	  
	  
	  ‚{\tiny $_{lb}$}‚ 
	    
	    \stanza[\smallbreak]
	  {\normalfontlatin\large ``\qquad}न त‚त्र ग‚म्य‚ते क‚श्चिद्विशिष्टः केन‚चित्प‚रः ॥{\normalfontlatin\large\qquad{}"}\&[\smallbreak]
	  
	  
	  
	    \pstart  \leavevmode% new par for following
	    \hphantom{.}
	  \href{http://sarit.indology.info/?cref=pv.3.125-126}{प्र० वा० १ । १२८}‚{\tiny $_{lb}$}‚ इति ग्र‚न्थ‚विरोधात् । नीरूप‚स्य चास्तित्व‚विरोधाच्छ‚श‚विषाण‚व‚त् । नीरूप‚त्वा‚{\tiny $_{५}$}‚‚{\tiny $_{lb}$}‚देव च न त‚स्याः प्र‚त्य‚क्षं ग्राह‚कं नाप्य‚नुमानं । स‚म्ब‚न्धाभावात् ।
	{\color{gray}{\rmlatinfont\textsuperscript{§~\theparCount}}}
	\pend% ending standard par
      ‚{\tiny $_{lb}$}‚

	  
	  \pstart \leavevmode% starting standard par
	नापि निय‚त‚रूपान्य‚थानुप‚प‚त्त्या त‚त्क‚ल्प‚ना स‚म्ब‚न्धाभावादेव । स्व‚हेतुभ्य एव‚{\tiny $_{lb}$}‚ च निय‚त‚रूपाणामुत्प‚न्न‚त्वादिति \textbf{स‚र्व‚भावा स्व‚भावेन} व्यावृत्तिभागिन \href{http://sarit.indology.info/?cref=pv.3.39}{१ । ४२}‚{\tiny $_{lb}$}‚ इत्य‚त्रान्त‚रेऽभिहित‚त्वात् ।
	{\color{gray}{\rmlatinfont\textsuperscript{§~\theparCount}}}
	\pend% ending standard par
      ‚{\tiny $_{lb}$}‚

	  
	  \pstart \leavevmode% starting standard par
	नापि च साऽप्र‚तिप‚न्ना विशेष‚ण‚म्भ‚वितुम‚र्ह‚ति । न हि द‚ण्डाप्र‚तीतौ द‚ण्डीति प्र‚ती‚{\tiny $_{lb}$}‚तिर्भ‚व‚ति ।‚{\tiny $_{६}$}‚ नापि सा क्व‚चिदाश्रिता नीरूप‚त्वात् । न चास‚म्ब‚द्ध‚म्विशेष‚ण‚म्भ‚व‚ति ।‚{\tiny $_{lb}$}‚ नाप्य‚न्य‚निवृत्तिग्र‚ह‚ण‚पुर‚स्स‚रं वृक्षादिषु वृक्ष‚श‚ब्दः प्र‚व‚र्त्त‚तेऽप्र‚तीतिरित्युक्त‚त्वात् ।‚{\tiny $_{lb}$}‚ निवृत्तेर्नीरूप‚त‚याऽप्र‚तिप‚न्न‚त्वेन संकेत‚स्याप्र‚वृत्तेश्च । क‚थं श‚ब्द‚विष‚य‚त्व‚न्त‚स्मा‚{\tiny $_{lb}$}‚‚{\tiny $_{lb}$}‚ \leavevmode\ledsidenote{\textenglish{258/s}}द‚नुभ‚व‚द्वारेण वृक्षोऽयं नावृक्ष इत्येवं निश्च‚य उत्प‚द्य‚ते । तेनान्य‚निवृत्तिः प्र‚तिषेध‚{\tiny $_{lb}$}‚\leavevmode\ledsidenote{\textenglish{95b/PSVTa}} विक‚{\tiny $_{७}$}‚ल्पेन क‚ल्पिता । य‚थासंकेतं च वृक्षादौ श‚ब्दः प्र‚व‚र्त्त‚मानोर्थाद‚न्य‚निवृत्तिमाक्षि‚{\tiny $_{lb}$}‚ति । अन्य‚निवृत्तिविक‚ल्प‚माक्षिप‚तीत्य‚र्थः । तेनान्य‚निवृत्त्या विशिष्टो । व‚स्तुभागो‚{\tiny $_{lb}$}‚ ग‚म्य‚त इत्युच्य‚त इति ।
	{\color{gray}{\rmlatinfont\textsuperscript{§~\theparCount}}}
	\pend% ending standard par
      ‚{\tiny $_{lb}$}‚

	  
	  \pstart \leavevmode% starting standard par
	य‚दि व्यावृत्त‚यः सामान्य‚भूता ब‚हिर्व‚स्तुत्वेन नेष्य‚न्ते । नापि प‚र‚प‚रिक‚ल्पितं‚{\tiny $_{lb}$}‚ सामान्य‚मेवं स‚ति बाह्य‚म्व‚स्त्वेव वाच्य‚माप‚तितं । त‚त्र च दोष इत्याह । \textbf{क‚थ‚न्त‚र्ही}ति ।‚{\tiny $_{lb}$}‚ \textbf{अ‚{\tiny $_{१}$}‚भिन्न‚स्य} निरंश‚स्य \textbf{व‚स्तुनः श‚ब्देन चोद}ने उप‚ल‚क्ष‚णं चैत‚त् लिंगेन प्र‚तिपाद‚ने ।‚{\tiny $_{lb}$}‚ \textbf{त‚स्यैवा}भिन्न‚स्य व‚स्तुन एक‚स्माद् भिन्न‚स्य पुन\textbf{र‚न्य‚तो}पि \textbf{भेदात्} । त‚था हि येन‚{\tiny $_{lb}$}‚ स्व‚भावेंन न श‚ब्दोऽकृत‚काद् भिन्न‚स्तेनैव मूर्त्तानित्य‚त्वाच्च । त‚स्यानंश‚त्वात् । \textbf{अनंश‚स्य}‚{\tiny $_{lb}$}‚ च व‚स्तुनः कृत‚क‚श‚ब्देनै\textbf{क‚स्या}कृत‚काद् \textbf{भेद}स्य भिन्न‚स्य स्व‚भाव‚स्य \textbf{चोद‚ने} । त‚था‚{\tiny $_{lb}$}‚ लिंगेन प्र‚तिपाद‚ने \textbf{स‚र्व‚भे‚{\tiny $_{२}$}‚द‚ग‚ते}स्स‚र्वेभ्यो भिन्न‚स्य स्व‚भाव‚स्य प्र‚तिप‚त्तेः । \textbf{त‚त्रानंशे}‚{\tiny $_{lb}$}‚ व‚स्तुनि \textbf{क‚थं श‚ब्द‚प्र‚माणान्त‚राणि व्य‚र्थानि न स्युः} । एकेन श‚ब्देन चोद‚ने श‚ब्दान्त‚राणां‚{\tiny $_{lb}$}‚ वैय‚र्थ्यं स्यात् । एकेन लिङ्गेन प्र‚तिपाद‚ने प्र‚माणान्त‚राणां वैय‚र्थ्यं स्यात् ।
	{\color{gray}{\rmlatinfont\textsuperscript{§~\theparCount}}}
	\pend% ending standard par
      ‚{\tiny $_{lb}$}‚

	  
	  \pstart \leavevmode% starting standard par
	\textbf{य‚स्मादि}त्यादिना प‚रिह‚र‚ति । त‚स्माद् \textbf{यो येन ध‚र्मेण} विशेषः संप्र‚तीय‚त‚{\tiny $_{lb}$}‚ \href{http://sarit.indology.info/?cref=pv.3.41}{१ । ४४} इत्यादिना प्रागेवेदं चोद्य‚म्प‚रिहृत‚म‚{\tiny $_{३}$}‚धिक‚विधानार्थ‚न्तु पुन‚रुप‚न्यासः ।‚{\tiny $_{lb}$}‚ अर्थेष्वाकारान्त‚र‚स‚मारोपो\textbf{र्थ‚श्ल‚षः} [।] स च प्र‚तिप‚त्तिभेदेनानेकः । \textbf{त‚त्रे}ति बुद्धि‚{\tiny $_{lb}$}‚प्र‚तिभासिनि ध‚र्मिणि बाह्य‚भिन्न‚त‚याऽख्येयास्ते । बाह्य‚त‚याध्य‚स्त‚स्यैव बुद्ध्याकार‚स्य‚{\tiny $_{lb}$}‚ श‚ब्द‚वाच्य‚त्वात् । न पुन‚र्बाह्य‚म्बुद्ध्याकारो वा केव‚लः श‚ब्द‚वाच्यः स्व‚ल‚क्ष‚ण‚त्वात् ।‚{\tiny $_{lb}$}‚ \textbf{त‚त्र} ध‚र्मिणि \textbf{विधिरूप‚त‚या} स्वार्थ‚प्र‚तिप‚त्तिद्वारेणै\textbf{कार्थ‚{\tiny $_{४}$}‚श्लेष‚विच्छेदे}ऽप‚न‚य‚न \textbf{एको‚{\tiny $_{lb}$}‚ ध्व‚निर्व्याप्रिय‚ते} । \href{http://sarit.indology.info/?cref=pv.3.127}{। १३० ॥}
	{\color{gray}{\rmlatinfont\textsuperscript{§~\theparCount}}}
	\pend% ending standard par
      ‚{\tiny $_{lb}$}‚

	  
	  \pstart \leavevmode% starting standard par
	\textbf{लिङ्गं} चैकार्थ‚श्लेष‚विच्छेदे व्याप्रिय‚ते । \textbf{त‚त्र} स्वार्थाभिधान‚द्वारेण स‚मारोप‚{\tiny $_{lb}$}‚व्य‚व‚च्छेदे श‚ब्द‚प्र‚माणान्त‚राणां साफ‚ल्य‚मिति याव‚त् । \textbf{न} पुन‚र्ज्ञानाद् व्य‚तिरिक्त‚{\tiny $_{lb}$}‚\textbf{म्बाह्य‚म्व‚स्तु} स्व‚ल‚क्ष‚णं [।] स्व‚ल‚क्ष‚णाद् वा व्य‚तिरिक्त‚म्बाह्य‚म्व‚स्तु सामान्य‚ल‚क्ष‚ण‚{\tiny $_{lb}$}‚म्वाच्यं \textbf{किञ्व}नास्ति । \textbf{य‚स्य} व‚स्तुनो\textbf{ऽभिधान‚तोऽखिले} व‚स्तुस्व‚{\tiny $_{५}$}‚भावे \textbf{ग‚तिर्भ‚वेत् ।‚{\tiny $_{lb}$}‚ ‚{\tiny $_{lb}$}‚ \leavevmode\ledsidenote{\textenglish{259/s}}व‚स्तुसाम‚र्थ्याद्} व‚स्तुव‚शात् । य‚त‚श्च श‚ब्दः स्वाभिधान‚द्वारेण व्य‚व‚च्छेदं क‚रोति‚{\tiny $_{lb}$}‚ \textbf{त‚तः} कार‚णात् तं तं व्य‚व‚च्छेदं कुर्व\textbf{न्नानाफ‚लः श‚ब्दो} भ‚व‚त्\textbf{येकाधार‚श्च} । क‚थं ।‚{\tiny $_{lb}$}‚ \textbf{अर्थ‚क्रियायोग्य‚म‚ध्य‚व‚साय} । अन‚र्थ‚क्रियाकारिणापि स्वाभास‚म‚र्थ‚क्रियाकारित्वेन‚{\tiny $_{lb}$}‚ स्व‚ल‚क्ष‚ण‚रूप‚त्त्वेनाध्य‚स्येत्य‚र्थः । \textbf{त‚त्रै}वेति बुद्धिप्र‚तिभासे बाह्य‚{\tiny $_{७}$}‚त‚याध्य‚स्ते । किंभूते । \leavevmode\ledsidenote{\textenglish{96a/PSVTa}}‚{\tiny $_{lb}$}‚ \textbf{तैस्तैर्भ्रान्तिकार‚णैः संसृष्ट‚रूप इव भा}ति । \href{http://sarit.indology.info/?cref=pv.3.128}{। १३१ ॥}
	{\color{gray}{\rmlatinfont\textsuperscript{§~\theparCount}}}
	\pend% ending standard par
      ‚{\tiny $_{lb}$}‚

	  
	  \pstart \leavevmode% starting standard par
	स‚दृशाप‚रोत्प‚त्त्यादिभि\textbf{र्भ्रान्तिहेतु}भिर्नित्याद्याकारेण \textbf{संसृष्ट‚रूप इव प्र‚तिभास‚{\tiny $_{lb}$}‚मा}ने । \textbf{ते}ऽनित्यादि\textbf{श‚ब्दाः} य\textbf{थासंके}तं य‚स्य य‚स्य स‚मारोप‚स्य \textbf{व्य‚व‚च्छेदार्थं} ।‚{\tiny $_{lb}$}‚ स्व‚प्र‚तिभासे संकेतः कुतः । त‚स्य त‚स्य प्र‚तियोगिनो व्य‚व‚च्छेदाय \textbf{व्याप्रिय‚न्}ते ।‚{\tiny $_{lb}$}‚ संकेतानुरूप‚मेव प्र‚तिपाद‚{\tiny $_{१}$}‚य‚न्नाह । \textbf{न चेत्}यादि । ह्य‚र्थे च‚श‚ब्दः । य‚तो \textbf{य‚थासंकेतं}‚{\tiny $_{lb}$}‚ \textbf{व्यापार‚स्त‚तो न ह्ये}क‚श‚ब्द\textbf{साध्}यं \textbf{व्य‚व‚च्छेद‚म‚न्यः} श‚ब्दः \textbf{क‚रो}ति । किङ्कार‚णं [।]‚{\tiny $_{lb}$}‚ \textbf{संकेत‚प्र‚तिनिय‚मा}त् । एकैक‚व्य‚व‚च्छेदार्थ‚म्बुद्ध्याकारोऽविभ‚क्त‚बाह्य‚रूपे श‚ब्द‚निवेश‚{\tiny $_{lb}$}‚नात् । व्य‚व‚हार‚कालेपि स्वार्थाभिधान‚द्वारेण त‚न्त‚मेव व्य‚व‚च्छेदं प्र‚त्याय‚य‚तीत्य‚र्थः ।‚{\tiny $_{lb}$}‚ श‚ब्द‚ग्र‚ह‚ण‚मुप‚ल‚क्ष‚ण‚मेवं लिङ्ग‚{\tiny $_{२}$}‚म‚पीति द्र‚ष्ट‚व्यं ।
	{\color{gray}{\rmlatinfont\textsuperscript{§~\theparCount}}}
	\pend% ending standard par
      ‚{\tiny $_{lb}$}‚

	  
	  \pstart \leavevmode% starting standard par
	न‚न्व‚ध्य‚व‚सित‚बाह्य‚रूप‚त्वाच्छ‚ब्दार्थ‚स्य त‚त‚श्च श‚ब्देन‚न्य‚ल‚क्ष‚ण‚स्य स‚र्वात्म‚ना‚{\tiny $_{lb}$}‚ विष‚यीक‚र‚णात् क‚थं न श‚ब्दान्त‚राणाम्वैय‚र्थ्य‚मित्याह \textbf{न चे}त्यादि । अव‚धार‚ण‚श्च‚{\tiny $_{lb}$}‚ श‚ब्दः । \textbf{नैव विच्छिन्नं} ज्ञानांशाद् भिन्नं \textbf{किञ्चिद् व‚स्तु} । स्व‚ल‚क्ष‚णं स्व‚ल‚क्ष‚णाद्‚{\tiny $_{lb}$}‚ व्य‚तिरिक्तं सामान्य‚ल‚क्ष‚ण‚म‚न्य‚निवृत्तिल‚क्ष‚णं व्\textbf{आक्षिप्य‚ते} गृह्य‚ते श‚ब्देन लिङ्गेन वा ।‚{\tiny $_{lb}$}‚ \textbf{य‚स्या‚{\tiny $_{३}$}‚भिधानाद् व‚स्तुब‚लेने}ति व‚स्तुग्र‚हे निरंश‚त्वाद् व‚स्तुनः \textbf{स‚र्व‚था ग‚तिः स्यात्} ।‚{\tiny $_{lb}$}‚ ध‚र्माणान्त‚तो व्य‚तिरेकात् । व्य‚तिरेकेप्युपाधीनां नानोपाध्युप‚काराङ्ग‚श‚क्त्य‚भिन्ना‚{\tiny $_{lb}$}‚त्म‚नो ग्र‚ह \href{http://sarit.indology.info/?cref=}{१ । ५४} इत्यादिना स‚र्व‚थाग्र‚ह‚ण‚स्योक्त‚त्वात् । क‚स्माच्छ‚ब्दैर्विच्छिन्न‚{\tiny $_{lb}$}‚म्व‚स्तुनाक्षिप्य‚त इत्य‚त आह । श\textbf{ब्दाना}मित्यादि ।
	{\color{gray}{\rmlatinfont\textsuperscript{§~\theparCount}}}
	\pend% ending standard par
      ‚{\tiny $_{lb}$}‚

	  
	  \pstart \leavevmode% starting standard par
	एत‚दुक्त‚म्भ‚व‚ति । य‚तो बुद्ध्याकार‚म‚बाह्य‚म्बाह्य‚म‚ध्य‚व‚स्य‚{\tiny $_{४}$}‚न्ति श‚ब्दास्त‚तो‚{\tiny $_{lb}$}‚ ‚{\tiny $_{lb}$}‚ \leavevmode\ledsidenote{\textenglish{260/s}}विच्छिन्न‚व‚स्तुग्राह‚का इव भ‚व‚न्तीत्य‚र्थः । बुद्धेर्विप्ल‚व‚श्च बुद्ध्याकार‚स्य ब‚हीरूप‚त‚या‚{\tiny $_{lb}$}‚ ग्र‚हः । \textbf{त‚द्विष‚य‚त्वाच्छ‚ब्दानां} । बुद्धिविप्ल‚वेपि वाच्य\textbf{व‚स्तुसाम‚र्थ्यांद‚खिले ग‚तिः} ।‚{\tiny $_{lb}$}‚ किन्नेति चेदाह । \textbf{त‚त्र चेत्}यादि । \textbf{त‚त्र} बुद्धिविप्ल‚वे\textbf{ऽव‚स्तुनि व‚स्तुसाम‚र्थ्याभावात्} ।
	{\color{gray}{\rmlatinfont\textsuperscript{§~\theparCount}}}
	\pend% ending standard par
      ‚{\tiny $_{lb}$}‚

	  
	  \pstart \leavevmode% starting standard par
	य‚द्य‚पि बुद्ध्याकारो ज्ञान‚स्व‚ल‚क्ष‚ण‚त्वाद्व‚स्तु । त‚थाप्य‚सौ श‚ब्दैर्विक‚ल्पैर्वा बाह्या‚{\tiny $_{५}$}‚‚{\tiny $_{lb}$}‚भिन्न‚त‚याध्य‚स्तोऽव‚स्त्वेव । तेन श‚ब्दो विक‚ल्पो वा न स्व‚ल‚क्ष‚ण‚विष‚यो य‚थाव‚देक‚{\tiny $_{lb}$}‚स्यापि बाह्य‚स्य ज्ञानाकार‚स्य वाऽग्र‚ह‚णादिति ।
	{\color{gray}{\rmlatinfont\textsuperscript{§~\theparCount}}}
	\pend% ending standard par
      ‚{\tiny $_{lb}$}‚

	  
	  \pstart \leavevmode% starting standard par
	य‚दि बुद्धिविप्ल‚व‚विष‚या एव स‚र्व‚श‚ब्दाः क‚थं कृत‚कानित्यादिश‚ब्दानान्त‚थाभूते‚{\tiny $_{lb}$}‚ व‚स्तुन्य‚व्य‚भिचार इत्य‚त आह । \textbf{त‚थाभूते}त्यादि । त‚त‚स्त‚तोऽकृत‚क‚नित्यादेर्भिन्न‚स्या‚{\tiny $_{lb}$}‚र्थ‚स्यानुभ‚व\textbf{द्वारेणे}त्य‚र्थः । \textbf{अयं व्य‚{\tiny $_{६}$}‚व‚हा}र इति स‚म्ब‚न्धः । किम्भूतः [।] \textbf{नानैके}त्यादि ।‚{\tiny $_{lb}$}‚ नाना एक‚श्च नानैकं । त‚च्च \textbf{ध‚र्म}श्चेति क‚र्म‚धार‚यः । त‚तो \textbf{भेदाभे}द‚श‚ब्दाभ्यान्त्रिप‚दो‚{\tiny $_{lb}$}‚ द्व‚न्द्वः । नानैक‚ध‚र्म\textbf{भेदाभेदा} एव बुद्धौ \textbf{प्र‚तिभास}न्त इति प्र‚तिभासास्त एव विप्ल‚वो‚{\tiny $_{lb}$}‚ \leavevmode\ledsidenote{\textenglish{96b/PSVTa}} भ्रान्त‚त्वात् । \textbf{त‚द‚नुसारी} तेनाकारेण प्र‚वृत्तः । \textbf{इति} हेतोस्\textbf{त‚स्य} व्य‚व‚हार‚स्य \textbf{त‚त्प्र‚तिब‚न्धे}‚{\tiny $_{७}$}‚‚{\tiny $_{lb}$}‚ त‚मिँस्त‚थाभूते स्व‚ल‚क्ष‚णे पार‚म्प‚र्येणोत्प‚त्तिप्र‚तिब‚न्धे स‚ति \textbf{त‚द‚व्य‚भिचारः} ।‚{\tiny $_{lb}$}‚ व‚स्त्व‚व्य‚भिचारः । य‚था हि कृत‚काद्याकाराः श‚ब्दा विप्ल‚वास्त‚थाव‚स्तूनाम‚पि‚{\tiny $_{lb}$}‚ कृत‚कादिरूपेण प‚र‚मार्थ‚तोव‚स्थान‚मित्य‚नेनाकारेणाव्\textbf{य‚भिचारो} द्र‚ष्ट‚व्यः ।
	{\color{gray}{\rmlatinfont\textsuperscript{§~\theparCount}}}
	\pend% ending standard par
      ‚{\tiny $_{lb}$}‚

	  
	  \pstart \leavevmode% starting standard par
	एत‚देवाह । \textbf{त‚तोपीत्या}दि । त‚तो बुद्धिविप्ल‚व‚विष‚याच्छ‚ब्दाद् व्य‚व‚हारात्‚{\tiny $_{lb}$}‚ प‚रार्थानुमान‚ल‚क्ष‚णा‚{\tiny $_{१}$}‚द् \textbf{वित‚थादि}ति भ्रान्तात् । \textbf{प्र‚वृत्त‚स्यान्ते} प्र‚वृत्तिप‚रिस‚माप्तौ‚{\tiny $_{lb}$}‚ \textbf{त‚थाभूत एव} कृत‚कादिरूप \textbf{एव व‚स्तुनि} न‚त्व‚कृत‚कादिरूपे । अनेनाव्य‚भिचार‚स्व‚रूप‚{\tiny $_{lb}$}‚मुक्तं । \textbf{ज्ञान‚स‚म्वादात्} स्व‚ल‚क्ष‚ण‚ग्राहिज्ञानोत्प‚त्तेः श‚ब्द‚स्य वा ज्ञान‚स्य स‚म्वादात् ।
	{\color{gray}{\rmlatinfont\textsuperscript{§~\theparCount}}}
	\pend% ending standard par
      ‚{\tiny $_{lb}$}‚

	  
	  \pstart \leavevmode% starting standard par
	अस्मिन्नेवान्यापोहे श‚ब्दार्थे सामानाधिक‚र‚ण्यं सिध्य‚ति न तु व‚स्तुनीत्याह । \textbf{न‚{\tiny $_{lb}$}‚ पुन‚र्भि}न्नेत्यादि । \textbf{भि}न्ना \textbf{आ}कारा जाति‚{\tiny $_{२}$}‚गुणाद‚यः श‚ब्द‚प्र‚वृत्तिनिमित्त‚भूताः ।‚{\tiny $_{lb}$}‚ त‚द्\textbf{ग्राहिणां ज्ञान‚श‚ब्दानामेक‚व‚स्तुविष‚य‚त्वात्} । श‚ब्द एकाधार इति स‚म्ब‚न्धः ।‚{\tiny $_{lb}$}‚ य‚दि हि भिन्नाकारं ज्ञान‚मेक‚व‚स्तुविष‚य‚म्प्र‚व‚र्त्तेत त‚त‚स्त‚द‚नुसारेण श‚ब्दोपि त‚था‚{\tiny $_{lb}$}‚ स्यात् । त‚था च \textbf{नानाफ}लः \textbf{श‚ब्द एकाधा}र इति भ‚वेत्सामानाधिक‚र‚ण्यं त‚च्च \textbf{नास्ति‚{\tiny $_{lb}$}‚ व्याघातात्} । त‚था हि नीलोत्प‚ल‚श‚ब्द‚योरेकं वा‚{\tiny $_{३}$}‚ व‚स्तुवाच्यं स्याद‚नेक‚म्वा । आद्ये‚{\tiny $_{lb}$}‚ प‚क्षे एकेनैव श‚ब्देन निरंश‚स्य व‚स्तुनः स‚र्वात्म‚नाभिधानात् । द्वितीय‚स्य श‚ब्द‚स्याप्र‚वृत्तिः‚{\tiny $_{lb}$}‚ स्यात् प्र‚वृत्तौ वा प‚र्याय‚तेति न नानाफ‚ल‚त्वं । द्वितीये प‚क्ष एकाधार‚ता नास्ति ।‚{\tiny $_{lb}$}‚ ‚{\tiny $_{lb}$}‚ \leavevmode\ledsidenote{\textenglish{261/s}}घ‚ट‚प‚टादिश‚ब्द‚व‚न्नानाविष‚य‚त्वात् ।
	{\color{gray}{\rmlatinfont\textsuperscript{§~\theparCount}}}
	\pend% ending standard par
      ‚{\tiny $_{lb}$}‚

	  
	  \pstart \leavevmode% starting standard par
	अथ म‚तं [।] नील‚श‚ब्दो नील‚गुण‚विशिष्ट‚न्द्र‚व्य‚माह । उत्प‚ल‚श‚ब्दोप्युत्प‚ल‚{\tiny $_{lb}$}‚जातिविशि‚{\tiny $_{४}$}‚ष्ट‚न्त‚देव द्र‚व्य‚माह । अतो विशेष‚ण‚योर्भेदान्नानाफ‚लः श‚ब्दो विशेष्या‚{\tiny $_{lb}$}‚भेदादेकाधार इति ।
	{\color{gray}{\rmlatinfont\textsuperscript{§~\theparCount}}}
	\pend% ending standard par
      ‚{\tiny $_{lb}$}‚

	  
	  \pstart \leavevmode% starting standard par
	त‚द‚प्य‚स‚त् । य‚तो नील‚गुणेन विशिष्ट‚द्र‚व्य‚न्नील‚श‚ब्देनाभिधीय‚मानं स‚र्वात्म‚{\tiny $_{lb}$}‚नाभिधीय‚ते निरंश‚त्वात् । त‚तः कोप‚रो द्र‚व्य‚स्योत्प‚ल‚जातिविशिष्ट आत्मान‚भि‚{\tiny $_{lb}$}‚हितोस्ति य‚दुत्प‚ल‚श‚ब्देनाभिधीयेतेति [।] त‚थैव प‚र्याय‚ता स्याद‚थ विशेष‚ण‚भेदाद्‚{\tiny $_{lb}$}‚ वि‚{\tiny $_{५}$}‚शेष्य‚द्र‚व्य‚स्य भेद‚स्त‚दाप्येकाधार‚ता न स्यात् । अपोह‚वादिन‚स्त्व‚य‚म‚दोष इत्याह ।‚{\tiny $_{lb}$}‚ \textbf{य‚थाव‚र्ण्णित} इत्यादि । \textbf{बुद्धिप्र‚तिभा}स \textbf{आश्र}यो य‚स्य श‚ब्दार्थ‚स्य स त‚थोक्तः ।
	{\color{gray}{\rmlatinfont\textsuperscript{§~\theparCount}}}
	\pend% ending standard par
      ‚{\tiny $_{lb}$}‚

	  
	  \pstart \leavevmode% starting standard par
	य‚था च न दोष‚स्त‚था प्र‚तिपाद‚य‚न्नाह । \textbf{विच्छेद}मित्यादि । एको नील‚श‚ब्द‚{\tiny $_{lb}$}‚ \textbf{एकं} व्य‚व‚च्छेद‚म‚नील‚व्य‚व‚च्छिन्नं नील‚स्व‚भावं \textbf{सूच‚य‚न्न‚न्य}म‚नुत्प‚ल‚व्य‚व‚च्छिन्न‚मुत्प‚ल‚{\tiny $_{lb}$}‚स्व‚भाव\textbf{म‚प्र‚तिक्षिप्य‚{\tiny $_{६}$}‚ व‚र्त्त‚ते} न निराकारोतीत्य‚र्थः । \href{http://sarit.indology.info/?cref=pv.3.129}{। १३२ ॥}
	{\color{gray}{\rmlatinfont\textsuperscript{§~\theparCount}}}
	\pend% ending standard par
      ‚{\tiny $_{lb}$}‚

	  
	  \pstart \leavevmode% starting standard par
	\textbf{स} इत्य‚नुत्प‚ल‚व्य‚व‚च्छिन्नः स्व‚भावः । \textbf{तेन} नील‚श‚ब्देन \textbf{व्याप्}त आक्रान्तः आक्षिप्तः‚{\tiny $_{lb}$}‚ स‚न्नुत्प‚ल‚श‚ब्द‚प्र‚योगे बुद्धावे\textbf{क‚त्वेन प्र‚तिभास}ते । एक‚स्यैव ध‚र्मिणः व्य‚व‚च्छेद‚द्व‚या‚{\tiny $_{lb}$}‚यात‚नीलोत्प‚ल‚ध‚र्म‚द्व‚य‚युक्त‚स्य विक‚ल्प‚बुद्धौ प्र‚तिभास‚नात् । य‚दा चैव\textbf{न्त‚दा सामा‚{\tiny $_{lb}$}‚नाधिक‚र‚ण्यं स्याद् बुद्ध्य‚नुरोध‚तः} ।
	{\color{gray}{\rmlatinfont\textsuperscript{§~\theparCount}}}
	\pend% ending standard par
      ‚{\tiny $_{lb}$}‚

	  
	  \pstart \leavevmode% starting standard par
	एत‚दुक्त‚म्भ‚व‚{\tiny $_{७}$}‚ति [।] नील‚श‚ब्द‚प्र‚योगाद् बुद्धिप्र‚तिभासी ध‚र्मी नील‚रूप एव \leavevmode\ledsidenote{\textenglish{97a/PSVTa}}‚{\tiny $_{lb}$}‚ प्र‚तिभास‚ते । त‚त्रार्थाद‚नीलं व्याव‚र्त्त्य‚ते न त्व‚नुत्प‚ल‚व्य‚व‚च्छिन्नः स्व‚भाव उत्प‚ल‚{\tiny $_{lb}$}‚श‚ब्द‚प्र‚योगाद‚प्युत्प‚ल‚रूप‚त‚या प्र‚तीय‚मानोनुत्प‚लं व्याव‚र्त्त्य‚ते न त्व‚नील‚व्यावृत्तः‚{\tiny $_{lb}$}‚ स्व‚भावः । श‚ब्द‚द्व‚य‚प्र‚योगे तु नीलोत्प‚ल‚ध‚र्म‚द्व‚य‚युक्तैक‚ध‚र्मिप्र‚तिभासिनी विक‚ल्प‚बुद्धि‚{\tiny $_{lb}$}‚रुत्प‚द्य‚ते त‚तो नानाफ‚लः‚{\tiny $_{१}$}‚ श‚ब्द एकाधारो भ‚व‚तीति भ‚वेद् बुद्ध्य‚नुरोधेन सामा‚{\tiny $_{lb}$}‚नाधिक‚र‚ण्य‚मिति ।
	{\color{gray}{\rmlatinfont\textsuperscript{§~\theparCount}}}
	\pend% ending standard par
      ‚{\tiny $_{lb}$}‚

	  
	  \pstart \leavevmode% starting standard par
	तेन य‚दु द्यो त क रे णोच्य‚ते [।] य‚स्य चान्यापोहः श‚ब्दार्थ‚स्तेनानीलानुत्प‚ल‚{\tiny $_{lb}$}‚व्युदासौ क‚थं स‚मानाधिक‚र‚णाविति व‚क्त‚व्यं । य‚स्य पुन‚र्विधीय‚मानः श‚ब्दार्थ‚स्य‚{\tiny $_{lb}$}‚ जातिगुण‚विशिष्टं नीलोत्प‚ल‚श‚ब्दाभ्यां द्र‚व्य‚म‚भिधीय‚ते जातिगुणौ च द्र‚व्ये व‚र्त्तेते ।‚{\tiny $_{lb}$}‚ ‚{\tiny $_{lb}$}‚ \leavevmode\ledsidenote{\textenglish{262/s}}न पुन‚र‚नीलानुत्प‚ल‚व्युदा‚{\tiny $_{२}$}‚सौ । त‚स्मात् स‚मानाधिक‚र‚णार्थो नास्तीति\edtext{}{\edlabel{pvsvt_262-1}\label{pvsvt_262-1}\lemma{ति}\Bfootnote{\href{http://sarit.indology.info/?cref=nv}{ Nyāyavārtika. }}} निर‚स्तं ।‚{\tiny $_{lb}$}‚ विधीय‚मान‚स्य श‚ब्दार्थ‚स्याभ्युप‚ग‚मात् । \href{http://sarit.indology.info/?cref=pv.3.130}{। १३३ ॥}
	{\color{gray}{\rmlatinfont\textsuperscript{§~\theparCount}}}
	\pend% ending standard par
      ‚{\tiny $_{lb}$}‚

	  
	  \pstart \leavevmode% starting standard par
	किं च पुनः [।] \textbf{श‚ब्द‚स्}य स्वार्थाभिधान‚द्वारेण स‚मारोप‚व्\textbf{य‚व‚च्छेद‚क‚र‚णे}भ्युप‚ग‚म्य‚{\tiny $_{lb}$}‚माने । य‚द्वा \textbf{व्य‚व‚च्छेद‚क‚र‚णे} व्य‚व‚च्छिन्न‚स्व‚भाव‚विष‚यीक‚र‚णे । लेश‚तो \textbf{व‚स्तुध‚र्म‚स्य}‚{\tiny $_{lb}$}‚ व‚स्तुस्व‚भाव‚स्य विजातीय‚व्यावृत्त‚स्य \textbf{संस्प‚र्शः स्यात्} प्राप्तिल‚क्ष‚णः [।] किं कार‚णं‚{\tiny $_{lb}$}‚ [।] \textbf{स‚त्त्य}‚{\tiny $_{३}$}‚मिति विद्य‚मानः \textbf{स ह्य}ध्य‚व‚सीय‚मानः व्य‚व‚च्छिन्नः स्व‚भावः । \textbf{त‚त्र} व‚स्तुनीति‚{\tiny $_{lb}$}‚ कृत्वा \textbf{नैक‚व‚स्त्व‚भिधायिनि} श‚ब्देभ्युप‚ग‚म्य‚माने व‚स्तुध‚र्म‚स्य संस्प‚र्शः सामान्य‚स्यैव‚{\tiny $_{lb}$}‚ व‚स्तुनोऽभावात् ।
	{\color{gray}{\rmlatinfont\textsuperscript{§~\theparCount}}}
	\pend% ending standard par
      ‚{\tiny $_{lb}$}‚

	  
	  \pstart \leavevmode% starting standard par
	क‚थ‚मिति चेत् । \textbf{बुद्धा}वित्यादि । उप‚ल‚ब्धिल‚क्ष‚ण‚प्राप्त‚स्यानुप‚ल‚म्भाद‚स‚त्त्व‚मिति‚{\tiny $_{lb}$}‚ याव‚त् । य‚त एव‚म्व‚स्तुनि श‚ब्दार्थे दोष\textbf{स्तेन} कार‚णे\textbf{नान्यापोह‚विष‚या} विक‚ल्प‚{\tiny $_{४}$}‚‚{\tiny $_{lb}$}‚बुद्धिप्र‚तिभास‚विष‚याः \textbf{श‚ब्दा बुद्ध‚य‚श्च} प्रोक्ता आचार्य दि ङ् ना गे न । किम्भूता‚{\tiny $_{lb}$}‚ बुद्ध‚यः \textbf{सामान्य‚गोच‚रा} विक‚ल्पिका इत्य‚र्थः । बुद्धीनामेवैत‚द् विशेष‚णं न तु श‚ब्दा‚{\tiny $_{lb}$}‚नान्तेषां सामान्य‚विष‚य‚व्य‚भिचारात् । किङ्कार‚णं । व\textbf{स्तुन्येषां} श‚ब्दानां विक‚ल्पानां‚{\tiny $_{lb}$}‚ च \textbf{स‚म्भ‚वा}त् ।
	{\color{gray}{\rmlatinfont\textsuperscript{§~\theparCount}}}
	\pend% ending standard par
      ‚{\tiny $_{lb}$}‚

	  
	  \pstart \leavevmode% starting standard par
	\textbf{य‚दि हीत्या}दिना व्याच‚ष्टे । \textbf{व‚स्त्वेव य‚दि विष‚यीक्रियेत} गृह्येत न तु विधिरूपे‚{\tiny $_{५}$}‚‚{\tiny $_{lb}$}‚णाध्य‚व‚सीयेत । \textbf{सोय}मित्य‚न‚न्त‚रोक्तः \textbf{स‚र्वार्थानां स‚र्वेणाकारे}ण व‚स्तुसाम‚र्थ्यात्‚{\tiny $_{lb}$}‚ \textbf{प्र‚तीतिप्र‚संगः । आदि}श‚ब्दाद् अविशेष‚ण‚विशेष्य‚भाव‚प्र‚माणान्त‚राप्र‚वृत्यादिप‚रिग्र‚हः ।‚{\tiny $_{lb}$}‚ \textbf{प्र‚णेता} आचार्य दि ङ् ना गः । \textbf{एता}विति बुद्धिश‚ब्दौ । अन्योपोह्य‚तेनेनेति विक‚ल्पाकार‚{\tiny $_{lb}$}‚ उच्य‚ते त‚द्विष‚यौ । त‚था भिन्नाकाराभिर्बुद्धिभिरेक‚म्व‚स्तु य‚दि विष‚यीक्रि‚{\tiny $_{६}$}‚येत ।‚{\tiny $_{lb}$}‚ त‚थैव श‚ब्देन चाभिधीयेत । त‚दा भिन्न‚फ‚ल‚योः श‚ब्द‚योरेक‚त्र द्र‚व्ये वृत्त‚त्वात् स्यात्‚{\tiny $_{lb}$}‚ सामानाधिक‚र‚ण्य‚मेत‚त्तु न स‚म्भ‚व‚ति । त‚था हि त‚च्छ‚ब्द‚वाच्यं सामान्यं स्व‚ल‚क्ष‚णाद‚{\tiny $_{lb}$}‚‚{\tiny $_{lb}$}‚ ‚{\tiny $_{lb}$}‚ \leavevmode\ledsidenote{\textenglish{263/s}}भिन्नं भिन्न‚म्वा स्यात् । त‚त्राद्ये प‚क्षे । \textbf{एक‚त्वाद् व‚स्तुरूप}स्य \textbf{भिन्न‚रू}पा सामान्य‚{\tiny $_{lb}$}‚विशेषाकारा भिन्ना \textbf{म‚तिः कुतः} । \href{http://sarit.indology.info/?cref=pv.3.133}{। १३६ ॥}
	{\color{gray}{\rmlatinfont\textsuperscript{§~\theparCount}}}
	\pend% ending standard par
      ‚{\tiny $_{lb}$}‚

	  
	  \pstart \leavevmode% starting standard par
	स्व‚ल‚क्ष‚णाच्च सामान्य‚स्याव्य‚तिरेके । शाव‚लेयात्म‚को‚{\tiny $_{७}$}‚ भेदो य‚तो बाहुले- \leavevmode\ledsidenote{\textenglish{97b/PSVTa}}‚{\tiny $_{lb}$}‚ यात्म‚काद् भेदाद् व्याव‚र्त्त‚ते । बाहुलेये शाव‚लेयात्म‚कं गोत्व‚म‚न्वेतीत्येक‚स्यार्थ‚स्यै‚{\tiny $_{lb}$}‚काधिक‚र‚णाव‚न्व‚य‚व्य‚तिरेकौ प्राप्नुतः । त‚च्चायुक्त‚मित्याह । \textbf{अन्व‚य‚व्य‚तिरेका}‚{\tiny $_{lb}$}‚वित्यादि । \textbf{एकोर्थो गोच}रो विष‚यो य‚योर‚न्व‚य‚व्य‚तिरेक‚योस्तौ त‚थोक्तौ ।
	{\color{gray}{\rmlatinfont\textsuperscript{§~\theparCount}}}
	\pend% ending standard par
      ‚{\tiny $_{lb}$}‚

	  
	  \pstart \leavevmode% starting standard par
	\textbf{त‚देक}मित्यादिना व्याच‚ष्टे । एक‚त्वादेवा\textbf{नंश}मेक‚स्यांशाभावात् । आक्रिय‚त‚{\tiny $_{lb}$}‚ इत्याकारो बुद्धिप्र‚ति‚{\tiny $_{१}$}‚भासः । त‚स्य \textbf{भेद}स्त‚द्\textbf{आश्र‚या}द् भेद‚स्य । व‚स्तुनानात्व‚स्य ।‚{\tiny $_{lb}$}‚ य‚दि भिन्नाकाराभिर्बुद्धिभिर्गृह्येत भिन्नं स्यात् । न चैव‚न्त‚स्य चाभावाद् व‚स्तुनः ।‚{\tiny $_{lb}$}‚ सामान्य‚विशेषोभ‚यात्म‚क‚त्वाद् व‚स्तुन एक‚स्यापि भिन्नाकार‚बुद्धिग्राह्य‚त्व‚मिति‚{\tiny $_{lb}$}‚ चेदाह । \textbf{त‚दात्म‚नोपी}त्यादि । स्व‚ल‚क्ष‚णात्म‚न‚स्त\textbf{देक‚योग‚क्षेम‚त्वात्} । स्व‚ल‚क्ष‚णेनैक‚{\tiny $_{lb}$}‚योग‚क्षेम‚त्वात् त‚द्व‚देवाभिन्न‚त्वं । \textbf{त‚दि}ति त‚स्मा\textbf{द‚यं}‚{\tiny $_{२}$}‚ सामानाधिक‚र‚ण्यादिर्न‚{\tiny $_{lb}$}‚ स्यादिति स‚म्ब‚न्धः । किं कार‚ण‚म् [।] \textbf{अन्योन्यार्थ‚प‚रिहारेण} प‚र‚स्प‚रार्थ‚प‚रि‚{\tiny $_{lb}$}‚हारेण भिन्न‚प्र‚वृत्तिनिमित्त‚त्वेनेति याव‚त् । श‚ब्द‚यो\textbf{रेक‚विष‚य‚यो}रेक‚द्र‚व्याधार‚यो‚{\tiny $_{lb}$}‚\textbf{र्वृत्त्य‚स‚म्भ‚वा}त् ।
	{\color{gray}{\rmlatinfont\textsuperscript{§~\theparCount}}}
	\pend% ending standard par
      ‚{\tiny $_{lb}$}‚

	  
	  \pstart \leavevmode% starting standard par
	\textbf{अन्व‚ये}त्यादि श्लोक‚भागं \textbf{न चे}त्यादिना व्याच‚ष्टे । य\textbf{दैक} एव व\textbf{स्त्वात्मा} स्व‚ल‚{\tiny $_{lb}$}‚क्ष‚णं सामान्यं च त‚दा शाव‚लेय‚स्व‚ल‚क्ष‚ण‚स्य गोत्व‚सामान्यात्म‚क‚{\tiny $_{३}$}‚त्वात् । \textbf{त‚त्रै}व बाहु‚{\tiny $_{lb}$}‚लेये \textbf{वृत्तिः} पुनः स्व‚ल‚क्ष‚णात्म‚क‚त्वात् त‚त्रै\textbf{वावृ}त्तिस्त‚स्मिन् काले प्र‚युक्ता । सा‚{\tiny $_{lb}$}‚ \textbf{चायुक्ता । व्याघातात्} । प्र‚माण‚बाधित‚त्वात् । स्व‚ल‚क्ष‚णाद‚भिन्न‚त्वा\textbf{न्नैव} सामान्य‚{\tiny $_{lb}$}‚\textbf{म‚न्य‚त्र व‚र्त्त}ते । त‚तो नैक‚स्यैक‚त्र वृत्त्य‚वृत्ती इति चेदाह । \textbf{न चेत्}यादि । \textbf{सामान्य‚स्ये}‚{\tiny $_{lb}$}‚त्यादि । त‚तो नैक‚स्य वृत्त्य‚वृत्ती इत्य‚भिप्रायः ।
	{\color{gray}{\rmlatinfont\textsuperscript{§~\theparCount}}}
	\pend% ending standard par
      ‚{\tiny $_{lb}$}‚

	  
	  \pstart \leavevmode% starting standard par
	\textbf{नेत्}या चा र्यः सामान्य‚विशेष‚योः सां ख्या‚{\tiny $_{४}$}‚दिद‚र्श‚नेन \textbf{भेदाभावात् । त‚दि}त्याद्य‚{\tiny $_{lb}$}‚‚{\tiny $_{lb}$}‚ \leavevmode\ledsidenote{\textenglish{264/s}}स्यैव स‚म‚र्थ‚नं । \textbf{त‚द्धि} व‚स्त्\textbf{वेक‚रूप‚मे}कात्म‚कं स‚त् । \textbf{सामान्य‚म्वा भ‚वेद् विशेषो वे}ति ।‚{\tiny $_{lb}$}‚ सामान्याद् विशेष‚स्याव्य‚तिरेकात् सामान्य‚मेव स्यात् । विशेष एव वा विशेषा‚{\tiny $_{lb}$}‚द‚व्य‚तिरेकात् सामान्य‚स्य । न त्वेकं द्विरूपं । य‚तो \textbf{न ह्य‚स}ति \textbf{रूप‚भेदेऽयं प्र‚विभाग}‚{\tiny $_{lb}$}‚ इति सामान्यं विशेष इति च । स‚ति वा प्र‚विभागे सामान्य‚विशे‚{\tiny $_{५}$}‚ष‚यो\textbf{र‚व्य‚तिरेको न‚{\tiny $_{lb}$}‚ स्यादित्युक्तं प्राक् । त‚दि}ति त‚स्मा\textbf{द‚य‚म्}व‚स्त्वात्मा । \textbf{अविभाग} इत्य‚नंशः ।
	{\color{gray}{\rmlatinfont\textsuperscript{§~\theparCount}}}
	\pend% ending standard par
      ‚{\tiny $_{lb}$}‚

	  
	  \pstart \leavevmode% starting standard par
	य‚दि सामान्य‚मेव त‚दान्वियाद् व्य‚क्त्य‚न्त‚र‚ङ्ग‚च्छेत् । अथ विशेषात्म‚क एव ।‚{\tiny $_{lb}$}‚ त‚दा न वान्वियात् । एक‚स्य तु विरुद्ध‚ध‚र्म‚द्व‚यास‚म्भ‚व इति याव‚त् । त‚देवाह [।]‚{\tiny $_{lb}$}‚ \textbf{न पुन}रित्यादि ।
	{\color{gray}{\rmlatinfont\textsuperscript{§~\theparCount}}}
	\pend% ending standard par
      ‚{\tiny $_{lb}$}‚

	  
	  \pstart \leavevmode% starting standard par
	\textbf{योपी}ति वै शे षि कादिः । द्र‚व्याद् \textbf{भिन्न‚मेव सामान्}यं श‚ब्द‚वाच्य‚माह । व्य‚क्तेः‚{\tiny $_{lb}$}‚ सामान्या‚{\tiny $_{६}$}‚न्\textbf{आम्भे}देऽभ्युप‚ग‚म्य‚माने\textbf{ऽभेद‚व्य‚व‚हाराः} सामानाधिक‚र‚ण्यादिव्य‚व‚हाराः‚{\tiny $_{lb}$}‚ \textbf{स्युर‚निब‚न्ध‚नाः । य‚थास्व}मिति । वीप्सायाम‚व्य‚यीभावः । त‚था हि नीलोत्प‚लादि\textbf{श‚ब्दा‚{\tiny $_{lb}$}‚ य‚थास्व}न्नीलोत्प‚लादिम्प‚र‚स्प‚र‚भिन्नं सामान्य‚माहुः । य‚दापि नील‚श‚ब्दो नील‚गुणाभि‚{\tiny $_{lb}$}‚\leavevmode\ledsidenote{\textenglish{98a/PSVTa}} धायीष्य‚ते त‚दापि भिन्नार्थाभिधान‚म‚स्त्येव । \textbf{ए}को ध‚र्मी \textbf{अर्थो} विष‚यो य‚स्या‚{\tiny $_{lb}$}‚ \textbf{बु}द्धेस्सा [।] \textbf{एकार्था} चासौ \textbf{बुद्धि}श्चेति क‚र्म‚धार‚यः पुम्व‚द्भाव‚श्च । अस्या \textbf{आश्र‚याः}‚{\tiny $_{lb}$}‚ कार‚णं \textbf{क‚थं स्युः} । त‚त‚श्च सामानाधिक‚र‚ण्यं न स्यादिति भावः ।
	{\color{gray}{\rmlatinfont\textsuperscript{§~\theparCount}}}
	\pend% ending standard par
      ‚{\tiny $_{lb}$}‚

	  
	  \pstart \leavevmode% starting standard par
	य‚दि नीलोत्प‚लादिश‚ब्दा विशेष‚ण‚द्व‚य‚युक्तैक‚ध‚र्मिविष‚यां बुद्धिं ज‚न‚येयुः । त‚दै‚{\tiny $_{lb}$}‚कार्थ‚प्र‚तिपाद‚नेन स्यात् सामानाधिक‚र‚ण्य‚न्त‚च्च नास्ति । व्य‚क्तेर‚र्थान्त‚रं सामान्य‚{\tiny $_{lb}$}‚न्त‚द‚भिधायिन‚श्चानाक्षेप‚कास्त‚द्ग‚तानां भेदानान्त‚द‚प‚रि‚{\tiny $_{१}$}‚त्यागेन वृत्तिराक्षेपः न त‚था ।‚{\tiny $_{lb}$}‚ क‚स्मात् [।] \textbf{निराकांक्ष‚त्वात्} । य‚दा वृक्ष‚श‚ब्दो वृक्ष‚त्व‚मेवाभिध‚त्ते । त‚दा त‚स्य‚{\tiny $_{lb}$}‚ निर्विशेष‚ण‚त्वात् ताव‚तैवासौ निराकांक्ष इति क‚थ‚न्ध‚वादीनाक्षिपेत् । अनाक्षिप्ताश्च‚{\tiny $_{lb}$}‚ क‚थं वृक्ष‚श‚ब्दार्थ‚स्य भेदा ध‚वाद‚योऽत‚द्भेद‚त्वाच्च क‚थं वृक्षः शिंश‚पेति विशेष‚ण‚{\tiny $_{lb}$}‚विशेष्य‚भावः । त‚दाह । \textbf{क‚थ}मित्यादि । सामान्य‚विशिष्ट‚स्य द्र‚व्य‚स्याभिधा‚{\tiny $_{२}$}‚नान्न‚{\tiny $_{lb}$}‚ ‚{\tiny $_{lb}$}‚ \leavevmode\ledsidenote{\textenglish{265/s}}य‚थोक्तो दोष इति चेन्न । उक्तोत्त‚र‚त्वात् । विशेष‚ण‚विशिष्ट‚स्यापि द्र‚व्य‚स्याभिधाने‚{\tiny $_{lb}$}‚ व‚स्तुसाम‚र्थ्यादेक‚स्माद‚पि श‚ब्दाद‚खिल‚ग‚तेः श‚ब्दान्त‚र‚स्य त‚त्राप्र‚वृत्तिः प्र‚वृत्तौ चाप‚र्या‚{\tiny $_{lb}$}‚य‚तेति । त‚देवं व‚स्तुश‚ब्दार्थ‚वादिनो न क‚थंचित्सामानाधिक‚र‚ण्यादिस‚म्भ‚वः ।
	{\color{gray}{\rmlatinfont\textsuperscript{§~\theparCount}}}
	\pend% ending standard par
      ‚{\tiny $_{lb}$}‚

	  
	  \pstart \leavevmode% starting standard par
	सामान्य‚म‚पि तेषां न स‚म्भ‚व‚तीत्युक्तं । त‚था हि य‚दा ताव‚त्स्व‚ल‚क्ष‚णाद‚व्य‚तिरि‚{\tiny $_{lb}$}‚रिक्तं सामान्य‚{\tiny $_{३}$}‚न्त‚दा स्व‚ल‚क्ष‚ण‚व‚द् व्य‚क्त्य‚न्त‚रान‚नुग‚माद‚सामान्यं [।] व्य‚तिरेकेपि‚{\tiny $_{lb}$}‚ क‚थ‚म‚न्य‚स्य स‚मान्य‚म‚तिप्र‚स‚ङ्गादित्यादि प्रागुक्तं ।
	{\color{gray}{\rmlatinfont\textsuperscript{§~\theparCount}}}
	\pend% ending standard par
      ‚{\tiny $_{lb}$}‚

	  
	  \pstart \leavevmode% starting standard par
	व्यावृत्तिवादिन‚स्त्व‚य‚म‚दोष इत्याह । \textbf{स‚र्व‚त्रे}त्यादि । एते दोषा इति सामान्य‚{\tiny $_{lb}$}‚सामानाधिक‚र‚ण्याभावाद‚यः । \textbf{य‚था} हीत्यादि । \textbf{एको} गोभेदः शाव‚लेय‚स्त\textbf{स्माद}‚{\tiny $_{lb}$}‚गोस्व‚भावाद् भिन्न‚स्\textbf{त‚थान्योपि} बाहुलेयादिः [।] अतो विजातीय‚व्यावृत्तः‚{\tiny $_{४}$}‚ स्व‚भावः‚{\tiny $_{lb}$}‚ स‚र्व‚त्र तुल्य इति \textbf{भेद‚स्य} विजातीय‚भिन्न‚स्य स्व‚भाव‚स्य विक‚ल्प‚बुद्ध्या स‚र्व‚त्र स्वाकारा‚{\tiny $_{lb}$}‚भेदेनाध्य‚स्त‚स्या\textbf{सामान्य‚दोषो नास्ति । प‚रिशिष्टाभाव} इति सामानाधिक‚र‚ण्याद्य‚{\tiny $_{lb}$}‚भावः प्रागेवोक्तः । विच्छेदं सूच‚य‚न्नेक‚म‚प्र‚तिक्षिप्य व‚र्त्त‚त \href{http://sarit.indology.info/?cref=pv.3.130}{१ । १३३} इत्यादिना ।
	{\color{gray}{\rmlatinfont\textsuperscript{§~\theparCount}}}
	\pend% ending standard par
      ‚{\tiny $_{lb}$}‚

	  
	  \pstart \leavevmode% starting standard par
	एव‚न्ताव‚द् विजातीय‚व्यावृत्तं स्व‚भावं स‚र्व‚त्र बुद्ध्या स्वाकाराभेदेनाध्य‚स्त‚मेकं‚{\tiny $_{lb}$}‚ श‚ब्दाभिधेयं प्र‚ति‚{\tiny $_{५}$}‚पाद्याधुनाऽभिन्नाकार‚म‚न्त‚रेणाप्येक‚कार्येषु भावेष्वेकः श‚ब्दो‚{\tiny $_{lb}$}‚ नियुज्य‚त इत्याह । \textbf{अपि चे}त्यादि । \textbf{त‚त्कार्य‚प‚रिचोद‚ने} एक‚कार्य‚ताप‚रिचोद‚नार्थं ।‚{\tiny $_{lb}$}‚ य‚द्वैक‚कार्याणां प‚रिचोद‚नार्थं । \textbf{एक‚कार्येषु भेदे}ष्वेक‚स्य भाव‚र‚हितेष्व‚पि \textbf{स‚मा} एका‚{\tiny $_{lb}$}‚ श्रुतिः । \textbf{कृता} संकेतिता । \textbf{वृद्धै}र्व्य‚व‚हार‚ज्ञैः । \textbf{त‚त्कार्या}णाम्भेदानाम‚त‚त्कार्येभ्यो या‚{\tiny $_{lb}$}‚ यावृत्तिस्त\textbf{न्निब‚न्ध‚ना‚{\tiny $_{६}$}‚} विजातीय‚व्यावृत्त‚त‚यैक‚कार्येष्वेका श्रुतिर्निब‚ध्य‚त इत्य‚र्थः ।‚{\tiny $_{lb}$}‚ \href{http://sarit.indology.info/?cref=pv.3.135}{। १३८ ॥}
	{\color{gray}{\rmlatinfont\textsuperscript{§~\theparCount}}}
	\pend% ending standard par
      ‚{\tiny $_{lb}$}‚

	  
	  \pstart \leavevmode% starting standard par
	न‚नु य‚दि न सामान्ये श‚ब्द‚निवेशः स्व‚ल‚क्ष‚णे त‚र्हि श‚ब्द‚निवेशः स्याद‚न्य‚स्याभा‚{\tiny $_{lb}$}‚वात् । न च स्व‚ल‚क्ष‚णं श‚ब्द‚वाच्यं ।
	{\color{gray}{\rmlatinfont\textsuperscript{§~\theparCount}}}
	\pend% ending standard par
      ‚{\tiny $_{lb}$}‚

	  
	  \pstart \leavevmode% starting standard par
	नैत‚द‚स्ति । य‚तः प्र‚तिपाद‚क‚स्ताव‚त् त्रिकाल‚स्थान् भावान् एक‚कार्यात् संकेत‚{\tiny $_{lb}$}‚क‚र‚णाभिप्रायेण विष‚यीकृत्य तेष्वेव संकेतं क‚रोति व्य‚व‚हार‚काले प‚रिचोद‚नार्थं ।‚{\tiny $_{lb}$}‚ ‚{\tiny $_{lb}$}‚ \leavevmode\ledsidenote{\textenglish{266/s}}\leavevmode\ledsidenote{\textenglish{98b/PSVTa}} तेन‚{\tiny $_{७}$}‚ य‚द्य‚पि बुद्धिप‚रिव‚र्त्तिनो भावाः सामान्य‚रूपास्त‚थापि तेष्वेव ब‚हुषु ब‚हिरिव‚{\tiny $_{lb}$}‚ प‚रिस्फुर‚त्स्वेकः श‚ब्दो निवेश्य‚ते [।] न तु तेषु स‚र्वेषु भिन्न‚रूपे सामान्ये स्थिते‚{\tiny $_{lb}$}‚ प्र‚तिव्य‚क्ति भिन्नेव श्रुतिः क‚स्मान्न संकेतितेत्याह । \textbf{गौर‚वे}त्यादि । \textbf{गौर‚वाद्}‚{\tiny $_{lb}$}‚ \add{सामान्यं}\edtext{\textsuperscript{*}}{\edlabel{pvsvt_266-1}\label{pvsvt_266-1}\lemma{*}\Bfootnote{In the margin. }} \textbf{अश‚क्तेर्वैफ‚ल्या}च्च \textbf{भेदाख्याया} भिन्नायाः श्रुतेः । य‚द्वा भेदाख्याया‚{\tiny $_{lb}$}‚ भेद‚क‚थ‚न‚स्य । एत‚च्च वृत्तौ स्प‚ष्ट‚यिष्यामः । \textbf{न भावे} व‚स्तुभू‚{\tiny $_{१}$}‚ते सामान्ये स‚मा श्रुतिः‚{\tiny $_{lb}$}‚ कृता । किं कार‚णं स‚र्व‚भावानां \textbf{स्व‚भाव‚स्य} स्व‚रूप‚स्य \textbf{व्य‚व‚स्थिते}र‚सांक‚र्यात् ।
	{\color{gray}{\rmlatinfont\textsuperscript{§~\theparCount}}}
	\pend% ending standard par
      ‚{\tiny $_{lb}$}‚

	  
	  \pstart \leavevmode% starting standard par
	\textbf{य‚द्रूपं शाव‚लेय‚स्ये}त्यादिना व्य‚व‚स्थित‚स्व‚भाव‚त्व‚माह । त‚तो नाव्य‚तिरिक्तं‚{\tiny $_{lb}$}‚ सामान्यं । व्य‚तिरिक्त‚म‚पि स्व‚स्मिन् स्व‚भावेव‚स्थितं त‚द‚पि क‚थं व्य‚क्तीनां स‚मानं‚{\tiny $_{lb}$}‚ रूपं । न ह्य‚न्येनान्ये स‚माना इत्युक्तं । सास्नाद्याकार‚प्र‚त्य‚य‚स्य हेत‚वोऽ\textbf{त‚त्कार्या}‚{\tiny $_{lb}$}‚स्तेभ्यो \textbf{व्यावृ‚{\tiny $_{२}$}‚त्ति}र्व्यावृत्तः स्व‚भावः । \textbf{द्व‚यो}रिति शाव‚लेय‚बाहुलेय‚योः । त‚स्माद‚{\tiny $_{lb}$}‚त‚त्कार्य‚व्यावृत्तिर्भिन्नानाम‚प्य‚विरुद्धेति । सैवार्थाभेदः श‚ब्दाभेद‚स्य कार‚ण‚मेष्ट‚व्यं‚{\tiny $_{lb}$}‚ य‚तो\textbf{र्थाद‚भेदेन विना श‚ब्दाभेदो न युज्य‚ते} । क‚थ‚न्त‚र्हि ब‚हुष्वेका श्रुतिर‚र्थाभेद एव‚{\tiny $_{lb}$}‚ प्र‚वृत्तेरित्य‚त आह । \textbf{त‚स्मादि}त्यादि । यापीय\textbf{न्त‚त्कार्य‚तै}क‚कार्य\textbf{तेष्टा} य‚स्याः प‚रिचो‚{\tiny $_{lb}$}‚द‚नार्थ‚म्ब‚हुष्वेका श्रुतिरित्यु‚{\tiny $_{३}$}‚क्तं साप्य\textbf{त‚त्कार्यादेव भिन्न‚ता} द्र‚ष्ट‚व्या । ब‚हूनाम‚त‚त्का‚{\tiny $_{lb}$}‚र्यादेव भिन्नः स्व‚भावो द्र‚ष्ट‚व्यः ।
	{\color{gray}{\rmlatinfont\textsuperscript{§~\theparCount}}}
	\pend% ending standard par
      ‚{\tiny $_{lb}$}‚

	  
	  \pstart \leavevmode% starting standard par
	न तु त‚त्कार्य‚ता नाम सामान्य‚म‚स्ति । विनापि च सामान्ये य‚था विल‚क्ष‚णेष्वेक‚{\tiny $_{lb}$}‚श‚ब्द‚निवेशो न विरुद्ध‚स्त‚था द‚र्श‚य‚न्नाह । \textbf{च‚क्षुरादावि}त्यादि । रूप‚विज्ञान‚मेकं‚{\tiny $_{lb}$}‚ फ‚लं य‚स्य च‚क्षुरादेरितिविग्र‚हः । \textbf{क्व‚चि}दिति य‚स्मिन् काले विज्ञान‚ज‚न‚न‚स‚म‚र्थास्ते‚{\tiny $_{lb}$}‚ चोद‚यितु‚{\tiny $_{४}$}‚मिष्टाः । अथ‚वा क्व‚चित् काले सांकेतिकीं श्रुतिं कुर्यादिति स‚म्ब‚न्धः ।‚{\tiny $_{lb}$}‚ किम‚र्थं कुर्यादित्याह । \textbf{अविशेषेण} सामान्येन । \textbf{त‚त्कार्य‚स्य} च‚क्षुर्विज्ञानैक‚कार्य‚स्य‚{\tiny $_{lb}$}‚ कार‚ण‚क‚लाप‚स्य प‚रेभ्यः प्र‚काश‚न\textbf{स‚म्भ‚वे स‚ति} । य‚दा तु च‚क्षुरादीनाम‚साधार‚ण‚{\tiny $_{lb}$}‚‚{\tiny $_{lb}$}‚ ‚{\tiny $_{lb}$}‚ \leavevmode\ledsidenote{\textenglish{267/s}}कार्य‚त्वं चोद्य‚ते । त‚दा नैका श्रुतिस्तेषु संकेत्य‚त इत्य‚र्थः । \textbf{स‚कृदे}क‚कालं \textbf{स‚र्व‚स्य}‚{\tiny $_{lb}$}‚ कार‚ण‚क‚लाप‚स्य \textbf{प्र‚तीत्य‚{\tiny $_{५}$}‚र्थं । ऋतेपी}त्यादि । तेषां च‚क्षुरादीना\textbf{न्त‚द्रूप‚सामान्य}स्येत्येक‚{\tiny $_{lb}$}‚कार्य‚त्व‚ल‚क्ष‚ण‚स्य सामान्य‚स्य व्य‚तिरिक्त‚स्याभावेपीत्य‚र्थः । स‚त्तैवं तेषां सामान्य‚{\tiny $_{lb}$}‚मिति चेन्न त‚स्या अविशेषात् स‚र्व‚दा स‚र्व‚त्र च‚क्षुर्विज्ञान‚प्र‚स‚ङ्गात् । \textbf{अभिन्न‚म‚र्थ‚म‚न्त‚रे}‚{\tiny $_{lb}$}‚णेति सामान्य‚म्व‚स्तुभूत‚म्विना । \textbf{ब‚हुषु} शाव‚लेयादिषु । \textbf{तेषामि}ति शाव‚लेयादि‚{\tiny $_{lb}$}‚भेदानां । य‚दि तेषां सामान्यं स्या‚{\tiny $_{६}$}‚त्त‚दा त‚त्र सामान्ये श‚ब्द‚निवेशात् स‚र्व‚त्र भेदे निवे‚{\tiny $_{lb}$}‚शितः स्यात् । एक‚न्तु सामान्यं विना ब‚हुष्वेक‚श‚ब्द‚स‚न्निवेशो न युक्तः । त‚त‚श्चासौ‚{\tiny $_{lb}$}‚ श‚ब्दः संकेत्य‚मान एक‚त्रैव भेदे संकेतितः स्यात् । त‚था \textbf{चैक‚वृत्तेरे}क‚त्र भेदे कृत‚स‚न्निवे‚{\tiny $_{lb}$}‚श‚स्यान्य‚त्र भेदे विल‚क्ष‚णे प्र‚त्\textbf{य‚याज‚न‚नात्} । द्वितीया गोव्य‚क्तिस्त‚तः श‚ब्दाद् गौरित्येवं‚{\tiny $_{lb}$}‚ न प्र‚तीयेत । त‚त्र प्र‚त्यास‚त्तिनि‚{\tiny $_{७}$}‚ब‚न्ध‚न‚स्य सामान्य‚स्याभावात् । स्व‚भावानुग‚मा- \leavevmode\ledsidenote{\textenglish{99a/PSVTa}}‚{\tiny $_{lb}$}‚ भावेपि शाव‚लेये निवेशितोऽप्र‚त्यास‚न्ने बाहुलेये प्र‚त्य‚यं ज‚न‚यिष्य‚तीति चेदाह । \textbf{अप्र‚{\tiny $_{lb}$}‚त्यास‚त्तिके} चेत्यादि । \textbf{अतिप्र‚स‚ङ्गात्} । गोश‚ब्दाद‚श्वेपि प्र‚तीतिः स्यात् । एव‚न्ताव‚{\tiny $_{lb}$}‚देकं सामान्यं विना ब‚हुष्वेक‚श‚ब्द‚निवेशाभाव उक्तः ।
	{\color{gray}{\rmlatinfont\textsuperscript{§~\theparCount}}}
	\pend% ending standard par
      ‚{\tiny $_{lb}$}‚

	  
	  \pstart \leavevmode% starting standard par
	अभ्युप‚ग‚म्य वैफ‚ल्य‚माह । \textbf{तेषु चे}त्यादि । त‚था हि ब‚हुष्वेका श्रुतिर्निवेश्य‚ते‚{\tiny $_{lb}$}‚ऽनेक‚वृत्ति‚{\tiny $_{१}$}‚मेक‚म‚र्थं प्र‚तिपाद‚यितुं । तानेव वा भेदान् असंक‚रेण । त‚त्राद्य‚स्याभाव‚{\tiny $_{lb}$}‚माह । \textbf{एकार्थ‚नियोगाभावा}दिति । य‚द्येक‚स्मिन्न‚र्थे श‚ब्द‚स्य नियोगः स्यात् त‚दा भ‚वे‚{\tiny $_{lb}$}‚देकार्थ‚प्र‚तिपाद‚नं । द्वितीयाभाव‚माह । \textbf{भिन्ने}त्यादि । \textbf{भिन्न‚स्व‚भावाना}म‚संकीर्ण्णानां‚{\tiny $_{lb}$}‚ शाव‚लेयादीनां प्र‚तिप‚त्त‚य इत्य‚ध्याहारः । \textbf{पृथ}गित्येकैक‚स्मिन् भेदे । एक‚स्य श‚ब्द‚स्य‚{\tiny $_{lb}$}‚ \textbf{नियोगे} । संके‚{\tiny $_{२}$}‚ते कृते । प‚श्चाद् व्य‚व‚हार‚काले । \textbf{य‚थाचोदिताना}मित्य‚संक‚रेण‚{\tiny $_{lb}$}‚ प्र‚तिप‚त्त्य‚र्थं चोदितानां \textbf{विभागाप‚रिज्ञानात्} त‚स्मान्न तेषु श‚ब्द‚नियोगः फ‚ल‚वान् ।‚{\tiny $_{lb}$}‚ एवं हि स्व‚ल‚क्ष‚णेषु पृथ‚क् पृथ‚क्छ‚ब्द‚नियोगः फ‚ल‚वान्भ‚व‚ति ।
	{\color{gray}{\rmlatinfont\textsuperscript{§~\theparCount}}}
	\pend% ending standard par
      ‚{\tiny $_{lb}$}‚

	  
	  \pstart \leavevmode% starting standard par
	य‚दि त‚स्माच्छ‚ब्दाद‚संक‚रेण स्व‚ल‚क्ष‚णानि प्र‚तीयेर‚न्निति [।] क्रिय‚ते च ब‚हु‚{\tiny $_{lb}$}‚ष्वेक‚श‚ब्द‚नियोग‚स्त‚स्माद् व‚स्तुभूतेन सामान्येन भाव्य‚मित्य‚भिप्रायः ।
	{\color{gray}{\rmlatinfont\textsuperscript{§~\theparCount}}}
	\pend% ending standard par
      ‚{\tiny $_{lb}$}‚

	  
	  \pstart \leavevmode% starting standard par
	\textbf{त‚स्यापी}त्या चा र्यः । एक‚म्व‚स्तु सामान्य‚म‚स्तीत्येव कृत्वान्त‚रेणापि प्र‚योज‚नं ।‚{\tiny $_{lb}$}‚ \textbf{लोकेन श‚ब्दो निवेश‚नीय} इत्येको विक‚ल्पः । द्वितीय‚माह । \textbf{त‚द्वे}त्यादि । त‚दिति‚{\tiny $_{lb}$}‚ ‚{\tiny $_{lb}$}‚ \leavevmode\ledsidenote{\textenglish{268/s}}सामान्यं \textbf{व‚स्तुश‚क्त्यैवे}ति । पुरुष‚व्यापार‚म‚न‚पेक्ष्य । एकां \textbf{श्रुतिं} ध्व‚न‚य\textbf{त्युत्थाप‚य‚ति ।‚{\tiny $_{lb}$}‚ नास्त्येव त‚द्} द्व‚य‚म‚पि । अन्त‚रेण प्र‚योज‚नं पुरुष‚व्यापार‚ञ्च प्र‚योगाभावात् । त‚देव‚{\tiny $_{lb}$}‚ \textbf{किन्त‚र्ही}त्या‚{\tiny $_{४}$}‚दिनाह । केन‚चित्प्र‚योज‚न‚नेति । एभ्यः श‚ब्देभ्यो व्य‚व‚हारे । त‚द‚त‚त्सा‚{\tiny $_{lb}$}‚ध‚न‚म‚र्थं ज्ञात्वा प्र‚तिप‚द्येतेति । अनेन प्र‚थ‚म‚विक‚ल्पाभाव‚माह ।
	{\color{gray}{\rmlatinfont\textsuperscript{§~\theparCount}}}
	\pend% ending standard par
      ‚{\tiny $_{lb}$}‚

	  
	  \pstart \leavevmode% starting standard par
	निर्देश्य‚न्त इत्य‚नेन द्वितीय‚स्य । निर्देश्य‚न्ते संकेत्य‚न्ते व्य‚व‚ह‚र्त्तुकामैरित्य‚ध्या‚{\tiny $_{lb}$}‚हारः । \textbf{त‚त्रै}त‚स्मिन् न्यायेऽ\textbf{नेकं} कार‚ण\textbf{मेक‚त्र} कार्ये \textbf{उप‚युज्येत} व्याप्रियेत । \textbf{त‚दि}त्य‚नेकं ।‚{\tiny $_{lb}$}‚ कुतः कार‚ण‚क‚लापात् त‚दु‚{\tiny $_{५}$}‚त्प‚द्य‚त इत्येव । \textbf{त‚त्रै}त‚स्मिन् कार्ये व्य‚व‚ह‚र्त्तुकामै\textbf{र‚व‚श्य‚{\tiny $_{lb}$}‚न्त}च्चानेक‚ङ्कार‚ण‚म‚त‚त्कार्य‚व्यावृत्तिल‚क्ष‚ण‚मेक‚कार्य‚तामाश्रित्यै\textbf{केनैव श‚ब्देन चोद‚यितुं‚{\tiny $_{lb}$}‚ युक्तं} । अनेकेन चोद‚ने दोष‚माह । \textbf{त‚स्येत्या}दि । \textbf{पृथ‚क् पृथ}गिति भिन्नैः श‚ब्दैः ।‚{\tiny $_{lb}$}‚ एत‚च्च स्व‚ल‚क्ष‚ण‚श‚ब्द‚निवेश‚म‚भ्युप‚ग‚म्योक्त‚म् [।]
	{\color{gray}{\rmlatinfont\textsuperscript{§~\theparCount}}}
	\pend% ending standard par
      ‚{\tiny $_{lb}$}‚

	  
	  \pstart \leavevmode% starting standard par
	एत‚देव न स‚म्भ‚व‚तीत्याह । \textbf{न चे}त्यादि । \textbf{अस्य} व‚स्तुनः । \textbf{अ‚{\tiny $_{६}$}‚न‚न्य‚साधार‚णं‚{\tiny $_{lb}$}‚ रूप‚मि}ति । नान्य‚साधार‚ण‚म‚न‚न्य‚साधार‚णं स्व‚ल‚क्ष‚ण‚मित्य‚र्थः । \textbf{नाप्य‚स्ये}त्य‚नेक‚{\tiny $_{lb}$}‚श‚ब्द‚निवेश‚न‚स्य ।
	{\color{gray}{\rmlatinfont\textsuperscript{§~\theparCount}}}
	\pend% ending standard par
      ‚{\tiny $_{lb}$}‚

	  
	  \pstart \leavevmode% starting standard par
	\textbf{एत‚च्च गौर‚वाश‚क्तिवैफ‚ल्यादि}त्य‚स्य य‚थाक्र‚मं विव‚र‚णं । क‚स्माद् वैफ‚ल्य‚मिति‚{\tiny $_{lb}$}‚ चेदाह । \textbf{केवंल}मित्यादि । \textbf{अनेन} प्र‚योक्त्रा त‚त्रैत‚स्मिन् कार्ये । \textbf{तेर्थाः} कार‚ण‚भूता‚{\tiny $_{lb}$}‚\leavevmode\ledsidenote{\textenglish{99b/PSVTa}} \textbf{श्चोद‚नीया} इत्येताव‚त् प्र‚योज‚नं ।‚{\tiny $_{७}$}‚ \textbf{ते} तु कार‚ण‚भूताः प‚दार्था \textbf{एकेन वा श‚ब्देन चोद्येर‚न्‚{\tiny $_{lb}$}‚ ब‚हुभिर्वेति स्वात‚न्त्र्य‚म‚त्र} चोद‚ने \textbf{व‚क्तुः} । य‚त एव‚न्त‚दिति त‚स्मा\textbf{दिय‚मेका श्रुतिर्ब‚हुषु}‚{\tiny $_{lb}$}‚ वाच्येषु \textbf{व‚क्तुर‚भिप्राय}व‚शाद् हेतोः \textbf{प्र‚व‚र्त्त‚माना नोप\textbf{आ}ल‚म्भ‚म‚र्ह‚ति । ब‚हुष्वेका‚{\tiny $_{lb}$}‚ श्रुतिर्न} श‚क्या प्र‚व‚र्त्त‚यितुमिति चेदाह । \textbf{न चेय}मित्यादि । इय‚मेका श्रुतिः । \textbf{अश‚क्यं‚{\tiny $_{lb}$}‚ प्र}व‚र्त्त‚न‚म‚य‚स्या इति विग्र‚हः ।‚{\tiny $_{१}$}‚ क‚स्मान्नाश‚क्य‚प्र‚व‚र्त्त‚नेत्याह । इच्छाधीन‚त्वाद्‚{\tiny $_{lb}$}‚ इच्छाया । अधीन‚म‚श‚क्य‚प्र‚व‚र्त्त‚नं । \textbf{य‚दि ह्येक‚त्रापी}त्यादि । त‚था हीच्छ‚यैव त‚त्र‚{\tiny $_{lb}$}‚ प‚र‚प‚रिक‚ल्पिते सामान्ये न व‚स्तुस्थित्यैव एक‚स्याः श्रुतेः प्र‚व‚त्तिः [।] किन्त‚र्हि‚{\tiny $_{lb}$}‚‚{\tiny $_{lb}$}‚ \leavevmode\ledsidenote{\textenglish{269/s}}व‚क्तुरिच्छाव‚शात् । त‚था \textbf{न} य‚दि \textbf{प्र‚योक्तुरिच्छा भ‚वेत् । क‚थ}मिय‚मेका श्रुतिरेक‚त्रापि‚{\tiny $_{lb}$}‚ \textbf{प्र‚व‚र्त्तेत} । नैवेत्य‚भिप्रायः । त‚थैक‚{\tiny $_{२}$}‚त्रापि प्र‚व‚र्त्त‚यितुमिच्छैव कार‚णं । न व‚स्तु‚{\tiny $_{lb}$}‚श‚क्ति [।] त‚दा \textbf{ब‚हुष्व‚पि} न क‚श्चिद् \textbf{व्याघात} इत्याह । \textbf{इच्छायां} वेत्यादि । एक‚{\tiny $_{lb}$}‚ त्रापि प्र‚व‚र्त्त‚यितुमिच्छायां कार‚ण‚त्वेन क‚ल्प्य‚मानायां । एनामेकां श्रुतिं । \textbf{प्र‚योज‚ना‚{\tiny $_{lb}$}‚भावादित्यादि} । न ह्येक‚म्व‚स्तुविना \textbf{ब‚हुष्वे}क‚श‚ब्द‚निवेश‚नं फ‚ल‚व‚त् । \textbf{उक्तं प्राक्} ।‚{\tiny $_{lb}$}‚ किमुक्त‚मित्याह । \textbf{भिन्नेष्वि}त्यादि । \textbf{भिन्नेषु} शाव‚लेयादिभेदे‚{\tiny $_{३}$}‚ष्\textbf{वेक‚स्माद्} गोश‚ब्दात्‚{\tiny $_{lb}$}‚ प्र‚तीतिर\textbf{त‚त्प्र‚योज‚नेभ्यो भेदेन} । त‚द्गोभेद‚साध्यं वाह‚दोहादिकं कार्यं प्र‚योज‚नं न‚{\tiny $_{lb}$}‚ भ‚व‚ति येषाम‚श्वादीनान्तेभ्यो भेदेनेति । किं पुन‚र्व‚स्तुभूत‚मेवैक‚त्वं न चोद्य‚त इत्य‚त‚{\tiny $_{lb}$}‚ आह । \textbf{न पुनः स्व‚भाव‚स्यैक‚त्वान्न} पुन‚र्व‚स्तुभूत‚स्य सामान्य‚स्यैक‚त्वाद् भिन्नेष्वेका‚{\tiny $_{lb}$}‚ श्रुतिः । त‚स्यैव सामान्य‚व्य‚तिरिक्त‚स्याव्य‚तिरिक्त‚स्य चायो‚{\tiny $_{४}$}‚गात् । \href{http://sarit.indology.info/?cref=pv.3.136}{। १३९ ॥}
	{\color{gray}{\rmlatinfont\textsuperscript{§~\theparCount}}}
	\pend% ending standard par
      ‚{\tiny $_{lb}$}‚

	  
	  \pstart \leavevmode% starting standard par
	भ‚व‚तु नाम सामान्यं व्य‚तिरिक्त‚न्त‚द‚पि त‚स्मिन् स्व‚भावे व्य‚व‚स्थित‚म‚मिश्र‚मेव ।‚{\tiny $_{lb}$}‚ त‚देवाह । \textbf{य‚थास्व‚मि}त्यादि । य‚स्य य‚ल्ल‚क्ष‚ण‚न्तेन \textbf{व्य‚व‚स्थित‚स्व‚भावानाम‚न्योन्य‚{\tiny $_{lb}$}‚रूपासंश्लेषात् क‚थ‚मेक‚निमित्तः} सामान्य‚निब‚न्ध‚नः \textbf{श‚ब्दो भिन्नेषु भ‚वेत्} । नैवेत्य‚भि‚{\tiny $_{lb}$}‚प्रायः । न ह्य‚न्येनान्ये स‚माना भ‚व‚न्ति । एत‚च्चोक्तं प्राक् । \textbf{स‚र्वे भावाः स्व‚भावेन‚{\tiny $_{lb}$}‚ स्व‚स्व‚भाव‚व्य‚व‚स्थि}‚{\tiny $_{५}$}‚तेरित्य‚त्र \href{http://sarit.indology.info/?cref=pv.3.39}{१ । ४२} प्र‚स्तावे ।
	{\color{gray}{\rmlatinfont\textsuperscript{§~\theparCount}}}
	\pend% ending standard par
      ‚{\tiny $_{lb}$}‚

	  
	  \pstart \leavevmode% starting standard par
	अन्यापोह‚वादिन‚स्त्व‚य‚म‚दोष इत्याह । \textbf{अत‚त्प्र‚योज‚नेत्या}दि । अत‚त्प्र‚योज‚नेभ्यो‚{\tiny $_{lb}$}‚ \textbf{व्यावृत्तिस्तु भिन्नानाम‚प्य‚विरुद्धेति} कृत्वा स \textbf{ए}वात‚त्प्र‚योज‚नेभ्यो भेद‚स्त\textbf{त्प्र‚योज‚ना}‚{\tiny $_{lb}$}‚नाम\textbf{भेद}स्त‚द्व्यावृत्तेः स‚र्व‚त्र भावात् । \textbf{श‚ब्दाभेद‚स्य कार‚ण‚म्भ‚व‚तु} । य‚त‚श्च न क‚थं‚{\tiny $_{lb}$}‚चिद् व‚स्तुभूतं सामान्यं घ‚ट‚ते । \textbf{तेनेमे} गोभेदास्त‚त्\textbf{प्र‚योज‚{\tiny $_{६}$}‚ना} वाह‚दोह‚प्र‚योज‚ना इति‚{\tiny $_{lb}$}‚ य‚दुक्त‚न्त‚त् । अत‚त्\textbf{प्र‚योज‚नेभ्यो}ऽश्वादिभ्यो \textbf{भिन्ना एवोक्ताः} । \href{http://sarit.indology.info/?cref=pv.3.137}{। १४० ॥}
	{\color{gray}{\rmlatinfont\textsuperscript{§~\theparCount}}}
	\pend% ending standard par
      ‚{\tiny $_{lb}$}‚

	  
	  \pstart \leavevmode% starting standard par
	\textbf{न पुन‚रेषाम‚न्या त‚त्कार्य‚तान्य}त्रान्य‚तो भेदात् । अन्य‚व्यावृत्त एव स्व‚भाव‚{\tiny $_{lb}$}‚ ‚{\tiny $_{lb}$}‚ \leavevmode\ledsidenote{\textenglish{270/s}}एषाम‚भेद इति याव‚त् । एतेन \textbf{त‚त्कार्य‚ता}पीत्यादि का रि का भागो व्याख्यातः ।‚{\tiny $_{lb}$}‚ \href{http://sarit.indology.info/?cref=pv.3.138}{। १४१ ॥}
	{\color{gray}{\rmlatinfont\textsuperscript{§~\theparCount}}}
	\pend% ending standard par
      ‚{\tiny $_{lb}$}‚

	  
	  \pstart \leavevmode% starting standard par
	\textbf{य‚थे}त्यादिना च‚क्षुरादौ य‚था रूप‚विज्ञानैक‚फ‚ल इत्यादि व्याच‚ष्टे । \textbf{आत्मेन्द्रिय‚{\tiny $_{lb}$}‚\leavevmode\ledsidenote{\textenglish{100a/PSVTa}} म‚नो‚{\tiny $_{७}$}‚र्थ‚स‚न्निक}र्षेष्विति प‚र‚प्र‚सिद्ध्योक्तं । \textbf{रूप‚विज्ञान‚मेकं कार्य} येषामिति विग्र‚हः ।‚{\tiny $_{lb}$}‚ \href{http://sarit.indology.info/?cref=pv.3.139}{। १४२ ॥} त‚द्रूप‚विज्ञानं कार्यं येषान्तानि त‚त्कार्याणि तेषां \textbf{\unclear{सा}मान्य‚म‚विशेषेणैक‚कार्य}‚{\tiny $_{lb}$}‚क‚र‚ण‚साम‚र्थ्य‚न्त‚स्य \textbf{चोद‚ना} प्र‚काश‚ना । त‚स्याः \textbf{स‚म्भ‚वे} स‚ति [।] केन पुनः प्र‚स्तावेन‚{\tiny $_{lb}$}‚ चोद‚नास‚म्भ‚व इत्याह । \textbf{कुतो रूप‚विज्ञा}न‚मित्य‚विशेषेण साम‚ग्रीग‚ते प्र‚श्ने स‚तीत्य‚र्थः ।‚{\tiny $_{१}$}‚‚{\tiny $_{lb}$}‚ \textbf{व्य‚व‚हार‚लाघ‚वार्थ}मेकेन श‚ब्देन ब‚हूनां प्र‚तिपाद‚नार्थं \textbf{रूप‚विज्ञान}स्य हेतुश्च‚क्षुरादि‚{\tiny $_{lb}$}‚क‚लापः स‚रः श‚रो \textbf{वेत्येवं श्रुतिन्निवेश‚येदि}ति स‚म्ब‚न्धः ।
	{\color{gray}{\rmlatinfont\textsuperscript{§~\theparCount}}}
	\pend% ending standard par
      ‚{\tiny $_{lb}$}‚

	  
	  \pstart \leavevmode% starting standard par
	\textbf{अपि नामे}त्यादिना व्य‚व‚हार‚लाघ‚व‚मेव व्याच‚ष्टे । \textbf{त‚द्धेतूनां च‚क्षुर्विज्ञान}‚{\tiny $_{lb}$}‚हेतूनां । \textbf{न चात्रेति} च‚क्षुरादौ । येना\textbf{नुगामिना रूपे}णैकं च‚क्षुर्विज्ञानं ज‚न‚य‚न्ति ।‚{\tiny $_{lb}$}‚ त‚थाभूतं सामान्य‚ञ्च‚{\tiny $_{२}$}‚क्षुरादीनान्नास्ति [।] स‚त्ता विद्य‚त इति चेत् । त‚स्यास्त‚र्हि‚{\tiny $_{lb}$}‚ स‚र्व‚त्राविशेषात् स‚र्व‚त्र च‚क्षुरादिविज्ञानं स्यात् । न च स‚म्ब‚न्धिभेदात् स‚त्ताया भेदो‚{\tiny $_{lb}$}‚ नित्य‚त्वेनानाधेयातिश‚य‚त्वात् । केव‚ल‚न्त‚द‚र्थ‚त‚या रूप‚विज्ञानैक‚कार्य‚त‚या भावाश्च‚क्षु‚{\tiny $_{lb}$}‚राद‚यः । \textbf{अत‚द‚र्थेभ्यो} रूप‚विज्ञानाज‚न‚केभ्यो \textbf{भिन्ना} इति कृत्वा भेद \textbf{एवात‚त्कार्येभ्यो‚{\tiny $_{lb}$}‚ व्यावृ}त्तिरेव । एषां च‚क्षुरादीनाम‚भेदः स‚र्वेषाम‚त‚त्कार्य‚व्यावृत्तेस्तुल्य‚त्वात् । य‚था‚{\tiny $_{lb}$}‚ च‚क्षुरादीनाम‚भेदः स‚र्वेषाम‚त‚त्कार्य‚व्यावृत्तेस्तुल्य‚त्वात् । य‚था च‚क्षूरूपादिष्वेकं‚{\tiny $_{lb}$}‚ सामान्यं विनाप्येक‚कार्य‚त्व‚ख्याप‚नायैकः श‚ब्दो निवेश्य‚ते । एवंजातीयाः स‚र्व‚{\tiny $_{lb}$}‚ इत्यादि । स‚मूह‚स्य स‚न्तान‚स्याव‚स्थाविशेष‚स्य च वाच‚काः श‚ब्दाः \textbf{स‚मूह‚स‚न्ताना‚{\tiny $_{lb}$}‚व‚स्थाविशेष‚श‚ब्दाः} । त‚त्र स‚मूहाभिधायिनं श‚ब्द‚म‚धिकृत्याह । \textbf{ये स‚{\tiny $_{४}$}‚म‚स्ता}‚{\tiny $_{lb}$}‚ इत्यादि । \textbf{ये} रूप‚र‚स‚ग‚न्धाद‚य‚स्स‚म‚स्ताः \textbf{किञ्चि}देक‚मुद‚काद्याह‚र‚णादि \textbf{कार्यं । तेषां}‚{\tiny $_{lb}$}‚ रूपादीनां । \textbf{त‚त्र} कार्ये कार‚ण‚त‚या \textbf{विशेषाभावात्} । स‚र्वेषां ह्येकं कार्य‚मिति कृत्वा‚{\tiny $_{lb}$}‚ तेनैक‚कार्य‚त्वेन विशेषाभाव उच्य‚ते । न तु स्व‚ल‚क्ष‚ण‚स्याविशेषात् ।
	{\color{gray}{\rmlatinfont\textsuperscript{§~\theparCount}}}
	\pend% ending standard par
      ‚{\tiny $_{lb}$}‚

	  
	  \pstart \leavevmode% starting standard par
	तेषामेक‚कार्य‚क‚र‚ण‚श‚क्तिख्याप‚न‚मात्रे क‚र्त्त‚व्येऽ\textbf{पार्थिका विशेष‚चोद‚{\tiny $_{५}$}‚ना} । प्र‚ति‚{\tiny $_{lb}$}‚‚{\tiny $_{lb}$}‚ \leavevmode\ledsidenote{\textenglish{271/s}}रूपादिभिन्नेन श‚ब्देन चोद‚नानिष्फ‚लेति कृत्वा \textbf{स‚कृदे}क‚कालं \textbf{स‚र्वेषां} क्व‚चित् कार्ये‚{\tiny $_{lb}$}‚ नियोज‚नार्थं रूपादिविशेषेषूद‚क‚धार‚णादिकार्य‚स‚मेषु \textbf{घ‚ट इत्येकं श‚ब्दं प्र‚युंक्ते}ऽयं \textbf{लोक}‚{\tiny $_{lb}$}‚ इति व्य‚व‚ह‚र्त्ता । \href{http://sarit.indology.info/?cref=pv.3.140}{। १४३ ॥}
	{\color{gray}{\rmlatinfont\textsuperscript{§~\theparCount}}}
	\pend% ending standard par
      ‚{\tiny $_{lb}$}‚

	  
	  \pstart \leavevmode% starting standard par
	न‚नु भिन्ना एव रूपाद‚यः क‚थ‚मेक‚स्माद् घ‚ट‚श‚ब्दाद् अभेदेन प्र‚तीय‚न्त इति‚{\tiny $_{lb}$}‚ च‚दाह । \textbf{तेपी}त्यादि । स‚जातीया‚{\tiny $_{६}$}‚द् रूपाद\textbf{न्य‚त‚श्च} र‚सादे\textbf{र्भेदाविशेषेपि । त‚त्प्र‚यो‚{\tiny $_{lb}$}‚नाङ्ग‚त‚या} । विशिष्टोद‚काद्याह‚र‚ण‚कार्याङ्ग‚त‚या । \textbf{त‚द‚न्येभ्य} इति त‚त्कार्य‚क‚र‚णा‚{\tiny $_{lb}$}‚स‚म‚र्थेभ्यः प‚टादिभ्यो \textbf{भिद्य‚न्त इति} भेद एवैषाम‚भेद‚स्त‚तोऽ\textbf{भेदात्} । त‚स्माद‚भेदाद‚{\tiny $_{lb}$}‚\textbf{विशेषेणैव} सामान्येनैवैक‚स्माद् घ‚ट‚श‚ब्दात् स‚र्वे रूपाद‚यः \textbf{प्र‚तीय‚न्ते} ।
	{\color{gray}{\rmlatinfont\textsuperscript{§~\theparCount}}}
	\pend% ending standard par
      ‚{\tiny $_{lb}$}‚

	  
	  \pstart \leavevmode% starting standard par
	य‚दि रू‚{\tiny $_{७}$}‚पाद‚य एव केव‚ला घ‚टो न तु त‚द्व्य‚तिरिक्तं द्र‚व्यं [।] क‚थ‚न्त‚र्हि घ‚ट‚स्य \leavevmode\ledsidenote{\textenglish{100b/PSVTa}}‚{\tiny $_{lb}$}‚ रूपाद‚य इति व्य‚तिरेक इति चेदाह । \textbf{त‚त्रे}त्यादि । \textbf{घ‚ट‚स्य रूपाद‚य इत्य‚पि} यो‚{\tiny $_{lb}$}‚ व्य‚तिरेक‚स्त‚स्याय‚म‚र्थो \textbf{घ‚ट‚स्व‚भावा रूपाद‚यो} न प‚टादिस्व‚भावा इति ।
	{\color{gray}{\rmlatinfont\textsuperscript{§~\theparCount}}}
	\pend% ending standard par
      ‚{\tiny $_{lb}$}‚

	  
	  \pstart \leavevmode% starting standard par
	एत‚देव व्याच‚ष्टे । \textbf{उद‚के}त्यादि । \textbf{उद‚काहार‚ण‚स्}य विशेषो घ‚टाद‚न्येनासा‚{\tiny $_{lb}$}‚ध्य‚त्वं । आदिश‚ब्दाद‚न्य‚स्यापि घ‚ट‚{\tiny $_{१}$}‚साध्य‚स्य \textbf{कार्य}स्य प‚रिग्र‚हः । त‚स्मिन् \textbf{कार्ये‚{\tiny $_{lb}$}‚ स‚म‚र्थाः} स‚प्त‚मीति योग‚विभागात्स‚मासः । अय‚म‚त्रार्थः [।] रूपादिश‚ब्दा रूपादीन्‚{\tiny $_{lb}$}‚ रूप‚साध्य‚कार्य‚मात्र‚श‚क्तियुक्तान‚विशेषेण प्र‚तिपाद‚य‚न्ति । घ‚ट‚श‚ब्द‚स्तु विशिष्ट‚कार्य‚{\tiny $_{lb}$}‚श‚क्तियोगेन प‚टादिस्व‚भावेभ्योपि रूपादिभ्यो भेदेन विशिष्टान् रूपादीनाह । अतो‚{\tiny $_{lb}$}‚ घ‚ट‚स्य रूपाद‚य इति श‚ब्द‚द्व‚य‚व्यापा‚{\tiny $_{२}$}‚रेण सामान्य‚विशेषाकार‚बुद्ध्युत्प‚त्तेः सामान्य‚{\tiny $_{lb}$}‚विशेष‚भावो व्य‚तिरेक‚विम‚तिश्च प्र‚युज्य‚त इति । एत‚मेव \textbf{सामान्ये}त्यादिनाह ।‚{\tiny $_{lb}$}‚ \textbf{सामान्य‚कार्यं} रूपादिमात्र‚साध्य‚न्त‚स्य सिद्धिः \textbf{साध‚न‚न्त‚स्मिन् प्र‚सिद्धेनात्म‚ना} स्व‚भा‚{\tiny $_{lb}$}‚वेन । इत्थंभूत‚ल‚क्ष‚णा तृतीया । हेतौ वा । इत्थंभूतेन रूपेण हेतुना वा \textbf{रूपादि‚{\tiny $_{lb}$}‚श‚ब्दैः} क‚र‚ण‚भूतैः \textbf{प्र}सिद्धास्स‚न्तः‚{\tiny $_{३}$}‚ \textbf{विशिष्टं कार्यं} घ‚ट‚साध्यं घ‚ट‚साध्य‚मेवोद‚काह‚र‚{\tiny $_{lb}$}‚णादि । त‚स्य \textbf{साध‚नं} साध्य‚तेनेनेति कृत्वा । त‚थाभूता \textbf{आख्या} संज्ञा \textbf{य‚स्य} स त‚था‚{\tiny $_{lb}$}‚ तेन विशिष्टाः । त इति रूपाद‚य \textbf{एव‚मुच्य}न्त इति ।
	{\color{gray}{\rmlatinfont\textsuperscript{§~\theparCount}}}
	\pend% ending standard par
      ‚{\tiny $_{lb}$}‚

	  
	  \pstart \leavevmode% starting standard par
	\textbf{न पुन‚र‚त्र} रूपादिसंह‚तौ ह्य‚त्र वा घ‚ट इति व्य‚व‚हारे \textbf{य‚थाव‚र्ण्णित‚ल‚क्ष‚ण‚मि}ति‚{\tiny $_{lb}$}‚ ‚{\tiny $_{lb}$}‚ \leavevmode\ledsidenote{\textenglish{272/s}}रूपादिव्य‚तिरि\textbf{क्त‚न्द्र‚व्यं} । त‚स्याव‚य‚विन‚स्तादृश‚स्येति रूपादिव्य‚तिरि‚{\tiny $_{४}$}‚क्त‚स्य । उप‚{\tiny $_{lb}$}‚ल‚ब्धिल‚क्ष‚ण‚प्राप्त‚श्चाव‚य‚वी प‚रैरिष्टो दार्श‚नं स्पार्श‚नं द्र‚व्य‚मिति व‚च‚नात् । तेनो‚{\tiny $_{lb}$}‚प‚ल‚ब्धिल‚क्ष‚ण‚प्राप्त‚त्वेनाभ्युप‚ग‚त‚स्य रूपादिव्य‚तिरेकेणानुप‚ल‚म्भादिति वाक्यार्थः ।‚{\tiny $_{lb}$}‚ य‚थावान्त‚रेणाप्य‚व‚य‚विनं प‚र‚माण‚व एव प्र‚त्य‚क्ष‚स्य विष‚य‚स्त‚था द्वितीये प‚रिच्छेदे‚{\tiny $_{lb}$}‚ प्र‚तिपाद‚यिष्य‚ते ।
	{\color{gray}{\rmlatinfont\textsuperscript{§~\theparCount}}}
	\pend% ending standard par
      ‚{\tiny $_{lb}$}‚

	  
	  \pstart \leavevmode% starting standard par
	य‚दि रूपाद‚य एव संह‚ता घ‚टः क‚{\tiny $_{५}$}‚थ‚न्त‚र्हि ब‚हुषु घ‚ट इत्येक‚व‚च‚न‚मिति चेदाह ।‚{\tiny $_{lb}$}‚ \textbf{एक‚व‚च‚न‚मि}त्यादि । य‚था ब‚हुष्वेकः श‚ब्द एक‚श‚क्तिसूच‚नार्थ‚स्त‚थैक‚व‚च‚न‚म‚पि । तेषां‚{\tiny $_{lb}$}‚ रूपादीना\textbf{मेक}स्मिन्नुद‚काह‚र‚ण‚कार्ये या श‚क्तिस्त‚स्याः सूच‚नार्थ । एक‚कार्य‚क‚र्त्तृत्वेन‚{\tiny $_{lb}$}‚ तेष्वेक‚त्व‚मारोप्यैक‚व‚च‚न‚मित्य‚र्थः । न पुन‚स्तेष्वेका श‚क्तिर्विद्य‚ते । अन‚पेक्षित‚वाह्या‚{\tiny $_{lb}$}‚र्य‚म‚क‚व‚च‚नं \textbf{सं‚{\tiny $_{६}$}‚केत‚प‚र‚त‚न्त्र‚म्वा} । एत‚च्च \textbf{येषां व‚स्तुव‚शा वाच} \href{http://sarit.indology.info/?cref=pv.3.64}{१ । ६६} इत्यादिना‚{\tiny $_{lb}$}‚ प्र‚तिपादितं । स‚न्तानाभिधायिनः श‚ब्दान‚धिकृत्याह । \textbf{त‚थे}त्यादि । \textbf{हेतुश्च फ‚लं च}‚{\tiny $_{lb}$}‚ हेतुफ‚ले । त‚यो\textbf{र्विशेष} उपादानोपादेय‚भावेनैक‚स‚न्तान{......}नाश्र‚य‚त्वं । \textbf{त‚म्भूताः}‚{\tiny $_{lb}$}‚ प्राप्ताः प्राप्तिव‚च‚नो भ‚व‚तिः स‚क‚र्म‚कः । साध‚नं कृतेति द्वितीयात‚त्पुरुषः । \textbf{हेतुफ‚ल‚{\tiny $_{lb}$}‚\leavevmode\ledsidenote{\textenglish{101a/PSVTa}} वि}शेषो‚{\tiny $_{७}$}‚ वा भूतो निष्प‚न्नो येषामिति ब‚ह‚व्रीहिः । आहितादेराकृतिग‚ण‚त्वाद्‚{\tiny $_{lb}$}‚ भूत‚श‚ब्द‚स्य प‚र‚निपातः । \textbf{किंचिदेकं साध‚य‚न्तीति} । य‚थांकुर‚नाड‚प‚त्राद‚यः फ‚ल‚मेकं ।‚{\tiny $_{lb}$}‚ \textbf{साध्य‚न्ते चै}केन । य‚था त एवोपादान‚भूतेन बीजेन । तेप्य‚कुराद‚यो नैक‚क्ष‚णात्म‚काः‚{\tiny $_{lb}$}‚ \textbf{स‚कृत्प्र‚तीत्य‚र्थः} । तेनैक‚कार्य‚त्वेनैक‚कार‚ण‚त्वेन वा साम्येन \textbf{ब्रीह्यादिश‚ब्दैः}‚{\tiny $_{lb}$}‚ स‚न्तानाभिधायिभिः \textbf{कृत‚संकेताः} सं‚{\tiny $_{१}$}‚केत‚काले । प‚श्चाद् व्य‚व‚हार‚काले क‚थ्य‚न्ते‚{\tiny $_{lb}$}‚ व्य‚व‚हार‚लाघ‚वार्थं । अभेदेन प्र‚ब‚न्ध‚जिज्ञासायां बीजांकुरादिभेदेनानेक‚श‚ब्द‚प्र‚योग‚स्य‚{\tiny $_{lb}$}‚ वैफ‚ल्यात् । आदिग्र‚ह‚णेन म‚नुष्यादिश‚ब्द‚ग्र‚ह‚णं । तैर‚पि बाल‚कुमारादिभेद‚भिन्न‚स्य‚{\tiny $_{lb}$}‚ प्र‚ब‚न्ध‚स्याभिधानात् ।
	{\color{gray}{\rmlatinfont\textsuperscript{§~\theparCount}}}
	\pend% ending standard par
      ‚{\tiny $_{lb}$}‚

	  
	  \pstart \leavevmode% starting standard par
	न‚नु ब्रीह्यादिश‚ब्दा अपि स‚मुदाय‚श‚ब्दा एव रूपादिस‚मुदायाभिधायित्वात् ।
	{\color{gray}{\rmlatinfont\textsuperscript{§~\theparCount}}}
	\pend% ending standard par
      ‚{\tiny $_{lb}$}‚

	  
	  \pstart \leavevmode% starting standard par
	स‚त्यं । किन्तु हेतुफ‚ल‚विशेष‚{\tiny $_{२}$}‚फ‚ल‚प्र‚ब‚न्धाभिधानादेव‚मुच्य‚ते [।] त‚था स‚मुदाय‚{\tiny $_{lb}$}‚श‚ब्दोनेक‚स‚मुदायापेक्ष‚या जातिश‚ब्दो भ‚व‚त्येव‚म‚व‚स्थाश‚ब्दोपि [।] केव‚लं विशि‚{\tiny $_{lb}$}‚ष्टार्थ‚विव‚क्ष‚या क‚श्चिच्छ‚ब्द इत्युच्य‚त इत्य‚दोषः । य‚था च घ‚ट‚स्य रूपाद‚यः घ‚ट इति‚{\tiny $_{lb}$}‚ चैक‚व‚च‚नं येन निब‚न्ध‚नेनोक्तं । त‚था ब्रीहे रूपाद‚यो ब्रीहिरिति नैक‚व‚च‚नं द्र‚ष्ट‚व्य‚म‚त‚{\tiny $_{lb}$}‚ एवाह । \textbf{पूर्व‚व‚द्वा}च्य‚मिति । अव‚स्थाश‚ब्दान‚{\tiny $_{३}$}‚धिकृत्याह । \textbf{येपी}त्यादि । येपि नीला‚{\tiny $_{lb}$}‚‚{\tiny $_{lb}$}‚ \leavevmode\ledsidenote{\textenglish{273/s}}दिप‚र‚माण‚वः \textbf{पृथ‚गि}ति नील‚पीताद‚यः प‚र‚स्प‚रान‚पेक्षाः \textbf{स‚म‚स्ता} वेति प‚र‚स्प‚र‚स‚हिताः ।‚{\tiny $_{lb}$}‚ \textbf{क्व‚चिदि}ति च‚क्षुर्विज्ञाने स्व‚देशे प‚र‚स्योत्प‚त्तिप्र‚तिब‚न्धे वा \textbf{स‚कृदेव प्र‚त्याय‚नार्थं} ।‚{\tiny $_{lb}$}‚ एक‚स्माच्छ‚ब्दाद् ब‚हूनां निश्च‚यार्थं । त‚त्र ये च‚क्षुर्विज्ञाने \textbf{उप‚युज्य‚न्ते । तेव‚स्था‚{\tiny $_{lb}$}‚विशेष‚वाचिनः स‚निद‚र्श‚ना} इत्युच्य‚न्ते । ये‚{\tiny $_{४}$}‚ स्व‚देशे प‚र‚स्योत्प‚त्तिं प्र‚तिघ्न‚न्ति ।‚{\tiny $_{lb}$}‚ ते \textbf{स‚प्र‚तिघा} इति ।
	{\color{gray}{\rmlatinfont\textsuperscript{§~\theparCount}}}
	\pend% ending standard par
      ‚{\tiny $_{lb}$}‚

	  
	  \pstart \leavevmode% starting standard par
	न‚नु नील‚पीताद‚योऽत्य‚न्त‚भिन्नास्ते क‚थ‚मेकेन स‚निद‚र्श‚नादिश‚ब्देनोच्य‚न्त इत्य‚त‚{\tiny $_{lb}$}‚ आह ।
	{\color{gray}{\rmlatinfont\textsuperscript{§~\theparCount}}}
	\pend% ending standard par
      ‚{\tiny $_{lb}$}‚

	  
	  \pstart \leavevmode% starting standard par
	\textbf{त‚द‚न्येभ्यो भेद‚सामान्येने}ति । त‚द‚न्येभ्योऽनिद‚र्श‚नाप्र‚तिघेभ्यो यो भेद‚स्य एव‚{\tiny $_{lb}$}‚ तेषां सामान्यं स‚र्वेषान्त‚तो व्यावृत्त‚त्वात् । तेन हेतुना । स‚निद‚र्श‚नादिश‚ब्दा अपि‚{\tiny $_{lb}$}‚ प‚र‚माणुस‚मुद‚या‚{\tiny $_{५}$}‚ऽभिधानात् स‚मुदाय‚श‚ब्दा एवेति चेत् [।] न । एक‚स्यापि‚{\tiny $_{lb}$}‚ प‚र‚माणोः स‚प्र‚तिघादिश‚ब्दैर‚भिधानात् ।
	{\color{gray}{\rmlatinfont\textsuperscript{§~\theparCount}}}
	\pend% ending standard par
      ‚{\tiny $_{lb}$}‚

	  
	  \pstart \leavevmode% starting standard par
	कार्य‚द्वारेण श‚ब्द‚प्र‚वृत्तिमुक्त्वा कार‚ण‚द्वारेणाह । \textbf{य‚थैक‚कार्या} रूपाद‚य‚स्त\textbf{त्कार्य‚{\tiny $_{lb}$}‚चोद‚नायां} । त‚दुद‚क‚धार‚णाद्येकं कार्यं य‚स्य रूपादिसाम‚र्थ्य‚स्य त‚स्य चोद‚नायामेक‚{\tiny $_{lb}$}‚श‚क्तिचोद‚नायामित्य‚र्थः । \textbf{त‚द‚न्य‚स्मात् घ‚टादेर्भेदेन} घ‚टादिश‚ब्दैः । आदि‚{\tiny $_{६}$}‚ग्र‚ह‚णाद्‚{\tiny $_{lb}$}‚ ब्रीह्यादिप‚रिग्र‚हः । \textbf{कृत‚स‚म‚याः} ख्याप्य‚न्त इति प्र‚कृतं । \textbf{त‚था कार‚णापेक्ष‚या}प्य‚ने‚{\tiny $_{lb}$}‚\textbf{कोर्थः एकेन} श‚ब्देन कृत‚स‚म‚यः ख्याप्य‚त इति व‚च‚न‚प‚रिणामेन स‚म्ब‚न्धः \textbf{व्य‚व‚हारार्थ‚{\tiny $_{lb}$}‚मेव} लाघ‚वेनेत्य‚र्थाद् द्र‚ष्ट‚व्यं । \textbf{य‚था श‚व‚लाया} गोर‚प‚त्यानि स‚र्वाण्येवैक‚कार‚ण‚त्वेन‚{\tiny $_{lb}$}‚ शाव‚लेय‚श‚ब्देनोच्य‚न्ते \textbf{ब‚हुलायाश्चाप‚त्यानि} बाहुलेय‚श‚ब्देन ।‚{\tiny $_{७}$}‚ यावांश्च पुरुष‚प्र‚य‚त्नेन \leavevmode\ledsidenote{\textenglish{101b/PSVTa}}‚{\tiny $_{lb}$}‚ कार‚णेन ज‚नितः श‚ब्दः स‚र्वः स‚मान‚कार‚ण‚ज‚न्य‚त्वेन \textbf{प्र‚य‚त्नान‚न्त‚रीय‚कः} क‚थ्य‚ते ।‚{\tiny $_{lb}$}‚ क‚रिष्यामीति चेत‚ना प्र‚य‚त्नः । त‚स्यान‚न्त‚र‚म‚व्य‚व‚धान‚न्त‚त्र भ‚व इति ग्र‚हादेराकृति‚{\tiny $_{lb}$}‚ग‚ण‚त्वाच्छः । देश‚ग्र‚ह‚ण‚न्त‚त्र न स्म‚र्य‚ते । त‚स्य स्वार्थिकः क‚न् । एत‚च्च कार‚ण‚{\tiny $_{lb}$}‚विशेषापेक्ष‚योक्तं ।
	{\color{gray}{\rmlatinfont\textsuperscript{§~\theparCount}}}
	\pend% ending standard par
      ‚{\tiny $_{lb}$}‚

	  
	  \pstart \leavevmode% starting standard par
	कार‚ण‚मात्राश्र‚येणाह । \textbf{कृत‚को} वेति । कार‚णा‚{\tiny $_{१}$}‚य‚त्त‚ज‚न्म‚नः प्र‚य‚त्नान‚न्त‚रीय‚{\tiny $_{lb}$}‚क‚स्यान्य‚स्य च स‚र्व‚स्य कृत‚क इत्य‚भिधानात् ।
	{\color{gray}{\rmlatinfont\textsuperscript{§~\theparCount}}}
	\pend% ending standard par
      ‚{\tiny $_{lb}$}‚

	  
	  \pstart \leavevmode% starting standard par
	एव‚न्ताव‚द्विधिमुखेनोक्तं ।
	{\color{gray}{\rmlatinfont\textsuperscript{§~\theparCount}}}
	\pend% ending standard par
      ‚{\tiny $_{lb}$}‚‚{\tiny $_{lb}$}‚\textsuperscript{\textenglish{274/s}}

	  
	  \pstart \leavevmode% starting standard par
	प्र‚तिषेध‚मुखेनाह । \textbf{त‚थे}त्यादि । त‚स्य चाक्षुष‚स्य नीलादेर्य‚त् कार्य‚ञ्च‚क्षुर्विज्ञा‚{\tiny $_{lb}$}‚न‚न्त‚स्य \textbf{प्र‚तिषेधेनाचाक्षुषः श‚ब्दः} । न स‚म‚र्थ‚ञ्च‚क्षुर्विज्ञानं प्र‚तीत्येव‚म‚चाक्षुष‚श‚ब्देन‚{\tiny $_{lb}$}‚ सामान्येनोच्य‚ते । अनित्य‚श‚ब्दोपि नित्य‚व्य‚व‚च्छेदेन व्य‚व‚स्था‚{\tiny $_{२}$}‚प्य‚मानः । त‚त्कार्य‚{\tiny $_{lb}$}‚प्र‚तिषेधेनैव । त‚था हि नित्यं प‚रैर्व‚स्त्वेवेष्ट‚न्त‚च्चासाध्य‚साध‚न‚भूतं व्य‚व‚हार‚प‚थं नाव‚{\tiny $_{lb}$}‚त‚र‚तीति साध्य‚साध‚नं चाङ्गीक‚र्त‚व्यं । नित्य‚कार्य‚प्र‚तिषेधेनानित्यः । आत्म‚श‚ब्दोपि‚{\tiny $_{lb}$}‚ क्व‚चित् कार्ये स्व‚त‚न्त्र‚स्य ख्याप‚नाय कृत इत्य‚नात्म‚श‚ब्दोऽत‚त्कार्य‚व्य‚व‚च्छेदेन स्यात् ।
	{\color{gray}{\rmlatinfont\textsuperscript{§~\theparCount}}}
	\pend% ending standard par
      ‚{\tiny $_{lb}$}‚

	  
	  \pstart \leavevmode% starting standard par
	एवं कार्य‚प्र‚तिषेधेनाभिधाय कार‚ण‚प्र‚तिषेधेनाह । \textbf{त‚दि}त्यादि । त‚{\tiny $_{३}$}‚स्य स‚स्वा‚{\tiny $_{lb}$}‚मिक‚स्याशून्य‚स्य च य\textbf{त्कार‚ण}न्त‚स्य \textbf{प्र‚तिषेधे}नायं श‚ब्दादिको भावो\textbf{स्वामिकः शून्य‚{\tiny $_{lb}$}‚ इति} व्य‚व‚हारार्थं ख्याप्य‚त इति स‚म्ब‚न्धः । त‚था हि स्व‚त‚न्त्रेणात्मादिना योधिष्ठि‚{\tiny $_{lb}$}‚त‚स्स स‚स्वामिकः प‚रैरिष्य‚ते । एव‚म‚शून्योपि त‚थाभूतेनाधिष्ठात्राधिष्ठित‚त्वादेवा‚{\tiny $_{lb}$}‚धिष्ठिता चाधिष्ठात‚व्य‚स्वी\textbf{क‚र‚ण‚म‚न्य}थाधिष्ठातृत्वायोगात् । त‚स्मात् स‚स्वामिका‚{\tiny $_{lb}$}‚दि‚{\tiny $_{४}$}‚श‚ब्दाः कार‚ण‚द्वार‚प्र‚वृत्ताः प‚रेषां । न प्र‚तिक्ष‚ण‚विश‚रारुषु भावेषु साम‚ग्रीमात्र‚{\tiny $_{lb}$}‚प्र‚तिब‚द्धेषु व्य‚व‚स्थित‚स्व‚भावः क‚श्चिद‚धिष्ठातास्ति य‚त्प्र‚तिब‚द्धास्संस्काराः प्र‚व‚र्त्त‚न्ते ।‚{\tiny $_{lb}$}‚ त‚तोस्वामिकाः शून्याश्च य‚थोक्त‚कार‚ण‚प्र‚तिषेधेन व्य‚व‚स्थाप्य‚न्त इति । एव‚म‚न्य‚द‚{\tiny $_{lb}$}‚पीति । दुःखाशून्यानाथाप्र‚तिश‚र‚णादिक‚म‚पि । \textbf{य‚थायोग}मिति किंचित्कार्य‚प्र‚ति‚{\tiny $_{५}$}‚‚{\tiny $_{lb}$}‚षेधेन किंचित्कार‚ण‚प्र‚तिषेधेनेत्य‚र्थः । दुःखाशून्यादिकार्य‚प्र‚तिषेधेन सुख‚शून्यादीनाम‚{\tiny $_{lb}$}‚प्रातिकूल्य‚कार्य‚त्वेन व्य‚व‚स्थाप्य‚मान‚त्वात् । स‚र्व‚स्य च संस्कृत‚स्य विप‚रिणाम‚ध‚र्मित्वेन‚{\tiny $_{lb}$}‚ प्र‚तिकूल‚त्वात् । \textbf{अनाथाप्र‚तिश‚र}णादि । कार्य‚प्र‚तिषेधेन स्व‚त‚न्त्र‚स्य नाथादेर‚भावात् ।‚{\tiny $_{lb}$}‚ त‚देवं कार्य‚कार‚ण‚योर्विधिप्र‚तिषेध‚मुखेन च‚तुष्ट‚यी श‚ब्दानाम्प्र‚{\tiny $_{६}$}‚वृत्तिराख्याता भ‚व‚ति ।
	{\color{gray}{\rmlatinfont\textsuperscript{§~\theparCount}}}
	\pend% ending standard par
      ‚{\tiny $_{lb}$}‚

	  
	  \pstart \leavevmode% starting standard par
	न‚नु चाशून्य‚नित्यादेर्व्य‚व‚च्छेद्य‚स्याभावात् क‚थं शून्यादिश‚ब्देष्व‚न्य‚व्य‚व‚च्छेदाभि‚{\tiny $_{lb}$}‚धान‚मिति चेदाह । \textbf{शून्ये}त्यादि । \textbf{य‚थाक‚थित}मिति य‚स्य यादृशी सिद्धान्ताश्र‚य‚ण‚{\tiny $_{lb}$}‚क‚ल्प‚ना त‚या \textbf{स‚मीहितो} र‚चितो शून्य‚नित्यादीनां य \textbf{आका}र‚स्तं \textbf{विक‚ल्प्य} बुद्धावारोप्य‚{\tiny $_{lb}$}‚ \leavevmode\ledsidenote{\textenglish{102a/PSVTa}} \textbf{त‚द्व्य‚व‚च्छेदेन} प‚र‚प‚रिक‚ल्पिताऽशून्याद्याकार‚व्य‚व‚च्छेदेन शून्यादिव्य‚{\tiny $_{७}$}‚प‚देशः क्रिय‚ते ।‚{\tiny $_{lb}$}‚ क‚स्मादित्याह । \textbf{बुद्धी}त्यादि । बुद्धेस्स‚मीहा इम‚म‚र्थ‚मारोप‚यामीति संक‚ल्पः । त‚था‚{\tiny $_{lb}$}‚ स{र्व्वं...} सांक‚र्यं य‚स्य श‚ब्द‚स्य स त‚था त‚द्भाव‚स्त‚स्मात् । स‚र्व‚{\tiny $_{lb}$}‚ग्र‚ह‚णादेत‚दाह । \textbf{य‚त्रापि} व‚स्तुभूते{......}ईति ।
	{\color{gray}{\rmlatinfont\textsuperscript{§~\theparCount}}}
	\pend% ending standard par
      ‚{\tiny $_{lb}$}‚

	  
	  \pstart \leavevmode% starting standard par
	एत‚दुक्त‚म्भ‚व‚ति । न व‚स्तुस्व‚ल‚क्ष‚णं श‚ब्दैः स्व‚रूपेण विधीय‚तेऽप‚नीय‚ते वा [।]‚{\tiny $_{lb}$}‚ ‚{\tiny $_{lb}$}‚ \leavevmode\ledsidenote{\textenglish{275/s}}केव‚लं विक‚ल्प‚बुद्धिस‚न्द‚र्शित एव स‚र्वो विधिप्र‚तिषे‚{\tiny $_{१}$}‚ध‚व्य‚व‚हारः । त‚त‚श्चानित्या‚{\tiny $_{lb}$}‚दिश‚ब्देष्व‚नित्यादिप्र‚तिप‚क्षो नित्यादिर्व्य‚व‚च्छेद्यो नास्तीत्य\textbf{प्र‚तिप‚क्ष‚दोष}स्त‚स्यो\textbf{प‚क्षेप}‚{\tiny $_{lb}$}‚ उद्भाव‚नं । आदिश‚ब्दान्नास्त्यात्मेति प्र‚तिषेधे चाप्र‚तिषेध‚दोष इत्येव‚माद्युप‚क्षेप‚श्च [।]‚{\tiny $_{lb}$}‚ \textbf{दुर्म‚तीनामु} द्यो त क र प्र‚भृतीनां \textbf{विस्प‚न्दितानि} विजृम्भितान्य‚स‚म्ब‚द्धानीति याव‚त् ।‚{\tiny $_{lb}$}‚ न हि न्यायानुग‚त‚बुद्धिर‚स‚म्ब‚द्ध‚मुद्भाव‚येत् । अत‚श्च ते दोषो‚{\tiny $_{२}$}‚प‚क्षेपा \textbf{उपेक्ष‚णीया}‚{\tiny $_{lb}$}‚ नाव‚धानार्हा इत्य‚र्थः ।
	{\color{gray}{\rmlatinfont\textsuperscript{§~\theparCount}}}
	\pend% ending standard par
      ‚{\tiny $_{lb}$}‚

	  
	  \pstart \leavevmode% starting standard par
	\textbf{अथे}त्यादि प‚रः । अपिश‚ब्दो भिन्न‚क्र‚मः । \textbf{एक}स्य व‚स्तुनः सामान्य‚स्य \textbf{वृत्ते}र‚पि‚{\tiny $_{lb}$}‚ कार‚णाद‚नेको व्य‚क्तिभेदः । एका श्रुति\textbf{रेक‚श्रुतिः} । सा चाधिका अस्या\textbf{नेक}स्या‚{\tiny $_{lb}$}‚स्तीत्\textbf{येक‚श्रुतिमान्} । एक‚श‚ब्द‚वाच्यो \textbf{य‚दि भ‚वेदि}त्य‚र्थः । एक‚कार्य‚त्वेनैकः श‚ब्द‚{\tiny $_{lb}$}‚ब‚हुष्वेकेन वा सामान्येनेति न क‚श्चिद् विशेष इति म‚न्य‚ते । अत एव व्याच‚ष्टे । न‚{\tiny $_{३}$}‚‚{\tiny $_{lb}$}‚ \textbf{केव‚ल‚मि}त्यादि । \textbf{त‚द‚न्य‚स्मा}द‚त‚त्कार्याद्यो भेद‚स्स एव स‚र्वेषां त‚त्कार्याणाम\textbf{विशेष‚{\tiny $_{lb}$}‚स्त‚स्मादेक‚श‚ब्देनोच्य‚न्ते । अपि त्वेक‚वृत्त्याप्}येक‚स्य सामान्य‚स्य व‚र्त्त‚नेना\textbf{प्य‚नेकः प‚दार्थ}‚{\tiny $_{lb}$}‚ एक‚श‚ब्देनोच्\textbf{येत को विरोधः स्यात्} । य‚थैक‚कार्य‚त्वेन ब‚हुष्वेक‚श‚ब्द‚प्र‚वृत्तौ नास्ति‚{\tiny $_{lb}$}‚ विरोध‚स्त‚था व‚स्तुभूतेनापि सामान्येन । त‚स्माद् व‚स्तुभूत‚सामान्य‚क‚ल्प‚नापि युक्तै‚{\tiny $_{lb}$}‚वेति भावः ।
	{\color{gray}{\rmlatinfont\textsuperscript{§~\theparCount}}}
	\pend% ending standard par
      ‚{\tiny $_{lb}$}‚

	  
	  \pstart \leavevmode% starting standard par
	\textbf{उ‚{\tiny $_{४}$}‚क्त}मित्या चा र्यः । त‚स्य व‚स्तुभूत‚स्य सामान्य‚स्य \textbf{उप‚ल‚भ्य‚ते} रूपेणा\textbf{भिम‚तं} ।‚{\tiny $_{lb}$}‚ अभिम‚त‚त्वे उप‚ल‚भ्य‚त्वं क‚र‚ण‚त्वेन विव‚क्षित‚मिति क‚र्त्तृक‚र‚णे कृतेत्येव स‚मासः ।‚{\tiny $_{lb}$}‚ उप‚ल‚ब्धिल‚क्ष‚ण‚प्राप्त‚स्य व्य‚क्तिव्य‚तिरेके\textbf{णानुप‚ल‚ब्धेर‚भावः} सामान्य‚स्येति वाक्यार्थः ।‚{\tiny $_{lb}$}‚ \textbf{अनुप‚ल‚भ्य‚मान‚तायाम्वा}ऽङ्गीक्रिय‚माणाया\textbf{न्त‚द्द‚र्श‚नाश्र‚या} इति सामान्य‚द‚र्श‚नाश्र‚या‚{\tiny $_{lb}$}‚ \textbf{व्य‚प‚दे‚{\tiny $_{५}$}‚श‚प्र‚त्य‚भिज्ञानाद‚यो न भ‚वेयु}रिति । \textbf{उक्त}मिति स‚म्ब‚न्धः । ब‚हुष्वेक‚श‚ब्दो‚{\tiny $_{lb}$}‚ व्य‚प‚देश‚स्तुल्याकारं ज्ञानं प्र त्य भि ज्ञा नं । \textbf{आदि}श‚ब्दात् सामान्याश्र‚या व्य‚क्तौ‚{\tiny $_{lb}$}‚ प्र‚वृत्तिर्न भ‚वेदित्यादेः प‚रिग्र‚हः ।
	{\color{gray}{\rmlatinfont\textsuperscript{§~\theparCount}}}
	\pend% ending standard par
      ‚{\tiny $_{lb}$}‚

	  
	  \pstart \leavevmode% starting standard par
	न हि स्व‚य‚म‚नुप‚ल‚भ्य‚मान‚मुप‚ल‚म्भ‚निब‚न्ध‚नं व्य‚प‚देश‚प्र‚त्य‚भिज्ञान‚म‚न्य‚त्र प्र‚व‚र्त्त‚{\tiny $_{lb}$}‚‚{\tiny $_{lb}$}‚ \leavevmode\ledsidenote{\textenglish{276/s}}य‚ति । न भ‚वेयुरित्यादीन्य‚नेन चादिश‚ब्देनान्य‚स्यापि पूर्वोक्त‚स्य दोष‚स्य ग्र‚ह‚{\tiny $_{६}$}‚णं ।‚{\tiny $_{lb}$}‚ \textbf{न ह्य‚न्ये}नान्ये स‚मानानाम‚त‚द्व‚न्तो नाम स्युः [।] त‚था न जातिर्वाह‚दोहादावुप‚युज्य‚त‚{\tiny $_{lb}$}‚ इत्यादि । अनेनैत‚दाह [।] जातिक‚ल्प‚नायाम्बाध‚कं प्र‚माण‚म‚स्ति । त‚तो न त‚न्नि‚{\tiny $_{lb}$}‚ब‚न्ध‚नो व्य‚प‚देशादिः । एक‚कार्य‚त्वे तु विरोधाभावात् त‚त्कृत‚मेव व्य‚प‚देश‚प्र‚त्य‚भिज्ञा‚{\tiny $_{lb}$}‚नादिकं युक्त‚मिति । दूष‚णान्त‚र‚म‚प्याह । \textbf{अपि चे}त्यादि । व‚स्तुभूते सामान्य‚मिच्छ‚ता‚{\tiny $_{lb}$}‚ \leavevmode\ledsidenote{\textenglish{102b/PSVTa}} स्वा‚{\tiny $_{७}$}‚श्र‚ये नैक‚स्मिंस्त‚स्य प्र‚वृत्तिरेष्ट‚व्या । न हि त‚त्राव‚र्त्त‚मान‚माश्र‚ये व्य‚प‚देशादि‚{\tiny $_{lb}$}‚कार‚णं युक्तं । सा च सामान्य‚स्य स्वाश्र‚ये \textbf{प्र‚वृत्तिराधेय‚ता} वा भ‚वेत् । त‚द्ब‚लेना‚{\tiny $_{lb}$}‚व‚स्थानात् । आश्र‚य‚ब‚लेनोप‚ल‚ब्धि\textbf{र्व्य‚क्तिः} सा वा वृत्तिर्भ‚वेत् । एत‚द् द्व‚य‚म‚पि \textbf{त‚स्मि‚{\tiny $_{lb}$}‚न्सा}मान्ये न \textbf{युज्य‚ते} । \href{http://sarit.indology.info/?cref=pv.3.141-142}{। १४४-४५ ॥}
	{\color{gray}{\rmlatinfont\textsuperscript{§~\theparCount}}}
	\pend% ending standard par
      ‚{\tiny $_{lb}$}‚

	  
	  \pstart \leavevmode% starting standard par
	\textbf{य‚देत‚दि}त्यादिना व्याच‚ष्टे । \textbf{य‚देत‚देक‚मि}ति व‚स्तुभूतं सामान्य‚म\textbf{नेक‚त्राश्र‚ये} व‚र्त्त‚{\tiny $_{lb}$}‚मान\textbf{मेकां श्रुतिं‚{\tiny $_{१}$}‚ व‚र्त्त‚य}ति [।] त‚स्य सामान्य‚स्य स्वाश्र‚ये \textbf{केयं वृत्तिरि}ति प्र‚श्न‚यित्वा‚{\tiny $_{lb}$}‚ स्व‚य‚मेव विक‚ल्प‚द्व‚य‚माह । \textbf{आधेय‚ता चे}त्यादि । अथ‚वा किंश‚ब्दः प्र‚तिक्षेपे [।]‚{\tiny $_{lb}$}‚ \textbf{केयं वृत्तिर्न} काचिदित्य‚र्थः । त‚था हि वृत्तेः स्वा\textbf{श्र‚ये} आधेय‚ता वा \textbf{स्यात् । य‚था‚{\tiny $_{lb}$}‚ कुण्डे} आधारे \textbf{ब‚द‚राणि व‚र्त्त‚न्त} इति । व्य‚क्तिर्वा त‚स्य सामान्य‚स्या\textbf{श्र‚ये} वृत्तिः \textbf{स्यात्}‚{\tiny $_{lb}$}‚ तैराश्र‚यैर्व्य‚क्तेः प्र‚काश‚नात् । त‚त्र य‚द्याधेय‚ता वृत्तिरिष्य‚ते ।‚{\tiny $_{२}$}‚ त‚दा व्य‚क्त‚य‚स्त‚दा‚{\tiny $_{lb}$}‚धार‚त्वेनैष्ट‚व्याः । नित्यं च सामान्य‚म‚भ्युप‚ग‚तं व्य‚क्त्युत्प‚त्तेः पूर्व्व‚न्त‚द‚नाधेय‚न्त‚तो‚{\tiny $_{lb}$}‚ \textbf{नित्य‚स्या}श्र‚यै\textbf{र‚नुप‚कार्य‚त्वा}द्धेतोराश्र‚याभिम‚ता व्य‚क्त‚यो \textbf{नाधारः} ।
	{\color{gray}{\rmlatinfont\textsuperscript{§~\theparCount}}}
	\pend% ending standard par
      ‚{\tiny $_{lb}$}‚

	  
	  \pstart \leavevmode% starting standard par
	\textbf{नित्यं} हीत्यादिना व्याच‚ष्टे । अथ नित्यं नेष्य‚ते त‚दाप्य\textbf{नित्य‚त्वेऽप‚राप‚रोत्प‚त्ते}‚{\tiny $_{lb}$}‚र‚न्य‚स्यान्य‚स्योत्प‚त्तेर‚नेकं सामान्य‚म\textbf{नेक‚त्वात्} कार‚णाद् \textbf{भेदेष्वि}व भेद‚व‚त् त‚स्मिन्‚{\tiny $_{lb}$}‚ सामान्ये \textbf{एक‚प्र‚त्य‚यायोगा}देक‚{\tiny $_{३}$}‚स्य ज्ञान‚स्यायोगात् कार‚णात् । \textbf{नित्यं सामान्य‚मिष्य‚त}‚{\tiny $_{lb}$}‚ इत्य‚नेन स‚म्ब‚न्धः । \textbf{नित्य‚स्य च} सामान्य‚स्य \textbf{किंकुर्वाण आश्र‚य आधारः‚{\tiny $_{lb}$}‚ स्यान्नैवे}त्य‚भिप्रायः ।
	{\color{gray}{\rmlatinfont\textsuperscript{§~\theparCount}}}
	\pend% ending standard par
      ‚{\tiny $_{lb}$}‚‚{\tiny $_{lb}$}‚\textsuperscript{\textenglish{277/s}}

	  
	  \pstart \leavevmode% starting standard par
	नोप‚कार‚क‚त्वादाधारः किन्तु \textbf{त‚स्य} सामान्य‚स्य त‚त्राश्र‚ये \textbf{स‚म‚वायात्} ।‚{\tiny $_{lb}$}‚ य‚दा हो द्यो त क रः । क‚थं त‚र्हि गोत्वं गोषु प्र‚व‚र्त्त‚ते । आश्र‚याश्र‚यिभावेन [।]‚{\tiny $_{lb}$}‚ \textbf{कः पु}न‚राश्र‚याश्र‚यिभावः \textbf{स‚म‚वायः} । त‚त्र वृत्तिम‚द् गोत्वं ।‚{\tiny $_{४}$}‚ वृत्तिः स‚म‚वाय‚{\tiny $_{lb}$}‚ इह प्र‚त्य‚य‚हेतुत्वादिति ।\edtext{\textsuperscript{*}}{\edlabel{pvsvt_277-1}\label{pvsvt_277-1}\lemma{*}\Bfootnote{\href{http://sarit.indology.info/?cref=nv}{ Nyāyavārtika. }}} उप‚कार्योप‚कार‚क‚त्वाभावे स‚म‚वाय‚म‚स‚म्भाव‚य‚न्नाह ।‚{\tiny $_{lb}$}‚ \textbf{कोय‚मि}त्यादि ।
	{\color{gray}{\rmlatinfont\textsuperscript{§~\theparCount}}}
	\pend% ending standard par
      ‚{\tiny $_{lb}$}‚

	  
	  \pstart \leavevmode% starting standard par
	\textbf{अपृथ‚गि}त्यादि प‚रः । अभिन्न‚देश‚त्वेन सिद्धा \textbf{अपृथ‚क्सिद्धाः} । तेषां योय‚मा‚{\tiny $_{lb}$}‚\textbf{श्र‚याश्र‚यिभा}व‚स्स‚म‚वायः ।
	{\color{gray}{\rmlatinfont\textsuperscript{§~\theparCount}}}
	\pend% ending standard par
      ‚{\tiny $_{lb}$}‚

	  
	  \pstart \leavevmode% starting standard par
	\textbf{त‚दि}त्यादि सि द्धा न्त वा दी । \textbf{त‚देवेद‚माश्र‚य‚त्व‚म‚नुप‚कार‚क‚स्याश्र‚य‚स्य न स‚म्भाव}‚{\tiny $_{lb}$}‚यामः । क‚स्माद् [।] \textbf{अतिप्र‚स‚ङ्गात्} । य‚द्य‚नुप‚का‚{\tiny $_{५}$}‚र‚क‚स्याश्र‚य‚त्व‚मिष्य‚ते । त‚दा‚{\tiny $_{lb}$}‚ स‚र्वः स‚र्व‚स्याश्र‚यः स्यात् । न भ‚व‚ति [।] स‚र्व‚स्य स‚र्वास‚म‚वेत‚त‚या प्र‚तीतेरिति‚{\tiny $_{lb}$}‚ चेत् । न‚नूप‚कार‚काभावे गोत्व‚व‚त् स‚र्व‚स्यैव स‚र्व‚स‚म‚वेत‚त्वेनैक‚स्मान्न प्र‚तीतिर्भ‚व‚ती‚{\tiny $_{lb}$}‚तीद‚मेव चोद्य‚ते ।
	{\color{gray}{\rmlatinfont\textsuperscript{§~\theparCount}}}
	\pend% ending standard par
      ‚{\tiny $_{lb}$}‚

	  
	  \pstart \leavevmode% starting standard par
	अथोप‚कार्योप‚कार‚क‚भावादेर‚न्य एवायं स‚म‚वाय‚ल‚क्ष‚ण‚स्स‚म्ब‚न्धः । स च न स‚र्व‚{\tiny $_{lb}$}‚त्रास्तीति क‚थ‚म‚तिप्र‚संगः ।
	{\color{gray}{\rmlatinfont\textsuperscript{§~\theparCount}}}
	\pend% ending standard par
      ‚{\tiny $_{lb}$}‚

	  
	  \pstart \leavevmode% starting standard par
	उच्य‚ते । स‚त्यं [।] केव‚लं क्व‚चित् स‚{\tiny $_{६}$}‚म‚वेत‚स्य स‚म‚वायो भ‚व‚ति । त‚त्स‚म‚{\tiny $_{lb}$}‚वेत‚त्वं च त‚दाय‚त्त‚त‚या [।] त‚दाय‚त्त‚त्व‚ञ्चार्थान्त‚र‚स्य त‚दुत्प‚त्तिरेव । तेनोप‚र्युप‚रि‚{\tiny $_{lb}$}‚भावेनोत्प‚त्तिरेवेह बुद्धेर्निब‚न्ध‚न‚न्न स‚म‚वाय इत्य‚र्थाप‚त्तिक्ष‚यः । उप‚र्युप‚रिभावे{... इ...}स्यात् । नाप्य‚स‚म‚वेतानां स‚म‚वायोस्ति येन स‚म‚वेत‚त्वं स्यात्‚{\tiny $_{lb}$}‚ स‚र्वेषां स‚र्व‚त्र स‚म‚वेत‚त्व‚प्र‚स‚ङ्गात् ।
	{\color{gray}{\rmlatinfont\textsuperscript{§~\theparCount}}}
	\pend% ending standard par
      ‚{\tiny $_{lb}$}‚

	  
	  \pstart \leavevmode% starting standard par
	उप‚संह‚र‚न्नाह । \textbf{त‚स्मा}‚{\tiny $_{७}$}‚दित्यादि । अपृथ‚क्सिद्ध‚योः स‚म‚वायो य‚थार‚भ्यार‚म्भ‚क- \leavevmode\ledsidenote{\textenglish{103a/PSVTa}}‚{\tiny $_{lb}$}‚ योर्द्र‚व्य‚योः पृथ‚क्सिद्धानां संयोगः । य‚थाग्निधूम‚योरेक‚स्मिन्न‚र्थे स‚म‚वाय \textbf{एकार्थ‚स‚म‚{\tiny $_{lb}$}‚वायः} । य‚था रूप‚र‚स‚योरेक‚स्मिन् द्र‚व्ये । \textbf{आदि}श‚ब्दात संयुक्त‚स‚म‚वेतस्य प‚रि‚{\tiny $_{lb}$}‚ग्र‚हः । व‚स्तुभूताः स‚म्ब‚न्धा व‚स्तूनां वा स‚म्ब‚न्धा इति विशेष‚ण‚स‚मासः ष‚ष्ठीस‚मासो‚{\tiny $_{lb}$}‚ वा । व‚स्तुग्र‚ह‚णं क‚ल्प‚नाकृत‚निवृत्त्य‚र्थं । \textbf{कार्य‚{\tiny $_{१}$}‚कार‚ण‚भावान्न व्य‚तिरिच्य‚न्ते} न व‚था‚{\tiny $_{lb}$}‚ ‚{\tiny $_{lb}$}‚ ‚{\tiny $_{lb}$}‚ \leavevmode\ledsidenote{\textenglish{278/s}}भ‚व‚न्ति । एत‚देव साध‚य‚न्नाह । \textbf{प‚र‚स्प‚र}मित्यादि । \textbf{प‚र‚स्प‚र}म‚न्योन्य‚मुप‚कारिणा\textbf{म‚{\tiny $_{lb}$}‚न्य‚तो} वाऽश्र‚याभिम‚ता\textbf{द‚नुप‚कारिणाम‚प्र‚तिब‚न्धा}द‚नाय‚त्त‚त्वात् । \textbf{अप्र‚तिब‚ध‚न्न‚स्य चा‚{\tiny $_{lb}$}‚स‚म्ब‚न्धा}त् कार‚णात् स‚र्व‚व‚स्तुस‚म्ब‚न्धाः कार्य‚कार‚ण‚भावान्न व्य‚तिरिच्य‚न्त इति प्र‚कृतेन‚{\tiny $_{lb}$}‚ संब‚न्धः ।
	{\color{gray}{\rmlatinfont\textsuperscript{§~\theparCount}}}
	\pend% ending standard par
      ‚{\tiny $_{lb}$}‚

	  
	  \pstart \leavevmode% starting standard par
	न‚नु चाश्र‚यात् स‚त्युप‚कारे आश्रित‚योः प‚र‚स्प‚र‚मुप‚कार्यो‚{\tiny $_{२}$}‚प‚कार‚क‚भावो नैवा‚{\tiny $_{lb}$}‚स्तीति किम‚र्थ‚म‚न्य‚तो वेत्य‚स्योप‚न्यासः ।
	{\color{gray}{\rmlatinfont\textsuperscript{§~\theparCount}}}
	\pend% ending standard par
      ‚{\tiny $_{lb}$}‚

	  
	  \pstart \leavevmode% starting standard par
	स‚त्य‚मेत‚त् । किन्तु \textbf{य‚द्य‚पि} साक्षा\textbf{द‚न्योन्यं नोप‚का}र‚स्त‚थाप्येक‚कार‚णाय‚त्त‚त‚या‚{\tiny $_{lb}$}‚ पार‚म्प‚र्येणापि स‚म्ब‚न्धं क‚ल्प‚येदित्युप‚न्यासः । \textbf{एकार्थ‚स‚म‚वायिनः} प‚र‚स्प‚र‚मुप‚कार्यो‚{\tiny $_{lb}$}‚प‚कार‚क‚भावो नैवेष्य‚त इति चेदाह । \textbf{य‚द्य‚पी}त्यादि । त‚त एक‚स्मादाश्र‚यादुप‚कार‚स्या‚{\tiny $_{lb}$}‚\textbf{भावे य‚थोक्त‚दोष‚प्र‚स‚ङ्गात्} । अतिप्र‚स‚ङ्ग‚{\tiny $_{३}$}‚भ‚यादित्युक्तो दोषः य‚त‚श्च स्वाश्र‚यादे‚{\tiny $_{lb}$}‚\textbf{कार्थ‚स‚म‚वायिनो}र‚व‚श्य‚मुप‚कारोऽतः स्वाश्र‚य‚कृतः स‚म‚वायिनोर्यः \textbf{स्वोप‚कार} आत्मोप‚{\tiny $_{lb}$}‚कार‚स्तेन \textbf{द्वारेण प‚र‚म‚पि} द्वितीय‚म‚पि स‚म‚वायिनं \textbf{संघ‚ट‚य्य} प्र‚तिपाद‚यित्रा \textbf{ख्याप्य‚ते}‚{\tiny $_{lb}$}‚ स‚म‚वायिनाविह स‚म्ब‚द्धाविति ।
	{\color{gray}{\rmlatinfont\textsuperscript{§~\theparCount}}}
	\pend% ending standard par
      ‚{\tiny $_{lb}$}‚

	  
	  \pstart \leavevmode% starting standard par
	एत‚दुक्त‚म्भ‚व‚ति । य‚था प‚र‚स्यैकार्थ‚स‚म‚वायिनोः प‚र‚स्प‚रास‚म्ब‚द्धेप्येकार्थ‚स‚म‚{\tiny $_{lb}$}‚वायात् स‚म्ब‚{\tiny $_{४}$}‚न्ध‚स्त‚थास्माक‚मेक‚कार्य‚त्वेन त‚योः \textbf{स‚म्ब‚न्धः} । य‚त एव‚न्त\textbf{स्मात् त‚त्रा‚{\tiny $_{lb}$}‚प्ये}कार्थ‚स‚म‚वायिनि \textbf{कार्य‚कार‚ण‚भाव‚कृत एवा}श्र‚येण स‚ह यः कार्य‚कार‚ण‚भाव‚स्त‚त्कृत‚{\tiny $_{lb}$}‚ \textbf{एव} य‚द्द्वारेणारोपित एव स‚म्ब‚न्धः । य‚स्मादुप‚कार‚द्वारेणैवाधारादिभावः । \textbf{त‚स्मा‚{\tiny $_{lb}$}‚द‚य‚माश्र‚यः} शाव‚लेयादिः । \textbf{स्वात्म‚नि} सामान्य‚स्व‚भावे\textbf{नुप‚कुर्वाणः} सामान्य\textbf{स्यान‚{\tiny $_{lb}$}‚पेक्ष‚स्याधार इति या चि त क म ण्ड न मे त‚त्} । म‚ण्ड‚न‚म‚ल‚ङ्कारो म‚ण्ड्य‚तेनेनेति‚{\tiny $_{lb}$}‚ कृत्वा [।] त‚स्य याचित‚क‚श‚ब्देन क‚र्म‚धार‚यः । क‚स्मात् प‚र‚स्माद् याचित‚क‚म्म‚ण्ड‚{\tiny $_{lb}$}‚न‚न्द‚रिद्र‚स्यात्म‚न्य‚विद्य‚मानं । त‚द्व‚त् सामान्याश्र‚य‚स्यापि सामान्यं प्र‚त्याधार‚त्वं ।‚{\tiny $_{lb}$}‚ भाव‚साध‚नो वा तृतीयास‚मास‚श्च । य‚था याचित‚केनाल‚ङ्कारेण म‚ण्ड‚न‚क्रिया ।‚{\tiny $_{lb}$}‚ त‚था सामान्याश्र‚य‚स्य प‚र‚स्मात् प्रार्थितेनाधार‚भावेना‚{\tiny $_{६}$}‚धार‚व्य‚प‚देशो न व‚स्तुस्थित्या‚{\tiny $_{lb}$}‚ त‚त उप‚काराभावादिति ।
	{\color{gray}{\rmlatinfont\textsuperscript{§~\theparCount}}}
	\pend% ending standard par
      ‚{\tiny $_{lb}$}‚

	  
	  \pstart \leavevmode% starting standard par
	\textbf{१--क‚थ‚मि}त्यादिप‚रः । न हि \textbf{कुण्डं ब‚द‚राणां ज‚न‚कं} । तेषां स्व‚हेतोरेव‚{\tiny $_{lb}$}‚ ‚{\tiny $_{lb}$}‚ \leavevmode\ledsidenote{\textenglish{279/s}}निष्प‚त्तेः [।] त‚त‚श्च य‚दुक्तं स‚र्व‚त्र व‚स्तुस‚म्ब‚न्धाः कार्य‚कार‚ण‚भावान्न व्य‚तिरि‚{\tiny $_{lb}$}‚च्य‚न्त इति त‚द‚नेकान्तिक‚मिति म‚न्य‚ते । \textbf{प्र‚विस‚र्प्प‚तो} देशान्त‚र‚विस‚र्प्प‚ण‚शील‚स्य‚{\tiny $_{lb}$}‚ ब‚द‚रादेस्त‚द्देश‚ज‚न‚{\tiny $_{७}$}‚न‚मुपादान‚भूत‚स्य पूर्व‚क‚स्य ब‚द‚र‚ल‚क्ष‚ण‚स्य यो देशः कुण्ड‚स‚म्ब‚द्ध- \leavevmode\ledsidenote{\textenglish{103b/PSVTa}}‚{\tiny $_{lb}$}‚ स्त‚स्मिन्नेव \textbf{देशे ज‚न}न‚म‚न्य‚त्राग‚म‚नात् । इयं \textbf{श‚क्तिः कुण्डादे}राधाराभिम‚त‚स्य‚{\tiny $_{lb}$}‚ \textbf{ब‚द‚रादि}ष्वाधेयेषु ।
	{\color{gray}{\rmlatinfont\textsuperscript{§~\theparCount}}}
	\pend% ending standard par
      ‚{\tiny $_{lb}$}‚

	  
	  \pstart \leavevmode% starting standard par
	\textbf{प्र‚कृत्येवे}त्यादिना व्याच‚ष्टे । \textbf{प्र‚कृ}त्या स्व‚भावेनैवास‚मानो \textbf{देशे} य‚स्य त‚त्त‚थोक्तं ।‚{\tiny $_{lb}$}‚ प्र‚कृतिश‚ब्द‚म‚पेक्ष‚माण‚स्यापि ग‚म‚क‚त्वाद् ब‚हुब्रीहिः । त‚थाभूतं च त‚त्कार्यं चेति क‚र्म‚{\tiny $_{lb}$}‚धार‚यः । कार्यं च \textbf{ब‚द‚रादि}क‚{\tiny $_{१}$}‚मेवोत्त‚रोत्त‚र‚क्ष‚ण‚संगृहीतं । त‚स्यो\textbf{त्पाद}नं त‚देव ध‚र्मः‚{\tiny $_{lb}$}‚ स्व‚भावो य‚स्य \textbf{गुरुणो द्र‚व्य}स्य ब‚द‚रादेः पूर्व‚क्ष‚ण‚संगृहीत‚स्य । \textbf{स‚मान‚देश‚कार्योत्पाद‚न‚{\tiny $_{lb}$}‚भाव आधार‚कृतः} । आत्म‚ना तुल्य‚देश‚स्योत्पाद‚क‚त्व‚माधार‚कृत‚मित्य‚र्थः । य‚त एव‚{\tiny $_{lb}$}‚\textbf{न्त‚स्मात्} पाश्चात्य‚स्य ब‚द‚र‚कार्य‚स्य यः \textbf{पूर्व‚क्ष‚णः} उपादान‚भूत‚स्त\textbf{स्य स‚ह‚कारि कुण्डं ।‚{\tiny $_{lb}$}‚ त‚त्रै}वोपादान‚क्ष‚ण‚देश एव \textbf{ब‚द‚र‚{\tiny $_{२}$}‚कार्य ज‚न‚य‚त्} कुण्ड\textbf{माधा}र इत्युच्य‚ते ।
	{\color{gray}{\rmlatinfont\textsuperscript{§~\theparCount}}}
	\pend% ending standard par
      ‚{\tiny $_{lb}$}‚

	  
	  \pstart \leavevmode% starting standard par
	अनेन चैक‚साम‚ग्र्य‚धीन‚योः कुण्ड‚ब‚द‚र‚क्ष‚ण‚योराधाराधेय‚भाव इत्युक्त‚म्भ‚व‚ति ।‚{\tiny $_{lb}$}‚ \textbf{अन्य}था य‚दि \textbf{कुण्}डेन \textbf{ब‚द‚रा}णां य‚थोक्त उप‚कारो \textbf{न क्रि}य‚ते त‚देह \textbf{कुण्डे ब‚द‚राणीत्येवं‚{\tiny $_{lb}$}‚ व्याप‚देशो न स्यात्} । निय‚ताधार‚स्य व्य‚प‚देश‚स्य निमित्त‚म‚न्त‚रेणायोगात् ।‚{\tiny $_{lb}$}‚ \textbf{त‚दुप‚कार‚कृत} इत्याधारोप‚कार‚कृ\textbf{तोयं व्याप‚दे}श इह कुण्डे ब‚द‚रा‚{\tiny $_{३}$}‚णीति । किन्त‚र्हि‚{\tiny $_{lb}$}‚ कुण्ड‚ब‚द‚र‚योर्यः संयोग‚स्त‚त्कृतः ।
	{\color{gray}{\rmlatinfont\textsuperscript{§~\theparCount}}}
	\pend% ending standard par
      ‚{\tiny $_{lb}$}‚

	  
	  \pstart \leavevmode% starting standard par
	\textbf{किम्पुन‚रि}त्यादि सि द्धा न्त वा दी । पृच्छ‚त‚श्चाय‚म‚भिप्रायो क्ष‚णिक‚त्वे स‚ति‚{\tiny $_{lb}$}‚ संयोगादीनाम्भ‚व‚द्भिः क‚ल्प‚नेष्य‚ते । अक्ष‚णिक‚त्वं चेद् भावानाम‚भ्युप‚ग‚म्य‚ते संयो‚{\tiny $_{lb}$}‚गादीनामेवोत्प‚त्तिर्न स्यादि\textbf{त्य‚र्थः । त‚योरि}ति कुण्ड‚ब‚द‚र‚योः \textbf{संयोग} इत्य‚पि व्य‚प‚देश‚{\tiny $_{lb}$}‚निमित्तं नास्त्युप‚कार्योप‚कार‚क‚{\tiny $_{४}$}‚त्वाभावादित्य‚भिप्रायः ।
	{\color{gray}{\rmlatinfont\textsuperscript{§~\theparCount}}}
	\pend% ending standard par
      ‚{\tiny $_{lb}$}‚‚{\tiny $_{lb}$}‚\textsuperscript{\textenglish{280/s}}

	  
	  \pstart \leavevmode% starting standard par
	\textbf{ताभ्यामि}त्यादि प‚रः । ताभ्यां कुण्ड‚ब‚द‚राभ्यां संयोग‚स्य \textbf{ज‚न‚नात्त‚योः} संयोग‚{\tiny $_{lb}$}‚ इष्य‚ते । द्वाभ्यामेव संयोग‚स्य ज‚न‚न‚मुभ‚य‚त्र स‚म‚वायः । प‚रेणोक्त इत्यवमृश्य‚{\tiny $_{lb}$}‚ सि द्धा न्त वा द्या ह । \textbf{स} इत्यादि । स संयोग \textbf{एक‚त्रैव} कुण्डे ब‚द‚रे वा \textbf{किन्न स‚म‚वैति}‚{\tiny $_{lb}$}‚ ज‚न्य‚ते वा । एकेन कुण्डेन ब‚द‚रेण वा पृच्छ‚त‚श्चायं भावो य‚दि तौ कुण्ड‚ब‚द‚रा‚{\tiny $_{५}$}‚ख्यौ‚{\tiny $_{lb}$}‚ भावौ संयोग‚ज‚न‚ने । आधार‚भावोप‚ग‚म‚ने वा । प्र‚त्येकं स‚म‚र्थ‚स्व‚भावौ त‚दा किमि‚{\tiny $_{lb}$}‚त्य‚न्योन्य‚म‚पेक्ष‚त इति । पृथ‚ग‚{\tiny $_{६}$}‚र्थ‚न्त‚दुभ‚यं प‚र‚स्प‚र‚स‚हित‚मेव स‚म‚र्थ‚मिति चेदाह ।‚{\tiny $_{lb}$}‚ \textbf{त‚दि}त्यादि । य‚त्कुण्ड‚ब‚द‚र‚व‚स्तु\textbf{पृथ‚ग‚स‚म‚र्थ‚म्} त‚त्प‚र‚स्प‚र\textbf{स‚हित‚म}पि तादृश‚मेवास‚म‚र्थ‚{\tiny $_{lb}$}‚मेवाक्ष‚णिक‚त्वादिति भावः । क्ष‚णिकास्तु प्र‚त्येकं पृथ‚ग‚स‚म‚र्थाः प‚{\tiny $_{६}$}‚श्चात् स‚ह‚कारि‚{\tiny $_{lb}$}‚कृत‚विशेषास्स‚हितास्स‚म‚र्था इत्य‚विरुद्धं । त‚त‚श्च संयोगं प्र‚त्य‚नु\textbf{प‚कार‚क‚त्वात्} ।‚{\tiny $_{lb}$}‚ कुण्ड‚ब‚द‚राख्य‚म्व‚स्तु । \textbf{न संयोगेन} त‚द्ध‚र्म्यात् ।
	{\color{gray}{\rmlatinfont\textsuperscript{§~\theparCount}}}
	\pend% ending standard par
      ‚{\tiny $_{lb}$}‚

	  
	  \pstart \leavevmode% starting standard par
	\textbf{स‚हित‚स्ये}त्यादि प‚रः । स‚हित‚स्य कुण्ड‚स्य ब‚द‚र‚स्य च त‚द‚न्योप‚कारात् । त‚स्मात्‚{\tiny $_{lb}$}‚ कुण्डाद् यो \textbf{यो} ब‚द‚रात्मा \textbf{त‚स्मात्} कुण्ड‚स्योप‚कारात् । त‚स्माद्वा ब‚द‚राद् य‚द‚न्य‚त्‚{\tiny $_{lb}$}‚ \leavevmode\ledsidenote{\textenglish{104a/PSVTa}} कुण्ड‚न्त‚स्माद् ब‚द‚र‚स्योप‚कारात् । \textbf{विशेषो‚{\tiny $_{७}$}‚त्प}त्तेर्हेतोः कुण्ड‚ब‚द‚र‚योः संयोग‚स्य ज‚न‚ने ।‚{\tiny $_{lb}$}‚ आधार‚भावोप‚ग‚म‚ने वा \textbf{साम‚र्थ्यं} न केव‚ल‚योरिति ।
	{\color{gray}{\rmlatinfont\textsuperscript{§~\theparCount}}}
	\pend% ending standard par
      ‚{\tiny $_{lb}$}‚

	  
	  \pstart \leavevmode% starting standard par
	\textbf{कोय}मित्या चा र्यः । ब‚द‚राणां कुण्डादीनां चा\textbf{ज‚न्य‚ज‚न‚क‚भूतानां कोय‚मुप‚{\tiny $_{lb}$}‚कारः} [।] नैवास्ति । अज‚न्य‚ज‚न‚क‚त्व‚मेव क‚थ‚मिति चेदाह । स्व‚रूपेत्यादि ।‚{\tiny $_{lb}$}‚ ब‚द‚र\textbf{स्व‚रूप}स्यान्य‚तः हेतोरेव \textbf{सिद्धेर‚कार्य}त्वान्न ब‚द‚राणां ज‚न्य‚त्वं नापि ज‚न‚क‚त्वं‚{\tiny $_{lb}$}‚ कुण्ड‚स्येति भावः । न‚{\tiny $_{१}$}‚ हि प‚रो ब‚द‚रादीनां कुण्डादेः स‚काशात् स्व‚रूपोत्प‚त्तिम्वा‚{\tiny $_{lb}$}‚ञ्छ‚ति । स्व‚हेतोरेव तेषान्निष्प‚त्तेः । न कुण्डेन ब‚द‚र‚रूप‚मेव क्रिय‚ते किन्तु त‚तोन्य‚{\tiny $_{lb}$}‚द्रूप‚मित्य‚त आह । \textbf{प‚र‚रूपे}त्यादि । \textbf{त‚त्रे}त्याधेये । न ह्य‚र्थान्त‚रे कृतेर्थान्त‚र‚मुप‚कृतं‚{\tiny $_{lb}$}‚ स्यात् । \textbf{उभ‚य‚थेति} स्व‚रूप‚प‚र‚रूप‚क्रियाभ्यां । अन‚न्त‚रोक्तेन विधिना\textbf{नुप‚कार‚क‚स्य}‚{\tiny $_{lb}$}‚ कुण्डादेर\textbf{किञ्चित्क‚र‚त्वात्} ।
	{\color{gray}{\rmlatinfont\textsuperscript{§~\theparCount}}}
	\pend% ending standard par
      ‚{\tiny $_{lb}$}‚

	  
	  \pstart \leavevmode% starting standard par
	एत\textbf{च्चोक्त‚प्रायं} । प्रा‚{\tiny $_{२}$}‚य‚श‚ब्दो बाहुल्य‚व‚च‚नः । प्रायेणोक्त‚मुक्त‚प्रायं । राज‚{\tiny $_{lb}$}‚द‚न्तादेराकृतिग‚ण‚त्वात् प्राय‚श‚ब्द‚स्य प‚र‚निपातः । अयं चार्थः कार्य‚श्च तासां प्राप्तो‚{\tiny $_{lb}$}‚सौ ज‚न‚नं य‚दुप‚क्रिये \href{http://sarit.indology.info/?cref=pv.3.105}{१ । १०८}त्यादि विस्त‚रेणोक्तः । प्राय‚श‚ब्दं स‚दृशार्थ‚म‚न्ये‚{\tiny $_{lb}$}‚ प्राहुः । उक्तेन स‚दृश‚मुक्त‚प्रायं । प्रागुक्तेनाश्र‚य‚कृतेन सामान्य‚स्य स्व‚रूपोप‚कारेणेदं‚{\tiny $_{lb}$}‚ ‚{\tiny $_{lb}$}‚ \leavevmode\ledsidenote{\textenglish{281/s}}कुण्डादिकृत‚माधेय‚स्योप‚कार‚क‚र‚ण‚न्तुल्य‚मित्य‚र्थः ।‚{\tiny $_{३}$}‚ \textbf{स‚र्व एवे}त्यादिनोप‚संहारः । \textbf{स‚र्व‚{\tiny $_{lb}$}‚ एव व‚स्तुस‚म्ब‚न्धाः} कार्य‚कार‚ण‚भावाद्धेतोर्व्य‚व‚स्थाप्य‚न्त इति स‚म्ब‚न्धः । विभाग‚स्तेषां‚{\tiny $_{lb}$}‚ न स्यादिति चेदाह । \textbf{ज‚न‚क‚स्यै}वेत्यादि । य‚द्वा \textbf{कार्य‚कार‚ण‚भावा}त् स‚काशात् \textbf{प्र‚विभागेन}‚{\tiny $_{lb}$}‚ भेदेन \textbf{व्य‚व‚स्थाप्य‚न्ते} । क‚थ‚म्भेद इत्याह । \textbf{ज‚न‚क‚स्यै}वेत्यादि । कार‚ण‚कृतः कार्य‚स्य‚{\tiny $_{lb}$}‚ य उप‚कार‚विशेष‚स्त‚स्य ब‚लादित्य‚र्थः । त‚था हि प्र‚विस‚र्प्प‚ण‚ध‚र्म‚{\tiny $_{४}$}‚णो ब‚द‚रादेः स्वोपा‚{\tiny $_{lb}$}‚दान‚देशोत्पाद‚न‚ल‚क्ष‚णेनोप‚योगेनाधाराधेय‚भावः । प्र‚दीप‚कृतेन च विज्ञान‚ज‚न‚न‚स‚म‚र्थ‚{\tiny $_{lb}$}‚स्व‚रूपोत्पादेन घ‚ट‚प्र‚दीपादीनां व्य‚ङ्ग्य‚व्य‚ञ्ज‚क‚ल‚क्ष‚णः स‚म्ब‚न्ध इत्येव‚म‚न्य‚स्मिन्न‚पि‚{\tiny $_{lb}$}‚ स‚म्ब‚न्धे य‚थायोगं वाच्यं ।
	{\color{gray}{\rmlatinfont\textsuperscript{§~\theparCount}}}
	\pend% ending standard par
      ‚{\tiny $_{lb}$}‚

	  
	  \pstart \leavevmode% starting standard par
	न‚नु \textbf{स‚र्व एव व‚स्तुस‚म्ब‚न्धा} इत्यादिना न संयोग‚ल‚क्ष‚ण‚स्य स‚म्ब‚न्ध‚स्य कार्य‚{\tiny $_{lb}$}‚कार‚ण‚भावेन्त‚र्भावः स‚मान‚काल‚भाविनोरेवास्य स‚त्त्वा‚{\tiny $_{५}$}‚त् । अथाक्ष‚णिक‚प‚क्षे संयो‚{\tiny $_{lb}$}‚गोत्प‚त्तिर्न युज्य‚ते [।] क्ष‚णिके त‚र्हि भ‚विष्य‚ति संयुक्तासंयुक्ताव‚स्थ‚योश्च कुण्ड‚{\tiny $_{lb}$}‚ब‚द‚र‚योर्न स्व‚रूप‚भेदः प्र‚तीय‚ते । तेनाक्ष‚णिकेपि संयोगोस्त्येव प्र‚तीतेः ।
	{\color{gray}{\rmlatinfont\textsuperscript{§~\theparCount}}}
	\pend% ending standard par
      ‚{\tiny $_{lb}$}‚

	  
	  \pstart \leavevmode% starting standard par
	\hphantom{.}य‚दाहो द्यो त क रः । य‚दि संयोगो न नार्थान्त‚र‚म्भ‚वेत्त‚दा क्षेत्र‚बीजोद‚काद‚यो‚{\tiny $_{lb}$}‚ निर्विशिष्ट‚त्वात् स‚र्व‚दैवांकुरादिकार्यं कुर्यु र्न‚चैवं । त‚स्मात् स‚र्व‚दा कार्यानार‚म्भात्‚{\tiny $_{lb}$}‚ क्षेत्रादीन्य‚ङ्कुरोत्प‚त्तौ‚{\tiny $_{६}$}‚ कार‚णान्त‚र‚सापेक्षाणि । य‚था मृत्पिण्डादिसाम‚ग्री घ‚टादि‚{\tiny $_{lb}$}‚क‚र‚णे कुलालादिसापेक्षा । यो सौ क्षेत्रादिनिर‚पेक्षः स संयोग इति सिद्धं । किं‚{\tiny $_{lb}$}‚ चासौ संयोगो द्र‚व्य‚योर्विशेष‚ण‚भावेन प्र‚तीय‚मान‚त्वात्त‚तोर्थान्त‚र‚त्वेन प्र‚त्य‚क्ष‚सिद्ध‚{\tiny $_{lb}$}‚ एव । त‚था हि क‚श्चित् केन‚चित् संयुक्ते द्र‚व्ये आह‚रेत्युक्तो य‚योरेव द्र‚व्य‚योः संयोग‚{\tiny $_{lb}$}‚मुप‚ल‚भ‚ते ते एवाह‚र‚ति । न द्र‚व्य‚मात्रं । किं च‚{\tiny $_{७}$}‚ दूर‚त‚र‚व‚र्त्तिनः पुंसः सान्त‚रेपि व‚ने \leavevmode\ledsidenote{\textenglish{104b/PSVTa}}‚{\tiny $_{lb}$}‚ निर‚न्त‚र‚रूपाव‚सायिनी सेयं बुद्धिरुद‚य‚मासाद‚य‚ति मिथ्याबुद्धिर्मुख्य‚प‚दार्थानुभ‚व‚{\tiny $_{lb}$}‚म‚न्त‚रेण न क्व‚चिदुप‚जाय‚ते । न ह्य‚न‚नुभूत‚गोद‚र्श‚न‚स्य ग‚व‚ये गौरिति विभ्र‚मो‚{\tiny $_{lb}$}‚ भ‚व‚ति त‚स्माद‚व‚श्यं संयोगो मुख्योभ्युप‚ग‚न्त‚व्यः । त‚था न चैत्रः कुण्ड‚लीत्य‚नेन‚{\tiny $_{lb}$}‚ प्र‚तिषेध‚वाक्येन न कुण्ड‚लं प्र‚तिषिध्य‚ते त‚स्यान्य‚त्र देशादौ स‚त्त्वात् । त‚स्मा‚{\tiny $_{१}$}‚च्चैत्र‚स्य‚{\tiny $_{lb}$}‚ कुण्ड‚ल‚संयोगः प्र‚तिषिध्य‚ते । त‚था चैत्रः कुण्ड‚लीत्य‚नेनापि विधिवाक्येन न चैत्र‚कु‚{\tiny $_{lb}$}‚ण्ड‚ल‚योर‚न्य‚त‚र‚विधान‚न्त‚योः सिद्ध‚त्वात् । पारिशेष्यात् संयोग‚विधानं । त‚स्मा‚{\tiny $_{lb}$}‚द‚स्त्येव संयोग इति ।\edtext{\textsuperscript{*}}{\edlabel{pvsvt_281-1}\label{pvsvt_281-1}\lemma{*}\Bfootnote{\href{http://sarit.indology.info/?cref=nv}{ Nyāyavārtika. }}}
	{\color{gray}{\rmlatinfont\textsuperscript{§~\theparCount}}}
	\pend% ending standard par
      ‚{\tiny $_{lb}$}‚

	  
	  \pstart \leavevmode% starting standard par
	अत्रोच्य‚ते । १ य‚था क्षेत्रादीनां विशिष्टाव‚स्थाप्र‚तिल‚म्भेन संयोगार‚म्भ‚{\tiny $_{lb}$}‚‚{\tiny $_{lb}$}‚ ‚{\tiny $_{lb}$}‚ \leavevmode\ledsidenote{\textenglish{282/s}}क‚त्व‚मिष्य‚ते त‚था संयोग‚मंत‚रेण कार्यार‚म्भ‚क‚त्व‚मेव किन्नेष्य‚ते । अन्य‚था स‚र्व‚दा‚{\tiny $_{lb}$}‚ संयोगार‚म्भ‚{\tiny $_{२}$}‚क‚त्वं स्यात् । २ नापि निर्विक‚ल्प‚केन प्र‚त्य‚क्षेण संयुक्ते द्र‚व्ये स्व‚रूपेण‚{\tiny $_{lb}$}‚ गृह्य‚माणे तृतीयः संयोगः प्र‚तिभास‚ते । ३ नापि स‚विक‚ल्प‚के ज्ञाने संयुक्ते द्र‚व्ये‚{\tiny $_{lb}$}‚ मुक्त्वा संयोग‚श‚ब्दं चाप‚रः संयोगो विशेष‚ण‚भावेन प्र‚तिभास‚ते । ४ नापि संयुक्त‚{\tiny $_{lb}$}‚\textbf{प्र‚त्य‚यान्य‚था}नुप‚प‚त्त्या संयोग‚क‚ल्प‚ना । उत्प‚न्न‚निर‚न्त‚राव‚स्थ‚योरेव भाव‚योः संयुक्त‚{\tiny $_{lb}$}‚प्र‚त्य‚य‚हेतुत्वात् । याव‚च्च त‚स्याम‚व‚स्था‚{\tiny $_{३}$}‚यां संयोग‚ज‚न‚क‚त्वेन संयुक्त‚प्र‚त्य‚य‚विष‚यो‚{\tiny $_{lb}$}‚ ताविष्येते ताव‚त् संयोग‚म‚न्त‚रेण संयुक्त‚प्र‚त्य‚य‚हेतुत्वेन त‚द्विष‚यौ किं नेष्येते । किम्पार‚{\tiny $_{lb}$}‚म्प‚र्येण । ५ नापि सान्त‚रे व‚ने निर‚न्त‚राव‚भासिनी बुद्धिर्मुख्य‚प‚दार्थानुभ‚व‚{\tiny $_{lb}$}‚पूर्विका स्ख‚ल‚त्प्र‚त्य‚य‚विष‚य‚त्वेनानुप‚च‚रित‚त्वात् । ६ त‚था न चैत्रः कुण्ड‚लीत्यादौ‚{\tiny $_{lb}$}‚ चैत्र‚स‚म्ब‚न्धिकुण्ड‚लं प्र‚तिषिध्य‚ते विधीय‚ते वा । न संयोगः । त‚{\tiny $_{४}$}‚स्मादेक‚साम‚{\tiny $_{lb}$}‚ग्र्य‚धीन‚योरेव संयुताविति प्र‚तीतिः । य‚था कुण्ड‚ब‚द‚र‚योस्त‚स्मात् संयोग‚स्यापि कार्य‚{\tiny $_{lb}$}‚कार‚ण‚भाव एवान्त‚र्भावः । केव‚लं भेदान्त‚र‚प्र‚तिक्षेपेण संयुक्तावेतौ संयोग‚स्येति‚{\tiny $_{lb}$}‚ वा प्र‚तीतिर्न पुन‚र्व‚स्तुभूत‚संयोग‚ब‚लात् । य‚त‚श्च नास्ति कार्य‚कार‚ण‚भाव‚म‚न्त‚रेण‚{\tiny $_{lb}$}‚ वास्त‚वः स‚म्ब‚न्धः ।
	{\color{gray}{\rmlatinfont\textsuperscript{§~\theparCount}}}
	\pend% ending standard par
      ‚{\tiny $_{lb}$}‚

	  
	  \pstart \leavevmode% starting standard par
	\textbf{त‚द‚य‚मि}त्यादि । \textbf{ब‚द‚रादिषु ज‚न‚न‚श‚{\tiny $_{५}$}‚क्तिरेव कुण्डादीनामाधार इति स‚म्ब‚न्धः}‚{\tiny $_{lb}$}‚ सामान्य‚स्याश्र‚यो ज‚न‚न‚श‚क्त्यैवाधारोस्त्विति चेदाह । \textbf{नेत्या}दि । सेति ज‚न‚न‚श‚क्तिः ।‚{\tiny $_{lb}$}‚ \textbf{अत्रे}ति सामान्ये ।
	{\color{gray}{\rmlatinfont\textsuperscript{§~\theparCount}}}
	\pend% ending standard par
      ‚{\tiny $_{lb}$}‚

	  
	  \pstart \leavevmode% starting standard par
	\textbf{न ही}त्यादिना व्याच‚ष्टे । स्वोपादान‚देश एव ज‚न‚नं \textbf{ज‚न‚न‚विशेषः स ल‚क्ष‚णं}‚{\tiny $_{lb}$}‚ य‚स्या\textbf{धार‚भाव‚स्य} स त‚था । \textbf{त‚स्याज‚न्य}त्वादिति सामान्य‚स्य नित्य‚त्वेनाज‚न्य‚त्वात् ।‚{\tiny $_{lb}$}‚ \textbf{त‚द‚भावे}न्याश्र‚याभावेपि सामान्य‚स्या\textbf{व‚स्थिते}र्हेतोरा‚{\tiny $_{६}$}‚श्र‚य‚व‚शेन न \textbf{स्थितिः} सामान्य‚स्य ।
	{\color{gray}{\rmlatinfont\textsuperscript{§~\theparCount}}}
	\pend% ending standard par
      ‚{\tiny $_{lb}$}‚

	  
	  \pstart \leavevmode% starting standard par
	\textbf{अथेत्या}दिना व्याच‚ष्टे । \textbf{त‚स्ये}ति सामान्य‚स्य \textbf{त‚द‚भावेपी}ति व्य‚क्त्य‚भावेपि‚{\tiny $_{lb}$}‚ व्य‚क्तिश‚न्ये देशे सामान्य‚स्य \textbf{स्थाना}त् । य‚दि हि व्य‚क्तिशून्ये देशे सामान्यं न भ‚वेत्त‚दा‚{\tiny $_{lb}$}‚ त‚त्रापूर्व‚व्य‚क्त्युत्पादे सामान्य‚स‚म्ब‚न्धो न भ‚वेत । न हि त‚स्यान्य‚त आग‚म‚नं निष्क्रिय‚{\tiny $_{lb}$}‚‚{\tiny $_{lb}$}‚ \leavevmode\ledsidenote{\textenglish{283/s}}त्वात् । न च भिन्न‚देशाव‚स्थिता व्य‚क्तिस्त‚स्य सामान्य‚स्याधारो भिन्न‚देश‚त्वात् ।‚{\tiny $_{७}$}‚ \leavevmode\ledsidenote{\textenglish{105a/PSVTa}}‚{\tiny $_{lb}$}‚ एक‚त्वात् सामान्य‚स्य नास्ति भिन्न‚देश‚तेति चेत् । स‚र्वास्त‚र्हि व्य‚क्त‚य एक‚जातिम‚त्य‚{\tiny $_{lb}$}‚ एक‚देशाः प्राप्नुव‚न्ति । न च स‚र्वा जातिम‚त्यः स्युः । एक‚स्य क‚थ‚म्भिन्न‚देशाव‚स्थित‚{\tiny $_{lb}$}‚त्व‚मिति चेद‚य‚म‚प‚रोस्य दोषोस्तु ।
	{\color{gray}{\rmlatinfont\textsuperscript{§~\theparCount}}}
	\pend% ending standard par
      ‚{\tiny $_{lb}$}‚

	  
	  \pstart \leavevmode% starting standard par
	\textbf{प‚त‚न‚ध‚र्मे}त्यादिनोप‚च‚य‚हेतुमाह । \textbf{हि} श‚ब्द‚श्चार्थे । \textbf{अपि}श‚ब्दोभ्युप‚ग‚म‚{\tiny $_{lb}$}‚सूच‚नार्थः । अभ्युप‚ग‚म्याप्य‚यं प्र‚कारः सामान्ये व्य‚व‚स्थाप‚यितुम‚श‚क्यः । सामान्य‚{\tiny $_{lb}$}‚स्याप‚त‚न‚ध‚र्म्म‚{\tiny $_{१}$}‚त्वादित्येव‚म‚र्थ‚मुप‚न्यासः । न \textbf{त्व‚ज‚न‚क‚स्य स्थाप‚क}त्वं स\textbf{म्भ‚व}ति ।‚{\tiny $_{lb}$}‚ अत एवाह । \textbf{अत्रापी}त्यादि । \textbf{अत्रापि} पात‚प्र‚तिब‚न्धात् स्थाप‚काभ्युप‚ग‚मे \textbf{य‚दि} न्याय‚{\tiny $_{lb}$}‚वादी क‚श्चित् \textbf{पात‚प्र‚तिब‚न्धं न प‚र्य‚नुयुञ्जीत} । त‚दा भ‚वेद‚ज‚न‚कोपि \textbf{स्थाप‚कः} ।‚{\tiny $_{lb}$}‚ स्व‚स‚म‚यानुरोधेनेत्याकूतं । अत्राप्य‚यं प‚र्य‚नुयोगः स‚म्भ‚व‚ति । यः \textbf{स्थाप‚यित्रा} क्रिय‚ते‚{\tiny $_{lb}$}‚ \textbf{पात‚प्र‚तिब‚न्धः} स स्थाप्य‚स्यात्म‚भूतो वा स्यात् त‚तो\textbf{र्थान्त‚रं} वा‚{\tiny $_{२}$}‚ पाताभाव‚मात्र‚म्वा ।‚{\tiny $_{lb}$}‚ न ताव‚दात्म‚भूत‚स्त‚त्स्व‚भाव‚स्यान्य‚तो निष्प‚त्त्य‚भ्युप‚ग‚मात् नाप्य‚र्थान्त‚र‚मित्याह ।‚{\tiny $_{lb}$}‚ \textbf{अर्यान्त‚र‚त्वेभ्युप‚ग‚म्य‚माने त‚त्रैव} प्र‚तिब‚न्धेर्थान्त‚र‚भूतेऽस्याधार‚स्यो\textbf{प‚योग इति कः‚{\tiny $_{lb}$}‚ प‚त‚तो} ब‚द‚रादेः \textbf{प्र‚तिब‚न्धो} विघातः [।] नैव क‚श्चित् । त‚त‚श्च कुण्डादिस्थ‚म‚पि‚{\tiny $_{lb}$}‚ ब‚द‚रादि प‚तेदेवेति भावः । आधार‚कृतेनार्थान्त‚रेण पात‚प्र‚तिब‚न्धेन ब‚द‚रादेर‚पातः‚{\tiny $_{३}$}‚‚{\tiny $_{lb}$}‚ क्रिय‚त इति चेदाह । \textbf{प्र‚तिब‚न्धाद‚पातेपी}त्यादि । प्र‚तिब‚न्धाख्यात् य‚दार्थाद्‚{\tiny $_{lb}$}‚ ब‚द‚रादेर‚पातेभ्युप‚ग‚म्य‚माने \textbf{तुल्यः प‚र्य‚नुयोगः} । योयं प्र‚तिब‚न्धाख्येन प‚दार्थेनापातः‚{\tiny $_{lb}$}‚ क्रिय‚ते स किं ब‚द‚रादेरात्म‚भूतोर्थान्त‚र‚म्वा । अर्थान्त‚र‚त्वे त‚त्रैवास्य प्र‚तिब‚न्ध‚स्योप‚योग‚{\tiny $_{lb}$}‚ इत्यादि । स‚र्व‚म‚न‚न्त‚रोक्तं तुल्यं ।
	{\color{gray}{\rmlatinfont\textsuperscript{§~\theparCount}}}
	\pend% ending standard par
      ‚{\tiny $_{lb}$}‚

	  
	  \pstart \leavevmode% starting standard par
	अथ तेनाप्य‚पाताख्येनार्थेन ब‚द‚रादेर‚पातः क्रिय‚ते त‚त्रापि तुल्यः‚{\tiny $_{४}$}‚ प‚र्य‚नुयोग‚{\tiny $_{lb}$}‚ इत्याह । \textbf{अन‚व‚स्था} चेति ।
	{\color{gray}{\rmlatinfont\textsuperscript{§~\theparCount}}}
	\pend% ending standard par
      ‚{\tiny $_{lb}$}‚

	  
	  \pstart \leavevmode% starting standard par
	\textbf{त‚स्मादित्या}दिना तृतीय‚प‚क्षोप‚न्यासः । स \textbf{पाताभावः क‚थं केन‚चित् क्रिय‚ते} ।‚{\tiny $_{lb}$}‚ ‚{\tiny $_{lb}$}‚ \leavevmode\ledsidenote{\textenglish{284/s}}नैव केन‚चित् । अभाव‚स्याकार्य‚त्वादिति भावः । क‚थ‚न्त‚र्ह्य\textbf{भावं क‚रोतीति} व्य‚प‚देश‚{\tiny $_{lb}$}‚ इति \textbf{चेदा}ह । \textbf{अभाव}मित्यादि । अभाव‚ङ्क‚रोतीति व्य‚प‚देशे \textbf{नाभावो नाम क‚श्चित्‚{\tiny $_{lb}$}‚ कार्य} इष्य‚ते । क‚स्मादित्याह । \textbf{त‚स्ये}त्यादि । \textbf{त‚स्ये}त्य‚भाव‚स्य कार्य‚त्वाद् \textbf{भाव एव‚{\tiny $_{५}$}‚‚{\tiny $_{lb}$}‚ स्या}दित्य‚भिप्रायः ।
	{\color{gray}{\rmlatinfont\textsuperscript{§~\theparCount}}}
	\pend% ending standard par
      ‚{\tiny $_{lb}$}‚

	  
	  \pstart \leavevmode% starting standard par
	न‚नु य‚था घ‚ट‚व‚त् कार्य‚त्वात् प‚ट‚स्य न घ‚ट‚रूप‚ता । त‚था भाव‚व‚न्नाभाव‚स्य‚{\tiny $_{lb}$}‚ कार्य‚त्वाद‚भाव‚रूप‚ता भ‚विष्य‚तीति चेत् [।] न । घ‚टादेर‚पि हि भाव‚रूप‚त्व‚म्भ‚व‚न‚{\tiny $_{lb}$}‚ध‚र्म‚त्वादेव [।] त‚च्चाभावेप्य‚स्तीति क‚थं न भाव‚रूप‚त्व‚म‚भाव‚रूप‚त्वेन प्र‚तिभास‚{\tiny $_{lb}$}‚नान्न भाव‚रूप‚तेति चेत् [।] न [।] \textbf{अभाव‚स्य} प्र‚तिभासाभावात् । अभा‚{\tiny $_{lb}$}‚वानाम्प‚र‚स्प‚र‚विभाग‚प्र‚तीतेर्घ‚टाभावः प‚टाभाव‚{\tiny $_{६}$}‚ इत्य‚त्र प‚टादीनाम्भेदो नाभा‚{\tiny $_{lb}$}‚वानामेक‚त्वेन प्र‚तिभास‚नादित्युक्तं । य‚त एव\textbf{न्त‚स्माद् भा}व‚स्य या \textbf{क्रिया} त‚स्याः‚{\tiny $_{lb}$}‚ \textbf{प्र‚तिषेध‚निर्देशो}ऽभावंक‚रोतीति ।
	{\color{gray}{\rmlatinfont\textsuperscript{§~\theparCount}}}
	\pend% ending standard par
      ‚{\tiny $_{lb}$}‚

	  
	  \pstart \leavevmode% starting standard par
	अत एव स्प‚ष्ट‚य‚ति । \textbf{भावं न क‚रोतीति याव}दिति । यावानेवास्य वाक्य‚स्या‚{\tiny $_{lb}$}‚र्थ‚स्तावानेवाभावं क‚रोतीत्य‚स्यापीत्य‚र्थः । \textbf{त‚था चे}ति [।] पात‚प्र‚तिब‚न्ध‚स्याभाव‚{\tiny $_{lb}$}‚\leavevmode\ledsidenote{\textenglish{105b/PSVTa}} मात्र‚त्वेनाकार्य‚त्वे । \textbf{अय‚मि}ति कुण्डादिः । तेन कार‚{\tiny $_{७}$}‚णे\textbf{नाय‚मि}ति ब‚द‚रादिः । \textbf{केन‚{\tiny $_{lb}$}‚ चित्}कुण्डादिनाधारेण \textbf{प्र‚तिब}द्धः । पाताद‚निवारितो \textbf{न क‚दाचित्तिष्ठे}त् । स‚दैव प‚ते‚{\tiny $_{lb}$}‚दित्य‚र्थः । \textbf{त‚स्मादित्या}दिनोप‚संहारः । \textbf{अपि}श‚ब्दादाधेय इत्य‚नेनापि व्य‚प‚देशेन‚{\tiny $_{lb}$}‚ \textbf{क्ष‚णिकानां} पूर्व‚क्ष‚ण‚संगृहीतेनो\textbf{पादानेन स‚मान‚देश‚स्यो}त्त‚र‚क्ष‚ण‚संगृहीत‚स्य कार्य\textbf{स्यो‚{\tiny $_{lb}$}‚त्पाद‚न}मुच्य‚ते । \textbf{त‚स्मात्} सामान्येऽय‚म‚पि प्र‚कारो न स‚म्भ‚व‚{\tiny $_{१}$}‚तीति ख्याप‚नायाभ्युप‚{\tiny $_{lb}$}‚ग‚म्यैत‚दुक्तं पात‚प्र‚तिब‚न्धाद\textbf{ज‚न‚कोपि} स्थाप‚क इति ।
	{\color{gray}{\rmlatinfont\textsuperscript{§~\theparCount}}}
	\pend% ending standard par
      ‚{\tiny $_{lb}$}‚

	  
	  \pstart \leavevmode% starting standard par
	त‚मेवास‚म्भ‚व‚न्द‚र्श‚यितुमाह । \textbf{अस्तु नामे}त्यादि । \textbf{पातिना}म्ब‚द‚स‚दीनान्त\textbf{त्प्र‚ति‚{\tiny $_{lb}$}‚ब‚न्धः} पात‚प्र‚तिब‚न्धोस्तु नामाज‚न‚न‚स्व‚भावः । \textbf{त‚त्क‚र‚णा}दिति पात‚प्र‚तिब‚न्ध‚क‚र‚णात् ।‚{\tiny $_{lb}$}‚ \textbf{ग‚तिम‚तो द्र‚व्य‚स्येति} स‚क्रिय‚स्य सामान्य‚स्य पुन‚र‚मूर्त्त‚त्वाद\textbf{क्रिय‚स्य किं ल‚क्ष‚णां स्थितिं‚{\tiny $_{lb}$}‚ कुर्वाण} आश्र‚यः \textbf{स्थाप‚{\tiny $_{२}$}‚कः स्यात्} । न हि सामान्य‚स्य पातोऽस्ति येन त‚त्प्र‚तिब‚न्धः‚{\tiny $_{lb}$}‚ ‚{\tiny $_{lb}$}‚ \leavevmode\ledsidenote{\textenglish{285/s}}स्थितिर्भ‚वेत् । किन्तु \textbf{स्थितिर्हि त‚स्य} सामान्य‚स्य \textbf{स्व‚रूपाप्र‚च्युतिरेवो}च्य‚ते । \textbf{सा च}‚{\tiny $_{lb}$}‚ स्व‚रूपाप्र‚च्युति\textbf{र्नाश्र‚याय‚त्ता} सामान्य‚स्य \textbf{नित्य‚त्वात्} । अभ्युप‚ग‚म्याप्युच्य‚ते ।‚{\tiny $_{lb}$}‚ \textbf{साप्या}श्र‚याय‚त्ता सामान्य‚स्य स्थिति\textbf{र‚युक्तैव} । सामान्यात् त‚स्याः स्थिते\textbf{र्भेदा‚{\tiny $_{lb}$}‚भेद‚विवेच‚ने} । अन्य‚त्त्वान‚न्य‚त्त्व‚विचारे क्रिय‚माणे ।
	{\color{gray}{\rmlatinfont\textsuperscript{§~\theparCount}}}
	\pend% ending standard par
      ‚{\tiny $_{lb}$}‚

	  
	  \pstart \leavevmode% starting standard par
	\textbf{अस्तु नामे}‚{\tiny $_{३}$}‚त्यादिना व्याच‚ष्टे । \textbf{आश्र‚य‚हेतुके}त्याश्र‚याय‚त्ता । \textbf{सेति} स्थितिः ।‚{\tiny $_{lb}$}‚ तामेवाश्र‚याद‚न्यां स्थितिं स आश्र‚यः क‚रोति न सामान्यं । सा स्थितिः सामान्ये‚{\tiny $_{lb}$}‚ प्र‚तिब‚द्धा त‚तः स‚म्ब‚न्ध‚स‚म्ब‚द्धात् सामान्य‚मुप‚कृत‚मेवेत्य‚त आह । \textbf{सा चे}त्यादि ।‚{\tiny $_{lb}$}‚ सेत्य‚र्थान्त‚र‚भूता स्थितिः । न हि त‚स्याः सामान्ये प्र‚तिब‚न्ध‚कार‚णं किंचिद‚स्ति‚{\tiny $_{lb}$}‚ किं सामान्य‚स्याश्र‚येण कृत‚म्भ‚व‚तीत्य‚ध्याहारः ।
	{\color{gray}{\rmlatinfont\textsuperscript{§~\theparCount}}}
	\pend% ending standard par
      ‚{\tiny $_{lb}$}‚

	  
	  \pstart \leavevmode% starting standard par
	अभ्युप‚ग‚{\tiny $_{४}$}‚म्य‚त एव स्थितेः सामान्ये प्र‚तिब‚न्ध इति चेदाह । \textbf{प्र‚तिब‚न्धे चे}त्यादि ।‚{\tiny $_{lb}$}‚ प्र‚तिब‚न्धे वाभ्युप‚ग‚म्य‚माने । \textbf{स्थितिक‚र‚णं चेत्} । आश्र‚येण ज‚निता या स्थितिस्त‚स्याः‚{\tiny $_{lb}$}‚ स्थितेः स्थितिः सामान्येन क्रिय‚ते । त‚तः \textbf{सा}श्र‚य‚ज‚निता स्थितिः \textbf{सामा}न्ये‚{\tiny $_{lb}$}‚ \textbf{प्र‚तिब‚द्धेति । त‚त्रापि} स्थितेः स्थितिक‚र‚णे \textbf{तुल्यः प्र‚स‚ङ्गः} । या \textbf{सा चा}श्र‚य‚प्र‚ति‚{\tiny $_{lb}$}‚ब‚द्धायाः स्थितेः सामान्येन स्थितिः क्रिय‚ते सा‚{\tiny $_{५}$}‚ आश्र‚य‚हेतुकायाः स्थितेरात्म\textbf{भूता वा}‚{\tiny $_{lb}$}‚ भ‚वेद् \textbf{व्य‚तिरिक्ता वा} । आत्म‚भूत‚त्वे आश्र‚येणैव सा कृतेति क‚थं सामान्येन क्रिय‚ते ।‚{\tiny $_{lb}$}‚ \textbf{व्य‚तिरिक्त‚त्वे} च सैव स्थितिः सामान्येन कृता आश्र‚य‚ज‚निताया आद्यायाः स्थितेः‚{\tiny $_{lb}$}‚ \textbf{किं सामान्ये}न \textbf{कृतं स्या}त् । अथ सामान्येन द्वितीया स्थितिः क्रिय‚ते । सा आश्र‚येण‚{\tiny $_{lb}$}‚ ज‚नितायां स्थितौ प्र‚तिब‚द्धा । \textbf{त‚दा कः प्र‚तिब‚न्ध इति वाच्यं} । सामान्य‚{\tiny $_{६}$}‚ज‚नितायाः‚{\tiny $_{lb}$}‚ स्थितेराश्र‚य‚ज‚नित‚या स्थित्याऽप‚रा तृतीया स्थितिः य‚त इति \textbf{तुल्यः प्र‚स‚ङ्गः} इत्य‚न‚{\tiny $_{lb}$}‚\textbf{व‚स्था स्यात् । त‚तो}न‚व‚स्थानादाश्र‚य‚ज‚नितायां स्थितौ सामान्य‚कृत‚स्यो\textbf{प‚कार}स्या‚{\tiny $_{lb}$}‚\textbf{न‚व‚धार‚णाद‚स्य} सामान्य‚स्य स‚म्ब‚न्धिनीय‚माश्र‚येण ज‚निता \textbf{स्थितिरित्य‚प्र‚ती}तिः‚{\tiny $_{lb}$}‚ \textbf{ज‚न‚नं चेत्}प्र‚तिब‚न्ध इति प्र‚कृतं । न सामान्येनाश्र‚य‚ज‚निताया स्थितेर‚{\tiny $_{७}$}‚प‚रा स्थितिः \leavevmode\ledsidenote{\textenglish{106a/PSVTa}}‚{\tiny $_{lb}$}‚ क्रिय‚ते । किन्तु सैवाद्या स्थितिर्ज‚न्य‚त इति । त‚दा केव‚लं सामान्यं स‚म‚र्थं स्थितिं‚{\tiny $_{lb}$}‚ ‚{\tiny $_{lb}$}‚ \leavevmode\ledsidenote{\textenglish{286/s}}क‚रोतु \textbf{किमाश्र‚येण} शाव‚लेयादिनां स्थितिक‚र‚णाया\textbf{पेक्षितेन} । न \textbf{ह्य‚नुप‚कार‚रिण्य‚पेक्षा}‚{\tiny $_{lb}$}‚ युक्ता । त‚स्माद\textbf{पेक्षेति हि त‚त्प्र‚तिब‚न्धः} । अस्मिन्व‚स्तुन्य‚स्यापेक्षेति येय‚म‚पेक्षा सा‚{\tiny $_{lb}$}‚ त‚स्मिन्न‚पेक्ष्ये प्र‚तिब‚न्ध‚स्त‚दाय‚त्त‚ता । \textbf{स च} प्र‚तिब‚न्धो नित्य‚त्वाद‚नाधेयातिश‚य\textbf{स्या‚{\tiny $_{lb}$}‚युक्त} इति \textbf{केव‚लं} सामान्यं ‚{\tiny $_{१}$}‚स्थितिं \textbf{ज‚न‚येदिति नास्त्य‚न्य आश्र‚यः स्थितिहेतुः} ।‚{\tiny $_{lb}$}‚ त‚त‚श्च स्थितिक‚र‚णादाश्र‚य‚स्सामान्य‚स्याधार इत्येत‚द‚युक्त‚मिति भावः । एव\textbf{म्भेदा‚{\tiny $_{lb}$}‚भेद‚विवेच‚न} इति य‚दुक्त‚न्त‚तो भेद‚प‚क्ष‚स्ताव‚द‚प‚नीतः ।
	{\color{gray}{\rmlatinfont\textsuperscript{§~\theparCount}}}
	\pend% ending standard par
      ‚{\tiny $_{lb}$}‚

	  
	  \pstart \leavevmode% starting standard par
	द्वितीय‚प‚क्ष‚माश्रित्याह । \textbf{अभेद} इत्यादि । \textbf{सामान्याद‚भेदे वा स्थिते}र‚भ्युप‚{\tiny $_{lb}$}‚ग‚म्य‚माने \textbf{स्व‚रूप‚मेव त‚त्स्थि}तिरूपं सामान्य‚स्य [।] \textbf{त‚च्च} स्व‚रूपं सामान्य‚स्य‚{\tiny $_{lb}$}‚ \textbf{नित्य‚म‚स्तीति न स्थिति‚{\tiny $_{२}$}‚ र‚स्य} सामान्य‚स्य \textbf{केन}चिदाश्र‚येण \textbf{क्रिय}ते । य‚त एव\textbf{न्त‚स्मा‚{\tiny $_{lb}$}‚दि}त्यादि । \textbf{त‚दि}त्यादिनोप‚संहारः । \textbf{त‚दि}ति त‚स्मा\textbf{द‚स्ये}ति सामान्य‚स्य ।
	{\color{gray}{\rmlatinfont\textsuperscript{§~\theparCount}}}
	\pend% ending standard par
      ‚{\tiny $_{lb}$}‚

	  
	  \pstart \leavevmode% starting standard par
	त‚देवं वृत्तिराधेय‚ता व्य‚क्तिरिति य‚त्प‚क्ष‚द्व‚य‚मुक्त‚न्त‚त आद्य‚स्य निरासः कृतः [।]
	{\color{gray}{\rmlatinfont\textsuperscript{§~\theparCount}}}
	\pend% ending standard par
      ‚{\tiny $_{lb}$}‚

	  
	  \pstart \leavevmode% starting standard par
	२--द्वितीय‚प‚क्ष‚मा\textbf{श्रि}त्याह । \textbf{अथ पुन}रित्यादि । \textbf{अव्य‚क्त‚स्ये}त्य‚प्र‚काशित‚स्य ।‚{\tiny $_{lb}$}‚ व्य‚क्तेत्याश्र‚येण ज्ञान‚स्याकार‚ण‚त्वात् त‚द्व्य‚क्तेस्तेनाश्र‚येण प्र‚काश‚नं य‚त् । त‚{\tiny $_{३}$}‚देव‚{\tiny $_{lb}$}‚ त‚त्राश्र‚ये सामान्य‚स्य वृत्तिः स्यात् । \href{http://sarit.indology.info/?cref=pv.3.143-144}{। १४६-४७ ॥}
	{\color{gray}{\rmlatinfont\textsuperscript{§~\theparCount}}}
	\pend% ending standard par
      ‚{\tiny $_{lb}$}‚

	  
	  \pstart \leavevmode% starting standard par
	\textbf{ने}त्या चा र्यः । आंत्म‚नि स्व‚विष‚ये \textbf{विज्ञानोत्पा}द‚नं । त‚त्र \textbf{योग्य}त्वं सामान्य‚न्त‚द‚र्थ‚{\tiny $_{lb}$}‚\textbf{म‚न्यानुरोधि} । कार‚णान्त‚र‚सापेक्षं \textbf{य‚त्त}द्व‚स्तु \textbf{व्य‚ङ्ग्यं} प्र‚तीतं । त‚स्याश्च स्व‚विष‚य‚ज्ञान‚{\tiny $_{lb}$}‚ज‚न‚न\textbf{योग्य‚तायाः कार‚णं} य‚त्प्र‚दीपादि त‚द्व्य‚ङ्ग्य‚स्य \textbf{कार‚क‚मेव} ज‚न‚क‚मेव । पूर्व‚म‚{\tiny $_{lb}$}‚योग्य‚स्य प‚श्चाद् विज्ञान‚ज‚न‚न‚योग्य‚स्य घ‚टादेरुत्पाद‚नात् ।
	{\color{gray}{\rmlatinfont\textsuperscript{§~\theparCount}}}
	\pend% ending standard par
      ‚{\tiny $_{lb}$}‚

	  
	  \pstart \leavevmode% starting standard par
	य‚दि‚{\tiny $_{४}$}‚ पुनः प्र‚दीपादिस‚न्निधानात् प्राग‚पि घ‚टादि स्वाकार‚ज्ञान‚ज‚न‚न\textbf{योग्}य‚{\tiny $_{lb}$}‚न्त‚दा \textbf{प्रागेवास्य च} घ‚टादे\textbf{र्योग्य}त्वे \textbf{त‚द‚पेक्षे}ति प्र‚दीपापेक्षा । \textbf{सामान्य‚स्य} नित्य‚त्वाद‚{\tiny $_{lb}$}‚‚{\tiny $_{lb}$}‚ \leavevmode\ledsidenote{\textenglish{287/s}}\textbf{विकार्य}स्य । \textbf{त‚दि}ति य‚थोक्त‚ल‚क्ष‚णं व्य‚ङ्ग्य‚त्वं \textbf{सामान्य‚व}त इत्याश्र‚यात्स‚काशात् \textbf{कुतो}‚{\tiny $_{lb}$}‚ नैवेत्य‚र्थः । \textbf{अप‚र‚मिति} स्व‚स‚न्तानाद‚न्य\textbf{म्भाव‚मेव ज‚न‚य‚न् व्य‚ञ्ज}क उच्य‚ते । किम्भूतं‚{\tiny $_{lb}$}‚ \textbf{स्व‚विष‚य}स्य \textbf{विज्ञानोत्पाद‚{\tiny $_{५}$}‚न‚स‚म}र्थं । \textbf{स‚जातीयोपादानापेक्ष‚मि}ति स्व‚स‚न्तान‚स‚ङ्गृ‚{\tiny $_{lb}$}‚हीत‚पूर्व‚क्ष‚ण‚सापेक्षं । य‚थान्ध‚काराव‚स्थित‚घ‚टादिक्ष‚ण‚सापेक्ष‚म्विज्ञान‚ज‚न‚न‚स‚म‚र्थ‚मुत्त‚रं‚{\tiny $_{lb}$}‚ घ‚ट‚क्ष‚ण‚ञ्ज‚न‚य‚त्प्र‚काश‚कः ।
	{\color{gray}{\rmlatinfont\textsuperscript{§~\theparCount}}}
	\pend% ending standard par
      ‚{\tiny $_{lb}$}‚

	  
	  \pstart \leavevmode% starting standard par
	न‚नु प्र‚दीप‚कार्य‚त्वे घ‚ट‚स्य प्र‚दीपोय‚च‚येपि घ‚ट‚स्योप‚च‚योपि स्यादिति चेत् [।]‚{\tiny $_{lb}$}‚ न । उपादान‚ग‚ताद् विभेदात् कार्य‚स्य भेदो न स‚ह‚कारिग‚तात् । स‚ह‚कारिकार‚णं‚{\tiny $_{lb}$}‚ च प्र‚{\tiny $_{६}$}‚दीपादिर्व्य‚ङ्ग्य‚स्य घ‚ट‚स्येति कुतो म‚ह‚त्वादिप्र‚स‚ङ्गोस्य [।] य‚द्वाऽभिव्य‚क्ता‚{\tiny $_{lb}$}‚व‚पि क्रिय‚माणायां तुल्योयं प्र‚स‚ङ्ग इति य‚त्किञ्चिदेत‚त् । न च व्य‚ङ्ग्य‚क्ष‚ण‚स‚दृश‚स्य‚{\tiny $_{lb}$}‚ क्ष‚ण‚स्य मृत्पिण्डादुत्प‚त्तिर‚पि तु प्र‚दीपादेवेति कुतोन्यादृशात् तादृश‚स्योत्प‚त्तिः ।‚{\tiny $_{lb}$}‚ \textbf{अन‚पेक्षं चे}ति । य‚था स‚जातीयोपादानापेक्षं स्व‚विष‚य‚विज्ञान‚ज‚न‚न‚स‚म‚र्थं श‚ब्दं ज‚न‚{\tiny $_{lb}$}‚य‚न्न‚भिघातः । न ह्य‚{\tiny $_{७}$}‚भिघातात् प्राक् छ‚ब्दोस्ति येन स‚मान‚जातीयापेक्षः श‚ब्दो \leavevmode\ledsidenote{\textenglish{106b/PSVTa}}‚{\tiny $_{lb}$}‚ भ‚वेत् । श‚ब्दोपि हि व्य‚ङ्ग्यः प‚रैरिष्य‚त इत्येव‚मुक्तं ।
	{\color{gray}{\rmlatinfont\textsuperscript{§~\theparCount}}}
	\pend% ending standard par
      ‚{\tiny $_{lb}$}‚

	  
	  \pstart \leavevmode% starting standard par
	य‚दि त‚र्हि कार‚क एव व्य‚ञ्ज‚कः क‚स्त‚र्हि कार‚क‚व्य‚ञ्ज‚क‚योर्हेत्वोर्विशेष इत्य‚त‚{\tiny $_{lb}$}‚ आह । \textbf{प‚र‚त्रे}त्यादि । व्य‚ञ्ज‚काद‚न्य‚स्मिन् कार‚क‚त्वेनाभिम‚त इत्य‚र्थः । \textbf{ज्ञान‚ज‚न‚{\tiny $_{lb}$}‚श‚क्तिर‚नाक्षिप्ता ज‚न्य‚स्य} । न हि स्व‚विष‚य‚विज्ञान‚ज‚न‚न‚स‚म‚र्थ‚मेव कार्यं कार‚केण‚{\tiny $_{lb}$}‚ बीजा‚{\tiny $_{१}$}‚दिना ज‚न्य‚ते । त‚तो \textbf{ज‚न‚न‚मात्रेण कार‚क‚त्वं} स्व‚विष‚य‚विज्ञान‚ज‚न‚न‚स‚म‚र्थ‚{\tiny $_{lb}$}‚कार्योत्पाद‚न‚ल‚क्ष‚णेन तु विशेषेण व्य‚ञ्ज‚क‚त्व‚मिति । \href{http://sarit.indology.info/?cref=pv.3.144-3.145}{१४७-४८ ॥}
	{\color{gray}{\rmlatinfont\textsuperscript{§~\theparCount}}}
	\pend% ending standard par
      ‚{\tiny $_{lb}$}‚

	  
	  \pstart \leavevmode% starting standard par
	य‚द्य‚पि व्य‚ञ्ज‚काद् व्य‚ङ्ग्यो \textbf{विज्ञानोत्पाद‚न‚योग्य‚तां} प्र‚तिल‚भ‚ते त‚थापि न ज‚न्य‚त‚{\tiny $_{lb}$}‚ इति चेदाह । \textbf{य‚दि ही}त्यादि । \textbf{य‚त} इति व्य‚ञ्ज‚कात् । स चेत् व्य‚ङ्ग्यः । \textbf{त‚स्ये}ति‚{\tiny $_{lb}$}‚ व्य‚ञ्ज‚क‚स्य \textbf{सा योग्य}ताऽस्य व्य‚ङ्ग्य‚स्य व्य‚ञ्ज‚क‚स‚न्निधानात् \textbf{प्रागेवास्ति} ।‚{\tiny $_{२}$}‚ य‚तो‚{\tiny $_{lb}$}‚ ‚{\tiny $_{lb}$}‚ \leavevmode\ledsidenote{\textenglish{288/s}}व्य‚ङ्ग्य‚स्य \textbf{स्व‚भाव‚भूता सा} । य‚थाव्य‚ङ्ग्यः प्रागेवास्ति त‚था त‚त्स्व‚भाव‚भूतापि‚{\tiny $_{lb}$}‚ योग्य‚ता । \textbf{त‚म‚पेक्ष‚त} इति व्य‚ञ्ज‚कं । व्य‚ङ्ग्याद् व्य‚तिरिक्तैव योग्य‚ता व्य‚ञ्ज‚केन‚{\tiny $_{lb}$}‚ क्रिय‚त इति चेदाह । \textbf{प‚रे}त्यादि । \textbf{अस्यामि}ति योग्य‚तायां । \textbf{सैव} योग्य‚ता । \textbf{त‚त} इति‚{\tiny $_{lb}$}‚ व्य‚ञ्ज‚कात् \textbf{स्थितिव‚त्प्र‚स‚ङ्गः} । य‚दुक्त‚म् [।] अन्या चेत् स्थितिस्तामेवाश्र‚यः क‚रो‚{\tiny $_{lb}$}‚तीत्यादि त‚दिहापि प्र‚स‚ज्येत । \textbf{त}‚{\tiny $_{३}$}‚मिति व्य‚ङ्ग्यं । नापि व्य‚ङ्ग्याद‚न्य‚त् त‚त्क‚र‚णे‚{\tiny $_{lb}$}‚ व्य‚ङ्ग्य‚स्य न \textbf{किञ्चि}दिति \textbf{कृत्वाऽपेक्ष्य‚त} इत्य‚नेनोप‚कारित्व‚मुक्तं । अकिंचित्क‚र‚त्वेन‚{\tiny $_{lb}$}‚ त‚त्प्र‚तिषेध‚स्\textbf{त‚तो व्याह‚त‚मेत}त् ।
	{\color{gray}{\rmlatinfont\textsuperscript{§~\theparCount}}}
	\pend% ending standard par
      ‚{\tiny $_{lb}$}‚

	  
	  \pstart \leavevmode% starting standard par
	य‚दुक्तं [।] ज‚न‚क एव व्य‚ञ्ज‚क इति त‚स्य त‚त्त्वित्यादिना व्य‚भिचार‚माह ।‚{\tiny $_{lb}$}‚ न हि धूमोग्नेर्ज‚न‚कोऽथ च \textbf{कार्य‚त्वा}त्त‚स्य \textbf{व्य‚ञ्ज}कः । \textbf{आदि}श‚ब्दाद् ब‚लाकादिः‚{\tiny $_{lb}$}‚ स‚लिल‚स्य [।] \textbf{स‚त्त्य}मित्यादिना प‚रिह‚र‚ति । न तु धूमं लिङ्ग\textbf{म‚{\tiny $_{४}$}‚पेक्ष्याग्निरा}त्म‚नि‚{\tiny $_{lb}$}‚ स्व‚ल‚क्ष‚णे \textbf{ज्ञानं ज‚न‚य}ति । क‚स्मात् [।] \textbf{त‚थाभूत‚स्}यानुमेय‚त्वेनाभिम‚त‚स्\textbf{याग्नेः साक्षा‚{\tiny $_{lb}$}‚द‚ज‚न‚क‚त्वात्} । अन्य‚थाग्निस्व‚ल‚क्ष‚णाकार‚त्वात् प्र‚त्य‚क्षात् प्र‚तिभासाविशेषः स्यात् ।‚{\tiny $_{lb}$}‚ \textbf{केव‚ल‚मि}त्यादिनोपादान‚कार‚ण‚मेव त‚स्य साक्षाज्ज‚न‚क‚मित्याद‚र्श‚य‚ति । लिङ्ग‚ज्ञान‚{\tiny $_{lb}$}‚मुपादानं । \textbf{न विष‚य‚ब‚लेना}ग्निस्व‚ल‚क्ष‚ण‚ब‚लेन [।] किङ्कार‚ण‚म् [।] अ\textbf{स‚त्य}पि‚{\tiny $_{५}$}‚‚{\tiny $_{lb}$}‚ त‚स्मिन् व‚ह्नौ पूर्व‚ध्व‚स्तेपि भावाद‚ग्निज्ञान‚स्य । क‚थ‚मित्याह । \textbf{प‚र‚म्प‚र}येत्यादि ।‚{\tiny $_{lb}$}‚ लिङ्गानुसारी लिङ्गानुस्म‚र‚ण‚विक‚ल्प‚स्तेन । त‚था हि क‚स्य‚चित् पुरुष‚स्य क्व‚चिद्‚{\tiny $_{lb}$}‚ \textbf{धू}म‚न्दृष्ट‚व‚तो ध्व‚स्ते धूमे व‚ह्नौ च क‚थ‚म‚पि त‚त्र धूमानुस्म‚र‚ण‚विक‚ल्प उत्प‚न्ने प‚श्चा‚{\tiny $_{lb}$}‚द‚न्व‚य‚व्य‚तिरेकानुस्म‚र‚णाद‚भूद‚त्र धूम‚स्त‚स्माद् व‚ह्निर‚प्य‚त्रासीदित्येवं प‚र‚म्प‚र‚याग्नि‚{\tiny $_{lb}$}‚ज‚न्य‚{\tiny $_{५}$}‚धूमाद‚ग्निज्ञान‚मुत्प‚द्य‚त एव ।
	{\color{gray}{\rmlatinfont\textsuperscript{§~\theparCount}}}
	\pend% ending standard par
      ‚{\tiny $_{lb}$}‚

	  
	  \pstart \leavevmode% starting standard par
	\textbf{नापी}त्यादिना पूर्वोक्त‚म‚त्रैव योज‚य‚ति । \textbf{सामान्याकाराव‚भासि} चानुमा\textbf{न‚ज्ञानं} ।‚{\tiny $_{lb}$}‚ न \textbf{स‚न्निहित‚विष}य‚ता । विन‚ष्टेपि हि विष‚ये अनुत्प‚न्ने च स‚म्भ‚वात् । य‚दापि स‚न्नि‚{\tiny $_{lb}$}‚हितो विष‚य‚स्त‚दापि \textbf{न विष‚य‚ब‚लेनोत्प‚त्तिरिति निवेदितं} प्राक् \textbf{न हि विक‚ल्पा‚{\tiny $_{lb}$}‚ ‚{\tiny $_{lb}$}‚ \leavevmode\ledsidenote{\textenglish{289/s}}य‚थाभाव‚मेव प्र‚व‚र्त्त‚न्त} इत्यादिना । \textbf{भावाभावान‚विधानाच्च सा‚{\tiny $_{७}$}‚म‚र्थ्य‚न्न प्र‚ति- \leavevmode\ledsidenote{\textenglish{107a/PSVTa}}‚{\tiny $_{lb}$}‚ भासा}दित्यादिना तृतीये प‚रिच्छेदे \href{http://sarit.indology.info/?cref=}{ }\textbf{प्र‚तिपाद‚यिष्य‚ते च । साक्षादुप‚योगेन}‚{\tiny $_{lb}$}‚ स्व‚रूपानुकारि\textbf{विज्ञान‚ज‚न‚न‚साम‚र्थ्येन । त‚त्रे}ति स्व‚विष‚य‚ज्ञान‚ज‚न‚ने । प‚र‚मिति प्र‚दी‚{\tiny $_{lb}$}‚पादिकं । \textbf{त‚त} इत्य‚पेक्ष्यात् प्र‚दीपादेः ।
	{\color{gray}{\rmlatinfont\textsuperscript{§~\theparCount}}}
	\pend% ending standard par
      ‚{\tiny $_{lb}$}‚

	  
	  \pstart \leavevmode% starting standard par
	एत‚दुक्त‚म्भ‚व‚ति । न स‚र्वो व्य‚ञ्ज‚को ज‚न‚क इत्युच्य‚ते [।] किन्तु स्वाकार‚{\tiny $_{lb}$}‚ज्ञान‚ज‚न‚क‚स्य प‚र‚स्य साहाय्यं यः प्र‚तिप‚द्य‚ते स एव । त‚तो नास्ति व्य‚भिचार इ‚{\tiny $_{१}$}‚ति ।‚{\tiny $_{lb}$}‚ सामान्य‚म‚पि स्वाकार‚ज्ञान‚ज‚न‚नाश्र‚य\textbf{म‚पेक्ष‚त इति व्य‚ङ्ग्य}मिष्ट‚न्त‚त‚स्तेनाश्र‚यादुप‚{\tiny $_{lb}$}‚ल‚म्भ‚योग्य \textbf{आत्मा ल‚ब्ध‚व्यः [।] न चाय‚मात्म‚प्र‚तिल‚म्भः सामान्य‚स्य कुत‚श्चित्‚{\tiny $_{lb}$}‚ स‚म्भ‚व‚ति} । नित्य‚त्वेनाभ्युप‚ग‚त‚त्वात् । \textbf{त‚दि}ति सामान्यं ।
	{\color{gray}{\rmlatinfont\textsuperscript{§~\theparCount}}}
	\pend% ending standard par
      ‚{\tiny $_{lb}$}‚

	  
	  \pstart \leavevmode% starting standard par
	\textbf{नैवे}त्यादि प‚रः । स्वाश्र‚य‚स‚म‚वाय‚व्य‚क्तिं ब्रूम इति स‚म्ब‚न्धः । स्वाश्र‚य‚स‚म‚वायः‚{\tiny $_{lb}$}‚ क‚थं व्य‚क्तिरिति चेदाह । \textbf{स्वाश्र‚ये}त्यादि । \textbf{त‚दि}ति सामा‚{\tiny $_{२}$}‚न्यं । \textbf{अन्य‚त्रेति} स्वाश्र‚ये ।
	{\color{gray}{\rmlatinfont\textsuperscript{§~\theparCount}}}
	\pend% ending standard par
      ‚{\tiny $_{lb}$}‚

	  
	  \pstart \leavevmode% starting standard par
	\textbf{उक्त}मित्या चा र्यः । त‚देवेद‚म‚नुप‚कार‚क‚स्याश्र‚य‚त्वं न स‚म्भाव‚याम इत्यादि‚{\tiny $_{lb}$}‚नोक्त‚त्वात् । स्वाश्र‚य‚स‚म‚वेतं हि \textbf{त‚दात्म‚न्य‚न्य‚त्र वा विज्ञान‚हेतुरिति} ब्रुवाणेन \textbf{स्वाश्र‚य‚{\tiny $_{lb}$}‚सावायापेक्षः} सामान्य‚प‚दार्थः \textbf{विज्ञान‚हेतु}रिष्टः । त‚त‚श्च \textbf{तेन} स्वाश्र‚य‚स‚म‚वायेन‚{\tiny $_{lb}$}‚ सामान्यात्मा \textbf{ज‚न्य‚स्य स्या}त् । किङ्कार‚णं [।] \textbf{त‚द्धेतो}र्ज्ञान‚हेतोः स्व‚भाव‚स्य स्वा‚{\tiny $_{३}$}‚‚{\tiny $_{lb}$}‚श्र‚य‚स‚म‚वायात् \textbf{प्राग्भावा}त् । स्वाश्र‚य‚स‚म‚वाये स‚ति \textbf{प‚श्चाच्च त‚तः} स्वाश्र‚य‚स‚म‚वायाद्‚{\tiny $_{lb}$}‚ विज्ञान‚हेतोः स्व‚भाव‚स्य \textbf{भावा}त् । नित्यं सामान्य‚स्य विज्ञान‚ज‚न‚न‚स्व‚भाव‚त्वाद‚सिद्ध‚{\tiny $_{lb}$}‚मेत‚दिति चेदाह । \textbf{नित्त्य}मित्यादि । \textbf{त‚त्स्व‚भाव‚स‚द्भाव} इति विज्ञान‚ज‚न‚न‚स्व‚भावे‚{\tiny $_{lb}$}‚ \textbf{प्राग‚पि} स्वाश्र‚य\textbf{स‚म‚वायाद्} व्य‚क्तिशून्येपि देशे केव‚लात् सामान्यादित्य‚र्थः । सामान्या‚{\tiny $_{lb}$}‚कार\textbf{विज्ञा‚{\tiny $_{४}$}‚नोद‚य‚प्र‚स‚ङ्गा}त् । \href{http://sarit.indology.info/?cref=pv.3.145-3-146}{। १४८-४९ ॥}
	{\color{gray}{\rmlatinfont\textsuperscript{§~\theparCount}}}
	\pend% ending standard par
      ‚{\tiny $_{lb}$}‚

	  
	  \pstart \leavevmode% starting standard par
	\textbf{ने}त्यादिप‚रः । \textbf{व्य‚क्ति}राश्र‚यः \textbf{सामान्य‚स्य संस्का}रो योग्य‚ताधान‚न्त\textbf{स्माद्धे}तोर्न‚{\tiny $_{lb}$}‚ ‚{\tiny $_{lb}$}‚ \leavevmode\ledsidenote{\textenglish{290/s}}\textbf{व्य‚ञ्जिका} सामान्य‚स्य । येन त‚या ज‚न्यं स्यात् सामान्यं । \textbf{किन्त‚र्हि त‚द्ग्राहिण} इति‚{\tiny $_{lb}$}‚ सामान्य‚ग्राहिणः \textbf{संस्कारा}द् व्य‚ञ्ज‚केति प्र‚कृतं ।
	{\color{gray}{\rmlatinfont\textsuperscript{§~\theparCount}}}
	\pend% ending standard par
      ‚{\tiny $_{lb}$}‚

	  
	  \pstart \leavevmode% starting standard par
	\textbf{योपी}त्याद्या चा र्यः । \textbf{अञ्ज‚नादे}रिवेति वैध‚र्म्य‚दृष्टान्तः । \textbf{अञ्ज‚नादेः} स‚काशाद्‚{\tiny $_{lb}$}‚ य‚थेन्द्रिय\textbf{संस्कारो} युक्तो \textbf{नैवं व्य‚क्तेः} स‚का‚{\tiny $_{५}$}‚शात् । क‚स्मात् \textbf{प्र‚तिप‚त्ते}र्ज्ञान‚स्य व्य‚ञ्ज‚{\tiny $_{lb}$}‚क‚त्वेनाभिम‚ताया व्य‚क्ते\textbf{र्भावाभाव‚काल‚योः} स‚प्त‚मीद्विव‚च‚न‚मेत‚त् ।
	{\color{gray}{\rmlatinfont\textsuperscript{§~\theparCount}}}
	\pend% ending standard par
      ‚{\tiny $_{lb}$}‚

	  
	  \pstart \leavevmode% starting standard par
	\textbf{संस्कृत‚मि}त्यादिना व्याच‚ष्टे । \textbf{अञ्ज‚नादिभिः संस्कृत‚मिन्द्रियं कंचिद‚तिश‚य}‚{\tiny $_{lb}$}‚मात्म‚भूत\textbf{मासाद‚य}ति । \textbf{प्र‚तिप‚त्तौ} प्र‚तिप‚त्तिनिमित्तं । विशिष्ट‚ज्ञानोत्पाद‚नायेति‚{\tiny $_{lb}$}‚ याव‚त् । निमित्तात् क‚र्म्म‚संयोग इत्य‚नेनात्र स‚प्त‚मी । कुत एत‚दिति चेदाह ।‚{\tiny $_{६}$}‚‚{\tiny $_{lb}$}‚ \textbf{स्प‚ष्टे}त्यादि । \textbf{प्र‚तिप‚त्ते}रिति विभ‚क्तिविप‚रिणामेन स‚म्ब‚न्धः । त‚था हि तिमिराद्युप‚{\tiny $_{lb}$}‚\leavevmode\ledsidenote{\textenglish{107b/PSVTa}} ह‚त‚मिन्द्रिय‚म‚स्प‚ष्टं विज्ञान‚ञ्ज‚न‚य‚ति । त‚देवाञ्ज‚नादिसंस्कृतं स्प‚ष्ट‚तोऽव‚सीय‚ते [।]‚{\tiny $_{lb}$}‚ संस्कृत‚मिन्द्रियं प्र‚तिप‚त्त्य‚र्थ‚म‚तिश‚य‚मासाद‚य‚तीति ।
	{\color{gray}{\rmlatinfont\textsuperscript{§~\theparCount}}}
	\pend% ending standard par
      ‚{\tiny $_{lb}$}‚

	  
	  \pstart \leavevmode% starting standard par
	अथाञ्ज‚नादेः स‚काशान्न प्र‚तिप‚त्तिभेद‚स्त‚दा । \textbf{त‚द‚कारिण}श्च प्र‚तिप‚त्तिभेदा‚{\tiny $_{lb}$}‚कारिण‚श्चाञ्ज‚ना\textbf{देर‚त‚त्सं‚{\tiny $_{७}$}‚स्कार\textbf{क}त्वादि}न्द्रियासंस्कार‚क‚त्वात् । य‚थाञ्ज‚नादेः‚{\tiny $_{lb}$}‚ स‚काशा\textbf{दिन्द्रिय‚स्य संस्कारो नैवं व्य‚क्तेः} स‚काशात् [।] किङ्कार‚णं [।] \textbf{त‚द्भावाभाव‚{\tiny $_{lb}$}‚काल}योरित्यादि । त‚था हि व्य‚ञ्जिकाया गोव्य‚क्तेर‚भाव‚काले यादृशं च‚क्षुर्विज्ञानं‚{\tiny $_{lb}$}‚ वृक्षादावुत्प‚न्न‚न्त‚स्या गोव्य‚क्तेः अभावेपि वृक्षादौ तादृश‚मेव । य‚दि त‚द‚{\tiny $_{lb}$}‚भाव‚काले पूर्व‚म‚स्प‚ष्टं विज्ञानं वृक्षादिषूत्प‚न्नं प‚श्चाद् गोव्य‚क्तिकृत इन्द्रिय‚संस्कारो‚{\tiny $_{lb}$}‚ ग‚{\tiny $_{१}$}‚म्येत ।
	{\color{gray}{\rmlatinfont\textsuperscript{§~\theparCount}}}
	\pend% ending standard par
      ‚{\tiny $_{lb}$}‚

	  
	  \pstart \leavevmode% starting standard par
	विष‚य‚संस्कारेपि स‚ति विष‚यान्त‚रे नैव प्र‚तिप‚त्तिभेदोऽस्त्य‚तः सोपि न युक्त‚{\tiny $_{lb}$}‚ इति चेदाह । \textbf{विष‚ये}त्यादि । विष‚य‚स्य ग‚न्धादेर्यः केन‚चित् संस्कार‚स्स \textbf{इन्द्रियाविशे‚{\tiny $_{lb}$}‚षेपि} । य‚दि नामान्य‚त्र विष‚यान्त‚रे इन्द्रिय‚स्य प्र‚तिप‚त्तिं प्र‚ति विशेषो नास्ति । \textbf{त‚थापि‚{\tiny $_{lb}$}‚ त‚द्विवेषाधा}न‚न्त‚स्य संस्कृत‚विष‚य‚ग्राह‚क‚स्य ज्ञान‚स्य \textbf{विशेषाधानादुप‚कारी स्यात्} ।‚{\tiny $_{lb}$}‚ ‚{\tiny $_{lb}$}‚ \leavevmode\ledsidenote{\textenglish{291/s}}नेन्द्रिय‚संस्का‚{\tiny $_{२}$}‚र उप‚कारी स्यादिति स‚म्ब‚न्धः ।
	{\color{gray}{\rmlatinfont\textsuperscript{§~\theparCount}}}
	\pend% ending standard par
      ‚{\tiny $_{lb}$}‚

	  
	  \pstart \leavevmode% starting standard par
	एत‚दुक्त‚म्भ‚व‚ति । विष‚या हि विनिय‚तास्ते स्वाकार‚स्यैव विज्ञान‚स्य साध‚नं‚{\tiny $_{lb}$}‚ नाकारान्त‚र‚युक्त‚स्य [।] त‚तो विष‚य‚संस्कारः प्र‚तिनिय‚त‚त्वात् स्व‚विष‚यामेव विशिष्टां‚{\tiny $_{lb}$}‚ प्र‚तिप‚त्तिञ्ज‚न‚य‚न्न विरुध्य‚ते । इन्द्रियं तु स्व‚ग्राह्ये विष‚य‚भेदे तुल्यं साध‚न‚म‚त‚स्त‚{\tiny $_{lb}$}‚त्संस्कारः स‚र्व‚स्मिंस्त‚द्ग्राह्ये प्र‚तिप‚त्तेर्भेद‚कः प्राप्नोतीति । नेन्द्रिय‚स्य व्य‚क्तिभेद‚{\tiny $_{३}$}‚‚{\tiny $_{lb}$}‚स्त‚दिन्द्रिय‚ग्राह्ये स‚र्व‚स्मिन् दृश्ये विष‚ये स्प‚ष्टाकार‚ज्ञान‚ज‚न‚नाय संस्कार‚माध‚त्ते ।‚{\tiny $_{lb}$}‚ किन्त‚र्हि व्य‚क्त्युत्प‚त्तेः \textbf{प्राग‚दृश्ये} सामान्ये इन्द्रिय‚स्य \textbf{द‚र्श‚न‚श‚क्त्याधाना}त् कार‚णाद्‚{\tiny $_{lb}$}‚ व्य‚क्तिभेद \textbf{उप‚कार‚कं इति चेत् । अतीन्द्रिय‚द‚र्श‚नादे}व च स्प‚ष्ट‚मिन्द्रिय‚स्य संस्कारो‚{\tiny $_{lb}$}‚ ग‚म्य‚ते । व्य‚क्तिस‚न्निधानात् पूर्व‚म‚स‚म‚र्थ‚म्प‚श्चात्त‚त्स‚न्निधाने स‚म‚र्थ‚मिति स व्य‚क्ति‚{\tiny $_{lb}$}‚भेदोतीन्द्रियं सामान्या‚{\tiny $_{४}$}‚ख्य‚म‚र्थ‚न्द‚र्श‚य‚न् \textbf{क‚थं प्र‚तिप‚त्तेः} स‚र्व‚त्र \textbf{न भेद‚को} भेद‚क एवेत्य‚र्थः ।
	{\color{gray}{\rmlatinfont\textsuperscript{§~\theparCount}}}
	\pend% ending standard par
      ‚{\tiny $_{lb}$}‚

	  
	  \pstart \leavevmode% starting standard par
	एत‚दुक्त‚म्भ‚व‚ति । दृश्येपि ताव‚द‚स्प‚ष्टे स्प‚ष्टाकार‚द‚र्श‚न‚श‚क्त्याधानाद् अञ्ज‚{\tiny $_{lb}$}‚नादिकृतः इन्द्रिय‚संस्कारः स‚र्व‚त्र त‚दिन्द्रिय‚ग्राह्ये विष‚ये प्र‚तिप‚त्तेर्भेद‚को दृष्टः [।]‚{\tiny $_{lb}$}‚ किं पुन‚र्योतीन्द्रिय‚स्यार्थ‚स्य द‚र्श‚क‚स्त‚था चान्य‚स्याप्य‚तीन्द्रिय‚स्य प‚र‚माण्वादेर्द‚र्श‚कः‚{\tiny $_{lb}$}‚ स्यादिति भावः । सामान्य‚स्यैव‚{\tiny $_{५}$}‚ द‚र्श‚नायेन्द्रिय‚स्य संस्कार‚माध‚त्ते व्य‚क्तिभेदः‚{\tiny $_{lb}$}‚ त‚तो नास्त्य‚तिप्र‚स‚ङ्ग इति चेदाह । \textbf{एके}त्यादि । एक‚स्मिन् सामान्ये द्र‚ष्ट‚व्ये इन्द्रिय‚{\tiny $_{lb}$}‚संस्कार‚स्य \textbf{प्र‚तिनि}य‚म‚स्त\textbf{स्मिन्न}भ्युप‚ग‚म्य‚माने त‚स्मिन्नेव व्य‚क्तिभेदे शाव‚लेयादिके‚{\tiny $_{lb}$}‚स‚म‚वेतं य\textbf{त्सामान्यान्त‚रं} स‚त्ताद्र‚व्य‚त्वादि । \textbf{त‚स्य द‚र्श‚क} इन्द्रिय‚संस्कारो‚{\tiny $_{lb}$}‚ न स्यात् । इष्य‚ते च । आत्म‚स‚म‚वेतानामेव स‚र्व‚सामान्यानान्द‚र्श‚नाये‚{\tiny $_{६}$}‚न्द्रिय‚{\tiny $_{lb}$}‚संस्कारो नैक‚स्यैवेति चेदाह । \textbf{व्य‚क्त्या चे}त्यादि । \textbf{त‚द्द‚र्श‚न} इति तेषां व्य‚क्तिस‚म‚वे‚{\tiny $_{lb}$}‚तानां सामान्यानान्द‚र्श‚ने । \textbf{त‚द्व्य‚ङ्ग्येषु} त‚या व्य‚क्त्या व्य‚ङ्ग्येषु । दृष्ट‚श्च दूराद्‚{\tiny $_{lb}$}‚ द्र‚व्य‚मात्र‚द‚र्श‚ने द्र‚व्य‚त्व‚म‚र्थ‚त्व‚योर्निश्च‚येपि स‚ति गोत्वादाव‚निश्च‚यः । य‚स्मिन्न‚निश्च‚{\tiny $_{lb}$}‚य‚स्त‚स्य द‚र्श‚नाय नाहितः संस्कार‚भेद इति चेदाह । एक\textbf{निश्च‚यो वेति न स्यादि}ति‚{\tiny $_{lb}$}‚ स‚म्ब‚न्धः । एक‚श‚ब्दो‚{\tiny $_{७}$}‚न्यार्थोऽनिश्चिताद‚न्य‚स्यापि निश्चिताभिम‚त‚स्य द्र‚व्य‚त्वादे- \leavevmode\ledsidenote{\textenglish{108a/PSVTa}}‚{\tiny $_{lb}$}‚  ‚{\tiny $_{lb}$}‚ ‚{\tiny $_{lb}$}‚ \leavevmode\ledsidenote{\textenglish{292/s}}र्निश्च‚यो न स्यात् । किङ्कार‚णं [।] \textbf{त‚स्या} व्य‚क्तेर\textbf{विभाग‚म‚यास्तेषु} स्वात्म‚स‚म\textbf{वेतेषु}‚{\tiny $_{lb}$}‚ सामान्येषु \textbf{विशेषाभावात्} । न हि सा व्य‚क्तिः क्व‚चित् प्र‚त्यास‚न्ना क्व‚चिन्न । त‚तः‚{\tiny $_{lb}$}‚ स‚र्व‚स्य वा निश्च‚यः सामान्य‚स्य न वा क‚स्य‚चिद\textbf{पीत्ये}व‚न्ताव‚द् \textbf{व्य‚क्तेरिन्द्रिय‚सं‚{\tiny $_{lb}$}‚स्कारो न घ‚ट‚त इत्याख्या}त‚म् [।]
	{\color{gray}{\rmlatinfont\textsuperscript{§~\theparCount}}}
	\pend% ending standard par
      ‚{\tiny $_{lb}$}‚

	  
	  \pstart \leavevmode% starting standard par
	अधुनाभ्युप‚ग‚म्याप्युच्य‚ते । \textbf{व्य‚क्ते}रित्यादि । \textbf{व्य‚क्ते}स्स‚काशात् प‚क्ष‚{\tiny $_{१}$}‚द्व‚येपि‚{\tiny $_{lb}$}‚ सामान्य‚स्य \textbf{विज्ञान‚ज‚न‚न‚स्व‚भाव इति} कृत्वा [।] त‚स्माद् विज्ञान‚ज‚न‚नात् \textbf{स्व‚भा‚{\tiny $_{lb}$}‚वात् प्र‚च्युतेः} कार‚णान्न हि स‚म‚र्थ‚स्य स‚ह‚कार्य‚पेक्षा युक्ता । \textbf{संस्कृत‚मिन्द्रियं स‚ह}‚{\tiny $_{lb}$}‚कारि य‚स्य सामान्य‚स्य त‚त्त‚थोक्त‚न्त‚द्भा\textbf{व‚स्त‚स्मात्} । नित्य‚त्वाद\textbf{नाधेयातिश‚य‚स्य}‚{\tiny $_{lb}$}‚ सामान्य‚स्य \textbf{कोयं स‚ह‚कारार्थः} [।] नैव क‚श्चित् ।
	{\color{gray}{\rmlatinfont\textsuperscript{§~\theparCount}}}
	\pend% ending standard par
      ‚{\tiny $_{lb}$}‚

	  
	  \pstart \leavevmode% starting standard par
	\textbf{अनित्त्या ही}त्यादिना व्य‚तिरेक‚माह । \textbf{स‚ह‚कारिणः} स‚काशाद्विशि\textbf{ष्ट‚स्यात्म‚नो‚{\tiny $_{lb}$}‚ ला‚{\tiny $_{२}$}‚भात् । त‚मि}ति स‚ह‚कारिणं [।] क‚स्माद् [।] \textbf{यो ह्येषां} क्ष‚णिकानां \textbf{ज‚न‚क‚{\tiny $_{lb}$}‚ आत्मा} स‚ह‚कारिस‚न्निधेः प्राङ्नासीत् । \textbf{त‚दैव} स‚ह‚कारिस‚न्निधिकाले । त‚तः स‚ह‚कारिणः‚{\tiny $_{lb}$}‚ स‚काशाद् \textbf{भ‚व‚तीति} कृत्वा । \textbf{एषा}मिति क्ष‚णिकानां ।
	{\color{gray}{\rmlatinfont\textsuperscript{§~\theparCount}}}
	\pend% ending standard par
      ‚{\tiny $_{lb}$}‚

	  
	  \pstart \leavevmode% starting standard par
	न‚नु क्ष‚णिकानाम‚पि क‚थं स‚ह‚कारिणो विशिष्टात्म‚लाभापेक्षा । स‚ह‚भाविना‚{\tiny $_{lb}$}‚म्प‚र‚स्प‚र‚म‚नुप‚कार्योप‚कार‚क‚त्वात् । य‚श्च क्ष‚णो जाय‚ते न त‚स्य स‚द‚स‚त्त्व‚काल‚यो‚{\tiny $_{lb}$}‚स्स‚{\tiny $_{३}$}‚ह‚कार्य‚पेक्षेति [।]
	{\color{gray}{\rmlatinfont\textsuperscript{§~\theparCount}}}
	\pend% ending standard par
      ‚{\tiny $_{lb}$}‚

	  
	  \pstart \leavevmode% starting standard par
	अयुक्त‚मुक्त‚मुप‚कारी ह्य‚पेक्ष‚त इति नैष दोषः । स‚त्तापेक्ष‚यैत‚दुच्य‚ते । अत‚{\tiny $_{lb}$}‚ एवाह । ज‚न्य‚तैवैषां प‚र‚स्प‚र‚तोपेक्षेति । अनासाद्य \textbf{प‚र‚मि}ति स‚ह‚कारिणं । \textbf{त‚त्स्व‚भावं}‚{\tiny $_{lb}$}‚ स‚म‚र्थ‚स्व‚भावं । न हि त‚स्य सामान्य‚स्य केव‚ल‚स्य स‚ह‚कारिविक‚ल‚स्य प्राग् यो न‚{\tiny $_{lb}$}‚ विज्ञान‚ज‚न‚न‚स्व‚भावः स पुनः क‚थंचिद् भावी । \textbf{न हीति} स‚म्ब‚न्धः । नित्य‚त्वादिति‚{\tiny $_{lb}$}‚ भावः ।
	{\color{gray}{\rmlatinfont\textsuperscript{§~\theparCount}}}
	\pend% ending standard par
      ‚{\tiny $_{lb}$}‚‚{\tiny $_{lb}$}‚‚{\tiny $_{lb}$}‚\textsuperscript{\textenglish{293/s}}

	  
	  \pstart \leavevmode% starting standard par
	किञ्च [।] \textbf{व्य‚क्तेः सामान्य‚सं‚{\tiny $_{४}$}‚स्कारे स‚ति त‚ज्ज‚न्यं सामान्यं स्यादि}ति‚{\tiny $_{lb}$}‚ प‚रेणेन्द्रिय‚संस्कारोङ्गीकृतः व्य‚क्तेः स‚काशात् । त‚थापि व्य‚क्तिज‚न्य‚त्वं प्र‚स‚ज्य‚त‚{\tiny $_{lb}$}‚ इत्याह । \textbf{व्य‚क्ति}रित्यादि । \textbf{त‚त्स‚ह‚कारि} । व्य‚क्तिसंस्कृतेन्द्रिय‚स‚ह‚कारि सामान्य‚म्वि‚{\tiny $_{lb}$}‚ज्ञान‚हेतुरित्य‚भ्युप‚ग‚म्य‚माने । \textbf{व्य‚क्तिकार्य‚स्येन्द्रि}य‚स्य कार्य‚त्वात् \textbf{सामान्यं व्य‚क्तेः‚{\tiny $_{lb}$}‚ पार‚म्प‚र्येण कार्य‚मुक्तं स्यात्} । \href{http://sarit.indology.info/?cref=pv.3.146-3.147}{। १४९-५० ॥}
	{\color{gray}{\rmlatinfont\textsuperscript{§~\theparCount}}}
	\pend% ending standard par
      ‚{\tiny $_{lb}$}‚

	  
	  \pstart \leavevmode% starting standard par
	भ‚व‚तु नाम सामान्य‚स्य व्य‚ञ्जिका व्य‚क्तिः ताव‚द‚स्या जातिम‚त्त्वं न युक्त‚म‚{\tiny $_{५}$}‚‚{\tiny $_{lb}$}‚तिप्र‚स‚ङ्गादित्याह । \textbf{अपि चे}त्यादि । \textbf{जातीनां} सामान्यानां \textbf{व्य‚ञ्ज‚क‚स्य} व्य‚क्ति‚{\tiny $_{lb}$}‚भेद‚स्य \textbf{जातिम‚त्ता य‚दीष्य}ते । त‚दा गोत्वादेः प्र‚काश‚कः \textbf{व्य\edtext{}{\lemma{व्य}\Bfootnote{? प्र}}दीपा}दि\href{http://sarit.indology.info/?cref=}{ः}‚{\tiny $_{lb}$}‚ व्य‚ञ्ज‚क‚त्वात्तेन गोत्वादिना त‚द्वान् गोत्वादिमान् \textbf{प्राप्तः} । शाव‚लेयादिव‚त् ।‚{\tiny $_{lb}$}‚ गोत्वाधारः प्राप्त इत्य‚र्थः ।
	{\color{gray}{\rmlatinfont\textsuperscript{§~\theparCount}}}
	\pend% ending standard par
      ‚{\tiny $_{lb}$}‚

	  
	  \pstart \leavevmode% starting standard par
	\textbf{यो ही}त्यादिना व्याच‚ष्टे । \textbf{गोत्वादिषु} व्य‚ङ्ग्येषु \textbf{विज्ञान‚हेतुत्वं प्र‚दीपादेर‚प्य}स्ति ।‚{\tiny $_{lb}$}‚ क‚थ‚मिति चेदाह । \textbf{तेज} इत्यादि‚{\tiny $_{६}$}‚ । त‚स्माद‚स्त्या\textbf{लोक‚स्य विज्ञान‚म्प्र‚ति हेतुत्वं ।‚{\tiny $_{lb}$}‚ त‚त} इति ज्ञान‚हेतुत्वात् । \textbf{प्र‚दीपाद‚य} इति [।] आदिश‚ब्दादिन्द्रिय‚संस्कारादिप‚रि‚{\tiny $_{lb}$}‚ग्र‚हः । तेषाम‚पि ज्ञान‚हेतुत्वात् । व्य‚क्तेः स‚काशाद् विशिष्ट‚स्यैवाभिव्य‚क्तिः सामान्य‚स्य‚{\tiny $_{lb}$}‚ भ‚व‚ति न त‚था प्र‚दीपादेरिति चेदाह । न हीत्यादि । \textbf{व्य‚क्तेर‚पि} स‚काशात् \textbf{सामा‚{\tiny $_{lb}$}‚न्य‚स्याभिव्य‚क्तिर्ज्ञान‚हेतुतां मुक्त्वा न ह्य‚न्या काचित्} । य‚दि हि सामा‚{\tiny $_{७}$}‚न्य‚स्या- \leavevmode\ledsidenote{\textenglish{108b/PSVTa}}‚{\tiny $_{lb}$}‚ तिश‚याधानं व्य‚क्त्या क्रिय‚ते न प्र‚दीपादिना । त‚दा भ‚वेद्विशेष‚स्त‚च्च नास्ति । \textbf{स्व‚भा‚{\tiny $_{lb}$}‚वातिश‚य‚स्याधातुम‚श‚क्य‚त्वात्} । नित्य‚त्वात् सामान्य‚स्येति भावः । \textbf{स‚म‚वाय}‚{\tiny $_{lb}$}‚ इत्यादि । व्य‚क्तौ च स‚म‚वेतं गोत्वं न प्र‚दीपादौ । त‚स्येति सामान्य‚स्य । अज‚न्य‚{\tiny $_{lb}$}‚‚{\tiny $_{lb}$}‚ \leavevmode\ledsidenote{\textenglish{294/s}}ज‚न‚क‚योः कोयं स‚म‚वाय \textbf{इत्युक्त‚त्वात्} ।
	{\color{gray}{\rmlatinfont\textsuperscript{§~\theparCount}}}
	\pend% ending standard par
      ‚{\tiny $_{lb}$}‚

	  
	  \pstart \leavevmode% starting standard par
	भ‚व‚तु नाम स‚म‚वाय‚स्त‚थाप्य‚स्य सामान्य‚स्य \textbf{स‚म‚वाय‚मात्रं व्य‚क्त्या स‚ह जातं‚{\tiny $_{lb}$}‚ नान्यः क‚श्चिद्विशेषो}‚{\tiny $_{१}$}‚ विज्ञान‚ज‚न‚न‚ल‚क्ष‚णः । \textbf{पूर्व‚व}दिति व्य‚क्तिस‚म‚वायात् प्राग्व‚त्‚{\tiny $_{lb}$}‚ \textbf{प‚श्चाद‚पि} व्य‚क्तिस‚म‚वायेपि य‚द्य‚पि सामान्य‚स्य \textbf{न क‚श्चिद् विशेष}स्त‚थापि‚{\tiny $_{lb}$}‚ स‚म‚वाय‚ब‚लादेव स्व‚विष‚य‚ज्ञान‚ज‚न‚न‚मिति चेदाह । \textbf{स‚म‚वायादेवे}त्यादि । \textbf{ज्ञान‚हेतुत्वेन}‚{\tiny $_{lb}$}‚ सामान्य‚स्याभ्युप‚ग‚म्य‚माने । \textbf{स्वाश्र‚यो} यो य‚त्र स‚म‚वेत‚स्त‚त्स‚म‚वायिनं । सामान्य‚{\tiny $_{lb}$}‚स्याभ्युप‚ग‚म्य‚माने । \textbf{स्वाश्र}यो यो य‚त्र स‚म‚वेत‚स्त‚त्स‚म‚वायिनं । सामान्या\textbf{द‚न्येषा}म‚पि‚{\tiny $_{lb}$}‚ प‚र‚माणु\textbf{स‚म‚वेतानां} रूपादीना\textbf{म‚पि दृश्य‚ता}प‚त्तिः‚{\tiny $_{२}$}‚ \textbf{स्यात्} । स‚म‚वाय‚स्यैक‚त्वेन‚{\tiny $_{lb}$}‚ स‚र्व‚त्राविशेषात् । य‚था हि सामान्य‚म‚तीन्द्रिय‚म‚पि केव‚ल‚स्स‚म‚वायो द‚र्श‚य‚त्येवं‚{\tiny $_{lb}$}‚ प‚र‚माणुग‚तान‚पि रूपादीन् किन्न द‚र्श‚येत् । य‚त‚श्च स‚म‚वाय‚प‚क्षेऽय‚न्दोष\textbf{स्त‚स्मात्‚{\tiny $_{lb}$}‚ ज्ञान‚हेतुतैव} सामान्ये व्य‚क्ते\textbf{र्व्य‚ञ्ज‚क‚त्वं} । त‚च्चेत्थं भूतं व्य‚ञ्ज‚क‚त्व‚न्तुल्य\textbf{म्प्र‚दीपां‚{\tiny $_{lb}$}‚दाव}पीति \textbf{स एव प्र‚स‚ङः} प्राप्तो गोत्वादिना त‚द्वानित्यादिकः । \href{http://sarit.indology.info/?cref=pv.3.147-3.148}{। १५०-५१ ॥}
	{\color{gray}{\rmlatinfont\textsuperscript{§~\theparCount}}}
	\pend% ending standard par
      ‚{\tiny $_{lb}$}‚

	  
	  \pstart \leavevmode% starting standard par
	\textbf{त‚दि}ति त‚स्मा\textbf{न्नाधे}य‚ता \textbf{सामान्य}स्य स्वा‚{\tiny $_{३}$}‚श्र‚ये वृत्तिर्नापि स्वा\textbf{श्र‚येण व्य‚क्ति} र‚भि‚{\tiny $_{lb}$}‚व्य‚क्तिः स्वाश्र‚ये सामान्य‚स्य वृत्तिः । अत‚श्चावृत्तेः कार‚णात् सामान्य\textbf{न्नाने}क‚त्र‚{\tiny $_{lb}$}‚ व्य‚क्तिभेदे एकाकार\textbf{ज्ञान‚हेतुः} । न हि यो य‚त्र न व‚र्त्त‚ते स त‚त्रात्म‚वृत्तिद्वारेण‚{\tiny $_{lb}$}‚ ज्ञान‚हेतुर्युक्तः ।
	{\color{gray}{\rmlatinfont\textsuperscript{§~\theparCount}}}
	\pend% ending standard par
      ‚{\tiny $_{lb}$}‚

	  
	  \pstart \leavevmode% starting standard par
	\textbf{अत एवाने}क‚त्र एक‚स्य सामान्य‚स्यावृत्तेः कार‚णात् । \textbf{व्य‚क्तेः} स‚काशाद‚न्या‚{\tiny $_{lb}$}‚ य‚दि \textbf{जातिर‚थ‚वान}न्या । \textbf{येषां} वादिनां । \textbf{व्य‚क्तिष्व‚पूर्वासु} संप्र‚त्युत्प‚न्नासु प‚श्चाद्वा‚{\tiny $_{lb}$}‚ दृ‚{\tiny $_{४}$}‚श्य‚मानासु । \textbf{तु} श‚ब्दो विद्य‚ते श‚ब्दात्प‚रेण द्र‚ष्ट‚व्योव\textbf{धार‚णार्थः} । अत एवाह । \textbf{विद्य‚त}‚{\tiny $_{lb}$}‚ एवेत्यादि । \textbf{व‚स्तुस‚तो} येषां जातिरिति याव‚त् । \textbf{स्व‚साम}र्थ्ये स‚ति स्व‚प्र‚तिप‚त्तिद्वारेण ।‚{\tiny $_{lb}$}‚ \textbf{अन्य‚त्रे}ति व्य‚क्तिभेदे बुद्धिञ्ज‚न‚य‚न् । किं विशिष्टां \textbf{स्व‚रूपानुकारिणी}मेक‚रूपानुग‚तां‚{\tiny $_{lb}$}‚ ‚{\tiny $_{lb}$}‚ \leavevmode\ledsidenote{\textenglish{295/s}}य‚त्र बुद्धिञ्ज‚न‚य‚ति तेन स‚म्ब‚न्ध‚म‚पेक्ष‚ते । स‚म्ब‚न्ध‚म‚न्त‚रेण स्वाकार‚बुद्धिज‚न‚नेऽति‚{\tiny $_{lb}$}‚\textbf{प्र‚स‚ङ्गात्} । \href{http://sarit.indology.info/?cref=pv.3.148-3.149}{। १५१-५२ ॥}
	{\color{gray}{\rmlatinfont\textsuperscript{§~\theparCount}}}
	\pend% ending standard par
      ‚{\tiny $_{lb}$}‚

	  
	  \pstart \leavevmode% starting standard par
	\textbf{स चे}ति स‚{\tiny $_{५}$}‚म्ब‚न्धो व्य‚क्त्य‚न्त‚रेण \textbf{सामान्य}स्य \textbf{स‚तो} विद्य‚मान‚स्य । \textbf{त‚त्त्व}प‚क्षे‚{\tiny $_{lb}$}‚ऽन्य‚त्त्व\textbf{प‚क्षे} च न \textbf{स‚म्भ}व‚ति [।] क‚स्माद् [।] \textbf{एक‚त्र} व्य‚क्तिभेद‚स्यार्थान्त‚र‚भूत‚स्यान‚{\tiny $_{lb}$}‚र्थान्त‚र‚स्य \textbf{चान्य}त्र व्य‚क्त्य‚न्त‚रे \textbf{द‚र्श‚नास‚म्भ‚वात् । सा ही}त्यादिनैत‚देव साध‚य‚ति ।‚{\tiny $_{lb}$}‚ \textbf{सा हि बुद्धिर्भूत‚ग्राहिणी} व‚स्तुभूत‚सामान्य‚ग्राहि\textbf{ण्येक‚भाविनी} । एक‚त्र व्य‚क्तिभेदे‚{\tiny $_{lb}$}‚ उत्प‚न्ना \textbf{व्य‚क्त्य‚न्त‚र‚मेव} सा \textbf{स्क‚न्देत्} ग‚च्छेद् [।] व्य‚क्त्य‚न्त‚र‚म‚पि सामान्याका‚{\tiny $_{६}$}‚रेण‚{\tiny $_{lb}$}‚ गृह्णीयादिति याव‚त् । य‚दि \textbf{त‚त्रै}क‚स्मिन् व्य‚क्तिभेदे \textbf{दृष्टं किञ्चिद्} व‚स्तुभूतं सामा‚{\tiny $_{lb}$}‚न्य\textbf{म‚न्य}त्रेति य‚त्र व्य‚क्तिभेदे त‚या स्क‚न्दित‚व्यं । \textbf{त‚च्चां}न्य‚त्र द‚र्श‚नं । \textbf{स‚त} इति‚{\tiny $_{lb}$}‚ व‚स्तुभूत‚स्य सामान्य‚स्य न स‚म्भ‚व‚तीति स‚म्ब‚न्धः । किङ्कार‚ण‚म् [।] आश्र‚याद‚{\tiny $_{lb}$}‚\textbf{न‚न्य‚त्वे}भ्युप‚ग‚म्य‚माने\textbf{न्व‚याभावा}त् । न ह्येक‚स्माद‚व्य‚तिरिक्त‚स्त‚दात्म‚भूतोन्य‚{\tiny $_{lb}$}‚द‚न्वेति । आश्र‚या\textbf{द‚न्य‚त्त्वे}पि साम‚न्य‚स्य व्य‚क्ताव‚{\tiny $_{७}$}‚\textbf{न‚पाश्र‚या}त् । अन‚न्त‚रोक्तेना‚{\tiny $_{lb}$}‚धाराधेयादिभाव‚निषेधेनाश्र‚य‚भाव‚स्य निषिद्ध‚त्वात् । य‚दि व्य‚तिरिक्त‚स्य सामा‚{\tiny $_{lb}$}‚न्य‚स्यास‚म्ब‚न्धान्न व्य‚क्त्य‚न्त‚रे स्वाकार‚ज्ञान‚ज‚न‚न‚म् [।] एव‚न्त‚र्ह्याद्यायाम‚पि‚{\tiny $_{lb}$}‚ व्य‚क्तौ त‚त्तुल्य‚मिति किमुच्य‚ते व्य‚क्तिष्व‚पूर्वास्विति व्य‚क्तिष्वित्येव व‚क्त‚व्यं ।‚{\tiny $_{lb}$}‚ त‚थैक‚त्र दृष्ट‚स्यान्य‚त्र द‚र्श‚नास‚म्भ‚वादित्य‚पि व‚क्त‚व्यं । एक‚त्रापि व्य‚क्तिभेदे‚{\tiny $_{lb}$}‚ स‚म्ब‚न्ध‚म‚न्त‚रेण द‚र्श‚नास‚म्भ‚वात् ।
	{\color{gray}{\rmlatinfont\textsuperscript{§~\theparCount}}}
	\pend% ending standard par
      ‚{\tiny $_{lb}$}‚

	  
	  \pstart \leavevmode% starting standard par
	स‚त्य‚मेत‚त् । अभ्युप‚ग‚म्यैत‚दुक्त‚मित्य‚दोषः ।
	{\color{gray}{\rmlatinfont\textsuperscript{§~\theparCount}}}
	\pend% ending standard par
      ‚{\tiny $_{lb}$}‚

	  
	  \pstart \leavevmode% starting standard par
	योपि म‚न्य‚ते [।] त‚त्त्वान्य‚त्त्व‚प‚क्षे सामान्य‚स्यान्य‚द‚र्श‚नं न स‚म्भ‚व‚त्य‚स्माक‚न्तु‚{\tiny $_{lb}$}‚ भिन्नाभिन्न‚मेव सामान्यं । त‚था हि [।]
	{\color{gray}{\rmlatinfont\textsuperscript{§~\theparCount}}}
	\pend% ending standard par
      ‚{\tiny $_{lb}$}‚
	  \bigskip
	  \begingroup
	
	    
	    \stanza[\smallbreak]
	  {\normalfontlatin\large ``\qquad}निर्विशेषं न सामान्यं भ‚वेच्छ‚श‚विषाण‚व‚त् ।&‚{\tiny $_{lb}$}‚केन‚चिच्चात्म‚नैक‚त्वं नानात्वं चास्य केन‚चित् ।&‚{\tiny $_{lb}$}‚य‚दा च श‚ब‚ल‚म्व‚स्तु युग‚प‚त् प्र‚तिभास‚ते ।&‚{\tiny $_{lb}$}‚‚{\tiny $_{lb}$}‚\leavevmode\ledsidenote{\textenglish{296/s}}त‚दान्यान‚न्य‚भेदादि स‚र्व‚मेव प्र‚लीय‚त इति ।\edtext{\textsuperscript{*}}{\edlabel{pvsvt_296-1}\label{pvsvt_296-1}\lemma{*}\Bfootnote{\href{http://sarit.indology.info/?cref=\%C5\%9Bv-\%C4\%81k\%E1\%B9\%9Bti.10}{ Ślokavārtika, Ākṛti 10. }}}{\normalfontlatin\large\qquad{}"}\&[\smallbreak]
	  
	  
	  
	  \endgroup
	‚{\tiny $_{lb}$}‚

	  
	  \pstart \leavevmode% starting standard par
	त‚न्निषेधार्थ‚माह । \textbf{स्व‚भा‚{\tiny $_{२}$}‚वो ही}त्यादि । स्व‚भावात् \textbf{त‚त्त्व‚मेवान्य‚त्त्व‚मेव वा न‚{\tiny $_{lb}$}‚ लंघ‚य‚ती}ति स‚म्ब‚न्धः ।
	{\color{gray}{\rmlatinfont\textsuperscript{§~\theparCount}}}
	\pend% ending standard par
      ‚{\tiny $_{lb}$}‚

	  
	  \pstart \leavevmode% starting standard par
	न‚नु देश‚काल‚स्व‚भावाभेदेपि सामान्य‚विशेष‚योर‚नुग‚त‚व्यावृत्तिरूपाभ्यां‚{\tiny $_{lb}$}‚ भेदोपीष्य‚त इत्य‚त आह । \textbf{रूप‚स्यानुग‚त‚स्य} स्व‚भाव‚स्या\textbf{त‚द्भूत‚स्य} व्यावृत्त‚रूप‚{\tiny $_{lb}$}‚स्व‚भाव‚स्या\textbf{न्य‚त्त्वाव्य‚तिक्र‚मात्} । व्यावृत्तेभ्यो विशेष‚रूपेभ्यो भिन्न‚स्यानुग‚त‚{\tiny $_{lb}$}‚रूप‚स्य सामान्य‚स्यान्य‚त्त्व‚मेव स्यादित्य‚र्थः । अस्त्य‚त‚द्रूप‚{\tiny $_{३}$}‚त्व‚म‚न्य‚त्त्व‚मेव [।] क‚थ‚{\tiny $_{lb}$}‚मित्य‚त आह । \textbf{इद‚मेवे}त्यादि । \textbf{य‚न्न त}दित्य‚त‚द्रूप‚मित्य‚र्थः । एताव‚देवान्य‚त्त्व‚ल‚क्ष‚ण‚मि‚{\tiny $_{lb}$}‚त्य‚र्थः । \textbf{आकारान्त‚र‚व‚त्} । ष‚ष्ठ्य‚र्थे व‚तिः । त‚था हि सुखाद् दुःख‚स्याप्य‚न्य‚त्त्व‚म‚सुख‚{\tiny $_{lb}$}‚ रूपं दुःख‚मिति कृत्वा । इय‚ता चात‚द्रूप‚स्यान्य‚त्त्व‚ल‚क्ष‚णेन व्याप्तिरुक्ता । अस्य‚{\tiny $_{lb}$}‚ चान्य‚त्त्व‚ल‚क्ष‚ण‚स्या\textbf{विशेषाद}भिम‚तेपि सामान्ये ।
	{\color{gray}{\rmlatinfont\textsuperscript{§~\theparCount}}}
	\pend% ending standard par
      ‚{\tiny $_{lb}$}‚

	  
	  \pstart \leavevmode% starting standard par
	एतेन प‚क्ष‚ध‚र्म उक्तः । प्र‚यो‚{\tiny $_{४}$}‚ग‚स्तु । य‚द्व‚स्तुत्वे स‚त्य‚त‚द्रूप‚न्त‚स्य त‚तोन्य‚त्त्व‚मेव‚{\tiny $_{lb}$}‚ त‚द्य‚था सुखाद् दुःख‚स्य । व‚स्तुत्वे स‚त्य‚व्य‚क्तिरूपं चेष्य‚ते सामान्य‚मित्य‚त‚द्रूप‚त्वेनान्य‚{\tiny $_{lb}$}‚त्त्वे व्य‚व‚हार‚स्य साध्य‚त्वात् स्व‚भाव‚हेतुः । एव‚न्ताव‚द‚त‚द्रूप‚त्वे सामान्य‚स्यान्य‚त्त्व‚मेवा‚{\tiny $_{lb}$}‚पादित‚म् [।] अथान्य‚त्त्वं नेष्य‚ते त‚दा त‚त्त्वं प्राप्नोतीत्याह । \textbf{त‚च्चे}त्यादि । व्य‚क्तेर‚{\tiny $_{lb}$}‚\textbf{न‚न्य‚त्त‚दा त‚देव} व्य‚क्तिरूप‚मेव \textbf{त‚त्} सामान्य‚म्भ‚व‚ति । \textbf{अत‚त्त्वे} इत्य‚व्य‚क्ति‚{\tiny $_{५}$}‚रूप‚त्वे‚{\tiny $_{lb}$}‚ \textbf{व‚स्त्व‚न्त‚र‚व‚द‚न्य‚त्त्व‚प्र‚स‚ङ्गात्} । एत‚च्चान‚न्त‚र‚मेवोक्तं ।
	{\color{gray}{\rmlatinfont\textsuperscript{§~\theparCount}}}
	\pend% ending standard par
      ‚{\tiny $_{lb}$}‚

	  
	  \pstart \leavevmode% starting standard par
	अस्त्व‚न‚न्य‚त्त्वं सामान्य‚स्य त‚थापि व्य‚क्त्य‚न्त‚र‚म‚नुयास्य‚तीत्य‚त आह । \textbf{न‚{\tiny $_{lb}$}‚ चैके}त्यादि । \textbf{एक‚व्य‚क्तिस्व‚भाव‚स्य व्य‚क्त्य‚न्त‚रान्वावेशो}नुग‚मो व्य‚क्त्य‚न्त‚र‚स्व‚भात्व‚{\tiny $_{lb}$}‚मिति याव‚त् । क‚स्मात् त‚स्याव‚ग‚म्य‚मान‚स्या\textbf{व्य‚क्त्य‚न्त‚र‚त्व‚प्र‚स‚ङ्गात्} । य‚दि शाब‚ले‚{\tiny $_{lb}$}‚यात्म‚कं सामान्यं बाहुलेय‚स्यात्म‚भूतं भ‚वेत्त‚दा बाहुलेयः शाब‚ले‚{\tiny $_{६}$}‚य एव जातः शाब‚ले‚{\tiny $_{lb}$}‚यात्म‚कात् सामान्याद‚व्य‚क्तिरेकाच्छाब‚लेय‚व‚दिति कुतोस्य व्य‚क्त्य‚न्त‚र‚त्वं । \textbf{त‚त}‚{\tiny $_{lb}$}‚ इति त‚स्माद् \textbf{व्य‚क्तेर‚व्य‚तिरेकिणः} सामान्यात्स‚काशाद् \textbf{अन्व‚यिनी}त्य‚नुगामिनी ।‚{\tiny $_{lb}$}‚ ‚{\tiny $_{lb}$}‚ ‚{\tiny $_{lb}$}‚ \leavevmode\ledsidenote{\textenglish{297/s}}\textbf{नापि व्य‚तिरेकि}णः सामान्याद् अन्व‚यिनी बुद्धिरिति प्र‚कृतं । क‚स्मात् [।] \textbf{त‚स्य}‚{\tiny $_{lb}$}‚ व्य‚तिरिक्त‚स्य सामान्य‚स्य \textbf{क्व‚चि}द् भेदे\textbf{ऽनाश्र‚या}द‚प्र‚वृत्तेः [।] स‚म्ब‚न्ध‚म‚न्त‚रेण प्र‚वृ‚{\tiny $_{७}$}‚‚{\tiny $_{lb}$}‚त्त्य‚योगात् । वृत्तिराधेय‚ता व्य‚क्तिरिति त‚स्मिन्न युज्य‚त इत्यादिना च व्य‚ङ्ग्य- \leavevmode\ledsidenote{\textenglish{109b/PSVTa}}‚{\tiny $_{lb}$}‚ व्य‚ञ्ज‚क‚स्याधाराधेय‚भाव‚स्य च स‚म्ब‚न्ध‚स्य निषिद्ध‚त्वात् । अन्य‚स्त‚र्हि सामान्य‚{\tiny $_{lb}$}‚त‚द्व‚तोः स‚म्ब‚न्धो भ‚विष्य‚तीति चेदाह । \textbf{अन्य‚स्यापी}त्यादि । \textbf{व्य‚ङ्ग्य‚व्य‚ञ्ज‚क‚भावा}‚{\tiny $_{lb}$}‚देरिति प‚ञ्च‚मी । आदिश‚ब्दादाधाराधेय‚भाव‚प‚रिग्र‚हः । एत‚स्मात् पूर्व‚निषिद्धात्‚{\tiny $_{lb}$}‚ स‚म्ब‚न्ध‚द्व‚याद\textbf{न्य‚स्यापि} य‚स्य क‚स्य‚{\tiny $_{१}$}‚चित् \textbf{स‚म्ब‚न्ध‚स्य} व्य‚क्तिं प्र‚ति सामान्य‚स्या‚{\tiny $_{lb}$}‚\textbf{भावात्} । किङ्कार‚णं [।] नित्य‚त्वात् केन‚चिद् व्य‚क्तिभेदेनानुप‚कार्य‚स्य सामान्य‚स्या‚{\tiny $_{lb}$}‚\textbf{प्र‚तिब‚न्धेन} । न ह्य‚प्र‚तिब‚न्ध‚स्य क‚श्चित् स‚म्ब‚न्धोस्तीत्युक्तं ।
	{\color{gray}{\rmlatinfont\textsuperscript{§~\theparCount}}}
	\pend% ending standard par
      ‚{\tiny $_{lb}$}‚

	  
	  \pstart \leavevmode% starting standard par
	एतेन \textbf{चान्य‚त्त्वेऽपाश्र‚या}दिति श्लोक‚भागो व्याख्यातः । \href{http://sarit.indology.info/?cref=pv.3.149-3.150}{। १५२-५३ ॥}
	{\color{gray}{\rmlatinfont\textsuperscript{§~\theparCount}}}
	\pend% ending standard par
      ‚{\tiny $_{lb}$}‚

	  
	  \pstart \leavevmode% starting standard par
	अस‚त्स‚म्ब‚न्ध‚म‚पि सामान्यं व्य‚क्तिषु स्व‚रूपानुकारिणीं प्र‚तीतिं ज‚न‚य‚तीति चेदाह ।‚{\tiny $_{lb}$}‚ \textbf{अस‚म्ब‚न्धा}दित्यादि । नास्य स‚म्ब‚न्धोस्तीति विग्र‚हः । स‚{\tiny $_{२}$}‚र्व‚स्मात् स‚र्व‚त्र प्र‚तीतिः‚{\tiny $_{lb}$}‚ स्यादित्य‚र्थः । \textbf{त‚दि}ति त‚स्मा\textbf{द‚य‚मि}ति सा मा न्य वा दी । \textbf{एक‚स्य} सामान्य‚स्य \textbf{द‚र्श‚नेन}‚{\tiny $_{lb}$}‚ हेतुना । \textbf{एक}स्मिन् शाब‚लेये व्य‚क्तिभेदे \textbf{वृत्ति}र्य‚स्य त‚स्यान्य‚त्र व्य‚क्त्य‚न्त‚रे \textbf{वृत्तिम‚{\tiny $_{lb}$}‚न्विच्छिन्} व‚स्तुत्वे\textbf{नेष्ट‚स्य} सामान्य‚स्य व्य‚क्तेः स‚काशाद् ये \textbf{त‚त्त्वान्य‚त्त्वे} । ते \textbf{नाक्र‚म‚ति}‚{\tiny $_{lb}$}‚ व‚स्तुनो ग‚त्य‚न्त‚राभावात् । चोक्तो दोष इत्\textbf{य‚युक्}त‚मेत‚द्व‚स्तुभूतात् सामान्याद‚नुयायि‚{\tiny $_{lb}$}‚ज्ञा‚{\tiny $_{३}$}‚न‚मिति ।
	{\color{gray}{\rmlatinfont\textsuperscript{§~\theparCount}}}
	\pend% ending standard par
      ‚{\tiny $_{lb}$}‚

	  
	  \pstart \leavevmode% starting standard par
	य‚त एव\textbf{न्त‚स्माद‚र्थेषु} प‚र‚स्प‚र‚विवेकिष्विय\textbf{मेक‚रूपैकाकारा} प्र‚तीति\textbf{र्भ्रान्तिरेव} [।]‚{\tiny $_{lb}$}‚ भिन्नेष्व‚भेदाध्यारोपेण वृत्तेः । कुत‚स्त‚र्हि सोत्म‚न्नेत्याह । \textbf{विक‚ल्पे}त्यादि । विजातीय‚{\tiny $_{lb}$}‚व्यावृत्त‚प‚दार्थानुभ‚वेन या त‚थाभूत‚विक‚ल्प‚स्य प्र‚कृत्या ज‚निका वास‚नाहिता त‚तः‚{\tiny $_{lb}$}‚ स‚मुत्थिताः । एत‚च्च प्रागेवोक्त‚मित्याह । \textbf{भाव‚भेद} इत्यादि । भावानान्त‚त्कार्याणाम‚{\tiny $_{lb}$}‚त‚त्कार्येभ्यो‚{\tiny $_{४}$}‚ भेदः । त‚थाभूतानां चानुभावेनाहिता या वास‚ना त‚स्याः \textbf{प्र‚कृतिश्च}‚{\tiny $_{lb}$}‚ स्व‚भाव‚श्चा\textbf{स्या आश्र‚य इति निर्णीत‚मेत‚त्} प्राक त‚त्र भाव‚भेदः पार‚म्प‚र्येण कार‚णं‚{\tiny $_{lb}$}‚ ‚{\tiny $_{lb}$}‚ \leavevmode\ledsidenote{\textenglish{298/s}}वास‚ना प्र‚कृतिः साक्षादिति द्व‚य‚मुप‚न्य‚स्तं । य‚द्य‚न्यापोह एव श‚ब्द‚वाच्यः क‚थ‚न्त‚र्ही‚{\tiny $_{lb}$}‚दानीमित्यादि । \textbf{प्र‚धानेश्व‚रादिकार्य‚श‚ब्दा} इति प्र धा न कार्य मी श्व र कार्य‚ञ्ज‚ग‚{\tiny $_{lb}$}‚दिति । आदिश‚ब्दाच्छ ब्द ब्र ह्म‚{\tiny $_{५}$}‚प‚रिणाम इत्यादिश‚ब्दानां प‚रिग्र‚हः ।
	{\color{gray}{\rmlatinfont\textsuperscript{§~\theparCount}}}
	\pend% ending standard par
      ‚{\tiny $_{lb}$}‚

	  
	  \pstart \leavevmode% starting standard par
	\textbf{भावे}ष्वाध्यात्मिक‚बाह्येषु । \textbf{अत‚द्भूतो}ऽप्र‚धानादिकार्यात्म‚को भेदो येषान्ते‚{\tiny $_{lb}$}‚ त‚थोक्ताः । य‚द्य‚प्य‚प्र‚धान‚कार्यात्तेषां भेदः स्यात् त‚दा भ‚वेत् प्र‚धानादिकार्यात्म‚को‚{\tiny $_{lb}$}‚ भेदः [।] स एव च स‚र्वेषाम‚भेदः । तेना\textbf{भेदेना}निमित्तेन स‚र्व‚त्र \textbf{व‚र्त्त‚न्ते} । स च‚{\tiny $_{lb}$}‚ नास्ति भावानाम‚न्यापोह‚वादिनो म‚तेनाप्र‚धानादिकार्यात्म‚क‚त्वात् । त‚त‚श्च‚{\tiny $_{६}$}‚ क‚थ‚{\tiny $_{lb}$}‚मेवंभूतेष्व‚भेदेन व‚र्त्त‚न्ते । नैवेत्य‚भिप्रायः । त‚त‚श्चाव्यापिन्य‚पोह‚व्य‚व‚स्थेति भावः ।
	{\color{gray}{\rmlatinfont\textsuperscript{§~\theparCount}}}
	\pend% ending standard par
      ‚{\tiny $_{lb}$}‚

	  
	  \pstart \leavevmode% starting standard par
	\textbf{तेपी}त्यादिना प‚रिह‚र‚ति । तेपि प्र‚धानादिकार्य‚श‚ब्दा विक‚ल्प‚विज्ञान‚प्र‚तिभा‚{\tiny $_{lb}$}‚सिन्य‚र्थे प्र‚व‚र्त्त‚न्ते इति स‚म्ब‚न्धः । क‚थ‚मित्याह [।] \textbf{संकेते}त्यादि । व‚स्तुन्य\textbf{त‚था‚{\tiny $_{lb}$}‚भूते} । इच्छाव‚शाद् यः संकेतः प्र‚धान‚कार्य \textbf{ज‚ग‚दिति} । तेनाहिता या वास‚नाश‚क्ति‚{\tiny $_{lb}$}‚\leavevmode\ledsidenote{\textenglish{110a/PSVTa}} स्त‚योत्त‚रोत्त‚र‚क्ष‚ण‚विप‚{\tiny $_{७}$}‚रिणामे\textbf{नोप‚स्कृत‚त्वा}द् विज्ञान‚स‚न्त‚तेः स‚र्वेषां बाह्याध्य‚त्मिका‚{\tiny $_{lb}$}‚नाम\textbf{र्थानां द‚र्श‚ने}ष्व‚नुभ‚वेषु स‚त्स्व‚प्य\textbf{न‚पेक्ष्य त‚द्भेद}म‚प्र‚धान‚कार्याद् भेद‚म्व‚स्तुग‚तं ।‚{\tiny $_{lb}$}‚ य‚दि नामार्थानाम‚प्र‚धान‚कार्याणामेव द‚र्श‚न‚म्व‚स्तुध‚र्मेण त‚थापि व‚स्तुस्व‚भाव‚{\tiny $_{lb}$}‚म‚न‚पेक्ष्येत्य‚र्थः [।] \textbf{त‚थाध्य‚व‚सायाद् य‚थासंकेत}म‚प्र‚धान‚कार्यान‚पि भावान् प्र‚धान‚{\tiny $_{lb}$}‚कार्य‚त्वेनाध्य‚व‚सानात् । अत‚थाभूत‚क‚ल्पित‚प्र‚धान‚कार्य‚त्वेन \textbf{क‚ल्पि‚{\tiny $_{१}$}‚तं} चैत‚न्यं सां ख्ये न‚{\tiny $_{lb}$}‚ त‚स्य \textbf{व्य‚व‚च्छेदेन} प्र‚धान‚कार्याभावा इति य‚द् \textbf{विक‚ल्प‚विज्ञान‚न्त‚त्प्र‚तिभासिन्य‚र्थे} ।‚{\tiny $_{lb}$}‚ स एव विजातीय‚व्य‚व‚च्छेदेनान्यापोह इति भावः । किं भूतास्ते श‚ब्दा इत्याह ।‚{\tiny $_{lb}$}‚ \textbf{उपादाने}त्यादि । \textbf{विक‚ल्प}हेतोर्वास‚नाया दार्ढ्य‚मु\textbf{पादान‚ब‚ल}न्त‚स्मात् \textbf{प्र‚भ‚व} उत्पादो‚{\tiny $_{lb}$}‚ य‚स्य विक‚ल्प‚स्य त‚स्मात् \textbf{स‚मुत्थिताः}
	{\color{gray}{\rmlatinfont\textsuperscript{§~\theparCount}}}
	\pend% ending standard par
      ‚{\tiny $_{lb}$}‚

	  
	  \pstart \leavevmode% starting standard par
	एत‚दुक्त‚म्भ‚व‚ति । य‚दि नाम व‚स्तुनि त‚थाभूत‚भेदाभा‚{\tiny $_{२}$}‚व‚स्त‚थापि विक‚ल्पा‚{\tiny $_{lb}$}‚रोपित एवान्यापोहः श‚ब्दानां प्र‚वृत्तेर‚ङ्ग‚न्त‚तो नास्याव्यापितादोष इति । प्र‚धा‚{\tiny $_{lb}$}‚नादिकार्य‚श‚ब्दानाम‚भेदेन प्र‚वृत्तौ तेषु भावेषु सामान्य‚मेव व‚स्तुभूतं किन्नेष्य‚त इति‚{\tiny $_{lb}$}‚ चेदाह । \textbf{न ही}त्यादि । \textbf{तेषु} प्र‚धान‚कार्य‚त्वेनाध्यारोपितेषु । \textbf{अत‚थाभुतेषु प्र‚धान‚कार्येषु}‚{\tiny $_{lb}$}‚ व्य‚क्त्य‚भावात् सामान्य‚स्याभाव इति भावः ।
	{\color{gray}{\rmlatinfont\textsuperscript{§~\theparCount}}}
	\pend% ending standard par
      ‚{\tiny $_{lb}$}‚‚{\tiny $_{lb}$}‚\textsuperscript{\textenglish{299/s}}

	  
	  \pstart \leavevmode% starting standard par
	अन्यापोह‚वादिन‚स्तु न‚{\tiny $_{३}$}‚ दोष इत्याह । \textbf{त‚थेत्}यादि । प्र‚धानादिकार्य‚त्व‚क‚ल्प‚{\tiny $_{lb}$}‚न‚येत्य‚र्थः । \textbf{त‚द‚न्य‚स्या}प्र‚धानादिकार्य‚स्य \textbf{भेदो} व्य‚व‚च्छेदः प्र‚धानादिकार्य‚त्वेनारोपि‚{\tiny $_{lb}$}‚तानाम्भावानां \textbf{प्र‚तिप‚त्तृणाम‚ध्य‚व‚साय‚व‚शात् स्यात्} ।
	{\color{gray}{\rmlatinfont\textsuperscript{§~\theparCount}}}
	\pend% ending standard par
      ‚{\tiny $_{lb}$}‚

	  
	  \pstart \leavevmode% starting standard par
	त‚द‚ध्य‚व‚साय‚व‚शादेव प्र‚तिप‚पअध्य‚व‚साय‚व‚शादेव शाव‚लेयादिष्व‚नुवृत्तिप्र‚त्य‚य‚{\tiny $_{lb}$}‚निमित्तं \textbf{सामान्यं किन्नेति} चेत् ।
	{\color{gray}{\rmlatinfont\textsuperscript{§~\theparCount}}}
	\pend% ending standard par
      ‚{\tiny $_{lb}$}‚

	  
	  \pstart \leavevmode% starting standard par
	नैत‚द‚स्ति । य‚स्मात् \textbf{तेन} सामान्य‚वादिना‚{\tiny $_{४}$}‚ सामान्यं क‚ल्प‚य‚ताप्य\textbf{व‚श्यं} त‚त्र‚{\tiny $_{lb}$}‚ शाब‚लेयादिषु विजातीयाद् \textbf{भेदो नान्त‚रीय‚क‚त‚येष्ट}व्योन्य‚था गोत्वादेर‚सिद्धिः स्यात् ।‚{\tiny $_{lb}$}‚ \textbf{स एव} भेदः \textbf{सामान्य‚कार्ये}ऽभिन्न‚श‚ब्द‚प्र‚वृत्त्यादिल‚क्ष‚णे \textbf{प‚र्याप्तः} श‚क्तः । \textbf{इत्येवं निष्प्र‚{\tiny $_{lb}$}‚योज‚ना सामान्य‚क‚ल्प‚ना} ।
	{\color{gray}{\rmlatinfont\textsuperscript{§~\theparCount}}}
	\pend% ending standard par
      ‚{\tiny $_{lb}$}‚

	  
	  \pstart \leavevmode% starting standard par
	एत‚दुक्त‚म्भ‚व‚ति । य‚था प्र‚धान‚कार्येष्व‚पि भावेषु सामान्य‚म‚न्त‚रेण प्र‚धानादि‚{\tiny $_{lb}$}‚कार्य‚श‚ब्दास्त‚द्बुद्ध‚य‚श्चैकाकाराः‚{\tiny $_{५}$}‚ प्र‚व‚र्त्त‚न्ते । त‚था ग‚वादिषु ग‚वादिश‚ब्दास्त‚द्‚{\tiny $_{lb}$}‚बुद्ध‚य‚श्चैकाकाराः किन्नेष्य‚न्ते किं सामान्येन पार‚मार्थिकेन क‚ल्पितेन । दृष्टा च‚{\tiny $_{lb}$}‚ प‚रैर‚पि सामान्य‚म‚न्त‚रेण ब‚हुषु सामान्येष्व‚भिन्नाभिधान‚प्र‚त्य‚य‚वृत्तिः ।
	{\color{gray}{\rmlatinfont\textsuperscript{§~\theparCount}}}
	\pend% ending standard par
      ‚{\tiny $_{lb}$}‚

	  
	  \pstart \leavevmode% starting standard par
	त‚दुक्त‚म्भ ट्टो द्यो त क रा भ्यां
	{\color{gray}{\rmlatinfont\textsuperscript{§~\theparCount}}}
	\pend% ending standard par
      ‚{\tiny $_{lb}$}‚
	  \bigskip
	  \begingroup
	
	    
	    \stanza[\smallbreak]
	  {\normalfontlatin\large ``\qquad}त‚स्मादेक‚स्य भिन्नेषु या वृत्तिस्त‚न्निब‚न्ध‚नः ।&‚{\tiny $_{lb}$}‚सामान्य‚श‚ब्दः स‚त्तादावेक‚धीक‚र‚णेन वेति ।{\normalfontlatin\large\qquad{}"}\&[\smallbreak]
	  
	  
	  \href{http://sarit.indology.info/?cref=\%C5\%9Bv-\%C4\%81k\%E1\%B9\%9Bti.24}{श्लोक वा० आकृ० २४}
	  \endgroup
	‚{\tiny $_{lb}$}‚

	  
	  \pstart \leavevmode% starting standard par
	तेनाय‚म‚र्थः [।] य‚था प्र‚त्येक‚{\tiny $_{६}$}‚म‚नेकार्थ‚स‚म‚वायित्वेन स‚त्त्व‚द्र‚व्य‚त्वादौ सामान्य‚{\tiny $_{lb}$}‚श‚ब्द‚स्त‚द्बुद्धिश्च सामान्य‚म‚न्त‚रेण प्र‚व‚र्त्तेते [।] न चानेकार्थ‚स‚म‚वायित्वं सामान्यं‚{\tiny $_{lb}$}‚ य‚देव स‚त्त्वे त‚देव द्र‚व्य‚त्वादाव‚स्ति । \textbf{निःसामान्यानि सामान्यानीति} व‚च‚नात् ।‚{\tiny $_{lb}$}‚ न चोप‚चारात्त‚योः श‚ब्द‚ज्ञान‚योः वृत्तिर‚स्ख‚ल‚द्बुद्धिग्राह्य‚त्वात् । त‚स्माद् य‚था‚{\tiny $_{lb}$}‚ सामान्यं विना त‚योः स‚त्तादौ वृत्ति‚{\tiny $_{७}$}‚स्त‚था शाब‚लेयादिषु सामान्य‚म‚न्त‚रेण [।] \leavevmode\ledsidenote{\textenglish{110b/PSVTa}}‚{\tiny $_{lb}$}‚ य‚था शाब‚लेयोऽगोव्यावृत्त‚स्त‚था बाहुलेयोऽगोव्यावृत्त‚स्त‚था ख‚ण्डोऽगोव्यावृत्त इत्य‚{\tiny $_{lb}$}‚ भिन्नाभिधान‚प्र‚त्य‚य‚वृत्तिः किन्नेष्य‚त इत्य‚र्थः ।
	{\color{gray}{\rmlatinfont\textsuperscript{§~\theparCount}}}
	\pend% ending standard par
      ‚{\tiny $_{lb}$}‚

	  
	  \pstart \leavevmode% starting standard par
	स्यादेत‚त् [।] स‚र्व‚त्र सामान्य‚बुद्धिर्निर्विष‚येष्टैव केव‚लं सामान्य‚म‚न्त‚रेण क्व‚चि‚{\tiny $_{lb}$}‚द‚विस‚म्वादो न स्यादित्य‚त आह । \textbf{य‚दी}त्यादि । य‚दिश‚ब्दोभ्युप‚ग‚म‚द्योत‚नार्थः ।‚{\tiny $_{१}$}‚‚{\tiny $_{lb}$}‚ य‚द्य‚स्यास्स‚र्व‚त्र निर्विष‚य‚त्व‚म‚भ्युप‚ग‚म्य‚ते । \textbf{स‚त्स्विति} विद्य‚मानेषु भावेष्व‚गोव्यावृ‚{\tiny $_{lb}$}‚त्तेषु । \textbf{अस‚त्स्वि}ति प‚र‚मार्थ‚तः प्र धा ने श्व रा दिकार्य‚त‚याऽविद्य‚मानेषु भावेषु । \textbf{नेय‚{\tiny $_{lb}$}‚‚{\tiny $_{lb}$}‚ \leavevmode\ledsidenote{\textenglish{300/s}}म‚र्थ‚व‚ती सामान्य‚बुद्धिः} । त‚या य‚थारोपित‚स्याभिन्नाकार‚स्य बाह्येष्व‚भाव\textbf{भा}\edtext{}{\lemma{भाव}\Bfootnote{?}}‚{\tiny $_{lb}$}‚ वाद‚त‚श्चाभूत‚ग्र‚हाद् । \textbf{विप्ल‚वो} भ्रान्तिरेवेति कृत्वा । \textbf{नास्या}स्सामान्य‚बुद्धेर्निर्विष‚{\tiny $_{lb}$}‚याया \textbf{विष‚य‚निरूप‚णं प्र‚ति क‚श्चि‚{\tiny $_{२}$}‚दाद‚रः} । य‚दि स‚र्वैव सामान्य‚बुद्धिर्भ्रान्ता क‚थ‚{\tiny $_{lb}$}‚न्त‚र्ह्य‚नुमानाद् व‚स्तुस‚म्वाद इत्याह । \textbf{क्व‚चिदि}त्यादि । क्व‚चिद् \textbf{व‚स्तुन्य‚स्य} बुद्धेः स‚का‚{\tiny $_{lb}$}‚शाद‚विस‚म्वादो \textbf{य‚स्म}आत् \textbf{कार्य‚कार‚ण‚स‚म्ब‚द्धाद्} य‚थोक्तात् । एत‚च्च निवेदितं‚{\tiny $_{lb}$}‚ प्राक् । य‚त्रास्ति व‚स्तुस‚म्ब‚न्धो य‚थोक्तानुमितौ य‚थेत्यादिना \href{http://sarit.indology.info/?cref=pv.3.80}{१ । ८३} । न‚{\tiny $_{lb}$}‚ \textbf{त‚थाभूत‚स्या}भिन्न‚रूप‚स्यानुमान‚ग्राह्य‚स्य व‚स्तुनि स‚मावेशाद् \textbf{विद्य‚मान‚त्वाद}नु‚{\tiny $_{lb}$}‚मान‚विक‚ल्प‚स्य‚{\tiny $_{३}$}‚ व‚स्तुविस‚म्वादः । \textbf{प्र‚त्य‚क्ष‚व‚दिति} । वैध‚र्म्य‚दृष्टान्तः । किङ्कार‚{\tiny $_{lb}$}‚ण‚म् [।] \textbf{अत‚थाभावे}पीति व्य‚तीतेपि व‚स्तुनि प‚र‚म्प‚र‚या लिङ्गानुसारेण भावात् ।‚{\tiny $_{lb}$}‚ य‚द्वाऽ\textbf{त‚थाभावेपि} सामान्य‚र‚हितेऽपि व‚स्तुन्य‚भिन्नाकाराया बुद्धे\textbf{र्भावात्} । इति‚{\tiny $_{lb}$}‚ एवं । \textbf{निवेद‚यिष्यामः । निवेदितं च} प्राक् । \textbf{भावाभावानुविधानात् साम‚र्थ्य}‚{\tiny $_{lb}$}‚मित्य‚न्त‚रे [।]
	{\color{gray}{\rmlatinfont\textsuperscript{§~\theparCount}}}
	\pend% ending standard par
      ‚{\tiny $_{lb}$}‚

	  
	  \pstart \leavevmode% starting standard par
	य‚दि सामान्य‚बुद्धिः स्वाकाराभेदेन भिन्नान् भावान‚भिन्नान‚ध्य‚स्य विधिरूप‚{\tiny $_{४}$}‚त‚या‚{\tiny $_{lb}$}‚ प्र‚तिप‚द्य‚ते क‚थ‚म‚स्या अन्यापोह‚विष‚य‚त्व‚मुक्त‚मित्य‚त आह । \textbf{भेदे}त्यादि । अस्याः‚{\tiny $_{lb}$}‚ सामान्य‚बुद्धेर्भिन्न‚प‚दार्थ‚द‚र्श‚न‚ब‚लेनेति विजातीय‚व्यावृत्त‚स्व‚ल‚क्ष‚णानुभ‚व‚साम‚र्थ्येन ।‚{\tiny $_{lb}$}‚ उत्प‚त्तेरित्य‚ध्याहारः । ब‚हुल‚ग्र‚ह‚ण‚म्व‚स्त्व‚भावेपि श‚श‚विषाणादौ विक‚ल्प‚बुद्धेः प्र‚वृत्ति‚{\tiny $_{lb}$}‚ख्याप‚नार्थं । \textbf{तेषु} भिन्नेषु स्व‚ल‚क्ष‚णेषु \textbf{भावाध्य‚व‚सायात्} स्वाकाराभेदे‚{\tiny $_{५}$}‚न स्व‚रूपाध्य‚{\tiny $_{lb}$}‚व‚सायात् । दृश्य‚विक‚ल्प‚योरेकीकृत्य प्र‚वृत्तेरित्य‚र्थः । य‚स्माद् \textbf{भिन्न‚व‚स्तुद‚र्श‚न‚ब‚ले‚{\tiny $_{lb}$}‚नो}त्प‚द्य‚ते बुद्धिरुत्प‚न्ना च तान्येव भिन्न‚व‚स्तूनि स्वाकाराभेदेन प्र‚तिप‚द्य‚ते [।] त‚स्माद्‚{\tiny $_{lb}$}‚ भेद‚विष‚य‚त्व‚म्भिन्न‚विष‚य‚त्व‚मित्य‚र्थः । युक्त‚न्ताव‚द्व‚स्तुद‚र्श‚न‚द्वारायातेष्व‚नित्यादिवि‚{\tiny $_{lb}$}‚क‚ल्पेषु व‚स्तुद‚र्श‚न‚ब‚लोत्प‚त्तेर्भेद‚विष‚य‚त्वं । य‚त्र तु न त‚थाभूत‚म्भिन्न‚{\tiny $_{६}$}‚म्व‚स्तु । य‚था‚{\tiny $_{lb}$}‚ नित्यादिविक‚ल्पेषु । श‚श‚विषाणादिविक‚ल्पेषु च [।] त‚त्र क‚थ‚म्भेद‚विष‚य‚त्वं तेषामिति‚{\tiny $_{lb}$}‚ चेदाह । \textbf{त‚था भावे}त्यादि । य‚थैव भिन्न‚व‚स्तुस्व‚भाव‚ग्राह्यानुभ‚व‚ब‚लेनोत्प‚न्ना अनि‚{\tiny $_{lb}$}‚त्यादिबुद्ध‚यः स्व‚प्र‚तिभासे भिन्न‚भावाध्य‚व‚सायेन प्र‚व‚र्त्त‚माना भेद‚विष‚याः । \textbf{एव‚न्त‚था‚{\tiny $_{lb}$}‚ \leavevmode\ledsidenote{\textenglish{111a/PSVTa}} भाव‚क‚ल्प‚नाया}मेव स्व‚प्र‚तिभास एव भिन्न‚बाह्य‚भावाध्य‚व‚सा‚{\tiny $_{७}$}‚य‚क‚ल्प‚नायां स‚त्यां‚{\tiny $_{lb}$}‚ भिन्न‚विष‚य‚त्वे \textbf{स‚त्य‚प‚र‚त्रेति} अविद्य‚मानेषु नित्य‚प्र‚धान‚कार्यादिषु श‚श‚विषाणादिषु च‚{\tiny $_{lb}$}‚ ‚{\tiny $_{lb}$}‚ \leavevmode\ledsidenote{\textenglish{301/s}}नित्य‚प्र‚धानादिकार्य‚श‚श‚विषाणादिविक‚ल्पानाम्भेद‚विष‚य‚त्व‚स्य \textbf{भावा}त् । \textbf{स्वाश्र‚य‚मा}‚{\tiny $_{lb}$}‚त्र‚ग‚त‚व्य‚क्तिभेद एव स्थितं । न तु व्य‚क्तिशून्ये देशे । त‚दुक्तं भ ट्टे न ।
	{\color{gray}{\rmlatinfont\textsuperscript{§~\theparCount}}}
	\pend% ending standard par
      ‚{\tiny $_{lb}$}‚
	  \bigskip
	  \begingroup
	
	    
	    \stanza[\smallbreak]
	  {\normalfontlatin\large ``\qquad}पिण्डेष्वेव च सामान्यं नान्त‚रा गृह्य‚ते य‚तः ।&‚{\tiny $_{lb}$}‚न ह्याकाश‚व‚दिच्छ‚न्ति सामान्य‚न्नाम केच‚न ॥ \href{http://sarit.indology.info/?cref=\%C5\%9Bv-\%C4\%81k\%E1\%B9\%9Bti}{आकृ० २५}&‚{\tiny $_{lb}$}‚प्र‚त्येक‚स‚म‚वेत‚त्व‚न्दृष्ट‚त्वान्न निरोत्स्य‚ते ।&‚{\tiny $_{lb}$}‚त‚था च स‚ति नानात्व‚न्नैक‚बुद्धेर्भ‚विष्य‚ति ॥ \href{http://sarit.indology.info/?cref=\%C5\%9Bv-vana.30}{व‚न० ३०}&‚{\tiny $_{lb}$}‚य‚था च व्य‚क्तिरेकैव दृश्य‚माना पुनः पुनः ।&‚{\tiny $_{lb}$}‚काल‚भेदेप्य‚भिन्नैव जातिर्भिन्नाश्र‚या स‚तीति ॥\edtext{\textsuperscript{*}}{\edlabel{pvsvt_301-1}\label{pvsvt_301-1}\lemma{*}\Bfootnote{\href{http://sarit.indology.info/?cref=\%C5\%9Bv-vana.33}{ Ślokavārtika. }}} \href{http://sarit.indology.info/?cref=\%C5\%9Bv}{व‚न० ३३}{\normalfontlatin\large\qquad{}"}\&[\smallbreak]
	  
	  
	  
	  \endgroup
	‚{\tiny $_{lb}$}‚

	  
	  \pstart \leavevmode% starting standard par
	उ द्यो त क रो प्याह ।\edtext{\textsuperscript{*}}{\edlabel{pvsvt_301-2}\label{pvsvt_301-2}\lemma{*}\Bfootnote{\href{http://sarit.indology.info/?cref=nv}{ Nyāyavārtika. }}} केन स‚र्व‚ग‚त‚त्वं जातेर‚भ्युप‚ग‚म्य‚ते येन मृत्पिण्डे‚{\tiny $_{lb}$}‚ मृद्ग‚व‚के गोत्वं स्याद‚पि तु स्व‚विष‚ये स‚र्व‚त्र वृत्तिर्व‚र्त्त‚त इति स‚र्व‚ग‚तेत्युच्य‚ते । कः‚{\tiny $_{lb}$}‚ पुन‚र्गोत्व‚स्य स्वो विष‚यः । य‚त्र गो‚{\tiny $_{२}$}‚त्व‚म्भ‚व‚ति । क्व पुन‚र्गोत्व‚म्व‚र्त्त‚ते । य‚त्र गोत्व‚{\tiny $_{lb}$}‚निमित्तोनुवृत्तिप्र‚त्य‚य‚यो भ‚व‚ति । क्व पुन‚र‚नुवृत्तिप्र‚त्य‚यं गोत्वं क‚रोति [।] य‚त्त‚स्य‚{\tiny $_{lb}$}‚ साध‚नं । कः पुन‚र्नित्ये गोत्वे गोस्साध‚नार्थः । य‚त्तेन व्य‚ज्य‚ते । न हि क‚कुदादिम‚द‚र्थ‚{\tiny $_{lb}$}‚व्य‚तिरेकेण गोत्व‚स्य व्य‚क्तिरिति । न पिण्डेभ्योर्थान्त‚रं गोत्व‚म्पिण्डान्त‚रालेष्व‚{\tiny $_{lb}$}‚ग्र‚ह‚णादिति बौ द्धो ब्रुवाणः पिण्डान्त‚राल‚म्प‚र्य‚नुयोज्यः । किमिद‚म्पि‚{\tiny $_{३}$}‚ण्डान्त‚रालं ।‚{\tiny $_{lb}$}‚ किमाकाश‚माहोस्विद‚भाव उत द्र‚व्यान्त‚र‚मिति । य‚द्याकाशं न त‚त्र गोत्वं न ह्याकाशं‚{\tiny $_{lb}$}‚ गौरिति प्र‚तीय‚ते । एतेनाभावो द्र‚व्यान्त‚रं च व्याख्यातं ।‚{\tiny $_{lb}$}‚ विशेष‚प्र‚त्य‚यानामाक‚स्मिक‚त्वाच्च । अयं पिण्ड‚प्र‚त्य‚य‚व्य‚तिरेक‚भाक् प्र‚त्य‚य उप‚{\tiny $_{lb}$}‚जाय‚मानो निमित्तान्त‚राद् भ‚व‚ति । दृष्टा ख‚लु पिण्ड‚व्य‚तिरेक‚भाजां प्र‚त्य‚यानां‚{\tiny $_{lb}$}‚ निमित्तान्त‚रादुत्प‚त्तिर्य‚था च‚र्म‚व‚स्त्र‚क‚{\tiny $_{४}$}‚म्ब‚लेषु नील‚प्र‚त्य‚य‚स्त‚च्च निमित्तान्त‚रं सामा‚{\tiny $_{lb}$}‚न्य‚मिति [।] त‚स्माद् व्य‚क्तिस‚र्व‚ग‚तं सामान्यं । व्य‚क्तिशून्ये‚{\tiny $_{५}$}‚पि देशे विद्य‚मान‚{\tiny $_{lb}$}‚त्वात् । न च त‚त्र सामान्य‚स्य प्र‚तीतिर्व्य‚ञ्जिकाया व्य‚क्तेर‚भावात् । य‚त्रैव च‚{\tiny $_{lb}$}‚ व्य‚क्तौ सामान्यं प्र‚तीय‚ते सैव सामान्याभिव्य‚क्तौ स‚म‚र्था सामान्य‚प्र‚तिप‚त्त्य‚न्य‚थानुप‚{\tiny $_{lb}$}‚त्त्या ग‚म्य‚ते नान्येति ।\edtext{\textsuperscript{*}}{\lemma{*}\Bfootnote{\href{http://sarit.indology.info/?cref=nv}{Nyāyavārtika.}}}
	{\color{gray}{\rmlatinfont\textsuperscript{§~\theparCount}}}
	\pend% ending standard par
      ‚{\tiny $_{lb}$}‚

	  
	  \pstart \leavevmode% starting standard par
	त‚दाह भ ट्टः ।
	{\color{gray}{\rmlatinfont\textsuperscript{§~\theparCount}}}
	\pend% ending standard par
      ‚{\tiny $_{lb}$}‚
	  \bigskip
	  \begingroup
	
	    
	    \stanza[\smallbreak]
	  {\normalfontlatin\large ``\qquad}य‚द्वा स‚र्व‚ग‚त‚त्वेपि व्य‚क्तिः श‚क्त्य‚नु‚{\tiny $_{५}$}‚रोध‚तः ।&‚{\tiny $_{lb}$}‚श‚क्तिः कार्यानुमेयादिव्य‚क्तिर्द‚र्श‚न‚हेतुका ।&‚{\tiny $_{lb}$}‚\leavevmode\ledsidenote{\textenglish{302/s}}तेन य‚त्रैव दृश्येत व्य‚क्तिः श‚क्त‚न्त‚देव तु ।&‚{\tiny $_{lb}$}‚तेनैव च न स‚र्वासु व्य‚क्तिष्वेत‚त् प्र‚तीय‚ते ।&‚{\tiny $_{lb}$}‚भिन्न‚त्वेपि हि कासांचिच्छ‚क्तिः काश्चिद‚श‚क्तिकाः ।&‚{\tiny $_{lb}$}‚न च प‚र्य‚नुयोगोस्ति व‚स्तुश‚क्तेः क‚दाच‚न ।&‚{\tiny $_{lb}$}‚व‚ह्निर्द‚ह‚ति र्ना\edtext{}{\lemma{र्ना}\Bfootnote{? ना}}काशं कोत्र प‚र्य‚नुयुज्य‚तामिति ।\edtext{\textsuperscript{*}}{\edlabel{pvsvt_302-1}\label{pvsvt_302-1}\lemma{*}\Bfootnote{\href{http://sarit.indology.info/?cref=\%C5\%9Bv-\%C4\%81k\%E1\%B9\%9Bti.25-29}{ Ślokavārtika. Ākṛti. 25-29 }}}{\normalfontlatin\large\qquad{}"}\&[\smallbreak]
	  
	  
	  
	  \endgroup
	‚{\tiny $_{lb}$}‚

	  
	  \pstart \leavevmode% starting standard par
	\textbf{त‚त्र} त‚योः \textbf{प‚क्ष‚यो}र्म‚ध्ये \textbf{य‚दि स्वाश्र‚य‚मात्र}ग‚तं अपूर्व‚घ‚टाद्युत्प‚{\tiny $_{६}$}‚त्तौ घ‚ट‚त्वादिशून्ये‚{\tiny $_{lb}$}‚ प्र‚देशे प‚श्चादुत्प‚न्नाद् घ‚टादे\textbf{र्भिन्न‚देशं} य‚द् द्र‚व्य‚न्त‚द्व‚र्त्तिनः \textbf{सामान्य‚स्य । क‚थ‚न्तेषु}‚{\tiny $_{lb}$}‚ प‚श्चादुत्प‚न्नेषु घ‚टादिषु \textbf{स‚म्भ‚वो} नैवेत्य‚भिप्रायः । भ‚वेत् स‚म्भ‚वो य‚दि त‚स्मात् \textbf{पूर्व‚{\tiny $_{lb}$}‚द्र‚व्या}त् त‚त्सामान्य‚म्प‚श्चा\textbf{दुत्प‚द्य‚मान‚न्द्र‚व्यं याति} । त‚च्च नास्तीत्याह । \textbf{य‚स्मादि}‚{\tiny $_{lb}$}‚त्यादि । \textbf{त‚दि}ति सामान्यं \textbf{पूर्व‚द्र‚व्या}दिति य‚त्र त‚त्पूर्वं स‚म‚वेतं त‚स्मा\textbf{दुत्पित्सु} द्र‚व्य‚{\tiny $_{७}$}‚‚{\tiny $_{lb}$}‚\leavevmode\ledsidenote{\textenglish{111b/PSVTa}} मुत्प‚त्तुमिच्छु पूर्वं घ‚टादिकं न \textbf{याति} । अमूर्त्त‚त्वेन \textbf{निष्क्रिय‚त्वा}त् सामान्य‚स्य [।]‚{\tiny $_{lb}$}‚ पूर्व‚द्र‚व्याद‚च‚ल‚तोपि भिन्न‚देशेन योगो भ‚विष्य‚ति विम्ब‚स्याद‚र्श इवेति चेदाह । \textbf{न‚{\tiny $_{lb}$}‚ ही}त्यादि । \textbf{अन्य‚द्र‚व्य‚वृत्ते}रित्युत्पित्सुद्र‚व्याद् भिन्न‚देश‚द्र‚व्य‚वृत्तेर्भाव‚स्य सामान्याख्य‚स्य‚{\tiny $_{lb}$}‚ त‚तः पूर्व‚कादाश्र‚या\textbf{द‚च‚ल‚त‚स्त‚दुभ‚यान्त‚रालाव्यापिनः} पूर्व‚प‚श्चादुत्प‚न्न‚द्व‚यान्त‚राला‚{\tiny $_{lb}$}‚व्यापिनः स्वाश्र‚याद् \textbf{भिन्न‚दे‚{\tiny $_{१}$}‚शेन} द्र‚व्येण योगो \textbf{न हि युक्त} इति \textbf{स‚म्ब‚न्धः} । बिम्ब‚स्य‚{\tiny $_{lb}$}‚ तु भिन्न‚देशे नाद‚र्शेन योगोस्तीति ब्रुवाणः क‚थ‚न्नोन्म‚तः स्यात् । साम‚ग्रीब‚लाद्‚{\tiny $_{lb}$}‚ भ्रान्तं ज्ञानं प्र‚तिबिम्बानुग‚ताद‚र्श‚प्र‚तिभासि त‚त्र जाय‚ते । य‚थोक्तं [।]
	{\color{gray}{\rmlatinfont\textsuperscript{§~\theparCount}}}
	\pend% ending standard par
      ‚{\tiny $_{lb}$}‚
	  \bigskip
	  \begingroup
	
	    
	    \stanza[\smallbreak]
	  {\normalfontlatin\large ``\qquad}विरुद्ध‚प‚रिणामेषु व‚ज्राद‚र्श‚त‚लादिषु [।]&‚{\tiny $_{lb}$}‚प‚र्व‚तादिस्व‚भावानां भावानां नास्ति स‚म्भ‚व इति ।{\normalfontlatin\large\qquad{}"}\&[\smallbreak]
	  
	  
	  
	  \endgroup
	‚{\tiny $_{lb}$}‚

	  
	  \pstart \leavevmode% starting standard par
	येपि त‚त्र भावान्त‚रोत्प‚त्तिमिच्छ‚न्ति [।] तेषाम‚पि न बिम्बेन योगोस्तीति‚{\tiny $_{lb}$}‚ य‚त्किञ्चिदेत‚{\tiny $_{२}$}‚त् । उत्पित्सुद्र‚व्यात् \textbf{प्राक्} सामान्यात्मा \textbf{न च त‚त्रो}त्पित्सुदेशे‚{\tiny $_{lb}$}‚ \textbf{आसीत्} । व्य‚क्तिशून्ये देशे त‚स्य स्थानान‚भ्युप‚ग‚मात् । अस्ति प‚श्चात् त‚त्सामान्यं‚{\tiny $_{lb}$}‚ ‚{\tiny $_{lb}$}‚ ‚{\tiny $_{lb}$}‚ \leavevmode\ledsidenote{\textenglish{303/s}}व्य‚क्तावुत्प‚न्नायां । सामान्य‚शून्याया व्य‚क्तेर‚न‚भ्युप‚ग‚मात् । न च त‚त्र देशे व्य‚क्त्या‚{\tiny $_{lb}$}‚ स‚होत्प‚न्नं नित्य‚त्वात् । न च व्य‚क्त्युत्पाद एव सामान्य‚स्योत्पादो भिन्न‚त्वात् ।‚{\tiny $_{lb}$}‚ अभिन्न‚त्वे वा त‚तो न सामान्य‚विशेष‚भावः स्यात् । \textbf{न च कुत‚श्चित्} पूर्व‚का‚{\tiny $_{३}$}‚द्‚{\tiny $_{lb}$}‚ व्य‚क्तिविशेषाद्\textbf{आग‚तं} । एत‚न्न \textbf{याती}ति य‚दुक्त‚न्त‚स्यैवोप‚संहार‚द्वारेणोप‚न्यासः ।
	{\color{gray}{\rmlatinfont\textsuperscript{§~\theparCount}}}
	\pend% ending standard par
      ‚{\tiny $_{lb}$}‚

	  
	  \pstart \leavevmode% starting standard par
	याव‚द्भिः प्र‚कारैः सामान्य‚स्य व्य‚क्त्य‚न्त‚रे स‚म्भ‚व‚स्ते प्र‚कारा नेष्य‚न्ते त‚त्र च‚{\tiny $_{lb}$}‚ सामान्य‚मिष्य‚त इति व्याघातः । स च प्राज्ञानान्दुःस‚ह‚त्वाद् भारः । अत एवाह ।‚{\tiny $_{lb}$}‚ \textbf{क इम‚मि}त्यादि । प्राज्ञो हि क‚थ‚म‚युक्तं स‚ह‚ते । ज‚ड‚स्त्व‚ज्ञानाद् युक्तायुक्त‚विचार‚{\tiny $_{lb}$}‚णाक्ष‚मः स‚हेतापि । य‚दाहान्य‚{\tiny $_{४}$}‚त्र \textbf{जाड्या}दिति ।
	{\color{gray}{\rmlatinfont\textsuperscript{§~\theparCount}}}
	\pend% ending standard par
      ‚{\tiny $_{lb}$}‚

	  
	  \pstart \leavevmode% starting standard par
	न‚नु चोत्पित्सुद्र‚व्ये सामान्य‚स्योत्प‚त्ताव‚पि स‚म‚वेत‚त्वं प्र‚तिभासादेवाव‚ग‚न्त‚व्यं [।]‚{\tiny $_{lb}$}‚ स च व्य‚क्तिस‚म‚वेत‚त्व‚प्र‚तिभासोनुत्पादेपि सामान्य‚स्य विद्य‚त एवेति किमुत्पादेन‚{\tiny $_{lb}$}‚ व्य‚क्तिस‚म‚वेत‚ञ्च सामान्य‚स्य रूप‚मिष्य‚ते । तेन त‚त्पूर्व‚द्र‚व्य‚स‚म‚वेत‚म‚पि त‚तोऽविच‚ल‚{\tiny $_{lb}$}‚दुत्पित्सुद्र‚व्य‚स‚म‚वेतं च प्र‚तिभास‚त इति क‚थ\textbf{न्न याती}त्यादि दूष‚णायो‚{\tiny $_{५}$}‚च्य‚तेभीष्ट‚त्वात् ।
	{\color{gray}{\rmlatinfont\textsuperscript{§~\theparCount}}}
	\pend% ending standard par
      ‚{\tiny $_{lb}$}‚

	  
	  \pstart \leavevmode% starting standard par
	स‚त्त्यं । यो हि सामान्य‚स्य प्र‚तिभासं नेच्छ‚ति त‚स्येदं दूष‚णं स्यात् \textbf{प्र‚तिभास‚त}‚{\tiny $_{lb}$}‚ इति । य‚स्तु सामान्य‚प्र‚तिभासोलीक इति म‚न्य‚ते त‚स्य क‚थं दूष‚णं । अलीक‚त्वं‚{\tiny $_{lb}$}‚ चोत्पित्सु द्र‚व्यं न याति न च त‚त्रासीन्न चोत्प‚न्न‚मित्यादिना ग्र‚न्थेन सामान्याभावेपि‚{\tiny $_{lb}$}‚ सामान्याव‚भासिनो ज्ञान‚स्योत्प‚त्तेः प्र‚तिपादित‚मा चा र्ये ण [।] न च प्र‚तिभास‚ना‚{\tiny $_{lb}$}‚देव‚{\tiny $_{६}$}‚ स‚त्य‚त्वं । द्विच‚न्द्रादेर‚पि स‚त्य‚त्व‚प्र‚स‚ङ्गात् ।
	{\color{gray}{\rmlatinfont\textsuperscript{§~\theparCount}}}
	\pend% ending standard par
      ‚{\tiny $_{lb}$}‚

	  
	  \pstart \leavevmode% starting standard par
	नापि प्र‚त्य‚क्ष‚बाधैका बाधा । अनुमान‚बाधाया अपि बाधात्वात् [।] य‚दि‚{\tiny $_{lb}$}‚ त‚दंश‚व‚त् स्यात्त‚दैकेनांशेन पूर्व‚म्भिन्नाधारे स्थित‚मंशान्त‚रेणोत्पित्सु द्र‚व्यं व्याप्नुयात् ।‚{\tiny $_{lb}$}‚ अनंश‚म्वा पूर्व‚माधारं हित्वा । द्व‚य‚म‚प्येत‚न्नास्तीत्याह । \textbf{न चे}त्यादि । \textbf{पूर्व‚माधार}‚{\tiny $_{lb}$}‚मिति सू त्र भागं । \textbf{उत्पित्सुदेशाद् भिन्न‚देश‚मि}ति मिश्र‚के‚{\tiny $_{७}$}‚ण स्प‚ष्ट‚य‚ति । \textbf{त‚योश्चे}ति \leavevmode\ledsidenote{\textenglish{112a/PSVTa}}‚{\tiny $_{lb}$}‚ पूर्व‚प‚श्चादुत्प‚न्न‚योर्द्र‚व्य‚योः ।
	{\color{gray}{\rmlatinfont\textsuperscript{§~\theparCount}}}
	\pend% ending standard par
      ‚{\tiny $_{lb}$}‚

	  
	  \pstart \leavevmode% starting standard par
	\textbf{भिन्ने}त्यादिना व्याच‚ष्टे । \textbf{द्विधा भ‚वे}दिति । नानाव‚य‚वात्म‚त‚या । पूर्वाधार‚{\tiny $_{lb}$}‚‚{\tiny $_{lb}$}‚ \leavevmode\ledsidenote{\textenglish{304/s}}त्यागेन वा । प्र‚थ‚म‚न्ताव‚त् प‚क्ष‚माह । \textbf{नाने}त्यादि । एत‚च्च प‚र‚प्र‚सिद्ध्योच्य‚ते ।‚{\tiny $_{lb}$}‚ न त्वेक‚म‚नेकाव‚य‚वात्म‚क‚मिष्य‚त इत्युक्तं । \textbf{अन्यान्योभ्या}म‚व‚य‚वाभ्याम्प‚र‚स्प‚र‚भिन्ना‚{\tiny $_{lb}$}‚भ्यामंशाख्या‚{\tiny $_{१}$}‚\textbf{न्त‚त्स‚म्ब‚न्धा}त् । ताभ्याम्भिन्न‚देशाभ्यां स‚म्ब‚न्धात् । \textbf{आलोको} हि‚{\tiny $_{lb}$}‚ साव‚य‚व‚त्वाद‚न्येनाव‚य‚वेन घ‚टेन स‚म्ब‚ध्य‚ते । अन्येन घ‚टादिभिः । एवं \textbf{र‚ज्जुवंश}‚{\tiny $_{lb}$}‚द‚ण्डादाव‚पि स्व‚स‚म्ब‚न्धिभिः ।
	{\color{gray}{\rmlatinfont\textsuperscript{§~\theparCount}}}
	\pend% ending standard par
      ‚{\tiny $_{lb}$}‚

	  
	  \pstart \leavevmode% starting standard par
	\textbf{न ही}त्याद्य‚स्यैव स‚म‚र्थ‚नं । अथ साव‚य‚व‚त्वेन सामान्य‚म‚नेक‚वृत्तीष्येत । त‚थापि‚{\tiny $_{lb}$}‚ क‚थ‚मेक‚म‚नेक‚त्र व‚र्त्तेत । य‚स्मादेक‚देशाः सामान्य‚स्य व‚र्त्त‚न्त इति [।] ये च त‚दैक‚{\tiny $_{lb}$}‚दे‚{\tiny $_{२}$}‚शाः सामान्य‚स्य प्र‚त्येक‚म्पिण्डेषु व‚र्त्त‚न्ते । ते किं सामान्यात्म‚का उत नेति [।]‚{\tiny $_{lb}$}‚ य‚दि सामान्यात्म‚का एक‚मेक‚त्र व‚र्त्त‚त इति प्राप्तं । न चैक‚मेक‚त्र व‚र्त्त‚मानं सामान्य‚{\tiny $_{lb}$}‚मिति युक्त‚म्व‚क्तुं । अथ न सामान्यात्म‚कास्ते । क‚थं सामान्य‚म‚नेक‚त्र व‚र्त्त‚त इत्यु‚{\tiny $_{lb}$}‚च्य‚ते । एक‚देशेषु च सामान्य‚स्य य‚द्येक‚देशान्त‚रेण वृत्तिस्त‚दान‚व‚स्था स्यात् । \textbf{न च‚{\tiny $_{lb}$}‚ साव‚य‚व‚त्व‚म‚न्त‚{\tiny $_{३}$}‚रेणैक}स्यानेक‚त्र \textbf{वृत्तिर्युक्ता} । अथान‚व‚य‚वं प्र‚तिपिण्डं प‚रिस‚माप्त्या‚{\tiny $_{lb}$}‚ पिण्ड‚व‚द‚साधार‚ण‚त्वान्न सामान्य‚म्भ‚वितुम‚र्ह‚ति । किं कार‚ण\textbf{न्त‚स्या}न‚व‚य‚व‚स्य सामान्य‚{\tiny $_{lb}$}‚स्यैकेन द्र‚व्येण स‚म्ब‚न्धो य आत्मा । त‚द्व्य‚तिरेकेण \textbf{दितीयात्माभावात् । एकात्म‚न‚श्च}‚{\tiny $_{lb}$}‚ त‚स्य सामान्य‚स्य \textbf{त‚त्प्र‚देश‚व‚र्त्तिस‚म्ब‚न्ध‚रूप‚त्वात्} । उत्पित्सुघ‚ट‚देशात् । पूर्व‚देश‚व‚र्त्ति‚{\tiny $_{lb}$}‚ य‚द् घ‚ट‚द्र‚व्य‚{\tiny $_{४}$}‚न्त‚त्स‚म्ब‚न्धिरूप‚त्वात् । नास्ति भिन्न‚देशेन युग‚प‚द्योगः । \textbf{अन्य‚थे}त्युत्पित्सु‚{\tiny $_{lb}$}‚देश‚द्र‚व्य‚स‚म्ब‚न्ध‚रूप‚त्वे \textbf{त‚त्स‚म्ब‚धायोगात्} । तेन पूर्व‚द्र‚व्येण स‚म्ब‚न्धायोगात् । त‚स्मादे‚{\tiny $_{lb}$}‚क‚व्य‚क्तिनिय‚तात्म‚नः सामान्य‚स्य नास्ति \textbf{त‚स्मिन्नेव काले} भिन्न‚देशेन द्र‚व्येण स‚म्ब‚न्धः ।‚{\tiny $_{lb}$}‚ स‚म्ब‚न्धे वा पूर्व‚व्य‚क्तिनिय‚तैकात्म‚क‚त्वेन सामान्य‚स्य पूर्व‚व्य‚क्तौ स्थितिस्त‚स्मिन्ने‚{\tiny $_{५}$}‚व‚{\tiny $_{lb}$}‚ काले भिन्न‚देश‚व्य‚क्तिस‚म्ब‚न्धेनास्थितिरेत‚च्च‚विरुद्ध‚मित्याह । \textbf{एक‚स्याधेय}स्येत्यादि ।‚{\tiny $_{lb}$}‚ \textbf{त‚त्र स्थान‚मि}ति पूर्व‚व्य‚क्तौ । \textbf{त‚दैव} त‚स्मिन्नेव काले व्य‚क्त्य‚न्त‚रे त्व‚याभ्य‚प‚ग‚मात् ।‚{\tiny $_{lb}$}‚ पूर्व‚व्य‚क्तित्याग‚म‚न्त‚रेणैक‚स्य चान्य‚त्रान्व‚यायोगात् । त‚त्र पूर्व‚स्यां व्य‚क्तौ । \textbf{तेनैव}‚{\tiny $_{lb}$}‚ पूर्व‚व्य‚क्तिनिय‚तेनात्म‚ना । त‚स्य सामान्य‚स्यास्थान‚मित्य\textbf{युक्त‚मे}त‚त् । किङ्कार‚णं ।‚{\tiny $_{६}$}‚‚{\tiny $_{lb}$}‚ \textbf{त‚त्स्थिते}त्यादि । त‚स्यामेव व्य‚क्तौ \textbf{स्थितास्थितात्म}नोः स्व‚भाव‚यो\textbf{रेक‚स्य} सामान्य‚स्य‚{\tiny $_{lb}$}‚ \textbf{युग‚प‚द् विरोधा}त् ।
	{\color{gray}{\rmlatinfont\textsuperscript{§~\theparCount}}}
	\pend% ending standard par
      ‚{\tiny $_{lb}$}‚‚{\tiny $_{lb}$}‚\textsuperscript{\textenglish{305/s}}

	  
	  \pstart \leavevmode% starting standard par
	न‚नु य‚थैक‚त्यागेनाप‚र‚त्र वृत्तिरेव‚म‚प‚राप‚राव‚य‚वैर्ब‚हुषु च वृत्तिर्दृष्ट‚त्वादिति‚{\tiny $_{lb}$}‚ द्विविधाभ्युप‚ग‚म्य‚ते । त‚था सामान्यं य‚दा येनैव रूपेणैक‚त्र वृत्त‚न्त‚दैव तेनैव रूपेणान्य‚त्र‚{\tiny $_{lb}$}‚ व‚र्त्त‚ते दृष्ट‚त्वादिति सापि तृतीया वृत्तिः‚{\tiny $_{७}$}‚ किन्नाभ्युप‚ग‚म्य‚ते । त‚दुक्त‚म्भ ट्टेन ॥
	{\color{gray}{\rmlatinfont\textsuperscript{§~\theparCount}}}
	\pend% ending standard par
      \textsuperscript{\textenglish{112b/PSVTa}}‚{\tiny $_{lb}$}‚
	  \bigskip
	  \begingroup
	
	    
	    \stanza[\smallbreak]
	  {\normalfontlatin\large ``\qquad}न हि द्वैविध्य‚मेवेति वृत्तेर‚स्ति नियाम‚कं ।&‚{\tiny $_{lb}$}‚त्रिविधापि हि दृष्ट‚त्वात् स‚म्भ‚वेद् द्विविधा य‚थेति ।\edtext{\textsuperscript{*}}{\edlabel{pvsvt_305-1}\label{pvsvt_305-1}\lemma{*}\Bfootnote{\href{http://sarit.indology.info/?cref=\%C5\%9Bv}{ Ślokavārtika. }}}{\normalfontlatin\large\qquad{}"}\&[\smallbreak]
	  
	  
	  
	  \endgroup
	‚{\tiny $_{lb}$}‚

	  
	  \pstart \leavevmode% starting standard par
	\hphantom{.}उ द्यो त क रो प्याह । न गोत्व‚म‚व‚य‚वी न च स‚मुदाय‚स्त‚स्मान्न त‚त्र कृत्स्नैक‚{\tiny $_{lb}$}‚देश‚श‚ब्दौ स्तः । न चेत् त‚त्रैतौ श‚ब्दौ स्तः त‚स्माद् गोत्वं किं कृत्स्न‚म्व‚र्त्त‚ते उतैक‚देशे‚{\tiny $_{lb}$}‚नेति न युक्तः प्र‚श्नः । क‚थ‚न्त‚र्हि गोत्वं गोषु व‚र्त्त‚ते । आश्र‚याश्र‚यिभावेन‚{\tiny $_{१}$}‚ । कः‚{\tiny $_{lb}$}‚ पुन‚राश्र‚याश्र‚यिभावः [।] स‚म‚वायः । त‚त्र वृत्तिम‚द् गोत्वं । व‚त्तिः स‚म‚वाय इह‚{\tiny $_{lb}$}‚ प्र‚त्य‚य‚हेतुत्वात्तेन स‚र्व‚त्र पूर्व‚द्र‚व्य उत्पित्सुद्र‚व्ये च स‚म‚वाय एव वृत्तिर‚तः क‚थ‚मुच्य‚ते [।]‚{\tiny $_{lb}$}‚ स्थितास्थितात्म‚नोरेक‚त्र विरोधाद‚युक्त‚मेत‚दि ति ।\edtext{\textsuperscript{*}}{\edlabel{pvsvt_305-2}\label{pvsvt_305-2}\lemma{*}\Bfootnote{\href{http://sarit.indology.info/?cref=nv}{ Nyāyavārtika. }}}
	{\color{gray}{\rmlatinfont\textsuperscript{§~\theparCount}}}
	\pend% ending standard par
      ‚{\tiny $_{lb}$}‚

	  
	  \pstart \leavevmode% starting standard par
	एत‚देवाह । स‚र्व‚त्रेत्यादि । \textbf{स‚र्व‚त्र} पूर्व‚व्य‚क्तावुत्पित्सुद्र‚व्ये च । \textbf{स‚र्व‚दे}वोत्य‚त्सु‚{\tiny $_{lb}$}‚द्र‚व्योत्पादेपि य‚दा व‚र्त्त‚ते त‚दापि द्र‚व्य‚{\tiny $_{२}$}‚न्न \textbf{ज‚हाति} । तेन स्थितास्थितात्म‚नोर्नैक‚त्र‚{\tiny $_{lb}$}‚ विरोधो स्थितात्म‚नोऽभावादिति । त‚द‚युक्तं । न ह्येक‚स‚म‚वेत‚त्व‚मेवान्य‚व्य‚क्तिस‚म‚{\tiny $_{lb}$}‚वेत‚त्व‚म‚न्य‚स्यास्त‚त्र प्र‚तिभास‚न‚प्र‚स‚ङ्गात् [।] त‚स्मादेक‚स‚म‚वेत‚त्वान्य‚स‚म‚वेत‚त्व‚योः‚{\tiny $_{lb}$}‚ प‚र‚स्प‚रं भेद एव । त‚च्चाभिन्नं सामान्यादेक‚स‚म‚वेत‚त्वाद‚न‚नुग‚म‚व‚द‚न्य‚त्र सामान्य‚स्या‚{\tiny $_{lb}$}‚प्य‚न‚नुग‚म‚प्र‚स‚ङ्गः । य‚द्वैक‚व्य‚{\tiny $_{३}$}‚क्तिकालादिस‚म्ब‚न्धेन ज्ञान‚ज‚न‚न‚श‚क्तिर्यासाम‚न्य‚स्य । न‚{\tiny $_{lb}$}‚ साऽन्य‚व्य‚क्त्यादिस‚म्ब‚न्ध‚त्वेन । तेनैक‚स्यां व्य‚क्तौ सामान्य‚स्य ज्ञान‚ज‚न‚न‚श‚क्तिर‚{\tiny $_{lb}$}‚न्य‚स्यां ज्ञान‚ज‚न‚न‚श‚क्तिविरोधिनी । श‚क्तिश्च श‚क्तिम‚तोऽभिन्ना । श‚क्तिल‚क्ष‚ण‚{\tiny $_{lb}$}‚त्वाच्च व‚स्तुनः । तेन य‚द्व‚स्त्वेक‚न्त‚देक‚वृत्त्येवेति व्याप्तिसिद्धिः । व‚स्तु चैकं सामान्यं‚{\tiny $_{lb}$}‚ य‚दि क‚थ‚म‚न्य‚त्रापि व‚र्त्तेत । त‚थाभूत‚{\tiny $_{४}$}‚स्य प्र‚तिभासादिति चेत् [।] न । प्र‚तिभासो‚{\tiny $_{lb}$}‚ ह्य‚प्र‚तिभास‚स्य बाध‚को नाव‚स्तुन‚स्त‚स्यापि प्र‚तिभास‚नात् । अस्य तु व‚स्तुप्र‚तिभासो‚{\tiny $_{lb}$}‚ बाध‚को न चानुग‚तं व‚स्त्व‚स्तीत्युक्तं । अत एव न प्र‚तिज्ञायाः प्र‚त्य‚क्ष‚बाधा । सामान्य‚{\tiny $_{lb}$}‚ज्ञान‚स्य प्र‚त्य‚क्ष‚त्वाभासा\edtext{}{\lemma{त्वाभासा}\Bfootnote{?}}च्च ।
	{\color{gray}{\rmlatinfont\textsuperscript{§~\theparCount}}}
	\pend% ending standard par
      ‚{\tiny $_{lb}$}‚

	  
	  \pstart \leavevmode% starting standard par
	न‚नु याव‚द‚स्याप्रामाण्यं न ताव‚द‚नुमान‚स्य प्र‚बृत्तिर्याव‚च्च नानुमान‚स्य प्र‚वृत्ति‚{\tiny $_{lb}$}‚स्ताव‚न्नास्य‚{\tiny $_{५}$}‚ प्र‚त्य‚क्षाभास‚तेत्य‚न्योन्याश्र‚य‚त्वं स्यादिति चेत् [।] न । य‚तोनुमानं प्र‚ति‚{\tiny $_{lb}$}‚भास‚मान‚स्य व‚स्तुत्व‚स‚न्देह‚मात्रेणैव प्र‚व‚र्त्त‚ते । नाप्य‚स्याप्रामाण्य‚निमित्त‚म‚नुमान‚म्प्र‚{\tiny $_{lb}$}‚‚{\tiny $_{lb}$}‚ ‚{\tiny $_{lb}$}‚ ‚{\tiny $_{lb}$}‚ \leavevmode\ledsidenote{\textenglish{306/s}}व‚र्त्त‚तेपि तु स्व‚साध्य‚प्र‚तिब‚द्ध‚लिङ्ग‚निमित्त‚म् [।] अतः सामान्य‚ज्ञान‚स्य बाध‚क‚न्त‚स्मा‚{\tiny $_{lb}$}‚न्नास्ति प‚र‚मार्थ‚त एक‚म्व‚स्त्वेक‚दाऽनेक‚वृत्तिः । वृत्तौ तु त‚त्स्थितास्थितात्म‚नोर्विरोध‚{\tiny $_{lb}$}‚ एव ।‚{\tiny $_{६}$}‚ आ चा र्य स्त्व‚भ्युप‚ग‚म्यापि दोष‚माह । \textbf{त‚त्स्व‚भावेत्}यादि । सामान्य‚{\tiny $_{lb}$}‚स्व‚भाव‚स्य \textbf{द‚र्श‚न‚माश्र‚यो} य‚स्य प्र‚त्य‚य‚स्य स \textbf{स‚र्व‚त्र} भिन्न‚जातीयेपि द्र‚व्ये \textbf{स‚र्वाकार‚{\tiny $_{lb}$}‚स्यात् । त‚था च स‚ति गाम‚प्य‚श्व} इत्यादि । किङ्कार‚ण‚म् [।] इत्याह । \textbf{अश्वे‚{\tiny $_{lb}$}‚ \leavevmode\ledsidenote{\textenglish{113a/PSVTa}} स्थित} आत्मा य‚स्य द्र‚व्य‚त्व‚स्येति विग्र‚हः । ग‚म‚क‚त्वाद् व्य‚धिक‚र‚ण‚स्यापि‚{\tiny $_{७}$}‚ ब‚हुब्रीहिः\edtext{}{\edlabel{pvsvt_306-1}\label{pvsvt_306-1}\lemma{हुब्रीहिः}\Bfootnote{\href{http://sarit.indology.info/?cref=p\%C4\%81.2.2}{ Pāṇini 2.2 }}} ।‚{\tiny $_{lb}$}‚ अश्वे स्थित इति वा \textbf{साध‚नं कृते}ति स‚मासः । प‚श्चादात्म‚श‚ब्देन द्विप‚दो ब‚हुब्रीहिः ।‚{\tiny $_{lb}$}‚ \textbf{त‚त्स्व‚भाव‚प्र‚तिप‚त्त्या} चाश्व‚स्थित‚स्व‚भाव‚द्र‚व्य‚त्व‚प्र‚तिप‚त्त्या च \textbf{त‚था निश्च‚याद्}‚{\tiny $_{lb}$}‚ गौर्द्र‚व्य‚मिति निश्च‚यात् ।
	{\color{gray}{\rmlatinfont\textsuperscript{§~\theparCount}}}
	\pend% ending standard par
      ‚{\tiny $_{lb}$}‚

	  
	  \pstart \leavevmode% starting standard par
	स्यादेत‚त् [।] नाश्व‚स‚म‚वेत‚द्र‚व्य‚त्व‚प्र‚तिप‚त्त्या द्र‚व्य‚मिति प्र‚तीतिः [।] कि‚{\tiny $_{lb}$}‚न्त‚र्हि [।] द्र‚व्य‚त्व‚मात्र‚प्र‚तिप‚त्त्येत्य‚त आह । \textbf{त‚स्य चे}त्यादि । \textbf{त‚स्य} चाश्वे द्र‚व्य‚त्व\textbf{स्यै‚{\tiny $_{lb}$}‚क‚{\tiny $_{१}$}‚स्यादृ}ष्ट‚स्याप्र‚तिप‚न्न‚स्याश्व‚स‚म‚वेत‚त्व‚व्य‚तिरेके\textbf{णाकारान्त‚र}स्याश्वासःस म‚वेत‚त्व‚{\tiny $_{lb}$}‚ल‚क्ष‚ण‚स्या\textbf{भावात्} । त‚स्माद‚श्व‚स‚म‚वेतेनैव द्र‚व्य‚त्वेन विशिष्टां गां द्र‚व्य‚मिति प्र‚तिप‚द्य‚{\tiny $_{lb}$}‚मानोश्व इति प्र‚तीयात् । य‚स्य त्व‚श्व‚व्य‚तिरिक्त‚मेव द्र‚व्य‚त्व‚सामान्य‚न्तेन च विशि‚{\tiny $_{lb}$}‚ष्ट‚म‚सौ गां प्र‚तिप‚द्य‚मानो निय‚मेनाश्व इति प्र‚तीयात् ।
	{\color{gray}{\rmlatinfont\textsuperscript{§~\theparCount}}}
	\pend% ending standard par
      ‚{\tiny $_{lb}$}‚

	  
	  \pstart \leavevmode% starting standard par
	\textbf{त‚स्मादि}त्युप‚संहारः । अन‚व‚य‚{\tiny $_{२}$}‚वं सामान्य\textbf{म‚नेक‚देशे}ऽनेको देशोऽस्येति त‚स्मिन्‚{\tiny $_{lb}$}‚ घ‚टादौ \textbf{युग‚प‚न्नाधीय‚ते} । नाधेय‚तां प्र‚तिप‚द्य‚ते । इय‚ता च न \textbf{चांश‚व‚दि}त्येत‚द्‚{\tiny $_{lb}$}‚ व्याख्यातं ।
	{\color{gray}{\rmlatinfont\textsuperscript{§~\theparCount}}}
	\pend% ending standard par
      ‚{\tiny $_{lb}$}‚

	  
	  \pstart \leavevmode% starting standard par
	\textbf{ज‚हाति पूर्व‚न्नाधार}मित्येत‚त् पूर्वेत्यादिना व्याच‚ष्टे । स चेति पूर्वाधार‚त्यागः‚{\tiny $_{lb}$}‚ सामान्य‚स्य \textbf{नाभिम‚तः} । \href{http://sarit.indology.info/?cref=pv.3.151-3.152}{१५४-५५}
	{\color{gray}{\rmlatinfont\textsuperscript{§~\theparCount}}}
	\pend% ending standard par
      ‚{\tiny $_{lb}$}‚

	  
	  \pstart \leavevmode% starting standard par
	\textbf{अन्य‚त्रेति} पूर्व‚व्य‚क्तौ \textbf{व‚र्त्त‚मान‚स्य} सामान्य‚स्य \textbf{स्व‚स्मात्} पूर्वाधार‚देशाद् \textbf{अच‚ल}‚{\tiny $_{lb}$}‚त‚स्त‚तः पूर्वाधार‚{\tiny $_{३}$}‚देशाद‚न्य‚त्र स्थाने ज‚न्म य‚स्य द्र‚व्य‚स्य त‚स्मिन् वृत्तिरित्य‚तियुक्ति‚{\tiny $_{lb}$}‚‚{\tiny $_{lb}$}‚ ‚{\tiny $_{lb}$}‚ \leavevmode\ledsidenote{\textenglish{307/s}}म‚दित्युप‚ह‚स‚ति \href{http://sarit.indology.info/?cref=pv.3.153}{१५६}
	{\color{gray}{\rmlatinfont\textsuperscript{§~\theparCount}}}
	\pend% ending standard par
      ‚{\tiny $_{lb}$}‚

	  
	  \pstart \leavevmode% starting standard par
	पूर्व‚व्य‚क्तिदेशाद‚विच‚ल‚द‚पि सामान्य‚न्त‚तोन्य‚देश‚न्द्र‚व्यं व्याप्नोतीति चेदाह ।‚{\tiny $_{lb}$}‚ य‚त्रेत्यादि । य‚त्र देशेऽसौ प‚श्चात्काल‚भावी भावो व‚र्त्त‚ते । तेन देशेन सामान्यं न‚{\tiny $_{lb}$}‚ स‚म्ब‚ध्य‚ते स्व‚व्य‚क्तिस‚र्व‚ग‚त‚त्वाभ्युप‚ग‚मात् । य‚त्र देशे सामान्यं न व‚र्त्त‚ते त‚द्देशिनं च‚{\tiny $_{lb}$}‚ प‚श्चात् काल‚{\tiny $_{४}$}‚भाविन‚म्भावं व्याप्नोतीति न्यायातिक्रान्त‚त्वात् किम‚प्येत‚न्म‚हाद्भुत‚{\tiny $_{lb}$}‚मिति प्र‚कारान्त‚रेणोप‚ह‚स‚ति । न हि यो य‚त्र देशे न व‚र्त्त‚ते स त‚द्देशं व्याप्नोतीति‚{\tiny $_{lb}$}‚ न्यायानुसारिणा श‚क्य‚म‚व‚सातुं \href{http://sarit.indology.info/?cref=pv.3.154}{१५७}
	{\color{gray}{\rmlatinfont\textsuperscript{§~\theparCount}}}
	\pend% ending standard par
      ‚{\tiny $_{lb}$}‚

	  
	  \pstart \leavevmode% starting standard par
	स‚र्व‚ग‚त‚त्व‚क‚ल्प‚नाम‚पि निराचिकीर्ष‚न्नाह । \textbf{य‚स्ये}त्यादि । त‚स्यापि स‚र्व‚ग‚त‚{\tiny $_{lb}$}‚सामान्य‚वादिनः स‚र्व‚त्र‚गा य‚दि जातिस्त‚दैक‚त्र शाव‚लेयादौ या त‚स्याव्य‚क्तिर‚भि‚{\tiny $_{lb}$}‚व्य‚क्तिस्त‚{\tiny $_{५}$}‚या क‚र‚ण‚भूत‚या । सा जातिस्स‚र्व‚त्र व्य‚क्तिशून्येपि देशे । विजातीय‚{\tiny $_{lb}$}‚व्य‚क्तिभेदे च \textbf{व्य‚क्तैव} प्र‚काशितै\textbf{वाभेदादेक}त्वात् स‚र्व‚त्र व्य‚क्तिशून्येपि देशे दृश्येत ।
	{\color{gray}{\rmlatinfont\textsuperscript{§~\theparCount}}}
	\pend% ending standard par
      ‚{\tiny $_{lb}$}‚

	  
	  \pstart \leavevmode% starting standard par
	एत‚दुक्त‚म्भ‚व‚ति । य‚द्य‚पि व्य‚क्तिशून्ये प्र‚देशे विजातीय‚व्य‚क्तौ च स्व‚व्य‚क्तेर्व्य‚{\tiny $_{lb}$}‚ञ्जिकाया अभाव‚स्त‚थापि स्व‚व्य‚क्त्य‚भिव्य‚क्तेनैव रूपेण त‚त्राव‚स्थानाज्जातेरुप‚ल‚म्भः‚{\tiny $_{lb}$}‚ स्यान्नो चेत् स्व‚भाव‚नानात्वं‚{\tiny $_{६}$}‚ प्राप्नोतीत्येक‚रूपा चेष्य‚त इति भावः । \textbf{न जा}तेर्नित्याया‚{\tiny $_{lb}$}‚ अनाधेयातिश‚य‚त्वेन \textbf{क‚दाचिद‚भिव्य‚क्तिरिति निषिद्ध‚मेत‚त्} ।वृत्तिराधेय‚ता व्य‚क्ति‚{\tiny $_{lb}$}‚रिति त‚स्मिन्न युज्य‚त \href{http://sarit.indology.info/?cref=pv.3.143}{१ । १४६} इत्य‚त्रान्त‚रे । य‚त एव\textbf{न्त‚स्मात्} सा जाति\textbf{र्नित्य‚{\tiny $_{lb}$}‚म‚न‚पेक्षित‚प‚रोप‚स्कारा}नाधेयातिश‚या । एव‚म्भूता य‚दि स्व‚भावेन स्व‚विज्ञान‚ज‚न‚न‚{\tiny $_{lb}$}‚योग्या । त‚दा नित्य\textbf{न्दृश्येत}‚{\tiny $_{७}$}‚ व्य‚क्तेः प्राक् प‚श्चाच्च । अथ न योग्या त‚दा क‚दाचिद् \leavevmode\ledsidenote{\textenglish{113b/PSVTa}}‚{\tiny $_{lb}$}‚ द‚श्येत । किं कार‚णं [।] \textbf{त‚स्मिन्} विज्ञान‚ज‚न‚न‚योग्य\textbf{स्व‚भावे} त‚द्विप‚रीते \textbf{चाव‚स्थानात्} ।‚{\tiny $_{lb}$}‚ स‚र्व‚काल‚मेक‚रूप‚त्वादित्य‚र्थः । अस‚म‚र्था व्य‚क्त्य‚स‚न्निधाने त‚त्स‚न्निधाने तु स‚म‚र्था भ‚व‚ति ।‚{\tiny $_{lb}$}‚ ‚{\tiny $_{lb}$}‚ \leavevmode\ledsidenote{\textenglish{308/s}}त‚तो न नित्य‚न्द‚र्श‚न‚म‚द‚र्श‚न‚म्वा जातेरित्य‚त आह । \textbf{स्व‚भावे} त्यादि । नित्य‚त्वेनाना‚{\tiny $_{lb}$}‚धेयातिश‚य‚त्वादिति भावः ॥ \href{http://sarit.indology.info/?cref=pv.3.154-3.155}{१५७-५८}
	{\color{gray}{\rmlatinfont\textsuperscript{§~\theparCount}}}
	\pend% ending standard par
      ‚{\tiny $_{lb}$}‚

	  
	  \pstart \leavevmode% starting standard par
	य‚दि जातेर्नास्ति व्य‚क्ति‚{\tiny $_{१}$}‚स्त‚त्किं व्य‚क्त्यैवैक‚त्र सा व्य‚क्तेत्याद्युच्य‚त इत्य‚त‚{\tiny $_{lb}$}‚ आह । \textbf{अभ्युप‚ग‚म्या}पीत्यादि । \textbf{व्यापि}न्य‚पि जातिः । \textbf{एक‚त्रा}श्र‚ये व्य‚क्ता \textbf{भेदाभावा‚{\tiny $_{lb}$}‚दे}क‚त्वाज्जाते\textbf{र्व्य‚क्तैव} प्र‚काशितैव \textbf{स‚र्व‚त्र व्य‚क्तिशून्ये दे}शे । विजातीये च व्य‚क्तिभेदे ।‚{\tiny $_{lb}$}‚ \textbf{व्य‚क्तिशून्येष्व‚पीत्य‚पि} श‚ब्दाद् विजातीयेपि व्य‚क्तिभेदे । अपि च \textbf{न च सा} जाति‚{\tiny $_{lb}$}‚\textbf{र्व्य‚क्त्य‚पेक्षिणी} । व्य‚ञ्जिका व्य‚क्तिर्न्नापेक्ष्येत । व्य‚क्तेर्जातिव्य‚ञ्ज‚क‚त्वा‚{\tiny $_{२}$}‚भावा‚{\tiny $_{lb}$}‚दिति भावः । य‚दि हि व्य‚ञ्जिकां व्य‚क्तिम‚पेक्षेत । त‚दा \textbf{व्य‚ञ्ज‚काप्र‚तिप‚त्तौ न‚{\tiny $_{lb}$}‚ व्य‚ङ्ग्य‚स्य प्र‚ती}तिः स्यान्न हि प्र‚दीपाद्य‚प्र‚तीतौ घ‚टादेः प्र‚तीतिर्भ‚व‚ति । त‚थेहापि‚{\tiny $_{lb}$}‚ व्य‚क्त्य‚प्र‚तीतौ न जातिप्र‚तीतिः स्यात् । \textbf{सामान्य‚त‚द्व‚तो}स्तु व्य‚ङ्ग्य‚व्य‚ञ्ज‚क‚यो\textbf{र्विप‚{\tiny $_{lb}$}‚र्य‚यः पुनः क‚स्मादिष्टः} । त‚था हि नागृहीत‚विशेष‚णाविशेष्ये बुद्धिर्व‚र्त्त‚त इति निय‚{\tiny $_{lb}$}‚मात् ।‚{\tiny $_{३}$}‚ प्रागेव सामान्य‚ग्र‚ह‚ण‚मिष्ट‚न्त‚द्द्वारेण तु व्य‚क्तेः । त‚तो व्य‚ञ्जिकाया‚{\tiny $_{lb}$}‚ व्य‚क्तेर्ग्र‚ह‚ण‚म‚न्त‚रेणापि व्य‚ङ्ग्याभिम‚त‚स्य सामान्य‚स्य प्र‚तिप‚त्तिरिष्टेति विप‚र्य‚यः ।
	{\color{gray}{\rmlatinfont\textsuperscript{§~\theparCount}}}
	\pend% ending standard par
      ‚{\tiny $_{lb}$}‚

	  
	  \pstart \leavevmode% starting standard par
	\textbf{यो ही}त्यादिना व्याच‚ष्टे । \textbf{स्वाश्र}यो य‚त्र स‚म‚वेतं सामान्यं । सामान्य‚ग्राह‚क‚{\tiny $_{lb}$}‚\textbf{मिन्द्रियं} च त‚योस्\textbf{संयोग}स्त‚द\textbf{पेक्षा} प्र‚तीतिर्य‚स्य \textbf{सामान्य‚स्य} त‚त्त‚थोक्तं । \textbf{आश्र‚य‚शून्याः}‚{\tiny $_{lb}$}‚ प्र‚देशा विजातीय‚{\tiny $_{४}$}‚व्य‚क्त्य‚ध्यासिता व्य‚क्तिशून्याश्च । तेषु \textbf{न दृश्य‚ते} । य‚थोक्त‚{\tiny $_{lb}$}‚संयोगाभावात् । त‚स्याप्येवं वादिनः । क्व‚चिद् व्य‚क्तिर्द‚र्श‚ने स‚त्य‚स्त्येव्\textbf{आश्र‚येन्द्रिय‚{\tiny $_{lb}$}‚संयोगो} जातेः स‚र्व‚त्र स्थिताया उ\textbf{प‚कार}क इति । त‚त आश्र‚येन्द्रिय‚संयोगाद्धेतो\textbf{स्त‚द्द‚र्शी}‚{\tiny $_{lb}$}‚ क्व‚चिद् व्य‚क्तिभेदे जातिद‚र्शी \textbf{य‚थास्थितां} स‚र्व‚देश‚व्यापिनीञ्जाति\textbf{म्प‚श्येत्} ।‚{\tiny $_{lb}$}‚ य‚त्रैव व्य‚ञ्जिका व्य‚क्तिस्त‚त्रैव जातेः‚{\tiny $_{५}$}‚ स्व‚रूपं दृश्यं नान्य‚त्रेति चेदाह । \textbf{न ही}त्यादि ।‚{\tiny $_{lb}$}‚ \textbf{त‚स्यामिति} जातौ । क्व‚चिद् व्य‚क्तौ \textbf{दृश्य‚मानाया}न्त‚दीयामिति सामान्य‚स‚म्ब‚न्धि ।‚{\tiny $_{lb}$}‚ एक‚स्य दृष्टादृष्ट‚विरोधात् । एव‚न्ताव‚द् व्य‚क्तेर्व्य‚ञ्जिकात्व‚म‚भ्युप‚ग‚म्य च श‚ब्दो‚{\tiny $_{lb}$}‚पात्तोर्थो व्याख्यातः ।
	{\color{gray}{\rmlatinfont\textsuperscript{§~\theparCount}}}
	\pend% ending standard par
      ‚{\tiny $_{lb}$}‚‚{\tiny $_{lb}$}‚\textsuperscript{\textenglish{309/s}}

	  
	  \pstart \leavevmode% starting standard par
	अधुना न सा व्य‚क्त्य‚पेक्षिणीत्यादि व्याख्यातुमाह । \textbf{व्य‚क्तिव्य‚ङ्ग्य}त्वादित्यादि ।‚{\tiny $_{lb}$}‚ त‚स्य च मिथ्यात्व‚म‚न‚न्त‚रोक्तेनैव‚{\tiny $_{६}$}‚ प्र‚तिपादितं । न ह्येक‚स्य दृष्टादृष्ट‚म‚स्त्य‚तोऽ‚{\tiny $_{lb}$}‚पूर्व‚प‚क्ष एवायं केव‚ल‚न्दोषान्त‚राभिधानार्थं ग‚ज‚निमील‚नं कृत्वोप‚न्य‚स्तं । त‚थाभूत‚स्येति‚{\tiny $_{lb}$}‚ न्याय्य‚स्य । \textbf{त‚त्रेति} जातित‚द्व‚ति । न्याय्य‚स्य \textbf{व्य‚ङ्ग्य‚व्य‚ञ्ज‚क‚भाव‚स्याभावात्} ।‚{\tiny $_{lb}$}‚ किङ्कार‚णं । \textbf{स्वेत्या}दि । हि य‚स्मात् । \textbf{स्व‚रूप‚शून्ये} देशे प्र‚दीपादिर‚हिते देशे ।‚{\tiny $_{lb}$}‚ स्व‚व्य‚ङ्ग्यं घ‚टादिकं । \textbf{नैवं} य‚थो‚{\tiny $_{७}$}‚\textbf{क्ते}न न्यायेन \textbf{व्य‚क्ति}र्व्य‚ञ्जिका \textbf{सामान्य‚स्य} । \leavevmode\ledsidenote{\textenglish{114a/PSVTa}}‚{\tiny $_{lb}$}‚ किङ्कार‚ण‚म् [।] विप‚र्य‚यात् । य‚स्माद‚गृहीत्वापि व्य‚क्तिं सामान्य‚मादौ गृह्य‚त‚{\tiny $_{lb}$}‚ इतीष्य‚ते प‚रेण । सामान्य‚ग्र‚ह‚ण‚द्वारेणैव व्य‚क्तेर्ग्र‚ह‚णाभ्युप‚ग‚माद‚त‚श्च व्य‚ञ्ज‚का‚{\tiny $_{lb}$}‚प्र‚तिप‚त्त्यापि व्य‚ङ्ग्य‚स्य ग्र‚ह‚णात् । व्य‚ञ्ज‚क‚ध‚र्मातिक्र‚मो व्य‚क्तेः ।
	{\color{gray}{\rmlatinfont\textsuperscript{§~\theparCount}}}
	\pend% ending standard par
      ‚{\tiny $_{lb}$}‚

	  
	  \pstart \leavevmode% starting standard par
	एत‚देवाह । \textbf{क‚थं} हीत्यादि । सेति व्य‚क्तिः सामान्य‚स्य व्य‚ञ्जिका च स्या‚{\tiny $_{lb}$}‚दिति \textbf{स‚{\tiny $_{१}$}‚म्ब‚न्धः । त‚त्प्र‚तिप‚त्तिद्वारेणेति} सामान्य‚प्र‚तिप‚त्तिद्वारेण । सा व्य‚क्तिर्दृश्या‚{\tiny $_{lb}$}‚ स्यादिति विरुद्ध‚मेत‚त् । \textbf{एव}मिति सामान्य‚द‚र्श‚न‚ब‚लेन । व्य‚क्तेर्द‚र्श‚नेभ्युप‚ग‚म्य‚माने ।‚{\tiny $_{lb}$}‚ \textbf{व्य‚ङ्ग्या} सा व्य‚क्तिः \textbf{प्र‚स‚ज्य}ते सामान्य‚ञ्चेत्य‚ध्याहारः । \textbf{प्र‚दीपेन घ‚ट‚व‚दि}ति ।‚{\tiny $_{lb}$}‚ तृतीयेति योग‚विभागात् स‚मासः । सुप्सुपेति वा स‚मासः । य‚था प्र‚दीपेन घ‚टो‚{\tiny $_{lb}$}‚ व्य‚ङ्ग्य‚स्त‚द्व‚त्सा‚{\tiny $_{२}$}‚मान्येन व्य‚क्तिर्व्य‚ङ्ग्या प्राप्तेर्थः । त‚त्\textbf{प्र‚तिप‚त्तिम‚न्त‚रे}ण सामान्य‚{\tiny $_{lb}$}‚प्र‚तिप‚त्तिम्विना व्य‚क्ते\textbf{र‚दृश्य‚रूप‚त्वात् । अन्ये त्वाहुः} । व्य‚ङ्ग्या च सैवं प्र‚स‚ज्य‚त‚{\tiny $_{lb}$}‚ इत्य‚त्र च‚श‚ब्देन सामान्य‚व्य‚ञ्ज‚क‚मित्येत‚दुपात्तं । त‚तः प्र‚दीप‚घ‚टाभ्यां तुल्य‚मिति‚{\tiny $_{lb}$}‚ द्व‚न्द्वादेव व‚तिर्द्र‚ष्ट‚व्यः । पूर्व‚निपात‚ल‚क्ष‚ण‚स्य व्य‚भिचारित्वाद् घ‚ट‚श‚ब्द‚स्यापूर्व‚निपातः ।‚{\tiny $_{lb}$}‚ प्र‚दीप‚व‚{\tiny $_{३}$}‚त् सामान्यं व्य‚ञ्ज‚कं । घ‚ट‚व‚च्च व्य‚क्तिर्व्य‚ङ्ग्या । \textbf{प्र‚स‚ज्य‚त} इति वाक्यार्थ‚{\tiny $_{lb}$}‚ इति । अनेनेति \textbf{सामान्य‚वादि}ना । सामान्य‚म्विना \textbf{कि}म‚स‚म्भ‚व‚त्कार्य‚म\textbf{भिस‚मीक्ष्य} ।‚{\tiny $_{lb}$}‚ एव‚मित्युक्त‚विधिना । \textbf{ब‚ह्वायासः} । अश‚क्य‚साध‚न‚त‚या ब‚हुदुःख‚हेतुः । \href{http://sarit.indology.info/?cref=pv.3.155-3.156}{१५८-५९}
	{\color{gray}{\rmlatinfont\textsuperscript{§~\theparCount}}}
	\pend% ending standard par
      ‚{\tiny $_{lb}$}‚

	  
	  \pstart \leavevmode% starting standard par
	\textbf{प‚र‚स्प‚रे}त्यादि प‚रः । \textbf{भेदा}द्विल‚क्ष‚ण‚त्वाद्धेतोर्व्य‚तिरेकिणीष्व‚न‚न्व‚यिनीषु । \textbf{अन्व‚{\tiny $_{lb}$}‚यिन} एकाकार‚स्य \textbf{प्र‚त्य‚य}स्य प्र‚{\tiny $_{४}$}‚त्य‚य‚ग्र‚ह‚ण‚मुप‚ल‚क्ष‚ण‚मेवं श‚ब्द‚स्य । सामान्य‚म‚न्त‚{\tiny $_{lb}$}‚रेणा\textbf{योगा}त् । सामान्य‚वाद आश्रित इति स‚म्ब‚न्धः ।
	{\color{gray}{\rmlatinfont\textsuperscript{§~\theparCount}}}
	\pend% ending standard par
      ‚{\tiny $_{lb}$}‚

	  
	  \pstart \leavevmode% starting standard par
	\textbf{क‚थ}मित्याचा र्यः । ये \textbf{पाच‚का}दिश‚ब्दा न क्रियानिमित्तानिच्छ‚न्ति तान् प्र‚त्ये‚{\tiny $_{lb}$}‚‚{\tiny $_{lb}$}‚ \leavevmode\ledsidenote{\textenglish{310/s}}त‚दुक्तं । अभिन्नेन सामान्याख्ये\textbf{नार्थेन विना पाच‚कादिषु} क‚थ‚मेकः श‚ब्दो \textbf{वाच‚कः} ।‚{\tiny $_{lb}$}‚ वाच‚क‚ग्र‚ह‚णेन प्र‚त्य‚यो\textbf{न्व‚यी} गृहीत एव तेन विना श‚ब्द‚स्याप्र‚वृत्तेः‚{\tiny $_{५}$}‚ । अत एव‚{\tiny $_{lb}$}‚ वृ त्तौ \textbf{श‚ब्द‚प्र‚त्य‚यानुवृत्तिर‚स्ती}त्याह ।
	{\color{gray}{\rmlatinfont\textsuperscript{§~\theparCount}}}
	\pend% ending standard par
      ‚{\tiny $_{lb}$}‚

	  
	  \pstart \leavevmode% starting standard par
	उ द्यो त क रे णापि ग‚वादिष्व‚नुवृत्तिप्र‚त्य‚यः पिण्डादिव्य‚तिरिक्त‚निमित्ताद्‚{\tiny $_{lb}$}‚ भ‚व‚ति विशेष‚प्र‚त्य‚यानाम‚नाक‚स्मिक‚त्वान्नीलादिप्र‚त्य‚य‚व‚त् । य‚त्त‚न्निमित्त‚न्त‚त्सामान्य‚{\tiny $_{lb}$}‚मिति सामान्य‚सिद्धौ प्र‚माणे कृते स्व‚य‚मेवाशंकित‚म् [।]
	{\color{gray}{\rmlatinfont\textsuperscript{§~\theparCount}}}
	\pend% ending standard par
      ‚{\tiny $_{lb}$}‚

	  
	  \pstart \leavevmode% starting standard par
	अथ म‚न्य‚से य‚था पाच‚कादिश‚ब्दा अनुवृत्ताश्च भ‚व‚न्ति न च‚{\tiny $_{६}$}‚ पाच‚क‚त्व‚न्नाम‚{\tiny $_{lb}$}‚ सामान्य‚म‚स्ति [।] य‚दि स्यात् । भावोत्प‚त्तिकाल एवाभिव्य‚क्तं स्यात् त‚था ग‚वा‚{\tiny $_{lb}$}‚दिष्व‚नुवृत्तिप्र‚त्य‚या इति । न [।] हेत्व‚र्थाप‚रिज्ञानात् । विशेष‚प्र‚त्य‚यानाम‚ना‚{\tiny $_{lb}$}‚क‚स्मिक‚त्वादित्य‚स्य हेतोः पिण्ड‚प्र‚त्य‚य‚व्य‚तिरिक्त‚स्य प्र‚त्य‚य‚स्य निमित्तान्त‚रादुत्पाद‚{\tiny $_{lb}$}‚ \leavevmode\ledsidenote{\textenglish{114b/PSVTa}} इत्य‚य‚म‚र्थः । न पुनः स‚र्वानुवृत्तिप्र‚त्य‚यः सामान्यादेव भ‚व‚तीति । एव‚{\tiny $_{७}$}‚ञ्च स‚ति‚{\tiny $_{lb}$}‚ प‚च‚न‚क्रियाया य‚त्प्र‚धानं साध‚न‚न्त‚त्पाच‚क‚श‚ब्देनोच्य‚ते । त‚च्च प्राधान्यं पाच‚कान्त‚{\tiny $_{lb}$}‚रेष्व‚प्य‚स्तीति न दोष इति व‚द‚ता पाच‚क‚त्वादिसामान्य‚म्विना पाच‚कादिश‚ब्दानां‚{\tiny $_{lb}$}‚ वृत्तिरिष्टैवो द्यो त क रे णेत्य‚नेनाभिप्रायेणा चा र्ये णाप्युक्तं क‚थ‚मित्यादि ।
	{\color{gray}{\rmlatinfont\textsuperscript{§~\theparCount}}}
	\pend% ending standard par
      ‚{\tiny $_{lb}$}‚

	  
	  \pstart \leavevmode% starting standard par
	न च प‚च‚न‚क्रियायां प्राधान्य‚निमित्तायां पाच‚कादिश‚ब्द‚प्र‚वृत्तिर्युक्ता । त‚{\tiny $_{१}$}‚‚{\tiny $_{lb}$}‚ न्निमित्त‚त्वे हि प्र‚धानं प्र‚धान‚मित्य‚नुगामी श‚ब्दः स्यान्न पाच‚क इति । अत‚{\tiny $_{lb}$}‚ एवाह [।] \textbf{न ही}त्यादि । \textbf{तेष्वि}ति पाच‚कादिषु । \textbf{अन्य‚दिति} द्र‚व्याद् व्य‚तिरिक्तं ।‚{\tiny $_{lb}$}‚ एवं स‚र्व‚श‚क्तिष्व\textbf{भिन्नं येनै}केन ते पाच‚काद‚यो भिन्नास्स‚न्तोपि \textbf{त‚थे}त्य‚भेदेन \textbf{प्र‚तीयेर‚न्}‚{\tiny $_{lb}$}‚ ज्ञानेन । उप‚ल‚क्ष‚ण‚मेत‚त् त‚थाभिधीयेर‚न् । पाच‚केष्व‚धिश्र‚य‚णादिल‚क्ष‚णं । पाठ‚केष्व‚{\tiny $_{lb}$}‚ध्य‚य‚नात्म‚क‚{\tiny $_{२}$}‚मेव‚म‚न्येष्व‚पि य‚थायोग्यं । क‚र्मैक‚प्र‚त्य‚यादिनिमित्त‚म‚स्तीति \textbf{चेत् ।‚{\tiny $_{lb}$}‚ स} इत्य‚न्व‚यी । प्र‚त्य‚य‚ग्र‚ह‚ण‚मुप‚ल‚क्ष‚ण‚मेवं श‚ब्दोपि । अन्येन वेति \textbf{क‚र्म‚णो} हेतुना‚{\tiny $_{lb}$}‚ प्र‚य‚त्नादिना । \textbf{भिन्न}मित्यादिनैत‚देव स‚म‚र्थ‚य‚ते । \textbf{त‚दि}ति बाह्योप‚न्यासे । त‚त्क‚र्मेति‚{\tiny $_{lb}$}‚ ‚{\tiny $_{lb}$}‚ \leavevmode\ledsidenote{\textenglish{311/s}}वा स‚म्ब‚न्ध‚नीयं । प्र‚तिपाच‚कं क‚र्म‚णो भेदात् । येनाप‚राधेन \textbf{ता} व्य‚क्त‚य\textbf{स्त‚थे}त्य‚भिन्न‚{\tiny $_{lb}$}‚प्र‚त्य‚य‚{\tiny $_{३}$}‚ हेतुत्वेन ।
	{\color{gray}{\rmlatinfont\textsuperscript{§~\theparCount}}}
	\pend% ending standard par
      ‚{\tiny $_{lb}$}‚

	  
	  \pstart \leavevmode% starting standard par
	स‚त्त्यं [।] न क‚श्चिद‚प‚राधः किन्तु तासां व्य‚क्तीना\textbf{मेक‚रूप‚त्वा}त् । त‚था हि‚{\tiny $_{lb}$}‚ द्र‚व्य‚मेक‚रूप‚म‚नंश‚त्वात् । एत‚देव चेद\textbf{भिन्न‚प्र‚त्य}य‚निब‚न्ध‚नं न तु त‚तो व्य‚तिरिक्तं‚{\tiny $_{lb}$}‚ ध‚र्मान्त‚र‚न्त‚दा पाच‚क‚स्य पाच‚क‚त्व‚मिति \textbf{व्य‚तिरेक‚प्र‚तीतिर्न स्यात्} । न हि त‚स्यैव‚{\tiny $_{lb}$}‚ त‚तो व्य‚तिरेको युक्तः । त‚स्याव्य‚क्तेराकार‚स्त\textbf{दाका}र‚स्त‚स्माद‚न्योऽभेदाकार‚स्त‚स्य‚{\tiny $_{lb}$}‚ \textbf{वि‚{\tiny $_{४}$}‚शेषः} सोस्ति य‚स्यां \textbf{सा अत‚दाकार‚विशेष‚व‚ती} । द्र‚व्याकाराद‚न्याकारेत्य‚र्थः ।
	{\color{gray}{\rmlatinfont\textsuperscript{§~\theparCount}}}
	\pend% ending standard par
      ‚{\tiny $_{lb}$}‚

	  
	  \pstart \leavevmode% starting standard par
	एत‚दुक्त‚म्भ‚व‚ति । द्र‚व्येभ्य एव प्र‚त्य‚यो द्र‚व्य‚मित्येव‚माकारः । त‚तोन्येनैवा‚{\tiny $_{lb}$}‚कारेण पाच‚क‚प्र‚त्य‚यः प्र‚तिषेध‚प्र‚त्य‚यस्स य‚दि द्र‚व्य‚निमित्त‚मेव स्यात् त‚दा द्र‚व्य‚मात्र‚{\tiny $_{lb}$}‚प्र‚त्य‚याविशिष्टः स्यात् । अथ किम‚र्थ‚म‚त‚दाकार‚विशेष‚व‚तीत्युभ‚य‚मुक्त‚म‚त‚दाकारेत्येव‚{\tiny $_{lb}$}‚ व‚क्त‚व्यं ।‚{\tiny $_{५}$}‚ विशेष‚व‚तीत्येव वा । उच्य‚तेऽभेदाकारेत्युक्ते द्र‚व्य‚स्याभाव इति प्र‚ति‚{\tiny $_{lb}$}‚षेध‚प्र‚त्य‚योप्य‚त‚दाकार इति श‚क्येत व्य‚प‚देष्टुं न चासौ व‚स्त्व‚न्त‚र‚निबंन्ध‚नः प‚रेणेष्टः ।‚{\tiny $_{lb}$}‚ विशेष‚प्र‚त्य‚यानामेव ध‚र्मान्त‚र‚निब‚न्ध‚न‚त्वात् । विशेष‚ग्र‚ह‚णे च केव‚ले क्रिय‚माणे ।‚{\tiny $_{lb}$}‚ चैत्र‚प्र‚त्य‚यो मैत्रापेक्ष‚या भ‚व‚ति विशेष‚वान् । न त्व‚त‚दाकारः । चैत्राद्य‚भिधा‚{\tiny $_{६}$}‚नेन‚{\tiny $_{lb}$}‚ द्र‚व्य‚स्यैव प्र‚तिपाद‚नात् । उभ‚योपादानात्त्व‚य‚म‚र्थो भ‚व‚त्य‚द्र‚व्याकार‚श्चासौ प्र‚त्य‚यो‚{\tiny $_{lb}$}‚ व‚स्तुस्प‚र्शाद् विशेष‚वांश्चेति । त‚स्मात् त‚त्र द्र‚व्य‚व्य‚तिरिक्ते न निमित्तान्त‚रेण‚{\tiny $_{lb}$}‚ भाव्य‚मिति ।
	{\color{gray}{\rmlatinfont\textsuperscript{§~\theparCount}}}
	\pend% ending standard par
      ‚{\tiny $_{lb}$}‚

	  
	  \pstart \leavevmode% starting standard par
	\textbf{उक्त‚मि}त्या चा र्यः । य‚था व्य‚तिरेको गोर्गोत्वं पाच‚क‚स्य पाच‚क‚त्व‚मित्यादिको‚{\tiny $_{lb}$}‚ य‚था च विशेष‚प्र‚त्य‚या अन‚न्त‚रोक्तास्त‚थोक्त‚मिति स‚म्ब‚न्धः ।‚{\tiny $_{७}$}‚ क‚थ‚मुक्त‚मित्याह । \leavevmode\ledsidenote{\textenglish{115a/PSVTa}}‚{\tiny $_{lb}$}‚ \textbf{य‚थास्व‚मि}त्यादि । \textbf{अर्थान्त‚र‚विवेको}र्थान्त‚र‚व्य‚व‚च्छेदः । \textbf{य‚थास्व‚मिति} य‚स्य श‚ब्द‚स्य‚{\tiny $_{lb}$}‚ \textbf{य‚थासंकेतं} यो व्य‚व‚च्छेद‚स्त‚स्मादित्य‚र्थः । त‚था हि पाच‚क‚श‚ब्दोऽपाच‚क‚व्य‚व‚च्छिन्न‚म‚{\tiny $_{lb}$}‚प्र‚तिक्षिप्त‚भेदान्त‚रं प्र‚तिपाद‚य‚न् ध‚र्मिव‚च‚नः [।] पाच‚क‚त्व‚श‚ब्द‚स्तु त‚मेव व्य‚व‚च्छिन्नं‚{\tiny $_{lb}$}‚ प्र‚तिक्षिप्त‚भेदान्त‚र‚माहेति ध‚र्म‚व‚च‚नः । त‚तो ध‚र्म‚ध‚र्मिभेद‚क‚ल्प‚न‚या पाच‚क‚स्य‚{\tiny $_{१}$}‚ पाच‚{\tiny $_{lb}$}‚क‚त्व‚मिति व्य‚तिरेक‚विभ‚क्तिः प्र‚युज्य‚ते । एवं द्र‚व्य‚श‚ब्द‚स्याप्य‚द्र‚व्य‚व्य‚व‚च्छिन्ने स्व‚भावे‚{\tiny $_{lb}$}‚ संकेतित‚त्वात् त‚द‚नुसारेणाद्र‚व्याव्य‚व‚च्छेदानुसारेणाद्र‚व्य‚व्य‚व‚च्छेदानुकारिणी बुद्धि‚{\tiny $_{lb}$}‚‚{\tiny $_{lb}$}‚ \leavevmode\ledsidenote{\textenglish{312/s}}रुत्प‚द्य‚ते । पाच‚क‚श‚ब्दात् त्व‚पाच‚क‚व्य‚व‚च्छिन्नानुकारिण्येव बुद्धिर‚तो य‚थाव्य‚व‚च्छेदं‚{\tiny $_{lb}$}‚ संकेतानुसारेण विशेष‚व‚ती बुद्धिरेक‚त्राप्य‚विरुद्धा । एत‚च्च भेदान्त‚र‚प्र‚तिक्षेपाप्र‚ति‚{\tiny $_{lb}$}‚क्षेपेत्यादिषु प्र‚तिपादितं‚{\tiny $_{२}$}‚ ।
	{\color{gray}{\rmlatinfont\textsuperscript{§~\theparCount}}}
	\pend% ending standard par
      ‚{\tiny $_{lb}$}‚

	  
	  \pstart \leavevmode% starting standard par
	\textbf{त‚स्मा}दित्यादिनोप‚संहारः । य‚था \textbf{व्य‚क्ती}नां भेद‚स्त‚द्व‚त् क‚र्म‚णोपि भेदाद्धेतो‚{\tiny $_{lb}$}‚र‚स्य \textbf{पाच‚काद्य‚भेद‚प्र‚त्य‚य‚स्य} न हेतुः क‚र्मेति स‚म्ब‚न्धः । \textbf{तेषां} पाच‚कानां यानि‚{\tiny $_{lb}$}‚ \textbf{क‚र्मा}णि पाकाख्यानि तेषु क‚र्म‚सु या पाच‚क‚त्व\textbf{जातिः} स‚म‚वेता सैवाभेदाद्धेतुः पाच‚का‚{\tiny $_{lb}$}‚भेद‚प्र‚त्य‚य‚स्य । \textbf{नेत्या}दिना प्र‚तिषेध‚ति । न जातिर्हेतुरिति प्र‚कृतं । किङ्कार‚णं [।]‚{\tiny $_{lb}$}‚ \textbf{क‚र्म‚संश्र}यात् । क‚र्म‚णि स‚म‚{\tiny $_{३}$}‚वेत‚त्वात् । द्र‚व्या\textbf{द‚र्थान्त‚रं} क‚र्म त‚त्स‚म्ब‚न्धिनी । अर्थान्त‚र‚{\tiny $_{lb}$}‚ इति द्र‚व्ये । \textbf{गोत्व‚मिवेति} निद‚र्श‚नं । न हि गोत्वं शाब‚लेयादिस‚म्ब‚न्धि । क‚र्क्का‚{\tiny $_{lb}$}‚दिष्व‚श्व‚भेदेषु गोप्र‚त्य‚य‚हेतुः । \textbf{पाच‚क‚क‚र्म‚सु} पाकाख्येषु क‚र्म‚जातिस्स‚म‚वेता । न च‚{\tiny $_{lb}$}‚ तानि क‚र्माणीति पाकाख्यानि । श‚ब्द‚ग्र‚ह‚ण‚मुप‚ल‚क्ष‚णं । त‚था पाच‚क‚प्र‚त्य‚येन प‚रि‚{\tiny $_{lb}$}‚ च्छिद्य‚न्ते । त‚स्य पाकाख्य‚स्य क‚र्म‚ण आश्र‚यो द्र‚व्यं‚{\tiny $_{४}$}‚ पाच‚क‚श‚ब्देनोच्य‚ते । न च त‚त्र‚{\tiny $_{lb}$}‚ द्र‚व्ये \textbf{क‚र्म‚जातिस्स}म‚वेता ।
	{\color{gray}{\rmlatinfont\textsuperscript{§~\theparCount}}}
	\pend% ending standard par
      ‚{\tiny $_{lb}$}‚

	  
	  \pstart \leavevmode% starting standard par
	एव‚न्ताव‚द\textbf{र्थान्त‚र‚स‚म्ब}न्धित्वं क‚र्म‚जातेराश्रित्य द्र‚व्य‚विष‚यं पाच‚काभिधान‚{\tiny $_{lb}$}‚प्र‚त्य‚यं प्र‚त्य‚य‚निमित्त‚त्व‚मुक्त‚म् [।] \href{http://sarit.indology.info/?cref=pv.3.156-3.157}{१५९-६०}
	{\color{gray}{\rmlatinfont\textsuperscript{§~\theparCount}}}
	\pend% ending standard par
      ‚{\tiny $_{lb}$}‚

	  
	  \pstart \leavevmode% starting standard par
	अधुना प्र‚कारान्त‚रेणाह । \textbf{त‚स्ये}त्यादि । \textbf{पाच‚क‚श्र}तेर‚न्या श्रुतिः \textbf{श्रुत्य‚न्त‚रं} । श्रुति‚{\tiny $_{lb}$}‚ग्र‚ह‚ण‚मुप‚ल‚क्ष‚ण‚मेवं ज्ञानान्त‚र‚निमित्त‚त्वात् । श्रुत्य‚न्त‚र‚मेवाह । \textbf{पाक} इत्यादि । त‚त इति‚{\tiny $_{lb}$}‚ क‚र्म‚जातेः क‚र्म‚विष‚{\tiny $_{५}$}‚य‚स्याभिधान‚स्य प्र‚त्य‚य‚स्य च हेतुत्वात् क‚र्म‚जातेरित्य‚भिप्रायः ।
	{\color{gray}{\rmlatinfont\textsuperscript{§~\theparCount}}}
	\pend% ending standard par
      ‚{\tiny $_{lb}$}‚

	  
	  \pstart \leavevmode% starting standard par
	स्यान्म‚तं [।] न क‚र्म‚जातिः पाच‚क‚प्र‚त्य‚यं ज‚न‚य‚ति किन्तु क‚र्म‚जातिस‚माश्र‚यात्‚{\tiny $_{lb}$}‚ क‚र्मैवेत्य‚त आह । \textbf{त‚स्ये}त्यादि । त‚स्येति पाच‚काद्य‚भेद‚प्र‚त्य‚य‚स्य । \textbf{क‚र्म‚निमित्तं य‚स्येति}‚{\tiny $_{lb}$}‚ विग्र‚हः । प्रोक्तं व्य‚क्तिव‚द् भेदान्न \textbf{हेतुः क‚र्मास्ये}त्यादि ।
	{\color{gray}{\rmlatinfont\textsuperscript{§~\theparCount}}}
	\pend% ending standard par
      ‚{\tiny $_{lb}$}‚

	  
	  \pstart \leavevmode% starting standard par
	न‚नूक्तं जातिस‚माश्र‚याद् भिन्न‚म‚पि क‚र्माभिन्न‚प्र‚त्य‚य‚हेतुरिति ।
	{\color{gray}{\rmlatinfont\textsuperscript{§~\theparCount}}}
	\pend% ending standard par
      ‚{\tiny $_{lb}$}‚‚{\tiny $_{lb}$}‚\textsuperscript{\textenglish{313/s}}

	  
	  \pstart \leavevmode% starting standard par
	उक्त‚मिद‚म‚युक्त‚न्तूक्तं । जातिस‚म्ब‚न्धेपि क‚र्म‚ण‚स्त‚थैव भिन्न‚त्वात् । \textbf{किञ्चे}‚{\tiny $_{lb}$}‚त्यादिनोप‚च‚य‚हेतुमाह । \textbf{त‚स्य} पाच‚काद्य‚भेद‚प्र‚त्य‚य‚स्य क‚र्म‚निमित्त‚त्वेऽभ्युप‚ग‚म्य‚माने ।‚{\tiny $_{lb}$}‚ \textbf{निरुद्धे क‚र्म‚णि} पुरुषः \textbf{पाच‚क इति नोच्येत} । उच्य‚ते च योग्य‚तामात्रेण [।] त‚तो न‚{\tiny $_{lb}$}‚ व‚स्तुभूत‚क्रियानिमित्तोयं व्य‚प‚देशः । अतीत‚स्यापि क‚र्म‚णोस्तित्वाद‚दोष इति चेदाह‚{\tiny $_{lb}$}‚ \textbf{प‚च‚त} एवेत्यादि‚{\tiny $_{७}$}‚ [।] य‚द्य‚तीत‚स्य स‚त्त्वं स्याद् व‚र्त्त‚मान‚व‚दुप‚ल‚भ्येतोप‚ल‚ब्धिल‚क्ष‚ण- \leavevmode\ledsidenote{\textenglish{115b/PSVTa}}‚{\tiny $_{lb}$}‚ प्राप्तं च क‚र्मेष्य‚ते ।
	{\color{gray}{\rmlatinfont\textsuperscript{§~\theparCount}}}
	\pend% ending standard par
      ‚{\tiny $_{lb}$}‚

	  
	  \pstart \leavevmode% starting standard par
	स्यान्म‚तं [।] क‚र्म‚जातिः क‚र्म‚णि स‚म‚वेता क‚र्मापि द्र‚व्ये स‚म‚वेत‚न्त‚तः स‚म्ब‚द्ध‚{\tiny $_{lb}$}‚स‚म्ब‚न्धात् क‚र्म‚जातिर्द्र‚व्य‚विष‚य‚स्य पाच‚क‚प्र‚त्य‚य‚स्य हेतुरिति चेदाह । \textbf{त‚त एवे}त्यादि ।‚{\tiny $_{lb}$}‚ \textbf{त‚त एवे}ति क‚र्म‚णो विन‚ष्ट‚त्वादेव न सामान्य‚स्य क‚र्म‚णा स‚म्ब‚न्धः साक्षात् । नापि‚{\tiny $_{lb}$}‚ क‚र्म‚द्वारेण पार‚म्प‚र्येण द्र‚व्य‚स‚म्ब‚न्धोस्या\textbf{स‚{\tiny $_{१}$}‚म्ब‚न्धात्} कार‚णा\textbf{न्न सामान्यं} पाच‚काद्य‚भि‚{\tiny $_{lb}$}‚धान‚प्र‚त्य‚य‚स्य हेतुः । अस‚म्ब‚द्ध‚म‚पि हेतुरिति चेदाह । \textbf{ने}त्यादि । अयुक्त‚मित्य‚{\tiny $_{lb}$}‚स‚म्ब‚द्धं । श‚ब्द‚ग्र‚ह‚ण‚मुप‚ल‚क्ष‚ण‚म‚स‚म्ब‚द्धं सामान्यं न ज्ञान‚श‚ब्द‚कार‚ण‚मित्य‚र्थः ।‚{\tiny $_{lb}$}‚ कुतः [।] \textbf{अतिप्र‚स‚ङ्गात्} । गोत्व‚म‚प्य‚श्व‚ज्ञान‚स्य हेतुः स्यात् । \href{http://sarit.indology.info/?cref=pv.3.157-3.158}{१६०-६१}
	{\color{gray}{\rmlatinfont\textsuperscript{§~\theparCount}}}
	\pend% ending standard par
      ‚{\tiny $_{lb}$}‚

	  
	  \pstart \leavevmode% starting standard par
	विन‚ष्टे हीत्यादिना व्याच‚ष्टे । \textbf{त‚त् सामान्य‚मि}ति क‚र्म‚सामान्य‚न्न क‚र्म‚णि स‚म‚{\tiny $_{lb}$}‚वेत‚न्त‚स्यास‚त्त्वात् । क‚र्माभावादेव क‚र्त्त‚रि पा‚{\tiny $_{२}$}‚च‚के पार‚म्प‚र्येणापि स‚म‚वेत‚म् [।]‚{\tiny $_{lb}$}‚ अतः स‚म्ब‚द्ध‚स‚म्ब‚न्धोप्य‚स्य सामान्य‚स्य द्र‚व्येण स‚ह नास्ति । \textbf{अन्य‚थे}त्य‚स‚म्ब‚द्ध‚स्यापि‚{\tiny $_{lb}$}‚ ज्ञानादिहेतुत्वे । स्थित्य‚भावाच्च क‚र्म‚ण इत्यादि य‚दुक्तं [।] त‚म\edtext{}{\lemma{म}\Bfootnote{? द}}तीतेत्या‚{\tiny $_{lb}$}‚दिना प‚क्षान्त‚र‚माशंक‚ते । \textbf{अतीतं} य‚द्विन‚ष्टं । \textbf{अनाग‚तं} य‚द् भ‚विष्य‚ति क‚र्म । \textbf{त‚यो‚{\tiny $_{lb}$}‚रिति} श‚ब्द‚ज्ञान‚योः [।] क‚र्माप्य‚तीतानाग‚त‚म‚स‚त् । \textbf{ज्ञानाभिधान‚योर्नि}मित्त‚मिति‚{\tiny $_{lb}$}‚‚{\tiny $_{lb}$}‚ \leavevmode\ledsidenote{\textenglish{314/s}}स‚म्ब‚न्धः । किं कार‚णं‚{\tiny $_{३}$}‚ [।] त‚योरित्यादि । त‚योर्ज्ञानाभिधान‚योः ।
	{\color{gray}{\rmlatinfont\textsuperscript{§~\theparCount}}}
	\pend% ending standard par
      ‚{\tiny $_{lb}$}‚

	  
	  \pstart \leavevmode% starting standard par
	\textbf{अस‚ती}त्यादिना व्याच‚ष्टे । उपाख्याय‚ते प्र‚काश्य‚ते व‚स्त्व‚न‚येत्युपाख्यार्थ‚क्रिया‚{\tiny $_{lb}$}‚श‚क्तिः । सा निर्ग‚ता य‚स्माद‚स‚त‚स्त‚त्त‚थोक्तं । \textbf{अस‚द्} य‚स्माद‚र्थ‚क्रिया\textbf{श‚क्तिविक‚लं} ।‚{\tiny $_{lb}$}‚ त‚देवंभूतं क‚थं श‚ब्द‚ज्ञान‚योर्निमित्तं स्यादित्य‚र्थः ।
	{\color{gray}{\rmlatinfont\textsuperscript{§~\theparCount}}}
	\pend% ending standard par
      ‚{\tiny $_{lb}$}‚

	  
	  \pstart \leavevmode% starting standard par
	अथास‚तोपि हेतुत्व‚मिष्य‚ते त‚दा त‚स्य व‚स्तुत्व‚मेव स्यान्नास‚त्त्वं । किङ्कार‚ण‚{\tiny $_{lb}$}‚मित्याह । कार्येत्यादि । ल‚{\tiny $_{४}$}‚क्ष‚ण‚श‚ब्दः स्व‚भाव‚व‚च‚नः । त‚दिति त‚स्मात् । अतीतं‚{\tiny $_{lb}$}‚ प्र‚च्युत‚रूपं । अनाग‚त‚म‚संप्राप्त‚रूपं । क‚र्म‚णः स‚काशाद‚न्य‚च्च व्य‚क्त्यादिकं ज्ञानाभि‚{\tiny $_{lb}$}‚धान‚योर्निमित्त‚त्वेन \textbf{नेष्टं} सामान्य‚वादिना ।
	{\color{gray}{\rmlatinfont\textsuperscript{§~\theparCount}}}
	\pend% ending standard par
      ‚{\tiny $_{lb}$}‚
	  \bigskip
	  \begingroup
	
	    
	    \stanza[\smallbreak]
	  {\normalfontlatin\large ``\qquad}व्य‚क्तिः क‚र्माश्र‚यो द्र‚व्यं ।{\normalfontlatin\large\qquad{}"}\&[\smallbreak]
	  
	  
	  
	  \endgroup
	‚{\tiny $_{lb}$}‚

	  
	  \pstart \leavevmode% starting standard par
	आदिश‚ब्दात् संकेत‚वास‚ना त‚त्प‚रिपाक‚योर्ग्र‚ह‚णं । ते इति श‚ब्द‚ज्ञाने । त‚था‚{\tiny $_{lb}$}‚ \textbf{चेत्य}न्व‚यिनोः पाच‚कादिश‚ब्द‚ज्ञान‚योर‚निमित्त‚त्वे स‚ति न जातिसिद्धिः ।‚{\tiny $_{५}$}‚ च‚श‚ब्दात्‚{\tiny $_{lb}$}‚ नित्यं स‚त्त्व‚म‚स‚त्त्व‚म्वा श‚ब्द‚ज्ञान‚योः स्यात् । क‚स्मान्न जातिसिद्धिरित्याह । त‚स्या‚{\tiny $_{lb}$}‚ इत्यादि । \textbf{त‚स्या जाते}र‚भिन्न‚स्य \textbf{ज्ञान‚स्याभिधान‚स्य च निमित्त‚त्वेनेष्ट‚त्वात् । य‚था}‚{\tiny $_{lb}$}‚ च पाच‚कादिविष‚ये । ते अनिमित्ते प्र‚व‚र्त्तेते त‚था ग‚वादाव‚पीति केन निब‚न्ध‚नेन जातिः‚{\tiny $_{lb}$}‚ क‚ल्प्येत ।
	{\color{gray}{\rmlatinfont\textsuperscript{§~\theparCount}}}
	\pend% ending standard par
      ‚{\tiny $_{lb}$}‚

	  
	  \pstart \leavevmode% starting standard par
	\textbf{श‚क्ति}रित्यादिना प‚क्षान्त‚र‚माशंक‚ते । क‚र्माश्र‚य‚स्य द्र‚व्य‚स्य श‚क्तिः । श‚ब्द‚{\tiny $_{lb}$}‚ग्र‚ह‚ण‚मुप‚ल‚क्ष‚णं‚{\tiny $_{६}$}‚ [।] पाच‚कादि ज्ञान‚स्यापि श‚क्तिर्निमित्तं । नेत्यादिना प्र‚तिषे‚{\tiny $_{lb}$}‚ध‚ति । न पाच‚कादिश‚क्तिः पाच‚कादिश‚ब्द‚निमित्तं [।] किङ्कार‚णं [।] श‚क्ते‚{\tiny $_{lb}$}‚र्द्र‚व्याव्य‚तिरेकेण द्र‚व्य‚व‚देवान‚न्व‚याद‚न‚न्व‚यिन‚श्चार्थ‚स्यान्व‚यिज्ञानाभिधानं प्र‚ति‚{\tiny $_{lb}$}‚ निमित्त‚त्वान‚भ्युप‚ग‚मात् । अभ्युप‚ग‚मे वा जातिक‚ल्प‚नाया निर्निब‚न्ध‚न‚त्व‚प्र‚स‚ङ्गात् ।‚{\tiny $_{lb}$}‚ ‚{\tiny $_{lb}$}‚ \leavevmode\ledsidenote{\textenglish{315/s}}भिन्नैव श‚क्तिरिति चेदाह । \textbf{न ही}त्यादि । न हि द्र‚व्याद‚{\tiny $_{७}$}‚न्यैव श‚क्तिर्य‚दि स्यात्त‚दा \leavevmode\ledsidenote{\textenglish{116a/PSVTa}}‚{\tiny $_{lb}$}‚ \textbf{त‚स्या} एक‚श‚क्तेः \textbf{पाकाद्य‚र्थ‚क्रियासूप‚योगेन} कार‚णेन \textbf{द्र‚व्य‚स्य} श‚क्त्याधार\textbf{स्यानुप‚यो‚{\tiny $_{lb}$}‚गित्व‚प्र‚स‚ङ्गात्} । त‚स्यां पाकादिनिर्व‚र्त्तिकायां श‚क्तौ त‚स्य द्र‚व्य‚स्योप‚योगः । एव‚म‚पि‚{\tiny $_{lb}$}‚ पार‚म्प‚र्येण पाकादौ द्र‚व्य‚मुप‚युक्तं स्यादिति भावः ।
	{\color{gray}{\rmlatinfont\textsuperscript{§~\theparCount}}}
	\pend% ending standard par
      ‚{\tiny $_{lb}$}‚

	  
	  \pstart \leavevmode% starting standard par
	\textbf{किमि}त्यादि सि द्धा न्त वा दी । अर्थान्त‚र‚भूत‚या \textbf{श‚क्त्या} न किञ्चित् प्र‚योज‚नं ।‚{\tiny $_{lb}$}‚ त‚था हि पाकादिनिर्व‚र्त्तिकायां प्र‚थ‚मायां \textbf{श‚क्तौ} द्र‚व्यं‚{\tiny $_{१}$}‚ य‚या श‚क्त्योप‚युज्येत । सापि‚{\tiny $_{lb}$}‚ श‚क्तिर्य‚दि व्य‚तिरिक्ताऽभ्युप‚ग‚म्येत त‚दा पाकादिनिर्व‚र्त्तिकायां \textbf{श‚क्तौ} द्र‚व्य‚स्यो\textbf{प‚योगाय‚{\tiny $_{lb}$}‚ श‚क्त्य‚न्त‚र‚स्य} द्र‚व्याद् \textbf{व्य‚तिरेकिणोऽभ्युप‚ग‚मेऽतिप्र‚स‚ङ्गात्} । त‚स्याम‚पि श‚क्तावुप‚{\tiny $_{lb}$}‚योगायाप‚रा व्य‚तिरिक्ता श‚क्तिः क‚ल्प‚नीया त‚त्राप्य‚प‚रेत्य‚न‚व‚स्था स्यादित्य‚र्थः ।‚{\tiny $_{lb}$}‚ त‚स्माद‚न्त‚रेण व्य‚तिरिक्तं श‚क्तिं \textbf{द्र‚व्य‚मेव} प्र‚थ‚मायां पाकादिनिर्विर्त्तिकायां श‚{\tiny $_{२}$}‚क्ता‚{\tiny $_{lb}$}‚\textbf{वुप‚युज्य‚त इति वाच्यं} । एवं च द्र‚व्य‚स्योप‚योगे \textbf{श‚क्ता}विष्य‚माणे । त‚द्द्र‚व्य‚म\textbf{र्थ‚क्रियायां}‚{\tiny $_{lb}$}‚ पाकादिल‚क्ष‚णाया\textbf{मेवोप‚युज्य‚त इति किन्नेष्य‚ते} । द्र‚व्य‚स्यार्थ‚क्रियायाश्चान्त‚राले \textbf{किम‚{\tiny $_{lb}$}‚न‚र्थिक‚या श‚क्त्या} क‚ल्पित‚या ।
	{\color{gray}{\rmlatinfont\textsuperscript{§~\theparCount}}}
	\pend% ending standard par
      ‚{\tiny $_{lb}$}‚

	  
	  \pstart \leavevmode% starting standard par
	य‚त एव‚न्त\textbf{स्मात्} पाकाद्य‚र्थ‚क्रियाश‚क्तिरित्य‚नेन द्र‚व्य‚मेवोच्य‚ते । किम्भूत‚न्त‚{\tiny $_{lb}$}‚त्कार्यं त‚त्पाकादि कार्यं य‚स्य । त‚च्च द्र‚व्यं व्य‚क्त्य‚न्त‚रं नान्वेतीति कृत्वा । त‚तो‚{\tiny $_{३}$}‚‚{\tiny $_{lb}$}‚ द्र‚व्यात् पाच‚कः पाच‚क इत्य\textbf{न्व‚यी श‚ब्दो न स्या}ज्ज्ञान‚ञ्च । श‚ब्द‚ग्र‚ह‚णं तूप‚ल‚क्ष‚णं ।‚{\tiny $_{lb}$}‚ \href{http://sarit.indology.info/?cref=pv.3.158-3.159}{१६१-६२} ॥
	{\color{gray}{\rmlatinfont\textsuperscript{§~\theparCount}}}
	\pend% ending standard par
      ‚{\tiny $_{lb}$}‚

	  
	  \pstart \leavevmode% starting standard par
	पाच‚कादिषु द्र‚व्येषु पाच‚क‚त्वादिसामान्य‚म‚स्ति त‚द‚न्व‚यि श‚ब्द‚ज्ञान‚निब‚न्ध‚न‚मिति‚{\tiny $_{lb}$}‚ चेदाह । \textbf{सामान्य‚मि}त्यादि । \textbf{सामान्यं पाच‚क‚त्वादि} य‚दीष्य‚ते । त‚दा पाकादिनिर्व‚{\tiny $_{lb}$}‚र्त्त‚न‚श‚क्त्य‚व‚स्थायाः \textbf{प्रागेव} द्र‚व्य‚स्योत्प‚त्तिस‚म‚काल एव द्र‚व्य‚स‚म‚वेत\textbf{न्त‚द् भ‚वेदि}त्य‚र्थः ।‚{\tiny $_{lb}$}‚ त‚था च त‚द‚ह‚र्जातो‚{\tiny $_{४}$}‚पि बालः पाच‚कादिज्ञानाभिधान‚विष‚यः स्यादिति भावः । \textbf{नो चेत्}‚{\tiny $_{lb}$}‚ प्रागेव भ‚वेत् त‚दा \textbf{प‚श्चाद}पि न भ‚वेत् । त‚स्य द्र‚व्य‚स्याविशेषात् । अस्त्येव स‚र्व‚कालं‚{\tiny $_{lb}$}‚ ‚{\tiny $_{lb}$}‚ \leavevmode\ledsidenote{\textenglish{316/s}}द्र‚व्ये पाच‚क‚त्वादि । किन्तु प्राग‚न‚भिव्य‚क्त‚म‚तो न श‚ब्द‚ज्ञान‚योर्निमित्त‚मित्य‚त आह ।‚{\tiny $_{lb}$}‚ \textbf{व्य‚क्त‚मि}ति । प्रागेवाभिव्य‚क्त‚म्भ‚वेदित्य‚र्थः । \textbf{स‚त्तादिव‚त्} । य‚था स‚त्ताद्र‚व्य‚त्वादि ।‚{\tiny $_{lb}$}‚ याव‚द्द्र‚व्य‚भावि । अर्थ‚क्रियायाश्च प्रागे‚{\tiny $_{५}$}‚व योग्य‚देशाव‚स्थितं द्र‚व्यं ।
	{\color{gray}{\rmlatinfont\textsuperscript{§~\theparCount}}}
	\pend% ending standard par
      ‚{\tiny $_{lb}$}‚

	  
	  \pstart \leavevmode% starting standard par
	\textbf{अथापी}त्यादिना व्याच‚ष्टे । \textbf{स‚त्य‚र्थे} जात्याश्र‚ये \textbf{त‚त्स‚म‚वाय‚स्य} सामान्य‚स‚म‚वा‚{\tiny $_{lb}$}‚य‚स्या\textbf{कादाचित्क‚त्वात्} स‚र्व‚काल‚भावित्वात् । एत‚देव द्र‚ढ‚य‚न्नाह । \textbf{याव‚न्ति} हीत्यादि ।‚{\tiny $_{lb}$}‚ \textbf{अर्थे} जात्याश्र‚ये । \textbf{स‚म‚वाय‚ध‚र्माणि} स‚म्ब‚न्ध‚योग्यानि \textbf{तानि} सामान्यानि । \textbf{अस्या}र्थ‚स्य‚{\tiny $_{lb}$}‚ य उत्पादः । \textbf{तेन स‚ह स‚म‚व‚य‚न्ति} । अस्मिन् सामान्याश्र‚य इति विभ‚{\tiny $_{६}$}‚क्तिविप‚रि‚{\tiny $_{lb}$}‚णामेन स‚म्ब‚न्धः । उत्पाद‚स‚म‚काल‚मेव द्र‚व्येण स‚ह स‚म्ब‚ध्य‚त इति याव‚त् । \textbf{इति‚{\tiny $_{lb}$}‚ स‚म‚यः} । सा मा न्य वा दि नः सिद्धान्तः । य‚दाहो द्यो त क रः [।] प्राग्गोत्वान्नासौ‚{\tiny $_{lb}$}‚ गौर्नाप्य‚गौरिति । किङ्कार‚ण‚म् [।] अभावे तौ विशेष‚ण‚प्र‚त्य‚यौ न च विशेष‚ण‚{\tiny $_{lb}$}‚प्र‚त्य‚यौ विशेष्य‚स‚म्ब‚न्ध‚म‚न्त‚रेण भ‚व‚तो न च प्राग् गोत्व‚योगाद् व‚स्तु विद्य‚ते । न‚{\tiny $_{lb}$}‚ \leavevmode\ledsidenote{\textenglish{116b/PSVTa}} चाविद्य‚मानं गौरिति वाऽगौरिति वा श‚{\tiny $_{७}$}‚क्य‚ते व्य‚प‚देष्टुं । य‚दैव व‚स्तूत्प‚द्य‚ते त‚दैव‚{\tiny $_{lb}$}‚ गोत्वेनाभिस‚म्ब‚ध्य‚त इति । त‚था न स‚त‚स्स‚त्तास‚म्ब‚न्धो नास‚तः । य‚दैव च व‚स्तु त‚दैव‚{\tiny $_{lb}$}‚ स‚त्त‚या स‚म्ब‚ध्य‚त इति ।\edtext{\textsuperscript{*}}{\lemma{*}\Bfootnote{\href{http://sarit.indology.info/?cref=nv}{ Nyāyavārtīka. }}}
	{\color{gray}{\rmlatinfont\textsuperscript{§~\theparCount}}}
	\pend% ending standard par
      ‚{\tiny $_{lb}$}‚

	  
	  \pstart \leavevmode% starting standard par
	अथ सिद्धान्त‚म‚तिक्र‚म्य प‚श्चाद् भावित्वं सामान्य‚स्य क‚ल्प्य‚ते । \textbf{त‚दा त‚द्व्य‚तिक्र‚मे}‚{\tiny $_{lb}$}‚ सिद्धान्त‚व्य‚तिक्र‚मे । त‚स्य सामान्य‚स्याश्र‚य‚स्य द्र‚व्य‚स्य \textbf{प‚श्चाद}प्य‚विशेषा\textbf{न्न‚{\tiny $_{lb}$}‚ त‚त्स‚म‚वायः स्यात्} । तेन सामान्येन स‚म‚वायो न स्यात् ।
	{\color{gray}{\rmlatinfont\textsuperscript{§~\theparCount}}}
	\pend% ending standard par
      ‚{\tiny $_{lb}$}‚

	  
	  \pstart \leavevmode% starting standard par
	य‚था फ‚लैक‚स्व‚भा‚{\tiny $_{१}$}‚व‚स्यापि र‚क्त‚ता प्राङ् न भ‚व‚ति । प‚श्चाच्च भ‚व‚ति ।‚{\tiny $_{lb}$}‚ त‚द्व‚त्पुरुष‚स्य पाच‚क‚त्वादिसामान्य‚मित्य‚त आह । \textbf{त‚त्स‚म्ब‚न्धी}त्यादि । \textbf{त‚त्स‚म्ब‚न्धि‚{\tiny $_{lb}$}‚स्व‚भाव‚वैगुण्यात्} । पाच‚क‚त्वादिसामान्य‚स‚म्ब‚न्धिस्व‚भाव‚वैगुण्यात् । स इति सामान्य‚{\tiny $_{lb}$}‚स‚म‚वायः । \textbf{त‚स्येति} पुंसः । न ह्य‚विगुणे स्व‚भावे स्थित‚स्य त‚त्स‚म्ब‚न्धो न भ‚वेत् ।‚{\tiny $_{lb}$}‚ \textbf{त‚त्रैव} च सामान्य‚स‚म‚वाय‚विगुणे \textbf{स्व‚भावे स्थित‚स्य} द्र‚व्य‚स्य प‚श्चात् सामान्य‚{\tiny $_{२}$}‚स‚म‚{\tiny $_{lb}$}‚वायो भ‚विष्य‚तीति \textbf{दुर‚न्व}य‚न्दुर्बोध\textbf{मेत‚त्} । फ‚ल‚स्याप्याम्रादेः पूर्वं प‚श्चाच्च य‚द्येक‚{\tiny $_{lb}$}‚स्व‚भाव‚ता । त‚त्रापि तुल्यं चोद्यं । सामान्य‚स‚म्ब‚द्ध‚मेव त‚दा द्र‚व्यं क्रियोप‚कारापेक्ष‚न्तु‚{\tiny $_{lb}$}‚ सामान्यं व्य‚न‚क्ति । \href{http://sarit.indology.info/?cref=pv.3.159-3.160}{१६२-६३}
	{\color{gray}{\rmlatinfont\textsuperscript{§~\theparCount}}}
	\pend% ending standard par
      ‚{\tiny $_{lb}$}‚‚{\tiny $_{lb}$}‚\textsuperscript{\textenglish{317/s}}

	  
	  \pstart \leavevmode% starting standard par
	सा च प्राङ्नास्तीति न प्राक् सामान्याभिव्य‚क्तिरित्य‚त आह । \textbf{क्रियोप‚कारे}‚{\tiny $_{lb}$}‚त्यादि । पाकादिल‚क्ष‚णा क्रिया । त‚त्कृतो य उप‚कार‚स्त\textbf{द‚पेक्ष‚स्य} सामान्यं प्र‚ति‚{\tiny $_{lb}$}‚ \textbf{व्य‚{\tiny $_{३}$}‚ञ्ज‚क‚त्वे}ऽभ्युप‚ग‚म्य‚माने । त‚स्य द्र‚व्य‚स्याक्ष‚णिक‚त्वाद\textbf{विकारिणोन‚पेक्षा} स‚ह‚{\tiny $_{lb}$}‚कारिणं प्र‚ति ।
	{\color{gray}{\rmlatinfont\textsuperscript{§~\theparCount}}}
	\pend% ending standard par
      ‚{\tiny $_{lb}$}‚

	  
	  \pstart \leavevmode% starting standard par
	अथ विक्रियेत त‚दाप्य\textbf{तिश‚येस्य} द्र‚व्य‚स्य क्ष‚णिक‚त्व‚माप‚द्य‚ते । \textbf{क्ष‚णिक‚त्वा}च्चो‚{\tiny $_{lb}$}‚त्पादान‚न्त‚रं ध्वंसिनः \textbf{कुतः क्रिया} । येन त‚दुप‚कारापेक्षं जातेर्व्य‚ञ्ज‚कं स्यात् ।
	{\color{gray}{\rmlatinfont\textsuperscript{§~\theparCount}}}
	\pend% ending standard par
      ‚{\tiny $_{lb}$}‚

	  
	  \pstart \leavevmode% starting standard par
	\textbf{क‚र्मोप‚कारे}त्यादिना व्याच‚ष्टे । अधिश्र‚य‚णादिल‚क्ष‚णो व्यापारः क‚र्म । त‚त्कृत‚{\tiny $_{lb}$}‚ उप‚कारोतिश‚य‚स्त‚म\textbf{पेक्ष्य} स्थिर‚{\tiny $_{४}$}‚स्व‚भाव‚स्य पूर्व‚स्व‚भावाद‚च‚ल‚तोन‚तिश‚यात् स्व‚भावा‚{\tiny $_{lb}$}‚न्त‚रानुपादानात् । \textbf{अविशेषाधायिनि} क‚र्म‚णि । \textbf{कापेक्षा} । नैव । \textbf{अतिश‚ये वा}‚{\tiny $_{lb}$}‚ द्र‚व्य‚स्य क्रियाकृतेभ्युप‚ग‚म्य‚मानेऽतिश‚याधाय‚क‚स्य क‚र्म‚णः \textbf{क्ष‚णिक‚त्वात्} त‚स्याप्युप‚{\tiny $_{lb}$}‚कार्य‚स्य द्र‚व्य‚स्य स्व‚भाव‚भूतेनान्येनातिश‚येनोत्प‚त्त‚व्यं ।
	{\color{gray}{\rmlatinfont\textsuperscript{§~\theparCount}}}
	\pend% ending standard par
      ‚{\tiny $_{lb}$}‚

	  
	  \pstart \leavevmode% starting standard par
	य‚दि क्रियाकृतोऽतिश‚यो न स्व‚भाव‚भूतो \textbf{द्र‚व्य‚स्य} त‚द‚र्थान्त‚र‚स्य क‚र‚णाद् द्र‚व्यं‚{\tiny $_{lb}$}‚ नैवोप‚{\tiny $_{५}$}‚कृतं स्यात् । त‚स्माद् य‚थाक्रियाक्ष‚णं प्र‚तिक्ष‚णं \textbf{स्व‚भाव‚भूत‚स्यान्यान्य‚स्याति‚{\tiny $_{lb}$}‚श‚योत्प‚त्तेस्त‚द‚पि} द्र‚व्यं देव‚द‚त्तादि \textbf{क्ष‚णिकं स्यात् । त‚त} इति क्ष‚णिक‚त्वात् । \textbf{स्वो‚{\tiny $_{lb}$}‚त्प‚त्तिस्थान‚विनाशिनः} स्व‚स्मिन्नेवोत्प‚त्तिदेशे विनाशिनः पुंसः \textbf{कृतः} पाक‚ल‚क्ष‚णा‚{\tiny $_{lb}$}‚ \textbf{क्रिया} । य‚द‚पेक्ष‚न्त‚त् क्रियासापेक्षं पाच‚कादिद्र‚व्य‚सामान्य‚स्य \textbf{व्य‚ञ्ज‚कं स्यात} ।
	{\color{gray}{\rmlatinfont\textsuperscript{§~\theparCount}}}
	\pend% ending standard par
      ‚{\tiny $_{lb}$}‚

	  
	  \pstart \leavevmode% starting standard par
	न‚नु च प्र‚थ‚मादिक्रियाक्ष‚ण‚द्वारेण द्र‚व्य‚स्य‚{\tiny $_{६}$}‚ क्ष‚णिक‚त्वं क्रियाऽभावे च क‚थं क्ष‚णि‚{\tiny $_{lb}$}‚क‚त्व‚मिति चेत् [।] न । य‚तः क्रियास‚म्ब‚न्धोत्प‚न्नानां क्ष‚णानाम‚न्य‚स्याः क्रियाया‚{\tiny $_{lb}$}‚स्सामान्याभिव्य‚ञ्जिकाया अभावादित्य‚र्थः ।
	{\color{gray}{\rmlatinfont\textsuperscript{§~\theparCount}}}
	\pend% ending standard par
      ‚{\tiny $_{lb}$}‚

	  
	  \pstart \leavevmode% starting standard par
	त‚स्मात् स्थित‚मेत‚द् य‚था व‚स्तुभूता जातिर्नास्तीति । \href{http://sarit.indology.info/?cref=pv.3.160-3.161}{१६३-६४}
	{\color{gray}{\rmlatinfont\textsuperscript{§~\theparCount}}}
	\pend% ending standard par
      ‚{\tiny $_{lb}$}‚

	  
	  \pstart \leavevmode% starting standard par
	\textbf{क‚थ‚न्त‚र्ही}त्यादि प‚रः ।
	{\color{gray}{\rmlatinfont\textsuperscript{§~\theparCount}}}
	\pend% ending standard par
      ‚{\tiny $_{lb}$}‚

	  
	  \pstart \leavevmode% starting standard par
	य‚थेत्यादि सि द्धा न्त वा दी । \textbf{य‚था-पाच‚कादिषु} पाच‚क‚त्वादिसामान्य‚न्नास्ति त‚था‚{\tiny $_{lb}$}‚ ‚{\tiny $_{lb}$}‚ \leavevmode\ledsidenote{\textenglish{318/s}}\leavevmode\ledsidenote{\textenglish{117a/PSVTa}} प्र‚साधित‚म‚थ च त‚त्र प्र‚व‚र्त्तेते \textbf{अन्व‚यिनौ‚{\tiny $_{७}$}‚ ज्ञान‚श‚ब्दौ} । त‚थान्य‚त्राप्य‚न्त‚रेण सामान्य‚न्तौ‚{\tiny $_{lb}$}‚ भिव‚ष्य‚तः ।
	{\color{gray}{\rmlatinfont\textsuperscript{§~\theparCount}}}
	\pend% ending standard par
      ‚{\tiny $_{lb}$}‚

	  
	  \pstart \leavevmode% starting standard par
	\textbf{त‚द्व}दित्यादि प‚रः । तेष्विति पाच‚कादिषु सामान्य‚म्विना क‚थ‚म‚न्व‚यिनोर्ज्ञान‚{\tiny $_{lb}$}‚श‚ब्द‚योर्वृत्तिरिति । त‚तोन्व‚य‚ज्ञान‚श‚ब्द‚वृत्तेः पाच‚कादिष्व‚पि पाच‚क‚त्वादिसामान्य‚{\tiny $_{lb}$}‚म‚स्तीति चिन्तित‚मेत‚द‚न‚न्त‚रं । त‚था तेषु पाच‚क‚त्वादि सामान्यं स न स‚म्भ‚व‚तीति ।
	{\color{gray}{\rmlatinfont\textsuperscript{§~\theparCount}}}
	\pend% ending standard par
      ‚{\tiny $_{lb}$}‚

	  
	  \pstart \leavevmode% starting standard par
	य‚द्य‚न्व‚यि रूप‚न्नास्ति‚{\tiny $_{१}$}‚ त‚त्किमिदानी\textbf{म‚निमित्ते} ते श‚ब्द‚ज्ञाने \textbf{स्यातां । नेत्यादि}‚{\tiny $_{lb}$}‚ सि द्धा न्त वा दी । अस्त्येव त‚योर्निमित्तं य‚त् प‚रेणेष्य‚ते त‚स्य प्र‚तिक्षेपः । त‚देवाह ।‚{\tiny $_{lb}$}‚ \textbf{किन्त‚र्ही}त्यादि । व‚स्तुभूतं सामान्य\textbf{म्बाह्य‚त‚त्वं निमित्ते} ते न भ‚व‚तः । किन्त‚र्हि‚{\tiny $_{lb}$}‚ त‚योर्निमित्त‚मित्य‚त आह । \textbf{य‚था}स्व‚मित्यादि । यो य आत्मीयो \textbf{वास‚नाप्र‚बोध‚स्त}‚{\tiny $_{lb}$}‚स्माद‚न्व‚यिनो विक‚ल्प‚स्योत्प‚त्तिः । \textbf{त‚तो विक‚ल्पोत्प‚त्तेः} स‚काशाद् य‚थाविक‚{\tiny $_{२}$}‚ल्पं‚{\tiny $_{lb}$}‚ \textbf{श‚ब्दा भ‚व‚न्ति । न पुन‚र्विक‚ल्पाभिधान‚योर्व‚स्तुस‚त्ता} । अन्व‚यिप‚दार्थ\textbf{स‚त्तास‚माश्र‚य‚{\tiny $_{lb}$}‚ इत्युक्त‚प्राय‚मेत‚त} । अव‚श्यं चैत‚देष्ट‚व्यं । त‚था हि य‚थास्वं स‚म‚वाय\textbf{वास‚नाव‚शात्}‚{\tiny $_{lb}$}‚ सिद्धान्ताश्र‚येण ज्ञान‚वास‚नानुरोधाद् \textbf{विरोधिरूप‚स‚मावेशेन} प‚र‚स्प‚र‚विरुद्ध‚रूपाध्या‚{\tiny $_{lb}$}‚रोपेण प्र धा न कार्य‚मी श्व र कार्य‚म‚हेतुकं स‚म्वृत्तिमात्रं ज‚ग‚दित्येवं स‚र्व‚भेदेष्व‚न्व‚{\tiny $_{lb}$}‚यिनो\textbf{स्त‚यो}‚{\tiny $_{३}$}‚रिति ज्ञानाभिधान‚योः । \textbf{अप‚राप‚र‚द‚र्श‚ने}पीति प‚र‚स्प‚र‚भिन्नानाम‚र्थाना‚{\tiny $_{lb}$}‚न्द‚र्श‚नेपि । \textbf{न च त‚त्रेति} प्र‚धानादिकार्य‚त्वेन प‚र‚स्प‚र‚विरुद्धेन रूपेण क‚ल्पितेष्व‚र्थेषु ।‚{\tiny $_{lb}$}‚ \textbf{त‚न्निब‚न्ध}नः श‚ब्द‚ज्ञान‚योर्निब‚न्ध‚नः । क‚स्मान्नास्तीत्याह । \textbf{प‚र‚स्प‚र‚विरोधि}नोरित्यादि ।
	{\color{gray}{\rmlatinfont\textsuperscript{§~\theparCount}}}
	\pend% ending standard par
      ‚{\tiny $_{lb}$}‚

	  
	  \pstart \leavevmode% starting standard par
	\textbf{अनिय‚मेने}त्यादि प‚रः । स‚र्वं स‚र्व‚त्रान्व‚यि ज्ञान‚म‚भिधानं च स्यात् ।
	{\color{gray}{\rmlatinfont\textsuperscript{§~\theparCount}}}
	\pend% ending standard par
      ‚{\tiny $_{lb}$}‚

	  
	  \pstart \leavevmode% starting standard par
	एत‚देव साध‚य‚न्नाह ।‚{\tiny $_{४}$}‚ न ह्य‚निमित्त‚मित्यादि ।
	{\color{gray}{\rmlatinfont\textsuperscript{§~\theparCount}}}
	\pend% ending standard par
      ‚{\tiny $_{lb}$}‚

	  
	  \pstart \leavevmode% starting standard par
	न‚नु य‚थास्वं वास‚नाप्र‚बोधाद् विक‚ल्पोत्प‚त्तेरित्यादिना त‚योर‚निमित्त‚त्वं प्र‚तिषि‚{\tiny $_{lb}$}‚द्ध‚मित्य‚न‚व‚काश‚मेव चोद्यं । एव‚म्म‚न्य‚ते । आन्त‚र‚मेव निमित्त‚न्त‚योरिष्य‚ते त‚स्य च‚{\tiny $_{lb}$}‚ निमित्त‚स्य केन‚चिदास‚त्तिविप्र‚क‚र्षाभावात् । स‚र्व‚त्र स‚र्व‚विक‚ल्प‚हेतुत्वं स्यादिति ।‚{\tiny $_{lb}$}‚ ‚{\tiny $_{lb}$}‚ \leavevmode\ledsidenote{\textenglish{319/s}}न ह्य‚निमित्ते भ‚व‚दित्य‚त्र बाह्य‚निमित्ताभावाद‚निमित्त‚मिति द्र‚ष्ट‚व्यं ।
	{\color{gray}{\rmlatinfont\textsuperscript{§~\theparCount}}}
	\pend% ending standard par
      ‚{\tiny $_{lb}$}‚

	  
	  \pstart \leavevmode% starting standard par
	\textbf{नानिमित्ते} इति सि द्धा न्त वा दी । अविशिष्ट‚निमित्ते । न भ‚व‚त इत्य‚र्थः ।‚{\tiny $_{lb}$}‚ अत एवाह । \textbf{वास‚नाविशेष‚निमित्त‚त्वादिति} । य‚थाभूत‚द‚र्श‚न‚द्वारायाता वास‚ना‚{\tiny $_{lb}$}‚ सा त‚त्रैवाध्य‚व‚सित‚त‚द्भाव‚म्विक‚ल्पं ज‚न‚य‚ति । न स‚र्व‚त्रेति स‚मुदायार्थः । त‚थाभू‚{\tiny $_{lb}$}‚त‚मित्य‚न्व‚यि रूपं । \textbf{न चास‚ति त‚स्मिन्न}न्व‚यिनि बाह्ये निमित्ते विक‚ल्पेन न भ‚वित‚{\tiny $_{lb}$}‚व्य‚म्भ‚वित‚व्य‚मे‚{\tiny $_{६}$}‚व ।
	{\color{gray}{\rmlatinfont\textsuperscript{§~\theparCount}}}
	\pend% ending standard par
      ‚{\tiny $_{lb}$}‚

	  
	  \pstart \leavevmode% starting standard par
	त‚देव \textbf{सुप्ते}त्यादिना साध‚य‚ति । सुप्त‚श्च \textbf{तैमिरिक}श्च ताभ्यामुप\textbf{ल‚ब्धे}ष्व‚र्थेषु‚{\tiny $_{lb}$}‚ ग‚वादिषु केश‚म‚क्षिकादिषु च य‚थाक्र‚मं । \textbf{अभावे}षु श‚श‚विषाणादिषु । स‚म‚वा\textbf{य‚वास}ना ।‚{\tiny $_{lb}$}‚ य‚था स्वं सिद्धान्तं संकेत‚वास‚ना त‚द्ब‚ले\textbf{नारोपित‚रूप‚विशेषे} प्र‚धान‚कार्यादिषु ।‚{\tiny $_{lb}$}‚ \textbf{त‚था} विक‚ल्पोत्प‚त्तेर‚न्व‚यिनो \textbf{विक‚ल्प‚स्योत्प‚{\tiny $_{७}$}‚त्तेः} । न ह्येतेषु य‚थोक्तेषु बाह्य‚म‚न्व‚यि \leavevmode\ledsidenote{\textenglish{117b/PSVTa}}‚{\tiny $_{lb}$}‚ रूप‚म‚स्ति । स्व‚प्न‚तिमिरोप‚ल‚ब्धानामेवास‚त्त्वात् । तेषां चास‚त्त्वं तृतीये प‚रिच्छेदे‚{\tiny $_{lb}$}‚ \href{http://sarit.indology.info/?cref=}{३ । ८५} प्र‚तिपाद‚यिष्य‚ते । सिद्धान्त‚स‚मारोपित‚स्य तु प‚र‚स्प‚र‚विरोधिनोर्युग‚{\tiny $_{lb}$}‚प‚देक‚त्रेत्यादिना प्र‚तिपादित‚मेवास‚त्त्वं । \textbf{न च ते} विक‚ल्पाः स्व‚प्नाद्युप‚ल‚ब्धे\textbf{ष्व‚स‚त्सु}‚{\tiny $_{lb}$}‚ व‚स्तुभूतान्व‚यिरूप‚म‚न्त‚रेणाप्युत्प‚द्यंत इति \textbf{स‚र्व‚त्रा‚{\tiny $_{१}$}‚}र्थाः \textbf{स‚र्वाकारा} भ‚व‚न्त्य‚पि तु‚{\tiny $_{lb}$}‚ प्र‚तिनिय‚ता एव । निय‚म‚हेतुं चाह । \textbf{विभागेनैवेत्}यादि । \textbf{त‚थैवोप‚ल‚ब्धाना}मिति‚{\tiny $_{lb}$}‚ विभागेनोप‚ल‚ब्धानां । विभागेनैव \textbf{विक‚ल्प‚नात्} । सुप्त‚तिमिराव‚स्थायान्ताव‚द्‚{\tiny $_{lb}$}‚ भ्रान्त‚ज्ञानारूढानाम‚र्थानां विभागेनोप‚ल‚म्भः । सिद्धान्तारोपितानाम‚पि य‚थास्वं‚{\tiny $_{lb}$}‚ सिद्धान्त‚श्र‚व‚ण‚काले । \textbf{श‚श‚विषाण‚मि}त्यादिष्व‚पि । श‚श‚विषाणं ब‚न्ध्यासुत इ‚{\tiny $_{२}$}‚ति‚{\tiny $_{lb}$}‚ व्य‚व‚हार‚व्युत्प‚त्तिकालेऽनादित्वाद् व्य‚व‚हार‚वास‚नायाः । उक्तं चात्रेत्यादि ।‚{\tiny $_{lb}$}‚ 
	    \pend% close preceding par
	  
	    
	    \stanza[\smallbreak]
	  {\normalfontlatin\large ``\qquad}एक‚प्र‚त्य‚म‚र्शार्थ‚ज्ञानाद्येकार्थ‚साध‚न{\normalfontlatin\large\qquad{}"}\&[\smallbreak]
	  
	  
	  
	    \pstart  \leavevmode% new par for following
	    \hphantom{.}
	   \href{http://sarit.indology.info/?cref=pv.3.72}{१ । ७५} इत्य‚त्र ।
	{\color{gray}{\rmlatinfont\textsuperscript{§~\theparCount}}}
	\pend% ending standard par
      ‚{\tiny $_{lb}$}‚

	  
	  \pstart \leavevmode% starting standard par
	अपि च य‚था ध‚व‚ख‚दिराद‚यः प‚र‚स्प‚र‚भिन्नास्त‚था ग‚वाद‚यः । त‚त्र \textbf{तुल्ये भेदे}‚{\tiny $_{lb}$}‚ ‚{\tiny $_{lb}$}‚ \leavevmode\ledsidenote{\textenglish{320/s}}क‚स्माद् वृक्ष‚त्वं ध‚वादिष्वेव व‚र्त्त‚ते न ग‚वादिष्विति पृष्टेन प‚रेणैत‚देव व‚क्त‚व्यं‚{\tiny $_{lb}$}‚ भाव‚श‚क्तिरेव सा ध‚वादीनां येन त एव वृक्ष‚त्वं प्र‚ति‚{\tiny $_{३}$}‚ प्र‚त्यास‚न्ना न ग‚वाद‚य इति ।‚{\tiny $_{lb}$}‚ त‚दा तुल्ये भेद्ये \textbf{य‚या प्र‚त्यास‚त्त्या} भाव‚श‚क्तिल‚क्ष‚ण‚या \textbf{जातिः क्व‚चित्}‚{\tiny $_{lb}$}‚ स्वाश्र‚याभिम‚तेऽर्थ‚राशौ । \textbf{प्र‚स}र्प्प‚ति । व्याप्य व‚र्त्त‚ते । सैव भाव‚श‚क्तिर‚न्व‚यि\textbf{श‚ब्द‚{\tiny $_{lb}$}‚ज्ञान‚निब‚न्ध‚न‚म‚स्तु} । किं सामान्येन क‚ल्पितेन [।]
	{\color{gray}{\rmlatinfont\textsuperscript{§~\theparCount}}}
	\pend% ending standard par
      ‚{\tiny $_{lb}$}‚

	  
	  \pstart \leavevmode% starting standard par
	तेन य‚दुच्य‚ते ।
	{\color{gray}{\rmlatinfont\textsuperscript{§~\theparCount}}}
	\pend% ending standard par
      ‚{\tiny $_{lb}$}‚
	  \bigskip
	  \begingroup
	
	    
	    \stanza[\smallbreak]
	  {\normalfontlatin\large ``\qquad}विष‚येण हि बुद्धीनां विना नोत्प‚त्तिरिष्य‚ते ।&‚{\tiny $_{lb}$}‚विशेषाद‚न्य‚दिच्छ‚न्ति सामान्य‚न्तेन त‚द् ध्रुवं ।&‚{\tiny $_{lb}$}‚ता हि‚{\tiny $_{४}$}‚ तेन विनोत्प‚न्ना मिथ्या स्युर्विष‚यादृते ।&‚{\tiny $_{lb}$}‚न त्व‚न्येन विना वृत्तिस्सामान्य‚स्येह दुष्य‚तीति ।{\normalfontlatin\large\qquad{}"}\&[\smallbreak]
	  
	  
	  
	  \endgroup
	‚{\tiny $_{lb}$}‚

	  
	  \pstart \leavevmode% starting standard par
	त‚द‚पास्तं । न हि य‚था सामान्य‚म‚न्त‚रेण केषुचित् सामान्य‚वृत्तिरिष्य‚ते ।‚{\tiny $_{lb}$}‚ त‚था सामान्य‚म्विना सामान्य‚बुद्धिरिष्य‚तामिति प‚रोभ्युप‚ग‚मं कार्य‚ते । येन ता‚{\tiny $_{lb}$}‚ हि तेन विनोत्प‚न्ना मिथ्या स्युरिति प‚र‚स्योत्त‚रं स्यात् । केव‚लं य‚था प‚र‚स्य‚{\tiny $_{lb}$}‚ सामान्य‚म‚न्त‚रेण के‚{\tiny $_{५}$}‚षुचित् प‚दार्थ‚षु य‚या प्र‚त्यास‚त्त्या सामान्य‚वृत्तिः सैव‚{\tiny $_{lb}$}‚ भ्रान्तान्व‚यिज्ञान‚श‚ब्द‚निमित्त‚म‚स्तु किं सामान्येनेत्य‚य‚म‚र्थोत्र विव‚क्षित इति न‚{\tiny $_{lb}$}‚ क‚श्चिद्दोषः ।
	{\color{gray}{\rmlatinfont\textsuperscript{§~\theparCount}}}
	\pend% ending standard par
      ‚{\tiny $_{lb}$}‚

	  
	  \pstart \leavevmode% starting standard par
	त‚स्मात् स्थित‚मेत‚द् [।] व्यावृत्तेरेवैक‚त्वाध्य‚व‚सायाद् भावेष्व‚न्व‚यो नान्य‚स्येति ।‚{\tiny $_{lb}$}‚ \href{http://sarit.indology.info/?cref=pv.3.161-3.162}{। १६४-६५ ॥}
	{\color{gray}{\rmlatinfont\textsuperscript{§~\theparCount}}}
	\pend% ending standard par
      ‚{\tiny $_{lb}$}‚

	  
	  \pstart \leavevmode% starting standard par
	अत्र सां ख्यः प्राह । \textbf{न निवृत्ति}मित्यादि । \textbf{भावान्व‚यो} भावानामेक‚रूप‚त्वं ।‚{\tiny $_{lb}$}‚ \textbf{अप‚र} इति व‚स्तुभूतः । त‚दे\textbf{क‚स्य} बीज‚स्य य‚{\tiny $_{६}$}‚त्\textbf{कार्य‚न्त‚द‚न्य‚स्य} पृथिव्यादे\textbf{र्न स्यात्} ।‚{\tiny $_{lb}$}‚ क‚स्मात् [।] त‚योर्बीज‚पृथिव्योर\textbf{त्य‚न्त‚भेद‚तः} । [।] \href{http://sarit.indology.info/?cref=pv.3.162-3.163}{१६५-६६}
	{\color{gray}{\rmlatinfont\textsuperscript{§~\theparCount}}}
	\pend% ending standard par
      ‚{\tiny $_{lb}$}‚

	  
	  \pstart \leavevmode% starting standard par
	\textbf{य‚द्येत} इत्यादिना व्याच‚ष्टे । \textbf{एषा}मिति भावानां य‚था बीजादेकादीनामेक‚मं‚{\tiny $_{lb}$}‚कुराख्यं कार्यं । \textbf{यो ही}त्यादिना त‚देव साध‚य‚ति । \textbf{यो हि त‚स्य} बीज‚स्यांकुर‚ज‚न‚न‚{\tiny $_{lb}$}‚\edtext{}{\lemma{न}\Bfootnote{\href{http://sarit.indology.info/?cref=\%C5\%9Bv-\%C4\%81k\%E1\%B9\%9Bti.37-38}{ Śloka, Ākṛti 37, 38 }}}‚{\tiny $_{lb}$}‚ ‚{\tiny $_{lb}$}‚ \leavevmode\ledsidenote{\textenglish{321/s}}\textbf{स्व‚भावो न हि सोन्य‚स्य} पृथिव्यादेर‚स्ति ।‚{\tiny $_{७}$}‚ \textbf{योस्ति} बुद्ध्यारोपितो व्यावृत्तिल‚क्ष‚णो \textbf{न \leavevmode\ledsidenote{\textenglish{118a/PSVTa}}‚{\tiny $_{lb}$}‚ स ज‚न‚कः} । क‚स्माद् । \textbf{व्य‚तिरेक}स्यान्य‚व्यावृत्तिल‚क्ष‚ण‚स्य \textbf{निःस्व‚भाव‚त्वात्} । त‚स्माद्‚{\tiny $_{lb}$}‚ बीज‚स्व‚ल‚क्ष‚ण‚मेव \textbf{ज‚न‚कं य‚च्च} ज‚न‚क‚रूप\textbf{न्त‚देव व‚स्तु । त‚ज्ज‚न‚कं चे}त्य‚ङ्कुर‚ज‚न‚कं‚{\tiny $_{lb}$}‚ स्व‚ल‚क्ष‚णं । \textbf{अप‚र‚वे}ति पृथिव्यादौ । अप‚रं पृथिव्यादिक‚मंकुरं ज‚न‚येत् । \textbf{स ही}त्यादि ।‚{\tiny $_{lb}$}‚ हि श‚ब्द एव‚कारार्थः । \textbf{त‚स्येति} बीज‚स्य । \textbf{अ‚{\tiny $_{१}$}‚न्य}स्य पृथिव्यादेः [।] स पृथिव्यादिस्ते‚{\tiny $_{lb}$}‚नांकुर‚ज‚न‚नेन बीज\textbf{स्व‚भावेन त‚तो} बीजाद\textbf{भिन्नः स्यात् । इत्य‚स्तिस्व‚भावान्व‚यः} ।‚{\tiny $_{lb}$}‚ तेन केचित् स्व‚भाव‚भेदेपि प्र‚कृत्यैक‚कार्य‚कारिण इन्द्रियादिव‚दित्य‚युक्त‚मुक्त‚मिति ।
	{\color{gray}{\rmlatinfont\textsuperscript{§~\theparCount}}}
	\pend% ending standard par
      ‚{\tiny $_{lb}$}‚

	  
	  \pstart \leavevmode% starting standard par
	\textbf{य‚दी}त्यादिना सि द्धा न्त वा दी । \textbf{आत्मैक‚त्रा}पीति । कार‚ण‚क‚लाप‚स्य येना‚{\tiny $_{lb}$}‚भिन्नेनात्म‚ना ज‚न‚क‚त्व‚मिष्य‚ते । \textbf{स} आत्मा तेषां कार‚णा‚{\tiny $_{२}$}‚नाम्म‚ध्ये एक‚त्रापि कार‚णे‚{\tiny $_{lb}$}‚\textbf{स्तीति} । तेनैकेन कार्यं कृत‚मिति कृत्वा \textbf{व्य‚र्थाः स्युः स‚ह‚कारिणः} ।
	{\color{gray}{\rmlatinfont\textsuperscript{§~\theparCount}}}
	\pend% ending standard par
      ‚{\tiny $_{lb}$}‚

	  
	  \pstart \leavevmode% starting standard par
	न‚नु व्यावृत्तिवादिनोप्य‚न्त्याव‚स्थायां स‚र्वेषां ज‚न‚क‚त्वात् कार‚णान्त‚र‚वैय‚र्थ्यं ।
	{\color{gray}{\rmlatinfont\textsuperscript{§~\theparCount}}}
	\pend% ending standard par
      ‚{\tiny $_{lb}$}‚

	  
	  \pstart \leavevmode% starting standard par
	नैत‚द‚स्ति । स‚मुदितानामेव तेषान्तादृशं साम‚र्थ्यं क्ष‚णिकानां । हेतुप्र‚त्य‚या‚{\tiny $_{lb}$}‚य‚त्त‚स‚न्निधित्वात् । प‚र‚स्य तु नित्य‚वादिनः स‚दा त‚द् रूप‚म‚स्तीति भ‚वेत्कार‚णान्त‚रा‚{\tiny $_{lb}$}‚णामान‚र्थ‚क्यं । अत एवोक्त\textbf{मेक‚त्रापि सोस्ती}ति कार‚णान्त‚र‚विक‚लाव‚स्थायाम‚{\tiny $_{lb}$}‚पीत्य‚र्थः ।
	{\color{gray}{\rmlatinfont\textsuperscript{§~\theparCount}}}
	\pend% ending standard par
      ‚{\tiny $_{lb}$}‚

	  
	  \pstart \leavevmode% starting standard par
	\textbf{य‚दी}त्यादिना व्याच‚ष्टे । \textbf{अनेकः} प‚दार्थो \textbf{य‚द्येक‚स्व‚भाव‚त्वादेक‚स्य} कार्य‚स्य \textbf{कार‚कः}‚{\tiny $_{lb}$}‚ ज‚न‚कः स \textbf{तेषां} कार‚णाभिम‚तानाम\textbf{भिन्नो} ज‚न‚कः \textbf{स्व‚भाव एक}कार‚ण\textbf{स‚न्निधानेप्य‚{\tiny $_{lb}$}‚स्तिं । त‚त}श्च स‚र्व‚स्याम‚व‚स्थाया\textbf{र्म‚वैक‚ल्यात्} कार‚ण‚स्य य‚त्र त‚त्राव‚स्थिति\textbf{रेको‚{\tiny $_{४}$}‚पि‚{\tiny $_{lb}$}‚ ज‚न‚कः स्यात्} । \href{http://sarit.indology.info/?cref=pv.3.163-3.164}{। १६६-६७}
	{\color{gray}{\rmlatinfont\textsuperscript{§~\theparCount}}}
	\pend% ending standard par
      ‚{\tiny $_{lb}$}‚

	  
	  \pstart \leavevmode% starting standard par
	एत‚देव द्र‚ढ‚य‚न्नाह । \textbf{य‚स्मा}दित्यादि । एक‚स्मिन्न‚पि बीजादौ स‚न्निहिते \textbf{नापै‚{\tiny $_{lb}$}‚त्य‚भिन्नं त‚त्}कार्य‚ज‚न‚नं सामान्य\textbf{रूपं । विशेषा} व्य‚क्तिभेदा \textbf{अपायिनः । न हि त‚स्या‚{\tiny $_{lb}$}‚‚{\tiny $_{lb}$}‚ \leavevmode\ledsidenote{\textenglish{322/s}}भिन्न‚स्य} कार्य‚ज‚न‚न\textbf{स्व‚भाव‚स्य} बीजाद\textbf{र्थान्त‚रेपि} पृथिव्यादौ \textbf{विशेषोस्ति} । किङ्कार‚ण‚म्‚{\tiny $_{lb}$}‚ [।] \textbf{विशेषे स‚त्य‚भेद‚हानेः । स ह्य}भिन्नो ज‚न‚काभिम‚तः स्व‚भाव‚स्त\textbf{त्रा‚{\tiny $_{५}$}‚पि} बीजेपि‚{\tiny $_{lb}$}‚ केव‚ले\textbf{स्तीति । नैक‚स्य स्थिताव‚पि त‚स्ये}त्य‚भिन्न‚स्य रूप‚स्य ज‚न‚काभिम‚त‚स्या\textbf{पायो}‚{\tiny $_{lb}$}‚ विनाशोस्ति । त्रैगुण्य‚स्य स‚र्वात्म‚ना स‚र्व‚त्र स‚र्व‚दा स‚त्त्वात् । \textbf{ये विशेषा अव‚स्था‚{\tiny $_{lb}$}‚भेदास्तेषां स‚ह‚स्थितिनिय‚माभावात् स्याद‚पायः} ।
	{\color{gray}{\rmlatinfont\textsuperscript{§~\theparCount}}}
	\pend% ending standard par
      ‚{\tiny $_{lb}$}‚

	  
	  \pstart \leavevmode% starting standard par
	एत‚च्चाभ्युप‚ग‚म्योक्तं । अन्य‚था नित्याद‚व‚स्थान्त‚र‚व्य‚तिरिक्तानां विशेषा‚{\tiny $_{lb}$}‚णाम‚पि क‚थ‚म‚पायः । \textbf{न च ‚{\tiny $_{६}$}‚ते ज‚न‚का} इति विशेषाः क‚स्मा\textbf{न्नेष्टा} इत्याह । \textbf{स‚ह‚{\tiny $_{lb}$}‚कारिणा}मित्यादि । \textbf{त‚स्मादेक}स्मिन्न‚पि बीजादौ स्थिते \textbf{ज‚न‚क‚स्या}त्म‚नः \textbf{स्थानात् ।‚{\tiny $_{lb}$}‚ अस्थायिन} इति विशेष‚स्य । \textbf{एक‚स्थिताव}पीति बीजादेर‚न्य‚त‚म‚स्य स्थिताव‚पि \textbf{कार्यो‚{\tiny $_{lb}$}‚त्प‚त्तिः स्यात् । न च भ‚व‚ति} कार्योत्प‚त्तिः । त‚तः सामान्य‚स्थितेपि स‚ह‚कारिणाम्म‚{\tiny $_{lb}$}‚\leavevmode\ledsidenote{\textenglish{118b/PSVTa}} ध्ये । \textbf{एक}स्य विशेष‚स्या\textbf{पा‚{\tiny $_{७}$}‚ये । फ‚लाभावाद् विशेषेभ्य‚स्त‚दुद्भ‚वः} कार्योद्भ‚वः ।‚{\tiny $_{lb}$}‚ न सामान्यात् । त‚त्कार्य‚म‚ङ्कुरादिकं । किं भूतं । \textbf{अनेक}मित्यादि । अनेक‚स्य स‚ह‚{\tiny $_{lb}$}‚कारिणः \textbf{साधार‚णं} । अनेक‚स‚ह‚कारिज‚न्य‚मित्य‚र्थः । \textbf{एक‚विशेषापायेपी}ति स‚ह‚का‚{\tiny $_{lb}$}‚रिणाम‚न्य‚त‚म‚भेदापायेपि । अनेन व्य‚तिरेक‚माह ।
	{\color{gray}{\rmlatinfont\textsuperscript{§~\theparCount}}}
	\pend% ending standard par
      ‚{\tiny $_{lb}$}‚

	  
	  \pstart \leavevmode% starting standard par
	\textbf{पुन‚रित्या}दिनान्व‚यं । त‚स्माद् विशेषेष्वेवान्व‚य‚व्य‚तिरेकौ कार्य‚स्य न तु‚{\tiny $_{lb}$}‚ सामा‚{\tiny $_{१}$}‚न्ये । त‚दाह [।] न‚न्व‚विक‚ल इति । एक‚विशेष‚स्थिताव\textbf{विक‚लेप्य‚भिन्ने‚{\tiny $_{lb}$}‚ रूपे} त‚त्कार्य\textbf{न्न भ‚व‚ति । कार्यं} हीत्यादिनैत‚देव विभ‚ज‚ते । \textbf{कुत‚श्चिद्} भाव उत्पादः‚{\tiny $_{lb}$}‚ स एव ध‚र्मः । स य‚स्यास्ति त‚त् त‚द्भाव‚ध‚र्मि । क‚दाचित् य‚त्र \textbf{भ‚व‚ति त‚त्त‚स्य} ज‚न‚{\tiny $_{lb}$}‚क‚स्य वैक‚ल्यान्न \textbf{चाभिन्न‚स्य} रूप‚स्य ज‚न‚काभिम‚त‚स्य स‚ह‚कारिणां म‚ध्ये । \textbf{एक‚स्य‚{\tiny $_{lb}$}‚ स्थिताव‚पि वैक‚ल्य‚म‚स्ति । अविक‚ले त‚स्मिन्} सामान्य‚रूपे‚{\tiny $_{२}$}‚ कार्य\textbf{म‚भ‚व‚त् त‚स्य} सामा‚{\tiny $_{lb}$}‚‚{\tiny $_{lb}$}‚ \leavevmode\ledsidenote{\textenglish{323/s}}न्य‚स्या\textbf{ज‚न‚कात्म‚तां स‚च‚य‚ति} । \href{http://sarit.indology.info/?cref=pv.3.164-3.165}{। १६७-६८ ॥}
	{\color{gray}{\rmlatinfont\textsuperscript{§~\theparCount}}}
	\pend% ending standard par
      ‚{\tiny $_{lb}$}‚

	  
	  \pstart \leavevmode% starting standard par
	न‚नु न सामान्य‚मेव ज‚न‚क‚मिष्य‚ते येनाय‚न्दोषः स्याद‚पि तु अनेकात्म‚त‚या‚{\tiny $_{lb}$}‚ ज‚न‚क इत्युक्त‚मित्य‚त्राह । \textbf{य‚त्साक‚ल्ये}त्यादि । येषां च विशेषाणां \textbf{साक‚ल्य‚वैक‚ल्या‚{\tiny $_{lb}$}‚भ्यां} कार्य‚म्भावाभाव‚व‚त् । \textbf{त‚त एव} विशेषेभ्यः \textbf{कार्य‚स्योत्प‚त्तिः । त‚स्मिन् स‚तीति}‚{\tiny $_{lb}$}‚ हेतुभाव‚योग्ये विशेषे स‚ति भ‚व‚तः कार्य‚स्य । \textbf{त‚द‚न्य‚स्मा}‚{\tiny $_{३}$}‚दिति । विशेषाद‚न्य‚स्मात्‚{\tiny $_{lb}$}‚ सामान्या\textbf{द‚तिप्र‚संगात्} स‚र्वः स‚र्व‚स्य कार‚णं स्यात् । य‚त एवं \textbf{त‚स्माद् विशेषा एव‚{\tiny $_{lb}$}‚ ज‚न‚काः । न सामान्यं} ज‚न‚कं । \textbf{त‚तो}ऽज‚न‚क‚त्वात् \textbf{त एव} विशेषा \textbf{व‚स्तु} । प‚र‚मार्थ‚{\tiny $_{lb}$}‚स‚न्त इत्य‚र्थः । किं कार‚णं [।] य‚स्मात् पार‚मार्थिको भावः प‚र‚मार्थ‚स‚न्नित्य‚र्थः ।‚{\tiny $_{lb}$}‚ \textbf{स एवार्थ‚क्रियाक्ष‚मः} ।
	{\color{gray}{\rmlatinfont\textsuperscript{§~\theparCount}}}
	\pend% ending standard par
      ‚{\tiny $_{lb}$}‚

	  
	  \pstart \leavevmode% starting standard par
	इद‚मेव \textbf{ही}त्यादिना व्याच‚ष्टे । \textbf{अर्थ‚क्रियायोग्य‚ता व‚स्तुनो ल‚क्ष‚णं । अ‚{\tiny $_{४}$}‚योग्य‚ता}‚{\tiny $_{lb}$}‚ त्व‚व‚स्तुनो ल‚क्ष‚णं । \textbf{व‚क्ष्याम} इति स‚म्ब‚न्धः [।]
	{\color{gray}{\rmlatinfont\textsuperscript{§~\theparCount}}}
	\pend% ending standard par
      ‚{\tiny $_{lb}$}‚

	  
	  \pstart \leavevmode% starting standard par
	अर्थ‚क्रियाक्ष‚मं य‚त्त‚द‚त्र प‚र‚मार्थ‚स‚दि \href{http://sarit.indology.info/?cref=}{३ । ३} त्यादिना । स‚र्वार्थ‚क्रियायोग्योर्थो‚{\tiny $_{lb}$}‚ विशेषात्म‚को \textbf{नान्वेति} । विशेष‚स्य व्य‚क्त्य‚न्त‚रान‚नुयायित्वात् । \textbf{योन्वेति}‚{\tiny $_{lb}$}‚ सामान्यात्मा । \textbf{त‚स्मात्} सामान्यात्म‚नो \textbf{न कार्य‚स्य स‚म्भ‚वः} ।
	{\color{gray}{\rmlatinfont\textsuperscript{§~\theparCount}}}
	\pend% ending standard par
      ‚{\tiny $_{lb}$}‚

	  
	  \pstart \leavevmode% starting standard par
	\textbf{त‚स्मादि}त्यादिनोप‚संहारः । \textbf{अन‚र्थ‚क्रियायोग्य‚त्वादि}त्य‚र्थ‚क्रियायाम‚योग्य‚त्वा‚{\tiny $_{lb}$}‚दित्य‚र्थः ।‚{\tiny $_{५}$}‚ \textbf{त‚त एवेति} विशेषादेव \textbf{त‚न्निष्प‚त्ते}र‚र्थ‚क्रियाया निष्प‚त्तेः ।
	{\color{gray}{\rmlatinfont\textsuperscript{§~\theparCount}}}
	\pend% ending standard par
      ‚{\tiny $_{lb}$}‚

	  
	  \pstart \leavevmode% starting standard par
	त‚देव‚म्प‚रैः क‚ल्पित‚स्याज‚न‚क‚त्वं प्र‚तिपादितं । \href{http://sarit.indology.info/?cref=pv.3.165-3.166}{१६८-६९}
	{\color{gray}{\rmlatinfont\textsuperscript{§~\theparCount}}}
	\pend% ending standard par
      ‚{\tiny $_{lb}$}‚

	  
	  \pstart \leavevmode% starting standard par
	\hphantom{.}अधुना य‚त्प‚रेणोक्तं एक‚स्य कार्य‚म‚न्य‚स्य न स्याद‚त्य‚न्त‚भेद‚त इति त‚त्प‚रिह‚र्त्तुं‚{\tiny $_{lb}$}‚ त‚देव चोद्य‚माव‚र्त्त‚य‚ति । \textbf{स्व‚भावान‚न्व‚यात्त‚र्ही}त्यादिना ।
	{\color{gray}{\rmlatinfont\textsuperscript{§~\theparCount}}}
	\pend% ending standard par
      ‚{\tiny $_{lb}$}‚

	  
	  \pstart \leavevmode% starting standard par
	\hphantom{.}ज्व‚रादिश‚म‚ने क‚श्चित् स‚ह प्र‚त्येक‚मेव वा \href{http://sarit.indology.info/?cref=pv.3.73}{१ । ७६}
	{\color{gray}{\rmlatinfont\textsuperscript{§~\theparCount}}}
	\pend% ending standard par
      ‚{\tiny $_{lb}$}‚‚{\tiny $_{lb}$}‚\textsuperscript{\textenglish{324/s}}

	  
	  \pstart \leavevmode% starting standard par
	इत्यादिना प्राग‚प्येत‚त् प‚रिहृत‚न्त‚था‚{\tiny $_{६}$}‚प्य‚धिक‚विधानार्थः पुन‚रुप‚न्यासः ।‚{\tiny $_{lb}$}‚ \textbf{एक‚स्य} वीजादेर्य\textbf{ज्ज‚न‚कं रूप‚न्त‚द‚न्य‚स्य} पृथिव्यादेर्नास्तीति कृत्वाऽन्यः स‚ह‚कारी‚{\tiny $_{lb}$}‚ \textbf{अज‚न‚कः स्यात्} । ज‚न‚क‚स्व‚भावाद् भिन्न‚स्व‚भाव‚स्य \textbf{ज‚न‚क‚त्वे} चाभ्युप‚ग‚म्य‚माने‚{\tiny $_{lb}$}‚ \textbf{भेदाविशेषात् स‚र्वो} य‚व‚बीजादिर‚पि शाल्य‚ङ्कुर‚स्य \textbf{ज‚न‚कः स्यात्} ।
	{\color{gray}{\rmlatinfont\textsuperscript{§~\theparCount}}}
	\pend% ending standard par
      ‚{\tiny $_{lb}$}‚

	  
	  \pstart \leavevmode% starting standard par
	\textbf{नैत‚दि}त्यादिना प‚रिह‚र‚ति । शालिबीज‚स्यैक‚स्य ज‚न‚क‚स्य य आत्मा \textbf{तेना‚{\tiny $_{७}$}‚‚{\tiny $_{lb}$}‚त्म‚ना} पृथिव्यादेः य‚व‚बीजादेश्चात्य‚न्त\textbf{भेदेपि हेतुः क‚श्चित्} पृथिव्यादिः शाल्यंकुर‚स्य‚{\tiny $_{lb}$}‚ \textbf{नाप‚रो} य‚व‚बीजादिः । \textbf{च}श‚ब्दः श्लोक‚पूर‚णार्थः । एव‚कारार्थो वा । किङ्कार‚णं [।]‚{\tiny $_{lb}$}‚ \textbf{स्व‚भावोयं} भावानां एक‚स्य यो ज‚न‚क आत्मा त‚स्मादात्म‚नः स्व‚भावाद् भिद्य‚मानाः‚{\tiny $_{lb}$}‚ स‚र्वे \textbf{स‚म‚न्तुल्यं ज‚न‚काः} प्राप्नुव‚न्ति [।] भेदाविशेषान्न वा क‚श्चिज्ज‚न‚क इति [।]
	{\color{gray}{\rmlatinfont\textsuperscript{§~\theparCount}}}
	\pend% ending standard par
      ‚{\tiny $_{lb}$}‚

	  
	  \pstart \leavevmode% starting standard par
	\leavevmode\ledsidenote{\textenglish{119a/PSVTa}} स्यादेत‚च्चो‚{\tiny $_{७}$}‚द्यं । य‚द्येषामेक‚स्माज्ज‚न‚कादात्म‚नो भिन्नानान्त‚द‚त‚त्कार्य‚{\tiny $_{lb}$}‚ज‚न‚नं प्र‚ति निय‚म‚ल‚क्ष‚णो विशेषो न स‚म्भ‚वेत् । किन्तु स‚म्भ‚वेदेव । \textbf{त‚त} इति विशेष‚{\tiny $_{lb}$}‚स‚म्भ‚वात । \textbf{भेदाविशेषे}पि \textbf{कुत‚श्चिदात्मातिश‚या}द्विशिष्ट‚कार्य‚प्र‚तिनिय‚त‚ल‚क्ष‚णात्‚{\tiny $_{lb}$}‚ \textbf{क‚श्चिज्ज‚न‚कः} पृथिव्यादिः शाल्यंकुर‚स्य \textbf{नाप‚रो} य‚व‚बीजादिः [।] क‚स्माच्छाल्य‚{\tiny $_{lb}$}‚ङ्कुर‚ज‚न‚ना‚{\tiny $_{१}$}‚विशेष\textbf{स्त‚स्य} पृथिव्यादेः स‚ह‚कारिणः \textbf{स्व‚भावो नाप‚र‚स्य} य‚व‚बीजादेः ।‚{\tiny $_{lb}$}‚ अय‚मेव विभागः किंकृत इति चेदाह । \textbf{न ही}त्यादि । \textbf{किम‚ग्निर्द‚ह‚त्युष्णो वा नोद‚{\tiny $_{lb}$}‚क}न्द‚ह‚त्युष्णं \textbf{चेति न प‚र्य‚नुयोग‚म‚र्ह‚ति} प्र‚त्य‚क्षादिप्र‚माण‚सिद्ध‚त्वात् । एताव‚त्तु प्र‚ष्टुं‚{\tiny $_{lb}$}‚ युक्तं स्यात् [।] कुतो हेतोर‚यं य‚थोक्त‚स्व‚भाव इति । अव‚श्यं हि स्व‚भाव‚भेद‚स्य‚{\tiny $_{lb}$}‚ हेतुना भाव्यं । य‚तो‚{\tiny $_{२}$}‚ \textbf{निर्हेतुक‚त्वेऽन‚पेक्षिणो} देशादि\textbf{निय‚मेनातिप्र‚संगात्} । स‚र्व‚त्र‚{\tiny $_{lb}$}‚ स‚र्व‚दा स‚र्वात्म‚ना भाव‚प्र‚संगात् । \textbf{त‚स्मात् स्व‚भावोस्य} कार‚णाभिम‚त‚स्य \textbf{स्व‚हेतोः}‚{\tiny $_{lb}$}‚ स‚काशाद् भ‚व‚ती\textbf{त्युच्य‚ते । त‚स्यापि} स्व‚हेतोस्त\textbf{ज्ज‚न‚नात्म‚ता} त‚थाभूत‚कार‚ण‚स्व‚भाव‚{\tiny $_{lb}$}‚ज‚न‚नात्म‚ता \textbf{त‚द‚न्य‚स्मात्} स्व‚हेतोरि\textbf{त्य‚नादिर्हेतुप‚र‚म्प‚रा} ।
	{\color{gray}{\rmlatinfont\textsuperscript{§~\theparCount}}}
	\pend% ending standard par
      ‚{\tiny $_{lb}$}‚‚{\tiny $_{lb}$}‚\textsuperscript{\textenglish{325/s}}

	  
	  \pstart \leavevmode% starting standard par
	\textbf{न हि भिन्नाना}म्भावानां हेतुप्र‚विभागे बाध‚{\tiny $_{३}$}‚कं प्र‚माण‚म‚स्ति । त‚देवाह । \textbf{भिन्ने}‚{\tiny $_{lb}$}‚त्यादि । \textbf{स्व‚भावा}दिति व‚स्तुस्थितेः । एक‚त्वे तु बा\textbf{ध‚क‚म}स्तीत्याह । \textbf{अभेदे} त्वित्यादि ।‚{\tiny $_{lb}$}‚ प‚र‚स्प‚र‚म‚भेदादेक‚स्य नाशे स‚र्व‚स्य \textbf{युग‚प‚न्नाशः} उत्पादे स‚र्व‚स्योत्पादः स्यादित्य‚र्थः ।
	{\color{gray}{\rmlatinfont\textsuperscript{§~\theparCount}}}
	\pend% ending standard par
      ‚{\tiny $_{lb}$}‚

	  
	  \pstart \leavevmode% starting standard par
	\textbf{अभेदादि}त्यादिना व्याच‚ष्टे । \textbf{विश्व‚स्य} स‚र्व‚स्य प‚दार्थ‚राशेः स्व‚भावेनाभेदात् ।‚{\tiny $_{lb}$}‚ \textbf{विभागोत्प‚त्ती}त्यादि । एक‚स्योत्प‚त्तिर‚{\tiny $_{४}$}‚न्य‚स्या\textbf{नुत्प‚त्ति}रेक‚स्य \textbf{स्थिति}र‚न्य‚स्य \textbf{निरोध}‚{\tiny $_{lb}$}‚ इत्येव‚म्विभागेनोत्प‚त्त्याद‚यो \textbf{न} स्युः । \textbf{स्वात्म‚व‚दिति} । य‚थैक‚स्याव‚स्थाविशेष‚स्य‚{\tiny $_{lb}$}‚ विभागेन नोत्प‚त्त्याद‚य‚स्त‚द्व‚दित्य‚र्थः । आदिश‚ब्दादेक‚स्य ग्र‚ह‚ण‚म‚न्य‚स्याग्र‚ह‚ण‚मित्यादि ।‚{\tiny $_{lb}$}‚ सूत्रे तु नाशोत्पाद‚ग्र‚ह‚ण‚मुप‚ल‚क्ष‚णार्थं । \textbf{त‚था} तेनैव विभागोत्प‚त्त्यादिना \textbf{उप‚ल‚क्ष‚णा}‚{\tiny $_{lb}$}‚न्निश्च‚याद‚भेद‚स्य ।‚{\tiny $_{५}$}‚ अन्य‚था भेदाभेदौ केन ल‚क्ष्येते । \textbf{एकाकार‚स्यापी}ति तुल्या‚{\tiny $_{lb}$}‚कार‚स्यापि \textbf{व्य‚तिरेको} विभागः पृथ‚गुत्प‚त्त्यादिम‚त्त्वं । त‚द् भेद‚स्य ल‚क्ष‚णं ।‚{\tiny $_{lb}$}‚ \textbf{अव्य‚तिरेको}ऽविभागः पृथ‚गुत्प‚त्त्यादिम‚त्त्वाभावः । त‚द‚भेद‚स्य ल‚क्ष‚णं । स‚त्य‚पि‚{\tiny $_{lb}$}‚ विभागोत्प‚त्त्यादिम‚त्त्वे क‚स्माद् भेद इत्याह । \textbf{विरोधिनो}रित्यादि । विरोधिनो‚{\tiny $_{lb}$}‚रुत्पादानुत्पाद‚प्र‚भृ‚{\tiny $_{६}$}‚तिक‚योर्युग‚प‚देकात्म‚नि विरोधात् । नाभेद एवार्थानां किन्तु‚{\tiny $_{lb}$}‚ भेदोप्य‚स्ति । त‚दुक्तं ।
	{\color{gray}{\rmlatinfont\textsuperscript{§~\theparCount}}}
	\pend% ending standard par
      ‚{\tiny $_{lb}$}‚
	    
	    \stanza[\smallbreak]
	  स‚र्वं हि व‚स्तुरूपेण भिद्य‚ते न प‚र‚स्प‚रं ।&‚{\tiny $_{lb}$}‚स्व‚रूपापेक्ष‚या चैषां प‚र‚स्प‚र‚विभिन्न‚तेति ।\&[\smallbreak]
	  
	  
	  ‚{\tiny $_{lb}$}‚

	  
	  \pstart \leavevmode% starting standard par
	तेन कार‚णेन । नैवं [।] न स‚कृदुत्पादादिप्र‚संग इति चेदाह । \textbf{नेत्या}द्य‚स्यैव‚{\tiny $_{lb}$}‚ व्याख्यानं । \textbf{न वै स‚र्वेणाकारे}णाव्य‚तिरेक\textbf{म‚भेदं ब्रूमः । ये‚{\tiny $_{७}$}‚नैवं स्यात्} । स‚कृन्ना- \leavevmode\ledsidenote{\textenglish{119b/PSVTa}}‚{\tiny $_{lb}$}‚ शोत्पादादि स्यात् । किन्त्व‚स्य बाह्याध्यात्मिक‚स्य भेद‚स्य \textbf{क‚श्चिदात्मा} घ‚टा‚{\tiny $_{lb}$}‚द्य‚व‚स्थाविशेष‚ल‚क्ष‚णो भिन्नो नान्य‚स्त्रैगुण्यात्म‚कः सुख‚दुःख‚मोहात्म‚ताया व‚स्तु‚{\tiny $_{lb}$}‚ रूप‚ताया द्र‚व्य‚रूप‚ताया स‚त्तारूप‚तायाश्च स‚र्व‚त्रानुग‚मात् । तेनाय‚म‚र्थोव‚स्थात‚द्व‚तोः‚{\tiny $_{lb}$}‚ ‚{\tiny $_{lb}$}‚ \leavevmode\ledsidenote{\textenglish{326/s}}प‚र‚स्प‚र‚म‚भेदोप्य‚व‚स्थानान्तु प‚र‚स्प‚र‚म्भेद एव । इति य‚थोक्ताद‚व‚स्थादिल‚क्ष‚णाद्‚{\tiny $_{lb}$}‚ \textbf{भेदा‚{\tiny $_{१}$}‚त्} । \href{http://sarit.indology.info/?cref=pv.3.166-3.167}{१६९-७०}
	{\color{gray}{\rmlatinfont\textsuperscript{§~\theparCount}}}
	\pend% ending standard par
      ‚{\tiny $_{lb}$}‚

	  
	  \pstart \leavevmode% starting standard par
	एवं त‚र्हि सामान्य‚स्य नित्य‚त्वात् स‚र्व‚त्र स्थानं । विशेषाणान्तु विनाश इत्येत‚{\tiny $_{lb}$}‚द‚ङ्गीकृतं । त‚त‚श्चैक‚स्मिन् विशेषे विन‚श्य‚ति स‚ति य‚स्तिष्ठ‚ति सामान्यात्मा न स‚{\tiny $_{lb}$}‚ त‚स्य विशेष‚स्य स‚म्भ‚वः । विरुद्ध‚ध‚र्माध्यासात् सामान्य‚विशेष‚योर्भेद एव स्यात् ।‚{\tiny $_{lb}$}‚ अन्य‚था \textbf{स्थानास्थान‚योरेकात्माश्र‚य‚त्वे}भ्युप‚ग‚म्य‚माने विरुद्धाव‚पि ध‚र्मावेकात्म‚{\tiny $_{lb}$}‚न्य‚ङ्गीकृतौ भ‚{\tiny $_{२}$}‚व‚त‚श्च [।] \textbf{कोन्यो ध‚र्मो भेद‚को} नैव क‚श्चित् । नैव विरुद्धौ ध‚र्मा‚{\tiny $_{lb}$}‚वेक‚त्राङ्गीक्रियेते त‚योर्ल‚क्ष‚ण‚भेदाद् भिन्न‚त्वात् । त‚था हि विशेषाः पृथ‚गुत्पादा‚{\tiny $_{lb}$}‚दिना स‚र्वाकार‚विवेकिनः सामान्य‚न्तु पृथ‚गुत्प‚त्त्याद्य‚भावात् स‚र्व‚त्राविवेकीत्य‚त‚{\tiny $_{lb}$}‚ आह । \textbf{स‚र्वाकारे}त्यादि । \textbf{स‚र्वाकार‚विवेकाविवेकिनो}र्य‚थाक्र‚म‚म्भेद‚सामान्य‚यो\textbf{र‚र्थ‚यो‚{\tiny $_{lb}$}‚र‚भ्युप‚ग‚मात् । नाम केव‚ल}मिति‚{\tiny $_{३}$}‚ भेद‚सामान्ये भिन्ने इत्येत‚द‚भिधान‚मात्रं \textbf{नेष्टं‚{\tiny $_{lb}$}‚ स्या}न्न तु व‚स्तु । \textbf{व‚स्तु तु} भेद‚सामान्यात्म‚क‚म्प‚र‚स्प‚र‚म्भिन्न‚मेवेष्टं । एत\textbf{च्चोक्तं‚{\tiny $_{lb}$}‚ प्राक्} । नामान्त‚रं वार्थ‚भेद‚म‚भ्युप‚ग‚म्य त‚थाभिधानादित्यादिना ।
	{\color{gray}{\rmlatinfont\textsuperscript{§~\theparCount}}}
	\pend% ending standard par
      ‚{\tiny $_{lb}$}‚

	  
	  \pstart \leavevmode% starting standard par
	\textbf{त‚दिति} त‚स्मात् । \textbf{इमे भावा} इति विशेषाः सामान्य‚म्वा प‚र‚स्प‚रंभिन्ना एव ।‚{\tiny $_{lb}$}‚ किंभूता \textbf{नैक‚योग‚क्षेमा} विरुद्ध‚ध‚र्माध्यासिता इति याव‚त् । अ‚{\tiny $_{४}$}‚तः कार‚णान्न \textbf{स्या‚{\tiny $_{lb}$}‚त्सामान्य‚भेद‚धीः} । सामान्य‚विशेष‚योः प‚र‚स्प‚र‚म‚स‚म्ब‚ध्न‚न्ती बुद्धिर्न स्यादित्य‚र्थः ।‚{\tiny $_{lb}$}‚ बुद्ध्य‚भावाच्च व्य‚प‚देशोपि प्र‚तिक्षिप्त एव । बुद्धिनिब‚न्ध‚न‚त्वात्त‚स्य । तेन ।
	{\color{gray}{\rmlatinfont\textsuperscript{§~\theparCount}}}
	\pend% ending standard par
      ‚{\tiny $_{lb}$}‚
	  \bigskip
	  \begingroup
	
	    
	    \stanza[\smallbreak]
	  {\normalfontlatin\large ``\qquad}अन्योन्यापेक्षिता नित्यं स्यात् सामान्य‚विशेष‚योः ।&‚{\tiny $_{lb}$}‚विशेषाणान्तु सामान्य‚न्ते च त‚स्य भ‚व‚न्ति हि \href{http://sarit.indology.info/?cref=\%C5\%9Bv-\%C4\%81k\%E1\%B9\%9Bti.9}{आकृ० ९}{\normalfontlatin\large\qquad{}"}\&[\smallbreak]
	  
	  
	  
	  \endgroup
	‚{\tiny $_{lb}$}‚

	  
	  \pstart \leavevmode% starting standard par
	इति निर‚स्तं । \textbf{त‚दि}ति त‚स्मात् । इदं सामान्य‚म्भेदेभ्यो‚{\tiny $_{५}$}‚र्थान्त‚रं । भेदेष्व‚ना‚{\tiny $_{lb}$}‚य‚त्तं । क‚स्मात् [।] तैर्भेदैस्त‚स्य सामान्य‚स्याज‚न्य‚त्वात् ।
	{\color{gray}{\rmlatinfont\textsuperscript{§~\theparCount}}}
	\pend% ending standard par
      ‚{\tiny $_{lb}$}‚‚{\tiny $_{lb}$}‚\textsuperscript{\textenglish{327/s}}

	  
	  \pstart \leavevmode% starting standard par
	द्वितीयोर्थः [।] \textbf{त‚दिदं} भेदात्म‚कं व‚स्तु सामान्याद\textbf{र्थान्त‚र}न्त‚स्मिन् सामान्ये‚{\tiny $_{lb}$}‚ \textbf{अनाय‚त्त}न्तेना\textbf{ज‚न्य‚त्वात्} । नित्य‚स्यार्थ‚क्रियाऽसाम‚र्थ्यात् । अस्य भेद‚स्येदं \textbf{सामान्य‚{\tiny $_{lb}$}‚मिति व्य‚प‚देशं नार्ह‚ति । भेदो वास्य} सामान्य‚स्येति ॥
	{\color{gray}{\rmlatinfont\textsuperscript{§~\theparCount}}}
	\pend% ending standard par
      ‚{\tiny $_{lb}$}‚

	  
	  \pstart \leavevmode% starting standard par
	\textbf{अन्यापोहे}पि सामान्ये \textbf{एष प्र‚संग} इति । य एक‚स्मिन् विन‚श्य‚{\tiny $_{६}$}‚ति तिष्ठ‚त्या‚{\tiny $_{lb}$}‚त्मेत्यादिकः । त‚था हि ध‚वे ख‚दिरे वा विन‚श्य‚त्य‚वृक्ष‚व्यावृत्तिस्तिष्ठ‚त्येव वृक्षान्त‚रे ।
	{\color{gray}{\rmlatinfont\textsuperscript{§~\theparCount}}}
	\pend% ending standard par
      ‚{\tiny $_{lb}$}‚

	  
	  \pstart \leavevmode% starting standard par
	\textbf{नेत्या}दिना प‚रिह‚र‚ति । अय‚म‚त्रार्थो द्विविधोन्यापोह एको विजातीय‚व्या‚{\tiny $_{lb}$}‚वृत्तौ बाह्यः स्वाकार‚भेदेनाध्य‚स्तो विक‚ल्प‚बुद्ध्या यो विष‚यीक्रिय‚ते श‚ब्देन च ।‚{\tiny $_{lb}$}‚ त‚स्यैवार्थ‚क्रियाकारित्वेन च प्र‚वृत्तिविष‚य‚त्वान्न बुद्धि‚{\tiny $_{७}$}‚प्र‚तिभास‚मात्र‚स्यार्थ‚कारित्वा- \leavevmode\ledsidenote{\textenglish{120a/PSVTa}}‚{\tiny $_{lb}$}‚ भावात् । अप‚रोर्थाद् य‚त् प्र‚तीय‚तेऽन्य‚निवृत्तिमात्रं । \href{http://sarit.indology.info/?cref=pv.3.167-3.168}{१७०-७१}
	{\color{gray}{\rmlatinfont\textsuperscript{§~\theparCount}}}
	\pend% ending standard par
      ‚{\tiny $_{lb}$}‚

	  
	  \pstart \leavevmode% starting standard par
	य‚च्चैत‚द‚न्य‚निवृत्तिमात्र‚न्त‚स्य निःस्व‚भाव‚त्वान्नैत‚च्चोद्यं । आ चा र्ये ण वा‚{\tiny $_{lb}$}‚ य‚द‚भिम‚त‚न्त‚स्य निःस्व‚भाव‚त्वाद‚भावादित्य‚र्थः । त‚दाह [।] \textbf{निवृत्तेर्निःस्व‚भाव‚त्वा‚{\tiny $_{lb}$}‚दिति न स्थानास्थान‚क‚ल्प‚ना} ।
	{\color{gray}{\rmlatinfont\textsuperscript{§~\theparCount}}}
	\pend% ending standard par
      ‚{\tiny $_{lb}$}‚

	  
	  \pstart \leavevmode% starting standard par
	विशेषे विन‚श्य‚ति किम‚न्यापोह‚स्तिष्ठ‚ति । किम्वा निव‚र्त्त‚त इति । इयं स्थाना‚{\tiny $_{lb}$}‚स्थान‚{\tiny $_{१}$}‚क‚ल्प‚ना युक्ता । \textbf{त‚स्ये}त्य‚न्यापोह‚स्य । स्व‚भावानुष‚ङ्गिण्यो व‚स्त्व‚नुपातिन्यः ।‚{\tiny $_{lb}$}‚ विक‚ल्प‚बुद्ध्यारोपितं य‚त्सामान्य‚न्त‚त्र त‚र्हि \textbf{स्व‚रूप‚स्थितिप्र‚च्युतिक‚ल्प‚ना} भ‚विष्य‚ती‚{\tiny $_{lb}$}‚त्य‚त आह । \textbf{उप‚प्ल‚व‚श्चे}त्यादि । यः सामान्याकारोऽनेक‚प‚दार्थाभिन्नः प्र‚तिभास‚ते [।]‚{\tiny $_{lb}$}‚ \textbf{सामान्य‚धिय उप‚प्ल‚वो} भ्रान्तोऽतः सोपि ब‚हिर्नास्त्येव [।] य‚त एव\textbf{न्तेनापि} विप्ल‚व‚त्वेन‚{\tiny $_{lb}$}‚ कार‚णेन‚{\tiny $_{२}$}‚ सामान्य‚धीः । \textbf{अदूष‚णा} । नास्यां य‚थोक्तं दूष‚ण‚म‚स्तीति विग्र‚हः ।
	{\color{gray}{\rmlatinfont\textsuperscript{§~\theparCount}}}
	\pend% ending standard par
      ‚{\tiny $_{lb}$}‚

	  
	  \pstart \leavevmode% starting standard par
	इद‚मेकाकारं विक‚ल्प‚विज्ञानं \textbf{निर्विष‚यं} । य‚स्मा\textbf{न्मिथ्याज्ञान}म्मिथ्यार्थ‚मेव‚{\tiny $_{lb}$}‚ क‚थ‚मिति चेदाह । \textbf{य‚दि}त्यादि । य‚स्माद\textbf{नेक‚त्रैकाकारं । त‚द्विष‚य‚स्य} विक‚ल्प‚विज्ञान‚{\tiny $_{lb}$}‚विष‚य‚स्य सामान्य‚स्य \textbf{न स्थितिर‚स्थितिर्वा} । क‚स्मात् [।] \textbf{त‚स्य} मिथ्याज्ञान‚{\tiny $_{lb}$}‚\textbf{विष‚य‚स्याभावात्} । \href{http://sarit.indology.info/?cref=pv.3.168-3.169}{१७१-७२}
	{\color{gray}{\rmlatinfont\textsuperscript{§~\theparCount}}}
	\pend% ending standard par
      ‚{\tiny $_{lb}$}‚‚{\tiny $_{lb}$}‚\textsuperscript{\textenglish{328/s}}

	  
	  \pstart \leavevmode% starting standard par
	स‚मान‚दोष‚{\tiny $_{३}$}‚ताम‚प‚नीय पुनः प्र‚कारान्त‚रेण प्र‚क्रान्त‚ञ्चोद्यं प‚रिह‚र्त्तुमाह ।‚{\tiny $_{lb}$}‚ \textbf{य‚त्पुन‚रि}त्यादि । \textbf{त‚ज्ज‚न‚को हि} त‚स्य शाल्य‚ङ्कुर‚स्य ज‚न‚को हि स \textbf{त‚स्य} शालिबीज‚स्य‚{\tiny $_{lb}$}‚ \textbf{स्व‚भावः । य‚च्च त‚स्य} शालिबीज‚स्य शाल्य\textbf{ङ्कुर‚ज‚न‚कं} रूपं । \textbf{त‚तो} ज‚न‚काद् \textbf{रूपा‚{\tiny $_{lb}$}‚द‚न्यः} पृथिव्यादि\textbf{र्ज‚न‚कः क‚थ‚मिति । त‚त्रे}त्युप‚न्यासे । त‚त्र वा चोद्ये प्र‚तिवि‚{\tiny $_{lb}$}‚धीय‚ते । शालिबीजाद\textbf{न्य‚स्य‚{\tiny $_{४}$}‚} पृथिव्यादे\textbf{स्त‚ज्ज‚न‚क}म‚ङ्कुर‚ज‚न‚कं \textbf{रूपं नास्तीति‚{\tiny $_{lb}$}‚ न ब्रूमः} । किन्त‚र्हि \textbf{य‚देक‚स्य} शालिबीज‚स्य \textbf{ज‚न‚कं रूप‚न्त‚द‚न्य‚स्य} पृथिव्या\textbf{देर्नेति‚{\tiny $_{lb}$}‚ ब्रूमः} । भावानां प‚र‚स्प‚र‚म‚न‚न्व‚यात् । \textbf{अन्योपी}ति । पृथिव्यादिः । \textbf{न प‚र‚रूपे‚{\tiny $_{lb}$}‚णे}ति न शालिबीज‚रूपेण । किं कार‚ण‚म् [।] \textbf{अत‚त्वाद}त‚त्स्व‚भाव‚त्वात् । न चात्र‚{\tiny $_{lb}$}‚ बाध‚कं प्र‚माण‚म‚स्तीत्याह । \textbf{ते पृथिव्याद‚यो य‚था‚{\tiny $_{५}$}‚स्व}मिति [।] य‚स्य य‚त्स्व‚ल‚क्ष‚णं‚{\tiny $_{lb}$}‚ तेन \textbf{भिन्ना}श्च प‚र‚स्प‚र‚मेक‚स्य कार्य‚स्य \textbf{ज‚न‚काश्च स्व‚भावेनेति कोत्र विरोधो} न क‚श्चिद्‚{\tiny $_{lb}$}‚ बाध‚क‚प्र‚माणाभावात् । न तु शाल्य‚ङ्कुर‚ज‚न‚काभिम‚तेन शालिबीज‚रूपेण विक‚{\tiny $_{lb}$}‚ल‚स्य पृथिव्यादेः शाल्य‚ङ्कुर‚कार्य‚त्व‚म्विरुद्ध‚मित्य‚त आह । एक‚रूपेत्यादि । \textbf{एक‚स्य}‚{\tiny $_{lb}$}‚ शालिबीज‚स्यं ज‚न‚कं य‚द्रूप‚न्तेन \textbf{विक‚लः} पृथिव्यादि\textbf{स्त‚{\tiny $_{६}$}‚द्रूपः} शालिबीजादिस्व‚भावो‚{\tiny $_{lb}$}‚ \textbf{न स्यात् । नात‚त्कार्यः} किन्तु त‚त्कार्य एव । स शाल्य‚ङ्कुरः कार्य‚म‚स्येति विग्र‚हः ।‚{\tiny $_{lb}$}‚ य‚दि तु बीज‚स्यैवाङ्कुर‚ज‚न‚क‚त्वं स्यात् स्याद् विरोध‚स्त‚च्च नास्ति । त‚दाह । \textbf{तेनै‚{\tiny $_{lb}$}‚वेत्यादि} । शालिबीजेनैव त‚त्कार्य‚म‚ङ्कुराख्यं ।
	{\color{gray}{\rmlatinfont\textsuperscript{§~\theparCount}}}
	\pend% ending standard par
      ‚{\tiny $_{lb}$}‚

	  
	  \pstart \leavevmode% starting standard par
	\textbf{अपि चेत्या}दिना पूर्वोक्तं स्मार‚य‚ति ।
	{\color{gray}{\rmlatinfont\textsuperscript{§~\theparCount}}}
	\pend% ending standard par
      ‚{\tiny $_{lb}$}‚

	  
	  \pstart \leavevmode% starting standard par
	\leavevmode\ledsidenote{\textenglish{120b/PSVTa}} एकापाये फ‚लाभावाद् विशेषेभ्य‚स्त‚दुद्भ‚व इति [।] प्र‚माण‚ब‚लेन \textbf{वि‚{\tiny $_{७}$}‚शेषा‚{\tiny $_{lb}$}‚ ज‚न‚का} इत्युक्तं प्राक् । \textbf{तेनात्म‚ने}ति तेन ज‚न‚क‚रूपेण । \textbf{य‚देक‚स्य} बीज‚स्य \textbf{ज‚न‚कं‚{\tiny $_{lb}$}‚ ‚{\tiny $_{lb}$}‚ \leavevmode\ledsidenote{\textenglish{329/s}}रूप}म‚न्य‚स्य पृथिव्यादे\textbf{स्त‚न्नास्ति । न ताव‚ते}ति शालिबीज‚रूप‚वैक‚ल्य‚मात्रेणा \textbf{ज‚न‚कः}‚{\tiny $_{lb}$}‚ पृथिव्याद‚यः । \textbf{अप्य‚भेद} इत्यादि । तेषु भेदेष्व‚भेदोप्य‚स्तीत्य‚र्थः ।
	{\color{gray}{\rmlatinfont\textsuperscript{§~\theparCount}}}
	\pend% ending standard par
      ‚{\tiny $_{lb}$}‚

	  
	  \pstart \leavevmode% starting standard par
	\textbf{स्यादेत‚दित्या}दिना व्याच‚ष्टे । \textbf{त‚दे}वाभिन्नं रूप\textbf{मेका श‚क्ति}स्त‚या \textbf{योगात्} ।
	{\color{gray}{\rmlatinfont\textsuperscript{§~\theparCount}}}
	\pend% ending standard par
      ‚{\tiny $_{lb}$}‚

	  
	  \pstart \leavevmode% starting standard par
	\textbf{तेने}त्या चा र्यः । \textbf{तेना}भिन्नेन‚{\tiny $_{१}$}‚ रूपेण ते विशेषा \textbf{अज‚न‚काः प्रोक्तः} ।
	{\color{gray}{\rmlatinfont\textsuperscript{§~\theparCount}}}
	\pend% ending standard par
      ‚{\tiny $_{lb}$}‚

	  
	  \pstart \leavevmode% starting standard par
	\textbf{स‚त्य‚पी}त्यादिना व्याच‚ष्टे । \textbf{तेन सामान्य‚रूपेण ते} विशेषा \textbf{अज‚न‚काः} [।]‚{\tiny $_{lb}$}‚ किङ्कार‚णं [।] \textbf{त‚स्य} सामान्य‚रूप‚स्य नित्य‚त्वेना\textbf{न‚पायात्} । एत‚च्च नापैत्य‚भिन्न‚{\tiny $_{lb}$}‚न्त‚द्रूप‚म्वि\textbf{शेषाः ख‚ल्व‚पायिन} \href{http://sarit.indology.info/?cref=}{१ । १६७} इत्यादिना प्रागुक्तं । \href{http://sarit.indology.info/?cref=pv.3.169-3.170}{१७२-७३}
	{\color{gray}{\rmlatinfont\textsuperscript{§~\theparCount}}}
	\pend% ending standard par
      ‚{\tiny $_{lb}$}‚

	  
	  \pstart \leavevmode% starting standard par
	\hphantom{.}\textbf{स्यातां नाशोद्भ‚वौ} स‚कृदि \href{http://sarit.indology.info/?cref=}{१ । १७०}त्यादिना \textbf{विरुद्ध‚ध‚र्माध्यासाद्} भेद‚{\tiny $_{lb}$}‚म्प्र‚साध्य प्र‚तिभास‚भेदेनापि साध‚यितु‚{\tiny $_{२}$}‚माह । \textbf{किं चे}त्यादि । किम्विशिष्टः प्र‚ति‚{\tiny $_{lb}$}‚भास‚भेदः [।] \textbf{अन‚न्य‚भाक्} । प्र‚तिव्य‚क्ति भिन्नः । \textbf{अपि}श‚ब्दादुत्प‚त्त्यादिभेद‚श्च‚{\tiny $_{lb}$}‚ भेद‚कः । एत‚च्च पूर्वोक्त‚मेव स‚मुच्चीय‚ते ।
	{\color{gray}{\rmlatinfont\textsuperscript{§~\theparCount}}}
	\pend% ending standard par
      ‚{\tiny $_{lb}$}‚

	  
	  \pstart \leavevmode% starting standard par
	योपि वे दा न्त वा द्याह । अभाव‚ग्र‚ह‚ण‚निमित्त‚को हि भेद‚ग्र‚हो न चाभावः‚{\tiny $_{lb}$}‚ प्र‚त्य‚क्ष‚ग्राह्यः । तेन प्र‚त्य‚क्षेण स‚त्तामात्र‚ङ्गृह्य‚ते न भेदः । त‚दुक्तं ।
	{\color{gray}{\rmlatinfont\textsuperscript{§~\theparCount}}}
	\pend% ending standard par
      ‚{\tiny $_{lb}$}‚

	  
	  \pstart \leavevmode% starting standard par
	\hphantom{.}ग‚व्य‚श्वे वोप‚जात‚न्तु प्र‚त्य‚क्षं न विशिष्य‚ते [।]
	{\color{gray}{\rmlatinfont\textsuperscript{§~\theparCount}}}
	\pend% ending standard par
      ‚{\tiny $_{lb}$}‚

	  
	  \pstart \leavevmode% starting standard par
	इत्य‚त आह‚{\tiny $_{३}$}‚ । \textbf{न्याय}मित्यादि । \textbf{त‚स्ये}त्य‚भेद‚वादिनः । \textbf{बुद्धिप्र‚तिभास‚भेदो}‚{\tiny $_{lb}$}‚ बुद्धेराकार‚भेदः । विरुद्ध‚ध‚र्माध्यास‚श्चेति पृथ‚गुत्प‚त्तिविनाशादिकः । \textbf{स‚ति वा त‚स्मिन्}‚{\tiny $_{lb}$}‚ प्र‚तिभासादिभेदे भावाना\textbf{म‚भेदे}भ्युप‚ग‚म्य‚माने \textbf{न क्व‚चिद्} भेदः \textbf{स्यात्} । लोक‚प्र‚ती‚{\tiny $_{lb}$}‚त‚श्च भेदः । \textbf{त‚था चे}त्य‚भेदे स‚ति \textbf{अयं प्र‚विभाग} इति प्र‚तिभासादिप्र‚विभागः ।‚{\tiny $_{lb}$}‚ \textbf{एकात्म‚व‚त्} । य‚थैक‚स्मिन् सुखात्म‚{\tiny $_{४}$}‚नि न प्र‚तिभासादिभेद‚स्त‚द्व‚त् । \textbf{त‚स्माद‚यं} बाह्या‚{\tiny $_{lb}$}‚ध्यात्मिको भेदो \textbf{विशेष एव} प‚र‚स्प‚र‚विल‚क्ष‚ण एव । किंभूतः [।] \textbf{भिन्न‚प्र‚तिभा‚{\tiny $_{lb}$}‚‚{\tiny $_{lb}$}‚ \leavevmode\ledsidenote{\textenglish{330/s}}सादिः} । भिन्नः प्र‚तिभासादिर्य‚स्येति विग्र‚हः ।
	{\color{gray}{\rmlatinfont\textsuperscript{§~\theparCount}}}
	\pend% ending standard par
      ‚{\tiny $_{lb}$}‚

	  
	  \pstart \leavevmode% starting standard par
	निर्विक‚ल्प‚क‚बोधेन द्व्यात्म‚क‚स्यापि व‚स्तुनो ग्र‚ह‚णाद‚भेदोपीति चेदाह ।‚{\tiny $_{lb}$}‚ न चात्रेत्यादि । अत्रेति भेदेषु । अप‚र‚मिति द्वितीयं । य‚द्व‚लेनाभिन्न‚प्र‚तिभास‚{\tiny $_{lb}$}‚व‚लेन । त‚तो विशेष‚{\tiny $_{५}$}‚ एव भेद एव । न त्व‚भेदोस्ति । स एव विशेषो व‚स्तु ।
	{\color{gray}{\rmlatinfont\textsuperscript{§~\theparCount}}}
	\pend% ending standard par
      ‚{\tiny $_{lb}$}‚

	  
	  \pstart \leavevmode% starting standard par
	ये त्व‚प‚रे सामान्य‚ध‚र्मा व‚स्तुत्वाद‚य‚स्त‚स्यैव व्यावृत्त‚यः क‚ल्पिताः । \href{http://sarit.indology.info/?cref=pv.3.170-3.171}{१७३-७४}
	{\color{gray}{\rmlatinfont\textsuperscript{§~\theparCount}}}
	\pend% ending standard par
      ‚{\tiny $_{lb}$}‚

	  
	  \pstart \leavevmode% starting standard par
	त‚त्कार्य‚मित्यादि । कार्यादिप‚द‚सामानाधिक‚र‚ण्यान्न‚पुम‚क । अन्य‚था विशे‚{\tiny $_{lb}$}‚श‚स्य प्र‚क्रान्त‚त्वात्स इति स्यात् । त‚देव विशेष‚रूपं कार्यं कार‚णं चोक्त‚न्त‚देव स्व‚ल‚{\tiny $_{lb}$}‚क्ष‚ण‚मुच्य‚ते [।] त‚त्त्यागाप्तिफ‚ला इति त‚स्यैव विशेष‚स्य हेय‚स्यो‚{\tiny $_{६}$}‚ पादेय‚स्य वा‚{\tiny $_{lb}$}‚ य‚थाक्र‚मं त्याग आप्तिश्च फ‚लं यासाम्प्र‚वृत्तीनान्तास्त‚थोक्ताः । स चार्थ‚क्रिया‚{\tiny $_{lb}$}‚कारी \textbf{विशेष एव । त‚स्यैवेति} विशेष‚स्याप‚र‚स्माद् विजातीयाद् भेदो व्यावृति‚{\tiny $_{lb}$}‚मात्रं । न तु व‚स्तुभूतं किञ्चित् सामान्यं नाम । \href{http://sarit.indology.info/?cref=pv.3.171-3.172}{१७४-७५}
	{\color{gray}{\rmlatinfont\textsuperscript{§~\theparCount}}}
	\pend% ending standard par
      ‚{\tiny $_{lb}$}‚

	  
	  \pstart \leavevmode% starting standard par
	य‚दि हि स्यात् त‚दुप‚ल‚ब्धिल‚क्ष‚ण‚प्राप्त‚म्भेद‚व्य‚तिरेकिणोप‚ल‚भ्येत । हि य‚स्मा‚{\tiny $_{lb}$}‚\leavevmode\ledsidenote{\textenglish{121a/PSVTa}} त् । न हि त‚स्य सामान्य‚स्यार्थ‚त्वे व‚स्तुत्वे स‚ति \textbf{दृश्य‚स्य स‚तः । रू‚{\tiny $_{७}$}‚पानुप‚ल‚क्ष‚णं}‚{\tiny $_{lb}$}‚ स्व‚रूपाग्र‚ह‚णं युक्तं । क‚स्मात् । त‚दुप‚ल‚क्ष‚ण‚कृत‚त्वात् सामान्योप‚ल‚क्ष‚ण‚कृत‚त्वाद्‚{\tiny $_{lb}$}‚ \textbf{भेदेषु} भेद‚प्र‚त्य‚य‚स्य । न हि स्व‚य‚म‚गृहीत‚म्प‚र‚त्र ज्ञान‚हेतुः ।
	{\color{gray}{\rmlatinfont\textsuperscript{§~\theparCount}}}
	\pend% ending standard par
      ‚{\tiny $_{lb}$}‚

	  
	  \pstart \leavevmode% starting standard par
	एव‚न्ताव‚त् मी मां स का दिम‚तेन प्रातिभासिकं सामान्यं निराकृत्यानुमानिक‚{\tiny $_{lb}$}‚म‚पि पूर्वोक्तं सां ख्य म‚तेन निराक‚र्त्तुमाह । \textbf{अपि चे}त्यादि । य‚था सांख्य‚स्य भेदा‚{\tiny $_{lb}$}‚विशेषेपि न \textbf{स‚र्व स‚र्व‚साध‚नं । त‚था} बौ द्ध‚{\tiny $_{१}$}‚स्य \textbf{भेदाविशेषेपि न स‚र्व स‚र्व‚साध‚नं । क‚स्य}‚{\tiny $_{lb}$}‚ पुन‚श्चोद्य‚स्यायं स‚माधिरित्याह । य‚दुक्त‚मित्यादि । त‚ज्ज‚न‚क‚स्व‚भावादिति [।]‚{\tiny $_{lb}$}‚ ‚{\tiny $_{lb}$}‚ \leavevmode\ledsidenote{\textenglish{331/s}}शाल्य‚ङ्कुर‚ज‚न‚न‚स्व‚भावाद् भिन्नः पृथिव्यादिः । अस्येत्य‚ङ्कुर‚स्य । \textbf{ज‚न‚क‚त्वे} वाभ्यु‚{\tiny $_{lb}$}‚प‚ग‚म्य‚माने भेद‚स्या\textbf{विशेषात् स‚र्वो ज‚न‚कः स्या}दिति । \textbf{अत्र} चोद्ये उक्त‚मेवोत्त‚रं ।‚{\tiny $_{lb}$}‚ इद‚न्तु द्वितीय‚मुच्य‚ते । किं पुन‚रुक्त‚मित्याह । य‚दीत्यादिं । प्र‚तिनिय‚त‚{\tiny $_{२}$}‚कार्य‚ज‚न‚का‚{\tiny $_{lb}$}‚ज‚न‚क‚त्व‚ल‚क्ष‚णो विशेषो न स्यात् । \href{http://sarit.indology.info/?cref=pv.3.172-3.173}{१७५-७६}
	{\color{gray}{\rmlatinfont\textsuperscript{§~\theparCount}}}
	\pend% ending standard par
      ‚{\tiny $_{lb}$}‚

	  
	  \pstart \leavevmode% starting standard par
	स्यादेत‚च्चोद्य‚मिति ।
	{\color{gray}{\rmlatinfont\textsuperscript{§~\theparCount}}}
	\pend% ending standard par
      ‚{\tiny $_{lb}$}‚

	  
	  \pstart \leavevmode% starting standard par
	य‚थेत्यादिना श्लोकार्थ‚माह । \textbf{त‚था विशेषेपि भ‚विष्य‚ति । न स‚र्वः स‚र्व‚ज‚न‚क}‚{\tiny $_{lb}$}‚ इति स‚म्ब‚न्धः । \textbf{व‚स्तुध‚र्म‚त‚ये}ति व‚स्तुश‚क्त्या । भावाना\textbf{म‚भेदे} त्व‚भ्युप‚ग‚म्य‚माने ।‚{\tiny $_{lb}$}‚ त‚स्य स‚र्व‚त्राभिन्न‚त्वेनाभ्युप‚ग‚त‚स्यैक‚त्र क्रियाक्रिये विरुध्येते । \href{http://sarit.indology.info/?cref=pv.3.173-3.174}{१७६-७७}
	{\color{gray}{\rmlatinfont\textsuperscript{§~\theparCount}}}
	\pend% ending standard par
      ‚{\tiny $_{lb}$}‚

	  
	  \pstart \leavevmode% starting standard par
	\textbf{भेद‚मात्रे}त्यादिना व्याच‚ष्टे । हेतुरुपादान‚कार‚णं । \textbf{प्र‚त्य‚या}‚{\tiny $_{३}$}‚स्स‚ह‚कारिणः [।]‚{\tiny $_{lb}$}‚ \textbf{स्वेहेतुप्र‚त्य‚याः} स्व‚हेतुप्र‚त्य‚यास्तै\textbf{निंय‚मितो} विशिष्ट‚कार्य‚निर्व‚र्त्त‚न‚स‚म‚र्थः कृतः \textbf{स्व‚भावो}‚{\tiny $_{lb}$}‚ येषान्ते त‚थोक्ताः । त‚द्भाव‚स्त‚स्मात् । \textbf{नान्य} इत्य‚कार‚काभिम‚ता न \textbf{कार‚का स्युः ।‚{\tiny $_{lb}$}‚ किङ्कार‚ण‚म्} [।] अत‚त्स्व‚भाव‚त्वात् । अत‚त्कार्य‚ज‚न‚न‚स्व‚भाव‚त्वात् । त‚स्येत्येक‚स्य‚{\tiny $_{lb}$}‚ त्रैगुण्य‚स्य । \textbf{त‚त्रैवे}त्येक‚स्मिन्नेव कार्ये । त‚थेति तेनैवाभिन्नेन प्र‚कारेण । त‚थापि‚{\tiny $_{lb}$}‚ य‚दा शा‚{\tiny $_{४}$}‚लिबीजं शाल्य‚ङ्कुरं ज‚न‚य‚ति त‚दैव न य‚व‚वीजं शाल्य‚ङ्कुरं ज‚न‚य‚ति ।‚{\tiny $_{lb}$}‚ य‚श्च शालिबीज‚स्यात्मा । स एव य‚व‚बीज‚स्येत्येक‚त्रैक‚स्य क्रियाक्रिये प्र‚स‚ज्येते ।‚{\tiny $_{lb}$}‚ ‚{\tiny $_{lb}$}‚ \leavevmode\ledsidenote{\textenglish{332/s}}त्रैगुण्य‚स्य तेन तेन शालिवीजादिस‚न्निवेशेन भेदोप्य‚स्ति । अतो भेदात् क‚स्य‚चिद्‚{\tiny $_{lb}$}‚ क्रिया चेत् । भेद‚श्चेद‚क्रियाहेतुर्न कुर्युः स‚ह‚कारिणः । तेषाम‚पि प‚र‚स्प‚रं भेदात् ।
	{\color{gray}{\rmlatinfont\textsuperscript{§~\theparCount}}}
	\pend% ending standard par
      ‚{\tiny $_{lb}$}‚

	  
	  \pstart \leavevmode% starting standard par
	\textbf{ने}त्यादिना व्याच‚ष्टे । स‚र्वाकारा धियः‚{\tiny $_{५}$}‚ किन्त‚स्यैवैक‚स्य व‚स्तुनः । नैवेत्य‚र्थः ।‚{\tiny $_{lb}$}‚ भेदाधिष्ठान‚त्वात् । प‚र्याय‚स्येति भावः ।
	{\color{gray}{\rmlatinfont\textsuperscript{§~\theparCount}}}
	\pend% ending standard par
      ‚{\tiny $_{lb}$}‚

	  
	  \pstart \leavevmode% starting standard par
	\textbf{अथे}त्यादिना व्याच‚ष्टे । \textbf{स‚र्वेषा}म्भेदानां \textbf{स‚र्व‚त्र} कार्ये \textbf{प‚र्यायेण क्र‚मेणोप‚योगात्} ।‚{\tiny $_{lb}$}‚ एत‚च्च य‚दा प्र धा न श‚क्त्याधिष्ठितानामेव भेदानाम‚प‚राप‚र‚प‚रिणामेन कार्य‚क‚र्त्तृत्व‚{\tiny $_{lb}$}‚मिति द‚र्श‚न‚न्त‚दोक्तं । य‚दा त्विदं द‚र्श‚न‚म्प्र‚धान‚श‚क्तिरेवाप‚राप‚र‚रूपेण प‚रिणामार्थ‚{\tiny $_{lb}$}‚\leavevmode\ledsidenote{\textenglish{121b/PSVTa}} क्रिया‚{\tiny $_{७}$}‚यामुप‚युज्य‚ते त‚देद‚मुच्य‚ते [।] \textbf{श‚क्ते}र्वेत्यादि । त्रैगुण्य‚ल‚क्ष‚णायास्त\textbf{न्निवेशिन्याः}‚{\tiny $_{lb}$}‚ पूर्व‚म‚कार‚काभिम‚त‚प‚दार्थ‚निवेशिन्याः प‚श्चाद् \textbf{रूपान्त}रेण कार‚काभिम‚त‚रूपेण‚{\tiny $_{lb}$}‚ \textbf{प‚रिण‚ताया} उप‚योगान्नैव क‚श्चिद‚कार‚कोस्तीत्य‚नेन स‚म्ब‚न्धः । \textbf{भेदो} नानात्व‚माश्र‚यो‚{\tiny $_{lb}$}‚ य‚स्य \textbf{प‚र्याय‚स्य} स त‚थोक्तः । एक‚स्याभेद‚स्य \textbf{क‚थ‚न्नैव} । शालिबी\textbf{ज‚स्यैक‚स्य}‚{\tiny $_{lb}$}‚ य‚व‚बीजादिरूप‚त‚या प‚रिणामो न यु‚{\tiny $_{१}$}‚क्त इत्य‚र्थ ।
	{\color{gray}{\rmlatinfont\textsuperscript{§~\theparCount}}}
	\pend% ending standard par
      ‚{\tiny $_{lb}$}‚

	  
	  \pstart \leavevmode% starting standard par
	एतेन \textbf{स किन्त‚स्यैव व‚स्तुन} इत्येत‚द् विवृतं । \href{http://sarit.indology.info/?cref=pv.3.174-3.175}{१७७-७८}
	{\color{gray}{\rmlatinfont\textsuperscript{§~\theparCount}}}
	\pend% ending standard par
      ‚{\tiny $_{lb}$}‚

	  
	  \pstart \leavevmode% starting standard par
	\textbf{श‚क्ते}र्वेति य‚दुक्त‚न्त‚त्राह । \textbf{प‚रिणामो} वेति । \textbf{अव्य‚तिरेकि}ण्या इति निर्वि‚{\tiny $_{lb}$}‚भागायाः श‚क्तेः प‚रिणामो वाव‚स्थानान्त‚र‚प्राप्तिर्वा क‚थं । अथेष्य‚तेऽव‚स्थान्त‚राणां‚{\tiny $_{lb}$}‚ प्राप्तिरात्म‚भूतैव त्रैगुण्य‚स्य । त‚तो \textbf{विशेषे} वा \textbf{क‚थंचिद}भ्युप‚ग‚म्य‚ते । प्र‚धान\textbf{स्यैक‚{\tiny $_{lb}$}‚त्व‚हानिरिति} ।
	{\color{gray}{\rmlatinfont\textsuperscript{§~\theparCount}}}
	\pend% ending standard par
      ‚{\tiny $_{lb}$}‚

	  
	  \pstart \leavevmode% starting standard par
	एव‚न्ताव‚त्प‚रिणाम‚प‚क्षं निराकृत्याधुनाऽभि‚{\tiny $_{२}$}‚न्न‚म्वा भिन्नाभिन्न‚म्वा भिन्न‚म्वा‚{\tiny $_{lb}$}‚ स‚र्वासु चोत्त‚रोत्त‚राव‚स्थास्व‚नुयायित्वादू\textbf{र्द्ध}\edtext{}{\lemma{नुयायित्वादू}\Bfootnote{? र्ध्व}}वृत्ति वा । \textbf{स‚मं स‚र्वासु‚{\tiny $_{lb}$}‚ व्य‚क्ति}ष्व‚नुयायित्वात् तिर्य‚ग्वृत्ति वा सामान्य‚म‚भ्युप‚ग‚म्य सां ख्य मी मां स क नै‚{\tiny $_{lb}$}‚ या यि काद्य‚भिम‚तं दूष‚यितुमाह । किंचेत्यादि ।
	{\color{gray}{\rmlatinfont\textsuperscript{§~\theparCount}}}
	\pend% ending standard par
      ‚{\tiny $_{lb}$}‚

	  
	  \pstart \leavevmode% starting standard par
	तेन योपि दि ग म्ब रो म‚न्य‚ते [।] नास्त्याभिर्घ‚ट‚प‚टादिष्वेकं सामान्य‚मिष्य‚ते‚{\tiny $_{lb}$}‚ तेषामेकान्त‚भेदात् । किन्त्व‚प‚राप‚रेण प‚र्यायेणाव‚स्थासं‚{\tiny $_{३}$}‚ज्ञितेन प‚रिणामि द्र‚व्य‚मेत‚देव‚{\tiny $_{lb}$}‚ च स‚र्व‚प‚र्यायानुयायित्वात् सामान्य‚मुच्य‚ते । त‚था हि सुव‚र्ण्णात्म‚कं घ‚टं भ‚ङ्क्त्त्वा‚{\tiny $_{lb}$}‚ ‚{\tiny $_{lb}$}‚ \leavevmode\ledsidenote{\textenglish{333/s}}मौलिनिर्व‚र्त्त‚ने त‚देव सुव‚र्ण्ण‚द्र‚व्यं घ‚ट‚रूप‚त‚या विन‚श्य मौलिरूप‚त‚योत्प‚द्य‚मानं‚{\tiny $_{lb}$}‚ सुव‚र्ण‚स्व‚भावेन तिष्ठ‚तीत्य‚प‚राप‚राव‚स्थायाः प‚रिणामि । त‚त्सामान्य‚मित्युच्य‚ते ।‚{\tiny $_{lb}$}‚ प‚रिणामित्वादेव चाव‚स्थात‚द्व‚तोर‚भेदोन्य‚थाव‚स्थातुः स‚काशाद‚व‚स्था‚{\tiny $_{४}$}‚या भेदे‚{\tiny $_{lb}$}‚ प‚रिणामायोगात् । घ‚टात्म‚त‚या च सुव‚र्ण्ण‚द्र‚व्य‚स्य विनाशुप‚टार्थी शोकं प्र‚तिप‚द्य‚ते ।‚{\tiny $_{lb}$}‚ मौलिरूप‚त‚योत्पादे त‚द‚र्थी प्रामोद्यं प्र‚तिप‚द्य‚ते [।] सुव‚र्ण्ण‚त‚या च विनाशोत्पादाभावे‚{\tiny $_{lb}$}‚ सुव‚र्ण्णार्थी माध्य‚स्थ्यं प्र‚तिप‚द्य‚ते [।] तेन युग‚प‚दुत्पाद‚व्य‚य‚ध्रौव्य‚युक्तं स‚दिति‚{\tiny $_{lb}$}‚ व‚स्तुनो ल‚क्ष‚ण‚मिति । त‚दाह ।
	{\color{gray}{\rmlatinfont\textsuperscript{§~\theparCount}}}
	\pend% ending standard par
      ‚{\tiny $_{lb}$}‚
	  \bigskip
	  \begingroup
	
	    
	    \stanza[\smallbreak]
	  {\normalfontlatin\large ``\qquad}घ‚ट‚मौलिसुव‚र्ण्णार्थी वि\edtext{}{\lemma{वि}\Bfootnote{?}}नाशोत्पाद‚स्थितिष्व‚यं ।&‚{\tiny $_{lb}$}‚शोक‚प्र‚मोद‚माध्य‚स्थं ज‚नो याति स‚हेतुकं ।&‚{\tiny $_{lb}$}‚न नाशेन विना शोको नोत्पादेन विना सुखं ।&‚{\tiny $_{lb}$}‚स्थित्या विना न माध्य‚स्थ‚न्त‚स्माद् व‚स्तु त्र‚यात्म‚कं ।&‚{\tiny $_{lb}$}‚प‚योव्र‚तो न द‚ध्य‚त्ति न प‚योत्ति द‚धिव्र‚तः ।&‚{\tiny $_{lb}$}‚अगोर‚स‚व्र‚तो नोभे त‚स्माद् व‚स्तु त्र‚यात्म‚कं ।&‚{\tiny $_{lb}$}‚न सामान्यात्म‚नोदेति न व्येति व्य‚क्त‚म‚न्व‚यात् ।&‚{\tiny $_{lb}$}‚वे व्येत्युदेति विशेषेण स‚हैक‚त्रोद‚यादि स‚दिति ।{\normalfontlatin\large\qquad{}"}\&[\smallbreak]
	  
	  
	  
	  \endgroup
	‚{\tiny $_{lb}$}‚

	  
	  \pstart \leavevmode% starting standard par
	सोप्य‚त्र निराकृत‚{\tiny $_{६}$}‚ एव द्र‚ष्ट‚व्यः । \textbf{त‚द्व‚ति} सामान्य‚विशेष‚व‚ति व‚स्तुन्य‚भ्युप‚{\tiny $_{lb}$}‚ग‚म्य‚माने । \textbf{अत्य‚न्त‚म‚भेदाभेदौ} स्यातां । विशेषेभ्यो घ‚ट‚प‚टादिभ्यः सामान्य‚स्य‚{\tiny $_{lb}$}‚ त्रैगुण्यादिल‚क्ष‚ण‚स्याव्य‚तिरेकात् सामान्य‚म‚पि विशेषात्म‚क‚मित्य‚त्य‚न्त‚भेदः स्यात् ।‚{\tiny $_{lb}$}‚ सामान्य‚स्याभावात् । सामान्याद् विशेषाणाम‚व्य‚क्तिरेकादैक्य‚मित्य‚न्ताभेदो विशेषा‚{\tiny $_{lb}$}‚णाम‚भावात् । एक‚म्भेद‚सामान्यात्म‚कं‚{\tiny $_{७}$}‚ नास्तीति याव‚त् । \leavevmode\ledsidenote{\textenglish{122a/PSVTa}}
	{\color{gray}{\rmlatinfont\textsuperscript{§~\theparCount}}}
	\pend% ending standard par
      ‚{\tiny $_{lb}$}‚

	  
	  \pstart \leavevmode% starting standard par
	अथ सामान्य‚विशेष‚योः क‚थंचिद् भेद इष्य‚ते । अत्राप्याह । \textbf{अन्योन्य}मित्यादि ।‚{\tiny $_{lb}$}‚ \textbf{स‚दृशास‚दृशात्म‚नो}स्सामान्य‚विशेष‚योर्य‚दि क‚थंचिद‚न्योन्य‚म्प‚र‚स्प‚र‚म्भेद‚स्त‚दैकान्तेन‚{\tiny $_{lb}$}‚ त‚योर्भेद एव स्यात् । घ‚ट‚प‚ट‚व‚त् । न चार्थान्त‚रं सामान्यं प्र‚तीय‚ते । त‚स्मान्नैकं सामा‚{\tiny $_{lb}$}‚न्य‚विशेषात्म‚कं व‚स्तु विद्य‚ते । दि ग म्ब र स्यापि त‚द्व‚ति व‚स्तुन्य‚भ्युप‚ग‚म्य‚मानेऽत्य‚{\tiny $_{lb}$}‚न्त‚{\tiny $_{१}$}‚भेदाभेदौ स्यातां । य‚दा घ‚टाद्य‚व‚स्थाभेदेभ्यः सुव‚र्ण्ण‚त्व‚सामान्य‚स्याभेद‚स्त‚दात्य‚{\tiny $_{lb}$}‚न्त‚मेकान्तेन भेदः स्याद् घ‚ट‚मौल्यादेः । सुव‚र्ण्ण‚त्व‚सामामान्य‚स्याभावात् । अथ‚{\tiny $_{lb}$}‚ सुव‚र्ण्ण‚त्व‚सामान्याद् घ‚ट‚मौल्याद्य‚व‚स्था भेदानाम‚भेद‚स्त‚दात्य‚न्त‚म‚भेद एकान्तेनैक‚त्वं‚{\tiny $_{lb}$}‚ सुव‚र्ण्ण‚रूप‚तैव स्यादित्य‚र्थः । अथाव‚स्थात‚द्व‚तोः स्व‚भावाभेदे स‚त्य‚पि ल‚क्ष‚ण‚भेदाद्‚{\tiny $_{lb}$}‚ भेद इष्य‚ते । त‚था हि सुव‚र्ण्ण‚त्व‚{\tiny $_{२}$}‚ सामान्य‚स्य स्व‚रूपं स‚र्वाव‚स्थानुयायि प्र‚तीय‚ते [।]‚{\tiny $_{lb}$}‚ ‚{\tiny $_{lb}$}‚ \leavevmode\ledsidenote{\textenglish{334/s}}घ‚टाद्य‚व‚स्थानां स्व‚रूपं व्यावृत्तं प्र‚तीय‚ते । तेनाव‚स्थात‚द्व‚तोर्ल‚क्ष‚ण‚भेदाद् भेदोस्त्येवे‚{\tiny $_{lb}$}‚त्य‚त्राह । \textbf{अन्योन्य}मित्यादि । \textbf{स‚दृशास‚दृशात्म‚नो}रित्य‚व‚स्थात‚द्व‚तोर्य‚दि भेद‚स्त‚दा‚{\tiny $_{lb}$}‚ त‚योर‚न्योन्य‚म्भेदः प‚र‚स्प‚र‚मेकान्तेन भेदः स्यात् । अनुग‚त‚व्यावृत्तिरूप‚योः प‚र‚स्प‚रा‚{\tiny $_{lb}$}‚संश्लेषात् । न चाप‚रः स्व‚भावोस्वि येन त‚योर‚भे‚{\tiny $_{३}$}‚दः स्याद‚न‚न्त‚ध‚र्मात्म‚क‚स्य ध‚र्मिणो‚{\tiny $_{lb}$}‚ऽप्र‚तीते \href{http://sarit.indology.info/?cref=}{ः} । \textbf{भावाश्चेद् भेदिन} इति स‚म्ब‚न्धः । \textbf{अभिन्नेनात्म‚ना} प्र‚धानाख्ये‚{\tiny $_{lb}$}‚नान्येन वा । व‚स्तुत्वादिना सुव‚र्ण्ण‚त्वेन वा तेषामेव घ‚टादीनाम्भेदानां \textbf{स्वात्म‚{\tiny $_{lb}$}‚भूतेना}व्य‚तिरिक्तेन त‚द्व‚न्तः स्यु\textbf{र‚भिन्न‚स्व‚भाव‚व‚न्तः} स्युः । तेषां प्र‚धानादीनाम‚भिन्नः‚{\tiny $_{lb}$}‚ स्व‚भाव‚स्त‚द‚भिन्न‚स्व‚भावः आत्मा रूपं य‚स्य भेद‚स्य घ‚टादिल‚क्ष‚ण‚स्य स \textbf{त‚द‚भि‚{\tiny $_{४}$}‚न्न‚{\tiny $_{lb}$}‚स्व‚भावात्मा} । त‚द्भाव‚स्त‚स्माद् [।] भेद‚स्यापि कुतो भेदः प‚र‚स्प‚रं । नैव । अने‚{\tiny $_{lb}$}‚नात्य‚न्ताभेदो व्याख्यातः ।
	{\color{gray}{\rmlatinfont\textsuperscript{§~\theparCount}}}
	\pend% ending standard par
      ‚{\tiny $_{lb}$}‚

	  
	  \pstart \leavevmode% starting standard par
	अथ त‚स्य भेद‚स्य घ‚टादिल‚क्ष‚ण‚स्य स‚मान एक आत्मा न भ‚व‚ति । भेद‚स्य‚{\tiny $_{lb}$}‚ घ‚टादिरूपेणानेकात्म‚क‚त्वात् । त‚था स‚ति \textbf{त‚दात्म‚ना} भेद‚स्व‚भावेन \textbf{तेनापि} सामान्य‚{\tiny $_{lb}$}‚प‚दार्थेन त्रैगुण्यादिना \textbf{त‚थेति} सामान्यात्म‚ना \textbf{भ‚वितुन्न युक्तं} । भेदाद‚व्य‚तिरिक्त‚{\tiny $_{५}$}‚‚{\tiny $_{lb}$}‚त्वात् सामान्य‚स्य स‚मान‚ता न प्राप्नोतीत्य‚र्थः । एतेनात्य‚न्त‚भेदो व्याख्यातः ।
	{\color{gray}{\rmlatinfont\textsuperscript{§~\theparCount}}}
	\pend% ending standard par
      ‚{\tiny $_{lb}$}‚

	  
	  \pstart \leavevmode% starting standard par
	\textbf{त‚थाभावे} हीति सामान्यात्म‚क‚त्वे प्र‚धानादेरिष्य‚माणे प्र‚धानाद्यात्मा\textbf{ऽत‚द्ध‚र्मा}‚{\tiny $_{lb}$}‚ भेद‚ध‚र्मा \textbf{न स्यात्} । अव्य‚तिरेकिणाव‚स्थाख्येन ध‚र्मेण त‚द्वान्न स्यादित्य‚र्थः । अव‚{\tiny $_{lb}$}‚स्थात‚द्व‚तोः प‚र‚स्प‚र‚तो भेदः स्यादिति याव‚त् । त‚मेव साध‚य‚न्नाह । \textbf{न ह्य‚य}मित्यादि ।‚{\tiny $_{lb}$}‚ \textbf{अय‚मेकः} स्व‚{\tiny $_{६}$}‚भावः \textbf{प्र‚वृत्तिनिवृत्तिमान्न युक्त} इति स‚म्ब‚न्धः । \textbf{स्थानं} प्र‚वृत्तिः ।‚{\tiny $_{lb}$}‚ \textbf{विग‚मो} निवृत्तिः । त‚था हि प्र‚धान‚श‚क्तौ स्थितायां सुव‚र्ण्ण‚द्र‚व्य‚त्वादौ च स्थितेऽ‚{\tiny $_{lb}$}‚व‚स्थानान्निवृत्तिरिष्य‚ते । \textbf{एतेनान्योन्य‚म्वा त‚योर्भेद} इत्यादि व्याख्यातं ।
	{\color{gray}{\rmlatinfont\textsuperscript{§~\theparCount}}}
	\pend% ending standard par
      ‚{\tiny $_{lb}$}‚

	  
	  \pstart \leavevmode% starting standard par
	\textbf{ने}त्यादिना प‚राभिप्राय‚माशंक‚ते । \textbf{न स‚र्वात्म‚ना} सामान्य‚विशेष‚यो\textbf{र‚भेद एव}‚{\tiny $_{lb}$}‚ \leavevmode\ledsidenote{\textenglish{122b/PSVTa}} किन्तु \textbf{त‚योर‚पि} भेद‚सामान्य‚यो\textbf{र्भेदो भ‚वे‚{\tiny $_{७}$}‚द्य‚दि । न ही}त्यादिना व्याच‚ष्टे । क्व‚चिद्‚{\tiny $_{lb}$}‚ द्र‚व्ये सामान्य‚विशेष‚स्य प‚र‚स्प‚र‚म्\textbf{भेदोऽभेदो वैकान्तिको} न हीति स‚म्ब‚न्धः । किङ्का‚{\tiny $_{lb}$}‚र‚णं [।] \textbf{विवेकिने}त्यादि । \textbf{सामान्यं} श‚क्तिः सुव‚र्ण्ण‚त्व‚न्द्र‚व्य‚त्व‚ञ्चाविशेषो घ‚टाद‚य‚{\tiny $_{lb}$}‚ इत्येवंभेदेन \textbf{व्य‚व‚स्थाप‚नात्} ।
	{\color{gray}{\rmlatinfont\textsuperscript{§~\theparCount}}}
	\pend% ending standard par
      ‚{\tiny $_{lb}$}‚‚{\tiny $_{lb}$}‚\textsuperscript{\textenglish{335/s}}

	  
	  \pstart \leavevmode% starting standard par
	\textbf{येने}त्या चा र्यः । \textbf{त‚योर्भेद}सामान्य‚योर‚यं \textbf{भेद} इदं \textbf{सामान्य‚मित्येत‚द्ये}नात्म‚ना‚{\tiny $_{lb}$}‚ व्यावृत्तेनानुग‚तेन च स्व‚भावेन भेदो व्य‚व‚{\tiny $_{१}$}‚स्थाप्य‚ते । \textbf{य‚दि तेनात्म‚ना} सामान्य‚{\tiny $_{lb}$}‚विशेष‚योर्भेद‚स्त‚दा \textbf{भेद} एवात्य‚न्तं । \href{http://sarit.indology.info/?cref=pv.3.176}{१७९}
	{\color{gray}{\rmlatinfont\textsuperscript{§~\theparCount}}}
	\pend% ending standard par
      ‚{\tiny $_{lb}$}‚

	  
	  \pstart \leavevmode% starting standard par
	\textbf{य‚दी}त्यादिना व्याच‚ष्टे । \textbf{य‚मात्मान}मित्य‚नुग‚तं व्यावृत्त‚ञ्च । \textbf{तेनात्म‚ना‚{\tiny $_{lb}$}‚ सामान्य‚विशेष‚योर्य}दि भेद इति स‚म्ब‚न्धः । एत‚देव स्फुट‚य‚न्नाह । \textbf{य‚स्मा}दित्यादि ।‚{\tiny $_{lb}$}‚ तौ भेद‚व्य‚व‚स्थाप‚कावात्मानौ \textbf{त‚यो}रिति सामान्य‚विशेष‚योः । \textbf{स्वात्म\textbf{आ}नौ} स्व‚भा‚{\tiny $_{lb}$}‚व‚भूतौ । \textbf{तौ चे}द‚नुग‚त‚व्यावृत्तावात्मानौ \textbf{व्य‚तिरे‚{\tiny $_{२}$}‚किणौ} प‚र‚स्प‚र‚व्यावृत्तौ \textbf{त‚दा व्य‚ति‚{\tiny $_{lb}$}‚रेक एव} भेद एव । किङ्कार‚णं [।] \textbf{स्व‚भाव‚भेदात्} । \href{http://sarit.indology.info/?cref=pv.3.176-3.177}{१७९-८०}
	{\color{gray}{\rmlatinfont\textsuperscript{§~\theparCount}}}
	\pend% ending standard par
      ‚{\tiny $_{lb}$}‚

	  
	  \pstart \leavevmode% starting standard par
	स्यान्म‚त‚म् [।] अव‚स्थाऽव‚स्थात्रोर्भेद‚व्य‚व‚स्थाप‚को हि स्व‚भाव एव भिद्य‚ते‚{\tiny $_{lb}$}‚ न भाव इत्याह । \textbf{स्व‚भावो ही}त्यादि । स्व‚भाव एव भाव इत्य‚र्थः । त‚था चेति [।]‚{\tiny $_{lb}$}‚ \textbf{भेद‚सामान्य‚योर}त्य‚न्त‚भेदे स‚ति । भेद‚स्य \textbf{निस्सामा}न्य‚ता । सामान्य‚स्य च निर्वि‚{\tiny $_{lb}$}‚शेष‚ता स्यादिति स‚म्ब‚न्धः [।] सामान्य‚स्य भेद‚व‚त्त्वं भे‚{\tiny $_{३}$}‚दानां च सामान्य‚व‚त्त्वं न‚{\tiny $_{lb}$}‚ स्यात् स‚म्ब‚न्धाभावादिति याव‚त् । य‚द्व‚द् \textbf{घ‚टादीनां} भेदानां स‚म्ब‚न्धाभावात् \textbf{प‚र‚{\tiny $_{lb}$}‚स्प‚र}न्त‚द्व‚त्ता नास्ति । \href{http://sarit.indology.info/?cref=pv.3.178}{१८१}
	{\color{gray}{\rmlatinfont\textsuperscript{§~\theparCount}}}
	\pend% ending standard par
      ‚{\tiny $_{lb}$}‚

	  
	  \pstart \leavevmode% starting standard par
	\textbf{व्य‚तिरेके चे}त्यादिनार्थ‚माह । भेद‚सामान्य‚योर‚ज‚न्य‚ज‚न‚क‚त्वेन \textbf{स‚म्ब‚न्धाभावात्} ।‚{\tiny $_{lb}$}‚ भ‚व‚ति च त‚योस्स‚म्ब‚न्धित‚या प्र‚तीतिस्त‚स्माद् भ्रान्त‚त्व‚मुक्त‚मिति \textbf{न स्यात् सामा‚{\tiny $_{lb}$}‚न्य‚भेद‚धीरि त्य‚त्रान्त‚रे} \href{http://sarit.indology.info/?cref=}{१ । १७१} । \href{http://sarit.indology.info/?cref=pv.3.177-3.178}{१८०-८१}
	{\color{gray}{\rmlatinfont\textsuperscript{§~\theparCount}}}
	\pend% ending standard par
      ‚{\tiny $_{lb}$}‚‚{\tiny $_{lb}$}‚\textsuperscript{\textenglish{336/s}}

	  
	  \pstart \leavevmode% starting standard par
	एव‚मूर्ध्व‚सामान्य‚वादं दि ग म्व रा द्य‚भिम‚{\tiny $_{४}$}‚तं । तिर्य‚क्सामान्य‚वाद‚ञ्च‚{\tiny $_{lb}$}‚ सां ख्या द्य‚भिम‚तं साधार‚ण‚दूष‚णेन निराकृत्य पुन‚स्तिर्य‚क्सामान्य‚वाद‚मेव दूष‚यितु‚{\tiny $_{lb}$}‚माह । \textbf{अपि} चेत्यादि ।
	{\color{gray}{\rmlatinfont\textsuperscript{§~\theparCount}}}
	\pend% ending standard par
      ‚{\tiny $_{lb}$}‚

	  
	  \pstart \leavevmode% starting standard par
	एत‚दुक्त‚म्भ‚व‚ति [।] अर्थ‚क्रियार्थिनः सामान्य‚विष‚य‚भेदाभेद‚चिन्त‚या न‚{\tiny $_{lb}$}‚ किञ्चित् प्र‚योज‚न‚म‚र्थ‚क्रियार‚हित‚त्वात् । किन्तु । \textbf{य‚मात्मान}म‚र्थ‚क्रियायोग्यं \textbf{पुर‚{\tiny $_{lb}$}‚स्कृत्या}ल‚म्ब‚नीकृत्य । \textbf{त‚त्साध्य‚फ‚ल‚वाञ्छावान्} । तेनात्म‚ना‚{\tiny $_{५}$}‚ य‚त्साध्य‚म्फ‚ल‚न्त‚{\tiny $_{lb}$}‚द‚भिलाषावान् । \textbf{अयं पुरुषः प्र‚व‚र्त्त‚ते । त‚दाश्र‚या}व‚र्थ‚क्रियास‚म‚र्थाधिष्ठानौ \textbf{भेदा‚{\tiny $_{lb}$}‚भेदौ} चिंत्येते । त‚स्य चार्थ‚क्रियायोग्य‚स्य \textbf{स्वात्म‚ना} स्वेन रूपेण \textbf{भेद} आत्य‚न्तिको‚{\tiny $_{lb}$}‚स्त्येव । \textbf{व्यावृत्त्या} च विजातीय‚व्यावृत्तेन रूपेण \textbf{स‚मान‚ता}स्त्येवाध्य‚व‚सितैक‚त्व‚{\tiny $_{lb}$}‚रूप‚या । इय‚तैवार्थ‚क्रियार्थिनो भेदाभेद‚चिन्ता स‚माप्ता । त‚तोन‚र्थ‚क्रियाकारिणः‚{\tiny $_{lb}$}‚ सामान्य‚स्य किं‚{\tiny $_{६}$}‚ स्व‚ल‚क्ष‚णे भेदाभेद‚चिन्त‚येति । \href{http://sarit.indology.info/?cref=pv.3.178-3.179}{१८१-८२}
	{\color{gray}{\rmlatinfont\textsuperscript{§~\theparCount}}}
	\pend% ending standard par
      ‚{\tiny $_{lb}$}‚

	  
	  \pstart \leavevmode% starting standard par
	न‚नु चार्थ‚क्रियार्थिनः पुरुष‚स्य व्यावृत्त्यापि स‚मान‚तायाः किम्प्र‚योज‚न‚म‚र्थ‚क्रिया‚{\tiny $_{lb}$}‚र‚हित‚त्वात् ।
	{\color{gray}{\rmlatinfont\textsuperscript{§~\theparCount}}}
	\pend% ending standard par
      ‚{\tiny $_{lb}$}‚

	  
	  \pstart \leavevmode% starting standard par
	स‚त्त्यं [।] स्व‚ल‚क्ष‚णान्येव व्यावृत्त्या सामान्य‚मुच्य‚ते श‚ब्दात्त‚त्रैक‚त्वाध्य‚व‚{\tiny $_{lb}$}‚सायेन प्र‚वृत्तिर्य‚थास्यादित्य‚दोषः ।
	{\color{gray}{\rmlatinfont\textsuperscript{§~\theparCount}}}
	\pend% ending standard par
      ‚{\tiny $_{lb}$}‚

	  
	  \pstart \leavevmode% starting standard par
	अथ स्यात् स्व‚ल‚क्ष‚ण‚मेव स्व‚ल‚क्ष‚णान्त‚रानुयायीति किं प‚रिक‚ल्पित‚या व्यावृ‚{\tiny $_{lb}$}‚\leavevmode\ledsidenote{\textenglish{123a/PSVTa}} त्त्येत्याह । स्व‚ल‚क्ष‚णानाम्प‚र‚स्प‚र‚म्भेदात्‚{\tiny $_{७}$}‚ । य‚दि घ‚ट‚रूप‚म्प‚टे स्यादुद‚काह‚र‚णार्थी‚{\tiny $_{lb}$}‚ प‚टेपि प्र‚व‚र्त्तेत । त‚दाह । \textbf{प्र‚वृत्त्या}दीत्यादि । \textbf{आदि}श‚ब्दात् तुल्योत्प‚त्तिनिरोधा‚{\tiny $_{lb}$}‚दिप्र‚स‚ङ्गः ।
	{\color{gray}{\rmlatinfont\textsuperscript{§~\theparCount}}}
	\pend% ending standard par
      ‚{\tiny $_{lb}$}‚

	  
	  \pstart \leavevmode% starting standard par
	\textbf{स‚र्व एवे}त्यादिना व्याच‚ष्टे । विशेष‚मेवार्थ‚क्रियायोग्यं स्व‚भावाख्य‚मात्म‚{\tiny $_{lb}$}‚भूत‚मित्य‚र्थः । क‚स्य भाव‚स्य व‚स्तुनो\textbf{धिकृत्य प्र‚व‚र्त्त‚ते} । स \textbf{एव ही}त्य‚र्थ‚क्रियाकारी‚{\tiny $_{lb}$}‚ विशेषः । \textbf{त‚थेति} । गौरित्यादिश‚ब्दैः । अर्थ‚क्रियार्थी हि स्व‚{\tiny $_{१}$}‚ल‚क्ष‚ण‚प्र‚तिपाद‚ना‚{\tiny $_{lb}$}‚भिप्राय एव श‚ब्दं प्र‚युङ्क्ते दृश्य‚विक‚ल्प्य‚योरेकीकृत्य । प्र‚तिप‚त्तापि त‚थैव प्र‚ति‚{\tiny $_{lb}$}‚‚{\tiny $_{lb}$}‚ \leavevmode\ledsidenote{\textenglish{337/s}}प‚द्य‚ते । त‚तो व्य‚व‚ह‚र्त्तृणाम‚ध्य‚व‚साय‚व‚शाच्छ‚ब्द‚व्यापारापेक्ष‚यैत‚दुक्तं । श‚ब्दे तु ज्ञान‚{\tiny $_{lb}$}‚ \textbf{स्व‚ल‚क्ष‚ण‚प्र‚तिभासो} नास्तीति स्व‚ल‚क्ष‚ण‚म‚वाच्य‚मुक्त‚मित्य‚दोषः ।
	{\color{gray}{\rmlatinfont\textsuperscript{§~\theparCount}}}
	\pend% ending standard par
      ‚{\tiny $_{lb}$}‚

	  
	  \pstart \leavevmode% starting standard par
	द्र‚व्याद‚य‚स्तु न त‚त्रेति । गौरित्यादिश‚ब्दैर्गंवादिचोद‚नायां । क‚स्मात् य‚थास्वं‚{\tiny $_{lb}$}‚ द्र‚व्य‚त्वादिश‚ब्दैस्तेषाङ्ग‚{\tiny $_{२}$}‚वादेः \textbf{पृथ‚ग‚भिधानात्} गोद्र‚व्य‚मित्यादिना । क‚थ‚न्त‚र्हि‚{\tiny $_{lb}$}‚ गौरित्यादिप‚द‚प्र‚योगे स‚त्ताद्र‚व्य‚त्वाद‚यः प्र‚तीय‚न्त इत्याह । \textbf{अर्थ‚स्ये}त्यादि । अर्थ‚स्य‚{\tiny $_{lb}$}‚ ग‚वादेः । \textbf{तेन} स‚त्त्व‚द्र‚व्य‚त्वादिनाऽ\textbf{व्य‚भिचारात्} । त‚तोर्थाद् ग‚तिः सामान्यानां‚{\tiny $_{lb}$}‚ स्यात् । न तु विशेष‚श‚ब्दः सामान्ये व्याप्रिय‚ते । \textbf{निर्लोठितं चैत‚दाचार्य} दि ङ् ना गे न‚{\tiny $_{lb}$}‚ सा मा न्य प री क्षा दौ य‚था न विशेष‚श‚ब्दानां सा‚{\tiny $_{३}$}‚मान्ये वृत्तिरिति ।
	{\color{gray}{\rmlatinfont\textsuperscript{§~\theparCount}}}
	\pend% ending standard par
      ‚{\tiny $_{lb}$}‚

	  
	  \pstart \leavevmode% starting standard par
	अत्र चोद्य‚ते । क‚स्य पुनः सामान्य‚स्य विशेषेणाव्य‚भिचारः । य‚त्ताव‚त् प‚र‚{\tiny $_{lb}$}‚प‚रिक‚ल्पित‚न्नास्त्येव । य‚च्चान्य‚व्यावृत्तिल‚क्ष‚णं प्र‚स‚ज्य मात्र‚न्त‚द‚पि नास्त्येव ।‚{\tiny $_{lb}$}‚ नापि विक‚ल्प‚बुद्धिप्र‚तिभासिनो बाह्येनाव्य‚भिचारोस्ति त‚स्य स्व‚त‚न्त्र‚त्वात् ।
	{\color{gray}{\rmlatinfont\textsuperscript{§~\theparCount}}}
	\pend% ending standard par
      ‚{\tiny $_{lb}$}‚

	  
	  \pstart \leavevmode% starting standard par
	उच्य‚ते । स्व‚ल‚क्ष‚ण‚मेव स‚जातीय‚व्यावृत्त‚म्विशेषः । त‚देव विजातीय‚व्यावृत्ति‚{\tiny $_{lb}$}‚म‚पेक्ष्याभेदे‚{\tiny $_{४}$}‚नोपात्तं सामान्य‚मित्युच्य‚ते । त‚तः सामान्य‚विशेष‚योर्व‚स्तुत एक‚त्वात् ।‚{\tiny $_{lb}$}‚ कृत‚क‚त्वानित्य‚त्व‚योरिवाव्य‚भिचारः । श‚ब्द‚व्यापार‚भेदात्तु केव‚लं क्व‚चिच्छाब्दी‚{\tiny $_{lb}$}‚ प्र‚तिप‚त्तिः क्व‚चिदार्थीत्युच्य‚ते । \textbf{त‚दिति} त‚स्मादेयं पुरुषः । \textbf{ग‚वादिश‚ब्द‚प्र‚त्युप‚स्था‚{\tiny $_{lb}$}‚पित‚ङ्} ग‚वादिश‚ब्द‚स‚न्निधापित‚म‚र्थ‚म‚र्थ‚क्रियाश्र‚यं । \textbf{अर्थान्त‚र‚स्य} सामान्य‚स्यो\textbf{प‚न्यासेन}‚{\tiny $_{lb}$}‚ भेद‚सा‚{\tiny $_{५}$}‚मान्याकार‚त‚या द्विमुखा बुद्धिर्य‚स्य स त‚थोक्तः । \textbf{योस्य} ग‚वा\textbf{देरात्मा}‚{\tiny $_{lb}$}‚ स्व‚भावः । \textbf{अन‚न्य}भाव\textbf{साधार}ण‚स्य एव स्व‚भावः श‚ब्द‚चोदित इति व‚क्ष्य‚माणेन‚{\tiny $_{lb}$}‚ स‚म्ब‚न्धः । य‚म‚र्थं सास्नादिम‚न्त्तं \textbf{पुर‚स्कृत्या}ल‚म्ब‚नीकृत्य विशिष्टार्थ‚क्रियार्थी । त‚मेवाह ।‚{\tiny $_{lb}$}‚ \textbf{य‚थे}त्यादि । य‚था गोर्वाह‚दोहादाव‚र्थी गाम‚धिकृत्य प्र‚व‚र्त्त‚ते । \textbf{अन्य‚स‚म्भ‚विन} इति‚{\tiny $_{lb}$}‚ गोर‚न्य‚स्मिन्न‚श्वे स‚{\tiny $_{६}$}‚म्भ‚विनोर्थ‚स्यार्थी गां पुर‚स्कृत्य न प्र‚व‚र्त्त‚त इति वाक्यार्थः‚{\tiny $_{lb}$}‚ स‚म‚र्थ‚नीयः । कोर्थोन्य‚स‚म्भ‚वीत्याह । \textbf{य‚था युद्ध‚प्र‚वेश} इति । य‚थास्व‚मिति य‚स्य‚{\tiny $_{lb}$}‚ यः श‚ब्दो वाच‚कः । \textbf{न द‚व्य‚त्वादि सामान्यं । चोदित}मिति लिङ्ग‚विप‚रिणामेन‚{\tiny $_{lb}$}‚ स‚म्ब‚न्धः । त‚च्चोद‚न‚या ग‚वादीनां ग‚वादिश‚ब्दैश्चोद‚न‚या । \textbf{त‚दे}त्य‚र्थ‚क्रियार्थिनः‚{\tiny $_{lb}$}‚ ‚{\tiny $_{lb}$}‚ \leavevmode\ledsidenote{\textenglish{338/s}}\leavevmode\ledsidenote{\textenglish{123b/PSVTa}} प्र‚वृत्तिकाले । \textbf{प्राप्तुम‚न‚भिप्रेत‚त्वाद्} द्र‚व्य‚त्वादिसामा‚{\tiny $_{७}$}‚न्य‚स्येति । विभ‚क्तिविप‚{\tiny $_{lb}$}‚रिणामेन स‚म्ब‚न्धः । क‚स्मात् पुन‚र्ग‚वादिश‚ब्देन द्र‚व्य‚त्वादिसामान्यं चोद‚यितुं ना‚{\tiny $_{lb}$}‚भिप्रेत‚मित्याह । \textbf{ग‚वादिस‚मावेशाद्} ग‚वादिश‚ब्द‚स्य ग‚वादौ लोके संकेतित‚त्वात् ।‚{\tiny $_{lb}$}‚ ग‚वादिस्व‚भाव‚त्वाद् द्र‚व्य‚त्वादिसामान्य‚स्य ग‚वादिचोद‚न‚याऽभिधान‚मिति चेदाह ।‚{\tiny $_{lb}$}‚ \textbf{त‚दात्म‚भूतानां} चेति । ग‚वादिस्व‚भावानां द्र‚व्य‚त्वादिसामान्यानां ग‚वादिव‚देवा‚{\tiny $_{lb}$}‚न‚न्व‚येन हेतुना‚{\tiny $_{१}$}‚ \textbf{त‚त्रे}ति त‚स्य ग‚वादिभेद‚स्य । \textbf{अनुभ‚य‚रूप‚त्वा}द‚सामान्य‚विशेष‚{\tiny $_{lb}$}‚रूप‚त्वादेवेति याव‚त् । त‚त‚श्च विशेष एव चोद्य‚ते ।
	{\color{gray}{\rmlatinfont\textsuperscript{§~\theparCount}}}
	\pend% ending standard par
      ‚{\tiny $_{lb}$}‚

	  
	  \pstart \leavevmode% starting standard par
	त‚देवाह । \textbf{त‚मेवे}त्यादि । त‚मेव चान‚न्य‚साधार‚ण\textbf{म्भाव}म‚र्थ‚क्रियार्थी पुरुषो‚{\tiny $_{lb}$}‚ भेदाभेद\textbf{प्र‚कारैः प‚र्य‚नुयुङ्क्ते} । अ न्या पो ह वा दि नोपि व्यावृत्तिल‚क्ष‚णो द्र‚व्य‚त्वा‚{\tiny $_{lb}$}‚द्य‚भेदः स्व‚ल‚क्ष‚णानामिष्ट‚स्त‚तोऽत्य‚न्त‚भेदो विशेषाणां विरुद्ध इत्य‚त आह‚{\tiny $_{२}$}‚ ।‚{\tiny $_{lb}$}‚ \textbf{त‚स्ये}त्यादि । त‚स्यार्थ‚क्रियाकारिणोर्थ‚स्य \textbf{भेदे} प्र‚कृत्या स्थिते स‚ति व्यावृत्तिल‚क्ष‚णो‚{\tiny $_{lb}$}‚ \textbf{द्र‚व्य‚त्वाद्य‚भे}दोस्य विशेष‚स्या\textbf{बाध‚क एव} । त‚स्य क‚ल्पित‚त्वात् । त‚स्मात् पार‚मा‚{\tiny $_{lb}$}‚र्थिको भेदः । स्व‚ल‚क्ष‚णानामुप‚क‚ल्पित‚मेक‚त्व‚म‚नेन च प्र‚कारेण भेदाभेदाविशिष्टा‚{\tiny $_{lb}$}‚व‚स्माकं [।] त‚देव द‚र्श‚य‚न्नाह । \textbf{स‚र्व‚त्रे}त्यादि । \textbf{स्व‚भावेन भेदः} स्व‚जातीय‚विजा‚{\tiny $_{lb}$}‚तीयात् । \textbf{सामान्य‚स्य च व्यावृत्तिल‚क्ष‚ण‚स्याभ्युप‚ग‚मादिति} स‚म्ब‚न्धः ।
	{\color{gray}{\rmlatinfont\textsuperscript{§~\theparCount}}}
	\pend% ending standard par
      ‚{\tiny $_{lb}$}‚

	  
	  \pstart \leavevmode% starting standard par
	न व्यावृत्तिरूपेण सामान्येनाभेदः किन्तु व‚स्तुभूतेनैवेति चेदाह । \textbf{स्व‚भाव}‚{\tiny $_{lb}$}‚भूत‚स्येत्यादि । व‚स्तुभूत‚स्य \textbf{सामान्य‚स्याभेद इति} व्य‚क्तिभ्योन‚र्थान्त‚र‚त्वे । \textbf{उक्त}‚{\tiny $_{lb}$}‚मिति सामान्याद‚व्य‚तिरेकाद् भेदानामैक्यं । भेद‚व‚देव वा सामान्य‚स्याप्य‚नेक‚त्व‚{\tiny $_{lb}$}‚मित्युक्तं प्राक् । न सामान्य‚द्वारेण भेदानामैक्य‚मित्युच्य‚ते । किन्तु योसौ विशेष‚{\tiny $_{४}$}‚‚{\tiny $_{lb}$}‚स्तेनैवाभेद इत्याह । \textbf{स्वात्म‚नै}वेत्यादि । स्वेनैव विशेष‚रूपेण ग‚वाश्वादीनाम‚भेदे‚{\tiny $_{lb}$}‚ \textbf{त‚द्} गोद्र‚व्यं \textbf{निब‚न्ध‚नं} य‚स्या अर्थ\textbf{क्रियाया} वाह‚दोहादिल‚क्ष‚णायास्त\textbf{यार्थी} पुरुषः‚{\tiny $_{lb}$}‚ [।] \textbf{स‚म‚मि}त्युभ‚य‚त्राप्य‚व‚सित‚ग‚वादिभावः । \textbf{द्व‚योर‚पी}ति ग‚वि चाश्वे चैव । य‚स्मा‚{\tiny $_{lb}$}‚\textbf{देकोपि} हि कार‚ण‚त्वेनाभिम‚तो गोप‚दार्थ‚स्ताम‚र्थ‚क्रियाम्वाह‚दोहादिस्व‚भावां ।‚{\tiny $_{lb}$}‚ ‚{\tiny $_{lb}$}‚ \leavevmode\ledsidenote{\textenglish{339/s}}\textbf{त‚त्स्व‚भाव‚त्वात्त}द‚र्थ‚{\tiny $_{५}$}‚क्रियाक‚र‚ण‚स्व‚भाव‚त्वादेव क‚रोति । \textbf{त‚द‚न्य‚स्यापि} त‚स्माद्‚{\tiny $_{lb}$}‚ गोद्र‚व्याद‚न्य‚स्याप्य‚श्व‚स्य \textbf{त‚द्वा}ह‚दोहादिक‚र‚ण‚स्व‚भाव‚त्व\textbf{न्तुल्य‚मिति सो}प्य‚श्वः‚{\tiny $_{lb}$}‚ गोसाध्याम‚र्थ‚क्रियां \textbf{किन्न क‚रोति} ॥ ० ॥ \href{http://sarit.indology.info/?cref=pv.3.179-3.180}{१८२-८३}
	{\color{gray}{\rmlatinfont\textsuperscript{§~\theparCount}}}
	\pend% ending standard par
      ‚{\tiny $_{lb}$}‚

	  
	  \pstart \leavevmode% starting standard par
	\textbf{एतेनै}वेति स‚र्व‚स्यार्थ‚स्य भेद‚साध‚नेन । \textbf{अह्रीका} न‚ग्न‚त‚या निर्ल‚ज्जाः क्ष प ण काः ।‚{\tiny $_{lb}$}‚ अयुक्ताभिधान‚स्य कुत्सित‚त्वात् किम‚पीत्याह । \textbf{अश्लील‚ङ्}ग्राम्यं । स‚र्वः‚{\tiny $_{६}$}‚ स‚र्व‚स्व‚भावो‚{\tiny $_{lb}$}‚ न च स‚र्वः स‚र्व‚स्व‚भाव इति य‚त् \textbf{प्र‚ल‚प‚न्ति प्र‚तिक्षिप्त‚न्त‚द}पि त‚स्मा\textbf{देकान्त‚स‚म्भ‚वात्} ।‚{\tiny $_{lb}$}‚ एक‚स्यैवान्त‚स्यात्य‚न्त‚भेद‚प्र‚कार‚स्य स‚म्भ‚वात् । \href{http://sarit.indology.info/?cref=pv.3.180-3.181}{१८३-८४}
	{\color{gray}{\rmlatinfont\textsuperscript{§~\theparCount}}}
	\pend% ending standard par
      ‚{\tiny $_{lb}$}‚

	  
	  \pstart \leavevmode% starting standard par
	न‚नु दि ग म्व रा णां स‚र्वं स‚र्वात्म‚कं न स‚र्वं स‚र्वात्म‚क‚मिति नैत‚द्द‚र्श‚न‚न्त‚त्किम‚र्थ‚{\tiny $_{lb}$}‚मिद‚मा चा र्ये णो च्य‚ते ।
	{\color{gray}{\rmlatinfont\textsuperscript{§~\theparCount}}}
	\pend% ending standard par
      ‚{\tiny $_{lb}$}‚

	  
	  \pstart \leavevmode% starting standard par
	स‚त्त्यं [।] य‚था द‚र्श‚न‚न्त्वं त्य‚न्त‚भेदाभेदौ च स्यातामि \href{http://sarit.indology.info/?cref=}{१ । १७८}त्यादिना‚{\tiny $_{lb}$}‚ पूर्व‚मेव दूषितं ।
	{\color{gray}{\rmlatinfont\textsuperscript{§~\theparCount}}}
	\pend% ending standard par
      ‚{\tiny $_{lb}$}‚

	  
	  \pstart \leavevmode% starting standard par
	य‚त्पुर‚न‚रेत‚दुक्तं‚{\tiny $_{७}$}‚ [।] त‚द्य‚था क‚ट‚केयूरादिषु सुव‚र्ण्ण‚प्र‚त्य‚य‚स्यानुयायिन‚स्स‚द्भा- \leavevmode\ledsidenote{\textenglish{124a/PSVTa}}‚{\tiny $_{lb}$}‚ वास्सुव‚र्ण्ण‚त्व‚सामान्यं क‚ल्प्य‚ते । त‚था घ‚ट‚प‚टादिषु द्र‚व्य‚त्वादिप्र‚त्य‚य‚स्यान्व‚यिनः‚{\tiny $_{lb}$}‚ स‚द्भावाद् द्र‚व्य‚त्व‚सामान्य‚मेकं किन्नेष्य‚ते । न चेष्य‚तेऽभिन्न‚प्र‚त्य‚य‚स‚द्भावेपि‚{\tiny $_{lb}$}‚ त‚था क‚ट‚क‚केयूरादिषु सामान्य‚क‚ल्प‚ना माभूदित्येव‚म्प‚र‚मेत‚त् ।
	{\color{gray}{\rmlatinfont\textsuperscript{§~\theparCount}}}
	\pend% ending standard par
      ‚{\tiny $_{lb}$}‚

	  
	  \pstart \leavevmode% starting standard par
	स्या\textbf{दुष्ट्रो द‚धि} । द्र‚व्यादिरूप‚त‚यैक‚त्वात् । स्यान्न द‚धि उष्ट्राव‚स्थातो द‚ध्य‚{\tiny $_{lb}$}‚व‚स्था‚{\tiny $_{१}$}‚या भिन्न‚त्वात् । अश्लील‚मित्य‚स्य व्याख्यान\textbf{म‚युक्त}मिति । त‚स्यायुक्त‚त्वात् ।‚{\tiny $_{lb}$}‚ विद्व‚ज्ज‚नायोग्य‚त‚या ग्राम्य‚मिति भावः । अश्लील‚मित्य‚स्य ग्राम्य‚प‚र्याय‚त्वात् । \textbf{अहे‚{\tiny $_{lb}$}‚योपादेय}मिति । अत्याज्य‚म‚ग्राह्य‚ञ्च । क‚स्माद\textbf{प‚रिनिष्ठानात्} । य‚दि हि किंचित्सुख‚{\tiny $_{lb}$}‚साध‚न‚त्वेन निश्चित‚म‚न्य‚च्च दुःख‚साध‚न‚त्वेन त‚दा य‚थाक्र‚मं हेय‚मुपादेयं वा स्यात् ।‚{\tiny $_{lb}$}‚ त‚च्च नास्ति य‚तः स‚र्व‚स्य‚{\tiny $_{२}$}‚ स‚र्व‚स्व‚भाव‚त्वं [।] न च स‚र्व‚स्य स‚र्व‚स्व‚भाव‚त्वं । अत‚{\tiny $_{lb}$}‚ एवाकुल‚मेक‚स्यापि स्व‚भाव‚भेद‚स्य गृहीतुम‚श‚क्य‚त्वात् ।
	{\color{gray}{\rmlatinfont\textsuperscript{§~\theparCount}}}
	\pend% ending standard par
      ‚{\tiny $_{lb}$}‚‚{\tiny $_{lb}$}‚\textsuperscript{\textenglish{340/s}}

	  
	  \pstart \leavevmode% starting standard par
	एत‚दुक्त‚म्भ‚व‚ति । य‚दाव‚स्थात‚द्व‚तोस्स‚र्वात्म‚नाऽभेदोव‚स्थानात्तु प‚र‚स्प‚र‚म्भेद‚स्त‚{\tiny $_{lb}$}‚दाय‚न्दोष‚स्त‚दाह । \textbf{त‚द‚न्व‚ये} वेति । त‚स्य स्व‚भाव‚भेद‚स्य प‚र‚स्प‚रान्व‚ये वा । द‚ध्या‚{\tiny $_{lb}$}‚दिस्व‚भाव‚स्य द्र‚व्य‚स्योष्ट्रादिषु तादात्म्येनानुग‚मादिति याव‚त् ।
	{\color{gray}{\rmlatinfont\textsuperscript{§~\theparCount}}}
	\pend% ending standard par
      ‚{\tiny $_{lb}$}‚

	  
	  \pstart \leavevmode% starting standard par
	\textbf{स‚र्व‚स्यो‚{\tiny $_{३}$}‚भ‚य‚रूप‚त्वं} । उभ‚य‚ग्र‚ह‚ण‚म‚नेक‚त्वोप‚ल‚क्ष‚णार्थ‚न्त‚स्मिन् स‚ति त‚द्वि‚{\tiny $_{lb}$}‚श‚ष‚स्य उष्ट्र उष्ट्र एव न द‚धि । द‚धि द‚ध्येव नोष्ट्र इत्येवं ल‚क्ष‚ण‚स्य निराकृतेः ।‚{\tiny $_{lb}$}‚ \textbf{द‚धि खादे}त्येवं चोदितः पुरुषः \textbf{किमुष्ट्रं} खादितुं \textbf{नाभिधाव‚ति । उष्ट्रोपि} द‚ध्य‚भि‚{\tiny $_{lb}$}‚न्नाद् द्र‚व्य‚त्वाद् अव्य‚तिरेकात् \textbf{स्याद् द‚धि । नापि स एवे}ति । \textbf{उष्ट्र} एवोष्ट्र इत्ये‚{\tiny $_{lb}$}‚कान्त‚वादः । \textbf{येनान्योपि} द‚ध्यादिकः \textbf{स्यादुष्ट्रः‚{\tiny $_{४}$}‚ त‚था द‚ध्य‚पि स्यादुष्ट्रः} । उष्ट्राभिन्नेन‚{\tiny $_{lb}$}‚ द्र‚व्य‚त्वेन द‚ध्न‚स्तादात्म्येनाभिस‚म्ब‚न्धात् । \textbf{नापि त‚देवे}ति द‚ध्येव \textbf{द‚धि । येनान्य‚{\tiny $_{lb}$}‚द‚प्यु}ष्ट्रादिकं \textbf{स्याद् द‚धि} । एतेन स‚र्व‚स्योभ‚य‚रूप‚त्वं व्याख्यातं ।
	{\color{gray}{\rmlatinfont\textsuperscript{§~\theparCount}}}
	\pend% ending standard par
      ‚{\tiny $_{lb}$}‚

	  
	  \pstart \leavevmode% starting standard par
	\textbf{त‚द्विशेष‚निराकृते}रित्येत\textbf{द‚न‚योरि}त्यादिना व्याच‚ष्टे । उभ‚य‚थापि द‚ध्युष्ट्र‚विशेषः‚{\tiny $_{lb}$}‚ स्यात् । द‚धिरूपाभावो वोष्ट्रे स्यात् । उष्ट्र‚रूपं वा द‚ध्य‚स‚म्भ‚वि य‚द्युष्ट्र‚स्व‚रूप‚{\tiny $_{lb}$}‚ एव निय‚त‚{\tiny $_{५}$}‚म्भ‚वेत् । एवं द‚ध्नोपि वाच्यं ।
	{\color{gray}{\rmlatinfont\textsuperscript{§~\theparCount}}}
	\pend% ending standard par
      ‚{\tiny $_{lb}$}‚

	  
	  \pstart \leavevmode% starting standard par
	आद्य‚स्य ताव‚द‚स‚म्भ‚व\textbf{स्त‚दि}त्यादिना क‚थ्य‚ते । त‚देव‚म‚नेक‚योर्द‚ध्युष्ट्र‚योर्न क‚श्चिद्‚{\tiny $_{lb}$}‚ विशेष इति स‚म्ब‚न्धः । \textbf{एक‚स्यापी}ति द‚ध्न उष्ट्र‚स्य वा \textbf{क‚स्य‚चित् त‚द्रुपाभाव‚स्}येति ।‚{\tiny $_{lb}$}‚ उष्ट्र‚रूपाभाव‚स्य द‚धिरूपाभाव‚स्य \textbf{चाभावात्} ।
	{\color{gray}{\rmlatinfont\textsuperscript{§~\theparCount}}}
	\pend% ending standard par
      ‚{\tiny $_{lb}$}‚

	  
	  \pstart \leavevmode% starting standard par
	द्वितीय‚स्यापि प्र‚काराभाव‚माह । \textbf{स्व‚रूप‚स्ये}त्यापि । \textbf{अत‚द्भावि}नो द‚ध्य‚भाविन‚{\tiny $_{lb}$}‚ उष्ट्र‚स्व‚रूप‚स्य । उष्ट्राभाविनो‚{\tiny $_{६}$}‚ वा द‚धिस्व‚रूप‚स्य । \textbf{स्व‚निय‚त‚स्य} उष्ट्र‚स्व‚भाव‚{\tiny $_{lb}$}‚निय‚त‚स्य । द‚धिस्व‚भाव‚निय‚त‚स्य \textbf{चाभावात्} \href{http://sarit.indology.info/?cref=pv.3.181-3.182}{१८४-८५}
	{\color{gray}{\rmlatinfont\textsuperscript{§~\theparCount}}}
	\pend% ending standard par
      ‚{\tiny $_{lb}$}‚

	  
	  \pstart \leavevmode% starting standard par
	\textbf{अथास्ति} द‚ध्युष्ट्र‚यो\textbf{र‚तिश‚यः क‚श्चिद् येना}तिश‚येन \textbf{द‚धि खादेति} चोदितः पुरुषो‚{\tiny $_{lb}$}‚ \textbf{भेदेन व‚र्त्त‚ते} । उष्ट्र‚प‚रिहारेण द‚ध्न्येव प्र‚व‚र्त‚त्ते ।
	{\color{gray}{\rmlatinfont\textsuperscript{§~\theparCount}}}
	\pend% ending standard par
      ‚{\tiny $_{lb}$}‚

	  
	  \pstart \leavevmode% starting standard par
	एत‚दुक्त‚म्भ‚व‚ति । य‚था द‚ध्युष्ट्र‚योः प‚र‚स्प‚रं स्व‚रुप‚म्भिन्न‚न्त‚था द्र‚व्य‚प‚र्याय‚यो‚{\tiny $_{lb}$}‚‚{\tiny $_{lb}$}‚ \leavevmode\ledsidenote{\textenglish{341/s}}र्ल‚क्ष‚ण‚भेदाद् भेदो य‚दीष्य‚ते‚{\tiny $_{७}$}‚ त‚दा \textbf{स एव} स्व‚रूपातिश‚यो द‚धि । स चान्य‚त्रोष्ट्रे \leavevmode\ledsidenote{\textenglish{124b/PSVTa}}‚{\tiny $_{lb}$}‚ नास्ति । नापि द्र‚व्य‚त्वं द‚ध्यादिव्य‚तिरिक्त प्र‚तिभास‚ते । इत्य‚नेन द्वारेणानुभ‚यं‚{\tiny $_{lb}$}‚ सामान्य‚विशेष‚र‚हितं स‚र्व‚म्व‚स्तु । प‚रं केव‚लं । एक‚त्व‚न्तु क‚ल्पित । अन‚योरिति‚{\tiny $_{lb}$}‚ द‚ध्युष्ट्र‚योः । \textbf{त‚था चोदित} इति द‚धि खादेति चोदितः । \textbf{क्षीर‚विकारो द‚धि ।‚{\tiny $_{lb}$}‚ नान्य‚त्रे}त्युष्ट्रे । स एवातिश‚यो द‚धि । किम्भूतः [।] \textbf{अर्थ‚कियार्विप्र‚वृत्तिविय‚{\tiny $_{१}$}‚यः} ।‚{\tiny $_{lb}$}‚ द‚धिसाध्यार्थ‚क्रिया त‚या योर्थी पुरुष‚स्त‚स्य प्र‚वृत्तिविप‚यः [।] किङ्कार‚णं [।] त‚त्फ‚{\tiny $_{lb}$}‚लेत्यादि । द‚धिसाध्यं फ‚लं त‚देव विशिव्य‚तेन्य‚स्मादिति विशेषः । त‚स्योपादान‚{\tiny $_{lb}$}‚भावो हेतुभाव‚स्तेन ल‚क्षितः स्व‚भावो य‚स्य व‚स्तुन । त‚देव द‚धीति कृत्वा । स च‚{\tiny $_{lb}$}‚ तादृश इत्य‚न‚न्त‚रोक्तो द‚धिस्व‚भावः । अन्य‚त्रेत्युष्ट्रे । क‚स्माद् [।] द‚ध्य‚र्थिन‚स्त‚{\tiny $_{lb}$}‚त्रोष्ट्रे \textbf{प्र‚वृत्त्य‚भावात्} । \href{http://sarit.indology.info/?cref=pv.3.182-3.183}{१८५-८६}
	{\color{gray}{\rmlatinfont\textsuperscript{§~\theparCount}}}
	\pend% ending standard par
      ‚{\tiny $_{lb}$}‚

	  
	  \pstart \leavevmode% starting standard par
	\textbf{स‚र्वात्म‚त्व‚{\tiny $_{२}$}‚ इत्येक‚रूप‚त्वे स‚तीत्य‚र्थः । भिन्नो} निय‚तार्थं । धीष्व‚नी । ज्ञानं‚{\tiny $_{lb}$}‚ श‚ब्द‚श्च । त‚द‚भावाद् भिन्न‚वुद्धिश‚ब्दाभावात् । भेदानां स‚हार‚वाद‚स्य । एकी‚{\tiny $_{lb}$}‚क‚र‚ण‚वाद‚स्यास‚म्भ‚वः । भेदेन गृहीत‚योः श्रुत‚योर्वा । एक‚त्वेनोप‚संहारो निर्देशः ।‚{\tiny $_{lb}$}‚ स्यादुष्ट्रो द‚धीत्यादि ।
	{\color{gray}{\rmlatinfont\textsuperscript{§~\theparCount}}}
	\pend% ending standard par
      ‚{\tiny $_{lb}$}‚

	  
	  \pstart \leavevmode% starting standard par
	सोय‚मित्यादिना व्याच‚ष्टे । क्व‚चिद‚पि द‚ध्न्युष्ट्रे वा प्र‚तिनिय‚त‚मेक‚माकार‚म‚{\tiny $_{lb}$}‚पृश्य‚न् क‚थं बुद्ध्याधिमुच्येता‚{\tiny $_{३}$}‚र्थानिति स‚म्व‚न्धः । किम्विशिष्ट‚या बुद्ध‚येत्याह ।‚{\tiny $_{lb}$}‚ असंसृष्टेत्यादि । असंसृष्टोन्याकारो य‚स्मिन्न‚र्थे स त‚थोक्तः । स य‚स्या बुद्धेर‚स्ति‚{\tiny $_{lb}$}‚ सा संसृष्टान्याकार‚व‚ती । विभ‚क्तार्थ‚ग्राहिण्येवेति याव‚त् । अभिल‚पेद्वा क‚थ ।‚{\tiny $_{lb}$}‚ प्र‚त्य‚र्थ प्र‚तिनिय‚त‚संकेतेन ध्व‚निनेत्याकूतं । क‚स्मान्नाधिमुच्येतेत्याह ।‚{\tiny $_{lb}$}‚ ‚{\tiny $_{lb}$}‚ \leavevmode\ledsidenote{\textenglish{342/s}}\textbf{विभागाभावाद् भावाना}मिति । \textbf{त‚त्संहार‚वाद} इति भेद‚संहार‚वादो न स्यात् [।]‚{\tiny $_{lb}$}‚ स्यादुष्ट्रः स्याद् द‚धीत्यादिकः ।
	{\color{gray}{\rmlatinfont\textsuperscript{§~\theparCount}}}
	\pend% ending standard par
      ‚{\tiny $_{lb}$}‚

	  
	  \pstart \leavevmode% starting standard par
	अथ पुन‚र‚संसृष्टौ द‚ध्युष्ट्रौ प्र‚तिप‚द्य संह‚रेत् । स्यादुष्ट्रः स्याद् द‚धीति ।‚{\tiny $_{lb}$}‚ \textbf{त‚दाप्येक‚रूप‚संस‚र्गिण्या} । उष्ट्र‚रूपेणैव द‚धिरूपेणैव वा संस‚र्गिण्या बुद्धेर‚संसृष्टाकार‚{\tiny $_{lb}$}‚ग्राहिण्या क्व‚चिदुष्ट्रे द‚ध‚नि च । प्र‚तिनिय‚मात् त‚त्प्र‚तिभास‚भेद‚कृत एव त‚योर्द‚ध्युष्ट्र‚योः‚{\tiny $_{lb}$}‚ स्व‚भाव‚भेदोपि । \textbf{एकानेकेत्या}दि । प्र‚ति‚{\tiny $_{५}$}‚भास‚भेद‚स्यानेक‚व्य‚व‚स्थितिर्विष‚यः ।‚{\tiny $_{lb}$}‚ एक‚व्य‚व‚स्थितिः प्र‚तिभास‚भेद‚स्येति योज्यं । भिन्न‚प्र‚तिभास‚विष‚यौ च द‚ध्युष्ट्रौ त‚था‚{\tiny $_{lb}$}‚ च स‚ति नैक उष्ट्रो द‚धि वा त‚दुभ‚य‚रूपः त‚दुभ‚य‚न्द‚ध्युष्ट्रात्म‚कं रूपं य‚स्येति विग्र‚हः ।‚{\tiny $_{lb}$}‚ मिथ्यावाद एव स्या द्वा दः ॥ ० ॥ \href{http://sarit.indology.info/?cref=pv.3.183-3.184}{१८६-८७}
	{\color{gray}{\rmlatinfont\textsuperscript{§~\theparCount}}}
	\pend% ending standard par
      ‚{\tiny $_{lb}$}‚

	  
	  \pstart \leavevmode% starting standard par
	भेद‚ल‚क्ष‚ण‚मिति व्यावृत्तिल‚क्ष‚ण‚म्विजातीय‚व्यावृत्तान्येव स्व‚ल‚क्ष‚णानि सामान्य‚{\tiny $_{lb}$}‚मित्यु‚{\tiny $_{६}$}‚च्य‚न्ते । \textbf{प्र‚कृत्या} स्व‚भावेन । आदिश‚ब्दादुद‚काद्याह‚र‚णाद्येक‚फ‚लाः ॥
	{\color{gray}{\rmlatinfont\textsuperscript{§~\theparCount}}}
	\pend% ending standard par
      ‚{\tiny $_{lb}$}‚

	  
	  \pstart \leavevmode% starting standard par
	\textbf{भ‚व‚तु नामे}त्यादिना चा चा र्यः प‚राभिप्राय‚माशंक‚ते । भावानाम्व‚स्तूनां‚{\tiny $_{lb}$}‚ स्व‚भाव‚भेदः स्व‚भावे नैवान्य‚स्माद् व्यावृत्तिः । त‚त्रेति निरुपाख्येषु [।] क‚थं‚{\tiny $_{lb}$}‚ स्व‚भाव‚भेद‚विष‚या व्यावृत्तिविष‚याः श‚ब्दाः ।
	{\color{gray}{\rmlatinfont\textsuperscript{§~\theparCount}}}
	\pend% ending standard par
      ‚{\tiny $_{lb}$}‚

	  
	  \pstart \leavevmode% starting standard par
	\leavevmode\ledsidenote{\textenglish{125a/PSVTa}} न‚नु निरुपाख्येषु श‚ब्दानां प्र‚वृत्तिरेव नास्ति त‚त्किम‚र्थ‚मिद‚माशंकित‚मि‚{\tiny $_{७}$}‚ति‚{\tiny $_{lb}$}‚ क‚दाचित् प‚रो ब्रूते । त‚न्निराक‚र‚णार्थ‚माचार्यः प्राह । \textbf{तेष्व‚व}श्यं \textbf{श‚ब्द‚प्र‚वृ}त्या भाव्यं ।‚{\tiny $_{lb}$}‚ ये तु न चान्यापोह‚वादिना श‚क्य‚म्व‚क्तुन्नैव निरुपाख्येषु श‚ब्दानां प्र‚वृत्तिरिति ।‚{\tiny $_{lb}$}‚ य‚त‚स्तेष्व‚व‚श्यं श‚ब्द‚प्र‚वृत्त्या भाव्य‚मिति चोद‚को ब्रूत इति व्याच‚क्ष‚ते । तेषाम‚नेन क्र‚मेण‚{\tiny $_{lb}$}‚ देश‚काल‚निषेध एव स‚र्व‚भावेषु क्रिय‚ते । \textbf{त‚था स‚म्ब‚न्ध‚स्य स्व‚रूप‚णान‚भिधान‚मुक्तं} ।‚{\tiny $_{lb}$}‚  ‚{\tiny $_{lb}$}‚ ‚{\tiny $_{lb}$}‚ \leavevmode\ledsidenote{\textenglish{343/s}}एवं य‚त्पुन‚रेत‚त् त‚द‚र्थ‚{\tiny $_{१}$}‚निषेधेऽन‚र्थ‚क‚श‚ब्दाप्र‚योगात् । निर्विष‚य‚स्य न‚ञोऽप्र‚योग‚{\tiny $_{lb}$}‚ इत्य‚त्रोत्त‚र‚म्व‚क्ष्य‚त इत्यादि ग्र‚न्थो व‚क्ष्य‚माण‚श्चोद‚काभिप्रायेणावाच‚कः स्यात् ।
	{\color{gray}{\rmlatinfont\textsuperscript{§~\theparCount}}}
	\pend% ending standard par
      ‚{\tiny $_{lb}$}‚

	  
	  \pstart \leavevmode% starting standard par
	\hphantom{.}त‚स्मादा चा र्य एव निरुपाख्येषु श‚ब्द‚प्र‚वृत्तिं स‚म‚र्थ‚य‚ते तेष्व‚व‚श्यं श‚ब्द‚प्र‚वृत्त्या‚{\tiny $_{lb}$}‚ भाव्य‚मित्या दिना ग्र‚न्थेन । क‚स्मात् [।] क‚थंचिज्ज्ञान‚श‚ब्द‚विष‚य‚त्वेनाव्य‚व‚स्थापितेषु‚{\tiny $_{lb}$}‚ निरुपाख्येषु स‚र्व‚त्रार्थे विधिप्र‚तिषेधे वा योगात्‚{\tiny $_{२}$}‚ । य‚दि क्व‚चिद‚स‚त आकार‚स्य‚{\tiny $_{lb}$}‚ निषेधे ज्ञानाभिधाने स्यातां । त‚दा निषिद्धाकार‚प‚रिहृतेर्थे विधिः स्यात् । \textbf{त‚था चे}ति‚{\tiny $_{lb}$}‚ विधिप्र‚तिषेधाभावे । अन्व‚य‚व्य‚तिरेकौ विधिप्र‚तिषेधौ आश्र‚यो य‚स्य व्य‚व‚हार‚स्य‚{\tiny $_{lb}$}‚ स त‚थोक्तः । त‚मेव व्य‚व‚हार‚भाव‚मुष्णेत्यादिनाऽह । उष्ण‚स्व‚भावोग्निरित्य‚न्व‚याश्र‚यो‚{\tiny $_{lb}$}‚ व्य‚व‚हारः । नानुष्ण इति व्य‚तिरेकाश्र‚यः । अय‚म‚प्य‚तिप्र‚सिद्धो‚{\tiny $_{३}$}‚ लोक‚व्य‚व‚हारो न‚{\tiny $_{lb}$}‚ स्यादित्य‚पिश‚ब्देनाह । उष्ण‚व्य‚व‚स्था ह्य‚नुष्ण‚व्य‚व‚च्छेदेन [।]
	{\color{gray}{\rmlatinfont\textsuperscript{§~\theparCount}}}
	\pend% ending standard par
      ‚{\tiny $_{lb}$}‚

	  
	  \pstart \leavevmode% starting standard par
	त‚स्य चानुष्ण‚स्योष्णाभाव‚ल‚क्ष‚ण‚स्य क‚थ‚ञ्चिद् व्य‚व‚स्थानात् क‚थ‚न्त‚द्व्य‚व‚च्छेदे‚{\tiny $_{lb}$}‚नोष्णं व्य‚व‚स्थाप्येत । त‚दाह । \textbf{स्व‚भावान्त‚रेत्या}दि । उष्णाभाव एवोष्ण‚स्व‚भावा‚{\tiny $_{lb}$}‚द‚न्त‚र‚म्विल‚क्ष‚ण‚न्त‚द्विर‚ह‚रूपेणेति स्व‚भावान्त‚र‚मुक्तं । अत एवास‚त इत्येत‚द् घ‚ट‚ते ।
	{\color{gray}{\rmlatinfont\textsuperscript{§~\theparCount}}}
	\pend% ending standard par
      ‚{\tiny $_{lb}$}‚

	  
	  \pstart \leavevmode% starting standard par
	न‚नु स्व‚भाव‚विशेषः स्व‚भावान्त‚{\tiny $_{४}$}‚र‚न्त‚स्य क‚थंचिद‚पि विक‚ल्प‚बुद्धेः श‚ब्द‚स्य च‚{\tiny $_{lb}$}‚ विष‚य‚त्वेनाव्य‚व‚स्थाप‚नात् । स‚र्व‚थानुष्ण‚स्याप्र‚तिप‚त्तेर‚निश्च‚यात् त‚द्व्य‚व‚च्छेद‚ल‚क्ष‚{\tiny $_{lb}$}‚ण‚स्याग्निस्व‚भाव‚स्याप्र‚तिप‚त्तिर‚निश्च‚यः । य‚थाग्निस्व‚भाव‚स्यैवं स‚र्व‚स्य प‚दार्थ‚स्य ।‚{\tiny $_{lb}$}‚ त‚तो व्यामूढ‚म‚निश्चित‚रूपं ज‚ग‚त् स्यात् ॥
	{\color{gray}{\rmlatinfont\textsuperscript{§~\theparCount}}}
	\pend% ending standard par
      ‚{\tiny $_{lb}$}‚

	  
	  \pstart \leavevmode% starting standard par
	स्यादेत‚त् [।] न त‚त्र व‚ह्न्यादौ क‚स्य‚चिद‚स‚तो निषेधो येनाभावेप्य‚व‚श्यं प्र‚वृत्त्या‚{\tiny $_{lb}$}‚ भाव्य‚मिति चोद्य‚ते ।‚{\tiny $_{५}$}‚ किन्त्व‚नुष्णं स्प‚र्शाख्यं \textbf{स‚देव} व‚स्त्वेव । अग्नेश्चार्थान्त‚रं‚{\tiny $_{lb}$}‚ निषिध्य‚त इति ।
	{\color{gray}{\rmlatinfont\textsuperscript{§~\theparCount}}}
	\pend% ending standard par
      ‚{\tiny $_{lb}$}‚

	  
	  \pstart \leavevmode% starting standard par
	\textbf{क‚थ}मित्याद्या चा र्यः । \textbf{स‚देवे}ति व‚च‚नात् स‚त्त्व‚मिष्टं । निषिध्य‚त इति‚{\tiny $_{lb}$}‚ व‚च‚नाद‚स‚त्त्व‚मेकं च क‚थं स‚द‚स‚न्नाम ॥ नेत्यादिना प‚रः प‚रिह‚र‚ति । त‚त्राग्नाव‚नुष्णं‚{\tiny $_{lb}$}‚ नास्तीत्य‚नेन स‚र्व‚त्रानुष्ण‚म‚स‚दिति ब्रूमः । एवं ह्युच्य‚माने स‚त्त्वं प्र‚तिज्ञाय पुनः स‚र्व‚त्र‚{\tiny $_{lb}$}‚ स‚त्त्व‚निषेधेस‚द‚स‚त्त्व‚मेक‚त्र प्र‚{\tiny $_{६}$}‚तिज्ञात‚म्भ‚वेत् । केव‚ल‚न्त‚त्र त्व‚ग्नाव‚नुष्ण‚न्नास्तीति ब्रूमः ।‚{\tiny $_{lb}$}‚ त‚त‚श्चान्य‚त्र स‚तोन्य‚त्रास‚त्त्व‚म‚विरुद्धं । अय‚मेव च देश ह\edtext{}{\lemma{ह}\Bfootnote{?}}इत्याह । इह नास्तीति‚{\tiny $_{lb}$}‚ देश‚स्य निषेधः । इदानीन्नास्तीति काल‚स्य । अनेन प्र‚कारेण नास्तीति ध‚र्म‚स्य ।‚{\tiny $_{lb}$}‚ ध‚र्मिणो निषेधः । क‚स्मात् [।] \textbf{त‚न्निषेधे} ध‚र्मिणो निषेधे । \textbf{त‚द्विष‚य‚स्य} ध‚र्मिविष‚य‚स्य‚{\tiny $_{lb}$}‚ श‚ब्द‚स्य निर्विष‚य‚त‚या प्र‚वृ‚{\tiny $_{७}$}‚त्त्य‚भावात् । त‚त‚श्च ध‚र्मिश‚ब्दाप्र‚वृत्तेर‚निर्दिष्टो विष‚यो \leavevmode\ledsidenote{\textenglish{125b/PSVTa}}‚{\tiny $_{lb}$}‚ ‚{\tiny $_{lb}$}‚ \leavevmode\ledsidenote{\textenglish{344/s}}य‚स्य न‚ञ‚स्त‚स्याप्र‚योगात् । इद‚मिह नास्तीत्य‚व‚श्य‚मिद‚मादिप‚दैर्विष‚यः प्र‚तिषेध‚स्यो‚{\tiny $_{lb}$}‚प‚स्थाप्योन्य‚था किम्विष‚योयं प्र‚तिषेध इत्येवं स न ज्ञाय‚ते । एत‚च्च स‚र्व मु द्यो त क‚{\tiny $_{lb}$}‚ रादीनाम्म‚त‚मुप‚न्य‚स्तं ।
	{\color{gray}{\rmlatinfont\textsuperscript{§~\theparCount}}}
	\pend% ending standard par
      ‚{\tiny $_{lb}$}‚

	  
	  \pstart \leavevmode% starting standard par
	सोपीत्या चा र्यः । \textbf{त‚त्रा}पीत्या चा र्यः । \textbf{त‚त्रापी}त्यादिदेश‚काल‚ध‚र्म‚निषेधेन‚{\tiny $_{lb}$}‚ देशादीनां निषेधः स‚त्त्वात् । \textbf{आदि}श‚ब्दात् काल‚स्य प‚{\tiny $_{१}$}‚रिग्र‚हः । व्य‚क्तिभेदाद्‚{\tiny $_{lb}$}‚ ब‚हुव‚च‚नं । नार्थ‚स्येति ध‚र्म‚स्य । क्व‚चित् स‚त्त्वादेव । न त्व‚र्थ‚श‚ब्देन ध‚र्मिणो निर्देशः ।‚{\tiny $_{lb}$}‚ प‚रेणापि ध‚र्मिनिषेध‚स्यानिष्ट‚त्वात् ॥
	{\color{gray}{\rmlatinfont\textsuperscript{§~\theparCount}}}
	\pend% ending standard par
      ‚{\tiny $_{lb}$}‚

	  
	  \pstart \leavevmode% starting standard par
	स्यादेत‚न्न देशादिनिषेधः क्रिय‚ते नाप्युष्ण‚स्य निषेधः । किन्त्व‚नुष्णेन स‚हाग्नेर्यः‚{\tiny $_{lb}$}‚ स‚म्ब‚न्ध‚स्स निषिध्य‚ते ।
	{\color{gray}{\rmlatinfont\textsuperscript{§~\theparCount}}}
	\pend% ending standard par
      ‚{\tiny $_{lb}$}‚

	  
	  \pstart \leavevmode% starting standard par
	\textbf{न‚त्वि}त्याद्या चा र्यः । \textbf{त‚न्निषेधेपी}ति स‚म्ब‚न्ध‚निषेधेपि तुल्यो दोषः । ध‚र्म्मिव‚त्‚{\tiny $_{lb}$}‚ स‚म्ब‚न्ध‚स्याप्य‚निषेधात् । त‚देवास‚{\tiny $_{२}$}‚ स\edtext{}{\lemma{स}\Bfootnote{?}}तीत्यादिना साध‚य‚ति । अस‚ति स‚म्ब‚न्धे‚{\tiny $_{lb}$}‚ श‚ब्दाप्र‚वृत्तिः । आदिश‚ब्दाद‚निर्दिष्ट‚विष‚य‚स्य न‚ञोऽप्र‚योगात् । अथ‚वा तुल्यो दोषः [।]‚{\tiny $_{lb}$}‚ क‚थं । निषेधाद‚स‚ति यो निषेध‚स्त‚स्य त्व‚यैव निषेधादिति व्याख्येयं । क‚थं निषेध‚{\tiny $_{lb}$}‚ इत्याह ॥ \textbf{अस‚ती}त्यादि । अस‚तो वास्येति स‚म्ब‚न्ध‚स्य निषेधे । \textbf{त‚द्व‚त् स‚म्ब‚न्ध‚व‚त्} ।‚{\tiny $_{lb}$}‚ ध‚र्मिणोपि निषेधः ॥
	{\color{gray}{\rmlatinfont\textsuperscript{§~\theparCount}}}
	\pend% ending standard par
      ‚{\tiny $_{lb}$}‚

	  
	  \pstart \leavevmode% starting standard par
	नेत्यादि प‚रः । स‚म्ब‚न्धो नास्तीत्येव स्व‚श‚ब्देन न वै स‚म्ब‚{\tiny $_{३}$}‚न्ध‚स्य निषेधः ।
	{\color{gray}{\rmlatinfont\textsuperscript{§~\theparCount}}}
	\pend% ending standard par
      ‚{\tiny $_{lb}$}‚

	  
	  \pstart \leavevmode% starting standard par
	एत‚दुक्त‚म्भ‚व‚ति । य‚था स‚म्ब‚न्ध‚स्य स्व‚श‚ब्देन स्व‚रूपेणाभिधान‚न्नास्ति त‚था‚{\tiny $_{lb}$}‚ निषेधेपि । त‚देवाह । किन्त‚र्हि नेह प्र‚देशे घ‚टो [।] नेदानीं काल इत्येवं प्र‚तिषे‚{\tiny $_{lb}$}‚धोक्तौ स‚त्यां । नानेन देशेन कालेन वास्य घ‚ट‚स्य स‚म्ब‚न्धोस्तीति प्र‚तीतिः । त‚था‚{\tiny $_{lb}$}‚ नैव‚न्नाऽनेन प्र‚कारेण घ‚टोस्तीत्युक्तौ नैत‚द्ध‚र्मा घ‚ट इति प्र‚तीतिः । \textbf{त‚थे}त्येवं प्र‚तीतौ‚{\tiny $_{lb}$}‚ स‚त्यां ।
	{\color{gray}{\rmlatinfont\textsuperscript{§~\theparCount}}}
	\pend% ending standard par
      ‚{\tiny $_{lb}$}‚

	  
	  \pstart \leavevmode% starting standard par
	\textbf{त‚था}पीत्या चा‚{\tiny $_{४}$}‚र्यः । नेदानीमित्यादिनापि क‚थं स‚म्ब‚न्धो निषिद्धो याव‚द‚स्य‚{\tiny $_{lb}$}‚ पुंसः स‚म्ब‚न्धो ध‚र्मो वा नास्तीति म‚तिर्भ‚व‚ति । नेह नेदानीमिति प्र‚तिषेधे स‚म्ब‚न्धो‚{\tiny $_{lb}$}‚ नास्तीति म‚तिः । नैव‚मिति प्र‚तिषेधे ध‚र्मो नास्तीति म‚तिः ॥
	{\color{gray}{\rmlatinfont\textsuperscript{§~\theparCount}}}
	\pend% ending standard par
      ‚{\tiny $_{lb}$}‚

	  
	  \pstart \leavevmode% starting standard par
	न‚नु ध‚र्म‚निषेधोऽपि स‚म्ब‚न्ध‚निषेध एव । ध‚र्म‚ध‚र्मिणोः स‚म्ब‚न्ध‚निषेधात् ।‚{\tiny $_{lb}$}‚ स‚त्त्यं । संयोग‚स‚म‚वाय‚ल‚क्ष‚ण‚स‚म्ब‚न्ध‚भेदात्तु स‚म्ब‚न्धो ध‚र्मो वेति भेदेनोक्तं‚{\tiny $_{५}$}‚ ।‚{\tiny $_{lb}$}‚ न चास्यास्स‚म्ब‚न्धो नास्तीति म‚तेः क‚थंचिद् भावे स‚म्ब‚न्धादिस‚त्तायां स‚म्भ‚वः ।‚{\tiny $_{lb}$}‚ क‚स्माद\textbf{भावेष्वि}त्यादि । त‚थाश‚ब्दो य‚थाश‚ब्दार्थ‚माक्षिप‚ति । य‚त्त‚दोर्नित्याभि‚{\tiny $_{lb}$}‚स‚म्ब‚न्धात् । स‚त्सु देशादिषु य‚था नास्तीति बुद्धेर‚भावः । स‚त्स्व‚भाव‚बुद्धेर्विरोधात् ।‚{\tiny $_{lb}$}‚ त‚द्व‚द‚भावेषु । अस‚ति बुद्धिप्र‚वृत्तेर‚न‚भ्युप‚ग‚मात् । त‚स्मात् स‚म्ब‚न्धाभाव‚प्र‚तीतेः‚{\tiny $_{lb}$}‚ स‚काशान्नाय‚मिहेत्याद्या प्र‚तीतिः‚{\tiny $_{६}$}‚ । सा त‚द‚भावे स‚म्ब‚न्धाभाव‚प्र‚तीत्य‚भावे न स्यात्‚{\tiny $_{lb}$}‚ ‚{\tiny $_{lb}$}‚ \leavevmode\ledsidenote{\textenglish{345/s}}य‚द्वा \textbf{से}ति स‚म्ब‚न्धाभाव‚प्र‚तीतिः । \textbf{त‚द‚भाव} इति स‚म्ब‚न्धाभावे । प्र‚तीतौ वा‚{\tiny $_{lb}$}‚ \textbf{त‚द‚भाव‚स्य} स‚म्ब‚न्धाभाव‚स्य । तादृशी स‚म्ब‚न्धाभाव‚प्र‚तीतिस्सा विद्य‚ते य‚स्य‚{\tiny $_{lb}$}‚ पुंसः । त‚स्य य‚थाप्र‚तीतिम‚तः । त‚त्प्र‚भ‚वास्स‚म्ब‚न्धाभाव‚प्र‚तीतिज‚न्मान‚स्स‚म्ब‚न्धाभाव‚{\tiny $_{lb}$}‚विष‚याः श‚ब्दाः केन वार्य‚न्ते । विक‚ल्पानाम्वि‚{\tiny $_{७}$}‚ष‚यः स‚म्ब‚न्धाभावो न श‚ब्दा- \leavevmode\ledsidenote{\textenglish{126a/PSVTa}}‚{\tiny $_{lb}$}‚ नामिति चेदाह । \textbf{स एव ही}त्यादि । यो न वित‚र्काणाम्विष‚यः स एव न श‚ब्दा‚{\tiny $_{lb}$}‚नाम्विष‚यः । विक‚ल्प‚विष‚य‚स्त्व‚व‚श्यं श‚ब्द‚विष‚य इत्य‚र्थः । \textbf{ते चेत् प्र‚वृत्ता} इति‚{\tiny $_{lb}$}‚ वित‚र्काः ॥
	{\color{gray}{\rmlatinfont\textsuperscript{§~\theparCount}}}
	\pend% ending standard par
      ‚{\tiny $_{lb}$}‚

	  
	  \pstart \leavevmode% starting standard par
	न‚नु पुरोव‚स्थिते नीलादौ नील‚मित्यादिविक‚ल्पः । स्व‚ल‚क्ष‚ण‚विष‚यो न च‚{\tiny $_{lb}$}‚ स्व‚ल‚क्ष‚णं श‚ब्द‚वाच्य‚मित्याह । न हीत्यादि । अवाच्य‚म‚र्थ‚मिति स्व‚ल‚क्ष‚णं । विक‚{\tiny $_{lb}$}‚ल्पा‚{\tiny $_{१}$}‚धिकाराद् विक‚ल्प‚बुद्ध‚यो गृह्य‚न्ते । स‚मीह‚न्त इत्याल‚म्ब‚न्ते । सामान्याकारैव‚{\tiny $_{lb}$}‚ स‚दा विक‚ल्प‚बुद्धिर्यापि स‚न्निहिते स्व‚ल‚क्ष‚णे । स‚विक‚ल्प‚निर्विक‚ल्प‚योस्तु यौग‚{\tiny $_{lb}$}‚प‚द्याद‚भिमान एष म‚न्द‚म‚तीनां विक‚ल्पः स्व‚ल‚क्ष‚णाकार इति [।] विचारितं चैत‚त्‚{\tiny $_{lb}$}‚ \textbf{प्र‚माण‚विनिश्च‚ये} शास्त्र‚कारेणेति नेह प्र‚त‚न्य‚ते ।
	{\color{gray}{\rmlatinfont\textsuperscript{§~\theparCount}}}
	\pend% ending standard par
      ‚{\tiny $_{lb}$}‚

	  
	  \pstart \leavevmode% starting standard par
	न‚नु य‚दि विक‚ल्प‚बुद्ध्या विष‚यीकृत‚त्वात् स‚म्ब‚न्धाभावो वाच्यः । एव‚न्त‚र्हि‚{\tiny $_{lb}$}‚ स‚म्ब‚न्धोपि वाच्यः स्याद् विक‚ल्प‚बुद्ध्या विष‚यीकृत‚त्वाद्[।]अथेष्य‚त एव । क‚थ‚न्त‚{\tiny $_{lb}$}‚र्ह्याचार्य दि ङ् ना गे न त‚स्यावाच्य‚त्व‚मुक्त‚मित्याह । \textbf{स‚म्ब‚न्ध}स्येत्यादि । \textbf{स्वेन रूपे}णेति‚{\tiny $_{lb}$}‚ स‚म्ब‚न्ध‚रूपेण [।] स‚म्ब‚न्धिनं स‚म्ब‚न्ध इति प‚र‚स्प‚रापेक्षाल‚क्ष‚ण‚म्भाव‚मात्र‚म‚द्र‚व्य‚भूतः‚{\tiny $_{lb}$}‚ स‚म्ब‚न्ध‚स्य स्व‚भावः । तेन च रूपेण त‚स्याभिधाय‚कः श‚ब्दो नास्ति । स‚म्ब‚न्ध‚श‚ब्दो‚{\tiny $_{lb}$}‚ हि त‚स्याभिधाय‚क‚{\tiny $_{३}$}‚ एष्ट‚व्यः स च प्र‚युक्तः क‚योरित्याक्षिप‚ति । त‚त्र राज‚पुरुष‚योः‚{\tiny $_{lb}$}‚ स‚म्ब‚न्ध इत्युच्य‚माने राज‚पुरुष‚योरित्य‚स्य व्य‚तिरेक‚स्य हेतुः स‚म्ब‚न्ध‚स्त‚दा स‚{\tiny $_{lb}$}‚ स‚म्ब‚न्धः स‚म्ब‚न्धिरूपेण प्र‚तीय‚ते । त‚दाह । \textbf{अभिधानेन स‚म्ब‚न्धित्वेने}त्यादि ।‚{\tiny $_{lb}$}‚ राज‚पुरुष‚योः स‚म्ब‚न्ध इत्य‚भिधाने राज‚पुरुषाभ्यां प‚र‚स्प‚र‚सापेक्षाभ्यां निष्कृष्ट‚{\tiny $_{lb}$}‚रूप‚स्यैव स‚म्ब‚न्ध‚स्य स‚म्ब‚न्धित्वेन \textbf{बु‚{\tiny $_{४}$}‚द्धावुप‚स्थानात् । य‚थाभिप्राय‚म‚प्र‚तीतः} ।‚{\tiny $_{lb}$}‚ राज्ञः पुरुष इति प‚र‚स्प‚रापेक्षाल‚क्ष‚ण‚स्स‚म्ब‚न्धो य‚था ज्ञातुमिष्ट‚स्तेन रूपेणाप्र‚तीतिः ।‚{\tiny $_{lb}$}‚ \textbf{त‚दि}ति त‚स्माद‚यं स‚म्ब‚न्धः \textbf{प्र‚तीय‚मानोपि} स‚म्ब‚न्ध‚श‚ब्दात् । य‚थोक्त‚विधिना \textbf{स‚म्ब‚{\tiny $_{lb}$}‚न्धिरूप एवे}ति न स‚म्ब‚न्धे\textbf{नाभिधीय‚ते । त‚स्मान्नाभाव‚व‚त् स‚म्ब‚न्धेपि प्र‚संग‚{\tiny $_{lb}$}‚ इति} स‚म्ब‚न्धाभावो य‚था बुद्ध्या विष‚यीक्रिय‚त इ‚{\tiny $_{५}$}‚ति \textbf{वाच्यः} प्र‚स‚क्तो नैवं स‚म्ब‚न्धेपि‚{\tiny $_{lb}$}‚ वाच्य‚त्व‚मित्य‚र्थः । स‚म्ब‚न्धाभावो हि \textbf{स्वेन रूपेण} बुद्ध्या विष‚यीक्रिय‚ते । श‚ब्दे‚{\tiny $_{lb}$}‚नापि त‚थैवाभिधीय‚ते । स‚म्ब‚न्ध‚स्तु स्व‚रूपेण गृह्य‚ते । नाप्य‚भिधीय‚ते । स‚म्ब‚न्धि‚{\tiny $_{lb}$}‚रूपाप‚न्न‚स्यैव विष‚यीक‚र‚णाद‚भिधानाच्च । त‚था चाह ।
	{\color{gray}{\rmlatinfont\textsuperscript{§~\theparCount}}}
	\pend% ending standard par
      ‚{\tiny $_{lb}$}‚

	  
	  \pstart \leavevmode% starting standard par
	अस‚त्त्व‚भूत‚स्स‚म्ब‚न्धो रूप‚न्त‚स्य न गृह्य‚ते ।
	{\color{gray}{\rmlatinfont\textsuperscript{§~\theparCount}}}
	\pend% ending standard par
      ‚{\tiny $_{lb}$}‚‚{\tiny $_{lb}$}‚\textsuperscript{\textenglish{346/s}}

	  
	  \pstart \leavevmode% starting standard par
	नाभिधानं स्व‚रूपेण स‚म्ब‚न्ध‚स्य क‚थ‚ञ्च‚नेति ।
	{\color{gray}{\rmlatinfont\textsuperscript{§~\theparCount}}}
	\pend% ending standard par
      ‚{\tiny $_{lb}$}‚

	  
	  \pstart \leavevmode% starting standard par
	त‚स्मात् स्थित‚मेत‚द् विक‚ल्प‚विष‚योव‚श्य‚म्वाच्य इति । त‚त‚श्च य‚दि नास्ति‚{\tiny $_{lb}$}‚ स‚म्ब‚न्ध इति म‚तिस्त‚दा त‚त्प्र‚भ‚वोपि श‚ब्दः प्र‚व‚र्त्त‚त एव । त‚थाचाभाव‚विष‚यः‚{\tiny $_{lb}$}‚ श‚ब्द आप‚तित एव ।
	{\color{gray}{\rmlatinfont\textsuperscript{§~\theparCount}}}
	\pend% ending standard par
      ‚{\tiny $_{lb}$}‚

	  
	  \pstart \leavevmode% starting standard par
	अथ माभूद‚य‚न्दोष इति स‚म्ब‚न्ध‚स्य नास्तीति बुद्ध्या विष‚यीक‚र‚णं नेष्य‚ते ।‚{\tiny $_{lb}$}‚ नेह घ‚ट इत्य‚त्र क‚स्य निषेधो [।] न ताव‚द् देशादेस्त‚स्य स‚त्त्वात् [।] न स‚म्ब‚न्ध‚स्य‚{\tiny $_{lb}$}‚ \leavevmode\ledsidenote{\textenglish{126b/PSVTa}} त‚द्भाव‚स्याग्र‚ह‚णादि‚{\tiny $_{७}$}‚ति य‚त्किञ्चिदेत‚त् ॥
	{\color{gray}{\rmlatinfont\textsuperscript{§~\theparCount}}}
	\pend% ending standard par
      ‚{\tiny $_{lb}$}‚

	  
	  \pstart \leavevmode% starting standard par
	\textbf{अपि चाभाव‚म‚भिधेयं यो ब्रूते तं} ब्रुवाणं \textbf{प्र‚ति} अय‚म‚भावा\textbf{न‚भिधान‚वादी} अभावो‚{\tiny $_{lb}$}‚ न वाच्य इति \textbf{प्र‚तिविद‚ध‚न्} प्र‚तिक्षि\textbf{प‚न्न‚ब्रुवाणः क‚थं प्र‚तिविद‚ध्यात्} । न ह्य‚भाव‚{\tiny $_{lb}$}‚श‚ब्द‚मुच्चार‚य‚ता अभाव‚स्य वाच्य‚त्वं श‚क्यं प्र‚तिपाद‚यितुं । अथेच्छ‚त्य‚भाव‚स्य‚{\tiny $_{lb}$}‚ व‚च‚न‚न्त‚दा व‚च‚ने वास्याभाव‚स्याभ्युप‚ग‚म्य‚माने \textbf{क‚थ‚म‚भावोनुक्तः} [।] उक्त एव ।‚{\tiny $_{lb}$}‚ अभावो न वाच्य इत्य‚{\tiny $_{१}$}‚नेनैवाभाव‚श‚ब्देन त‚स्योक्तेः ॥ अथ प‚रेणाभाव‚स्य वाच्य‚त्वं‚{\tiny $_{lb}$}‚ य‚दुच्य‚ते त‚द‚नुवादेन निषेधः क्रिय‚ते [।]--
	{\color{gray}{\rmlatinfont\textsuperscript{§~\theparCount}}}
	\pend% ending standard par
      ‚{\tiny $_{lb}$}‚

	  
	  \pstart \leavevmode% starting standard par
	--नाभावो वाच्य‚स्तेनादोष इति चेत् ।
	{\color{gray}{\rmlatinfont\textsuperscript{§~\theparCount}}}
	\pend% ending standard par
      ‚{\tiny $_{lb}$}‚

	  
	  \pstart \leavevmode% starting standard par
	न‚न्व‚नुवादेपि किम‚भाव‚स्य वाच्य‚ता न भ‚व‚ति येनैव‚मुच्य‚ते । त‚स्मादिष्ट‚स्यै‚{\tiny $_{lb}$}‚वाभाव‚स्य वाच्य‚ता । स्व‚भावो नैवास्ति \textbf{तेना}स‚त्त्वाद\textbf{व‚च‚न}म‚भाव‚स्येति चेदाह ।‚{\tiny $_{lb}$}‚ \textbf{अथाभाव‚मेवे}त्यादि । \textbf{तेने}त्य‚भाव‚स्या\textbf{स‚त्त्वेन} ।‚{\tiny $_{२}$}‚ इदानीमित्य‚भाव‚स्यास‚त्त्वे \textbf{त‚देवाभावो‚{\tiny $_{lb}$}‚ नास्तीति व‚च‚नं क‚थं} । अभावो नास्तीत्य‚स्यैवाभाव‚श‚ब्द‚स्य प्र‚योगो न स्यात् [।]‚{\tiny $_{lb}$}‚ क‚थ‚ञ्च न स्याद‚भाव‚स्यैवान‚भ्युप‚ग‚मात् । अभावो ह्य‚स्य वाच्यः स च नाभ्यु‚{\tiny $_{lb}$}‚प‚ग‚म्य‚ते । प‚र‚प‚रिक‚ल्पित‚स्याभाव‚स्य प्र‚तिषेध इति चेदिष्ट‚स्ताव‚द‚भाव‚विष‚यः‚{\tiny $_{lb}$}‚ श‚ब्दः । त‚स्मात् क‚थंचिद‚भाव‚व्य‚व‚हारं प्र‚व‚र्त्त‚य‚ताऽव‚श्य‚म‚भाव‚विष‚या ज्ञा‚{\tiny $_{३}$}‚न‚श‚ब्दा‚{\tiny $_{lb}$}‚ एष्ट‚व्याः ।
	{\color{gray}{\rmlatinfont\textsuperscript{§~\theparCount}}}
	\pend% ending standard par
      ‚{\tiny $_{lb}$}‚

	  
	  \pstart \leavevmode% starting standard par
	\textbf{य‚त्पुन}रेत‚दुक्त‚म् [।] \textbf{अर्थ‚निषेधे} स‚त्य‚न‚र्थ\textbf{क‚श‚ब्दाप्र‚योगात्} कार‚णा\textbf{न्निर्विष‚य‚स्य‚{\tiny $_{lb}$}‚ न‚ञोप्र‚योग इत्य‚त्रोत्त‚र‚म्व‚क्ष्य‚तें} ।
	{\color{gray}{\rmlatinfont\textsuperscript{§~\theparCount}}}
	\pend% ending standard par
      ‚{\tiny $_{lb}$}‚

	  
	  \pstart \leavevmode% starting standard par
	\hphantom{.}अनादिवास‚नोद्भूत‚विक‚ल्प‚प‚रिनिष्ठ‚त \href{http://sarit.indology.info/?cref=pv.3.204}{१ । २०७} इत्यादिना ।
	{\color{gray}{\rmlatinfont\textsuperscript{§~\theparCount}}}
	\pend% ending standard par
      ‚{\tiny $_{lb}$}‚

	  
	  \pstart \leavevmode% starting standard par
	\textbf{त‚स्मा}दित्युप‚संहारः । इय‚ता च ग्र‚न्थेन य‚दुक्त‚न्तेष्व‚व‚श्यं श‚ब्द‚प्र‚वृत्त्या भाव्य‚{\tiny $_{lb}$}‚मिति त‚देवाचा र्ये ण स‚म‚र्थितं ।
	{\color{gray}{\rmlatinfont\textsuperscript{§~\theparCount}}}
	\pend% ending standard par
      ‚{\tiny $_{lb}$}‚

	  
	  \pstart \leavevmode% starting standard par
	अत्र प‚रः प्राह । य‚द्य\textbf{भावेष्व‚पि श‚ब्दास्स‚न्ति ते}ष्व‚{\tiny $_{४}$}‚भावेषु \textbf{क‚थं स्व‚भाव‚भेदः}‚{\tiny $_{lb}$}‚ श‚ब्द‚प्र‚व‚त्तिहेतुर्येनापोह‚विष‚य‚त्व‚म‚भाव‚प्र‚तिपाद‚कानां स्यात् ।‚{\tiny $_{lb}$}‚ ‚{\tiny $_{lb}$}‚ \leavevmode\ledsidenote{\textenglish{347/s}}अत्रोत्त‚र‚माहा चा र्यः । \textbf{त‚त्रापी}त्यादि । \textbf{रूपाभावा}दिति स्व‚रूपाभावाद‚भाव‚स्य ।‚{\tiny $_{lb}$}‚ \textbf{रूपाभिधायिनः} स्व‚भाव‚ग्राह‚काः \textbf{श‚ब्दा नाशंक्या एव} । य‚त‚स्ते श‚ब्दा अभाव‚विष‚या‚{\tiny $_{lb}$}‚ \textbf{व्य‚व‚च्छेद‚स्या}न्यापोह‚स्य \textbf{वाच‚काः सिद्धा} एव ।
	{\color{gray}{\rmlatinfont\textsuperscript{§~\theparCount}}}
	\pend% ending standard par
      ‚{\tiny $_{lb}$}‚

	  
	  \pstart \leavevmode% starting standard par
	एत‚दुक्त‚म्भ‚व‚ति [।] अभाव‚विष‚या‚{\tiny $_{५}$}‚णां \textbf{श‚ब्दानां} भाव‚स्व‚रूपाग्राह‚क‚त्वा‚{\tiny $_{lb}$}‚द‚पोह‚विष‚य‚त्व‚न्त‚था भाव‚विष‚याणाम‚पि श‚ब्दानाम्भाव‚स्व‚रूपाग्राह‚क‚त्वाद‚पोह‚विष‚{\tiny $_{lb}$}‚य‚त्व‚मेव [।] केव‚लं केचिच्छ‚ब्दा भावाध्य‚व‚सायाद् भाव‚विष‚याः केचिद‚भावाध्य‚{\tiny $_{lb}$}‚व‚सायाद‚भाव‚विष‚या उच्य‚न्ते । \textbf{व‚स्तुनि वृत्ति}र्व्यापारो येषां श‚ब्दानान्तेषां \textbf{किं‚{\tiny $_{lb}$}‚ रूप‚म‚भिधेयं} विधिरूपेण व‚स्तुग्राह्य\textbf{माहोस्विद् भेदो}ऽन्य‚{\tiny $_{६}$}‚व्यावृत्तः स्व‚भावोध्य‚व‚{\tiny $_{lb}$}‚सीय‚त \textbf{इति श‚ङ्का स्यात् । अभाव‚स्तु विवेक‚ल‚क्ष‚ण} इति स्व‚भाव‚विर‚ह‚ल‚क्ष‚णः क‚स्मा‚{\tiny $_{lb}$}‚\textbf{न्निमित्तीक‚र्त्त‚व्य‚स्य रूप‚स्य} व‚स्तुस्व‚भाव‚स्य । \textbf{त‚द्भावे} त‚स्य रूप‚स्य स‚त्त‚या ह्य\textbf{भावा‚{\tiny $_{lb}$}‚योगात्} । त‚स्य रूप‚स्य भाव‚स्त‚द्भावः स एव ल‚क्ष‚णं य‚स्य भाव‚स्य स त‚थोक्तः । अय‚{\tiny $_{lb}$}‚मेव स मुख्यो विवेकोन्यापोहः । अन्य‚स्तु ग‚वादिश‚ब्द‚विष‚यो‚{\tiny $_{७}$}‚पोह‚निमित्त‚त्वाद‚पोह इत्यु- \leavevmode\ledsidenote{\textenglish{127a/PSVTa}}‚{\tiny $_{lb}$}‚ प‚च‚रितः स‚र्व‚भाव‚विर‚ह‚ल‚क्ष‚णः । त‚स्य विवेक‚स्य \textbf{त‚थाभाव‚ख्यापिन} इति विवेक‚रूपा‚{\tiny $_{lb}$}‚भिधायिनः ॥ \textbf{विवेक‚विष‚या इत्य}न्यापोह‚विष‚या । विक‚ल्पाश्च \textbf{विवेक‚विष‚या} इति‚{\tiny $_{lb}$}‚ स‚म्ब‚न्धः । ते श‚ब्दा विक‚ल्पाश्च । एकं व्यावृत्तिस‚माश्र‚य‚भूत‚म्व‚स्तु । \textbf{प्र‚तिस}\edtext{\textsuperscript{*}}{\lemma{*}\Bfootnote{? श}}‚{\tiny $_{lb}$}‚ \textbf{र‚ण‚म}धिष्ठानं येषां श‚ब्दानां विक‚ल्पानाञ्च । ते त‚थोक्ताः । त‚था ह्य‚कृत‚क‚व्य‚{\tiny $_{lb}$}‚व‚च्छेदेन य‚देव व‚स्तु कृत‚क‚श‚ब्द‚स्य विक‚ल्प‚स्य वाधिष्ठान‚न्त‚देवानित्यानात्मादि‚{\tiny $_{lb}$}‚श‚ब्दानाम्विक‚ल्पानां च । ते त‚थाभूता अपि भिन्न‚विष‚या एवेति स‚म्ब‚न्धः ।‚{\tiny $_{lb}$}‚ क‚स्माद् \textbf{य‚थास्व‚मि}त्यादि । या व्यावृत्तिर्य‚तो व्य‚व‚स्थाप्य‚ते सा त‚स्या अव‚धिः ।‚{\tiny $_{lb}$}‚ य‚था कृत‚काख्य‚स्य व्य‚व‚च्छेद‚स्याकृत‚कः । एव‚म‚नित्य‚त्व‚ल‚क्ष‚ण‚स्य व्य‚व‚च्छेद‚स्य‚{\tiny $_{lb}$}‚ नित्य इत्यादि । तेषां‚{\tiny $_{२}$}‚ \textbf{य‚थास्व‚म‚व}धीनां भेदास्तै\textbf{र्भेदैरुप‚क‚ल्पिता} र‚चिता अनित्य‚{\tiny $_{lb}$}‚‚{\tiny $_{lb}$}‚ \leavevmode\ledsidenote{\textenglish{348/s}}त्वादीनां विवेकिनां भेदाः प‚र‚स्प‚रं विशेषाः । \textbf{तैर्भेदैर्भिन्नेष्विव} विक‚ल्प\textbf{बुद्धौ प्र‚ति‚{\tiny $_{lb}$}‚भात्सु} प्र‚तिभास‚मानेषु ध‚र्मिषु । तेषां श‚ब्दाना\textbf{म्विवेकेषु} भेदेषु विक‚ल्पानां \textbf{चोप‚स्था‚{\tiny $_{lb}$}‚प‚नात्} । य‚थाक्र‚मं वाच‚क‚त्वेन ग्राह‚क‚त्वेन चोप‚श्लेषात् ।
	{\color{gray}{\rmlatinfont\textsuperscript{§~\theparCount}}}
	\pend% ending standard par
      ‚{\tiny $_{lb}$}‚

	  
	  \pstart \leavevmode% starting standard par
	न‚नु च कृत‚कानित्य‚त्व‚योर्नैवाव‚धिभेदोस्त्य‚कृत‚क‚स्येव नित्य‚रूप‚त्वात् ।‚{\tiny $_{३}$}‚‚{\tiny $_{lb}$}‚ त‚त‚श्च प्र‚तिज्ञार्थैक‚देश एव हेतुः ।
	{\color{gray}{\rmlatinfont\textsuperscript{§~\theparCount}}}
	\pend% ending standard par
      ‚{\tiny $_{lb}$}‚

	  
	  \pstart \leavevmode% starting standard par
	नैष दोषो य‚स्माद‚कृत‚क‚स्यापि प्राग्भाव‚स्यानित्य‚त्वात् । कृत‚क‚स्यापि प्र‚ध्वं‚{\tiny $_{lb}$}‚साभाव‚स्य नित्य‚त्वाद‚स्त्येवाव‚धिभेदः । य‚द्वा कार‚णेन कृतः श‚ब्दो न भ‚व‚तीत्य‚स्य‚{\tiny $_{lb}$}‚ स‚मारोप‚स्य व्य‚व‚च्छेदेन कृत‚को द्वितीयादिक्ष‚णे स्थायित्व‚स‚मारोप‚व्य‚व‚च्छेदेनानित्य‚{\tiny $_{lb}$}‚ इत्युच्य‚त इत्य‚स्त्येवाव‚धिभेदः । \textbf{ते}नेति य‚थोक्ते‚{\tiny $_{४}$}‚न व्यावृत्तिभेदेन स्व‚भाव‚हेतौ‚{\tiny $_{lb}$}‚ \textbf{स्व‚भाव‚स्यैव साध्य‚साध‚न‚भावेपि न साध्यासाध‚न‚योः संस‚र्ग} एक‚त्वं । \textbf{त‚त‚श्च} य‚दुक्तं‚{\tiny $_{lb}$}‚ स्व‚भावे साध्ये \textbf{प्र‚तिज्ञार्थैक‚देशो हेतुः स्यादिति} स दोषो नास्तीत्याह ।--
	{\color{gray}{\rmlatinfont\textsuperscript{§~\theparCount}}}
	\pend% ending standard par
      ‚{\tiny $_{lb}$}‚

	  
	  \pstart \leavevmode% starting standard par
	\textbf{त‚न्ने}ति [।] \textbf{त‚दि}ति त‚स्मात् ॥ \textbf{स चायं स्व‚भाव} इति स‚म्ब‚न्धः । स्व‚भाव‚{\tiny $_{lb}$}‚ इत्य‚व्य‚तिरिक्तो ध‚र्मः । स क‚दाचित् स‚त्त्व‚म‚न्यो वा । य‚द्य‚पि कृत‚के स‚त्त्व‚म‚स्ति‚{\tiny $_{lb}$}‚ स‚त्सु च कृत‚क‚त्व‚न्त‚था‚{\tiny $_{५}$}‚पि हेतुकृतोयं स्व‚भाव इत्येताव‚न्मात्र‚विवाक्षायां कृत‚को‚{\tiny $_{lb}$}‚ हेतुरुच्य‚ते । न तु साम‚र्थ्य‚विव‚क्षायां । प्र‚मेय‚त्वादिव‚त् । साम‚र्थ्य‚म‚स्त्येताव‚न्मात्र‚{\tiny $_{lb}$}‚विव‚क्षायाञ्च स‚त्त्वं हेतुरुच्य‚ते तेन तु हेतुकृत‚त्व‚विव‚क्षायां । \href{http://sarit.indology.info/?cref=pv.3.184-3.185}{१८७-८८}
	{\color{gray}{\rmlatinfont\textsuperscript{§~\theparCount}}}
	\pend% ending standard par
      ‚{\tiny $_{lb}$}‚

	  
	  \pstart \leavevmode% starting standard par
	तेन य‚दुच्य‚ते [।] कृत‚के स‚त्त्वं विद्य‚ते न च त‚स्यानित्य‚त्वे व्य‚भिचारोस्ति‚{\tiny $_{lb}$}‚ [।] त‚त्किमित्युपाधिभेदेन विशेष्य‚त इति त‚द‚पास्तं । कृत‚क‚त्वादौ‚{\tiny $_{६}$}‚ साम‚र्थ्य‚स्या‚{\tiny $_{lb}$}‚विव‚क्षित‚त्वादिति । हेतुत्वेनाप‚दिश्य‚मान उच्य‚मानः । \textbf{उपाधिभेदापेक्षो} विशेष‚ण‚{\tiny $_{lb}$}‚भेदापेक्षः । \textbf{केव‚लो} वेत्युपाध्य‚न‚पेक्षः । \textbf{साध्य‚सिद्ध्य‚र्थ‚मुच्य‚ते} । अनित्य‚त्वे साध्ये‚{\tiny $_{lb}$}‚ कृत‚क‚त्व‚मुपाधिभेदापेक्ष‚न्त‚द‚न‚पेक्ष‚न्तु स‚त्त्वं ॥ अपेक्षितेत्यादिनोपाधिभेदापेक्ष‚त्वं कृत‚{\tiny $_{lb}$}‚\leavevmode\ledsidenote{\textenglish{127b/PSVTa}} क‚त्व‚स्याह । प‚र‚स्याहेत्व‚भिम‚त‚स्य ज‚न‚न‚{\tiny $_{७}$}‚श‚क्तिरेव व्यापारः । अन्व‚य‚व्य‚तिरेकानु‚{\tiny $_{lb}$}‚विधान‚मेव चापेक्षा । \textbf{स्व‚भाव‚निष्प‚त्तौ} स्व‚भाव‚निष्प‚त्तिनिमित्त\textbf{म‚पेक्षितः प‚र‚व्यापारो}‚{\tiny $_{lb}$}‚ येन \textbf{भावेन} स कृत‚कः । संज्ञायां क‚नो विधानात् संज्ञाश‚ब्दोयं \textbf{कृत‚क} इति । य‚त‚{\tiny $_{lb}$}‚ ‚{\tiny $_{lb}$}‚ \leavevmode\ledsidenote{\textenglish{349/s}}एव‚न्तेनेयं कृत‚क‚श्रुतिः स्व‚भावाभिधायिन्य‚पि स‚ती \textbf{प‚रोपाधिम}त्य‚न्त‚विशेष‚ण‚मेनं‚{\tiny $_{lb}$}‚ स्व‚भाव‚मा\textbf{क्षिप‚ति । एतेनेति} कृत‚क‚त्व‚स्योपाधिभेदापेक्ष‚त्व‚प्र‚तिपाद‚नेन । \textbf{प्र‚त्य‚यानां}‚{\tiny $_{lb}$}‚ कार‚ण‚नाम्भेद‚स्तेन भेत्तुं शीलं य‚स्य स त‚थोक्त‚स्त‚द्भाव‚स्त‚त्त्वं । स्थान‚क‚र‚णादि‚{\tiny $_{lb}$}‚भेदाद् भिद्य‚ते श‚ब्दः । आदिश‚ब्दात् प्र‚य‚त्नान‚न्त‚रीय‚क\textbf{त्वाद‚यो व्याख्याताः} ।
	{\color{gray}{\rmlatinfont\textsuperscript{§~\theparCount}}}
	\pend% ending standard par
      ‚{\tiny $_{lb}$}‚

	  
	  \pstart \leavevmode% starting standard par
	\textbf{एव}मित्यादिनोप‚संहारः । \textbf{क्व‚चित् प्र‚योगे उपाधिभेदंप्र‚त्य‚न‚पेक्षः} । अत‚{\tiny $_{lb}$}‚एवाह । सामान्येन \textbf{अनित्य एव} साध्ये \textbf{य‚था स‚त्त्वं । स्व‚भाव‚भूत‚श्}चासौ \textbf{ध‚र्म}श्च‚{\tiny $_{lb}$}‚ त‚स्य \textbf{प‚रिग्र‚हेण}‚{\tiny $_{२}$}‚ । क्व‚चित्स्व‚भावो हेतुरुच्य‚ते इति प्र‚कृतं । \textbf{य‚था त‚त्रैवे}त्य‚नित्य‚त्वे‚{\tiny $_{lb}$}‚ साध्य \textbf{उत्प‚त्तिः} । न चोत्प‚त्तिरुत्प‚त्तिम‚तोन्याभाव‚स्याज‚न्य‚त्वेनोत्प‚त्त्य‚भाव‚प्र‚स‚ङ्गात् ।‚{\tiny $_{lb}$}‚ केव‚ल‚म‚र्थान्त‚र‚भूतेव‚क‚ल्पिताविशेष‚ण‚त्वेन तेनोत्प‚त्तेरित्युत्प‚त्तिम‚त्वादित्य‚र्थः । अय‚मु‚{\tiny $_{lb}$}‚पाध्य‚पेक्ष एव स्व‚भावो द्र‚ष्ट‚व्यः । कृत‚क‚त्वादौ प‚र‚भूत उपाधिर‚हित‚त्वात्म‚भूत एव‚{\tiny $_{lb}$}‚ ध‚र्म‚विशेष इत्येता‚{\tiny $_{३}$}‚वान् विशेषः ।
	{\color{gray}{\rmlatinfont\textsuperscript{§~\theparCount}}}
	\pend% ending standard par
      ‚{\tiny $_{lb}$}‚

	  
	  \pstart \leavevmode% starting standard par
	\textbf{अन‚या दि}शेति । उपाध्य‚पेक्षान‚पेक्ष‚हेतुप्र‚विभाग‚दिशा ॥ \href{http://sarit.indology.info/?cref=pv.3.186}{१८९}
	{\color{gray}{\rmlatinfont\textsuperscript{§~\theparCount}}}
	\pend% ending standard par
      ‚{\tiny $_{lb}$}‚

	  
	  \pstart \leavevmode% starting standard par
	य‚दि \textbf{स‚त्ताख्यः स्व‚भावो हेतुः} स‚त्त्व‚मिति याव‚त् । प्र‚धानादि\textbf{स‚त्ता क‚थं न‚{\tiny $_{lb}$}‚ साध्य‚ते} । अथ स‚त्ता-सामान्ये साध्ये सिद्ध‚साध्य‚ता स्याद‚तः स‚त्ताविशेष‚स्साध्य‚{\tiny $_{lb}$}‚स्त‚स्मिंश्च साध्ये विशेष‚स्यान‚न्व‚यात् साध्य‚शून्यो दृष्टान्तः स्याद‚तो न स‚त्ता \textbf{साध्य‚ते} ।‚{\tiny $_{lb}$}‚ त‚दा हेताव‚पि स‚त्त्वे विशेष‚स्यान‚न्व‚या‚{\tiny $_{४}$}‚त् साध‚न‚शून्यो दृष्टान्तः स्यात् । त‚दाह‚{\tiny $_{lb}$}‚ [।] \textbf{अन‚न्व‚यो ही}त्यादि । \textbf{भेदाना}म्विशेषाणां \textbf{व्याह‚तो} दुष्टो \textbf{हेतुसाध्य‚योः} । हेतौ‚{\tiny $_{lb}$}‚ साध्ये चेत्य‚र्थः ॥ अन्य‚त्र \textbf{चे}त्य‚नात्मादौ । \textbf{त‚दि}ति स‚त्त्वं । \textbf{किल}श‚ब्दोन‚भिम‚तार्थे‚{\tiny $_{lb}$}‚ एव । प्र‚साध्य‚मान‚मिति । \href{http://sarit.indology.info/?cref=pv.3.186-3.187}{१८९-९०}
	{\color{gray}{\rmlatinfont\textsuperscript{§~\theparCount}}}
	\pend% ending standard par
      ‚{\tiny $_{lb}$}‚

	  
	  \pstart \leavevmode% starting standard par
	अस्ति \textbf{प्र‚धान‚मि}त्यादिना \textbf{प्र‚धान}ल‚क्ष‚ण‚योगेन \textbf{विशेषीभ‚व‚ति न च विशेषः‚{\tiny $_{lb}$}‚ ‚{\tiny $_{lb}$}‚ \leavevmode\ledsidenote{\textenglish{350/s}}साध‚यितुं श‚क्य‚ते त‚स्यान‚न्व‚यात् । य‚{\tiny $_{५}$}‚थाऽहे}त्याचार्य दि ग्ना गः । अस्ति प्र‚धान‚{\tiny $_{lb}$}‚मित्य‚नेन प्र‚धान‚स्व‚ल‚क्ष‚ण‚मेव साध्य‚त इति य‚त्सां ख्ये नोक्तं त\textbf{त्प्र‚माण}स्यानुमान‚स्य‚{\tiny $_{lb}$}‚ \textbf{विष‚याज्ञानात्} । सामान्य‚विष‚यं ह्य‚नुमानं स्व‚ल‚क्ष‚ण‚विष‚यं । \textbf{व्याह‚न्य‚ते} दुष्य‚ति ।‚{\tiny $_{lb}$}‚ \textbf{किन्त‚र्हि [।] हेताव‚पि तुल्य‚दोष‚त्वात्} । त‚देवाह [।] \textbf{न हि हेतुरि}त्यादि । न विद्य‚ते‚{\tiny $_{lb}$}‚ ऽन्व‚योस्येत्य‚न‚न्व‚यः \textbf{सिद्धेः} साध्य‚साध‚न‚स्य \textbf{ना‚{\tiny $_{६}$}‚ङ्गं} । कुत इत्य‚साधार‚णाद्धेतोः ।‚{\tiny $_{lb}$}‚ \textbf{भावः} स‚त्ता सआ \textbf{उपादान}म्विशेष‚णं य‚स्य ध‚र्मिण‚स्त\textbf{न्मात्रे} । त‚न्मात्र‚त्वादेव \textbf{सामा‚{\tiny $_{lb}$}‚न्य‚रूपे ध‚र्मिणि} साध्ये । सां ख्य स्य \textbf{न क‚श्चिद‚र्थः सिद्धः स्यात्} । त्रैगुण्यादिल‚क्ष‚ण‚{\tiny $_{lb}$}‚स्यासिद्धेः । \textbf{अनिषिद्ध‚ञ्च तादृशं} । तादृश‚मिति सामान्य‚मात्रं । अनेन सिद्ध‚सा‚{\tiny $_{lb}$}‚\leavevmode\ledsidenote{\textenglish{128a/PSVTa}} ध्य‚तामाह ॥‚{\tiny $_{७}$}‚ \href{http://sarit.indology.info/?cref=pv.3.187-3.188}{१९०-९१}
	{\color{gray}{\rmlatinfont\textsuperscript{§~\theparCount}}}
	\pend% ending standard par
      ‚{\tiny $_{lb}$}‚

	  
	  \pstart \leavevmode% starting standard par
	\textbf{न स‚र्व‚थे}त्यादिना व्याच‚ष्टे । \textbf{स‚त्तासाध‚न} इति स‚त्तासिद्धौ । \textbf{भाव‚मात्र‚विशे‚{\tiny $_{lb}$}‚ष‚ण} इति स‚त्तामात्र‚विशेष‚णः । \textbf{अनिर्दिष्टः स्व‚भाव‚विशेषो} य‚स्येति विग्र‚हः । \textbf{नेहेति}‚{\tiny $_{lb}$}‚ व‚स्तुमात्र‚साध‚ने । \textbf{स‚त्तासाध‚न‚प्र‚तिषेधः । किन्तु} स वादी \textbf{त‚था} सामान्ये नास्ति‚{\tiny $_{lb}$}‚ क‚श्चिदिति क‚ञ्च‚ना\textbf{स्य} ध‚र्मिणो \textbf{भेदं} विशेषं नित्य‚त्वादिक‚म\textbf{प‚रामृश‚न्न}संस्पृश‚न् ।‚{\tiny $_{lb}$}‚ \textbf{अनेने}ति वादिना ॥ \textbf{उपात्त‚भे‚{\tiny $_{१}$}‚द} इत्युपात्त‚विशेषः । त्रिगुणात्म‚को नित्य इत्यादि‚{\tiny $_{lb}$}‚ नोपात्त‚भेदे \textbf{साध्ये}स्मिन् प्र‚धानादिके ध‚र्मिणि । \textbf{भ‚वेद्धेतुर‚न‚न्व‚यः} । नास्य साध्य‚{\tiny $_{lb}$}‚ध‚र्म‚विशेष‚ण‚दृष्टान्तेन्व‚योस्तीत्य‚न‚न्व‚यः । य‚त एवं [।]
	{\color{gray}{\rmlatinfont\textsuperscript{§~\theparCount}}}
	\pend% ending standard par
      ‚{\tiny $_{lb}$}‚

	  
	  \pstart \leavevmode% starting standard par
	\textbf{स‚त्तायान्तेन साध्यायाम्विशेषः साधितो भ‚वेत्} ।
	{\color{gray}{\rmlatinfont\textsuperscript{§~\theparCount}}}
	\pend% ending standard par
      ‚{\tiny $_{lb}$}‚

	  
	  \pstart \leavevmode% starting standard par
	अन्ये तु \textbf{स‚त्ताया}मित्यादि प‚श्चाद् ध‚र्मादौ व्याख्याय । पूर्वार्द्ध‚मु\textbf{पात्त‚भेद‚मि}त्यादि‚{\tiny $_{lb}$}‚ प‚श्चाद् व्याच‚क्ष‚ते ।
	{\color{gray}{\rmlatinfont\textsuperscript{§~\theparCount}}}
	\pend% ending standard par
      ‚{\tiny $_{lb}$}‚‚{\tiny $_{lb}$}‚\textsuperscript{\textenglish{351/s}}

	  
	  \pstart \leavevmode% starting standard par
	स ही‚{\tiny $_{२}$}‚त्यादिना व्याच‚ष्टे । \textbf{एको} मूल‚प्र‚कृतेर्भेदाभावात् । \textbf{नित्यो} निर‚न्व‚य‚{\tiny $_{lb}$}‚विनाशाभावात् । त्रिगुण‚त्वात्म‚क‚त्वात् सुख‚दुःख‚मोहा\textbf{त्म‚कः । अन्यो} वेति क‚र्त्तृ‚{\tiny $_{lb}$}‚त्वादियुक्तः । \textbf{य‚थाक‚थंचिद‚पी}ति । य‚थोक्तैर्द्ध‚र्मैः स‚म‚स्तैर्व्य‚स्तैर्वा विशेषितः ।‚{\tiny $_{lb}$}‚ \textbf{त‚त्स्व‚भाव} इति य‚थोक्त‚विशेष‚ण‚विशिष्ट‚स्व‚भावः [।] \textbf{स च} ध‚र्मी \textbf{त‚थेति} विशिष्टेन‚{\tiny $_{lb}$}‚ स्व‚भावेन नान्वेति स‚प‚क्षे । त‚था भूत‚स्य‚{\tiny $_{३}$}‚ दृष्टान्त‚ध‚र्मिणोसिद्धेः । \textbf{न तेन सिद्धेने}ति‚{\tiny $_{lb}$}‚ स‚त्तामात्रेण । स‚त्तामात्रे विवादाभावात् । \href{http://sarit.indology.info/?cref=pv.3.188-3.189}{१९१-९२}
	{\color{gray}{\rmlatinfont\textsuperscript{§~\theparCount}}}
	\pend% ending standard par
      ‚{\tiny $_{lb}$}‚

	  
	  \pstart \leavevmode% starting standard par
	\textbf{न‚न्वि}त्यादि प‚रः । \textbf{एवं प्र‚स‚ङ्ग} इति यः सामान्य‚विशेष‚विक‚ल्पेन स‚त्ताया‚{\tiny $_{lb}$}‚मुक्तः । त‚दुक्तं [।]
	{\color{gray}{\rmlatinfont\textsuperscript{§~\theparCount}}}
	\pend% ending standard par
      ‚{\tiny $_{lb}$}‚
	  \bigskip
	  \begingroup
	
	    
	    \stanza[\smallbreak]
	  {\normalfontlatin\large ``\qquad}विशेषानुग‚माभावः सामान्ये सिद्ध‚साध्य‚तेति ।{\normalfontlatin\large\qquad{}"}\&[\smallbreak]
	  
	  
	  
	  \endgroup
	‚{\tiny $_{lb}$}‚

	  
	  \pstart \leavevmode% starting standard par
	य‚स्मात्त\textbf{त्राप्य}ग्न्यादिषु साध्येषु ना\textbf{ग्निसात्तायां क‚श्चिद्विवा}दोस्त्य‚ग्निमात्र‚स्य‚{\tiny $_{lb}$}‚ सिद्ध‚त्वात् । न च तेन सिद्धेन किञ्चित्‚{\tiny $_{४}$}‚ त‚स्याप्र‚वृत्त्य‚ङ्ग‚त्वात् । देशादिविशिष्टो‚{\tiny $_{lb}$}‚ ह्य‚ग्निः प्र‚वृत्त्य‚ङ्गं नाग्निमात्रं । स एव त‚र्हि साध्य इत्याह । विशिष्ट आधारो‚{\tiny $_{lb}$}‚ य‚त्राग्न्यादिकं साध्य‚ते स विशेष‚णं य‚स्याग्न्यादिक‚स्य स \textbf{विशिष्टाधार‚विशेष‚णं} [।]‚{\tiny $_{lb}$}‚ त‚स्य । साध्य‚त्वेना\textbf{भिम‚त‚स्य} । स‚प‚क्षेऽन‚न्व‚याद‚सिद्धिः ।
	{\color{gray}{\rmlatinfont\textsuperscript{§~\theparCount}}}
	\pend% ending standard par
      ‚{\tiny $_{lb}$}‚

	  
	  \pstart \leavevmode% starting standard par
	\textbf{ने}त्या चा र्यः । \textbf{न वै स आधारो} विशेष‚ण‚भावेन गृहीतोपि त‚म\textbf{ग्निम्विशेषी‚{\tiny $_{lb}$}‚क‚रोति} ।‚{\tiny $_{५}$}‚ येनान्व‚यः स्यात् । किं कार‚णं । \textbf{त‚द‚योग} इत्यादि । त‚स्य ध‚र्म‚स्य‚{\tiny $_{lb}$}‚ त‚स्मिन् ध‚र्मिण्य‚योगो य आशंकित‚स्त‚स्य \textbf{व्य‚व‚च्छ‚देन} विशेष‚णात् । एत‚च्च \textbf{प‚क्ष‚ध‚र्म}‚{\tiny $_{lb}$}‚ \href{http://sarit.indology.info/?cref=}{१ । ३} इत्य\textbf{त्रोक्तं । व‚क्ष्य‚ते च} च‚तुर्थे प‚रिच्छेदे \href{http://sarit.indology.info/?cref=}{४ । १४९} ॥
	{\color{gray}{\rmlatinfont\textsuperscript{§~\theparCount}}}
	\pend% ending standard par
      ‚{\tiny $_{lb}$}‚

	  
	  \pstart \leavevmode% starting standard par
	न‚नु च व्याप्तिग्र‚ह‚ण‚काले प्र‚देशायोग‚व्य‚व‚च्छिन्नो व‚ह्निर‚सिद्धः । त‚त्सिद्धौ‚{\tiny $_{lb}$}‚ वा किम‚र्थोन्व‚यानुग‚मः । क‚थ‚म् [।] असिद्धो य‚स्मात् । य‚त्र य‚त्र धूम‚स्त‚त्र त‚{\tiny $_{६}$}‚-त्राग्नि‚{\tiny $_{lb}$}‚रिति व्याप्तिं प्र‚तिय‚ता सामान्येनाभिम‚त‚देशायोग‚व्य‚व‚च्छिन्नोपि ब‚ह्निराक्षिप्त एव ।‚{\tiny $_{lb}$}‚ केव‚ल‚मिदानीम‚स्मिन्देशे व‚ह्निरित्येवं विशेष‚प्र‚तीत्य‚र्थ‚म‚न्व‚यानुग‚म‚न‚मिष्य‚ते ।
	{\color{gray}{\rmlatinfont\textsuperscript{§~\theparCount}}}
	\pend% ending standard par
      ‚{\tiny $_{lb}$}‚‚{\tiny $_{lb}$}‚\textsuperscript{\textenglish{352/s}}

	  
	  \pstart \leavevmode% starting standard par
	\textbf{त‚स्मा}दित्युप‚संहारः । \textbf{त‚त्रे}ति प्र‚देशादौ । त‚द‚योग‚व्य‚व‚च्छेदेनेति त‚स्मिन् प्र‚देशादौ‚{\tiny $_{lb}$}‚ \leavevmode\ledsidenote{\textenglish{128b/PSVTa}} ध‚र्मिणि साध्य‚ध‚र्म‚स्या\textbf{योग‚व्य‚व‚च्छेदेन} सामान्य‚स्याग्निमात्र‚स्य साध‚नात्‚{\tiny $_{७}$}‚ नास्त्य‚न्व‚य‚{\tiny $_{lb}$}‚दोषः । नापि सिद्ध‚साध्य‚ता प्र‚देशायोग‚व्य‚व‚च्छेद‚स्यासिद्ध‚त्वात् ॥ प्र‚धानादिके ध‚र्मिण्य‚{\tiny $_{lb}$}‚योग‚व्य‚व‚च्छेदेन स‚त्तामात्रं साध्य‚मिति चेदाह । न \textbf{त‚थे}त्यादि । क्व‚चिदिति प्र‚धाना‚{\tiny $_{lb}$}‚दिके ध‚र्मिणि । क‚स्मात् [।] \textbf{प्र‚धानादिश‚ब्द‚वाच्य‚स्यैवार्थ‚स्य} त्रैगुण्यादिल‚क्ष‚ण‚{\tiny $_{lb}$}‚स्यैवा\textbf{भावात् । निर्विशेषेणैव सा} स‚त्ता । विशेष‚ण‚भूत‚स्याधार‚स्या\textbf{भावात्} [।]
	{\color{gray}{\rmlatinfont\textsuperscript{§~\theparCount}}}
	\pend% ending standard par
      ‚{\tiny $_{lb}$}‚

	  
	  \pstart \leavevmode% starting standard par
	\textbf{क‚थ‚मित्या}दि प‚रः । सोपि ध‚{\tiny $_{१}$}‚र्मी \textbf{क‚थं । विज्ञात‚व्यः} । ज्ञात्वा च श‚ब्दे\textbf{नाभि}‚{\tiny $_{lb}$}‚धात‚व्यः । अभिहितः प्र‚माणेन \textbf{निश्चेत‚व्यः} । त‚स्माज्ज्ञेय‚त्वादिभिः सोपि सिद्ध‚{\tiny $_{lb}$}‚ \textbf{एव} । त‚स्मिन् स‚त्ता सामान्यं साध्य‚ते । \textbf{त‚त्कि}मिदानीं \textbf{ज्ञेय}न्निर्विशेष‚ण‚म\textbf{स्तीत्ये}‚{\tiny $_{lb}$}‚ताव‚ता प्र धा न स्य \textbf{सिद्धिर‚स्तु} । ज्ञेयाद्य‚र्थो हि प्र‚धानार्थः श‚ब्दार्थ‚रूपः स्यान्न‚{\tiny $_{lb}$}‚ नित्यादिगुणोपेतः ॥ प्र‚धान‚न्ताव‚त्सिद्ध‚म्भ‚व‚त्य‚न्ये च ध‚र्मा अन्यैः प्र‚माणैः से‚{\tiny $_{२}$}‚त्स्यंते‚{\tiny $_{lb}$}‚ इति चेदाह । \textbf{त‚थापि} ज्ञेय‚त्वादिना \textbf{किं सिद्धं स्यात्} [।] नैवाभिम‚त‚स्य प्र‚धा‚{\tiny $_{lb}$}‚न‚स्य स्व‚ल‚क्ष‚ण‚स्य सिद्धिः स्यात् । अस्य श‚ब्दार्थ‚रूप‚त्वात् । अग्न्य‚नुमानेपि त‚र्ह्य‚{\tiny $_{lb}$}‚ग्निमात्रं सिद्ध‚मित्य‚साध्यं स्यादित्याह । \textbf{अन्य‚त्र} तु \textbf{त‚देवाग्निसामान्य}निय‚ता‚{\tiny $_{lb}$}‚धार‚म‚सिद्ध\textbf{न्त‚त्र} देशे न \textbf{सिद्ध‚मिति साध्य‚ते} ॥
	{\color{gray}{\rmlatinfont\textsuperscript{§~\theparCount}}}
	\pend% ending standard par
      ‚{\tiny $_{lb}$}‚

	  
	  \pstart \leavevmode% starting standard par
	न‚नु \textbf{त‚त्रापि त‚द‚योग‚विर‚हिणेति} तेन प्र‚देशेनायोग‚स्त‚द‚यो‚{\tiny $_{३}$}‚ग‚स्तेन विर‚हः‚{\tiny $_{lb}$}‚ प्र‚देशेन योग इत्य‚र्थः । सोस्ति य‚स्य सामान्य‚स्य त‚त्त‚थोक्तं । तेना\textbf{न्व‚यो न सिद्धः} ।
	{\color{gray}{\rmlatinfont\textsuperscript{§~\theparCount}}}
	\pend% ending standard par
      ‚{\tiny $_{lb}$}‚

	  
	  \pstart \leavevmode% starting standard par
	\textbf{नेत्या}दिना प‚रिह‚र‚ति । \textbf{न वै क‚श्चि}न्न्याय‚ज्ञः \textbf{त‚थाभूतेन} प्र‚देश‚स‚म्ब‚न्धिना‚{\tiny $_{lb}$}‚ऽग्निसामान्येन व्याप्तिं \textbf{क‚रोति । त‚स्मात् प‚रं} प्र‚तिवादिनं \textbf{प्र‚तिपाद‚य‚ता धूमो‚{\tiny $_{lb}$}‚ग्निनान्त‚रीय‚कोग्न्य‚विनाभावी द‚र्श‚नीयः । य‚त्र धूम‚स्त‚त्राग्नि}रित्येवं । \textbf{स} धूम‚{\tiny $_{lb}$}‚\textbf{स्त‚थे}ति स‚{\tiny $_{४}$}‚र्वोप‚संहार‚व्याप्तिप्र‚द‚र्श‚नेना\textbf{ग्निमात्रेण व्याप्तः सिद्धो य‚त्रैव} प्र‚देशे‚{\tiny $_{lb}$}‚ ‚{\tiny $_{lb}$}‚ \leavevmode\ledsidenote{\textenglish{353/s}}ध‚र्मिणि \textbf{स्व‚यं} स्वेन \textbf{रूपेण दृश्य‚ते त‚त्रैवाग्निबुद्धिञ्ज‚न‚य‚ति । त‚त्रै}त‚स्यां साम‚र्थ्या‚{\tiny $_{lb}$}‚द‚नुमेय‚प्र‚तीतौ \textbf{साध्य‚निर्देशेन न किञ्चित् प्र‚योज‚न‚न्तेन} विनापि साध्य‚सिद्धेः‚{\tiny $_{lb}$}‚ एत‚देवाह । \textbf{त‚त्रे}त्यादि । \textbf{त‚त्र} साध्य‚ध‚र्मिणि लिङ्ग‚स्य \textbf{द‚र्श‚नात् स‚म्ब‚न्धाख्यान‚मात्रा‚{\tiny $_{lb}$}‚च्चेष्ट‚स्य} साध्य‚स्य \textbf{सिद्धेः‚{\tiny $_{५}$}‚} । य‚त‚श्च न साध्य‚स्य ध‚र्म‚ध‚र्मिस‚मुदाय‚स्य निर्देश‚स्त‚दा‚{\tiny $_{lb}$}‚ \textbf{त‚द‚निर्देशे च क‚थ‚न्त‚द्विशिष्टे} साध्य‚ध‚र्मेणा\textbf{न्व‚यः} [।] नैव [।] य‚तोन‚न्व‚य‚दोषः‚{\tiny $_{lb}$}‚ स्यात् । त‚दिति \textbf{त‚स्माद‚य‚न्धूमोग्न्य‚विनाभावित‚या सिद्धः साम‚र्थ्यादेव तेन प्र‚देशे‚{\tiny $_{lb}$}‚नायोग‚म्व्य‚व‚च्छिन‚त्ति} । त‚स्मात् स‚मुदायः साध्य उच्य‚ते ।
	{\color{gray}{\rmlatinfont\textsuperscript{§~\theparCount}}}
	\pend% ending standard par
      ‚{\tiny $_{lb}$}‚

	  
	  \pstart \leavevmode% starting standard par
	य‚दि हि त‚त्र नाग्निः स्यान्नैव धूमो भ‚वेदिति साम‚र्थ्यं । अन्व‚य‚{\tiny $_{६}$}‚स्तु केव‚ले‚{\tiny $_{lb}$}‚नैव साध्य‚ध‚र्मेण द‚र्श‚नीयो न साध्य‚ध‚र्म‚ध‚र्मिस‚मुदायेन । त‚स्य दृष्टान्तेऽसिद्ध‚{\tiny $_{lb}$}‚त्वात् । स‚मुदायेन च व्याप्तिप्र‚द‚र्श‚ने प्र‚योज‚नाभावात् । अत एवाह । \textbf{न पुन‚र}‚{\tiny $_{lb}$}‚स्यास्त‚थेति प्र‚देश‚विशिष्ट\textbf{स्योप‚न्यास‚पूर्व‚कोन्व‚यः} । किं कार‚णं [।] \textbf{साध्योक्ते‚{\tiny $_{lb}$}‚ रिहान्व}य‚प्र‚द‚र्श‚न‚कालेऽ\textbf{न‚ङ्ग‚त्वात्} । साध्य‚निर्देश‚पूर्व‚काले वान्व‚य‚स्येष्य‚माणे नैव‚{\tiny $_{lb}$}‚ सा‚{\tiny $_{७}$}‚ध‚न‚वाक्यात् \textbf{क‚श्चित्} प्र‚तिज्ञावाक्य\textbf{म‚प‚न‚येत्} । \leavevmode\ledsidenote{\textenglish{129a/PSVTa}}
	{\color{gray}{\rmlatinfont\textsuperscript{§~\theparCount}}}
	\pend% ending standard par
      ‚{\tiny $_{lb}$}‚

	  
	  \pstart \leavevmode% starting standard par
	त‚स्मात् स्थित‚मेत‚त् [।] प‚क्ष‚म‚नुप्र‚द‚र्श्यैव साध्य‚ध‚र्मेण लिङ्ग‚स्य व्याप्तिः‚{\tiny $_{lb}$}‚ क‚थ‚नीयेति । त‚था चाह आचार्य दि ग्ना गः । लिंग‚स्य धूमादेः साध्येना\textbf{व्य‚भिचारो‚{\tiny $_{lb}$}‚न्य‚त्र} सामान्ये न ध‚र्मिमात्रे द‚र्श‚यित‚व्यः । \textbf{निश्चिताव्य‚भिचारं च लिङ्ग‚न्त‚त्रे}ति‚{\tiny $_{lb}$}‚ साध्य‚ध‚र्मिणि \textbf{प्र‚सिद्धं स‚त् । तेन व्याप‚क‚ध}र्मेण युक्तं साध्य\textbf{ध‚र्मिणं} ग‚म‚यिष्य‚ति ।
	{\color{gray}{\rmlatinfont\textsuperscript{§~\theparCount}}}
	\pend% ending standard par
      ‚{\tiny $_{lb}$}‚

	  
	  \pstart \leavevmode% starting standard par
	\textbf{त‚स्मा}दित्यादिनोप‚संहारः । \textbf{य‚थाग्निसाध‚न}म्विशेष‚प‚रिग्र‚हाद‚न‚व‚द्यं \textbf{नैव स‚त्ता‚{\tiny $_{lb}$}‚साध‚न‚म}न‚व‚द्यं विशेषासिद्धैः । त‚देवं सिद्ध‚साध्य‚ताप्र‚संगाद‚न‚न्व‚याच्च न प्र धा ना देः‚{\tiny $_{lb}$}‚ स‚त्ता साध्या ॥
	{\color{gray}{\rmlatinfont\textsuperscript{§~\theparCount}}}
	\pend% ending standard par
      ‚{\tiny $_{lb}$}‚‚{\tiny $_{lb}$}‚\textsuperscript{\textenglish{354/s}}

	  
	  \pstart \leavevmode% starting standard par
	साध‚नं पुनः स‚र्व‚म‚त्र‚म‚नुद्दिष्ट‚मिति प्र‚तिपाद‚यितुमाह । \textbf{अप‚रा}मृष्टेत्यादि ।‚{\tiny $_{lb}$}‚ अप‚रामृष्टोनुपात्त‚स्\textbf{त‚द्भेदः} । य‚स्मिन् \textbf{व‚स्तुमात्रे} स‚त्तामात्रे । त‚स्मिन् \textbf{सा‚{\tiny $_{२}$}‚ध‚ने}‚{\tiny $_{lb}$}‚ क्रिय‚माणेन्व‚यो न विह‚न्य‚ते । क‚स्य [।] \textbf{त‚न्मात्र‚व्यापिनः} स‚त्तामात्र‚व्यापिन‚स्\textbf{सा‚{\tiny $_{lb}$}‚ध्य‚स्य} । त‚देव विवृण्व‚न्नाह । \textbf{व‚स्तुमात्र‚व्यापिनि साध्य‚ध‚र्मे । स्व‚भाव‚विशेषा‚{\tiny $_{lb}$}‚प‚रिग्र‚हेण} पुन‚स्स‚त्त्वे क्रिय‚माणे \textbf{नान्व‚य‚व्याघातः} । न साध्य‚शून्यो दृष्टान्त इत्य‚र्थः ।
	{\color{gray}{\rmlatinfont\textsuperscript{§~\theparCount}}}
	\pend% ending standard par
      ‚{\tiny $_{lb}$}‚

	  
	  \pstart \leavevmode% starting standard par
	\textbf{न हि त‚त्रे}ति । स‚त्त्वे \textbf{साध‚ने} । क‚स्मात् [।] \textbf{स‚न्मात्राश्र‚येपि} स‚त्त्व‚मात्र‚स्य‚{\tiny $_{lb}$}‚ हेतुत्वेनाश्र‚य‚णेपीत्य‚र्थः । न पुनः‚{\tiny $_{३}$}‚ साध्य‚त्वे स‚त्ताया विशेषानाश्र‚यः [।] क‚स्माद्‚{\tiny $_{lb}$}‚ [।] \textbf{वैफ‚ल्यात्} \href{http://sarit.indology.info/?cref=pv.3.189-3.190}{१९२-९३}
	{\color{gray}{\rmlatinfont\textsuperscript{§~\theparCount}}}
	\pend% ending standard par
      ‚{\tiny $_{lb}$}‚

	  
	  \pstart \leavevmode% starting standard par
	स‚त्तायां साध्यायां पुन‚र्दोषान्त‚र‚न्दातुमाह । \textbf{अपि चेति । असिद्धे} प्र‚धानादौ‚{\tiny $_{lb}$}‚ ध‚र्मिणि \textbf{न भाव‚ध‚र्मोस्ती}त्य‚सिद्धो हेतुः । य‚स्तु भावाभावो\textbf{भ‚याश्र‚यो} ध‚र्मः स‚{\tiny $_{lb}$}‚ स‚त्त्वे साध्ये \textbf{व्य‚भिचार्य}नैकान्तिको योप्\textbf{य‚भाव}स्य \textbf{ध‚र्मः} स स‚त्त्वे साध्ये \textbf{विरुद्धो} स‚त्त्व‚{\tiny $_{lb}$}‚स्यैव साध‚नात् । य‚स्यां प्र‚धानादिस‚त्तायां साध्यायां हेतु‚{\tiny $_{४}$}‚र्यः क‚श्चिदुपादीय‚ते स‚{\tiny $_{lb}$}‚ स‚र्वो दोष‚त्र‚यं नातिव‚र्त्त‚ते [।] \textbf{सा स‚त्ता साध्य‚ते क‚थं} । इह च हेतोः सिद्ध‚त्व‚म‚भ्यु‚{\tiny $_{lb}$}‚प‚ग‚म्य विरुद्धानैकान्तिकान्तिक‚त्वे उक्ते हेत्व‚सिद्धाव‚न‚योर‚स‚म्भ‚वात् । \href{http://sarit.indology.info/?cref=pv.3.189-3.190}{१९२-९३}
	{\color{gray}{\rmlatinfont\textsuperscript{§~\theparCount}}}
	\pend% ending standard par
      ‚{\tiny $_{lb}$}‚

	  
	  \pstart \leavevmode% starting standard par
	न‚नु स‚र्व‚ज्ञादिस‚त्तायाम‚पि साध्यायां हेतोर‚सिद्ध‚तादिदोष‚स्य तुल्य‚त्वात् क‚थ‚न्त‚{\tiny $_{lb}$}‚त्स‚त्तासिद्धिः ।
	{\color{gray}{\rmlatinfont\textsuperscript{§~\theparCount}}}
	\pend% ending standard par
      ‚{\tiny $_{lb}$}‚

	  
	  \pstart \leavevmode% starting standard par
	नैष दोषः । य‚द्य‚दुप‚दिश्य‚ते त‚ज्ज्ञान‚पूर्व‚क‚मेव य‚थाऽन्य‚त् किञ्चित् । उप‚{\tiny $_{५}$}‚‚{\tiny $_{lb}$}‚दिश्य‚ते च च‚तुरार्य‚स‚त्यं । त‚स्मात्त‚द‚पि ज्ञान‚पूर्व‚क‚मेव य‚स्य त‚ज्ज्ञानं सोस्माभिः स‚र्व‚{\tiny $_{lb}$}‚ज्ञोभ्युप‚ग‚म्य‚त इति । न क‚श\edtext{}{\lemma{श}\Bfootnote{? काचित्}} क्ष‚तिः । \textbf{त‚द्धेतु}रिति स‚त्ताहेतुः । \textbf{त्र‚यी}मिति‚{\tiny $_{lb}$}‚ त्र्य‚व‚य‚वान्दोष‚जातिन्दोष‚प्र‚कारा\textbf{न्नाभिव‚र्त्त‚ते} । तामाह । \textbf{असिद्ध}मित्यादि । \textbf{विरोध}‚{\tiny $_{lb}$}‚‚{\tiny $_{lb}$}‚ \leavevmode\ledsidenote{\textenglish{355/s}}मिति विरुद्ध‚तां । \textbf{त‚त्र भाव‚ध‚र्मो हेतुर‚सिद्ध‚स‚त्ताके क‚थं सिध्येत्} ।
	{\color{gray}{\rmlatinfont\textsuperscript{§~\theparCount}}}
	\pend% ending standard par
      ‚{\tiny $_{lb}$}‚

	  
	  \pstart \leavevmode% starting standard par
	स्यादेत‚द् [।] भाव‚ध‚र्मः प्र‚{\tiny $_{६}$}‚ धा ना देस्सिद्धो न तु भाव इत्याह । \textbf{यो ही}‚{\tiny $_{lb}$}‚त्यादि । \textbf{यो हि} प्र‚धानादि\textbf{भाव‚ध‚र्म्म}हेतु\textbf{न्त‚त्र} प्र‚धानादा\textbf{विच्छ‚ति स क‚थं} वादी प्र‚धा‚{\tiny $_{lb}$}‚नादिक‚म्\textbf{भावं} सिद्धं \textbf{नेच्छेत्} । त‚स्माद् भाव एव ध‚र्मः । क‚थ‚न्त‚र्हि भाव‚स्यायं ध‚र्म‚{\tiny $_{lb}$}‚ इति क‚थ्य‚त इति चेदाह । \textbf{क‚दाचिद}पेक्ष‚येति भेदान्त‚र‚प्र‚तिक्षेप‚ल‚क्ष‚ण‚या \textbf{व्य‚तिरे‚{\tiny $_{lb}$}‚कीव} भिन्न‚रूप इव \textbf{ध‚र्मिणः} स‚काशाद् \textbf{ध‚र्मो निर्दिश्य‚{\tiny $_{७}$}‚ते । य‚था} कृत‚क‚त्व‚म‚स्ये- \leavevmode\ledsidenote{\textenglish{129b/PSVTa}}‚{\tiny $_{lb}$}‚ त्य‚कृत‚क‚व्यावृत्त एव भाव उच्य‚ते । न‚त्व‚न्य एव ध‚र्मो ध‚र्म‚श‚ब्देनोच्य‚तेऽन्य‚श्च‚{\tiny $_{lb}$}‚ ध‚र्मी ध‚र्मिश‚ब्देनेत्याह । \textbf{न ही}त्यादि । य‚स्माद‚न्य‚व्यावृत्तिनिर‚पेक्षः पुमान् य‚दा‚{\tiny $_{lb}$}‚ श‚ब्द‚स्याकृत‚कादेवैक‚स्माद् व्यावृत्तिं जिज्ञास‚ते । त‚दा कृत‚क‚त्व‚म‚स्येत्युच्य‚ते ।‚{\tiny $_{lb}$}‚ य‚दान्य‚व्यावृत्तिसाकांक्षोऽकृत‚क‚त्वादेवैक‚स्माद् व्यावृत्तिं जिज्ञास‚ते त‚दा कृत‚कः‚{\tiny $_{lb}$}‚ श‚ब्द इत्यु‚{\tiny $_{१}$}‚च्य‚त इति । एत‚च्च प्रागेवोक्त‚म् [।] भेदान्त‚र‚प्र‚तिक्षेपाप्र‚तिक्षेपेत्या‚{\tiny $_{lb}$}‚दिना ।
	{\color{gray}{\rmlatinfont\textsuperscript{§~\theparCount}}}
	\pend% ending standard par
      ‚{\tiny $_{lb}$}‚

	  
	  \pstart \leavevmode% starting standard par
	अथ पुन‚रुभ‚योर्भावाभाव‚योर्द्ध‚र्मं हेतुम्बूयात् । क‚थं पुन‚रेको ध‚र्मो भावाभाव‚यो‚{\tiny $_{lb}$}‚र्भ‚व‚ति [।] ध‚र्मो हि स्व‚भावो य‚श्च भाव‚स्य स्व‚भावः क‚थ‚म‚भाव‚स्य स्यादित्याह ।‚{\tiny $_{lb}$}‚ अनाश्रितेत्यादि [।] \textbf{अनाश्रि}त‚म्व‚स्तु य‚स्मिन् व्य‚तिरेक‚मात्रे । त‚स्य व्य‚तिरेक‚{\tiny $_{lb}$}‚मात्र‚स्य \textbf{प्र‚तिषेध‚मात्र‚स्य} ध‚र्म‚त्वेन क‚ल्पि‚{\tiny $_{२}$}‚त‚स्या\textbf{भावेप्य‚विरोधात्} । व्य‚तिरेक‚मात्र‚मेव‚{\tiny $_{lb}$}‚ क‚थ‚म्भ‚व‚तीत्याह । अप‚र्युदासेन प्र‚स‚ज्य‚प्र‚तिषेधेन । प्र‚स‚ज्योप‚स‚र्ज‚नो विधिः \textbf{प‚र्युदासः}‚{\tiny $_{lb}$}‚ स चेह नाश्रितः । \textbf{य‚था न भ‚व‚ति मूर्त्त इत्य‚मूर्त्त‚त्वं} मूर्त्त‚त्व‚निवृत्तिमात्रं भावेपि‚{\tiny $_{lb}$}‚ विज्ञाने \textbf{निरूपाख्ये}प्य‚भावे\textbf{पि स्यात्} ॥
	{\color{gray}{\rmlatinfont\textsuperscript{§~\theparCount}}}
	\pend% ending standard par
      ‚{\tiny $_{lb}$}‚

	  
	  \pstart \leavevmode% starting standard par
	प‚रः प्र‚तिब‚द्धुमाह । \textbf{निरुपाख्याभावान्न प्र‚तिषेध‚विष‚य‚त्वं} । य‚द‚धिक‚र‚णादि‚{\tiny $_{lb}$}‚श‚{\tiny $_{३}$}‚क्तियुक्त‚न्त‚त्क‚स्य‚चिद्विष‚यः स्यात् । निरुपाख्यं च स‚र्व‚श‚क्तिर‚हित‚न्त‚त्क‚थ‚म्विष‚यः‚{\tiny $_{lb}$}‚ स्यात् । \textbf{संप्र‚ति} प्र‚तिषेध‚विष‚य‚त्वे प्र‚तिषिद्धे \textbf{किम्विधिविष‚योस्ति} निरुपाख्यं । \textbf{त‚द‚पि}‚{\tiny $_{lb}$}‚ ‚{\tiny $_{lb}$}‚ \leavevmode\ledsidenote{\textenglish{356/s}}विधिविष‚य‚त्व‚न्निरुपाख्य‚स्य \textbf{नेति चेत् । क‚थ‚मि}दानीम‚भावो \textbf{न प्र‚तिषेध‚विष‚यः} । विधि‚{\tiny $_{lb}$}‚विष‚य‚त्व‚निषेधादेव हि प्र‚तिषेध‚विष‚य‚त्वं । किं कार‚ण‚म् [।] \textbf{विधिनिवृत्तिरूप‚त्वात्‚{\tiny $_{lb}$}‚ प्र‚तिषे‚{\tiny $_{४}$}‚ध‚स्य । त‚दि}ति त‚स्मादेत‚द‚न‚न्त‚रोक्त‚म\textbf{व्य‚व‚च्छेद‚मात्रं द्व‚योर‚पि} भावाभाव‚योः‚{\tiny $_{lb}$}‚ स‚म्भ‚व‚त् स‚त्त्वे साध्ये ग‚म‚क‚त्वं क‚थ‚मात्म‚सात् कुर्यात् । किम्बिशिष्टं ग‚म‚क‚त्वं ।‚{\tiny $_{lb}$}‚ \textbf{विप‚क्षे}त्यादि । \textbf{विप‚क्षे} प्र‚योगे \textbf{वृत्ति}र्हेतोस्त‚स्य \textbf{श‚ङ्का} त‚स्या अपि \textbf{व्य‚व‚च्छेदेन न ल‚भ्य‚न्न}‚{\tiny $_{lb}$}‚ चोभ‚य‚ध‚र्म‚स्य व्य‚व‚च्छेद‚मात्र‚स्य विप‚क्षाद् व्यावृत्तिर‚स्तीति \textbf{क‚थ}न्त‚द् ग‚म‚क‚त्व‚मा\textbf{त्म‚{\tiny $_{lb}$}‚सात् कुर्यात् । स च} वा‚{\tiny $_{५}$}‚दी \textbf{स्व‚वाचा}न्य‚व‚च‚नेन । स‚त्तासाध‚न‚स्य हेतो\textbf{रुभ‚य‚ध‚र्म‚तां‚{\tiny $_{lb}$}‚ ब्रुवाण‚स्स‚तः} साध्या\textbf{द‚न्य‚त्राप्य}स‚ति अस्योभ‚य‚ध‚र्म‚स्य हेतो\textbf{र्वृत्तिम्}भाष‚ते [।] स एव‚{\tiny $_{lb}$}‚ च \textbf{स‚त्तायां} साध्यायाम\textbf{व्य‚भिचार‚निब‚न्ध‚न‚त्वाद्} ग‚म‚क‚त्व‚स्येत्य‚व्य‚भिचार‚म्भा\edtext{}{\lemma{म्भा}\Bfootnote{?}}एत‚{\tiny $_{lb}$}‚ इति हेतोः \textbf{क‚थं नो}न्म‚त्तः ॥
	{\color{gray}{\rmlatinfont\textsuperscript{§~\theparCount}}}
	\pend% ending standard par
      ‚{\tiny $_{lb}$}‚

	  
	  \pstart \leavevmode% starting standard par
	\textbf{अभाव‚ध‚र्म‚न्तु} हेतुं \textbf{स‚त्ताया}म्व‚द‚तोस्य वादिनो विरुद्धः स्यात् । स‚त्त्व‚विप‚रीत‚{\tiny $_{lb}$}‚स्यास‚त्त्व‚स्य साध‚नात् ।‚{\tiny $_{६}$}‚ कः पुन‚र‚स्यैव अभाव‚स्यैव ध‚र्म इत्याह । व्य‚व‚च्छेदं‚{\tiny $_{lb}$}‚ कीदृश‚म्\textbf{भाव‚मात्र‚व्यापी साम‚र्थ्य}ल‚क्ष‚ण‚स्त‚स्य \textbf{व्य‚व‚च्छेदो} निय‚मेनाभाव‚स्यैव‚{\tiny $_{lb}$}‚ भ‚व‚ति [।] न तु मूर्त्त‚त्वादेर्व्य‚व‚च्छेद‚स्त‚स्यैवोभ‚य‚ध‚र्म‚त्वात् । क‚स्माद्विरुद्ध इत्याह ।‚{\tiny $_{lb}$}‚ \textbf{त‚स्य} भाव‚मात्र‚व्याप्य‚र्थ‚व्य‚व‚च्छेद‚स्य \textbf{भावे क्व‚चिद‚भावाद‚भावे} च स‚र्व‚त्र भावाद्‚{\tiny $_{lb}$}‚ \leavevmode\ledsidenote{\textenglish{130a/PSVTa}} विरुद्ध‚त्वं ॥ \textbf{त‚स्माद‚यं त्रिप्र‚कारोपि} भावाभावोभ‚य‚स‚म्ब‚न्धी ध‚र्मः \textbf{स‚त्ता‚{\tiny $_{७}$}‚यास्साध्र‚नेन‚{\tiny $_{lb}$}‚ हेतुल‚क्ष‚ण‚मुक्तः} । न च त्रिप्र‚काराद‚ध‚र्मा\textbf{द‚न्या ग‚ति}र‚न्यः प्र‚कारोस्ति य‚त\textbf{स्त‚स्मान्न‚{\tiny $_{lb}$}‚ स‚त्ता साध्य‚ते} । साध‚न‚त्वे लिङ्ग‚त्वेऽस्यास्स‚त्तायाः \textbf{सामान्ये}नानुपात्त‚विशेष‚ण‚त्वेन ।‚{\tiny $_{lb}$}‚ \textbf{सिद्ध‚स‚त्ताके ध‚र्मिणि नासिद्धिः} । अनित्य‚त्वादिके \textbf{व‚स्तुध‚र्मे}साध्ये । किम्विशिष्टे‚{\tiny $_{lb}$}‚ \textbf{त‚न्मात्र‚व्यापिनि । तेन च साध्य‚ध‚र्मेण} लिंग‚स्य \textbf{व्याप्तिः । क‚थंचिदि}त्य‚न्व‚य‚{\tiny $_{१}$}‚मुखेन‚{\tiny $_{lb}$}‚ ‚{\tiny $_{lb}$}‚ \leavevmode\ledsidenote{\textenglish{357/s}}व्य‚तिरेक‚मुखेण\edtext{}{\lemma{मुखेण}\Bfootnote{? न}} \textbf{वा य‚दि निश्चीय‚ते} । न विरोध‚व्य‚भिचारौ । न विरुद्ध‚त्व‚{\tiny $_{lb}$}‚म\textbf{नैकान्तिक‚त्व‚म्वा} [।] इति हेतो\textbf{र्नायं प्र‚संगो}सिद्धि\edtext{}{\lemma{सिद्धि}\Bfootnote{? द्ध}} विरुद्धानैकान्तिक‚ल‚क्ष‚णः ।‚{\tiny $_{lb}$}‚ अनिश्चितायान्तु साध्य‚ध‚र्मेण लिङ्ग‚स्या\textbf{व्याप्तौ । ध‚र्मिस‚माश्र‚ये वा} प‚क्ष‚ध‚र्म‚त्वे‚{\tiny $_{lb}$}‚ \textbf{वाऽनिश्चिते स‚ति त‚त्स्व‚भाव‚त‚या} निश्चित‚त्रैरूप्य‚स्व‚भाव‚त‚या यो \textbf{ग‚म‚को}ऽभिम‚तो‚{\tiny $_{lb}$}‚ \textbf{हेतुर्न स क‚श्चिद् ग‚म‚कः ॥ अत‚{\tiny $_{२}$}‚ एव} कार‚णात् \textbf{स्व‚ध‚र्मेण} स्वेन साध्य‚रूपेण \textbf{व्याप्तः}‚{\tiny $_{lb}$}‚ साध्य‚ध‚र्मिणि \textbf{सिद्धो} निश्चितो हेतु\textbf{स्व‚भावो वाच्यः} ।
	{\color{gray}{\rmlatinfont\textsuperscript{§~\theparCount}}}
	\pend% ending standard par
      ‚{\tiny $_{lb}$}‚

	  
	  \pstart \leavevmode% starting standard par
	एत‚देव द्र‚ढ‚य‚न्नाह । \textbf{न ही}त्यादि । \textbf{प्र‚काश‚त‚या} प्र‚भास्व‚र‚त‚या घ‚टादीन्प्र‚काश‚य‚न्‚{\tiny $_{lb}$}‚ य‚दा क‚दाचिद् घ‚टाद्युद‚रान्त‚र्व‚र्त्ती भ‚व‚ति त‚दा \textbf{त‚द्रूपाप्र‚तिप‚त्तौ} प्र‚भास्व‚र‚ताऽप्र‚तिप‚त्तौ‚{\tiny $_{lb}$}‚ स‚त्यां \textbf{स्वाम‚र्थ‚क्रियां} घ‚टादिप्र‚काश‚न‚ल‚क्ष‚णां \textbf{न हि क‚रोति} ॥
	{\color{gray}{\rmlatinfont\textsuperscript{§~\theparCount}}}
	\pend% ending standard par
      ‚{\tiny $_{lb}$}‚

	  
	  \pstart \leavevmode% starting standard par
	ग‚{\tiny $_{३}$}‚म‚क‚स्व‚रूप‚म‚भिधायाधुना ग‚म्य‚स्व‚रूप‚माह । \textbf{व्याप‚क‚स्त‚स्ये}ति । योसौ ग‚म‚को‚{\tiny $_{lb}$}‚ व्याप्यः स्व‚भाव‚स्त‚स्य \textbf{व्याप‚कः} स्व‚भावः निश्चितो ग‚म्यः ।
	{\color{gray}{\rmlatinfont\textsuperscript{§~\theparCount}}}
	\pend% ending standard par
      ‚{\tiny $_{lb}$}‚

	  
	  \pstart \leavevmode% starting standard par
	\textbf{त‚द्ध‚र्मे}त्यादिना व्याच‚ष्टे । ध‚र्मिणो \textbf{ध‚र्मो} ग‚म्यः [।] कीदृश‚स्त‚स्य ग‚म‚क‚त्वे‚{\tiny $_{lb}$}‚नाभिम‚त‚स्य \textbf{व्याप‚क‚त्वेन निश्चितः} । क‚थं पुन‚र्व्याप‚क‚त्वेन निश्चित इत्याह ।‚{\tiny $_{lb}$}‚ \textbf{त‚द्ध‚र्म‚निश्च‚यादेव} व्याप्य‚ध‚र्म‚निश्च‚यादेव ।
	{\color{gray}{\rmlatinfont\textsuperscript{§~\theparCount}}}
	\pend% ending standard par
      ‚{\tiny $_{lb}$}‚

	  
	  \pstart \leavevmode% starting standard par
	इय‚ता ग‚म्य‚{\tiny $_{४}$}‚ग‚म‚क‚योः स्व‚रूप‚न्द‚र्शितं ॥
	{\color{gray}{\rmlatinfont\textsuperscript{§~\theparCount}}}
	\pend% ending standard par
      ‚{\tiny $_{lb}$}‚

	  
	  \pstart \leavevmode% starting standard par
	निव‚र्त्त्य‚निव‚र्त्त‚क‚योर‚पि स्व‚रूप‚माह । \textbf{त‚स्ये}त्यादि । अय‚मिति व्याप‚को ध‚र्मः‚{\tiny $_{lb}$}‚ स्व‚यं \textbf{निवृत्तौ} स‚त्यां \textbf{त‚स्य} व्याप्य‚स्य \textbf{निव‚र्त्त‚कः} ।
	{\color{gray}{\rmlatinfont\textsuperscript{§~\theparCount}}}
	\pend% ending standard par
      ‚{\tiny $_{lb}$}‚

	  
	  \pstart \leavevmode% starting standard par
	\textbf{त‚स्ये}त्यादिना व्याच‚ष्टे । \textbf{अयं व्याप‚को ध‚र्मः स्व‚य‚न्निव‚र्त्त‚मान}स्त‚स्य \textbf{व्याप्य‚स्य‚{\tiny $_{lb}$}‚ निव‚र्त्त‚क} इति स‚म्ब‚न्धः । किङ्कार‚णं । य‚स्मा\textbf{देवं ह्य}स्या\textbf{यं} साध्यो \textbf{ध‚र्मो व्याप‚कः‚{\tiny $_{lb}$}‚ सिद्धो भ‚व‚ति । य‚द्य‚स्य} व्याप‚क‚स्याभा‚{\tiny $_{५}$}‚वे व्याप्यो \textbf{न भ‚वेत्} ॥ त‚दिति वाक्योप‚न्यासे ।
	{\color{gray}{\rmlatinfont\textsuperscript{§~\theparCount}}}
	\pend% ending standard par
      ‚{\tiny $_{lb}$}‚‚{\tiny $_{lb}$}‚\textsuperscript{\textenglish{358/s}}

	  
	  \pstart \leavevmode% starting standard par
	अनेनान‚न्त‚रोक्तेनानुव‚र्त्त्यानुव‚र्त्त‚क‚भाव‚स्य निव‚र्त्त्य‚निव‚र्त्त‚क‚भाव‚स्य च प्र‚द‚र्श‚नेन‚{\tiny $_{lb}$}‚ \textbf{द्विविध‚स्यापि} साध‚र्म्य‚व‚तो वैध‚र्म्य‚व‚त‚श्च \textbf{साध‚न‚प्र‚योग‚स्य ग‚म‚क‚ल‚क्ष‚णं} साध्य‚साध‚क‚त्व‚{\tiny $_{lb}$}‚ल‚क्ष‚ण‚मुक्त‚म्वेदित‚व्यं ।
	{\color{gray}{\rmlatinfont\textsuperscript{§~\theparCount}}}
	\pend% ending standard par
      ‚{\tiny $_{lb}$}‚

	  
	  \pstart \leavevmode% starting standard par
	त‚द्व्याच‚ष्टे । \textbf{द्विविधो} हीत्यादि । \textbf{य‚थाहुरेके} इति नै या यि काः । \textbf{साध‚र्म्य‚वानेव}‚{\tiny $_{lb}$}‚ हि प्र‚योगोन्व‚यी‚{\tiny $_{६}$}‚ । \textbf{वैध‚र्म्य‚वानेव} च व्य‚तिरेकी ।
	{\color{gray}{\rmlatinfont\textsuperscript{§~\theparCount}}}
	\pend% ending standard par
      ‚{\tiny $_{lb}$}‚

	  
	  \pstart \leavevmode% starting standard par
	न‚नु साध‚र्म्य‚प्र‚योगे प‚क्ष‚ध‚र्म‚त्व‚म‚न्व‚य‚श्चेति [।] त‚था वैध‚र्म्य‚प्र‚योगेपि प‚क्ष‚ध‚र्म‚त्वं‚{\tiny $_{lb}$}‚ व्य‚तिरेक‚श्चेति द्विरूप‚न्त‚र्हि लिङ्ग‚म्प्राप्त‚मित्याह । \textbf{नान‚यो}रित्यादि । अन‚योरित्य‚न्व‚य‚{\tiny $_{lb}$}‚व्य‚तिरेकिणोर्हेत्वोर्व‚स्तुतः \textbf{प‚र‚मार्थ‚तो न क‚श्चिद् भेदः} । द्व‚योर‚प्य\textbf{न्व‚य‚व्य‚तिरेक‚व‚त्त्वात्} ।‚{\tiny $_{lb}$}‚ \leavevmode\ledsidenote{\textenglish{130b/PSVTa}} अन्य‚त्र संयोग‚भेदात्‚{\tiny $_{७}$}‚ । त‚स्मात्तावेवान्व‚य‚व्य‚तिरेकौ क‚दाचित्साध‚र्म्य‚प्र‚योगेण‚{\tiny $_{lb}$}‚ प्र‚तिपाद्येते क‚दाचिद् वैध‚र्म्य‚प्र‚योगेणेति प्र‚योग‚मात्र‚म्भिद्य‚ते न त्व‚र्थः । किं कार‚णं [।]‚{\tiny $_{lb}$}‚ य‚स्मात् \textbf{साध‚र्म्येणापि हि प्र‚योगेऽर्थात्} साम‚र्थ्यात् । साध्य‚विप‚क्षाद्धेतोर्व्यावृत्तिर्वै‚{\tiny $_{lb}$}‚ध‚र्म्य‚न्त‚स्य ग‚तिः ।
	{\color{gray}{\rmlatinfont\textsuperscript{§~\theparCount}}}
	\pend% ending standard par
      ‚{\tiny $_{lb}$}‚

	  
	  \pstart \leavevmode% starting standard par
	त‚देव साम‚र्थ्य‚माह । \textbf{अस‚ती}त्यादि । \textbf{त‚स्मिन्निति} वैध‚र्म्ये । एवं हि साध्ये‚{\tiny $_{lb}$}‚नान्वितो \textbf{हे‚{\tiny $_{१}$}‚तुः} स्याद् य‚दि \textbf{साध्याभावे न भ‚वेत्} । त‚था वैध‚र्म्य इति वैध‚र्म्य‚प्र‚योगे ।‚{\tiny $_{lb}$}‚ त‚स्मिन्नित्य‚न्व‚ये य‚दि हि साध्येन हेतोर‚न्व‚यः स्यात् त‚दायं साध्य‚निवृत्तौ निव‚र्त्तेत ।‚{\tiny $_{lb}$}‚ \textbf{एत‚च्च} व्य ति रे क चि न्ता \textbf{याम्व‚क्ष्यामः} ॥ \href{http://sarit.indology.info/?cref=pv.3.191-3.192}{१९४-९५}
	{\color{gray}{\rmlatinfont\textsuperscript{§~\theparCount}}}
	\pend% ending standard par
      ‚{\tiny $_{lb}$}‚

	  
	  \pstart \leavevmode% starting standard par
	\textbf{अनित्य‚त्वे य‚था कार्यं । अनित्य एव कृत‚क‚त्वं} । एत‚च्चान्व‚यिन उदाह‚र‚णं ।‚{\tiny $_{lb}$}‚ अकार्य‚म्वा । अविनाशिनीति व्य‚तिरेकिण उदाह‚र‚णं । तेनाय‚म‚र्थो भ‚व‚ति [।]‚{\tiny $_{lb}$}‚ \textbf{अविनाशिनि‚{\tiny $_{२}$}‚} विनाशाभावे स‚ति । अकार्यं कृत‚क‚त्व‚न्न भ‚व‚ति । त‚देवाह [।]‚{\tiny $_{lb}$}‚ \textbf{अनेने}त्यादि । \textbf{अन‚योरि}त्य‚न्व‚य‚व्य‚तिरेकिणोः । य‚त्किञ्चिदिति स‚र्वोप‚संहारेण‚{\tiny $_{lb}$}‚ ‚{\tiny $_{lb}$}‚ \leavevmode\ledsidenote{\textenglish{359/s}}व्याप्तिक‚थ‚नेनार्थान्नित्याद् व्यावृत्तिः कृत‚क‚त्व‚स्यो\textbf{क्तेति} व्य‚तिरेक‚म‚तिः । \textbf{श‚ब्द‚श्च‚{\tiny $_{lb}$}‚ कृत‚क} इति प‚क्ष‚ध‚र्म‚क‚थ‚नं । प‚क्षः क‚स्मान्नोच्य‚त इत्याह । \textbf{कृत‚क}स्येत्यादि । \textbf{अनि‚{\tiny $_{lb}$}‚त्य‚त्वेन व्याप्तं कृत‚क‚त्वं} य‚दा श‚ब्देन \textbf{क‚थि}त‚न्त‚दा‚{\tiny $_{३}$}‚ निय‚मेन स्वं व्याप‚कं स‚न्निधाप‚{\tiny $_{lb}$}‚य‚तीति \textbf{साम‚र्थ्यादेवानित्यः श‚ब्द} इति भ‚व‚ति । \textbf{त‚स्मान्नाव‚श्य‚मि}त्यादि । इहेति‚{\tiny $_{lb}$}‚ साध‚र्म्य‚प्र‚योगे [।]
	{\color{gray}{\rmlatinfont\textsuperscript{§~\theparCount}}}
	\pend% ending standard par
      ‚{\tiny $_{lb}$}‚

	  
	  \pstart \leavevmode% starting standard par
	वैध‚र्म्य‚व‚न्तं प्र‚योग‚माह । \textbf{व्य‚तिरेके}पीत्यादि । एत‚च्चाकार्य‚म्वा ऽविनाशी‚{\tiny $_{lb}$}‚त्येत‚स्य विव‚र‚णं । इहापि न प्र‚तिज्ञाव‚च‚नं । य‚स्मात् सिद्ध‚स्व‚भाव‚त‚या निश्चित‚या‚{\tiny $_{lb}$}‚ नित्य‚स्व‚भाव‚त‚या हेतुभूत‚या । \textbf{त‚द‚भाव} इत्य‚नित्य‚त्वाभावे स‚ति \textbf{न भ‚{\tiny $_{४}$}‚व‚तः‚{\tiny $_{lb}$}‚ कृत‚क‚त्व‚स्य श‚ब्दे च भाव‚ख्यातौ} स‚द्भाव‚क‚थ‚ने कृते स‚ति \textbf{त‚दात्म‚नः} स‚त‚{\tiny $_{lb}$}‚ इत्य‚नित्य‚स्व‚भाव‚स्य स‚तः कृत‚क‚त्व‚स्य श‚ब्दे \textbf{भाव} इति । \textbf{साम‚र्थ्याद}नित्यः श‚ब्द‚{\tiny $_{lb}$}‚ इति सिद्धेः । \textbf{पूर्व‚व‚दि}ति साध‚र्म्य‚प्र‚योग‚व‚त् ।
	{\color{gray}{\rmlatinfont\textsuperscript{§~\theparCount}}}
	\pend% ending standard par
      ‚{\tiny $_{lb}$}‚

	  
	  \pstart \leavevmode% starting standard par
	न‚न्व‚त्र वैध‚र्म्य‚प्र‚योगेऽन्व‚यो नोक्त इत्याह [।] अन्व‚य‚स्त्वित्यादि । \textbf{अन्व‚य}‚{\tiny $_{lb}$}‚म‚न्त‚रेण वैध‚र्म्य‚स्यानुप‚प‚त्तिर\textbf{र्थाप‚त्तिः} । किङ्कार‚णं [।] न हीत्यादि । य‚{\tiny $_{५}$}‚स्मा‚{\tiny $_{lb}$}‚\textbf{द‚त‚दात्म‚निय‚त‚स्या}नित्य‚स्व‚भावेऽप्र‚तिब‚द्ध‚स्य । \textbf{त‚न्निवृत्ता}व‚नित्य‚त्व‚निवृत्तौ \textbf{निवृत्ति}‚{\tiny $_{lb}$}‚र्युक्ता । य‚त एव‚न्त‚स्मात् स्व‚य‚न्तादात्म्य‚त‚दुत्प‚त्तिभ्यां हेतोः साध्ये \textbf{निय‚मं} प्र‚माणेन‚{\tiny $_{lb}$}‚ \textbf{प्र‚साध्य} साध्य‚निवृत्त्या मूढं प्र‚ति हेतो\textbf{र्निवृत्तिर्व‚क्त‚व्या} ।
	{\color{gray}{\rmlatinfont\textsuperscript{§~\theparCount}}}
	\pend% ending standard par
      ‚{\tiny $_{lb}$}‚

	  
	  \pstart \leavevmode% starting standard par
	तेन य‚दुच्य‚ते [।] प्र‚माणेन चेन्निय‚मः प्र‚साधितः किन्निष्फ‚लेन निवृत्तिव‚च‚{\tiny $_{lb}$}‚नेन । क‚थं वा निय‚मं न प्र‚तिप‚द्य‚ते ।‚{\tiny $_{६}$}‚ न तु त‚न्निवृत्तौ निवृत्तिमिति त‚द‚पास्तं ॥
	{\color{gray}{\rmlatinfont\textsuperscript{§~\theparCount}}}
	\pend% ending standard par
      ‚{\tiny $_{lb}$}‚

	  
	  \pstart \leavevmode% starting standard par
	अन्ये त्व‚न्य‚था व्याच‚क्ष‚ते । प्र‚साध्य श‚ब्दार्थादाक्षेप‚व‚च‚न‚स्तेनाय‚म‚र्थो निय‚मं‚{\tiny $_{lb}$}‚ प्र‚साध्य निय‚म‚म‚र्थादाक्षिप्य निवृत्तिर्व‚क्त‚व्येति । त‚थाभूतेन व‚च‚नेन निवृत्तिर्व‚क्त‚व्या ।‚{\tiny $_{lb}$}‚ निवृत्त्युक्तिः साम‚र्थ्यान्निय‚म‚माक्षिप‚तीति । अत एवाह । \textbf{सा चे}त्यादि । \textbf{से}ति निय‚म‚{\tiny $_{lb}$}‚स्याक्षेपिका निवृत्तिः सिध्य‚ति प‚र‚म्प्र‚ति । य‚दि त‚था‚{\tiny $_{७}$}‚ भूतेन व‚च‚नेन प्र‚काश‚ते । \textbf{त‚दा- \leavevmode\ledsidenote{\textenglish{131a/PSVTa}}‚{\tiny $_{lb}$}‚ ‚{\tiny $_{lb}$}‚ \leavevmode\ledsidenote{\textenglish{360/s}}त्म‚निय‚मं} साध्यात्म‚निय‚म\textbf{म‚र्था}दुक्तिसाम‚र्थ्या\textbf{दाक्षिप‚ति । इति} हेतोः \textbf{सिद्धोन्व‚यः} ॥
	{\color{gray}{\rmlatinfont\textsuperscript{§~\theparCount}}}
	\pend% ending standard par
      ‚{\tiny $_{lb}$}‚

	  
	  \pstart \leavevmode% starting standard par
	\textbf{क‚थ‚मि}त्यादि प‚रः । \textbf{इदानी}मिति निश्चिते व्याप्य‚व्याप‚क‚भावे ग‚म‚को हेतु‚{\tiny $_{lb}$}‚रित्य‚भ्युप‚ग‚मे स‚तीत्य‚र्थः । क‚थं केन प्र‚माणेन \textbf{कृत‚कोव‚श्य‚म‚नित्य इति} प्र‚त्येत‚व्यो‚{\tiny $_{lb}$}‚ निश्चेत‚व्यो \textbf{येन} त्व‚यैव‚मुच्य‚ते । कृत‚कः श‚ब्दोऽनित्यः । य‚त्कृत‚क‚न्त‚द‚नित्य‚{\tiny $_{१}$}‚मित्येवं‚{\tiny $_{lb}$}‚ पृष्टो व्याप्तिविष‚यं बाध‚कं प्र‚माण‚न्द‚र्श‚यितुमाह । \textbf{य‚स्मादि}त्यादि । य‚स्माद्‚{\tiny $_{lb}$}‚ विनाश‚स्य निवृत्तिध‚र्म‚क‚त्व‚ल‚क्ष‚ण‚स्य स्व‚भावात् स्व‚रूप‚मात्राद‚नुब‚न्धिता । य‚द्वा‚{\tiny $_{lb}$}‚ भ‚व‚त्य‚स्मादिति भावः [।] स्वो भावः स्व‚भावः स्व‚हेतुरित्य‚र्थः । त‚स्मादेवानु‚{\tiny $_{lb}$}‚ब‚न्धिता विनाश‚स्य व‚स्तुनि स‚द्भाव‚स्त‚स्मात् कृत‚कोऽनित्यः । कुत एव त‚द\textbf{हेतुत्वाद्}‚{\tiny $_{lb}$}‚ य‚तो न ज‚न‚काद्धेतोर‚न्यो \textbf{विनाश‚{\tiny $_{२}$}‚स्य} हेतुः [।] त‚स्मात् \textbf{स्व‚भावाद‚न‚ब‚न्धः} ।
	{\color{gray}{\rmlatinfont\textsuperscript{§~\theparCount}}}
	\pend% ending standard par
      ‚{\tiny $_{lb}$}‚

	  
	  \pstart \leavevmode% starting standard par
	त‚द‚य‚म‚त्र स‚मुदायार्थः । मुद्ग‚र‚व्यापारान‚न्त‚रं द्व‚यं प्र‚तीय‚ते घ‚ट‚निवृत्तिः ।‚{\tiny $_{lb}$}‚ क‚पालं च [।] त‚थैते विनाश‚रूप‚त‚या प्र‚तीयेते । त‚त्र घ‚ट‚निवृत्तेर्नीरूप‚त्वेनाकार्य‚{\tiny $_{lb}$}‚त्वादिति व‚क्ष्य‚ति । त‚त्कार्य‚त्वेन तु प्र‚तीतिभ्रान्तिरेव । कार्य‚त्वे वास्या न घ‚ट‚{\tiny $_{lb}$}‚निवृत्तिरूप‚त्वं स्यात् । घ‚ट‚स‚म्ब‚न्धित्वेन कृत‚क‚त्वात् । विनाश‚रूप‚{\tiny $_{३}$}‚त‚या च न‚{\tiny $_{lb}$}‚ प्र‚तीतिः स्यात् घ‚ट‚स्य स‚त्त्वात् । क‚पाल‚स्यापि मुद्ग‚र‚कार्य‚त्वे स‚त्य‚पि विनाश‚{\tiny $_{lb}$}‚रूप‚ता । घ‚ट‚स्यानिवृत्त‚त्वादिति च व‚क्ष्य‚ति । त‚त‚श्च क‚थ‚म‚स्य विनाश‚रूप‚त‚या‚{\tiny $_{lb}$}‚ प्र‚तीतिः । निर्हेतुके तु विनाशे स्व‚र‚स‚तो निव‚र्त्त‚मान एव घ‚टो मुद्ग‚रादिस‚ह‚कारी‚{\tiny $_{lb}$}‚ क‚पाल‚ज‚न‚क‚त्वेन स‚दृश‚क्ष‚णानार‚म्भ‚क‚त्वात् मुद्ग‚र‚व्यापारान‚न्त‚रं घ‚ट‚निवृत्तेः क‚पा‚{\tiny $_{lb}$}‚ल‚स्य‚{\tiny $_{४}$}‚ च स‚द्भावात्त‚योर्विनाश‚रूप‚त‚या विनाश‚स्य च स‚हेतुक‚त्वेन म‚न्द‚म‚तीना‚{\tiny $_{lb}$}‚म‚व‚सायो युज्य‚त एव ।
	{\color{gray}{\rmlatinfont\textsuperscript{§~\theparCount}}}
	\pend% ending standard par
      ‚{\tiny $_{lb}$}‚

	  
	  \pstart \leavevmode% starting standard par
	मुद्ग‚र‚व्यापारान‚न्त‚रं स‚न्तान‚विच्छेदात् । त‚त्क‚थं \textbf{निर्हेतुक}विनाशाभ्युप‚ग‚म‚वा‚{\tiny $_{lb}$}‚दिनां प्र‚तीतिबाधा चोद्य‚त इति । एत‚मेवार्थ‚माह । \textbf{न ही}त्यादि । न‚श्य‚न्त इति‚{\tiny $_{lb}$}‚ येषान्ताव‚त् कृत‚कानां नाशो दृश्य‚ते ते विन‚श्य‚न्तः । \textbf{त‚द्भावे विन‚श्व‚र‚स्व‚भावे}‚{\tiny $_{lb}$}‚ स्व‚रूप‚{\tiny $_{५}$}‚ज‚न‚काद‚न्य‚त्र \textbf{हेतुम‚पेक्ष‚ते} । कुतः । \textbf{स्व‚हेतोरेव विन‚श्व‚राणां} निवृत्तिध‚र्माणां‚{\tiny $_{lb}$}‚ \textbf{भावात्} । य‚त‚श्च न‚श्व‚र‚स्व‚भावं प्र‚त्य‚न‚पेक्ष‚ता भावाना\textbf{न्त‚स्माद् यः क‚श्चिद् कृत‚कः‚{\tiny $_{lb}$}‚ स स्व‚भावेनैव न‚श्व‚रः} । \href{http://sarit.indology.info/?cref=pv.3.192-3.193}{१९५-९६}
	{\color{gray}{\rmlatinfont\textsuperscript{§~\theparCount}}}
	\pend% ending standard par
      ‚{\tiny $_{lb}$}‚‚{\tiny $_{lb}$}‚\textsuperscript{\textenglish{361/s}}

	  
	  \pstart \leavevmode% starting standard par
	य‚दि स‚हेतुको विनाश‚स्त‚दाव‚श्यंभावी न स्यादित्याह । सापेक्षाणामित्यादि ।‚{\tiny $_{lb}$}‚ हीति य‚स्मात् [।] सापेक्षाणाम्भावानां नाव‚श्य‚म्भाविता [।] त‚स्मात् \textbf{निर‚पेक्षो}‚{\tiny $_{lb}$}‚भावो वि‚{\tiny $_{६}$}‚नाशे विन‚श्व‚रे स्व‚भावो हेतुसापेक्ष‚त्वे हि घ‚टादीनाम्म‚ध्ये \textbf{केषाञ्चिन्नि‚{\tiny $_{lb}$}‚त्य‚तापि स्यात्} । येषां नाश‚कार‚ण‚म‚स‚न्निहितं । वाहुल्यात् विनाश‚कार‚णानां न‚{\tiny $_{lb}$}‚ क्व‚चिद‚स‚न्निधान‚मिति चेदाह । \textbf{येने}त्यादि । \textbf{त‚द्धेतोरिति विनाश‚हेतोस्तेषा‚{\tiny $_{lb}$}‚म‚पि} विनाश‚कार‚णानां \textbf{नाव‚श्यं स‚न्निधान}मिति स‚म्ब‚न्धः । क‚स्मात् [।] \textbf{स्व‚प्र‚त्य}‚{\tiny $_{lb}$}‚याधीन‚स‚न्निधित्वात् स्व‚कार‚णाय‚{\tiny $_{७}$}‚त‚स‚न्निधित्वात् । न च विनाश‚कार‚णान्तं \leavevmode\ledsidenote{\textenglish{131b/PSVTa}}‚{\tiny $_{lb}$}‚ कार‚णानि स‚र्व‚त्र स‚न्निहितानि । त‚त‚श्च विनाश‚हेतोर‚म‚न्निधानात् \textbf{क‚श्चिन्न न‚श्ये‚{\tiny $_{lb}$}‚द‚पि} । स‚त्य‚पि विनाश‚हेतुस‚न्निधानं न निय‚तो विनाशः [।] य‚तो न \textbf{ह्य‚व‚श्यं‚{\tiny $_{lb}$}‚ हेत‚वः फ‚ल‚व‚न्तः} विनाशाख्य‚कार्य‚व‚न्तः । क‚स्मात् [।] स‚ह‚कार्य‚स‚न्निधानं \textbf{वैक‚ल्यं} ।‚{\tiny $_{lb}$}‚ विरुद्धोप‚निपातः \textbf{प्र‚तिब‚न्धः} । एतेन सापेक्ष‚स्य नाव‚श्य‚म्भावित्वेन व्य‚भिचा‚{\tiny $_{१}$}‚रित्व‚{\tiny $_{lb}$}‚मुक्तं । \href{http://sarit.indology.info/?cref=pv.3.192-3.193}{१९५-९६}
	{\color{gray}{\rmlatinfont\textsuperscript{§~\theparCount}}}
	\pend% ending standard par
      ‚{\tiny $_{lb}$}‚

	  
	  \pstart \leavevmode% starting standard par
	स‚र्वेषां नाश‚हेतूनां नाश‚स्य लिङ्ग‚त्वेन ये हेत‚व उपादीय‚न्ते तेषां । क‚स्मात्‚{\tiny $_{lb}$}‚ [।] \textbf{कार्याव्य‚व‚स्थितेः} । नाश‚ल‚क्ष‚ण‚कार्योत्प‚त्तिनिय‚माभावात् । \textbf{हेतुम‚न्नाश‚वादिनां}‚{\tiny $_{lb}$}‚ हेतुम‚न्त‚न्नाशं ये व‚द‚न्ति तेषां । य‚त‚श्चाहेतुको विनाशः । \textbf{त‚त्त‚स्माद‚य‚म्भावः}‚{\tiny $_{lb}$}‚ \textbf{कृत‚कोन‚पेक्ष‚स्त‚द्भाव‚म्प्र‚ति} विन‚श्व‚र‚स्व‚भाव‚म्प्र‚ति । \textbf{त‚द्भाव‚निय‚तो} निवृत्तिध‚र्म‚{\tiny $_{lb}$}‚क‚तायां निय‚तः । दृष्टा‚{\tiny $_{२}$}‚न्त‚माह [।] \textbf{अस‚म्भ‚वे}त्यादि । न स‚म्भ‚व‚ति प्र‚तिब‚द्धो‚{\tiny $_{lb}$}‚ य‚स्यां सा कार‚ण‚साम‚ग्री । \textbf{स‚क‚लेति} स‚ह‚कारिप्र‚त्य‚येन स‚न्तान‚प‚रिणामेन च‚{\tiny $_{५}$}‚ प‚रि‚{\tiny $_{lb}$}‚पूर्ण्णेत्य‚र्थः । प्र‚योग‚स्तु । ये य‚द्भावं प्र‚त्य‚न‚पेक्षास्ते त‚द‚भाव‚निय‚ताः । \textbf{त‚द्य‚था‚{\tiny $_{lb}$}‚ ‚{\tiny $_{lb}$}‚ \leavevmode\ledsidenote{\textenglish{362/s}}ऽस‚म्भ‚व‚त्प्र‚तिब‚न्धा कार‚ण‚साम‚ग्री कार्योत्पाद‚ने} । अन्यान‚पेक्ष‚श्च कृत‚को भावो‚{\tiny $_{lb}$}‚ विनाश इति स्व‚भाव‚हेतुः । \href{http://sarit.indology.info/?cref=pv.3.193-3.194}{१९६-९७}
	{\color{gray}{\rmlatinfont\textsuperscript{§~\theparCount}}}
	\pend% ending standard par
      ‚{\tiny $_{lb}$}‚

	  
	  \pstart \leavevmode% starting standard par
	\textbf{न‚न्वि}त्यादिना नैकान्तिक‚त्व‚मा‚{\tiny $_{३}$}‚शंक‚ते । \textbf{क्व‚चित्} कार्येऽ\textbf{न‚पेक्षाणाम‚पि केषां‚{\tiny $_{lb}$}‚चित्} कार‚णानां \textbf{नाव‚श्य‚न्त‚द्भाव} इत्याह । \textbf{भूमी}त्यादि । सा हि कार्य‚ज‚न‚नेऽपेक्षा‚{\tiny $_{lb}$}‚ साम‚ग्री । त‚स्याम‚स‚त्याम‚पि \textbf{क‚दाचित्} प्र‚तिब‚न्ध‚कालेऽ\textbf{ङ्कुरानुत्प‚त्तेः} । एत‚च्च स‚न्ता‚{\tiny $_{lb}$}‚न‚स्यैक‚त्व‚म‚ध्य‚व‚सायोक्तं ।
	{\color{gray}{\rmlatinfont\textsuperscript{§~\theparCount}}}
	\pend% ending standard par
      ‚{\tiny $_{lb}$}‚

	  
	  \pstart \leavevmode% starting standard par
	\textbf{ने}त्यादिना प‚रिह‚र‚ति । \textbf{त‚त्र} य‚थोक्तायां साम‚ग्र्यां स‚न्तान‚स्य प‚रिणामः स्व‚भा‚{\tiny $_{lb}$}‚वान्त‚रोत्प‚त्तिल‚क्ष‚ण‚स्त‚त्र \textbf{सापेक्ष‚त्वा‚{\tiny $_{४}$}‚त्} । त‚तोऽन‚पेक्ष‚त्वादित्य‚स्य हेतोस्त‚त्रावृत्तिः ।‚{\tiny $_{lb}$}‚ कृत‚क‚स्याप्य‚स्ति विनाशं प्र‚ति कालान्त‚रापेक्षा त‚तो हेतुर‚सिद्ध इत्याह । \textbf{नैव}‚{\tiny $_{lb}$}‚मित्यादि । कृत‚क‚स्य \textbf{भाव‚स्य} नाशे \textbf{काचित्} कालान्त\textbf{रापेक्षेति} व‚क्ष्य‚ति ।
	{\color{gray}{\rmlatinfont\textsuperscript{§~\theparCount}}}
	\pend% ending standard par
      ‚{\tiny $_{lb}$}‚

	  
	  \pstart \leavevmode% starting standard par
	स्यादेत‚द् [।] एक‚स्व‚भावा एव भूमिबीजाद‚यः कुत‚स्तेषां स‚न्तान‚प‚रिणा‚{\tiny $_{lb}$}‚मापेक्ष‚त्वं [।] अतो व्य‚भिचार एव हेतोरित्याह । \textbf{त‚त्रापी}त्यादि ।
	{\color{gray}{\rmlatinfont\textsuperscript{§~\theparCount}}}
	\pend% ending standard par
      ‚{\tiny $_{lb}$}‚

	  
	  \pstart \leavevmode% starting standard par
	एत‚दुक्त‚म्भ‚{\tiny $_{५}$}‚व‚ति । न भूमिबीजाद‚य एक‚स्व‚भावाः प‚श्चादिव प्राग‚पि कार्यो‚{\tiny $_{lb}$}‚त्पाद‚न‚प्र‚स‚ङ्गात् । किन्तूत्त‚रोत्त‚र‚प‚रिणामेन भिन्नाः । त‚त्रेति त‚स्यां स‚न्तान‚प‚रिणामेन‚{\tiny $_{lb}$}‚ भिन्नायां \textbf{साम‚ग्र्याम‚न्त्या या} साम‚ग्री । कार्योत्पाद‚ने ल‚क्ष‚णान्त‚रेणा\textbf{व्य‚व‚हिता} सा‚{\tiny $_{lb}$}‚ \textbf{फ‚ल‚व‚त्येवेति} कुतो हेतोर‚नैकान्तिक‚त्वं ।
	{\color{gray}{\rmlatinfont\textsuperscript{§~\theparCount}}}
	\pend% ending standard par
      ‚{\tiny $_{lb}$}‚

	  
	  \pstart \leavevmode% starting standard par
	स्यादेत‚त् [।] पूर्वा साम‚ग्री ज‚निकापि स‚ती न निय‚तेत्याह । \textbf{सैवे}त्य‚न्त्या‚{\tiny $_{lb}$}‚ साम‚{\tiny $_{६}$}‚ग्री त‚त्र \textbf{तासु म‚ध्येऽङ्कुर‚हेतु}र्नान्या काचित् । किम‚र्थ‚न्त‚र्ह्य‚ङ्कुरार्थिभिः पूर्वा‚{\tiny $_{lb}$}‚ साम‚ग्र्युपादीय‚त इत्याह । \textbf{अन्यास्त्वि}त्यादि । पूर्वः प‚रिणामः पूर्वोव‚स्थाविशेष\textbf{स्त‚द‚र्थ‚{\tiny $_{lb}$}‚एवा}ङ्कुर‚ज‚न‚न‚स‚म‚र्थान्त्य‚साम‚ग्र्य‚र्थ एव । तेनार्थिभिरुपादीय‚ते साम‚ग्री रूप‚त‚या‚{\tiny $_{lb}$}‚ चाध्य‚व‚सीय‚ते ।
	{\color{gray}{\rmlatinfont\textsuperscript{§~\theparCount}}}
	\pend% ending standard par
      ‚{\tiny $_{lb}$}‚

	  
	  \pstart \leavevmode% starting standard par
	स्यादेत‚द् [।] अन्त्याया अपि साम‚ग्र्याः प्र‚तिब‚न्धः स‚म्भ‚व‚ति । तेन‚{\tiny $_{lb}$}‚ \leavevmode\ledsidenote{\textenglish{132a/PSVTa}} कार्योत्पाद‚नि‚{\tiny $_{७}$}‚य‚माभावात् साध्य‚शून्यो दृष्टान्त इत्याह \textbf{न चे}त्यादि । \textbf{ताम}न्त्यां‚{\tiny $_{lb}$}‚ ‚{\tiny $_{lb}$}‚ \leavevmode\ledsidenote{\textenglish{363/s}}साम‚ग्री\textbf{न्त‚त्र} कार्ये ज‚न्ये । \textbf{एक‚त्र भाव} इत्य‚न्त्ये क्ष‚णे । विकार‚स्यो\textbf{त्प‚त्तौ वा} त‚स्या‚{\tiny $_{lb}$}‚न्त्य‚स्य क्ष‚ण‚स्यै\textbf{क‚त्व‚हानेः} पूर्व‚स्य प्र‚च्युतेर्विकाराख्य‚स्य च द्वितीय‚स्योत्प‚त्तेः । त‚त‚श्च‚{\tiny $_{lb}$}‚ नासाव‚न्त्यः स्यात् ।
	{\color{gray}{\rmlatinfont\textsuperscript{§~\theparCount}}}
	\pend% ending standard par
      ‚{\tiny $_{lb}$}‚

	  
	  \pstart \leavevmode% starting standard par
	अथ न त‚स्यान्त्य‚स्य ज‚न‚क‚स्व‚भावात् प्र‚च्युतिरिष्य‚ते । त‚दा त‚दात्म‚नो ज‚न‚का‚{\tiny $_{lb}$}‚त्म‚नः । स्व‚भावाद\textbf{प्र‚च्युत‚स्य त‚{\tiny $_{१}$}‚दुत्पाद‚नं} कार्योत्पाद‚नं \textbf{प्र‚ति वैगुण्य‚म‚कुर्वाण‚स्य‚{\tiny $_{lb}$}‚ प्र‚तिब‚न्ध‚हेतोर}प्र‚तिब‚न्ध‚क‚त्वाद् विघात‚क‚र‚णात् ।
	{\color{gray}{\rmlatinfont\textsuperscript{§~\theparCount}}}
	\pend% ending standard par
      ‚{\tiny $_{lb}$}‚

	  
	  \pstart \leavevmode% starting standard par
	पुन‚र‚पि व्य‚भिचार‚माशंक‚ते । \textbf{य‚व‚बीजाद‚यो न सापेक्षाः} [।] \textbf{क‚स्मिन्} [।]‚{\tiny $_{lb}$}‚ \textbf{शाल्य‚ङ्कुरे} कार्ये ज‚न्ये । क‚स्मात् [।] \textbf{त‚दुत्प‚त्तिप्र‚त्य‚यानां क‚दाचित् त‚त्रापि}‚{\tiny $_{lb}$}‚ य‚व‚बीजादौ \textbf{स‚न्निधानात्} । ते \textbf{निर‚पेक्षा} अपि न शाल्य‚कुरं ज‚न‚य‚न्तीत्य‚नैकान्तिक‚{\tiny $_{lb}$}‚ एवेति ।
	{\color{gray}{\rmlatinfont\textsuperscript{§~\theparCount}}}
	\pend% ending standard par
      ‚{\tiny $_{lb}$}‚

	  
	  \pstart \leavevmode% starting standard par
	\textbf{क‚थ‚मिति} सि द्धा न्त वा दी ।‚{\tiny $_{२}$}‚ सापेक्षा एवेत्य‚र्थः । \textbf{एषा}मिति य‚व‚बीजादीनां‚{\tiny $_{lb}$}‚ \textbf{शालिबीज‚स्य} य‚स्त\textbf{दुत्पाद‚नः} शाल्य‚ङ्कुरोत्पाद‚नः स्व‚भावः स एवैषां नास्तीति‚{\tiny $_{lb}$}‚स‚म्ब‚न्धः । \textbf{त‚त्स्व‚भावापेक्षा} इति शाल्यंकुरोत्पाद‚न‚स्व‚भावापेक्षाः ।
	{\color{gray}{\rmlatinfont\textsuperscript{§~\theparCount}}}
	\pend% ending standard par
      ‚{\tiny $_{lb}$}‚

	  
	  \pstart \leavevmode% starting standard par
	क‚दा च क‚थं निर‚पेक्ष‚त्वं स्या\textbf{देव‚न्त‚र्हीति} प‚रः । \textbf{कृत‚कानां च केषांचित् स‚ताम्वा}‚{\tiny $_{lb}$}‚ केषांचित् । \textbf{स एव} स्व‚भावो \textbf{नास्ति यो न‚श्व‚रः । त‚स्मात् त‚त्स्व‚भावापेक्ष‚त्वाद्}‚{\tiny $_{lb}$}‚ विन‚श्व‚{\tiny $_{३}$}‚र‚स्व‚भावापेक्ष‚त्वान्न \textbf{विन‚श्व‚रा} इत्य‚सिद्ध‚त्वं हेतोरिति ।
	{\color{gray}{\rmlatinfont\textsuperscript{§~\theparCount}}}
	\pend% ending standard par
      ‚{\tiny $_{lb}$}‚

	  
	  \pstart \leavevmode% starting standard par
	एत‚न्निराक‚र्त्तुम्प्र‚क्र‚म‚ते । \textbf{शालिबीजे}त्यादि । आदिश‚ब्दाद् य‚व‚बीजादीनां । स‚{\tiny $_{lb}$}‚ \textbf{स्व‚भाव} इत्य‚भिम‚तेत‚र‚कार्य ज‚न‚नाज‚न‚न‚स्व‚भावः \textbf{स्व‚हेतो}रिति कृत्वा । यो य‚व‚बी‚{\tiny $_{lb}$}‚जादिर्न \textbf{त‚द्धेतुः । स} शालिबीज‚हेतुर्य‚स्य हेतु\textbf{र्न‚भ‚व‚ती}त्य‚र्थः । सोऽत‚त्स्व‚भाव इत्य‚शा‚{\tiny $_{lb}$}‚ल्यंकुर‚ज‚न‚न‚स्व‚भावः ।
	{\color{gray}{\rmlatinfont\textsuperscript{§~\theparCount}}}
	\pend% ending standard par
      ‚{\tiny $_{lb}$}‚

	  
	  \pstart \leavevmode% starting standard par
	न‚न्व‚त‚द्धेतुश्च स्या‚{\tiny $_{४}$}‚त् त‚त्स्व‚भाव‚श्चेत्याह । \textbf{निय‚त‚श‚क्ति}श्चेत्यादि । निय‚ता‚{\tiny $_{lb}$}‚ प्र‚तिनिय‚ता श‚क्तिर्य‚स्य स त‚था [।] \textbf{स हेतुरिति} शालिय‚व‚बीज‚ज‚न‚न‚स्व‚भावः ।‚{\tiny $_{lb}$}‚ \textbf{स्व‚रूपेण} विभ‚क्तेनैव स्व‚भावेन \textbf{प्र‚तीतः} प्र‚त्य‚क्ष‚तः ।
	{\color{gray}{\rmlatinfont\textsuperscript{§~\theparCount}}}
	\pend% ending standard par
      ‚{\tiny $_{lb}$}‚‚{\tiny $_{lb}$}‚\textsuperscript{\textenglish{364/s}}

	  
	  \pstart \leavevmode% starting standard par
	स्यादेत‚त् [।] न हेतुकृतः स्व‚भाव‚भेदो भावानां किन्तु स्व‚भाव एव क‚स्य‚{\tiny $_{lb}$}‚चित् [।] तादृश‚स्व‚भावोन्य‚स्य चान्यादृश इत्याह । \textbf{न चे}त्यादि । \textbf{आक‚स्मिक}‚{\tiny $_{lb}$}‚ इति निर्हेतुकः । \textbf{अन‚पेक्ष‚स्या}हेतोः‚{\tiny $_{५}$}‚ क्व‚चि\textbf{द्देशे} । क्व‚चित्\textbf{काले} । क्व‚चिच्च शालि‚{\tiny $_{lb}$}‚बीजादौ \textbf{द्र‚व्ये} शाल्यंकुरोत्पाद‚न‚स्य स्व‚भाव‚स्य \textbf{निय‚मो} न स्यात् । किन्तु स‚र्व्व‚स्य‚{\tiny $_{lb}$}‚ स‚र्व‚दा स‚र्व‚त्र भ‚वेद‚पेक्षाभावात् । त‚स्माद् देशादिक‚म‚पेक्ष्य भ‚व‚न्निय‚मो हेतुमानिति‚{\tiny $_{lb}$}‚ ग‚म्य‚ते । य‚था शालिबीजादीनां स्व‚भाव‚निय‚म‚स्\textbf{त‚थात्रापि} कृत‚केषु, स‚त्सु वा‚{\tiny $_{lb}$}‚ \textbf{निय‚म‚हेतुर्व‚क्त‚व्यो य‚तो} नियाम‚काद्धेतोः कृत‚कास्स‚न्तो वा \textbf{केचि‚{\tiny $_{६}$}‚न्न‚श्व‚रात्मानो‚{\tiny $_{lb}$}‚ जाता} नान्ये ।
	{\color{gray}{\rmlatinfont\textsuperscript{§~\theparCount}}}
	\pend% ending standard par
      ‚{\tiny $_{lb}$}‚

	  
	  \pstart \leavevmode% starting standard par
	स्यादेत‚द् [।] य‚दि नाम नियाम‚को हेतुर्न श‚क्य‚ते द‚र्श‚यितुन्त‚थापि स‚म्भाव्य‚त‚{\tiny $_{lb}$}‚ इत्याह । \textbf{न चात्र} लोके न‚श्व‚र‚स्य स्व‚भाव‚स्य \textbf{नियाम‚को हेतुर‚स्ति} । न स‚म्भाव्य‚त‚{\tiny $_{lb}$}‚ एवेति याव‚त् । स‚र्वेषां ज‚न्म‚व‚तां नाश‚स्य सिद्धेर्दृष्ट‚त्वात् । अनिय‚त‚हेतुको विनाश‚{\tiny $_{lb}$}‚ इति याव‚त् ।
	{\color{gray}{\rmlatinfont\textsuperscript{§~\theparCount}}}
	\pend% ending standard par
      ‚{\tiny $_{lb}$}‚

	  
	  \pstart \leavevmode% starting standard par
	\leavevmode\ledsidenote{\textenglish{132b/PSVTa}} य‚दि \textbf{स‚र्व‚ज‚न्मिनां विनाश‚सिद्धिरे}व‚न्त‚र्हि स‚त्त्वादिति हेतु‚{\tiny $_{७}$}‚र‚नैकान्तिकः स्यात्त‚{\tiny $_{lb}$}‚दाह । \textbf{ज‚न्मी}त्यादि । ज‚न्म‚व‚ता\textbf{मेव स्व‚भावो नाशी} नाज‚न्म‚व‚तां । नाकाशादीनां‚{\tiny $_{lb}$}‚ स‚ताम‚पीति प‚रो म‚न्य‚ते । आ चा र्य आह । \textbf{न वै ज‚न्मे}ति । न हि ज‚न्म‚व‚शाद् भाव‚स्य‚{\tiny $_{lb}$}‚ स्व‚भाव उत्प‚द्य‚ते । त‚स्मान्न ज‚न्म \textbf{नाश‚स्य हेतुः} । नाप्याकाशादौ स‚त्त्व‚म‚स्तीत्याह ।‚{\tiny $_{lb}$}‚ \textbf{न चे}त्यादि । \textbf{अहेतो}राकाशादेः \textbf{स्व‚भाव‚निय‚मः} स्व‚रूप‚निय‚मोऽहेतोर्देश‚काल‚प्र‚कृति‚{\tiny $_{lb}$}‚निय‚मा‚{\tiny $_{१}$}‚योगात् । य‚त‚श्च स‚र्व‚ज‚न्मिनां विनाश‚सिद्धिराकाशादीनां चास‚त्त्वं ।‚{\tiny $_{lb}$}‚ \textbf{त‚स्मान्नात्र} कृत‚केषु स‚त्सु वा \textbf{हेतो}र्न‚श्व‚रान‚श्व‚र‚ज‚न‚क‚त्वेन \textbf{स्व‚भाव‚प्र‚विभागः ।‚{\tiny $_{lb}$}‚ त‚द्भावाद्धे}तुप्र‚विभागाभावात् \textbf{फ‚ल‚स्य} कृत‚क‚स्य \textbf{स‚तो वा} न‚श्व‚रान‚श्व‚र‚प्र‚विभागो‚{\tiny $_{lb}$}‚ \textbf{नास्तीत्य‚स‚मानं} य‚व‚बीजादिना । \textbf{सेय‚म्विनाश‚स्य निर‚पेक्ष‚ता क्व‚चिद्} व‚स्तुनि ।‚{\tiny $_{lb}$}‚ \textbf{क‚दाचित्} काले विनाश‚स्य यो \textbf{भा‚{\tiny $_{२}$}‚व}स्तेन \textbf{विरोधिनी} क‚रोति \textbf{त‚द‚भावं । त‚स्य क्व‚चित‚{\tiny $_{lb}$}‚ ‚{\tiny $_{lb}$}‚ \leavevmode\ledsidenote{\textenglish{365/s}}क‚दाचिच्च} विनाश‚स्याभावं \textbf{स्व‚भावेन} स‚त्त‚या \textbf{साध‚य‚ति} । स‚र्व‚त्र स‚र्व‚काल‚म्भावं‚{\tiny $_{lb}$}‚ साध‚य‚तीति याव‚त् । किं कार‚णं । \textbf{यो ही}त्यादि । \textbf{त‚त्काल‚द्र‚व्यापेक्ष} इति य‚स्मिन्‚{\tiny $_{lb}$}‚ काले भ‚व‚ति य‚त्र वा द्र‚व्ये । तं कालं द्र‚व्य‚ञ्चापेक्ष‚त इति \textbf{निर‚पेक्ष एव न स्यादित्युक्तं}‚{\tiny $_{lb}$}‚ प्राक् ।
	{\color{gray}{\rmlatinfont\textsuperscript{§~\theparCount}}}
	\pend% ending standard par
      ‚{\tiny $_{lb}$}‚

	  
	  \pstart \leavevmode% starting standard par
	न‚नु विनाश‚क‚हेत्व‚न‚पेक्ष‚त्वेन विनाश‚स्यान‚पे‚{\tiny $_{३}$}‚क्ष‚त्वं, न तु कालाद्य‚न‚पेक्ष‚त्वेन‚{\tiny $_{lb}$}‚ [।] त‚त्क‚थ‚मुच्य‚ते त‚त्काल‚द्र‚व्यापेक्ष इति निर‚पेक्ष एव विनाशो न स्यादिति । य‚दि‚{\tiny $_{lb}$}‚ च कालान‚पेक्षो विनाशः द्वितीयेपि क्ष‚णे विनाशो न स्यात् त‚त्कालापेक्ष‚त्वात् ।‚{\tiny $_{lb}$}‚ द्र‚व्यान‚पेक्ष‚त्वे च क‚स्य त‚र्हि विनाशः स्यात् ।
	{\color{gray}{\rmlatinfont\textsuperscript{§~\theparCount}}}
	\pend% ending standard par
      ‚{\tiny $_{lb}$}‚

	  
	  \pstart \leavevmode% starting standard par
	एव‚म्म‚न्य‚ते [।] जात‚स्य त‚द्भावेऽन्योन‚पेक्ष‚णादितिं व‚च‚नात् । द्वितीय एव‚{\tiny $_{lb}$}‚ क्ष‚णे विनाशो भ‚व‚ति नान्य‚स्मिन् क्ष‚णे ।‚{\tiny $_{४}$}‚ त‚था स‚र्व‚स्य जात‚स्य भ‚व‚ति न‚{\tiny $_{lb}$}‚ द्र‚व्य‚विशेष‚स्य । तेन द्र‚व्याऽन‚पेक्ष‚त्वे क‚स्य त‚र्हि विनाशो भ‚व‚तु [।]
	{\color{gray}{\rmlatinfont\textsuperscript{§~\theparCount}}}
	\pend% ending standard par
      ‚{\tiny $_{lb}$}‚

	  
	  \pstart \leavevmode% starting standard par
	इति निर‚स्तं । कालान्त‚रे द्र‚व्य‚विशेषे च नाश‚स्य भावे कालान्त‚र‚स्य‚{\tiny $_{lb}$}‚ द्र‚व्य‚विशेष‚स्य च विनाश‚क‚त्व‚मेव स्यात् । \textbf{विनाश‚स्य त‚द्भाव} एव भावात् ।‚{\tiny $_{lb}$}‚ \textbf{त‚स्मात्} त‚त्काल‚द्र‚व्यापेक्ष‚त्वे \textbf{निर‚पेक्ष} एव न स्यादित्युच्य‚ते ।
	{\color{gray}{\rmlatinfont\textsuperscript{§~\theparCount}}}
	\pend% ending standard par
      ‚{\tiny $_{lb}$}‚

	  
	  \pstart \leavevmode% starting standard par
	\textbf{त‚र्ही}ति प‚रः । नेत्या चा र्यः । \textbf{स‚त्ताया} यो हे\textbf{तुर्भाव}स्त‚स्मा‚{\tiny $_{५}$}‚\textbf{देव त‚थोत्प‚त्ते}‚{\tiny $_{lb}$}‚र्न‚श्व‚र‚स्व‚भाव‚स्योत्प‚त्तेः । एत‚देव स्प‚ष्ट‚य‚ति । \textbf{स‚तो हि भ‚व‚त} इति स‚त्तां‚{\tiny $_{lb}$}‚ प्र‚तिप‚द्य‚मान‚स्य \textbf{तादृश‚स्यैव} न‚श्व‚र‚स्व‚भाव‚स्यैवआ \textbf{भावात् । नाव‚श्यं स‚तः} प‚दार्थ‚स्य‚{\tiny $_{lb}$}‚ \textbf{कुत‚श्चित्} कार‚णाद् \textbf{भाव} उत्पाद‚न \textbf{इति चेत्} । केचिद्धि स‚न्तोपि नोत्प‚त्तिम‚न्तो‚{\tiny $_{lb}$}‚ य‚थाकाशाद‚य इति प‚रो म‚न्य‚ते । \textbf{आक‚स्मिकी}त्य‚हेतुका । \textbf{नेयं} स‚त्ता \textbf{क‚स्य}चिद‚र्थ‚स्य‚{\tiny $_{lb}$}‚ \textbf{क‚दाचित्} काले क्व‚चिद् द्र‚व्ये‚{\tiny $_{६}$}‚ \textbf{विर‚मेत} ।
	{\color{gray}{\rmlatinfont\textsuperscript{§~\theparCount}}}
	\pend% ending standard par
      ‚{\tiny $_{lb}$}‚

	  
	  \pstart \leavevmode% starting standard par
	न‚नु च घ‚टादीनां स्व‚हेतुतः स‚त्तानिय‚मे क‚थ‚माकाशादिस‚त्ताप्र‚स‚ङ्गः ।
	{\color{gray}{\rmlatinfont\textsuperscript{§~\theparCount}}}
	\pend% ending standard par
      ‚{\tiny $_{lb}$}‚

	  
	  \pstart \leavevmode% starting standard par
	स‚त्त्यं [।] किन्तु न क‚स्य‚चिद् विर‚मेतेत्य‚स्याय‚म‚र्थः । न स क‚श्चिद् भावो‚{\tiny $_{lb}$}‚ य‚त्स्व‚भावोसावाकाशादिर्न स्यात् स‚र्वात्म‚कः प्र‚स‚ज्य‚त इत्येके । त‚द‚युक्तं [।]‚{\tiny $_{lb}$}‚ य‚दि हि घ‚टादिरूप‚माकाशे स्याद‚हेतुकं स्यात् त‚स्य मृत्पिण्डादिक‚म‚न्त‚रेण भावात् ।‚{\tiny $_{lb}$}‚ \textbf{त‚स्मान्नेय‚मा}काशादेस्स‚त्ता । \textbf{क‚स्य‚चि‚{\tiny $_{७}$}‚}दात्मादेः \textbf{क‚दाचित्} काले \textbf{क्व‚चिद्} देशे \leavevmode\ledsidenote{\textenglish{133a/PSVTa}}‚{\tiny $_{lb}$}‚ ‚{\tiny $_{lb}$}‚ \leavevmode\ledsidenote{\textenglish{366/s}}विर‚मेत । देश‚काल‚द्र‚व्य‚निय‚ता न भ‚वेदित्य‚र्थः ।
	{\color{gray}{\rmlatinfont\textsuperscript{§~\theparCount}}}
	\pend% ending standard par
      ‚{\tiny $_{lb}$}‚

	  
	  \pstart \leavevmode% starting standard par
	एत‚देव द्र‚ढ‚य‚न्नाह । त‚द्धीत्यादि । त‚द्धि व‚स्तु । किंचिदुप‚लीयेताश्र‚येत् ।‚{\tiny $_{lb}$}‚ \textbf{य‚स्य य‚त्र किंचिदु}त्पादादिकं प्र‚तिब‚द्ध‚माय‚त्तं । न चोप‚लीयेत य‚स्य य‚त्राप्र‚तिब‚द्धं ।‚{\tiny $_{lb}$}‚ \textbf{सेयं स‚त्ता क्व‚चिद‚प्र‚तिब‚न्धिनी चेत्} । द्र‚व्य‚कालापेक्ष‚या \textbf{न निय‚म‚व‚ती स्यात्} । त‚था‚{\tiny $_{lb}$}‚ चाकाश‚स्येयं स‚त्ता नात्म‚नः । आ‚{\tiny $_{१}$}‚त्म‚न‚स्स‚त्ता न काल‚स्येत्यादि न स्यात् । \textbf{य‚त‚श्चै‚{\tiny $_{lb}$}‚व‚न्त‚स्मान्नेयं स‚त्ताक‚स्मिकी क्व‚चि}न्नित्याभिम‚तेष्व‚पि ।
	{\color{gray}{\rmlatinfont\textsuperscript{§~\theparCount}}}
	\pend% ending standard par
      ‚{\tiny $_{lb}$}‚

	  
	  \pstart \leavevmode% starting standard par
	\textbf{य‚दि} स‚त्ताहेतोरेव विन‚श्व‚र‚स्योत्पादः \textbf{क‚थ}मिदानी\textbf{म‚हेतुको विनाश उक्त} इति‚{\tiny $_{lb}$}‚ व्याघात‚माह ।
	{\color{gray}{\rmlatinfont\textsuperscript{§~\theparCount}}}
	\pend% ending standard par
      ‚{\tiny $_{lb}$}‚

	  
	  \pstart \leavevmode% starting standard par
	\textbf{जात‚स्ये}त्यादिना प‚रिह‚र‚ति । \textbf{जात‚स्य} निष्प‚न्न‚स्य \textbf{त‚द्भावे} विन‚श्व‚र‚ताभावे‚{\tiny $_{lb}$}‚ ज‚न‚काद्धेतो\textbf{र‚न्य‚स्यान‚पेक्ष‚णात्} ।
	{\color{gray}{\rmlatinfont\textsuperscript{§~\theparCount}}}
	\pend% ending standard par
      ‚{\tiny $_{lb}$}‚

	  
	  \pstart \leavevmode% starting standard par
	अहेतुको विनाश उक्तः । उक्तं \textbf{चात्र} प्राक् य‚था‚{\tiny $_{२}$}‚ \textbf{न विनाशो नामान्य एव‚{\tiny $_{lb}$}‚ क‚श्चिद्} भावात् किन्तु \textbf{भाव एव हि नाशः} ।
	{\color{gray}{\rmlatinfont\textsuperscript{§~\theparCount}}}
	\pend% ending standard par
      ‚{\tiny $_{lb}$}‚

	  
	  \pstart \leavevmode% starting standard par
	न‚नु च प्र‚ध्वंसाभावो नाशः स क‚थ‚म्भाव‚स्व‚भावो भ‚व‚तीत्याह ।
	{\color{gray}{\rmlatinfont\textsuperscript{§~\theparCount}}}
	\pend% ending standard par
      ‚{\tiny $_{lb}$}‚

	  
	  \pstart \leavevmode% starting standard par
	\textbf{स एव ही}त्यादि । य‚स्मात् स्व‚हेतोरेव \textbf{क्ष‚ण‚स्थायी जात‚स्त‚स्माद्} भाव एव नाश‚{\tiny $_{lb}$}‚ उक्तः । य एव स‚त्ताकालो भाव‚स्य त‚मेवैकं क्ष‚णं स्थातुं शीलं य‚स्य स त‚था । न‚{\tiny $_{lb}$}‚ पुन‚रुत्प‚द्य क्ष‚ण‚म‚पि तिष्ठ‚ति । य‚दि तिष्ठेन्न क‚दाचिन्न निव‚र्त्तेतेति व‚क्ष्याम‚{\tiny $_{३}$}‚ः ।‚{\tiny $_{lb}$}‚ य‚द्येक‚क्ष‚ण‚स्थायी भावो विनाशः क‚स्मात् प्र‚वाह‚विच्छेदात् प्राग‚पि त‚था न‚{\tiny $_{lb}$}‚ निश्चीय‚तेऽनिश्च‚याच्चाक्ष‚णिकः स इत्याह । \textbf{त‚म‚स्ये}त्यादि । \textbf{तं} नाश‚स्व‚भाव‚म‚स्य‚{\tiny $_{lb}$}‚ घ‚टादेरासंसार‚म‚विद्यानुब‚न्धात् \textbf{म‚न्दा उर्ध्वं} प्र‚वाह‚विच्छेद‚काले \textbf{व्य‚व‚स्य‚न्ति न प्राक्}‚{\tiny $_{lb}$}‚ स‚त्ताकाले । \textbf{द‚र्श‚नेपीति} न‚श्व‚र‚स्व‚भाव‚स्य स‚त्य‚पि \textbf{द‚र्श‚ने} । न द‚र्श‚न‚कालेऽध्य‚व‚{\tiny $_{lb}$}‚सायोस्ति । अविद्यासाम\textbf{र्थ्या}‚{\tiny $_{४}$}‚त्स‚दृशाप‚रोत्प‚त्त्या च द‚र्श‚न‚पाट‚व‚स्याभावात् ।‚{\tiny $_{lb}$}‚ य‚त‚श्च स‚न्तान‚विच्छेद‚काले नाश‚स्व‚भाव‚स्य निश्च‚य\textbf{स्त‚स्मात् त‚द्व‚शेन} निश्च‚य‚{\tiny $_{lb}$}‚‚{\tiny $_{lb}$}‚ \leavevmode\ledsidenote{\textenglish{367/s}}व‚शेन । \textbf{प‚श्चा}दिति य‚स्मिंन् काले नाश‚स्व‚भाव‚स्य निश्च‚य‚स्त‚त्कालोपाधिरेव‚{\tiny $_{lb}$}‚ स भाव‚स्य विनाश‚स्व‚भावो व्य‚व‚स्थाप्य‚ते [।] दृष्टान्त‚माह । \textbf{विकारे}त्यादि ।‚{\tiny $_{lb}$}‚ य‚था विष‚द्र‚व्यं गृहीत‚म‚पि भ्रान्तिस‚द्भावात् प्राग‚न‚व‚धारित‚म‚ज्ञैः पुरुषैः प‚श्चाद्‚{\tiny $_{lb}$}‚ विष‚कृ‚{\tiny $_{५}$}‚त‚स्य विकार‚स्य लालास्रुत्यादेर्द‚र्श‚नेन विषं व्य‚व‚स्थाप्य‚ते त‚द्व‚त् । एताव‚{\tiny $_{lb}$}‚न्मात्रेणायं दृष्टान्तो न तु मार‚ण‚श‚क्तिर्गृहीता प‚श्चाद‚व‚धार्य‚त इति दृष्टान्तः ।‚{\tiny $_{lb}$}‚ त‚दिति त‚स्माद् \textbf{अय‚म्विनाश} इति स‚म्ब‚न्धः । व‚स्तुनो या स‚त्ता त‚द्व्य‚तिरेकेण न‚{\tiny $_{lb}$}‚ किचिद् विनाश‚कार‚ण\textbf{म‚पेक्ष‚त इति त‚द्व्यापी} स‚त्ताव्यापी ।
	{\color{gray}{\rmlatinfont\textsuperscript{§~\theparCount}}}
	\pend% ending standard par
      ‚{\tiny $_{lb}$}‚

	  
	  \pstart \leavevmode% starting standard par
	\textbf{क‚थ‚मित्या}दि प‚रः । \textbf{असाम‚र्थ्याच्चे}ति प्र‚तिव‚च‚नं । त‚द्धेतोरिति नाश‚हे‚{\tiny $_{६}$}‚तोः ।‚{\tiny $_{lb}$}‚ च‚काराद‚कार‚क‚त्वाच्च । एत‚देव विवृण्व‚न्नाह । अभाव‚कारिण इत्य‚भाव‚कारि‚{\tiny $_{lb}$}‚त‚याभिम‚त‚स्य क्रियायाः कार‚क‚त्व‚स्य \textbf{प्र‚तिषेधाच्चेति} । अव‚स्तुकारी योभिम‚तः‚{\tiny $_{lb}$}‚ सोकार‚क एव भ‚व‚ति । असाम‚र्थ्य‚न्द‚र्श‚यितुमाह । सिद्धे हीत्यादि । नाश‚हेतुर्भावा‚{\tiny $_{lb}$}‚द‚भिन्न‚म्वा विनाशं कुर्यात्त‚तोन्य‚म्वा । न ताव‚दाद्यः प‚क्षः सिद्धे हि \textbf{भावे} कार‚को‚{\tiny $_{lb}$}‚ नाश‚हेतुस्त‚म्भावं \textbf{न क‚रो‚{\tiny $_{७}$}‚ति} सिद्ध‚त्वादेव । नापि द्वितीयः प‚क्षः । य‚तो \textbf{नाप्य‚न्य- \leavevmode\ledsidenote{\textenglish{133b/PSVTa}}‚{\tiny $_{lb}$}‚ क्रियान्त‚स्य} भाव‚स्य \textbf{न किञ्चित्} । त‚द‚व‚स्थ‚त्वात् । त‚द‚त‚द्रूपेत्यादि । भाव‚रूप‚स्य‚{\tiny $_{lb}$}‚ त‚तोन्य‚स्य च विनाश‚स्य \textbf{कार‚णाच्चाकिञ्चि}त्क‚रो विनाश‚हेतुर‚तो नापेक्ष्य‚त इति‚{\tiny $_{lb}$}‚ सिद्ध‚म‚साम‚र्थ्य ।
	{\color{gray}{\rmlatinfont\textsuperscript{§~\theparCount}}}
	\pend% ending standard par
      ‚{\tiny $_{lb}$}‚

	  
	  \pstart \leavevmode% starting standard par
	\textbf{क्रियाप्र‚तिषेध‚स्तु क‚थ}मित्याह । \textbf{विनाश इति हि भाव‚स्याभाव‚न्ते} हेतुम‚न्ना‚{\tiny $_{lb}$}‚श‚वादिनो \textbf{म‚न्य‚न्ते} । अस्माभिर्भाव‚स्व‚भाव एव विनाश इत्यु‚{\tiny $_{१}$}‚क्तं । त‚दिति त‚स्मा‚{\tiny $_{lb}$}‚द‚यं विनाश‚हेतुर्विनाशं क‚रोत्य‚भावं क‚रोतीति प्राप्तं । त‚त्रैत‚स्मिन् प्राप्ते स‚ति य‚द्य‚{\tiny $_{lb}$}‚भावो नाम क‚श्चित् स्व‚भावः कार्यः स्यात् । त‚दा कार्य‚त्वादंकुरादिव‚त् स एव भाव‚{\tiny $_{lb}$}‚ इति नाभावः स्यात् । अथ य‚था घ‚टो घ‚ट‚रूपेण कार्यः प‚ट‚श्च प‚ट‚रूपेण कार्यो न तु‚{\tiny $_{lb}$}‚ कार्य‚त्वाद् घ‚टः प‚टो भ‚व‚ति । त‚था भावो भाव‚रूपेण कार्योऽभावोप्य‚भाव‚रूपेण‚{\tiny $_{lb}$}‚ कार्यः‚{\tiny $_{२}$}‚ स्यात् । न तु भाव एव भ‚व‚तीति । त‚द‚युक्त‚म्भ‚व‚तीति हि भावो न भ‚व‚तीति‚{\tiny $_{lb}$}‚ ‚{\tiny $_{lb}$}‚ \leavevmode\ledsidenote{\textenglish{368/s}}चाभाव‚स्तेनाभावो भाव‚विरोधी । न चाभाव‚रूप‚त‚या त‚स्य प्र‚तिभास‚नाद‚भाव‚{\tiny $_{lb}$}‚रूप‚ता । भ‚व‚न‚ध‚र्म‚त्वेनाभाव‚रूप‚त‚या प्र‚तिभास एव न स्यादितीद‚मेव चोद्य‚ते ।‚{\tiny $_{lb}$}‚ न च प‚र‚स्प‚र‚विविक्त‚रूप‚त‚याऽभावानां प्र‚तिभासः । य‚त‚श्चाभाव‚स्य नीरूप‚त्वा‚{\tiny $_{lb}$}‚द‚कार्य‚त्व\textbf{न्त‚स्माद‚भावं क‚रोतीति भावं‚{\tiny $_{३}$}‚ न क‚रोतीति} वाक्यार्थः स्यात् । तेन \textbf{क्रिया‚{\tiny $_{lb}$}‚प्र‚तिषेधोस्य} नाश‚हेतोः \textbf{कृतः स्यात् । त‚थापि} क्रियाप्र‚तिषेधेप्य‚यं विनाश‚हेतु\textbf{र‚कि‚{\tiny $_{lb}$}‚ञ्चित्क‚रः । किमिति} नाशेऽपेक्ष्य‚ते भावेनेति \textbf{सिद्धा विनाशं प्र‚त्य‚न‚पेक्षा भाव‚स्य} ।
	{\color{gray}{\rmlatinfont\textsuperscript{§~\theparCount}}}
	\pend% ending standard par
      ‚{\tiny $_{lb}$}‚

	  
	  \pstart \leavevmode% starting standard par
	न‚नु निर्हेतुकेपि विनाशे क‚थ‚म्विनाशं प्र‚त्य‚न‚पेक्षा भाव‚स्य । स्व‚भावो हि स‚{\tiny $_{lb}$}‚ त‚स्येत्थं येनापेक्ष्य निव‚र्त्त‚ते विरोधिनं [।] य‚थाऽन्येषां प्र‚वाहो मुद्ग‚रा‚{\tiny $_{४}$}‚दिकं । तेन‚{\tiny $_{lb}$}‚ पूर्व‚म्विरोधिनोऽभावे नानिवृत्तेः क‚थं क्ष‚णिक‚त्व‚मिति ।
	{\color{gray}{\rmlatinfont\textsuperscript{§~\theparCount}}}
	\pend% ending standard par
      ‚{\tiny $_{lb}$}‚

	  
	  \pstart \leavevmode% starting standard par
	त‚द‚युक्तं । य‚तो विरोध्य‚पेक्ष‚स्व‚भाव‚त्वं य‚दि व‚स्तुनो न पूर्व‚म‚पि त‚दास्य पूर्वो‚{\tiny $_{lb}$}‚त्त‚र‚रूप‚योर्भेदाद् नित्य‚त्व‚मेव । अथ पूर्व‚म‚पि स स्व‚भावोस्ति त‚दा पूर्व‚म‚प्य‚स्य निवृत्तिः‚{\tiny $_{lb}$}‚ स्यात् । अथ त‚दा विरोध्य‚भावान्न निव‚र्त्त‚ते । क‚थ‚न्त‚र्हि विरोध्य‚पेक्ष‚स्व‚भा‚{\tiny $_{lb}$}‚व‚त्वं । स‚त्येव विरोधिनि । विरोध्य‚पेक्ष‚स्व‚भाव‚{\tiny $_{५}$}‚त्व‚स्य भावान्नान्य‚दा । य‚दि‚{\tiny $_{lb}$}‚ विरोधी व‚स्तुनो नोप‚कार‚कः क‚थ‚न्त‚न्त‚द‚पेक्ष‚ते । उप‚कारे वा विरोधिस‚न्निधाने‚{\tiny $_{lb}$}‚प्य‚प‚र‚स्य भाव‚स्योत्प‚त्तिरिति पूर्व‚को भाव‚स्त‚द‚व‚स्थो दृश्येत । विरोधे स‚न्निधाना‚{\tiny $_{lb}$}‚भावेनानिवृत्तेः । अथ निव‚र्त्त‚ते । न त‚र्हि विरोध्य‚पेक्ष‚या भाव‚स्य निवृत्तिः [।] य‚दि‚{\tiny $_{lb}$}‚ च न भाव‚म्विनाश‚य‚ति क‚थ‚म्विरोधी । न च क्ष‚णिक‚वादिनां विरोधिस‚न्निधाने‚{\tiny $_{lb}$}‚ स‚त्ता‚{\tiny $_{६}$}‚ नो निव‚र्त्त‚ते । किन्त‚र्हि नोत्प‚द्य‚ते । त‚था हि निरोध‚मुप‚ग‚च्छ‚न्नेव घ‚टो‚{\tiny $_{lb}$}‚ मुद्ग‚रादिस‚ह‚कार्य‚पेक्षः क‚पाल‚ज‚न‚क‚त्वेन स‚दृश‚क्ष‚णानार‚म्भ‚को भ‚व‚तीति स‚न्तानानुत्प‚{\tiny $_{lb}$}‚त्तिर्न तु विरोधिन‚म‚पेक्ष्य प्र‚वाहो निव‚र्त्त‚ते । य‚त‚श्च पूर्वंस‚न्तानेनोत्पित्सोर्भाव‚स्य‚{\tiny $_{lb}$}‚ विरोधिस‚न्निधाने स‚न्तानानुत्प‚त्तिर‚त एव म‚न्द‚म‚तीनां स‚हेतुक‚नाशाध्य‚व‚सायो‚{\tiny $_{lb}$}‚ \leavevmode\ledsidenote{\textenglish{134a/PSVTa}} मुद्ग‚रादौ च विरो‚{\tiny $_{७}$}‚धित्वाव‚साय इति स‚र्वं सुस्थं ॥ त‚स्मान्निर‚पेक्ष‚त्वादेव य‚त्र नाम‚{\tiny $_{lb}$}‚ क्व‚चिद् भ‚व‚द्दृष्टो विनाश‚स्त‚त्रैष स्व‚भाव‚त एव भ‚व‚ति ।
	{\color{gray}{\rmlatinfont\textsuperscript{§~\theparCount}}}
	\pend% ending standard par
      ‚{\tiny $_{lb}$}‚

	  
	  \pstart \leavevmode% starting standard par
	न‚न्व‚हेतुकेपि नाशे य‚दैव घ‚टादेर्नाशः प्र‚तीय‚ते त‚दैवाहेतुकः स्यान्न पूर्व‚म‚प्र‚तीते‚{\tiny $_{lb}$}‚र‚र्थैक‚क्ष‚ण‚स्थायित्वेन घ‚टादेरुत्प‚त्तेः पूर्व‚म‚पि नाशः [।] न‚नु य‚थैक‚क्ष‚ण‚स्थायित्वे‚{\tiny $_{lb}$}‚ नोत्प‚त्तिः स्व‚हेतुभ्य‚स्त‚था ऽनेक‚क्ष‚ण‚स्थायित्वेनाप्यु‚{\tiny $_{१}$}‚त्प‚त्तिः स्यात् । विचित्र‚श‚क्त‚यो‚{\tiny $_{lb}$}‚ हि साम‚ग्र्यो दृश्य‚न्ते । त‚त्र काचित् स्याद‚पि याऽन‚श्व‚रात्मानं ज‚न‚येदित्याह ।
	{\color{gray}{\rmlatinfont\textsuperscript{§~\theparCount}}}
	\pend% ending standard par
      ‚{\tiny $_{lb}$}‚‚{\tiny $_{lb}$}‚\textsuperscript{\textenglish{369/s}}

	  
	  \pstart \leavevmode% starting standard par
	\textbf{अस्मादि}त्यादि । अस्माच्च स्व‚भाव‚मात्र‚भावाद\textbf{न्य‚त्रापि} देशादिव्य‚व‚धानेना‚{\tiny $_{lb}$}‚\textbf{दृष्टे} । त‚था दृष्टे [।] विरोधि स‚न्निधानात् पूर्व‚म‚पि \textbf{स्व‚भाव‚त} एव विनाशो भ‚व‚ति ।
	{\color{gray}{\rmlatinfont\textsuperscript{§~\theparCount}}}
	\pend% ending standard par
      ‚{\tiny $_{lb}$}‚

	  
	  \pstart \leavevmode% starting standard par
	एव‚म्म‚न्य‚ते । येषान्ताव‚द्विनाशो दृश्य‚ते तेषां य‚दि न प्र‚तिक्ष‚णं विनाशः स्यात्‚{\tiny $_{lb}$}‚ त‚दा विना‚{\tiny $_{२}$}‚श‚प्र‚तीतिरेव न स्यात् । त‚था हि य‚दि द्वितीये क्ष‚णे भाव‚स्य स्थिति‚{\tiny $_{lb}$}‚स्त‚दासौ स‚र्व‚दैव तिष्ठेत् । द्वितीयेपि क्ष‚णे क्ष‚ण‚द्व‚य‚स्थायी स्व‚भाव‚त्वात् । त‚दा‚{\tiny $_{lb}$}‚ च तेनाप‚र‚क्ष‚ण‚द्व‚यं स्थात‚व्यं । अप‚र‚स्मिन्न‚पि क्ष‚णे क्ष‚ण‚द्व‚य‚स्थायिस्व‚भाव‚त्वा‚{\tiny $_{lb}$}‚द‚प‚र‚स्मिन् क्ष‚णेऽव‚स्थानं स्यादेव‚मुत्त‚रोत्त‚रे क्ष‚णे द्र‚ष्ट‚व्य‚मित्यासंसार‚म्भाव‚स्य‚{\tiny $_{lb}$}‚ स्थितिः स्यात् ।
	{\color{gray}{\rmlatinfont\textsuperscript{§~\theparCount}}}
	\pend% ending standard par
      ‚{\tiny $_{lb}$}‚

	  
	  \pstart \leavevmode% starting standard par
	अथ प्र‚थ‚मे क्ष‚णे भाव‚{\tiny $_{३}$}‚स्य क्ष‚ण‚द्व‚य‚स्थायी स्व‚भावो द्वितीये क्ष‚ण एक‚क्ष‚ण‚स्थायी ।‚{\tiny $_{lb}$}‚ त‚थापि त‚योः स्व‚भाव‚योर्भेदात् क्ष‚णिक‚त्वं स्यात् । न त्वेव‚म‚पि प्र‚थ‚मे क्ष‚णे भाव‚स्या‚{\tiny $_{lb}$}‚नेक‚क्ष‚णाव‚स्थायिस्व‚भावाद‚क्ष‚णिक‚त्वं स्यान्नासंसारं स्थितिप्र‚स‚ङ्गादित्युक्त‚त्वात् ।
	{\color{gray}{\rmlatinfont\textsuperscript{§~\theparCount}}}
	\pend% ending standard par
      ‚{\tiny $_{lb}$}‚

	  
	  \pstart \leavevmode% starting standard par
	न‚नु द्वितीयेपि क्ष‚णे भाव‚स्योत्त‚र‚क्ष‚णान‚व‚स्थानेपि पूर्व‚क्ष‚ण‚स्थायी रूपाभेदेन‚{\tiny $_{lb}$}‚ पूर्व‚म‚पि स्थानात् क्ष‚ण‚द्व‚य‚{\tiny $_{४}$}‚स्थायित्व‚म‚स्त्येवेति चेत् [।]
	{\color{gray}{\rmlatinfont\textsuperscript{§~\theparCount}}}
	\pend% ending standard par
      ‚{\tiny $_{lb}$}‚

	  
	  \pstart \leavevmode% starting standard par
	न । पूर्व‚म्भाव‚स्य ह्येष स्व‚भावो य‚देक‚स्मिन् क्ष‚णे तिष्ठ‚त्य‚न्य‚स्मिन् क्ष‚णे स्था‚{\tiny $_{lb}$}‚स्य‚ति । स एव चेत्स्व‚भावो द्वितीये क्ष‚णे त‚दाप्येक‚स्मिन् क्ष‚णे तिष्ठ‚त्य‚न्य‚स्मिन्‚{\tiny $_{lb}$}‚ क्ष‚णे द्व‚यं स्थास्य‚तीत्येव‚मुत्त‚रोत्त‚र‚क्ष‚णेपि द्र‚ष्ट‚व्य‚मिति न विनाशो भाव‚स्य स्यात् ।‚{\tiny $_{lb}$}‚ दृश्य‚ते च [।] त‚स्माद् विनाश‚प्र‚तीत्य‚न्य‚थानुप‚प‚त्त्या प्र‚तिक्ष‚ण‚विनाशानुमानं ।
	{\color{gray}{\rmlatinfont\textsuperscript{§~\theparCount}}}
	\pend% ending standard par
      ‚{\tiny $_{lb}$}‚

	  
	  \pstart \leavevmode% starting standard par
	अदृष्टेषु त‚र्हि कृत‚केषु क‚थं प्र‚{\tiny $_{५}$}‚तिक्ष‚ण‚विनाशित्वानुमानं विनाश‚स्यैवाद‚र्श‚नात् ।
	{\color{gray}{\rmlatinfont\textsuperscript{§~\theparCount}}}
	\pend% ending standard par
      ‚{\tiny $_{lb}$}‚

	  
	  \pstart \leavevmode% starting standard par
	नैष दोषो य‚स्मात् तेषाम‚पि प्र‚थ‚मे क्ष‚णे य एव स्व‚भावः स एव चेत् द्वितीयादि‚{\tiny $_{lb}$}‚क्ष‚णे त‚दाऽभूत्वा भ‚व‚न‚मेव प्र‚थ‚म‚क्ष‚ण‚व‚त् । अथ प्र‚थ‚मे क्ष‚णे त‚स्य ज‚न्मैव न स्थिति‚{\tiny $_{lb}$}‚र्द्वितीये च क्ष‚णे स्थितिरेव न ज‚न्म । एव‚म‚पि क्ष‚णिक‚त्वं स्यात् । ज‚न्म‚ज‚न्मिनोः‚{\tiny $_{lb}$}‚ स्थितिस्थितिम‚तोश्चाभेदात् । न च द्वि‚{\tiny $_{६}$}‚तीये क्ष‚णे ज‚न्म विना स्थितिर्युक्ता । ज‚न्म‚{\tiny $_{lb}$}‚ चेन्न त‚दास्थितिस्त‚स्या द्वितीयादिक्ष‚ण‚भावित्वात् । द्वितीयादौ क्ष‚णेप्येव‚मिति स‚र्व‚{\tiny $_{lb}$}‚त्रोत्प‚त्तिरेव न स्थितिरिति क्ष‚णिक‚त्वं । किं च प‚र‚स्प‚र‚भिन्न‚स्याप‚राप‚र‚काल‚{\tiny $_{lb}$}‚स‚म्ब‚न्धित्व‚स्य कृत‚काद् भावाद‚भिन्न‚त्वात् क्ष‚णिक‚त्व‚मेव ।
	{\color{gray}{\rmlatinfont\textsuperscript{§~\theparCount}}}
	\pend% ending standard par
      ‚{\tiny $_{lb}$}‚

	  
	  \pstart \leavevmode% starting standard par
	\textbf{सोय‚मि}त्यादिना का रि का र्थ‚माह । \textbf{अय‚मि}ति विनाशः । \textbf{क्व‚चिद्} घ‚टादौ‚{\tiny $_{lb}$}‚ मुद्ग‚रादिस‚न्निधाने‚{\tiny $_{७}$}‚ \textbf{त‚थान्य‚त्रा}पीति मुद्ग‚र‚स‚न्निधानात् पूर्व‚म‚पि । एव‚न्देशादि- \leavevmode\ledsidenote{\textenglish{134b/PSVTa}}‚{\tiny $_{lb}$}‚ ‚{\tiny $_{lb}$}‚ \leavevmode\ledsidenote{\textenglish{370/s}}व्य‚व‚धानेपि । \href{http://sarit.indology.info/?cref=pv.3.195-3.196}{१९८-९९}
	{\color{gray}{\rmlatinfont\textsuperscript{§~\theparCount}}}
	\pend% ending standard par
      ‚{\tiny $_{lb}$}‚

	  
	  \pstart \leavevmode% starting standard par
	य‚त्पूर्व‚मुप‚न्य‚स्त‚न्त‚त्र द्वौ व‚स्तुसाध‚नाविति त‚दुप‚संह‚र‚न्नाह । \textbf{त‚त} इति त‚स्माद्‚{\tiny $_{lb}$}‚ \textbf{या काचिद् भाव‚विष‚या} कार्य‚स्व‚भावाभ्यां लिङ्गाभ्यां \textbf{द्विधैवानुमितिः} । क‚स्मात्त‚योः‚{\tiny $_{lb}$}‚ कार्य‚स्व‚भाव‚योरेव \textbf{स्व‚साध्ये स‚म्ब‚न्ध‚निय‚मात् । कार्यं लिङ्गं} य‚स्या अनुमितेः । एवं‚{\tiny $_{lb}$}‚ \textbf{स्व‚भावो लिङ्गं} य‚स्या इति विग्र‚हः‚{\tiny $_{१}$}‚ । \textbf{य‚थास्वं व्यापिनि साध्य} इति कार्य‚स्य कार‚णं‚{\tiny $_{lb}$}‚ व्याप‚कं साध्यं [।] स्व‚भाव‚स्यापि स्व‚भावो व्याप‚कः साध्य‚स्त‚स्मिन् साध्ये \textbf{लिङ्गिनि‚{\tiny $_{lb}$}‚ त‚योरेव} कार्य‚स्व‚भाव‚योर्लिङ्ग‚योः \textbf{प्र‚तिब‚न्धात्} ॥ \href{http://sarit.indology.info/?cref=pv.3.195-3.196}{१९८-९९}
	{\color{gray}{\rmlatinfont\textsuperscript{§~\theparCount}}}
	\pend% ending standard par
      ‚{\tiny $_{lb}$}‚

	  
	  \pstart \leavevmode% starting standard par
	अनुप‚ल‚ब्धिम‚धिकृत्याह । \textbf{प्र‚वृत्ते}रित्यादि । प्र‚वृत्तेः । स‚न्निश्च‚य‚श‚ब्द‚व्य‚व‚हार‚{\tiny $_{lb}$}‚ल‚क्ष‚ण‚योः \textbf{बुद्धिपूर्व‚त्वात्} प्र‚माण‚पूर्व‚त्वात् । \textbf{त‚द्भावानुप‚ल‚म्भ‚ने} त‚स्य प्र‚वृत्तिविष‚य‚स्य‚{\tiny $_{lb}$}‚ भाव‚{\tiny $_{२}$}‚स्य प्र‚त्य‚क्षानुमानाभ्याम‚नुप‚ल‚म्भ‚ने प्रेक्षाव‚ता \textbf{प्र‚व‚र्त्तित‚व्यं नेतीय‚ता} लेशेनादृश्य‚{\tiny $_{lb}$}‚विष‚याया अप्य\textbf{नुप‚ल‚ब्धेरुक्ता प्र‚माण‚ता ॥ तृतीय‚स्तु हेतुर}नुप‚ल\textbf{ब्धिर्ग‚म‚क इत्युच्य‚त}‚{\tiny $_{lb}$}‚ इति स‚म्ब‚न्धः । \textbf{अविशेषे}णेति दृश्य‚विष‚येत्य‚मुम्विशेष‚न्त्य‚क्त्वा सामान्येन \textbf{क्व‚चिद‚र्थे}‚{\tiny $_{lb}$}‚ साध्ये । \textbf{स‚न्निश्च‚ये}त्यादिना त‚मेवार्थ‚न्द‚र्श‚य‚ति । प्र‚माण‚पृष्ठ‚भावी स‚दिति‚{\tiny $_{३}$}‚ विक‚ल्पः‚{\tiny $_{lb}$}‚ स‚न्निश्च‚यः । त‚त्पृष्ठ‚भाव्येव स‚दिति \textbf{ध्व‚निः} स‚च्छ‚ब्दः । त‚थैव स‚दित्य‚नुष्ठानं‚{\tiny $_{lb}$}‚ \textbf{स‚द्व्य‚व‚हारः} । तेषां \textbf{प्र‚तिषेधे हि स‚र्वैव} दृश्य‚विष‚याऽदृश्य‚विष‚या च ।
	{\color{gray}{\rmlatinfont\textsuperscript{§~\theparCount}}}
	\pend% ending standard par
      ‚{\tiny $_{lb}$}‚

	  
	  \pstart \leavevmode% starting standard par
	न‚नु का रि का यां प्र‚वृत्तिरित्युक्त‚न्त‚त्क‚थं वृत्तौ \textbf{स‚न्निश्च‚ये}त्यादि व्याख्याय‚त‚{\tiny $_{lb}$}‚ इत्याह । \textbf{स‚न्निश्च‚या}द्धीत्यादि । य‚स्मात् स‚तां विद्य‚मानानां निश्च‚या\textbf{च्छ‚ब्द‚व्य‚व‚हाराः‚{\tiny $_{lb}$}‚ ‚{\tiny $_{lb}$}‚ \leavevmode\ledsidenote{\textenglish{371/s}}प्र‚व‚र्त्त‚न्ते त‚{\tiny $_{४}$}‚स्मात्ते} स‚न्निश्च‚य‚श‚ब्द‚व्य‚व‚हारास्त्र‚योपि प्र‚वृत्त्य‚ङ्ग‚त्वात् पुरुष‚प्र‚वृत्तेर्नि‚{\tiny $_{lb}$}‚मित्त‚त्वात् \textbf{प्र‚वृत्तिरित्युक्तः} । त‚स्मान्न सू त्र वृ त्त्योर्व्याघात इति ।
	{\color{gray}{\rmlatinfont\textsuperscript{§~\theparCount}}}
	\pend% ending standard par
      ‚{\tiny $_{lb}$}‚

	  
	  \pstart \leavevmode% starting standard par
	किम्पुनः कार‚ण‚मुप‚ल‚म्भ‚निवृत्तौ स‚द्व्य‚व‚हारो निव‚र्त्त‚ते । \textbf{त‚था ह्य‚नुप‚ल‚ब्धि‚{\tiny $_{lb}$}‚रेव} द्विप्र‚काराप्य‚विशेषेणा\textbf{स‚त्त्व‚मित्युक्तं} प्राक् । \textbf{त‚च्चा}स‚त्त्व‚म्विप्र‚कृष्टायाम‚नुप‚ल‚ब्धौ‚{\tiny $_{lb}$}‚ \textbf{प्र‚तिप‚त्तृव‚शादु}क्तं । प्र‚तिप‚त्ता ह्य‚नुप‚{\tiny $_{५}$}‚ल‚भ्य‚मान‚न्नास्तीत्य‚ध्य‚व‚स्य‚ति । \textbf{न व‚स्तुव‚{\tiny $_{lb}$}‚शात्} । त‚था हि विप्र‚कृष्टेर्थे स‚त्त्व‚म‚स‚त्त्व‚ञ्च स‚न्\textbf{दिग्धं} । त‚स्मात् \textbf{ताव‚द्धि} स‚{\tiny $_{lb}$}‚ विप्र‚कृष्टो \textbf{भावोस्य} प्र‚तिप‚त्तु\textbf{र्नास्ति याव‚द‚त्राप्र‚तिप‚त्तिः} । क‚स्मात् [।] \textbf{स‚तापि तेन} विप्र‚{\tiny $_{lb}$}‚कृष्टेनार्थेनानुप‚ल‚ब्धेन \textbf{त‚द‚र्थाक‚र‚णात्} । त‚स्य पुंसोर्थाक‚र‚णात् । स‚न्न‚प्य‚स‚त्क‚ल्पः ।‚{\tiny $_{lb}$}‚ \textbf{व‚स्तुत‚स्त्व‚नुप‚ल‚भ्य‚मानो न स‚न्नास‚न्} । क‚स्मात् [।] \textbf{स‚ताम‚पि} क‚दाचि‚{\tiny $_{६}$}‚त्‚{\tiny $_{lb}$}‚ \textbf{स्व‚भावादिविप्र‚क‚र्षाद‚नुप‚ल‚म्भान्ना}स‚त्तानिश्च‚यः । \textbf{क्वापि} स‚त्तानिश्च‚य‚स्त‚स्यास्यानु‚{\tiny $_{lb}$}‚प‚ल‚म्भ‚स्या\textbf{स‚त्स्व‚पि तुल्य‚त्वात् । त‚दि}ति त‚स्मात् । \textbf{एत‚त् स‚त्त्व}मिति स‚म्ब‚न्धः ।‚{\tiny $_{lb}$}‚ किम्भूत‚म‚नु\textbf{प‚ल‚ब्धिल‚क्ष‚ण‚न्}देशादिविप्र‚कृष्टं \textbf{प्र‚तिप‚त्तुः प्र‚माणाभावात्} । प्र‚त्य‚क्षानु‚{\tiny $_{lb}$}‚मानाभावा\textbf{न्निवृत्त‚न्न} व‚स्तुव‚शात् । किं क‚रोति [।] \textbf{स्व‚निमित्तान् श‚ब्द‚{\tiny $_{७}$}‚व्य‚व‚हा- \leavevmode\ledsidenote{\textenglish{135a/PSVTa}}‚{\tiny $_{lb}$}‚ रान् निव‚र्त्त‚य‚ति} ।
	{\color{gray}{\rmlatinfont\textsuperscript{§~\theparCount}}}
	\pend% ending standard par
      ‚{\tiny $_{lb}$}‚

	  
	  \pstart \leavevmode% starting standard par
	\textbf{उप‚ल‚ब्धिल‚क्ष‚ण}मित्य‚न्ये प‚ठ‚न्ति । उप‚ल‚ब्धिरेव स‚त्त्व‚मुप‚चारात् त‚थाभूतं‚{\tiny $_{lb}$}‚ स‚त्त्वं निवृत्त‚मित्य‚र्थः ।
	{\color{gray}{\rmlatinfont\textsuperscript{§~\theparCount}}}
	\pend% ending standard par
      ‚{\tiny $_{lb}$}‚

	  
	  \pstart \leavevmode% starting standard par
	येनैव\textbf{न्तेन} कार‚णेन \textbf{यापीय‚म‚नुप‚ल‚ब्धिः} [।] केषाम् [।] \textbf{अनुप‚ल‚ब्धिल‚क्ष‚ण‚{\tiny $_{lb}$}‚प्राप्तानां} । या \textbf{व‚स्तुतोप्य‚स‚त्त्व‚रूपा} [।] अपिश‚ब्दात् प्र‚तिप‚त्तृव‚शाद‚पि । किं‚{\tiny $_{lb}$}‚ कार‚ण‚म् [।] \textbf{अप्र‚वृत्तियोग्य‚त्वाद}स‚द्व्य‚व‚हार‚योग्य‚त्वात् । त‚स्या अप्य‚नुप‚ल‚ब्धेरेत‚त्स‚{\tiny $_{lb}$}‚द्व्य‚व‚हार‚प्र‚तिषेध‚क‚त्वेन \textbf{तुल्यं प्रामाण्यं । अत्र विष‚ये} स‚द्व्य‚व‚हार‚प्र‚तिषेधे ।
	{\color{gray}{\rmlatinfont\textsuperscript{§~\theparCount}}}
	\pend% ending standard par
      ‚{\tiny $_{lb}$}‚

	  
	  \pstart \leavevmode% starting standard par
	येय‚म‚नुप‚ल‚ब्धिल‚क्ष‚ण‚प्राप्तानु-प‚ल‚ब्धि\textbf{र‚स‚न्निश्च‚य‚फ‚लापि सा} । अस‚न्निश्च‚यः‚{\tiny $_{lb}$}‚ ‚{\tiny $_{lb}$}‚ \leavevmode\ledsidenote{\textenglish{372/s}}फ‚लं य‚स्या इति विग्र‚हः । \href{http://sarit.indology.info/?cref=pv.3.197-3.198}{२००-२०१}
	{\color{gray}{\rmlatinfont\textsuperscript{§~\theparCount}}}
	\pend% ending standard par
      ‚{\tiny $_{lb}$}‚

	  
	  \pstart \leavevmode% starting standard par
	\textbf{अस‚त्ताया}मित्यादिना व्याच‚ष्टे । अस‚त्तायाम‚प्य‚निश्च‚येपि । य‚स्मान्न ह्य‚स्ति‚{\tiny $_{lb}$}‚ स‚म्भ‚वो \textbf{य‚दुप‚ल‚ब्धियोग्यो} भावः \textbf{स‚क‚लेष्व‚न्येषूप}ल‚म्भ\textbf{कार‚णेषु स‚न्} विद्य‚मानो \textbf{नोप‚{\tiny $_{lb}$}‚ल‚भ्येत‚{\tiny $_{२}$}‚ । न पुनः पूर्वा} विप्र‚कृष्ट‚विष‚यानुप‚ल‚ब्धिर\textbf{स‚त्तासाध‚नी} । त‚था हि [।]‚{\tiny $_{lb}$}‚ प्र‚त्य‚क्षानुमानाग‚म‚निवृत्तिल‚क्ष‚णैवानुप‚ल‚ब्धिः । त‚त्र \textbf{शास्त्र‚स्याधिका}रोस्मिन् प्र‚क‚र‚णे‚{\tiny $_{lb}$}‚ त‚त्रा\textbf{स‚म्ब‚द्धा} अनान्त‚रीय‚का \textbf{ब‚ह‚वोर्थाः} स्व‚भावादिविप्र‚क‚र्षिणः शास्त्रे नाधिक्रिय‚न्त‚{\tiny $_{lb}$}‚ इति याव‚त् । प्र‚त्य‚क्ष‚स्यापि न ते विष‚या इत्याहा\textbf{तीन्द्रियाः} । नाप्य‚नुमान‚स्य य‚स्मा‚{\tiny $_{lb}$}‚द‚लिङ्गाः ।‚{\tiny $_{३}$}‚ नैषां लिङ्ग‚म‚स्तीत्\textbf{य‚लिङ्गाः} । तेषाम‚तीन्द्रियाणाम‚र्थानां प्र‚माण‚त्र‚या‚{\tiny $_{lb}$}‚निवृत्तिल‚क्ष‚णाया \textbf{अनुप‚ल‚ब्धितः} क‚थ‚म‚भावः [।] नैव । \href{http://sarit.indology.info/?cref=pv.3.197-3.198}{२००-१}
	{\color{gray}{\rmlatinfont\textsuperscript{§~\theparCount}}}
	\pend% ending standard par
      ‚{\tiny $_{lb}$}‚

	  
	  \pstart \leavevmode% starting standard par
	\textbf{सोय}मिति वादी । \textbf{स‚र्वार्थाना}मिति देश‚काल‚स्व‚भाव‚विप्र‚क‚र्षिणां \textbf{प्र‚माण‚त्र‚य‚{\tiny $_{lb}$}‚निवृत्त्येति} प्र‚त्य‚क्षानुमानाग‚म‚निवृत्त्या । \textbf{त‚स्येति शास्त्र‚स्य क्व‚चित्} पुरुषार्थानुप‚{\tiny $_{lb}$}‚योगिन्य‚र्थेऽ\textbf{न‚धिकारे} विनियोगाभावे \textbf{प्र‚{\tiny $_{४}$}‚वृत्तेः । य‚स्माच्छास्त्रं ही}त्यादि । \textbf{अन्य‚थे}ति‚{\tiny $_{lb}$}‚ \textbf{पुरुषार्थानुप‚योगिन‚म‚र्थ‚माश्रित्य शास्त्र‚प्र‚वृत्तौ । अब‚द्ध‚प्र‚लाप‚स्या}स‚म्ब‚द्धाभिधायिनः‚{\tiny $_{lb}$}‚ शास्त्र‚स्या\textbf{प्रामाण्यात्} ।
	{\color{gray}{\rmlatinfont\textsuperscript{§~\theparCount}}}
	\pend% ending standard par
      ‚{\tiny $_{lb}$}‚

	  
	  \pstart \leavevmode% starting standard par
	स्यादेत‚त् [।] स‚र्व एवार्थाः पुरुषार्थोप‚योगिन इत्य‚त्राह । \textbf{त‚त्रे}त्यादि । \textbf{त‚त्र‚{\tiny $_{lb}$}‚ प्र‚क‚र‚ण} इति पुरुषार्थ‚चिन्ताप्र‚स्तावे । \textbf{प्र‚त्यात्म‚निय‚ता} इति प्र‚तिपुरुष‚निय‚ताः ।‚{\tiny $_{lb}$}‚ एतेन ‚{\tiny $_{५}$}‚पुरुषाणामान‚न्त्यादान‚न्त्यं चेतोवृत्तीनामाह । \textbf{अनिय‚ता}न्निमित्ताद् भ‚वितुं‚{\tiny $_{lb}$}‚ शीलं यासामिति विग्र‚हः । अनेनैक‚स्मिन्न‚पि पुंसि निमित्त‚भेदाद् ब‚हुत्वं । एवं‚{\tiny $_{lb}$}‚भूताश्चेतोवृत्त‚यो \textbf{नाव‚श्यं} साक‚ल्येन प्र‚तिप‚दं \textbf{निर्देश्या} अश‚क्य‚त्वात् । काल‚देश‚{\tiny $_{lb}$}‚व्य‚व‚हिता वा पुरुषार्थानुप‚योगिनो \textbf{द्र‚व्य‚विशेषा नाव‚श्यं निर्देश्या}स्त‚तो \textbf{न त‚च्छा‚{\tiny $_{lb}$}‚‚{\tiny $_{lb}$}‚ \leavevmode\ledsidenote{\textenglish{373/s}}स्त्र‚म्विष‚यीक‚रो‚{\tiny $_{६}$}‚ति} । ताश्च चेतोवृत्त‚य‚स्ते च विशेषास्तानिति पुलिंगेनोक्ताः ।‚{\tiny $_{lb}$}‚ पुमान् स्त्रिये ति\edtext{}{\edlabel{pvsvt_373-1}\label{pvsvt_373-1}\lemma{ति}\Bfootnote{\href{http://sarit.indology.info/?cref=p\%C4\%81.1.2.67}{ Pāṇini 1. 2. 67 }}} पुंसः शेषं कृत्वा ॥
	{\color{gray}{\rmlatinfont\textsuperscript{§~\theparCount}}}
	\pend% ending standard par
      ‚{\tiny $_{lb}$}‚

	  
	  \pstart \leavevmode% starting standard par
	नापि प्र‚त्य‚क्ष‚स्य विष‚या इत्याह । \textbf{न चे}त्यादि । \textbf{त‚था विप्र‚कृष्टेष्वि}ति देशादि‚{\tiny $_{lb}$}‚विप्र‚कृष्टेषु \textbf{स्व‚साम‚र्थ्योप‚धानात्} । स्व‚रूप‚स‚न्निधानात् । \textbf{ज्ञानोत्पाद‚न‚श‚क्तिर्नास्ति} ।‚{\tiny $_{lb}$}‚ एतेना\textbf{तीन्द्रिया} इत्येत‚द् व्याख्यातं ॥
	{\color{gray}{\rmlatinfont\textsuperscript{§~\theparCount}}}
	\pend% ending standard par
      ‚{\tiny $_{lb}$}‚

	  
	  \pstart \leavevmode% starting standard par
	अनुमान‚स्यापि न ते विष‚या इत्याह ।‚{\tiny $_{७}$}‚ \textbf{स चाव‚श्य}मित्यादि । \textbf{एषा}मिति \leavevmode\ledsidenote{\textenglish{135b/PSVTa}}‚{\tiny $_{lb}$}‚ देशादिव्य‚व‚हितानां । \textbf{येनेति} कार्योप‚ल‚म्भेन । \textbf{न च त} इति विप्र‚क‚र्षिणः ।‚{\tiny $_{lb}$}‚ \textbf{स‚र्वे}त्य‚दृश्यानुप‚ल‚ब्धिर‚पि । \textbf{निवृत्तिनि}श्च‚य‚स्याभाव‚निश्च‚य‚स्य । य‚त एव\textbf{न्त‚दिति}‚{\tiny $_{lb}$}‚ त‚स्मात् । \textbf{इय}मित्य‚दृश्यानुप‚ल‚ब्धिस्स‚द‚स‚न्निश्च‚य‚फ‚ला नेति [।] \textbf{स‚न्निश्च‚य‚फ‚ला न}‚{\tiny $_{lb}$}‚ भ‚व‚ति स‚द्व्य‚व‚हार‚निमित्ता । न चाप्य‚स‚न्निश्च‚य‚फ‚ला स‚न्देहात् । \textbf{इति} हेतो‚{\tiny $_{१}$}‚\href{http://sarit.indology.info/?cref=}{ः}‚{\tiny $_{lb}$}‚ \textbf{स्याद्वा}नुप‚ल‚ब्धे\textbf{र‚प्र‚माण‚ता । व्य‚व‚साय‚फ‚ल‚त्वादि}ति निश्च‚य‚फ‚ल‚त्वात् \textbf{प्र‚माणानां} ।‚{\tiny $_{lb}$}‚ प्र‚त्य‚क्ष‚म‚पि हि प्र‚माणं स‚र्वाकार‚ग्र‚हेपि येष्वाकारेषु निश्च‚य‚माव‚ह‚ति तेष्वेव ।
	{\color{gray}{\rmlatinfont\textsuperscript{§~\theparCount}}}
	\pend% ending standard par
      ‚{\tiny $_{lb}$}‚

	  
	  \pstart \leavevmode% starting standard par
	न‚नु प्र‚वृत्तिनिषेध‚प्र‚माणं स्यादित्याह । \textbf{न ही}त्यादि । \textbf{इय}मित्य‚नुप‚ल‚ब्धिः ।‚{\tiny $_{lb}$}‚ \textbf{निःशंक‚प‚रिच्छेद}मिति निःशंकः प‚रिच्छेदो य‚स्य चेत‚स इति विग्र‚हः । संश‚ये स‚ति‚{\tiny $_{lb}$}‚ न प्र‚व‚र्त्तित‚व्य‚म‚{\tiny $_{२}$}‚ व‚श्य‚मित्येवं निश्चितं चेतो न क‚रोतित्य‚र्थः । \textbf{संश‚याद‚पि क्व‚चित्}‚{\tiny $_{lb}$}‚ कृषीव‚लादेर्लोक‚स्य \textbf{प्र‚वृत्तेः} ॥ क‚थ‚न्त‚र्ह्य‚प्र‚वृत्तिफ‚ल‚त्वेनास्याः प्रामाण्य‚मुक्त‚मित्य‚त‚{\tiny $_{lb}$}‚ आह । \textbf{त‚थात्वे-त‚दि}त्यादि । त‚था तेन रूपेणै\textbf{त‚द‚प्र}वृत्तिकार‚ण‚म‚नुप‚ल‚म्भाख्यं \textbf{निर‚व‚द्यं}‚{\tiny $_{lb}$}‚ निर्दोषं य‚दि \textbf{निश्च‚य‚पूर्वं व्य‚व‚ह‚रेत्} क‚श्चित् । प्र‚माण‚पूर्वं स‚द्व्य‚व‚हारादि प्र‚व‚र्त‚येत् ।‚{\tiny $_{lb}$}‚ इ\textbf{त्य}नेन द्वारेण से‚{\tiny $_{३}$}‚य‚म‚दृश्यानुप‚ल‚ब्धिर\textbf{प्र‚वृत्तिफ‚ला प्रोक्ता} निश्चित‚स‚द्व्य‚व‚हारादि‚{\tiny $_{lb}$}‚प्र‚तिषेध‚फ‚ला प्रोक्ता [।] स‚न्दिग्ध‚स्तु स‚द्व्य‚व‚हारादिर्न निषिध्य‚त इति पुरुष‚स्य‚{\tiny $_{lb}$}‚ प्र‚वृत्तिर्भ‚व‚त्य‚पि ।
	{\color{gray}{\rmlatinfont\textsuperscript{§~\theparCount}}}
	\pend% ending standard par
      ‚{\tiny $_{lb}$}‚

	  
	  \pstart \leavevmode% starting standard par
	\textbf{लिङ्गातिश‚य‚भाविनी}ति लिङ्ग‚म‚नुप‚ल‚ब्धिस्त‚स्या अतिश‚यो विशेष उप‚ल‚ब्धि‚{\tiny $_{lb}$}‚ल‚क्ष‚ण‚प्राप्त‚त्व‚न्त‚स्य भाव‚स्स‚द्भाव‚स्स य‚स्याम‚स्ति सा त‚थोक्ता । \textbf{लिङ्ग‚विशेष‚{\tiny $_{lb}$}‚‚{\tiny $_{lb}$}‚ ‚{\tiny $_{lb}$}‚ \leavevmode\ledsidenote{\textenglish{374/s}}व‚ती}त्य‚र्थः । उप‚ल‚ब्धिल‚क्ष‚{\tiny $_{४}$}‚ण‚प्राप्तानुप‚ल‚ब्धिरिति याव‚त् ।
	{\color{gray}{\rmlatinfont\textsuperscript{§~\theparCount}}}
	\pend% ending standard par
      ‚{\tiny $_{lb}$}‚

	  
	  \pstart \leavevmode% starting standard par
	\textbf{अत्रे}त्यादिना व्याच‚ष्टे । \textbf{अत्रे}ति निवृत्तिनिश्च‚ये । \textbf{य‚थोदाहृता प्रागिति} ।‚{\tiny $_{lb}$}‚ \href{http://sarit.indology.info/?cref=pv.3.199-3.200}{२०२-३}
	{\color{gray}{\rmlatinfont\textsuperscript{§~\theparCount}}}
	\pend% ending standard par
      ‚{\tiny $_{lb}$}‚
	  \bigskip
	  \begingroup
	
	    
	    \stanza[\smallbreak]
	  {\normalfontlatin\large ``\qquad}अस‚ज्ज्ञान‚फ‚ला काचिद्धेतुभेद‚व्य‚पेक्ष‚येत्यादिना [।]{\normalfontlatin\large\qquad{}"}\&[\smallbreak]
	  
	  
	  
	  \endgroup
	‚{\tiny $_{lb}$}‚

	  
	  \pstart \leavevmode% starting standard par
	य‚त्पुन‚रुक्त‚म् [।] अप्र‚माण‚म‚नुप‚ल‚ब्धिरिति त‚न्नाविशेषेण बोद्ध‚व्यं किन्तु‚{\tiny $_{lb}$}‚ स्व‚भावेत्यादि । देशादिविप्र‚कृष्टः स्व‚भावः [।] ज्ञाप‚कं लिङ्गं । त‚योर‚ज्ञानं ।
	{\color{gray}{\rmlatinfont\textsuperscript{§~\theparCount}}}
	\pend% ending standard par
      ‚{\tiny $_{lb}$}‚
	  \bigskip
	  \begingroup
	
	    
	    \stanza[\smallbreak]
	  {\normalfontlatin\large ``\qquad}स्व‚भाव‚ज्ञाप‚काज्ज्ञान‚स्यायं न्याय उदाहृतः ।‚{\tiny $_{५}$}‚ \href{http://sarit.indology.info/?cref=pv.3.200}{१ । २०३}{\normalfontlatin\large\qquad{}"}\&[\smallbreak]
	  
	  
	  
	  \endgroup
	‚{\tiny $_{lb}$}‚

	  
	  \pstart \leavevmode% starting standard par
	अस‚त्त्वे साध्ये नास्ति प्रामाण्य‚मिति । स्व‚भावाज्ञानं प्र‚त्य‚क्ष‚निवृत्तिः । ज्ञाप‚{\tiny $_{lb}$}‚काज्ञान‚म‚नुमान‚निवृत्तिः । अदृश्य‚विष‚यायाः प्र‚त्य‚क्षानुमान‚निवृत्तेर‚यं न्याय उदाहृत‚{\tiny $_{lb}$}‚ इति स‚मुदायार्थः ।
	{\color{gray}{\rmlatinfont\textsuperscript{§~\theparCount}}}
	\pend% ending standard par
      ‚{\tiny $_{lb}$}‚

	  
	  \pstart \leavevmode% starting standard par
	\textbf{य‚स्ये}त्यादिना व्याच‚ष्टे । \textbf{य‚स्य क‚स्य‚चित्} पिशाचादेः । स इति स्व‚भावः । \textbf{त‚द‚नु‚{\tiny $_{lb}$}‚प‚ल‚म्भ‚मात्रे}णेति तेन त‚द्विष‚यानुप‚ल‚म्भ‚मात्रेण प्र‚त्य‚क्ष‚निवृत्ति‚{\tiny $_{६}$}‚रूपे\textbf{णास‚न्नाम । य‚थोक्तं‚{\tiny $_{lb}$}‚ प्रागिति} । स‚ताम‚पि क‚दाचिद‚नुप‚ल‚म्भादि त्यादिना । स्व‚भाव्राज्ञान‚म‚नेन व्याख्यातं ।
	{\color{gray}{\rmlatinfont\textsuperscript{§~\theparCount}}}
	\pend% ending standard par
      ‚{\tiny $_{lb}$}‚

	  
	  \pstart \leavevmode% starting standard par
	ज्ञाप‚काज्ञान‚म्व्याख्यातुमाह । \textbf{योपी}त्यादि । \textbf{ज्ञाप‚क}स्येत्य‚स्य विव‚र‚णं लिङ्ग‚{\tiny $_{lb}$}‚स्येति । \textbf{अतीन्द्रियः प्र‚तिक्षिप्य‚तेऽर्थः । य‚था नास्ति विर‚क्तं चेत} इत्यादि । त‚था‚{\tiny $_{lb}$}‚ \leavevmode\ledsidenote{\textenglish{136a/PSVTa}} ज्ञाप‚क‚स्य लिङ्ग‚स्याभावात् स्व‚भाव‚विशेषो वा‚{\tiny $_{७}$}‚ प्र‚तिक्षिप्य‚ते । अत्र विशेषः‚{\tiny $_{lb}$}‚ प्र‚तिक्षिप्य‚ते न धार्मिमात्रं । य‚था \textbf{नास्ति दाने}त्यादि । दानं च हिंसा-विर‚तिश्चे‚{\tiny $_{lb}$}‚ति द्व‚न्द्वः । त‚द्विष‚याश्चेत‚नाः । \textbf{दान}चेत‚नानां \textbf{हिंसाविर‚तिचेत‚नानां} चेत्य‚र्थः ।‚{\tiny $_{lb}$}‚ ‚{\tiny $_{lb}$}‚ \leavevmode\ledsidenote{\textenglish{375/s}}\textbf{अभ्युद‚य‚हेतुता} स्व‚र्गादिफ‚ल‚हेतुना । अत्र चेत‚नानां न स्व‚रूपं प्र‚तिक्षिप्य‚ते तासां‚{\tiny $_{lb}$}‚ प्र‚त्य‚क्ष‚त्वात् । किन्त्व‚भ्युद‚य‚हेतुत्वं स्व‚भावो-विशेषो नास्तीत्युच्य‚ते ।
	{\color{gray}{\rmlatinfont\textsuperscript{§~\theparCount}}}
	\pend% ending standard par
      ‚{\tiny $_{lb}$}‚

	  
	  \pstart \leavevmode% starting standard par
	न‚नु चेत‚नानां प्र‚त्य‚क्ष‚त्वा‚{\tiny $_{१}$}‚द‚भ्युद‚य‚हेतुतापि त‚दात्म‚भूत‚त्वात् । प्र‚त्य‚क्ष‚स्यैवेति‚{\tiny $_{lb}$}‚ कुतो भ्रान्तिरित्यादि ।
	{\color{gray}{\rmlatinfont\textsuperscript{§~\theparCount}}}
	\pend% ending standard par
      ‚{\tiny $_{lb}$}‚

	  
	  \pstart \leavevmode% starting standard par
	प्र‚त्य‚क्षेप्य‚र्थे \textbf{विप‚र्य‚स्तोऽप‚व‚देतापीति} स‚म्ब‚न्धः । क‚थ‚म्विप‚र्य‚स्त इत्याह ।‚{\tiny $_{lb}$}‚ \textbf{अत‚त्फ‚ले}त्यादि । अत‚त्फ‚ला अन‚भ्युद‚य‚फ‚ला ये दृष्टाश्चेत‚ना-विशेषा अव्याकृताः ।‚{\tiny $_{lb}$}‚ तैस्सा\textbf{ध‚र्म्यात्} । साध‚र्म्य‚मेव क‚थ‚मित्याह । \textbf{फ‚ल‚स्यान‚न्त‚र्याभावा}दिति । या अन्या‚{\tiny $_{lb}$}‚ अत‚त्फ‚लाश्चेत‚नाः । याश्च त‚त्फ‚ला । उभ‚य‚त्र त‚त्फ‚{\tiny $_{२}$}‚ल‚स्यान‚न्त‚र्य‚न्न दृश्य‚ते । \textbf{न‚{\tiny $_{lb}$}‚ ताव‚तेति} फ‚ल‚स्यान‚न्त‚र्याद‚र्श‚न‚मात्रेण । \textbf{त‚द‚भावः} फ‚लाभावः । क‚स्माद् [।] \textbf{व्य‚व‚{\tiny $_{lb}$}‚हिताना}मित्यादि । \textbf{हेतोः} स‚काशात् कालान्त‚रेणोत्त‚रोत्त‚राव‚स्थाप‚रिणाम‚ल‚क्ष‚णेन‚{\tiny $_{lb}$}‚ \textbf{व्य‚व‚हितानां फ‚लानान्द‚र्श‚नात् । मूषिक‚स्या}ल‚र्क्क‚स्य \textbf{चोन्म‚त्त‚कुक्कुर‚स्य विष‚विकार‚{\tiny $_{lb}$}‚ इव} स हि न विष‚संचार‚काल एव भ‚व‚ति किन्तु स‚ह‚कारिणः काल‚विशेष‚स्य स‚न्निधौ‚{\tiny $_{lb}$}‚‚{\tiny $_{३}$}‚निष्प‚द्य‚ते । न तु हेत्व‚न्त‚र‚मेव त‚द्व‚त् ॥
	{\color{gray}{\rmlatinfont\textsuperscript{§~\theparCount}}}
	\pend% ending standard par
      ‚{\tiny $_{lb}$}‚

	  
	  \pstart \leavevmode% starting standard par
	स्यादेत‚त् [।] मूषिकादिविष‚विकार‚स्य कालान्त‚रे दृष्ट‚त्वात् स‚द्भावो‚{\tiny $_{lb}$}‚ युक्त एव । दानादिफ‚ल‚न्तु न क‚दाच‚नापि दृष्ट‚मिति क‚थ‚न्त‚स्य स‚द्भाव इत्याह ।‚{\tiny $_{lb}$}‚ त‚दित्यादि । \textbf{त‚द्भाव} इति । त‚योर्दानादिफ‚लातीन्द्रिय‚व‚स्तुनोर्भावे \textbf{विरोधाभा}वात् ।‚{\tiny $_{lb}$}‚ \textbf{अत्र} दानादिफ‚लेऽतीन्द्रिय‚भावे \textbf{वानुप‚ल‚ब्धिमात्र‚म‚प्र‚माणं} ।
	{\color{gray}{\rmlatinfont\textsuperscript{§~\theparCount}}}
	\pend% ending standard par
      ‚{\tiny $_{lb}$}‚

	  
	  \pstart \leavevmode% starting standard par
	\textbf{य‚दि} बाध‚कं प्र‚माणं नास्ती‚{\tiny $_{४}$}‚त्य‚प्र‚तिक्षेपो ॥ \textbf{भावे}ऽस्तित्वे किं \textbf{प्र‚माण‚न्नैवास्ति‚{\tiny $_{lb}$}‚ प्र‚माण}म‚तः स‚त्तानिश्च‚यो न युक्त इति प‚रः ।
	{\color{gray}{\rmlatinfont\textsuperscript{§~\theparCount}}}
	\pend% ending standard par
      ‚{\tiny $_{lb}$}‚

	  
	  \pstart \leavevmode% starting standard par
	\textbf{अत एवे}त्या चा र्यः । अत्य‚न्त‚म‚तीन्द्रिय‚स्यार्थ‚स्य साध‚क‚बाध‚क‚प्र‚माणाभावात्‚{\tiny $_{lb}$}‚ प्रेक्षाव‚तः \textbf{संश‚यो} युक्तः । य‚दि वा य‚द् व‚स्तु निराक‚र्त्तुन्न श‚क्य‚ते न च त‚स्य साध‚कं‚{\tiny $_{lb}$}‚ प्र‚माणं प्र‚तिभाति । त‚स्यैवं स‚म्भाव‚ना युक्ता भ‚वेद‚स्य क‚दाचित् साध‚कं प्र‚माणं‚{\tiny $_{lb}$}‚ ‚{\tiny $_{lb}$}‚ \leavevmode\ledsidenote{\textenglish{376/s}}क‚स्य‚चित् प्र‚तिभाव‚तः‚{\tiny $_{५}$}‚ । त‚स्माद‚र्थ‚संश‚यात् प्र‚माण‚संश‚याद्वाऽप्र‚तिक्षेपः ।
	{\color{gray}{\rmlatinfont\textsuperscript{§~\theparCount}}}
	\pend% ending standard par
      ‚{\tiny $_{lb}$}‚

	  
	  \pstart \leavevmode% starting standard par
	अन्ये तु विर‚क्त‚चित्तेऽभ्युद‚य‚हेतुत्वे चात एवेति साध‚क‚बाध‚क‚प्र‚माणाभावात्‚{\tiny $_{lb}$}‚ संश‚योस्त्विति व्याख्याय । विर‚क्तं चित्तं स‚र्व‚ज्ञ‚त्वे भ‚वेद्वा प्र‚माण‚मित्य‚प्र‚तिक्षेपः ।‚{\tiny $_{lb}$}‚ त‚च्च प्र‚माणं द्वितीये प‚रिच्छेदेऽभिधास्य‚त इति व्याच‚क्ष‚ते ।
	{\color{gray}{\rmlatinfont\textsuperscript{§~\theparCount}}}
	\pend% ending standard par
      ‚{\tiny $_{lb}$}‚

	  
	  \pstart \leavevmode% starting standard par
	य‚त एव व्य‚व‚हित‚स्यापि कार्योत्प‚त्तिः । \textbf{त‚त्त‚स्माद‚त्र} एवातीन्द्रियेषु म‚ध्ये \textbf{केषां‚{\tiny $_{lb}$}‚चिद‚र्थानां स्व‚भावानां चेति} य‚द्य‚पि पाठ‚क्र‚मः । त‚थापि य‚थायोगं स‚म्ब‚न्धः ।
	{\color{gray}{\rmlatinfont\textsuperscript{§~\theparCount}}}
	\pend% ending standard par
      ‚{\tiny $_{lb}$}‚

	  
	  \pstart \leavevmode% starting standard par
	केषांचित् स्व‚भाव‚नाम‚भ्युद‚य‚हेत्वादीनां भ‚वेज्ज्ञाप‚कासिद्धिः । क‚थं [।] \textbf{द‚र्श‚न‚{\tiny $_{lb}$}‚पाट‚वाभावात्} । त‚द्विष‚य‚स्यानुभ‚व‚स्य य‚थागृहीत‚स्व‚रूप‚निश्च‚योत्पाद‚ने साम‚र्थ्या‚{\tiny $_{lb}$}‚भावात् । केषांचिद‚र्थानां विर‚क्त‚चित्त‚त्वादीनां । प्र‚त्य‚क्षानुमान‚ल‚क्ष‚णं कार्यं‚{\tiny $_{lb}$}‚ ज्ञाप‚क‚न्त‚स्याभावात् ।
	{\color{gray}{\rmlatinfont\textsuperscript{§~\theparCount}}}
	\pend% ending standard par
      ‚{\tiny $_{lb}$}‚

	  
	  \pstart \leavevmode% starting standard par
	\leavevmode\ledsidenote{\textenglish{136b/PSVTa}} त‚थाभूत‚स्य कार्य‚स्य क‚स्माद‚भाव इत्याह । \textbf{कार‚णाना}मित्यादि । न हि कार‚{\tiny $_{lb}$}‚णैर‚व‚श्य‚मात्म‚ज्ञाप‚क‚ङ्कार्यं स‚र्व‚पुरुष‚ग्राह्य‚मार‚ब्ध‚व्य‚मिति निय‚मः ।
	{\color{gray}{\rmlatinfont\textsuperscript{§~\theparCount}}}
	\pend% ending standard par
      ‚{\tiny $_{lb}$}‚

	  
	  \pstart \leavevmode% starting standard par
	अन्ये त्व‚न्य‚था व्याच‚क्ष‚ते । केषां चित् स्व‚भावानाम‚भ्युद‚य‚हेतुवादिनां । \textbf{अर्थानां‚{\tiny $_{lb}$}‚ च} विर‚क्त‚चित्तादीनाम्भ‚वे\textbf{ज्ज्ञाप‚क‚स्य} निश्चाय‚क‚स्य प्र‚माण‚स्या\textbf{सिद्धिः} । कुतः [।]‚{\tiny $_{lb}$}‚ द‚र्श‚न‚पाट‚वाभावात् ।
	{\color{gray}{\rmlatinfont\textsuperscript{§~\theparCount}}}
	\pend% ending standard par
      ‚{\tiny $_{lb}$}‚

	  
	  \pstart \leavevmode% starting standard par
	एत‚दुक्त‚{\tiny $_{४}$}‚म्भ‚व‚ति । द‚र्श‚न‚मे‚{\tiny $_{१}$}‚व ज्ञाप‚कं क‚स्य‚चिद‚र्थ‚स्य त‚च्चाप‚टुत्वात्‚{\tiny $_{lb}$}‚ स‚र्वाकार‚निश्च‚य‚न्नोत्पाद‚य‚ति । कार‚णानां च कार्योत्पाद‚न‚निय‚माभाव‚स्तेनान‚न्त‚र‚{\tiny $_{lb}$}‚कार्याद‚र्श‚नात् । कार्य‚द्वारेणाप्य‚भ्युद‚य‚हेतुत्व‚न्न श‚क्यं निषेद्धुं । त‚थातीन्द्रियाणा‚{\tiny $_{lb}$}‚म‚र्थानां द‚र्श‚न‚स्य म‚नोविज्ञान‚ल‚क्ष‚ण‚स्यापाट‚वात् प्र‚त्य‚क्षेणाग्र‚ह‚णं । न हि कार्य‚{\tiny $_{lb}$}‚द्वारेणैषां निश्च‚यः कार‚णानां च कार्योत्पाद‚{\tiny $_{२}$}‚न‚निय‚माभावात् । न हि कार‚णैर‚{\tiny $_{lb}$}‚व‚श्य‚मात्म‚ज्ञाप‚कं कार्यं स‚न्तानान्त‚रे ज‚न‚यित‚व्य‚मिति निय‚मः । त‚स्मान्न प्र‚त्य‚क्षं‚{\tiny $_{lb}$}‚ नानुमानं तेषाम‚स्ति ।
	{\color{gray}{\rmlatinfont\textsuperscript{§~\theparCount}}}
	\pend% ending standard par
      ‚{\tiny $_{lb}$}‚

	  
	  \pstart \leavevmode% starting standard par
	\textbf{नेय‚तेति} य‚थोक्त‚ज्ञाप‚काभाव‚मात्रेण \textbf{त‚द‚भावो}तीन्द्रियाणाम‚भावः । य‚स्मात्‚{\tiny $_{lb}$}‚ प्र‚त्य‚क्षानुमानाभ्याम‚नुप‚ल‚ब्धानाम‚पि \textbf{केषांचि}द‚र्थानां \textbf{पुन‚र‚पि प‚र्यायेण} क्र‚मेण‚{\tiny $_{lb}$}‚ कुड्य‚विव‚राव‚स्थितानाम‚र्थानां प्र‚त्य‚क्षेणानु‚{\tiny $_{३}$}‚मानेना\textbf{भिव्य‚क्तेः} प्र‚तीतेः । त‚देवं‚{\tiny $_{lb}$}‚ विप्र‚कृष्टे स्व‚भावानुप‚ल‚म्भो नास‚त्तासाध‚नं । नापि कार्यानुप‚ल‚म्भः ।
	{\color{gray}{\rmlatinfont\textsuperscript{§~\theparCount}}}
	\pend% ending standard par
      ‚{\tiny $_{lb}$}‚‚{\tiny $_{lb}$}‚\textsuperscript{\textenglish{377/s}}

	  
	  \pstart \leavevmode% starting standard par
	कार‚णानुप‚ल‚म्भ‚स्तु त‚त्राप्य‚भाव‚साध‚न‚मित्याह । कार्ये त्वित्यादि । कार्ये तु‚{\tiny $_{lb}$}‚ स्व‚भावादिविप्र‚क‚र्षिण्य‚पि कार‚काज्ञानं कार‚णानुप‚ल‚ब्धिर‚भाव‚स्य साध‚न‚मेव ।
	{\color{gray}{\rmlatinfont\textsuperscript{§~\theparCount}}}
	\pend% ending standard par
      ‚{\tiny $_{lb}$}‚

	  
	  \pstart \leavevmode% starting standard par
	\textbf{स्व‚भावे}त्यादिना व्याच‚ष्टे । विप्र‚कृष्ट‚विष‚य‚स्य \textbf{स्व‚भाव‚स्याभावे साध्ये}‚{\tiny $_{lb}$}‚ स्व‚भावा\textbf{नुप‚{\tiny $_{४}$}‚ल‚म्भ एवाप्र‚माण‚मुच्य‚ते । कार‚णानुप‚ल‚म्भ‚स्तु प्र‚माण‚मेव} ।
	{\color{gray}{\rmlatinfont\textsuperscript{§~\theparCount}}}
	\pend% ending standard par
      ‚{\tiny $_{lb}$}‚

	  
	  \pstart \leavevmode% starting standard par
	न‚नु विप्र‚कृष्ट‚विष‚ये कार‚णानुप‚ल‚म्भ एव निश्चेतुम‚श‚क्य‚स्त‚त्क‚थं कार्याभावं‚{\tiny $_{lb}$}‚ साध‚येत् । स‚त्त्य‚म् [।] एताव‚द् व‚क्तुं श‚क्य‚ते [।] कार‚ण‚म‚न्त‚रेणानुद्दिष्ट‚विष‚ये‚{\tiny $_{lb}$}‚ कार्य‚म‚व‚श्यं न भ‚व‚तीतीय‚ता लेशेनास्योप‚न्यासः । अत एव सामान्येनाह । \textbf{न‚{\tiny $_{lb}$}‚ ह्य‚स्ती}त्यादि ॥
	{\color{gray}{\rmlatinfont\textsuperscript{§~\theparCount}}}
	\pend% ending standard par
      ‚{\tiny $_{lb}$}‚

	  
	  \pstart \leavevmode% starting standard par
	\textbf{न‚न्वि}त्यादि प‚रः । \textbf{अग्नेर्विनाशेपि‚{\tiny $_{५}$}‚} वास‚ग्रृहादौ धूम‚स्य द‚र्श‚नात् । त‚था‚{\tiny $_{lb}$}‚ चास‚त्य‚पि कार‚णे कार्यं दृष्ट‚मिति व्य‚भिचारः ॥ अस‚ति कार‚णे कार्य‚न्न स्यादित्य‚{\tiny $_{lb}$}‚नेन वाक्येन \textbf{कार‚ण‚स्थितिकाल‚भावि कार्यं} । याव‚त् कार‚ण‚स‚त्ता ताव‚त् कार्य‚स‚त्तेत्ये‚{\tiny $_{lb}$}‚\textbf{व‚न्न ब्रूमो} येन कार‚ण‚नाशेपीत्यादिकः प्र‚संगः स्यात् । किन्त‚र्ह्य‚नेन वाक्येनोच्य‚त‚{\tiny $_{lb}$}‚ इत्याह । \textbf{हेतुर‚हिते}त्यादि ।
	{\color{gray}{\rmlatinfont\textsuperscript{§~\theparCount}}}
	\pend% ending standard par
      ‚{\tiny $_{lb}$}‚

	  
	  \pstart \leavevmode% starting standard par
	एव‚न्ताव‚त् स्थिर‚ताम‚भ्युप‚ग‚{\tiny $_{६}$}‚म्योक्तं । क्ष‚णिक‚त्वे तु कार‚णे विन‚ष्टे कार्यं‚{\tiny $_{lb}$}‚स्थान‚मेव नास्तीति कुतो व्य‚भिचाराशंका । त‚था हि योग्निज‚न्यो धूम‚क्ष‚ण‚स्त‚स्या‚{\tiny $_{lb}$}‚ग्निविनाशे नास्त्येवाव‚स्थानं । क्ष‚णिक‚त्वेन विनाशात् । य‚श्च प‚श्चात् स्थायी धूमः‚{\tiny $_{lb}$}‚ स धूम‚हेतुक एव नाग्निहेतुकः । त‚देव द‚र्श‚य‚न्नाह । \textbf{न च त‚थे}त्यादि । \textbf{त‚था स्थायी}ति‚{\tiny $_{lb}$}‚ न‚ष्टेप्याद्ये कार‚णे कालान्त‚र‚स्थायी । \textbf{त‚दुपा‚{\tiny $_{७}$}‚दानः} पूर्व‚निरुद्ध‚हेतूपादानः ।
	{\color{gray}{\rmlatinfont\textsuperscript{§~\theparCount}}}
	\pend% ending standard par
      \textsuperscript{\textenglish{137a/PSVTa}}‚{\tiny $_{lb}$}‚

	  
	  \pstart \leavevmode% starting standard par
	क‚थ‚न्त‚र्हि पाश्चात्योपि धूमोग्निहेतुक इत्युच्य‚त इत्याह । \textbf{पार‚म्प‚र्ये}त्यादि ।‚{\tiny $_{lb}$}‚ आद्य‚न्ताव‚द् धूम‚क्ष‚ण‚म्व‚ह्निरेव ज‚न‚य‚ति [।] स धूम‚क्ष‚णोऽप‚रं सोप्य‚प‚र‚मित्येवं‚{\tiny $_{lb}$}‚ पार‚म्प‚र्येण । एत‚देव स्फुट‚य‚न्नाह । \textbf{स‚न्तानोप‚कारा}दिति । प्र‚ब‚न्ध‚स्य प्र‚थ‚म‚तो ज‚न‚{\tiny $_{lb}$}‚नात् \textbf{त‚त्कार्य‚व्य‚प‚देशः} । त‚स्माद् य‚स्य कार‚ण‚स्य पाश्चात्य‚म‚पि कार्य‚मित्येवं‚{\tiny $_{१}$}‚ व्य‚प‚{\tiny $_{lb}$}‚‚{\tiny $_{lb}$}‚ \leavevmode\ledsidenote{\textenglish{378/s}}देशः । \textbf{य‚द्य‚स्य} हेतोः \textbf{क‚थंचि}त्प्र‚माणेना\textbf{भावः सिध्येत्} त‚दा \textbf{त‚त्फ‚ल‚न्त‚स्य} हेतोः \textbf{फ‚ल‚{\tiny $_{lb}$}‚न्नास्तीति निश्चीय‚ते} ।
	{\color{gray}{\rmlatinfont\textsuperscript{§~\theparCount}}}
	\pend% ending standard par
      ‚{\tiny $_{lb}$}‚

	  
	  \pstart \leavevmode% starting standard par
	एत‚च्चोद्दिष्ट‚विष‚य‚स्याभाव‚स्य साध‚न‚म‚भिप्रेत्योक्त‚म् [।] अनुद्दिष्ट‚विष‚ये तु‚{\tiny $_{lb}$}‚ नैत‚त्प्र‚माणं प्र‚तिब‚न्ध‚फ‚ल‚त्वात् ।
	{\color{gray}{\rmlatinfont\textsuperscript{§~\theparCount}}}
	\pend% ending standard par
      ‚{\tiny $_{lb}$}‚

	  
	  \pstart \leavevmode% starting standard par
	\textbf{स्व‚भावे}त्यादि । \textbf{अर्थ}स्येति व्याप‚क‚स्य स्व‚भावेऽव्य‚तिरिक्ते \textbf{लिङ्गि}न्य‚स‚त्त्वेन‚{\tiny $_{lb}$}‚ साध्ये \textbf{स्व‚भावा}नुप\textbf{ल‚म्भ}श्च व्याप‚कानुप‚ल‚म्भ‚श्चा‚{\tiny $_{२}$}‚भाव‚स्य साध‚न‚मिति प्र‚कृतं ।‚{\tiny $_{lb}$}‚ \href{http://sarit.indology.info/?cref=pv.3.200-3.201}{२०३-४}
	{\color{gray}{\rmlatinfont\textsuperscript{§~\theparCount}}}
	\pend% ending standard par
      ‚{\tiny $_{lb}$}‚

	  
	  \pstart \leavevmode% starting standard par
	\textbf{स्व‚भावाभावे}त्यादिना व्याच‚ष्टे । \textbf{क‚श्चिदि}ति व्याप‚कानुप‚ल‚म्भः । य‚द्य‚नुप‚{\tiny $_{lb}$}‚ल‚भ्य‚मानो व्याप‚कः स्व‚भावोस्य व्याप्य‚स्य सिद्धः स्यात् त‚दा भ‚वेत् प्र‚माणं ॥ कार‚ण‚{\tiny $_{lb}$}‚व्याप‚कानुप‚ल‚म्भ‚श्च भ‚वेत् प्र‚माणं य‚दि त‚द‚भाव‚स्त‚योः कार‚ण‚व्याप‚क‚योर‚भावः‚{\tiny $_{lb}$}‚ प्र‚तीयेत हेतुना केन‚चित् । स्व‚भावानुप‚ल‚म्भाख्येन ।
	{\color{gray}{\rmlatinfont\textsuperscript{§~\theparCount}}}
	\pend% ending standard par
      ‚{\tiny $_{lb}$}‚

	  
	  \pstart \leavevmode% starting standard par
	\textbf{य‚दी}त्यादिना का रि‚{\tiny $_{३}$}‚ का र्थ माह । \textbf{य‚द्य‚स्य} कार‚क‚स्या\textbf{भावः सिद्ध्ये}दिति‚{\tiny $_{lb}$}‚ स‚म्ब‚न्धः ॥ \textbf{व्याप‚क‚स्य च स्व‚भाव‚स्याभावः । कुत‚श्चिद् ग‚म‚काद्धेतो}रित्युप‚ल‚ब्धि‚{\tiny $_{lb}$}‚ल‚क्ष‚ण‚प्राप्तानुप‚ल‚म्भात् । सोयं कार‚को व्याप‚को वाऽस‚न्नेव सिद्धो य‚थाक्र‚मं कार्यं‚{\tiny $_{lb}$}‚ व्याप्य‚ञ्च निव‚र्त्त‚य‚ति । \textbf{त‚द‚भावासिद्धौ} कार‚क‚व्याप‚क‚योर‚भावासिद्धौ \textbf{निव‚र्त्त्येपि}‚{\tiny $_{lb}$}‚ ‚{\tiny $_{lb}$}‚ \leavevmode\ledsidenote{\textenglish{379/s}}कार्ये व्याप्ये \textbf{च संश‚यात्} । \href{http://sarit.indology.info/?cref=pv.3.201-3.202}{२०४-५}
	{\color{gray}{\rmlatinfont\textsuperscript{§~\theparCount}}}
	\pend% ending standard par
      ‚{\tiny $_{lb}$}‚

	  
	  \pstart \leavevmode% starting standard par
	य‚दि स्व‚भावाभावे साध्ये त‚द‚{\tiny $_{४}$}‚नुप‚ल‚म्भ एवाप्र‚माण‚मुच्य‚ते । \textbf{क‚थ}मिदानीम्भा‚{\tiny $_{lb}$}‚\textbf{व‚स्य} घ‚टादेः \textbf{स्व‚य‚म‚नुप‚ल‚ब्धेर‚भाव‚सिद्धिः} ।
	{\color{gray}{\rmlatinfont\textsuperscript{§~\theparCount}}}
	\pend% ending standard par
      ‚{\tiny $_{lb}$}‚

	  
	  \pstart \leavevmode% starting standard par
	उत्त‚र‚माह । \textbf{दृश्य‚स्ये}त्यादि । विप्र‚कृष्टे विष‚ये स्व‚भावानुप‚ल‚म्भे प्र‚माण‚मुक्तं ।‚{\tiny $_{lb}$}‚ न तु दृश्य‚विष‚य इत्य‚र्थः । \textbf{दृश्य}स्येति स्व‚भावाद्य‚विप्र‚कृष्ट‚स्य \textbf{भाव‚स्यानुप‚ल‚ब्ध‚स्य}‚{\tiny $_{lb}$}‚ स‚तः । \textbf{भाव‚स्य} स‚त्ताया \textbf{अभावः प्र‚तीय‚ते} । क‚दा \textbf{द‚र्श‚नाभाव‚कार‚णास‚म्भ‚वे स‚ति} ।‚{\tiny $_{lb}$}‚ द‚{\tiny $_{५}$}‚र्श‚नाभाव‚स्य कार‚णं । कार‚णान्त‚राणां वैक‚ल्य‚न्त‚स्यास‚म्भ‚वे स‚ति । उप‚ल‚म्भ‚{\tiny $_{lb}$}‚प्र‚त्य‚यान्त‚र‚साक‚ल्ये स‚तीत्य‚र्थः ।
	{\color{gray}{\rmlatinfont\textsuperscript{§~\theparCount}}}
	\pend% ending standard par
      ‚{\tiny $_{lb}$}‚

	  
	  \pstart \leavevmode% starting standard par
	\textbf{भावो ही}त्यादि विर‚णं । स्व‚भावाद्य‚विप्र‚कृष्टो भावो \textbf{य‚दि भ‚वेत् । य‚थास्व‚{\tiny $_{lb}$}‚ ग्राह‚केण क‚र‚णेने}ति य‚स्य य‚द् ग्राह‚क‚मिन्द्रिय‚न्ते\textbf{नोप‚ल‚भ्य} एव भ‚वेत् । स इति य‚थोक्तो‚{\tiny $_{lb}$}‚ भावः । \textbf{द‚र्श‚न‚प्र‚पिब‚न्धिषु व्य‚व‚धानादिषु} [।] आदिश‚ब्दाद् वैक‚ल्य‚प्र‚तिब‚न्धादिष्व‚{\tiny $_{lb}$}‚\textbf{स‚त्सु} । उ‚{\tiny $_{६}$}‚प‚ल‚म्भ‚प्र‚त्य‚येषु स‚त्स्विति याव‚त् । \href{http://sarit.indology.info/?cref=pv.3.202-3.203}{२०५-६}
	{\color{gray}{\rmlatinfont\textsuperscript{§~\theparCount}}}
	\pend% ending standard par
      ‚{\tiny $_{lb}$}‚

	  
	  \pstart \leavevmode% starting standard par
	त‚था भूतोनुप\textbf{ल‚ब्ध‚स्त्व‚स‚न्निति निश्चीय‚ते} । किङ्कार‚णं [।] \textbf{तादृशः स‚त} उप‚{\tiny $_{lb}$}‚ल‚ब्धिल‚क्ष‚ण‚प्राप्त‚स्य स‚तः । \textbf{उप‚ल‚म्भाव्य‚भिचारात्} । य एवायं स्व‚भाव‚स्याभाव‚{\tiny $_{lb}$}‚निश्च‚ये दृश्य‚स्य द‚र्श‚नेत्यादिनोक्तोऽ\textbf{य‚मेव हेतुर्वेदित‚व्यः} । क‚स्मिन् साध्ये [।]‚{\tiny $_{lb}$}‚ \textbf{हेतुव्याप‚क‚योर‚भावेपि} साध्ये ॥
	{\color{gray}{\rmlatinfont\textsuperscript{§~\theparCount}}}
	\pend% ending standard par
      ‚{\tiny $_{lb}$}‚

	  
	  \pstart \leavevmode% starting standard par
	\textbf{विरुद्ध‚स्य चेत्यादि} ।‚{\tiny $_{७}$}‚ य‚स्याभाव‚स्साध्य‚स्तेन यो \textbf{विरुद्ध‚स्त‚स्योप‚ल‚ब्धौ च} \leavevmode\ledsidenote{\textenglish{137b/PSVTa}}‚{\tiny $_{lb}$}‚ स्या\textbf{द‚स‚त्तायाः} प्र‚तिषेध्याभाव‚स्य \textbf{निश्च‚यः} । किङ्कार‚ण‚म् [।] \textbf{विरुद्ध‚स्य भाव‚स्य‚{\tiny $_{lb}$}‚ भावे} स‚त्ताया\textbf{न्त‚द्भाव‚बाध‚नात्} । त‚स्य निषेध्याभिम‚त‚स्य स‚त्ताबाध‚नात् ।
	{\color{gray}{\rmlatinfont\textsuperscript{§~\theparCount}}}
	\pend% ending standard par
      ‚{\tiny $_{lb}$}‚

	  
	  \pstart \leavevmode% starting standard par
	\textbf{यो ही}त्यादि विव‚र‚णं । क‚स्मान्नाव‚तिष्ठ‚त इत्याह । \textbf{त‚दि}त्यादि । त‚योर्वि‚{\tiny $_{lb}$}‚रुद्ध‚योर्ये \textbf{उपादाने} त‚योर‚न्योन्यं \textbf{प‚र‚स्प‚रं} य‚द् \textbf{वैगुण्य‚न्त‚स्याश्र‚य‚त्वेन} । य‚था शीतो‚{\tiny $_{lb}$}‚पादान‚मुष्णोपादान‚वैगुण्य‚स्याश्र‚य इत‚र‚श्चेत‚र‚स्येत्य‚र्थः । तेन कार‚णेन विरुद्ध‚{\tiny $_{lb}$}‚‚{\tiny $_{lb}$}‚ \leavevmode\ledsidenote{\textenglish{380/s}}योरेक‚त्र युग‚प‚दा\textbf{र‚म्भ‚विरोधात् । त‚योर्विरुद्ध‚योरेक‚स्य भावेप्य‚न्याभाव‚ग‚तिर्भ‚{\tiny $_{lb}$}‚व‚ति । य‚थोक्तं प्राग}नुप‚ल‚ब्धिप्र‚भेदे । न शीत‚स्प‚र्शोत्राग्नेरित्यादि ।‚{\tiny $_{lb}$}‚ य‚द्य‚प्य‚त्रानुप‚ल‚ब्धिरिति न श्रूय‚ते त‚थापीदं स्व‚भाव‚विरुद्धाख्यं लिङ्ग‚म‚नुप‚ल‚ब्धे‚{\tiny $_{lb}$}‚स्स‚काशान्न‚{\tiny $_{२}$}‚ पृथ‚गुच्य‚ते [।] किंङ्कार‚णं [।] \textbf{त‚त एवा}नुप‚ल‚म्भाद् \textbf{विरोध‚ग‚तेः ।‚{\tiny $_{lb}$}‚ विरोधाच्चाभाव‚साध‚नादि}त्युक्तं ॥
	{\color{gray}{\rmlatinfont\textsuperscript{§~\theparCount}}}
	\pend% ending standard par
      ‚{\tiny $_{lb}$}‚

	  
	  \pstart \leavevmode% starting standard par
	\textbf{भ‚व‚तु नामैव‚म्विधाया} दृश्या\textbf{नुप‚ल‚ब्धेः} स‚काशाद् \textbf{भाव‚ग‚तिः} । सा पुनः क‚थ‚म‚{\tiny $_{lb}$}‚नुमानं [।] नैवानुमानं किन्तु प्र‚माणान्त‚र‚मेवेति भावः ।
	{\color{gray}{\rmlatinfont\textsuperscript{§~\theparCount}}}
	\pend% ending standard par
      ‚{\tiny $_{lb}$}‚

	  
	  \pstart \leavevmode% starting standard par
	\textbf{क‚थ‚न्न स्या}दित्या चा र्यः ।
	{\color{gray}{\rmlatinfont\textsuperscript{§~\theparCount}}}
	\pend% ending standard par
      ‚{\tiny $_{lb}$}‚

	  
	  \pstart \leavevmode% starting standard par
	\textbf{दृष्टान्ते}त्यादि प‚रः । \textbf{दृष्टान्तापेक्षं} ह्य‚नुमान‚म‚न्व‚य‚व्य‚तिरेक‚व‚त्त्वात् । दृष्टान्ता‚{\tiny $_{lb}$}‚पेक्ष‚ण‚मेवा‚{\tiny $_{३}$}‚ह । \textbf{न हीति । अस्या}मित्य‚नुप‚ल‚ब्धौ ।
	{\color{gray}{\rmlatinfont\textsuperscript{§~\theparCount}}}
	\pend% ending standard par
      ‚{\tiny $_{lb}$}‚

	  
	  \pstart \leavevmode% starting standard par
	\textbf{किन्ने}त्याचा र्यः ।
	{\color{gray}{\rmlatinfont\textsuperscript{§~\theparCount}}}
	\pend% ending standard par
      ‚{\tiny $_{lb}$}‚

	  
	  \pstart \leavevmode% starting standard par
	\textbf{त‚दि}त्यादि प‚रः । त‚द् \textbf{व्योम‚कुसुमादि} । अस\textbf{दिति क‚थं} केन प्र‚माणेना\textbf{व‚ग‚न्त‚व्यं‚{\tiny $_{lb}$}‚ येनैवं स्याद्}दृष्टान्तः स्यात् । \textbf{अन‚प‚ल‚ब्धेरेव} लिङ्गाद् व्योम‚कुसुमाद्य‚स‚द‚व‚ग‚न्त‚व्य‚{\tiny $_{lb}$}‚मिति \textbf{चेत् । त‚त्रे}ति व्योम‚कुसुमादौ । \textbf{क‚थ‚म‚दृष्टान्तिका} दृष्टान्त‚र‚हिताऽ\textbf{स‚त्ता‚{\tiny $_{lb}$}‚सिद्धिः । स‚दृष्टान्त‚त्वे वान‚व‚स्थाप्र‚संग} इति त‚त्रा‚{\tiny $_{४}$}‚प्यपरोपीति कृत्वा । \textbf{त‚था‚{\tiny $_{lb}$}‚ चे}त्य‚न‚व‚स्थायां स‚त्यां साध्य‚स्या\textbf{प्र‚तिप‚त्तिः} । य‚त‚श्च दृष्टान्त‚त्वेन‚व‚स्थादोषः ।‚{\tiny $_{lb}$}‚ \textbf{त‚स्मान्निरुपाख्याभाव‚सिद्धिर}दृष्टान्तिका क‚र्त्त‚व्या । \textbf{त‚द्व‚द‚न्य‚त्रापि} [।] नेह घ‚टो‚{\tiny $_{lb}$}‚नुप‚ल‚ब्धिल‚क्ष‚ण‚प्राप्त‚स्यानुप‚ल‚ब्धेरित्यादाव‚पि प्र‚योगे \textbf{दृष्टान्तान‚पेक्ष‚णाद‚न‚नुमान‚म}‚{\tiny $_{lb}$}‚नुप‚ल‚ब्धिः [।]
	{\color{gray}{\rmlatinfont\textsuperscript{§~\theparCount}}}
	\pend% ending standard par
      ‚{\tiny $_{lb}$}‚

	  
	  \pstart \leavevmode% starting standard par
	\textbf{श्रृण्व‚न्न}पीत्या चा र्यः । अस‚कृदुक्त‚मेत‚त् । य‚{\tiny $_{५}$}‚था स्व‚भावानुप‚ल‚ब्धौ नाभावः‚{\tiny $_{lb}$}‚ साध्य‚ते किन्त्व‚भाव‚व्य‚व‚हार इति श्रृण्व‚न्न‚पि \textbf{देवानां} मूर्खाणां \textbf{प्रियो नाव‚धार‚ण‚{\tiny $_{lb}$}‚‚{\tiny $_{lb}$}‚ \leavevmode\ledsidenote{\textenglish{381/s}}प‚टुर्येन} स‚त्य‚पि दृष्टान्ते त‚द‚सिद्धिश्चोद्य‚ते । दृष्टान्त‚मेव द‚र्श‚यितुमुप‚क्र‚म‚ते ।‚{\tiny $_{lb}$}‚ \textbf{निमित्तं} हीत्यादि । उप‚ल‚भ्यानुप‚ल‚ब्धिर्दृश्यानुप‚ल‚ब्धिर्या \textbf{निमित्तं} कार‚ण‚म\textbf{स‚द्व‚{\tiny $_{lb}$}‚य‚व‚हाराणां । से}त्य‚नुप‚ल‚ब्धिः \textbf{स्व‚स‚न्निधाना}दात्म‚स‚न्नि‚{\tiny $_{२}$}‚धानात् । \textbf{स्व‚निमित्तान्}‚{\tiny $_{lb}$}‚ स्व‚म‚नुप‚ल‚ब्धिरूपं निमित्तं येषान्तानेतान\textbf{स‚द्व्य‚व‚हारान्} साध‚य‚तीति कृत्वा \textbf{स‚र्वोत्र‚{\tiny $_{lb}$}‚ दृष्टान्तः} । किंभूतः [।] \textbf{स्व‚निमित्त‚साम‚ग्रीयोग्य‚स‚न्निधानः} । स्व‚कार‚णानां साम‚ग्री‚{\tiny $_{lb}$}‚ त‚स्यां योग्यं स‚न्निधानं य‚स्याङ्कुरादेस्स त‚थोक्तः [।] प्र‚योगः पुनः । य‚स्य य‚त्र‚{\tiny $_{lb}$}‚ निमित्तं स‚क‚ल‚म‚प्र‚तिब‚द्व‚म‚स्ति त‚त्र तेन भ‚वित‚व्य‚न्त‚द्य‚थाङ्कुरादि ।‚{\tiny $_{७}$}‚ अस्ति \leavevmode\ledsidenote{\textenglish{381a/PSVTa}}‚{\tiny $_{lb}$}‚ चोप‚ल‚ब्धिल‚क्ष‚ण‚प्राप्त‚स्यानुप‚ल‚ब्धाव‚स‚द्व्य‚व‚हाराणां निमित्त‚त्व‚मिति स्व‚भाव‚हेतुः ।‚{\tiny $_{lb}$}‚ क‚स्माद् [।] अत्राभाव‚व्य‚व‚हार एव साध्य‚ते न पुन‚र‚भाव एवेत्याह । \textbf{अस‚त्ते}त्यादि ।‚{\tiny $_{lb}$}‚ \textbf{अत्र} दृश्यानुप‚ल‚ब्धाव\textbf{नुप‚ल‚ब्धिरेवा}स‚त्ता । य‚थोक्तं प्राक् । \textbf{त‚स्माद}भावो \textbf{न} \edtext{\textsuperscript{*}}{\lemma{*}\Bfootnote{?}}‚{\tiny $_{lb}$}‚ साध्य‚ते । य‚त एवात्राभाव‚व्य‚व‚हारः साध्य‚तेऽ\textbf{त एवेय‚म‚नु}प‚ल‚ब्धिः स्व‚भाव‚{\tiny $_{lb}$}‚हेताव‚न्त‚र्भ‚व‚तीति‚{\tiny $_{१}$}‚ स‚म्ब‚न्धः । \textbf{कार‚णाद्} दृश्यानुप‚ल‚म्भात् \textbf{कार्य‚स्या}स‚द्व्य‚व‚हार‚{\tiny $_{lb}$}‚स्यानु\textbf{मानं} त‚देव \textbf{ल‚क्ष‚णं} य‚स्येति सामान्येनान्य‚प‚दार्थ‚मुप‚द‚र्श्य प‚श्चाद् भाव‚प्र‚त्य‚यः ।‚{\tiny $_{lb}$}‚ स‚म‚ग्रात् \textbf{कार‚णात् कार्यानुमाने} च योग्य‚तानुमान‚मिति स्व‚भाव‚हेताव‚न्त‚र्भावः [।]‚{\tiny $_{lb}$}‚ व‚क्ष्याम‚श्च‚तुर्थे प‚रिच्छेदे [।]
	{\color{gray}{\rmlatinfont\textsuperscript{§~\theparCount}}}
	\pend% ending standard par
      ‚{\tiny $_{lb}$}‚

	  
	  \pstart \leavevmode% starting standard par
	दृश्यानुप‚ल‚ब्धौ भ‚व‚तु दृष्टान्तोऽदृश्यानुप‚ल‚ब्धौ तुं क‚थ‚मित्याह । \textbf{स‚च्छ‚ब्दे}‚{\tiny $_{lb}$}‚त्यादि देशादिविप्र‚कृष्टेषु‚{\tiny $_{२}$}‚ \textbf{प्र‚माण‚निवृत्त्या स‚द्व्य‚व‚हार‚निषेधे} साध्ये कार‚णाभावात्‚{\tiny $_{lb}$}‚ कार्य‚स्याभावः साध्य‚स्तेनात्र न केव‚ल‚न्निरुपाख्यं दृष्टान्तः किन्तु \textbf{निमित्त}स्य कार‚ण‚स्य‚{\tiny $_{lb}$}‚ \textbf{वैक‚ल्ये}नाभाविनो\textbf{ङ्कुराद‚योपि दृष्टान्तः} । प्र‚योग‚स्तु [।] य‚द्विक‚ल‚कार‚ण‚न्त‚न्न‚{\tiny $_{lb}$}‚ भ‚व‚ति य‚था बीज‚र‚हितोङ्कुरः । विक‚ल‚कार‚ण‚श्चादृश्यानुप‚ल‚ब्धौ स‚द्व्य‚व‚हार इति‚{\tiny $_{lb}$}‚ कार‚णानुप‚ल‚ब्धिः ।
	{\color{gray}{\rmlatinfont\textsuperscript{§~\theparCount}}}
	\pend% ending standard par
      ‚{\tiny $_{lb}$}‚

	  
	  \pstart \leavevmode% starting standard par
	य‚त्पुन‚रुक्तं [।] त‚द‚स‚न्निरुपा‚{\tiny $_{३}$}‚ख्यं क‚थं प्र‚तिप‚त्त‚व्य‚मिति त‚त्रापि \textbf{निरुपाख्ये}‚{\tiny $_{lb}$}‚ दृष्टान्ते । \textbf{इय‚मेव}म‚ज्ञान‚व्य‚व‚हार‚ल‚क्ष‚णा \textbf{प्र‚वृत्तिर्निषिध्य‚ते} । अनुप‚ल‚ब्धितो न त्व‚{\tiny $_{lb}$}‚भावः साध्य‚ते । किङ्कार‚ण‚म् [।] \textbf{अनुप‚ल‚ब्धी}त्यादि । विषाणादिविविक्त‚श‚श‚{\tiny $_{lb}$}‚‚{\tiny $_{lb}$}‚ \leavevmode\ledsidenote{\textenglish{382/s}}म‚स्त‚काद्युप‚ल‚ब्धिरेवानुप‚ल‚ब्धिः प‚र्युदास‚वृत्त्या । सैव ल‚क्ष‚णं य‚स्याः श‚श‚विषा‚{\tiny $_{lb}$}‚णास‚त्तायाः सा \textbf{सिद्धैव} ।
	{\color{gray}{\rmlatinfont\textsuperscript{§~\theparCount}}}
	\pend% ending standard par
      ‚{\tiny $_{lb}$}‚

	  
	  \pstart \leavevmode% starting standard par
	स्यादेत‚द् [।] अदृश्यानुप‚ल‚ब्धाव‚स‚त्त्वे‚{\tiny $_{४}$}‚ विष‚य‚भूते सिद्धे । त‚न्निमित्तोप्य‚{\tiny $_{lb}$}‚स‚द्व्य‚व‚हारः सिद्ध एवेति किन्तेनास‚द्व्य‚व‚हारेण । अदृश्यानुप‚ल‚ब्धाव‚पि स‚द्व्य‚व‚{\tiny $_{lb}$}‚हार‚निमित्त‚स्याभावाद् व्य‚व‚हाराप्र‚वृत्तिः सिद्धैवेति स‚द्व्य‚व‚हार‚निषेधेनापि साधि‚{\tiny $_{lb}$}‚तेन किं ।
	{\color{gray}{\rmlatinfont\textsuperscript{§~\theparCount}}}
	\pend% ending standard par
      ‚{\tiny $_{lb}$}‚

	  
	  \pstart \leavevmode% starting standard par
	त‚था पूर्व‚प्र‚सिद्ध‚विष‚योप‚द‚र्श‚न‚ल‚क्ष‚णेन दृष्टान्तेनाप्य‚नुप‚ल‚ब्धौ न किंचित्‚{\tiny $_{lb}$}‚ प्र‚योज‚न‚मित्य‚त आह । \textbf{सोय}मित्यादि । मूढं प्र‚त्येत‚त् साध्य‚त इति याव‚त् ।
	{\color{gray}{\rmlatinfont\textsuperscript{§~\theparCount}}}
	\pend% ending standard par
      ‚{\tiny $_{lb}$}‚

	  
	  \pstart \leavevmode% starting standard par
	क्व‚चिद‚प्य‚स‚द्व्य‚व‚हार\textbf{निमित्तं} दृश्या\textbf{नुप‚ल‚म्भ‚म‚भ्युप‚ग‚म्या}स‚द्व्य‚व‚हार\textbf{प्र‚वृत्तिम्वि‚{\tiny $_{lb}$}‚लोम‚य}न्न‚कुर्व‚न् । अदृश्यानुप‚ल‚ब्धौ त‚द‚भावं च स‚द्व्य‚व‚हार‚निमित्त‚स्य प्र‚माण‚{\tiny $_{lb}$}‚स्याभावं चाभ्युप‚ग‚म्य स‚द्व्य‚व‚हार\textbf{निवृत्तिं च विलोम‚य‚न्न}कुर्व‚न् । \textbf{य‚थाभ्युप‚ग‚मं‚{\tiny $_{lb}$}‚ प्र‚तिपाद्य‚त इति} दृश्यानुप‚ल‚म्भेन निमित्तेन त्व‚याऽस‚द्व्य‚व‚हारोन्य‚त्र कृत‚स्त‚दिहाप्य‚स्ति‚{\tiny $_{lb}$}‚ [।] त‚स्मा‚{\tiny $_{६}$}‚द‚स‚द्व्य‚व‚हार‚ङ् कुर्विति प्र‚तिपाद्य‚ते । त‚था प्र‚माणेन निमित्तेन त्व‚या‚{\tiny $_{lb}$}‚ क्व‚चित् स‚द्व्य‚व‚हारः कृत‚स्त‚दिहाप्य‚निरुपाख्ये नास्ति । त‚स्माद् स‚द्व्य‚वाहार‚म्मा‚{\tiny $_{lb}$}‚ कुर्विति प्र‚तिपाद्य‚ते । किमिव \textbf{निरुपाख्य‚व‚त्} । य‚था श‚श‚विषाणादाव‚स‚द्व्य‚व‚हार‚{\tiny $_{lb}$}‚निमित्त‚स्य दृश्यानुप‚ल‚म्भ‚स्य भावाद‚स‚द्व्य‚व‚हार‚स्त‚थेहापीति ।
	{\color{gray}{\rmlatinfont\textsuperscript{§~\theparCount}}}
	\pend% ending standard par
      ‚{\tiny $_{lb}$}‚

	  
	  \pstart \leavevmode% starting standard par
	\leavevmode\ledsidenote{\textenglish{138b/PSVTa}} \textbf{अन्य‚द्वेति} [।] य‚था स‚न्निहित‚निमित्त‚स्याऽङ्कुरादेः प्र‚{\tiny $_{७}$}‚तिप‚त्तिरेव स‚द्व्य‚व‚हार‚{\tiny $_{lb}$}‚स्यापीति । य‚था च निरुपाख्ये प्र‚माणाभावात् स‚द्व्य‚व‚हार‚स्य निवृत्तिः । \textbf{अन्य}‚{\tiny $_{lb}$}‚स्मिन् वा कार‚ण‚विक‚ले कार्ये निवृत्ति\textbf{स्त‚द्व‚द्} विप्र‚कृष्टेषु स‚द्व्य‚व‚हार‚स्य निवृत्ति‚{\tiny $_{lb}$}‚रिति \textbf{प्र‚तिपाद्य‚ते} ॥
	{\color{gray}{\rmlatinfont\textsuperscript{§~\theparCount}}}
	\pend% ending standard par
      ‚{\tiny $_{lb}$}‚

	  
	  \pstart \leavevmode% starting standard par
	\textbf{स एव ताव‚दुप‚ल‚ब्ध्य‚भावो}नुप‚ल‚ब्ध्याख्यः \textbf{क‚थं} केन प्र‚माणेन \textbf{सिद्धः} [।] न‚{\tiny $_{lb}$}‚ प्र‚त्य‚क्षेणाभाव‚विष‚य‚त्व‚विरोधात् । नाप्य‚नुमानेन प्र‚त्य‚क्ष‚पूर्व‚क‚त्वाद‚स्य‚{\tiny $_{१}$}‚ । नापि‚{\tiny $_{lb}$}‚ प्र‚देश‚स‚म्ब‚न्ध्य‚नुप‚ल‚म्भोऽभाव‚स्य नीरूप‚त्वेन स‚म्ब‚न्धित्वायोगात् । प्र‚देश‚स्या‚{\tiny $_{lb}$}‚नुप‚ल‚म्भे च ध‚र्म्म‚सिद्धेराश्र‚यासिद्धो हेतुरित्य‚नुप‚ल‚ब्धेर्न प‚क्ष‚ध‚र्म‚त्व‚मित्युद्यो त क र‚{\tiny $_{lb}$}‚ प्र‚भृत‚यः ।
	{\color{gray}{\rmlatinfont\textsuperscript{§~\theparCount}}}
	\pend% ending standard par
      ‚{\tiny $_{lb}$}‚

	  
	  \pstart \leavevmode% starting standard par
	अत्राह । \textbf{एत‚दुत्त‚र‚त्र व‚क्ष्याम}स्त‚द्विशिष्टोप‚ल‚म्भोऽत‚स्त‚स्याप्य‚नुप‚ल‚म्भ‚न‚न्त‚{\tiny $_{lb}$}‚‚{\tiny $_{lb}$}‚ \leavevmode\ledsidenote{\textenglish{383/s}}स्माद‚नुप‚ल‚म्भोयं प्र‚त्य‚क्षेणैव सिंध्य‚तीति ॥ य‚त्पुन‚रुक्त‚म‚नुप‚ल‚ब्धौ किन्दृष्टान्तोप‚{\tiny $_{२}$}‚‚{\tiny $_{lb}$}‚द‚र्श‚नेनेत्याह । \textbf{अन्य‚त्रापी}त्यादि । विधिसाध‚नेप्य‚नुमाने \textbf{साध्य‚ध‚र्मेण व्याप‚केन}‚{\tiny $_{lb}$}‚ व्याप्य‚ध‚र्मं कृत‚क‚त्वादिकं साध‚न\textbf{मिच्छ‚न्} प्र‚तिपाद्यः \textbf{किमिति} दृष्टान्तेन \textbf{प्र‚त्याय्यः} ।‚{\tiny $_{lb}$}‚ किङ्कार‚णं [।] य‚थास्वं साध्येन \textbf{व्याप्य}स्य हेतो\textbf{र्निर्देशादेव व्याप्नुव‚तः} साध्य‚ध‚र्म‚स्य‚{\tiny $_{lb}$}‚ \textbf{सिद्धेः । निश्चितार्थ‚स्स‚म्ब‚न्ध}ल‚क्ष‚णो येन \textbf{त‚स्य} व्यामूढ‚म्प्र‚ति \textbf{स्मृत्य‚र्थः} । त‚देत‚द्‚{\tiny $_{lb}$}‚ \textbf{दृष्टान्तेन मूढं प्र‚ति स्मृतिज‚न‚न‚म‚त्रा}नुप‚ल‚ब्धौ । एत‚देवाह । \textbf{सोय}मित्यादि । \textbf{अन्य‚{\tiny $_{lb}$}‚त्रापी}ति श‚श‚विषाणादौ \href{http://sarit.indology.info/?cref=pv.3.202-3.203}{। २०५-६}
	{\color{gray}{\rmlatinfont\textsuperscript{§~\theparCount}}}
	\pend% ending standard par
      ‚{\tiny $_{lb}$}‚

	  
	  \pstart \leavevmode% starting standard par
	\hphantom{.}अथ य‚दिद‚माचार्य दि ग्ना ग प्र‚भृतिभिरुक्तं । न स‚न्ति प्र‚धानाद‚योनुप‚ल‚ब्धे‚{\tiny $_{lb}$}‚ रिति त‚त्र प्र‚योगे । आ चा र्य आह । \textbf{क‚थं च न स्यादि}ति ।
	{\color{gray}{\rmlatinfont\textsuperscript{§~\theparCount}}}
	\pend% ending standard par
      ‚{\tiny $_{lb}$}‚

	  
	  \pstart \leavevmode% starting standard par
	चोद‚कः स्वाभिप्रायेणाह । \textbf{त‚द‚र्थे}त्यादि । त‚द‚र्थ‚स्य प्र‚धानार्थ‚स्य \textbf{निषेधे} स‚ति‚{\tiny $_{lb}$}‚ प्र‚धानादिध\textbf{र्मिवाचिनोभिधान‚{\tiny $_{४}$}‚स्याप्र‚योगात्} । न हि वाच्य‚म‚न्त‚रेण वाच‚क‚स्य‚{\tiny $_{lb}$}‚ प्र‚योगोस्ति । य‚दा च प्र‚धानादिश‚ब्दानाम‚प्र‚योग‚स्त‚दा न स‚न्तीति प्र‚तिषेध‚वा‚{\tiny $_{lb}$}‚च्येव श‚ब्दोव‚शिष्य‚ते । त‚स्य च प्र‚तिषेध्यासंकीर्त्त‚ने \textbf{निर्विष‚य‚स्य प्र‚तिषे}ध‚स्येति‚{\tiny $_{lb}$}‚ प्र‚तिषेध‚वाचिनः श‚ब्द‚स्या\textbf{प्र‚योगात्} । त‚था च न स‚न्ति प्र‚धानाद‚य इति द्व‚योर‚पि‚{\tiny $_{lb}$}‚ प्र‚तिज्ञाप‚द‚योर‚प्र‚योगात् । कुत्र किं साध्य‚त‚{\tiny $_{५}$}‚ इति व्य‚र्थानुप‚ल‚ब्धेः साध‚न‚योगः ।‚{\tiny $_{lb}$}‚ प्र‚धानादीनां च प‚क्ष‚भूतानाम‚भावाद् प‚क्ष‚ध‚र्मो हेतुः स्यात् । प्र‚तिज्ञाप‚द‚योर्वा प‚र‚{\tiny $_{lb}$}‚स्प‚रं विरोध‚स्त‚था हि प्र‚धानाद‚य इति प्र‚योगात् त‚द‚र्थ‚स‚न्निधानं । पुन‚र्न स‚न्तीति‚{\tiny $_{lb}$}‚ व‚च‚नात् तेषाम‚स‚न्निधानं । एते च स‚न्निधानास‚न्निधाने युग‚प‚देक‚त्र विरुध्येते ।‚{\tiny $_{lb}$}‚ \href{http://sarit.indology.info/?cref=pv.3.203-3.204}{२०६-७}
	{\color{gray}{\rmlatinfont\textsuperscript{§~\theparCount}}}
	\pend% ending standard par
      ‚{\tiny $_{lb}$}‚‚{\tiny $_{lb}$}‚‚{\tiny $_{lb}$}‚\textsuperscript{\textenglish{384/s}}

	  
	  \pstart \leavevmode% starting standard par
	\textbf{नैष दोष} इति प‚रिहारः । य‚स्मा\textbf{द‚नादेर्वास‚ना‚{\tiny $_{६}$}‚}यास्स\textbf{मुद्भूते विक‚ल्पे प‚रिनि‚{\tiny $_{lb}$}‚ष्ठित}स्स‚मारुढः \textbf{श‚ब्दार्थः} स च \textbf{भावाभावोभ‚याश्रित}त्वात् \textbf{त्रिविधो} ध‚र्म्मः । \textbf{त‚स्मिन्}‚{\tiny $_{lb}$}‚ श‚ब्देर्थे बाह्य‚प्र‚धानादिको भावोनुपादान‚म‚स्येति त‚स्मिन् \textbf{भावानुपादाने साध्ये ।‚{\tiny $_{lb}$}‚ अस्ये}ति बुद्धिप‚रिव‚र्त्तिनः प्र‚धानादिध‚र्मिणः । \textbf{त‚थे}ति बाह्य‚प्र‚धानाद्युपादान‚त्वे‚{\tiny $_{lb}$}‚\leavevmode\ledsidenote{\textenglish{139a/PSVTa}} \textbf{नानुप‚ल‚म्भ‚नं हेतुः} । वास‚नोपादान‚त्वेनोप‚ल‚म्भ‚न‚{\tiny $_{७}$}‚मेवात्रानुप‚ल‚म्भ‚नं । \textbf{न त‚स्यैव}‚{\tiny $_{lb}$}‚ विक‚ल्प‚प्र‚तिनिष्ठित‚स्य ध‚र्मिणोऽ\textbf{भाव}स्साध्य‚ते । क‚स्मात् [।] प्र‚धानादिश‚ब्दानां‚{\tiny $_{lb}$}‚ \textbf{प्र‚योग‚तः} । त‚न्निषेधे हि निर्विष‚य‚त्वाच्छ‚ब्द‚प्र‚योग एव न स्यात् ।
	{\color{gray}{\rmlatinfont\textsuperscript{§~\theparCount}}}
	\pend% ending standard par
      ‚{\tiny $_{lb}$}‚

	  
	  \pstart \leavevmode% starting standard par
	\textbf{निवेदित}मित्यादिना व्याच‚ष्टे । य‚दि हि स्व‚ल‚क्ष‚ण‚म‚भिधेयं स्यात् त‚दा स्व‚ल‚क्ष‚ण‚{\tiny $_{lb}$}‚प्र‚तिषेधे श‚ब्दार्थ‚स्य प्र‚तिषेधात् त‚द्वाचिनः श‚ब्द‚स्याप्र‚योगः स्यात् । याव‚ता निवेदित‚{\tiny $_{lb}$}‚मेत‚द् \textbf{य‚थैते श‚ब्दा न स्व‚ल‚क्ष‚{\tiny $_{१}$}‚ण‚विष‚या} इति । किन्त्व\textbf{नादिवास‚नायाः प्र‚भ‚व}‚{\tiny $_{lb}$}‚ उत्पादो य‚स्य \textbf{विक‚ल्प‚स्य} त‚स्मिन् \textbf{प्र‚तिभासी} योर्थ\textbf{स्त‚म्विष‚य‚त्वेनात्म‚सात्कुर्व‚न्ति} ।
	{\color{gray}{\rmlatinfont\textsuperscript{§~\theparCount}}}
	\pend% ending standard par
      ‚{\tiny $_{lb}$}‚

	  
	  \pstart \leavevmode% starting standard par
	कुत एत‚द् विक‚ल्प‚विष‚य‚मेवार्थ‚म्विष‚यीकुर्व‚न्तीत्याह । \textbf{व‚क्तुस्त‚द्विक‚ल्प‚भाज}‚{\tiny $_{lb}$}‚ इति भावाभावोभ‚याश्र‚य‚सामान्याकार‚विक‚ल्प‚भाजः [।] कार‚ण‚माह । \textbf{य‚थे}त्यादि ।‚{\tiny $_{lb}$}‚ \textbf{य‚थाप्र‚तिभासि व‚स्तु} । प्र‚तिभासिव‚स्त्व‚न‚तिक्र‚मेण या \textbf{प्र‚{\tiny $_{२}$}‚तिपाद‚न‚स‚मीहा} त‚या‚{\tiny $_{lb}$}‚ श‚ब्द\textbf{प्र‚योगात्} । त‚स्माद् विक‚ल्प‚प्र‚तिभासिन एवार्थान् श‚ब्दा विष‚यीकुर्व‚न्ति ।
	{\color{gray}{\rmlatinfont\textsuperscript{§~\theparCount}}}
	\pend% ending standard par
      ‚{\tiny $_{lb}$}‚

	  
	  \pstart \leavevmode% starting standard par
	त‚था श्रोतुश्च त‚द्विक‚ल्प‚भाज इति स‚म्ब‚न्धः । किं कार‚णं [।] \textbf{त‚दाकार‚विक‚{\tiny $_{lb}$}‚ल्प‚ज‚न‚नात्} । भावाभावोभ‚याश्र‚य‚सामान्याकार‚विक‚ल्प‚ज‚न‚नाच्च विक‚ल्प‚प्र‚तिभासिन‚{\tiny $_{lb}$}‚ एवार्थान् श‚ब्दा विष‚यीकुर्व‚न्ति ।
	{\color{gray}{\rmlatinfont\textsuperscript{§~\theparCount}}}
	\pend% ending standard par
      ‚{\tiny $_{lb}$}‚

	  
	  \pstart \leavevmode% starting standard par
	स्यादेत‚द् [।] य‚दि नामैव‚न्त‚थापि क‚थ‚न्न स्व‚ल‚क्ष‚ण‚विष‚या श‚{\tiny $_{३}$}‚ब्दाः स्व‚ल‚क्ष‚{\tiny $_{lb}$}‚ण‚प्र‚तीतेरित्याह । \textbf{न चे}त्यादि । \textbf{उपादानं} च \textbf{कार्यं} च ते एव \textbf{प्र‚त्य‚यो} ज्ञान‚न्त‚त्रा‚{\tiny $_{lb}$}‚‚{\tiny $_{lb}$}‚ \leavevmode\ledsidenote{\textenglish{385/s}}\textbf{प्र‚तिभासि य‚द्रूप‚न्त‚द्विष‚य‚त्वेन त‚न्न} श‚क्य\textbf{न्निश्चेतुं} । त‚त्रोपादान‚प्र‚त्य‚यो व‚क्तुः‚{\tiny $_{lb}$}‚ प्र‚तिपाद‚न‚स‚मीहारूपो विक‚ल्पः श‚ब्द‚स्य कार‚ण‚त्वात् । कार्य‚प्र‚त्य‚यः श‚ब्दोत्थः‚{\tiny $_{lb}$}‚ श्रोतुर्विक‚ल्पः । न हि स श‚ब्दार्थो यः श‚ब्दे प्र‚त्य‚ये न प्र‚तिभास‚ते । न च त‚त्र स्व‚ल‚{\tiny $_{lb}$}‚क्ष‚{\tiny $_{४}$}‚णं प्र‚तिभास‚ते । स्प‚ष्टाकार‚विवेकात् ।
	{\color{gray}{\rmlatinfont\textsuperscript{§~\theparCount}}}
	\pend% ending standard par
      ‚{\tiny $_{lb}$}‚

	  
	  \pstart \leavevmode% starting standard par
	न‚नु क‚थं विक‚ल्पाभावाश्र‚योऽर्थ‚ज‚न्य‚त्वेनाविक‚ल्प‚क‚त्व‚प्र‚संगात् । क‚थ‚{\tiny $_{lb}$}‚ञ्चाभावाश्र‚योऽभाव‚स्याकार‚ण‚त्वात् । क‚थ‚ञ्चैक उभ‚याश्र‚योऽहेतुक‚त्व‚प्र‚स‚ङ्गा‚{\tiny $_{lb}$}‚दित्य‚त आह । \textbf{स त्वि}त्यादि । स‚द‚स‚दुभ‚याकारो \textbf{विक‚ल्पः स‚द‚स‚दुभ‚य‚प्र‚त्य‚य‚स्ते‚{\tiny $_{lb}$}‚नाहिता वास‚ना त‚तः प्र‚भ‚व} उत्पादो य‚स्य स त‚थोक्तः । \textbf{इति} हेतोर्भावाभावो‚{\tiny $_{५}$}‚‚{\tiny $_{lb}$}‚भ‚य‚ध‚र्म इत्युच्य‚ते ।
	{\color{gray}{\rmlatinfont\textsuperscript{§~\theparCount}}}
	\pend% ending standard par
      ‚{\tiny $_{lb}$}‚

	  
	  \pstart \leavevmode% starting standard par
	द्वितीयं कार‚ण‚माह । \textbf{त‚दि}त्यादि । \textbf{त}स्मिन्नेव विक‚ल्पे \textbf{प्र‚तिभासी} य \textbf{आकार}‚{\tiny $_{lb}$}‚स्त‚स्य स‚द‚स‚दुभ‚य‚रूप‚त‚या\textbf{ध्य‚व‚सा}य‚स्त\textbf{द्व‚शेन च । भावाभावोभ‚य‚ध‚र्म इत्युच्य‚ते} ।‚{\tiny $_{lb}$}‚ \href{http://sarit.indology.info/?cref=pv.3.206}{२०९}
	{\color{gray}{\rmlatinfont\textsuperscript{§~\theparCount}}}
	\pend% ending standard par
      ‚{\tiny $_{lb}$}‚

	  
	  \pstart \leavevmode% starting standard par
	एत‚दुक्त‚म्भ‚व‚ति । स‚त्प्र‚त्य‚याहित‚वास‚नाप्र‚भ‚व‚स्स‚दाकाराध्य‚व‚सायीव भाव‚{\tiny $_{lb}$}‚ध‚र्म इत्युच्य‚ते । एव‚म‚भावोभ‚य‚ध‚र्म‚योर‚पि द्र‚ष्ट‚व्यं पूर्व‚पूर्व‚स‚दादिप्र‚त्य‚{\tiny $_{६}$}‚याहित‚{\tiny $_{lb}$}‚वास‚नाप्र‚भ‚व‚त्वादेव स‚दादिविक‚ल्पानाम‚नादित्व‚म् [।] अतो न भावादिज‚न्य‚त्वं ।‚{\tiny $_{lb}$}‚ य‚त‚श्च श‚ब्दार्थ‚स्त्रिविधः । \textbf{त‚त्त}स्मा\textbf{द‚त्र} श‚ब्दार्थे \textbf{ध‚र्मिणि} व्य‚व‚स्थिता विवा\textbf{दाश्र‚{\tiny $_{lb}$}‚य‚त्वेन} स्थिताः पुरुषा\textbf{स्स‚द‚स‚त्त्वं} प्र‚धानादे\textbf{श्चिन्त‚य‚न्ति । क‚थं} चिन्त‚य‚न्तीत्याह ।‚{\tiny $_{lb}$}‚ \textbf{किम‚यं प्र‚धान‚श‚ब्दादु}च्च‚रिताद् विक‚ल्प\textbf{प्र‚तिभास‚मानोर्थो} बाह्य‚प्र‚धान‚निब‚न्ध‚नो‚{\tiny $_{lb}$}‚ भ‚व‚ति न वेति । \textbf{त‚स्य} प्र‚{\tiny $_{७}$}‚धानादिविक‚ल्प‚प्र‚तिभास‚स्य \textbf{भावानुपादान‚त्वे} ध‚र्मे \textbf{साध्ये \leavevmode\ledsidenote{\textenglish{139b/PSVTa}}‚{\tiny $_{lb}$}‚ स एव} विक‚ल्प‚प्र‚तिभासो बुद्धिस्व‚भाव‚त‚या \textbf{प्र‚त्यात्म‚वेद्य‚त्वात्} स्व‚स‚म्वेद‚न‚प्र‚त्य‚क्ष‚{\tiny $_{lb}$}‚सिद्ध‚त्वाद\textbf{प्र‚तिक्षेपार्होर्थो ध‚र्मी} ।
	{\color{gray}{\rmlatinfont\textsuperscript{§~\theparCount}}}
	\pend% ending standard par
      ‚{\tiny $_{lb}$}‚

	  
	  \pstart \leavevmode% starting standard par
	न‚नु च स एव विक‚ल्प‚ग्राह्यार्थः स्व‚ल‚क्ष‚णं स्व‚ल‚क्ष‚ण‚रूप‚त‚याध्य‚व‚सायात् ।‚{\tiny $_{lb}$}‚ त‚त‚श्च प्र‚त्य‚क्ष‚सिद्ध एव प्र‚धानादिः श‚ब्द‚श्च स्व‚ल‚क्ष‚ण‚विष‚य इत्याह । \textbf{न च स‚{\tiny $_{lb}$}‚ एवे}त्यादि ।‚{\tiny $_{१}$}‚ स एव विक‚ल्प‚प्र‚तिभास्य‚र्थो बाह्यं \textbf{स्व‚ल‚क्ष‚ण‚मिति श‚क्य‚म्व‚क्तुं} । किं‚{\tiny $_{lb}$}‚ ‚{\tiny $_{lb}$}‚ \leavevmode\ledsidenote{\textenglish{386/s}}कार‚ण‚म् [।] \textbf{असंप्राप्ते}ऽनुत्प‚न्ने स्व‚ल‚क्ष‚णे \textbf{निरुद्धे} च \textbf{त‚स्य} विक‚ल्प‚प्र‚तिभास‚स्यान‚{\tiny $_{lb}$}‚\textbf{पाया}त् । किं च \textbf{व‚स्तुनि} घ‚टादौ \textbf{विप‚रीताकारः} प‚र‚स्प‚र‚विरोधिनो नित्यानित्या‚{\tiny $_{lb}$}‚द‚यः । तान\textbf{भिनिवे}ष्टुं शीलं येषान्ती\textbf{र्थान्त‚रीय‚प्र‚त्य‚यानान्तेष} भावाच्छ‚व्दार्थ‚{\tiny $_{lb}$}‚प्र‚तिभास‚स्य । न च स एवार्थः स्व‚{\tiny $_{२}$}‚ल‚क्ष‚ण‚मिति प्र‚कृतेन स‚म्ब‚न्धः ॥ \href{http://sarit.indology.info/?cref=pv.3.206-3.207}{२०९-१०}
	{\color{gray}{\rmlatinfont\textsuperscript{§~\theparCount}}}
	\pend% ending standard par
      ‚{\tiny $_{lb}$}‚

	  
	  \pstart \leavevmode% starting standard par
	अमुमेवार्थं संग्र‚हीतुमाह । \textbf{प‚र‚मार्थः स्व‚ल‚क्ष‚ण}न्त‚स्मिन्नेक‚स्थानः प्र‚वृत्तिर्ये‚{\tiny $_{lb}$}‚षान्त‚द्भाव‚स्त‚त्त्व‚न्त‚स्मिन्स‚ति \textbf{श‚ब्दानाम‚निब‚न्ध‚ना} प‚र‚मार्थ‚निब‚न्ध‚न‚र‚हिता‚{\tiny $_{lb}$}‚ प्र‚वृ\textbf{त्तिर्न स्यात् । द‚र्श‚नान्त‚र‚भि}न्नेष्व‚र्थे\textbf{षु} । सिद्धान्त‚भेद‚भिन्नेषु । \textbf{अतीताजात‚यो‚{\tiny $_{lb}$}‚र्वापी}ति अतीतान‚ग‚ते \textbf{चा}र्थे श‚ब्दानाम्प्र‚वृत्ति\textbf{र्न स्यात्} । त‚था \textbf{क‚स्या‚{\tiny $_{३}$}‚श्चिद् वाचो\textbf{ऽ}‚{\tiny $_{lb}$}‚ भूतार्थ‚ता} मृषार्थ‚ता न स्यात् । \textbf{इति} य‚थोक्ताद्धेतोरेषा वाक् । \textbf{बौद्धार्थ‚विष‚या} ।‚{\tiny $_{lb}$}‚ विक‚ल्प‚प्र‚तिभास्य‚र्थ‚विष‚या म‚ता । \href{http://sarit.indology.info/?cref=pv.3.207-3.208}{२१०-११}
	{\color{gray}{\rmlatinfont\textsuperscript{§~\theparCount}}}
	\pend% ending standard par
      ‚{\tiny $_{lb}$}‚

	  
	  \pstart \leavevmode% starting standard par
	\textbf{त‚स्य} चेति प्र‚धानादिविक‚ल्प‚प्र‚तिभास‚स्य । \textbf{य‚थास‚मीहितं रूपं} य‚थाप‚रिक‚{\tiny $_{lb}$}‚ल्पित‚म्बाह्य‚प्र‚धानादिरूप‚न्त‚द\textbf{नुपादान‚त्वे साध्ये । त‚थानुप‚ल‚म्भ} इति य‚था स‚मी‚{\tiny $_{lb}$}‚हित‚रूपोपादान‚त्वेनानुप‚ल‚म्भोस्तीति कृत्वा ‚{\tiny $_{४}$}‚न साध‚न\textbf{ध‚र्मा}सिद्धिर्नाप‚क्ष‚ध‚र्मो‚{\tiny $_{lb}$}‚ हेतुरित्य‚र्थः ।
	{\color{gray}{\rmlatinfont\textsuperscript{§~\theparCount}}}
	\pend% ending standard par
      ‚{\tiny $_{lb}$}‚

	  
	  \pstart \leavevmode% starting standard par
	\textbf{न पुन‚र‚त्र} प्र‚योगेऽय‚मेव \textbf{श‚ब्दे}त्यादि । \textbf{श‚ब्दा}दुत्प‚द्य‚ते यो विक‚ल्प‚स्त‚त्प्र‚ति‚{\tiny $_{lb}$}‚\textbf{भास्य‚र्थोप‚ह‚नूय‚ते} । किं कार‚णं [।] \textbf{त‚स्य} विक‚ल्प‚प्र‚तिभास्य‚र्थ‚स्य श्रोतृ\textbf{बुद्धा}वुप‚{\tiny $_{lb}$}‚  ‚{\tiny $_{lb}$}‚ ‚{\tiny $_{lb}$}‚ \leavevmode\ledsidenote{\textenglish{387/s}}स्थाप‚नाय \textbf{श‚ब्द‚प्र‚योगात् । त‚द‚भावे} विक‚ल्प‚प्र‚तिभास्य‚र्थाभावे । \textbf{त‚द‚योगा}च्छ‚ब्द‚{\tiny $_{lb}$}‚प्र‚योगायोगात् ।
	{\color{gray}{\rmlatinfont\textsuperscript{§~\theparCount}}}
	\pend% ending standard par
      ‚{\tiny $_{lb}$}‚

	  
	  \pstart \leavevmode% starting standard par
	\textbf{अपि च} य‚दि \textbf{श‚ब्तार्थ‚स्य वाप‚ह्न‚वः सा‚{\tiny $_{५}$}‚ध्य‚ते} । त‚दा श‚ब्दार्थाप‚ह्न‚वे साध्ये‚{\tiny $_{lb}$}‚ त‚स्य साध्य\textbf{ध‚र्म}स्य नास्तित्व‚स्य य \textbf{आधारो} ध‚र्मी श‚ब्दार्थ‚ल‚क्ष‚ण‚स्त‚स्य \textbf{निराकृतेर्न‚{\tiny $_{lb}$}‚ साध्यः} ध‚र्मिध‚र्मात्म‚कः \textbf{स‚मुदायः स्यात् । केव‚ल‚श्च} साध्य\textbf{ध‚र्म‚स्सिद्ध} एव । नास्ति‚{\tiny $_{lb}$}‚त्व‚मात्र‚स्य क्व‚चित् सिद्ध‚त्वात् ।
	{\color{gray}{\rmlatinfont\textsuperscript{§~\theparCount}}}
	\pend% ending standard par
      ‚{\tiny $_{lb}$}‚

	  
	  \pstart \leavevmode% starting standard par
	\textbf{य‚दी}त्यादि विव‚र‚णं । \textbf{त‚दि}ति त‚स्मा\textbf{द‚यं} साध्य‚ध‚र्मः \textbf{आधा}रेण यो \textbf{व्य‚व‚च्छेदो}‚{\tiny $_{lb}$}‚ विशेष‚ण‚म‚स्येदं नास्तित्व‚न्त\textbf{द‚न‚पेक्षः । इ‚{\tiny $_{६}$}‚ति} हेतो\textbf{र्नोप‚न्य‚स‚नीय एव} विवाद‚काले ।‚{\tiny $_{lb}$}‚ \href{http://sarit.indology.info/?cref=pv.3.208-3.209}{२११-१२}
	{\color{gray}{\rmlatinfont\textsuperscript{§~\theparCount}}}
	\pend% ending standard par
      ‚{\tiny $_{lb}$}‚

	  
	  \pstart \leavevmode% starting standard par
	\textbf{स‚द‚स‚त्प‚क्ष‚भेदे}नेति स‚द‚स‚त्प्र‚तिज्ञाभेदेन \textbf{व‚स्त्वेव चिन्त्य‚ते । श‚ब्दार्थान‚प‚वादिभि}‚{\tiny $_{lb}$}‚र‚थंक्रियार्थिभिः प्रेक्षाव‚द्भिः । य‚स्माद‚त्र व‚स्तुनि \textbf{प्र‚तिब‚द्धः फ‚लोद‚यः} । \href{http://sarit.indology.info/?cref=pv.3.209-3.210}{२१२-१३}
	{\color{gray}{\rmlatinfont\textsuperscript{§~\theparCount}}}
	\pend% ending standard par
      ‚{\tiny $_{lb}$}‚

	  
	  \pstart \leavevmode% starting standard par
	य‚दि नामैव‚न्त‚थापि किन्न श‚ब्दार्थो विचार‚णीय इत्याह [।] अर्थेत्यादि ।‚{\tiny $_{lb}$}‚ \textbf{अर्थ‚क्रियां} प्र‚त्य\textbf{स‚म‚र्थ‚स्य} श‚ब्दार्थ‚स्य \textbf{विचारैः} स‚द‚स‚त्त्व‚विचारैः \textbf{किन्त‚द‚{\tiny $_{७}$}‚द‚र्थि}नाम‚र्थ- \leavevmode\ledsidenote{\textenglish{140a/PSVTa}}‚{\tiny $_{lb}$}‚ क्रियार्थिनां [।] निद‚र्श‚न‚माह \textbf{ष‚ण्ढ‚स्ये}त्यादि । पुंस्त्व‚र‚हित‚स्य य‚द् \textbf{रूप‚वैरूप्य‚न्त}‚{\tiny $_{lb}$}‚स्मिन् विष‚ये \textbf{कामिन्या} मैथुनाभिलाषिण्याः \textbf{किम्प‚रीक्ष‚या} । \href{http://sarit.indology.info/?cref=pv.3.210-3.211}{२१३-१४}
	{\color{gray}{\rmlatinfont\textsuperscript{§~\theparCount}}}
	\pend% ending standard par
      ‚{\tiny $_{lb}$}‚

	  
	  \pstart \leavevmode% starting standard par
	\textbf{न ही}त्यादि विव‚र‚णं । न हि \textbf{श‚ब्दार्थोऽस‚न् कंचित् पुरुषार्थ‚मुप‚रुण‚द्धि} निव‚र्त्त‚{\tiny $_{lb}$}‚य‚ति । \textbf{स‚न् वा स‚माद‚धाति} क‚रोतीति य‚थायोगं स‚म्ब‚न्ध‚नीयं । किं कार‚णं [।]‚{\tiny $_{lb}$}‚ \textbf{य‚थाभिनिवेशं} पुरुष‚स्तं‚{\tiny $_{१}$}‚ श‚ब्दार्थ‚म्बाह्य‚त‚याऽभिनिविश‚तेऽध्य‚व‚स्य‚ति त‚था \textbf{त‚स्या‚{\tiny $_{lb}$}‚‚{\tiny $_{lb}$}‚ \leavevmode\ledsidenote{\textenglish{388/s}}स‚त्त्वाद}विद्य‚मान‚त्वात् । \textbf{य‚थात‚त्त्वं} वा स‚मीहित‚त्वात् ।
	{\color{gray}{\rmlatinfont\textsuperscript{§~\theparCount}}}
	\pend% ending standard par
      ‚{\tiny $_{lb}$}‚

	  
	  \pstart \leavevmode% starting standard par
	--य‚थार्थ‚स्व‚ल‚क्ष‚णं ज्ञान‚स्व‚ल‚क्ष‚ण‚म्वा स्थित‚न्त‚था श‚ब्देनाविष‚यीकृत‚त्वात् ।‚{\tiny $_{lb}$}‚ य‚त एव\textbf{न्त‚स्माद‚यं} पुरुष‚स्स‚द‚स‚च्चिन्तायां \textbf{प्र‚व‚र्त्त‚मानः स‚र्व‚दाऽव‚धीरित‚विक‚ल्प‚प्र‚ति‚{\tiny $_{lb}$}‚भासो}ऽन‚पेक्षित‚विक‚ल्प‚प्र‚तिभासो \textbf{व‚स्त्वेवाधिष्ठानीक‚रो‚{\tiny $_{२}$}‚ति} विष‚यीक‚रोति \textbf{य‚त्र}‚{\tiny $_{lb}$}‚ व‚स्तु\textbf{न्य‚यं पुरुषार्थः प्र‚तिब‚द्धः । अग्नावि}त्य‚ग्निस्व‚ल‚क्ष‚णे । आदिग्र‚ह‚णाद्‚{\tiny $_{lb}$}‚ दाह‚पाकादि ।
	{\color{gray}{\rmlatinfont\textsuperscript{§~\theparCount}}}
	\pend% ending standard par
      ‚{\tiny $_{lb}$}‚

	  
	  \pstart \leavevmode% starting standard par
	स्यादेत‚द् [।] ईदृशोर्थः श‚ब्दार्थेनापि साध्य‚त इत्याह । \textbf{न ही}त्यादि । \textbf{अत्रे}ति‚{\tiny $_{lb}$}‚ शीत‚प्र‚तिघातादौ । क‚स्मात् \textbf{त‚द‚नुभ‚वाप्ताव‚पि} । त‚स्य श‚ब्दार्थ‚स्यानुभ‚वाप्ताव‚पि ।‚{\tiny $_{lb}$}‚ \textbf{त‚द‚भावात्} त‚स्य शीत‚प्र‚तीकारादेर‚भावात् । \textbf{त‚त्} त‚स्मा\textbf{द‚य‚म‚र्थ‚क्रियार्थी‚{\tiny $_{३}$}‚} पुरुषः ।‚{\tiny $_{lb}$}‚ \textbf{त‚द‚स‚म‚र्थ}न्त‚स्याम‚र्थ‚क्रियायाम‚स‚म‚र्थं श‚ब्दार्थं \textbf{प्र‚ति द‚त्तानुयोगो} द‚त्ताव‚धानो भ‚वि‚{\tiny $_{lb}$}‚\textbf{तुन्न युक्तः । न हि वृष‚स्य‚न्ती} मैथुन‚मिच्छ‚न्ती \textbf{ष‚ण्ढ‚स्य} प‚रिच‚रितुम‚श‚क्त‚स्य \textbf{रूप‚{\tiny $_{lb}$}‚वैरूप्य‚प‚रीक्षायाम‚व‚ध‚त्ते}ऽव‚धान‚व‚ती भ‚व‚ति ।
	{\color{gray}{\rmlatinfont\textsuperscript{§~\theparCount}}}
	\pend% ending standard par
      ‚{\tiny $_{lb}$}‚

	  
	  \pstart \leavevmode% starting standard par
	उ द्यो त क रा द्युक्त‚दूष‚ण‚निरासार्थं पृच्छ‚ति । \textbf{य‚त्पुन‚रेत‚दुक्त}माचार्य दि ग्ना‚{\tiny $_{lb}$}‚ गेन [।] \textbf{क‚ल्पित‚स्ये}त्यादि । त‚स्य‚{\tiny $_{४}$}‚ कोर्थः ।
	{\color{gray}{\rmlatinfont\textsuperscript{§~\theparCount}}}
	\pend% ending standard par
      ‚{\tiny $_{lb}$}‚

	  
	  \pstart \leavevmode% starting standard par
	उत्त‚र‚माह । \textbf{श‚ब्दार्थ} इत्यादि । \textbf{क‚ल्प‚नाज्ञान‚विष‚य‚त्वाच्छ‚ब्दार्थः} प्र‚धानादि‚{\tiny $_{lb}$}‚श‚ब्दार्थः । \textbf{क‚ल्पितो । व‚स्तुनो} बाह्य‚स्य प्र‚धान‚ल‚क्ष‚ण‚स्या\textbf{श्र‚य}ण‚न्तेना\textbf{सिद्धि}र‚नुप‚ल‚ब्धि‚{\tiny $_{lb}$}‚र‚स्य प्र‚धानादिश‚ब्दार्थ‚स्य ध‚र्म \textbf{उक्तो} लिङ्ग‚भूतो भावानुपादान‚त्वे साध्ये । \textbf{न्याय‚{\tiny $_{lb}$}‚वादिना}चार्य दि ग्ना गे न ।
	{\color{gray}{\rmlatinfont\textsuperscript{§~\theparCount}}}
	\pend% ending standard par
      ‚{\tiny $_{lb}$}‚‚{\tiny $_{lb}$}‚\textsuperscript{\textenglish{389/s}}

	  
	  \pstart \leavevmode% starting standard par
	तेन य‚दुच्य‚ते उ द्यो त क रा दिभिः । य‚दि प्र‚माणेन प्र‚धानं स‚त्त्वेन क‚{\tiny $_{५}$}‚ल्पितं‚{\tiny $_{lb}$}‚ क‚थ‚म‚स्यानुप‚ल‚ब्धिर्द्ध‚र्मः प्र‚तीय‚ते । अथास‚त्त्वेन क‚ल्पितं प्र‚धान‚न्त‚थापि क‚थ‚म‚स्यानु‚{\tiny $_{lb}$}‚प‚ल‚ब्धिर्ध‚र्मोऽस‚त्त्वादिति [।]
	{\color{gray}{\rmlatinfont\textsuperscript{§~\theparCount}}}
	\pend% ending standard par
      ‚{\tiny $_{lb}$}‚

	  
	  \pstart \leavevmode% starting standard par
	त‚द‚पास्तं । य‚स्मात् \textbf{क‚ल्प}नाज्ञान\textbf{विष‚य‚त्वाच्छ‚ब्दार्थ एव क‚ल्पितः} । न तु‚{\tiny $_{lb}$}‚ प्र‚माणेन बाह्यं प्र‚धानं स‚त्त्वेन क‚ल्पितं । \textbf{त‚स्ये}ति प्र‚धान‚श‚ब्दार्थ‚स्य बाह्य‚प्र‚धान‚{\tiny $_{lb}$}‚\textbf{व‚स्त्वाश्र‚य‚णानुल‚म्भ} इत्य‚य‚म‚भिप्राय आचार्य दि ग्ना ग स्य ॥ \href{http://sarit.indology.info/?cref=pv.3.211-3.212}{२१४-१५}
	{\color{gray}{\rmlatinfont\textsuperscript{§~\theparCount}}}
	\pend% ending standard par
      ‚{\tiny $_{lb}$}‚

	  
	  \pstart \leavevmode% starting standard par
	\textbf{य‚दुक्त}मित्यादि प‚रः । प्र‚{\tiny $_{६}$}‚माण‚त्र‚यं प्र‚त्य‚क्षानुमानाग‚म‚ल‚क्ष‚णं । \textbf{अन्य‚प्र‚माण‚{\tiny $_{lb}$}‚निवृत्ताविति} प्र‚त्य‚क्षानुमान‚निवृत्तौ निवृत्तिर्देशादिविप्र‚कृष्टानां । \textbf{त‚यो}रिति प्र‚त्य‚{\tiny $_{lb}$}‚क्षानुमान‚योः । \textbf{न किञ्च‚द व्याप्नोति} स‚र्व‚मेव विष‚यीक‚रोति । त‚न्निनिवृत्तिरा‚{\tiny $_{lb}$}‚ग‚म‚निवृत्तिः ।
	{\color{gray}{\rmlatinfont\textsuperscript{§~\theparCount}}}
	\pend% ending standard par
      ‚{\tiny $_{lb}$}‚

	  
	  \pstart \leavevmode% starting standard par
	\textbf{उक्त‚म}त्रेति सि द्धा न्त वा दी । \textbf{अप्र‚क‚र‚णाप‚न्न‚त्वादिति} पुरुषार्थ‚चिन्ताप्र‚{\tiny $_{lb}$}‚प्र‚स्तावानुप‚योगित्वात् । \textbf{शास्त्राधिकारास}‚{\tiny $_{७}$}‚म्ब‚न्धा ब‚ह‚वोर्था इत्य‚त्रान्त‚रे \href{http://sarit.indology.info/?cref=pv.3.198}{१ । २०१} \leavevmode\ledsidenote{\textenglish{140b/PSVTa}}‚{\tiny $_{lb}$}‚ उक्त‚त्वात् ।
	{\color{gray}{\rmlatinfont\textsuperscript{§~\theparCount}}}
	\pend% ending standard par
      ‚{\tiny $_{lb}$}‚

	  
	  \pstart \leavevmode% starting standard par
	एत‚च्च बाह्येषु आग‚म‚स्य प्रामाण्य‚म‚भ्युप‚ग‚म्योक्त‚म् [।]
	{\color{gray}{\rmlatinfont\textsuperscript{§~\theparCount}}}
	\pend% ending standard par
      ‚{\tiny $_{lb}$}‚

	  
	  \pstart \leavevmode% starting standard par
	अधुना नैव बाह्येर्थ‚स्य प्रामाण्य‚मित्याह । \textbf{अपि चे}त्यादि । \textbf{व‚स्तुभि}स्स्व‚ल‚क्ष‚{\tiny $_{lb}$}‚\textbf{णैस्स‚ह} । श‚ब्दा\textbf{नान्त‚रीय‚क‚ताया} अविनाभाव‚स्याभावात् तेभ्यः श‚ब्देभ्यो \textbf{नार्थ‚सिद्धिर्न‚{\tiny $_{lb}$}‚ बा}ह्य‚व‚स्तुनिश्च‚यः । य‚स्मा\textbf{त्ते हि व‚क्त्र‚भिप्राय‚सूच‚काः} ।
	{\color{gray}{\rmlatinfont\textsuperscript{§~\theparCount}}}
	\pend% ending standard par
      ‚{\tiny $_{lb}$}‚

	  
	  \pstart \leavevmode% starting standard par
	य‚द्य‚पि घ‚ट‚विव‚क्षातः प‚ट‚श‚ब्द‚स्योत्प‚त्ति‚{\tiny $_{१}$}‚स्त‚थापि स्थान‚क‚र‚णाभिघातादेरेव‚{\tiny $_{lb}$}‚ साक्षात् क‚र‚णात् त‚दुत्प‚त्तेर्व्य‚भिचाराभावान्नाहेतुक‚त्वं [।] य‚श्च घ‚ट‚विव‚क्षाज‚न्यं‚{\tiny $_{lb}$}‚ घ‚ट‚श‚ब्द‚म‚व‚धार‚य‚ति । त‚स्य प‚ट‚श‚ब्दात् प‚ट‚विव‚क्षानुमान‚न्त‚द‚व‚धार‚णं च प्र‚क‚र‚णा‚{\tiny $_{lb}$}‚दिना लोक‚स्य विद्य‚त एवेति विव‚क्षानुमानेऽव्य‚भिचार एव ।
	{\color{gray}{\rmlatinfont\textsuperscript{§~\theparCount}}}
	\pend% ending standard par
      ‚{\tiny $_{lb}$}‚‚{\tiny $_{lb}$}‚\textsuperscript{\textenglish{390/s}}

	  
	  \pstart \leavevmode% starting standard par
	\textbf{य‚थाभावं} य‚थास्व‚ल‚क्ष‚णं । \textbf{य‚त} इति य‚थाभावं प्र‚वृत्तेः श‚ब्देभ्यो\textbf{र्थ‚प्र‚कृति}र‚र्थ‚{\tiny $_{lb}$}‚स्व‚भावो \textbf{नि‚{\tiny $_{२}$}‚श्चीयेत । ते ही}ति श‚ब्दाः । विव‚क्ष‚या वृत्तिर्येषान्ते त‚थोक्ताः ।‚{\tiny $_{lb}$}‚ \textbf{त‚त्रान्त‚रीय‚का} विव‚क्षानान्त‚रीय‚काः । \textbf{तामेव} विव‚क्षां । सैव विव‚क्षा स्व‚ल‚क्ष‚ण‚{\tiny $_{lb}$}‚म‚न्त‚रेण न भ‚व‚ति [।] अतोस्त्येव श‚ब्दानाम‚र्थाव्य‚भिचार इत्याह । \textbf{न चे}त्यादि ।‚{\tiny $_{lb}$}‚ \textbf{य‚थार्थं} य‚थाव‚स्तु भ‚वितुं शीलं यासान्तास्त‚थोक्ताः । क्षीण‚दोष‚स्य कृपालोर्य‚थार्थ‚{\tiny $_{lb}$}‚भाविनी स‚र्वा इत्याह । अर्थेऽप्र‚तिब‚द्धोपि श‚ब्दार्थं ग‚म‚{\tiny $_{३}$}‚यिष्य‚तीति चेदाह । \textbf{न‚{\tiny $_{lb}$}‚ चे}त्यादि । त‚स्मिन् व‚स्तुन्य‚प्र‚तिब‚द्धः स्व‚भावो य‚स्य श‚ब्द‚ल‚क्ष‚ण‚स्य । अन्यं य‚त्रासौ न‚{\tiny $_{lb}$}‚ प्र‚तिब‚द्धः ॥
	{\color{gray}{\rmlatinfont\textsuperscript{§~\theparCount}}}
	\pend% ending standard par
      ‚{\tiny $_{lb}$}‚

	  
	  \pstart \leavevmode% starting standard par
	य‚दि बाह्ये व‚स्तुनि श‚ब्द‚स्य नास्ति प्रामाण्यं य‚त्त‚र्हीद\textbf{माप्त‚वादाविस‚म्वाद‚{\tiny $_{lb}$}‚सामान्यात्} । यो य आप्त‚वादः सोऽ\textbf{विस‚म्वादी} । य‚था क्ष‚णिकाः स‚र्वे संस्कारा इत्या‚{\tiny $_{lb}$}‚दिकः । आप्त‚वाद‚श्चाय‚म‚त्य‚न्त‚प‚रोक्षेप्य‚र्थे त‚स्माद‚य‚म‚प्य‚वि‚{\tiny $_{४}$}‚स‚म्वादीत्येव‚माप्त‚{\tiny $_{lb}$}‚वाद‚स्याविस‚म्वाद‚सामान्याद‚विस‚म्वादित्वाद‚नुमान‚तेत्याग‚म‚स्य बाह्येर्थेनुमान‚त्व‚मु‚{\tiny $_{lb}$}‚क्त‚माचार्य दि ग्ना गे न ।
	{\color{gray}{\rmlatinfont\textsuperscript{§~\theparCount}}}
	\pend% ending standard par
      ‚{\tiny $_{lb}$}‚

	  
	  \pstart \leavevmode% starting standard par
	\textbf{त‚त्क‚थ}मित्य‚नेनाभ्युपेत‚बाधामाह । \textbf{नाय}मित्यादिना प‚रिह‚र‚ति । एत‚त् क‚थ‚{\tiny $_{lb}$}‚य‚ति [।] नाचार्येण भाविकं प्रामाण्यं क‚थ‚य‚ता अनुमान‚त्व‚माग‚म‚स्योक्त‚म‚पि तु‚{\tiny $_{lb}$}‚ पुरुष‚प्र‚वृत्तिम‚पेक्ष्य । य‚स्मा\textbf{न्नाय‚म्पुरुषः} प्र‚वृत्तिकामः‚{\tiny $_{५}$}‚ \textbf{आग‚म‚प्रामाण्य‚म‚नाश्रित्या‚{\tiny $_{lb}$}‚सितुं स‚म‚र्थः} । किङ्कार‚णं । \textbf{प्र‚त्य‚क्षं} प‚रोक्षं फ‚लं येषान्तेषां केषांचित् \textbf{प्र‚वृत्तिनिवृ‚{\tiny $_{lb}$}‚त्त्योरिति} य‚थायोगं स‚म्ब‚न्धः । हिंसादिचेत‚नाविष‚याणां निवृत्तेः स्व‚र्गादिफ‚ल‚त्वेन‚{\tiny $_{lb}$}‚ \textbf{म‚हानुशंसाश्र‚व‚णात्} । हिंसाचेत‚नाविशेषाणां प्र‚वृत्तेर्न‚र‚कादिफ‚ल‚त्वेन म‚हा\textbf{पाप‚श्र‚व‚ण‚{\tiny $_{lb}$}‚त्वात्} । न चात्र व‚स्तुब‚ल‚प्र‚वृत्त‚म‚न्य‚त्प्र‚माणं साध‚क‚म‚{\tiny $_{६}$}‚स्ति येनाग‚म‚म‚न‚पेक्ष्यान्य‚तः‚{\tiny $_{lb}$}‚ प्र‚माण‚म्प्र‚व‚र्त्तेत ।
	{\color{gray}{\rmlatinfont\textsuperscript{§~\theparCount}}}
	\pend% ending standard par
      ‚{\tiny $_{lb}$}‚

	  
	  \pstart \leavevmode% starting standard par
	नापि बाध‚क‚म‚स्ति य‚तो निव‚र्त्तेत । बाध‚काभाव‚मेवाह । \textbf{त‚द्भाव} इति य‚थो‚{\tiny $_{lb}$}‚क्ताभ्यां प्र‚वृत्तिनिवृत्तिभ्यामिष्टानिष्ट‚स्य फ‚ल‚स्य भावे \textbf{विरोधाद‚र्श‚नाच्च} । इच्छ‚ति‚{\tiny $_{lb}$}‚ चाय‚माग‚म‚व‚शेन प्र‚व‚र्त्तितुं त‚त् \textbf{स‚ति प्र‚व‚र्त्तित‚व्ये} । \href{http://sarit.indology.info/?cref=pv.3.212-3.213}{२१५-१६} ?
	{\color{gray}{\rmlatinfont\textsuperscript{§~\theparCount}}}
	\pend% ending standard par
      ‚{\tiny $_{lb}$}‚‚{\tiny $_{lb}$}‚\textsuperscript{\textenglish{391/s}}

	  
	  \pstart \leavevmode% starting standard par
	व‚र\textbf{मेव‚मा}ग‚म‚म्प‚रीक्ष्य \textbf{प्र‚वृत्त इत्या}ग‚म‚स्य \textbf{प‚रीक्ष‚या प्रामाण्य‚मा}हा चा र्यः‚{\tiny $_{७}$}‚ । \leavevmode\ledsidenote{\textenglish{141a/PSVTa}}‚{\tiny $_{lb}$}‚ न च स‚र्वं शास्त्रं प‚रीक्ष‚याधिकृतं । किन्तु \textbf{त‚च्चे}त्यादि शास्त्रं । प‚दार्थानाम्प‚र‚स्प‚र‚{\tiny $_{lb}$}‚\textbf{स‚म्ब‚न्धात्} स‚म्ब‚द्धं । \textbf{अनुगुणः} साध‚यितुं श‚क्यः फ‚ल‚साध‚नो\textbf{पायो} य‚स्मिंस्त‚द‚नु‚{\tiny $_{lb}$}‚गुणोपायं स‚म्ब‚द्धं च त‚द‚नुगुणोपायं चेति विशेष‚ण‚स‚मासः । एवं भूत‚म‚पि य‚दि पुरु‚{\tiny $_{lb}$}‚\textbf{षार्थाभिधाय‚क}न्त‚दा \textbf{प‚रीक्षाधिकृतं वाक्यं । अतो} य‚थोक्त‚स्व‚भावाद‚प‚र‚म‚न्य‚द्वाक्य‚{\tiny $_{lb}$}‚\textbf{म‚न‚धि‚{\tiny $_{१}$}‚कृत‚म्}प‚रीक्षायां ।
	{\color{gray}{\rmlatinfont\textsuperscript{§~\theparCount}}}
	\pend% ending standard par
      ‚{\tiny $_{lb}$}‚

	  
	  \pstart \leavevmode% starting standard par
	\textbf{स‚म्ब‚न्ध} इत्यादिना व्याच‚ष्टे । \textbf{वाक्याना}म‚ङ्गाङ्गिभावे\textbf{नैक‚स्मिन्न‚र्थे} विधेय‚{\tiny $_{lb}$}‚प्र‚तिषेध्य‚ल‚क्ष‚णे \textbf{उप‚संहारो} मील‚न‚न्ते\textbf{नोप‚कारः} प‚र‚स्प‚रं वाक्यानां स‚म्ब‚न्धः । अनु‚{\tiny $_{lb}$}‚प‚कार‚कः पुनः केषामित्याह । \textbf{द‚श दाडि}मेत्यादि द‚श दाडिमानि ष‚ड‚पूपाः कुण्ड‚{\tiny $_{lb}$}‚म‚जाजिनं प‚ल‚ल‚मित्येव‚मादीनि वाक्यानि । न ह्येषामेकार्थोप‚संहारोस्ति प‚र‚स्प‚{\tiny $_{lb}$}‚र‚म‚{\tiny $_{२}$}‚स‚म्ब‚न्धात् । \textbf{अन‚य‚थे}त्य‚स‚म्ब‚द्ध‚त्वे \textbf{व‚क्तुः} शास्त्र‚कार‚स्य \textbf{वैगुण्य}म‚स‚म्ब‚द्धा‚{\tiny $_{lb}$}‚भिधायित्व\textbf{मुद्भाव‚येत्} । अश‚क्योपायो येषां फ‚लान‚न्तान्य‚श‚क्योपायानि । एवं‚{\tiny $_{lb}$}‚ भूतानि फ‚लानि येषां शास्त्राणान्तानि । \textbf{फ‚लार्थी} पुमान्नाद्रियेत \textbf{विचार‚यितुं} । एत‚{\tiny $_{lb}$}‚च्चानुगुणोपाय‚मित्येत‚स्य वैध‚र्म्येण विव‚र‚णं ।
	{\color{gray}{\rmlatinfont\textsuperscript{§~\theparCount}}}
	\pend% ending standard par
      ‚{\tiny $_{lb}$}‚

	  
	  \pstart \leavevmode% starting standard par
	पुरुषार्थः फ‚लं येषां शास्त्राणां तानि त‚थोक्तानि । त‚तोन्या\textbf{न्य‚पुरु‚{\tiny $_{३}$}‚षार्थ‚{\tiny $_{lb}$}‚फ‚लानि} । तानि \textbf{च} नाद्रियेत विचार‚यितुं ।
	{\color{gray}{\rmlatinfont\textsuperscript{§~\theparCount}}}
	\pend% ending standard par
      ‚{\tiny $_{lb}$}‚

	  
	  \pstart \leavevmode% starting standard par
	अश‚क्योपाय‚फ‚ल‚स्योदाह‚र‚णं । \textbf{विषे}त्यादि । एवं ह्य‚स्य विषं शाम्य‚ति य‚दि‚{\tiny $_{lb}$}‚ \textbf{त‚क्ष‚क}नाग‚राज‚स्य क‚र्णाव‚स्थितेन र‚त्नेनाल‚ङ्कारः क्रिय‚त इति । एत‚च्चाश‚क्य‚{\tiny $_{lb}$}‚साध‚नं ।
	{\color{gray}{\rmlatinfont\textsuperscript{§~\theparCount}}}
	\pend% ending standard par
      ‚{\tiny $_{lb}$}‚

	  
	  \pstart \leavevmode% starting standard par
	अपुरुषार्थ‚फ‚ल‚स्योदाह‚र‚णं । \textbf{काक‚द‚न्त‚प‚रीक्षाव‚च्चेति} ।
	{\color{gray}{\rmlatinfont\textsuperscript{§~\theparCount}}}
	\pend% ending standard par
      ‚{\tiny $_{lb}$}‚

	  
	  \pstart \leavevmode% starting standard par
	\textbf{त‚द्विप‚र्य‚येण} तेषां य‚थोक्तानान्त्र‚याणां विप‚र्येयेणो\textbf{प‚संहार‚व‚त्} । ए‚{\tiny $_{४}$}‚तेन स‚म्ब‚द्ध‚{\tiny $_{lb}$}‚मित्येत‚द् व्याख्यातं । \textbf{श‚क्योपाय‚मि}त्य‚नेनानुगुणोपाय‚मिति । \textbf{अन्य‚त्रे}ति स‚म्ब‚न्धादि‚{\tiny $_{lb}$}‚‚{\tiny $_{lb}$}‚ \leavevmode\ledsidenote{\textenglish{392/s}}त्र‚य‚र‚हिते । \textbf{अव‚धान‚स्यै}वेत्याद‚र‚स्य । \textbf{त‚दि}ति य‚थोक्त‚गुण‚त्र‚य‚युक्तं शास्त्रं‚{\tiny $_{lb}$}‚ \textbf{प‚रीक्षायां} स‚त्यां न विस‚म्वाद‚भाग् भ‚व‚ति । \textbf{त‚स्मिन्न‚विस‚म्वाद‚भाजि} शास्त्रे \textbf{प्र‚व‚र्त्त‚{\tiny $_{lb}$}‚मानः} पुरुषः \textbf{शोभ‚ते} ।
	{\color{gray}{\rmlatinfont\textsuperscript{§~\theparCount}}}
	\pend% ending standard par
      ‚{\tiny $_{lb}$}‚

	  
	  \pstart \leavevmode% starting standard par
	\textbf{कः पुन‚र‚स्य} शास्त्र‚स्या\textbf{विस‚म्वाद} इत्याह । \textbf{प्र‚त्य‚क्षे}णेत्यादि प्र‚थ‚मोर्थः श‚{\tiny $_{५}$}‚ब्दो‚{\tiny $_{lb}$}‚ व‚स्तुव‚च‚नः । द्वितीयो विष‚य‚व‚च‚नः । तेनाय‚म‚र्थः । दृष्टादृष्ट‚योर्व‚स्तुनोस्त‚द‚र्थ‚योः‚{\tiny $_{lb}$}‚ प्र‚त्य‚क्षानुमान‚विष‚य‚योर्य‚थाक्र‚मं प्र‚त्य‚क्षेणानुमानेन च द्विविधेन व‚स्तुब‚ल‚प्र‚वृत्तेनाग‚{\tiny $_{lb}$}‚माश्रितेन चाबाध‚न‚म‚स्य शास्त्र‚स्याविस‚म्वादः । \href{http://sarit.indology.info/?cref=pv.3.213-3.214}{२१६-१७} ।
	{\color{gray}{\rmlatinfont\textsuperscript{§~\theparCount}}}
	\pend% ending standard par
      ‚{\tiny $_{lb}$}‚

	  
	  \pstart \leavevmode% starting standard par
	\textbf{प्र‚त्य‚क्षे}त्यादिना व्याच‚ष्टे । शास्त्रे \textbf{प्र‚त्य‚क्षाभिम‚ताना}म्प्र‚त्य‚क्ष‚त्वेनोप‚ग‚ता‚{\tiny $_{lb}$}‚ना\textbf{म‚र्थानान्त‚थाभावः} प्र‚त्य‚क्ष‚भा‚{\tiny $_{६}$}‚वः \textbf{प्र‚त्य‚क्षेणाबाध‚नं} ।
	{\color{gray}{\rmlatinfont\textsuperscript{§~\theparCount}}}
	\pend% ending standard par
      ‚{\tiny $_{lb}$}‚

	  
	  \pstart \leavevmode% starting standard par
	\textbf{य‚थे}त्यादिना स्व‚सिद्धान्ते प्र‚त्य‚क्षाभिम‚त‚म‚र्थं प‚ञ्च\textbf{स्क‚न्ध}संगृहीत‚न्द‚र्श‚य‚ति ।‚{\tiny $_{lb}$}‚ \textbf{नीलादी}त्य‚नेन रूपादीन् प‚ञ्च विष‚यानाह । अनेन च रूप‚स्क‚न्ध उक्तः । सुख‚{\tiny $_{lb}$}‚\textbf{दुःखे} इति वे द ना स्क‚न्धः । \textbf{निमित्त}स्य स्त्रीपुरुषादिचिह्न‚स्यो\textbf{प‚ल‚क्ष‚णं} निमित्तो‚{\tiny $_{lb}$}‚प‚ल‚क्ष‚णं । अनेन सं ज्ञा स्क‚न्धः । \textbf{रागादि}ग्र‚ह‚णेन सं स्का र स्क‚न्धः । आदिश‚ब्दाद्‚{\tiny $_{lb}$}‚ \leavevmode\ledsidenote{\textenglish{141b/PSVTa}} द्वेष‚{\tiny $_{७}$}‚मोहादिप‚रिग्र‚हः । \textbf{बुद्धि}ग्र‚ह‚णेन वि ज्ञा न स्क‚न्धः । नीलादि च सुख‚दुःखे च‚{\tiny $_{lb}$}‚ निमित्तोप‚ल‚क्ष‚णं च रागादि च बुद्धिश्चेति द्व‚न्द्वः । एवं शास्त्रे प्र‚त्य‚क्षाभिम‚तानां‚{\tiny $_{lb}$}‚ प्र‚त्य‚क्ष‚त्व‚मेव । नीलादीनां च‚क्षुर्विज्ञानादिप्र‚त्य‚क्ष‚त्वात् [।] सुखादीनां स्व‚स‚म्वेद‚न‚{\tiny $_{lb}$}‚प्र‚त्य‚क्ष‚त्वात् । \textbf{अत‚थाभिम‚तानां} चाप्र‚त्य‚क्षाभिम‚ता नां चार्थाना\textbf{म‚प्र‚त्य‚क्ष‚ता} [।]‚{\tiny $_{lb}$}‚ प्र‚त्य‚क्षेणाव‚धान‚मिति प्र‚कृतेन स‚म्ब‚न्धः‚{\tiny $_{१}$}‚ । \textbf{य‚थे}ति विष‚योप‚द‚र्श‚नं । श‚ब्दादिरूपेण‚{\tiny $_{lb}$}‚ स‚न्निवेष्टुं शीलं येषां सुखादीनान्तेषाम‚प्र‚त्य‚क्ष‚ता । श‚ब्दादिस्व‚भावानां सुख‚दुःख‚{\tiny $_{lb}$}‚मोहानां प्र‚त्य‚क्षेणाप्र‚तीतेः
	{\color{gray}{\rmlatinfont\textsuperscript{§~\theparCount}}}
	\pend% ending standard par
      ‚{\tiny $_{lb}$}‚

	  
	  \pstart \leavevmode% starting standard par
	एत‚त् सां ख्य द‚र्श‚न‚प्र‚तिक्षेपेणोक्तं ।
	{\color{gray}{\rmlatinfont\textsuperscript{§~\theparCount}}}
	\pend% ending standard par
      ‚{\tiny $_{lb}$}‚

	  
	  \pstart \leavevmode% starting standard par
	वै शे षि का दि द‚र्श‚न‚प्र‚तिक्षेपेणाह । \textbf{द्र‚व्यं} द्विविधं । अद्र‚व्यं द्र‚व्यं य‚थाका‚{\tiny $_{lb}$}‚‚{\tiny $_{lb}$}‚ \leavevmode\ledsidenote{\textenglish{393/s}}शादि । अनेक‚द्र‚व्यं च द्र‚व्यं । य‚थाव‚य‚वि द्र‚व्यं । \textbf{क‚र्मो}त्क्षेप‚णादिकं । \textbf{सामा‚{\tiny $_{२}$}‚न्यं}‚{\tiny $_{lb}$}‚ स‚त्ता गोत्वादिकं [।] \textbf{आदि}श‚ब्दाद् विभागादिप‚रिग्र‚हः ।
	{\color{gray}{\rmlatinfont\textsuperscript{§~\theparCount}}}
	\pend% ending standard par
      ‚{\tiny $_{lb}$}‚

	  
	  \pstart \leavevmode% starting standard par
	न हि नीलादिविष‚यं पंच व्य‚तिरेकेणान्य‚स्य प्र‚त्य‚क्ष‚तास्ति । त‚द्व्य‚तिरेकेणानु‚{\tiny $_{lb}$}‚प‚ल‚ब्धेः । \textbf{व‚स्तुब‚ल‚प्र‚वृत्त‚म‚नुमान‚म‚नाग‚मापेक्षानुमानं} । त‚स्य \textbf{विष‚य}त्वेना\textbf{भिम‚ता‚{\tiny $_{lb}$}‚नान्त‚था}भावोनुमान‚विष‚य‚भावोनुमानेनाबाध‚कं । \textbf{च‚त्वारि चा र्य स‚त्यानि} । उत्त‚र‚त्र‚{\tiny $_{lb}$}‚ प्र‚तिपाद‚यिष्य‚ते । \textbf{अन‚नुमेयानान्त‚{\tiny $_{३}$}‚थाभावो}ऽन‚नुमेय‚त्व‚म‚नुमानेनाबाध‚नं । \textbf{य‚थात्मा‚{\tiny $_{lb}$}‚दीनाम्} [।] आदिश‚ब्दात् प्र‚धानेश्व‚रादिप‚रिग्र‚हः । न ह्येषां किञ्चिल्लिङ्ग‚म‚स्ति‚{\tiny $_{lb}$}‚ येनानुमेयाः स्युः । एत‚द‚पि प्र‚तिपाद‚यिष्य‚ति । विशुद्धे विष‚य‚द्व‚ये \href{http://sarit.indology.info/?cref=}{ } ऽत्य‚न्त‚{\tiny $_{lb}$}‚प‚रोक्षे चाग‚म‚विष‚ये पौर्वाप‚र्य‚विरोधेन य‚स्मिन् चिन्तां प्र‚व‚र्त्त‚य‚ति त‚स्मि\textbf{न्नाग‚मापेक्ष‚{\tiny $_{lb}$}‚म‚नुमान‚म‚पि} । अबाध‚न‚मिति प्र‚कृतं ।
	{\color{gray}{\rmlatinfont\textsuperscript{§~\theparCount}}}
	\pend% ending standard par
      ‚{\tiny $_{lb}$}‚

	  
	  \pstart \leavevmode% starting standard par
	कीदृश‚न्त‚द‚बाध‚{\tiny $_{४}$}‚न‚मित्याह । \textbf{रागादि}रूपं राग‚द्वेष‚मोह‚स्व‚भाव\textbf{म‚ध‚र्म‚म‚भ्युप‚ग‚म्य‚{\tiny $_{lb}$}‚ त‚त्प्र‚भ‚वं} रागादिस‚मुत्थापितं काय‚वाक्क‚र्म चाध‚र्म‚म‚भ्युप‚ग‚प्य । \textbf{त‚त्प्र‚हाणाय} त‚स्या‚{\tiny $_{lb}$}‚ध‚र्म‚स्याप‚ग‚माय \textbf{स्नानाग्निहोत्रादेः} । तीर्थ‚स्नानेन पाप‚क्ष‚यो भ‚व‚ति । य‚मुद्दिश्याग्नौ‚{\tiny $_{lb}$}‚ घृतादिकं हूय‚ते त‚स्य पाप‚क्ष‚यो भ‚व‚तीत्येव‚मादे\textbf{र‚नुप‚देशः} । आदिश‚ब्दादुप‚वासा‚{\tiny $_{lb}$}‚दिप‚रिग्र‚ह‚{\tiny $_{५}$}‚ः । त‚था हि न स्नानादि पाप‚म‚प‚न‚य‚ति [।] पाप‚निदानेन रागादिना‚{\tiny $_{lb}$}‚ विरोधाभावात् ।
	{\color{gray}{\rmlatinfont\textsuperscript{§~\theparCount}}}
	\pend% ending standard par
      ‚{\tiny $_{lb}$}‚

	  
	  \pstart \leavevmode% starting standard par
	\textbf{सेय‚म}न‚न्त‚रोक्ताप्र‚त्य‚क्षेणानुमानेन द्विविधेन \textbf{श‚क्य‚प‚रिच्छेद‚स्य} श‚क्य‚निश्च‚य‚{\tiny $_{lb}$}‚स्या\textbf{शेष‚स्य विष‚य‚स्य} बाधाल‚क्ष‚णा ॥ त‚स्यास्ताव‚द‚स्या विस‚म्वादाद् \textbf{सामान्यात्} । य‚था‚{\tiny $_{lb}$}‚ श‚क्य‚प‚रिच्छेदेर्थे \textbf{आप्त‚वाद‚स्याविस‚म्वाद}स्त‚थात्य‚न्त‚प‚रोक्षेपि आप्त‚वाद‚त्वादेव ।‚{\tiny $_{lb}$}‚ त‚त‚श्चाप्त‚वाद‚ल‚क्ष‚णा‚{\tiny $_{६}$}‚ल्लिङ्गादुत्प‚न्नाया अविस‚म्वाद\textbf{बुद्धेर‚नुमान‚ता}चार्य दि ग्ना‚{\tiny $_{lb}$}‚गे ना \textbf{भिहिता} \href{http://sarit.indology.info/?cref=pv.3.214-3.215}{२१७-१८}
	{\color{gray}{\rmlatinfont\textsuperscript{§~\theparCount}}}
	\pend% ending standard par
      ‚{\tiny $_{lb}$}‚

	  
	  \pstart \leavevmode% starting standard par
	\textbf{प‚रोक्षेप्य‚र्थ‚स्य गोच‚र} इत्य‚त्य‚न्त‚प‚रोक्षेप्य‚स्य शास्त्र‚स्य गोच‚रे विष‚ये । सा‚{\tiny $_{lb}$}‚ \textbf{चाग‚त्या}भिहितान्येन प्र‚कारेणात्य‚न्त‚प‚रोक्षे प्र‚वृत्त्य‚स‚म्भ‚वात् । स‚त्यां प्र‚वृत्तौ व‚र‚मेवं‚{\tiny $_{lb}$}‚ ‚{\tiny $_{lb}$}‚ \leavevmode\ledsidenote{\textenglish{394/s}}प्र‚वृत्त इति ।
	{\color{gray}{\rmlatinfont\textsuperscript{§~\theparCount}}}
	\pend% ending standard par
      ‚{\tiny $_{lb}$}‚

	  
	  \pstart \leavevmode% starting standard par
	\leavevmode\ledsidenote{\textenglish{142a/PSVTa}} \textbf{त‚स्ये}त्यादि विव‚र‚णं । त‚स्याग‚म‚स्याचार्य दि ग्ना गे न निर्दिष्टानुमान‚भाव‚{\tiny $_{७}$}‚‚{\tiny $_{lb}$}‚स्य । \textbf{अस्ये}त्य‚स्माभि\textbf{स्स‚म्ब‚न्धाद‚नुगुणोपाय}मित्यादिना विचारित‚स्य । अत एवाह‚{\tiny $_{lb}$}‚ [।] \textbf{एवंभूत‚स्येति} स‚म्ब‚न्धादिगुण‚युक्त‚स्येत्य‚र्थः । प्र‚त्य‚क्षानुमान‚ग‚म्ये त‚स्मिन्‚{\tiny $_{lb}$}‚ व‚स्तु\textbf{न्य‚विस‚म्वाद‚सामान्याद}विस‚म्वाद‚त्वात् कार‚णाद् \textbf{दृष्ट‚व्य‚भिचार‚स्या}प‚वाद‚स्य‚{\tiny $_{lb}$}‚ \textbf{प्र‚त्य‚क्षानुमानाग‚म्येर्थे} विष‚ये । आप्त‚वादाद‚नुत्प‚न्नायाः \textbf{प्र‚तिप‚त्ते}र्बुद्धेर‚विस‚म्वादो‚{\tiny $_{lb}$}‚नुमीय‚ते । क‚{\tiny $_{१}$}‚स्मात् [।] \textbf{त‚दाश्र‚य‚त्वा}दाप्त‚वादाश्र‚य‚त्वं चाचार्य‚पार‚म्प‚र्योप‚देशात्‚{\tiny $_{lb}$}‚ सिद्धं । \textbf{त‚द‚न्य‚प्र‚तिप‚त्तिव‚त्} । अत्य‚न्त‚प‚रोक्षाद‚न्य‚स्मिन् विष‚ये प्र‚त्य‚क्षानुमान‚विष‚ये‚{\tiny $_{lb}$}‚ प्र‚तिप‚त्तिव‚त् । \textbf{त‚तो} य‚थोक्तादाप्त‚वादादत्य‚न्त‚प‚रोक्षेर्थे य‚थोक्ताग‚माश्रिता बुद्धिः‚{\tiny $_{lb}$}‚ \textbf{श‚ब्द‚प्र‚भ‚वापि} स‚ती श‚ब्दादुत्प‚न्नापि । \textbf{शाब्द‚व‚दि}ति य‚थान्यः शाब्दः प्र‚त्य‚योभि‚{\tiny $_{lb}$}‚प्राय‚मात्रं‚{\tiny $_{२}$}‚ निवेद‚य‚ति । \textbf{त‚था ने}यं बुद्धि\textbf{र‚भिप्राय‚मेव निवेद‚य‚ति} । एव‚कार‚स्य‚{\tiny $_{lb}$}‚ भिन्न‚क्र‚म‚त्वात् ।
	{\color{gray}{\rmlatinfont\textsuperscript{§~\theparCount}}}
	\pend% ending standard par
      ‚{\tiny $_{lb}$}‚

	  
	  \pstart \leavevmode% starting standard par
	किन्त‚र्ही‚{\tiny $_{४}$}‚ष्ट‚स्य प्र‚त्य‚क्षानुमानाग‚म्य‚स्यार्थ‚स्यान‚न्त‚रोक्तेन न्याये\textbf{नाविस‚म्वा‚{\tiny $_{lb}$}‚दाद‚नुमान‚म‚पि} प्र‚वृत्तिकाम‚स्य पुंसोभिप्राय‚व‚शात् । व‚स्तुत‚स्त्व‚न‚नुमानं श‚ब्दाना‚{\tiny $_{lb}$}‚म‚र्थैस्स‚ह स‚म्ब‚न्धाभावात् ।
	{\color{gray}{\rmlatinfont\textsuperscript{§~\theparCount}}}
	\pend% ending standard par
      ‚{\tiny $_{lb}$}‚

	  
	  \pstart \leavevmode% starting standard par
	अस्यैवार्थ‚स्य ख्याप‚नार्थोऽपिश‚ब्दः प्र‚कारान्त‚रेणाप्त‚{\tiny $_{३}$}‚वादाविस‚म्वादं द‚र्श‚य‚{\tiny $_{lb}$}‚न्नाह । \textbf{अथ‚वे}त्यादि । हेयं स‚र्व‚न्दुःखं । उपादेयं स‚र्वं\textbf{क्लेश‚प्र‚हाण}न्निर्वाणं । त‚यो‚{\tiny $_{lb}$}‚स्त‚त्त्व‚म‚विप‚रीतं रूप‚न्दुःख‚स‚त्य‚स्य निरोध‚स‚त्य‚स्य च । स‚ह उपायेन य‚द् व‚र्त्त‚ते हेयो‚{\tiny $_{lb}$}‚पादेय‚त‚त्त्व‚न्त‚स्योपायं । \textbf{हेय}स्योपायः स मु द य स‚त्त्यं । उपादेय‚स्योपादेय\edtext{}{\lemma{स्योपादेय}\Bfootnote{? यो}}‚{\tiny $_{lb}$}‚ मार्ग‚स‚त्यं । अस्य \textbf{हेयोपादेय‚त‚त्त्व‚स्य सोपाय‚स्य} भ ग ‚{\tiny $_{४}$}‚व द्द‚र्शित‚स्य व‚स्तुब‚लाया‚{\tiny $_{lb}$}‚तेन प्र‚माणेन \textbf{प्र‚सिद्धितो} निश्च‚य‚तः कार‚णात् । भ‚ग‚व‚द्व‚च‚ने स‚त्त्य‚च‚तुष्ट‚य‚ल‚क्ष‚ण‚स्य‚{\tiny $_{lb}$}‚ प्र‚धान‚स्यार्थ‚स्याविस‚म्वादः [।] स‚त्त्य‚च‚तुष्ट‚य‚ल‚क्ष‚णोर्थः प्र‚धान‚न्त‚द‚धिग‚मेन निर्वाण‚{\tiny $_{lb}$}‚प्राप्तेः । त‚स्मात् \textbf{प्र‚धानार्थाविस‚म्वादात्} । भ‚ग‚व‚द्व‚च‚नादुत्प‚न्नं ज्ञानं \textbf{प‚र‚त्रा}प्य‚{\tiny $_{lb}$}‚‚{\tiny $_{lb}$}‚ \leavevmode\ledsidenote{\textenglish{395/s}}त्य‚न्त‚प‚रोक्षेप्य‚र्थे\textbf{नुमान‚म्वा} श‚ब्दः पूर्व‚प्र‚कारापेक्ष‚{\tiny $_{५}$}‚ या विक‚ल्पार्थः ।
	{\color{gray}{\rmlatinfont\textsuperscript{§~\theparCount}}}
	\pend% ending standard par
      ‚{\tiny $_{lb}$}‚

	  
	  \pstart \leavevmode% starting standard par
	\textbf{त‚योर्हेयोपादेय‚योरुपायौ त‚दुपायौ} । हेयं चोपादेयं च त‚दुपायौ चेति द्व‚न्द्वः ।‚{\tiny $_{lb}$}‚ \textbf{तेषान्त‚दुप‚दिष्टाना}न्तेनाप्तेनोप‚दिष्टाना\textbf{म‚वैप‚रीत्य}म‚नुमानेन निरूप्य‚माणानाम‚वि‚{\tiny $_{lb}$}‚त‚थ‚त्व‚म\textbf{विंस‚म्वादः । य‚था च‚तुर्ण्णा}न्दुःखादीना\textbf{मार्य‚स‚त्यानां} द्वितीये प‚रिच्छेदे ।‚{\tiny $_{lb}$}‚ \textbf{व‚क्ष्य‚माण‚या नीत्या} विचारेण । \textbf{त‚स्या}स्येति भ‚ग‚व‚ता‚{\tiny $_{६}$}‚ पूर्व‚निर्द्दिष्ट‚स्याधुना विभ‚क्त‚{\tiny $_{lb}$}‚ त्वाद‚स्य स त्त्य च तु ष्ट‚य‚ल‚क्ष‚ण‚स्य । किंभूत‚स्य \textbf{पुरुषार्थोप‚योगिनः} । पुरुषार्थो‚{\tiny $_{lb}$}‚ निर्वाणं त‚त्रोप‚योगः कार‚ण‚त्वं स य‚स्यास्ति त‚था । अत एवा\textbf{भियो}गार्ह‚स्याभ्यासा‚{\tiny $_{lb}$}‚र्ह‚स्या\textbf{विस‚म्वादाद् विष‚यान्त‚रेपि} प्र‚त्य‚क्षानुमानाग‚म्ये ।‚{\tiny $_{७}$}‚ \textbf{त‚थात्वोप‚ग‚म} इत्य‚विस‚म्वा- \leavevmode\ledsidenote{\textenglish{142b/PSVTa}}‚{\tiny $_{lb}$}‚ दोप‚ग‚मो न विप्र‚ल‚म्भाय । न विस‚{\tiny $_{७}$}‚म्वादाय भ‚व‚ति । किं कार‚ण‚म् [।] \textbf{अनु‚{\tiny $_{lb}$}‚प‚रोधात्} । प्र‚माणेनाबाध‚नात् । प्र‚धाने च स‚त्त्य‚च‚तुष्ट‚य‚ल‚क्ष‚णेर्थे पुरुष‚म‚विस‚म्वाद्य‚{\tiny $_{lb}$}‚ पुन‚स्तृतीये स्थाने \textbf{व‚क्तुर्निष्प्र‚योज‚नं} य‚द्वित‚था\textbf{भिधान}न्त‚स्य \textbf{वैफ‚ल्यात्} ।
	{\color{gray}{\rmlatinfont\textsuperscript{§~\theparCount}}}
	\pend% ending standard par
      ‚{\tiny $_{lb}$}‚

	  
	  \pstart \leavevmode% starting standard par
	क‚दाचित् त‚त्राज्ञानाद‚पि त‚थाभिधानं स्यादिति चेदाह । \textbf{त‚दि}त्यादि । \textbf{उभ‚य‚था}‚{\tiny $_{lb}$}‚ पीति श्लोक‚द्व‚य‚निर्दिष्टेन प्र‚कारेण । त‚देत\textbf{दाग‚म‚स्यानुमान‚{\tiny $_{१}$}‚त्व}म‚ग‚त्यो\textbf{प‚व‚र्ण्णित‚{\tiny $_{lb}$}‚मिति} स‚म्ब‚न्धः । अनुमान‚कार‚ण‚त्वाद‚नुमान‚मिति द्र‚ष्ट‚व्यं । \textbf{आग‚मात् प्र‚वृत्तौ}‚{\tiny $_{lb}$}‚ स‚त्याम्व‚र‚मेवंयुक्तादाग‚मात् प्र‚वृत्तो न तु प्र‚माण‚ग‚म्य एवार्थे विस‚म्वाद‚का‚{\tiny $_{lb}$}‚दिति ।
	{\color{gray}{\rmlatinfont\textsuperscript{§~\theparCount}}}
	\pend% ending standard par
      ‚{\tiny $_{lb}$}‚

	  
	  \pstart \leavevmode% starting standard par
	\textbf{न ख‚ल्वे}व‚मुक्तेनापि प्र‚कारेणाग‚माश्र‚ये ज्ञान\textbf{म‚नुमान‚म‚न‚पायं} निर्दोषं । क‚स्माद्‚{\tiny $_{lb}$}‚ [।] अनान्त‚रीय‚क‚त्वाद‚स‚म्वाद‚त्वा\textbf{द‚र्थेषु} श‚ब्दानामिति निवेदित‚मेत\textbf{न्नान्त‚री‚{\tiny $_{२}$}‚‚{\tiny $_{lb}$}‚ य‚क‚ताभावाच्छ‚ब्दानाम्व‚स्तुभिः स‚हे} \href{http://sarit.indology.info/?cref=}{१ । २१३} त्यादिना [।] \textbf{पुरुष}स्या\textbf{तिश‚याः}‚{\tiny $_{lb}$}‚ ‚{\tiny $_{lb}$}‚ \leavevmode\ledsidenote{\textenglish{396/s}}क्षीण‚दोषादिक‚त्व‚न्त‚म‚पेक्ष्य‚ते य‚द्व‚च‚न‚न्त‚द् \textbf{य‚थार्थ‚म‚प‚रे} वादिनो \textbf{विदु}र्जानीयुः । \textbf{य‚थार्थं}‚{\tiny $_{lb}$}‚ य‚थाव‚स्तु व्य‚व‚स्थित‚न्त‚थैव द‚र्श‚नं ज्ञानं \textbf{य‚थार्थ‚द‚र्श‚नं} । त‚दादिर्य‚स्य कृपावैराग्यादे‚{\tiny $_{lb}$}‚स्त\textbf{द्य‚थार्थ‚द‚र्श‚नादि} । स एव \textbf{गुण}स्तेन युक्तः \textbf{पुरुष} आप्त‚स्ते\textbf{न प्र‚ण‚य}न‚न्तेनाप्त‚व‚च‚{\tiny $_{३}$}‚‚{\tiny $_{lb}$}‚ न‚माग‚म‚स्या\textbf{विस‚म्वाद इत्य‚न्ये} । \href{http://sarit.indology.info/?cref=pv.3.217-3.218}{२२०-२१}
	{\color{gray}{\rmlatinfont\textsuperscript{§~\theparCount}}}
	\pend% ending standard par
      ‚{\tiny $_{lb}$}‚

	  
	  \pstart \leavevmode% starting standard par
	\textbf{इष्ट} इत्यादि \textbf{सि द्धा न्त वा दी । योय}म‚न‚न्त‚रोक्तोर्थः स इष्टोस्माकं । किन्तु‚{\tiny $_{lb}$}‚ \textbf{श‚क्येत ज्ञातुं} पुरुष‚नैय‚म्येन \textbf{योतिश‚यो} य‚था द‚र्श‚नादिल‚क्ष‚ण‚स्य तु श‚क्यः ।
	{\color{gray}{\rmlatinfont\textsuperscript{§~\theparCount}}}
	\pend% ending standard par
      ‚{\tiny $_{lb}$}‚

	  
	  \pstart \leavevmode% starting standard par
	\textbf{स‚र्व एवेत्या}दिना व्याच‚ष्टे । \textbf{स‚र्व एव प्रेक्षापूर्व‚कारी । आग‚म‚म‚नाग‚म‚म्वान्वेष‚ते}‚{\tiny $_{lb}$}‚ निरूप‚य‚ति । अनाग‚म‚न्त्य‚क्त्वाग‚म‚द्वारेण \textbf{प्र‚वृत्तिकामः ।‚{\tiny $_{४}$}‚ न व्य‚स‚नेना}स‚क्तिमात्रेण ।‚{\tiny $_{lb}$}‚ प्र‚वृत्तिकाम‚ता\textbf{म‚पि ने}त्यादिनाह । \textbf{अपिना}मेति क‚थ‚न्नाम । \textbf{अत} इत्याग‚मा‚{\tiny $_{lb}$}‚\textbf{द‚नुष्ठेयं} साक्षात् क‚र्त्त‚व्य‚म‚र्थं \textbf{ज्ञात्वा प्र‚वृत्त स‚न्न‚र्थ‚वा}न् फ‚ल‚वान् \textbf{स्या}म्भ‚वेय\textbf{मिति} ।‚{\tiny $_{lb}$}‚ सोन्वेष‚माणः पुरुषः \textbf{श‚क्यं द‚र्श‚नं} निश्च‚यो य‚स्मिन्न‚र्थे प्र‚त्य‚क्षानुमान‚ग‚म्ये त‚स्या‚{\tiny $_{lb}$}‚\textbf{विस‚म्वादः} प्र‚त्य‚क्षानुमानाभ्याम‚व‚धानं । स एव प्र‚त्य‚योव‚{\tiny $_{५}$}‚ल‚म्ब‚न‚न्तेना\textbf{न्य‚त्रापि}‚{\tiny $_{lb}$}‚ प्र‚त्य‚क्षानुमानाग‚म्येप्य‚र्थे \textbf{प्र‚व‚र्त्तेत} । किं कार‚ण‚म् [।] एवं\textbf{प्राय‚त्वाल्लोक‚व्य‚व‚हार‚स्य}‚{\tiny $_{lb}$}‚ [।] एव‚मित्येक‚देशाविस‚म्वाद‚द‚र्श‚नेनान्य‚त्र प्र‚व‚र्त्त‚नं । \textbf{प्रायो} बाहुल्येन य‚स्मिन्‚{\tiny $_{lb}$}‚ लोक‚व्य‚व‚हारे स त‚थोक्तः ।
	{\color{gray}{\rmlatinfont\textsuperscript{§~\theparCount}}}
	\pend% ending standard par
      ‚{\tiny $_{lb}$}‚

	  
	  \pstart \leavevmode% starting standard par
	पुरुष‚प‚रीक्ष‚या पुनः प्र‚वृत्ताव‚भ्युप‚ग‚म्य‚मानायां \textbf{प्र‚वृत्तिरेव न स्यात्} । किं‚{\tiny $_{lb}$}‚ कार‚णं [।] त‚स्य पुंस\textbf{स्त‚थाभूत‚स्य} य‚थार्थ‚द‚र्श‚ना‚{\tiny $_{६}$}‚ दिगुण‚युक्त‚स्य । \textbf{नानिष्टे}र‚प्र‚वृत्तिरेव‚{\tiny $_{lb}$}‚ स्यादिति स‚म्ब‚ध्य‚ते । तेनाय‚म‚र्थो न पुन‚र्य‚थार्थ‚द‚र्श‚नादिगुण‚युक्तानां पुंसाम‚वित‚था‚{\tiny $_{lb}$}‚भिधायित्वेनानिष्टेर‚प्र‚वृत्तिरेव स्यात् । किं कार‚ण‚म् [।] \textbf{तादृशां} य‚थार्थ‚द‚र्श‚नादि‚{\tiny $_{lb}$}‚गुण‚युक्तानाम\textbf{वित‚थाभिधानात्} । य‚थाव‚स्थित‚व‚स्तुप्र‚काश‚क‚वात् ।
	{\color{gray}{\rmlatinfont\textsuperscript{§~\theparCount}}}
	\pend% ending standard par
      ‚{\tiny $_{lb}$}‚

	  
	  \pstart \leavevmode% starting standard par
	\leavevmode\ledsidenote{\textenglish{143a/PSVTa}} किं पुन कार‚ण‚न्त‚थाभूतः पुमान् ज्ञातुम‚श‚क्य इत्याह ।‚{\tiny $_{७}$}‚ \textbf{त‚था ही}त्यादि । \textbf{अयं}‚{\tiny $_{lb}$}‚ पुमाने\textbf{व‚न्दो}ष‚वान् । \textbf{न वा} । एव‚न्दोष‚वान् । किन्तु निर्दोष \textbf{इत्येव‚म‚न्य‚दोषानिर्दोष‚तापि‚{\tiny $_{lb}$}‚ ‚{\tiny $_{lb}$}‚ \leavevmode\ledsidenote{\textenglish{397/s}}वा दुर्बोधेत्य‚प‚रे विदुः} । निर्दोष‚तेत्येत‚द‚पेक्ष‚या दुर्बोधेत्येक‚व‚च‚नेन स्त्रीलिंगेन च‚{\tiny $_{lb}$}‚ निर्देशः । अन्य‚दोषा इत्येत‚द‚पेक्ष‚या तु दुर्बोधा इति पुल्लिङ्ग‚ब‚हुव‚च‚नाभ्यां विप‚रि‚{\tiny $_{lb}$}‚णामः क‚र्त्त‚व्यः । क‚स्माद् [।] दुर्बोध‚त्वाद् दुःप्राप्य‚त्वाद‚न्य‚{\tiny $_{१}$}‚गुण‚दोष‚निश्चाय‚कानां‚{\tiny $_{lb}$}‚ \textbf{प्र‚माणानां} [।]
	{\color{gray}{\rmlatinfont\textsuperscript{§~\theparCount}}}
	\pend% ending standard par
      ‚{\tiny $_{lb}$}‚

	  
	  \pstart \leavevmode% starting standard par
	\textbf{चैत‚सेभ्य} इत्यादिना व्याच‚ष्टे । \textbf{स‚म्य‚ग् मिथ्या} च प्र‚वृत्तिः काय‚वाक्क‚र्म‚{\tiny $_{lb}$}‚ल‚क्ष‚णा येषां पुंसान्ते त‚था । चेत‚सि भ‚वाः चैत‚सा गुण‚दोषाः । चैत‚सेभ्यो गुणेभ्यः‚{\tiny $_{lb}$}‚ कृपावैराग्य‚बोधादिभ्यो हेतुभ्यः \textbf{स‚म्य‚क्प्र‚वृत्त‚यः} य‚थार्थ‚प्र‚वृत्त‚यः । चैत‚सेभ्यो दोषेभ्यो‚{\tiny $_{lb}$}‚ रागादिभ्यो मिथ्याप्र‚वृत्त‚यो विप‚रीत‚प्र‚वृत्त‚{\tiny $_{२}$}‚ यो भ‚व‚न्ति । \textbf{ते चे}ति प‚रेषां चैत‚सा‚{\tiny $_{lb}$}‚ गुण‚दोषाश्चेतोध‚र्म‚त्वेना\textbf{तीन्द्रियाः} । त‚तो न प्र‚त्य‚क्ष‚ग‚म्याः । नाप्य‚नुमान‚ग‚म्या‚{\tiny $_{lb}$}‚ अतीन्द्रिय‚त्वादेव स्व‚भाव‚लिङ्ग‚स्यासिद्धेः । किन्तु \textbf{स्व}स्माद् गुण‚दोष‚रूपात् \textbf{प्र‚भ‚व}‚{\tiny $_{lb}$}‚ उत्पादो य‚स्य \textbf{काय‚वाक्क‚र्म‚णः} । तेन कार्य‚लिङ्गेना\textbf{नुमेयाः} । त‚च्च नास्ति ।‚{\tiny $_{lb}$}‚ य‚स्माद् \textbf{व्य‚व‚हाराश्च} काय‚वाक्क‚र्म‚ल‚क्ष‚णाः \textbf{प्राय‚शो} वा‚{\tiny $_{३}$}‚ ब‚हुल्येन । \textbf{बुद्धिपूर्व}मिति‚{\tiny $_{lb}$}‚ कृत्वा प्र‚तिसंख्यानेना\textbf{न्य‚थापि क‚र्त्तुं श‚क्य‚न्ते} । त‚था हि स‚रागा अपि वीत‚राग‚व‚द्‚{\tiny $_{lb}$}‚ आत्मान‚न्द‚र्श‚य‚न्ति । वीत‚रागाश्च स‚राग‚व‚त् । किं कार‚णं [।] \textbf{पुरुषेच्छावृत्तित्वाद्}‚{\tiny $_{lb}$}‚ व्य‚व‚हाराणां पुरुषेच्छ‚या \textbf{वृत्तिः} प्र‚वृत्तिर्येषामिति विग्र‚हः ।
	{\color{gray}{\rmlatinfont\textsuperscript{§~\theparCount}}}
	\pend% ending standard par
      ‚{\tiny $_{lb}$}‚

	  
	  \pstart \leavevmode% starting standard par
	य‚दि नाम पुरुषेच्छावृत्त‚यो व्य‚व‚हारास्त‚थापि किमित्य‚न्य‚था क्रिय‚न्त इत्याह ।‚{\tiny $_{lb}$}‚ \textbf{तेषां चे}ति पुंसां \textbf{चित्राभि‚{\tiny $_{४}$}‚स‚न्धित्वा}च्चित्राभिप्राय‚त्वात् । त‚तो य‚थेष्टं व्य‚व‚हाराः‚{\tiny $_{lb}$}‚ प्र‚व‚र्त्त‚न्त इति नास्ति गुण‚दोष‚प्र‚भ‚वाणां व्य‚व‚हाराणाम्विवेक‚निश्च‚यः [।] त‚दिति‚{\tiny $_{lb}$}‚ त‚स्माद‚य‚म‚नुमाता पुमान् \textbf{लिङ्ग‚संक‚रा}ल्लिङ्ग‚व्य‚भिचारा\textbf{द‚निश्चिन्व‚न्} क्षीण‚दोषः‚{\tiny $_{lb}$}‚ \textbf{क‚थ}माग‚म‚स्य क‚र्त्तारं \textbf{प्र‚तिप‚द्येत} नैवेति निग‚म‚नीयं ।
	{\color{gray}{\rmlatinfont\textsuperscript{§~\theparCount}}}
	\pend% ending standard par
      ‚{\tiny $_{lb}$}‚

	  
	  \pstart \leavevmode% starting standard par
	\textbf{अथ किमि}त्यादि । \textbf{यो निर्दोषो} रागादिदोष‚र‚हित‚स्ता\textbf{दृशः पुरु}षः \textbf{किन्नैवा‚{\tiny $_{५}$}‚स्ति} ।‚{\tiny $_{lb}$}‚ अस्तीति प्र‚तिपाद‚य‚न्नाह । \textbf{स‚र्वेषामि}त्यादि । प्र‚तिप‚क्ष‚स‚म्मुखीभावे निर्ह्रास‚म‚प‚च‚यं‚{\tiny $_{lb}$}‚  ‚{\tiny $_{lb}$}‚ ‚{\tiny $_{lb}$}‚ \leavevmode\ledsidenote{\textenglish{398/s}}प्र‚तिप‚क्षास‚म्मुखीभावे चातिश‚य‚मुप‚च‚यं श्र‚य‚न्ते ये रागाद‚य‚स्ते \textbf{निर्ह्रासातिश‚याश्रिताः} ।‚{\tiny $_{lb}$}‚ तेषां \textbf{स‚र्वेषां विप‚क्ष‚त्वा}त् । य‚स्य च स‚म्मुखीभावास‚म्मुखीभावाभ्यां निर्ह्रासातिश‚{\tiny $_{lb}$}‚य‚म्भ‚ज‚न्ते । स एव तेषां विप‚क्षो बाध‚क‚स्तेन स‚ह व‚र्त्त‚त इति स‚वि‚{\tiny $_{६}$}‚प‚क्षः ।‚{\tiny $_{lb}$}‚ त‚द्भाव‚स्त‚त्त्व‚न्त‚स्मात् । स बाध‚क‚त्वादिति याव‚त् । येन च बाध‚केन दोषाणां‚{\tiny $_{lb}$}‚ स‚विप‚क्ष‚त्व‚न्त‚स्य \textbf{बाध‚क‚स्याभ्यासात्} पुनः पुन‚र्नैर‚न्त‚र्येणोत्पाद‚नाद् य \textbf{सात्मीभाव}स्त‚{\tiny $_{lb}$}‚दौर्ज्जित्य‚न्त‚न्म‚य‚ता । त\textbf{स्माद्} बाध‚क‚सात्मीभावा\textbf{द्धीयेर‚न्} । क्षीय‚रेन् । \textbf{आस्र‚वा}‚{\tiny $_{lb}$}‚ \leavevmode\ledsidenote{\textenglish{143b/PSVTa}} रागाग‚द‚यः \textbf{क्व‚चित्} स‚न्ताने सात्मीभूत‚दोष‚प्र‚तिप‚क्षे दोषाणां च बाध‚कं नैरात्म्य‚{\tiny $_{lb}$}‚ज्ञान‚मिति प्र‚तिपाद‚यिष्य‚ति । त‚स्मान्न त‚थाभूतः पुरुषो नेष्य‚ते एताव‚त्तु ब्रूमः \textbf{स तु‚{\tiny $_{lb}$}‚ क्षीणास्र‚वो दुर्ज्ञान} इति ।
	{\color{gray}{\rmlatinfont\textsuperscript{§~\theparCount}}}
	\pend% ending standard par
      ‚{\tiny $_{lb}$}‚

	  
	  \pstart \leavevmode% starting standard par
	\textbf{दोषो ही}त्यादि विव‚र‚णं । \textbf{दोषा हि} रागाद‚यः । किंभूता [।] \textbf{निर्ह्रासातिश‚य‚{\tiny $_{lb}$}‚ध‚र्माणः} । अप‚क‚र्षोत्क‚र्ष‚स्व‚भावाः स‚न्तो \textbf{विप‚क्षाभिभ‚वोत्क‚र्ष} विप‚क्ष‚कृतो‚{\tiny $_{lb}$}‚ योभिभ‚व‚{\tiny $_{१}$}‚स्तिर‚स्कार‚स्त‚स्योत्क‚र्षाप‚क‚र्षं \textbf{साध‚य‚न्ति} ग‚म‚य‚न्ति । तेनाय‚म‚र्थः [।]‚{\tiny $_{lb}$}‚ निर्ह्रास‚ध‚र्माणः विप‚क्षाभिभ‚वोत्क‚र्षं साध‚य‚न्ति \textbf{बाध‚काभिभ‚वो}त्क‚र्षेण दोषाणां‚{\tiny $_{lb}$}‚ निर्ह्रासात् अतिश‚य‚ध‚र्माणो बाध‚काभिभ‚वाप‚क‚र्षं साध‚य‚न्ति । बाध‚काभिभ‚व‚मान्द्येन‚{\tiny $_{lb}$}‚ तेषाम‚तिश‚य‚ध‚र्म‚त्वात् । \textbf{ज्वालादिव‚त्} । आदिश‚ब्दाच्छीतोष्ण‚स्प‚र्शादिप‚रिग्र‚हः ।‚{\tiny $_{lb}$}‚ य‚था ज्वालाद‚यो बाध‚क‚स्योद‚कादेरुत्क‚र्षाप‚क‚र्षे स‚ति निर्ह्रासातिश‚य‚ध‚र्माणो‚{\tiny $_{lb}$}‚ य‚थाक्र‚म‚मुद‚काद्य‚भिभ‚वोत्क‚र्षं साध‚य‚न्ति त‚द्व‚त् ।
	{\color{gray}{\rmlatinfont\textsuperscript{§~\theparCount}}}
	\pend% ending standard par
      ‚{\tiny $_{lb}$}‚

	  
	  \pstart \leavevmode% starting standard par
	न‚नु च बाह्यार्थ‚प्र‚तिब‚द्धा रागाद‚यः बाह्यं च व‚स्तु नित्यं स‚न्निहित‚मेव [।]‚{\tiny $_{lb}$}‚ त‚त्क‚थं रागादीनामुच्छेद इत्य‚त आह । ते हीत्यादि । हिश‚ब्दो य‚स्माद‚र्थः । ते रागा‚{\tiny $_{lb}$}‚द‚यो \textbf{विक‚ल्प‚प्र‚भ‚वाः} । विक‚ल्पाद‚योनिशोम‚न‚स्कार‚ल‚क्ष‚णात् प्र‚भ‚व उत्पाद एषामि‚{\tiny $_{lb}$}‚ति विग्र‚हः । त‚था ह्य‚योनिशोम‚न‚स्कार‚म‚न्त‚रेण \textbf{स‚त्य‚पि} बाह्येर्थे नोत्प‚द्य‚न्ते रागाद‚यः‚{\tiny $_{lb}$}‚ त‚त्स‚म्मुखीभावे च विनाप्य‚र्थेनोत्प‚द्य‚न्त इति विक‚ल्प‚प्र‚भ‚वा रागाद‚यः । त‚तः स‚त्य‚{\tiny $_{lb}$}‚\textbf{प्युपादाने} य‚थोक्त‚ल‚क्ष‚णे । अनाद‚र‚विव‚क्षायां चेयं स‚प्त‚मी । \textbf{क‚स्य‚चित् म‚नोगुण‚स्य}‚{\tiny $_{lb}$}‚ नैरात्म्य‚द‚र्श‚न‚ल‚क्ष‚ण‚स्या\textbf{भ्यासात् । अ‚{\tiny $_{४}$}‚प‚क‚र्षिणः} अप‚च‚य‚व‚न्तो भ‚व‚न्तीत्य‚र्थः ।
	{\color{gray}{\rmlatinfont\textsuperscript{§~\theparCount}}}
	\pend% ending standard par
      ‚{\tiny $_{lb}$}‚

	  
	  \pstart \leavevmode% starting standard par
	एत‚दुक्त‚म्भ‚व‚ति । य‚द्य‚पि ताव‚द् दोष‚निदान‚स्य स‚र्व‚दा नोच्छेदः प्र‚तिप‚क्ष‚स्या‚{\tiny $_{lb}$}‚त्य‚न्त‚पाट‚वाभावात् [।] त‚थापि प्र‚तिप‚क्षाभ्यासात् म‚न्दीकृत‚साम‚र्थ्यादुपादानाद्‚{\tiny $_{lb}$}‚ अप‚क‚र्षिणः क्षाम‚क्षाम‚त‚रा दोषा भ‚व‚न्तीत्य‚नेन च हेतुरुक्तः । य‚दा तु य‚थोक्त‚स्य‚{\tiny $_{lb}$}‚ ‚{\tiny $_{lb}$}‚ \leavevmode\ledsidenote{\textenglish{399/s}}म‚नोगुण‚स्य भाव‚नाप्र‚क‚र्ष‚प‚र्य‚न्त‚व‚र्त्तित‚या पाट‚{\tiny $_{५}$}‚वं जात‚न्त‚दा \textbf{त‚त्पाट‚वे} । त‚स्य‚{\tiny $_{lb}$}‚ म‚नोगुण‚स्य पाट‚वे स‚ति । अन्व‚यः क्लेश‚बीज‚म‚न्वेत्युत्प‚द्य‚तेऽस्माद्दोष इति कृत्वा [।]‚{\tiny $_{lb}$}‚ निर्ग‚तोन्व‚यो य‚स्मिन् विनाशे स निर‚न्व‚य‚विनाशः [।] स ध‚र्मो येषान्दोषाणान्ते‚{\tiny $_{lb}$}‚ \textbf{निर‚न्व‚य‚विनाश‚ध‚र्माणः} । वास‚न‚या स‚ह विनाश‚ध‚र्माण इत्य‚र्थः । \textbf{ज्वालादिव‚त्} । य‚था‚{\tiny $_{lb}$}‚ ज्वालाद‚यः प्र‚तिप‚क्ष‚स्योद‚कादेरुत्क‚र्षे स‚त्य‚त्य‚न्त‚विनाश‚ध‚र्माण‚स्त‚द्व‚त् । प्र‚योगः पुनः [।]‚{\tiny $_{lb}$}‚ ये य‚दुप‚धानाद‚प‚क‚र्षिण‚स्ते त‚द‚त्य‚न्त‚वृद्धौ त‚द‚भिभ‚वान्निर‚न्व‚य‚विनाश‚ध‚र्माणः ।‚{\tiny $_{lb}$}‚ त‚द्य‚था ज्वालाद‚यः स‚लिलाभिभ‚व‚वृद्धौ । नैरात्म्य‚द‚र्श‚नोप‚धानाच्चाप‚क‚र्ष‚ध‚र्माणो‚{\tiny $_{lb}$}‚ दोषा इति स्व‚भाव‚हेतुः ।
	{\color{gray}{\rmlatinfont\textsuperscript{§~\theparCount}}}
	\pend% ending standard par
      ‚{\tiny $_{lb}$}‚

	  
	  \pstart \leavevmode% starting standard par
	य‚त एव\textbf{न्तेन} कार‚णेन \textbf{स्याद‚पि} क‚श्चि\textbf{न्निर्दोषः । क‚थ‚मित्या}दि प‚रः ।‚{\tiny $_{lb}$}‚ \textbf{याव}तेत्य‚य‚न्निपातो य‚{\tiny $_{७}$}‚देत्य‚स्मिन्न‚र्थे व‚र्त्त‚ते । \textbf{दोष‚सात्म}नो \textbf{विप‚क्षोत्प‚त्तिव}दिति \leavevmode\ledsidenote{\textenglish{144a/PSVTa}}‚{\tiny $_{lb}$}‚ चानादिकालाभ्यासात् । दोष‚सात्म‚नः पुंसः विप‚क्षोत्प‚त्तिर्नैरात्म्य‚द‚र्श‚नोत्प‚तिः ।‚{\tiny $_{lb}$}‚ एव\textbf{न्दोष‚विप‚क्ष}स्य नैरात्म्य‚द‚र्श‚न‚स्य \textbf{सात्म‚त्वेपि य‚थाप्र‚त्य‚यं । य‚था}कार‚ण‚स‚न्निधा‚{\tiny $_{lb}$}‚न\textbf{न्दोषोत्प‚त्तिर‚पि} स्यात् ।
	{\color{gray}{\rmlatinfont\textsuperscript{§~\theparCount}}}
	\pend% ending standard par
      ‚{\tiny $_{lb}$}‚

	  
	  \pstart \leavevmode% starting standard par
	\textbf{नाय‚मि}त्या चा र्यः । स‚र्व‚सांसारिकोप‚द्र‚व‚{\tiny $_{१}$}‚र‚हित‚त्वान्निरुप‚द्र‚वः । भूत‚विप‚रीत‚{\tiny $_{lb}$}‚म‚नित्यादिस्व‚ल‚क्ष‚ण‚म‚र्थो विष‚योस्येति भूतार्थः । भूतार्थ‚ग्र‚ह‚णादेव च मार्ग‚श्चित्त‚{\tiny $_{lb}$}‚स्व‚भावः । निरुप‚द्र‚व‚श्चासौ भूतार्थ‚श्चेति \textbf{निरुप‚द्र}व\textbf{भूता}र्थ‚स्त‚भूत‚श्चासौ‚{\tiny $_{lb}$}‚ \textbf{स्व‚भा}व‚श्चेति क‚र्म‚धार‚य‚ग‚र्भ एव क‚र्म‚धार‚य‚स‚मासः निरुप‚द्र‚व‚स्य भूतार्थ‚स्य‚{\tiny $_{lb}$}‚ भूतार्थ‚त्वेनाभ्यासात् सात्मीभाव‚ग‚ते [।] अनेन च चित्त\textbf{स्व‚भाव‚{\tiny $_{२}$}‚स्य} दोष‚प्र‚तिप‚क्ष‚स्य‚{\tiny $_{lb}$}‚ \textbf{विप‚र्य‚यै}र्य‚थोक्तात् त्र‚याद् विल‚क्ष‚णैः सोप‚द्र‚वैर‚भूतार्थैर‚स्व‚भावैश्च दोषै\textbf{र्न बाध‚नं} ।‚{\tiny $_{lb}$}‚ सात्मीभूतं मार्ग‚म‚भिभूय न दोषाणामुत्प‚त्तिरित्य‚र्थः । किङ्कार‚णं [।] \textbf{य‚त्न‚व‚त्त्वे‚{\tiny $_{lb}$}‚ पीत्यादि} । एत‚दाह । सात्मीभूत‚स्य मार्ग‚स्य दोषोत्पाद‚नाय य‚त्न एव न स‚म्भ‚व‚ति ।
	{\color{gray}{\rmlatinfont\textsuperscript{§~\theparCount}}}
	\pend% ending standard par
      ‚{\tiny $_{lb}$}‚

	  
	  \pstart \leavevmode% starting standard par
	त‚थाप्य‚भ्युप‚ग‚म्योच्य‚ते [।] दोषोत्पाद‚ने य‚त्न‚व‚त्त्वे \textbf{बुद्धेस्त‚त्प‚क्ष‚पात‚तः} ।‚{\tiny $_{lb}$}‚ त‚स्मि‚{\tiny $_{३}$}‚न् दोष‚प्र‚तिप‚क्ष‚भूते गुण‚व‚ति नैरात्म्य‚मार्गे । प‚क्ष‚पातेन ब‚हुमान‚तः । दोषो‚{\tiny $_{lb}$}‚  ‚{\tiny $_{lb}$}‚ ‚{\tiny $_{lb}$}‚ \leavevmode\ledsidenote{\textenglish{400/s}}त्पाद‚ने य‚त्न‚न्निव‚र्त्त्य । गुण‚प‚क्ष‚पातेन दोष‚प्र‚तिप‚क्ष एव य‚त्नाधानादिति याव‚त् ।
	{\color{gray}{\rmlatinfont\textsuperscript{§~\theparCount}}}
	\pend% ending standard par
      ‚{\tiny $_{lb}$}‚

	  
	  \pstart \leavevmode% starting standard par
	स्व‚भाव‚प‚द‚मेव ताव‚दादौ व्याच‚ष्टे । \textbf{न हि स्व‚भावो} नैराप्म्य‚द‚र्श‚न‚ल‚क्ष‚णः‚{\tiny $_{lb}$}‚ प्र‚तिप‚क्ष‚सात्म‚नि व्य‚व‚स्थितेन पुरुषेण । \textbf{अय‚न्तेन} प्र‚य‚त्नेन \textbf{विना निव‚र्त्त‚यितुं श‚क्यः‚{\tiny $_{lb}$}‚ श्रोत्रि‚{\tiny $_{४}$}‚य‚कापालिक‚घृणाव‚दि}ति । यः श्रोत्रियः स‚न् कापालिको भ‚व‚ति त‚स्य‚{\tiny $_{lb}$}‚ श्रोत्रियाव‚स्थायां या घृणा सा य‚था य‚त्न‚म‚न्त‚रेण न श‚क्य‚ते निव‚र्त्त‚यितुन्त‚द्व‚त् ।‚{\tiny $_{lb}$}‚ मार्ग‚स्व‚भाव‚निव‚र्त्त‚नाय \textbf{य‚त्न‚श्च} क्रिय‚माणः । \textbf{प्रा}प्य‚स्य रागादिस्व‚भाव‚स्य \textbf{गुण}द‚र्श‚नेन‚{\tiny $_{lb}$}‚ \textbf{निव‚र्त्त्य}स्य विप‚श्य‚नास्व‚भाव‚स्य \textbf{दोष‚द‚र्श‚नेन क्रियेत । त‚च्च} दोष‚द‚र्श‚नं \textbf{विप‚क्ष‚सात्म‚नः}‚{\tiny $_{lb}$}‚ दोष‚प्र‚ति‚{\tiny $_{५}$}‚प‚क्ष‚सात्म‚नः \textbf{पुंसो} दोष‚प्र‚तिप‚क्षे\textbf{न स‚म्भ‚व}ति । त‚था त‚च्च गुण‚द‚र्श‚न‚म्वि‚{\tiny $_{lb}$}‚प‚क्ष‚सात्म‚नो \textbf{दोषेषु} न स‚म्भ‚व‚ति ।
	{\color{gray}{\rmlatinfont\textsuperscript{§~\theparCount}}}
	\pend% ending standard par
      ‚{\tiny $_{lb}$}‚

	  
	  \pstart \leavevmode% starting standard par
	क‚स्मात् पुनः प्र‚तिप‚क्षे दोष‚द‚र्श‚न‚न्न स‚म्भ‚व‚तीत्याह । त‚स्येत्यादि । \textbf{त‚स्य}‚{\tiny $_{lb}$}‚ प्र‚तिप‚क्ष‚स्य \textbf{निरुप‚द्र‚व‚त्वात्} । त्रिविधो ह्युप‚द्र‚वो य‚स्याभावान्निरुप्र‚द्र‚वो मार्गः ।‚{\tiny $_{lb}$}‚ त‚था हि चित्त‚म्विब‚द्धुं हेतुर्दोषोप‚द्र‚वो यैश्चित्त‚म्विव‚द्ध‚म्भूतार्थ‚द‚र्श‚ने न प्र‚व‚र्त्त‚ते ।‚{\tiny $_{६}$}‚‚{\tiny $_{lb}$}‚ काय‚चित्त‚व्य‚थाहेतुर्दुःख‚दौर्म‚न‚स्योप‚द्र‚वः । सास्र‚व‚सुख‚स्याप्र‚शान्त‚त‚या त‚दुप‚भोगे‚{\tiny $_{lb}$}‚ वैर‚स्योद्वेग‚श्च ।
	{\color{gray}{\rmlatinfont\textsuperscript{§~\theparCount}}}
	\pend% ending standard par
      ‚{\tiny $_{lb}$}‚

	  
	  \pstart \leavevmode% starting standard par
	त‚त्र प्र‚थ‚म‚स्योप‚द्र‚व‚स्याभाव‚माह । \textbf{स‚र्व‚दोष‚हाने}रिति । स‚र्व‚स्य रागादिदोष‚स्य‚{\tiny $_{lb}$}‚ हानेर्विग‚मात् ।
	{\color{gray}{\rmlatinfont\textsuperscript{§~\theparCount}}}
	\pend% ending standard par
      ‚{\tiny $_{lb}$}‚

	  
	  \pstart \leavevmode% starting standard par
	\textbf{प‚र्य‚व‚स्था}नेत्यादिना द्वितीय‚स्याभाव‚माह । रागादिस‚म्मुखीभावः प‚र्य‚व‚स्थानं‚{\tiny $_{lb}$}‚ \leavevmode\ledsidenote{\textenglish{144b/PSVTa}} च ज‚न्म च \textbf{प‚र्य‚व‚स्थान‚ज‚न्म}नी ।‚{\tiny $_{७}$}‚ त‚योर्य‚त्\textbf{प्र‚तिव‚द्ध‚न्दुःख}न्त‚स्य \textbf{विवेकात्} । रागा‚{\tiny $_{lb}$}‚द्युत्प‚त्तिकाले य‚द् दुःखं काय‚चित्त‚प‚रिदाह‚ल‚क्ष‚ण‚न्त‚त्प‚र्य‚व‚स्थान‚प्र‚तिब‚द्ध‚जातिज‚राव्या‚{\tiny $_{lb}$}‚ध्यादिदुःख‚न्तु ज‚न्म‚प्र‚तिब‚द्धं ।
	{\color{gray}{\rmlatinfont\textsuperscript{§~\theparCount}}}
	\pend% ending standard par
      ‚{\tiny $_{lb}$}‚

	  
	  \pstart \leavevmode% starting standard par
	त‚तीय‚स्योप‚द्र‚व‚स्याभाव‚माह । प्र‚श‚मेत्यादि \textbf{प्र‚श‚मो} रागादिविर‚ह‚ल‚क्ष‚णं निर्वाणं ।‚{\tiny $_{lb}$}‚ त‚स्मिन् य\textbf{त्सुख}म‚नास्र‚व‚न्त‚स्य \textbf{र‚स} अस्वाद‚स्त\textbf{स्यानुद्वेज‚नात्} । अवैमुख्य‚क‚र‚णात् ।‚{\tiny $_{lb}$}‚ \textbf{अभूतार्थ}म‚भ‚{\tiny $_{१}$}‚त‚विष‚यं \textbf{ख‚ल्व‚पि} रागादि । न स्यादिति स‚म्ब‚न्धः ।
	{\color{gray}{\rmlatinfont\textsuperscript{§~\theparCount}}}
	\pend% ending standard par
      ‚{\tiny $_{lb}$}‚

	  
	  \pstart \leavevmode% starting standard par
	य‚द्य‚भूतार्थं क‚थ‚न्त‚र्हि त‚स्योत्प‚त्तिरित्याह । \textbf{उपादान‚ब‚ल‚भावी}ति वित‚थ‚विक‚ल्प‚{\tiny $_{lb}$}‚‚{\tiny $_{lb}$}‚ \leavevmode\ledsidenote{\textenglish{401/s}}वास‚नाब‚ल‚भावि । त‚देवंभूत‚रागादिबीजाश्र‚य‚स्य विज्ञान\textbf{स‚न्तान‚स्य विप‚र्य‚यो‚{\tiny $_{lb}$}‚पादानात्} रागादिविप‚र्य‚य‚ल‚क्ष‚ण‚स्य प्र‚तिप‚क्ष‚स्य प‚रिग्र‚हान्न स्यान्नोत्प‚द्येत । न तु‚{\tiny $_{lb}$}‚ \textbf{भूतार्थ‚म}विप‚रीत‚विष‚यं \textbf{न} भ‚वेत् । किन्तु भ‚वे‚{\tiny $_{२}$}‚देव । किं कार‚ण‚म् [।] \textbf{व‚स्तुब‚ल‚{\tiny $_{lb}$}‚प्र‚वृत्तेः} । य‚थाव‚स्थित‚व‚स्तुसाम‚र्थ्येनोत्प‚त्तेः । दोष‚सात्म्येपि ताव‚त् स्थित‚स्य । प्र‚मा‚{\tiny $_{lb}$}‚णान्य‚नित्यादिभ‚ताकार‚ग्राहीणि । प्र‚तिप‚क्ष‚मार्ग‚माव‚ह‚न्ति । \href{http://sarit.indology.info/?cref=pv.3.219-3.220}{२२२-२३}
	{\color{gray}{\rmlatinfont\textsuperscript{§~\theparCount}}}
	\pend% ending standard par
      ‚{\tiny $_{lb}$}‚

	  
	  \pstart \leavevmode% starting standard par
	किं पुन‚र्विप‚श्य‚ना सात्म‚नि स्थित‚स्य । \textbf{अभूतार्थ‚श्च} दोषा रागाद‚यः । आत्मा‚{\tiny $_{lb}$}‚त्मीयाध्यारोपितेर्थे प्र‚वृत्तेः । ते \textbf{प्र‚तिप‚क्ष}स्य भूतार्थ‚स्य य‚त् \textbf{सात्म्}यं स्व‚भाव‚त्व‚न्त‚स्य‚{\tiny $_{lb}$}‚ \textbf{वा‚{\tiny $_{३}$}‚चि}नो न भ‚व‚न्ति । य‚त एव\textbf{न्त‚स्मान्न पुनः} प्र‚हीण‚दोषाणां \textbf{दोषोत्प‚त्तिः} । दोषो‚{\tiny $_{lb}$}‚त्पाद‚न\textbf{य‚त्नेपि बुद्धेर्गुण‚प‚क्ष‚पातेन} कार‚णेन रागादि\textbf{प्र‚तिप‚क्ष एव य‚त्नाधानात्} प्र‚य‚त्न‚स्य‚{\tiny $_{lb}$}‚ क‚र‚णात् । क‚स्य \textbf{प‚रीक्षाव‚तो} युक्त्या विचार‚क‚स्य दोष‚सात्म्येपि ताव‚त् स्थित‚स्य ।‚{\tiny $_{lb}$}‚ \textbf{विशेषेणा}तिश‚येन गुणेष्वेव य‚त्नाधान\textbf{म‚दुष्टात्म‚नः} प्र‚तिप‚क्ष‚सात्म‚नि स्थित‚स्य‚{\tiny $_{४}$}‚ ।
	{\color{gray}{\rmlatinfont\textsuperscript{§~\theparCount}}}
	\pend% ending standard par
      ‚{\tiny $_{lb}$}‚

	  
	  \pstart \leavevmode% starting standard par
	\textbf{कः पुन‚रेषान्दोषाणां प्र‚भ‚व} उत्प‚त्तिकार‚णं । प्र‚भ‚व‚त्युत्प‚द्य‚तेस्मादिति कृत्वा ।‚{\tiny $_{lb}$}‚ य‚स्य दोष‚हेतोः \textbf{प्र‚तिप‚क्षाभ्यासेन प्र‚हीयंते} ।
	{\color{gray}{\rmlatinfont\textsuperscript{§~\theparCount}}}
	\pend% ending standard par
      ‚{\tiny $_{lb}$}‚

	  
	  \pstart \leavevmode% starting standard par
	उत्त‚र‚माह । \textbf{स‚र्वासामि}त्यादि । \textbf{दोष‚जातीना}न्दोष‚प्र‚काराणां \textbf{जाति}रुत्प‚त्तिः‚{\tiny $_{lb}$}‚ \textbf{स‚त्काय‚द‚र्श‚नात्} । आत्मात्मीयाभिनिवेशात् ।
	{\color{gray}{\rmlatinfont\textsuperscript{§~\theparCount}}}
	\pend% ending standard par
      ‚{\tiny $_{lb}$}‚

	  
	  \pstart \leavevmode% starting standard par
	न‚नु चाविद्याहेतुकाः क्लेशा आग‚मे उक्तास्त‚त्क‚थं न व्याघात इत्याह ।‚{\tiny $_{lb}$}‚ \textbf{साऽविद्ये}ति सैव‚{\tiny $_{५}$}‚ \textbf{स‚त्काय‚दृष्टि}र‚विद्या । त‚तो नास्ति विरोधः । केन पुनः क्र‚मेण‚{\tiny $_{lb}$}‚ दोषाणां स‚त्काय‚द‚र्श‚नादुत्प‚त्तिरित्याह । \textbf{त}त्रेत्यादि । \textbf{त‚त्रा}त्मात्मीय‚त्वेनाभिनिवि‚{\tiny $_{lb}$}‚ ष्टे विष‚ये । \textbf{त‚त्स्नेहः} । आत्मात्मीय‚स्नेहः । \textbf{त‚स्मादा}त्मात्मीय‚स्नेहात् \textbf{द्वेषादिस‚म्भ‚व}‚{\tiny $_{lb}$}‚ इति क्र‚मः ।
	{\color{gray}{\rmlatinfont\textsuperscript{§~\theparCount}}}
	\pend% ending standard par
      ‚{\tiny $_{lb}$}‚

	  
	  \pstart \leavevmode% starting standard par
	\textbf{न ही}त्यादिना व्याच‚ष्टे । \textbf{नाह}मित्यात्माकार‚प्र‚तिषेधः । \textbf{न म‚मे}त्यात्मीयाका‚{\tiny $_{lb}$}‚र‚स्य । अनात्माकारेण नात्मीयाकारेण च \textbf{प‚श्य‚तः} पुरुष‚स्य । \textbf{प‚रिग्र‚ह‚म‚न्त‚रेणे}ति ।‚{\tiny $_{lb}$}‚ ‚{\tiny $_{lb}$}‚ \leavevmode\ledsidenote{\textenglish{402/s}}आत्मात्मीय‚त्वेन त‚द‚नुग्राह‚क‚त्वेन प‚रिक‚ल्प्य ग्र‚हः तेन विना \textbf{न क्व‚चित्} विष‚ये‚{\tiny $_{lb}$}‚ \textbf{स्नेहः । न चान‚नुरागिणः} आत्मात्मीयादिस्नेह‚र‚हित‚स्य \textbf{क्व‚चिद् द्वेषः} । अनुन‚य‚{\tiny $_{lb}$}‚\leavevmode\ledsidenote{\textenglish{145a/PSVTa}} म‚न्त‚रेण त‚स्याभावात् । किङ्कार‚ण‚म् [।] \textbf{आ‚{\tiny $_{७}$}‚त्मात्मीये}त्यादि । \textbf{आत्मात्मीय‚यो‚{\tiny $_{lb}$}‚र‚नुप‚रोधिन्य}प्र‚तिकूल‚व‚र्त्तिनि । उदासीन‚प‚क्षे । \textbf{त‚द‚भावात्} त‚स्य द्वेष‚स्याभावात् ।‚{\tiny $_{lb}$}‚ उप‚रोध\textbf{प्र‚तिघातिनि} चेति । आत्मात्मीय‚त्वेन गृहीत‚स्य य उप‚रोधः पीडा ।‚{\tiny $_{lb}$}‚ त‚त्\textbf{प्र‚तिघातिनि} त‚त्प्र‚तिषेधं कुर्व‚ति । मित्र‚प‚क्षे \textbf{त‚द‚भावात्} । द्वेषाभावात् । किंत्वा‚{\tiny $_{lb}$}‚त्मात्मीय‚स्नेह‚विष‚य‚भूत‚विरोधेन । यः स्थितः प्र‚{\tiny $_{१}$}‚तिकूल‚व‚र्त्ती । त‚त्रैव द्वेषः । त‚स्मा‚{\tiny $_{lb}$}‚न्नात्मात्मीय‚स्नेह‚म‚न्त‚रेण द्वेष इति । \textbf{त‚स्मा}दित्यादिना निग‚म‚नं । य‚त एव‚न्त\textbf{स्मात्तौ‚{\tiny $_{lb}$}‚ चे}ति । \textbf{आत्म‚द}र्श\textbf{ना}त्मीय‚ग्र‚हौ । \textbf{स्ने}हं \textbf{प्र‚सुवाते} इति व‚च‚न‚विप‚रिणामेन स‚म्ब‚न्धः ।‚{\tiny $_{lb}$}‚ \textbf{स} च स्नेहो \textbf{द्वेषादीन् प्र‚सूते} ज‚न‚य‚ति ।
	{\color{gray}{\rmlatinfont\textsuperscript{§~\theparCount}}}
	\pend% ending standard par
      ‚{\tiny $_{lb}$}‚

	  
	  \pstart \leavevmode% starting standard par
	त‚योस्त‚र्ह्यात्म‚द‚र्श‚नात्मीय‚ग्र‚ह‚योः को हेतुरित्याह । \textbf{स‚माने}त्यादि । \textbf{स‚मा}न‚{\tiny $_{lb}$}‚\textbf{जाती}य‚न्त‚{\tiny $_{२}$}‚देवात्म‚द‚र्श‚न‚न्त‚स्या\textbf{भ्या}सः पौनःपुन्येनादिकाल‚मुत्प‚त्तिः । त‚द्वास‚ना च ।‚{\tiny $_{lb}$}‚ त‚स्माज्जा\textbf{त‚मात्म‚द‚र्श‚न}मात्मीय‚ग्र‚हं प्र‚सूते । त‚स्मात् \textbf{स‚त्काय‚द‚र्श‚न‚जाः स‚र्वे क्लेशाः} ।‚{\tiny $_{lb}$}‚ \href{http://sarit.indology.info/?cref=pv.3.220-3.221}{२२३-२४}
	{\color{gray}{\rmlatinfont\textsuperscript{§~\theparCount}}}
	\pend% ending standard par
      ‚{\tiny $_{lb}$}‚

	  
	  \pstart \leavevmode% starting standard par
	\textbf{त‚देव} च स‚त्काय‚द‚र्श‚न‚म\textbf{ज्ञान}म‚विद्येत्युच्य‚ते सिद्धान्ते ।
	{\color{gray}{\rmlatinfont\textsuperscript{§~\theparCount}}}
	\pend% ending standard par
      ‚{\tiny $_{lb}$}‚

	  
	  \pstart \leavevmode% starting standard par
	येनैव स‚त्काय‚द‚र्श‚न‚मेवाऽविद्याऽ\textbf{त एव} कार‚णात् \textbf{मोहोनिदानं} प्र‚धान‚कार‚ण‚{\tiny $_{lb}$}‚\textbf{न्दोषाणां} रागादीना\textbf{म‚भिधीय‚ते}‚{\tiny $_{३}$}‚ सू त्रा न्त रे\edtext{}{\edlabel{pvsvt_402-1}\label{pvsvt_402-1}\lemma{रे}\Bfootnote{बुद्ध‚व‚च‚ने}}-- अविद्याहेतुकाः स‚र्व‚क्लेशा‚{\tiny $_{lb}$}‚ इति \textbf{स‚त्काय‚दृष्टि}र्दोषाणां निदान‚म\textbf{न्य‚त्र} सूत्रान्त‚रेभि धीय‚ते । क‚स्मात् । \textbf{त‚त्प्र‚हाणे}‚{\tiny $_{lb}$}‚ स‚त्काय‚दृष्टेः प्र‚हाणे दोषाणां \textbf{प्र‚हाण‚तः । मोह‚न्दोष‚निदानं} दोष कार‚ण‚माहुर्बुद्धा‚{\tiny $_{lb}$}‚ भ‚ग‚व‚न्तः । क‚स्माद् [।] \textbf{अमूढ‚स्या}ज्ञान‚र‚हित‚त्य \textbf{दोषानुत्प‚त्तेः पुन‚र‚न्य‚त्र} प्र‚देशे‚{\tiny $_{lb}$}‚ \textbf{स‚त्काय‚दृष्टि}न्दोष‚निदान‚माहुः । \textbf{त‚च्चैत‚{\tiny $_{४}$}‚त्का}र‚ण‚त्वं मोह‚स्य स‚त्काय‚दृष्टेश्च \textbf{प्र‚धान}‚{\tiny $_{lb}$}‚‚{\tiny $_{lb}$}‚ ‚{\tiny $_{lb}$}‚ \leavevmode\ledsidenote{\textenglish{403/s}}हेतु\textbf{निर्देशे स‚ति स्या}न्न‚हेतुमात्र‚निर्देशे । किं कार‚ण‚म् [।] \textbf{अनेक}स्मादिन्द्रिय‚विष‚यो‚{\tiny $_{lb}$}‚ऽयोनिशोम‚न‚स्कार‚क‚लापाज्ज‚न्म येषान्दोषाणान्तेषा\textbf{मेक}स्मात् मोहात् स‚त्काय‚दृष्टे‚{\tiny $_{lb}$}‚श्चो\textbf{त्प‚तिविरोधात्} । त‚स्माद‚न्य‚कार‚ण‚स‚म्भ‚वेपि प्राधान्यं गृहीत्वा मोह‚स‚त्काय‚{\tiny $_{lb}$}‚दृष्ट्योः कार‚ण‚त्व‚मुक्त‚मिति ग‚म्य‚ते ।
	{\color{gray}{\rmlatinfont\textsuperscript{§~\theparCount}}}
	\pend% ending standard par
      ‚{\tiny $_{lb}$}‚

	  
	  \pstart \leavevmode% starting standard par
	य‚दि चान्यो मो‚{\tiny $_{५}$}‚होन्या च स‚त्काय‚दृष्टिस्त‚योश्च प्राधान्य‚न्त‚दा \textbf{न च द्व‚यो}‚{\tiny $_{lb}$}‚र्मोह‚स‚त्काय‚दृष्ट्योः \textbf{प्राधान्ये} स‚त्ये\textbf{कैक‚निर्देशः} । क्व‚चित् मोह‚स्यैव निर्देशः क्व‚चित्‚{\tiny $_{lb}$}‚ स‚त्काय‚दृष्टेरेवेत्य‚र्थः । न \textbf{प‚र‚भाग‚भाक्} शोभाभाक् व‚क्तुर‚कौश‚ल‚मेवाव‚ह‚तीति‚{\tiny $_{lb}$}‚ याव‚त् । य‚दा पुन‚र‚न‚योर्न स्व‚भाव‚भेदः । त‚दा मोह‚श‚ब्देन स‚त्काय‚दृष्टिश‚ब्देन‚{\tiny $_{lb}$}‚ \textbf{चोभ‚य‚थाप्येक‚स्या}र्थ‚स्य प‚र्यायेण \textbf{निर्दे‚{\tiny $_{६}$}‚शेन न विरोधः} प्राधान्य‚स्य । \textbf{प्राधान्यं पुन}‚{\tiny $_{lb}$}‚स्स‚त्कार्य‚द‚र्श‚न‚स्य \textbf{त‚दुपादान‚त्वेन} दोषाणामुपादान‚त्वेन [।] त‚च्चान‚न्त‚र‚मेव प्र‚ति‚{\tiny $_{lb}$}‚पादितं । प्र‚तिपाद‚यिष्य‚ते च द्वितीये प‚रिच्छेदे ।
	{\color{gray}{\rmlatinfont\textsuperscript{§~\theparCount}}}
	\pend% ending standard par
      ‚{\tiny $_{lb}$}‚

	  
	  \pstart \leavevmode% starting standard par
	त‚स्य च स‚त्काय‚द‚र्श‚न‚स्य \textbf{प्र‚हाणे} स‚ति \textbf{दोषाणां प्र‚हाण‚तः} प्राधान्यं । य‚त‚{\tiny $_{lb}$}‚ एव\textbf{न्त‚स्मात् स‚म्भ‚व}ति \textbf{स‚त्काय‚द‚र्श‚नाज्ज‚न्म} ये\textbf{षान्दोषाणा}न्त्तेषां त‚स्य स‚त्काय- \leavevmode\ledsidenote{\textenglish{145b/PSVTa}}‚{\tiny $_{lb}$}‚ द‚र्श‚न‚स्य \textbf{प्र‚ति‚{\tiny $_{७}$}‚प‚क्षो नैरात्म्य‚द‚र्श}न‚न्त‚स्या\textbf{भ्यासात् प्र‚हाणं} ।
	{\color{gray}{\rmlatinfont\textsuperscript{§~\theparCount}}}
	\pend% ending standard par
      ‚{\tiny $_{lb}$}‚

	  
	  \pstart \leavevmode% starting standard par
	\textbf{स च क्षीण‚दोषः} पुमानौद्देशिको \textbf{दुर‚न्व‚यो} दुर्बोधो \textbf{य‚दुप‚देशाद्} य‚स्य क्षीण‚{\tiny $_{lb}$}‚दोष‚स्योप‚देशा\textbf{द‚यं} प्र‚वृत्तिकामः \textbf{प्र‚तिप‚द्येत} । तेनोप‚दिष्ट‚म‚र्थ‚म‚नुतिष्ठेत् ॥ ० ॥‚{\tiny $_{lb}$}‚ \href{http://sarit.indology.info/?cref=pv.3.222}{२५}
	{\color{gray}{\rmlatinfont\textsuperscript{§~\theparCount}}}
	\pend% ending standard par
      ‚{\tiny $_{lb}$}‚

	  
	  \pstart \leavevmode% starting standard par
	\textbf{मा भूत् पुरुषाश्र‚यं} पुरुष‚हेतुक‚म्\textbf{व‚च‚न‚माग‚मः} । किं कार‚ण‚म [।] अन‚न्त‚रोक्तेन‚{\tiny $_{lb}$}‚ न्यायेनाग‚म\textbf{प्र‚णेतुर्दुर‚न्व‚य‚त्वात्} । दुर्बोध‚त्वात् ।
	{\color{gray}{\rmlatinfont\textsuperscript{§~\theparCount}}}
	\pend% ending standard par
      ‚{\tiny $_{lb}$}‚

	  
	  \pstart \leavevmode% starting standard par
	\textbf{गिरा}म्व‚च‚सां \textbf{मिथ्यात्व‚स्य} मृषार्थ‚त्व‚स्य ये \textbf{हेत}वो \textbf{दोषा} रागाद‚य\textbf{स्तेषां} [।]‚{\tiny $_{lb}$}‚ क‚र्त्त‚रि चेयं ष‚ष्ठी । आश्र‚य‚ण‚माश्र‚यः । \textbf{पुरुष}स्या\textbf{श्र‚य} इति स‚मासः । पुरुष‚श‚ब्दाच्च‚{\tiny $_{lb}$}‚ क‚र्म‚णि ष‚ष्ठी । न चोभ‚य‚प्राप्तौ क‚र्म‚णीति निय‚मः । शेष विभा\edtext{}{\edlabel{pvsvt_403-2}\label{pvsvt_403-2}\lemma{विभा}\Bfootnote{२ }}षेति विक‚ल्प‚नात् ।‚{\tiny $_{lb}$}‚ ‚{\tiny $_{lb}$}‚ \leavevmode\ledsidenote{\textenglish{404/s}}तेनाय‚म‚र्थो [।] मिथ्याहेतुभिर्दोषैः पुरुष‚प‚रिगृहीत‚त्वादिति ।
	{\color{gray}{\rmlatinfont\textsuperscript{§~\theparCount}}}
	\pend% ending standard par
      ‚{\tiny $_{lb}$}‚

	  
	  \pstart \leavevmode% starting standard par
	अथ‚वा दोषाणा‚{\tiny $_{२}$}‚मिति क‚र्म‚णि ष‚ष्ठी दोषाणां पुरुषेणाश्र‚यात् प‚रिग्र‚हात् ।
	{\color{gray}{\rmlatinfont\textsuperscript{§~\theparCount}}}
	\pend% ending standard par
      ‚{\tiny $_{lb}$}‚

	  
	  \pstart \leavevmode% starting standard par
	\textbf{अपौरुषेय}म्व‚च‚स्\textbf{स‚त्यार्थ}म्मिथ्यात्व‚हेतोर्दोष‚स्याभावा\textbf{दिति केचित्} मी मां स का‚{\tiny $_{lb}$}‚ \textbf{आच‚क्ष‚ते} ।
	{\color{gray}{\rmlatinfont\textsuperscript{§~\theparCount}}}
	\pend% ending standard par
      ‚{\tiny $_{lb}$}‚

	  
	  \pstart \leavevmode% starting standard par
	एत‚दुक्त‚म्भ‚व‚ति । त्रिविध‚म‚प्य‚प्रामाण्य‚म्मिथ्यात्वाज्ञान‚स‚ङ्श‚य‚ल‚क्ष‚णे वादे‚{\tiny $_{lb}$}‚ नास्त्येव । य‚तः श‚ब्दानां द्विविधः स्व‚भावो निस‚र्ग‚सिद्ध औपाधिक‚श्च त‚त्र निस‚र्ग‚{\tiny $_{lb}$}‚सिद्धो यो य‚थार्थ‚प्र‚तिपाद‚क‚त्व‚म् [।] अय‚थार्थ‚प्र‚तिपाद‚क‚त्वं पुन‚{\tiny $_{३}$}‚रौपाधिकः ।‚{\tiny $_{lb}$}‚ स्व‚भावः पुरुषाधीन‚त्वात् । त‚दाह ।
	{\color{gray}{\rmlatinfont\textsuperscript{§~\theparCount}}}
	\pend% ending standard par
      ‚{\tiny $_{lb}$}‚
	  \bigskip
	  \begingroup
	
	    
	    \stanza[\smallbreak]
	  {\normalfontlatin\large ``\qquad}श‚ब्द‚दोषोद्भ‚व‚स्ताव‚द् व‚क्त्र्य‚धीन इति स्थितं ।&‚{\tiny $_{lb}$}‚त‚द‚भावः क्व‚चित् ताव‚द् गुण‚व‚द्व‚क्त्रृक‚त्व‚तः । \href{http://sarit.indology.info/?cref=\%C5\%9Bv-codan\%C4\%81.62}{६२}&‚{\tiny $_{lb}$}‚त‚द्गुणैर‚प‚कृष्टानां श‚ब्दे संक्रान्त्य‚स‚म्भ‚वात् ।&‚{\tiny $_{lb}$}‚य‚द्वा व‚क्तुर‚भावेन न स्युर्दोषा निराश्र‚या इति\edtext{}{\edlabel{pvsvt_404-1}\label{pvsvt_404-1}\lemma{इति}\Bfootnote{\href{http://sarit.indology.info/?cref=\%C5\%9Bv-codan\%C4\%81}{ Ślokavārtika, Choda. }}} \href{http://sarit.indology.info/?cref=\%C5\%9Bv-codan\%C4\%81.63}{६३}{\normalfontlatin\large\qquad{}"}\&[\smallbreak]
	  
	  
	  
	  \endgroup
	‚{\tiny $_{lb}$}‚

	  
	  \pstart \leavevmode% starting standard par
	तेन वे दे पुरुष‚निवृत्तौ मिथ्यात्व‚निवृत्तिः । नाप्य‚नुत्प‚तिल‚क्ष‚ण‚म‚प्रामाण्य‚{\tiny $_{lb}$}‚म्वेदाद‚र्थाव‚ग‚तेः । नापि सं‚{\tiny $_{४}$}‚श‚य‚ल‚क्ष‚ण‚म‚प्रामाण्यं वेदाद‚र्थ‚ग‚तौ संश‚य‚स्याप्र‚तिभास‚{\tiny $_{lb}$}‚नात् । त‚दाह ।
	{\color{gray}{\rmlatinfont\textsuperscript{§~\theparCount}}}
	\pend% ending standard par
      ‚{\tiny $_{lb}$}‚
	  \bigskip
	  \begingroup
	
	    
	    \stanza[\smallbreak]
	  {\normalfontlatin\large ``\qquad}एवंभूत‚स्य वेद‚स्य ज्ञानोत्प‚त्तिं च कुर्व‚तः ।&‚{\tiny $_{lb}$}‚स्व‚रूप‚विप‚रीत‚त्व‚संश‚यौ भाष्य‚वारिताविति ॥\edtext{\textsuperscript{*}}{\edlabel{pvsvt_404-2}\label{pvsvt_404-2}\lemma{*}\Bfootnote{Ibid. }}{\normalfontlatin\large\qquad{}"}\&[\smallbreak]
	  
	  
	  
	  \endgroup
	‚{\tiny $_{lb}$}‚

	  
	  \pstart \leavevmode% starting standard par
	य‚त‚श्चाप्रामाण्यं त्र‚य‚न्निवृत्तं । निस‚र्ग‚सिद्ध‚श्च य‚यार्थ‚प्र‚तिपाद‚न‚ल‚क्ष‚णः‚{\tiny $_{lb}$}‚ स्व‚भावो वेद‚स्यास्ति त‚स्मात् स्व‚त एवास्य प्रामाण्य‚म‚र्थ‚प्र‚तिपाद‚क‚त्वात ।‚{\tiny $_{lb}$}‚ य‚त‚श्च श‚ब्दे व‚{\tiny $_{५}$}‚क्तृदोषेण बाध‚दुष्ट‚कार‚ण‚त्व‚ल‚क्ष‚ण‚स्य दोष‚स्य स‚म्भ‚व‚स्तेन वेदे‚{\tiny $_{lb}$}‚ पुरुष‚निवृत्तौ दोष‚निवृत्तेः स्व‚तः । प्रामाण्याप‚वाद‚क‚योर्बाध‚कार‚ण‚दुष्ट‚त्व‚ज्ञान‚{\tiny $_{lb}$}‚योर्निवृत्तेर्नाप्रामाण्याश‚ङ्का । त‚दाह ।
	{\color{gray}{\rmlatinfont\textsuperscript{§~\theparCount}}}
	\pend% ending standard par
      ‚{\tiny $_{lb}$}‚
	  \bigskip
	  \begingroup
	
	    
	    \stanza[\smallbreak]
	  {\normalfontlatin\large ``\qquad}\edtext{\textsuperscript{*}}{\edlabel{pvsvt_404-1b}\label{pvsvt_404-1b}\lemma{*}\Bfootnote{\href{http://sarit.indology.info/?cref=\%C5\%9Bv-codan\%C4\%81}{Ślokavārtika, Choda.}}}त‚त्राप‚वाद‚निर्मुक्तिर्व‚क्त्र्य‚भावाल्ल‚घीय‚सी ।&‚{\tiny $_{lb}$}‚वेदे तेनाप्र‚माण‚त्वं नाश‚ङ्काम‚पि ग‚च्छ‚ति । \href{http://sarit.indology.info/?cref=\%C5\%9Bv-codan\%C4\%81.68}{६८}&‚{\tiny $_{lb}$}‚प्रामाण्यं पौरुषेये तु प्र‚माणान्त‚र‚भाव‚तः ।&‚{\tiny $_{lb}$}‚\leavevmode\ledsidenote{\textenglish{405/s}}त‚द‚भावे तु त‚द्दुष्येद् वैदिकं न क‚दाच‚न ।&‚{\tiny $_{lb}$}‚तेनेत‚र‚प्र‚माणैर्या चोद‚नानाम‚संग‚तिः ।&‚{\tiny $_{lb}$}‚त‚यैव स्यात् प्र‚माण‚त्व‚म‚नुवाद‚त्व‚म‚न्य‚था ।&‚{\tiny $_{lb}$}‚चोद‚नार्थान्य‚थाभावं कुर्व‚त‚श्चानुमान‚तः ।&‚{\tiny $_{lb}$}‚त‚ज्ज्ञानेनैव यो बाधः स क‚थं विनिवार्य‚ते । \href{http://sarit.indology.info/?cref=\%C5\%9Bv-codan\%C4\%81.89}{८९}&‚{\tiny $_{lb}$}‚त‚न्मिथ्यात्वाद‚बाध‚श्चेत् प्राप्त‚म‚न्योन्य‚संश्र‚यं ।&‚{\tiny $_{lb}$}‚नानुमानाद‚तोन्य‚द्धि बाध‚कं किंच‚द‚{\tiny $_{७}$}‚स्ति ते । \href{http://sarit.indology.info/?cref=\%C5\%9Bv-codan\%C4\%81.90}{९०}{\normalfontlatin\large\qquad{}"}\&[\smallbreak]
	  
	  
	  
	  \endgroup
	\textsuperscript{\textenglish{146a/PSVTa}}‚{\tiny $_{lb}$}‚

	  
	  \pstart \leavevmode% starting standard par
	न चान्याप्र‚माणैर्वेदार्थ‚स्याग्र‚हेऽभावो र‚सादिव‚त् । अथ र‚सादेर‚प‚र‚या र‚स‚बुद्ध‚या‚{\tiny $_{lb}$}‚ ग्र‚हात् पूर्विकाया र‚स‚बुद्धेः प्रामाण्य‚म्वेदार्थेप्येव‚म्भ‚विष्य‚ति । त‚दाह ॥
	{\color{gray}{\rmlatinfont\textsuperscript{§~\theparCount}}}
	\pend% ending standard par
      ‚{\tiny $_{lb}$}‚
	  \bigskip
	  \begingroup
	
	    
	    \stanza[\smallbreak]
	  {\normalfontlatin\large ``\qquad}न चान्यैर‚ग्र‚हेर्थ‚स्य स्याद‚भावो र‚सादिव‚त् ।&‚{\tiny $_{lb}$}‚त‚द्धियैवार्थ‚बोध‚श्चेत्तादृग्ध‚र्मे भ‚विष्य‚ति ।&‚{\tiny $_{lb}$}‚म‚मासिद्ध‚मितीदं चेद् वेदाज्जातेऽव‚बोध‚ने ।&‚{\tiny $_{lb}$}‚व‚क्तुन्न द्वेष‚मात्रेण युज्य‚ते स‚त्य‚वादिनेति । \href{http://sarit.indology.info/?cref=\%C5\%9Bv-codan\%C4\%81.91-92}{९१, ९२}{\normalfontlatin\large\qquad{}"}\&[\smallbreak]
	  
	  
	  
	  \endgroup
	‚{\tiny $_{lb}$}‚

	  
	  \pstart \leavevmode% starting standard par
	\textbf{न ख‚ल्वि}त्यादिना कारिकार्थ‚माच‚ष्टे । \textbf{स‚र्व एवे}ति पौरुषेयो‚{\tiny $_{१}$}‚ऽ पौरुषेय‚श्च ।‚{\tiny $_{lb}$}‚ \textbf{स‚म्भाव्य‚विप्र‚ल‚म्भः} स‚म्भाव्य अशंक‚नीयो विप्र‚ल‚म्भो विस‚म्वादोस्येति । किन्तु‚{\tiny $_{lb}$}‚ पौरुषेय एव स‚म्भाव्य‚विप्र‚ल‚म्भः । \textbf{विप्र‚ल‚म्भ‚हेतू}नां विस‚म्वाद‚हेतूना\textbf{न्दोषाणां‚{\tiny $_{lb}$}‚ पुरुषाश्र‚यात्} । \href{http://sarit.indology.info/?cref=pv.3.224}{२२७}
	{\color{gray}{\rmlatinfont\textsuperscript{§~\theparCount}}}
	\pend% ending standard par
      ‚{\tiny $_{lb}$}‚

	  
	  \pstart \leavevmode% starting standard par
	य‚त्पुन\textbf{र‚पौरुषेय‚न्त‚त्स‚त्यार्थ‚मित्येके} । य‚स्माद् वेदेषु मिथ्यात्व\textbf{कार‚णा}नां पुरुषा‚{\tiny $_{lb}$}‚णाम\textbf{भावः कार्य}स्य मिथ्यात्व‚स्या\textbf{भावं} साध‚य‚तीत्य‚पौरुषेयं स‚त्यार्थ‚मिति‚{\tiny $_{२}$}‚ [।]
	{\color{gray}{\rmlatinfont\textsuperscript{§~\theparCount}}}
	\pend% ending standard par
      ‚{\tiny $_{lb}$}‚

	  
	  \pstart \leavevmode% starting standard par
	य \textbf{एव‚म्वादि}न‚स्तानेव मी मां स का न् प्र\textbf{त्य‚न्ये प्र‚च‚क्ष‚ते} । प‚र‚मुखेन\edtext{}{\lemma{मुखेन}\Bfootnote{? न}}‚{\tiny $_{lb}$}‚ शास्त्र‚कार एवाह । \textbf{गिरां स‚त्त्य‚त्व}स्य ये \textbf{हेत‚वो गुणा}स्ते\textbf{षां} पु\textbf{रुषाश्र‚या}द‚पौरुषेयेषु‚{\tiny $_{lb}$}‚ वाक्येषु पुरुष‚निवृत्त्या स‚त्त्य‚त्व‚कार‚ण‚स्य गुण‚स्य निवृत्तेः कार्य‚स्यापि स‚त्त्य‚त्व‚स्य‚{\tiny $_{lb}$}‚ निवृत्तिरि\textbf{त्य‚पौरुषेय}म्वाक्यं \textbf{मिथ्यात्वं किन्न} भ‚व‚ति ।
	{\color{gray}{\rmlatinfont\textsuperscript{§~\theparCount}}}
	\pend% ending standard par
      ‚{\tiny $_{lb}$}‚

	  
	  \pstart \leavevmode% starting standard par
	१ एत‚दुक्त‚म्भ‚व‚ति । श‚ब्दे स‚त्य‚त्त्व‚मिथ्यात्व‚योः पुरुषाय‚त्त‚त्वा‚{\tiny $_{३}$}‚द् य‚दि‚{\tiny $_{lb}$}‚ ‚{\tiny $_{lb}$}‚ \leavevmode\ledsidenote{\textenglish{406/s}}पुरुष‚निवृत्तौ स‚त्त्यार्थ‚त्व‚मिष्य‚ते मिथ्यार्थ‚त्वं किन्नेष्य‚त इत्युच्य‚ते । प‚र‚मार्थ‚त‚स्तु‚{\tiny $_{lb}$}‚ पुरुष‚निवृत्त्या स‚त्त्यार्थ‚त्व‚मिथ्यार्थ‚योर्निवृत्तेरान‚र्थ‚क्याद‚नुत्प‚त्तिल‚क्ष‚ण‚मेवाप्रामाण्यं ।‚{\tiny $_{lb}$}‚ तेन [।]
	{\color{gray}{\rmlatinfont\textsuperscript{§~\theparCount}}}
	\pend% ending standard par
      ‚{\tiny $_{lb}$}‚
	  \bigskip
	  \begingroup
	
	    
	    \stanza[\smallbreak]
	  {\normalfontlatin\large ``\qquad}म‚मासिद्ध‚मितीदं चेद् वेदाज्जातेव‚बोध‚ने ।&‚{\tiny $_{lb}$}‚व‚क्तुन्न द्वेष‚मात्रेण युज्य‚ते स‚त्त्य‚वादिनेति । \href{http://sarit.indology.info/?cref=\%C5\%9Bv-codan\%C4\%81.92-93}{९२, ९३}{\normalfontlatin\large\qquad{}"}\&[\smallbreak]
	  
	  
	  
	  \endgroup
	‚{\tiny $_{lb}$}‚
	    
	    \stanza[\smallbreak]
	  निर‚स्तं । वेदात् स्व‚भाव‚तोर्थाव‚बोध‚स्यानुत्प‚त्तेः ।\&[\smallbreak]
	  
	  
	  ‚{\tiny $_{lb}$}‚

	  
	  \pstart \leavevmode% starting standard par
	२ किञ्च [।] व‚र्ण्णानाम‚वाच‚क‚{\tiny $_{४}$}‚रूप‚त्वं प्र‚त्येक स‚म‚स्तानां चावाच‚क‚त्वाद्‚{\tiny $_{lb}$}‚ व‚र्ण्ण‚रूप‚श्च वेद इति क‚थ‚म‚तोर्थ‚ज्ञानं ।
	{\color{gray}{\rmlatinfont\textsuperscript{§~\theparCount}}}
	\pend% ending standard par
      ‚{\tiny $_{lb}$}‚

	  
	  \pstart \leavevmode% starting standard par
	न‚न्व‚गृहीत‚स‚म‚य‚स्यापि वाक्यादुच्चारितात् कोप्य‚र्थोनेनोक्त इति स‚न्देहो‚{\tiny $_{lb}$}‚ दृश्य‚ते ।
	{\color{gray}{\rmlatinfont\textsuperscript{§~\theparCount}}}
	\pend% ending standard par
      ‚{\tiny $_{lb}$}‚

	  
	  \pstart \leavevmode% starting standard par
	स चैवं स्याद य‚द्य‚र्थ‚प्र‚तिपाद‚ने श‚ब्द‚स्य स्व‚भावेन श‚क्तिः स्यात् । एव‚न्त‚र्हि‚{\tiny $_{lb}$}‚ स‚न्देह‚ल‚क्ष‚ण‚म‚स्याप्रामाण्यं स्यात् । इष्टेनिष्टेचार्थे प्र‚काश‚न‚श‚क्तिस‚म्भ‚वात् । य‚दि‚{\tiny $_{lb}$}‚ चास्य स्व‚भाव‚{\tiny $_{५}$}‚ त एव सा श‚क्तिः किं संकेतेन । य‚था दीप‚स्यार्थ‚प्र‚काश‚ने‚{\tiny $_{lb}$}‚ श‚क्त‚स्येन्द्रियापेक्षा त‚था श‚ब्द‚स्यापि संकेतापेक्षेति चेत् । न । प्र‚दीपेन्द्रिय‚योः‚{\tiny $_{lb}$}‚ प्र‚त्येक‚म‚भावेप्य‚र्थ‚प्र‚काश‚क‚त्वाभावात् । त‚त्रान्योन्यापेक्ष‚त्वं युक्तं । नैवं श‚ब्द‚श‚क्ति‚{\tiny $_{lb}$}‚संकेत‚योः । संकेत‚मात्रेणैवार्थ‚प्र‚तीतेरुत्प‚त्तेः । त‚स्मान्न स्व‚भाव‚तः श‚ब्दोर्थ‚प्र‚तिपाद‚न‚{\tiny $_{lb}$}‚स‚म‚र्थ इत्यु‚{\tiny $_{६}$}‚त्प‚त्तिल‚क्ष‚ण‚म‚प्रामाण्य‚म् ।
	{\color{gray}{\rmlatinfont\textsuperscript{§~\theparCount}}}
	\pend% ending standard par
      ‚{\tiny $_{lb}$}‚

	  
	  \pstart \leavevmode% starting standard par
	३ न‚न्व‚ग्निहोत्र‚ञ्जुहुयात् स्व‚र्ग‚काम इत्यादि वाक्येष्व‚ग्निहोत्रादेः‚{\tiny $_{lb}$}‚ स्व‚र्गादिसाध‚नोपाय‚त्वं प्र‚तीय‚त एवेति क‚थ‚म‚नुत्प‚त्तिल‚क्ष‚ण‚म‚प्रामाण्यं ।
	{\color{gray}{\rmlatinfont\textsuperscript{§~\theparCount}}}
	\pend% ending standard par
      ‚{\tiny $_{lb}$}‚

	  
	  \pstart \leavevmode% starting standard par
	स‚त्त्य‚मेत‚त् [।] केव‚लं प्र‚तीतिर्ह्य‚प्र‚तीतेर्बाधिका । न तु मिथ्यात्व‚स्य ।‚{\tiny $_{lb}$}‚ त‚स्यापि प्र‚तीतेः । तेन किमेभिर्वाक्यैर‚ग्निहोत्रादिः स्व‚र्ग‚साध‚नानुपाय एवोपाय‚त‚या‚{\tiny $_{lb}$}‚ \leavevmode\ledsidenote{\textenglish{146b/PSVTa}} प्र‚द‚{\tiny $_{७}$}‚र्श्य‚तेऽथोपाय एवेति मिथ्यात्वाशंका न निव‚र्त्त‚त एव । बाध‚क‚प्र‚माणाभावा‚{\tiny $_{lb}$}‚द‚पौरुषेय‚त्व‚स्य मिथ्यार्थ‚त्वेन स‚ह विरोधाभाव‚च्च । दृष्ट‚श्चापौरुषेयाणाम्वित‚थ‚{\tiny $_{lb}$}‚ज्ञान‚हेतुत्वं । त‚द्य‚था ज्योत्स्नादीनां शुक्ल‚व‚स्त्रादौ पीत‚ज्ञान‚हेतुत्वं । तेन चोद‚नार्था‚{\tiny $_{lb}$}‚न्य‚थाभावोनुमान‚तः क्रिय‚त इति क‚थ‚न्त‚ज्ज्ञानेन बाधा । र‚सादिज्ञानानान्तु तृप्त्यादि‚{\tiny $_{lb}$}‚कार्याविस‚म्वादा‚{\tiny $_{१}$}‚देव प्र‚थ‚मं प्रामाण्य‚निश्च‚याद‚न्य‚दा त्व‚भ्यासादिना स्व‚त एव‚{\tiny $_{lb}$}‚ प्रामाण्य‚निश्च‚यो युक्त इति न मिथ्यात्वाश‚ङ्का । वेदे तु नैव क‚दाचिद‚प्य‚विस‚म्वादः‚{\tiny $_{lb}$}‚ प्र‚तिप‚न्न इति क‚थ‚म‚न्य‚दापि स्व‚तः प्रामाण्य‚निश्च‚यः । तेन स‚त्य‚पि विज्ञाने प्र‚ति‚{\tiny $_{lb}$}‚भादेर्य‚थैव हि स्वात‚न्त्र्यान्न प्र‚माण‚त्वं त‚था वेदेपि दृश्य‚तां ।
	{\color{gray}{\rmlatinfont\textsuperscript{§~\theparCount}}}
	\pend% ending standard par
      ‚{\tiny $_{lb}$}‚

	  
	  \pstart \leavevmode% starting standard par
	४ किंच [।] चोद‚नार्थ‚ज्ञान‚स्याविद्य‚मानोप‚ल‚म्भ‚न‚{\tiny $_{२}$}‚रूप‚त्वात् मिथ्यात्वं । त‚था‚{\tiny $_{lb}$}‚ ‚{\tiny $_{lb}$}‚ \leavevmode\ledsidenote{\textenglish{407/s}}हि कार्येर्थे वेद‚स्य प्रामाण्य‚मिष्य‚ते । कार्य‚श्चार्थानुष्ठेय एव, स च भावित्वेनाविद्य‚{\tiny $_{lb}$}‚मान‚त्वान्न चोद‚नाज्ञान‚काल‚भावीति । त‚त्क‚थ‚म‚विद्य‚मान‚विष‚य‚त्वाच्चोद‚नाज्ञान‚स्य‚{\tiny $_{lb}$}‚ न मिथ्यात्वं । स‚र्व‚विक‚ल्पानां च पूर्व‚म‚व‚स्तुविष‚य‚त्व‚स्य प्र‚तिपादित‚त्वात् तेनापि‚{\tiny $_{lb}$}‚ चोद‚नाज्ञान‚स्य मिथ्यात्व‚मेव । किं च । लोक‚वेद‚योर्व‚र्ण्णाः प‚दानि चाभि‚{\tiny $_{३}$}‚न्नान्येव‚{\tiny $_{lb}$}‚ वाक्य‚भेद‚स्तु केव‚ल‚मिष्य‚ते । लोके च प‚दानाम‚र्थः संकेत‚व‚शात् । लौकिक‚प‚दार्थ‚श्च‚{\tiny $_{lb}$}‚ वैदिकानां प‚दानान्तेन पौरुषेय एवार्थ‚स‚म्ब‚न्धः । लौकिक‚प‚दार्थ‚द्वारेण च वैदिक‚{\tiny $_{lb}$}‚वाक्यार्थाव‚ग‚मो भ‚व‚तीति पौरुषेय एवासौ लौकिक‚वाक्य इव । लोके च प‚दान्य‚ने‚{\tiny $_{lb}$}‚कार्थानीति वैदिक‚वाक्य‚स्याप्य‚नेकार्थ‚तास‚म्भ‚वाद् विप‚रीतार्था‚{\tiny $_{४}$}‚ शंका न निव‚र्त्त‚त‚{\tiny $_{lb}$}‚ इति संश‚य‚ल‚क्ष‚ण‚म‚प्र‚माण्यं वेद‚स्येति ।
	{\color{gray}{\rmlatinfont\textsuperscript{§~\theparCount}}}
	\pend% ending standard par
      ‚{\tiny $_{lb}$}‚

	  
	  \pstart \leavevmode% starting standard par
	\textbf{य‚थे}त्यादिना कारिकार्थं व्याच‚ष्टे । \textbf{रागादिप‚रीतो} रागादियुक्तः । \textbf{द‚ये}ति‚{\tiny $_{lb}$}‚ क‚रुणा । सैव सात्मीभूता \textbf{ध‚र्म‚ता । आदि}श‚ब्दात् प्र‚ज्ञाश्र‚द्धादि । तै\textbf{र्युक्तः । त}त्त‚स्माद्‚{\tiny $_{lb}$}‚ \textbf{य‚था व‚च‚न‚स्य पुरुषाश्र‚यात् मिथ्यार्थ‚ता । त‚था स‚त्त्यार्थ‚तापि । स} इति पुरुषः । \textbf{ताम‚पि}‚{\tiny $_{lb}$}‚ स‚त्त्यार्थ‚ता\textbf{न्निव‚र्त्त‚य‚ति} । न केव‚{\tiny $_{५}$}‚लं मिथ्यार्थ‚तां । इति पुरुष‚निवृत्तौ त‚द्ध‚र्म‚योः‚{\tiny $_{lb}$}‚ स‚त्त्यार्थ‚त्व‚मिथ्यार्थ‚त्व‚योर्निवृत्ता\textbf{वान‚र्थ‚क्यं} वेदे \textbf{स्यात्} ।
	{\color{gray}{\rmlatinfont\textsuperscript{§~\theparCount}}}
	\pend% ending standard par
      ‚{\tiny $_{lb}$}‚

	  
	  \pstart \leavevmode% starting standard par
	अथ पुरुष‚निवृत्ताव‚पि स‚त्त्यार्थ‚तेष्य‚ते । त‚दा स‚त्त्यार्थ‚ता\textbf{विप‚र्य‚यो} मिथ्यार्थ‚ता‚{\tiny $_{lb}$}‚ \textbf{वा} स्यात् ।
	{\color{gray}{\rmlatinfont\textsuperscript{§~\theparCount}}}
	\pend% ending standard par
      ‚{\tiny $_{lb}$}‚

	  
	  \pstart \leavevmode% starting standard par
	५ स्यान्म‚तं [।] न पुरुषापेक्ष‚या श‚ब्दानाम‚र्थ‚व‚त्ता किन्तु स्व‚भाव‚त एवे‚{\tiny $_{lb}$}‚त्याह । \textbf{न ही}त्यादि । \textbf{प्र‚कृत्या} स्व‚भावे\textbf{नार्थ‚व‚न्तः} [।] किं कार‚णं [।] \textbf{स‚म‚यात्}‚{\tiny $_{lb}$}‚ सं‚{\tiny $_{६}$}‚ केतात् । \textbf{त‚तः} श‚ब्देभ्यो\textbf{र्थ‚ख्याते}र‚र्थ‚प्र‚तीतेः । \textbf{काय‚संज्ञादिव‚त्} । ह‚स्त‚विकाराक्षिनि‚{\tiny $_{lb}$}‚कोचाद‚यः काय‚संज्ञा । य‚था त‚त्र स‚म‚यात् क्व‚चिद‚र्थ‚प्र‚तिप‚त्तिस्त‚द्व‚त् । य‚दि पुनः‚{\tiny $_{lb}$}‚ स्व‚भाव‚त एव श‚व्दा अर्थ‚प्र‚काश‚काः स्युस्त‚दा न संकेत‚म‚पेक्षेर‚न् । त‚स्मान्न‚{\tiny $_{lb}$}‚ स्व‚तोर्थ‚प्र‚काश‚न‚योग्य‚ता श‚व्दानां [।] किन्त\textbf{र्ह्य‚प्रातिकूल्य‚न्तु} य‚थासंकेतं प्र‚वृ‚{\tiny $_{७}$}‚त्ति- \leavevmode\ledsidenote{\textenglish{147a/PSVTa}}‚{\tiny $_{lb}$}‚ रेव श‚ब्दानां \textbf{योग्य‚ता} । किं कार‚णं [।] \textbf{स‚म‚ये} संकेत‚काले \textbf{त}स्य संकेत‚क‚र्त्तुर्य‚त्र‚{\tiny $_{lb}$}‚ नियोक्तु\textbf{मिच्छा} त‚या य‚थेष्टं श‚व्दानां \textbf{प्र‚ण‚य‚नात्} प्र‚वृत्तेः । य‚दि पुन‚र्निस‚र्ग‚सिद्धाः‚{\tiny $_{lb}$}‚ स्व‚भाव‚सिद्धाः क्व‚चिद‚र्थे श‚ब्दाः स्युस्त‚दा \textbf{निस‚र्ग‚सिद्धेषु} पुरुषे\textbf{च्छाव‚शाद्} य‚थेष्टं‚{\tiny $_{lb}$}‚ ‚{\tiny $_{lb}$}‚ \leavevmode\ledsidenote{\textenglish{408/s}}संकेतेनार्थ\textbf{प्र‚तिपाद‚नायोगात्} ।
	{\color{gray}{\rmlatinfont\textsuperscript{§~\theparCount}}}
	\pend% ending standard par
      ‚{\tiny $_{lb}$}‚

	  
	  \pstart \leavevmode% starting standard par
	त‚स्माद\textbf{न‚र्थ‚काः} स्व‚तः श‚ब्दास्तेन‚र्थ‚कास्स‚न्तः \textbf{पुरुष‚संस्कारात्} पुरुष‚संकेता‚{\tiny $_{lb}$}‚\textbf{द‚र्थ‚व‚न्तः‚{\tiny $_{१}$}‚} । अथ माभूद‚न‚र्थ‚क‚त्व‚मिति पुरुष‚संस्कारापेक्ष‚यार्थ‚व‚त्त्व‚न्तेषामिष्य‚ते ।‚{\tiny $_{lb}$}‚ त‚दा पौरुषेय‚त‚यैव स्यात् । य‚स्मात् \textbf{त‚त्संस्कार्य‚तैव} पुरुष‚संस्कार्य‚तैव \textbf{चैषां}‚{\tiny $_{lb}$}‚ श‚ब्दानां \textbf{पौरुषेय‚ता युक्ता । न} पुरुषा\textbf{दुत्प‚त्तिः} । किं कार‚णं [।] \textbf{त‚त एव}‚{\tiny $_{lb}$}‚ पुरुष‚स‚म‚यादे\textbf{वार्थ‚विप्र‚ल‚म्भाद्} विस‚म्वादात् । न पुरुषादुत्प‚त्तेर्विप्र‚ल‚म्भः । य‚स्मा‚{\tiny $_{lb}$}‚\textbf{दुत्प‚न्नोपि} पुरुषाच्छ‚ब्दो\textbf{न्य‚था स‚मित} इति मि‚{\tiny $_{१}$}‚थ्यार्थ‚ताविरोधेन य‚थाभावं \textbf{स‚मितः}‚{\tiny $_{lb}$}‚ संकेतितः । \textbf{नोप‚रोधी} न विप्र‚ल‚म्भ‚कः । \textbf{त‚द‚न्य} इति श‚ब्दाद‚न्यः \textbf{पुरुष‚ध‚र्म} उन्मेष‚{\tiny $_{lb}$}‚निमेषादिः स्व‚तोन‚र्थ‚कोपि य‚थाभावं पुरुषेण स‚मितो न विप्र‚ल‚म्भ‚क‚स्त‚द्व‚त् ।
	{\color{gray}{\rmlatinfont\textsuperscript{§~\theparCount}}}
	\pend% ending standard par
      ‚{\tiny $_{lb}$}‚

	  
	  \pstart \leavevmode% starting standard par
	उप‚संह‚र‚न्नाह । \textbf{त‚दि}ति । त‚स्मा\textbf{द‚यं} पुरुषो \textbf{निव‚र्त्त‚मानः स्व‚कृत‚स‚म‚यात्‚{\tiny $_{lb}$}‚ स‚म्भ‚व} उत्पादो य‚स्या \textbf{अर्थ‚प्र‚तिभा}याः अर्थ‚बुद्धेस्\textbf{ताम‚पि नि‚{\tiny $_{३}$}‚व‚र्त्त‚य‚ति । त‚दि}ति‚{\tiny $_{lb}$}‚ त‚स्मात् । \textbf{कुत‚स्त‚न्निवृत्त्या} त‚स्य पुरुष‚स्य निवृत्त्या \textbf{स‚त्त्यार्थ‚ता} किन्त्वान‚र्थ‚क्य‚मेव‚{\tiny $_{lb}$}‚ स्यात् ।
	{\color{gray}{\rmlatinfont\textsuperscript{§~\theparCount}}}
	\pend% ending standard par
      ‚{\tiny $_{lb}$}‚
	    
	    \stanza[\smallbreak]
	  एतेन य‚दुच्य‚ते ।&‚{\tiny $_{lb}$}‚य‚त्पूर्वाप‚र‚योः कोट्योः प‚रैः साध‚न‚मुच्य‚ते ।&‚{\tiny $_{lb}$}‚त‚न्निराक‚र‚णं कृत्वा कृतार्था वेद‚वादिनः ॥&‚{\tiny $_{lb}$}‚पूर्वा वेद‚स्य या कोटिः पौरुषेय‚त्व‚ल‚क्ष‚णा [।]&‚{\tiny $_{lb}$}‚प‚रा विनाश‚रूपा च त‚द‚भावो हि नित्य‚तेति ।\edtext{\textsuperscript{*}}{\edlabel{pvsvt_408-1}\label{pvsvt_408-1}\lemma{*}\Bfootnote{\href{http://sarit.indology.info/?cref=\%C5\%9Bv}{ Ślokavārtika. }}}\&[\smallbreak]
	  
	  
	  ‚{\tiny $_{lb}$}‚

	  
	  \pstart \leavevmode% starting standard par
	त‚द‚पास्तं स‚त्य‚पौरुषेय‚त्वे नित्य‚{\tiny $_{४}$}‚त्वे च वेद‚स्यान‚र्थ‚क्येनाकृतार्थ‚त्वात् ।‚{\tiny $_{lb}$}‚ ६ \textbf{अथ पुन‚रुत्प‚त्तिरेव पौरुषेय‚ता न स‚म‚याख्यान‚न्न} संकेत‚क‚र‚णं पौरुषेय‚ता‚{\tiny $_{lb}$}‚ तेनापौरुषेय‚त्वादेव य‚थार्थो वेद इति भावः ।
	{\color{gray}{\rmlatinfont\textsuperscript{§~\theparCount}}}
	\pend% ending standard par
      ‚{\tiny $_{lb}$}‚

	  
	  \pstart \leavevmode% starting standard par
	एव‚म‚प्य\textbf{र्थ‚ज्ञाप‚न‚हेतु}र‚र्थाभिव्य‚क्तिहेतुः \textbf{संकेतः पुरुषाश्र‚य} पुरुषेच्छानुरोधी‚{\tiny $_{lb}$}‚ ‚{\tiny $_{lb}$}‚ ‚{\tiny $_{lb}$}‚ \leavevmode\ledsidenote{\textenglish{409/s}}न य‚थार्थ‚म‚व‚श्य‚म्व‚र्त्त‚त इति \textbf{गिराम‚पौरुषेय‚त्}वेभ्युप‚ग‚म्य‚माने । \textbf{अ‚{\tiny $_{५}$}‚तः} पुरुषेच्छा‚{\tiny $_{lb}$}‚नुरोधिनः संकेतात् \textbf{मिथ्यात्व‚स‚म्भ‚वः} ।
	{\color{gray}{\rmlatinfont\textsuperscript{§~\theparCount}}}
	\pend% ending standard par
      ‚{\tiny $_{lb}$}‚

	  
	  \pstart \leavevmode% starting standard par
	७ \textbf{कि}मित्यादि विव‚र‚णं । \textbf{अस्ये}ति वेद‚स्य य\textbf{तो हि स‚म‚यात्} संकेताद‚र्थ‚{\tiny $_{lb}$}‚\textbf{प्र‚तीतिर}र्थ‚प्र‚काश‚नं \textbf{स} स‚म‚यः \textbf{पौरुषेय}स्त‚तो \textbf{वित‚थोप्य}लीकोपि \textbf{स्यात्} । स्वात‚न्त्र्यात् ।‚{\tiny $_{lb}$}‚ त‚त‚श्च \textbf{शीलं साध‚नं} हेतुर्य‚स्य \textbf{स्व‚र्ग}स्य । त‚थाभूत‚श्चासौ स्व‚र्ग‚श्च सुमेरुपृष्ठ‚ल‚क्ष‚ण‚{\tiny $_{lb}$}‚स्त‚स्य य‚द्वेदे व\textbf{च‚न}न्त‚द\textbf{न्य‚था स‚म‚ये‚{\tiny $_{६}$}‚ न विप‚र्यास‚येत्} । निर‚तिश‚या प्रीतिः स्व‚र्ग इत्ये‚{\tiny $_{lb}$}‚ व‚म्विप‚रीतार्थं कुर्याद‚ग्निहोत्रादिसाध‚नेन च विप‚रीतार्थं कुर्या\textbf{त्तेन} कार‚णे\textbf{नाय‚{\tiny $_{lb}$}‚थार्थ‚म}पि \textbf{प्र‚काश‚न‚स‚म्भ‚वात्} [।]
	{\color{gray}{\rmlatinfont\textsuperscript{§~\theparCount}}}
	\pend% ending standard par
      ‚{\tiny $_{lb}$}‚

	  
	  \pstart \leavevmode% starting standard par
	\textbf{स एव दोषो} यः पौरुषेयेषूक्तः [।] पुरुष‚दोषात् स‚म्भाव्य‚विप्र‚ल‚म्भः‚{\tiny $_{lb}$}‚ पौरुषेय इति । \href{http://sarit.indology.info/?cref=pv.3.225}{२२८}
	{\color{gray}{\rmlatinfont\textsuperscript{§~\theparCount}}}
	\pend% ending standard par
      ‚{\tiny $_{lb}$}‚

	  
	  \pstart \leavevmode% starting standard par
	८ न‚न्व‚पौरुषेय एव श‚ब्दार्थ‚योः स‚म्ब‚न्धः । त‚था हि य‚थेदानीन्त‚ना वृद्धाः‚{\tiny $_{७}$}‚ \leavevmode\ledsidenote{\textenglish{147b/PSVTa}}‚{\tiny $_{lb}$}‚ पूर्व‚प्र‚सिद्ध‚मेव श‚ब्दार्थ‚स‚म्ब‚न्ध‚मुप‚दिश‚न्ति । त‚था पूर्व‚पूर्व‚वृद्धा अपीत्य‚नादित्वाद‚{\tiny $_{lb}$}‚पौरुषेय एव स‚म्ब‚न्धः । त‚दुक्तं ।
	{\color{gray}{\rmlatinfont\textsuperscript{§~\theparCount}}}
	\pend% ending standard par
      ‚{\tiny $_{lb}$}‚
	  \bigskip
	  \begingroup
	
	    
	    \stanza[\smallbreak]
	  {\normalfontlatin\large ``\qquad}श‚ब्दार्थानादितां मुक्त्वा स‚म्ब‚न्धानादिकार‚णं ।&‚{\tiny $_{lb}$}‚अस्ति नान्य‚द‚तो वेदे स‚म्ब‚न्धादिर्न विद्य‚त इति\edtext{}{\edlabel{pvsvt_409-1}\label{pvsvt_409-1}\lemma{इति}\Bfootnote{\href{http://sarit.indology.info/?cref=\%C5\%9Bv-sa\%E1\%B9\%83bandha}{ Ślokavartika. }}}{\normalfontlatin\large\qquad{}"}\&[\smallbreak]
	  
	  
	  
	  \endgroup
	‚{\tiny $_{lb}$}‚

	  
	  \pstart \leavevmode% starting standard par
	स च स‚म्ब‚न्ध‚स्त्रिप्र‚माण‚कः । त‚था हि [।] श्रोतुर‚र्थ‚प्र‚तिप‚त्त‚ये केन‚चिद्‚{\tiny $_{lb}$}‚ वृद्धेन श‚ब्दे प्र‚युक्तेऽन्यः पार्श्व‚स्थः प्र‚तिप‚त्ता प्र‚योक्तार‚म्वाच्यं वाच‚कं प्र‚त्य‚क्षे‚{\tiny $_{१}$}‚ ण‚{\tiny $_{lb}$}‚ प्र‚तिप‚द्य‚ते । श्रोतुश्च प्र‚तिप‚न्न‚त्वं प्र‚वृत्तिद्वारेणाव‚ग‚च्छ‚ति । अर्थ‚प्र‚तिप‚त्त्य‚न्य‚थानु‚{\tiny $_{lb}$}‚प‚प‚त्त्या च श‚व्दार्थाश्रिता वाच्य‚वाच‚क‚श‚क्तिं चाव‚ग‚च्छ‚तीति त्रिप्र‚माण‚क एव‚{\tiny $_{lb}$}‚ स‚म्ब‚न्धः । त‚दुक्तं ॥
	{\color{gray}{\rmlatinfont\textsuperscript{§~\theparCount}}}
	\pend% ending standard par
      ‚{\tiny $_{lb}$}‚
	  \bigskip
	  \begingroup
	
	    
	    \stanza[\smallbreak]
	  {\normalfontlatin\large ``\qquad}श‚ब्दआन् वृद्धामिधेयांश्च प्र‚त्य‚क्षेणात्र प‚श्य‚ति ।&‚{\tiny $_{lb}$}‚श्रोतुश्च प्र‚तिप‚न्न‚त्व‚म‚नुमानेन चेष्ट‚या ॥&‚{\tiny $_{lb}$}‚अन्य‚थानुप‚प‚त्त्या च बुद्ध्येच्छ‚क्तिं द्व‚याश्रितां&‚{\tiny $_{lb}$}‚अर्थाप‚त्त्या च बुध्य‚न्ते स‚म्ब‚न्ध‚न्त्रि‚{\tiny $_{२}$}‚ प्र‚माण‚क‚मिति ।{\normalfontlatin\large\qquad{}"}\&[\smallbreak]
	  
	  
	  
	  \endgroup
	‚{\tiny $_{lb}$}‚‚{\tiny $_{lb}$}‚‚{\tiny $_{lb}$}‚\textsuperscript{\textenglish{410/s}}

	  
	  \pstart \leavevmode% starting standard par
	त‚त्राह । स\textbf{म्ब‚न्धापौरुषेय‚त्}व इत्यादि । श‚ब्दार्थ‚योस्स\textbf{म्ब‚न्धापौरुष‚य‚त्वेभ्}युप‚{\tiny $_{lb}$}‚ग‚म्य‚माने \textbf{स्यात् प्र‚तीति}र‚र्थ‚प्र‚तिप‚त्तिर\textbf{स‚म्विदः} । संकेत‚ज्ञानं स‚म्वित् । सा न‚{\tiny $_{lb}$}‚ विद्य‚ते य‚स्य त‚स्य । \textbf{अकार्यः} पुरुषैर‚ज‚न्यः \textbf{स‚म्ब}न्धो येषान्ते त‚था । \textbf{ते श‚ब्दा अर्थेषु}‚{\tiny $_{lb}$}‚ वाच्येषु \textbf{पुरुषै}र्य‚थार्थ‚प्र‚काश‚क‚त्वात् । स्व‚भावा\textbf{द‚न्य‚था विप‚र्य‚स्य‚न्ते} वित‚थ‚ज्ञा‚{\tiny $_{३}$}‚न‚हेत‚वो‚{\tiny $_{lb}$}‚ न क्रिय‚न्ते । \textbf{तेना}विप‚र्यासेना\textbf{दोषो} विस‚म्वाद‚ल‚क्ष‚णो दोषो नास्ति । \textbf{इदानीमिति}‚{\tiny $_{lb}$}‚ स‚म्ब‚न्धापौरुषेय‚त्वे \textbf{किं संकेतेन} प्र‚योज‚नं । \textbf{य‚स्मात् स हि} श‚ब्दार्थ‚यो\textbf{स्स‚म्ब‚न्धः} स‚{\tiny $_{lb}$}‚ चेद\textbf{र्थ‚प्र‚तीति}हेतुः स‚म्ब‚न्धोपौरुषेय‚स्त‚दा \textbf{नायं} पुरुषो निस‚र्ग‚सिद्ध‚स‚म्ब‚न्ध‚त्वाच्छ‚{\tiny $_{lb}$}‚ब्दाद‚र्थं प्र‚तिप‚द्य‚मानः \textbf{स‚म‚य‚म‚पेक्षेत} । प्र‚दीपादिव‚त् । अपेक्ष‚ते च संकेत‚न्त‚स्मा‚{\tiny $_{४}$}‚न्न‚{\tiny $_{lb}$}‚ श‚ब्दानाम‚र्थेन स‚हापौरुषेयः स‚म्ब‚न्धो राज‚चिह्नादिव‚त् ।
	{\color{gray}{\rmlatinfont\textsuperscript{§~\theparCount}}}
	\pend% ending standard par
      ‚{\tiny $_{lb}$}‚

	  
	  \pstart \leavevmode% starting standard par
	९ स्यान्म‚त‚म् [।] अपौरुषेय एव स‚म्ब‚न्धः स तु संकेत‚निर‚पेक्षो प्र‚तीत्याश्र‚य‚{\tiny $_{lb}$}‚ इत्याह । \textbf{अप्र‚तीत्याश्र‚योर्थ}प्र‚तीतेरेवाश्र‚यो वा \textbf{क‚थं स‚म्ब‚न्धः} [।] नैव ।‚{\tiny $_{lb}$}‚ प्र‚तीत्य‚न्य‚थानुप‚प‚त्त्या स‚म्ब‚न्ध‚क‚ल्प‚नात् । प्र‚तीत्य‚भावे क‚थं स‚म्ब‚न्धः । अथ [।]
	{\color{gray}{\rmlatinfont\textsuperscript{§~\theparCount}}}
	\pend% ending standard par
      ‚{\tiny $_{lb}$}‚
	  \bigskip
	  \begingroup
	
	    
	    \stanza[\smallbreak]
	  {\normalfontlatin\large ``\qquad}ज्ञाप‚क‚त्वाद्धि स‚म्ब‚न्धः स्वात्म‚ज्ञान‚म‚पेक्ष‚ते ।&‚{\tiny $_{lb}$}‚तेनासौ‚{\tiny $_{५}$}‚ विद्य‚मानोपि नागृहीतः प्र‚काश‚कः ।&‚{\tiny $_{lb}$}‚स‚र्वेषाम‚न‚भिव्य‚क्तानां पूर्व‚पूर्व‚प्र‚योक्त्रृतः [।]&‚{\tiny $_{lb}$}‚सिद्धः स‚म्ब‚न्ध इत्येवं स‚म्ब‚न्धादिर्न विद्य‚त इति ।{\normalfontlatin\large\qquad{}"}\&[\smallbreak]
	  
	  
	  
	  \endgroup
	‚{\tiny $_{lb}$}‚

	  
	  \pstart \leavevmode% starting standard par
	त‚त्राह । \textbf{संकेते}त्यादि । \textbf{संकेते}न \textbf{त}स्य स‚म्ब‚न्ध‚स्या\textbf{भिव्य‚क्तौ । अस‚म‚र्था} निष्फ‚ला‚{\tiny $_{lb}$}‚ संकेता\textbf{द‚न्य}स्य स‚म्ब‚न्ध‚स्य \textbf{क‚ल्प‚ना} ।
	{\color{gray}{\rmlatinfont\textsuperscript{§~\theparCount}}}
	\pend% ending standard par
      ‚{\tiny $_{lb}$}‚

	  
	  \pstart \leavevmode% starting standard par
	एत‚दुक्त‚म्भ‚व‚ति । संकेते स‚त्य‚र्थ‚प्र‚तीतेरुत्प‚त्तिर‚स‚ति चानुत्प‚त्तिरित्य‚न्व‚य‚{\tiny $_{lb}$}‚व्य‚तिरेका‚{\tiny $_{६}$}‚भ्यां संकेत एवार्थ‚प्र‚तीतेः कार‚ण‚मिति किं स‚म्ब‚न्धेनान्येन ।
	{\color{gray}{\rmlatinfont\textsuperscript{§~\theparCount}}}
	\pend% ending standard par
      ‚{\tiny $_{lb}$}‚

	  
	  \pstart \leavevmode% starting standard par
	\textbf{ने}त्यादिना व्याच‚ष्टे । \textbf{अन‚भिव्य‚क्तो} प्र‚काशितोर्थ\textbf{प्र‚तीतिहेतुः । एन‚मिति} स‚म्ब‚न्धं‚{\tiny $_{lb}$}‚ \textbf{स त‚र्ही}ति स‚म्ब‚न्धः । संकेतादेवार्थ‚प्र‚तिपाद‚ने \textbf{सिद्धे} स‚ति \textbf{उप‚स्थातुं शीलं} य‚स्य स‚{\tiny $_{lb}$}‚ ‚{\tiny $_{lb}$}‚ \leavevmode\ledsidenote{\textenglish{411/s}}स‚म्ब‚न्धः \textbf{किम‚कार}णं क‚स्मान्निष्फ‚लं पोष्य‚ते ।
	{\color{gray}{\rmlatinfont\textsuperscript{§~\theparCount}}}
	\pend% ending standard par
      ‚{\tiny $_{lb}$}‚

	  
	  \pstart \leavevmode% starting standard par
	१० नैष्फ‚ल्य‚मेवाह । \textbf{न‚न्वि}त्यादि । \textbf{इयाने}ताव‚न्मात्र एव । \textbf{य‚द‚र्थे} वाच्ये \leavevmode\ledsidenote{\textenglish{148a/PSVTa}}‚{\tiny $_{lb}$}‚ \textbf{प्र‚{\tiny $_{७}$}‚तीते}र्ज्ञान‚स्य \textbf{ज‚न‚नं । त}त्प्र‚तीतिज‚न‚नं \textbf{स‚म‚येन कृत‚मिति} निष्फ‚ल‚स्स‚म्ब‚न्धः ।‚{\tiny $_{lb}$}‚ \textbf{नायोग्ये स‚म‚य‚स्स‚म‚र्थः} । किन्त‚र्हि य एवार्थ‚प्र‚तिप‚त्य‚न्य‚थानुप‚प‚त्या प्र‚तिपाद‚केन‚{\tiny $_{lb}$}‚ स‚म‚र्थः प्र‚तिप‚न्नः श‚ब्द‚स्त‚त्रैव संकेतः स‚म‚र्थ इति \textbf{योग्य‚ता स‚म्ब}न्ध‚श्चेत् । त‚दुक्तं ।
	{\color{gray}{\rmlatinfont\textsuperscript{§~\theparCount}}}
	\pend% ending standard par
      ‚{\tiny $_{lb}$}‚
	  \bigskip
	  \begingroup
	
	    
	    \stanza[\smallbreak]
	  {\normalfontlatin\large ``\qquad}श‚क्तिरेव हि स‚म्ब‚न्धो भेद‚श्चास्या न दृश्य‚ते ।&‚{\tiny $_{lb}$}‚सा हि कार्यानुमेय‚त्वात् त‚द्भेद‚म‚नुव‚र्त्त‚ते इति ॥ \href{http://sarit.indology.info/?cref=\%C5\%9Bv-sambandha.28}{संबं० २८}{\normalfontlatin\large\qquad{}"}\&[\smallbreak]
	  
	  
	  
	  \endgroup
	‚{\tiny $_{lb}$}‚

	  
	  \pstart \leavevmode% starting standard par
	अत्राह । \textbf{त‚त्किमि}त्यादि ।
	{\color{gray}{\rmlatinfont\textsuperscript{§~\theparCount}}}
	\pend% ending standard par
      ‚{\tiny $_{lb}$}‚

	  
	  \pstart \leavevmode% starting standard par
	एव‚म्म‚न्य‚ते । नैका श‚क्तिः श‚ब्दार्थ‚योः स्थिता किन्तु श‚ब्द‚स्था हि वाच‚क‚{\tiny $_{lb}$}‚श‚क्तिर‚न्यार्थे स्थातुं वाच्य‚श‚क्तिर‚न्यैव । य‚दा तु श‚ब्द‚श‚क्तिः स‚म्ब‚न्ध‚स्त‚दा य‚{\tiny $_{lb}$}‚ एव स‚म्ब‚न्ध‚स्स एव स‚म्ब‚न्धी प्राप्त इति पृच्छ‚ति [।] \textbf{त‚त्किम्वै श‚ब्दः स‚म्ब‚न्धोस्तु} ।
	{\color{gray}{\rmlatinfont\textsuperscript{§~\theparCount}}}
	\pend% ending standard par
      ‚{\tiny $_{lb}$}‚

	  
	  \pstart \leavevmode% starting standard par
	न‚न्व‚न्या हि योग्य‚ताऽन्य‚श्च श‚ब्द‚स्त‚त्क‚थं श‚ब्द‚स्स‚म्ब‚न्ध इत्य‚त आह ।
	{\color{gray}{\rmlatinfont\textsuperscript{§~\theparCount}}}
	\pend% ending standard par
      ‚{\tiny $_{lb}$}‚

	  
	  \pstart \leavevmode% starting standard par
	य‚स्मात् \textbf{स‚म‚र्थं हि रूपं श‚ब्द}स्यार्थ‚प्र‚तिपाद‚नं प्र‚ति \textbf{योग्य‚ता । कार्य‚कार‚ण‚यो‚{\tiny $_{lb}$}‚ग्य‚ता‚{\tiny $_{२}$}‚व‚त्} । कार्य‚क‚र‚णाय योग्य‚ता कार्य‚क‚र‚ण‚योग्य‚ता । य‚था कार‚ण‚स्यात्म‚भ‚ता ।‚{\tiny $_{lb}$}‚ त‚द्व‚त् । \textbf{सा चे}दिति योग्य‚ता त‚तः श‚ब्दाद\textbf{र्थान्त‚रं} त‚दा \textbf{किं श‚ब्द‚स्य} भ‚व‚तीति श‚ब्द‚{\tiny $_{lb}$}‚योग्य‚त‚यो\textbf{स्स‚म्ब‚न्धो वाच्यः} । अन्य‚था श‚ब्द‚स्येयं योग्य‚तेति न सिध्येत् । यार्थान्त‚र‚भूता‚{\tiny $_{lb}$}‚ \textbf{योग्य‚ता} त‚स्यां श‚ब्द‚स्यो\textbf{प‚कार इति चेत्} । श‚ब्देन योग्य‚ताया उप‚कारः क्रिय‚त इति‚{\tiny $_{lb}$}‚ याव‚त् । तेन श‚ब्द‚ज‚न्य‚त्वा‚{\tiny $_{३}$}‚च्छ‚ब्द‚स्य योग्य‚तेत्युच्य‚त इति भावः । \textbf{न} श‚ब्देनोप‚कारो‚{\tiny $_{lb}$}‚ योग्य‚तायाः [।] कुतः [।] \textbf{नित्याया} योग्य‚ताया \textbf{निर‚तिश‚य‚त्वात्} । अथ योग्य‚ताया‚{\tiny $_{lb}$}‚ अपि श‚ब्देन व्य‚तिरिक्त एवोप‚कारः क्रिय‚ते । त‚दा त‚स्य श‚ब्द‚कृत‚स्योप‚कार‚स्य‚{\tiny $_{lb}$}‚ योग्य‚त‚या कः स‚म्ब‚न्ध इति वाच्यं । अथं त‚त्राप्युप‚कारे योग्य‚त‚यान्य उप‚कारः क्रिय‚ते‚{\tiny $_{lb}$}‚ त‚दा \textbf{त‚त्राप्यु}प‚कारे य‚थोक्त‚विधिनाऽप‚राप‚र‚स्योप‚का‚{\tiny $_{४}$}‚र‚स्य क‚ल्प‚नाया\textbf{म‚तिप्र‚स‚ङ्गात्} ।‚{\tiny $_{lb}$}‚ त‚तोन‚व‚स्थाया\textbf{मुप‚कारासिद्धेः} सैव योग्य‚त‚या स‚ह श‚ब्द‚स्य स‚म्ब‚न्धासिद्धिः । किञ्च‚{\tiny $_{lb}$}‚ व्य‚तिरिक्तां योग्य‚तामुप‚कुर्वाणः श‚ब्दः स्व‚रूपेणैवोप‚क‚रोति न प‚र‚रूपेण । श‚ब्द‚स्या‚{\tiny $_{lb}$}‚‚{\tiny $_{lb}$}‚ \leavevmode\ledsidenote{\textenglish{412/s}}नुप‚कार‚क‚त्व‚प्र‚संगात् । त‚दा \textbf{योग्य‚तायां च स्व‚तः} स्व‚रूपेण श‚ब्द‚स्य \textbf{योग्य‚त्वेऽर्थ एव}‚{\tiny $_{lb}$}‚ श‚ब्द‚स्व‚भावः प्र‚तीतिज‚न‚न‚योग्यः \textbf{किन्नेष्य‚ते} । किं पा‚{\tiny $_{५}$}‚र‚म्प‚र्येणेति याव‚त् । त‚स्मान्न‚{\tiny $_{lb}$}‚ योग्य‚ता स‚म्ब‚न्धो न च सार्थ‚प्र‚तीतिहेतुः । स‚म‚वायादेवार्थ‚प्र‚तीतेः ।
	{\color{gray}{\rmlatinfont\textsuperscript{§~\theparCount}}}
	\pend% ending standard par
      ‚{\tiny $_{lb}$}‚

	  
	  \pstart \leavevmode% starting standard par
	बौद्ध‚स्यापि \textbf{त‚र्हि स‚म‚यः क‚थं श‚ब्दार्थ‚योस्स‚म्ब‚न्धः} [।] क‚थं च न स्यात्‚{\tiny $_{lb}$}‚ \textbf{पुरुषेषु वृत्तेः} । अस्यार्थ‚स्यायं वाच‚क इत्य‚र्थ‚क‚थ‚नं स म यः । स च पुरुषेषु व‚र्त्त‚ते‚{\tiny $_{lb}$}‚ न श‚ब्दार्थ‚योः । न चान्य‚ध‚र्मोन्य‚स्य ध‚र्मोऽश्व‚ध‚र्म इव गोः ।
	{\color{gray}{\rmlatinfont\textsuperscript{§~\theparCount}}}
	\pend% ending standard par
      ‚{\tiny $_{lb}$}‚

	  
	  \pstart \leavevmode% starting standard par
	१ आ चा र्य स्तु न केव‚लं स‚म‚यो न‚{\tiny $_{६}$}‚ स‚म्ब‚न्धोन्योपि भाविकः स‚म्ब‚न्धो‚{\tiny $_{lb}$}‚ नास्तीत्याह । \textbf{नेत्या}दि । त‚था हि भावानां रूप‚श्लेषो वा स‚म्ब‚न्धः स्यात् ।‚{\tiny $_{lb}$}‚ पार‚त‚न्त्र्यं वा प‚र‚स्प‚रापेक्ष‚णं वा । त\textbf{त्रामिश्राणा}म्प‚र‚स्प‚र‚भिन्नानां न क‚श्चिद्‚{\tiny $_{lb}$}‚ रूप‚श्लेष‚ल‚क्ष‚ण‚स्स‚म्ब‚न्धः । तेषाम‚भेद‚प्र‚स‚ङ्गात् । एक‚त्वाप‚त्तेः । त‚था \textbf{सिद्धानां}‚{\tiny $_{lb}$}‚ निष्प‚न्नानां रूपाणां न \textbf{क‚श्चित्} पार‚त‚न्त्र्य‚ल‚क्ष‚णः \textbf{स‚म्ब‚न्धः} [।] सिद्धे पार‚त‚न्त्र्या‚{\tiny $_{lb}$}‚\leavevmode\ledsidenote{\textenglish{148b/PSVTa}} योगात् ।‚{\tiny $_{७}$}‚ प‚र‚स्प‚रापेक्षाल‚क्ष‚णोपि स‚म्ब‚न्धो नास्तीत्याह । \textbf{अन‚पेक्ष‚णा}च्चेति ।‚{\tiny $_{lb}$}‚ सिद्ध‚स्य स‚र्व‚निर‚पेक्ष‚त्वात् ।
	{\color{gray}{\rmlatinfont\textsuperscript{§~\theparCount}}}
	\pend% ending standard par
      ‚{\tiny $_{lb}$}‚

	  
	  \pstart \leavevmode% starting standard par
	२ क य‚दि त‚र्हि श‚ब्दार्थ‚योर्नास्ति स‚म्ब‚न्धः क‚थ‚न्त‚र्हि श‚ब्दार्थ‚प्र‚तीतिः ।‚{\tiny $_{lb}$}‚ क‚थं च शाब्दं ज्ञान‚म‚नुमानेऽन्त‚र्भाव्य‚ते । स‚म‚यो वा त‚दा किंप्र‚योज‚न इति [।]
	{\color{gray}{\rmlatinfont\textsuperscript{§~\theparCount}}}
	\pend% ending standard par
      ‚{\tiny $_{lb}$}‚

	  
	  \pstart \leavevmode% starting standard par
	आह । \textbf{अर्थ‚विशेषे}त्यादि [।] \textbf{अर्थ‚विशेषो} यः प्र‚तिपाद‚नाभिप्रायेण विष‚यीकृतः ।‚{\tiny $_{lb}$}‚ \textbf{त‚स्य स‚मीहा} प्र‚तिपाद‚नेच्छा । त‚या‚{\tiny $_{१}$}‚ \textbf{प्रेरिता} ज‚निता वाक् सूच‚य‚ति प्र‚काश‚य‚ति‚{\tiny $_{lb}$}‚ \textbf{स्व‚निदान‚प्र‚तिभास‚न‚म‚र्थं} । वाचः स्व‚निदानं प्र‚तिपाद‚नाभिप्रायः । त‚त्प्र‚काशिनं क‚स्य‚{\tiny $_{lb}$}‚ \textbf{सूच‚य}ति । \textbf{अत इद‚मिति} विदुषः । \textbf{अतः} प्र‚तिपाद‚नाभिप्रायात् स‚काशा\textbf{दिद}म्व‚च‚न‚{\tiny $_{lb}$}‚माग‚त‚मिति यो \textbf{विद्वान् त‚स्य} । इति एवं \textbf{बुद्धिरूप}स्याभिप्राय‚ल‚क्ष‚ण‚स्य \textbf{वाग्विज्ञ‚प्तेश्च‚{\tiny $_{lb}$}‚ ज‚न्य‚ज‚न‚क}ल‚क्ष‚ण‚स्स\textbf{म्ब‚न्धः । त‚तः श‚ब्दात् प्र‚{\tiny $_{२}$}‚तिप‚त्तिर‚विनाभावादिति} [।] त‚तो‚{\tiny $_{lb}$}‚ ज‚न्य‚ज‚न‚क‚भावाद् वाक्याद‚र्थ‚प्र‚तिप‚त्तिः ।
	{\color{gray}{\rmlatinfont\textsuperscript{§~\theparCount}}}
	\pend% ending standard par
      ‚{\tiny $_{lb}$}‚

	  
	  \pstart \leavevmode% starting standard par
	तेन य‚दुक्तं [।] प‚दाद‚र्थ‚म‚तिर्य‚द्य‚प्य‚नुमानं वाक्यात्त्व‚र्थ‚प्र‚तीतिः प्र‚माणान्त‚रं‚{\tiny $_{lb}$}‚ ‚{\tiny $_{lb}$}‚ \leavevmode\ledsidenote{\textenglish{413/s}}स‚म्ब‚न्धाग्र‚हात् [।] न चात्राविनाभाव उप‚योगीति [।]
	{\color{gray}{\rmlatinfont\textsuperscript{§~\theparCount}}}
	\pend% ending standard par
      ‚{\tiny $_{lb}$}‚

	  
	  \pstart \leavevmode% starting standard par
	त‚द‚पास्तं । अविनाभाव‚म‚न्त‚रेण वाच्य‚वाच‚क‚भाव‚स्याभावात् ।
	{\color{gray}{\rmlatinfont\textsuperscript{§~\theparCount}}}
	\pend% ending standard par
      ‚{\tiny $_{lb}$}‚

	  
	  \pstart \leavevmode% starting standard par
	य‚द‚प्युक्तं [।] क‚थं श‚ब्दाद‚र्थ‚प्र‚तिप‚त्तिः । क‚थं च शाब्दं ज्ञान‚म‚नुमानेन्त‚र्भ‚व‚{\tiny $_{lb}$}‚तीति । त‚त्प‚रिहृतं‚{\tiny $_{३}$}‚ [।]
	{\color{gray}{\rmlatinfont\textsuperscript{§~\theparCount}}}
	\pend% ending standard par
      ‚{\tiny $_{lb}$}‚

	  
	  \pstart \leavevmode% starting standard par
	ख इदानीं स‚म‚य‚प्र‚योज‚न‚माह । \textbf{त‚दाख्यान}मित्यादि । \textbf{त‚दाख्यान}म‚विना‚{\tiny $_{lb}$}‚भावाख्यानं \textbf{स‚म‚यः} । आख्याय‚तेनेनेति कृत्वा । \textbf{त‚त} इति संम‚यात् \textbf{प्र‚त्याय‚क}स्यार्थ‚{\tiny $_{lb}$}‚प्र‚तिपाद‚नाङ्ग‚स्याविनाभाव‚ल‚क्ष‚ण‚स्य \textbf{सिद्धेः} प्र‚तीतेः । उप‚चारेण स‚म‚य‚स्य \textbf{स‚म्ब‚न्धा‚{\tiny $_{lb}$}‚ख्यानात्} स‚म्ब‚न्ध‚व्य‚प‚देशः । न तु पुनः स एव स‚म‚यो मुख्यः स‚म्ब‚न्धः ।
	{\color{gray}{\rmlatinfont\textsuperscript{§~\theparCount}}}
	\pend% ending standard par
      ‚{\tiny $_{lb}$}‚

	  
	  \pstart \leavevmode% starting standard par
	तेन य‚दुच्य‚ते ।‚{\tiny $_{lb}$}‚ 
	    \pend% close preceding par
	  
	    
	    \stanza[\smallbreak]
	  {\normalfontlatin\large ``\qquad}स‚म‚यः प्र‚ति‚{\tiny $_{४}$}‚म‚र्त्त्य‚म्वा प्र‚त्युच्चार‚ण‚मेव वा ।&‚{\tiny $_{lb}$}‚क्रिय‚ते ज‚ग‚दादौ वा स‚कृदेकेन केन‚चिदिति [।]{\normalfontlatin\large\qquad{}"}\&[\smallbreak]
	  
	  
	  
	    \pstart  \leavevmode% new par for following
	    \hphantom{.}
	   निर‚स्तं ।
	{\color{gray}{\rmlatinfont\textsuperscript{§~\theparCount}}}
	\pend% ending standard par
      ‚{\tiny $_{lb}$}‚

	  
	  \pstart \leavevmode% starting standard par
	ग न‚नु य‚द्य‚विनाभावेन श‚ब्दार्थ‚प्र‚तीतिस्त‚दा श‚ब्द‚स्यार्थ‚प्र‚तिपाद‚नं वाच‚क‚त्वेन‚{\tiny $_{lb}$}‚ न स्यात् । धूम‚स्येवाग्निप्र‚तिपाद‚नं । य‚दि च श‚ब्दार्थ‚योस्स‚म‚येन विना भावाख्यानात्‚{\tiny $_{lb}$}‚ स‚म‚य‚स्स‚म्ब‚न्ध उच्य‚ते । अग्निधूम‚योर‚पि स‚म‚यः स्यादिति ।
	{\color{gray}{\rmlatinfont\textsuperscript{§~\theparCount}}}
	\pend% ending standard par
      ‚{\tiny $_{lb}$}‚

	  
	  \pstart \leavevmode% starting standard par
	नैष दोषो य‚स्मात् ।‚{\tiny $_{५}$}‚ इम‚म‚र्थ‚म‚कृत‚स‚म‚येनापि श‚ब्देन प्र‚तिपाद‚यामीत्येव‚म‚र्थ‚स्य‚{\tiny $_{lb}$}‚ वाच्य‚त्वं श‚ब्द‚स्य च वाच‚क‚त्व‚मारोप्यार्थ‚प्र‚तिपाद‚नाभिप्राये स‚ति य‚दा श‚ब्दं‚{\tiny $_{lb}$}‚ प्र‚युङ्क्ते त‚दा श‚ब्द‚स्य वाच‚क‚रूप‚स्यैवोत्प‚त्तिः ।
	{\color{gray}{\rmlatinfont\textsuperscript{§~\theparCount}}}
	\pend% ending standard par
      ‚{\tiny $_{lb}$}‚

	  
	  \pstart \leavevmode% starting standard par
	घ न‚न्व‚र्थ‚प्र‚तिपाद‚नाभिप्रायेण व‚र्ण्णा एव ज‚न्य‚न्ते न च वाच‚का व‚र्ण्णा‚{\tiny $_{lb}$}‚ इष्य‚न्ते त‚त्क‚थ‚मुच्य‚तेऽभिप्रायाद् वाच‚क‚स्य श‚ब्द‚स्योत्प‚त्तिरिति ।
	{\color{gray}{\rmlatinfont\textsuperscript{§~\theparCount}}}
	\pend% ending standard par
      ‚{\tiny $_{lb}$}‚

	  
	  \pstart \leavevmode% starting standard par
	स‚त्त्यं [।] केव‚ल‚म्व‚क्तृश्रो‚{\tiny $_{६}$}‚त्रोर्व‚र्ण्णेष्वेव वाच‚काभिमानाद् वाच‚क‚स्यैवोत्प‚त्ति‚{\tiny $_{lb}$}‚रुच्य‚ते । त‚स्माद‚स्यैव वाच्य‚वाच‚क‚भाव‚ल‚क्ष‚ण‚स्याविनाभाव‚स्य श‚ब्दार्थ‚स‚म्ब‚न्ध‚स्य‚{\tiny $_{lb}$}‚ मूढं प्र‚ति आख्यानं स‚म‚य‚स्स‚म्ब‚न्ध उच्य‚ते । न तु स‚र्व‚मेव कार्य‚कार‚ण‚भावाख्यानं‚{\tiny $_{lb}$}‚ स‚म‚यः [।] तेन न धूमादौ स‚म्ब‚न्धः स‚म‚य उच्य‚ते । य‚द्य‚पि संकेत‚व्य‚व‚हार‚काल‚योः‚{\tiny $_{lb}$}‚ श‚ब्दार्थ‚स‚म्ब‚न्ध‚स्य भेद‚स्त‚थापि सादृश्यादेक‚त्वाध्य‚{\tiny $_{७}$}‚व‚सायेन लोक‚स्य प्र‚वृत्तिः । \leavevmode\ledsidenote{\textenglish{149a/PSVTa}}‚{\tiny $_{lb}$}‚ अत एव च य‚मेव श‚ब्दार्थ‚स‚म्ब‚न्धं पूर्व‚प्र‚तिप‚न्नं व‚क्ता क‚थ‚य‚ति त‚मेव श्रोता‚{\tiny $_{lb}$}‚ प्र‚तिप‚द्य‚ते । तेन य‚दुच्य‚ते ॥
	{\color{gray}{\rmlatinfont\textsuperscript{§~\theparCount}}}
	\pend% ending standard par
      ‚{\tiny $_{lb}$}‚
	  \bigskip
	  \begingroup
	
	    
	    \stanza[\smallbreak]
	  {\normalfontlatin\large ``\qquad}प्र‚त्येकं स च स‚म्ब‚न्धो भिद्येतैकोऽथ‚वा भ‚वेत् ।&‚{\tiny $_{lb}$}‚एक‚त्वे कृत‚को न स्याद् भिन्न‚श्चेद् भेद‚धीर्भ‚वेत् । \href{http://sarit.indology.info/?cref=\%C5\%9Bv-sambandha.14}{संबंध० १४}&‚{\tiny $_{lb}$}‚व‚क्तृश्रोतृधियो भेदाद् व्य‚व‚हार‚श्च न दुष्य‚ति ।&‚{\tiny $_{lb}$}‚व‚क्तुर‚न्यो हि स‚म्ब‚न्धो बुद्धौ श्रोतुस्त‚थाप‚रः ।&‚{\tiny $_{lb}$}‚‚{\tiny $_{lb}$}‚\leavevmode\ledsidenote{\textenglish{414/s}}श्रोतुः क‚र्तुञ्च स‚म्ब‚न्ध‚म्व‚क्ता कं‚{\tiny $_{१}$}‚ प्र‚तिप‚द्य‚ते ।&‚{\tiny $_{lb}$}‚पूर्व‚दृष्टो हि य‚स्तेन तं श्रोतुर्न क‚रोत्य‚सौ ।&‚{\tiny $_{lb}$}‚यं क‚रोति न‚वं सोपि न दृष्टः प्र‚तिपाद‚क इति [।]{\normalfontlatin\large\qquad{}"}\&[\smallbreak]
	  
	  
	  
	  \endgroup
	‚{\tiny $_{lb}$}‚

	  
	  \pstart \leavevmode% starting standard par
	त‚द‚पास्तं । स‚र्व‚त्र वाच्य‚वाच‚क‚स‚म्ब‚न्धानाम्भिन्नानाम‚प्येक‚त्वाध्य‚व‚सायेन‚{\tiny $_{lb}$}‚ लोक‚स्य प्र‚वृत्तेः [।] न चाप्य‚नादिता तेषां प्र‚त्य‚भिज्ञाया अप्र‚माण‚त्वादिति । \href{http://sarit.indology.info/?cref=pv.3.226}{२२९}
	{\color{gray}{\rmlatinfont\textsuperscript{§~\theparCount}}}
	\pend% ending standard par
      ‚{\tiny $_{lb}$}‚

	  
	  \pstart \leavevmode% starting standard par
	ङ \textbf{अस्तु वा}ऽविनाभाव‚स‚म्ब‚न्धा\textbf{द‚न्य एव नित्यः} श‚ब्दार्थ‚योस्\textbf{स‚म्ब‚न्धः । तेन}‚{\tiny $_{lb}$}‚ स‚म्ब‚न्धेन \textbf{गिरां} श‚ब्दाना\textbf{मे‚{\tiny $_{२}$}‚कार्थ‚निय‚मे} स‚ति पुनः स‚म‚य‚व‚शा\textbf{न्न स्याद‚र्थान्त‚रे} य‚त्रासौ‚{\tiny $_{lb}$}‚ श‚ब्दो न निय‚मित‚स्त‚त्र \textbf{ग‚तिः} प्र‚तिप‚त्तिः । किं कार‚णं । \textbf{न हि तेनै}कार्थ‚निय‚तेन‚{\tiny $_{lb}$}‚ \textbf{स‚म्ब‚न्धेनास‚म्ब‚द्धेर्थे प्र‚तीतिर्युक्ता} । क‚स्मात् [।] \textbf{त‚स्य} स‚म्ब‚न्ध‚स्य \textbf{वैफ‚ल्य‚प्र‚स‚ङ्गात्} ।‚{\tiny $_{lb}$}‚ न भ‚व‚त्येव त‚त्र प्र‚तीतिरिति [।]
	{\color{gray}{\rmlatinfont\textsuperscript{§~\theparCount}}}
	\pend% ending standard par
      ‚{\tiny $_{lb}$}‚

	  
	  \pstart \leavevmode% starting standard par
	आह । \textbf{दृष्ट‚श्चे}त्यादि । पुरुषे\textbf{च्छाव‚शात् कृतः स‚म‚यो}स्येति \textbf{दीप‚कः} प्र‚काश‚कः ।
	{\color{gray}{\rmlatinfont\textsuperscript{§~\theparCount}}}
	\pend% ending standard par
      ‚{\tiny $_{lb}$}‚

	  
	  \pstart \leavevmode% starting standard par
	\textbf{अनेके}त्या‚{\tiny $_{३}$}‚दि । \textbf{अनेकार्थे}न श‚ब्द‚स्या\textbf{भिस‚म्ब‚न्धे}भ्युप‚ग‚म्य‚माने य‚द्येकार्थ‚निय‚मेन‚{\tiny $_{lb}$}‚ स‚म‚य‚कारोभिव्य‚क्तिं क‚रोति । त‚दाभिम‚त एवार्थे क‚रोति न त्व‚न्य‚स्मिन्नेव विरुद्धे‚{\tiny $_{lb}$}‚ इति नास्ति निय‚म‚स्त‚तो \textbf{विरुद्ध‚व्य‚क्तिस‚म्भ‚वः} । अभीष्टाद् विरुद्ध‚स्याप्य‚र्थ‚स्याभि‚{\tiny $_{lb}$}‚व्य‚क्तिः स‚म्भाव्य‚ते ।
	{\color{gray}{\rmlatinfont\textsuperscript{§~\theparCount}}}
	\pend% ending standard par
      ‚{\tiny $_{lb}$}‚

	  
	  \pstart \leavevmode% starting standard par
	\textbf{अथे}त्यादिना व्याच‚ष्टे । एकार्थ‚प्र‚तिनिय‚मे स‚म‚य‚व‚शाद‚नेकार्थ‚प्र‚तिपा‚{\tiny $_{४}$}‚द‚नं दृष्टं‚{\tiny $_{lb}$}‚ विरुध्य‚ते [।] त‚स्मात्त‚स्य दृष्ट‚स्य विरोधो \textbf{मा भूदि}ति \textbf{स‚र्वे} श‚ब्दाः \textbf{स‚र्व‚स्या}र्थ‚स्य‚{\tiny $_{lb}$}‚ \textbf{वाच‚काः} ।
	{\color{gray}{\rmlatinfont\textsuperscript{§~\theparCount}}}
	\pend% ending standard par
      ‚{\tiny $_{lb}$}‚

	  
	  \pstart \leavevmode% starting standard par
	च \textbf{त‚थे}त्यादिना विरुद्ध‚व्य‚क्तिप्र‚तिपाद‚नं । य‚था स‚र्वे स‚र्व‚स्य वाच‚का इष्य‚न्ते‚{\tiny $_{lb}$}‚ \textbf{त‚थे}द‚म‚प्य‚व‚श्य‚मेष्ट‚व्यं [।] \textbf{न} च \textbf{स‚र्वा}र्थः \textbf{स‚र्व}स्य कार्य‚स्य \textbf{साध‚क} इति न्याय‚{\tiny $_{lb}$}‚प्राप्त‚त्वात् । न्यायं चाह । \textbf{संक‚रा}दिति । प्र‚तिनिय‚त‚त्वात् \textbf{कार्य‚कार‚ण‚तायाः ।‚{\tiny $_{lb}$}‚ त‚त्रै}त‚स्मिन् न्या‚{\tiny $_{५}$}‚ये स‚ति \textbf{प्र‚तिनिय}त‚म्विशिष्ट‚मेव व‚स्तु साध‚नं कार‚णं य‚स्या\textbf{भिम‚त}‚{\tiny $_{lb}$}‚स्यार्थ‚स्य \textbf{साध्य‚स्य} स्व‚र्गादेः । \textbf{त‚स्मिन्} विष‚य‚भूते । श‚ब्द‚स्य किम्विशिष्ट‚स्य [।]‚{\tiny $_{lb}$}‚ ‚{\tiny $_{lb}$}‚ \leavevmode\ledsidenote{\textenglish{415/s}}स‚र्वेत्यादि । \textbf{स‚र्वेषां साध्यानां} कार्याणां य‚थास्वं यानि साध‚नानि कार‚णानि तेषां‚{\tiny $_{lb}$}‚ वाच‚क‚त्वेन \textbf{साधार‚ण}स्य त‚स्य \textbf{श‚ब्द‚स्येष्ट‚व्य‚क्तिमेव} विशिष्ट‚साध्य‚साध्य‚क‚त्वेना‚{\tiny $_{lb}$}‚भिम‚त एवार्थेभिव्य‚क्तिं \textbf{स‚म‚य‚{\tiny $_{६}$}‚कारः क‚रोतीति कुत एत‚त्} ।
	{\color{gray}{\rmlatinfont\textsuperscript{§~\theparCount}}}
	\pend% ending standard par
      ‚{\tiny $_{lb}$}‚

	  
	  \pstart \leavevmode% starting standard par
	एत‚दुक्त‚म्भ‚व‚ति । य एवार्थो व‚स्तुस्थित्या स्व‚र्ग‚साध‚नः किन्त‚त्रैव स‚म‚य‚कारे‚{\tiny $_{lb}$}‚णाग्निहोत्रादिश‚ब्दोभिव्य‚क्तः किम्वान्य‚स्मिन्नेव स्व‚र्ग‚साध‚न‚विरुद्धेर्थे बुद्धिमान्द्या‚{\tiny $_{lb}$}‚दिति स‚न्देह एव ।
	{\color{gray}{\rmlatinfont\textsuperscript{§~\theparCount}}}
	\pend% ending standard par
      ‚{\tiny $_{lb}$}‚

	  
	  \pstart \leavevmode% starting standard par
	\textbf{स} इति श‚ब्दः स‚र्व‚स्मिन् वाच‚क‚त्वेना\textbf{निय‚तो निय‚मं} क्व‚चिद‚र्थे \textbf{पुरुषात्} पुरुष‚{\tiny $_{lb}$}‚संकेतात् \textbf{प्र‚तिप‚द्य‚ते} । स च पुरुषोऽ‚{\tiny $_{७}$}‚विरुद्धेप्य‚र्थे संकेतं कुर्यात् । त‚था च न केव‚लं \leavevmode\ledsidenote{\textenglish{149b/PSVTa}}‚{\tiny $_{lb}$}‚ विरुद्ध‚व्य‚क्तिस‚म्भ‚वो यापीय‚म\textbf{पौरुषेय‚ता} वेद‚स्येष्टा त‚स्या \textbf{व्य‚र्था स्यात् प‚रि‚{\tiny $_{lb}$}‚क‚ल्प‚ना । अपि नामे}ति । क‚थ‚न्नाम । \textbf{असंकीर्ण्ण‚म}निष्टेनासंसृष्टं । इष्ट\textbf{मेवार्थ‚{\tiny $_{lb}$}‚म}पौरुषेयेभ्यः श‚ब्देभ्यो \textbf{जानीयामिति} कृत्वा संक‚र‚स्येष्टानिष्ट‚व्य‚तिक‚र‚स्य हेतुः‚{\tiny $_{lb}$}‚ \textbf{पुरुषोपाकीर्ण्णो} व‚हिस्कृ\edtext{}{\lemma{हिस्कृ}\Bfootnote{? ष्कृ}}तो वैदिकेभ्यः श‚ब्देभ्यः । त‚{\tiny $_{१}$}‚त्रैव‚म‚व‚स्थिते‚{\tiny $_{lb}$}‚ \textbf{यादृशाः} श‚ब्दाः पौरुषेयाभिम‚ताः \textbf{पुरुषैः क्व‚चिद्} विव‚क्षितेर्थे \textbf{प्र‚युक्तास्संकीर्य‚न्तेऽ}‚{\tiny $_{lb}$}‚निष्टाभिधाय‚क‚त्व‚स‚म्भाव‚न‚या । \textbf{तादृशा} एवापौरुषेयाभिम‚ता \textbf{अपि} श‚ब्दाः‚{\tiny $_{lb}$}‚ \textbf{स‚र्वार्थ‚साधार‚णास्स‚न्तः क्व‚चिद‚र्थे तैः} पुरुषैः स‚म‚येन य‚थेष्टं \textbf{विनिय‚मिताः} । किं का‚{\tiny $_{lb}$}‚र‚णं [।] \textbf{तेषां} पुंसा\textbf{न्त‚त्त्वाप‚रिज्ञानात् । प्र‚कृत्यैव} स्व‚भावेनैव \textbf{वैदि‚{\tiny $_{२}$}‚काः} श‚ब्दा \textbf{निय‚ता}‚{\tiny $_{lb}$}‚ अभिम‚तेर्थे [।] त‚तो न पुरुष‚संस्कार‚कृतो दोष \textbf{इति चेत्} । एवं स‚त्य‚र्थ\textbf{प्र‚काश‚ने}‚{\tiny $_{lb}$}‚ नोप‚देश‚म‚पेक्षेर‚न् । अपेक्ष‚न्ते च । स्व‚त‚स्तेभ्योर्थ‚प्र‚तीतेर‚भावात् । य‚दि च ते‚{\tiny $_{lb}$}‚ स्व‚भाव‚त एव प्र‚तिनिय‚ताः स्युस्त‚दा य‚त्र क्व‚चिद‚र्थे एक‚दा स‚मिताः पुनः क‚थंचित्‚{\tiny $_{lb}$}‚ त‚तो\textbf{न्य‚थासंकेतेना}र्थान्त‚रं \textbf{न प्र‚काश‚येयुः} । प्र‚काश‚य‚न्ति च । त‚तो न प्र‚कृ‚{\tiny $_{३}$}‚त्यै‚{\tiny $_{lb}$}‚  ‚{\tiny $_{lb}$}‚ ‚{\tiny $_{lb}$}‚ \leavevmode\ledsidenote{\textenglish{416/s}}कार्थ‚निय‚ता इति । स्व‚भाव‚त‚श्चैकार्थ‚निय‚मे । योय‚म्वैदिकेषु वाक्येषु व्याख्यातॄणां‚{\tiny $_{lb}$}‚ \textbf{व्याख्याविक‚ल्प‚श्चा}प‚राप‚र‚व्याख्याभेद‚श्च \textbf{न स्यात्} । एकार्थ‚प्र‚तिनिय‚मात् । भ‚व‚ति‚{\tiny $_{lb}$}‚ च । त‚स्मात् पौरुषेय‚वाक्य‚व‚न्नैकार्थ‚निय‚ता वैदिकाः श‚ब्दा इति ।
	{\color{gray}{\rmlatinfont\textsuperscript{§~\theparCount}}}
	\pend% ending standard par
      ‚{\tiny $_{lb}$}‚

	  
	  \pstart \leavevmode% starting standard par
	छ अथ स्यात् [।] निय‚त एवार्थे तेषामुप‚देश इत्य‚त आह । \textbf{उप‚देश}स्येत्यादि ।‚{\tiny $_{lb}$}‚ व्याख्याभेदेन श‚क्योर्थ‚ना‚{\tiny $_{४}$}‚नात्व‚विक‚ल्पो य‚स्मिन् वैदिके वाक्ये त‚त्त‚थोक्त‚न्त‚स्मिन्‚{\tiny $_{lb}$}‚ \textbf{श‚क्य‚विक‚ल्पे} वैदिके वाक्ये । व्याख्यातॄणां य \textbf{उप‚देश‚स्त‚स्येष्ट‚स‚म्वादो} नास्त्य‚य‚म‚पि‚{\tiny $_{lb}$}‚ क‚दाचित् स्याद् अस्यार्थोय‚मन्यो वेति निय‚माभावात् । \textbf{इति} हेतो\textbf{व्य‚र्थैवापौ‚{\tiny $_{lb}$}‚रुषेय‚ता} । ताम‚पि क‚ल्प‚यित्वा व्य‚भिचाराशंकाया अनिवृत्तेः ।
	{\color{gray}{\rmlatinfont\textsuperscript{§~\theparCount}}}
	\pend% ending standard par
      ‚{\tiny $_{lb}$}‚

	  
	  \pstart \leavevmode% starting standard par
	ज य‚श्च श‚ब्दार्थ‚योः स‚म्ब‚न्ध‚मिच्छ‚ति तेन \textbf{वाच्य‚श्च हेतुः स‚म्ब‚न्ध‚स्य व्य‚व‚{\tiny $_{lb}$}‚स्थि‚{\tiny $_{५}$}‚तेः} । स‚म्ब‚न्ध‚व्य‚व‚स्थायाः । केषाम् [।] \textbf{भिन्नानां} प‚र‚स्प‚र‚भिन्नानां श‚ब्दा‚{\tiny $_{lb}$}‚र्थानां । १ न ताव‚च्छ‚ब्दार्थ‚योस्तादात्म्य‚ल‚क्ष‚णः स‚म्ब‚न्धः । य‚स्माद‚र्था हि‚{\tiny $_{lb}$}‚ \textbf{बाह्या} घ‚ट‚प‚टाद‚यः \textbf{श‚ब्द‚स्य न रूपं} न स्व‚भावः । श‚ब्द‚रूप‚त्वे हि घ‚टादीनाम‚{\tiny $_{lb}$}‚भावः स्यात् । \textbf{नापि श‚व्दोर्थानां} रूप‚म‚र्थ‚रूप‚त्वे हि श‚ब्द‚रूप‚ताहानिप्र‚स‚ङ्गात् ।‚{\tiny $_{lb}$}‚ \textbf{येनाभिन्नात्म‚त‚यान्त‚रीय‚क‚ता स्यात् । अ‚{\tiny $_{६}$}‚विनाभाविता} स्यात् । \textbf{व्य‚व‚स्थाभेदे}पीति‚{\tiny $_{lb}$}‚ व्यावृत्तिभेद‚स‚माश्र‚येण साध्य‚साध‚न‚भेदेपि । किमिव । \textbf{कृत‚क‚त्वाऽनित्य‚त्व‚व‚त्} । य‚था‚{\tiny $_{lb}$}‚ कृत‚क‚त्वानित्य‚त्व‚योर्व्यावृत्तिभेदेप्येकात्म‚त‚या नान्त‚रीय‚क‚ता । त‚द्व‚त् ।
	{\color{gray}{\rmlatinfont\textsuperscript{§~\theparCount}}}
	\pend% ending standard par
      ‚{\tiny $_{lb}$}‚

	  
	  \pstart \leavevmode% starting standard par
	२ त‚दुत्प‚त्तिल‚क्ष‚णोपि स‚म्ब‚न्धो नास्तीत्याह । \textbf{नापी}त्यादि । एत‚दाह ।‚{\tiny $_{lb}$}‚ \leavevmode\ledsidenote{\textenglish{150a/PSVTa}} त‚दुत्प‚त्तिल‚क्ष‚णो हि \textbf{श‚ब्दानां} स‚म्ब‚न्धो भ‚व‚न् विव‚क्षाप्र‚तिष्ठिते‚{\tiny $_{७}$}‚न चार्थेन स्याद्‚{\tiny $_{lb}$}‚ बाह्येन वा ।
	{\color{gray}{\rmlatinfont\textsuperscript{§~\theparCount}}}
	\pend% ending standard par
      ‚{\tiny $_{lb}$}‚

	  
	  \pstart \leavevmode% starting standard par
	क न ताव‚दाद्यः प‚क्षः । य‚स्मान्नाप्येते वैदिका ध्व‚न‚यो \textbf{विव‚क्षा‚{\tiny $_{lb}$}‚ज‚न्मान} इष्य‚न्ते । विव‚क्षातो ज‚न्म येषामिति विग्र‚हः । नित्य‚त्वांभ्युप‚ग‚मात् । \textbf{अज‚{\tiny $_{lb}$}‚न्मानो वा} अनुत्प‚न्ना वा स‚न्तो नापि \textbf{विव‚क्षाव्य‚ङ्ग्याः} । नित्य‚त्व‚हानिप्र‚संगात् ।‚{\tiny $_{lb}$}‚ व्य‚ङ्ग्यानामुत्पाद्य‚त्वादित्युक्तं प्राक् ।
	{\color{gray}{\rmlatinfont\textsuperscript{§~\theparCount}}}
	\pend% ending standard par
      ‚{\tiny $_{lb}$}‚

	  
	  \pstart \leavevmode% starting standard par
	ख नापि द्वितीयः प‚क्ष इत्याह । \textbf{नार्थाय‚त्ता} इति । नापि बाह्यार्थाय‚त्ताः ।‚{\tiny $_{lb}$}‚ नित्य‚त्वादेव ।
	{\color{gray}{\rmlatinfont\textsuperscript{§~\theparCount}}}
	\pend% ending standard par
      ‚{\tiny $_{lb}$}‚

	  
	  \pstart \leavevmode% starting standard par
	अन्ये त्वेक‚मेव ग्र‚न्थं कृत्वा व्याच‚क्ष‚ते । य‚स्मान्न विव‚क्षाज‚न्मानो नापि‚{\tiny $_{lb}$}‚ ‚{\tiny $_{lb}$}‚ \leavevmode\ledsidenote{\textenglish{417/s}}त‚द्व्य‚ङ्ग्याः । त‚स्मादुभ‚य‚थापि नार्थाय‚त्ता न बाह्ये व‚स्तुनि प्र‚तिब‚द्धा वैदिकाः‚{\tiny $_{lb}$}‚ श‚ब्दाः । य‚त‚श्च नार्थाय‚त्तास्त‚तः त‚स्मात् । \textbf{इदानी}मिति स‚म्ब‚न्धाभावे । \textbf{त‚द‚न्व‚यं‚{\tiny $_{lb}$}‚ त‚स्यार्थ‚स्यान्व‚यं} स‚द्भावं ते श‚ब्दाः क‚थं साध‚येयुः । नैव साध‚येयुः । किम्भूतं [।]‚{\tiny $_{lb}$}‚ त‚त्प्र‚तिनिय‚म‚संसाध्यं । त‚स्मिन् बाह्येर्थे तादात्म्य‚{\tiny $_{२}$}‚त‚दुत्प‚त्तिभ्यां यः प्र‚तिनिय‚मः‚{\tiny $_{lb}$}‚ श‚ब्दानां तेन संसाध्यं । बाह्येर्थेऽप्र‚तिब‚न्धेन निय‚माभावात् । \href{http://sarit.indology.info/?cref=pv.3.229}{२३२}
	{\color{gray}{\rmlatinfont\textsuperscript{§~\theparCount}}}
	\pend% ending standard par
      ‚{\tiny $_{lb}$}‚

	  
	  \pstart \leavevmode% starting standard par
	३. उक्त‚मेवार्थ श्लोकेन संगृह्णाति । \textbf{असंस्कार्य‚त‚ये}त्यादि । १ \textbf{स‚र्व‚{\tiny $_{lb}$}‚येति} । य‚दि पुरुषैर‚र्थाभिप्रायेण श‚ब्दा न क्रिय‚न्ते नापि संकेत्य‚न्ते । त‚दा पुम्भि‚{\tiny $_{lb}$}‚र‚संस्कार्य‚त‚या हेतुभूत‚या वैदिकानां श‚ब्दानान्निर‚र्थ‚ता स्यात् । य‚स्मात् पुरुष‚सं‚{\tiny $_{lb}$}‚स्कार‚प्र‚ब‚द्धे श‚ब्दानां स‚त्त्यार्थ‚त्व‚मिथ्यार्थ‚त्वे‚{\tiny $_{३}$}‚ । तेन त‚द‚भावान्निर‚र्थ‚तैव स्यात् ।‚{\tiny $_{lb}$}‚ २ निर‚र्थ‚ताप‚रिहारार्थ पुनः पुरुष‚संस्कारोप‚ग‚मे क्रिय‚माणे मिथ्यार्थ‚तापि स्या‚{\tiny $_{lb}$}‚दिति मुख्यं ग ज स्ना न मिद‚म्भ‚वेत् । ग‚जो हि प‚ङ्काप‚न‚य‚नाय स्नात्वा पुनः प‚ङ्के‚{\tiny $_{lb}$}‚नात्मान‚म‚व‚किर‚तीति । नित्योऽनित्यो वा स्यादिति व‚स्तुनो ग‚त्य‚न्त‚राभावात् ।‚{\tiny $_{lb}$}‚ ३ \textbf{पुरुषेच्छावृत्तिः} पुरुषेच्छाव‚शादुत्प‚न्नः स्यादाकुञ्च‚नादिव‚त् । अवृत्तिर्वा ।‚{\tiny $_{lb}$}‚ नास्य पुरु‚{\tiny $_{४}$}‚षेच्छ‚या वृत्तिः । अङ्कुरादिव‚त् । त‚त्रानित्या पुरुषाधीन‚त्व‚प‚क्षे देशा‚{\tiny $_{lb}$}‚दिप‚रावृत्त्या । देशादीनाम‚न्य‚थात्वेन आदिश‚ब्दात् कालाव‚स्थाग्र‚ह‚णं । \textbf{पुरुषाणां‚{\tiny $_{lb}$}‚ य‚थाभिप्राय‚मिति} येन पुरुषेण य‚था प्र‚तिपाद‚यितुमिष्ट‚न्त‚था तेन श‚ब्देन‚{\tiny $_{lb}$}‚ प्र‚तिपाद‚न‚न्न स्यादिति स‚म्ब‚न्धः । किं कार‚णं [।] पुरुष‚प्र‚तिपाद‚नेच्छायां स‚त्या‚{\tiny $_{lb}$}‚\textbf{म‚प्य‚नाय‚त्त‚स्य} श‚ब्द‚स्य \textbf{क‚दाचिद‚योगात्} पुरुषेण नियो‚{\tiny $_{५}$}‚क्तुङ् क‚दाचिद‚प्य‚श‚क्य‚{\tiny $_{lb}$}‚त्वादित्य‚र्थः । \textbf{प‚र्व‚तादिव‚त्} । य‚था प‚र्व‚ताद‚यः पुरुषानाय‚त्ताः स‚त्याम‚पीच्छायां‚{\tiny $_{lb}$}‚ न य‚थेष्टं नियोक्तुम्पार्य‚न्ते [।] त‚द्व‚त् । दृष्ट‚श्च देशादिप‚रावृत्या य‚थाभिप्रायं‚{\tiny $_{lb}$}‚ प्र‚तिपाद‚नं । त‚स्मात् पुरुषेष्व‚नाय‚त्तः स‚म्ब‚न्ध इति ।
	{\color{gray}{\rmlatinfont\textsuperscript{§~\theparCount}}}
	\pend% ending standard par
      ‚{\tiny $_{lb}$}‚‚{\tiny $_{lb}$}‚\textsuperscript{\textenglish{418/s}}

	  
	  \pstart \leavevmode% starting standard par
	अनित्य‚स्य स‚म्ब‚न्ध‚स्यापुरुषाधीन‚त्वे योय‚म‚न‚न्त‚रोक्तो दोषोऽ\textbf{य‚मेव} स‚म्ब‚न्ध‚स्या‚{\tiny $_{lb}$}‚\textbf{नित्य‚त्वेपि दोषः} । किं कार‚णं [।] \textbf{त‚स्य} स‚म्ब‚न्ध‚स्य \textbf{स्थिर‚स्व‚भा}व‚स्य देशादि‚{\tiny $_{lb}$}‚प‚रावृत्त्या प‚रावृत्त्यायोगा\textbf{द‚न्य‚था} त्व‚स्या\textbf{योगात्} । आकाश‚व‚त् ।
	{\color{gray}{\rmlatinfont\textsuperscript{§~\theparCount}}}
	\pend% ending standard par
      ‚{\tiny $_{lb}$}‚

	  
	  \pstart \leavevmode% starting standard par
	४. क अथ स‚र्वेष्वेवार्थेषु स‚म्ब‚द्धः श‚ब्दः [।] त‚त्राप्याह । \textbf{स‚म‚मि}त्यादि ।‚{\tiny $_{lb}$}‚ \textbf{स‚म‚मेक}कालं \textbf{स‚र्व‚स्मि}न्न‚र्थे स‚म्ब‚न्ध‚स्या\textbf{व‚स्थानेपि} क‚ल्प्य‚मान \textbf{इष्टो} योर्थ‚स्त‚स्मिन्‚{\tiny $_{lb}$}‚ \textbf{प्र‚तिनियामाभावात् [।] त‚तः} स‚र्वार्थ‚साधार‚णाद‚र्थ\textbf{विशेष}स्याभिम‚त‚स्य \textbf{प्र‚तीतिर्न‚{\tiny $_{lb}$}‚\leavevmode\ledsidenote{\textenglish{150b/PSVTa}} स्यादिति} कृत्वा‚{\tiny $_{७}$}‚ अनेकार्थ‚स‚म्ब‚द्धोपि श‚ब्दः पुरुष‚संस्काराद् इष्टार्थ‚निय‚मं प्र‚तिप‚द्य‚त‚{\tiny $_{lb}$}‚ इत्य‚ङ्गीक‚र्त्त‚व्य‚न्त‚त्र च \textbf{पूर्व‚व‚त् प्र‚संगः} । अनेकार्थाभिस‚म्ब‚न्धे विरुद्ध‚व्य‚क्तिस‚म्भ‚वे‚{\tiny $_{lb}$}‚ \href{http://sarit.indology.info/?cref=pv.3.228}{१ । २३१} इत्यादिना य उक्तः । एव‚न्ताव‚न्नित्य‚त्वे स‚म्ब‚न्ध‚स्यानित्य‚त्वेप्य‚{\tiny $_{lb}$}‚पुरुषाधीन‚त्वे दोष उक्तः । \textbf{पुरुषेच्छावृत्तौ च} स‚म्ब‚न्ध‚स्य \textbf{पौरुषेय‚त्व‚मिति} कृत्वा‚{\tiny $_{lb}$}‚ \textbf{विप्र‚ल‚म्भ‚न‚श‚ङ्का} । विस‚म्वाद‚हेतोः पुरुष‚स्याभ्युप‚ग‚मात् ।
	{\color{gray}{\rmlatinfont\textsuperscript{§~\theparCount}}}
	\pend% ending standard par
      ‚{\tiny $_{lb}$}‚

	  
	  \pstart \leavevmode% starting standard par
	ख \textbf{अपि चे}त्यादि‚{\tiny $_{१}$}‚ त‚त्रैव दूष‚णान्त‚र‚माह । द्विविधो हि श‚ब्दानां विष‚यः‚{\tiny $_{lb}$}‚ साक्षाज्जातिस्त‚ल्ल‚क्षिता च व्य‚क्तिरिति । १ त‚त्र य‚दि व्य‚क्त्या स‚ह स‚म्ब‚न्ध‚{\tiny $_{lb}$}‚स्त‚त्राह । \textbf{स‚म्ब‚न्धिनां} वाच्यानाम‚र्थानाम\textbf{नित्य‚त्वात्} । तेषु विन‚श्य‚त्सु स‚म्ब‚न्ध‚{\tiny $_{lb}$}‚स्यापि त‚दाश्रित‚स्य विनाश इति \textbf{न स‚म्ब‚न्धेस्ति नित्य‚ता । प‚राश्र‚य} इति प‚र‚स्स‚म्ब‚न्धी‚{\tiny $_{lb}$}‚ आश्र‚योस्येति कृत्वा । \textbf{स‚म्ब‚न्धिनि} स‚म्ब‚न्ध‚स्याप्र‚तिब‚न्धे स‚ति \textbf{त‚योः} स‚म्ब‚न्धि‚{\tiny $_{lb}$}‚नो‚{\tiny $_{२}$}‚\textbf{स्स‚म्ब‚न्धिताया अयोगात्} । न ह्य‚प्र‚तिब‚द्धेन केन‚चित् क‚श्चित् त‚द्वान् भ‚व‚ति‚{\tiny $_{lb}$}‚ गौरिवाश्वेन । स \textbf{चाश्र‚यः} श‚ब्दार्थ‚स‚म्ब‚न्ध‚स्यानित्योर्थानाम‚नित्य‚त्वात् । \textbf{अपाये}‚{\tiny $_{lb}$}‚ विना\textbf{शेस्या}श्र‚य‚स्य \textbf{स‚म्ब‚न्ध‚स्याप्या}श्रित‚स्या\textbf{पायः} । प्र‚दीपापाये प्र‚भाया इव‚{\tiny $_{lb}$}‚ त‚दाश्रितायाः । \textbf{अन्य‚था} य‚द्याश्र‚यापायेप्याश्रितो \textbf{नापैती}तीष्य‚ते त‚दाश्र‚याभिम‚ते‚{\tiny $_{lb}$}‚ स‚म्ब‚न्धो नाश्रितः स्यात्‚{\tiny $_{३}$}‚ । त‚त् आश्र‚य‚विनाशेविनाशान्न नित्यः स‚म्ब‚न्धः ।‚{\tiny $_{lb}$}‚ ‚{\tiny $_{lb}$}‚ \leavevmode\ledsidenote{\textenglish{419/s}}\textbf{त‚दाश्र‚यार्थ‚श्च व‚क्त‚व्यः} । त‚स्य स‚म्ब‚न्ध‚स्य स‚म्ब‚न्धिनौ केनार्थेनाश्र‚यादित्याश्र‚{\tiny $_{lb}$}‚यार्थो वाच्यः । उप‚कारार्थो ह्याश्र‚यार्थः । स चेह नास्ति [।] किं कार‚णं [।] नि‚{\tiny $_{lb}$}‚त्य‚स्य स‚म्ब‚न्ध‚स्या\textbf{नुप‚कार्य‚त्वा}द‚नाधेयातिश‚य‚त्वा\textbf{द‚नुप‚कुर्वाण‚श्चा}श्र‚याभिम‚तो \textbf{नाश्र‚यः‚{\tiny $_{lb}$}‚ स्यात् ।}
	{\color{gray}{\rmlatinfont\textsuperscript{§~\theparCount}}}
	\pend% ending standard par
      ‚{\tiny $_{lb}$}‚

	  
	  \pstart \leavevmode% starting standard par
	२ स्यादेत‚द् [।] \textbf{य‚दि} व्य‚क्तिर्वाच्या त‚दा स्याद‚न‚न्त‚रोक्तो दोषः । या‚{\tiny $_{४}$}‚‚{\tiny $_{lb}$}‚व‚ता नित्याया \textbf{जातेर्वाच्य‚त्वाद‚दोषः} स‚म्ब‚न्धिनाम‚पायेन स‚म्ब‚न्ध‚स्यानित्य‚तादोषो‚{\tiny $_{lb}$}‚ नास्ती\textbf{ति चेत्} ।
	{\color{gray}{\rmlatinfont\textsuperscript{§~\theparCount}}}
	\pend% ending standard par
      ‚{\tiny $_{lb}$}‚

	  
	  \pstart \leavevmode% starting standard par
	\textbf{ने}त्यादिना प्र‚तिव‚च‚नं । \textbf{त‚द्व‚च‚ने} जातिव‚च‚ने \textbf{प्र‚योज‚नाभावादिति निर्लोठित‚{\tiny $_{lb}$}‚मेत‚द}न्यापो ह चि न्ता याम् \href{http://sarit.indology.info/?cref=}{३ । ५५}
	{\color{gray}{\rmlatinfont\textsuperscript{§~\theparCount}}}
	\pend% ending standard par
      ‚{\tiny $_{lb}$}‚

	  
	  \pstart \leavevmode% starting standard par
	अपि प्र‚व‚र्त्त‚त पुमान् विज्ञायार्थ‚क्रियाक्ष‚मान् [।] इत्य‚त्रान्त‚रे । \textbf{स‚र्व‚त्र‚{\tiny $_{lb}$}‚ चा}भिधाने \textbf{जातेर‚स‚म्भ‚वात्} कार‚णात् । जातिचोद‚नाया \textbf{अयोगः} । य‚था \textbf{यादृच्छिकेषु‚{\tiny $_{lb}$}‚ व्य‚क्तिवाचिषु}‚{\tiny $_{५}$}‚ बाह्यं निमित्त‚म‚न्त‚रेण श‚ब्द‚प्र‚योगेच्छा य‚दृच्छा । त‚स्यां भ‚वा‚{\tiny $_{lb}$}‚ यादृच्छिकाः । तेषु देव‚द‚त्तादिश‚ब्देषु व्य‚क्तिवाचिषु । \href{http://sarit.indology.info/?cref=pv.3.231}{२३४}
	{\color{gray}{\rmlatinfont\textsuperscript{§~\theparCount}}}
	\pend% ending standard par
      ‚{\tiny $_{lb}$}‚

	  
	  \pstart \leavevmode% starting standard par
	५. च‚तुष्ट‚यी श‚ब्दानां प्र‚वृत्तिरिति केषांचिद्द‚र्श‚नं । जातिश‚ब्दा गुण‚श‚ब्दाः‚{\tiny $_{lb}$}‚ क्रियाश‚ब्दा य‚दृच्छाश‚ब्दा इति । तेषां म‚तेनैत‚दुक्तं । अथ देव‚द‚त्तादिश‚ब्दोप्य‚व‚{\tiny $_{lb}$}‚स्थाभेदेन जातिवाच‚क इष्य‚ते । त‚दा \textbf{स‚र्व‚दा जातिचो}द‚नेभ्युप‚ग‚म्य‚माने \textbf{विशे‚{\tiny $_{lb}$}‚षान्त‚{\tiny $_{६}$}‚र‚व्युदासेन} व्य‚क्त्य‚न्त‚र‚प‚रित्यागेन क्व‚चिद‚भिम‚ते व्य‚क्त्य‚न्त‚रं \textbf{प्र‚वृत्य‚योगाच्च}‚{\tiny $_{lb}$}‚ न जात्य‚भिधानं । दृष्टा च गोस्वामिना गामान‚येत्युक्तेऽन्य‚स्वामिक‚गोव्युदासेन‚{\tiny $_{lb}$}‚ विनिय‚ता एव गोरान‚य‚नार्थ‚म्प्र‚वृत्तिः सा चैवं स्यात् । य‚दि प्र‚क‚र‚णादिना गोश‚ब्दो‚{\tiny $_{lb}$}‚ विशेष‚वृत्तिः स्यात् । न च गोर्थ‚स्यान‚य‚न‚म‚स्तीति वाक्यार्थ‚प्र‚तीतिर‚पि न स्यात् ।‚{\tiny $_{lb}$}‚ त‚स्मान्न स‚र्व‚त्र जा‚{\tiny $_{७}$}‚तिश्चोद्य‚ते । य‚त एवं वास्त‚वः स‚म्ब‚न्धो न संग‚च्छ‚ते प्र‚कृति- \leavevmode\ledsidenote{\textenglish{151a/PSVTa}}‚{\tiny $_{lb}$}‚ भिन्नानां । \textbf{त‚स्माद‚न्व‚य‚व्य‚तिरेकिण} इति भावाभाव‚व‚तो भाव‚स्य कार्योभिम‚त‚स्य [।]‚{\tiny $_{lb}$}‚ कार‚णाभिम‚त‚भावाभाव‚द्वारेण यौ \textbf{भावाभावौ} स एव \textbf{स‚म्ब‚न्धः} ज‚न्य‚ज‚न‚क‚भाव‚{\tiny $_{lb}$}‚ ‚{\tiny $_{lb}$}‚ \leavevmode\ledsidenote{\textenglish{420/s}}एवा\edtext{}{\edlabel{pvsvt_420-1}\label{pvsvt_420-1}\lemma{एवा}\Bfootnote{In the margin. }} भिन्नानां स‚म्ब‚न्ध इति याव‚त् ।
	{\color{gray}{\rmlatinfont\textsuperscript{§~\theparCount}}}
	\pend% ending standard par
      ‚{\tiny $_{lb}$}‚

	  
	  \pstart \leavevmode% starting standard par
	य‚त एव\textbf{म‚तः} कार‚णा\textbf{द‚र्थैः} कार‚ण‚भूतैस्स‚ह कार्यात्म‚नां \textbf{श‚ब्दानां} स‚म्ब‚न्धः \textbf{पुरुषैः}‚{\tiny $_{lb}$}‚ क‚र्तृभि\textbf{र्द्धिया} बुद्ध्‚{\tiny $_{१}$}‚या \textbf{संस्कार्यो} व्य‚व‚स्थाप्यः । अर्थे स‚ति श‚ब्द‚स्य प्र‚योगाद‚स‚ति‚{\tiny $_{lb}$}‚ चाप्र‚योगात् । अथ \textbf{भावाभाव‚द्वारेण} श‚ब्द‚स्य यौ \textbf{भावाभावौ} तावा\textbf{श्रित्य} बुद्ध्या‚{\tiny $_{lb}$}‚ श‚ब्दार्थ‚योः स‚म्ब‚न्धो व्य‚व‚स्थाप्य‚ते इति वाक्यार्थः ।
	{\color{gray}{\rmlatinfont\textsuperscript{§~\theparCount}}}
	\pend% ending standard par
      ‚{\tiny $_{lb}$}‚

	  
	  \pstart \leavevmode% starting standard par
	६. एत‚देव व्याच‚ष्टे । \textbf{तावेवे}त्यादि । अर्थ‚द्वारेण श‚ब्द‚स्य यौ \textbf{भावाभावौ}‚{\tiny $_{lb}$}‚ तावा\textbf{श्रित्य} स्व‚भावेना\textbf{संसृष्टाव‚पि} श‚ब्दार्थाव‚स‚म्ब‚न्धिनौ \textbf{संसृष्टौ} स‚म्ब‚द्धौ \textbf{पुरुष‚स्य‚{\tiny $_{lb}$}‚ प्र‚ति‚{\tiny $_{२}$}‚भातः} । विक‚ल्प‚बुद्धौ प्र‚तिभासेते । कुतः [।] \textbf{व्य‚व‚हार‚भाव‚नातः} । अनादि‚{\tiny $_{lb}$}‚कालीन‚व्य‚व‚हाराभ्यास‚तः । \textbf{इति} हेतोः \textbf{पौरुषेयो भावानां} श‚ब्दार्थानां संश्लेषः‚{\tiny $_{lb}$}‚ स‚म्ब‚न्धः । व्य‚व‚हार‚वास‚नाब‚लेनाव‚स्थाप्य‚मान‚त्वात ।
	{\color{gray}{\rmlatinfont\textsuperscript{§~\theparCount}}}
	\pend% ending standard par
      ‚{\tiny $_{lb}$}‚

	  
	  \pstart \leavevmode% starting standard par
	७. व‚स्तुभूत‚स‚म्ब‚न्धाभ्युप‚ग‚मेऽय‚म‚प‚रो दोष इत्याह । \textbf{किंचे}त्यादि । वाच्य‚त्वे‚{\tiny $_{lb}$}‚नाभिम‚त‚स्या\textbf{श्र‚य‚स्य विनाशाद}विन‚ष्टे स‚म्ब‚न्धे । स त‚स्य वा‚{\tiny $_{३}$}‚च‚क‚त्वेनाभिम‚तः‚{\tiny $_{lb}$}‚ शाब्दो स‚म्ब‚न्धः । नास्य स‚म्ब‚न्धोस्तीति कृत्वा त‚द‚भाव‚स्त‚स्माद‚स‚म्ब‚न्धात् पुन‚र‚{\tiny $_{lb}$}‚\textbf{पूर्वेण} वाच्येन \textbf{न योज्येत} । त‚त‚श्चो\textbf{त्प‚न्नोत्प‚न्नाश्च भावा अवाच्याः} स्युर‚स\textbf{म्ब‚न्धिनो‚{\tiny $_{lb}$}‚ य‚तः} । अस‚म्ब‚न्धिन एव कुतः [।] \textbf{स्थित‚स‚म्ब‚न्धाभावात्} । ये य उत्प‚द्य‚न्ते तेषु‚{\tiny $_{lb}$}‚ तेषु न ताव‚दुत्पादात् पूर्वं स‚म्ब‚न्धः स्थितो द्विष्ठ‚स्य त‚स्य स‚म्ब‚न्धिन‚म‚न्त‚रेण स्थाना‚{\tiny $_{lb}$}‚योगात् ।‚{\tiny $_{४}$}‚ \textbf{त‚त्रा}पीत्यादि । त‚त्राप्युत्प‚न्नोत्प‚न्नेष्व‚र्थेषु \textbf{तैरेवार्थैस्स‚ह} स‚म्ब‚न्ध‚स्यो\textbf{त्पादे‚{\tiny $_{lb}$}‚ ‚{\tiny $_{lb}$}‚ ‚{\tiny $_{lb}$}‚ \leavevmode\ledsidenote{\textenglish{421/s}}क‚ल्प्य‚माने । न स्व‚भाव‚विप‚र्य‚यः श‚ब्देषु युक्तः} । योसौ प‚श्चाद‚र्थेन स‚म्ब‚न्ध‚{\tiny $_{lb}$}‚ उत्प‚द्य‚ते । त‚त्स‚म्ब‚न्धिस्व‚भाव‚ता पूर्व‚न्नास्ति स‚म्ब‚न्धाभावात् प‚श्चात्तु भ‚व‚तीति‚{\tiny $_{lb}$}‚ स्व‚भाव‚विप‚र्य‚यः श‚ब्दानां स्यात् । स च नित्य‚त्वान्न युज्य‚ते ।
	{\color{gray}{\rmlatinfont\textsuperscript{§~\theparCount}}}
	\pend% ending standard par
      ‚{\tiny $_{lb}$}‚

	  
	  \pstart \leavevmode% starting standard par
	\textbf{अथे}त्यादि व्याख्यानं । पूर्व\textbf{स‚म्ब‚न्धिविनाशे} विन‚{\tiny $_{५}$}‚ष्टः स‚म्ब‚न्धो य‚स्य श‚ब्द‚स्य‚{\tiny $_{lb}$}‚ त‚स्या\textbf{र्थान्त‚रे} प‚श्चादुत्प‚न्ने वैगुण्यं स‚म्ब‚न्ध‚वैक‚ल्यं त‚स्माच्चा\textbf{र्थाना}म्प‚श्चादुत्प‚न्ना‚{\tiny $_{lb}$}‚नाम\textbf{वाच्य‚ता मा भूदिति कृत्वोत्प‚न्नोत्प‚न्नोर्थः स‚म्ब‚न्ध‚वान्} अगृहीत‚स‚म्ब‚न्ध‚{\tiny $_{lb}$}‚ एव \textbf{य‚द्युत्प‚द्येत} । त‚थापि \textbf{स‚म्ब‚न्ध उत्प‚न्नोपि न श‚ब्दे स्यात्} । श‚ब्द‚स्तेन स‚म्ब‚न्धेन‚{\tiny $_{lb}$}‚ त‚द्वान्न स्यात् । किङ्कार‚णं [।] त‚स्य श‚ब्द‚स्य \textbf{तेन} प‚श्चादुत्प‚न्नेन \textbf{स‚म्ब‚न्धेना‚{\tiny $_{lb}$}‚स‚म्ब‚न्धि‚{\tiny $_{६}$}‚स्व‚भाव‚स्य त‚द्भावायोगात्} । तेन स‚म्ब‚न्धेन स‚म्ब‚न्धिस्व‚भावायोगात् ।‚{\tiny $_{lb}$}‚ क‚दा [।] \textbf{स्व‚भाव‚विप‚र्य‚य‚म‚न्त‚रेण} । पूर्वास‚म्ब‚न्धिस्व‚भाव‚त्याग‚म‚न्त‚रेण दोषान्त‚र‚{\tiny $_{lb}$}‚म‚प्याह । अर्थेन \textbf{स‚होत्प‚न्न‚स्य} च स‚म्ब‚न्ध‚स्य श‚ब्दा\textbf{द‚न्य‚तः सिद्ध‚स्यानुप‚कारिणि‚{\tiny $_{lb}$}‚ श‚ब्दे} त‚स्या\textbf{स‚माश्र‚य‚त्वाच्च} । त‚द्भावायोग एव । \href{http://sarit.indology.info/?cref=pv.3.231}{२३४}
	{\color{gray}{\rmlatinfont\textsuperscript{§~\theparCount}}}
	\pend% ending standard par
      ‚{\tiny $_{lb}$}‚

	  
	  \pstart \leavevmode% starting standard par
	अथ \textbf{त‚स्यापि} श‚ब्द‚स्य \textbf{त‚दुत्प‚त्तिस‚ह‚कारित्वे} स‚म्ब‚न्धो‚{\tiny $_{७}$}‚त्पत्तिं प्र‚ति स‚ह‚का- \leavevmode\ledsidenote{\textenglish{151b/PSVTa}}‚{\tiny $_{lb}$}‚ रित्वे क‚ल्प्य‚माने । \textbf{स‚म‚र्थ‚स्य} श‚ब्द‚स्य \textbf{नित्योत्पाद‚न‚प्र‚संगः} । स‚र्व‚कालं स‚म्ब‚न्ध‚ज‚न‚न‚{\tiny $_{lb}$}‚प्र‚संगः । किं कार‚णं [।] \textbf{स‚ह‚कार्य‚न‚पेक्ष‚त्वात्} । अन‚पेक्ष‚त्वं पुन\textbf{र्नित्य‚स्य} श‚ब्द‚स्य‚{\tiny $_{lb}$}‚ स‚ह‚कारिभि\textbf{र‚नुप‚कारात्} । अथास‚म‚र्थः स‚ह‚कार्य‚पेक्ष‚या ज‚न‚येत् । त‚दा प्राक्स‚म्ब‚{\tiny $_{lb}$}‚न्ध‚ज‚न‚नं प्र‚त्य\textbf{साम‚र्थ्येपि प‚श्चाद}प्य‚र्थे स‚न्निधिकालेप्य\textbf{श‚क्तिः} । किं कार‚णं [।]‚{\tiny $_{lb}$}‚ पूर्वास‚म‚र्थ\textbf{स्व‚भावात्यागात्} । व‚स्तुभू‚{\tiny $_{१}$}‚ते \textbf{स‚म्ब‚न्धे} यो \textbf{दोषोय‚म्विक‚ल्पिते} बुद्धिस‚न्द‚{\tiny $_{lb}$}‚र्शिते स‚म्ब‚न्धे \textbf{ना}स्ति ।
	{\color{gray}{\rmlatinfont\textsuperscript{§~\theparCount}}}
	\pend% ending standard par
      ‚{\tiny $_{lb}$}‚

	  
	  \pstart \leavevmode% starting standard par
	\textbf{न ही}त्यादि व्याख्यानं । त‚द\textbf{पेक्षा} तेषाम्भावानां प‚र‚स्प‚रापेक्षा कुत‚श्चिन्नि‚{\tiny $_{lb}$}‚मित्ताद् बुद्धिप‚रिक‚ल्पितात् । \textbf{त‚ल्ल‚क्ष‚ण‚स्स‚म्ब‚न्धः पुरुष‚भाव‚नाप्र‚तिभासी} [।]‚{\tiny $_{lb}$}‚ पुरुष‚स्य भाव‚नाभ्यास‚व‚ती विक‚ल्प‚बुद्धिस्त‚त्र प्र‚तिभासितुं शील‚म‚स्येति कृत्वा । स‚{\tiny $_{lb}$}‚ चैवं पौरुषेय‚स्स‚म्ब‚न्धो न भाव‚श्लेषापेक्षी । भाव‚श्लेषो मिश्र‚ता त‚द‚पेक्षी न भ‚व‚ति‚{\tiny $_{२}$}‚‚{\tiny $_{lb}$}‚ ‚{\tiny $_{lb}$}‚ \leavevmode\ledsidenote{\textenglish{422/s}}स‚र्वेषाम्भावानां प्र‚कृतिभेदेन स्व‚स्व‚रूपाव‚स्थानात् । केव‚लं \textbf{सोयं} पुरुषो \textbf{नित्याना‚{\tiny $_{lb}$}‚म‚पि स्व‚भाव‚मा}त्मीय\textbf{म‚प‚राव‚र्त्त}य‚न्न स‚म्ब‚न्धिस्व‚भावं स्थिर‚म‚प‚नीयान्यं स‚म्ब‚न्धि‚{\tiny $_{lb}$}‚स्व‚भाव‚म‚नाद‚ध‚त् । \textbf{कुत}श्चि\textbf{दिति} त‚द्भावे क‚स्य‚चिद् भाव‚द‚र्श‚नात् । अन्त‚स्त‚थाभूत‚{\tiny $_{lb}$}‚व्य‚व‚हार‚वास‚नाप‚रिपाकाद्वा । \textbf{स्व‚य‚मुत्प्रेक्ष्ये}द‚मिह स‚म्ब‚न्ध‚मिति \textbf{घ‚ट‚येदिति} । पुरुष‚{\tiny $_{lb}$}‚व्य‚व‚हाराभ्यासात्तेपि नित्याभि‚{\tiny $_{३}$}‚म‚तास्त‚था स्युः । पुरुषोप‚क‚ल्पित‚स‚म्ब‚न्ध‚व‚न्तः स्युः ।‚{\tiny $_{lb}$}‚ \textbf{न च ताव‚ता ते} पूर्व‚व्य‚व‚स्थिताद‚स‚म्ब‚न्धिस्व‚भावाच्\textbf{च्य‚व‚न‚ध‚र्माणः} ।
	{\color{gray}{\rmlatinfont\textsuperscript{§~\theparCount}}}
	\pend% ending standard par
      ‚{\tiny $_{lb}$}‚

	  
	  \pstart \leavevmode% starting standard par
	एत‚दुक्त‚म्भ‚व‚ति । बुद्धिप‚रिक‚ल्पितः स‚म्ब‚न्ध‚स्त्व‚त्प‚रिक‚ल्पितः स‚म्ब‚न्ध‚स्त्व‚त्प‚{\tiny $_{lb}$}‚रिक‚ल्पितेष्व‚पि नित्येषु न विरुध्य‚ते । त‚स्मात् स एवाश्र‚यितुं युक्तो न वास्त‚व इति ।‚{\tiny $_{lb}$}‚ \textbf{य‚दुक्तं प्रागाश्र‚य‚स्यापायेन} कार‚णेना\textbf{श्रित‚स्य स‚म्ब‚न्ध‚स्य विनाशाद‚नित्यः} स‚म्ब‚न्ध‚{\tiny $_{lb}$}‚ \textbf{इति । त‚त्रै}त‚स्मिन् दूष‚णे \textbf{नित्य‚त्वात्} स‚{\tiny $_{४}$}‚म्ब‚न्ध‚स्या\textbf{श्र‚यापायेप्य‚नाशो य‚दि} जातिव‚दि‚{\tiny $_{lb}$}‚त्युच्य‚ते । य‚था जाते\textbf{र्नित्य‚त्वादाश्र‚य‚ना}शेप्य‚नाश‚स्त‚द्व‚त् स‚म्ब‚न्ध‚स्येति । \href{http://sarit.indology.info/?cref=pv.3.232}{२३५}
	{\color{gray}{\rmlatinfont\textsuperscript{§~\theparCount}}}
	\pend% ending standard par
      ‚{\tiny $_{lb}$}‚

	  
	  \pstart \leavevmode% starting standard par
	अत्राप्याह । \textbf{नित्ये}ष्वाश्र‚याभिम‚तेषु जात्यादिष्व‚पि काय‚षु किमा\textbf{श्र}य‚स्य‚{\tiny $_{lb}$}‚ \textbf{साम‚र्थ्य}न्नैव । \textbf{येनेष्टः सो}किञ्चित्क‚र \textbf{आश्र‚यः । श्रूय‚त} इत्यादि विव‚र‚णं । श्रूय‚त‚{\tiny $_{lb}$}‚ इत्य‚नेन प्र‚सिद्धिमात्र‚मेत‚न्निर्व‚स्तुक‚मित्येत‚दाह [।] व्य‚क्त्याश्रिता । केव‚ल\textbf{न्नित्येषु}‚{\tiny $_{lb}$}‚ जात्यादि‚{\tiny $_{५}$}‚ष्वा\textbf{श्र‚य‚साम‚र्थ्य‚मा}श्र‚य‚कृत‚मुप‚कार‚न्न \textbf{प‚श्यामो येना}सावाश्र‚याभिम‚त \textbf{आश्र‚यः‚{\tiny $_{lb}$}‚ स्यात्} । किं कार‚णं [।] \textbf{कृत‚स्य} सिद्ध‚स्व‚भाव‚स्य \textbf{कार‚णाभावात् । अकार‚क‚स्य‚{\tiny $_{lb}$}‚ चा}श्र‚य\textbf{स्यान‚पेक्ष‚त्वात्} । \href{http://sarit.indology.info/?cref=pv.3.233}{२३६}
	{\color{gray}{\rmlatinfont\textsuperscript{§~\theparCount}}}
	\pend% ending standard par
      ‚{\tiny $_{lb}$}‚

	  
	  \pstart \leavevmode% starting standard par
	आश्र‚यात् स‚काशा\textbf{ज्जातेः स‚म्ब‚न्ध‚स्य च व्य‚क्ति}र‚भिव्य‚क्तिरूप‚स्य म्भ\edtext{}{\lemma{म्भ}\Bfootnote{? तु}}‚{\tiny $_{lb}$}‚ योग्य‚ता भ‚व‚ति । सैव चाश्र‚य‚कृत \textbf{उप‚कार}स्तेन कार‚णेन प‚दार्थ \textbf{आश्र‚य} इति चेत् ।
	{\color{gray}{\rmlatinfont\textsuperscript{§~\theparCount}}}
	\pend% ending standard par
      ‚{\tiny $_{lb}$}‚

	  
	  \pstart \leavevmode% starting standard par
	उत्त‚र‚माह । \textbf{ज्ञाने}त्या‚{\tiny $_{६}$}‚दि । \textbf{स‚ह‚कारिणां ज्ञानोत्पाद‚न‚हेतूनां} प्र‚दीपादीनां‚{\tiny $_{lb}$}‚ ‚{\tiny $_{lb}$}‚ \leavevmode\ledsidenote{\textenglish{423/s}}\textbf{स‚म्ब‚न्धाद्} योग्य‚देशाव‚स्थानात् स‚काशात् \textbf{त‚दुत्पाद‚न‚योग्य‚त्वेन} स्वानुरूप‚ज्ञानो‚{\tiny $_{lb}$}‚त्पाद‚न‚साम‚र्थ्येन \textbf{घ‚टादिष्व‚पि} व्यंग्येषु \textbf{योत्प‚त्तिः} सैव \textbf{युक्तिज्ञै}र्न्याय‚विद्भि\textbf{र्व्य‚क्तिरि‚{\tiny $_{lb}$}‚ष्य‚ते} । जात्यादीनान्तु व्य‚ङ्ग्यानान्नित्य‚त्वा\textbf{द‚विकारिणां} व्य‚ञ्ज‚कात्स‚काशा\textbf{द‚विशेषे} ।‚{\tiny $_{lb}$}‚ ज्ञानोत्पाद‚न‚योग्य‚तानुत्प‚त्तौ । \textbf{स्वैर्व्य‚ञ्ज}कैस्तेषां जा‚{\tiny $_{७}$}‚त्यादीनां \textbf{कोर्थः क} उप‚कारः \leavevmode\ledsidenote{\textenglish{152a/PSVTa}}‚{\tiny $_{lb}$}‚ कृतः [।] नैव क‚श्चित् । य‚त‚स्ते जात्याद‚य‚स्तैर्व्य‚ञ्ज‚कै\textbf{र्व्य‚क्ता म‚ताः} । \href{http://sarit.indology.info/?cref=pv.3.234}{२३७}
	{\color{gray}{\rmlatinfont\textsuperscript{§~\theparCount}}}
	\pend% ending standard par
      ‚{\tiny $_{lb}$}‚

	  
	  \pstart \leavevmode% starting standard par
	\textbf{स‚ह‚कारिण} इत्यादिना व्याच‚ष्टे । स‚ह‚कारिणः प्र‚दीपादेः \textbf{स‚काशात्} किम्भूता\textbf{दु‚{\tiny $_{lb}$}‚पादानापेक्षात्} पूर्व‚को\add{ज्ञान‚ज‚न‚नास‚म‚र्थो}\edtext{}{\edlabel{pvsvt_423-1}\label{pvsvt_423-1}\lemma{को}\Bfootnote{In the margin. }} ‚{\tiny $_{१}$}‚घ‚टादिल‚क्ष‚ण उपादान‚कार‚णं स‚म‚र्थ‚स्य‚{\tiny $_{lb}$}‚ घ‚टादिल‚क्ष‚ण‚स्य घ‚टापेक्ष‚त्वात् ।\edtext{\textsuperscript{*}}{\edlabel{pvsvt_423-2}\label{pvsvt_423-2}\lemma{*}\Bfootnote{त‚द‚पेक्ष‚त्वात् is missing.}} स्व‚विष‚य\textbf{ज्ञान‚ज‚न‚नं} प्र‚ति योग्य‚स्य \textbf{क्ष‚णान्त‚र‚स्यो‚{\tiny $_{lb}$}‚त्प‚त्तिरेव घ‚टादीनाम‚भिव्य‚क्तिः । अन्य‚था} य‚दि प्र‚दीपादेस्स‚काशात् ज्ञानोत्पाद‚{\tiny $_{lb}$}‚न‚योग्य‚तां न ल‚भ‚न्ते घ‚टाद‚य‚स्त‚दा\textbf{न‚पेक्ष्य त‚दुप‚कारं} प्र‚दीपोप‚कारं घ‚टादीनां \textbf{ज्ञान‚{\tiny $_{lb}$}‚ज‚न‚न‚प्र‚स‚ङ्गात्} ।
	{\color{gray}{\rmlatinfont\textsuperscript{§~\theparCount}}}
	\pend% ending standard par
      ‚{\tiny $_{lb}$}‚

	  
	  \pstart \leavevmode% starting standard par
	अथ प्र‚दीपादिः प्राग‚स‚म‚र्थ‚स्य व्यंग्य‚स्य साम‚र्थ्यं क‚रोतीतीष्य‚ते । त‚दा \textbf{साम‚र्थ्य‚{\tiny $_{lb}$}‚कारिण‚श्च} प्र‚दीपादेर्घ‚टादीन्प्र‚ति \textbf{ज‚न‚क‚त्वात्} । किं कार‚णं [।] \textbf{त‚स्य} साम‚र्थ्य‚स्य‚{\tiny $_{lb}$}‚ \textbf{त‚दात्म‚क‚त्वाद}व्यं‚{\tiny $_{२}$}‚ग्यात्म‚क‚त्वात् । त‚स्य ज‚न‚ने घ‚टादिरेव ज‚नितः स्यात् ।
	{\color{gray}{\rmlatinfont\textsuperscript{§~\theparCount}}}
	\pend% ending standard par
      ‚{\tiny $_{lb}$}‚

	  
	  \pstart \leavevmode% starting standard par
	अथ माभूदेष दोष इति प्र‚दीपादिकृत‚स्य साम‚र्थ्य‚स्य व्यंग्याद‚र्थान्त‚र‚त्व‚मि‚{\tiny $_{lb}$}‚ष्य‚ते । त‚दा\textbf{र्थान्त‚र‚त्वे च} साम‚र्थ्य‚स्याभ्युप‚ग‚म्य‚माने । भाव‚स्य घ‚टादेः प्र‚दीपा‚{\tiny $_{lb}$}‚दिभि\textbf{र‚नुप‚कार‚प्र‚स‚ङ्गात्} । न ह्य‚न्य‚स्मिन् कृतेन्य उप‚कृतो भ‚व‚त्य‚तिप्र‚संगात् ।‚{\tiny $_{lb}$}‚ य‚च्च प्र‚दीपादिकृतं \textbf{साम‚र्थ्य}म‚र्थान्त‚र‚न्त‚स्मा\textbf{च्च} घ‚टादि\textbf{ज्ञानो‚{\tiny $_{३}$}‚त्प‚त्ते}स्साम‚र्थ्य‚मेव स‚र्व‚{\tiny $_{lb}$}‚कालं गृह्येतेति स्व‚विष‚य‚ज्ञानाज‚न‚नानां \textbf{घ‚टादीनां नित्य‚म‚ग्र‚ह‚ण}प्र‚स‚ङ्गात् । इष्य‚ते‚{\tiny $_{lb}$}‚ च घ‚टादीनां ग्र‚ह‚णात् । त‚दाप्य‚नालोकापेक्ष‚ग्र‚ह‚ण‚प्र‚संगात् । \textbf{आलोक‚म‚न‚पेक्ष्य}‚{\tiny $_{lb}$}‚ घ‚टादीनाङ्\textbf{ग्र‚ह‚ण‚म्प्र‚स‚ज्येते}त्य‚र्थः । आलोकान‚पेक्षैव क‚थ‚मिति चेदाह । \textbf{अन‚पे}‚{\tiny $_{lb}$}‚‚{\tiny $_{lb}$}‚ ‚{\tiny $_{lb}$}‚ ‚{\tiny $_{lb}$}‚ \leavevmode\ledsidenote{\textenglish{424/s}}क्षेत्यादि । येयं घ‚टादीनामालोकान‚पेक्षा प्र‚स‚क्ता सा आ\textbf{त्मानुप‚कारा‚{\tiny $_{४}$}‚त्} प्र‚दीपा‚{\tiny $_{lb}$}‚दिभिर्घ‚टाद्यात्म‚नोनुप‚काराद् व्य‚तिरिक्त‚स्य हि साम‚र्थ्य‚स्य क‚र‚णेन घ‚टादीनां क‚श्चि‚{\tiny $_{lb}$}‚दुप‚कारः । \textbf{त‚दिति} त‚स्माद् \textbf{इमे} व्यंग्याः \textbf{स्व‚विष‚य‚ज्ञान‚ज‚न‚ने । प‚र}म्प्र‚दीपादि\textbf{क‚म‚{\tiny $_{lb}$}‚पेक्ष‚माणाः । त‚त} इति प‚र‚स्मात् \textbf{स्व‚भावातिश‚यं} ज्ञान‚ज‚न‚न‚योग्यं स्व‚रूपं \textbf{स्वीकु‚{\tiny $_{lb}$}‚र्व‚न्ति [।] तेन} कार‚णेनास्य प्र‚दीपादेस्ते व्यंग्याभिम‚ता \textbf{ज‚न्या एव} । य‚दि ज‚न्या‚{\tiny $_{५}$}‚ः‚{\tiny $_{lb}$}‚ क‚स्माद् व्य‚ङ्ग्या इत्येव‚म्व्य‚प‚दिश्य‚न्त इत्याह । \textbf{ज्ञेय‚रूपे}त्यादि । व्य‚ञ्ज‚कात्‚{\tiny $_{lb}$}‚ प्र‚दीपादेस्स‚काशाज्ज्ञेय‚रूप‚स्य ग्राह्य‚रूप‚स्या\textbf{साद‚ना}ल्लाभात्तु कार‚णाद‚व‚श्यं त‚द्विष‚यं‚{\tiny $_{lb}$}‚ ज्ञान‚म्भ‚व‚ति [।] अतो \textbf{ज्ञान‚व‚शेन । कार्यातिश‚य‚वाचिना} । अगृहीत‚ज्ञानं कार्यं कार्या‚{\tiny $_{lb}$}‚तिश‚य‚स्त‚द्वाचिना व्यंग्यादि\textbf{श‚ब्देन} । अनागृहीत‚ज्ञानेभ्यः कार्येम्यो \textbf{विशेष‚ख्यात्य‚र्थ} ।‚{\tiny $_{lb}$}‚ य‚दाव‚श्य‚केणो‚{\tiny $_{६}$}‚\edtext{}{\lemma{केणो}\Bfootnote{? नो}}पात्त‚ज्ञान‚न्त‚देव ज‚न्य‚म‚पि स‚द् \textbf{व्य‚ङ्ग्य‚मित्युच्य‚ते} । य‚त्तु‚{\tiny $_{lb}$}‚ नैवंभूत‚न्त‚त्कार्य‚मेवेत्युच्य‚त इत्येवं प्र‚सिद्ध्य‚र्थं व्यङ्ग्याः ख्याप्य‚न्त इत्य‚र्थः । \textbf{नैवं‚{\tiny $_{lb}$}‚ जातिस‚म्ब‚न्धाद‚य} इति । जातिश्च स‚म्ब‚न्ध‚श्च तावादी येषां । आदिश‚ब्दाद‚न्य‚स्यापि‚{\tiny $_{lb}$}‚ नित्याभिम‚त‚स्याश्रित‚स्य प‚रिग्र‚हः । \textbf{क‚थंचिदा}श्र‚याभिम‚तेना\textbf{नुप‚कार्य‚त्वात् । तेनानु‚{\tiny $_{lb}$}‚\leavevmode\ledsidenote{\textenglish{152b/PSVTa}} प‚कारिणा नैव व्य‚क्ता युज्य‚न्त} इति स‚{\tiny $_{७}$}‚म्ब‚न्धः ।
	{\color{gray}{\rmlatinfont\textsuperscript{§~\theparCount}}}
	\pend% ending standard par
      ‚{\tiny $_{lb}$}‚

	  
	  \pstart \leavevmode% starting standard par
	न च स‚म्ब‚न्ध‚स्त्रिप्र‚माण‚क इति द‚र्श‚यितुमाह । \textbf{स‚म्ब‚न्ध‚स्य च व‚स्तुत्वे} स‚म्ब‚{\tiny $_{lb}$}‚न्धिभ्यां स‚म्ब‚न्ध‚स्य \textbf{भेदात्} तृतीयः स‚म्ब‚न्धाख्यो भांवो जातः । स च य‚द्युप‚ल‚ब्धि‚{\tiny $_{lb}$}‚ल‚क्ष‚ण‚प्राप्त‚स्त‚दा प‚दार्थ‚त्र‚याल‚म्ब‚न‚त्वेन \textbf{स्याद् बुद्धिचित्र‚ता । स चायं} श‚ब्दार्थ‚यो‚{\tiny $_{lb}$}‚\textbf{स्स‚म्ब‚न्धो व‚स्तु भ‚व‚न्निय‚मेन श‚ब्दार्थाभ्यां भेदाभेदौ नातिव‚र्त्त‚ते} क्राम‚ति । भेदा‚{\tiny $_{lb}$}‚भेद‚व्य‚तिरिक्तः प्र‚कारो भ‚विष्य‚तीति चेदाह ।‚{\tiny $_{१}$}‚ \textbf{रूपं ही}ति य‚तो रूपं स्व‚भावो‚{\tiny $_{lb}$}‚ व‚स्तु \textbf{त‚स्य स्व‚भाव‚स्यात‚त्त्व‚मे}वात‚द्भाव \textbf{एवान्य‚त्त्व‚मित्युक्तं} प्राक् । रूप‚स्यात‚द्‚{\tiny $_{lb}$}‚‚{\tiny $_{lb}$}‚ \leavevmode\ledsidenote{\textenglish{425/s}}भूत‚स्यान्य‚त्त्वाव्य‚तिक्र‚मादि त्य‚त्रान्त‚रे ।\edtext{\textsuperscript{*}}{\edlabel{pvsvt_425-1}\label{pvsvt_425-1}\lemma{*}\Bfootnote{\cite{pvb-B} य‚च्च ।}}
	{\color{gray}{\rmlatinfont\textsuperscript{§~\theparCount}}}
	\pend% ending standard par
      ‚{\tiny $_{lb}$}‚

	  
	  \pstart \leavevmode% starting standard par
	\textbf{स चायं} स‚म्ब‚न्ध ऐन्द्रियः स‚न् \textbf{स्व‚बुद्धौ} स‚म्ब‚न्धाल‚म्ब‚नायां बुद्धौ । \textbf{त‚द‚न्य‚विवेके}‚{\tiny $_{lb}$}‚नेति । त‚स्मात्स‚म्ब‚न्धा\textbf{द‚न्य‚स्स‚म्ब‚न्धी} त‚तो \textbf{विवेके}नार्थान्त‚रेण \textbf{रूपेणाप्र‚तिभास‚मानः‚{\tiny $_{lb}$}‚ क‚थ‚न्त‚था स्यात्} त‚द‚न्य‚विवेकि रूपं क‚थं स्यात् । किं‚{\tiny $_{२}$}‚ कार‚णं [।] दृश्य‚स्य प्र‚त्य‚क्षाद‚{\tiny $_{lb}$}‚र्थाद‚विवेकोऽपृथ‚ग्भावः [।] य‚च्चाद‚र्श‚न‚न्त‚योर्दृश्या\textbf{विवेकाद‚र्श‚न‚यो}र्य‚थाक्र‚मं \textbf{विवेक‚{\tiny $_{lb}$}‚स‚त्ता विप‚र्य‚याश्र‚य‚त्वात्} । विवेक‚विप‚र्य‚यो विवेकाभाव‚स्त‚स्य दृश्याविवेक आश्र‚यः ।‚{\tiny $_{lb}$}‚ स‚त्ताविप‚र्य‚योस‚त्त्व‚न्त‚स्य दृश्याद‚र्श‚न‚माश्र‚यः । तेनाय‚म‚र्थः [।] य‚द्य‚तो भेदेन नोप‚ल‚भ्य‚ते‚{\tiny $_{lb}$}‚ त‚त्त‚तो नान्य‚त् । य‚द्य‚त्\edtext{}{\edlabel{pvsvt_425-2}\label{pvsvt_425-2}\lemma{त्}\Bfootnote{\cite{pvb-B} ते चेद् ।}} दृश्यं स‚न्नोप‚ल‚भ्य‚ते त‚न्नास्तीति या\edtext{}{\edlabel{pvsvt_425-3}\label{pvsvt_425-3}\lemma{या}\Bfootnote{\cite{pvb-B} प्राप्य ।}}व‚त् ।
	{\color{gray}{\rmlatinfont\textsuperscript{§~\theparCount}}}
	\pend% ending standard par
      ‚{\tiny $_{lb}$}‚

	  
	  \pstart \leavevmode% starting standard par
	\textbf{अन्य‚थेति} । य‚द्य‚तोर्थान्त‚र‚म‚भ्युप‚ग‚तं दृश्यं च त‚स्य त‚स्माद‚विवेके स‚त्य‚द‚र्श‚ने‚{\tiny $_{lb}$}‚ च य‚दि विवेकः स‚त्ता च क‚ल्प्य‚ते । त‚दा \textbf{त‚त्स्थिते}र‚विवेकाभाव‚योर्व्य‚व‚स्थिते\textbf{र‚भाव‚{\tiny $_{lb}$}‚प्र‚स‚ङ्गः} । त‚था हि दृश्याविवेकाद‚र्श‚ने अस्या व्य‚व‚स्थाया निमित्ते । \textbf{ते चेति\edtext{}{\edlabel{pvsvt_425-3b}\label{pvsvt_425-3b}\lemma{चेति}\Bfootnote{\cite{pvb-B} प्राप्य ।}}} विवे‚{\tiny $_{lb}$}‚काभाव‚योर्न साध‚न‚मिष्टे त‚दा त‚द्व्य‚व‚स्थोच्छिद्य‚ते । \textbf{अतीन्द्रिय‚त्वात्} स‚म्ब‚न्ध‚स्य‚{\tiny $_{lb}$}‚ विवेकेन बुद्धाव\textbf{व‚प्र‚तिभास‚{\tiny $_{४}$}‚नेपि न} य‚थोक्त\textbf{दोष इन्द्रियादिव‚दिति चेत्} । य‚थेन्द्रि‚{\tiny $_{lb}$}‚य‚ञ्च‚क्षुरादि रूपादिभ्यो विवेकेन बुद्धौ न प्र‚तिभास‚तेऽथ च व्य‚तिरिक्त‚म‚स्ति ।‚{\tiny $_{lb}$}‚ त‚द्व‚त् स‚म्ब‚न्धो भ‚विष्य‚तीत्य‚र्थः
	{\color{gray}{\rmlatinfont\textsuperscript{§~\theparCount}}}
	\pend% ending standard par
      ‚{\tiny $_{lb}$}‚

	  
	  \pstart \leavevmode% starting standard par
	\textbf{ने}त्या चा र्यः । नातीन्द्रिय‚स्स‚म्ब‚न्धः । \textbf{त‚तो}तीन्द्रियात् स‚म्ब‚न्धाद‚र्थ‚स्या\textbf{प्र‚ति‚{\tiny $_{lb}$}‚प‚त्तिप्र‚स‚ङ्गात्} । किं कार‚णं । \textbf{अप्र‚सिद्ध‚स्य} स्वेन रूपेणानिश्चित‚स्या\textbf{ज्ञाप‚क‚त्वात्} ।‚{\tiny $_{lb}$}‚ न हि येन स‚ह य‚स्य स‚म्ब‚न्धो‚{\tiny $_{५}$}‚ न गृह्य‚ते त‚द्द्वारेण त‚स्य प्र‚तीतिर्युक्ता ।
	{\color{gray}{\rmlatinfont\textsuperscript{§~\theparCount}}}
	\pend% ending standard par
      ‚{\tiny $_{lb}$}‚

	  
	  \pstart \leavevmode% starting standard par
	अथाज्ञात एव स‚म्ब‚न्धोर्थं ज्ञाप‚य‚तीन्द्रिय‚व‚दित्याह । स‚न्निधिमात्रेणेत्यादि ।‚{\tiny $_{lb}$}‚ स‚म्ब‚न्ध‚स्य \textbf{स‚न्निधिमात्रेण} स‚त्तामात्रेणार्थ\textbf{ज्ञाप‚ने}भ्युप‚ग‚म्य‚माने श‚ब्दार्थ‚स‚म्ब‚न्ध‚{\tiny $_{lb}$}‚म्प्र‚त्य\textbf{व्युत्प‚न्नानाम\edtext{}{\edlabel{pvsvt_425-4}\label{pvsvt_425-4}\lemma{न्नानाम}\Bfootnote{४ }}}प्य‚स्यार्थ‚स्याय‚म्वाच‚क इति प्र‚तिप‚त्तिः \textbf{स्यात्} ।
	{\color{gray}{\rmlatinfont\textsuperscript{§~\theparCount}}}
	\pend% ending standard par
      ‚{\tiny $_{lb}$}‚

	  
	  \pstart \leavevmode% starting standard par
	स्यान्म‚त‚म् [।] अस्त्येव स‚म्ब‚न्ध‚स्य विवेकेनोप‚ल‚ब्धिः । सा तु न प्र‚त्य‚क्षात्‚{\tiny $_{lb}$}‚ \leavevmode\ledsidenote{\textenglish{426/s}}किन्त‚र्ह्य‚नुमानादित्य‚त आह । \textbf{नानुमा‚{\tiny $_{६}$}‚नात् प्र‚तिप‚त्ति}स्स‚म्ब‚न्ध‚स्य । कुतो लिङ्गा‚{\tiny $_{lb}$}‚\textbf{भावात्} । न हि स‚म्ब‚न्ध‚साध‚नं किञ्चिल्लिङ्ग‚म‚स्ति । अर्थ‚प्र‚तीतिर‚पि न लिङ्गं‚{\tiny $_{lb}$}‚ \textbf{दृष्टान्तासिद्धेः} । न हि क्व‚चिद् दृष्टान्ते स‚म्ब‚न्ध‚कार्याऽर्थ‚प्र‚तीतिः प्र‚तिप‚न्ना । कि‚{\tiny $_{lb}$}‚ङ्कार‚णं [।] \textbf{त‚त्रापि} दृष्टान्त‚त्वेनोप‚नीते स‚म्ब‚न्ध‚स्या\textbf{तीन्द्रिय‚त्वेन} कार‚णेन \textbf{साध‚ना‚{\tiny $_{lb}$}‚पेक्ष‚णात्} । न चास्ति साध‚नं [।] त‚त्रापि दृष्टान्तासिद्धेः ।
	{\color{gray}{\rmlatinfont\textsuperscript{§~\theparCount}}}
	\pend% ending standard par
      ‚{\tiny $_{lb}$}‚

	  
	  \pstart \leavevmode% starting standard par
	\leavevmode\ledsidenote{\textenglish{153a/PSVTa}} त‚देत‚द् दृष्टान्त‚र‚हित‚त्व\textbf{मिन्द्रियादि}ष्व‚तीन्द्रिये‚{\tiny $_{७}$}‚षु स‚त्तासाध‚केनुमाने क्रिय‚माणे‚{\tiny $_{lb}$}‚ \textbf{तुल्य‚मिति चेत् । न} तुल्यं । कुतः । \textbf{तेषा}मिन्द्रियादीना\textbf{म‚न्य‚थानुमानात्} । न‚{\tiny $_{lb}$}‚प्र‚त्य‚क्षाणामिन्द्रियादीनामिद‚न्त‚या किंचिद् रूपं प्र‚साध्य‚ते । येन तुल्यो दोषः स्यात् ।‚{\tiny $_{lb}$}‚ किन्तु \textbf{ज्ञानं} कार्य‚भूतं प्र‚त्य‚क्षं \textbf{केषुचिदा}लोकादिषु \textbf{स‚त्सु} व्य‚तिरेका\textbf{न्व‚य‚व‚त्} । निमी‚{\tiny $_{lb}$}‚लित‚लोच‚नाद्य‚व‚स्थासु \textbf{व्यंतिरेक‚व‚त्} । उन्मीलित\edtext{}{\edlabel{pvsvt_426-1}\label{pvsvt_426-1}\lemma{उन्मीलित}\Bfootnote{\cite{pvb-B} कार्य‚त‚न्मात्रा० ।}}लोच‚नाद्य‚व‚स्थास‚म‚न्व‚य‚व‚त् ।‚{\tiny $_{lb}$}‚ त‚देवंभूतं ज्ञा‚{\tiny $_{१}$}‚नं कार्य‚न्त‚न्मात्रास‚म्भ‚वं । येषु स‚त्स्व‚प्य‚भ‚व‚द् दृष्ट\textbf{न्त‚न्मात्राद‚स‚{\tiny $_{lb}$}‚म्भ‚व}म‚नुत्प‚त्तिमात्म‚नः साध‚य‚ति । \textbf{त‚द्व्य‚तिरिक्तापेक्षां च} । य‚था स‚न्निहित‚{\tiny $_{lb}$}‚कार‚णाद् व्य‚तिरिक्त‚कार‚णापेक्षाञ्च \textbf{साध‚य‚ति} ।
	{\color{gray}{\rmlatinfont\textsuperscript{§~\theparCount}}}
	\pend% ending standard par
      ‚{\tiny $_{lb}$}‚

	  
	  \pstart \leavevmode% starting standard par
	अस्ति किम‚पि कार‚णान्त‚र‚मिति । \textbf{त‚तो} य‚थोक्तान्व‚यात्\edtext{}{\edlabel{pvsvt_426-2}\label{pvsvt_426-2}\lemma{यात्}\Bfootnote{\cite{pvb-B} ०क्तान्न्यायात् ।}} \textbf{कार्य‚द्वारेणेन्द्रिय‚{\tiny $_{lb}$}‚सिद्धिः} । कार‚णान्त‚र‚वैक‚ल्यास‚म्भ‚विन‚श्चांकुराद‚योत्र दृष्टान्तः । \textbf{नैवं स‚म्ब‚न्ध‚स्य}‚{\tiny $_{lb}$}‚ च‚क्षुरादिव‚{\tiny $_{२}$}‚त्कार्य‚व्य‚तिरेकेणानुमानं । विशेषानुमानात् । त‚था हि स‚म्ब‚न्धोस्तीति‚{\tiny $_{lb}$}‚ य‚द‚नुमान‚न्त‚द्विशेष‚स्यैवानुमानं ।
	{\color{gray}{\rmlatinfont\textsuperscript{§~\theparCount}}}
	\pend% ending standard par
      ‚{\tiny $_{lb}$}‚

	  
	  \pstart \leavevmode% starting standard par
	त‚च्चायुक्तं । किं कार‚णं । \textbf{त‚स्यैव} स‚म्ब‚न्ध‚स्या\textbf{सिद्धौ} स‚त्या\textbf{न्त‚त्कार्य‚स्यैव}‚{\tiny $_{lb}$}‚ स‚म्ब‚न्ध‚कार्य‚स्यैव \textbf{ज्ञान‚स्याभावात्} । श‚ब्दार्थो लिङ्ग‚मिति चेदाह । \textbf{न ही}त्यादि ।‚{\tiny $_{lb}$}‚ न हि \textbf{त‚त्र} स‚म्ब‚न्ध‚विशेषे \textbf{श‚ब्द‚रूंप‚म‚र्थो वा लिङ्गं} । किं कार‚णं [।] \textbf{त‚योः} श‚ब्दार्थ‚यो‚{\tiny $_{lb}$}‚\textbf{स्स‚र्व‚त्र योग्य‚त्वात्} । स‚र्व‚स्य श‚ब्द‚स्य स‚र्व‚स्मिन्न‚र्थे वाच‚क‚त्वेंन योग्य‚त्वात् स‚र्व‚स्य‚{\tiny $_{lb}$}‚ चार्थ‚स्य स‚र्व‚स्मिन् श‚ब्दे वाच्य‚त्वेन योग्य‚त्वात् ।\edtext{\textsuperscript{*}}{\edlabel{pvsvt_426-3}\label{pvsvt_426-3}\lemma{*}\Bfootnote{\cite{pvb-B} स‚र्व‚स्य चार्थ‚स्य स‚र्व‚स्मिन् श‚ब्दे वाच्य‚त्वेन योग्य‚त्वात्-- added}} अर्थ‚विशेष‚प्र‚तीतेश्च कार‚णं स‚म्ब‚{\tiny $_{lb}$}‚न्ध‚विशेष‚स्त‚स्य चा\textbf{र्थ‚विशेष‚प्र‚तीतिस‚माश्र‚य‚स्य} स‚म्ब‚न्ध‚स्यानिय‚ताभ्यां श‚ब्दार्था‚{\tiny $_{lb}$}‚‚{\tiny $_{lb}$}‚ \leavevmode\ledsidenote{\textenglish{427/s}}भ्याम\textbf{प्र‚त्याय‚नात्} । प्र‚त्याय‚ने वा विशेषाभावे न स‚र्व‚स‚म्ब‚न्ध‚प्र‚तीतः स‚र्वार्थ‚ग‚तिः‚{\tiny $_{lb}$}‚ स्यात् । न चैव‚म् [।] त‚स्माद‚निय‚ता‚{\tiny $_{४}$}‚भ्यां श‚ब्दार्थाभ्याम‚प्र‚तीतिर‚स्य स‚म्ब‚न्ध‚स्य ।
	{\color{gray}{\rmlatinfont\textsuperscript{§~\theparCount}}}
	\pend% ending standard par
      ‚{\tiny $_{lb}$}‚

	  
	  \pstart \leavevmode% starting standard par
	य‚दि च श‚ब्दार्थानां स‚म्ब‚न्धेन स‚ह स‚म्ब‚न्ध‚विशेषः सिद्धः स्यात् क्व‚चित्त‚दा‚{\tiny $_{lb}$}‚ स‚म्ब‚न्ध‚विशेष‚प्र‚तीतिः स्यात् । \textbf{न ह्य‚स‚त्यां स‚म्ब‚न्ध‚विशेषेण} श‚ब्दानां स‚म्ब‚न्ध‚सिद्धौ‚{\tiny $_{lb}$}‚ सा स‚म्ब‚न्ध‚विशेष‚प्र‚तीति\textbf{र्युक्ता} ।
	{\color{gray}{\rmlatinfont\textsuperscript{§~\theparCount}}}
	\pend% ending standard par
      ‚{\tiny $_{lb}$}‚

	  
	  \pstart \leavevmode% starting standard par
	अथ पुन‚स्स‚म्ब‚न्ध‚म‚न्त‚रेण श‚ब्दात् स‚म्ब‚न्ध‚विशेष‚प्र‚तीतिरिष्य‚ते \textbf{त‚स्याम्वा}‚{\tiny $_{lb}$}‚ स‚म्ब‚न्ध‚प्र‚तीता\textbf{व‚निमित्ताया}मिष्य‚माणाया‚{\tiny $_{५}$}‚\textbf{न्त‚द्विशेषः प्र‚तीतिनिय‚म‚व‚त्} स‚म्ब‚न्ध‚{\tiny $_{lb}$}‚विशेष‚प्र‚तीतिप्र‚तिनिय‚म‚व‚द्\edtext{}{\edlabel{pvsvt_427-1}\label{pvsvt_427-1}\lemma{द्}\Bfootnote{\cite{pvb-B} त‚द्विशेष‚प्र‚तीतिप्र‚तिनिय‚त‚व‚त् ।}} अर्थ\textbf{प्र‚तिपाद‚न‚म‚प्य‚निमित्तं श‚ब्दानां किन्नेष्य‚ते} ।‚{\tiny $_{lb}$}‚ त‚च्छ‚ब्दार्थ‚स्व‚भावं \textbf{लिङ्गं स‚दृश‚न्तुल्यं स‚र्व‚स‚म्ब‚न्धे\edtext{}{\edlabel{pvsvt_427-2}\label{pvsvt_427-2}\lemma{न्धे}\Bfootnote{\cite{pvb-B} तुल्यं स‚म्ब‚न्धः ।}} [।] त‚त‚श्चाविशेषेण} श‚ब्दः‚{\tiny $_{lb}$}‚ \textbf{स‚र्वं स‚म्ब‚न्ध‚ङ्ग‚म‚येत् त‚दाऽविशेषेणैव स‚र्व‚स्यार्थ‚स्य प्र‚तीतिः स्यात्} । स‚र्व‚स्य पुरु‚{\tiny $_{lb}$}‚ष‚स्य गृहीत‚स‚म‚य‚स्यागृहीत‚स‚म‚य‚स्य\edtext{}{\edlabel{pvsvt_427-3}\label{pvsvt_427-3}\lemma{स्य}\Bfootnote{\cite{pvb-B} च--added.}} स‚र्वार्थ‚प्र‚तीतिः स्यात् । \textbf{त‚स्माद्} य‚थोक्ते‚{\tiny $_{६}$}‚न‚{\tiny $_{lb}$}‚ न्यायेन \textbf{स‚म्ब‚न्ध‚सिद्ध्याऽर्थ‚प्र‚तीतेः} कार‚णान्न \textbf{क‚श्चित्} पुरुषोर्थ‚प्र‚तीतौ \textbf{संप्र‚दायं}‚{\tiny $_{lb}$}‚ प‚रोप‚देश\textbf{म‚पेक्षेत} ।
	{\color{gray}{\rmlatinfont\textsuperscript{§~\theparCount}}}
	\pend% ending standard par
      ‚{\tiny $_{lb}$}‚

	  
	  \pstart \leavevmode% starting standard par
	न केव‚ल‚स्य श‚ब्द‚स्य स‚म्ब‚न्ध‚सिद्धौ लिङ्ग‚त्वं किन्तु \textbf{संप्र‚दाय‚स‚हित‚स्य लिङ्ग‚{\tiny $_{lb}$}‚त्व‚मिति चेत् । त‚त्कि}मिदानी\textbf{म‚न‚या प‚र‚म्प‚र‚या} । स‚म्प्र‚दाय‚स्त‚तः श‚ब्द‚स्य लिङ्ग‚त्व‚{\tiny $_{lb}$}‚न्त‚स्मात् स‚म्ब‚न्ध‚प्र‚तीतिस्त‚तोर्थ‚स्य प्र‚त्याय‚न‚मिति किम‚न‚या प‚र‚म्प‚र‚या ।‚{\tiny $_{७}$}‚ \textbf{स एव} \leavevmode\ledsidenote{\textenglish{153b/PSVTa}}‚{\tiny $_{lb}$}‚ श‚ब्दः केव‚लो व‚स्तुभूत‚स‚म्ब‚न्ध‚र‚हित‚स्स\textbf{म्प्र‚दायापेक्षोर्थ‚ज्ञाप‚नं किन्न क‚रोति} येन‚{\tiny $_{lb}$}‚ स‚म्ब‚न्धोप‚रः क‚ल्प्य‚ते ।
	{\color{gray}{\rmlatinfont\textsuperscript{§~\theparCount}}}
	\pend% ending standard par
      ‚{\tiny $_{lb}$}‚

	  
	  \pstart \leavevmode% starting standard par
	अत एवार्थ‚प्र‚तीत्य‚न्य‚थानुप‚प‚त्त्यापि श‚क्तिस‚म्ब‚न्ध‚क‚ल्प‚ना निर‚स्ता । श‚क्ति‚{\tiny $_{lb}$}‚म‚न्त‚रेण संकेत‚ब‚लादेवार्थ‚प्र‚तीतिस‚म्भ‚वात् ।\edtext{\textsuperscript{*}}{\edlabel{pvsvt_427-4}\label{pvsvt_427-4}\lemma{*}\Bfootnote{\cite{pvb-B} स‚द्भावात् ।}} तेन स‚म्ब‚न्ध‚स्त्रिप्र‚माण‚क इति‚{\tiny $_{lb}$}‚ य‚दुच्य‚ते त‚द‚पास्तं ।
	{\color{gray}{\rmlatinfont\textsuperscript{§~\theparCount}}}
	\pend% ending standard par
      ‚{\tiny $_{lb}$}‚‚{\tiny $_{lb}$}‚‚{\tiny $_{lb}$}‚\textsuperscript{\textenglish{428/s}}

	  
	  \pstart \leavevmode% starting standard par
	\textbf{स च श‚ब्दो य‚द‚भिप्रायै}र्य‚द‚र्थ‚प्र‚तिपाद‚नाभिप्रायैः पुरुषैः \textbf{प्र‚{\tiny $_{१}$}‚युज्य‚मानो दृष्टः}‚{\tiny $_{lb}$}‚ स‚म‚य‚कालेऽ\textbf{न्य‚था न दृष्ट इति} विव‚क्षितार्थ‚विप‚र्य‚येण प्र‚युज्य‚मानो न दृष्टः । एते‚{\tiny $_{lb}$}‚नान्व‚य‚व्य‚तिरेकावुक्तौ । \href{http://sarit.indology.info/?cref=pv.3.235}{२३८}
	{\color{gray}{\rmlatinfont\textsuperscript{§~\theparCount}}}
	\pend% ending standard par
      ‚{\tiny $_{lb}$}‚

	  
	  \pstart \leavevmode% starting standard par
	इति य‚थोक्ताभ्या\textbf{न्द‚र्श‚नाद‚र्श‚नाभ्या}न्त‚स्यार्थ‚स्य \textbf{प्र‚तीतिञ्ज‚न‚य‚ति धूमादिव‚त्} ।‚{\tiny $_{lb}$}‚ स एव द‚र्श‚नाद‚र्श‚न‚श‚ब्दाभ्यां सूचितः श‚ब्दार्थ‚योर\textbf{विनाभावाख्यः स‚म्ब‚न्धः । न‚{\tiny $_{lb}$}‚ चात्रा}प्र‚तीतिज‚न‚ने य‚थोक्त‚म‚विनाभावं मुक्त्वा\textbf{न्य‚स्य} व‚स्तुभूत‚स्य‚{\tiny $_{२}$}‚ स‚म्ब‚न्ध‚स्य \textbf{साम‚र्थ्य‚{\tiny $_{lb}$}‚म्प‚श्यामः} । नापि त‚स्य स‚म्ब‚न्ध‚स्य सिद्ध्युपायं सिद्धिनिमित्तं किंचित् प‚श्यामः ।
	{\color{gray}{\rmlatinfont\textsuperscript{§~\theparCount}}}
	\pend% ending standard par
      ‚{\tiny $_{lb}$}‚

	  
	  \pstart \leavevmode% starting standard par
	एव‚न्ताव‚त् स‚म्ब‚न्धिभ्यां स‚म्ब‚न्ध‚भेदाभ्युप‚ग‚मे दोष‚मुक्त्वाऽभेदाभ्युप‚ग‚मेपि‚{\tiny $_{lb}$}‚ दोष‚माह । \textbf{अथे}त्यादि । \textbf{ताभ्या}मिति स‚म्ब‚न्धिभ्यां स‚म्ब‚न्ध‚स्या\textbf{भेदे} स‚ति । \textbf{तावेव}‚{\tiny $_{lb}$}‚ स‚म्ब‚न्धिनावेव श‚ब्दार्थौ केव‚ल‚मिति न स‚म्ब‚न्धो नाम क‚श्चित् । त‚त्त्वान्य‚त्त्व‚र‚हित‚{\tiny $_{lb}$}‚स्त‚र्हि स‚म्ब‚न्धो भ‚वि‚{\tiny $_{३}$}‚ष्य‚तीति चेदाह । \textbf{ना}त इत्यादि । अत‚स्तेत्त्वान्य‚त्व‚विक‚{\tiny $_{lb}$}‚ल्पा\textbf{द‚न्या} नास्ति \textbf{व‚स्तुनो ग‚तिः} ।
	{\color{gray}{\rmlatinfont\textsuperscript{§~\theparCount}}}
	\pend% ending standard par
      ‚{\tiny $_{lb}$}‚

	  
	  \pstart \leavevmode% starting standard par
	\textbf{रूपे}त्यादि विर‚णं ।\edtext{\textsuperscript{*}}{\edlabel{pvsvt_428-1}\label{pvsvt_428-1}\lemma{*}\Bfootnote{\cite{pvb-B} व‚च‚नं ।}} \textbf{रूप‚भेदः} स्व‚भाव‚भेदः । त\textbf{न्निब‚न्ध‚न‚त्वाद्} व्य‚व‚स्थान्त‚{\tiny $_{lb}$}‚र‚स्येति स्व‚भावान्त‚र‚व्य‚व‚स्थान‚स्य । य‚त्तु न भिन्न‚रूपं किन्तु त‚द्रूपं स‚म्ब‚न्धिरूप‚{\tiny $_{lb}$}‚मेवेष्टं स‚म्ब‚न्धाख्य‚म्व‚स्तु । त‚त्त‚देव स्यात् । स‚म्ब‚न्धिस्व‚भाव‚मेव स्यान्नान्य‚त् ।‚{\tiny $_{lb}$}‚ क‚थ‚न्त‚र्ह्य‚न‚योः स‚म्ब‚न्ध इति‚{\tiny $_{४}$}‚ प्र‚तीतिरित्याह । \textbf{ध‚र्म‚भेद‚स्तु} प‚रिक‚ल्पितः \textbf{स्यात् ।‚{\tiny $_{lb}$}‚ पूर्वोक्तेन क्र‚मेणा}न्यापोह‚विहितेन व्यावृत्तिभेद‚स‚माश्र‚येण । \textbf{स च} व्यावृत्तिभेदः ।‚{\tiny $_{lb}$}‚ क‚ल्प‚नाकृत एक‚स्मिन्न‚प्य\textbf{विरुद्धः । न तु व‚स्तुभेद} एक‚त्राविरुद्धः किन्तु विरुद्ध \textbf{एव} ।‚{\tiny $_{lb}$}‚ एक‚स्य प‚र‚मार्थेन नानात्वायोगात् \textbf{न भेदाभेदौ मुक्त्वा व‚स्तुनोन्या ग‚तिः} ।
	{\color{gray}{\rmlatinfont\textsuperscript{§~\theparCount}}}
	\pend% ending standard par
      ‚{\tiny $_{lb}$}‚‚{\tiny $_{lb}$}‚‚{\tiny $_{lb}$}‚\textsuperscript{\textenglish{429/s}}

	  
	  \pstart \leavevmode% starting standard par
	कुत‚स्त\textbf{स्य व‚स्तुनो} रूप‚ल\textbf{क्ष‚ण‚त्वात्} स्व‚भा‚{\tiny $_{५}$}‚व‚ल‚क्ष‚ण‚त्वात् । \textbf{रूप‚स्य चैत‚द्‚{\tiny $_{lb}$}‚ विक‚ल्पान‚तिवृत्तेः} । भेदाभेद‚विक‚ल्पान‚तिवृत्तेः । \textbf{भिन्न‚त्वाद् व‚स्तुरूप‚स्य} श‚ब्दार्थ‚{\tiny $_{lb}$}‚स्व‚रूप‚स्य । न रूप‚श्लेष‚ल‚क्ष‚ण‚स्स\textbf{म्ब‚न्धो} भाविकः किन्तु \textbf{क‚ल्प‚नाकृत एवेत्युक्तं प्राक्} ।‚{\tiny $_{lb}$}‚ पुरुष‚स्य व्य‚व‚हाराभ्यासाद‚संसृष्टाव‚पि संसृष्टौ तौ भासेते त‚द्व‚शात्स‚म्ब‚न्ध‚{\tiny $_{lb}$}‚व्य‚व‚स्थेत्या दिना । \textbf{न हि श्लेष‚ल‚क्ष‚ण‚स्स‚म्ब‚न्धिनोः} प‚र‚स्प‚र‚म्मिश्र‚{\tiny $_{६}$}‚ताल‚क्ष‚णः‚{\tiny $_{lb}$}‚ स‚म्ब‚न्धोऽ\textbf{श्लिष्टेष्व}संसृष्टेषु \textbf{प‚दार्थेषु स‚म्भ‚व‚ति} । श‚ब्दार्थानां रूप‚श्लेषाद‚र्थान्त‚र‚मेव‚{\tiny $_{lb}$}‚ तृतीय‚म्व‚स्तु स‚म्ब‚न्ध इत्याह । \textbf{न चे}त्यादि । \href{http://sarit.indology.info/?cref=pv.3.236}{२३९}
	{\color{gray}{\rmlatinfont\textsuperscript{§~\theparCount}}}
	\pend% ending standard par
      ‚{\tiny $_{lb}$}‚

	  
	  \pstart \leavevmode% starting standard par
	य‚स्मान्निष्प‚न्नं स‚त् त‚द‚र्थान्त‚र‚म्\textbf{प‚राधीनं} क‚थं । स‚म्ब‚न्धाधीनं क‚थ‚म्भ‚वेत् ।‚{\tiny $_{lb}$}‚ स‚म्ब‚न्धाधीन‚श्च स‚म्ब‚न्ध इष्य‚ते द्विष्ठ‚त्वात् । नापि त‚द‚र्थान्त‚र‚म्प‚र‚श्लेष‚रूप‚त्वा‚{\tiny $_{lb}$}‚त्स‚म्ब‚न्धो युज्य‚त इत्याह । \textbf{द्र‚व्य}म्प‚दार्थान्त‚र‚ञ्च‚{\tiny $_{७}$}‚ \textbf{क‚थ‚म‚न्य‚स्य} स‚म्ब‚न्धिनः \textbf{स‚म्ब‚न्धः \leavevmode\ledsidenote{\textenglish{154a/PSVTa}}‚{\tiny $_{lb}$}‚ स्यात्} ।
	{\color{gray}{\rmlatinfont\textsuperscript{§~\theparCount}}}
	\pend% ending standard par
      ‚{\tiny $_{lb}$}‚

	  
	  \pstart \leavevmode% starting standard par
	एतेनार्थान्त‚र‚त्वे स‚म्ब‚न्ध‚स्य स‚म्ब‚न्ध्याश्रित‚त्व‚म्प‚र‚श्लेष‚रूप‚त्व‚ञ्च य‚त्प‚रेणे‚{\tiny $_{lb}$}‚ष्ट‚न्त‚दुभ‚यं निर‚स्तं ।
	{\color{gray}{\rmlatinfont\textsuperscript{§~\theparCount}}}
	\pend% ending standard par
      ‚{\tiny $_{lb}$}‚

	  
	  \pstart \leavevmode% starting standard par
	य‚दि नामार्थान्त‚रं क‚स्मात् प‚राधीनं न भ‚व‚तीत्याह । \textbf{न ही}त्यादि । य‚स्मान्न‚{\tiny $_{lb}$}‚ हि सिद्धं \textbf{स‚त् प‚र‚म‚पेक्ष‚ते} । निष्प‚न्न‚स्य स‚र्व‚निराशंस‚त्वात् । \textbf{अन‚पेक्ष‚त्वेन स्व‚त‚न्त्र‚{\tiny $_{lb}$}‚श्चा}न्य‚स्य \textbf{न स‚म्ब‚न्धः} । ‚{\tiny $_{१}$}‚न चार्थान्त‚रं स‚म्ब‚न्धो युज्य‚ते [।] य‚स्माद् \textbf{द्र‚व्य‚मिति‚{\tiny $_{lb}$}‚ स्व‚भाव उच्य‚ते} प‚दार्थान्त‚र‚मेवोच्य‚ते [।] \textbf{स क‚थं प‚र‚भाव‚स्य} स‚म्ब‚न्धिनोः \textbf{श्लेषः‚{\tiny $_{lb}$}‚ स्यात्} । नैव स्यात् । नापि श्लेष‚हेतुर्भ‚व‚ति । किं कार‚णं । य‚स्मा\textbf{न्न स्व‚भा‚{\tiny $_{lb}$}‚वान्त‚र‚स्य} तृतीय‚स्य \textbf{स‚त्त‚यान्यः} स‚म्ब‚न्धिनोः स्व‚भावः \textbf{श्लिष्टो नामाभूद‚श्लिष्टे}‚{\tiny $_{lb}$}‚नास‚म्ब‚द्धेन स‚म्ब‚न्धाख्येन श्लिष्टो भावः । \textbf{श्लिष्टेन तु} श्लिष्टः \textbf{स्यादिति‚{\tiny $_{२}$}‚ चेत्} ।
	{\color{gray}{\rmlatinfont\textsuperscript{§~\theparCount}}}
	\pend% ending standard par
      ‚{\tiny $_{lb}$}‚

	  
	  \pstart \leavevmode% starting standard par
	\textbf{नै}त‚देवं । किं कार‚णं [।] \textbf{त‚स्यैव} स‚म्ब‚न्धाख्य‚स्य \textbf{ताभ्यां} स‚म्ब‚न्धिभ्यां \textbf{श्लेषा‚{\tiny $_{lb}$}‚‚{\tiny $_{lb}$}‚ \leavevmode\ledsidenote{\textenglish{430/s}}सिद्धेः । य‚स्स}म्ब‚न्धः सृष्टः स‚न् तौ स‚म्ब‚न्धिनौ \textbf{श्लेष‚येत् । त‚द‚य}मित्यादि । \textbf{य‚द्य‚र्था‚{\tiny $_{lb}$}‚न्त‚रेण} तृतीयेन स‚म्ब‚न्धिनौ \textbf{श्लिष्य‚त‚स्त‚दातिप्र‚संगः} । स‚र्वो येन केन‚चित् तृतीयेन‚{\tiny $_{lb}$}‚ श्लिष्टः स्यात् । \textbf{विशेषाभावात्} । न हि स‚म्ब‚न्धाभिम‚त‚स्यान्य‚स्य च प‚दार्थान्त‚रेण‚{\tiny $_{lb}$}‚ स‚म्ब‚द्ध‚त्वे क‚श्चिद् विशेषोस्ति । \href{http://sarit.indology.info/?cref=pv.3.237}{२४०}
	{\color{gray}{\rmlatinfont\textsuperscript{§~\theparCount}}}
	\pend% ending standard par
      ‚{\tiny $_{lb}$}‚

	  
	  \pstart \leavevmode% starting standard par
	\textbf{किञ्च ।‚{\tiny $_{३}$}‚ व‚र्ण्णा} ये \textbf{स‚न्तो} व‚स्तुस‚न्त‚स्ते ताव\textbf{न्निर‚र्थ‚का}स्त‚तो न ते वाच‚कास्तेन‚{\tiny $_{lb}$}‚ न त‚त्र वाच्य‚वाच‚क‚स‚म्ब‚न्ध‚स्य वृत्तिः । प‚दादेस्त‚र्हि सार्थ‚क‚त्वात् त‚त्र स‚म्ब‚न्ध‚वृत्ति‚{\tiny $_{lb}$}‚र्भ‚विष्य‚तीत्याह । \textbf{प‚दादिप‚रिक‚ल्पित}मादिश‚ब्दाद् वाक्य‚म्वाच‚क‚म्भ‚वेत् । त‚स्मिंश्च‚{\tiny $_{lb}$}‚ प‚रिक‚ल्पिते प‚दे वाक्ये वाऽव‚स्तुन्य‚व‚स्तुस्व‚भावे । क‚थं \textbf{स‚म्ब‚न्ध‚स्य} वाच्य‚वाच‚क‚त्व‚{\tiny $_{lb}$}‚ल‚क्ष‚ण‚स्य \textbf{व‚स्तुनो} व‚स्तुस्व‚भाव‚स्य \textbf{क‚थ‚म्प्र‚वृ‚{\tiny $_{४}$}‚त्ति}र्नैव ।
	{\color{gray}{\rmlatinfont\textsuperscript{§~\theparCount}}}
	\pend% ending standard par
      ‚{\tiny $_{lb}$}‚

	  
	  \pstart \leavevmode% starting standard par
	\textbf{वाच‚को ही}त्यादि विव‚र‚णं । वाच‚को हि \textbf{व‚च‚नांगेनो}क्तिनिमित्तेन स‚म्ब‚न्धा‚{\tiny $_{lb}$}‚ख्येन \textbf{त‚द्वान्} स‚म्ब‚न्ध‚वान् \textbf{स्यात् । स‚न्तोपि} विद्य‚माना अपि \textbf{व‚र्ण्णाः} प्र‚त्येक‚म‚र्था‚{\tiny $_{lb}$}‚प्र‚तिपाद‚क‚त्वात् । साहित्याभावात् । नानाप्र‚योक्तृप्र‚युक्तेभ्य‚श्चार्थ‚प्र‚तिप‚त्त्य‚द‚र्श‚ना‚{\tiny $_{lb}$}‚\textbf{द‚वाच‚काः} ।
	{\color{gray}{\rmlatinfont\textsuperscript{§~\theparCount}}}
	\pend% ending standard par
      ‚{\tiny $_{lb}$}‚

	  
	  \pstart \leavevmode% starting standard par
	\textbf{त‚दि}ति त‚स्मा\textbf{न्न तेषु} व‚र्ण्णेषु \textbf{वाच्य‚वाच‚क‚भाव‚स‚म्ब‚न्धो व‚र्त्त‚ते । त‚द्वृत्तौ} तेषु‚{\tiny $_{lb}$}‚ वाच‚{\tiny $_{५}$}‚केषु व‚र्ण्णेषु स‚म्ब‚न्ध‚स्य वृत्तौ स‚त्यां स‚म्ब‚न्ध‚स्य यंद्वाच‚क‚त्वाङ्ग‚त्व‚न्त‚स्य‚{\tiny $_{lb}$}‚ हानिप्र‚स‚ङ्गात् ।
	{\color{gray}{\rmlatinfont\textsuperscript{§~\theparCount}}}
	\pend% ending standard par
      ‚{\tiny $_{lb}$}‚

	  
	  \pstart \leavevmode% starting standard par
	\textbf{क्र‚म‚विशेषेणा}नुपूर्वी विशेषेणैक‚प्र‚योक्तृप्र‚युक्ता \textbf{व‚र्ण्णा एव वाच}कास्त‚तो न‚{\tiny $_{lb}$}‚ य‚थोक्त‚दोष \textbf{इति चेत्} । त‚दुक्तं ।‚{\tiny $_{lb}$}‚ 
	    \pend% close preceding par
	  
	    
	    \stanza[\smallbreak]
	  {\normalfontlatin\large ``\qquad}याव‚न्तो यादृशा ये च य‚द‚र्थ‚प्र‚तिपाद‚ने ।&‚{\tiny $_{lb}$}‚व‚र्ण्णाः प्र‚ज्ञात‚साम‚र्थ्यास्ते त‚थैवाव‚बोध‚काः ॥ \href{http://sarit.indology.info/?cref=\%C5\%9Bv-spho\%E1\%B9\%ADa.69}{स्फोट० ६९}{\normalfontlatin\large\qquad{}"}\&[\smallbreak]
	  
	  
	  
	    \pstart  \leavevmode% new par for following
	    \hphantom{.}
	  ‚{\tiny $_{lb}$}‚ एत‚देव स्प‚ष्ट‚य‚ति ।
	{\color{gray}{\rmlatinfont\textsuperscript{§~\theparCount}}}
	\pend% ending standard par
      ‚{\tiny $_{lb}$}‚
	  \bigskip
	  \begingroup
	
	    
	    \stanza[\smallbreak]
	  {\normalfontlatin\large ``\qquad}तेषान्तु गुण‚भूतानाम‚र्थ‚{\tiny $_{६}$}‚प्र‚त्याय‚नं प्र‚ति ।&‚{\tiny $_{lb}$}‚साहित्य‚मेक‚क‚र्त्त्रादिक्र‚म‚श्चापि विव‚क्षितः ॥ \href{http://sarit.indology.info/?cref=\%C5\%9Bv-spho\%E1\%B9\%ADa.70}{स्फोट० ७०}&‚{\tiny $_{lb}$}‚‚{\tiny $_{lb}$}‚\leavevmode\ledsidenote{\textenglish{431/s}}क‚र्त्त्रैक‚त्व‚निमित्ते च क्र‚मे स‚ति नियाम‚कं ।&‚{\tiny $_{lb}$}‚प्र‚युञ्जान‚स्य य‚त्पूर्व‚म्वृद्धेभ्यः क्र‚म‚द‚र्श‚नं ॥ \href{http://sarit.indology.info/?cref=\%C5\%9Bv-spho\%E1\%B9\%ADa.71}{स्फोट० ७१}&‚{\tiny $_{lb}$}‚युग‚प‚द् दृष्ट‚साम‚र्थ्यान्नैव\edtext{}{\edlabel{pvsvt_431-1}\label{pvsvt_431-1}\lemma{र्थ्यान्नैव}\Bfootnote{\cite{pvb-B} साम‚र्थ्या नैव ।}} श‚क्ताः क्र‚मे य‚था ।&‚{\tiny $_{lb}$}‚भावास्त‚था क्र‚मे श‚क्ता यौग‚प‚द्ये न श‚क्नुयुः ॥ \href{http://sarit.indology.info/?cref=\%C5\%9Bv-spho\%E1\%B9\%ADa.73}{स्फोट० ७३}{\normalfontlatin\large\qquad{}"}\&[\smallbreak]
	  
	  
	  
	  \endgroup
	‚{\tiny $_{lb}$}‚

	  
	  \pstart \leavevmode% starting standard par
	किञ्चार्थ‚प्र‚त्याय‚नं प्र‚ति ।
	{\color{gray}{\rmlatinfont\textsuperscript{§~\theparCount}}}
	\pend% ending standard par
      ‚{\tiny $_{lb}$}‚
	  \bigskip
	  \begingroup
	
	    
	    \stanza[\smallbreak]
	  {\normalfontlatin\large ``\qquad}अव‚श्य‚म्भाविनी नित्यं प्र‚त्यास‚त्तिश्च‚{\tiny $_{७}$}‚ क‚स्य‚चित् ।&\leavevmode\ledsidenote{\textenglish{154b/PSVTa}}‚{\tiny $_{lb}$}‚न ताव‚ता व्य‚पेत‚त्वादित‚रेषाम‚न‚ङ्ग‚ता ॥ \href{http://sarit.indology.info/?cref=\%C5\%9Bv-spho\%E1\%B9\%ADa.83}{स्फोट० ८३}&‚{\tiny $_{lb}$}‚य‚था विस‚र्ज‚नीय‚स्य व्य‚व‚धाने न श‚क्त‚ता ।&‚{\tiny $_{lb}$}‚त‚थैव श‚क्तिर‚न्येषामान‚न्त‚र्ये न विद्य‚ते ॥ \href{http://sarit.indology.info/?cref=\%C5\%9Bv-spho\%E1\%B9\%ADa.85}{स्फोट० ८५}&‚{\tiny $_{lb}$}‚न च य‚त्रैक‚शोऽश‚क्तिस्त‚त्र स‚र्वेषाम\edtext{}{\edlabel{pvsvt_431-2}\label{pvsvt_431-2}\lemma{र्वेषाम}\Bfootnote{\cite{pvb-B} स‚र्वेष्व‚श‚क्त‚ता ।}} श‚क्त‚ता ।&‚{\tiny $_{lb}$}‚र‚थाङ्गानि हि दृश्य‚न्ते श‚क्तानि\edtext{}{\edlabel{pvsvt_431-3}\label{pvsvt_431-3}\lemma{क्तानि}\Bfootnote{\cite{pvb-B} श‚क्तानि दृश्य‚न्ते ।}} व‚ह‚नादिष्विति ॥ \href{http://sarit.indology.info/?cref=\%C5\%9Bv-spho\%E1\%B9\%ADa.86}{स्फोट० ८६}{\normalfontlatin\large\qquad{}"}\&[\smallbreak]
	  
	  
	  
	  \endgroup
	‚{\tiny $_{lb}$}‚

	  
	  \pstart \leavevmode% starting standard par
	नेत्यादिना प‚रिहार‚माह । नैत‚देवं । य‚स्माद् व‚र्ण्णेभ्यः क्र‚म‚स्यान‚र्थान्त‚र‚त्वं‚{\tiny $_{lb}$}‚ स्याद‚र्थान्त‚र‚त्व‚म्वा\edtext{}{\edlabel{pvsvt_431-4}\label{pvsvt_431-4}\lemma{म्वा}\Bfootnote{\cite{pvb-B} च}} । त‚त्र प्र‚थ‚मे प‚क्षे \textbf{क्र‚म‚स्यान‚{\tiny $_{१}$}‚र्थान्त‚र‚त्वेन} कार‚णे\textbf{नाभेद‚क‚त्वाद}‚{\tiny $_{lb}$}‚विशेष‚क‚त्वात् । न हि य‚द्य‚तोन‚र्थान्त‚र‚न्त‚त्त‚त्स्व‚भाव‚स्य भेद‚क‚म्भ‚व‚ति । त‚त‚श्च‚{\tiny $_{lb}$}‚ \textbf{त‚द्रूप‚स्य} व‚र्ण्णात्म‚क‚स्य क्र‚म‚स्य । र‚स इत्य‚त्र य‚द्रूप‚न्त‚तः \textbf{क्र‚मान्त‚रे} सा इत्येत‚स्मि\textbf{न्न‚पि}‚{\tiny $_{lb}$}‚ व‚र्ण्ण‚व\textbf{द‚विशेषात् तुल्या} स्याद‚र्थ\textbf{प्र‚तिप‚तिः} ।
	{\color{gray}{\rmlatinfont\textsuperscript{§~\theparCount}}}
	\pend% ending standard par
      ‚{\tiny $_{lb}$}‚

	  
	  \pstart \leavevmode% starting standard par
	व्य‚तिरिक्त‚स्त‚र्हि व‚र्ण्णेभ्यः क्र‚म इत्य‚त आह \textbf{अर्थान्त‚र‚त्व‚म}पीति । व‚र्ण्णेभ्योर्था‚{\tiny $_{lb}$}‚न्त‚र‚त्व‚म‚पि \textbf{क्र‚म‚स्य‚{\tiny $_{२}$}‚ प‚श्चात् निषेत्स्य‚मान‚त्वात्} । य‚तो न युग‚प‚दुत्प‚न्नानाम्व‚र्ण्णानां\edtext{}{\edlabel{pvsvt_431-5}\label{pvsvt_431-5}\lemma{र्ण्णानां}\Bfootnote{\cite{pvb-B} नास्ति व‚र्णानां ।}}‚{\tiny $_{lb}$}‚ क्र‚मः स‚म्भ‚व‚त्य‚प्र‚तीतेः । अयुग‚प‚दुत्प‚न्नानाम‚पि नार्थान्त‚र‚भूतः क्र‚मोऽयुग‚प‚दुत्पा‚{\tiny $_{lb}$}‚द‚स्यैव क्र‚म‚रूप‚त्वात् । त‚था हि लौकिकाः क्र‚मं क‚थ‚य‚न्तोऽयुग‚प‚दुत्पाद‚मेव क‚थ‚य‚न्ति ।‚{\tiny $_{lb}$}‚ \textbf{त‚स्मा}द‚युग‚प‚दुत्पाद एव क्र‚मः । नापि क्र‚मोऽयुग‚प‚दुत्प‚न्न‚योरेक‚स्य ध‚र्म एक‚प्र‚तीतौ‚{\tiny $_{lb}$}‚ क्र‚म‚स्याप्र‚ती‚{\tiny $_{३}$}‚तेः । नाप्युभ‚य‚ध‚र्मः । एक‚काल‚मुभ‚य‚स्यास‚त्त्वाद‚स‚त‚श्च क‚थं ध‚र्मः ।‚{\tiny $_{lb}$}‚ त‚स्मात् पूर्वाप‚र‚योर्भाव‚योः स्व‚रूप‚मेव क्र‚म उच्य‚ते इति व‚क्ष्य‚ति । पूर्वाप‚र‚रूपे च‚{\tiny $_{lb}$}‚ क्र‚मे त‚थापि न व‚र्ण्णाः क्र‚मेणार्थाधिग‚म‚निमित्त‚म्भ‚वंति । प्र‚त्येक‚म‚र्थाप्र‚तिपाद‚क‚{\tiny $_{lb}$}‚त्वात् । साहित्याभावात् । निय‚त‚क्र‚म‚व‚र्त्तिनाम‚यौग‚प‚द्येन स‚म्भूय कारित्वानुप‚प‚त्तेश्च ।‚{\tiny $_{lb}$}‚ \leavevmode\ledsidenote{\textenglish{432/s}}स्यादेत‚द् [।] य‚था के‚{\tiny $_{४}$}‚व‚ल‚स्य बीज‚स्यांकुरं प्र‚त्य‚कार‚क‚त्वेपि स‚ह‚कारिस‚न्निधाने‚{\tiny $_{lb}$}‚ विशिष्ट‚त्वात् कार‚क‚त्वं । त‚था व‚र्ण्णाः प्र‚त्येक‚म‚स‚म‚र्था अप्यानुपूर्वीविशेषेण‚{\tiny $_{lb}$}‚ विशिष्टा अर्थ‚प्र‚तीतिहेत‚व इति ।
	{\color{gray}{\rmlatinfont\textsuperscript{§~\theparCount}}}
	\pend% ending standard par
      ‚{\tiny $_{lb}$}‚

	  
	  \pstart \leavevmode% starting standard par
	त‚द‚युक्तं । अन्त्य‚स्य हि व‚र्ण्ण‚स्य व‚र्ण्णान्त‚र‚स‚हित‚स्य केव‚ल‚स्य चोच्चार‚णे‚{\tiny $_{lb}$}‚ को विशेषो य‚त्कृताव‚र्थ‚प्र‚तीतिभावाभावौ स्यातां ।
	{\color{gray}{\rmlatinfont\textsuperscript{§~\theparCount}}}
	\pend% ending standard par
      ‚{\tiny $_{lb}$}‚
	    
	    \stanza[\smallbreak]
	  न‚न्व‚य‚मेव विशेषो ये स‚हितास‚हित‚{\tiny $_{५}$}‚ ते ।\&[\smallbreak]
	  
	  
	  ‚{\tiny $_{lb}$}‚

	  
	  \pstart \leavevmode% starting standard par
	स‚त्त्यं । कार्य‚क‚र‚णे हि ख‚लु तेषां साहित्यं । न च ते य‚दा स‚न्त‚स्त‚दा व्याप्रिय‚{\tiny $_{lb}$}‚न्तेर्थ‚प्र‚तीतौ । प्र‚त्येक‚म‚स‚म‚र्थ‚त्वात् । नाप्य‚न्य‚व‚र्ण्ण‚कालेऽस‚त्त्वात् ।
	{\color{gray}{\rmlatinfont\textsuperscript{§~\theparCount}}}
	\pend% ending standard par
      ‚{\tiny $_{lb}$}‚

	  
	  \pstart \leavevmode% starting standard par
	एष त‚र्हि विशेषो येयं क्व‚चित् प्र‚वृत्ता पूर्व‚व‚र्ण्णोप‚ल‚ब्धिः क्व‚चिन्नेति ।
	{\color{gray}{\rmlatinfont\textsuperscript{§~\theparCount}}}
	\pend% ending standard par
      ‚{\tiny $_{lb}$}‚

	  
	  \pstart \leavevmode% starting standard par
	नैत‚द‚पि सारं । न हि प्र‚वृत्तापूर्व‚व‚र्ण्णोप‚ल‚ब्धिर‚न्त्य‚म्व‚र्ण्ण‚म्भेत्तुम‚र्ह‚त्य‚स‚त्त्वात् ।‚{\tiny $_{lb}$}‚ अविशेषे च य‚त्र कार्ये व‚र्ण्णानां प्र‚त्येक‚म‚श‚क्तिस्त‚त्र‚{\tiny $_{६}$}‚ स‚हितानाम‚प्य‚विशेषात् ।‚{\tiny $_{lb}$}‚ अन्धानामिवादित्य‚द‚र्श‚ने ।
	{\color{gray}{\rmlatinfont\textsuperscript{§~\theparCount}}}
	\pend% ending standard par
      ‚{\tiny $_{lb}$}‚

	  
	  \pstart \leavevmode% starting standard par
	तेन न च य‚त्रैक‚शो श‚क्तिरित्यादि निर‚स्तं । र‚थाङ्गानां हि विशेषोत्प‚त्तौ स‚त्यां‚{\tiny $_{lb}$}‚ साहित्याव‚स्थायाम्व‚ह‚नादौ साम‚र्थ्य‚म‚न्य‚था प्र‚त्येक‚व‚त् साहित्येपि साम‚र्थ्य‚न्न स्यात् ।‚{\tiny $_{lb}$}‚ न च प‚र‚स्प‚र‚म्व‚र्ण्णानां कार्य‚कार‚ण‚भावो येन पूर्वे व‚र्ण्णाः पार‚म्प‚र्येणार्थ‚प्र‚तीतौ‚{\tiny $_{lb}$}‚ \leavevmode\ledsidenote{\textenglish{155a/PSVTa}} श‚क्ताः स्युः ।‚{\tiny $_{७}$}‚ नापि पूर्व‚व‚र्ण्ण‚ज‚नित‚संस्कार‚स‚हित‚स्यान्त्य‚स्य व‚र्ण्ण‚स्यार्थ‚प्र‚तीति‚{\tiny $_{lb}$}‚हेतुत्वात् पूर्व‚व‚र्ण्णानां पार‚म्प‚र्येण साम‚र्थ्यं । व‚र्ण्णानुभ‚वाहित‚संस्कार‚स्य व‚र्ण्णे‚{\tiny $_{lb}$}‚ष्वेव स्मृतिहेतुत्वान्नार्थे । न हि ग‚वानुभ‚वाहित‚संस्कारोऽश्वे स्म‚र‚ण‚मुप‚क‚ल्प‚य‚ति ।‚{\tiny $_{lb}$}‚ न च पूर्व‚व‚र्ण्णाहित‚संस्कार‚स‚हितान्त्य‚व‚र्ण्ण‚द‚र्श‚ने स‚त्य‚र्थ‚प्र‚तीतेर्दृष्ट‚वात् त‚द्धेतुत्वं ।‚{\tiny $_{lb}$}‚ संकेताभावेर्थ‚प्र‚तीतेर‚भा‚{\tiny $_{१}$}‚वात् । संकेत‚श्च सामान्य‚विष‚यो न व‚र्ण्ण‚स्व‚ल‚क्ष‚ण‚विष‚य इति‚{\tiny $_{lb}$}‚ क‚थ‚म्व‚र्ण्णाः क्र‚म‚विशेषेण वाच‚काः ।
	{\color{gray}{\rmlatinfont\textsuperscript{§~\theparCount}}}
	\pend% ending standard par
      ‚{\tiny $_{lb}$}‚

	  
	  \pstart \leavevmode% starting standard par
	किञ्च [।] केव‚ल‚स्य व‚र्ण्ण‚स्यार्थाप्र‚तिपाद‚क‚त्वे संस्कार‚स‚हित‚स्यापि न‚{\tiny $_{lb}$}‚ त‚त् स्यात् । विशेषानुत्प‚त्तेः [।] त‚त्क‚थं क‚श्चिद्व‚र्ण्णः साक्षाद‚र्थ‚प्र‚तिपाद‚ने स‚म‚र्थः‚{\tiny $_{lb}$}‚ क‚श्चित् पार‚म्प‚र्येणेत्युच्य‚ते । य‚द‚प्युच्य‚ते ।
	{\color{gray}{\rmlatinfont\textsuperscript{§~\theparCount}}}
	\pend% ending standard par
      ‚{\tiny $_{lb}$}‚
	  \bigskip
	  \begingroup
	
	    
	    \stanza[\smallbreak]
	  {\normalfontlatin\large ``\qquad}इत्थं क्र‚म‚गृहीतानां युग‚प‚द् याथ‚वा स्थितिः ।&‚{\tiny $_{lb}$}‚त‚तः सा कार‚णं नः स्यान्नित्य‚म‚र्थ‚धिय‚म्प्र‚ति ॥\edtext{\textsuperscript{*}}{\edlabel{pvsvt_432-1}\label{pvsvt_432-1}\lemma{*}\Bfootnote{\href{http://sarit.indology.info/?cref=\%C5\%9Bv-spho\%E1\%B9\%ADa.108}{ Śloka, Sphot. 108. }}}{\normalfontlatin\large\qquad{}"}\&[\smallbreak]
	  
	  
	  
	  \endgroup
	‚{\tiny $_{lb}$}‚

	  
	  \pstart \leavevmode% starting standard par
	एवं क्र‚म‚प्र‚तिप‚न्नानाम्व‚र्ण्णानां नित्य‚त्वाद् व्यापित्वाच्चाकाश‚देशे या युग‚प‚त्‚{\tiny $_{lb}$}‚ स्थितिर‚व‚स्थान‚न्त‚देव निमित्त‚म‚र्थ‚प्र‚तीतिं प्र‚तीति ।
	{\color{gray}{\rmlatinfont\textsuperscript{§~\theparCount}}}
	\pend% ending standard par
      ‚{\tiny $_{lb}$}‚

	  
	  \pstart \leavevmode% starting standard par
	त‚द‚युक्तं । प्र‚तीय‚मानो हि श‚ब्दार्थं प्र‚तिपाद‚य‚ति न स‚न्निधान‚मात्रेण । स‚र्व‚{\tiny $_{lb}$}‚‚{\tiny $_{lb}$}‚ ‚{\tiny $_{lb}$}‚ \leavevmode\ledsidenote{\textenglish{433/s}}प‚दार्थ‚प्र‚तिपाद‚न‚प्र‚स‚ङ्गात् । न चैक‚क‚र्त्तृकाणां यौग‚प‚द्यं प्र‚तिभास‚ते । नापि नित्य‚{\tiny $_{३}$}‚‚{\tiny $_{lb}$}‚त्वं व्यापित्वं च युज्य‚त इति व‚क्ष्य‚तीत्य‚सार‚मेत‚त् ।
	{\color{gray}{\rmlatinfont\textsuperscript{§~\theparCount}}}
	\pend% ending standard par
      ‚{\tiny $_{lb}$}‚

	  
	  \pstart \leavevmode% starting standard par
	य‚च्चाप्युच्य‚ते ।‚{\tiny $_{lb}$}‚ 
	    \pend% close preceding par
	  
	    
	    \stanza[\smallbreak]
	  {\normalfontlatin\large ``\qquad}य‚द्वा प्र‚त्य‚क्ष‚तः पूर्वं क्र‚म‚ज्ञातेषु य‚त्प‚रं ।&‚{\tiny $_{lb}$}‚स‚म‚स्त‚व‚र्ण्ण‚विज्ञान‚न्त‚द‚र्थ‚ज्ञान‚कार‚णं ॥&‚{\tiny $_{lb}$}‚त‚त्र ज्ञाने च व‚र्ण्णानां यौग‚प‚द्यं प्र‚तीय‚ते ।&‚{\tiny $_{lb}$}‚नाव‚श्यं यौग‚प‚द्येन प्र‚त्य‚क्ष‚स्थेन त‚द् भ‚वेत् ॥\edtext{\textsuperscript{*}}{\edlabel{pvsvt_433-1}\label{pvsvt_433-1}\lemma{*}\Bfootnote{\href{http://sarit.indology.info/?cref=\%C5\%9Bv-spho\%E1\%B9\%ADa.109-110}{ Śloka, Sphot. 109-10 }}}{\normalfontlatin\large\qquad{}"}\&[\smallbreak]
	  
	  
	  
	    \pstart  \leavevmode% new par for following
	    \hphantom{.}
	  ‚{\tiny $_{lb}$}‚ य‚त्प‚र‚मित्युत्त‚रं पूर्वाप‚र‚प‚राम‚र्शे\textbf{न}\edtext{}{\lemma{र्शे}\Bfootnote{? ण}} स‚म‚स्त‚व‚र्ण्ण‚विष‚यं विज्ञान‚न्त‚द‚र्थ‚प्र‚तीति‚{\tiny $_{lb}$}‚निमित्तं । त‚{\tiny $_{४}$}‚द्भ‚वेदित्य‚र्थ‚प्र‚तिपाद‚न‚म्भ‚वेद‚न्येनापि यौग‚प‚द्य‚ज्ञानेनार्थ‚प्र‚तिपाद‚न‚म्भ‚वे‚{\tiny $_{lb}$}‚दित्य‚र्थ इति ।
	{\color{gray}{\rmlatinfont\textsuperscript{§~\theparCount}}}
	\pend% ending standard par
      ‚{\tiny $_{lb}$}‚

	  
	  \pstart \leavevmode% starting standard par
	एत‚द‚प्य‚युक्तं । क्र‚मो हि प्र‚योक्तृप्र‚युक्तो न यौग‚प‚द्यं । प्र‚योक्तृप्र‚युक्ताव‚{\tiny $_{lb}$}‚स्थेभ्य‚श्च व‚र्ण्णेभ्योर्थ‚प्र‚तीतिरिति न यौग‚प‚द्याद‚र्थ‚प्र‚तीतिः स्यात् । स‚क्र‚माणाञ्च‚{\tiny $_{lb}$}‚ व‚र्ण्णानां यौग‚प‚द्येन ग्र‚ह‚णे भ्रान्त‚त्व‚प्र‚संगात् । न च तेषां यौग‚प‚द्य‚म‚स्ति नित्य‚त्वा‚{\tiny $_{५}$}‚‚{\tiny $_{lb}$}‚योगादिति ।
	{\color{gray}{\rmlatinfont\textsuperscript{§~\theparCount}}}
	\pend% ending standard par
      ‚{\tiny $_{lb}$}‚
	  \bigskip
	  \begingroup
	
	    
	    \stanza[\smallbreak]
	  {\normalfontlatin\large ``\qquad}चित्र‚रूपां च तां बुद्धिं स‚द‚स‚द्व‚र्ण्ण‚गोच‚रां ।&‚{\tiny $_{lb}$}‚केचिदाहुर्य‚या व‚र्ण्णो गृह्य‚तेऽन्त्यः प‚दे प‚दे ॥ \href{http://sarit.indology.info/?cref=\%C5\%9Bv-spho\%E1\%B9\%ADa.11}{स्फोट० ११}{\normalfontlatin\large\qquad{}"}\&[\smallbreak]
	  
	  
	  
	  \endgroup
	‚{\tiny $_{lb}$}‚

	  
	  \pstart \leavevmode% starting standard par
	प्र‚तिप‚द‚म‚न्त्यो व‚र्ण्णो य‚या बुद्ध्या गृह्य‚ते सा स‚न्निहितास‚न्निहित‚व‚र्ण्ण‚विष‚य‚त्वेन‚{\tiny $_{lb}$}‚ स्म‚र‚ण‚प्र‚त्य‚क्ष‚रूपाभ्यामुभ‚य‚रूपेति केचिदाहुः ।
	{\color{gray}{\rmlatinfont\textsuperscript{§~\theparCount}}}
	\pend% ending standard par
      ‚{\tiny $_{lb}$}‚

	  
	  \pstart \leavevmode% starting standard par
	त‚द‚प्य‚युक्त‚म् [।] एक‚स्य ज्ञान‚स्य प्र‚त्य‚क्षाप्र‚त्य‚क्ष‚रूप‚विरोधात् । न च प्र‚त्य‚क्ष‚{\tiny $_{lb}$}‚मेवै‚{\tiny $_{६}$}‚कं स‚द‚स‚द्व‚र्ण‚विष‚यं । अभाव‚विष‚य‚त्व‚विरोधात् । नापि स्मृतिरूपं स‚न्निहित‚{\tiny $_{lb}$}‚विष‚य‚त्वेनानिष्ट‚त्वात् । अत एव च प‚दादिग्राह‚क‚ज्ञानं क‚ल्पित‚विष‚यं स्यादिति ।
	{\color{gray}{\rmlatinfont\textsuperscript{§~\theparCount}}}
	\pend% ending standard par
      ‚{\tiny $_{lb}$}‚

	  
	  \pstart \leavevmode% starting standard par
	अन्ये त्व‚न्त्य‚व‚र्ण्ण‚प‚रिज्ञाने स‚ति पूर्व‚व‚र्ण्णानुभ‚वाहित‚संस्कार‚प्र‚बोध‚कारितं स्म‚र‚णं‚{\tiny $_{lb}$}‚ स‚र्व‚व‚र्णेष्व‚र्थ‚प्र‚त्याय‚क‚माच‚क्ष‚ते । त‚दाह ।
	{\color{gray}{\rmlatinfont\textsuperscript{§~\theparCount}}}
	\pend% ending standard par
      ‚{\tiny $_{lb}$}‚

	  
	  \pstart \leavevmode% starting standard par
	अन्त्य‚व‚र्ण्णे हि विज्ञाने स‚र्व‚संस्कार‚{\tiny $_{७}$}‚कारितं स्म‚र‚णं यौग‚प‚द्येन स‚र्वेष्व‚न्ये \leavevmode\ledsidenote{\textenglish{155b/PSVTa}}‚{\tiny $_{lb}$}‚ प्र‚च‚क्ष‚ते ॥ क‚थं क्र‚मेणानुभूतानां युग‚प‚त्स्म‚र‚ण‚मिति चेदाह ।
	{\color{gray}{\rmlatinfont\textsuperscript{§~\theparCount}}}
	\pend% ending standard par
      ‚{\tiny $_{lb}$}‚
	  \bigskip
	  \begingroup
	
	    
	    \stanza[\smallbreak]
	  {\normalfontlatin\large ``\qquad}स‚र्वेषु चैव‚म‚र्थेषु मान‚सं स‚र्व‚वादिनां ।&‚{\tiny $_{lb}$}‚इष्टं स‚मुच्च‚य‚ज्ञानं क्र‚म‚ज्ञान‚षु स‚त्स्व‚पि ॥ \href{http://sarit.indology.info/?cref=\%C5\%9Bv-spho\%E1\%B9\%ADa.113}{स्फो० ११३}&‚{\tiny $_{lb}$}‚तेन श्रोत्र‚म‚नोभ्यां च क्र‚माद् व‚र्ण्णेषु य‚द्य‚पि ।&‚{\tiny $_{lb}$}‚पूर्वं ज्ञानं प‚र‚स्तात्तु युग‚प‚त् स्म‚र‚ण‚म्भ‚वेत् ॥&‚{\tiny $_{lb}$}‚‚{\tiny $_{lb}$}‚‚{\tiny $_{lb}$}‚\leavevmode\ledsidenote{\textenglish{434/s}}त‚दारुढास्त‚तो व‚र्ण्णा न दूरेर्थाव‚बोध‚नात् ।&‚{\tiny $_{lb}$}‚श‚ब्दाद‚र्थ‚म‚तिस्तेन लौ‚{\tiny $_{१}$}‚किकैर‚भिधीय‚त इति ॥\edtext{\textsuperscript{*}}{\edlabel{pvsvt_434-2}\label{pvsvt_434-2}\lemma{*}\Bfootnote{\href{http://sarit.indology.info/?cref=\%C5\%9Bv}{ Ślokavārtika } \href{http://sarit.indology.info/?cref=\%C5\%9Bv-spho\%E1\%B9\%ADa}{Ślok Sphot.}}}{\normalfontlatin\large\qquad{}"}\&[\smallbreak]
	  
	  
	  
	  \endgroup
	‚{\tiny $_{lb}$}‚

	  
	  \pstart \leavevmode% starting standard par
	एत‚द‚प्य‚युक्तं । एक‚क‚र्त्तृप्र‚युक्तानामेवार्थ‚प्र‚तिपाद‚क‚त्वेनायुग‚प‚द्व‚र्त्तिनामेवार्थ‚प्र‚{\tiny $_{lb}$}‚तिपाद‚क‚त्वात् । न च स्म‚र‚ण‚विष‚याणां व‚र्ण्णानां यौग‚प‚द्य‚म‚ध्य‚व‚सीय‚ते । निय‚त‚क्र‚{\tiny $_{lb}$}‚माणामेव स्म‚र्य‚माण‚त्वात् । नापि स्मृत्या व‚र्ण्ण‚स्व‚ल‚क्ष‚ण‚ग्र‚ह‚ण‚म्प्र‚त्य‚क्ष‚व‚त् स्प‚ष्ट‚{\tiny $_{lb}$}‚प्र‚तिभासाभावात् । एक‚स्य च स्प‚ष्टास्प‚ष्टानेकाकारायोगाच्च । केव‚लं स्म‚र‚णेना‚{\tiny $_{lb}$}‚स्प‚ष्ट‚स्व‚भावा‚{\tiny $_{२}$}‚नाम्व‚र्ण्णानां स्वाकार‚रूपाणां बाह्य‚व‚र्ण्णाभेदेनाध्य‚व‚सायात् बाह्य‚{\tiny $_{lb}$}‚व‚र्ण्णानामेव वाच‚क‚त्व‚मुच्य‚ते । अवाह्येषु च व‚र्ण्णेषु बाह्य‚व‚र्ण्णाध्य‚व‚सायेन प‚दादि‚{\tiny $_{lb}$}‚प‚रिक‚ल्पित‚म‚स्माभिरिष्य‚ते ।
	{\color{gray}{\rmlatinfont\textsuperscript{§~\theparCount}}}
	\pend% ending standard par
      ‚{\tiny $_{lb}$}‚

	  
	  \pstart \leavevmode% starting standard par
	एवं मी मां स क प‚क्षे व‚र्ण्णानां वाच‚क‚त्वे निर‚स्ते प‚दाद्य‚पि निर‚स्त‚मेव ।‚{\tiny $_{lb}$}‚ व‚र्ण्णादिव्य‚तिरेकेण प‚दादेर‚भावात् । त‚दुक्तं ।
	{\color{gray}{\rmlatinfont\textsuperscript{§~\theparCount}}}
	\pend% ending standard par
      ‚{\tiny $_{lb}$}‚
	    
	    \stanza[\smallbreak]
	  न व‚र्ण्ण‚व्य‚तिरेकेण प‚द‚म‚न्य‚द्धि विद्य‚ते ।&‚{\tiny $_{lb}$}‚वाक्य‚म्व‚र्ण्ण‚प‚दाभ्यां च व्य‚तिरि‚{\tiny $_{३}$}‚क्त‚न्न किञ्च‚नेति ॥\&[\smallbreak]
	  
	  
	  ‚{\tiny $_{lb}$}‚

	  
	  \pstart \leavevmode% starting standard par
	संप्र‚ति वैयाक‚र‚णानां व‚र्ण्णादिव्य‚तिरिक्त‚म्प‚दादि निराक‚र्त्तुमाह ।‚{\tiny $_{lb}$}‚ \textbf{त‚द‚स‚तीति} । त‚दित्युप‚न्यासे । त‚स्माद‚र्थे वा । \textbf{अस‚ति व‚र्ण्णानाम्वाच‚क‚त्वे प‚दादि‚{\tiny $_{lb}$}‚ वाच‚कं स्यात्त‚च्च} प‚दादि \textbf{न किञ्चित्} । किं कार‚ण‚म् [।] व‚र्ण्णेभ्य‚स्त‚स्य प‚दादे‚{\tiny $_{lb}$}‚\textbf{र्व्य‚तिरेकाव्य‚तिरेक‚योर्विरोधात्} । व्य‚तिरेके भेदेनोप‚ल‚म्भः स्याद् दृश्य‚स्य ।‚{\tiny $_{lb}$}‚ अदृश्य‚त्वेप्य‚वाच‚क‚त्व‚म‚गृहीत‚स्य ज्ञाप‚क‚त्वायोगात् ।‚{\tiny $_{४}$}‚ अव्य‚तिरेकेपि व‚र्ण्ण‚व‚{\tiny $_{lb}$}‚देवावाच‚क‚त्व‚प्र‚संगः ।
	{\color{gray}{\rmlatinfont\textsuperscript{§~\theparCount}}}
	\pend% ending standard par
      ‚{\tiny $_{lb}$}‚

	  
	  \pstart \leavevmode% starting standard par
	य‚त एव\textbf{न्त‚स्मात् । इन्द्रिय‚विज्ञान‚विशेषः} क्र‚म‚व‚र्ण्ण‚ग्राहिप‚टीयः श्रोत्र‚विज्ञा‚{\tiny $_{lb}$}‚न‚न्त‚द‚नु\textbf{ब‚न्धी} त‚द‚नुभ‚व‚द्वारायातः । स‚भाग‚वास‚ना स‚जातीय‚विक‚ल्प‚श‚क्तिरुपादानं‚{\tiny $_{lb}$}‚ य‚स्य विक‚ल्प‚स्य स त‚थोक्तः । \textbf{स‚भाग‚वास‚नोपादान}श्चासौ \textbf{विक‚ल्प‚श्च त‚स्य} प्र‚ति‚{\tiny $_{lb}$}‚भास‚विभ्र‚मः । प्र‚तिभास\textbf{भ्रान्तिरेव प‚दं} वाक्यं \textbf{चैकाव‚भासि‚{\tiny $_{५}$}‚ मिथ्यैव} ।
	{\color{gray}{\rmlatinfont\textsuperscript{§~\theparCount}}}
	\pend% ending standard par
      ‚{\tiny $_{lb}$}‚‚{\tiny $_{lb}$}‚‚{\tiny $_{lb}$}‚‚{\tiny $_{lb}$}‚\textsuperscript{\textenglish{435/s}}

	  
	  \pstart \leavevmode% starting standard par
	एत‚दुक्त‚म्भ‚व‚ति । क्र‚म‚व‚र्ण्णानुभ‚व‚पृष्ठ‚भावि म‚नोविज्ञान‚न्तान् व‚र्ण्णान् प‚दादि‚{\tiny $_{lb}$}‚रूप‚त‚यैक‚स्व‚भावान‚ध्य‚व‚स्य‚तीति प‚दादिप‚रिक‚ल्पितं मिथ्यैव ।
	{\color{gray}{\rmlatinfont\textsuperscript{§~\theparCount}}}
	\pend% ending standard par
      ‚{\tiny $_{lb}$}‚

	  
	  \pstart \leavevmode% starting standard par
	न‚नु व‚र्ण्णानाम्भिन्नानामेवानुभ‚वात् क‚थ‚मेक‚प‚दाद्य‚व‚भासी विक‚ल्प उत्प‚द्य‚ते ।‚{\tiny $_{lb}$}‚ उत्प‚द्य‚ते च । त‚स्माद् व‚र्ण्णेष्वेक‚प‚दाद्य‚नुभ‚वेन भाव्य‚मिति ।
	{\color{gray}{\rmlatinfont\textsuperscript{§~\theparCount}}}
	\pend% ending standard par
      ‚{\tiny $_{lb}$}‚

	  
	  \pstart \leavevmode% starting standard par
	नैष दोषः । प्र‚तिपाद‚को हि संकेत‚काले व‚र्ण्ण‚क्र‚म‚मेक‚प‚दादिरूप‚{\tiny $_{६}$}‚त‚या प्र‚ति‚{\tiny $_{lb}$}‚प‚न्न‚मेव प‚रं प्र‚त्येक‚मिदं प‚दादीति संकेत‚य‚ति । त‚दा च प‚र‚स्यापि त‚त्र व‚र्ण्ण‚क्र‚मे‚{\tiny $_{lb}$}‚ एक‚पंदाध्यारोपिका बुद्धिरुत्प‚द्य‚ते । त‚स्य चैक‚प‚दाद्य‚ध्यारोपितैकाकारानुभ‚वाहित‚{\tiny $_{lb}$}‚संस्कार‚स्य पुंसो व्य‚व‚हार‚कालेपि व‚र्ण्ण‚क्र‚म‚श्र‚व‚णादेक‚मिदं प‚द‚म्वाक्य‚म्वेत्येकाकार‚स्य‚{\tiny $_{lb}$}‚ विक‚ल्प‚स्योत्प‚त्तिर्भ‚व‚ति । एवं पूर्व‚पूर्व‚श्रोतॄणां पूर्व‚पूर्व‚व‚क्तृभ्यो‚{\tiny $_{७}$}‚ व‚र्ण्ण‚क्र‚मेष्वेक‚त्वारो- \leavevmode\ledsidenote{\textenglish{156a/PSVTa}}‚{\tiny $_{lb}$}‚ पेण प्र‚तीतिर्भ‚व‚तीत्य‚नादित्वं प‚दादिव्य‚व‚हार‚स्य ।
	{\color{gray}{\rmlatinfont\textsuperscript{§~\theparCount}}}
	\pend% ending standard par
      ‚{\tiny $_{lb}$}‚

	  
	  \pstart \leavevmode% starting standard par
	अत एवोच्य‚ते । अनादिस‚भाग‚वास‚नो विक‚ल्प‚प्र‚तिभास‚विभ्र‚मः प‚दं वाक्यं‚{\tiny $_{lb}$}‚ चैकाव‚भासि मिथ्यैवेति । मिथ्यात्वं च भिन्नानाम्व‚र्ण्णानामेक‚प‚दादिरूप‚त‚या स्म‚{\tiny $_{lb}$}‚र‚ण‚ज्ञाने प्र‚तिभास‚नात् । ताव‚त‚श्\textbf{चैकानेक‚त्व‚यो}र्विरोधे\textbf{नायोगात्} ।
	{\color{gray}{\rmlatinfont\textsuperscript{§~\theparCount}}}
	\pend% ending standard par
      ‚{\tiny $_{lb}$}‚

	  
	  \pstart \leavevmode% starting standard par
	अथ स्यादेक‚मेव प‚दादि प्र‚त्य‚क्ष‚ग्राह्य‚न्त‚त्क‚थं मिथ्येति ।
	{\color{gray}{\rmlatinfont\textsuperscript{§~\theparCount}}}
	\pend% ending standard par
      ‚{\tiny $_{lb}$}‚

	  
	  \pstart \leavevmode% starting standard par
	त‚द‚युक्तं । य‚स्मान्न ह्येकं प‚दादि । किं कार‚णं । \textbf{अनेक‚या} व‚र्ण्ण‚क्र‚म‚ग्राहिण्या‚{\tiny $_{lb}$}‚ \textbf{बुद्ध्या} क्र‚मेण \textbf{ग्र‚ह‚णायोगात्} । एक‚त्वे ह्ये\textbf{क‚यैव} बुद्ध्या स‚कृद् \textbf{गृह्येत} । न त्वेक‚यैव‚{\tiny $_{lb}$}‚ बुद्ध्या प‚दादेर्ग्र‚ह‚ण‚मिति चेदाह । \textbf{न त‚देक‚ये}त्यादि । \textbf{त‚त्} प‚दादि । \textbf{नैक‚या} बुद्ध्या‚{\tiny $_{lb}$}‚ \textbf{ग्राह्यं} । किं कार‚णं । \textbf{व‚र्ण्णानुक्र‚मेण} व‚र्ण्ण‚प‚रिपाट्या प‚द‚वाक्य‚यो\textbf{र्ग्र‚ह‚णा‚{\tiny $_{२}$}‚त्} ।
	{\color{gray}{\rmlatinfont\textsuperscript{§~\theparCount}}}
	\pend% ending standard par
      ‚{\tiny $_{lb}$}‚

	  
	  \pstart \leavevmode% starting standard par
	एक‚व‚र्ण्ण‚रूप‚न्त‚र्हि प‚द‚मेक‚बुद्धिग्राह्य‚म्भ‚विष्य‚तीत्य‚त आह । \textbf{एक‚व‚र्ण्णे}त्यादि ।‚{\tiny $_{lb}$}‚ एक‚व‚र्ण‚निष्प‚त्तिकाले\textbf{प्य‚नेक‚बुद्धिव्य‚तिक्र‚मा}न्नैक‚व‚र्ण्णः । त‚था हि भागित्युक्तेऽर्द्ध‚मा‚{\tiny $_{lb}$}‚त्राकालो निर‚च्को ग‚कारः प्र‚तीय‚ते । साच्क‚स्तु मात्राकालः प्र‚तीय‚त इति क‚थ‚मे‚{\tiny $_{lb}$}‚क‚व‚र्ण्ण‚रूपं प‚द‚म्विद्य‚ते य‚देक‚बुद्धिग्राह्यं स्यात् । तेन य‚दुच्य‚ते । स‚क‚ल‚मेव \textbf{गृह्णा}ति ।
	{\color{gray}{\rmlatinfont\textsuperscript{§~\theparCount}}}
	\pend% ending standard par
      ‚{\tiny $_{lb}$}‚
	  \bigskip
	  \begingroup
	
	    
	    \stanza[\smallbreak]
	  {\normalfontlatin\large ``\qquad}अल्पीय‚सापि य‚त्नेन श‚{\tiny $_{३}$}‚ब्द‚मुच्च‚रित‚म्म‚तिः ।&‚{\tiny $_{lb}$}‚य‚दि वा नैव गृह्णाति व‚र्ण्ण‚म्वा स‚क‚लं स्फुटं ।&‚{\tiny $_{lb}$}‚पृथ‚क् च नोप‚ल‚भ्य‚न्ते व‚र्ण्ण‚स्याव‚य‚वाः क्व‚चिदिति [।]\edtext{}{\edlabel{pvsvt_435-1}\label{pvsvt_435-1}\lemma{चिदिति}\Bfootnote{\href{http://sarit.indology.info/?cref=\%C5\%9Bv-spho\%E1\%B9\%ADa.10-11}{ Śloka, Sphoṭ. 10, 11 }}}{\normalfontlatin\large\qquad{}"}\&[\smallbreak]
	  
	  
	  
	  \endgroup
	‚{\tiny $_{lb}$}‚

	  
	  \pstart \leavevmode% starting standard par
	त‚द‚पास्तं । य‚थोक्तेन न्यायेन साव‚य‚व‚त्वाद् व‚र्ण्ण‚स्य । न चैक‚या बुद्ध्या क्र‚म‚व‚तां‚{\tiny $_{lb}$}‚ व‚र्ण्ण‚भागानां ग्र‚ह‚णं \textbf{क्ष‚णिक‚त्वाद् बुद्धीनां} ।
	{\color{gray}{\rmlatinfont\textsuperscript{§~\theparCount}}}
	\pend% ending standard par
      ‚{\tiny $_{lb}$}‚\textsuperscript{\textenglish{436/s}}

	  
	  \pstart \leavevmode% starting standard par
	स्यादेत‚द् [।] याव‚ता कालेन व‚र्ण्ण‚निष्प‚त्तिस्ताव‚त्काल एकः क्ष‚ण‚स्त‚त एक‚या‚{\tiny $_{lb}$}‚ बुद्ध्या प‚द‚स्य ग्र‚ह‚ण‚म्भ‚विष्य‚तीत्य‚त आह । \textbf{क्ष‚ण‚{\tiny $_{४}$}‚स्ये}त्यादि । याव‚ता कालेनैकः‚{\tiny $_{lb}$}‚ प‚र‚माणुः प‚र‚माण्व‚न्त‚र‚म‚तिक्राम‚ति ताव‚त्काल‚त्वात् क्ष‚ण‚स्य । विभाग‚र‚हितः‚{\tiny $_{lb}$}‚ कालः स \textbf{चैक‚प‚र‚माण्व‚तिक्र‚म‚काल} एव युज्य‚ते । य‚थोक्तात्कालादा\textbf{धिक्ये} क्ष‚ण‚{\tiny $_{lb}$}‚स्याभ्युप‚ग‚म्य‚माने । \textbf{विभाग‚व‚तः} श‚क्य‚विभाग‚स्य क्ष‚ण‚स्य काल‚प\textbf{र्य‚व‚सानायोगात्} ।‚{\tiny $_{lb}$}‚ तेनैक‚स्याप्य‚तिनिष्कृष्ट‚स्य व‚र्ण्ण‚स्यानेक‚क्ष‚णेन \textbf{निष्प‚त्तिः} । किं कार‚णं । \textbf{अनेके}‚{\tiny $_{lb}$}‚त्यादि । \textbf{अनेक‚{\tiny $_{५}$}‚स्या}णोर्व्य‚त्य‚यो व्य‚तिक्र‚मो य‚स्मिन्निमेषे सो\textbf{नेकाणुव्य‚त्य‚यो निमेषः} ।‚{\tiny $_{lb}$}‚ तेन तुल्य\textbf{काल‚त्वाद}न्त्य‚स्य \textbf{निष्कृष्ट‚स्या}प्याकारादे\textbf{र्व‚र्ण्ण‚स्य} प‚रिस‚माप्तेः । त‚स्मान्नैक‚{\tiny $_{lb}$}‚व‚र्ण्ण‚रूपं प‚द‚मेक‚बुद्धिग्राह्यं । नाप्य‚नेकात्म‚क‚मेक‚प‚दं स्मृतिग्राह्यं । किं कार‚णं [।]‚{\tiny $_{lb}$}‚ \textbf{य‚थानुभ‚वं स्म‚र‚णात्} । य‚थानुभ‚वो व‚र्ण्णानाम‚नुक्र‚मेण त‚था स्मृतिर‚पि त‚त्र क्र‚म‚{\tiny $_{lb}$}‚भाविन्येवेति \textbf{स्मृतिर‚पि त‚त्कालैव} । स एवा‚{\tiny $_{६}$}‚नुभ‚व‚क्र‚म‚कालोस्या इति कृत्वा ।‚{\tiny $_{lb}$}‚ एत‚च्चान्यां प्र‚वृत्तिम‚धिकृत्योक्त‚म‚भ्यास‚व‚त्यान्तु प्र‚वृत्तौ क्र‚मेणानुभूतानाम‚पि‚{\tiny $_{lb}$}‚ व‚र्ण्णानां य‚द्य‚पि युग‚प‚त्स्म‚र‚ण‚म्भ‚व‚ति त‚थाप्य‚नु\textbf{भ‚व‚स्म‚र‚णानुक्र‚म‚योर्विशेषानुप‚ल‚क्ष‚{\tiny $_{lb}$}‚णाच्च नैक‚म्प‚दादि} । तेनाय‚म‚र्थः [।] अनुभ‚वे योय‚म्व‚र्ण्णानाम‚नुक्र‚मः प्र‚तिभास‚ते‚{\tiny $_{lb}$}‚ \leavevmode\ledsidenote{\textenglish{156b/PSVTa}} स्म‚र‚णे च यो व‚र्ण्णानुक्र‚मः प्र‚तिभास‚ते त‚योर्विशेषो भेदो नोप‚{\tiny $_{७}$}‚ल‚क्ष्य‚तेऽतः क‚थ‚मेकं‚{\tiny $_{lb}$}‚ प‚दाद्येक‚बुद्धिग्राह्य‚मुच्य‚ते ।
	{\color{gray}{\rmlatinfont\textsuperscript{§~\theparCount}}}
	\pend% ending standard par
      ‚{\tiny $_{lb}$}‚

	  
	  \pstart \leavevmode% starting standard par
	नाप्य‚नेक‚मेव प‚दादि । किं कार‚ण‚म् [।] \textbf{अभेद‚प्र‚तिभास‚त्वात् बुद्धेः} । अभेदे‚{\tiny $_{lb}$}‚नैक‚त्वेन प्र‚तिभास‚नाद् बुद्धेः प‚द‚वाक्याकारायाः त‚था हि प‚दे वाक्ये चोच्चारिते‚{\tiny $_{lb}$}‚ एक‚मिदं प‚दं वाक्यं चेति लोक‚स्य म‚तिर्भ‚व‚ति ।
	{\color{gray}{\rmlatinfont\textsuperscript{§~\theparCount}}}
	\pend% ending standard par
      ‚{\tiny $_{lb}$}‚

	  
	  \pstart \leavevmode% starting standard par
	तेन य‚दुच्य‚ते ।‚{\tiny $_{lb}$}‚ 
	    \pend% close preceding par
	  
	    
	    \stanza[\smallbreak]
	  {\normalfontlatin\large ``\qquad}शैध्र्याद‚ल्पान्त‚र‚त्वाच्च गोश‚ब्दे सा भ‚वेद‚पि ।&‚{\tiny $_{lb}$}‚देव‚द‚त्तादिश‚ब्देषु स्फुटो भेदः प्र‚तीय‚ते इति \href{http://sarit.indology.info/?cref=\%C5\%9Bv-spho\%E1\%B9\%ADa.121}{स्फोट० १२१}{\normalfontlatin\large\qquad{}"}\&[\smallbreak]
	  
	  
	  
	    \pstart  \leavevmode% new par for following
	    \hphantom{.}
	  ‚{\tiny $_{lb}$}‚ त‚द‚पास्तं ।‚{\tiny $_{१}$}‚ व‚र्ण्णानुभ‚वोत्त‚र‚काल‚मेक‚प‚दाध्यारोपिकाया बुद्धेरुत्प‚त्तेः । \textbf{त‚द‚ने‚{\tiny $_{lb}$}‚क‚त्व‚स्य} प‚दाद्य‚नेक‚त्व‚स्योत्त‚र‚त्र \textbf{निषेत्स्य‚मान‚त्वाच्च} ।
	{\color{gray}{\rmlatinfont\textsuperscript{§~\theparCount}}}
	\pend% ending standard par
      ‚{\tiny $_{lb}$}‚‚{\tiny $_{lb}$}‚‚{\tiny $_{lb}$}‚\textsuperscript{\textenglish{437/s}}

	  
	  \pstart \leavevmode% starting standard par
	नानेक‚मेव प‚दादि । \textbf{त‚दि}ति त‚स्माद् [।] एकानेक‚त्वेन प्र‚तिभास‚नादेकानेक‚यो‚{\tiny $_{lb}$}‚र्विरोधेनायोगात् प‚दादि \textbf{न व‚स्तु} । य‚द्वा त‚त्प‚दादि न व‚स्तु । एकानेक‚त्वायोगादिति‚{\tiny $_{lb}$}‚ भावः । किं कार‚णं [।] त‚स्य व‚स्तुनः \textbf{एत‚द्विक‚ल्पान‚तिक्र‚मात्} । य‚स्माद् व‚स्त्वेक‚रूपं‚{\tiny $_{२}$}‚‚{\tiny $_{lb}$}‚ वा स्याद‚नेक‚रूपं वा क‚दाचित् स्यान्न तूभ‚य‚रूपं विरोधात् । \textbf{व‚स्तु च} श‚ब्दार्थ\textbf{स‚म्ब‚न्धः}‚{\tiny $_{lb}$}‚ प‚रेणेष्टः \textbf{स क‚थ‚न्त‚दाश्र‚यः स्यात्} । अव‚स्तुभूत‚प‚द‚वाक्याश्र‚यः स्यात् । त‚त्प‚द‚वाक्य‚{\tiny $_{lb}$}‚माश्र‚योस्येति विग्र‚हः । किं कार‚णं । अस‚त्त्वेन प‚दादे\textbf{राश्र‚य‚णीय‚स्यायोगात्} । एव‚मि‚{\tiny $_{lb}$}‚त्याश्र‚य‚णीया\textbf{भावेऽनाश्रितः} स‚म्ब‚न्धः \textbf{स्यात् । त‚था चा}नाश्रित‚त्वा\textbf{द‚स‚म्ब‚न्धः} स‚म्ब‚न्धः‚{\tiny $_{lb}$}‚ \textbf{स्यात्} । स‚म्ब‚न्धिपा‚{\tiny $_{३}$}‚र‚त‚न्त्र्याभावात् ।
	{\color{gray}{\rmlatinfont\textsuperscript{§~\theparCount}}}
	\pend% ending standard par
      ‚{\tiny $_{lb}$}‚

	  
	  \pstart \leavevmode% starting standard par
	य‚त एव\textbf{न्त‚स्मान्न स्वाभाविको}पौरुषेयः \textbf{श‚ब्दार्थ‚योस्स‚म्ब‚न्धः} । किन्तु पौरु‚{\tiny $_{lb}$}‚षेय एव स‚म्ब‚न्धः । य‚स्मात् । \textbf{त‚द‚भिप्राय‚स्या}र्थ‚प्र‚तिपाद‚नाभिप्राय‚स्य यः \textbf{प्र‚योगान्तः}‚{\tiny $_{lb}$}‚ प‚रिस्प‚न्दादिः । त‚स्मा\textbf{दुत्प‚न्नः} श‚ब्द एत‚त् स्व‚द‚र्श‚नेनोक्तं । \textbf{अभिव्य‚क्तो वा श‚ब्द}‚{\tiny $_{lb}$}‚ एत‚त्प‚राभिप्रायेणोक्तं । \textbf{त‚द‚व्य‚भिचारी} । अर्थ‚प्र‚तिपाद‚नाव्य‚भिचारीति कृत्वा \textbf{त‚त्त्व}‚{\tiny $_{lb}$}‚म‚र्थ‚प्र‚तिपाद‚{\tiny $_{४}$}‚नाभिप्राय‚कार्य‚त्व\textbf{म‚स्य} श‚ब्द‚स्य \textbf{स‚म्ब‚न्धः} । अर्थ‚प्र‚तिपाद‚नाभिप्रायेण‚{\tiny $_{lb}$}‚ श‚ब्द‚प्र‚योगात् । \textbf{सा चोत्प‚त्तिर‚भिव्य‚क्तिर्वा} श‚ब्द‚स्यार्थ‚प्र‚तिपाद‚न‚म्प्र‚त्य\textbf{व्य‚भिचारा‚{\tiny $_{lb}$}‚श्र‚यो}ऽव्य‚भिचार‚स्य निमित्तं \textbf{पौरुषेयी} पुरुष‚कृता । \textbf{इति} । एवं \textbf{पौरुषेय एव स‚म्ब‚न्धः}‚{\tiny $_{lb}$}‚ श‚व्दार्थ‚योः । \textbf{त‚द्द्वारेण च} य‚थोक्त‚स‚म्ब‚न्ध‚द्वारेणा\textbf{र्थ‚प्र‚त्याय‚ने श‚ब्दानान्न निय‚म‚{\tiny $_{lb}$}‚ इत्य‚पौरुषेय‚त्वे}पि श‚ब्दानां \textbf{स‚{\tiny $_{५}$}‚ एव विप्र‚ल‚म्भो} विस‚म्वादः । त‚था चापौरुषेय‚त्व‚{\tiny $_{lb}$}‚क‚ल्प‚ना व्य‚र्थैवेति भावः । \href{http://sarit.indology.info/?cref=pv.3.238-3.239}{२४१-२४२}
	{\color{gray}{\rmlatinfont\textsuperscript{§~\theparCount}}}
	\pend% ending standard par
      ‚{\tiny $_{lb}$}‚

	  
	  \pstart \leavevmode% starting standard par
	\textbf{अपौरुषेय‚तापि} वेद‚वाक्यानां मी मां स कै \textbf{रिष्टा । क‚र्त्तृणां} वेद‚स्य प्र‚णेतॄणा‚{\tiny $_{lb}$}‚\textbf{म‚स्मृते}र्लिङ्गात् । \textbf{किल} श‚ब्द‚श्चायुक्त‚ताख्याप‚नाय ।
	{\color{gray}{\rmlatinfont\textsuperscript{§~\theparCount}}}
	\pend% ending standard par
      ‚{\tiny $_{lb}$}‚

	  
	  \pstart \leavevmode% starting standard par
	\textbf{यापी}त्यादि व्याख्यानं । ब‚हूनाम‚र्थानां क‚र्त्ता न स्म‚र्य‚ते । न च ते ताव‚ताऽ‚{\tiny $_{lb}$}‚कृत‚काः । त‚द्य‚था जीर्ण्ण‚कूपाद‚यः । एवं हेतोर्व्य‚भिचाराद‚युक्त‚रूपा‚{\tiny $_{६}$}‚\textbf{पीय‚म‚पौरुषेय‚ता ।‚{\tiny $_{lb}$}‚ वेद‚वाक्यानां क‚र्त्तुर‚स्म‚र‚णाद् व‚र्ण्य‚ते} जै मि नि ना । अस्यैव‚म्विध‚स्य व‚स्तुनः \textbf{स‚न्त्य}‚{\tiny $_{lb}$}‚‚{\tiny $_{lb}$}‚ \leavevmode\ledsidenote{\textenglish{438/s}}द्य‚त्वेप्य\textbf{नुव‚क्तार} इति । किम‚त्र व‚क्त‚व्यं केव‚लं \textbf{धिग्व्याप‚क‚न्त‚मः} । त‚था हि [।]‚{\tiny $_{lb}$}‚ यः क‚र्त्तुर‚स्म‚र‚णाद‚पौरुषेय‚तामाह जै मि निः । \textbf{त‚स्यैव ताव‚दी}दृश‚म‚तिस्थ‚लं \textbf{प्र‚ज्ञा‚{\tiny $_{lb}$}‚\leavevmode\ledsidenote{\textenglish{157a/PSVTa}} स्ख‚लितं क‚थं वृत्तं} जात\textbf{मिति‚{\tiny $_{७}$}‚} कृत्वा स‚ह विस्म‚येनानुक‚म्प‚या व‚र्त्त‚त इति \textbf{स‚विस्म‚या‚{\tiny $_{lb}$}‚नुक‚म्पं नोस्माकं चेतः} । श्रुत‚व‚तोप्येव‚म‚विद्याविल‚सित‚मिति स‚विस्म‚यं । गाढे‚{\tiny $_{lb}$}‚नाविद्याब‚न्धेन स‚त्त्वाः पीड्य‚न्त इति कृत्वा सानुक‚म्पं । त‚द‚त्रा\textbf{प‚रेपी}दानीन्त‚न्म‚ता‚{\tiny $_{lb}$}‚नुसारिणः कु मा रि ल प्र‚भृत‚यः प‚रीक्ष‚कंम‚न्या एव‚मेत‚द\textbf{नुव‚द‚न्तीति निर्द‚यं निष्कृ‚{\tiny $_{lb}$}‚प‚माक्रान्तं भुव‚नं} ज‚ग‚द् येन \textbf{त‚म‚सा} त‚त्त‚थोक्तं धि‚{\tiny $_{१}$}‚ग्व्याप‚क‚न्त‚मः । अज्ञान‚स्यैवात्र‚{\tiny $_{lb}$}‚ धिग्वादो युक्तो न प्राणिनः । य‚स्मात् \textbf{कः प्राणिन} एवं वादिनोपि \textbf{हितेप्साविप्र‚ल‚ब्ध‚{\tiny $_{lb}$}‚स्य} हित‚प्राप्तीच्छ‚या विप्र‚ल‚ब्ध‚स्य विस‚म्वादित‚स्या\textbf{प‚राधः} । किन्त्व‚ज्ञान‚स्यैवाय‚{\tiny $_{lb}$}‚न्दोषः । किं पुन‚स्त‚स्यैव‚म्व‚द‚तः प्र‚ज्ञास्ख‚लितं ।
	{\color{gray}{\rmlatinfont\textsuperscript{§~\theparCount}}}
	\pend% ending standard par
      ‚{\tiny $_{lb}$}‚

	  
	  \pstart \leavevmode% starting standard par
	य‚स्मादिदं साध‚न‚म‚सिद्ध‚म‚नैकान्तिक‚ञ्च । \href{http://sarit.indology.info/?cref=pv.3.239}{२४२}
	{\color{gray}{\rmlatinfont\textsuperscript{§~\theparCount}}}
	\pend% ending standard par
      ‚{\tiny $_{lb}$}‚

	  
	  \pstart \leavevmode% starting standard par
	त‚त्रासिद्ध‚म‚धिकृत्याह । \textbf{त‚था ही}त्यादि । स्म‚र‚न्ति सौ ग ता वेद‚{\tiny $_{२}$}‚स्य \textbf{क‚र्त्तृ‚{\tiny $_{lb}$}‚न ष्ट का दीन्} । आदिश‚ब्दाद् वा म क वा म दे व वि श्वा मि त्र प्र‚भृतीन् । \textbf{हिर‚{\tiny $_{lb}$}‚ण्य‚ग‚र्भ} ब्र ह्मा णं वेद‚स्य क‚र्त्तारं \textbf{स्म‚र‚न्ति का णा दा वैशेषिकाः} । त‚त‚श्चासिद्धं क‚र्त्तु‚{\tiny $_{lb}$}‚र‚स्म‚र‚णं । \textbf{तेषां} सौग‚तानाञ्च स वेद‚स्य क‚र्त्तृस्म‚र‚ण‚वादो \textbf{मिथ्यावा}द‚स्त‚तः ।‚{\tiny $_{lb}$}‚ सिद्धिहेतो\textbf{रिति चेत् । क इदानी}म्वेदाद‚न्योपि \textbf{पौरुषेयः} श‚ब्दः । न क‚श्चित् पौरु‚{\tiny $_{lb}$}‚षेय इत्य‚र्थः‚{\tiny $_{३}$}‚ । \textbf{एव}मिति क‚र्त्तुः स्म‚र‚ण‚वाद‚स्य मिथ्यात्वे । एत‚देव स्प‚ष्ट‚य‚न्नाह ।‚{\tiny $_{lb}$}‚ कु मा र स म्भ वे त्या दिष्वित्यादि । \textbf{कुमार‚स‚म्भ‚वादिषु} ग्र‚न्थेषु का लि \textbf{दा सा द‚य‚{\tiny $_{lb}$}‚ आत्मान‚म‚न्य‚म्वा प्र‚णेतारं} क‚र्त्तारं \textbf{व्य‚प‚दिश‚न्तो य‚देव‚म्प्र‚तिव्यूह्येर‚न् । प्र‚तिक्षि}‚{\tiny $_{lb}$}‚‚{\tiny $_{lb}$}‚ \leavevmode\ledsidenote{\textenglish{439/s}}प्येर‚न् । मिथ्यावादो युष्माकं न यूयं प्र‚णेतार इति । \textbf{त‚त्र} कुमार‚स‚म्भ‚वादौ‚{\tiny $_{lb}$}‚ क‚र्त्तुः \textbf{प्र‚तिव‚ह‚नेभ्युपेत‚बाधा} ।‚{\tiny $_{४}$}‚ कुमार‚स‚म्भ‚वादीनां पौरुषेय‚त्वेनाभ्युप‚ग‚त‚त्वादिति‚{\tiny $_{lb}$}‚चेत् । \textbf{न‚न्विद‚मेव} क‚र्त्तुर‚स्म‚र‚ण‚म‚पोरुषेया\textbf{भ्युप‚ग‚मेङ्गं} साध‚नं । त‚च्च कुमार‚स‚म्भ‚{\tiny $_{lb}$}‚वादाव‚स्तीति य‚थोक्तेन न्यायेनेति क‚थ‚म‚नेन पौरुषेयः कुमार‚स‚म्भ‚वादिरिष्ट इति‚{\tiny $_{lb}$}‚ क‚स्य \textbf{केन बाधा} । अथ तुल्येपि न्याये कुमार‚स‚म्भ‚वादौ क‚र्त्तुः प्र‚तिव‚ह‚नेभ्युपेत‚{\tiny $_{lb}$}‚बाध‚न‚मिष्य‚ते । \textbf{त}देत‚द‚{\tiny $_{५}$}‚भ्युपेत‚बाध‚न‚म्प‚र‚स्यापि वेद‚वादिनोपि वेद‚वाक्येषु प्र‚णेतृ‚{\tiny $_{lb}$}‚प्र‚तिव‚ह‚ने \textbf{तुल्य‚मेव । त‚स्य} वेद‚वादिनो वेदापौरुषेय‚त्व‚मिष्ट‚म‚तो पौरुषेय‚त्व‚{\tiny $_{lb}$}‚स्ये\textbf{ष्ट‚त्वात्} क‚र्त्तुः प्र‚तिव‚ह‚नेप्य\textbf{दोषः} । अभ्युपेत‚वाधादोषो नास्तीति \textbf{चेत् कुतोस्य}‚{\tiny $_{lb}$}‚ वेद‚वादिनः आग‚मोपादान‚निमित्त‚तायाः प‚रीक्षायाः प्रागिय‚म‚पौरुषेयो वेद इत्येव‚{\tiny $_{lb}$}‚ \textbf{मिष्टि}र‚भ्यु‚{\tiny $_{६}$}‚प‚ग‚तिः । \textbf{अप्र‚माणिका} प्र‚माण‚र‚हिता आसीत् । त‚था हि वेद‚स्यापौरुषे‚{\tiny $_{lb}$}‚य‚त्वाभ्युप‚ग‚मे क‚र्त्तुर‚स्म‚र‚णं प्र‚माण‚मुक्तं । त‚त्र चान‚न्त‚र‚मुक्तो दोष इत्य\textbf{प्र‚माणि‚{\tiny $_{lb}$}‚केय‚मिष्टिः} ।
	{\color{gray}{\rmlatinfont\textsuperscript{§~\theparCount}}}
	\pend% ending standard par
      ‚{\tiny $_{lb}$}‚

	  
	  \pstart \leavevmode% starting standard par
	अथ प्र‚माण‚म‚न्त‚रेण वेद‚स्यापौरुषेय‚त्व‚म‚ङ्गीकृत‚वान् वेद‚वादी । त‚दाऽ\textbf{क‚स्माद्‚{\tiny $_{lb}$}‚ ग्राही} युक्त्या विना ग्राह‚क‚श्\textbf{चायं} मी मां स कः \textbf{किम्पुनः क्व‚चित्} पौरुषेयापौरु‚{\tiny $_{७}$}‚षेय- \leavevmode\ledsidenote{\textenglish{175b/PSVTa}}‚{\tiny $_{lb}$}‚ त्वादौ \textbf{साध‚नं} प्र‚माण\textbf{म‚पेक्ष‚ते । य‚दिति पौरुषेयापौरुषेय‚चिन्त‚ये}ति पौरुषेयापौरु‚{\tiny $_{lb}$}‚ष‚य‚त्व‚साध‚नोप‚न्यासेना\textbf{त्मान‚मा}साद‚य‚ति । यो ह्य‚युक्तिग्राही स स‚र्व‚त्र त‚थैव प्र‚व‚{\tiny $_{lb}$}‚र्त्त‚तां । किमिति क्व‚चित् प्र‚माणाव‚तार‚णेनात्मानं \textbf{दुःख‚य‚तीति} स‚मुदायार्थः ।
	{\color{gray}{\rmlatinfont\textsuperscript{§~\theparCount}}}
	\pend% ending standard par
      ‚{\tiny $_{lb}$}‚

	  
	  \pstart \leavevmode% starting standard par
	\textbf{त‚त एवा}प्र‚म‚णिकाया वेद‚स्यापौरुषेय\textbf{त्वेष्टे}र्हेतोर्वेद‚वादिनो वेद‚स्य क‚र्त्तुः प्र‚ति‚{\tiny $_{१}$}‚‚{\tiny $_{lb}$}‚व‚ह‚नेप्य\textbf{न‚भ्युपेत‚बाधायामिष्य‚माणायान्त‚द‚न्य‚स्यापि} त‚स्मात् मी मां स काद‚न्य‚स्यापि‚{\tiny $_{lb}$}‚ पुंसः कु मा र स म्भ वादिम‚पौरुषेय‚मिच्छ‚त‚स्त‚त्प्र‚णेतृप्र‚तिव‚ह‚नेप्य‚न‚भ्युपेत‚बाध‚नं‚{\tiny $_{lb}$}‚ \textbf{तुल्य‚मित्य‚नुपाल‚म्भः} । त‚त्र प्र‚तिव‚ह‚नेभ्युपेत‚बाधेत्य‚य‚मुपाल‚म्भो नास्तीत्य‚र्थः ।
	{\color{gray}{\rmlatinfont\textsuperscript{§~\theparCount}}}
	\pend% ending standard par
      ‚{\tiny $_{lb}$}‚

	  
	  \pstart \leavevmode% starting standard par
	\textbf{किं चान‚तिश‚य‚द‚र्शी}त्यादि । \textbf{एवंप्र‚काराणां क‚र्त्तुर‚स्म‚र‚णादित्येव‚मादीनाम‚पौ‚{\tiny $_{lb}$}‚रुष‚य‚त्व‚सा‚{\tiny $_{२}$}‚ध‚नानां} वाक्येषु पौरुषेयापौरुषेय‚त्वाभिम‚तेष्व‚न‚तिश‚य‚द‚र्शीति स‚म्ब‚न्धः ।‚{\tiny $_{lb}$}‚ त‚था हि य‚था पौरुषेयाणाम‚नेकेषां चिर‚कालातीत‚क‚र्त्तृकाणां क‚र्तुर‚स्म‚र‚ण‚म‚स्ति ।‚{\tiny $_{lb}$}‚ ‚{\tiny $_{lb}$}‚ \leavevmode\ledsidenote{\textenglish{440/s}}त‚था वेद‚वाक्येष्वेवं क‚र्त्तुर‚स्म‚र‚णादिसाध‚न‚स्यान‚तिश‚य‚द‚र्शी विशेष‚द‚र्शी स‚न्‚{\tiny $_{lb}$}‚ मी मां स कः । पुरुष‚कार्याणां वा श‚ब्दानां ध‚र्माः कार्य‚ध‚र्माः पुरुषान्व‚य‚व्य‚तिरेकानुवि‚{\tiny $_{lb}$}‚धायित्वाद‚य‚{\tiny $_{३}$}‚स्तेषां \textbf{कार्य‚ध‚र्माणाम्वाक्येषु} लौकिक‚वैदिकेष्व‚न‚तिश‚य‚द‚र्शी स‚न् \textbf{क्व‚चिद्}‚{\tiny $_{lb}$}‚ वैदिके श‚ब्दे\textbf{तिश‚यं} विशेष‚म‚पौरुषेय‚त्व‚ल‚क्ष‚ण\textbf{म‚भ्युपेति} नान्य‚त्रेति न किञ्चिद‚भ्युप‚ग‚मे‚{\tiny $_{lb}$}‚ साध‚न‚म‚स्ती\textbf{त्य‚प्र‚त्य‚येवा}युक्ते\textbf{वास्य} वेद‚वादिनो \textbf{वृत्तिः} ।
	{\color{gray}{\rmlatinfont\textsuperscript{§~\theparCount}}}
	\pend% ending standard par
      ‚{\tiny $_{lb}$}‚

	  
	  \pstart \leavevmode% starting standard par
	त‚देवं य‚थोक्त‚विधिना क‚र्त्तुर‚स्म‚र‚णादित्य‚सिद्धो हेतुः । अनैकान्तिक‚त्व‚म‚प्याह ।‚{\tiny $_{lb}$}‚ \textbf{दृश्य‚न्ते चे}त्यादि । उप‚देश‚पार‚म्प‚{\tiny $_{४}$}‚र्यं स म्प्र दा यः । विच्छिन्नः क्रियासंप्र‚दायः‚{\tiny $_{lb}$}‚ पुरुष‚कृत‚त्व‚संप्र‚दायो येषां व‚टे व‚टे वै श्र व णादि श‚ब्दानान्ते त‚था । अनेनास्म‚र्य‚माण‚{\tiny $_{lb}$}‚क‚र्त्तृत्व‚माह । कृत‚काश्च पौरुषेयाश्च । त‚तः पौरुषेयेपि वाक्ये क‚र्त्तुर‚स्म‚र‚ण‚म्व‚र्त्त‚त‚{\tiny $_{lb}$}‚ इत्य‚नैकान्तिकोयं हेतुः । \textbf{तानि}ति \textbf{विच्छिन्न‚क्रियास‚म्प्र‚दायान् । कृत‚कान्} श‚ब्दान् ।‚{\tiny $_{lb}$}‚ \textbf{य‚त्न‚व‚न्तः} पुमांस\textbf{मुप‚ल‚भ‚न्ते}ऽनेन कृ‚{\tiny $_{५}$}‚ता \textbf{इति} । नैत‚देवं । किङ्कार‚णं । य‚त्न‚व‚तोपि‚{\tiny $_{lb}$}‚ क‚र्त्तुः स्म‚र‚णे \textbf{निय‚माभावात्} । नाव‚श्यं क‚र्त्तार‚मुप‚ल‚भ‚ते य‚त्न‚वान‚पीति स‚न्देह‚{\tiny $_{lb}$}‚ एव ।
	{\color{gray}{\rmlatinfont\textsuperscript{§~\theparCount}}}
	\pend% ending standard par
      ‚{\tiny $_{lb}$}‚

	  
	  \pstart \leavevmode% starting standard par
	किञ्च [।] \textbf{अन्य‚त्रा}पौरुषेयाभिम\textbf{तेपि} श‚ब्देस्य क‚र्त्ता नोप‚ल‚भ्य‚त इत्य‚नुप‚{\tiny $_{lb}$}‚ल‚म्भ‚स्य । उप‚ल‚भ्य‚ते वास्यापौरुषेय‚स्य क‚र्त्तेत्युप‚ल‚म्भ‚स्य न प्र‚माणात् कुत‚श्चिन्नि‚{\tiny $_{lb}$}‚श्च‚यः । किन्तु \textbf{प‚रोप‚देशात्} किंभूताद\textbf{प्र‚त्य‚याद}प्र‚{\tiny $_{६}$}‚माण‚कात् । एवंभूताच्चोप‚देशात्‚{\tiny $_{lb}$}‚ क‚र्त्तु\textbf{रुप‚ल‚म्भानुप‚ल‚म्भ‚स्यानिश्च‚यार्ह‚त्वात्} । म‚या वेद‚वाक्यानि कृतानीत्येवंवादिनो‚{\tiny $_{lb}$}‚नुप‚ल‚म्भाद् वेद‚वाक्येषु क‚र्त्तुर‚भावो निश्चीय‚त इत्येत‚द‚पि नास्ति । \textbf{स्व‚यंकृतानाम‚पि}‚{\tiny $_{lb}$}‚ श‚ब्दानाम\textbf{प‚ह्नोतृद‚र्श‚नात्} । स्व‚यंकृत्वापि श‚ब्दा न म‚यैते कृता इत्य‚प‚ल‚पितारो‚{\tiny $_{lb}$}‚ \leavevmode\ledsidenote{\textenglish{158a/PSVTa}} दृश्य‚न्ते । त‚त्र च किम‚नेनैते कृताः किम्वान्येने‚{\tiny $_{७}$}‚ति \textbf{निष्ठाग‚म‚न‚स्य} निश्च‚य‚ग‚म‚न‚{\tiny $_{lb}$}‚\textbf{स्याश‚क्य‚त्वात्} । \href{http://sarit.indology.info/?cref=pv.3.239}{२४२}
	{\color{gray}{\rmlatinfont\textsuperscript{§~\theparCount}}}
	\pend% ending standard par
      ‚{\tiny $_{lb}$}‚

	  
	  \pstart \leavevmode% starting standard par
	त‚देवं क‚र्त्तुर‚स्म‚र‚णादिति हेतुन्निराकृत्यान्य‚द‚पि साध‚नं ।
	{\color{gray}{\rmlatinfont\textsuperscript{§~\theparCount}}}
	\pend% ending standard par
      ‚{\tiny $_{lb}$}‚
	  \bigskip
	  \begingroup
	
	    
	    \stanza[\smallbreak]
	  {\normalfontlatin\large ``\qquad}वेद‚स्याध्य‚नं स‚र्वं गुर्व‚ध्य‚य‚न‚पूर्व‚कं [।]&‚{\tiny $_{lb}$}‚वेदाध्य‚य‚न‚वाच्य‚त्वाद‚धुनाध्य‚य‚नं य‚था [।]{\normalfontlatin\large\qquad{}"}\&[\smallbreak]
	  
	  
	  ‚{\tiny $_{lb}$}‚ ‚{\tiny $_{lb}$}‚ \textsuperscript{\textenglish{441/s}}

	  
	  \pstart \leavevmode% starting standard par
	\edtext{\textsuperscript{*}}{\edlabel{pvsvt_441-1}\label{pvsvt_441-1}\lemma{*}\Bfootnote{\href{http://sarit.indology.info/?cref=\%C5\%9Bv-v\%C4\%81kya.366}{ Ślokavārtika, Vākya, 366 }}}इति दूष‚यितुमुप‚न्य‚स्य‚ति । य‚थेत्यादि । य‚थाय‚मिदानीन्त‚नो वेद‚स्याध्येतान्य‚तः‚{\tiny $_{lb}$}‚ स‚काशाद् \textbf{श्रुत्वा} इम‚म्वैदिक‚म्व‚र्ण्ण‚क्र‚मं । \textbf{व‚र्ण‚प}द‚योः \textbf{क्र‚मं व‚क्तु}म‚ध्येतुं \textbf{न स‚म‚{\tiny $_{१}$}‚‚{\tiny $_{lb}$}‚र्थः । त‚थान्योपि} वेद‚स्य क‚र्त्तृत्वेनाभिम‚तः सोप्य‚न्य‚त उप‚देश‚म‚पेक्ष‚ते सोप्य‚न्य‚त‚{\tiny $_{lb}$}‚ इत्य‚नादित्वात् सिद्ध‚म‚पौरुषेय‚त्व‚मिति एवं \textbf{क‚श्च‚ना}ह । \textbf{त‚स्या}प्येव‚म्वादिन‚स्\textbf{त‚देवो‚{\tiny $_{lb}$}‚त्त‚रं य‚त्} क‚र्त्तुर‚स्म‚र‚णादित्य‚त्रोक्तं । \textbf{एव}म‚न‚न्त‚रोक्त‚प्र‚कारेणा\textbf{पौरुषेय‚त्वेपि कि}मि‚{\tiny $_{lb}$}‚दानीम्\textbf{पौरुषेय}म्वाक्यं स‚र्व‚म‚पौरुषेयं \textbf{स्यात्} । अन्य‚स्यापि कु मा र स म्भ वाध्य‚य‚न‚{\tiny $_{lb}$}‚स्याध्य‚य‚{\tiny $_{२}$}‚न‚पूर्व‚क‚त्वेनानादित्व‚प्र‚साध‚नात् । त‚त्र प्र‚साध‚नेभ्युपेत‚बाधेति चेत् ।‚{\tiny $_{lb}$}‚ न‚न्विद‚मेवाभ्युप‚ग‚माङ्ग‚मित्यादि स‚र्व‚म्वाच्यं । अस्यैव संग्र‚हायादिश‚ब्दः प्र‚युक्तः ।‚{\tiny $_{lb}$}‚ \href{http://sarit.indology.info/?cref=pv.3.239-240}{२४२-२४३}
	{\color{gray}{\rmlatinfont\textsuperscript{§~\theparCount}}}
	\pend% ending standard par
      ‚{\tiny $_{lb}$}‚ 

	  
	  \pstart \leavevmode% starting standard par
	अतिप्र‚संग‚मेव द‚र्श‚य‚न्नाह । \textbf{त‚था ही}त्यादि । \textbf{अन्यो वा} पुरुष\textbf{र‚चितः} कुमार‚{\tiny $_{lb}$}‚स‚म्भ‚वादिको \textbf{ग्र‚न्थः । संप्र‚दायादृते} । प‚रोप‚देश‚म‚न्त‚रेण \textbf{प‚रैः कोभिहितो दृष्टो}‚{\tiny $_{lb}$}‚ नैव क‚श्चिद् दृष्टः । \textbf{येन} कार‚{\tiny $_{३}$}‚णेन \textbf{सोपि} वेदाद‚न्यो ग्र‚न्थः । \textbf{एव}मित्य‚पौरुषेयः‚{\tiny $_{lb}$}‚ किन्ना\textbf{नुमीय‚ते} ।
	{\color{gray}{\rmlatinfont\textsuperscript{§~\theparCount}}}
	\pend% ending standard par
      ‚{\tiny $_{lb}$}‚ 

	  
	  \pstart \leavevmode% starting standard par
	\textbf{न ख‚ल्वि}त्यादिना व्याच‚ष्टे । \textbf{न ख‚लु किञ्चिद‚पौरुषेय‚त्वाश्र‚यो} पौरुषेय‚त्व‚स्य‚{\tiny $_{lb}$}‚ सिद्धिनिमित्त\textbf{म‚न्य‚त्रेदानीन्त‚नानाम‚नुप‚देशेन} यः पाठ‚स्त\textbf{त्राश‚क्तेः} । न हि प‚रोप‚देश‚{\tiny $_{lb}$}‚म‚न्त‚रेण वेदं प‚ठितुं श‚क्त इत्य‚पौरुषेय‚त्व‚म्वेद‚वाक्यानामिष्टं । \textbf{सा चा}नुप‚देश‚पाठा‚{\tiny $_{lb}$}‚श‚क्ति\textbf{र‚न्य‚त्रा}पि‚{\tiny $_{४}$}‚ पौरुषेयाभिम‚ते । \textbf{एकेन} केन‚चित् पुरुषेण \textbf{र‚चितेन्य‚स्या}ध्येतुस्तु‚{\tiny $_{lb}$}‚‚{\tiny $_{lb}$}‚ ‚{\tiny $_{lb}$}‚ \leavevmode\ledsidenote{\textenglish{442/s}}\textbf{ल्या । त‚द‚नुसारि}णेति । अनुप‚देश‚पाठाश‚क्तिम‚पौरुषेय‚त्व‚साध‚क‚त्वेन योनुस‚र‚ति तेन‚{\tiny $_{lb}$}‚ \textbf{स‚र्वो} लौकिक‚वैदिकः श‚ब्द\textbf{स्त‚था} पौरुषेय‚त्वे\textbf{नानुमेयः । न वा क‚श्चिद्} वैदिको‚{\tiny $_{lb}$}‚ विशेषाभावात् । \textbf{त‚स्य} लौकिक‚स्य वाक्य‚स्य \textbf{त‚थे}त्य‚पौरुषेय‚त्वेना\textbf{न‚ष्ट‚त्वा}दित्यादौ ।‚{\tiny $_{५}$}‚‚{\tiny $_{lb}$}‚ आदिश‚ब्देनाभ्युपेत‚बाधाप‚रिग्र‚ह‚स्त‚त्रोक्त‚म‚न‚न्त‚र‚मेव \textbf{इष्टेस्त‚दाश्र‚य‚त्वा}द‚पौरुषेय‚त्व‚{\tiny $_{lb}$}‚साध‚नाश्र‚य‚त्वा\textbf{दि}\textbf{त्यादि\edtext{}{\edlabel{pvsvt_442-2}\label{pvsvt_442-2}\lemma{त्यादि}\Bfootnote{\cite{pvb-B} साध‚न‚त्वादित्यादि ।}}} । \href{http://sarit.indology.info/?cref=pv.3.240-241}{२४३-२४४}
	{\color{gray}{\rmlatinfont\textsuperscript{§~\theparCount}}}
	\pend% ending standard par
      ‚{\tiny $_{lb}$}‚ 

	  
	  \pstart \leavevmode% starting standard par
	\textbf{अपि च । य‚ज्जातीयो} य‚द्द्र‚व्य‚स‚मान‚जातीयः । \textbf{य‚तो} हेतोः \textbf{सिद्धो}न्व‚य‚व्य‚ति‚{\tiny $_{lb}$}‚रेकाभ्यां । स त‚ज्जातीय‚त्वेना\textbf{विशिष्टो}न्योप्य‚दृष्ट‚हेतुर‚पि त‚स्माद्धेतोर्न\edtext{}{\edlabel{pvsvt_442-3}\label{pvsvt_442-3}\lemma{स्माद्धेतोर्न}\Bfootnote{\cite{pvb-B} ०तोर्भ‚व‚ती०}} भ‚व‚ती‚{\tiny $_{lb}$}‚त्येवं \textbf{संप्र‚तीय‚ते} । किमिव [।] \textbf{अग्निकाष्ठ‚व‚त्} । य‚थेन्ध‚नादेको‚{\tiny $_{६}$}‚ व‚ह्निर्दृष्ट‚स्त‚त्स‚{\tiny $_{lb}$}‚मान‚स्व‚भावोऽ\textbf{प‚रोपि} त‚त्स‚मान‚हेतुरेवा\textbf{दृष्ट‚हेतुर‚पि} स‚म्प्र‚तीय‚ते । अनेन वेद‚स्या‚{\tiny $_{lb}$}‚पौरुष‚य‚त्व‚साध‚ने प्र‚तिज्ञाया अनुमान‚बाधामाह ।
	{\color{gray}{\rmlatinfont\textsuperscript{§~\theparCount}}}
	\pend% ending standard par
      ‚{\tiny $_{lb}$}‚ 

	  
	  \pstart \leavevmode% starting standard par
	\textbf{ने}त्यादिना व्याच‚ष्टे । \textbf{हेतोर‚द‚र्श‚नान्नाहेतुको नाम} । य‚स्माद‚दृष्ट‚हेत‚वोपि ।‚{\tiny $_{lb}$}‚ न दृष्टो हेतुरेषामिति विग्र‚हः । त एवं भ‚ता अपि \textbf{भावा}स्त‚द‚न्यैर्दृष्ट‚हेतुभिः \textbf{स्व‚भा‚{\tiny $_{lb}$}‚\leavevmode\ledsidenote{\textenglish{158b/PSVTa}} वाभेद‚{\tiny $_{७}$}‚म‚नुभ‚व}न्त‚स्तुल्य‚रूपा इत्य‚र्थः । \textbf{त‚थाविधा} इति त‚त्स‚मान‚हेत‚व\textbf{स्स‚मुन्नीय‚न्ते} ।‚{\tiny $_{lb}$}‚ अय‚म‚त्र स‚मुदायार्थः । लौकिकेन श‚ब्देन स‚मान‚ध‚र्मो वैदिकोपि श‚ब्दो लौकिक‚व‚त्‚{\tiny $_{lb}$}‚ पुरुष‚हेतुकः स्यान्ना वा क‚श्चिद‚पीति ।
	{\color{gray}{\rmlatinfont\textsuperscript{§~\theparCount}}}
	\pend% ending standard par
      ‚{\tiny $_{lb}$}‚ 

	  
	  \pstart \leavevmode% starting standard par
	\textbf{अथ हेतुरूप‚स्य} हेतुस्व‚भाव‚स्य \textbf{निवृत्ताव‚पि त‚द्रूपं} पुरुष‚हेतुश‚ब्द‚स‚मानं रूपंन‚{\tiny $_{lb}$}‚ \textbf{निवृत्तं} वैदिक‚स्य श‚ब्द‚स्येष्य‚ते । त‚दा \textbf{कार्य‚ध‚र्म‚व्य‚तिक्र‚मः} । अयं हि कार्य‚स्य‚{\tiny $_{१}$}‚ ध‚र्मो‚{\tiny $_{lb}$}‚ य‚त्कार‚ण‚निवृत्तौ निवृत्तिः । य‚दा तु निवृत्तेपि पुरुषे वैदिकेषु श‚ब्देषु पौरुषेयं‚{\tiny $_{lb}$}‚ रूपं स्यात् त‚दा तेन कार्य‚ध‚र्मो व्य‚तिवृत्तः स्यात् । \textbf{त‚तः} कार्य‚ध‚र्म‚व्य‚तिक्र‚मात्‚{\tiny $_{lb}$}‚  ‚{\tiny $_{lb}$}‚ \leavevmode\ledsidenote{\textenglish{443/s}}त‚तः पुरुषान्न किञ्चिद्वाक्यं \textbf{स्यादिति न क‚श्चि}च्छ‚ब्दो लौकिक‚स्त‚थेति पौरुषेय‚त्वेन‚{\tiny $_{lb}$}‚ \textbf{व‚च‚नीयः स्यात् । रूप‚विशेषो वा} पौरुषेयाणां वैदिकाद् भिन्नो \textbf{द‚र्श‚नीयो यो} रूप‚{\tiny $_{lb}$}‚विशेष एनं पुरुषाख्यं‚{\tiny $_{२}$}‚ \textbf{हेतुम‚नुविद‚ध्यात् । येन} विशेषेणेष्ट‚स्यापौरुषेय‚त्वेन वेद‚स्य‚{\tiny $_{lb}$}‚ अनिष्ट‚स्य च लौकिक‚स्य । \textbf{इष्ट‚विप‚र्य‚यो न स्यात्} । य‚थाक्र‚मं पौरुष‚य‚त्व‚म‚पौरुषेय‚{\tiny $_{lb}$}‚त्व‚म्वा स्यात् ।\edtext{\textsuperscript{*}}{\edlabel{pvsvt_443-1}\label{pvsvt_443-1}\lemma{*}\Bfootnote{\cite{pvb-B} न स्यात् ।}} न च लौकिक‚वैदिकानां क‚श्चित् स्व‚भाव‚भेदोस्तीत्युक्तं ।
	{\color{gray}{\rmlatinfont\textsuperscript{§~\theparCount}}}
	\pend% ending standard par
      ‚{\tiny $_{lb}$}‚ 

	  
	  \pstart \leavevmode% starting standard par
	किं च पुरुषाख्य‚स्य \textbf{हेतोर्यः} स्व‚भाव‚स्त‚स्य निवृतेर्निवृत्ताव‚पि प‚ष्ठीस‚प्त‚म्योर‚{\tiny $_{lb}$}‚भेदात् । य‚था वृक्षे शाखा वृक्ष‚स्य‚शाखेति ।‚{\tiny $_{३}$}‚ वैदिकानां वाक्यानां पौरुषेयैर्वाक्यै‚{\tiny $_{lb}$}‚\textbf{र‚भेदेन} तुल्य‚रूप‚त्वेभ्युप‚ग‚म्य‚माने । \textbf{स} तेषां लौकिकानाम्वाक्यानाम्भेवः पुरुष‚{\tiny $_{lb}$}‚कृतो विशेष \textbf{आक‚स्मिकः स्याद‚हेतुकः} स्यात् । पुरुष‚म‚न्त‚रेणापि वैदिकेषु वाक्येषु‚{\tiny $_{lb}$}‚ त‚स्य विशेष‚स्य भावात् । \textbf{त‚था च न क्व‚चिन्निव‚र्त्तेताकाशादौ । न चैव‚न्त‚स्माद् यः‚{\tiny $_{lb}$}‚ स्व‚भावो} य‚ज्ज‚न्मा । य‚स्माज्ज‚न्म य‚स्ये\edtext{}{\edlabel{pvsvt_443-2}\label{pvsvt_443-2}\lemma{स्ये}\Bfootnote{\cite{pvb-B} ज‚न्माऽस्येति ।}}ति विग्र‚हः । \textbf{सोन्य‚त्रा‚{\tiny $_{४}$}‚प्य‚दृष्ट‚हेताव‚{\tiny $_{lb}$}‚प्य‚विभ‚ज्य‚मानः} । दृष्ट‚हेतुना कार्येणापृथ‚क्क्रिय‚माण‚स्त‚त्कार्य‚तां \textbf{यातो भ‚व‚न्‚{\tiny $_{lb}$}‚ दृष्ट‚स्त‚स्कार्य‚तां स्वात्म‚ना} स्वेन रूपेण \textbf{नातिव‚र्त्त‚ते । किमिव [।] अग्नी‚{\tiny $_{lb}$}‚न्ध‚न‚व‚त्} । अग्निश्चेन्ध‚नं चेत्य‚ग्नीन्ध‚न‚न्तेन तुल्य‚न्त‚द्व‚त् । दृष्टेनेन्ध‚न‚कार‚णेनाग्निना‚{\tiny $_{lb}$}‚ भेद‚म‚नुभ‚व‚न्न‚दृष्ट‚कार‚णोप्य‚ग्निर्य‚थेन्ध‚न‚कार्य‚तां नातिव‚र्त्त‚ते त‚द्व‚त् । \href{http://sarit.indology.info/?cref=pv.3.241-3-242}{२४४-२४५}
	{\color{gray}{\rmlatinfont\textsuperscript{§~\theparCount}}}
	\pend% ending standard par
      ‚{\tiny $_{lb}$}‚ 

	  
	  \pstart \leavevmode% starting standard par
	त‚त्रैत‚स्मिन् न्याये स्थिते । लौ‚{\tiny $_{५}$}‚किक‚वैदिक‚योर्वाक्य‚योर्भेद‚म‚प्र‚द‚र्श्य अपो‚{\tiny $_{lb}$}‚रुषेय‚त्व‚साध‚नाय \textbf{ये हेत‚वः प्र‚वित‚न्य‚न्ते} । विस्त‚रेणाभिधीय‚न्ते । त‚द्य‚था क‚र्त्तु‚{\tiny $_{lb}$}‚र‚स्म‚र‚णात् ।
	{\color{gray}{\rmlatinfont\textsuperscript{§~\theparCount}}}
	\pend% ending standard par
      ‚{\tiny $_{lb}$}‚ 
	    
	    \stanza[\smallbreak]
	  {\normalfontlatin\large ``\qquad}वेद‚स्याध्य‚य‚नं स‚र्व‚ङ् गुर्व‚ध्य‚य‚न‚पूर्व‚क‚म् [।]&‚{\tiny $_{lb}$}‚वेदाध्य‚य‚न‚वाच्य‚त्वात् अधुनाध्य‚य‚नं य‚था । \href{http://sarit.indology.info/?cref=\%C5\%9Bv-v\%C4\%81kya.366}{वाक्य० ३६६}&‚{\tiny $_{lb}$}‚अतीतानाग‚तौ कालौ वेद‚कार‚वियोगिनो ।&‚{\tiny $_{lb}$}‚काल‚त्वात् त‚द्य‚था कालो व‚र्त्त‚मान‚स्स‚मीक्ष्य‚ते ॥&‚{\tiny $_{lb}$}‚\leavevmode\ledsidenote{\textenglish{444/s}}ब्र‚ह्माद‚यो न वेदा‚{\tiny $_{६}$}‚नां क‚र्त्तार इति ग‚म्य‚तां ।&‚{\tiny $_{lb}$}‚पुरुष‚त्वादिहेतुभ्य‚स्त‚द्य‚था प्राकृता न‚रा\edtext{}{\edlabel{pvsvt_444-1}\label{pvsvt_444-1}\lemma{रा}\Bfootnote{\href{http://sarit.indology.info/?cref=\%C5\%9Bv}{ Ślokavārtika. }}} इति ।{\normalfontlatin\large\qquad{}"}\&[\smallbreak]
	  
	  
	  
	  \endgroup
	‚{\tiny $_{lb}$}‚

	  
	  \pstart \leavevmode% starting standard par
	\textbf{स‚र्वे ते} हेत‚वो \textbf{व्य‚भिचारिणो}ऽनेकान्तिका एव ।\edtext{\textsuperscript{*}}{\edlabel{pvsvt_444-2}\label{pvsvt_444-2}\lemma{*}\Bfootnote{\cite{pvb-B} किं कार‚णं-- added.}} \textbf{कार्य‚सामान्य‚द‚र्श‚नात्} ।‚{\tiny $_{lb}$}‚ पुरुष‚कार्यैः श‚ब्दैः सामान्य‚स्य तुल्य\edtext{}{\edlabel{pvsvt_444-3}\label{pvsvt_444-3}\lemma{तुल्य}\Bfootnote{\cite{pvb-B} तुल्य‚त्वात्}} स्य वैदिकेषु श‚ब्देषु द‚र्श‚नात् ।
	{\color{gray}{\rmlatinfont\textsuperscript{§~\theparCount}}}
	\pend% ending standard par
      ‚{\tiny $_{lb}$}‚

	  
	  \pstart \leavevmode% starting standard par
	किम्व‚द‚नैकान्तिका इत्याह । \textbf{य‚थे}त्यादि ।
	{\color{gray}{\rmlatinfont\textsuperscript{§~\theparCount}}}
	\pend% ending standard par
      ‚{\tiny $_{lb}$}‚

	  
	  \pstart \leavevmode% starting standard par
	\leavevmode\ledsidenote{\textenglish{159a/PSVTa}} य‚द्वा \textbf{त‚त्रे}ति । य‚द्वेदाध्य‚य‚न‚न्त‚द्वेदाध्य‚य‚न‚पूर्व‚क‚मित्य‚त्र प्र‚योगे । अप्र‚द‚र्श्य‚{\tiny $_{७}$}‚‚{\tiny $_{lb}$}‚ भेद‚मिति वेद‚क्रियाप्र‚तिभार‚हितात् पुरुषाद् विशेष‚म‚प्र‚द‚र्श्य । इदानीं वेदाध्य‚य‚नं‚{\tiny $_{lb}$}‚ वेदाध्य‚य‚न‚पूर्व‚क‚न्त‚थान्य‚दापीत्येवं वेदाध्य‚य‚न‚त्व‚ल‚क्ष‚ण‚स्य कार्य‚सामान्य‚स्य द‚र्शाना‚{\tiny $_{lb}$}‚देवंप्र‚कारा हेत‚वः प्र‚वित‚न्य‚ते स‚र्वे ते व्य‚भिचारिणः । \textbf{य‚थाऽन्योपि प‚थिक‚कृताग्नि}‚{\tiny $_{lb}$}‚र‚दृष्ट‚हेतुत्वात् । \textbf{ज्वालान्त‚र‚पूर्व‚को न काष्ठ‚निर्म‚थ‚न‚पूर्व‚कः} । कुतः [।] प‚थि‚{\tiny $_{lb}$}‚काग्निव‚त् । किमिव [।] \textbf{अ‚{\tiny $_{१}$}‚न‚न्त‚राग्निव‚दि}ति ज्वालान्त‚र‚संभ‚त‚दृश्य‚मानाग्नि‚{\tiny $_{lb}$}‚व‚त् ।
	{\color{gray}{\rmlatinfont\textsuperscript{§~\theparCount}}}
	\pend% ending standard par
      ‚{\tiny $_{lb}$}‚

	  
	  \pstart \leavevmode% starting standard par
	\textbf{क‚थ}मित्यादि । य‚स्मा\textbf{ज्ज्वालोद्भ‚व‚साम‚र्थ्यं ह्याश्रि}त्येति ज्वालायाः स‚काशा‚{\tiny $_{lb}$}‚दुद्भ‚व‚साम‚र्थ्य‚माश्रित्य \textbf{प‚थिक‚कृत‚द‚ह‚न‚स्य हेत्व‚न्त‚र‚म‚र}णिनिर्म‚थ‚नं \textbf{प्र‚तिक्षिप्य‚ते} ।‚{\tiny $_{lb}$}‚ किं कार‚णं [।] \textbf{य‚दि ह्य‚य‚म‚ग्निर्विना} ज्वाल‚या \textbf{स्याद}त्रापीति ज्वालापूर्व‚क‚प‚थि‚{\tiny $_{lb}$}‚काग्निस्थानेपि ज्वालाम‚न्त‚रेणैव \textbf{स्यादि}ति । \textbf{त‚त्रै}त‚स्मिन् साध‚नेऽनैकान्तिक‚त्व‚{\tiny $_{lb}$}‚मु‚{\tiny $_{२}$}‚च्य‚ते । क‚थं \textbf{ज्वालेत‚र‚ज‚न्म‚नोर्ज्वालाया योत्प‚त्तिः} । इत‚र‚स्माद‚र‚णिनिर्म‚थ‚{\tiny $_{lb}$}‚नाद् योत्प‚त्तिस्त‚योरुत्प‚त्योर‚ग्निसामान्ये प‚र‚स्प‚र‚म‚बाध्य\textbf{बाध‚क‚त्वात्} । को ह्य‚त्र‚{\tiny $_{lb}$}‚विरोधोग्निश्च स्यान्न च ज्वालान्त‚र‚पूर्व‚क इति । एवं स‚ति \textbf{ज्वालाप्र‚भ‚व‚त्व}‚{\tiny $_{lb}$}‚‚{\tiny $_{lb}$}‚ \leavevmode\ledsidenote{\textenglish{445/s}}म्व‚ह्ने रूप\textbf{म‚न्य‚थापि स्याद}र‚णि\edtext{}{\edlabel{pvsvt_445-1}\label{pvsvt_445-1}\lemma{णि}\Bfootnote{\cite{pvb-B} स्यादिति । अर‚णि० ।}} निर्म‚थ‚नाद‚पि स्यात् । इति\edtext{}{\edlabel{pvsvt_445-2}\label{pvsvt_445-2}\lemma{इति}\Bfootnote{\cite{pvb-B} त्य‚क्तः ।}} \textbf{एवंध‚र्म‚यो}र्ज्वाले‚{\tiny $_{lb}$}‚त‚र‚स‚म्भ‚विनोर्द्व‚यो\textbf{रेक‚त्रार्थे} व‚ह्निसामान्ये \textbf{स‚म्भ‚{\tiny $_{३}$}‚वात्} कार‚णात् स \textbf{प‚थिकाग्नि‚{\tiny $_{lb}$}‚र‚न्यो} वा चेद\edtext{}{\edlabel{pvsvt_445-3}\label{pvsvt_445-3}\lemma{चेद}\Bfootnote{\cite{pvb-B} वेदाध्य‚य‚नादिः ।}}ध्य‚य‚नादिः । \textbf{एक‚प्र‚तिनिय‚त} इति ज्वालापूर्व‚क एव । वेदाध्य‚य‚नं ।‚{\tiny $_{lb}$}‚ वा वेदाध्य‚य‚न‚पूर्व‚क‚मेवेत्येत\textbf{न्न स्यादित्याशंक्य‚ते व्य‚भिचारः} [।] वेदाध्य‚य‚नं च स्यात् ।‚{\tiny $_{lb}$}‚ न च वेदाध्य‚य‚न‚पूर्व‚कं । त‚था प‚थिकाग्निश्च स्यान्न न च ज्वालापूर्व‚क इति ।‚{\tiny $_{lb}$}‚ विरोधाभावात् ।
	{\color{gray}{\rmlatinfont\textsuperscript{§~\theparCount}}}
	\pend% ending standard par
      ‚{\tiny $_{lb}$}‚

	  
	  \pstart \leavevmode% starting standard par
	न‚नु य‚ज्ज्वालाप्र‚भ‚व‚म्व‚ह्ने\edtext{}{\edlabel{pvsvt_445-4}\label{pvsvt_445-4}\lemma{ह्ने}\Bfootnote{\cite{pvb-B} रूपं व‚न्हे० ।}}र्न त‚द‚र‚णिनिर्म‚थ‚न‚प्र‚भ‚व‚मिति क‚{\tiny $_{४}$}‚थं न विरोध‚{\tiny $_{lb}$}‚ इत्याह । \textbf{सोपी}त्यादि । \textbf{सोप्य‚न्योन्य‚व्य‚तिरेकी} प‚र‚स्प‚र‚विरुद्धो \textbf{ध‚र्म}द्व‚य‚स्य ज्वाला‚{\tiny $_{lb}$}‚प्र‚भ‚व‚त्वार‚णिनिर्म‚थ‚न‚प्र‚भ‚व‚त्व‚ल‚क्ष‚ण‚स्या\textbf{व‚तारो}व‚काशो \textbf{व‚स्तुसामान्ये\edtext{}{\edlabel{pvsvt_445-5}\label{pvsvt_445-5}\lemma{स्तुसामान्ये}\Bfootnote{\cite{pvb-B} अग्निसामान्य-- added.}}ऽविरुद्ध इत्यु‚{\tiny $_{lb}$}‚च्य‚ते । नाव‚स्थाभेदिनि} व‚ह्नि\textbf{विशेषे} ज्वालाज‚न्म‚न्य‚र‚णिनिर्म‚थ‚न‚ज‚न्म‚नि \textbf{वा}ऽविरुद्ध‚{\tiny $_{lb}$}‚ उच्य‚ते । किन्तु विरुद्ध एव । किं कार‚णं [।] \textbf{निष्क‚ल‚स्यात्म‚नो} निर्विभाग‚स्य स्व‚{\tiny $_{५}$}‚भा‚{\tiny $_{lb}$}‚व‚स्य \textbf{त‚द‚त‚त्त्व‚विरोधात्} । ज्वालाज‚न्म‚नो हि ज्वालापूर्व‚क‚त्व‚म‚त‚त्पूर्व‚क‚त्वं च विरुध्य‚ते ।‚{\tiny $_{lb}$}‚ अर‚णिज‚न्म‚न‚श्चार‚णिपू\edtext{}{\edlabel{pvsvt_445-6}\label{pvsvt_445-6}\lemma{णिपू}\Bfootnote{\cite{pvb-B} र‚णि० ।}} र्व‚क‚त्व‚म‚त‚त्पूर्व‚क‚त्वं च विरुध्य‚ते प‚थिकाग्निसामान्येपि‚{\tiny $_{lb}$}‚ द्व‚य‚म्विरुध्य‚त इत्याह\edtext{}{\edlabel{pvsvt_445-7}\label{pvsvt_445-7}\lemma{इत्याह}\Bfootnote{\cite{pvb-B} इत्य‚त आह ।}} । \textbf{न चे}त्यादि । \textbf{ज्वालेत‚र‚ज‚न्म‚नोः} ज्वालोत्पाद‚स्यार‚णि‚{\tiny $_{lb}$}‚निर्म‚थ‚नोत्पाद‚स्य च \textbf{प‚थिकाग्नौ} प‚थिकाग्निसामान्ये \textbf{बाध्य‚बाध‚क‚ता}\edtext{}{\edlabel{pvsvt_445-8}\label{pvsvt_445-8}\lemma{थिकाग्निसामान्ये}\Bfootnote{\cite{pvb-B} भाव‚ता ।}} । किं कार‚णं‚{\tiny $_{lb}$}‚ [।] \textbf{त‚स्य} प‚थिकाग्नि‚{\tiny $_{६}$}‚सामान्य‚स्य \textbf{ज्वालाप्र‚भ‚व‚व्य‚तिरेकेण} ज्वालोत्प‚त्तिव्य‚तिरे‚{\tiny $_{lb}$}‚केण\edtext{}{\edlabel{pvsvt_445-9}\label{pvsvt_445-9}\lemma{केण}\Bfootnote{\cite{pvb-B} ज्वालो-- Omitted.}} योऽस‚म्भ‚व‚स्त‚स्याभावात्\edtext{}{\edlabel{pvsvt_445-10}\label{pvsvt_445-10}\lemma{स्याभावात्}\Bfootnote{\cite{pvb-B} योऽस‚म्भ‚व‚स्त‚स्याभावाद‚र‚णिनिर्भ‚थ‚नाद‚पि-- added.}} । ज्वालोत्प‚त्तिव्य‚तिरेकेणार‚णिनिर्म‚थ‚नाद‚पि\edtext{}{\edlabel{pvsvt_445-11}\label{pvsvt_445-11}\lemma{पि}\Bfootnote{\cite{pvb-B} दृष्ट एवंभूतोऽन्योपि ज्वालाप्र‚भ‚व-- added.}}‚{\tiny $_{lb}$}‚ भावादित्य‚र्थः । यादृश‚स्तु \textbf{ज्वालाप्र‚भ‚व इति स्यान्न स‚र्वो} व‚ह्निर‚विशेषेण ।‚{\tiny $_{lb}$}‚ \leavevmode\ledsidenote{\textenglish{446/s}}किं कार‚णं [।] \textbf{त‚त्र} हेतुभेद‚भिन्ने व‚ह्नौ \textbf{विशेष‚प्र‚तिक्षेप‚स्या}र‚णिनिर्म‚थ‚न‚कृत‚विशेषा‚{\tiny $_{lb}$}‚\leavevmode\ledsidenote{\textenglish{159b/PSVTa}} प‚ह्न‚व‚{\tiny $_{७}$}‚स्य \textbf{क‚र्त्तुम‚श‚क्य‚त्वात् । स‚म्भ}व‚त्य‚र‚णिकृतो विशेषो य‚स्य व‚ह्नेस्त‚स्य च \textbf{ताद‚{\tiny $_{lb}$}‚व‚स्थ्यानिय‚मात्} । ज्वालाप्र‚भ‚व‚त्व‚ल‚क्ष‚णायाम‚व‚स्थायान्निय‚माभावात् ।
	{\color{gray}{\rmlatinfont\textsuperscript{§~\theparCount}}}
	\pend% ending standard par
      ‚{\tiny $_{lb}$}‚

	  
	  \pstart \leavevmode% starting standard par
	य‚द‚प्युक्त‚म् [।] आद्यः प‚थिकाग्निर्विना ज्वाल‚या य‚दि \textbf{स्याद‚न्य‚त्रा}पि ज्वाला‚{\tiny $_{lb}$}‚र‚हितेपि प्र‚देशे \textbf{स्यादिति} ।
	{\color{gray}{\rmlatinfont\textsuperscript{§~\theparCount}}}
	\pend% ending standard par
      ‚{\tiny $_{lb}$}‚

	  
	  \pstart \leavevmode% starting standard par
	अत्रोच्य‚ते [।] भ‚व‚त्येव\edtext{}{\edlabel{pvsvt_446-1}\label{pvsvt_446-1}\lemma{त्येव}\Bfootnote{\cite{pvb-B} अत्रोच्य‚ते । अन्य‚त्र ज्वालार‚हितेपि प्र‚देशे प‚थिकाग्निर्भ‚व‚त्येव ।}} । \textbf{य‚या}र‚णिनिर्म‚थ‚न‚ल‚क्ष‚ण‚या \textbf{साम‚ग्र्या स} प‚थि‚{\tiny $_{lb}$}‚काग्निः \textbf{स‚म्भ‚व‚ति सा} साम‚ग्री \textbf{य‚दि स्यात्‚{\tiny $_{१}$}‚} । य‚दि पुन\textbf{र‚स्याः} साम‚ग्र्याः \textbf{स‚म्भ‚वं‚{\tiny $_{lb}$}‚ प्र‚द‚र्श्य त‚द‚भाव}म्व‚ह्न्य‚भावं \textbf{क‚थ‚येत् । त‚त्र वा} य‚थोक्त‚साम‚ग्रीस‚म्भ‚विनि देशे‚{\tiny $_{lb}$}‚ \textbf{ज्वालान्द‚र्श‚येत् [।] त‚दा स्यादेव} ज्वालापूर्व‚क‚त्व‚मेव व‚ह्नेर्न चैवं । त‚स्मान्न स‚र्वः‚{\tiny $_{lb}$}‚ प‚थिकाग्निर्ज्वालापूर्व‚क इति व्य‚भिचारः ।
	{\color{gray}{\rmlatinfont\textsuperscript{§~\theparCount}}}
	\pend% ending standard par
      ‚{\tiny $_{lb}$}‚

	  
	  \pstart \leavevmode% starting standard par
	य‚त एव\textbf{न्त‚स्मान्नैक‚स्य} वेद‚क्रियाश‚क्तिर‚हित‚स्य \textbf{प‚र‚पूर्व‚क‚मु}प‚देष्टृपूर्व‚क\textbf{म‚ध्य‚य‚{\tiny $_{lb}$}‚न}न्दृष्टं \textbf{स‚र्व‚स्य} हि र ण्य ग‚र्भा‚{\tiny $_{२}$}‚देर‚प्य‚ध्य‚य‚न‚स्य \textbf{त‚थाभावं} प‚र‚पूर्व‚क‚त्वं \textbf{साध‚य‚ति} ।‚{\tiny $_{lb}$}‚ किं कार‚णं । \textbf{त‚स्या}ध्य‚य‚न‚स्या\textbf{न्य‚था} प‚र‚पूर्व‚क‚त्व‚म‚न्त‚रेण यो \textbf{स‚म्भ‚व}स्त‚स्या\textbf{भावात्} ।‚{\tiny $_{lb}$}‚ स्व‚य‚मुप‚र‚च‚य्याध्य‚य‚नं न स‚म्भ‚वेदित्य‚र्थः । हिर‚ण्य‚ग‚र्भादीनाम्वेद‚र‚च‚नायां श‚क्ति‚{\tiny $_{lb}$}‚स‚म्भ‚वात् । य‚स्तु श‚क्तिविक‚ल इदानीन्त‚न‚स्त‚स्य त‚थाविध‚स्य स्व‚यं कृत्वा वेद‚म‚{\tiny $_{lb}$}‚ध्येतुम‚स‚म‚र्थ‚स्य । \textbf{त‚त्क्रिया} वेद‚{\tiny $_{३}$}‚क्रिया त‚स्यां या \textbf{प्र‚तिभा} त‚या \textbf{र‚हि}त‚स्य वेद‚क‚र‚ण‚{\tiny $_{lb}$}‚स‚म‚र्थ‚या बुद्ध्या र‚हित‚स्येत्य‚र्थः । \textbf{एवंभूत‚स्य} पुरुष‚स्य य‚द‚ध्य‚य‚न‚न्त\textbf{त्त‚था} स्याद‚{\tiny $_{lb}$}‚ध्य‚य‚नान्त‚र‚पूर्व‚कं स्या\textbf{दिति} कृत्वा । \textbf{त‚थाभूत‚मिति} वेद‚क्रियाश‚क्तिर‚हित‚स्य य‚द‚ध्य‚य‚{\tiny $_{lb}$}‚\textbf{न‚न‚न्त‚देव} । एवं वाक्यं स्याद‚ध्य‚य‚नान्त‚र‚पूर्व‚कं \textbf{वाच्यं स्यान्नाविशेषेण} । य‚त्पुन‚र‚ध्य‚य‚नं‚{\tiny $_{lb}$}‚ ‚{\tiny $_{lb}$}‚  ‚{\tiny $_{lb}$}‚ ‚{\tiny $_{lb}$}‚ \leavevmode\ledsidenote{\textenglish{447/s}}पुरुषातिश‚य‚स‚म्भ‚वेन‚{\tiny $_{४}$}‚ स्व‚यं कृत्वाध्य‚य‚नात् \textbf{स‚म्भ‚व‚द्विशेषं} । अविशेषेण स‚र्व‚म‚{\tiny $_{lb}$}‚ध्य‚य‚न‚म‚ध्य‚य‚नान्त‚र‚पूर्व‚क‚मित्\textbf{युच्य‚मानं} व्याप्त्य‚सिद्ध्या व्य‚भिचारित्वान्न \textbf{छायां‚{\tiny $_{lb}$}‚ पुष्णाति} । विव‚क्षित‚साध्यासाध‚नात् ।
	{\color{gray}{\rmlatinfont\textsuperscript{§~\theparCount}}}
	\pend% ending standard par
      ‚{\tiny $_{lb}$}‚

	  
	  \pstart \leavevmode% starting standard par
	\textbf{क‚थ‚मि}त्यादि प‚रः । \textbf{विशेष‚स्य} स्व‚यं कृत्वा वेद‚बाह्यानाम‚ध्य‚य‚न‚स्य \textbf{स‚म्भ‚वः}‚{\tiny $_{lb}$}‚ क‚थं । \textbf{याव‚ते}ति य‚देत्य‚र्थः । \textbf{तेषाम‚पि पुरुषाणां} वेद‚स्य क‚र्त्तृत्वेनाभिम‚तानाम्वेद‚{\tiny $_{lb}$}‚र‚{\tiny $_{५}$}‚च‚नाया\textbf{म‚श‚क्तिः} पुरुष‚त्वा\textbf{दिदानीन्त‚न‚पुरुष‚व‚त्} ।
	{\color{gray}{\rmlatinfont\textsuperscript{§~\theparCount}}}
	\pend% ending standard par
      ‚{\tiny $_{lb}$}‚

	  
	  \pstart \leavevmode% starting standard par
	\textbf{अत्रा}पीत्या चा र्यः । अत्रापि प्र‚योगे । न \textbf{श‚क्ति}पुरुष‚योरिति । वेद‚क‚र‚{\tiny $_{lb}$}‚ण‚स्य श‚क्तेः पुरुष‚स्य च प‚र‚स्प‚रं । \textbf{न किञ्चिद् विरोध‚द‚र्श‚न‚म‚स्ति । त‚त‚श्च}‚{\tiny $_{lb}$}‚ ब्र ह्मा दिषु पुरुष‚त्वं हेतुत्वेनोक्त\textbf{म‚विरुद्ध‚त्वान्न} वेद‚क‚र‚ण‚श‚क्तिम‚प‚न‚य‚ति । \textbf{त‚स्मान्ना‚{\tiny $_{lb}$}‚विरुद्ध‚विधिः} । अविरुद्ध‚स्य विधिर्य‚स्मिन्न‚नु‚{\tiny $_{६}$}‚प‚ल‚ब्धिप्र‚योगे स एवंभूतो\textbf{नुप‚ल‚ब्धि‚{\tiny $_{lb}$}‚प्र‚योगो न ग‚म‚कः} । विरोधाभाव एव क‚थ‚मित्याह । \textbf{न ही}त्यादि । \textbf{अतीन्द्रिये}ष्व‚{\tiny $_{lb}$}‚त्य‚न्त‚प‚रोक्षेषु ब्र‚ह्मादिषु वेद‚क‚र‚ण‚श‚क्त्या स‚ह । स‚हान‚व‚स्थान‚ल‚क्ष‚ण‚स्य \textbf{विरोध‚स्य‚{\tiny $_{lb}$}‚ प्र‚तीतिः} । अतीन्द्रिय‚त्वादेव । नापि प‚र‚स्प‚र‚प‚रिहार‚स्थितिल‚क्ष‚ण‚स्य विरोध‚स्य‚{\tiny $_{lb}$}‚ प्र‚तीतिः । श‚क्त्य‚श‚क्त्योः पुरुषापुरुष‚{\tiny $_{७}$}‚त्व‚योश्च प‚र‚स्प‚र‚म्विरोधात् । न श‚क्ति- \leavevmode\ledsidenote{\textenglish{160a/PSVTa}}‚{\tiny $_{lb}$}‚ पुरुष‚योर्यः पुरुषः स वेद‚क‚र‚णं प्र‚त्य‚श‚क्तो य‚थेदानीन्त‚नः पुरुष इति । \textbf{न चायं प्र‚योगः}‚{\tiny $_{lb}$}‚ पूर्व‚प्र‚योगादिति । य‚द्वेदाध्य‚य‚न‚न्त‚द्वेदाध्य‚य‚न‚पूर्व‚क‚मिदानीन्त‚न‚वेदाध्य‚य‚न‚व‚दित्येत‚स्मात्‚{\tiny $_{lb}$}‚ \textbf{प‚र्व‚प्र‚योगाद् भिद्य‚ते} । त‚स्मादुभ‚योरुपादानं व्य‚र्थ‚मेवेत्य‚भिप्रायः ।
	{\color{gray}{\rmlatinfont\textsuperscript{§~\theparCount}}}
	\pend% ending standard par
      ‚{\tiny $_{lb}$}‚

	  
	  \pstart \leavevmode% starting standard par
	य‚त्पुन‚रुच्य‚ते । \textbf{य‚दि पुरुषाः} प्राक्त‚ना वेदं कृत्वा स्व‚य‚{\tiny $_{१}$}‚म‚ध्येतुं \textbf{श‚क्ताः स्युस्त‚{\tiny $_{lb}$}‚देदानीन्त‚ना अपि स्युरिति} ।
	{\color{gray}{\rmlatinfont\textsuperscript{§~\theparCount}}}
	\pend% ending standard par
      ‚{\tiny $_{lb}$}‚

	  
	  \pstart \leavevmode% starting standard par
	अत्रोच्य‚ते । पुरुषाणा\textbf{म्विशेषास‚म्भ‚वे स‚त्येत‚द}न‚न्त‚रोक्तं \textbf{स्यात् । स च}‚{\tiny $_{lb}$}‚ पुरुषाणां विशेषास‚म्भ‚वाद्\textbf{दुस्साध्य}श‚क्य‚साध‚नः । बाध‚काभावात् । त‚स्माद्‚{\tiny $_{lb}$}‚ \textbf{य‚त्रैक‚स्य} पुरुष‚स्याश\textbf{क्तिस्त‚त्र स‚र्व‚पुरुषाणाम‚श‚क्तिः} [।] पुरुष‚त्वा\textbf{दित्य}स्मिन्न‚पि साध‚ने‚{\tiny $_{lb}$}‚ \textbf{पूर्व‚व‚द}ध्य‚न‚त्वादिव‚त् पुरुष‚त्वं लिङ्गं \textbf{व्य‚भिचारि} । किं कार‚णं । \textbf{भा र ता दिष्व‚पि‚{\tiny $_{२}$}‚}‚{\tiny $_{lb}$}‚ पौरुषेयाभिम‚तेष्\textbf{विदानीन्त‚नानां} पुरुषाणा\textbf{म‚श‚क्ताव‚पि क‚स्य‚चिद्} व्या सा देः पुरुषा‚{\tiny $_{lb}$}‚‚{\tiny $_{lb}$}‚ \leavevmode\ledsidenote{\textenglish{448/s}}तिश‚य‚स्य \textbf{श‚क्तिसिद्धेः} ।
	{\color{gray}{\rmlatinfont\textsuperscript{§~\theparCount}}}
	\pend% ending standard par
      ‚{\tiny $_{lb}$}‚

	  
	  \pstart \leavevmode% starting standard par
	य‚त एव\textbf{न्त‚स्मात् कार‚णानि विवेच‚य‚ता} वैदिकानां वाक्यानां ताल्वादिव्यापारं‚{\tiny $_{lb}$}‚ कार‚ण‚म‚प‚न‚य‚ता । \textbf{अर्थेषु} लौकिक‚वैदिकेषु श‚ब्देषु [।] किम्भूतेषु । \textbf{त‚द‚त‚त्प्र‚तिभ‚वेषु}‚{\tiny $_{lb}$}‚ ताल्वादिकार‚णेष्व‚त‚त्कार‚णेषु \textbf{स्व‚भाव‚भेदो द‚र्श‚नीयः} । ताल्वादिकार‚णानामी‚{\tiny $_{३}$}‚‚{\tiny $_{lb}$}‚ दृशः स्व‚भावो त‚त्कार‚णानाम‚न्यादृशः स्व‚भाव इत्येवं स्व‚भाव‚नानात्वं द‚र्श‚नीयं ।‚{\tiny $_{lb}$}‚ येन त‚द‚त‚त्प्र‚भ‚व‚त्व‚म्विभागेन जाय‚ते । \textbf{त‚द‚भावे} स्व‚भाव‚भेदाभावे \textbf{स‚र्व‚स्त‚दात्मा‚{\tiny $_{lb}$}‚ भ‚वेत्} । स‚र्वः श‚ब्दः पौरुषेयः स्या\textbf{न्न वा क‚श्चि}ल्लौकिकोपि ।
	{\color{gray}{\rmlatinfont\textsuperscript{§~\theparCount}}}
	\pend% ending standard par
      ‚{\tiny $_{lb}$}‚

	  
	  \pstart \leavevmode% starting standard par
	अथ स्याद् [।] अस्त्येव त‚योः स्व‚भाव‚भेद इत्याह । \textbf{न चात्रे}त्यादि । अत्र‚{\tiny $_{lb}$}‚ ज‚ग‚ति \textbf{लौकिक‚वैदिक‚योर्वाक्य‚योः स्व‚भाव‚नानात्वं प‚{\tiny $_{४}$}‚श्यामः । अस‚ति त‚स्मिन्}‚{\tiny $_{lb}$}‚ स्व‚रूप‚भेदे \textbf{त‚यो}र्लौकिक‚वैदिक‚वाक्य‚यो\textbf{स्सामान्य‚स्यैव} तुल्य‚रूप‚स्यैव व‚र्ण्णानुक्र‚म‚{\tiny $_{lb}$}‚ल‚क्ष‚ण‚स्य \textbf{द‚र्श‚नादेक‚स्य} लौकिक‚वैदिक‚स्य वाक्य‚स्य \textbf{कंचिद् ध‚र्म्मं विवेच‚य‚न्}‚{\tiny $_{lb}$}‚ पौरुषेय‚त्व‚म‚पौरुषेय‚त्व‚म्वा विभागेन व्य‚व‚स्थाप‚य‚न् पुरुष \textbf{आशंक्य व्याभिचार‚वादः‚{\tiny $_{lb}$}‚ क्रिय‚ते । आशंक्य} व्य‚भिचारो वादो य‚स्य पुरुष‚स्य स त‚थोच्य‚{\tiny $_{५}$}‚ते । केन क्रिय‚ते ।‚{\tiny $_{lb}$}‚ \textbf{त‚त्स्व‚भाव‚स‚म्भ‚विना} तेन । लौकिक‚वैदिक‚वाक्य‚स‚म्भ‚विना तेन व‚र्ण्ण‚प‚द‚र‚च‚ना‚{\tiny $_{lb}$}‚ल‚क्ष‚णेन सामान्येन । पौरुषेय‚तुल्य‚ध‚र्म‚क‚स्य वेद‚स्यापौरुषेय‚त्व‚म्व‚द‚न् व्य‚भिचार्य‚ते‚{\tiny $_{lb}$}‚ इति याव‚त् ।
	{\color{gray}{\rmlatinfont\textsuperscript{§~\theparCount}}}
	\pend% ending standard par
      ‚{\tiny $_{lb}$}‚

	  
	  \pstart \leavevmode% starting standard par
	न‚नु \textbf{वेदावेद‚योस्त}त्त्वान्य‚त्त्व\textbf{ल‚क्ष‚णो} वेदावेद‚ल‚क्ष‚णो \textbf{विशेषो}स्त्येव । त‚तो विशे‚{\tiny $_{lb}$}‚षाल्लौकिक‚वैदिक‚योर्य‚थाक्र‚मं पौरुषेय‚त्व‚म‚पौरुषेय‚{\tiny $_{६}$}‚त्व‚म्भ‚विष्य‚तीति प‚रो म‚न्य‚ते ।
	{\color{gray}{\rmlatinfont\textsuperscript{§~\theparCount}}}
	\pend% ending standard par
      ‚{\tiny $_{lb}$}‚

	  
	  \pstart \leavevmode% starting standard par
	\textbf{स‚त्त्य}मित्या चा र्यः । न‚न्वीदृशो विशेष\textbf{स्त‚योः} पौरुषेय‚त्वापौरुषेय‚त्व‚साध‚को‚{\tiny $_{lb}$}‚ य‚स्मा\textbf{न्न केव‚ल‚म‚न‚योरेव} लौकिक‚वैदिक‚योर्विशेषः । किन्त‚र्हि [।] \textbf{डि ण्डि क‚{\tiny $_{lb}$}‚ पुराणेत‚र‚योर‚पि} । डि ण्डि कै र्न‚ग्नाचार्यैः कृत‚स्य पुराण‚स्येत‚र‚स्य च पुराण‚स्य । ईदृशो‚{\tiny $_{lb}$}‚ ‚{\tiny $_{lb}$}‚ \leavevmode\ledsidenote{\textenglish{449/s}}विशेषोस्ति । \textbf{न च ताव‚ता} स्व‚यं‚{\tiny $_{७}$}‚ व्य‚व‚हारार्थं \textbf{स्व‚प्र‚क्रियाभेद‚दीप‚नः} स‚म‚य‚प‚रिक‚ल्पितो \leavevmode\ledsidenote{\textenglish{160b/PSVTa}}‚{\tiny $_{lb}$}‚ \textbf{नाम‚भेदः} संज्ञाभेदः पुरुष‚कृति\textbf{म्बाध‚ते} वेद‚स्य । किं कार‚ण‚म् [।] \textbf{अन्य‚त्रापि}‚{\tiny $_{lb}$}‚ पुराणेऽपौरुषेय‚त्व\textbf{प्र‚संगात्} । डिण्डिकेत‚र‚पुराणानां नाम‚भेद‚स्य विद्य‚मान‚त्वात् ।
	{\color{gray}{\rmlatinfont\textsuperscript{§~\theparCount}}}
	\pend% ending standard par
      ‚{\tiny $_{lb}$}‚

	  
	  \pstart \leavevmode% starting standard par
	\textbf{य‚दि} तु या वेद‚वाक्ये व‚र्ण्ण‚प‚द‚र‚च‚ना दृश्य‚ते \textbf{तादृशीं र‚च‚नां पुरुषाः क‚र्त्तु न‚{\tiny $_{lb}$}‚ श‚क्नुयुः । कृताम्वा} निष्पादिताम्वा व‚र्ण्ण‚प‚द‚र‚च‚ना‚{\tiny $_{१}$}‚ \textbf{म‚कृत‚संकेतः} श्र‚व‚ण‚मात्राद्‚{\tiny $_{lb}$}‚ \textbf{विवेच‚येदि}यं पुरुष‚पूर्विकेति । \textbf{त‚दा व्य‚क्त‚म‚पौरुषेयो वेदः स्यात्} [।] न विवे‚{\tiny $_{lb}$}‚च‚य‚ति तां र‚च‚नान्त‚त्क‚थ‚म‚पौरुषेयो वेदः स्यात् ।
	{\color{gray}{\rmlatinfont\textsuperscript{§~\theparCount}}}
	\pend% ending standard par
      ‚{\tiny $_{lb}$}‚

	  
	  \pstart \leavevmode% starting standard par
	\textbf{न‚न्वि}त्यादि प‚रः । म न्त्रा अपि पुरुष‚कृता \textbf{एवेत्येत‚दुत्त‚र‚त्र विचार‚यिष्यामः ।‚{\tiny $_{lb}$}‚ अपि च न म‚न्त्रो नामान्य‚देव किञ्चित्} । किन्त‚र्हि [।] \textbf{स‚त्त्ये}त्यादि । य‚थाभूता‚{\tiny $_{lb}$}‚ख्यानं स‚त्त्यं । इन्द्रिय‚म‚न‚सोर्द‚म‚न‚न्त‚पः । त‚योः प्र‚भावो‚{\tiny $_{२}$}‚ विष‚स्त‚म्भ‚नादिसाम‚र्थ्यं‚{\tiny $_{lb}$}‚ स विद्य‚ते येषां पुंसान्ते त‚था । तेषां \textbf{स‚त्त्य‚त‚पःप्र‚भाव‚व‚तां} पुंसां \textbf{स‚मीहितार्थ‚स्य‚{\tiny $_{lb}$}‚ साध‚न}न्त‚देव म‚न्त्रः । त‚द्व‚च‚नं म‚न्त्र‚ल‚क्ष‚ण‚म\textbf{द्य‚त्वेपि पुरुषेषु दृश्य‚त एव} । किं कार‚णं ।‚{\tiny $_{lb}$}‚ \textbf{य‚थास्वं स‚त्त्याधिष्ठान‚ब‚लाद् विष‚द‚ह‚नादे\href{http://sarit.indology.info/?cref=}{ः} स्त‚म्भ‚न‚स्य} साम‚र्थ्योप‚घात‚स्य‚{\tiny $_{lb}$}‚ द‚र्श‚नात् । त‚था श ब रा णां च \textbf{केषांचित्} स्व‚निय‚म‚स्थानाम‚द्यापि विषाद्य‚प‚{\tiny $_{lb}$}‚न‚य‚न‚श‚क्तियु‚{\tiny $_{३}$}‚क्त‚स्य कार‚णाच्छ‚क्नुव‚न्त्येव पुरुषा म‚न्त्रान् क‚र्त्तुं । \textbf{अवैदिकानाञ्च}‚{\tiny $_{lb}$}‚ वेदाद‚न्येषां बौ द्धा दी नामिति [।] आदिश‚ब्दाद् आ र्ह त गा रु ड मा हे‚{\tiny $_{lb}$}‚श्व रा दीनां \textbf{म‚न्त्र‚क‚ल्पानां} । म‚न्त्राणां म‚न्त्र‚क‚ल्पाना\textbf{ञ्च द‚र्श‚नात्} । विद्याक्ष‚राणि‚{\tiny $_{lb}$}‚ म‚न्त्राः । त‚त्साध‚न‚विधानोप‚देशा म‚न्त्र‚क‚ल्पाः । \textbf{तेषां च} बौद्धादीना म्म‚न्त्र‚क‚ल्पानां‚{\tiny $_{lb}$}‚ \textbf{पुरुष‚कृतेः} पुरुषैः क‚र‚णात् ।
	{\color{gray}{\rmlatinfont\textsuperscript{§~\theparCount}}}
	\pend% ending standard par
      ‚{\tiny $_{lb}$}‚

	  
	  \pstart \leavevmode% starting standard par
	त‚स्मान्न लौकिभ्यो वैदिकानां स्व‚भाव‚भेदः ।
	{\color{gray}{\rmlatinfont\textsuperscript{§~\theparCount}}}
	\pend% ending standard par
      ‚{\tiny $_{lb}$}‚

	  
	  \pstart \leavevmode% starting standard par
	\textbf{त‚त्रे}त्यादि प‚रः । त‚त्रापि बौद्धादिम‚न्त्र‚क‚ल्पेप्\textbf{य‚पौरुषेय‚त्वे} क‚ल्प्य‚माने । \textbf{क‚थ}‚{\tiny $_{lb}$}‚मिदानी\textbf{म‚पौरुषेयं} वाक्यं स‚र्व\textbf{म‚वित‚थं} । किन्तु मिथ्यार्थ‚म‚पि स्यात् । \textbf{त‚था ही}त्या‚{\tiny $_{lb}$}‚‚{\tiny $_{lb}$}‚ \leavevmode\ledsidenote{\textenglish{450/s}}दिनैत‚देव बोध‚य‚ति । \textbf{बौद्ध‚म‚न्त्र‚क‚ल्पे हिंसा मैथुनात्म‚द‚र्श‚नाद}यः आदिश‚ब्दाद‚नृत‚{\tiny $_{lb}$}‚व‚च‚नाद‚य \textbf{अन‚भ्युद‚य‚हेत‚वो} दुःख‚हेत‚वो \textbf{व‚र्ण्य‚न्ते} । इत‚र‚स्मिंस्त्व‚बौद्ध‚म‚न्त्र‚क‚ल्पे त‚{\tiny $_{lb}$}‚ एव हिंसाद‚योन्य‚{\tiny $_{५}$}‚था चाभ्युद‚य‚हेत‚वो व‚र्ण्य‚य‚न्ते । य‚दि च स‚र्वे म‚न्त्र‚क‚ल्पा‚{\tiny $_{lb}$}‚ अपौरुषेयाः स्युस्त‚दा चैत‚द् \textbf{विरुद्धाभिधा}यि वाक्य\textbf{द्व‚य‚मेक‚त्रा}पौरुषेये \textbf{क‚थं‚{\tiny $_{lb}$}‚ स‚त्त्यं स्यात्} ।
	{\color{gray}{\rmlatinfont\textsuperscript{§~\theparCount}}}
	\pend% ending standard par
      ‚{\tiny $_{lb}$}‚

	  
	  \pstart \leavevmode% starting standard par
	स्यादेत‚द् [।] बौद्ध‚म‚न्त्र‚क‚ल्पें हिंसादिश‚ब्दानाम‚न्य एवाप्र‚सिद्धोर्थो यो वैदिकेन‚{\tiny $_{lb}$}‚ म‚न्त्र‚क‚ल्पेनाविरुद्ध इति [।] अत आह । \textbf{त‚त्रे}त्यादि । त‚त्र बौद्धे म‚न्त्र‚क‚ल्पे‚{\tiny $_{lb}$}‚ प्र‚सिद्धाद‚र्थाद‚न्य\textbf{स्यार्थान्त‚र‚स्य क‚ल्प‚ने} क्रिय‚माणे । \textbf{त‚द}र्थान्त‚र‚{\tiny $_{६}$}‚क‚ल्प‚न\textbf{म‚न्य‚त्रा}बौद्धे‚{\tiny $_{lb}$}‚ वैदिके म‚न्त्र‚क‚ल्पे \textbf{तुल्य‚मि}ति कृत्वा स‚र्व‚त्र म‚न्त्र‚क‚ल्पेष्व‚र्थान्त‚र‚क‚ल्प‚नास‚म्भ‚वेनार्था‚{\tiny $_{lb}$}‚निर्ण्ण‚यात् । त‚त्प्र‚तिपादितेर्थे क्व‚चित् प्र‚तिप‚त्तिर‚नुष्ठानं न स्यात् । \textbf{त‚था चे‚{\tiny $_{lb}$}‚त्य‚र्थानिश्च}येनानुष्ठानाभावे स‚द‚प्य\textbf{पौरुषेय}म्वाक्यं पुरुषार्थं प्र‚त्य\textbf{नुप‚योगं} ।
	{\color{gray}{\rmlatinfont\textsuperscript{§~\theparCount}}}
	\pend% ending standard par
      ‚{\tiny $_{lb}$}‚

	  
	  \pstart \leavevmode% starting standard par
	\textbf{बौद्धादीनां म‚न्त्र‚त्व‚मेव नास्तीति चेदाह । बौद्धादीनाम‚म‚न्त्र‚त्व} इति ।‚{\tiny $_{lb}$}‚ \leavevmode\ledsidenote{\textenglish{161a/PSVTa}} \textbf{त‚द‚न्य‚त्रापि} त‚स्माद् बौद्धा‚{\tiny $_{७}$}‚दिम‚न्त्राद‚न्य‚त्रापि वैदिके म‚न्त्रे म‚न्त्र‚त्व‚प्र‚तिपाद‚नाय‚{\tiny $_{lb}$}‚ \textbf{कोश‚पानं क‚र‚णीयं} । न हि काचिद् व्य‚क्तिर‚स्तीत्य‚भिप्रायः । दृष्ट‚विरुद्धं चैत‚द्‚{\tiny $_{lb}$}‚ बौद्धाद‚यो न म‚न्त्रा इति । त‚था हि \textbf{विषादिक‚र्म‚कृतो} विष‚क‚र्मादीन् कुर्व‚न्तो \textbf{बौद्धा‚{\tiny $_{lb}$}‚ अपि} म‚न्त्रा \textbf{दृश्य‚न्ते} । तेन \textbf{त‚त्र} बौद्धादिषु म‚न्त्र‚क‚ल्पेष्व\textbf{म‚न्त्र‚त्व‚म‚पि} विप्र‚तिषिद्धं ।‚{\tiny $_{lb}$}‚ विष‚क‚र्मादिक‚र‚ण‚द्वारेण वैदिकानाम‚पि म‚न्त्र‚त्व‚व्य‚व‚स्थाप‚नात् । न च‚{\tiny $_{१}$}‚ विष‚स्त‚म्भ‚ना‚{\tiny $_{lb}$}‚दिसाम‚र्थ्य‚योगात् वेद‚वाक्यं लौकिवाक्याद‚तिश‚य‚व‚दित्येवापौरुषेयं युक्तं ।
	{\color{gray}{\rmlatinfont\textsuperscript{§~\theparCount}}}
	\pend% ending standard par
      ‚{\tiny $_{lb}$}‚

	  
	  \pstart \leavevmode% starting standard par
	त‚था हि पाण्य‚ङ्गुल‚स‚न्निवेशो \textbf{मु द्रा । म ण्ड लं} देव‚तादिर‚च‚नाविशेषः ।‚{\tiny $_{lb}$}‚ \textbf{ध्यान}न्देव‚तादिरूप‚चिन्त‚नं । \textbf{तैर‚न‚क्ष‚रै}र‚श‚ब्द‚स्व‚भावैः स्व\textbf{क‚र्माणि} विषाद्य‚प‚न‚य‚ना‚{\tiny $_{lb}$}‚दिल‚क्ष‚णानि क्रिय‚न्ते । न च तानि मुद्राम‚ण्ड‚ल‚ध्याना\textbf{न्य‚पौरुषेयाणि} युज्य‚न्ते‚{\tiny $_{२}$}‚ [।]
	{\color{gray}{\rmlatinfont\textsuperscript{§~\theparCount}}}
	\pend% ending standard par
      ‚{\tiny $_{lb}$}‚

	  
	  \pstart \leavevmode% starting standard par
	स्यादेत‚त् [।] मुद्रादिष्वेव पुंसां क‚र‚ण‚साम‚र्थ्य‚न्न व‚र्ण्ण‚क्र‚मेषु म‚न्त्रेष्विति ।
	{\color{gray}{\rmlatinfont\textsuperscript{§~\theparCount}}}
	\pend% ending standard par
      ‚{\tiny $_{lb}$}‚‚{\tiny $_{lb}$}‚\textsuperscript{\textenglish{451/s}}

	  
	  \pstart \leavevmode% starting standard par
	त‚न्न । य‚स्मात् \textbf{तेषां} मुद्रादीनां \textbf{क्रियास‚म्भंवे स‚त्य‚क्ष‚र‚र‚च‚नायां} स‚त्त्यादिम‚तां‚{\tiny $_{lb}$}‚ पुंसां \textbf{कः प्र‚तिघातो} विशेषाभावात् । \textbf{त‚स्मान्न किञ्चिद‚श‚क्य‚क्रिय‚मेषां} पुंसां ।‚{\tiny $_{lb}$}‚ येन पुरुषेणाकृत‚म‚तिश‚य‚मुप‚ल‚भ्य लौकिकेभ्यो वैदिकानां स्व‚भाव‚भेदः क‚ल्प्येत ।
	{\color{gray}{\rmlatinfont\textsuperscript{§~\theparCount}}}
	\pend% ending standard par
      ‚{\tiny $_{lb}$}‚

	  
	  \pstart \leavevmode% starting standard par
	य‚दि बौद्धेत‚रौ म‚न्त्र‚क‚ल्पौ‚{\tiny $_{३}$}‚ द्वाव‚पि पौरुषेयो \textbf{तौ च स‚त्त्य‚प्र‚भ‚वौ} । अवित‚था‚{\tiny $_{lb}$}‚भिधायिपुरुषादुत्प‚न्नौ । \textbf{त‚त्क‚थ}मिदानीन्तावेव स‚त्त्य‚प्र‚भ‚वौ \textbf{म‚न्त्र‚क‚ल्पौ} बौद्धेत‚रौ‚{\tiny $_{lb}$}‚\textbf{प‚र‚स्प‚र‚विरुद्धौ} युज्येते । एक‚त्र हिंसादीनाम‚न‚भ्युद‚य‚हेतुत्वेन द‚र्श‚नाद‚न्य‚त्राभ्यु‚{\tiny $_{lb}$}‚द‚य‚हेतुत्वेन ।
	{\color{gray}{\rmlatinfont\textsuperscript{§~\theparCount}}}
	\pend% ending standard par
      ‚{\tiny $_{lb}$}‚

	  
	  \pstart \leavevmode% starting standard par
	\textbf{ने}त्यादिना प‚रिह‚र‚ति । \textbf{न वै स‚र्व‚त्र तौ} म‚न्त्र‚क‚ल्पौ \textbf{स‚त्त्य‚प्र‚भ‚वौ} येनायं विरोधः ।‚{\tiny $_{lb}$}‚ किन्तु \textbf{प्र‚भाव‚युक्त‚{\tiny $_{४}$}‚पुरुष‚प्र‚तिज्ञाल‚क्ष‚णाव‚पि} तौ म‚न्त्र‚क‚ल्पौ स्तः । प्र‚भाव‚व‚ता पुरुषेण‚{\tiny $_{lb}$}‚ य इमां व‚र्ण्ण‚प‚द‚र‚च‚नाम‚भ्य‚स्य‚ति त‚द्विधिं चानुतिष्ठ‚ति त‚स्याहं य‚थाप्र‚तिज्ञात‚म‚र्थं‚{\tiny $_{lb}$}‚ स‚म्पाद‚यिष्यामीति या प्र‚तिज्ञा त‚ल्ल‚क्ष‚णाव‚पि म‚न्त्र‚क‚ल्पौ भ‚व‚तः । त‚तोन्य‚था‚{\tiny $_{lb}$}‚वाद्य‚पि प्र‚भाव‚युक्तौ म‚न्त्र‚क‚ल्पौ कुर्यादेवेत्य‚विरोधः । स एव स‚त्त्याभावात् प्र‚भावः‚{\tiny $_{lb}$}‚ कुत इति चेदाह । \textbf{स प्र‚{\tiny $_{५}$}‚भावो ग‚तिसिद्धिविशेषाभ्याम‚पि स्यात्} । पुण्येन ग‚ति‚{\tiny $_{lb}$}‚विशेष एव स तादृशो ल‚ब्धो देव‚तादिस‚ङ्गृहीतो म‚न्त्र‚सिद्धिविशेषो येन स तादृशः‚{\tiny $_{lb}$}‚ प्र‚भावो भ‚व‚तीति ।
	{\color{gray}{\rmlatinfont\textsuperscript{§~\theparCount}}}
	\pend% ending standard par
      ‚{\tiny $_{lb}$}‚

	  
	  \pstart \leavevmode% starting standard par
	\textbf{य‚दि पौरुषेया म‚न्त्रास्त‚दा} पुरुष‚त्वात् \textbf{स‚र्वे पुरुषाः किन्न म‚न्त्र‚कारिणः} [।]‚{\tiny $_{lb}$}‚ न च कुर्व‚न्ति । त‚स्माद‚भिम‚ता अपि पुरुषा न म‚न्त्रान‚कार्षुरित्य‚भिप्रायः ।
	{\color{gray}{\rmlatinfont\textsuperscript{§~\theparCount}}}
	\pend% ending standard par
      ‚{\tiny $_{lb}$}‚

	  
	  \pstart \leavevmode% starting standard par
	\textbf{त‚दि}त्यादि सि द्धा न्त वा दी ।‚{\tiny $_{६}$}‚ तेषाम्म‚न्त्राणां य‚त् \textbf{क्रियासाध‚नं} स‚त्त्य‚त‚पःप्र‚भा‚{\tiny $_{lb}$}‚वादि तेन \textbf{वैक‚ल्या}न्न स‚र्वे पुरुषा म‚न्त्र‚कारिणः ।
	{\color{gray}{\rmlatinfont\textsuperscript{§~\theparCount}}}
	\pend% ending standard par
      ‚{\tiny $_{lb}$}‚

	  
	  \pstart \leavevmode% starting standard par
	\textbf{य‚दि पुन‚स्तादृशैः स‚त्त्य‚त‚पःप्र‚भृतिभिर्म}न्त्र‚हेतुभिः पुरुषा \textbf{युक्ताः} स्युस्त‚दा ते‚{\tiny $_{lb}$}‚ म‚न्त्रान् कुर्व‚न्त्येव ।
	{\color{gray}{\rmlatinfont\textsuperscript{§~\theparCount}}}
	\pend% ending standard par
      ‚{\tiny $_{lb}$}‚‚{\tiny $_{lb}$}‚\textsuperscript{\textenglish{452/s}}

	  
	  \pstart \leavevmode% starting standard par
	\textbf{अपि च काव्यानि} त‚त्क्रियाप्र‚तिभायुक्तः \textbf{पुरुषः क‚रोती}ति कृत्वा त‚त्क्रिया‚{\tiny $_{lb}$}‚\leavevmode\ledsidenote{\textenglish{161b/PSVTa}} प्र‚तिभार‚हितोपि \textbf{स‚र्वः पुरुषः} पुरुष‚त्व‚साम्यात् \textbf{काव्य‚कृत् स्यात् । अक‚र‚{\tiny $_{७}$}‚णे वा‚{\tiny $_{lb}$}‚ क‚स्य‚चिद}न्योपि नैव कुर्यात् । त‚द्व‚त् । काव्य‚क‚र‚णास‚म‚र्थ‚पुरुष‚व‚त् । \textbf{इत्य‚पूर्वेषा वा‚{\tiny $_{lb}$}‚ चो युक्तिः} । व्य‚भिचारिणीत्य‚र्थः ।
	{\color{gray}{\rmlatinfont\textsuperscript{§~\theparCount}}}
	\pend% ending standard par
      ‚{\tiny $_{lb}$}‚

	  
	  \pstart \leavevmode% starting standard par
	\textbf{स‚त्त्य}मित्यादि प‚रः । \textbf{म‚न्त्र‚क्रियासाध‚नेन विक‚लाः} पुरुषा \textbf{म‚न्त्रान्न कुर्व‚ते} ।‚{\tiny $_{lb}$}‚ केव‚ल\textbf{न्त‚देवा}त्र म‚न्त्र‚क्रियासाध‚न‚स्य स‚त्त्यादेः \textbf{साक‚ल्यं क‚स्य‚चित्} पुरुष‚स्य \textbf{न प‚श्यामः ।‚{\tiny $_{lb}$}‚ स‚र्व‚पुरुषाणां स‚मान‚ध‚र्म‚त्वात्} ।
	{\color{gray}{\rmlatinfont\textsuperscript{§~\theparCount}}}
	\pend% ending standard par
      ‚{\tiny $_{lb}$}‚

	  
	  \pstart \leavevmode% starting standard par
	\textbf{उक्त‚मि}ति सि द्धा न्त वा दी । \textbf{अ‚{\tiny $_{१}$}‚त्र} चोद्य उक्त‚मुत्त‚रं । किमुक्तं । \textbf{न म‚न्त्रो‚{\tiny $_{lb}$}‚ नामे}त्यादि । व‚च‚नं च स‚म‚य‚श्चेति द्व‚न्द्वः । \textbf{स‚त्त्यादिम‚तां} पुरुषाणां स‚मीहितार्थ‚{\tiny $_{lb}$}‚साध‚ना\textbf{द्व‚च‚नात्} । त‚था स‚त्त्यादियुक्त‚पुरुष‚प्र‚तिज्ञाल‚क्ष‚णाच्च स‚म‚यान्न म‚न्त्रोनामा‚{\tiny $_{lb}$}‚\textbf{न्य‚देव किंचित् । तानि च} स‚त्त्य‚त‚पोग‚तिसिद्धिविशेष‚ल‚क्ष‚णानि म‚न्त्र‚क्रियासाध‚नानि‚{\tiny $_{lb}$}‚ \textbf{क्व‚चिदेव पुरुषेषु दृश्य‚न्ते} ।
	{\color{gray}{\rmlatinfont\textsuperscript{§~\theparCount}}}
	\pend% ending standard par
      ‚{\tiny $_{lb}$}‚

	  
	  \pstart \leavevmode% starting standard par
	स्यादेत‚द् [।] यो यः पुरुष‚स्य‚{\tiny $_{२}$}‚ म‚न्त्र‚क्रियासाध‚न‚र‚हित‚स्त‚द्य‚था र‚थ्यापुरुषः ।‚{\tiny $_{lb}$}‚ पुरुष‚श्चायं म‚न्त्र‚क‚र्त्तृत्वेनाभिम‚तः पुरुष इति । त‚त्रापि \textbf{स‚र्व‚पुरुषास्त‚द्र‚हिता‚{\tiny $_{lb}$}‚स्तेन} म‚न्त्र‚क्रियासाध‚नेन र‚हिता \textbf{इत्य‚पि त‚त्स‚म्भ‚व‚स्य} म‚न्त्र‚क्रियासाध‚न‚स‚म्भ‚व‚स्य‚{\tiny $_{lb}$}‚ पुरुष‚त्वेन स‚ह \textbf{विरोधाभावात् । अनिर्ण्ण}योऽनिश्च‚यः । म‚न्त्र‚क्रियासाध‚न‚स्य‚{\tiny $_{lb}$}‚ स्व‚भावानुप‚ल‚म्भादेव पुरुषे स्व‚भाव‚निश्च‚य इति चेदाह । \textbf{न चे‚{\tiny $_{३}$}‚}त्यादि । \textbf{अत्य‚क्ष‚{\tiny $_{lb}$}‚स्व‚भावेषु} अक्षातिक्रान्तः स्व‚भावो येषान्तेष्व‚त्य‚न्त‚प‚रोक्षेष्वित्य‚र्थः । \textbf{अनुप‚ल‚ब्धि‚{\tiny $_{lb}$}‚र्नाभाव‚निश्च‚य‚स्य हेतुः} । आत्म‚नि म‚न्त्र‚क्रियासाध‚नानां स्मृत्यादीनाम‚नुप‚ल‚म्भेन‚{\tiny $_{lb}$}‚ प‚र‚त्राप्य‚भावो निश्चीय‚त इति चेदाह । \textbf{न चे}त्यादि । अतिक्रान्त‚ज‚न्मादिस्म‚र‚णं‚{\tiny $_{lb}$}‚ स्मृतिः । प‚र‚चित्ताव‚बोधो म तिः । अदृष्टेषु प‚दार्थ‚त‚त्त्व‚द‚र्श‚नं \textbf{प्र‚तिवेधः । स‚त्त्य}‚{\tiny $_{lb}$}‚म‚न‚न्य‚थावा‚{\tiny $_{४}$}‚दित्वं । \textbf{श‚क्तिः} प्र‚भावः । ता म‚न्त्र‚हेत‚वः स‚र्व‚भाविन्यः । \textbf{स‚र्व‚पुरुष}‚{\tiny $_{lb}$}‚‚{\tiny $_{lb}$}‚ \leavevmode\ledsidenote{\textenglish{453/s}}स‚न्तान\textbf{भाविन्यो भ‚व‚न्ति} । येन ता आत्म‚नि न दृष्टा इत्य‚न्य‚त्रापि प्र‚तिक्षिप्येर‚न् ।
	{\color{gray}{\rmlatinfont\textsuperscript{§~\theparCount}}}
	\pend% ending standard par
      ‚{\tiny $_{lb}$}‚

	  
	  \pstart \leavevmode% starting standard par
	इदानीं स्मृत्यादीनां क्व‚चिदेव पुरुषे स‚म्भ‚व‚माह । \textbf{त‚त्साध‚न}मित्यादि । तेषां‚{\tiny $_{lb}$}‚ स्मृत्यादीनां य‚थोक्तानां य‚त्साध‚न‚मुत्प‚त्तिकार‚ण‚न्त‚स्य \textbf{संप्र‚दाय उप‚देश}स्त‚स्य भेदो‚{\tiny $_{lb}$}‚ विशेषः क्व‚चिदेवाग‚मे स‚म्भ‚वो न स‚र्व‚त्र‚{\tiny $_{५}$}‚ । त‚द्व‚द् \textbf{गुणान्त‚र‚साध‚नान्य‚पि स्युः} ।‚{\tiny $_{lb}$}‚ क्व‚चिदेव पुरुषे भ‚वेयुः । सिद्धिः साध‚नं । स्मृत्यादिकार‚णानां कार्य‚भूतानि यानि‚{\tiny $_{lb}$}‚ गुणान्त‚राणि स्मृत्यादिरूपाणि । तेषां साध‚नानि निष्प‚त्त‚योपि क्व‚चिदेव पुरुषे‚{\tiny $_{lb}$}‚ स्युः स्मृत्यादिकार‚णानुष्ठानात् । य‚दि स्मृत्यादि साध‚नं स्यात् किन्न दृश्य‚त इति‚{\tiny $_{lb}$}‚ चेदाह । \textbf{नापी}त्यादि । \textbf{स‚न्न‚पि} विद्य‚मानोपि स‚न्तानान्त‚र‚स्थो म‚नोगुणो द्र‚{\tiny $_{६}$}‚ष्टुं‚{\tiny $_{lb}$}‚ पुरुष‚मात्रेण \textbf{न श‚क्यः । अत एव} कार‚णा\textbf{द‚दृष्ट}स्य स‚न्तानान्त‚र‚स्य म‚नोगुण‚स्यान‚{\tiny $_{lb}$}‚\textbf{प‚ह्न‚वो}प्र‚तिक्षेपः । \textbf{नापि पुरुषेषु} म‚नो\textbf{गुण‚स्योत्पित्सोरुत्प‚त्तिमिच्छोः प्र‚तिरोद्धा} बाध‚{\tiny $_{lb}$}‚कोस्ति येन त‚द‚न्य‚गुणातिशायी म‚नोगुणः \textbf{क‚स्य‚चिद‚पि} नास्तीति स्यात् । पुरुष‚त्वा‚{\tiny $_{lb}$}‚दिक एव ध‚र्मो बाध‚क इति चेत् [।] न । किं कार‚णं । त‚स्यान्य‚स‚न्तान‚भाविनो‚{\tiny $_{lb}$}‚ बाध्य‚स्य‚{\tiny $_{७}$}‚ गुण‚स्य पुरुष‚मात्रेणा\textbf{दृष्टेः} । पुरुष‚त्वादिना ध‚र्मेण \textbf{बाध्य‚बाध‚क‚भावा}- \leavevmode\ledsidenote{\textenglish{162a/PSVTa}}‚{\tiny $_{lb}$}‚ सिद्धेः । अबाध‚काच्चाप्र‚तिक्षेपः ।
	{\color{gray}{\rmlatinfont\textsuperscript{§~\theparCount}}}
	\pend% ending standard par
      ‚{\tiny $_{lb}$}‚

	  
	  \pstart \leavevmode% starting standard par
	\textbf{एतेना}न्त‚रोक्तेन \textbf{स‚र्व}स्यार्थ‚स्य य‚ज्ज्ञान‚न्त‚स्य \textbf{प्र‚तिषेधः} । आदिश‚ब्दाद् वीत‚{\tiny $_{lb}$}‚रागादि\textbf{प्र‚तिषेधाद‚यो निर्व‚र्ण्णितोत्त‚राः} ।
	{\color{gray}{\rmlatinfont\textsuperscript{§~\theparCount}}}
	\pend% ending standard par
      ‚{\tiny $_{lb}$}‚

	  
	  \pstart \leavevmode% starting standard par
	य‚था न व‚क्तृत्वादिलिंगेन स‚र्व‚ज्ञ‚त्वादीनां प्र‚तिक्षेप इति । \textbf{त‚त्रा}पि बीत‚रागं‚{\tiny $_{lb}$}‚त्वादिप्र‚तिक्षेप \textbf{एवंभूतः} पुरुषो वीत‚राग‚त्वा‚{\tiny $_{१}$}‚दिगुण‚युक्तो नेति न्यायो युक्तः ।‚{\tiny $_{lb}$}‚ किंभूतः । यादृशोय‚म‚स‚म्भ‚व‚न्त‚त्साध‚न‚संप्र‚दायः । \textbf{अस‚म्भ‚व‚न्त‚त्साध‚न‚संप्र‚दायो} वीत‚{\tiny $_{lb}$}‚राग‚त्वादिसाध‚न‚संप्र‚दायो य‚स्येति विग्र‚हः । वीत‚राग‚त्वादिसाध‚नेनोपायेन विक‚ल‚स्स‚{\tiny $_{lb}$}‚ वीत‚रागादि\textbf{र्न भ‚व‚त्येवं न्याय इति} याव‚त् । \textbf{न दृष्ट‚ज्ञाप‚को}त‚त्स्व‚भाव \textbf{इत्य‚पि} । अदृष्टं‚{\tiny $_{lb}$}‚ ज्ञाप‚कं वीत‚राग‚त्वादि लिङ्गं य‚स्य । स ज्ञाप‚को द‚र्श‚न‚मा‚{\tiny $_{२}$}‚त्रेणात‚त्स्व‚भावो वीत‚रा‚{\tiny $_{lb}$}‚ग‚त्वादिगुण‚वियुक्त‚स्व‚भावो भ‚व‚तीत्य‚पि न युक्त‚म्व‚क्तुं । न हि ज्ञाप‚कानुप‚ल‚म्भ‚मा‚{\tiny $_{lb}$}‚त्रेण ज्ञाप्य‚स्याभावो न्याय्यः । किं कार‚णं । \textbf{स‚ताम‚पि} केषांचिद‚र्थानां लिंग‚भूत‚स्य‚{\tiny $_{lb}$}‚ \textbf{कार्य‚स्यानार‚म्भ‚स‚म्भ‚वात्} । आर‚ब्ध‚न्नाम तैर‚तीन्द्रियैः कार्य‚न्त‚थापि \textbf{स्व‚भाव‚विप्र‚{\tiny $_{lb}$}‚‚{\tiny $_{lb}$}‚ \leavevmode\ledsidenote{\textenglish{454/s}}क‚र्षेणा}मीषामिदं कार्य‚मिति \textbf{द्र‚ष्टुम‚श‚क्य‚त्वाच्च} । त‚स्मान्म‚न्त्र‚क्रियासाध‚न‚वैक‚ल्यं‚{\tiny $_{३}$}‚‚{\tiny $_{lb}$}‚ य‚थैक‚स्य त‚था स‚र्व‚स्येत्येत‚द‚श‚क्य‚निश्च‚य‚मिति स्थितं ।
	{\color{gray}{\rmlatinfont\textsuperscript{§~\theparCount}}}
	\pend% ending standard par
      ‚{\tiny $_{lb}$}‚

	  
	  \pstart \leavevmode% starting standard par
	य‚त एव‚न्त\textbf{स्माद‚ध्य‚य‚न‚म}ध्य‚य‚नान्त‚र‚व‚द् अ\textbf{ध्य‚य‚नान्त‚र‚पूर्व‚क‚मि}ति साध्ये \textbf{अध्य‚{\tiny $_{lb}$}‚य‚ना}दिति लिङ्गं \textbf{व्य‚भिचारि । भा र ता द्य‚ध्य‚य‚ने} पौरुषेयाध्य‚य‚न‚त्व‚स्य \textbf{भावात्} ।
	{\color{gray}{\rmlatinfont\textsuperscript{§~\theparCount}}}
	\pend% ending standard par
      ‚{\tiny $_{lb}$}‚

	  
	  \pstart \leavevmode% starting standard par
	\textbf{वेदेन विशेष‚णाद‚दोषः} । अध्य‚य‚न‚मात्र‚स्य हि व्य‚भिचारो न वेदेन विशिष्ट‚{\tiny $_{lb}$}‚स्याध्य‚य‚न‚न‚स्येत्य‚भिप्रायः । \textbf{कः पुन‚रि}त्यादि सिद्धान्त‚वादी । \textbf{को‚{\tiny $_{४}$}‚तिश‚यो वेदा‚{\tiny $_{lb}$}‚ध्य‚य‚न‚स्य ये}न त‚द्वेदा\textbf{ध्य‚य‚न‚म‚न्य‚थे}ति स्व‚यं कृत्वाध्येतुं \textbf{न श‚क्य‚ते} । नैव क‚श्चि‚{\tiny $_{lb}$}‚द‚तिश‚यः [।] त‚तो वेदाध्य‚य‚नं च स्यान्न चाध्य‚य‚न‚पूर्व‚क‚मिति विरोधाभावात् स‚{\tiny $_{lb}$}‚ एव व्य‚भिचारः । य‚स्मा\textbf{न्न हि विशेष‚णं} वेद‚त्व\textbf{म‚विरुद्धं विप‚क्षेणा}न‚ध्य‚य‚नान्त‚र‚पूर्व‚{\tiny $_{lb}$}‚क‚त्वेन \textbf{स‚ह । अस्माद्} विप‚क्षा\textbf{द्धेतुन्निव‚र्त्त‚य‚ति} । किं कार‚णं । \textbf{अविरुद्ध‚यो}र्वेद‚त्वा‚{\tiny $_{lb}$}‚न‚ध्य‚{\tiny $_{५}$}‚य‚नान्त‚र‚पूर्व‚क‚त्व‚यो\textbf{रेक‚त्र} वेद‚वाक्ये \textbf{स‚म्भ‚वात्} । को ह्य‚त्र विरोधो य‚द् वेदा‚{\tiny $_{lb}$}‚ध्य‚य‚नं च स्यान्न चाध्य‚य‚नान्त‚र‚पूर्व‚क‚मिति । \textbf{इदानीन्त‚नानां} पुरुषाणाम\textbf{न‚ध्य‚य‚{\tiny $_{lb}$}‚नात्} । अध्य‚य‚नान्त‚र‚पूर्व‚क‚त्वेनैवाध्य‚य‚नात् । \textbf{उक्तोत्त‚र‚मेत‚त्} । भा र ता ध्य‚य‚नेपि‚{\tiny $_{lb}$}‚ प्र‚स‚ङ्गात् । त‚द‚पि हीदानीन्त‚नाः प‚रोप‚देशेनैवाधीय‚त इति । त‚स्याप्याद्याभिम‚त‚{\tiny $_{lb}$}‚म‚ध्य‚य‚न‚म‚ध्य‚य‚{\tiny $_{६}$}‚नान्त‚र‚पूर्व‚क‚त्वेन वेद‚व‚द‚पौरुषेयं स्यात् । वेदाध्य‚य‚न‚पूर्व‚क‚मेव वेदा‚{\tiny $_{lb}$}‚ध्य‚य‚नं क‚र्त्तुर\textbf{द‚र्श‚नादिति चेत् । इद‚म‚पि प्राक् प्र‚त्यूढं} प्र‚तिक्षिप्तं । दृश्य‚न्ते हि‚{\tiny $_{lb}$}‚ विच्छिन्न‚क्रियासंप्र‚दायाः । \textbf{कृत‚काश्चे}त्यादिना । \textbf{नाप्य‚द‚र्श‚न‚मात्र‚म‚भावं ग‚म‚य‚तीति}‚{\tiny $_{lb}$}‚ कृत्वा \textbf{व्य‚भिचार एव} वेदाध्य‚य‚न‚त्वादित्य‚स्य हेतोः । \textbf{त‚स्मात्} वेद‚त्वं \textbf{विशेष‚ण}‚{\tiny $_{lb}$}‚\leavevmode\ledsidenote{\textenglish{162b/PSVTa}} म‚ध्य‚य‚न‚स्य हेतोर\textbf{तिश‚य‚{\tiny $_{७}$}‚भाग् न भ‚व‚ति} विशेषाधाय‚क‚न्न भ‚व‚ति । विप‚क्ष‚विरो‚{\tiny $_{lb}$}‚धाभावेन विप‚क्षाद‚व्याव‚र्त्त‚नात् । उपात्त‚म‚पि विशेष‚ण\textbf{म‚नुपात्त‚स‚मं} ।
	{\color{gray}{\rmlatinfont\textsuperscript{§~\theparCount}}}
	\pend% ending standard par
      ‚{\tiny $_{lb}$}‚

	  
	  \pstart \leavevmode% starting standard par
	किञ्च । \textbf{य‚त्किञ्चिद वेदाध्य‚य‚नं स‚र्व‚न्त‚द‚ध्य‚य‚नान्त‚र‚प‚र्व‚क‚मिति} वेदेन‚{\tiny $_{lb}$}‚ ‚{\tiny $_{lb}$}‚ \leavevmode\ledsidenote{\textenglish{455/s}}विशेषितेपि हेतौ \textbf{व्याप्तिर्न सिध्य‚ति} । विप‚र्य‚ये बाध‚क‚प्र‚माणाभावेन स‚र्व‚स्य वेदा‚{\tiny $_{lb}$}‚ध्य‚य‚न‚स्य त‚थाभाव‚सिद्धेर‚ध्य‚य‚नान्त‚र‚पूर्व‚क‚त्वासिद्धेः । \textbf{यादृशं} त्व‚ध्य‚य‚नं स्व‚यं‚{\tiny $_{१}$}‚‚{\tiny $_{lb}$}‚ क‚र्त्तुम‚श‚क्त‚स्य त‚न्निमित्त‚म‚ध्य‚य‚नान्त‚र‚निमित्तं \textbf{दृष्टं त‚त्त‚थेत्य}ध्य‚य‚नान्त‚र‚पूर्व‚क‚मे‚{\tiny $_{lb}$}‚\textbf{वेति स्यात्} । स्व‚यं कृत्वाध्येतुम‚श‚क्त‚स्य य‚द‚ध्य‚य‚न‚न्त‚स्य दृष्टे विशेष‚जाड्यादि‚{\tiny $_{lb}$}‚ल‚क्ष‚णं त‚न्निमित्त‚त‚या प‚र‚पूर्वाध्य‚य‚न‚निमित्त‚त‚या । त‚त्त्यागेन त‚स्य जाड्यादिनि‚{\tiny $_{lb}$}‚मित्त‚स्याध्य‚य‚न‚स्य य‚था प‚रिदृष्टे विशेष‚स्य त्यागेन । य‚द्वा \textbf{त‚न्निमित्त‚त‚या} श‚क्ति‚{\tiny $_{lb}$}‚निमित‚त‚या । दृष्टेऽव‚ग‚ते विशेषेस्व‚यं कृत्वाध्य‚य‚न‚ल‚क्ष‚णे‚{\tiny $_{२}$}‚ \textbf{त‚त्त्यागेन} त‚स्य विशे‚{\tiny $_{lb}$}‚ष‚स्य त्यागेन वेदाध्य‚य‚न‚त्व‚सामा\textbf{न्य‚स्य ग्र‚ह‚णं} श‚क्त‚स्याश‚क्त‚स्य वा स‚र्वं वेदाध्य‚य‚न‚{\tiny $_{lb}$}‚म‚ध्य‚य‚नान्त‚र‚पूर्व‚क‚म्वेदाध्य‚य‚न‚त्व‚सामान्यादिति क्रिय‚माणं \textbf{व्य‚भिचार्येव} । किमिव‚{\tiny $_{lb}$}‚ [।] हुताश‚न‚सिद्धौ । अग्निसिद्धौ \textbf{पाण्डुद्र‚व्य‚त्व‚व‚त्} । अग्निसाध्ये धूमे यः पाण्डु‚{\tiny $_{lb}$}‚विशेषो दृष्ट‚स्त‚त्त्यागेन पाण्डुद्र‚व्य‚सामान्य‚मुपादीय‚मान‚म‚ग्निसिद्धौ य‚था व्य‚भिचा‚{\tiny $_{३}$}‚रि‚{\tiny $_{lb}$}‚ त‚द्व‚दित्य‚र्थः ।
	{\color{gray}{\rmlatinfont\textsuperscript{§~\theparCount}}}
	\pend% ending standard par
      ‚{\tiny $_{lb}$}‚

	  
	  \pstart \leavevmode% starting standard par
	एतेनान‚न्त‚रोक्तेन व्य‚भिचारित्व‚प्र‚तिपाद‚नेन व‚च‚नाद‚यः । आदिश‚ब्दात् पुरुष‚{\tiny $_{lb}$}‚त्वाद‚यः \textbf{प्र‚त्युक्ताः} । य‚था तेपि व्य‚भिचारिण इति ।
	{\color{gray}{\rmlatinfont\textsuperscript{§~\theparCount}}}
	\pend% ending standard par
      ‚{\tiny $_{lb}$}‚

	  
	  \pstart \leavevmode% starting standard par
	क‚स्मिन् साध्ये [।] रागादिसाध‚ने । रागादिसिद्धौ । यादृशो रागादिप्र‚भ‚वो‚{\tiny $_{lb}$}‚ व‚च‚न‚विशेषो दृष्ट‚स्त‚त्त्यागेन व‚क्तृत्व‚सामान्य‚स्य व्य‚भिचारात् ।
	{\color{gray}{\rmlatinfont\textsuperscript{§~\theparCount}}}
	\pend% ending standard par
      ‚{\tiny $_{lb}$}‚

	  
	  \pstart \leavevmode% starting standard par
	अस्तु \textbf{वे}त्य‚भ्युप‚ग‚म्याप्याह । स‚र्व‚थाप्येवंकृत्वा वेद‚स्या\textbf{नादिता सिध्ये}‚{\tiny $_{४}$}‚दादिर‚हि‚{\tiny $_{lb}$}‚त‚त्व‚मात्रं सिध्येत् । \textbf{नापुरुषाश्र‚यः} । अपौरुषेय‚त्व‚न्तु न सिध्येत् ।
	{\color{gray}{\rmlatinfont\textsuperscript{§~\theparCount}}}
	\pend% ending standard par
      ‚{\tiny $_{lb}$}‚

	  
	  \pstart \leavevmode% starting standard par
	अथ त‚स्माद‚पौरुषेय‚मात्रादेवापौरुषेय‚त्व‚मिष्य‚ते । त‚दा \textbf{स्याद‚न्योपि} लोक‚व्य‚व‚{\tiny $_{lb}$}‚हारोनादिप्र‚वृत्त‚त्वाद‚न‚राश्र‚योपौरुषेयः । \textbf{पुरुष एव हि स्व‚य‚म‚भ्युह्यो}प‚क‚ल्प्याधीय‚ते ।‚{\tiny $_{lb}$}‚ प‚र‚तो वा श्रुत्वाधीय‚ते तेषां पुंसा\textbf{म‚व्याप‚त‚क‚र‚णाना}म‚व्यापृत‚ताल्वादीनां \textbf{स्व‚यं‚{\tiny $_{lb}$}‚ ‚{\tiny $_{lb}$}‚ \leavevmode\ledsidenote{\textenglish{456/s}}श‚ब्दा ध्व‚{\tiny $_{५}$}‚न‚य‚न्ति} । स्व‚रूपं प्र‚काश‚य‚न्ति । \textbf{येन} स्व‚यं ध्व‚न‚नेना\textbf{पौरुषेयाः स्युः} । किन्तु‚{\tiny $_{lb}$}‚ पुरुष‚व्यापारेणैषां वैदिकानां श‚ब्दानां ध्व‚न‚नाल्लौकिक‚वाक्य‚व‚त् पौरुषेय‚त्व‚मेव ।
	{\color{gray}{\rmlatinfont\textsuperscript{§~\theparCount}}}
	\pend% ending standard par
      ‚{\tiny $_{lb}$}‚

	  
	  \pstart \leavevmode% starting standard par
	\textbf{अपि स्युर‚पौरुषेया}स्स‚म्भाव्य‚त एषाम‚पौरुषेय‚त्वं \textbf{य‚दि पुरुषाणामादिः स्यात्} ।‚{\tiny $_{lb}$}‚ अध्य‚य‚नं चानादि\textbf{स्त‚दाप्या}द्य‚स्य पुरुष‚स्याध्य‚य‚न\textbf{म‚न्य‚पूर्व‚क}म‚ध्य‚य‚नान्त‚र‚पूर्व‚कं \textbf{न‚{\tiny $_{lb}$}‚ सिध्य‚ति} । किङ्कार‚{\tiny $_{६}$}‚ण‚म् [।] \textbf{अध्याप‚यितुर}न्य‚स्य पुरुष‚स्या\textbf{भावात् । त‚त्‚{\tiny $_{lb}$}‚प्र‚थ‚मोध्येता} त‚स्य वेद‚स्य प्र‚थ‚मोध्येता स्व‚य‚म‚भ्युह्य वेद‚म‚धीत इति \textbf{क‚र्तैव स्याद्}‚{\tiny $_{lb}$}‚ वेद‚स्य । \textbf{त‚दि}ति त‚स्मा\textbf{द‚यं} वेदाध्य‚य‚न‚ल‚क्ष‚णो व्य‚व‚हार एक‚स्माद‚धीत्याप‚र‚म‚ध्या‚{\tiny $_{lb}$}‚प‚य‚ति । सोप्य‚न्य‚मिति पूर्व‚पूर्व‚द‚र्श‚न‚प्र‚वृत्तो\textbf{नादिः} पुरुष‚व्य‚व‚हार इति पुरुषैरेवायं‚{\tiny $_{lb}$}‚ \leavevmode\ledsidenote{\textenglish{163a/PSVTa}} र‚चितो व्य‚व‚हार इति स्यान्नापौरु‚{\tiny $_{७}$}‚षेय एव । किमिव [।] \textbf{डिम्भ‚क‚पांसुक्रीडा‚{\tiny $_{lb}$}‚व‚त्} । डिम्भ‚का बालास्तेषां पांसुक्रीडा य‚था \textbf{पूर्व‚व‚त्द‚र्श‚न‚प्र‚वृ}त्त‚त्वाद‚नादिः \textbf{पुरुष‚{\tiny $_{lb}$}‚व्य‚व‚हार}स्त‚द्व‚त् । आदिश‚ब्दाद् भोज‚नादिव्य‚व‚हारः ।
	{\color{gray}{\rmlatinfont\textsuperscript{§~\theparCount}}}
	\pend% ending standard par
      ‚{\tiny $_{lb}$}‚

	  
	  \pstart \leavevmode% starting standard par
	\textbf{अनादित्वादि}त्यादि । \textbf{अनादित्वाद्} वेद‚स्या\textbf{पौरुषेय‚त्वे}भ्युप‚ग‚म्य‚माने \textbf{म्ले च्छा‚{\tiny $_{lb}$}‚ दिव्य‚व‚हाराणा}मिति स्व‚कुल‚क्र‚माग‚तानां मातृविवाहादिल‚क्ष‚णानाम‚नादित्वात्‚{\tiny $_{lb}$}‚ त‚थाभावो वेद‚{\tiny $_{१}$}‚व‚द‚पौरुषेय‚त्वं स्यात् । आदिश‚ब्दादार्य‚व्य‚व‚हार‚स्यानादेः प‚रिग्र‚हः ।‚{\tiny $_{lb}$}‚ त‚था \textbf{नास्तिक्य‚व‚च‚साम‚पि} ध‚र्माध‚र्म‚प‚र‚लोकाप‚वाद‚प्र‚वृत्ताना\textbf{म‚नादित्वात्त‚थाभावः} ।‚{\tiny $_{lb}$}‚ अपौरुषेय‚त्वं स्यात् । अनादित्व‚मेव तेषां क‚थ‚मिति चेदाह । \textbf{पूर्व‚संस्कार‚स‚न्त‚तेः}‚{\tiny $_{lb}$}‚ पूर्व‚संस्कार‚व‚शात् स‚न्तानेन प्र‚वृत्तेरित्य‚र्थः । \textbf{म्लेच्छ‚व्य‚व‚हा}रा अनाद‚यः । \textbf{के}‚{\tiny $_{lb}$}‚ पुन‚स्त इत्याह । मृते‚{\tiny $_{२}$}‚ पित‚रि पुत्रेण \textbf{मातृविवाहः} कार्य इति । म्लेच्छानां केषां‚{\tiny $_{lb}$}‚चिद् व्य‚व‚हारः । आदिश‚ब्दाद् वृद्धानाम्मार‚णं संसार‚मोच‚नार्थ‚मित्यादिव्य‚व‚हार‚{\tiny $_{lb}$}‚प‚रिग्र‚हः । आदिश‚ब्दोपात्त‚माह । \textbf{म‚द‚ने}त्यादि । म द न त्र यो द‚श्याम्प‚र्व‚णि \textbf{म‚द‚नो‚{\tiny $_{lb}$}‚त्स‚वः} । अत्राप्यादिश‚ब्दात् पुत्र‚ज‚न्मोत्स‚वाद‚योप्य‚नाद‚यः । \textbf{नास्ति}कानां लौ का य‚{\tiny $_{lb}$}‚  ‚{\tiny $_{lb}$}‚ ‚{\tiny $_{lb}$}‚ \leavevmode\ledsidenote{\textenglish{457/s}}ति का ना\textbf{म्व‚चांसि} च । किंभूतान्य\textbf{पूर्व‚प‚र‚लोका‚{\tiny $_{३}$}‚द्य‚प‚वादीनि} । अपूर्व‚स्य ध‚र्माध‚{\tiny $_{lb}$}‚र्म‚स्य प‚र‚लोक‚स्य चाप‚वादीनि प्र‚तिक्षेप‚काणि । तान्य‚प्य‚नादीनीति । लिंग‚विप‚रि‚{\tiny $_{lb}$}‚णामेन स‚म्ब‚न्धः । \href{http://sarit.indology.info/?cref=pv.3.245}{२४८}
	{\color{gray}{\rmlatinfont\textsuperscript{§~\theparCount}}}
	\pend% ending standard par
      ‚{\tiny $_{lb}$}‚

	  
	  \pstart \leavevmode% starting standard par
	क‚थं पुन‚र्म्लेच्छादिव्य‚व‚हारादीनाम‚नादित्व‚मित्याह । \textbf{न ही}त्यादि । ते च‚{\tiny $_{lb}$}‚ व्य‚व‚हारा\textbf{स्तानि} च नास्तिक्य‚व‚चांसीति न‚पुंस‚क‚म‚न‚पुंस‚केनैक‚व‚च्चान्य‚त‚र‚स्यामि\edtext{}{\edlabel{pvsvt_457-1}\label{pvsvt_457-1}\lemma{चांसीति}\Bfootnote{\href{http://sarit.indology.info/?cref=P\%C4\%81.1.2.69}{ Pāṇini 1. 2. 69 }}}‚{\tiny $_{lb}$}‚ति न‚पुंस‚क‚स्यैक‚शेषः । तेनाय‚म‚र्थः [।] तान् व्य‚व‚हारां‚{\tiny $_{४}$}‚स्तानि च नास्तिक्य‚व‚{\tiny $_{lb}$}‚चांसि \textbf{प‚रै}र‚न्यैः पुरुषैर\textbf{नाहित‚संस्कारा} अव्युत्प‚न्न‚बुद्ध‚य इदानीन्त‚ना \textbf{न प्र‚व‚र्त्त‚य‚न्ति} ।‚{\tiny $_{lb}$}‚ किन्तु व्युत्पादित‚बुद्ध‚य एव । तेप्य‚प‚रैस्तेप्य‚प‚रैरिति सिद्ध‚म‚नादित्वं ।
	{\color{gray}{\rmlatinfont\textsuperscript{§~\theparCount}}}
	\pend% ending standard par
      ‚{\tiny $_{lb}$}‚

	  
	  \pstart \leavevmode% starting standard par
	येप्य‚पूर्वं काव्यादिकं कुर्व‚न्ति । तेषाम‚प्य‚न्य‚कृतेनैव संस्कारेण प्र‚वृत्तेस्त‚त्कृतोपि‚{\tiny $_{lb}$}‚ व्य‚व‚हारोनादिरिति क‚थ‚य‚न्नाह । \textbf{स्व‚प्र‚तिभे}त्यादि । \textbf{स्व‚प्र‚तिभ}या स्व‚बुद्ध्या‚{\tiny $_{lb}$}‚ \textbf{र‚चि}त‚{\tiny $_{५}$}‚\textbf{स्स‚म‚यः} काव्यादिल‚क्ष‚णो यैस्ते\textbf{षाम‚पि} ताव‚त् पुरुषाणां \textbf{य‚थाश्रुतः} प‚र‚स्मात्‚{\tiny $_{lb}$}‚ स‚म‚स्तो व्य‚स्तो वा \textbf{योर्थः} । त‚त्र ये \textbf{विक‚ल्पा}स्तेषां \textbf{संहार} एक‚त्रोपादान‚म्व‚र्गीक‚र‚ण‚मिति‚{\tiny $_{lb}$}‚ याव‚त् । ते\textbf{नैव} प्र‚कारेण \textbf{प्र‚वृत्ते}र्ग्र‚न्थादीनां क‚र‚णात् । स्व‚प्र‚तिभार‚चितोपि ग्र‚न्थो‚{\tiny $_{lb}$}‚ व‚स्तुतः प‚र‚पूर्व‚क एव । क‚थ‚न्त‚र्हि स्व‚कृत इत्युच्य‚ते इत्याह । \textbf{त‚त्}काव्यादिक\textbf{म‚प‚र‚{\tiny $_{lb}$}‚पूर्व‚क‚{\tiny $_{६}$}‚मित्युच्य‚त} इति स‚म्ब‚न्धः । केन‚चित् स्व‚यं कृत‚मित्युच्य‚ते । किं कार‚णं [।]‚{\tiny $_{lb}$}‚ \textbf{कुत‚श्चिदु}प‚देष्टुः किञ्चिद‚र्थ‚जात‚मा\textbf{ग‚त‚मिति} कृत्वा । \textbf{एक‚स्योप‚देष्टुः प्र‚ब‚न्धेनाभा‚{\tiny $_{lb}$}‚वात्} । त‚देवं स्व‚प्र‚तिभार‚चितोपि ताव‚द् ग्र‚न्थः प‚र‚मार्थ‚तः प‚र‚पूर्व‚क एव । \textbf{प्रागेव}‚{\tiny $_{lb}$}‚ किम्पुन\textbf{र्य‚थाद‚र्श‚न‚प्र‚वृत्त‚यः} । प‚रेभ्यो य‚थाद‚र्श‚न‚मेव प्र‚वृत्तिर्निवृत्तिर्येषां ते त‚थोक्ताः‚{\tiny $_{७}$}‚ । \leavevmode\ledsidenote{\textenglish{163b/PSVTa}}‚{\tiny $_{lb}$}‚ \textbf{स‚म्य‚ग्मिथ्याप्र‚वृत्त‚यः} । स‚म्य‚ग् मिथ्या च प्र‚वृत्तिराच‚र‚णं येषां \textbf{लोक‚व्य‚व‚हा}राणान्ते‚{\tiny $_{lb}$}‚ त‚थोक्ताः । त‚त्र स‚म्य‚क्प्र‚वृत्त‚यः पूज्य‚पूजाद‚यः । मिथ्याप्र‚वृत्त‚यः कामोप‚संहिता‚{\tiny $_{lb}$}‚द‚यः । एते च स्फुट‚मेव प‚र‚पूर्व‚काः ।
	{\color{gray}{\rmlatinfont\textsuperscript{§~\theparCount}}}
	\pend% ending standard par
      ‚{\tiny $_{lb}$}‚

	  
	  \pstart \leavevmode% starting standard par
	अत्र व्य‚भिचार‚माशंक‚ते । \textbf{न‚न्वि}त्यादि । प्र‚थ‚म‚क‚ल्पे भ‚वा \textbf{आदिक‚ल्पिकाः} ।‚{\tiny $_{lb}$}‚ ‚{\tiny $_{lb}$}‚ ‚{\tiny $_{lb}$}‚ \leavevmode\ledsidenote{\textenglish{458/s}}तेष्\textbf{व‚दृष्टा व्य‚व‚हाराः इष्य‚न्ते} । न हि तैः पूर्वेभ्यो‚{\tiny $_{१}$}‚ व्य‚व‚हारा उप‚ल‚ब्धास्तेषामेव‚{\tiny $_{lb}$}‚ प्र‚थ‚म‚त्वात् ।
	{\color{gray}{\rmlatinfont\textsuperscript{§~\theparCount}}}
	\pend% ending standard par
      ‚{\tiny $_{lb}$}‚

	  
	  \pstart \leavevmode% starting standard par
	\textbf{तेषाम‚पी}त्यादिना प‚रिह‚र‚ति । \textbf{तेषाम}प्यादिक‚ल्पिकानाम्पुंसा\textbf{म‚न्य‚संस्कारा‚{\tiny $_{lb}$}‚हितानां पू}र्व‚ज‚न्म‚प्र‚स‚रेषु पूर्व‚दृष्ट‚व्य‚व‚हारेणाहित‚संस्काराणां प‚श्चाद् \textbf{य‚थाप्र‚त्य‚य‚यं} ।‚{\tiny $_{lb}$}‚ य‚था स‚ह‚कारिस‚न्निधानं \textbf{प्र‚बोधात्} प्र‚वृत्तेः । तेपि नाप‚र‚पूर्व‚काः । \href{http://sarit.indology.info/?cref=pv.3.245}{२४८ a.b.}
	{\color{gray}{\rmlatinfont\textsuperscript{§~\theparCount}}}
	\pend% ending standard par
      ‚{\tiny $_{lb}$}‚

	  
	  \pstart \leavevmode% starting standard par
	\textbf{भ‚व‚त्व}नादित्वात् \textbf{स‚र्वेषां} म्लेच्छादिव्य‚व‚हाराणाम\textbf{पौरुषेय‚त्व‚{\tiny $_{१}$}‚मिति चेत् ।‚{\tiny $_{lb}$}‚ तादृशे}नादित्व‚मात्रेण स‚र्व‚व्य‚व‚हाराणा\textbf{म‚पौरुषेय‚त्वे सिद्धेपि को गुणो भ‚वेत्} [।]‚{\tiny $_{lb}$}‚ नैव क‚श्चित् । त‚था हि \textbf{कामं भ‚वेद‚विस‚म्वाद‚क‚मित्य‚पौरुषेय‚त्व‚मिष्टं । त‚च्चा}‚{\tiny $_{lb}$}‚पौरुषेय‚त्वं \textbf{विस‚म्वाद‚कानाम‚पि केषांचि}ल्लोक‚व्य‚व‚हार‚णा\textbf{म‚स्ती}ति नापौरुषेय‚त्व‚म‚{\tiny $_{lb}$}‚वित‚थ‚त्व‚स्य साध‚कं व्य‚भिचारादिति \textbf{कि}न्तेना\textbf{पौरुषेय‚त्वेन} क‚ल्पितेन ।
	{\color{gray}{\rmlatinfont\textsuperscript{§~\theparCount}}}
	\pend% ending standard par
      ‚{\tiny $_{lb}$}‚

	  
	  \pstart \leavevmode% starting standard par
	\textbf{अथ‚{\tiny $_{३}$}‚} वेद‚वाक्यानामेवापौरुषेय‚त्व‚मिष्य‚ते । त‚दा \textbf{वेद‚वाक्यानामेवापौरुषे}‚{\tiny $_{lb}$}‚य‚त्वे \textbf{स‚त्य}पि त‚द्वाच्येष्व‚र्थेषु \textbf{संश‚य} एव \textbf{पुन}रिति भूयः । अपौरुषेय‚त्व‚म‚पि क‚ल्प‚यित्वा‚{\tiny $_{lb}$}‚ भूयः संश‚य एव प्राप्त इत्य‚र्थः । किं कार‚ण‚म् [।] अर्थ\textbf{भेदा}नां वेदार्थ‚व्याख्यान‚विक‚{\tiny $_{lb}$}‚ल्पानामाचार्य‚भेदेन \textbf{द‚र्श‚नात्} ।
	{\color{gray}{\rmlatinfont\textsuperscript{§~\theparCount}}}
	\pend% ending standard par
      ‚{\tiny $_{lb}$}‚

	  
	  \pstart \leavevmode% starting standard par
	\textbf{य‚दी}त्यादिना व्याच‚ष्टे ।
	{\color{gray}{\rmlatinfont\textsuperscript{§~\theparCount}}}
	\pend% ending standard par
      ‚{\tiny $_{lb}$}‚

	  
	  \pstart \leavevmode% starting standard par
	\textbf{अपौरुषेय‚त्वेपि य‚दि} वेद‚वाक्यं य‚था‚{\tiny $_{४}$}‚स्वं \textbf{प्र‚तिनिय‚तामेव । त‚द‚र्थ‚प्र‚तिभां} ।‚{\tiny $_{lb}$}‚ वेद‚वाच्यार्थाल‚म्ब‚नाम्बुद्धिप्र‚वृत्तिकाम‚स्य य‚दि \textbf{ज‚न‚ये}त्त‚दा विप‚रीतार्थ‚स‚मारोपाभा‚{\tiny $_{lb}$}‚वादा\textbf{श्वास‚नं स्या}ल्ल‚ब्धाश्वासः पुरुषो भ‚वेत् । त‚त्तु नास्ति । य‚स्माद् । \textbf{य‚थे‚{\tiny $_{lb}$}‚ष्ट‚न्तु स‚मारोपाप‚वादाभ्या}म‚धिक‚श‚ब्द‚प्र‚क्षेपेण श‚ब्दान्त‚राप‚ह्न‚वेन वेत्य‚र्थः । विरु‚{\tiny $_{lb}$}‚द्ध‚शास्त्र‚व्य‚व‚हारिणो \textbf{नै रु क्ताः । आदि}श‚ब्दाद् वै या क र णा दि‚{\tiny $_{५}$}‚प‚रिग्र‚हः ।‚{\tiny $_{lb}$}‚ \textbf{वेद‚वाक्यानि विश‚स‚न्तो} नानार्थान् कुर्व‚न्तो \textbf{दृश्य‚न्ते । न च ते} प‚र‚स्प‚र‚विरोधिनो‚{\tiny $_{lb}$}‚ व्याख्याभेदोप‚नीता \textbf{अर्थास्तेषा}म्वेद‚वाक्याना\textbf{न्न संघ‚ट‚न्त} एव स‚म्भ‚व‚न्त्येवेत्य‚र्थः ।‚{\tiny $_{lb}$}‚ ‚{\tiny $_{lb}$}‚ \leavevmode\ledsidenote{\textenglish{459/s}}किं कार‚ण‚म् [।] \textbf{अर्थ‚निवेश}स्यार्थ‚वाच‚क‚त्वेन प्र‚व‚र्त्त‚न‚स्य \textbf{स‚म‚य‚प्राधान्यात्} ।‚{\tiny $_{lb}$}‚ संकेत‚प्र‚तिब‚द्ध‚त्वात् । \textbf{एक‚स्या}पि \textbf{वाक्य}स्य य‚थास‚म‚य\textbf{म‚नेकार्थ‚विक‚ल्प‚स‚म्भ‚वात्}‚{\tiny $_{lb}$}‚ संश‚य एव । प्र‚कृतिप्र‚त्य‚यानुसारेण च वेद‚वाक्यानां व्याख्यानात् । तेषां च निय‚{\tiny $_{lb}$}‚तार्थ‚त्वान्न वेद‚वाक्येष्व‚नेकार्थ‚विक‚ल्प‚स‚म्भ‚व इत्य‚पि मिथ्या । किं कार‚णं [।]‚{\tiny $_{lb}$}‚ \textbf{प्र‚कृतिप्र‚त्य‚यानाम‚नेकार्थ‚पा}ठ‚स‚म्भ‚वात् । एकापि हि प्र‚कृतिर‚नेकेष्व‚र्थेषु प‚ठ्य‚ते ।‚{\tiny $_{lb}$}‚ त‚था प्र‚त्य‚योपीति [।] त‚द‚व‚स्थ एव य‚थाभिप्राय‚म‚र्थ‚संस्कार‚भेदात् संश‚यः ।
	{\color{gray}{\rmlatinfont\textsuperscript{§~\theparCount}}}
	\pend% ending standard par
      ‚{\tiny $_{lb}$}‚

	  
	  \pstart \leavevmode% starting standard par
	स्यादेत‚{\tiny $_{७}$}‚द् [।] रुढिमाश्रित्य वेदार्थ‚व्याख्यानात् प्र‚कृतिप्र‚त्य‚यानाम‚नेकार्थ- \leavevmode\ledsidenote{\textenglish{164a/PSVTa}}‚{\tiny $_{lb}$}‚ पाठेपि न संश‚य इत्य‚प्य‚स‚त् । किं कार‚णं [।] \textbf{रुढेर‚प्येकान्तेन} स्व‚य‚मेवा\textbf{न‚नुम‚ते}‚{\tiny $_{lb}$}‚र‚न‚ङ्गीक‚र‚णात् । एत‚देव कुतः । \textbf{अरूढ‚श‚ब्द‚बाहुल्यात्} । अरूढा एव ये लोके‚{\tiny $_{lb}$}‚ श‚ब्दास्ते वेदे बाहुल्येन दृश्य‚न्ते । त‚द्य‚था ज‚र्भुराण\edtext{}{\edlabel{pvsvt_459-1}\label{pvsvt_459-1}\lemma{र्भुराण}\Bfootnote{In the Vedic mantras. }}प्र‚भृत‚यः । त‚तो न त‚त्र रूढि‚{\tiny $_{lb}$}‚श‚ब्दान्निर्ण्ण‚यः । त‚त्र \textbf{त‚द‚र्थ‚स्या}रूढ‚श‚ब्दार्थ‚{\tiny $_{१}$}‚स्य निर्ण्ण‚ये व्याख्यातृ\textbf{पुरुषोप‚देशापेक्ष‚{\tiny $_{lb}$}‚णात् । त‚दुप‚देश‚स्य} च पुरुषोप‚देश‚स्य च । \textbf{त‚दिच्छानुवृत्तेः} पुरुषेच्छान‚वृत्ते\textbf{र‚नि‚{\tiny $_{lb}$}‚र्ण‚य एव} वेद\textbf{वाक्यार्थेषु} । \href{http://sarit.indology.info/?cref=pv.3.246}{२४९}
	{\color{gray}{\rmlatinfont\textsuperscript{§~\theparCount}}}
	\pend% ending standard par
      ‚{\tiny $_{lb}$}‚

	  
	  \pstart \leavevmode% starting standard par
	\textbf{अपि चायं} वे द वा द्य\textbf{पौरुषेत्वं साध‚य‚न् व‚र्ण्णानाम्वा साध‚येद् वाक्य‚स्य वा} ।‚{\tiny $_{lb}$}‚ वाक्य‚विक‚ल्पेनैव प‚द‚स्याप्य‚भिधानं द्र‚ष्ट‚व्यं ।
	{\color{gray}{\rmlatinfont\textsuperscript{§~\theparCount}}}
	\pend% ending standard par
      ‚{\tiny $_{lb}$}‚

	  
	  \pstart \leavevmode% starting standard par
	१ व‚र्ण्ण‚विक‚ल्प‚म‚धिकृत्याह । \textbf{त‚त्रे}त्यादि । \textbf{अन्याविशे}षादिति [।] लौकि‚{\tiny $_{lb}$}‚के‚{\tiny $_{२}$}‚भ्यो व‚र्ण्णेभ्यो वैदिकानाम‚विशेषात् । \textbf{व‚र्ण्णाना}म‚पौरुषेय‚त्व\textbf{साध‚ने किम्फ‚ल‚म्भे}त्‚{\tiny $_{lb}$}‚ [।] नैव किञ्चित् ।
	{\color{gray}{\rmlatinfont\textsuperscript{§~\theparCount}}}
	\pend% ending standard par
      ‚{\tiny $_{lb}$}‚

	  
	  \pstart \leavevmode% starting standard par
	\textbf{न ही}त्यादिना व्याच‚ष्टे । य‚स्मान्न \textbf{हि लौकिक‚वैदिक‚वाक्य‚योर्नाना व‚र्ण्णाः} ।‚{\tiny $_{lb}$}‚ ‚{\tiny $_{lb}$}‚ ‚{\tiny $_{lb}$}‚ \leavevmode\ledsidenote{\textenglish{460/s}}किन्त‚र्हि [।] य‚था वैदिका अकाराद‚योऽभिन्नास्त‚था लौकिका अपि । एक‚त्वेन‚{\tiny $_{lb}$}‚ प्र‚त्य‚भिज्ञाय‚मान‚त्वात् । स‚त्य‚पि प्र‚त्य‚भिज्ञाने य‚दि लौकिकेभ्यो वैदिकानां व‚र्ण्णांनां‚{\tiny $_{lb}$}‚ भे‚{\tiny $_{३}$}‚द इष्य‚ते । त‚दा \textbf{भेदेपि । त‚तः} प्र‚त्य‚भिज्ञानाद् वैदिकानाम‚कारादीनां प्र‚त्यु‚{\tiny $_{lb}$}‚च्चार‚णं य‚दे\textbf{क‚त्व}न्त‚स्या\textbf{सिद्धिप्र‚संगा}त् । किं कार‚णं [।] \textbf{प्र‚त्य‚भिज्ञाविशेषा}त् ।‚{\tiny $_{lb}$}‚ लौकिक‚वैदिक‚व‚र्ण्ण‚भेदे दृष्ट‚स्य प्र‚त्य‚भिज्ञान‚स्य वैदिकेषु व‚र्ण्णेष्व‚विशेषात् । एक‚{\tiny $_{lb}$}‚त्व‚व्य‚भिचारिणः प्र‚त्य‚भिज्ञानात् क‚थ‚म्वैदिकानामेक‚त्वं सिध्य‚तीत्य‚र्थः । \textbf{भेदानु‚{\tiny $_{lb}$}‚प‚ल‚क्ष‚णाच्च वैदिक‚व‚र्ण्णासिद्धिः} । लौकिक‚वैदिक‚योर्भेदानुप‚ल‚क्ष‚णात् ।
	{\color{gray}{\rmlatinfont\textsuperscript{§~\theparCount}}}
	\pend% ending standard par
      ‚{\tiny $_{lb}$}‚

	  
	  \pstart \leavevmode% starting standard par
	अथ स्याद् वैदिकेषु व‚र्ण्णेष्वेक‚त्व‚निमित्त‚त्वात् प्र‚त्य‚भिज्ञानं प्र‚माण‚मेवान्य‚त्र‚{\tiny $_{lb}$}‚ तु सादृश्येन भ्रान्त‚त्वाद‚प्र‚माण‚मित्य‚त आह । \textbf{प्र‚त्य‚भिज्ञानाद‚प्र‚तीतिप्र‚स‚ङ्गादि}ति ।‚{\tiny $_{lb}$}‚ य‚द्य‚न्य‚त्राप्र‚माणं घ‚टादाव‚पि त‚र्हि प्र‚त्य‚भिज्ञानाद् क्ष‚णिक‚त्वाप्र‚तिप‚त्तिप्र‚स‚ङ्गात् ।‚{\tiny $_{lb}$}‚ अन्य‚त्त्वादेव । भ‚व‚द्भि र्मी मां स कैर्ल्लौकि‚{\tiny $_{५}$}‚क‚वैदिक‚व‚र्ण्ण‚भेदा\textbf{न‚भ्युप‚ग‚माच्च ।‚{\tiny $_{lb}$}‚ तेषाञ्च} व‚र्ण्णानाम\textbf{पौरुषेय‚त्व‚साध‚ने}भ्युप‚ग‚म्य‚माने \textbf{ते} व‚र्ण्णाः \textbf{स‚र्व}त्र लोके वेदे च‚{\tiny $_{lb}$}‚ \textbf{तुल्या इति किम‚नेन} मी मां स के नैव‚म‚पौरुषेय‚त्वं साध‚य‚ता \textbf{प‚रिशेषितं} प‚रि‚{\tiny $_{lb}$}‚त्य‚क्त‚म्व‚र्ण्ण‚जातं य‚त् पौरुषेयं स्यात् । \textbf{त‚था वै} लौकिक‚वैदिक‚व‚र्ण्णानाम‚पौरुषेय‚त्वे‚{\tiny $_{lb}$}‚ स‚ति \textbf{स‚र्वः} शाब्दो \textbf{व्य‚व‚हारो} लौकिको वै‚{\tiny $_{६}$}‚दिक‚श्\textbf{चापौरुषेयो न च स‚र्वोऽवित}थो‚{\tiny $_{lb}$}‚ न च स‚र्वः स‚त्त्यार्थः । अपौरुषेय‚त्वेपि वित‚थार्थ‚स्य स‚म्भ‚वात् । \textbf{इति} हेतो\textbf{र्व्य‚र्थः‚{\tiny $_{lb}$}‚ प‚रिश्र‚मो}ऽपौरुषेय‚त्व‚क‚ल्प‚नायाः
	{\color{gray}{\rmlatinfont\textsuperscript{§~\theparCount}}}
	\pend% ending standard par
      ‚{\tiny $_{lb}$}‚

	  
	  \pstart \leavevmode% starting standard par
	२ \textbf{अथे}त्यादिना द्वितीय‚प‚क्षोप‚न्यासः । \textbf{अथ वाक्य}म\textbf{पौरुषेय‚मिष्टं} ।‚{\tiny $_{lb}$}‚ त‚द‚स‚त् । त‚था हि \textbf{वाक्य‚न्न भिन्न‚म्व‚र्ण्णेभ्यो विद्य‚ते} । किं कार‚णं । दृश्य\textbf{स्यानु‚{\tiny $_{lb}$}‚प‚ल‚म्भ‚नात्} ।
	{\color{gray}{\rmlatinfont\textsuperscript{§~\theparCount}}}
	\pend% ending standard par
      ‚{\tiny $_{lb}$}‚

	  
	  \pstart \leavevmode% starting standard par
	\leavevmode\ledsidenote{\textenglish{164b/PSVTa}} \textbf{न ही}त्यादिना व्या‚{\tiny $_{७}$}‚च‚ष्टे । \textbf{न हि व‚य‚न्देव‚द‚त्तादिप‚द‚वाक्येषु} । देव‚द‚त्तादिप‚देषु‚{\tiny $_{lb}$}‚ वाक्येषु च \textbf{द‚कारा-दी}नां व‚र्ण्णानां यः \textbf{प्र‚तिभास‚स्तं} मुक्त्वाऽन्य‚व‚र्ण्णात्म‚कं प‚द‚{\tiny $_{lb}$}‚‚{\tiny $_{lb}$}‚ \leavevmode\ledsidenote{\textenglish{461/s}}वाक्य‚प्र‚तिभासं \textbf{बुद्धेः प‚श्यामः । द्वितीय‚व‚र्ण्ण‚प्र‚तिभास‚व‚त्} । य‚था द‚कारे प्र‚तिभास‚{\tiny $_{lb}$}‚माने त‚त्स‚मान‚काल‚मेव द्वितीयो व‚र्ण्णो न प्र‚तिभास‚ते । त‚द्व‚न्न प‚द‚वाक्यं प्र‚ति‚{\tiny $_{lb}$}‚भास‚ते । \textbf{न चाप्र‚तिभास‚मानं} ग्र‚ह‚णे बुद्धौ \textbf{ग्राह्य‚{\tiny $_{१}$}‚त‚येष्ट}मुप‚ल‚ब्धिल‚क्ष‚ण‚प्राप्तं‚{\tiny $_{lb}$}‚ स\textbf{द‚स्ती}ति श‚क्य‚म‚व‚सातुं । त‚था व‚र्ण्णेभ्यो\textbf{न्य‚द् वेति श‚क्य‚म‚व‚सातुं} । न चेति स‚म्ब‚{\tiny $_{lb}$}‚न्धः । अस्तित्वे निषिद्धेन्य‚त्त्व‚म‚पि निषिद्ध‚मेव । त‚थापि द्व‚योरुपादान‚म‚त्य‚न्त‚स‚त्त्व‚{\tiny $_{lb}$}‚प्र‚तिपाद‚नार्थं । \textbf{आकारान्त‚व‚त्} । य‚थैक‚स्मिन्नाकारे भास‚माने त‚त्राप्र‚तिभास‚मानं‚{\tiny $_{lb}$}‚ दृश्य‚माकारान्त‚र‚म‚न्य‚न्नास्ति त‚द्व‚त् ।
	{\color{gray}{\rmlatinfont\textsuperscript{§~\theparCount}}}
	\pend% ending standard par
      ‚{\tiny $_{lb}$}‚

	  
	  \pstart \leavevmode% starting standard par
	\textbf{अन्यास‚म्भ‚वीत्या}दि‚{\tiny $_{२}$}‚ [।] अन्येषु व‚र्ण्णेष्व‚स‚म्भ‚वि । अर्थ‚प्र‚त्याय‚न\textbf{कार्यं}ल‚क्ष‚णं‚{\tiny $_{lb}$}‚ \textbf{कार्य}व्य‚तिरिक्त‚स्य प‚द‚वाक्य‚स्य \textbf{ग‚म‚क‚मिति चेत्} । त‚था ह्य‚प्र‚तिप‚त्तिर्देव‚द‚त्तादिप‚द‚{\tiny $_{lb}$}‚वाक्येषु दृष्टा । न चेय‚म्व‚र्ण्णेभ्य‚स्तेषां प्र‚त्येक‚म‚न‚र्थ‚क‚त्वात् । एक‚व‚र्ण्ण‚कालेऽप‚र‚{\tiny $_{lb}$}‚व‚र्ण्णाभावेन साम‚स्त्याभावाच्चातोव‚ग‚म्य‚तेऽस्ति त‚त्प‚द‚वाक्यं य‚त इय‚म‚र्थ‚प्र‚ती‚{\tiny $_{lb}$}‚तिर्भ‚व‚तीति ।
	{\color{gray}{\rmlatinfont\textsuperscript{§~\theparCount}}}
	\pend% ending standard par
      ‚{\tiny $_{lb}$}‚

	  
	  \pstart \leavevmode% starting standard par
	\textbf{स्यादि}त्यादिना प्र‚तिविध‚त्ते । \textbf{स्याद्} व‚र्णेभ्यो‚{\tiny $_{३}$}‚र्थान्त‚रं प‚दादि । \textbf{य‚दि} तेषु‚{\tiny $_{lb}$}‚ \textbf{व‚र्ण्णेषु स‚त्स्व‚पि त‚द‚र्थ}प्र‚तीतिल‚क्ष‚णं \textbf{कार्य‚न्न स्यात्} । यावान् व‚र्ण्ण‚स‚मुदायोर्थ‚प्र‚ति‚{\tiny $_{lb}$}‚पाद‚नाय संकेतित‚स्ताव‚तो य‚द्य‚र्थ‚प्र‚तीतिर्न स्यात् स्यादेत‚त् । याव‚ता भ‚व‚त्येव ।‚{\tiny $_{lb}$}‚ त‚दुक्तं [।]
	{\color{gray}{\rmlatinfont\textsuperscript{§~\theparCount}}}
	\pend% ending standard par
      ‚{\tiny $_{lb}$}‚
	    
	    \stanza[\smallbreak]
	  नान्य‚थानुप‚प‚त्तिस्तु भ‚व‚त्य‚र्थ‚म‚तिं प्र‚ति ।&‚{\tiny $_{lb}$}‚त‚देवास्यानिमित्तं स्याज्जाय‚ते य‚द‚न‚न्त‚र‚मिति । \href{http://sarit.indology.info/?cref=\%C5\%9Bv-spho\%E1\%B9\%ADa.95}{स्फोट० ९५}\&[\smallbreak]
	  
	  
	  ‚{\tiny $_{lb}$}‚

	  
	  \pstart \leavevmode% starting standard par
	\textbf{न भ‚व‚ती}त्यादि प‚रः । \textbf{न भ‚व‚ति} व‚र्ण्णेभ्योर्थ‚प्र‚तीतिः ।‚{\tiny $_{४}$}‚ किं कार‚णं [।] \textbf{तेषा}‚{\tiny $_{lb}$}‚म्व‚र्ण्णाना\textbf{म‚विशेषेपि प‚द‚वाक्यान्त‚रे}र्थ‚प्र‚तीतेर\textbf{स‚म्भ‚वात्} । य‚दि हि व‚र्ण्णेभ्योर्थ‚प्र‚{\tiny $_{lb}$}‚तीतिः स्यात् त‚दा स‚र इत्य‚स्मिन् प‚दे यादृश्य‚र्थ‚प्र‚तीतिस्तादृश्येव र‚स इत्य‚त्रापि‚{\tiny $_{lb}$}‚ स्याद् उभ‚य‚त्र व‚र्ण्णानान्तुल्य‚त्वात् । एवं वाक्येपि स‚दृश‚व‚र्ण्णे बोद्ध‚व्यं । न च‚{\tiny $_{lb}$}‚ भ‚व‚ति । त‚स्मान्न व‚र्ण्णेभ्योर्थ‚प्र‚तीतिल‚क्ष‚णं कार्य‚मिति ।
	{\color{gray}{\rmlatinfont\textsuperscript{§~\theparCount}}}
	\pend% ending standard par
      ‚{\tiny $_{lb}$}‚

	  
	  \pstart \leavevmode% starting standard par
	\textbf{ने}त्यादिना प‚रिह‚र‚ति । \textbf{तेषा}म्व‚र्ण्णा‚{\tiny $_{५}$}‚नां वाक्यान्त‚रेष्\textbf{व‚विशेषासिद्धेः} । त‚था हि‚{\tiny $_{lb}$}‚ य एक‚त्र वाक्ये व‚र्ण्णा न त एव वाक्यान्त‚रेषु पुरुष‚प्र‚य‚त्न‚भेदेन व‚र्ण्णानां प्र‚तिवा‚{\tiny $_{lb}$}‚क्य‚म्भिन्नानामेवोत्प‚त्तेः ।
	{\color{gray}{\rmlatinfont\textsuperscript{§~\theparCount}}}
	\pend% ending standard par
      ‚{\tiny $_{lb}$}‚‚{\tiny $_{lb}$}‚\textsuperscript{\textenglish{462/s}}

	  
	  \pstart \leavevmode% starting standard par
	स एवाय‚म्व‚र्ण्ण इति \textbf{प्र‚त्य‚भिज्ञानात्} प्र‚तिवाक्यं व‚र्ण्णाना\textbf{म‚विश‚षोऽ}भेदः \textbf{सिद्ध‚{\tiny $_{lb}$}‚ इति चेत्} ।
	{\color{gray}{\rmlatinfont\textsuperscript{§~\theparCount}}}
	\pend% ending standard par
      ‚{\tiny $_{lb}$}‚

	  
	  \pstart \leavevmode% starting standard par
	\textbf{नैत}देवं । किं कार‚णं । \textbf{त‚स्य} प्र‚त्य‚भिज्ञान‚स्य \textbf{व्य‚भिचारित्वात्} । दृश्य‚ते हि‚{\tiny $_{lb}$}‚ लून‚पुन‚र्जातेषु केशेषु भिन्नेष्व‚पि‚{\tiny $_{६}$}‚ सादृश्य‚ग्र‚ह‚णाद् विप्र‚ल‚ब्ध‚स्य प्र‚त्य‚भिज्ञानं ।‚{\tiny $_{lb}$}‚ सादृश्य‚ग्र‚ह‚णं च स‚दृश‚स्य स्व‚रूप‚ग्र‚ह‚णं न त्व‚न्य‚स‚दृश इति ग्र‚ह‚णं । \textbf{अनिद‚र्श‚न‚त्वा}‚{\tiny $_{lb}$}‚च्चादृष्टान्त‚त्वाच्च प्र‚त्य‚भिज्ञान‚स्यालिंग‚स्य । न ह्येकः प्र‚त्य‚भिज्ञाय‚मानो वादि‚{\tiny $_{lb}$}‚प्र‚तिवादिसिद्धो दृष्टान्तोस्ति । नापि प्र‚तिप‚दं व‚र्ण्णैक‚त्व‚ग्राह‚कं प्र‚त्य‚क्षं प्र‚त्य‚भिज्ञानं‚{\tiny $_{lb}$}‚ \leavevmode\ledsidenote{\textenglish{165a/PSVTa}} स‚म्भ‚व‚ति । पूर्व‚काल‚स‚म्ब‚न्धित्व‚स्येदानीम‚स‚न्नि‚{\tiny $_{७}$}‚हित‚त्वेनाग्र‚ह‚णात् । ग्र‚ह‚णे वा श्रोत्र‚{\tiny $_{lb}$}‚ज्ञान‚व‚त् स्प‚ष्ट‚प्र‚तिभासः स्यात् [।] न च भ‚व‚ति [।] त‚स्मान्न पूर्व‚काल‚व‚र्ण्ण‚ग्रा‚{\tiny $_{lb}$}‚ह‚कं । दृश्य‚मान‚स्य चेदानीन्त‚न‚काल‚त्वाद् य‚श्चेदानीन्त‚न‚काल‚स‚म्ब‚न्धी स्व‚भावः‚{\tiny $_{lb}$}‚ स क‚थं पूर्व‚काल‚स‚म्ब‚न्धी । पूर्वाप‚र‚काल‚योः प‚र‚स्प‚र‚विरोधात् क‚थं प्र‚त्य‚क्षेण त‚त्त्व‚{\tiny $_{lb}$}‚ग्र‚ह‚ण उच्य‚ते । स‚न्निहित‚विष‚यं च प्र‚त्य‚क्ष‚मिष्य‚ते [।] न च व‚र्ण्ण‚स्य स‚न्निधानं‚{\tiny $_{lb}$}‚ स‚{\tiny $_{१}$}‚म्भ‚व‚ति सांश‚त्वात् । अन्त्य‚व‚र्ण्ण‚भाग‚काले च पूर्व्व‚व‚र्ण्ण‚भागानाम‚स‚त्त्वात् । तेन‚{\tiny $_{lb}$}‚ न व‚र्ण्णेषु प्र‚तिप‚द‚मेक‚त्व‚ग्राह‚कं प्र‚त्य‚क्षं प्र‚त्य‚भिज्ञानं स‚म्भ‚व‚ति ।
	{\color{gray}{\rmlatinfont\textsuperscript{§~\theparCount}}}
	\pend% ending standard par
      ‚{\tiny $_{lb}$}‚

	  
	  \pstart \leavevmode% starting standard par
	त‚स्मात् स्थित‚मेत‚त् प्र‚तिवाक्यं भिन्ना एव व‚र्ण्णास्तेषामेव भेदार्थ‚प्र‚तीते‚{\tiny $_{lb}$}‚र्भेद इति ।
	{\color{gray}{\rmlatinfont\textsuperscript{§~\theparCount}}}
	\pend% ending standard par
      ‚{\tiny $_{lb}$}‚

	  
	  \pstart \leavevmode% starting standard par
	न‚नु व‚र्ण्णा निर‚र्थ‚का इत्युक्त‚न्त‚त्क‚थ‚न्तेषामेव भेदाद‚र्थ‚प्र‚तीतेर्भेद इत्युच्य‚ते ।
	{\color{gray}{\rmlatinfont\textsuperscript{§~\theparCount}}}
	\pend% ending standard par
      ‚{\tiny $_{lb}$}‚

	  
	  \pstart \leavevmode% starting standard par
	स‚त्त्यं । स‚न्तो व‚र्ण्णा निर‚र्थ‚का वि‚{\tiny $_{२}$}‚क‚ल्प‚विष‚यास्तु सामान्य‚रूपा एव प्र‚ति‚{\tiny $_{lb}$}‚वाक्यं भिन्ना व‚र्ण्णा व‚र्ण्ण‚स्व‚ल‚क्ष‚णा भेदेनाध्य‚स्ता वाच‚का इष्य‚न्ते । तेन व‚र्ण्णाना‚{\tiny $_{lb}$}‚मेव भेदाद‚र्थ‚प्र‚तीतेर्भेद इत्युच्य‚ते । य‚दि तु व‚र्ण्ण‚भेदाद‚य‚म‚र्थ‚प्र‚तीतिभेदो नेष्य‚ते‚{\tiny $_{lb}$}‚ किन्तु \textbf{व‚र्ण्णाविशेषेपि} [।] त‚तो व्य‚तिरिक्त‚स्य \textbf{वाक्य}स्य \textbf{भेदाद}र्थ\textbf{प्र‚तिप‚त्तिभे}दः‚{\tiny $_{lb}$}‚ [।] स एव \textbf{कार्य‚भेदः स्यात्} । सा चार्थ‚प्र‚तीति\textbf{र्वाक्याद्} भ‚वेत् । \textbf{त‚च्च‚{\tiny $_{३}$}‚} वाक्य‚म‚{\tiny $_{lb}$}‚\textbf{तीन्द्रिय}म्व‚र्ण्ण‚व्य‚तिरेकेणेन्द्रिय‚बुद्धाव‚प्र‚तिभास‚नात् । \textbf{इति} एवं \textbf{कुतः स्या}त् । वाक्यात्‚{\tiny $_{lb}$}‚ स प्र‚तीतिर्न स्यात् । स‚म्ब‚न्ध‚स्यागृहीत‚त्वात् ।
	{\color{gray}{\rmlatinfont\textsuperscript{§~\theparCount}}}
	\pend% ending standard par
      ‚{\tiny $_{lb}$}‚

	  
	  \pstart \leavevmode% starting standard par
	स्यादेत‚द् [।] अदृश्य‚म‚पि त‚द्वाक्य‚मिन्द्रिय‚व‚त् स‚न्निधिमात्रेण प्र‚तीतिं ज‚न‚{\tiny $_{lb}$}‚य‚ति । प्र‚तीत्य‚न्य‚थानुप‚प‚त्या च वाक्य‚क‚ल्प‚नेत्य‚त आह ।
	{\color{gray}{\rmlatinfont\textsuperscript{§~\theparCount}}}
	\pend% ending standard par
      ‚{\tiny $_{lb}$}‚

	  
	  \pstart \leavevmode% starting standard par
	\textbf{स‚न्निधिमात्रे}ण वाक्य‚स्य प्र‚तीति\textbf{ज‚न‚ने}ऽभ्युप‚ग‚म्य‚माने । इन्द्रिया‚{\tiny $_{४}$}‚दिव\textbf{द‚व्युत्प‚{\tiny $_{lb}$}‚‚{\tiny $_{lb}$}‚ \leavevmode\ledsidenote{\textenglish{463/s}}न्न‚स्याप्य}कृत‚संकेत‚स्यापि पुंसोर्थ‚प्र‚तीतिर्वाक्यात् \textbf{स्यात्} [।] न च भ‚व‚ति । त‚स्माद्‚{\tiny $_{lb}$}‚ व‚र्ण्णेभ्यः संकेत‚ब‚लादेवार्थंप्र‚तीतेर्भावात् क‚थ‚म‚न्य‚थानुप‚प‚त्त्या वाक्य‚क‚ल्प‚ना ।‚{\tiny $_{lb}$}‚ \textbf{त‚स्मान्न वाक्य‚न्नाम किञ्चिद‚र्थान्त‚र‚म्व‚र्ण्णेभ्यो य‚स्या}न्य‚स्या\textbf{पौरुषेय‚त्वं साध्येत ।‚{\tiny $_{lb}$}‚ त‚द‚भावाद्} वाक्याभावाद् व\textbf{र्ण्णा} एव केव‚ल‚म‚व‚शिष्य‚न्ते । ते चाविशिष्टाः स‚र्व‚त्र‚{\tiny $_{lb}$}‚ तेषा\textbf{म‚पौरुषेय‚त्व}सा‚{\tiny $_{५}$}‚ध‚ने ।
	{\color{gray}{\rmlatinfont\textsuperscript{§~\theparCount}}}
	\pend% ending standard par
      ‚{\tiny $_{lb}$}‚

	  
	  \pstart \leavevmode% starting standard par
	\textbf{वेद‚नाविशिष्ट}रूपाणां लौकिकानाम‚पि व\textbf{र्ण्णा}नाम\textbf{पौरुषेय‚त्वं} साध‚यित‚व्य‚म्‚{\tiny $_{lb}$}‚ [।] अत्र च \textbf{प्र‚थ‚म‚प‚क्षे} व‚र्ण्णापौरुषेय‚त्व‚साध‚न‚प‚क्षे \textbf{प्र‚त्युक्तं} । व्य‚र्थः प‚रिश्र‚म इति ।‚{\tiny $_{lb}$}‚ \href{http://sarit.indology.info/?cref=pv.3.246-3.247}{२४९-२५०}
	{\color{gray}{\rmlatinfont\textsuperscript{§~\theparCount}}}
	\pend% ending standard par
      ‚{\tiny $_{lb}$}‚

	  
	  \pstart \leavevmode% starting standard par
	\textbf{अपि चे}त्यादि । \textbf{अनेकाव‚य‚वात्म‚त्वे} वाक्य‚स्य क‚ल्प्य‚माने \textbf{तेषा}म‚व‚य‚वानां‚{\tiny $_{lb}$}‚ \textbf{पृथ‚क्} प्र‚त्येकं \textbf{निर‚र्थ‚का} य‚दि ।
	{\color{gray}{\rmlatinfont\textsuperscript{§~\theparCount}}}
	\pend% ending standard par
      ‚{\tiny $_{lb}$}‚

	  
	  \pstart \leavevmode% starting standard par
	\textbf{तेपी}त्यादिना व्याच‚ष्टे । \textbf{त‚स्य} वाक्य‚स्य ब\textbf{ह‚वोव‚य‚वाः पृथ‚क् प्र‚{\tiny $_{६}$}‚कृत्या} स्व‚{\tiny $_{lb}$}‚भावेन य‚द्य‚न‚र्थ‚कास्त‚दा वाक्य‚म‚प्य‚नेकाव‚य‚व‚स‚मुदायात्म‚कं त‚द्व‚देवान‚र्थ‚कं । त‚त‚{\tiny $_{lb}$}‚श्चा\textbf{त‚द्रूप} इत्य‚न‚र्थ‚क‚त्वेनावाच‚क‚रूपेऽव‚य‚व‚स‚ङ्घाते \textbf{ताद्रूप्यं} वाच‚क‚वाक्य‚रूप‚म‚र्थ‚व‚त्त्व‚{\tiny $_{lb}$}‚मिति याव‚त् । \textbf{क‚ल्पितं} स‚मारोपित‚म्भ‚वेत् । \textbf{सिंह‚तादिव‚त्} । य‚था सिंहो माण‚व‚क‚{\tiny $_{lb}$}‚ इत्यादिष‚प‚चारेषु । माण‚व‚कादिष्व‚त‚द्रूपेषु सिंहादिक‚मारो‚{\tiny $_{७}$}‚पित‚न्त‚द्व‚त् ।
	{\color{gray}{\rmlatinfont\textsuperscript{§~\theparCount}}}
	\pend% ending standard par
      \textsuperscript{\textenglish{165b/PSVTa}}‚{\tiny $_{lb}$}‚

	  
	  \pstart \leavevmode% starting standard par
	\textbf{अर्थ‚वानि}त्यादिना व्याच‚ष्टे । \textbf{अर्थ‚वानेवात्मा} । वाच‚क एव स्व‚भावो \textbf{वाक्यं ।‚{\tiny $_{lb}$}‚ ते चाव‚य‚वा} वाक्य‚स्य \textbf{स्व‚य‚म‚न‚र्थ‚काः । तेषु च} स्व‚य‚म‚न‚र्थ‚केष्व‚व‚य‚वेषु सोर्थ‚वान् वा‚{\tiny $_{lb}$}‚क्या\textbf{त्मा क‚ल्प‚नास‚मारोपितः स्या}त् । \textbf{सिंह‚तादिव‚त् माण‚व‚कादिषु । इति} हेतोस्स‚{\tiny $_{lb}$}‚ वाच‚क आत्मा क‚ल्प‚नार‚चित‚त्वात् \textbf{पौरुषेय एव} । \href{http://sarit.indology.info/?cref=pv.3.248}{२५१}
	{\color{gray}{\rmlatinfont\textsuperscript{§~\theparCount}}}
	\pend% ending standard par
      ‚{\tiny $_{lb}$}‚

	  
	  \pstart \leavevmode% starting standard par
	\textbf{अथ माभूदेष दो}ष \textbf{इति प्र‚त्येकं} वाक्य‚{\tiny $_{१}$}‚स्या\textbf{व‚य‚वाः} वाक्यार्थेन \textbf{सार्थ‚का इष्य‚न्ते} ।‚{\tiny $_{lb}$}‚ ‚{\tiny $_{lb}$}‚ \leavevmode\ledsidenote{\textenglish{464/s}}त‚दा \textbf{प्र‚त्येक}म‚व‚य‚वानां \textbf{सार्थ‚क‚त्वे मिथ्यानेक‚त्व‚क‚ल्प‚ना} एक‚स्याप्य‚व‚य‚व‚स्य प‚रिस‚मा‚{\tiny $_{lb}$}‚प्तार्थ‚त्वाद‚व‚य‚वान्त‚रापेक्षा वाक्य‚स्य न युज्य‚त इत्य‚र्थः । य‚दा \textbf{चैकाव‚य‚व‚ग‚त्या च} ।‚{\tiny $_{lb}$}‚ एक‚स्यापि वाक्याव‚य‚व‚स्य ग्र‚ह‚णे \textbf{वाक्यार्थ‚प्र‚तिप}त्प्र‚तीतिर्भ‚वेत् ।
	{\color{gray}{\rmlatinfont\textsuperscript{§~\theparCount}}}
	\pend% ending standard par
      ‚{\tiny $_{lb}$}‚

	  
	  \pstart \leavevmode% starting standard par
	अथ स्याद् [।] एकाव‚य‚व‚ग‚त्यापि सामान्येन‚{\tiny $_{२}$}‚ वाक्यार्थ‚प्र‚तीतिर्भ‚व‚त्येव ।‚{\tiny $_{lb}$}‚ य‚दाह । भ र्त्तृ ह रिः । स‚र्वेषाम्पृथ‚ग‚र्थ‚व‚त्ता स‚र्वेषु प्र‚तिश‚ब्दं कृत्स्नार्थ‚प‚रिस‚माप्तेः ।‚{\tiny $_{lb}$}‚ त‚था य‚देव प्र‚थ‚मं प‚द‚मुपादीय‚ते त‚स्मिन् स‚र्व‚रूपार्थोप‚ग्राहिणि निय‚मानुवाद‚निब‚न्ध‚{\tiny $_{lb}$}‚नानि प‚दान्त‚राणि विज्ञाय‚न्त\edtext{\textsuperscript{*}}{\edlabel{pvsvt_464-1}\label{pvsvt_464-1}\lemma{*}\Bfootnote{Bhāgavṛtti [[?]]}} इति । त‚त्क‚थ‚मुच्य‚ते \textbf{वृथानेक‚त्व‚क‚ल्प}नेति ।
	{\color{gray}{\rmlatinfont\textsuperscript{§~\theparCount}}}
	\pend% ending standard par
      ‚{\tiny $_{lb}$}‚
	    
	    \stanza[\smallbreak]
	  नैष दोषो य‚स्मात् । विव‚क्षितार्थ‚विशेषापेक्ष‚यैत‚दुच्य‚ते ।&‚{\tiny $_{lb}$}‚प्र‚त्येकं सा‚{\tiny $_{३}$}‚र्थ‚क‚त्वेपि मिथ्यानेक‚त्व‚क‚ल्प‚ना ।&‚{\tiny $_{lb}$}‚एकाव‚य‚व‚ग‚त्या च वाक्यार्थ‚प्र‚तिप‚द् भ‚वेदिति ।\&[\smallbreak]
	  
	  
	  ‚{\tiny $_{lb}$}‚

	  
	  \pstart \leavevmode% starting standard par
	नापि क‚श्चिद‚व‚य‚वः कार‚क‚विशेष‚स्याभिधाय‚कोन्य‚श्च क्रियाविशेष‚स्या‚{\tiny $_{lb}$}‚भिधाय‚क इति वाक्याव‚य‚वानां प्र‚त्येकं सार्थ‚क‚त्वात् साफ‚ल्यं युक्तं । क्रियाविशेषा‚{\tiny $_{lb}$}‚न‚न्वित‚स्य कार‚क‚विशेष‚स्याभिधातुम‚श‚क्य‚त्वात् । त‚द‚न्वित‚स्य त्व‚भिधाने मिथ्या‚{\tiny $_{lb}$}‚नेक‚त्व‚क‚ल्प‚नेत्यादिदोष‚{\tiny $_{४}$}‚स्त‚द‚व‚स्थ एवेति ।
	{\color{gray}{\rmlatinfont\textsuperscript{§~\theparCount}}}
	\pend% ending standard par
      ‚{\tiny $_{lb}$}‚

	  
	  \pstart \leavevmode% starting standard par
	\textbf{प‚रिस‚माप्तार्थेत्या}दिना व्याच‚ष्टे । प\textbf{रिस‚माप्ते}र्थो य‚स्य \textbf{श‚ब्द‚रू}प‚स्य त‚त्त‚था ।‚{\tiny $_{lb}$}‚ \textbf{ते चाव‚य‚वा} वाक्य‚ग‚तास्त\textbf{थाविधा} इति प‚रिस‚माप्तार्थ‚रूपाः \textbf{पृथ‚क् प्र‚त्येकं} । इति‚{\tiny $_{lb}$}‚ हेतोः प्र‚त्येक‚न्तेऽव‚य‚वा \textbf{वाक्यं} प्र‚स‚क्ताः [।] \textbf{त‚था} च ना\textbf{नेकाव‚य‚वं वाक्यं} । अनेके‚{\tiny $_{lb}$}‚नाव‚य‚वेन युक्त‚मेक‚म्वाक्यं न स्यादित्य‚र्थः । प्र‚त्येकं चाव‚य‚वानां सा‚{\tiny $_{५}$}‚र्थ‚क‚त्वे \textbf{एकाव‚{\tiny $_{lb}$}‚य‚व‚प्र‚तिप‚त्त्या} स‚म‚स्त‚वाक्यार्थ‚प्र‚तिप‚त्त्या स‚म‚स्त\textbf{वाक्यार्थ‚प्र‚तिप‚त्तेर‚व‚य‚वात्त‚रं प्र‚ति‚{\tiny $_{lb}$}‚ अपेक्षा} श्रोतुर्न स्यात् । \textbf{काल‚क्षेप‚श्च न स्यात्} । काल‚ह‚र‚णेन वाक्यार्थ‚प्र‚तीतिर्न‚{\tiny $_{lb}$}‚ स्यादित्य‚र्थः । किङ्कार‚णं [।] \textbf{त‚स्य} वाक्यार्थ‚स्य \textbf{निष्क‚लात्म‚नो} निर्विभाग‚स्य‚{\tiny $_{lb}$}‚ \textbf{क्ष‚णेनैकेन प्र‚तिप‚त्तेः} । एत‚देव कुत [।]\textbf{एक‚ज्ञानोत्प‚त्तौ} त‚स्य वाक्यार्थ‚स्य \textbf{निः‚{\tiny $_{lb}$}‚शेषाग‚मात्} ।
	{\color{gray}{\rmlatinfont\textsuperscript{§~\theparCount}}}
	\pend% ending standard par
      ‚{\tiny $_{lb}$}‚‚{\tiny $_{lb}$}‚‚{\tiny $_{lb}$}‚\textsuperscript{\textenglish{465/s}}

	  
	  \pstart \leavevmode% starting standard par
	अन्ये त्व‚न्य‚था व्याच‚क्ष‚ते । एका‚{\tiny $_{६}$}‚व‚य‚व‚प्र‚तिप‚त्त्या च वाक्यार्थ‚प्र‚तिप‚त्तौ त‚स्या‚{\tiny $_{lb}$}‚व‚य‚व‚स्य काल‚क्षेप‚श्च न स्यात् । किङ्कार‚णं । त‚स्याव‚य‚व‚स्य निःक‚लात्म‚नः क्ष‚णे‚{\tiny $_{lb}$}‚नैकेन प्र‚तिप‚त्तेः । किं कार‚ण‚म् [।] एक‚ज्ञानोत्प‚त्तौ त‚स्य निर्भाग‚स्याव‚य‚व‚स्य निः‚{\tiny $_{lb}$}‚शेषाव‚ग‚मात् ।
	{\color{gray}{\rmlatinfont\textsuperscript{§~\theparCount}}}
	\pend% ending standard par
      ‚{\tiny $_{lb}$}‚

	  
	  \pstart \leavevmode% starting standard par
	\textbf{अन्य‚थेति} य‚द्येक‚ज्ञान‚क्ष‚णेन स‚र्व‚स्य ग्र‚ह‚णं न स्यात् त‚दा गृहीतागृहीत‚स्व‚भाव‚{\tiny $_{lb}$}‚योरे\textbf{क‚त्व‚विरोधात्} । विरुद्ध‚योरेक‚{\tiny $_{७}$}‚त्वायोगात् । \href{http://sarit.indology.info/?cref=pv.3.249}{२५२}
	{\color{gray}{\rmlatinfont\textsuperscript{§~\theparCount}}}
	\pend% ending standard par
      \textsuperscript{\textenglish{166a/PSVTa}}‚{\tiny $_{lb}$}‚

	  
	  \pstart \leavevmode% starting standard par
	अथ मा भूद‚य‚न्दोष इति \textbf{स‚कृच्छ्र‚रुति}रिष्य‚ते । त‚दा \textbf{स‚कृच्छ्रुतौ च स‚र्वेषा}म‚व‚{\tiny $_{lb}$}‚य‚वानां क‚ल्प्य‚मानायां \textbf{काल‚क्षेपो न युज्य‚ते} । \href{http://sarit.indology.info/?cref=pv.3.249}{२५२}
	{\color{gray}{\rmlatinfont\textsuperscript{§~\theparCount}}}
	\pend% ending standard par
      ‚{\tiny $_{lb}$}‚

	  
	  \pstart \leavevmode% starting standard par
	\textbf{मा भूदि}त्यादिना व्याच‚ष्टे । \textbf{अव‚य‚वान्त‚रा}णाम\textbf{प्र‚तीक्ष‚णेनैक‚स्मादेवाव‚य‚वाद्‚{\tiny $_{lb}$}‚ वाक्यार्थ‚सिद्धे}र्वाक्यार्थ‚निश्च‚यात् कार‚णाद् अ\textbf{नेकाव‚य‚व‚त्व‚हानिर्वाक्य‚स्येति} कृत्वा‚{\tiny $_{lb}$}‚ \textbf{स‚र्वे}षाम्वाक्या\textbf{व‚य‚वानां स‚कृच्छ्र‚व‚ण‚मिष्य}ते । \textbf{त‚दापि काल‚क्षेपो न युक्त एव} ।‚{\tiny $_{१}$}‚‚{\tiny $_{lb}$}‚ किं कार‚ण‚म् [।] \textbf{एकाव‚य‚व‚प्र‚तिप‚त्तिकाल एव स‚र्वेषाम}व‚य‚वानां \textbf{श्र‚व‚णात्} । क्र‚मेण‚{\tiny $_{lb}$}‚ च श्र‚व‚णं दृष्टं । \textbf{क्र‚म‚श्र‚व‚णे चा}व‚य‚वानां पृथ‚क् \textbf{पृथ‚ग‚र्थ‚व‚तां} स‚ता\textbf{मेक‚स्मादे}वाव‚य‚वा‚{\tiny $_{lb}$}‚\textbf{त्त‚द‚र्थ‚सिद्धे}र्वाक्यार्थ‚सिद्धे\textbf{र‚न्य‚स्}याव‚य‚व‚स्य \textbf{वैय‚र्थ्यात्} । एत‚च्चान‚न्त‚र‚मेवोक्तं ।
	{\color{gray}{\rmlatinfont\textsuperscript{§~\theparCount}}}
	\pend% ending standard par
      ‚{\tiny $_{lb}$}‚

	  
	  \pstart \leavevmode% starting standard par
	स‚कृत्स‚र्वाव‚य‚व‚श्र‚व‚णे प‚र‚न्दोष‚न्द‚र्श‚य‚न्नाह । \textbf{स‚कृच्छ्रुतौ} स‚र्वाव‚य‚वानां युग‚{\tiny $_{lb}$}‚प‚द्ग्र‚ह‚णेभ्युप‚{\tiny $_{२}$}‚ग‚म्य‚माने \textbf{पृथ‚क्} प्र‚त्ये\textbf{क‚म‚र्थेषु} वाच्येष्व\textbf{दृष्ट‚साम‚र्थ्या}नाम‚व‚य‚वानां‚{\tiny $_{lb}$}‚ स‚हितानाम‚प्\textbf{य‚र्थ‚व‚त्ता च न सिध्य‚ति} ।
	{\color{gray}{\rmlatinfont\textsuperscript{§~\theparCount}}}
	\pend% ending standard par
      ‚{\tiny $_{lb}$}‚

	  
	  \pstart \leavevmode% starting standard par
	स्यादेत‚त् [।] \textbf{स‚हिते}ष्व‚व‚य‚वेष्व\textbf{र्थ‚द‚र्श‚नाद}र्थ‚प्र‚तीतेः पृथ‚ग‚प्य‚व‚य‚वानाम‚र्थ‚प्र‚ती‚{\tiny $_{lb}$}‚तिज‚न‚न‚साम‚र्थ्य‚म‚स्त्य‚तोय\textbf{म‚दोष} इति ।
	{\color{gray}{\rmlatinfont\textsuperscript{§~\theparCount}}}
	\pend% ending standard par
      ‚{\tiny $_{lb}$}‚

	  
	  \pstart \leavevmode% starting standard par
	त‚न्न । किं कार‚ण‚म् [।] \textbf{पृथ‚क्} प्र‚त्येकं तेष्व‚व‚य‚वेष्\textbf{व‚स‚तो रूप‚स्या}र्थ‚प्र‚तिपाद‚न‚{\tiny $_{lb}$}‚स्व‚भाव‚स्य \textbf{संह‚तेष्व‚स‚म्भ‚वा}त् ।
	{\color{gray}{\rmlatinfont\textsuperscript{§~\theparCount}}}
	\pend% ending standard par
      ‚{\tiny $_{lb}$}‚

	  
	  \pstart \leavevmode% starting standard par
	केव‚{\tiny $_{३}$}‚लानाम‚व‚य‚वानां य‚द्रूप‚न्त‚तोन्य‚देव स‚मुदितानाम‚र्थ‚प्र‚तिपाद‚न‚स‚म‚र्थं रूप‚मु‚{\tiny $_{lb}$}‚‚{\tiny $_{lb}$}‚ \leavevmode\ledsidenote{\textenglish{466/s}}प‚प‚द्य‚त इत्य‚त आह । \textbf{अर्थान्त‚रानुत्प‚त्तेश्च} । पूर्व‚काद‚स‚म‚र्थ‚रूपाद‚र्थान्त‚र‚स्य स‚म‚र्थ‚स्य‚{\tiny $_{lb}$}‚ रूप‚स्यानुत्प‚त्तेश्च । नित्य‚त्वाद्व‚र्ण्णानामिति भावः ।
	{\color{gray}{\rmlatinfont\textsuperscript{§~\theparCount}}}
	\pend% ending standard par
      ‚{\tiny $_{lb}$}‚

	  
	  \pstart \leavevmode% starting standard par
	अनित्य‚वादिनोप्य‚य‚न्दोषः किन्नेत्याह । \textbf{श‚ब्दोत्प}त्तीत्यादि । \textbf{श‚ब्दोत्प‚त्तिवादि‚{\tiny $_{lb}$}‚न‚स्ताव‚द‚य}म‚न‚न्त‚रोक्तो \textbf{न दोष ए}व । किङ्कार‚णं‚{\tiny $_{४}$}‚ [।] त‚स्य वादिनः \textbf{पृथ‚ग‚स‚म‚र्था‚{\tiny $_{lb}$}‚नाम‚प्य}स‚म‚र्थानां पुनः पुरुष‚प्र‚य‚त्न‚कृता\textbf{दुप‚कार‚विशेषा}त् स‚हिताव‚स्थायाम‚र्थ‚प्र‚तिपाद‚{\tiny $_{lb}$}‚न‚साम‚र्थ्य‚ल‚क्ष‚णेनातिश‚येना\textbf{तिश‚य‚व}ताम‚र्थ‚प्र‚तीतिल‚क्ष‚णे \textbf{कार्य‚विशेष उप‚योगात्} ।‚{\tiny $_{lb}$}‚ नित्य‚वादिन‚स्तु \textbf{प्र‚त्येक‚म‚व‚य‚वेषु स‚म‚र्थेष्}वेक‚स्माद‚प्य‚व‚य‚वाद‚र्थ‚प्र‚तीते\textbf{व्य‚र्था स्याद‚न्य}‚{\tiny $_{lb}$}‚स्याव‚य‚व‚स्य \textbf{क‚ल्प‚ना} ।
	{\color{gray}{\rmlatinfont\textsuperscript{§~\theparCount}}}
	\pend% ending standard par
      ‚{\tiny $_{lb}$}‚

	  
	  \pstart \leavevmode% starting standard par
	एव‚न्ता‚{\tiny $_{५}$}‚व‚त्साव‚य‚व‚वाक्य‚प‚क्षे दोष उक्तः ।
	{\color{gray}{\rmlatinfont\textsuperscript{§~\theparCount}}}
	\pend% ending standard par
      ‚{\tiny $_{lb}$}‚

	  
	  \pstart \leavevmode% starting standard par
	\textbf{अथ पुन‚रेक‚मेवान‚व‚य‚व‚म्वाक्यं स्यात् । त‚त्रैक‚त्वेपि हि} वाक्य‚स्याभ्युप‚ग‚म्य‚{\tiny $_{lb}$}‚माने । त‚स्या\textbf{भिन्न‚स्य} निर्भाग‚स्य \textbf{क्र‚म‚शः} क्र‚मेण \textbf{ग‚त्य‚स‚म्भ‚वात्} । ग्र‚ह‚णास‚म्भ‚वात्‚{\tiny $_{lb}$}‚ \textbf{काल‚भेद एव न युज्य‚ते} । य‚तो \textbf{न ह्येक‚स्य क्र‚मेण प्र‚तिप‚त्तिर्युक्ता} । किं कार‚णं [।]‚{\tiny $_{lb}$}‚ \textbf{गृहीतागृहीत‚योर‚भेदात्} । न हि त‚स्य गृहीतात् स्व‚भावाद‚गृहीतोन्यः‚{\tiny $_{६}$}‚ स्व‚भावोस्ति‚{\tiny $_{lb}$}‚ य‚स्य क्र‚मेण ग्र‚ह‚णं स्यात् । भ‚व‚त्व‚क्र‚मेण वाक्य‚स्य ग्र‚ह‚ण‚मिति चेदाह । \textbf{क्र‚मेण‚{\tiny $_{lb}$}‚ चे}त्यादि । किं कार‚णं । \textbf{स‚र्व}स्य \textbf{वाक्य}स्य यो व्य‚व‚हार‚कालो व‚क्तुः श्रोतुश्च \textbf{श्र‚व‚ण}‚{\tiny $_{lb}$}‚कालः \textbf{स्म‚र‚ण‚काल}श्च । त‚स्या\textbf{नेक‚क्ष‚ण‚निमेषानुक्र‚म‚स‚माप्तेः} । अनेकः क्ष‚णो‚{\tiny $_{lb}$}‚ य‚स्मिन्न‚क्षिनिमेषे सोनेक‚क्ष‚ण‚निमेषः त‚स्यानुक्र‚मः प‚रिपाटिस्तेनानुक्र‚म‚णोत्प‚त्तेः‚{\tiny $_{lb}$}‚ \leavevmode\ledsidenote{\textenglish{166b/PSVTa}} कार‚णात्‚{\tiny $_{७}$}‚ ।
	{\color{gray}{\rmlatinfont\textsuperscript{§~\theparCount}}}
	\pend% ending standard par
      ‚{\tiny $_{lb}$}‚

	  
	  \pstart \leavevmode% starting standard par
	व‚र्ण्णानामिदं क्र‚मेण ग्र‚ह‚णं वाक्य‚स्य त्व‚क्र‚मेणैवेति चेदाह । \textbf{व‚र्ण्णे}त्यादि ।‚{\tiny $_{lb}$}‚ \textbf{व‚र्ण्ण‚रूपासंस्प‚र्शिनो} व‚र्ण्ण‚रूप‚व्य‚तिरिक्त‚स्\textbf{यैक‚बु}द्धिक्ष‚ण‚प्र\textbf{तिभासिनः श‚ब्दात्म‚नोप्र‚ति‚{\tiny $_{lb}$}‚भास‚नात्} । एत‚देव कुतः । व‚र्ण्णानुक्र‚म‚प्र‚तीतेः \textbf{व‚र्ण्णानुक्र‚मे}णैव वाक्य‚स्य \textbf{प्र‚ती‚{\tiny $_{lb}$}‚‚{\tiny $_{lb}$}‚ \leavevmode\ledsidenote{\textenglish{467/s}}तेः} । न हि क्र‚म‚प्र‚तिभासं व‚र्ण्ण‚कृतं मुक्त्वाऽप‚रो क्र‚म प्र‚तिभास‚स्स‚म्प‚द्य‚ते श्रोत्र‚{\tiny $_{lb}$}‚ज्ञाने । इत‚श्च नाक्र‚म‚स्य वाक्य‚स्य प्र‚ति‚{\tiny $_{१}$}‚भासः । य‚त\textbf{स्त‚द‚विशेषेपि} त्व‚न्म‚ते न तेषां‚{\tiny $_{lb}$}‚ व‚र्ण्णानाम‚विशेषेपि व‚र्ण्णा\textbf{नुक्र‚म‚कृत‚त्वाद् वाक्य}भेद\textbf{स्यानुक्र‚म‚व‚ती वाक्य‚प्र‚तीतिर्न}‚{\tiny $_{lb}$}‚ युग‚प‚द्भाविनी । \textbf{व‚र्ण्णानुक्र‚मोप‚कारान‚पेक्ष‚णे} । व‚र्ण्णानुक्र‚म‚कृत‚मुप‚कारं वाक्यं य‚दि‚{\tiny $_{lb}$}‚ नापेक्षेत । त‚दा \textbf{तै}र्व‚र्ण्णै\textbf{र्य‚थाक‚थंचित्} त‚त्क्र‚मैर‚न्य‚क्र‚मै\textbf{र‚पि प्र‚युक्तैर्य‚त्किञ्चिद्‚{\tiny $_{lb}$}‚ वाक्यं प्र‚तीयेत} । स‚रोस्तीति प्र‚युक्ते र‚सोस्तीति प्र‚तीयेत । व‚{\tiny $_{२}$}‚र्ण्णोप‚कारान‚पेक्ष‚{\tiny $_{lb}$}‚त्वाद् विनापि वा व‚र्ण्णैर्वाक्यं प्र‚तीयेत । न च व‚र्ण्णोप‚कारापेक्ष‚या वाक्य‚प्र‚तीतिः ।‚{\tiny $_{lb}$}‚ किं कार‚णं । तै\textbf{र्व‚र्ण्णैर‚नुक्र‚म‚व‚द्भिर‚स्य} वाक्य‚स्योप‚कारा\textbf{योगात्} । क्र‚म‚व‚द्भिः क्र‚म‚{\tiny $_{lb}$}‚वानेवोप‚कारः क‚र्त्त‚व्य‚स्त‚था चोप‚कार्य‚स्य क्र‚म‚व‚त्त्वं स्यात् [।] न चैव‚मिष्य‚ते ।
	{\color{gray}{\rmlatinfont\textsuperscript{§~\theparCount}}}
	\pend% ending standard par
      ‚{\tiny $_{lb}$}‚

	  
	  \pstart \leavevmode% starting standard par
	अक्र‚मा एव व‚र्ण्णा वाक्य‚स्योप‚कार‚का भ‚विष्य‚न्तीति चेदाह । \textbf{अक्र‚मेण‚{\tiny $_{lb}$}‚ चे}त्यादि । \textbf{अक्र‚मेण} व‚र्ण्णानां‚{\tiny $_{३}$}‚ \textbf{व्याह‚र्त्तुमु}च्चार‚यितु\textbf{म‚श‚क्य‚त्वात्} । न च क्र‚मा‚{\tiny $_{lb}$}‚क्र‚मोप‚कार‚व्य‚तिरेकेणान्यः प्र‚कारोस्तीति ग\textbf{त्य‚न्त‚राभावान्नो}प‚कार‚का व‚र्ण्णा‚{\tiny $_{lb}$}‚ वाक्य‚स्येति स्थितं । \textbf{नैव वाक्ये व‚र्ण्णाः स‚न्ति} । नैव व‚र्ण्णात्म‚कं वाक्यं । किन्त‚र्हि‚{\tiny $_{lb}$}‚ व‚र्ण्णेभ्योर्थान्त\textbf{र‚मेक}मेव \textbf{श‚ब्द‚रूपं} वाक्यं । \textbf{व्य‚ञ्ज}का ध्व‚न‚यो\textbf{नुक्र‚म}व‚न्तो विशिष्टे‚{\tiny $_{lb}$}‚नानुक्र‚मेण व्य‚ञ्ज‚य‚न्ति न व्युत्क्र‚मेण । त‚दुक्तं ।
	{\color{gray}{\rmlatinfont\textsuperscript{§~\theparCount}}}
	\pend% ending standard par
      ‚{\tiny $_{lb}$}‚
	  \bigskip
	  \begingroup
	
	    
	    \stanza[\smallbreak]
	  {\normalfontlatin\large ``\qquad}य‚थानुपूर्वीनिय‚मो विकारे‚{\tiny $_{४}$}‚ क्षीर‚बीज‚योः ।&‚{\tiny $_{lb}$}‚त‚थैव प्र‚तिप‚त्तॄणान्निय‚तो बुद्धिषु क्र‚मः [।]\edtext{}{\edlabel{pvsvt_467-1}\label{pvsvt_467-1}\lemma{मः}\Bfootnote{Kumārila. }}{\normalfontlatin\large\qquad{}"}\&[\smallbreak]
	  
	  
	  
	  \endgroup
	‚{\tiny $_{lb}$}‚
	    
	    \stanza[\smallbreak]
	  तेन य‚थाक‚थ‚ञ्चित् प्र‚युक्तैरित्यादिर‚दोष इति ।\&[\smallbreak]
	  
	  
	  ‚{\tiny $_{lb}$}‚

	  
	  \pstart \leavevmode% starting standard par
	\textbf{व्य‚ञ्ज‚कानुक्र‚म‚व‚शात्} त‚देक‚म‚पि वाक्यं व्य‚क्त्य‚नुक्र‚माद\textbf{नुक्र‚म‚व‚त्} । स्फोट‚{\tiny $_{lb}$}‚रूपाविभागेन व‚र्ण्णानां नाद‚रूपाणां ग्र‚ह‚णाद् \textbf{व‚र्ण्ण‚विभाग‚व‚च्च} पुरुष‚स्य प्र\textbf{तिभाति}‚{\tiny $_{lb}$}‚ [।] प‚र‚मार्थ‚तोनुक्र‚म‚व‚र्ण्ण‚विभागाभ्यां र‚हित‚म‚पि । त‚दुक्तं ।
	{\color{gray}{\rmlatinfont\textsuperscript{§~\theparCount}}}
	\pend% ending standard par
      ‚{\tiny $_{lb}$}‚
	  \bigskip
	  \begingroup
	
	    
	    \stanza[\smallbreak]
	  {\normalfontlatin\large ``\qquad}नाद‚स्य क्र‚म‚ज‚न्य‚त्वान्न पू‚{\tiny $_{५}$}‚र्वो नाप‚र‚श्च सः ।&‚{\tiny $_{lb}$}‚अक्र‚मः क्र‚म‚रूपेण भेद‚वानिव जाय‚ते ॥&‚{\tiny $_{lb}$}‚‚{\tiny $_{lb}$}‚‚{\tiny $_{lb}$}‚\leavevmode\ledsidenote{\textenglish{468/s}}त‚स्माद‚भिन्न‚कालेषु व‚र्ण्ण‚वाक्य‚प‚दादिषु ।&‚{\tiny $_{lb}$}‚श‚ब्द‚काल‚स्व‚भाव‚श्च नाद‚भेदाद् विभिद्य‚त इति ॥\edtext{\textsuperscript{*}}{\edlabel{pvsvt_468-1}\label{pvsvt_468-1}\lemma{*}\Bfootnote{Kumārila. }}{\normalfontlatin\large\qquad{}"}\&[\smallbreak]
	  
	  
	  
	  \endgroup
	‚{\tiny $_{lb}$}‚

	  
	  \pstart \leavevmode% starting standard par
	अत्रोत्त‚र‚माह । अनु\textbf{क्र‚म‚व‚ते}त्यादि । एव‚म्म‚न्य‚ते । अव‚धार‚ण‚रूपा वाभि‚{\tiny $_{lb}$}‚\textbf{व्य‚क्ति}र‚न‚व‚धार‚ण‚रूपा वा [।] त‚दाव‚धार‚ण‚रूपाभिव्य‚क्ति\textbf{र‚क्र‚म‚स्य} वाक्य‚स्यानु‚{\tiny $_{lb}$}‚क्र‚म‚व‚ता \textbf{व्य‚ञ्ज‚केन प्र‚त्युक्ता} प्र‚तिक्षिप्ता ।‚{\tiny $_{६}$}‚ किं कार‚णं [।] \textbf{व्य‚क्ताव्य‚क्त}रूप‚यो‚{\tiny $_{lb}$}‚र‚व‚धृतान‚व‚धृत‚रूप‚योरेक‚त्र \textbf{विरोधात्} । न ह्य‚व‚धृत‚रूपाद‚न्य‚द‚न‚व‚धृतं रूपान्त‚र‚{\tiny $_{lb}$}‚मेक‚स्यास्ति येन त‚त्प‚श्चाद् व्य‚ज्येत । तेन य‚दुच्य‚ते । प्र‚थ‚मेन व‚र्ण्णेनाभिव्य‚{\tiny $_{lb}$}‚क्त‚स्यान‚व‚धार‚णाद‚व‚धार‚णार्थ‚म‚न्येषाम्व‚र्ण्णानां व्यापार इति त‚द‚पास्तं । प्र‚थ‚{\tiny $_{lb}$}‚\leavevmode\ledsidenote{\textenglish{167a/PSVTa}} मेनैव व‚र्ण्णेनाव‚धार‚ण‚रूप‚या व्य‚क्तेर्निष्पादित‚त्वात् । अन‚व‚धार‚ण‚{\tiny $_{७}$}‚रूपायां व्य‚क्तौ‚{\tiny $_{lb}$}‚ स‚म‚स्त‚व‚र्ण्णेत्यादिनोत्त‚र‚म्व‚क्ष्य‚ति ।
	{\color{gray}{\rmlatinfont\textsuperscript{§~\theparCount}}}
	\pend% ending standard par
      ‚{\tiny $_{lb}$}‚

	  
	  \pstart \leavevmode% starting standard par
	अथ स्याद् [।] व‚र्ण्णेभ्यो भिन्न‚मेव वाक्यं प्र‚तिभास‚ते न तु ध्व‚निसंसृष्टं ।‚{\tiny $_{lb}$}‚ त‚दुक्तं [।]
	{\color{gray}{\rmlatinfont\textsuperscript{§~\theparCount}}}
	\pend% ending standard par
      ‚{\tiny $_{lb}$}‚

	  
	  \pstart \leavevmode% starting standard par
	कैश्चिद् ध्व‚निर‚स‚म्वेद्यः स्व‚त‚न्त्रोन्यैः प्र‚क‚ल्पित इति ।
	{\color{gray}{\rmlatinfont\textsuperscript{§~\theparCount}}}
	\pend% ending standard par
      ‚{\tiny $_{lb}$}‚

	  
	  \pstart \leavevmode% starting standard par
	अत्राप्याह । \textbf{अव‚र्ण्णे}त्यादि । \textbf{अ}विद्य‚माना \textbf{व‚र्ण्ण}रूपा \textbf{भागा} य‚स्मिन् \textbf{वाक्ये}‚{\tiny $_{lb}$}‚ त‚स्मिन्न‚भ्युप‚ग‚म्य‚माने पुरुष‚स्या\textbf{स‚क‚ल‚श्राविणो} स‚म‚स्त‚व‚र्ण्णानुक्र‚म‚श्राविणः ख‚ण्ड‚शः‚{\tiny $_{lb}$}‚ श्रोतुरित्य‚र्थः । क‚दाचिद‚प्य‚स‚{\tiny $_{१}$}‚क‚ल‚स्य \textbf{वाक्य}स्य \textbf{ग‚तिः} श्रु\textbf{तिर्न स्यात्} । किं कार‚णं‚{\tiny $_{lb}$}‚ [।] व‚र्ण्ण‚व्य‚तिरिक्त\textbf{स्यैक‚स्य} वाक्य‚स्य \textbf{श‚क‚लाभावाद्} भागाभावात् । भ‚व‚ति च‚{\tiny $_{lb}$}‚ लोके क‚तिप‚य‚व‚र्ण्ण‚श्र‚व‚णे पूर्व‚वाक्य‚भाग‚श्र‚व‚ण‚प्र‚तीतिः । अथ व‚र्ण्णैर्भाग‚व‚तो वाक्य‚{\tiny $_{lb}$}‚स्याभ्युप‚ग‚मात् । क‚तिप‚य‚व‚र्ण्ण‚श्र‚व‚णे पूर्व‚वाक्य‚भाग‚श्र‚व‚ण‚मिष्य‚ते ।
	{\color{gray}{\rmlatinfont\textsuperscript{§~\theparCount}}}
	\pend% ending standard par
      ‚{\tiny $_{lb}$}‚

	  
	  \pstart \leavevmode% starting standard par
	त‚द‚युक्त‚म् [।] एक‚त्वाद् वाक्य‚स्य य‚दि पूर्व‚भाग‚श्र‚व‚ण‚न्त‚दा \textbf{स‚क‚ल‚श्रुतिः}‚{\tiny $_{lb}$}‚ स‚र्वात्म‚{\tiny $_{२}$}‚ना वाक्य‚स्य श्र‚व‚णं स्यात् । पूर्व‚भागाव्य‚तिरेकात् । अथ न स‚क‚ल‚{\tiny $_{lb}$}‚श्रुतिस्त‚दा \textbf{न वा क‚स्य}चिछ्रुतिः स्यात् । पूर्व‚स्यापि भाग‚स्य श्रुतिर्न स्याद् वाक्य‚{\tiny $_{lb}$}‚व्य‚तिरिक्त‚त्वादिति ।
	{\color{gray}{\rmlatinfont\textsuperscript{§~\theparCount}}}
	\pend% ending standard par
      ‚{\tiny $_{lb}$}‚

	  
	  \pstart \leavevmode% starting standard par
	\hphantom{.}तेन य‚दुच्य‚ते म ण्ड ने न । व्य‚ञ्ज‚क‚सादृश्याच्च वाक्ये त‚दात्म‚ग्र‚ह‚णाभि‚{\tiny $_{lb}$}‚मान‚स्तेन नाश्र‚व‚णं स‚क‚ल‚श्र‚व‚णं वेति [।]
	{\color{gray}{\rmlatinfont\textsuperscript{§~\theparCount}}}
	\pend% ending standard par
      ‚{\tiny $_{lb}$}‚

	  
	  \pstart \leavevmode% starting standard par
	त‚द‚पास्तं । स‚क‚लास‚क‚ल‚व‚र्ण्ण‚भाग‚प्र‚तिप‚त्तिकाले निष्क‚ल‚स्य वाक्य‚स्याश्र‚व‚{\tiny $_{३}$}‚‚{\tiny $_{lb}$}‚‚{\tiny $_{lb}$}‚ ‚{\tiny $_{lb}$}‚ \leavevmode\ledsidenote{\textenglish{469/s}}णात् । न हि व्य‚ङ्ग्य‚व्य‚ञ्ज‚क‚योः सादृश्य‚म्व‚र्ण्णाव‚र्ण्णात्म‚क‚त्वेन विस‚दृश‚त्वात् त‚त्क‚थं‚{\tiny $_{lb}$}‚ वाक्ये व‚र्ण्णात्म‚ग्र‚ह‚णाभिमान इति य‚त्किञ्चिदेत‚त् ।
	{\color{gray}{\rmlatinfont\textsuperscript{§~\theparCount}}}
	\pend% ending standard par
      ‚{\tiny $_{lb}$}‚

	  
	  \pstart \leavevmode% starting standard par
	अन्ये त्व‚न्य‚था व्याच‚क्ष‚ते । अथोप‚कार्योप‚काराभावेनायुक्त‚म‚पि क्र‚म‚व‚द्‚{\tiny $_{lb}$}‚ व्य‚ञ्ज‚कानुविधान‚म‚क्र‚म‚स्य वाक्य‚स्याभ्युप‚ग‚म्य‚ते । त‚त‚श्चास‚क‚ल‚श्रुतिरित्य‚त आह ।‚{\tiny $_{lb}$}‚ स‚क‚लेत्यादि । ख‚ण्ड‚शः श्रोतुर‚पि \textbf{स‚क‚ल}स्य निष्क‚ल‚स्य‚{\tiny $_{४}$}‚ वाक्य‚स्य \textbf{श्रुतिः} स्यात् ।‚{\tiny $_{lb}$}‚ अथ नेष्य‚ते त‚दा \textbf{न वा क‚स्य‚चित्} पुंसः स्यात् । स‚क‚ल‚व‚र्ण्णाश्राविणोपि न वा‚{\tiny $_{lb}$}‚ निष्क‚ल‚स्य वाक्य‚स्य श्रुतिः स्यात् । अन्त्याव‚स्थायाम‚पि युग‚प‚द् व‚र्ण्णानाम‚श्र‚व‚णेन‚{\tiny $_{lb}$}‚ भाग‚स्यैव श्र‚व‚णात् ।
	{\color{gray}{\rmlatinfont\textsuperscript{§~\theparCount}}}
	\pend% ending standard par
      ‚{\tiny $_{lb}$}‚

	  
	  \pstart \leavevmode% starting standard par
	अथ स्याद् [।] य‚था श्लोक एक‚दा प्र‚काशितो नाव‚धारितोन्य‚दा प्र‚काश‚ने‚{\tiny $_{lb}$}‚ त्व‚व‚धार‚ण‚स‚हो भ‚व‚ति । पुनः पुनः प्र‚काश‚ने त्व‚व‚धार्य‚ते । त‚था वाक्यं पूर्व‚ध्व‚नि‚{\tiny $_{lb}$}‚भा‚{\tiny $_{५}$}‚वान‚भिव्य‚क्त‚म‚पि नाव‚धारितं । तेन पूर्व‚पूर्व‚वाक्याभिव्य‚क्त्याहितैस्तु संस्कारै‚{\tiny $_{lb}$}‚र्वाक्याव‚धार‚णंप्र‚ति प्र‚त्य‚य‚भूतैर‚न्त्य‚व‚र्ण्ण‚श्र‚व‚ण‚काले त‚द‚व‚धार्य‚ते । त‚स्माद् व‚र्ण्णेना‚{\tiny $_{lb}$}‚नुक्र‚म‚व‚ताऽक्र‚म‚स्य वाक्य‚स्य व्य‚क्तिर्युज्य‚त एव । त‚दुक्तं ।
	{\color{gray}{\rmlatinfont\textsuperscript{§~\theparCount}}}
	\pend% ending standard par
      ‚{\tiny $_{lb}$}‚
	  \bigskip
	  \begingroup
	
	    
	    \stanza[\smallbreak]
	  {\normalfontlatin\large ``\qquad}य‚थानुवाकः श्लोको वा सोढ‚त्व‚मुप‚ग‚च्छ‚ति ।&‚{\tiny $_{lb}$}‚आवृत्त्या न तु स ग्र‚न्थ‚प्र‚त्यावृत्तिर्निरुच्य‚ते ॥&‚{\tiny $_{lb}$}‚प्र‚त्य‚यैर‚नुपाख्येयैर्ग्र‚ह‚णानुगुणैस्त‚{\tiny $_{६}$}‚था ।&‚{\tiny $_{lb}$}‚ध्व‚निः प्र‚काशिते श‚ब्दे स्व‚रूप‚म‚व‚धार्य‚ते ॥&‚{\tiny $_{lb}$}‚नादैराहित‚बीजायाम‚न्त्येन ध्व‚निना स‚ह ।&‚{\tiny $_{lb}$}‚आवृत्त‚प‚रिपाकायाम्वुद्धौ श‚ब्दोव‚धार्य‚त इति ॥\edtext{\textsuperscript{*}}{\edlabel{pvsvt_469-1}\label{pvsvt_469-1}\lemma{*}\Bfootnote{Kumārila. }}{\normalfontlatin\large\qquad{}"}\&[\smallbreak]
	  
	  
	  
	  \endgroup
	‚{\tiny $_{lb}$}‚

	  
	  \pstart \leavevmode% starting standard par
	एत‚देवाह । \textbf{स‚म‚स्ते}त्यादि । स‚म‚स्तैर्व‚र्ण्णैः प्र‚त्येकं वाक्याभिव्य‚क्तिपूर्व‚का ये‚{\tiny $_{lb}$}‚ कृताः संस्कारा विद्य‚न्ते य‚स्या बुद्धेस्सा त‚था । त‚या \textbf{स‚म‚स्त‚व‚र्ण्ण‚संस्कार‚व‚त्यान्त्य‚या}‚{\tiny $_{lb}$}‚ ऽन्त्य‚व‚र्ण्ण‚विष‚य‚या \textbf{बुद्ध्या} निष्क‚ल‚स्य \textbf{वाक्य}स्या\textbf{व‚धार‚ण‚मि‚{\tiny $_{७}$}‚त्य‚पि} क‚ल्प‚ना \textbf{मिथ्या} । \leavevmode\ledsidenote{\textenglish{167b/PSVTa}}‚{\tiny $_{lb}$}‚ किं कार‚णं [।] \textbf{त‚स्य} वाक्य‚स्या\textbf{व‚र्ण्ण‚रूप‚संस्प‚र्शिनः} । व‚र्ण्ण‚रूप‚संस्प‚र्श‚र‚हित‚स्य श्रोत्र‚{\tiny $_{lb}$}‚ज्ञाने \textbf{क‚स्य‚चित्} पुरुष‚स्य \textbf{क‚दाचिद}प्य\textbf{प्र‚तिप‚त्तेः} प्र‚तिव‚र्ण्णोच्चार‚णं प्र‚तिभासाभाव‚{\tiny $_{lb}$}‚ इत्य‚र्थः श्लोक‚स्य तूच्चार‚णं प्र‚तिभासोस्ति ।
	{\color{gray}{\rmlatinfont\textsuperscript{§~\theparCount}}}
	\pend% ending standard par
      ‚{\tiny $_{lb}$}‚

	  
	  \pstart \leavevmode% starting standard par
	अथ स्याद् [।] व‚र्ण्णात्म‚क‚मेव वाक्य‚न्तेनेन्द्रिय‚ज्ञान‚विष‚य‚मेवेत्य‚त आह ।‚{\tiny $_{lb}$}‚ ‚{\tiny $_{lb}$}‚ ‚{\tiny $_{lb}$}‚ \leavevmode\ledsidenote{\textenglish{470/s}}\textbf{व‚र्ण्णानां चाक्र‚मेणाप्र‚तिप‚त्तेः} क्र‚मेणैव प्र‚तिप‚त्तेः कार‚णात् \textbf{कुतोक्र‚{\tiny $_{१}$}‚म‚मेक‚बुद्धिग्राह्य}‚{\tiny $_{lb}$}‚म्वाक्य\textbf{न्नाम} ।
	{\color{gray}{\rmlatinfont\textsuperscript{§~\theparCount}}}
	\pend% ending standard par
      ‚{\tiny $_{lb}$}‚

	  
	  \pstart \leavevmode% starting standard par
	अथ स्याद् [।] \textbf{अन्त्य‚व‚र्ण्ण‚प्र‚तिप‚त्तेरुर्ध्व} मान‚सेन ज्ञानेन निर‚व‚य‚व‚स्य वाक्य‚{\tiny $_{lb}$}‚स्याव‚धार‚ण‚म‚स्त्येवेति चेदाह । \textbf{न चे}त्यादि । अन्त्य‚व‚र्ण्ण‚प्र‚तिप‚त्तेरूर्ध्व\textbf{म‚न्य}म्व‚र्ण्ण‚{\tiny $_{lb}$}‚व्य‚तिरिक्त\textbf{म‚श‚क‚ल}म‚ख‚ण्डं निर्विभाग‚मित्य‚र्थः । \textbf{श‚ब्दात्मानं न चोप‚ल‚क्ष‚यामः} ।
	{\color{gray}{\rmlatinfont\textsuperscript{§~\theparCount}}}
	\pend% ending standard par
      ‚{\tiny $_{lb}$}‚

	  
	  \pstart \leavevmode% starting standard par
	\textbf{नापि स्व‚य‚म‚य‚म्व‚क्ता} य‚थोक्तं श‚ब्दात्मान\textbf{म्विभाव‚य‚ति} । त‚था हि त‚दापि‚{\tiny $_{lb}$}‚ वाक्य‚म‚व‚धार‚य‚{\tiny $_{२}$}‚न् व‚र्ण्णानुक्र‚म‚मेव बाह्य‚रूप‚त‚याव‚धार‚य‚ति [।] न तु व‚र्ण्ण‚व्य‚{\tiny $_{lb}$}‚तिरिक्त‚न्निर्विभाग‚म्वाक्य‚म‚व‚धार‚य‚ति । केव‚ल‚म‚यं व‚क्ता य‚था म‚योक्तं \textbf{स‚माप्त‚{\tiny $_{lb}$}‚क‚लः श‚ब्दोन्त्यायाम्बुद्धौ भातीत्येवं य‚दि स्यात् । साधु} मे स्यादिति या \textbf{क‚ल्याण‚{\tiny $_{lb}$}‚काम}ताभिप्रेतार्थाशंसा । \textbf{त‚या मूढ‚म‚तिः स्व‚प्नाय‚ते} । अस्व‚प‚न्न‚पि स्व‚प्ने व्य‚व‚{\tiny $_{lb}$}‚स्थित‚मिवात्मान‚माच‚र‚ति । अधिक‚र‚णाच्चे\edtext{}{\edlabel{pvsvt_470-1}\label{pvsvt_470-1}\lemma{णाच्चे}\Bfootnote{Restored. }}ति व‚क्त‚व्य‚मिति स‚{\tiny $_{३}$}‚प्त‚म्य‚न्ताद‚पि ।‚{\tiny $_{lb}$}‚ क्य‚ज् । क्य‚ज् विधानेप्येत‚द्व‚क्त‚व्यं स्म‚र्य‚त इत्येके । अन्ये त्वाहुः [।] स्व‚प्न‚वाने‚{\tiny $_{lb}$}‚वाभेदोप‚च‚रात् । अथ‚वा म‚त्व‚र्थीय‚स्यार्श आदिद‚र्श‚नेन विधानात् । स्व‚प्न‚श‚ब्दे‚{\tiny $_{lb}$}‚नोक्तः । तेन क‚र्त्तुरेवोप‚मानात् क्य‚ज् प्र‚त्य‚यः । सुप्त इवाच‚र‚ति स्व‚प्नाय‚त इति‚{\tiny $_{lb}$}‚ याव‚त् । अनेनोप‚ह‚स‚ति ।
	{\color{gray}{\rmlatinfont\textsuperscript{§~\theparCount}}}
	\pend% ending standard par
      ‚{\tiny $_{lb}$}‚

	  
	  \pstart \leavevmode% starting standard par
	स्म‚र‚ण‚ज्ञानेन त‚र्हि प‚द‚वाक्य‚म‚क्र‚मं गृह्य‚त इति चेदाह । \textbf{न ही}त्यादि । \textbf{न हि‚{\tiny $_{४}$}‚‚{\tiny $_{lb}$}‚ स्म‚र्य‚माण‚योर‚पि प‚द‚वाक्य‚योः} स‚म्ब‚न्धिनो \textbf{व‚र्ण्णाः} प‚द‚वाक्य‚योर्भेद‚व्य‚व‚स्थाप‚काः‚{\tiny $_{lb}$}‚ \textbf{क्र‚म‚विशेष}म‚न्त‚रेणाक्र‚मायामेक‚स्यां बुद्धौ न हि \textbf{विभाव्य‚न्ते} किन्त्व‚नुभ‚व‚क्र‚म‚व‚त्‚{\tiny $_{lb}$}‚ स्म‚र‚ण‚म‚पि क्र‚मेणैवेति याव‚त् । य‚दि त्व‚क्र‚मायाम‚न्त्यायां बुद्धौ प‚द‚वाक्य‚योर्व‚र्ण्णाः‚{\tiny $_{lb}$}‚ क्र‚म‚विशेष‚म‚न्त‚रेण‚विभाव्य‚न्ते । त‚दा त‚स्या\textbf{म‚क्र‚मायां बुद्धौ पौर्वाप‚र्याभावाद्} व‚र्ण्णा‚{\tiny $_{lb}$}‚ युग‚प‚देव वि‚{\tiny $_{५}$}‚भाव्य‚न्त इति कृत्वा तेषां \textbf{प‚द‚वाक्य‚भेदानां} प‚द‚भेदानां वाक्य‚भेदानां‚{\tiny $_{lb}$}‚ \textbf{च त‚त्कृतो} व‚र्ण्ण‚पौर्वाप‚र्य‚प्र‚तिभास‚कृतो \textbf{भेदो} विशेषो \textbf{न स्यात्} । व‚र्ण्णानां क्र‚म‚{\tiny $_{lb}$}‚‚{\tiny $_{lb}$}‚ ‚{\tiny $_{lb}$}‚ \leavevmode\ledsidenote{\textenglish{471/s}}विशेष‚प्र‚तिभासादेव प‚द‚वाक्यानाम्प‚र‚स्प‚र‚म्भेद‚स्त‚द‚भावे स न स्यादिति याव‚त् ।
	{\color{gray}{\rmlatinfont\textsuperscript{§~\theparCount}}}
	\pend% ending standard par
      ‚{\tiny $_{lb}$}‚

	  
	  \pstart \leavevmode% starting standard par
	\textbf{नाप्य‚क्र‚म}मित्यादि । न विद्य‚ते व‚र्ण्ण‚क्र‚मो य‚स्मिन् श‚ब्द‚रूपे त‚द‚क्र‚मं \textbf{श‚ब्द‚रूप‚{\tiny $_{lb}$}‚म्व‚र्ण्णे}भ्योन्य\textbf{न्न प‚श्याम इत्युक्तं} ।‚{\tiny $_{६}$}‚ त‚स्याव‚र्ण्ण‚रूप‚संस्प‚र्शिनः क‚स्य‚चिद‚प्य‚प्र‚ति‚{\tiny $_{lb}$}‚प‚त्तेरित्युक्त‚त्वात् ।
	{\color{gray}{\rmlatinfont\textsuperscript{§~\theparCount}}}
	\pend% ending standard par
      ‚{\tiny $_{lb}$}‚

	  
	  \pstart \leavevmode% starting standard par
	जातिस्फोट‚स्तु जात्य‚भावादेव निर‚स्तः [।] \textbf{स‚ति वा} त‚स्मिन्न‚व‚र्ण्ण‚क्र‚मे‚{\tiny $_{lb}$}‚ श‚ब्द‚रूपे । \textbf{त‚च्}छ‚ब्द‚रूप‚म\textbf{नित्य‚म्वा स्यात् नित्य‚म्वा} । व‚स्तुनो ग‚त्य‚न्त‚राभावात् ।‚{\tiny $_{lb}$}‚ य‚द्य\textbf{नित्य‚न्त‚दा} पुरुष‚प्र\textbf{य‚त्न‚स‚म्भूतं पौरुषेयं क‚थं न त}द्वाक्यं । पौरुषेय‚मेव स्यात् ।‚{\tiny $_{lb}$}‚ \textbf{अव‚श्यं ह्य‚नित्य‚मुत्प‚त्तिम}दिति \textbf{कुत‚श्चित्} स्व‚हेतो\textbf{र्भ‚व‚{\tiny $_{७}$}‚ति} । त‚था \textbf{ह्याक‚स्मिक‚त्वे} \leavevmode\ledsidenote{\textenglish{168a/PSVTa}}‚{\tiny $_{lb}$}‚ हेतुर‚हित‚त्वे \textbf{स‚त्व}\edtext{}{\lemma{त्वे}\Bfootnote{? श‚ब्द}}स्याभ्युप‚ग‚म्य‚माने \textbf{देशादिनिय‚मः} । आदिश‚ब्दात् काल‚{\tiny $_{lb}$}‚व‚स्तुनिय‚मी न \textbf{स्यादित्युक्तं} ।
	{\color{gray}{\rmlatinfont\textsuperscript{§~\theparCount}}}
	\pend% ending standard par
      ‚{\tiny $_{lb}$}‚

	  
	  \pstart \leavevmode% starting standard par
	\textbf{त‚च्च} वाक्यं पुरुष\textbf{प्र‚य‚त्नेन प्रेरितान्य‚विगुणानि क‚र‚णानि येषां} पुंसान्तेषाम्भ‚{\tiny $_{lb}$}‚व‚द् \textbf{दृष्टं} पुन\textbf{र‚न्य‚था} व‚क्तुकाम‚ताभावे क‚र‚ण‚वैगुण्ये वा नेति । \textbf{न दृष्ट}मिति पुरुष‚{\tiny $_{lb}$}‚व्यापारान्व‚य‚व्य‚तिरेक‚ल‚क्ष‚ण‚स्य \textbf{कार‚ण‚ध‚र्म‚स्य} वाक्यं प्र‚ति \textbf{द‚र्श‚नात पु‚{\tiny $_{१}$}‚रुष‚व्यापार‚{\tiny $_{lb}$}‚ एव} वाक्य‚स्य \textbf{कार‚ण‚म‚तः} कार‚णात् \textbf{पौरुषेय}म‚पि वाक्यं । \href{http://sarit.indology.info/?cref=pv.3.250}{२५३}
	{\color{gray}{\rmlatinfont\textsuperscript{§~\theparCount}}}
	\pend% ending standard par
      ‚{\tiny $_{lb}$}‚

	  
	  \pstart \leavevmode% starting standard par
	अथ नित्य‚न्त‚द् वाक्यं त‚दास्य \textbf{नित्य‚त्वे}भ्युप‚ग‚म्य‚माने \textbf{नित्योप‚ल‚ब्धि}र्वाक्य‚स्य‚{\tiny $_{lb}$}‚ \textbf{स्यात्} । किं कार‚णं [।] त‚स्य नित्य‚स्य स‚तो \textbf{नाव‚र‚ण‚स‚म्भ‚वात्} । आव‚र‚णाभावात् ।
	{\color{gray}{\rmlatinfont\textsuperscript{§~\theparCount}}}
	\pend% ending standard par
      ‚{\tiny $_{lb}$}‚

	  
	  \pstart \leavevmode% starting standard par
	\textbf{अथे}त्यादि व्याख्यानं । अथ \textbf{त‚च्छ‚ब्द‚रूप‚म्वाक्यात्म‚क‚न्नित्यं स्यादुप‚ल‚भ्य‚{\tiny $_{lb}$}‚स्व‚भावं च} । उप‚ल‚भ्यः स्व‚भावोस्येति विग्र‚हः । [।]‚{\tiny $_{२}$}‚ \textbf{स} उप‚ल‚भ्यः \textbf{स्व‚भाव‚{\tiny $_{lb}$}‚स्त‚स्य} वाक्य‚स्य \textbf{क‚दाचिन्नापैति} न हीय‚त \textbf{इति} कृत्वा \textbf{नित्य‚मुप‚ल‚भ्येत} । य‚स्मा\textbf{देवं‚{\tiny $_{lb}$}‚ ‚{\tiny $_{lb}$}‚ \leavevmode\ledsidenote{\textenglish{472/s}}हि स नित्यः स्याद् न कुत‚श्चिद‚पि} ज्ञान‚ज‚न‚न‚ल‚क्ष‚णाद‚पि \textbf{साम‚र्थ्यात् प्र‚च्य‚वेत्} ।‚{\tiny $_{lb}$}‚ किं कार‚ण‚म् [।] \textbf{त‚स्य ज्ञान‚ज‚न‚न‚साम‚र्थ्य}स्य त‚दा\textbf{त्म‚क‚त्वा}न्नित्य‚श‚ब्द‚स्व‚भावात् ।‚{\tiny $_{lb}$}‚ नापि श‚ब्दाज्ज्ञान‚ज‚न‚न‚साम‚र्थ्य‚म‚र्थान्त‚रं य‚स्माद\textbf{र्थान्त‚र‚त्व‚स्य प्रागेव निषिद्ध‚त्वात्} ।‚{\tiny $_{३}$}‚‚{\tiny $_{lb}$}‚ भावानुप‚कार‚क‚त्व‚प्र‚स‚ङ्गा दित्य‚त्रान्त‚रे ।
	{\color{gray}{\rmlatinfont\textsuperscript{§~\theparCount}}}
	\pend% ending standard par
      ‚{\tiny $_{lb}$}‚

	  
	  \pstart \leavevmode% starting standard par
	स्तिमितेन वायुनाव‚र‚णान्नित्यं नोप‚ल‚भ्य‚न्त इति चेदाह । \textbf{नापी}त्यादि ।‚{\tiny $_{lb}$}‚ \textbf{त‚स्य} बाह्य‚स्यो\textbf{प‚ल‚भ्यात्म‚नो} दृश्य‚स्य \textbf{किञ्चिदुप‚ल‚म्भाव‚र‚णं स‚म्भ‚व‚ति} । त‚त्सिद्धौ‚{\tiny $_{lb}$}‚ प्र‚माणाभावात् । \textbf{स‚तोपि वा} विद्य‚मान‚स्यापि चाव‚र‚ण‚स्य \textbf{त‚दात्मान‚म‚ख‚ण्ड‚त‚यो}‚{\tiny $_{lb}$}‚ नित्य‚श‚ब्दात्मान‚म‚प्र‚च्याव‚य‚तः । \textbf{साम‚र्थ्य‚तिर‚स्कारा‚{\tiny $_{४}$}‚योगात्} । ज्ञान‚ज‚न‚न‚श‚क्त्य‚भि‚{\tiny $_{lb}$}‚भ‚वायोगात् । य‚स्मा\textbf{न्न हि त‚त्र} श‚ब्दात्म‚न्य\textbf{तिश‚य‚म}नुत्पाद‚य‚न्नाव‚र‚णाभिम‚तः‚{\tiny $_{lb}$}‚ \textbf{किञ्चित्क‚रो} नाम । \textbf{अकिंचित्क‚र‚श्चा}र्थः कः क‚स्या\textbf{व‚र‚णं} ज्ञान‚विब‚न्ध‚क\textbf{म‚न्य‚द्वेति}‚{\tiny $_{lb}$}‚ प्र‚कारान्त‚रेणोप‚घात‚कं नैवेति याव‚त् । निर्लोठित‚प्राय‚मेत‚त् । \textbf{विचारित‚प्राय‚मेत‚त्}‚{\tiny $_{lb}$}‚ प्राक् । अकिञ्चित्क‚र‚स्याव‚र‚ण‚त्व‚न्दृष्ट‚मिति क‚थ‚य‚न्नाह प‚रः । \textbf{कु‚{\tiny $_{५}$}‚ड्याद‚य} इत्या‚{\tiny $_{lb}$}‚दि । \textbf{कुड्याद‚यो घ‚टादीनां क‚म‚तिश‚य‚मुत्पाद‚य‚न्ति । क‚म्वा} साम‚र्थ्यातिश‚यं \textbf{ख‚ण्ड‚{\tiny $_{lb}$}‚य‚न्ति येनाव‚र‚ण‚मिष्य‚न्ते} । त‚स्माद् य‚था तेऽतिश‚य‚म‚नुत्पाद‚य‚न्तो घ‚टादीनामा‚{\tiny $_{lb}$}‚व‚र‚ण‚मिष्य‚न्ते । त‚था नित्य‚स्यापि श‚ब्द‚स्य किंचिदाव‚र‚ण‚म्भ‚विष्य‚तीत्य‚भिप्रायः ।
	{\color{gray}{\rmlatinfont\textsuperscript{§~\theparCount}}}
	\pend% ending standard par
      ‚{\tiny $_{lb}$}‚

	  
	  \pstart \leavevmode% starting standard par
	\textbf{न ब्रूम} इत्यादिना प‚रिह‚र‚ति । \textbf{ते} कुड्याद‚यः \textbf{क‚ञ्चिद्} घ‚टादिक‚म‚तिशाय‚{\tiny $_{lb}$}‚य‚न्ति वि‚{\tiny $_{६}$}‚शिष्टं स्व‚भावं कुर्व‚न्तीति न ब्रूमः । क‚थ‚न्त‚र्ह्याव‚र‚ण‚मुच्य‚न्त इत्याह ।‚{\tiny $_{lb}$}‚ \textbf{अपि तु न स‚र्व} इत्यादि । न स‚र्व‚घ‚ट‚क्ष‚णास्\textbf{स‚र्व‚स्य} पुरुष‚स्ये\textbf{न्द्रिय‚ज्ञान‚हेत‚वः} [।]‚{\tiny $_{lb}$}‚ किन्त‚र्हि [।] \textbf{प‚र‚स्प‚र‚स‚हितास्तु विष‚येन्द्रियालोकाः} । प‚र‚स्प‚र‚तो \textbf{विशिष्ट‚{\tiny $_{lb}$}‚क्ष‚णान्त‚रोत्पादात्} कार‚णाद् \textbf{विज्ञान‚हेत‚वः} । किं कार‚ण‚म् [।] \textbf{अनुप‚कार्य‚स्य}‚{\tiny $_{lb}$}‚ \leavevmode\ledsidenote{\textenglish{168b/PSVTa}} प‚रैर‚नाधेयातिश‚य‚स्य प‚र‚म्प्र‚त्य\textbf{न‚पेक्षा‚{\tiny $_{७}$}‚योगात्} । प‚रैश्चानाधेयातिश‚यः श‚क्त‚स्व‚भावो‚{\tiny $_{lb}$}‚ ‚{\tiny $_{lb}$}‚ \leavevmode\ledsidenote{\textenglish{473/s}}वा स्याद‚श‚क्त‚स्व‚भावो वा । त‚त्र \textbf{श‚क्त‚स्व‚भाव‚स्य नित्यं} कार्य\textbf{ज‚न‚नं} स्या\textbf{द‚ज‚न‚न‚{\tiny $_{lb}$}‚म‚न्य‚स्ये}त्य‚श‚क्त‚स्व‚भाव‚स्य \textbf{स्यादित्युक्तं प्राक् । ते च} विष‚येन्द्रियाद‚यः । तेन‚{\tiny $_{lb}$}‚ \textbf{प्र‚तिघातिना} कुड्यादिनाऽ\textbf{व्य‚व‚हिता} य‚दा भ‚व‚न्ति त‚दा\textbf{न्योन्य‚स्योप‚कारिणः} [।]‚{\tiny $_{lb}$}‚ किं कार‚ण‚म् [।] \textbf{अव्य‚व‚धाने}त्यादि । न विद्य‚ते व्य‚व‚धानं य‚स्य देश‚स्य‚{\tiny $_{१}$}‚ सो\textbf{व्य‚व‚{\tiny $_{lb}$}‚धान‚देश}स्त‚स्य \textbf{योग्य‚ता} साम‚र्थ्य‚न्त\textbf{त्स‚ह‚कारित्वात् तेषां} विष‚यादीनाम\textbf{न्योन्यातिश‚{\tiny $_{lb}$}‚योत्प‚त्तेः} । तेषां पुन‚रालोकादीनां कुड्यादिकृते \textbf{व्य‚व‚धाने स‚ति} । अव्य‚व‚धान‚{\tiny $_{lb}$}‚देश‚योग्य‚ताल‚क्ष‚ण‚स्य \textbf{हेतोर‚भावात् स‚म‚र्थ‚क्ष‚णान्त‚रानुत्प‚त्तेः} कार‚णाद् घ‚टादि‚{\tiny $_{lb}$}‚\textbf{ज्ञानानुत्प‚त्तिः} । य‚त एवं क्ष‚णिकेषु न स‚र्व‚काल‚मेक‚स्व‚रूपानुवृत्ति\textbf{स्त‚स्मात्‚{\tiny $_{lb}$}‚ प‚{\tiny $_{२}$}‚र्वोत्प‚न्न‚स्य स‚म‚र्थ‚स्ये}न्द्रियादिक्ष‚ण‚स्य स्व‚र‚स‚त एव \textbf{निरोधात् । स‚ति} च व्य‚व‚{\tiny $_{lb}$}‚धाय‚के \textbf{कुड्येन्य‚स्योत्पित्सोः} स‚म‚र्थ‚स्य क्ष‚ण‚स्य य‚थोक्त\textbf{कार‚णाभावेनानुत्प‚त्ते}र्ज्ञान‚{\tiny $_{lb}$}‚कार‚ण‚वैक‚ल्य‚म‚तः कार‚ण‚वैक‚ल्यात् । घ‚टादिषु कुड्यादिव्य‚व‚हितेषु \textbf{ज्ञानानुत्प‚त्ति‚{\tiny $_{lb}$}‚रिति} कृत्वा \textbf{कुड्याद‚य आव‚र‚ण}मुच्य‚ते । \textbf{न पुनः प्राग्} विज्ञान‚ज‚न‚न‚योग्य‚स्य घ‚टा‚{\tiny $_{३}$}‚देः‚{\tiny $_{lb}$}‚ \textbf{प्र‚तिब‚न्धात्} । किङ्कार‚ण‚म् [।] त‚स्य घ‚टादेर्योग्य‚स्व‚भावे स्थित‚स्य कुड्यादिस‚न्नि‚{\tiny $_{lb}$}‚धानेपि स्व‚भावाद‚प्र‚च्युतेः । य‚स्स‚म‚र्थः स स‚म‚र्थ एव । न त‚स्यान्य‚थात्वं क‚र्त्तुं श‚क्य‚ते ।‚{\tiny $_{lb}$}‚ त‚देवं क्ष‚णिकेषु प‚दार्थेषु य‚थोक्त‚विधिनातिश‚य‚म‚कुर्व‚द‚प्याव‚र‚ण‚मुच्य‚ते ।
	{\color{gray}{\rmlatinfont\textsuperscript{§~\theparCount}}}
	\pend% ending standard par
      ‚{\tiny $_{lb}$}‚

	  
	  \pstart \leavevmode% starting standard par
	अधुनातिश‚य‚क‚र‚णेनैवाव‚र‚ण‚मित्याह । \textbf{अथ‚वे}त्यादि । स‚म्भ‚व‚त्य‚पि \textbf{भावा‚{\tiny $_{lb}$}‚नां‚{\tiny $_{४}$}‚} घ‚टादीनां \textbf{क्ष‚णिकानाम‚न्योन्योप‚कारः} कुड्यादिकृतोप्युकारः स‚ह‚कारिकृते‚{\tiny $_{lb}$}‚ उप‚कारे विवादाभावात् । न त्वाव‚र‚ण‚मिन्द्रिय‚विष‚याभ्यां दूर‚व‚र्ति । त‚त् क‚थ‚{\tiny $_{lb}$}‚मिन्द्रिय‚विष‚यावुप‚क‚रोतीत्याह । \textbf{अचिन्त्य‚त्वा}दित्यादि । नैवं चिन्त‚यितुं श‚क्यं दूर‚{\tiny $_{lb}$}‚देश‚व‚र्त्याव‚र‚णं क‚थं विष‚य‚स्योप‚कार‚कं । दूर‚व‚र्त्तिनाप्य‚य‚स्कान्तेनाय‚सः स‚माक‚र्ष‚{\tiny $_{lb}$}‚णात् ।‚{\tiny $_{५}$}‚ \textbf{हेतु}रुपादान‚कार‚णं । \textbf{प्र‚त्य‚यः} स‚ह‚कारिकार‚ण‚न्त‚योः \textbf{साम‚र्थ्य‚स्याचिन्त्य‚{\tiny $_{lb}$}‚त्वाद‚स‚र्व‚विदा}ऽस‚र्व‚ज्ञेन ।
	{\color{gray}{\rmlatinfont\textsuperscript{§~\theparCount}}}
	\pend% ending standard par
      ‚{\tiny $_{lb}$}‚‚{\tiny $_{lb}$}‚\textsuperscript{\textenglish{474/s}}

	  
	  \pstart \leavevmode% starting standard par
	य‚त्र एव‚न्तेन कार‚णेन \textbf{य‚दिन्द्रिय‚विष‚य‚योर्म‚ध्ये स्थित‚माव‚र‚णं} । त‚त्ताविन्द्रिय‚वि‚{\tiny $_{lb}$}‚ष‚याव‚तिशाय‚येद‚पि । केन प्र‚कारेण [।] विज्ञानोत्प‚त्तिवैगुण्य‚तार‚त‚म्येन । अपि‚{\tiny $_{lb}$}‚श‚ब्दः स‚म्भाव‚नायां स‚म्भाव्य‚तेय‚म‚र्थो न ह्य‚त्र किञ्चिद् बाध‚क‚म‚स्तीति ।
	{\color{gray}{\rmlatinfont\textsuperscript{§~\theparCount}}}
	\pend% ending standard par
      ‚{\tiny $_{lb}$}‚

	  
	  \pstart \leavevmode% starting standard par
	न‚नु स‚न्निहितेनाव‚र‚णेन द्वितीयादिक्ष‚णे त‚द् द्र‚व्यं ज्ञान‚ज‚न‚नास‚म‚र्थ‚ञ्ज‚न्य‚ते‚{\tiny $_{lb}$}‚ [।] न तु स‚म्प‚र्क्क‚क्ष‚ण एवानुप‚कारात् । त‚त‚श्च प्र‚थ‚मे क्ष‚णे त‚द् द्र‚व्य‚माव‚र‚ण‚स‚न्नि‚{\tiny $_{lb}$}‚धानेपि दृश्यं स्यात् । ज्ञान‚ज‚न‚न‚साम‚र्थ्य‚स्याप्र‚तिब‚न्धात् ।
	{\color{gray}{\rmlatinfont\textsuperscript{§~\theparCount}}}
	\pend% ending standard par
      ‚{\tiny $_{lb}$}‚

	  
	  \pstart \leavevmode% starting standard par
	नैष दोषः [।] यो ह्याव‚र‚ण‚क्ष‚ण‚स्य ज‚न‚को दृष्टः स आव्रिय‚माण‚स्यापि क्ष‚ण‚{\tiny $_{lb}$}‚\leavevmode\ledsidenote{\textenglish{169a/PSVTa}} स्यास‚म‚र्थ‚स्यैव ज‚न‚को दृष्टो य‚था द्विती‚{\tiny $_{७}$}‚यादिषु क्ष‚णेषु [।] तेनादाव‚प्याव‚र‚ण‚क्ष‚ण‚{\tiny $_{lb}$}‚ज‚न‚क आव्रिय‚माण‚क्ष‚ण‚म‚स‚म‚र्थ ज‚न‚येद् [।] अत एवोच्य‚ते [।] अचिन्त्य‚त्वाद्धेतु‚{\tiny $_{lb}$}‚प्र‚त्य‚य‚साम‚र्थ्य‚स्येति । तेन कुतः प्र‚थ‚म‚क्ष‚णे द्र‚व्य‚स्याव‚र‚ण‚स‚न्निधाने द‚र्श‚नं स्यात् ।‚{\tiny $_{lb}$}‚ तार‚त‚म्य‚ग्र‚ह‚णे चाय‚म‚र्थ उप‚द‚र्शितः [।] वैगुण्य‚मादाव‚र्थ‚स्याव‚र‚ण‚कार‚णेत्रापि कृतं ।‚{\tiny $_{lb}$}‚ द्वितीयादिक्ष‚णेषु त‚दाव‚र‚ण‚म‚तिश‚य‚माध‚{\tiny $_{१}$}‚त्त इति ।
	{\color{gray}{\rmlatinfont\textsuperscript{§~\theparCount}}}
	\pend% ending standard par
      ‚{\tiny $_{lb}$}‚

	  
	  \pstart \leavevmode% starting standard par
	स्यादेत‚द् [।] आव‚र‚ण‚स्य वैगुण्याधाने साम‚र्थ्य‚म‚न्व‚य‚व्य‚तिरेकाभ्याम‚नु‚{\tiny $_{lb}$}‚ ग‚न्त‚व्यं [।] न चान्व‚य‚व्य‚तिरेको विद्येते इत्याह । \textbf{आव‚र‚ण}भेदेनेत्यादि । क‚र्प्प‚ट‚{\tiny $_{lb}$}‚प‚ट‚कुड्यादि व्य‚व‚धान‚भेदेन श‚ब्दादौ श‚ब्द‚ग‚न्ध‚स्प‚र्शेषु । श्रुतिग्र‚ह‚ण‚मुप‚ल‚क्ष‚णार्थ ।‚{\tiny $_{lb}$}‚ तेन श्र‚व‚ण‚द‚र्श‚नादीनां \textbf{मान्द्य}त‚त्पा\textbf{ट‚व‚योर्द‚र्श‚ना}दावाव‚र‚ण‚साम‚र्थ्य‚म‚नुग‚म्य‚ते‚{\tiny $_{२}$}‚ । अन्य‚था‚{\tiny $_{lb}$}‚ य‚द्याव‚र‚णेन विशेषो नाधीय‚ते । त‚दा त‚स्याव‚र‚ण‚स्या\textbf{किंचित्क‚र‚स्य य‚त्स‚न्निधान‚{\tiny $_{lb}$}‚न्त‚स्य स‚न्निधान‚स्याप्य‚स‚न्निधान‚तुल्य‚त्वात् । त‚स्य श‚ब्द‚स्येद‚माव‚र‚ण‚मित्युप‚संहारः‚{\tiny $_{lb}$}‚ स‚म्ब‚न्धो विक‚ल्प‚निर्मित एव स्यान्न व‚स्त्वाश्र‚यः} ।
	{\color{gray}{\rmlatinfont\textsuperscript{§~\theparCount}}}
	\pend% ending standard par
      ‚{\tiny $_{lb}$}‚

	  
	  \pstart \leavevmode% starting standard par
	विक‚ल्पारोपितार्थ‚क्रियाश्र‚यो भ‚विष्य‚तीति चेदाह । \textbf{न चे}त्यादि । न च स‚मा‚{\tiny $_{lb}$}‚रोपानुविधायिन्यो न विक‚{\tiny $_{३}$}‚ल्प‚स‚मारोपितार्थाश्र‚या अर्थ‚क्रियास्तासाम्व‚स्त्वा‚{\tiny $_{lb}$}‚श्र‚य‚त्वात् । य‚स्मान्न \textbf{हि माण‚व‚को द‚ह‚नोप‚चाराद}ग्निर्माण‚व‚क इत्युप‚चारात् \textbf{पाके‚{\tiny $_{lb}$}‚ साध्ये आधीय‚ते} नियुज्य‚ते । य‚त एव\textbf{न्त‚स्मात् स‚त्याम‚पि क‚ल्प‚नायाम‚त‚त्प‚रावृत्त‚यो‚{\tiny $_{lb}$}‚ भावाः} । त‚या क‚ल्प‚न‚या प‚रावृत्तिर्येषान्ते त‚था । त‚द‚भावाद‚त‚त्प‚रा वृत्त‚यः \textbf{किन्तु‚{\tiny $_{lb}$}‚ य‚थास्व‚भा}व‚वृत्त‚य \textbf{एव‚{\tiny $_{४}$}‚ स्युः} । य‚थास्व‚भावं वृत्तिर्येषामिति विग्र‚हः ।
	{\color{gray}{\rmlatinfont\textsuperscript{§~\theparCount}}}
	\pend% ending standard par
      ‚{\tiny $_{lb}$}‚‚{\tiny $_{lb}$}‚\textsuperscript{\textenglish{475/s}}

	  
	  \pstart \leavevmode% starting standard par
	\textbf{त‚दि}ति त‚स्मात् । य‚द्याव‚र‚णेन न विशेष आधीय‚ते त‚दा \textbf{स‚त्य‚प्याव‚र‚णे ज्ञाप‚{\tiny $_{lb}$}‚येयुर्ज्ञानं} ज‚न‚येयु\textbf{रेवेन्द्रियाद‚यः । न चैवं [।] त‚स्मात् तेना}व‚र‚णेना\textbf{धेय‚विशेषा}‚{\tiny $_{lb}$}‚ ज‚न्म‚विशेषा इन्द्रियाद‚य इति ग‚म्य‚न्ते ।
	{\color{gray}{\rmlatinfont\textsuperscript{§~\theparCount}}}
	\pend% ending standard par
      ‚{\tiny $_{lb}$}‚

	  
	  \pstart \leavevmode% starting standard par
	\textbf{न ख‚ल्वेव‚न्नित्यानां श‚ब्दानां क‚स्मिंश्चिदा}व‚र‚ण‚विशेषे \textbf{स‚त्य‚तिश‚य‚हानि}रुत्प‚{\tiny $_{५}$}‚‚{\tiny $_{lb}$}‚त्तिर्वातिश‚य‚स्य ।
	{\color{gray}{\rmlatinfont\textsuperscript{§~\theparCount}}}
	\pend% ending standard par
      ‚{\tiny $_{lb}$}‚

	  
	  \pstart \leavevmode% starting standard par
	\textbf{त‚दि}ति त‚स्मात् । \textbf{य‚दि तेषां} नित्यानां श‚ब्दानां \textbf{ज्ञान‚ज‚न‚नः स्व‚भावः । स‚र्व‚स्य}‚{\tiny $_{lb}$}‚ पुरुष‚स्य \textbf{स‚र्व‚दा} स‚र्वाणि \textbf{स्व‚विष‚याणि ज्ञानानि स‚कृज्ज‚न‚येयुः} । नो चेद् विज्ञान‚{\tiny $_{lb}$}‚ज‚न‚न‚स्व‚भाव‚स्त‚दा \textbf{न क‚दाचित्} क‚स्यचित्पुरुष‚स्य किञ्चिद् विज्ञानं \textbf{ज‚न‚येयुरि‚{\tiny $_{lb}$}‚त्येकान्त एषः । क‚स्य‚चित् स‚ह‚कारिणो विक‚ल‚त्वान्नि}त्य‚स्यापि श‚ब्द‚स्य‚{\tiny $_{६}$}‚ स‚र्व‚काल‚{\tiny $_{lb}$}‚\textbf{म‚श्रुतिरिति} चेत् ।
	{\color{gray}{\rmlatinfont\textsuperscript{§~\theparCount}}}
	\pend% ending standard par
      ‚{\tiny $_{lb}$}‚

	  
	  \pstart \leavevmode% starting standard par
	\textbf{स्यादेत}दित्यादिना व्याच‚ष्टे । \textbf{अपि तु किंचिदेषां} नित्यानां श‚ब्दानां \textbf{प्र‚तिप‚त्तौ}‚{\tiny $_{lb}$}‚ प्र‚तिप‚त्तिनिमित्तं \textbf{स‚ह‚कारि प्र‚तिनिय}तं । क‚स्य‚चित् किञ्चिदेव व‚स्तु स्थित्या‚{\tiny $_{lb}$}‚ निय‚त‚म‚स्ति । त‚त्स‚ह‚कारि । \textbf{क‚दा}चित्काले \textbf{क‚स्य‚चि}च्छ‚ब्द‚स्य \textbf{भ‚व‚तीति य‚त्कृतं}‚{\tiny $_{lb}$}‚ स‚ह‚कारिकृत\textbf{मेषां} श‚ब्दानां \textbf{क‚दाचित् क्व‚चित्} प्र‚देशे \textbf{श्र‚व‚ण‚मिति} । \href{http://sarit.indology.info/?cref=pv.3.251}{२५४}
	{\color{gray}{\rmlatinfont\textsuperscript{§~\theparCount}}}
	\pend% ending standard par
      ‚{\tiny $_{lb}$}‚

	  
	  \pstart \leavevmode% starting standard par
	\textbf{काम}‚{\tiny $_{७}$}‚मित्यादि सि द्धा न्त वा दी । काम‚मेव‚मित्य‚र्थः । अन्य‚स्य स‚ह‚कारिणः \leavevmode\ledsidenote{\textenglish{169b/PSVTa}}‚{\tiny $_{lb}$}‚ प्र‚तीक्षा प्र‚तीक्ष‚ण‚म‚स्तु न निवार्य‚ते । केव‚लं निय‚म‚स्तु विरुध्य‚ते । पूर्व‚स्व‚भाव एव‚{\tiny $_{lb}$}‚ श‚ब्दः । स्थित इत्य‚यं निय‚मो न स्यादुप‚कार‚क‚स्यापेक्ष‚णीय‚त्वात् ।
	{\color{gray}{\rmlatinfont\textsuperscript{§~\theparCount}}}
	\pend% ending standard par
      ‚{\tiny $_{lb}$}‚

	  
	  \pstart \leavevmode% starting standard par
	\textbf{न वै व‚म‚य}मित्यादिनैत‚देव व्याच‚ष्टे । \textbf{न वै कार‚णानां स‚ह‚कारीणि प्र‚तिक्षि‚{\tiny $_{lb}$}‚‚{\tiny $_{lb}$}‚ \leavevmode\ledsidenote{\textenglish{476/s}}पामः । किन्त्}व‚पेक्ष‚न्त एव \textbf{कार‚णानि स‚ह‚कारिणं किं} भूतं \textbf{त‚द‚{\tiny $_{१}$}‚व‚स्थोप‚कारिणं} ।‚{\tiny $_{lb}$}‚ य‚थाभिम‚त‚कार्य‚ज‚न‚न‚स्व‚भावाव‚स्थोप‚कारिणं [।] किं कार‚णं [।] त‚तः‚{\tiny $_{lb}$}‚ स‚ह‚कारिणः स‚काशाल्ल‚भ्य‚स्यातिश‚य‚स्य कार्ये ज‚न्ये उप‚योगाद् व्यापारात् ।‚{\tiny $_{lb}$}‚ \textbf{त‚था श‚ब्दोपि} वैदिको \textbf{य‚दि किञ्चित्} स‚ह‚कारिण‚म\textbf{पेक्ष्य कार्य}मात्म‚विष‚यं ज्ञानं‚{\tiny $_{lb}$}‚ \textbf{कुर्यात्} । क‚रोतु कः प्र‚तिषेद्धा [।] केव‚लं \textbf{पूर्व‚स्व‚भाव‚निय‚त इति} पूर्व‚स्मिन्नेव‚{\tiny $_{lb}$}‚ स्व‚भावे स्थित \textbf{इत्येत‚न्न स्यात्} । किं कार‚णं [।] त‚स्य पूर्व‚{\tiny $_{२}$}‚स्व‚भाव‚स्य प्र‚च्युतेः ।‚{\tiny $_{lb}$}‚ \textbf{अपेक्षाच्च स‚ह‚कारिण}स्स‚काशात् \textbf{स्व‚भावान्त‚र‚स्या}पूर्व‚क‚स्य \textbf{प्र‚तिल‚म्भात् ।‚{\tiny $_{lb}$}‚ अतिश‚य‚प्र‚तिल‚म्भाभावेऽपेक्षायोगात्} । य‚स्मान्न ह्य‚नुप‚र्य‚पेक्ष‚त इति । \textbf{उक्त‚{\tiny $_{lb}$}‚मेत‚त्प्राक्} ।
	{\color{gray}{\rmlatinfont\textsuperscript{§~\theparCount}}}
	\pend% ending standard par
      ‚{\tiny $_{lb}$}‚

	  
	  \pstart \leavevmode% starting standard par
	\textbf{अथ} स‚ह‚कारिणः स‚काशा\textbf{च्छ‚ब्दोर्थान्त‚र‚भूत‚मुप‚कारं ल‚भ‚ते} । त‚दोप‚कार‚स्य‚{\tiny $_{lb}$}‚ चार्थान्त‚र‚त्वे । त‚स्याय‚मुप‚कार \textbf{इति स‚म्ब‚न्धा\textbf{द्य}भावोप्युक्तः} । आदिश‚ब्दाद् य‚दि‚{\tiny $_{lb}$}‚ स‚म्ब‚न्ध‚{\tiny $_{३}$}‚सिद्ध्य‚र्थं स‚ह‚कारिकृत उप‚कारे श‚ब्द‚कृत उप‚कारः क‚ल्प्य‚ते त‚दा त‚त्राप्य‚{\tiny $_{lb}$}‚प‚र‚स्त‚त्राप्य‚प‚र इत्य‚न‚व‚स्थादोषाद‚योप्युक्ताः । त‚स्य च श‚ब्द‚स्याज्ञेय‚त्वं प्र‚स‚क्तं ।‚{\tiny $_{lb}$}‚ किं कार‚णं [।] स‚ह‚कारिकृतादेवोप‚काराद‚र्थान्त‚र‚भूताज्ज्ञानोत्प‚त्तैः ।
	{\color{gray}{\rmlatinfont\textsuperscript{§~\theparCount}}}
	\pend% ending standard par
      ‚{\tiny $_{lb}$}‚

	  
	  \pstart \leavevmode% starting standard par
	य‚त एव‚न्त‚स्मात् । \textbf{एष श‚ब्दो नेन्द्रियं} श्रोत्राख्यं [।] नेन्द्रियार्थ‚यो\textbf{स्स‚न्निक‚र्ष ।‚{\tiny $_{lb}$}‚ नात्मानं} । एत‚च्च प‚र‚प्र‚सिद्ध्योक्तं । \textbf{अन्य‚च्चेति} प्र‚य‚त्नादि‚{\tiny $_{४}$}‚कं । किम्भूत‚म् [।]‚{\tiny $_{lb}$}‚ ‚{\tiny $_{lb}$}‚ \leavevmode\ledsidenote{\textenglish{477/s}}\textbf{यिज्ञानोत्प‚त्तिस‚माश्र‚य}म्विज्ञानोत्प‚त्तिस‚ह‚कारिणं \textbf{स्व‚ज्ञान‚ज‚न‚नेऽपेक्ष‚ते} । किं कार‚णं‚{\tiny $_{lb}$}‚ [।] स‚र्व‚स्य त‚त्र नित्ये श‚ब्देनुप‚योगात् ।
	{\color{gray}{\rmlatinfont\textsuperscript{§~\theparCount}}}
	\pend% ending standard par
      ‚{\tiny $_{lb}$}‚

	  
	  \pstart \leavevmode% starting standard par
	\textbf{अपि} चेत्यादि [।] \textbf{य‚द्य‚व्यापिता} त‚दा \textbf{स‚र्व‚त्र} देशे \textbf{तेषां श‚ब्दानाम‚नुप‚ल‚म्भः‚{\tiny $_{lb}$}‚ स्यात् । त‚था} हि \textbf{क‚थ‚मेक‚देश‚व‚र्त्तिनं} श‚ब्दं \textbf{त‚च्छून्य‚देश‚स्थितः} पुरुष \textbf{उप‚ल‚भेत ।‚{\tiny $_{lb}$}‚ अप्राप्त‚ग्र‚ह‚ण‚प‚क्षेऽय‚म‚दोष इति चेत्} । अप्राप्त एव श्रोत्र‚दे‚{\tiny $_{५}$}‚शं श‚ब्दः श्रोत्रेन्द्रियेण‚{\tiny $_{lb}$}‚ गृह्य‚ते त‚तः श‚ब्द‚शून्य‚देशाव‚स्थितोपि श‚ब्दं गृह्णीयाद‚त‚स्स‚र्व‚त्रानुप‚ल‚म्भ‚दोषो न‚{\tiny $_{lb}$}‚ भ‚व‚तीति ।
	{\color{gray}{\rmlatinfont\textsuperscript{§~\theparCount}}}
	\pend% ending standard par
      ‚{\tiny $_{lb}$}‚

	  
	  \pstart \leavevmode% starting standard par
	\textbf{नैत‚देवं} । किं कार‚णं [।] \textbf{त‚त्राप्य‚प्रा}प्त‚ग्र‚ह‚ण‚प‚क्षेपि न व्य‚व‚हित‚स्य ग्र‚ह‚णं‚{\tiny $_{lb}$}‚ स‚म्भ‚व‚ति । किं कार‚णं [।] त‚स्य श‚ब्द‚स्य \textbf{योग्य‚दे}शे याव‚त् \textbf{स्थि}तिस्त‚स्या‚{\tiny $_{lb}$}‚स्तार‚त‚म्य‚स्या\textbf{पेक्ष‚णा}दिन्द्रिय‚स्य । किमिव [।] \textbf{अय‚स्कान्तादिव‚त्} । य‚थाय‚स्कान्त‚{\tiny $_{lb}$}‚स्याप्राप्ताक‚र्ष‚क‚त्वेपि नायो‚{\tiny $_{६}$}‚ग्य‚देशाव‚स्थित‚लोहाक‚र्ष‚ण‚न्त‚द्व‚त् । आदिश‚ब्दाद्‚{\tiny $_{lb}$}‚ आशीविषादिर्दीपाद्युप‚घातं कुर्व‚न् गृह्य‚ते । \textbf{अन्य‚थे}ति य‚दि श‚ब्द‚स्य योग्य‚{\tiny $_{lb}$}‚देशाव‚स्थान‚न्त‚द्ग्राह‚क‚मिन्द्रियं नोपेक्ष‚त । त‚दा योग्य‚देशाव‚स्थान‚तार‚त‚म्य‚भेदेन‚{\tiny $_{lb}$}‚ \textbf{स्प‚ष्टास्प‚ष्ट}प्र‚तीति\textbf{भेदो न स्यात्} । भ‚व‚ति च [।] त‚स्मात् योग्य‚देशापेक्ष‚त्वं ।‚{\tiny $_{lb}$}‚ योग्य‚देशाव‚स्थित‚स्याप्राप्त‚स्य श‚ब्द‚स्य ग्र‚ह‚णेपि स्प‚ष्टास्प‚ष्ट‚प्र‚ति‚{\tiny $_{७}$}‚भास‚भेदो न \leavevmode\ledsidenote{\textenglish{170a/PSVTa}}‚{\tiny $_{lb}$}‚ स्यादित्याह । \textbf{स‚ति चोप‚ल‚म्भ‚प्र‚त्य‚ये} ताल्वादिव्यापार‚ल‚क्ष‚णे \textbf{स‚र्व‚देशे} स‚मीपे दूरे च‚{\tiny $_{lb}$}‚ श‚ब्दास्तु\textbf{ल्य‚मुप‚ल‚भ्येर‚न्} । न चैवं [।] \textbf{त‚स्मात् नाव्यापिनः} । न तु बौ द्धै रिन्द्रिय‚{\tiny $_{lb}$}‚देश‚म‚प्राप्त‚स्यैव श‚ब्द‚स्येन्द्रियेण ग्र‚ह‚ण‚मिष्य‚ते क‚थ‚न्त‚स्य स्प‚ष्टास्प‚ष्ट‚श्रुतिभेदः ।‚{\tiny $_{lb}$}‚ त‚दुक्तं ।
	{\color{gray}{\rmlatinfont\textsuperscript{§~\theparCount}}}
	\pend% ending standard par
      ‚{\tiny $_{lb}$}‚
	  \bigskip
	  \begingroup
	
	    
	    \stanza[\smallbreak]
	  {\normalfontlatin\large ``\qquad}येषाम‚प्राप्त एवायं श‚ब्दः श्रोत्रेण गृह्य‚ते ।&‚{\tiny $_{lb}$}‚तेषाम‚प्राप्तितुल्य‚त्वं दूर‚व्य‚व‚हितादिषु ॥&‚{\tiny $_{lb}$}‚त‚त्र दूर‚स‚मीप‚स्थ‚ग्र‚ह‚णाग्र‚ह‚णे स‚मे [।]&‚{\tiny $_{lb}$}‚स्यातान्न च क्र‚मो नापि तीव्र‚म‚न्दादिस‚म्भ‚व इति ।\edtext{\textsuperscript{*}}{\edlabel{pvsvt_477-1}\label{pvsvt_477-1}\lemma{*}\Bfootnote{Kumārila. }}{\normalfontlatin\large\qquad{}"}\&[\smallbreak]
	  
	  
	  
	  \endgroup
	‚{\tiny $_{lb}$}‚

	  
	  \pstart \leavevmode% starting standard par
	एव‚म्म‚न्य‚ते । य‚स्य स्प‚ष्टास्प‚ष्ट‚प्र‚तिभासानि स‚र्वाण्येव विज्ञानान्य‚भ्रान्तानि‚{\tiny $_{lb}$}‚ त‚स्याय‚न्दोषो न बौ द्ध स्यास्प‚ष्ट‚प्र‚तिभास‚स्य ज्ञान‚स्य भ्रान्त‚त्वाभ्युप‚ग‚मात् । अप‚रा‚{\tiny $_{lb}$}‚प‚र‚देशोत्प‚त्त्या चाग‚च्छ‚तः श‚ब्द‚स्य ग्र‚ह‚णात् क्र‚मो गृह्य‚ते क‚र्ण्ण‚दे‚{\tiny $_{२}$}‚शे च तीव्र‚स्य‚{\tiny $_{lb}$}‚ ‚{\tiny $_{lb}$}‚ ‚{\tiny $_{lb}$}‚ \leavevmode\ledsidenote{\textenglish{478/s}}श‚ब्द‚स्य म‚न्द‚स्य चोत्प‚त्तेस्तीव्र‚म‚न्दादिस‚म्भ‚व इति न काचित् क्ष‚तिः ।
	{\color{gray}{\rmlatinfont\textsuperscript{§~\theparCount}}}
	\pend% ending standard par
      ‚{\tiny $_{lb}$}‚

	  
	  \pstart \leavevmode% starting standard par
	स्यादेत‚द् [।] य‚था दूरे रूपं र‚जोनीहारादिसंसृष्टं गृह्य‚ते स‚मीपे तु त‚द‚भा‚{\tiny $_{lb}$}‚वात् स्प‚ष्टं । त‚था श‚ब्दोपि [।]
	{\color{gray}{\rmlatinfont\textsuperscript{§~\theparCount}}}
	\pend% ending standard par
      ‚{\tiny $_{lb}$}‚

	  
	  \pstart \leavevmode% starting standard par
	दूरास‚न्नादिभेदेन स्प‚ष्टास्प‚ष्टः प्र‚तीय‚त इति ।
	{\color{gray}{\rmlatinfont\textsuperscript{§~\theparCount}}}
	\pend% ending standard par
      ‚{\tiny $_{lb}$}‚

	  
	  \pstart \leavevmode% starting standard par
	त‚द‚युक्तं । य‚तो रूपस्यर‚जोनीहारादेस्संसृष्ट‚ताग्र‚ह‚णं य‚दि ताव‚त्त‚योः‚{\tiny $_{lb}$}‚ पृथ‚क् पृथ‚ग् ग्र‚ह‚ण‚न्त‚दा दूरास‚न्न‚व‚र्त्तिनोः‚{\tiny $_{३}$}‚ पुरुष‚योस्तुल्यो रूप‚प्र‚तिभासः स्याद्‚{\tiny $_{lb}$}‚ य‚थाव‚स्थितेन स्व‚रूपेण ग्र‚ह‚णात् । अथैक‚त्वेन त‚योर्ग्र‚ह‚णं संसृष्ट‚ताग्र‚ह‚णं क‚थ‚म‚स्प‚ष्ट‚{\tiny $_{lb}$}‚प्र‚तिभासं ज्ञानं भ्रान्त‚न्न स्यात् । भिन्नानामेक‚त्वे ग्र‚ह‚णात् क‚थं चैक रूप‚स्यानेकाकारः‚{\tiny $_{lb}$}‚ प्र‚तिभासः । त‚दुक्तं ।
	{\color{gray}{\rmlatinfont\textsuperscript{§~\theparCount}}}
	\pend% ending standard par
      ‚{\tiny $_{lb}$}‚
	  \bigskip
	  \begingroup
	
	    
	    \stanza[\smallbreak]
	  {\normalfontlatin\large ``\qquad}जातो नामाश्र‚योन्यान्य‚श्चेत‚सां त‚स्य व‚स्तुन ः।&‚{\tiny $_{lb}$}‚एक‚स्यैव कुतो रूप‚म्भिन्नाकाराव‚भासि त‚दिति ।\edtext{\textsuperscript{*}}{\edlabel{pvsvt_478-1}\label{pvsvt_478-1}\lemma{*}\Bfootnote{Kumārila. }}{\normalfontlatin\large\qquad{}"}\&[\smallbreak]
	  
	  
	  
	  \endgroup
	‚{\tiny $_{lb}$}‚
	    
	    \stanza[\smallbreak]
	  न‚नु देश‚कालाव्यापि‚{\tiny $_{४}$}‚नः श‚ब्दाः ।\&[\smallbreak]
	  
	  
	  ‚{\tiny $_{lb}$}‚
	  \bigskip
	  \begingroup
	
	    
	    \stanza[\smallbreak]
	  {\normalfontlatin\large ``\qquad}य‚स्माच्छ‚व्द‚स्य नित्य‚त्वं श्रोत्र‚ज‚प्र‚त्य‚भिज्ञ‚या ।&‚{\tiny $_{lb}$}‚विभुत्वं च स्थित‚न्त‚स्य कोध्य‚व‚स्येद् विप‚र्य‚यं ॥&‚{\tiny $_{lb}$}‚देश‚भेदेन भिन्न‚त्व‚मित्येत‚च्चानुमानिकं&‚{\tiny $_{lb}$}‚प्र‚त्य‚क्ष‚स्तु स एवेति प्र‚त्य‚य‚स्त‚स्य बाध‚कः ।&‚{\tiny $_{lb}$}‚प‚र्यायेण य‚था लोके भिन्नान्देशान् ब्र‚ज‚न्न‚पि ।&‚{\tiny $_{lb}$}‚देव‚द‚त्तो न भिद्येत त‚था श‚ब्दो न भिद्य‚ते ।&‚{\tiny $_{lb}$}‚त‚स्माद्या स‚र्व‚कालेषु स‚र्व‚देशेषु चैक‚ता ।&‚{\tiny $_{lb}$}‚प्र‚त्य‚क्ष‚प्र‚त्य‚भिज्ञान‚प्र‚सिद्धा सास्य बाधि‚{\tiny $_{५}$}‚का ।\edtext{\textsuperscript{*}}{\edlabel{pvsvt_478-2}\label{pvsvt_478-2}\lemma{*}\Bfootnote{Ibid. }}{\normalfontlatin\large\qquad{}"}\&[\smallbreak]
	  
	  
	  
	  \endgroup
	‚{\tiny $_{lb}$}‚

	  
	  \pstart \leavevmode% starting standard par
	त‚स्माद् व्यापिनः श‚ब्दा इति ।
	{\color{gray}{\rmlatinfont\textsuperscript{§~\theparCount}}}
	\pend% ending standard par
      ‚{\tiny $_{lb}$}‚

	  
	  \pstart \leavevmode% starting standard par
	अत्राप्याह । स‚र्वेषां पुंसां \textbf{युग‚प}त्स‚र्व‚श‚ब्दोप‚ल‚म्भः स्यात्, तेषां श‚ब्दानां \textbf{व्या‚{\tiny $_{lb}$}‚पिता य‚दि । न हि} क‚श्चिच्छ‚ब्दः क्व‚चिद्देशे नास्ति किन्तु स‚र्वः श‚ब्दः स‚र्व‚त्रास्ति‚{\tiny $_{lb}$}‚ \leavevmode\ledsidenote{\textenglish{479/s}}व्यापित्वात् । इति हेतोः । स‚र्व‚श‚ब्दा युग‚प‚दुप‚ल‚भ्येर‚न् स‚र्व‚देशाव‚स्थितैश्च पुरु‚{\tiny $_{lb}$}‚षैरुप‚ल‚भ्येर‚न् । किं कार‚णं [।] योग्येन्द्रिय‚त्वात् पुंसां । विष‚य‚स्य श‚ब्द‚ल‚क्ष‚ण‚स्य‚{\tiny $_{lb}$}‚ नित्य‚स्य‚{\tiny $_{६}$}‚ स‚तो व्यापित्वेन स‚दा स‚र्व‚त्र \textbf{स‚न्निहित‚त्वात्} । नित्य‚त्वादेव चानाधेया‚{\tiny $_{lb}$}‚तिश‚य‚स्य प्र‚ब‚न्धाच्च ।
	{\color{gray}{\rmlatinfont\textsuperscript{§~\theparCount}}}
	\pend% ending standard par
      ‚{\tiny $_{lb}$}‚

	  
	  \pstart \leavevmode% starting standard par
	\textbf{संस्कृत‚स्ये}त्यादि । क‚र्म‚णि क‚र्त्त‚रि वा-ष‚ष्ठी । तेनाय‚म‚र्थः [।] प्र‚य‚त्ना‚{\tiny $_{lb}$}‚भिह‚त‚वायुना संस्कृत‚स्य श‚ब्द‚स्य संस्कृतेनैवेन्द्रियेणो\textbf{प‚ल‚म्भे चा}भ्युप‚ग‚म्य‚माने ।‚{\tiny $_{lb}$}‚ न य‚थोक्त‚दोष इति ।
	{\color{gray}{\rmlatinfont\textsuperscript{§~\theparCount}}}
	\pend% ending standard par
      ‚{\tiny $_{lb}$}‚

	  
	  \pstart \leavevmode% starting standard par
	उत्त‚र‚माह । \textbf{कः संस्क‚र्त्ता विकारिणः} श‚ब्द‚स्य । नैव क‚श्चित् ।
	{\color{gray}{\rmlatinfont\textsuperscript{§~\theparCount}}}
	\pend% ending standard par
      ‚{\tiny $_{lb}$}‚

	  
	  \pstart \leavevmode% starting standard par
	स्यादेत‚दित्यादिना व्याच‚ष्टे । स‚{\tiny $_{७}$}‚र्व‚कालं \textbf{स‚न्न‚पि न स‚र्वः श‚ब्द उप‚ल‚भ्य‚ते} \leavevmode\ledsidenote{\textenglish{170b/PSVTa}}‚{\tiny $_{lb}$}‚ स‚र्वेण पुरुषेण । किं कार‚णं [।] \textbf{संस्कृत‚स्य} श‚ब्द‚स्य प्र‚य‚त्नाभिह‚तेन वायुना \textbf{संस्कृते‚{\tiny $_{lb}$}‚नेवेन्द्रियेणोप‚ल‚म्भादिति । त‚त्र} त‚योर्म‚ध्ये न ताव‚त् \textbf{संस्कृत‚स्य} श‚ब्द‚स्यो\textbf{प‚ल‚म्भः} ।‚{\tiny $_{lb}$}‚ किं कार‚ण‚म् [।] \textbf{अनाधेय‚विकार‚स्य} श‚ब्द‚स्य \textbf{संस्कार‚योगात् । इन्द्रिय‚स्य} त्व‚नि‚{\tiny $_{lb}$}‚त्य‚त्वादाधेय‚विशेष‚स्य प्र‚य‚त्नाभिह‚तेन वायुना \textbf{स्यात् संस्कारः} । य‚दाह [।]
	{\color{gray}{\rmlatinfont\textsuperscript{§~\theparCount}}}
	\pend% ending standard par
      ‚{\tiny $_{lb}$}‚

	  
	  \pstart \leavevmode% starting standard par
	प्र‚य‚त्नाभि‚{\tiny $_{१}$}‚ह‚तो वायुः कोष्ठ्यो यातीत्य‚संश‚यं \href{http://sarit.indology.info/?cref=\%C5\%9Bv-\%C5\%9Babda.122}{श्लो० श‚ब्द० १२२}
	{\color{gray}{\rmlatinfont\textsuperscript{§~\theparCount}}}
	\pend% ending standard par
      ‚{\tiny $_{lb}$}‚

	  
	  \pstart \leavevmode% starting standard par
	क‚र्ण्ण‚व्योम‚नि संप्राप्तः श‚क्तिं श्रोत्रे निय‚च्छ‚ति \href{http://sarit.indology.info/?cref=\%C5\%9Bv-\%C5\%9Babda.124}{श्लो० श‚ब्द० १२४}
	{\color{gray}{\rmlatinfont\textsuperscript{§~\theparCount}}}
	\pend% ending standard par
      ‚{\tiny $_{lb}$}‚

	  
	  \pstart \leavevmode% starting standard par
	श‚ब्द‚रूप‚प्र‚तिप‚त्त्य‚न्य‚थानुप‚त्त्या चेन्द्रिय‚स्य श‚क्तिः क‚ल्प्य‚ते । श‚क्तिरूप‚श्च‚{\tiny $_{lb}$}‚ संस्कार इष्य‚त इति ।
	{\color{gray}{\rmlatinfont\textsuperscript{§~\theparCount}}}
	\pend% ending standard par
      ‚{\tiny $_{lb}$}‚

	  
	  \pstart \leavevmode% starting standard par
	त‚त्राह । त‚द‚पि संस्कृत‚मिन्द्रियं \textbf{शृणुयान्निखिलं} निर‚व‚शेषं श‚ब्दं ।
	{\color{gray}{\rmlatinfont\textsuperscript{§~\theparCount}}}
	\pend% ending standard par
      ‚{\tiny $_{lb}$}‚

	  
	  \pstart \leavevmode% starting standard par
	\textbf{त‚त्रे}त्यादिना व्याच‚ष्टे । \textbf{य‚दि संस्कृतेनैवे}न्द्रियेण श‚ब्द\textbf{स्योप‚ल‚म्भ इति} कृत्वा‚{\tiny $_{lb}$}‚ ऽसंस्कृतेन्द्रियः पुरुषो \textbf{नोप‚ल‚भ‚ते} । त‚दा \textbf{य‚स्येन्द्रिय‚संस्कारः‚{\tiny $_{२}$}‚} कृतः \textbf{स स‚र्व‚श‚ब्दान्‚{\tiny $_{lb}$}‚ युग‚प‚च्छ‚णुयादिति} पूर्वः \textbf{प्र‚स‚ङ्गोऽनिवृत्त एव} ।
	{\color{gray}{\rmlatinfont\textsuperscript{§~\theparCount}}}
	\pend% ending standard par
      ‚{\tiny $_{lb}$}‚

	  
	  \pstart \leavevmode% starting standard par
	अथ स्याद् [।] य‚था श‚ब्द‚प्र‚तिप‚त्त्य‚न्य‚थानुप‚प‚त्त्यैन्द्रिय‚स्य संस्कार‚क‚ल्प‚ना‚{\tiny $_{lb}$}‚ ‚{\tiny $_{lb}$}‚ \leavevmode\ledsidenote{\textenglish{480/s}}[।] त‚था श‚ब्द‚विशेष‚प्र‚तिप‚त्त्य‚न्य‚थानुप‚प‚त्त्या संस्कार‚विशेष‚क‚ल्प‚ना । य‚दाह ।
	{\color{gray}{\rmlatinfont\textsuperscript{§~\theparCount}}}
	\pend% ending standard par
      ‚{\tiny $_{lb}$}‚

	  
	  \pstart \leavevmode% starting standard par
	त‚थैव त‚द्विशेषोपि विशिष्ट‚श्र‚व‚णाद् भ‚वेदिति ।
	{\color{gray}{\rmlatinfont\textsuperscript{§~\theparCount}}}
	\pend% ending standard par
      ‚{\tiny $_{lb}$}‚

	  
	  \pstart \leavevmode% starting standard par
	त‚स्मात् संस्कार‚भेदात् प्र‚तिविष‚य‚म्भिन्न‚त्वादिन्द्रिय‚स्यैकार्थ‚निय‚मः । एक‚स्यैव‚{\tiny $_{lb}$}‚ श‚ब्द‚स्य ग्र‚{\tiny $_{३}$}‚ह‚णं य‚दि । \href{http://sarit.indology.info/?cref=pv.3.255}{२५८}
	{\color{gray}{\rmlatinfont\textsuperscript{§~\theparCount}}}
	\pend% ending standard par
      ‚{\tiny $_{lb}$}‚

	  
	  \pstart \leavevmode% starting standard par
	एवं स‚त्य‚नेक‚श‚ब्द‚संघाते । विचित्र‚श‚ब्द‚मूहात्म‚के क‚ल‚क‚ल‚श‚ब्दे श्रुतिः क‚थं‚{\tiny $_{lb}$}‚ नैव स्यात् । दृष्टा च ।
	{\color{gray}{\rmlatinfont\textsuperscript{§~\theparCount}}}
	\pend% ending standard par
      ‚{\tiny $_{lb}$}‚

	  
	  \pstart \leavevmode% starting standard par
	\textbf{अथापी}त्यादिना व्याच‚ष्टे । इन्द्रिय‚स्य ये \textbf{संस्कारा}स्ते \textbf{श‚ब्दानां प्र‚तिनिय‚ता‚{\tiny $_{lb}$}‚स्त‚त्रै}त‚स्मिन् \textbf{संस्कार‚प्र‚तिनिय‚मे केन‚चित् संस्कृत‚मिन्द्रियं क‚स्य‚चिदे}व श‚ब्द‚स्य‚{\tiny $_{lb}$}‚ \textbf{ग्राह‚क‚मिति न युग‚प‚त् स‚र्व‚श‚ब्द‚श्रुतिरि}ति । एवं \textbf{संस्कार‚विशेषाच्छ्रुतिनिय‚म‚{\tiny $_{lb}$}‚ इन्द्रियाणाम‚{\tiny $_{४}$}‚}भ्युप‚ग‚म्य‚माने \textbf{अनेक‚श‚ब्द‚स‚ङ्घात‚स्य क‚ल‚क‚ल‚श‚ब्द‚स्य श्रुतिर्न स्यात्} ।‚{\tiny $_{lb}$}‚ य‚स्मा\textbf{न्न ह्येकः श‚ब्दः क‚ल‚क‚लो ना}म । किं कार‚ण‚म् [।] \textbf{भिन्न‚स्व‚भावानां}‚{\tiny $_{lb}$}‚ वेणुमृद‚ङ्ग‚काव्य‚पाठ‚गीत‚श‚ब्दानां क‚ल‚क‚ले \textbf{युग‚प‚च्छ्र‚व‚णात्} । नापि भिन्न‚स्व‚भाव‚{\tiny $_{lb}$}‚ग्र‚ह‚णेप्य‚भेदो य‚तः \textbf{स्व‚भाव‚भेदाश्र‚य‚त्वाच्च भेद‚व्य‚व‚स्थितेः} ।
	{\color{gray}{\rmlatinfont\textsuperscript{§~\theparCount}}}
	\pend% ending standard par
      ‚{\tiny $_{lb}$}‚

	  
	  \pstart \leavevmode% starting standard par
	न‚नु य‚दानेकः श‚ब्दः श्रूय‚ते । त‚दानेक‚श‚ब्द‚श्र‚व‚णान्य‚थानु‚{\tiny $_{५}$}‚प‚प‚त्त्यापीन्द्रिय‚स्या‚{\tiny $_{lb}$}‚नेकः संस्कारः क‚ल्प्य‚ते त‚तोनेक‚श‚ब्द‚श्र‚व‚ण‚म‚विरुद्ध‚मेव ।
	{\color{gray}{\rmlatinfont\textsuperscript{§~\theparCount}}}
	\pend% ending standard par
      ‚{\tiny $_{lb}$}‚

	  
	  \pstart \leavevmode% starting standard par
	एव‚र्म्म‚न्य‚ते । ये प्र‚य‚त्नाभिह‚तैर्वायुभिः संस्कारा आधीय‚न्ते । ते य‚दीन्द्रि‚{\tiny $_{lb}$}‚याद‚भिन्नास्त‚दा संस्कार‚व‚हुत्वं कुतः । इन्द्रिय‚स्यैक‚त्वाद् [।] अथ भिन्नाः क‚थं‚{\tiny $_{lb}$}‚ त‚र्हीन्द्रियं संस्कृतं । त‚स्य च संस्कारा इति \textbf{स‚म्ब‚न्ध‚श्च न सिध्य‚ति} ये च निष्प‚न्ने‚{\tiny $_{lb}$}‚ भ‚व‚न्ति ते क‚थ‚न्त‚त्स्व‚भावा विरुद्ध‚ध‚र्माध्यासात् । ते‚{\tiny $_{६}$}‚न भिन्नाभिन्ना अपि संस्कारा‚{\tiny $_{lb}$}‚ न युज्य‚न्त इति य‚त्किञ्चिदेत‚त् । न क‚ल‚क‚ले युग‚प‚द‚नेक‚श‚ब्द‚ग्र‚ह‚णं किन्तु‚{\tiny $_{lb}$}‚ क्र‚मेणैव त‚त्रैकैकः श‚ब्दः श्रूय‚ते । तानि च श्र‚व‚ण‚ज्ञानानि ल‚घुवृत्तीनि । त‚तो‚{\tiny $_{lb}$}‚ \textbf{ल‚घुवृत्तेः} कार‚णात् तेषु क्र‚मेण गृह्य‚माणेष्व‚पि \textbf{स‚कृच्छ्रुतिर्भ्रान्तिरिति चेत्} । त‚दा‚{\tiny $_{lb}$}‚ \leavevmode\ledsidenote{\textenglish{171a/PSVTa}} \textbf{वंशादिस्व‚र‚धारा}यां ये \textbf{ग‚मॄ}काः स्व‚र‚विशेषास्तेषां येऽ\textbf{व‚य‚वा}स्तेषा\textbf{म‚पि} ल‚घुवृत्तित्वे‚{\tiny $_{७}$}‚न‚{\tiny $_{lb}$}‚ \textbf{संहारादे}कीक‚र‚णात् \textbf{संकुला प्र‚तिप‚त्तिः स्यात्} । न त्व‚संसृष्ट‚ग‚म‚काव‚य‚वानुक्र‚म‚व‚ती‚{\tiny $_{lb}$}‚ ‚{\tiny $_{lb}$}‚ \leavevmode\ledsidenote{\textenglish{481/s}}स्यात् । \textbf{व‚क्ष्य‚ते चात्र प्र‚तिषेध}स्तृतीये प‚रिच्छेदे । ह्र‚स्व‚द्व‚योच्चार‚णे स्यादि‚{\tiny $_{lb}$}‚ त्यादिना \href{http://sarit.indology.info/?cref=pv.2.493}{३ । ४९३} ।
	{\color{gray}{\rmlatinfont\textsuperscript{§~\theparCount}}}
	\pend% ending standard par
      ‚{\tiny $_{lb}$}‚

	  
	  \pstart \leavevmode% starting standard par
	य‚त एव\textbf{न्त‚स्मादेक‚श‚ब्द‚ग‚तौ श‚क्तिप्र‚तिनिय‚मादिन्द्रिय‚स्यानेकात्मा । अनेक}‚{\tiny $_{lb}$}‚श‚ब्द‚स्व‚भावः \textbf{क‚ल‚क‚लो न श्रूय‚ते} । श्रूय‚ते च [।] त‚स्मान्नेन्द्रिय‚संस्कारोऽपि तु‚{\tiny $_{lb}$}‚ ताल्वादिना श‚ब्द‚क‚र‚णं । तेन याव‚{\tiny $_{१}$}‚न्तः श‚ब्दाः कृतास्ताव‚न्त एव श्रूय‚न्त इति‚{\tiny $_{lb}$}‚ क‚ल‚क‚ल‚ग्र‚ह‚णं । \textbf{ध्व‚न‚यः केव‚ल‚न्त‚त्र श्रूय‚न्ते न वाच‚काः} श‚ब्दा य‚दि ।
	{\color{gray}{\rmlatinfont\textsuperscript{§~\theparCount}}}
	\pend% ending standard par
      ‚{\tiny $_{lb}$}‚

	  
	  \pstart \leavevmode% starting standard par
	\textbf{ने}त्यादिना व्याच‚ष्टे । \textbf{न क‚ल‚क‚ले} वाच‚कानि \textbf{व‚र्ण्ण‚प‚द‚वाक्यानि श्र‚य‚न्ते} ।‚{\tiny $_{lb}$}‚ किङ्कार‚णं [।] \textbf{ध्व‚नीनां केव‚लानाम}वाच‚कानान्त‚त्र \textbf{श्र‚व‚णात्} । \href{http://sarit.indology.info/?cref=pv.3.255}{२५८}
	{\color{gray}{\rmlatinfont\textsuperscript{§~\theparCount}}}
	\pend% ending standard par
      ‚{\tiny $_{lb}$}‚

	  
	  \pstart \leavevmode% starting standard par
	एकंग‚तिश‚क्तिप्र‚तिनिय‚मे ध्व‚नीनाम‚पि क‚थं युग‚प‚च्छ्र‚व‚ण‚मिति चेदाह । \textbf{वाच‚{\tiny $_{lb}$}‚केत्}यादि । \textbf{वाच‚के च} श‚ब्दे \textbf{प्र‚{\tiny $_{२}$}‚तिनिय‚त‚श‚क्तीन्द्रिय‚म}स्माभिरुच्य‚ते । \textbf{न तु ध्व‚नि‚{\tiny $_{lb}$}‚ष्व‚वाच‚केषु} ।
	{\color{gray}{\rmlatinfont\textsuperscript{§~\theparCount}}}
	\pend% ending standard par
      ‚{\tiny $_{lb}$}‚

	  
	  \pstart \leavevmode% starting standard par
	\textbf{त‚त्रे}त्यादिना प्र‚तिविध‚त्ते । \textbf{ध्व‚न‚य} एव हि विशिष्टा व‚र्ण्ण‚रूपा वाच‚काः ।‚{\tiny $_{lb}$}‚ तेभ्यो \textbf{भिन्न}म‚र्थान्त‚र‚वाच‚कं श‚ब्द‚रूप\textbf{म‚स्ती}त्येत‚त्स‚त्ताग्राह‚क‚प्र‚माणाभावाद् अति‚{\tiny $_{lb}$}‚ब‚ह्वियं \textbf{श्र‚द्धेयं} । किं कार‚णं ।
	{\color{gray}{\rmlatinfont\textsuperscript{§~\theparCount}}}
	\pend% ending standard par
      ‚{\tiny $_{lb}$}‚

	  
	  \pstart \leavevmode% starting standard par
	य‚तो \textbf{न व‚य‚म}वाच‚कं \textbf{ध्व‚निं श‚ब्दं च वाच‚कं पृथ‚ग्रूप}मिति ध्व‚निभ्यो‚{\tiny $_{lb}$}‚ भिन्न‚स्व‚भाव\textbf{मुप‚ल‚क्ष‚यामः} । किन्त्\textbf{वेक‚दैक}‚{\tiny $_{३}$}‚स्मिन् \textbf{व‚र्ण्णानुक्र‚म‚श्र‚व‚ण}काले \textbf{एक‚मेव‚{\tiny $_{lb}$}‚ श‚ब्दात्मा}न‚म्व‚र्ण्णानुक्र‚म‚ल‚क्ष‚णं \textbf{व्य‚व‚स्यामः । त‚त्क‚थं} पुन‚र्ध्वंनिव्य‚तिरिक्तं श‚ब्दा‚{\tiny $_{lb}$}‚त्मान‚म‚ध्य‚व‚स्य‚न्तो प‚रिच्छिन्द‚न्तः । \textbf{व्य‚व‚साय‚पूर्व्व‚कं निश्च‚य‚पूर्व‚कं ध्व‚निभ्यो‚{\tiny $_{lb}$}‚ ‚{\tiny $_{lb}$}‚ \leavevmode\ledsidenote{\textenglish{482/s}}भिन्नं} श‚ब्द‚रूप‚म‚निब‚न्ध‚नं क‚थ\textbf{म्प्र‚व‚र्त्त‚यामः । त‚स्माद् घ्व‚निविशेष} एवाकारा‚{\tiny $_{lb}$}‚दिरूपेण स्थितः \textbf{व‚र्ण्णाख्यः} व‚र्ण्णादिरित्याख्या य‚स्येति विग्र‚हः । आदि‚{\tiny $_{४}$}‚ग्र‚ह‚णात्‚{\tiny $_{lb}$}‚ प‚द‚वाक्यादिप‚रिग्र‚हः ।
	{\color{gray}{\rmlatinfont\textsuperscript{§~\theparCount}}}
	\pend% ending standard par
      ‚{\tiny $_{lb}$}‚

	  
	  \pstart \leavevmode% starting standard par
	\textbf{अपि च} [।] य‚दि क‚ल‚क‚ले ध्व‚न‚यः श्रूय‚न्ते न \textbf{वाच‚का} । य‚दा त‚र्हि त‚त्र‚{\tiny $_{lb}$}‚ ब‚हूनां व्याह‚र्त्तृणान्तूष्णीम‚व‚स्थानात् । \textbf{स्थितेष्व‚न्येषु श‚ब्देष्वेकः} पुरुषो व्याह‚र‚ति‚{\tiny $_{lb}$}‚ त‚स्यैक‚स्य \textbf{श्र‚व‚णे वाच‚कः क‚थं} ।
	{\color{gray}{\rmlatinfont\textsuperscript{§~\theparCount}}}
	\pend% ending standard par
      ‚{\tiny $_{lb}$}‚

	  
	  \pstart \leavevmode% starting standard par
	अथ स्यात् [।] त‚दा ध्व‚निर‚पि प्र‚तीय‚त इत्य‚त आह । \textbf{न ध्व‚निर‚तो} वाच‚का‚{\tiny $_{lb}$}‚द्भिन्नो रूप‚न्तेन वाच‚केन स‚ह पृथ‚ग् वा श्रूय‚ते । ध्व‚निभ्यः श्रूय‚{\tiny $_{५}$}‚त एवेति चेदाह ।‚{\tiny $_{lb}$}‚ \textbf{न हि प्र‚त्य‚क्षेर्थे प‚रोप‚देशो ग‚रीयान्} । येन स्व‚य‚म्विवेकेनाश्रृण्व‚न्न‚पि त्व‚द्व‚च‚न‚मा‚{\tiny $_{lb}$}‚त्राद् ध्व‚नेः श्र‚व‚णं व्य‚तिरिक्त‚स्य प्र‚तिप‚द्य‚ते । त‚दिति त‚स्मा\textbf{द‚यं} श्रोता \textbf{स्थितेष्व‚न्येषु‚{\tiny $_{lb}$}‚ व्य‚व‚ह‚र्त्तृष्वे}क‚स्यैव व्याह‚र‚तः । \textbf{केव‚ल‚मे}वार्थान्त‚र‚ध्व‚निविविक्त‚मेव \textbf{श‚ब्दं शृण्वं‚{\tiny $_{lb}$}‚स्त‚दुप‚ल‚म्भ‚प्र‚त्य‚यानां} व्य‚व‚ह‚र्त्तृग‚तानां क‚र‚ण‚साङ्ग‚ल्यादीनां श‚ब्दोप‚ल‚म्भ‚{\tiny $_{६}$}‚प्र‚त्य‚{\tiny $_{lb}$}‚यानां \textbf{साम‚र्थ्याभावं प्र‚त्येति} । क‚स्मिन् क‚र्त्त‚व्ये । \textbf{त‚द‚न्य‚निष्पाद‚ने} श्र‚य‚माणा‚{\tiny $_{lb}$}‚च्छ‚ब्दाद‚न्य‚स्य ध्व‚नेर्निष्पाद‚ने । किं कार‚णं [।] \textbf{य‚दि} त‚दुप‚ल‚म्भ‚प्र‚त्य‚यास्त‚द‚न्य‚{\tiny $_{lb}$}‚निष्पाद‚ने \textbf{स‚म‚र्थाः स्युस्त‚दा त‚त् साधित‚न्तैः} श‚ब्दोप‚ल‚म्भ‚प्र‚त्य‚यैः साधितं ध्व‚निरूप‚{\tiny $_{lb}$}‚\textbf{मुप‚ल‚भ्येत} । न चोप‚ल‚भ्य‚ते ।
	{\color{gray}{\rmlatinfont\textsuperscript{§~\theparCount}}}
	\pend% ending standard par
      ‚{\tiny $_{lb}$}‚

	  
	  \pstart \leavevmode% starting standard par
	अथ स्यात् [।] क‚ल‚क‚ले ते ध्व‚न्यार‚म्भ‚का इत्याह । \textbf{त‚त्स्व‚भावा} इत्यादि ।‚{\tiny $_{lb}$}‚ \leavevmode\ledsidenote{\textenglish{171b/PSVTa}} ध्व‚निर‚हि‚{\tiny $_{७}$}‚त‚श‚ब्द‚ज‚न‚न‚स्व‚भावा एव पुनः श‚ब्दोप‚ल‚म्भ‚प्र‚त्य‚या व्याह‚र‚त्स्व‚पि \textbf{ब‚हुषु‚{\tiny $_{lb}$}‚ क‚ल‚क‚ले} स्व‚कार्यं श‚ब्दं मुक्त्वा \textbf{कार्यान्त‚रं} ध्व‚निं \textbf{क‚थ‚मार‚भेर‚न्} । नैवार‚भे‚{\tiny $_{lb}$}‚र‚न् । य‚स्मान्न \textbf{हि कार‚णाभेदे कार्य‚भेदो युक्तः} । त‚स्मिन्नेव कार‚णे कार्य‚भेदः‚{\tiny $_{lb}$}‚ श‚ब्द‚ध्व‚निल‚क्ष‚णो न युक्तः । किं कार‚णं [।] कार‚ण‚भेदान‚पेक्षिणः कार्य\textbf{भेद‚स्याहेतु‚{\tiny $_{lb}$}‚क‚त्व‚प्र‚स‚ङ्गादित्युक्तं प्राक्} । त‚स्मात् क‚ल‚क‚ले वाच‚का एव श्रूय‚{\tiny $_{१}$}‚न्ते न ध्व‚न‚यः ।
	{\color{gray}{\rmlatinfont\textsuperscript{§~\theparCount}}}
	\pend% ending standard par
      ‚{\tiny $_{lb}$}‚

	  
	  \pstart \leavevmode% starting standard par
	न‚नु य‚दि क‚ल‚क‚ले वाच‚का एव स‚न्तीत्य‚भ्युप‚ग‚म्य‚ते । क‚थ‚न्त‚र्हि दूर‚व‚र्त्तिनां‚{\tiny $_{lb}$}‚ ध्व‚निमात्र‚श्र‚व‚णं स‚मीप‚व‚र्त्तिनां वाच‚कानां ध्व‚नीनां श्र‚व‚ण‚मिति ।
	{\color{gray}{\rmlatinfont\textsuperscript{§~\theparCount}}}
	\pend% ending standard par
      ‚{\tiny $_{lb}$}‚‚{\tiny $_{lb}$}‚‚{\tiny $_{lb}$}‚\textsuperscript{\textenglish{483/s}}

	  
	  \pstart \leavevmode% starting standard par
	स‚त्त्यं । य एव वाच‚काः प्र‚य‚त्न‚निष्प‚न्नास्त एव प‚र‚स्प‚र‚संह‚र्षेण ध्व‚न्यार‚म्भ‚{\tiny $_{lb}$}‚काः [।] तेन क‚ल‚क‚ले केषांचिद् ध्व‚निमात्र‚स्य प्र‚तीतिर‚न्येषामुभ‚य‚प्र‚तीतिरित्य‚दोषः ।‚{\tiny $_{lb}$}‚ \href{http://sarit.indology.info/?cref=pv.3.256}{२५९}
	{\color{gray}{\rmlatinfont\textsuperscript{§~\theparCount}}}
	\pend% ending standard par
      ‚{\tiny $_{lb}$}‚

	  
	  \pstart \leavevmode% starting standard par
	य‚द‚प्युक्तं स‚मीप‚व‚र्तिनापि क‚ल‚क‚ले‚{\tiny $_{२}$}‚ ध्व‚न‚य एव केव‚लं श्रूय‚न्ते न वाच‚काः‚{\tiny $_{lb}$}‚ श‚ब्दा इति ।
	{\color{gray}{\rmlatinfont\textsuperscript{§~\theparCount}}}
	\pend% ending standard par
      ‚{\tiny $_{lb}$}‚

	  
	  \pstart \leavevmode% starting standard par
	त‚द‚प्य‚युक्तं । य‚स्मान्न \textbf{च क‚ल‚क‚ले वाच‚को न श्रूय‚ते} । किन्तु श्रूय‚त एव ।‚{\tiny $_{lb}$}‚ किं कार‚णं । \textbf{प‚द‚वाक्य‚विच्छेदानामुप‚ल‚क्ष‚णात्} । अपि च \textbf{क‚थं} चेन्द्रिय‚स्यैक‚श‚क्ति‚{\tiny $_{lb}$}‚प्र‚ति\textbf{निय‚माद् भिन्न‚ध्व‚निग‚तिर्भ‚वे}त् । ब‚हूनां ध्व‚नीनां ग्र‚ह‚ण‚म्भ‚वेत् । नैव भ‚वेत् ।
	{\color{gray}{\rmlatinfont\textsuperscript{§~\theparCount}}}
	\pend% ending standard par
      ‚{\tiny $_{lb}$}‚

	  
	  \pstart \leavevmode% starting standard par
	\textbf{तानी}त्यादिना व्याच‚ष्टे । \textbf{तानि प्र‚तिनिय‚त‚श‚क्तीन्य‚पीन्द्रियाणि} युग‚प‚{\tiny $_{३}$}‚\textbf{न्ना‚{\tiny $_{lb}$}‚नारूपान्} ध्व‚नीन् श्रृण्व‚न्ति । कीदृशान् [।] \textbf{प्र‚तिश‚ब्द‚निय‚तान्} । श‚ब्दं श‚ब्दं प्र‚ति‚{\tiny $_{lb}$}‚ व्य‚ञ्ज‚क‚त्वेन निय‚तान् । \textbf{न त्वे}व श‚ब्दान् युग‚प\textbf{च्छ्र‚ण्व‚न्तीति} कः \textbf{श‚ब्देष्वेषा}मिन्द्रि‚{\tiny $_{lb}$}‚याणां \textbf{निर्वेदो} वैमुख्यं येन तान् न श्रृण्व‚न्ति । न च भाव‚श‚क्तिरीदृशीति श‚क्य‚{\tiny $_{lb}$}‚म्व‚क्तुं [।] \textbf{क‚दाचिद्} ब‚हूनाम‚पि \textbf{वाच‚कानां श्र‚व‚णात्} \href{http://sarit.indology.info/?cref=pv.3.257}{२६०}
	{\color{gray}{\rmlatinfont\textsuperscript{§~\theparCount}}}
	\pend% ending standard par
      ‚{\tiny $_{lb}$}‚

	  
	  \pstart \leavevmode% starting standard par
	\textbf{य‚दुक्त}मित्यादि प‚रः । य‚दुक्त‚म्बौ द्धे न \textbf{वाच‚केभ्यः} व‚र्ण्ण‚प‚द‚वाक्येभ्यो \textbf{भेदेन}‚{\tiny $_{lb}$}‚ ध्व‚{\tiny $_{४}$}‚न‚यो न सिद्धा इति । क‚थ‚न्न सिद्धाः [।] सिद्धा एव । किं कार‚णं । व‚च‚नाद‚र्थ‚{\tiny $_{lb}$}‚प्र‚तीतेः । श‚ब्दादुच्च‚रिताद‚र्थ‚स्य वाच्य‚स्य ग‚तेः । \textbf{न चेय‚म‚र्थ‚ग‚तिर्ध्व‚निभ्यः स‚म्भ‚{\tiny $_{lb}$}‚व‚ति} । किं कार‚णं [।] \textbf{न हि ध्व‚निभागाद}ल्पीय‚सो व‚र्ण्ण‚व्य‚ञ्ज‚काद‚र्थ\textbf{प्र‚तीतिः} ।‚{\tiny $_{lb}$}‚ व‚र्ण्णोप्येक‚स्ताव‚त् प्रायेणान‚र्थ‚कः [।] प्रागेव व्य‚ञ्ज‚कोल्पीयान् ध्व‚निभागः ।‚{\tiny $_{lb}$}‚ स‚हिता प्र‚तिपाद‚का इति चेदाह । \textbf{न च सोन्यं स‚{\tiny $_{५}$}‚मेति} [।] सोल्पीयान् ध्व‚निभागः‚{\tiny $_{lb}$}‚ क्ष‚णिक‚त्वाद‚न्य‚मुत्त‚र‚काल‚भाविनं ध्व‚निभागं स‚मेति संश्लिष्य‚ति । \textbf{त‚दि}ति त‚स्मा‚{\tiny $_{lb}$}‚दिय‚म‚र्थ‚प्र‚तीतिः स‚म‚स्तानि प‚रिपूर्ण्णानि प‚द‚वाक्य‚रूपाणि य‚स्मिन् वाच‚के त‚त्त‚था ।‚{\tiny $_{lb}$}‚ \textbf{तेन साध्या} ध्व‚निषु न स‚म्भ‚व‚ति । कीदृशेषु । \textbf{अस‚म‚स्ता} असंश्लिष्टा भागा उत्प‚{\tiny $_{lb}$}‚न्नोत्प‚न्न‚ध्व‚निभाग‚स्य क्ष‚णिक‚त्वेन द्वितीय\textbf{ध्व‚निभागान‚व‚स्थानाद्} येषान्ते‚{\tiny $_{६}$}‚षु । इति‚{\tiny $_{lb}$}‚ एव‚म‚र्थ\textbf{प्र‚तिप‚त्त्य}न्य‚थानुप‚प‚त्त्या सिद्ध‚म\textbf{क्र‚म‚स‚त्त्वं} । अक्र‚मं स‚त्त्वं य‚स्य \textbf{श‚ब्द‚रूप‚स्य}‚{\tiny $_{lb}$}‚ त‚त्त‚था । निर्विभाग‚मिति याव‚त् । \textbf{क्र‚म‚व‚द्} विभाग‚श्च वाच‚क‚व्य‚तिरिक्तो ध्व‚निः‚{\tiny $_{lb}$}‚ ‚{\tiny $_{lb}$}‚ \leavevmode\ledsidenote{\textenglish{484/s}}[।] क्र‚म‚व‚न्तो भागा य‚स्येति विग्र‚हः ।
	{\color{gray}{\rmlatinfont\textsuperscript{§~\theparCount}}}
	\pend% ending standard par
      ‚{\tiny $_{lb}$}‚

	  
	  \pstart \leavevmode% starting standard par
	\textbf{त‚न्ने}त्यादिना प्र‚तिषेध‚ति । त‚देत‚द‚न‚न्त‚रोक्तं न स‚म्भ‚व‚ति । क‚स्मात् । \textbf{क्र‚म}‚{\tiny $_{lb}$}‚\leavevmode\ledsidenote{\textenglish{172a/PSVTa}} व‚न्तो ये व‚र्ण्णास्त‚द्व्\textbf{य‚तिरेकिणा क्र‚म‚स्य} श‚ब्द‚स्य । न हि व‚य‚न्देव‚द‚त्ता‚{\tiny $_{७}$}‚दि प‚द‚वा‚{\tiny $_{lb}$}‚क्येषु द‚कारादिप्र‚तिभासं मुक्त्वा प‚रं प्र‚तिभास‚मुप‚ल‚क्ष‚याम इत्यादिना \textbf{प्रागेव‚{\tiny $_{lb}$}‚ निषिद्ध‚त्वात्} ।
	{\color{gray}{\rmlatinfont\textsuperscript{§~\theparCount}}}
	\pend% ending standard par
      ‚{\tiny $_{lb}$}‚

	  
	  \pstart \leavevmode% starting standard par
	य‚दि चास‚म‚स्त‚भागेषु ध्व‚निष्व‚र्थ‚प्र‚तीतेर‚स‚म्भ‚वाद‚क्र‚म‚स‚त्त्वं श‚ब्द‚रूपं क‚ल्प्य‚ते ।‚{\tiny $_{lb}$}‚ त‚दातिप्र‚स‚ङ्ग‚श्चैवं क‚ल्प्य‚माने । \textbf{त‚था} हि ह‚स्तादीनां य‚था संकेत‚ग‚म‚नाग‚म‚नादि‚{\tiny $_{lb}$}‚सूच‚कानि यानि क‚र्माणि तेषां ये भागास्तेषां क्ष‚णिक‚त्वात् \textbf{पूर्वेण क‚र्म‚भागेनाप‚र‚{\tiny $_{lb}$}‚स्यो}त्त‚र‚{\tiny $_{१}$}‚स्य क‚र्म‚भाग‚स्याप्र‚तिस‚न्धानाद‚घ‚ट‚नात् । \textbf{एकांशाच्चाप्र‚तीतेः} । एक‚स्मा‚{\tiny $_{lb}$}‚च्चाल्पीय‚सः क‚र्म‚भागाद् य‚था संकेत‚स्य ग‚म‚नाग‚म‚नादिल‚क्ष‚ण‚स्यार्थ‚स्याप्र‚तिप‚त्तेः ।‚{\tiny $_{lb}$}‚ \textbf{त‚द्व्य‚तिरेकी} । क‚र्म‚भागेभ्योन्यः । य‚थासंकेतं \textbf{ह‚स्त‚संज्ञाद‚यः} । आदिश‚ब्दाद‚र्थ‚प्र‚तीतौ‚{\tiny $_{lb}$}‚ शिरःक‚म्पाद‚यो गृह्य‚न्ते । तेष्व‚र्थ‚प्र‚तीतिहेतुः \textbf{स‚म‚स्त‚रूप‚क‚र्मात्माभ्युप‚ग‚न्त‚व्यः स्यात् ।‚{\tiny $_{lb}$}‚ श‚ब्द‚{\tiny $_{२}$}‚व‚देव} । ध्व‚निव्य‚तिरिक्त‚श‚ब्द‚क‚ल्प‚नाव‚त् ।
	{\color{gray}{\rmlatinfont\textsuperscript{§~\theparCount}}}
	\pend% ending standard par
      ‚{\tiny $_{lb}$}‚

	  
	  \pstart \leavevmode% starting standard par
	\hphantom{.}य‚त्पुन‚रुक्त‚म्म ण्ड ने न । य‚दा त्रैविद्य‚वृद्धा ह‚स्त‚संज्ञादिविष‚यानुत्क्षेप‚ण‚त्वादि‚{\tiny $_{lb}$}‚श‚ब्द‚निर्देश्यान् सामान्य‚विशेषान‚भ्युप‚ग‚च्छ‚न्ति त‚दा कोयं प्र‚स‚ङ्गः । एकः क‚र्मात्मा‚{\tiny $_{lb}$}‚भ्युप‚ग‚न्त‚व्य\edtext{\textsuperscript{*}}{\edlabel{pvsvt_484-1}\label{pvsvt_484-1}\lemma{*}\Bfootnote{\href{http://sarit.indology.info/?cref=spho\%E1\%B9\%ADasiddhi.33}{Sphoṭasiddhi 33 pp. 253-54}.}} इति ।
	{\color{gray}{\rmlatinfont\textsuperscript{§~\theparCount}}}
	\pend% ending standard par
      ‚{\tiny $_{lb}$}‚

	  
	  \pstart \leavevmode% starting standard par
	त‚द‚युक्तं । य‚तो य‚द्येक‚मुत्क्षेप‚ण‚रूप‚ङ्क‚र्म सिद्ध‚म्भ‚वेत् । त‚था प‚राप‚र‚म‚पि‚{\tiny $_{lb}$}‚य‚दि सिद्धं स्यात् त‚दा तेषु ब‚हुषूत्क्षेप‚{\tiny $_{३}$}‚णेषु प्र‚त्येक‚मुत्क्षेप‚ण‚त्व‚सामान्य‚म्व‚र्त्तेत ।‚{\tiny $_{lb}$}‚ त‚देव तु न सिद्धं पूर्वाप‚र‚क‚र्म‚भागानाम‚न‚न्व‚यात् । न च विशेषाभावे सामान्य‚स‚द्‚{\tiny $_{lb}$}‚भावः । नापि क‚र्म‚भागेषु प्र‚त्येक‚मुत्क्षेप‚णादिरूप‚त‚या प्र‚तीतिः [।] किन्त‚र्हि [।]‚{\tiny $_{lb}$}‚ त‚द्भाग‚रूप‚त‚या [।] त‚त्क‚थ‚न्तेषु भागेषूत्क्षेप‚ण‚त्व‚सामान्य‚म‚भ्युप‚ग‚म्येत [।]‚{\tiny $_{lb}$}‚ अभ्युप‚ग‚मे वा एक‚स्माद‚पि क‚र्म‚भागाद् ग‚म‚नादिल‚क्ष‚ण‚स्यार्थ‚स्य‚{\tiny $_{४}$}‚ प्र‚तिप‚त्तिः स्या‚{\tiny $_{lb}$}‚द‚र्थाभिधाय‚क‚स्य सामान्य‚स्य भावादितिं य‚त्किञ्चिदेत‚त् । य‚था च न क‚र्म‚भागेषु‚{\tiny $_{lb}$}‚ व्य‚तिरिक्तं क‚र्मात्मा त‚था ध्व‚निभागेष्व‚पि न व्य‚तिरिक्तः श‚ब्दात्मा । क‚थ‚न्त‚र्ह्य‚र्थ‚{\tiny $_{lb}$}‚‚{\tiny $_{lb}$}‚ ‚{\tiny $_{lb}$}‚ \leavevmode\ledsidenote{\textenglish{485/s}}प्र‚तीतिरित्याह । \textbf{क्र‚म‚भाविन एवे}त्यादि । \textbf{य‚थास्वं} य‚स्य य‚त्क‚र‚ण‚न्ताल्वादि ।‚{\tiny $_{lb}$}‚ त‚स्य \textbf{प्र‚यो}गो व्यापार‚स्त‚स्माद् \textbf{भिन्ना व‚र्ण्ण‚भागाः । क‚र्म‚भागा वा य‚थास्वं क‚र‚ण‚{\tiny $_{lb}$}‚प्र‚योगात्} । क‚र्म‚{\tiny $_{५}$}‚हेतोः प्र‚योगात् \textbf{क्र‚म‚भाविनो} भिन्ना इत्य‚त्रापि स‚म्ब‚न्ध‚नीयं । ते‚{\tiny $_{lb}$}‚ य‚थोक्ता व‚र्ण्ण‚भागाः क‚र्म‚भागा वा \textbf{क्र‚मेण विक‚ल्प‚विष‚याद}त्य‚नुभ‚व‚ज्ञानानुक्र‚मानु‚{\tiny $_{lb}$}‚सारिणां विक‚ल्पानां क्र‚मेण विष‚य‚मुप‚ग‚ता \textbf{य‚थासंकेत‚मेवार्थ‚प्र‚तीतिं ज‚न‚य‚न्तीति‚{\tiny $_{lb}$}‚ न्याय्यं} । युक्त्य‚पेत‚त्वात् ।
	{\color{gray}{\rmlatinfont\textsuperscript{§~\theparCount}}}
	\pend% ending standard par
      ‚{\tiny $_{lb}$}‚

	  
	  \pstart \leavevmode% starting standard par
	\textbf{किं चे}ति दोषान्त‚र‚म‚प्याह । यैः कैर‚पि दोषैः पूर्व‚पूर्व‚स्य ध्व‚निभाग‚स्योत्त‚रो-‚{\tiny $_{६}$}‚‚{\tiny $_{lb}$}‚ त्त‚रेण ध्व‚निभागेनाप्र‚तिस‚न्धानादित्यादिकैः क‚र‚ण‚भूतैस्ते \textbf{ध्व‚न‚यो} वै या क र णा‚{\tiny $_{lb}$}‚ दीनाम‚वाच‚कास्\textbf{स‚म्म‚ताः} । दृष्टाः [।] तैः क्र‚म‚भाविभि\textbf{र्ध्व‚निभिर्व्य‚ज्य‚मानेस्मिन्}‚{\tiny $_{lb}$}‚ध्व‚निव्य‚तिरिक्तेपि \textbf{वाच‚के क‚थ‚न्न ते} । ध्व‚निभाविनो दोषा न स‚न्ति भ‚व‚न्त्येव ।
	{\color{gray}{\rmlatinfont\textsuperscript{§~\theparCount}}}
	\pend% ending standard par
      ‚{\tiny $_{lb}$}‚

	  
	  \pstart \leavevmode% starting standard par
	न‚नु ध्व‚न‚यः प्र‚त्येकं स‚मुदिता वा पूर्वोक्तेन न्यायेन नार्थ‚स्य प्र‚तिपाद‚काः ।‚{\tiny $_{lb}$}‚ वाच‚क‚स्य तु ते प्र‚त्येक‚म‚भि‚{\tiny $_{७}$}‚व्य‚ञ्ज‚का इष्य‚न्ते । एकेन ध्व‚निनाभिव्य‚क्त‚स्य वांच- \leavevmode\ledsidenote{\textenglish{172b/PSVTa}}‚{\tiny $_{lb}$}‚ क‚स्यान‚व‚धृत‚त्वाद‚न्यान्यैर‚भिव्य‚क्त‚स्य संस्काराधान‚तार‚त‚म्य‚प्र‚बोधेनाव‚धार‚ण‚मिति‚{\tiny $_{lb}$}‚ ध्व‚निभिर्व्य‚ज्य‚माने वाच‚केपि कुत‚स्ते दोषा इति । त‚दुक्त‚म्म ण्ड ने न ।
	{\color{gray}{\rmlatinfont\textsuperscript{§~\theparCount}}}
	\pend% ending standard par
      ‚{\tiny $_{lb}$}‚
	  \bigskip
	  \begingroup
	
	    
	    \stanza[\smallbreak]
	  {\normalfontlatin\large ``\qquad}नानेकाव‚य‚वं वाक्यं प‚दं वा स्फोट‚वादिनां ।&‚{\tiny $_{lb}$}‚एक‚त्वेपि ह्य‚भिन्न‚स्य क्र‚म‚शो द‚र्शिता ग‚तिरिति ।{\normalfontlatin\large\qquad{}"}\&[\smallbreak]
	  
	  
	  
	  \endgroup
	‚{\tiny $_{lb}$}‚

	  
	  \pstart \leavevmode% starting standard par
	त‚द‚युक्त‚म् [।] अभिव्य‚क्तिर्हि ज्ञानं [।] न च श‚ब्दानुग‚मेन विना‚{\tiny $_{१}$}‚ ज्ञान‚{\tiny $_{lb}$}‚मिष्य‚ते भ‚व‚द्भिः । तेनाभिव्य‚क्तिरिति निश्च‚य एवोच्य‚ते । न च प्र‚थ‚म‚ध्व‚न्य‚{\tiny $_{lb}$}‚न‚न्त‚र‚म्वाच‚क‚निश्च‚यः । प्र‚तिभास‚त इति [।] त‚त्क‚थ‚म‚स्याभिव्य‚क्तिः ।
	{\color{gray}{\rmlatinfont\textsuperscript{§~\theparCount}}}
	\pend% ending standard par
      ‚{\tiny $_{lb}$}‚

	  
	  \pstart \leavevmode% starting standard par
	त‚स्मात् स्थित‚मेत‚द् य‚था ध्व‚न‚यः प्र‚त्येकं स‚म‚स्ता वार्थ‚प्र‚तिपाद‚नेऽश‚क्तास्त‚था‚{\tiny $_{lb}$}‚ वाच‚काभिव्य‚क्ताविति ।
	{\color{gray}{\rmlatinfont\textsuperscript{§~\theparCount}}}
	\pend% ending standard par
      ‚{\tiny $_{lb}$}‚

	  
	  \pstart \leavevmode% starting standard par
	क्र‚मेत्यादि विव‚र‚णं [।] \textbf{क्र‚मोत्पादिभिर्ध्व‚निभागैर्व्य‚क्तः प्र‚काशितः किला}‚{\tiny $_{lb}$}‚क्र‚मः श‚ब्दात्मा \textbf{वाच‚को‚{\tiny $_{२}$}‚ र्थ‚म्व‚क्ति} । न स‚न्निधान‚मात्रेण \href{http://sarit.indology.info/?cref=pv.3.258}{२६१}
	{\color{gray}{\rmlatinfont\textsuperscript{§~\theparCount}}}
	\pend% ending standard par
      ‚{\tiny $_{lb}$}‚

	  
	  \pstart \leavevmode% starting standard par
	\textbf{त‚म‚पि} ध्व‚निव्य‚तिरिक्तं श‚ब्दात्मानं \textbf{ते} ध्व‚न‚यो \textbf{न स‚कृत् प्र‚काश‚य‚न्ति} । किं‚{\tiny $_{lb}$}‚ कार‚णं [।] तेषां ध्व‚निभागानां \textbf{क्र‚म‚भावात्} । ना\textbf{प्येक एव} ध्व‚निभागः श‚ब्दं \textbf{व्य‚{\tiny $_{lb}$}‚\edtext{}{\lemma{व्य}\Bfootnote{Sphoṭasiddhi (by Maṇḍana) 29.}}‚{\tiny $_{lb}$}‚ ‚{\tiny $_{lb}$}‚ \leavevmode\ledsidenote{\textenglish{486/s}}न‚क्ति} । निश्चाय‚य‚ति । किं कार‚णं [।] \textbf{त‚द‚न्य‚स्य} व्य‚ञ्ज‚क‚त्वेनाभिम‚त‚स्य ध्व‚नि‚{\tiny $_{lb}$}‚भाग‚स्य \textbf{वैय‚र्थ्य‚प्र‚स‚ङ्गात्} । इत‚श्चैको ध्व‚निभागो न स‚म‚स्त‚स्य श‚ब्द‚स्य व्य‚ञ्ज‚को‚{\tiny $_{lb}$}‚ य‚स्मा\textbf{देक‚व‚र्ण्ण‚{\tiny $_{३}$}‚भाग‚काले च स‚म‚स्त‚स्य} वाच‚क‚रूप‚स्या\textbf{नुप‚ल‚क्ष‚णात्} ।
	{\color{gray}{\rmlatinfont\textsuperscript{§~\theparCount}}}
	\pend% ending standard par
      ‚{\tiny $_{lb}$}‚

	  
	  \pstart \leavevmode% starting standard par
	\textbf{त‚दि}ति त‚स्माद‚यं श‚ब्दात्मा । ध्व‚निभागैः क्र‚म‚भाविभिः क्र‚मेण व्य‚ज्य‚मान‚{\tiny $_{lb}$}‚त्वात् । \textbf{अप्र‚तिसंहितो} न संघ‚टित\textbf{स्स‚क‚लोप‚ल‚म्भो} य‚स्य श‚ब्दात्म‚नः स एव‚म्भूतः‚{\tiny $_{lb}$}‚ श‚ब्दात्मा । \textbf{उप‚ल‚म्भ‚साक‚ल्य‚साध्य‚म‚र्थ} स्वाभिधेयं प्र‚काश‚न‚ल‚क्ष‚णं \textbf{क‚थं साध‚येत्}‚{\tiny $_{lb}$}‚ [।] नैव साध‚येत् । किमिव [।] \textbf{ध्व‚निव‚त्} । य‚था ध्व‚नि‚{\tiny $_{४}$}‚भागास्त्व‚न्म‚तेन पूर्व्वा‚{\tiny $_{lb}$}‚प‚रेणाप्र‚तिस‚न्धानाद‚र्थ‚न्न प्र‚काश‚येयुस्त‚द्व‚त् ।
	{\color{gray}{\rmlatinfont\textsuperscript{§~\theparCount}}}
	\pend% ending standard par
      ‚{\tiny $_{lb}$}‚

	  
	  \pstart \leavevmode% starting standard par
	\textbf{को ही}त्यादिनैत‚देव स‚म‚र्थ‚य‚ते । \textbf{उप‚ल‚म्भ‚साध्येष्व‚र्थेषु को हि संद‚स‚तोर‚त्य‚न्ता‚{\tiny $_{lb}$}‚नुप‚ल‚म्भे} स‚ति \textbf{विशेषो} नैव क‚श्चित् । य‚था हि क्ष‚णिका ध्व‚निभागा उत्त‚रोत्त‚र‚भा‚{\tiny $_{lb}$}‚गाव‚स्थायाम‚स‚त्त्वाद‚स‚म‚स्तोप‚ल‚म्भ‚नान्न स‚म‚र्थास्त‚थैवाक्र‚मोपि श‚ब्दात्मा स‚न्न‚प्य‚स्वी‚{\tiny $_{lb}$}‚कृत‚स‚{\tiny $_{५}$}‚म‚स्तोप‚ल‚म्भ‚नो न स‚म‚र्थ एवेति । \textbf{न चा}यं श‚ब्दात्मा । उप‚ल‚म्भ‚निर‚पेक्षः‚{\tiny $_{lb}$}‚ \textbf{स‚न्निधिमात्रेणा}र्थ‚प्र‚तीति\textbf{साध‚नः} । किं कार‚ण [।] त‚स्या \textbf{व्य‚पेक्ष‚णात्} । सा चेयं‚{\tiny $_{lb}$}‚ व्य‚क्तिः क्र‚म‚भाविनी स‚द‚स‚तोः [।] \textbf{स‚तः} श‚ब्दात्म‚नः । \textbf{अस‚त}श्च ध्व‚निभाग‚स्य \textbf{क्र‚मेण‚{\tiny $_{lb}$}‚ भ‚व‚न्ती तुल्योप‚योगा} । तुल्य‚फ‚लेति कृत्वा \textbf{ध्व‚निभिर‚श‚क्य‚साध‚नं} साध‚यितुम‚श‚क्यं‚{\tiny $_{lb}$}‚ य‚त् कार्य‚म‚र्थ‚{\tiny $_{६}$}‚प्र‚तीतिल‚क्ष‚णं । त\textbf{त्रापि त‚था} । ध्व‚निभिर्व्य‚ज्य‚मानेपि श‚ब्दात्म‚नि‚{\tiny $_{lb}$}‚ त‚था । अश‚क्य‚साध‚न‚मेवे\textbf{त्य‚ल‚म‚न्येन} श‚ब्देन ध्व‚निव्य‚तिरिक्तेन क‚ल्पितेन ।
	{\color{gray}{\rmlatinfont\textsuperscript{§~\theparCount}}}
	\pend% ending standard par
      ‚{\tiny $_{lb}$}‚

	  
	  \pstart \leavevmode% starting standard par
	\textbf{त‚स्मान्न व‚र्ण्णेष्व‚पौरुषेय‚ता} । नापि वाक्य इत्युप‚संहारः ।
	{\color{gray}{\rmlatinfont\textsuperscript{§~\theparCount}}}
	\pend% ending standard par
      ‚{\tiny $_{lb}$}‚

	  
	  \pstart \leavevmode% starting standard par
	न व‚र्ण्ण‚व्य‚तिरिक्त‚म्वाक्यं किन्तु \textbf{व‚र्ण्णानुपूर्वी वाक्यं} [।] त‚च्चापौरुषेय‚{\tiny $_{lb}$}‚मिति \textbf{चेत्} ।
	{\color{gray}{\rmlatinfont\textsuperscript{§~\theparCount}}}
	\pend% ending standard par
      ‚{\tiny $_{lb}$}‚‚{\tiny $_{lb}$}‚\textsuperscript{\textenglish{487/s}}

	  
	  \pstart \leavevmode% starting standard par
	\textbf{त‚न्न} । किं कार‚णं [।] \textbf{व‚र्ण्णानामा}नुपूर्व्याः स‚काशा\textbf{द‚भेद‚तः‚{\tiny $_{७}$}‚} [।]
	{\color{gray}{\rmlatinfont\textsuperscript{§~\theparCount}}}
	\pend% ending standard par
      \textsuperscript{\textenglish{173a/PSVTa}}‚{\tiny $_{lb}$}‚

	  
	  \pstart \leavevmode% starting standard par
	\textbf{न} व‚र्ण्णेभ्यो\textbf{र्थान्त‚र‚मेव श‚ब्द‚रूप‚म्वाक्य‚म‚पौरुषेयं} । किन्त‚र्हि [।] \textbf{व‚र्ण्णानुक्र‚म‚{\tiny $_{lb}$}‚ल‚क्ष‚णं हि नो}स्माकं मी मां स का नाम्\textbf{वाक्यं । त‚द‚पौरुषेयं साध्य‚मिति चेत्} ।
	{\color{gray}{\rmlatinfont\textsuperscript{§~\theparCount}}}
	\pend% ending standard par
      ‚{\tiny $_{lb}$}‚

	  
	  \pstart \leavevmode% starting standard par
	\textbf{न । व‚र्ण्णानामानुपूर्व्याः} स‚काशा\textbf{द‚भेदात् । नेय‚मा}नुपूर्वी \textbf{अर्थान्त‚र‚म्व‚र्ण्णेभ्यः} ।‚{\tiny $_{lb}$}‚ किं कार‚णं । \textbf{दृश्या}याम‚नुप‚ल‚ब्धिल‚क्ष‚ण‚प्राप्तायान्त‚स्यामानुपूर्व्याम‚ङ्गीक्रिय‚माणायां‚{\tiny $_{lb}$}‚ व‚र्ण्णेभ्यो विभागेन \textbf{भेदो नोप‚ल‚म्भ‚प्र‚स‚ङ्गात्} ।‚{\tiny $_{१}$}‚ न चोप‚लंभ्य‚त इत्य‚भाव‚सिद्धौ स्व‚भा‚{\tiny $_{lb}$}‚वानुप‚ल‚ब्धिर्वाच्या । \textbf{अथादृश्या}नुपूर्वी [।] त‚दाप्य‚दृश्यायामानुपूर्व्यां । \textbf{त‚त} अनुपूर्व्या‚{\tiny $_{lb}$}‚ अर्था\textbf{प्र‚तिप‚त्तिप्र‚स‚ङ्गात्} [।] न च दृश्याया आनुपूर्व्या ग्राह‚कं प्र‚त्य‚क्ष‚म‚दृश्य‚त्वा‚{\tiny $_{lb}$}‚देव । नाप्य‚नुमानं लिङ्गाभावात् । व‚र्ण्णेभ्यो \textbf{भेद‚व‚त्याश्चानुपूर्व्या अभावे व‚र्ण्ण‚{\tiny $_{lb}$}‚मात्र‚म‚विशिष्टं} स‚र्व‚त्र लौकिक‚वैदिक‚वाक्ये\textbf{ष्विति पूर्व‚व‚त् प्र‚स‚ङ्गः‚{\tiny $_{२}$}‚} यः किम‚नेन‚{\tiny $_{lb}$}‚ प‚रिशोषितं स्यादित्यादिनोक्तः ।
	{\color{gray}{\rmlatinfont\textsuperscript{§~\theparCount}}}
	\pend% ending standard par
      ‚{\tiny $_{lb}$}‚

	  
	  \pstart \leavevmode% starting standard par
	अथ स्याद् [।] क्र‚मो व‚र्ण्णानां ध‚र्म‚मात्र‚न्न व‚स्त्व‚न्त‚रं तेनादोषः । त‚दुक्तं [।]
	{\color{gray}{\rmlatinfont\textsuperscript{§~\theparCount}}}
	\pend% ending standard par
      ‚{\tiny $_{lb}$}‚
	  \bigskip
	  \begingroup
	
	    
	    \stanza[\smallbreak]
	  {\normalfontlatin\large ``\qquad}ध‚र्म‚मात्र‚म‚सौ तेषान्न व‚स्त्व‚न्त‚र‚मिष्य‚ते ।&‚{\tiny $_{lb}$}‚क्र‚मेण ज्ञाय‚मानाः स्युर्व‚र्ण्णास्तेनाव‚बोध‚काः ।&‚{\tiny $_{lb}$}‚न च क्र‚म‚स्य कार्य‚त्वं पूर्व‚सिद्ध‚प‚रिग्र‚हात् ।&‚{\tiny $_{lb}$}‚व‚क्ता न हि क्र‚मं क‚श्चित् स्वात‚न्त्र्येण प्र‚प‚द्य‚ते ।&‚{\tiny $_{lb}$}‚य‚थैवास्य प‚रैरुक्त‚स्त‚थैवैनं विव‚क्ष‚ति ।&‚{\tiny $_{lb}$}‚प‚रोप्ये‚{\tiny $_{३}$}‚वं स‚त‚श्चास्य स‚म्ब‚न्ध‚व‚द‚नादिता ।\edtext{\textsuperscript{*}}{\edlabel{pvsvt_487-1}\label{pvsvt_487-1}\lemma{*}\Bfootnote{Kumārila. }}{\normalfontlatin\large\qquad{}"}\&[\smallbreak]
	  
	  
	  
	  \endgroup
	‚{\tiny $_{lb}$}‚

	  
	  \pstart \leavevmode% starting standard par
	तेन पूर्व‚पूर्व‚वृद्ध‚द‚र्श‚नायातोनादिव‚र्ण्ण‚क्र‚मो पौरुषेय एवेत्य‚त्राह ।
	{\color{gray}{\rmlatinfont\textsuperscript{§~\theparCount}}}
	\pend% ending standard par
      ‚{\tiny $_{lb}$}‚

	  
	  \pstart \leavevmode% starting standard par
	\textbf{तेषां च न व्य‚व‚स्थानं} [।] तेषां व‚र्ण्णानां न व्य‚व‚स्थित‚क्र‚म‚त्वं । किं कार‚णं‚{\tiny $_{lb}$}‚ [।] व्य‚व‚स्थितादेक‚स्मात् \textbf{क्र‚मान्त‚र‚स्य विरोध‚तः} ।
	{\color{gray}{\rmlatinfont\textsuperscript{§~\theparCount}}}
	\pend% ending standard par
      ‚{\tiny $_{lb}$}‚‚{\tiny $_{lb}$}‚‚{\tiny $_{lb}$}‚\textsuperscript{\textenglish{488/s}}

	  
	  \pstart \leavevmode% starting standard par
	\textbf{य‚दी}त्यादिना व्याच‚ष्टे । \textbf{व‚र्ण्णानामानुपूर्वी} य‚दि \textbf{कृत‚का ते} च व‚र्ण्णा न ब‚ह‚व‚{\tiny $_{lb}$}‚स्स‚मान‚जातीया येन केन‚{\tiny $_{४}$}‚चिद् व‚र्ण्णा \textbf{व्य‚व‚स्थित‚क्र‚माः} स्युर्वैदिकाः । अन्ये पुन‚{\tiny $_{lb}$}‚र्ल्लौकिका \textbf{य‚थेष्ट‚प‚रावृत्त‚यः} । य‚थेष्टं प‚रावृत्तिः क्र‚मान्त‚रेण प्र‚योगो येषामिति‚{\tiny $_{lb}$}‚ विग्र‚हः । \textbf{किन्त‚र्हि त्रैलोक्य एक एवाकार‚स्त‚था ग‚कारः} । त‚दुक्तं ।
	{\color{gray}{\rmlatinfont\textsuperscript{§~\theparCount}}}
	\pend% ending standard par
      ‚{\tiny $_{lb}$}‚
	  \bigskip
	  \begingroup
	
	    
	    \stanza[\smallbreak]
	  {\normalfontlatin\large ``\qquad}देश‚काल‚प्र‚योक्तॄणाम्भेदेपि च न भेद‚वान् ।&‚{\tiny $_{lb}$}‚गादिव‚र्ण्णो य‚त‚स्त‚त्र प्र‚त्य‚भिज्ञा प‚रिस्फुटेति ।\edtext{\textsuperscript{*}}{\edlabel{pvsvt_488-1}\label{pvsvt_488-1}\lemma{*}\Bfootnote{Kumārila. }}{\normalfontlatin\large\qquad{}"}\&[\smallbreak]
	  
	  
	  
	  \endgroup
	‚{\tiny $_{lb}$}‚

	  
	  \pstart \leavevmode% starting standard par
	य‚दा चैव‚न्त‚दा व्य‚व‚स्थित‚क्र‚म‚त्वे व‚र्ण्णानाम‚{\tiny $_{५}$}‚\textbf{ग्निरित्येव स्यान्न ग‚ग‚न‚मिति} ।‚{\tiny $_{lb}$}‚ किंकार‚ण‚म् [।] \textbf{अकार‚ग‚कार‚योः पूर्वाप‚र‚भाव‚स्य व्य‚व‚स्थित‚त्वात्} । अकारो‚{\tiny $_{lb}$}‚ ग‚कारात् पूर्व‚मेवाकाराच्च ग‚कारः प‚रेणैव व्य‚व‚स्थित इत्य‚र्थः । ग‚ग‚न‚मित्य‚त्र‚{\tiny $_{lb}$}‚ ग‚कारात्प‚रेणाकारः स्यादिति क्र‚मान्त‚र‚न्न स्यात् ।
	{\color{gray}{\rmlatinfont\textsuperscript{§~\theparCount}}}
	\pend% ending standard par
      ‚{\tiny $_{lb}$}‚

	  
	  \pstart \leavevmode% starting standard par
	एत‚देव द्र‚ढ‚य‚न्नाह । \textbf{कृत‚कानाम}पीत्यादि । आस्तान्ताव‚द‚कृत‚कानामिय‚{\tiny $_{lb}$}‚ञ्चिन्ता । येषाम‚न्य‚थाभा‚{\tiny $_{६}$}‚वः क‚थ‚ञ्चिद‚पि क‚र्त्तुं न श‚क्य‚ते । कृत‚कानाम‚पि‚{\tiny $_{lb}$}‚ ताव‚द् भावानां कीदृशं \textbf{हेतुप‚रिणाम‚निय‚म‚व‚तां} हेतोः प‚रिणामः । उत्त‚रोत्त‚रा‚{\tiny $_{lb}$}‚व‚स्थाप्र‚तिल‚म्भः । त‚स्मान्निय‚मः कार्य‚स्य हेत्व‚न‚न्त‚रं स‚त्ता । स येषां विद्य‚ते ।‚{\tiny $_{lb}$}‚ ते त‚थोच्य‚न्ते । तेषाम‚प्य‚श\textbf{क्यः क्र‚म‚विप‚र्य‚यः क‚र्त्तुं । य‚था बीजाङ्कु}रादीनां । बी‚{\tiny $_{lb}$}‚\leavevmode\ledsidenote{\textenglish{173b/PSVTa}} जात् प‚श्चाद‚ङ्कुरोङ्कुरात् \textbf{काण्डं} य‚त्र पुष्पादी‚{\tiny $_{७}$}‚नि न विप‚र्य‚यः । त‚था \textbf{ऋतुस‚म्व‚{\tiny $_{lb}$}‚स्स‚रादीनां} व्य‚व‚स्थित‚क्र‚म‚त्वं । ऋतूनां हेम‚न्तादिल‚क्ष‚णानां । स‚म्व‚त्स‚राणाञ्च‚{\tiny $_{lb}$}‚ शौ क्र बा र्ह स्प‚त्यादीनां । आदिश‚ब्दाद् ग्र‚ह‚न‚क्ष‚त्र‚प्र‚भृतीनां । \textbf{किं पुन‚र‚च‚लिता‚{\tiny $_{lb}$}‚व‚स्थास्व‚भावानाम‚कृत‚काना}म्व‚र्ण्णानां । अच‚लिताक्र‚माद्य‚व‚स्था स्व‚भाव‚श्च‚{\tiny $_{lb}$}‚ येषाम‚कृत‚कानामिति विग्र‚हः । \textbf{क‚थंदिद् व्य‚व‚स्थितानां} । विनिय‚तेन क्र‚मेण‚{\tiny $_{१}$}‚ ।‚{\tiny $_{lb}$}‚ \textbf{पूर्वाव‚स्थायास्त्याग‚म‚न्त‚रेणान्य‚थाभावायोगात्} क्र‚मान्त‚रेणाव‚स्थान‚स्यायोगात् ।‚{\tiny $_{lb}$}‚ \textbf{पूर्वाव‚स्थात्यागे वा} व‚र्ण्णानाम‚भ्युप‚ग‚म्य‚माने तेषा\textbf{म्विनाश‚प्र‚स‚ङ्गात्} । व‚र्ण्णानाम‚पि‚{\tiny $_{lb}$}‚ ताव‚न्न पूर्वाव‚स्थात्याग‚म‚न्त‚रेण क्र‚म‚विप‚र्य‚यो \textbf{विशेषेण नित्यायामानुपूर्व्या} ।‚{\tiny $_{lb}$}‚ ‚{\tiny $_{lb}$}‚ ‚{\tiny $_{lb}$}‚ \leavevmode\ledsidenote{\textenglish{489/s}}त‚देत‚त् क्र‚मान्य‚र्त्त्वं \textbf{प्र‚तिप‚द‚म्व‚र्ण्णान्य‚त्वे} स्यान्नित्या अपि व‚र्ण्णाः प्र‚तिप‚द‚म्भि‚{\tiny $_{२}$}‚न्ना‚{\tiny $_{lb}$}‚ इति कृत्वा । \textbf{अपूर्वेषा}म्व‚र्ण्णानाम्प्र‚तिप‚द‚मु\textbf{त्पादाद् व‚र्ण्ण‚बाहुल्यं} । त‚स्माद्वा क्र‚मान्य‚त्त्वं‚{\tiny $_{lb}$}‚ स्यात् । \textbf{त‚च्चै}त‚दुभ‚य‚म‚पि \textbf{नाभिम‚तं} मी मां स का ना मेक‚त्वान्नित्य‚त्वाच्च‚{\tiny $_{lb}$}‚ व‚र्ण्णानां । \href{http://sarit.indology.info/?cref=pv.3.259}{२६२.}
	{\color{gray}{\rmlatinfont\textsuperscript{§~\theparCount}}}
	\pend% ending standard par
      ‚{\tiny $_{lb}$}‚

	  
	  \pstart \leavevmode% starting standard par
	\textbf{अपि चे}त्यादिना दूष‚णान्त‚र‚माह । \textbf{देश‚का}लाभ्यां यः कृतः \textbf{क्र‚म}स्त‚स्य व‚र्ण्णे‚{\tiny $_{lb}$}‚ष्व\textbf{भावः} क‚थं । \textbf{व्याप्तिनित्य‚त्व‚व‚र्ण्ण‚नात्} । त‚दुक्तं ।
	{\color{gray}{\rmlatinfont\textsuperscript{§~\theparCount}}}
	\pend% ending standard par
      ‚{\tiny $_{lb}$}‚
	  \bigskip
	  \begingroup
	
	    
	    \stanza[\smallbreak]
	  {\normalfontlatin\large ``\qquad}किञ्च श‚ब्द‚स्य नित्य‚त्वं श्रोत्र‚ज‚प्र‚त्य‚भिज्ञ‚{\tiny $_{३}$}‚या ।&‚{\tiny $_{lb}$}‚विभुत्वं च स्थितं त‚स्य को व्य‚व‚स्येद्विप‚र्य‚य‚मिति ।\edtext{\textsuperscript{*}}{\edlabel{pvsvt_489-1}\label{pvsvt_489-1}\lemma{*}\Bfootnote{Kumārila. }}{\normalfontlatin\large\qquad{}"}\&[\smallbreak]
	  
	  
	  
	  \endgroup
	‚{\tiny $_{lb}$}‚

	  
	  \pstart \leavevmode% starting standard par
	व‚र्ण्णानामाकाश‚व‚द् व्याप्तिव‚र्ण्ण‚नान्न देश‚कृतः क्र‚मः । नित्य‚व‚र्ण्ण‚नान्न काल‚कृतः ।‚{\tiny $_{lb}$}‚ \textbf{सा चेय}मित्यादिना व्याच‚ष्टे । \textbf{सा चेय‚म्व‚र्ण्णानामानु}पूर्वी । \textbf{देश‚कृता वा} स्यात् ।‚{\tiny $_{lb}$}‚ \textbf{य‚था}न्योन्य‚देश‚प‚रिहारेण स्थितानां \textbf{पिपीलिका}दीनाम्प‚ङ्क्तौ । \textbf{काल‚कृता वा}‚{\tiny $_{lb}$}‚ स्यादानुपूर्वी । \textbf{य‚था बीजाङ्कुरादीनां} । य‚दा‚{\tiny $_{४}$}‚ बीजं न त‚दांकुरो य‚दांकुरो न‚{\tiny $_{lb}$}‚ त‚दा प‚त्राद‚य इति \textbf{सेय‚मा}नुपूर्वी \textbf{द्विधा} । देश‚काल‚कृता \textbf{व‚र्ण्णेषु न स‚म्भ‚व‚ति} ।‚{\tiny $_{lb}$}‚ कुतः [।] व्याप्तेर्न्नित्य‚त्वाच्च । त‚त्र \textbf{न} ताव‚द्देश‚कृतानुपूर्वी \textbf{व‚र्ण्णानां स‚म्भ‚व‚ति ।‚{\tiny $_{lb}$}‚ य‚स्माद‚न्योन्य‚देश‚प‚रिहारेण} भावानां \textbf{वृत्तिर्हि देश‚पौर्वाप‚र्यं । त‚दि}त्थ‚म्भूतं पौर्वा‚{\tiny $_{lb}$}‚प‚र्य‚म्व\textbf{र्ण्ण‚षु न स‚म्भ‚व‚ति} । किं कार‚णं [।] व्यापित्वेन \textbf{स‚र्व‚स्य} व‚र्ण‚स्य \textbf{संर्वेण} व‚र्ण्णेन‚{\tiny $_{५}$}‚‚{\tiny $_{lb}$}‚ \textbf{तुल्य‚देश‚त्वात् । वातात‚प‚व‚त्} । लौकिको दृष्टान्तः । शास्त्रीय‚माह । \textbf{आत्मादि‚{\tiny $_{lb}$}‚व‚च्चे}ति । आदिश‚ब्दादाकाशादिप‚रिग्र‚हः । \textbf{त‚था} काल‚कृतानुपूर्वी व‚र्ण्णानान्न‚{\tiny $_{lb}$}‚ ‚{\tiny $_{lb}$}‚ ‚{\tiny $_{lb}$}‚ \leavevmode\ledsidenote{\textenglish{490/s}}स‚म्भ‚व‚ति । य‚स्माद\textbf{न्योन्यं काल‚प‚रिहारेण वृत्तिः काल‚पौर्वाप‚र्यं} । एत‚देव कुतः ।‚{\tiny $_{lb}$}‚ \textbf{य‚दे}त्यादि । य‚स्मिन् काले \textbf{एको नास्ति त‚दान्य‚स्य भावात्} कार‚णात् । \textbf{त‚द‚पि} काल‚{\tiny $_{lb}$}‚पौर्वाप‚र्य\textbf{न्नि‚{\tiny $_{६}$}‚त्येषु} व‚र्ण्णेषु न \textbf{स‚म्भ‚व‚ति । स‚र्व‚दा स‚र्व‚स्य} व‚र्ण्ण‚स्य \textbf{भावात् । न च}‚{\tiny $_{lb}$}‚ देश‚काल‚कृतात् क्र‚मा\textbf{द‚न्या} व‚र्ण्णानुपूर्वी \textbf{ग‚तिः} प्र‚कारोस्ति । \textbf{त‚त्क‚थ‚म्व‚र्ण्ण‚पौर्वाप‚र्यं‚{\tiny $_{lb}$}‚ वाक्यं} य‚द्भ‚व‚द्भि\textbf{र‚पौरुषेयं साध्ये}तेति ।
	{\color{gray}{\rmlatinfont\textsuperscript{§~\theparCount}}}
	\pend% ending standard par
      ‚{\tiny $_{lb}$}‚

	  
	  \pstart \leavevmode% starting standard par
	न च ध्व‚निकृतो युग‚प‚द्भाविनाम्व‚र्ण्णानां क्र‚मो युक्तोऽनित्य‚त्व‚प्र‚संगात् ।‚{\tiny $_{lb}$}‚ त‚दुक्त‚म् [।]
	{\color{gray}{\rmlatinfont\textsuperscript{§~\theparCount}}}
	\pend% ending standard par
      ‚{\tiny $_{lb}$}‚
	  \bigskip
	  \begingroup
	
	    
	    \stanza[\smallbreak]
	  {\normalfontlatin\large ``\qquad}अनित्य‚ध्व‚निकार्य‚त्वात् क्र‚म‚स्यातो विनाशिता ।&‚{\tiny $_{lb}$}‚\leavevmode\ledsidenote{\textenglish{174a/PSVTa}}पुरुषाधीन‚ता चास्य त‚द्विव‚क्षाव‚शा‚{\tiny $_{७}$}‚द् भ‚वेदिति ।\edtext{\textsuperscript{*}}{\edlabel{pvsvt_490-1}\label{pvsvt_490-1}\lemma{*}\Bfootnote{Kumārila. }}{\normalfontlatin\large\qquad{}"}\&[\smallbreak]
	  
	  
	  
	  \endgroup
	‚{\tiny $_{lb}$}‚

	  
	  \pstart \leavevmode% starting standard par
	तेनाय‚म‚र्थो भ‚व‚ति [।] व्यापित्वाद् व‚र्ण्णानां यौग‚प‚द्य‚म‚तो व्यापित्व‚विरोधी‚{\tiny $_{lb}$}‚ क्र‚मः [।] क्र‚म‚विरोधि च व्यापित्वं । क्र‚म‚श्चेद् व‚र्ण्णानामिष्य‚ते व्यापित्व‚ग्राहि‚{\tiny $_{lb}$}‚ प्र‚त्य‚भिज्ञानं भ्रान्तं स्यात् [।] त‚था च देश‚काल‚प्र‚योक्तृभेदेन व‚र्ण्णानाम्भिन्न‚त्वात्‚{\tiny $_{lb}$}‚ कार्य‚त्व‚मिति क‚थ‚म‚नादित्वं क्र‚म‚स्य ।
	{\color{gray}{\rmlatinfont\textsuperscript{§~\theparCount}}}
	\pend% ending standard par
      ‚{\tiny $_{lb}$}‚

	  
	  \pstart \leavevmode% starting standard par
	तेन य‚दुच्य‚ते ।
	{\color{gray}{\rmlatinfont\textsuperscript{§~\theparCount}}}
	\pend% ending standard par
      ‚{\tiny $_{lb}$}‚
	  \bigskip
	  \begingroup
	
	    
	    \stanza[\smallbreak]
	  {\normalfontlatin\large ``\qquad}न च क्र‚म‚स्य कार्य‚त्वं पूर्व‚सिद्ध‚प‚रिग्र‚हात् ।&‚{\tiny $_{lb}$}‚व‚क्ता न हि क्र‚मं क‚श्चित्स्वात‚न्त्र्येण प्र‚प‚द्य‚ते ।&‚{\tiny $_{lb}$}‚य‚थैवास्य प‚रैरुक्त‚स्त‚थैवैन‚म्विव‚क्ष‚ति ।&‚{\tiny $_{lb}$}‚प‚रोप्येव‚म‚त‚श्चास्य स‚म्ब‚न्ध‚व‚द‚नादितेति [।]{\normalfontlatin\large\qquad{}"}\&[\smallbreak]
	  
	  
	  
	  \endgroup
	‚{\tiny $_{lb}$}‚

	  
	  \pstart \leavevmode% starting standard par
	त‚द‚पास्तं । क्र‚मे स‚ति व‚र्ण्णैक‚त्व‚प्र‚त्य‚भिज्ञान‚स्याप्रामाण्येन प्र‚त्युच्चार‚णं‚{\tiny $_{lb}$}‚ व‚र्ण्णानां कार्य‚त्वात् क्र‚म‚स्य च तेभ्योन‚र्थान्त‚र‚त्वात् । न च क्र‚मः क्र‚मिणान्ध‚र्मः [।]‚{\tiny $_{lb}$}‚ ध‚र्म‚स्यापि ध‚र्मिण‚स्स‚काशाद् भेदात् भेदेन श्रोत्र‚ज्ञानेऽव‚भासः स्यात् । न च भ‚व‚ति ।‚{\tiny $_{lb}$}‚ त‚स्माद‚युग‚प‚दुत्प‚न्ना एव‚{\tiny $_{२}$}‚ भावाः क्र‚मः तेन प्र‚त्युच्चार‚ण‚म्व‚र्ण्णानामुत्प‚त्तिभेदात्‚{\tiny $_{lb}$}‚ क्र‚म‚भेदेपि । पूर्व‚दृष्ट एवायं क्र‚म इति प्र‚त्य‚भिज्ञानं सादृश्य‚निब‚न्ध‚नं । न प्र‚त्य‚{\tiny $_{lb}$}‚भिज्ञानं सादृश्य‚निब‚न्ध‚नं । न प्र‚त्य‚भिज्ञानं प्र‚माण‚मिति प्र‚तिपाद‚यिष्य‚ते च ।‚{\tiny $_{lb}$}‚ त‚त्क‚थं क्र‚म‚स्यानादित्वाद‚पौरुषेय‚त्व‚मिति । \href{http://sarit.indology.info/?cref=pv.3.260}{२६३}
	{\color{gray}{\rmlatinfont\textsuperscript{§~\theparCount}}}
	\pend% ending standard par
      ‚{\tiny $_{lb}$}‚‚{\tiny $_{lb}$}‚‚{\tiny $_{lb}$}‚\textsuperscript{\textenglish{491/s}}

	  
	  \pstart \leavevmode% starting standard par
	अथ माभूदेष दोष इत्य‚नित्यान् व्यापिन‚श्च व‚र्ण्णानिच्छेत् वे द वा दी ।‚{\tiny $_{lb}$}‚ त‚दाऽ\textbf{नित्याव्यापितायान्दोषः प्रागेव कीर्तितः} ।
	{\color{gray}{\rmlatinfont\textsuperscript{§~\theparCount}}}
	\pend% ending standard par
      ‚{\tiny $_{lb}$}‚

	  
	  \pstart \leavevmode% starting standard par
	अथेत्यादि‚{\tiny $_{३}$}‚ना व्याच‚ष्टे । \textbf{माभूदेष} व‚र्ण्णानुपूर्व्य‚भाव\textbf{दोष इत्य‚नित्यान‚व्यापि‚{\tiny $_{lb}$}‚न‚श्च व‚र्ण्णानिच्छेद्} वे द वा दी । अनित्य‚त्वात् काल‚कृत‚पौर्वाप‚र्य‚म‚व्यापित्वाद् देश‚{\tiny $_{lb}$}‚कृत‚मिद‚मिति म‚न्य‚मानः । \textbf{ताव‚प्य}नित्याव्यापि\textbf{प‚क्षौ प्रागेव} । अनित्यं य‚त्न‚स‚म्भू‚{\tiny $_{lb}$}‚त‚म्पौरुषेयं क‚थं न त‚द् । {...} इत्यादिना । \textbf{स‚र्व‚त्रानुप‚ल‚म्भः स्या}दित्यादिना‚{\tiny $_{lb}$}‚ च य‚थाक्र‚म\textbf{न्निराकृतावित्य‚प‚रिहारः} । \href{http://sarit.indology.info/?cref=pv.3.261}{२६४}
	{\color{gray}{\rmlatinfont\textsuperscript{§~\theparCount}}}
	\pend% ending standard par
      ‚{\tiny $_{lb}$}‚

	  
	  \pstart \leavevmode% starting standard par
	व‚र्ण्णानां व्य‚क्तिर‚{\tiny $_{४}$}‚\textbf{भिव्य‚क्ति}स्त‚स्याः \textbf{क्र‚मोपि वाक्य‚न्न} भ‚व‚ति । य‚दा क‚र्म‚स्था‚{\tiny $_{lb}$}‚ क्रियाभिव्य‚क्तिस्त‚दा व‚र्ण्णानां व्य‚क्तिविष‚य‚त्व‚क्र‚मो वाक्य‚मित्य‚पि न भ‚व‚तीत्य‚र्थः ।‚{\tiny $_{lb}$}‚ क‚स्मात् [।] \textbf{नित्य‚व्य‚क्तिनिराकृतेः} ।
	{\color{gray}{\rmlatinfont\textsuperscript{§~\theparCount}}}
	\pend% ending standard par
      ‚{\tiny $_{lb}$}‚

	  
	  \pstart \leavevmode% starting standard par
	\textbf{ने}त्यादिना व्याच\textbf{ष्टे । न व‚र्ण्णानां रूपानुपूर्वी} स्व‚रूपानुपूर्वी \textbf{वाक्यं} येनाय‚{\tiny $_{lb}$}‚न्दोषः [।] \textbf{किन्त‚र्हि [।] त‚द्व्य‚क्तेः} । व‚र्ण्ण‚रूप‚व्य‚क्तेर्व्य‚क्त‚त्व‚ल‚क्ष‚णाया यानुपूर्वी‚{\tiny $_{lb}$}‚ त‚द् वाक्यं । तामेव द‚र्श‚य‚न्नाह ।‚{\tiny $_{५}$}‚ \textbf{ने}त्यादि । \textbf{सा} व्य‚क्ति\textbf{र्य‚था स्व‚व‚र्ण्णाभिव्य‚क्ति‚{\tiny $_{lb}$}‚प्र‚त्य‚यानां} । येन य‚स्य व‚र्ण्णाभिव्य‚क्तिप्र‚त्य‚यास्ताल्वादिव्यापारास्तेषां \textbf{क्र‚माद् भ‚व‚न्ती‚{\tiny $_{lb}$}‚ क्र‚म‚योगिनीति} कृत्वा \textbf{त‚दानुपूर्वी} तेषां व्य‚क्तानां व‚र्ण्णानामानुपूर्वी \textbf{वाक्य‚म् [।]‚{\tiny $_{lb}$}‚ इत्य‚पि मिथ्या} । किं कार‚णं [।] \textbf{त‚स्या} व्य‚क्ते\textbf{र्नित्येषु प्रागेव} सा मा न्य व्य क्ति‚{\tiny $_{lb}$}‚ चि न्तास्थाने \textbf{निषिद्ध‚त्वात्} । त‚देव स्मार‚य‚न्नाह । \textbf{कार्य‚तेत्}यादि । व्य‚ञ्ज‚क‚कृतेन‚{\tiny $_{६}$}‚‚{\tiny $_{lb}$}‚ \textbf{साक्षा}ज्ज‚न‚न\textbf{श‚क्त्युप‚धानेन । ज्ञान‚ज‚न‚ना}स‚म‚र्थानां घ‚टादीनां \textbf{कार्य‚विशेष एव‚{\tiny $_{lb}$}‚ व्य‚क्तिरित्याख्यात‚मेत‚त्} । किञ्च \textbf{क‚र‚णाना}न्ताल्वादीनां \textbf{व्यापारादेव त‚स्माद्धेतो}स्ते‚{\tiny $_{lb}$}‚षाम्व‚र्ण्णानामुप‚ल‚ब्धेस्तेषाम्व‚र्ण्णानां \textbf{कार्य‚ता} प्राप्ता [।] य‚स्मात् ।
	{\color{gray}{\rmlatinfont\textsuperscript{§~\theparCount}}}
	\pend% ending standard par
      ‚{\tiny $_{lb}$}‚‚{\tiny $_{lb}$}‚‚{\tiny $_{lb}$}‚\textsuperscript{\textenglish{492/s}}

	  
	  \pstart \leavevmode% starting standard par
	\textbf{य‚त् ख‚लु रूपं य‚त एवोप‚ल‚भ्य‚ते त‚स्य} रूप‚स्य \textbf{लोकः कार्य‚तां प्र‚ज्ञाप‚य‚ती}ति स‚म्ब‚{\tiny $_{lb}$}‚\leavevmode\ledsidenote{\textenglish{174b/PSVTa}} न्धः । य‚त एवोप‚ल‚भ्य‚त इति य‚च्छ‚ब्देन यो निर्दि‚{\tiny $_{७}$}‚ष्ट‚स्त‚स्योप‚ल‚ब्धि\textbf{स्त‚दुप‚ल‚ब्धि}स्त‚{\tiny $_{lb}$}‚न्ना\textbf{न्त‚रीयिका}मेवो\textbf{प‚ल‚ब्धिमाश्रित्ये}ति भिन्न‚क्र‚म एव‚कारः । \textbf{सा} य‚थोक्तोप‚ल‚ब्धि‚{\tiny $_{lb}$}‚\textbf{र्व‚र्ण्णेष्व‚प्य‚स्ति} [।] प्र‚य‚त्न‚व्यापारोप‚ल‚ब्धिनान्त‚रीय‚क‚त्वादेव व‚र्ण्णोप‚ल‚ब्धेः ।‚{\tiny $_{lb}$}‚ \textbf{सा चो}प‚ल‚ब्धि\textbf{र‚न्य‚त्रापि} कार्य‚त्वेन प्र‚सिद्धे व‚स्तुनि[।] \textbf{त‚दाश्र‚यः} कार्य‚ताप्र‚ज्ञ‚प्तेराश्र‚यः ।‚{\tiny $_{lb}$}‚ \textbf{नातो}धिको \textbf{विशेषः [।] त‚त्क‚थ‚न्तुल्ये} कार्य\textbf{ताभ्युप‚ग‚म‚निब‚न्ध‚ने न व‚र्ण्णाः कार्याः} ।
	{\color{gray}{\rmlatinfont\textsuperscript{§~\theparCount}}}
	\pend% ending standard par
      ‚{\tiny $_{lb}$}‚

	  
	  \pstart \leavevmode% starting standard par
	\textbf{ने}त्यादि प‚रः ।‚{\tiny $_{५}$}‚ \textbf{न चैत‚दुप‚ल‚ब्ध्याश्र‚या} । एकोप‚ल‚ब्धिनान्त‚रीयिका याऽप‚र‚{\tiny $_{lb}$}‚स्योप‚ल‚ब्धिस्त‚दाश्र‚या \textbf{कार्य‚तास्थितिः [।] किन्त‚र्हि [।] य‚त्स‚त्येव भ‚व‚ति} । य‚स्मिन्‚{\tiny $_{lb}$}‚ स‚त्येव य‚द् भ‚व‚ति । \textbf{इति} एवं \textbf{स‚त्ताश्र‚या} कार्य‚तास्थितिः । न च व‚र्ण्णानां क‚र‚णेभ्यः‚{\tiny $_{lb}$}‚ स‚त्ता भ‚व‚ति किन्तूप‚ल‚ब्धिरेवेति म‚न्य‚ते ।
	{\color{gray}{\rmlatinfont\textsuperscript{§~\theparCount}}}
	\pend% ending standard par
      ‚{\tiny $_{lb}$}‚

	  
	  \pstart \leavevmode% starting standard par
	\textbf{सा स‚त्ते}त्या\textbf{चा र्यः} । स‚त्ताश्र‚य‚स्यैव कार्य‚ता प्र‚ज्ञ‚प्तिरित्य‚स‚त्त्य‚मेत‚त् । केव‚लं‚{\tiny $_{lb}$}‚ \textbf{सा स‚त्ता कुतः} प्र‚माणात् \textbf{सिद्धा येने}यं \textbf{कार्य‚तां साध‚{\tiny $_{२}$}‚येत् । न ह्य‚सिद्धायाम‚स्यां}‚{\tiny $_{lb}$}‚ स‚त्ताया\textbf{मेव‚म्भ‚व‚ति} । स‚त्येवास्मिन्निदं भ‚व‚तीत्येव‚म्भ‚व‚ति । \textbf{त‚स्मात् स‚त्तासिद्धिस्त‚{\tiny $_{lb}$}‚त्साध‚नी} । त‚स्याः कार्य‚तायाः साध‚नी । \textbf{सा} सिद्धि\textbf{रुप‚ल‚ब्धिरेव} सिद्धेर्ज्ञान‚{\tiny $_{lb}$}‚स्व‚भाव‚त्वात् । त‚त‚श्च य‚न्नान्त‚रीयिकैव य‚त्स‚त्तोप‚ल‚ब्धिस्त‚त्त‚स्य कार्य‚मित्येताव‚त्‚{\tiny $_{lb}$}‚ स्थितं [।] त‚च्च व‚र्ण्णेष्व‚पि तुल्य‚मिति क‚थं न व‚र्ण्णाः कार्याः ।
	{\color{gray}{\rmlatinfont\textsuperscript{§~\theparCount}}}
	\pend% ending standard par
      ‚{\tiny $_{lb}$}‚

	  
	  \pstart \leavevmode% starting standard par
	\textbf{स‚त्त्य}मित्यादि प‚रः । \textbf{स‚त्त्य‚मेवं} क‚र‚ण‚व्या‚{\tiny $_{३}$}‚पारादेव श‚ब्दोप‚ल‚ब्धौ त‚त्कार्य‚ता‚{\tiny $_{lb}$}‚ स्यात् । \textbf{य‚दि त‚स्य} श‚ब्द‚स्य ताल्वादिव्यापारात् \textbf{स‚त्ता न सिद्धा स्यात्} । किन्तु‚{\tiny $_{lb}$}‚ सिद्धैव प्र‚माणेन । त‚था हि पूर्वं गोश‚ब्दं श्रुत‚व‚तः पुंसोन्य‚दा गोश‚ब्द‚श्र‚व‚णे स एवायं‚{\tiny $_{lb}$}‚ गोश‚ब्द इति त‚त्त्व‚ग्राहिणी प्र‚त्य‚भिज्ञोत्प‚द्य‚ते । त‚त्त्व‚ग्र‚ह‚ण‚मेवान्य‚था न स्यात् । य‚दि‚{\tiny $_{lb}$}‚ पूर्वोत्त‚र‚श्र‚व‚ण‚काल‚योर‚न्त‚राले श‚ब्दो न स्यादित्य‚र्थाप‚त्त्या प्राक्छ‚ब्द‚स्य‚{\tiny $_{४}$}‚ स‚त्ता‚{\tiny $_{lb}$}‚ सिद्धैव । न च सिद्धिपूर्विका सिद्धिः कार्य‚तासाध‚नी । किन्तु \textbf{सा हि स‚त्तासिद्धिः}‚{\tiny $_{lb}$}‚ कार्य‚त्व‚प्र‚ज्ञ‚प्तेर्निमित्तं या \textbf{सिद्धिपूर्विका} । य‚था घ‚ट‚स्य प्राग‚स‚तः कुलालादिव्या‚{\tiny $_{lb}$}‚पारादेव प‚श्चात् सिद्धिः ।
	{\color{gray}{\rmlatinfont\textsuperscript{§~\theparCount}}}
	\pend% ending standard par
      ‚{\tiny $_{lb}$}‚

	  
	  \pstart \leavevmode% starting standard par
	\textbf{न‚न्वि}त्यादि सि द्धा न्त वा दी । एत‚त् क‚थ‚य‚ति [।] न ताव‚त् प्र‚त्य‚भि\textbf{ज्ञा}‚{\tiny $_{lb}$}‚ प्र‚माण‚मिति प्रातिपाद‚यिष्य‚ते त‚त्क‚थं स‚त्ता सिद्धा । भ‚व‚तु नाम प्राक्छ‚ब्द‚स्य स‚त्ता‚{\tiny $_{lb}$}‚ सिद्धा । त‚था‚{\tiny $_{५}$}‚पि श‚ब्द‚स्य \textbf{त‚द्रूप‚म‚सिद्धं} । क‚त‚र‚त् त‚द् रूप‚मित्याह । \textbf{य‚त् त‚था‚{\tiny $_{lb}$}‚‚{\tiny $_{lb}$}‚ \leavevmode\ledsidenote{\textenglish{493/s}}विधे}त्यादि । \textbf{त‚थाविध‚स्य} श‚ब्द‚स्व‚ल‚क्ष‚ण‚प्र‚तिभासि\textbf{ज्ञान}स्य \textbf{व्य‚व‚धाने}नोप‚युक्तं शीलं‚{\tiny $_{lb}$}‚ य‚स्येति विग्र‚हः । य‚दि हि त‚थाभूतं रूपं प्राक् सिद्धं स्यात् त‚दा नित्यं श‚ब्दो‚{\tiny $_{lb}$}‚प‚ल‚म्भः स्यात् ।
	{\color{gray}{\rmlatinfont\textsuperscript{§~\theparCount}}}
	\pend% ending standard par
      ‚{\tiny $_{lb}$}‚

	  
	  \pstart \leavevmode% starting standard par
	\textbf{सिद्ध‚मेव त‚च्छ}ब्द‚स्य य‚थोक्तं रूपं केव‚ल\textbf{म‚न्य‚स्}य स‚ह‚कारिणो \textbf{वैक‚ल्या}च्छ्रोत्र‚{\tiny $_{lb}$}‚विज्ञाने कार‚ण‚त्वेन \textbf{नोप‚युक्त‚मिति चेत्} ।
	{\color{gray}{\rmlatinfont\textsuperscript{§~\theparCount}}}
	\pend% ending standard par
      ‚{\tiny $_{lb}$}‚

	  
	  \pstart \leavevmode% starting standard par
	य‚द्येवं स‚ह‚कारिस‚न्निधाने या प‚श्चात् स्व‚ज्ञाने उप‚युक्ता । या प्र‚य‚त्नात्‚{\tiny $_{lb}$}‚ प्राग‚नुप‚युक्ताव‚स्था । ते प‚र‚स्प‚र‚विरुद्धे । \textbf{क‚थ‚मिदानीमुप‚युक्तानुप‚युक्त‚यो}र‚व‚स्थ‚{\tiny $_{lb}$}‚योर्विरुद्ध‚यो\textbf{र‚भेदः} [।] अपि तु भेद एव त‚त‚श्च नानात्वात् स तादृशः श‚ब्द‚स्य‚{\tiny $_{lb}$}‚ स्व‚भावः कृत इति कार्य एव श‚ब्दः स्यात् ।
	{\color{gray}{\rmlatinfont\textsuperscript{§~\theparCount}}}
	\pend% ending standard par
      ‚{\tiny $_{lb}$}‚

	  
	  \pstart \leavevmode% starting standard par
	अथापि स्याद् [।] योसाव‚तिश‚यो भ‚व‚ति न स श‚ब्द‚स्यात्म‚भूतोपि त्व‚र्थान्त‚र‚{\tiny $_{lb}$}‚मिति पूर्व‚क‚{\tiny $_{७}$}‚स्व‚भावाद‚प्र‚च्युत एवासावित्याह । \leavevmode\ledsidenote{\textenglish{175a/PSVTa}}
	{\color{gray}{\rmlatinfont\textsuperscript{§~\theparCount}}}
	\pend% ending standard par
      ‚{\tiny $_{lb}$}‚

	  
	  \pstart \leavevmode% starting standard par
	\textbf{नापि भेदो}ऽकार‚काव‚स्थातः कार‚काव‚स्थाल‚क्ष‚णोतिश‚यः श‚ब्द\textbf{स्व‚भावा‚{\tiny $_{lb}$}‚संस्प‚र्शी} श‚ब्द‚स्व‚भावान्न व्य‚तिरिक्त इति याव‚त् । व्य‚तिरेके हि \textbf{त‚स्यैवातिश‚य‚स्य}‚{\tiny $_{lb}$}‚ श‚ब्द‚ज्ञाने \textbf{उप‚योग‚सिद्धेः} कार‚ण‚त्व‚सिद्वे\textbf{स्त‚स्य} श‚ब्द‚स्\textbf{याकार‚ण‚त्व‚प्र‚संगात्} । त‚था हि‚{\tiny $_{lb}$}‚ \textbf{य‚स्यैव भावे साध्य‚सिद्धिस्त‚देव त‚त्र} साध्य \textbf{उप‚योगि युक्त}न्नाप‚रं । अतिश‚यो ज्ञान‚{\tiny $_{१}$}‚‚{\tiny $_{lb}$}‚ उप‚युज्य‚ते [।] साक्षाद‚तिश‚ये तु श‚ब्द उप‚युज्य‚त इति पार‚म्प‚र्येण श‚ब्दोपि ज्ञाने‚{\tiny $_{lb}$}‚ उप‚युक्त एवेत्य‚त आह ।
	{\color{gray}{\rmlatinfont\textsuperscript{§~\theparCount}}}
	\pend% ending standard par
      ‚{\tiny $_{lb}$}‚

	  
	  \pstart \leavevmode% starting standard par
	\textbf{त‚द‚तिश‚ये}त्यादि । \textbf{त‚स्मिन्न‚तिश‚य‚स्य} श‚ब्द‚स्यो\textbf{प‚योगेपि} क‚ल्प्य‚माने । \textbf{त‚द्व‚त्प्र‚{\tiny $_{lb}$}‚स‚ङ्गः} । ज्ञान‚व‚त्प्र‚स‚ङ्गः । य‚था विज्ञाने क‚र्त्त‚व्येर्थान्त‚र‚भूते नातिश‚येन श‚ब्द उप‚यु‚{\tiny $_{lb}$}‚ज्य‚ते । त‚द्व‚द‚तिश‚येपि क‚र्त्त‚व्येर्थान्त‚र‚भूतोतिश‚यः क‚ल्प‚नीयः । त‚था चान‚व‚स्था‚{\tiny $_{lb}$}‚ स्याद‚{\tiny $_{२}$}‚तोतिश‚यः श‚ब्दाद‚भिन्नः [।] य‚त‚श्चाभिन्नः । \textbf{त‚स्मात्} त‚द‚न्यं स्व‚विष‚य‚{\tiny $_{lb}$}‚ज्ञान‚ज‚न‚नं श‚ब्द‚स्व‚भाव‚म\textbf{तिशेत एव} । स्व‚रूप‚भेदेन । कोऽ\textbf{व्य‚व‚हित‚साम‚र्थ्यो‚{\tiny $_{lb}$}‚‚{\tiny $_{lb}$}‚ \leavevmode\ledsidenote{\textenglish{494/s}}प‚योगोव‚स्थाभेदोऽव्य‚व‚हित‚साम‚र्थ्य उप‚योगो य‚स्याव‚स्थाभेद‚स्येति विग्र‚हः} ।
	{\color{gray}{\rmlatinfont\textsuperscript{§~\theparCount}}}
	\pend% ending standard par
      ‚{\tiny $_{lb}$}‚

	  
	  \pstart \leavevmode% starting standard par
	अथ स्यात् [।] नैव विज्ञान‚ज‚निकाव‚स्थोत्प‚द्य‚ते नित्य‚त्वात् [।] किन्तु ताल्वा‚{\tiny $_{lb}$}‚दिक‚म‚पेक्ष्यासौ ज्ञानं ज‚न‚य‚तीति [।]
	{\color{gray}{\rmlatinfont\textsuperscript{§~\theparCount}}}
	\pend% ending standard par
      ‚{\tiny $_{lb}$}‚

	  
	  \pstart \leavevmode% starting standard par
	अत आह । अतिश‚य‚{\tiny $_{३}$}‚स्येत्यादि । नास्य स‚ह‚कारिकृतोतिश‚योस्तीत्य‚न‚ति‚{\tiny $_{lb}$}‚श‚य‚स्य स‚ह‚कारिणं प्र‚त्य‚पेक्षा प्रागेव निर‚स्ता [।] \textbf{स च} ज‚न‚कः श‚ब्द‚स्व‚भावः \textbf{क‚र‚ण‚{\tiny $_{lb}$}‚व्यापारादेव सिद्ध इति कृत्वा स‚र्व‚कार्य‚तुल्य‚ध‚र्मा} । स‚र्वे कार्ये तुल्य‚ध‚र्मा य‚स्येति विग्र‚हः ।‚{\tiny $_{lb}$}‚ त‚स्य तादृश‚स्य कार्य‚तुल्य‚ध‚र्म‚णः श‚ब्द‚स्य व्य‚क्ताविष्य‚माणायां स‚र्व‚म‚ङ्कुराद्य‚पि‚{\tiny $_{lb}$}‚ व्य‚ङ्ग्यं स्यात् । न वा किंचिद् व्य‚ङ्ग्यं ।‚{\tiny $_{४}$}‚ श‚ब्दोपि कार्यः स्याद् विशेषाभावात् ।
	{\color{gray}{\rmlatinfont\textsuperscript{§~\theparCount}}}
	\pend% ending standard par
      ‚{\tiny $_{lb}$}‚

	  
	  \pstart \leavevmode% starting standard par
	त‚था हि [।] स्व‚ज्ञानेन क‚र‚णेनान्य‚धीहेतुर‚र्थो व्य‚ञ्ज‚को म‚तः । क‚दा [।]‚{\tiny $_{lb}$}‚ सिद्धेर्थे । य‚द्य‚सो व्य‚ङ्ग्यः कार‚णाल्ल‚ब्ध‚स‚त्ताको भ‚व‚ति । य‚था दीपः कुलालादि‚{\tiny $_{lb}$}‚सिद्धे घ‚टे त‚ज्ज्ञान‚हेतुर्व्य‚ञ्ज‚कः । अन्य‚था वापि य‚दि व्य‚ङ्ग्यः प्राग‚सिद्धः स्यात् ।‚{\tiny $_{lb}$}‚ त‚दा को विशेषोस्य व्य‚ञ्ज‚क‚स्य कार‚काद्धेतोः [।]
	{\color{gray}{\rmlatinfont\textsuperscript{§~\theparCount}}}
	\pend% ending standard par
      ‚{\tiny $_{lb}$}‚

	  
	  \pstart \leavevmode% starting standard par
	स्व‚प्र‚तिप‚त्तीत्यादिना व्याच‚ष्टे । स्व‚{\tiny $_{५}$}‚प्र‚तिप‚त्तिरेव द्वार‚मुपाय‚स्तेन क‚र‚णेना‚{\tiny $_{lb}$}‚न्य‚स्य घ‚टादेः प्र‚तिप‚त्तिहेतुर्लोके व्य‚ञ्ज‚कः सिद्धः । दीपादिव‚त् । स चेद् व्य‚ङ्ग्यः‚{\tiny $_{lb}$}‚ व्य‚ञ्ज‚क‚व्यापारात् प्राक् सिद्धः स्यात् ।
	{\color{gray}{\rmlatinfont\textsuperscript{§~\theparCount}}}
	\pend% ending standard par
      ‚{\tiny $_{lb}$}‚

	  
	  \pstart \leavevmode% starting standard par
	न‚नु च प्र‚दीपादिर‚प्युप‚ल‚ब्धियोग्यं घ‚ट‚क्ष‚णं प्राग‚सिद्ध‚मेव ज‚न‚य‚ति । त‚त्कि‚{\tiny $_{lb}$}‚मुच्य‚ते स चेत् प्राक् सिद्ध इति [।]
	{\color{gray}{\rmlatinfont\textsuperscript{§~\theparCount}}}
	\pend% ending standard par
      ‚{\tiny $_{lb}$}‚

	  
	  \pstart \leavevmode% starting standard par
	अत आह । स‚मान‚जातीयेत्यादि । अनुप‚ल‚म्भ‚योग्यः पूर्व‚को घ‚टादिक्ष‚{\tiny $_{६}$}‚णः‚{\tiny $_{lb}$}‚ स‚मान‚जातीय उपादान‚क्ष‚ण‚स्त‚स्य व्य‚ञ्ज‚क‚व्यापारात् प्राक् सिद्धेः कार‚णात् स चेत्‚{\tiny $_{lb}$}‚ प्राक् सिद्धः स्यादित्युच्य‚ते । न त‚स्यैव व्य‚ञ्ज‚काल्ल‚भ्य‚स्य ज्ञान‚हेतोर‚तिश‚य‚स्य‚{\tiny $_{lb}$}‚ प्राक् सिद्धेः सिद्ध उच्य‚ते । किं कार‚णं [।] त‚स्य य‚थोक्त‚स्यातिश‚य‚स्य । त‚त्साम‚{\tiny $_{lb}$}‚ग्रीप्र‚त्य‚य‚त्वात् । सा व्य‚ञ्ज‚क‚साम‚ग्री प्र‚त्य‚यः कार‚णं य‚स्येति विग्र‚हः । ये पुनः‚{\tiny $_{lb}$}‚ \leavevmode\ledsidenote{\textenglish{175b/PSVTa}} स्व‚व्यापारात् प्राग् असिद्ध‚स्यो‚{\tiny $_{७}$}‚प‚ल‚म्भ‚काः कार‚का एव ते । किमिव [।] कुला‚{\tiny $_{lb}$}‚लादिव‚द् घ‚टादौ श‚ब्द‚स्याप्युप‚ल‚म्भ‚हेत‚वः कुलालादितुल्या इति । श‚ब्दोपि घ‚टा‚{\tiny $_{lb}$}‚दिव‚त् कार्य एवं ।
	{\color{gray}{\rmlatinfont\textsuperscript{§~\theparCount}}}
	\pend% ending standard par
      ‚{\tiny $_{lb}$}‚

	  
	  \pstart \leavevmode% starting standard par
	न‚न्वेक‚दा श्रुत‚स्य श‚ब्द‚स्यान्य‚दा श्र‚व‚णे च स एवाय‚मिति त‚त्त्वं प्र‚त्य‚क्ष‚प्र‚त्य‚भि‚{\tiny $_{lb}$}‚‚{\tiny $_{lb}$}‚ \leavevmode\ledsidenote{\textenglish{495/s}}ज्ञ‚या प्र‚तीय‚ते । त‚त्त्व‚प्र‚तिप‚त्त्य‚न्य‚थानुप‚प‚त्त्या च ताल्वादिव्यापारात् प्राक् स‚त्त्वं‚{\tiny $_{lb}$}‚ श‚ब्द‚स्यापि निश्चित‚मिति क‚थ‚न्ताल्वाद‚यो सिद्धोप‚ल‚म्भ‚नाः [।] तेन व्य‚ञ्ज‚{\tiny $_{१}$}‚का‚{\tiny $_{lb}$}‚ एव युक्ताः ।
	{\color{gray}{\rmlatinfont\textsuperscript{§~\theparCount}}}
	\pend% ending standard par
      ‚{\tiny $_{lb}$}‚

	  
	  \pstart \leavevmode% starting standard par
	एव‚म्म‚न्य‚ते । प्र‚थ‚मे क्ष‚णे श‚ब्द‚ग्र‚ह‚णं द्वितीय‚क्ष‚णे पूर्व‚गृहीत‚श‚ब्दाहित‚संस्कार‚{\tiny $_{lb}$}‚प्र‚बोध‚स्त‚तोन्य‚स्मिन् क्ष‚णे श‚ब्द‚स्म‚र‚णं । त‚त‚त‚श्च‚तुर्थे क्ष‚णे तिरोहिते त‚स्मिन् स‚{\tiny $_{lb}$}‚ एवायं घ‚ट‚श‚ब्द इति प्र‚त्य‚भिज्ञानं क‚थं प्र‚त्य‚क्षं स्याद‚स‚न्निहित‚विष‚य‚त्वात् ।
	{\color{gray}{\rmlatinfont\textsuperscript{§~\theparCount}}}
	\pend% ending standard par
      ‚{\tiny $_{lb}$}‚

	  
	  \pstart \leavevmode% starting standard par
	नापि प्राक्प्र‚बुद्ध‚संस्कार‚स्य पुंसो व‚र्ण्ण‚ग्राह‚कं प्र‚त्य‚भिज्ञानं स‚म्भ‚व‚ति । व‚र्ण्ण‚स्य‚{\tiny $_{lb}$}‚ सांश‚त्वादित्युक्तं । अन्त्य‚व‚{\tiny $_{२}$}‚र्ण्ण‚भाग‚काले च पूर्व‚व‚र्ण्ण‚भागानाम‚स‚त्त्वेनान्त्य‚स्यापि‚{\tiny $_{lb}$}‚ व‚र्ण्ण‚स्यास‚न्निहित‚त्वात् । अत एव प‚द‚वाक्य‚योर‚पि ग्राह‚कं प्र‚त्य‚क्षं प्र‚त्य‚भिज्ञानं न‚{\tiny $_{lb}$}‚ स‚म्भ‚व‚ति व‚र्ण्ण‚स‚मुदाय‚त्वात् प‚दादेर‚न्त्य‚व‚र्ण्ण‚काले च पूर्व‚पूर्व‚व‚र्ण्णानाम‚स‚त्त्वात्‚{\tiny $_{lb}$}‚ स‚न्निहित‚विष‚य‚ञ्च प्र‚त्य‚क्ष‚मिष्य‚ते । त‚स्मान्न प्र‚त्य‚क्षं प्र‚त्य‚भिज्ञान‚म्व‚र्ण्ण‚प‚द‚वाक्येषु‚{\tiny $_{lb}$}‚ त‚त्त्व‚ग्राह‚कं स‚म्भ‚व‚ति [।] अत एव चा चा र्ये ण नो‚{\tiny $_{३}$}‚प‚न्य‚स्तं [।]
	{\color{gray}{\rmlatinfont\textsuperscript{§~\theparCount}}}
	\pend% ending standard par
      ‚{\tiny $_{lb}$}‚

	  
	  \pstart \leavevmode% starting standard par
	भ‚व‚तु वा तेषु प्र‚त्य‚भिज्ञानं प्र‚त्य‚क्ष‚न्त‚थापि त‚त्त्व‚ग्र‚ह‚णान्य‚थानुप‚प‚त्त्या न ताल्वा‚{\tiny $_{lb}$}‚दिव्यापारात् प्राक्छ‚ब्द‚स्य स‚त्त्व‚क‚ल्प‚ना युक्ता । स‚दृशाप‚र‚ग्र‚ह‚णेनापि त‚त्त्व‚ग्र‚ह‚ण‚स्य‚{\tiny $_{lb}$}‚ स‚म्भ‚वात् स‚दृशाप‚र‚ग्र‚ह‚ण‚मेवाव्याप्य‚सिद्ध‚मिति चेत् ।
	{\color{gray}{\rmlatinfont\textsuperscript{§~\theparCount}}}
	\pend% ending standard par
      ‚{\tiny $_{lb}$}‚

	  
	  \pstart \leavevmode% starting standard par
	न‚न्वेक‚त्व‚म‚पि नैव सिद्धं । त‚त्त्व‚ग्र‚ह‚णात् सिद्ध‚मिति चेन्न [।] भिन्नेष्व‚पि‚{\tiny $_{lb}$}‚ लून‚पुन‚र्जातेषु केशेषु त‚त्त्व‚ग्र‚ह‚ण‚स्य द‚र्श‚नात् संश‚य एवातः । क‚थ‚म‚{\tiny $_{४}$}‚र्थाप‚त्त्या प्राक्‚{\tiny $_{lb}$}‚ स‚त्त्व‚क‚ल्प‚ना ।
	{\color{gray}{\rmlatinfont\textsuperscript{§~\theparCount}}}
	\pend% ending standard par
      ‚{\tiny $_{lb}$}‚

	  
	  \pstart \leavevmode% starting standard par
	अथ प्र‚त्य‚भिज्ञाय‚मान‚त्वाच्छ‚ब्द‚स्य नित्य‚त्व‚म् [।] अनित्य‚त्वे ह्य‚नेक‚त्वात्‚{\tiny $_{lb}$}‚ प्र‚त्य‚भिज्ञान‚मेव न स्यात् । त‚था । यः प‚रार्थ‚म्प्र‚युज्य‚ते स प्र‚योगात् प्राग् विद्य‚{\tiny $_{lb}$}‚मानो य‚था वास्यादिच्छिदायां । प्र‚युज्य‚ते च श‚ब्दः प‚र‚प्र‚त्याय‚नाय । त‚स्मात्‚{\tiny $_{lb}$}‚ सोपि प्राग् विद्य‚त एव चेति [।]
	{\color{gray}{\rmlatinfont\textsuperscript{§~\theparCount}}}
	\pend% ending standard par
      ‚{\tiny $_{lb}$}‚

	  
	  \pstart \leavevmode% starting standard par
	अत आह । \textbf{प्र‚त्य‚भिज्ञाने}त्यादि । श‚ब्द‚स्य स‚दा स‚त्तासिद्धिहेत‚वः । \textbf{तेपि न‚{\tiny $_{lb}$}‚ हेतुल‚क्ष‚णं पुष्ण‚न्ति} । त‚था‚{\tiny $_{५}$}‚ ह्य‚नित्येपि प्र‚दीपादौ प्र‚त्य‚भिज्ञान‚न्दृष्टं । त‚स्माद‚नै‚{\tiny $_{lb}$}‚कान्तिक‚मेत‚त् [।] त‚था क्ष‚णिकेपि क‚र्म‚णि प्र‚योगे दृश्य‚ते । तेन प्र‚युज्य‚मान‚त्वा‚{\tiny $_{lb}$}‚दित्य‚पि हेतुर‚नैकान्तिक एव ।
	{\color{gray}{\rmlatinfont\textsuperscript{§~\theparCount}}}
	\pend% ending standard par
      ‚{\tiny $_{lb}$}‚

	  
	  \pstart \leavevmode% starting standard par
	\textbf{य‚द‚पि किञ्चि}ल्लिङ्गं श‚ब्द‚स्यैक‚त्व‚साध‚नायोपादीय‚ते । \textbf{उत्त‚रा} प‚श्चाद्भाविन्य‚{\tiny $_{lb}$}‚\textbf{कार‚प्र‚तीति}र्या सा पूर्वाभिन्न‚विष‚या । पूर्व‚या अकार‚प्र‚तीत्या एक‚विष‚या ।
	{\color{gray}{\rmlatinfont\textsuperscript{§~\theparCount}}}
	\pend% ending standard par
      ‚{\tiny $_{lb}$}‚

	  
	  \pstart \leavevmode% starting standard par
	एतेन श‚ब्दानामेक‚त्व‚साध‚नान्नित्य‚त्वं‚{\tiny $_{६}$}‚साधित‚मिति म‚न्य‚ते । अकार\textbf{प्र‚तीतिरि}ति‚{\tiny $_{lb}$}‚ ‚{\tiny $_{lb}$}‚ \leavevmode\ledsidenote{\textenglish{496/s}}हेतुः । \textbf{त‚द्व‚दि}ति पूर्वाकार‚प्र‚तीतिव‚दि\textbf{त्यादि} । आदिश‚ब्दाद् द्रुत‚म‚ध्य‚विल‚म्बिताव‚स्था‚{\tiny $_{lb}$}‚यामेक एव ग‚कारादिव‚र्ण्ण‚स्स एवायं ग‚कारादिव‚र्ण्णो द्रुतादिभेद‚भिन्न इति प्र‚तीतेः‚{\tiny $_{lb}$}‚ [।] प्र‚योग‚स्तु या या अकार‚प्र‚तीतिः सा पूर्वाकार‚प्र‚तीत्य‚भिन्न‚विष‚या । त‚द्य‚था‚{\tiny $_{lb}$}‚ \leavevmode\ledsidenote{\textenglish{176a/PSVTa}} पूर्व अकार‚प्र‚तीतिः । अकार‚प्र‚तीतिश्चोत्त‚राप्य‚कार‚प्र‚तीतिरि‚{\tiny $_{७}$}‚ति स्व‚भाव‚हेतुप्र‚ति‚{\tiny $_{lb}$}‚रूप‚कः । \textbf{त‚द‚पि} साध‚नं पूर्वाप‚र‚योर‚कार\textbf{स्व‚ल‚क्ष‚ण‚योर‚भेद‚साध‚ने} न \textbf{स‚म‚र्थ} । त‚था‚{\tiny $_{lb}$}‚ ह्य‚कार‚प्र‚तीतेरित्य‚यं हेतुर्विशेषेण वा स्यात् पूर्वाकार‚प्र‚तीतिरूप‚त्वादिति । सामान्येन‚{\tiny $_{lb}$}‚ वा स्याद‚कार‚प्र‚तीतिमात्र‚त्वादिति । आद्ये प‚क्षे हेतुर‚सिद्धः । किं कार‚णं [।]‚{\tiny $_{lb}$}‚ \textbf{त‚त्स्व‚भाव‚त्वासिद्धेः} पूर्वाकार‚प्र‚तीतित्वासिद्धेः । य‚द्वा विशेषेण वा हेतुरुत्त‚राकार‚{\tiny $_{lb}$}‚प्र‚तीतिरूप‚त्वादिति ।
	{\color{gray}{\rmlatinfont\textsuperscript{§~\theparCount}}}
	\pend% ending standard par
      ‚{\tiny $_{lb}$}‚

	  
	  \pstart \leavevmode% starting standard par
	त‚{\tiny $_{१}$}‚द‚पि न साध‚नं । किङ्कार‚णं [।] त‚त्स्व‚भाव‚त्वासिद्धेः साध्य‚स्व‚भाव‚त्वा‚{\tiny $_{lb}$}‚सिद्धेः । अनैकान्तिक‚त्वं व्याप्तेर‚सिद्ध‚त्वादित्य‚र्थः ।
	{\color{gray}{\rmlatinfont\textsuperscript{§~\theparCount}}}
	\pend% ending standard par
      ‚{\tiny $_{lb}$}‚

	  
	  \pstart \leavevmode% starting standard par
	अथ \textbf{सामान्येन} लिङ्ग‚स्य \textbf{व‚च‚ने भिन्न‚विष‚य‚त्व‚स्याप्य‚विरोधः} । अकार‚प्र‚तीतिश्च‚{\tiny $_{lb}$}‚ स्यात् पूर्वाकार‚प्र‚तीतिविष‚याद् भिन्न‚विष‚या चेति को विरोधः ।
	{\color{gray}{\rmlatinfont\textsuperscript{§~\theparCount}}}
	\pend% ending standard par
      ‚{\tiny $_{lb}$}‚

	  
	  \pstart \leavevmode% starting standard par
	अन्ये त्व‚कार‚प्र‚तीतित्वं सामान्यं य‚था त‚योः प्र‚तीत्योरेव‚म‚{\tiny $_{२}$}‚कार‚विष‚य‚त्व‚म‚वि‚{\tiny $_{lb}$}‚रुद्ध‚मिति व्याच‚क्ष‚ते ।
	{\color{gray}{\rmlatinfont\textsuperscript{§~\theparCount}}}
	\pend% ending standard par
      ‚{\tiny $_{lb}$}‚

	  
	  \pstart \leavevmode% starting standard par
	किञ्च । \textbf{एक‚विष‚य‚योश्च प्र‚तीत्योः} पूर्व‚व्य‚व‚स्थितैकाकार‚विष‚य‚योः पूर्वोत्त‚र‚{\tiny $_{lb}$}‚काल‚भाविन्योः प्र‚तीत्योः \textbf{पूर्वाप‚र‚भावः} प्राक् प‚श्चाद्भावे \textbf{विरुध्य}ते । किं कार‚णं‚{\tiny $_{lb}$}‚ [।] \textbf{स‚न्निहितास‚न्निहित‚कार‚ण‚त्वेन} य‚थाक्र‚मं कार्य‚स्यो\textbf{त्पादानुत्पादात्} । स‚न्नि‚{\tiny $_{lb}$}‚हित‚कार‚ण‚त्वे च त‚योर्युग‚प‚द् भावः स्यात् । अथ स‚न्निहितेपि कार‚{\tiny $_{३}$}‚णे पूर्वैवाकार‚{\tiny $_{lb}$}‚प्र‚तीतिरुत्प‚द्य‚ते नोत्त‚रा । त‚दा प‚श्चाद‚पि सा न स्यात् । किं कार‚णं । पूर्वाप‚र‚{\tiny $_{lb}$}‚प्र‚तीतिकार‚ण\textbf{स‚न्निधानेप्य‚नुत्प‚न्न‚स्यो}त्त‚राकार‚प्र‚तीतिविशेष‚स्या\textbf{त‚त्कार‚ण‚त्वात्} । पूर्वा‚{\tiny $_{lb}$}‚कार‚प्र‚तीतिकार‚णं नास्य कार‚ण‚मित्य‚र्थः । त‚स्मात् \textbf{त‚योः} पूर्वाप‚र‚भाविन्योः प्र‚तीत्यो‚{\tiny $_{lb}$}‚\textbf{र्भिन्नाखिल‚कार‚ण‚त्वं} । भिन्न‚म‚खिलं कार‚ण‚न्त‚योरिति विग्र‚हः ।
	{\color{gray}{\rmlatinfont\textsuperscript{§~\theparCount}}}
	\pend% ending standard par
      ‚{\tiny $_{lb}$}‚

	  
	  \pstart \leavevmode% starting standard par
	स्यान्म‚तं [।] त‚यो‚{\tiny $_{४}$}‚र‚कार‚प्र‚तीत्योः श‚ब्द एवैकः कार‚णं केव‚लं स‚ह‚कारि‚{\tiny $_{lb}$}‚स‚न्निधान‚क्र‚मादुत्प‚त्तिक्र‚म इति [।]
	{\color{gray}{\rmlatinfont\textsuperscript{§~\theparCount}}}
	\pend% ending standard par
      ‚{\tiny $_{lb}$}‚

	  
	  \pstart \leavevmode% starting standard par
	अत आह । त‚त्रेत्यादि । \textbf{त‚त्र} त‚स्मिन् पूर्वोत्त‚राकार‚प्र‚तीत्युत्प‚त्तिकाले । एक‚स्य‚{\tiny $_{lb}$}‚ कार‚ण‚स्य स्व‚रूपेणा\textbf{भेदेपि} प्र‚तीत्योर्युग‚प‚द् भाव एव स्यात् । किं कार‚णं [।]‚{\tiny $_{lb}$}‚ ‚{\tiny $_{lb}$}‚ \leavevmode\ledsidenote{\textenglish{497/s}}त‚स्यैक‚स्य \textbf{श‚क्त‚स्य} कार‚ण‚स्य स‚ह‚कार्य\textbf{प्र‚तीक्ष‚णात्} । त‚त‚श्च \textbf{युक्तिविरुद्धं पूर्वाप‚र‚योः‚{\tiny $_{lb}$}‚ प्र‚तीत्योरेक‚{\tiny $_{५}$}‚विष‚य‚त्वं} ।
	{\color{gray}{\rmlatinfont\textsuperscript{§~\theparCount}}}
	\pend% ending standard par
      ‚{\tiny $_{lb}$}‚

	  
	  \pstart \leavevmode% starting standard par
	एतेन च स‚र्वेणोत्त‚राकार‚प्र‚तीते पूर्वाकार‚प्र‚तीत्य‚भिन्न‚विष‚य‚त्वे साध्येनुमान‚{\tiny $_{lb}$}‚बाधित‚त्वं प्र‚तिज्ञाया उक्तं । अनुमान‚न्त्वीदृशं । य‚त् क्र‚म‚भावि त‚न्नैक‚विष‚यं ।‚{\tiny $_{lb}$}‚ य‚था क्र‚मेण भ‚व‚च्च‚क्षुःश्रोत्र‚विज्ञानं [।] क्र‚म‚भाविन्यौ च पूर्वोत्त‚रे अकार‚प्र‚तीती ।‚{\tiny $_{lb}$}‚ एक‚विष‚य‚त्व‚म‚क्र‚म‚भावित्वेन व्याप्त‚न्त‚द्विरुद्धं च क्र‚म‚भावित्व‚मिति व्याप‚क‚विरु‚{\tiny $_{lb}$}‚द्ध‚मेव ।
	{\color{gray}{\rmlatinfont\textsuperscript{§~\theparCount}}}
	\pend% ending standard par
      ‚{\tiny $_{lb}$}‚

	  
	  \pstart \leavevmode% starting standard par
	न‚न्व‚{\tiny $_{६}$}‚त्र प्र‚त्य‚क्ष‚प्र‚त्य‚भिज्ञाबाधित‚त्वात् प्र‚तिज्ञाया अनुमान‚स्योत्थान‚मेव नास्तीति‚{\tiny $_{lb}$}‚ चेत् [।]
	{\color{gray}{\rmlatinfont\textsuperscript{§~\theparCount}}}
	\pend% ending standard par
      ‚{\tiny $_{lb}$}‚

	  
	  \pstart \leavevmode% starting standard par
	न । स एवाय‚मिति ज्ञान‚स्य पूर्वाप‚र‚काल‚स‚म्ब‚न्धिविष‚य‚त्वेन भेद‚विष‚य‚{\tiny $_{lb}$}‚त्वात् । अन्य‚देव हि पूर्व‚काल‚स‚म्ब‚न्धित्व‚म‚न्य‚देव चाप‚र‚काल‚स‚म्ब‚न्धित्वं । अन्य‚था‚{\tiny $_{lb}$}‚ पूर्व‚काल‚स‚म्ब‚न्धित्वाद्वाऽप‚र‚काल‚स‚म्ब‚न्धित्व‚स्याभेदेधुना भावाद् भाव‚स्य प्र‚तिभासो‚{\tiny $_{lb}$}‚ न स्यात् । स एवेति च‚{\tiny $_{७}$}‚ ज्ञान‚स्योत्प‚त्तिः स्यात् [।] न स एवाय‚मिति । अप‚र- \leavevmode\ledsidenote{\textenglish{176b/PSVTa}}‚{\tiny $_{lb}$}‚ काल‚स‚म्ब‚न्धित्वाद्वा पूर्व‚काल‚स‚म्ब‚न्धित्व‚स्याभेदे पूर्व‚म‚स्य प्र‚तिभासो न स्याद् [।]‚{\tiny $_{lb}$}‚ अय‚मेवेति च ज्ञान‚स्योत्प‚त्तिः स्यात् [।] न स एवाय‚मिति ।
	{\color{gray}{\rmlatinfont\textsuperscript{§~\theparCount}}}
	\pend% ending standard par
      ‚{\tiny $_{lb}$}‚

	  
	  \pstart \leavevmode% starting standard par
	त‚स्माद् य‚त्पूर्व‚काल‚स‚म्ब‚न्धित्व‚न्त‚द‚प‚र‚काल‚स‚म्ब‚न्धित्व‚न्न भ‚व‚ति । य‚च्चाप‚र‚{\tiny $_{lb}$}‚काल‚स‚म्ब‚न्धित्व‚न्त‚त्पूर्व‚काल‚स‚म्ब‚न्धित्वं न भ‚व‚तीति पूर्वाप‚र‚काल‚स‚म्ब‚न्धिविष‚य‚त्वेन‚{\tiny $_{lb}$}‚ भेद‚विष‚य‚त्वात् क‚{\tiny $_{१}$}‚थ‚म्प्र‚त्य‚भिज्ञातः प्र‚तिज्ञाबाधा ।
	{\color{gray}{\rmlatinfont\textsuperscript{§~\theparCount}}}
	\pend% ending standard par
      ‚{\tiny $_{lb}$}‚

	  
	  \pstart \leavevmode% starting standard par
	\hphantom{.}उ म्बे क स्त्वाह । य‚दि स एवाय‚मित्येकानुभ‚व‚स्त‚थाप्य‚य‚म‚तीत‚ज्ञान‚{\tiny $_{lb}$}‚क‚र्म‚ताऽप‚रोक्ष‚ते एकाधिक‚र‚णे गृह्ण‚न् स‚म्वेद्य‚ते । अथापि प्र‚त्य‚य‚द्व‚य‚मिदं ग्र‚ह‚ण‚{\tiny $_{lb}$}‚स्म‚र‚ण‚रूपं । त‚थापि घ‚ट‚स्म‚र‚ण‚प‚ट‚ग्र‚ह‚ण‚योर्निर‚न्त‚रोत्प‚न्न‚योर्विल‚क्ष‚ण‚मिद‚म्प‚र‚स्प‚र‚{\tiny $_{lb}$}‚विष‚य‚त्वेन प्र‚तिभास‚नात् । अप‚रोक्ष एव ह्य‚र्थोतीत‚ज्ञान‚विशिष्ट‚त‚{\tiny $_{२}$}‚या स्मृतौ‚{\tiny $_{lb}$}‚ प्र‚तिभास‚ते । अतीत‚ज्ञान‚विष‚य‚श्चाप‚रोक्ष‚त‚या प्र‚त्य‚क्षे । त‚द‚हं स्म‚राम्येत‚दिति‚{\tiny $_{lb}$}‚ प्र‚तिभास‚नात् । त‚स्माद‚निमिषितदृष्टेः पुरुष‚स्य य‚दुत्प‚त्तिविनाश‚र‚हितानुवृत्ता‚{\tiny $_{lb}$}‚व‚सायः स एव बाध‚कः क्ष‚ण‚भ‚ङ्ग‚साध‚क‚स्यानुमान‚स्येति ।
	{\color{gray}{\rmlatinfont\textsuperscript{§~\theparCount}}}
	\pend% ending standard par
      ‚{\tiny $_{lb}$}‚

	  
	  \pstart \leavevmode% starting standard par
	त‚द‚युक्त‚म् [।] उत्त‚रोत्त‚र‚प्र‚त्य‚क्षाणां य‚थाक्र‚म‚मुत्त‚रोत्त‚र‚व‚स्त्व‚व‚स्थाभेद‚विष‚{\tiny $_{lb}$}‚य‚त्वेन स एवाय‚मिति त‚त्त्वारोप‚{\tiny $_{३}$}‚स्य भ्रान्त‚त्वात् । त‚था हि प्र‚थ‚म‚द‚र्शिनः प्र‚त्य‚क्षे‚{\tiny $_{lb}$}‚ य‚थाऽप‚रोक्षाव्र‚स्था प्र‚तिभास‚ते नातीत‚ज्ञान‚विष‚याव‚स्था । त‚था भूयो द‚र्शिनोपि ।‚{\tiny $_{lb}$}‚ इदानीन्त‚नेन च रूपेण व‚स्त्व‚व‚स्थित‚न्न प्राक्त‚नेन । अव‚स्थिते च रूपे प्राक्त‚न‚रूप‚{\tiny $_{lb}$}‚स्यान‚व‚स्थान‚मेव विनाशः । य‚था वृद्धाव‚स्थायाम्बाल‚रूप‚स्य प्राक्त‚न‚ञ्च रूप‚{\tiny $_{lb}$}‚म‚तीत‚ज्ञान‚क‚र्म । इदानीन्त‚नं च रूप‚म‚{\tiny $_{४}$}‚प‚रोक्ष‚म‚थ च बालाद्य‚व‚स्थायां दृष्टः‚{\tiny $_{lb}$}‚ ‚{\tiny $_{lb}$}‚ \leavevmode\ledsidenote{\textenglish{498/s}}पुरुषो वृद्धाद्य‚व‚स्थायां प्र‚त्य‚भिज्ञाय‚त इति क‚थ‚म‚तीत‚ज्ञान‚क‚र्म‚ताऽप‚रोक्ष‚ते एका‚{\tiny $_{lb}$}‚धिक‚र‚णे प्र‚तिभासेते । क‚थं वाऽप‚रोक्ष एवार्थोतीत‚ज्ञान‚विशिष्ट‚त‚या स्मृतौ प्र‚ति‚{\tiny $_{lb}$}‚भास‚त इत्याद्युच्य‚ते ।
	{\color{gray}{\rmlatinfont\textsuperscript{§~\theparCount}}}
	\pend% ending standard par
      ‚{\tiny $_{lb}$}‚

	  
	  \pstart \leavevmode% starting standard par
	य‚त्राप्य‚निमिष‚दृष्टेश्चिर‚त‚र‚कालं प‚श्य‚तोनुवृत्ताव‚साय‚स्त‚त्रापीदानीन्त‚न‚प्र‚त्य‚क्ष‚{\tiny $_{lb}$}‚ज्ञान‚स‚म्ब‚न्धेनार्थ‚{\tiny $_{५}$}‚स्याप‚रोक्ष‚तोत्प‚द्य‚ते [।] अतीत‚ज्ञानाभावेनातीत‚ज्ञान‚क‚र्म्म‚ताया‚{\tiny $_{lb}$}‚श्चेदानीम‚भाव एव विनाश इति क‚थ‚मुच्य‚ते [।] उत्प‚त्तिविनाश‚र‚हितानुवृत्ता‚{\tiny $_{lb}$}‚व‚साय एव बाध‚कः क्ष‚णिक‚त्वानुमान‚स्येति ।
	{\color{gray}{\rmlatinfont\textsuperscript{§~\theparCount}}}
	\pend% ending standard par
      ‚{\tiny $_{lb}$}‚

	  
	  \pstart \leavevmode% starting standard par
	य‚द‚प्युच्य‚ते [।] यः प्र‚तिक्ष‚ण‚म‚न्य‚त्व‚म्व‚द‚ति त‚स्य चाय‚म्बाधः प्र‚त्य‚भिज्ञान‚{\tiny $_{lb}$}‚मात्रेणान‚न्य‚त्त्वे तु विन‚ष्ट‚स्यापि त‚त्त्वाव‚ग‚मात् । मृत‚प्र‚त्य‚भिज्ञायामि‚{\tiny $_{६}$}‚वेति [।]
	{\color{gray}{\rmlatinfont\textsuperscript{§~\theparCount}}}
	\pend% ending standard par
      ‚{\tiny $_{lb}$}‚

	  
	  \pstart \leavevmode% starting standard par
	त‚द‚पि निर‚स्तं । अन‚न्य‚त्व‚स्यैवाभावात् । नापि विन‚ष्टाविन‚ष्ट‚योर‚न‚न्य‚त्वं‚{\tiny $_{lb}$}‚ विरोधात् । न च त‚त्त्वाव‚ग‚मान्य‚थानुप‚प‚त्यान‚न्य‚त्वं सादृश्येनापि त‚त्त्वाव‚ग‚म‚स्य‚{\tiny $_{lb}$}‚ स‚म्भ‚वात् । स इत्य‚ङ्श‚श्च न प्र‚त्य‚क्षोऽस‚न्निहित‚विष‚य‚त्वात् । स्म‚र‚ण‚रूप‚त्वे‚{\tiny $_{lb}$}‚ चास्य न पूर्व‚दृष्टार्थ‚ग्राहित्वं स्प‚ष्ट‚प्र‚तिभासाभावात् । दृष्टार्थाध्य‚व‚साय‚क‚त्वेन‚{\tiny $_{lb}$}‚ \leavevmode\ledsidenote{\textenglish{177a/PSVTa}} तु स्मृति‚{\tiny $_{७}$}‚रूप‚त्वे भ्रान्त‚त्वं [।] स्वआकाराभेदेन दृष्टार्थाध्य‚व‚सायात् । अय‚मिति‚{\tiny $_{lb}$}‚ चांशः प्र‚त्य‚क्ष इष्य‚ते [।] स्म‚र‚ण‚प्र‚त्य‚क्ष‚योश्चैक‚त्व‚म्विरुध्य‚ते । त‚स्मात् पूर्व‚विज्ञा‚{\tiny $_{lb}$}‚न‚विष‚य‚त्व‚र‚हिते पुरोव‚स्थितेर्थे सादृश्येन पूर्व‚ज्ञान‚विष‚य‚त्व‚मारोप्य स एवाय‚मिति‚{\tiny $_{lb}$}‚ मान‚सं ज्ञानं गृह्णाति । आरोप‚ब‚लेन चातीत‚ज्ञान‚क‚र्म‚ताऽप‚रोक्ष‚ते एकाधिक‚र‚णे‚{\tiny $_{lb}$}‚ प्र‚तिभासेते । म‚रीचिकायां ज‚ल‚प्र‚त्य‚भि‚{\tiny $_{१}$}‚ज्ञान इव । आरोपाभावे त्वेते भिन्नाधि‚{\tiny $_{lb}$}‚क‚र‚णे एव प्र‚तिभासेते । ज‚ल‚स्म‚र‚ण‚म‚रीचिकाग्र‚ह‚ण‚योरिव ।
	{\color{gray}{\rmlatinfont\textsuperscript{§~\theparCount}}}
	\pend% ending standard par
      ‚{\tiny $_{lb}$}‚

	  
	  \pstart \leavevmode% starting standard par
	त‚स्मात् स्थित‚मेत‚द् [।] भ्रान्त‚त्वाद‚प्र‚त्य‚क्ष‚त्वाच्च न प्र‚त्य‚भिज्ञातः क्ष‚णि‚{\tiny $_{lb}$}‚क‚त्वानुमान‚बाधेति । तेन पूर्वोत्त‚रे अकार‚प्र‚तीती भिन्न‚विष‚ये एव । त‚था द्रुत‚{\tiny $_{lb}$}‚म‚ध्य‚बिल‚म्बितानाङ्ग‚कारादिप्र‚तिप‚त्तीनां भिन्न‚विष‚य‚त्वं । द्रुतादिभेद‚भिन्न‚ग‚कारा‚{\tiny $_{lb}$}‚ल‚म्ब‚न‚त्वा‚{\tiny $_{२}$}‚त् । ग‚कार एव द्रुतो ग‚कार एव विल‚म्वितं इति ग‚कारैक‚त्व‚प्र‚तीतिस्तु‚{\tiny $_{lb}$}‚ सादृश्य‚निमित्तैव ।
	{\color{gray}{\rmlatinfont\textsuperscript{§~\theparCount}}}
	\pend% ending standard par
      ‚{\tiny $_{lb}$}‚

	  
	  \pstart \leavevmode% starting standard par
	तेन य‚दुच्य‚ते ।
	{\color{gray}{\rmlatinfont\textsuperscript{§~\theparCount}}}
	\pend% ending standard par
      ‚{\tiny $_{lb}$}‚
	  \bigskip
	  \begingroup
	
	    
	    \stanza[\smallbreak]
	  {\normalfontlatin\large ``\qquad}न हि द्रुतादिभेदेपि निष्प‚न्ना संप्र‚तीय‚ते ।&‚{\tiny $_{lb}$}‚ग‚व्य‚क्त्य‚न्त‚र‚विच्छिन्ना ग‚व्य‚क्तिर‚प‚रा स्फुटा ।&‚{\tiny $_{lb}$}‚तेनैक‚त्वेन व‚र्ण्ण‚स्य बुद्धिरेकोप‚जाय‚ते ।&‚{\tiny $_{lb}$}‚विशेष‚बुद्धिस‚द्भावो भ‚वेद् व्य‚ञ्ज‚क‚भेद‚त इति [।]\edtext{}{\edlabel{pvsvt_498-1}\label{pvsvt_498-1}\lemma{इति}\Bfootnote{Kumārila. }}{\normalfontlatin\large\qquad{}"}\&[\smallbreak]
	  
	  
	  
	  \endgroup
	‚{\tiny $_{lb}$}‚‚{\tiny $_{lb}$}‚‚{\tiny $_{lb}$}‚\textsuperscript{\textenglish{499/s}}

	  
	  \pstart \leavevmode% starting standard par
	त‚द‚पास्तं । य‚तो ध्व‚निविशेष एव व‚र्ण्ण उच्य‚ते । तेन द्रुतो‚{\tiny $_{३}$}‚च्चारिता ध्व‚नि‚{\tiny $_{lb}$}‚विशेषा द्रुता ग‚व्य‚क्तिरुच्य‚ते । म‚ध्योच्चारिता म‚ध्य‚ग‚व्य‚क्तिः [।] विल‚म्बितो‚{\tiny $_{lb}$}‚च्चारिता ध्व‚निविशेषा विल‚म्विता ग‚व्य‚क्तिः [।] न तु व्य‚ञ्जेकेभ्यो ध्व‚निभ्यो‚{\tiny $_{lb}$}‚न्यो ग‚कारः प्र‚तिभास‚ते [।] ग‚कारो ग‚कार इति तेषु नाम‚साम्य‚मेव केव‚लं प्र‚ती‚{\tiny $_{lb}$}‚य‚ते । त‚था ह्र‚स्व‚दीर्घ‚प्लुतादिषु नैकाकारः । य‚तो ध्व‚निविशेषा एव मात्रा‚{\tiny $_{lb}$}‚कालं प्र‚युज्य‚माना ह्र‚स्वोका‚{\tiny $_{४}$}‚रो भ‚व‚ति । त‚थाप‚रे ध्व‚निविशेषा द्विमात्राकालं‚{\tiny $_{lb}$}‚ प्र‚युज्य‚माना दीर्घ आकारो भ‚व‚ति [।] त्रिमात्राकालं प्र‚युज्य‚माना ध्व‚निविशेषाः‚{\tiny $_{lb}$}‚ प्लुतो भ‚व‚ति । तेन ह्र‚स्व‚दीघ्र‚प्लुतानां स्व‚भाव‚भेद एव प्र‚तिभास‚ते । न त्व‚कारो‚{\tiny $_{lb}$}‚ऽभिन्न‚स्तेषु प्र‚तिभास‚ते । अकार एव तु मात्रादिकाल‚मुच्चार्य‚माणो य‚थाक्र‚मं‚{\tiny $_{lb}$}‚ ह्र‚स्व‚दीर्घ‚प्लुताः प्र‚तीय‚न्त इति श‚ब्द‚मात्र‚मेव‚{\tiny $_{५}$}‚ केव‚लं । तेन य‚दुच्य‚ते ।
	{\color{gray}{\rmlatinfont\textsuperscript{§~\theparCount}}}
	\pend% ending standard par
      ‚{\tiny $_{lb}$}‚
	  \bigskip
	  \begingroup
	
	    
	    \stanza[\smallbreak]
	  {\normalfontlatin\large ``\qquad}स्व‚तो ह्र‚स्वादिभेद‚स्तु नित्य‚वादे विरुध्य‚ते ।&‚{\tiny $_{lb}$}‚स‚र्व‚दा य‚स्य स‚द्भावः स क‚थं मात्रिकः स्व‚यं ।&‚{\tiny $_{lb}$}‚त‚स्मादुच्चार‚ण‚न्त‚स्य मात्राकालं प्र‚तीय‚तां ।&‚{\tiny $_{lb}$}‚द्विमात्र‚म्वा त्रिमात्र‚म्वा न श‚ब्दो मात्रिकः स्व‚य‚मिति\edtext{}{\edlabel{pvsvt_499-1}\label{pvsvt_499-1}\lemma{मिति}\Bfootnote{\href{http://sarit.indology.info/?cref=\%C5\%9Bv-spho\%E1\%B9\%ADa.50-51}{ Śloka-Sphoṭavāda 50, 51 }}} [।]{\normalfontlatin\large\qquad{}"}\&[\smallbreak]
	  
	  
	  
	  \endgroup
	‚{\tiny $_{lb}$}‚

	  
	  \pstart \leavevmode% starting standard par
	त‚द‚पि निर‚स्तं । ह्र‚स्व‚दीर्घ‚प्लुतेष्व‚कारोकार इत्य‚नुयायिनोर्ज्ञानाभिधान‚यो‚{\tiny $_{lb}$}‚र‚प्र‚वृत्तेः । अथापि स्यात् [।] पूर्वोत्त‚र‚काल‚{\tiny $_{६}$}‚भाविन्योः प्र‚तीत्योर्नाम‚साम्यादेक‚विष‚{\tiny $_{lb}$}‚य‚त्व‚मिति [।]
	{\color{gray}{\rmlatinfont\textsuperscript{§~\theparCount}}}
	\pend% ending standard par
      ‚{\tiny $_{lb}$}‚

	  
	  \pstart \leavevmode% starting standard par
	अत आह । \textbf{प्र‚तीत्या}दि । पूर्वोत्त‚र‚योर‚कार‚प्र‚तीत्योः प्र‚तिभास‚भेदः पूर्वोत्त‚र‚रूप‚{\tiny $_{lb}$}‚त‚या । स्व‚भाव‚भेदो द्रुत‚म‚ध्य‚विल‚म्वितादिभेदेन । त‚स्मिन् \textbf{प्र‚तीतिप्र‚तिभास‚स्व‚भाव‚भे‚{\tiny $_{lb}$}‚देपि} । अकार‚प्र‚तीतिर‚कार‚प्र‚तीतिरित्येवं \textbf{नाम‚साम्यादेक‚विष‚य‚त्व‚म‚युक्तं} । किं कार‚णं‚{\tiny $_{lb}$}‚ [।] \textbf{घ‚टादिष्व‚पि‚{\tiny $_{७}$}‚ प्र‚स‚ङ्गात्} । या पूर्वा घ‚ट‚प्र‚तीतिर्या च प‚श्चाद् अन्य‚घ‚ट‚प्र‚तीतिस्त- \leavevmode\ledsidenote{\textenglish{177b/PSVTa}}‚{\tiny $_{lb}$}‚ योर‚पि घ‚ट‚प्र‚तीतिर्घ‚ट‚प्र‚तीतिरिति नाम‚साम्यादेक‚विष‚य‚त्वं स्यात् [।] त‚था चैको‚{\tiny $_{lb}$}‚ घ‚टः स‚र्व‚त्र प्राप्नोति । त‚त्र घ‚टादावेक‚त्व‚साध‚ने दृष्ट‚विरोधो घ‚टादीनाम‚नेक‚त्व‚स्य‚{\tiny $_{lb}$}‚ दृष्ट‚त्वात् । त‚स्मात् त‚त्रा\textbf{साध‚न‚मेक‚त्व‚स्}येति \textbf{चेत् । इहापि} व‚र्ण्णेष्व‚प्येक‚त्व‚साध‚ने‚{\tiny $_{lb}$}‚ दृष्ट\textbf{विरोधाभावः केन} प्र‚माणेन \textbf{सिद्धः} । अत्रापि क‚{\tiny $_{१}$}‚र‚णानां प्र‚तिपुरुषं भेदेन भेदः‚{\tiny $_{lb}$}‚ सिद्ध एव [।] लून‚पुन‚र्जातेषु केशेष्विव सादृश्यादेक‚त्वाध्य‚व‚साय इति याव‚त् ।
	{\color{gray}{\rmlatinfont\textsuperscript{§~\theparCount}}}
	\pend% ending standard par
      ‚{\tiny $_{lb}$}‚‚{\tiny $_{lb}$}‚‚{\tiny $_{lb}$}‚\textsuperscript{\textenglish{500/s}}

	  
	  \pstart \leavevmode% starting standard par
	नाम‚साम्यादित्य‚यं हेतुर‚नैकान्तिक इत्याह । \textbf{याव‚दि}त्यादि । \textbf{त‚थाभिधे}य‚तेति‚{\tiny $_{lb}$}‚ अकार‚प्र‚तीत्ये\edtext{}{\lemma{तीत्ये}\Bfootnote{? ति}}र‚कार‚प्र‚तीतिरित्येवं नाम‚साम्येनाभिधेय‚ता । अर्थाभेदेन‚{\tiny $_{lb}$}‚ विष‚यैक‚त्वेन \textbf{व्याप्त्या न साध्य‚ते ताव‚त् स‚न्दिग्धो व्य‚तिरेकः} । नाम‚साम्यं च स्या‚{\tiny $_{२}$}‚द्‚{\tiny $_{lb}$}‚ भेद‚श्चेति । किञ्च । \textbf{प्र‚तिक‚र‚ण‚भेदं} पुरुष‚भेदेन क‚र‚ण‚भेदं प्र‚ति \textbf{भिन्न‚स्व‚भावः श‚ब्दः‚{\tiny $_{lb}$}‚ श्रुतौ} श्रोत्र‚विज्ञाने \textbf{निविश‚मानः} स‚मारोह‚न् \textbf{य‚दैकः साध्य‚ते किन्न} घ‚टाद‚योप्येक‚{\tiny $_{lb}$}‚रूपास्साध्य‚न्ते । तेपि साध्य‚न्तां । विशेषोपि वा वाच्यः ।
	{\color{gray}{\rmlatinfont\textsuperscript{§~\theparCount}}}
	\pend% ending standard par
      ‚{\tiny $_{lb}$}‚

	  
	  \pstart \leavevmode% starting standard par
	एक‚त्वेपि श‚ब्द‚स्य व्य‚ञ्ज‚क‚भेदात् प्र‚तिभास‚भेद इति चेदाह । \textbf{त‚त्रापी}‚{\tiny $_{lb}$}‚त्यादि । \textbf{त‚त्रापि} भिन्ने घ‚टादौ \textbf{श‚क्य‚मेवं व्य‚ञ्ज‚क‚भे‚{\tiny $_{३}$}‚दात् प्र‚तिभास‚भेद इति‚{\tiny $_{lb}$}‚ प्र‚त्य‚व‚स्थातुं} [।] \href{http://sarit.indology.info/?cref=pv.3.263}{२६६}
	{\color{gray}{\rmlatinfont\textsuperscript{§~\theparCount}}}
	\pend% ending standard par
      ‚{\tiny $_{lb}$}‚

	  
	  \pstart \leavevmode% starting standard par
	किञ्च [।] \textbf{कार‚णानां स‚म‚ग्राणां व्यापारात्} प‚रिस्प‚न्दादिल‚क्ष‚णा\textbf{न्निय‚मेन}‚{\tiny $_{lb}$}‚ श‚ब्द‚स्यैवउप\textbf{ल‚ब्धितः} कार‚णात् \textbf{कार्य‚त्व}म्प्राप्तं । किं कार‚णं [।] \textbf{व्य‚ञ्ज‚के} हेतौ‚{\tiny $_{lb}$}‚ \textbf{त‚द‚स‚म्भ‚वात्} । निय‚मेन व्य‚ङ्ग्य‚स्योप‚ल‚म्भास‚म्भ‚वात् ।
	{\color{gray}{\rmlatinfont\textsuperscript{§~\theparCount}}}
	\pend% ending standard par
      ‚{\tiny $_{lb}$}‚

	  
	  \pstart \leavevmode% starting standard par
	\textbf{न ही}त्यादिना व्याच‚ष्टे । \textbf{व्यापृतेषु क‚र‚णेषु} न हि \textbf{क‚दाचिच्छ‚ब्दानुप‚ल‚ब्धिः}‚{\tiny $_{lb}$}‚ किन्तूप‚ल‚ब्धिरेव । \textbf{न चाव‚श्यं व्य‚ञ्ज‚क‚व्यापारो‚{\tiny $_{४}$}‚र्थ‚मुप‚ल‚म्भ‚य‚ति} ग्राह‚य‚ति । किं‚{\tiny $_{lb}$}‚ कार‚णं । \textbf{क्व‚चिद्} घ‚टादिशून्ये देशे \textbf{प्र‚काशे} प्र‚दीपादिल‚क्ष‚णे स‚त्\textbf{य‚पि घ‚टाद्य‚नुप‚ल‚ब्धेः} ।‚{\tiny $_{lb}$}‚ त‚स्माद‚विक‚ल‚विज्ञानोत्पाद‚स‚ह‚कारिकार‚ण‚स्य पुंसः । \textbf{सेयं} श‚ब्द‚स्य त‚द्व्यापारात्‚{\tiny $_{lb}$}‚ कार‚ण‚व्यापारा\textbf{न्निय‚मेनोप‚ल‚ब्धिस्त‚दुद्भ‚वे} । क‚र‚ण‚व्यापारा\textbf{च्छ‚ब्द}स्योत्प‚त्तौ स‚त्यां‚{\tiny $_{lb}$}‚ \textbf{स्यात्} । त‚त‚श्च ज‚न्य एव श‚ब्दो न व्य‚ङ्ग्यः ।
	{\color{gray}{\rmlatinfont\textsuperscript{§~\theparCount}}}
	\pend% ending standard par
      ‚{\tiny $_{lb}$}‚

	  
	  \pstart \leavevmode% starting standard par
	न‚नु पूर्वं ज‚न‚न‚{\tiny $_{५}$}‚मात्रेण कार‚कं ज्ञान‚ज‚न‚न‚योग्य‚त्वेनोत्पाद‚क‚न्तु व्य‚ञ्ज‚क‚मेवे‚{\tiny $_{lb}$}‚त्युक्तं । तेन ताल्वादीनां व्य‚ञ्ज‚क‚त्व‚मेव युक्तं ।
	{\color{gray}{\rmlatinfont\textsuperscript{§~\theparCount}}}
	\pend% ending standard par
      ‚{\tiny $_{lb}$}‚

	  
	  \pstart \leavevmode% starting standard par
	नैष दोषो य‚तः [।] कार्य‚मात्र‚म‚भिप्रेत्य ज‚न‚न‚मात्रेण कार‚कं [।] ज्ञान‚ज‚न‚न‚{\tiny $_{lb}$}‚योग्य‚त्वेन तु व्य‚ञ्ज‚क इत्युक्तं । न तु दृश्य‚कार्यापेक्ष‚या । त‚था ह्य‚विक‚ल‚स‚ह‚कारि‚{\tiny $_{lb}$}‚कार‚ण‚स्य पुंसः प्र‚दीपादिज‚न‚को निय‚मेन प्र‚दीपादेरुप‚ल‚म्भ‚कः कार‚{\tiny $_{६}$}‚को न‚{\tiny $_{lb}$}‚ व्य‚ञ्ज‚क इत्य‚दोषः ।
	{\color{gray}{\rmlatinfont\textsuperscript{§~\theparCount}}}
	\pend% ending standard par
      ‚{\tiny $_{lb}$}‚‚{\tiny $_{lb}$}‚\textsuperscript{\textenglish{501/s}}

	  
	  \pstart \leavevmode% starting standard par
	अथ पुनः क‚र‚णंश‚ब्द‚स्याक‚र्त्तृ । त‚स्या\textbf{क‚र्तुः} क‚र‚ण‚स्य \textbf{व्यापारेण त‚त्सिद्ध्य‚योगात्} ।‚{\tiny $_{lb}$}‚ श‚ब्द‚स्य सिद्ध्य‚योगात् । व्यापिनः श‚ब्दा नित्याश्च । त‚तो \textbf{व्यापिनित्य‚त्वा}च्छ‚ब्दानां ।‚{\tiny $_{lb}$}‚ व्य‚ञ्ज‚क‚स्य क‚र‚ण‚स्य व्यापारात् स‚र्व‚त्रोप‚ल‚ब्धिः । घ‚टाद‚य‚स्तु न व्यापिनो नापि‚{\tiny $_{lb}$}‚ नित्याः । तेन ते व्य‚ञ्ज‚क‚व्यापारेण नाव‚श्य‚मुप‚ल‚भ्य‚न्त इति ।
	{\color{gray}{\rmlatinfont\textsuperscript{§~\theparCount}}}
	\pend% ending standard par
      ‚{\tiny $_{lb}$}‚

	  
	  \pstart \leavevmode% starting standard par
	य‚{\tiny $_{७}$}‚द्येवं \textbf{क इदानीं घ‚टादिषु स‚माश्वासः} । निश्च‚यः । य‚था ते न नित्या नापि \leavevmode\ledsidenote{\textenglish{178a/PSVTa}}‚{\tiny $_{lb}$}‚ व्यापिन इति । याव‚ता तेपि नित्या व्यापिन‚श्च भ‚व‚न्तु । क‚थं स‚र्व‚दा नोप‚ल‚भ्य‚न्त‚{\tiny $_{lb}$}‚ इति चेत् । एत‚च्छ‚ब्देष्व‚पि तुल्यं । य‚त्त‚त्र प्र‚तिविधानं त‚द् घ‚टादिष्व‚पि भ‚विष्य‚ति ।‚{\tiny $_{lb}$}‚ \textbf{तेषां} घ‚टादीना\textbf{न्त‚था} व्यापिनित्य‚त्वेना\textbf{निष्टेरिति चेत् । श‚ब्दो} व्यापिनित्य‚त्वेन‚{\tiny $_{lb}$}‚ \textbf{किमिष्टः} [।] क‚स्मादिष्ट\textbf{स्त‚त्स‚मान‚ध‚र्मा} ।‚{\tiny $_{१}$}‚ घ‚टादिस‚मान‚ध‚र्मा । \textbf{प्र‚तिषिद्धे च व्या‚{\tiny $_{lb}$}‚पिनित्य‚त्वे} प्रागिति य‚त्किञ्चिदेत‚त् ।
	{\color{gray}{\rmlatinfont\textsuperscript{§~\theparCount}}}
	\pend% ending standard par
      ‚{\tiny $_{lb}$}‚

	  
	  \pstart \leavevmode% starting standard par
	\textbf{घ‚टादीना}मित्यादि प‚रः । कार‚क‚व्य‚तिरेकेण \textbf{व्य‚ञ्ज‚कान्त‚र‚स‚द्भावाद‚दोषः} ।‚{\tiny $_{lb}$}‚ श‚ब्देन तुल्य‚त्व‚प्र‚स‚ङ्ग‚दोषो नास्ति । व्य‚ञ्ज‚कान्त‚र‚मेव द‚र्श‚य‚न्नाह । \textbf{प्र‚काशो‚{\tiny $_{lb}$}‚ ही}त्यादि । \textbf{प्र‚काशो ह्येषां} घ‚टादीनां \textbf{व्य‚ञ्ज‚को} लोके \textbf{सिद्धो} न कुलालाद‚यः । \textbf{कुला‚{\tiny $_{lb}$}‚लादीनां व्य‚ञ्ज‚क‚त्वे} । ते कुलालाद‚य‚{\tiny $_{२}$}‚\textbf{स्तादृशा एव} स्युः [।] य‚था प्र‚दीपाद‚यो न‚{\tiny $_{lb}$}‚ निय‚मेन घ‚ट‚मुप‚ल‚म्भ‚य‚न्ति । क्व‚चित् प्र‚काशेपि घ‚ट‚स्याभावात् । त‚था कुलाला‚{\tiny $_{lb}$}‚द‚योपि भ‚वेयुः [।] न चैव‚म् [।] अ\textbf{तिशेर‚ते च} कुलालाद‚यः । कुलालादिव्यापारे‚{\tiny $_{lb}$}‚ स‚र्व‚दा घ‚टादेर्भावात् । \textbf{त‚तो व्य‚ञ्ज‚कातिश‚यात्} । व्य‚ञ्ज‚काद् भेदेन वृत्तेः । \textbf{कार‚{\tiny $_{lb}$}‚का एव} कुलालाद‚यः । किं कार‚ण‚म् [।] \textbf{उप‚कार‚क‚स्य ग‚त्य‚न्त‚राभावात्} । कार‚{\tiny $_{lb}$}‚क‚{\tiny $_{३}$}‚व्य‚ञ्ज‚क‚त्व‚व्य‚तिरेकेण प्र‚कारान्त‚राभावात् । त‚त्र व्य‚ञ्ज‚क‚त्वे निषिद्धे पारि‚{\tiny $_{lb}$}‚शेष्यात् कार‚क‚त्वं कुलालादीनां [।] नैवं श‚ब्द‚स्य क‚र‚ण‚मुक्त्वान्य‚द् व्य‚ञ्ज‚कान्त‚रं‚{\tiny $_{lb}$}‚ सिद्धं येन क‚र‚ण‚मेव श‚ब्द‚स्य कार‚कं क‚ल्प्येत । त‚स्माद् घ‚टादिवैल‚क्ष‚ण्याच्छ‚ब्दो‚{\tiny $_{lb}$}‚ व्य‚ङ्ग्य एव ।
	{\color{gray}{\rmlatinfont\textsuperscript{§~\theparCount}}}
	\pend% ending standard par
      ‚{\tiny $_{lb}$}‚

	  
	  \pstart \leavevmode% starting standard par
	\textbf{त‚दि}त्यादि सिद्धान्त‚वादी । \textbf{त‚देत‚द्} व्य‚ञ्ज‚कान्त‚र‚स‚म्भ‚व‚नं \textbf{श‚ब्देपि तुल्यं} ।‚{\tiny $_{lb}$}‚ ‚{\tiny $_{lb}$}‚ \leavevmode\ledsidenote{\textenglish{502/s}}य‚स्मात् \textbf{त‚त्रापि} श‚ब्दे \textbf{इन्द्रिय‚योग्य‚दे‚{\tiny $_{४}$}‚श‚तादिभ्यः} श्रोत्रेन्द्रियाच्छ्रोत्र‚योग्य‚देशाव‚स्था‚{\tiny $_{lb}$}‚नात् । आदिश‚ब्दात् म‚न‚स्काराच्च । \textbf{क‚र‚णानाम‚तिश‚यात्} । अतिश‚य एव क‚थ‚मिति‚{\tiny $_{lb}$}‚ चेदाह । \textbf{घ‚टादी}त्यादि । ह्य‚र्थे च‚श‚ब्दः । \textbf{घ‚टादे}र्ये \textbf{कार‚काः} कुलालाद‚यः स‚म‚ग्रास्तेषां‚{\tiny $_{lb}$}‚ यो \textbf{ध‚र्मो} निय‚मेन स्व‚कार्यार‚म्भ‚क‚त्व\textbf{न्त‚स्य क‚र‚णेषु दृष्टेः} । तान्य‚पि हि व्यापृतानि‚{\tiny $_{lb}$}‚ श‚ब्दं निय‚मेन ज‚न‚य‚न्ति । त‚स्मात् तान्य‚पि कुलालादि‚{\tiny $_{५}$}‚व‚त् कार‚काण्येव ।
	{\color{gray}{\rmlatinfont\textsuperscript{§~\theparCount}}}
	\pend% ending standard par
      ‚{\tiny $_{lb}$}‚

	  
	  \pstart \leavevmode% starting standard par
	य‚दि च श‚ब्द‚स्य व्य‚ञ्ज‚कान्त‚राभावात् क‚र‚णानि व्य‚ञ्ज‚कानीष्य‚न्ते । त‚दा‚{\tiny $_{lb}$}‚ \textbf{त‚स्यैव} व्य‚ञ्ज‚क‚स्य \textbf{प्र‚दीपादेर्विष‚यान्त‚र‚स्य च क‚स्य‚चिदि}ति र‚सादेर्व्य‚ञ्ज‚कान्त‚र‚{\tiny $_{lb}$}‚\textbf{म्प्र‚दी}पादिर्नास्ति । त‚तो \textbf{व्य‚ञ्ज‚कान्ताराभावात् । त‚त्कार‚णानि} प्र‚दीपादिकार‚णानि‚{\tiny $_{lb}$}‚ \textbf{चैषां} प्र‚दीपादीनां \textbf{व्य‚ञ्ज‚कानि स्युः} ।
	{\color{gray}{\rmlatinfont\textsuperscript{§~\theparCount}}}
	\pend% ending standard par
      ‚{\tiny $_{lb}$}‚

	  
	  \pstart \leavevmode% starting standard par
	य‚त एव\textbf{न्त‚स्मान्न व्य‚क्तिः श‚ब्द‚स्य} क‚र‚णेभ्यः किन्तूत्प‚त्तिरेव । \textbf{भ‚व‚न्ती‚{\tiny $_{६}$}‚ वा‚{\tiny $_{lb}$}‚ क‚र‚णेभ्यः} स‚काशाद् व्य‚क्तिस्त्रिधा भ‚वेत् । १ पूर्वाव‚स्थात्यागे\textbf{नातिश‚य‚व‚त्ता‚{\tiny $_{lb}$}‚ वा श‚ब्द‚स्य व्य‚क्ति}र्भ‚वेत् । २ उप‚ल‚म्भा\textbf{व‚र‚ण‚विग‚मो वा} । ३ श‚ब्दाल‚म्ब‚नं‚{\tiny $_{lb}$}‚ \textbf{\textbf{वि}ज्ञान‚म्}वा व्य‚क्तिः । प्र‚कार‚त्र‚य‚व्य‚तिरेकेण \textbf{ग‚त्य‚न्त‚राभावात्} ।
	{\color{gray}{\rmlatinfont\textsuperscript{§~\theparCount}}}
	\pend% ending standard par
      ‚{\tiny $_{lb}$}‚

	  
	  \pstart \leavevmode% starting standard par
	१ \textbf{त‚त्र नातिश‚योत्प‚त्तिः} श‚ब्द‚स्य व्य‚क्तिर\textbf{नित्य‚ताप्र‚स‚ङ्गात्} । भ‚व‚त्व‚ति‚{\tiny $_{lb}$}‚\leavevmode\ledsidenote{\textenglish{178b/PSVTa}} श‚योत्प‚त्तिर‚{\tiny $_{७}$}‚नित्य‚त्व‚न्तु क‚थ‚मिति चेदाह । \textbf{त‚स्या} अतिश‚योत्प‚त्तेः \textbf{पूर्व}रूप‚स्य या‚{\tiny $_{lb}$}‚ \textbf{हानिर‚प‚र}स्य पाश्चात्य‚स्य रूप‚स्य यद् \textbf{उप‚ज‚न‚नं} त‚ल्ल\textbf{क्ष‚ण‚त्वात्} ।
	{\color{gray}{\rmlatinfont\textsuperscript{§~\theparCount}}}
	\pend% ending standard par
      ‚{\tiny $_{lb}$}‚

	  
	  \pstart \leavevmode% starting standard par
	२ द्वितीय‚प‚क्ष‚माह । \textbf{अथे}त्यादि । \textbf{त}स्य श‚ब्द‚स्य य‚ज्ज‚न‚कं \textbf{रूप}न्त‚स्योप‚ल‚{\tiny $_{lb}$}‚म्भ‚प्र‚तिघातीनि स्तिमित‚वाय‚वीयाव‚य‚व‚संयोग‚रूपाण्या\textbf{व‚र‚णा}नि [।] ते\textbf{षां विग‚मः}‚{\tiny $_{lb}$}‚ प्र‚य‚त्न‚प्रेरितेन वायुना वियोगः । स \textbf{य‚दि} श‚{\tiny $_{१}$}‚ब्द‚स्य \textbf{व्य‚क्तिस्ते} त‚व मी मां स क स्य‚{\tiny $_{lb}$}‚ म‚ता । त‚दुक्तं ।
	{\color{gray}{\rmlatinfont\textsuperscript{§~\theparCount}}}
	\pend% ending standard par
      ‚{\tiny $_{lb}$}‚
	  \bigskip
	  \begingroup
	
	    
	    \stanza[\smallbreak]
	  {\normalfontlatin\large ``\qquad}प्र‚य‚त्नाभिह‚तो वायुः कोष्ठ्यो यातीत्य‚संश‚यं ।&‚{\tiny $_{lb}$}‚स संयोग‚विभागौ च ताल्वादेर‚नुव‚र्त्त‚ते ॥&‚{\tiny $_{lb}$}‚वेग‚व‚त्वाच्च सोव‚श्यं याव‚द्वेगं प्र‚तिष्ठ‚ते ।&‚{\tiny $_{lb}$}‚त‚स्यात्माव‚य‚वानाञ्च स्तिमितेन च वायुना ।&‚{\tiny $_{lb}$}‚संयोगाश्च वियोगाश्च जाय‚न्ते ग‚म‚नाद् ध्रुव‚मिति ॥ \href{http://sarit.indology.info/?cref=\%C5\%9Bv-\%C5\%9Babda.122-124}{श्लो० श‚ब्द १२२-२४}{\normalfontlatin\large\qquad{}"}\&[\smallbreak]
	  
	  
	  
	  \endgroup
	‚{\tiny $_{lb}$}‚‚{\tiny $_{lb}$}‚\textsuperscript{\textenglish{503/s}}

	  
	  \pstart \leavevmode% starting standard par
	सा व्य‚क्तिः क‚थं क्रिय‚ते । य‚स्मादाव‚र‚ण‚विग‚मोऽभाव‚स्त‚स्मिन्न\textbf{भावे} क‚थं‚{\tiny $_{lb}$}‚चिद‚प्य‚{\tiny $_{२}$}‚कार्ये \textbf{क‚र‚ण‚ग्राम}स्य क‚र‚ण‚स‚ङ्घात‚स्य \textbf{साम‚र्थ्य किन्न त‚द् भ‚वेत्} [।] नैवेति‚{\tiny $_{lb}$}‚ याव‚त् ।
	{\color{gray}{\rmlatinfont\textsuperscript{§~\theparCount}}}
	\pend% ending standard par
      ‚{\tiny $_{lb}$}‚

	  
	  \pstart \leavevmode% starting standard par
	\textbf{न ही}त्यादिना व्याच‚ष्टे । एत‚दाह [।] \textbf{आव‚र‚ण}स्व‚रूपे निष्प‚न्ने\textbf{ऽकिंचित्क‚रा}ण्येव‚{\tiny $_{lb}$}‚ \textbf{क‚र‚णानि} न हि \textbf{स‚म‚र्थानि} भ‚व‚न्ति । आव‚र‚ण‚विग‚मेपि न तेषां साम‚र्थ्यं । य‚स्माद्‚{\tiny $_{lb}$}‚ \textbf{विग‚म‚श्चाभावो न चाभावः कार्य इति निवेदित‚मेत‚त् साम‚र्थ्य‚चिन्तायां} ।
	{\color{gray}{\rmlatinfont\textsuperscript{§~\theparCount}}}
	\pend% ending standard par
      ‚{\tiny $_{lb}$}‚

	  
	  \pstart \leavevmode% starting standard par
	अभ्युप‚ग‚म्य चैत‚दुक्तं । त‚देव ना‚{\tiny $_{३}$}‚स्तीत्याह । \textbf{नापी}त्यादि । \textbf{न श‚ब्द‚स्य‚{\tiny $_{lb}$}‚ नित्य‚स्}यानाधेयातिश‚य‚त्वात् \textbf{किञ्चिदाव‚र‚ण}म‚स्ति येनाव‚र‚ण‚विग‚मो व्य‚क्तिः‚{\tiny $_{lb}$}‚ स्यात् । किं कार‚णं [।] त‚स्याव‚र‚ण‚स्य नित्य‚व‚स्तुन्य\textbf{साम‚र्थ्यादित्य‚प्युक्तं} ।
	{\color{gray}{\rmlatinfont\textsuperscript{§~\theparCount}}}
	\pend% ending standard par
      ‚{\tiny $_{lb}$}‚

	  
	  \pstart \leavevmode% starting standard par
	य‚त एव\textbf{न्त‚स्मान्नाव‚र‚णे क‚र‚णा}नामु\textbf{प‚क्षेपः} । क‚र‚णान्याव‚र‚ण‚विग‚मं श‚ब्द‚स्य‚{\tiny $_{lb}$}‚ कुर्व‚न्तीत्येत‚न्नोप‚न्य‚स‚नीय‚मित्य‚र्थः ।
	{\color{gray}{\rmlatinfont\textsuperscript{§~\theparCount}}}
	\pend% ending standard par
      ‚{\tiny $_{lb}$}‚

	  
	  \pstart \leavevmode% starting standard par
	\textbf{नाप्येवं} कार‚णानां श‚ब्दंप्र‚त्\textbf{य‚साम‚र्थ्य‚मे}व । किं कार‚{\tiny $_{४}$}‚णं [।] \textbf{त‚द्व्यापाराभावे}‚{\tiny $_{lb}$}‚ क‚र‚णानां व्यापाराभावे \textbf{श‚ब्दानुप‚ल‚ब्धेः [।] अतो युक्त‚मेते} क‚र‚ण‚व्यापारा‚{\tiny $_{lb}$}‚ \textbf{य‚च्छ‚ब्दान् कुर्युः} [।]
	{\color{gray}{\rmlatinfont\textsuperscript{§~\theparCount}}}
	\pend% ending standard par
      ‚{\tiny $_{lb}$}‚

	  
	  \pstart \leavevmode% starting standard par
	\textbf{अन्य‚था} य‚दि क‚र‚णानि न कार‚काणि किन्तु व्य‚ञ्ज‚कान्येव । त‚दा \textbf{श‚ब्दावि‚{\tiny $_{lb}$}‚शेषाद‚न्येषाम‚पि} घ‚टादीनां \textbf{व्य‚क्तिः} कुलालादिभ्यः \textbf{प्र‚स‚ज्य‚ते} ।
	{\color{gray}{\rmlatinfont\textsuperscript{§~\theparCount}}}
	\pend% ending standard par
      ‚{\tiny $_{lb}$}‚

	  
	  \pstart \leavevmode% starting standard par
	अथ पुन‚स्तेपि घ‚टाद‚यो व्य‚ज्य‚न्त एव कुलालादिभिरितीष्य‚ते । त‚दा \textbf{त‚था‚{\tiny $_{lb}$}‚भ्युप‚ग‚मे स‚र्व‚कार‚णानान्निर‚{\tiny $_{५}$}‚र्थ‚ता} ।
	{\color{gray}{\rmlatinfont\textsuperscript{§~\theparCount}}}
	\pend% ending standard par
      ‚{\tiny $_{lb}$}‚

	  
	  \pstart \leavevmode% starting standard par
	त‚था हि [।] व्य‚ङ्ग्ये व‚स्तुन्य‚तिश‚य‚स्य कार‚को वाऽव‚र‚णाभाव‚स्य कार‚को‚{\tiny $_{lb}$}‚ वा ज्ञान‚स्य वा कार‚को व्य‚ञ्ज‚कः स्यात् । अतिश‚यादेर्व्य‚क्तिस्व‚रूप‚स्य चाकार्य‚{\tiny $_{lb}$}‚त्वात् । स‚र्वेषां व्य‚क्तिकार‚काणां स्व‚रूप‚कार‚काणां च निर‚र्थ‚ता ।
	{\color{gray}{\rmlatinfont\textsuperscript{§~\theparCount}}}
	\pend% ending standard par
      ‚{\tiny $_{lb}$}‚

	  
	  \pstart \leavevmode% starting standard par
	\textbf{य‚दी}त्यादिना व्याख्यानं । \textbf{य‚दि} श‚ब्द‚स्य \textbf{क‚र‚णानि व्य‚ञ्ज‚कानि} कीदृशानि‚{\tiny $_{lb}$}‚ ‚{\tiny $_{lb}$}‚ \leavevmode\ledsidenote{\textenglish{504/s}}[।] \textbf{स‚र्व‚कार‚ण‚स‚मान‚ध‚र्माण्य}पि स‚र्वे कार‚णानां स‚माना ध‚र्मा‚{\tiny $_{६}$}‚ येषां क‚र‚णाना‚{\tiny $_{lb}$}‚मिति विग्र‚हः । त‚दा \textbf{न किञ्चिद्} घ‚टादिक‚म‚पी\textbf{दानीं} क‚स्य‚चित् कुलालादेः \textbf{कार्यं‚{\tiny $_{lb}$}‚ स्या}च्छ‚ब्देनाविशेषात् । स‚र्व‚स्य व्य‚ङ्ग्य‚त्व‚मिष्ट‚मिति चेत् [।] \textbf{न चेत‚द्} युक्तं ।‚{\tiny $_{lb}$}‚ किं कार‚णं [।] स‚र्व‚कार‚काणां व्य‚ञ्ज‚क‚त्वेनाभिम‚तानामान‚र्थ‚क्य‚प्र‚स‚ङ्गात् । त‚था‚{\tiny $_{lb}$}‚हि व्य‚ञ्ज‚कानां त्र‚यो विक‚ल्पाः । व्य‚ङ्ग्ये व‚स्तुन्य‚तिश‚य‚स्य वाऽव‚र‚णाभाव‚स्य वा‚{\tiny $_{lb}$}‚ \leavevmode\ledsidenote{\textenglish{179a/PSVTa}} ज्ञान‚स्य वा क‚र‚णाद् व्य‚ञ्ज‚कः स्या‚{\tiny $_{७}$}‚दिति ।
	{\color{gray}{\rmlatinfont\textsuperscript{§~\theparCount}}}
	\pend% ending standard par
      ‚{\tiny $_{lb}$}‚

	  
	  \pstart \leavevmode% starting standard par
	न ताव‚द् व्य‚ङ्ग्य‚स्यातिश‚य‚क‚र‚णाद् व्य‚ञ्ज‚कः । किं कार‚ण‚म् [।] व‚स्तुनो‚{\tiny $_{lb}$}‚व‚स्थित‚रूप‚स्या\textbf{नाधेयातिश‚य‚त्वात्} । नाप्याव‚र‚ण‚विग‚म‚क‚र‚णात् । \textbf{आव‚र‚णा‚{\tiny $_{lb}$}‚भाव‚स्याकार्य‚त्वात्} । नापि ज्ञान‚क‚र‚णाद् व्य‚ञ्ज‚कः । किं कार‚णं । \textbf{व‚स्तुव‚देव}‚{\tiny $_{lb}$}‚ त‚द्विष‚य‚स्यापि \textbf{ज्ञान‚स्य} स‚त्कार्य‚वादिद‚र्श‚ने \textbf{सिद्ध‚त्वात्} । अथास‚देव ज्ञानं क्रिय‚ते ।‚{\tiny $_{lb}$}‚ त‚दा \textbf{ज्ञानंप्र‚ति कार‚क‚त्वे क‚स्य‚चिदि}ष्य‚माणे । \textbf{त‚थाभूतानां} ज्ञान‚स्य कार‚{\tiny $_{१}$}‚कैस्तुल्य‚{\tiny $_{lb}$}‚ध‚र्माणा\textbf{म‚न्येषाम‚पि} कुलालादीना\textbf{न्त‚थाभाव‚प्र‚संगेन} । घ‚टादीन् प्र‚ति कार‚क‚त्व‚{\tiny $_{lb}$}‚प्र‚संगेन \textbf{स‚र्व‚स्य} व‚स्तुनः \textbf{कार्य‚ताप्र‚संगात्} । विशेषो वा वाच्यो येन ज्ञानं प्र‚ति कार‚क‚{\tiny $_{lb}$}‚त्वं न घ‚टादीन् प्र‚ति । न चान्यो व्य‚क्तेः प्र‚कारः स‚म्भ‚व‚ति ।
	{\color{gray}{\rmlatinfont\textsuperscript{§~\theparCount}}}
	\pend% ending standard par
      ‚{\tiny $_{lb}$}‚

	  
	  \pstart \leavevmode% starting standard par
	\textbf{त‚स्माद‚यं कार‚काभिम‚तोर्थ‚क‚लापो} घ‚टादेः क‚स्य‚चिद‚पि न \textbf{व्य‚क्ता}वुप‚युज्य‚ते ।‚{\tiny $_{lb}$}‚ व‚स्तुनो नाधेय‚विशेष‚त्वादिना व्य‚{\tiny $_{२}$}‚क्तेर्निषिद्ध‚त्वात् । \textbf{नापि क्रियायामु}प‚युज्य‚ते ।‚{\tiny $_{lb}$}‚ कार्य‚क‚त्वान‚भ्युप‚ग‚मा\textbf{दिति व्य‚र्थ एव स्यात्} ।
	{\color{gray}{\rmlatinfont\textsuperscript{§~\theparCount}}}
	\pend% ending standard par
      ‚{\tiny $_{lb}$}‚

	  
	  \pstart \leavevmode% starting standard par
	\textbf{त‚था} चेति कार‚काणां वैक‚ल्ये स‚ति । \textbf{इदं ज‚ग‚न्निरीहं} निर्व्यापारं \textbf{स्यात्} । किं‚{\tiny $_{lb}$}‚ भूत‚म् [।] \textbf{अनुप‚कार्योप‚कार‚कं} । न विद्य‚ते उप‚कार्य‚मुप‚कार‚कं च य‚स्मिन्निति‚{\tiny $_{lb}$}‚ विग्र‚हः ।
	{\color{gray}{\rmlatinfont\textsuperscript{§~\theparCount}}}
	\pend% ending standard par
      ‚{\tiny $_{lb}$}‚

	  
	  \pstart \leavevmode% starting standard par
	किञ्च [।] \textbf{श‚ब्द‚नित्य‚त्वे} साध्ये[।]\textbf{साध‚नं प्र‚त्य‚भिज्ञान}म‚प्र‚योगादि \textbf{य‚न्म‚त}मिष्टं ।‚{\tiny $_{lb}$}‚ य‚था नित्यः श‚ब्द एक‚त्वेन प्र‚त्य‚{\tiny $_{३}$}‚भिज्ञाय‚मान‚त्वात् । त‚दुक्तं [।] संख्याभावात् ।‚{\tiny $_{lb}$}‚ अष्ट‚कृत्वो गोश‚ब्द उच्च‚रित इति हि व‚द‚न्ति । नाष्टौ गोश‚ब्दा इत्य‚नेनाव‚ग‚म्य‚ते‚{\tiny $_{lb}$}‚ ‚{\tiny $_{lb}$}‚ \leavevmode\ledsidenote{\textenglish{505/s}}प्र‚त्य‚भिजान‚न्तीति । स‚तः प्र‚योगात् । य‚त् प्र‚युज्य‚ते त‚त् प्राक् स‚त् । य‚था वास्यादि‚{\tiny $_{lb}$}‚ च्छिदायां । प्र‚युज्य‚ते च श‚ब्दोर्थ‚प्र‚तिपाद‚ने त‚स्मात्सोपि प्र‚योगात् प्राक् स‚न्निति ।‚{\tiny $_{lb}$}‚ आदिश‚ब्दात् प‚रार्थ‚मुच्चार्य‚माण‚त्वादित्यादिप‚रिग्र‚हः । त‚दुक्तं [।] \textbf{नित्य‚स्तु‚{\tiny $_{lb}$}‚ स्या‚{\tiny $_{४}$}‚द्द‚र्श‚न‚स्य प‚रार्थ‚त्वात्} । द‚र्श‚न‚मुच्चार‚णं [।] त‚त्प‚रार्थ‚म‚र्थं प्र‚त्याय‚यितुं ।‚{\tiny $_{lb}$}‚ उच्चारित‚मात्र एव विन‚ष्टे श‚ब्दे । न त‚तोर्थ‚म्प्र‚त्याय‚यितुं श‚क्नुयाद‚तो न प‚रार्थ‚{\tiny $_{lb}$}‚मुच्चार्येतेति । \textbf{अनुदाह‚र‚ण}मित्य‚दृष्टान्तं । न हि नित्यं किंचिद‚स्ति य‚त्रैत‚त्सा‚{\tiny $_{lb}$}‚ध‚न‚म्व‚र्त्तेत । किं कार‚णं [।] \textbf{स‚र्व‚भावानां क्ष‚ण‚भ‚ङ्ग‚तः} ।
	{\color{gray}{\rmlatinfont\textsuperscript{§~\theparCount}}}
	\pend% ending standard par
      ‚{\tiny $_{lb}$}‚

	  
	  \pstart \leavevmode% starting standard par
	एत‚देव ताव‚त् प‚दं विवृण्व‚न्नाह । \textbf{क्ष‚ण‚भ‚ङ्गिनो ही}त्यादि । \textbf{विनाश‚स्याक‚र‚{\tiny $_{lb}$}‚णादित्युक्तं प्राक् । व‚क्ष्य‚ते च} प‚श्चात् । विनाश‚द्वारेणानित्य‚तां प्र‚द‚र्श्य \textbf{उत्प}त्ति‚{\tiny $_{lb}$}‚द्वारेणापि द‚र्श‚य‚न्नाह । उत्प‚त्तीत्यादि । \textbf{प‚र‚तः} कार‚णा\textbf{दुत्प‚त्तिम‚न्त‚श्च} भावास्त‚{\tiny $_{lb}$}‚तोपि न नित्यः । प‚र‚त उत्प‚द्य‚न्त इति कुत एत‚त् । \textbf{स‚त्ताया आक‚स्मिक‚त्वायोगात्} ।‚{\tiny $_{lb}$}‚ आक‚स्मिक‚त्वे देशादिनिय‚मो न स्यादिति प्रागेवोक्तं । \textbf{त‚दि}ति त‚स्मात् \textbf{प्र‚त्य‚भि‚{\tiny $_{lb}$}‚ज्ञानं स‚त्प्र‚योगादिकं} लिङ्गं ।‚{\tiny $_{६}$}‚ \textbf{स्थिरैक‚रूपे} व‚स्तुनि स‚प‚क्ष‚भूते । \textbf{न क्व‚चिद‚न्वेति ।‚{\tiny $_{lb}$}‚ अप‚राप‚रेणा}न्येनान्येन \textbf{स्व‚भावे}न \textbf{प‚रावृत्ति}रुत्प‚त्तिर्य‚था \textbf{प्र‚दीपादी}नान्ते\textbf{ष्वेव} प्र‚त्य‚{\tiny $_{lb}$}‚भिज्ञानादिलिङ्ग‚म्भेदेन व्याप्त\textbf{न्दृष्ट‚मिति विरुद्ध‚मेव} । न चात्र दीप‚त्वादिसामा‚{\tiny $_{lb}$}‚न्य‚निमित्तं प्र‚त्य‚भिज्ञानं सामान्य‚स्य पूर्व‚मेव निषिद्ध‚त्वात् ।
	{\color{gray}{\rmlatinfont\textsuperscript{§~\theparCount}}}
	\pend% ending standard par
      ‚{\tiny $_{lb}$}‚

	  
	  \pstart \leavevmode% starting standard par
	\textbf{ने}त्यादि प‚रः । \textbf{न} विरुद्धं प्र‚त्य‚भिज्ञानं [।] किं कार‚णं । अभिन्नात् स्थिरैक‚रू‚{\tiny $_{lb}$}‚पाज्ज‚न्म य‚स्य‚{\tiny $_{७}$}‚ त‚स्या\textbf{भिन्न‚ज‚न्म‚नः} प्र‚त्य‚भिज्ञान‚स्य \textbf{दीपादिषु भ्रान्त्या भावात्} । \leavevmode\ledsidenote{\textenglish{179b/PSVTa}}‚{\tiny $_{lb}$}‚ भ्रान्तिः क‚थ‚मिति चेत् । \textbf{साध‚र्म्य‚विप्र‚ल‚म्भात्} पूर्वोत्त‚र‚योः क्ष‚ण‚योर्य‚त्सादृश्य‚न्तेन‚{\tiny $_{lb}$}‚ विप्र‚ल‚म्भाद् व‚ञ्चित‚त्वात् । एत‚त्क‚थ‚य‚ति [।] अभ्रान्तं प्र‚त्य‚भिज्ञानं लिंग‚त्वेनो‚{\tiny $_{lb}$}‚पात्त‚न्त‚च्च नैव प्र‚दीपादिषु व‚र्त्त‚ते ।
	{\color{gray}{\rmlatinfont\textsuperscript{§~\theparCount}}}
	\pend% ending standard par
      ‚{\tiny $_{lb}$}‚

	  
	  \pstart \leavevmode% starting standard par
	\textbf{अभिन्न‚ज‚न्मे}त्या चा र्यः । \textbf{केनाव‚ष्ट‚म्भेन} केन प्र‚माणेना\textbf{भिन्न‚ज‚न्म}प्र‚त्य‚भि‚{\tiny $_{lb}$}‚ज्ञान\textbf{मित्युच्य‚ते} । नित्य‚स्य साम‚र्थ्याभा‚{\tiny $_{१}$}‚वात् । एत‚च्च प्र‚त्य‚क्ष‚स्यापि प्र‚त्य‚भि‚{\tiny $_{lb}$}‚ज्ञान‚स्य दूष‚ण‚न्द्र‚ष्ट‚व्यं । प्र‚त्य‚भिज्ञाय‚मान‚स्याभेदेन प्र‚तिभास‚नादिति चेत् [।]‚{\tiny $_{lb}$}‚ न [।] \textbf{त‚स्यैवाभेद‚स्य स‚र्व‚त्र व‚ज्रोप‚लादि}ष्व‚पि \textbf{पौर्वाप‚र्येण चिन्त्य‚त्वात्} । य‚था‚{\tiny $_{lb}$}‚ ‚{\tiny $_{lb}$}‚ \leavevmode\ledsidenote{\textenglish{506/s}}पूर्व‚म्ब‚ज्रादिषु स्व‚रूपं किं प‚श्चाद‚पि त‚देवाहोस्विद‚न्य‚देव केव‚लं साद‚श्यादेक‚त्व‚{\tiny $_{lb}$}‚विभ्र‚मः प्र‚दीपादिष्विवेति चिन्त्य‚मेत‚त् । य‚था व‚ज्रादिष्व‚भेद‚स्य चिन्त्य‚त्व\textbf{न्त‚था‚{\tiny $_{lb}$}‚ भेद‚स्यापि} चि‚{\tiny $_{२}$}‚न्त्य‚त्वा\textbf{दिति चेत्} । किम्भेदः पौर्वाप‚र्येण प्र‚तिभास‚त इत्येत‚द‚पि‚{\tiny $_{lb}$}‚ निरूप‚णीय‚मेव ।
	{\color{gray}{\rmlatinfont\textsuperscript{§~\theparCount}}}
	\pend% ending standard par
      ‚{\tiny $_{lb}$}‚

	  
	  \pstart \leavevmode% starting standard par
	\textbf{तेनै}वेत्याचार्यः । य‚त‚श्च नैकान्तेन भेदोऽभेदो वाव‚धार‚यितुं श‚क्य‚स्तेनैवान‚{\tiny $_{lb}$}‚व‚धार‚णेन \textbf{संश‚योस्तु} । प्र‚त्य‚भिज्ञाय‚मानेष्व‚र्थेषु भेदाभेद‚संश‚यो भ‚व‚तु । संश‚यादेव‚{\tiny $_{lb}$}‚ प्र‚त्य‚भिज्ञाय‚मान‚त्वाद् भेद‚निश्च‚य इति चेदाह । \textbf{न च} सं\textbf{श‚यितात्} संश‚य‚विष‚यात्‚{\tiny $_{lb}$}‚ प्र‚{\tiny $_{३}$}‚त्य‚भिज्ञान‚लिङ्गाच्छ‚ब्द‚स्यैक‚त्व‚सिद्धिः ।
	{\color{gray}{\rmlatinfont\textsuperscript{§~\theparCount}}}
	\pend% ending standard par
      ‚{\tiny $_{lb}$}‚

	  
	  \pstart \leavevmode% starting standard par
	पूर्व‚क्ष‚णादुत्त‚र‚स्य क्ष‚ण‚स्य \textbf{विवेकाद‚र्श‚नाद्} विवेकाप्र‚तिभास‚नात् पूर्वोत्त‚र‚{\tiny $_{lb}$}‚कालेषु भाव‚स्यै\textbf{क‚त्वं} सिद्ध\textbf{मिति चेत्} ।
	{\color{gray}{\rmlatinfont\textsuperscript{§~\theparCount}}}
	\pend% ending standard par
      ‚{\tiny $_{lb}$}‚

	  
	  \pstart \leavevmode% starting standard par
	\textbf{ने}त्यादि प्र‚तिव‚च‚नं । तेनाय‚म‚र्थो भाव‚स्येदानीम्प्र‚तिभास एव क्ष‚ण‚प्र‚ति‚{\tiny $_{lb}$}‚भासः । पूर्वाप‚र‚काल‚स‚म्ब‚न्धित्वेनाप्र‚तिभास‚नात् । क्ष‚ण‚स्य च स्व‚रूपेण प्र‚तिभास‚{\tiny $_{lb}$}‚ एव पूर्वादिक्ष‚णाद् विवेकेन प्र‚तिभास‚{\tiny $_{४}$}‚स्सु मे रु\edtext{}{\lemma{रु}\Bfootnote{?}} भिन्न‚प्र‚तिभास‚वान्न त्व‚विना‚{\tiny $_{lb}$}‚भावेन पूर्वादिक्ष‚णात् प्र‚तिभास‚मानात् । केव‚लं स विवेको नाव‚धार्य‚त इति ।‚{\tiny $_{lb}$}‚ त‚द‚र्थ‚म‚नुमानं प्र‚व‚र्त्त‚ते । त‚दाह न व‚ज्रादिष्व‚विवेक‚स्याद‚र्श‚न‚म‚स्ति । किं कार‚णं‚{\tiny $_{lb}$}‚ [।] \textbf{पौर्वाप‚र्येण} व‚ज्रादि\textbf{ज्ञाना}नां पूर्वाप‚र‚भावेन व‚ज्रादेः \textbf{स‚द‚स‚त्त्व‚सिद्धेः} । त‚था हि‚{\tiny $_{lb}$}‚ [।] व‚ज्राद्याल‚म्ब‚न‚मुत्त‚रं ज्ञानं प्राग‚भ‚व‚त् स्व‚कार‚ण‚विशेष‚स्य प्राग‚स‚त्त्वं साध‚{\tiny $_{५}$}‚य‚ति ।‚{\tiny $_{lb}$}‚ प‚श्चाद् भ‚व‚च्च स‚त्त्व‚मित्येवं ज्ञान‚पौर्वाप‚र्येण व‚ज्रादिषु स‚द‚स‚त्त्व‚सिद्धेः । \textbf{विवेक‚{\tiny $_{lb}$}‚स‚द्भावाद्} भेद‚स‚द्भावात् ।
	{\color{gray}{\rmlatinfont\textsuperscript{§~\theparCount}}}
	\pend% ending standard par
      ‚{\tiny $_{lb}$}‚

	  
	  \pstart \leavevmode% starting standard par
	एत‚देव स्फुट‚य‚न्नाह । \textbf{य‚दी}त्यादि । \textbf{अप‚राण्यु}त्त‚र‚काल‚भावीनि \textbf{ज्ञानानि‚{\tiny $_{lb}$}‚ प्राक्} पूर्व‚ज्ञान‚काले \textbf{स‚न्निहित‚कार‚णानि} स्युः \textbf{पूर्व‚ज्ञान‚व‚ज्जातान्येव स्युः} [।] न‚{\tiny $_{lb}$}‚ चैवं । त‚स्माद\textbf{जातानि तु} तानि ज्ञानानि प्राक् । स्व\textbf{कार‚ण}स्य \textbf{वैक‚ल्यं सूच‚य‚न्ति} ।‚{\tiny $_{lb}$}‚ अन्य‚था य‚{\tiny $_{६}$}‚दि तेषां कार‚णं प्राग‚पि स्यात् । त‚त्स‚म‚र्थ‚म्वा भ‚वेद‚स‚म‚र्थ‚म्वा । य‚दि‚{\tiny $_{lb}$}‚ ‚{\tiny $_{lb}$}‚ \leavevmode\ledsidenote{\textenglish{507/s}}स‚म‚र्थं प्राग‚पि ज‚न‚येत् । किं कार‚णं [।] \textbf{स‚म‚र्थ‚स्य ज‚न‚नात्} । अथास‚म‚र्थ‚म् [।]‚{\tiny $_{lb}$}‚ प‚श्चाद‚पि न ज‚न‚येत् । क‚स्माद् [।] \textbf{अस‚म‚र्थ‚स्याप्य}नाधेयातिश‚य‚त्वेन \textbf{पुनः} कुत‚{\tiny $_{lb}$}‚श्चित् \textbf{साम‚र्थ्य\textbf{आ}प्र‚तिल‚म्भात्} । अथ कुत‚श्चित् साम‚र्थ्यं प्र‚तिल‚भेत । त‚दा‚{\tiny $_{lb}$}‚ साम‚र्थ्य‚स्य \textbf{प्र‚तिल‚म्भे वा स्थैर्यायोगात्} । पूर्वास‚म‚र्थ‚स्व‚भाव‚हाने‚{\tiny $_{७}$}‚र‚न्य‚स्य च स‚म‚र्थ- \leavevmode\ledsidenote{\textenglish{180a/PSVTa}}‚{\tiny $_{lb}$}‚ स्योत्पादात् ।
	{\color{gray}{\rmlatinfont\textsuperscript{§~\theparCount}}}
	\pend% ending standard par
      ‚{\tiny $_{lb}$}‚

	  
	  \pstart \leavevmode% starting standard par
	त‚स्मात् क्र‚म‚भावीनि विज्ञानानि स्व‚विष‚य‚स्यापि क्र‚मं साध‚य‚न्तीति स‚र्व‚प‚दा‚{\tiny $_{lb}$}‚र्थानाम्भेद‚सिद्धेर‚नित्य‚त्वं [।] स एवाय‚मिति ज्ञानं तु स‚दृश‚द‚र्श‚न‚निमित्तं ।
	{\color{gray}{\rmlatinfont\textsuperscript{§~\theparCount}}}
	\pend% ending standard par
      ‚{\tiny $_{lb}$}‚

	  
	  \pstart \leavevmode% starting standard par
	य‚त एव‚न्\textbf{त‚स्माद‚यं स‚त्प्र‚योग इत्य‚पि} योयं द्वितीयो हेतुरुच्य‚ते । तेनापि श‚ब्द‚स्य‚{\tiny $_{lb}$}‚ \textbf{ज‚न‚न‚मे}वोच्य‚ते । किं कार‚णं [।] \textbf{प्र‚योक्तुः} स‚काशाच्छ‚ब्द‚स्य \textbf{साम‚र्थ्यात्} । अभि‚{\tiny $_{lb}$}‚म‚त‚{\tiny $_{१}$}‚कार्य‚क‚र‚णे श‚ब्द‚स्य साम‚र्थ्य‚प्र‚तिल‚म्भादित्य‚र्थः । अन्य‚या य‚दि प्र‚योक्तुर्व्या‚{\tiny $_{lb}$}‚पारात् प्रागेव वास्यादिकं श‚ब्दो वा स्व‚कार्ये स‚म‚र्थं स्यात्त‚दा \textbf{स्व‚यं साम‚र्थ्ये त‚स्य}‚{\tiny $_{lb}$}‚ प्र‚योक्तुर\textbf{नुप‚योगा}त् पुरुषान‚पेक्षाणां स्व‚य‚मेव वास्यादीनां प्र‚वृत्तिः स्यात् । न च‚{\tiny $_{lb}$}‚ भ‚व‚ति । त‚स्मात् प्र\textbf{योग इत्य‚पि । इष्ट}स्याभिम‚त‚स्यार्थ‚स्य च्छिदादेः \textbf{साध‚नं}‚{\tiny $_{lb}$}‚ सिद्धिस्त‚त्र \textbf{स‚म‚र्थ}स्व‚भाव‚स्योत्पा\textbf{द‚न‚{\tiny $_{२}$}‚मेव} वास्यादेः \textbf{क‚थ्य‚ते} । किम्भूत‚मुत्पाद‚नं [।]‚{\tiny $_{lb}$}‚ स‚मान‚जातीयं स‚दृश‚मुपादानं पूर्वं कार‚ण‚भूतं क्ष‚ण‚म‚पेक्ष‚त इति \textbf{स‚मान‚जातीयो‚{\tiny $_{lb}$}‚पादानापेक्षं} वा\textbf{स्यादिप्र‚योग‚व‚त्} । छिदादिषु प्र‚युज्य‚मानानां वास्यादीनां स‚मान‚{\tiny $_{lb}$}‚जातीय‚पूर्व‚क्ष‚णापेक्ष‚णात् । \textbf{उपादानान‚पेक्षं वा । क‚र्मादिप्र‚योग‚व‚च्च} । आदिश‚ब्दाद्‚{\tiny $_{lb}$}‚ वीणादिश‚ब्द‚प‚रिग्र‚हः । न हि क‚{\tiny $_{३}$}‚र्मादिषु प्र‚थ‚मं प्र‚युज्य‚मानेषु पूर्व‚स‚दृश‚क्ष‚णापे‚{\tiny $_{lb}$}‚क्षास्ति ।
	{\color{gray}{\rmlatinfont\textsuperscript{§~\theparCount}}}
	\pend% ending standard par
      ‚{\tiny $_{lb}$}‚

	  
	  \pstart \leavevmode% starting standard par
	\textbf{योपि म‚न्य‚ते} [।] मा भूत् प्र‚त्य‚भिज्ञान‚म‚नुमानं व्य‚भिचारात् य‚त् पुरो‚{\tiny $_{lb}$}‚व‚स्थिते व‚स्तुनि \textbf{स‚म‚क्षे प्र‚त्य‚भिज्ञान}न्त\textbf{त्प्र‚त्य‚क्षं} प्र‚माणं । \textbf{त‚तः प्र‚त्य‚क्षादेव} प्र‚माणाद्‚{\tiny $_{lb}$}‚ भावानां \textbf{स्थैर्य‚सिद्धिरिति । त‚द‚प्युत्त‚र‚त्र निषेत्स्यामः} ।
	{\color{gray}{\rmlatinfont\textsuperscript{§~\theparCount}}}
	\pend% ending standard par
      ‚{\tiny $_{lb}$}‚

	  
	  \pstart \leavevmode% starting standard par
	अन‚या दिशा स्थैर्य‚साध‚नायोप‚नीतः \textbf{कुहेतु}र्द्र‚ष्ट‚व्यः । य‚स्मा\textbf{न्नैवं क‚श्चिद्} सा‚{\tiny $_{४}$}‚‚{\tiny $_{lb}$}‚‚{\tiny $_{lb}$}‚ \leavevmode\ledsidenote{\textenglish{508/s}}\textbf{ध‚न‚ध‚र्मोस्त} स्थैर्य‚साध‚नो \textbf{यः स‚मान‚जातीयं} स्थिरैक‚स्व‚भाम्व‚स्त्व‚न्वेति । क‚स्मात्‚{\tiny $_{lb}$}‚ [।] \textbf{स‚र्व‚ध‚र्माणामेत‚द‚व‚स्थानात्} । अप‚राप‚र‚स्व‚भाव‚हान्युत्पाद‚स्व‚भाव‚त्वात् ।‚{\tiny $_{lb}$}‚ \textbf{स‚र्व}स्याः \textbf{स्थैर्य‚प्र‚त्य‚भिज्ञा}याश्च \textbf{युक्तिविरोधा}द‚नुमान‚विरोधात् । क‚थं [।] \textbf{य‚थाभि‚{\tiny $_{lb}$}‚धानं} । य‚थेह शास्त्रे क्ष‚णिक‚त्व‚साध‚न‚म‚भिहित‚म‚भिधास्य‚ते च । त‚था युक्तिविरो‚{\tiny $_{lb}$}‚धात् कार‚णा\textbf{द‚न्येपि} प‚{\tiny $_{५}$}‚र‚प‚रिक‚ल्पिताः स्थैर्य‚साध‚न\textbf{हेत‚वो वाच्य‚दोषाः} ।
	{\color{gray}{\rmlatinfont\textsuperscript{§~\theparCount}}}
	\pend% ending standard par
      ‚{\tiny $_{lb}$}‚

	  
	  \pstart \leavevmode% starting standard par
	एव‚न्ताव‚द् व्य‚क्तिक्र‚मो वाक्यं नेति प्र‚क्र‚म्य व्य‚क्तिस्त्रिविधा क‚ल्पिता ।‚{\tiny $_{lb}$}‚ श‚ब्द‚स्यातिश‚योत्पाद‚नं । त‚दाव‚र‚ण‚विग‚मो [।] ज्ञानं चेति । त‚त्र नित्य‚त्वाच्छ‚{\tiny $_{lb}$}‚ब्द‚स्य नातिश‚योत्पाद‚नं । आव‚र‚णाभाव‚स्य चाकार्य‚त्वान्नाप्याव‚र‚ण‚विग‚मो व्य‚क्ति‚{\tiny $_{lb}$}‚रिति विक‚ल्प‚द्व‚ये प्र‚तिक्षिप्ते । ज्ञानं व्य‚क्तिरित्य‚व‚शिष्य‚ते ।
	{\color{gray}{\rmlatinfont\textsuperscript{§~\theparCount}}}
	\pend% ending standard par
      ‚{\tiny $_{lb}$}‚

	  
	  \pstart \leavevmode% starting standard par
	३ त‚दा च व्य‚क्तिक्र‚{\tiny $_{६}$}‚मो वाक्यं । बुद्धीनामानुपूर्वी वाक्य‚माप‚द्य‚ते ।
	{\color{gray}{\rmlatinfont\textsuperscript{§~\theparCount}}}
	\pend% ending standard par
      ‚{\tiny $_{lb}$}‚

	  
	  \pstart \leavevmode% starting standard par
	न चैत‚द् युक्त‚म् [।] अबुद्धिस्व‚भाव‚त्वाद् वाक्य‚स्य । त‚थाप्य‚भ्युप‚ग‚म्यो‚{\tiny $_{lb}$}‚च्य‚ते । व्य‚क्तिक्र‚म‚स्य च वाक्य‚स्यापौरुषेय‚त्वे साध्ये बुद्धेरेवापौरुषेय‚त्वं साध्यं‚{\tiny $_{lb}$}‚ स्यात् । त‚त्र \textbf{बुद्धेर‚पुरुषाश्र‚ये} पुरुषानाश्र‚य‚णे साध्ये प्र‚तिज्ञाया बाधा । कैः [।]‚{\tiny $_{lb}$}‚ \leavevmode\ledsidenote{\textenglish{180b/PSVTa}} \textbf{अभ्युपेत‚प्र‚त्य‚क्ष‚प्र‚तीतानुमितैः स‚म‚मे}क‚कालं अभ्युपेतेनाभ्युप‚ग‚तेन‚{\tiny $_{७}$}‚ । प्र‚त्य‚क्ष‚प्र‚तीते‚{\tiny $_{lb}$}‚नानुमितेन च । \textbf{त‚दानुपूर्वी} बुद्ध्य‚नुपूर्वी \textbf{वाक्यं} । सा च नार्थान्त‚रं बुद्धिभ्य इति‚{\tiny $_{lb}$}‚ \textbf{त‚स्या} आनुपूर्व्या \textbf{अपौरुषेय‚स्व‚प्र‚साध‚ने} । बुद्धेरेवापौरुषेय‚त्वं साधितं स्यात् । त‚त्र‚{\tiny $_{lb}$}‚ \textbf{च स‚म‚यः} सिद्धान्तोस्य मी मां स क स्य \textbf{बाध्य‚ते} । किं कार‚णं [।] \textbf{बुद्धीनां} स्व‚{\tiny $_{lb}$}‚\textbf{सिद्धान्ते पुरुष‚गुण‚त्वेनाभ्युप‚ग‚मात्} ।
	{\color{gray}{\rmlatinfont\textsuperscript{§~\theparCount}}}
	\pend% ending standard par
      ‚{\tiny $_{lb}$}‚

	  
	  \pstart \leavevmode% starting standard par
	प्र‚त्य‚क्ष‚बाधान्द‚र्श‚य‚न्नाह । \textbf{प्र‚त्य‚क्षं ख‚ल्व‚प्येत‚द् य‚दि ता बुद्ध‚यो म‚न‚स्कारा‚{\tiny $_{१}$}‚‚{\tiny $_{lb}$}‚दिभ्यो भ‚व‚न्तीति} । आदिग्र‚ह‚णादिन्द्रिय‚प‚रिग्र‚हः । कीदृशेभ्यो म‚न‚स्कारादिभ्यः‚{\tiny $_{lb}$}‚ ‚{\tiny $_{lb}$}‚ \leavevmode\ledsidenote{\textenglish{509/s}}पुरुष इति व्य‚व‚हार‚लाघ‚वार्थं कृत‚संकेतेभ्यः । एत‚त्स्व‚म‚तेनोक्तं । \textbf{पुरुष‚गुणेभ्यो}‚{\tiny $_{lb}$}‚ वेति व्य‚तिक्र‚म्य पुरुष‚स्य गुणेभ्यः । एत‚त्तु प‚र‚म‚तेनोक्तं । त‚स्मात् म‚न‚स्कारादिभ्य‚{\tiny $_{lb}$}‚ उत्प‚द्य‚मानाया बुद्धेः कार्य‚त्वं प्र‚त्य‚क्ष‚सिद्ध‚मिति त‚स्या अपौरुषेय‚त्वे साध्ये प्र‚त्य‚{\tiny $_{२}$}‚‚{\tiny $_{lb}$}‚क्ष‚बाधा ।
	{\color{gray}{\rmlatinfont\textsuperscript{§~\theparCount}}}
	\pend% ending standard par
      ‚{\tiny $_{lb}$}‚

	  
	  \pstart \leavevmode% starting standard par
	स्यान्म‚तं [।] कार्य‚ताया अप्र‚त्य‚क्ष‚त्वान्न प्र‚त्य‚क्ष‚बाधेति चेदाह । न‚न्वित्यादि ।‚{\tiny $_{lb}$}‚ कार‚णाभिम‚त‚स्य भाव एव भावः त‚द‚भावे चाभाव इत्येतौ \textbf{भावाभाव‚विशेषौ ।‚{\tiny $_{lb}$}‚ ताभ्यां नान्या कार्य‚ता भाव‚स्य [।] स} य‚थोक्तो \textbf{भावः प्र‚त्य‚क्ष}सिद्धः । त‚द‚भावे‚{\tiny $_{lb}$}‚ त्व‚भावः क‚थं प्र‚त्य‚क्ष‚सिद्ध इति चेदाह । \textbf{अभावोपि । अनुप‚ल‚ब्धि}रेव ल‚क्ष‚णं‚{\tiny $_{lb}$}‚ स्व‚भावो य‚स्याभाव‚स्येति विग्र‚हः सोपि \textbf{प्र‚त्य‚क्ष‚{\tiny $_{३}$}‚साम‚र्थ्य‚सिद्धं इ}त्युत्त‚र‚त्र \textbf{व‚क्ष्यामः} ।‚{\tiny $_{lb}$}‚ त‚द‚न्य‚विविक्त‚रूप‚म्भाव‚मेव प्र‚तिपाद‚य‚त् प्र‚त्य‚क्षं साम‚र्थ्याद‚भावं ग‚म‚य‚तीति साम‚र्थ्य‚{\tiny $_{lb}$}‚ग्र‚ह‚णं कृतं । य‚त‚श्च म‚न‚स्कारादिभावाभावाभ्यां बुद्धिभावाभावौ । \textbf{त‚त} एव‚{\tiny $_{lb}$}‚ त‚द्भाव‚भावित्वात् \textbf{पुरुष‚कार्य‚ता बुद्धीनाम‚नुमेया} । किं कार‚ण‚म् [।] \textbf{अन्व‚य‚{\tiny $_{lb}$}‚व्य‚तिरेक‚लिंग‚त्वाद‚स्याः} कार्य‚तायाः । त‚द‚नेनानुमान‚बाधोक्ता ।
	{\color{gray}{\rmlatinfont\textsuperscript{§~\theparCount}}}
	\pend% ending standard par
      ‚{\tiny $_{lb}$}‚

	  
	  \pstart \leavevmode% starting standard par
	अथ स्यान्न व्य‚क्ति‚{\tiny $_{४}$}‚क्र‚मो वाक्यं । किन्तु व‚र्ण्णानुपूर्वी वाक्य‚मित्य‚त आह ।‚{\tiny $_{lb}$}‚ \textbf{किं चे}त्यादि । \textbf{व‚र्ण्णेभ्यः} स‚काशा\textbf{दानुपूर्व्या भेदः स्फो टे ने} ति पूर्वोक्तेन स्फोट‚वि‚{\tiny $_{lb}$}‚चारेण । अभिन्नापि प्रागेव निषिद्धा[।]भेदाभेदं च मुक्त्वा व‚स्तुनो नान्या ग‚ति‚{\tiny $_{lb}$}‚र‚स्ति । त‚दा च व्यापिनाम्व‚र्ण्णानामानुपूर्वी \textbf{क‚ल्प‚नारोपिता स्या}त् । \textbf{क‚थं वा‚{\tiny $_{lb}$}‚ त‚दानीम‚पुरुषाश्र‚या} । पुरुषाश्र‚यैव स्यात् ।
	{\color{gray}{\rmlatinfont\textsuperscript{§~\theparCount}}}
	\pend% ending standard par
      ‚{\tiny $_{lb}$}‚

	  
	  \pstart \leavevmode% starting standard par
	व‚र्ण्णेत्यादि‚{\tiny $_{५}$}‚ना व्याच‚ष्टे । \textbf{व‚र्ण्णे}भ्यः स‚काशाद् \textbf{व्य‚तिरेकिणी} भिन्न‚स्व‚भावा‚{\tiny $_{lb}$}‚\textbf{नुपूर्वी} । पूर्वोक्तेन \textbf{स्फोट‚विचारानुक्र‚मेणैव प्र‚तिविहिता} ।
	{\color{gray}{\rmlatinfont\textsuperscript{§~\theparCount}}}
	\pend% ending standard par
      ‚{\tiny $_{lb}$}‚

	  
	  \pstart \leavevmode% starting standard par
	वाक्य‚न्न भिन्न‚म्व‚र्णेभ्यो विद्य‚तेनुप‚ल‚म्भ‚नाद् [।] इत्यादिना दूष‚णेनानुपूर्व्य‚पि‚{\tiny $_{lb}$}‚ प्र‚तिक्षिप्ता ।
	{\color{gray}{\rmlatinfont\textsuperscript{§~\theparCount}}}
	\pend% ending standard par
      ‚{\tiny $_{lb}$}‚

	  
	  \pstart \leavevmode% starting standard par
	\textbf{नापि सा व‚र्ण‚स्व‚भावा} । स‚रो र‚स इति प्र‚तिप‚त्तिभेद‚भाव‚प्र‚स‚ङ्गात् । न चापि‚{\tiny $_{lb}$}‚ सा त‚त्त्वान्य‚त्त्वाभ्याम‚वाच्या । \textbf{व‚स्तुस्व‚भाव‚स्यैत‚द्विक‚ल्पान‚{\tiny $_{६}$}‚तिक्र‚मात्} । त‚त्त्वान्य‚त्त्व‚{\tiny $_{lb}$}‚‚{\tiny $_{lb}$}‚ \leavevmode\ledsidenote{\textenglish{510/s}}विक‚ल्पान‚तिक्र‚मात् । त‚स्मा\textbf{द‚त‚द्रूपेषु} व‚स्तुभूत‚भिन्नानुपूर्वीर‚हितेषु व‚र्ण्णेषु \textbf{त‚दूप‚{\tiny $_{lb}$}‚स‚मारोप‚प्र‚तिभासिन्या} आनुपूर्वीस‚मारोप‚प्र‚तिभासिन्या \textbf{बुद्धेर‚य‚म्विभ्र‚मः स्यादा‚{\tiny $_{lb}$}‚नुपूर्वीति । सा चा}नुपूर्वी \textbf{क‚थ‚म‚पौरुषेयी} पौरुषेय्येव । किं कार‚णं [।] बुद्धेर्विठं‚{\tiny $_{lb}$}‚ \leavevmode\ledsidenote{\textenglish{181a/PSVTa}} \edtext{\textsuperscript{*}}{\lemma{*}\Bfootnote{?}}प‚नेन व्यापारेण \textbf{प्र‚त्युप‚स्थाप‚नात्} । स‚न्द‚र्शित‚त्वात् ।
	{\color{gray}{\rmlatinfont\textsuperscript{§~\theparCount}}}
	\pend% ending standard par
      ‚{\tiny $_{lb}$}‚

	  
	  \pstart \leavevmode% starting standard par
	\textbf{अपि चात्य‚न्तिक‚स्य} नित्य‚स्य \textbf{क‚स्य‚चित्स्व‚भाव‚स्याभावात् । भ‚व‚ता} विद्य‚मानेन‚{\tiny $_{lb}$}‚ \textbf{ध्व‚निना} श‚ब्देनाव‚श्य‚म\textbf{नात्य‚न्तिकेना}स्थिरेण \textbf{भ‚वित‚व्यं । स च} ध्व‚नि\textbf{र‚हेतुकः} स्यात् ।‚{\tiny $_{lb}$}‚ पुरुष‚व्य‚तिरेकेणान्यो हेतुर‚स्येत्य\textbf{न्य‚हेतुको वा} । त‚त्राहेतुक‚त्वे \textbf{नित्य‚म्भ‚वेत्} । अन्या‚{\tiny $_{lb}$}‚न‚पेक्ष‚णात् । अन्य‚हेतुक‚त्वे तु \textbf{न च पुरुष‚व्यापाराद्} भ‚वेत् । भ‚व‚ति च पुरुष‚व्यापा‚{\tiny $_{lb}$}‚रात् । \textbf{त‚स्मात् पौरुषे‚{\tiny $_{१}$}‚य} इति ग‚म्य‚ते ।
	{\color{gray}{\rmlatinfont\textsuperscript{§~\theparCount}}}
	\pend% ending standard par
      ‚{\tiny $_{lb}$}‚

	  
	  \pstart \leavevmode% starting standard par
	\textbf{क‚थ‚मि}त्यादि प‚रः । \textbf{अनात्य‚न्तिको ध्व‚निर‚न्यो वा} पृथिव्यादिका\textbf{भाव इति‚{\tiny $_{lb}$}‚ क‚थ‚मिदं} ग‚म्य‚ते ।
	{\color{gray}{\rmlatinfont\textsuperscript{§~\theparCount}}}
	\pend% ending standard par
      ‚{\tiny $_{lb}$}‚

	  
	  \pstart \leavevmode% starting standard par
	\textbf{स‚त्ते}त्या चा र्यः । \textbf{नाश‚स्य स‚त्तामात्रानुब‚न्धित्वात्} । कार‚णा\textbf{द‚नित्य‚ता ध्व‚नेः} ।‚{\tiny $_{lb}$}‚ स‚न्नित्येव कृत्वा नाशो भ‚व‚ति न कार‚णान्त‚र‚म‚पेक्ष‚ते । संश्च श‚ब्दः । त‚स्मान्न‚{\tiny $_{lb}$}‚ नित्य इति स‚मुदायार्थः ।
	{\color{gray}{\rmlatinfont\textsuperscript{§~\theparCount}}}
	\pend% ending standard par
      ‚{\tiny $_{lb}$}‚

	  
	  \pstart \leavevmode% starting standard par
	क‚स्मात् स‚त्तामात्रानुब‚न्धी विनाश इत्याह । \textbf{न ही}त्यादि । य‚स्मान्न‚{\tiny $_{२}$}‚ \textbf{भावानां‚{\tiny $_{lb}$}‚ नाशो} नाम ध‚र्मान्त‚रं \textbf{कुत‚श्चि}न्नाश‚कार‚णाद् \textbf{भ‚व‚ति} ।
	{\color{gray}{\rmlatinfont\textsuperscript{§~\theparCount}}}
	\pend% ending standard par
      ‚{\tiny $_{lb}$}‚

	  
	  \pstart \leavevmode% starting standard par
	य‚त एवं [।] \textbf{त‚दि}ति त‚स्माद् \textbf{भाव‚स्व‚भाव} एव नाशो \textbf{भ‚वेत्} । कुत एत‚द् [।]‚{\tiny $_{lb}$}‚ \textbf{भाव‚स्यैव स्व‚हेतुभ्यः} स‚काशात् \textbf{त‚द्ध‚र्म‚णो} विनाश‚ध‚र्म‚णो \textbf{भावा}दुत्प‚त्तेः । एक‚क्ष‚ण‚{\tiny $_{lb}$}‚स्थितिध‚र्म‚क‚त्व‚मेव विनाशः [।] त‚च्च हेतुभ्य एवोत्प‚द्य‚त इति याव‚त् ।
	{\color{gray}{\rmlatinfont\textsuperscript{§~\theparCount}}}
	\pend% ending standard par
      ‚{\tiny $_{lb}$}‚‚{\tiny $_{lb}$}‚\textsuperscript{\textenglish{511/s}}

	  
	  \pstart \leavevmode% starting standard par
	कृत‚कानामेव स‚तां विनाशो नान्येषां स‚तां । त‚दुक्तं [।] स‚द‚का‚{\tiny $_{३}$}‚र‚ण व‚न्नि‚{\tiny $_{lb}$}‚त्य‚मि ति [।]
	{\color{gray}{\rmlatinfont\textsuperscript{§~\theparCount}}}
	\pend% ending standard par
      ‚{\tiny $_{lb}$}‚

	  
	  \pstart \leavevmode% starting standard par
	\textbf{न चेत्या}दि । \textbf{न च भाव‚विशेष}स्य क‚स्य‚चित्\textbf{स्व‚भावो} विनाशः । किं कार‚णं‚{\tiny $_{lb}$}‚ [।] \textbf{त‚स्यो}त्त‚र‚त्र \textbf{निषेत्स्य‚मान‚त्वात् । त‚स्माद् भाव‚मात्र‚स्व‚भावः स्याद्} विनाशः ।‚{\tiny $_{lb}$}‚ स‚त्तामात्र‚स्व‚भावः स्यात् । \textbf{तेन} कार‚णेन \textbf{श‚ब्दोन्यो वा स‚र्व एव भावः स‚त्ताभाज‚नः}‚{\tiny $_{lb}$}‚ स‚त्ताधारः स‚न्निति याव‚त् । \textbf{अनात्य‚न्तिक इति सिद्धं} ।
	{\color{gray}{\rmlatinfont\textsuperscript{§~\theparCount}}}
	\pend% ending standard par
      ‚{\tiny $_{lb}$}‚

	  
	  \pstart \leavevmode% starting standard par
	\textbf{न सिद्ध}मिति प‚{\tiny $_{४}$}‚रः । किं कार‚णं [।] \textbf{त‚स्यैव विनाश‚स्याप‚र‚ज‚न्मासिद्धेः} ।‚{\tiny $_{lb}$}‚ प‚र‚स्माज्ज‚न्म प‚र‚ज‚न्म । न प‚र‚ज‚न्माप‚र‚ज‚न्म । त‚स्यासिद्धेः । विनाश‚स्याहेतुक‚त्वा‚{\tiny $_{lb}$}‚सिद्धेरिति याव‚त् ।
	{\color{gray}{\rmlatinfont\textsuperscript{§~\theparCount}}}
	\pend% ending standard par
      ‚{\tiny $_{lb}$}‚

	  
	  \pstart \leavevmode% starting standard par
	\textbf{त‚था ह्य‚ग्निना काष्ठं} द‚ग्धं । \textbf{द‚ण्डेन घ‚टो} भ‚ग्न \textbf{इति विनाश‚हेत‚वो}ऽग्न्याद‚यः‚{\tiny $_{lb}$}‚ काष्ठादीना\textbf{म्भावानान्दृश्य‚न्ते} । त‚था ह्य‚ग्न्यादिभावे काष्ठादीनां नाश‚स्त‚द‚भावे‚{\tiny $_{lb}$}‚ चानाश इ‚{\tiny $_{५}$}‚त्य\textbf{न्व‚य‚व्य‚तिरेकानुविधानं} नाश‚स्यास्ति । एत‚च्च \textbf{हेतुत‚द्व‚तो}र्हेतुम‚तो‚{\tiny $_{lb}$}‚\textbf{र्ल‚क्ष‚ण‚माहुः} । त‚दुक्तं ।
	{\color{gray}{\rmlatinfont\textsuperscript{§~\theparCount}}}
	\pend% ending standard par
      ‚{\tiny $_{lb}$}‚
	  \bigskip
	  \begingroup
	
	    
	    \stanza[\smallbreak]
	  {\normalfontlatin\large ``\qquad}अभिघाताग्निसंयोग‚नाश‚प्र‚त्य‚य‚स‚न्निधिं ।&‚{\tiny $_{lb}$}‚विना, संस‚र्गितां याति विनाशो न घ‚टादिभिरिति [।]\edtext{}{\edlabel{pvsvt_511-2}\label{pvsvt_511-2}\lemma{टादिभिरिति}\Bfootnote{नो ऋओओत्नोते ऋओउन्द् ओन् थिस् प‚गे!}}{\normalfontlatin\large\qquad{}"}\&[\smallbreak]
	  
	  
	  
	  \endgroup
	‚{\tiny $_{lb}$}‚

	  
	  \pstart \leavevmode% starting standard par
	\textbf{ने}त्यादिना प्र‚तिपेध‚ति । नाग्न्याद‚यः काष्ठादेर्विना क‚र‚णाल्लोके विनाश‚{\tiny $_{lb}$}‚हेत‚वः प्र‚तीय‚न्ते । किन्तु पूर्व\textbf{पूर्व‚स्य} काष्ठादिक्ष‚ण‚{\tiny $_{६}$}‚स्य \textbf{स्व‚र‚स‚निरोधे} स्व‚य‚मेव निरोधे‚{\tiny $_{lb}$}‚ स‚ति । \textbf{अन्य‚स्यो}त्त‚र‚स्य क्ष‚ण‚स्य \textbf{निकृत}स्य भ‚स्मादि\textbf{रूप‚स्योत्प‚त्ते}र‚ग्न्याद‚यः काष्ठा‚{\tiny $_{lb}$}‚दीनां विनाश‚हेत‚वः प्र‚ज्ञाय‚न्ते न तु विनाश‚स्य क‚र‚णात् । कुतः पुन‚स्त‚स्य विकृत‚{\tiny $_{lb}$}‚स्योत्प‚त्तिरित्याह । \textbf{विशिष्टे}त्यादि । \textbf{विशिष्टः प्र‚त्य}योऽग्न्यादिः स‚ह‚कारी‚{\tiny $_{lb}$}‚ \textbf{त‚दाश्र‚येण} ।
	{\color{gray}{\rmlatinfont\textsuperscript{§~\theparCount}}}
	\pend% ending standard par
      ‚{\tiny $_{lb}$}‚

	  
	  \pstart \leavevmode% starting standard par
	अभ्युप‚ग‚म्या‚{\tiny $_{७}$}‚पि \textbf{ब्रूमः [।] अस्तु वाग्निः काष्ठ‚विनाश‚हेतुः [।] स नाशोग्नि- \leavevmode\ledsidenote{\textenglish{181b/PSVTa}}‚{\tiny $_{lb}$}‚ ‚{\tiny $_{lb}$}‚ ‚{\tiny $_{lb}$}‚ \leavevmode\ledsidenote{\textenglish{512/s}}ज‚न्मा} । अग्नेर्ज‚न्म य‚स्येति विग्र‚हः । \textbf{किं काष्ठ‚मेवाहोस्वि}त्काष्ठा\textbf{द‚र्थान्त}र‚न्त‚त्रा\textbf{ग्ने}‚{\tiny $_{lb}$}‚र्विनाश‚क‚म् [।]
	{\color{gray}{\rmlatinfont\textsuperscript{§~\theparCount}}}
	\pend% ending standard par
      ‚{\tiny $_{lb}$}‚

	  
	  \pstart \leavevmode% starting standard par
	हेतोस्स‚काशान्नाश‚स्या\textbf{र्थान्त‚र‚स्योत्प‚त्तौ भ‚वेत् काष्ठ‚स्य द‚र्श‚नं} । किं कार‚ण‚म्‚{\tiny $_{lb}$}‚ [।] \textbf{अविनाशात्} । काष्ठ‚स्य \textbf{किमित्य‚र्थान्त‚रात्} काष्ठा\textbf{द‚र्थान्त‚र‚स्य} नाश‚स्योत्प‚त्तौ‚{\tiny $_{lb}$}‚ \textbf{ज‚न्म‚नि} स‚ति \textbf{काष्ठ‚म‚भूतं} विन‚ष्टं \textbf{नाम} ।‚{\tiny $_{१}$}‚ नैवाभूत‚मिति याव‚त् ।
	{\color{gray}{\rmlatinfont\textsuperscript{§~\theparCount}}}
	\pend% ending standard par
      ‚{\tiny $_{lb}$}‚

	  
	  \pstart \leavevmode% starting standard par
	य‚दि नामाविन‚ष्ट‚न्त‚थाप्य‚र्थान्त‚रोत्प‚त्या त‚स्य द‚र्श‚न‚मिति चेदाह । \textbf{न दृश्य‚ते‚{\tiny $_{lb}$}‚ वेति} [।] किमिति \textbf{न दृश्य‚ते} । दृश्य‚त एव । य‚दि त्व‚र्थान्त‚रोत्प‚त्यार्थान्त‚रं विन‚ष्टं‚{\tiny $_{lb}$}‚ \textbf{न दृश्य‚ते वा} । त\textbf{दातिप्र‚स‚ङ्गो ह्येवं स्यात्} । अर्थान्त‚र‚स्य य‚स्य क‚स्य‚चिदुत्त्प‚त्या‚{\tiny $_{lb}$}‚ स‚र्व‚म‚भूतं स्यात् । न वा दृश्येत । \textbf{स एव} प‚दार्थोग्निज‚न्मा । न स‚र्वः । \textbf{अस्य}‚{\tiny $_{lb}$}‚ काष्ठ‚स्य \textbf{विनाशो‚{\tiny $_{२}$}‚} लोके विनाश‚रूप‚त‚या प्र‚तीते\textbf{रिति} याव‚त् ।
	{\color{gray}{\rmlatinfont\textsuperscript{§~\theparCount}}}
	\pend% ending standard par
      ‚{\tiny $_{lb}$}‚

	  
	  \pstart \leavevmode% starting standard par
	एत‚देव ग्र‚ह‚ण‚क‚वाक्यं \textbf{य‚दी}त्यादिना व्याच‚ष्टे । \textbf{य‚दि स एवाग्निज‚न्मा} काष्ठ‚{\tiny $_{lb}$}‚स्या\textbf{भावो} विनाशः । त‚दिति त‚स्मा\textbf{दिदं} काष्ठ‚म\textbf{भूत‚त्वाद्} विन‚ष्ट‚त्वा\textbf{न्न दृश्य‚त}‚{\tiny $_{lb}$}‚ इति ।
	{\color{gray}{\rmlatinfont\textsuperscript{§~\theparCount}}}
	\pend% ending standard par
      ‚{\tiny $_{lb}$}‚

	  
	  \pstart \leavevmode% starting standard par
	\textbf{भ‚व}त्वित्यादिना प्र‚तिष‚ध‚ति । \textbf{भ‚व‚तु त‚स्या}ग्निज‚न्म‚नोर्थ‚स्\textbf{येद‚न्नाम} संज्ञा‚{\tiny $_{lb}$}‚ य‚दिद\textbf{म‚भाव इति । त‚थापि} नाम‚मात्रेण \textbf{क‚थ‚म‚न्योन्य‚स्य}‚{\tiny $_{३}$}‚ विनाशः । \textbf{न} हि \textbf{क‚स्य‚{\tiny $_{lb}$}‚चिद‚र्थ}स्याग्निज‚न्म‚नो विनाश इति \textbf{नाम‚क‚र‚ण‚मात्रेण काष्ठं न दृश्य‚त इति युक्तं} ।
	{\color{gray}{\rmlatinfont\textsuperscript{§~\theparCount}}}
	\pend% ending standard par
      ‚{\tiny $_{lb}$}‚

	  
	  \pstart \leavevmode% starting standard par
	न‚नु लोक‚प्र‚तीत‚त्वाद्विनाश एवासौ न त‚स्य विनाश इति नाम‚क‚र‚ण‚मात्र‚{\tiny $_{lb}$}‚मित्य‚त आह ।
	{\color{gray}{\rmlatinfont\textsuperscript{§~\theparCount}}}
	\pend% ending standard par
      ‚{\tiny $_{lb}$}‚

	  
	  \pstart \leavevmode% starting standard par
	\textbf{न चान्यः} प‚दार्थो\textbf{ऽन्य‚स्य विनाशोऽतिप्र‚संगात्} । स‚र्वे प‚दार्थाः काष्ठ‚स्य‚{\tiny $_{lb}$}‚ विनाशः स्याद् [।] एत‚च्चान‚न्त‚रोक्त‚मेव स्म‚र‚य‚ति ।
	{\color{gray}{\rmlatinfont\textsuperscript{§~\theparCount}}}
	\pend% ending standard par
      ‚{\tiny $_{lb}$}‚‚{\tiny $_{lb}$}‚\textsuperscript{\textenglish{513/s}}

	  
	  \pstart \leavevmode% starting standard par
	एव‚म्म‚न्य‚ते । य‚था‚{\tiny $_{४}$}‚ स‚र्व‚प‚दार्थानाम‚र्थान्त‚र‚त्वात् न काष्ठ‚विनाश‚रूप‚त‚या प्र‚ती‚{\tiny $_{lb}$}‚तिस्त‚थाऽग्निकृत‚स्याप्य‚र्थान्त‚र‚त्वान्न काष्ठ‚विनाश‚रूप‚त‚या प्र‚तीतिः स्यात् ।
	{\color{gray}{\rmlatinfont\textsuperscript{§~\theparCount}}}
	\pend% ending standard par
      ‚{\tiny $_{lb}$}‚

	  
	  \pstart \leavevmode% starting standard par
	स्यादेत‚द् [।] य‚द्य‚र्थान्त‚र‚त्वाद‚ग्निकृत‚स्यार्थ‚स्य न विनाश‚रूप‚ता । धूम‚स्यापि‚{\tiny $_{lb}$}‚ त‚र्ह्य‚ग्निकार्य‚ता न स्याद‚र्थान्त‚र‚त्वाद् घ‚ट‚व‚त् । भ‚व‚ति च त‚द‚र्थान्त‚र‚त्वाविशेषे‚{\tiny $_{lb}$}‚प्य‚ग्निकृत‚स्य काष्ठ‚विनाश‚रूप‚ता भ‚वि‚{\tiny $_{५}$}‚ष्य‚तीति [।]
	{\color{gray}{\rmlatinfont\textsuperscript{§~\theparCount}}}
	\pend% ending standard par
      ‚{\tiny $_{lb}$}‚

	  
	  \pstart \leavevmode% starting standard par
	अत आह । \textbf{अविशेषात् । त‚स्या}ग्निकृत‚स्य व‚स्तुभूत‚स्य काष्ठाद\textbf{र्थान्त‚र‚त्वेन‚{\tiny $_{lb}$}‚ त‚द‚न्येभ्यो} घ‚टादिभ्यो \textbf{विशेषाभावात्} । क‚थ‚म्विनाश‚रूप‚ता निवृत्तिरूप‚त्वाद् विना‚{\tiny $_{lb}$}‚श‚स्येति भावः । धूम‚स्य त्व‚र्थान्त‚र‚त्वेप्य‚ग्निकार्य‚त्वं युक्त‚मेव । अर्थान्त‚र‚स्याग्नि‚{\tiny $_{lb}$}‚कार्य‚त्वेन स‚ह विरोधाभावादिति य‚त्किञ्चिदेत‚त् ।
	{\color{gray}{\rmlatinfont\textsuperscript{§~\theparCount}}}
	\pend% ending standard par
      ‚{\tiny $_{lb}$}‚

	  
	  \pstart \leavevmode% starting standard par
	\textbf{काष्ठेग्निकृतः स्व‚भावो नाशो न}‚{\tiny $_{६}$}‚ स‚र्वः घ‚टादिस्त‚तो नातिप्र‚संग \textbf{इति चेत्} ।
	{\color{gray}{\rmlatinfont\textsuperscript{§~\theparCount}}}
	\pend% ending standard par
      ‚{\tiny $_{lb}$}‚

	  
	  \pstart \leavevmode% starting standard par
	\textbf{काष्ठे}ऽय‚म‚ग्निकृतो विनाश \textbf{इति} काष्ठ‚विनाश‚योः \textbf{क‚स्स‚म्ब‚न्धः} । प‚र‚स्प‚र‚म‚नु‚{\tiny $_{lb}$}‚प‚कार्योप‚कार‚त्वात् । नैव स‚म्ब‚न्धोस्ति ।
	{\color{gray}{\rmlatinfont\textsuperscript{§~\theparCount}}}
	\pend% ending standard par
      ‚{\tiny $_{lb}$}‚

	  
	  \pstart \leavevmode% starting standard par
	काष्ठ‚माश्र‚य आश्र‚योस्यास्तीत्याश्र‚यी विनाशः । त‚त \textbf{आश्र‚याश्र‚यिस‚म्ब‚न्धो‚{\tiny $_{lb}$}‚स्तीति चेत्} ।
	{\color{gray}{\rmlatinfont\textsuperscript{§~\theparCount}}}
	\pend% ending standard par
      ‚{\tiny $_{lb}$}‚

	  
	  \pstart \leavevmode% starting standard par
	\textbf{नै}त‚देवं । त‚स्याश्र‚याश्र‚यिस‚म्ब‚न्ध‚स्य \textbf{निषेत्स्य‚मान‚त्वात्} ।
	{\color{gray}{\rmlatinfont\textsuperscript{§~\theparCount}}}
	\pend% ending standard par
      ‚{\tiny $_{lb}$}‚

	  
	  \pstart \leavevmode% starting standard par
	विनाशो‚{\tiny $_{७}$}‚ ज‚न्यः । त‚स्य काष्ठं ज‚न‚कं । त‚तो नाश‚काष्ठ‚यो\textbf{र्ज‚न्य‚ज‚न‚क‚भाव}- \leavevmode\ledsidenote{\textenglish{182a/PSVTa}}‚{\tiny $_{lb}$}‚ स‚म्ब‚न्ध\textbf{श्चेत्} ।
	{\color{gray}{\rmlatinfont\textsuperscript{§~\theparCount}}}
	\pend% ending standard par
      ‚{\tiny $_{lb}$}‚

	  
	  \pstart \leavevmode% starting standard par
	त\textbf{दाग्नेरिति किं} । अग्नेः स‚काशान्नाशो भ‚व‚तीति किमुच्य‚ते । किं कार‚णं‚{\tiny $_{lb}$}‚ [।] \textbf{काष्ठादेव} त‚स्य नाश‚स्य \textbf{भावा}दुत्प‚त्तेः ।
	{\color{gray}{\rmlatinfont\textsuperscript{§~\theparCount}}}
	\pend% ending standard par
      ‚{\tiny $_{lb}$}‚

	  
	  \pstart \leavevmode% starting standard par
	\textbf{त‚द‚पेक्षा}द‚ग्न्य‚पेक्षात् काष्ठान्नाश‚स्यो\textbf{त्प‚त्तेर‚दोषः} । अग्निकृतो नाशो न स्या‚{\tiny $_{lb}$}‚दिति यो दोष उक्तः स नास्ती\textbf{ति चेत्} ।
	{\color{gray}{\rmlatinfont\textsuperscript{§~\theparCount}}}
	\pend% ending standard par
      ‚{\tiny $_{lb}$}‚

	  
	  \pstart \leavevmode% starting standard par
	व‚ह्नेः स‚काशाद\textbf{\textbf{न}तिश‚य‚लाभि‚{\tiny $_{१}$}‚नः} काष्ठ‚स्य व‚ह्निम्प्र‚ति \textbf{कापेक्षा} । नैव‚{\tiny $_{lb}$}‚ काचित् । व‚ह्नेः स‚काशात् काष्ठ‚स्यातिश‚य\textbf{लाभे वाऽप‚र}स्य द्वितीय‚स्य \textbf{काष्ठ}‚{\tiny $_{lb}$}‚स्यातिश‚य‚संज्ञ‚क‚स्य \textbf{ज‚न्म स्यात्} । त‚था च \textbf{पूर्व}काष्ठ\textbf{म‚प्र‚च्युतिकार}णं । नास्य प्र‚च्यु‚{\tiny $_{lb}$}‚तिकार‚ण‚म‚स्तीति विग्र‚हः । \textbf{त‚थैव} प्राग्व‚द् \textbf{दृश्य‚ते} ।
	{\color{gray}{\rmlatinfont\textsuperscript{§~\theparCount}}}
	\pend% ending standard par
      ‚{\tiny $_{lb}$}‚

	  
	  \pstart \leavevmode% starting standard par
	स्यादेत‚द् [।] य‚त एवाग्नेर‚तिश‚य‚व‚तो द्वितीय‚स्य काष्ठ‚स्य ज‚न्म । \textbf{त‚त एवाग्नेः‚{\tiny $_{lb}$}‚ ‚{\tiny $_{lb}$}‚ \leavevmode\ledsidenote{\textenglish{514/s}}पूर्व‚स्य} का‚{\tiny $_{२}$}‚ष्ठ‚स्य विनाश इति चेत् ।
	{\color{gray}{\rmlatinfont\textsuperscript{§~\theparCount}}}
	\pend% ending standard par
      ‚{\tiny $_{lb}$}‚

	  
	  \pstart \leavevmode% starting standard par
	\textbf{कः पूर्वेण} काष्ठेनास्य व‚ह्निकृत‚स्य विनाश‚स्य \textbf{स‚म्ब‚न्ध इति स एव प्र‚संगः} ।‚{\tiny $_{lb}$}‚ काष्ठ इति कः स‚म्ब‚न्ध इत्य‚न‚न्त‚र‚मेवोक्तः ।
	{\color{gray}{\rmlatinfont\textsuperscript{§~\theparCount}}}
	\pend% ending standard par
      ‚{\tiny $_{lb}$}‚

	  
	  \pstart \leavevmode% starting standard par
	अथाप्याश्र‚याश्र‚यिभावादिक‚माश्रीय‚ते । त‚दा त‚स्य निषेत्स्य‚मान‚त्वादित्यादि‚{\tiny $_{lb}$}‚ स‚र्वं पुन‚राव‚र्त्त‚ते इत्य‚प‚र्य‚व‚सान‚श्च प्र‚स‚ङ्गः स्यात् ।
	{\color{gray}{\rmlatinfont\textsuperscript{§~\theparCount}}}
	\pend% ending standard par
      ‚{\tiny $_{lb}$}‚

	  
	  \pstart \leavevmode% starting standard par
	त‚दिति त‚स्मा\textbf{द‚व‚श्यं विनाश‚स‚म्ब‚न्ध}स्य \textbf{योग्य‚ङ् काष्ठ‚स्यो‚{\tiny $_{३}$}‚त्त‚र‚म‚तिश‚यं‚{\tiny $_{lb}$}‚ प्र‚त्युप‚कुर्वाणोग्निर‚पूर्व‚मेव} काष्ठ\textbf{ञ्ज‚न‚य‚तीति पूर्वं} काष्ठ‚न्त‚द‚व‚स्थ‚न्दृश्येतेत्युप‚संहारः ।
	{\color{gray}{\rmlatinfont\textsuperscript{§~\theparCount}}}
	\pend% ending standard par
      ‚{\tiny $_{lb}$}‚

	  
	  \pstart \leavevmode% starting standard par
	किञ्च । \textbf{काष्ठ‚विनाश इति काष्ठाभाव उच्य‚ते [।] न चाभावः कार्यः} । विधिना‚{\tiny $_{lb}$}‚ कार्य‚त्वोप‚ग‚मे त‚स्य भाव‚त्व‚प्र‚स‚ङ्गात् । त‚स्माद‚भावं क‚रोति भाव‚न्न क‚रोतीति ।‚{\tiny $_{lb}$}‚ क्रियाप्र‚तिषेध‚मात्रं । त‚था च \textbf{त‚त्कारी} चाभाव‚कारी वाकार एव क्रियाप्र‚ति‚{\tiny $_{४}$}‚‚{\tiny $_{lb}$}‚षेध‚मात्र‚त्वा\textbf{दिति} कृत्वा काष्ठ‚विनाशेन व‚ह्न्यादि\textbf{र‚न‚पेक्ष‚णीय इत्युक्तं सामान्य‚{\tiny $_{lb}$}‚त‚द्व‚तोराधाराधेय‚चिन्ता}स्थाने ।
	{\color{gray}{\rmlatinfont\textsuperscript{§~\theparCount}}}
	\pend% ending standard par
      ‚{\tiny $_{lb}$}‚

	  
	  \pstart \leavevmode% starting standard par
	किञ्च [।] \textbf{स्व‚भावाभाव‚स्य} काष्ठादिस्व‚भाव‚स्य योऽभावो नाश‚स्त‚स्य त‚तः‚{\tiny $_{lb}$}‚ काष्ठादिस्व‚भावाद् \textbf{भेदे}भ्युप‚ग‚म्य‚माने । त‚तोर्थान्त‚राद‚भावात् काष्ठादिर्भावो‚{\tiny $_{lb}$}‚ निव‚र्त्त‚ते । त‚त‚स्त‚स्माद‚भाव\textbf{न्निव‚र्त्त‚मान‚स्य} काष्ठादेः \textbf{स्व‚भाव एव स‚म‚र्थितः‚{\tiny $_{५}$}‚‚{\tiny $_{lb}$}‚ स्यात्} । अस‚तो निव‚र्त्त‚मान‚स्य स‚त्त्व‚मेव स‚म‚र्थितं स्यादिति कृत्वा क‚थ‚म‚ग्न्यादिकृतेन‚{\tiny $_{lb}$}‚ विनाशेन काष्ठादि\textbf{र‚भूतो नाम} ।
	{\color{gray}{\rmlatinfont\textsuperscript{§~\theparCount}}}
	\pend% ending standard par
      ‚{\tiny $_{lb}$}‚

	  
	  \pstart \leavevmode% starting standard par
	य‚त एव\textbf{न्त‚स्मान्न अन्योन्य‚स्य विनाशः} । अर्थान्त‚र‚म‚र्थान्त‚र‚स्य न विनाश‚{\tiny $_{lb}$}‚ इत्य‚र्थः ।
	{\color{gray}{\rmlatinfont\textsuperscript{§~\theparCount}}}
	\pend% ending standard par
      ‚{\tiny $_{lb}$}‚

	  
	  \pstart \leavevmode% starting standard par
	अभ्युप‚ग‚म्यापि ब्रूमः [।] अस्त्व‚न्यो विनाश‚स्त‚स्मिन्न‚र्थान्त‚रे व‚ह्निकृते‚{\tiny $_{lb}$}‚ काष्ठ‚न्त‚द‚व‚स्थ‚मेवेति \textbf{क‚स्मान्न दृश्य‚ते} । एत‚देव साध‚य‚न्नाह । \textbf{कोय‚{\tiny $_{६}$}‚म्विरोधः ।‚{\tiny $_{lb}$}‚ ‚{\tiny $_{lb}$}‚ \leavevmode\ledsidenote{\textenglish{515/s}}अग्निज‚नित‚स्य विनाश‚स्यार्थान्त‚र}स्य यो \textbf{भावः} । य‚च्च \textbf{काष्ठ}स्य \textbf{द‚र्श‚न}न्त\textbf{योः ।‚{\tiny $_{lb}$}‚ त‚था च काष्ठ‚न्त‚द‚व‚स्थं} दृश्येत ।
	{\color{gray}{\rmlatinfont\textsuperscript{§~\theparCount}}}
	\pend% ending standard par
      ‚{\tiny $_{lb}$}‚

	  
	  \pstart \leavevmode% starting standard par
	\hphantom{.}न‚नुयोसाव‚र्थान्त‚रं भावो व‚ह्निकृतः स काष्ठ‚विनाशः । विनाश‚रूप‚त‚या प्र‚ती‚{\tiny $_{lb}$}‚तेः । विनाश‚श्चाभावो य‚श्च काष्ठाभावः । स काष्ठ‚विरोधिरूप एव क्रिय‚ते । न‚{\tiny $_{lb}$}‚ चाय‚म‚र्थान्त‚र‚त्वाद् घ‚ट‚व‚द् विरोधिरूप‚त‚या क‚र्त्तुम‚श‚क्यः‚{\tiny $_{७}$}‚ [।] न हि घ‚ट‚व‚द‚र्था- \leavevmode\ledsidenote{\textenglish{182b/PSVTa}}‚{\tiny $_{lb}$}‚ न्त‚र‚त्वाद् धूमोग्निकार्यो न भ‚व‚ति । त‚स्माद् य‚थार्थान्त‚र‚भूतोपि धूमोग्निना क्रि‚{\tiny $_{lb}$}‚य‚ते, त‚था विरोधिरूपो विनाशः क्रिय‚ते । य‚योश्च प‚र‚स्प‚र‚प‚रिहारेण विरोध‚{\tiny $_{lb}$}‚स्त‚योरेक‚भाव एवाप‚र‚स्याद‚र्श‚न‚मिति क‚थ‚म‚ग्निकृत‚स्यार्थान्त‚र‚स्य विनाश‚संज्ञि‚{\tiny $_{lb}$}‚त‚स्य विरोधिनो भावे काष्ठ‚स्य द‚र्श‚नं स्यादित्युच्य‚त इति श ङ्क रः ।\edtext{\textsuperscript{*}}{\edlabel{pvsvt_515-1}\label{pvsvt_515-1}\lemma{*}\Bfootnote{नो ऋओओत्नोते ऋओउन्द् ओन् थिस् प‚गे!}}
	{\color{gray}{\rmlatinfont\textsuperscript{§~\theparCount}}}
	\pend% ending standard par
      ‚{\tiny $_{lb}$}‚

	  
	  \pstart \leavevmode% starting standard par
	त‚द‚युक्तं ‚{\tiny $_{१}$}‚ [।] य‚तोर्थान्त‚र‚स्याग्निकार्य‚त्वेन स‚ह विरोधाभावाद् [।] धूम‚{\tiny $_{lb}$}‚म्यार्थान्त‚र‚त्वेप्य‚ग्निकार्य‚त्व‚म‚विरुद्ध‚मेव । व‚ह्निकृत‚स्य त्व‚र्थान्त‚र‚स्य भ‚व‚न‚ध‚र्म‚{\tiny $_{lb}$}‚त‚या भाव‚रूप‚ता । य‚श्च भावः स क‚थ‚म‚भावो वः [।] विरोधाद् विनाश‚श्चाभाव‚{\tiny $_{lb}$}‚ इष्य‚ते । त‚तोर्थान्त‚र‚भावेन विरुद्धो विनाशः । न चार्थान्त‚र‚स्यापि विनाश‚रूप‚{\tiny $_{lb}$}‚त‚या प्र‚तिभास‚नात् । काष्ठादिविनाश‚रू‚{\tiny $_{२}$}‚प‚ता । स्व‚र‚स‚निरोधो हि निमित्त‚{\tiny $_{lb}$}‚म्विनाश‚प्र‚तिभासे ।
	{\color{gray}{\rmlatinfont\textsuperscript{§~\theparCount}}}
	\pend% ending standard par
      ‚{\tiny $_{lb}$}‚

	  
	  \pstart \leavevmode% starting standard par
	स्व‚र‚स‚निरोधान‚भ्युप‚ग‚मे तु क‚थ‚म‚र्थान्त‚र‚स्यापि विनाश‚रूप‚त‚या प्र‚तिभासो‚{\tiny $_{lb}$}‚ भाव‚रूप‚त्वादित्यादावेवोक्तं । त‚त्क‚थ‚मुच्य‚ते [।] विनाशाख्य‚स्यार्थान्त‚र‚स्य विरो‚{\tiny $_{lb}$}‚धिनः कृत‚क‚त्वात् काष्ठ‚स्याद‚र्श‚न‚मिति । नीरूप‚त्वे तु विनाश‚स्य स्याद् भावेन‚{\tiny $_{lb}$}‚ स‚हायं विरोधः [।] किन्तु त‚दाप्य‚र्थान्त‚र‚{\tiny $_{३}$}‚त्वं हेतुज‚न्य‚त्वं चास्य न स्यान्नीरूप‚त्वा‚{\tiny $_{lb}$}‚देव । त‚स्माद‚ग्निनार्थान्त‚र‚स्य क‚र‚णे काष्ठ‚न्त‚द‚व‚स्थं दृश्येत ।
	{\color{gray}{\rmlatinfont\textsuperscript{§~\theparCount}}}
	\pend% ending standard par
      ‚{\tiny $_{lb}$}‚

	  
	  \pstart \leavevmode% starting standard par
	तेनाग्निकृतेनार्थान्त‚रेण \textbf{प‚रिग्र‚ह‚तः} स्वीकारात् काष्ठं न दृश्य‚त इति \textbf{चेत्} ।
	{\color{gray}{\rmlatinfont\textsuperscript{§~\theparCount}}}
	\pend% ending standard par
      ‚{\tiny $_{lb}$}‚

	  
	  \pstart \leavevmode% starting standard par
	एव स‚ति त‚द‚र्थान्त‚रं काष्ठ‚स्याव‚र‚ण‚मिति प्राप्तं । त‚च्च न युक्तं [।] \textbf{य‚तो‚{\tiny $_{lb}$}‚ न तेना}र्थान्त‚रेण काष्ठ‚स्या\textbf{व‚र‚णं} स‚म्भ‚व‚ति ।
	{\color{gray}{\rmlatinfont\textsuperscript{§~\theparCount}}}
	\pend% ending standard par
      ‚{\tiny $_{lb}$}‚

	  
	  \pstart \leavevmode% starting standard par
	\textbf{य‚दी}त्यादिना व्याच‚ष्टे । \textbf{तेनार्थान्त‚रेणा}ग्नि‚{\tiny $_{४}$}‚ना कृतेन । \textbf{त‚दि}त्य‚ग्निकृत‚म‚{\tiny $_{lb}$}‚र्थान्त‚रं । न \textbf{चैत‚दाव}र‚ण‚क‚ल्पितं \textbf{युक्तं} । य‚स्मादा\textbf{व‚र‚णं हि} । आव्रिय‚माणे\textbf{र्थ‚द‚र्श‚न‚{\tiny $_{lb}$}‚म्विब‚ध्नीयात् । नाभिघातावीनि द्र‚व्य‚साम‚र्थ्यानि} विब‚ध्नीयात् । अन्ध‚कारा‚{\tiny $_{lb}$}‚‚{\tiny $_{lb}$}‚ ‚{\tiny $_{lb}$}‚ \leavevmode\ledsidenote{\textenglish{516/s}}वृतानां घ‚टादीनाम‚भिधातादिद‚र्श‚नात् । त‚त्र स्व‚देशे प‚र‚स्योत्प‚त्तिविब‚न्धोभि‚{\tiny $_{lb}$}‚घातः । आदिश‚ब्दाद् ग‚न्ध‚र‚सादिप‚रिग्र‚हः । अथाव‚र‚{\tiny $_{५}$}‚णं स‚र्व‚साम‚र्थ्यं काष्ठ‚स्य‚{\tiny $_{lb}$}‚ निब‚ध्नीयात् । त‚दा \textbf{स‚र्व‚प्र‚तिब‚न्धे चा}भ्युप‚ग‚म्य‚माने । \textbf{न-त्व‚नेनै}वाव‚र‚णेन काष्ठ‚{\tiny $_{lb}$}‚\textbf{न्नाशितं स्या}न्न व‚ह्निना । किं कार‚णं [।] तेनैवास्य काष्ठ‚स्य \textbf{स‚र्व‚श‚क्तिप्र‚च्या‚{\tiny $_{lb}$}‚व‚नात्} । त‚था च स‚ति \textbf{पुन‚स्त‚त्राप्य‚ग्निकृ}तेर्थान्त‚रे नाश‚हेता\textbf{व‚ग्नाविव प्र‚संगात्} ।‚{\tiny $_{lb}$}‚ काष्ठ‚विनाशं प्र‚ति योग्नौ दोषो विस्त‚रेणोक्तः सोर्थान्त‚रेणाप्य‚{\tiny $_{६}$}‚ग्निकृतेन काष्ठ‚{\tiny $_{lb}$}‚नाशे क्रिय‚माणे स्यात् । त‚था \textbf{चान‚व‚स्था} । तेनाप्य‚र्थान्त‚रेणाग्निकृतेन नाशेना‚{\tiny $_{lb}$}‚प‚र‚म‚र्थान्त‚र‚न्नाशाख्यं क‚र्त्त‚व्य‚न्तेनाप्य‚प‚र‚मित्य‚न‚व‚स्था स्याद् ।
	{\color{gray}{\rmlatinfont\textsuperscript{§~\theparCount}}}
	\pend% ending standard par
      ‚{\tiny $_{lb}$}‚

	  
	  \pstart \leavevmode% starting standard par
	अथ मा भूदेष दोष इत्य‚प्र‚च्युता एव काष्ठ‚स्याभिधातादिसाम‚र्थ्याद‚यः ।
	{\color{gray}{\rmlatinfont\textsuperscript{§~\theparCount}}}
	\pend% ending standard par
      ‚{\tiny $_{lb}$}‚

	  
	  \pstart \leavevmode% starting standard par
	त‚दा\textbf{प्य‚प्र‚च्युतेषु वास्य} काष्ठ‚स्या\textbf{भिघात‚साम‚र्थ्यादिषु । स‚ता वा} तेनान्येनाग्नि‚{\tiny $_{lb}$}‚\leavevmode\ledsidenote{\textenglish{183a/PSVTa}} ज‚नि‚{\tiny $_{७}$}‚तेन काष्ठ‚स्य \textbf{किं विनाशितं} येन त‚दाव‚र‚ण‚न्त‚था च काष्ठं दृश्येत ॥ \textbf{य‚दि‚{\tiny $_{lb}$}‚ चे}त्यादि । [।] \textbf{अग्नेः स‚मुद्भ}वो य‚स्येति विग्र‚हः । अग्निस‚मुद्भूतेन \textbf{विनाशा‚{\tiny $_{lb}$}‚ख्ये}ना\textbf{र्थेन प‚रिग्र‚हा}दित्य‚र्थः । त‚दा \textbf{विनाश‚स्य विनाशित्वं स्यात्} । किं कार‚ण‚म्‚{\tiny $_{lb}$}‚ [।] \textbf{उत्प‚त्तेः} । उत्प‚त्तिम‚त्वाद् विनाशोपि काष्ठ‚व‚द् विनाशी स्यादिति याव‚त् ।‚{\tiny $_{lb}$}‚ \textbf{त‚तो} विनाश‚विनाशात् \textbf{पुनः काष्ठ‚स्य द‚र्श‚नं} स्यात् ।
	{\color{gray}{\rmlatinfont\textsuperscript{§~\theparCount}}}
	\pend% ending standard par
      ‚{\tiny $_{lb}$}‚

	  
	  \pstart \leavevmode% starting standard par
	\textbf{अ‚{\tiny $_{१}$}‚व‚श्य}मित्यादिना व्याच‚ष्टे । \textbf{उत्प‚त्तिम‚ता} स‚ता काष्ठ\textbf{विनाशेनाव‚श्यं विन‚{\tiny $_{lb}$}‚ष्ट‚व्यं । त‚स्मिन्} काष्ठ‚नाशे \textbf{विन‚ष्टे} स‚ति \textbf{पुनः काष्ठादीनामुन्म‚ज्ज‚नं स्यात्} ।‚{\tiny $_{lb}$}‚ प्रादुर्भावो भ‚वेत् ।
	{\color{gray}{\rmlatinfont\textsuperscript{§~\theparCount}}}
	\pend% ending standard par
      ‚{\tiny $_{lb}$}‚

	  
	  \pstart \leavevmode% starting standard par
	\textbf{ह‚न्तुघाते}त्यादिना प‚र‚माशंक‚ते । चैत्र‚स्य यो ह‚न्ता त‚स्य \textbf{ह‚न्तुर्घाते} स‚ति \textbf{य‚था} ।‚{\tiny $_{lb}$}‚ ह‚त‚स्य \textbf{चैत्र}स्या\textbf{पुन‚र्भावः\textbf{न}} पुन‚र‚नुत्प‚त्तिः । \textbf{अत्रा}पि काष्ठ‚नाशे विन‚ष्टे\textbf{प्येवं} काष्ठ‚{\tiny $_{lb}$}‚स्यापुन‚र्भाव \textbf{इ‚{\tiny $_{२}$}‚ति चेत्} ।
	{\color{gray}{\rmlatinfont\textsuperscript{§~\theparCount}}}
	\pend% ending standard par
      ‚{\tiny $_{lb}$}‚‚{\tiny $_{lb}$}‚\textsuperscript{\textenglish{517/s}}

	  
	  \pstart \leavevmode% starting standard par
	\textbf{ह‚न्तुरि}त्यादिना प्र‚तिविध‚त्ते । नेदं स‚माधानं युक्तं । किं कार‚णं [।] \textbf{ह‚न्तुर‚म‚{\tiny $_{lb}$}‚र‚ण‚त्व‚तः} । न हि ह‚न्ता चैत्र‚स्य म‚र‚ण‚स्व‚भावः । किन्त‚र्हि [।] मार‚यिता । त‚तो‚{\tiny $_{lb}$}‚ युक्तं य‚त् त‚न्नाशे चैत्र‚स्यापुन‚र्भ‚व‚नं । म‚र‚णे त्व‚निवृत्तेऽव‚श्यं पुन‚र्भ‚व‚नं स्यात् ।
	{\color{gray}{\rmlatinfont\textsuperscript{§~\theparCount}}}
	\pend% ending standard par
      ‚{\tiny $_{lb}$}‚

	  
	  \pstart \leavevmode% starting standard par
	\textbf{विनाशे}त्यादिना व्याच‚ष्टे । \textbf{विनाश}स्य \textbf{विनाशेपि न व‚स्तुनः प्र‚त्याप}त्तिर्न‚{\tiny $_{lb}$}‚ पूर्व‚रूप‚ग‚म‚नं । य‚स्मा\textbf{न्न हि‚{\tiny $_{३}$}‚ ह‚न्त‚रि ह‚ते}पि \textbf{त‚द्व‚त}स्तेन ह‚न्त्रा पुरुषेण ह‚तः \textbf{प्र‚त्यु‚{\tiny $_{lb}$}‚ज्जीव‚ति । ना}य‚म्प‚रिहारो युक्तः । क‚स्माद् [।] \textbf{ह‚न्तुः} पुरुष‚स्य \textbf{त‚द्घात‚हेतुत्वात्} ।‚{\tiny $_{lb}$}‚ त‚स्य चैत्र‚स्य य‚न्म‚र‚ण‚न्त‚द्धेतुत्वात् । न त्व‚सौ ह‚न्ता म‚र‚ण‚स्व‚भावः ।
	{\color{gray}{\rmlatinfont\textsuperscript{§~\theparCount}}}
	\pend% ending standard par
      ‚{\tiny $_{lb}$}‚

	  
	  \pstart \leavevmode% starting standard par
	एत‚देव स्प‚ष्ट‚य‚न्नाह । \textbf{ने}त्यादि । \textbf{नाश‚हेतोर‚ग्निद‚ण्डादेर्निवृत्तौ} स‚त्या‚{\tiny $_{lb}$}‚म्विन‚ष्टेन \textbf{भावेन} पुन\textbf{र्भ‚व‚तिव्य‚मिति न ब्रूमः} । एव‚म‚भिधाने भ‚वेदेष प‚रि‚{\tiny $_{४}$}‚हारः ।‚{\tiny $_{lb}$}‚ \textbf{किन्त‚र्हि} [।] व‚ह् न्यादिना काष्ठादे\textbf{र्भाव}स्या\textbf{भावो} यः क्रिय‚ते \textbf{त‚स्य} । किम्भूत‚स्य‚{\tiny $_{lb}$}‚ [।] \textbf{अत्य‚न्तानुप‚ल‚ब्धिल‚क्ष‚ण‚स्य} । क‚र्म्म‚स्था च क्रियात्रोप‚ल‚ब्धिः । त‚त्प्र‚तिषेधे‚{\tiny $_{lb}$}‚नात्य‚न्तानुप‚ल‚ब्धिः स‚र्व‚साम‚र्थ्य‚विर‚ह उच्य‚ते । \textbf{त‚स्यै}वंभूत‚स्याभाव‚स्य \textbf{निवृत्तौ}‚{\tiny $_{lb}$}‚ स‚त्यां । स्व‚भावाव‚स्थितेः स‚काशाद् भाव‚स्य \textbf{कान्या ग‚तिः । स्व‚भाव‚स्थितिरे}व‚{\tiny $_{lb}$}‚ ग‚तिरिति याव‚त् ।
	{\color{gray}{\rmlatinfont\textsuperscript{§~\theparCount}}}
	\pend% ending standard par
      ‚{\tiny $_{lb}$}‚

	  
	  \pstart \leavevmode% starting standard par
	ह‚न्त‚रि तु विन‚ष्टे न‚{\tiny $_{५}$}‚युक्तं पुन‚र्भ‚व‚नं । य‚स्मा\textbf{द्ध‚न्ता हि चैत्र‚स्य न नाश‚क‚ल्पः ।‚{\tiny $_{lb}$}‚ किन्त‚र्हि [।] द‚ण्डादिक‚ल्पः} द‚ण्डादितुल्यः नाश‚हेतुत्वात् । \textbf{नाश‚क‚ल्पं ह्य‚स्य} चै‚{\tiny $_{lb}$}‚त्र‚स्य \textbf{म‚र‚णं [।] त‚न्निवृत्तौ} त‚स्य नाश‚क‚ल्प‚स्य म‚र‚ण‚स्य निवृत्तौ \textbf{स्यादेवास्य} चैत्र‚स्य‚{\tiny $_{lb}$}‚ \textbf{पुन‚र्भावः} ।
	{\color{gray}{\rmlatinfont\textsuperscript{§~\theparCount}}}
	\pend% ending standard par
      ‚{\tiny $_{lb}$}‚

	  
	  \pstart \leavevmode% starting standard par
	एव‚न्ताव‚त् नाश‚स्यार्थान्त‚र‚त्वे दोष उक्तः ।
	{\color{gray}{\rmlatinfont\textsuperscript{§~\theparCount}}}
	\pend% ending standard par
      ‚{\tiny $_{lb}$}‚

	  
	  \pstart \leavevmode% starting standard par
	अन‚र्थान्त‚र‚त्व‚म‚धिकृत्याह । \textbf{अन‚न्य‚त्वेपी}त्यादि । व‚स्तुनो \textbf{नाश‚स्यान‚न्य‚{\tiny $_{lb}$}‚‚{\tiny $_{६}$}‚त्वेपि स्यान्नाशः काष्ठ‚मेव तु । त‚स्य} च काष्ठ‚स्य स्व‚हेतोरुत्प‚न्न‚स्य \textbf{स‚त्वात्} । न‚{\tiny $_{lb}$}‚ ‚{\tiny $_{lb}$}‚ \leavevmode\ledsidenote{\textenglish{518/s}}व‚ह्न्यादिभिः किञ्चित् क‚र्त्त‚व्य‚मिति त‚त्स्व‚भाव‚स्य नाश‚स्या\textbf{हेतुत्वं । नात}स्त‚त्त्वा‚{\tiny $_{lb}$}‚न्य‚त्व‚विक‚ल्पान्नाश‚स्य व‚स्तुध‚र्म‚स्य \textbf{विद्य‚तेन्या ग‚तिः} ।
	{\color{gray}{\rmlatinfont\textsuperscript{§~\theparCount}}}
	\pend% ending standard par
      ‚{\tiny $_{lb}$}‚

	  
	  \pstart \leavevmode% starting standard par
	\textbf{अन‚र्थान्त‚र} इत्यादिना व्याच‚ष्टे । \textbf{काष्ठाद‚न‚र्थान्त‚र‚भूतो} य‚दा \textbf{विनाश}स्त‚दा‚{\tiny $_{lb}$}‚ \leavevmode\ledsidenote{\textenglish{183b/PSVTa}} \textbf{त‚देव} काष्ठ‚मेव \textbf{त}द्विनाशाख्य‚म्व‚स्तु‚{\tiny $_{७}$}‚ \textbf{भ‚व‚ति । त‚च्च} काष्ठादि । व‚ह्न्यादिस‚न्निधा‚{\tiny $_{lb}$}‚नात् \textbf{प्रागेवास्तीति । किम}त्र काष्ठादौ विनाश्ये \textbf{साम‚र्थ्य‚म्व‚ह्न्यादीना}मिति‚{\tiny $_{lb}$}‚ द्र‚ष्ट‚व्यं । क्व‚चिद् द‚ण्डादीनामिति पाठः स तु घ‚टादीन् पुरोधाय व्याख्येयः ।‚{\tiny $_{lb}$}‚ \textbf{त‚स्मात् त‚द‚नुकारात्} । त‚त्र काष्ठादौ विनाश‚हेतूनाम‚नुप‚कारात् \textbf{तेन} काष्ठादिना‚{\tiny $_{lb}$}‚ विनाश‚हेत‚वो \textbf{नापेक्ष्य‚न्ते क‚थंचित्} केनापि रूपेण ।‚{\tiny $_{१}$}‚ \textbf{नाप्य‚स्य} काष्टादे\textbf{रिद}म्व‚ह्न्‚{\tiny $_{lb}$}‚यादिकं विनाश‚हेतु\textbf{रिति स‚म्ब‚न्ध‚म‚र्ह‚ति} । किं कार‚णं [।] \textbf{त‚स्योप‚कार‚निब‚न्ध‚न‚त्वात् ।‚{\tiny $_{lb}$}‚ अन्य‚थो}प‚कार‚म‚न्त‚रेण स‚म्ब‚न्ध‚क‚ल्प‚नायाम\textbf{तिप्र‚संगात्} । स‚र्वः स‚र्व‚स्य स‚म्ब‚न्धी स्यात् ।
	{\color{gray}{\rmlatinfont\textsuperscript{§~\theparCount}}}
	\pend% ending standard par
      ‚{\tiny $_{lb}$}‚

	  
	  \pstart \leavevmode% starting standard par
	स्यादेत‚त् [।] न साक्षाद् व‚न्ह्यादिः काष्ठादेरुप‚कार‚कः किन्तु त‚त्स‚म्ब‚न्धि‚{\tiny $_{lb}$}‚भूतोप‚कार‚क‚र‚णादिति [।]
	{\color{gray}{\rmlatinfont\textsuperscript{§~\theparCount}}}
	\pend% ending standard par
      ‚{\tiny $_{lb}$}‚

	  
	  \pstart \leavevmode% starting standard par
	अत आह । \textbf{पार‚म्प‚र्येणे}त्यादि । व‚ह्न्यादिना काष्ठादेः स‚{\tiny $_{२}$}‚म्ब‚न्धिभूत उप‚कारः‚{\tiny $_{lb}$}‚ क्रिय‚ते न साक्षादिति [।] एवं \textbf{पार‚म्प‚र्येणोप‚कारेपि} क‚ल्प्य‚माने\textbf{ऽव‚श्य‚म‚य‚म्विक‚ल्पो‚{\tiny $_{lb}$}‚न्वेत्य}नुग‚च्छ‚ति । स किम्पार‚म्प‚र्येणा\textbf{प्युप‚कारोर्थान्त‚र‚माहोस्वित् त‚देव} काष्ठादिक‚{\tiny $_{lb}$}‚मि\textbf{ति} । त‚त्र \textbf{त}स्मात् काष्ठादेर\textbf{र्थान्त‚र‚त्वेप्यु}प‚कार‚स्य । \textbf{त‚स्य} काष्ठादेर‚य‚मुप‚कार‚{\tiny $_{lb}$}‚ \textbf{इति} क‚स्स‚म्ब‚न्ध इति । त‚त्र काष्ठादौ त‚स्याग्निकृत‚स्\textbf{योप‚कार}स्योप‚कार‚क‚त्वं‚{\tiny $_{३}$}‚‚{\tiny $_{lb}$}‚ \textbf{प‚र्य‚नुयोज्यं} । त‚द‚न्त‚रेण स‚म्ब‚न्धाभावात् । आदिश‚ब्दात् त‚त्राप्य‚प‚रोप‚कार‚क‚ल्प‚ने‚{\tiny $_{lb}$}‚त्य‚न‚व‚स्थादोषादिप‚रिग्र‚हः । \textbf{त‚थान‚न्य‚त्वेप्यु}प‚कार‚स्य \textbf{त‚द‚व‚स्थः प‚र्य‚नुयोगः} स्यान्नाशः‚{\tiny $_{lb}$}‚ काष्ठ‚मेवेत्यादिना य उक्तः \textbf{त‚स्मात् स‚तो} विद्य‚मान‚स्य \textbf{रूप‚स्य त‚त्त्वान्य‚त्त्वाव्य‚{\tiny $_{lb}$}‚तिक्र‚मात्} कार‚णात् ।
	{\color{gray}{\rmlatinfont\textsuperscript{§~\theparCount}}}
	\pend% ending standard par
      ‚{\tiny $_{lb}$}‚

	  
	  \pstart \leavevmode% starting standard par
	स्यादेत‚त् [।] स‚तो रूप‚स्य त‚त्त्वान्य‚त्त्वाव्य‚तिक्र‚माद् विनाश‚हेतुकृतं तूप‚कारो‚{\tiny $_{४}$}‚‚{\tiny $_{lb}$}‚त्पाद‚न‚म‚स‚देवेति [।]
	{\color{gray}{\rmlatinfont\textsuperscript{§~\theparCount}}}
	\pend% ending standard par
      ‚{\tiny $_{lb}$}‚‚{\tiny $_{lb}$}‚\textsuperscript{\textenglish{519/s}}

	  
	  \pstart \leavevmode% starting standard par
	अत आह । \textbf{उप‚कारे}त्यादि । \textbf{रूप‚निष्पाद‚न‚ल‚क्ष‚ण}त्वात् । स‚द्रूप‚निष्पाद‚न‚ल‚क्ष‚{\tiny $_{lb}$}‚ण‚त्वात् । त‚त‚श्च त‚द्वा व‚स्तु तेन विनाश‚केन क‚र्त्त‚व्य‚म‚न्य‚द्वा । उभ‚य‚था चोक्तो‚{\tiny $_{lb}$}‚ दोष इति । \textbf{त‚द‚त‚त्क्रियाविक‚लो} नाश‚हेतु\textbf{र्न क‚र्तैवैति न क‚स्य‚चिद्धेतुः । अहेतुश्च}‚{\tiny $_{lb}$}‚ द‚ण्डादि \textbf{नापेक्ष्य‚ते} विन‚श्व‚रेण घ‚टादिना । \textbf{त‚स्मात् स्व‚यं} स‚त्तामात्रे\textbf{णाय‚म्भाव‚स्त‚{\tiny $_{lb}$}‚त्त‚त्स्व‚भावो‚{\tiny $_{५}$}‚} विन‚श्व‚र‚स्व‚भाव \textbf{इति} ॥
	{\color{gray}{\rmlatinfont\textsuperscript{§~\theparCount}}}
	\pend% ending standard par
      ‚{\tiny $_{lb}$}‚

	  
	  \pstart \leavevmode% starting standard par
	प्र‚ध्वंसाभाव‚न्नाशं गृहीत्वा प‚र‚स्य चोद्य‚माशंक‚ते । \textbf{अहेतुत्वेपी}त्यादि अहेतु‚{\tiny $_{lb}$}‚र्हि भ‚व‚न्नित्य‚म्भ‚वेत् । \textbf{नित्य‚त्वा}च्च भाव‚कालेपि नाशो भ‚वेदित्येव‚म्\textbf{भाव‚नाश‚योः‚{\tiny $_{lb}$}‚ स‚ह‚भाव‚प्र‚स‚ङ्ग‚श्चेत्} ।
	{\color{gray}{\rmlatinfont\textsuperscript{§~\theparCount}}}
	\pend% ending standard par
      ‚{\tiny $_{lb}$}‚

	  
	  \pstart \leavevmode% starting standard par
	नाय‚न्दोषः [।] किं कार‚ण‚म् [।] \textbf{अस‚तः} प्र‚ध्वंस‚ल‚क्ष‚ण‚स्य नाश‚स्य \textbf{नित्य‚ता‚{\tiny $_{lb}$}‚ कुतः} ।
	{\color{gray}{\rmlatinfont\textsuperscript{§~\theparCount}}}
	\pend% ending standard par
      ‚{\tiny $_{lb}$}‚

	  
	  \pstart \leavevmode% starting standard par
	\textbf{स्यादेत}दित्यादिना व्याच‚ष्टे । \textbf{य‚द्य‚पि नाशः} क्ष‚णिक‚वादिनोऽ\textbf{हेतुकः सोव‚श्यं‚{\tiny $_{lb}$}‚ नित्य इति} कृत्वा \textbf{भाव‚स्त‚द‚भाव‚ल‚क्ष‚णो} विनाश‚निवृत्तिरूपः । \textbf{विनाश‚श्च} त‚द‚भाव‚{\tiny $_{lb}$}‚ल‚क्ष‚णो भाव‚निवृत्तिरूपः । एक‚स्य \textbf{स‚ह स्यातामिति} ।
	{\color{gray}{\rmlatinfont\textsuperscript{§~\theparCount}}}
	\pend% ending standard par
      ‚{\tiny $_{lb}$}‚

	  
	  \pstart \leavevmode% starting standard par
	\textbf{नैत}देवं । क‚स्मात् [।] \textbf{त‚स्या}भाव‚स्याव‚स्तुत्वेन \textbf{नित्यादिध‚र्मायोगात् । न‚{\tiny $_{lb}$}‚ ह्य‚स‚त्य‚य}न्नित्यानित्य\textbf{विक‚ल्प‚स्स‚म्भ‚व‚ति । त‚यो}र्नित्य‚यो\textbf{र्व‚स्तुध‚र्म‚त्वात् । विनाश‚स्य‚{\tiny $_{lb}$}‚ च} भाव‚निवृत्तिल‚क्ष‚ण‚स्या\textbf{कि‚{\tiny $_{७}$}‚ञ्चित्त्वात्} ।
	{\color{gray}{\rmlatinfont\textsuperscript{§~\theparCount}}}
	\pend% ending standard par
      \textsuperscript{\textenglish{184a/PSVTa}}‚{\tiny $_{lb}$}‚

	  
	  \pstart \leavevmode% starting standard par
	किं च [।] \textbf{भ‚व‚तो} ह्युत्प‚द्य‚मान‚स्य नित्यं स‚त्त्वात् \textbf{केन‚चित् स‚ह‚भावः स्यात् [।]‚{\tiny $_{lb}$}‚ न च विनाशो भ‚व‚ति} । केव‚ल‚मेक‚क्ष‚ण‚स्थितिध‚र्मा भावः स्व‚य‚मेव न भ‚व‚तीति‚{\tiny $_{lb}$}‚ क्रियाप्र‚तिषेध‚मात्र‚मेत‚त् । \textbf{त‚स्माद‚दोषो}न‚न्त‚रोक्तः ।
	{\color{gray}{\rmlatinfont\textsuperscript{§~\theparCount}}}
	\pend% ending standard par
      ‚{\tiny $_{lb}$}‚

	  
	  \pstart \leavevmode% starting standard par
	पुन‚र‚पि प‚राभिप्राय‚माशंक‚ते । य‚दि विनाशो स‚न्निष्य‚ते त‚दा विनाश‚स्या‚{\tiny $_{lb}$}‚‚{\tiny $_{lb}$}‚ \leavevmode\ledsidenote{\textenglish{520/s}}\textbf{स‚त्त्वे} स‚त्य\textbf{भाव‚नाशित्व‚प्र‚संगः} । भाव‚स्य नाशित्वं न स्यादित्य‚य‚म‚पि‚{\tiny $_{१}$}‚ प्र‚स‚ङ्गो \textbf{न‚{\tiny $_{lb}$}‚ युज्य‚ते । य‚स्माद् भाव‚स्य नाशेना}र्थान्त‚रेण \textbf{ना}स्माभि\textbf{र्विनाश‚न‚मिष्य‚ते} ।
	{\color{gray}{\rmlatinfont\textsuperscript{§~\theparCount}}}
	\pend% ending standard par
      ‚{\tiny $_{lb}$}‚

	  
	  \pstart \leavevmode% starting standard par
	\textbf{क‚थ‚मि}त्यादिना व्याच‚ष्टे । \textbf{क‚थ‚म‚स‚न् विनाशो भावं नाश‚ये}द‚स‚तो व्यापारा‚{\tiny $_{lb}$}‚योगात् । \textbf{अतः} कार‚णाद\textbf{विनाशी भावः स्यादित्य‚प्र‚संग एव} । किं कार‚ण‚म् [।]‚{\tiny $_{lb}$}‚ \textbf{विनाशा}द‚र्थान्त‚र‚भूताद् \textbf{भाव}स्य \textbf{नाशान‚भ्युप‚ग‚मात्} ।
	{\color{gray}{\rmlatinfont\textsuperscript{§~\theparCount}}}
	\pend% ending standard par
      ‚{\tiny $_{lb}$}‚

	  
	  \pstart \leavevmode% starting standard par
	\textbf{यो ही}त्यादिनैत‚देव स‚म‚र्थ‚य‚ते । \textbf{यो हि} वादी \textbf{विनाश इ‚{\tiny $_{२}$}‚ति किंचिन्नेत्याह ।‚{\tiny $_{lb}$}‚ स क‚थ‚न्त‚तो} निःस्व‚भावान्नाशाद् \textbf{भाव}स्य \textbf{नाश‚मिच्छेत्} [।] नेच्छेत् ।
	{\color{gray}{\rmlatinfont\textsuperscript{§~\theparCount}}}
	\pend% ending standard par
      ‚{\tiny $_{lb}$}‚

	  
	  \pstart \leavevmode% starting standard par
	\textbf{क‚थ}मित्यादि प‚रः । \textbf{अस‚त्}य‚विद्य‚माने \textbf{विनाशे क‚थ‚म्भावो न‚ष्टो नाम} । नैव‚{\tiny $_{lb}$}‚ विन‚ष्टः स्यात् । त‚था हि \textbf{प्र‚त्युत्प‚न्नाव‚स्थायाम}स‚द्विनाशाः । \textbf{अस‚द्विना}शो येषा‚{\tiny $_{lb}$}‚मिति विग्र‚हः । ते न हि न‚ष्टा ग‚ण्य‚न्ते । य‚दा च भाव‚स्य नाशो नास्ति त‚दा क‚थ‚न्तेन‚{\tiny $_{lb}$}‚ स व्य‚प‚दिश्य‚ते नाश‚वा‚{\tiny $_{३}$}‚निति । \textbf{न हि यो येन} स्व‚भावेना\textbf{त‚द्वान्} अस‚म्ब‚न्ध‚वान् ।‚{\tiny $_{lb}$}‚ \textbf{स} प‚दार्थ\textbf{स्तेना}स‚म्ब‚न्धिना \textbf{त‚था व्य‚प‚दिश्य‚ते} । त‚द्वानिति व्य‚प‚दिश्य‚ते श‚ब्देन ।‚{\tiny $_{lb}$}‚ \textbf{प्र‚तीय‚ते वा} ज्ञानेन ।
	{\color{gray}{\rmlatinfont\textsuperscript{§~\theparCount}}}
	\pend% ending standard par
      ‚{\tiny $_{lb}$}‚

	  
	  \pstart \leavevmode% starting standard par
	नेत्यादिना प‚रिह‚र‚ति । \textbf{न वै} भाव‚स्य \textbf{नाशो नास्त्ये}वापि त्व‚स्त्येव नाशः ।‚{\tiny $_{lb}$}‚ क‚थ‚न्त‚र्हि नास्तीत्युच्य‚ते । \textbf{स तु नास्ति} नाशो \textbf{यो भाव‚स्य भ‚व‚ति} ।‚{\tiny $_{lb}$}‚ य‚दि विनाशो न भ‚व‚ति क‚थ‚न्त‚र्हि विनाशोस्तीत्युच्य‚{\tiny $_{४}$}‚त इति [।]
	{\color{gray}{\rmlatinfont\textsuperscript{§~\theparCount}}}
	\pend% ending standard par
      ‚{\tiny $_{lb}$}‚

	  
	  \pstart \leavevmode% starting standard par
	आह । \textbf{भाव एव तु क्ष‚ण‚स्थितिध‚र्मा} । एक‚क्ष‚ण‚स्थायी \textbf{नाशः} ।
	{\color{gray}{\rmlatinfont\textsuperscript{§~\theparCount}}}
	\pend% ending standard par
      ‚{\tiny $_{lb}$}‚

	  
	  \pstart \leavevmode% starting standard par
	य‚दि भाव एव नाशः क‚थ‚न्त‚र्हि भाव‚स्य नाशो भूत इति लोको व्य‚प‚दिश‚तीति [।]
	{\color{gray}{\rmlatinfont\textsuperscript{§~\theparCount}}}
	\pend% ending standard par
      ‚{\tiny $_{lb}$}‚

	  
	  \pstart \leavevmode% starting standard par
	अत आह । \textbf{त‚म}स्येत्यादि । \textbf{अस्य} भाव‚स्य \textbf{त‚मे}क‚क्ष‚ण‚स्थायि\textbf{स्व‚भा}वं स‚दृशाप‚{\tiny $_{lb}$}‚रोत्प‚त्तिविप्र‚ल‚म्भादुप‚ल‚क्षितं । \textbf{उत्त‚र‚कालं} स‚न्तानोच्छित्ताव‚नुप‚ल‚म्भेनास्थितिप्र‚ति‚{\tiny $_{lb}$}‚प‚त्त्या । \textbf{विभाव‚य‚न्तो} निश्चिन्व‚न्तः । \textbf{विना‚{\tiny $_{५}$}‚शोस्य} भाव‚स्य \textbf{भूत इति य‚था प्र‚तीति‚{\tiny $_{lb}$}‚ व्य‚प‚दिश‚न्ति} व्य‚व‚हारिणः पुरुषा \textbf{इत्युक्तं} प्राक् ।
	{\color{gray}{\rmlatinfont\textsuperscript{§~\theparCount}}}
	\pend% ending standard par
      ‚{\tiny $_{lb}$}‚‚{\tiny $_{lb}$}‚\textsuperscript{\textenglish{521/s}}

	  
	  \pstart \leavevmode% starting standard par
	य‚स्मा\textbf{न्न हि भाव‚स्य} निष्प‚न्न‚स्य \textbf{किञ्चि}द्रूपान्त‚र‚म्विनाशाख्य‚म‚न्य‚द्वा \textbf{क‚दाचिद्‚{\tiny $_{lb}$}‚ भ‚व‚ति । स एव} भावः \textbf{केव‚लं स्व‚हेतुभ्य‚स्त‚थाभूत} एक‚क्ष‚ण‚स्थायी \textbf{भ‚व‚ति । त‚दि}ति‚{\tiny $_{lb}$}‚ त‚स्मा\textbf{न्न केन‚चि}द्विनाशाख्येन \textbf{भ‚व‚ता । स} भावो \textbf{न‚ष्टो} नाम । \textbf{किन्त‚र्हि [।] स्व‚भाव‚{\tiny $_{lb}$}‚ एवास्य} भाव‚स्य स एक‚{\tiny $_{६}$}‚क्ष‚णाव‚स्थान‚शीलः । \textbf{येन स} भावो \textbf{न‚ष्टो नाम} । अन्य‚था‚{\tiny $_{lb}$}‚ स्व‚य‚म‚त‚त्स्व‚भाव‚त्वेन्य‚स‚न्निधानेप्य‚नाशात् ।
	{\color{gray}{\rmlatinfont\textsuperscript{§~\theparCount}}}
	\pend% ending standard par
      ‚{\tiny $_{lb}$}‚

	  
	  \pstart \leavevmode% starting standard par
	य‚दि नाशो नाम न किञ्चित् । \textbf{क‚थ‚न्त‚र्हीदानीम‚हेतुको नाशो भ‚व‚तीत्युच्य‚ते}‚{\tiny $_{lb}$}‚ भ‚व‚द्भिः । य‚स्य हि स्व‚भाव एव नास्ति त‚स्य किम\textbf{हेतुकः} स‚हेतुको वेति चिन्त‚या । \leavevmode\ledsidenote{\textenglish{184b/PSVTa}}‚{\tiny $_{lb}$}‚ भाव‚स्य नाश इति व्य‚तिरेको वा क‚थं ।
	{\color{gray}{\rmlatinfont\textsuperscript{§~\theparCount}}}
	\pend% ending standard par
      ‚{\tiny $_{lb}$}‚

	  
	  \pstart \leavevmode% starting standard par
	\textbf{न‚श्य‚न्नि}त्यादिना प‚रिह‚र‚ति । \textbf{भावो न‚श्य‚न्न}‚{\tiny $_{७}$}‚प‚रापेक्षः । प‚रं विनाश‚हेतुं‚{\tiny $_{lb}$}‚ नापेक्ष‚त \textbf{इति} कृत्वा । \textbf{न ज्ञाप‚नाये}त्य‚प‚रापेक्ष‚त्व‚ज्ञाप‚नाय । \textbf{सा} नाश्\textbf{आव‚स्था}स्माभिर‚{\tiny $_{lb}$}‚\textbf{हेतुरुक्ता} । त‚स्या नाशाव‚स्थायाश्\textbf{चेत‚सा} विक‚ल्प‚बुद्ध्या भावाद् \textbf{भेदं} व्य‚तिरेक‚{\tiny $_{lb}$}‚\textbf{मारोप्य} [।]
	{\color{gray}{\rmlatinfont\textsuperscript{§~\theparCount}}}
	\pend% ending standard par
      ‚{\tiny $_{lb}$}‚

	  
	  \pstart \leavevmode% starting standard par
	एत‚दुक्त‚म्भ‚व‚ति [।] अहेतुको भाव‚स्य विनाशो भ‚व‚तीति स‚हेतुकोस्य विनाशो‚{\tiny $_{lb}$}‚ न भ‚व‚तीत्य‚र्थः ।
	{\color{gray}{\rmlatinfont\textsuperscript{§~\theparCount}}}
	\pend% ending standard par
      ‚{\tiny $_{lb}$}‚

	  
	  \pstart \leavevmode% starting standard par
	\textbf{ने}त्यादिना व्याच‚ष्टे । \textbf{न भावो जातः} स‚न्न\textbf{प‚र‚स्माद्} विनाश‚हेतो\textbf{र्ना‚{\tiny $_{१}$}‚शं प्र‚ति‚{\tiny $_{lb}$}‚ल‚भ‚ते} । किं कार‚णं [।] \textbf{त‚थाभूत‚स्यैव} न‚श्व‚र‚स्व‚भाव‚स्यैव \textbf{स्व‚यं} स‚त्ताहेतोरेव‚{\tiny $_{lb}$}‚ \textbf{जातेरु}त्प‚त्तेः । \textbf{इति} हेतोर‚प‚र‚म‚न्य‚म्विनाश‚हेतुत्वेन क‚ल्पित‚म‚पेक्ष‚त इत्य‚प‚रापेक्षः ।‚{\tiny $_{lb}$}‚ त‚थाभूत‚श्चासौ ध‚र्म‚श्च विनाशाख्यः । \textbf{अप‚रापेक्ष‚ध}र्म‚स्त‚स्य \textbf{प्र‚तिषेधार्थं} ।‚{\tiny $_{lb}$}‚ स‚हेतुक‚विनाश‚प्र‚तिषेधार्थ‚मिति याव‚त् । \textbf{त‚त्स्व‚भाव‚ज्ञाप‚नेने}ति भाव‚स्य विन‚श्व‚र‚{\tiny $_{lb}$}‚स्व‚भाव‚ज्ञा‚{\tiny $_{२}$}‚प‚नेन । \textbf{स्व‚भाव एव} त‚थोच्य‚त इत्य‚नेन स‚म्ब‚न्धः । \textbf{त‚थोच्य‚त} इत्य‚हेतु‚{\tiny $_{lb}$}‚कोस्य विनाशो भ‚व‚तीत्युच्य‚ते । क‚दाचित्त‚न्मात्र‚जिज्ञासायां । भाव‚स्यान्य‚स्मात्‚{\tiny $_{lb}$}‚ किम्विनाशो भ‚व‚ति न चेत्ये\textbf{ताव‚न्मात्र‚जिज्ञासायां} । केन प्र‚कारेणोच्य‚ते । \textbf{ध‚र्मि}णः‚{\tiny $_{lb}$}‚ स‚काशाद् \textbf{अर्थान्त‚र‚मिव} विनाशाख्यं \textbf{ध‚र्मं चेत}सा बुद्ध्या \textbf{विभ‚ज्या}स्य भाव‚स्य‚{\tiny $_{lb}$}‚ विनाश इति विभागं कृत्वा । \textbf{त‚देत‚द्} य‚थोक्तेन‚{\tiny $_{३}$}‚ प्र‚कारेणाभावाद‚व्य‚तिरिक्त‚{\tiny $_{lb}$}‚‚{\tiny $_{lb}$}‚ \leavevmode\ledsidenote{\textenglish{522/s}}न्नाशित्व‚न्त‚त्व‚तो व्य‚व‚स्थापित‚म‚पि \textbf{म‚न्द‚म‚त‚यो} नाशं गुणं ध‚र्मं स‚मारोप्यात्मान‚{\tiny $_{lb}$}‚माकुल‚य‚न्तीत्य‚नेन स‚म्ब‚न्धः ।
	{\color{gray}{\rmlatinfont\textsuperscript{§~\theparCount}}}
	\pend% ending standard par
      ‚{\tiny $_{lb}$}‚

	  
	  \pstart \leavevmode% starting standard par
	क‚स्मात् पुन‚स्त एव‚माकुल‚य‚न्तीतित्याह । \textbf{क्व‚चि}दित्यादि । राज्ञः पुरुष‚{\tiny $_{lb}$}‚ इत्यादौ व्य‚तिरेक‚विभ‚क्तिप्र‚योगे \textbf{त‚थाद‚र्श‚नात्} । स‚म्ब‚न्धिनोर्विभाग‚द‚र्श‚नात् । इहापि‚{\tiny $_{lb}$}‚ भा‚{\tiny $_{४}$}‚व‚स्य नाशो भ‚व‚तीत्य‚नेन \textbf{घोष‚णामात्रेण विप्र‚ल‚ब्धाः} । भाव‚स्य व्य‚तिरिक्तं‚{\tiny $_{lb}$}‚ \textbf{नाशं गुणं} ध‚र्मं \textbf{स‚मारोप्य । त‚स्य च} य‚था क‚ल्पित‚स्य गुण‚स्य \textbf{भावं} स‚त्तां \textbf{स‚मा‚{\tiny $_{lb}$}‚रोप्य} । तं नाशाख्यं गुणं \textbf{स‚हेतुक‚म‚हेतुक‚म्वा} द‚र्श‚न‚भेदेन स‚मारोप्य \textbf{भाव‚चिन्त‚या}‚{\tiny $_{lb}$}‚ व‚स्तुचिन्त‚या । किंभूत‚या [।] \textbf{अप्र‚तिष्ठित‚त‚त्त्व‚या} । अप्र‚तिष्ठित‚न्त‚त्त्वं य‚स्यां‚{\tiny $_{lb}$}‚ चिन्तायां । त‚यात्\textbf{मान‚माकुल‚य‚न्ति} ।
	{\color{gray}{\rmlatinfont\textsuperscript{§~\theparCount}}}
	\pend% ending standard par
      ‚{\tiny $_{lb}$}‚

	  
	  \pstart \leavevmode% starting standard par
	\textbf{स्व‚{\tiny $_{५}$}‚तोपी}त्यादिना प‚राभिप्राय‚माशंक‚ते । य‚स्यापि स्व‚य‚मेवाहेतुको नाशो‚{\tiny $_{lb}$}‚ भ‚व‚ति । त‚स्यापि \textbf{स्व‚तोप्य‚भाव‚स्य} विनाश‚स्य \textbf{भावे}ङ्गीक्रिय‚माणे । अय‚न्त‚त्त्वान्य‚त्त्व‚{\tiny $_{lb}$}‚ल‚क्ष‚णो \textbf{विक‚ल्प‚श्चे}त‚सः ।
	{\color{gray}{\rmlatinfont\textsuperscript{§~\theparCount}}}
	\pend% ending standard par
      ‚{\tiny $_{lb}$}‚

	  
	  \pstart \leavevmode% starting standard par
	\textbf{न‚न्वि}त्यादिना व्याच‚ष्टे । न प‚र‚भावित्व\textbf{म‚प‚र‚भावित्व}म‚हेतुक‚त्वेपीत्य‚र्थः । \textbf{भाव‚{\tiny $_{lb}$}‚स्य} व‚स्तुनो यो \textbf{नाश}स्त\textbf{स्य स्व‚त एव भ‚व‚तः । अय‚न्त‚त्त्वान्य‚त्त्व‚विक‚ल्प‚{\tiny $_{६}$}‚स्तुल्यः} [।]
	{\color{gray}{\rmlatinfont\textsuperscript{§~\theparCount}}}
	\pend% ending standard par
      ‚{\tiny $_{lb}$}‚

	  
	  \pstart \leavevmode% starting standard par
	किम‚र्थान्त‚र‚न्नाशो भावादुत भाव एवेति । त‚त्र य‚द्य‚र्थान्त‚र\textbf{न्त‚दा किम‚र्था‚{\tiny $_{lb}$}‚न्त‚र}स्य नाश‚स्य \textbf{भावे} स‚त्तायाम्\textbf{भावो न दृश्य‚ते} । अथान‚र्थान्त‚रं विनाश‚स्त\textbf{दान‚{\tiny $_{lb}$}‚र्थान्त‚र‚त्वेपि त‚देव} घ‚टादिक‚मेव \textbf{त‚न्ना}श‚ख्य‚म्\textbf{भ‚व‚ति । त‚त्त}स्मा\textbf{न्न किञ्चिद‚स्य}‚{\tiny $_{lb}$}‚ प‚दार्थ‚स्य \textbf{जात‚मिति क‚थं विन‚ष्टो नाम} ।
	{\color{gray}{\rmlatinfont\textsuperscript{§~\theparCount}}}
	\pend% ending standard par
      ‚{\tiny $_{lb}$}‚

	  
	  \pstart \leavevmode% starting standard par
	\textbf{न‚न्वि}त्यादिना प‚रिह‚र‚ति । \textbf{अत्र} प्र‚स्तावे । उक्तं [।] किमुक्तं [।] \textbf{न त‚स्य}‚{\tiny $_{lb}$}‚ \leavevmode\ledsidenote{\textenglish{185a/PSVTa}} भाव‚स्य \textbf{किंचि‚{\tiny $_{७}$}‚द्} व्य‚तिरिक्त‚म्वा नाशाख्यं ध‚र्म‚रूप‚म्\textbf{भ‚व‚ति} । क‚थ‚न्त‚र्हि विनाशी‚{\tiny $_{lb}$}‚ भाव इत्याह । \textbf{न भ‚व‚त्येव केव‚ल‚मि}त्युक्तं प्राक् ।
	{\color{gray}{\rmlatinfont\textsuperscript{§~\theparCount}}}
	\pend% ending standard par
      ‚{\tiny $_{lb}$}‚‚{\tiny $_{lb}$}‚\textsuperscript{\textenglish{523/s}}

	  
	  \pstart \leavevmode% starting standard par
	एत‚देव स्फुट‚य‚न्नाह । \textbf{न ही}त्यादि । \textbf{न ह्य‚य‚म}हेतुक‚विनाश‚वादी भाव‚स्य‚{\tiny $_{lb}$}‚ स्व‚हेतोर्निष्प‚न्न‚स्य क‚श्चिद् भाव‚रूपोऽभाव‚रूपो वा \textbf{विनाशोन्यो} वा स्थित्य‚न्य‚था‚{\tiny $_{lb}$}‚त्वादिको ध‚र्मो \textbf{भ‚व‚तीत्याह । किन्त‚र्हि स एव भावो न भ‚व‚तीति} भाव‚निवृत्ति‚{\tiny $_{१}$}‚‚{\tiny $_{lb}$}‚मात्र‚माह । तेनाय‚म‚र्थः [।] प्र‚थ‚मे क्ष‚णे भावोऽभूतो भ‚व‚ति । द्वितीये क्ष‚णे त‚स्य‚{\tiny $_{lb}$}‚ न भावो भ‚व‚ति नाभावो वा । नापि स्व‚र‚स‚हानिर्वा भ‚व‚ति । केव‚लं स्व‚य‚मेव निव‚र्त्त‚ते ।
	{\color{gray}{\rmlatinfont\textsuperscript{§~\theparCount}}}
	\pend% ending standard par
      ‚{\tiny $_{lb}$}‚

	  
	  \pstart \leavevmode% starting standard par
	\textbf{य‚दि} पुन‚र्नाशाभिधानेन \textbf{क‚स्य}चिद्ध‚र्म‚स्य \textbf{भाव}मुत्पादं \textbf{ब्रूया}न्न \textbf{भावोनेन} वादिना‚{\tiny $_{lb}$}‚ \textbf{निव‚र्तितः स्यात्} । भाव‚निवृत्तिर्न क‚थितेति याव‚त् । \textbf{त‚था च भाव‚निवृत्तौ प्र‚स्तु‚{\tiny $_{lb}$}‚ताया}म‚र्थान्त‚र‚स्यान्य‚स्य‚{\tiny $_{२}$}‚ विधाना\textbf{द‚प्र‚स्तुत‚मेवोक्तं स्यात्} । किं कार‚णं [।] \textbf{न हि‚{\tiny $_{lb}$}‚ क‚स्य‚चिद्} विनाशाख्य‚स्याभाव‚स्य भाव‚निवृत्तिरूप‚स्य वा \textbf{भावे}नोत्पादेन \textbf{भावः}‚{\tiny $_{lb}$}‚ प‚दार्थो \textbf{न भूतो नाम} । येन त‚द्विधानेन भाव‚स्य स्व‚निवृत्तिः स्यात् ।
	{\color{gray}{\rmlatinfont\textsuperscript{§~\theparCount}}}
	\pend% ending standard par
      ‚{\tiny $_{lb}$}‚

	  
	  \pstart \leavevmode% starting standard par
	एत‚दुक्त‚म्भ‚व‚ति । य‚था भाव‚स्य विज्ञान‚भावे भावो न निव‚र्त्त‚ते केव‚ल‚न्त‚{\tiny $_{lb}$}‚द्विज्ञान‚न्त‚त्स‚म्बंन्धि स्यात् । त‚था भाव‚स्य निवृत्तिर्भ‚व‚तीत्य‚भ्युप‚ग‚मे स एव निवृ‚{\tiny $_{lb}$}‚त्त्या‚{\tiny $_{३}$}‚ख्यो ध‚र्म‚स्त‚त्स‚म्ब‚न्धी स्यान्न तु भावो निव‚र्त्तेतेति क‚थ‚म‚स्य निवृत्तिः स्यात् ।‚{\tiny $_{lb}$}‚ त‚स्मात्त‚दा स भावो \textbf{न भूतो निवृत्तो य‚दि स्व‚यं न भ‚वेत्} ।
	{\color{gray}{\rmlatinfont\textsuperscript{§~\theparCount}}}
	\pend% ending standard par
      ‚{\tiny $_{lb}$}‚

	  
	  \pstart \leavevmode% starting standard par
	तेन य‚दुच्य‚ते । न‚न्व‚भ‚व‚न‚म‚पि य‚दि भाव‚स्य न भ‚व‚ति । त‚दाऽविनाशित्वं ।‚{\tiny $_{lb}$}‚ अथ भ‚व‚ति । त‚द्भिन्न‚म्वा स्याद‚भिन्न‚म्वाऽन‚योश्च प‚क्ष‚योर्भाव‚स्य स‚र्व‚दा द‚र्श‚नं‚{\tiny $_{lb}$}‚ स्याद‚विनाशात् । त‚स्मान्नाभाव‚स्य विनाशः [।] क‚थ‚न्त‚र्हि भावः स‚र्व‚दा न प्र‚{\tiny $_{४}$}‚‚{\tiny $_{lb}$}‚तीय‚ते प्र‚माणाभावादिति [।]
	{\color{gray}{\rmlatinfont\textsuperscript{§~\theparCount}}}
	\pend% ending standard par
      ‚{\tiny $_{lb}$}‚

	  
	  \pstart \leavevmode% starting standard par
	त‚द‚पास्तं । दृश्य‚स्य हि स‚त्तायाः प्र‚माण‚विष‚य‚त्वेन व्याप्त‚त्वात् । त‚द‚भावा‚{\tiny $_{lb}$}‚द‚भावः । भावे वाव‚श्यं प्र‚माण‚विष‚य‚त्व‚मिति क‚थ‚म‚प्र‚तिप‚त्तिः । अथादृश्य‚रूप‚त‚यास्य‚{\tiny $_{lb}$}‚ भाव‚स्त‚दा त‚र्हि दृश्य‚रूप‚ताया निवृत्तिः । सा च भावाद्भिन्नाऽभिन्ना वा [।]‚{\tiny $_{lb}$}‚ अन‚योश्च प‚क्ष‚योर्भाव‚स्य द‚र्श‚नं स्यादिति दोष‚स्त‚द‚व‚स्थ एव । त‚स्माद् भाव‚स्या‚{\tiny $_{lb}$}‚भ‚व‚न‚{\tiny $_{५}$}‚म‚पि न भ‚व‚ति । नाप्य‚विनाशित्व‚दोषः । स्व‚रूपेण निवृत्तेः ।
	{\color{gray}{\rmlatinfont\textsuperscript{§~\theparCount}}}
	\pend% ending standard par
      ‚{\tiny $_{lb}$}‚

	  
	  \pstart \leavevmode% starting standard par
	न‚नु भाव‚निवृत्तेर्नीरूप‚त्वेन रूपिणो भावाद‚न्य‚त्त्व‚मिति चेत् ।
	{\color{gray}{\rmlatinfont\textsuperscript{§~\theparCount}}}
	\pend% ending standard par
      ‚{\tiny $_{lb}$}‚

	  
	  \pstart \leavevmode% starting standard par
	न‚नु य‚स्य रूप‚मेव न विद्य‚ते त‚स्य क‚थ‚म‚न्य‚त्त्वं । त‚त्किमेक‚त्व‚म‚स्तु । त‚द‚पि‚{\tiny $_{lb}$}‚ नास्त्य‚रूप‚त्वात् । त‚स्माद् भावेन स‚हास्यास्त‚त्त्वान्य‚त्त्व‚निषेध‚मात्रं क्रिय‚ते । श‚श‚वि‚{\tiny $_{lb}$}‚षाण‚व‚त् ।
	{\color{gray}{\rmlatinfont\textsuperscript{§~\theparCount}}}
	\pend% ending standard par
      ‚{\tiny $_{lb}$}‚‚{\tiny $_{lb}$}‚\textsuperscript{\textenglish{524/s}}

	  
	  \pstart \leavevmode% starting standard par
	न‚त्वेव‚म‚पि क‚थं द्वितीय‚क्ष‚णे भावो न भ‚व‚ती‚{\tiny $_{६}$}‚तीष्य‚ते [।] य‚तो य‚दि द्वितीय‚{\tiny $_{lb}$}‚क्ष‚णे भाव‚स्त‚दा क‚थ‚न्त‚त्र नास्तीतीष्य‚ते विरोधात् । अथ नास्ति त‚दा क‚थं भावो‚{\tiny $_{lb}$}‚ नास्तीत्युच्य‚तेऽस‚त्त्वादिति ।
	{\color{gray}{\rmlatinfont\textsuperscript{§~\theparCount}}}
	\pend% ending standard par
      ‚{\tiny $_{lb}$}‚

	  
	  \pstart \leavevmode% starting standard par
	त‚द‚युक्तं । य‚तः प्र‚थ‚मेपि क्ष‚णे भावो भ‚व‚तीति लोकेभिधीय‚ते । त‚त्र च य‚दि‚{\tiny $_{lb}$}‚ भावः क‚थ‚म्भ‚व‚तीत्युच्य‚ते । त‚स्मात् स‚र्व‚त्र बुद्धिस्थ‚मेव भावं कृत्वा विधिप्र‚तिषेध‚{\tiny $_{lb}$}‚व्य‚व‚हार इति य‚त्किञ्चिदेत‚त् ।
	{\color{gray}{\rmlatinfont\textsuperscript{§~\theparCount}}}
	\pend% ending standard par
      ‚{\tiny $_{lb}$}‚

	  
	  \pstart \leavevmode% starting standard par
	त‚स्मात् स्थित‚मेत‚त् [।] त‚दा स भावो न भूतो य‚दि स्व‚यं न भ‚वेदिति ।
	{\color{gray}{\rmlatinfont\textsuperscript{§~\theparCount}}}
	\pend% ending standard par
      ‚{\tiny $_{lb}$}‚

	  
	  \pstart \leavevmode% starting standard par
	\leavevmode\ledsidenote{\textenglish{185b/PSVTa}} न‚नु स्व‚य‚म्भावो न भ‚व‚तीत्य‚नेनापि वाक्येन स्व‚य‚मेवाभावो भाव‚स्य भ‚व‚ती‚{\tiny $_{lb}$}‚त्युच्य‚ते । त‚दा च स एव दोष इत्य‚त आह । \textbf{न भ‚व‚तीति चे}त्यादि । च‚श‚ब्दो य‚स्मा‚{\tiny $_{lb}$}‚ द‚र्थे । य‚स्माद् भावो न भ‚व‚तीति च \textbf{प्र‚स‚ज्य‚प्र‚तिषेध एषः । न प‚र्युदासः} । य‚त्र‚{\tiny $_{lb}$}‚प्र‚स‚क्त‚स्य निवृत्तिमात्र‚मेव क्रिय‚ते न व‚स्त्वंश‚स्य संस्प‚र्शः स प्र‚{\tiny $_{१}$}‚स‚ज्य‚प्र‚तिषेधः ।‚{\tiny $_{lb}$}‚ य‚त्र त्वेक‚निषेधेनान्य‚विधानं स प‚र्युदासः । अन्य‚थे\textbf{हापि} प्र‚स‚ज्य‚प्र‚तिषेधेपि \textbf{क‚स्य‚चिद्व}‚{\tiny $_{lb}$}‚स्तुनो \textbf{भावे} । विधाने स‚ति । न \textbf{प्र‚तिषेध‚प‚र्युदास‚यो रूप‚भेदः} स्व‚भाव‚भेदः \textbf{स्यात्} ।‚{\tiny $_{lb}$}‚ प्र‚स‚ज्य‚प्र‚तिषेधः प्र‚तिषेध‚श‚ब्देनोक्तः । किं कार‚ण‚म् [।] \textbf{उभ‚य‚त्रापि} प्र‚स‚ज्ये प‚र्युदासे‚{\tiny $_{lb}$}‚ \textbf{च । विधेः प्राधान्यात्} ।
	{\color{gray}{\rmlatinfont\textsuperscript{§~\theparCount}}}
	\pend% ending standard par
      ‚{\tiny $_{lb}$}‚

	  
	  \pstart \leavevmode% starting standard par
	य‚दि च प्र‚स‚ज्य‚प्र‚तिषेधेपि विधि‚{\tiny $_{२}$}‚स्त‚दा प्र‚तिषेध एव नास्ति । \textbf{एवं चाप्र‚ति‚{\tiny $_{lb}$}‚षेधात् क‚स्य‚चित् प‚र्युदासोपि न स्यात् क्व‚चित्} । किं कार‚ण‚म् । \textbf{य‚दि हि किं‚{\tiny $_{lb}$}‚चिद्व‚स्}तु कुत‚श्चि\textbf{न्निव‚र्त्तेत । त‚दा त‚द्व्य‚तिरेकि} । निव‚र्त्त‚मानाद् व‚स्तुनो व्य‚तिरेकि‚{\tiny $_{lb}$}‚ \textbf{संस्पृश्येत । त‚त्प‚र्युदासेन} निव‚र्त्त्य‚मान‚प‚र्युदासेन । य‚थाऽब्राह्म‚ण‚मान‚येति ब्राह्म‚ण‚{\tiny $_{lb}$}‚प‚र्युदासेन क्ष‚त्रियादेः संस्प‚र्शात् [।] \textbf{त‚च्च} क‚स्य‚चिन्निव‚र्त्त‚न‚मेव \textbf{नास्ति} । किं कार‚णं‚{\tiny $_{lb}$}‚ [।] \textbf{स‚र्व‚त्र}‚{\tiny $_{३}$}‚ क‚स्य‚चि\textbf{न्निवृत्तिर्भ‚व‚तीत्युक्तेपि} न भाव‚व्य‚व‚च्छेदः \textbf{क‚स्य‚चित्} प्र‚तीय‚ते‚{\tiny $_{lb}$}‚ऽपि तु निवृत्तिश‚ब्देनापि क‚स्य‚चिद् \textbf{भाव‚स्यैव प्र‚तीतेः} । न चानेन वादिना भाव‚स्य‚{\tiny $_{lb}$}‚ निवृत्तिं ब्रुवाणेनापि निवृत्तिर्नैवोक्ता किन्त्व\textbf{र्थान्त‚र‚भाव एवोक्तः स्यात्} । त‚था‚{\tiny $_{lb}$}‚ च य‚स्य प‚र्युदासेन य‚द्विविक्त‚मुच्य‚ते \textbf{न त‚योः प‚र‚स्प‚र‚विवेकः} सिद्धः । अस‚ति च‚{\tiny $_{lb}$}‚ \textbf{विवेके न प‚र्युदासः} ।‚{\tiny $_{४}$}‚ त‚द‚न्य‚विवेकेनान्योपाद‚न‚ल‚क्ष‚ण‚त्वात् प‚र्युदास‚स्य । \textbf{त‚देवं}‚{\tiny $_{lb}$}‚ ‚{\tiny $_{lb}$}‚ \leavevmode\ledsidenote{\textenglish{525/s}}य‚थोक्तेन प्र‚कारेण \textbf{व्य‚तिरेकाभावाद‚न्व‚योपि न स्यात्} । अन्व‚यः क‚स्य‚चिद‚र्थ‚स्यानु‚{\tiny $_{lb}$}‚ग‚मो विधान‚न्त‚न्न स्यादित्य‚र्थः । किं कार‚णं [।] \textbf{त‚स्या}न्व‚य‚स्यै\textbf{क‚स्व‚भाव‚स्थितिल‚{\tiny $_{lb}$}‚क्ष‚ण‚त्वात् । त‚त्स्थितिश्चैक}स्व‚भाव‚स्थितिश्च त‚स्मा\textbf{द‚न्य‚स्य व्य‚तिरेके} प‚रिहारे \textbf{स‚ति‚{\tiny $_{lb}$}‚ स्यात् । स च} त‚द‚न्य‚व्य‚तिरेको‚{\tiny $_{५}$}‚ \textbf{नास्ति} त्व‚न्म‚तेन । इति एवं श‚ब्दाद\textbf{प्र‚वृत्तिनि‚{\tiny $_{lb}$}‚वृत्तिकं ज‚ग‚त् स्यात्} । शाब्द‚स्य विधिप्र‚तिषेध‚व्य‚व‚हार‚स्याभावः स्यात् । न चैव‚{\tiny $_{lb}$}‚मित्य‚व‚श्यं क‚स्य‚चिद् व्य‚व‚च्छेद‚मात्रं श‚ब्द‚वाच्य‚म‚भ्युप‚ग‚न्त‚व्यं ।
	{\color{gray}{\rmlatinfont\textsuperscript{§~\theparCount}}}
	\pend% ending standard par
      ‚{\tiny $_{lb}$}‚

	  
	  \pstart \leavevmode% starting standard par
	य‚त‚श्चैव\textbf{न्त‚स्माद् य‚स्य} भाव‚स्य \textbf{नाशो भ‚व‚तीत्युच्य‚ते स स्व‚य‚मेव न भ‚व‚ती‚{\tiny $_{lb}$}‚त्युक्तं स्यात्} [।] न पुन‚र‚स्य ध‚र्मान्त‚रं किञ्चिन्नाशो नाम विधीय‚ते । चैत्र‚स्य‚{\tiny $_{lb}$}‚ पुत्र इत्य‚त्र य‚था‚{\tiny $_{६}$}‚ वास्त‚वो भेद‚स्त‚था भाव‚स्य नाश इत्य‚त्रापि व्य‚तिरेक‚विभ‚क्तेस्तु‚{\tiny $_{lb}$}‚ल्य‚त्वादित्य‚त आह । \textbf{नेत्या}दि । \textbf{न वै घोष}मात्रेण चैत्र‚स्य पुत्र इत्य‚नेन श‚ब्देन‚{\tiny $_{lb}$}‚ \textbf{साम्याद् विष‚यान्त‚र‚दृष्टो विधिः} । चैत्र‚स्य पुत्र इत्य‚त्र दृष्टोविधिर्वास्त‚वो यः स‚{\tiny $_{lb}$}‚ \textbf{स‚र्व‚त्र} भाव‚स्य नाश इत्य‚त्रापि \textbf{योज‚नाम‚र्ह‚ति} [।] श‚ब्द‚प्र‚वृत्तिमात्रेण व‚स्तुयोज‚नाया‚{\tiny $_{lb}$}‚ अयोगात् ।
	{\color{gray}{\rmlatinfont\textsuperscript{§~\theparCount}}}
	\pend% ending standard par
      ‚{\tiny $_{lb}$}‚

	  
	  \pstart \leavevmode% starting standard par
	एत‚देव \textbf{न ही}त्या‚{\tiny $_{७}$}‚दिना प्राह । क‚स्य‚चित् पुरुष‚स्य \textbf{ग‚र्द‚भ इति नाम‚क‚र‚णात् \leavevmode\ledsidenote{\textenglish{186a/PSVTa}}‚{\tiny $_{lb}$}‚ बालेय‚ध‚र्मा} ग‚र्द‚भ‚स्य ध‚र्मा \textbf{म‚नुष्येपि न हि योज्याः । त‚था न चैत्र‚स्य पुत्रो भ‚व‚ती‚{\tiny $_{lb}$}‚त्य‚त्र} वाक्ये \textbf{दृष्टो विधि}र‚र्थान्त‚र‚स्य पुत्र‚स्य विधानं दृष्ट‚मिति \textbf{नाशेपि} योज्यः ।‚{\tiny $_{lb}$}‚ भाव‚स्य नाशो भ‚व‚तीत्य‚त्रापि भावाद् व्य‚तिरेको नाशो विधेयः । किं कार‚ण‚म् [।]‚{\tiny $_{lb}$}‚ \textbf{विरोधात्} । नाश‚स्याभाव‚रूप‚त्वाद‚भाव‚स्य भ‚व‚न‚{\tiny $_{१}$}‚विरोधादित्युक्तं ।
	{\color{gray}{\rmlatinfont\textsuperscript{§~\theparCount}}}
	\pend% ending standard par
      ‚{\tiny $_{lb}$}‚

	  
	  \pstart \leavevmode% starting standard par
	य‚दि नाशो नार्थान्त‚रं क‚स्माद् भाव‚स्य नाशो भ‚व‚तीत्येव‚म‚भिधीय‚त इति [।]
	{\color{gray}{\rmlatinfont\textsuperscript{§~\theparCount}}}
	\pend% ending standard par
      ‚{\tiny $_{lb}$}‚

	  
	  \pstart \leavevmode% starting standard par
	अत आह । \textbf{एवं चे}त्यादि । भाव‚स्य नाश इत्य\textbf{भिधानेपि प्र‚योज‚न‚मावेदित‚{\tiny $_{lb}$}‚मेव} । अर्थान्त‚र‚मिव ध‚र्मिणो ध‚र्मं चेत‚साविभ‚ज्य त‚न्मात्र‚जिज्ञासायां स्व‚भाव‚{\tiny $_{lb}$}‚ एव त‚थोच्य‚त\edtext{\textsuperscript{*}}{\edlabel{pvsvt_525-1}\label{pvsvt_525-1}\lemma{*}\Bfootnote{नो ऋओओत्नोते ऋओउन्द् ओन् थिस् प‚गे!}} इत्यादिना निवेदित‚त्वात् ॥
	{\color{gray}{\rmlatinfont\textsuperscript{§~\theparCount}}}
	\pend% ending standard par
      ‚{\tiny $_{lb}$}‚

	  
	  \pstart \leavevmode% starting standard par
	त‚स्माद‚भाव‚स्याकिंचित्त्वात् त‚त्त्वान्य‚त्त्व‚विक‚ल्पो न तुल्यः‚{\tiny $_{२}$}‚ । अतो \textbf{भावे}‚{\tiny $_{lb}$}‚ ‚{\tiny $_{lb}$}‚ ‚{\tiny $_{lb}$}‚ \leavevmode\ledsidenote{\textenglish{526/s}}व‚स्तुनो भ‚व‚ने \textbf{एष} त‚त्त्वान्य‚त्त्व\textbf{विक‚ल्पः स्यात्} । किं कार‚ण‚म् [।] भ‚व‚न‚स्य \textbf{विधे‚{\tiny $_{lb}$}‚र्व‚स्त्व‚नुरोध‚तः} ।
	{\color{gray}{\rmlatinfont\textsuperscript{§~\theparCount}}}
	\pend% ending standard par
      ‚{\tiny $_{lb}$}‚

	  
	  \pstart \leavevmode% starting standard par
	\textbf{न‚न्व‚तिश‚योत्प‚त्ताव‚पि स एव त‚स्यातिश‚य उत्प‚न्न इति क‚थं} न‚ष्टो नाम [।]‚{\tiny $_{lb}$}‚ \textbf{तेन नायं त‚द‚व‚स्थो न‚ष्टो नाम । येन स्व‚यं न भ‚व‚ति । तेन न‚ष्टो नार्थान्त‚रोत्पा‚{\tiny $_{lb}$}‚दादित्युक्तं । न ह्य‚तिश‚योत्प‚त्त्या स्व‚यं न भूतो नाम[।]अभाव‚स्य स‚र्वातिश‚{\tiny $_{lb}$}‚योपाख्या निवृ‚{\tiny $_{४}$}‚त्त्या स‚र्व‚भाव‚ध‚र्म‚विवेक‚ल‚क्ष‚ण‚त्वात् । भाव‚स्य चोत्प‚त्तिस‚मावेश‚ल‚{\tiny $_{lb}$}‚क्ष‚ण‚त्वात्} ।
	{\color{gray}{\rmlatinfont\textsuperscript{§~\theparCount}}}
	\pend% ending standard par
      ‚{\tiny $_{lb}$}‚

	  
	  \pstart \leavevmode% starting standard par
	\textbf{भाव} इत्यादिना व्याच‚ष्टे । \textbf{भावो} भ‚व‚न‚मुत्पाद इति याव‚त् । \textbf{सोव‚श्य‚म्भ‚व‚न्त‚{\tiny $_{lb}$}‚म‚पेक्ष‚ते} । भ‚वितार‚म‚पेक्ष‚ते [।] भ‚वितार‚म‚न्त‚रेण भ‚व‚न‚स्याभावात् । \textbf{स च} भावः‚{\tiny $_{lb}$}‚ \textbf{व्यापारे स्व‚भाव एव} व‚स्त्वेव । किं कार‚णं [।] \textbf{निः‚{\tiny $_{३}$}‚स्व‚भाव‚स्य क्व‚चिद्} भ‚व‚{\tiny $_{lb}$}‚तीत्यादिके \textbf{स‚मावेशाभावात्} । स‚म्ब‚न्धाभावात् । न च व्यापारो नामार्थान्त‚रं ।‚{\tiny $_{lb}$}‚ किन्तु \textbf{व्यापार इति} हि \textbf{य‚थाभूत‚स्व‚भावोत्प‚त्ति}र्विशिष्ट‚स्व‚भावोत्प‚त्तिः [।] \textbf{सा}‚{\tiny $_{lb}$}‚ चोत्प‚त्ति\textbf{र्निःस्व‚भाव‚स्य} नाश‚स्य \textbf{क‚थं स्यात्} ।
	{\color{gray}{\rmlatinfont\textsuperscript{§~\theparCount}}}
	\pend% ending standard par
      ‚{\tiny $_{lb}$}‚

	  
	  \pstart \leavevmode% starting standard par
	य‚दि निःस्व‚भाव‚स्य नास्ति व्यापार‚स‚मावेशः \textbf{क‚थ‚मिदानीम्भ‚व‚त्य‚भावः श‚श‚{\tiny $_{lb}$}‚विषाण‚मित्यादिव्य‚व‚हारः} श‚{\tiny $_{४}$}‚श‚विषाण‚म‚भावो भ‚व‚तीति भ‚व‚न‚ल‚क्ष‚णेन व्यापारेण‚{\tiny $_{lb}$}‚ व्य‚व‚हार इत्य‚र्थः । आदिश‚ब्दाद् व‚न्ध्यासुतोऽभावो भ‚व‚तीति प‚रिग्र‚हः ।
	{\color{gray}{\rmlatinfont\textsuperscript{§~\theparCount}}}
	\pend% ending standard par
      ‚{\tiny $_{lb}$}‚

	  
	  \pstart \leavevmode% starting standard par
	\textbf{नेत्}यादिना प‚रिह‚र‚ति । \textbf{न वै श‚श‚विषाणं किंचि}द‚भावोन्य‚द्वा \textbf{भ‚व‚तीति} विधि‚{\tiny $_{lb}$}‚\textbf{नोच्य‚ते । अपि त्वेव}मिति श‚श‚विषाण‚म‚भावो भ‚व‚तीत्य‚नेन वाक्येना\textbf{स्ये}ति श‚श‚{\tiny $_{lb}$}‚विषाण‚स्याभावो भ‚व‚ति [।] भावो \textbf{न भ‚व‚तीति भाव‚प्र‚ति‚{\tiny $_{५}$}‚षेध एव क्रिय}ते । \textbf{अपि‚{\tiny $_{lb}$}‚ च व्य‚व‚ह‚र्त्तारः} पुरुषाः । \textbf{एत}च्छ‚श‚विषाणादिक\textbf{मेव}म‚भावो भ‚व‚तीति \textbf{व्यापार‚व‚दिव‚{\tiny $_{lb}$}‚ स‚मारोप्याद‚र्श‚य‚न्ति । केन‚चित् प्र‚क‚र‚णेन} । किं श‚श‚विषाणादिक‚म‚भावो भ‚व‚ति न‚{\tiny $_{lb}$}‚ भ‚व‚तीति प्र‚स्ताव‚स‚माश्र‚येण । \textbf{न तु त‚च्}छ‚श‚विषाणादिकं व्य‚व‚हार‚मात्रेण \textbf{त‚था}‚{\tiny $_{lb}$}‚ व्यापार‚युक्त‚म्भ‚व‚ति । य‚स्मात् \textbf{स‚र्वार्थ‚विवेच‚नं} स‚र्वार्थ‚स्व‚भाव‚{\tiny $_{६}$}‚विर‚ह\textbf{स्त‚त्र} श‚श‚वि‚{\tiny $_{lb}$}‚षाणादौ \textbf{त‚त्त्वं । न} त्व‚स‚तः \textbf{क‚स्य‚चिद्} भ‚व‚नादेः \textbf{स‚मावेशः} ।
	{\color{gray}{\rmlatinfont\textsuperscript{§~\theparCount}}}
	\pend% ending standard par
      ‚{\tiny $_{lb}$}‚‚{\tiny $_{lb}$}‚\textsuperscript{\textenglish{527/s}}

	  
	  \pstart \leavevmode% starting standard par
	स‚हेतुकोपि विनाश एव‚म्भ‚विष्य‚तीति चेदाह । \textbf{ने}त्यादि । \textbf{एवं} श‚श‚विषा‚{\tiny $_{lb}$}‚ण‚व‚त् स‚र्वार्थ‚विर‚ह‚ल‚क्ष‚णो \textbf{विनाशः} प‚रेष्टः । किं कार‚ण‚म् [।] \textbf{व‚स्तुनि त}स्य विना‚{\tiny $_{lb}$}‚श‚स्य \textbf{भावा}दुत्प‚त्तेः । य‚श्च भ‚व‚ति स क‚थ‚म‚भावो विरोधात् । \leavevmode\ledsidenote{\textenglish{186b/PSVTa}}
	{\color{gray}{\rmlatinfont\textsuperscript{§~\theparCount}}}
	\pend% ending standard par
      ‚{\tiny $_{lb}$}‚

	  
	  \pstart \leavevmode% starting standard par
	\textbf{य‚दि} पुन\textbf{र‚साव‚पि} विनाशो निःस्व‚भाव एव केव‚लं‚{\tiny $_{७}$}‚ \textbf{व‚क्तृभिरेव‚म्}भ‚व‚तीति व्या‚{\tiny $_{lb}$}‚पार‚वानिव \textbf{ख्याप्य‚ते । न तु स्व‚य‚न्त‚था}भ‚व‚न‚ध‚र्मा नीरूप‚त्वाद‚स्य । \textbf{त‚दै}व‚मिष्य‚{\tiny $_{lb}$}‚माणेऽभावो भ‚व‚तीत्य‚पि ब्रुवाणेन । \textbf{न किञ्चिद् भ‚व‚तीतीष्ट‚मेव} । क्रियाप्र‚तिषेध‚{\tiny $_{lb}$}‚मात्र‚त्वाद‚स्य वाक्य‚स्य । \textbf{त‚स्मात् स्व‚य}म‚न‚ध्यारोपितेनाकारेण क्व‚चिद् व‚स्तुनि‚{\tiny $_{lb}$}‚ \textbf{भ‚व‚न् स्व‚भावो विक‚ल्प‚द्व‚यं नातिव‚र्त्तेते त‚त्त्व‚म‚न्य‚त्त्वं चेति} प्र‚कारान्त‚राभावात्‚{\tiny $_{lb}$}‚ ० ॥ \href{http://sarit.indology.info/?cref=pv.3.277}{२८०}
	{\color{gray}{\rmlatinfont\textsuperscript{§~\theparCount}}}
	\pend% ending standard par
      ‚{\tiny $_{lb}$}‚

	  
	  \pstart \leavevmode% starting standard par
	रूपादिस्क‚न्ध‚स्व‚भावः पुद्ग‚लो न भ‚व‚{\tiny $_{१}$}‚त्य‚थ च रूपादिभ्यो नान्यः । त‚स्म‚{\tiny $_{lb}$}‚ त‚त्त्वान्य‚त्त्व‚म‚तिव‚र्त्त‚त एव स्व‚भाव इति चेत् ।
	{\color{gray}{\rmlatinfont\textsuperscript{§~\theparCount}}}
	\pend% ending standard par
      ‚{\tiny $_{lb}$}‚

	  
	  \pstart \leavevmode% starting standard par
	त‚न्न [।] य‚स्मा\textbf{द‚त‚त्त्व‚मेवा}त‚त्स्व‚भाव‚त्व‚मेव \textbf{स्व‚भाव‚स्यान्य‚त्त्व‚मिति} ।
	{\color{gray}{\rmlatinfont\textsuperscript{§~\theparCount}}}
	\pend% ending standard par
      ‚{\tiny $_{lb}$}‚

	  
	  \pstart \leavevmode% starting standard par
	य‚दि पुद्ग‚लोपि न स्क‚न्ध‚स्व‚भाव‚स्त‚दा स्क‚न्धेभ्योन्य एव । य‚तो \textbf{न हि} प्र‚सि‚{\tiny $_{lb}$}‚द्धान्य‚त्त्व‚यो \textbf{रूप‚र‚स‚योर‚प्य‚न्य‚देव प‚र‚स्प‚र‚म‚न्य‚त्त्वं} [।] किन्त्व‚त‚त्स्व‚भाव‚त्व‚मेवान्य‚{\tiny $_{lb}$}‚त्त्व‚न्त‚च्च पुद्ग‚लेप्य‚स्तीति सोपि स्क‚न्धेभ्योन्य एवेष्ट‚व्यः ॥
	{\color{gray}{\rmlatinfont\textsuperscript{§~\theparCount}}}
	\pend% ending standard par
      ‚{\tiny $_{lb}$}‚

	  
	  \pstart \leavevmode% starting standard par
	न‚न्व‚त‚त्स्व‚भाव‚त्वे‚{\tiny $_{२}$}‚पि प‚र‚स्प‚रं \textbf{स्व‚भावाप्र‚तिब‚न्धोन्य‚त्त्व‚मिति चेत्} । स च‚{\tiny $_{lb}$}‚प्र‚तिब‚न्धः पुद्ग‚ल‚स्य स्क‚न्धेष्व‚स्ति त‚तो त‚त्स्व‚भाव‚त्वेपि नान्य‚त्त्वं स्क‚न्धेभ्यः पुद्‚{\tiny $_{lb}$}‚ग‚ल‚स्येति ।
	{\color{gray}{\rmlatinfont\textsuperscript{§~\theparCount}}}
	\pend% ending standard par
      ‚{\tiny $_{lb}$}‚

	  
	  \pstart \leavevmode% starting standard par
	\textbf{कोय‚मि}त्यादिना प्र‚तिषेध‚ति । \textbf{कोयं प्र‚तिब‚न्धो नाम} पुद्ग‚ल‚स्य स्क‚न्धेषु \textbf{येन}‚{\tiny $_{lb}$}‚ प्र‚तिब‚न्धेन । \textbf{स च न स्यादि}ति स्क‚न्ध‚स्व‚भाश्च पुद्ग‚लो न स्यात् । \textbf{नान्य}स्व‚भाव‚श्\textbf{च}‚{\tiny $_{lb}$}‚ स्क‚न्धेभ्यः । अन्यः स्व‚भावोस्येति विग्र‚हः ।
	{\color{gray}{\rmlatinfont\textsuperscript{§~\theparCount}}}
	\pend% ending standard par
      ‚{\tiny $_{lb}$}‚

	  
	  \pstart \leavevmode% starting standard par
	स्क‚न्धेभ्यः पुद्ग‚ल‚स्य ज‚{\tiny $_{३}$}‚न्म त‚देव \textbf{ज‚न्म} प्र‚तिब‚न्ध \textbf{इति चेत्} ।
	{\color{gray}{\rmlatinfont\textsuperscript{§~\theparCount}}}
	\pend% ending standard par
      ‚{\tiny $_{lb}$}‚‚{\tiny $_{lb}$}‚\textsuperscript{\textenglish{528/s}}

	  
	  \pstart \leavevmode% starting standard par
	एवं स‚ति कार्य‚त्वात् स्क‚न्धेभ्यः पुद्ग‚ल‚स्य त‚त्त्वान्य‚त्त्वेनावाच्य‚त्व‚मिष्टं । त‚था‚{\tiny $_{lb}$}‚ च स‚ति \textbf{स‚र्व‚कार‚णानाम्प‚र‚स्प‚र‚म‚वाच्य‚ता स्यात् । त‚था चे}ति कार्य‚त्वाद‚वाच्य‚त्वे ।‚{\tiny $_{lb}$}‚ \textbf{स‚र्वः स‚र्व‚स्य क‚थंचि}दिति साक्षात् पार‚म्प‚र्येण \textbf{चोप‚योगीति} स‚र्व‚त्र कार्य‚कार‚ण‚भावा‚{\tiny $_{lb}$}‚\textbf{न्न} च \textbf{क‚श्चित्} कुत‚श्चि\textbf{द‚न्यः स्यात् । एवं चा}न‚न्त‚रोक्तेनावाच्य‚ताल‚क्ष‚णेना\textbf{वाच्य‚{\tiny $_{४}$}‚‚{\tiny $_{lb}$}‚तेत्य‚पि} ब्रुव‚ता \textbf{कार्य‚कार‚ण‚भाव एव श‚ब्दान्त‚रेणोक्तः स्यान्नार्थ‚भेदः} क‚श्चित् ।‚{\tiny $_{lb}$}‚ अन्य‚त्त्व‚न्तु न निषिद्ध‚मेव । य‚स्मात् \textbf{स्व‚भाव}योः प‚र‚स्प‚र‚म\textbf{न‚नुग‚म}न‚म‚मिश्रीभ‚व‚न‚{\tiny $_{lb}$}‚\textbf{म‚न्य‚त्त्व‚म्ब्रूमः} । स च स्व‚भावान‚नुग‚मः \textbf{स्व‚भाव‚व‚तां} स‚र्व‚प‚दार्थाना\textbf{म‚स्त्येवे}ति \textbf{प‚र‚{\tiny $_{lb}$}‚स्प‚र‚म‚न्य‚त्त्व‚मेव} । न चान्यः प्र‚तिब‚न्धः पुद्ग‚ल‚स्य स्क‚न्धेषु । य‚स्मा\textbf{न्न} हि \textbf{ज‚न्म‚{\tiny $_{lb}$}‚ल‚क्ष‚णा}ज्ज‚न्म‚स्व‚भावात् \textbf{स्व‚भाव‚प्र‚तिब‚न्धाद‚न्यः‚{\tiny $_{५}$}‚ प्र‚तिब‚न्धो नाम} । किं कार‚ण‚म् [।]‚{\tiny $_{lb}$}‚ \textbf{अनाय‚त्त‚स्य} त‚दुत्प‚त्त्या त‚त्राप्र‚तिब‚द्ध‚स्य । तेन स‚ह यो \textbf{व्य‚भिचार}स्त‚स्य्\textbf{आविरोधात् ।‚{\tiny $_{lb}$}‚ त‚तो}प्र‚तिब‚न्धात् पुद्ग‚ल‚स्य स्क‚न्धेभ्योन्य‚त्त्वं । \textbf{ध‚र्म‚भेदाच्चा}न्य‚त्त्वं । त‚था ह्य‚वाच्य‚त्वं‚{\tiny $_{lb}$}‚ पुद्ग‚ल‚स्य ध‚र्मः स्क‚न्धानान्तु प‚र‚स्प‚र‚म्वाच्य‚त्व‚मिति ध‚र्म‚भेदः ॥
	{\color{gray}{\rmlatinfont\textsuperscript{§~\theparCount}}}
	\pend% ending standard par
      ‚{\tiny $_{lb}$}‚

	  
	  \pstart \leavevmode% starting standard par
	य‚द्य‚पि न ज‚न्म‚कृतः प्र‚तिब‚न्ध‚स्त‚थापि पुद्ग‚ल‚स्य स्क‚न्धेषु ज्ञा\textbf{न‚कृतः प्र‚तिब‚न्ध‚{\tiny $_{lb}$}‚ इति चेत्} ।
	{\color{gray}{\rmlatinfont\textsuperscript{§~\theparCount}}}
	\pend% ending standard par
      ‚{\tiny $_{lb}$}‚

	  
	  \pstart \leavevmode% starting standard par
	\textbf{स्यादित्या}दि‚{\tiny $_{६}$}‚नैत‚देव व्याच‚ष्टे । य‚स्य रूपादेः प्र‚तिप‚त्ति\textbf{र्य‚त्प्र‚तिप‚त्ति}स्त‚या‚{\tiny $_{lb}$}‚ \textbf{नान्त‚रीय‚क}म‚विनाभावि \textbf{य‚ज्ज्ञानं} य‚स्य पुद्ग‚ल‚स्य ज्ञानं । \textbf{त‚द्ग‚ता}विति रूपादिग‚तौ‚{\tiny $_{lb}$}‚ \textbf{निय‚मेन} त‚स्य पुद्ग‚ल‚स्य \textbf{प्र‚तिभास‚नात्} । ज्ञान‚कृतः प्र‚तिब‚न्ध‚स्त‚था हि रूप‚श‚ब्दादि‚{\tiny $_{lb}$}‚ग्र‚ह‚णेनैव पुद्ल‚ग्र‚ह‚ण‚मिष्य‚ते । च‚क्षुरादिविज्ञान‚विज्ञेय‚त्वात् पुद्ग‚ल‚स्येति । तेन‚{\tiny $_{lb}$}‚ \leavevmode\ledsidenote{\textenglish{187a/PSVTa}} ज्ञान‚कृतात् प्र‚तिब‚न्धात् । त‚त्पु‚{\tiny $_{७}$}‚द्ग‚लाख्य‚म्व‚स्तु स्क‚न्धेभ्योन्य‚त्त्वेना\textbf{वाच्य‚म‚त‚द्रूप‚{\tiny $_{lb}$}‚म‚प्}य‚स्क‚न्ध‚स्व‚भाव‚म‚पि ।
	{\color{gray}{\rmlatinfont\textsuperscript{§~\theparCount}}}
	\pend% ending standard par
      ‚{\tiny $_{lb}$}‚

	  
	  \pstart \leavevmode% starting standard par
	\textbf{ने}त्यादिना प्र‚तिषेध‚ति । \textbf{न} पुद्ग‚ल‚स्य रूपादिप्र‚तिप‚त्तिनान्त‚रीय‚कं ज्ञानं । किं‚{\tiny $_{lb}$}‚ ‚{\tiny $_{lb}$}‚ \leavevmode\ledsidenote{\textenglish{529/s}}कार‚णं [।] \textbf{त‚स्य} पुद्ग‚ल‚स्य रूपादिस्व‚भाव‚म‚प‚हाय \textbf{निःस्व‚भाव‚त्वात् स्व‚यं} । य‚स्मात्‚{\tiny $_{lb}$}‚ \textbf{स एव हि त‚स्य स्व‚भावो यो} रूपादिरूपः \textbf{प्र‚तिभाति} ।
	{\color{gray}{\rmlatinfont\textsuperscript{§~\theparCount}}}
	\pend% ending standard par
      ‚{\tiny $_{lb}$}‚

	  
	  \pstart \leavevmode% starting standard par
	अथारूपादिस्व‚भावः पुद्ग‚लः । त‚दाऽरूपादि\textbf{स्व‚भाव‚त्वेऽस्य} पुद्ग‚ल‚स्या\textbf{त‚द्व‚{\tiny $_{१}$}‚द्}‚{\tiny $_{lb}$}‚रूपादिव‚त् पृथ‚क्\textbf{प्र‚तिभाव‚प्र‚स‚ङ्गात्} । न च प्र‚तिभास‚ते त‚तो नास्त्येव पुद्ग‚लः ।‚{\tiny $_{lb}$}‚ य‚तो \textbf{दृश्य‚स्याप्र‚तिभास‚मान‚स्य चाभावात्} । अथादृश्यः पुद्ग‚ल इष्य‚ते [।] त‚दा‚{\tiny $_{lb}$}‚ \textbf{अदृश्य‚त्वेपि} पुद्ग‚ल‚स्येष्य‚माणे \textbf{न त‚द्रूपं} ज्ञान‚न्न पुद्ग‚लाकारं \textbf{ज्ञान‚मिति क‚स्य‚{\tiny $_{lb}$}‚ किमाय‚त्ता प्र‚तीतिः} । न रूपादिज्ञान‚नान्त‚रीय‚कं पुद्ग‚ल‚ज्ञान‚मित्य‚र्थः । त‚था च न‚{\tiny $_{lb}$}‚ ज्ञान‚कृतः प्र‚तिब‚न्ध इति भावः‚{\tiny $_{२}$}‚ ।
	{\color{gray}{\rmlatinfont\textsuperscript{§~\theparCount}}}
	\pend% ending standard par
      ‚{\tiny $_{lb}$}‚

	  
	  \pstart \leavevmode% starting standard par
	रूपाद्याय‚त्त‚प्र‚तीतित्वादेव पृथ‚क् पुद्ग‚लो न प्र‚तिभास‚त इति चेद् [।]
	{\color{gray}{\rmlatinfont\textsuperscript{§~\theparCount}}}
	\pend% ending standard par
      ‚{\tiny $_{lb}$}‚

	  
	  \pstart \leavevmode% starting standard par
	आह । \textbf{न चे}त्यादि । \textbf{य‚द्} व‚स्तु \textbf{य‚दाय‚त्त‚प्र‚तीतिकं} य‚त्प्र‚तिब‚द्धोप‚ल‚म्भ‚न\textbf{न्त‚स्य‚{\tiny $_{lb}$}‚ स्व‚भाव‚प्र‚तिभास एव} न च \textbf{न‚श्य‚तीति} स‚म्ब‚न्धः । \textbf{किमिवे}त्याह । \textbf{प्र‚काशे}त्यादि ।‚{\tiny $_{lb}$}‚ य‚था \textbf{नीलादीनामा}लोक‚प्र‚तिब‚द्ध‚ज्ञानानामालोके प्र‚तिभास‚मानेपि स्व‚प्र‚तिभासो न‚{\tiny $_{lb}$}‚ न‚श्य‚ति । आलोक‚व्य‚तिरेकेण तेषां प्र‚{\tiny $_{३}$}‚तिभास‚नात् । त‚द्व‚त् पुद्ग‚ल‚स्यापि स्यात् ॥
	{\color{gray}{\rmlatinfont\textsuperscript{§~\theparCount}}}
	\pend% ending standard par
      ‚{\tiny $_{lb}$}‚

	  
	  \pstart \leavevmode% starting standard par
	अपि च \textbf{का वा त‚स्य} पुद्ग‚ल‚स्य \textbf{प्र‚त्यास}त्तिः स‚म्ब‚न्ध\textbf{स्त‚त्र} स्क‚न्धे । \textbf{य}दिति‚{\tiny $_{lb}$}‚ येन प्र‚त्यास‚त्तिकार‚णेन \textbf{त‚स्मिं}स्क‚न्धेऽ\textbf{नात्म‚रूपे}ऽपुद्ग‚ल‚स्व‚भावे प्र‚तिभास‚माने \textbf{स्व}य‚{\tiny $_{lb}$}‚\textbf{म्प्र‚त्युप‚तिष्ठ‚ते} । आत्मानं ग्राह‚य‚तीति याव‚त् ।
	{\color{gray}{\rmlatinfont\textsuperscript{§~\theparCount}}}
	\pend% ending standard par
      ‚{\tiny $_{lb}$}‚

	  
	  \pstart \leavevmode% starting standard par
	\textbf{अतिप्र‚संगो ह्येवं स्यात्} । अप्र‚तिब‚द्धे प्र‚तिभास‚माने य‚दि निय‚मेन पुद्ग‚लः‚{\tiny $_{lb}$}‚ प्र‚तिभासेत । त‚दा य‚स्य क‚स्य‚चि‚{\tiny $_{४}$}‚त्प्र‚तिभास‚नेन्योप्य‚त्य‚न्तास‚म्ब‚न्धः प्र‚तीय‚त‚{\tiny $_{lb}$}‚ इत्य‚र्थः । \textbf{प्र‚तीय‚मान‚स्य} पुद्ग‚ल‚स्य \textbf{त‚दुपादान‚ता}रूपाद्युपादान‚ताप्र‚त्यास‚त्ति\textbf{रिति}‚{\tiny $_{lb}$}‚ चेत् [।] \textbf{कोय‚मुपादानार्थः । न} ताव‚त् पुद्ग‚ल‚स्य रूपादीनाञ्च य‚थाक्र‚म‚ङ्\textbf{कार्य‚{\tiny $_{lb}$}‚कार‚ण‚भावः} । त‚स्यान‚भ्युप‚ग‚मात् । कार्य‚कार‚ण‚भावा\textbf{भ्युप‚ग‚मे वा न} रूपादिद‚र्श‚ने‚{\tiny $_{lb}$}‚ निय‚मेन पुद्ग‚ल‚स्य द‚र्श‚नं । किं कार‚णं [।] य‚तो‚{\tiny $_{५}$}‚ \textbf{न कार्य‚कार‚णे । अन्योन्य‚{\tiny $_{lb}$}‚प्र‚तीतिप्र‚त्युप‚स्थाप‚ने} । य‚था कार्यात् कार‚ण‚प्र‚तीतिस्त‚था न कार‚णात् कार्य‚प्र‚ती‚{\tiny $_{lb}$}‚‚{\tiny $_{lb}$}‚ \leavevmode\ledsidenote{\textenglish{530/s}}तिर्भ‚व‚तीत्य‚र्थः ॥ न पुद्ग‚ल‚स्य रूपादिनान्त‚रीय‚क‚ता किन्तु पुद्ग‚ल‚स्य या प्र‚तीति‚{\tiny $_{lb}$}‚स्त‚स्याः । \textbf{त‚न्नान्त‚रीय‚क‚ता} रूपादिनान्त‚रीय‚क‚ता ।
	{\color{gray}{\rmlatinfont\textsuperscript{§~\theparCount}}}
	\pend% ending standard par
      ‚{\tiny $_{lb}$}‚

	  
	  \pstart \leavevmode% starting standard par
	सैव \textbf{प्र‚त्यास‚त्तिरिति चेत्} ।
	{\color{gray}{\rmlatinfont\textsuperscript{§~\theparCount}}}
	\pend% ending standard par
      ‚{\tiny $_{lb}$}‚

	  
	  \pstart \leavevmode% starting standard par
	\textbf{न‚नु सैव} प्र‚तीतेस्त‚न्नान्त‚रीय‚क‚ता । रूपादिषु पुद्ग‚ल‚स्य्\textbf{आस‚ति प्र‚तिब‚न्धे‚{\tiny $_{lb}$}‚ न यु‚{\tiny $_{६}$}‚क्तेत्युच्य‚ते} ।
	{\color{gray}{\rmlatinfont\textsuperscript{§~\theparCount}}}
	\pend% ending standard par
      ‚{\tiny $_{lb}$}‚

	  
	  \pstart \leavevmode% starting standard par
	अकार्य‚कार‚ण‚योर‚पि पुद्ग‚ल‚रूपाद्योः प्र‚तिब‚न्धो भ‚विष्य‚तीति [।]
	{\color{gray}{\rmlatinfont\textsuperscript{§~\theparCount}}}
	\pend% ending standard par
      ‚{\tiny $_{lb}$}‚

	  
	  \pstart \leavevmode% starting standard par
	अत आह । \textbf{अकार्य‚कार‚ण‚योर्न क‚श्चिद्} वास्त‚वः \textbf{प्र‚तिब‚न्ध इत्}य‚स‚कृ\textbf{दुक्तं य‚त्प्र‚ति‚{\tiny $_{lb}$}‚प‚त्तिनान्त‚रीय‚कं य‚ज्ज्ञान‚मित्य‚पि} य‚दुच्य‚ते । \textbf{त‚ज्ज्ञाने} रूपादिविवेके । पुद्ग‚ल‚ज्ञाने‚{\tiny $_{lb}$}‚ \textbf{स‚ति स्यात्} । त‚च्च नास्ति । य‚तो \textbf{यः} पुद्ग‚ले \textbf{विज्ञाने} स्व‚रूपेण न प्र‚ति\textbf{भास‚ते‚{\tiny $_{lb}$}‚ \leavevmode\ledsidenote{\textenglish{187b/PSVTa}} स्व‚रूपासंस‚र्गिणा}न्यासंस‚{\tiny $_{७}$}‚र्गेणेत्य‚र्थः । \textbf{त‚स्य किञ्चिज्ज्ञानं न ही}ति स‚म्ब‚न्धः ।‚{\tiny $_{lb}$}‚ \textbf{त‚द‚भावाद्} य‚थोक्त‚ज्ञानाभावा\textbf{द‚र्थ‚रूप‚स्य} पुद्ग‚लाख्य‚स्या\textbf{वाच्य‚ताल‚क्ष‚णं न सिध्य‚ति} ।
	{\color{gray}{\rmlatinfont\textsuperscript{§~\theparCount}}}
	\pend% ending standard par
      ‚{\tiny $_{lb}$}‚

	  
	  \pstart \leavevmode% starting standard par
	\textbf{त‚दि}ति त‚स्माद् \textbf{व‚स्तुतः} प‚र‚मार्थ‚तः क्व‚चिद् \textbf{भ‚व‚ता} केन‚चिद‚र्थेन त‚त्र \textbf{त‚त्त्वान्य‚{\tiny $_{lb}$}‚त्व‚भाजा भ‚वित‚व्यं} । व‚स्तुनो ग‚त्य‚न्त‚राभावात् । \textbf{य‚स्य तु} क्ष‚णिक‚वादिनो \textbf{विन‚श्य‚तो‚{\tiny $_{lb}$}‚ भाव‚स्य न किञ्चिद् भ‚व‚ति} केव‚लं स भावः स्व‚य‚मेव न भ‚व‚ती‚{\tiny $_{१}$}‚ति म‚तं । \textbf{तेनाभावो‚{\tiny $_{lb}$}‚ भ‚व‚तीत्य‚पि} ब्रुव‚ता \textbf{न भावो भ‚व‚ती}ति प्र‚तिषेध‚मात्र‚मे\textbf{वोक्तं} न क‚स्य‚चिद् विधानं ।‚{\tiny $_{lb}$}‚ त‚तो नाभावंप्र‚ति क्ष‚णिक‚वादिन‚स्त‚त्त्वान्य‚त्व‚विक‚ल्प‚स्याव‚तारोस्तीति म‚न्य‚ते ।
	{\color{gray}{\rmlatinfont\textsuperscript{§~\theparCount}}}
	\pend% ending standard par
      ‚{\tiny $_{lb}$}‚

	  
	  \pstart \leavevmode% starting standard par
	\textbf{य‚द‚पी}त्यादिना व्याच‚ष्टे । \textbf{य‚द‚प्य‚यं} क्ष‚णिक‚वादी \textbf{भाव‚स्याभावो भ‚व‚तीति}‚{\tiny $_{lb}$}‚ विधिसंस्प‚र्शिनेव श‚ब्दे\textbf{नाह । त‚द‚पि भावो न भ‚व‚तीत्येवोक्तं भ‚व‚ति । एवं हि स}‚{\tiny $_{lb}$}‚ भा‚{\tiny $_{२}$}‚वो \textbf{निव‚र्तितो भ‚व‚ति} य‚दि किञ्चिन्न विधीय‚ते । किं कार‚णं [।] \textbf{प्र‚तिषेधे}‚{\tiny $_{lb}$}‚ भाव‚मात्र‚व्य‚व‚च्छेदे \textbf{विधेर‚स‚म्भ‚वात्} ।
	{\color{gray}{\rmlatinfont\textsuperscript{§~\theparCount}}}
	\pend% ending standard par
      ‚{\tiny $_{lb}$}‚

	  
	  \pstart \leavevmode% starting standard par
	य‚त‚श्च विन‚श्य‚तो भाव‚स्य न क‚श्चिद् व‚स्तुध‚र्मो भ‚व‚ति । \textbf{त‚त एवास्य} भाव‚स्य‚{\tiny $_{lb}$}‚ \textbf{विनाशे न क‚श्चिद्धेतुः । त‚था हि} विन‚श्य‚ता भावेना\textbf{पेक्षेत प‚रो} विनाश‚हेतुः । \textbf{य‚दि}‚{\tiny $_{lb}$}‚ ‚{\tiny $_{lb}$}‚ \leavevmode\ledsidenote{\textenglish{531/s}}तेन भाव‚स्य \textbf{कार्यं} क‚र्त्त‚व्य\textbf{म्विद्येत किञ्च‚न} । न तु किञ्चित् कार्य‚म‚स्ति ।‚{\tiny $_{lb}$}‚ त‚स्माद\textbf{किञ्चित्क‚र}म्वि‚{\tiny $_{३}$}‚नाश‚कार‚णं । \textbf{य‚च्चाकिञ्चित्क‚र‚म्व‚स्तु । त‚त्किं केन‚{\tiny $_{lb}$}‚चिद‚पेक्ष्य‚ते} । नैवापेक्ष्य‚ते ।
	{\color{gray}{\rmlatinfont\textsuperscript{§~\theparCount}}}
	\pend% ending standard par
      ‚{\tiny $_{lb}$}‚

	  
	  \pstart \leavevmode% starting standard par
	\textbf{स‚ती}त्यादिना व्याच‚ष्टे । \textbf{स‚ति हि} क‚र्त्त‚व्ये \textbf{कार‚क‚म्भ}व‚ति । \textbf{न च न‚श्य‚तो‚{\tiny $_{lb}$}‚ भाव‚स्य किंचित् कार्य‚मित्युक्तं । त‚स्माद् यो नाम} क‚श्चि\textbf{न्नाश‚हेतुः स भावे न‚{\tiny $_{lb}$}‚ किञ्चित् क‚रोतीत्य‚किञ्चित्क‚रो नापेक्ष‚णीयो} विनाश‚हेतुः ॥
	{\color{gray}{\rmlatinfont\textsuperscript{§~\theparCount}}}
	\pend% ending standard par
      ‚{\tiny $_{lb}$}‚

	  
	  \pstart \leavevmode% starting standard par
	\textbf{त‚त्क‚थ}मित्यादि प‚रः । य‚दि विन‚श्य‚तो नातिश‚यः क‚श्चिदुत्प‚द्य‚ते । \textbf{क‚थ‚{\tiny $_{४}$}‚}‚{\tiny $_{lb}$}‚मिदानीम‚नुत्प‚न्नातिश‚यः । \textbf{अनुत्प‚न्नोतिश‚योस्}येति विग्र‚हः । \textbf{त‚द‚व‚स्थ एव} पूर्व‚रूपा‚{\tiny $_{lb}$}‚व‚स्थ एव \textbf{भावो} वि\textbf{न‚ष्टो नाम} ।
	{\color{gray}{\rmlatinfont\textsuperscript{§~\theparCount}}}
	\pend% ending standard par
      ‚{\tiny $_{lb}$}‚

	  
	  \pstart \leavevmode% starting standard par
	\textbf{न‚न्वि}त्यादि सि द्धा न्त वा दी । विनाश‚हेतोः स‚काशाद् भाव‚स्या\textbf{तिश‚योत्प‚{\tiny $_{lb}$}‚त्ताव‚प्य}ङ्गीक्रिय‚माणायां \textbf{स एव त‚स्यातिश‚यो} नाशाख्यः । \textbf{उत्प‚न्न इति} स भावः‚{\tiny $_{lb}$}‚ \textbf{क‚थं न‚ष्टो नाम} । न ह्य‚न्य‚भावेन्य‚स्य नाशः । य‚त एव\textbf{न्तेन} कार‚णे\textbf{नाय}‚{\tiny $_{५}$}‚म्भाव‚स्\textbf{त‚द‚{\tiny $_{lb}$}‚व‚स्थो न न‚ष्टो नाम} । किन्तु \textbf{येन} य‚स्मात् \textbf{स्व‚यं न भ‚व‚ति तेन न‚ष्टो नार्थान्त}र‚स्य‚{\tiny $_{lb}$}‚ नाशाख्य‚स्यो\textbf{त्पादादित्य}न‚न्त‚र‚मे\textbf{वोक्तं} । य‚तो \textbf{न हि} नाशाख्य‚स्या\textbf{तिश}य‚स्यो\textbf{त्प‚त्त्या}‚{\tiny $_{lb}$}‚ भावः \textbf{स्व‚यं न भूतो नाम} । किं कार‚ण‚म् [।] \textbf{अभाव‚स्य स‚र्वे येतिश‚याः} ।‚{\tiny $_{lb}$}‚ साम‚र्थ्य‚ल‚क्ष‚णाः । याश्\textbf{चोपाख्याः} व्य‚प‚देशास्तेषां \textbf{निवृत्त्या} । स‚र्व‚साम‚र्थ्य‚व्य‚प‚देश‚नि‚{\tiny $_{lb}$}‚वृत्त्येत्य‚र्थः । \textbf{स‚र्व}स्य \textbf{भाव‚ध‚{\tiny $_{६}$}‚र्म}स्य भ‚व‚न‚रूप‚स्य ध‚र्म‚स्य यो \textbf{विवेको} विर‚ह‚स्त\textbf{ल्ल‚क्ष‚{\tiny $_{lb}$}‚ण‚त्वात् । भाव}स्य \textbf{चोत्प‚त्तिस‚मावेश‚ल‚क्ष‚ण‚त्वात्} । य‚स्माद् भ‚व‚तीति भाव उच्य‚ते ।‚{\tiny $_{lb}$}‚ तेनोत्प‚त्तियोगी भावः । य‚त‚श्चैव\textbf{न्त‚स्मान्नाभावे} भाव‚स्य विनाशे \textbf{क‚स्य‚चिद‚न्य‚स्य‚{\tiny $_{lb}$}‚ भावो} भ‚व‚न‚न्त‚स्यो\textbf{प‚क्षेपः} । न क‚स्य‚चिद् भ‚व‚न‚मित्य‚र्थः ।
	{\color{gray}{\rmlatinfont\textsuperscript{§~\theparCount}}}
	\pend% ending standard par
      ‚{\tiny $_{lb}$}‚‚{\tiny $_{lb}$}‚\textsuperscript{\textenglish{532/s}}

	  
	  \pstart \leavevmode% starting standard par
	\leavevmode\ledsidenote{\textenglish{188a/PSVTa}} \textbf{एतेने}ति स्व‚भाव‚प्र‚तिपाद‚नेन । अहेतु\textbf{क‚त्वेपि} नाश‚{\tiny $_{७}$}‚स्याङ्गीक्रिय‚माणे स नाशः‚{\tiny $_{lb}$}‚ प्र‚थ‚म‚म\textbf{भूत्वा} भ‚व‚तीत्येव‚म‚भूत्वा \textbf{ना}श‚स्य \textbf{भाव‚तः} कार‚णात् त‚स्य नाश‚स्यांकुरादिव‚त्‚{\tiny $_{lb}$}‚ स‚त्ता स्यात् । नाशित्वं चेति । \textbf{स‚त्तानाशित्व‚दोष‚स्य} य‚त् \textbf{प्र‚स‚ञ्ज}न‚न्त‚त्\textbf{प्र‚त्याख्यात‚{\tiny $_{lb}$}‚मेते}नैव ।
	{\color{gray}{\rmlatinfont\textsuperscript{§~\theparCount}}}
	\pend% ending standard par
      ‚{\tiny $_{lb}$}‚

	  
	  \pstart \leavevmode% starting standard par
	\textbf{योपी}त्यादिना व्याच‚ष्टे । \textbf{अहेतुकेपि नाशे}ऽस्य नाश\textbf{स्याभूत्वा भावात् स‚त्ताऽ‚{\tiny $_{lb}$}‚नित्य‚त्वं च दुर्निवारं । अभूत्वा भ‚व‚न्न‚हेतुको भ‚व‚तीत्य‚पि विरुद्धं} कादाचित्क‚स्या‚{\tiny $_{१}$}‚हेतु‚{\tiny $_{lb}$}‚त्व‚विरोधात् । \textbf{सोपि} दोषोप‚न्यासो\textbf{ऽनेनैव} विनाश‚स्य नीरूप‚त्व‚प्र‚तिपाद‚नेन‚{\tiny $_{lb}$}‚ \textbf{प्र‚त्याख्यातः} । किं कार‚णं [।] विनाश‚काले \textbf{क‚स्य‚चिद्} ध‚र्म‚स्य \textbf{भावान‚भ्युप‚ग‚मात्} ॥
	{\color{gray}{\rmlatinfont\textsuperscript{§~\theparCount}}}
	\pend% ending standard par
      ‚{\tiny $_{lb}$}‚

	  
	  \pstart \leavevmode% starting standard par
	\textbf{य‚था} तुल्ये व‚स्तुत्वे \textbf{केषांचिदेव ज‚न्मिना}मुत्प‚त्तिम‚तां \textbf{प्र‚तिघो} नाम स्व‚देशे‚{\tiny $_{lb}$}‚ प‚र‚स्योत्प‚त्तिविव‚न्ध‚ल‚क्ष‚ण \textbf{इष्टो} न स‚र्वेषां । \textbf{त‚था भावानामुत्प‚त्तिम‚ता}मेव \textbf{नाश‚{\tiny $_{lb}$}‚स्व‚भावो} भ‚व‚तु [।] न त्व‚नुत्प‚त्तिम‚{\tiny $_{२}$}‚तामाकाशादीनां । त‚था च य‚त् स‚त्त‚त्क्ष‚{\tiny $_{lb}$}‚णिकमित्येत‚द् व्य‚भिचारीति ।
	{\color{gray}{\rmlatinfont\textsuperscript{§~\theparCount}}}
	\pend% ending standard par
      ‚{\tiny $_{lb}$}‚

	  
	  \pstart \leavevmode% starting standard par
	\textbf{अथे}त्यादिना व्याच‚ष्टे । \textbf{भ‚व‚तु नाम । स्व‚भाव एष भावा}नां [।] कोसौ‚{\tiny $_{lb}$}‚ स्व‚भाव इत्याह । \textbf{य इमे क्ष‚ण‚स्थितिध‚र्माणः} । क्ष‚ण‚स्थितिर्ध‚र्मो येषामिति विग्र‚हः ।‚{\tiny $_{lb}$}‚ \textbf{स तु} क्ष‚ण‚स्थितिध‚र्म‚स्व‚भाव \textbf{उत्प‚त्तिम‚तामेव} भावानाम्\textbf{भ‚विष्य‚ति} । न त्व‚नुत्प‚त्ति‚{\tiny $_{lb}$}‚म‚तामाकाशादीनां ।
	{\color{gray}{\rmlatinfont\textsuperscript{§~\theparCount}}}
	\pend% ending standard par
      ‚{\tiny $_{lb}$}‚

	  
	  \pstart \leavevmode% starting standard par
	य‚स्मान्न \textbf{हि स्व‚भाव इत्}येव कृ‚{\tiny $_{३}$}‚त्वा \textbf{स‚र्वः स‚र्व‚स्य स्व‚भावो भ‚व‚ति । प्र‚तिघात्म‚{\tiny $_{lb}$}‚ताव‚त्} । य‚था प्र‚तिघात्म‚ता व‚स्तुस्व‚भाव‚त्वेपि न स‚र्व‚स्य भ‚व‚ति त‚द्व‚दित्य‚र्थः ।
	{\color{gray}{\rmlatinfont\textsuperscript{§~\theparCount}}}
	\pend% ending standard par
      ‚{\tiny $_{lb}$}‚

	  
	  \pstart \leavevmode% starting standard par
	\textbf{स‚त्त्य}मित्या चा र्यः । स‚र्वः स‚र्व‚स्य स्व‚भावो न भ‚व‚तीति स‚त्त्य‚मेत‚त् । \textbf{त‚थाप्य‚यं}‚{\tiny $_{lb}$}‚ स‚प्र‚तिघ‚स्य ज‚न‚कोऽयं नेति \textbf{स्व‚भाव‚निय‚माद्धेतोः स्व‚भाव‚निय‚मः फ‚ले}ऽयं प्र‚ति‚{\tiny $_{lb}$}‚घोऽयं नेति । \textbf{नानित्ये} तु नानित्य‚त्व‚वि‚{\tiny $_{४}$}‚ष‚ये कृत‚कानां \textbf{रूप‚भेदोस्ति} येन क‚स्य‚चिन्न‚{\tiny $_{lb}$}‚‚{\tiny $_{lb}$}‚ \leavevmode\ledsidenote{\textenglish{533/s}}श्व‚रः स्व‚भावः स्यान्नान्य‚स्य [।] किं कार‚ण‚म् [।] अनित्य‚स्व‚भाव‚स्य \textbf{भेद‚कानां}‚{\tiny $_{lb}$}‚ हेतूनाम\textbf{भाव‚तः} । स‚र्वेषाम्विन‚श्व‚र‚स्व‚भाव‚स्य ज‚न‚नादितियाव‚त् ।
	{\color{gray}{\rmlatinfont\textsuperscript{§~\theparCount}}}
	\pend% ending standard par
      ‚{\tiny $_{lb}$}‚

	  
	  \pstart \leavevmode% starting standard par
	न‚नु नाश‚स्व‚भावो भावानान्नानुत्प‚त्तिम‚तां य‚दीति चोद्यं प्र‚कृतं । न च त‚{\tiny $_{lb}$}‚त्रानित्ये रूप‚भेदोस्ति भेद‚कानाम‚भाव‚त इत्य‚य‚म्प‚रिहारो युक्तः । न ह्याकाशा‚{\tiny $_{५}$}‚दीनां‚{\tiny $_{lb}$}‚ स्व‚हेतुकृतो विन‚श्व‚र‚स्व‚भावोनुत्प‚त्तिम‚त्त्वात् । त‚त्क‚थं स‚त्त्य‚मित्याद्युक्त‚मिति ।
	{\color{gray}{\rmlatinfont\textsuperscript{§~\theparCount}}}
	\pend% ending standard par
      ‚{\tiny $_{lb}$}‚

	  
	  \pstart \leavevmode% starting standard par
	एव‚म्म‚न्य‚ते [।] य‚था स‚त्त्वं व्य‚भिचार्युक्त‚न्त‚था कृत‚कोपि क‚श्चिन्न‚श्व‚रः‚{\tiny $_{lb}$}‚ क‚श्चिन्नेत्याशंक‚ते । तेनादावेव कृत‚क‚त्व‚स्य व्य‚भिचार‚न्ताव‚त् प‚रिह‚र्त्तुं स‚त्त्य‚मि‚{\tiny $_{lb}$}‚त्याद्युक्त‚मित्य‚दोषः ।
	{\color{gray}{\rmlatinfont\textsuperscript{§~\theparCount}}}
	\pend% ending standard par
      ‚{\tiny $_{lb}$}‚

	  
	  \pstart \leavevmode% starting standard par
	\textbf{नेत्या}दिना व्याच‚ष्टे । \textbf{अन्य} इति स‚निद‚र्श‚नादिकः । \textbf{अक‚स्मा}दिति हेतुम‚न्त‚रेण‚{\tiny $_{lb}$}‚ \textbf{नि‚{\tiny $_{६}$}‚य‚म‚वान्} । क्व‚चित् स्यात् क्व‚चिन्नेति । \textbf{यादृशी तु} प्र‚तिनिय‚त‚विष‚या ।‚{\tiny $_{lb}$}‚ प्र‚तिघादिध‚र्म्म‚ज‚न‚क‚स्य \textbf{हेतोः श‚क्तिस्थितिः} । श‚क्तिनिय‚म‚स्\textbf{तादृशं} हेतुश‚क्त्य‚{\tiny $_{lb}$}‚नुरूपं \textbf{फ‚ल‚म्भ‚व‚तीति} कृत्वा \textbf{हेतुस्व‚भाव‚निय‚मात् फ‚ल‚स्व‚भाव‚निय‚म} इष्टः ।‚{\tiny $_{lb}$}‚ \textbf{आक‚स्मिल‚त्वे} तु निर्हेतुक‚त्\textbf{वेस्य} फ‚ल‚स्व‚भाव‚निय‚म‚स्योक्तो \textbf{दोषः} । देश‚काल‚प्र‚कृति‚{\tiny $_{lb}$}‚निव‚मो न युज्य‚त इति ।‚{\tiny $_{७}$}‚ त‚स्मात् \textbf{प्र‚तिघातात्म}ताया हेतुस्त‚स्य \textbf{स्व‚भाव}स्त‚स्य \leavevmode\ledsidenote{\textenglish{188b/PSVTa}}‚{\tiny $_{lb}$}‚ \textbf{प्र‚तिनिय‚म‚व‚त्} । न \textbf{न‚श्व‚र‚ज‚न‚न‚प्र‚तिनिय‚त‚स्व‚भावं} हेतुभूत‚म्भाव‚स्य \textbf{प‚श्यामः} ।‚{\tiny $_{lb}$}‚ क‚श्चिदेव न‚श्न‚रं ज‚न‚येन्न स‚र्व‚मित्येवं न‚श्व‚र‚ज‚न‚ने प्र‚तिनिय‚तः स्व‚भावो य‚स्य‚{\tiny $_{lb}$}‚ भाव‚स्येति विग्र‚हः । \textbf{येन} हेतुप्र‚तिनिय‚मेन । \textbf{त‚ज्ज‚न्मा} विन‚श्व‚र‚ज‚न‚नाद्धेतोर्ज‚न्म‚{\tiny $_{lb}$}‚ य‚स्य स त‚था स्यान्न‚श्व‚रः स्या\textbf{न्नान्यो} य‚{\tiny $_{१}$}‚स्त‚द्विल‚क्ष‚ण‚हेतुज‚न्मेति । किं कार‚णं [।]‚{\tiny $_{lb}$}‚ \textbf{स‚र्वाकार‚ज‚न्म‚नां नाश‚द‚र्श‚नात्} । स‚र्वाकारेभ्यो हेतुभ्यो ज‚न्म येषामिति विग्र‚हः ॥
	{\color{gray}{\rmlatinfont\textsuperscript{§~\theparCount}}}
	\pend% ending standard par
      ‚{\tiny $_{lb}$}‚

	  
	  \pstart \leavevmode% starting standard par
	\textbf{न‚न्}वित्यादि प‚रः । \textbf{स‚र्वाकारेभ्यो} हेतुभ्यो \textbf{ज‚न्म} येषाम्भावानान्ते \textbf{न‚श्य‚न्ती‚{\tiny $_{lb}$}‚तीद‚म‚प्य‚निश्चेय‚मेव} । किं कार‚णं [।] \textbf{तासां} हेतुसाम‚ग्रीणाम‚र्वाग्द‚र्श‚नै\textbf{र‚निःशेष‚{\tiny $_{lb}$}‚द‚र्श‚नात्} साक‚ल्येनाद‚र्श‚नात् । क‚स्याश्चित् साम‚ग्र्या न‚श्व‚र‚ज‚निकाया‚{\tiny $_{२}$}‚ द‚र्श‚नाद‚{\tiny $_{lb}$}‚दृष्टा अपि त‚ज्जातीय‚त‚या त‚थाभूता निश्चीय‚न्त इति चेद् [।]
	{\color{gray}{\rmlatinfont\textsuperscript{§~\theparCount}}}
	\pend% ending standard par
      ‚{\tiny $_{lb}$}‚‚{\tiny $_{lb}$}‚\textsuperscript{\textenglish{534/s}}

	  
	  \pstart \leavevmode% starting standard par
	आह \textbf{विचित्रे}त्यादि । \textbf{विचित्रा श‚क्ति}र्यासामिति विग्र‚हः । \textbf{साम‚ग्र्यो दृश्य‚न्ते ।‚{\tiny $_{lb}$}‚ त‚त्र} विचित्र‚श‚क्तिषु साम‚ग्रीषु म‚ध्ये \textbf{काचित्} साम‚ग्री \textbf{स्याद‚पि या} भाव\textbf{म‚न‚श्व‚रा‚{\tiny $_{lb}$}‚त्मानं ज‚न‚येत्} ।
	{\color{gray}{\rmlatinfont\textsuperscript{§~\theparCount}}}
	\pend% ending standard par
      ‚{\tiny $_{lb}$}‚

	  
	  \pstart \leavevmode% starting standard par
	\textbf{ने}त्यादिना प‚रिह‚र‚ति । अयं च प‚रिहारो नाश‚स्व‚भावो भावानां नानुत्प‚त्ति‚{\tiny $_{lb}$}‚म‚तां । य‚दीत्य‚त्रापि चो‚{\tiny $_{३}$}‚द्ये द्र‚ष्ट‚व्यः साधार‚ण‚त्वात् । नाय‚न्दोषः [।] क‚स्मात्‚{\tiny $_{lb}$}‚ [।] \textbf{ज्ञेयाधिकारात्} ।
	{\color{gray}{\rmlatinfont\textsuperscript{§~\theparCount}}}
	\pend% ending standard par
      ‚{\tiny $_{lb}$}‚

	  
	  \pstart \leavevmode% starting standard par
	एत‚देव स्प‚ष्ट‚य‚न्नाह । ये \textbf{क‚दाचित्} काले \textbf{क्व‚चिद्} देशे \textbf{केन‚चित्} पुरुषेणार्था‚{\tiny $_{lb}$}‚ \textbf{ज्ञाताः स‚न्तः} पुन‚र्न \textbf{ज्ञाय‚न्ते तेषां स‚न्तानानुब‚न्धी नाश इति ब्रूमः} । ये चाज्ञाताः स‚न्तो‚{\tiny $_{lb}$}‚ ज्ञाय‚न्ते ज्ञाता वा पुन‚र्न ज्ञाय‚न्ते [।] \textbf{त एव कृत‚का अनित्यास्साध्य‚न्ते} । अनेन च‚{\tiny $_{lb}$}‚ कृत‚क‚त्व‚स्य क्ष‚णिक‚त्वेन व्याप्तिः स‚त्त्वादि‚{\tiny $_{४}$}‚त्य‚नेन निश्चीय‚त इत्युक्त‚म्भ‚व‚ति ।
	{\color{gray}{\rmlatinfont\textsuperscript{§~\theparCount}}}
	\pend% ending standard par
      ‚{\tiny $_{lb}$}‚

	  
	  \pstart \leavevmode% starting standard par
	न‚नु च य‚द्य‚पि ते ज्ञातास्स‚न्तो न ज्ञाय‚न्ते त‚थापि क‚थ‚न्तेषाम‚नित्य‚त्व‚मिति [।]
	{\color{gray}{\rmlatinfont\textsuperscript{§~\theparCount}}}
	\pend% ending standard par
      ‚{\tiny $_{lb}$}‚

	  
	  \pstart \leavevmode% starting standard par
	अत आह । य‚स्मान्न \textbf{ह्य‚स्ति स‚म्भ‚वो य‚त् ते ज्ञान‚ज‚न‚न‚स्व‚भावाः} पूर्व\textbf{म्पुन‚र‚न‚ष्टा}‚{\tiny $_{lb}$}‚स्त‚स्मिन्नेव स्व‚भावे स्थिता \textbf{न ज‚न‚येयुः} ।
	{\color{gray}{\rmlatinfont\textsuperscript{§~\theparCount}}}
	\pend% ending standard par
      ‚{\tiny $_{lb}$}‚

	  
	  \pstart \leavevmode% starting standard par
	स‚ह‚कार्य‚भावान्न ज‚न‚य‚न्तीति चेद् [।]
	{\color{gray}{\rmlatinfont\textsuperscript{§~\theparCount}}}
	\pend% ending standard par
      ‚{\tiny $_{lb}$}‚

	  
	  \pstart \leavevmode% starting standard par
	आह । \textbf{अपेक्षेर‚न्नाप‚रं} । न ह्य‚स्य स‚म्भ‚वोस्तीति स‚म्ब‚न्धः । किं कार‚णं [।]‚{\tiny $_{lb}$}‚ \textbf{त‚ज्ज‚{\tiny $_{५}$}‚न‚न‚स्व‚भाव‚स्य निष्प‚न्न}त्वात् ।
	{\color{gray}{\rmlatinfont\textsuperscript{§~\theparCount}}}
	\pend% ending standard par
      ‚{\tiny $_{lb}$}‚

	  
	  \pstart \leavevmode% starting standard par
	अथ स्यात् [।] तेष्व‚न‚पेक्षेष्व‚पि क‚स्य‚चित् क‚दाचिज्ज्ञान‚म्भ‚विष्य‚तीति [।]
	{\color{gray}{\rmlatinfont\textsuperscript{§~\theparCount}}}
	\pend% ending standard par
      ‚{\tiny $_{lb}$}‚

	  
	  \pstart \leavevmode% starting standard par
	अत आह । \textbf{न च तेषु} नैव तेषु ज्ञान‚ज‚न‚न‚स्व‚भावेषु व्य‚व‚स्थितेषु स‚ह‚कार्य‚{\tiny $_{lb}$}‚न‚पेक्षेषु \textbf{क‚स्य‚चित्} पुंसः । \textbf{क‚दाचित्} काले किंचिज्\textbf{ज्ञानं निव‚र्त्तेत । स‚र्व‚स्य} स‚र्व‚दा‚{\tiny $_{lb}$}‚ स्व‚विष‚याणि ज्ञानानि ज‚न‚येयुः । न चैवं । क‚दाचित् ज्ञान‚द‚र्श‚नात् । त‚त‚श्च ज्ञान‚{\tiny $_{lb}$}‚म‚ज‚न‚य‚{\tiny $_{६}$}‚न्तो ज‚न‚क‚स्व‚भावात् प्र‚च्युता इति ग‚म्य‚ते ।
	{\color{gray}{\rmlatinfont\textsuperscript{§~\theparCount}}}
	\pend% ending standard par
      ‚{\tiny $_{lb}$}‚

	  
	  \pstart \leavevmode% starting standard par
	य‚त्त‚र्ह्य‚ज्ञेयं कृत‚क‚म‚कृत‚क‚म्वा त‚न्नैव‚म्भ‚विष्य‚तीति [।]
	{\color{gray}{\rmlatinfont\textsuperscript{§~\theparCount}}}
	\pend% ending standard par
      ‚{\tiny $_{lb}$}‚

	  
	  \pstart \leavevmode% starting standard par
	अत आह । \textbf{र्ने}त्यादि । न चैवंभूत‚म‚ज्ञेयं किंचिद‚स्ति । किं कार‚णं [।] \textbf{स‚र्व‚{\tiny $_{lb}$}‚स्यार्थ‚स्य केन‚चित्पु}रुषेण \textbf{क‚दाचिञ्ज्ञानात्} । अथ त‚ज्ज्ञान‚म‚पि न ज‚न‚येत् । त‚दा‚{\tiny $_{lb}$}‚ ‚{\tiny $_{lb}$}‚ \leavevmode\ledsidenote{\textenglish{535/s}}\textbf{ज्ञान‚मात्रार्थ‚क्रियायाम‚प्य‚साम‚र्थ्ये} त‚त्साम‚र्थ्य‚र‚हित‚म्\textbf{व‚स्त्वेव न स्यात् ।‚{\tiny $_{७}$}‚ त‚था हि \leavevmode\ledsidenote{\textenglish{189a/PSVTa}}‚{\tiny $_{lb}$}‚ त‚ल्ल‚क्ष‚ण}म‚र्थ‚क्रियासाम‚र्थ्य‚ल‚क्ष‚ण‚म्\textbf{व‚स्त्विति व‚क्ष्यामः । त‚स्य चार्थ}क्रियास‚म‚र्थ‚स्य‚{\tiny $_{lb}$}‚ व‚स्तुनः क्र‚मेणार्थ‚क्रियां कुर्व‚तो \textbf{विनाशाव्य‚भिचारा}द‚हेतुत्वाच्च विनाश‚स्य \textbf{स‚त्तानु‚{\tiny $_{lb}$}‚ब‚न्धी} विनाशः सिद्धः ॥ अत एवान‚न्त‚रोक्तात् स‚र्व‚भावानां क्ष‚णिक‚त्व‚साध‚नाच्छ‚{\tiny $_{lb}$}‚ब्दार्थ‚योस्स‚म्ब‚न्ध‚स्यापि निंत्य‚ता प्र‚त्याख्येया । दूष्या ।
	{\color{gray}{\rmlatinfont\textsuperscript{§~\theparCount}}}
	\pend% ending standard par
      ‚{\tiny $_{lb}$}‚

	  
	  \pstart \leavevmode% starting standard par
	नेत्यादिना व्याच‚ष्टे । अत एव य‚थोक्ताद् विनाश‚स्य व‚स्तुमात्रानु‚{\tiny $_{१}$}‚‚{\tiny $_{lb}$}‚ ब‚न्धात् । श‚ब्द‚व‚द् [।] य‚था श‚ब्द‚स्य नित्य‚ता प्र‚तिक्षिप्ता त‚द्व‚त् स‚म्ब‚न्ध‚स्यापि‚{\tiny $_{lb}$}‚ नित्य‚ता प्र‚त्याख्येया ।
	{\color{gray}{\rmlatinfont\textsuperscript{§~\theparCount}}}
	\pend% ending standard par
      ‚{\tiny $_{lb}$}‚

	  
	  \pstart \leavevmode% starting standard par
	एव‚न्ताव‚त् स‚म्ब‚न्धं व्य‚तिरिक्त‚म‚भ्युप‚ग‚म्य दोष उक्तोऽधुनाऽव्य‚तिरिक्त‚{\tiny $_{lb}$}‚ एव स‚म्ब‚न्धो न युज्य‚त इति [।]
	{\color{gray}{\rmlatinfont\textsuperscript{§~\theparCount}}}
	\pend% ending standard par
      ‚{\tiny $_{lb}$}‚

	  
	  \pstart \leavevmode% starting standard par
	आह । या च श‚ब्द‚श‚क्तिर्योग्य‚ताख्या योग्य‚तासंज्ञितार्थ‚प्र‚तिप‚त्त्याश्र‚यो जै‚{\tiny $_{lb}$}‚ मि नी यै र्व र्ण्ण्य‚ते । सा योग्य‚ताश‚ब्दार्थान्त‚र‚मेव न भ‚व‚ति । त‚था हि भावानां‚{\tiny $_{lb}$}‚स्व‚{\tiny $_{२}$}‚भावातिश‚य एव विव‚क्षितार्थ‚क्रियास‚म‚र्थो योग्य‚तेत्यावेदितं प्राक् । स‚म‚र्थं‚{\tiny $_{lb}$}‚ हि रूपं श‚ब्द‚स्य योग्य‚ते त्यादिना ।
	{\color{gray}{\rmlatinfont\textsuperscript{§~\theparCount}}}
	\pend% ending standard par
      ‚{\tiny $_{lb}$}‚

	  
	  \pstart \leavevmode% starting standard par
	तेन श‚ब्द‚स्यानित्य‚त्वे योग्य‚ताया अप्य‚नित्य‚त्व‚म‚व्य‚तिरेकादिति भावः ।
	{\color{gray}{\rmlatinfont\textsuperscript{§~\theparCount}}}
	\pend% ending standard par
      ‚{\tiny $_{lb}$}‚‚{\tiny $_{lb}$}‚\textsuperscript{\textenglish{536/s}}

	  
	  \pstart \leavevmode% starting standard par
	\textbf{अस्तु वा} श‚ब्दा\textbf{द‚र्थान्त‚रं} योग्य‚ता । \textbf{त‚थापि श‚ब्द‚श‚क्तिश्च दूषिता} वेदित‚व्या ।‚{\tiny $_{lb}$}‚ कैः [।] \textbf{स‚म्ब‚न्ध‚दोषैः प्रागुक्तैः} ।
	{\color{gray}{\rmlatinfont\textsuperscript{§~\theparCount}}}
	\pend% ending standard par
      ‚{\tiny $_{lb}$}‚

	  
	  \pstart \leavevmode% starting standard par
	\textbf{उक्तो हीत्या}दिना व्याच‚ष्टे । \textbf{स‚म्ब‚न्धः} स‚म्ब‚न्धिभ्यो\textbf{र्था‚{\tiny $_{३}$}‚न्त‚र}मित्येव\textbf{म्वादे‚{\tiny $_{lb}$}‚ऽनेक‚प्र‚कारो दोष उक्तः} ।
	{\color{gray}{\rmlatinfont\textsuperscript{§~\theparCount}}}
	\pend% ending standard par
      ‚{\tiny $_{lb}$}‚

	  
	  \pstart \leavevmode% starting standard par
	\hphantom{.}स‚म्ब‚न्धिनाम‚नित्य‚त्वान्न स‚म्ब‚न्धेस्ति नित्य‚ता इत्यादिना । तेनैव‚{\tiny $_{lb}$}‚ प्रागुक्तेन दोषेण श‚ब्द‚श‚क्तिर‚पि स‚म्ब‚न्ध‚रूपेण क‚ल्पिता दूषितेति कृत्वा \textbf{न पुनः}‚{\tiny $_{lb}$}‚ पृथ\textbf{गुच्य‚ते} दोषः ।
	{\color{gray}{\rmlatinfont\textsuperscript{§~\theparCount}}}
	\pend% ending standard par
      ‚{\tiny $_{lb}$}‚

	  
	  \pstart \leavevmode% starting standard par
	त‚देवं ना\textbf{पौरुषेयो} वेदः ।
	{\color{gray}{\rmlatinfont\textsuperscript{§~\theparCount}}}
	\pend% ending standard par
      ‚{\tiny $_{lb}$}‚

	  
	  \pstart \leavevmode% starting standard par
	भ‚व‚तु नामापौरुषेय‚स्त‚थापि न त‚स्य स‚त्यार्थ‚ता निश्चेतुं श‚क्या । य‚स्मा\textbf{द‚{\tiny $_{lb}$}‚पौरुषेय‚मित्येव} कृत्वा \textbf{न} वै‚{\tiny $_{४}$}‚दिक‚म्व‚च‚नं \textbf{य‚थार्थ‚ज्ञान‚साध‚नं} । अविप‚रीतार्थ‚ज्ञान‚हेतुः ।‚{\tiny $_{lb}$}‚ य‚स्मात् \textbf{पुरुषाग‚सा} पुरुष‚दोषेणा\textbf{दुष्टो व‚ह्न्या}दिना [।] आदिश‚ब्दाज्ज्योत्स्नादिः‚{\tiny $_{lb}$}‚ [।] \textbf{अन्य‚थापि दृष्टो} वित‚थ‚ज्ञान‚हेतुर्दृष्ट इत्य‚र्थः ।
	{\color{gray}{\rmlatinfont\textsuperscript{§~\theparCount}}}
	\pend% ending standard par
      ‚{\tiny $_{lb}$}‚

	  
	  \pstart \leavevmode% starting standard par
	\textbf{भ‚व‚त्वि}त्यादिना व्याच‚ष्टे । \textbf{भ‚व‚न्तु नामापौरुषेया वैदिकाः श‚ब्दास्त‚थापि‚{\tiny $_{lb}$}‚ स‚म्भाव्य‚मेवैषां} वैदिकानां श‚ब्दाना\textbf{म‚य‚थार्थ‚ज्ञान‚हेतुत्वं} । य‚तो \textbf{न हि पुरुष‚दोषो‚{\tiny $_{lb}$}‚प‚धानादेव} । पुरुष‚दोषे रागादिभिरुप‚धानात् । संस्कारादेव । \textbf{अर्थेषु} ज्ञाप्येषु‚{\tiny $_{lb}$}‚ ज्ञाप‚कानां श‚ब्दानां \textbf{ज्ञान‚विभ्र‚मो} ज्ञान‚विप‚र्यासः । प्र‚कृत्यापि मिथ्याज्ञान‚ज‚न‚न‚स्य‚{\tiny $_{lb}$}‚ स‚म्भाव्य‚त्वात् । य‚स्मात् \textbf{त‚द्र‚हितानाम‚पि} पुरुष‚दोषोप‚धान‚र‚हितानाम‚पि \textbf{प्र‚दी‚{\tiny $_{lb}$}‚पादीनाम्वित‚थार्थ‚ज्ञान‚ज‚न्नात्} । आदिश‚ब्दाज्ज्योत्स्नादीनां [।] कुत्र [।] \textbf{नीलो‚{\tiny $_{lb}$}‚त्प‚{\tiny $_{६}$}‚लादिषु} । त‚था हि रात्रौ प्र‚दीपो नीलोत्प‚ले र‚क्त‚प्र‚तिभास‚ज्ञान‚हेतुः । ज्योत्स्ना‚{\tiny $_{lb}$}‚ पीते व‚स्त्रे शुक्ल‚ज्ञान‚हेतुः । \textbf{त‚दि}ति त‚स्मा\textbf{दिमे} वैदिकाः \textbf{श‚ब्दाः} पुरुष\textbf{संस्कार‚निर‚{\tiny $_{lb}$}‚पेक्षाः} स्युरिति स‚म्ब‚न्धः \textbf{प्र‚कृत्या} च स्व‚भावेन \textbf{चार्थेषु प्र‚तीतिहेत‚वो} ज्ञान‚हेत‚वः स्युः ।‚{\tiny $_{lb}$}‚ \leavevmode\ledsidenote{\textenglish{189b/PSVTa}} किं कार‚णं [।] \textbf{स्व‚भाव‚विशेषात्} स्व‚रूप‚विशेषात् । किमिव [।] \textbf{व‚ह्न्यादिव‚त् ।‚{\tiny $_{७}$}‚‚{\tiny $_{lb}$}‚ ‚{\tiny $_{lb}$}‚ \leavevmode\ledsidenote{\textenglish{537/s}}वित‚थ‚व्य‚क्त‚य‚श्च} स्यु\textbf{र्निय‚मे}न । वित‚था व्य‚क्त‚योर्थ‚प्र‚तिप‚त्त‚यो येभ्यः श‚ब्देभ्य इति‚{\tiny $_{lb}$}‚ विग्र‚हः । वित‚थ‚व्य‚क्त‚य एव स‚दा वैदिकाः श‚ब्दा इत्य‚स्य निय‚म‚स्य कार‚णं नास्ति‚{\tiny $_{lb}$}‚ [।] त‚तो \textbf{निय‚म‚कार‚णाभावाद्} वित‚थ‚व्य‚क्त‚य एव वैदिकाः श‚ब्दा इत्येत‚त् क‚ल्प‚न‚{\tiny $_{lb}$}‚\textbf{म‚युक्त‚मिति चेत्} । त‚वापि मी मां स क स्यावित‚थ‚ज्ञान‚हेत‚वो वैदिकाः श‚ब्दा इत्य‚{\tiny $_{lb}$}‚स्मि\textbf{न्न‚वित‚थ‚व्य‚क्तिनिय‚मे किं कार‚णं} [।] नै‚{\tiny $_{१}$}‚व किञ्चित् । \textbf{त‚स्मात्} त्व‚न्म‚तेन‚{\tiny $_{lb}$}‚ \textbf{य‚थार्थ‚व्य‚क्तिनिय‚म‚त्वात् प्र‚कृत्या} स्व‚भावेना\textbf{य‚थार्थ‚व्य‚क्तिनिय‚मः किन्न क‚ल्प्य‚ते} ।
	{\color{gray}{\rmlatinfont\textsuperscript{§~\theparCount}}}
	\pend% ending standard par
      ‚{\tiny $_{lb}$}‚

	  
	  \pstart \leavevmode% starting standard par
	\textbf{अथ‚वा} य‚था व‚ह्न्याद‚यो घ‚टादिषु नीलोत्प‚लादिषु \textbf{चार्थेषु} य‚थायोगं स‚त्यास‚त्य‚{\tiny $_{lb}$}‚ज्ञान‚हेत‚व‚स्त‚था वैदिकानां श‚ब्दाना\textbf{म‚र्थेषु} मिथ्येत‚र‚ज्ञान‚हेतुत्वेनो\textbf{भ‚य‚ज्ञान‚हेतुत्वं‚{\tiny $_{lb}$}‚ स्यात्} । य‚स्मा\textbf{न्न ह्य‚पौरुषेया अपि व‚ह्न्याद‚यो} व‚न‚द‚व‚च‚न्द्रालोकादिरूपाः । \textbf{एक‚त्र‚{\tiny $_{२}$}‚}‚{\tiny $_{lb}$}‚ घ‚टादौ दिवा \textbf{य‚थार्थ‚ज्ञान‚हेत‚वोपि} स‚न्तः \textbf{स‚र्व‚त्र} नीलोत्प‚लादाव‚पि रात्रौ । \textbf{त‚था‚{\tiny $_{lb}$}‚ भ‚व‚न्ति} य‚थार्थ‚ज्ञान‚हेत‚वो भ‚व‚न्ति । \textbf{त‚था} वैदिकानां \textbf{श‚ब्दानाम‚प्य‚पौरुषेय‚त्वेप्युभ‚यं‚{\tiny $_{lb}$}‚ स्यात्} । य‚थार्थाय‚थार्थ‚ज्ञान‚हेतुत्वं स्यात् ।
	{\color{gray}{\rmlatinfont\textsuperscript{§~\theparCount}}}
	\pend% ending standard par
      ‚{\tiny $_{lb}$}‚

	  
	  \pstart \leavevmode% starting standard par
	\textbf{भ‚व‚त्वि}त्यादि प‚रः । \textbf{व‚ह्न्यादीनां कृत‚क‚त्वात्} कार‚णाद् \textbf{य‚थाप्र‚त्य‚यं} य‚स्य‚{\tiny $_{lb}$}‚ य‚द्भ्रान्तिकार‚णं । त‚द्व‚शा\textbf{द‚न्य‚त्र} नीलोत्प‚लादौ । \textbf{अन्य‚थात्वं} वित‚थ‚ज्ञान‚{\tiny $_{३}$}‚हेतुत्वं‚{\tiny $_{lb}$}‚ \textbf{न पुन‚र्नित्येषु श‚ब्देष्वेत‚त्} स‚ह‚कारिप्र‚त्य‚य‚ब‚लेन मिथ्याज्ञान‚हेतुत्व\textbf{म‚स्ति} । नित्यानां‚{\tiny $_{lb}$}‚ स‚ह‚कारिब‚लेनान्य‚था प्र‚वृत्त्य‚स‚म्भ‚वात् ।
	{\color{gray}{\rmlatinfont\textsuperscript{§~\theparCount}}}
	\pend% ending standard par
      ‚{\tiny $_{lb}$}‚

	  
	  \pstart \leavevmode% starting standard par
	\textbf{न‚न्वि}त्यादि सि द्धा न्त वा दी । \textbf{एवंविध} इति स‚ह‚कारिब‚लेनार्थेष्य‚न्य‚था‚{\tiny $_{lb}$}‚ प‚रिवृत्तिल‚क्ष‚णोस्त्\textbf{येव ध‚र्म्मः} [।] \textbf{त‚त्रापीति} वैदिकेष्व‚पि श‚ब्देषु । किं कार‚णं‚{\tiny $_{lb}$}‚ [।] \textbf{तेषाम‚पि} वैदिकानां \textbf{संकेत‚ब‚लाद‚न्य‚थावृत्तेः} पुरुषेच्छानुविधायि‚{\tiny $_{४}$}‚संकेत‚ब‚ले‚{\tiny $_{lb}$}‚नान्य‚था प्र‚तीतिज‚न‚नादित्य‚र्थः ।
	{\color{gray}{\rmlatinfont\textsuperscript{§~\theparCount}}}
	\pend% ending standard par
      ‚{\tiny $_{lb}$}‚

	  
	  \pstart \leavevmode% starting standard par
	अथ संकेत‚ब‚लान्न तेषाम‚र्थेषु प‚रावृत्तिरिष्य‚ते । किन्तु नित्य‚त्वात् स्व‚भाव‚त‚{\tiny $_{lb}$}‚ एव स्व‚विष‚य‚ज्ञान‚ज‚न‚न‚स्व‚भावा वैदिकाः श‚ब्दाः । त‚दा \textbf{कार्य}स्य स्व‚विष‚य‚ज्ञान‚स्य‚{\tiny $_{lb}$}‚ ‚{\tiny $_{lb}$}‚ \leavevmode\ledsidenote{\textenglish{538/s}}यो \textbf{ज‚न‚न‚स्व‚भा}व‚स्त‚त्र \textbf{स्थितौ चैषां} वैदिकानां श‚ब्दानां \textbf{स‚म‚यादेः} [।] आदिश‚ब्दा‚{\tiny $_{lb}$}‚द‚न्य‚स्यापि क‚र‚ण‚व्यापारादेर‚पेक्ष‚णीय‚स्या\textbf{भावात्} कार‚णात् । \textbf{त‚तो} वै‚{\tiny $_{५}$}‚दिकाच्छ‚{\tiny $_{lb}$}‚ब्दात्\textbf{प्र‚तीति}र्ज्ञान\textbf{म‚र्थेषु स‚र्व‚स्य} पुंसः \textbf{स‚र्व‚दा स्यात् । न चास्ति} स‚र्व‚स्य स‚र्व‚दार्थ‚प्र‚{\tiny $_{lb}$}‚तीतिः । \textbf{त‚स्मान्न} वैदिकाः \textbf{श‚ब्दा} अर्थ‚प्र‚तीतिप्र‚तिष्ठित‚स्\textbf{व‚भा}वाः किन्तु स‚म‚यादिक‚{\tiny $_{lb}$}‚म‚पेक्ष्य‚न्त एवेति । तेपि मिथ्याज्ञान‚स्य हेत‚व इति त‚द‚व‚स्थो दोषः ।
	{\color{gray}{\rmlatinfont\textsuperscript{§~\theparCount}}}
	\pend% ending standard par
      ‚{\tiny $_{lb}$}‚

	  
	  \pstart \leavevmode% starting standard par
	\textbf{अपि च [।] त‚स्मिन्} श‚ब्दे\textbf{ऽकृत‚के म‚ते} इष्टे स‚ति \textbf{न ज्ञान‚हेतुतैव स्यात्} ।‚{\tiny $_{lb}$}‚ य‚स्मान्न हि \textbf{नित्येभ्यो व‚स्तुसाम‚र्थ्यात्} स्व‚रू‚{\tiny $_{६}$}‚पोप‚धान‚साम‚र्थ्येन \textbf{ज‚न्मास्ति क‚स्य}‚{\tiny $_{lb}$}‚चित् । ज्ञान‚स्यान्य‚स्य वा [।]
	{\color{gray}{\rmlatinfont\textsuperscript{§~\theparCount}}}
	\pend% ending standard par
      ‚{\tiny $_{lb}$}‚

	  
	  \pstart \leavevmode% starting standard par
	\textbf{य‚द्य‚कृत‚कं} इत्यादिना व्याच‚ष्टे । \textbf{य‚द्य‚कृत‚क‚श्श‚ब्दो} वैदिक‚स्त‚तो\textbf{र्थेषु प्र‚तीतिरेव‚{\tiny $_{lb}$}‚ न स्यात्} । किं कार‚णं [।] \textbf{प्र‚तीतीत्या}दि । इत‚र‚द‚ज‚न्माज‚न्म च इत‚र‚च्चेति‚{\tiny $_{lb}$}‚ विग्र‚हः । त‚योः कालो \textbf{प्र‚तीतेर्ज‚न्मेत‚र‚काल‚योस्तुल्य‚रूप‚स्य} नित्य‚त्वादेक‚रूप‚स्य \textbf{प्र‚ती‚{\tiny $_{lb}$}‚\leavevmode\ledsidenote{\textenglish{190a/PSVTa}} तिर्ज‚न्म‚नि साम‚र्थ्य‚स‚म्भाव‚नाऽयोगात्} । प्र‚तीतेर‚{\tiny $_{७}$}‚ज‚न्म‚काले य‚त्त‚स्य ज‚न‚कं रूप‚न्त‚{\tiny $_{lb}$}‚स्मिन्नेव स्व‚भावे स्थित‚स्य ज‚न‚क‚त्व‚विरोधात् । किं कार‚ण‚म् [।] \textbf{एव‚म}नेन रूपे‚{\tiny $_{lb}$}‚\textbf{णायं} नित्याभिम‚तो \textbf{ज‚न‚को नैव}म‚नेन रूपेणाज‚न‚क \textbf{इत्ये}वं \textbf{विवेच‚नीय‚स्य} पृथ‚ग्‚{\tiny $_{lb}$}‚ व्य‚व‚स्थाप्य‚स्य \textbf{रूप‚भेद‚स्य} स्व‚भाव‚भेद‚स्या\textbf{भावात्} । नित्य‚स्य स‚र्व‚दैक‚रूप‚त्वात् ।
	{\color{gray}{\rmlatinfont\textsuperscript{§~\theparCount}}}
	\pend% ending standard par
      ‚{\tiny $_{lb}$}‚

	  
	  \pstart \leavevmode% starting standard par
	एक‚स्व‚भावोपि पूर्व‚म‚ज‚न‚कः प‚श्चाज्ज‚न‚को भ‚विष्य‚तीति चेद् [।]
	{\color{gray}{\rmlatinfont\textsuperscript{§~\theparCount}}}
	\pend% ending standard par
      ‚{\tiny $_{lb}$}‚

	  
	  \pstart \leavevmode% starting standard par
	आह \textbf{ने}त्यादि । \textbf{अस्य‚{\tiny $_{१}$}‚} नित्य‚स्या\textbf{ज‚न}को \textbf{यादृशः} स्व‚भाव‚स्\textbf{तादृश एव ज‚न‚को‚{\tiny $_{lb}$}‚न युक्तः} । एक‚रूप‚त्वात् । स‚ह‚कारिण‚म‚धिकं प्राप्य प‚श्चाज्ज‚न‚य‚तीत्यादि मिथ्या ।‚{\tiny $_{lb}$}‚ य‚तो\textbf{न्यापेक्षापि} स‚ह‚कार्य‚पेक्षापि नित्य‚स्य \textbf{निषिद्धैव} प्राक् ।
	{\color{gray}{\rmlatinfont\textsuperscript{§~\theparCount}}}
	\pend% ending standard par
      ‚{\tiny $_{lb}$}‚

	  
	  \pstart \leavevmode% starting standard par
	य‚त एव\textbf{न्त‚स्मान्नित्यानां} श‚ब्दानां \textbf{क्व‚चि}द‚र्थेषु पुरुषे \textbf{ज्ञान‚ज‚न‚न‚साम‚र्थ्यं} । किं‚{\tiny $_{lb}$}‚ कार‚णं [।] \textbf{क‚दाचि}ज्ज्ञान‚स्\textbf{याज‚न‚ने} स‚ति प‚श्चाद‚पि त‚त्स्व‚भाव‚त्वा\textbf{न्नित्य‚म‚ज‚न‚न‚{\tiny $_{lb}$}‚प्र‚स‚ङ्गात्} ।
	{\color{gray}{\rmlatinfont\textsuperscript{§~\theparCount}}}
	\pend% ending standard par
      ‚{\tiny $_{lb}$}‚

	  
	  \pstart \leavevmode% starting standard par
	अथ माभूदेष दोष इति नित्यं स्व‚कार्यं कुर्व‚न्त्येवेतीष्य‚ते ।
	{\color{gray}{\rmlatinfont\textsuperscript{§~\theparCount}}}
	\pend% ending standard par
      ‚{\tiny $_{lb}$}‚‚{\tiny $_{lb}$}‚\textsuperscript{\textenglish{539/s}}

	  
	  \pstart \leavevmode% starting standard par
	त‚द‚पि नास्ति । स्व‚विष‚य‚ज्ञान\textbf{कार्य}स्य \textbf{सात‚त्याद‚र्श‚नाच्च । न ते} श‚ब्दाः‚{\tiny $_{lb}$}‚ क‚थंचित् केन‚चित् प्र‚कारेण \textbf{क‚र्त्तार इत्}येत‚च्चो\textbf{क्तं प्राक्} ।
	{\color{gray}{\rmlatinfont\textsuperscript{§~\theparCount}}}
	\pend% ending standard par
      ‚{\tiny $_{lb}$}‚

	  
	  \pstart \leavevmode% starting standard par
	स्यादेत‚त् [।] नित्येभ्योप्याकाशादिभ्यो बुद्ध‚यो भ‚व‚न्त्येव क‚स्य‚चित् क‚दा‚{\tiny $_{lb}$}‚चिदिति [।]
	{\color{gray}{\rmlatinfont\textsuperscript{§~\theparCount}}}
	\pend% ending standard par
      ‚{\tiny $_{lb}$}‚

	  
	  \pstart \leavevmode% starting standard par
	अत आह । \textbf{या अप्येता नित्याभिम‚तेष्वाकाशादिषु प्र‚तिप‚त्त‚यो} बुद्ध‚यो भ‚वि‚{\tiny $_{lb}$}‚ष्य‚न्तीतीष्य‚ते । \textbf{ता अपि‚{\tiny $_{३}$}‚न त‚त्स्व‚भाव‚भाविन्यो} नाकाशादिस्व‚भावाय‚त्त‚ज‚न्मानः ।‚{\tiny $_{lb}$}‚ नित्यानां क्र‚म‚यौग‚प‚द्याभ्याम‚र्थ‚क्रियाविरोधात् ।
	{\color{gray}{\rmlatinfont\textsuperscript{§~\theparCount}}}
	\pend% ending standard par
      ‚{\tiny $_{lb}$}‚

	  
	  \pstart \leavevmode% starting standard par
	किम्पुन‚रुत्प‚त्तौ तासां निमित्त‚मिति [।]
	{\color{gray}{\rmlatinfont\textsuperscript{§~\theparCount}}}
	\pend% ending standard par
      ‚{\tiny $_{lb}$}‚

	  
	  \pstart \leavevmode% starting standard par
	आह । \textbf{न ही}त्यादि । अनादिः स‚मान‚जातीयो यो \textbf{विक‚ल्प}स्तेनाहिता \textbf{या‚{\tiny $_{lb}$}‚ वास‚ना} श‚क्तिस्त‚त \textbf{उद्भूता} उत्प‚न्ना । य‚थाग‚मं \textbf{स‚मारोपितो} य आकाशाद्याकार‚{\tiny $_{lb}$}‚स्त‚द्\textbf{गोच}रास्त‚त्प्र‚तिभासिन्य एव \textbf{केव‚लं} ग‚ताः । त‚त्र‚{\tiny $_{४}$}‚ बाह्य‚त्वेन क‚ल्पितेष्वाकाशादिषु‚{\tiny $_{lb}$}‚ \textbf{जाय‚न्ते । न} तु ता \textbf{बुद्ध‚योर्थ‚गोच‚रा} नाकाशादिस्व‚ल‚क्ष‚ण‚विष‚याः ।
	{\color{gray}{\rmlatinfont\textsuperscript{§~\theparCount}}}
	\pend% ending standard par
      ‚{\tiny $_{lb}$}‚

	  
	  \pstart \leavevmode% starting standard par
	\textbf{स्व‚ल‚क्ष‚णे}त्यादिना व्याच‚ष्टे । \textbf{स्व‚ल‚क्ष‚ण‚विष‚या हि बुद्धिर्निय‚मेन त}स्य स्व‚ल‚क्ष‚ण‚{\tiny $_{lb}$}‚स्य \textbf{योग्य‚ता} साम‚र्थ्य‚न्त‚स्यो\textbf{प‚स्थानं} स‚न्निधान‚न्त\textbf{द‚नुविधायिनी} त‚द्भाव एव भावि‚{\tiny $_{lb}$}‚\textbf{नीति} कृत्वा । \textbf{अस्या} बुद्धेर्य‚त् कार‚णं स्व‚ल‚क्ष‚णं योग्यं स‚म‚र्थ‚न्\textbf{त‚स्मिन् कार‚णे} योग्ये \textbf{स‚ति}‚{\tiny $_{lb}$}‚ सा‚{\tiny $_{५}$}‚ बुद्धिर्भ\textbf{व‚त्ये}व । \textbf{त‚दे}वं न्याये स्थिते \textbf{य‚दि नित्यानां प‚दार्थानां स्व‚ल‚क्ष‚णे क‚स्य‚चित्}‚{\tiny $_{lb}$}‚ पुंसो \textbf{ज्ञानं स्यात्} । त‚दा नित्यं कार‚ण‚स्य संन्निधात् \textbf{स‚र्व‚स्य} पुंसः \textbf{स‚र्व‚दा स्यात्} ।
	{\color{gray}{\rmlatinfont\textsuperscript{§~\theparCount}}}
	\pend% ending standard par
      ‚{\tiny $_{lb}$}‚

	  
	  \pstart \leavevmode% starting standard par
	नापि स‚ह‚कार्य‚पेक्ष‚या नित्यानां क‚स्य‚चित् क‚दाचिज्ज्ञान‚ज‚न‚न‚न्त‚था हि \textbf{कार्यो}‚{\tiny $_{lb}$}‚ ज‚न्यः स‚ह‚कारिभि\textbf{र्विशेषो} यासां व्य‚क्तीनान्ता \textbf{हि व्य‚क्त‚यः क‚थंचिद्दे}श‚कालाव‚स्था‚{\tiny $_{lb}$}‚निय‚मेन \textbf{क्व‚चित्} कार्ये \textbf{उप‚{\tiny $_{६}$}‚युज्य‚माना} हेतुत्वं प्र‚तिप‚द्य‚मानास्त‚स्य कार्य‚स्यो\textbf{प‚ज‚न‚ने‚{\tiny $_{lb}$}‚ ‚{\tiny $_{lb}$}‚ \leavevmode\ledsidenote{\textenglish{540/s}}यो}ग्यो योतिश‚य आत्म‚भूत‚स्त‚स्य \textbf{प्र‚ल‚म्भे हेतुम्व‚स्तुविशेषं} स‚ह‚कारिण‚मिति याव‚त् ।
	{\color{gray}{\rmlatinfont\textsuperscript{§~\theparCount}}}
	\pend% ending standard par
      ‚{\tiny $_{lb}$}‚

	  
	  \pstart \leavevmode% starting standard par
	\textbf{त‚थे}त्य‚नित्य‚त्व‚व‚त् । नि\textbf{त्यो भावो कार्य‚विशेषो}नाधेयातिश‚यः \textbf{केन‚चित्} पुंसा‚{\tiny $_{lb}$}‚ \textbf{गृह्य‚माण‚स्त‚त्कार‚णापेक्षः} ग्र‚ह‚ण‚स‚ह‚कारिकार‚णापेक्षो \textbf{य‚दि ग्र‚ह‚ण‚म‚स्य} पुंसो \textbf{ज‚न‚येत्} ।‚{\tiny $_{lb}$}‚ \leavevmode\ledsidenote{\textenglish{190b/PSVTa}} \textbf{युक्तं य‚त्ते‚{\tiny $_{७}$}‚नैव} पुंसा \textbf{गृह्येत} नान्येन स‚ह‚कारिप्र‚तिनिय‚मात् । \textbf{त‚च्च} स‚ह‚कार्य‚पेक्ष‚या‚{\tiny $_{lb}$}‚ ज‚न‚नं नित्य‚स्य \textbf{न स‚म्भ}व‚ति । किं कार‚णं [।] \textbf{स्थित‚स्व‚भाव‚त्वाद्} । नित्य‚स्य‚{\tiny $_{lb}$}‚ स‚ह‚कारिणा नाधेयातिश‚य‚त्वादिति याव‚त् ।
	{\color{gray}{\rmlatinfont\textsuperscript{§~\theparCount}}}
	\pend% ending standard par
      ‚{\tiny $_{lb}$}‚

	  
	  \pstart \leavevmode% starting standard par
	त‚त‚श्च नित्यं ज‚न‚न‚स्व‚भावे स्थित‚त्वात् \textbf{स‚र्वेण} पुंसा \textbf{स‚म‚मे}क‚कालं \textbf{गृह्ये}ताथ‚{\tiny $_{lb}$}‚ स‚र्वेणाज‚न‚क‚त्वान्न गृह्येत । त‚दा स एवास्य स्व‚भाव इति \textbf{न वा केन‚चित्} पुरुषेण‚{\tiny $_{lb}$}‚ क‚दाचिद् गृह्ये‚{\tiny $_{१}$}‚त । \textbf{इति} हेतोस्\textbf{स‚न्} विद्य‚मानो नित्यो भावो य‚दि \textbf{केन‚चित्} योगि‚{\tiny $_{lb}$}‚नापि \textbf{दृष्ट}स्त‚दा \textbf{न क‚श्चिन्नित्योऽतीन्द्रियः स्या}त्स‚र्वेषाम‚व‚श्यं केन‚चिद् द‚र्श‚नात् ।‚{\tiny $_{lb}$}‚ त‚था चासौ नित्यं स‚र्व‚पुरुषाणामिन्द्रिय‚ग्राह्य एव स्यात् स‚र्व‚पुरुष‚म्प्र‚ति ज्ञान‚ज‚न‚न‚{\tiny $_{lb}$}‚साम‚र्थ्याविशेषात् । \textbf{न चेदं} स‚र्व‚पुरुष‚ग्राह्य‚त्व‚न्नित्य‚स्या\textbf{स्ति । त‚स्माद‚र्थ‚साम‚र्थ्यान‚{\tiny $_{lb}$}‚पेक्षा} आकाशादिस्व‚ल‚क्ष‚ण‚साम‚र्थ्यान‚पे‚{\tiny $_{२}$}‚क्ष‚काः \textbf{स‚मारोपित‚गोच‚राः} । य‚थाग‚म‚म‚{\tiny $_{lb}$}‚ध्यारोपिताऽकाशाद्याकार‚प्र‚तिभासिन्य इत्य‚र्थः । \textbf{आन्त‚र‚मेवोपादान}कार‚ण‚मा\textbf{श्रित्य}‚{\tiny $_{lb}$}‚ कीदृशं \textbf{विक‚ल्प‚वास‚नाप्र‚बोधं} । आकाशादिविक‚ल्प‚नानादिता । या आहिता वास‚ना‚{\tiny $_{lb}$}‚ त‚स्याः प्र‚बोधः कार्योत्पाद‚नं प्र‚त्याभिमुख्यं । य‚त एवार्थ‚साम‚र्थ्यान‚पेक्षा अत एव‚{\tiny $_{lb}$}‚ \textbf{बाह्यार्थ‚शून्या भ्रान्त‚य एवाकाशादिषु स‚र्व‚स्य} पुंसो \textbf{भ‚{\tiny $_{३}$}‚व‚न्ति} ।
	{\color{gray}{\rmlatinfont\textsuperscript{§~\theparCount}}}
	\pend% ending standard par
      ‚{\tiny $_{lb}$}‚

	  
	  \pstart \leavevmode% starting standard par
	स्थित‚मेत‚त् [।] नास्ति नित्येभ्यः कार्योत्पाद इति । श‚ब्दात्तु दृश्य‚ते क‚दा‚{\tiny $_{lb}$}‚चिज्ज्ञान‚कार्य\textbf{न्त‚स्मान्नाप‚रावृत्तिध‚र्माणः श‚ब्दाः} । एक‚रूप‚तायां अप‚रावृत्तिरेव ध‚र्मो‚{\tiny $_{lb}$}‚ येषामिति विग्र‚हः । किन्तु ज्ञानं ज्ञानं प्र‚त्य‚र्था भिन्न‚वृत्त‚य एव ।
	{\color{gray}{\rmlatinfont\textsuperscript{§~\theparCount}}}
	\pend% ending standard par
      ‚{\tiny $_{lb}$}‚

	  
	  \pstart \leavevmode% starting standard par
	अथ नित्य‚मेक‚रूपा एव त‚दा \textbf{त‚त्त्वे वा} । एक रूप‚त्वे वाऽभ्युप‚ग‚म्य‚मानेऽवि‚{\tiny $_{lb}$}‚त‚थार्थ‚प्र‚तीत‚य एव वैदिकाः श‚ब्दा इति \textbf{कुत एत‚त् । अ‚{\tiny $_{४}$}‚वित‚था अर्थ‚प्र‚ती}तिर्येभ्य‚{\tiny $_{lb}$}‚ \textbf{इति विग्र‚हः} ।
	{\color{gray}{\rmlatinfont\textsuperscript{§~\theparCount}}}
	\pend% ending standard par
      ‚{\tiny $_{lb}$}‚‚{\tiny $_{lb}$}‚\textsuperscript{\textenglish{541/s}}

	  
	  \pstart \leavevmode% starting standard par
	स्यान्म‚त‚म् [।] अग्निर्हिम‚स्य भेष‚जमित्यादिवैदिक‚वाक्य‚स्यावित‚थ‚त्वात्‚{\tiny $_{lb}$}‚ स‚र्व‚स्यावित‚थ‚त्व‚मिति [।]
	{\color{gray}{\rmlatinfont\textsuperscript{§~\theparCount}}}
	\pend% ending standard par
      ‚{\tiny $_{lb}$}‚

	  
	  \pstart \leavevmode% starting standard par
	अत आह । \textbf{न ही}त्यादि । \textbf{न ह्य‚ग्निर्हिम‚स्य भेष‚जं} प्र‚तिप‚क्ष \textbf{इ}त्येव‚मा\textbf{दिषु}‚{\tiny $_{lb}$}‚ वेद‚वाक्येष्व‚ग्नेः \textbf{शीत‚प्र‚तिघात‚साम‚र्थ्य}म्वेद‚वाक्यात् । प्राग‚पि य‚थासंकेतं \textbf{लोक‚{\tiny $_{lb}$}‚प्र‚सिद्धं ख्याप्य‚त इति} कृत्वा \textbf{स‚र्व}म‚दृष्टार्थ‚म‚पि वेद‚वा‚{\tiny $_{५}$}‚क्य\textbf{न्त‚था भ‚व‚ति} । अवित‚{\tiny $_{lb}$}‚थ‚म्भ‚व‚ति । लोक‚प्र‚सिद्धे ह्य‚र्थे लोक‚स्य संकेतानुसारेण व्य‚व‚हारो दृष्टः । त‚तो‚{\tiny $_{lb}$}‚\textbf{लोक‚स्य स्वेच्छाकृतो} यः \textbf{संकेत}स्ते\textbf{नानु} प‚श्चाद् व्य‚व‚हार‚काले \textbf{व्य‚व‚हारात्} । स‚न्देह‚{\tiny $_{lb}$}‚ एव कि\textbf{म‚यं लोकः स्व‚संकेत‚म‚नुविद‚ध‚त्} । अनुस‚र‚न् ।
	{\color{gray}{\rmlatinfont\textsuperscript{§~\theparCount}}}
	\pend% ending standard par
      ‚{\tiny $_{lb}$}‚

	  
	  \pstart \leavevmode% starting standard par
	अग्निर्हिम‚स्य भेष‚ज‚मित्यादिवाक्या\textbf{देव‚म्प्र‚त्ये}त्य‚ग्नेः शीताप‚नोद‚साम‚र्थ्य‚मिति‚{\tiny $_{lb}$}‚ निश्चिनोत्या\textbf{होस्विच्छ‚ब्द‚{\tiny $_{६}$}‚स्व‚भाव‚स्थितेः} श‚ब्द‚स्य स्व‚भावेन प्र‚कृत्या साम‚र्थ्य‚निय‚मा‚{\tiny $_{lb}$}‚देवं प्र‚त्येत्\textbf{ईति} । य‚दा च दृष्ट एवार्थे वैदिक‚स्य श‚ब्द‚स्य स्व‚तोर्थ‚प्र‚तिपाद‚न‚श‚क्ति‚{\tiny $_{lb}$}‚स्स‚न्दिग्धा त‚दात्य‚न्त‚प‚रोक्षेप्य‚र्थे नित‚रां स‚म्भाव्य‚त इति भावः ।
	{\color{gray}{\rmlatinfont\textsuperscript{§~\theparCount}}}
	\pend% ending standard par
      ‚{\tiny $_{lb}$}‚

	  
	  \pstart \leavevmode% starting standard par
	न‚न्व‚नादिलोक‚प्र‚सिद्ध्य‚नुविधानेनैव वैदिकानां श‚ब्दानाम‚र्थ‚व‚त्ता न च त‚त्र‚{\tiny $_{lb}$}‚ स‚न्देहः प्र‚तिभास‚त इति [।]
	{\color{gray}{\rmlatinfont\textsuperscript{§~\theparCount}}}
	\pend% ending standard par
      ‚{\tiny $_{lb}$}‚

	  
	  \pstart \leavevmode% starting standard par
	अत आह । \textbf{लोकेच्छ‚या प‚राव‚र्त्त्य‚मा‚{\tiny $_{७}$}‚ना} य‚थास‚म‚य‚म‚र्थेषु निवेश्य‚मानाः पुन- \leavevmode\ledsidenote{\textenglish{191a/PSVTa}}‚{\tiny $_{lb}$}‚ र‚न्य‚त्र देशादिप‚रावृत्ताव\textbf{न्य‚थे}त्य‚र्थान्त‚र‚निवेशेन प‚राव‚र्त्य‚मानाः श‚ब्दा \textbf{दृश्य‚न्ते} ।‚{\tiny $_{lb}$}‚ इति हेतो\textbf{र्लोक‚प्र‚सिद्ध्यानुविधाने}प्य‚ङ्गीक्रिय‚माणे \textbf{स‚म्भ‚व‚त्येवैषां} वैदिकानां श‚ब्दा‚{\tiny $_{lb}$}‚नाम\textbf{न्य‚थाभावो} मिथ्यात्वं । प्र‚सिद्धेरेवानिय‚त‚त्वात् ।
	{\color{gray}{\rmlatinfont\textsuperscript{§~\theparCount}}}
	\pend% ending standard par
      ‚{\tiny $_{lb}$}‚

	  
	  \pstart \leavevmode% starting standard par
	\textbf{त‚स्मात् क‚स्य‚चिद्} वैदिक‚स्य वाक्य‚स्या ग्निर्हिम‚स्य‚भेष‚जमित्यादिक‚स्या\textbf{वै‚{\tiny $_{lb}$}‚प‚रीत्य‚द‚र्श‚ने‚{\tiny $_{१}$}‚पि स‚र्वेषां} वेद‚वाक्यानान्त\textbf{थाभाव}स्स‚त्यार्थ‚त्व\textbf{न्न सिध्य‚ति} ।
	{\color{gray}{\rmlatinfont\textsuperscript{§~\theparCount}}}
	\pend% ending standard par
      ‚{\tiny $_{lb}$}‚

	  
	  \pstart \leavevmode% starting standard par
	अकृत‚क‚त्वादेव स‚त्यार्थ‚त्व‚मिति चेद् [।]
	{\color{gray}{\rmlatinfont\textsuperscript{§~\theparCount}}}
	\pend% ending standard par
      ‚{\tiny $_{lb}$}‚‚{\tiny $_{lb}$}‚\textsuperscript{\textenglish{542/s}}

	  
	  \pstart \leavevmode% starting standard par
	आह । \textbf{अकृत‚क‚स्व‚भाव‚त्वेह्येषां} वैदिकानां श‚ब्दानां \textbf{मिथ्यार्थ‚निय‚तो}पि \textbf{क‚श्चि‚{\tiny $_{lb}$}‚च्छ}ब्दः स्यात् । इतिहेतोः स्व‚भाव‚प‚रिज्ञानाद‚यं स‚त्यार्थोयं मिथ्यार्थ इत्येवं विवेकेन‚{\tiny $_{lb}$}‚ श‚ब्द\textbf{स्व‚भावानिश्च‚यात् स‚र्व‚त्र} श‚ब्दे \textbf{संश‚यः स्यात्} । स‚त्यार्थ‚म्वैदिक‚म्वाक्य‚म‚कृत‚क‚{\tiny $_{२}$}‚‚{\tiny $_{lb}$}‚त्वादिति प्र‚योगे क्रिय‚माणेन्व‚याभावात् ।
	{\color{gray}{\rmlatinfont\textsuperscript{§~\theparCount}}}
	\pend% ending standard par
      ‚{\tiny $_{lb}$}‚

	  
	  \pstart \leavevmode% starting standard par
	व्य‚तिरेकिप्र‚योग‚माह । \textbf{मिथ्यात्व}मित्यादिना । \textbf{मिथ्यात्वं कृत‚केष्वेव दृष्ट‚मिति}‚{\tiny $_{lb}$}‚ हेतोर\textbf{कृत‚क‚म्व‚चः स‚त्यार्थं य‚दी}ति स‚म्ब‚न्धः । किं कार‚णं [।] \textbf{व्य‚तिरेकेस्य‚{\tiny $_{lb}$}‚ विरोधिव्याप‚नात्} । अकृत‚क‚स्य हेतोर्यो व्य‚तिरेकः कृत‚क‚त्व‚न्तेन स‚त्यार्थ‚त्वं य‚त्सा‚{\tiny $_{lb}$}‚ध्य‚न्त‚स्य विरोधिमिथ्यात्व‚न्त‚स्य व्याप‚नात् । व्य‚तिरेक‚स्येति क‚र्त्त‚{\tiny $_{३}$}‚रि ष‚ष्ठी ।‚{\tiny $_{lb}$}‚ हेतुव्य‚तिरेकेण कृत‚क‚त्वेन स‚त्यार्थ‚विरोधिनो मिथ्यार्थ‚त्व‚स्य व्याप्त‚त्वात् । वैदिके‚{\tiny $_{lb}$}‚ श‚ब्देऽकृत‚क‚त्वात् कृत‚क‚त्व‚निवृत्तौ मिथ्यार्थ‚त्व‚निवृत्तेः स‚त्यार्थ‚त्वं सिध्य‚त्येव ।
	{\color{gray}{\rmlatinfont\textsuperscript{§~\theparCount}}}
	\pend% ending standard par
      ‚{\tiny $_{lb}$}‚

	  
	  \pstart \leavevmode% starting standard par
	\textbf{य‚थे}त्यादिना व्याच‚ष्टे । \textbf{य}त्किञ्चि\textbf{न्मिथ्यार्थ‚म्व‚चः त‚द‚खिल‚न्निः}शेषं \textbf{कृत‚{\tiny $_{lb}$}‚क‚मिति} कृत्वा । \textbf{हेतो}र‚कृत‚क‚त्व‚स्य \textbf{व्य‚तिरेकेण} कृत‚क‚त्वेन \textbf{साध्य‚व्य‚तिरेक‚स्य} ।‚{\tiny $_{lb}$}‚ साध्यं स‚{\tiny $_{४}$}‚त्यार्थ‚त्वं त‚स्य व्य‚तिरेको मिथ्यात्व‚न्त‚स्य \textbf{व्याप्तेर‚न्य‚त्र्}आकृत‚के मिथ्यार्थ‚{\tiny $_{lb}$}‚त्व‚स्या\textbf{स‚म्भ‚वात्} कार‚णा\textbf{द‚कृत‚कं स‚त्यार्थ‚मिति स्याद् विनाप्य‚न्व‚येन} । स‚त्यार्थ‚{\tiny $_{lb}$}‚म्वैदिक‚म्व‚चो कृत‚क‚त्वादित्य‚त्र प्र‚योगे य‚द्य‚प्य‚न्व‚यो नास्ति । त‚थाप्य‚न्व‚येन विना‚{\tiny $_{lb}$}‚ सिध्य‚त्येवेत्य‚र्थः । य‚स्माद् \textbf{यो ह्य}र्थो मिथ्यात्व‚ल‚क्ष‚णो \textbf{येन} कृत‚क‚त्वेना\textbf{व्याप्त‚स्त‚त्र}‚{\tiny $_{lb}$}‚ मिथ्यात्वे \textbf{त‚द्व्य‚तिरे}‚{\tiny $_{५}$}‚क‚स्त‚स्याव्याप‚क‚स्य कृत‚क‚त्व‚स्य व्य‚तिरेको कृत‚क‚त्व‚{\tiny $_{lb}$}‚ल‚क्ष‚णो ध‚र्म \textbf{आशंक्ये}ताय‚म‚पि मिथ्यात्वे भ‚वेदिति । त‚च्चेह नास्ति कृत‚क‚त्वेन‚{\tiny $_{lb}$}‚ मिथ्यार्थ‚ताया व्याप्तेः । \textbf{न च} विरुद्धेन व्याप्ते विरुद्ध‚स्य स‚म्भ‚वो य‚तो न च \textbf{विरु‚{\tiny $_{lb}$}‚द्ध‚योः} कृत‚क‚त्वाकृत‚क‚त्व‚यो\textbf{रेक‚त्र} मिथ्यात्वे \textbf{स‚म्भ‚वोस्ति} । तेनाकृत‚के स‚त्या‚{\tiny $_{lb}$}‚र्थ‚त्वं \textbf{विजातीय‚स्य} मिथ्यार्थ‚त्व‚स्या\textbf{स‚म्भ‚वे} । स‚त्यार्थ‚मिथ्या‚{\tiny $_{६}$}‚र्थ‚त्वाभ्यां नान्या ग‚ति‚{\tiny $_{lb}$}‚र‚स्तीत \textbf{ग‚त्य‚न्त‚राभावाद}कृत‚क‚त्वेन स‚त्यार्थ एव भ‚वित‚व्य‚मित्य‚कृत‚क‚त्वा\textbf{दिष्टा‚{\tiny $_{lb}$}‚‚{\tiny $_{lb}$}‚ \leavevmode\ledsidenote{\textenglish{543/s}}र्थ}स्य स‚त्यार्थ‚त्व‚ल‚क्ष‚ण‚स्य \textbf{सिद्धेः} किम‚न्व‚येन । \textbf{त‚त्साध‚न‚त्वाच्च लिङ्ग‚स्य} इष्टार्थ‚{\tiny $_{lb}$}‚साध‚न‚त्वाच्च लिङ्ग‚स्य \textbf{व्य‚र्थ‚म‚न्व‚य‚द‚र्श‚नं} । क‚स्माद् [।] य‚थोक्त‚विधिना \textbf{व्य‚ति‚{\tiny $_{lb}$}‚रेक‚मात्रेणैव} साध्य\textbf{सिद्धेरिति} ।‚{\tiny $_{७}$}‚ \leavevmode\ledsidenote{\textenglish{191b/PSVTa}}
	{\color{gray}{\rmlatinfont\textsuperscript{§~\theparCount}}}
	\pend% ending standard par
      ‚{\tiny $_{lb}$}‚

	  
	  \pstart \leavevmode% starting standard par
	\textbf{स‚त्त्य‚मेत}दित्या चा र्यः । विप‚क्षाद्धेतोर्व्य‚तिरेके सिद्धे स‚ति साध्यं सिध्येदिति‚{\tiny $_{lb}$}‚ \textbf{स}त्त्य‚मेत‚त् । \textbf{य‚दि} स‚त्त्यार्थ‚ताऽकृत‚क‚त्व\textbf{विप‚क्ष‚यो}र्मिथ्यात्व‚कृत‚क‚त्व‚यो\textbf{र्व्याप्य‚व्याप‚क‚{\tiny $_{lb}$}‚भावः सिध्येत्} । त‚दा कृत‚क‚त्वेन व्याप्तान्मिथ्यार्थ‚त्वाद‚कृत‚क‚न्निव‚र्त्त‚ते । \textbf{स तु}‚{\tiny $_{lb}$}‚ व्याप्य‚व्याप‚क‚भावो विप‚क्ष‚योर्न \textbf{सिद्धः} । किं कार‚णं ।‚{\tiny $_{१}$}‚ \textbf{य‚स्मात्} मिथ्यात्वेऽकृत‚{\tiny $_{lb}$}‚क‚त्व‚स्या\textbf{स‚म्भ‚वे}ऽस‚म्भ‚व‚निमित्तं बाध‚के \textbf{हेताव‚नुक्ते} स‚ति । \textbf{भाव}स्स‚त्त्व\textbf{न्त‚स्याप्य}कृत‚{\tiny $_{lb}$}‚क‚त्व‚स्य मिथ्यात्वे \textbf{श‚क्य‚ते} । अकृत‚कं च स्यान्मिथ्यार्थं चेति । \href{http://sarit.indology.info/?cref=pv.3.287}{२९० ab}
	{\color{gray}{\rmlatinfont\textsuperscript{§~\theparCount}}}
	\pend% ending standard par
      ‚{\tiny $_{lb}$}‚

	  
	  \pstart \leavevmode% starting standard par
	न‚नु च मिथ्यात्वे कृत‚क‚त्वं दृष्टं [।] य‚त्र च कृत‚क‚त्व‚न्त‚त्र क‚थ‚म‚कृत‚क‚त्व‚{\tiny $_{lb}$}‚मिति [।]
	{\color{gray}{\rmlatinfont\textsuperscript{§~\theparCount}}}
	\pend% ending standard par
      ‚{\tiny $_{lb}$}‚

	  
	  \pstart \leavevmode% starting standard par
	अत आह । \textbf{विरुद्धाना}मित्यादि । \textbf{विरुद्धानाम‚पि प‚दार्थाना}मेक‚व्\textbf{याप‚क‚द‚र्श‚नात्} ।‚{\tiny $_{lb}$}‚ य‚था प्र‚य‚त्ना‚{\tiny $_{२}$}‚प्र‚यंत्न‚नान्त‚रीय‚काणाम‚नित्यानामेकेन कृत‚क‚त्वेन व्याप्तिः ।
	{\color{gray}{\rmlatinfont\textsuperscript{§~\theparCount}}}
	\pend% ending standard par
      ‚{\tiny $_{lb}$}‚

	  
	  \pstart \leavevmode% starting standard par
	\textbf{य‚दी}त्यादिना व्याच‚ष्टे । \textbf{य‚द्य}कृत‚क‚त्वाख्य‚स्य \textbf{हेतोस्साध्य‚विप‚क्षे} मिथ्यार्थ‚{\tiny $_{lb}$}‚त्वे\textbf{ऽभावः सिध्ये}त्त‚दा \textbf{साध्य‚स्य} स‚त्यार्थ‚त्व‚स्य \textbf{व्य‚ति}रेकं मिथ्यात्वं । \textbf{हेतुव्य‚तिरेकः}‚{\tiny $_{lb}$}‚ हेतोर‚कृत‚क‚त्व‚स्य व्य‚तिरेकः कृत‚क‚त्वाख्यो \textbf{व्याप्नुयात् । न च त‚स्या}कृत‚क‚त्व‚स्य \textbf{त‚त्र}‚{\tiny $_{lb}$}‚ मिथ्यार्थ‚तायाम\textbf{स‚म्भ‚वे} बाध‚कं \textbf{प्र‚मा‚{\tiny $_{३}$}‚णं प‚श्यामः} ।
	{\color{gray}{\rmlatinfont\textsuperscript{§~\theparCount}}}
	\pend% ending standard par
      ‚{\tiny $_{lb}$}‚

	  
	  \pstart \leavevmode% starting standard par
	अकृत‚क‚त्वादेव वेद‚स्य मिथ्यार्थ‚तायाम‚वृत्तिरिति चेद् [।]
	{\color{gray}{\rmlatinfont\textsuperscript{§~\theparCount}}}
	\pend% ending standard par
      ‚{\tiny $_{lb}$}‚

	  
	  \pstart \leavevmode% starting standard par
	आह । \textbf{न चे}त्यादि । मिथ्यार्थ‚त्वेन्\textbf{आविरुद्ध}स्याकृत‚त्व‚स्य \textbf{विधि}र्मिथ्यार्थ‚ता‚{\tiny $_{lb}$}‚\textbf{प्र‚तिषेध}स्य \textbf{साध‚नो युक्तः} । क‚स्माद् [।] \textbf{अतिप्र‚स‚ङ्गात्} । एवं हि य‚स्य क‚स्य‚चिद्‚{\tiny $_{lb}$}‚ विधानेन य‚स्य क‚स्य‚चिद‚भावः प्र‚तीयेत ।
	{\color{gray}{\rmlatinfont\textsuperscript{§~\theparCount}}}
	\pend% ending standard par
      ‚{\tiny $_{lb}$}‚

	  
	  \pstart \leavevmode% starting standard par
	कृत‚के दृष्ट‚स्य मिथ्यात्व‚स्याकृत‚के क‚थं वृत्तिरिति चेद् [।]
	{\color{gray}{\rmlatinfont\textsuperscript{§~\theparCount}}}
	\pend% ending standard par
      ‚{\tiny $_{lb}$}‚

	  
	  \pstart \leavevmode% starting standard par
	आह । \textbf{न चैक‚त्र} कृत‚के \textbf{दृष्ट‚स्य} मिथ्यात्व‚स्य पु‚{\tiny $_{४}$}‚न‚र\textbf{न्य}त्राकृत‚केऽ\textbf{स‚म्भ‚व} एव‚{\tiny $_{lb}$}‚ [।] किन्तु स‚म्भ‚व एव । किं कार‚ण‚म् [।] \textbf{पृथ‚ग्विरुद्ध‚स‚ह‚भाविना}म्पृथ‚गिति‚{\tiny $_{lb}$}‚ ‚{\tiny $_{lb}$}‚ \leavevmode\ledsidenote{\textenglish{544/s}}व्य‚क्तिभेदेन विरुद्ध‚स‚ह‚भाविनाम्विरुद्धैर‚र्थैरेक‚त्र भाविना\textbf{म‚पि द‚र्श‚नात् । अनित्य‚त्व‚व‚त्‚{\tiny $_{lb}$}‚ प्र‚य‚त्नान‚न्त‚रीय‚केत‚र‚योरि}त‚र‚द‚प्र‚य‚त्नान‚न्त‚रीय‚कं । अनित्य‚त्वं प्र‚य‚त्नान‚न्त‚रीय‚क‚त्वेन‚{\tiny $_{lb}$}‚ स‚ह दृष्ट‚म‚प्र‚य‚त्नान‚रीय‚क‚त्वेन स‚ह दृश्य‚ते ।
	{\color{gray}{\rmlatinfont\textsuperscript{§~\theparCount}}}
	\pend% ending standard par
      ‚{\tiny $_{lb}$}‚

	  
	  \pstart \leavevmode% starting standard par
	य‚{\tiny $_{५}$}‚द्येव‚म‚कृत‚के मिथ्यात्व‚स्याद‚र्श‚नाद‚भावः स्यादिति [।]
	{\color{gray}{\rmlatinfont\textsuperscript{§~\theparCount}}}
	\pend% ending standard par
      ‚{\tiny $_{lb}$}‚

	  
	  \pstart \leavevmode% starting standard par
	अत आह । \textbf{न च त‚थाविध‚स्}येत्य‚कृत‚क‚स्य स‚तो मिथ्यात्व‚स्या\textbf{द‚र्श‚नाद‚स‚त्व‚मेव} ।
	{\color{gray}{\rmlatinfont\textsuperscript{§~\theparCount}}}
	\pend% ending standard par
      ‚{\tiny $_{lb}$}‚

	  
	  \pstart \leavevmode% starting standard par
	\textbf{य‚स्मान्न} विप‚क्षे हेतो\textbf{र‚स‚त्तासिद्धिः स‚र्व‚तोनुप‚ल‚म्भ‚नादित्युक्तं} प्राक् । त‚त‚श्चा‚{\tiny $_{lb}$}‚\textbf{सिद्धायां} विप‚क्षाद्धेतोर\textbf{स‚त्तायां स‚न्दिग्ध}विप‚क्षाद् \textbf{व्य‚तिरेकिता} ।
	{\color{gray}{\rmlatinfont\textsuperscript{§~\theparCount}}}
	\pend% ending standard par
      ‚{\tiny $_{lb}$}‚

	  
	  \pstart \leavevmode% starting standard par
	\textbf{न ही}त्यादिना व्याच‚ष्टे । \textbf{अय‚म्पुरुष‚मात्र‚क} इत्य‚र्वा‚{\tiny $_{६}$}‚ग्द‚र्शी \textbf{स‚र्व}म्व‚स्तु \textbf{द्र‚ष्टुं‚{\tiny $_{lb}$}‚ स‚म‚र्थो येनास्य} पुंसो \textbf{द‚र्श‚न‚निवृत्त्या न त‚था स्यात्} । अदृष्टो न स्यात् । य‚स्माद् [।]‚{\tiny $_{lb}$}‚ \textbf{य‚स्य} हि पुंसो \textbf{ज्ञानं ज्ञेय‚स‚त्तां न व्य‚भिच‚र‚ति} । स‚दित्येव कृत्वा य‚स्य स‚र्व‚स्मिन् \textbf{ज्ञ}‚{\tiny $_{lb}$}‚\edtext{}{\lemma{स्मिन्}\Bfootnote{? ज्ञे}}ये ज्ञानं प्र‚व‚र्त्त‚ते त‚स्य ज्ञानं ज्ञेय‚व्याप‚कं निव‚र्त्त‚मानं ज्ञेय‚म‚पि निव‚र्त्त‚य‚ति ।‚{\tiny $_{lb}$}‚ अतो\textbf{साव‚द‚र्श‚नान्नास्तीत्येवं व्रुवाणः शोभेत} । न स‚र्व‚म् [।] \textbf{त‚दि}ति । त‚स्माद्‚{\tiny $_{lb}$}‚ \leavevmode\ledsidenote{\textenglish{192a/PSVTa}} [।] \textbf{इमे} भावाः \textbf{स‚{\tiny $_{७}$}‚न्तोप्य‚नुप‚ल‚क्ष्याः} अपोह्याः स्युः । क‚थं [।] \textbf{स्व‚भाव‚देश‚काल‚{\tiny $_{lb}$}‚विप्र‚क‚र्षेण} । स्व‚भाव‚श्च देश‚श्च काल‚श्च तैर्विप्र‚क‚र्षो व्य‚व‚धान‚मिति विग्र‚हः ।‚{\tiny $_{lb}$}‚ \textbf{स‚न्न‚पि} क‚श्चिद्देश‚कालाभ्यां स्व‚भावेन च विप्र‚कृष्टः पिशाचादिव‚त् । \href{http://sarit.indology.info/?cref=pv.3.288}{२९१}
	{\color{gray}{\rmlatinfont\textsuperscript{§~\theparCount}}}
	\pend% ending standard par
      ‚{\tiny $_{lb}$}‚

	  
	  \pstart \leavevmode% starting standard par
	न‚नु काल‚व्य‚व‚हितानाम‚तीतानाग‚तानाम‚नुप‚ल‚म्भाद‚स‚त्वं युक्त‚म‚भावादेव । स‚त्यं‚{\tiny $_{lb}$}‚ [।] केव‚ल‚म‚तीतानाग‚तानामिदानीम‚नुप‚ल‚म्भेपि क‚{\tiny $_{१}$}‚दाचित् स‚त्त्वं भूतं क‚दा‚{\tiny $_{lb}$}‚चिद् भ‚विष्य‚तीत्य‚नुप‚ल‚म्भाभाव इति ।
	{\color{gray}{\rmlatinfont\textsuperscript{§~\theparCount}}}
	\pend% ending standard par
      ‚{\tiny $_{lb}$}‚

	  
	  \pstart \leavevmode% starting standard par
	\textbf{त‚था ही}त्यादिनैत‚देव बोध‚य‚ति । \textbf{को ह्य‚स‚र्व‚द‚र्शी । अत्य‚न्त‚प‚रोक्षेर्थे व‚च‚न‚स्याकृत‚{\tiny $_{lb}$}‚क‚स्य स‚म्वाद‚नं । इत‚र‚स्य} कृत‚क‚स्य व‚च‚न‚स्यात्य‚न्त‚प‚रोक्षेर्थे । \textbf{इत‚र‚द्वेत्य}स‚म्वाद‚नं‚{\tiny $_{lb}$}‚ \textbf{भाव‚यितुं} निश्चेतुं \textbf{स‚म‚र्थः} [।] नैव । प‚रेण कृत‚क‚स्यास‚म्वाद‚न‚मिष्ट‚मिति द्व‚य‚मुक्तं ।‚{\tiny $_{lb}$}‚ ‚{\tiny $_{lb}$}‚ \leavevmode\ledsidenote{\textenglish{545/s}}स्यादेत‚द्[।]एक‚स्य‚{\tiny $_{२}$}‚ वेद‚वाक्य‚स्य स‚म्वाद‚द‚र्श‚नात् स‚र्व‚त्र वेदे स‚म्वाद‚न‚मिति [।]
	{\color{gray}{\rmlatinfont\textsuperscript{§~\theparCount}}}
	\pend% ending standard par
      ‚{\tiny $_{lb}$}‚

	  
	  \pstart \leavevmode% starting standard par
	अत आह । \textbf{प्र‚तिपादितं चैत‚च्छेष‚व‚द‚नुमान‚चिन्तायां} पूर्व‚मेव । किं \textbf{प्र‚ति}‚{\tiny $_{lb}$}‚पादित‚मित्याह । \textbf{क्व‚चि}दित्यादि । \textbf{क्व‚चिद्} देश‚काले वा । \textbf{त‚थे}त्येतेन प्र‚कारेण‚{\tiny $_{lb}$}‚ \textbf{दृष्टानाम‚र्थानां पुन‚र‚न्य‚थाभावः} । पूर्व‚दृष्टाकाराद् वैप‚रीत्यं । एत‚देवाह । \textbf{य‚थे}त्यादि‚{\tiny $_{lb}$}‚ \textbf{संस्कार‚विशेषात्} क्षीर‚म‚ध्वादिप‚{\tiny $_{३}$}‚रिष्कारादिल‚क्ष‚णात् । \textbf{आम‚ल‚कीफ‚लानि च‚{\tiny $_{lb}$}‚ क्व‚चिद्देशे म‚धुराणी}ति स‚म्ब‚न्धः । \textbf{न चेदानीम‚त‚द्द‚र्शिना} । म‚धुर‚निम्ब‚फ‚लाद्य‚द‚र्शिना ।‚{\tiny $_{lb}$}‚ \textbf{तानि म‚धुराणि निम्ब‚फ‚लानि प्र‚तिक्षेप्त‚व्यान्ये}व । त‚था वेद‚वाक्यानाम्मिथ्यात्वं‚{\tiny $_{lb}$}‚ य‚दि नाम दृष्ट‚न्त‚थापि न श‚क्य‚म्प्र‚तिक्षेप्तुं । \textbf{त‚स्माद‚कृत‚कं च स्यान्मिथ्यार्थं} चेत्य‚{\tiny $_{lb}$}‚कृत‚क‚मिथ्यार्थ‚त्व‚योर्न \textbf{विरो‚{\tiny $_{४}$}‚ध‚म्प‚श्यामः} ।
	{\color{gray}{\rmlatinfont\textsuperscript{§~\theparCount}}}
	\pend% ending standard par
      ‚{\tiny $_{lb}$}‚

	  
	  \pstart \leavevmode% starting standard par
	न‚नु मिथ्यार्थ‚तायाम‚कृत‚क‚त्व‚स्यानुप‚ल‚म्भाद‚भाव इत्य‚त आह । \textbf{न हीय‚{\tiny $_{lb}$}‚म‚नुप‚ल‚ब्धिर‚दृश्यात्म}स्व‚दृश्य‚स्व‚भावेर्थेष्व‚भाव‚स्य साधिकेत्युक्तं प्राक् । य‚त‚श्चा‚{\tiny $_{lb}$}‚कृत‚क‚त्व‚स्य न मिथ्यात्वेऽभावः सिद्धः । \textbf{तेन य‚त्किंचिन्मिथ्यात्व‚न्त‚त्स‚र्वं पौरुषेयं ।‚{\tiny $_{lb}$}‚ इति} एव‚म्मिथ्यार्थ‚त्व‚स्य कृत‚क‚त्वेना\textbf{व्याप्तिः} । किङ्कार‚ण‚म\textbf{निश्च‚यात्} । अकृत-‚{\tiny $_{५}$}‚‚{\tiny $_{lb}$}‚ क‚त्व‚स्य मिथ्यार्थ‚त्वे व्य‚तिरेकानिश्च‚यादित्य‚र्थः । न चानिश्चित‚व्य‚तिरेकाद्धेतो‚{\tiny $_{lb}$}‚स्स‚काशात् साध्य‚सिद्धिः ।
	{\color{gray}{\rmlatinfont\textsuperscript{§~\theparCount}}}
	\pend% ending standard par
      ‚{\tiny $_{lb}$}‚

	  
	  \pstart \leavevmode% starting standard par
	\textbf{त‚था हि} स्व‚साध्येन हेतो\textbf{र‚न्व‚यो} व्याप्तिः । \textbf{व्य‚तिरेको वा} विप‚क्षाद् व्यावृ‚{\tiny $_{lb}$}‚त्तिर्वा । \textbf{स‚त्वं वा} हेतोः \textbf{साध्य‚ध‚र्मिणि} । प‚क्ष‚ध‚र्म्म‚मित्य‚र्थः । एतानि च त्रीणि‚{\tiny $_{lb}$}‚ रूपाणि \textbf{ज्ञानैः} प्र‚माणै\textbf{र्य‚दि सिध्य‚न्ति} कीदृशैस्त\textbf{न्निश्च‚य‚फ‚लै}स्त‚स्य रूप‚त्र‚य‚स्य‚{\tiny $_{lb}$}‚ नि‚{\tiny $_{६}$}‚श्च‚यः फ‚लं येषामिति विग्र‚हः । त‚दा तानि त्रीणि रूपाणि विव‚क्षित‚स्य‚{\tiny $_{lb}$}‚ साध्य‚स्य \textbf{साध‚न}म्भ‚व‚न्ति । य\textbf{थोक्त}माचार्य दि ग्ना गे न । \textbf{य एव ह्युभ‚य‚निश्चि‚{\tiny $_{lb}$}‚त‚वाची} । वादिप्र‚तिवादिभ्यां निश्चित‚स्य हेतुल‚क्ष‚ण‚युक्त‚स्यार्थ‚स्य वाच‚कः \textbf{स‚{\tiny $_{lb}$}‚ साध‚न‚न्दूष‚ण‚म्वा} । असिद्ध‚त्वादिल‚क्ष‚ण‚युक्त‚स्यार्थ‚स्य वाच‚कः श‚ब्दः प्र‚तिवादिना‚{\tiny $_{lb}$}‚ दूष‚णाभिप्राये‚{\tiny $_{७}$}‚ण प्र‚युक्तः । \textbf{नान्य‚त‚र‚प्र‚सिद्धः स‚न्दिग्ध‚वादो} वादिप्र‚तिवादिभ्या- \leavevmode\ledsidenote{\textenglish{192b/PSVTa}}‚{\tiny $_{lb}$}‚ म‚न्य‚त‚र‚प्र‚सिद्ध‚स्य स‚न्दिग्ध‚स्य चोभ‚योर‚न्य‚त‚र‚स्य च यो वाच‚कः श‚ब्दः स न साध‚नं‚{\tiny $_{lb}$}‚ ‚{\tiny $_{lb}$}‚ \leavevmode\ledsidenote{\textenglish{546/s}}नापि दूष‚णं । किं कार‚णं [।] \textbf{पुनः साध‚नापेक्ष‚त्वात्} । अनिश्चित‚स्य निश्च‚यार्थं‚{\tiny $_{lb}$}‚ पुनः प्र‚माणापेक्ष‚त्वात् । न चाकृत‚क‚त्वेन स‚ह मिथ्यार्थ‚त्व‚स्य विरोधो निश्चितो‚{\tiny $_{lb}$}‚ येनाकृत‚के मिथ्यात्व‚स्य व्य‚ति‚{\tiny $_{१}$}‚रेकः सिद्धः स्यात् ।
	{\color{gray}{\rmlatinfont\textsuperscript{§~\theparCount}}}
	\pend% ending standard par
      ‚{\tiny $_{lb}$}‚

	  
	  \pstart \leavevmode% starting standard par
	अथ स्याद् [।] अकृत‚क‚त्वादेव वेदे मिथ्यात्वं न भ‚व‚तीति चेद् [।]
	{\color{gray}{\rmlatinfont\textsuperscript{§~\theparCount}}}
	\pend% ending standard par
      ‚{\tiny $_{lb}$}‚

	  
	  \pstart \leavevmode% starting standard par
	आह । \textbf{को} हीत्यादि । य‚स्मात् \textbf{को हि} स‚चेता अकृत‚क‚त्वेन स‚हा\textbf{दृष्ट‚विरोध‚स्य}‚{\tiny $_{lb}$}‚ मिथ्यात्व‚स्य \textbf{स‚म्भ‚व‚म्}वेद‚वाक्येषु \textbf{प्र‚त्याच‚क्षीत । त‚दि}ति त‚स्मा\textbf{द‚य‚म}कृत‚क‚त्व‚स्य‚{\tiny $_{lb}$}‚ हेतोर्य‚थोक्तो \textbf{व्य‚तिरेको न साध‚नं} । किं कार‚णं [।] \textbf{संश‚यात्} । इत‚श्च व्य‚तिरेकी‚{\tiny $_{lb}$}‚ हेतोर्नास्तीति द‚र्श‚य‚{\tiny $_{२}$}‚न्नाह ।
	{\color{gray}{\rmlatinfont\textsuperscript{§~\theparCount}}}
	\pend% ending standard par
      ‚{\tiny $_{lb}$}‚

	  
	  \pstart \leavevmode% starting standard par
	\textbf{अपि चे}त्यादि । \textbf{य‚त्र} विष‚ये \textbf{साध्य‚विप‚क्ष‚स्य} । साध्यं स‚त्यार्थ‚त्व‚न्त‚द्विप‚क्ष‚स्य‚{\tiny $_{lb}$}‚ मिथ्यात्व‚स्य \textbf{व‚र्ण्ण‚य‚ते व्य‚तिरेकिता} । य‚त्कृत‚कं न भ‚व‚ति त‚न्मिथ्यार्थ‚न्न भ‚व‚तीति ।‚{\tiny $_{lb}$}‚ य एव मिथ्यार्थ‚त्व‚व्य‚व‚च्छेद‚स्य विष‚यः । \textbf{स एवास्य} कृत‚क‚स्य हेतोः \textbf{स‚प‚क्षः स्याद्‚{\tiny $_{lb}$}‚ [।] अतः} कार‚णात् \textbf{स‚र्वो हेतुर‚न्व‚यी} । अन्व‚य‚व्य‚तिरेकी [।] व्य‚तिरेक‚स्य प्र‚कृत‚{\tiny $_{lb}$}‚त्वात् । न व्य‚तिरे‚{\tiny $_{३}$}‚क्येवेत्य‚र्थः ।
	{\color{gray}{\rmlatinfont\textsuperscript{§~\theparCount}}}
	\pend% ending standard par
      ‚{\tiny $_{lb}$}‚

	  
	  \pstart \leavevmode% starting standard par
	न‚नु साध्य‚ध‚र्म‚सामान्येन स‚मानोर्थः स‚प‚क्षः साध‚र्म्य‚दृष्टान्त उच्य‚ते । न‚{\tiny $_{lb}$}‚ चाय‚मिहास्ति [।] त‚त्क‚थं स एवास्य स‚प‚क्षः स्यादित्युच्य‚ते ।
	{\color{gray}{\rmlatinfont\textsuperscript{§~\theparCount}}}
	\pend% ending standard par
      ‚{\tiny $_{lb}$}‚

	  
	  \pstart \leavevmode% starting standard par
	स‚त्यं [।] स‚प‚क्ष‚साध्य‚त्वेनान्व‚य एव स‚प‚क्ष उच्य‚ते । अत एवाह [।] \textbf{स‚र्वो‚{\tiny $_{lb}$}‚ हेतुर‚तोन्व‚यीति} ।
	{\color{gray}{\rmlatinfont\textsuperscript{§~\theparCount}}}
	\pend% ending standard par
      ‚{\tiny $_{lb}$}‚

	  
	  \pstart \leavevmode% starting standard par
	\textbf{य‚दि}त्यादिना व्याच‚ष्टे । \textbf{य‚त्किञ्चिन्मिथ्यार्थ‚न्त‚त्स‚र्व‚म्पौरुषेय‚मिति} । एवं‚{\tiny $_{lb}$}‚ \textbf{हेतो}र‚कृत‚क‚त्व‚स्य \textbf{विप‚क्षेण} कृत‚{\tiny $_{४}$}‚क‚त्वेन \textbf{साध्य‚विप‚क्ष‚स्य} मिथ्यात्व‚स्य या \textbf{व्याप्तिः‚{\tiny $_{lb}$}‚ सा त‚द‚भावे} कृत‚क‚त्वाभावे मिथ्यात्व‚स्या\textbf{भाव‚सिद्धौ} स‚त्यां \textbf{स्यात्} । किं कार‚णं [।]‚{\tiny $_{lb}$}‚ कृत‚क‚त्वाभावे \textbf{भ‚व‚तो} मिथ्यार्थ‚त्व‚स्य \textbf{तेन} कृत‚क‚त्वेन \textbf{व्याप्त्य‚योगात्} । त‚स्माद्‚{\tiny $_{lb}$}‚ व्याप्तिमिच्छ‚ता । मिथ्यार्थ‚त्व‚स्य कृत‚क‚त्व‚निवृत्त्या निवृत्तिरेष्ट‚व्या । \textbf{यैव} च‚{\tiny $_{lb}$}‚ साध्य‚साध‚न\textbf{विजातीय‚यो}र्मिथ्यात्व‚कृत‚क‚त्व‚यो\textbf{र्व्यावृत्तिसि‚{\tiny $_{५}$}‚द्धिः} । कृत‚क‚त्व‚निवृत्या‚{\tiny $_{lb}$}‚ ‚{\tiny $_{lb}$}‚ \leavevmode\ledsidenote{\textenglish{547/s}}मिथ्यार्थ‚त्व‚न्निव‚र्त्त‚त इत्येवंरूपा । \textbf{सैवा}कृत‚क‚स्य हेतोः स‚त्यार्थ‚त्वेनान्व‚य‚स्थितिर‚{\tiny $_{lb}$}‚\textbf{न्व‚य‚व्य‚व‚स्थितिः} । किं कार‚ण‚म् [।] \textbf{विप‚क्ष‚व्य‚व‚च्छेद‚ल‚क्ष‚ण‚त्वात् साध्य‚स्य} स‚त्या‚{\tiny $_{lb}$}‚र्थ‚त्व‚स्य । विप‚क्षो मिथ्यात्व‚न्त‚स्य कृत‚क‚त्व‚निव‚र्त्त‚नो कृत‚के यो व्य‚व‚च्छेदो व्यावृ‚{\tiny $_{lb}$}‚त्तिस्त‚ल्ल‚क्ष‚ण‚त्वात् ।
	{\color{gray}{\rmlatinfont\textsuperscript{§~\theparCount}}}
	\pend% ending standard par
      ‚{\tiny $_{lb}$}‚

	  
	  \pstart \leavevmode% starting standard par
	किञ्च [।] स‚त्यार्थ‚ताप्र‚तिषेधो मिथ्यार्थ‚त्वं । त‚{\tiny $_{६}$}‚स्य च मिथ्यार्थ‚त्व‚स्या‚{\tiny $_{lb}$}‚कृत‚के य‚दा प्र‚तिषेधः कृत‚स्त‚दा \textbf{प्र‚तिषेध‚द्व‚य}ञ्जातं । अस्मा\textbf{च्च} स‚त्यार्थ‚ताव्य‚व‚{\tiny $_{lb}$}‚च्छेद‚रूपात् प्र‚तिषेध‚द्व‚यात् स‚त्यार्थ‚ता\textbf{विधिसिद्धे}रिति हेतोः \textbf{कान‚न्व‚या}न्व‚य‚र‚हिता ।‚{\tiny $_{lb}$}‚ साध‚न‚व्य‚तिरेकेण कृत‚क‚त्वेन साध्य\textbf{व्य‚तिरे}क‚स्य मिथ्यात्व‚स्य व्याप्तिसिद्धिर्नैवान‚{\tiny $_{lb}$}‚न्व‚या \textbf{व्याप्तिसिद्धिः} ।
	{\color{gray}{\rmlatinfont\textsuperscript{§~\theparCount}}}
	\pend% ending standard par
      ‚{\tiny $_{lb}$}‚

	  
	  \pstart \leavevmode% starting standard par
	\textbf{त‚दि}ति त‚स्मा\textbf{न्न क‚श्चिद्धेतुर‚न‚{\tiny $_{७}$}‚न्व‚यो नाम} । किन्तु स‚र्वोन्व‚य‚व्य‚तिरेक‚वानेव । \leavevmode\ledsidenote{\textenglish{193a/PSVTa}}‚{\tiny $_{lb}$}‚ किं कार‚ण‚म् [।] \textbf{एक‚व्य‚व‚च्छेद‚स्या}कृत‚के मिथ्यार्थ‚त्व‚व्य‚व‚च्छेद‚स्य । \textbf{विजातीय‚{\tiny $_{lb}$}‚सिद्धिनान्त‚रीय‚क‚त्वात्} । मिथ्यार्थ‚ता विजातीय‚स्य स‚त्यार्थ‚त्व‚स्य या सिद्धिस्त‚न्ना‚{\tiny $_{lb}$}‚न्त‚रीय‚क‚त्वात् ।
	{\color{gray}{\rmlatinfont\textsuperscript{§~\theparCount}}}
	\pend% ending standard par
      ‚{\tiny $_{lb}$}‚

	  
	  \pstart \leavevmode% starting standard par
	य‚द्येक‚व्य‚व‚च्छेद‚स्त‚द्विजातीय‚सिद्धिनान्त‚रीय‚क‚स्त‚दा\textbf{ऽनित्य‚निरात्मादिव्य‚व‚च्छे‚{\tiny $_{lb}$}‚देपि} अनित्य‚ताया निरात्म‚ताया आदिश‚ब्दाद् दुःख‚त्वादीनाञ्च । य‚{\tiny $_{१}$}‚दा क्व‚चि‚{\tiny $_{lb}$}‚च्छ‚श‚विषाणादौ व्य‚व‚च्छेदः क्रिय‚ते त‚दापि \textbf{त‚था स्यात्} । मिथ्यार्थ‚ताव्य‚व‚च्छेदेन‚{\tiny $_{lb}$}‚ स‚त्यआर्थ‚सिद्धिव‚द‚नित्य‚त्वादिविजातीयानान्नित्य‚सात्म‚क‚सुख‚त्वादीनां सिद्धिः‚{\tiny $_{lb}$}‚ स्यात् । न चैत‚दिष्टं [।] त‚थात्रापि माभूदिति प‚रो म‚न्य‚ते ।
	{\color{gray}{\rmlatinfont\textsuperscript{§~\theparCount}}}
	\pend% ending standard par
      ‚{\tiny $_{lb}$}‚

	  
	  \pstart \leavevmode% starting standard par
	\textbf{ने}त्यादिना प‚रिह‚र‚ति । नाय‚न्दोषः । किं कार‚णं [।] \textbf{व्य‚तिरेक‚व्य‚व‚च्छेद‚स्य‚{\tiny $_{lb}$}‚ भाव‚रूप‚त्वात्} । व्य‚तिरेकोऽभावो [।] अभाव‚स्य च यो व्य‚{\tiny $_{२}$}‚व‚च्छेदो निवृत्तिस्त‚स्य‚{\tiny $_{lb}$}‚ भाव‚रूप‚त्वात् । अभाव‚निवृत्त्या भाव‚व्य‚व‚स्थेति याव‚त् । त‚द‚नेन‚भाव‚ल‚क्ष‚ण‚मुक्तं ।‚{\tiny $_{lb}$}‚ अस्मादेव व‚च‚नादिद‚म‚प्य‚र्थादुक्त‚म्भ‚व‚ति । भाव‚व्य‚व‚च्छेद‚स्याभाव‚रूप‚त्वादिति ।‚{\tiny $_{lb}$}‚ त‚द‚नेन भावाभाव‚योस्ताव‚ल्ल‚क्ष‚ण‚मुक्तं । त‚त्र य‚स्मिन् व्य‚व‚च्छिद्य‚माने । न भावा‚{\tiny $_{lb}$}‚नुष‚ङ्गः । य‚स्मिंश्च व्य‚व‚च्छिद्य‚माने भावानुष‚ङ्ग‚स्त‚द्द‚र्श‚य‚न्नाह । \textbf{न भा}वेत्या‚{\tiny $_{३}$}‚दि ।‚{\tiny $_{lb}$}‚ \textbf{भाव}स्व‚भाव‚स्य \textbf{व्य‚व‚च्छेदे} स‚ति न \textbf{भावानुष‚ङ्गो} न भाव‚स्याक्षेपः । भावो न भ‚व‚{\tiny $_{lb}$}‚‚{\tiny $_{lb}$}‚ \leavevmode\ledsidenote{\textenglish{548/s}}तीति त‚त्प्र‚तिषेध‚स्यैव स‚म्भ‚वात् । \textbf{अभाव}व्य‚व‚च्छेद‚स्व‚भाव‚स्य विर‚ह‚मात्र‚स्य \textbf{तु}‚{\tiny $_{lb}$}‚ यो \textbf{व्य‚व‚च्छेद‚स्स निय‚मेन भाव}स्यो\textbf{प‚स्थाप‚नो} भाव‚स्याक्षेप‚कः । किङ्कार‚ण‚म्‚{\tiny $_{lb}$}‚ [।] \textbf{भावाभाव‚योर‚न्योन्यं} यो \textbf{विवेको} विर‚ह‚स्त‚द्रू\textbf{प‚त्वात्} । त‚त्स्व‚भाव‚त्वात् । भाव‚{\tiny $_{lb}$}‚विवेक‚स्याभाव‚रूप‚त्वाद‚{\tiny $_{४}$}‚भाव‚विवेक‚स्य च भाव‚रूप‚त्वादित्य‚र्थः ।
	{\color{gray}{\rmlatinfont\textsuperscript{§~\theparCount}}}
	\pend% ending standard par
      ‚{\tiny $_{lb}$}‚

	  
	  \pstart \leavevmode% starting standard par
	एव‚न्ताव‚द् भावाभाव‚व्य‚व‚च्छेद‚योर्भावानाक्षेपाक्षेप‚क‚त्व\edtext{}{\edlabel{pvsvt_548-1}\label{pvsvt_548-1}\lemma{त्व}\Bfootnote{\cite{pvb-B} भावानाक्षेप‚क‚त्व‚मुक्त्वा ।}}मुक्त्वा प्र‚कृतं योज‚{\tiny $_{lb}$}‚य‚त्य\textbf{भाव‚रूप‚स्}त्वित्यादिना । स‚त्यार्थ‚ताया\textbf{स्तु} यो \textbf{व्य‚तिरेको} मिथ्यार्थ‚ताल‚क्ष‚णः ।‚{\tiny $_{lb}$}‚ \textbf{स} स‚त्यार्थ‚ताऽभाव‚रूपा स \textbf{व्य‚तिरिच्य‚मानः} कृत‚क‚त्व‚निवृत्या निव‚र्त्त‚मानो \textbf{भावं}‚{\tiny $_{lb}$}‚ स‚त्यार्थ‚ताल‚क्ष‚ण\textbf{मुप‚स्थाप‚य‚त्य}कृत‚के ।
	{\color{gray}{\rmlatinfont\textsuperscript{§~\theparCount}}}
	\pend% ending standard par
      ‚{\tiny $_{lb}$}‚

	  
	  \pstart \leavevmode% starting standard par
	य‚दि तु स‚त्यार्थ‚{\tiny $_{५}$}‚त्व‚विप‚रीत‚रूपं मिथ्यार्थ‚त्व‚म्प‚रेणाभ्युप‚ग‚म्य‚ते । त‚दा वेद‚वा‚{\tiny $_{lb}$}‚ क्येषु कृत‚क‚निवृत्तौ मिथ्यात्व\edtext{}{\edlabel{pvsvt_548-2}\label{pvsvt_548-2}\lemma{मिथ्यात्व}\Bfootnote{\cite{pvb-B} मिथ्यार्थ‚त्व‚मेव ।}}मेव न स्यान्न तु स‚त्यार्थ‚त्व‚न्त‚त‚श्चान‚र्थ‚क्यं स्यादिति‚{\tiny $_{lb}$}‚ व‚क्ष्य‚ति । त‚स्माद‚व‚श्यं स‚त्यार्थ‚ता भाव\textbf{रूपा\edtext{}{\edlabel{pvsvt_548-3}\label{pvsvt_548-3}\lemma{रूपा}\Bfootnote{\cite{pvb-B} रूपो ।}}} । मिथ्यार्थ‚ताल‚क्ष‚णो ध‚र्म एष्ट‚व्यः‚{\tiny $_{lb}$}‚ [।] स च व्य‚तिरिच्य‚मान‚स्स‚त्यार्थ‚ताया भाव‚मुप‚स्थाप‚य‚ति । \textbf{नैवं नैरात्म्याद‚यो}‚{\tiny $_{lb}$}‚ विर‚ह‚रूपाः [।] किं कार‚णं [।] \textbf{स्व‚भाव‚विशेषात्} । ‚{\tiny $_{६}$}‚स्व‚भाव‚विशेषो हि नैरात्म्यं ।‚{\tiny $_{lb}$}‚ त‚मेवाह । \textbf{क्रिया} शुभाशुभ‚क‚र‚णं । \textbf{भोगः} सुख‚दुःखानुभ‚व‚स्त‚यो\textbf{र‚धिष्ठानं} स्वीकार‚{\tiny $_{lb}$}‚स्त‚त्रा\textbf{स्व‚त‚न्त्रः} प‚र‚व‚शो \textbf{ह्यात्मा} स्व‚भावो \textbf{निरात्मा} । स्व‚भाव‚प‚र्यायोत्रात्म‚श‚ब्दः ।‚{\tiny $_{lb}$}‚ किं पुनः कार‚ण‚म् [।] एवंभूतः स्व‚भावो निरात्मा भ‚व‚तीत्याह । \textbf{त‚त्स्वात‚न्त्र्ये}‚{\tiny $_{lb}$}‚त्यादि । \textbf{त‚स्मिन्} क्रियाभोगाधिष्ठाने य‚त्\textbf{स्वात‚न्त्र्यं} त‚ल्\textbf{ल‚क्ष \textbf{ण} \edtext{\textsuperscript{*}}{\edlabel{pvsvt_548-4}\label{pvsvt_548-4}\lemma{*}\Bfootnote{\cite{pvb-B}}}त्वात्} । त‚त्स्व‚{\tiny $_{lb}$}‚\leavevmode\ledsidenote{\textenglish{193b/PSVTa}} भाव‚त्वादेवा\textbf{त्म‚नः‚{\tiny $_{७}$}‚} प‚रैः क‚ल्पित‚स्य । अत‚स्त‚द्वैध‚र्म्येणास्व‚त‚न्त्र‚स्व‚भावो निरात्मा‚{\tiny $_{lb}$}‚ भ‚व‚तीत्य‚भिप्रायः ।
	{\color{gray}{\rmlatinfont\textsuperscript{§~\theparCount}}}
	\pend% ending standard par
      ‚{\tiny $_{lb}$}‚

	  
	  \pstart \leavevmode% starting standard par
	य‚त एव‚न्त\textbf{दि}ति त‚स्माद् \textbf{रूपं} स्व‚भावो \textbf{नैरात्म्यं नात्म‚निवृत्तिमात्रं\edtext{}{\edlabel{pvsvt_548-5}\label{pvsvt_548-5}\lemma{निवृत्तिमात्रं}\Bfootnote{\cite{pvb-B} नात्म‚व्य‚व‚च्छेद‚मात्रं-- added}} । अन्य‚था}‚{\tiny $_{lb}$}‚ य‚दि निवृत्तिमात्रं\edtext{}{\edlabel{pvsvt_548-6}\label{pvsvt_548-6}\lemma{निवृत्तिमात्रं}\Bfootnote{\cite{pvb-B} स्वात‚न्त्र्यः ।}} नैरात्म्यं स्यात्त‚दा निःस्व‚भाव‚त्वान्निरुपाख्य‚मेव त‚द् भ‚वेत् ।‚{\tiny $_{lb}$}‚ \textbf{निरुपाख्ये} च \textbf{कृत‚क‚त्वादेः} [।] आदिश‚ब्दात्प्र‚य‚त्नान‚न्त‚रीय‚क‚त्वादेर्व‚स्तुध‚र्म‚स्य‚{\tiny $_{lb}$}‚ हेतोर\textbf{योगात् । त‚तः} कृत‚क‚त्वादेर्हेतोस्स‚{\tiny $_{१}$}‚काशान्नैरात्म्यं ग‚ति\textbf{र्नैरात्म्य‚सिद्धिर्न‚{\tiny $_{lb}$}‚ \leavevmode\ledsidenote{\textenglish{549/s}}स्यात्} । इष्य‚ते च [।] त‚स्मान्न बौद्ध‚स्यात्म‚विर‚ह‚मात्रं नैरात्म्यं । न व्य‚व‚च्छेद‚मात्रं‚{\tiny $_{lb}$}‚ विव‚क्ष्य‚ते [।] किन्त्\textbf{वात्म‚व्य‚व‚च्छेदेन निरात्म‚नो भाव}स्य व‚स्तुनो निरात्म‚श‚ब्देन‚{\tiny $_{lb}$}‚ \textbf{प‚राम‚र्शाद}भिधानात् कृत‚क‚त्वादेस्स‚काशान्नैरात्म्य‚ग‚तिर्न प्राप्नोतीत्य‚य\textbf{म‚दोष इति‚{\tiny $_{lb}$}‚ चेत्} । एव‚म‚प्यात्म\textbf{प‚र्युदासेन व‚स्तुसंस्प‚र्शात् । त‚देवा}स्म‚दुक्त‚म्\textbf{व‚स्तुरूप}म्भा‚{\tiny $_{२}$}‚व‚स्व‚भा‚{\tiny $_{lb}$}‚व\textbf{न्नैरात्म्य‚मायातं} । न च भाव‚व्य‚व‚च्छेदे भावान्त‚रानुष‚ङ्गः । त‚स्माद् बौद्धानां‚{\tiny $_{lb}$}‚ श‚श‚विषाणादौ नैरात्म्य‚व्य‚व‚च्छेदेपि नात्म‚नो भाव‚सिद्धिः । तेन य‚दुक्त म‚नित्य‚{\tiny $_{lb}$}‚निरात्म‚ताव्य‚व‚च्छेदेपि त‚था स्यादि ति त‚त्प‚रिहृतं ।
	{\color{gray}{\rmlatinfont\textsuperscript{§~\theparCount}}}
	\pend% ending standard par
      ‚{\tiny $_{lb}$}‚

	  
	  \pstart \leavevmode% starting standard par
	\textbf{य‚स्यापि} नै या यि का दे \textbf{र्नाभाव‚रूप} आत्मआ \textbf{व्य‚तिरेकः} [।] किन्त‚र्हि [।]‚{\tiny $_{lb}$}‚ स्व‚भावाभाव एव \textbf{त‚स्य} नैयायिकादे\textbf{र्भाव‚रूप}स्य‚{\tiny $_{३}$}‚ नैरात्म्य‚स्य \textbf{व्य‚व‚च्छेदे ना}त्म‚नो‚{\tiny $_{lb}$}‚ \textbf{भाव‚सिद्धिः स्यात् । इति} हेतोः श‚श‚विषाणादौ नैरात्म्य‚व्य‚व‚च्छेदेप्यात्म‚नो \textbf{नान्व‚{\tiny $_{lb}$}‚यानुष‚ङ्गः} । आत्म‚नान्व‚योनुग‚मो न भ‚व‚तीत्य‚र्थः । य‚था च श‚श‚विषाणादौ नैरात्म्य‚{\tiny $_{lb}$}‚व्य‚व‚च्छेदेप्यात्म‚नो नान्व‚यानुग‚म\textbf{स्त‚था} जीव‚च्छ‚रीरेऽ\textbf{नैरात्म्येपि} नैरात्म्याभावेपि‚{\tiny $_{lb}$}‚ \textbf{ना}त्म‚नो \textbf{भाव‚सिद्धिः स्यात्} ।
	{\color{gray}{\rmlatinfont\textsuperscript{§~\theparCount}}}
	\pend% ending standard par
      ‚{\tiny $_{lb}$}‚

	  
	  \pstart \leavevmode% starting standard par
	अथ‚वान्य‚था व्याख्याय‚ते । मी मां स को क्त‚व्य‚तिरेकिनिराक‚र‚ण‚प्र‚स्तावेन‚{\tiny $_{lb}$}‚ नैयायिकोक्त‚म‚पि व्य‚तिरेकिणं निराक‚र्त्तुमाह । \textbf{य‚स्यापी}त्यादि । य‚स्यापि नैयायि‚{\tiny $_{lb}$}‚कादे\textbf{र्नाभाव‚रूप} आत्मा \textbf{व्य‚तिरेकः} [।] किन्त‚र्हि [।] नैरात्म्य‚म्भाव‚स्व‚भाव एव ।‚{\tiny $_{lb}$}‚ \textbf{त‚स्यै}व‚म्वादिनो नैयायिक‚स्य \textbf{भाव‚रूप‚स्य} नैरात्म्य‚स्य \textbf{व्य‚व‚च्छेदे ना}त्म‚नो \textbf{भाव‚{\tiny $_{lb}$}‚सिद्धिः स्यात्} । भा‚{\tiny $_{५}$}‚व‚व्य‚व‚च्छेदे भावान्त‚रानुष‚ङ्गाभावात् । \textbf{इति} हेतोर्य‚त्र प्राणा‚{\tiny $_{lb}$}‚दिम‚त्व‚न्त‚त्रात्मेति \textbf{नान्व‚यानुष‚ङ्गः} । य‚था च नान्व‚यानुष‚ङ्ग\textbf{स्त‚था} साध्य‚ध‚र्मि\textbf{ण्य‚{\tiny $_{lb}$}‚नैरात्म्येपि} नैरात्म्य‚निवृत्ताव‚पि \textbf{ना}त्म‚नो \textbf{भाव‚सिद्धिः} स्यात् ।
	{\color{gray}{\rmlatinfont\textsuperscript{§~\theparCount}}}
	\pend% ending standard par
      ‚{\tiny $_{lb}$}‚

	  
	  \pstart \leavevmode% starting standard par
	एत‚देव प्र‚योग‚पूर्व‚क‚न्द‚र्श‚य‚न्नाह । \textbf{य‚थे}त्यादि । \textbf{इदं जीव‚च्छ‚रीरं न निरात्म‚कं‚{\tiny $_{lb}$}‚ प्राणादिम‚त्वादिति} । आश्वासः प्राणः । आदिश‚ब्दात्‚{\tiny $_{६}$}‚ प्र‚श्वासादिप‚रिग्र‚हः ।‚{\tiny $_{lb}$}‚ अत्र प्र‚योगे जीव‚च्छ‚रीर‚स्य सात्म‚क‚त्वं साध्य‚न्न चोभ‚य‚सिद्धः । स प‚क्ष‚भूत‚{\tiny $_{lb}$}‚ आत्मास्तीति नान्व‚य‚श्चिन्त्य‚ते । केव‚लं साध्य‚स्यात्म‚नो हेतोश्च प्राणादिम‚त्त्व‚स्य‚{\tiny $_{lb}$}‚ य‚थाक्र‚मं यौ \textbf{विप‚क्षौ} नैरात्म्य‚म‚प्राणादिम‚त्वं च \textbf{त‚योर्व्याप्य‚व्याप‚क‚भाव‚चिन्तायां}‚{\tiny $_{lb}$}‚ क्रिय‚माणायाम\textbf{प्र‚माणादिम‚त्व एव नैरात्म्यं दृष्ट‚न्त‚द‚भावे} व्याप‚क‚स्या‚{\tiny $_{७}$}‚प्राणादि- \leavevmode\ledsidenote{\textenglish{194a/PSVTa}}‚{\tiny $_{lb}$}‚ ‚{\tiny $_{lb}$}‚ \leavevmode\ledsidenote{\textenglish{550/s}}म‚त्त्व‚स्याभावे \textbf{च} व्याप्यं नैरात्म्यं । प्राणादिम‚ति \textbf{नास्तीति} न्यायात् । \textbf{स्व‚यं न भ‚व‚{\tiny $_{lb}$}‚द‚पि} नैरात्म्यं \textbf{प्राणादीनां} हेतुत्वेनाभिम‚तानाम्\textbf{आत्म‚नि} स‚प‚क्ष‚भूते \textbf{न सिद्धिमुप‚स्था‚{\tiny $_{lb}$}‚प‚य‚ति} । नैरात्म्य‚स्य भाव‚रूप‚स्य व्य‚व‚च्छेदेप्य‚नात्म\edtext{}{\edlabel{pvsvt_550-1}\label{pvsvt_550-1}\lemma{नात्म}\Bfootnote{\cite{pvb-B} ०प्यात्म‚ल‚क्ष‚ण‚स्य ।}}ल‚क्ष‚ण‚स्य भाव‚स्यानाक्षे‚{\tiny $_{lb}$}‚पात् । त‚तः स‚त्य‚पि व्य‚तिरेकेन्व‚यानुष‚ङ्गाभावाद् व्य‚तिरेक्येव हेतुः प्राणादिरिति‚{\tiny $_{lb}$}‚ न या यि क स्येष्ट‚सि‚{\tiny $_{१}$}‚द्धिरेवेय‚ता ग्र‚न्थेन द‚र्शिता ।
	{\color{gray}{\rmlatinfont\textsuperscript{§~\theparCount}}}
	\pend% ending standard par
      ‚{\tiny $_{lb}$}‚

	  
	  \pstart \leavevmode% starting standard par
	एत‚स्मिन्न‚भ्युप‚ग‚मे दोष‚माह । \textbf{त‚थे}त्यादि । अप्राणादिम‚त्व‚निवृत्त्या प्राणा‚{\tiny $_{lb}$}‚दिभ्यो निव‚र्त्त्य‚मान‚म‚पि नैरात्म्यं । य‚था प्राणादीनामात्म‚नि स‚प‚क्ष‚भूते न सिद्धि‚{\tiny $_{lb}$}‚मुप‚स्थाप‚य‚ति । \textbf{त‚था साध्येपि} जीव‚च्छ‚रीरे \textbf{प्राणादि}हेतुभिर्न्नैरात्म्यं \textbf{व्युद‚स्य‚मानं‚{\tiny $_{lb}$}‚ केव‚लं स्यात्} । न त्वात्म‚न उप‚स्थाप‚कं ।
	{\color{gray}{\rmlatinfont\textsuperscript{§~\theparCount}}}
	\pend% ending standard par
      ‚{\tiny $_{lb}$}‚

	  
	  \pstart \leavevmode% starting standard par
	एत‚देव ग्र‚ह‚ण‚क‚वाक्यं स्प‚{\tiny $_{२}$}‚ष्ट‚य‚न्नाह । \textbf{नैरात्म्येऽभावा}दित्यादि । \textbf{नैरात्म्ये}‚{\tiny $_{lb}$}‚ प्राणादीना\textbf{म‚भावात् प्राणाद‚य‚स्त‚न्निर}स‚ना नैरात्म्य‚मात्र‚व्य‚व‚च्छेद‚का \textbf{नात्मोप‚स्था‚{\tiny $_{lb}$}‚प‚काः} । किं कार‚णं [।] \textbf{त‚त्र} सात्म‚के प्राणादे\textbf{र्भावासिद्धेः । न च} जीव‚च्छ‚रीरे‚{\tiny $_{lb}$}‚ \textbf{नैरात्म्य}स्यात्म‚विरुद्ध‚स्य \textbf{निवृत्त्यात्म‚सिद्धिः} । किं कार‚णं [।] \textbf{विप‚क्षा}न्नैरात्म्यात्‚{\tiny $_{lb}$}‚ प्राणादे\textbf{र्व्य‚तिरेक‚द‚र्श‚नेपि} क्रिय‚माणे स‚प‚क्षेप्यात्म‚नि प्रा‚{\tiny $_{३}$}‚णादीनां सिद्धि\textbf{प्र‚स‚ङ्गात्} ।
	{\color{gray}{\rmlatinfont\textsuperscript{§~\theparCount}}}
	\pend% ending standard par
      ‚{\tiny $_{lb}$}‚

	  
	  \pstart \leavevmode% starting standard par
	य‚त एवं [।] त‚स्मा\textbf{न्न विप‚र्य‚य‚व्याप्तिः} । सात्म‚क‚त्व‚प्राणादिम‚त्व‚विप‚क्ष‚योर्नैरा‚{\tiny $_{lb}$}‚त्म्यात् प्राणादिम‚त्त्व‚योर्न व्याप्तिसिद्धिरित्य‚र्थः । क‚दा [।] नैरात्म्यात् प्राणा‚{\tiny $_{lb}$}‚\textbf{देर्व्य‚तिरेकासिद्धौ} । एवं हि हेतुविप‚क्षेणाप्राणादिम‚त्त्वेन साध्य‚विप‚क्ष‚स्य नैरात्म्य‚स्य‚{\tiny $_{lb}$}‚ व्याप्तिसिद्धिः स्यात् य‚दि प्राणादिभ्यो नैरात्म्यं । निव‚र्त्तेत । सा च निवृ‚{\tiny $_{४}$}‚त्तिर‚न्व‚ये‚{\tiny $_{lb}$}‚ स‚ति स्यात् । त‚दाह [।] \textbf{त‚त्सिद्धिरेव चे}त्यादि । त‚स्य व्य‚तिरेक‚स्य \textbf{सिद्धिरेवान्व‚य‚{\tiny $_{lb}$}‚ सिद्धिः} । त‚न्नान्त‚रीय‚क‚त्वात्त‚स्य । व्य‚तिरेक‚सिद्धिरेवान्व‚य‚सिद्ध‚रुक्ता ।
	{\color{gray}{\rmlatinfont\textsuperscript{§~\theparCount}}}
	\pend% ending standard par
      ‚{\tiny $_{lb}$}‚

	  
	  \pstart \leavevmode% starting standard par
	अथान्व‚य‚सिद्धिर्नेष्य‚ते । त‚दाऽ\textbf{सिद्धौ} चान्व‚य‚स्य । त‚द्व्य‚तिरेक‚वृत्तिसिद्धा‚{\tiny $_{lb}$}‚व‚पि । \textbf{त}स्यात्म‚नो \textbf{व्य‚तिरे}को नैरात्म्य‚न्त‚स्य प्राणादेः स‚काशा\textbf{न्निवृत्तिसिद्धाव‚पि ।‚{\tiny $_{lb}$}‚ त‚द‚सिद्धि}स्त‚स्यात्म‚न‚{\tiny $_{५}$}‚स्स‚प‚क्ष‚भूत‚स्यासिद्धि\textbf{रिति} कृत्वा \textbf{साध्येपि} जीच्छ‚रीरे नैरा‚{\tiny $_{lb}$}‚‚{\tiny $_{lb}$}‚ ‚{\tiny $_{lb}$}‚ \leavevmode\ledsidenote{\textenglish{551/s}}त्म्य‚निवृत्ताव‚प्यात्म‚नोऽसिद्धि\textbf{प्र‚संगः} । त‚दिति त‚स्मा\textbf{द‚न्व‚या}ऽन्व‚य‚र‚हिता \textbf{न व्य‚ति‚{\tiny $_{lb}$}‚रेक‚व्याप्तिः} साध्य‚साध‚न‚विप‚क्ष‚योर्न व्याप्तिरित्य‚र्थः ।
	{\color{gray}{\rmlatinfont\textsuperscript{§~\theparCount}}}
	\pend% ending standard par
      ‚{\tiny $_{lb}$}‚

	  
	  \pstart \leavevmode% starting standard par
	अभ्युप‚ग‚म्यापीति तु ब्रूमः । भ‚व‚तु नामान‚न्व‚या विप‚र्य‚य‚व्याप्तिः । जीव‚च्छ‚{\tiny $_{lb}$}‚रीराच्च नैरात्म्य‚निवृत्तौ सात्म‚क‚त्वं जीव‚च्छ‚रीर‚स्य प्र‚कारान्त‚राभा‚{\tiny $_{६}$}‚वात् ।‚{\tiny $_{lb}$}‚ स‚त्यार्थ‚तासाध‚ने त्व‚कृत‚के हेताव‚य‚म्प्र‚कारो न स‚म्भ‚व‚तीत्याह । \textbf{मिथ्यार्थ‚ताया}‚{\tiny $_{lb}$}‚ इत्यादि । \textbf{मिथ्यार्थ‚तायास्तु} साध्य‚विप‚क्ष‚भूतायाः \textbf{पौरुषेय‚त्वेन व्याप्त्या} हेतुभूत‚या‚{\tiny $_{lb}$}‚ श‚ब्दा\textbf{द‚पौरुषेयान्निवृत्ताव‚पि न स‚त्यार्थ‚त्व‚म}कृत‚क‚स्य सिध्येत् । किं कार‚णं [।]‚{\tiny $_{lb}$}‚ स‚त्यार्थ‚त्व‚व्य‚तिरेकेणान‚र्थ‚क्य‚स्य \textbf{प्र‚कारान्त}र‚स्य \textbf{स‚म्भ‚वात्}‚{\tiny $_{७}$}‚ । स‚त्यार्थ‚त्व‚मिथ्यार्थ- \leavevmode\ledsidenote{\textenglish{194b/PSVTa}}‚{\tiny $_{lb}$}‚ त्वे \textbf{न द्वैराश्ये} तु \textbf{श‚ब्दानामेत‚त्} स्यात् । मिथ्यार्थ‚त्व‚निवृत्तौ स‚त्यार्थ‚त्वं स्यात् । किं‚{\tiny $_{lb}$}‚ कार‚ण‚म् [।] \textbf{एक‚निवृत्तौ} मिथ्यार्थ‚तानिवृत्तौ स‚त्यार्थ‚ताव्य‚तिरेकेण \textbf{ग‚त्य‚न्त‚रा‚{\tiny $_{lb}$}‚भावात्} । द्वैराश्य‚मेव तु नास्ति ग‚त्य‚न्त‚र‚स‚म्भ‚वाद् [।] अत‚स्ते श‚ब्दा \textbf{अन‚र्था अपि‚{\tiny $_{lb}$}‚ स्युरिति नेष्ट‚सिद्धिर्न} स‚त्यार्थ‚तासिद्धिः ।
	{\color{gray}{\rmlatinfont\textsuperscript{§~\theparCount}}}
	\pend% ending standard par
      ‚{\tiny $_{lb}$}‚

	  
	  \pstart \leavevmode% starting standard par
	श‚ब्देभ्यो\textbf{र्थ‚प्र‚तीतेः} कार‚णान्ना\textbf{न‚र्थ‚का इति चेत्} ।‚{\tiny $_{१}$}‚ \textbf{ना}यं स्वाभाविकोर्थ‚प्र‚त्य‚यः‚{\tiny $_{lb}$}‚ किन्त्\textbf{वे}ष \textbf{पुरुष‚व्यापारः स्यात्} । संकेतः पुरुष‚व्यापार‚स्त‚त्फ‚ल‚त्वात्पुरुष‚व्यापार इत्य‚{\tiny $_{lb}$}‚भेदेनोक्तः पुरुष‚व्यापारात् स्यादित्य‚र्थः । किम्व‚त् । प्र‚सिद्धाद‚र्थाद\textbf{र्थान्त‚र‚विक‚ल्प‚व‚त्} ।
	{\color{gray}{\rmlatinfont\textsuperscript{§~\theparCount}}}
	\pend% ending standard par
      ‚{\tiny $_{lb}$}‚

	  
	  \pstart \leavevmode% starting standard par
	एत‚देव व्याच‚ष्टे । \textbf{य‚थे}त्यादि । जै मि नी यै रिष्टेनार्थेन \textbf{भ र तो र्व श्या दि‚{\tiny $_{lb}$}‚च‚रितादि}क‚स्या\textbf{त‚द‚र्थ‚त्वेपि} त‚द्भ‚र‚त‚च‚रित‚मुर्व‚शीच‚रि‚{\tiny $_{२}$}‚तं । आदिश‚ब्दाद‚न्य‚स्यापि‚{\tiny $_{lb}$}‚ पु रू र व श्च‚रितादेर्ग्र‚ह‚णं । द्वितीयेनादिग्र‚ह‚णे न भ‚र‚तादीनां च‚रिताद‚न्य‚द‚प्य‚{\tiny $_{lb}$}‚व‚स्थाविशेषादि\textbf{क‚म‚र्थ‚म‚न्ये} मी मां स का लोक‚प्र‚सिद्धाद‚र्थाद\textbf{न्य‚था व्याच‚क्ष‚ते} । भ‚र‚तो‚{\tiny $_{lb}$}‚ यूपः । उर्व‚र्शी पात्री । अर‚णिर्वेत्यादिना ।
	{\color{gray}{\rmlatinfont\textsuperscript{§~\theparCount}}}
	\pend% ending standard par
      ‚{\tiny $_{lb}$}‚
	    
	    \stanza[\smallbreak]
	  व्याख्यातृविक‚ल्पिताद् अर्थ‚प्र‚तीतिर्न भ‚व‚त्येवेति चेद् [।]\&[\smallbreak]
	  
	  
	  ‚{\tiny $_{lb}$}‚

	  
	  \pstart \leavevmode% starting standard par
	आह । \textbf{त‚द‚नुसारेणे}त्यादि । व्याख्यातृभिर्वि‚{\tiny $_{३}$}‚क‚ल्पितार्था\textbf{नुसारेण । केषांचि}च्छ्रोतॄणां‚{\tiny $_{lb}$}‚ \textbf{प्र‚तीति}र्भ‚व‚त्येव [।] \textbf{त}थेति भ‚र‚तादिश‚ब्द‚व्याख्याव‚त् । स्व‚भावाद\textbf{न‚र्थ‚केष्व‚पि} वेद‚{\tiny $_{lb}$}‚वाक्येष्\textbf{व‚र्थ‚विक‚ल्पः पुरुष‚कृतः स्यान्न श‚ब्द‚स्व‚भाव‚कृतः} । किं कार‚णं [।] वैदिके‚{\tiny $_{lb}$}‚‚{\tiny $_{lb}$}‚ \leavevmode\ledsidenote{\textenglish{552/s}}भ्योर्थ‚प्र‚तिप‚त्तौ \textbf{पुरुष}स्य जै मि नि प्र‚भृते\textbf{रुप‚देशापेक्ष‚णात्} । किमिव [।] \textbf{अर्थान्त‚{\tiny $_{lb}$}‚र‚व‚देव} । भ‚र‚तोर्व‚श्यादिश‚ब्दानाम‚र्थान्त‚र‚विक‚ल्प‚व‚दि‚{\tiny $_{४}$}‚त्य‚र्थः । स्व‚भाव‚तोर्थ‚प्र‚तिपा‚{\tiny $_{lb}$}‚द‚नेपि पुरुषोप‚देश‚म‚पेक्षिष्य‚न्त इत्य‚पि मिथ्या । य‚तो \textbf{न हि प्र‚कृत्या} स्व‚भावेनार्थ‚{\tiny $_{lb}$}‚\textbf{प्र‚काश‚नास्तं} पुरुषोप‚देश\textbf{म‚पेक्ष‚न्ते व‚ह्न्याद‚यः । पुरुष‚स्तु} स्व‚मात्मीयं \textbf{स‚म‚य‚व्यापार}‚{\tiny $_{lb}$}‚संकेत‚साम‚र्थ्येना\textbf{च‚क्षाणः} प‚र‚स्मै \textbf{उप‚दिश‚ति} । य‚थायं श‚ब्दोस्मिन्न‚र्थे म‚या प्र‚युक्त‚{\tiny $_{lb}$}‚ \textbf{इति} न्याय्यं [।] न तु प्र‚कृत्यार्थ‚प्र‚काश‚ने पु‚{\tiny $_{५}$}‚रुषोप‚देशो न्याय्यः ।
	{\color{gray}{\rmlatinfont\textsuperscript{§~\theparCount}}}
	\pend% ending standard par
      ‚{\tiny $_{lb}$}‚

	  
	  \pstart \leavevmode% starting standard par
	अथ पुरुष‚स‚मितार्थ‚व‚न्निस‚र्ग‚सिद्धोपि वेदोर्थेषु पुरुषोप‚देश‚म‚पेक्ष‚ते । त‚दा‚{\tiny $_{lb}$}‚ य‚श्च \textbf{पुरुषे}ण \textbf{स‚मितः} संकेतितो य‚श्च \textbf{निस‚र्गे}ण स्व‚भावेन \textbf{सि}द्धोर्थः । त‚योर्द्व‚योर‚पि‚{\tiny $_{lb}$}‚ पुरुषोप‚देशा\textbf{पेक्ष‚णं} प्र‚त्य‚विशेषात् । \textbf{अन्य}श्च क‚श्चिद्विशेषो नास्ति येनायं पुरुष‚{\tiny $_{lb}$}‚स‚मितोयं निस‚र्ग‚सिद्ध इति प्र‚तीयेत । \textbf{अतः} कार‚णा\textbf{दे‚{\tiny $_{६}$}‚को} वैदिकोर्थे \textbf{नैस‚र्गि}कः ।‚{\tiny $_{lb}$}‚ निस‚र्गे भ‚व इत्याध्यात्मादित्वाट्ठ‚क् । \textbf{अन्यो} लौकिक‚श‚ब्दोर्थे \textbf{पौरुषेय इति दुर‚{\tiny $_{lb}$}‚व‚सानं} । दुर्बोधं । विभाग‚साध‚क‚प्र‚माणाभावात् ।
	{\color{gray}{\rmlatinfont\textsuperscript{§~\theparCount}}}
	\pend% ending standard par
      ‚{\tiny $_{lb}$}‚

	  
	  \pstart \leavevmode% starting standard par
	\textbf{अस्ति विशेषो} निस‚र्ग‚सिद्ध‚स्य वैदिक‚स्य कोपि प्र‚माण‚स‚म्वादः । प्र‚मा‚{\tiny $_{lb}$}‚णान्त‚रानुग‚म‚नं । य‚स्तु नैवं स पौरुषेय इति चेत् । \textbf{एत‚दुत्त‚र‚त्र निषेत्स्या}मो य‚था‚{\tiny $_{lb}$}‚ \leavevmode\ledsidenote{\textenglish{195a/PSVTa}} \textbf{नास्त्य‚न्त‚प‚{\tiny $_{७}$}‚रोक्षेर्थे प्र‚माणान्त‚र‚वृत्तिरि}त्यादिना ग्र‚न्थेन । पुरुषोप‚देशापेक्ष‚णा‚{\tiny $_{lb}$}‚दिना च \textbf{स‚मान‚ध‚र्म‚णो}र्लौकिक‚वैदिक‚योः \textbf{प्र‚माण‚स‚म्वाद‚मात्र‚विशेषादेक‚त्र} वैदिके‚{\tiny $_{lb}$}‚ \textbf{ऽपौरुषेय‚त्वे}न क‚ल्प्य‚माने । \textbf{ब‚हुत‚र‚मिदानीं} लौकिक‚मेक‚वाक्यं प्र‚माण‚स‚म्वा\textbf{द्य‚पौ‚{\tiny $_{lb}$}‚रुषेयं} क‚ल्प्यं स्यात् । य‚स्मात् \textbf{स‚न्ति पुरुष‚कृतान्य‚पि वाक्यानि कानिचिद्} अनि‚{\tiny $_{lb}$}‚त्या व‚त संस्कारा\edtext{\textsuperscript{*}}{\edlabel{pvsvt_552-1}\label{pvsvt_552-1}\lemma{*}\Bfootnote{ध‚म्म‚प‚दे}} इत्येव‚मादी‚{\tiny $_{१}$}‚नि । \textbf{एव‚म्विधानीति} प्र‚माण‚स‚म्वादीनीति कृत्वा‚{\tiny $_{lb}$}‚ \textbf{तेष्व}पौरुषेय‚त्व\textbf{प्र‚संगः} ।
	{\color{gray}{\rmlatinfont\textsuperscript{§~\theparCount}}}
	\pend% ending standard par
      ‚{\tiny $_{lb}$}‚

	  
	  \pstart \leavevmode% starting standard par
	अथ प्र‚माण‚स‚म्वादिनोपि लौकिक‚स्य पौरुषेय‚त्व‚न्त‚दा \textbf{त‚द्व‚देषाम‚पि} वैदिकानां‚{\tiny $_{lb}$}‚ ‚{\tiny $_{lb}$}‚ ‚{\tiny $_{lb}$}‚ \leavevmode\ledsidenote{\textenglish{553/s}}श‚ब्दानाम\textbf{भिप्रेतार्थ‚व‚त्ता पौरुषेयी च स्यात् प्र‚माणानुरोधिनी च} प्र‚माणास‚म्वा‚{\tiny $_{lb}$}‚दिनी चेति न विशेषं प‚श्यामो वैदिकानां श‚ब्दानां लौकिकेभ्यः ।
	{\color{gray}{\rmlatinfont\textsuperscript{§~\theparCount}}}
	\pend% ending standard par
      ‚{\tiny $_{lb}$}‚

	  
	  \pstart \leavevmode% starting standard par
	वैदिकानां श‚ब्दानां म‚न्त्र‚त्वादेवापौरुषे‚{\tiny $_{२}$}‚य‚त्व‚मिति द‚र्श‚य‚न्नाह । \textbf{अपि चे}त्यादि ।‚{\tiny $_{lb}$}‚ \textbf{व्याह‚तं} प‚र‚स्प‚र‚विरुद्धं । त‚था हि [।] \textbf{म‚न्त्राणां क‚स्य‚चि}त्पुरुष‚स्य \textbf{स‚म‚य‚त्वे} प्र‚तिज्ञा‚{\tiny $_{lb}$}‚व्य‚व‚स्थापित‚त्वे स‚ति \textbf{कार्य‚साध‚नं युक्त}म‚भिम‚त‚कार्य‚सिद्धिर्म‚न्त्र‚प्र‚योगाद् युक्तेति ।
	{\color{gray}{\rmlatinfont\textsuperscript{§~\theparCount}}}
	\pend% ending standard par
      ‚{\tiny $_{lb}$}‚

	  
	  \pstart \leavevmode% starting standard par
	\textbf{य‚द्येत} इत्यादिना व्याख्यानं । \textbf{एते म‚न्त्रा य‚दि क‚स्य‚चित्} प्र‚भाव‚व‚तः क‚र्त्तुः‚{\tiny $_{lb}$}‚ \textbf{स‚म‚यः} स्यादिति व‚क्ष्य‚माणेन स‚म्ब‚न्धः । स‚म‚य‚व्य‚व‚स्थापित‚त्वात्स‚{\tiny $_{३}$}‚म‚य इत्युच्य‚न्ते ।‚{\tiny $_{lb}$}‚ क‚थं पुन‚र‚सौ स‚म‚यः कृत इत्याह । \textbf{प‚रार्थे}त्यादि । \textbf{प‚रार्थ‚प}र‚ता प‚रार्थ‚प्र‚धान‚ता ।‚{\tiny $_{lb}$}‚ कृपालुतेति याव‚त् । \textbf{त}स्या \textbf{अनुरोधेन । अन्य‚तो वा कुत‚श्चि}द् य‚शःप्र‚भृते\textbf{र्हेतोः}‚{\tiny $_{lb}$}‚ कृतः \textbf{स्यात्} । कीदृशोसौ स‚म‚य इत्याह । \textbf{य‚थे}त्यादि । \textbf{य‚था म‚त्प्र‚णीत‚मेत}द् \textbf{वाक्यं}‚{\tiny $_{lb}$}‚ [।] किं भूत‚म् [।] \textbf{अभिम‚तार्थोप‚निब‚न्ध‚नं} । म‚न्त्र‚स्य क‚र्त्तुर‚भिम‚तो योर्थो विधि‚{\tiny $_{lb}$}‚विशेष‚{\tiny $_{४}$}‚ः । आवाह‚न‚विस‚र्ज‚नादिल‚क्ष‚णः स निब‚न्ध‚नं प्र‚वृत्त्य‚ङ्गं य‚स्मिन् वाक्ये‚{\tiny $_{lb}$}‚ त‚त्त‚थोक्तं । \textbf{एव}मित्य‚नेनानुक्र‚मेण \textbf{नियुंजानं} प्र‚युंजानं पुरुष‚म\textbf{नेनार्थेन} पुरुषाभिम‚तेन‚{\tiny $_{lb}$}‚ फ‚लेन \textbf{योज‚यामीति} स‚म‚यः स्यात् । \textbf{त‚दा म‚न्त्र‚प्र‚योगात् क‚दाचित्} विधिस‚माप्तौ ।‚{\tiny $_{lb}$}‚ अभिम‚त्\textbf{आर्थ‚निष्प‚त्तिः} स्यात् । \textbf{क‚विस‚म‚यादिव} । य‚था केन‚चित् काव्यं कृत्वैवं‚{\tiny $_{lb}$}‚ स‚म‚यः कृतः [।] म‚त्प्र‚णी‚{\tiny $_{५}$}‚तं काव्यं यः प‚ठ‚ति । त‚स्मै म‚येदं दात‚व्य‚मित्य‚त‚स्त‚त्कृत‚{\tiny $_{lb}$}‚काव्य\textbf{पाठ‚कानां} य‚था प्र‚तिज्ञातार्थ‚निष्प‚त्तिस्त‚द्व‚त् ।
	{\color{gray}{\rmlatinfont\textsuperscript{§~\theparCount}}}
	\pend% ending standard par
      ‚{\tiny $_{lb}$}‚

	  
	  \pstart \leavevmode% starting standard par
	\textbf{अथ भाव‚श‚क्तिः} श‚ब्द‚स्व‚भाव‚स्यैव सा तादृशी श‚क्तिर्येनाभिम‚त‚म्फ‚ल‚म्भ‚व‚ति ।‚{\tiny $_{lb}$}‚ न य‚थाभिम‚तात् पुरुष‚स‚म‚यात् । त‚दा स्या\textbf{द‚न्य‚त्रापि} । य‚था क‚थंचित्प्र‚युक्तान्म‚{\tiny $_{lb}$}‚न्त्राद‚पि । किं कार‚णं [।] व‚र्ण्णात्म‚क‚स्य म‚न्त्र‚स्या\textbf{विशेष‚तः} ।
	{\color{gray}{\rmlatinfont\textsuperscript{§~\theparCount}}}
	\pend% ending standard par
      ‚{\tiny $_{lb}$}‚

	  
	  \pstart \leavevmode% starting standard par
	\textbf{ने}त्यादि व्याख्यानं‚{\tiny $_{६}$}‚ । \textbf{न वै पुरुष‚स‚म‚या}द्धेतो\textbf{र्म‚न्त्रे}भ्योर्थ \textbf{\textbf{सिद्धिः} । किन्त‚{\tiny $_{lb}$}‚र्हि [।] भाव}स्य व‚र्ण्ण‚रूप‚स्य म‚न्त्र‚स्य \textbf{स्व‚भाव एष [।] य‚दि न} म‚न्त्राः क‚स्य‚चिद्‚{\tiny $_{lb}$}‚ ‚{\tiny $_{lb}$}‚ \leavevmode\ledsidenote{\textenglish{554/s}}विधिपूर्व\textbf{न्नियुक्ताः फ‚ल‚दाः} ।
	{\color{gray}{\rmlatinfont\textsuperscript{§~\theparCount}}}
	\pend% ending standard par
      ‚{\tiny $_{lb}$}‚

	  
	  \pstart \leavevmode% starting standard par
	\textbf{त‚त्त‚र्ही}त्यादि [।] सि द्धा न्त वा दी । \textbf{त‚त्त‚र्हि} म‚न्त्राख्यानं \textbf{व‚र्ण्णानां रूपं स‚र्व‚{\tiny $_{lb}$}‚त्रेति} विधिर‚हिते काले । विप‚रीतादिप्र‚योगे वा\textbf{ऽविशिष्ट‚मिति । य‚थाक‚थंचि}त्पा‚{\tiny $_{lb}$}‚\leavevmode\ledsidenote{\textenglish{195b/PSVTa}} ठानु‚{\tiny $_{७}$}‚क्र‚मं विधिं चोल्लंघ्य \textbf{प्र‚युक्ताद‚पि} म‚न्त्राद‚भिम‚तं \textbf{फ‚लं स्यात्} । य‚स्माद् \textbf{व‚र्ण्णा‚{\tiny $_{lb}$}‚ एव हि म‚न्त्रो नान्य‚त् किञ्चिद्} व‚र्ण्ण‚व्य‚तिरिक्तं ।
	{\color{gray}{\rmlatinfont\textsuperscript{§~\theparCount}}}
	\pend% ending standard par
      ‚{\tiny $_{lb}$}‚

	  
	  \pstart \leavevmode% starting standard par
	\textbf{त‚त्क्र‚मो} व‚र्ण्ण‚क्र‚मो न व‚र्ण्णा एवे\textbf{ति चेत्} [।]
	{\color{gray}{\rmlatinfont\textsuperscript{§~\theparCount}}}
	\pend% ending standard par
      ‚{\tiny $_{lb}$}‚

	  
	  \pstart \leavevmode% starting standard par
	त‚द‚स‚त् [।] य‚स्माद् व‚र्ण्णेभ्यः \textbf{क्र‚म‚स्यार्थान्त‚र‚त्वं च} व‚र्ण्णानुपूर्वी वाक्यं चेदि‚{\tiny $_{lb}$}‚त्य‚त्रान्त‚रे \textbf{पूर्व‚मेव निराकृतं} ।
	{\color{gray}{\rmlatinfont\textsuperscript{§~\theparCount}}}
	\pend% ending standard par
      ‚{\tiny $_{lb}$}‚

	  
	  \pstart \leavevmode% starting standard par
	\textbf{ने}त्यादि व्याख्यानं । \textbf{व‚र्ण्णे}भ्यः क्र‚म‚स्या\textbf{व्य‚तिरेके च व‚र्ण्णा एव म‚न्त्रास्ते च‚{\tiny $_{१}$}‚}‚{\tiny $_{lb}$}‚ व‚र्ण्णा \textbf{अविशिष्टाः स‚र्व‚त्र} प्र‚तिलोम‚पाठादाविति \textbf{स‚र्व‚था} य‚थाक‚थ‚ञ्चित् प्र‚युक्ता‚{\tiny $_{lb}$}‚ \textbf{फ‚ल‚दाः स्युः} । न च फ‚ल‚दा भ‚व‚न्ति । न केव‚लं विधिभ्रंशे न फ‚ल‚दाः प्र‚त्युतान‚र्थ‚{\tiny $_{lb}$}‚कारिण एव भ‚व‚न्तीत्याह । \textbf{उप‚प्ल‚व} इत्यादि । उप‚द्र‚वः । \textbf{उप‚प्ल‚वः} । तु श‚ब्दो‚{\tiny $_{lb}$}‚तिश‚ये । \textbf{अल्पीय‚सोपि} विधि\textbf{क्र‚म‚स्य भ्रंशाद् दृष्टः} । स च पुरुष‚स‚म‚य‚त्वे म‚न्त्राणां‚{\tiny $_{lb}$}‚ युज्य‚ते । नापौरुषेय‚त्वे‚{\tiny $_{२}$}‚ ।
	{\color{gray}{\rmlatinfont\textsuperscript{§~\theparCount}}}
	\pend% ending standard par
      ‚{\tiny $_{lb}$}‚

	  
	  \pstart \leavevmode% starting standard par
	क‚स्य‚चित् स‚म‚य‚त्वेपि क‚थ‚मुप‚प्ल‚व इति चेदाह । \textbf{क‚स्य‚चिदि}त्यादि । म‚न्त्र‚स्य‚{\tiny $_{lb}$}‚ क‚र्त्त्रा ये विध‚यो निर्दिष्टास्तेषां म‚ध्ये \textbf{क‚स्य‚चि}द्विधे\textbf{र‚नुष्ठाना}द् \textbf{देव‚ता}याः स‚न्नि‚{\tiny $_{lb}$}‚धिर्भ‚व‚ति । त‚त\textbf{स्स‚न्नि}धेर‚न्य‚स्य विधिविशेष\textbf{स्यासाक‚ल्ये}नास‚म्पाद‚नेन । देव‚ताया‚{\tiny $_{lb}$}‚ \textbf{विराध‚नात्} खेद‚ना\textbf{च्चो}प‚प्ल‚वः स्यात् । त‚त्रावीत‚रागा देव‚ता । विरागिता स्व‚य‚{\tiny $_{lb}$}‚मेवान‚र्थं क‚रोति । वीत‚रा‚{\tiny $_{३}$}‚गाः तु न स्व‚यं । त‚द‚भिप्र‚स‚न्नास्त्व‚न्ये देव‚ताद‚यः‚{\tiny $_{lb}$}‚ कुर्व‚न्तीति द्र‚ष्ट‚व्यं ।
	{\color{gray}{\rmlatinfont\textsuperscript{§~\theparCount}}}
	\pend% ending standard par
      ‚{\tiny $_{lb}$}‚

	  
	  \pstart \leavevmode% starting standard par
	य‚त्र म‚न्त्रे न क‚श्चिद‚पि विधिः क्रिय‚ते त‚त्र क‚थ‚मिति [।]
	{\color{gray}{\rmlatinfont\textsuperscript{§~\theparCount}}}
	\pend% ending standard par
      ‚{\tiny $_{lb}$}‚

	  
	  \pstart \leavevmode% starting standard par
	आह । \textbf{स‚र्वे}त्यादि । \textbf{स‚र्व}विधि\textbf{भ्रंशे तु क‚स्य‚चिदे}व \textbf{स‚म‚य‚स्य} म‚न्त्र‚प्र‚णेतृकृत‚स्य‚{\tiny $_{lb}$}‚ ‚{\tiny $_{lb}$}‚ \leavevmode\ledsidenote{\textenglish{555/s}}विधेर\textbf{नुष्ठानाद्} देव‚ताया \textbf{अस‚न्निधेर्नार्थान‚र्थौ} ।
	{\color{gray}{\rmlatinfont\textsuperscript{§~\theparCount}}}
	\pend% ending standard par
      ‚{\tiny $_{lb}$}‚

	  
	  \pstart \leavevmode% starting standard par
	\textbf{किंचे}त्यादिना दोषान्त‚र‚माह । निवारितं \textbf{क्र‚म}स्यार्थान्त‚र‚त्वं । भ‚व‚तु वा व‚र्ण्णे‚{\tiny $_{lb}$}‚भ्यः क्र‚म‚स्यार्थान्त‚र‚त्व‚म‚न‚र्था‚{\tiny $_{४}$}‚न्त‚र‚त्व‚म्वा । त‚त्रा\textbf{न‚र्थान्त‚र‚त्वे} क्र‚म‚स्य व‚र्ण्णात्मा‚{\tiny $_{lb}$}‚ व‚र्ण्ण‚स्व‚भाव एव म‚न्त्रः । अर्थान्त‚र‚त्वे तु त‚त्क्र‚मात्मा । व‚र्ण्ण‚क्र‚मात्मा । त‚स्य‚{\tiny $_{lb}$}‚ \textbf{व‚र्णात्म‚न‚स्त‚त्क्र‚मात्म‚नो वा म‚न्त्र‚स्य} कीदृश‚स्य्\textbf{आर्थ‚हेतोः} पुरुषार्थ‚कार‚ण‚स्य हेतुभि\textbf{र‚{\tiny $_{lb}$}‚कृत‚त्वान्नित्य‚स्य नित्यं स‚न्निधान‚मिति} कृत्वा \textbf{नित्य‚न्त‚द‚र्थ‚सिद्धिः स्यात्} । तेभ्यो‚{\tiny $_{lb}$}‚ म‚न्त्रेभ्यः पुरुषार्थ‚स्य निष्प‚त्तिः स्यात् । किङ्कार‚णं [।] \textbf{य‚तो} य‚स्या \textbf{हि भा}व‚{\tiny $_{५}$}‚‚{\tiny $_{lb}$}‚\textbf{श‚क्ते}र्म‚न्त्र‚श‚क्तेः स‚काशात् म‚न्त्र‚साध्य‚स्य \textbf{फ‚ल}स्यो\textbf{त्प‚त्तिः सा} भाव‚श‚क्ति\textbf{र‚विक‚लेति‚{\tiny $_{lb}$}‚ न फ‚ल‚वैक‚ल्यं स्यात्} । य‚स्मा\textbf{न्न हि कार‚ण}स्य \textbf{साक‚ल्ये} स‚ति \textbf{कार्य}स्य \textbf{वैक‚ल्य}म‚स‚त्त्वं‚{\tiny $_{lb}$}‚ \textbf{युक्तं} । किं कार‚णं [।] \textbf{त‚स्या}विक‚ल‚स्य कार‚ण‚स्य कार्य‚म‚कुर्व‚तो \textbf{कार‚ण‚त्व‚{\tiny $_{lb}$}‚प्र‚स‚ङ्गात्} । नित्य‚त्वेपि म‚न्त्राणां \textbf{न केव‚लान्म‚न्त्र‚प्र‚योगा}दिति म‚न्त्र‚स‚म्ब‚न्धा\textbf{दिष्ट‚{\tiny $_{lb}$}‚सिद्धिः} । किङ्कार‚णं [।] \textbf{त‚स्य} म‚न्त्र‚स्य \textbf{विधा‚{\tiny $_{६}$}‚नापेक्ष‚त्वादिति चेत्} । त‚द‚युक्तं‚{\tiny $_{lb}$}‚ य‚स्मान्म‚न्त्र‚स्य विधानापेक्ष‚त्वेऽभ्युप‚ग‚म्य‚माने त‚स्या\textbf{पेक्ष}णीय‚स्य विधानादेर्म‚न्त्रं‚{\tiny $_{lb}$}‚ \textbf{प्र‚त्य‚साम‚र्थ्य}म‚नाधेयातिश‚य‚त्वान्म‚न्त्र‚स्य ।
	{\color{gray}{\rmlatinfont\textsuperscript{§~\theparCount}}}
	\pend% ending standard par
      ‚{\tiny $_{lb}$}‚

	  
	  \pstart \leavevmode% starting standard par
	त‚द्व्याच‚ष्टे । \textbf{य‚दि ही}त्यादि । विधानादिभि\textbf{र्म‚न्त्र} क‚ल्पो\edtext{}{\edlabel{pvsvt_555-1}\label{pvsvt_555-1}\lemma{ल्पो}\Bfootnote{In the margin. }} विधिस्त‚स्मा‚{\tiny $_{lb}$}‚\textbf{द‚न्य‚तो वे}ति \textbf{कुत}श्चित्स‚ह‚कारिणः स्थान‚विशेषादेस्स‚काशात् क‚ञ्चित् \textbf{स्व‚भावा‚{\tiny $_{lb}$}‚तिश‚य‚मासाद‚ये}युर्ल‚भेर‚न् ।‚{\tiny $_{७}$}‚ त‚दा स‚ह‚कारी \textbf{त}त्रोत्पाद्ये म‚न्त्र‚स्य स्व‚भावातिश‚ये \leavevmode\ledsidenote{\textenglish{196a/PSVTa}}‚{\tiny $_{lb}$}‚ \textbf{स‚म‚र्थोपेक्ष्यः स्या}त् । \textbf{न च नित्येष्वेत}द‚तिश‚योत्पाद‚न\textbf{म‚स्तीत्युक्तं} प्राक् । \textbf{त‚त्कि‚{\tiny $_{lb}$}‚म}यं स‚ह‚कार्य‚तिश‚योत्पाद‚नं प्र\textbf{त्य‚स‚म}र्थो म‚न्त्रै\textbf{र‚पेक्ष्य‚त इति} कृत्व्\textbf{आन‚पेक्षा} म‚न्त्राः \textbf{स‚दा}‚{\tiny $_{lb}$}‚ कार्यं \textbf{कुर्युः} । य‚दि\edtext{}{\edlabel{pvsvt_555-1b}\label{pvsvt_555-1b}\lemma{दि}\Bfootnote{१ In the margin.}} कार‚क‚स्व‚भावा [।] नो चे\textbf{न्न वा क‚दाचित्} कार्यं कुर्यु\textbf{र‚न‚ति‚{\tiny $_{lb}$}‚‚{\tiny $_{lb}$}‚ ‚{\tiny $_{lb}$}‚ \leavevmode\ledsidenote{\textenglish{556/s}}श‚यात्} स्व‚भास्य स‚र्व‚दा तुल्य‚त्वात् ।
	{\color{gray}{\rmlatinfont\textsuperscript{§~\theparCount}}}
	\pend% ending standard par
      ‚{\tiny $_{lb}$}‚

	  
	  \pstart \leavevmode% starting standard par
	किं च [।] \textbf{स‚र्व‚स्य} पुंसः पात‚कादियुक्त‚{\tiny $_{१}$}‚स्यापि \textbf{साध‚नं} फ‚ल‚हेत‚व\textbf{स्ते} न म‚न्त्राः‚{\tiny $_{lb}$}‚ \textbf{स्युः} । स‚म‚य‚निर‚पेक्षा \textbf{य‚दीदृशी} म‚न्त्राणाम्\textbf{भाव‚श‚क्तिः} । अथ स्याद् य‚ज‚मानेनैव‚{\tiny $_{lb}$}‚ प्र‚युक्ताः फ‚ल‚दा इति [।]
	{\color{gray}{\rmlatinfont\textsuperscript{§~\theparCount}}}
	\pend% ending standard par
      ‚{\tiny $_{lb}$}‚

	  
	  \pstart \leavevmode% starting standard par
	अत आह । \textbf{प्र‚योक्तुर्भेदो} विशेषो य‚ज‚मान‚त्व‚न्त\textbf{द‚पेक्षा च । नासंस्कार्य‚स्य}‚{\tiny $_{lb}$}‚ म‚न्त्र‚स्य \textbf{युज्य‚ते । य‚दि भाव‚श‚क्त्यैव} स‚म‚यान‚पेक्ष‚या \textbf{म‚न्त्राः} फ‚ल‚दा \textbf{न ते} म‚न्त्राः‚{\tiny $_{lb}$}‚ फ‚ल‚दानं प्र‚ति \textbf{क‚ञ्चित्} पुरुषं \textbf{प‚रिह‚रेयुः । अन्यं वा} शूद्रादि‚{\tiny $_{२}$}‚कं । य‚स्मा\textbf{न्न ह्य‚न्य‚{\tiny $_{lb}$}‚म}य‚ज‚मानं \textbf{प्र‚ति स्व‚भावो} म‚न्त्राणां कार्य‚क‚र‚ण‚स्व‚भावो\textbf{ऽत‚द्भावो भ‚व}त्य‚ज‚न‚क‚स्व‚{\tiny $_{lb}$}‚भावो भ‚व‚ति । किं कार‚णं [।] \textbf{त‚स्य} ज‚न‚क‚स्व‚भाव‚स्य \textbf{तेना}ब्राह्म‚णेन चान‚प‚क‚र्ष‚{\tiny $_{lb}$}‚णात् । अख‚ण्ड‚नात् । \textbf{अन्येन} च य‚ज‚मानादिना\textbf{नुत्क‚र्ष‚णा}त् । अत‚श्च कार‚णात्‚{\tiny $_{lb}$}‚ \textbf{केन‚दित्} पुरुषेण \textbf{स‚ह} म‚न्त्राणां \textbf{कार्य‚कार‚ण‚भाव}योगः । त\textbf{द‚योगाच्च} य‚थाक्र‚मं‚{\tiny $_{lb}$}‚ ब्राह्म‚णेनान्ये‚{\tiny $_{३}$}‚न च म‚न्त्राणां \textbf{प्र‚त्यास‚त्तिविप्र‚क‚र्षाभावात्} स‚र्व‚स्य साध‚नं स्युरिति ।‚{\tiny $_{lb}$}‚ \textbf{अत एव} न नित्य‚त्वादेवा\textbf{स्य} म‚न्त्र‚स्या\textbf{संस्कार्य‚त्वात् प्र‚योक्ता} । कार्ये नियोक्ता‚{\tiny $_{lb}$}‚ क‚श्चिद् ब्राह्म‚णोन्यो वा \textbf{नास्ति य‚तः प्र‚योक्ता} म‚न्त्र‚साध्यं \textbf{फ‚ल‚म‚श्नुवीत} ल‚भेत‚{\tiny $_{lb}$}‚ ब्राह्म‚ण एव नान्यः । त‚था हि य‚था शूद्रादिर‚पाठ‚को न किंचित्क‚रोतीति न प्र‚योक्ता‚{\tiny $_{lb}$}‚ त‚था ब्राह्म‚णोपि [।] त‚तः संस्काराप्र‚तिप‚त्तेरिति ।
	{\color{gray}{\rmlatinfont\textsuperscript{§~\theparCount}}}
	\pend% ending standard par
      ‚{\tiny $_{lb}$}‚

	  
	  \pstart \leavevmode% starting standard par
	कि‚{\tiny $_{४}$}‚ञ्च \textbf{संस्कार्य‚स्या}प्याधेयातिश‚य‚स्यापि \textbf{भाव‚स्य व‚स्तुभेदो} हि कार‚ण‚भेदो‚{\tiny $_{lb}$}‚ \textbf{हि भेद‚को} न च ब्राह्म‚ण‚शूद्रादीनां स्व‚भाव‚भेदः प‚र‚मार्थ‚तोस्ति । केव‚लं लोक‚व्य‚{\tiny $_{lb}$}‚व‚हार‚कृतो विप्र‚शूद्रादिभेदः । तेन लोक‚व्य‚व‚हार‚भिन्नानां ब्राह्म‚णादीनां \textbf{प्र‚योक्त‚{\tiny $_{lb}$}‚णाम्भेदान्निय‚मो} म‚न्त्र\textbf{श‚क्तौ न} स‚म्भ‚व‚ति । येन ब्राह्म‚ण एव फ‚ल‚मासाद‚येन्न शूद्रः ।
	{\color{gray}{\rmlatinfont\textsuperscript{§~\theparCount}}}
	\pend% ending standard par
      ‚{\tiny $_{lb}$}‚‚{\tiny $_{lb}$}‚\textsuperscript{\textenglish{557/s}}

	  
	  \pstart \leavevmode% starting standard par
	क्व त‚र्ह्य‚य‚न्निय‚मः स्या‚{\tiny $_{५}$}‚दिति [।]
	{\color{gray}{\rmlatinfont\textsuperscript{§~\theparCount}}}
	\pend% ending standard par
      ‚{\tiny $_{lb}$}‚

	  
	  \pstart \leavevmode% starting standard par
	आह । \textbf{स‚म‚ये भ‚वेत्} । य‚दा स‚म‚यो म‚न्त्र‚स्त‚दा स‚म‚य‚स्य क‚र्त्ता व‚स्तुस्व‚भावान‚पेक्षः‚{\tiny $_{lb}$}‚ स‚म‚यं क‚रोति । य‚थालोके ये ब्राह्म‚णाः प्र‚सिद्धास्तेभ्य एव प्र‚योक्तृभ्यः फ‚ल‚न्दास्यामि‚{\tiny $_{lb}$}‚ नान्येभ्य इति स्यान्निय‚मः ।
	{\color{gray}{\rmlatinfont\textsuperscript{§~\theparCount}}}
	\pend% ending standard par
      ‚{\tiny $_{lb}$}‚

	  
	  \pstart \leavevmode% starting standard par
	\textbf{आधेये}त्यादिना व्याख्यानं । \textbf{आधेयो} ज‚न्यो \textbf{विशे}षो येषान्ते भावाः । \textbf{त‚द्धेतो}‚{\tiny $_{lb}$}‚र्विशेष‚हेतोः \textbf{स्व‚भाव‚भेदे} स‚ति । \textbf{त‚तो} विशेष‚हेतोः स‚का‚{\tiny $_{६}$}‚शाद्\textbf{आसादितातिश‚य‚त्वादे}‚{\tiny $_{lb}$}‚क‚त्रान्य‚थाभूतः पुन‚र\textbf{न्य‚त्र} कार‚णान्त‚रे\textbf{ऽन्य‚था} स्यु\textbf{र्न} त्व\textbf{भेदे} कार‚ण‚स्यान्य‚थाभावः ।‚{\tiny $_{lb}$}‚ किं कार‚णं \textbf{कार‚णाविशेषे} स‚ति \textbf{कार्य‚स्याविशेषात्} ।
	{\color{gray}{\rmlatinfont\textsuperscript{§~\theparCount}}}
	\pend% ending standard par
      ‚{\tiny $_{lb}$}‚

	  
	  \pstart \leavevmode% starting standard par
	अथ कार‚णाविशेषे कार्य‚स्य विशेष‚स्त‚दा विशेषे कार्य‚स्याप्य‚भ्युप‚ग‚म्य‚माने ।‚{\tiny $_{lb}$}‚ \textbf{त‚स्य विशेष‚स्याहेतुक‚त्व‚प्र‚संगादि}त्युक्त‚प्रायं ।
	{\color{gray}{\rmlatinfont\textsuperscript{§~\theparCount}}}
	\pend% ending standard par
      ‚{\tiny $_{lb}$}‚

	  
	  \pstart \leavevmode% starting standard par
	\textbf{त‚दि}ति त‚स्मा\textbf{दिमे मंत्राः कार्या अप्य}नि‚{\tiny $_{७}$}‚त्या अपि । हेतुकृतात् \textbf{स्व‚भाव‚भे}- \leavevmode\ledsidenote{\textenglish{196b/PSVTa}}‚{\tiny $_{lb}$}‚ दात् \textbf{फ‚ल‚दायिनो}ऽपि \textbf{न शूद्रादिप्र‚योगेप्य‚न्य‚था स्यु}र‚फ‚ल‚दाः स्युः । किं कार‚णं [।]‚{\tiny $_{lb}$}‚ \textbf{शूद्र} इति \textbf{विप्र} इति ब्राह्म‚ण इत्य\textbf{भिधानं} संज्ञा य\textbf{योः पुरुष‚यो}स्त‚योर्बुद्धीन्द्रिय‚देहेषु‚{\tiny $_{lb}$}‚ \textbf{स्व‚भाव‚भेदाभावात्} । प्र‚तिव्य‚क्ति स्व‚ल‚क्ष‚ण‚भेदोस्तीति चेत् [।] न । त‚स्य‚{\tiny $_{lb}$}‚ ब्राह्म‚णेष्व‚पि प्र‚तिव्य‚क्ति स‚म्भ‚वात् । जातिकृत‚स्तु भेदो नास्तीत्युच्य‚ते ।
	{\color{gray}{\rmlatinfont\textsuperscript{§~\theparCount}}}
	\pend% ending standard par
      ‚{\tiny $_{lb}$}‚

	  
	  \pstart \leavevmode% starting standard par
	न‚न्व‚यं ब्राह्म‚णोऽयं‚{\tiny $_{१}$}‚ शूद्र इति लोके नाम‚भेदोस्ति । त‚था स‚म्मानादिव्य‚व‚{\tiny $_{lb}$}‚हार‚भेद‚श्च [।] त‚तः स्व‚भाव‚भेदोव‚सीय‚त इति । चेद् [।]
	{\color{gray}{\rmlatinfont\textsuperscript{§~\theparCount}}}
	\pend% ending standard par
      ‚{\tiny $_{lb}$}‚

	  
	  \pstart \leavevmode% starting standard par
	आह । \textbf{न ही}त्यादि । \textbf{पुरुषेच्छानुरोधिनो} ब्राह्म‚णादि\textbf{ना}म‚भेदात् स‚त्कारा‚{\tiny $_{lb}$}‚दि\textbf{व्य‚व‚हार‚भेदा}च्च \textbf{स्व‚भाव‚भेदानुब‚न्धिनां} य‚था कार‚ण‚मुत्प‚न्नेन स्व‚भाव‚भेदेनानु‚{\tiny $_{lb}$}‚ग‚ताना\textbf{म‚र्थानाम‚न्य‚थात्वं} । न हि स्व‚भाव‚भेदोस्तीति स‚म्ब‚न्धः ।
	{\color{gray}{\rmlatinfont\textsuperscript{§~\theparCount}}}
	\pend% ending standard par
      ‚{\tiny $_{lb}$}‚

	  
	  \pstart \leavevmode% starting standard par
	\textbf{त‚यो}र्ब्राह्म‚ण‚श‚द्र‚योः पुरुष‚{\tiny $_{२}$}‚यो\textbf{र्जातिभेदोस्तीति चेत्} ।
	{\color{gray}{\rmlatinfont\textsuperscript{§~\theparCount}}}
	\pend% ending standard par
      ‚{\tiny $_{lb}$}‚‚{\tiny $_{lb}$}‚\textsuperscript{\textenglish{558/s}}

	  
	  \pstart \leavevmode% starting standard par
	त‚न्नैवं । य‚स्मात् \textbf{स ख‚ल्वि}त्यादि । सामान्य‚निषेधान्निषिद्धैव जातिः [।]‚{\tiny $_{lb}$}‚ केव‚ल‚म‚भ्युप‚ग‚म्योच्य‚ते । \textbf{स ख‚लु} भ‚व‚न्न‚पि जातिभेद‚स्त्रिधा \textbf{इष्टः । आकृतिगुण‚{\tiny $_{lb}$}‚श‚क्तिभेदे} स‚ति । \textbf{ग‚वाश्व‚व‚त्} । ग‚वाश्व‚स्येव ग‚वाश्व‚व‚त् । त‚त्राकृतिभेदः संस्थान‚{\tiny $_{lb}$}‚विशेषः स प्र‚तिव्य‚क्ति भेद‚व‚तीष्व‚पि गोष्व‚नुगामी विद्य‚ते [।] नाश्व‚व्य‚क्तिषु ।‚{\tiny $_{lb}$}‚ गुण‚भेदो\edtext{}{\lemma{भेदो}\Bfootnote{? दः}} क्षीरादीनां‚{\tiny $_{३}$}‚ र‚स‚वीर्य‚विपाकादिभेदेन स च स‚मान‚जातीयासु‚{\tiny $_{lb}$}‚ व्य‚क्तिष्व‚नुगामी दृष्टः । न विजातीयासु । श‚क्तिभेदानुरूप‚कार्य‚साम‚र्थ्य‚ल‚क्ष‚णः ।‚{\tiny $_{lb}$}‚ य‚था ग‚वान्दोहादिसाम‚र्थ्यं नाश्वानां । त‚देवं स‚मान‚जातीय‚व्य‚क्त्य‚नुगामिनामा‚{\tiny $_{lb}$}‚कृत्यादिभेदानामुप‚ल‚म्भात् कामं ग‚वाश्वादिष्व‚स्तु जातिभेदो नैवं ब्राह्म‚णादिषु‚{\tiny $_{lb}$}‚ प्र‚तिनिय‚त आकृत्यादिभेदोस्ति [।] येन जातिभेदः‚{\tiny $_{४}$}‚ क‚ल्प्येत । स‚कृच्च ग‚वादिषु‚{\tiny $_{lb}$}‚ व्युत्प‚न्नो देश‚कालादिभेदे\textbf{प्य‚नुप‚देश}मित्युप‚देश‚म‚न्त‚रेणैनं जातिभेदं \textbf{लोकः प्र‚ति‚{\tiny $_{lb}$}‚प‚द्य‚ते} । अय‚ङ्गौर‚य‚म‚श्व इति । नैवं ब्राह्म‚णादिभेद‚म‚नुप‚देशं प्र‚तिप‚द्य‚ते । \textbf{त‚द्व‚दि}ति‚{\tiny $_{lb}$}‚ ग‚वाश्व‚व‚त् । \textbf{अन‚यो}र्ब्राह्म‚ण‚शूद्र‚योः \textbf{कंचिद‚पि} गुणं विनिय‚तं स‚मान‚जातीयास्वेव‚{\tiny $_{lb}$}‚ व्य‚क्तिषु स्थित\textbf{म्प‚श्यामः} । गुण‚{\tiny $_{५}$}‚ग्र‚ह‚ण‚मुप‚ल‚क्ष‚णं । एव‚माकृतिभेदं श‚क्तिभेदं च‚{\tiny $_{lb}$}‚ विनिय‚तं न प‚श्यामः । \textbf{अप‚श्य‚न्त‚श्चा}कृत्यादि\textbf{भेदं क‚थं} शूद्र‚विप्र‚योर्जातिभेदं‚{\tiny $_{lb}$}‚ \textbf{प्र‚तिप‚द्येम‚हि} । नैवेति याव‚त् । त‚था ह्य‚ध्य‚य‚न‚शौचाचारादिविशेषः स‚र्वो व्य‚भि‚{\tiny $_{lb}$}‚चारी । य‚श्च गौर‚पिङ्ग‚ल‚केश‚त्वादिल‚क्ष‚ण आकार‚भेदः क‚ल्प्य‚ते स ब्राह्म‚णेष्व‚पि‚{\tiny $_{lb}$}‚ केषुचिन्नास्ति । शूद्रेषु च विद्य‚ते केषुचि‚{\tiny $_{६}$}‚त् ।
	{\color{gray}{\rmlatinfont\textsuperscript{§~\theparCount}}}
	\pend% ending standard par
      ‚{\tiny $_{lb}$}‚

	  
	  \pstart \leavevmode% starting standard par
	य‚दि न जातिभेदः विप्र‚शूद्र‚योः क‚थ‚न्त‚र्ह्य‚यं ब्राह्म‚णादिश‚ब्द‚श्र‚व‚णाद् भिन्ना‚{\tiny $_{lb}$}‚ प्र‚तीतिर्भ‚व‚तीति [।]
	{\color{gray}{\rmlatinfont\textsuperscript{§~\theparCount}}}
	\pend% ending standard par
      ‚{\tiny $_{lb}$}‚

	  
	  \pstart \leavevmode% starting standard par
	आह । \textbf{योपी}त्यादि । \textbf{नाम‚भेदान्व‚यो} ब्राह्म‚णादिसंज्ञाविशेष‚हेतुको \textbf{योप्य‚यं‚{\tiny $_{lb}$}‚ प्र‚तीतिभेदो} \add{बुद्धिभेदो अयं ब्राह्म‚णोऽयं शूद्र}\edtext{\textsuperscript{*}}{\edlabel{pvsvt_558-1}\label{pvsvt_558-1}\lemma{*}\Bfootnote{In the margin. }} इति प्र‚तीतिभेदो\textbf{स‚त्य‚पि जाति‚{\tiny $_{lb}$}‚\leavevmode\ledsidenote{\textenglish{197a/PSVTa}} भेदे व्यापार‚विशेषानुष्ठाना}ज्ज‚प‚होमादिक्रियाविशेषानुष्ठानात् \textbf{स्यात् । अन्व‚{\tiny $_{७}$}‚‚{\tiny $_{lb}$}‚याच्चे}ति । त‚थाभूत‚व्यापारानुष्ठायिनः कुलादुत्प‚त्तेश्च । \textbf{वैद्य‚व‚णिग्व्य‚प‚देशा‚{\tiny $_{lb}$}‚दिव‚त्} । य‚था तुल्य‚जातीयेषु शुद्रेषु त‚स्य त‚स्य व्यापार‚विशेष‚स्यानुष्ठानाद‚{\tiny $_{lb}$}‚न्व‚याच्च वैद्यादिव्य‚प‚देशाः प्र‚व‚र्त्त‚न्ते [।] न ताव‚ता जातिभेदः । त‚द्व‚त् ब्रा\edtext{}{\edlabel{pvsvt_558-1b}\label{pvsvt_558-1b}\lemma{ब्रा}\Bfootnote{१ In the margin.}}‚{\tiny $_{lb}$}‚ह्म‚णादिष्व‚पि स्यात् ।
	{\color{gray}{\rmlatinfont\textsuperscript{§~\theparCount}}}
	\pend% ending standard par
      ‚{\tiny $_{lb}$}‚‚{\tiny $_{lb}$}‚‚{\tiny $_{lb}$}‚\textsuperscript{\textenglish{559/s}}

	  
	  \pstart \leavevmode% starting standard par
	\textbf{त‚दिति} त‚स्मा\textbf{दिमे} म‚न्त्रा \textbf{अविशिष्टेन} ब्राह्म‚ण‚शूद्रादिना \textbf{प्र‚युज्य‚मानास्त‚तो}‚{\tiny $_{lb}$}‚ ब्राह्म‚णादे\textbf{र‚विशिष्ट‚{\tiny $_{१}$}‚मेव स्व‚भाव‚मासाद‚य‚न्ति । तेन} कार‚णे\textbf{नाविशेषेणैव} शूद्रादिषु‚{\tiny $_{lb}$}‚ \textbf{फ‚ल‚दाः स्यु}रित्युप‚संहारः । त‚देवं म‚न्त्राणाम्भाव‚श‚क्त्या फ‚लोत्पाद‚नेनायं दोषो‚{\tiny $_{lb}$}‚ \textbf{य‚दा तु} य‚थोक्तात् पुरुष\textbf{स‚म‚यादेभ्यो} वैदिकेभ्योन्येभ्यो वा म‚न्त्रेभ्यः \textbf{फ‚ल}मिष्य‚ते ।‚{\tiny $_{lb}$}‚ \textbf{त‚दाय}म‚विशेषेण फ‚ल‚दाः स्युरि\textbf{त्य‚य‚म‚दोषः} । किं कार‚णं [।] \textbf{स‚म‚य‚कार‚स्य} म‚न्त्र‚{\tiny $_{lb}$}‚प्र‚णेतू \textbf{रुचेः फ‚लोत्प‚त्ति‚{\tiny $_{२}$}‚निय‚मात्} । स‚म‚य‚कार‚स्यैव‚म‚भिरुचितं य ईदृशो ब्राह्म‚ण‚{\tiny $_{lb}$}‚व्य‚प‚देश‚भागेव‚म्विधं निय‚म‚म‚नुतिष्ठ‚ति त‚स्यैवाहं फ‚ल‚योगेन प्र‚त्युप‚स्थितो नान्येभ्य‚{\tiny $_{lb}$}‚ इत्येवं रुचेः फ‚ल‚नियाम‚कः । प‚र्य‚नुयोगः । त‚था हि [।] \textbf{स्व‚भाव‚वृत्त‚यो} भावाः‚{\tiny $_{lb}$}‚ पुरुष‚व्यापारान‚पेक्षा व‚स्तुस्थित्यैव कार्य‚कारिण इत्य‚र्थः । ते \textbf{त‚न्मुखेन} स्व‚भाव‚द्वारेण‚{\tiny $_{lb}$}‚ \textbf{प्र‚संग‚म‚र्ह‚न्ति‚{\tiny $_{३}$}‚} । य‚था स्व‚भाव‚विशेषादिहाप्येवं किन्न भ‚व‚तीति । \textbf{न पुरुषेच्छा‚{\tiny $_{lb}$}‚ वृत्त‚यो} भावाः प्र‚संग‚म‚र्ह‚न्ति । किङ्कार‚णं [।] \textbf{तेषां} पुरुषाणां स्वेच्छानुविधायिनां‚{\tiny $_{lb}$}‚ य‚था\textbf{क‚थंचिद् वृत्तेः} ।
	{\color{gray}{\rmlatinfont\textsuperscript{§~\theparCount}}}
	\pend% ending standard par
      ‚{\tiny $_{lb}$}‚

	  
	  \pstart \leavevmode% starting standard par
	\textbf{य‚द‚पि} विशिष्टः \textbf{प्र‚योक्ता} म‚न्त्र‚फ\textbf{ल‚श्म‚नुत इ}त्युच्य‚ते । त‚त्रापि \textbf{प्र‚योग}मेवं‚{\tiny $_{lb}$}‚ ल‚क्ष‚णं \textbf{प‚श्यामः} । य‚था \textbf{स‚मीहिते} पुरुषा\textbf{र्थे योग्यो} यो म‚न्त्र‚स्व‚भाव‚स्त\textbf{स्योत्पाद‚नं} ।‚{\tiny $_{lb}$}‚ उत्प‚न्न‚स्याप्यु‚{\tiny $_{४}$}‚त्त‚रोत्त‚र‚विशेषोत्पाद‚नेन \textbf{स‚न्तान‚प‚रिणाम}न‚म‚न्य‚थात्व‚म्वा \textbf{त‚दुभ‚य‚मु}‚{\tiny $_{lb}$}‚त्पाद‚न‚म्विप‚रिण‚म‚नं वा \textbf{विशेष‚ज‚न्म‚नि} विशेषोत्प‚त्तौ स‚त्यां \textbf{स्यात् । अन्य‚था}‚{\tiny $_{lb}$}‚ विशेषानुत्प‚त्ताव\textbf{नाधेयातिश‚यानां} म‚न्त्राणां । \textbf{किं कुर्वाणः प्र‚योज‚कः} [।] नैव ।
	{\color{gray}{\rmlatinfont\textsuperscript{§~\theparCount}}}
	\pend% ending standard par
      ‚{\tiny $_{lb}$}‚

	  
	  \pstart \leavevmode% starting standard par
	\textbf{येने}त्यादिना व्याच‚ष्टे । \textbf{येन} कार‚णेन \textbf{त‚तो} म‚न्त्र‚प्र‚योगात् क‚श्चित् प्र‚योक्ता‚{\tiny $_{lb}$}‚ ब्राह्म‚णः \textbf{फ‚ल‚म‚श्नुतेऽन्यः} शू‚{\tiny $_{५}$}‚द्रादि\textbf{र्ना}श्नुते । नोत्पाद‚न‚म्म‚न्त्राणां \textbf{प्र‚योगः} किन्त्व\textbf{भिव्य‚{\tiny $_{lb}$}‚क्तिः} प्र‚योगो य‚दीष्य‚ते । \textbf{साभि}व्य‚क्तिः \textbf{प्र‚गेव} सा मा न्य चि न्ता यां\edtext{}{\edlabel{pvsvt_559-1}\label{pvsvt_559-1}\lemma{यां}\Bfootnote{In the margin. }} \textbf{निराकृता} ।
	{\color{gray}{\rmlatinfont\textsuperscript{§~\theparCount}}}
	\pend% ending standard par
      ‚{\tiny $_{lb}$}‚‚{\tiny $_{lb}$}‚‚{\tiny $_{lb}$}‚\textsuperscript{\textenglish{560/s}}

	  
	  \pstart \leavevmode% starting standard par
	त‚द्व्याच‚ष्टे । \textbf{न हि नित्याना}म्प‚दार्थानां \textbf{काचिद‚भिव्य‚क्तिरित्युक्तं । य‚तः}‚{\tiny $_{lb}$}‚ कार‚णा\textbf{दाभिव्य‚ञ्ज‚कः} श‚ब्द‚स्य \textbf{प्र‚योक्ता स्यात्} ।
	{\color{gray}{\rmlatinfont\textsuperscript{§~\theparCount}}}
	\pend% ending standard par
      ‚{\tiny $_{lb}$}‚

	  
	  \pstart \leavevmode% starting standard par
	भ‚व‚तु वाभिव्य‚क्तिः [।] सा च ताव‚द् योग्य‚तोत्प‚त्तिनित्य‚त्वात् । किन्तु‚{\tiny $_{lb}$}‚ \textbf{व्य‚क्तिश्च} श‚{\tiny $_{६}$}‚ब्द‚विष‚या \textbf{बुद्धि सा य‚स्मात्} पुरुषात् प्र‚योक्तुः \textbf{स} पुरुषो म‚न्त्र\textbf{फ‚लैर्य‚दि‚{\tiny $_{lb}$}‚ युज्य‚ते । स्याच्छेतु}र‚पि \textbf{फ‚ल‚स‚म्ब‚न्धः} । योन्येन प‚ठ्य‚मानं म‚न्त्रं श्रृणोति केव‚लं ।‚{\tiny $_{lb}$}‚ त‚स्यापि म‚न्त्र‚फ‚लेन योगः स्यान्न तु व‚क्तुरेव । य‚स्मात् \textbf{व‚क्ता हि व्य‚क्तिकार‚णं}‚{\tiny $_{lb}$}‚ ज्ञान‚कार‚ण‚म‚त‚श्च फ‚लेन प्र‚युज्य‚ते । त‚च्च म‚न्त्र‚विष‚य‚ज्ञान‚कार‚ण‚त्वं श्रोतुर‚पि‚{\tiny $_{lb}$}‚ \leavevmode\ledsidenote{\textenglish{197b/PSVTa}} तुल्य‚{\tiny $_{७}$}‚मिति सोपि व‚क्तैवेति क‚स्मान्न फ‚लेन युज्य‚ते ।
	{\color{gray}{\rmlatinfont\textsuperscript{§~\theparCount}}}
	\pend% ending standard par
      ‚{\tiny $_{lb}$}‚

	  
	  \pstart \leavevmode% starting standard par
	\textbf{न ही}त्यादिना व्याच‚ष्टे । \textbf{न हि} नित्य‚स्य \textbf{श‚ब्द‚स्यान्य‚तः} कार‚णात् \textbf{स्व‚रूप‚प‚रि‚{\tiny $_{lb}$}‚णामः} स्व‚रूपान्य‚थात्व‚म्व्य‚क्तिः । \textbf{नाप्याव‚र‚ण‚विग‚म‚नं व्य‚क्तिः} । नित्य‚स्याव‚र‚णा‚{\tiny $_{lb}$}‚नुप‚प‚त्तेः । \textbf{किन्तु त‚द्विष‚या} श‚ब्द‚विष‚या \textbf{प्र‚तीति}र्बुद्धिर‚भिव्य‚क्तिः । किं कार‚ण‚म् [।]‚{\tiny $_{lb}$}‚ \textbf{अश्रूय‚माणे} श‚ब्दे\textbf{ऽव्य‚क्त‚व्य‚प‚देशात्} । न ह्य‚नुप‚ल‚{\tiny $_{१}$}‚भ्य‚मानः श‚ब्दोभिव्य‚क्त इत्युच्य‚ते ।‚{\tiny $_{lb}$}‚ \textbf{त‚त्रैवं} व्य‚व‚स्थिते \textbf{य‚दि} श‚ब्द‚विष‚य\textbf{बुद्धिहेतुर्व‚क्ता स्यात्} । त‚दा त‚च्छ‚ब्द‚विष‚य‚{\tiny $_{lb}$}‚बुद्धिहेतुस्व‚भाव‚व‚क्त्तृल‚क्ष‚णं \textbf{श्रोत‚र्य‚प्य}स्तीति \textbf{सोपि} श्रोता म‚न्त्र‚प्र‚योग‚स‚म्भ‚व‚म‚{\tiny $_{lb}$}‚भिम‚तं \textbf{फ‚ल‚म्व‚क्तृव‚द‚श्नुवीत । न हि व‚क्तुः क‚श्चिद‚न्य‚स्त‚द्भावो} व‚क्तृत्व‚भावो\textbf{न्य‚त्र‚{\tiny $_{lb}$}‚ त‚द्बुद्धिहेतुत्वात्} । श‚ब्द‚बुद्धिहेतुत्वात् । अतो नास्ति व‚क्तृश्रोत्रोः श‚ब्द‚{\tiny $_{२}$}‚ज्ञान‚हेतु‚{\tiny $_{lb}$}‚त्वे विशेष इति तुल्यः फ‚ल‚स‚म्ब‚न्धः स्यात् । प‚रो व‚क्ता । उपाधिर्हेतुर्य‚स्या बुद्धेः ।‚{\tiny $_{lb}$}‚ सा \textbf{प‚रोपाधिबुद्धिः श्रोतुर्न व‚क्तुः} प‚रोपाधिर्बुद्धिः । \textbf{इति} हेतोर्व‚क्तृश्रोत्रो\textbf{र्विशेष‚{\tiny $_{lb}$}‚ इति चेत् । कः पुन‚रुप‚योगः} साम‚र्थ्य\textbf{म्व‚क्तुः श्रोत‚रि} बुद्धिज‚न‚न‚म्प्र‚ति । \textbf{येनो}प‚यो‚{\tiny $_{lb}$}‚\edtext{}{\lemma{यो}\Bfootnote{In the margin. }}‚{\tiny $_{lb}$}‚ ‚{\tiny $_{lb}$}‚ \leavevmode\ledsidenote{\textenglish{561/s}}गेनो\textbf{पाधिरिष्य‚ते} व‚क्ता । \textbf{त‚तो} व‚क्तुः स‚काशा\textbf{च्छ‚ब्द‚श्रुतिः} श‚ब्दोप‚ल‚ब्धिः श्रोतुर्भ‚{\tiny $_{lb}$}‚व‚त्य‚तो‚{\tiny $_{३}$}‚सौ व‚क्ता उपाधिरि\textbf{ति चेत्} ।
	{\color{gray}{\rmlatinfont\textsuperscript{§~\theparCount}}}
	\pend% ending standard par
      ‚{\tiny $_{lb}$}‚

	  
	  \pstart \leavevmode% starting standard par
	\textbf{न‚नु त‚देवेदं प‚र्य‚नुयुज्य‚ते क‚थ‚न्त‚तो} व‚क्तुः स‚काशाच्छ‚ब्द‚श्रुतिः श्रोतुर्भ‚व‚ति ।‚{\tiny $_{lb}$}‚ क‚थं च त‚तो न भ‚वेत् । \textbf{स‚म्ब‚न्धाभावात्} । उप‚कार्योप‚कार‚क‚भावाभावात् । त‚द‚भा‚{\tiny $_{lb}$}‚व‚मेव द‚र्श‚य‚न्नाह । \textbf{विष‚योप‚न‚याद्} विष‚य‚स‚न्निधाप‚ना\textbf{द‚य‚म्}व‚क्त्\textbf{आस्य} श्रोतुः \textbf{श्राव‚कः‚{\tiny $_{lb}$}‚ स्यात्} । श्राव‚यिता भ‚वेत् । \textbf{त‚च्च} प्र‚त्युप‚स्थाप‚नं \textbf{न श‚क्य‚न्त‚स्य} श‚{\tiny $_{४}$}‚ब्द‚स्य । नित्य‚स्य‚{\tiny $_{lb}$}‚ \textbf{क‚थंचिद‚प्य‚प‚रिणामात्} । अन्य‚थात्वाभावात् ।
	{\color{gray}{\rmlatinfont\textsuperscript{§~\theparCount}}}
	\pend% ending standard par
      ‚{\tiny $_{lb}$}‚

	  
	  \pstart \leavevmode% starting standard par
	श्रोतुरिन्द्रिय‚संस्कार‚ङ्कुर्व‚न्नाव‚र‚ण‚विग‚म‚नं वा श‚ब्द‚स्य स‚म्पाद‚य‚न् ब्र‚जेदुप‚यो‚{\tiny $_{lb}$}‚ग‚म्व‚क्तेति चेत् [।]
	{\color{gray}{\rmlatinfont\textsuperscript{§~\theparCount}}}
	\pend% ending standard par
      ‚{\tiny $_{lb}$}‚

	  
	  \pstart \leavevmode% starting standard par
	\textbf{त‚न्न} । य‚स्मा\textbf{दिन्द्रिय‚संस्काराद‚योप्युक्ताः} प्र‚तिक्षिप्ताः । इन्द्रिय‚स्य स्यात्‚{\tiny $_{lb}$}‚ \textbf{संकारः} शृणुयान्निखिल‚न्त‚दि \edtext{\textsuperscript{*}}{\edlabel{pvsvt_561-2}\label{pvsvt_561-2}\lemma{*}\Bfootnote{\href{http://sarit.indology.info/?cref=pv.3.255}{प्र० वा० १ । २५८}}}त्यादिना । माम‚य‚म्व‚क्ता श्राव‚य‚तीति श्रोतुः प्र‚त्य‚यो‚{\tiny $_{lb}$}‚ बुद्धि\textbf{र‚ह‚मेनं} श्रोतारं \textbf{श्राव‚{\tiny $_{५}$}‚यामीति} व‚क्तुः संप्र‚त्य‚यो भ‚व‚ति [।] अतः प्र‚त्य‚य‚{\tiny $_{lb}$}‚द्व‚याद् य‚थायोग\textbf{म्व‚क्तृश्रोत्रोर्भेद इति चेत्} ।
	{\color{gray}{\rmlatinfont\textsuperscript{§~\theparCount}}}
	\pend% ending standard par
      ‚{\tiny $_{lb}$}‚

	  
	  \pstart \leavevmode% starting standard par
	\textbf{अनुप‚कार्ये}त्यादिना प्र‚तिषेध‚ति । मां श्राव‚य‚त्य‚हं श्राव‚यामीति \textbf{भ्रान्तिमात्र‚{\tiny $_{lb}$}‚मेत‚त्} । किम्भूत\textbf{म‚नुप‚कार्योप‚कार‚कं} । उप‚कार्यः श्रोता । उप‚कार‚को व‚क्ता न भ‚व‚ति‚{\tiny $_{lb}$}‚ य‚स्मिन् भ्रान्तिमात्रे त‚त्त‚थोक्तं । नित्ये च श‚ब्दे बुद्धिज‚न्म‚नि पुंसः स‚र्व‚था व्या‚{\tiny $_{६}$}‚‚{\tiny $_{lb}$}‚पाराभावाद‚नुप‚कार्योप‚कार‚क‚भावः प्र‚तिपादितः । \textbf{त‚स्मा}देवंभूताद् भ्रान्तिमात्रा\textbf{त्त‚{\tiny $_{lb}$}‚द्भावे} । व‚क्तृश्रोतृभेद‚भावे\textbf{ऽतिप्र‚स‚ङ्गः} । स‚र्व‚स्याश्राव‚यितुर‚शृण्व‚त‚श्चैवं \textbf{स्यात्} [।]‚{\tiny $_{lb}$}‚ किङ्कार‚णं । \textbf{अ}\add{\textbf{न्य‚त्रा}प्युन्म‚त्तादौ विनैव श‚ब्द}\edtext{\textsuperscript{*}}{\edlabel{pvsvt_561-3}\label{pvsvt_561-3}\lemma{*}\Bfootnote{In the margin. }} श्र‚व‚णेन \textbf{भ्रान्त्या}ऽहं शृणोमीत्यादि‚{\tiny $_{lb}$}‚\textbf{प्र‚त्य‚य‚द‚र्श‚नात्} । त‚स्मात् \textbf{स‚र्व‚थोप‚काराभावे च त‚थाप्र‚त्य‚य} इत्य‚हं श्राव‚{\tiny $_{७}$}‚यामीत्यादि \leavevmode\ledsidenote{\textenglish{198a/PSVTa}}‚{\tiny $_{lb}$}‚ प्र‚त्य‚यो \textbf{न युक्तः} । किं कार‚णं [।] \textbf{स‚र्वेषां प‚र‚स्प‚र‚म}नुप‚कार्योप‚कार‚काणा\textbf{मेव‚म}ह‚म‚तः‚{\tiny $_{lb}$}‚ शृणोमीत्यादि \textbf{प्र‚स‚ङ्गात्} । य‚था त‚र्ह्युन्म‚त्तेषु भ्रान्त्या प्र‚त्य‚योत्प‚त्तिस्त‚था नित्येष्व‚पि‚{\tiny $_{lb}$}‚ भ‚व‚त्वि\add{ति चेदाह । \textbf{भ्रान्तिरि}त्यादि । \textbf{भ्रान्ति}}\edtext{}{\edlabel{pvsvt_561-3b}\label{pvsvt_561-3b}\lemma{त्वि}\Bfootnote{३ In the margin.}} र‚पि या पुरुष‚स्योन्म‚त्त‚स्य भ‚व‚ति‚{\tiny $_{lb}$}‚ सापि स्व‚स्थाव‚स्थायां \textbf{कुत‚श्चित् पुरुषादुप‚कारे स‚ति} त‚था प्र‚त्य‚योत्प‚त्तौ त‚दाहि‚{\tiny $_{१}$}‚त‚{\tiny $_{lb}$}‚  ‚{\tiny $_{lb}$}‚ ‚{\tiny $_{lb}$}‚ ‚{\tiny $_{lb}$}‚ \leavevmode\ledsidenote{\textenglish{562/s}}संस्कार‚व‚शेन पुन‚रुन्म‚त्ताव‚स्थाया\textbf{म‚न्य‚त्राप्य}व‚क्त‚र्य‚पि । \textbf{क‚याचित् प्र‚त्यास‚त्त्या} केन‚{\tiny $_{lb}$}‚चित्सादृश्येन भ‚व‚ति । \textbf{सापि} भ्रान्तिर्नित्येषु म‚न्त्रेष्व‚त्य‚न्तं \textbf{पार‚म्प‚र्येणा}प्य‚नुप‚कारे‚{\tiny $_{lb}$}‚ \textbf{न स्यात्} ।
	{\color{gray}{\rmlatinfont\textsuperscript{§~\theparCount}}}
	\pend% ending standard par
      ‚{\tiny $_{lb}$}‚

	  
	  \pstart \leavevmode% starting standard par
	\textbf{त‚स्मा}दित्यादिना प्र‚कृत‚मुप‚संह‚र‚ति । य‚थोक्त‚विधिना \textbf{व‚क्तृश्रोत्रोर्व्य‚क्तिहेतुत्वे} ।‚{\tiny $_{lb}$}‚ श‚ब्द‚ज्ञान‚हेतुत्वे \textbf{विशेषाभावात्तुल्यो} म‚न्त्र\textbf{फ‚लेन स‚म्ब‚न्धः स्यात्} । नित्येषु व्य‚{\tiny $_{२}$}‚क्ति‚{\tiny $_{lb}$}‚हेतुत्व‚म‚पि नैवास्ति [।] केव‚ल‚म‚भ्युप‚ग‚म्यैव‚मुच्य‚ते ।
	{\color{gray}{\rmlatinfont\textsuperscript{§~\theparCount}}}
	\pend% ending standard par
      ‚{\tiny $_{lb}$}‚

	  
	  \pstart \leavevmode% starting standard par
	\textbf{अपि चे}त्यादिना दूष‚णान्त‚र‚माह । \textbf{अन‚भिव्य‚क्तः}नश्रोत्र‚विष‚य‚न्नीतः \textbf{श‚ब्दो‚{\tiny $_{lb}$}‚ यैः} क‚र‚णैस्ताल्वादिभिस्तान्य‚न‚भिव्य‚क्त‚श‚ब्दानि । तेषां \textbf{क‚र‚णानां प्र‚योज‚नं} व्या‚{\tiny $_{lb}$}‚पार‚णं व्य‚र्थं स्यादिति लिङ्ग‚विप‚रिणामेन स‚म्ब‚न्धः । य‚त्रौष्ठादिप्र‚स्प‚न्द‚मात्रेण‚{\tiny $_{lb}$}‚ उपांशुज‚पः क्रिय‚ते । स \textbf{व्य‚र्थः स्यादिति} वाक्या‚{\tiny $_{३}$}‚र्थः । य‚त्रौष्ठादिप्र‚स्प‚न्दोपि नास्ति‚{\tiny $_{lb}$}‚ केव‚लं म‚न‚सा म‚न्त्र‚चिन्त‚नं स \textbf{म‚नोज‚पः । वा} श‚ब्दः स‚मुच्च‚ये । म‚नोज‚प‚श्च व्य‚र्थः‚{\tiny $_{lb}$}‚ स्यात् । य‚स्मा\textbf{च्छ‚ब्दो हि श्रोत्र‚गोच‚रः} श्रोत्र‚ग्राह्य एव श‚ब्दः [।] श‚ब्द‚स्व‚भाव‚श्च‚{\tiny $_{lb}$}‚ म‚न्त्रः । उपांशुमान‚स‚योश्च ज‚प‚योः श्रोत्र‚ग्र‚ह‚णाभावाद‚श‚ब्द‚त्व‚म् [।] अश‚ब्द‚त्वा‚{\tiny $_{lb}$}‚च्चाम‚न्त्र‚त्वं च ज‚प‚न् क‚थं फ‚ल‚वान् स्यात् । गृह्य‚त इति ग्र‚ह‚णं । \textbf{श्रो‚{\tiny $_{४}$}‚त्र‚ग्राह्य}‚{\tiny $_{lb}$}‚ इत्य‚र्थः । त‚देव ल‚क्ष‚णं य‚स्य श‚ब्द‚स्य स त‚थोक्तः । त‚द‚ति\textbf{क्र‚मेतिप्र‚संगात्} ।‚{\tiny $_{lb}$}‚ श्रोत्र‚ग्राह्यं श‚ब्दं मुक्त्वा म‚नोज‚पादेर्ज्ञानात्म‚क‚स्य श‚ब्द‚त्व इष्य‚माणे र‚सादीनाम‚पि‚{\tiny $_{lb}$}‚ श‚ब्द‚त्वं स्यात् ।
	{\color{gray}{\rmlatinfont\textsuperscript{§~\theparCount}}}
	\pend% ending standard par
      ‚{\tiny $_{lb}$}‚

	  
	  \pstart \leavevmode% starting standard par
	\textbf{न‚न्वेम‚पि} श्रोत्र‚गोच‚र‚स्य श‚ब्द‚स्व‚भाव‚त्वे श‚ब्द‚त्व\textbf{सामान्ये} श‚ब्द‚स्व‚भाव‚ता‚{\tiny $_{lb}$}‚\textbf{प्र‚स‚ङ्गः} ।
	{\color{gray}{\rmlatinfont\textsuperscript{§~\theparCount}}}
	\pend% ending standard par
      ‚{\tiny $_{lb}$}‚

	  
	  \pstart \leavevmode% starting standard par
	\textbf{ने}ति सि द्धा न्त वा दी । \textbf{न ब्रूमः} श्रोत्र‚गोच‚रः \textbf{श‚ब्द एवेति । श‚ब्द‚स्त्व‚{\tiny $_{५}$}‚व‚श्य‚{\tiny $_{lb}$}‚न्त‚ल्ल‚क्ष‚णः} श्रोत्र‚ग्र‚ह‚ण‚ल‚क्ष‚ण इति ब्रूमः । क‚स्मात् । \textbf{त‚स्य} श‚ब्द‚स्य श्रोत्र‚ग्राह्य‚त्व‚{\tiny $_{lb}$}‚‚{\tiny $_{lb}$}‚ \leavevmode\ledsidenote{\textenglish{563/s}}म्मुक्त्वा ल\textbf{क्ष‚णान्त‚राभावात् । त‚त्रै}त‚स्मिन् श‚ब्द‚स्व‚ल‚क्ष‚णे \textbf{य‚दि श‚ब्दात्म‚नां} श‚ब्द‚{\tiny $_{lb}$}‚स्व‚भावाना\textbf{म्म‚न्त्राणाम‚भिव्य‚क्तिहेतुः} श‚ब्द‚स्व‚रूप‚ग्राहिज्ञान‚हेतुः पुरुषः \textbf{प्र‚योक्ते}‚{\tiny $_{lb}$}‚ष्य‚ते य‚स्य फ‚लेन स‚म्ब‚न्धः । त‚दा\textbf{न‚भिव्य‚क्त‚श्रुतिविष‚याणां} । श्रुतिविष‚यः श‚ब्दः‚{\tiny $_{६}$}‚‚{\tiny $_{lb}$}‚ सोन‚भिव्य‚क्तो यैः कार‚णैरिति विग्र‚हः । तेषां कार‚णानां \textbf{प्र‚योक्ता} । ओष्ठादिस्प‚{\tiny $_{lb}$}‚न्द‚मात्रेण व्यापार‚यिता । \textbf{उपांशुजापी न म‚न्त्र‚फ‚लेन युज्य‚ते} नापि म‚न‚सा ज‚प‚न्‚{\tiny $_{lb}$}‚ म‚न्त्र‚फ‚लेन युज्येत । य‚स्मान्न \textbf{हि त‚दा} उपांशुमौन‚ज‚प‚काले \textbf{श्रोत्रेण कंचिद‚र्थं}‚{\tiny $_{lb}$}‚ श‚ब्दाख्य\textbf{म्विभाव‚यामो} गृह् णीमः ।
	{\color{gray}{\rmlatinfont\textsuperscript{§~\theparCount}}}
	\pend% ending standard par
      ‚{\tiny $_{lb}$}‚

	  
	  \pstart \leavevmode% starting standard par
	मान‚सोपि ज‚पो म‚न्त्र इत्याह । \textbf{न च} मान‚{\tiny $_{७}$}‚सो विक‚ल्पो\textbf{ऽश‚ब्दात्मा म‚न्त्रः} । \leavevmode\ledsidenote{\textenglish{198b/PSVTa}}‚{\tiny $_{lb}$}‚ श‚ब्द‚स्य श्रोत्र‚ग्राह्य‚त्वापौरुष‚य‚त्व‚नित्य‚त्वेनाभ्युप‚ग‚मात् । त‚द्विप‚रीत‚त्वाच्च विक‚{\tiny $_{lb}$}‚ल्प‚स्य ।
	{\color{gray}{\rmlatinfont\textsuperscript{§~\theparCount}}}
	\pend% ending standard par
      ‚{\tiny $_{lb}$}‚

	  
	  \pstart \leavevmode% starting standard par
	\textbf{पार‚म्प‚र्ये}त्यादिना प‚राभिप्राय‚माशंक‚ते । याप्युपांशुम‚नोजाप‚काले म‚न्त्रा‚{\tiny $_{lb}$}‚भासा \textbf{म‚ति}र्बुद्धि\textbf{स्सापि त‚द्व्य‚क्तिः} । त‚स्य श‚ब्द‚स्य व्य‚क्तिज्ञानं । क‚स्मात् [।]‚{\tiny $_{lb}$}‚ पार‚म्प‚र्येण \textbf{त‚ज्ज‚त्वात्} । श‚ब्द‚जात‚त्वात् । त‚था हि त‚स्य प्राक् श‚ब्द‚ज्ञान‚मुत्प‚न्न‚{\tiny $_{lb}$}‚न्तेन ज्ञानेन चाहित‚सं‚{\tiny $_{१}$}‚स्कार‚स्य क्र‚मेण म‚नोज‚पे श‚ब्द‚प्र‚तिभासोत्प‚त्तेः ।
	{\color{gray}{\rmlatinfont\textsuperscript{§~\theparCount}}}
	\pend% ending standard par
      ‚{\tiny $_{lb}$}‚

	  
	  \pstart \leavevmode% starting standard par
	\textbf{नेत्या}दिना व्याच‚ष्टे । \textbf{न हि म‚न‚सा ध्याय‚तोपि} ज‚पिनो या \textbf{म‚न्त्राभासा बुद्धिः}‚{\tiny $_{lb}$}‚ सा \textbf{श्र‚व‚णादृते} । श‚ब्द‚श्र‚व‚णं विना । \textbf{त‚तः श‚ब्द‚प्र‚भ‚वा}च्छ‚ब्दादुत्प‚त्ते\textbf{स्सापि} म‚नो‚{\tiny $_{lb}$}‚ज‚प‚काले म‚तिः \textbf{श‚ब्द‚व्य‚क्तिरेव} श‚ब्द‚ज्ञान‚मेव । त‚त‚श्च म‚न्त्र‚स्व‚भाव‚त्वात् म‚नो‚{\tiny $_{lb}$}‚ज‚पादेः प्र‚योक्ता फ‚ल‚वान् स्यादिति भावः । \textbf{एव‚मि}‚{\tiny $_{२}$}‚ति पार‚म्प‚र्येण म‚न्त्र‚त्वेऽ\textbf{न‚{\tiny $_{lb}$}‚व‚स्था स्यात्} ।
	{\color{gray}{\rmlatinfont\textsuperscript{§~\theparCount}}}
	\pend% ending standard par
      ‚{\tiny $_{lb}$}‚

	  
	  \pstart \leavevmode% starting standard par
	त‚मेवाह \textbf{श‚ब्दे}त्यादि । म‚न्त्र‚ल‚क्ष‚णाच्छ‚ब्दाद‚र्थ‚विक‚ल्पाः \textbf{श‚ब्दार्थ‚विक‚ल्पाः} ।‚{\tiny $_{lb}$}‚ त‚था हि [।] अग्न‚ये स्वाहा [।] इत्युक्ते । अग्निर्ज्व‚ल‚द्भासुरादिरूप इत्यादि‚{\tiny $_{lb}$}‚विक‚ल्पाः क‚दाचिदुत्प‚द्य‚न्ते । \textbf{तेषाम‚पि प‚र‚म्प‚र}या श‚ब्द‚प्र‚सूतिर‚स्तीति \textbf{तेपि} विक‚{\tiny $_{lb}$}‚ल्पास्त\textbf{था स्यु}र्म‚न्त्र‚व्य‚क्तिल‚क्ष‚णाः प्र‚योगाः स्युः । श‚ब्द‚प्र‚भ‚वापि स‚ती या \textbf{त‚द‚र्था‚{\tiny $_{३}$}‚‚{\tiny $_{lb}$}‚‚{\tiny $_{lb}$}‚ \leavevmode\ledsidenote{\textenglish{564/s}}स} श‚ब्दः अर्थः विष‚यो य‚स्या इति विग्र‚हः । सैव \textbf{चेन्म}तिर्म‚न्त्र‚व्य‚क्तिर्न चार्थ‚विक‚ल्पा‚{\tiny $_{lb}$}‚ ये श‚ब्द‚विष‚याः । \textbf{असिद्ध}मिति सि द्धा न्त वा दी । श‚ब्द‚विक‚ल्प‚स्यासिद्धं श‚ब्द‚स्व‚{\tiny $_{lb}$}‚ल‚क्ष‚ण‚विष‚य‚त्वं । किं कार‚णं [।] \textbf{क‚ल्प‚नान्व‚यात्} । स‚जातीय‚विक‚ल्प‚हेतुक‚त्वात् ।‚{\tiny $_{lb}$}‚ अध्यारोपिताकार‚विष‚या एव म‚नोविक‚ल्पाः ।
	{\color{gray}{\rmlatinfont\textsuperscript{§~\theparCount}}}
	\pend% ending standard par
      ‚{\tiny $_{lb}$}‚

	  
	  \pstart \leavevmode% starting standard par
	\textbf{ने}त्यादिना व्याच‚ष्टे । \textbf{न ब्रूमः स‚र्वा श‚ब्द‚प्र‚भ‚वा‚{\tiny $_{४}$}‚ बुद्धिस्त‚द्व्य‚क्तिः} श‚ब्द‚{\tiny $_{lb}$}‚व्य‚क्ति\textbf{रिति} [।] किन्तु \textbf{या त‚द्विष‚या} श‚ब्द‚विष‚या विक‚ल्प‚बुद्धिः । \textbf{सा त‚स्य} श‚ब्द‚स्य‚{\tiny $_{lb}$}‚ \textbf{व्य‚क्तिरिति} ।
	{\color{gray}{\rmlatinfont\textsuperscript{§~\theparCount}}}
	\pend% ending standard par
      ‚{\tiny $_{lb}$}‚

	  
	  \pstart \leavevmode% starting standard par
	त‚देत‚द‚स‚त् । य‚तो \textbf{म‚नो}विज्ञान‚स्य त‚द्विष‚य‚त्वं \textbf{श‚ब्द‚विष‚य‚त्व‚म‚सिद्धं} । य‚स्मा‚{\tiny $_{lb}$}‚\textbf{न्न हि स्व‚ल‚क्ष‚ण}श‚ब्दानां \textbf{वृत्तिरिति निवेद‚यिष्यामः} य‚दि बाह्योर्थो \textbf{विक‚ल्पानां‚{\tiny $_{lb}$}‚ न} हेतुः क‚थ‚न्त‚र्ह्युत्प‚द्य‚न्त इति [।]
	{\color{gray}{\rmlatinfont\textsuperscript{§~\theparCount}}}
	\pend% ending standard par
      ‚{\tiny $_{lb}$}‚

	  
	  \pstart \leavevmode% starting standard par
	आह । \textbf{ते ही}त्यादि । \textbf{ते हि} विक‚ल्पा \textbf{य‚था‚{\tiny $_{५}$}‚ स्व‚मि}ति य‚स्य यो वास‚नाप्र‚बोधो‚{\tiny $_{lb}$}‚ हेतुः । विज्ञान‚प्र‚तिष्ठित‚त्वेनान्त‚रात् । \textbf{विक‚ल्प‚वास‚नाप्र‚बोधाद् भ‚व‚ति । अन‚{\tiny $_{lb}$}‚पेक्षितो बाह्यार्थोप‚निधिः} स‚न्निधानं यैरिति विग्र‚हः । क‚स्मात् । \textbf{बाह्ये}त्यादि ।‚{\tiny $_{lb}$}‚ \textbf{अपायो} निरोधः । \textbf{अनाग‚मो}नुत्प‚त्तिः । बाह्य‚स्य निरोधेऽनुत्प‚त्तौ च विक‚ल्पाना‚{\tiny $_{lb}$}‚\textbf{म्भावात्} । य‚त‚श्चार्थ‚म‚न्त‚रेण भ‚व‚न्ति त‚स्मान्नार्थ‚हेत‚वः । किं कार‚णं‚{\tiny $_{६}$}‚ [।] \textbf{न हि‚{\tiny $_{lb}$}‚ यो य‚स्य स‚त्तोप‚धानं} स‚त्तास‚न्निधानं \textbf{नापेक्ष‚ते सो}न‚पेक्ष्य‚माण\textbf{स्त‚स्य} निर‚पेक्ष‚स्य \textbf{हेतु‚{\tiny $_{lb}$}‚र‚हेतुश्च} विक‚ल्पानां \textbf{क‚थ}म्विष‚यो नैव । य‚त एवं न श‚ब्द‚विष‚यो विक‚ल्प‚स्\textbf{त‚स्मान्न‚{\tiny $_{lb}$}‚ म‚नोविक‚ल्पो} \add{म‚नोज\textbf{ल्या}\edtext{}{\lemma{नोज}\Bfootnote{? पा}} दिल‚क्ष‚णः}\edtext{\textsuperscript{*}}{\edlabel{pvsvt_564-1}\label{pvsvt_564-1}\lemma{*}\Bfootnote{In the margin. }} \textbf{श‚ब्द‚व्य‚क्तिर्य‚तो} येन । \textbf{त‚द्वान्} ।‚{\tiny $_{lb}$}‚ म‚नोज‚प‚वान् पुरुषः । म‚न्त्र‚स्य \textbf{प्र‚योक्ता स्यात्} । नैव स्यादिति याव‚त् ।
	{\color{gray}{\rmlatinfont\textsuperscript{§~\theparCount}}}
	\pend% ending standard par
      ‚{\tiny $_{lb}$}‚

	  
	  \pstart \leavevmode% starting standard par
	\leavevmode\ledsidenote{\textenglish{199a/PSVTa}} श‚ब्दाभिव्य‚{\tiny $_{७}$}‚क्तिम‚भ्युप‚ग‚म्यापि दोष‚माह । श‚ब्द\textbf{प्र‚सूता त‚द्विष‚या च बुद्धिः‚{\tiny $_{lb}$}‚ श‚ब्द‚व्य‚क्तिस्त‚दाश्र‚य}स्त‚स्या व्य‚क्तेराश्र‚यो य‚तः पुरुष‚स्त‚स्मात् \textbf{प्र‚योक्ते}त्य‚त्रापि‚{\tiny $_{lb}$}‚ \textbf{प‚क्षे उक्तं} । किमुक्तं [।]श्रो \add{त्र‚प्र‚योक्तृत्व‚प्र‚स‚ङ्ग इति श्रोता}\edtext{}{\edlabel{pvsvt_564-1b}\label{pvsvt_564-1b}\lemma{श्रो}\Bfootnote{१ In the margin.}}पि हि श‚ब्द‚प्र‚सूता‚{\tiny $_{lb}$}‚‚{\tiny $_{lb}$}‚ ‚{\tiny $_{lb}$}‚ \leavevmode\ledsidenote{\textenglish{565/s}}यास्त‚द्विष‚याया बुद्धेः स्व‚स‚न्तान‚भाविन्या आश्र‚य‚स्त‚था च त‚स्यापि म‚न्त्र‚फ‚लेन‚{\tiny $_{lb}$}‚ योगः‚{\tiny $_{१}$}‚ स्यादिति ।
	{\color{gray}{\rmlatinfont\textsuperscript{§~\theparCount}}}
	\pend% ending standard par
      ‚{\tiny $_{lb}$}‚

	  
	  \pstart \leavevmode% starting standard par
	किं च नित्यानां म‚न्त्राणां नातिश‚योत्पाद‚नं प्र‚योगः किन्तु त‚द्विष‚यं ज्ञानं प्र‚योगः ।‚{\tiny $_{lb}$}‚ \textbf{त‚स्मिन्न ज्ञाने च प्र‚योगे}भ्युप‚म्य‚माने \textbf{श‚ब्दः पुरुषे व्याप्रिय‚ते} । क‚थं [।] \textbf{त‚स्य} पुरु‚{\tiny $_{lb}$}‚ष‚स्य \textbf{ज्ञान‚ज‚न‚नात् । न पुरुषः श‚ब्दे} व्याप्रिय‚ते । क‚स्मात् । \textbf{त‚दात्म‚नि} नित्ये‚{\tiny $_{lb}$}‚ श‚ब्दात्म‚नि क‚थंचिद‚पि पुरुषेणा\textbf{नुप‚कारात्} कार‚णात् । अथ च \textbf{पुरुषः श‚ब्दानां‚{\tiny $_{lb}$}‚ प्र‚योक्तेत्य‚{\tiny $_{२}$}‚लौकिकोयं व्य‚व‚हारः} । लोके हि यो य‚त्र व्याप्रिय‚ते स त‚स्य प्र‚योक्ते‚{\tiny $_{lb}$}‚त्युच्य‚ते ।
	{\color{gray}{\rmlatinfont\textsuperscript{§~\theparCount}}}
	\pend% ending standard par
      ‚{\tiny $_{lb}$}‚

	  
	  \pstart \leavevmode% starting standard par
	\textbf{स‚र्व‚थे}त्यादिनोप‚संहारः । य‚दि साक्षाच्छ‚ब्द‚प्र‚सूता बुद्धिः श‚ब्द‚बुद्धिः । अथ‚{\tiny $_{lb}$}‚ पार‚म्प‚र्येण [।] \textbf{स‚र्व‚था श‚ब्द‚स्व‚भावानां म‚न्त्राणां प्र‚योगात्} त‚द्विष‚य‚ज्ञान‚ज‚न‚नात्‚{\tiny $_{lb}$}‚ \textbf{फ‚लावाप्ता}विष्य‚माणायां \textbf{व्य‚र्थो म‚नोज‚पः} । किङ्कार‚ण‚म् [।] म‚नोज‚प‚ल‚क्ष‚ण‚स्य‚{\tiny $_{lb}$}‚ \textbf{विक‚ल्प‚स्य श‚ब्द‚स्व‚रू‚{\tiny $_{३}$}‚पासंस्प‚र्शात्} ।
	{\color{gray}{\rmlatinfont\textsuperscript{§~\theparCount}}}
	\pend% ending standard par
      ‚{\tiny $_{lb}$}‚

	  
	  \pstart \leavevmode% starting standard par
	य‚स्यापि स‚म‚यात् फ‚ल‚न्त‚स्यापि क‚थ‚म्म‚नोज‚पो न व्य‚र्थ इत्याह । \textbf{स्व}स्व‚भावः‚{\tiny $_{lb}$}‚ श‚ब्द‚स्व‚ल‚क्ष‚ण‚म्विक‚ल्प‚प्र‚तिभास्याकारः \textbf{सामान्य‚स्व‚भाव‚न्तेषां} स्व‚सामान्य‚स्व‚भावा‚{\tiny $_{lb}$}‚नामेक‚भाव‚विव‚क्ष‚या । दृश्य‚विक‚ल्प‚योरेकीकृत्य \textbf{स‚म‚य‚कार‚णामुक्ते}र्म‚न्त्र‚प्र‚ण‚य‚नात्‚{\tiny $_{lb}$}‚ म‚नोज‚पो व्य‚र्थः स्यादित्य‚य‚म\textbf{विरोधः} । स‚म‚य‚कार‚स्याभिप्राय‚स‚{\tiny $_{४}$}‚म्पाद‚नेन फ‚ल‚नि‚{\tiny $_{lb}$}‚ष्प‚त्तेः । \textbf{न} तु \textbf{व‚स्तुन्य}विरोधः किन्तु विरोध एव । त‚था हि [।] श‚ब्द‚स्व‚ल‚क्ष‚णाद्‚{\tiny $_{lb}$}‚ व‚स्तुनः फ‚लावाप्तौ म‚नोज‚पो व्य‚र्थ‚स्त‚स्याव‚स्तुसंस्प‚र्शात् ।
	{\color{gray}{\rmlatinfont\textsuperscript{§~\theparCount}}}
	\pend% ending standard par
      ‚{\tiny $_{lb}$}‚

	  
	  \pstart \leavevmode% starting standard par
	\textbf{स‚म‚ये}त्यादिना व्याच‚ष्टे । श‚ब्द\textbf{स्व‚ल‚क्ष‚णं} श्रोत्रे\textbf{न्द्रिय‚विष‚यं सामान्य‚ल‚क्ष‚ण‚ञ्च‚{\tiny $_{lb}$}‚ विक‚ल्प‚प्र‚तिभासं य‚थाव्य‚व‚हारं} लोक‚व्य‚व‚हारान‚तिक्र‚मेण \textbf{संवृत्या संक‚ल‚य्येति}‚{\tiny $_{lb}$}‚ ‚{\tiny $_{lb}$}‚ \leavevmode\ledsidenote{\textenglish{566/s}}विक‚ल्प‚बुद्ध्या दृश्य‚विक‚{\tiny $_{५}$}‚ल्प‚योः श‚ब्द‚स्व‚ल‚क्ष‚ण‚सामान्याकार‚योरेकीकृत्य \textbf{स‚म‚य‚{\tiny $_{lb}$}‚मारोच‚येत्} । यो वाचा म‚न‚सा चाभिल‚प‚न‚म्म‚त्प्र‚णीत‚स्य म‚न्त्र‚स्यानुतिष्ठ‚ति त‚स्या‚{\tiny $_{lb}$}‚य‚म्म‚न्त्रो \textbf{य‚थास‚म‚य‚म‚र्थ‚न्निष्पाद‚येदिति} [।] त‚तो य एनं म‚न्त्र‚म्वाचा म‚न‚सा चाभि‚{\tiny $_{lb}$}‚ल‚प‚ति त‚स्य \textbf{य‚था}स‚म‚यं चार्थं \textbf{निष्पाद‚येदिति न म‚नोज‚पादौ दोषः} । आदिग्र‚ह‚णा‚{\tiny $_{lb}$}‚दुपांशुप्र‚यो‚{\tiny $_{६}$}‚गादिप‚रिग्र‚हः । \textbf{व‚स्तुस्व‚भावात्तु} श‚ब्द‚स्व‚भावात् म‚न्त्रात् \textbf{फ‚लोत्प‚त्ता}वि‚{\tiny $_{lb}$}‚ष्य‚माणायाम\textbf{त‚त्स्व‚भाव‚संस्प‚र्शे} स‚ति म‚नोज‚पादौ न फ‚लं \textbf{स्या}त् ।
	{\color{gray}{\rmlatinfont\textsuperscript{§~\theparCount}}}
	\pend% ending standard par
      ‚{\tiny $_{lb}$}‚

	  
	  \pstart \leavevmode% starting standard par
	\textbf{य‚दुक्त‚मित्}यादिना प‚राभिप्राय‚माशंक‚ते । य‚दुक्त‚म् [।] व‚र्ण्णा एव हि‚{\tiny $_{lb}$}‚ म‚न्त्र इत्य‚त्रान्त‚रे ।
	{\color{gray}{\rmlatinfont\textsuperscript{§~\theparCount}}}
	\pend% ending standard par
      ‚{\tiny $_{lb}$}‚

	  
	  \pstart \leavevmode% starting standard par
	\textbf{न व‚र्ण्णेभ्योन्या} काचिदा\textbf{नुपूर्वीति} त‚त्रोच्य‚ते । \textbf{आनुपूर्व्याम‚स‚त्यां स‚र इति‚{\tiny $_{lb}$}‚ \leavevmode\ledsidenote{\textenglish{199b/PSVTa}} श्रुतौ र‚स इति श्रुतौ च न कार्य‚भेदः}‚{\tiny $_{७}$}‚ प्र‚तीतिभेद‚ल‚क्ष‚णः स्यादिति चेत् । य‚स्मा\textbf{न्न‚{\tiny $_{lb}$}‚ हि स‚रो र‚स इत्यादि प‚देषु क‚श्चिद्व‚र्ण्ण‚भेदः} । य एव हि व‚र्ण्णाः स‚र इत्य‚त्र प‚दे । त‚{\tiny $_{lb}$}‚ एव र‚स इत्य‚त्र प‚दे ।
	{\color{gray}{\rmlatinfont\textsuperscript{§~\theparCount}}}
	\pend% ending standard par
      ‚{\tiny $_{lb}$}‚

	  
	  \pstart \leavevmode% starting standard par
	\textbf{न च व्य‚तिरिक्त‚म‚न्य‚त्} प‚द‚म‚स्ति \textbf{य‚तः कार्य‚भेदो} बुद्धिभेद‚ल‚क्ष‚णः \textbf{स्यात्} । अस्ति‚{\tiny $_{lb}$}‚ च कार्य‚स्य भेदः । य‚तो \textbf{भिन्नाञ्च त‚यो}स्स‚रो र‚स इति प‚द‚योः \textbf{प्र‚तिभां} बुद्धिम्प‚श्यामः ।‚{\tiny $_{lb}$}‚ \textbf{आनुपूर्वीमे‚{\tiny $_{१}$}‚व चातुल्यां} भिन्नान्त‚योः प‚श्यामो व‚र्ण्णाः पुन‚स्त एव । \textbf{न च कार‚ण‚स्य}‚{\tiny $_{lb}$}‚ व‚र्ण्ण‚स्या\textbf{भेदे कार्य‚भेदो} बुद्धिभेद‚ल‚क्ष‚णो \textbf{युक्तः । त‚स्माद्} व‚र्ण्णाभेदेप्य‚स्ति \textbf{भेद‚व‚ती}‚{\tiny $_{lb}$}‚ प्र‚तिप‚द‚म्भिन्ना । \textbf{से}त्यानुपूर्वी । \textbf{य‚तः} प्र‚तिप‚द‚म्भेद‚व‚त्या आनुपूर्व्या अयं स‚र इत्या‚{\tiny $_{lb}$}‚दिप‚देषु \textbf{प्र‚तीतिभेदो} बुद्धिभेदः ।
	{\color{gray}{\rmlatinfont\textsuperscript{§~\theparCount}}}
	\pend% ending standard par
      ‚{\tiny $_{lb}$}‚

	  
	  \pstart \leavevmode% starting standard par
	\textbf{स‚त्य}मिति सि द्धा न्त वा दी । स‚त्यं प्र‚तिप‚द‚म्भेद‚व‚त्य‚स्त्यानुपूर्वी‚{\tiny $_{२}$}‚ । केव‚लं‚{\tiny $_{lb}$}‚ ‚{\tiny $_{lb}$}‚ \leavevmode\ledsidenote{\textenglish{567/s}}सानुपूर्वी \textbf{पुरुषाश्र‚या} पौरुषेयी । अव्य‚तिरिक्तैव व‚र्ण्णेभ्यः । व‚र्ण्णाश्च प्र‚तिप‚द‚{\tiny $_{lb}$}‚म‚न्ये चान्ये चोत्प‚द्य‚न्ते कार‚ण‚भेदात् । केव‚ल‚न्तेषु सादृश्यादेक‚त्वाध्य‚व‚सायो‚{\tiny $_{lb}$}‚ म‚न्द‚म‚तीनां ।
	{\color{gray}{\rmlatinfont\textsuperscript{§~\theparCount}}}
	\pend% ending standard par
      ‚{\tiny $_{lb}$}‚

	  
	  \pstart \leavevmode% starting standard par
	एत‚देव द‚र्श‚य‚न्नाह । \textbf{त‚था ही}त्यादि । अय‚म‚त्र स‚मुदायार्थः । व‚क्तृस्थेन पूर्व‚पूर्व‚{\tiny $_{lb}$}‚व‚र्ण्ण‚स‚मुत्थाप‚क‚चित्तेनोत्त‚रोत्त‚र‚व‚र्ण्ण‚स‚मुत्थाप‚कं चित्त‚ञ्ज‚न्य‚त इति स‚मुत्था‚{\tiny $_{३}$}‚प‚क‚चि‚{\tiny $_{lb}$}‚त्त‚क्र‚मात् । त‚त्स‚मुत्थाप्यानाम्व‚र्ण्णानामुत्प‚त्तिक्र‚मः क्र‚मोत्प‚न्नैश्च व‚र्ण्णैः स्व‚विष‚याणि‚{\tiny $_{lb}$}‚ क्र‚म‚भावीन्येव श्रोत्र‚विज्ञानानि साक्षाज्ज‚न्य‚न्ते । क्र‚म‚भाविन्य एव व‚र्ण्णाल‚म्ब‚नाः‚{\tiny $_{lb}$}‚ स्मृत‚य‚श्च पार‚म्प‚र्येण । त‚तो व‚र्ण्णानां स‚मुत्थाप‚क‚ज्ञान‚क्र‚माद् या क्र‚मे कार्य‚ता ।‚{\tiny $_{lb}$}‚ स्व‚विष‚य‚ज्ञानेषु च या क्र‚मेण कार‚ण‚ता सैवानुपूर्वीति व्य‚व‚स्थाप्य‚त इति ।
	{\color{gray}{\rmlatinfont\textsuperscript{§~\theparCount}}}
	\pend% ending standard par
      ‚{\tiny $_{lb}$}‚

	  
	  \pstart \leavevmode% starting standard par
	स‚म्प्र‚त्य‚व ? प‚दार्थो विभ‚ज्य‚ते [।] यो ध्व‚निर्जाय‚त इति स‚म्ब‚न्धः । य‚था‚{\tiny $_{lb}$}‚ स‚र इत्य‚त्र प‚दे स‚कारात् प‚रोऽकारः । कुतो जाय‚ते [।] \textbf{य‚द्व‚र्ण्ण‚स‚मुत्थान‚ज्ञान‚{\tiny $_{lb}$}‚जाज्ज्ञान‚तः} । पूर्व‚काल‚भावी व‚र्ण्णः स‚कारः । य‚श्चासौ व‚र्ण्ण‚श्चेति य‚द्व‚र्ण्णः ।‚{\tiny $_{lb}$}‚ य‚द्व‚र्ण्ण‚स्य स‚मुत्थानं कार‚णं स‚मुत्तिष्ठ‚तेनेनेति कृत्वा । य‚द्व‚र्ण्ण‚स‚मुत्थानं । य‚द्व‚र्ण्ण‚{\tiny $_{lb}$}‚स‚मुत्थान‚ञ्च त‚ज्ज्ञानं चेति क‚र्म‚धार‚यः । त‚स्मा‚{\tiny $_{५}$}‚ज्जातं य‚ज्ज्ञानं । त‚द् य‚द्व‚र्ण्ण‚स‚मु‚{\tiny $_{lb}$}‚त्थान‚ज्ञान‚जं ज्ञानं । त‚स्माज्ज्ञान‚तो \textbf{जाय‚ते} । स‚कार‚स्य स‚मुत्थाप‚कं य‚ज्ज्ञान‚न्त‚स्माद‚{\tiny $_{lb}$}‚कार‚स‚मुत्थाप‚कं ज्ञानं य‚दुत्प‚न्न‚न्तेनाकारो ज‚न्य‚त इत्य‚र्थः । एव‚म‚न्योपि पूर्व‚पूर्व‚व‚र्ण्ण‚{\tiny $_{lb}$}‚स‚मुत्थान‚ज्ञान‚जादुत्त‚रोत‚रो व‚र्ण्णो जाय‚त इति योज्यं । एव‚न्ताव‚द् व‚क्तृस‚न्तान‚{\tiny $_{lb}$}‚स्थ‚स्य स‚मुत्थान‚ज्ञान‚स्य क्र‚माद् व‚र्ण्णानां ‚{\tiny $_{६}$}‚क्र‚मेणोत्प‚त्तेः कार्य‚त्व‚मुक्तं । ते च‚{\tiny $_{lb}$}‚ क्र‚मेणोत्प‚न्नाः श्रोतृस‚न्तान‚स्थानां स्व‚विष‚य‚ज्ञानानां क्र‚मेण हेत‚वो भ‚व‚न्तो ज्ञाय‚न्त‚{\tiny $_{lb}$}‚ इति द‚र्श‚य‚न्नाह । \textbf{त‚दुपाधिरि}त्यादि । पूर्वो व‚र्ण्ण उपाधिविशेष‚णं \add{य‚स्योत्त‚र‚स्य‚{\tiny $_{lb}$}‚ व‚र्ण्ण‚स्य}\edtext{}{\edlabel{pvsvt_567-1}\label{pvsvt_567-1}\lemma{णं}\Bfootnote{In the margin. }} स त‚थोक्तः । स इत्युत्त‚रो व‚र्ण्णः \textbf{श्रुत्या} श्रोत्र‚ज्ञानेन \textbf{स‚म‚व‚सीय‚ते} गृह्य‚ते ।
	{\color{gray}{\rmlatinfont\textsuperscript{§~\theparCount}}}
	\pend% ending standard par
      ‚{\tiny $_{lb}$}‚

	  
	  \pstart \leavevmode% starting standard par
	न‚नु च पूर्वो व‚र्ण्ण उत्त‚र‚व‚र्ण्ण‚काले‚{\tiny $_{७}$}‚ नैवास्ति [।] त‚त्क‚थ‚न्त‚दुपाधि पूर्व \leavevmode\ledsidenote{\textenglish{200a/PSVTa}}‚{\tiny $_{lb}$}‚ व‚र्णोपाधि रुत्त‚रो व‚र्ण्णो गृह्य‚त इति [।]
	{\color{gray}{\rmlatinfont\textsuperscript{§~\theparCount}}}
	\pend% ending standard par
      ‚{\tiny $_{lb}$}‚

	  
	  \pstart \leavevmode% starting standard par
	आह । \textbf{त‚ज्ज्ञान‚ज‚नित‚ज्ञान} इति । तेन पूर्व‚व‚र्ण्णाविष‚येण ज्ञानेन ज‚नितं स्व‚{\tiny $_{lb}$}‚विष‚यं ज्ञानं य‚स्येति विग्र‚हः ।
	{\color{gray}{\rmlatinfont\textsuperscript{§~\theparCount}}}
	\pend% ending standard par
      ‚{\tiny $_{lb}$}‚‚{\tiny $_{lb}$}‚‚{\tiny $_{lb}$}‚\textsuperscript{\textenglish{568/s}}

	  
	  \pstart \leavevmode% starting standard par
	एत‚दुक्त‚म्भ‚व‚ति । स‚काराल‚म्ब‚नं य‚दा \add{श्रोत्र‚विज्ञानं त‚स्मिन्नेव‚काले}\edtext{}{\edlabel{pvsvt_568-1}\label{pvsvt_568-1}\lemma{दा}\Bfootnote{In the margin. }}‚{\tiny $_{lb}$}‚ अकार‚स‚मुत्थाप‚न‚चित्तेनाकारो ज‚नित‚स्तेनाकार‚स्स‚काराल‚म्ब‚न‚श्च प्र‚त्य‚यः स‚मान‚{\tiny $_{lb}$}‚कालः । त‚त्र साकाराल‚म्ब‚ने‚{\tiny $_{१}$}‚न श्रोत्र‚विज्ञानेन स‚ह‚कारिणाऽकारः स्व‚विष‚यं‚{\tiny $_{lb}$}‚ ज्ञानं ज‚न‚य‚न् पूर्व‚व‚र्ण्णोपाधिः प्र‚तीय‚त इत्युच्य‚त इत्य‚र्थः । त‚देवं पूर्व‚ज्ञानेन स‚ह‚कारि‚{\tiny $_{lb}$}‚णा ज‚नितात्म‚ज्ञानः । स इत्युत्त‚रो व‚र्ण्णः [।] कीदृशः श्र‚व‚ण‚काले । \textbf{अप‚टुश्रुति}‚{\tiny $_{lb}$}‚रित्य‚त्व‚रितं श‚नैः श‚नैरुच्चारितो य‚दा व‚र्ण्णो भ‚व‚ति । त‚दाऽप‚ट्वी म‚न्द‚चारिणी‚{\tiny $_{lb}$}‚ प्र‚विभ‚क्त‚व‚र्ण्ण‚ग्राहिणी श्रुतिः श्रो‚{\tiny $_{२}$}‚त्र‚विज्ञानं य‚स्य श‚ब्द‚स्येत्य‚प‚टुश्रुतिः । य‚स्याम‚{\tiny $_{lb}$}‚व‚स्थायाम्विभ‚क्ता व‚र्ण्णा अव‚धार्य‚न्त इति याव‚त् । अति त्व‚रित‚न्तूच्चार्य‚माणे‚{\tiny $_{lb}$}‚ विभ‚क्त‚व‚र्ण्णाप‚रिच्छेदात् कुतः क्र‚मेण स्मृतिज‚न‚न‚मित्य‚स्य स‚न्द‚र्श‚नार्थं । अप‚टु‚{\tiny $_{lb}$}‚श्रुतिग्र‚ह‚णं । स एवंभूतो व‚र्ण्णः किंकारीति [।]
	{\color{gray}{\rmlatinfont\textsuperscript{§~\theparCount}}}
	\pend% ending standard par
      ‚{\tiny $_{lb}$}‚

	  
	  \pstart \leavevmode% starting standard par
	आह । \textbf{अपेक्ष्य त‚त्स्मृतिं} पूर्व‚व‚र्ण्ण‚स्मृतिं । \textbf{प‚श्चादाध‚त्ते} । ज‚न‚य‚ति \textbf{स्मृतिमा‚{\tiny $_{lb}$}‚त्म‚नि} । स्व‚विष‚ये‚{\tiny $_{३}$}‚ । पार‚म्प‚र्येणेति द्र‚ष्ट‚व्यं ।
	{\color{gray}{\rmlatinfont\textsuperscript{§~\theparCount}}}
	\pend% ending standard par
      ‚{\tiny $_{lb}$}‚

	  
	  \pstart \leavevmode% starting standard par
	एतेन च स्व‚विष‚याणि ज्ञानानि प्र‚ति व‚र्ण्णानां क्र‚मेण कार‚ण‚तोक्ता । इत्यु‚{\tiny $_{lb}$}‚क्तेन क्र‚मे\textbf{णैषा कार्य‚कार‚ण‚ता । व‚र्ण्णेष्वि}ति व‚र्ण्णाधारा व‚र्ण्णानामिति याव‚त् ।‚{\tiny $_{lb}$}‚ \textbf{पौरुषे}येप्येवं । पुरुष‚कृतैवा\textbf{नुप‚र्वीति} लोके \textbf{क‚थ्य‚ते} ।
	{\color{gray}{\rmlatinfont\textsuperscript{§~\theparCount}}}
	\pend% ending standard par
      ‚{\tiny $_{lb}$}‚

	  
	  \pstart \leavevmode% starting standard par
	किम‚पेक्ष‚या व‚र्ण्णानां \textbf{कार्य‚ता कार‚ण‚ता} चेत्याह । \textbf{त‚द्धेतुग्राहिचेत‚सा}मिति ।‚{\tiny $_{lb}$}‚ हेत‚व‚श्च ग्राहीणि‚{\tiny $_{४}$}‚ चेति द्व‚न्द्वः । तेषाम्व‚र्ण्णानां हेतुग्राहीणीति ष‚ष्ठीस‚मासः ।‚{\tiny $_{lb}$}‚ प‚श्चाच्चेतःश‚ब्देन विशेष‚ण‚स‚मासः । व‚र्ण्ण‚हेत‚वः क्र‚मेण य‚दि चेतांसि तेषां स‚म्ब‚न्धेन‚{\tiny $_{lb}$}‚ व‚र्ण्णानां क्र‚मेण कार्य‚ता । तैश्चेतोभिर्व‚र्ण्णानाञ्ज‚न्य‚त्वात् । तेषां व‚र्ण्णानां ग्राहीणि‚{\tiny $_{lb}$}‚ यानि चेतांसि । तेषां स‚म्ब‚न्धेन व‚र्ण्णानां क्र‚मेण कार‚ण‚ता । व‚र्ण्णौस्तेषाम्व‚र्ण्ण‚{\tiny $_{lb}$}‚ग्राहिणां चेत‚सां ज‚न्य‚{\tiny $_{५}$}‚त्वात् ।
	{\color{gray}{\rmlatinfont\textsuperscript{§~\theparCount}}}
	\pend% ending standard par
      ‚{\tiny $_{lb}$}‚

	  
	  \pstart \leavevmode% starting standard par
	\textbf{चित्ते}त्यादिना व्याच‚ष्टे । \textbf{चित्तं स‚मुत्थानं} कार‚णं य‚स्या \textbf{वाग्विज्ञ}प्तेस्सा‚{\tiny $_{lb}$}‚ त‚थोक्ता । वागेव विज्ञ‚प्तिः प‚र‚विज्ञाप‚नात् [।] सा च त्रिधा लोक इत्याह । \textbf{व\unclear{र्ण्णे}}‚{\tiny $_{lb}$}‚ त्यादि \textbf{व‚र्ण्णाः प‚दं वाक्यं} चेत्य‚भिधानं य‚स्येति विग्र‚हः । त‚त्राक्ष‚राणि व‚र्ण्णाः ।‚{\tiny $_{lb}$}‚ अर्थाव‚च्छिन्नो व‚र्ण्ण‚स‚मुदायः प‚दं । प‚द‚स‚मुदायो वाक्यं । \textbf{त‚त्रे}त्यादि स‚र इत्य‚त्र‚{\tiny $_{lb}$}‚ ‚{\tiny $_{lb}$}‚ ‚{\tiny $_{lb}$}‚ \leavevmode\ledsidenote{\textenglish{569/s}}प‚दे य‚ञ्च व‚{\tiny $_{६}$}‚र्ण्णाः स‚कार‚स्त‚स्मात् प‚रो अकार‚स्त‚तो रेफ‚स्त‚स्माद‚कार‚स्त‚स्मात्प‚रो‚{\tiny $_{lb}$}‚ विस‚र्ज‚नीय इति । \textbf{त‚त्र स‚कार‚स्य स‚मुत्थाप‚नं} कार‚णं य‚च्चेतः । तेन \textbf{चेत‚सा स‚म‚{\tiny $_{lb}$}‚न‚न्त‚र‚प्र‚त्य‚येन} । स‚म‚न‚न्त‚र‚ग्र‚ह‚ण‚माल‚म्ब‚न‚प्र‚त्य‚य‚व्य‚व‚च्छेदार्थं । \textbf{अकारोत्थाप‚न‚चित्तं} ।‚{\tiny $_{lb}$}‚ अकार उत्थाप्य‚ते \textbf{ज‚न्य‚ते} येन चेत‚सा । त‚दुत्पाद्य‚ते । \textbf{त‚थे}त्युक्तेन क्र‚मेण‚{\tiny $_{७}$}‚ \textbf{रेफाका- \leavevmode\ledsidenote{\textenglish{200b/PSVTa}}‚{\tiny $_{lb}$}‚ र‚विस‚र्ज‚नीया उत्थाप्य‚न्ते} यैश्चित्तैस्तानि \textbf{पूर्व‚पूर्व‚प्र‚त्य‚यानि} । पूर्वं पूर्वं चित्तं प्र‚त्य‚यः‚{\tiny $_{lb}$}‚ कार‚णं येषामिति विग्र‚हः । त‚त्राकार‚स‚मुत्थाप‚न‚चेत‚सा रेफ उत्प‚द्य‚ते । रेफ‚स‚मु‚{\tiny $_{lb}$}‚त्थाप‚न‚चेत‚सा स‚म‚न‚न्त‚र‚प्र‚त्य‚येनाकारः । अकार‚स‚मुत्थाप‚न‚चेत‚सा विस‚र्ज‚नीय‚{\tiny $_{lb}$}‚ उत्पाद्य‚त इति ।
	{\color{gray}{\rmlatinfont\textsuperscript{§~\theparCount}}}
	\pend% ending standard par
      ‚{\tiny $_{lb}$}‚

	  
	  \pstart \leavevmode% starting standard par
	न‚नु चेद‚म्प‚द‚मुच्चार‚यामीति विव‚क्ष‚{\tiny $_{१}$}‚या प‚द‚मुच्चार्य‚ते । तेनैक‚यैव विव‚क्ष‚या‚{\tiny $_{lb}$}‚ व‚र्ण्ण‚क्र‚म उच्चार्य‚ते । न तु व‚र्ण्णानां प्र‚त्येकं विव‚क्षापूर्व‚क‚त्व‚म‚प्र‚तीतेः [।] त‚त्क‚थ‚{\tiny $_{lb}$}‚मुच्य‚ते [।] स‚कारादिस‚मुत्थाप‚क‚चित्तेनाकारादिस‚मुत्थाप‚कं चित्तं ज‚न्य‚त इति ।
	{\color{gray}{\rmlatinfont\textsuperscript{§~\theparCount}}}
	\pend% ending standard par
      ‚{\tiny $_{lb}$}‚

	  
	  \pstart \leavevmode% starting standard par
	एव‚म्म‚न्य‚ते [।] व‚र्ण्णोच्चार‚णे ताव‚द‚य‚मेव क्र‚मः । प‚दोच्चार‚णेपि प्र‚थ‚म‚म‚य‚{\tiny $_{lb}$}‚मेव क्र‚मः [।] त‚था हि [।] स‚कार‚विव‚क्ष‚या स‚कार‚मु‚{\tiny $_{२}$}‚च्चार‚य‚त्येव‚मुत्त‚रोत्त‚र‚{\tiny $_{lb}$}‚व‚र्ण्ण‚विव‚क्ष‚योत्त‚रोत्त‚र‚म्व‚र्ण्ण‚मुच्चार‚य‚ति । अभ्यासात्तु प‚दोच्चार‚णे प‚द‚विव‚क्षैवैका‚{\tiny $_{lb}$}‚ कार‚ण‚मित्येके ।
	{\color{gray}{\rmlatinfont\textsuperscript{§~\theparCount}}}
	\pend% ending standard par
      ‚{\tiny $_{lb}$}‚

	  
	  \pstart \leavevmode% starting standard par
	अन्ये त्व‚न्य‚था [।] प‚दोच्चार‚णे । एकैव‚विव‚क्षा कार‚ण‚मिति [।]
	{\color{gray}{\rmlatinfont\textsuperscript{§~\theparCount}}}
	\pend% ending standard par
      ‚{\tiny $_{lb}$}‚

	  
	  \pstart \leavevmode% starting standard par
	स‚त्य‚मेत‚त् । केव‚लं स‚कारोच्चार‚ण‚कालेऽव‚श्यं चित्तं विद्य‚तेऽन्य‚था म‚र‚ण‚{\tiny $_{lb}$}‚प्र‚स‚ङ्गात् । त‚देव च चित्तं स‚कार‚स‚मुत्थाप‚क‚मुच्य‚ते त‚द‚न‚{\tiny $_{३}$}‚न्त‚रं स‚कार‚स्योत्प‚त्तेः ।‚{\tiny $_{lb}$}‚ एव‚मुत्त‚रोत्त‚र‚व‚र्ण्णेषु चित्त‚स‚मुत्थाप‚क‚त्वं द्र‚ष्ट‚व्य‚मिति । अत्र च स‚कार‚स‚मुत्थाप‚{\tiny $_{lb}$}‚क‚चेत‚सा स‚म‚न‚न्त‚र‚प्र‚त्य‚येनेत्या दिना ग्र‚न्थेनैक‚क‚र्त्तृक एव व‚र्ण्ण‚क्र‚मोर्थ‚प्र‚तीतिहेतुर्न‚{\tiny $_{lb}$}‚ भिन्न‚क‚र्त्तृक इत्युक्त‚म्भ‚व‚ति । तेन य‚दुच्य‚ते म ण्ड ने न ।
	{\color{gray}{\rmlatinfont\textsuperscript{§~\theparCount}}}
	\pend% ending standard par
      ‚{\tiny $_{lb}$}‚
	  \bigskip
	  \begingroup
	
	    
	    \stanza[\smallbreak]
	  {\normalfontlatin\large ``\qquad}कार्य‚कार‚ण‚भाव‚श्चेत् क्र‚म‚स्त‚द्ग्राहिचेत‚सां ।&‚{\tiny $_{lb}$}‚त‚द्धेतुरात्म‚भेदो वा व‚क्त्तृभे‚{\tiny $_{४}$}‚देपि धीर्भ‚वेदिति \href{http://sarit.indology.info/?cref=\%C5\%9Bv-spho\%E1\%B9\%ADa.31}{स्फोट‚सिद्धिः ३१}{\normalfontlatin\large\qquad{}"}\&[\smallbreak]
	  
	  
	  
	  \endgroup
	‚{\tiny $_{lb}$}‚

	  
	  \pstart \leavevmode% starting standard par
	त‚द‚पास्तं । भिन्न‚क‚र्त्तृक‚व‚र्ण्ण‚ग्राहिचेत‚सामात्म‚भाव‚स्यार्थ‚प्र‚तीतिहेतुत्वान‚भ्युप‚{\tiny $_{lb}$}‚ग‚मात् ।
	{\color{gray}{\rmlatinfont\textsuperscript{§~\theparCount}}}
	\pend% ending standard par
      ‚{\tiny $_{lb}$}‚

	  
	  \pstart \leavevmode% starting standard par
	य‚च्चाप्युक्त म‚म ण्ड ने न [।] अथ स‚मुत्थाप‚क‚चित्त‚क्र‚मेण व‚र्ण्ण‚क्र‚माद‚र्थ‚{\tiny $_{lb}$}‚प्र‚तीतिस्त‚थापि स च क्र‚मो ज्ञाप‚क‚त्वाज्ज्ञान‚म‚पेक्ष‚ते । दृश्य‚ते च तिरोहित‚व्य‚व‚हित‚{\tiny $_{lb}$}‚व‚क्तृप्र‚युक्ताच्छ‚ब्दाद‚र्थ‚ज्ञानं [।] न च त‚त्र स‚मुत्थाप‚क‚चित्त‚कार्य‚कार‚ण‚तां‚{\tiny $_{५}$}‚ क‚श्च‚न‚{\tiny $_{lb}$}‚ निश्चेतुम‚र्ह‚ति । च‚क्षुरेक‚त्वे हि सा निश्चीयेतान्त‚रेण श‚ब्द‚ज्ञानात् । न च तिरोहित‚{\tiny $_{lb}$}‚‚{\tiny $_{lb}$}‚ \leavevmode\ledsidenote{\textenglish{570/s}}व्य‚व‚हित‚योर्व‚क्तुरेक‚त्वे प्र‚माण‚म‚स्ती ति ।
	{\color{gray}{\rmlatinfont\textsuperscript{§~\theparCount}}}
	\pend% ending standard par
      ‚{\tiny $_{lb}$}‚

	  
	  \pstart \leavevmode% starting standard par
	त‚द‚युक्तं । य‚त‚स्तिरोहित‚व्य‚व‚हिताव‚स्थायाम्व‚क्तुरेक‚त्वं कैश्चिद‚व‚धार्य‚त एव ।‚{\tiny $_{lb}$}‚ त‚था हि [।] देव‚द‚त्तो मां श‚ब्द‚य‚ति न य‚ज्ञ‚द‚त्त इति लोके प्र‚तीतिपूर्विकैव प्र‚वृ‚{\tiny $_{lb}$}‚त्तिर्दृश्य‚ते । य‚त्र नाव‚धार‚णं न त‚त्र‚{\tiny $_{६}$}‚ प्र‚वृत्तिरितीष्ट‚सिद्धिरेव ।
	{\color{gray}{\rmlatinfont\textsuperscript{§~\theparCount}}}
	\pend% ending standard par
      ‚{\tiny $_{lb}$}‚

	  
	  \pstart \leavevmode% starting standard par
	अपि च स्फो ट वा दिनोपि तिरोहिताव‚स्थादौ क‚थं स्फोटाभिव्य‚क्तिर्व्य‚ञ्ज‚{\tiny $_{lb}$}‚कानां व‚र्ण्णानामेक‚क‚र्त्तृक‚त्वान‚व‚धार‚णात् । अव‚श्यं च त‚त्र स्फोटाभिव्य‚क्ति\add{रेष्ट‚व्या ।‚{\tiny $_{lb}$}‚ अन्य‚थार्थाव‚ग‚तिर्न स्या}\edtext{}{\edlabel{pvsvt_570-1}\label{pvsvt_570-1}\lemma{क्ति}\Bfootnote{In the margin. }} दिति कैश्चिद‚व‚धार्य‚त एव [।] य‚त्किञ्चिदेत‚त् ।
	{\color{gray}{\rmlatinfont\textsuperscript{§~\theparCount}}}
	\pend% ending standard par
      ‚{\tiny $_{lb}$}‚

	  
	  \pstart \leavevmode% starting standard par
	त‚दिति त‚स्मा\textbf{दिमे व‚र्ण्णा} अन्यान्य‚हेत‚व इति । अन्य‚द‚न्य‚त्स‚मुत्थाप‚कं चित्तं‚{\tiny $_{lb}$}‚ \leavevmode\ledsidenote{\textenglish{201a/PSVTa}} हेतुर्येषा‚{\tiny $_{७}$}‚मित्य‚र्थः । \textbf{स्व‚कार‚णानि} स‚मुत्थाप‚कान्येव ज्ञानानि तेषामानुपूर्वी‚{\tiny $_{lb}$}‚ क्र‚म‚स्त‚स्या ज‚न्म येषान्ते त‚थोक्ताः । कार‚ण‚क्र‚मात् क्र‚म‚भाविनो व‚र्ण्णा इत्य‚र्थः ।‚{\tiny $_{lb}$}‚ इय‚ता च [।]
	{\color{gray}{\rmlatinfont\textsuperscript{§~\theparCount}}}
	\pend% ending standard par
      ‚{\tiny $_{lb}$}‚

	  
	  \pstart \leavevmode% starting standard par
	यो य‚द्व‚र्ण्ण‚स‚मुत्थान‚ज्ञान‚जाज्ज्ञातो ध्व‚निः [।] इत्येत‚द्व्याख्यातं ।
	{\color{gray}{\rmlatinfont\textsuperscript{§~\theparCount}}}
	\pend% ending standard par
      ‚{\tiny $_{lb}$}‚

	  
	  \pstart \leavevmode% starting standard par
	\textbf{स श्रुत्या स‚म‚व‚सीय‚त}\edtext{\textsuperscript{*}}{\edlabel{pvsvt_570-2}\label{pvsvt_570-2}\lemma{*}\Bfootnote{नो ऋओओत्नोते ऋओउन्द् ओन् थिस् प‚गे!}} इत्येत‚द् विवृण्व‚न्नाह । \textbf{श्रुती}त्यादि । \textbf{श्रुतिकालेपि}‚{\tiny $_{lb}$}‚ श्र‚व‚ण‚कालेपि \textbf{य‚दा म‚न्द‚चारिण} इति य‚दा नातिद्रुत‚मुच्चार्य‚न्त इत्य‚{\tiny $_{१}$}‚र्थः । \textbf{पूर्व‚व‚र्ण्णा}‚{\tiny $_{lb}$}‚ल‚म्ब‚नं ज्ञान‚न्त‚देव स‚ह‚कारिप्र‚त्य‚य‚स्त‚म‚पेक्ष‚न्ते ये व‚र्ण्णास्ते त‚थोक्ताः ।
	{\color{gray}{\rmlatinfont\textsuperscript{§~\theparCount}}}
	\pend% ending standard par
      ‚{\tiny $_{lb}$}‚

	  
	  \pstart \leavevmode% starting standard par
	त एवंभूताः किं कुर्व‚न्ति [।] \textbf{स्व‚ज्ञानं} स्व‚विष‚यं श्रोत्र‚विज्ञानं ज‚न‚य‚न्ति ।‚{\tiny $_{lb}$}‚ त‚था हि [।] स‚काराल‚म्ब‚नं श्रोत्र‚विज्ञानं य‚स्मिन्नेव काले त‚थैवाकारोप्य‚कार‚{\tiny $_{lb}$}‚स‚मुत्थाप‚न‚चेत‚सा ज‚नित‚स्तेनाकारः स‚काराल‚म्ब‚नं च ज्ञान‚मेक‚काल‚न्त‚स्माद‚कारः‚{\tiny $_{lb}$}‚ स‚काराल‚म्ब‚न‚ज्ञानेन स‚ह‚कारिणा‚{\tiny $_{२}$}‚ स्व‚विष‚यं ज्ञानं ज‚न‚य‚ति [।] एव‚म‚न्येष्व‚पि‚{\tiny $_{lb}$}‚ व‚र्ण्णेष्व‚यं न्यायो योज्यः ।
	{\color{gray}{\rmlatinfont\textsuperscript{§~\theparCount}}}
	\pend% ending standard par
      ‚{\tiny $_{lb}$}‚

	  
	  \pstart \leavevmode% starting standard par
	\textbf{त‚दे}ति य‚दा स्व‚विष‚य‚म‚नुभ‚व‚ज्ञानं ज‚नित‚व‚न्त‚स्त‚दा \textbf{पूर्व‚व‚र्ण्ण‚विष‚यं य‚त् स्म‚र‚ण‚{\tiny $_{lb}$}‚न्त‚द‚पेक्षा एव स्मृतिमुप‚लीय‚न्ते} स्मृतावारोह‚न्ति । येनैव क्र‚मेणानुभूतास्तेनैव‚{\tiny $_{lb}$}‚ क्र‚मेण स्म‚र्य‚न्त इत्य‚र्थः । स एषो युग‚प‚द‚भाव‚ल‚क्ष‚णो व‚र्ण्णानां स्व‚भाव इति स‚म्ब‚न्धः ।‚{\tiny $_{lb}$}‚ कीदृश इत्याह । \textbf{भिन्ने}त्यादि‚{\tiny $_{३}$}‚ । पूर्व‚पूर्व‚विज्ञान‚ज‚न्य‚त्वाद् भिन्नः कार्य‚भावः ।‚{\tiny $_{lb}$}‚ उत्त‚रोत्त‚र‚ज्ञान‚स्य हेतुत्वाद् भिन्नः \textbf{कार‚ण}भाव‚श्च येषां स‚कारादिस‚मुत्थाप‚क‚{\tiny $_{lb}$}‚ज्ञानानान्तानि भिन्न‚कार्य‚कार‚ण‚भावानि । तान्येव \textbf{प्र‚त्य‚या} हेत‚वः । तेभ्यो निर्वृ‚{\tiny $_{lb}$}‚‚{\tiny $_{lb}$}‚ ‚{\tiny $_{lb}$}‚ ‚{\tiny $_{lb}$}‚ \leavevmode\ledsidenote{\textenglish{571/s}}त्तिरुत्प‚त्तिः सैव \textbf{ध‚र्मो} ल‚क्ष‚णं य‚स्य स्व‚भाव‚स्येति विग्र‚हः । एतेन त‚द्धेतुचेतां‚{\tiny $_{lb}$}‚स्य‚पेक्ष्य व‚र्णानां कार्य‚त्व‚मुक्तं ।
	{\color{gray}{\rmlatinfont\textsuperscript{§~\theparCount}}}
	\pend% ending standard par
      ‚{\tiny $_{lb}$}‚

	  
	  \pstart \leavevmode% starting standard par
	त‚द्ग्राहिचेतांस्य‚पे‚{\tiny $_{४}$}‚क्ष्य कार‚ण‚त्व‚माह । \textbf{भिन्न}स्य विज्ञान‚कार्य‚स्य निर्व‚र्त्त‚नं‚{\tiny $_{lb}$}‚ \textbf{ज‚न‚नं} स एव \textbf{ध‚र्मो} ल‚क्ष‚णं य‚स्य स्व‚भाव‚स्येति विग्र‚हः । \textbf{स} एवंभूतो \textbf{व‚र्ण्ण‚स्व‚भावः‚{\tiny $_{lb}$}‚ पुरुष‚संस्कार‚भेद‚भिन्नः} पुरुष‚प्र‚य‚त्न‚भेद‚भिन्नः \textbf{क्र‚म इत्युच्य‚ते} ।
	{\color{gray}{\rmlatinfont\textsuperscript{§~\theparCount}}}
	\pend% ending standard par
      ‚{\tiny $_{lb}$}‚

	  
	  \pstart \leavevmode% starting standard par
	न‚नु क्र‚मो व‚र्ण्णानां ध‚र्म‚स्तेन क‚थं स एवंभूतो व‚र्ण्ण‚स्व‚भावः क्र‚म इत्युच्य‚ते ।
	{\color{gray}{\rmlatinfont\textsuperscript{§~\theparCount}}}
	\pend% ending standard par
      ‚{\tiny $_{lb}$}‚

	  
	  \pstart \leavevmode% starting standard par
	एव‚म्म‚न्य‚ते [।] न युग‚प‚दुत्प‚न्नानां क्र‚मोस्त्य‚प्र‚तीतेः । त‚स्माद्‚{\tiny $_{५}$}‚ युग‚प‚दुत्प‚{\tiny $_{lb}$}‚न्नानामेव व‚र्ण्णानां क्र‚मः । अयुग‚प‚दुत्प‚न्नाश्चेद् व‚र्ण्णा इष्य‚न्ते त एव लोके क्र‚मो‚{\tiny $_{lb}$}‚ न व‚र्ण्णेभ्योर्थान्त‚र‚भूतोसाव‚प्र‚तीतेः । नापि तेषां क्र‚म एको ध‚र्मोऽस‚ह‚भावात् ।‚{\tiny $_{lb}$}‚ नापि प्र‚त्येकं ध‚र्मः क्र‚मोप्र‚तीतेः । त‚स्माद‚युग‚प‚दुत्प‚न्ना एव व‚र्ण्णाः क्र‚म इत्युच्य‚ते‚{\tiny $_{lb}$}‚ इत्युक्तं । केव‚ल‚मेषां क्र‚म इति क‚ल्पितोयं व्य‚व‚हारः । न च य एव स‚र इति प‚दे‚{\tiny $_{६}$}‚‚{\tiny $_{lb}$}‚ क्र‚मो लोके प्र‚तीय‚ते स एव र‚स इति प‚दे । नापि क्र‚म‚व्य‚तिरिक्तं पूर्वाप‚र‚व‚र्ण्णानां‚{\tiny $_{lb}$}‚ स्व‚रूपं । त‚स्मात् प्र‚तिप‚दं व‚र्ण्णानाम‚न्य‚देव स्व‚रूपं । लोक‚श्च स‚र इति प‚दाद् र‚स‚{\tiny $_{lb}$}‚प‚द‚स्यान्य‚त्व‚म‚व‚धार‚य‚त्येव ।
	{\color{gray}{\rmlatinfont\textsuperscript{§~\theparCount}}}
	\pend% ending standard par
      ‚{\tiny $_{lb}$}‚

	  
	  \pstart \leavevmode% starting standard par
	तेन य‚दुच्य‚ते म ण्ड \textbf{ने} न ।
	{\color{gray}{\rmlatinfont\textsuperscript{§~\theparCount}}}
	\pend% ending standard par
      ‚{\tiny $_{lb}$}‚
	  \bigskip
	  \begingroup
	
	    
	    \stanza[\smallbreak]
	  {\normalfontlatin\large ``\qquad}उत्प‚त्तिवादिनो व‚र्ण्णाः काम‚न्ते स‚न्तु भेदिनः ।&‚{\tiny $_{lb}$}‚न त्व‚साधार‚ण‚स्तेषाम्भेदोर्थ‚ज्ञान‚कार‚णं । \href{http://sarit.indology.info/?cref=spho\%E1\%B9\%ADasiddhi.30}{स्फोट‚सिद्धिः ३०}{\normalfontlatin\large\qquad{}"}\&[\smallbreak]
	  
	  
	  
	  \endgroup
	‚{\tiny $_{lb}$}‚

	  
	  \pstart \leavevmode% starting standard par
	त‚स्यान‚व‚धार‚णात् संकेत‚काले‚{\tiny $_{७}$}‚ चादृष्ट‚त्वादिति [।]
	{\color{gray}{\rmlatinfont\textsuperscript{§~\theparCount}}}
	\pend% ending standard par
      \textsuperscript{\textenglish{201b/PSVTa}}‚{\tiny $_{lb}$}‚

	  
	  \pstart \leavevmode% starting standard par
	त‚द‚पास्तं । य‚त‚स्स‚र इति व‚र्ण्ण‚क्र‚माद् र‚स इति व‚र्ण्ण‚क्र‚मो भिन्न एवाव‚धार्य‚ते ।‚{\tiny $_{lb}$}‚ नापिक्र‚म‚व्य‚तिरेकेण व‚र्ण्णाः प्र‚तिभास‚न्ते । त‚स्मात् क्र‚म‚भेदाव‚धार‚ण‚मेव व‚र्ण्ण‚{\tiny $_{lb}$}‚भेदाव‚धार‚णं [।] केव‚लं र‚स‚प‚दाद् र‚स‚प‚दान्त‚र‚स्य भेदः सादृश्यान्नाव‚धार्य‚ते [।]‚{\tiny $_{lb}$}‚ अत एव संकेत‚काले दृष्ट‚त्वात् र‚स‚प‚दार्थ‚प्र‚तिपाद‚कं युक्त‚म् [।] एव‚म‚न्य‚स्यापि‚{\tiny $_{lb}$}‚ प‚द‚स्येति य‚त्किञ्चिदेत‚त् ।
	{\color{gray}{\rmlatinfont\textsuperscript{§~\theparCount}}}
	\pend% ending standard par
      ‚{\tiny $_{lb}$}‚

	  
	  \pstart \leavevmode% starting standard par
	एत‚देव‚{\tiny $_{१}$}‚ द‚र्श‚य‚न्नाह । \textbf{अन्य‚देवे}त्यादि । य‚तो \textbf{व‚र्ण्णानां} स्व‚भावो य‚थोक्तः क्र‚म‚{\tiny $_{lb}$}‚ इत्युच्य‚ते त‚तः कार‚णात् । \textbf{त‚दि}ति सादृश्यादेक‚त्वेनाध्य‚व‚सित‚म‚पि \textbf{रूप‚म्}व‚र्ण्णा‚{\tiny $_{lb}$}‚नाम‚न्य‚देव \textbf{प‚दं प‚दं} प्र‚तिप‚दं । किं कार‚णं [।] \textbf{क‚र्त्तृसंस्कार‚तो भिन्नं} य‚तः [।]‚{\tiny $_{lb}$}‚ स‚मुत्थाप‚क‚चित्त‚मेव क‚र्त्तृ त‚स्य श‚क्तिभेदाद् भिन्नं । \textbf{स‚हित‚मिति} पूर्वोत्त‚र‚क्र‚मेणो‚{\tiny $_{lb}$}‚‚{\tiny $_{lb}$}‚ \leavevmode\ledsidenote{\textenglish{572/s}}च्चारितं । \textbf{कार्य‚भेद‚कृ}दिति । बुद्धिभेदं क‚रोति‚{\tiny $_{२}$}‚ [।]
	{\color{gray}{\rmlatinfont\textsuperscript{§~\theparCount}}}
	\pend% ending standard par
      ‚{\tiny $_{lb}$}‚

	  
	  \pstart \leavevmode% starting standard par
	\hphantom{.}न च व‚र्ण्णा निर‚र्थ‚कास्स‚न्त इति \href{http://sarit.indology.info/?cref=pv.3.238}{१ । २४१} पूर्व‚मेव प्र‚तिपादितं [।]‚{\tiny $_{lb}$}‚ त‚त्क‚थ‚म्व‚र्ण्ण‚स्व‚रूपं स‚हितं कार्य‚भेद‚कृदित्युच्य‚ते ।
	{\color{gray}{\rmlatinfont\textsuperscript{§~\theparCount}}}
	\pend% ending standard par
      ‚{\tiny $_{lb}$}‚

	  
	  \pstart \leavevmode% starting standard par
	अत्रैके म‚न्य‚न्ते [।] प्र‚तिप‚द‚म्व‚र्ण्णानां स्व‚रूप‚म्भिन्नं पौरुषेय‚म्वाच‚क । नापौ‚{\tiny $_{lb}$}‚रुषेय‚मिति [।] य‚दाह [।] स‚त्य‚म् [।] अस्ति सा किन्तु पुरुषाश्र‚येति [।]
	{\color{gray}{\rmlatinfont\textsuperscript{§~\theparCount}}}
	\pend% ending standard par
      ‚{\tiny $_{lb}$}‚

	  
	  \pstart \leavevmode% starting standard par
	त‚द‚युक्त‚म् [।] व‚र्ण्णानां स‚हितास‚हितानाम‚र्थाप्र‚तिपाद‚क‚त्वात् । त‚स्माद‚{\tiny $_{lb}$}‚य‚म‚भिप्रायः [।] य‚दि प‚र‚मार्थ‚तो व‚र्ण्ण‚क्र‚मः स्या‚{\tiny $_{३}$}‚त् त‚दासावेक‚प‚दादिरूप‚त‚या‚{\tiny $_{lb}$}‚ क‚ल्पितोर्थ‚स्य प्र‚तिपाद‚कः स्यात् । य‚त‚श्चैकेन विक‚ल्पेन विष‚यीकृताः क्र‚मिणो‚{\tiny $_{lb}$}‚ व‚र्ण्णाः प्र‚तिपाद‚का अत एवैक‚विक‚ल्पाव‚भासित्वात् । क्र‚मिणाम्व‚र्ण्णानां रूपं स‚हितं‚{\tiny $_{lb}$}‚ कार्य‚भेद‚कृदित्युच्य‚ते इत्य‚दोषः । य‚त‚श्च स‚मुत्थाप‚क‚भेदाद् भेदः । \textbf{त‚स्मान्न‚{\tiny $_{lb}$}‚ ख‚ल्वेक एव स्व‚भावो व‚र्ण्णानां} स‚रो र‚स इत्यादिप‚देषु । किं कार‚णं [।] क‚र्त्तृचित्त‚{\tiny $_{lb}$}‚संस्कार‚भेदेन भेदात् । क‚र्त्तृ च त‚च्चित्त‚ज्ञात‚स्य संस्कार‚भेदः स‚म‚न‚न्त‚र‚प्र‚त्य‚य‚भेदेन‚{\tiny $_{lb}$}‚ श‚क्तिभेद‚स्तेन व‚र्ण्णानां स्व‚भाव‚स्य भेदात् । स च व‚र्ण्णानाम्प्र‚तिप‚द‚म्भिन्नः‚{\tiny $_{lb}$}‚ स्व‚भावः क्र‚म‚रूप एक‚विक‚ल्पारूढ‚त्वात् । प‚र‚स्प‚र‚स‚हितः कार्य‚भेव‚स्यार्थ‚विष‚य‚बुद्धि‚{\tiny $_{lb}$}‚भेद‚स्य हेतुः ।
	{\color{gray}{\rmlatinfont\textsuperscript{§~\theparCount}}}
	\pend% ending standard par
      ‚{\tiny $_{lb}$}‚

	  
	  \pstart \leavevmode% starting standard par
	या चैवं कार्य‚कार‚ण‚ता ल‚क्ष‚णानुपू‚{\tiny $_{४}$}‚र्वी सा चानुपूर्वी व‚र्ण्णानां प्र‚वृत्तेत्युत्प‚न्ना‚{\tiny $_{lb}$}‚ \textbf{र‚च‚नाकृतः} पुरुषात् । र‚च‚नां क‚रोतीति \textbf{र‚च‚नाकृत्} त‚स्मात् । क‚स्मादित्याह । \textbf{इच्छे‚{\tiny $_{lb}$}‚त्यादि} । पुरुषेच्छ‚या येषां व‚र्ण्णानाम‚विरुद्धा सिद्धिस्तेषां स्थित‚स्य क्र‚म‚स्य विरोध‚तः ।
	{\color{gray}{\rmlatinfont\textsuperscript{§~\theparCount}}}
	\pend% ending standard par
      ‚{\tiny $_{lb}$}‚

	  
	  \pstart \leavevmode% starting standard par
	\textbf{कार्ये}त्यादिना व्याच‚ष्टे । \textbf{कार्य‚कार‚ण‚भूताश्च ते प्र‚त्य‚याश्चेति विग्र‚हः} ।‚{\tiny $_{lb}$}‚ व‚र्ण्ण‚स‚मुत्थाप‚क‚चित्तान्येव‚मुच्य‚न्ते । तानि हि पूर्व‚विज्ञानापेक्ष‚या कार्य‚{\tiny $_{५}$}‚भूतान्युत्त‚र‚{\tiny $_{lb}$}‚विज्ञानापेक्ष‚या कार‚ण‚भूतानि । तेभ्य \textbf{उत्प‚न्नः स्व‚भाव‚विशेष आनुपूर्वीत्युक्तं ।‚{\tiny $_{lb}$}‚ सा चा}नुपूर्वी पुरुष‚स्य यौ वित‚र्क‚विचारौ \textbf{त‚त्कृतेति कृत्वान‚स्थित‚क्र‚मा} व‚र्ण्णाः ।‚{\tiny $_{lb}$}‚ किमिद‚मिदं वेति विम‚र्शाकारो विक‚ल्पो वित‚र्क्कः । इद‚मेवेति निश्च‚याकारो‚{\tiny $_{lb}$}‚ विचारः । क‚स्मान्न स्थित‚क्र‚मा । इच्छेत्यादि । इच्छ‚या अविरुद्धा सिद्धिर्य‚स्य‚{\tiny $_{lb}$}‚ \textbf{क्र‚म‚स्य स इच्छाऽविरुद्ध‚सि‚{\tiny $_{६}$}‚द्धिः} । इच्छाविरुद्ध‚सिद्धिः क्र‚मो येषान्ते त‚थोक्ताः ।‚{\tiny $_{lb}$}‚ ‚{\tiny $_{lb}$}‚ \leavevmode\ledsidenote{\textenglish{573/s}}त‚द्भाव‚स्त‚स्मात् । त‚था हि [।] य‚थेच्छ‚म्व‚र्ण्णानां क्र‚मो व्य‚व‚स्थाप्य‚ते । किमिव‚{\tiny $_{lb}$}‚ [।] \textbf{क‚र्म‚विशेषानुक्र‚म‚व‚त्} । य‚था क‚र्म‚विशेषाणामाकुञ्च‚नादीनामिच्छा व्य‚व‚स्थितेः‚{\tiny $_{lb}$}‚ क्र‚म‚स्त‚द्व‚त् । \leavevmode\ledsidenote{\textenglish{202a/PSVTa}}
	{\color{gray}{\rmlatinfont\textsuperscript{§~\theparCount}}}
	\pend% ending standard par
      ‚{\tiny $_{lb}$}‚

	  
	  \pstart \leavevmode% starting standard par
	\textbf{न ही}त्यादिना वैध‚र्म्य‚माह । \textbf{देश‚काल‚यो}रिति देशेस्थित\edtext{}{\edlabel{pvsvt_573-1}\label{pvsvt_573-1}\lemma{देशेस्थित}\Bfootnote{One side of the leaf is torn, about 32 letters in every‚{\tiny $_{lb}$}‚ line are missing.}}{... ... ... ...}‚{\tiny $_{१}$}‚\textbf{म‚र‚च‚ना श‚क्य‚ते क‚र्त्तुं न}‚{\tiny $_{lb}$}‚ ही ति स‚म्ब‚न्धः । न हि हि म व त्स्थाने वि न्ध्यो भ‚व‚तु म ल य स्थाने विन्ध्यादि‚{\tiny $_{lb}$}‚ रित्येवं पूर्व‚म‚ङ‚कुरो भ‚व‚तु प‚श्चाद् बीजात्त‚ज्ज‚न‚क‚मिति पुरुषेच्छ‚या श‚क्य‚ते‚{\tiny $_{lb}$}‚ विप‚र्यासः क‚र्तु । व‚र्ण्णास्तु श‚क्य‚न्ते य‚थेच्छं विप‚र्यास‚यितुं । \textbf{त‚स्मान्न} व्य‚व‚स्थित‚{\tiny $_{lb}$}‚क्र‚मा व‚र्ण्णाः । त‚त ए \href{http://sarit.indology.info/?cref=}{ }{व... ... ... ... ... ...३०८ ... ... ...का}र्य‚कार‚ण‚भूते \textbf{विक‚ल्पानुक्र‚मे स‚ति} व‚र्ण्ण‚{\tiny $_{lb}$}‚क्र‚म‚स्य भावाद‚स‚ति च विक‚ल्पानुक्र‚मे व‚र्ण्ण‚क्र‚म‚स्या\textbf{भावात्} । लौकिक‚वाक्येषु \textbf{पुंसा‚{\tiny $_{lb}$}‚म्व‚र्ण‚क्र‚म‚स्य च कार्य‚कार‚ण‚तासिद्धिः} । पुरुषः कार‚णं व‚र्ण्णानुक्र‚मः कार्यः । त‚तः‚{\tiny $_{lb}$}‚ कार्य‚कार‚ण‚तासिद्धेः कार‚णाद‚न्योपि वैदिकः \textbf{स‚र्वो व‚र्ण्ण} {... ... ... ... ... ...}‚{\tiny $_{३}$}‚ ति स‚र्वो \textbf{द‚ह‚न इन्ध‚न}पूर्व‚क एवेति \textbf{युक्ति}स्त\textbf{द्व‚त्} ।
	{\color{gray}{\rmlatinfont\textsuperscript{§~\theparCount}}}
	\pend% ending standard par
      ‚{\tiny $_{lb}$}‚

	  
	  \pstart \leavevmode% starting standard par
	\textbf{स‚ती}त्यादिना व्याच‚ष्टे । \textbf{स‚तीन्ध‚ने दाह‚वृत्ते} द‚ह्य‚तेनेनेति दाहो द‚ह‚न एवोक्तः ।‚{\tiny $_{lb}$}‚ \textbf{अस‚ती}न्ध‚ने द‚ह‚न‚स्या\textbf{भावात्} । क्व‚चिद् द‚ह‚नेन्ध‚न‚योः कार्य‚कार‚ण‚भाव‚सिद्धौ स‚त्या‚{\tiny $_{lb}$}‚\textbf{म‚दृष्टेन्ध‚नोपि} य‚स्यापि द‚ह‚न‚स्येन्ध‚नं न दृष्टं सापि {... ... ... ... ... ...}‚{\tiny $_{४}$}‚त् । त‚दा \textbf{त‚स्य} द‚ह‚न‚स्याहेतो\textbf{र्देश‚{\tiny $_{lb}$}‚निय‚म‚स्य काल‚निय‚म‚स्यायोगात् ।} स‚र्व‚त्र स‚र्व‚दा भावः स्यात् । अथ निय‚मेनैव‚{\tiny $_{lb}$}‚ ‚{\tiny $_{lb}$}‚ ‚{\tiny $_{lb}$}‚ \leavevmode\ledsidenote{\textenglish{574/s}}क्व‚चिद्देशादौ भ‚व‚तीतीष्य‚ते । त‚दा देशादिनिय‚मे च द‚ह‚न‚स्येष्य‚माणे \textbf{त‚स्यैव} देशादे‚{\tiny $_{lb}$}‚\textbf{रिन्ध‚न‚त्वात्} । किं कार‚णं [।] \textbf{द‚ह‚ने}त्यादि । द‚ह‚न‚स्योपादानं क {... ... ... ... ... ...}‚{\tiny $_{५}$}‚ न‚पेक्ष‚त्वे देश‚काल‚निय‚मायोगा‚{\tiny $_{lb}$}‚\textbf{स्त‚थाय‚म‚पि व‚र्ण्ण‚क्र‚मो} हेतुभूतं \textbf{पुरुष‚प्र‚य}त्नं \textbf{य‚दि नापेक्षेत । त‚दा निराल‚म्बो} निरा‚{\tiny $_{lb}$}‚श्र‚यः पुरुष‚प्र‚य‚त्नान‚पेक्षः \textbf{स्व‚यं प्र‚काशेत} व‚र्ण्ण‚क्र‚म‚मुच्चार‚यामीत्येव‚म्पुरुष‚स्य \textbf{प्र‚य‚{\tiny $_{lb}$}‚त्नेपि न श‚क्येत} व‚र्ण्ण‚क्र‚मः प्र‚काश‚यितुं । क‚स्माद् [।] अत {त्प्र... ... ... ... ... ...}‚{\tiny $_{६}$}‚ क्र‚मे पुरुष‚स्य श‚क्तिस्त‚दा \textbf{क्व‚चि‚{\tiny $_{lb}$}‚च्छ‚क्ता}विष्य‚माणायां \textbf{स‚र्वो} वैदिकोपि व‚र्ण्ण‚क्र‚म‚स्\textbf{त‚था स्यात्} पौरुषेयः स्यात् ।‚{\tiny $_{lb}$}‚ क‚स्मात् [।] लौकिक‚वैदिक‚व‚र्ण्णानुक्र‚म‚यो\textbf{र्विशेषाभावात्} । विषाद्य‚प‚न‚य‚नादिल‚क्ष‚{\tiny $_{lb}$}‚ण‚स्य विशेष‚स्य लौकिकेष्व‚पि दृष्टेः ।
	{\color{gray}{\rmlatinfont\textsuperscript{§~\theparCount}}}
	\pend% ending standard par
      ‚{\tiny $_{lb}$}‚

	  
	  \pstart \leavevmode% starting standard par
	किञ्च \textbf{त‚द्भाव‚भाविनः} पुरुष{... ... ... ... ... ...विषा}‚{\tiny $_{७}$}‚ \edtext{\textsuperscript{*}}{\edlabel{pvsvt_574-1}\label{pvsvt_574-1}\lemma{*}\Bfootnote{Illegible. }}द्य‚प‚न‚य‚नादिना लौकिक‚व‚र्ण्ण‚क्र‚मा\textbf{द‚{\tiny $_{lb}$}‚विशिष्ट‚स्य च} वैदिक‚व‚र्ण्ण‚क्र‚म‚स्या\textbf{त‚त्कृतौ} प्र‚क‚र्षेण कृतौ । \textbf{स‚र्व‚त्रे}ति य‚त्रापि पुरुष‚{\tiny $_{lb}$}‚कृत‚त्व‚मिष्ट‚न्त‚त्रापि \textbf{कार्य‚कार‚ण‚भाव‚श्च निराकृतः स्यात्} । क‚स्माद् [।] \textbf{अन्व‚ये‚{\tiny $_{lb}$}‚त्यादि} । त‚द्भावे भावोन्व‚यः [।] त‚द‚भावेऽभावो \textbf{व्य ति रे कः} [।] \textbf{त‚ल्ल‚क्ष‚ण‚त्वा‚{\tiny $_{lb}$}‚त्त‚स्येति} का{र्य... ... ... ... ... ...}‚{\tiny $_{१}$}‚कार्य‚ता नेष्य‚ते [।] त‚दा त‚द्भाव‚भावित्व‚{\tiny $_{lb}$}‚व्य‚तिरिक्तं कार्य‚कार‚ण‚भाव‚स्य \textbf{ल‚क्ष‚णान्त‚र‚म्वाच्यं} । य‚द्विर‚हाद् वैदिकानां पुरुष‚{\tiny $_{lb}$}‚प्र‚य‚त्नेन स‚ह कार्य‚कार‚ण‚भावो न स्यात् । न चान्य‚ल्ल‚क्ष‚णं कार्य‚कार‚ण‚भाव‚स्यास्ति ।
	{\color{gray}{\rmlatinfont\textsuperscript{§~\theparCount}}}
	\pend% ending standard par
      ‚{\tiny $_{lb}$}‚

	  
	  \pstart \leavevmode% starting standard par
	अथ तुल्यें कार्य‚कार‚ण‚भाव‚ल‚क्ष‚णे लौकिको व‚र्ण्ण‚क्र‚मः कृत्रिम इष्टो थे{... ... ... ...}‚{\tiny $_{२}$}‚ः \textbf{प्र‚स‚ज}न्ति । किं कार‚णं [।]‚{\tiny $_{lb}$}‚ \textbf{त‚त्रापि} घ‚टादिष्\textbf{वेवं क‚ल्प‚नाया} वैदिक‚श‚ब्द‚क्र‚म‚व‚त् क‚ल्प‚नायाः \textbf{स‚म्भ‚वात्} । य‚था‚{\tiny $_{lb}$}‚ पुरुष‚व्यापारेण स एव वैदिकः क्र‚मो व्य‚ज्य‚त इति क‚ल्प‚ना । त‚था घ‚टाद‚योपीति ।‚{\tiny $_{lb}$}‚ ‚{\tiny $_{lb}$}‚ ‚{\tiny $_{lb}$}‚ \leavevmode\ledsidenote{\textenglish{575/s}}\textbf{विशेषाभावाच्च} । न हि पुरुष‚व्यापारानुविधायित्वेन घ‚टादेः स‚काशा{... ... ... ... ... ...}‚{\tiny $_{३}$}‚ स्त‚स्य विशेष इति चेदाह ।‚{\tiny $_{lb}$}‚ \textbf{तानि}त्यादि । \textbf{तान‚पि} घ‚टादीन् प‚रैः पूर्व‚कैः कुलालादिभिर्घ‚टादीनां र‚च‚ना । त‚स्या‚{\tiny $_{lb}$}‚\textbf{द‚र्श‚नं पूर्व‚मेव} । तान् दृष्ट‚वैवेति याव‚त् । \textbf{अन्यः} पाश्चात्योपि कुलालादिः \textbf{क‚रोति} ।‚{\tiny $_{lb}$}‚ नानुप‚देशं । प‚रोप‚देश‚पूर्विका च येषां प्र‚तिप‚त्तिस्तेषाम‚पौरुषेय‚त्वे {... ... ... ... ...वै}‚{\tiny $_{४}$}‚ दिकः क्र‚मः क‚र्त्तुर‚स्म‚{\tiny $_{lb}$}‚र‚णाद‚पौरुषेय इष्टः । \textbf{एव‚न्त‚र्हि} घ‚टाद‚योप्य\textbf{विदित‚क‚र्त्तार‚श्च केचि}त् । ब‚हुत‚र‚काला‚{\tiny $_{lb}$}‚च्छ‚न्न‚देशे घ‚टाद‚यो दृश्य‚न्ते [।] न च तेषां क‚र्त्ता स्म‚र्य‚ते । त‚स्मात् तुल्ये पुरुष‚{\tiny $_{lb}$}‚व्यापारानुविधाने य‚द्य‚पौरुषेयो वैदिक‚क्र‚म‚स्त‚दा स‚र्वेषां घ‚टादीनाम‚क्रि{... ... ... ... ... ...}‚{\tiny $_{५}$}‚य‚था वा क‚र्त्तु‚{\tiny $_{lb}$}‚र‚स्म‚र‚णाद् वैदिक‚क्र‚म‚स्याक्रिया । एवं \textbf{केषांचित्} क‚दाचित् घ‚टादीनाम‚स्म‚र्य‚माण‚{\tiny $_{lb}$}‚क‚र्त्तृकाणा\textbf{म‚क्रियाभिनिवेशोस्तु} । न चैवं [।] \textbf{त‚स्मा}न्न चैवेयं लौकिकी वैदिकी‚{\tiny $_{lb}$}‚ \textbf{च व‚र्ण्णानुपूर्वी} पुरुष‚कृतेति स‚म्ब‚न्धः । क‚स्मादिति [।] आह । \textbf{प्र‚सिद्धे}त्यादि ।‚{\tiny $_{lb}$}‚ \textbf{प्र‚सिद्धिः कार्य‚कार‚ण‚भावो} {... ... ... ... ... ...वि}‚{\tiny $_{६}$}‚शेष‚ण‚स‚मासः । तेषां \textbf{ध‚र्मः} स्व‚कार‚णान्व‚य‚व्य‚ति‚{\tiny $_{lb}$}‚रेकानुविधान‚न्त‚स्या\textbf{न‚तिक्र‚मात्} ।
	{\color{gray}{\rmlatinfont\textsuperscript{§~\theparCount}}}
	\pend% ending standard par
      ‚{\tiny $_{lb}$}‚

	  
	  \pstart \leavevmode% starting standard par
	य‚त एव पौरुषेया म‚न्त्रास्\textbf{त‚त एव} प्राकृतेभ्यः पुरुषेभ्यो \textbf{साधार‚ण‚ता} विशिष्ट‚ता‚{\tiny $_{lb}$}‚ \textbf{सिद्धा} । केषां [।] म‚न्त्राख्य\textbf{क्र‚म‚कारिणां पुंसां} । मंत्र‚संज्ञित‚म्व‚र्ण्ण‚क्र‚मं कुर्व‚न्ति‚{\tiny $_{lb}$}‚ ये । तेषां केना {... ... ... ... ... ...}‚{\tiny $_{७}$}‚म‚स्य प‚रिज्ञानं । स‚मीहितार्थ‚स‚म्पाद‚न्- \leavevmode\ledsidenote{\textenglish{203a/PSVTa}}‚{\tiny $_{lb}$}‚ श‚क्तिः \textbf{प्र‚भावः} । किं कार‚ण‚म् [।] अन्येभ्योसाधार‚ण‚तेत्याह । \textbf{अन्येषां} प्राकृता‚{\tiny $_{lb}$}‚दीनां पुरुषाणान्त‚योः । ज्ञान‚प्र‚भाव‚योर\textbf{भाव‚तः} ।
	{\color{gray}{\rmlatinfont\textsuperscript{§~\theparCount}}}
	\pend% ending standard par
      ‚{\tiny $_{lb}$}‚

	  
	  \pstart \leavevmode% starting standard par
	\textbf{अय‚मित्}यादिना व्याच‚ष्टे । \textbf{व‚र्ण्णानाम‚यं क्र‚मो विष‚निर्घातादिस‚म‚र्थो नान्य‚{\tiny $_{lb}$}‚ इन्येवं}विभागेन \textbf{य‚द्य‚न्योपि} प्राकृत‚पुरुषो \textbf{जानीयात् त‚दा} त‚म्व‚र्ण्ण‚क्र‚म\textbf{न्त‚थैव} विभागे‚{\tiny $_{lb}$}‚‚{\tiny $_{lb}$}‚ \leavevmode\ledsidenote{\textenglish{576/s}}नैव \textbf{प्र‚तिपाद्ये‚{\tiny $_{१}$}‚ता}नुतिष्ठेत् [।] \textbf{न चैवं} प्र‚तिप‚द्य‚ते [।] \textbf{त‚स्माद‚यं} म‚न्त्राख्यो‚{\tiny $_{lb}$}‚ व‚र्ण्णा\textbf{नुक्र‚मः स्व‚भाव‚तो} य‚दि नाम \textbf{कार्य‚कृद}भ्युप‚ग‚तो मी मां स कै स्त‚थापि \textbf{क‚श्चिदे}व‚{\tiny $_{lb}$}‚ पुरुषैर‚तीन्द्रिय‚श‚क्तिभेद‚युक्तो \textbf{विज्ञात इति} कृत्वास्ति \textbf{प‚रोक्षार्थ‚द‚र्शी पुरुषो} यो‚{\tiny $_{lb}$}‚ म‚न्त्र‚न्त‚त्साम‚र्थ्य‚न्त‚द‚नुष्ठान‚ञ्च वेत्ति ।
	{\color{gray}{\rmlatinfont\textsuperscript{§~\theparCount}}}
	\pend% ending standard par
      ‚{\tiny $_{lb}$}‚

	  
	  \pstart \leavevmode% starting standard par
	य‚द्य‚पि स‚त्य‚त‚पःप्र‚भाव‚व‚तां स‚मीहितार्थ‚साध‚नं व‚च‚नं म‚न्त्र इत्युक्तं [।]‚{\tiny $_{lb}$}‚ त‚थाप्य‚भ्युप‚{\tiny $_{२}$}‚ग‚म्योच्य‚ते [।] \textbf{स्व‚भाव‚तोपि} कार्य‚कृन्म‚न्त्र इति । कार्य‚कृन्म‚न्त्र‚{\tiny $_{lb}$}‚क‚र‚णेन वा म‚न्त्र‚फ‚ल‚श‚क्तिज्ञानेन वा पुरुषातिश‚य इत्युक्त‚म्भ‚व‚ति । अनुमानात्‚{\tiny $_{lb}$}‚ म‚न्त्राम‚न्त्र‚प‚रिज्ञानात् प‚रोक्ष‚द‚र्शिनोऽभाव इत्य‚त्राह । \textbf{न ही}त्यादि । \textbf{अय‚म‚र्थः}‚{\tiny $_{lb}$}‚ [।] व‚र्ण्णानुक्र‚म‚ल‚क्ष‚णो विषाद्य‚प‚न‚य‚ने \textbf{स‚म‚र्थो}ऽय‚न्तु व‚र्ण्णानुक्र‚मो न स‚म‚र्थ \textbf{इति} ।‚{\tiny $_{lb}$}‚ एवं \textbf{न हि श‚क्यं} लिङ्गात् \textbf{प्र‚त्येतुं} । किं कार‚णं [।] म‚{\tiny $_{३}$}‚न्त्राम‚न्त्र‚विभागेना\textbf{संकीर्ण्ण‚स्य‚{\tiny $_{lb}$}‚ लिङ्ग‚विशेष‚स्यासिद्धेः} ।
	{\color{gray}{\rmlatinfont\textsuperscript{§~\theparCount}}}
	\pend% ending standard par
      ‚{\tiny $_{lb}$}‚

	  
	  \pstart \leavevmode% starting standard par
	न‚नु च व‚र्ण्ण‚रूप‚योर्म‚न्त्राम‚न्त्र‚योः प्र‚त्य‚क्षेण ग्र‚ह‚णे फ‚ल‚दान‚शंक्तिर‚पि त‚द‚व्य‚ति‚{\tiny $_{lb}$}‚रेकात् प्र‚त्य‚क्ष‚गृहीतैवेति [।]
	{\color{gray}{\rmlatinfont\textsuperscript{§~\theparCount}}}
	\pend% ending standard par
      ‚{\tiny $_{lb}$}‚

	  
	  \pstart \leavevmode% starting standard par
	अत आह । \textbf{प्र‚त्य‚क्ष‚यो}रित्यादि । \textbf{अनुप‚दिष्ट‚यो}रित्य‚यं म‚न्त्रो नाय‚म्म‚न्त्र‚{\tiny $_{lb}$}‚ इत्येव‚म‚क‚थित‚यो\textbf{र‚प‚रिज्ञानात्} । अय‚म्विषाद्य‚प‚न‚य‚ने श‚क्तोऽय‚म‚श‚क्त इत्येव‚{\tiny $_{lb}$}‚म‚निश्च‚यात् । \textbf{उप‚देशे‚{\tiny $_{४}$}‚पी}ति [।] य‚दि नाम केन‚चिदुप‚दिष्ट‚म्भ‚व‚त्य‚यं म‚न्त्र एवं‚{\tiny $_{lb}$}‚कार्य‚कारीति । त‚थापि क‚थंचित् \textbf{केन‚चिद‚प्याकारेण} म‚न्त्र‚स्य यः \textbf{स्व‚भाव‚विवेक}‚{\tiny $_{lb}$}‚स्त‚स्या\textbf{प्र‚तीतेर}निश्च‚यात् । उप‚देश‚स्याप्रामाण्यात् । \textbf{अन्य‚त्र कार्य‚स‚म्वादात्} ।‚{\tiny $_{lb}$}‚ म‚न्त्र‚साध्य‚कार्य‚प्राप्त्या त्व‚य‚म्म‚न्त्र इति निश्च‚यः स्यात् । \textbf{त‚स्य च} म‚न्त्र‚स्य साध्य‚स्य‚{\tiny $_{lb}$}‚ \textbf{कार्य‚स्य} क‚र‚णात् \textbf{प्रागि}ति म‚न्त्रानुष्ठात् प्राग् \textbf{द्र‚ष्टुम‚श‚क्य‚त्वात्} ।‚{\tiny $_{५}$}‚ त‚दानीन्त‚स्य‚{\tiny $_{lb}$}‚ स‚त्वादिति भावः । न चानिश्चित‚स्य म‚न्त्र‚स्यानुष्ठानं स‚म्भ‚व‚ति । त‚स्माद‚व‚श्यं‚{\tiny $_{lb}$}‚ मी मां स केनातीन्द्रियार्थ‚द‚र्शी पुरुषोभ्युप‚ग‚न्त‚व्यो यो म‚न्त्राम‚न्त्र‚स्व‚भावं विवेच‚य‚ति‚{\tiny $_{lb}$}‚ [।] अन्य‚था म‚न्त्रानुष्ठानं न स्यात् ।
	{\color{gray}{\rmlatinfont\textsuperscript{§~\theparCount}}}
	\pend% ending standard par
      ‚{\tiny $_{lb}$}‚

	  
	  \pstart \leavevmode% starting standard par
	\textbf{न चाय‚म्}म‚न्त्राख्यो व‚र्ण्ण\textbf{क्र‚मः स्व‚भाव‚तो} विषाद्य‚प‚न‚य‚नादेः \textbf{कार‚कः} किन्तु‚{\tiny $_{lb}$}‚ ‚{\tiny $_{lb}$}‚ \leavevmode\ledsidenote{\textenglish{577/s}}पुरुष‚स‚म‚यात् । किं कार‚णं [।] \textbf{क‚स्य‚चि}न्म‚न्त्रानुष्ठायिनः पु‚{\tiny $_{६}$}‚रुष‚स्याशुम‚न्त्र‚सिद्धेः‚{\tiny $_{lb}$}‚ \textbf{अन्य‚स्य} त‚त्तुल्य‚म‚न्त्र‚क्रियानुष्ठायिनः \textbf{चिरात् सिद्धेः} । स्व‚तो हि फ‚ल‚दानेऽयं काल‚{\tiny $_{lb}$}‚भेदो न युज्य‚ते म‚न्त्र‚स्य केन‚चिदुत्क‚र्षानुत्क‚र्षाक‚र‚णात् । \textbf{अन्य‚स्य} पुरुष‚स्य म‚न्त्रात्‚{\tiny $_{lb}$}‚ फ‚ल‚निष्प‚त्तौ \textbf{व्र‚त‚च‚र्याद्य‚पेक्ष‚णान्न} स्व‚भाव‚तो म‚न्त्रः कार‚कः । व्र‚तं म‚न्त्र‚क‚ल्प‚विहितो‚{\tiny $_{lb}$}‚ निय‚म‚स्त‚स्य च‚र‚ण‚म‚नुष्ठानं । अपिश‚ब्दाद्धोमादिप‚{\tiny $_{७}$}‚रिग्र‚हः । स्व‚भाव‚तो हि \leavevmode\ledsidenote{\textenglish{203b/PSVTa}}‚{\tiny $_{lb}$}‚ फ‚ल‚दाने । किं व्र‚त‚च‚र्याद्य‚नुष्ठानेनापेक्षितेन । त‚तोतिश‚यानुत्प‚त्तेः । \textbf{त‚थैक‚स्मा‚{\tiny $_{lb}$}‚द‚पि} म‚न्त्र‚विष‚याज्ज‚प‚होमादि\textbf{क‚र्म्म‚णः} स‚काशात् \textbf{क‚योश्चित्} पुरुष‚योस्तुल्यं विधि‚{\tiny $_{lb}$}‚म‚नुतिष्ठ‚तोर‚प्येक‚स्या\textbf{र्थ}द‚र्श‚नाद‚न्य‚स्य्\textbf{आन‚र्थ‚द‚र्श‚ना}न्न स्व‚भाव‚तः फ‚लं । स्व‚भाव‚{\tiny $_{lb}$}‚त‚स्तु फ‚ल‚दाने तुल्योर्थ‚योगः स्यात् । \textbf{व‚ह‚ताम‚पी}ति [।] विषाद्य‚प‚{\tiny $_{१}$}‚न‚य‚नं कुर्व‚{\tiny $_{lb}$}‚ताम‚पि \textbf{म‚न्त्राणां} पुनः कालान्त‚रे तेनैव पुरुषेण प्र‚युक्तानां \textbf{क‚स्य‚चित्} कार्य‚स्य \textbf{विस‚{\tiny $_{lb}$}‚म्वादा}द‚क‚र‚णात् । \textbf{न ह्य‚यं प्र‚कारो} व‚स्तु\textbf{स्व‚भावे युक्तः} [।] किं कार‚णं [।] \textbf{स्व‚भा‚{\tiny $_{lb}$}‚व‚स्य स‚र्व‚त्र} पुरुषादौ \textbf{तुल्य‚त्वात्} । यो हि य‚स्य स्व‚भावो न स क‚ञ्चिद‚पेक्ष्य स्व‚भावो‚{\tiny $_{lb}$}‚ भ‚व‚ति । य‚दा तु पुरुष‚कृतात् स‚म‚यात् फ‚ल‚मिष्य‚ते त‚दाय‚म‚दोषः । त‚था हि [।]‚{\tiny $_{lb}$}‚ \textbf{पुरुष‚स्तु} म‚{\tiny $_{२}$}‚न्त्र‚स‚म‚य‚स्य क‚र्त्ता । \textbf{स्वेच्छावृत्ति}रिति कृत्वा \textbf{कंचित्} पुरुष‚म\textbf{नुगृह्णाति‚{\tiny $_{lb}$}‚ नाप‚र‚मिति युक्तं} । केन कार‚णेन । \textbf{स‚त्त्व‚स‚भाग‚ताविशेषात्} । म‚न्त्र‚क्रिया‚{\tiny $_{lb}$}‚नुष्ठातुः स‚त्त्व‚स्य म‚न्त्र‚प्र‚णेत्रा स‚ह स‚भाग‚ता तुल्य‚शीलाचारादिना । आदिश‚ब्दा‚{\tiny $_{lb}$}‚दुप‚प्र‚दानादिप‚रिग्र‚हः । त‚द्व‚शात्त‚द‚नुरागात् । \textbf{सेवाविशेषाद्वा} । ज‚प‚होमादिना‚{\tiny $_{lb}$}‚ देव‚ताराध‚नं सेवाविशेष‚स्त‚{\tiny $_{३}$}‚स्माद्वा । \textbf{क‚ञ्चिद‚नुगृह्णाति} पुरुषं । \textbf{नाप‚रं} स‚त्त्व‚{\tiny $_{lb}$}‚स‚भाग‚तादिर‚हितं ।
	{\color{gray}{\rmlatinfont\textsuperscript{§~\theparCount}}}
	\pend% ending standard par
      ‚{\tiny $_{lb}$}‚

	  
	  \pstart \leavevmode% starting standard par
	\textbf{व्र‚ते}त्यादिना प‚र‚माशंक‚ते । निय‚म‚स्यानुष्ठानं \textbf{व्र‚त‚च‚र्या} । व्र‚त‚च‚र्या च व्र‚त‚{\tiny $_{lb}$}‚च‚र्या\textbf{भ्र‚ङ‚श}श्चेति विरूपैक‚शेषः । \textbf{आदि}श‚ब्दः प्र‚त्येक‚म‚भिस‚म्ब‚ध्य‚ते । व्र‚त‚च‚र्या‚{\tiny $_{lb}$}‚दिना व्र‚त‚च‚र्याभ्रंशादिना चेत्य‚र्थः । व्र‚त‚च‚र्यादिना \textbf{ध‚र्मो}प‚च‚ये स‚ति सिद्धिरिति‚{\tiny $_{lb}$}‚ स‚म्ब‚न्धः । त‚था \textbf{प्र‚{\tiny $_{४}$}‚कृत्या} स्व‚भावेन \textbf{ध‚र्मा}त्म‚नो वा पुंसः \textbf{सिद्धिः} । व्र‚त‚च‚र्याभ्र‚ङ्‚{\tiny $_{lb}$}‚शादिना त्व‚ध‚र्मोप‚च‚ये स‚ति । \textbf{अध‚र्मा}त्म‚नो वा प्रा\textbf{कृत्या} पुरुष‚स्या\textbf{सिद्धि}रिति वाक्यार्थो‚{\tiny $_{lb}$}‚ योज्यः । स्व‚भाव‚तोपि म‚न्त्रात् फ‚ल‚निष्प‚त्तौ य‚थोक्तेन प्र‚कारेण \textbf{सिद्ध्}य‚सिद्धिभेदो‚{\tiny $_{lb}$}‚ भ‚विष्य‚तीति प‚रो म‚न्य‚ते ।
	{\color{gray}{\rmlatinfont\textsuperscript{§~\theparCount}}}
	\pend% ending standard par
      ‚{\tiny $_{lb}$}‚‚{\tiny $_{lb}$}‚\textsuperscript{\textenglish{578/s}}

	  
	  \pstart \leavevmode% starting standard par
	\textbf{नेत्या}दिना प्र‚तिषेध‚ति । न ध‚र्मापेक्षात् म‚न्त्रात् फ‚ल‚सिद्धिः । किं कार‚णं‚{\tiny $_{lb}$}‚ [।] \textbf{ध‚र्म‚विरुद्धान‚म‚{\tiny $_{५}$}‚पि} म‚न्त्र‚सिद्धिहेतूनां व्र‚तानां \textbf{डा कि नी भ गि नी त‚न्त्रादिषु‚{\tiny $_{lb}$}‚ द‚र्श‚नात्} । डा कि नी त‚न्त्रे । च‚तुर्भ‚गिनीत‚न्त्रे । आदिश‚ब्दात् । चौर्य‚हेतुषु क‚म्बु‚{\tiny $_{lb}$}‚कि नी त‚न्त्रादिषु द‚र्श‚नात् । कानि पुन‚स्तानि ध‚र्म‚विरुद्धानीत्याह । क्रौर्येत्यादि ।‚{\tiny $_{lb}$}‚ \textbf{क्रौर्य} प्राणिब‚धः । \textbf{स्तेयं} चौर्यं । \textbf{द्विं\edtext{}{\lemma{द्विं}\Bfootnote{? द्वीं}}द्रिय‚स‚माप‚त्ति}र्मैथुनं । \textbf{हीन‚क‚र्म} मार्जारा‚{\tiny $_{lb}$}‚शुचिधूम‚प्र‚दानादि । आदिश‚ब्दाद‚न्य‚स्यापि ध‚र्म‚{\tiny $_{६}$}‚विरुद्ध‚स्य ग्र‚ह‚णं । तानि क्रौर्या‚{\tiny $_{lb}$}‚दीनि \textbf{ब‚हुलानि} भूयांसि येषां ब्र‚तानान्तानि त‚थोक्तानि । \textbf{तैश्च} त‚थोक्तैर्व्र‚तैर्म‚न्त्र‚{\tiny $_{lb}$}‚\textbf{सिद्धिविशेषात्} । त‚था हि [।] डा कि नी त‚न्त्रे स‚म‚य‚व्य‚व‚स्था । य‚दा प्राणिनं‚{\tiny $_{lb}$}‚ ह‚त्वा खाद‚ति त‚दा म‚न्त्र‚सिद्धिमासाद‚य‚ति । त‚था क‚म्बु कि नी त‚न्त्रे स्तेयाच‚र‚{\tiny $_{lb}$}‚\leavevmode\ledsidenote{\textenglish{204a/PSVTa}} णात् सिद्धिरुक्ता । त‚था मैथुनाच‚र‚णात् सिद्धिप्र‚दा काचिद्देव‚तेति भ‚गिनी‚{\tiny $_{७}$}‚त‚न्त्रा‚{\tiny $_{lb}$}‚न्त‚रे क्व‚चित् स‚म‚यः ।
	{\color{gray}{\rmlatinfont\textsuperscript{§~\theparCount}}}
	\pend% ending standard par
      ‚{\tiny $_{lb}$}‚

	  
	  \pstart \leavevmode% starting standard par
	क्रौर्याद्येव ध‚र्मो भ‚विष्य‚तीति चेद् [।]
	{\color{gray}{\rmlatinfont\textsuperscript{§~\theparCount}}}
	\pend% ending standard par
      ‚{\tiny $_{lb}$}‚

	  
	  \pstart \leavevmode% starting standard par
	आह । \textbf{न चे}त्यादि । \textbf{एव‚म्विधो} क्रौर्यादिल‚क्ष‚णो \textbf{ध‚र्म‚स्व‚भाव इति य‚था‚{\tiny $_{lb}$}‚व‚स‚रं} प‚श्चा\textbf{न्निवेद‚यिष्यामः} । एव‚न्ताव‚द‚ध‚र्माद‚पि सिद्धिर्दृष्टा ।
	{\color{gray}{\rmlatinfont\textsuperscript{§~\theparCount}}}
	\pend% ending standard par
      ‚{\tiny $_{lb}$}‚

	  
	  \pstart \leavevmode% starting standard par
	ध‚र्माद‚पि सिद्धिर्न दृष्टेत्याह । \textbf{मैत्री}त्यादि । स‚त्वानां हित‚सुख‚चिन्त‚न‚{\tiny $_{lb}$}‚म्मै त्री । \textbf{शौ चं} द्विविधं । बाह्य‚मान्त‚र‚ञ्च । बाह्यं स्नानादि । आन्त‚रं स्तेयादि‚{\tiny $_{lb}$}‚निवृत्तिः । दा‚{\tiny $_{१}$}‚नादिना प‚रानुग्र‚हो ध‚र्मः । मैत्रीशौच‚ध‚र्माः प‚रे प्र‚दानानि येषान्ते‚{\tiny $_{lb}$}‚ त‚थोक्ताः । त‚थाभूतानां पुरुषाणां । \textbf{त‚न्निमित्त‚मेव} मैत्र्यादिक‚मेव \textbf{निमित्तं} कृत्वा‚{\tiny $_{lb}$}‚ \textbf{क‚स्याश्चित् सिद्धे}रिति मैत्री\textbf{विप‚र्य‚येण} या ल‚भ्या त‚स्या \textbf{असिद्धेः । विप‚र्य‚येण च}‚{\tiny $_{lb}$}‚ द्वेषादिना पुनः \textbf{सिद्धेः} । न ध‚र्मोप‚च‚यापेक्षात् म‚न्त्रात् फ‚ल‚सिद्धिरिति । द्वेषादि‚{\tiny $_{lb}$}‚स‚मुत्थितोपि क्रौर्यादिर्म‚न्त्र‚वि‚{\tiny $_{२}$}‚धानेनानुष्ठित‚स्स‚न् ध‚र्म एवेति [।]
	{\color{gray}{\rmlatinfont\textsuperscript{§~\theparCount}}}
	\pend% ending standard par
      ‚{\tiny $_{lb}$}‚

	  
	  \pstart \leavevmode% starting standard par
	अत आह । \textbf{न चे}त्यादि । एक‚रूपाद्धि सादिल‚क्ष‚णात् \textbf{क‚र्म‚णः} स‚काशात्‚{\tiny $_{lb}$}‚ \textbf{स त‚द्विरोध्य}ध‚र्म‚विरोधी \textbf{ध‚र्मो युक्तः} । अध‚र्म‚श्चेति म‚न्त्र‚विधानाद‚न्य‚त्र त‚त एव‚{\tiny $_{lb}$}‚ हिंसादेः स‚काशाद‚ध‚र्म‚श्च \textbf{न युक्तः} । न ह्य‚ध‚र्म‚हेतोर्ध‚र्मो भ‚व‚ति विरोधात् । त‚था हि‚{\tiny $_{lb}$}‚ [।] येनैव द्वेषाद्याश‚येन म‚न्त्र‚विधानाद‚न्य‚त्र हिंसादिकुर्व‚तोऽध‚र्मो भ‚व‚ति । ‚{\tiny $_{३}$}‚तेनै‚{\tiny $_{lb}$}‚वाश‚येन म‚न्त्र‚विधानानुष्ठानेपि हिंसादिकं क्रिय‚त इति क‚थ‚न्त‚स्य ध‚र्माङ्ग‚त्व‚मिति ।
	{\color{gray}{\rmlatinfont\textsuperscript{§~\theparCount}}}
	\pend% ending standard par
      ‚{\tiny $_{lb}$}‚‚{\tiny $_{lb}$}‚\textsuperscript{\textenglish{579/s}}

	  
	  \pstart \leavevmode% starting standard par
	य‚दि द्वेषादिकृत‚त्वान्म‚न्त्र‚विधानेनानुष्ठितोपि क्रौर्यादिर‚ध‚र्म एव \textbf{क‚थ‚मिदानी‚{\tiny $_{lb}$}‚म‚ध‚र्मात्म‚नो व्र‚तादेरि}ति पूर्वोक्तात् । क्रौर्यादिल‚क्ष‚णाद् व्र‚तांत् । आदिश‚ब्दाद‚{\tiny $_{lb}$}‚न्य‚स्मान्निहीनाद‚शुचिधूप‚दानादिल‚क्ष‚णात् । \textbf{ध‚र्म}स्य \textbf{फ‚ल‚मि}ष्ट‚स‚म्भोगादिल‚क्ष‚णं‚{\tiny $_{lb}$}‚ \textbf{क‚थ‚म‚श्नुते‚{\tiny $_{४}$}‚} भ‚ज‚ते जापी ।
	{\color{gray}{\rmlatinfont\textsuperscript{§~\theparCount}}}
	\pend% ending standard par
      ‚{\tiny $_{lb}$}‚

	  
	  \pstart \leavevmode% starting standard par
	\textbf{ने}त्यादि सि द्धा न्त वा दी । \textbf{न चैत‚स्या}ध‚र्मात्म‚नो \textbf{व्र‚तादेस्त‚दिष्टं फ‚ल‚म्वि‚{\tiny $_{lb}$}‚पाकः} क‚र्म‚फ‚लं । \textbf{किन्तु पूर्व‚कृत‚स्य} शुभ‚स्य क‚र्म‚णो विपाकः [।] किम्व‚दिति [।]‚{\tiny $_{lb}$}‚ आह । \textbf{ब्र‚ह्म‚ह‚त्या}या \textbf{आदेश}स्त‚स्य्\textbf{आनुष्ठानात्} स‚म्पाद‚नात् । \textbf{ग्राम‚प्र‚तिल‚म्भ‚व‚त्} ।‚{\tiny $_{lb}$}‚ य‚था क‚श्चित् क‚ञ्चित् पुरुषं नियुंक्ते । मार‚येमं ब्राह्म‚णं अह‚न्ते ग्राम‚न्दास्यामीति ।‚{\tiny $_{lb}$}‚ स त‚स्यादेश‚स्यानुष्ठाना‚{\tiny $_{५}$}‚द् ग्रामं प्र‚तिल‚भ‚ते । न च त‚द् ब्र‚ह्म‚ह‚त्याया फ‚लं । किन्तु‚{\tiny $_{lb}$}‚ त‚द्ब्रह्मह‚त्याच‚र‚णेनाराधितं पुरुषं स‚ह‚कारिणं प्राप्य पूर्व‚कं शुभ‚मेव क‚र्म त‚था‚{\tiny $_{lb}$}‚ फ‚ल‚ति ।
	{\color{gray}{\rmlatinfont\textsuperscript{§~\theparCount}}}
	\pend% ending standard par
      ‚{\tiny $_{lb}$}‚

	  
	  \pstart \leavevmode% starting standard par
	अध‚र्मात्म‚न‚स्त‚र्हि त‚स्य व्र‚त‚स्य किंफ‚ल‚मित्याह । \textbf{त‚स्य तु} क्रौर्यादिल‚क्ष‚ण‚स्या‚{\tiny $_{lb}$}‚\textbf{ध‚र्मात्म‚नो व्र‚त‚स्यागामि} भ‚विष्य‚ज्ज‚न्भ‚भावि फ‚ल‚मिष्टं न‚र‚कादि ।
	{\color{gray}{\rmlatinfont\textsuperscript{§~\theparCount}}}
	\pend% ending standard par
      ‚{\tiny $_{lb}$}‚

	  
	  \pstart \leavevmode% starting standard par
	य‚दि शुभ‚स्य क‚र्म‚ण इष्टं फ‚लं कि\textbf{न्तेना}ध‚र्मात्म‚ना \textbf{म‚न्त्रादिप्र‚{\tiny $_{६}$}‚योगेणा}पेक्षि‚{\tiny $_{lb}$}‚तेनेति चेद् [।]
	{\color{gray}{\rmlatinfont\textsuperscript{§~\theparCount}}}
	\pend% ending standard par
      ‚{\tiny $_{lb}$}‚

	  
	  \pstart \leavevmode% starting standard par
	आह । स त्व‚ध‚र्मात्मा । डा कि नी \textbf{म‚न्त्रादिप्र‚योगः} । आदिश‚ब्दात् क्रौर्यादि‚{\tiny $_{lb}$}‚व्र‚त‚प्र‚योगः । \textbf{त‚स्येष्ट‚फ‚ल‚स्य} शुभ‚स्य \textbf{क‚र्म‚णः} । इष्ट‚म्फ‚लं य‚स्य क‚र्म‚ण इति विग्र‚हः ।‚{\tiny $_{lb}$}‚ \textbf{क‚थ‚ञ्चित्} केन‚चित् \textbf{प्र‚कारेणोप‚कारात् पाच‚कः} फ‚ल‚स्य दाय‚कः ।
	{\color{gray}{\rmlatinfont\textsuperscript{§~\theparCount}}}
	\pend% ending standard par
      ‚{\tiny $_{lb}$}‚

	  
	  \pstart \leavevmode% starting standard par
	क‚थं पुनः कुश‚ल‚स्याकुश‚ल‚मुप‚कार‚क‚म्भ‚व‚तीति [।]
	{\color{gray}{\rmlatinfont\textsuperscript{§~\theparCount}}}
	\pend% ending standard par
      ‚{\tiny $_{lb}$}‚

	  
	  \pstart \leavevmode% starting standard par
	आह । \textbf{चित्र‚त्वादुप‚कार‚श‚क्तेः} ।‚{\tiny $_{७}$}‚ स‚ह‚कारिभावो हि चित्रः । क‚दाचित् कुश- \leavevmode\ledsidenote{\textenglish{204b/PSVTa}}‚{\tiny $_{lb}$}‚ ल‚स्याकुश‚लं स‚ह‚कारि । अकुश‚ल‚स्यापि कुश‚लं । य‚थात्य‚र्थ‚मुदार‚कुश‚ल‚कारिणो‚{\tiny $_{lb}$}‚ न न‚र‚कादिदुःख‚फ‚ल‚म‚शुभं क‚र्म कुश‚लं स‚ह‚कारि प्राप्येहैव ज‚न्म‚नि । व्याध्यादि‚{\tiny $_{lb}$}‚दुःख‚मात्रं द‚त्वा क्षीय‚ते । अध‚र्मात्म‚को म‚न्त्रादिप्र‚योगः । क‚थंचिदिष्ट‚फ‚ल‚स्य क‚र्म‚ण‚{\tiny $_{lb}$}‚ उप‚कार‚क इत्युक्तं ।
	{\color{gray}{\rmlatinfont\textsuperscript{§~\theparCount}}}
	\pend% ending standard par
      ‚{\tiny $_{lb}$}‚

	  
	  \pstart \leavevmode% starting standard par
	य‚त्र येन प्र‚कारेणोप‚कार‚स्त‚न्द‚र्श‚य‚न्नाह । \textbf{पुरुष‚{\tiny $_{१}$}‚विशेषो} म‚न्त्र‚स्य प्र‚णेता स‚{\tiny $_{lb}$}‚ ‚{\tiny $_{lb}$}‚ \leavevmode\ledsidenote{\textenglish{580/s}}एवाश्र‚य‚स्तेन विपाकः फ‚ल‚दानं स एव ध‚र्मः स्व‚भावो य‚स्य स पुरुषा\textbf{श्र‚य‚विपा‚{\tiny $_{lb}$}‚क‚ध‚र्मा । ध‚र्म} इति पुण्य‚विशेषः । तेनेति पुरुषेण \textbf{कृतः} पूर्व‚ज‚न्म‚नि । \textbf{स त‚था त‚दा‚{\tiny $_{lb}$}‚राध‚ने}नेति । स ध‚र्म‚स्त‚था क्रौर्याद्याच‚र‚णात् । त‚दाराध‚नेन म‚न्त्र‚प्र‚णेतृपुरुषाराध‚नेन‚{\tiny $_{lb}$}‚ फ‚ल‚तीति ।
	{\color{gray}{\rmlatinfont\textsuperscript{§~\theparCount}}}
	\pend% ending standard par
      ‚{\tiny $_{lb}$}‚

	  
	  \pstart \leavevmode% starting standard par
	एत‚देव स्प‚ष्ट‚य‚न्नाह । \textbf{त‚दि}त्यादि । तेन म‚न्त्र‚प्र‚णेत्रा पुरु‚{\tiny $_{२}$}‚षेण विहिता क्रोर्य‚{\tiny $_{lb}$}‚युक्त‚व्र‚तादिप्र‚योग‚स्ते\textbf{नोप‚कारः} । क‚र्म‚णः प‚रिपोषः । तेन विपाकः फ‚ल‚दानं ध‚र्मः‚{\tiny $_{lb}$}‚ स्व‚भावो य‚स्य त‚स्यैवंध‚र्म‚णः । \textbf{त‚त्फ‚ल}स्येति । इष्ट‚फ‚ल‚स्य \textbf{क‚र्म‚णः कृत‚त्वात्}‚{\tiny $_{lb}$}‚ कार‚णात् । त‚दाराध‚नेन फ‚ल‚तीति ।
	{\color{gray}{\rmlatinfont\textsuperscript{§~\theparCount}}}
	\pend% ending standard par
      ‚{\tiny $_{lb}$}‚

	  
	  \pstart \leavevmode% starting standard par
	विनापि म‚न्त्र‚प्र‚णेत्रा पुरुवेण त‚दुप‚कारात् । त‚स्यैव य‚न्त्र‚स्य केव‚ल‚स्य ज‚पादिना‚{\tiny $_{lb}$}‚ व्यापारेणोप‚कारान्म‚न्त्रात्फ‚{\tiny $_{३}$}‚ल‚मिति चेत । त‚त‚श्च पुरुषाराध‚नेन फ‚ल‚तीति य‚दु‚{\tiny $_{lb}$}‚क्त‚न्त‚द‚युक्त‚मिति प‚रो म‚न्य‚ते ।
	{\color{gray}{\rmlatinfont\textsuperscript{§~\theparCount}}}
	\pend% ending standard par
      ‚{\tiny $_{lb}$}‚

	  
	  \pstart \leavevmode% starting standard par
	\textbf{ने}त्यादि सि द्धा न्त वा दी । \textbf{नैत‚देवं} । किङ्कार‚णं [।] पुरुषेत्यादि । म‚न्त्र‚प्र‚णेतुः‚{\tiny $_{lb}$}‚ \textbf{पुरुष‚स्याकारो} व‚र्ण्ण‚संस्थानादिः । \textbf{स्व‚भावः} शान्त‚रोद्रादिः । \textbf{च‚र्या} काय‚वाग्व्यापार‚{\tiny $_{lb}$}‚ल‚क्ष‚णा चेष्टा । तेषामाकारादीनाम‚धि\textbf{मोक्षो}ऽधिमुक्तिस्त‚स्या \textbf{वैय‚र्थ्य‚प्र‚स‚ङ्गात्} ।‚{\tiny $_{lb}$}‚ म‚न्त्र‚स्याधिष्ठाता पुरुष‚श्चेन्ना‚{\tiny $_{४}$}‚स्ति । किम‚र्थ‚न्त‚स्याकारादीन‚धिमुच्येत् । अथेष्य‚ते‚{\tiny $_{lb}$}‚ त‚स्यापि पुरुष‚स्यो\textbf{प‚कार‚क‚त्व‚न्त‚दा} त‚स्यापि म‚न्त्र‚प्र‚णेतुः पुरुष‚स्य म‚न्त्रात् फ‚ल‚{\tiny $_{lb}$}‚सिद्धिं प्र‚त्युप‚कार‚क‚त्वेऽङ्गीक्रिय‚माणे सिद्धः पुरुष‚विशेषोसाधार‚ण‚गुणः । असाधार‚णा‚{\tiny $_{lb}$}‚ गुणा अस्येति विग्र‚हः । किं कार‚णं [।] \textbf{त‚द‚धिमुक्तेरेव} पुरुष‚विशेषाकार‚स्व‚भाव‚{\tiny $_{lb}$}‚च‚र्याधिमुक्तेरेव विष‚क‚र्मादिक‚{\tiny $_{५}$}‚र‚णात् । त‚स्मान्न म‚न्त्राः पुरुष‚प्र‚णीता अपि त‚दुप‚{\tiny $_{lb}$}‚योग‚निर‚पेक्षाः पुरुष‚विशेषोप‚योग‚निर‚पेक्षाः स्व‚भावेन प्र‚कृत्यैव फ‚ल‚दाः । किन्तु‚{\tiny $_{lb}$}‚ पुरुष‚विशेषोप‚योग‚सापेक्षा एव ।
	{\color{gray}{\rmlatinfont\textsuperscript{§~\theparCount}}}
	\pend% ending standard par
      ‚{\tiny $_{lb}$}‚

	  
	  \pstart \leavevmode% starting standard par
	य‚द्य‚साधार‚ण‚गुण एव पुरुषो म‚न्त्र‚स्य प्र‚णेता । क‚थं प्र‚भावादिविशेष‚र‚हिता‚{\tiny $_{lb}$}‚ अपि त‚न्त्र‚विदो म‚न्त्रान् भाष‚न्त इति [।]
	{\color{gray}{\rmlatinfont\textsuperscript{§~\theparCount}}}
	\pend% ending standard par
      ‚{\tiny $_{lb}$}‚‚{\tiny $_{lb}$}‚\textsuperscript{\textenglish{581/s}}

	  
	  \pstart \leavevmode% starting standard par
	अत आह । \textbf{येपी}त्यादि । म‚न्त्र‚प्र‚तिब‚द्धा‚{\tiny $_{६}$}‚नि शास्त्राणि त‚न्त्राणि तानि‚{\tiny $_{lb}$}‚ \textbf{विद‚न्तीति ते त‚न्त्र‚विदः केचिद}द्य‚त्वेपि \textbf{म‚न्त्रा}न‚पूर्वान् \textbf{कांश्च‚न कुर्व‚ते} । त‚न्न \textbf{तेषां}‚{\tiny $_{lb}$}‚ केव‚लानाम‚साम‚र्थ्यं । किन्तु य‚त् त‚न्त्र‚माश्रितास्ते त‚स्य त‚न्त्र‚स्य प्र‚णेता यः पुरुषा‚{\tiny $_{lb}$}‚\textbf{तिश‚य‚स्त‚स्य प्र‚भोः} स्वामिनः स \textbf{प्र‚भावः} साम‚र्थ्यं । क‚स्मात् [।] \textbf{त‚दुक्त‚न्याय‚वृत्तितः} ।‚{\tiny $_{lb}$}‚ त‚स्मात् प्र‚भुणा य‚स्तेभ्य‚स्स‚म‚यादिको न्याय उप‚दिष्ट‚स्त‚स्यानुव‚र्त्त‚नात् । प्र‚भु‚{\tiny $_{७}$}‚- \leavevmode\ledsidenote{\textenglish{205a/PSVTa}}‚{\tiny $_{lb}$}‚ स्तुष्ट‚स्त‚त्प्र‚णीतान‚पि म‚न्त्रान‚धितिष्ठ‚तीति भावः ।
	{\color{gray}{\rmlatinfont\textsuperscript{§~\theparCount}}}
	\pend% ending standard par
      ‚{\tiny $_{lb}$}‚

	  
	  \pstart \leavevmode% starting standard par
	\textbf{र‚थ्यापुरुषे}त्यादिना पूर्व‚प‚क्षोप‚न्यास‚पूर्व‚कं कारिकार्थं व्याच‚ष्टे । \textbf{र‚थ्यापुरुषा‚{\tiny $_{lb}$}‚ अपी}ति सामान्य‚पुरुषा अपि \textbf{केच‚न} गारुडिक‚प्र‚भृत‚यो म‚न्त्र‚ल‚क्ष‚ण\textbf{त‚न्त्र‚ज्ञाः किञ्चिद्}‚{\tiny $_{lb}$}‚ विषादिश‚म‚न‚ल‚क्ष‚णं \textbf{क‚र्म कुर्व‚न्ति} । न च ते विशिष्टाः सुरापानाद्य‚नुष्ठानात् ।‚{\tiny $_{lb}$}‚ \textbf{त‚थान्यो}पीति प्र‚भाव‚व‚त्त्वेनाभिम‚तो \textbf{म‚न्त्र‚स्य प्र‚{\tiny $_{१}$}‚णेताऽन‚तिश‚य‚श्च स्यात्} । र‚थ्या‚{\tiny $_{lb}$}‚पुरुष‚व‚द‚तिश‚य‚र‚हित‚श्च स्यात् म‚न्त्राणां च क‚र्त्तेति । त‚था च प्र‚भाव‚वान् पुरुषो न‚{\tiny $_{lb}$}‚ सिध्य‚तीति म‚न्य‚ते ।
	{\color{gray}{\rmlatinfont\textsuperscript{§~\theparCount}}}
	\pend% ending standard par
      ‚{\tiny $_{lb}$}‚

	  
	  \pstart \leavevmode% starting standard par
	नेत्यादिना प‚रिह‚र‚ति । न प्र‚भाव‚र‚हितानां म‚न्त्र‚क‚र‚णं [।] ये तु र‚थ्यापुरुषा‚{\tiny $_{lb}$}‚ अपि म‚न्त्रान् कुर्व‚न्ति \textbf{तेषां} पुरुषाणां \textbf{प्र‚भाव‚व‚तैव} त‚न्त्र‚स्य प्र‚णेत्रा\textbf{धिष्ठानात्} म‚न्त्र‚{\tiny $_{lb}$}‚क‚र‚ण‚साम‚र्थ्यं ।
	{\color{gray}{\rmlatinfont\textsuperscript{§~\theparCount}}}
	\pend% ending standard par
      ‚{\tiny $_{lb}$}‚

	  
	  \pstart \leavevmode% starting standard par
	एत‚देव द‚र्श‚य‚न्नाह । \textbf{त‚त्कृतं ही}त्यादि । तेन प्र‚{\tiny $_{२}$}‚भावातिश‚य‚व‚ता पुरुषेण‚{\tiny $_{lb}$}‚ \textbf{कृत‚स‚म‚य‚म‚नुपाल‚य‚न्तो} र‚क्ष‚न्तः । \textbf{त‚दुप‚देशेन चे}ति प्र‚भाव‚युक्त‚पुरुषोप‚देशेन च‚{\tiny $_{lb}$}‚ \textbf{व‚र्त‚माना} म‚न्त्र‚क्रियास‚म‚र्थाः कुत एत‚त् ।
	{\color{gray}{\rmlatinfont\textsuperscript{§~\theparCount}}}
	\pend% ending standard par
      ‚{\tiny $_{lb}$}‚

	  
	  \pstart \leavevmode% starting standard par
	\textbf{त}दित्यादि त‚स्य प्र‚भाव‚तो य‚स्स\textbf{म‚यः} । य‚श्चो\textbf{प‚देश}स्त‚त्र \textbf{निर‚पेक्षाणां} पुंसां‚{\tiny $_{lb}$}‚ म‚न्त्र‚र‚च‚नाया\textbf{म‚साम‚र्थ्यात्} । त‚त्र स‚म‚यो य‚स्यातिक्र‚मात् पुन‚र्म‚ण्ड‚ल‚प्र‚वेशादिः क‚र्त्त‚व्यो‚{\tiny $_{lb}$}‚ जाय‚ते । त‚तोन्य‚द्विधा‚{\tiny $_{३}$}‚नामुप‚देश इत्य‚न‚योर्भेदः । \textbf{त‚त्रापी}ति र‚थ्यापुरुष‚कृतेष्व‚पि‚{\tiny $_{lb}$}‚ \textbf{म‚न्त्रेषु त‚दाकार‚ध्यानादेव} । प्र‚भाव‚व‚तः । पुंस‚श्चाकार‚ध्यानादेरेव [।]‚{\tiny $_{lb}$}‚ आदिश‚ब्दात् स्व‚भाव‚च‚र्याध्यान‚स्य प‚रिग्र‚हः । तेन म‚न्त्र‚स्य \textbf{प्र‚योगात्} प्र‚व‚र्त्त‚नात् ।‚{\tiny $_{lb}$}‚ \textbf{य‚त एव‚न्त‚स्मात्त‚द‚धिष्ठान‚मेव} प्र‚भाव‚व‚त्पुरुषाधिष्ठान‚मेव \textbf{त‚त्तादृश‚मुन्नेयं} बोद्ध‚व्यं ।‚{\tiny $_{lb}$}‚ \textbf{य‚त्ते स्व‚यंकृ}तैर्म‚न्त्रैः क‚र्म कुर्व‚न्तीति ।
	{\color{gray}{\rmlatinfont\textsuperscript{§~\theparCount}}}
	\pend% ending standard par
      ‚{\tiny $_{lb}$}‚‚{\tiny $_{lb}$}‚\textsuperscript{\textenglish{582/s}}

	  
	  \pstart \leavevmode% starting standard par
	\textbf{अपि च‚{\tiny $_{४}$}‚} [।] केचित् त‚न्त्र‚ज्ञा म‚न्त्रं कुर्व‚न्तीत्य‚भिद‚ध‚ता \textbf{पुरुषातिश‚य एव स‚म‚{\tiny $_{lb}$}‚र्थितः स्यात्} । य‚स्मात् सोपि \textbf{तादृश‚स्}त‚न्त्र‚ज्ञो म‚न्त्र‚स्य क‚र्त्ता \textbf{प्र‚भाव‚वानेव । त‚द‚न्यै}र‚{\tiny $_{lb}$}‚त‚न्त्र‚ज्ञैः पुरुषै\textbf{र‚साधार‚ण‚श‚क्तित्वादि}ति कृत्वा । य‚त‚श्च पुरुषाधिष्ठितानामेव म‚न्त्राणां‚{\tiny $_{lb}$}‚ फ‚लं । त‚स्मात् \textbf{कृत‚काः पौरुषेयाश्च} फ‚ल‚दा इत्येव\textbf{म्म‚न्त्रा वाच्याः फ‚लेप्सुना} ।‚{\tiny $_{lb}$}‚ म‚न्त्राद् फ‚ल‚मिच्छ‚ता न नित्या म‚{\tiny $_{५}$}‚न्त्राः किन्तु कृत‚काः कृत‚क‚त्वेपि न फ‚ल‚दाने‚{\tiny $_{lb}$}‚ पुरुष‚निर‚पेक्षा इत्य‚र्थ‚द्व‚य‚माद‚र्श‚यितुं कृत‚काः पौरुषेयाश्चेति द्व‚योपादानं ।
	{\color{gray}{\rmlatinfont\textsuperscript{§~\theparCount}}}
	\pend% ending standard par
      ‚{\tiny $_{lb}$}‚

	  
	  \pstart \leavevmode% starting standard par
	\textbf{न ही}त्यादिना व्याच‚ष्टे । न हि नित्यानाम्वैदिकानां म‚न्त्राणां \textbf{प्र‚योग} उच्चार‚णं‚{\tiny $_{lb}$}‚ \textbf{स‚म्भ‚व‚त्य}नाधेयातिश‚य‚त्वात् । न चा\textbf{प्र‚युक्तेभ्यो} म‚न्त्रेभ्यः \textbf{फ‚ल‚मि}ति कृत्वा \textbf{प्र‚योगात्‚{\tiny $_{lb}$}‚ फ‚ल‚मिच्छ‚ता कृत‚का म‚न्त्रा वाच्याः । पौरुषेयाश्च} ।‚{\tiny $_{६}$}‚ पुरुषाधिष्ठिताश्च फ‚ल‚दा‚{\tiny $_{lb}$}‚ वाच्याः । किं कार‚णं [।] \textbf{पुरुषाधिष्ठान‚म‚न्त‚रेण} विनान्य‚तो भाव‚श‚क्त्यादे‚{\tiny $_{lb}$}‚\textbf{र‚स‚म्भ‚व‚त्फ‚लाना}म्म‚न्त्राणां पुरुषाधिष्ठानादेव \textbf{फ‚ल‚द‚र्श‚नात्} । य‚था च न भाव‚श‚क्त्या‚{\tiny $_{lb}$}‚ म‚न्त्रेभ्यः फ‚लोत्प‚त्तिस्त‚था प्र‚तिपादितं [।]
	{\color{gray}{\rmlatinfont\textsuperscript{§~\theparCount}}}
	\pend% ending standard par
      ‚{\tiny $_{lb}$}‚

	  
	  \pstart \leavevmode% starting standard par
	\hphantom{.}स‚र्व‚स्य साध‚न‚न्ते स्युर्भाव‚श‚क्तिर्य‚दीदृशी त्यादिना \href{http://sarit.indology.info/?cref=pv.3.294}{१ । २९७} ।
	{\color{gray}{\rmlatinfont\textsuperscript{§~\theparCount}}}
	\pend% ending standard par
      ‚{\tiny $_{lb}$}‚

	  
	  \pstart \leavevmode% starting standard par
	\leavevmode\ledsidenote{\textenglish{205b/PSVTa}} निद‚र्श‚नं चाह । यो म‚दीयं काव्याद्येवं प‚ठिष्य‚ति‚{\tiny $_{७}$}‚ त‚स्य म‚याय‚म‚र्थः स‚म्पाद‚नीय‚{\tiny $_{lb}$}‚ इत्येवं\textbf{कृतः स‚म‚यो} य‚स्मिन् \textbf{काव्या}दौ आदिश‚ब्दाच्छिल्प‚स्थानादौ । स कृत‚स‚म‚यः‚{\tiny $_{lb}$}‚ काव्यादिः त‚स्मिन्निव त‚द्व‚त् । य‚था त‚त्र काव्यादिपाठ‚कानां पुरुषाधिष्ठानात्‚{\tiny $_{lb}$}‚ फ‚ल‚न्त‚द्व‚न्म‚न्त्रेष्व‚पीत्य‚र्थः । \textbf{पुंसाम}तीन्द्रियार्थ‚द‚र्श‚नं प्र‚ति \textbf{श‚क्ति}र्नास्तीत्येव‚म‚श‚क्ति‚{\tiny $_{lb}$}‚साध‚न‚म‚साम‚र्थ्य‚स्य \textbf{साध‚नं} य‚न्नाम किञ्चित् मी मां स कै रुच्य‚ते त‚त्स‚र्व\textbf{म‚नेनैव}‚{\tiny $_{lb}$}‚ म‚{\tiny $_{१}$}‚न्त्र‚कारिणां ज्ञान‚प्र‚भावातिश‚य‚साध‚नेन \textbf{निराकृतं} ।
	{\color{gray}{\rmlatinfont\textsuperscript{§~\theparCount}}}
	\pend% ending standard par
      ‚{\tiny $_{lb}$}‚

	  
	  \pstart \leavevmode% starting standard par
	\textbf{प्र‚तिपादिता ही}त्यादिना व्याच‚ष्टे । \textbf{प्र‚तिपादिता हि पुरुष‚कृता म‚न्त्रास्त‚द‚{\tiny $_{lb}$}‚धिष्ठानाच्च फ‚ल‚दा} म‚न्त्रा इत्येत‚द‚पि प्र‚तिपादितं ।
	{\color{gray}{\rmlatinfont\textsuperscript{§~\theparCount}}}
	\pend% ending standard par
      ‚{\tiny $_{lb}$}‚

	  
	  \pstart \leavevmode% starting standard par
	न च स‚र्वे पुरुषा म‚न्त्रान् क‚र्त्तुम‚धिष्ठातुं वा श‚क्ताः । \textbf{त‚दि}ति त‚स्मा\textbf{द‚स्ति क‚श्चि‚{\tiny $_{lb}$}‚द‚तिश‚य‚वान्} पुरुषो म‚न्त्र‚स्य क‚र्त्ते\textbf{ति । त‚स्या}तिश‚य‚व‚तः पुंसः \textbf{प्र‚तिक्षेप‚साध‚नान्य‚पि‚{\tiny $_{lb}$}‚ ‚{\tiny $_{lb}$}‚ \leavevmode\ledsidenote{\textenglish{583/s}}प्र‚तिव्यू‚{\tiny $_{२}$}‚ढानि} प्र‚तिक्षिप्तानि । एतेन च प‚रोक्त‚स्यातिश‚य‚प्र‚तिक्षेप‚साध‚न‚स्य न‚{\tiny $_{lb}$}‚ विरुद्धाव्य‚भिचारित्व‚मुद्भाव्य‚ते [।] किन्त‚र्हि पुरुषातिश‚य‚प्र‚तिक्षेप‚साध‚नानि व‚स्तु‚{\tiny $_{lb}$}‚ब‚लायातानि न स‚न्त्येवेत्य‚नेन व्याजेन क‚थ्य‚ते । न हि व‚स्तुब‚लायातं पुरुषातिश‚यं‚{\tiny $_{lb}$}‚ निराक‚र्त्तृ किंचित् साध‚न‚म‚स्ति ।
	{\color{gray}{\rmlatinfont\textsuperscript{§~\theparCount}}}
	\pend% ending standard par
      ‚{\tiny $_{lb}$}‚

	  
	  \pstart \leavevmode% starting standard par
	न‚नु चेद‚म‚स्ति विव‚क्षितः पुरुषो नातिश‚य‚वान् बुद्धिम‚त्त्वात् । इन्द्रिय‚व‚त्वात् ।‚{\tiny $_{lb}$}‚ व‚{\tiny $_{३}$}‚च‚नात् पुंस्त्वात् । र‚थ्यापुरुष‚व‚दिति [।]
	{\color{gray}{\rmlatinfont\textsuperscript{§~\theparCount}}}
	\pend% ending standard par
      ‚{\tiny $_{lb}$}‚

	  
	  \pstart \leavevmode% starting standard par
	अत आह । \textbf{बुद्धी}न्द्रियेत्यादि । बुद्धिश्चेन्द्रियं च \textbf{उक्ति}श्च \textbf{पुंस्त्वं} चेति द्व‚न्द्वः ।‚{\tiny $_{lb}$}‚ \textbf{आदि}श‚ब्दात् प्राणादिम‚त्त्वादि । पुरुषातिश‚य‚निराक‚र‚ण\textbf{साध‚नं य‚त्तु व‚र्ण्ण्य‚ते} [।]‚{\tiny $_{lb}$}‚ त‚त्स‚र्वं \textbf{प्र‚माणाभं} प्र‚माणाभास‚म‚नैकान्तिक‚मिति याव‚त् । किं कार‚णं [।] विप‚क्ष‚{\tiny $_{lb}$}‚वृत्तेः स‚न्देहेन स‚र्व‚स्य शेष‚व‚त्त्वात् । \textbf{न हि शेष‚व‚त} इत्य‚नैकान्तिक‚त्वात् । \textbf{य‚थार्थे}त्य‚{\tiny $_{lb}$}‚वि‚{\tiny $_{४}$}‚प‚रीता \textbf{ग‚ति}र‚नुमेय‚प्र‚तिप‚त्तिर‚स्ति ।
	{\color{gray}{\rmlatinfont\textsuperscript{§~\theparCount}}}
	\pend% ending standard par
      ‚{\tiny $_{lb}$}‚

	  
	  \pstart \leavevmode% starting standard par
	\textbf{य‚त्त्वि}त्यादिना व्याच‚ष्टे । \textbf{य‚त्तु पुरुषातिश‚य‚प्र‚तिक्षेप‚साध‚न‚न्त‚त्त्व‚ग‚म‚क‚मेवे}ति‚{\tiny $_{lb}$}‚ स‚म्ब‚न्धः । त‚त्पुनः साध‚नं । \textbf{बुद्धी}न्द्रिय‚योगादित्यादि । क‚स्माद‚ग‚म‚क‚मित्याह ।‚{\tiny $_{lb}$}‚ \textbf{प्र‚तिक्षे}पेत्यादि । प्र‚तिक्षेप‚श्च सामान्यं च \textbf{प्र‚तिक्षेप‚सामान्ये} । त‚योः \textbf{साध‚ने}‚{\tiny $_{lb}$}‚ त‚यो\textbf{र‚योगात्} । त‚था हि [।] बुद्धिम‚त्त्वादिना साध‚नेन नास्ति‚{\tiny $_{५}$}‚ पुरुषातिश‚य इति‚{\tiny $_{lb}$}‚ प्र‚तिक्षेपो वा साध्येत । य‚द्वा योसौ पुरुषातिश‚यः स र‚थ्यापुरुषैः स‚मान इतीत‚र‚{\tiny $_{lb}$}‚पुरुष‚सामान्यं साध्येत ।
	{\color{gray}{\rmlatinfont\textsuperscript{§~\theparCount}}}
	\pend% ending standard par
      ‚{\tiny $_{lb}$}‚

	  
	  \pstart \leavevmode% starting standard par
	त‚त्र \textbf{न ही}त्यादिना प्र‚तिक्षेप‚साध‚न‚स्याभाव‚माह । \textbf{न ह्य‚तीन्द्रियेष्व‚र्थेष्व‚{\tiny $_{lb}$}‚त‚द्द‚र्शिनो}तीन्द्रियार्थाद‚र्शिनः \textbf{प्र‚तिक्षेपः स‚म्भ‚व‚ति} । किं कार‚णं [।] \textbf{स‚ताम‚प्येषा}‚{\tiny $_{lb}$}‚म‚तीन्द्रियाणाम‚र्थानाम‚र्वाग्द‚र्श‚न‚स्य्\textbf{आज्ञानात्} । त‚स्मान्ना‚{\tiny $_{६}$}‚द‚र्श‚न‚मात्रात्प्र‚तिक्षेप इति‚{\tiny $_{lb}$}‚ भावः ।
	{\color{gray}{\rmlatinfont\textsuperscript{§~\theparCount}}}
	\pend% ending standard par
      ‚{\tiny $_{lb}$}‚

	  
	  \pstart \leavevmode% starting standard par
	नापि विरुद्ध‚विधानात् पुरुषातिश‚य‚स्य प्र‚तिक्षेपः य‚स्मा\textbf{द‚त एवा}तीन्द्रिय‚त्वादेव‚{\tiny $_{lb}$}‚ पुरुषातिश‚य‚स्य बुद्धित्वादिना हेतुना । द्विविध‚स्यापि \textbf{विरोध‚स्यासिद्धेः । अविरो‚{\tiny $_{lb}$}‚धिना च} व‚क्तृत्वादिना पुरुषातिश‚य\textbf{स्यैक‚त्र स‚म्भ‚वाविरोधादित्युक्तं प्राक्} ।
	{\color{gray}{\rmlatinfont\textsuperscript{§~\theparCount}}}
	\pend% ending standard par
      ‚{\tiny $_{lb}$}‚‚{\tiny $_{lb}$}‚\textsuperscript{\textenglish{584/s}}

	  
	  \pstart \leavevmode% starting standard par
	\leavevmode\ledsidenote{\textenglish{206a/PSVTa}} नापीत्यादिना सामान्य‚साध‚न‚स्याभाव‚माह । नापीत‚रे‚{\tiny $_{७}$}‚णार्वाग्द‚र्शिना पुरुषेण‚{\tiny $_{lb}$}‚ त‚स्यातिश‚य‚व‚तः सामान्य‚सिद्धिस्तुल्य‚तासिद्धिः । किं कार‚णं [।] अतीन्द्रिय‚द‚र्श‚नादि‚{\tiny $_{lb}$}‚ल‚क्ष‚ण‚स्य विशेष‚स्य यो स‚म्भ‚व‚स्त‚स्य ज्ञातुम‚श‚क्य‚त्वात् । ईदृशेषु च प‚र‚स‚न्तान‚{\tiny $_{lb}$}‚व‚र्त्तिषु पुरुष‚मात्राप्र‚त्य‚क्षेष्व‚तीन्द्रियार्थ‚द‚र्श‚नादिषु । या काचिद‚स‚म्भ‚व‚प्र‚माध‚न्य‚नुप‚{\tiny $_{lb}$}‚ल‚ब्धिरुपादीय‚ते । त‚स्या अनुप‚ल‚ब्धेः प्रागेव हेतुत्व‚प्र‚तिक्षेपात् ।
	{\color{gray}{\rmlatinfont\textsuperscript{§~\theparCount}}}
	\pend% ending standard par
      ‚{\tiny $_{lb}$}‚

	  
	  \pstart \leavevmode% starting standard par
	किं च [।] पुंस्त्वादि‚{\tiny $_{१}$}‚ साम्येपि य‚थास्वं संस्कारात् क‚स्य‚चित् प्र‚ज्ञामेवादेर‚ति‚{\tiny $_{lb}$}‚श‚य‚द‚र्श‚नात् त‚थान्य‚स्याप्य‚तिश‚य‚स्य स‚म्भाव्य‚त्वात् । त‚स्मात् स‚म्भ‚व‚द्विशेषाः पुरुपा‚{\tiny $_{lb}$}‚स‚म्भ‚व‚द्विशेषे चेत‚र‚पुरुष‚सामान्यासिद्धेरित्य‚प्युक्तं प्राक् ।
	{\color{gray}{\rmlatinfont\textsuperscript{§~\theparCount}}}
	\pend% ending standard par
      ‚{\tiny $_{lb}$}‚

	  
	  \pstart \leavevmode% starting standard par
	त‚स्माच्छेष‚व‚द‚नुमान‚मेत‚द् व‚क्तृत्वाद्य‚स‚म‚र्थ पुरुषातिश‚यास‚म्भ‚व‚प्र‚तिपाद‚नाय ।‚{\tiny $_{lb}$}‚ विप‚क्ष‚वृत्तेर‚द‚र्श‚नेपि ।
	{\color{gray}{\rmlatinfont\textsuperscript{§~\theparCount}}}
	\pend% ending standard par
      ‚{\tiny $_{lb}$}‚

	  
	  \pstart \leavevmode% starting standard par
	य‚दि नाम विप‚क्षे पुरुषातिश‚ये व‚क्तृ‚{\tiny $_{२}$}‚त्वादेर्वृत्तिनं दृश्य‚ते । त‚थापि वाध‚का‚{\tiny $_{lb}$}‚भावेन व्य‚तिरेक‚स्य स‚न्देहाद‚स‚म‚र्थ । अपि चैवंवादिन इति नास्त्य‚तीन्द्रियार्थ‚द‚र्शी‚{\tiny $_{lb}$}‚ पुरुष इत्येवंवादिनो जै मि नी याः स्व‚मेव वाद‚मिति क‚थ‚चिद‚तिश‚य‚व‚तो जै मि‚{\tiny $_{lb}$}‚ न्यादेः स‚काशाद् वेदार्थ‚ग‚तिर्भ‚व‚तीति पुरुषातिश‚याभ्युप‚ग‚म‚वादं पुन‚र्नास्त्य‚तीन्द्रि‚{\tiny $_{lb}$}‚यार्थ‚ज्ञः पुरुषः क‚श्चिदित्य‚न‚या स्व‚वाचा \textbf{बिधुर‚य‚न्ति‚{\tiny $_{३}$}‚} बाध‚न्तेऽतिश‚य‚व‚त्पुरुष‚प्र‚ति‚{\tiny $_{lb}$}‚क्षेपेण वेदार्थ‚ग‚तेर‚स‚म्भ‚वात् [।] त‚था हि [।] अय‚म‚र्थोऽस्माक‚न्नाय‚म‚र्थ इति स्व‚य‚{\tiny $_{lb}$}‚म्वैदिकाः श‚ब्दा न व‚द‚न्ति । तेनाग्निहोत्र‚श‚ब्दानां योभिम‚तोर्थः स क‚ल्प्यो भ‚वेत्‚{\tiny $_{lb}$}‚ पुरुषै र्मी मां स कैः । त‚च्च नास्ति । य‚त‚स्ते हि पुरुषा रागादिसंयुता रागादि‚{\tiny $_{lb}$}‚युक्ताः । त‚तो न त‚त्क‚ल्पितोऽर्थः प्र‚माणं ।
	{\color{gray}{\rmlatinfont\textsuperscript{§~\theparCount}}}
	\pend% ending standard par
      ‚{\tiny $_{lb}$}‚

	  
	  \pstart \leavevmode% starting standard par
	अथ त‚स्य वेदार्थ‚स्य क‚श्चि ज्जै मि न्या दि‚{\tiny $_{४}$}‚ रेव वेता क‚ल्प्येत । त‚त्रैक‚पुरुषो‚{\tiny $_{lb}$}‚  ‚{\tiny $_{lb}$}‚ ‚{\tiny $_{lb}$}‚ \leavevmode\ledsidenote{\textenglish{585/s}}\textbf{भिम‚त‚स्त‚त्व‚वित्} । वेदार्थ‚त‚त्त्व‚ज्ञो नान्यः पुरुष इति किंकृतः । नात्र किञ्चित्‚{\tiny $_{lb}$}‚ कार‚ण‚म‚स्ति मी मां स क स्य पुरुष‚त्वाविशेषात् स‚र्वो वा वेत्ति । न वा क‚श्चि‚{\tiny $_{lb}$}‚दिति भावः ।
	{\color{gray}{\rmlatinfont\textsuperscript{§~\theparCount}}}
	\pend% ending standard par
      ‚{\tiny $_{lb}$}‚

	  
	  \pstart \leavevmode% starting standard par
	अथ पुरुष‚त्वादिसाम्येप्य‚साधार‚ण‚श‚क्तियुक्तो वैदिकानां श‚ब्दानाम‚तीन्द्रि‚{\tiny $_{lb}$}‚यैर‚र्थेः स‚ह स‚म्ब‚न्ध‚स्य वेता क‚श्चिज्जै मि न्यादिः क‚ल्प्य‚ते [।] त‚दा त‚द्व‚ज्जैमि‚{\tiny $_{५}$}‚न्या‚{\tiny $_{lb}$}‚दिव‚त् । पुंस्त्वे पुरुष‚त्वे तुल्योपि क‚थ‚म‚पीति निर्निमित्त‚म‚न्योपि क‚श्चिज्ज्ञानी ज्ञाना‚{\tiny $_{lb}$}‚तिश‚य‚वान् । क‚स्मान्न वो न युष्माक‚म‚भिम‚तो जै मि न्या दिव‚द‚न्योपि ज्ञान‚वान्‚{\tiny $_{lb}$}‚ प्र‚स‚ज्य‚त इति याव‚त् ।
	{\color{gray}{\rmlatinfont\textsuperscript{§~\theparCount}}}
	\pend% ending standard par
      ‚{\tiny $_{lb}$}‚

	  
	  \pstart \leavevmode% starting standard par
	नेत्यादिना व्याच‚ष्टे । एत आग‚च्छ‚त भ‚व‚न्तो ब्राह्म‚णा अय‚म‚स्माक‚म‚र्थो भ‚व‚{\tiny $_{lb}$}‚द्भिर्ग्राह्यो नान्य इत्येवंवैदिकाः श‚ब्दा न विक्रोश‚न्ति न क‚थ‚य‚न्ति येन तेभ्योर्थ‚{\tiny $_{lb}$}‚ग‚तिः स्यात् ।‚{\tiny $_{६}$}‚ केव‚ल‚मित्य‚व‚धार‚णे । अन‚भिव्य‚क्तार्थ‚विशेष‚संस‚र्गा एव श्रुतिं श्रोत्र‚{\tiny $_{lb}$}‚विज्ञान‚म‚भिप‚त‚न्त्यारोह‚न्ति [।] अन‚भिव्य‚क्तोर्थ‚विशेषेण स‚ह संस‚र्गः स‚म्ब‚न्धो‚{\tiny $_{lb}$}‚ येषामिति विग्र‚हः । त‚त्राज्ञातार्थ‚स‚म्ब‚न्धेषु श‚ब्देषु श्रुतिम‚भिप‚त‚त्स्वेकः पुरुषः स्व‚यं‚{\tiny $_{lb}$}‚ क‚ञ्चिद‚र्थ स्वेच्छानुरूपं क‚ल्प‚य‚त्य‚न्योपि पुरुषोप‚र‚म‚र्थं क‚ल्प‚य‚तीत्य‚निर्ण्ण‚य एव‚{\tiny $_{lb}$}‚ प‚दार्थ‚स्य ।
	{\color{gray}{\rmlatinfont\textsuperscript{§~\theparCount}}}
	\pend% ending standard par
      \textsuperscript{\textenglish{206b/PSVTa}}‚{\tiny $_{lb}$}‚

	  
	  \pstart \leavevmode% starting standard par
	स्वाभाविकः श‚ब्दानाम‚र्थ‚स‚म्ब‚न्ध‚स्तेनैकार्थ‚प्र‚तिनिय‚मो भ‚विष्य‚तीत्याह ।‚{\tiny $_{lb}$}‚ नेत्याह । न च क‚श्चिच्छ‚ब्दानां स्व‚भाव‚प्र‚तिनिय‚मः स्व‚भावेन प्र‚कृत्यार्थैस्स‚ह स‚म्ब‚न्धो‚{\tiny $_{lb}$}‚ येनानेकार्थ‚क‚ल्प‚नायाम‚पि केव‚लं स‚म‚य‚व‚शात् त‚न्त‚म‚र्थ‚माविश‚न्तो वाच्य‚त्वेनोपा‚{\tiny $_{lb}$}‚द‚दाना दृश्य‚न्ते । तेषाम्वैदिकानां श‚ब्दा‚{\tiny $_{१}$}‚नां क‚श्चित् त‚त्त्व‚माच‚ष्टे नाप‚र इति न‚{\tiny $_{lb}$}‚ न्याव्य‚मिति स‚म्ब‚न्धः । कीदृशानाम‚विदितार्थ‚निय‚मानां । अविदितोर्थ‚निय‚मो‚{\tiny $_{lb}$}‚ येषामिति विग्र‚हः । किं कार‚ण‚म् [।] अत्य‚क्षावेशात् । अतीन्द्रिय‚स्य स्व‚र्गादि‚{\tiny $_{lb}$}‚साध‚न‚स्यार्थ‚स्य विष‚य‚त्वेनात्म‚सात्क‚र‚णात् । न ह्य‚तीन्द्रियार्थ‚स्य श‚ब्द‚स्यार्थ‚निय‚म‚{\tiny $_{lb}$}‚म‚र्वाग्दिर्श‚नः श‚क्तो ज्ञातु । त‚त्राविद्वानेव । रागादिदोषोप‚प्लुतः । क‚{\tiny $_{२}$}‚श्चिज्जै‚{\tiny $_{lb}$}‚मि निः श व र स्वा मी वा । तेषां श‚ब्दानान्त‚त्त्व‚माच‚ष्टे । अस्याय‚मेवार्थ इति‚{\tiny $_{lb}$}‚ ‚{\tiny $_{lb}$}‚ \leavevmode\ledsidenote{\textenglish{586/s}}नाप‚रः । अप‚रोपि पुरुषो जै मि न्याद्य‚विशिष्टो न त‚त्त्व‚माच‚ष्ट इति भेद‚व्य‚व‚स्थानं‚{\tiny $_{lb}$}‚ न न्याय्य‚म‚युक्तित्वात् ।
	{\color{gray}{\rmlatinfont\textsuperscript{§~\theparCount}}}
	\pend% ending standard par
      ‚{\tiny $_{lb}$}‚

	  
	  \pstart \leavevmode% starting standard par
	\textbf{अथ कुत‚श्चिद}निर्देश्य‚रूपाद् \textbf{बुद्धीन्द्रियादीनाम्} [।] आदिश‚ब्दाद‚भ्यास‚स्या‚{\tiny $_{lb}$}‚तिश‚यात् कार‚णात् \textbf{स एव} जै मि नि प्र‚भृतिर्वेदार्थ\textbf{म्वेत्ति नाप‚रः} प्राकृतः पुरुष‚{\tiny $_{lb}$}‚ इ‚{\tiny $_{३}$}‚तीष्य‚ते । त‚दा \textbf{त‚स्य} जै मि नि प्र‚भृतेः \textbf{कुतोय‚म‚तीन्द्रिय‚ज्ञानातिश‚यः} । अतीन्द्रिय‚स्य‚{\tiny $_{lb}$}‚ वेदार्थ‚त‚त्त्व‚स्य ज्ञानातिश‚योन्यैर‚विदित‚त‚त्त्वैर‚विशिष्ट‚स्य ।
	{\color{gray}{\rmlatinfont\textsuperscript{§~\theparCount}}}
	\pend% ending standard par
      ‚{\tiny $_{lb}$}‚

	  
	  \pstart \leavevmode% starting standard par
	भ‚व‚तु वा जै मि नि प्र‚भृतिः पुरुषोतीन्द्रियार्थ‚स्य वेत्ता । \textbf{त‚था} जै मि न्यादि‚{\tiny $_{lb}$}‚व‚द\textbf{न्योपि} पुरुषातिश‚यो बौ द्धा द्य भिम‚तो \textbf{देश‚काल‚स्व‚भाव‚विप्र‚कृष्टानाम‚र्थानां‚{\tiny $_{lb}$}‚ द्र‚ष्टा । किम‚स‚म्भ‚वी} । क‚स्माद‚वि‚{\tiny $_{४}$}‚द्य‚मानो दृष्टो येन प्र‚तिक्षिप्य‚ते । सोप्य‚ती‚{\tiny $_{lb}$}‚न्द्रियार्थ‚द‚र्श्य‚स्त्वितीष्य‚तां । न चेद‚भिम‚तोपि जै मि न्यादिर्मा भूत् । \textbf{य‚तो न हि‚{\tiny $_{lb}$}‚ त‚त्प्र‚तिक्षेप‚साध‚नानि} । बौ द्धा द्य‚भिम‚त‚पुरुषातिश‚य‚प्र‚तिक्षेप‚साध‚नानि पुरुष‚त्वा‚{\tiny $_{lb}$}‚दीनि \textbf{कानिचित्} स‚न्ति [।] यानि \textbf{नैन‚म्}वेदार्थ‚विवेक‚कारिणं जै मि नि प्र‚भृतिं‚{\tiny $_{lb}$}‚ नो\textbf{प‚लीय‚न्ते} । न विष‚यीकुर्व‚न्ति । किन्तूप‚ली‚{\tiny $_{५}$}‚य‚न्त एव । तेषाम‚पि पुरुष‚त्वा‚{\tiny $_{lb}$}‚दियोगात् ।
	{\color{gray}{\rmlatinfont\textsuperscript{§~\theparCount}}}
	\pend% ending standard par
      ‚{\tiny $_{lb}$}‚

	  
	  \pstart \leavevmode% starting standard par
	अथ पुरुष‚त्वादिसाध‚न‚स‚म्भ‚वेपि जैमिन्यादे\textbf{र्विशेष} इष्य‚ते । त‚दा \textbf{य‚थाय}म‚ती‚{\tiny $_{lb}$}‚न्द्रिय‚वेदार्थ‚विवेच‚न‚ल‚क्ष‚णे \textbf{विशेषोस्य} जैमिनिप्र‚भृतेरिष्टः । \textbf{त‚त्साध‚न‚स‚म्भ‚वेपी}ति ।‚{\tiny $_{lb}$}‚ त‚स्यातीन्द्रियार्थ‚द‚र्शिपुरुष‚प्र‚तिक्षेप‚साध‚न‚स्य पुरुषात्वादेः स‚म्भ‚वेपि । \textbf{त‚थान्य‚स्या}पि‚{\tiny $_{lb}$}‚ पुरुष‚स्यातीन्द्रिया‚{\tiny $_{६}$}‚र्थ‚द‚र्श‚नं \textbf{स्यादित्य‚न‚भिनिवेश एव} भ‚व‚तां मी मां स का नां‚{\tiny $_{lb}$}‚ \textbf{युक्तः} ।
	{\color{gray}{\rmlatinfont\textsuperscript{§~\theparCount}}}
	\pend% ending standard par
      ‚{\tiny $_{lb}$}‚

	  
	  \pstart \leavevmode% starting standard par
	नातीन्द्रियार्थ‚द‚र्शीति कृत्वा जै मि नि प्र‚भृतेर्वेदार्थ‚ज्ञान‚मिष्ट‚म‚पि तु । \textbf{य‚स्य}‚{\tiny $_{lb}$}‚ वाक्यं \textbf{प्र‚माण‚स‚म्वादि । स} पुरुषो वेदा\textbf{र्थ‚विद् य‚दी}ष्य‚ते ।
	{\color{gray}{\rmlatinfont\textsuperscript{§~\theparCount}}}
	\pend% ending standard par
      ‚{\tiny $_{lb}$}‚

	  
	  \pstart \leavevmode% starting standard par
	जै मि नि प्र‚भृतेरेव च वेदार्थ‚विवेच‚नं कुर्व‚तो व‚च‚नं प्र‚माण‚स‚म्वादीति प‚रो‚{\tiny $_{lb}$}‚ म‚न्य‚ते ।
	{\color{gray}{\rmlatinfont\textsuperscript{§~\theparCount}}}
	\pend% ending standard par
      ‚{\tiny $_{lb}$}‚

	  
	  \pstart \leavevmode% starting standard par
	\leavevmode\ledsidenote{\textenglish{207a/PSVTa}} नेत्या द्या चा र्यः । नेद‚मुत्त‚रं युक्तं य‚{\tiny $_{७}$}‚स्मा\textbf{न्न‚ह्य‚त्य‚न्त‚प‚रोक्षेषु} वेदार्थेषु स्व‚र्गा‚{\tiny $_{lb}$}‚दिसाध‚न‚त्वेषु \textbf{प्र‚माण‚स्यास्ति स‚म्भ‚वः} ।
	{\color{gray}{\rmlatinfont\textsuperscript{§~\theparCount}}}
	\pend% ending standard par
      ‚{\tiny $_{lb}$}‚‚{\tiny $_{lb}$}‚\textsuperscript{\textenglish{587/s}}

	  
	  \pstart \leavevmode% starting standard par
	\textbf{स्यादेत}दित्यादिना व्याच‚ष्टे । \textbf{न व‚यं पुरुष‚प्रामाण्याः क‚स्य‚चि}ज्जैमिन्यादे‚{\tiny $_{lb}$}‚\textbf{र्वेद‚व्याख्यान‚म‚भिनिविष्टाः [।] किन्त‚र्हि [।] प्र‚माण‚स‚म्वादाद्} व्याख्यान‚म‚{\tiny $_{lb}$}‚भिनिविष्टाः । एत‚देव व्य‚न‚क्ति । \textbf{ब‚हुष्व‚पि} वेद\textbf{व्याख्यातृषु} म‚ध्ये । \textbf{यो} वेद‚स्य‚{\tiny $_{lb}$}‚ व्याख्याता । य‚था व्याख्यातेर्थे \textbf{प्र‚माणं प्र‚त्य‚क्षा‚{\tiny $_{१}$}‚दिकं संस्य‚न्द‚य‚ति} योज‚य‚ति \textbf{स}‚{\tiny $_{lb}$}‚ तादृशो व्याख्याता\textbf{नुम‚न्य‚ते}ङ्गीक्रिय‚ते नान्य \textbf{इति} ।
	{\color{gray}{\rmlatinfont\textsuperscript{§~\theparCount}}}
	\pend% ending standard par
      ‚{\tiny $_{lb}$}‚

	  
	  \pstart \leavevmode% starting standard par
	\textbf{त‚न्ने}त्यादिना प्र‚तिषेध‚ति । ध‚र्माध‚र्माव\textbf{दृष्टं । आदि}श‚ब्दात् स्व‚र्गादिसाध‚क‚{\tiny $_{lb}$}‚त्वेष्व\textbf{तीन्द्रियेषु} प्र‚त्य‚क्षादि\textbf{प्र‚माणान्त‚रावृत्तेः} । न वेदार्थे क‚स्य‚चित् प्र‚माण‚स‚म्वा‚{\tiny $_{lb}$}‚दि व‚च‚नं । य‚स्मात् \textbf{त‚द}स‚म्भ‚वा\textbf{देव} हि । अत्य‚न्त‚प‚रोक्षे प्र‚त्य‚क्षादिप्र‚माण‚स्यास‚म्भ‚{\tiny $_{lb}$}‚वादेव हि । \textbf{त‚{\tiny $_{२}$}‚त्प्र‚तीत्य‚र्थ}न्त‚स्यातीन्द्रिय‚स्य प्र‚तीत्य‚र्थ‚म्\textbf{आग‚मः उप‚याच्य‚ते} प्रार्थ्य‚ते ।‚{\tiny $_{lb}$}‚ \textbf{अन्य‚थे}ति य‚द्याग‚म‚ग‚म्येप्य‚र्थे प्र‚माणान्त‚र‚स‚म्वादादेवार्थ‚निश्च‚य‚स्त‚दा \textbf{स‚त्य‚पि त‚स्मि}‚{\tiny $_{lb}$}‚न्नाग‚मे त‚द्ग‚म्\textbf{येर्थे य‚दि प्र‚माणान्त‚र‚स्यावृत्तिः} स्यात् [।] त‚दा प्र‚माणान्त‚रावृत्ता‚{\tiny $_{lb}$}‚वाग‚मात् केव‚ला\textbf{द‚प्र‚तिप‚त्तेः । त‚त‚श्चे}ति प्र‚माणान्त‚रात्\textbf{केव‚ला}दित्याग‚म‚र‚हितार्थ‚{\tiny $_{lb}$}‚\textbf{प्र‚तिप‚त्तेर‚सा‚{\tiny $_{३}$}‚ध‚न‚मेवाग‚मः स्यात्} । प्र‚माणान्त‚र‚भावाभावाभ्यामेवार्थ‚प्र‚तिप‚त्तेर्भावा‚{\tiny $_{lb}$}‚भावात् । \textbf{केव}लादाग‚म‚निर‚पेक्षा\textbf{द‚न्य‚तोपि} प्र‚त्य‚क्षादेः प्र‚माणाद\textbf{तीन्द्रि}येर्थे \textbf{प्र‚तिप‚त्तिः}‚{\tiny $_{lb}$}‚ किन्त्वाग‚म‚स‚हितात् प्र‚त्य‚क्षादेर‚तीन्द्रियार्थ\textbf{प्र‚तिप‚त्तिरिति चेत्} ।
	{\color{gray}{\rmlatinfont\textsuperscript{§~\theparCount}}}
	\pend% ending standard par
      ‚{\tiny $_{lb}$}‚

	  
	  \pstart \leavevmode% starting standard par
	\textbf{क‚थ}मिति सि द्धा न्त वा दी । \textbf{क‚थ‚म‚तीन्द्रिय‚श्च नाम} स्व‚र्गादिसिद्ध्युपायः‚{\tiny $_{lb}$}‚ \textbf{प्र‚त्य‚क्षादिविष‚य‚श्च} विरोधा‚{\tiny $_{४}$}‚त् ।
	{\color{gray}{\rmlatinfont\textsuperscript{§~\theparCount}}}
	\pend% ending standard par
      ‚{\tiny $_{lb}$}‚

	  
	  \pstart \leavevmode% starting standard par
	स्यान्म‚तं [।] नैवात्य‚न्त‚प‚रोक्षेर्थे प्र‚त्य‚क्षादीनां साध‚क‚त्वं किंन्तु \textbf{ते पुनः} प्र‚त्य‚क्षा‚{\tiny $_{lb}$}‚\textbf{द‚यः स्व‚विष‚येप्या}त्मीये विष‚येप्य्\textbf{आग‚म‚म‚पेक्ष्यैव साध‚काश्चेत्} त‚था चाग‚म‚स्यैव‚{\tiny $_{lb}$}‚ प्रामाण्य‚मिति प‚रो म‚न्य‚ते ।
	{\color{gray}{\rmlatinfont\textsuperscript{§~\theparCount}}}
	\pend% ending standard par
      ‚{\tiny $_{lb}$}‚

	  
	  \pstart \leavevmode% starting standard par
	\textbf{अनाग‚मेत्या}दि सि द्धा न्त वा दी । नास्मिन्नाग‚मोस्तीत्य\textbf{नाग‚मोग्न्यादिप्र‚त्य‚यो‚{\tiny $_{lb}$}‚ धूमादेर्लिङ्गान्न स्यात्} ।
	{\color{gray}{\rmlatinfont\textsuperscript{§~\theparCount}}}
	\pend% ending standard par
      ‚{\tiny $_{lb}$}‚‚{\tiny $_{lb}$}‚\textsuperscript{\textenglish{588/s}}

	  
	  \pstart \leavevmode% starting standard par
	\textbf{ने}त्यादि प‚रः । \textbf{न वै आग‚मे}तीन्द्रियेष्व‚{\tiny $_{५}$}‚र्थेषु \textbf{प्र‚वृत्ते} प्र‚त्य‚क्षादिप्र‚माण‚माग‚म‚वि‚{\tiny $_{lb}$}‚ष‚ये\textbf{न्विष्य}ते । येनाय‚न्दोषः । स्यात् ।
	{\color{gray}{\rmlatinfont\textsuperscript{§~\theparCount}}}
	\pend% ending standard par
      ‚{\tiny $_{lb}$}‚

	  
	  \pstart \leavevmode% starting standard par
	त‚त‚श्च केव‚लात् प्र‚तिप‚त्तेर‚साध‚न‚मेवाग‚मः स्यादिति । \textbf{किन्तु सैवाग‚म‚स्या}‚{\tiny $_{lb}$}‚तीन्द्रियेष्व‚पि \textbf{प्र‚वृत्तिर्न ज्ञाय‚ते} । तेनाग‚म‚प्र‚वृत्तिः प्र‚त्य‚क्षादिक‚म‚पेक्ष‚त \textbf{इति चेत्} ।
	{\color{gray}{\rmlatinfont\textsuperscript{§~\theparCount}}}
	\pend% ending standard par
      ‚{\tiny $_{lb}$}‚

	  
	  \pstart \leavevmode% starting standard par
	\textbf{स्व‚य}मित्यादिप्र‚तिव‚च‚न‚न्त‚दास्य प्र‚त्य‚क्षादेः स्व‚य‚म‚तीन्द्रियार्थ\textbf{प्र‚साध‚ने स‚म‚{\tiny $_{lb}$}‚र्थ‚स्य त‚दाग‚मो‚{\tiny $_{६}$}‚प‚धा}न‚न्त‚स्याग‚म‚स्योप‚धानं स‚न्निधानं प्र‚त्य‚क्षादेः \textbf{क‚म‚तिश‚यं पुष्णाति}‚{\tiny $_{lb}$}‚ येनाग‚म‚प्र‚वृत्तिम‚तीन्द्रियेर्थे ज्ञाप‚य‚ति ।
	{\color{gray}{\rmlatinfont\textsuperscript{§~\theparCount}}}
	\pend% ending standard par
      ‚{\tiny $_{lb}$}‚

	  
	  \pstart \leavevmode% starting standard par
	अथातीन्द्रिये प्र‚त्य‚क्षाद्य‚स‚म‚र्थं [।] त‚तो \textbf{स‚म‚र्थ‚न्तु} प्र‚त्य‚क्षादि \textbf{आग‚म‚प्र‚वृत्ति‚{\tiny $_{lb}$}‚म‚पि नैव साध‚यिष्य‚ति} य‚तो य‚था स्व‚र्गादिसाध‚क‚म‚तीन्द्रिय‚न्त‚था \textbf{सा चातीन्द्रिये‚{\tiny $_{lb}$}‚\leavevmode\ledsidenote{\textenglish{207b/PSVTa}} णार्थेन स‚म्ब‚द्धा आग‚म‚प्र‚वृत्तिर‚तीन्द्रिया । क‚थ‚म‚{\tiny $_{७}$}‚न्येन} प्र‚त्य‚क्षादिना सिद्धा । नैव‚{\tiny $_{lb}$}‚ सिद्धा । येन त‚द‚र्थं प्र‚त्य‚क्षादिर‚न्विष्य‚त इति य‚त्किञ्चिदेत‚त् ।
	{\color{gray}{\rmlatinfont\textsuperscript{§~\theparCount}}}
	\pend% ending standard par
      ‚{\tiny $_{lb}$}‚

	  
	  \pstart \leavevmode% starting standard par
	ब‚हुष्व‚पि व्याख्यातृषु यः प्र‚माणं प्र‚त्य‚क्षादिकं संस्य‚न्द‚य‚ति त‚स्य भाषितं गृह्य‚त‚{\tiny $_{lb}$}‚ इति ब्रुव‚तो पौरुषेय‚त्वादाग‚म‚ल‚क्ष‚णा\textbf{द‚न्य‚श्चैव‚माग‚म‚ल‚क्ष‚णं स्यात्} । एत‚देवाह ।‚{\tiny $_{lb}$}‚ \textbf{त‚था} हीत्यादि । \textbf{य‚स्य} पुरुष‚स्य \textbf{व‚च‚नं प्र‚माण‚स‚म्वादि । त‚त्कृत}न्तेन पुरुषेण संस्कृतं‚{\tiny $_{lb}$}‚ \textbf{व‚च‚{\tiny $_{१}$}‚} आग‚म \textbf{इति प्राप्तं} । व‚च‚न‚स्य च संस्कार‚स्त‚द‚र्थ‚स्य प्र‚माणानुगृहीत‚त्व‚ख्याप‚नं ।‚{\tiny $_{lb}$}‚ इद‚न्त‚द‚न्य‚दाग‚म‚ल‚क्ष‚णं । त‚था च \textbf{निर‚र्था} व्य‚र्थाऽपौरुषेय‚ता आग‚म‚ल‚क्ष‚ण‚त्वेनेष्टा ।
	{\color{gray}{\rmlatinfont\textsuperscript{§~\theparCount}}}
	\pend% ending standard par
      ‚{\tiny $_{lb}$}‚

	  
	  \pstart \leavevmode% starting standard par
	\textbf{तुल्ये}त्यादिना व्याच‚ष्टे । य‚श्च प्र‚माण‚स‚म्वादिव‚च‚नेन पुरुषेणार्थः क‚ल्पितो‚{\tiny $_{lb}$}‚ य‚श्चेत‚रेण त‚योर‚र्थ‚योस्तु\textbf{ल्येप्य}पौरुषेय‚त्वा\textbf{ग‚म‚वादे} स‚ति \textbf{प्र‚माण‚ब‚लादाग‚म‚स्या}पौ‚{\tiny $_{२}$}‚‚{\tiny $_{lb}$}‚रुषेय‚स्यापि \textbf{क्व‚चिद}र्थे प्र‚माण‚स‚म्वादिन्य्\textbf{आग‚म‚त्व} इष्य‚माणेऽग्निहोत्रादिवाक्यानां‚{\tiny $_{lb}$}‚ \textbf{प्र‚माण‚स‚म्वाद आग‚म‚ल‚क्ष‚णं स्यात् । नापुरुष‚क्रिया} । अपौरुषेय‚त्व‚माग‚म‚ल‚क्ष‚णं‚{\tiny $_{lb}$}‚ न स्यात् । किङ्कार‚णं [।] \textbf{त‚स्या} अपुरुष‚क्रियाया अभिम‚ताऽन‚भिम‚तेषु पुरुषोप‚{\tiny $_{lb}$}‚\textbf{दिष्टेषु स‚र्वार्थेषु तुल्य‚त्वेपि प्र‚माणेनाबाध‚नात् प्र‚तिप‚त्ते}रिष्ट‚त्वात् ।
	{\color{gray}{\rmlatinfont\textsuperscript{§~\theparCount}}}
	\pend% ending standard par
      ‚{\tiny $_{lb}$}‚‚{\tiny $_{lb}$}‚\textsuperscript{\textenglish{589/s}}

	  
	  \pstart \leavevmode% starting standard par
	एत‚दुक्त‚म्भ‚व‚ति ।‚{\tiny $_{३}$}‚ प्र‚माण‚स‚म्वादित्वेनाग‚मार्थ‚प्र‚तिप‚त्तेरिष्ट‚त्वादित्य‚र्थः । \textbf{त‚द्‚{\tiny $_{lb}$}‚भावे}प्य‚पौरुषेय‚त्व‚भावेपि तुल्ये\textbf{न्य‚त्राप्र‚माण‚स‚म्वादि}न्य‚र्थे प्र‚तिप‚त्तेर\textbf{निष्ठ‚त्वात्} प्र‚माण‚{\tiny $_{lb}$}‚स‚म्वादो व‚च‚नानामाग‚म‚ल‚क्ष‚णं स्यात् ।
	{\color{gray}{\rmlatinfont\textsuperscript{§~\theparCount}}}
	\pend% ending standard par
      ‚{\tiny $_{lb}$}‚

	  
	  \pstart \leavevmode% starting standard par
	\textbf{य‚दि चात्य‚न्त‚प‚रोक्षेर्थेऽनाग‚म‚ज्ञान‚स‚म्भ‚व} आग‚म‚निर‚पेक्ष‚स्य ज्ञान‚स्य स‚म्भ‚वः ।‚{\tiny $_{lb}$}‚ त‚दा\textbf{तीन्द्रियार्थ‚वित् क‚श्चिद‚स्तीति} स्व‚म\textbf{भिम‚त‚म्भ‚वे‚{\tiny $_{४}$}‚त्} ।
	{\color{gray}{\rmlatinfont\textsuperscript{§~\theparCount}}}
	\pend% ending standard par
      ‚{\tiny $_{lb}$}‚

	  
	  \pstart \leavevmode% starting standard par
	\textbf{य‚दि पुरुष‚स्य} जै मि न्या देर्वेदार्थ‚माख्यातुः \textbf{प‚रोक्षेर्थे} स्व‚र्ग्गादिसाध‚नोपाये \textbf{आग‚{\tiny $_{lb}$}‚मान‚पेक्षं ज्ञान‚याथात‚थ्यं} ज्ञान‚स्यावैप‚रीत्य\textbf{मिष्य‚ते} । त‚दा जै मि न्यादिव‚द‚न्येपि‚{\tiny $_{lb}$}‚ \textbf{पुरुषाः} संत्य\textbf{तीन्द्रियार्थ‚दृश इतीष्टं स्यात्} ।
	{\color{gray}{\rmlatinfont\textsuperscript{§~\theparCount}}}
	\pend% ending standard par
      ‚{\tiny $_{lb}$}‚

	  
	  \pstart \leavevmode% starting standard par
	स्यादेत‚त् [।] नातीन्द्रिय‚म‚र्थं प्र‚त्य‚क्ष‚तो जानात्य‚पि त्व‚नुमानेन ज्ञात्वोप‚दिश‚{\tiny $_{lb}$}‚तीति [।]
	{\color{gray}{\rmlatinfont\textsuperscript{§~\theparCount}}}
	\pend% ending standard par
      ‚{\tiny $_{lb}$}‚

	  
	  \pstart \leavevmode% starting standard par
	त‚न्न । य‚तः [।] \textbf{प्र‚त्य‚क्ष‚पूर्व‚काणाम}नुमानादीनाम‚{\tiny $_{५}$}‚\textbf{स‚म्भ‚वात्} । क‚दा\textbf{ऽत‚द्द‚र्श‚ने}‚{\tiny $_{lb}$}‚ तेन प्र‚त्य‚क्षेणातीन्द्रिय‚स्याद‚र्श‚ने [।] \textbf{प्र‚त्य‚क्षे}त्यादिनैत‚देव स्प‚ष्ट‚य‚ति । तेष्व‚तीन्द्रि‚{\tiny $_{lb}$}‚येष्व‚र्थेषु \textbf{प्र‚त्य‚क्षावृत्तेः} कार‚णात् प्र‚त्य‚क्ष‚पूर्व‚काणां \textbf{प्र‚माणान्त‚राणाम‚स‚म्भ‚वा}त् ।
	{\color{gray}{\rmlatinfont\textsuperscript{§~\theparCount}}}
	\pend% ending standard par
      ‚{\tiny $_{lb}$}‚

	  
	  \pstart \leavevmode% starting standard par
	अत‚दाल‚म्ब‚नं प्र‚तीय‚त इति त‚स्य प्र‚त्य‚क्षादेराल‚म्ब‚न‚म्विष‚यः । न त‚दाल‚म्ब‚न‚{\tiny $_{lb}$}‚\textbf{म‚त‚दाल‚म्ब‚नं} प्र‚त्य‚क्षाद्य‚विष‚य‚न्त‚स्य \textbf{प्र‚तीय‚ते} निश्च‚यार्थ‚माग‚{\tiny $_{६}$}‚मोन्विष्य‚त इत्य‚ध्या‚{\tiny $_{lb}$}‚हारः । \textbf{प्र‚माणान्त‚र‚स्य} त्व‚नुमानादेर‚तीन्द्रियेर्थे \textbf{वृत्तिः प्र‚त्य‚क्ष‚म‚ती}न्द्रिय‚विष‚य‚{\tiny $_{lb}$}‚\textbf{म‚न्वाक‚र्ष‚ति} साध‚य‚ति । प्र‚त्य‚क्ष‚पूर्व‚क‚त्वाद‚नुमानादेः । \textbf{इति} हेतोः \textbf{पुरुषातिश‚यो}‚{\tiny $_{lb}$}‚तीन्द्रियाणाम‚र्थानां द्र‚ष्टा मी मां स कै र\textbf{निवार्यः स्यात्} । निवारित‚श्च । \textbf{त‚स्मान्ना‚{\tiny $_{lb}$}‚स्त्य‚तीन्द्रिये}र्थे \textbf{प्र‚माणान्त‚र‚वृत्तिः} ।
	{\color{gray}{\rmlatinfont\textsuperscript{§~\theparCount}}}
	\pend% ending standard par
      ‚{\tiny $_{lb}$}‚

	  
	  \pstart \leavevmode% starting standard par
	\textbf{अत एवे}ति प्र‚मा‚{\tiny $_{७}$}‚णान्त‚रावृत्तेरेवाग‚म‚स्य वेद‚स्य्\textbf{आतीन्द्रिये} स्व‚र्गादिसाध‚न‚त्व- \leavevmode\ledsidenote{\textenglish{208a/PSVTa}}‚{\tiny $_{lb}$}‚ \textbf{ल‚क्ष‚णेर्थ‚विशेषे} या \textbf{वृत्ति}स्त‚स्या अनिश्च‚यात् । \textbf{अयं जै मि नि र‚न्यो वा} श ब र स्वा‚{\tiny $_{lb}$}‚  ‚{\tiny $_{lb}$}‚ ‚{\tiny $_{lb}$}‚ \leavevmode\ledsidenote{\textenglish{590/s}}म्यादिः पुरुषः । \textbf{स्व‚य‚मि}ति प‚रोप‚देश‚निर‚पेक्षः । वेद‚स्या\textbf{र्थं न वेत्ति । रागादिमान्}‚{\tiny $_{lb}$}‚ य‚तः । रागादिम‚त्त्व‚म‚प‚रिज्ञान‚कार‚णं । \textbf{अन्य‚तोपि न} वेत्ति वेद‚स्यार्थ‚न्त‚स्या‚{\tiny $_{lb}$}‚प्य‚स्य पुरुष‚स्य रागादिम‚त्त्वात् । \textbf{वेदोपि} स्व‚य‚म‚र्थं \textbf{न वेद‚य‚ति‚{\tiny $_{१}$}‚} न प्र‚काश‚य‚ति । तेन‚{\tiny $_{lb}$}‚ \textbf{वेदार्थ‚स्य कुतो ग‚तिः} । नैव ग‚तिर‚स्ति । य‚स्मात् \textbf{स‚र्व एव हि पुरुषोन‚तिक्रान्तो} रागा‚{\tiny $_{lb}$}‚\textbf{दिदोष}कृतो विप्ल‚वो \textbf{भ्रान्ति}र्य‚स्येति विग्र‚हः । त‚मिति वेद‚स्या\textbf{तीन्द्रियार्थ‚प्र‚ति‚{\tiny $_{lb}$}‚निय‚म}म‚य‚मेवास्यार्थो नाय‚मित्येवं \textbf{न वेत्ति स्व‚यं । नाप्येनं} रागादिम‚न्तं पुरुष\textbf{म‚न्यः}‚{\tiny $_{lb}$}‚ पुरुषो \textbf{वेद‚य‚ति त‚स्याप्य}न्य‚स्य तुल्य‚प्र‚संग‚तः । रागादिम‚त्त्वेन त‚स्याप्य‚ज्ञ‚त्वादित्य‚र्थः ।‚{\tiny $_{२}$}‚‚{\tiny $_{lb}$}‚ य‚तो \textbf{न ह्य‚न्धेन} स्व‚य‚म‚मार्ग‚ज्ञेना\textbf{कृष्य‚माणोऽन्धः प‚न्थानं प्र‚तिप‚द्य‚ते । नापि स्व‚य}मुप‚{\tiny $_{lb}$}‚देश‚निर‚पेक्षो \textbf{वेदः स्वार्थं प्र‚काश‚य‚ति} । कुतः [।] \textbf{उप‚देश‚वैय‚र्थ्य‚प्र‚संगात्} । जै मि‚{\tiny $_{lb}$}‚न्या दि व्याख्यान‚स्य निष्फ‚ल‚त्व‚प्र‚संगात् ।
	{\color{gray}{\rmlatinfont\textsuperscript{§~\theparCount}}}
	\pend% ending standard par
      ‚{\tiny $_{lb}$}‚

	  
	  \pstart \leavevmode% starting standard par
	\textbf{त‚त्त}स्मा\textbf{द‚य‚म‚प‚रिज्ञातार्थो} वेदाख्यः \textbf{श‚ब्द‚ग‚डुः} । घाटाम‚स्त‚क‚योर्म‚ध्ये मांस‚पिण्डो‚{\tiny $_{lb}$}‚ ग‚डुरुच्य‚ते निष्फ‚ल‚त्वात् । त‚द्व‚द्वैदिकोपि श‚ब्दोऽप‚रिज्ञाता‚{\tiny $_{३}$}‚र्थ‚त्वेन निष्फ‚ल‚त्वात् ।‚{\tiny $_{lb}$}‚ ग‚डुरिव ग‚डुः । \textbf{एव}मिति किम‚स्याय‚म‚र्थोथ‚वाय‚मिति संश‚यात् त‚द‚भिप्र‚स‚न्न‚स्य \textbf{श‚ल्य‚{\tiny $_{lb}$}‚भूतो} दुःख‚हेतुर‚त‚स्त‚मंगीकृत‚वेद‚म्पुरुषं \textbf{दुःख‚मासाद‚य‚ति} स्थाप‚य‚ति । कीदृशोऽ\textbf{स‚द्द‚{\tiny $_{lb}$}‚र्श‚न}मेव \textbf{स्नायुः} शिरा । तेनो\textbf{प‚निब}द्ध‚स्त‚त एव केन‚चित् कारुणिकेनाप्य‚प‚नेतु‚{\tiny $_{lb}$}‚मिच्छ‚ता \textbf{दुरुद्ध‚र} इत्युप‚ह‚स‚ति ।
	{\color{gray}{\rmlatinfont\textsuperscript{§~\theparCount}}}
	\pend% ending standard par
      ‚{\tiny $_{lb}$}‚

	  
	  \pstart \leavevmode% starting standard par
	\textbf{तेने}त्य‚प‚रिज्ञातार्थ‚त्वेन्\textbf{आग्निहोत्र‚ञ्जु‚{\tiny $_{४}$}‚हुयात् स्व‚र्ग‚काम इति श्रुतौ} वेद‚वाक्ये \textbf{खादे‚{\tiny $_{lb}$}‚च्छ्व‚मांस‚मित्येष नार्थः} किन्त्व‚न्योभिम‚तोर्थ \textbf{इत्य‚त्र का प्र‚मा} । नैव किञ्चित् प्र‚माणं ।
	{\color{gray}{\rmlatinfont\textsuperscript{§~\theparCount}}}
	\pend% ending standard par
      ‚{\tiny $_{lb}$}‚‚{\tiny $_{lb}$}‚‚{\tiny $_{lb}$}‚\textsuperscript{\textenglish{591/s}}

	  
	  \pstart \leavevmode% starting standard par
	\textbf{क्व‚चिदि}त्यादिना व्याच‚ष्टे । \textbf{अग्निहोत्रं जुहुयात् स्व‚र्ग‚काम} इत्यादि‚{\tiny $_{lb}$}‚ वाक्य‚स्य क्व‚चिद‚पि वाच्य‚त्वेनाभिम‚तेर्थे \textbf{प्र‚त्यास‚त्तिर‚हित‚स्या}न‚भिम‚तेर्थे विप्र‚क‚र्ष‚{\tiny $_{lb}$}‚र‚हित‚स्य वास्त‚व‚स्य स‚म्ब‚न्ध‚स्य निराक‚र‚णादुभ‚याभावः । \textbf{भूत‚विशे}षेग्नौ \textbf{य‚{\tiny $_{५}$}‚थाभि‚{\tiny $_{lb}$}‚म‚त}मिति म‚न्त्रादिपूतं \textbf{घृतादि प्र‚क्षिपेदित्य‚य‚म‚र्थो} न पुनः \textbf{श्व‚मांसं खादे}दित्य‚य‚म‚र्थ‚{\tiny $_{lb}$}‚ इति \textbf{नातिश‚य‚म्प‚श्यामः} ।
	{\color{gray}{\rmlatinfont\textsuperscript{§~\theparCount}}}
	\pend% ending standard par
      ‚{\tiny $_{lb}$}‚

	  
	  \pstart \leavevmode% starting standard par
	न‚न्व‚यं य‚थोक्तः । स‚र्व‚त्र पौरुषेयेप्याग‚मे स‚मानः प्र‚स‚ङ्गः । त‚मेव द‚र्श‚य‚न्नाह ।‚{\tiny $_{lb}$}‚ \textbf{प‚रोक्षो दैशि}को व‚क्ता । येषाम्व‚च‚नानामाग‚मानान्तानि त‚थोक्तानि । ते\textbf{षाम‚र्थं}‚{\tiny $_{lb}$}‚ य‚थाभिप्राय‚मिति य‚था दैशिकाभिम‚तः । इदानीन्त‚ना अ‚{\tiny $_{६}$}‚दृष्ट‚दैशिकाः पुरुषाः‚{\tiny $_{lb}$}‚ स‚म‚नुय‚न्त्य‚व‚ग‚च्छ‚न्त्याऽ\textbf{होस्विद् विप‚र्य‚य}न्दैशिकाभिप्राय‚विप‚रीत\textbf{मिति} निय‚माभावात्‚{\tiny $_{lb}$}‚ स‚र्व‚त्र संश‚य इति म‚न्य‚ते ।
	{\color{gray}{\rmlatinfont\textsuperscript{§~\theparCount}}}
	\pend% ending standard par
      ‚{\tiny $_{lb}$}‚

	  
	  \pstart \leavevmode% starting standard par
	\textbf{ने}त्यादिना प‚रिह‚र‚ति । \textbf{नायं स‚र्व‚त्र प्र‚संगः । उप‚देष्टु}राग‚म‚स्य प्र‚णेतु\textbf{र‚भिप्राय‚{\tiny $_{lb}$}‚प्र‚काश}नेनेति प्र‚थ‚मं प्र‚काश‚य‚तानेन स्वाभिप्राय‚स्त‚त्काल‚स‚न्निहितेभ्यः श्रोतृभ्यः‚{\tiny $_{lb}$}‚ क‚थितोस्य व‚च‚न‚स्याय‚म‚र्थ इति ।‚{\tiny $_{७}$}‚ तेपि श्रोतारोन्येभ्यः प्र‚काश‚य‚न्ति तेप्य‚न्येभ्य \leavevmode\ledsidenote{\textenglish{208b/PSVTa}}‚{\tiny $_{lb}$}‚ इत्याचार्योप‚देशे पार‚म्प‚र्य‚स्य \textbf{स‚म्भ‚वात् । न ह्य‚य‚म}भिप्राय‚क‚थ‚न‚ल‚क्ष‚णो वैदिकानां‚{\tiny $_{lb}$}‚ \textbf{श‚ब्दानाम‚दैशिकाना}मित्य‚पौरुषेयाणां \textbf{स‚म्भ‚व‚ति} दैशिक‚स्याभावात् ।
	{\color{gray}{\rmlatinfont\textsuperscript{§~\theparCount}}}
	\pend% ending standard par
      ‚{\tiny $_{lb}$}‚

	  
	  \pstart \leavevmode% starting standard par
	इत‚श्च पौरुषेयेषु य‚था प्र‚सिद्धार्थ‚स‚म्भ‚वो य‚स्मादाग‚म‚स्य प्र‚णेता लोको म‚या‚{\tiny $_{lb}$}‚ बोध‚नीय इति \textbf{लोक‚प्र‚त्याय‚नाभिप्राय‚श्च ब्रुवाणो लोके} य‚स्मिन्न‚र्थे श‚ब्द‚{\tiny $_{१}$}‚स्य \textbf{संकेत‚{\tiny $_{lb}$}‚स्त‚स्य प्र‚सिद्धिम‚नुपाल‚य‚ति} र‚क्ष‚त्य‚न्य‚था विफ‚ल‚न्त‚स्य प्र‚काश‚नं स्यात् । \textbf{त‚तोपि}‚{\tiny $_{lb}$}‚ लोक‚प्र‚सिद्ध‚संकेतानुपाल‚नाद‚पि \textbf{त‚द}र्थ‚सिद्धिः \textbf{स्यात्} । पुरुष‚प्र‚णीत‚स्याग‚म‚स्यार्थ‚{\tiny $_{lb}$}‚सिद्धिः स्यात् । अपिश‚ब्दात् पूर्वोक्ताच्च स‚म्प्र‚दाय‚स‚म्भ‚वात् । \textbf{नापौरुषेयाणां‚{\tiny $_{lb}$}‚ श‚ब्दानां} य‚थोक्तेन प्र‚कारेणार्थ‚सिद्धिः । किं कार‚णं । \textbf{त‚त्र} वैदिकेषु श‚ब्देषु \textbf{क‚स्य‚चित्}‚{\tiny $_{lb}$}‚ पुंसः स‚मीहाभआ‚{\tiny $_{२}$}‚वात् । व‚च‚न‚हेतो\textbf{र्विव‚क्षाया अभावात्} । एव‚न्ताव‚त् संप्र‚{\tiny $_{lb}$}‚दायादिस‚म्भ‚वेन पौरुषेय‚स्याग‚म‚स्यार्थ‚प‚रिज्ञान‚स‚म्भ‚वात् तुल्य‚प्र‚संग‚ता नास्तीत्यु‚{\tiny $_{lb}$}‚क्त‚म् [।]
	{\color{gray}{\rmlatinfont\textsuperscript{§~\theparCount}}}
	\pend% ending standard par
      ‚{\tiny $_{lb}$}‚‚{\tiny $_{lb}$}‚\textsuperscript{\textenglish{592/s}}

	  
	  \pstart \leavevmode% starting standard par
	अधुना न्यायानुसारेणैव पौरुषेयाणाम‚र्थ‚निश्च‚यो भ‚व‚तीत्याह । \textbf{अपि चे}त्यादि ।‚{\tiny $_{lb}$}‚ \textbf{न्याय‚मेव} युक्तिमेवा\textbf{नुपाल‚य‚न्तः प‚ण्डिताः} प्रेक्षापूर्व‚कारिणो बौ द्धा \textbf{हेया}दिषु \textbf{संघ}‚{\tiny $_{lb}$}‚ट‚न्ते । हेय‚स्य साश्र‚य‚स्य त्या‚{\tiny $_{३}$}‚गाय । \textbf{उपा}देय‚स्य साश्र‚य‚स्योपादानाय \textbf{प्र‚व‚र्त्त‚न्ते} ।‚{\tiny $_{lb}$}‚ त‚त्र हेयं दुःख‚मुपादेयो मोक्षः । त‚योराश्र‚यो य‚थासंख्यं क‚र्म‚क्लेशास्त‚त्त्व‚ज्ञान‚ञ्च ।
	{\color{gray}{\rmlatinfont\textsuperscript{§~\theparCount}}}
	\pend% ending standard par
      ‚{\tiny $_{lb}$}‚

	  
	  \pstart \leavevmode% starting standard par
	एत‚दुक्त‚म्भ‚व‚ति [।] अनेकार्थ‚त्व‚स‚म्भ‚वेपि श‚ब्दानां युक्तियुक्तं पुरुषार्थोप‚यो‚{\tiny $_{lb}$}‚गिन‚मेवाग‚मार्थ‚न्निश्चिन्व‚न्ति सौ ग ता न प‚रोप‚देश‚मात्रेण [।] त‚तोय‚म‚दोष इति‚{\tiny $_{lb}$}‚ [।]
	{\color{gray}{\rmlatinfont\textsuperscript{§~\theparCount}}}
	\pend% ending standard par
      ‚{\tiny $_{lb}$}‚

	  
	  \pstart \leavevmode% starting standard par
	त‚देवाह [।] \textbf{न प्र‚वाद‚मात्रेणेति} । न बृद्धानां प्र‚वाद‚{\tiny $_{४}$}‚मात्रेणेति \textbf{न स‚मानः‚{\tiny $_{lb}$}‚ प्र‚संगः । त‚च्च} न्यायानुपाल‚न‚पूर्व‚क‚मेवाग‚मे प्र‚व‚र्त्त‚नं \textbf{य‚थाऽव‚स‚रं प्र‚तिपाद‚यिष्यामः}‚{\tiny $_{lb}$}‚ प‚श्चात् ।
	{\color{gray}{\rmlatinfont\textsuperscript{§~\theparCount}}}
	\pend% ending standard par
      ‚{\tiny $_{lb}$}‚

	  
	  \pstart \leavevmode% starting standard par
	य‚दि न्यायानुपाल‚नेनाग‚मार्थ‚निश्च‚योऽत्य‚न्त‚प‚रोक्षे त‚र्ह्याग‚मार्थे निश्च‚यो‚{\tiny $_{lb}$}‚न स्यादित्याह । \textbf{न‚न्वि}त्यादि । \textbf{लोक‚स‚न्निवेशादि}रिति भाज‚न‚लोक‚स्य पृथि‚{\tiny $_{lb}$}‚व्यादेः स‚न्निवेशादिः । य‚थोक्तं [।]
	{\color{gray}{\rmlatinfont\textsuperscript{§~\theparCount}}}
	\pend% ending standard par
      ‚{\tiny $_{lb}$}‚
	  \bigskip
	  \begingroup
	
	    
	    \stanza[\smallbreak]
	  {\normalfontlatin\large ``\qquad}त‚त्र भाज‚न‚लोक‚स्य स‚न्निवेश‚मुश‚न्त्य‚धः ।&‚{\tiny $_{lb}$}‚ल‚क्ष‚षोड‚श‚कोद्बेध‚म‚संख्य‚म्वायुम‚ण्ड‚ल‚मित्यादि ।\edtext{\textsuperscript{*}}{\edlabel{pvsvt_592-1}\label{pvsvt_592-1}\lemma{*}\Bfootnote{\href{http://sarit.indology.info/?cref=ak.3}{ Abhidharmakośa, ch. 3}}}{\normalfontlatin\large\qquad{}"}\&[\smallbreak]
	  
	  
	  
	  \endgroup
	‚{\tiny $_{lb}$}‚

	  
	  \pstart \leavevmode% starting standard par
	आदिश‚ब्दात् । दान‚हिंसादिचेत‚नानामिष्टानिष्ट‚फ‚ल‚दानादि ।
	{\color{gray}{\rmlatinfont\textsuperscript{§~\theparCount}}}
	\pend% ending standard par
      ‚{\tiny $_{lb}$}‚

	  
	  \pstart \leavevmode% starting standard par
	देश‚स्व‚भावादिविप्र‚कृष्ट‚त्वाद\textbf{युक्तिविष‚योप्य‚र्थः} प्र‚तिप‚द्य‚ते भ‚व‚द्भिर्बौद्धैः [।]‚{\tiny $_{lb}$}‚ कुतः [।] \textbf{स‚म्भ‚व‚नीय‚पुरुष‚व‚च‚नात्} । य‚स्य प्र‚त्य‚क्षानुमान‚ग‚म्येर्थे व‚च‚न‚म‚वि‚{\tiny $_{lb}$}‚स‚म्वादि । त‚स्य तृतीय‚स्थाने व‚च‚नं स‚त्यार्थ‚त्वेन स‚म्भाव्य‚ते । त‚स्मात्‚{\tiny $_{६}$}‚स‚म्भाव‚नी‚{\tiny $_{lb}$}‚यात् पुरुष‚व‚च‚नाद‚र्थः प्र‚तिप‚द्य‚ते । त‚था च न न्यायानुपाल‚न‚पूर्विकाऽग‚मार्थे प्र‚वृत्ति‚{\tiny $_{lb}$}‚रिति स‚मान एव प्र‚संग इति म‚न्य‚ते ।
	{\color{gray}{\rmlatinfont\textsuperscript{§~\theparCount}}}
	\pend% ending standard par
      ‚{\tiny $_{lb}$}‚

	  
	  \pstart \leavevmode% starting standard par
	\textbf{ने}त्यादिना प‚रिह‚र‚ति । नात्य‚न्त‚प‚रोक्षोर्थः पुरुष‚व‚च‚नात् प्र‚तिप‚द्य‚ते । किं कार‚{\tiny $_{lb}$}‚ण‚म् [।] \textbf{अप्र‚त्य‚यात्} त‚द्विष‚य‚प्र‚माणाभावेनानिश्च‚यात् । प्र‚त्य‚क्षादिविष‚ये \textbf{स‚म्वादा‚{\tiny $_{lb}$}‚\leavevmode\ledsidenote{\textenglish{209a/PSVTa}} द‚त्य}न्त‚प‚रोक्षेपि स‚म्वादो निश्चीय‚त इत्य‚पि‚{\tiny $_{७}$}‚ मिथ्या य‚तो \textbf{न हि क्व‚चित्} प्र‚त्य‚क्षा‚{\tiny $_{lb}$}‚‚{\tiny $_{lb}$}‚ ‚{\tiny $_{lb}$}‚ \leavevmode\ledsidenote{\textenglish{593/s}}दिविष‚ये प्र‚माण‚स‚म्वादाद‚स्ख‚लितो दृष्ट इति कृत्वा \textbf{स‚र्व}न्त‚दुप‚दिष्ट‚म‚युक्तिग‚म्य‚म‚पि‚{\tiny $_{lb}$}‚ \textbf{त‚था भ‚व‚ति} । किं कार‚णं [।] \textbf{त‚त्प्र‚वृत्ते}रित्यादि । त‚स्य पुरुषातिश‚य‚स्य व‚च‚न\textbf{प्र‚वृ‚{\tiny $_{lb}$}‚त्तेर‚विस‚म्वादेन व्याप्त्या सिद्धे}र्विप‚र्य‚ये बाध‚क‚प्र‚माणाभावात् । न‚नु भ‚व‚तापि [।]
	{\color{gray}{\rmlatinfont\textsuperscript{§~\theparCount}}}
	\pend% ending standard par
      ‚{\tiny $_{lb}$}‚

	  
	  \pstart \leavevmode% starting standard par
	\hphantom{.}प्र‚त्य‚क्षेणानुमानेन द्विविधेनाप्य‚बाध‚न‚म् \href{http://sarit.indology.info/?cref=pv.3.214}{। १ । २१८} इत्यादिनैक‚देशावि‚{\tiny $_{lb}$}‚स‚म्वा‚{\tiny $_{१}$}‚द\textbf{न‚माग‚म}ल‚क्ष‚ण\textbf{मिष्ट‚मिति} [।]
	{\color{gray}{\rmlatinfont\textsuperscript{§~\theparCount}}}
	\pend% ending standard par
      ‚{\tiny $_{lb}$}‚

	  
	  \pstart \leavevmode% starting standard par
	अत आह । \textbf{अग‚त्या} चेत्यादि । प्र‚माण‚ग‚म्यार्थाविस‚म्वादेनात्य‚न्त‚प‚रोक्षेप्य‚वि‚{\tiny $_{lb}$}‚स‚म्वादिव‚च‚न‚मिती\textbf{द‚माग‚म‚ल‚क्ष‚ण‚म‚ग‚त्येष्ट}माग‚मात् प्र‚वृत्तौ व‚र‚मेवं प्र‚वृत्त \textbf{इति ।‚{\tiny $_{lb}$}‚ नातो} य‚थोक्तादाग‚माद‚तीन्द्रियार्थ\textbf{निश्च‚य}स्त‚स्याप्रामाण्यात् । \textbf{त‚दि}ति त‚स्मा\textbf{न्न‚{\tiny $_{lb}$}‚ प्र‚माण‚माग‚म} इत्युक्तं प्राक् ।
	{\color{gray}{\rmlatinfont\textsuperscript{§~\theparCount}}}
	\pend% ending standard par
      ‚{\tiny $_{lb}$}‚

	  
	  \pstart \leavevmode% starting standard par
	त‚देव‚म\textbf{पौरुषेयाणां श‚ब्दानाम‚र्थ‚ज्ञान‚न्ना}चा‚{\tiny $_{२}$}‚र्य\textbf{संप्र‚दायात्} । दैशिक‚स्याभावात् ।‚{\tiny $_{lb}$}‚ \textbf{न युक्ते}स्स‚काशाद‚त्य‚न्त‚प‚रोक्षेर्थे प्र‚माणाप्र‚वृत्तेः । \textbf{नापि लोका}द‚र्थ‚प्र‚तीतिर्लोक‚{\tiny $_{lb}$}‚प्र‚त्याय‚नाय प्र‚योगाभावेन लोक‚संकेतानुस‚र‚णायोगादिति । \textbf{त‚त्रा}पौरुषेयेषु श‚ब्दे‚{\tiny $_{lb}$}‚ष्व‚र्थानाम\textbf{प्र‚तिप‚त्तिरेव} न्याय्या । त‚त्रापि वैदिके श‚ब्दे \textbf{प्र‚सिद्धो लोक‚वादो} य‚था‚{\tiny $_{lb}$}‚ ग्न्यादि श‚ब्दाद् दाह‚पाकादिस‚म‚र्थेर्थे प्र‚व‚र्त्त‚न्त इत्यादि ।
	{\color{gray}{\rmlatinfont\textsuperscript{§~\theparCount}}}
	\pend% ending standard par
      ‚{\tiny $_{lb}$}‚

	  
	  \pstart \leavevmode% starting standard par
	यो \textbf{लोक‚व्य‚व‚हार}स्स चेद् वेदार्थ\textbf{प्र‚तिप‚त्तिहेतुः} ।
	{\color{gray}{\rmlatinfont\textsuperscript{§~\theparCount}}}
	\pend% ending standard par
      ‚{\tiny $_{lb}$}‚

	  
	  \pstart \leavevmode% starting standard par
	उत्त‚र‚माह । \textbf{त‚त्रे}त्यादि । \textbf{त‚त्र} लोकेऽविद्यात्वे कः पुरुषो\textbf{तीन्द्रियार्थ‚दृक्} ।‚{\tiny $_{lb}$}‚ येनातीन्द्रियार्थ‚दृशाऽनेकार्थेषु श‚ब्देष्वेकार्थ‚प्र‚तिनिय‚माभावादाशंक्य‚मानार्थ‚विशेषेषु‚{\tiny $_{lb}$}‚ वैदिके\textbf{ष्व‚र्थोय}म‚तीन्द्रियो \textbf{विवेचितो} विभ‚क्तोय‚मेवास्यार्थो नाय‚मिति [।] नैव‚{\tiny $_{lb}$}‚ तादृशः क‚श्चिद‚स्ति ।
	{\color{gray}{\rmlatinfont\textsuperscript{§~\theparCount}}}
	\pend% ending standard par
      ‚{\tiny $_{lb}$}‚

	  
	  \pstart \leavevmode% starting standard par
	\textbf{न ही}त्यादिना व्याच‚ष्टे । \textbf{न ह्य‚यं लोक‚व्य‚{\tiny $_{४}$}‚व‚हारोपौरुषेया}द‚कृत‚का\textbf{च्छ‚{\tiny $_{lb}$}‚  ‚{\tiny $_{lb}$}‚ ‚{\tiny $_{lb}$}‚ \leavevmode\ledsidenote{\textenglish{594/s}}ब्दार्थ‚स‚म्ब‚न्धाद्} भ‚व‚ति [।] \textbf{किन्त‚र्हि} [।] अभिप्राय‚क‚थ‚न‚ल‚क्ष‚णात् \textbf{स‚म‚यात्} ।‚{\tiny $_{lb}$}‚ किमिव [।] \textbf{स्व‚शास्त्र‚का}रेत्यादि । \textbf{स्व‚शास्त्र‚कारा}णां पा णि नि प्र‚भृतीनां \textbf{स‚म‚यात्}‚{\tiny $_{lb}$}‚ संकेतात् । त‚त्स‚म‚यानुसारिणां \textbf{पा णि नी या दीनां} वृद्धिगुण‚संज्ञादि\textbf{व्य‚व‚हार‚व‚त्} किं‚{\tiny $_{lb}$}‚ कार‚ण‚म् [।] \textbf{उप‚देशापेक्ष‚णात्} ।
	{\color{gray}{\rmlatinfont\textsuperscript{§~\theparCount}}}
	\pend% ending standard par
      ‚{\tiny $_{lb}$}‚

	  
	  \pstart \leavevmode% starting standard par
	य‚दि हि लौकिकोपि व्य‚व‚हारो निस‚र्ग‚सिद्धः स्यात् त‚दा प‚{\tiny $_{५}$}‚रोप‚देश‚न्नापेक्षेत ।‚{\tiny $_{lb}$}‚ न च य‚था साम‚यिकात् स‚म्ब‚न्धादुप‚देशापेक्षाद‚र्थ‚प्र‚तीतिस्त‚थाऽपौरुषेयाद‚पि‚{\tiny $_{lb}$}‚ स‚म्ब‚न्धादुप‚देशापेक्षादेव वेदार्थ‚प्र‚तीतिः [।] य‚तो \textbf{न ह्य‚पौरुषेये त‚स्मिन्} वैदिके‚{\tiny $_{lb}$}‚ श‚ब्दार्थ‚स‚म्ब‚न्धे प\textbf{रोप‚देशो युक्तः} । किङ्कार‚णं [।] \textbf{त‚स्य} वेदार्थ‚स्य \textbf{केन‚चिद‚पि}‚{\tiny $_{lb}$}‚ पुरुषेणाज्ञानाद‚ज्ञानां च वेदार्थ‚स्या\textbf{तीन्द्रिय‚त्वात्} । त‚दुक्तं ।
	{\color{gray}{\rmlatinfont\textsuperscript{§~\theparCount}}}
	\pend% ending standard par
      ‚{\tiny $_{lb}$}‚
	  \bigskip
	  \begingroup
	
	    
	    \stanza[\smallbreak]
	  {\normalfontlatin\large ``\qquad}श्रेयः साध‚न‚ता ह्येषान्नि‚{\tiny $_{६}$}‚त्य‚म्वेदात् प्र‚तीय‚ते ।&‚{\tiny $_{lb}$}‚ताद्रूप्येण च ध‚र्म‚त्व‚न्त‚स्मान्नेन्द्रिय‚गोच‚रः । \href{http://sarit.indology.info/?cref=\%C5\%9Bv}{श्लो० चो० १४}&‚{\tiny $_{lb}$}‚श्रेयो हि पुरुष‚प्रीतिः सा द्र‚व्य‚गुण‚क‚र्म‚भिः ।&‚{\tiny $_{lb}$}‚चोद‚नाल‚क्ष‚णैः साध्या त‚स्मात्तेष्वेव ध‚र्म‚तेति । \href{http://sarit.indology.info/?cref=\%C5\%9Bv}{श्लो० चो० १९१}{\normalfontlatin\large\qquad{}"}\&[\smallbreak]
	  
	  
	  
	  \endgroup
	‚{\tiny $_{lb}$}‚

	  
	  \pstart \leavevmode% starting standard par
	\textbf{ऐन्द्रिय‚क‚त्वे} तु वेदार्थ‚स्याभ्युप‚ग‚म्य‚माने \textbf{स्व‚य}मुप‚देश‚म‚न्त‚रेण वेदार्थ‚स्य \textbf{प्र‚तीति‚{\tiny $_{lb}$}‚\leavevmode\ledsidenote{\textenglish{209b/PSVTa}} प्र‚स‚ङ्गात् । रूपादिव‚त्} । य‚था रूपादीनामैन्द्रिय‚क‚त्वादुप‚देश‚म‚न्त‚रेण प्र‚तीतिस्त‚द्व‚त् ।
	{\color{gray}{\rmlatinfont\textsuperscript{§~\theparCount}}}
	\pend% ending standard par
      ‚{\tiny $_{lb}$}‚

	  
	  \pstart \leavevmode% starting standard par
	अथातीन्द्रियेपि वेदार्थे निश्च‚यार्थ‚मु प‚देशोऽभ्युप‚ग‚म्य‚ते । त\textbf{दोप‚देशे च} ।‚{\tiny $_{lb}$}‚ वेदार्थोप‚देष्टॄणां \textbf{पुंसां स्व‚त‚न्त्राणा}मिति रागाद्य‚भिभ‚वेन स‚म्य‚ग्ज्ञानाभावाद् य‚थेच्छं‚{\tiny $_{lb}$}‚ प्र‚वृत्तानां \textbf{य‚थात‚त्त्व‚मुप‚देशेनाविस‚म्वाद‚स्यासिद्धेर‚नाश्वासो} न निश्च‚यो वेदार्थे । न‚{\tiny $_{lb}$}‚ ह्य‚ज्ञा य‚थात‚त्त्व‚मुप‚देष्टुं स‚म‚र्था इति । न पुरुषाः स्वेच्छ‚योप‚दिश‚न्ति ।
	{\color{gray}{\rmlatinfont\textsuperscript{§~\theparCount}}}
	\pend% ending standard par
      ‚{\tiny $_{lb}$}‚

	  
	  \pstart \leavevmode% starting standard par
	किन्तु \textbf{वेद‚व‚द् वेद‚व्याख्यान‚म‚प्य‚पौरु‚{\tiny $_{१}$}‚षे}य‚मुप‚देश‚प‚र‚म्प‚रायात‚ल‚क्ष‚णात् \textbf{संप्र‚दाया‚{\tiny $_{lb}$}‚विच्छेदादाग‚तं} । त‚तो य‚थोक्ताद् वेद‚व्याख्यानाद् वेदार्थ‚सिद्धि\textbf{रिति चेद्} । [।]
	{\color{gray}{\rmlatinfont\textsuperscript{§~\theparCount}}}
	\pend% ending standard par
      ‚{\tiny $_{lb}$}‚

	  
	  \pstart \leavevmode% starting standard par
	उत्त‚र‚माह । \textbf{त‚स्यापी}त्यादि । \textbf{त‚स्यापि} वेद‚व्याख्यान‚स्य \textbf{श‚ब्दात्म‚क‚त्वे} श‚ब्द‚{\tiny $_{lb}$}‚स्व‚भाव‚त्वे वेदेन \textbf{तुल्यः प‚र्य‚नुयोगः । क‚थ‚म‚स्या}पौरुषेय‚स्य वेद‚व्याख्यान‚स्\textbf{यार्थो‚{\tiny $_{lb}$}‚ विदि}त \textbf{इति} कुतोस्माद् वेदार्थ‚निश्च‚यः । पौरुषेये त्वाग‚मे नाय‚न्दोषो‚{\tiny $_{२}$}‚ य‚तः \textbf{पुरुषो हि‚{\tiny $_{lb}$}‚ ‚{\tiny $_{lb}$}‚ \leavevmode\ledsidenote{\textenglish{595/s}}स्व‚यं स‚मितानां} संकेतितानां \textbf{श‚ब्दानाम}र्थं \textbf{शृङ्ग‚ग्राहिकायापी}त्य‚स्य श‚ब्द‚स्याय‚म‚र्थ‚{\tiny $_{lb}$}‚ इत्य‚नेन \textbf{ताव‚द‚बुध}म‚ज्ञं पुरुषं \textbf{बोध‚येदित्य‚स्ति पौरुषेयाणां} श‚ब्दाना\textbf{म‚र्थ‚ग‚ता}व‚र्थ‚{\tiny $_{lb}$}‚ज्ञाने \textbf{उपायो} नापौरुषेयाणां । त‚था \textbf{ह्य‚पौरुषेय‚स्तु श‚ब्दो नैवं क‚रोति} य‚थायं‚{\tiny $_{lb}$}‚ म‚मार्थो रागादिम‚ता ग्राह्य इति । \textbf{न चास्या}पौरुषेय‚स्य श‚ब्द‚स्य \textbf{क‚श्चि} ज्जै मि‚{\tiny $_{lb}$}‚न्या‚{\tiny $_{३}$}‚दी रागादिमान् \textbf{क्व‚चिद}तीन्द्रियेर्थे \textbf{स‚म्ब‚न्ध‚निय‚मं ज्ञातुमीशः} श‚क्त \textbf{इति ।‚{\tiny $_{lb}$}‚ अप्र‚तिप‚त्तिरेव त‚द‚र्थ‚स्य} वेदार्थ‚स्य ।
	{\color{gray}{\rmlatinfont\textsuperscript{§~\theparCount}}}
	\pend% ending standard par
      ‚{\tiny $_{lb}$}‚

	  
	  \pstart \leavevmode% starting standard par
	\textbf{अपि च} [।] भ‚व‚तु नामापौरुषेयो \textbf{वेद‚स्त‚द्व्याख्यान}ञ्च त‚थापि रागादिम‚ता‚{\tiny $_{lb}$}‚ \textbf{पुरुषेण पुरुषायोप‚दिश्य‚मान‚म‚न‚ष्ट‚संप्र‚दाय‚मे}वाद्य‚त्वे\textbf{प्य‚नुव‚र्त्त‚त इति \textbf{अत्रापि}}‚{\tiny $_{lb}$}‚ प्र‚माणाभावात् \textbf{स‚म‚यः} श‚प‚थादिः \textbf{श‚र‚णं । आग‚म‚भ्रंश‚कारिणा}मित्यादिना संप्र‚{\tiny $_{४}$}‚दाय‚{\tiny $_{lb}$}‚विच्छेदेन र‚च‚नान्त‚र‚स‚म्भ‚व‚मेव स‚म‚र्थ‚य‚ते । आग‚म‚भ्रंश‚कारिणां पुंसाम‚न्य‚था ।‚{\tiny $_{lb}$}‚ पूर्व‚र‚च‚नावैप‚रीत्येन र‚च‚नाद‚र्श‚नादिति स‚म्ब‚न्धः । अन्य‚था र‚च‚नायां कार‚ण‚माह ।‚{\tiny $_{lb}$}‚ आहोपुरुषिक‚येत्यादि । \textbf{आहोपुरुषि}क‚येत्य‚हंमानित्वेन । य‚था सां ख्य नाश‚क‚{\tiny $_{lb}$}‚मा ध वे न सांख्य‚सिद्धान्त‚स्यान्य‚था र‚च‚नं कृतं । \textbf{त‚द्द‚र्श‚न‚विद्वेषेण} वान्य‚था र‚च‚ना‚{\tiny $_{lb}$}‚स‚म्भ‚वात् ।‚{\tiny $_{५}$}‚ य‚था म हा या न विद्विष्टानां म‚हायान‚प्र‚तिरूप‚क‚सूत्रान्त‚र‚र‚च‚नं ।‚{\tiny $_{lb}$}‚ त‚त्प्र‚तिप‚न्न‚ख‚लीक‚र‚णायेति । \textbf{त}स्मिन् द‚र्श‚ने यः \textbf{प्र‚तिप‚न्नः} पुरुष‚स्त‚स्य \textbf{ख‚लीकारा‚{\tiny $_{lb}$}‚यान्य‚थार‚च‚नास‚म्भ‚वः} । त‚त्प्र‚तिप‚न्न‚ख‚लीकार एव क‚थं । \textbf{धूर्त‚व्य‚स‚नेन} । व्य‚स‚न‚मि‚{\tiny $_{lb}$}‚द‚न्धूर्त्तानां य‚त्प‚रः ख‚लीक‚र्त‚व्य इति । \textbf{अन्य‚तो वा कुत}श्चिल्लाभादिकात् ।
	{\color{gray}{\rmlatinfont\textsuperscript{§~\theparCount}}}
	\pend% ending standard par
      ‚{\tiny $_{lb}$}‚

	  
	  \pstart \leavevmode% starting standard par
	\textbf{अपि चात्र} वेदार्थ‚निर्ण्ण‚ये । \textbf{भ‚वा‚{\tiny $_{६}$}‚न्} वे द वा दी । \textbf{स्व‚वादानुरागात्कार‚णात् ।‚{\tiny $_{lb}$}‚ स्व‚मेव मुख‚व‚र्ण्णं} । मुखं व‚र्ण्ण‚य‚ति शोभ‚य‚तीति मुख‚व‚र्ण्णः । स्वाभ्युप‚ग‚म‚स्तं \textbf{नून‚{\tiny $_{lb}$}‚म्विस्मृत‚वान्} । येन रागादिम‚लिनेभ्यः पुरुषेभ्यो वेदार्थ‚निर्ण्ण‚यः प्रार्थ्य‚ते ।
	{\color{gray}{\rmlatinfont\textsuperscript{§~\theparCount}}}
	\pend% ending standard par
      ‚{\tiny $_{lb}$}‚

	  
	  \pstart \leavevmode% starting standard par
	\textbf{पुरुष} इत्यादिना मुख‚व‚र्ण्ण‚माच‚ष्टे । य‚स्मात् \textbf{पुरुषो रागादिभिरुप‚प्लुतो}‚{\tiny $_{lb}$}‚ विप‚र्य‚स्तो\textbf{ऽनृत‚म‚पि ब्रूयादिति} कृत्वा \textbf{नास्य} पुरुष‚स्य \textbf{व‚च‚{\tiny $_{७}$}‚न‚म्प्र‚माण‚मिति} [।] \leavevmode\ledsidenote{\textenglish{210a/PSVTa}}‚{\tiny $_{lb}$}‚ ‚{\tiny $_{lb}$}‚ \leavevmode\ledsidenote{\textenglish{596/s}}\textbf{त‚द}नृत‚वादित्व\textbf{मिहापि} वेदार्थ‚स‚म्प्र‚दायानुक्र‚मे । जै मि न्यादिना पुरुषेण क्रिय‚माणे‚{\tiny $_{lb}$}‚ \textbf{किन्न प्र‚त्य‚वेक्ष‚ते स‚म्भ‚व‚ति न वेति} । य‚स्मात् \textbf{स एव} पुरुषो रागादिमान् जै मि नि‚{\tiny $_{lb}$}‚प्र‚भृति\textbf{र्वेद‚म्वेदार्थ‚म्वोप‚दिश‚न्} रागा\textbf{द्युप‚प्ल‚वात्} कार‚णाद\textbf{न्य‚थाप्युप‚दिशेदिति} मिथ्या‚{\tiny $_{lb}$}‚र्थाशंका नैव निवार्य‚ते ।
	{\color{gray}{\rmlatinfont\textsuperscript{§~\theparCount}}}
	\pend% ending standard par
      ‚{\tiny $_{lb}$}‚

	  
	  \pstart \leavevmode% starting standard par
	आशंकाकार‚णान्येव द‚र्श‚य‚न्नाह । \textbf{श्रूय‚न्ते ही}त्यादि ‚{\tiny $_{१}$}‚\textbf{कैश्चित् पुरुषैर्या} ज्ञ व‚{\tiny $_{lb}$}‚ ल्क्य प्र‚भृतिभि\textbf{रुत्स‚न्नोद्धृतानि} । उत्स‚न्नान्य‚न्त‚रितानि स‚न्ति वेद‚स्य \textbf{शाखान्त‚{\tiny $_{lb}$}‚राणि} । उद्धृतानि स्मृत्वा स्मृत्वा पुन‚रार‚चितानि । तानि च य‚थात‚त्त्वं स्मृत्वो‚{\tiny $_{lb}$}‚द्धृतानीति किम‚त्र प्र‚माणं । \textbf{इदानीम‚पि कानिचिद्} आ हू र क प्र‚भृतीनि शाखा‚{\tiny $_{lb}$}‚न्त‚राणि \textbf{विर‚लाध्येतृका}णि । स्व‚ल्पाध्येतृका\textbf{नि}\edtext{}{\lemma{ल्पाध्येतृका}\Bfootnote{? णि}} दृश्य‚न्ते । ते क‚तिप‚या‚{\tiny $_{lb}$}‚ध्येतारो न स‚मारोप्योप‚दिश‚न्ती‚{\tiny $_{२}$}‚ति किम‚त्र प्र‚माणं । य‚था विर‚लाध्येतृकाणि‚{\tiny $_{lb}$}‚ शाखान्त‚राणि दृश्य‚न्ते । \textbf{त‚द्व‚त् प्र‚चुराध्येतृकाणाम‚पि} ब‚हुत‚राध्येतृकाणाम‚पि‚{\tiny $_{lb}$}‚ शाखान्त‚राणां \textbf{क‚स्मिंश्चि}दित्य‚तिक्रान्ते \textbf{काले संसार‚स‚म्भ‚वात्} । अल्पाध्येतृक‚त्व‚{\tiny $_{lb}$}‚स‚म्भाव‚नास‚म्भ‚वात् । तेषां प्राग‚पि प्र‚चुराध्येतृक‚त्वे प्र‚माणाभावात् । इदानीन्त‚र्हि‚{\tiny $_{lb}$}‚ क‚थं प्र‚चुराध्येतृकाणि तानीत्य‚त आह । \textbf{पुन}रित्यादि ।‚{\tiny $_{३}$}‚ \textbf{पुनः} कालान्त‚रेणाप्त‚{\tiny $_{lb}$}‚त्वेन \textbf{स‚म्भावित}स्य \textbf{पुरुष}स्य \textbf{प्र‚त्य‚यात्} प्रामाण्यात् त‚द‚न्यास‚म्भावित‚पुरुषाध्य‚य‚न‚{\tiny $_{lb}$}‚वैप‚रीत्येन संहृतानाम‚ध्येतॄणां \textbf{प्र‚चुर‚तोप‚ग‚म‚न}स्य बाहुल्योप‚ग‚म‚न‚स्य या \textbf{स‚म्भाव‚ना}‚{\tiny $_{lb}$}‚ त‚स्याः \textbf{स‚म्भ‚वाद}निश्च‚यः ।
	{\color{gray}{\rmlatinfont\textsuperscript{§~\theparCount}}}
	\pend% ending standard par
      ‚{\tiny $_{lb}$}‚

	  
	  \pstart \leavevmode% starting standard par
	किञ्च । ये ते पुरुषा विर‚लीभूताः शाखान्त‚राणां प्र‚तान‚यितार\textbf{स्तेषां प्र‚ता‚{\tiny $_{lb}$}‚न‚यितॄणाम‚न्य‚थोप‚{\tiny $_{४}$}‚देश‚स‚म्भ‚वा}दिति स‚म्ब‚न्धः । त‚था चानाश्वास इत्य‚भिप्रायः ।‚{\tiny $_{lb}$}‚ कीदृशानां प्र‚तान‚यितॄणां \textbf{क‚दाचिद‚धीत‚विस्मृताध्य‚य‚नानां} । अधीतं स‚द् विस्मृत‚{\tiny $_{lb}$}‚म‚ध्य‚य‚नं यैस्ते त‚थोक्ताः । य‚थाधीतं विस्मृतास्स‚न्त‚स्तेऽन्य‚थापि प्र‚तान‚येयुरित्य‚र्थः ।‚{\tiny $_{lb}$}‚ केन कार‚णेनेत्याह । \textbf{अन्येषा}न्त‚द‚भिप्र‚स‚न्नानाम‚ध्येतॄणान्त‚स्मिन्न‚ध्याप‚यित‚रि । या‚{\tiny $_{lb}$}‚ म‚ह‚त्व\textbf{स‚म्भाव‚{\tiny $_{५}$}‚ना} । त\textbf{स्या भ्रंश‚भ‚यात्} । य‚द्य‚ह‚म‚न्य‚थापि नोप‚दिशेयं । नून‚मेते‚{\tiny $_{lb}$}‚ म‚य्याप्त‚स‚म्भाव‚नां ज\textbf{हा}\edtext{}{\lemma{ज}\Bfootnote{? ज‚ह}}तीति । आदि श‚ब्दादाहोपुरुषादिकात् अन्य‚थोप‚{\tiny $_{lb}$}‚\textbf{देश‚स‚म्भ‚वः} ।
	{\color{gray}{\rmlatinfont\textsuperscript{§~\theparCount}}}
	\pend% ending standard par
      ‚{\tiny $_{lb}$}‚‚{\tiny $_{lb}$}‚\textsuperscript{\textenglish{597/s}}

	  
	  \pstart \leavevmode% starting standard par
	प्र‚तान‚य‚न्तु नाम तेऽन्य‚था । त‚थाप्य‚ध्येतारो न त‚था प्र‚तिप‚द्येर‚न्नित्याह ।‚{\tiny $_{lb}$}‚ \textbf{त‚दि}त्यादि । \textbf{त‚स्य} स‚म्भावित‚स्य पुरुष‚स्य \textbf{प्र‚त्य‚याच्च} त‚दुक्तानां । स‚म्भावित‚प्र‚णे‚{\tiny $_{lb}$}‚तृ‚{\tiny $_{६}$}‚पुरुषा\textbf{भिप्र‚स‚न्नानाम‚विचारेण} वेदाध्य‚य‚न\textbf{प्र‚तिप‚त्तेर}न्य‚थासंप्र‚दाय‚स‚म्भ‚वः । पुरुष‚{\tiny $_{lb}$}‚प्रामाण्यात् प्र‚वृत्तिमेव साध‚य‚न्नाह । \textbf{ब‚हुष्व‚प्य‚ध्येतृषु} म‚ध्ये \textbf{स‚म्भावितात् पुरुषाद्‚{\tiny $_{lb}$}‚ ब‚हुलं लोके प्र‚वृत्तिद‚र्श‚नात्} ।
	{\color{gray}{\rmlatinfont\textsuperscript{§~\theparCount}}}
	\pend% ending standard par
      ‚{\tiny $_{lb}$}‚

	  
	  \pstart \leavevmode% starting standard par
	न‚नु स‚म्भावितात् प्र‚तिप‚त्तौ स‚म्वाद एवेत्य‚त आह । \textbf{त‚तोपि} स‚म्भावितात्‚{\tiny $_{lb}$}‚ पुरुषात् \textbf{क‚थ‚ञ्चित्} केन‚चित् कार‚णेन स‚म्भा‚{\tiny $_{७}$}‚व‚ना भ्रंश‚भ‚यादिना । \textbf{विप्र‚ल‚म्भ}स्य \leavevmode\ledsidenote{\textenglish{210b/PSVTa}}‚{\tiny $_{lb}$}‚ विस‚म्वाद‚स्य \textbf{स‚म्भ‚वात्} ।
	{\color{gray}{\rmlatinfont\textsuperscript{§~\theparCount}}}
	\pend% ending standard par
      ‚{\tiny $_{lb}$}‚

	  
	  \pstart \leavevmode% starting standard par
	उप‚च‚य‚हेतुमाह । \textbf{किंचे}त्यादि । \textbf{प‚रिमिता}श्च ते \textbf{व्याख्यातृ}पु\textbf{रुषा}श्च तेषां‚{\tiny $_{lb}$}‚ \textbf{प‚र‚म्प‚रामेवात्र} वेद‚व्याख्याने \textbf{भ‚व‚तां} मी मां स का नां \textbf{श्रृणुमः । त‚त्र} तेषु म‚ध्ये \textbf{क‚श्चिद्}‚{\tiny $_{lb}$}‚ वेद‚स्य व्याख्याता \textbf{द्विष्टादीनाम‚न्य‚त‚मः स्यात्} । क‚श्चिद् वेद‚द‚र्श‚ने विद्विष्टः सोन्य‚{\tiny $_{lb}$}‚थाप्युप‚दिशेत् । त‚था क‚श्चिद‚ज्ञः‚{\tiny $_{१}$}‚ धूर्त्तो वा । त‚था च वेद‚व्याख्यायाम\textbf{नाश्वा}सः ।
	{\color{gray}{\rmlatinfont\textsuperscript{§~\theparCount}}}
	\pend% ending standard par
      ‚{\tiny $_{lb}$}‚

	  
	  \pstart \leavevmode% starting standard par
	य‚त एव\textbf{न्त‚स्मान्ना}पौरुषेयाद् वेद\textbf{व्याख्यानाद्} वेदार्थ‚सिद्धिः । \textbf{नापि साम‚यि‚{\tiny $_{lb}$}‚कात्} सांकेतिका\textbf{ल्लोक‚व्य‚व‚हाराद्वेदार्थ‚सिद्धिः} । लोक‚स्य रागाद्युप‚प्लुत‚त्वात् ।
	{\color{gray}{\rmlatinfont\textsuperscript{§~\theparCount}}}
	\pend% ending standard par
      ‚{\tiny $_{lb}$}‚

	  
	  \pstart \leavevmode% starting standard par
	भ‚व‚न्तु वा निस‚र्ग‚सिद्धा वैदिकाः श‚ब्दास्स‚म‚य‚निर‚पेक्षाः । एव‚म‚प्य\textbf{साम‚यि‚{\tiny $_{lb}$}‚क‚त्वे}भ्युप‚ग‚म्य‚माने । \textbf{व्य‚व‚हारे नानार्थानां} ग‚वादि\textbf{श‚ब्दानान्द‚र्श‚{\tiny $_{२}$}‚नात्} । स‚र्व‚त्र वैदि‚{\tiny $_{lb}$}‚केपि श‚ब्दे न त‚दाशंकाऽनिवृत्तेरिति स‚म्ब‚न्धः । नानार्थाशंकाया अनिवृत्तेरित्य‚र्थः ।‚{\tiny $_{lb}$}‚ लौकिकानामेव नानार्थ‚त्व‚न्न वैदिकानामिति चेदाह । \textbf{क‚स्य‚चिदि}त्यादि । त‚स्यापि‚{\tiny $_{lb}$}‚ वैदिक‚स्य \textbf{क‚स्य‚चिद प्र‚सिद्धार्थ}स्येति य‚त्र श‚ब्दः प्र‚सिद्धो नार्थ‚स्त‚स्या\textbf{प्र‚सिद्ध‚स्य} वा‚{\tiny $_{lb}$}‚ स्व‚रूपेण श‚ब्द‚स्य । \textbf{पुन‚र्व्युत्प‚त्तिप्र‚द‚र्श‚नेन्}आर्थः क‚ल्प‚नीयः पुरुषैः । त‚{\tiny $_{३}$}‚था च स‚ति‚{\tiny $_{lb}$}‚ पुनः किं य‚था स्थित‚मेवार्थं पुरुषो वैदिकानां श‚ब्दानामुप‚दिश‚ति किम्वाविप‚रीत‚{\tiny $_{lb}$}‚मिति \textbf{स‚र्व‚त्र} नानार्था \textbf{श‚ङ्काया अनिवृत्तेः} ।
	{\color{gray}{\rmlatinfont\textsuperscript{§~\theparCount}}}
	\pend% ending standard par
      ‚{\tiny $_{lb}$}‚

	  
	  \pstart \leavevmode% starting standard par
	अनिष्टेर्थे प्र‚युक्ता अपि वैदिकाश्श‚ब्दा न त‚त्र प्र‚तीतिं ज‚न‚य‚न्तीत्य‚प्य‚युक्तं ।‚{\tiny $_{lb}$}‚ ‚{\tiny $_{lb}$}‚ \leavevmode\ledsidenote{\textenglish{598/s}}य‚तः \textbf{स‚र्वेषां} श‚ब्दानां लौकिकानां वैदिकानां च \textbf{य‚थार्थ‚न्नियोगेपि} । वीप्सायाम‚{\tiny $_{lb}$}‚व्य‚यीभावः । य‚स्मिन् य‚स्मिन्न‚र्थे नियोग‚स्संकेत‚{\tiny $_{४}$}‚स्त‚स्मिन् स‚त्य\textbf{प्य‚वैगुण्येन} त‚त्र‚{\tiny $_{lb}$}‚ त‚त्रार्थे \textbf{य‚थास‚म‚यं} य‚थासंकेतं \textbf{प्र‚तीतिज‚न‚ना}त् । न चानिष्टेन्य‚थाप्र‚तीतिज‚न‚नं । य‚त‚{\tiny $_{lb}$}‚ \textbf{इष्टानिष्ट‚यो}र‚र्थ‚योः प्र‚त्यास‚त्तिविप्र‚क‚र्षाभावेन प्र‚तीत‚ज‚न‚न‚स्या\textbf{विशेषात्} । त‚त‚{\tiny $_{lb}$}‚श्चा\textbf{विशिष्टानां स‚र्वार्थेषु} वैदिकानां श‚ब्दानामे\textbf{क‚म‚र्थं} किम्विशिष्ट\textbf{म‚त्य‚क्ष‚संयोगं} [।]‚{\tiny $_{lb}$}‚ श‚ब्देन स‚ह स‚म्ब‚न्धो य‚स्य स‚{\tiny $_{५}$}‚ त‚थोक्तः । \textbf{अन‚त्य‚क्ष‚द‚र्शिनि} । अर्वाग्द‚र्शिनि \textbf{पुरुष‚{\tiny $_{lb}$}‚सामान्ये को विवेच‚ये}त् [।] नैव क‚श्चिद् विवेच‚येत् । \textbf{य‚तो लोकात् प्र‚तीतिः} स्यात् ।
	{\color{gray}{\rmlatinfont\textsuperscript{§~\theparCount}}}
	\pend% ending standard par
      ‚{\tiny $_{lb}$}‚

	  
	  \pstart \leavevmode% starting standard par
	\textbf{अपि च} [।] अयं मी मां स कः \textbf{स्व‚य‚म‚पि न स‚र्व‚त्र} वेदे \textbf{प्र‚सिद्धिम‚नुस‚र‚ति} । येन‚{\tiny $_{lb}$}‚ प्र‚सिद्धाल्लोक‚प्र‚वादाद् वेदार्थ‚ग‚तिः स्यात् । किं कार‚णं । \textbf{य‚स्मात् स्व‚र्गोर्व‚श्यादि‚{\tiny $_{lb}$}‚श‚ब्द‚श्च} । स्व‚र्ग‚श‚ब्द उर्व‚र्शी श‚ब्दः । आदिश‚ब्दान्न न्द न व ना‚{\tiny $_{६}$}‚ दि श‚ब्द‚श्चा‚{\tiny $_{lb}$}‚\textbf{रूढार्थ}स्याप्र‚सिद्धार्थ‚स्य \textbf{वाच‚कोऽनेन} वेद‚वादिना \textbf{निर्व‚र्ण्ण्य‚मानो} व्याख्याय‚मानो‚{\tiny $_{lb}$}‚ दृष्टः । त‚था हि प्राकृत\textbf{पुरुषातिशायिनो} ये \textbf{पुरुषं} विशेषास्तेषां \textbf{निकेतः} स्थानं ।‚{\tiny $_{lb}$}‚ मानुषातिक्रान्तं सुख‚म‚तिमानुषं त‚स्या\textbf{तिमानुष}स्य \textbf{सुख}स्या\textbf{धिष्ठान}माश्र‚यः । नाना‚{\tiny $_{lb}$}‚\leavevmode\ledsidenote{\textenglish{211a/PSVTa}} प्र‚काराण्युप‚क‚र‚णान्युप‚भोग‚व‚स्तूनि य‚स्मिन् । स \textbf{नानोप‚क‚र‚णः‚{\tiny $_{७}$}‚} स्व‚र्ग इति लोक‚प्र‚{\tiny $_{lb}$}‚वादः \textbf{त‚न्निवासिनी} स्व‚र्ग‚निवासि\textbf{न्य‚प्स‚रा उर्व‚र्शी नामेति लोक‚प्र‚वादः । तं} लोक‚{\tiny $_{lb}$}‚प्र‚वाद\textbf{म‚ना}दृ\textbf{त्य} म‚नुष्येष्वेव निर‚तिश‚या प्रीतिः \textbf{स्व‚र्गः} । उर्व‚शी चार‚णिः । पात्री‚{\tiny $_{lb}$}‚ वेत्यादिना । लोक‚प्र‚सिद्धाद‚र्था\textbf{द‚न्यामेवांर्थ‚क‚ल्प‚नाम‚यं} जै मि न्यादिः \textbf{कुर्वाणो}ग्निहो‚{\tiny $_{lb}$}‚त्रादि\textbf{श‚ब्दान्त‚रे}ष्व‚र्थ‚निर्ण्ण‚ये । \textbf{क‚थं प्र‚सिद्धिं प्र‚माण‚येत्} । नैव प्र‚माण‚येदि‚{\tiny $_{१}$}‚ति‚{\tiny $_{lb}$}‚ याव‚त् ।
	{\color{gray}{\rmlatinfont\textsuperscript{§~\theparCount}}}
	\pend% ending standard par
      ‚{\tiny $_{lb}$}‚

	  
	  \pstart \leavevmode% starting standard par
	\textbf{त‚त्रा}ग्निहोत्रादिश‚ब्देषु लोक‚प्र‚सिद्धार्थ‚क‚ल्प‚नाया \textbf{अविरोधात्} प्र‚तीत‚स्यैवार्थ‚{\tiny $_{lb}$}‚स्या\textbf{भ्युप‚ग‚म इति चेत्} । स्व‚र्गोर्व‚श्यादिश‚ब्देषु तु प्र‚सिद्धार्थाभ्युप‚ग‚मे प्र‚माण‚विरो‚{\tiny $_{lb}$}‚ धाद‚न‚भ्युप‚ग‚म इति प‚रो म‚न्य‚ते ।
	{\color{gray}{\rmlatinfont\textsuperscript{§~\theparCount}}}
	\pend% ending standard par
      ‚{\tiny $_{lb}$}‚‚{\tiny $_{lb}$}‚\textsuperscript{\textenglish{599/s}}

	  
	  \pstart \leavevmode% starting standard par
	\textbf{ने}त्यादिना सि द्धा न्त वा दी । भेद‚म‚न‚न्त‚रोक्तं युज्य‚ते । य‚स्मा\textbf{द‚त्रापि} स्व‚र्गो‚{\tiny $_{lb}$}‚र्व‚श्यादिश‚ब्देषु लोक‚प्र‚सिद्धे\textbf{तीन्द्रिये}र्थेभ्युप‚ग‚म्य‚माने । प्र‚मा‚{\tiny $_{२}$}‚णेन \textbf{विरोधासिद्धेः}‚{\tiny $_{lb}$}‚ न ह्य‚त्र प्र‚त्य‚क्ष‚म‚नुमानं वा बाध‚कं प्र‚माण‚म‚स्ति । \textbf{अन्य‚त्रापी}त्य‚ग्निहोत्रादिश‚ब्दे‚{\tiny $_{lb}$}‚ष्व‚पि लोक‚प्र‚सिद्धार्थ‚क‚ल्प‚नाया\textbf{म‚विरोध‚स्य दुर‚न्व‚य‚त्वात्} । साध‚क‚प्र‚माणाभावेन‚{\tiny $_{lb}$}‚ दुर्बोध‚त्वात् ।
	{\color{gray}{\rmlatinfont\textsuperscript{§~\theparCount}}}
	\pend% ending standard par
      ‚{\tiny $_{lb}$}‚

	  
	  \pstart \leavevmode% starting standard par
	एत‚देव साध‚य‚न्नाह । \textbf{विरुद्धाम‚पि} य‚थाप्र‚सिद्धाम\textbf{ग्निहोत्रात् स्व}र्गा\textbf{वाप्ति‚{\tiny $_{lb}$}‚ मान्द्याद‚यं} जै मि न्या दि\textbf{र्न ल‚क्ष‚येद‚पि} क‚दाचिदिति संश‚यः । किं च [।]‚{\tiny $_{lb}$}‚ \textbf{विरो‚{\tiny $_{३}$}‚धाविरोधौ च} नान्यावेव [।] किन्त‚र्हि [।] \textbf{बाध‚क‚साध‚क‚प्र‚माण‚वृत्ती} ।‚{\tiny $_{lb}$}‚ य‚थोक्त‚स्यानुप‚ल‚म्भाख्य‚स्य बाध‚क‚स्य प्र‚माण‚स्य वृत्तिर्विरोधः । साध‚क‚स्य प्र‚त्य‚{\tiny $_{lb}$}‚यानुमान‚स्य प्र‚माण‚स्य वृत्तिर‚विरोधः । ते च विरोध‚विरोध‚स्व‚भावे । बाध‚क‚{\tiny $_{lb}$}‚साध‚क‚प्र‚माण‚वृत्ती । \textbf{अत्य‚क्षे}ऽतीन्द्रिये व‚स्तुनि \textbf{नाभिम‚ते} । न हि देशादिविप्र‚कृष्टेषु‚{\tiny $_{lb}$}‚ स्व‚र्गादिसाध‚नेषु बाध‚कं साध‚कं च‚{\tiny $_{४}$}‚ प्र‚माणं प्र‚व‚र्त्त‚ते । य‚दा \textbf{चैव‚न्त‚त्क‚थं} नैव‚{\tiny $_{lb}$}‚ \textbf{त‚द्व‚शा}दिति विरोधाविरोध‚व‚शात् । विरोध‚व‚शात् स्व‚र्गादिश‚ब्देषु प्र‚सिद्धार्थ‚प्र‚ती‚{\tiny $_{lb}$}‚तिर‚विरोध‚व‚शाच्चाग्निहोत्रादिश‚ब्देषु प्र‚सिद्धार्थ\textbf{प्र‚तीतिः} ।
	{\color{gray}{\rmlatinfont\textsuperscript{§~\theparCount}}}
	\pend% ending standard par
      ‚{\tiny $_{lb}$}‚

	  
	  \pstart \leavevmode% starting standard par
	अथ म‚तं [।] न साध‚क‚प्र‚माण‚वृत्तिर‚विरोधोतीन्द्रियेर्थे किं त्वाग‚म‚संज्ञित‚स्या‚{\tiny $_{lb}$}‚ग्निहोत्रादिव‚च‚न‚स्य लोक‚प्र‚सिद्धार्थ‚वाच‚क‚त्वेन प्र‚वृत्तिरेवाविरोध इति [।]
	{\color{gray}{\rmlatinfont\textsuperscript{§~\theparCount}}}
	\pend% ending standard par
      ‚{\tiny $_{lb}$}‚

	  
	  \pstart \leavevmode% starting standard par
	अत आह ।‚{\tiny $_{५}$}‚ \textbf{न चे}त्यादि । लोक‚प्र‚सिद्धार्थ‚वाच‚क‚त्वेनाग्निहोत्रादि\textbf{व‚च‚न‚स्य‚{\tiny $_{lb}$}‚ प्र‚वृत्तिरेवाविरोधो न च} । किं कार‚ण‚म् [।] \textbf{अन्य‚त्रापि} स्व‚र्गोर्व‚श्यादिश‚ब्देष्व‚पि‚{\tiny $_{lb}$}‚ स्व‚र्गादिवाच‚क‚त्वेन प्र‚वृत्तेर‚विरोध\textbf{प्र‚स‚ङ्गात्} । त‚था हि [।] स्व‚र्गोर्व‚श्यादिश‚ब्दाः‚{\tiny $_{lb}$}‚स्थानाप्स‚रोविशेषादिष्वेव प्र‚वृत्ता लोके दृश्य‚न्ते । त‚त‚श्चाविशेषादुभ‚य‚त्र प्र‚सिद्धा‚{\tiny $_{lb}$}‚र्थ‚प‚रिग्र‚होस्तु । न चैक‚त्रापि ।
	{\color{gray}{\rmlatinfont\textsuperscript{§~\theparCount}}}
	\pend% ending standard par
      ‚{\tiny $_{lb}$}‚

	  
	  \pstart \leavevmode% starting standard par
	किं च [।] \textbf{अपौ‚{\tiny $_{६}$}‚रुषेयो} वेदाख्य \textbf{आग‚मः} स च स्व‚म‚र्थं स्व‚यं न प्र‚काश‚य‚ति‚{\tiny $_{lb}$}‚ किन्तु \textbf{त‚स्या}ग‚म‚स्य लोक\textbf{प्र‚वादाद‚र्थ‚सिद्धि}र‚भ्युप‚ग‚म्य‚ते । \textbf{त‚त्र पुन}र्लोक‚प्र‚सिद्ध्य‚ङ‚{\tiny $_{lb}$}‚गीक‚र‚णेपि \textbf{विरोध‚चिन्तायां} क्रिय‚माणायां स‚र्व\textbf{त्राग‚मेऽनाश्वासः} स्याद‚तीन्द्रिये‚{\tiny $_{lb}$}‚ विरोधाविरोध‚योर्निश्चेतुम‚श‚क्य‚त्वात् ।
	{\color{gray}{\rmlatinfont\textsuperscript{§~\theparCount}}}
	\pend% ending standard par
      ‚{\tiny $_{lb}$}‚

	  
	  \pstart \leavevmode% starting standard par
	अनाश्वास‚मेव साध‚य‚न्नाह । \textbf{स‚त्य‚पी}त्यादि । \textbf{स‚त्य‚पि त‚स्मिन्न}पौरु‚{\tiny $_{७}$}‚षेय \leavevmode\ledsidenote{\textenglish{211b/PSVTa}}‚{\tiny $_{lb}$}‚ ‚{\tiny $_{lb}$}‚ \leavevmode\ledsidenote{\textenglish{600/s}}आग‚मे । स्व‚र्गादिश‚ब्द‚वाच्य‚स्य स्व‚र्गाद्य‚र्थ‚स्यात‚थाभावात् । \textbf{त‚थाभाव‚स्त}थात्वं‚{\tiny $_{lb}$}‚ [।] य‚था लोकेप्र‚सिद्ध‚स्व‚र्गार्थ‚ग्र‚ह‚ण\textbf{म‚त‚थाभावाद}प्र‚सिद्धार्थ‚ग्र‚ह‚णाद‚न्य‚स्याप्य‚ग्निहो‚{\tiny $_{lb}$}‚त्रादिश‚ब्द‚स्य \textbf{शंक‚नीय‚त्वात्} । किम‚स्य लोक‚प्र‚सिद्ध एवार्थः किम्वा स्व‚र्गादिश‚ब्द‚{\tiny $_{lb}$}‚व‚द‚न्य एवेति । एत‚देव कुतः [।] \textbf{प्र‚माण‚वृत्तेः} । न ह्य‚त्र लोक‚प्र‚सिद्धार्थ‚ग्र‚ह‚णे‚{\tiny $_{lb}$}‚ प्र‚माणं प्र‚व‚र्त्त‚त इति ।
	{\color{gray}{\rmlatinfont\textsuperscript{§~\theparCount}}}
	\pend% ending standard par
      ‚{\tiny $_{lb}$}‚

	  
	  \pstart \leavevmode% starting standard par
	\textbf{य‚दुक्त}मित्यादि प‚रः । य‚दुक्त‚म् [।] \textbf{अग्निहोत्रं जुहुयादित्य‚त्र} वाक्ये‚{\tiny $_{lb}$}‚ \textbf{श्व‚मांस‚भ‚क्ष‚ण}स्य \textbf{चोद‚ना}भिधान‚न्त‚स्य \textbf{विक‚ल्पः} क‚ल्प‚ना \textbf{भ‚व‚त्विति । स} दोषो \textbf{न‚{\tiny $_{lb}$}‚ भ‚व‚ति} । किं कार‚ण‚म् [।] वेद‚स्यैव \textbf{प्र‚देशान्त‚रे त‚थे}ति भूत‚विशेषे घृतादिकं प्र‚क्षि‚{\tiny $_{lb}$}‚पेदित्येव‚म‚स्याग्निहोत्रादिवाक्यास्या\textbf{र्थ‚च‚र्च्च‚नात्} । व्याख्यानात् ।
	{\color{gray}{\rmlatinfont\textsuperscript{§~\theparCount}}}
	\pend% ending standard par
      ‚{\tiny $_{lb}$}‚

	  
	  \pstart \leavevmode% starting standard par
	\textbf{ने}त्यादि सि द्धा न्त वा दी । \textbf{ने}द‚मुत्त‚रं युज्य‚ते । \textbf{त‚स्य} प्र‚{\tiny $_{२}$}‚देशान्त‚र‚स्थ‚स्य‚{\tiny $_{lb}$}‚ व्याख्याभूत‚स्य वाक्य‚स्या\textbf{र्थाप‚रिज्ञानात्} । त‚त‚श्च \textbf{प्र‚देशान्तेरेष्व‚पि} व्याख्याभूतेषु‚{\tiny $_{lb}$}‚ \textbf{त‚थार्थ‚क‚ल्प‚नाया} इति । श्व‚मांस‚भ‚क्ष‚ण‚क‚ल्प‚नाया \textbf{अनिवार्य‚त्वात्} ।
	{\color{gray}{\rmlatinfont\textsuperscript{§~\theparCount}}}
	\pend% ending standard par
      ‚{\tiny $_{lb}$}‚

	  
	  \pstart \leavevmode% starting standard par
	\textbf{य‚दीत्याद्य}स्यैव स‚म‚र्थ‚नं । \textbf{य‚दि ह्य‚य‚म‚पौरुषेयो} वेदाख्यः \textbf{श‚ब्द‚राशिः क्व‚चित्}‚{\tiny $_{lb}$}‚ प्र‚देशान्त‚रे \textbf{विदितार्थः स्यात् त‚दा त‚तो} विदितार्थात् प्र‚देशान्त‚रात् प‚रिशिष्ट\textbf{स्यार्थ‚{\tiny $_{lb}$}‚प्र‚तीतिः स्यात्} । या‚{\tiny $_{३}$}‚व‚ता \textbf{ते तु} वैदिकाः श‚ब्दा \textbf{बाहुल्येप्य‚प}रिज्ञानार्थ‚त्वाद‚न्धा‚{\tiny $_{lb}$}‚ \textbf{एव स‚र्व इति} कृत्वा पुरुषेण य\textbf{थेष्टं प्र‚णीय‚न्ते} व्याख्याय‚न्ते ।
	{\color{gray}{\rmlatinfont\textsuperscript{§~\theparCount}}}
	\pend% ending standard par
      ‚{\tiny $_{lb}$}‚

	  
	  \pstart \leavevmode% starting standard par
	य‚त एव\textbf{न्त‚स्माच्छ‚ब्दान्त‚रेषु तादृक्ष्वि}ति । अग्निहोत्रं जुहुयादित्यादि वाक्या‚{\tiny $_{lb}$}‚नाम्व्याख्याभूतेषु \textbf{तादृश्येवास्तु क‚ल्प‚ना} । कीदृशी [।] इत्याह । \textbf{यादृश्य‚ग्नि‚{\tiny $_{lb}$}‚होत्र‚ञ्जुहुयात्स्व‚र्ग‚काम इत्य‚स्य} क‚ल्प‚ना कृता प्र‚देशान्त‚रेपि श्व‚मांस‚भ‚क्ष‚ण‚क‚ल्प‚{\tiny $_{४}$}‚‚{\tiny $_{lb}$}‚नास्त्विति याव‚त् ।
	{\color{gray}{\rmlatinfont\textsuperscript{§~\theparCount}}}
	\pend% ending standard par
      ‚{\tiny $_{lb}$}‚

	  
	  \pstart \leavevmode% starting standard par
	\textbf{अपि च प्र‚सिद्धिश्च} नान्य‚देव किंचित् । किन्त‚र्हि [।] \textbf{नृणां} पुरुषाणां \textbf{वादः‚{\tiny $_{lb}$}‚ ‚{\tiny $_{lb}$}‚ \leavevmode\ledsidenote{\textenglish{601/s}}स च} वादः \textbf{प्र‚माण‚न्नेष्य‚ते} भ‚व‚ता । स‚र्व‚पुरुषाणां रागाद्युप‚प्लुत‚त्वात् । \textbf{भूय}‚{\tiny $_{lb}$}‚ इति पुनः । \textbf{त‚त‚श्च} लोक‚प्र‚वादाद‚प्र‚माण‚त्वेन पूर्वं व्य‚व‚स्थापितात् पुन‚र्वेदा\textbf{र्थ‚ग‚ति}रिति‚{\tiny $_{lb}$}‚ युग‚प\textbf{त्किमेत‚द्विष्ट‚कामितं} । य‚देव व‚स्त्व‚प्र‚माण‚त्वेन द्विष्टं । त‚देव पुनः‚{\tiny $_{५}$}‚ प्र‚तिप‚त्ति‚{\tiny $_{lb}$}‚हेतुत्वेन कामित‚म‚भिल‚षित‚मिति प‚र‚स्प‚र‚विरोधः ।
	{\color{gray}{\rmlatinfont\textsuperscript{§~\theparCount}}}
	\pend% ending standard par
      ‚{\tiny $_{lb}$}‚

	  
	  \pstart \leavevmode% starting standard par
	\textbf{न प्र‚सिद्धिर्नामान्या ज‚न‚प्र‚वादात्} । किन्तु ज‚न‚प्र‚वाद एव प्र‚सिद्धिः । \textbf{ते च‚{\tiny $_{lb}$}‚ स‚र्वे ज‚ना अस‚म्भाव‚नीय‚याथात‚थ्य‚व‚च‚ना} अस‚म्भाव‚नीयं याथात‚थ्य‚म‚विप‚रीत‚त्वं‚{\tiny $_{lb}$}‚ य‚स्मिन् व‚च‚ने । त‚द‚स‚म्भाव‚नीय\textbf{याथात‚थ्यं व‚च}नं येषामिति विग्र‚हः । किं कार‚णं‚{\tiny $_{lb}$}‚ [।] \textbf{रागाविद्याप‚री‚{\tiny $_{६}$}‚त‚त्वात्} । रागाविद्याभ्यां व्याप्त‚त्वात् । \textbf{त}दिति त‚स्मा\textbf{देषां}‚{\tiny $_{lb}$}‚ ज‚नानां \textbf{प्र‚वादो न प्र‚माणं} ।
	{\color{gray}{\rmlatinfont\textsuperscript{§~\theparCount}}}
	\pend% ending standard par
      ‚{\tiny $_{lb}$}‚

	  
	  \pstart \leavevmode% starting standard par
	बाहुल्याज्ज‚न‚स्य त‚त्प्र‚वादः प्र‚माण‚मित्य‚पि मिथ्या । य‚तः [।] \textbf{न हि} ब‚हूनां‚{\tiny $_{lb}$}‚ ज‚नानाम्म‚ध्ये \textbf{क‚स्य‚चिदे}क‚स्य्\textbf{आपि} पुरुष‚स्य \textbf{स‚म्य‚क्प्र‚तिप‚त्तेर‚भावे} स‚ति लोक‚स्य‚{\tiny $_{lb}$}‚ \textbf{बाहुल्य‚म‚र्थ‚व‚द्} भ‚व‚ति प्र‚योज‚न‚व‚द् भ‚व‚ति । किमिव [।] \textbf{पा र सी के}त्यादि । य‚था‚{\tiny $_{lb}$}‚ ब‚हुभिः पार‚सीकै\textbf{र्मा‚{\tiny $_{७}$}‚त‚रि} मैथुनाच‚र‚णान्न त‚न्न्याय्य‚म्भ‚व‚ति । एवं बाहुल्येपि रागादि- \leavevmode\ledsidenote{\textenglish{212a/PSVTa}}‚{\tiny $_{lb}$}‚ म‚ताम‚तीन्द्रियेर्थे व‚च‚न‚म‚प्र‚माण‚मेवेति ।
	{\color{gray}{\rmlatinfont\textsuperscript{§~\theparCount}}}
	\pend% ending standard par
      ‚{\tiny $_{lb}$}‚

	  
	  \pstart \leavevmode% starting standard par
	अथ \textbf{तेषामेव} रागादिम‚तां \textbf{पुरुषाणां व‚च‚नात् पुनः प‚रोक्ष}स्याग्निहोत्रादिश‚ब्द‚{\tiny $_{lb}$}‚वाच्य‚स्यार्थ‚स्य प्र‚तिप‚त्तिरिति । \textbf{क‚थ‚न्त‚देव युग‚प‚दे}क‚स्मिन्नेव काले \textbf{द्वेष्यं च काम्यं‚{\tiny $_{lb}$}‚ च} युज्य‚ते ।
	{\color{gray}{\rmlatinfont\textsuperscript{§~\theparCount}}}
	\pend% ending standard par
      ‚{\tiny $_{lb}$}‚

	  
	  \pstart \leavevmode% starting standard par
	अग्निहोत्रादिश‚ब्द‚स्य य‚स्मिन्न‚र्थे लोक\textbf{प्र‚सिद्धि}स्ता\textbf{मुल्लंध्य‚{\tiny $_{१}$}‚} त्य‚क्त्वा । त‚तो‚{\tiny $_{lb}$}‚र्थान्त‚र‚स्य श्व‚मांस‚भ‚क्ष‚णादेः \textbf{क‚ल्प‚ने न निब‚न्ध‚नं} कार‚ण‚मिति । त‚स्मात् प्र‚सि‚{\tiny $_{lb}$}‚द्धिरेव गृह्य‚त इत्येत‚द‚प्य‚युक्तं । य‚तः \textbf{प्र‚सिद्धेर‚प्र‚माण‚त्वात् त‚द्ग्र‚हे} प्र‚सिद्धिग्र‚हे‚{\tiny $_{lb}$}‚ \textbf{किन्निब‚न्ध‚नं} [।] नैव किञ्चिदिति प्र‚सिद्धेर‚पि ग्र‚हो माभूदिति ।
	{\color{gray}{\rmlatinfont\textsuperscript{§~\theparCount}}}
	\pend% ending standard par
      ‚{\tiny $_{lb}$}‚‚{\tiny $_{lb}$}‚\textsuperscript{\textenglish{602/s}}

	  
	  \pstart \leavevmode% starting standard par
	\textbf{प्राप्त्ये}त्यादिना पूर्वार्द्ध‚न्ताव‚द् व्याच‚ष्टे । प्र‚सिद्ध्या \textbf{प्राप्त}स्यार्थ‚स्य \textbf{प्र‚तिलोम‚नेन}‚{\tiny $_{lb}$}‚ त्यागेनान्य‚त्राप्र‚सिद्धेर्थे‚{\tiny $_{२}$}‚ \textbf{प्र‚वृत्तिर्गुण‚दोष‚द‚र्श‚ने} स‚ति \textbf{युक्ता} । य‚दि प्राप्तेर्थे दोष‚द‚र्श‚नं‚{\tiny $_{lb}$}‚ स्याद‚प्राप्ते च गुण‚द‚र्श‚नं । न चाग्निहोत्रादिश‚ब्दानां प्र‚सिद्धेर्थे दोष‚द‚र्श‚न‚म‚स्त्य‚{\tiny $_{lb}$}‚प्र‚सिद्धे वा गुण‚द‚र्श‚न‚न्त‚तः \textbf{प्र‚सिद्धे}रेवा\textbf{न्व‚योनु}ग‚म‚न\textbf{मिति चेत्} ।
	{\color{gray}{\rmlatinfont\textsuperscript{§~\theparCount}}}
	\pend% ending standard par
      ‚{\tiny $_{lb}$}‚

	  
	  \pstart \leavevmode% starting standard par
	\textbf{ने}त्यादिना प्र‚तिषेध‚ति । एत‚च्च प‚श्चार्द्ध‚स्य विव‚र‚णं । \textbf{नैत}देवं [।] किंङ्का‚{\tiny $_{lb}$}‚र‚णं [।] \textbf{प्राप्तेः प्र‚माण‚वृत्तिल‚क्ष‚ण‚त्वात्}‚{\tiny $_{३}$}‚ । साध‚केन हि प्र‚माणेनार्थ‚स्य प्राप्ति‚{\tiny $_{lb}$}‚र्निश्चीय‚त इति प्राप्तेः प्र‚माण‚वृत्तिल‚क्ष‚ण‚त्व‚मुच्य‚ते । न च प्र‚सिद्धिः प्र‚माणं येन‚{\tiny $_{lb}$}‚ सिद्धोर्थो न्याय‚प्राप्तः स्यात् । त‚त‚श्च तामेनां \textbf{प्र‚सिद्धिम‚प्र‚माण‚य‚तो} मी मां स क स्य‚{\tiny $_{lb}$}‚ \textbf{त‚न्मुखेन} प्र‚सिद्धिद्वारेणाग्निहोत्रादिश‚ब्द‚वाच्य‚स्यार्थ‚स्य या \textbf{प्र‚तीतिः} सा य‚त्किञ्च‚नः‚{\tiny $_{lb}$}‚ ग्र‚ह‚ण‚म‚प्र‚माण‚क‚त्वात् । य‚था‚{\tiny $_{४}$}‚क‚थ‚ञ्चित् त‚द्ग्र‚ह‚ण‚मित्य‚र्थः ।
	{\color{gray}{\rmlatinfont\textsuperscript{§~\theparCount}}}
	\pend% ending standard par
      ‚{\tiny $_{lb}$}‚

	  
	  \pstart \leavevmode% starting standard par
	त‚त‚श्चाग्निहोत्रादिश‚ब्द‚स्य लोक‚प्र‚सिद्धोर्थो न्याय‚प्राप्तो न भ‚व‚तीत्येवं \textbf{न्याया}‚{\tiny $_{lb}$}‚ल्लोक‚प्र‚सिद्ध‚स्यार्थ‚स्य \textbf{प्राप्तिप्र‚तिषेधात्} कार‚णाद‚ग्निहोत्रादिश‚ब्दानामिच्छ‚यार्थः‚{\tiny $_{lb}$}‚ प‚रिक‚ल्प‚नीयः ।
	{\color{gray}{\rmlatinfont\textsuperscript{§~\theparCount}}}
	\pend% ending standard par
      ‚{\tiny $_{lb}$}‚

	  
	  \pstart \leavevmode% starting standard par
	त‚त्र मी मां स कैर‚ग्निहोत्रादिश‚ब्दानां योर्थः प‚रिक‚ल्प्य‚ते द‚ह‚न‚द्र‚व्यादिल‚क्ष‚णो‚{\tiny $_{lb}$}‚ य‚श्च प‚रेण श्व‚मांस‚भ‚क्ष‚ण‚ल‚क्ष‚ण‚स्त‚{\tiny $_{५}$}‚योः \textbf{स्व‚प‚र‚विक‚ल्प‚योरुभ‚य‚था}पीति य‚दि प्र‚सि‚{\tiny $_{lb}$}‚द्ध्य‚नुपाल‚न‚म‚थ नानुपाल‚न‚न्त‚थापि प्र‚माणाभावादिच्छ‚या प‚रिग्र‚हे \textbf{तुल्या वृत्ति}‚{\tiny $_{lb}$}‚रिति कृत्वा \textbf{कः प्र‚सिद्धानुरोधो} येन प्र‚सिद्धेर‚र्थ‚क‚ल्प‚ना क्रिय‚ते ।
	{\color{gray}{\rmlatinfont\textsuperscript{§~\theparCount}}}
	\pend% ending standard par
      ‚{\tiny $_{lb}$}‚

	  
	  \pstart \leavevmode% starting standard par
	\textbf{अपि च [।] प्र‚सिद्ध्यै}व स‚र्व\textbf{श‚ब्दा}ना\textbf{म‚र्थ‚निश्च‚ये} । इय‚न्नानार्थ‚त्वेन \textbf{शंकोत्पा‚{\tiny $_{lb}$}‚दिता} । किं कार‚णं [।] \textbf{य‚स्मान्ना\textbf{ना}र्थ}वृत्तित्व‚म‚ग्न्यादि\textbf{श‚ब्दानान्त‚त्र} प्र‚सिद्धौ‚{\tiny $_{lb}$}‚ \textbf{दृश्य‚ते} ।
	{\color{gray}{\rmlatinfont\textsuperscript{§~\theparCount}}}
	\pend% ending standard par
      ‚{\tiny $_{lb}$}‚

	  
	  \pstart \leavevmode% starting standard par
	\textbf{ने}त्यादिना व्याच‚ष्टे । \textbf{न प्र‚सिद्धे}स्स‚काशा\textbf{देकार्थ‚निश्च‚यः श‚ब्दा}नाम्वैदिकानां ।‚{\tiny $_{lb}$}‚ किं कार‚णं [।] \textbf{त‚त एव} प्र‚सिद्धेरेव \textbf{शंकोत्प‚त्तेः} । त‚था हि \textbf{नानार्था} अग्न्यादि‚{\tiny $_{lb}$}‚\textbf{श‚ब्दा लोके दृश्य‚न्ते} । त‚च्च नानार्थ‚द‚र्श‚नं लोक‚वादो \textbf{लोक‚प्र‚वाद‚श्च प्र‚तीतिर‚त एव}‚{\tiny $_{lb}$}‚‚{\tiny $_{lb}$}‚ \leavevmode\ledsidenote{\textenglish{603/s}}लोक‚वादा\textbf{न्ना\textbf{ना}र्थ‚तेति} कृत्वा \textbf{त‚तो} लोक‚वादाद् वैदिकानां श‚ब्दानामेकार्थ‚{\tiny $_{७}$}‚\textbf{निय‚मो \leavevmode\ledsidenote{\textenglish{212b/PSVTa}}‚{\tiny $_{lb}$}‚ न युक्तः} ।
	{\color{gray}{\rmlatinfont\textsuperscript{§~\theparCount}}}
	\pend% ending standard par
      ‚{\tiny $_{lb}$}‚

	  
	  \pstart \leavevmode% starting standard par
	अनिय‚तार्थ‚त्व‚मेव द्र‚ढ‚य‚न्नाह । \textbf{अन्य‚थे}त्यादि । \textbf{स्व‚यं} स्व‚भाव‚तो \textbf{नानाश‚क्}तेर‚ने‚{\tiny $_{lb}$}‚कार्थ‚प्र‚तिपाद‚न‚योग्य‚स्य ध्व‚नेः श‚ब्द‚स्या\textbf{न्य‚थास‚म्भ‚वाभावात्} । अभिम‚ताद‚र्थाद‚र्थान्त‚रे‚{\tiny $_{lb}$}‚ वृत्तिर‚न्य‚था । त‚स्या अर्थान्त‚र‚वृत्तेर‚स‚म्भ‚वोन्य‚थाऽस‚म्भ‚व‚स्त‚स्याभावात् । प्र‚तिषेध‚{\tiny $_{lb}$}‚द्व‚येन विधेर‚भ्युप‚ग‚माद‚न्य‚थापि स‚म्भ‚व‚स्य भावादिति याव‚त् । त‚त‚श्चातीन्द्रि‚{\tiny $_{१}$}‚यार्थ‚{\tiny $_{lb}$}‚वृत्तिषु वैदिकेषु श‚ब्देष्\textbf{व‚व‚श्य‚म}नेकार्थ‚वृत्तित्वेन \textbf{शंक‚या भाव्यं} । केषां \textbf{नियाम‚क‚म‚{\tiny $_{lb}$}‚प‚श्य‚तां} पुंसां प्र‚तिनिय‚त‚विष‚य‚साध‚कं प्र‚माण‚म‚प‚श्य‚तां । \href{http://sarit.indology.info/?cref=pv.3.323}{३२६}
	{\color{gray}{\rmlatinfont\textsuperscript{§~\theparCount}}}
	\pend% ending standard par
      ‚{\tiny $_{lb}$}‚

	  
	  \pstart \leavevmode% starting standard par
	य‚त एव\textbf{न्त‚स्माद् [।] अविदितः अर्थ‚विभागो} येषामिति विग्र‚हः । \textbf{तेष्व}वि‚{\tiny $_{lb}$}‚दितार्थ‚विभागेषु \textbf{श‚ब्देष्वेक‚म‚र्थ}म‚भिम‚तं ।ऽ किं भूत‚म् [।] \textbf{अत्य‚क्ष‚संयोगं} ।‚{\tiny $_{lb}$}‚ अत्य‚क्षोतीन्द्रिय‚स्संयोगः श‚ब्देन स‚ह स‚म्ब‚न्धो य‚स्येति वि‚{\tiny $_{२}$}‚ग्र‚हः । आल‚म्ब‚नं‚{\tiny $_{lb}$}‚ प्र‚माण‚न्त‚द‚भा\textbf{वाद‚नाल‚म्ब‚नो} निःप्र‚माण‚क\textbf{स्स‚मारोपो} य‚स्य स त‚थोक्तः । \textbf{त‚मेवं}‚{\tiny $_{lb}$}‚ भूत‚म‚र्थ\textbf{म्विनिश्च‚त्य व्याच‚क्षाणो जै मि निः । त‚द्व्याजेनेति} वेद एवं प्राहेति‚{\tiny $_{lb}$}‚ वेदोप‚क्षेपेण \textbf{स्व‚मेव म‚त‚माहेति} कृत्वा य‚स्तीर्थ्य\edtext{}{\lemma{स्तीर्थ्य}\Bfootnote{? र्थ}}क‚रो निर्व्याज‚मेव‚माह ।‚{\tiny $_{lb}$}‚ अह‚मेव स्व‚य‚म्व‚दामीति । त‚स्मात् \textbf{तीर्थ्य}\edtext{}{\lemma{स्मात्}\Bfootnote{? र्थ}}\textbf{क‚रान्त‚राद‚स्य} जै मि ने\textbf{र्न विशेषं‚{\tiny $_{lb}$}‚ प‚श्यामः} ।
	{\color{gray}{\rmlatinfont\textsuperscript{§~\theparCount}}}
	\pend% ending standard par
      ‚{\tiny $_{lb}$}‚

	  
	  \pstart \leavevmode% starting standard par
	अविशेष‚मेव‚{\tiny $_{३}$}‚ साध‚य‚न्नाह । \textbf{त‚था ही}त्यादि । स चासाव‚र्थ‚श्चेति \textbf{त‚द‚र्थ}स्त‚स्य‚{\tiny $_{lb}$}‚ प्र‚काश‚न\textbf{म्व‚च‚नं} । त‚त्र \textbf{व्यापार}स्साम‚र्थ्य‚न्तेन \textbf{शून्य‚स्य} र‚हित‚स्य वेद‚स्य । \textbf{त‚त्स‚मारो‚{\tiny $_{lb}$}‚पेणे}त्य‚र्थ‚प्र‚काश‚न‚व्यापार‚स‚मारोपेण{... ... ...}एव‚म्व‚क्तीति य‚द\textbf{भिधान}न्त‚{\tiny $_{lb}$}‚ {... ... ...} स्व‚व‚च‚न‚मेव त‚स्य त‚दिति याव‚त् । \textbf{त‚त्कारिणा} व‚च‚न‚व्यापार‚शून्ये‚{\tiny $_{lb}$}‚ वेदेर्थ‚प्र‚काश‚न‚व्यापार‚स‚मारोप‚कारिणा जै मि नि ना \textbf{केव‚{\tiny $_{४}$}‚ल‚म्मिथ्याविनीत‚तैवा‚{\tiny $_{lb}$}‚त्म‚नः स‚मुद्द्योतिता स्यात्} । न तु पौरुषेयाद् व‚च‚नाद‚स्य विशेषः ।
	{\color{gray}{\rmlatinfont\textsuperscript{§~\theparCount}}}
	\pend% ending standard par
      ‚{\tiny $_{lb}$}‚‚{\tiny $_{lb}$}‚\textsuperscript{\textenglish{604/s}}

	  
	  \pstart \leavevmode% starting standard par
	एत‚देव स्फुट‚य‚न्नाह । \textbf{त‚था ही}त्यादि । कः प‚न्था पा ट लि पु त्रं ग‚च्छ‚तीति‚{\tiny $_{lb}$}‚ पृष्टः \textbf{क‚श्च‚न} पुरुष आह । न जाने स्व‚य‚म‚हं केव‚ल‚मे\textbf{ष स्थाणुर‚यं मार्ग इति व‚क्ती}‚{\tiny $_{lb}$}‚त्येव‚मेको मार्गोप‚देश‚साम‚र्थ्य‚शून्य‚स्थाणुव्याजेन मार्ग‚माच‚ष्टे । \textbf{अन्य}स्त्वाह । न‚{\tiny $_{lb}$}‚ स्थाणोर्व‚च‚न‚साम‚र्थ्य‚{\tiny $_{५}$}‚म‚स्त्य‚ह‚मेव \textbf{स्व‚यं} ज्ञात्वायं मार्ग इति \textbf{ब्र‚वीमीति । त‚योरे}व‚म‚{\tiny $_{lb}$}‚भिद‚ध‚तोः स्व‚य‚म्व‚च‚न\textbf{भेदः प‚रीक्ष्य‚तां} य‚द्य‚स्ति नैवास्तीत्य‚भिप्रायः ।
	{\color{gray}{\rmlatinfont\textsuperscript{§~\theparCount}}}
	\pend% ending standard par
      ‚{\tiny $_{lb}$}‚

	  
	  \pstart \leavevmode% starting standard par
	\textbf{निर‚भिप्राय} इत्यादिना व्याच‚ष्टे । अभिप्राय इदं चेदं च क‚रिष्यामीति चेत‚ना ।‚{\tiny $_{lb}$}‚ त‚त्पूर्व‚कः प्र‚य‚त्नो व्यापारः । \textbf{अभिप्राय}श्च \textbf{व्यापार}श्च \textbf{व‚च‚नं} च । तानि \textbf{न} विद्य‚न्ते‚{\tiny $_{lb}$}‚ य‚स्मिन्स त‚थोक्तः । \textbf{त‚स्मि}न्नेव‚म्भूते \textbf{स्थाणौ} मार्ग‚{\tiny $_{६}$}‚प्र‚काश‚क‚त्वं \textbf{स‚मारोप्योप‚दिश‚त}‚{\tiny $_{lb}$}‚ एक‚स्य पुंसः \textbf{स्व‚त‚न्त्र‚स्य वा} स्थाणुनिर‚पेक्ष‚स्याप‚र‚स्य मार्ग‚मुप‚दिश‚त इति स‚म्ब‚{\tiny $_{lb}$}‚\textbf{न्धः} । एत‚योर्द्व‚योः पुरुष‚योः स्व‚य‚म्व\textbf{च‚नोप‚ग‚मे न क‚श्चिद्विशेषोन्य‚त्र ज‚ड‚स्य‚{\tiny $_{lb}$}‚ प्र‚तिप‚त्तिमान्द्यात्} ज‚ड‚स्य श्रोतुः प्र‚तिप‚त्तिमान्द्य‚मेव विशिष्य‚ते । य‚तोभिप्रायादि‚{\tiny $_{lb}$}‚\leavevmode\ledsidenote{\textenglish{213a/PSVTa}} शून्य‚स्य स्थाणोर्व‚च‚नं प्र‚तिप‚द्य‚ते स्थाणुरेव व‚क्तीति । एवं‚{\tiny $_{७}$}‚ जै मि नेर्वेद‚व्याजेन‚{\tiny $_{lb}$}‚ स्व‚म‚तं ब्रुव‚तो ज‚डः प्र‚तिप‚त्तिमान्द्यात् वेद एवं ब्रूत इति प्र‚तिप‚द्य‚ते ।
	{\color{gray}{\rmlatinfont\textsuperscript{§~\theparCount}}}
	\pend% ending standard par
      ‚{\tiny $_{lb}$}‚

	  
	  \pstart \leavevmode% starting standard par
	\textbf{अपि च} वैदिक‚स्य श‚ब्द‚स्यै\textbf{कार्थ}प्र‚ति\textbf{निय‚मे स‚त्येन}मेकार्थ‚प्र‚तिनिय‚मं \textbf{जै मि नि‚{\tiny $_{lb}$}‚र्जानीया}त् । याव‚ता \textbf{श‚ब्द‚स्य वाच‚क}स्य \textbf{स‚र्व‚त्रा}र्थे वाच‚क‚त्वेन \textbf{योग्य‚स्यैकार्थ‚द्योत‚ने‚{\tiny $_{lb}$}‚ निय‚तिः कुतो} नैव । \textbf{न हि श‚ब्द‚स्य क‚श्चिद‚र्थः स्व‚भावेन} निस‚र्ग‚सिद्ध्या \textbf{निय‚तो}स्ति ।‚{\tiny $_{lb}$}‚ किं कार‚णं [।] \textbf{स‚{\tiny $_{१}$}‚र्व‚त्र} वाच्येर्थे \textbf{योग्य‚त्वा}च्छ‚ब्द‚स्य ।
	{\color{gray}{\rmlatinfont\textsuperscript{§~\theparCount}}}
	\pend% ending standard par
      ‚{\tiny $_{lb}$}‚

	  
	  \pstart \leavevmode% starting standard par
	अथ पुन‚र्न योग्य‚ता स‚र्व‚त्रार्थे श‚ब्द‚स्य त‚दा\textbf{प्य‚योग्य‚त्वे च त‚द‚प्र‚च्युते}र‚योग्य‚ता‚{\tiny $_{lb}$}‚स्व‚भावान्नित्य‚स्य श‚ब्द‚स्याप्र‚च्युतेः कार‚णात् \textbf{पुरुषाणाम‚विधेय‚स्}यानाय‚त्त‚स्य‚{\tiny $_{lb}$}‚ \textbf{क्व‚चिद‚र्थे उप‚न‚य‚न}म्वाच‚क‚त्वेन नियोज‚नं । नियुक्त‚स्याप्य\textbf{प‚न‚य‚नं} । नेदानीम‚यं‚{\tiny $_{lb}$}‚ श‚ब्दो वाच‚क इति त‚स्\textbf{यास‚म्भ‚व्}आत् । भ‚व‚ति च [।] त‚स्मात् स‚र्व‚त्र श‚ब्दा यो‚{\tiny $_{२}$}‚ग्या‚{\tiny $_{lb}$}‚ ‚{\tiny $_{lb}$}‚ \leavevmode\ledsidenote{\textenglish{605/s}}इत्येकार्थ‚द्योत‚नं नास्ति ।
	{\color{gray}{\rmlatinfont\textsuperscript{§~\theparCount}}}
	\pend% ending standard par
      ‚{\tiny $_{lb}$}‚

	  
	  \pstart \leavevmode% starting standard par
	भ‚व‚तु वा वैदिकानामेकार्थ‚निय‚म‚स्त‚थाप्य‚तीन्द्रिय‚म‚र्थ‚निय‚मं पुरुषो ज्ञातुम‚{\tiny $_{lb}$}‚श‚क्तः । त‚देवाह । \textbf{ज्ञाता वातीन्द्रिया अर्थाः केन} पुरुषेण । न हि मी मां स को‚{\tiny $_{lb}$}‚तीन्द्रियार्थ‚द‚र्शिनं क‚ञ्चिदिच्छ‚ति । \textbf{विव‚क्षाव‚च‚नाद् ऋते} [।] विव‚क्षायाः प्र‚काश‚{\tiny $_{lb}$}‚न‚म्व‚च‚न‚म्विव‚क्षाव‚च‚न‚न्तेन विना । व‚क्तुर‚भिप्राय‚क‚थ‚न‚म‚न्त‚रेणातीन्द्रिया नैव केन‚{\tiny $_{lb}$}‚चि‚{\tiny $_{३}$}‚ज्ज्ञाता इत्य‚र्थः ।
	{\color{gray}{\rmlatinfont\textsuperscript{§~\theparCount}}}
	\pend% ending standard par
      ‚{\tiny $_{lb}$}‚

	  
	  \pstart \leavevmode% starting standard par
	\textbf{पुरुषे}त्यादिना व्याच‚ष्टे । क\textbf{याचिद् विव‚क्ष‚या पुरुषे}ण \textbf{प्र‚णीते} उच्चारिते \textbf{श‚ब्दे}‚{\tiny $_{lb}$}‚ स पुरुष‚स्ताम्विव‚क्षां \textbf{क‚दाचित्} क्व‚चिच्छ्रोत‚रि \textbf{निवेद‚येदि}दं म‚या वाच्य‚त्वेन विव‚क्षि‚{\tiny $_{lb}$}‚त\textbf{मिति । विव‚क्षापूर्व‚काणां श‚ब्दानाम‚र्थे निय‚मः प्र‚तीयेतापि । अपौरुषेये तु} श‚ब्दे‚{\tiny $_{lb}$}‚ विव‚क्षापूर्वं केन‚चिद‚प्र‚युक्ते \textbf{विद्य‚मानोप्य‚र्थ‚निय‚मः क‚थं ज्ञेयः} [।] नैव क‚थंचित् ।‚{\tiny $_{lb}$}‚ किं‚{\tiny $_{४}$}‚ कार‚णं [।] स्\textbf{व‚भाव‚भेद‚स्याभावा}त् । न हि वैदिक‚स्य श‚ब्द‚स्य क‚श्चित् स्व‚भा‚{\tiny $_{lb}$}‚वो भिन्नोस्ति स एक‚त्राभिम‚तेर्थे निय‚तो य‚द्द‚र्श‚नादिष्टार्थ‚प्र‚तीतिः स्यात् । \textbf{स‚त्य}‚{\tiny $_{lb}$}‚पि \textbf{वा} स्व‚भाव‚भेदे स स्व‚भाव‚भेदः प्र‚त्य‚क्षो वा स्याद‚प्र‚त्य‚क्षो वा । न ताव‚त्प्र‚त्य‚क्षः‚{\tiny $_{lb}$}‚ [।] किङ्कार‚णं [।] \textbf{प्र‚त्य‚क्ष‚स्य} स्व‚भाव‚स्योप‚देश‚निर‚पेक्ष‚स्य । \textbf{स्व‚यं प्र‚तीतिप्र‚सं‚{\tiny $_{lb}$}‚गात्} । अथाप्र‚त्य‚क्षः [।] \textbf{अप्र‚त्य‚क्ष‚त्वेपि} प्र‚माणान्त‚र‚{\tiny $_{५}$}‚स्याभावेन \textbf{केन‚चि}द‚प्य‚र्वा‚{\tiny $_{lb}$}‚ग्द‚र्श‚नेन \textbf{ज्ञातुम‚श‚क्य‚त्वात्} ।
	{\color{gray}{\rmlatinfont\textsuperscript{§~\theparCount}}}
	\pend% ending standard par
      ‚{\tiny $_{lb}$}‚

	  
	  \pstart \leavevmode% starting standard par
	अभ्युप‚ग‚म्यैत‚दुक्तं । \textbf{न चास्ति क‚श्चिद् विशेषो} य एकार्थ‚प्र‚तिनिय‚तः । \textbf{स‚र्व‚{\tiny $_{lb}$}‚श‚ब्दा हि स‚र्वार्थ‚प्र‚त्यास‚त्तिविप्र‚क‚र्ष‚र‚हिताः} ।
	{\color{gray}{\rmlatinfont\textsuperscript{§~\theparCount}}}
	\pend% ending standard par
      ‚{\tiny $_{lb}$}‚

	  
	  \pstart \leavevmode% starting standard par
	न हि केचिच्छ‚ब्दाः क्व‚चिद‚र्थे प्र‚त्यास‚न्ना विप्र‚कृष्टा वा \textbf{भाव‚तो}ऽपि तु \textbf{तेषां}‚{\tiny $_{lb}$}‚ श‚ब्दानां स‚र्वार्थेषु स्व‚भाव‚त‚स्तुल्यानामेकार्थ‚निय‚मे व‚क्तु\textbf{र्विव‚क्षा हेतुः । संकेत‚स्त‚{\tiny $_{lb}$}‚त्प्र‚का‚{\tiny $_{६}$}‚श‚न} इति त‚स्याश्च विव‚क्षायाः प्र‚काश‚नः संकेतः । त‚था ह्य‚य‚म‚र्थो\edtext{}{\lemma{र्थो}\Bfootnote{? र्थः}}‚{\tiny $_{lb}$}‚ तेन विव‚क्षित इति संकेताद‚व‚ग‚म्य‚ते । \textbf{सा} च विव‚क्षा\textbf{पौरुषेये} श‚ब्दे \textbf{नास्ती}ति कृत्वा‚{\tiny $_{lb}$}‚ ‚{\tiny $_{lb}$}‚ \leavevmode\ledsidenote{\textenglish{606/s}}\textbf{त‚स्या}पौरुषेय‚स्य श‚ब्द‚स्य \textbf{सा} य‚थोक्ता \textbf{एकार्थ‚ता कुतः} [।] नैव ।
	{\color{gray}{\rmlatinfont\textsuperscript{§~\theparCount}}}
	\pend% ending standard par
      ‚{\tiny $_{lb}$}‚

	  
	  \pstart \leavevmode% starting standard par
	\textbf{विव‚क्षा ही}त्यादिना व्याच‚ष्टे । \textbf{विव‚क्ष‚या} हेतुभूत‚या \textbf{श‚ब्दोर्थेषु निय‚म्य‚ते}‚{\tiny $_{lb}$}‚ \leavevmode\ledsidenote{\textenglish{213b/PSVTa}} ऽस्यैवार्थ‚स्यायं वाच‚क इति । \textbf{न} तु \textbf{स्व‚भाव‚तः} [।] किं कार‚णं [।] \textbf{त‚स्य} श‚{\tiny $_{७}$}‚ब्द‚स्य‚{\tiny $_{lb}$}‚ क्व‚चिद् व‚स्तुन्य\textbf{प्र‚तिब‚न्धे}न स‚म्ब‚न्ध‚र‚हित‚त्वेन कार‚णेन \textbf{स‚र्व‚त्रा}र्थ‚तु\textbf{ल्य‚त्वात् । य‚त्रापि}‚{\tiny $_{lb}$}‚ श‚ब्द‚स्य \textbf{प्र‚तिब‚न्धः} स्थान‚क‚र‚णेषु त‚तः श‚ब्दानामुत्प‚त्तेर‚भिव्य‚क्तेर्वा । तेषाम‚पि‚{\tiny $_{lb}$}‚ क‚र‚णानां \textbf{स‚र्वैः श‚ब्दै}र\textbf{भिधान‚निय‚म‚भावात्} । न हि स‚र्वे श‚ब्दाः \textbf{क‚र‚णाना}मे‚{\tiny $_{lb}$}‚\textbf{वाभिधाय}काः । क‚र‚णाभिधायी पुनः श‚ब्दः क‚र‚णानि प्र‚तिपाद‚येद‚पीति स‚र्व‚ग्र‚ह‚णं ।
	{\color{gray}{\rmlatinfont\textsuperscript{§~\theparCount}}}
	\pend% ending standard par
      ‚{\tiny $_{lb}$}‚

	  
	  \pstart \leavevmode% starting standard par
	य‚त एव\textbf{न्त‚स्माद् वि‚{\tiny $_{१}$}‚व‚क्षाप्र‚काश‚नाय संकेतः क्रिय‚ते} [।] कीदृशः [।]‚{\tiny $_{lb}$}‚ \textbf{अभिप्राय‚निवेद‚न‚ल‚क्ष‚णः} । व‚क्तुर‚भिप्राय‚प्र‚काश‚न‚ल‚क्ष‚णः । \textbf{अपौरुषेये} तु श‚ब्दे \textbf{न‚{\tiny $_{lb}$}‚ विव‚क्षा} नियामिका । \textbf{ना}पि \textbf{संकेत}स्त‚त्प्र‚काश‚नः । किं कार‚णं [।] पुरुष‚निवृत्तेरेव‚{\tiny $_{lb}$}‚ \textbf{क‚स्य‚चिद‚भिप्राय‚स्याभावादिति} कृत्वा \textbf{नै}कार्थ\textbf{निय‚मो} वैदिकानां श‚ब्दानां । निय‚मे‚{\tiny $_{lb}$}‚ वा । \textbf{न त‚ज्ज्ञानं} नैकार्थ‚प्र‚तिनिय‚म‚ज्ञानं क‚स्य‚चित् । अर्थे स्व‚{\tiny $_{२}$}‚भाव‚त एकार्थ‚निय‚ता‚{\tiny $_{lb}$}‚ वैदिकाः श‚ब्दा न विव‚क्षातः ।
	{\color{gray}{\rmlatinfont\textsuperscript{§~\theparCount}}}
	\pend% ending standard par
      ‚{\tiny $_{lb}$}‚

	  
	  \pstart \leavevmode% starting standard par
	त‚दा \textbf{स्व‚भाव‚निय‚मे}भ्युप‚ग‚म्य‚माने स श‚ब्दो\textbf{न्य‚त्रार्थे त‚या} विव‚क्ष‚या \textbf{न‚{\tiny $_{lb}$}‚ वियुज्य‚ते} ।
	{\color{gray}{\rmlatinfont\textsuperscript{§~\theparCount}}}
	\pend% ending standard par
      ‚{\tiny $_{lb}$}‚

	  
	  \pstart \leavevmode% starting standard par
	\textbf{य‚दी}त्यादिना व्याच‚ष्टे । \textbf{य‚दि संकेत‚निर‚पेक्षः स्व‚भाव‚त एव श‚ब्दोर्थेषु निलीनो}‚{\tiny $_{lb}$}‚ युक्तः \textbf{स्यात् । उक्त‚म}त्रोत्त‚रं [।] क्व‚चिद्व‚स्तु\textbf{न्य‚प्र‚तिब‚न्धाद‚निय‚त} इति ।
	{\color{gray}{\rmlatinfont\textsuperscript{§~\theparCount}}}
	\pend% ending standard par
      ‚{\tiny $_{lb}$}‚

	  
	  \pstart \leavevmode% starting standard par
	\textbf{अपि च [।] स्वाभाविकेन} निस‚र्ग‚सिद्धे \textbf{वाच्य‚वाच‚क‚भावे}ऽभ्युप‚ग‚म्य‚मा‚{\tiny $_{३}$}‚ने । \textbf{न‚{\tiny $_{lb}$}‚ पुन‚र्विव‚क्ष‚या य‚थेष्टं} श‚ब्दोन्य‚त्रार्थे \textbf{नियुज्य‚ते} । न हि रूप‚प्र‚काश‚ने स्व‚भाव‚तो निय‚तं‚{\tiny $_{lb}$}‚ च‚क्षुः श‚ब्द‚प्र‚काश‚ने नियोक्तुं श‚क्य‚ते । नियुज्य‚ते च य‚थेष्टं श‚ब्दः [।] त‚स्मान्न‚{\tiny $_{lb}$}‚ स्व‚भाव‚निय‚तः । ह‚स्त‚संज्ञादिव‚त् । स्व‚भाव‚तोर्थ‚प्र‚तिनिय‚मे श‚ब्दानां \textbf{संकेत‚श्च‚{\tiny $_{lb}$}‚ ‚{\tiny $_{lb}$}‚ \leavevmode\ledsidenote{\textenglish{607/s}}निर‚र्थो} निष्फ‚लः स्यात् । य‚तो \textbf{न हि स्व‚भाव‚भेदः} स्व‚भाव‚विशेष \textbf{इन्द्रिय‚ग‚म्यः स्व‚प्र‚{\tiny $_{lb}$}‚तीतौ प‚रिभाषा‚{\tiny $_{४}$}‚दिकं} । प‚रिभाषा संकेतः । आदिश‚ब्दात् संकेत‚स्मृत्यादिप‚रिग्र‚हः ।‚{\tiny $_{lb}$}‚ किमिव [।] \textbf{नीलादिभेद‚व‚त्} । य‚था नीलादिविशेषाः स्व‚प्र‚तीतौ संकेतादिकं‚{\tiny $_{lb}$}‚ नापेक्ष‚न्ते त‚द्व‚त् । \textbf{त‚द‚पेक्ष‚प्र‚तीत‚य‚स्तु} संकेतापेक्ष‚प्र‚तीत‚य‚स्तु ये । ते \textbf{न व‚स्तुस्व‚{\tiny $_{lb}$}‚भावाः} । न व‚स्तुनो निस‚र्ग‚सिद्धाः विशेषाः [।] \textbf{किन्त‚र्हि [।] साम‚यिकाः} संकेत‚{\tiny $_{lb}$}‚कृताः । किमिव [।] \textbf{राज‚चिह्नादिव‚त्} । य‚था राज्ञा‚{\tiny $_{५}$}‚ स्व‚प्र‚तीत‚ये स‚मिता ध्व‚जा‚{\tiny $_{lb}$}‚द‚य‚श्चिन्ह‚भेदाः । आदिश‚ब्दाद्ध‚स्त‚संज्ञादिप‚रिग्र‚हः । \textbf{य}श्च \textbf{साम‚यिक}स्स \textbf{स्व‚भाव‚{\tiny $_{lb}$}‚निय‚तो युक्तः} । निस‚र्ग‚सिद्धो न \textbf{युक्तः} । किं कार‚णं [।] \textbf{त‚स्य} साम‚यिक‚स्य‚{\tiny $_{lb}$}‚ पुरुष‚स्य पुरुषे\textbf{च्छ}या \textbf{प्र‚वृत्तेः} ।
	{\color{gray}{\rmlatinfont\textsuperscript{§~\theparCount}}}
	\pend% ending standard par
      ‚{\tiny $_{lb}$}‚

	  
	  \pstart \leavevmode% starting standard par
	वैदिक‚स्य श‚ब्द‚स्यार्थे निस‚र्ग‚त एव स्व‚भाव‚विशेषो निय‚तः स तु संकेतेन‚{\tiny $_{lb}$}‚ व्य‚ज्य‚त इति चेद् [।]
	{\color{gray}{\rmlatinfont\textsuperscript{§~\theparCount}}}
	\pend% ending standard par
      ‚{\tiny $_{lb}$}‚

	  
	  \pstart \leavevmode% starting standard par
	आह । \textbf{अत एवेत्}यादि । य‚स्मादिच्छावृत्तिस्स‚{\tiny $_{६}$}‚ङ्केतो\textbf{ऽत एव} कार‚णात् ।‚{\tiny $_{lb}$}‚ \textbf{संकेता}देकार्थ‚निय‚त‚स्य \textbf{स्व‚भाव‚विशेष‚स्य व्य‚क्तौ निय‚तिः कुतः} [।] नैव ।
	{\color{gray}{\rmlatinfont\textsuperscript{§~\theparCount}}}
	\pend% ending standard par
      ‚{\tiny $_{lb}$}‚

	  
	  \pstart \leavevmode% starting standard par
	त‚द् व्याच‚ष्टे [।] \textbf{स्वेच्छावृत्तिस्संकेत‚स्स इहैवा}भिम‚तेर्थे \textbf{क‚र्त्तुं श‚क्य‚ते नान्य‚{\tiny $_{lb}$}‚त्रेति नोप‚रोधोस्ति} [।] न वाच‚क‚म‚स्ति । त‚त‚श्च \textbf{स} संकेतः \textbf{पुरुषैः स्वेच्छ‚या‚{\tiny $_{lb}$}‚ क्रिय‚माण‚स्त‚मेवै}कार्थ‚निय‚तं \textbf{स्व‚भावं व्य‚न‚क्ति नान्य‚मिति न निय‚मोस्ति} ।
	{\color{gray}{\rmlatinfont\textsuperscript{§~\theparCount}}}
	\pend% ending standard par
      ‚{\tiny $_{lb}$}‚

	  
	  \pstart \leavevmode% starting standard par
	\textbf{य‚त्र} संकेते \textbf{स्वात‚न्त्र्य‚मिच्छाया‚{\tiny $_{७}$}‚ निय‚मो नाम त‚त्र कः} । नैवास्ति निय‚म \leavevmode\ledsidenote{\textenglish{214a/PSVTa}}‚{\tiny $_{lb}$}‚ इत्य‚र्थः । \textbf{तेना}निय‚त‚त्वेन कार‚णेन्\textbf{आस्य वै}दिक‚स्य श‚ब्द‚स्ये\textbf{ष्टामेवा}भिम‚तार्थ‚विष‚{\tiny $_{lb}$}‚यामेव । \textbf{योग्य‚तां संकेतो न द्योत‚ये}दिति ।
	{\color{gray}{\rmlatinfont\textsuperscript{§~\theparCount}}}
	\pend% ending standard par
      ‚{\tiny $_{lb}$}‚

	  
	  \pstart \leavevmode% starting standard par
	त‚देव‚म‚पौरुषेय‚त्वं नाग‚म‚ल‚क्ष‚ण‚मिति प्र‚तिपादित‚म् ।
	{\color{gray}{\rmlatinfont\textsuperscript{§~\theparCount}}}
	\pend% ending standard par
      ‚{\tiny $_{lb}$}‚‚{\tiny $_{lb}$}‚\textsuperscript{\textenglish{608/s}}

	  
	  \pstart \leavevmode% starting standard par
	\textbf{इदा}नीमेक‚देशाविस‚म्वाद‚न‚माग‚म‚ल‚क्ष‚णं दूष‚यितुमुप‚न्य‚स्य‚ति । \textbf{य‚स्मा}दित्यादि ।‚{\tiny $_{lb}$}‚ \textbf{किल} श‚ब्दोन‚भिप्राय‚द्योत‚कः ।‚{\tiny $_{१}$}‚ \textbf{अग्निः शीत‚नोद‚नः} शीत‚स्य निवार‚कः । एतेना‚{\tiny $_{lb}$}‚ग्निर्हिम‚स्य भेष‚ज‚मित्येत\textbf{द्वाक्यं} य‚था स‚त्यं त‚था\textbf{न्य‚द‚पि वाक्य‚म}ग्निहोत्रादिकं जुहु‚{\tiny $_{lb}$}‚यादित्यादिक‚म‚वित‚थ‚मेत‚त् साध्यं । \textbf{वेदैक‚देश‚त्वादि}ति हेतुः । एव‚म‚प‚रो वृ द्ध मी‚{\tiny $_{lb}$}‚ मां स को ब्र‚वीत् । उक्त‚वान् ।
	{\color{gray}{\rmlatinfont\textsuperscript{§~\theparCount}}}
	\pend% ending standard par
      ‚{\tiny $_{lb}$}‚

	  
	  \pstart \leavevmode% starting standard par
	\textbf{अन्य‚स्त्वि}त्यादिना व्याच‚ष्टे । \textbf{अन्य‚स्तु} मीमांस‚कः । य‚थोक्त‚दोषोप‚ह‚त‚त्वात् ।‚{\tiny $_{lb}$}‚ \textbf{अपौरुषेय‚माग‚म‚ल‚क्ष‚{\tiny $_{२}$}‚णं प‚रित्य‚ज्यान्य‚था प्रामाण्य‚म्वेद‚स्य साध‚यितुकामः प्राह ।‚{\tiny $_{lb}$}‚ अवित‚थानी}त्यादि । \textbf{य‚त्राप्र‚तिप‚त्ति}रिति येषु वेद‚वाक्येष्व‚वित‚थ‚त्वेन बौ द्ध स्याप्र‚ति‚{\tiny $_{lb}$}‚प‚त्तिस्ता\textbf{न्य‚वित‚थानीत्य}नेन विशेष‚स्य प‚क्षीक‚र‚णात् । \textbf{वैदैक‚देश‚त्वा}दिति । सामा‚{\tiny $_{lb}$}‚न्य‚स्य हेतुत्वेनोपादानान्न प्र‚तिज्ञार्थैक‚देश‚ता हेतोर‚स्ति । \textbf{य‚था [।] अग्निर्हिम‚स्य‚{\tiny $_{lb}$}‚ भेष‚जं} प्र‚तिप‚क्ष \textbf{इत्यादि} वाक्य‚वत्‚{\tiny $_{३}$}‚ । आदिश‚ब्दाद् द्वाद‚श मासा स‚म्व‚त्स‚र‚{\tiny $_{lb}$}‚ इत्यादिवाक्य‚प‚रिग्र‚हः ।
	{\color{gray}{\rmlatinfont\textsuperscript{§~\theparCount}}}
	\pend% ending standard par
      ‚{\tiny $_{lb}$}‚

	  
	  \pstart \leavevmode% starting standard par
	उत्त‚र‚माह । \textbf{त‚स्ये}त्यादि । \textbf{त‚स्य} वादिन \textbf{इदं} साध‚नं शेष‚व‚त् । क‚स्माद् [।]‚{\tiny $_{lb}$}‚ \textbf{व्य‚भिचारित्वात् । ईदृश‚म}नुमानं \textbf{न्याय‚विदा} आचार्य दि ग्ना गे न प्र मा ण स मु च्च ये‚{\tiny $_{lb}$}‚ प्र‚ति\textbf{क्षिप्त}मिति स‚म्ब‚न्धः । किमिव शेष‚व‚दित्याह । \textbf{र‚स‚व‚दि}त्यादि । य‚था । स्वा‚{\tiny $_{lb}$}‚दितेन फ‚लेन तु\textbf{ल्य‚रूप‚त्वाद}नास्वादित‚{\tiny $_{४}$}‚म‚पि फ‚लं तुल्य‚मित्येत‚द‚नुमानं शेष‚व‚त् ।‚{\tiny $_{lb}$}‚ त‚द्व‚त् । अदृष्टा अपि त‚ण्डुलाः प‚क्वा एक‚भाण्डे प‚च‚नात् । दृष्ट‚प‚क्व‚त‚ण्डुल‚व‚दित्ये‚{\tiny $_{lb}$}‚\textbf{व‚मेक‚भाण्डे च पाक‚व‚त्} । य‚थैव त‚द‚नुमानं \textbf{शेष‚व‚त्} त‚था मी मां स को क्त मिति ।
	{\color{gray}{\rmlatinfont\textsuperscript{§~\theparCount}}}
	\pend% ending standard par
      ‚{\tiny $_{lb}$}‚

	  
	  \pstart \leavevmode% starting standard par
	\textbf{स्व}य‚मित्यादिना व्याख्यानं । \textbf{ईदृश‚म‚नुमानं स्व‚य‚माचार्येणा}साध‚न‚मुक्त‚मिति‚{\tiny $_{lb}$}‚ ‚{\tiny $_{lb}$}‚ \leavevmode\ledsidenote{\textenglish{609/s}}स‚म्ब‚न्धः । क्व [।] \textbf{नै या यि का} नां \textbf{शेष‚व‚द‚नुमा‚{\tiny $_{५}$}‚न}स्य \textbf{व्य‚भिचार‚मुद्भाव‚य‚ता}‚{\tiny $_{lb}$}‚ प्र मा ण स मु च्च ये । किमिव । य‚था \textbf{तुल्य‚रूप‚त‚या} हेतुभूत‚याऽनास्वादिताना‚{\tiny $_{lb}$}‚म‚पि \textbf{फ‚लाना}मास्वादित‚फ‚लेन । \textbf{तुल्य‚र‚स‚साध‚न‚व‚त् । एक‚स्थाल्य‚न्त‚र्ग‚मादिति च}‚{\tiny $_{lb}$}‚ हेतुना । \textbf{दृष्ट}प‚रिप‚क्व‚त‚ण्डुल\textbf{व‚द‚दृष्ट‚त‚ण्डुला}नां \textbf{पाक‚साध‚न}व‚त् । \textbf{त‚द‚साध‚न‚त्व‚न्याय}‚{\tiny $_{lb}$}‚श्चेति त‚स्य शेष‚व‚तोनुमान‚स्यासाधन‚{\tiny $_{६}$}‚त्व‚न्याय‚श्च । य‚स्माद‚द‚र्श‚न‚मात्रेण व्य‚ति‚{\tiny $_{lb}$}‚रेकः प्र‚द‚र्श्य‚त इत्यादिना \textbf{पूर्व‚मेवोक्तः} । न‚नु च [।]
	{\color{gray}{\rmlatinfont\textsuperscript{§~\theparCount}}}
	\pend% ending standard par
      ‚{\tiny $_{lb}$}‚
	  \bigskip
	  \begingroup
	
	    
	    \stanza[\smallbreak]
	  {\normalfontlatin\large ``\qquad}आप्त‚वादाविस‚म्वाद‚सामान्याद‚नुमान‚ता \href{http://sarit.indology.info/?cref=pv.3.216}{प्र० वा० १। २१८}{\normalfontlatin\large\qquad{}"}\&[\smallbreak]
	  
	  
	  
	  \endgroup
	‚{\tiny $_{lb}$}‚

	  
	  \pstart \leavevmode% starting standard par
	इत्यादिना आचार्य दि ग्ना गे नाप्येक‚देशाविस‚म्वाद‚न‚माग‚म‚ल‚क्ष‚ण‚मुक्त‚{\tiny $_{lb}$}‚मेवेति [।]
	{\color{gray}{\rmlatinfont\textsuperscript{§~\theparCount}}}
	\pend% ending standard par
      ‚{\tiny $_{lb}$}‚
	    
	    \stanza[\smallbreak]
	  आह । \textbf{उक्तं चेद}मित्यादि । \textbf{उक्तं चेद‚म्} [।]&‚{\tiny $_{lb}$}‚
	  \bigskip
	  \begingroup
	एक‚देशाविस‚म्वादिरूप‚माग‚म‚ल‚क्ष‚ण‚म् [।]
	  \endgroup
	\&[\smallbreak]
	  
	  
	  ‚{\tiny $_{lb}$}‚

	  
	  \pstart \leavevmode% starting standard par
	\textbf{अस्माभि}र्नायं‚{\tiny $_{७}$}‚ पुरुषो नाश्रित्याग‚म‚प्रामाण्य‚मासितुं स‚म‚र्थ इत्य‚त्रान्त‚रे ।
	{\color{gray}{\rmlatinfont\textsuperscript{§~\theparCount}}}
	\pend% ending standard par
      \textsuperscript{\textenglish{214b/PSVTa}}‚{\tiny $_{lb}$}‚

	  
	  \pstart \leavevmode% starting standard par
	त‚त्रैक‚देशाविसंवाद‚न‚माग‚म‚ल‚क्ष‚णं नात्य‚न्त‚प्र‚सिद्धैक‚विष‚य‚स‚त्य‚ताश्र‚य‚म‚पि तु‚{\tiny $_{lb}$}‚ \textbf{त}त्त्वाग‚म‚ल‚क्ष‚णं योऽर्वाग्द‚र्श‚नेन प्र‚माण‚तः श‚क्य‚प‚रिच्छेदः । अशेषो विष‚य‚स्त‚स्य‚{\tiny $_{lb}$}‚ \textbf{स‚र्व‚स्य श‚क्य‚विचार‚स्}य \textbf{विष‚य‚स्य} [।] श‚क्यो विचारोस्येति विग्र‚हः । \textbf{य‚थास्वं‚{\tiny $_{lb}$}‚ प्र‚माणेन विधिप्र‚तिषेध‚स्वीकृत‚{\tiny $_{१}$}‚}सिद्धिव‚त् \edtext{}{\lemma{त्}\Bfootnote{?}} विशुद्धा शास्त्रे प्र‚त्य‚क्ष‚विष‚य‚त्वे‚{\tiny $_{lb}$}‚नाभिम‚तानां प्र‚त्य‚क्ष‚त्वं । य‚था बौ द्ध सि द्धा न्ते बुद्ध्यादीनां । त‚था व‚स्तु‚{\tiny $_{lb}$}‚व‚लायातानुमान‚विष‚याभिम‚तानां व‚स्तुब‚लानुमान‚विष‚य‚त्वं । य‚था दुःख‚स‚त्यादीनां ।‚{\tiny $_{lb}$}‚ आग‚मापेक्षानुमान‚विष‚याभिम‚तानां च त‚थाभावः । त्रिविध‚स्य विष‚य‚स्य य‚थास्वं‚{\tiny $_{lb}$}‚ प्र‚माणेन विधिसिद्धिः । प्र‚त्य‚क्षादिप्र‚माणा वि{ष‚य...}‚{\tiny $_{२}$}‚ ता नाम प्र‚त्य‚क्षादित्वं‚{\tiny $_{lb}$}‚ य‚थास्वं प्र‚माणेन प्र‚तिषेध‚सिद्धिः । एवं विधिप्र‚तिषेध‚सिद्धौ स‚त्याम्प‚रिशिष्टेष्व‚{\tiny $_{lb}$}‚त्य‚न्त‚प‚रोक्षेष्\textbf{व‚र्थेषु श‚ब्दानां नान्त‚र‚तीय‚क‚ताभावे} स‚म्ब‚न्धाभावे स‚त्\textbf{य‚पि} व‚र‚म‚ग‚त्या‚{\tiny $_{lb}$}‚ \textbf{संश‚यित‚स्य} पुरुष‚स्य \textbf{प्र‚वृत्तिस्त‚त्रे}ति स‚र्व‚स्मिन् व‚स्तुन्य‚दृष्ट‚व्य‚भिचार \textbf{आग‚मे‚{\tiny $_{lb}$}‚ क‚दाचिद‚विस‚म्वाद‚स‚म्भ‚वात्} ।
	{\color{gray}{\rmlatinfont\textsuperscript{§~\theparCount}}}
	\pend% ending standard par
      ‚{\tiny $_{lb}$}‚

	  
	  \pstart \leavevmode% starting standard par
	\textbf{न‚न्व‚न्य‚त्रे}ति [।] य‚त्राग‚मे य‚थोक्त‚विष‚य‚श{... ...त}‚{\tiny $_{३}$}‚प्र‚त्य‚क्षादिविष‚येपि‚{\tiny $_{lb}$}‚ ‚{\tiny $_{lb}$}‚ \leavevmode\ledsidenote{\textenglish{610/s}}\textbf{दृष्टः प्र‚मा}णेनो\textbf{प‚रोधो} वा । येन पुरुषेण त\textbf{स्य पुरुष‚स्य प्र‚वृत्ति}र्न युक्तेत्येव‚माग‚{\tiny $_{lb}$}‚म‚ल‚क्ष‚ण‚मुक्त‚म‚स्माभिः ।
	{\color{gray}{\rmlatinfont\textsuperscript{§~\theparCount}}}
	\pend% ending standard par
      ‚{\tiny $_{lb}$}‚

	  
	  \pstart \leavevmode% starting standard par
	\textbf{यः पुन} र्मी मां स कादिः \textbf{प्राकृत}पुरुषाणां \textbf{विष‚य‚स्य व‚ह‚नेर्य‚च्छीत‚प्र‚तिघात‚सा‚{\tiny $_{lb}$}‚म‚र्थ्य}न्त\textbf{स्याभिधान}म‚ग्निर्हिम‚स्य भेष‚ज‚मित्येत‚द् वाक्य‚म्वेदैक‚देश‚भूतं \textbf{स‚त्यार्थ} ।‚{\tiny $_{lb}$}‚ दृष्टान्त‚त्वेनो\textbf{प‚द‚र्श्य स‚{\tiny $_{४}$}‚र्वं स‚त्यार्थ}म्वेद‚ल‚क्ष‚णं शास्त्र‚मि\textbf{त्याह} । किम्भूतं शास्त्रं‚{\tiny $_{lb}$}‚ \textbf{श‚क्य‚प‚रिच्छेदे} प्र‚माण‚ग‚म्ये\textbf{पि विष‚ये प्र‚माण‚विरोधाद् ब‚हुत‚र‚म‚युक्त‚म‚पि} त‚देवंभूतं‚{\tiny $_{lb}$}‚ शास्त्रं प्र‚तिजानानो मीमांस‚कादिर्ज‚येद् धार्ष्ट्येन‚ब‚न्ध‚कीमिति व‚क्ष्य‚माणेन स‚म्ब‚न्धः ।
	{\color{gray}{\rmlatinfont\textsuperscript{§~\theparCount}}}
	\pend% ending standard par
      ‚{\tiny $_{lb}$}‚

	  
	  \pstart \leavevmode% starting standard par
	ब‚हुत‚र‚म‚युक्त‚म‚पीत्युक्त‚न्त‚द्द‚र्श‚य‚न्नाह । \textbf{नित्य‚स्ये}त्यादि । व‚द‚दित्येत‚त्प‚दं स‚र्व‚त्र‚{\tiny $_{lb}$}‚ स‚म्ब‚न्धनीयं ।‚{\tiny $_{५}$}‚ नित्यः क‚र्त्ता पुरुषोस्तीत्येव\textbf{न्नित्य‚स्य पुंसः क‚र्त्तृत्व}म्व‚द‚{\tiny $_{lb}$}‚च्छास्त्रं । त‚था \textbf{नित्यान् भावान्} व‚द‚त् । \textbf{अतीन्द्रियान}प्र‚त्य‚क्षान‚र्था\textbf{नैद्रियान्}‚{\tiny $_{lb}$}‚ प्र‚त्य‚क्षान् व‚द‚त् । त‚था \textbf{विष‚म}म‚युक्तं \textbf{हेतुं} भावानां व‚द‚त् । त‚था \textbf{भावा‚{\tiny $_{lb}$}‚नाम्विष‚मां स्थितिं । निवृत्तिञ्च} भावानाम्विष‚माम्व‚द‚त् । एत‚च्च वृत्तौ स्प‚ष्ट‚{\tiny $_{lb}$}‚यिष्यामः । य‚थोक्ताद\textbf{न्य‚द्वा} व‚स्तु \textbf{व्य‚स्त‚गोच‚रं} । व्य‚स्तः प्र‚{\tiny $_{६}$}‚तिक्षिप्तो गोच‚रो‚{\tiny $_{lb}$}‚व‚काशो य‚स्य त‚त्त‚थोक्तं । केन व्य‚स्त‚गोच‚र‚मित्याह । \textbf{प्र‚माणाभ्यां} प्र‚त्य‚क्षानु‚{\tiny $_{lb}$}‚मानाभ्यां निर‚स्तमस‚म्भ‚व‚मिति याव‚त् । त‚देव‚म्भूतं व‚स्तु व‚द‚त् । त‚था\textbf{ग‚मा‚{\tiny $_{lb}$}‚पेक्षेणानुमानेन विरुद्ध‚म्व‚द}च्छास्त्रं । त‚देव‚म‚नेकायुक्तार्थाभिधाय‚कं शास्त्रं \textbf{स‚त्यार्थं‚{\tiny $_{lb}$}‚ प्र‚तिजानानो} वादी । \textbf{ज‚येद् धार्ष्ट्येन ब‚न्ध‚कीं} ।
	{\color{gray}{\rmlatinfont\textsuperscript{§~\theparCount}}}
	\pend% ending standard par
      ‚{\tiny $_{lb}$}‚

	  
	  \pstart \leavevmode% starting standard par
	\leavevmode\ledsidenote{\textenglish{215a/PSVTa}} किम‚कृत्वा प्र‚तिजानान‚{\tiny $_{७}$}‚ इत्याह । \textbf{विरोध}मित्यादि । श‚क्य‚विचारे व‚स्तुनि‚{\tiny $_{lb}$}‚ शास्त्र‚स्य \textbf{विरोध‚म‚स‚माधाया}प‚रिहृत्य पुरुष‚शास्त्र‚प्र‚वृत्तौ निमित्तं \textbf{शास्त्रार्थ‚ञ्च}‚{\tiny $_{lb}$}‚ स‚म्ब‚न्धानुगुणोपाय‚पुरुषार्थ‚ल‚क्ष‚ण\textbf{म‚प्र‚द‚र्श्य प्र‚तिजानानः} ।
	{\color{gray}{\rmlatinfont\textsuperscript{§~\theparCount}}}
	\pend% ending standard par
      ‚{\tiny $_{lb}$}‚

	  
	  \pstart \leavevmode% starting standard par
	\textbf{अप्र‚च्युते}रित्यादिना व्याच‚ष्टे । \textbf{अप्र‚च्युतं पूर्वं} रूपं । \textbf{अनुत्प‚न्नं चाप‚रं रूपं}‚{\tiny $_{lb}$}‚ ‚{\tiny $_{lb}$}‚ \leavevmode\ledsidenote{\textenglish{611/s}}य‚स्येति विग्र‚हः । ईदृशः किल \textbf{पुमान्} सुकृत‚दुष्कृतानां \textbf{क‚र्म‚णां क्र‚मेण‚{\tiny $_{१}$}‚ क‚र्त्ता । क‚र्म‚{\tiny $_{lb}$}‚फ‚लानां च भोक्ता} । केन प्र‚कारेण भोक्ता क‚र्त्ता चेत्याह । पूर्व‚क‚र्म‚ज‚नित‚सुख‚{\tiny $_{lb}$}‚दुःखादिस‚म्वित्तिं प्र‚ति स‚म‚वायिकार‚ण‚भावेनात्मा क‚र्म फ‚लानाम्भोक्ताः । त‚दुक्तं ।
	{\color{gray}{\rmlatinfont\textsuperscript{§~\theparCount}}}
	\pend% ending standard par
      ‚{\tiny $_{lb}$}‚
	  \bigskip
	  \begingroup
	
	    
	    \stanza[\smallbreak]
	  {\normalfontlatin\large ``\qquad}सुख‚दुःखादिस‚म्वित्तिस‚म‚वाय‚स्तु भोक्तृतेति ।{\normalfontlatin\large\qquad{}"}\&[\smallbreak]
	  
	  
	  
	  \endgroup
	‚{\tiny $_{lb}$}‚

	  
	  \pstart \leavevmode% starting standard par
	शुभाशुभ‚क‚र्म‚क‚र‚णे ज्ञान‚प्र‚य‚त्नादिकं प्र‚ति अधिष्ठान‚भावेनात्मा क‚र्म‚णा क‚र्त्ता ।‚{\tiny $_{lb}$}‚ त‚दुक्तं [।]
	{\color{gray}{\rmlatinfont\textsuperscript{§~\theparCount}}}
	\pend% ending standard par
      ‚{\tiny $_{lb}$}‚
	  \bigskip
	  \begingroup
	
	    
	    \stanza[\smallbreak]
	  {\normalfontlatin\large ``\qquad}ज्ञान‚य‚त्नाभिस‚म्ब‚न्धः क‚{\tiny $_{२}$}‚र्त्तृत्व‚न्त‚स्य भ‚ण्य‚त इति ।{\normalfontlatin\large\qquad{}"}\&[\smallbreak]
	  
	  
	  
	  \endgroup
	‚{\tiny $_{lb}$}‚

	  
	  \pstart \leavevmode% starting standard par
	आदिग्र‚ह‚णात् । ज‚ड‚रूप‚स्याप्यात्म‚न‚श्चेत‚नायोगेन भोक्तृत्वं गृह्य‚ते । त‚दु‚{\tiny $_{lb}$}‚क्त‚म् [।]
	{\color{gray}{\rmlatinfont\textsuperscript{§~\theparCount}}}
	\pend% ending standard par
      ‚{\tiny $_{lb}$}‚
	  \bigskip
	  \begingroup
	
	    
	    \stanza[\smallbreak]
	  {\normalfontlatin\large ``\qquad}भोक्ता च चेत‚नायोगात् चेत‚नं न स्व‚रूप‚त इति ।{\normalfontlatin\large\qquad{}"}\&[\smallbreak]
	  
	  
	  
	  \endgroup
	‚{\tiny $_{lb}$}‚

	  
	  \pstart \leavevmode% starting standard par
	त‚देवं \textbf{स‚म‚वायिकार‚णाधिष्ठान‚भावादिने}त्याह वेदः । \textbf{त‚च्चै}त‚द\textbf{युक्त‚मित्यावे‚{\tiny $_{lb}$}‚दित‚प्रायं} । स्व‚य‚मेव शास्त्र‚कारेण नित्यानां कार्य‚कार‚ण‚भावास‚म्भ‚व‚न्द‚र्श‚य‚ता ।‚{\tiny $_{lb}$}‚ \textbf{नित्य‚त्वं चायुक्तं केषां\textbf{चिद् भा}‚{\tiny $_{३}$}‚ वानाम्}वेद आहेति स‚म्ब‚न्ध‚नीयं । क‚स्माद‚{\tiny $_{lb}$}‚युक्त‚म् [।] \textbf{अक्ष‚णिक‚स्य} क्र‚म‚यौग‚प‚द्याभ्याम‚र्थ‚क्रियाविरोधेन \textbf{व‚स्तुध‚र्मातिक्र‚मात्} ।‚{\tiny $_{lb}$}‚ अर्थ‚क्रियास‚म‚र्थं हि व‚स्तु । त‚च्चार्थ‚क्रियासाम‚र्थ्य‚म‚क्ष‚णिक‚स्य न स‚म्भ‚व‚तीत्य‚{\tiny $_{lb}$}‚स‚देव [।] त‚त्कुत‚स्त‚स्य व‚स्तुध‚र्मः । \textbf{अप्र‚त्य‚क्षाण्येव सामान्या}दीनीति [।] आदि‚{\tiny $_{lb}$}‚श‚ब्दात् क्रियागुणादीनि \textbf{प्र‚त्य‚क्षाणीत्}याह वेदः । \textbf{ज‚न्म} च \textbf{स्थि}तिश्च \textbf{नि\textbf{वृ}‚{\tiny $_{४}$}‚‚{\tiny $_{lb}$}‚ त्तिश्च} ताश्च भावाना\textbf{म्विष‚माः} प्राह वेदः । ज‚न्म‚नो वैष‚म्येण विष‚यो हेतुर्भावा‚{\tiny $_{lb}$}‚नामुक्तः सूत्रे ।
	{\color{gray}{\rmlatinfont\textsuperscript{§~\theparCount}}}
	\pend% ending standard par
      ‚{\tiny $_{lb}$}‚

	  
	  \pstart \leavevmode% starting standard par
	त‚मेव विष‚मं हेतुमाह । \textbf{अनाधेय‚स्ये}त्यादि । नित्य‚त्वा\textbf{द‚नाधेयातिश‚य}स्य‚{\tiny $_{lb}$}‚ \textbf{प्रागि}त्य‚र्थ‚क्रियाकालाव‚स्थायाः पूर्व‚म\textbf{क‚र्त्तुः} प‚श्चात् \textbf{प‚रापेक्ष‚या} स‚ह‚कार्य‚पेक्ष‚या ।‚{\tiny $_{lb}$}‚ \textbf{ज‚न‚क‚त्व}माह वेदः । त‚च्चैत‚द‚युक्त‚मिति विस्त‚रेण प्र‚तिपादितं‚{\tiny $_{५}$}‚ ।
	{\color{gray}{\rmlatinfont\textsuperscript{§~\theparCount}}}
	\pend% ending standard par
      ‚{\tiny $_{lb}$}‚

	  
	  \pstart \leavevmode% starting standard par
	\textbf{निष्प‚त्ते}रित्यादिना विष‚म‚स्थित्य‚भिधायित्व‚म्वेद‚स्याह । स्व‚हेतुतो \textbf{निष्प‚त्ते-}‚{\tiny $_{lb}$}‚ र्निष्प‚न्न‚त्वाद\textbf{कार्य‚रूप‚स्य} भाव‚स्या\textbf{श्र‚य‚व‚शेन स्थान}माह वेदः । त‚च्चैत‚द‚युक्तं‚{\tiny $_{lb}$}‚ स‚र्व‚निर‚शंस्य नान्य‚ब‚लेन स्थान‚मिति प्राक् प्र‚तिपादितं ।
	{\color{gray}{\rmlatinfont\textsuperscript{§~\theparCount}}}
	\pend% ending standard par
      ‚{\tiny $_{lb}$}‚

	  
	  \pstart \leavevmode% starting standard par
	विष‚मां निवृत्तिन्द‚र्श‚य‚न्नाह । \textbf{कार‚णाच्च} विनाश‚हेतोः स‚काशाद् भावानाम्वि‚{\tiny $_{lb}$}‚‚{\tiny $_{lb}$}‚ \leavevmode\ledsidenote{\textenglish{612/s}}नाश इत्याह वेदः । त‚च्चायुक्त‚म{... ...}‚{\tiny $_{६}$}‚द्विनाश‚स्येति प्र‚तिपादितं ।
	{\color{gray}{\rmlatinfont\textsuperscript{§~\theparCount}}}
	\pend% ending standard par
      ‚{\tiny $_{lb}$}‚

	  
	  \pstart \leavevmode% starting standard par
	एव‚मा\textbf{दिक‚म‚न्य‚द‚पि} स‚दित्याह वेदः । किंभूतं [।] \textbf{प्र‚सिद्ध‚विप‚र्य‚यं} । अस‚त्त्वं‚{\tiny $_{lb}$}‚ हि स‚त्त्व‚विप‚र्य‚यः । प्र‚सिद्धो विप‚र्य‚यो य‚स्य [।] केन [।] \textbf{प्र‚त्य‚क्षानुमानाभ्यां} [।]‚{\tiny $_{lb}$}‚ त‚स्य स‚त्त्व‚माह वेद इत्य‚र्थः । \textbf{अग्निहोत्रादेः} [।] आदिश‚ब्दात् तीर्थ‚स्नानादेः \textbf{पाप‚{\tiny $_{lb}$}‚\leavevmode\ledsidenote{\textenglish{215b/PSVTa}} शोध‚न‚साम‚र्थ्यादिक‚म्}आह वेदः । अत्राप्यादिश‚ब्दाद् ध‚र्मोप‚च‚या{दि... ...}‚{\tiny $_{७}$}‚‚{\tiny $_{lb}$}‚ \textbf{ग‚माश्र‚येणानुमानेन बाधितं} । त‚था हि [।] अध‚र्मो रागादिरूप‚स्त‚त्प्र‚भ‚वं च क‚र्मे‚{\tiny $_{lb}$}‚त्याग‚म‚व्य‚व‚स्था । द्व‚य‚म‚प्येत‚द‚ग्निहोत्रादिना न बाध्य‚त इति क‚थ‚म‚ध‚र्म‚स्य तेन‚{\tiny $_{lb}$}‚ विशुद्धिः । ध‚र्म‚वृद्धिर्वा क‚थ‚न्त‚तः [।] ध‚र्म‚स्यालोभादित‚त्प्र‚भ‚व‚क‚र्म‚स्व‚भाव‚त्वात् ।‚{\tiny $_{lb}$}‚ तीर्थ‚स्नानादीनां चात‚त्स्व‚भाव‚त्वात् । आदिश‚ब्दाद‚न्य‚द‚प्येवंजातीय‚क‚म‚युक्ताभि‚{\tiny $_{lb}$}‚धानं द्र‚ष्ट‚व्यं ।
	{\color{gray}{\rmlatinfont\textsuperscript{§~\theparCount}}}
	\pend% ending standard par
      ‚{\tiny $_{lb}$}‚

	  
	  \pstart \leavevmode% starting standard par
	\textbf{त‚स्यैव}{... ...} । \textbf{वादिनो वेद‚स्य स‚र्व‚त्र} प्र‚त्य‚क्षादिविष‚ये त्रिविधेपि‚{\tiny $_{lb}$}‚ \textbf{शास्त्र‚श‚रीरे} शास्त्र‚प्र‚तिपाद्ये व‚स्तुनि । \textbf{प्र‚माण‚विरोध‚म‚स‚माधाया}प‚रिहृत्य । \textbf{शास्त्रे}‚{\tiny $_{lb}$}‚ प्र‚वृत्त्य‚ङ्ग‚भूता \textbf{ध‚र्मा}स्ता\textbf{न‚प्र‚द‚र्श्य} ।
	{\color{gray}{\rmlatinfont\textsuperscript{§~\theparCount}}}
	\pend% ending standard par
      ‚{\tiny $_{lb}$}‚

	  
	  \pstart \leavevmode% starting standard par
	के पुन‚स्ते ध‚र्मा इत्याह । \textbf{स‚म्ब‚न्धे}त्यादि । प‚र‚स्प‚रं प‚दार्थानां स‚ङ्ग‚तार्थ‚ता‚{\tiny $_{lb}$}‚ \textbf{स‚म्ब‚न्धः} । श‚क्य‚साध‚न उपायः । \textbf{अनुगुणोपायः} । यः शास्त्रे \textbf{पुरुषार्थ}साध‚न‚{\tiny $_{lb}$}‚ उ{क्त... ...दृ}‚{\tiny $_{२}$}‚ ष्ट‚स्स पुरुषेण साध‚यितुं श‚क्य‚त इति याव‚त् । अभ्युद‚य‚निः‚{\tiny $_{lb}$}‚श्रेय‚सं पुरुषार्थः । स‚म्ब‚न्ध‚श्चानुगुणोपाय‚श्च पुरुषार्थ‚श्चेति द्व‚न्द्वः । तेषा\textbf{म‚भिधा‚{\tiny $_{lb}$}‚नानि} ।
	{\color{gray}{\rmlatinfont\textsuperscript{§~\theparCount}}}
	\pend% ending standard par
      ‚{\tiny $_{lb}$}‚

	  
	  \pstart \leavevmode% starting standard par
	न‚नु विरोधास‚माधानादेव शास्त्र‚स्याग्राह्य‚त्व‚मुक्त‚न्त‚त्किं शास्त्र‚ध‚र्माप्र‚द‚{\tiny $_{lb}$}‚र्श‚नेनोक्तेन ।
	{\color{gray}{\rmlatinfont\textsuperscript{§~\theparCount}}}
	\pend% ending standard par
      ‚{\tiny $_{lb}$}‚

	  
	  \pstart \leavevmode% starting standard par
	एव‚म्म‚न्य‚ते [।] प्र‚त्य‚क्षानुमान‚विष‚ये विशुद्ध‚मुप‚द‚र्श्य क‚दाचिद‚न्य‚त्र स‚म्ब‚{\tiny $_{lb}$}‚न्धादिर‚हिते ब्रू{... ... ...}‚{\tiny $_{३}$}‚ द‚र्थं शास्त्र‚ध‚र्माप्र‚द‚र्श‚न‚मुक्तं ।
	{\color{gray}{\rmlatinfont\textsuperscript{§~\theparCount}}}
	\pend% ending standard par
      ‚{\tiny $_{lb}$}‚

	  
	  \pstart \leavevmode% starting standard par
	एवं स‚म्ब‚न्धाद्य‚भिधानानि च शास्त्र‚ध‚र्मान‚प्र‚द‚र्श्य्\textbf{आत्य‚न्त‚प्र‚सिद्ध‚विष‚य‚स‚त्यार्थ‚{\tiny $_{lb}$}‚ताभिधान‚मात्रेणे}त्य‚त्य‚न्त‚प्र‚सिद्धो विष‚यो व‚ह्नेः शीताप‚नोद‚साम‚र्थ्यं त‚स्याभिधानं‚{\tiny $_{lb}$}‚ स‚त्य‚न्तेन स‚त्याभिधान‚मात्रेण । प्र‚ज्ञाप्र‚क‚र्षेणापि दुःखेनाव‚गाह्य‚त इति \textbf{प्र‚ज्ञाप्र‚क‚र्ष‚{\tiny $_{lb}$}‚दुर‚व‚ग्राहः । त}त एव \textbf{ग‚ह}न‚न्त\textbf{र्स्मिं}स्त‚थाभूतेपिविष‚{\tiny $_{४}$}‚येऽत्य‚न्त‚प‚रोक्षेपीति याव‚त् ।‚{\tiny $_{lb}$}‚ ‚{\tiny $_{lb}$}‚ \leavevmode\ledsidenote{\textenglish{613/s}}\textbf{निर‚त्य‚य‚तां} स‚त्यार्थ‚तां \textbf{साध‚यितुकामो} मी मां स को ब\textbf{न्ध‚कीम‚पि प्राग‚ल्भ्येन}‚{\tiny $_{lb}$}‚ धार्ष्ट्येन \textbf{विज‚य‚ते} [।] यादृशीञ्च ब‚न्ध‚कीम्विज‚य‚ते तां \textbf{क‚ञ्चि}दित्यादिना द‚र्श‚{\tiny $_{lb}$}‚य‚ति । \textbf{ब‚न्ध‚की} दुश्चारिणी । \textbf{स्व‚यं स्वामिना विप्र‚तिप‚त्तिस्थाने दृष्टे}ति विप्र‚ति‚{\tiny $_{lb}$}‚प‚त्त्य‚व‚स्थायान्दृष्ट्वा प‚र‚पुरुषेण स‚ङ्ग‚ता त्व‚मित्यु\textbf{पाल‚ब्धा} स‚ती । \textbf{सा तँ} स्वामिनं‚{\tiny $_{lb}$}‚ \textbf{प्र‚त्युवाच} । प्र‚त्यु {त्त... ... ।}‚{\tiny $_{५}$}‚ क‚थं प्र‚त्युवाचेत्याह । \textbf{प‚श्य‚तै}‚{\tiny $_{lb}$}‚त्यादि । पार्श्व‚स्थाः स्त्रियो मात इत्य‚नेनाम‚न्त्र्य‚न्ते [।] \textbf{मातः प‚श्य‚त पुरुष‚स्य}‚{\tiny $_{lb}$}‚ म‚दीय‚स्य स्वामिनो \textbf{वैप‚रीत्यं} । न‚नु प‚श्य‚तेति लोड्म‚ध्य‚म‚पुरुष‚ब‚हुव‚च‚नान्त‚मेत‚त् ।‚{\tiny $_{lb}$}‚ त‚त‚श्च मातृश‚ब्दाद‚पि ब‚हुव‚च‚न‚मेव युक्त‚म्मात‚र इति [।] त‚त्रैके प्र‚तिप‚न्ना मातृ‚{\tiny $_{lb}$}‚श‚ब्देनाम‚न्त्रितैक‚व‚च‚नान्तेन स‚मानार्थो मातः श‚ब्दोस्ति । वि{... ...}‚{\tiny $_{६}$}‚ स्व‚र‚{\tiny $_{lb}$}‚प्र‚तिरूप‚काश्च निपाता इत्य‚नेन न्यायेन । स चाव्य‚य‚त्वात् स‚र्वेषु व‚च‚नेषु तुल्य‚रूप‚{\tiny $_{lb}$}‚ इति ब‚हुव‚च‚नेनापि प‚श्य‚त श‚ब्देन स‚म्ब‚ध्य‚मानो मात‚रित्येव प्र‚युक्त इति । अन्ये‚{\tiny $_{lb}$}‚ तु प‚श्य‚त मात‚रः पुरुष‚स्येति प‚ठ‚न्ति । ध‚र्म‚स्य साध‚न‚भूता प‚त्नी ध‚र्म‚प‚त्नीति म‚ध्य‚{\tiny $_{lb}$}‚प‚द‚लोपी स‚मासः । \textbf{म‚यि ध‚र्म‚प‚त्न्यां प्र‚त्य‚य‚म‚कृत्वा आत्मीय‚योर्ज}ल‚बु{द्बुद...}‚{\tiny $_{७}$}‚‚{\tiny $_{lb}$}‚ द्धाञ्\textbf{ज‚ल‚बुद्बुद्यो}र्द्व‚योर्नेत्राभिधान‚योः क\textbf{रोति} प्र‚त्य‚य‚मिति प्र‚कृतं । नेत्र‚मित्य‚भि- \leavevmode\ledsidenote{\textenglish{216a/PSVTa}}‚{\tiny $_{lb}$}‚ धानं य‚योरिति विग्र‚हः । प‚र‚पुरुषेणासंग‚तेः कार‚ण‚माह । \textbf{तेने}त्यादि । ज‚रंश्चासौ‚{\tiny $_{lb}$}‚ काण‚श्चेति \textbf{ज‚र‚त्काणः} । वृद्ध‚काणेनेत्य‚र्थः । त‚त्र ज‚र‚द्ग्र‚ह‚णेन व‚योवैक‚ल्य‚मु‚{\tiny $_{lb}$}‚क्तं । प‚रं रूप‚स्थानं च‚क्षुरिति त‚द्वैक‚ल्यात् काण‚ग्र‚ह‚णेन वैरूप्यं । \textbf{ग्राम्य}ग्र‚ह‚णेन‚{\tiny $_{lb}$}‚ वैद‚ग्ध्यादिगुण‚{\tiny $_{१}$}‚वैक‚ल्यं । \textbf{काष्ठ‚हार‚क}ग्र‚ह‚णेन कृच्छ्र‚जीवित्वात् दारिद्र्य‚मुक्तं ।‚{\tiny $_{lb}$}‚ त‚देवं वृद्ध‚त्वादिगुण‚युक्तेन पुरुषेण स‚ङ्ग‚त्य‚र्थं \textbf{प्रार्थितापि} स‚ती । \textbf{न्}आह‚न्तेन‚{\tiny $_{lb}$}‚ स‚ह \textbf{स‚ङ‚ग‚ता} प्राक् । \textbf{रूप‚गुणानुरागेण} । रूपं प्रासादिक‚ता । गुणो वैद‚ग्ध्यादिको‚{\tiny $_{lb}$}‚ ध‚र्मः । रूप‚गुण‚योर‚नुरागोभिलाष‚स्तेन हेतुना । \textbf{मंत्रिमुख्य‚दार‚कं} म‚न्त्रिप्र‚धान‚{\tiny $_{lb}$}‚दार‚कं युवानं पुत्रं । म‚न्त्रिमु‚{\tiny $_{२}$}‚ख्य‚श्चासौ दार‚क‚श्चेति विग्र‚हः । त‚मेवंभूतं‚{\tiny $_{lb}$}‚ दार‚कं \textbf{काम‚येह‚मिति} क‚थ‚मिदं स‚म्भाव्य‚ते । त‚त्र वृद्धादिदोष‚च‚तुष्ट‚य‚वैप‚रीत्येन‚{\tiny $_{lb}$}‚ म‚न्त्रिमुख्य‚दार‚के गुण‚च‚तुष्ट‚य‚मुक्तं । रूप‚ग्र‚ह‚णेन प्रासादिक‚त्वं गुण‚ग्र‚ह‚णेन वैद‚{\tiny $_{lb}$}‚ग्ध्यादिः । म‚न्त्रिमुख्य‚ग्र‚ह‚णेनैश्व‚र्यं । दार‚क‚ग्र‚ह‚णेन व‚योगुणः । एत‚च्च ब‚न्ध‚क्या‚{\tiny $_{lb}$}‚ धार्ष्ट्यात् प्रेरित‚मेव केव‚ल‚स्व‚म्व‚चनं न‚{\tiny $_{३}$}‚ तु युक्तियुक्तं । रूपादीनामेव काम‚हेतु‚{\tiny $_{lb}$}‚‚{\tiny $_{lb}$}‚ \leavevmode\ledsidenote{\textenglish{614/s}}त्वान्न तु वार्द्ध‚क्यादीनां ।
	{\color{gray}{\rmlatinfont\textsuperscript{§~\theparCount}}}
	\pend% ending standard par
      ‚{\tiny $_{lb}$}‚

	  
	  \pstart \leavevmode% starting standard par
	\textbf{एवंजातीय‚क}मित्यादिना दृष्टान्तार्थं दार्ष्टान्तिके योज‚य‚ति । \textbf{एवंजातीय‚{\tiny $_{lb}$}‚क‚मि}ति ब‚न्ध‚कीप्र‚तिव‚च‚न‚तुल्यं । \textbf{एत‚द‚त्य‚न्त‚प‚रोक्षेर्थे} वेद‚स्या\textbf{विस‚म्वादानुमानं} [।]‚{\tiny $_{lb}$}‚ किम्भूत‚स्य वेद‚स्य श‚क्य‚विचारे व‚स्तुनि \textbf{दृष्ट‚प्र‚माण‚विरोध‚स्य} । दृष्टः प्र‚मा‚{\tiny $_{lb}$}‚ण‚विरोधोस्येति विग्र‚हः । व‚ह्निशीत‚नोद‚न‚{\tiny $_{४}$}‚दृष्टान्तेनानुमानं । व‚ह्नेः \textbf{शीत‚प्र‚ती‚{\tiny $_{lb}$}‚कार‚व‚च‚नेन} । य‚थाग्निर्हिम‚स्य भेष‚ज‚मिति वाक्य‚म‚विस‚म्वादि । त‚थान्य‚द‚पि वेद‚{\tiny $_{lb}$}‚वाक्य‚म‚विस‚म्वादीति ।
	{\color{gray}{\rmlatinfont\textsuperscript{§~\theparCount}}}
	\pend% ending standard par
      ‚{\tiny $_{lb}$}‚

	  
	  \pstart \leavevmode% starting standard par
	त‚त्र ध‚र्म‚प‚त्नीस्थानीयो वेदः । विप्र‚तिप‚त्तितुल्य‚न्नित्य‚स्य पुंसः क‚र्त्तृत्वा‚{\tiny $_{lb}$}‚द्य‚भिधानं । नेत्र‚तुल्ये प्र‚त्य‚क्षानुमाने । न च दृष्ट‚व्य‚भिचारायाः प‚त्न्या व‚च‚नं ग‚री‚{\tiny $_{lb}$}‚य‚स्त‚स्य पुरुष‚स्य येन{... ...}‚{\tiny $_{५}$}‚ य‚म्विप्र‚तिप‚र्त्तिं दृष्ट्वापि स्व‚द‚र्श‚न‚म‚{\tiny $_{lb}$}‚प्र‚माणीकृत्य त‚स्या व‚च‚नं य‚थार्थं कुर्यात् । एव‚म्वेदोक्तार्थ‚बाध‚क‚योः प्र‚माण‚योर‚{\tiny $_{lb}$}‚प्र‚माण्यं कृत्वा न वेद‚स्य प‚त्नीस्थानीय‚स्य दृष्ट‚व्य‚भिचार‚स्य व‚च‚नाद‚त्य‚न्त‚प‚रोक्षं‚{\tiny $_{lb}$}‚ प्र‚तिप‚द्येम‚हीति । \href{http://sarit.indology.info/?cref=pv.3.333}{३३६}
	{\color{gray}{\rmlatinfont\textsuperscript{§~\theparCount}}}
	\pend% ending standard par
      ‚{\tiny $_{lb}$}‚

	  
	  \pstart \leavevmode% starting standard par
	अत्रैव दोषान्त‚र‚माह । \textbf{सिध्ये}दित्यादि । \textbf{एव‚मि}ति य‚था दृष्टैक‚स‚त्याभि‚{\tiny $_{lb}$}‚धान‚मात्रेण \textbf{य‚दि} स‚र्वो वेदः । {... ...प्र}‚{\tiny $_{६}$}‚ \textbf{माणं सिध्ये}त्त‚दा स‚र्वः पुरुष‚स्स‚र्व‚{\tiny $_{lb}$}‚त्राथ प्र‚माणं स्यात् । य‚स्मा\textbf{न्न हि पुरुषे ब‚हुभाषिण्येकं} व‚च‚नं \textbf{स‚त्यार्थं नास्ति} [।]‚{\tiny $_{lb}$}‚ किन्त्व‚स्त्येव ।
	{\color{gray}{\rmlatinfont\textsuperscript{§~\theparCount}}}
	\pend% ending standard par
      ‚{\tiny $_{lb}$}‚

	  
	  \pstart \leavevmode% starting standard par
	\textbf{य‚थे}त्यादिना व्याच‚ष्टे । \textbf{य‚थेद‚म}त्य‚न्त‚म‚भिधायित्वं । एकान्तेन स‚त्य‚वादित्व‚{\tiny $_{lb}$}‚\textbf{म‚तिदुष्क‚रं} । अत्य‚न्त‚दुःखेन क्रिय‚त इति कृत्वा । त‚था\textbf{त्य‚न्}तं । \textbf{स‚त्याभिधान}म‚ति‚{\tiny $_{lb}$}‚\leavevmode\ledsidenote{\textenglish{216b/PSVTa}} दुष्क‚र\textbf{न्त‚त्रै}वंस्थिते न्याये । \textbf{एक‚स्य{... ...व‚च}‚{\tiny $_{७}$}‚ न‚स्य क‚थंचिदि}ति का क ता‚{\tiny $_{lb}$}‚ ली य न्यायेनापि य‚स्स\textbf{म्वाद}स्स‚त्यार्थ‚त्व\textbf{न्तेन} हेतुना । त‚स्माद‚विस‚म्वाद‚काद् व‚च‚{\tiny $_{lb}$}‚ना\textbf{द‚विशिष्ट‚स्य त‚द्व‚च‚न‚राशे}रिति य‚स्य त‚देक‚म्व‚च‚न‚म‚विस‚म्वादि दृष्ट‚न्त‚स्य पुरुष‚स्य‚{\tiny $_{lb}$}‚ व‚च‚न‚राशेः । त\textbf{थाभावे}ऽविस‚म्वादित्वेऽभ्युप‚ग‚म्य‚माने । \textbf{न क‚श्चित् पुरुषो नाप्तः‚{\tiny $_{lb}$}‚ स्या}त् । किन्तु स‚र्व एवाप्तः स्यात् । न चैवं । त‚स्मान्नैक‚देशाविस‚म्वा{दात्... ...प्रामा}‚{\tiny $_{१}$}‚ ण्य‚मिति ।
	{\color{gray}{\rmlatinfont\textsuperscript{§~\theparCount}}}
	\pend% ending standard par
      ‚{\tiny $_{lb}$}‚‚{\tiny $_{lb}$}‚\textsuperscript{\textenglish{615/s}}

	  
	  \pstart \leavevmode% starting standard par
	\textbf{अपि च} यो नाम क‚श्चित् क‚स्य‚चिद् ग‚म‚कः स त‚त्स्व‚भाव‚स्त‚ज्ज‚न्यो वा स‚न्‚{\tiny $_{lb}$}‚ ग‚म‚येन्नान्य‚था । \textbf{न चायं} ध्व‚निर्वाच्य‚त्वेनाभिम‚तानां व‚स्तूनां \textbf{स्व‚भावः} [।] किं‚{\tiny $_{lb}$}‚ कार‚णं [।] य‚स्माद् व‚क्त‚रि ध्व‚निः स्थितो न ह्य‚न्य‚स्व‚भावोन्य‚त्र व‚र्त्त‚ते । \textbf{व‚स्त‚नां‚{\tiny $_{lb}$}‚ कार्य वा} [।] नायं ध्व‚निः [।] किं कार‚णं [।] य‚स्माद् \textbf{व‚क्त‚रि} म‚ति\textbf{ध्व‚नि}र्भ‚व‚ति ।‚{\tiny $_{lb}$}‚ तेनाय‚म‚र्थः । [।] य‚स्माद् व‚क्तुरिच्छामात्र‚प्र‚तिब‚द्धो न बाह्य‚व‚स्तु{... ...}‚{\tiny $_{२}$}‚‚{\tiny $_{lb}$}‚ इत्य‚र्थः । \textbf{न च त‚द्व्य‚तिरिक्त‚स्ये}ति स्व‚भाव‚कार्य‚व्य‚तिरिक्त‚स्यार्थ‚स्या\textbf{व्य‚भि‚{\tiny $_{lb}$}‚ चारिता विद्य‚ते} ।
	{\color{gray}{\rmlatinfont\textsuperscript{§~\theparCount}}}
	\pend% ending standard par
      ‚{\tiny $_{lb}$}‚

	  
	  \pstart \leavevmode% starting standard par
	\textbf{ने}त्यादिना व्याच‚ष्टे । ताव‚च्छ‚ब्दः क्र‚मे । एत‚द् व‚क्तृस्थ‚म्व\textbf{च‚नं न ताव‚द्‚{\tiny $_{lb}$}‚ वाच्याना}म‚र्थानां \textbf{स्व‚भावः । नाप्येषां वाच्यानां कार्यं} । किङ् कार‚णं [।]‚{\tiny $_{lb}$}‚ \textbf{त‚द‚भावेपि} । व‚स्तूनाम‚भावेपि \textbf{विव‚क्षामात्रेण भावा}दुत्प‚त्तेर्\textbf{न च} कार्य‚स्व‚भावा‚{\tiny $_{lb}$}‚भ्या\textbf{म‚न्यः क‚श्चित्} क‚स्य‚चिद\textbf{व्य‚भिचारी} हेतुर\textbf{स्ति} । वा{... ...}‚{\tiny $_{३}$}‚ \textbf{व्य‚भिचारे}‚{\tiny $_{lb}$}‚ \textbf{च स}ति श‚ब्द‚स्य \textbf{त‚तोन्य‚थापी}ति । त‚स्माद् बाह्यार्थाद‚न्य‚थापि वाह्यार्थ‚भावेपीत्य‚र्थः ।‚{\tiny $_{lb}$}‚ \textbf{त‚त्स‚म्भ‚वात्} त‚स्य श‚ब्द‚स्य स‚म्भ‚वात् कार‚णात् । \textbf{त‚स्य} श‚ब्द‚स्य \textbf{भावात् त‚त्प्र‚ती‚{\tiny $_{lb}$}‚ति}र्वाह्यार्थ‚प्र‚तीति\textbf{र‚युक्ता} ।
	{\color{gray}{\rmlatinfont\textsuperscript{§~\theparCount}}}
	\pend% ending standard par
      ‚{\tiny $_{lb}$}‚

	  
	  \pstart \leavevmode% starting standard par
	स्यादेत‚द् [।] \textbf{य‚द्य}पि वाच्याद् व‚स्तुनोर्थान्त‚रं व‚च‚न‚न्त‚थापि त‚स्य वाच्य‚स्य‚{\tiny $_{lb}$}‚ \textbf{कार्य‚मेवे}ति ग‚म‚क‚मेव । य‚स्माद् \textbf{वाच‚कानां} श‚ब्दानां या \textbf{प्र‚वृत्ति}रुत्प‚त्तिर‚भि‚{\tiny $_{lb}$}‚व्य‚क्तिर्वा सा वाच्य \textbf{दृ}‚{\tiny $_{४}$}‚\textbf{ष्टिकृता} । वाच्य‚स्यार्थ‚स्य य‚द्द‚र्श‚न‚न्त‚त्कृता । \textbf{वाच्ये हि}‚{\tiny $_{lb}$}‚ स‚ति त‚द्द‚र्श‚न‚न्त‚द्द‚र्श‚नं [।] त‚द्द‚र्श‚नात् त‚द्विव‚क्षा । विव‚क्षातो व‚च‚न‚मिति पार‚म्प‚र्येण‚{\tiny $_{lb}$}‚ व‚च‚न‚म‚र्थ\textbf{कार्य}मिति पूर्व‚प‚क्षः ।
	{\color{gray}{\rmlatinfont\textsuperscript{§~\theparCount}}}
	\pend% ending standard par
      ‚{\tiny $_{lb}$}‚

	  
	  \pstart \leavevmode% starting standard par
	\textbf{स्यादेत}दित्यादिना व्याच‚ष्टे । त‚स्माद् \textbf{वाच‚क‚स्य} श‚ब्द‚स्य \textbf{वाच्य‚द‚र्श‚नेन प्र‚वृत्तेः}
	{\color{gray}{\rmlatinfont\textsuperscript{§~\theparCount}}}
	\pend% ending standard par
      ‚{\tiny $_{lb}$}‚‚{\tiny $_{lb}$}‚‚{\tiny $_{lb}$}‚\textsuperscript{\textenglish{616/s}}

	  
	  \pstart \leavevmode% starting standard par
	\textbf{एव}मित्यादिना प्र‚तिविध‚त्ते । \textbf{एवं स‚तीति} य‚द्य‚र्थ‚व‚शेनैव व‚च‚न‚प्र‚वृत्तेस्स‚र्वा‚{\tiny $_{lb}$}‚ व‚च‚न‚प्र‚वृत्तिः स‚त्यार्था{... ...}‚{\tiny $_{५}$}‚दैक‚त्राभिधेये आग‚म‚भेदेन \textbf{प‚र‚स्प‚रं विरुद्धार्था सा}‚{\tiny $_{lb}$}‚ व‚च‚न‚वृत्तिः \textbf{क‚थ‚म्भ‚वेत्} । नैव भ‚वेत् । स‚र्व‚प्र‚वादेष्वेकार्थैव भ‚वेदिति याव‚त् ।
	{\color{gray}{\rmlatinfont\textsuperscript{§~\theparCount}}}
	\pend% ending standard par
      ‚{\tiny $_{lb}$}‚

	  
	  \pstart \leavevmode% starting standard par
	\textbf{य‚दी}त्यादिना व्याच‚ष्टे । \textbf{य‚द्येष निय‚मो} वाच्य‚म्व‚स्त्व‚न्त‚रेण \textbf{वाच्येन व‚स्तुना‚{\tiny $_{lb}$}‚ विना श‚ब्दो न प्र‚व‚र्त्त‚त इति} त‚दा प‚र‚स्प‚र‚विरुद्धार्थाभिधानाद् \textbf{भिन्नेषु प्र‚वादेषु}‚{\tiny $_{lb}$}‚ सिद्धान्तेष्\textbf{वेक‚त्र व‚स्तुनि} । नित्यानित्या{दे... ... ...वि}‚{\tiny $_{६}$}‚रुद्ध‚स्य \textbf{स्व‚भाव‚स्योप‚{\tiny $_{lb}$}‚संहारेण} स‚मारोपेण व‚च‚न\textbf{वृत्तिर्न स्यात्} । य‚तो \textbf{न ह्य‚स्त्य‚यं स‚म्भ‚वो य‚देकः श‚ब्दो‚{\tiny $_{lb}$}‚ निःप‚र्याय‚मि}ति प्र‚कारान्त‚रेण विनेत्य‚र्थः । \textbf{नित्य‚श्च स्याद‚नित्य‚श्चेति} । श‚ब्द‚ग्र‚ह‚ण‚{\tiny $_{lb}$}‚मुप‚ल‚क्ष‚णार्थं [।] तेन घ‚टादिर‚पि निःप‚र्याय‚न्नित्य‚श्च । नित्य‚श्च न स‚म्भ‚व‚त्येव ।‚{\tiny $_{lb}$}‚ \leavevmode\ledsidenote{\textenglish{217a/PSVTa}} भ‚व‚ति च क‚स्य‚चित् प्र‚वादे नित्यः श‚ब्द इति {व... ...}‚{\tiny $_{७}$}‚तिर‚न्य‚स्यानित्य इत्येवं‚{\tiny $_{lb}$}‚ सात्म‚को निरात्म‚क इत्यादि ।
	{\color{gray}{\rmlatinfont\textsuperscript{§~\theparCount}}}
	\pend% ending standard par
      ‚{\tiny $_{lb}$}‚

	  
	  \pstart \leavevmode% starting standard par
	त‚स्मान्नास्ति श‚ब्दानां बाह्यैर‚र्थैस्स‚ह स‚म्ब‚न्धः ।
	{\color{gray}{\rmlatinfont\textsuperscript{§~\theparCount}}}
	\pend% ending standard par
      ‚{\tiny $_{lb}$}‚

	  
	  \pstart \leavevmode% starting standard par
	य‚त एव\textbf{न्तेन} कार‚णेन \textbf{प्र‚तिप‚त्तुः} पुरुष‚स्य । \textbf{व‚स्तुभिः स‚हाग‚मा नान्त‚रीय‚का}‚{\tiny $_{lb}$}‚ अविनाभाविनो न क‚थंचित्प्र‚तिप‚त्तुः सिध्य‚न्ति \textbf{त‚त्कुस्तेभ्य} आग‚मेभ्यो व‚स्त्व‚नान्त‚{\tiny $_{lb}$}‚रीय‚केभ्यो\textbf{र्थ‚निश्च‚यो} न वा निश्च‚यः । \textbf{आग‚म‚स्य} प्र‚माणां‚{\tiny $_{१}$}‚ न स‚र्वोन्वेष‚ते किन्त्\textbf{व‚ज्ञो‚{\tiny $_{lb}$}‚ ज‚नः । स‚म‚न्वेष‚ते} किम‚र्थं [।] \textbf{त‚दाद‚र्शितार्थ‚प्र‚तिप‚त‚ये} । तेनाग‚मेनोप‚द‚र्शित‚स्या‚{\tiny $_{lb}$}‚र्थ‚स्य \textbf{प्र‚ति}प‚त्त्य‚र्थं । किङ्कार‚ण‚म‚ज्ञ एव स‚म‚न्वेष‚ते नान्य इत्याह । \textbf{स‚म}धिग‚तं‚{\tiny $_{lb}$}‚ \textbf{याथात}थ्यं प‚दार्थानाम‚विप‚रीतं रूपं यैस्तेषाम‚धिग‚त‚प‚र‚मार्थानां प‚रोप\textbf{देशान‚पेक्ष‚{\tiny $_{lb}$}‚णात्} । येनाप्य‚ज्ञेनान्वेष‚णीन्त‚स्य {... ...आ}‚{\tiny $_{२}$}‚ तीन्द्रिया गुणा य‚स्य पुरुष‚स्य‚{\tiny $_{lb}$}‚ \textbf{सोतीन्द्रिय‚गुणः} [।] प‚श्चात् पुरुष‚श‚ब्देन विशेष‚ण‚स‚मासः । त‚स्य \textbf{पुरुष}स्य विवे‚{\tiny $_{lb}$}‚  ‚{\tiny $_{lb}$}‚ ‚{\tiny $_{lb}$}‚ \leavevmode\ledsidenote{\textenglish{617/s}}\textbf{च‚ने}ऽयं स‚र्व‚ज्ञो नान्यो वा वित‚थाभिधायीत्येवं विभाग‚क्रियायाम\textbf{साम‚र्थ्यात्} । त‚त्कु‚{\tiny $_{lb}$}‚त‚स्त‚थाभूत‚पुरुष‚प्र‚णीतं व‚च‚न‚माग‚म‚त्वेन निश्चित्य प्र‚व‚र्त्तेत ।
	{\color{gray}{\rmlatinfont\textsuperscript{§~\theparCount}}}
	\pend% ending standard par
      ‚{\tiny $_{lb}$}‚

	  
	  \pstart \leavevmode% starting standard par
	न च स‚र्व‚स्य स‚र्वाणि व‚च‚नानि य‚थार्थं प्र‚व‚र्त्त‚मानानि प‚र{... ...}‚{\tiny $_{३}$}‚‚{\tiny $_{lb}$}‚ नां च \textbf{स‚मीहितो}भीष्टो यो\textbf{र्थ}स्त‚स्य \textbf{स‚त्ताम‚न्त‚रेणापि वृत्तिम्}प‚श्य‚तः प्र‚तिप‚त्तुः पुरु‚{\tiny $_{lb}$}‚ष‚स्य \textbf{भ‚वित‚व्य‚मेव वाचि शंक‚या । किमियं य‚थार्था} वाङ् \textbf{नेति वेति} । केषां \textbf{वाचि ।‚{\tiny $_{lb}$}‚ अदृष्ट‚व्य‚भिचार‚व‚च‚साम‚पि} । श‚क्य‚विचारे व‚स्तुन्य‚दृष्ट‚व्य‚भिचारं व‚चो येषान्ते‚{\tiny $_{lb}$}‚षाम‚पि वाचि । अपिश‚ब्दाद् दृष्ट‚व्य‚भिचार‚व‚च‚साम‚पि वाचि सुत‚रां शंक‚या‚{\tiny $_{lb}$}‚ {... ...}‚{\tiny $_{४}$}‚ येन शंका \textbf{तेन} कार‚णेन \textbf{न युक्त‚म‚नेना}ज्ञेन प्र‚तिप‚त्त्रा‚{\tiny $_{lb}$}‚ \textbf{क‚स्य‚चित्} पुरुष‚स्य व‚च‚नेन \textbf{किञ्चिद्} व‚स्तु \textbf{निश्चेतुं} । य‚त‚श्चानिश्च‚य‚स्\textbf{त‚स्मा‚{\tiny $_{lb}$}‚द‚स्य} प्र‚तिप‚त्तेस्त\textbf{न्निवृत्त्याप्या}ग‚म्निवृत्त्यापि प्र‚तिषेध्याभिम‚त‚स्य भाव‚स्या\textbf{भावो न‚{\tiny $_{lb}$}‚ प्र‚सिध्य‚ति} ।
	{\color{gray}{\rmlatinfont\textsuperscript{§~\theparCount}}}
	\pend% ending standard par
      ‚{\tiny $_{lb}$}‚

	  
	  \pstart \leavevmode% starting standard par
	\textbf{य‚दुक्त}मित्यादिना व्याच‚ष्टे । य‚दुक्तं प्राक् [।] \textbf{स‚र्व‚विष‚य‚त्वादाग‚म‚स्य‚{\tiny $_{lb}$}‚ स‚ति व‚स्तुन्य}वि{स‚म्वादेना... ...}‚{\tiny $_{५}$}‚ क‚र‚णात् । \textbf{त‚न्निवृत्तिल‚क्ष}णेत्याग‚म‚निवृ‚{\tiny $_{lb}$}‚त्तिल‚क्ष‚णा\textbf{नुप‚ल‚ब्धिः} स्व‚भावादिविप्र‚क‚र्षिणोप्य‚र्थ‚स्या\unclear{भा}वं \textbf{साध‚यिष्य}तीति । \textbf{त‚दि}‚{\tiny $_{lb}$}‚त्य‚भाव‚निश्चाय‚क‚त्व\textbf{म‚स्ये}त्याग‚म‚स्य नैवास्याग‚म‚स्य । \textbf{स‚र्व‚विष‚य‚त्व‚म}स्ति । पुरुषा‚{\tiny $_{lb}$}‚र्थास‚म्ब‚द्धानाम‚र्थानाम‚विष‚यीक‚र‚णात् ।
	{\color{gray}{\rmlatinfont\textsuperscript{§~\theparCount}}}
	\pend% ending standard par
      ‚{\tiny $_{lb}$}‚

	  
	  \pstart \leavevmode% starting standard par
	अभ्युप‚ग‚म्याप्युच्य‚ते । आग‚म‚स्य स‚र्व‚विष‚य‚त्वे {पि य‚दि... ... ...य‚द्य}‚{\tiny $_{lb}$}‚  ‚{\tiny $_{lb}$}‚ ‚{\tiny $_{lb}$}‚ \leavevmode\ledsidenote{\textenglish{618/s}}वृत्तिः स्यात्त‚दा \textbf{व‚स्त्व‚न्त‚रेणावृत्तौ} स‚त्यां स्यादाग‚म‚निवृत्तिल‚क्ष‚ण‚स्यानुप‚ल‚म्भ‚स्या‚{\tiny $_{lb}$}‚भाव‚निश्चाय‚क‚त्वं [।] त‚च्च नास्ति व‚स्त्व‚न्त‚रेणाप्याग‚म‚स्य वृत्तेः ।
	{\color{gray}{\rmlatinfont\textsuperscript{§~\theparCount}}}
	\pend% ending standard par
      ‚{\tiny $_{lb}$}‚

	  
	  \pstart \leavevmode% starting standard par
	एवं च स‚ति \textbf{त‚तो}र्थ‚व्य‚भिचारिण आग‚मात् \textbf{प्र‚तिप‚त्तुकाम‚स्य} पुंसो\textbf{भिम‚तार्था‚{\tiny $_{lb}$}‚सिद्धिरित्युक्तं} । तेन कार‚णेन विप्र‚कृष्टेष्व‚स‚न्निश्च‚य‚फ‚लेत्य‚स‚द्व्य‚व‚हार {विष ... ... ...}‚{\tiny $_{७}$}‚ न सिध्य‚ति ।
	{\color{gray}{\rmlatinfont\textsuperscript{§~\theparCount}}}
	\pend% ending standard par
      ‚{\tiny $_{lb}$}‚

	  
	  \pstart \leavevmode% starting standard par
	\textbf{त‚स्मान्न प्र‚माण‚त्र‚य‚निवृत्ताव‚पि} प्र‚त्य‚क्षानुमानाग‚म‚संज्ञित‚स्य प्र‚माण‚त्र‚य‚स्य‚{\tiny $_{lb}$}‚ निवृत्ताव‚पि देश‚काल‚स्व‚भाव\textbf{विप्र‚कृष्टेष्व‚र्थेष्व‚निश्च‚य} इत्युप‚संहारः ।
	{\color{gray}{\rmlatinfont\textsuperscript{§~\theparCount}}}
	\pend% ending standard par
      ‚{\tiny $_{lb}$}‚

	  
	  \pstart \leavevmode% starting standard par
	अयुक्ताभिधायित्वे दिङ्मात्र‚न्तीर्थिकानान्द‚र्श‚य‚न्नाह ।
	{\color{gray}{\rmlatinfont\textsuperscript{§~\theparCount}}}
	\pend% ending standard par
      ‚{\tiny $_{lb}$}‚

	  
	  \pstart \leavevmode% starting standard par
	१ \textbf{वेद} इत्यादि । इति श‚ब्दो वाद‚श‚ब्द‚श्चात्र व‚क्ष्य‚माण‚क‚स्स‚म्ब‚न्ध‚नीयः ।‚{\tiny $_{lb}$}‚ तेनाय‚म‚र्थः [।] \textbf{वेद‚प्रामा{ण्य... ... ...जाड्ये}‚{\tiny $_{१}$}‚} लिङ्गं ।
	{\color{gray}{\rmlatinfont\textsuperscript{§~\theparCount}}}
	\pend% ending standard par
      ‚{\tiny $_{lb}$}‚

	  
	  \pstart \leavevmode% starting standard par
	न‚नु गुणेन ष‚ष्ठीस‚मास‚प्र‚तिषेधाद् वेद‚स्य प्रामाण्य‚मिति भ‚वित‚व्यं ।
	{\color{gray}{\rmlatinfont\textsuperscript{§~\theparCount}}}
	\pend% ending standard par
      ‚{\tiny $_{lb}$}‚

	  
	  \pstart \leavevmode% starting standard par
	नायं दोषो लोक‚प्र‚सिद्धानाम्विशिष्टानामेव गुणानान्त‚त्र ग्र‚ह‚णात् । अस्य च‚{\tiny $_{lb}$}‚ लिङ्ग‚म‚धिक‚र‚णैताव‚त्व इत्यादिको निर्देशः ।
	{\color{gray}{\rmlatinfont\textsuperscript{§~\theparCount}}}
	\pend% ending standard par
      ‚{\tiny $_{lb}$}‚

	  
	  \pstart \leavevmode% starting standard par
	२ \textbf{क‚स्य‚चित्} नै या यि का देरीश्व‚र‚स्त‚त्त्वादीनां \textbf{क‚र्त्ते}त्य‚य‚म‚पि वादो‚{\tiny $_{lb}$}‚ जाड्ये लिङ्गं ।
	{\color{gray}{\rmlatinfont\textsuperscript{§~\theparCount}}}
	\pend% ending standard par
      ‚{\tiny $_{lb}$}‚

	  
	  \pstart \leavevmode% starting standard par
	३ तीर्थ\textbf{स्नाने ध‚र्मेच्छा} जाड्य‚लिङ्ग‚म‚प{र... ... ...}‚{\tiny $_{२}$}‚ मोहा‚{\tiny $_{lb}$}‚दिस्त‚ज्ज‚नितं च काय‚वाक्क‚र्म्म ध‚र्म‚स्त‚द्विप‚रीत‚ञ्च ज‚ल‚संश्लेष‚मात्र‚ल‚क्ष‚णं स्नान‚{\tiny $_{lb}$}‚मिति कुत‚स्त‚तो ध‚र्म‚प्राप्तिः । विस्त‚रेण निराकृत‚श्चाय‚न्तीर्थ‚स्नान‚वाद आचार्य‚{\tiny $_{lb}$}‚ व सु ब न्धु प्र‚भृत्तिभिरिति नेह प्र‚त‚न्य‚ते ।
	{\color{gray}{\rmlatinfont\textsuperscript{§~\theparCount}}}
	\pend% ending standard par
      ‚{\tiny $_{lb}$}‚

	  
	  \pstart \leavevmode% starting standard par
	४ शीलादिगुण‚वैक‚ल्येपि ब्राह्म‚णोह‚मिति \textbf{जातिवादेनाव‚लेपो} द‚र्पो‚{\tiny $_{lb}$}‚ जाड्य‚लिङ्ग‚म‚युक्त‚त्वात् । {त‚था हि प्र...... ...गृ}‚{\tiny $_{३}$}‚ हीत्वा जाति‚{\tiny $_{lb}$}‚वादाव‚लेपः स्यात् । ब्राह्म‚णेन पित्रा व्राह्म‚ण्या ग‚र्भे य उत्पाद‚स्तं वा स‚माश्रित्य ।‚{\tiny $_{lb}$}‚ त‚त्र व‚स्तुभूत‚सामान्य‚निषेधान्न पूर्वः प‚क्षः । नाप्युत्त‚रः ब्राह्म‚ण‚ब्राह्म‚णीश‚रीर‚योर‚{\tiny $_{lb}$}‚शुचिस्व‚भाव‚त्वेन श‚रीरान्त‚राद‚विशेषात् ।
	{\color{gray}{\rmlatinfont\textsuperscript{§~\theparCount}}}
	\pend% ending standard par
      ‚{\tiny $_{lb}$}‚‚{\tiny $_{lb}$}‚‚{\tiny $_{lb}$}‚\textsuperscript{\textenglish{619/s}}

	  
	  \pstart \leavevmode% starting standard par
	५ अन‚श‚नादिना श‚रीर‚पीड‚नं \textbf{स‚न्ताप}स्त‚स्या\textbf{र‚म्भो}नुष्ठानं \textbf{पाप‚हानाय} ।‚{\tiny $_{lb}$}‚ पाप‚स्य {... ... ...}‚{\tiny $_{४}$}‚ \textbf{जाड्ये} लिङ्गं । त‚था हि [।] स‚र्व‚स्य‚{\tiny $_{lb}$}‚ पाप‚स्य निदानं लोभ‚द्वेष‚मोहाः । तैश्चाविरुद्धः स‚न्तापार‚म्भ इति कुत‚स्तेन पाप‚{\tiny $_{lb}$}‚शुद्धिः ।
	{\color{gray}{\rmlatinfont\textsuperscript{§~\theparCount}}}
	\pend% ending standard par
      ‚{\tiny $_{lb}$}‚

	  
	  \pstart \leavevmode% starting standard par
	एतानि \textbf{पंच लिङ्गानि जाड्ये} । किंविशिष्टे जाड्ये । \textbf{ध्व‚स्त‚प्र‚ज्ञाने} । प्र‚माणा‚{\tiny $_{lb}$}‚व‚ल‚म्विज्ञानं प्र‚ज्ञान‚न्त‚द् ध्व‚स्तं य‚स्मिन् जाड्ये त‚त्त‚थोक्तं । न्यायानुसारिज्ञान‚{\tiny $_{lb}$}‚ \textbf{र‚हित इत्य‚र्थः} । य‚थास्थ‚ल‚मुपादानात् प‚ञ्चेत्युक्त‚म{... ... ...}‚{\tiny $_{५}$}‚ लिङ्गा‚{\tiny $_{lb}$}‚\textbf{नीति} ॥ ० ॥
	{\color{gray}{\rmlatinfont\textsuperscript{§~\theparCount}}}
	\pend% ending standard par
      ‚{\tiny $_{lb}$}‚
	    
	    \stanza[\smallbreak]
	  भ‚ट्टोद्योत‚क‚रादिषु प्र‚विच‚यो येषां म‚हान् विद्य‚ते&‚{\tiny $_{lb}$}‚स‚द्व‚स्त्वाक‚र‚ध‚र्म‚कीर्तिज‚ल‚धेर‚न्त‚र्निम‚ग्न‚म्म‚नः ।&‚{\tiny $_{lb}$}‚पौर्वाप‚र्य‚विम‚र्षिणी स्मृतिर‚लं प्र‚ज्ञापि चोत्क‚र्षिणी&‚{\tiny $_{lb}$}‚य‚त्किञ्चिद् ग‚दिन्त‚द‚त्र निपुणैस्तैरेव विज्ञास्य‚ते ॥&‚{\tiny $_{lb}$}‚अर्थोत्खात‚प‚र‚म्प‚रासु म‚ह‚ती वाचः प्र‚स‚न्नात्म‚ता&‚{\tiny $_{lb}$}‚य‚स्याम‚न्य‚सुभाषिताऽ{... ... ... ...}‚{\tiny $_{६}$}‚तिः ।&‚{\tiny $_{lb}$}‚द‚र्पाध्मात\edtext{}{\lemma{र्पाध्मात}\Bfootnote{?}}स‚म‚स्त‚तीर्थिक‚म‚त‚ध्वंस‚श्च संवेद्यते&‚{\tiny $_{lb}$}‚टीकेयं सुविल‚क्ष‚णोदित‚धियामाव‚र्ज‚नीया क‚थ‚मिति ॥ ० ॥\&[\smallbreak]
	  
	  
	  ‚{\tiny $_{lb}$}‚

	  
	  \pstart \leavevmode% starting standard par
	क ण् र्ण क गो मिविर‚चितायाम्प्र‚मा{ण‚वार्त्तिक‚वृत्ति... ... ... ... ... ...प्र‚माण... ... ...} ॥ ० ॥
	{\color{gray}{\rmlatinfont\textsuperscript{§~\theparCount}}}
	\pend% ending standard par
      
	    
	    \endnumbering% ending numbering from div
	    
	  % running endDocumentHook
     \backmatter 
	 \chapter{The TEI Header}
	 \begin{minted}[fontfamily=rmfamily,fontsize=\footnotesize,breaklines=true]{xml}
       <teiHeader xmlns="http://www.tei-c.org/ns/1.0" xml:lang="en">
   <fileDesc>
      <titleStmt>
         <title type="main" subtype="commentary">Pramāṇavārttikasvavṛttiṭīkā</title>
         <author role="base author">Dharmakīrti</author>
         <author role="commentator">Karṇakagomin</author>
         <funder>Deutsche Forschungsgemeinschaft</funder>
         <funder>The National Endowment for the Humanities</funder>
         <principal>
	           <persName>Birgit Kellner</persName>
	        </principal>
         <respStmt>
            <resp>data entry by</resp>
            <name key="name aurorachana">Aurorachana, Auroville</name>
         </respStmt>
         <respStmt xml:id="sarit-encoder-pvsvt">
            <resp>prepared for SARIT by</resp>
            <persName key="name person lo">Liudmila Olalde</persName>
         </respStmt>
      </titleStmt>
      <editionStmt>
         <p> </p>
      </editionStmt>
      <publicationStmt>
         <publisher>SARIT: Search and Retrieval of Indic Texts. DFG/NEH Project (NEH-No.
	HG5004113), 2013-2016 </publisher>
         <idno>2015-08-06</idno>
         <availability status="restricted">
            <p>Copyright Notice:</p>
            <p>Copyright 2015-2016 SARIT</p>
            <licence> 
	              <p>Distributed under a <ref target="https://creativecommons.org/licenses/by-sa/4.0/">Creative Commons Attribution-ShareAlike 4.0 International licence.</ref> Under this licence, you are free to:</p>
	              <list>
                  <item>Share — copy and redistribute the material in any medium or format.</item>
                  <item>Adapt — remix, transform, and build upon the material for any purpose, even commercially.</item>
               </list>
	              <p>The licensor cannot revoke these freedoms as long as you follow the license terms.</p>
	              <p>Under the following terms:</p>
	              <list>
                  <item>Attribution — You must give appropriate credit, provide a link to the license, and indicate if changes were made. You may do so in any reasonable manner, but not in any way that suggests the licensor endorses you or your use.</item>
                  <item>ShareAlike — If you remix, transform, or build upon the material, you must distribute your contributions under the same license as the original.</item>
               </list>
	              <p>More information and fuller details of this license are given on the Creative Commons website.</p>
	           </licence>
            <p>SARIT assumes no responsibility for unauthorised use that infringes the rights of any copyright owners, known or unknown.</p>
         </availability>
         <date>2015</date>
      </publicationStmt>
      <sourceDesc>
         <bibl xml:id="pvsvt-sankrtyayana-book">
	           <title type="main">Ācāryadharmakīrteḥ Pramāṇavārttikam (svārthānumānaparicchedaḥ) svopajñavṛttyā, Karṇakagomiviracitayā taṭṭīkayā ca sahitam</title>
	           <author>Dharmakīrti</author>
	           <author>Karṇakagomin</author>
	           <editor xml:id="ed-rs">Rāhula Sāṅkṛtyāyana</editor>
	           <publisher>Kitāb Mahal</publisher>
	           <pubPlace>Ilāhābād</pubPlace>
	           <date>1943</date>
	           <note>Sāṅkṛtyāyana's edition is based on the manuscript PSVTa. He restored three-fourth of the text from its Tibetan translation. The manuscripts he consulted are described below (sigla and descriptions are take from Sāṅkṛtyāyana's foreword).</note>
	        </bibl>
         <listWit>
            <witness xml:id="PSVTa">
	              <msDesc>
                  <msIdentifier>
                     <idno>PSVTa</idno>
                  </msIdentifier>
                  <msContents>
                     <msItem>
                        <author>Karṇakagomin</author>
                        <title>Pramāṇavārttikasvavṛttiṭīkā</title>
                     </msItem>
                  </msContents>
                  <physDesc>
                     <objectDesc>
                        <p>217 leaves (each side 7 lines) in Māgadhī script. Only leaves 12 and 37 are missing and there are a few lacunae.</p>
                     </objectDesc>
                  </physDesc>
                  <history>
                     <p>Manuscript discovered by Sāṅkṛṭyāyana in  the Sa-skya Chhag-pe-lha-khang.</p>
                  </history>
               </msDesc>
	           </witness>
            <witness xml:id="PSVTb">
	              <msDesc>
                  <msIdentifier>
                     <idno>PSVTb</idno>
                  </msIdentifier>
                  <msContents>
                     <msItem>
                        <author>Karṇakagomin</author>
                        <title>Pramāṇavārttikasvavṛttiṭīkā</title>
                     </msItem>
                  </msContents>
                  <physDesc>
                     <objectDesc>
                        <p>Only 6 leaves corresponding to PVSVTa:
	      <list>
                              <item>151a6-153a4 (इत्यादि-योग्य)</item>
                              <item>153a4-154a3 (अर्य&#130;विशेष-श्लेष&#130;येत्)</item>
                              <item>154a3-155a2 (य&#130;मित्यादि-सामान्य)</item>
                              <item>158a5-159a1 नानुमेयः-त्येतिज्वाला)</item>
                              <item>159a2-169b6 (याः स&#130;काशात्-अत्रापि प्र)</item>
                              <item>193a3-194a2 (न भ&#130;वेदित्यादि-द&#130;र्शिता एत)</item>
                           </list>
                        </p>
                     </objectDesc>
                  </physDesc>
                  <history>
                     <p>Manuscript discovered by Sāṅkṛṭyāyana in the Sa-skya Chhag-pe-lha-khang.</p>
                  </history>
               </msDesc>
	           </witness>
            <witness xml:id="PSVTc">
	              <msDesc>
                  <msIdentifier>
                     <idno>PSVTc</idno>
                  </msIdentifier>
                  <msContents>
                     <msItem>
                        <author>Karṇakagomin</author>
                        <title>Pramāṇavārttikasvavṛttiṭīkā</title>
                     </msItem>
                  </msContents>
                  <physDesc>
                     <objectDesc>
                        <p>24 leaves.</p>
                     </objectDesc>
                  </physDesc>
                  <history>
                     <p>Manuscript placed at Sāṅkṛṭyāyana's disposal by the Rājaguru Pandit Hemarāja Śarman (Nepal).</p>
                  </history>
               </msDesc>
	           </witness>
            <witness xml:id="pvb-B">
	              <msDesc>
                  <msIdentifier>
                     <idno>B</idno>
                  </msIdentifier>
                  <msContents>
                     <msItem>
                        <author>Prajñākaragupta</author>
                        <title>Pramāṇavārttikabhāṣya</title>
                     </msItem>
                  </msContents>
                  <physDesc>
                     <objectDesc>
                        <p>Complete.</p>
                     </objectDesc>
                  </physDesc>
                  <history>
                     <p>Manuscript discovered by Sāṅkṛṭyāyana in  the Sa-skya Chhag-pe-lha-khang.</p>
                  </history>
               </msDesc>
	           </witness>
         </listWit>
      </sourceDesc>
   </fileDesc>
   <encodingDesc>
      <p>Only the commentary was digitized. The base-text, which is included in the printed edition, was not keyboarded.</p>
      <p>The front was supplied by SARIT.</p>
      <p>
         <list>
            <item>Line breaks: In the source file, there were two types of line breaks: returns (and possible surrounding space) and hyphens+returns. These were replaced with lb-elements. I didn't check whether the source was consequent in this respect. The ed-attribute "s" refers to Sāṅkṛtyāyana's edition<ref sameAs="#pvsvt-sankrtyayana-book"/>.</item>
            <item>The folio numbers on the margins were encoded as pb-elements. The ed-attribute "PSVTa" refers to the <ref target="#PSVTa">manuscript used by Sāṅkṛtyāyana</ref>. The line numbers in the manuscript were encoded as lb-elements with the ed-attribute "PSVTa".</item>
            <item>Round and square brackets were often used indistinctively in the printed edition and were replaced by SARIT with the following TEI-elements:
	<list>
                  <item>Bracketed references to other works were enclosed in &lt;ref cRef=""&gt;. The attribute cert="unknown" indicates that cRef was not checked by the encoder; whereas  cert="high" indicates that the value of the cRef was checked by the encoder.</item>
                  <item>Bracketed text preceded by a question mark in the print edition was enclosed in &lt;note type="correction"&gt;. The question mark was kept.</item>
                  <item>Bracketed text followed by a question mark in the print edition was enclosed in &lt;supplied&gt;.The question mark was kept.</item>
                  <item>Bracketed text followed by a footnote indicating that the text was written on the margin of the manuscript was encoded as &lt;add place="margin"&gt;.</item>
                  <item>Bracketed text with preceding or following suspension points ("...") was encoded as &lt;supplied reason="gap"&gt;.</item>
                  <item>Question marks enclosed in round brackets were encoded as: &lt;note type="uncertain" resp="#ed-rs #sarit-encoder-pvsvt" rend="brackets"&gt;.</item>
                  <item>Punctuation marks enclosed in brackets were enclosed in &lt;supplied resp="#ed-rs #sarit-encoder-pvsvt"&gt;.</item>
                  <item>All other round brackets (227 occurrences) were encoded as &lt;hi rend="brackets"&gt;.</item>
                  <item>All other square brackets (19 occurrences) were encoded as &lt;hi rend="squarebrackets"&gt;.</item>
                  <item>All suspension points ("...") were enclosed in &lt;supplied reason="gap"&gt;.</item>
               </list>
	           </item>
            <item>Bold characters were enclosed in &lt;hi rend="bold"&gt;</item>
            <item>The footnotes were encoded as note-elements with their corresponding n-attribute. The footnotes listed below are missing in the printed edition. Since the footnote references appear in the book, they were encoded as note-elements and contain only the footnote number:
	<list>
                  <item>30-3</item>
                  <item>403-2</item>
                  <item>425-4</item>
               </list>
	           </item>
            <item>Characters that were not readable in the printed edition available to SARIT were enclosed &lt;unclear reason="illegible"&gt;.</item>
            <item>The text is structured in 1  div-level.</item>
         </list>
      </p>
      <p>Abbreviations used in the attributes ed, cRef and xml:id's in this file: <!-- this is a provisory list and has to be replaced by a refsDecl -->
      <list ana="abbreviations">
            <item>ak = Abhidharmakośa</item>
            <item>nbh = Nyāyabhāṣya; verse numbers correspond to the <ref target="https://www.worldcat.org/title/srigautamamunipranitanyayasutrani-va&#130;tsya&#130;na&#130;mu&#130;ni&#130;krta&#130;bha&#130;sya&#130;vi&#130;sva&#130;na&#130;tha&#130;bha&#130;tta&#130;ca&#130;rya&#130;krta&#130;vrtti&#130;sa&#130;meta&#130;ni/oclc/644135949">1922 edition</ref>
            </item>
            <item>nyāma = Nyāyamañjarī </item>
            <item>nv = Uddyotakara's Nyāyavārttika </item>
            <item>nsū = Nyāyasūtra; verse numbers correspond to the <ref target="https://www.worldcat.org/title/srigautamamunipranitanyayasutrani-va&#130;tsya&#130;na&#130;mu&#130;ni&#130;krta&#130;bha&#130;sya&#130;vi&#130;sva&#130;na&#130;tha&#130;bha&#130;tta&#130;ca&#130;rya&#130;krta&#130;vrtti&#130;sa&#130;meta&#130;ni/oclc/644135949">1922 edition</ref> 
            </item>
            <item>Pā = Pāṇini's Aṣṭādhyāyī</item>
            <item>ps = Dignāga's Pramāṇasamuccaya</item>
            <item>pv = Dharmakīrti's Pramāṇavārttika; the verse numbers correspond to <ref target="http://east.uni-hd.de/buddh/ind/36/120/193/">Sāṅkṛtyāyana's edition</ref>.</item>
            <item>pv-pandey = Dharmakīrti's Pramāṇavārttika; the verse numbers correspond to <ref target="http://east.uni-hd.de/buddh/ind/7/16/658/">Pandey's edition</ref>.</item>
            <item>vk-mbh = Vyākaraṇa-mahābhāṣya</item>
            <item>vsū = Kaṇāda's Vaiśeṣikasūtra; the verse numbers correspond to: <bibl>Jambūvijaya, ed. Vaiśeṣikasūtra of Kaṇāda: with the commentary of Candrānanda. Baroda: Oriental Institiute, 1982.</bibl>
            </item>
            <item>sphotasiddhi = Maṇḍana's Sphoṭasiddhi</item>
            <item>śv = Kumārila's Ślokavārttika, chapters:
	<list>
                  <item>śv-ākṛti</item>
                  <item>śv-apoha</item>
                  <item>śv-codanā</item>
                  <item>śv-niralambana</item>
                  <item>śv-pratyakṣa</item>
                  <item>śv-śabda</item>
                  <item>śv-saṃbandha</item>
                  <item>śv-sphoṭa </item>
                  <item>śv-vākya</item>
                  <item>śv-vana</item>
               </list>
            </item>
         </list>
      </p>
   </encodingDesc>
   <profileDesc><!-- ... --></profileDesc>
   <revisionDesc>
      <change who="#sarit-encoder-pvsvt" when="2015-03-17">Corrected typo in note 1-3 ("parady" to "parody").</change>
      <change who="#sarit-encoder-pvsvt" when="2015-05-04">On p. 526 the line number "2" of the manuscript was printed in Devanāgarī and looks like a footnote reference. I encoded it as lb-element.</change>
      <change who="#sarit-encoder-pvsvt" when="2015-12-30">Added @xml:lang to the front-element.</change>
      <change who="#sarit-encoder-pvsvt" when="2016-04-25">Updated some abbreviations used in @cRef.</change>
      <change who="#sarit-encoder-pvsvt" when="2016-05-31">Replaced <tag>add</tag> that I had used for additions by the editor with <tag>supplied</tag> and updated the encoding description.</change>
   </revisionDesc>
</teiHeader>
	 \end{minted}
       
      \clearpage
      \begin{english}
      \printshorthands
      \printbibliography
      \end{english}
    
\end{document}
