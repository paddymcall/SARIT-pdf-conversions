\documentclass[article,12pt,a4paper]{memoir}
	  \usepackage{euler}
	  \usepackage{xltxtra}
  \usepackage{polyglossia}
  \PolyglossiaSetup{sanskrit}{
  hyphenmins={2,3},% default is {1,3}
  }
  \setdefaultlanguage{sanskrit}
  % english should be available, notes and stuff
  \setotherlanguage{english}
  \setotherlanguage[numerals=arabic]{tibetan}
  \usepackage{fontspec}
  \catcode`⃥=\active \def⃥{\textbackslash}
  \catcode`❴=\active \def❴{\{}
  \catcode`〔=\active \def〔{{[}}% translate 〔OPENING TORTOISE SHELL BRACKET
  \catcode`〕=\active \def〕{{]}}% translate 〕CLOSING TORTOISE SHELL BRACKET
  \catcode`❴=\active \def❴{\{}
  \catcode`❵=\active \def❵{\}}
  \catcode` =\active \def {\,}
  %% show a lot of tolerance
  \tolerance=9000
  \def\textJapanese{\fontspec{Kochi Mincho}}
  \def\textChinese{\fontspec{HAN NOM A}}
  \def\textKorean{\fontspec{Baekmuk Gulim} }
  % make sure English font is there
  \newfontfamily\englishfont[Mapping=tex-text]{TeX Gyre Schola}
    % set up a devanagari font
  \newfontfamily\devanagarifont{TeX Gyre Pagella}
	\newfontfamily\rmlatinfont[Mapping=tex-text]{TeX Gyre Pagella}
	\newfontfamily\tibetanfont[Script=Tibetan,Scale=1.2]{Tibetan Machine Uni}
  \newcommand\bo\tibetanfont
  
    \defaultfontfeatures{Scale=MatchLowercase,Mapping=tex-text}
	\setmainfont{TeX Gyre Pagella}
    \setsansfont{TeX Gyre Bonum}
  
  \setmonofont{DejaVu Sans Mono}
	  %% page layout start: fit to a4 and US letterpaper (example in memoir.pdf)
	  %% page layout start
	  % stocksize (actual size of paper in the printer) is a4 as per class
	  % options;
	  
	  % trimming, i.e., which part should be cut out of the stock (this also
	  % sets \paperheight and \paperwidth):
	  % \settrimmedsize{0.9\stockheight}{0.9\stockwidth}{*}
	  % \settrimmedsize{225mm}{150mm}{*}
	  % % say where you want to trim
	  \setlength{\trimtop}{\stockheight}    % \trimtop = \stockheight
	  \addtolength{\trimtop}{-\paperheight} %           - \paperheight
	  \setlength{\trimedge}{\stockwidth}    % \trimedge = \stockwidth
	  \addtolength{\trimedge}{-\paperwidth} %           - \paperwidth
	  % % this makes trims equal on top and bottom (which means you must cut
	  % % twice). if in doubt, cut on top, so that dust won't settle when book
	  % % is in shelf
	  \settrims{0.5\trimtop}{0.5\trimedge}

	  % figure out which font you're using
	  \setxlvchars
	  \setlxvchars
	  % \typeout{LENGTH: lxvchars: \the\lxvchars}
	  % \typeout{LENGTH: xlvchars: \the\xlvchars}

	  % set the size of the text block next:
	  % this sets \textheight and \textwidth (not the whole page including
	  % headers and footers)
	  \settypeblocksize{230mm}{130mm}{*}

	  % left and right margins:
	  % this way spine and edge margins are the same
	  % \setlrmargins{*}{*}{*}
	  \setlrmargins{*}{*}{1.5}

	  % upper and lower, same logic as before
	  % \setulmargins{*}{*}{*}% upper = lower margin
	  % \uppermargin = \topmargin + \headheight + \headsep
	  %\setulmargins{*}{*}{1.5}% 1.5*upper = lower margin
	  \setulmargins{*}{*}{1.5}% 

	  % header and footer spacings
	  \setheadfoot{2\baselineskip}{2\baselineskip}

	  % \setheaderspaces{ headdrop }{ headsep }{ ratio }
	  \setheaderspaces{*}{*}{1.5}

	  % see memman p. 51 for this solution to widows/orphans 
	  \setlength{\topskip}{1.6\topskip}
	  % fix up layout
	  \checkandfixthelayout
	  \sloppybottom
	  %% page layout end
	
	    % numbering depth
	    \maxtocdepth{section}
	    \setsecnumdepth{all}
	    \newenvironment{docImprint}{\vskip 6pt}{\ifvmode\par\fi }
	    \newenvironment{docDate}{}{\ifvmode\par\fi }
	    \newenvironment{docAuthor}{\ifvmode\vskip4pt\fontsize{16pt}{18pt}\selectfont\fi\itshape}{\ifvmode\par\fi }
	    % \newenvironment{docTitle}{\vskip6pt\bfseries\fontsize{18pt}{22pt}\selectfont}{\par }
	    \newcommand{\docTitle}[1]{#1}
	    \newenvironment{titlePart}{ }{ }
	    \newenvironment{byline}{\vskip6pt\itshape\fontsize{16pt}{18pt}\selectfont}{\par }
	    % setup title page; see CTAN /info/latex-samples/TitlePages/, and memoir
	  \newcommand*{\plogo}{\fbox{$\mathcal{SARIT}$}}
	  \newcommand*{\makeCustomTitle}{\begin{english}\begingroup% from example titleTH, T&H Typography
	  \thispagestyle{empty}
	  \raggedleft
	  \vspace*{\baselineskip}
	  
	      % author(s)
	    {\Large Ratnakīrti}\\[0.167\textheight]
	    % maintitle
	    {\Huge Ratnakīrtinibandhāvali}\\[\baselineskip]
	    % titlesubtitle
	    {\small  — A SARIT edition}\\[\baselineskip]
	    {\Large SARIT}\\\vspace*{\baselineskip}\plogo\par
	  \vspace*{3\baselineskip}
	  \endgroup
	  \end{english}}
	  \newcommand{\gap}[1]{}
	  \newcommand{\corr}[1]{($^{x}$#1)}
	  \newcommand{\sic}[1]{($^{!}$#1)}
	  \newcommand{\reg}[1]{#1}
	  \newcommand{\orig}[1]{#1}
	  \newcommand{\abbr}[1]{#1}
	  \newcommand{\expan}[1]{#1}
	  \newcommand{\unclear}[1]{($^{?}$#1)}
	  \newcommand{\add}[1]{($^{+}$#1)}
	  \newcommand{\deletion}[1]{($^{-}$#1)}
	  \newcommand{\pratIka}[1]{\textcolor{cyan}{#1}}
	  \newcommand{\name}[1]{#1}
	  \newcommand{\persName}[1]{#1}
	  \newcommand{\placeName}[1]{#1}
	  % running latexPackages template
     \usepackage[x11names]{xcolor}
     \definecolor{shadecolor}{gray}{0.95}
     \usepackage{longtable}
     \usepackage{ctable}
     \usepackage{rotating}
     \usepackage{lscape}
     \usepackage{ragged2e}
     
	 \usepackage{titling}
	 \usepackage{marginnote}
	 \renewcommand*{\marginfont}{\color{black}\rmlatinfont\scriptsize}
	 \setlength\marginparwidth{.75in}
	 \usepackage{graphicx}
	 \usepackage{csquotes}
       
	 \def\Gin@extensions{.pdf,.png,.jpg,.mps,.tif}
       %% biblatex stuff start
	 \usepackage[backend=biber,%
	 citestyle=authoryear,%
	 bibstyle=authoryear,%
	 language=english,%
	 sortlocale=en_US,%
	 ]{biblatex}
	 
		 \addbibresource[location=remote]{https://raw.githubusercontent.com/paddymcall/Stylesheets/HEAD/profiles/sarit/latex/bib/sarit.bib}
	 \renewcommand*{\citesetup}{%
	 \rmlatinfont
	 \biburlsetup
	 \frenchspacing}
	 \renewcommand{\bibfont}{\rmlatinfont}
	 \DeclareFieldFormat{postnote}{:#1}
	 \renewcommand{\postnotedelim}{}
	 %% biblatex stuff end
	 
	 \setcounter{errorcontextlines}{400}
       
	 \usepackage{lscape}
	 \usepackage{minted}
       
	   % pagestyles
	   \pagestyle{ruled}
	   
	   \makeoddfoot{ruled}{{\tiny\rmlatinfont \textit{Compiled: \today}}}{}{\rmlatinfont\thepage}
	   \makeevenfoot{ruled}{\rmlatinfont\thepage}{}{{\tiny\rmlatinfont \textit{Compiled: \today}}}
	   
	 
	   \usepackage{perpage}
           \MakePerPage{footnote}
	 
      \usepackage[noend,series={A,B}]{reledmac}
       % simplify what ledmac does with fonts, because it breaks. From the documentation of ledmac:
       % The notes are actually given seven parameters: the page, line, and sub-line num-
       % ber for the start of the lemma; the same three numbers for the end of the lemma;
       % and the font specifier for the lemma. 
       \makeatletter
       \def\select@lemmafont#1|#2|#3|#4|#5|#6|#7|%
       {}
       \makeatother
       \setlength{\stanzaindentbase}{20pt}
     \setstanzaindents{8,2,2,2,2,2,2,2,2,2,2,2,2,2,2,2,2,2,2,2,2,2,2,2,2,2,2,2,2,}
     % \setstanzapenalties{1,5000,10500}
     \lineation{page}
     % \linenummargin{inner}
     \linenumberstyle{arabic}
     \firstlinenum{5}
    \linenumincrement{5}
    \renewcommand*{\numlabfont}{\normalfont\scriptsize\color{black}}
    \addtolength{\skip\Afootins}{1.5mm}
    \Xnotenumfont{\bfseries\footnotesize}
    \sidenotemargin{outer}
    \linenummargin{inner}
    \Xarrangement{twocol}
    \arrangementX{twocol}
    
       \usepackage[destlabel=true,% use labels as destination names; ; see dvipdfmx.cfg, option 0x0010, if using xelatex
       pdftitle={Ratnakīrtinibandhāvali; A SARIT edition // Ratnakīrti},
       pdfauthor={SARIT (http://sarit.indology.info)}]{hyperref}
       \hyperbaseurl{}
       \usepackage[english]{cleveref}% clashes with eledmac < 1.10.1!
       % \newcommand{\cref}{\href}
     
\begin{document}
    
     \makeCustomTitle
     \let\tabcellsep&
	\frontmatter
	\tableofcontents
	% \listoffigures
	% \listoftables
	\cleardoublepage
        \mainmatter 
\chapter*{Ratnakīrtinibandhāvali}
	  
	% new div opening: depth here is 0
	
	    
	    \begingroup
	    \beginnumbering% beginning numbering from div depth=0
	    
	  
\chapter[{Sarvajñasiddhiḥ}]{Sarvajñasiddhiḥ}\label{Sarvajñasiddhiḥ}

	  \pstart namas tārāyai 
	\pend
      
	    
	    \stanza[\smallbreak]
	yasminn avajñā narakaprasūtir bhaktiś ca sarvābhimatapradāyinī |&avyāhataṃ yo jagadekabandhuḥ sa jñāyate sarvavid atra nirmalam ||\&[\smallbreak]


	

	  \pstart iha hi dharmajñād aparam anavaśeṣajñam anicchann api \persName{kumārilo} dharmajña eva kevale pratiṣiddhe vedam upādeyam abhimanyamānaḥ paṭhati
	\pend
      
	    
	    \stanza[\smallbreak]
	dharmajñatvaniṣedhas tu kevalo 'tropayujyate | &sarvam anyad vijānaṃs tu puruṣaḥ kena vāryate || iti | \&[\smallbreak]


	

	  \pstart tad ayam ācāryo 'pi sarvasarvajñacaraṇareṇusanāthaṃ yāvad ākāśaṃ jagadicchann api tribhuvanacūḍāmaṇībhūtasaparikaraheyopādeyatattvajñapuruṣapuṇḍarīka-prasādhanād apy apramāṇakajaḍavaidikaśabdarāśipramukhasakaladurmatipravādapratihatir ity antarnayann āha – 
	\pend
      
	    
	    \stanza[\smallbreak]
	heyopādeyatattvasya sābhyupāyasya vedakaḥ |&yaḥ pramāṇam asāviṣṭo na tu sarvasya vedakaḥ ||\edtext{}{\lemma{||}\Bfootnote{PV}}\&[\smallbreak]


	

	  \pstart ityādi || tad idānīm upayuktasarvajñam eva tāvat prasādhayāmaḥ | paryante tu sarvasarvajñadohadam apy apaneṣyāmaḥ | svāsthyam āsthīyatām | 
	\pend
      

	  \pstart yo yaḥ sādaranirantaradīrghakālābhyāsasahitacetoguṇaḥ sa sarvaḥ sphuṭībhāvayogyaḥ | 
	\pend
      

	  \pstart yathā yuvatyākāraḥ kāminaḥ puruṣasya | yathoktābhyāsasahitacetoguṇāś cāmī caturāryasatyaviṣayā ākārā iti svabhāvo hetuḥ | 
	\pend
      

	  \pstart tatra na tāvād āśrayadvāreṇa hetudvāreṇa vāsiddhisambhāvanā | saṃkalparūḍhānāṃ caturāryasatyākārāṇām \edlabel{ratnakīrtinibandhāvali__36r1PF7IMWZB49IEX1VY8ZG8MD3}\label{ratnakīrtinibandhāvali__36r1PF7IMWZB49IEX1VY8ZG8MD3}\edtext{}{\lemma{cetoguṇa}\xxref{ratnakīrtinibandhāvali__36r1PF7IMWZB49IEX1VY8ZG8MD3}{ratnakīrtinibandhāvali__36r1PF7IMWZ43B0FVQSP9MJDYEH}\Afootnote{ \cite{buehnemann80} cetoguṇatva \cite{thakur75} }}cetoguṇa\edlabel{ratnakīrtinibandhāvali__36r1PF7IMWZ43B0FVQSP9MJDYEH}\label{ratnakīrtinibandhāvali__36r1PF7IMWZ43B0FVQSP9MJDYEH}mātrasya ca hetoḥ pratyātmavedyatvāt | nāpi sādaranirantaradīrghakālābhyāsalakṣaṇaṃ hetuviśeṣaṇam asambhāvanīyam | tathā hi saṃsārasvabhāvaṃ duḥkhātiśayam apanetum iyaṃ saṃkalpārūḍhā caturāryasatyākārabhāvanā prārabdhā | asyāś cāsambhāvanā nāma kiṃ (1) bhāvyasya saṃkalpārūḍhatvāsambhavāt (2) anarthitvāt (3) heyarūpāniścayāt (4) heyasya nityatvāt (5) tasyāhetutvāt (6) taddhetor nityatvāt (7) heyahetvaparijñānāt (8) tadbādhakābhāvāt (9) bādhakāparijñānāt (10) cittasya doṣātmakatvāt (11) tasya vyavasthitaguṇatvāt (12) bhavāntarābhāvāt (13) dhvastadoṣapunarudbhavād veti trayodaśa vikalpāḥ ||
	\pend
      

	  \pstart tatra na tāvad ādyaḥ pakṣaḥ | saparikaraheyopādeyātmakasya caturādyasatyākārasya bhāvyasya vikalpārūḍhasya pratyātmavedyatvāt || 
	\pend
      

	  \pstart \edlabel{thakur75-2.9}\label{thakur75-2.9} nāpi dvitīyaḥ | duḥkhamātrasyāpi parityāgārthitvena vyāpteḥ sarvajanānubhavasiddhatvāt ||
	\pend
      

	  \pstart nāpi tṛtīyaḥ | saṃsārātmano duḥkhasvarūpasya pratīteḥ | katham asya duḥkhātmakatvam iti cet | saṃkṣepataḥ kathitaṃ 
	\pend
      
	    
	    \stanza[\smallbreak]
	sākṣād duḥkhaprakṛti narakaṃ pretatiryakkharūpaṃ martye śama kvacana tad api grastam evāsukhena |&devānāṃ ca kṣayam upagate puṇyapātheyapiṇḍe caṇḍajvālāvyatikaramuco hanta bhogāsta eva || \&[\smallbreak]


	

	  \pstart iti || 
	\pend
      

	  \pstart na ca caturthaḥ | vārtamānikapañcaskandhātmakasya duḥkhasyotpādadarśanāt || 
	\pend
      

	  \pstart na ca pañcamaḥ | duḥkhasya kādācitkatvāt || 
	\pend
      

	  \pstart nāpi ṣaṣṭhaḥ | kāryakādācitkatvasya anityahetukatvena vyāptatvāt || 
	\pend
      

	  \pstart nāpi saptamaḥ | duḥkhe viparyāsatṛṣṇāpravṛttiśaktikarmabhiḥ sahitasyātmadṛṣṭilakṣaṇasya hetoḥ sāṃsārikapañcaskandhalakṣaṇakāryānyathānupapattito niścayāt | yad āhuḥ 
	\pend
      
	    
	    \stanza[\smallbreak]
	ahaṃkāras tāvat tadanu mamakāras tadubhayaprasūto rāgādis tadahitamater dveṣadahanaḥ |&tataḥ śeṣaḥ kleśas tata udayinaḥ karmavisarādvisārī saṃsāraḥ śaraṇarahito dāruṇataraḥ ||\&[\smallbreak]


	
	    
	    \stanza[\smallbreak]
	tasmāt tṛṣṇāviparyāsāv ātmadṛṣṭipuraḥsarau |&aṃsāriskandhajanakau nirṇītau kāryahetutaḥ ||\&[\smallbreak]


	

	  \pstart ātmadarśanasya cāvidyātvam ātmapratikṣepato draṣṭavyam | tadabhāve 'pi kṣaṇabhaṅgaprastāve paralokādikam anākulam avasthāpitam || 
	\pend
      

	  \pstart na cāṣṭamaḥ | ātmadṛṣṭirūpāyā avidyāyāḥ pratipakṣabhūtasya nairātmyadarśanasya sambhavāt || 
	\pend
      

	  \pstart nāpi navamaḥ | nairātmyadarśanasya mārgaśabdavācyasya pramāṇato niścitatvāt || 
	\pend
      

	  \pstart daśamo 'py asambhavī | doṣāvasthāyāṃ cittasya saṃskārāpekṣatvāt | yo hi yatsvabhāvas tasmin svabhāve vyavasthito na saṃskāram apekṣate | yathā doṣam apanīya tapanīyam akṣayadaśāyām avasthitam | apekṣate ca cittam avidyāvasthāyāṃ saṃskāram iti vyāpakaviruddhopalabdhiḥ | pratiṣedhyasya tatsvabhāvatvasya yadvyāpakaṃ saṃskāranirapekṣatvaṃ tadviruddhaṃ tadapekṣatvam iti cittasya doṣātmakatvakṣatiḥ || 
	\pend
      

	  \pstart ekādaśo 'py ayuktaḥ | cetasas tattatsamskārātiśaye prajñātiśayadarśanāt || 
	\pend
      

	  \pstart na ca dvādaśaḥ | paralokaprasādhanāt | tathā hi, yac cittaṃ tat cittāntaraṃ pratisandhatte | yathedānīntanaṃ cittam | cittaṃ ca maraṇakālabhāvīti svabhāvahetuḥ | 
	\pend
      

	  \pstart na cārhaccaramacittena vyabhicāraḥ | tasyāgamamātrataḥ pratītatvāt | niḥkleśacittāntarajananād vā | hetor vā kleśe satīti viśeṣaṇād ity anāgatabhavasiddhiḥ | evaṃ yac cittaṃ tac cittāntarapūrvakaṃ yathedānīntanaṃ cittaṃ | cittaṃ ca janmasamayabhāvīty arthataḥ kāryahetur ity atītabhavasiddhiḥ || 
	\pend
      

	  \pstart na ca trayodaśaḥ | doṣakāraṇasyātmadarśanasya yadviruddhaṃ nairātmyadarśanaṃ tasya nirupadravatvāt | bhūtārthatvāt | svabhāvatvāc ca | sarvadāvasthiteḥ | tan nāyaṃ viśeṣaṇāsiddho 'pi hetuḥ | tathāpīdṛśo 'bhyāso na kasyacid dṛśyata iti cet | na dṛśyatām | sambhāvanā tāvad aśakyapratiṣedhā | idānīntanajanapravṛttiś cāvyāhateti nāparaṃ gamyate | ata evedaṃ sambhāvanānumānam ucyate || 
	\pend
      

	  \pstart na caiṣa viruddho hetuḥ | sapakṣe kāminy ākāre sambhavāt | 
	\pend
      

	  \pstart na cānaikāntikaḥ | abhyāsasahitacetoguṇasphuṭapratibhāsayoḥ kāryakāraṇayor ghaṭakumbhakārayor iva sarvopasaṃhāreṇa pratyakṣānupalambhataḥ kāryakāraṇabhāvasiddhāv abhyāsasahitacetoguṇatvasya sādhanasya sphuṭapratibhāsakaraṇayogyatayā vyāptisiddheḥ | tathā hi vyāptyadhikaraṇe kāmātur avartini yuvatyākāre sādaranirantaradīrghakālābhyāsasahitacetoguṇāt pūrvaṃ anupalabdhiḥ sphuṭābhasya | paścād abhyāsasaṃvedanaṃ sphuṭābhasaṃvedanam iti | trividhapratyakṣānupalambhasādhyaḥ kāryakāraṇabhāvaḥ sphuṭapratibhāsābhyāsasacivacittākārayor iyam upapannā sarvopasaṃhāravatī vyāptiḥ | ato 'naikāntikatāpy asambhavinīty anavadyo hetuḥ || 
	\pend
      

	  \pstart nanu katham anumānataḥ sarvajñasiddhipratyāśā | tasya parokṣatvena tatpratibaddhaliṅgāniścayāt | kiṃ ca sarvajñasattāsādhane sarvo hetur na trayīṃ doṣajātim atipatati | sarvajñe hi dharmiṇy asiddhatvam | asarvajñe hi viruddhatvam | ubhayātmake 'py anaikāntikatvam iti || 
	\pend
      

	  \pstart api ca abhyāsāt kāraṇāt kāryasya sphuṭābhasya pratītau nāvaśyaṃ kāraṇāni kāryavanti bhavantīty anaikāntikatā | atha sphuṭībhāvayogyatānumīyate | sāpi śaktir ucyate | sā ca kārye 'nantarā sāntarā vā | atrādyā kāryasamadhigamyā | na cādhigatakāryasya tayā kaścid upayogaḥ | dvitīyā tu kāryāvasāyam aikāntikaṃ na sādhayet || 
	\pend
      

	  \pstart na ca kāryāpratītau yogyatāniścayaḥ sambhavī | nāpi yogyatāmātrasādhane kṛtārthaḥ sādhanavādī | sarvajñajñāne kārye vivādasya tādavasthyād | bhavatu sphuṭībhāvasya siddhiḥ | tathāpi kaḥ prastāvaḥ sarvajñavivāde sādhanam ārabdhavataḥ sphuṭatvaṃ cetasaḥ sādhayitum || 
	\pend
      

	  \pstart kiṃ ca prasiddhānumāne bhūtalasya dharmiṇi kumbhakāraghaṭayor api dharmayoḥ pratītatvāt kāryakāraṇabhāvo grahītuṃ śakyata eva | prastute tu kāmātur asantānavartino yuvatyākārasya dharmiṇas tatpragatābhyāsasphuṭatvayor api dharmayoḥ parokṣatvāt | kathaṃ kāryakāraṇagṛhītiḥ | yathā ca naiyāyikaṃ prati yuṣmābhir ucyate pratyakṣato na kāryamātraṃ puruṣavyāptaṃ sidhyati | kiṃ tv avāntaram eva ghaṭajātīyaṃ kāryam iti tathā nākāramātram abhyāsapūrvakaṃ sidhyati | kiṃ tv avāntaram eva yuvatyākārasāmānyam iti vyaktam eva | na cābhyāsakāryaḥ sphuṭībhāvaḥ | tadabhāve 'pi svapne darśanāt || 
	\pend
      

	  \pstart kiṃ ca sarvavido 'pi yadi caturāryasatyaparijñānataḥ sarvajñatāsthitiḥ, tarhi ghaṭādikatipayavastujñāne 'pi sarvajñeti sādhvī śuddhiḥ | api ca 
	\pend
      

	  \pstart jñānavān mṛgyate kaścit taduktapratipattaye | 
	\pend
      
	    
	    \stanza[\smallbreak]
	ajñopadeśakaraṇe vipralambhanaśaṅkibhiḥ ||\edtext{}{\lemma{||}\Bfootnote{(PV II 30)}}\&[\smallbreak]


	

	  \pstart iti yuṣmābhir evocyate | na ca sarvajñānavān viśeṣaniṣṭhatayādhigantuṃ śakyate | na cāsya sattāmātrasiddhau kaścid upayogaḥ, pravṛtter anaṅgatvād iti sarvam asamañjasam || 
	\pend
      

	  \pstart atrocyate | na vayaṃ sākṣātsarvajñasattāpratijñāyāṃ hetuvyāpāram anumanyāmahe | bhūdharādhīnavahnisattāvat | kiṃ tu caturāryasatyākārasvarūpe dharmiṇi sphuṭābhatvasya sādhyasyāyogavyavacchedārthaṃ parvate 'gnimātrāyogavyavacchedavat | sphuṭābhatvaṃ tu kāminy ākārādidṛṣṭānte dṛṣṭam eva | tac ca parvatīyāgnivat | pakṣadharmatābalataḥ satyacatuṣṭayādhikaraṇaṃ sidhyat sarvajñatām ācakṣmahe | yathoktam 
	\pend
      
	    
	    \stanza[\smallbreak]
	ityabhyāsabalāt parisphuṭadaśākoṭiḥ sphurat sambhavī heyādeyatadaṅgalakṣaṇaguṇaḥ sarvajñatā saiva naḥ || \&[\smallbreak]


	

	  \pstart iti | 
	\pend
      

	  \pstart tad atrābhyāsasahitacaturāryasatyākāraḥ samagro dharmī sāmagryam abhyāsaviśiṣṭacetoguṇatvamātraṃ hetuḥ sphuṭībhāvayogyatāsādhyam | yathā sāgnitvānagnitvasandehe parvatātmā pramāṇapratīto dharmī | tathātrāpi sarvajñatvāsarvajñatvavivāde 'pi pratyātmaviditaḥ satyacatuṣṭayākāro dharmī | tasmāt sphuṭābhatvena sādhyena dṛṣṭānte vyāptisiddher asty eva tatpratibaddhaliṅganiścayaḥ | sādhyasandehe 'pi dharmiṇaś caturāryasatyākārasya siddher na trividhadoṣajāter avasaraḥ | yogyatāyāḥ prasādhanena ca kāraṇāt kāryapratītāv anaikāntikatvam ity apy anabhyupagamapratihatam | yogyatā ca sāntaraiva sādhyate | iyaṃ ca na gamayatu nāmaikāntataḥ kāryasattvam | anupapadyamānaṃ punar asya sambhavam ākṣipaty eva | tadā bhāvini kārye sandehe 'pi kāraṇayogyatā niścīyata eva | brīhyādau bhāviphalāniścaye 'pi yogyatāniścayena pravṛtteḥ | anyathā śilāśakalāder apy upādānaprasaṅgaḥ | 
	\pend
      

	  \pstart tajjātīyasya śarāvasthapaṅkoptasya sāmarthyam upalabdham iti cet | atrāpi kāminy ākāre bhāvanājātīyasya sphuṭībhāvakaraṇayogyatā dṛṣṭeti samānam | 
	\pend
      

	  \pstart evaṃ yogyatāmātrasādhanenaiva kṛtārthaḥ sādhanavādī | sarvajñakāraṇabhāvāt tadabhāvavādināṃ nirdalanāt | kāryasya ca traikālikasya sambhāvanāprasādhanāt | muttkyarthināṃ ca pravṛtter avirodhāt | vādino 'pi tanmātrasādhanasyābhipretatvāt | ata eva kaḥ prastāvaḥ sarvajñasattāvivāde sphuṭībhāvasādhanasyetyādy apy anavakāśam | sarvajñaśabdena sphuṭībhāvayogyatāyā vivakṣitatvāt | tathā kāryakāraṇapratītir api sambhavaty eva | tathā hi kāminy abhyāsasantatisahacāri sambhramkāryavacodarśanam eva kāminy ākārasya tadbhāvanāyāś ca darśanam | tathābhūtakāyavaco 'darśanam eva bhāvanāyā adarśanam | evaṃ sphuṭapratibhāsasantatisahacāriviśiṣṭakāyavacodarśanaṃ sphuṭapratibhāsadarśanam | tathāvasthitakāyavaco 'darśanam eva sphuṭapratibhāsādarśanam ity asaty eva prastute 'pi pratyakṣānupalambhataḥ kāryakāraṇabhāvapratītiḥ | iyaṃ ca tathāvasthakāmātur aśarīravacanagrahaṇe tadekadeśabhūtayuvatyākārābhyāsasphuṭapratibhāsagrahaṇavyavasthā vyāvahārikeṇāvaśyaṃ svīkartavyā | anyathā \edlabel{ratnakīrtinibandhāvali__36r1PF7IMWYWBFZSMXH00ZADHKO}\label{ratnakīrtinibandhāvali__36r1PF7IMWYWBFZSMXH00ZADHKO}\edtext{}{\lemma{}\xxref{ratnakīrtinibandhāvali__36r1PF7IMWYWBFZSMXH00ZADHKO}{ratnakīrtinibandhāvali__36r1PF7IMWYNUC1RBB38NANZLQM}\Afootnote{citta \cite{buehnemann80} ; citya \cite{thakur75} }}citta\edlabel{ratnakīrtinibandhāvali__36r1PF7IMWYNUC1RBB38NANZLQM}\label{ratnakīrtinibandhāvali__36r1PF7IMWYNUC1RBB38NANZLQM}caityarūparasagandhasparśaparamāṇupuñjādyātmakasya kumbhakāraghaṭapradeśāder api rūpaikadeśagrāhakaṃ cakṣuḥpratyakṣaṃ na samudāyavyavasthāpakam iti sarvavyāvahārikapramāṇocchedaprasaṅgaḥ | tathā bāhyaghaṭakām ityādīnāṃ śaktikṛtasya mahato jātibhedasya sambhavād anyajātīyavyāptigrahe 'nyajātīyād buddhimadanumānam ayuktam | saṃkalpārūḍhānāṃ tu jalajvalanayuvatyākārādīnāṃ bāhyatvenādhyastānām api vijñānaikasvarūpatayaikajātīyatvam astīti bhāvanāsahitākāramātreṇaiva vaiśadyavyāptir astu ||
	\pend
      

	  \pstart na ca svapne sphuṭatāvyabhicāraḥ | bhāvanāsiddhalakṣaṇayor hetvor jātibhede tatkāryayor ekatvābhimāne 'pi jātibhedasyāvaśyaṃ svīkartavyatvāt | dṛśyate hi siddhasādhyā vaiśadyajātir anapekṣya viparītabhāvanāṃ nidrāvicchede vicchidyamānā | bhāvanābhāvinī tu na vinā vipakṣābhyāsaṃ jāgrato 'pi | yad āhuḥ 
	\pend
      
	    
	    \stanza[\smallbreak]
	svapne 'pi sphuṭatā tathaiva na tathāpy ekatvam evānayor na prākārasamatvam eva samatāṃ jāteḥ samāmaṅgati |&anyanniddhanirodhabādhyam itaradbādhyaṃ pratyatnaiḥ punar vaiśadyaṃ viparītabhāvanabalān nairghṛṇyabhede yathā ||\&[\smallbreak]


	

	  \pstart iti ||
	\pend
      

	  \pstart yad api ghaṭādikatipayajñāne 'pi sarvajñaḥ syād ity uktam | tatrāpi 
	\pend
      
	    
	    \stanza[\smallbreak]
	ghaṭādiprakṛtāśeṣavedane 'pi bhayaṃ bhavād dheyata yadi ko doṣaḥ so 'pi sarvajñatāṃ vrajet |&saṃsāraduḥkhamokṣāya spṛhayanto vayaṃ punar bhajema tadupāyajñaṃ sthātuṃ tadgītavartamani ||\&[\smallbreak]


	

	  \pstart ity uttaraṃ draṣṭavyam | tathā sattāmātre vipratipannān prati sattaiva kevalā prasādhitā | viśeṣajijñāsāyāṃ tu pramāṇopapannakṣaṇikanairātmyavādina eva sugatasya bhagavataḥ sarvajñatā | ata etad api nirastaṃ yad āha Bhaṭṭaḥ 
	\pend
      
	    
	    \stanza[\smallbreak]
	\label{rna-ts-3149}\flagstanza{\tiny\textenglish{...s-3149}}sugato yadi sarvajñaḥ kapilo neti kā pramā |&athobhāv api sarvajñau matabhedaḥ kathaṃ tayoḥ || iti | \edtext{}{\lemma{|}\Bfootnote{(=TS 3149)}}\&[\smallbreak]


	

	  \pstart tasmāt 
	\pend
      
	    
	    \stanza[\smallbreak]
	uktakrameṇa munirājanaye pramāyāḥ śaktir vyanakti gatim apramitāṃ kṛpāṃ ca | &anyatra tu dvayam udastam ado 'stamāne tenaika eva śaraṇaṃ sa nirātmavādī || \&[\smallbreak]


	

	  \pstart iti viśeṣasiddhir apy anavadeyeti sarvam anākulam ākulādhayaḥ pare na pratipadyante | sādhane 'sminn avadye 'pi durnītidahanadagdhabuddhayaḥ punar apy etad ācakaṣate | bādhakapramāṇasadbhāvāt sarvajñasyāsadvyavahāro yuktaḥ sadvyavahārapratiṣedho vā prasādhakapramāṇābhāvād veti || 
	\pend
      

	  \pstart atra vicāryate kiṃ punar asya bhagavato bādhakaṃ pramāṇaṃ pratyakṣam anumānaṃ śabdādikaṃ veti vikalpāḥ || 
	\pend
      

	  \pstart na tāvat pratyakṣaṃ | pratyakṣaṃ hi kevalapradeśādau pravartamānaṃ svapravṛttiyogyam eva tatra vastu pratiṣedhati | na vastumātram | na ca sarvajñasya pratyakṣapravṛttiyogyatāsti | svabhāvaviprakṛṣṭatvāt tasya || 
	\pend
      

	  \pstart syād etat | na vayaṃ pratyakṣaṃ pravartamānam abhāvaṃ sādhayatīti brūmaḥ | kiṃ tarhi | nivartamānam | tathā hi yatra vastuni pratyakṣasya nivṛttis tasyāsadbhāvaḥ | yathā śaśaviṣāṇādeḥ | yatra tu pratyakṣasya pravṛttis tasya sadbhāvo yathā ghaṭādeḥ | asti ca sarvajñe pratyakṣanivṛttiḥ | tad asyāpy abhāvaḥ kena nivāryata iti || 
	\pend
      

	  \pstart ucyate | nivartamānaṃ pratyakṣam abhāvaṃ sādhyatīti ko 'rthaḥ | kiṃ pratyakṣasya yā nivṛttis tato 'bhāvasiddhiḥ, nivṛttisahitād vā pratyakṣāt, nivṛttād vā pratyakṣād iti | 
	\pend
      

	  \pstart nādyaḥ pakṣaḥ | saty api vastuni pratyakṣanivṛtter upalabhyamānāyā vastvabhāvaniyatatvāsiddheḥ || 
	\pend
      

	  \pstart nāpi dvitīyaḥ | svābhāvena saha kasyacit sāhityānupapatteḥ | anyathā tannivṛttatvānupapatteḥ || 
	\pend
      

	  \pstart na ca tṛtīyaḥ | tathā hi nivṛttāt pratyakṣād abhāvasiddhir ity asataḥ pratyakṣād ity uktaṃ bhavati | na cāsato hetubhāvaḥ sambhavati | sarvasamarthyavirahalakṣaṇtvāt tasya | na hi tac ca nāsti tena ca pratipattir iti nyāyam | ato na tāvat pratyakṣaṃ sarvajñabādhakam || 
	\pend
      

	  \pstart nāpy anumānam | tad dhi trividhaliṅgajatvena trividham | tatra kāryasvabhāvayor vidhisādhanatvāt, pratiṣedhe sādhye 'navasaraḥ | na ca dṛśyānupalambhaḥ tatprabhedo vā kāryānupalabdhyādir yogyānupalambho vā parābhimato 'tra pramāṇam | sarvajñatāyāḥ svabhāvaviprakṛṣṭatvenādṛśyatvāt || 
	\pend
      

	  \pstart nanu kāraṇānupalambhād eva sarvajñatāpratiṣedhaḥ sidhyati | tathā hi tatkāraṇam indriyavijñānaṃ vā mānasaṃ vā bhāvanābalajaṃ vā | bhāvanābalajam api cākṣuṣaṃ vā, mānasaṃ veti vikalpāḥ | 
	\pend
      

	  \pstart tatra na tāvac cakṣurindriyavijñānam aśeṣārthagrāhi | tasya pratiniyatārthaviṣayatvāt | deśāntare kālāntare ca tathaiva pratiniyamaḥ | anyathā hetuphalabhāvābhāvaprasaṅgāt | anekendriyavaiyarthyaprasaṅgāc ca | tathā ca kārikā 
	\pend
      

	  \pstart ekendriyapramāṇena sarvajño yena kalpyate | 
	\pend
      

	  \pstart nūnaṃ sa cakṣuṣā sarvān rasādīn pratipadyate || 
	\pend
      

	  \pstart yajjātīyaiḥ pramāṇaiś ca yajjātīyārthadarśanam | 
	\pend
      

	  \pstart bhaved idānīṃ lokasya tathā kālāntare 'py abhūt || iti | \edtext{}{\lemma{|}\Bfootnote{(ŚV II 112-113; =TS 3158-3159)}}
	\pend
      

	  \pstart tataś caivaṃ prayogaḥ kartavyaḥ | buddhacakṣurnātītādiviṣayam | cakṣustvāt | asmadādicakṣurvat | acakṣur vā | 
	\pend
      

	  \pstart atītādiviṣayatvāt | śabdavat | iti sarvam etat śrotrādāv api draṣṭavyam | na cakṣurādiprakarṣaḥ svārtham atikramya dṛṣṭaḥ | \name{Kārikā}
	\pend
      
	    
	    \stanza[\smallbreak]
	yatrāpy atiśayo dṛṣṭaḥ sa svārthānatilaṅghanāt | &dūrasūkṣmādivṛttau syān na rūpe śrotravṛttitaḥ || \edtext{}{\lemma{||}\Bfootnote{(ŚV II 114)}}\&[\smallbreak]


	

	  \pstart \name{Bṛhaṭṭīkā} ca
	\pend
      
	    
	    \stanza[\smallbreak]
	śrotragamyeṣu śabdeṣu dūrasūkṣmopalabdhitaḥ | &puruṣātiśayo dṛṣṭo na rūpādyupalambhanāt || &cakṣuṣāpi ca dūrasthasūkṣmarūpopalambhanam | &kriyate 'tiśayaprāptyā na tu śabdādidarśanam || \edtext{}{\lemma{||}\Bfootnote{(=TS 3162-63)}}\&[\smallbreak]


	

	  \pstart na caitad vaktavyam | yadi nāmaikaikenendriyeṇa tajjñānena vā sarvasyāgrhaṇaṃ tathāpi pañcabhir indriyais tajjñānair vā svasvaviṣayapravṛttair evātiśayaprāptair bhaviṣyatīti | ekaikasyāpi niḥśeṣasvaviṣayagrahaṇādarśanāt | paracittādyatīndriyāṇāṃ grahaṇābhāvāc ca | tad evam indriyavijñānaṃ vā nāśeṣagrāhīti na prathamaḥ pakṣaḥ || 
	\pend
      

	  \pstart nāpi dvitīyaḥ | tathā hi yady api tanmānasaṃ sarvārthaviṣayaṃ tathāpi na tasya svātantryeṇārthagrahaṇe vyāpāro 'sti | manaso bahirasvātantryāt | anyathāndhavadhirādyabhāvaprasaṅgaḥ | teṣām api manaso bhāvāt | pāratantrye cetndriyajñānaparigṛhītārthaviṣayatvād atītānāgatadūrasūkṣmavyavahitaparacittāder arthasyendriyaparijñānāgocarasya manasā paricchedo na prāpnotīti kathaṃ sarvajñatā || 
	\pend
      

	  \pstart na ca bhāvanābalajaṃ sarvārthagrāhīti tṛtīyaḥ pakṣaḥ | tathā hi tadbhāvanābalajam api yadīndriyāśritam iti caturthaḥ pakṣaḥ, tadā so 'saṅgataḥ | indriyasya tajjñānasya ca niyataviṣayaviṣayatvapratipādanāt || 
	\pend
      

	  \pstart atha bhāvanābalena tathāvidham utpannaṃ manovijñānaṃ sarvārthagrāhīti pañcamaḥ pakṣaḥ | tadānvarthatvāt pratyakṣaśabdasya tasya ca bhāvanābalāvalambino 'py anakṣajatvāt nārthasākṣātkāritvam astīti pratipādanīyam | kiṃ ca svaviṣayasīmānam anatipatyaiva prakarṣo 'pi dṛśyate | na tu sarvaviṣayatveneti | kathaṃ tenāpi sakalārthajātādivedanam | yato na kasyacid abhyāse 'py atīndriyārthadarśitvam upalabdham || 
	\pend
      

	  \pstart Bṛhaṭṭīkā 
	\pend
      

	  \pstart ye 'pi sātiśayā dṛṣṭāḥ prajñāmedhābalair narāḥ | 
	\pend
      

	  \pstart stokastokāntaratvena na te 'tīndriyadarśanāḥ || 
	\pend
      

	  \pstart prājño 'pi ca naraḥ sūkṣmān athān draṣṭuṃ kṣamo 'pi san | 
	\pend
      

	  \pstart sajātīr anatikrāman nātiśete parān api || \edtext{}{\lemma{||}\Bfootnote{(=TS 3160-61)}}
	\pend
      

	  \pstart ekāvavarakasthasya pratyakṣaṃ yat pravartate | 
	\pend
      

	  \pstart śaktis tatraiva tasya syān naivāvavarakāntare || 
	\pend
      

	  \pstart ye cārthā dūravicchinnā deśaparvatasāgaraiḥ | 
	\pend
      

	  \pstart varṣadvīpāntarair ye ca kas tān paśyed ihaiva san || \edtext{}{\lemma{||}\Bfootnote{(=TS 3170-71)}}
	\pend
      

	  \pstart atra varṣaḥ kālaviśeṣaḥ | 
	\pend
      

	  \pstart evaṃ śāstravicāreṣu dṛśyate 'tiśayo mahān | 
	\pend
      

	  \pstart na tu śāstrāntarajñānaṃ tanmātreṇaiva sidhyati || 
	\pend
      

	  \pstart jñātvā vyākaraṇaṃ dūraṃ buddhiḥ śabdāpaśabdayoḥ | 
	\pend
      

	  \pstart ākṛṣyate na nakṣatratithigrahaṇanirṇaye || 
	\pend
      

	  \pstart jyotirvic ca prakṛṣṭo 'pi candrārkagrahaṇādiṣu | 
	\pend
      

	  \pstart na bhavatyādiśabdānāṃ sādhutvaṃ jñātum arhati || 
	\pend
      

	  \pstart tathā vedetihāsādijñānātiśayavān api | 
	\pend
      

	  \pstart na svargadevatāpūrvapratyakṣīkaraṇe kṣamaḥ || 
	\pend
      

	  \pstart daśahastāntaraṃ vyomno ye nāmotplutya gacchati | 
	\pend
      

	  \pstart na yojanam asau gantuṃ śakto 'bhyāsaśatair api | 
	\pend
      

	  \pstart tasmād atiśayajñānair atidūragatair api | 
	\pend
      

	  \pstart kiñcid evādhikaṃ jñātuṃ śakyate na tv atīndriyam || iti | \edtext{}{\lemma{|}\Bfootnote{(=TS 3164-69)}}
	\pend
      

	  \pstart pratyakṣasūtre tu kāśikākāraḥ paramatam āśaṅkyāha, tan na, avagataviṣayatvād bhāvanāyāḥ | na cākasmād avagater utpattiḥ sambhavati | sarvotpattimatāṃ kāraṇavattvāt | atha pramāṇāntarāvagataṃ bhāvyate | kiṃ bhāvanayā | tata eva tatsiddheḥ | kiṃ ca tatpramāṇam | na tāvad anumānaṃ dharmādharmayoḥ pūrvam agrahaṇena tadvyāptaliṅgasaṃvedanāsambhavāt | jagadvaividhyārthāpatter api hi kim api kāraṇam astīti etāvad unnīyate | na tu kaścid viśeṣaḥ | na cānirdiṣṭaviśeṣaviṣayā bhāvanā bhavati | yogaśāstreṣv api hi viśeṣā eva dhyeyatayopadiśyante | 
	\pend
      

	  \pstart dhyeya ātmā prabhuryo 'sau hṛdi dīpa iva sthitaḥ | (Maitrī Up. 6,30) 
	\pend
      

	  \pstart ityādibhiḥ | āgamamānāt tarhi avagataṃ bhāvayiṣyate | yadi pramāṇāt tadā tata evāvagateḥ | kiṃ bhāvanayā | hānopādānārthaṃ hi vastu jijñāsyate | te ca tata eva siddhe iti vyarthā bhāvanā | kāruṇiko 'pi hi dharmāgamān eva śiṣyebhyo vyācakṣīta | na bhāvanābhedam anubhavet | 
	\pend
      

	  \pstart atha vipralambhabhūyiṣṭhatvād āgāmānāṃ pramāṇam āgamo na veti vicikitsamāno bhāvanayā jijñāsate | tan na | tato 'pi tadasiddheḥ | bhāvanābalapriniṣpannam api jñānam anāśvsanīyārtham eva | abhūtasyāpi bhāvyamānasyāparokṣārthavat prakāśanāt | yathā hi tair evoktam 
	\pend
      
	    
	    \stanza[\smallbreak]
	tasmād bhūtam abhūtaṃ vā yad yad evābhibhāvyate |&bhāvanāpariniṣpattau tat sphuṭā kalpadhīḥ phalam ||\edtext{\textsuperscript{*}}{\lemma{*}\Bfootnote{PV III 285; PVin I 30.}}\&[\smallbreak]


	

	  \pstart api ca bhāvanābalajam apramāṇam | gṛhītagrahaṇāt | yāvad eva hi gṛhītaṃ tāvad eva bhāvanayā viṣayīkriyate | mātrayāpy adhikaṃ na bhāvanā gocarayati | yogābhyāsāhitasaṃskārapāṭavanimittā hi smṛtir eva bhāvaneti gīyate | sā ca pramāṇam iti sthitam eva | na ca taduttarakālaṃ sākṣātkārijñānam udetīti pramāṇam asti | indriyasannikarṣam antareṇārthasākṣātkārasya kvacid adarśanāt | yogināṃ dharmādharmayor aparokṣapratibhāsaṃ jñānaṃ nāsti, indriyasannikarṣābhāvād asmadādivat || 
	\pend
      

	  \pstart Vācapatis tu Kaṇikāyām āha | satyaṃ śrutānumānagocaracāriṇī bhāvanā viśadābhajñānahetur iti nāvajānīmahe | kin tu yadviṣayajātaṃ tad eva viśadapratipattigocaraḥ | na jātu rūpabhāvanāprakarṣo rasaviṣayavijñānavaiśadyāya kalpate | 
	\pend
      

	  \pstart nanu na viṣayāntaravaiśadyahetubhāvaṃ bhāvanāyāḥ saṅgirāmahe | kintu śrutānumānaviṣayavaiśadyahetutām eva | tadviṣayaś ca samastavastunairātmyam iti tadbhāvanāprakarṣaḥ samastavastunairātmyaṃ viśadayan samastavastuviśadatām antareṇa tadupapatteḥ samastavastuvaiśadyam āvahatīty uktam | 
	\pend
      

	  \pstart satyam uktam | ayuktaṃ tu tat | tathā hi nāgamānumānagocaratvaṃ nirātmanāṃ vastubhedānāṃ paramārthasatām | na hi te eteṣām anyanivṛttimātrāvagāhinī paramārthasatsvalakṣaṇaṃ gocarayitum arhataḥ | nāpi tadviṣayā bhāvanā | tadagrāhyam api svalakṣaṇaṃ tadadhyavaseyatayā tadviṣaya iti tadyonir api bhāvanā tadviṣayeti tatprakarṣas tadvaiśadyahetur iti cet | na | tadadhyavaseyasyāpi paramārthasattvābhāvāt | tathā hi yad anumānena gṛhyate yac cādhyavasīyate te dve apy anyanivṛttī, na vastunī |\edtext{}{\edlabel{note-2objects-neither-real}\lemma{|}\Bfootnote{Cf. also \cref{Frauwallner37}, \cref{buehnemann80}, \cref{mccrea_patil06}.}} svalakṣaṇāvagāhitve 'bhilāpasaṃsargayogyapratibhāsānupapatteḥ ||
	\pend
      

	  \pstart mā bhūt tayoḥ svalakṣaṇaṃ viṣayaḥ | tatprabhavabhāvanāprakarṣaparyantajanmanas tu viśadābhasya cetaso bhaviṣyati | kāmīnīvikalpaprabhavabhāvanāprakarṣād iva kāmātur asya kāminīsvalakṣaṇasākṣātkāraḥ | karikumbhakaṭhorakucakalaśahāriṇi hariṇaśāvalolalocane campakadalāvadātagātralate lāvaṇyasarasi nirantaralagnalalitadoḥkandalīmūlamāliṅganam aṅgane preyasitare prayaccha | sañjīvaya jīviteṣvari, patito 'smi tava caraṇanalinayor iti vacanakāyaceṣṭayor upalabdheḥ | asti ca vikalpāvikalpayoḥ kathañcit samānaviṣayateti nātiprasaṅga iti cet | satyam | sambhavaty ayam anubhavo na punar asyārthe prāmāṇyasambhavaḥ | atadutpatter atadātmanas tadavyabhicāraniyamāyogāt | atādātmyaṃ cārthasya vijñānād atirekāt | anatireke 'pi ca vijñānānām anyonyasya bhedād atādātmyāt | ekasya vijñānasyetaravijñānavedanānupapatteḥ | vijñānasvalakṣaṇaikatvābhyupagame ca tannityam ekam advitīyaṃ brahmābhyasanīyam iti kṣaṇikanairātmyābhyāsābhyupagamo dattajalāñjaliḥ prasajyeta | tan na tādātmyāt tasyāvyabhicāraḥ | nāpi tatkāryatvāt | bhāvanāprakarṣakāryaṃ khalv evan na viṣayakāryam | yady ucyeta pāramparyeṇa tatkāryam anumānavat | yathā hi vahnisvalakṣaṇād dhūmasvalakṣaṇam | tato dhūmānubhavas tato dahanavikalpaḥ, tataś cānumānam utpannam iti pāramparyeṇa vahnipratibandhāt prāpakaṃ ca vahner dāhapākakāriṇaḥ tathedam api anumānajanitabhāvanāprakarṣaparyantajaṃ pāramparyeṇārthaprasūtatayā tadavyabhicāraniyamāt tatra pramāṇam iti | tat kim anumānena vahniṃ vyavasthāpya bhāvayato yad vahniviṣayamativiśadavijñānaṃ tat pramāṇam iti | om iti brubāṇasya parvatanitambārohaṇe satīndriyasannikarṣajanmano dahanavijñānasya bhāvanādhipatyaviśadābhavijñānena saha saṃvādaniyamaprasaṅgaḥ | visaṃvādaś ca bahulam upalabhyate | lakṣaṇayogini ca vyabhicārasambhave tallakṣaṇam eva bādhitam iti viśadābham api prātibham iva saṃśayākrāntam apramāṇam | tadbhāvanāyā bhūtārthatvaṃ na tajjaviśadābhavijñānaprāmāṇyahetuḥ, vyabhicārāt | etañ ca prāsarpakasyeva saktukarkarīprāptimūlalābhamanorathaparamparāhito draviṇasambhārasākṣātkāras tathāgatasya nirātmakasamastavastusākṣātkāra ity āpatitam | sarvārthavastubhāvanāparikarmitacittasantānavartivijñānaṃ pratyālambanapratyayatvam arthamātrasya | 
	\pend
      

	  \pstart tathā ca tadutpatteḥ tadavyabhicāraniyama iti cet | na | arthasya hy ālambanapratyayatvavijñānaṃ pratīndriyāpekṣatvena vyāptam | tac cāsmāt svaviruddhopalabdhyā vyāvartamānam ālambanapratyayatām apy arthasya nivartayati | na khalv indhanaviśeṣo dhūmahetur iti vināpi dahanaṃ sastreṇāpi saṃskārair dhūmam ādhatte | tadādhāne vā samastakāryahetvanumānocchedaprasaṅgaḥ | bhāvanāyāś ca bhūtārthāyā arthānapekṣāyā eva viśadavijñānajananasāmarthyam upalabdhaṃ kāmāturādivartinyā iti bhūtārthāpi tannirapekṣaiva samartheti nārthasyālambanapratyayatvaṃ śakyāvagamam | api ca ālambanapratyayāpi ta evāsya kṣaṇā yujyante, ye tasya purastāt tanā avyavadhānās tathā ca ta evāsya grāhyā na punaḥ pūrvatarāḥ | tatkālā anāgatāś ceti na sarvaviṣayatā | atha dṛśyamānā dhātutrayaparyāpannāḥ prāṇabhṛto janmāntaraparivartopāttātītānāgataskandhakadambakopādānopādeyātmāna iti taddarśanaṃ dṛśyamānatādātmyena tadviśeṣaṇatayātītānāgatam api gocarayati | na cāsmadādidarśanasyāpi tathātvaprasaṅgaḥ, rāgādimalāvṛtatvāt | tasya ca bhagavato nirmṛṣṭanikhilakleśopakleśamalaṃ vijñānamanāvaraṇaṃ paritaḥ pradyotamānam ālambanapratyayaṃ sarvākāraṃ gocarayet | tasya ca sākṣāt paramparayā ca kathañcit sarveṇa sambandhād deśakālaviprakīrṇavastumātraviśiṣṭasvabhāvatayā tathaiva gocarayet | na caitat sarvagrahaṇam antareṇeti sarvaviṣayam asya vijñānam anāvaraṇaṃ siddham | 
	\pend
      

	  \pstart tad anupapannam | vicārāsahatvāt | tathā hīyam ālambanapratyayasya sarvaviśiṣṭātmatā bhāvikī na vā | bhāvikī cet | na tāvat sarvasminn ālambanapratyaye caikā sambhavati | ekasyānekavṛttitvānupapatteḥ | nānā cet | ālambanapratyayāś ca sarve ceti tattvam | tathā ca na sambandha iti na tadgrahaṇe sarvagrahaṇam | vikalpāropitatayā tv avikalpakaṃ samastavastuviṣayaṃ sarvatra pratīyata iti subhāṣitam | svālambanapratyayamātragocaram evāvikalpakaṃ samastavastuviśiṣṭālambanādhyavasāyajananam tenādhyavasāyānugatavyāpāram avikalpakam api samastavastuviṣayaṃ bhavati | yad āha 
	\pend
      
	    
	    \stanza[\smallbreak]
	vyavasyantīkṣaṇād eva sarvākārān mahādhiyaḥ | \edtext{}{\lemma{|}\Bfootnote{(PV III 107)}}\&[\smallbreak]


	

	  \pstart iti cet | atha katipayavastvālambanānubhavasya kutastya eṣa mahimā yataḥ samastavastvavasāya iti | rāgādyāvaraṇavigamād iti cet | tarhi yathāvad vastūni paśyet | na punar asmād apārthatvam asyeti | tad ayuktaṃ vikalpanirmāṇakauśalam asya yujyeta | tattvāvarakatā hi sulabhamalānāṃ kleṣādīnāṃ na punarvikalpanirmāṇapratibandhatā | tasmād bhāvanāprakarṣamātrajatvāt, arthāvyabhicāraniyamābhāvāt, viśadābham api saṃśayākrāntatvād apramāṇam apratyakṣaṃ ceti sāmpratam || 
	\pend
      

	  \pstart yad api sadarthaprakāśanaṃ buddheḥ svabhāvo 'sadarthatvaṃ cāgantukam iti, asati bādhake sadarthatvam eveti, tad ayuktam | anumitabhāvitavahniviṣayaviśadābhajñānaprāmāṇyaprasaṅgāt tadvidhasya kvacid bādhadarśanād aprāmāṇyam ihāpi samānam | anyatrābhiniveśāt | tad iha yadi viśadābhavijñānahetutvaṃ bhāvanāyā viśeṣaṇatrayayogena sādhyate, tataḥ siddhasādhanam | bhavatu tathāgatas tathābhūtavijñānavān | na tv etad vijñānam asya pratyakṣam apramāṇatvāt | tathā cāpakṣadharmatayā hetor asiddhatā | prasiddhadharmaṇo dharmiṇo 'jijñāsitaviśeṣatayā anumeyatvābhāvāt | atha pratyakṣavijñānahetutā bhāvanāyāḥ paraṃ pratyasiddhā sādhyate, tathā ca sati sādhyaviparyayavyāpter viruddhatā hetoḥ, viśeṣaṇatrayavatyāpi bhāvanāyā viśadābhabhrāntavijñānajanakatvāt | dṛṣṭāntasya ca sādhyahīnatvāt | yadā ca bhūtārthabhāvanājanitatve 'pi nāsya prāmāṇyam abhūtārthatvāt, tadā yad ucyate, 
	\pend
      
	    
	    \stanza[\smallbreak]
	nirupadravabhūtārthasvabhāvasya viparyayaiḥ | &na bādhā yatnavattve 'pi buddhes tatpakṣapātataḥ || \edtext{}{\lemma{||}\Bfootnote{(PV I 223; II 210)}}\&[\smallbreak]


	

	  \pstart iti | tad anupapannam | bhūtārthatve 'pi hi buddheḥ tatpakṣapātitā bhūtārthaiḥ pratipakṣair bādho na bhavet | abhūtārthā tv iyaṃ sātmībhāvam āpannāpy ātmātmīyadṛṣṭir iva sambhavadbādhā | tasmāt pratipakṣavivṛddhimātram | na tv ātyantikī vivṛddhiḥ sambhavati | yayā samūlakāṣaṃ kaṣitā doṣā na punar udbhaviṣyanti | ata evāsthirāśrayatve 'pi apunaryatnāpekṣatve 'pi asya nātyantikī niṣṭhā sambhavati | ātmātmīyadṛśa iva virodhipratyayasambhavāt | tatsambhavaś cābhūtārthatvāt | śrutānumitaviṣayaṃ tu pratyakṣaṃ na sambhavaty eva | tayoḥ parokṣarūpāvagāhitvāt | pratyakṣasya ca tadviparītatvāt | tadgatabhūtābhūtārthānuvidhāyitvena svaviṣaye śrutānumānajñānāpekṣayā prāmāṇyānupapatteś ca || 
	\pend
      

	  \pstart tat siddham etat bhūtārthabhāvanāprakarṣaparyantajavijñānam apratyakṣam arthe 'prāmāṇyāt | yad apramāṇaṃ tad apratyakṣam arthe | yathā kāmātur asya kāminīvijñānam | apramāṇaṃ ca tat | nitāntaviśadābhatve sati bhāvanāprakarṣajatvāt | yan nitāntaviśadābhatve sati bhāvanāprakarṣajaṃ vijñānaṃ tad apramāṇam | 
	\pend
      

	  \pstart yathānumitabhāvitavahniviśadavijñānam iti | samānahetujatvaṃ samānarūpatayā vyāptam | yad āha 
	\pend
      
	    
	    \stanza[\smallbreak]
	tadatadrūpiṇo bhāvās tadatadrūpahetujāḥ \edtext{}{\lemma{tadatadrūpahetujāḥ}\Bfootnote{(PV III 251ab)}}\&[\smallbreak]


	

	  \pstart iti | tad asya prāmāṇyaṃ nivartamānaṃ tulyahetujatvam api nivartayati | na caiṣa bhūtārthabhāvanāprakarṣaparyantajo 'nindriyasannikṛṣṭānumitabhāvitavahnivaiśadye ca nirātmakasamastavastuvaiśadye ca viśiṣyate | na ca rāgādyāvaraṇaviraho viśeṣaḥ | na khalv ete kambalādivad āvarakā vijñānasya | kiṃ tu tadākṣiptamanā vividhaviṣayabhedatṛṣṇādiparipluto na śaknoti bhāvayitum iti bhāvanādaramātra eva tadvirahopayogaḥ | asti cehāpi śiśirabharasambhṛtajaḍimamantharatarakāyakāṇḍasyānumitavahnibhāvanābhiyoga iti na hetubhedataḥ pratibandhasiddhiḥ | na caikapārthivāṇusamavāyikāraṇajanmabhir abhinnauṣṇyāpekṣaikavahnisaṃyogāsamavāyikāraṇair gandharasarūpasparśair nānāsvabhāvair vyabhicāraḥ | sāmarthyavaicitryād ekatve 'pi pārthivasya paramāṇoḥ | tadvaicitryaṃ ca kāryavaicitryopalambhāt | tac ca nityasamavetaṃ nityam, kāraṇasāmarthyaprakrameṇa ca pārthivāvayavini kārye jāyata iti avadātam | pariśiṣṭaṃ tu granthavyākhyānasamaye vyākhyāsyāmaḥ | tadāstāṃ tāvat || 
	\pend
      

	  \pstart \persName{trilocanas} tu \name{nyāyaprakīrṇake} prāha | iha kila duḥkhasamudayanirodhamārgākhyānyāryāṇāṃ satyāni catvāri | teṣām satyānāṃ svarūpasākṣātkārijñānaṃ yogipratyakṣaṃ | tatra duḥkhaṃ phalabhūtāḥ pañcopādānaskandhāḥ | tac ca svarūpato jñātavyam | ta eva hetubhūtāḥ samudayaḥ | sa ca prahātavyaḥ | niḥkleśāvasthā cittasya nirodhaḥ | sa ca sākṣātkartavayaḥ | tadavasthāprāptihetur nairātmyakṣaṇikatvādyākāraś cittaviśeṣo mārgaḥ | sa ca bhāvayitavya iti saugatamatam |
	\pend
      

	  \pstart atrocyate | mārgas tāvat pramāṇapariśuddho na bhavatīty uktaṃ prāk | ato 'bhūtaviṣayasya vikalpasyābhyāsād asatyārthavijñānaṃ syān na saṃvādi | api ca pramāṇapariśuddhamārgavādī śākyaḥ pramāṇaṃ pṛṣṭaḥ san sattvākhyaliṅgajaṃ vikalpaṃ brūyāt | tato yāvad vikalpena darśitarūpaṃ tat sarvam asat | śabdasaṃsṛṣṭatvāt | tasmiṃs ca bhāvyamāne sattve bhāvakasya vikalpakasya bhāvanopahite viśadābhatve śabdasamsṛṣṭagrāhyanimittaṃ vikalpakatvaṃ nivartate | tadvyāvṛttau grāhyam api śabdasaṃsṛṣṭaṃ nivartate | ato nirvikalpakam api yogijñānaṃ nirviṣayaṃ prasaktam | yat tu pāramārthikaṃ vastvātmakaṃ na tatpramāṇapariśuddham | śuddhau vā bhāvanayā | bhāvyasya sākṣādvijñātatvāt | na cānyasmin śabdasaṃsṛṣṭe bhāvyamāne sphuṭam anyad rūpaṃ bhavati | śokātur asyāpi niruddhendriyavyāpārasya tanayabhāvanāyāṃ mitrādipratibhāsaprasaṅgāt | 
	\pend
      

	  \pstart kṣaṇikatve bhāvye samāropite vāstavaṃ kṣaṇikatvam eva yogivijñānapratibhāsīti cet | na | satyāsatyayor ekatvābhāvātmake hi bhede 'satyabhāvane 'pi yadi satyapratibhāsaḥ, tarhi satyatanayābhyāse 'pi śabdasāmyād abhedinas tanayasaṃjñakasya kasyacid aparasya svarūpapratibhāsaprasaṅgaḥ | tasmād abhūtaviṣayābhyāsaṃ nirvikalpakam api saṃvādān na pramāṇam iti na sarvajñasiddhiḥ | 
	\pend
      

	  \pstart api ca bhāvyasya vastunaḥ punaḥ punaś cetasi niveśanam abhyāsaḥ | sa ca brahmacaryeṇa tapasā sādaraṃ dīrghakālaṃ nirantaramāsevito dṛḍhabhūmir asphuṭākārasya vikalpasya sphuṭābhatvajanana iṣṭaḥ | sa kṣaṇikatvanairātmyavādinā draḍhayitum aśakyaḥ | tathā hi bhāvyagrāhī yādṛśo vikalpa utpannas tādṛśa eva niranvayaṃ nirudhyate | tasmiṃś ca niruddhe punaḥ punar utpadyamānaḥ pratyayas tādṛśa evāpūrva utpadyate | tad anena paryāyeṇa kalpasahasre 'py apūrvotpatter aviśeṣān na tajjanyaḥ saṃskāro 'bhyāsa utpadyate | etena viśiṣṭavijñānotpādo 'bhyāso vyākhyātaḥ | niranvayaniruddhaṃ hi pūrvapūrvavijñānaṃ katham uttarāvasthāntaraṃ viśiṣtaṃ janayet | sarvathā kramabhāvibhiḥ pratyayair avasthitam eva rūpaṃ śakyaṃ saṃskartum | anavasthitaṃ tu svotpādavyayayogimātram ity aviśiṣṭaṃ syāt | tasmāt pratyāvṛttibhāvyavastupratyayajaḥ saṃskāro vyutthānapratyayasaṃskāravirodhī yasyāsti tasyaivātmanaḥ prakṛṣṭo 'pi bhāvyasākṣātkāripratyayahetur iti yuktaṃ paśyāmaḥ | kiṃ ca cittam ekāgraṃ vyavasthāpayituṃ vikṣepatyāgārtham abhyāso 'nuṣṭhīyate | na ca kṣaṇikavādināṃ vikṣiptaṃ cittam asti | pratyarthaniyatatayā sarvasya vittaikāgratvāt | tathā hi yadi sākāraṃ vikalpavijñānaṃ svapratibhāsaniyatatvāt ekāgram eva tat kathaṃ vikṣipyate | atha nirākāraṃ tathāpi vikalpakaṃ prati vikalpyaṃ bhinnam eva | na tu sarvavikalpānāṃ vikalpyam asti | tato nirākāram api vijñānaṃ niyatālambanatvād ekāgram eva, na vikṣiptam | sarvathā nāsti kṣaṇikavādinām ekam anekārtham avasthitaṃ cittaṃ yad ekāgraṃ kartum iṣyate | tad evam abhyāsānupapatter asarvajñavatyāṃ cittasantatau na ca vijñānaviśeṣaḥ sarvajñaḥ sidhyatīti || 
	\pend
      

	  \pstart nyāyabhūṣaṇakāras tv āha | sarvajñānānāṃ nirālambanatve saṃvedamātratve ca yogītarapratyayayoḥ ko viśeṣaḥ | śuddhāśuddhatvam iti cet | bhavatu nāmaivam | tathāpi caturāryasatyādiviṣayatvam ayuktam | na hi svātmamātravedanena caturāryasatyādikam sākṣātkṛtam iti yuktam, atiprasaṅgāt. 
	\pend
      

	  \pstart tadākāratvena tadviṣayatvam iti cet, tat kim idānīṃ sautrāntikamatam abhyupagataṃ satyam | tathāpy atītānāgataviṣayatvaṃ katham | na hy asataḥ kaścid ākāro 'sti | dṛṣṭaśrutānumitākāraś ca yadi bhāvanābalataḥ spaṣṭa evāvabhāti, tathā ca sati bhrāntam eva yogipratyakṣaṃ syāt | avidyamānasya vidyamānākāratayā pratibhāsanāt, svapnavat | tathā 'visaṃvāditvān na bhrāntam | na | anumānajñānasya bhrāntatve 'pi avisaṃvāditvābhyupagamāt | 
	\pend
      

	  \pstart atha bhrāntasyāpi saṃvāditvena prāmāṇyam | tathāpi pratyakṣalakṣaṇasyābhrāntatvaviśeṣaṇaṃ virudhyate | na cāvisaṃvāditvam api tvanmate yuktam | yataḥ prāpyārthadarśakatvaṃ vā, pravṛttiviṣayopadarśakatvaṃ vā, avabhātād arthakriyāniṣpattir vā bhavatām avisaṃvāditvam abhipretam | na caitad atītādyarthajñāne sambhavati | vartamānārthajñānasyāpi kṣaṇikatvapakṣe nopapadyata eva | tasmāt saugatānāṃ yogipratyakṣopavarṇanam ayuktam eveti || 
	\pend
      

	  \pstart kiṃ cedam api vaktum ucitam | yady anumānapūrvakam artheṣu bhāvanābalajajñānam āśvāsabhājanaṃ, tadāstāṃ tāvad anumānapauruṣapratyāśā | pratyakṣeṇāpi cakṣurdahanādikaṃ gṛhītvā bhāvanāprakarṣaparyante jātaṃ sthirataraṃ tadākāravijñānaṃ syāt, yāvan na viparītabhāvanābhiyogaparyantaḥ | astaṃ gataś ca tadviṣayo 'vasthānataraprāpto veti kathaṃ pramāṇopanītavastugocaratve 'pi saṃvādāśvāsaḥ | api ca yadā hālika eva havyāśanam anumāya bhāvanayā sphuṭayet, tadā na tadyogijñānaṃ paramārthaviṣayābhāvād iti pratyakṣāntaraprasaṅgaḥ | 
	\pend
      

	  \pstart kiṃ ca tadyogijñānam indriyajñānād bhinnam abhinnaṃ vā | abhedapakṣe na yogijñānaṃ nāma pratyakṣeṇa bhinnam indriyajñānenaiva saṅgrahāt | na ca bhāvanopaskṛtasantānasya tathodayād bhedavyavasthā | rasāyanādisaṃskārāpekṣayāpi pratyakṣāntaravyavasthāprasaṅgāt | bhedapakṣe ca bhāvanāsambhavaṃ jñānaṃ kṣaṇikasākṣātkāri | indriyajñānaṃ ca syairyagrāhīti sādhvī siddhiḥ | indriyajñānasyāpi tadavasthāyām asthairyagrhaṇe kṛtaṃ yogijñānena | na ca tasyākasmikaḥ kṣaṇikatvāvabodhaḥ | bhāvanodbhūtavaiśadyasya hi tadbodhaḥ | na cendriyajñānasya bhāvanā | api tu manovijñāne | tām antareṇāpi sākṣāt kriyālābhe ca bhāvanāvaiyarthyam iti kāraṇābhāvād eva sarvajñapratihatiḥ || 
	\pend
      

	  \pstart atrābhidhīyate | yat tāvat sarvapadārthasaṃvedanasya kāraṇaṃ kim indriyajñānam ityādi valgitaṃ tatra bhāvanābalajaṃ manovijñānam eva sarvapadārthagrāhīti pañcama evāsmākaṃ pakṣaḥ | ataḥ pakṣāntarabhāvino doṣā anubhyupagamapratihatāḥ | yac cāsmadabhyupagate pañcame pakṣe dūṣaṇam uktam, anarthatvāt pratyakṣaśabdasya, tasya ca bhāvanābalāvalambino 'py anakṣajatvān nārthasākṣātkāritvam astīti, tad asaṅgatam | tathā hi pratyakṣaśabdasya tāvad akṣāśritatvaṃ vyutpattinimittam arthasākṣātkāritvaṃ tu pravṛttinimittam iti pratipāditam | na ca bhāvanābalāvalambino manovijñānasyānakṣāśritatve 'py arthasākṣārkaraṇe kaścid asti śaktipratighātaḥ | yathā hi cakṣurindriyaṃ svasāmarthyān atikrameṇa yogyadeśastham artham apekṣya svavijñānajanane pravartate, tathā sarvāvidyāparipanthibhūtārthabhāvanāsahitaṃ mana indriyam api yogyadeśastham arthaṃ prāpya svavijñānajanane pravartiṣyate | aprāpyakāritāyā ubhayoḥ sādhāraṇatvāt | arthavattāyāś ca manaso 'pi tadānīm iṣṭatvāt | pṛthagjanasya tu na tādṛśī śaktiḥ, yato netraśrotravanmano 'pi tādṛṅmaryādayā yogyadeśastham arthasahakāriṇam āsādya vedanam utpādayet, sarvāvidyonmūlakasya bhāvanāviśeṣasya sahakāriṇo 'bhāvād iti nātiprasaṅgaḥ | tadavasthāyāṃ tu śrutinayanayor iva manaso 'pi kiyaddūreṇa viṣayasannidhivyavasthitika eva pramātuṃ kṣamaḥ | kevalam etāvad ucyate | yāvat tena śakyam adhigantuṃ svākārārpaṇasamarthaṃ sahakāri vastu tāvad itarajanāsādhāraṇaṃ truṭyadrūpatayā tasya gocarībhavatīti | ata evārthākāro vastuto na bhāvanāmātrajanita iti na visaṃvādaśaṅkāpi | bhāvanayā punas tadīyasantāne netra ivāñjanaviśeṣeṇa śaktir atiśayavatī kācid arpitā yatparajanāsādhāraṇadarśanam asya | tasmād anakṣajatve 'pi amnovijñānasyārthasākṣātkāritvaṃ sambhavati | 
	\pend
      

	  \pstart nanu manaso bahirasvātantryam | anyathāndhabadhirādyabhāvaprasaṅgāt | uktaṃ ca yogināṃ dharmādharmayor aparokṣapratibhāsaṃ jñānaṃ nāsti | indriyasannikarṣābhāvād asmadādivad iti | 
	\pend
      

	  \pstart api ca arthasya hy ālambanapratyayatvam indriyāpekṣatvena vyāptam | tac cāsmāt svaviruddhopalabdhyā vyāvartamanam ālambanapratyayatām api tasya nivartayati | na khalv indhanaviśeṣo dhūmahetur iti vināpi dahanaṃ sahasreṇāpi saṃskārair dhūmam ādhatte | tadādhāne samastakāryahetukānumānocchedaprasaṅgaḥ | na ca bhāvanābalena kasyacid atīndriyadarśitvaṃ sarvajñatvaṃ vā dṛṣṭam iti cet | 
	\pend
      

	  \pstart atrocyate | manaḥśabdena tāvad asmākam anakṣajaṃ vijñānam evābhipretam | na cāsminn andhabadhirādyabhāvaprasaṅgaḥ | sarvāvidyāpratipakṣabhūtārthabhāvanālakṣaṇasya sahakāriviśeṣasyāndhādīnām abhāvāt | indriyasannikarṣābhāvād iti tv arthasākṣātkāritvamātrāpekṣayā sandigdhavyatirekitve anaikāntikī kāraṇānupalabdhiḥ | asmadvidhārthasākṣātkāritvāpekṣayā punaḥ siddhasādhanam || 
	\pend
      

	  \pstart asmadādiviśeṣaṇaśūnyasyārthasākṣātkāritvamātrasyaivendriyādhīnatva-darśanād anaikāntikatvam asambhavīti cet | yady evam arthasākṣātkāritvamātrasyendiryavadālokādhīnatvam upalabdham iti na santamase paśyeyur ulūkādayaḥ | atha vyabhicāradarśanād ālokasyāvyāpakatvam, vyabhicāraśaṅkayā tarhīndriyasyāpy avyāpakatvam | vyāptyā śaṅkā khaṇḍyata iti cet | śaṅkāsambhavād vyāptir evāsambhavinī yadi prathamata eva vyāptiḥ, vyabhicāro 'pi na dṛśyeta | 
	\pend
      

	  \pstart tasmād vyabhicāradarśanaṃ vyāptiśaithilyād eva | sati ca vyāptiśaithilye śaṅkāpi nyāyād āpatantī kena pratihanyate | ulūkādīnāṃ bhinnajātīyatvād ālokābhāve 'py arthasākṣātkāritvam astv iti cet | tarhi bhagavato 'pi bhūtārthabhāvanāprakarṣaparyantamahāpralayavāyunā nirastānādyāvipakṣasya saṃsārakūpapatitebhyaḥ prāṇibhyo 'sty evādbhūtavaijātyam iti yuktam asyāvidyāpratipakṣabhāvanātiśayasahitātmakāntarapratyayād ālambanapratyayāc ca sākṣādutpannasyendriyam antareṇārthasākṣātkāritvam | ataḥ kāraṇānupalabdhiḥ kāśikākārasya vyāpakaviruddhopalabdhiś ca vācaspateḥ sandigdhavyatirekitvād anaikāntikī | sandigdhavyatirekitvaṃ tu dūṣaṇam asmadīśvaradūṣaṇe prasādhitam || 
	\pend
      

	  \pstart tasmāt sādhāraṇakarmanirjātānām asmadādīnām arthasākṣātkāritvam indiryāpekṣatvena vyāptam iti siddhasādhanam | prasiddhānumānasya ca na kṣatir dṛśyatvopādher dhūmādeḥ pratyakṣānupalambhato vyāptigrahaṇāvirodhāt | sāṃsārikāgocarārthasākṣātkāritvamātrāpekṣayā tu sandigdhavyatirekitvam | adṛśyasya pratyakṣānupalambhābhyāṃ kenacid vyāptigrahaṇāyogāt | viparyaye bādhakapramāṇasya cāsambhavād iti | na cātīndriyadarśitvaṃ sarvajñatvaṃ vādarśane 'pi niṣeddhuṃ śakyate, adṛśyānupalambhato niṣedhāyogāt | kāraṇānupalambatas tanniśedha iti cet | kāraṇābhāvo 'pi adarśanamātrato na sidhyatīti tadavasthaḥ paribhavaḥ || 
	\pend
      

	  \pstart yad api kāśikākāreṇābhihitam, atha pramāṇāntarāvagataṃ bhāvyate, kiṃ bhāvanayā, tata eva tatsiddher iti | tad apy asaṅgataṃ | pramāṇāntaraṃ hy anumānam | na ca caturāryasatyasvarūpe vastutattve niścite sākṣātkāram antareṇa kleśajñeyāvaraṇakṣatir iti svārtham api tāvad bhāvanā yuktimatī | tattvasākṣātkāriṇi ca cittasantāne sati śakyasākṣātkriyam idam ity anye 'pi niścayānantaraṃ sākṣātkriyāyai pravartyante, tadupadiṣṭasvargasādhanaṃ cārthabhāvanayānusarantīti svargāpavargalakṣaṇaparārthasiddhaye ca bhāvanā saphaleti | anyathā tattvāsākṣātkāriṇo lokānatikrāntasya vacanam anādeyam eva syād iti kva parārthavārtāpi | yac ca kiṃ ca tatpramāṇam ityādy ārambhya tasmād bhūtam abhūtaṃ vety etatparyantena dharmādharmayor anumānāpravartanam uktam, tatra dharmādharmaśabdena kim abhipretam | yadi kṣaṇikanirātmakavastu tattvam, tadā tasya pratyakṣeṇāniścaye 'pi yathā viparyaye bādhakapramāṇabalena vyāptisaṃvedanaṃ tathā kṣaṇabhaṅgasādhanāvasare vyavasthāpitam | atha vastūnāṃ svargādisādhanatvam abhipretam, tadā tadviṣayaparijñānāprasādhane 'pi nāsmākaṃ kācit kṣatiḥ | saparikarasaṃsāranirvāṇaparijñānenaivopayuktasarvajñaprasādhanāt | yad āhuḥ: heyopadeyatattvasyetyādi (PV I 217a) | 
	\pend
      

	  \pstart yad api, api ca bhāvanābalajaṃ gṛhītagrahaṇād apramāṇam ity uktam, tatra gṛhītaṃ nāma pratyakṣeṇānumānena vā | pramāṇāntarasyābhāvāt | na tāvat pratyakṣaṃ kṣaṇikatvādāv arvācīnasya kasyacid asti | anumānena caikavyāvṛttiviśiṣṭe vastutattve 'vasite 'pi sarvātmanā spaṣṭavastutattvasākṣātkāri pratyakṣaṃ na gṛhītagrāhi, anumānena vastutattvāsparśanāt | na ca taduttarakālam ityādi tu kāraṇānupalabdhidūṣaṇaprastāve prativyūḍham iti | 
	\pend
      

	  \pstart yad api \persName{vācaspatinā} satyam ityādinā punaḥ punar uttarottaram āśaṅkya tat kim anumānena vahniṃ vyavasthāpyetyādinā bhāvanābalajasyānumānapūrvakatve visaṃvādam upadarśyopasaṃhṛtam, tan na bhāvanāyā bhūtārthatvaṃ tajjaviśadavijñānaprāmāṇyahetuḥ, vyabhicārād iti | tad asaṅgatam | tathā hy ayaṃ vahniviṣaye 'numānapūrvakabhāvanābalataḥ spaṣṭavahnipratyayaḥ kiṃ vahner apy utpannaḥ, tathābhūtabhāvanāmātrād eva vā |
	\pend
      

	  \pstart parathampakṣe visaṃvādaś ca bahulam upalabhyate iti yad uktaṃ tad durbhāṣitam | sākṣād arthād utpannasyāpi visaṃvādasambhave 'nyasyāpi pratyakṣasya hastakatyāgaprasaṅgāt | 
	\pend
      

	  \pstart dvitīyapakṣe tu bhāvanāprakarṣamātrajasyārthād anutpannasya bahulaṃ visaṃvādopalambhe 'pi bhāvanārthābhyāṃ sākṣād utpannasya yogipratyakṣasyāpi visaṃvādasambhava iti sthavīyasī bhrāntiḥ | 
	\pend
      

	  \pstart nanu yadīndriyaṃ vināpi bhāvanārthābhyāṃ yogijñānam utpadyate, tarhi parvate bhāvanāvahnibhyāṃ vahnijñānam utpadyatām avisaṃvādi | visaṃvādaś ca bahulam upalabhyata iti cet | na | sākṣād vahner utpāde sati visaṃvādābhāvāt | kevalam utpāda eva durāpaḥ | na hi vayaṃ pramāṇadṛṣṭavastubhāvanāsahitaṃ mana indiryam arthasvarūpagrāhijñānaṃ janayatīti brūmaḥ, api tv asaddṛṣṭilakṣaṇāvidyāparipanthikṣaṇikanairātmyalakṣaṇasarvavastutattvabhāvanāsahitam | na ca vahnitvaṃ sarvavastutattvam, kiṃ tu kṣaṇikanairātmyam eveti kṣaṇabhaṅgaprasādhanataḥ pratipāditam iti | kiṃ ca svamanīṣāparikalpitaḥ khalv ayam anumitabhāvitavahniviṣayaviśadaḥ pratyayaḥ | na punar asya loke sambhavaḥ | tathā hi niṣprayojanam anunmatto na kaścid bhāvayati | prayojanaṃ ca śiśirabharamanthakāyakāṇḍasyāpi dāhādimātram eva, tac cānumitenaiva vahninā taddeśopasarpaṇāt sidhyati | anupasarpaṇe bhāvānāvaiyarthyam | purastāt tu bhāvite parisphurati tadathāpekṣayā bhrāntiḥ prāsarpakasyevetyādy upahāsyam apy asya kṣatātmano durnītipūtigavībhakṣaṇādhmātajaradgomāyor udgāra iva satām asahyaḥ | 
	\pend
      

	  \pstart yad api tato 'nantaramāśaṅkyārthasyālambanapratyayatvam indriyāpekṣitvena vyāptam iti prasādhitam, tatpūrvam eva pratyuktam | tathā bhāvanayās cetyādyāśaṅkyārthasyālambanapratyayatvam aśakyāvagamam iti yad uktaṃ tad apy asambaddham | 
	\pend
      

	  \pstart cakṣurindriyasyāpy artham antareṇa dvicandrakeśoṇḍukādau viśadabhrāntajñanajananasāmarthyam upalabdham ity arthasahitam api kevalam eva samartham | ato ghaṭāder apy ālambanapratyayatvam aśakyāvagamam iti indriyapratyakṣam api pratihataṃ syād iti | tathāpi cālambanapratyayāpi ta eva yujyanta ityādir na punar vikalpanirmāṇapratibandhateti paryanto vyarthaḥ | asmābhir evaṃvidhasya prastute 'nabhyupagatatvāt | ata eva tasmād bhāvanāprakarśamātrajatvāt, arthāvyabhicāraniyamābhāvāt, viśadābham api saṃśayākrāntatvāt, apramāṇam apratyakṣaṃ ceti sāmpratam ity upasaṃhāro 'pi dhikkāraḥ | sarveṣām eva hetūnām asiddhatvāt | bhāvanābalajasyārthād apy utpatter indriyapratyakṣavat | sadarthaprakāśanaṃ buddheḥ svabhāva ityādy asmākam api manoharam | bhāvanāyāś ca sāmānyena sphuṭābhajñānahetutvaṃ sādhyate | pramāṇopannacaturāryasatyaviṣayaniṣṭhāyāṃ tu sāmarthyāt pratyakṣapramāṇahetutāpi sādhyate | ata eva kāminīpratibhāsasyāpramāṇatve 'py apratyakṣatve 'pi sphuṭābhatvasya sādhyadharmasāmānyasya sambhavāt na viruddho hetuḥ | nāpi dṛṣṭāntasya sādhyaśūnyateti | na ca nairātmyadṛṣṭiḥ sambhavadbādhā, arthād utpatter abhūtārthatvābhāvāt | 
	\pend
      

	  \pstart śrutānumitaviṣayaṃ pratyakṣaṃ na sambhavatīty apy ayuktam | āgamānumānayor dvividho viṣayaḥ grāhyo 'dhyavaseyaś ca | tatra grāhyaḥ svākāraḥ, adhyavaseyas tu pāramārthikavastusvalakṣaṇātmā | asya ca parokṣatve 'numānasāmagrīsambhave 'numānaviṣayatvam, pratyakṣasāmagrīsambhave ca krameṇa pratyakṣaviṣayatvaṃ dṛṣṭam eva | tat siddham ityādyupasaṃhāro 'pi paryākula eva | apramāṇatvād iti hetuś ca prathamo 'siddhaḥ | bhāvanābalajasyārthād apy utpatteḥ, pramāṇaśaktisambhavāt, indriyapratyakṣavat | bhāvanābalajatvād iti dvitīyas tu sandighavyatirekitvād anaikāntikaḥ | tathā yathānumitabhāvitavahniviṣayaviśadajñānam iti dṛṣṭānto 'py asambhavīti pratipāditam | bhavatu vā, tathāpi yogijñānasya tena saha tulyahetutvam asiddham | tad dhi pramāṇadṛṣṭavastubhāvanāmātrajam | yogijñānaṃ tv avidyāpratipakṣasarvavastutattvabhāvanāviṣayābhyām utpannam iti mahāntam api viśeṣam asau durmatiprapātapatito nāvagāhata ity upekṣaṇīyaḥ || 
	\pend
      

	  \pstart nyāyaprakīrṇe tu mārgas tāvat pramāṇapariśuddho na bhavatīty uktaṃ yat, tat tatprasādhakapramāṇenaiva prayuktam | 
	\pend
      

	  \pstart yac cāpi cetyādy ārabhya yogijñānaṃ nirviṣayaṃ prasaktam ity uktam tatra keyaṃ nirviṣayatā nāma | kiṃ vikalpākāranivṛttau nirākāratā, arthākārād visadṛśākāratā, atha tadākāratve 'pi tadvastusaṃsparśitā | 
	\pend
      

	  \pstart na tāvat prathamaḥ pakṣaḥ kṣamaḥ | jñānasya nirākāratānupapatteḥ | 
	\pend
      

	  \pstart nāpi dvitīyaḥ | kāminyādibhāvanāyās tadākārasyaiva viśadasya darśanāt | 
	\pend
      

	  \pstart na ca tṛtīyaḥ | arthasamarpitākārasaṃsparśam apāsyānyasyārthasaṃsparsasyāyogāt | 
	\pend
      

	  \pstart tathā coktam:
	\pend
      
	    
	    \stanza[\smallbreak]
	arthena ghaṭayatyenām | \edtext{\textsuperscript{*}}{\lemma{*}\Bfootnote{(PV III 305a)}}\&[\smallbreak]


	

	  \pstart ityādi
	\pend
      

	  \pstart tayoś caikatvenādhyavasāyād bāhya eva pravṛttinivṛttī, vyāvahārikasya sphuṭībhāvo 'pi bahirabhimatasya paryante vikalpopādeyakṣaṇasyaiva sphuṭasyodayaḥ | tāvataiva sa viṣayas tena sākṣātkṛta iti vyavahāraḥ kevalam arthād apy utpattau | anyathā vyabhicārād aprāmāṇyam | na ca vikalpopadarśitam api rūpam avastu jñānātmakatvāt | anātmakatve prakāśāyogāt | tadbhāvanaiva cārthabhāvanā, tatsphuṭībhāva eva bāhyasphuṭībhāvaḥ, prakārāntareṇa bāhyasparśāyogāt | etena yat pāramārthikam ityādi na sarvajñasiddhir itiparyantaṃ prayuktam | 
	\pend
      

	  \pstart yac cāpi cetyādi na yuktaṃ paśyāma itiparyantena dūṣaṇam uktam, tad apy asaṅgatam | tathā hi yādṛśa eva bhāvyagrāhī pratyayaḥ prathamo niranvayo niruddhas tādṛśa evāpara utpadyata iti niyamaniścayakāraṇaṃ na kiñcid asti caṇḍadevatāsparśād anyat, kṣaṇikatvād iti cet | nanu kṣaṇikatvaṃ sthāyitayā virudhyate na visadṛśotpādena, tad dhi prācīnaṃ niranvayanirodhe yathā sadṛśakṣaṇāntaram ārabhate tathā svahetugatasāmarthyayogāt kāryotpādānumeyād yadi viśeṣaleśaviśiṣṭaṃ kṣaṇāntaram utpādayati, tadā na kācit kṣatiḥ | na hi bhavata iva bhāvasyāpi kṣaṇikatāyāṃ pradveṣo nāma | tasmān na kṣaṇikatvottaraviśiṣṭakṣaṇajanakatvayor virodha iti nāpārthako 'bhyāsaḥ | 
	\pend
      

	  \pstart yac cedaṃ kiñcetyādinā kṣaṇikatve cittam avikṣiptam āveditam, tad apy asādhu | nairātmyāditattvaparāṅmukhasya sarvasyaiva vikṣiptatvāt | bhāvanābalena tattvasākṣātkāriṇaḥ samāhitatvāt | atha ca tattvasākṣātkriyālābhāt grāhakākārāvagrahasambhavāt ca vyāvahārikam api vikṣiptam asti cittam | yato mamaiva doṣakṣayo bhāvīti mārgāmyāsapravṛttir abhyāhateti | paramārthataḥ prāpyādīnām abhāve 'pi tatsaṃkalpasyaivānādyavidyāprabhāvitasya sarvatra pravartakatvāt | ata eva mārgasatyābhyāsāt siddhaḥ sarvajñaḥ | 
	\pend
      

	  \pstart nyāyabhūṣaṇasyāpi yogācārāpekṣayā dūṣaṇam aprastutam | bahirarthābhyupagamenaiva sādhanaprakramāt | yac coktam tathāpy atītānāgataviṣayatvaṃ katham, na hy asataḥ kaścid ākāro 'stīti, tad etat prastāvān avagāhanaphalam | upayuktasarvajñādhikāreṇa hi sarvakṣaṇikanirātmakavastubhāvanopakṣepaḥ, na sarvasarvajñāpekṣayā | tato 'tītānāgatam apratīyamānam api na bādhakam | tāvataiva duḥkhanirodhasiddheḥ | parasmai ca kṣaṇikatvādiniṣṭhakasya deśanāvatārāt | na ca sarvasarvajñahastakatyāgaḥ | tathā hi caturāryasatyasākṣātkāraprāptau nirāvaraṇāntaḥkaraṇasya kāruṇyātiśayāt sarvākāraparārthaparatayā sakalagocaracāriṇi cetasi ciravirūḍhotsāhasya tādṛgupāyaviśeṣādhigamo bhavaṣyati, yam anutiṣṭhatas tadutpattim antareṇāpi devatādhipatyāt satyasvapnavat | pratiparamāṇusarvaviṣayaṃ yathā deśakālākārapratyavasthānukāri sphuṭataraṃ jñānam udiyāt, tadā na tāvad vastuvyabhicārakṛtaṃ visaṃvāditvam, vastūnām eva pratibhāsanāt | utpattisārūpyābhyāṃ vedyasthitir iti tu pṛthagjanāpekṣayā | yoginas tu sārūpyamātreṇaiva grahaṇam iti nyāyaḥ | 
	\pend
      

	  \pstart yad Vārttikam 
	\pend
      
	    
	    \stanza[\smallbreak]
	aviśuddhadhiyaḥ prati |&grāhyagrāhakacinteyam acintyā yogināṃ gatiḥ || iti | \edtext{}{\lemma{|}\Bfootnote{(PV III 532)}}\&[\smallbreak]


	

	  \pstart tad evaṃ bhāvibhūtayor ajanakayor api yogijñāne sphuraṇam abādhyam | bhāvibhūtayos tarhi yadi svarūpasya sphuraṇam, vartamānataiva syāt | atha svarūpam asannihitaṃ jñānam eva tadākāram iti nirālambanaṃ niyamena | tad api nāsti | yasmād asannihite 'py arthe bhāvanābalāt taddeśakālākārānukāri vijñānaṃ katham anālambanam | tathātvenādhyavasāyāc ca, adhyavasitakālaviśiṣṭasyaiva satyasvapnavat tasya prāpteḥ | 
	\pend
      

	  \pstart yad \name{Bhāṣyam}
	\pend
      
	    
	    \stanza[\smallbreak]
	yathā sa dṛṣṭaḥ śaradādikālayuktas tathā tasya na bādhitatvam | &tatkālayuktas tu na tena dṛṣṭas tathāpratītāv api nāsti doṣaḥ || \edtext{}{\lemma{||}\Bfootnote{(PVA II 615)}}\&[\smallbreak]


	

	  \pstart jñānamātrasya tu tattvataḥ sphuraṇāc ca na vartamānatāprasaṅgaḥ saṅgataḥ | tathā kṣaṇikatvapakṣe 'pi ekatvādhyāropasāmarthyān na vyavahārikaṃ prati pramāṇasya kācit kṣatir iti śāstre prapañcitam | 
	\pend
      

	  \pstart yad api kiñ cedam api vaktum ucitam ityādy ārabhya bhāvanābalajasyānumānapūrvakatve 'pi pratyakṣapūrvakatve 'pi vyabhicārābhidhānam, tadarthād api bhāvanābalajasya sākṣādutpattisvīkārād apahastitam | yathendriyajasyāpi dvicandrādijñānasyārthād anutpatter aprāmāṇyam, arthendriyābhyām utpattau tu prāmāṇyam evaṃ pramāṇapūrvakasyāpi bhāvanāmātrād utpannasyāprāmāṇyam, bhāvanārthābhyām utpannasya tu prāmāṇyam | 
	\pend
      

	  \pstart yadi yogijñānasyārthād utpattiḥ, pramāṇapūrvakatvāpekṣayā na kiñcit prayojanam iti cet | na | deśakālavastuviśeṣam apāsya sāmāneyana sarvadikkālavartivastumātraṃ kṣaṇikanirātmakam ity aniścaye mahāprayāsasādhyapuruṣāyuṣavyāpinyāṃ bhāvanāyām eva pravṛtter abhāvāt | na ca hāliko havyāśanam anumāya sphuṭīkaroti yena pratyakṣāntaratvaprasaṅgaḥ | asāmarthyavaiyarthyābhyāṃ tadasambhavapratipādanāt | 
	\pend
      

	  \pstart yad apy uktaṃ yogino jñānam indriyajñanād abhinnaṃ bhinnaṃ vā | tatra prathamapakṣe tāvan na vastudoṣaḥ | tādṛkpuruṣaviśeṣasya siddhatvāt | vyavasthādūṣaṇam api nāsti | sādhyatayaiva tādṛgdaśāviśeṣasya lokātikrāntātiśayasya paramapuruṣārtharūpasya sādhanaviśeṣapratipādanāya pṛthagjanasādhāraṇendriyajñānād bhedena nirdeśāt | paramapuruṣārthaviṣayatvābhāvād eva ca rasāyanādisaṃskārajasyāpi jñānasya na pratyakṣāntaratā | bhedapakṣe 'pi na tāvat sthairyetarasphuraṇakṛtopālambhasambhavaḥ | indriyajñānenāpi vastu sarvātmanā gṛhṇatā truṭyadrūpasyaiva grahaṇāt | adhyavasāyo hi pūrvaṃ durllabhaḥ idānīṃ tu bhāvanābalanirdalitāvidye cittasantāne so 'pīndriyajñānena janyata iti viśeṣaḥ | 
	\pend
      

	  \pstart nanu yogino manovijñānendriyajñānābhyāṃ paśyata ākāradvayasphuraṇaprasaṅga iti cet | satyam | satyajñānākāras tāvad vastuno na bhinnadeśo 'nyatarabhrāntiprasaṅgāt | atas tāv ākārāv apratimau kayā gatyā sphurata iti ko nirṇetuṃ kṣamaḥ | yad āha: acintyā yogināṃ gatir iti |\edtext{}{\lemma{|}\Bfootnote{(PV III 530d)}}
	\pend
      

	  \pstart sarvathā tu na yogijñānasya kṣatir iti siddham | tad evaṃ kāraṇānupalambhād api na sarvajñatābhāvaḥ | 
	\pend
      

	  \pstart nanu yadi nāma yuṣmadabhimatasyānumānasya na bādhakam, tathāpy asaty evānumānaṃ bādhakam | tathā hi śakyam idam abhidhātum 
	\pend
      

	  \pstart sugato 'sarvajñaḥ | jñeyatvāt, prameyatvāt, sattvāt, puruṣatvāt, vakṛtvāt, idriyādimattvād ityādi | rathyāpuruṣavat | 
	\pend
      

	  \pstart tathā ca Bṛhaṭṭīkā 
	\pend
      

	  \pstart yasya jñeyaprameyatvavastusattvādilakṣaṇāḥ | 
	\pend
      

	  \pstart nihantuṃ hetavaḥ śaktāḥ ko nu taṃ kalpayiṣyati || \edtext{}{\lemma{||}\Bfootnote{(=TS 3157)}}
	\pend
      

	  \pstart Kārikāpi 
	\pend
      

	  \pstart pratyakṣādyavisaṃvādi prameyatvādi yasya ca | 
	\pend
      

	  \pstart sadbhāvavāraṇe śaktaṃ ko nu taṃ kalpayiṣyati | \edtext{}{\lemma{|}\Bfootnote{(ŚV II 132)}}
	\pend
      

	  \pstart atrocyate | kim ete jñeyatvādayaḥ sarvajñatvena sākṣād viruddhāḥ paramparayā vā | aviruddhavidhāne pratiṣedhāyogāt | sa ca sākṣād virodhaḥ parasparaparihārasthitilakṣaṇo vā, bhāvābhāvavat, sahānavasthānalakṣaṇo vā, dahanatuhinavad iti | 
	\pend
      

	  \pstart na tāvad ādyaḥ pakṣaḥ | yad vyavacchedanāntarīyako yasya paricchedas tayor eva parasparaparihārasthitilakṣaṇo virodhaḥ | na ca jñeyatvādi sarvajñatvavyavacchedena sthitam | kiṃ tarhi | ajñeyatvādivyacacchedena | tathā sarvajñatvam asarvajñatvavyavacchedena, na tu jñeyatvavyavacchedena | 
	\pend
      

	  \pstart nāpi dvitīyo virodhaḥ | yasya hy avikalakāraṇasya bhavato yat sannidhānād abhāvas tayor eva sahānavasthānalakṣaṇo virodhaḥ | na ca sarvajñatvaṃ prāk pravṛttam avikalakāraṇaṃ dṛṣṭaṃ yena paścāj jñeyatvādisadbhāve nirvartata iti syāt | tathātve sati deśādiniṣedha eva bhaven na tu sarvathoccheda iti | 
	\pend
      

	  \pstart na ca paramparayā virodhaḥ | sa hi bhavan niṣedhyasya sarvajñatvasya vyāpakaviruddhatvāt, kāraṇaviruddhatvāt, kāryaviruddhatvāt, svabhāvaviruddhakāryatvāt, vyāpakaviruddhakāryatvāt, kāraṇaviruddhakāryatvāt, kāryaviruddhakāryatvāt, svabhāvaviruddhavyāptatvāt, vyāpakaviruddhavyāptatvāt, kāraṇaviruddhavyāptatvāt, kāryaviruddhavyāptatvād vā bhavet | tatra sarvajñatvasyāsattvāt, vyāpakakāraṇakāryāṇām asiddhes tadviruddhakāryavyāpyābhāvāt na prameyatvādayaḥ sarvajñatvena paramparayāpi viruddhāḥ | 
	\pend
      

	  \pstart nanu vaktṛtvaṃ virudhyata eva sarvaviṣayanirvikalpajñānaviruddhavikalpakāryatvād vaktṛtvasya | naitad yuktam | savikalpāvikalpayor yugapadavṛtter vikalpatvena sarvajñasyāvirodhāt | 
	\pend
      

	  \pstart kas tarhi pṛthagjanād asya bheda iti cet | ucyate | yathā māyākāro nirmitāśvādiviṣayaṃ vijñānaṃ nirviṣayatvena niścinvannabhrāntaḥ, tadanyasmāc ca śreṣṭhaḥ, tathā bhagavān api śuddhalaukikavikalpasammukhībhāve 'pi na bhrānto nāpi pṛthagjanasamāna iti | tataś ca nirvikalpakasarvajñajñānavikalpayor virodhābhāvād vaktṛtvaṃ sarvajñatvena sahāviruddham eva || 
	\pend
      

	  \pstart etenaid api nirastam yad āha kāśikākāraḥ, samādher vyutthāyopadekṣyata iti cet | na | vyutthitasya hy abhilāpinī pratītir bhrāntabhāṣitam apramāṇaṃ bhaved iti || 
	\pend
      

	  \pstart yad apy uktaṃ Bṛhaṭṭīkāyām 
	\pend
      

	  \pstart yadā copadiśedekaṃ kiñcit sāmānyavaktṛvat | 
	\pend
      

	  \pstart ekadeśajñagītaṃ tan na syāt sarvajñabhāṣitam ||\edtext{}{\lemma{||}\Bfootnote{(=TS 3240)}}
	\pend
      

	  \pstart tad api nirastam, vikalpenaikasya kasyacid āmukhīkṛtvopadeśe 'pi nirvikalpena sarvam avabudhyamānasya vacanānāṃ sarvajñabhāṣitatvād eva || 
	\pend
      

	  \pstart yat punaḥ Kārikāyām uktam 
	\pend
      

	  \pstart sānnidhyamātratas tasya puṃsaś cintāmaṇer iva | 
	\pend
      

	  \pstart niścaranti yathākāmāṃ kuḍyādibhyo 'pi deśanāḥ || 
	\pend
      

	  \pstart evam ādyucyamānaṃ hi śraddadhānasya śobhate | 
	\pend
      

	  \pstart kuḍyādiniḥsṛtatvāt tu nāśvāso deśanāsu naḥ || 
	\pend
      

	  \pstart kin nu buddhapraṇītāḥ syuḥ kiṃ vā kaiścid durātmabhiḥ | 
	\pend
      

	  \pstart adṛśyair vipralambhārthaṃ piśācādibhir īritāḥ ||\edtext{}{\lemma{||}\Bfootnote{(ŚV II 138-140)}}
	\pend
      

	  \pstart Bṛhaṭṭīkāyām api 
	\pend
      

	  \pstart tasmin dhyānasamādhisthe cintāratnavadāsthite | 
	\pend
      

	  \pstart niścaranti yathākāmaṃ kuḍyādibhyo 'pi deśanāḥ || 
	\pend
      

	  \pstart tābhir jijñāsitān arthān sarvān jānanti mānavāḥ | 
	\pend
      

	  \pstart hitāni ca yathāyogaṃ kṣipramāsādayanti te || 
	\pend
      

	  \pstart ityādi kīrtamānaṃ tu śraddadhānasya śobhate | 
	\pend
      

	  \pstart vayam aśraddadhānās tu ye yuktīr arthayāmahe || 
	\pend
      

	  \pstart kuḍyādiniḥsṛtānāṃ ca na syād āptopadiṣṭatā | 
	\pend
      

	  \pstart viśvāsaś ca na tāsu syāt kenaitāḥ kīrtitā iti || 
	\pend
      

	  \pstart kin nu buddhapraṇītāḥ syuḥ kiṃ vā brāhmaṇavañcakaiḥ | 
	\pend
      

	  \pstart krīḍadbhir upadiṣṭāḥ syur dūrasthapratiśabdakaiḥ || 
	\pend
      

	  \pstart kiṃ vā kṣudrapiśācādyair adṛṣṭaiḥ parikalpitāḥ | 
	\pend
      

	  \pstart tasamān na tāsu viśvāsaḥ kartavyaḥ prājñamānibhiḥ ||\edtext{}{\lemma{||}\Bfootnote{(=TS 3241-46)}}
	\pend
      

	  \pstart etad apy anabhyupagamenaiva nirastam | śuddhalaukikavikalpasaṃmukhībhāvenaiva tasya deśakatvābhyupagamād iti || 
	\pend
      

	  \pstart atha vā yathā cakrasyoparate 'pi daṇḍapreraṇāvyāpāre pūrvāvegavaśād bhramaṇam | evaṃ bhagavati pratyastamitasamastavikalpajāle 'pi sthite yadi pūrvapraṇidhānāhitasatatānābhogavāhinī deśanā syāt tadā ko virodhaḥ | vivakṣābhāve kathaṃ vacanapravṛttir iti na vaktavyam | tadabhāve 'pi nidrāṇasya tattatpravyaktavacanasandarśanāt | vacanamātrasya vivakṣayā vyāpter abhāvāt | tasmād yathā pūrvābhyāsato jhaṭiti prabodhitasyāriṇā prahārādidānenānurūpa eva prakramaḥ śastroddharaṇādikaḥ, tathā sarvavedino 'pi sakalāḥ kalāḥ ity anākulam | 
	\pend
      

	  \pstart yad āhālaṅkāraḥ 
	\pend
      

	  \pstart śatrusānnidhyamātreṇa pravartante 'vikalpataḥ | 
	\pend
      

	  \pstart prāg eva tannirākāriprakramāḥ kopanirmitāḥ || \edtext{}{\lemma{||}\Bfootnote{(PVA III 275)}}
	\pend
      

	  \pstart yat punar uktam: piśācādikṛtaśaṅkayā nātrāśvāsaḥ satāṃ yukta iti | 
	\pend
      

	  \pstart tad asaṅgatam, yataḥ 
	\pend
      

	  \pstart sambhinnālāpahiṃsādikutsitārthopadarśanam | 
	\pend
      

	  \pstart krīḍāśīlapiśācādeḥ kāryaṃ tāsu na vidyate || 
	\pend
      

	  \pstart pramāṇadvayasaṃvādi mataṃ tadviṣaye 'khile | 
	\pend
      

	  \pstart yasya bādhā pramāṇābhyāmaṇīyasy api nekṣate || 
	\pend
      

	  \pstart yathātyantarokṣe 'pi na pūrvāparabādhitam | 
	\pend
      

	  \pstart karuṇādiguṇotpatteḥ sarvapuṃsāṃ pravartakam || 
	\pend
      

	  \pstart sarvānuśayasaṃdohapratipakṣābhidhāyakam | 
	\pend
      

	  \pstart nirvāṇagaradvārakapāṭapuṭabhedam || 
	\pend
      

	  \pstart tac cet krīḍanaśīlānāṃ rakṣasāṃ vā vaco bhavet | 
	\pend
      

	  \pstart ta eva santu sambuddhāḥ sarvatallakṣaṇasthiteḥ || \edtext{}{\lemma{||}\Bfootnote{(=TS 3613-18)}}
	\pend
      

	  \pstart na ca nāmni vivādaḥ | na ca nāmanivṛttau vastu nirvartate | pratyuta vedasyaiva krīḍanaśīlapiśācādipraṇītatvaṃ yuktaṃ sambhāvayitum | yena gośavādiṣu yogeṣv agamyāgamanādayo 'satyasamudācārāḥ saṃprakāśitāḥ | lokaprasiddhiś ca | trayo vedasya kartāro munibhaṇḍaniśācarāḥ | iti alam atinirbandhena || 
	\pend
      

	  \pstart nanu sarvajñatvaṃ vītarāgāditvena vyāptam iṣyate | tadviruddhaṃ ca rāgādiyogitvam, tatkāryaṃ ca vacanam | tad etad vyāpakaviruddhakāryabhūtaṃ vacanaṃ sarvajñābhāvaṃ sādhayati paramparayā viruddhatvād iti cet | na | rāgādīnāṃ vacasaś ca kāryakāraṇabhāvāsiddheḥ | tathā hi vacanaviśeṣo rāgādikāryam, yo rāgeṇaiva janitaḥ, vacanamātraṃ vā | 
	\pend
      

	  \pstart tatra na tāvat prathamaḥ pakṣaḥ | tādṛśasya vacanasya niścayopāyāsambhavāt | asabhyamaithunācāraprakāśakaṃ vacanaṃ tatkāryam iti cet | na | abhiprāyasya durlakṣyatvāt | virakto 'pi raktavac ceṣṭate, rakto 'pi viraktavad ity abhiprāyo durbodaḥ | tataś ca viśiṣṭavyavahārasya sāṃkaryeṇa na tatraikāntena rāgānumānaṃ yujyate | nāpi vacanamātraṃ rāgādikāryam | asaṃmukhībhūtarāgādayo 'pi hi svābhimatadevatāstutividhāne mātrādigurujanasambhāṣaṇādau ca vacanamātram uccārayantaḥ samupalabhyante | na ca yad yadabhāve bhavati tasya tatkāryatocyate, atiprasaṅgāt | rāgādiyogyatā tarhi vacasaḥ kāraṇam, tayā vinopalakhaṇḍalādau vacanasyādarśanād iti cen | na | karaṇaguṇavaktukāmate hi vacanasya hetuḥ | tadabhāvād evopalakhaṇḍalādau nivartate, na rāgādiyogyatāyā abhāvāt | yadi kāraṇaguṇādisakalatadanyakāraṇabhāve 'pi rāgādiyogyatābhāvān notpadyate vacanam iti sidhyet tasyāḥ kāraṇatvam | upalakhaṇḍalādau tu vaktukāmatā nāsti | tat kathaṃ tatkāraṇatvaṃ vacasām iti | evaṃ tarhi vaktukāmataiva rāgo 'stu | iṣṭatvān na kiñcid bādhitaṃ syāt, nāmni vivādābhāvāt | paramārthataḥ punar nityasukhātmātmīyadarśanākṣiptaṃ sāśravaviṣayaṃ cetaso 'bhiṣvaṅgaṃ rāgam āhuḥ | 
	\pend
      

	  \pstart niṣpannasarvasampatter vivakṣāpi na yujyata iti cet | adoṣo 'yam, parārthatvādivivakṣāyāḥ | vītarāge 'rthāsaṅgābhāvāt kathṃ parārthāpi pravṛttir iti cet | na | āsaṅgam antareṇa karuṇayāpi pravṛtteḥ | 
	\pend
      

	  \pstart saiva rāga iti cet | iṣṭatvād adoṣaḥ | rāgasya tu svarūpam uktam | kāruṇikasyāpi niṣphalārambho na yukta iti cet | na | parārthasyaiva phalatvāt | iṣṭalakṣaṇatvāt phalasyeti yat kiñcid etat | 
	\pend
      

	  \pstart nanu nirvikalpasya bhagavataḥ kathaṃ tasyām avasthāyāṃ karuṇāsambhavaḥ | duḥkhavikalpaprabhavā hi karuṇety anvayavyatirekābhyām anyatvena niścitam | 
	\pend
      

	  \pstart tataś ca kāraṇābhāvāt kathaṃ kāryasambhava iti cet | na | yathā kumbhakāranivṛttāv api svasantānamātrabhāvinī ghaṭādisthitis tathotthāpakavikalpābhāve 'pi samanantarapratyayabalād anālambanakaruṇāpravṛtter avāryatvāt | yad āhur guruvaḥ 
	\pend
      

	  \pstart sattāropakṛto 'pi bhāvanavaśāt kāṭhinyam āpat tathā śaithilye 'pi yathāsya duḥkhahataye sāndras tathaiva śramaḥ | 
	\pend
      

	  \pstart utpāde tu phalasya hetuniyamo no tu prabandhasthitau tasmād duḥkhadṛśaḥ kṣaye 'pi vilasanmaitryādaye 'smai namaḥ || 
	\pend
      

	  \pstart etenaitad api nirastaṃ yad āha Kārikāyām 
	\pend
      
	    
	    \stanza[\smallbreak]
	rāgādirahite cāsamin nirvyāpāre vyavasthite | &deśanānyapraṇītaiva syād ṛte pratyavekṣaṇāt || \edtext{}{\lemma{||}\Bfootnote{(ŚV II 137)}}\&[\smallbreak]


	

	  \pstart nanu yadi nāmaiva vaktṛtvaṃ sarvajñatvena sahāviruddhaṃ dehendriyabuddhyādiyogitvaṃ tu viruddham eva | sarvajñatāvyāpakavītarāgatvaviruddharāgādikāraṇatvād dehādīnām | 
	\pend
      

	  \pstart tataś ca pratiṣedhyavyāpakaviruddhakāraṇopalambhāt sarvajñābhāva iti cet | ucyate | dehādīnāṃ hetutve 'pi naiṣāṃ kevalānāṃ sahakārimātrāṇām ātmābhiniveśalakṣaṇopādānakāraṇavikalānāṃ rāgādijanakatvam ity agamakā eva dehādayaḥ sarvajñābhāvasya | tasmāj jñeyatvādīnām apy asāmarthyān na paraparikalpitānumānato 'pi sarvajñābhāvaḥ | 
	\pend
      

	  \pstart nāpi svavikalpitaṃ śābdādikaṃ bhagavato bādhakam | tathā hi yady api teṣāṃ sati prāmāṇye 'numāna evāntarbhāvaḥ, anantarbhāve cāprāmāṇyam eveti sthūlaṃ dūṣaṇam asti, tathāpi tatprāmāṇyam abhyupagamyāpi brūmaḥ | yat tāvat pauruṣeyavacanaṃ tadapramāṇam eva bhavatām | na ca vaidikaṃ kiñcid vacanaṃ sarvanarāsarvajñatvapratipādakam upalabhyate | pratyuta nimittanāmni śākhāntare sphuṭataram eva sarvajñaḥ pratipāditaḥ | 
	\pend
      

	  \pstart tathā hi: sa vetti viśvaṃ na ca tasya vettā ityādinā ca sarvajño vede pratipāditaḥ || 
	\pend
      

	  \pstart nāpy upamānāt tadabhāvaḥ sidhyati | tathā hi smaryamāṇam eva gavādivastu purovartigavayādisādṛśyopādhi gavādyupādhi vā sādṛśyam upamānena pratīyata iti sthitiḥ | na ca sarvajñasantānavartīni cetāṃsi kenacit sarvajñenānubhūtāni yataḥ smaraṇena viṣayīkriyeran, paracittavitter ayogāt || 
	\pend
      

	  \pstart yat punar uktaṃ Kumārilena 
	\pend
      

	  \pstart narān dṛṣṭvā tv asravajñān sarvān evādhunātanān | 
	\pend
      

	  \pstart tatsādṛśyopamānena śeṣāsarvajñaniścayaḥ || \edtext{}{\lemma{||}\Bfootnote{(=TS 3215)}}
	\pend
      

	  \pstart tad apy ayuktam, adhunātanasarvajñatvāniścayāt | niścaye cātmany eva sarvajñatvābhyupagamaprasaṅgāt | 
	\pend
      

	  \pstart nāpy arthāpattir bādhikā | yato dṛṣṭaḥ śruto vārtho 'nyathā nopapadyata iti adṛṣṭārthaparikalpanam arthāpattir ucyate | na cāsarvajñatvam antareṇa sarvanareṣu kaścid artho dṛṣṭaḥ śruto vā nopapadyate yatas tadarthāpattyā parikalpyeta | nanu saṃsārasya tāvad anāditvaṃ pramāṇena pratītam | tac ca na sarvajñena jñāyate, tajjñānāvadheḥ parastād asattve 'nāditākṣatiprasaṅgāt, tadanyathānupapadyamānaṃ sarvabhāvānām anāditvaṃ sarvajñābhāvaṃ sādhayatīti cet | 
	\pend
      

	  \pstart ucyate | upayuktasarvajñāpekṣayā tāvad idam adūṣaṇam | tasyānāditvājñāne 'pi upayuktasarvajñatvāvyāhateḥ | sarvasarvajñasyāpy abhāve sādhye 'samartheyam arthāpattiḥ | tathā hi yathā saṃsārasyānāditve pūrvapūrvavastusattāyā anavadhitvaṃ tathā sarvajñajñānasyāpi pūrvapūrvavastusattāvyāpakatvenānavadhiprasaratā iti | ajñātasyaikasyāpi vastuno 'navasthiteḥ | saty api sarvajñe 'nāditvam upapadyamānaṃ na sarvajñābhāvam ākṣipati | tataś cārthāpattir api na sarvajñasya bādhikā | 
	\pend
      

	  \pstart na cābhāvapramāṇabādhyaḥ sarvajñaḥ | pramāṇapañcakanivṛttir 
	\pend
      

	  \pstart abhāvapramāṇam iṣyate | tatra nivṛttir iti prasajyavṛttyā 
	\pend
      

	  \pstart pramāṇānutpattimātram abhipretam, atha vā paryudāsavṛttyā 
	\pend
      

	  \pstart vastvantaram, vastvantaram api jaḍarūpaṃ jñānarūpaṃ vā, jñānam api 
	\pend
      

	  \pstart jñānamātram, ekajñānasaṃsargivastujñānaṃ veti vikalpāḥ | 
	\pend
      

	  \pstart tatra na tāvan nivṛttimātram abhāvapramāṇam upapadyate | tat 
	\pend
      

	  \pstart khalu nikhilaśaktivikalatayā na kiñcit | yac ca na kiñcit tat kathaṃ 
	\pend
      

	  \pstart prameyaṃ paricchindyāt, tadviṣayaṃ vā vijñānaṃ janayet, pratītaṃ vā 
	\pend
      

	  \pstart tat katham iti sarvam andhakāranartanam | yathoktam: na hy abhāvaḥ 
	\pend
      

	  \pstart kasyacit pratipattiḥ pratipattihetur vā | tasyāpi vā kathaṃ 
	\pend
      

	  \pstart pratipattir \edtext{}{\lemma{pratipattir}\Bfootnote{(HB 25,12-14)}} iti |
	\pend
      

	  \pstart nāpi vastvantaratāpakṣe jaḍarūpaḥ pramāṇābhāvaḥ saṅgacchate, tasya prameyaparicchedāyogāt | paricchedasya jñānadharmatvāt | nāpi jñānamātrasvabhāvo 'bhāvaḥ | deśakālasvabhavaviprakṛṣṭasyāpi tato 'bhāvaprasaṅgāt | tadapekṣayāpi vijñānamātratvāt tasya | athaikajñānasaṃsargisvabhāvo 'numanyate, tadā kṣatam abhāvapramāṇapratyāśayā, adhyakṣaviśeṣasyaivābhāvapramāṇanāmakaraṇāt | tasya cāsmābhir dṛśyānupalambhākhyasādhanatvena svīkṛtatvāt | dṛśyānupalambhaś ca bhagavadabhāvasādhane 'samartha iti pūrvam evāveditam | 
	\pend
      

	  \pstart kiṃ ca, kaḥ punar ayaṃ pramāṇābhāvo 'bhimato bhavatām | svapramāṇagaṇanivṛttir atha sarvaprāṇigaṇapramāṇanivṛttiḥ | tatra svapramāṇagaṇanivṛttir vyabhicāriṇī, tasyāṃ satyām api vyavahitasyārthasyānapahnavatvāt | parapramāṇanivṛttis tv asarvavido 'siddhā | yad āha 
	\pend
      

	  \pstart sarvādṛṣṭiś ca sandigdhā svādṛṣṭir vyabhicāriṇī | 
	\pend
      

	  \pstart vindhyādrirandhradūrvāder adṛṣṭāv api sattvataḥ || iti || \edtext{}{\lemma{||}\Bfootnote{(=TS 122)}}
	\pend
      

	  \pstart tad evaṃ nābhāvapramāṇato 'pi sarvajñaniṣedha iti sthitam || 
	\pend
      

	  \pstart nanu tathāpi sadvyavahārārthaṃ sādhakam apy asya na vidyate | tathā hi sarvavido 'tīndriyatvāt na tāvad asmadādipratyakṣam asya sādhakam | yathā cāsmābhir asau nopalabhyate tathāsmajjātīyair apy apratyakṣasvabhāvaniyamāt | na cāyaṃ kālāntare 'bhūd iti ca kalpanā yujyate | yathā hi kālatvādidānīntanakālavad iti anenānumānena nirākartuṃ śakyate, na tathā sādhayitum | Kārikā 
	\pend
      
	    
	    \stanza[\smallbreak]
	sarvajñakalpanā tv anyair vede vāpauruṣeyatā | &tulyavat kalpyate yena tenedaṃ saṃpradhāryate || &sarvajño dṛśyate tāvan nedānīm asmadādibhiḥ | &nirākaraṇavac chakyā na cāsīd iti kalpanā ||\edtext{}{\lemma{||}\Bfootnote{(ŚV II 116-117)}}\&[\smallbreak]


	

	  \pstart iti ||
	\pend
      
	    
	    \stanza[\smallbreak]
	nāpy anumānataḥ sarvajñasiddhiḥ | tatpratibaddhaliṅgāniścayāt | \&[\smallbreak]


	

	  \pstart kiṃ ca sarvajñasattāsādhane sarvo hetuḥ trayīṃ doṣajātiṃ nātivartate asiddhatvaṃ viruddhatvam anaikāntika-tvaṃ ceti | tathā hi sarvajñe dharmaṇi kriyamāṇe na taddharmo hetuḥ siddhaḥ | tasyaiva dharmiṇaḥ sādhyatvenāsiddhatvāt | siddhau vā vaiyarthyaprasaṅgāt | asarvajñe dharmiṇi na sarvajñasiddhiḥ | hetoḥ sarvajñaviparītasādhanatvena viruddhatvāt | nāpi sarvajñāsarvajñadharmo hetuḥ | tasyānaikāntikatvāt | tasmān nānumānato 'pi sarvajñasiddhiḥ | 
	\pend
      

	  \pstart Kārikā
	\pend
      
	    
	    \stanza[\smallbreak]
	dṛṣṭo na caikadeśo 'sti liṅgaṃ yo vānumāpayet |\edtext{\textsuperscript{*}}{\lemma{*}\Bfootnote{(=TS 3125cd)}}\&[\smallbreak]


	

	  \pstart iti ||
	\pend
      

	  \pstart nāpy āgamagamyaḥ | āgamo hi dvividhaḥ pauruṣeyo nityaś ca | tatra pauruṣeyo 'py āgamaḥ tadīyo vā tatra pramāṇam, narāntarapraṇīto vā | na tāvat tadīyaḥ | anyonyasaṃśrayāpatteḥ | tathā hy āgamasya sarvajñoktatve prāmāṇyam | asya ca prāmāṇye satyasmāt sarvajñasiddhir iti | narāntarapraṇītas tu pramāṇatvenānabhimata evety ato 'pi na sarvajñasiddhiḥ || 
	\pend
      

	  \pstart kiṃ ca sarvajñapraṇītād vacanāt sarvajñasiddhau kim aparāddhaṃ svavacanena yenāto 'py asau na gamyeta | nāpi nityāgamagamyaḥ sarvajñaḥ, tathāvidhasya sarvajñapratipādakasya nityāgamasyābhāvāt | yac copaniṣadādau sarvajñapratipādakavākyaṃ tasyānyārthatvaṃ draṣṭavyam | na ca nityavākyasyānityasarvajñatvapratipādakatvam, nirviṣayatvaprasaṅgāt | 
	\pend
      

	  \pstart kiṃ ca yady aṅgīkṛto nityāgamaḥ, kiṃ sarvajñakalpanayā, nitya evāgamo dharme pramāṇaṃ bhaviṣyati | 
	\pend
      

	  \pstart Kārikā 
	\pend
      
	    
	    \stanza[\smallbreak]
	na cāgamena sarvajñas tadīye 'nyonyasaṃśrayāt | &narāntarapraṇītasya prāmāṇyaṃ gamyate katham || &na cāpy evaṃ paro nityaḥ śakyo labdhum ihāgamaḥ | &dṛṣṭaś ced arthavādatvaṃ tatpare syād anityatā || &āgamasya ca nityatve siddhe tatkalpanā vṛthā | &yatas taṃ pratipatsyante dharmam eva tato narāḥ || \&[\smallbreak]


	\footnote{\begin{english}(ŚV II 118-120)\end{english}}

	  \pstart Bṛhaṭṭīkāpi 
	\pend
      
	    
	    \stanza[\smallbreak]
	na cāgamavidhiḥ kaścin nityaḥ sarvajñabodhakaḥ | \edtext{\textsuperscript{*}}{\lemma{*}\Bfootnote{(=TS 3186ab)}}\&[\smallbreak]


	

	  \pstart ityādi saptacatvāriṃśat ślokāḥ saprapañcam etam arthaṃ pratipādayanti | tad evam āgamato 'pi na sarvajñasiddhiḥ | 
	\pend
      

	  \pstart nāpy upamānapramāṇasamadhigamyaḥ | upamānaṃ hi sadṛśagrahaṇanāntarīyakapravṛttikam asannikṛṣṭārthagocaram | yathā gavanagrahaṇadvāreṇa goḥ smaraṇam | na ca sarvajñasadṛśaḥ kaścid asti | Kārikā 
	\pend
      

	  \pstart sarvajñasadṛśaṃ kañcid yadi paśyema samprati | 
	\pend
      

	  \pstart upamānena sarvajñaṃ jānīyāmas tato vayam || \edtext{}{\lemma{||}\Bfootnote{(=TS 3215)}}
	\pend
      

	  \pstart nāpy arthāpattitaḥ sarvajñasiddhiḥ | dṛṣṭaḥ śruto vārtho 'nyathā nopapadyata ity adṛṣṭārthaparikalpanam arthāpattilakṣaṇam | na cātra pramāṇapratītaṃ kiñcid vastv asti yat sarvajñam anatareṇānupapadyamānaṃ tat sattām upanayet | tan nārthāpattir api sarvajñasādhanī | 
	\pend
      

	  \pstart na ca pramāṇapañcakābhāvasvabhāvād abhāvapramāṇād asya siddhiḥ, vastvabhāvasādha\leavevmode\ledsidenote{\textenglish{17a/RNAms}}\label{RNAms_17a}natvād asya | pratyutāyam evāsyābhāvaṃ sādhayatīti pratipāditam | yad apīdaṃ kārikābṛhaṭtīkayor ekaṣaṣṭyā ślokaiḥ sarvajñasiddhaye bauddhasya sādhanam āśaṅkya dūṣitaṃ tad api ghṛṇākaram iti granthavistarabhayān na likhitam |
	\pend
      

	  \pstart tathā hy etāni kila saugataiḥ sarvajñasādhanāya sādhanāny abhidhīyante | sarvajño 'stīti satyam, sarvajñoktatvāt, dharmābhyupadeśakatvāt, buddhaḥ sarvajña iti cirapravṛttadṛḍhasmṛteḥ, prathamataram aśeṣaśiṣyajanavargasyānekavidhacittacaittādiparijñānāt, sakalapadārtharāśitattvopadeśād iti || 
	\pend
      

	  \pstart tasmāt sthitam etat nātīndriyadarśī sākṣād asti, api tu nityavacanadvāreṇaiva tasya darśanam iti | tad evaṃ sarvathā sarvajñasādhakapramāṇāsabhavād ayukto bauddhānāṃ sarvajñe sadvyavahāra iti || 
	\pend
      

	  \pstart atrocyate | anumānād anyato 'siddhau siddhasādhanam | anumānād apīty asiddham, anumānasya pūrvam uktatvāt | tatpratibaddhaliṅgāniścayād ityādidūṣaṇaprabandho 'pi prativyūḍha ity upayuktasarvajñas tāvat trailokyālokaḥ siddhaḥ | 
	\pend
      

	  \pstart sarvasarvjñapakṣe 'pīdaṃ sādhanam | 
	\pend
      

	  \pstart yat pramāṇasaṃvādiniścitārthavacanaṃ tat sākṣāt paramparayā vā tadarthasākṣātkārijñānapūrvakam | yathā dahano dāhaka iti vacanam | pramāṇasaṃvādi niścitārthavacanaṃ cedam | kṣaṇikāḥ sarvajñasaṃskārā ity arthataḥ kāryahetuḥ | nāsyāsiddhiḥ, sarvabhāvakṣaṇabhaṅgaprasādhanād asya vacanasya satyārthatvāt | nāpi virodhaḥ, sapakṣe bhāvāt | na cānaikāntikaḥ, vacanamātrasya saṃśayaviparyāsapūrvakatve 'pi pramāṇaniścitārthavacanasya sākṣātpāramparyeṇa tadarthasākṣātkārijñānapūrvakatvāt | anyathā niyamena pramāṇasaṃvādāyogāt || 
	\pend
      

	  \pstart ayaṃ ca bhāṣyakārīyaḥ sarvasarvajñaprasādhakaprayogaḥ paṇḍitajitāribhiḥ prapañcita iti tata eva pracayato 'avadhārya iti | 
	\pend
      
	    
	    \stanza[\smallbreak]
	durvāraprativādivikramam anādṛtya pramāprauḍhitaḥ sarvajño jagadekacakṣurudagād eṣa prabhāvo 'tra ca | &sambuddhasthitimedinīkulagirer asmadguroḥ kin tv ayaṃ saṃkṣepo mama ratnakīrtikṛtinas tadvistaratrāsinaḥ || \&[\smallbreak]


	
	    
	    \stanza[\smallbreak]
	viśvam astu śubhād asmād yathecchaṃ ratimanmataḥ | &mañjuvajraś ca paryante tatpādaṃ satphalapradam || &ahañ ca mañjuvajraḥ syāṃ mañjughoṣo 'tha mañjuvāk | &mañjuśrīr vādirāṇmamañjukumāro jinadhūrdharaḥ | \&[\smallbreak]


	

	  \pstart || sarvjñasiddhiḥ samāptā ||\cref{thakur75-30.18}
	\pend
      
	    
	    \endnumbering% ending numbering from div
	    \endgroup
	    
	  \label{īśvarasādhanadūṣaṇam}
	  
	% new div opening: depth here is 0
	
	    
	    \begingroup
	    \beginnumbering% beginning numbering from div depth=0
	    
	  
\chapter[{Īśvarasādhanadūṣaṇam}]{Īśvarasādhanadūṣaṇam}\label{īśvarasādhanadūṣaṇam}\marginpar{\footnotesize\textenglish{pb in}}\label{RNAms_18b}

	  \pstart \edtext{\textsuperscript{*}}{\lemma{*}\Bfootnote{Mikogami\textunderscore Ms 18b1}} \edlabel{thakur75-32.4}\label{thakur75-32.4} oṃ namas tārāyai |
	\pend
      
	    
	    \stanza[\smallbreak]
	sūktaratnāśrayatvena jitaratnākarād idam |&guror vāgambudheḥ smartuṃ kiñcid ākṛṣya likhyate ||\&[\smallbreak]


	
	    
	    \stanza[\smallbreak]
	rītiḥ sudhānidhir iyaṃ sattame madhyavartini |&vidveṣiṇi viṣajvālā kiñcij jñe tu na kiñcana ||\&[\smallbreak]


	

	  \pstart ihaite naiyāyikādayo vivādapadasya kṣitidharādeḥ svarūpopādānopakaraṇasaṃpradānaprayojanavibhāgapravīṇaṃ sarvajñatādiguṇaviśiṣṭaṃ puruṣaviśeṣam icchanti | yad āhuḥ
	\pend
      
	    
	    \stanza[\smallbreak]
	\label{ratnakīrtinibandhāvali__īśvarakārikā}\flagstanza{\tiny\textenglish{...kārikā}}eko vibhuḥ sarvavidekabuddhisamāśrayaḥ śāśvata īśvarākhyaḥ |&pramāṇam iṣto jagato vidhātā svargāpavargārthibhir arthanīyaḥ ||\&[\smallbreak]


	

	  \pstart iti |
	\pend
      

	  \pstart sa ca kathaṃ sidhyatīti paryanuyuktāḥ sādhanam idam ācakṣate |
	\pend
      

	  \pstart vivādādhyāsitaṃ buddhimaddhetukam |
	\pend
      

	  \pstart kāryatvāt |
	\pend
      

	  \pstart yat kāryaṃ tadbuddhimadhetukam | yathā ghaṭaḥ |
	\pend
      

	  \pstart kāryaṃ cedam |
	\pend
      

	  \pstart tasmād buddhimadhetukam iti |
	\pend
      

	  \pstart hetoḥ parokṣārthapratipādakatvam anubhūteṣu hetvābhāseṣu na śakyam āvedayitum | hetvābhāsāś ca pañca | yathoktam
	\pend
      
	    
	    \stanza[\smallbreak]
	savyabhicāraviruddhaprakaraṇasamasādhyasamātītakālā iti |\&[\smallbreak]


	

	  \pstart tatra na tāvad ayaṃ sādhyasamo hetuḥ | asiddho hi sādhyasamaḥ kathyate | sa ca saṃkṣepato vibhajyamāno dvidhā vyavatiṣṭhate | āśrayāsiddhatvād vāsiddho yathā surabhi gaganāravindamaravindatvād iti | saty api cāśraye pramāṇena sambandhāsiddher asiddho yathā anityaḥ śabdaḥ sāvayavatvād iti | na cābhyāṃ prakārābhyāṃ prastutasya hetor asiddhir asti | \edlabel{ratnakīrtinibandhāvali__36r1N54T2H6NLF3JHCUF8B8TWUQ}\label{ratnakīrtinibandhāvali__36r1N54T2H6NLF3JHCUF8B8TWUQ}kṣmāruhādau dharmiṇi pramāṇasamadhigate kāryatvasya sādhanasya pramāṇpratītatvāt | cirotpannaparvatādau ca dharmiṇi kāryatvaṃ sāvayavatvena hetunā boddhavyam | tad yathā: vivādapadaṃ kāryam | sāvayavatvāt | yat sāvayavaṃ tat kāryam | yathā vastram | tathā cedam | tasmāt kāryam iti |
	\pend
      

	  \pstart nanu sāvayavatvena hetunā dravyāṇām eva kāryatvaṃ sidhyati | na tu tatsamavetānāṃ guṇakarmādīnām | teṣām avayavasambandhābhāvād iti cet | satyam | teṣāṃ kāryaguṇāditvena hetvantareṇa kāryatvam adhigantavyam | tathā hi; 
	\pend
      

	  \pstart janmabhājo vivādādhyāsitanityetarasamavāyino guṇādayaḥ |
	\pend
      

	  \pstart kāryaguṇāditvāt |
	\pend
      

	  \pstart yo yaḥ kāryaguṇādiḥ sa sarvas tathā, yathā ghaṭādirūpādiḥ |
	\pend
      

	  \pstart tathā caite |
	\pend
      

	  \pstart tasmāj janmabhājaḥ | iti |
	\pend
      

	  \pstart \edlabel{ratnakīrtinibandhāvali__36r1PF7IMWY3NMXO5SUWBH92TEW}\label{ratnakīrtinibandhāvali__36r1PF7IMWY3NMXO5SUWBH92TEW}\edtext{}{\lemma{}\xxref{ratnakīrtinibandhāvali__36r1PF7IMWY3NMXO5SUWBH92TEW}{ratnakīrtinibandhāvali__36r1PF7IMWXHVSXCQADYWUN3BV2}\Afootnote{kāryañca \cite{RNAms?citedRange=18b6} ; kāryatvaṃ ca \cite{ĪSD?citedRange=33.18} }}kāryañca\edlabel{ratnakīrtinibandhāvali__36r1PF7IMWXHVSXCQADYWUN3BV2}\label{ratnakīrtinibandhāvali__36r1PF7IMWXHVSXCQADYWUN3BV2} na svakāraṇasamavāyaḥ, sāmānyaviśeṣo vā \edlabel{ratnakīrtinibandhāvali__36r1PF7IMWWXB92JHQ54Y62T8O9}\label{ratnakīrtinibandhāvali__36r1PF7IMWWXB92JHQ54Y62T8O9}\edtext{}{\lemma{}\xxref{ratnakīrtinibandhāvali__36r1PF7IMWWXB92JHQ54Y62T8O9}{ratnakīrtinibandhāvali__36r1PF7IMWWBD3LBZ22781LLC70}\Afootnote{boddhavyaṃ\deletion{ḥ} \cite{RNAms?citedRange=18b6} ; boddhavyaḥ \cite{ĪSD?citedRange=33.18} }}boddhavyaṃ\edlabel{ratnakīrtinibandhāvali__36r1PF7IMWWBD3LBZ22781LLC70}\label{ratnakīrtinibandhāvali__36r1PF7IMWWBD3LBZ22781LLC70}, yenāsya pradhvaṃsāvyāpakatvād bhāgāsiddhatā syāt, kiṃ tu kāraṇādhīnasvarūpamātram | tac ca śabdādiṣv iva pradhvaṃsādāv api pratyakṣeṇādhigatam iti na tāvad ayam asiddho hetuḥ |
	\pend
      

	  \pstart nāpi viruddhaḥ | tathā \leavevmode\ledsidenote{\textenglish{19a/RNAms}}\label{RNAms_19a}hi yo vipakṣa eva vartate sa khalu sādhyaviparyayavyāpteḥ sādhyaviruddhaṃ sādhayan viruddho 'bhidhīyate | yathā nityaḥ śabdaḥ kṛtakatvād iti | na cāyaṃ tathā, prasiddhakartṛkeṣu ghaṭādiṣu sapakṣeṣu sadbhāvadarśanāt |
	\pend
      

	  \pstart \edlabel{thakur75-33.24}\label{thakur75-33.24} nanu \edlabel{sarit__ratnakīrtinibandhāvali__84822}\label{sarit__ratnakīrtinibandhāvali__84822}\edtext{}{\lemma{buddhimatpūrvakatve … tadviśeṣaṇatvānupapatteḥ |}\xxref{sarit__ratnakīrtinibandhāvali__84822}{sarit__ratnakīrtinibandhāvali__86003}\Afootnote{\label{sarit__ratnakīrtinibandhāvali__395787}\textenglish{---\textsc{Note} NK 150--151.}\textenglish{---\textsc{Note} Sanskrit and translation in \cite[92--97]{krasser02_zaGkar_Izvar_studie}.}  {\rmlatinfont [App type: parallel]}}}buddhimatpūrvakatve sādhye siddhasādhanam | abhimataṃ hi pareṣām api karmajatvaṃ kāryajātasya, karmaṇaś ca cetanātmakatvāt, cetanāhetukatvād vā | taddhetukatvaṃ ca jagataḥ | \edlabel{ratnakīrtinibandhāvali__36r1N6ESBB6GEXJBT5XVAMTHRS5}\label{ratnakīrtinibandhāvali__36r1N6ESBB6GEXJBT5XVAMTHRS5}sarvajñapūrvakatve tu sādhye vyāptiḥ svapne 'pi \edlabel{ratnakīrtinibandhāvali__36r1N6F0R3CZ6K2E77O6CXTJDTY}\label{ratnakīrtinibandhāvali__36r1N6F0R3CZ6K2E77O6CXTJDTY}\edtext{}{\lemma{nopalabdhā |}\xxref{ratnakīrtinibandhāvali__36r1N6F0R3CZ6K2E77O6CXTJDTY}{ratnakīrtinibandhāvali__36r1N6F0R3EBPFS8O3U3VB2EVLU}\Afootnote{\label{ratnakīrtinibandhāvali__36r1N6F0S0SBC9X1LL8VAVGOA73}nopalabdhā | \cite{RNAms} ; nopalabddhā | \cite{thakur75}   {\rmlatinfont [App type: typo]}}}nopalabdhā |\edlabel{ratnakīrtinibandhāvali__36r1N6F0R3EBPFS8O3U3VB2EVLU}\label{ratnakīrtinibandhāvali__36r1N6F0R3EBPFS8O3U3VB2EVLU} dṛṣṭāntaś ca sādhyahīnaḥ, kulālādīnām asarvajñatvāt | viruddhatā ca hetor asarvajñapūrvakatvenaiva kumbhādau kāryatvasya vyāpter upalabdheḥ | na copalabdhimatpūrvakatvamātraṃ sādhanaviṣayaḥ, tadviśeṣasya tu sarvajñapūrvakatvasyātadviṣayasyāpi tataḥ siddhir iti sāmpratam | tathā  hi yady asau viśeṣo na sādhanaviṣayaḥ katham atas tatsiddhiḥ, \edlabel{ratnakīrtinibandhāvali__36r1N6H3ZXWODAWI2R2XNTCQWH5}\label{ratnakīrtinibandhāvali__36r1N6H3ZXWODAWI2R2XNTCQWH5}\edtext{}{\lemma{siddhyan vā}\xxref{ratnakīrtinibandhāvali__36r1N6H3ZXWODAWI2R2XNTCQWH5}{ratnakīrtinibandhāvali__36r1N6H3ZYQH06DWRZ87HBU0W3H}\Afootnote{\label{ratnakīrtinibandhāvali__36r1N6H411992B9T0CM06YGJ2RD}sidhyanvā \cite{} ; siddhaṃ vā \cite{thakur75}   {\rmlatinfont [App type: var]}}}sidhyan vā\edlabel{ratnakīrtinibandhāvali__36r1N6H3ZYQH06DWRZ87HBU0W3H}\label{ratnakīrtinibandhāvali__36r1N6H3ZYQH06DWRZ87HBU0W3H} katham aviṣayaḥ, viṣayaś cet katham ananvayadoṣaṃ na spṛśed iti cet |
	\pend
      

	  \pstart \edlabel{thakur75-33.32}\label{thakur75-33.32} ucyate | sāmānyamātravyāptāv apy antarbhāvitaviśeṣasya sāmānyasya pakṣadharmatāvaśena sādhyadharmiṇy anumānāt viśeṣaviṣayam anumānaṃ bhavaty eva | itarathā sarvānumānocchedaprasaṅgāt | tathā hi vahnyanumānam api na sāmānyamātraviṣayam, tasya prāg eva siddhatvāt | nāpi tadviśiṣṭagirigocaram vahnitvasāmānyasya tatsambandhābhāvena tadviśeṣaṇatvānupapatteḥ |\edlabel{sarit__ratnakīrtinibandhāvali__86003}\label{sarit__ratnakīrtinibandhāvali__86003} itarathā gotvasamavāyād iva gāvaḥ śābaleyādayaḥ parvato 'pi vahnitvasamavāyād vahniḥ prasajyeta | asty eva girer vahnitvena saṃyuktasamavāyaḥ sambandha iti cet | tarhi nāpratipadya parvatasaṃyuktaṃ vahniviśeṣam asau śakyapratipattir iti vahniviśeṣasyāpy ananumānam | tathā cānanvayadoṣaprasaṅgaḥ | indriyānumāne 'py ayam eva nyāyo draṣṭavyaḥ, yathendriyalakṣaṇakaraṇaviśeṣasiddhiḥ | tathā hi tatrāpi nendriyakaraṇikā kācit kriyopalabdhā | na khalu \edlabel{ratnakīrtinibandhāvali__36r1NBNCTLXSLLK5Q0BPDLIM95A}\label{ratnakīrtinibandhāvali__36r1NBNCTLXSLLK5Q0BPDLIM95A}\edtext{}{\lemma{cchidādyaḥ}\xxref{ratnakīrtinibandhāvali__36r1NBNCTLXSLLK5Q0BPDLIM95A}{ratnakīrtinibandhāvali__36r1NBNCTLZ1SMUV2GW43CFNXI9}\Afootnote{\label{ratnakīrtinibandhāvali__36r1NBNCUJSJ7ITQZY3I4WX64RH}cchidādyaḥ \cite{RNAms,SVR,NK} ; cchidādyāḥ \cite{thakur75}   {\rmlatinfont [App type: var]}}}cchidādyaḥ\edlabel{ratnakīrtinibandhāvali__36r1NBNCTLZ1SMUV2GW43CFNXI9}\label{ratnakīrtinibandhāvali__36r1NBNCTLZ1SMUV2GW43CFNXI9} kriyā \edlabel{ratnakīrtinibandhāvali__36r1NBNDW1G2X6VFZF0NJ3EWJDN}\label{ratnakīrtinibandhāvali__36r1NBNDW1G2X6VFZF0NJ3EWJDN}\edtext{}{\lemma{indriyasādhanā vraścanādīnām}\xxref{ratnakīrtinibandhāvali__36r1NBNDW1G2X6VFZF0NJ3EWJDN}{ratnakīrtinibandhāvali__36r1NBNDW1H5HREMMTOXP3XHCJD}\Afootnote{\label{ratnakīrtinibandhāvali__36r1NBNDWK0VLOSL6051B56H1PP}indriyasādhanāvraścanādīnām \cite{RNAms} ; indriyasādhanāḥ, vraścanādīnām \cite{NK} ; indriyasādhanāḥ| vraścanādīnām \cite{SVR} ; indriyasādhanā, vraścanādīnām \cite{thakur75}   {\rmlatinfont [App type: var]}}}indriyasādhanāḥ, \edlabel{ratnakīrtinibandhāvali__36r1NBNGZ65MJHSVBYOUHFQKGUA}\label{ratnakīrtinibandhāvali__36r1NBNGZ65MJHSVBYOUHFQKGUA}\edtext{}{\lemma{vraścanādīnām anindriyatvāt |}\xxref{ratnakīrtinibandhāvali__36r1NBNGZ65MJHSVBYOUHFQKGUA}{ratnakīrtinibandhāvali__36r1NBNGZ66X54IW5MBNP1T2YNQ}\Afootnote{\label{ratnakīrtinibandhāvali__36r1NBNGZ69U89855WFDDRMM450}\textenglish{---\textsc{Note} \cite[408.26--27]{SVR}: \begin{sanskrit}paraśvadhādīnām anindriyatvāt |\end{sanskrit}}  {\rmlatinfont [App type: parallel]}}}vraścanādīnām\edlabel{ratnakīrtinibandhāvali__36r1NBNDW1H5HREMMTOXP3XHCJD}\label{ratnakīrtinibandhāvali__36r1NBNDW1H5HREMMTOXP3XHCJD} anindriyatvāt |\edlabel{ratnakīrtinibandhāvali__36r1NBNGZ66X54IW5MBNP1T2YNQ}\label{ratnakīrtinibandhāvali__36r1NBNGZ66X54IW5MBNP1T2YNQ} \edlabel{ratnakīrtinibandhāvali__36r1NBNFAY1WZTRAN8OW7C97F9D}\label{ratnakīrtinibandhāvali__36r1NBNFAY1WZTRAN8OW7C97F9D}\edtext{}{\lemma{na … kriyā |}\xxref{ratnakīrtinibandhāvali__36r1NBNFAY1WZTRAN8OW7C97F9D}{ratnakīrtinibandhāvali__36r1NBNFAY37HO66IDDUSZLKZSL}\Afootnote{\label{ratnakīrtinibandhāvali__36r1NBNFAY6REOBTPZAUSZEQ0TV}\textenglish{---\textsc{Note}  has plural: \begin{english}na ca vraścanādisādhanāḥ sambhavanti rūpādiparicchittilakṣaṇāḥ kriyāḥ |\end{english}. But \cite[408.27--28]{SVR}: \begin{english}na ca paraśvadhādisādhanā rūpādiparicchittirūpā kriyā sambhavati |\end{english}.}  {\rmlatinfont [App type: parallel]}}}na ca vraścanādisādhanā sambhavati rūpādiparicchittilakṣaṇā kriyā |\edlabel{ratnakīrtinibandhāvali__36r1NBNFAY37HO66IDDUSZLKZSL}\label{ratnakīrtinibandhāvali__36r1NBNFAY37HO66IDDUSZLKZSL} tasmād yathā kriyātvasāmānyasya karaṇamātrādhīnatvavyāptatve pakṣadharmatāvaśād indriyalakṣaṇakaraṇaviśeṣasiddhis \edlabel{sarit__ratnakīrtinibandhāvali__86683}\label{sarit__ratnakīrtinibandhāvali__86683}\edtext{}{\lemma{tathehāpi … āpatati |}\xxref{sarit__ratnakīrtinibandhāvali__86683}{sarit__ratnakīrtinibandhāvali__87284}\Afootnote{\label{sarit__ratnakīrtinibandhāvali__396860}---\textsc{Note} Cf. ---\textsc{Note} Sanskrit and translation in \href{krasser02_zaGkar_Izvar_studie}{krasser02\textunderscore zaGkar\textunderscore Izvar\textunderscore studie}  {\rmlatinfont [App type: parallel]}}}tathehāpi saty api kāryatvasyopādānopakaraṇasaṃpradānaprayojanajñakartṛmātravyāptatve 'pi vivādādhyāsiteṣu pakṣadharmatāvaśā\leavevmode\ledsidenote{\textenglish{19b/RNAms}}\label{RNAms_19b}d upādānādyabhijñasāmānyasyākṣiptaviśeṣasyaiva siddhiḥ | anyathā sāmānyasyāpi vyāpakābhimatasya na siddhiḥ syāt, \edlabel{ratnakīrtinibandhāvali__36r1NBN3IACWENPEYV4Q9H7K18R}\label{ratnakīrtinibandhāvali__36r1NBN3IACWENPEYV4Q9H7K18R}\edtext{}{\lemma{nirviśeṣasyāsambhavādviśeṣasya … tasyānupapatteḥ |}\xxref{ratnakīrtinibandhāvali__36r1NBN3IACWENPEYV4Q9H7K18R}{ratnakīrtinibandhāvali__36r1NBN3IBWUTMDSV5RRK1CHVSV}\Afootnote{\label{ratnakīrtinibandhāvali__36r1NBN3IC09L08K3LEKT2N1R0A}nirviśeṣasyāsambhavadviśeṣasya vā tasyānupapatteḥ | \cite{} \textenglish{---\textsc{Note} Cf. : \begin{english}nirviśeṣasyāsambhavāt, viśeṣasya cānyasyānupapatteḥ\end{english}.}\textenglish{---\textsc{Note} Cf. \cite[408?]{SVR}: \begin{english}nirviśeṣasya tasyānupapatteḥ\end{english}.}  {\rmlatinfont [App type: parallel]}}}nirviśeṣasyāsambhavādviśeṣasya vā tasyānupapatteḥ |\edlabel{ratnakīrtinibandhāvali__36r1NBN3IBWUTMDSV5RRK1CHVSV}\label{ratnakīrtinibandhāvali__36r1NBN3IBWUTMDSV5RRK1CHVSV} asarvajñasya cātrādṛṣṭādibhedavijñānasahitasyādhiṣṭhātṛbhāvāsambhavāt sarvajñātmaka eva viśeṣo balād āpatati |\edlabel{sarit__ratnakīrtinibandhāvali__87284}\label{sarit__ratnakīrtinibandhāvali__87284}
	\pend
      

	  \pstart \edlabel{thakur75-34.17}\label{thakur75-34.17}nanūpādānādyabhijñakartṛmātreṇevāsarvajñatvadehitvādibhir api vyāptir aśakyaparihārā, vyabhicārādarśanasya samānatvād iti cet | na | sarvajñatvāsarvajñatvayor dehitvādehitvayor vā kāryotpattāv anupayogāt | na hi sārvajñyaṃ kartṛṇām yogyatām upasthāpayati, asarvajñebhyaḥ kumbhakārādibhyaḥ kumbhādīnām aprasavaprasaṅgāt | nāpy asārvajñyaṃ kumbhakārād eva keyūrādīnām apy utpattiprasaṅgāt | tathā na dehitvaṃ kāryotpattāv upayogi kumbhakārād eva keyūrādīnām utpattiprasaṅgāt | nādehitvaṃ kumbhakārād ghaṭādīnām anutpādaprasaṅgāt | tataś copādānādyabhijñapuruṣapūrvakatvam eva kāryatvasya vyāpakam | tad eva ca buddhimatpuruṣapūrvakatvaśabdavācyam | \edlabel{sarit__ratnakīrtinibandhāvali__88081}\label{sarit__ratnakīrtinibandhāvali__88081}\edtext{}{\lemma{tena … °pratikṣepahetutvāt ||}\xxref{sarit__ratnakīrtinibandhāvali__88081}{sarit__ratnakīrtinibandhāvali__88447}\Afootnote{\label{sarit__ratnakīrtinibandhāvali__397298}---\textsc{Note} Cf. ---\textsc{Note} Sanskrit and translation in \cite[98--100]{krasser02_zaGkar_Izvar_studie}  {\rmlatinfont [App type: parallel]}}}tena yady api buddhimatpūrvakatvamātraṃ vyāptiviṣayas tathāpi tadviśeṣasya sarvajñatvasya pakṣadharmatābalāt pratilambha iti viśeṣaviṣayam anumānam | na coktadoṣaprasaṅgaḥ, tasya sādhyadṛṣṭāntayor dharmavikalpād utkarṣāpakarṣalakṣaṇaparyanuyogasya sarvānumānasādhāraṇyenānumānamātraprāmāṇyapratikṣepahetutvāt ||\edlabel{sarit__ratnakīrtinibandhāvali__88447}\label{sarit__ratnakīrtinibandhāvali__88447}
	\pend
      

	  \pstart etena yad uktaṃ kaṇikāyāṃ \edlabel{nk-150.9}\label{nk-150.9}yadi kulālādīnāṃ katipayopakaraṇādijñānam, na samastopakaraṇādijñatā, tarhi tenaiva nidarśanena īśvarasyāpi \edlabel{ratnakīrtinibandhāvali__36r1NLOHWTQ70VHBU3NJZWC5S9G}\label{ratnakīrtinibandhāvali__36r1NLOHWTQ70VHBU3NJZWC5S9G}\edtext{}{\lemma{tadupakaraṇādimātrajñānam | … sarvajñatāsiddhiḥ |}\xxref{ratnakīrtinibandhāvali__36r1NLOHWTQ70VHBU3NJZWC5S9G}{ratnakīrtinibandhāvali__36r1NLOHWTQZXFC6755VVL7IPZO}\Afootnote{\label{ratnakīrtinibandhāvali__36r1NLOI13GSGSHMPIZGYJ8Z7E2}tadupakaraṇādimātrajñānam | tanmātrajñāne na sarvajñatāsiddhiḥ \cite{ĪSD?citedRange=34.31--32} ; tadupakaraṇādimātrajñānaṃ tanmātrajñāne ca na sarvajñatāsiddhiḥ \cite{NK?citedRange=150.10--11} ; tadupakaraṇādimātrajñāne na sarvajñatāsiddhiḥ \cite{RNAms?citedRange=19b5}   {\rmlatinfont [App type: var]}}}tadupakaraṇādimātrajñānam | tanmātrajñāne na sarvajñatāsiddhiḥ |\edlabel{ratnakīrtinibandhāvali__36r1NLOHWTQZXFC6755VVL7IPZO}\label{ratnakīrtinibandhāvali__36r1NLOHWTQZXFC6755VVL7IPZO} katipayajño hi tathā sati syāt |
	\pend
      

	  \pstart na vā tanmātrajñānam apīśvarasya \edlabel{ratnakīrtinibandhāvali__36r1PF7IMWVPAYJT0FI4RBAXGDB}\label{ratnakīrtinibandhāvali__36r1PF7IMWVPAYJT0FI4RBAXGDB}\edtext{}{\lemma{bālādivad ity āha | bālonmattādīnāṃ}\xxref{ratnakīrtinibandhāvali__36r1PF7IMWVPAYJT0FI4RBAXGDB}{ratnakīrtinibandhāvali__36r1PF7IMWV213W4T6OZ0RND8CE}\Afootnote{bālonmattādīnāṃ \cite{RNAms?citedRange=19b5} ; bālādivad ity āha | bālonmattādīnāṃ \cite{ĪSD?citedRange=34.32} ; bālādivad ity aha---bālonmattādayaśca \cite{NK?citedRange=150.10--11}   {\rmlatinfont [App type: em]}}}bālādivad ity āha | bālonmattādīnāṃ\edlabel{ratnakīrtinibandhāvali__36r1PF7IMWV213W4T6OZ0RND8CE}\label{ratnakīrtinibandhāvali__36r1PF7IMWV213W4T6OZ0RND8CE} svakāryaprayojanāparijñāne 'pi nirabhiprāyāṇāṃ tatra tatra pravṛttidarśanāt | na ca kulālādayo nidarśanaṃ na bālādaya ity atra niyamahetur astī\edlabel{nk-150.14}\label{nk-150.14}ti tan nirastam ||\edtext{}{\lemma{---}\Afootnote{ \cite{NK} ---\textsc{Note} Cf.   {\rmlatinfont [App type: parallel]}}}\edtext{\textsuperscript{*}}{\lemma{*}\Bfootnote{Cf. }}
	\pend
      

	  \pstart īśvarasya hi katipayātīndriyopakaraṇādijñāne tatkāraṇasya sarvatra samānatvād aśeṣopakaraṇādijñatāyā durvāratvāt | kāraṇam ca tajjñāne sattām antareṇa nānyat, dharmādharmādīnāṃ laukikapratyāsattihetūnāṃ tatrāsambhavāt | kāraṇābhede ca kāryābhedaḥ | anyathā katipayātīndriyajñānam api na syāt | yathā hi \edlabel{ratnakīrtinibandhāvali__36r1PF7IMWUFC5S10GC5G797R9K}\label{ratnakīrtinibandhāvali__36r1PF7IMWUFC5S10GC5G797R9K}\edtext{}{\lemma{kulālādis}\xxref{ratnakīrtinibandhāvali__36r1PF7IMWUFC5S10GC5G797R9K}{ratnakīrtinibandhāvali__36r1PF7IMWTROD9VEEQNOUKG7MN}\Afootnote{kulālādis \cite{thakur75} ; ku\marginpar{\footnotesize\textenglish{20a/RNAms}}\label{RNAms_20a}lālādas \cite{RNAms}   {\rmlatinfont [App type: em]}}}kulālādis\edlabel{ratnakīrtinibandhāvali__36r1PF7IMWTROD9VEEQNOUKG7MN}\label{ratnakīrtinibandhāvali__36r1PF7IMWTROD9VEEQNOUKG7MN} tulyadarśanasāmagrīkeṣu nākiñcij\edlabel{ratnakīrtinibandhāvali__36r1PF7IMWT51K8YNUUTZQTQ86H}\label{ratnakīrtinibandhāvali__36r1PF7IMWT51K8YNUUTZQTQ86H}\edtext{}{\lemma{jñaḥ}\xxref{ratnakīrtinibandhāvali__36r1PF7IMWT51K8YNUUTZQTQ86H}{ratnakīrtinibandhāvali__36r1PF7IMWSHSBW8CCLYGRITU90}\Afootnote{jñaḥ \cite{RNAms} ; jñāḥ \cite{thakur75}   {\rmlatinfont [App type: var]}}}jñaḥ\edlabel{ratnakīrtinibandhāvali__36r1PF7IMWSHSBW8CCLYGRITU90}\label{ratnakīrtinibandhāvali__36r1PF7IMWSHSBW8CCLYGRITU90} tathātīndriyopakaraṇādiṣv apīśvaraḥ, sāmarthyasyāviśeṣāt | na ca bālonmattādinidarśanena katipayopakaraṇajñatāniṣedho yuktaḥ, bījadṛṣṭāntena buddhimanmātrasyāpi niṣedhābhidhānaprasaṅgāt | tasmād yathopādānādyabhijñasyāpi sambhavād bījādibhir na vyabhicārābhidhānam, tathā bālonmattādibhir apīti kulālādīnām eva dṛṣṭāntatā yuktimatī, upādānādyabhijñabuddhivanmātrakāryatvayoḥ sādhyasādhanayos tatra prasiddhatvāt | tathā jñānavad īśvarasya cikīrṣāprayatnau nityāv ity atrāpi |
	\pend
      

	  \pstart yad abhihitam: nityau cet kim īśvarasya jñānena cikīrṣāprayatnopayoginā, tayor nityatvāt, svotpādopayogānapekṣaṇādityādi | tad apy asāram | ajñātakartṛtvānupaptteḥ | jñānaṃ hi yatra cikīrṣāpratyatnāv anityau tatra tāv upasthāpayadupakaraṇādikam upadarśayati | yatra tu tau nityau tatropakaraṇādikam upadarśayad api saphalam | tasmāt saty api cikīrṣāpratyatnayor nityatve saphalam īśvarajñānaṃ sākṣātkāryopattāv anupayogy api | ata eva ca so 'yam īdṛśo viśeṣo vicārāsahaḥ kathaṃ pakṣadharmatābalād api sādhyadharmiṇy upasaṃhriyata ityādir api pralāpa eva | īśvarajñānasyāvyāhatau sarvajñatāviśeṣasya durvāratvāt |
	\pend
      

	  \pstart yad abhihitam: \edlabel{nk_3522_start}\label{nk_3522_start}prekṣāvatāṃ pravṛttiḥ prayojanavattayā vyāptā | na ceśvarasya prekṣāvato jagannirmāṇe prayojanam utpaśyāmaḥ, prāptanikhilaprāpaṇīyasya prāptavyābhāvāt |\edlabel{nk_3522_end}\label{nk_3522_end}\edtext{}{\lemma{---}\Afootnote{---\textsc{Note} Corresponds to NK   {\rmlatinfont [App type: ce2]}}}\edtext{\textsuperscript{*}}{\lemma{*}\Bfootnote{Corresponds to NK }} tad api sāvadyam, tadabhiprāyasya durbodhatvāt, prayojanābhāvāsiddheḥ, vyāpakānupalabdheḥ, sandigdhatvāt | vicitrā hi puruṣamātrasya cetovṛttiḥ prāg eva viśvasya kartuḥ | prāptanikhilaprāpaṇīyasyāpi karuṇayāpi parārtha\edlabel{ratnakīrtinibandhāvali__36r1PF7IMWRV4JZSWYNB05SZCTW}\label{ratnakīrtinibandhāvali__36r1PF7IMWRV4JZSWYNB05SZCTW}\edtext{}{\lemma{pravṛtteḥ}\xxref{ratnakīrtinibandhāvali__36r1PF7IMWRV4JZSWYNB05SZCTW}{ratnakīrtinibandhāvali__36r1PF7IMWR7I4SV0JG3GBCU654}\Afootnote{pravṛtteḥ \cite{RNAms} ; pravṛttaḥ \cite{thakur75}   {\rmlatinfont [App type: var]}}}pravṛtteḥ\edlabel{ratnakīrtinibandhāvali__36r1PF7IMWR7I4SV0JG3GBCU654}\label{ratnakīrtinibandhāvali__36r1PF7IMWR7I4SV0JG3GBCU654} sambhāvyamānatvāt | na cāsya narakādinirmāṇapravṛttiḥ kāruṇikatām upahanti, pratyuta pituḥ putragaṇḍapāṭanavṛttir ivālpaduḥkhadānena prabhūtadāruṇaduḥkhāpanayanāt karuṇātiśayam eva gamayati | prekṣāvatām ivāsyāpi niyatasthirapravṛttisiddheḥ prayojanānumitir eva nyāyaprāptā ||
	\pend
      

	  \pstart yac cedam udīritam: yadi hi sarvakāryāṇām ekaḥ kartā syāt tato 'jñasya tattvānupapatteḥ sarvajñatā syāt | \edlabel{ratnakīrtinibandhāvali__36r1PF7IMWQKR7W5N7YIKB3F4T2}\label{ratnakīrtinibandhāvali__36r1PF7IMWQKR7W5N7YIKB3F4T2}\edtext{}{\lemma{adya}\xxref{ratnakīrtinibandhāvali__36r1PF7IMWQKR7W5N7YIKB3F4T2}{ratnakīrtinibandhāvali__36r1PF7IMWPXIAKW6HP32AFTWQM}\Afootnote{adya \cite{RNAms} ; atha \cite{thakur75}   {\rmlatinfont [App type: var]}}}adya\edlabel{ratnakīrtinibandhāvali__36r1PF7IMWPXIAKW6HP32AFTWQM}\label{ratnakīrtinibandhāvali__36r1PF7IMWPXIAKW6HP32AFTWQM} punar ekaikaṃ kāryam ekaikena kartrā \leavevmode\ledsidenote{\textenglish{20b/RNAms}}\label{RNAms_20b} janyata iti yo yaj janayati sa tatkāraṇamātrajña eva na tu sarvajña iti |
	\pend
      

	  \pstart atrocyate | \edlabel{ratnakīrtinibandhāvali__36r1NSAQKYMBR7UQH8UZ9ZCFB5G}\label{ratnakīrtinibandhāvali__36r1NSAQKYMBR7UQH8UZ9ZCFB5G}kāryaliṅgāviśeṣād ekaḥ kartā \edlabel{ratnakīrtinibandhāvali__36r1NSAS7AJHGG1XWSGMMV9343A}\label{ratnakīrtinibandhāvali__36r1NSAS7AJHGG1XWSGMMV9343A}sad iti jñānāviśeṣāt sattaikatvavat | kutaścil liṅgād anumitasya vastuno nānātvasya liṅgāntarānumeyatvāt, nānātvam upapādayituṃ pramāṇāntaraṃ vaktavyam | yathātmanānātvam avasthāpayadbhiḥ \edlabel{ratnakīrtinibandhāvali__36r1PF7IMWPAVAXZV420L88W8KX}\label{ratnakīrtinibandhāvali__36r1PF7IMWPAVAXZV420L88W8KX}\edtext{}{\lemma{sukhādibhir nānātvavyavasthāpanam}\xxref{ratnakīrtinibandhāvali__36r1PF7IMWPAVAXZV420L88W8KX}{ratnakīrtinibandhāvali__36r1PF7IMWOMZHMJWONH2AU5GOY}\Afootnote{sukhādivyavasthāpanam \cite{RNAms} ; sukhādibhir nānātvavyavasthāpanam \cite{thakur75} ---\textsc{Note} It could be that there was a correction in the ms. The upper margin has some signs here (around three akṣaras), but they are completely illegible. Also, there is a dot in the middle above di and vya, which could have been a mark to insert something here.  {\rmlatinfont [App type: var]}}}sukhādibhir nānātvavyavasthāpanam\edlabel{ratnakīrtinibandhāvali__36r1PF7IMWOMZHMJWONH2AU5GOY}\label{ratnakīrtinibandhāvali__36r1PF7IMWOMZHMJWONH2AU5GOY} ucyate | na ceha kartur anekatvādhigame pramāṇāntaram asti | \edlabel{ratnakīrtinibandhāvali__36r1NSAZCVOOMC2YM2ZB4UM8USC}\label{ratnakīrtinibandhāvali__36r1NSAZCVOOMC2YM2ZB4UM8USC}ekatve tu na pramāṇātaram anveṣṭavyam, ekasya kartur abhāve \edlabel{ratnakīrtinibandhāvali__36r1NMMCGE4QHCX673V3O13PAR2}\label{ratnakīrtinibandhāvali__36r1NMMCGE4QHCX673V3O13PAR2}\edtext{}{\lemma{bahūnāṃ … syād}\xxref{ratnakīrtinibandhāvali__36r1NMMCGE4QHCX673V3O13PAR2}{ratnakīrtinibandhāvali__36r1NMMCGE62YQPZ3MBLV5PA87V}\Afootnote{\label{ratnakīrtinibandhāvali__36r1NMMCGE84C0Q7R96K5FVFGL5}\textenglish{---\textsc{Note} Corresponds to 269.}  {\rmlatinfont [App type: parallel]}}}bahūnāṃ vyāhatamanasāṃ \edlabel{ratnakīrtinibandhāvali__36r1PF7IMWNZT5VRHPNFFDTLK58}\label{ratnakīrtinibandhāvali__36r1PF7IMWNZT5VRHPNFFDTLK58}\edtext{}{\lemma{svātantryeṇa}\xxref{ratnakīrtinibandhāvali__36r1PF7IMWNZT5VRHPNFFDTLK58}{ratnakīrtinibandhāvali__36r1PF7IMWNC4M3SGL0Z4ELQ9P7}\Afootnote{svātantryeṇa \cite{RNAms} ; svātantreṇa \cite{thakur75}   {\rmlatinfont [App type: var]}}}svātantryeṇa\edlabel{ratnakīrtinibandhāvali__36r1PF7IMWNC4M3SGL0Z4ELQ9P7}\label{ratnakīrtinibandhāvali__36r1PF7IMWNC4M3SGL0Z4ELQ9P7} parasparavirodhena mithaḥ svānukūlābhiprāyānavabodhena yugapatkāryānutpattiḥ, utpannasya vā vilopādiprasaṅgaḥ syād\edlabel{ratnakīrtinibandhāvali__36r1NMMCGE62YQPZ3MBLV5PA87V}\label{ratnakīrtinibandhāvali__36r1NMMCGE62YQPZ3MBLV5PA87V} iti | ekatve tu siddhe sarvajñatāsiddhir avirodhinī | na ceśvarasya sakalakṣetrajñasamavāyidharmādharmajñānakāraṇābhāvena tadajñānam, tatsamavetānāṃ jñānacikīrṣāpratyatnānāṃ nityatvāt | na ca buddhitannityatvayoḥ kaścit virodhaḥ | na ca buddher anityatāyās tatra tatropalabdher īśvarabuddher api tathātvaṃ yuktam, rūpādīnām apy anityānāṃ tatra tatropalabdhes toyādiparamāṇusamavetānām api rūpādīnām anityatvaprasaṅgāt | parapuruṣasamavetadharmādharmādhiṣṭhānam apy asya yuktam eva, saṃyuktasaṃyogisamavāyasya sambandhasya sadbhāvāt | saṃyuktāḥ khalv īśvareṇa paramāṇavaḥ, taiś ca kṣetrajñāḥ, tatsamavetau ca dharmādharmāv iti ||
	\pend
      

	  \pstart tad evaṃ kaṇikāyāṃ vācaspater īśvaradūṣaṇaṃ yathāsāram utthāpya vyudastam asmābhiḥ | aparaṃ ca busaprāyam anabhyupagamaprasiddhasiddhāntagrastam iha granthavistarabhayān na likhitam | tad evam abhimatasyaiva sarvajñatālakṣaṇasya viśeṣasya siddher naiṣa viśeṣaviruddho hetuḥ | nāpi karmabhiḥ siddhasādhanam iti sthitam || 
	\pend
      

	  \pstart \edlabel{thakur75-36.21}\label{thakur75-36.21} na cānaikāntikaḥ | sa hi bhavann asādhāraṇo vā syāt, yathā nityā pṛthvī gandhavattvād iti, anupasaṃhāryo vā, yathā sarvaṃ nityaṃ prameyatvād iti, sādhāraṇo vā yathā nityaḥ śabdaḥ, asparśavattvād iti |
	\pend
      

	  \pstart tatra na tāvad ādimau pakṣau, sapakṣasadbhāvadarśanena pratikṣiptatvāt | nāpy antimaḥ, adhigatakartṛnivṛtter vyomāder vipakṣād vyāvṛtter upalabdheḥ | 
	\pend
      

	  \pstart nanu puruṣavyāpāram antareṇa tṛṇādīn udayamānānavalokayan lokaḥ kāryamātraṃ puruṣapūrvakam iti vyāptim eva na pratipadyata iti cet | evaṃ tarhi prasiddhānumānasthitir api dattajalāñjaliḥ | tatrāpi hi vyāptipratītikāla eva vyāghrā\leavevmode\ledsidenote{\textenglish{21a/RNAms}}\label{RNAms_21a}diparyākulātidurgapradeśe vahnivyāpāram antareṇa dhūmaṃ puruṣavyāpāraṃ vinā pūrvaṃ siddhaṃ ghaṭaṃ vā vilokayan loko dhūmamātraṃ vahnipūrvakaṃ ghaṭamātraṃ vā puruṣapūrvakam iti vyāptim eva na pratipadyata iti vaktuṃ śakyatvāt |
	\pend
      

	  \pstart tatra vahnipuruṣayor deśakālaviprakṛṣṭatvād apratikṣepa iti cet | yady evaṃ tṛṇādāv api puruṣasya svabhāvaviprakṛṣṭatvād apratikṣepa iti sarvaṃ samānam anyatrābhiniveśāt | puruṣavyāpārapūrvakatā tāvan na pratīyate tṛṇādīnām | sā ca puruṣasyādṛśyatvād asattvād vā na pratīyatām, kim anena vicāritena | sarvathā kiñcitkāryam apūrvapuruṣapūrvakam apaśyan na vyāptiṃ kāryamātrasya puruṣeṇa kaścit cetanāvān avagacchatīti cet | yady evaṃ vahnimātrapūrvakatā tāvan na pratīyate dhūmasya, puruṣamātrapūrvakatā ca ghaṭasya | sā ca vahner deśaviprakṛṣṭatvād asattvād vā puruṣasya kālaviprakṛṣṭatvād asattvād vā na pratīyatām, kim anena vicāritena | sarvathā dhūmamātraṃ vahnivyāpārapūrvakam apaśyan ghaṭamātraṃ vā puruṣapūrvakam apaśyann avyāptim eva dhūmasya vahnimātreṇa ghaṭasya puruṣamātreṇa vā kaścic cetanāvān adhigacchatīty apy ucyamānaṃ na vaktraṃ vakrīkaroti | tat kim anena prasiddhānumānāpalāpinā jātyuttareṇa ||
	\pend
      

	  \pstart \edlabel{rnā__96541}\label{rnā__96541}\edtext{}{\lemma{syād etat |  ... prabodhāśrayāyattatāsiddheḥ |}\xxref{rnā__96541}{rnā__97385}\Afootnote{\label{rnā__394381}See JNĀ 235. \cite{jna}   {\rmlatinfont [App type: parallel]}}}syād etat | na sapakṣāsapakṣayor darśanādarśanamātreṇāvyabhicāraniścayaḥ, atadātmano 'tadutpatteś cāvyabhicāraniyamābhāvāt | tad idaṃ kāryatvaṃ sandigdhavipakṣavyāvṛttikatvenāsādhanam |
	\pend
      

	  \pstart atrocyate | nāsti vipakṣād dhetor vyāvṛttisandehaḥ, dhūmānalayor iva kāryabuddhimator upalambhānupalambhasādhanasya \edlabel{ratnakīrtinibandhāvali__36r1PF7IMWMOO857NALIQLRG7O8}\label{ratnakīrtinibandhāvali__36r1PF7IMWMOO857NALIQLRG7O8}\edtext{}{\lemma{kāryakāraṇabhāvasya siddha}\xxref{ratnakīrtinibandhāvali__36r1PF7IMWMOO857NALIQLRG7O8}{ratnakīrtinibandhāvali__36r1PF7IMWM0G3TH23WNU1X5LKO}\Afootnote{ \cite{thakur75,jñānaśrī87} kāryakāraṇasiddha \cite{} }}kāryakāraṇabhāvasya siddha\edlabel{ratnakīrtinibandhāvali__36r1PF7IMWM0G3TH23WNU1X5LKO}\label{ratnakīrtinibandhāvali__36r1PF7IMWM0G3TH23WNU1X5LKO}tvāt |
	\pend
      

	  \pstart kāryaviśeṣasyaiva tadutpādasiddhir na kā\gap{}ryasāmānyasya, yathā dhūmādivartino vastutvāder nānalādijanyatvaniścaya iti cet | na | viśeṣahetvabhāvāt | upalambhānupalambhayos tadutpattisādhanatveneṣṭayor aviśeṣāt kāryaviśeṣasyeva kāryasāmānyasya prabodhāśrayāyattatāsiddheḥ |\edlabel{rnā__97385}\label{rnā__97385} yathā hi kāryaṃ vastrādyupādānavad \edlabel{ratnakīrtinibandhāvali__36r1PF7IMWLD7EYP7VS6KY50CFW}\label{ratnakīrtinibandhāvali__36r1PF7IMWLD7EYP7VS6KY50CFW}\edtext{}{\lemma{dṛṣṭam iti}\xxref{ratnakīrtinibandhāvali__36r1PF7IMWLD7EYP7VS6KY50CFW}{ratnakīrtinibandhāvali__36r1PF7IMWKOTSRMOPVD0OSH4J8}\Afootnote{ \cite{jñānaśrī87} dṛṣṭaṃ \cite{RNAms} ---\textsc{Note} This is also parallel to the tathā part.  {\rmlatinfont [App type: emendation]}}}dṛṣṭam iti\edlabel{ratnakīrtinibandhāvali__36r1PF7IMWKOTSRMOPVD0OSH4J8}\label{ratnakīrtinibandhāvali__36r1PF7IMWKOTSRMOPVD0OSH4J8} kāryāntaram apy adṛṣṭopādānam upādānavat \edlabel{ratnakīrtinibandhāvali__36r1PF7IMWK15LWUW1A687BTMQB}\label{ratnakīrtinibandhāvali__36r1PF7IMWK15LWUW1A687BTMQB}\edtext{}{\lemma{kāryatvād vyavasthāpyate}\xxref{ratnakīrtinibandhāvali__36r1PF7IMWK15LWUW1A687BTMQB}{ratnakīrtinibandhāvali__36r1PF7IMWJCSS30GYR134GAMFI}\Afootnote{ \cite{RNAms} kāryatvādy upasthāpyate \cite{thakur75} }}kāryatvād vyavasthāpyate\edlabel{ratnakīrtinibandhāvali__36r1PF7IMWJCSS30GYR134GAMFI}\label{ratnakīrtinibandhāvali__36r1PF7IMWJCSS30GYR134GAMFI}, tathā tad eva kāryaṃ vastrādi dṛṣṭakartṛkam ity adṛṣṭakartṛkam api kāryatvāt kartṛmad vyavasthāpyate | upādānasyeva kartur api kāryeṇānukṛtānvyavyatirekatvāt | tanmātranibandhanatvāc ca sarvatra kāryakāraṇavyavahārayoḥ | tasmād yathā kārya\leavevmode\ledsidenote{\textenglish{21b/RNAms}}\label{RNAms_21b}ṃ ca syān nirupādānaṃ ceti na śakyam āśaṅkitum, kāryamātrasyopādānamātrād utpādasiddheḥ tathā kāryaṃ ca bhaved akartṛkaṃ ceti nāśaṅkanīyam, kāryamātrasya kartṛmātrād utpādasiddher aviśeṣāt ||\edlabel{thakur-75-37.26}\label{thakur-75-37.26}\edtext{}{\lemma{||}\Bfootnote{\cref{thakur75-37.12} to \cref{thakur-75-37.26} corresponds to . The passage is introduced by Vittokas tv āha.}}
	\pend
      

	  \pstart nanu brūyā nāma kiñcit | tathāpi na kāryamātrād buddhimadanumānam, api tu kāryaviśeṣād eva | yaddarśanād akriyādarśino 'pi kṛtabuddhiḥ syāt | na cānapekṣitatattvānugamāc chabdamātrasāmyāt sādhyasiddhir yuktā | gośabdavācyatāmātreṇa vāgādīnāṃ viṣāṇitvānumitiprasaṅgād iti cet | tad etat svasthottaram anuttarārham, kāryasāmānyasyaiva vyāptiprasādhanāt | api ca kā punar iyaṃ kṛtabuddhiḥ, kim apekṣitaparavyāpārāvasāyo 'tha puruṣakṛtam etad iti pauruṣeyatvaniścaya iti |
	\pend
      

	  \pstart yady ādyaḥ pakṣaḥ, sa kathaṃ kṣityādiṣu nāsti, kāraṇavyāpārātmalābhalakṣaṇasya kāryatvasya kumbhādivat kṣityādiṣv aviśeṣāt | atha puruṣeṇa kṛtam iti pauruṣeyatvaniścayaḥ kṛtabuddhir abhimatā, tadāpi tādṛśī kṛtabuddhiḥ kasya nāstīti vaktavyam | kiṃ kāryatvād iti hetor avinābhāvavedina āhosvit tadviparītasya | nādyaḥ pakṣaḥ | avinābhāvavedinaḥ sādhyāpratipatter ayogāt | atha tadviparītasya sādhyabuddhir na bhavatīti kṛtabuddhihetukatvam avanitanumahīruhādiṣu nāstīti buddhimato 'numānaṃ pratikṣipyate |
	\pend
      

	  \pstart nanv evaṃ sati sarvānumānocchedaḥ syāt | sarvahetūnām agṛhītāvinābhāvaṃ praty \edlabel{ratnakīrtinibandhāvali__36r1PF7IMWIP0YA1GHFXFCWNYAY}\label{ratnakīrtinibandhāvali__36r1PF7IMWIP0YA1GHFXFCWNYAY}\edtext{}{\lemma{agamakatvāt}\xxref{ratnakīrtinibandhāvali__36r1PF7IMWIP0YA1GHFXFCWNYAY}{ratnakīrtinibandhāvali__36r1PF7IMWHZTAG0OSMDFIGKOEM}\Afootnote{agama\add{ka}tvāt \cite{} ; agamakatvāt \cite{thakur75} }}agamakatvāt\edlabel{ratnakīrtinibandhāvali__36r1PF7IMWHZTAG0OSMDFIGKOEM}\label{ratnakīrtinibandhāvali__36r1PF7IMWHZTAG0OSMDFIGKOEM} | tasmān na \edlabel{rnā__99759}\label{rnā__99759}\edtext{}{\lemma{kṛtabuddhihetutvaviśeṣaḥ}\xxref{rnā__99759}{rnā__99814}\Afootnote{\label{rnā__393999}kṛtabuddhihetutvaviśeṣaḥ \cite{RNAms} ; kṛtabuddhihetutvaṃ viśeṣaḥ \cite{thakur75} ---\textsc{Note} Emend to kṛtabuddhihetu-ka-tvaviśeṣaḥ?  {\rmlatinfont [App type: var]}}}kṛtabuddhihetutvaviśeṣaḥ\edlabel{rnā__99814}\label{rnā__99814} | bhavatu vā kaścid anirūpitarūpo viśeṣas tathāpi kim anena | kāryamātrasyaiva dhūmamātrasyeva vyāptipratīteḥ | na ca kāryatvena hetunā saha mṛdvikārasya samakakṣatā | tasya svasādhyena dṛśyakumbhakāreṇa saha vyabhicārasya śataśo darśanāt | kāryatvasya tu dṛśyādṛśyasādhāraṇena buddhimanmātreṇa tadyogād iti nāyam anaikāntikaḥ |
	\pend
      

	  \pstart nāpi prakaraṇasamaḥ, apratipakṣatvāt | na hy asya pratipakṣopasthāpakaṃ dharmāntaram asti | \edlabel{RNAms-add-1-start}\label{RNAms-add-1-start}yathā nityaḥ śabdo vastutve saty anupalabhyamā\edlabel{ratnakīrtinibandhāvali__36r1PF7IMWHBAID9C9MEN8DPBJV}\label{ratnakīrtinibandhāvali__36r1PF7IMWHBAID9C9MEN8DPBJV}\edtext{}{\lemma{nānityadharmatvād}\xxref{ratnakīrtinibandhāvali__36r1PF7IMWHBAID9C9MEN8DPBJV}{ratnakīrtinibandhāvali__36r1PF7IMWGLU1HYJ7TI9KDX9AW}\Afootnote{ \cite{} nanityadharmatvād \cite{thakur75} ---\textsc{Note} Cf. \href{thakur97:_gautam_with_bhAsy_vAtsy}{thakur97:\textunderscore gautam\textunderscore with\textunderscore bhAsy\textunderscore vAtsy} for a similar thought.  {\rmlatinfont [App type: em]}}}nānityadharmatvād\edlabel{ratnakīrtinibandhāvali__36r1PF7IMWGLU1HYJ7TI9KDX9AW}\label{ratnakīrtinibandhāvali__36r1PF7IMWGLU1HYJ7TI9KDX9AW} ity asya, anityaḥ śabdo vastutve saty anupalabhyamānanityadharmatvād iti pratipakṣakṛtaṃ dharmāntaram asti |\edlabel{RNAms-add-1-end}\label{RNAms-add-1-end} na cedaṃ bādhakaṃ vaktavyam | neśvarakartṛkaṃ jagat | vastutvasattvād ityādi | īśvarakartṛkatvasya vastutvād iti virodhābhāvāt | iti nāyaṃ prakaraṇasamo 'pi |
	\pend
      

	  \pstart na ca kālātyayāpadiṣṭaḥ pratyakṣānumānāgamair bādhitaviṣayasya tathābhāvāt | asya ca tair avirodhāt | tatra pratyakṣaviruddhaḥ, anuṣṇas tejo'vayavī kṛtakatvāt | anumānavi\leavevmode\ledsidenote{\textenglish{22a/RNAms}}\label{RNAms_22a}ruddhaḥ, sāvayavāḥ paramāṇavo mūrtatvāt | āgamaviruddhaḥ, śucina[ra]śiraḥkapālaṃ prāṇyaṅgatvād iti | tatra na tāvad ayaṃ pratyakṣaviruddhaḥ, sādhyaviparyayasya pratyakṣāviṣayatvāt | nāpy anumānaviruddhaḥ, dharmigrāhiṇānumānenābādhitaviṣayatvāt | na cāgamaviruddhaḥ, āgamena sādhyaviparyayasyāparicchedāt | saugatādyāgamair viparītaparicchedād iti cet | na, teṣāṃ kṣaṇikatvādyarthavisaṃvādopalambhena prāmāṇyābhāvāt | vedāgamo 'pi bādhakatvena nāśaṅkanīyaḥ,
	\pend
      
	    
	    \stanza[\smallbreak]
	sahasraśīrṣā puruṣaḥ \&[\smallbreak]


	

	  \pstart ityādinā tatra kartur eva pratipādanāt | tathābhūtapuruṣātiśayapūrvakatvābhāve satyaprāmāṇyāc ceti nāyam atikrāntakālo hetuḥ | tad evam apanītahetvabhāsavibhramād ataḥ sādhanād upādānādyabhijño buddhimān abhimataḥ kartā sidhyati | sa  eva bhagavān asmākam īśvara iti sthitam ||
	\pend
      

	  \pstart \edlabel{sarit__ratnakīrtinibandhāvali__102493}\label{sarit__ratnakīrtinibandhāvali__102493}\edtext{}{\lemma{tathāsya … tatheti |}\xxref{sarit__ratnakīrtinibandhāvali__102493}{sarit__ratnakīrtinibandhāvali__103091}\Afootnote{\label{sarit__ratnakīrtinibandhāvali__397794}---\textsc{Note} Sanskrit and translation in \cite[62--64]{krasser02_zaGkar_Izvar_studie}  {\rmlatinfont [App type: parallel]}}}tathāsya siddhaye śaṅkaraḥ sādhanam idam abhipraiti—
	\pend
      

	  \pstart jagad etat prabodhāśrayāyattaprasavam \edlabel{rnā__102330}\label{rnā__102330}\edtext{}{\lemma{abhilāpa}\xxref{rnā__102330}{rnā__102368}\Afootnote{\label{rnā__394697}abhilāpa \cite{RNAms} ; abhilāṣa \cite{thakur75}   {\rmlatinfont [App type: var]}}}abhilāṣa\edlabel{rnā__102368}\label{rnā__102368}prītiparamāṇumūrtyādhāraparatvāparatvānumeyasāmānyasamavāyāntyaviśeṣatadekārthasamavetaparimāṇaikatvapṛthaktvagurutvasnehāpārthivarūparasasparśāpyadravatvāmūrtasaṃyogataditaretarābhāvānutpattirūpārūpam asmadādivinirmitetarat |
	\pend
      

	  \pstart acetanopādānatvāt |
	\pend
      

	  \pstart yad itthaṃ tat tathā, yathā kalasaḥ |
	\pend
      

	  \pstart tathā cedam |
	\pend
      

	  \pstart tasmād idam api tatheti |\edlabel{sarit__ratnakīrtinibandhāvali__103091}\label{sarit__ratnakīrtinibandhāvali__103091}
	\pend
      

	  \pstart asyāyam arthaḥ | jagad iti dharmī | prabodhāśrayāyattaprasavam iti sādhyam | \edlabel{rnā__102859}\label{rnā__102859}\edtext{}{\lemma{abhilāpe}\xxref{rnā__102859}{rnā__102897}\Afootnote{\label{rnā__395000}abhilāpe \cite{RNAms} ; abhilāṣe \cite{thakur75}   {\rmlatinfont [App type: var]}}}abhilāṣe\edlabel{rnā__102897}\label{rnā__102897}tyādy anutpattirūpārūpaparyantena dharmiviśeṣeṇākāśādinityavargaparihāraḥ | asmadādivinirmitetarad ity anenāpi dharmiviśeṣeṇa prasiddhakartṛkaghaṭādiparihāraḥ | \edlabel{rnā__103110}\label{rnā__103110}\edtext{}{\lemma{abhilāpaś}\xxref{rnā__103110}{rnā__103149}\Afootnote{\label{rnā__395215}abhilāpaś \cite{ms} ; abhilāṣaś \cite{s} ; \texttibetan{not checked} \cite{tib}   {\rmlatinfont [App type: var]}}}abhilāpaś\edlabel{rnā__103149}\label{rnā__103149} ca prītiś ca paramāṇumūrtiś ca | \edlabel{ratnakīrtinibandhāvali__36r1PF7IMWFWV6FZAOIU39HWKEF}\label{ratnakīrtinibandhāvali__36r1PF7IMWFWV6FZAOIU39HWKEF}\edtext{}{\lemma{āsām ādhāraḥ |}\xxref{ratnakīrtinibandhāvali__36r1PF7IMWFWV6FZAOIU39HWKEF}{ratnakīrtinibandhāvali__36r1PF7IMWF7GMMH0PB0HUCRS3K}\Afootnote{ \cite{} āsāmadhāra  \cite{thakur75} }}āsām ādhāraḥ |\edlabel{ratnakīrtinibandhāvali__36r1PF7IMWF7GMMH0PB0HUCRS3K}\label{ratnakīrtinibandhāvali__36r1PF7IMWF7GMMH0PB0HUCRS3K}ākāśa ātmā paramāṇuḥ | paratvāparatvānumeyau dikkālau | sāmānyādayas tu yathāprasiddhā grahītavyāḥ |
	\pend
      

	  \pstart \edlabel{sarit__ratnakīrtinibandhāvali__103807}\label{sarit__ratnakīrtinibandhāvali__103807}\edtext{}{\lemma{tathā … tatheti |}\xxref{sarit__ratnakīrtinibandhāvali__103807}{sarit__ratnakīrtinibandhāvali__104267}\Afootnote{\label{sarit__ratnakīrtinibandhāvali__398247}---\textsc{Note} Sanskrit and translation in \cite[61, n.~80]{krasser02_zaGkar_Izvar_studie}  {\rmlatinfont [App type: parallel]}}}tathā narasiṃhaḥ prāha—
	\pend
      

	  \pstart vijñānādhārādhīna\edlabel{ratnakīrtinibandhāvali__36r1PF7IMWEJ1OLG89TI3U468WA}\label{ratnakīrtinibandhāvali__36r1PF7IMWEJ1OLG89TI3U468WA}\edtext{}{\lemma{janmājanmā}\xxref{ratnakīrtinibandhāvali__36r1PF7IMWEJ1OLG89TI3U468WA}{ratnakīrtinibandhāvali__36r1PF7IMWDOD3VAWL9JR4YL56Z}\Afootnote{jan\add{mā}janmā \cite{} }}janmājanmā\edlabel{ratnakīrtinibandhāvali__36r1PF7IMWDOD3VAWL9JR4YL56Z}\label{ratnakīrtinibandhāvali__36r1PF7IMWDOD3VAWL9JR4YL56Z}vacchinnātmobhayavādyavivādāspadapuruṣapūrvakavyatireki bhāvānubhāvi prameyajātam |
	\pend
      

	  \pstart utpattimattvāt |
	\pend
      

	  \pstart yad yad ākhyātasādhanasambandhi tat tad uktasādhyadharmādhikaraṇam | yathā vāsaḥ |
	\pend
      

	  \pstart tathā cedam |
	\pend
      

	  \pstart tasmād idam api tatheti |\edlabel{sarit__ratnakīrtinibandhāvali__104267}\label{sarit__ratnakīrtinibandhāvali__104267}
	\pend
      

	  \pstart asyāyam arthaḥ | prameyajātaṃ dharmi | vijñānādhārādhīnajanmeti sādhyam | ajanmāvacchinnātmeti dharmiviśeṣaṇam | etenākāśādinityavargaparihāraḥ | ubhayavādyavivādāspadapuruṣapūrvakavya\leavevmode\ledsidenote{\textenglish{22b/RNAms}}\label{RNAms_22b}tirekīty anenāpi prasiddhakartṛkaghaṭādiparihāraḥ | bhāvānubhāvīti vasturūpam | etena pradhvaṃsādiparihāraḥ | yad yadākhyātasādhanasambandhīti vyāptivacanaṃ yaddharmirūpam kathitasādhanayogīty arthaḥ |
	\pend
      

	  \pstart \label{sarit__ratnakīrtinibandhāvali__104795}\persName{trilocanas} tu vyatirekiṇam imaṃ prayogam āha —
	\pend
      

	  \pstart sarvaṃ kāryaṃ prabodhavaddhetukam |
	\pend
      

	  \pstart utpattidharmakatvāt |
	\pend
      

	  \pstart yan nityaṃ dṛṣṭam abodhavaddhetukaṃ tasyākāśādes tathotpattir nāstīti dṛṣṭam |
	\pend
      

	  \pstart utpattidharmakaṃ ca pakṣīkṛtam asmadādivinirmitetarat |
	\pend
      

	  \pstart tasmād bodhavaddhetukam iti |\edlabel{sarit__ratnakīrtinibandhāvali__105213}\label{sarit__ratnakīrtinibandhāvali__105213}
	\pend
      

	  \pstart punar dvyaṇukeśvarasiddhau \persName{trilocana} eva prāha—
	\pend
      

	  \pstart vivādāspadībhūtaṃ dvitvam ātmotpattau kasyacid ekaikaviṣayāṃ buddhim apekṣate |
	\pend
      

	  \pstart dvitvasaṃkhyātvāt |
	\pend
      

	  \pstart yad yad dvitvaṃ tat tathā | yathā dve dravye |
	\pend
      

	  \pstart tathā cedaṃ dvyaṇukagataṃ dvitvam |
	\pend
      

	  \pstart tasmāt tatheti |
	\pend
      

	  \pstart yasya cātra buddhir apekṣyate sa bhagavān īśvaraḥ ||
	\pend
      

	  \pstart tathā ca Vācaspatiḥ pramāṇayati—
	\pend
      

	  \pstart vivādādhyāsitatanutarugirisāgarādayaḥ upādānādyabhijñakartṛkāḥ |
	\pend
      

	  \pstart kāryatvāt |
	\pend
      

	  \pstart yad yat kāryaṃ tat tad upādānādyabhijñakartṛkam | yathā prāsādādi |
	\pend
      

	  \pstart tathā ca vivādādhyāsitās tanvādayaḥ |
	\pend
      

	  \pstart tasmāt tatheti |
	\pend
      

	  \pstart evaṃ sthitvā sthitvā pravṛttidharmakatvāt, sanniveśavattvāt, arthakriyākāritvād ityādayo hetavaḥ kathitapañcāvayavakrameṇa boddhavyā iti |
	\pend
      \label{īsd-uttarapakṣa}
	  
	% new div opening: depth here is 1
	\label{īsd-vyāptigrahaṇa}
	  
	% new div opening: depth here is 2
	

	  \pstart tad etad durmativispanditaṃ jagadandhīkaraṇaṃ na satām upekṣitum ucitam iti kiñcid ucyate | iha khalu buddhimatkāryamātrayoḥ sādhyasādhanayoḥ sarvopasaṃhāravatī vyāptis tāvad avaśyaṃ grahītavyā | anyathā gamyagamakabhāvāyogāt | sā ca gṛhyamāṇā kiṃ kāraṇakāryamātrayor iva viparyayabādhakapramāṇabalāt grāhyā | yad vā 'gnidhūmayor iva viśiṣṭānvayavyatirekagrahaṇapravaṇaviśiṣṭapratyakṣānupalambhābhyāṃ boddhavyā | uta svavyavasthayā sapakṣāsapakṣayor bhūyor darśanādarśanābhyāṃ pratyetavyā | āhosvit sapakṣāsapakṣayoḥ sakṛddarśanābhyāṃ jñātavyeti catvāro vikalpāḥ |
	\pend
      
	  
	% new div opening: depth here is 3
	

	  \pstart na tāvad ādyaḥ pakṣaḥ, sādhyaviparyaye buddhimadabhāve kāryatvasāmānyasya sādhanasya bādhakapramāṇābhāvāt. nanu bādhakapramāṇā\edlabel{ratnakīrtinibandhāvali__36r1PF7IMWCZ259HVW7MWE6RCOV}\label{ratnakīrtinibandhāvali__36r1PF7IMWCZ259HVW7MWE6RCOV}\edtext{}{\lemma{bhāvo 'siddhaḥ}\xxref{ratnakīrtinibandhāvali__36r1PF7IMWCZ259HVW7MWE6RCOV}{ratnakīrtinibandhāvali__36r1PF7IMWC93RHV8K3QH5DZGWS}\Afootnote{ \cite{thakur75} bhāvosiddhaḥ \cite{} }}bhāvo 'siddhaḥ\edlabel{ratnakīrtinibandhāvali__36r1PF7IMWC93RHV8K3QH5DZGWS}\label{ratnakīrtinibandhāvali__36r1PF7IMWC93RHV8K3QH5DZGWS}. tathā hīdaṃ kāryatvaṃ yathā buddhimatā vyāptam iṣyate tathā deśakālasvabhāvaniyatatvenāpi, kadācikakāraṇasannidhimattayāpi, sāmagrīkāryatvenāpi vyāptam upalabdham | sa ca deśādiniyamaḥ kādācitkakāraṇasanndhiḥ sāmagrī vā buddhimatpūrvikā siddhā | yadi punar acetanāni cetanānadhiṣṭhatāni kāryaṃ kuryuḥ tato yatra kvacanāvasthitāni janayeyur iti na deśakālasvabhāvaniyataprasavaṃ kāryam upalabhyeta |
	\pend
      

	  \pstart hetusamavadhānajanmatayā na kāryaṃ pratyekaṃ kāraṇair janyata iti cet | samavadhānam eva tu kāraṇānāṃ kutaḥ | kādācitkaparipākādadṛṣṭaviśeṣād iti cet | nanv ayam acetanaḥ kathaṃ yathāvat kāraṇāni sannidhāpayet | no khalu kvacid avasthitāni daṇḍādīni vinā kumbhakāraprayatnam adṛṣṭaviśeṣavaśād eva parasparaṃ sannidhīyante | sannihitāni vā kāryāya prabhavantīti buddhimatā deśakālasvabhāvaniyamasya kādācitkakāraṇasannidheḥ sāmagryāś ca vyāptisiddhiḥ | buddhimadabhāve caiṣāṃ vyāpakānāṃ nivṛttau nivartamānaṃ kāryatvaṃ buddhimatpūrvakatvena vyāpyata iti pratibandhasiddhaye vyāpakānupalambhatrayam upanyastam | \edlabel{ratnakīrtinibandhāvali__36r1PF7IMWBJY889FM807QZKIA6}\label{ratnakīrtinibandhāvali__36r1PF7IMWBJY889FM807QZKIA6}\edtext{}{\lemma{tathā na}\xxref{ratnakīrtinibandhāvali__36r1PF7IMWBJY889FM807QZKIA6}{ratnakīrtinibandhāvali__36r1PF7IMWATRHA5MXY189N75LV}\Afootnote{ \cite{RNAms} tathā ca na \cite{thakur75} }}tathā na\edlabel{ratnakīrtinibandhāvali__36r1PF7IMWATRHA5MXY189N75LV}\label{ratnakīrtinibandhāvali__36r1PF7IMWATRHA5MXY189N75LV} kāryaṃ buddhimatparityāgād ahetukam eva bhavatīti sambhāvyam, deśakālasvabhavaniyamābhāvaprasaṅgāt | nāpi buddhimato 'nyasmād eva bhavatīti śaṅkanīyam, sakṛd apy utpādābhāvaprasaṅgāt | na cānyasmād asmād api bhavatīti sambhāvyam, aniyatahetutve 'hetutvaprasaṅgāt | tathā buddhimantam antareṇāce tanena karaṇe sarvadā kriyāyā avirāmaprasaṅgaś cety api viparyayabādhakam atiprasaṅgacatuṣṭayaṃ vyāptiprasādhakam iti | kāryatvasya hetupūrvakatvam iva buddhimatpūrvakatvam apy avāryam iti cet |
	\pend
      

	  \pstart atrocyate | sidhyaty evedaṃ manorājyaṃ yadi deśakālasvabhāvaniyamasya kādācitkakāraṇasannidheḥ samagryāś ca buddhimatpūrvakatvena vyāptiḥ sidhyati | kevalam etad eva durāpam | buddhimadabhāve 'pi hi svahetubalasamutpannasannidheḥ pratiniyatadeśakālaśaktinācetanenāpi sāmagrīlakṣaṇakāraṇaviśeṣeṇa kriyamāṇāni deśakālasvabhāvaniyamakādcitkakāraṇasannidhisāmagrīkāryatvāni yujyanta iti sandigdhāsiddhā vyāpakānupalabdhayaḥ ||
	\pend
      

	  \pstart buddhimadabhāve samavadhānam eva kuta iti cet | tad api cetanānadhiṣṭhitayathoktācetanasāmagrīviśeṣād eva | so 'pi tādṛśād ity anādyacetanasāmagrīparamparāto 'pi deśādiniyamasambhāvanāyāṃ nāvaśyaṃ buddhimadapekṣā | ghaṭāder deśakālasvabhāvaniyamaḥ kādācitkakāraṇasannidhiś ca, sāmagrī ca buddhimatpūrvikā dṛṣṭā ity aparopi deśakālasvabhāvaniyamādis tathaiveti cet | yady evaṃ ghaṭādikam api kāryaṃ bahuśo buddhimatpūrvakam upalabdham iti sarvam eva kāryaṃ tathāstu, kim anena vyāpakānupalambhopanyāsadurvyasanena | ghaṭāder bahuśo buddhimatpūrvakatvadarśane 'pi na sarvatra kāryamātrasya tathābhāvaniścayaś cet | deśādiniyamādīnām apīdaṃ samānam iti katham atrāpi śaṅkāvyudāsaḥ ||
	\pend
      

	  \pstart astu tadā pratyakṣam eva sarvatra vyāptigrāhakam iti cet | na tarhi viparyayabādhakapramāṇabalād vyāptigrahanirvāhaḥ | pratyakṣaṃ ca tatrāśaktam iti dvitīyavikalpāvasare nivedayiṣyate | \edlabel{ratnakīrtinibandhāvali__36r1PF7IMWA4775R1JUSOZJ6SAL}\label{ratnakīrtinibandhāvali__36r1PF7IMWA4775R1JUSOZJ6SAL}\edtext{}{\lemma{tathāsiddhe}\xxref{ratnakīrtinibandhāvali__36r1PF7IMWA4775R1JUSOZJ6SAL}{ratnakīrtinibandhāvali__36r1PF7IMW9DJ9QCGXQ796B6FKO}\Afootnote{ \cite{RNAms} tathā siddhe \cite{thakur75} }}tathāsiddhe\edlabel{ratnakīrtinibandhāvali__36r1PF7IMW9DJ9QCGXQ796B6FKO}\label{ratnakīrtinibandhāvali__36r1PF7IMW9DJ9QCGXQ796B6FKO} kāryakāraṇabhāve dhūmasyāhetukotpattāv anyasmād evotpattāv anyasmād apy utpattau sambhāvyamānāyāṃ \edlabel{ratnakīrtinibandhāvali__36r1PF7IMW8O0QHG35NKE9RFYAP}\label{ratnakīrtinibandhāvali__36r1PF7IMW8O0QHG35NKE9RFYAP}\edtext{}{\lemma{deśādiniyamābhāvasakṛdutpādābhāvāhetutvaprasaṅgāḥ}\xxref{ratnakīrtinibandhāvali__36r1PF7IMW8O0QHG35NKE9RFYAP}{ratnakīrtinibandhāvali__36r1PF7IMW7WHQ84Y4DKMZ52NWG}\Afootnote{deśādiniyamābhāva\add{\unclear{sākṛ}dutpādāhetutva}prasaṅgāḥ \cite{RNAms} ; deśādiniyamābhāvakḷptahetutyāgānyahetutvaprasaṅgāḥ \cite{thakur75} ---\textsc{Note} Emended according to the options given in \cref{RNA-ms-23a-3} to \cref{ĪSD-41-9}.}}deśādiniyamābhāvasakṛdutpādābhāvāhetutvaprasaṅgāḥ\edlabel{ratnakīrtinibandhāvali__36r1PF7IMW7WHQ84Y4DKMZ52NWG}\label{ratnakīrtinibandhāvali__36r1PF7IMW7WHQ84Y4DKMZ52NWG} saṅgacchante | prastute tu buddhimatkāryamātrayoḥ kāryakāraṇabhāvo nādyāpi siddhaḥ | sādhayituṃ vā śakyaḥ | na cācetanasya kartṛtve kriyāyā \leavevmode\ledsidenote{\textenglish{42/thakur75}} avirāmaprasaṅgaḥ saṅgataḥ | na hy acetanam ity eva sarvadā sāmarthyayogi, tasyāpi svahetuparamparāpratibaddhasāmarthyatvād ity acetanakāraṇaviśeṣaparamparāsambhāvanāyāṃ nāvaśyaṃ buddhimadākṣepa iti svamatavyālopaviklavavikrośitamātram evedaṃ na punar atra nyāyagandho 'pi |
	\pend
      

	  \pstart tad evaṃ vyāptisādhanārtham upanyastaṃ vyāpakānupalambhatrayaṃ sandigdhāsiddham atiprasaṅgatucaṣṭayaṃ ca buddhimatkāryamātrayor vyāptyasiddhāv asaṅgatam | ataḥ kāryatvaṃ sādhanaṃ sandigdhavipakṣavyāvṛttikatvād anaikāntikam || 
	\pend
      

	  \pstart atra Vācaspatiḥ prāha: sandigdhavipakṣavyāvṛttikatvaṃ nāma hetudoṣa eva na bhavati | tat kathaṃ nirasyate | tathā hi ya eva vipakṣe dṛṣṭo hetuḥ sa eva prameyatvādivad abhimataṃ na sādhyet | yas tu mahatāpi prayatnena mṛgyamāṇo 'sapakṣe nopalakṣitaḥ sa kathaṃ sādhyaṃ na sādhayet | 
	    \pend
	  
	    
	    \stanza[\smallbreak]
	\label{sarit__ratnakīrtinibandhāvali__111722}\edtext{}{\lemma{avaśyaṃ ... apaśyatām |}\xxref{sarit__ratnakīrtinibandhāvali__111722}{sarit__ratnakīrtinibandhāvali__111823}\Afootnote{\label{sarit__ratnakīrtinibandhāvali__402429}\textenglish{---\textsc{Note} Quoted from .}  {\rmlatinfont [App type: parallel]}}}avaśyaṃ śaṅkayā bhāvyaṃ niyāmakam apaśyatām |\label{sarit__ratnakīrtinibandhāvali__111823}\&[\smallbreak]


	
	    \pstart
	   iti tu dattāvakāśā laukikam aryādātikrameṇa saṃśayapiśācī labdhaprasarā na kvacin nāstīti nāyaṃ kvacit pravarteta | sarvasyaivārthasya kathañcic chaṅkāspada\edlabel{ratnakīrtinibandhāvali__36r1PF7IMW74UT7SUMA8GL8PWHC}\label{ratnakīrtinibandhāvali__36r1PF7IMW74UT7SUMA8GL8PWHC}\edtext{}{\lemma{tvadarśanāt}\xxref{ratnakīrtinibandhāvali__36r1PF7IMW74UT7SUMA8GL8PWHC}{ratnakīrtinibandhāvali__36r1PF7IMW6DS4HL565OJR8D4FX}\Afootnote{ \cite{RNAms} tvādarśanāt \cite{thakur75} }}tvadarśanāt\edlabel{ratnakīrtinibandhāvali__36r1PF7IMW6DS4HL565OJR8D4FX}\label{ratnakīrtinibandhāvali__36r1PF7IMW6DS4HL565OJR8D4FX} | anarthaśaṅkāyāś ca prekṣāvatāṃ nivṛttyaṅgatvāt | antataḥ snigdhānnapānopayoge 'pi maraṇadarśanāt | tasmāt prāmāṇikalokayātrām anupālayatā yathādarśanaṃ śaṅkanīyam, na tv adṛṣṭam api | viśeṣasmṛtyapekṣo hi saṃśayo nāsmṛter bhavati | na ca smṛtir ananubhūtacare bhavati |
	\pend
      

	  \pstart tad uktaṃ mīmāṃsāvārttikakṛ\leavevmode\ledsidenote{\textenglish{24a/RNAms}}tā adhyuṣṭasahasrikāyām: 
	    \pend
	  
	    
	    \stanza[\smallbreak]
	nāśaṅkā niḥpramāṇiketi |\&[\smallbreak]


	
	    \pstart
	  \edtext{\textsuperscript{*}}{\lemma{*}\Bfootnote{(ŚV II 60cd)}}
	\pend
      

	  \pstart tathā tenaiva Bṛhaṭṭīkāyām:
	\pend
      
	    
	    \stanza[\smallbreak]
	utprekṣeta hi yo mohād ajātam api bādhakam |&sa sarvavyavahāreṣu saṃśayātmā kṣayaṃ vrajet || iti | \edtext{}{\lemma{|}\Bfootnote{(=TS 2871) \href{../../kamalaśīla/texts/tsp_ges.xml\#ts-2872}{../../kamalaśīla/texts/tsp\textunderscore ges.xml\#ts-2872}}}\&[\smallbreak]


	

	  \pstart tad etat pralāpamātram | na hi mahatāpi prayatnena vipakṣe mṛgyamāṇasya hetor adarśanamātreṇa vyatirekaḥ sidhyati | tathā hi vipakṣe hetur nopalabhyata ity anena tadupalambhakapramāṇanivṛttir ucyate |  pramāṇaṃ ca prameyasya kāryam, nākāraṇaṃ viṣaya iti nyāyāt | na ca kāryanivṛttau kāraṇanivṛttir upalabdhā, nirdhūmasyāpi vahner upalambhāt | yadi punaḥ pramāṇasattayā prameyasattā vyāptā syāt, tadā yuktam etat | kevalam iyam eva vyāptir asambhavinī, sarvasya sarvadarśitvaprasaṅgāt | tan nādarśanamātreṇa vyatirekasiddhiḥ | yathoktam: 
	    \pend
	  
	    
	    \stanza[\smallbreak]
	sarvādṛṣṭiś ca sandigdhā svādṛṣṭir vyabhicāriṇī |&vindhyādrirandhradūrvāder adṛṣṭāv api sattvataḥ || iti \edtext{}{\lemma{iti}\Bfootnote{(=TS 122)}}\&[\smallbreak]


	
	    \pstart
	   sakalavipakṣasyārvācīnaṃ praty adṛśyatvāt ||
	\pend
      

	  \pstart yac coktam: saṃśayapiśācī labdhaprasarā na kvacin nāstīti na kvacit pravarteteti | tad asaṅgatam | arthasaṃśayasyāpi prekṣāvatāṃ pravṛttyaṅgatvāt pravṛttir avirodhiny eva | anarthasandehaḥ sarvatra kartuṃ śakyate | antataḥ snigdhānnapānopayoge 'pi maraṇadarśanād apravṛttir iti cet | durjñānam etat | tathā hy arthasandeho 'narthasandeho veti nāyaṃ ṣaṣṭhīsamāsaḥ | kin tv arthonmukhaḥ sandeho 'rthasandehaḥ, anarthonmukhaḥ sandeho 'narthasandeha iti śākapārthivādivanmadhyapadalopī samāsaḥ | evaṃ sati snigdhānnapānādāv arthasandeha eva, tajjātīyasya svaparasantāne dṛṣṭipuṣṭyādyarthasya koṭiśaḥ karaṇadarśanāt, maraṇāder anarthsya kvacit kadācid darśanāt | \edlabel{ratnakīrtinibandhāvali__36r1PF7IMW5NG28AHCCLVFKB31B}\label{ratnakīrtinibandhāvali__36r1PF7IMW5NG28AHCCLVFKB31B}\edtext{}{\lemma{etadviparīto}\xxref{ratnakīrtinibandhāvali__36r1PF7IMW5NG28AHCCLVFKB31B}{ratnakīrtinibandhāvali__36r1PF7IMW4W5CE0I89H8GG6RQN}\Afootnote{ \cite{RNAms} etadviviparīto \cite{thakur75} }}etadviparīto\edlabel{ratnakīrtinibandhāvali__36r1PF7IMW4W5CE0I89H8GG6RQN}\label{ratnakīrtinibandhāvali__36r1PF7IMW4W5CE0I89H8GG6RQN} 'narthasandeho draṣṭavyaḥ | tasmāt pramāṇādivārthasaṃśayād api prekṣāvatāṃ tatra tatra pravṛttir durvāraiva ||
	\pend
      

	  \pstart yad apīdaṃ lapitaṃ yathādarśanaṃ śaṅkanīyaṃ nādṛṣṭapūrvam api viśeṣasmṛtyapekṣo hi saṃśaya ityādi | tad asambaddhaṃ | sādhakabādhakapramāṇābhāvād eva paryudāsavṛttyā \edlabel{ratnakīrtinibandhāvali__36r1PF7IMW45SFTLRLUVTIHXJ6N}\label{ratnakīrtinibandhāvali__36r1PF7IMW45SFTLRLUVTIHXJ6N}\edtext{}{\lemma{vastvantara}\xxref{ratnakīrtinibandhāvali__36r1PF7IMW45SFTLRLUVTIHXJ6N}{ratnakīrtinibandhāvali__36r1PF7IMW3EBY5XOPZTJQUZJ9I}\Afootnote{ \cite{RNAms} vasvantara \cite{thakur75} }}vastvantara\edlabel{ratnakīrtinibandhāvali__36r1PF7IMW3EBY5XOPZTJQUZJ9I}\label{ratnakīrtinibandhāvali__36r1PF7IMW3EBY5XOPZTJQUZJ9I}rūpāt sarvatra saṃśayotpatteḥ | kiṃ ca viśeṣasmṛtyapekṣa evāyaṃ saṃśayaḥ | tathā hi lakṣaṇayogitvāyogitvābhyām eva tajjātīyātajjātīye vaktavye | anyathā lakṣaṇapraṇayanam anarthakaṃ syāt | evaṃ ca sati tādātmyatadutpattilakṣaṇapratibandhaviyogitvena sādhāraṇena dharmeṇa prameyatvadhūmatvakārya\leavevmode\ledsidenote{\textenglish{24b/RNAms}}tvādīnāṃ tvanmatena sajātīyatvāt prameyatvavyabhicāradarśanam eva śaṅkām upasthāpayatīti yathādarśanam evedam āśaṅkitam |
	\pend
      

	  \pstart yaś ca Kumārilasya sākṣitvenopanyāsaḥ sa khalu 
	    \pend
	  
	    
	    \stanza[\smallbreak]
	dadhibhāṇḍe viḍālaḥ sākṣīti\&[\smallbreak]


	
	    \pstart
	   pravādaṃ nātipatatīti kim atra vaktavyam | tad evaṃ vipakṣe 'darśanamātreṇa hetor vyatirekāsiddheḥ sandigdhavipakṣavyāvṛttikatvaṃ nāma hetudūṣaṇaṃ durvāram eva | ata evāsyopanyāso 'doṣodbhāvanaṃ nāma nigrahasthānam iti yad anenāveditaṃ tad api sāvadyam | pratyutāsmin heto\gap{}ḥ saddūṣaṇe parihartavye nāyaṃ hetudoṣo 'to na parihartavyo 'sya copanyāso 'doṣodbhāvanaṃ nāma nigrahasthānam iti bruvann ayam eva tapasvī svamatena niranuyojyānuyogalakṣaṇena nigrahasthānena nigṛhyata iti kṛpām arhati | tad evaṃ viparyayabādhakapramāṇābhāvād avyāpter asiddheḥ sandigdhavipakṣavyāvṛttikatvād anaikāntikaḥ kāryatvalakṣaṇo hetuḥ ||
	\pend
      
	  
	% new div opening: depth here is 3
	

	  \pstart athāgnidhūmayor iva viśiṣṭānvayavyatirekagrahaṇapravaṇaviśiṣṭapratyakṣānupalambhābhyāṃ vyaptir niścīyata iti dvitīyaḥ pakṣaḥ | atrocyate | kiṃ dṛśyaśarīropādhinā buddhimanmātreṇa vyāptigṛhyate, āhosvit dṛśyaśarīropādhividhureṇa dṛśyādṛśyasādhāraṇeneti vikalpau | yady ādyaḥ pakṣah, tadā tathābhūtasādhyam antareṇāpy utpadyamāne viṭapādau kāryatvadarśanāt prameyatvādivat sādhāraṇānaikāntiko hetuḥ |
	\pend
      

	  \pstart \edlabel{thakur75-44.2}\label{thakur75-44.2} nanu vṛkṣādayaḥ pakṣīkṛtāḥ | kathaṃ tair vyabhicāraḥ | trividho hi bhāvarāśiḥ | sandigdhakartṛko yathā vṛkṣādiḥ | prasiddhakartṛko yathā ghaṭādiḥ | akartṛko yathā ākāśādiḥ | tatra prasiddhakartṛke ghaṭādau pratyakṣānupalambhābhyāṃ vyāptim ādāya sandehapade kṣmāruhādau kāryatvam upasaṃhṛtya buddhimān anumīyate | na punar asu vyabhicāraviṣayo bhavitum arhati | \edlabel{thakur75-44.8}\label{thakur75-44.8} yad āha: na sādhyenaiva vyabhicāra iti | ayuktam etat | na hi vyabhicāraviṣaya eva pakṣe bhavitum arhati:
	\pend
      
	    
	    \stanza[\smallbreak]
	sandigdhe hetuvacanād vyasto hetor anāśrayaḥ\edtext{}{\lemma{anāśrayaḥ}\Bfootnote{(PV IV 91)}}\&[\smallbreak]


	

	  \pstart iti nyāyāt | vyabhicāraviṣayatā ca dṛśyaśarīropādher buddhimanmātrasya tṛṇādyutpattau dṛśyānupalambhena pratikṣiptatvāt | tataś ca kṣmādharādir eva sandigdhakartṛkaḥ pakṣīkartum ucitaḥ kṣmāruhādis tv acetanakartṛka iti caturtho bhavarāśir neṣṭavyaḥ | atha vyabhicāracamat\edlabel{ratnakīrtinibandhāvali__36r1PF7IMW2MOB8FDBO73BBUMXX}\label{ratnakīrtinibandhāvali__36r1PF7IMW2MOB8FDBO73BBUMXX}\edtext{}{\lemma{kārāttri}\xxref{ratnakīrtinibandhāvali__36r1PF7IMW2MOB8FDBO73BBUMXX}{ratnakīrtinibandhāvali__36r1PF7IMW1RSAP4VW0YYBRBY6Z}\Afootnote{ \cite{RNAms} kārāstri \cite{thakur75} }}kārāttri\edlabel{ratnakīrtinibandhāvali__36r1PF7IMW1RSAP4VW0YYBRBY6Z}\label{ratnakīrtinibandhāvali__36r1PF7IMW1RSAP4VW0YYBRBY6Z}vidhabhāvarāśivyavasthāpanārthaṃ ca viṭapādau pratyakṣāpratikṣiptena dṛśyādṛśyasādhāraṇena buddhimanmātreṇa vyāptir avagamyata iti dvitīyaḥ saṅkalpaḥ | tadā viṭapādau buddhimanmātrasya sambhāvyamānatvād na sādhāraṇānaikāntikatāṃ brūmaḥ | kiṃ tarhi vyāptigrahaṇakāle dṛśyādṛśyasādhāraṇasya buddhimanmātrasya sādhyasyādṛśyatayā dṛśyānupalambhena vyatirekāsiddher vyāpter abhāvat sandigdhavyāvṛttikatvam ācakṣmahe | tathā hi | yadā kumbhakāravyāpārāt pūrvaṃ kumbhasya vyatirekaḥ pratyetavyas tadā na sādhyābhāvakṛto ghaṭavyatirekaḥ pratyetuṃ śakyaḥ | yathā hi viṭapādijanmasamaye buddhimanmātrāsyādṛśyatvena niṣeddhum aśakyatvāt sattāsambhāvanā tathā ghaṭādāv api vyatirekaniścayakāle buddhimanmātrasyādṛśyatvāt sattvasambhāvanāyāṃ sādhyābhāvaprayuktasya sādhanābhāvasyāsiddhatvena vyāpter abhāvāt kathaṃ na sandigdhavyatireko hetuḥ |
	\pend
      

	  \pstart \edlabel{ratnakīrtinibandhāvali__36r1PF7IMW0ZOFGWPS76K4TJPBP}\label{ratnakīrtinibandhāvali__36r1PF7IMW0ZOFGWPS76K4TJPBP}\edtext{}{\lemma{yaccoktannacaṃ}\xxref{ratnakīrtinibandhāvali__36r1PF7IMW0ZOFGWPS76K4TJPBP}{ratnakīrtinibandhāvali__36r1PF7IMW06M8WVY8B702VL98X}\Afootnote{yathoktam—na ca \cite{thakur75} }}yaccokta\edlabel{ratnakīrtinibandhāvali__36r1PF7IMW06M8WVY8B702VL98X}\label{ratnakīrtinibandhāvali__36r1PF7IMW06M8WVY8B702VL98X} yathā kāryaṃ ca syān nirupādānāṃ ceti nāśaṅkanīyam, tathā kāryaṃ ca bhaved akartṛkaṃ ceti nāśaṅkanīyam iti, tatrāpi kāryaṃ ca syān nirupādānaṃ ca bhaved iti na vaktavyam iti kenaivaṃ pratārito 'si | yadi hy atra pratyakṣānupalambhābhyāṃ vyāptir gṛhyate tadā katham upādānapūrvakaṃ kāryamātraṃ sidhyati | vyāptigrahaṇaprakārāntaraṃ ca tvayāpi nopanyastam | dṛśyādṛśyasādhāraṇayor upādānakāryamātrayor dṛśyaviṣayābhyāṃ pratyakṣānupalambhābhyāṃ vyāpter \edlabel{ratnakīrtinibandhāvali__36r1PF7IMVZEV49WMONFBJTJ1CG}\label{ratnakīrtinibandhāvali__36r1PF7IMVZEV49WMONFBJTJ1CG}\edtext{}{\lemma{abhyūhitum}\xxref{ratnakīrtinibandhāvali__36r1PF7IMVZEV49WMONFBJTJ1CG}{ratnakīrtinibandhāvali__36r1PF7IMVYM47AWWTF852YD57I}\Afootnote{ \cite{thakur75} \unclear{angra}hītum \cite{RNAms} }}abhyūhitum\edlabel{ratnakīrtinibandhāvali__36r1PF7IMVYM47AWWTF852YD57I}\label{ratnakīrtinibandhāvali__36r1PF7IMVYM47AWWTF852YD57I} aśakyatvāt | svamatavyālopaprasaṅgas tu pramāṇacintāvasare 'prāptāvakāśaḥ | viparyayabādhakapramāṇabalād vātra vyāptisiddhiḥ | tathā hi yathāṅkurādikaṃ kāryaṃ niyatadeśakālasvabhāvatvena vyāptaṃ tathā śālitvādināpi jātibhedena vyāptam upalabdham | tataś cānupādānapūrvakatvād vipakṣātmanaḥ śālitvādijātibhedasya vyāpakasya nivṛttau nivartamānaṃ kāryatvam upādānapūrvakatve viśrāmyat tena vyāptaṃ sidhyati | na cānupādānenāpi kriyamāṇaḥ śālitvādijātibhedo yujyate, upādānaṃ vinā \edlabel{ratnakīrtinibandhāvali__36r1PF7IMVXU9JL77WHT5LDVJ1B}\label{ratnakīrtinibandhāvali__36r1PF7IMVXU9JL77WHT5LDVJ1B}\edtext{}{\lemma{kṛtād anu}\xxref{ratnakīrtinibandhāvali__36r1PF7IMVXU9JL77WHT5LDVJ1B}{ratnakīrtinibandhāvali__36r1PF7IMVX1JV1BFBSIE9TYQY6}\Afootnote{ \cite{RNAms} kṛtānu \cite{thakur75} }}kṛtād anu\edlabel{ratnakīrtinibandhāvali__36r1PF7IMVX1JV1BFBSIE9TYQY6}\label{ratnakīrtinibandhāvali__36r1PF7IMVX1JV1BFBSIE9TYQY6}pādānād eva kevalād ekajātīyakāraṇāt tadatajjātīyakāryotpattau kāryabhedasyāhetukatvaprasaṅgāt | tad uktam: 
	    \pend
	  
	    
	    \stanza[\smallbreak]
	tadatadrūpiṇo bhāvās tadatadrūpahetujāḥ ||\&[\smallbreak]


	
	    \pstart
	   iti | \edtext{}{\lemma{|}\Bfootnote{(PV III 251ab)}}
	\pend
      

	  \pstart anyathānupādānād eva kṣityāder aṅkurādikam utpadyetety aṅkurārthino bījaṃ nānusareyuḥ | tasmād viparyayabādhakapramāṇabalād eva kāryatvasya hetumātrapūrvakatvenevopādānapūrvakatvenāpi vyāptisiddhir iti nyāyaḥ | na caivaṃ kāryamātrakartṛtvamātrayor api vyāptiprasādhakaṃ viparyaye bādhakaṃ pramāṇam asti, pūrvoktasya vyāpakānupalambhatrayasyātiprasaṅgacatuṣṭayasya ca prāg eva pratyākhyātatvāt | tasmāt kāryaṃ ca syāt na ca dhīmatkartṛpūrvakam iti śaṅkāṃ kurvāṇaḥ prativādī vinā caraṇamardanādinā niṣeddhum aśakyaḥ ||
	\pend
      

	  \pstart nanu yadi dṛśyāgnidhūmasāmānyayor iva \edlabel{ratnakīrtinibandhāvali__36r1PF7IMVW9VQKNUGR24RY6CLM}\label{ratnakīrtinibandhāvali__36r1PF7IMVW9VQKNUGR24RY6CLM}\edtext{}{\lemma{dṛśyātmanor}\xxref{ratnakīrtinibandhāvali__36r1PF7IMVW9VQKNUGR24RY6CLM}{ratnakīrtinibandhāvali__36r1PF7IMVVGFNWYPOLMDXQ8LLU}\Afootnote{}}dṛśyā\edlabel{ratnakīrtinibandhāvali__36r1PF7IMVVGFNWYPOLMDXQ8LLU}\label{ratnakīrtinibandhāvali__36r1PF7IMVVGFNWYPOLMDXQ8LLU} eva kāryakāraṇasāmānyayoḥ pratyakṣānupalambhato vyāptis tadā paracittānumānakṣatiḥ | svaparasantānasādhāra\edlabel{ratnakīrtinibandhāvali__36r1PF7IMVUOL26OEJJYFE6B0AJ}\label{ratnakīrtinibandhāvali__36r1PF7IMVUOL26OEJJYFE6B0AJ}\edtext{}{\lemma{ṇenādṛśyena}\xxref{ratnakīrtinibandhāvali__36r1PF7IMVUOL26OEJJYFE6B0AJ}{ratnakīrtinibandhāvali__36r1PF7IMVTV691SZ12AAL1XY0F}\Afootnote{ \cite{RNAms} ṇena dṛśyādṛśyena \cite{thakur75} \textenglish{---\textsc{Note} Thakur says ```dṛśyā' later addition." No trace of it in \cite{RNAms}, so probably his own emendation.}}}ṇenādṛśyena\edlabel{ratnakīrtinibandhāvali__36r1PF7IMVTV691SZ12AAL1XY0F}\label{ratnakīrtinibandhāvali__36r1PF7IMVTV691SZ12AAL1XY0F} cinmātreṇa pratyakṣato dṛśyaviṣayād \edlabel{ratnakīrtinibandhāvali__36r1PF7IMVSW4O6482CUH1BZRR4}\label{ratnakīrtinibandhāvali__36r1PF7IMVSW4O6482CUH1BZRR4}\edtext{}{\lemma{kampasya}\xxref{ratnakīrtinibandhāvali__36r1PF7IMVSW4O6482CUH1BZRR4}{ratnakīrtinibandhāvali__36r1PF7IMVSW4O6482CUH1BZRR4}\Afootnote{ \cite{RNAms,thakur75} }}vyāptigrahaṇāyogād ity api na vācyam | bāhyārthasthitau hi svaparasantānasādhāraṇasya cinmātrasya svarūpeṇādṛśyatve 'pi dṛśyaśarīreṇa sahaikasāmagrīpratibandhād avinirbhāgavartitvam asty eva | tato yathā ghaṭaviṣayaṃ pratyakṣaṃ rūpaikadeśapravṛttam apy avyabhicārāt samudāyopasthāpakam tathā dehagrāhakam eva pratyakṣaṃ dehāvinirbhāgavarti svaparasantānasādhāraṇaṃ cinmātraṃ kampāder vyāpakam adhigacchanti | tad evaṃ dṛśyātmano dṛśyāvinirbhāgavartino vā padārthasya vyāvahārikapaṭupratyakṣataḥ siddhir vyāptigrahaś ca, na tu tathātvavinākṛtādṛśyasādhāraṇacinmātrasyeti santānāntarānumānam ucitam | tasmād yadi pratyakṣānupalabhābhyāṃ vyāptigrahas tadā dṛśyenaiva dṛśyasyeti nyāyaḥ | \edlabel{thakur75-45.24}\label{thakur75-45.24} tad ayaṃ saṃkṣepārthaḥ:
	\pend
      
	    
	    \stanza[\smallbreak]
	kāryatvasya vipakṣavṛttihataye sambhāvyate 'tīndriyaḥ kartā ced vyatirekasiddhividhurā vyāptiḥ kathaṃ sidhyati |&dṛśyo 'tha vyatirekasiddhimanasā kartā samāśrīyate tattyāge 'pi tadā tṛṇādikam iti vyaktaṃ vipakṣe kṣaṇam ||\&[\smallbreak]


	\footnote{\begin{english}(JNA 285,7-10)\end{english}}

	  \pstart ato na pratyakṣānupalmbhābhyām api vyāptisiddhiḥ ||
	\pend
      
	  
	% new div opening: depth here is 3
	

	  \pstart nanu bhūyodarśanādarśanābhyāṃ pratibandhaḥ pratīyata iti tṛtīya evāsamākaṃ pakṣaḥ | kevalaṃ sa pratibandho na tadutpattilakṣaṇo grahītavyaḥ | kin tu svābhāvikaḥ | sa eva darśanādarśanābhyāṃ pratīyate | tathā caitam evārthaṃ Vācaspatiḥ prāha:\edtext{\textsuperscript{*}}{\lemma{*}\Bfootnote{The following collects material from \href{NVTṬ\#nvtṭ-nsū_1-1.5}{NVTṬ\#nvtṭ-nsū\textunderscore 1-1.5}, pp. 135--136.}} na sapakṣāsapakṣayor \edlabel{ratnakīrtinibandhāvali__36r1PF7IMVS4B46SOBOG8KXEGK8}\label{ratnakīrtinibandhāvali__36r1PF7IMVS4B46SOBOG8KXEGK8}\edtext{}{\lemma{darśanādarśanābhyāṃ}\xxref{ratnakīrtinibandhāvali__36r1PF7IMVS4B46SOBOG8KXEGK8}{ratnakīrtinibandhāvali__36r1PF7IMVRB34ML6ZYATM1PHF2}\Afootnote{ \cite{RNAms} darśanābhyāṃ \cite{thakur75} }}darśanādarśanābhyāṃ\edlabel{ratnakīrtinibandhāvali__36r1PF7IMVRB34ML6ZYATM1PHF2}\label{ratnakīrtinibandhāvali__36r1PF7IMVRB34ML6ZYATM1PHF2} kāryatvasya gamakatvam api tu svābhāvikapratibandhabalād iti brūmaḥ | sa eva tu sapakṣāsapakṣayor darśanādarśanābhyāṃ vakṣyamāṇena krameṇa pratīyata iti tadupakṣepo 'pi yuktaḥ | \edlabel{sarit__ratnakīrtinibandhāvali__122689}\label{sarit__ratnakīrtinibandhāvali__122689}\edtext{}{\lemma{tena … api ||}\xxref{sarit__ratnakīrtinibandhāvali__122689}{sarit__ratnakīrtinibandhāvali__124304}\Afootnote{\label{sarit__ratnakīrtinibandhāvali__399142}---\textsc{Note} Mentioned in the context of NVTṬ and VyN in \cite[171--172, n.~232]{krasser02_zaGkar_Izvar_studie}.  {\rmlatinfont [App type: parallel]}}}tena yasyāsau \edlabel{ratnakīrtinibandhāvali__36r1PF7IMVQJ1KWGQSXO9UY1JGL}\label{ratnakīrtinibandhāvali__36r1PF7IMVQJ1KWGQSXO9UY1JGL}\edtext{}{\lemma{svābhāvikaḥ}\xxref{ratnakīrtinibandhāvali__36r1PF7IMVQJ1KWGQSXO9UY1JGL}{ratnakīrtinibandhāvali__36r1PF7IMVPPPG3TFIIZHIM3XZ4}\Afootnote{ \cite{RNAms} svābhāvika \cite{thakur75} }}svābhāvikaḥ\edlabel{ratnakīrtinibandhāvali__36r1PF7IMVPPPG3TFIIZHIM3XZ4}\label{ratnakīrtinibandhāvali__36r1PF7IMVPPPG3TFIIZHIM3XZ4}pratibandho niyataḥ siddhaḥ sa eva gamako gamyaś cetaraḥ sambandhīti yujyate | tathā hi dhūmādīnāṃ vahnyādibhiḥ saha sambandhaḥ svābhāviko na tu vahnyādīnāṃ dhūmādibhiḥ | te hi vinā dhūmādibhir upalambhyante | yadā tv \edlabel{ratnakīrtinibandhāvali__36r1PF7IMVOXKCE0O54J1VV68NX}\label{ratnakīrtinibandhāvali__36r1PF7IMVOXKCE0O54J1VV68NX}\edtext{}{\lemma{ārdrendhanādisambandham}\xxref{ratnakīrtinibandhāvali__36r1PF7IMVOXKCE0O54J1VV68NX}{ratnakīrtinibandhāvali__36r1PF7IMVO3DAEZ0FQPWRJZXZ2}\Afootnote{ārdrendhanādisambandham \cite{} ; ārdrendhanasambandham \cite{thakur75,RNAms} }}ārdrendhanādisambandham\edlabel{ratnakīrtinibandhāvali__36r1PF7IMVO3DAEZ0FQPWRJZXZ2}\label{ratnakīrtinibandhāvali__36r1PF7IMVO3DAEZ0FQPWRJZXZ2} anubhavanti tadā dhūmādibhiḥ sambadhyante | tasmād vahnyādīnām ārdredhanādyupādhikṛtaḥ sambandho na tu svābhāvikas tato na niyataḥ | svābhāvikas tu dhūmādīnāṃ vahnyādibhiḥ sambandhaḥ, tadupādher anupalabhyamānatvāt | kvacid vyabhicārasyādarśanāt | anupalabhyamānasyāpi kalpanānupapatteḥ | na cānupalabhyamāno darśanānarhatayā sādhakabādhakapramāṇābhāvena sandihyamāna upādhiḥ sambandhasya svābhāvikatvaṃ pratibadhnātīti yuktam | yathoktaṃ prāk seyaṃ saṃśayapiśācītyādi | tasmād upādhiṃ prayatnenānviṣyanto 'nupalabhyamānā nāstīty avagamya svābhāvikatvaṃ niścinumaḥ ||
	\pend
      

	  \pstart syād etat | anyasyānyena sahakāraṇena cet svābhāvikaḥ sambandho bhavet, sarvaṃ sarveṇa sambadhyeta | tathā ca sarvaṃ sarvasmād gamyeta | athānyac ced anyasya kāryaṃ kasmāt sarvaṃ sarvasmān na bhavati, anyatvāviśeṣāt | tataś ca sa evātiprasaṅgaḥ | yady ucyeta svabhāvā na paryanuyojyāḥ | tasmād anyatvāviśeṣe 'pi kiñcid eva kāraṇaṃ kāryaṃ ca kiñcid iti | nanv eṣa svabhāvānanuyogo 'kāryakāraṇabhūtānām api svabhāvapratibandhe tulya eva | tasmād yat kiñcid etad api ||\edlabel{sarit__ratnakīrtinibandhāvali__124304}\label{sarit__ratnakīrtinibandhāvali__124304}
	\pend
      

	  \pstart kim asya sambandhasya vyāptigrāhakaṃ pramāṇam iti cet | ucyate
	\pend
      
	    
	    \stanza[\smallbreak]
	bhūyodarśanagamyā hi vyāptiḥ sāmānyadharmayoḥ | \edtext{}{\lemma{|}\Bfootnote{(ŚV, anumāna, 12)}}\&[\smallbreak]


	

	  \pstart iti prasiddham eva | asyāyam arthaḥ kāśikākāreṇa vyākhyātaḥ—prācīnānekadarśanajanitasaṃskārasahāye carame \edlabel{ratnakīrtinibandhāvali__36r1PF7IMVNAFWNL8S22XY2AOTX}\label{ratnakīrtinibandhāvali__36r1PF7IMVNAFWNL8S22XY2AOTX}\edtext{}{\lemma{cetasi}\xxref{ratnakīrtinibandhāvali__36r1PF7IMVNAFWNL8S22XY2AOTX}{ratnakīrtinibandhāvali__36r1PF7IMVMGY4ZI43PI8JBSRTA}\Afootnote{ \cite{RNAms} darśane cetasi \cite{thakur75,kāśikā} }}cetasi\edlabel{ratnakīrtinibandhāvali__36r1PF7IMVMGY4ZI43PI8JBSRTA}\label{ratnakīrtinibandhāvali__36r1PF7IMVMGY4ZI43PI8JBSRTA} cakāsti dhūmasyāgniniyatasvabhāvatvam, ratnatattvam iva parīkṣakasya, śabdatattvam iva vyākaraṇasmṛtisaṃskṛtasya, brāhmaṇatvam iva mātāpitṛsambandhasmaraṇasacivasyetyādi | na hy etat sarvam āpātato na pratibhātam iti purastād api pratibhāsamānam anyathā bhavatīti ||
	\pend
      

	  \pstart \persName{trilocanena} punar ayam arthaḥ kathitah – \edlabel{sarit__ratnakīrtinibandhāvali__125156}\label{sarit__ratnakīrtinibandhāvali__125156}\edtext{}{\lemma{bhūyodarśanena … iti ||}\xxref{sarit__ratnakīrtinibandhāvali__125156}{sarit__ratnakīrtinibandhāvali__125583}\Afootnote{\label{sarit__ratnakīrtinibandhāvali__399616}---\textsc{Note} Cf. \href{thakur75-106.16}{thakur75-106.16}.---\textsc{Note} Sanskrit, translation, and discussion of parallels in \cite[71--72]{krasser02_zaGkar_Izvar_studie}.  {\rmlatinfont [App type: parallel]}}}bhūyodarśanena bhūyodarśanasahāyena manasā tajjātīyānāṃ sambandho gṛhīto bhavati | ato dhūmo 'gniṃ na vyabhicarati | tadvyabhicāre 'py upādhirahitaṃ sambandham atikrāmet | hetor vipakṣaśaṅkānivartakaṃ pramāṇam upalabdhilakṣaṇaprāptopādhivirahaniścayahetur anupalambhākhyaṃ pratyakṣam eva | tataḥ siddhaḥ svābhāvikaḥ sambandhaḥ | tathehāpīti svamataṃ vyavasthāpitam iti ||\edlabel{sarit__ratnakīrtinibandhāvali__125583}\label{sarit__ratnakīrtinibandhāvali__125583}
	\pend
      

	  \pstart Vācaspatināpīdam uktam – abhijātamaṇibhedatattvavad bhūyodarśanajanitasaṃskārasahāyam indriyam eva dhūmādīnāṃ vahnyādibhiḥ svābhāvikasambandhagrāhīti yuktam iti ||
	\pend
      

	  \pstart atrocyate | \edlabel{ratnakīrtinibandhāvali__36r1PF7IMVLODEE7RZ9HXOTB6GW}\label{ratnakīrtinibandhāvali__36r1PF7IMVLODEE7RZ9HXOTB6GW}\edtext{}{\lemma{'bhede}\xxref{ratnakīrtinibandhāvali__36r1PF7IMVLODEE7RZ9HXOTB6GW}{ratnakīrtinibandhāvali__36r1PF7IMVKUPK1QFXFUAIMYDBD}\Afootnote{bhede \cite{thakur75} }}'bhede\edlabel{ratnakīrtinibandhāvali__36r1PF7IMVKUPK1QFXFUAIMYDBD}\label{ratnakīrtinibandhāvali__36r1PF7IMVKUPK1QFXFUAIMYDBD} sati tadutpatter anyaḥ svābhāvikaḥ sambandhaḥ śabdāsphālanamātram evedam | na khalu nirūpyamāṇaḥ prāpyate | tathā hi svābhāvikas tu dhūmādīnāṃ vahnyādibhiḥ sambandhaḥ tadupādher anupalabhyamānatvāt | kvacid vyabhicārasyādarśanād iti tvayaivāsya lakṣaṇam uktam | etac cāsiddham | yataḥ, upādhiśabdena svato 'rthāntaram evāpekṣaṇīyam abhidhātavyam | na cārthāntaraṃ dṛśyatāniyatam, adṛśyasyāpi deśakālasvabhāvaviprakṛṣṭasya sambhavāt | tataś ca dhūmasyāpi hutāśena saha sambandhe syād upādhiḥ, na copalakṣyata iti katham adarśanān nāsty eva yataḥ svābhāvikasambandhasiddhiḥ ||
	\pend
      

	  \pstart atha \edlabel{ratnakīrtinibandhāvali__36r1PF7IMVK24238RM491U9AF0B}\label{ratnakīrtinibandhāvali__36r1PF7IMVK24238RM491U9AF0B}\edtext{}{\lemma{yady arthā}\xxref{ratnakīrtinibandhāvali__36r1PF7IMVK24238RM491U9AF0B}{ratnakīrtinibandhāvali__36r1PF7IMVJ82PKM5MV8I8FCZI0}\Afootnote{ \cite{RNAms} yadyathā \cite{} }}yady arthā\edlabel{ratnakīrtinibandhāvali__36r1PF7IMVJ82PKM5MV8I8FCZI0}\label{ratnakīrtinibandhāvali__36r1PF7IMVJ82PKM5MV8I8FCZI0}nataram apekṣaṇīyaṃ syāt | \edlabel{sarit__ratnakīrtinibandhāvali__126671}\label{sarit__ratnakīrtinibandhāvali__126671}\edtext{}{\lemma{kathaṃ … veti}\xxref{sarit__ratnakīrtinibandhāvali__126671}{sarit__ratnakīrtinibandhāvali__126898}\Afootnote{\label{sarit__ratnakīrtinibandhāvali__400122}---\textsc{Note} Translation, and parallels in \cite[176, and n.~241]{krasser02_zaGkar_Izvar_studie}.  {\rmlatinfont [App type: parallel]}}}kathaṃ dhūma ity eva pāvakasattāniyama iti cet | nanv idam eva cintyate | tadutpatter asvīkāre sahasraśo darśane 'pi kiṃ sarvatra dhūme saty avaśyam agniḥ sambhavī na veti\edlabel{sarit__ratnakīrtinibandhāvali__126898}\label{sarit__ratnakīrtinibandhāvali__126898} kadācid arthāntaram upādhim apekṣya dhūmo 'pi syān nāgnir iti kim atra niṣṭaṅkakāraṇam | tadupādher anupalabhyamānatvāt | kvacid vyabhicārasyādarśanād iti tu yad uktaṃ tat pratyuktam eva | adṛśyasyāpy upādheḥ sambhāvyamānatvāt | vyabhicārasya ca pratyayāntaravaikalyenāhatyādarśane 'pi niṣeddham aśakyatvāt | ata eva tayor bādhakābhāve 'pi sādhakabādhakapramāṇābhāvāt śaṅkā sambhavaty eva | na punas tavāmunā viklavavikrośitamātreṇa vyāvartate | na caitavatā prāmāṇikalokayātrātikramaḥ | prāmāṇikair eva sādhakabādhakapramāṇābhāve nyāyaprāptasya saṃśayasya vihitatvāt | na ca sarvatrāpravṛttiprasaṅgaḥ, pamāṇād arthasaṃśayāc ca pravṛtter upapatteḥ | na cānarthasandehaḥ sarvatra kartuṃ śakyate, kvacid arthonmukhatāyā eva darśanāt ||
	\pend
      

	  \pstart yac cānyatvāviśeṣe 'pi kiñcid eva kāraṇaṃ kāryaṃ ca kiñcid iti svabhāvo yathā na paryanuyojyas tathaiṣa svabhāvānanuyogo 'kāryakāraṇabhūtā\leavevmode\ledsidenote{\textenglish{27a/RNAms}}nām api svabhāva pratibandhe tulya eveti grāmyajanadhandhīkaraṇaṃ \edlabel{ratnakīrtinibandhāvali__36r1PF7IMVIAT61KSGR7T0A55I7}\label{ratnakīrtinibandhāvali__36r1PF7IMVIAT61KSGR7T0A55I7}\edtext{}{\lemma{vandī}\xxref{ratnakīrtinibandhāvali__36r1PF7IMVIAT61KSGR7T0A55I7}{ratnakīrtinibandhāvali__36r1PF7IMVHGMKIKSCKKF1EMQRT}\Afootnote{ \cite{RNAms} prativandī \cite{thakur75} }}vandī\edlabel{ratnakīrtinibandhāvali__36r1PF7IMVHGMKIKSCKKF1EMQRT}\label{ratnakīrtinibandhāvali__36r1PF7IMVHGMKIKSCKKF1EMQRT}karaṇam atilāghavam āviskaroti vācaspateḥ | tathā hi vastutvāviśeṣe 'py agnir dahati nākāśam ity atra yathā nātiprasaṅgaḥ saṅgataḥ pramāṇasiddhatvād asyārthasya, tathā bhedāviśeṣe 'pi kiñcid eva kasyacit kāraṇaṃ kāryaṃ ca kiñcid ity atrāpi nātiprasaṅgāvatāraḥ | bhede sati tadanvayavyatirekānuvidhānalakṣaṇasya kāryakāraṇabhāvasya pramāṇasiddhatvād eva | na caivaṃ svābhāvikasambandhaśabdavācyo 'rthaḥ pramāṇasiddhaḥ kaścid asti, tallakṣaṇasyāsiddhatvād uktatvāt | na ca pratijñāsiddhe vastuny atiprasaṅgo nābhaidhātavyaḥ, sarveṣāṃ sarvatra tadrūpābhyupagamamātreṇa vijetṛtvaprasaṅgāt | yad āhālaṅkārakāraḥ:
	\pend
      
	    
	    \stanza[\smallbreak]
	yat kiñcid ātmābhimataṃ vidhāya niruttaras tatra kṛtaḥ pareṇa |&vastusvabhāvair iti vācyam itthaṃ tathottaraṃ syād vijayī samastaḥ ||\&[\smallbreak]


	

	  \pstart iti ||
	\pend
      

	  \pstart kiṃ ca svābhāvikasambandha iti ko 'rthaḥ | kiṃ svato bhūtaḥ svahetuto bhūto 'hetuko veti trayaḥ pakṣāh | na tāvad ādyaḥ pakṣaḥ, svātmani kāritravirodhāt | dvitīyapakṣe tu tadutpattir eva sambandho mukhāntareṇa svīkṛta iti na kaścid vivādaḥ |ahetukatve tu deśakālasvabhāvaniyamābhāvaprasaṅgād ity asaṅgataḥ svābhāvikaḥ sambandhaḥ ||
	\pend
      

	  \pstart etena yad uktam: na sapakṣāsapakṣayor darśanādarśanābhyāṃ kāryatvasya gamakatvam api tu svābhāvikasambandhabalād iti brūmaḥ, sa eva tu sapakṣāsapakṣayor darśanādarśanābhyāṃ vakṣyamāṇena krameṇa pratīyata iti, tadiṣṭakāmatāmātrāviṣkaraṇam iti mantavyam | svābhāvikasambandhasya hy upādhinirapekṣaniyatatvaṃ lakṣaṇam uktam | tasya coktanyāyenāsiddhau bhūyodarśanajanitasaṃskārasahāye carame cetasi manasi vā tathābhūtaṃ niyatatvaṃ parisphuratīti \edlabel{ratnakīrtinibandhāvali__36r1PF7IMVGNCTDB5V2TFKEVGNU}\label{ratnakīrtinibandhāvali__36r1PF7IMVGNCTDB5V2TFKEVGNU}\edtext{}{\lemma{sahṛdayena}\xxref{ratnakīrtinibandhāvali__36r1PF7IMVGNCTDB5V2TFKEVGNU}{ratnakīrtinibandhāvali__36r1PF7IMVFSDSSMA1F0VH8OHD1}\Afootnote{sa\deletion{ha}\add{hṛ}dayena \cite{RNAms} ; sadayena \cite{thakur75}   {\rmlatinfont [App type: var]}}}sahṛdayena\edlabel{ratnakīrtinibandhāvali__36r1PF7IMVFSDSSMA1F0VH8OHD1}\label{ratnakīrtinibandhāvali__36r1PF7IMVFSDSSMA1F0VH8OHD1} vaktum aśakyatvāt |
	\pend
      

	  \pstart yac ca śabdatattvam iva brāhmaṇatvam iveti dṛṣṭāntīkṛtaṃ tad dvayam apy asmān pratyasiddham iti dṛṣṭāntayitum anucitam | abhijātamaṇibhedatattvaṃ tu parisphuratīti yuktam | tasya hy upadeśaparamparāto māṇikyavattenāpi kaṣṭenendradhanurākārajyotirādikaṃ lakṣaṇaṃ niścitam | na caivaṃ svābhāvikasambandhalakṣaṇaṃ tvayā svakapolaracitam api pramāṇena niścitam | yenāsyāpi tādṛśī vyavasthā syād iti yat kiñcid etat ||
	\pend
      

	  \pstart kiṃ ca bhavatu tāvad ayam anavadhāritarūpaḥ svābhāvikaḥ sambandhaḥ, tathāpi darśanādarśanābhyām asya grahaṇam atidurlabham | tathā hi yadi prācīnānekadarśanajanitasaṃskārasahāyena caramacetasā dhūmasyāgniniyatatvaṃ grāhyaṃ tadā sapakṣāsapakṣayoḥ koṭiśaḥ pravṛttadarśanādarśanajanitasaṃskārasahāyena caramacetasā pārthivatvasyāpi lohalekhyatvaniyatatvaṃ gṛhyata iti pārthivatvād api lohalekhyatvasiddhir astu | atha pārthivatvasya lohalekhyatvaniyatatvam eva nāsti vajre vyabhicāradarśanāt | tat kathaṃ pratyakṣeṇa niyatatvagrahaḥ | tarhi dhūmasya vahniniyatatvam eva nāsti, vyabhicārābhāvasya darśayitum aśakyatvāt | tat kathaṃ caramacittena niyamagraha ity apy tulyam |
	\pend
      

	  \pstart vyabhicārādarśanād avyabhicāra iti cet | nanu vyabhicārādarśanād avyabhicāra iti ko 'rthaḥ | kiṃ vyabhicārādarśanād avyabhicāraḥ, vyabhicārābhāvād vā | prathame pakṣe vyabhicāro bhavatu mā vā vyabhicārādarśanād evāvyabhicāra iti niṣṇātaṃ pāṇḍityam | atha dvitīyaḥ pakṣaḥ | tadā vyabhicārābhāvaḥ kuto jñātaḥ | adarśanād iti cet | tat kim adarśanamātraṃ dṛśyādarśanaṃ vā | prathamam aśaktam | na hy adarśane 'pi vyabhicāro nāstīty abhidhātuṃ śakyate, cirakālanaṣṭabrāhmaṇīvyabhicāravat | āhatyādarśane 'py aticirakālavyavadhānena vyabhicāradarśanāt | dvitīyaṃ cāsambhavi, kvacit kadācit kenacid vyabhicāradarśanasāmagryāṃ satyāṃ vyabhicāradarśanāt | darśana\edlabel{ratnakīrtinibandhāvali__36r1PF7IMVEXVG9R4VA06AT0QJO}\label{ratnakīrtinibandhāvali__36r1PF7IMVEXVG9R4VA06AT0QJO}\edtext{}{\lemma{sāmagrībhāve}\xxref{ratnakīrtinibandhāvali__36r1PF7IMVEXVG9R4VA06AT0QJO}{ratnakīrtinibandhāvali__36r1PF7IMVE2Q2LD7GAVUX8TFLK}\Afootnote{°sāmagrībhāve \cite{RNAms} ; sāmagryabhāve \cite{thakur75}   {\rmlatinfont [App type: var]}}}sāmagrībhāve\edlabel{ratnakīrtinibandhāvali__36r1PF7IMVE2Q2LD7GAVUX8TFLK}\label{ratnakīrtinibandhāvali__36r1PF7IMVE2Q2LD7GAVUX8TFLK} tu pratyayāntaravaikalyāt deśakālāntaravartitvād vā vyabhicārasya \edlabel{ratnakīrtinibandhāvali__36r1PF7IMVD8O0SCK5F8GLLVI9X}\label{ratnakīrtinibandhāvali__36r1PF7IMVD8O0SCK5F8GLLVI9X}\edtext{}{\lemma{sarvaṃ pratyupalabdhi}\xxref{ratnakīrtinibandhāvali__36r1PF7IMVD8O0SCK5F8GLLVI9X}{ratnakīrtinibandhāvali__36r1PF7IMVCD7CDOE25WW75M15C}\Afootnote{sa\unclear{rvaṃ pratyu}palabdhi° \cite{RNAms} ; sarvaṃ pratyupalabdhi \cite{thakur75}   {\rmlatinfont [App type: var]}}}sa\edlabel{ratnakīrtinibandhāvali__36r1PF7IMVCD7CDOE25WW75M15C}\label{ratnakīrtinibandhāvali__36r1PF7IMVCD7CDOE25WW75M15C}lakṣaṇaprāptatvābhāvāt | tasmāt saty api vyabhicāre tadupalambhasāmagryabhāvād vyabhicārānupalambhaḥ | prakārāntareṇa vā tadutpattilakṣaṇenāvyabhicāre vyabhicārānupalambha ity ubhayathāpi vyabhicāropalambhanivṛttir astu | tvayā tu yad avyabhicārapratipattinibandhanaṃ darśanādarśanam upavarṇitaṃ tatpārthivatvādau vyabhicārād dhūme 'pi nāvyabhicāranibandhanam iti dhūmo 'pi tvanmate nāśvāsabhājam iti prasaktam |
	\pend
      

	  \pstart asmanmate tu pratyakṣānupalambhābhyām ekatra kāryakāraṇabhāvasiddhau na vyabhicāraśaṅkāsambhavaḥ | tadabhāve tu: hetumattāṃ vilaṅghayed \edtext{}{\lemma{vilaṅghayed}\Bfootnote{(PV I 34d)}} iti nyāyāt na saṃśayapiśācāvasaraḥ | tad evaṃ bhūyodarśanādarśanābhyām api na vyāptisiddhiḥ |
	\pend
      
	  
	% new div opening: depth here is 3
	

	  \pstart tarhi sakṛt sapakṣāsapakṣayor darśanādarśanābhyām vyāpter niścaya iti caturtha eva pakṣo 'stu | tathā hi kāryatvasya buddhimanmātrapūrvakatvenānvayo ghaṭādau dṛṣtaḥ, ākāśādau buddhimatkāraṇanivṛttau kāryatvasya vyatirekaḥ | tataś ca sakṛdanvayavyatirekasiddhau vyāpteḥ siddhatvāt kuto 'naikāntikatā |
	\pend
      

	  \pstart atrābhidhīyate | yadi buddhimatkāraṇakāryatvayor ekatra pratibandhaḥ pramāṇapratītaḥ syāt tadākāśādau buddhimannnivṛttau kāryatvasya nivṛttir iti yuktam | sa ca pratibandhaḥ tādātmyaṃ tadutpattiḥ svābhāviko 'nyo vā na sidhyati sādhakapramāṇābhāvād ity anantaram evāveditam | tataś cākāśādau buddhimannivṛttir api syāt | na ca kāryatvasya nivṛttir iti sandigdhavipakṣavyāvṛttikatvād anaikāntikaṃ kāryatvam |
	\pend
      

	  \pstart nanv ākāśasyāsamanmate nityatvaṃ tvanmate cāsattvam | tat katham ataḥ kāryatvavyatirekaḥ sandigdha iti cet | ucyate | na hy ākāśe \edlabel{ratnakīrtinibandhāvali__36r1PF7IMVBIDAQOHI7KQVIQHUI}\label{ratnakīrtinibandhāvali__36r1PF7IMVBIDAQOHI7KQVIQHUI}\edtext{}{\lemma{kāryatvavyā}\xxref{ratnakīrtinibandhāvali__36r1PF7IMVBIDAQOHI7KQVIQHUI}{ratnakīrtinibandhāvali__36r1PF7IMVAMYN340RKAH5IQ181}\Afootnote{kāryatvavyā° \cite{RNAms} ; kāryavyā° \cite{thakur75}   {\rmlatinfont [App type: var]}}}kāryatvavyā\edlabel{ratnakīrtinibandhāvali__36r1PF7IMVAMYN340RKAH5IQ181}\label{ratnakīrtinibandhāvali__36r1PF7IMVAMYN340RKAH5IQ181}vṛttimātraṃ vyatirekaḥ | kin tu sādhyābhāvaprayuktaḥ sādhanābhāvo vyatirekaḥ | sa cākāśe grahītum aśakyaḥ | yathā tatra buddhimatkāraṇanivṛttis tathā \edlabel{ratnakīrtinibandhāvali__36r1PF7IMV9SO7H54L924LP0P1J}\label{ratnakīrtinibandhāvali__36r1PF7IMV9SO7H54L924LP0P1J}\edtext{}{\lemma{'cetanasyāpi kāraṇasya nivṛttiḥ |}\xxref{ratnakīrtinibandhāvali__36r1PF7IMV9SO7H54L924LP0P1J}{ratnakīrtinibandhāvali__36r1PF7IMV8WQAP4PWOSGBTB038}\Afootnote{kāraṇa'cetanasyā\deletion{pi}kāraṇanivṛttiḥ | \cite{RNAms} ; kāraṇamātrasyāpi nivṛttiḥ | \cite{thakur75} ; acetanasyāpi kāraṇasya nivṛttir \cite{TBh-GOS}   {\rmlatinfont [App type: em]}}}'cetanasyāpi kāraṇasya nivṛttiḥ |\edlabel{ratnakīrtinibandhāvali__36r1PF7IMV8WQAP4PWOSGBTB038}\label{ratnakīrtinibandhāvali__36r1PF7IMV8WQAP4PWOSGBTB038} tat kasyābhāvaprayuktaḥ kāryābhāvaḥ pratīyatāṃ yena vyatirekaḥ sidhyati ||
	\pend
      

	  \pstart nanu satyam evaitat | yathākāśe buddhimatkāraṇanivṛttis tathā kāraṇamātrasyāpi tatra nivṛttir na buddhimatkāraṇavyatirekānuvidhāyitvaṃ kāryatvasya niścetuṃ śakyate | tathāpi ghatādau kāryatvasya buddhimatānvayadarśanākāśe 'pi buddhimadabhāvaprayuktaḥ kāryatvābhāvaḥ pratīyate | tat kathaṃ vyatirekāsiddhir iti cet | hanta ghaṭādāv api na kāryatvasya sattāmātram anvayaḥ | kiṃ tu sādhyasadbhāvaprayuktaḥ sādhanasadbhāvaḥ | sa ca ghaṭe grahītum aśakyaḥ | yathā hi tatra buddhimadbhāvas tathā kaṭakuḍyādibhāvo 'pi | tat ka evaṃ jānātu kiṃ buddhimadbhāve kāryatvasya bhāvo yad vā kaṭakuḍyādibhāve bhāva iti | tasmād atra viśiṣṭānvayavyatirekagrahaṇapravaṇaviśiṣṭapratyakṣānupalambhāv anusarta\edlabel{ratnakīrtinibandhāvali__36r1PF7IMV81NXEGJDO5D1CVNXF}\label{ratnakīrtinibandhāvali__36r1PF7IMV81NXEGJDO5D1CVNXF}\edtext{}{\lemma{vyau yad dṛśyayor eva kāryakāraṇayos}\xxref{ratnakīrtinibandhāvali__36r1PF7IMV81NXEGJDO5D1CVNXF}{ratnakīrtinibandhāvali__36r1PF7IMV75IDMDPIKQ4BXDK0Y}\Afootnote{vyau | ya\unclear{}\add{\unclear{dṛśyayor eva kāryakāraṇayo}}s \cite{RNAms} ; vyau yad dṛśyayor eva kāryakāraṇayos \cite{thakur75}   {\rmlatinfont [App type: var]}}}vyau yad dṛśyayor eva kāryakāraṇayos\edlabel{ratnakīrtinibandhāvali__36r1PF7IMV75IDMDPIKQ4BXDK0Y}\label{ratnakīrtinibandhāvali__36r1PF7IMV75IDMDPIKQ4BXDK0Y} tadutpattisiddhāv anvayavyatirekau sidhyataḥ ||
	\pend
      

	  \pstart na ca pratibandhasādhakaṃ pramāṇaṃ svapne 'py astīti caturtho 'pi pakṣaḥ kṣataḥ |
	\pend
      
	  
	% new div opening: depth here is 3
	

	  \pstart tad evaṃ buddhimatkāryamātrayor vyāpter asiddhāv adhikaraṇasiddhānta\edtext{}{\lemma{adhikaraṇasiddhānta}\Bfootnote{A separate hand adds \begin{sanskrit}yasminna[rthe] sidhyanti tadanuyāyīnya[rthā]ntarā[ṇi] sidhyanti so 'dhikaraṇasiddhāntaḥ |\end{sanskrit} Cf. \href{NVTṬ\#nvtṭ-ad-nsū_1-1.30}{NVTṬ\#nvtṭ-ad-nsū\textunderscore 1-1.30}.}}\edlabel{ratnakīrtinibandhāvali__36r1PF7IMV6ABHHQV02271R0OCL}\label{ratnakīrtinibandhāvali__36r1PF7IMV6ABHHQV02271R0OCL}\edtext{}{\lemma{nyāyād upā}\xxref{ratnakīrtinibandhāvali__36r1PF7IMV6ABHHQV02271R0OCL}{ratnakīrtinibandhāvali__36r1PF7IMV5ECMXMGGYFDTSM3IU}\Afootnote{nyāyādupā \cite{RNAms} ; nyāyādyupā \cite{thakur75}   {\rmlatinfont [App type: var]}}}nyāyād upā\edlabel{ratnakīrtinibandhāvali__36r1PF7IMV5ECMXMGGYFDTSM3IU}\label{ratnakīrtinibandhāvali__36r1PF7IMV5ECMXMGGYFDTSM3IU}dānādyabhijñaḥ sarvajñaḥ puruṣaviśeṣaḥ sidhyatīti pratyāśā durāśaiva ||
	\pend
      
	  
	% new div opening: depth here is 3
	

	  \pstart yac ca kriyāsāmānyasya pakṣadharmatāvaśāc cakṣurlakṣaṇakaraṇaviśeṣasiddhir iti dṛṣṭānto darśitaḥ so 'pi sādhyābhinnaḥ | tatra hi rūpajñānānyathānupapattyā siddhasya kāraṇāntarasyaiva cakṣur indriyam iti nāmakaraṇāt | rūpajñānajanakatvātiriktasya cakṣurlakṣaṇaviśeṣasyāsiddhatvāt | atha rūpajñānajanakatvam eva cakṣuṣṭvam ucyate | bhavatu ko doṣaḥ | etad evāsmābhiḥ kāraṇāntaram ucyate | tathaiva yadi tvayāpi buddhimatsāmānyāśrayamātrasya puruṣaviśeṣa iti nāma kriyate, tadā nāsmākaṃ kādacid vipratipattiḥ | paramārthato buddhimatsāmānyāśraye sarvajñatvādiviśeṣaś cakṣurādiviśeṣavat sidhyatīti tatra vivadāmahe | ubhayor api dṛṣṭāntadārṣṭāntikayor viśeṣasādhanasāmarthyābhāvāt ||
	\pend
      
	  
	% new div opening: depth here is 3
	

	  \pstart tad ayaṃ saṃkṣepārthaḥ:
	\pend
      
	    
	    \stanza[\smallbreak]
	dṛśye tu sādhye vyabhicāra eva dṛśyaṃ na cen na \label{ratnakīrtinibandhāvali__36r1PF7IMV4JB3OIRJAO5COAIMX}\edtext{}{\lemma{vyatireka}\xxref{ratnakīrtinibandhāvali__36r1PF7IMV4JB3OIRJAO5COAIMX}{ratnakīrtinibandhāvali__36r1PF7IMV3NGIPUSUJNFVNEHT1}\Afootnote{vyatireka \cite{RNAms} ; vyabhicāra \cite{thakur75}   {\rmlatinfont [App type: var]}}}vyatireka\label{ratnakīrtinibandhāvali__36r1PF7IMV3NGIPUSUJNFVNEHT1}siddhiḥ |&sādhāraṇatvād atha vā vipakṣasandehataḥ sādhyam ato na sidhyati ||\edtext{\textsuperscript{*}}{\lemma{*}\Bfootnote{Cf. \href{JNA\#jnā-sādhye-ca-dṛśye}{JNA\#jnā-sādhye-ca-dṛśye}.}}\&[\smallbreak]


	

	  \pstart itīśvaro dattājalāñjaliḥ ||
	\pend
      \label{īsd-sādhanasvarūpa}
	  
	% new div opening: depth here is 2
	

	  \pstart idānīṃ sādhanasvarūpaṃ nirūpyate | yad etan merumandaramedinīghaṭapaṭādisādhāraṇaṃ kāryamātraṃ sādhanam upanyastam yāvad asya buddhimadanvayavyatirekānuvidhānam ekatra nāvadhāryate tāvad gamakatvam ayuktam | na ca tat svapne 'pi pratyetuṃ śakyam | tathā hi kumbhakāravyāpāre sati mṛtpiṇḍād ghaṭalakṣaṇaṃ kāryam upalabhyatāṃ nāma | na tu vyāpārāt pūrvaṃ ghaṭavatkāryamātrasya vyatirekaḥ pratyetuṃ śakyaḥ, kumbhakāravyatireke 'pi śoṣabhaṅgādilakṣaṇasya kāryasya mṛtpiṇḍe darśanāt | na ca yad vinābhūtaṃ yad upalabhyate tat tasya kāryam atiprasaṅgāt | tṛṇādivanmṛtpiṇḍasya śoṣabhaṅgādikāryamātram api pakṣīkṛtam iti cet | kriyatāṃ buddhimadvyatireke kāryamātravyatirekas tv ektrāpi pratipādyatāṃ yena vyāptisiddhau \edlabel{ratnakīrtinibandhāvali__36r1PF7IMV2SFJR19V53YIQP3TF}\label{ratnakīrtinibandhāvali__36r1PF7IMV2SFJR19V53YIQP3TF}\edtext{}{\lemma{tṛṇādir}\xxref{ratnakīrtinibandhāvali__36r1PF7IMV2SFJR19V53YIQP3TF}{ratnakīrtinibandhāvali__36r1PF7IMV1W7FSO0IXX5GIKQIY}\Afootnote{tṛṇādi\deletion{|}\gap{}r \cite{RNAms} ; tṛṇādir \cite{thakur75}   {\rmlatinfont [App type: var]}}}tṛṇādir\edlabel{ratnakīrtinibandhāvali__36r1PF7IMV1W7FSO0IXX5GIKQIY}\label{ratnakīrtinibandhāvali__36r1PF7IMV1W7FSO0IXX5GIKQIY} iva śoṣabhaṅgāder api buddhimadanumānaṃ syāt | ākāśādivaidharmyadṛṣṭāntas tu pūrva pratihataḥ, buddhimatpūrvakatvasyeva kāraṇmātrapūrvakatvasyāpi tatra sambhavāt kiṃprayuktaḥ kāryatvābhāva ity aparijñānāt ||
	\pend
      

	  \pstart etena yad uktam - na vyabhicāropalambhāt prātisvikaviśeṣaparityāgena ghaṭādīnām abhūtvābhavanād anyarūpaṃ viśeṣam upalakṣayāmo yanniṣṭhaṃ puruṣapūrvakatvaṃ vyavasthāpayāma iti tad api prativyūḍham | kumbhakārādyabhāve 'pi mṛtpiṇḍādau śoṣabhaṅgādikāryadarśanād abhūtvā \leavevmode\ledsidenote{\textenglish{29a/RNAms}}\label{RNAms-29a} bhāvalakṣaṇasya kāryamātrasya vyatirekāsiddher vyāpter abhāvāt ||
	\pend
      

	  \pstart nanu yadi kāryatvamātrasya na buddhimatā pratyakṣato vyāptigrahaḥ vyatirekābhāvāt, tvayāpi tarhi kathaṃ kṛtakatvasyānityatvena vyāptir avadhārayta iti cet | anapekṣālakṣaṇaviparyayabādhakapramāṇabalād iti brūmaḥ | tac cātadrūpaparāvṛttasyaiva kṛtakatvasya vipakṣād vyatirekaṃ sādhayati | na ca tvayā viparyayabādhakapramāṇam abhidhātuṃ śakyata iti prāg eva pratipāditam | sandigdhavipakṣavyāvṛttikatvād anaikāntikam idaṃ kāryatvamātram ||
	\pend
      

	  \pstart etena yad etat naiyāyikānām ākṣepaparihāraviḍambanam | iha khalu dve kāryatve | kāryamātram | viśiṣṭaṃ ca | tatrādyasya pratibandhāsiddher anaikāntikatvam | viśiṣṭasya bhūdharādiṣv asambhavād asiddhatvam iti | tad asaṅgatam | kāryatvamātrasyaiva pratibandhopapādanāt ||
	\pend
      

	  \pstart yac coktaṃ viśiṣṭaṃ kāryatvam iti | kīdṛśaṃ punas tad iti vaktavyam | atha yat kāryaṃ puruṣānvayavyatirekānuvidhāyitayā tatpūrvakam upalabdham | yaddṛṣṭer akriyādarśino 'pi kṛtabuddhir utpadyate tat kāryaṃ sakalaprāsādādyanugataṃ bhūdharādivyāvṛttaṃ viśiṣṭam ity abhidhīyate | tad asundaram | vikalpānupapatteḥ ||
	\pend
      

	  \pstart \edlabel{sarit__ratnakīrtinibandhāvali__139880}\label{sarit__ratnakīrtinibandhāvali__139880}\edtext{}{\lemma{tathā … iti ||}\xxref{sarit__ratnakīrtinibandhāvali__139880}{sarit__ratnakīrtinibandhāvali__140214}\Afootnote{\label{sarit__ratnakīrtinibandhāvali__400579}---\textsc{Note} Sanskrit and translation in \cite[66--67]{krasser02_zaGkar_Izvar_studie}.  {\rmlatinfont [App type: parallel]}}}tathā cāha śaṅkaraḥ—kṛtabuddhiḥ kiṃ sādhyabuddhiḥ kiṃ vā sādhanabuddhiḥ | sādhyabuddhir api yadi gṛhītavyāptikasya, sā bhavaty eva | athāgṛhītavyāptikasya, kim anyatrāpi sā bhavantī dṛṣṭā | atha sādhanabuddhiḥ | tarhi svopagamavirodhaḥ, sarvasya bhāvasya kṛtakatvopagamād iti ||\edlabel{sarit__ratnakīrtinibandhāvali__140214}\label{sarit__ratnakīrtinibandhāvali__140214}
	\pend
      

	  \pstart vācaspatiḥ punar atrāha - idam atra nipuṇataram nirūpayatu bhavān kiṃ buddhimadanvayavyatirekānuvidhānaṃ viśeṣaḥ | āhosvit tad darśanaṃ yat parvatādiṣu nastīty abhidhīyate | yadi pūrvakaḥ kalpaḥ, sa buddhimaddhetukatvaṃ tanubhuvanādīnām ātiṣṭhamānair abhyupeyata eva | na hi kāraṇaṃ kāryānanuvihitabhāvābhāvam anyo vaktyahrīkāt | atha taddarśanam iti caramaḥ kalpaḥ | na tarhi akriyādarśinaḥ kṛtabuddhisambhavaḥ | ya eva hi ghaṭo 'nena buddhimadanvayavyatirekānuvidhāyī dṛṣṭaḥ, sa eva kāryo na tu vipaṇivartī | tajjātīyasya tadanvayavytirekānuvidhānadarśanād adṛṣṭānvayavyatirekānuvidhānam api tajjātīyaṃ tatheti cet | hantotpattimadghaṭādi buddhimadanvayavyatirekānuvidhāyīti anyad api tanubhuvanādikaṃ tathā bhavan na daṇḍena parāṇudya\leavevmode\ledsidenote{\textenglish{29b/RNAms}}\label{RNAms-29b}\edlabel{ratnakīrtinibandhāvali__36r1PF7IMV108LJMMASAEXOEIZO}\label{ratnakīrtinibandhāvali__36r1PF7IMV108LJMMASAEXOEIZO}\edtext{}{\lemma{te | ghaṭa}\xxref{ratnakīrtinibandhāvali__36r1PF7IMV108LJMMASAEXOEIZO}{ratnakīrtinibandhāvali__36r1PF7IMV03KHD6DEC57SLMJ4Y}\Afootnote{\unclear{ghaṭa} \cite{RNAms} ; te | ghaṭa \cite{s}   {\rmlatinfont [App type: var]}}}te | ghaṭa\edlabel{ratnakīrtinibandhāvali__36r1PF7IMV03KHD6DEC57SLMJ4Y}\label{ratnakīrtinibandhāvali__36r1PF7IMV03KHD6DEC57SLMJ4Y}jātīyam utpattimadbuddhimatpūrvakam iti cet | nanu prāsādādi taddhetukaṃ na bhavet | aghaṭajātīyatvāt | atha yajjātīyam anvayavyatirekānuvidhāyi dṛṣṭam,tajjātīyam evādṛṣṭānvayavyatirekam api taddhetukam | tat kiṃ kāryajātīyaṃ prāsādādi buddhimaddhetukaṃ na dṛṣṭam yenotpattimattanubhuvanādi tathā na syāt | na khalu tajjātīyatve kaścid viśeṣa iti ||
	\pend
      

	  \pstart \edlabel{sarit__ratnakīrtinibandhāvali__141640}\label{sarit__ratnakīrtinibandhāvali__141640}\edtext{}{\lemma{vittokas … etad}\xxref{sarit__ratnakīrtinibandhāvali__141640}{sarit__ratnakīrtinibandhāvali__143038}\Afootnote{\label{sarit__ratnakīrtinibandhāvali__401027}---\textsc{Note} Sanskrit and translation in \cite[133--136]{krasser02_zaGkar_Izvar_studie}.  {\rmlatinfont [App type: parallel]}}}vittokas tv āha—bhavatu vā kaścid anirūpitarūpo viśeṣaḥ | kiṃ punar anena viśeṣaṃ pratipādayatābhipretam | kiṃ kāryatvasāmānyasyāsiddhatvam | atha kāryaviśeṣasya | atha kāryamātrasya buddhimatkartṛvyabhicāraḥ | atha sādhyadṛṣṭāntayor vaidharmyamātram | kiṃ cātaḥ | yadi tāvat kāryasāmānyasyāsiddhatvam | tan nāsti | viśvambharādiṣv api kāraṇavyāpārajanyatvasyobhayasiddhatvāt | atha kāryaviśeṣasya kumbhādivartinaḥ pakṣe 'siddhir abhidhīyate | tadā na kācid atra kṣatir viśeṣasya hetutvenānupādānāt | yadi kāryasāmānyasya kartṛvyabhicāraḥ pratipādayitum iṣṭaḥ | sa na śakyo vipakṣe 'darśanāt | tṛṇādeś ca pakṣīkṛtatvāt | śaṅkāmātrasya \edlabel{ratnakīrtinibandhāvali__36r1PF7IMUZ7KUXZS9SZJ6Y0O7I}\label{ratnakīrtinibandhāvali__36r1PF7IMUZ7KUXZS9SZJ6Y0O7I}\edtext{}{\lemma{sarvathā'ni}\xxref{ratnakīrtinibandhāvali__36r1PF7IMUZ7KUXZS9SZJ6Y0O7I}{ratnakīrtinibandhāvali__36r1PF7IMUYASOPXV8CSA661H3U}\Afootnote{sarvathā'ni \cite{RNAms} ; sarvathāni \cite{RNA}   {\rmlatinfont [App type: var]}}}sarvathā'ni\edlabel{ratnakīrtinibandhāvali__36r1PF7IMUYASOPXV8CSA661H3U}\label{ratnakīrtinibandhāvali__36r1PF7IMUYASOPXV8CSA661H3U}ṣiddhatvāt | sandigdhavyatirekitvaṃ naiyāyikānāṃ niranuyojyānuyogo bauddhānām adoṣodbhāvanaṃ nigrahasthānam iti tu pratipāditam | tathāpi bādhakapramāṇāny abhiditāny eva |
	\pend
      

	  \pstart tasmān na pratibandhāsiddheḥ sarvatra vyabhicārāśaṅkā | atha sādhyadṛṣṭāntayor vaidharmyodbhāvanam | tan na | tasya sarvatra sulabhatvāt | yadi sādhyadṛṣṭāntayor vaidharmyamātrāt sādhyāsiddhiḥ nivṛttedānīm anumānavārtāpi nikuñjamahānasayor api dhūmavattve 'pi kathañcid vaidharmyopapatter iti sakalaṃ yat kiñcid etad\edlabel{sarit__ratnakīrtinibandhāvali__143038}\label{sarit__ratnakīrtinibandhāvali__143038} iti |
	\pend
      

	  \pstart tad ayam atra saṃkṣepārthaḥ | yat tāvat kāryatvamātraṃ tadevoktena krameṇa pratibandhasiddher bhūdharādiṣu dṛṣṭaṃ puruṣam anumāpayatīty asmākam abhimata\edlabel{ratnakīrtinibandhāvali__36r1PF7IMUXEUF5YNC0LPJ2WOMW}\label{ratnakīrtinibandhāvali__36r1PF7IMUXEUF5YNC0LPJ2WOMW}\edtext{}{\lemma{sādhyasiddhir}\xxref{ratnakīrtinibandhāvali__36r1PF7IMUXEUF5YNC0LPJ2WOMW}{ratnakīrtinibandhāvali__36r1PF7IMUWHI4JVCPCDFPY7PX4}\Afootnote{sādhyasiddhir \cite{RNAms} ; sādhyam asiddhir \cite{RNAms}   {\rmlatinfont [App type: var]}}}sādhyasiddhir\edlabel{ratnakīrtinibandhāvali__36r1PF7IMUWHI4JVCPCDFPY7PX4}\label{ratnakīrtinibandhāvali__36r1PF7IMUWHI4JVCPCDFPY7PX4} upapannaiveti | kim asmākam adhikacintayety aṅgīkṛtyāpy uktaṃ viśiṣṭakāryatvam | tad eva tu nāstīti punar vistareṇa pratipāditam iti tad api sarvam anavadheyam eva | tathā hi kāryatvamātrasya tāvad uktena krameṇa vyāpter asiddhatvād anaikāntikatvam anirvāyam | yac ca viśiṣṭakāryatvaṃ vikalpya dūṣitaṃ tasyāsmābhir anabhyupagatatvāt taddūṣaṇāya prabandhaḥ prayāsaikaphalaḥ | na hi kāryatvaṃ dvividham abhimatam | ekaṃ sarvakāryānugatam, aparaṃ parvatādivyāvṛttaṃ ghaṭapaṭaprāsādādyanuyāyīti | kiṃ tu kāryam anekajātīyakam | tatra yadi nāma paṭasya prāsādādibhiḥ saha vastutvasaṃsthānaviśeṣayogitvakāryatvādibhir dharmaiḥ sajātīyatvam asti tathāpi na tān dharmān buddhimatpūrvakānadhigacchati vyāvahārikaṃ pratyakṣaṃ, kāryatvādīnāṃ buddhimadvyatirekānuvidhānābhāvāt | tat kathaṃ prāsādaparvatādiṣu kāryatvādidarśanād buddhimadanumānam astu | kiṃ tu yasyaiva ghaṭajātīyakāryacakrasya vyatirekasiddhis tasya buddhimadvyāptatvaṃ pratyakṣataḥ sidhyatīty uktam | tena deśakālāntare ghaṭajātīyād eva buddhimadanumānam | yadā tu prāsādajātīyakam api buddhimaddhetukam ekatra pṛthag avadhāryate tadā tajjātīyād api buddhimatsiddhiḥ | evaṃ tattajjātīyasarāvodañcanaśakaṭapaṭakeyūraprabhṛtteḥ kāryacakrād buddhimatpūrvakatvena pṛthak pṛthag avadhāritād buddhimadanumānam anavadyam |
	\pend
      

	  \pstart amum evārtham abhisandhāyācāryapādair abhihitam:
	\pend
      
	    
	    \stanza[\smallbreak]
	siddhaṃ yādṛg adhiṣṭhātṛbhāvābhāvānuvṛttimat |&sanniveśādi tad yuktaṃ tasmād yad anumīyate || \edtext{}{\lemma{||}\Bfootnote{(PV II 11)}}\&[\smallbreak]


	

	  \pstart iti | evaṃ ghaṭapaṭaparvatadīnāṃ kāryatvavastutvādibhir dharmaiḥ sajātīyatve 'py avāntaraṃ ghaṭapaṭaparvatatvādijātibhedam ādāya lokasya vyāptigrāhakaṃ pratyakṣaṃ pravartata iti darśayituṃ saṃvyavahārapragalbhapuruṣabuddhyapekṣayā yaddarśanād akriyādarśino 'pi kṛtabuddhir bhavatīty uktam | na tu śāstraparavaśabuddhipuruṣāpekṣayā | tathā hi śāstrasaṃskārarahitasya vyavahārapragalbhasya puruṣasya devakulajātīyakaṃ puruṣapūrvakatayāvadhāritavato nagarād vanaṃ praviṣṭasya parvatadevakulayor darśane tayor dvayor apy akriyādarśino 'pi devakule kṛtabuddhir bhavati na parvate | tad anayor devakulaparvatayoḥ kāryatvādinā ekajātitve 'pi kṛtabuddhibhāvābhāvau na tayoḥ parvatadevakulatvalakṣaṇāvāntarajātibhedam anavasthāpya sthātuṃ prabhavataḥ | jātibhede ca siddhe devakulajātīye vyāpter grahaṇāt na parvatajātīyasya, na ca prāsādajātīyasya vyāptisiddhir iti na tato buddhimadanumānam | yadā tu prāsādasyāpi pṛthag vyāptigrahaḥ tadā tajjātīyād api buddhimadanumānam astu | na kṣitidharādijātīyasya svapne 'pi vyāptigrahaḥ | krīḍāparvatāder nāmamātrā\leavevmode\ledsidenote{\textenglish{30b/RNAms}}\label{RNAms_30b}bhede 'pi parvatādibhir ekāntato bhinnasvarūpatvāt | \edlabel{sarit__ratnakīrtinibandhāvali__146084}\label{sarit__ratnakīrtinibandhāvali__146084}\edtext{}{\lemma{yac … brūmaḥ |}\xxref{sarit__ratnakīrtinibandhāvali__146084}{sarit__ratnakīrtinibandhāvali__146582}\Afootnote{\label{sarit__ratnakīrtinibandhāvali__401478}---\textsc{Note} Sanskrit quoted in \cite[68, n.~94]{krasser02_zaGkar_Izvar_studie}.  {\rmlatinfont [App type: parallel]}}}yac ca pṛṣṭaṃ keyaṃ kṛtabuddhir ityādi | tatra kāmaṃ sādhyabuddhir eveti brūmaḥ | yac cātroktaṃ sādhyabuddhir api yadi gṛhītavyāptikasya sā bhavaty eva | athāgṛhītavyāptikasya kim anyatrāpi sā bhavatī dṛṣṭeti ||
	\pend
      

	  \pstart atrocyate | gṛhītavyāptikasyānumānaṃ bhavati, agṛhītavyāptikasya na bhavatīty atrāsmākaṃ na kācid vipratipattiḥ | kevalaṃ gṛhītavyāptiko 'smin viṣaye na sambhavatīti brūmaḥ |\edlabel{sarit__ratnakīrtinibandhāvali__146582}\label{sarit__ratnakīrtinibandhāvali__146582} uktakrameṇa vyatirekāsiddher vyāvahārikapratyakṣeṇa kāryatvasya vyāptatvāniścayāt | tasmād avāntarajātibhedaprasiddhyarthaṃ vyāvahārikapuruṣāpekṣayaivāsyā buddher bhāvābhāvāv uktau | jātibhede ca prayojanaṃ pūrvam eva pratipāditam |
	\pend
      

	  \pstart yad apy atra \edlabel{rnā__144871}\label{rnā__144871}\edtext{}{\lemma{nipuṇamanyena}\xxref{rnā__144871}{rnā__144915}\Afootnote{\label{rnā__390487}nipu\unclear{ṇaṃ}manyena \cite{RNAms} ; nipuṇammanyena \cite{thakur75}   {\rmlatinfont [App type: var]}}}nipuṇamanyena\edlabel{rnā__144915}\label{rnā__144915} vācaspattinā kathitaṃ tat kiṃ kāryajātīyaṃ prāsādādi buddhimaddhetukaṃ na dṛṣṭaṃ yenotpattimattanubhuvanādi tathā na syāt, na khalu tajjātīyakatve kascidviśeṣa iti | tad asaṅgatam | tathā hi bhavatu prāsādaparvatādīnāṃ kāryatvādinā sajātī\gap{}yatvam | tat tu na vyāvahārikapratyakṣeṇa buddhimadvyāptaṃ pratyetuṃ śakyam, vyāptigrahaṇasamaye dṛṣṭānte buddhimadabhāvaprayuktasya kāryamātravyatirekasya darśayitum aśakyatvāt |
	\pend
      

	  \pstart tad ayaṃ saṃkṣepārthaḥ | kāryatvamātrasyāvyatirekād avyāptasyāgamakatvam | avāntaraṃ tu ghaṭaprāsādādisādhāraṇaṃ kāryatvamātram asmābhir api na svīkṛtam eva | yathā tu ghaṭatvapaṭatvādiprātisvikānekajātipuraskāreṇa prasiddhānumānavyavasthā sā cānavadyam avasthāpiteti |
	\pend
      \label{rnā__īsd__sādhya}
	  
	% new div opening: depth here is 2
	

	  \pstart saṃprati sādhyātmā vicāryate | nanu vādinā sādhane samupanyaste taddūṣaṇopanyāsam apāsya sādhyasvarūpavikalpanaṃ nāma naiyāyikamate niranuyojyānuyogaḥ, saugatamate tv adoṣodbhāvanaṃ nigrahasthānam iti cet | tad etaj jālmajalpitam | tathā hi sādhyasvarūpe 'pariniṣṭhite tadanusāriṇī pakṣasapakṣavipakṣavyavasthā kutaḥ | tadasiddhau cāsiddhatādayo doṣāḥ pakṣadharmatādayaś ca guṇā na vyavasthitā ity uktam | nedānīṃ hetor doṣaguṇakatheti mūkena prativādinā sthātavyam | tasmād dhetudoṣopanyāsaiveyaṃ sādhyaniruktir ity ayam eva vādī svamate niranuyojyānuyogadūṣaṇena nigrahasthānena nigṛhyata iti kim atra nirbandhena | 
	\pend
      

	  \pstart yad etat kārya\leavevmode\ledsidenote{\textenglish{31a/RNAms}}\label{RNAms_31a}tvaṃ sādhanaṃ kim anena viśvasya buddhimanmātrapūrvakatvaṃ sādhyate | āhosvid ekatvavibhutvasarvajñatvanityatvādiguṇaviśiṣṭabuddhimatpūrvakatvam | prathamapakṣe siddhasādhanam | dvitīye tu vyāpter abhāvād anaikāntikatā |
	\pend
      

	  \pstart nanu sāmānyena vyāptau pratītāyām api pakṣadharmatābalād viśeṣasiddhiḥ | yathāgneḥ parvatāyogavyavacchedādisiddhiḥ | anyathā sarvānumānocchedaḥ | anumānadveṣī hy evaṃ jalpati:
	\pend
      
	    
	    \stanza[\smallbreak]
	anumānabhaṅgapaṅke 'smin nimagnā vādidantinaḥ |&viśeṣe 'nugamābhāvaḥ sāmānye siddhasādhyatā ||\&[\smallbreak]


	

	  \pstart atrocyate | \edlabel{sarit__ratnakīrtinibandhāvali__149196}\label{sarit__ratnakīrtinibandhāvali__149196}\edtext{}{\lemma{sidhyaty … parvatavartitvādiviśeṣo}\xxref{sarit__ratnakīrtinibandhāvali__149196}{sarit__ratnakīrtinibandhāvali__149477}\Afootnote{\label{sarit__ratnakīrtinibandhāvali__401919}---\textsc{Note} Sanskrit, translation, and relation to Īśvarāpākaraṇasaṅkṣepa specified in \cite[195, n.~282]{krasser02_zaGkar_Izvar_studie}.  {\rmlatinfont [App type: parallel]}}}sidhyaty eva pakṣadharmatābalato viśeṣaḥ | na tu sarvaḥ | yena hi vinā \edlabel{rnā__147220}\label{rnā__147220}\edtext{}{\lemma{pakṣasthaṃ sādhanaṃ}\xxref{rnā__147220}{rnā__147269}\Afootnote{\label{rnā__390951}pakṣasthaṃ sādhanaṃ \cite{RNAms,thakur75} \textenglish{---\textsc{Note} Thakur's note 2 is wrong, it is clealy °kṣastha° in the ms.}  {\rmlatinfont [App type: comment]}}}pakṣasthaṃ sādhanaṃ\edlabel{rnā__147269}\label{rnā__147269} nopapadyate sa viśeṣaḥ sidhyatu | yathā vahner eva parvatavartitvādiviśeṣo\edlabel{sarit__ratnakīrtinibandhāvali__149477}\label{sarit__ratnakīrtinibandhāvali__149477} na pañcavarṇaśikhākalāpakamanīyaḥ | na ca girīṇāṃ tarūṇāṃ kāryatvaṃ kartur ekatvavibhutvasarvajñatvādikam antareṇa nopapadyate, taditareṣv api darśanāt | tasmāt
	\pend
      
	    
	    \stanza[\smallbreak]
	\label{ratnakīrtinibandhāvali__36r1NFGMVPDL94KA62XUYBRHVL8}\flagstanza{\tiny\textenglish{...BRHVL8}}pakṣāyogavyavaccheda\label{rnā__147570}\edtext{}{\lemma{bhedamātre na}\xxref{rnā__147570}{rnā__147613}\Afootnote{\label{rnā__391297}bhedamātre\deletion{bhede}na \cite{RNAms} ; bhedamātre na \cite{thakur75}   {\rmlatinfont [App type: orth]}}}bhedamātre na\label{rnā__147613} dūṣaṇam |&iṣṭasiddhyanvayābhāvād atirikte tu dūṣaṇam ||\edtext{}{\lemma{||}\Bfootnote{(JNA 268,19)}}\&[\smallbreak]


	

	  \pstart yady evaṃ \edlabel{rnā__147773}\label{rnā__147773}\edtext{}{\lemma{svasvarūpopā°}\xxref{rnā__147773}{rnā__147815}\Afootnote{\label{rnā__391587}sva\add{svarūpo3}pā \cite{RNAms} ; svasvarūpopā \cite{thakur75} ---\textsc{Note} As pointed out in \cite{thakur75}. Perhaps the intended correction was \begin{sanskrit}svarūpo\end{sanskrit}, however.  {\rmlatinfont [App type: orth]}}}svasvarūpopā\edlabel{rnā__147815}\label{rnā__147815}dānopakaraṇasaṃpradānaprayojanābhijña eva kartā \edlabel{rnā__147893}\label{rnā__147893}\edtext{}{\lemma{sādhyate | svarūpam iha ca dvyaṇukaṃ kāryam | upādānam}\xxref{rnā__147893}{rnā__147977}\Afootnote{\label{rnā__391975} \cite{} sādhyate |\add{svarūpamihacadvyaṇukaṃkāryaṃ3}upādānam \cite{RNAms}   {\rmlatinfont [App type: orth]}}}sādhyate | svarūpam iha ca dvyaṇukaṃ kāryam | upādānam\edlabel{rnā__147977}\label{rnā__147977} iha paramāṇujāticatuṣṭayam | upakaraṇaṃ samastakṣetrajñasamavāyidharmādharmau | sampradanaṃ kṣetrajñāḥ, yānayaṃ bhagavān svakarma\edlabel{rnā__148159}\label{rnā__148159}\edtext{}{\lemma{bhir abhipraiti |}\xxref{rnā__148159}{rnā__148206}\Afootnote{\label{rnā__392347} \cite{thakur75} bhirapraiti | \cite{RNAms} \textenglish{---\textsc{Note} Possible that something is added in the top margin. Only the bottom of two akṣāras are visible there, because the folio is overlapped by the one on top on the photo.}  {\rmlatinfont [App type: em]}}}bhir abhipraiti |\edlabel{rnā__148206}\label{rnā__148206} prayojanaṃ sukhaduḥkhopabhogaḥ kṣetrajñānām | \edlabel{ratnakīrtinibandhāvali__36r1NJ252MXEUNK6K4TYO5KZWY7}\label{ratnakīrtinibandhāvali__36r1NJ252MXEUNK6K4TYO5KZWY7}\edtext{}{\lemma{evaṃbhūte}\xxref{ratnakīrtinibandhāvali__36r1NJ252MXEUNK6K4TYO5KZWY7}{ratnakīrtinibandhāvali__36r1NJ252MZ3SBLPSHXDBDAC4WY}\Afootnote{\label{ratnakīrtinibandhāvali__36r1NJ253DNHUHRNHC0HVCCV4OH}evaṃ bhūte \cite{thakur75}   {\rmlatinfont [App type: punctuation]}}}evaṃbhūte\edlabel{ratnakīrtinibandhāvali__36r1NJ252MZ3SBLPSHXDBDAC4WY}\label{ratnakīrtinibandhāvali__36r1NJ252MZ3SBLPSHXDBDAC4WY} buddhimati sādhye kutaḥ siddhasādhanam | na cāvyāptiḥ | \edlabel{rnā__148350}\label{rnā__148350}\edtext{}{\lemma{kulāladṛṣṭānte upādānā}\xxref{rnā__148350}{rnā__148404}\Afootnote{\label{rnā__392792}kulāladṛṣṭāntena upādānā \cite{RNAms,thakur75}   {\rmlatinfont [App type: emendation]}}}kulāladṛṣṭānte upādānā\edlabel{rnā__148404}\label{rnā__148404}dyabhijñatvasya sambhavāt |
	\pend
      

	  \pstart tathā ca vācaspatiḥ pramāṇyati:\edlabel{ratnakīrtinibandhāvali__36r1NJ1JDU74SXAW1GSGGXN9PN1}\label{ratnakīrtinibandhāvali__36r1NJ1JDU74SXAW1GSGGXN9PN1}\edtext{}{\lemma{vivādādhyāsitās … tatheti}\xxref{ratnakīrtinibandhāvali__36r1NJ1JDU74SXAW1GSGGXN9PN1}{ratnakīrtinibandhāvali__36r1NJ1JLUB0U2KHDR5ES91PSZB}\Afootnote{\label{ratnakīrtinibandhāvali__36r1NJ1JLUFISBCCCGEOBBH3N1M}\textenglish{---\textsc{Note} Similar to: \href{file://../../../nyAyA/vAcaspatimizra/texts/nyAyavArttikatAtparyaTIka.xml\#nvtṭ__36r1NJ1JB8C6NHNHPFGA5WCCXW4}{file://../../../nyAyA/vAcaspatimizra/texts/nyAyavArttikatAtparyaTIka.xml\#nvtṭ\textunderscore \textunderscore 36r1NJ1JB8C6NHNHPFGA5WCCXW4}.}  {\rmlatinfont [App type: parallel]}}} vivādādhyāsitās tanugirisāgarādayaḥ upādānādyabhijñakartṛkāḥ | kāryatvāt | yad yat kāryaṃ tat tad upādānādyabhijñakartṛkam | yathā prāsādādi | tathā ca vivādādhyāsitās tanvādayaḥ | tasmāt tatheti\edlabel{ratnakīrtinibandhāvali__36r1NJ1JLUB0U2KHDR5ES91PSZB}\label{ratnakīrtinibandhāvali__36r1NJ1JLUB0U2KHDR5ES91PSZB} |
	\pend
      

	  \pstart evam ataḥ sādhanād upādānādyabhijñakartṛmātraṃ prasādhya tasya sarvajñatvasādhanāya vācaspatir eva punar apīdam āha: \edlabel{ratnakīrtinibandhāvali__36r1NJ1GUNF6HVSRTM28VAP3DL9}\label{ratnakīrtinibandhāvali__36r1NJ1GUNF6HVSRTM28VAP3DL9}\edtext{}{\lemma{bhavatu … tādṛgīśvarād}\xxref{ratnakīrtinibandhāvali__36r1NJ1GUNF6HVSRTM28VAP3DL9}{ratnakīrtinibandhāvali__36r1NJ1HHMVH5RR7U2GTVWCF63V}\Afootnote{\label{ratnakīrtinibandhāvali__36r1NJ1HHN0UIZL59XVOFV2QXCK}\textenglish{---\textsc{Note} Cf. \href{file://../../../nyAyA/vAcaspatimizra/texts/nyAyavArttikatAtparyaTIka.xml\#nvtṭ__36r1NJ1GUNHV7JE0L7ACW6RSRE0}{file://../../../nyAyA/vAcaspatimizra/texts/nyAyavArttikatAtparyaTIka.xml\#nvtṭ\textunderscore \textunderscore 36r1NJ1GUNHV7JE0L7ACW6RSRE0}.}  {\rmlatinfont [App type: parallel]}}}bhavatu tāvad upādānādyabhijñakartṛmātrasiddhiḥ | pāriśeṣyāt tu vyatirekidvitīyanāmno 'numānād viśeṣasiddhiḥ | tathā hi: tanubhuvanādyupādānādyabhijñaḥ kartā nānityāsarvaviṣayabuddhimān | tatkartus tadupādānādyanabhijñatvaprasaṅgāt | na hy evaṃvidhas tadupādānādyabhijño \edlabel{ratnakīrtinibandhāvali__36r1NJAZS8ND3NTZPZDACR3GERA}\label{ratnakīrtinibandhāvali__36r1NJAZS8ND3NTZPZDACR3GERA}\edtext{}{\lemma{yathāsmadādiḥ | tadupādānādyabhijñaś}\xxref{ratnakīrtinibandhāvali__36r1NJAZS8ND3NTZPZDACR3GERA}{ratnakīrtinibandhāvali__36r1NJAZS90JKBU2BVD5ZRNFMIB}\Afootnote{\label{ratnakīrtinibandhāvali__36r1NJAZSYJ31M8C7Y763COK6UY}yathāsmadādiḥ | tadupādānādyabhijñaś \cite{thakur75} ; yathāsmadādiḥ | \deletion{tadupādānādyanabhijñatvaprasaṅgāt} tadupādānādyabhijñaś \cite{RNAms}   {\rmlatinfont [App type: correction]}}}yathāsmadādiḥ | tadupādānādyabhijñaś\edlabel{ratnakīrtinibandhāvali__36r1NJAZS90JKBU2BVD5ZRNFMIB}\label{ratnakīrtinibandhāvali__36r1NJAZS90JKBU2BVD5ZRNFMIB} cāyam | ta\leavevmode\ledsidenote{\textenglish{31b/RNAms}}\label{RNAms_31b}smāt tatheti |
	\pend
      

	  \pstart no khalu paramāṇubhedān kṣetrajñasamavāyinaś ca karmāśayabhedān aparimeyān anyaḥ śakto jñātum ṛte tādṛgīśvarād\edlabel{ratnakīrtinibandhāvali__36r1NJ1HHMVH5RR7U2GTVWCF63V}\label{ratnakīrtinibandhāvali__36r1NJ1HHMVH5RR7U2GTVWCF63V} iti |
	\pend
      

	  \pstart atrocyate | yāvanti dvyaṇukāni bhinnadeśakālasvabhāvāni kāryāṇi santi teṣu sarveṣv eva kim eka eva buddhimān vyāpriyate | aneko vā | yad vā svasvaviṣayamātropādānādivedinaḥ parasparavyāpārānabhijñā bhinnadeśakālasvabhāvāḥ pratidvyaṇukam anya eva buddhimanto vyāpriyante iti trayaḥ pakṣāḥ |
	\pend
      

	  \pstart na tāvat \edlabel{ratnakīrtinibandhāvali__36r1NM7W6RLH3UQ01ZEG4Q53FWC}\label{ratnakīrtinibandhāvali__36r1NM7W6RLH3UQ01ZEG4Q53FWC}\edtext{}{\lemma{prathamaḥ pakṣaḥ |}\xxref{ratnakīrtinibandhāvali__36r1NM7W6RLH3UQ01ZEG4Q53FWC}{ratnakīrtinibandhāvali__36r1NM7W6SG03AESJNSHQ1FK1L5}\Afootnote{\label{ratnakīrtinibandhāvali__36r1NM7W7SUFEKSHR9YQ6XSG9WS}prathamaḥ pakṣaḥ | \cite{RNAms} ; prathamapakṣaḥ | \cite{thakur75}   {\rmlatinfont [App type: var]}}}prathamaḥ pakṣaḥ |\edlabel{ratnakīrtinibandhāvali__36r1NM7W6SG03AESJNSHQ1FK1L5}\label{ratnakīrtinibandhāvali__36r1NM7W6SG03AESJNSHQ1FK1L5} de\gap{}śakālasvabhāvabhinnānāṃ sarveṣāṃ dvyaṇukānāṃ kartur ekatvāsiddheḥ | yac caikatvasādhanāya kāryaliṅgāviśeṣād ityādy api sādhanam upanyastaṃ tad asaṅgatam | dhūmaliṅgāviśeṣe 'pi hy agner anekatvavat tatrāpi tacchaṅkāsambhavāt | \label{ratnakīrtinibandhāvali__36r1NSAWNTV0PU093F550ENXSD2}\edtext{}{\lemma{sad iti liṅgāviśeṣād}\xxref{ratnakīrtinibandhāvali__36r1NSAWNTV0PU093F550ENXSD2}{ratnakīrtinibandhāvali__36r1NSAWNTWQ48AKDPN5RIVJIC3}\Afootnote{\label{ratnakīrtinibandhāvali__36r1NSAWNU07VM3DNQQNL6L15SZ}\textenglish{---\textsc{Note} \begin{sanskrit}sad iti jñānāviśeṣāt\end{sanskrit} \cref{ratnakīrtinibandhāvali__36r1NSAS7AJHGG1XWSGMMV9343A}.}  {\rmlatinfont [App type: parallel]}}}sad iti liṅgāviśeṣād\label{ratnakīrtinibandhāvali__36r1NSAWNTWQ48AKDPN5RIVJIC3} iti tu dṛṣṭānto 'smān pratyasiddha eva | tasmād yathā mayā nānātvasādhanāya pramāṇaṃ vaktavyaṃ tathā tvayāpy ekatvasādhanāya sādhanam abhidhānīyam |
	\pend
      

	  \pstart atha manyate anekatvasādhanābhāvād ekatvasiddhir iti | yady evam ekatvasādhanābhāvad anekatvam eva kiṃ nāvagacchasi |
	\pend
      

	  \pstart yad apy uktam: \edlabel{ratnakīrtinibandhāvali__36r1NSAZOF4PYRJJ9T4KL5GNBA4}\label{ratnakīrtinibandhāvali__36r1NSAZOF4PYRJJ9T4KL5GNBA4}\edtext{}{\lemma{ekatve … ityādi}\xxref{ratnakīrtinibandhāvali__36r1NSAZOF4PYRJJ9T4KL5GNBA4}{ratnakīrtinibandhāvali__36r1NSAZOF6VPMT1M1P47SSA79H}\Afootnote{\label{ratnakīrtinibandhāvali__36r1NSAZP9YLE8DI58KMDSHDTF3}\textenglish{---\textsc{Note} Quote of \cref{ratnakīrtinibandhāvali__36r1NSAZCVOOMC2YM2ZB4UM8USC}.}  {\rmlatinfont [App type: parallel]}}}ekatve tu na pramāṇāntaram anveṣṭavyam ekasya kartur abhāve bahūnāṃ vyāhatamanasām ityādi\edlabel{ratnakīrtinibandhāvali__36r1NSAZOF6VPMT1M1P47SSA79H}\label{ratnakīrtinibandhāvali__36r1NSAZOF6VPMT1M1P47SSA79H} | \edlabel{ratnakīrtinibandhāvali__36r1NMN5X0PC291R8Q4FO5BKCDN}\label{ratnakīrtinibandhāvali__36r1NMN5X0PC291R8Q4FO5BKCDN}tad api cintyatām | bahubhiḥ karaṇe yugapat kāryānutpattir iti kiṃ bhinnadeśakālānāṃ kāryāṇām anutpattir vivakṣitā | ekasyaiva vā mahāvayavinaḥ kṣitighaṭādirūpasya | tatra ekasminn api kārye bahubhiḥ karaṇe utpattivirodhinaṃ na paśyāmaḥ | bahūnāṃ parasparaṃ vaimatyaniyamābhāvāt | parasparāvyāghātapuruṣatvayor dvividhasyāpi virodhasyāsambhavāt | puruṣatvaṃ hi apuruṣatvena viruddham | na tu parasparāvyāghātena |
	\pend
      

	  \pstart ye tv anantadeśakālasvabhāvabhedabhinnāsteṣu sutarām evānekavyāpāraniṣedho 'sambhavīti dvitīyo 'pi pakṣo vyudastaḥ | na ca kartur ekatvena dṛṣṭā vyāptisiddhiḥ |  anekenāpi svatantreṇa svasvaprayojanārthinā grāmapraviṣṭahariṇādimāraṇaikakāryadarśanāt | tasyāpi pakṣīkaraṇe ekakartṛpūrvakābhimatasyāpi pakṣīkaraṇe ātmakartṛpūrvakatvam astu | tad evaṃ na sarvadvyaṇukānāṃ kartur ekatvasiddhiḥ | \edlabel{ratnakīrtinibandhāvali__36r1NMMFX7IAGR5V3DIY9Q16BLF}\label{ratnakīrtinibandhāvali__36r1NMMFX7IAGR5V3DIY9Q16BLF}\edtext{}{\lemma{tathā coktam ... āśrayam |}\xxref{ratnakīrtinibandhāvali__36r1NMMFX7IAGR5V3DIY9Q16BLF}{ratnakīrtinibandhāvali__36r1NMFLNZF26QG3OB18MFYZVHQ}\Afootnote{\label{ratnakīrtinibandhāvali__36r1NMFLNZHO0BSBWPYS4CX4X7Q}\textenglish{---\textsc{Note} Cf. : \begin{sanskrit}एककर्तुरसिद्धौ च सर्वज्ञत्वं किमाश्रयम् ।\end{sanskrit}. \cite{thakur75} notes that this is a ``marginal edition, separate hand".}  {\rmlatinfont [App type: parallel]}}}tathā coktam
	\pend
      
	    
	    \stanza[\smallbreak]
	\label{ratnakīrtinibandhāvali__36r1NMFLFX785HFFPE2O9XY3QCP}\flagstanza{\tiny\textenglish{...XY3QCP}}ekakartur na siddhau tu sarvajñatvaṃ kim āśrayam |\label{ratnakīrtinibandhāvali__36r1NMFLNZF26QG3OB18MFYZVHQ}\&[\smallbreak]


	

	  \pstart ata eva dvitīyo 'pi pakṣaḥ kṣīṇaḥ | saveṣu dvyaṇukeṣv ekasyāpi kartur apravṛttau bahūnāṃ sutarām apravṛtteḥ |
	\pend
      

	  \pstart tṛ\leavevmode\ledsidenote{\textenglish{pb in}}\label{RNAms_32a}tīyas tu pakṣo yadi bhavet tadā svasvavyāpāraviṣayamātropādānādyabhijñatve 'pi naikaḥ kaścit sarvajñaḥ sidhyati | na ca \edlabel{rnā__151710}\label{rnā__151710}\edtext{}{\lemma{jñānasattāmātreṇa}\xxref{rnā__151710}{rnā__151759}\Afootnote{\label{rnā__393154}jñānasattāmātreṇa \cite{RNAms} ; jñānaṃ sattāmātreṇa \cite{thakur75}   {\rmlatinfont [App type: var]}}}jñānasattāmātreṇa\edlabel{rnā__151759}\label{rnā__151759} katipayātīndriyadarśanavat sarvārthagrahaṇaṃ yena tadabhedāt prastutaparamāṇuvat sarvasyaivāviśeṣeṇa grahaṇāt sarvajñatā syāt | anumānato hi katipayātīndriyadarśane siddhe 'pīśvarasya tatkāraṇayogitvaṃ niścīyate | na tu jñānasāttāmātreṇa prakārāntareṇeti niścaya iti kutaḥ sarvajñatā |
	\pend
      

	  \pstart nanv atīndriyaṃ paramāṇvādikaṃ jānato na kathaṃ sārvajñyam iti cet | tat kim idānīm asarvadarśitveṣv atīndriyadarśanamātreṇa sarvajñatāpratyayāśā | evam eveti cet | hanta yadi nāma nyāyavihastena tvayā īdṛśo hastasamāracitaḥ sarvajñaḥ paribhāvitas tathāpy anyeṣām apāradūradeśakālavartināṃ dvyaṇukādīnām upādānādiṣu januṣāndhaprakhyasya paramapuruṣārthāvedino vā lokaiḥ prāmāṇikaiś ca nāsya sārvajñyam anumanyate ||
	\pend
      

	  \pstart asmākan tu nātīndriyadarśimātre pradveṣaḥ | evaṃ ca kartur ekatvāsiddhau vyatireky api hetur asamarthaḥ viśveṣām ekasya kartur asiddhau tadupādānādyabhijñabhāvasyāsiddhatvāt | yaś ca yanmātrakāraḥ sa tanmātropādānādyabhijño bhavan na sarvajñaḥ | anekāśrayeṇāpi upādānādyabhijñasāmānyasya caritārthatvāt | tad evam upādānādyabhijñapuruṣamātrasiddhāv api naikatvasarvajñatvādiviśiṣṭapuruṣaviśeṣasiddhiḥ | puruṣamātre ca siddhasādhanam uktam | buddhimanmātrapūrvakatām icchatām upādānādyabhijñabuddhimatpūrvakatve sādhye kathaṃ \edlabel{ratnakīrtinibandhāvali__36r1NMNKPK8I8K079OIC2E8BVU5}\label{ratnakīrtinibandhāvali__36r1NMNKPK8I8K079OIC2E8BVU5}\edtext{}{\lemma{siddhasādhanam}\xxref{ratnakīrtinibandhāvali__36r1NMNKPK8I8K079OIC2E8BVU5}{ratnakīrtinibandhāvali__36r1NMNKPKAMZXOQ29O4V14D9F6}\Afootnote{\label{ratnakīrtinibandhāvali__36r1NMNKQ1PS32JKPAF5U8FE58P}siddhasādhanam \cite{thakur75} ; siddhisādhanam \cite{RNAms}   {\rmlatinfont [App type: var]}}}siddhasādhanam\edlabel{ratnakīrtinibandhāvali__36r1NMNKPKAMZXOQ29O4V14D9F6}\label{ratnakīrtinibandhāvali__36r1NMNKPKAMZXOQ29O4V14D9F6} iti cet | na tadapekṣayā siddhasādhyatāyā janitatvāt kevalam asiddhoddhāre 'bhimate viśeṣe siddhe 'pi naiyāyikasyāpi nābhimatasiddhir iti brūmaḥ ||
	\pend
      

	  \pstart saugatasya tāvad aniṣṭasiddhir iti cet, na, svābhimatasādhyasādhanenaiva hi parasyāniṣṭam api sādhanīyam | anyathā mātṛśokasmaraṇādināpi tadaniṣṭasiddhiḥ syād iti | asya saṅgrahaḥ
	\pend
      
	    
	    \stanza[\smallbreak]
	pareṣṭasiddhir napareṣṭabādhakaṃ prasādhane vedanayatnamātrayoḥ |&ananvayo 'bhīṣṭaviśeṣasādhane vipakṣasandehasahantu sādhanam ||\&[\smallbreak]


	\label{īsd-sādhyacintā}
	  
	% new div opening: depth here is 2
	

	  \pstart sādhyacintādhikāras tṛtīyaḥ ||
	\pend
      

	  \pstart evam anye 'pi hetavo yathāyogam abhyūhya dūṣaṇīyāḥ | tad evaṃ tāvad īśvarasya sadvyavahāro niṣiddhaḥ | asadvyavahārārthan tu tallakṣaṇavilakṣaṇakṣaṇabhaṅgasādhakaṃ sattādisādhanam eva draṣṭavyam iti ||\leavevmode\ledsidenote{\textenglish{pb in}}\label{RNAms_32B1}
	\pend
      
	    
	    \stanza[\smallbreak]
	\label{thakur75-57.14}\flagstanza{\tiny\textenglish{...-57.14}}ity abodhajanakartṛvikalpa vyāpi mohatimirapratirodhi |&ratnakīrtir acanāmalaramya jyotir astu ciramapratirodhi ||\&[\smallbreak]


	
	    
	    \endnumbering% ending numbering from div
	    \endgroup
	    
	  
	  
	% new div opening: depth here is 0
	
	    
	    \begingroup
	    \beginnumbering% beginning numbering from div depth=0
	    
	  
\chapter[{Apohasiddhiḥ}]{Apohasiddhiḥ}\label{Apohasiddhiḥ}

	  \pstart || namas tārāyai || \edlabel{thakur75-58.5}\label{thakur75-58.5} apohaḥ śabdārtho nirucyate | nanu ko 'yam apoho nāma | kim idam anyasmād apohyate | asmād vānyad apohyate | asmin vānyad apohyata iti vyutpattyā vijātivyāvṛttaṃ bāhyam eva vivakṣitam | buddhyākāro vā | yadi vā apohanam apoha ity anyavyāvṛttimātram iti trayaḥ pakṣāḥ | \edlabel{thakur75-58.9}\label{thakur75-58.9} na tāvad ādimau pakṣau apohanāmnā vidher eva vivakṣitatvāt | antimo 'py asaṅgataḥ, pratītibādhitatvāt | tathā hi parvatoddeśe vahnir astīti śābdī pratītir vidhirūpam evollikhantī lakṣyate | nānagnir na bhavatīti nitrṛttimātram āmukhayantī | yac ca pratyakṣabādhitaṃ na tatra sādhanāntarāvakāśa ity atiprasiddham ||
	\pend
      

	  \pstart atha yady api nivṛttim ahaṃ pratyemīti na vikalpaḥ tathāpi nivṛttapadārthollekha eva nivṛttyullekhaḥ | na hy anantrbhāvitaviśeṣaṇapratītir viśiṣṭapratītiḥ | tato yathā sāmānyam ahaṃ pratyemīti vikalpābhāve 'pi sādhāraṇākāraparisphuraṇād vikalpabuddhiḥ sāmānyabuddhiḥ pareṣām, tathā nivṛttapratyayākṣiptā nivṛttibuddhir apohapratītivyavahāramātanotīti cet |
	\pend
      

	  \pstart nanu sādhāraṇākāraparisphuraṇe vidhirūpatayā yadi sāmānyabodhavyavasthā, tat kim āyātam asphuradabhāvākāre cetasi nivṛttipratītivyavasthāyāḥ | tato nivṛttim ahaṃ pratyemīty evam ākārābhāve 'pi nivṛttyākārasphuraṇaṃ yadi syāt ko nāma nivṛttipratītisthitim apalapet | anyathā asati pratibhāse tatpratītivyavahṛtir iti gavākāre 'pi cetasi turagabodha ity astu ||
	\pend
      

	  \pstart atha viśeṣaṇtayā antarbhūtā nivṛttipratītir ity uktam | tathāpi yady agavāpoḍha itīdṛśākāro vikalpas tadā viśeṣaṇatayā tadanupraveśo bhavatu kiṃ tu gaur iti pratītiḥ | tadā ca sato 'pi nivṛttilakṣaṇasya viśeṣaṇasya tatrānutkalanāt kathaṃ tatpratītivyavasthā |
	\pend
      

	  \pstart athaivaṃ matiḥ: yad vidhirūpaṃ sphurati tasya parāpoho 'py astīti tatpratītir ucyate | tadāpi sambandhamātram apohasya | vidhir eva sākṣān nirbhāsī | api caivam adhyakṣasyāpy apohaviṣayatvam anivāryam viśeṣato vikalpād ekavyāvṛttollekhino 'khilānyavyāvṛttam īkṣamāṇasya | tasmād vidhyākārāvagrahād adhyakṣavad vikalpasyāpi vidhiviṣayatvam eva nānyāpohaviṣayatvam iti katham apohaḥ śabdārtho ghuṣyate | 
	\pend
      

	  \pstart atrābhidhīyate | nāsmābhir apohaśabdena vidhir eva kevalo 'bhipretaḥ | nāpy anyavyāvṛttimātram | kin tv anyāpohaviśiṣṭo vidhiḥ śabdānām arthaḥ | tataś ca na pratyekapakṣopanipātidoṣāvakāśaḥ || \edlabel{thakur75-59.7}\label{thakur75-59.7} yat tu goḥ pratītau na tadātmāparātmeti sāmarthyād apohaḥ paścān niścīyata iti vidhivādināṃ matam, anyāpohapratītau vā sāmarthyād anyāpoḍho 'vadhāryate iti pratiṣedhavādināṃ matam | tad asundaram | prāthamikasyāpi pratipattikramādarśanāt | na hi vidhiṃ pratipadya kaścid arthāpattitaḥ paścād apoham avagacchati | apohaṃ vā pratipadyānyāpoḍham | tasmād goḥ pratipattir ity anyāpoḍhapratipattir ucyate | yady api cānyāpoḍhaśabdānullekha uktas tathāpi nāpratipattir eva viśeṣaṇabhūtasyāpohasya | agavāpoḍha eva gośabdasya niveśitatvāt | yathā nīlotpale niveśitād indīvaraśabdān nīlotpalapratītau tatkāla eva nīlimasphuraṇam anivāryaṃ tathā gośabdād apy agavāpoḍhe niveśitād gopratītau tulyakālam eva viśeṣaṇtvād ago 'pohasphuraṇam anivāryam | yathā pratyakṣasya prasajyarūpābhāvāgrahaṇam abhāvavikalpotpādanaśaktir eva tathā vidhivikalpānām api tadanurūpānuṣṭhānadānaśaktir evābhāvagrahaṇam abhidhīyate | paryudāsarūpābhāvagrahaṇaṃ tu niyatasvarūpasaṃvedanam ubhayor aviśiṣṭam | anyathā yadi śabdād arthapratipattikāle kalito na parāpohaḥ katham anyaparihāreṇa pravṛttiḥ | tato gāṃ badhāneti codito 'śvādīn api badhnīyāt || \edlabel{thakur75-59.21}\label{thakur75-59.21} yad apy avocad Vācaspatiḥ jātimatyo vyaktayo vikalpānāṃ śabdānāṃ ca gocaraḥ | tāsāṃ ca tadvatīnāṃ rūpam atajjātīyaparāvṛttim ity atas tadavagater na gāṃ badhāneti codito 'śvādīn badhnāti | tad apy anenaiva nirastam | yato jāter adhikāyāḥ prakṣepe 'pi vyaktīnāṃ rūpam atajjātīyaparāvṛttam eva cet, tadā tenaiva rūpeṇa śabdavikalpayor viṣayībhavantīnāṃ katham atadvyāvṛttiparihāraḥ || \edlabel{thakur75-59.26}\label{thakur75-59.26} atha na vijātīyavyāvṛttaṃ vyaktirūpaṃ tathāpratītaṃ vā tadā jātiprasāda eṣa iti katham arthato 'pi tadavagatir ity uktaprāyam | \edlabel{thakur75-59.28}\label{thakur75-59.28} atha jātibalād evānyato 'vyāvṛttam | bhavatu jātibalāt svahetuparamparābalād vānyavyāvṛttam | ubhayathāpi vyāvṛttapratipattau vyāvṛttipratipattir asty eva | \edlabel{thakur75-60.1}\label{thakur75-60.1} na cāgavāpoḍhe gośabdasaṅketavidhāv anyonyāśrayadoṣaḥ | sāmānye tadvati vā saṃkete 'pi taddoṣāv akāśāt | na hi sāmānyaṃ nāma sāmānyamātram abhipretam, turage 'pi gośabdasaṃketaprasaṅgāt | kiṃ tu gotvam | tāvatā ca sa eva doṣaḥ | gavādiparijñāne gotvasāmānyāparijñānāt | gotvasāmānyāparijñāne gośabdavācyāparijñānāt | \edlabel{thakur75-60.6}\label{thakur75-60.6} tasmād ekapiṇḍadarśanapūrvako yaḥ sarvavyaktisādhāraṇa iva bahiradhyasto vikalpabuddhyākāraḥ tatrāyaṃ gaur iti saṃketakaraṇe netaretarāśrayadoṣaḥ | \edlabel{thakur75-60.8}\label{thakur75-60.8} abhimate ca gośabdapravṛttāv agośabdena śeṣasyāpy abhidhānam ucitam | na cānyāpoḍhānyāpohayor virodho viśeṣyaviśeṣaṇabhāvakṣatir vā, parasparavyavacchedābhāvāt | sāmānādhikaraṇyasadbhāvāt | bhūtalaghaṭābhāvavat | svābhāvena hi virodho na parābhāvenety ābālaprasiddham | eṣa panthāḥ śrudhnam upatiṣṭhata ity atrāpy apoho gamyata eva | aprakṛtapathāntarāpekṣayā eṣa eva śrudhnapratyanīkāniṣṭasthānāpekṣayā śrudhnam eva | araṇyamārgavad vicchedābhāvād upatiṣṭhata eva | sārthadūtādivyavacchedena panthā eveti pratipadaṃ vyavacchedasya sulabhatvāt | tasmād apohadharmaṇo vidhirūpasya śabdād avagatiḥ puṇḍarīkaśabdād iva śvetim aviśiṣṭasya padmasya || \edlabel{thakur75-60.16}\label{thakur75-60.16} yady evaṃ vidhir eva śabdārtho vaktum ucitaḥ, katham apoho gīyata iti cet | uktamatrāpohaśabdenānyāpohaviśiṣṭo vidhir ucyate | tatra vidhau pratīyamāne viśeṣaṇatayā tulyakālam anyāpohapratītir iti | \edlabel{thakur75-60.19}\label{thakur75-60.19} na caivaṃ pratyakṣasyāpy apohaviṣayatvavyavasthā kartum ucitā | tasya śābdapratyayasyeva vastuviṣayatve vivādābhāvāt | vidhiśabdena ca yathādhyavasāyam atadrūpaparāvṛtto bāhyo 'rtho 'bhimataḥ, yathāpratibhāsaṃ buddhyākāraś ca | tatra bāhyo 'rtho 'dhyavasāyād eva śabdavācyo vyavasthāpyate | na svalakṣaṇaparisphūrtyā | pratyakṣavad deśakālāvasthāniyatapravyaktasvalakṣaṇāsphuraṇāt | yac chāstram
	\pend
      

	  \pstart śabdenāvyāpṛtākṣasya buddhāv apratibhāsanāt | arthasya dṛṣṭāv iva \edtext{}{\lemma{iva}\Bfootnote{(PVin I 15)}}
	\pend
      

	  \pstart iti | indriyaśabdasvabhāvopāyabhedād ekasyaivārthasya pratibhāsabheda iti cet | atrāpy uktam:
	\pend
      

	  \pstart jāto nāmāśrayo 'nyānyaḥ cetasāṃ tasya vastutaḥ | ekasyaiva kuto rūpaṃ bhinnākārāvabhāsi tat || \edtext{}{\lemma{||}\Bfootnote{(PV III 235)}} \edlabel{thakur75-60.29}\label{thakur75-60.29} na hi spaṣṭāspaṣṭe dve rūpe parasparaviruddhe ekasya vastunaḥ staḥ | yata ekenendriyabuddhau pratibhāsetānyena vikalpe | tathā sati vastuna eva bhedaprāpteḥ | na hi svarūpabhedād aparo vastubhedaḥ | na ca pratibhāsabhedād aparaḥ svarūpabhedaḥ | anyathā trailokyam ekam eva vastu syāt || \edlabel{thakur75-61.3}\label{thakur75-61.3} dūrāsannadeśavartinoḥ puruṣayor ekatra śākhini spaṣṭāspaṣṭapratibhāsabhede 'pi na śākhibheda iti cet | na brūmaḥ pratibhāsabhedo bhinnavastuniyataḥ, kiṃ tv ekaviṣayatvābhāvaniyata iti | tato yatrārthakriyābhedādisacivaḥ pratibhāsabhedas tatra vastubhedaḥ, ghaṭavat | anyatra punarniyamenaikaviṣayatāṃ pariharatīty ekapratibhāso bhrāntaḥ || \edlabel{thakur75-61.7}\label{thakur75-61.7} etena yad āha Vācaspatiḥ: na ca śabdapratyakṣayor vastugocaratve pratyayābhedaḥ kāraṇabhedena pārokṣyāpārokṣyabhedopapatter iti, tannopayogi | parokṣapratyayasya vastugocaratvāsamarthatāt | parokṣatāśrayas tu kāraṇabheda indriyagocaragrahaṇaviraheṇaiva kṛtārthaḥ | tan na | śābde pratyaye svalakṣaṇaṃ parisphurati | kiṃ ca svalakṣaṇātmani vastuni vācye sarvātmanā pratipatteḥ vidhiniṣedhayor ayogaḥ | tasya hi sadbhāve 'stīti vyartham, nāstīty asamartham | asadbhāve tu nāstīti vyartham, astīty asamartham | asti cāstyādipadaprayogaḥ | tasmāt śābdapratibhāsasya bāhyārthabhāvābhāvasādhāraṇyaṃ na tadviṣayatāṃ kṣamate || \edlabel{thakur75-61.15}\label{thakur75-61.15} yac ca Vācaspatinā jātimadvyaktivācyatāṃ svavācaiva prastutyāntaram eva na ca śabdārthasya jāter bhāvābhāvasādhāraṇyaṃ nopapadyate | sā hi svarūpato nityāpi deśakālaviprakīrṇānekavyaktyāśrayatayā bhāvābhāvasādhāraṇībhavanty astināstisambandhayogyā | vartamānavyaktisambandhitā hi jāter astitā | atītānāgatavyaktisambandhitā ca nāstiteti sandigdhavyatirekitvād anaikāntikaṃ bhāvābhāvasādhāraṇyam, anyathāsiddhaṃ veti vikalpitam | tad aprastutam | tāvatā tāvan na prakṛtakṣatiḥ | jātau bharaṃ nyasyatā svalakṣaṇavācyatvasya svayaṃ svīkārāt | kiṃ ca sarvatra padārthaya svalakṣaṇasvarūpeṇaivāstitvādikaṃ cintyate | jātes tu vartamānādivyaktisambadhī 'stitvādikam iti tu bālapratāraṇam | evaṃ jātimadvyaktivacane 'pi doṣaḥ | vyakteś cet pratītisiddhiḥ jātir adhikā pratīyatāṃ mā vā, na tu vyaktipratītidoṣānmuktiḥ | \edlabel{thakur75-61.25}\label{thakur75-61.25} etena yad ucyate Kaumārilaiḥ sabhāgatvād eva vastuno na sādhāraṇyadoṣaḥ | vṛkṣatvaṃ hy anirdhāritabhāvābhāvaṃ śabdād avagamyate | tayor anyatareṇa śabdāntarāvagatena sambadhyata iti | tad apy asaṅgatam | sāmānyasya nityasya pratipattāv anirdhāritabhāvābhāvatvāyogāt | \edlabel{thakur75-62.1}\label{thakur75-62.1} yac cedam - na ca pratyakṣasyeva śabdānām arthapratyāyanaprakāro yena taddṛṣṭa ivāstyādiśabdāpekṣā na syāt, vicitraśaktitvāt pramāṇānām iti | tad apy aindriyakaśābdapratibhāsayor ekasvarūpagrāhitve bhinnāvabhāsadūṣaṇena dūṣitam | vicitraśaktitvaṃ ca pramāṇānāṃ sākṣātkārādhyavasāyābhyām api caritārtham | tato yadi pratyakṣārthapratipādanaṃ śābdena tadvad evāvabhāsaḥ syāt | abhavaṃś ca na tadviṣayakhyāpanaṃ kṣamate || \edlabel{thakur75-62.6}\label{thakur75-62.6} nanu vṛkṣaśabdena vṛkṣatvāṃśo codite sattvādyaṃśaniścayanārtham astyādipadaprayoga iti cet | \edlabel{thakur75-62.8}\label{thakur75-62.8} niraṃśatvena pratyakṣasamadhigatasya svalakṣaṇasya ko 'vakāśaḥ padāntareṇa | dharmāntaravidhiniṣedhayoḥ pramāṇāntareṇa vā | pratyakṣe 'pi pramāṇāntarāpekṣā dṛṣṭeti cet | bhavatu tasyāniścayātmakatvād anabhyastasvarūpaviṣaye | vikalpas tu svayaṃ niścayātmako yatra grāhī tatra kim apareṇa | asti ca śabdaliṅgāntarāpekṣā | tato na vastusvarūpagrahaḥ || \edlabel{thakur75-62.13}\label{thakur75-62.13} nanu bhinnā jātyādayo dharmāḥ parasparaṃ dharmiṇaś ceti jātilakṣaṇaikadharmadvāreṇa pratīte 'pi śākhini dharmāntaravattayā na pratītir iti kiṃ na bhinnābhidhānādhīno dharmāntarasya nīlacaloccais taratvāder avabodhaḥ | tad etad asaṅgatam | akhaṇḍātmanaḥ svalakṣaṇasya pratyakṣe 'pi pratibhāsāt | dṛśyasya dharmadharmibhedasya pratyakṣapratikṣitpatatvāt | anyathā sarvaṃ sarvatra syād ity atiprasaṅgaḥ | kālpanikabhedāśrayas tu dharmadharmivyavahāra iti prasādhitaṃ śāstre \edtext{}{\lemma{śāstre}\Bfootnote{(PVin?)}} |
	\pend
      

	  \pstart bhavatu vā pāramārthiko 'pi dharmadharmibhedaḥ | tathāpy anayoḥ samavāyāder dūṣitatvād upakāralakṣaṇaiva pratyāsattir eṣitavyā | evaṃ ca yathendriyapratyāsattyā pratyakṣeṇa dharmipratipattau sakalataddharmapratipattis tathā śabdaliṅgābhyām api vācyavācakādisambandhapratibaddhābhyāṃ dharmipratipatau niravaśeṣataddharmapratipattir bhavet | pratyāsattimātrasyāviśeṣāt ||
	\pend
      

	  \pstart yac ca Vācaspatiḥ, na caikopādhinā sattvena viśiṣṭe tasmin gṛhīte upādhyantaraviśiṣṭas tadgrahaḥ | svabhāvo hi dravyasyopādhibhir viśiṣyate | na tūpādhayo vā viśeṣyatvaṃ vā tasya svabhāva iti | tad api plavata eva | na hy abhedād upādhyantaragrahaṇam āsañjitam | bhedaṃ punas kṛtyaivopakārakagrahaṇe upakāryagrahaṇaprasañjanāt | na cāgnidhūmayoḥ kāryakāraṇabhāva iva svabhāvata eva dharmadharmiṇoḥ pratipattiniyamakalpanam ucitam | tayor api pramāṇāsiddhatvāt | pramāṇsiddhe ca svabhāvopavarṇanam iti nyāyaḥ || \edlabel{thakur75-63.3}\label{thakur75-63.3} yac cātra Nyāyabhūṣaṇena sūryādigrahaṇe tadupakāryāśeṣavasturāśigrahaṇaprasañjanam uktam, tadabhiprāyānavagāhanaphalam | tathā hi tvanmate dharmadharmiṇor bhedaḥ, upakāralakṣaṇaiva ca pratyāsattis tadopakārakagrahaṇe samānadeśasyaiva dharmarūpasyaiva copakāryasya grahaṇam āsañjitam | tat kathaṃ sūryopakāryasya bhinnadeśasya dravyāntarasya vā dṛṣṭavyabhicārasya grahaṇaprasaṅgaḥ saṅgataḥ | tasmād ekadharmadvāreṇāpi vastusvarūpapratipattau sarvātmapratīteḥ kva śabdāntareṇa vidhiniṣedhāvakāśaḥ | asti ca | tasmān na svalakṣaṇsya śabdavikalpaliṅgapratibhāsitvam iti sthitam || \edlabel{thakur75-63.10}\label{thakur75-63.10} nāpi sāmānyaṃ śābdapratyayapratibhāsi | saritaḥ pāre gāvaś carantīti gavādiśabdāt sāsnāśṛṅgalāṅgūlādayo 'kṣarākāraparikaritāḥ sajātīyabhedāparāmarśanāt sampiṇḍitaprāyāḥ pratibhāsante | na ca tad eva sāmānyam |
	\pend
      
	    
	    \stanza[\smallbreak]
	varṇākṛtyakṣarākāraśūnyaṃ gotvaṃ hi kathyate | \edtext{}{\lemma{|}\Bfootnote{(PV III 147)}}\&[\smallbreak]


	

	  \pstart tad eva ca sāsnāśṛṅgādimātram akhilavyaktāv atyantavilakṣaṇam api svalakṣaṇenaikīkriyamāṇaṃ sāmānyam ity ucyate tādṛśasya bāhyasyāprāpter bhrāntir evāsau keśapratibhāsavat | tasmād vāsanāvaśād buddher eva tadātmanā vivarto 'yam astu | asad eva vā tadrūpaṃ khyātu | vyaktaya eva vā svajātīyabhedatiraskāreṇānyathā bhāsantām anubhavavyavadhānāt smṛtipramoṣo vābhidhīyatām | sarvathā nirviṣayaḥ khalv ayaṃ sāmānyapratyayaḥ | kva sāmānyavārtā | 
	\pend
      

	  \pstart yat punaḥ sāmānyābhāve sāmānyapratyayasyākasmikatvam uktaṃ tad ayuktam | yataḥ pūrvapiṇḍadarśanasmaraṇasahakāriṇātiricyamānaviśeṣapratyayajanikā sāmagrī nirviṣayaṃ sāmānyavikalpam utpādayati | tad evaṃ na śābde pratyaye jātiḥ pratibhāti | nāpi pratyakṣe | na cānumānato 'pi siddhiḥ | adṛśyatve pratibaddhaliṅgād adarśanāt | nāpīndriyavad asyāḥ siddhiḥ jñānakāryataḥ kādācitkasyaiva nimittāntarasya siddheḥ | yadā piṇḍāntare antarāle vā gobuddher abhāvaṃ darśayet tadā śāvaleyādisakalagopiṇḍānām evābhāvād abhāvo gobuddher upapadyamānaḥ katham arthāntaram ākṣipet | atha gotvād eva gopiṇḍaḥ | anyathā turago 'pi gopiṇḍaḥ syāt | yady evaṃ gopiṇḍād eva gotvam anyathā turagatvam api gotvaṃ syāt | tasmāt kāraṇaparamparāta eva gopiṇḍo gotvaṃ tu bhavatu mā vā | nanu sāmānyapratyayajananasāmarthyaṃ yady ekasmāt piṇḍād abhinnaṃ tadā vijātīyavyāvṛttaṃ piṇḍāntaram asamartham | atha bhinnam, tadā tad eva sāmānyam, nāmni paraṃ vivāda iti cet | abhinnaiva sā śaktiḥ prativastu | yathā tv ekaḥ śaktasvabhāvo bhāvas tathānyo 'pi bhavan kīdṛśaṃ doṣam āvahati | yathā bhavatāṃ jātir ekāpi samānadhvaniprasavahetuḥ, anyāpi svarūpeṇaiva jātyantaranirapekṣā, tathāsmākaṃ vyaktir api jātinirapekṣā svarūpeṇaiva bhinnā hetuḥ || \edlabel{thakur75-64.7}\label{thakur75-64.7} yat tu \persName{trilocanaḥ}: aśvatvagotvādīnāṃ sāmānyaviśeṣāṇāṃ svāśraye samavāyaḥ sāmānyaṃ sāmānyam ity abhidhānapratyayor nimittam iti | yady evaṃ vyaktiṣv apy ayam eva tathābhidhānapratyayahetus tu, kiṃ sāmānyasvīkārapramādena | na ca samavāyaḥ sambhavī |
	\pend
      

	  \pstart iheti buddheḥ samavāyasiddhir iheti dhīś ca dvayadarśanena | na ca kvacit tadviṣaye tv apekṣā svakalpanāmātramato 'bhyupāyaḥ || \edlabel{thakur75-64.15}\label{thakur75-64.15} etena seyaṃ pratyayānuvṛttir anuvṛttavastvanuyāyinī katham atyantabhedinīṣu vyaktiṣu vyāvṛttaviṣayapratyayabhāvānupātinīṣu bhavitum arhatīty ūhāpravartanam asya pratyākhyātam | jātiṣv eva parasparavyāvṛttatayā vyaktīyamānāsv anuvṛttapratyayena vyabhicārāt | \edlabel{thakur75-64.19}\label{thakur75-64.19} yat punar anena viparyaye bādhakam uktam, abhidhānapratyayānuvṛttiḥ kutaścin nivṛttya kvacid eva bhavantī nimittavatī, na cānyannimittam ityādi | tan na samyak | anuvṛttam anyatreṇāpy abhidhānapratyayānuvṛtter atadrūpaparāvṛttasvarūpaviśeṣād avaśyaṃ svīkārasya sādhitatvāt | tasmāt
	\pend
      
	    
	    \stanza[\smallbreak]
	tulye bhede yayā jātiḥ pratyāsattyā prasarpati |&kvacin nānyatra saivāstu śabdajñānanibandhanam ||\edtext{\textsuperscript{*}}{\lemma{*}\Bfootnote{PV I 162}}\&[\smallbreak]


	

	  \pstart yat punar atra Nyāyabhūṣaṇoktam: na hy evaṃ bhavati, yayā pratyāsattyā daṇḍasūtrādikaṃ prasarpati kvacin nānyatra saiva pratyāsattiḥ puruṣasphaṭikādiṣu daṇḍisūtritvādivyavahāranibandhanam astu, kiṃ daṇḍasūtrādineti | tad asaṅgatam | daṇḍasūtrayor hi puruṣasphaṭikapratyāsannyoḥ dṛṣṭayoḥ daṇḍisūtritvapratyayahetutvaṃ nāpalapyate | sāmānyaṃ tu svapne 'pi na dṛṣṭam | tad yadīdaṃ parikalpanīyaṃ tadā varaṃ pratyāsattir eva sāmānyapratyayahetuḥ parikalpyatām, kiṃ gurvyā parikalpanayety abhiprāyāparijñānāt |
	\pend
      

	  \pstart \edlabel{thakur75-65.1}\label{thakur75-65.1} athedaṃ jātiprasādhakam anumānam abhidhīyate | yad viśiṣṭajñānaṃ tadviśeṣaṇagrahaṇanāntarīyakam | yathā daṇḍijñānam | viśiṣṭajñānaṃ cedaṃ gaurayam ity arthataḥ kāryahetuḥ | viśeṣaṇānubhavakāryaṃ hi dṛṣṭānte viśiṣṭabuddhiḥ siddheti | atrānuyogaḥ | viśiṣṭabuddher bhinnaviśeṣaṇagrahaṇanāntarīyakatvaṃ vā sādhyaṃ viśeṣaṇamātrānubhavanāntarīyakatvaṃ vā |
	\pend
      

	  \pstart \edlabel{thakur75-65.6}\label{thakur75-65.6} prathamapakṣe pakṣasya pratyakṣabādhā sādhanāvadhānam anavakāśayati, vastugrāhiṇaḥ pratyakṣasyobhayapratibhāsābhāvāt | viśiṣṭabuddhitvaṃ ca sāmānyahetur anaikāntikaḥ, bhinnaviśeṣaṇagrahaṇam antareṇāpi darśanāt | yathā svarūpavān ghaṭaḥ, gotvaṃ sāmānyam iti vā |
	\pend
      

	  \pstart \edlabel{thakur75-65.10}\label{thakur75-65.10} dvitīyapakṣe tu siddhasādhanam | svarūpavān ghaṭa ityādivat gotvajātimān piṇḍa iti parikalpitaṃ bhedam upādāya viśeṣaṇaviśeṣyabhāvasyeṣṭatvād agovyāvṛttānubhavabhāvitvād gaurayam iti vyavahārasya | tad evaṃ na sāmānyasiddhiḥ | bādhakaṃ ca sāmānyaguṇakarmādyupādhicakrasya kevalavyaktigrāhakaṃ paṭupratyakṣaṃ dṛśyānulambho vā prasiddhaḥ |
	\pend
      

	  \pstart \edlabel{thakur75-65.15}\label{thakur75-65.15} tad evaṃ vidhir eva śabdārthaḥ | sa ca bāhyo 'rtho buddhyākāraś ca vivakṣitaḥ | tatra na buddhyākārasya tattvataḥ saṃvṛtyā vā vidhiniṣedhau, svasaṃvedanapratyakṣagamyatvāt | anadhyavasāyāc ca | nāpi tattvato bāhyasyāpi vidhiniṣedhau, tasya śābde pratyaye 'pratibhāsanāt | ata eva sarvadharmāṇāṃ tattvato 'nabhilāpyatvaṃ pratibhāsādhyavasāyābhāvāt | tasmād bāhyasyaiva sāṃvṛttau vidhiniṣedhau | anyathā saṃvyavahārahāniprasaṅgāt | tad evaṃ
	\pend
      
	    
	    \stanza[\smallbreak]
	\label{thakur75-65.21}\flagstanza{\tiny\textenglish{...-65.21}}nākārasya na bāhyasya tattvato vidhisādhanam |&bahir eva hi saṃvṛtyā samvṛtyāpi tu nākṛteḥ ||\edtext{\textsuperscript{*}}{\lemma{*}\Bfootnote{Corresponds to AP 229.3–4, SāSiŚā 443.13–14.}}\&[\smallbreak]


	

	  \pstart etena yad \persName{Dharmottaraḥ} āropitasya bāhyatvasya vidhiniṣedhāv ity alaukikam anāgamamatārkikīyaṃ kathayati, tad apy apahastitam | \edlabel{thakur75-65.26}\label{thakur75-65.26} nanv adhyavasāye yady adhyavaseyaṃ vastu na sphurati tadā tad adhyavasitam iti ko 'rthaḥ | apratibhāse 'pi pravṛttiviṣayīkṛtam iti yo 'rthaḥ | apratibhāsāviśeṣe viṣayāntaraparihāreṇa kathaṃ niyataviṣayā pravṛttir iti cet | ucyate | yady api viśvam agṛhītaṃ tathāpi vikalpasya niyatasāmagrīprasūtatvena niyatākāratayā, niyataśaktitvāt niyataiva jalādau pravṛttiḥ | dhūmasya parokṣāgnijñānajananavat | \edlabel{thakur75-66.1}\label{thakur75-66.1} niyataviṣayā hi bhāvāḥ pramāṇapariniṣṭhitasvabhāvā na śaktisāṃkaryaparyanuyogabhājaḥ | tasmāt tadadhyavasāyitvam ākāraviśeṣayogāt tatpravṛttijanakatvam | na ca sādṛśyād āropeṇa pravṛttiṃ brūmaḥ, yenākāre bāhyasya bāhye vākārasyāropadvāreṇa dūṣaṇāvakāśaḥ | kiṃ tarhi svavāsanāvipākavaśād upajāyamānaiva buddhir apaśyanty api bāhyaṃ bāhye pravṛttim ātanotīti viplutaiva | tad evam anyābhāvaviśiṣṭo vijātivyāvṛtto 'rtho vidhiḥ | sa eva cāpohaśabdavācyaḥ śabdānām arthaḥ pravṛttinivṛttiviṣayaś ceti sthitam | \edlabel{thakur75-66.8}\label{thakur75-66.8} atra prayogaḥ | yad vācakaṃ tat sarvam adhyavasitātadrūpaparāvṛttavastumātragocaram | yatheha kūpe jalam iti vacanam | vācakaṃ cedaṃ gavādiśabdarūpam iti svabhāvahetuḥ | nāyam asiddhaḥ | pūrvoktena nyāyena pāramārthikavācyavācakabhāvasyābhāve 'pi adhyavasāyakṛtasyaiva sarvavyavahāribhir avaśyaṃ svīkarttavyatvāt |anyathā sarvavyavahārocchedaprasaṅgāt | nāpi viruddhaḥ | sapakṣe bhāvāt | na cānaikāntikaḥ | tathā hi śabdānām adhyavasitavijātivyāvṛttavastumātraviṣayatvam anicchadbhiḥ paraiḥ paramārthato
	\pend
      

	  \pstart vācyaṃ svalakṣaṇam upādhir upādhiyogaḥ sopādhir astu yadi vā kṛtir astu buddhaḥ |
	\pend
      

	  \pstart gatyantarābhāvāt | aviṣayatve ca vācakatvāyogāt | tatra
	\pend
      

	  \pstart ādyantayor na samayaḥ phalaśaktihāner madhye 'py upādhivirahāt tritayena yuktaḥ || \edlabel{thakur75-66.19}\label{thakur75-66.19} tad evaṃ vācyāntarasyābhāvāt viṣayavattvalakṣaṇasya vyāpakasya nivṛttau vipakṣato nivarttamānaṃ vācakatvam adhyavasitabāhyaviṣayatvena vyāpyata iti vyāptisiddhiḥ |
	\pend
      

	  \pstart mahāpaṇḍitaratnakīrtipādaviracitam apohaprakaraṇaṃ samāptam || 
	\pend
      
	    
	    \endnumbering% ending numbering from div
	    \endgroup
	    
	  
	  
	% new div opening: depth here is 0
	
	    
	    \begingroup
	    \beginnumbering% beginning numbering from div depth=0
	    
	  
\chapter[{Kṣaṇabhaṅgasiddhiḥ Anvayātmikā}]{Kṣaṇabhaṅgasiddhiḥ Anvayātmikā}\label{Kṣaṇabhaṅgasiddhiḥ_Anvayātmikā}

	  \pstart namas tārāyai ||
	\pend
      
	    
	    \stanza[\smallbreak]
	\label{ratnakīrtinibandhāvali__36r1OI6NR531EDRWPU9XOG3N62S}\flagstanza{\tiny\textenglish{...G3N62S}}ākṣiptavyatirekā yā vyāptir anvayarūpiṇī |&sādharmyavati dṛṣṭānte sattvahetor ihocyate ||\&[\smallbreak]


	

	  \pstart yat sat tat kṣaṇikam, yathā ghaṭaḥ, santaś cāmī vivādāspadībhūtāḥ padārthā iti |
	\pend
      

	  \pstart hetoḥ parokṣārtha pratipādakatvaṃ hetvābhāsatvaśaṅkānirākaraṇam antareṇa na śakyate pratipādayitum | hetvābhāsāś ca asiddhaviruddhānaikāntikabhedena trividhāḥ |
	\pend
      

	  \pstart tatra na tāvad ayam asiddho hetuḥ |
	\pend
      

	  \pstart yadi nāma darśane darśane nānāprakāraṃ sattvalakṣaṇam uktam āste, arthakriyākāritvaṃ, sattāsamavāyaḥ, svarūpasattvam, utpādavyayadhrauvyayogitvaṃ, pramāṇaviṣayatvaṃ, sad upalambhaka pramāṇagocaratvaṃ, vyapadeśaviṣayatvam ityādi, tathāpi kim anenāprastutenedānīm eva niṣṭaṅkitena | yad eva hi pramāṇato nirūpyamāṇaṃ padārthānāṃ sattvam upapannaṃ bhaviṣyati tad eva vayam api svīkariṣyāmaḥ | 
	\pend
      

	  \pstart kevalaṃ tad etad arthakriyākāritvaṃ sarvajanaprasiddham āste
	\pend
      

	  \pstart tat khalv atra sattvaśabdenābhisandhāya sādhanatvenopāttam | tac ca
	\pend
      

	  \pstart yathāyogaṃ pratyakṣānumānapramāṇaprasiddhasadbhāveṣu bhāveṣu
	\pend
      

	  \pstart pakṣīkṛteṣu pratyakṣādinā pramāṇena pratītam iti na
	\pend
      

	  \pstart svarūpeṇāśrayadvāreṇa vāsiddhi sambhāvanāpi ||
	\pend
      

	  \pstart nāpi viruddhatā, sapakṣīkṛte ghaṭe sadbhāvāt |
	\pend
      

	  \pstart nanu katham asya sapakṣatvam, pakṣavad atrāpi kṣaṇabhaṅgāsiddheḥ | na hy asya pratyakṣataḥ kṣaṇabhaṅgasiddhiḥ, tathātvenāniścayāt | nāpi sattvānumānataḥ, punarnidarśanāntarāpekṣāyām anavasthāprasaṅgāt | na cānyad anumānam asti | sambhave vā tenaiva pakṣe 'pi kṣaṇabhaṅgasiddher alaṃ sattvānumāneneti cet |
	\pend
      

	  \pstart ucyate | anumānāntaram eva prasaṅgaprasaṅgaviparyayātmakaṃ ghaṭe kṣaṇabhaṅgaprasādhakaṃ pramāṇāntaram asti |
	\pend
      

	  \pstart tathā hi ghaṭo vartamānakṣaṇe tāvad ekām arthakriyāṃ karoti | atītānāgatakṣaṇayor api kiṃ tām evārthakriyāṃ kuryāt, anyāṃ vā, na vā kām api kriyām iti trayaḥ pakṣāḥ |
	\pend
      

	  \pstart nātra prathamaḥ pakṣo yuktaḥ, kṛtasya karaṇāyogāt |
	\pend
      

	  \pstart atha dvitīyo 'bhyupagamyate, tad idam atra vicāryatām | yadā ghaṭo vartamānakṣaṇabhāvi kāryaṃ karoti tadā kim atītānāgatakṣaṇabhāviny api kārye śakto 'śakto vā |
	\pend
      

	  \pstart yadi śaktas tadā vartamānakṣaṇabhāvikāryavad atītānāgatakṣaṇabhāvy api kāryaṃ tadaiva kuryāt | tatrāpi śaktatvāt | śaktasya ca kṣepāyogāt, anyathā varttamānakṣaṇabhāvino 'pi kāryasyākaraṇaprasaṅgāt pūrvāparakālayor api śaktatvenāviśeṣāt | samarthasya ca sahakāryapekṣāyā ayogāt | 
	\pend
      

	  \pstart athāśaktaḥ, tadaikatra kārye śaktāśaktatvaviruddhadharmādhyāsāt kṣaṇavidhvaṅso ghaṭasya durvāraprasaraḥ syāt |
	\pend
      

	  \pstart nāpi tṛtīyaḥ pakṣaḥ saṅgacchate , śaktasvabhāvānuvṛtter eva | yadā hi śaktasya padārthasya vilambo 'py asahyas tadā dūrotsāritam akaraṇam | anyathā vārtamānikasyāpi kāryasyākaraṇaṃ syād ity uktam | 
	\pend
      

	  \pstart tasmād yad yadā yajjananavyavahārapātraṃ tat tadā tat kuryāt | akurvac ca na jananavyavahārabhājanam | tad evam ekatra kārye samarthetarasvabhāvatayā pratikṣaṇaṃ bhedād ghaṭasya sapakṣatvam akṣatam |
	\pend
      

	  \pstart atra prayogaḥ | yad yadā yajjananavyavahārayogyaṃ tat tadā taj janayaty eva | yathā 'ntyā kāraṇasāmagrī svakāryam | atītānāgatakṣaṇabhāvikāryajananavyavahārayogyaś cāyaṃ ghaṭo vartamānakṣaṇabhāvikāryakaraṇakāle sakalakriyātikramakāle 'pīti svabhāvahetuprasaṅgaḥ |
	\pend
      

	  \pstart asya ca dvitīyādikṣaṇabhāvikāryakaraṇavyavahāragocaratvasya prasaṅgasādhanasya vārtamānikakāryakaraṇakāle sakalakriyātikramakāle ca ghaṭe dharmiṇi parābhyupagamamātrataḥ siddhatvād asiddhis tāvad asambhavinī |
	\pend
      

	  \pstart nāpi viruddhatā, sapakṣe 'ntya kāraṇasāmagryāṃ sadbhāvasambhavāt| 
	\pend
      

	  \pstart nanv ayaṃ sādhāraṇānaikāntiko hetuḥ | sākṣādajanake 'pi kuśūlādyavasthitabījādau vipakṣe samarthavyavahāragocaratvasya sādhanasya darśanād iti cet | 
	\pend
      

	  \pstart na | dvividho hi samarthavyavahāraḥ pāramārthika aupacārikaś ca | tatra yat pāramārthikaṃ jananaprayuktaṃ jananavyavahāragocaratvaṃ tad iha sādhanatvenopāttam | tasya ca kuśūlādyavasthitabījādau kāraṇakāraṇatvād aupacārikajananavyavahāraviṣayabhūte sambhavābhāvāt kutaḥ sādhāraṇānaikāntikatā | 
	\pend
      

	  \pstart na cāsya sandigdhavyatirekitā, viparyaye bādhakapramāṇasadbhāvat | 
	\pend
      

	  \pstart tathā hīdaṃ jananavyavahāragocaratvaṃ niyataviṣayatvena vyāptam iti sarvajanānubhavaprasiddham | na cedaṃ nirnimittam, deśakālasvabhāvaniyamābhāvaprasaṅgāt | na ca jananād anyan nimittam upalabhyate, tadanvayavyatirekānuvidhānadarśanāt | yadi ca jananam antareṇāpi jananavyavahāragocaratvaṃ syāt tadā sarvasya sarvatra jananavyavahāra ity aniyamaḥ syāt | niyataś cāyaṃ pratītaḥ | tato jananābhāve vipakṣe niyataviṣayatvasya vyāpakasya nivṛttau nivartamānaṃ jananavyavahāragocaratvaṃ janana eva viśrāmyatīti vyāptisiddher anavadyo hetuḥ |
	\pend
      

	  \pstart na caiṣa ghaṭo varttamānakāryakaraṇakṣaṇe sakalakriyātikramakāle cātītānāgatakṣaṇabhāvikāryaṃ janayati | tato na jananavyavahārayogyaḥ, sarvaḥ prasaṅgaḥ prasaṅgaviparyayaniṣṭha iti nyāyāt |
	\pend
      

	  \pstart atrāpi prayogaḥ | yad yadā yan na karoti na tat tadā tatra samarthavyavahārayogyam | yathā śālyaṅkuram akurvan kodravaḥ śālyaṅkure | na karoti caiṣa ghaṭo vartamānakṣaṇabhāvikāryakaraṇakāle sakalakriyātikramakāle cātītānāgatakṣaṇabhāvikāryam iti vyāpakānupalabdhir bhinatti samarthakṣaṇād asamarthakṣaṇam |
	\pend
      

	  \pstart atrāpy asiddhir nāsti, vartamānakṣaṇabhāvikāryakaraṇakāle sakalakriyātikramakāle cātītānāgatakṣaṇabhāvikāryakaraṇasyāyogāt |
	\pend
      

	  \pstart nāpi virodhaḥ, sapakṣe bhāvāt |
	\pend
      

	  \pstart na cānaikāntikatā, pūrvoktena nyāyena samarthavyavahāragocaratvajanakatvayor vidhibhūtayoḥ sarvopasaṃhāravatyā vyāpteḥ prasādhanāt ||
	\pend
      

	  \pstart yat punar atroktam yad yadā yan na karoti na tat tadā tatra samartham ity atra kaḥ karotyarthaḥ | kiṃ kāraṇatvam | uta kāryotpādānuguṇasahakārisākalyam | ahosvit kāryāvyabhicāraḥ | kāryasambandho veti | tatra kāraṇatvam eva karotyarthaḥ | tataḥ pakṣāntarabhāvino doṣā anabhyupagamapratihatāḥ |
	\pend
      

	  \pstart na cātra pakṣe kāraṇatvasāmarthyayoḥ paryāyatvena vyāpakānupalambhasya sādhyāviśiṣṭatvam abhidhātum ucitam, samarthavyavahāragocaratvābhāvasya sādhyatvāt | kāraṇatvasamarthavyavahāragocaratvayoś ca vṛkṣaśiṃśapayor iva vyāvṛttibhedo 'stīty anavasara evaivaṃvidhasya kṣudrapralāpasya |
	\pend
      

	  \pstart tad evaṃ prasaṅgaprasaṅgaviparyayahetudvayabalato ghaṭe dṛṣṭānte kṣaṇabhaṅgaḥ siddhaḥ | tat kathaṃ sattvād anyad anumānam dṛṣṭānte kṣaṇabhaṅgasādhakaṃ nāstīty ucyate | na caivaṃ sattvahetor vaiyarthyam, dṛṣṭāntamātra eva prasaṅgaprasaṅgaviparyayābhyāṃ kṣaṇabhaṅgaprasādhanāt ||
	\pend
      

	  \pstart nanv ābhyām eva pakṣe 'pi kṣaṇabhaṅgasiddhir astv iti cet | 
	\pend
      

	  \pstart astu, ko doṣaḥ | yo hi pratipattā prativastu yad yadā yajjananavyavahārayojyaṃ tat tadā taj janayatītyādikam upanyasitum analasas tasya tata eva kṣaṇabhaṅgasiddhiḥ | yas tu prativastu tannyāyopanyāsaprayāsabhīruḥ sa khalv ekatra dharmiṇi yad yadā yajjananavyavahārayogyaṃ tat tadā taj janayatītyādinyāyena sattvamātram asthairyavyāptam avadhārya sattvād evānyatra kṣaṇikatvam avagacchayatīi, katham apramatto vaiyarthyam asyācakṣīta |
	\pend
      

	  \pstart tad evam ekakāryakāriṇo ghaṭasya dvitīyādikṣaṇabhāvikāryāpekṣayā samarthetarasvabhāvaviruddhadharmādhyāsād bheda eveti kṣaṇabhaṅgitayā sapakṣatām āvahati ghaṭe sattvahetur upalabhyamāno na viruddhaḥ | 
	\pend
      

	  \pstart na cāyam anaikāntikaḥ, atraiva sādharmyavati dṛṣṭānte sarvopasaṃhāravatyā vyāpteḥ prasādhanāt |
	\pend
      

	  \pstart nanu viparyayabādhakapramāṇabalād vyāptisiddhiḥ | tasya copanyāsavārtāpi nāsti | tat kathaṃ vyāptiḥ prasādhiteti cet | 
	\pend
      

	  \pstart tad etat taralabuddhivilasitam | tathā hi uktam etad vartamānakṣaṇabhāvikāryakaraṇakāle 'tītānāgatakṣaṇabhāvikārye 'pi ghaṭasya śaktisambhave tadānīm eva tatkaraṇam , akaraṇe ca śaktāśaktasvabhāvatayā pratikṣaṇaṃ bheda iti kṣaṇikatvena vyāptaiva sā arthakriyāśaktiḥ || 
	\pend
      

	  \pstart nanv evam anvayamātram astu | vipakṣāt punar ekāntena vyāvṛttir iti kuto labhyata iti cet | 
	\pend
      

	  \pstart vyāptisiddher eva |
	\pend
      

	  \pstart vyatirekasandehe vyāptisiddhir eva katham iti cet | 
	\pend
      

	  \pstart na | dvividhā hi vyāptisiddhiḥ | anvayarūpā ca kartṛdharmaḥ sādhanadharmavati dharmiṇi sādhyadharmasyāvaśyambhāvo yaḥ, vyatirekarūpā ca karmadharmaḥ sādhyābhāve sādhanasyāvaśyamabhāvo yaḥ | enayoś caikatarapratītir niyamena dvītyapratītim ākṣipati, anyathaikasyā evāsiddheḥ |
	\pend
      

	  \pstart tasmād yathā viparyaye bādhakapramāṇabalāt niyamavati vyatireke siddhe 'nvayaviṣayaḥ saṃśayaḥ pūrvaṃ sthito 'pi paścāt parigalati tato 'nvayaprasādhārthaṃ na pṛthak sādhanam ucyate tathā prasaṅgatadviparyayahetudvayabalato niyamavaty anvaye siddhe vyatirekaviṣaye pūrvaṃ sthito 'pi sandehaḥ paścāt parigalaty eva | na ca vyatirekaprasādhakam anyat pramāṇaṃ vaktavyam | tataś ca sādhyābhāve sādhanasyaikāntiko vyatirekaḥ, sādhane sati
	\pend
      

	  \pstart sādhyasyāvaśyam anvayo veti na kaścid arthabhedaḥ |
	\pend
      

	  \pstart tad evaṃ viparyayabādhakapramāṇam antareṇāpi prasaṅgaprasaṅgaviparyayahetudvayabalād anvayarūpavyāptisiddhau sattvahetor anaikāntikatvasyābhāvād ataḥ sādhanāt kṣaṇabhaṅgasiddhir anavadyeti ||
	\pend
      

	  \pstart nanu ca sādhanam idam asiddham | na hi kāraṇabuddhyā kāryaṃ gṛhyate, tasya bhāvitvāt | na ca kāryabuddhyā \leavevmode\ledsidenote{\textenglish{\cite[39b]{RNAms}}} kāraṇam, tasyātītatvāt | na ca vartamānagrāhiṇā jñānenātītānāgatayor grahaṇaṃ atiprasaṅgāt |
	\pend
      

	  \pstart na ca pūrvāparayoḥ kālayor ekaḥ pratisandhātā asti, kṣaṇabhaṅgabhaṅgaprasaṅgāt | kāraṇābhāve tu kāryābhāvapratītiḥ svasaṃvedanavādino manorathasyāpy aviṣayaḥ |
	\pend
      

	  \pstart nanu ca pūrvottarakālayoḥ saṃvittī, tābhyāṃ vāsanā, tayā ca \edlabel{ratnakīrtinibandhāvali__36r1PF7IMUVC4DF8QV4LEE5R9OG}\label{ratnakīrtinibandhāvali__36r1PF7IMUVC4DF8QV4LEE5R9OG}\edtext{}{\lemma{hetu}\xxref{ratnakīrtinibandhāvali__36r1PF7IMUVC4DF8QV4LEE5R9OG}{ratnakīrtinibandhāvali__36r1PF7IMUU5ZOUZZQDXUR7YR87}\Afootnote{hetū}}hetu\edlabel{ratnakīrtinibandhāvali__36r1PF7IMUU5ZOUZZQDXUR7YR87}\label{ratnakīrtinibandhāvali__36r1PF7IMUU5ZOUZZQDXUR7YR87}phalāvasāyī vikalpa iti cet tad ayuktam | sa hi vikalpo gṛhītānusandhāyako 'tadrūpasamāropako vā |
	\pend
      

	  \pstart na prathamaḥ pakṣaḥ | ekasya pratisandhātur abhāve pūrvāparagrahaṇayor ayogāt, vikalpavāsanāyā evābhāvāt |
	\pend
      

	  \pstart nāpi dvitīyaḥ | marīcikāyām api jalavijñānasya prāmāṇyaprasaṅgāt |
	\pend
      

	  \pstart tad evam anvayavyatirekayor apratipatter arthakriyālakṣaṇaṃ sattvam asiddham iti ||
	\pend
      

	  \pstart kiṃ ca prakārāntarād apīdaṃ sādhanam asiddham | tathā hi bījādīnāṃ sāmarthyaṃ bījādijñānāt tatkāryād aṅkurāder vā niścetavyam |
	\pend
      

	  \pstart kāryatvaṃ ca vastutvasiddhau sidhyati | vastutvaṃ ca kāryāntarāt | kāryāntarasyāpi kāryatvaṃ vastutvasiddhau | tadvastutvaṃ ca tadaparakāryāntarād ity anavasthā |
	\pend
      

	  \pstart athānavasthābhayāt paryante kāryāntaraṃ nāpekṣate tadā tenaiva pūrveṣām asattvaprasaṅgān naikasyāpy arthakriyāsāmarthyaṃ sidhyati |
	\pend
      

	  \pstart nanu kāryatvasattvayor bhinnavyāvṛttikatvāt \edlabel{ratnakīrtinibandhāvali__36r1PF7IMUT19YJQJP8HLZNIM4Y}\label{ratnakīrtinibandhāvali__36r1PF7IMUT19YJQJP8HLZNIM4Y}\edtext{}{\lemma{sattvāsiddhāv}\xxref{ratnakīrtinibandhāvali__36r1PF7IMUT19YJQJP8HLZNIM4Y}{ratnakīrtinibandhāvali__36r1PF7IMURRIVO388ZTVNIPXXN}\Afootnote{satvāsiddhĀv°; sattāsiddhāv\textenglish{---\textsc{Note} tva and tta are very similar in this script. Corroboration will need more examples.}}}sattvāsiddhāv\edlabel{ratnakīrtinibandhāvali__36r1PF7IMURRIVO388ZTVNIPXXN}\label{ratnakīrtinibandhāvali__36r1PF7IMURRIVO388ZTVNIPXXN} api kāryatvasiddhau kā kṣatir iti cet |
	\pend
      

	  \pstart tad asaṅgatam | saty api kāryatvasattvayor vyāvṛttibhede \edlabel{ratnakīrtinibandhāvali__36r1PF7IMT62ZHYE66V9OBOAHB0}\label{ratnakīrtinibandhāvali__36r1PF7IMT62ZHYE66V9OBOAHB0}\edtext{}{\lemma{sattvāsiddhāu}\xxref{ratnakīrtinibandhāvali__36r1PF7IMT62ZHYE66V9OBOAHB0}{ratnakīrtinibandhāvali__36r1PF7IMT4VU9YSDDOV432L8BV}\Afootnote{satvāsiddhau; sattāsiddhāu}}sattvāsiddhāu\edlabel{ratnakīrtinibandhāvali__36r1PF7IMT4VU9YSDDOV432L8BV}\label{ratnakīrtinibandhāvali__36r1PF7IMT4VU9YSDDOV432L8BV} kutaḥ kāryatvasiddhiḥ | kāryatvaṃ hy abhūtvābhāvitvaṃ | bhavanaṃ ca sattā | sattā ca saugatānāṃ sāmarthyam eva | tataś ca sāmarthyasandehe bhavatīty eva vaktum aśakyam | katham abhūtvābhāvitvaṃ kāryatvaṃ setsyati |
	\pend
      

	  \pstart apekṣitaparavyāpāratvaṃ kāryatvam ity api nāsato dharmaḥ | sattvaṃ ca sāmarthyam | tac ca sandigdham iti kutaḥ kāryatvasiddhiḥ | tadasiddhau pūrvasya sāmarthyaṃ na sidhyatīti sandigdhāsiddho hetuḥ ||
	\pend
      

	  \pstart tathā viruddho 'py ayam | tathā hi kṣaṇikatve sati na tāvad ajātasyānanvayaniruddhasya vā kāryārambhakatvaṃ sambhavati |  na ca niṣpannasya tāvān kṣaṇo 'sti yam upādāya kasmaicit kāryāya vyāpāryeta | ataḥ kṣaṇikapakṣa evārthakriyānupapatter viruddhatā |
	\pend
      

	  \pstart athavā vikalpena yad upanīyate tat sarvam avastu | tataś ca vastvātmake kṣaṇikatve sādhye 'vastūpasthāpayann anumānavikalpo viruddhaḥ |
	\pend
      

	  \pstart yadvā sarvasyaiva hetoḥ kṣaṇikatve sādhye viruddhatvaṃ | deśakālāntarānanugame sādhyasādhanabhāvābhāvāt | anugame ca nānākālam ekam akṣaṇikaṃ kṣaṇikatvena virudhyata iti ||
	\pend
      

	  \pstart anaikāntiko 'py ayam, sattvasthairyayor virodhābhāvād iti |
	\pend
      

	  \pstart atrocyate | yat tāvad uktaṃ sāmarthyaṃ na pratīyata iti, tat kiṃ sarvathaiva na pratīyate kṣaṇabhaṅgapakṣe vā |
	\pend
      

	  \pstart prathamapakṣe sakalakārakajñāpakahetucakrocchedān mukhaspandanamātrasyāpy akaraṇaprasaṅgaḥ | anyathā yenaiva vacanena sāmarthyaṃ nāstīti pratipādyate tasyaiva tatpratipādanasāmarthyam avyāhatam āyātam | tasmāt paramapuruṣārthasamīhayā vastutattvanirūpaṇapravṛttasya śaktisvīkārapūrvakaiva pravṛttiḥ | tadasvīkāre tu na kaścit kvacit pravarteteti nirīhaṃ jagaj jāyeta |
	\pend
      

	  \pstart atha dvitīyaḥ pakṣaḥ, tadāsti tāvat sāmarthyapratītiḥ | sā ca kṣaṇikatve yadi nopapadyate tadā viruddhaṃ vaktum ucitam | asiddham iti tu nyāyabhūṣaṇīyaḥ prāyo vilāpaḥ |
	\pend
      

	  \pstart na ca saty api kṣaṇikatve sāmarthyapratītivyāghātaḥ | tathā hi kāraṇagrāhijñānopādeyabhūtena kāryagrāhiṇā jñānena tadarpitasaṃskāragarbheṇa asya bhāve asya bhāva ity anvayaniścayo janyate | tathā kāraṇāpekṣayā bhūtalakaivalyagrāhijñānopādeyabhūtena kāryāpekṣayā bhūtalakaivalyagrāhiṇā jñānena tadarpitasaṃskāragarbheṇa asyābhāve asyābhāva iti vyatirekaniścayo janyate |
	\pend
      

	  \pstart yad āhur guravaḥ
	\pend
      
	    
	    \stanza[\smallbreak]
	ekāvasāyasamanantarajātam anyavijñānam anvayavimarśam upādadhāti |&evaṃ tadekavirahānubhavodbhavānyavyāvṛttidhīḥ prathayati vyatirekabuddhim ||\&[\smallbreak]


	

	  \pstart evaṃ sati gṛhītānusandhāyaka evāyaṃ vikalpaḥ | upādānopādeyabhūtakramipratyakṣadvayagṛhītānusandhānāt |
	\pend
      

	  \pstart yad āhālaṅkāraḥ
	\pend
      
	    
	    \stanza[\smallbreak]
	\label{ratnakīrtinibandhāvali__lg__yadi_nāmaikam}\flagstanza{\tiny\textenglish{...maikam}}yadi nāmaikam adhyakṣaṃ na pūrvāparavittimat |&adhyakṣadvayasadbhāve prākparāvedanaṃ katham ||\edtext{}{\lemma{||}\Bfootnote{(PVA)}}\&[\smallbreak]


	

	  \pstart iti ||
	\pend
      

	  \pstart nāpi dvitīyo 'siddhaprabhedaḥ | sāmarthyaṃ hi sattvam iti saugatānāṃ sthitir eṣā | na caitatprasādhanārtham asmākam idānīm eva prārambhaḥ | kiṃ tu yatra pramāṇapratīte bījādau vastubhūte dharmiṇi pramāṇapratītaṃ sāmarthyaṃ tatra kṣaṇabhaṅgaprasādhanāya |\edlabel{thakur75-72.26}\label{thakur75-72.26} tataś cāṅkurādīnāṃ kāryādarśanād āhatya sāmarthyasandehe 'pi paṭupratyakṣaprasiddham sanmātratvam avadhāryam eva | anyathā na kvacid api vastumātrasyāpi pratipattiḥ syāt | \edlabel{thakur75-72.29}\label{thakur75-72.29} tasmāc chāstrīyasattvalakṣaṇasandehe 'pi paṭupratyakṣabalāvalambitavastubhāve 'ṅkurādau kāryatvam upalabhyamānaṃ bījādeḥ sāmarthyam upasthāpayatīti nāsiddhidoṣāvakāśaḥ ||
	\pend
      

	  \pstart nāpi kṣaṇikatve sāmarthyakṣatiḥ, yato viruddhatā syāt, kṣaṇikatvaniyataprāgbhāvitvalakṣaṇakāraṇatvayor virodhābhāvāt, kṣaṇamātrasthāyiny api sāmarthyasambhavād iti nādimo virodhaḥ |\edlabel{thakur75-73.5}\label{thakur75-73.5} nāpi dvitīyo virodhaprabhedaḥ | avastuno vastuno vā svākārasya grāhyatve 'pi adhyavaseyavastvapekṣayaiva sarvatra prāmāṇyapratipādanāt vastusvabhāvasyaiva kṣaṇikatvasya siddhir iti kva virodhaḥ |
	\pend
      

	  \pstart yac ca gṛhyate yac cādhyavasīyate te dve 'py anyanivṛttī na vastunī svalakṣaṇāvagāhitve 'bhilāpasaṃsargānupapatter iti cet |\edtext{}{\edlabel{RNA-n-0}\lemma{|}\Bfootnote{Cf. \ref{note-2objects-neither-real}.}} \edlabel{thakur75-73.9}\label{thakur75-73.9} na | adhyavasāyasvarūpāparijñānāt | agṛhīte 'pi vastuni [ {\corr mānasādi}]pravṛttikārakatvaṃ vikalpasyādhyavasāyitvam | apratibhāse 'pi pravṛttiviṣayīkṛtatvam adhyavaseyatvam | etac cādhyavaseyatvaṃ svalakṣaṇasyaiva yujyate, nānyasya, arthakriyārthitvād arthipravṛtteḥ | \edlabel{thakur75-73.12}\label{thakur75-73.12} evaṃ cādhyavasāye svalakṣaṇasyāsphuraṇam eva | na ca tasyāsphuraṇe 'pi sarvatrāviśeṣeṇa pravṛttyākṣepaprasaṅgaḥ, pratiniyatasāmagrīprasūtāt pratiniyatasvākārāt pratiniyataśaktiyogāt, pratiniyata evātadrūpaparāvṛtte 'pratīte 'pi pravṛttisāmarthyadarśanāt | yathā sarvasyāsattve 'pi bījād aṅkurasyaivotpattiḥ, dṛṣṭasya niyatahetuphalabhāvasya pratikṣeptum aśakyatvāt | paraṃ bāhyenārthena sati pratibandhe prāmāṇyam | anyathā tv aprāmāṇyam iti viśeṣaḥ ||
	\pend
      

	  \pstart tathā tṛtīyo 'pi pakṣaḥ prayāsaphalaḥ | nānākālasyaikasya vastuno vastuto 'sambhave 'pi sarvadeśakālavartinor atadrūpaparāvṛttayor eva sādhyasādhanayoḥ pratyakṣeṇa vyāptigrahaṇāt |\edlabel{thakur75-73.20}\label{thakur75-73.20} dvividho hi pratyakṣasya viṣayaḥ, grāhyo 'dhyavaseyaś ca | \edlabel{ratnakīrtinibandhāvali__36r1OKKLAM78U8UT7RZ1GZQXEDK}\label{ratnakīrtinibandhāvali__36r1OKKLAM78U8UT7RZ1GZQXEDK}\edtext{}{\lemma{sakalātadrūpaparāvṛttaṃ vastumātraṃ}\xxref{ratnakīrtinibandhāvali__36r1OKKLAM78U8UT7RZ1GZQXEDK}{ratnakīrtinibandhāvali__36r1OKKLAOUYW8OHNORUBKXVNLH}\Afootnote{\label{ratnakīrtinibandhāvali__36r1OKKLCHAUV99B3CBNC3D4VJL}sakalātadrūpaparāvṛttaṃ vastumātraṃ \cite{} ; sakalātadrūpaparāvṛttavastumātraṃ  {\rmlatinfont [App type: emendation]}}}sakalātadrūpaparāvṛttaṃ vastumātraṃ\edlabel{ratnakīrtinibandhāvali__36r1OKKLAOUYW8OHNORUBKXVNLH}\label{ratnakīrtinibandhāvali__36r1OKKLAOUYW8OHNORUBKXVNLH} sākṣād asphuraṇāt pratyakṣasya grāhyo viṣayo mā bhūt | tadekadeśagrahaṇe tu tanmātrayor vyāptiniścāyakavikalpajananād adhyavaseyo viṣayo bhavaty eva | kṣaṇagrahaṇe santānaniścayavat, rūpamātragrahaṇe rūparasagandhasparśātmakaghaṭaniścayavac ca | anyathā sarvānumānocchedaprasaṅgāt ||
	\pend
      

	  \pstart tathā hi vyāptigrahaḥ sāmānyayoḥ, viśeṣayoḥ, sāmānyaviśiṣṭaviśeṣayoḥ viśeṣaviśiṣṭasāmānyayor veti vikalpāḥ |
	\pend
      

	  \pstart nādyo vikalpaḥ, sāmānyasya bādhyatvāt | abādhyatve 'py adṛśyatvāt | dṛśyatve 'pi puruṣārthānupayogitayā tasyānumeyatvāyogāt | nāpy anumitāt sāmānyād viśeṣānumānam | sāmānyasarvaviśeṣayor vakṣyamāṇanyāyena pratibandhapratipatter ayogāt |
	\pend
      

	  \pstart nāpi dvitīyaḥ | viśeṣasyānanugāmitvāt |
	\pend
      

	  \pstart antime tu vikalpadvaye sāmānyādhāratayā dṛṣṭa eva viśeṣaḥ sāmānyasya viśeṣyo viśeṣaṇaṃ vā kartavyaḥ | adṛṣṭa eva vā deśakālāntaravartī | yadvā dṛṣṭādṛṣṭātmako atadrūpaparāvṛttaḥ sarvo viśeṣaḥ |
	\pend
      

	  \pstart na prathamaḥ pakṣo 'nanugāmitvāt | nāpi dvitīyaḥ, adṛṣṭatvāt | na ca tṛtīyaḥ, prastutaikaviśeṣadarśane 'pi deśakālāntaravartināṃ viśeṣāṇām adarśanāt |
	\pend
      

	  \pstart atha teṣāṃ sarveṣām eva viśeṣāṇāṃ sadṛśatvāt sadṛśasāmagrīprasūtatvāt sadṛśakāryakāritvād iti pratyāsattyā ekaviśeṣagrāhakaṃ pratyakṣam atadrūpaparāvṛttamātre niścayaṃ janayad atadrūpaparāvṛttaviśeṣamātrasya vyavasthāpakam |\edlabel{thakur75-74.9}\label{thakur75-74.9} yathaikasāmagrīpratibaddharūpamātragrāhakaṃ pratyakṣaṃ ghaṭe niścayaṃ janayad ghaṭagrāhakaṃ vyavasthāpyate | anyathā ghaṭo 'pi ghaṭasantāno 'pi pratyakṣato na sidhyet, sarvātmanā grahaṇābhāvāt | \edlabel{thakur75-74.12}\label{thakur75-74.12} tadekadeśagrahaṇaṃ tv atadrūpaparāvṛtte 'py aviśiṣṭam | yady evam anenaiva krameṇa sarvasya viśeṣasya viśeṣaṇaviśeṣyabhāvavad vyāptipratipattir apy astu | \edlabel{thakur75-74.13}\label{thakur75-74.13} tat kimarthaṃ nānākālam ekam akṣaṇikam abhyupagantavyaṃ, yena kṣaṇikatvasādhanasya viruddhatvaṃ syād iti na kaścid virodhaprabhedaprasaṅgaḥ ||
	\pend
      

	  \pstart na cāyam anaikāntiko 'pi hetuḥ, pūrvoktakrameṇa sādharmyadṛṣṭānte prasaṅgaviparyayahetubhyām anvayarūpavyāpteḥ prasādhanāt |\edlabel{thakur75-74.17}\label{thakur75-74.17} nanu yadi prasaṅgaviparyayahetudvayabalato ghaṭe dṛṣṭānte kṣaṇabhaṅgaḥ sidhyet tadā sattvasya niyamena kṣaṇikatvena vyāptisiddher anaikāntikatvaṃ na syād iti yuktam | kevalam idam evāsambhavi | tathā hi śakto 'pi ghaṭaḥ krami sahakāryapekṣayā kramikāryaṃ kariṣyati |
	\pend
      

	  \pstart na caitad vaktavyam, samartho 'rthaḥ svarūpeṇa karoti, svarūpaṃ ca sarvadāstīty anupakāriṇi sahakāriṇy apekṣā na yujyata iti | saty api svarūpeṇa kārakatve sāmarthyābhāvāt kathaṃ karotu | sahakārisākalyaṃ hi sāmarthyam, tadvaikalyaṃ cāsāmarthyam | na ca tayor āvirbhāvatirobhāvābhyāṃ tadvataḥ kācit kṣatiḥ, tasya tābhyām anyatvāt | tasmād arthaḥ samartho 'pi syāt, na ca karotīti sandigdhavyatirekaḥ prasaṅgahetuḥ ||
	\pend
      

	  \pstart atrocyate | bhavatu tāvat sahakārisākalyam eva sāmarthyam | tathāpi so 'pi tāvad bhāvaḥ svarūpeṇa kārakaḥ |\edlabel{thakur75-74.28}\label{thakur75-74.28} tasya ca yādṛśaś caramakṣaṇe 'kṣepakriyādharmā svabhāvas tādṛśa eva cet | \edlabel{thakur75-74.29}\label{thakur75-74.29} prathamakṣaṇe tadā tadāpi prasahya kurvāṇo brahmaṇāpy anivāryaḥ | na ca so 'py akṣepakriyādharmā svabhāvaḥ sākalye sati jāto bhāvād bhinna evābhidhātuṃ śakyaḥ, bhāvasyākartṛtvaprasaṅgāt | evaṃ yāvad yāvad dharmāntaraparikalpas tāvat tāvad udāsīno bhāvaḥ | \edlabel{thakur75-75.2}\label{thakur75-75.2} tasmād yadrūpam ādāya svarūpeṇāpi janayatīty ucyate tasya prāg api bhāve katham ajaniḥ kadācit | akṣepakriyāpratyanīkasvabhāvasya vā prācyasya paścād anuvṛttau kathaṃ kadācid api kāryasambhavaḥ ||
	\pend
      

	  \pstart nanu yadi sa evaikaḥ kartā syād yuktam etat | kiṃtu sāmagrī janikā | tataḥ sahakāryantaravirahavelāyāṃ balīyaso 'pi na kāryaprasava iti kim atra viruddham | na hi bhāvaḥ svarūpeṇa karotīti svarūpeṇaiva karoti , sahakārisahitād eva tataḥ kāryotpattidarśanāt | tasmād vyāptivat kāryakāraṇabhāvo 'py ekatrānyayogavyavacchedenānyatrāyogavyavacchedenāvaboddhavyaḥ, tathaiva laukikaparīkṣakāṇāṃ saṃpratipatter iti ||
	\pend
      

	  \pstart atrocyate | yadā militāḥ santaḥ kāryaṃ kurvate tadaikārthakaraṇalakṣaṇaṃ sahakāritvam eṣām astu | ko niṣeddhā | militair eva tu tatkāryaṃ kartavyam iti kuto labhyate | pūrvāparayor ekasvabhāvatvād bhāvasya sarvadā jananājananayor anyataraniyama prasaṅgasya durvāratvāt | tasmāt sāmagrī janikā, naikaṃ janakam iti sthiravādināṃ manorathasyāpy aviṣayaḥ |
	\pend
      

	  \pstart dṛśyate tāvad evam iti cet | dṛśyatām | kiṃ tu pūrvasthitād eva sāmagrīmadhyapraviṣṭād bhāvāt kāryotpattir anyasmād eva vā viśiṣṭād bhāvād utpannād iti vivādapadam | tatra prāg api sambhave sarvadaiva kāryotpattir na vā kadācid apīti virodham asamādhāya cakṣuṣī nimīlya tata eva kāryotpattidarśanād iti sādhyānuvādamātrapravṛttaḥ kṛpām arhatīti |
	\pend
      

	  \pstart na ca pratyabhijñā balād ekatvasiddhiḥ | tatpauruṣasya lūnapunarjātakeśanakhādāv apy upalambhato nirdalanāt | lakṣaṇabhedasya ca darśayitum aśakyatvāt | sthirasiddhi dūṣaṇe cāsmābhiḥ prapañcato nirastatvāt |\edlabel{thakur75-75.22}\label{thakur75-75.22} tasmāt sākṣāt kāryakāraṇabhāvāpekṣayobhayatrāpy anyayogavyavacchedaḥ | vyāptau tu sākṣāt paramparayā kāraṇamātrāpekṣayā kāraṇe vyāpake 'yogavyavacchedaḥ | kārye vyāpye 'nyayogavyavacchedaḥ | tathā tad atatsvabhāve vyāpake 'yogavyavacchedaḥ | tatsvabhāve ca vyāpye 'nyayogavyavacchedaḥ | vikalpārūḍharūpāpekṣayā vyāptau dvividham avadhāraṇam |
	\pend
      

	  \pstart nanu yadi pūrvāparakālayor ekasvabhāvo bhāvaḥ sarvadā janakatvenājanakatvena vā vyāpta upalabdhaḥ syāt, tadāyaṃ prasaṅgaḥ saṅgacchate | na ca kṣaṇabhaṅgavādinā pūrvāparakālayor ekaḥ kaścid upalabdha iti cet |\edlabel{thakur75-76.1}\label{thakur75-76.1} tad etad atigrāmyam | tathā hi pūrvāparakālayor ekasvabhāvatve satīty asyāyam arthaḥ, parakālabhāvī janako yaḥ svabhāvo bhāvasya sa eva yadi pūrvakālabhāvī, pūrvakālabhāvī vā yo 'janakaḥ svabhāvaḥ sa eva yadi parakālabhāvī, tadopalabdham eva jananam ajananaṃ vā syāt | tathā ca sati siddhayor eva svabhāvayor ekatvārope siddham eva jananam ajananaṃ vāsajyata iti |
	\pend
      

	  \pstart nanu kāryam eva sahakāriṇam apekṣate , na tu kāryotpattihetuḥ | yasmād dvividhaṃ sāmarthyaṃ nijam āgantukaṃ ca sahakāryantaram | tato 'kṣaṇikasyāpi kramavatsahakārinānātvād api kramavatkāryanānātvopapatter aśakyaṃ bhāvānāṃ pratikṣaṇam anyatvam upapādayitum iti cet |\edlabel{thakur75-76.9}\label{thakur75-76.9} ucyate | bhavatu tāvan nijāgantukabhedena dvividhaṃ sāmarthyam | tathāpi tat prātisvikaṃ vastusvalakṣaṇam arthakriyādharmakam avaśyam abhyupagantavyam | tat kiṃ prāg api paścād eva veti vikalpya yad dūṣaṇam udīritaṃ tatra kim uktam aneneti na pratīmaḥ | \edlabel{thakur75-76.12}\label{thakur75-76.12} yat tu kāryeṇaiva sahakāriṇo 'pekṣyanta ity upaskṛtaṃ tad api nirupayogam. yadi hi kāryam eva svajanmani svatantraṃ syād yuktam etat | kevalam evaṃ sati sahakārisākalyasāmarthyakalpanam aphalam | svātantryād eva hi kāryaṃ kādācitkaṃ bhaviṣyati | tathā ca sati santo hetavaḥ sarvathā 'samarthāḥ | asat tu kāryaṃ svatantram iti viśuddhā buddhiḥ |
	\pend
      

	  \pstart atha kāryasyaivāyam aparādho yad idaṃ samarthe kāraṇe saty api kadācin nopapadyata iti cet | na tat tarhi tatkāryaṃ, svātantryāt |\edlabel{thakur75-76.18}\label{thakur75-76.18} yad bhāṣyam,
	\pend
      
	    
	    \stanza[\smallbreak]
	sarvāvasthāsamāne 'pi kāraṇe yady akāryatā |&svatantraṃ kāryam evaṃ syān na tatkāryaṃ tathā sati || \edtext{}{\lemma{||}\Bfootnote{(PV II 396)}}\&[\smallbreak]


	

	  \pstart atha na tadbhāve bhavatīti tatkāryam ucyate, kiṃtu tadabhāve na bhavaty eveti vyatirekaprādhānyād iti cet |
	\pend
      

	  \pstart na | yadi hi svayaṃ bhavan bhāvayed eva hetuḥ svakāryam , tadā tadabhāvaprayukto 'syābhāva iti pratītiḥ syāt | no cet, yathā kāraṇe saty api kāryaṃ svātantryān na bhavati, tathā tadabhāve 'pi svātantryād eva na bhūtam iti śaṅkā kena nivāryeta |
	\pend
      

	  \pstart yad Bhāṣyam
	\pend
      
	    
	    \stanza[\smallbreak]
	tadbhāve 'pi na bhāvaś ced abhāve 'bhāvitā kutaḥ |&tadabhāvaprayukto 'sya so 'bhāva iti tat kutaḥ || \edtext{}{\lemma{||}\Bfootnote{(PVA II 411)}}\&[\smallbreak]


	

	  \pstart tasmād yathaiva tadabhāve niyamena na bhavati tathaiva tadbhāve niyamena bhaved eva | abhavac ca na tatkāraṇatām ātmanaḥ kṣamate |
	\pend
      

	  \pstart yac coktaṃ prathamakāryotpādanakāle hi uttarakāryotpādanasvabhāvaḥ, ataḥ prathamakāla evāśeṣāṇi kāryāṇi kuryād iti, tad idaṃ mātā me bandhyetyādivat svavacanavirodhād ayuktam | yo hi uttarakāryajananasvabhāvaḥ sa katham ādau kāryaṃ kuryāt | na tarhi tatkāryakaraṇasvabhāvaḥ | na hi nīlotpādanasvabhāvaḥ pītādikam api karotīti |
	\pend
      

	  \pstart artocyate | sthirasvabhāvatve hi bhāvasyottarakālam evedaṃ kāryaṃ na pūrvakālam iti kuta etat | tadabhāvāc ca kāraṇam apy uttarakāryakaraṇasvabhāvam ity api kutaḥ | 
	\pend
      

	  \pstart kiṃ kurmaḥ | uttarakālam eva tasya janmeti cet | astu, sthiratve tad anupapadyamānam, asthiratām ādiśatu |
	\pend
      

	  \pstart sthiratve 'py eṣa eva svabhāvas tasya yad uttarakṣaṇa eva karotīti cet | hatedānīṃ pramāṇapratyāśā | dhūmād atrāgnir ity atrāpi svabhāva evāsya yad idānīm atra niragnir api dhūma iti vaktuṃ śakyatvāt | tasmāt pramāṇasiddhe svabhāvāvalambanam | na tu svabhāvāvalambanena pramāṇavyālopaḥ | 
	\pend
      

	  \pstart tasmād yadi kāraṇasyottarakāryakārakatvam abhyupagamya kāryasya prathamakṣaṇabhāvitvam āsajyate, syāt svavacanavirodhaḥ | yadā tu kāraṇasya sthiratve kāryasyottarakālatvam evāsaṅgatam ataḥ kāraṇasyāpy uttarakāryajanakatvaṃ vastuto 'sambhavi tadā prasaṅgasādhanam idam | jananavyavahāragocaratvaṃ hi jananena vyāptam iti prasādhitam | uttarakāryajananavyavahāragocaratvaṃ ca tvad abhyupagamāt prathamakāryakaraṇakāla eva ghaṭe dharmiṇi siddhaṃ | atas tanmātrānubandhina uttarābhimatasya kāryasya prathame kṣaṇe 'sambhavād eva prasaṅgaḥ kriyate | 
	\pend
      

	  \pstart na hi nīlakārake 'pi pītakārakatvārope pītasambhavaprasaṅgaḥ svavacanavirodho nāma | 
	\pend
      

	  \pstart tad evaṃ śaktaḥ sahakāryanapekṣitatvād jananena vyāptaḥ | ajanayaṃś ca śaktāśaktatvaviruddhadharmādhyāsād bhinna eva ||
	\pend
      

	  \pstart nanu bhavatu prasaṅgaviparyayabalād ekakāryaṃ prati śaktāśaktatvalakṣaṇaviruddhadharmādhyāsaḥ | tathāpi na tato bhedaḥ sidhyati | 
	\pend
      

	  \pstart tathā hi bījam aṅkurādikaṃ kurvad yadi yenaiva svabhāvenāṅkuraṃ karoti tenaiva kṣityādikaṃ, tadā kṣityādīnām apy aṅkurasvābhāvyāpattiḥ | nānāsvabhāvatvena tu kārakatve svabhāvānām anyonyābhāvāvyabhicāritvād ekatra bhāvābhāvau parasparaviruddhau syātām ity ekam api bījaṃ bhidyeta |
	\pend
      

	  \pstart evaṃ pradīpo 'pi tailakṣayavarti dāhādikam | 
	\pend
      

	  \pstart tathā pūrvarūpam apy uttararūparasagandhādikam anekaiḥ svabhāvaiḥ parikaritaṃ karoti | 
	\pend
      

	  \pstart teṣāṃ ca svabhāvānām anyonyābhāvāvyabhicārād viruddhānāṃ yoge pradīpādikaṃ bhidyeta | na ca bhidyate | tan na viruddhadharmādhyāso bhedakaḥ | 
	\pend
      

	  \pstart tathā bījasyāṅkuraṃ prati kārakatvaṃ gardabhādikaṃ praty akārakatvam iti kārakatvākārakatve 'pi viruddhau dharmau | na ca tadyoge 'pi bījabhedaḥ | 
	\pend
      

	  \pstart tad evaṃ ekatra bīje pradīpe rūpe ca vipakṣe paridṛśyamānaḥ śaktāśaktatvādir viruddhadharmādhyāso na ghaṭāder bhedaka iti |
	\pend
      

	  \pstart atra brūmaḥ | bhavatu tāvad bījādīnām anekakāryakāritvād dharmabhūtānekasvabhāvabhedaḥ, tathāpi kaḥ prastāvo viruddhadharmādhyāsasya | svabhāvānāṃ hy anyonyābhāvāvyabhicāre bhedaḥ prāptāvasaro na virodhaḥ | virodhas tu yadvidhāne yanniṣedho yanniṣedhe ca yadvidhānaṃ tayor ekatra dharmiṇi parasparaparihārasthitatayā syāt | tad atraikaḥ svabhāvaḥ svābhāvena viruddho yukto bhāvābhāvavat | na tu svabhāvāntareṇa ghaṭatvavastutvavat | 
	\pend
      

	  \pstart evam aṅkurādikāritvaṃ tadakāritvena viruddhaṃ, na punar vastvantarakāritvena | pratyakṣavyāpāraś cātra yathā nānādharmair adhyāsitaṃ bhāvam abhinnaṃ vyavasthāpayati tathā tatkāryakāriṇaṃ kāryāntarākāriṇaṃ ca | 
	\pend
      

	  \pstart tad yadi pratiyogitvābhāvād anyonyābhāvāvyabhicāriṇāv api svabhāvāv aviruddhau tatkārakatvānyākārakatve vā viṣayabhedād aviruddhe tat kim āyātam, ekakāryaṃ prati śaktāśaktatvayoḥ parasparapratiyoginor viruddhayor dharmayoḥ | etayor api punar avirodhe virodho nāma dattajalāñ–jaliḥ ||
	\pend
      

	  \pstart bhavatu tarhy ekakāryāpekṣayaiva sāmarthyāsāmarthyayor virodhaḥ | kevalaṃ yathā tad eva kāryaṃ prati kvacid deśe śaktir deśāntare cāśaktir iti deśabhedād aviruddhe śaktyaśaktī tathaikatraiva kārye kālabhedād apy aviruddhe | yathā pūrvaṃ niṣkriyaḥ sphaṭikaḥ sa eva paścāt sakriya iti cet |
	\pend
      

	  \pstart ucyate | na hi vayaṃ paribhāṣāmātrād ekatra kārye deśabhedād aviruddhe śaktyaśaktī brūmaḥ, kiṃ tu virodhābhāvāt | taddeśakāryakāritvaṃ hi taddeśakāryākāritvena viruddham, na punar deśāntare tatkāryākāritvenānyakāryakāritvena vā ||
	\pend
      

	  \pstart yady evaṃ tatkālakāryakāritvaṃ tatkālakāryākāritvena viruddham | na punaḥ kālāntare tatkāryākāritvenānyakāryakāritvena vā | tat kathaṃ kālabhede 'pi virodha iti cet |
	\pend
      

	  \pstart ucyate | dvayor hi dharmayor ekatra dharmiṇy anavasthitiniyamaḥ parasparaparihārasthiti lakṣaṇo virodhaḥ | sa ca sākṣātparasparapratyanīkatayā bhāvābhāvavad vā bhavet, ekasya vā niyamena pramāṇāntareṇa bādhanān nityatvasattvavad vā bhaved iti na kaścid arthabhedaḥ | tad atraikadharmiṇi tatkālakāryakāritvādhāre kālāntare tatkāryākāritvasyā nyakāryakāritvasya vā niyamena pramāṇāntareṇa bādhanād virodhaḥ | 
	\pend
      

	  \pstart tathā hi yatraiva dharmiṇi tatkālakāryakāritvam upalabdhaṃ na tatraiva kālāntare tatkāryākāritvam anyakāryakāritvaṃ vā brahmaṇāpy upasaṃhartuṃ śakyate , yenānayor avirodhaḥ syāt | kṣaṇāntare
	\pend
      

	  \pstart kathitaprasaṅgaviparyayahetubhyām avaśyambhāvena dharmibhedaprasādhanāt ||
	\pend
      

	  \pstart na ca pratyabhijñānād ekatvasiddhiḥ, tatpauruṣasya nirmūlitatvāt | ata eva vajro 'pi pakṣakukṣau nikṣiptaḥ | katham asau sphaṭiko varākaḥ kālabhedenābhedaprasādhanāya dṛṣṭāntībhavitum arhati | 
	\pend
      

	  \pstart na caivaṃ samānakālakāryāṇāṃ deśabhede 'pi dharmibhedo yukto bhedaprasādhaka pramāṇābhāvāt indriyapratyakṣeṇa nirastavibhramāśaṅkenābhedaprasādhanāc ceti na kālabhede 'pi śaktyaśaktyor virodhaḥ svasamayamātrād apahastayituṃ śakyaḥ, samayapramāṇayor apravṛtter iti |
	\pend
      

	  \pstart tasmāt sarvatra viruddhadharmādhyāsasiddhir eva bhedasiddhiḥ | vipratipannaṃ prati tu viruddhadharmādhyāsād bhedavyavahāraḥ sādhyate ||
	\pend
      

	  \pstart nanu tathāpi sattvam idam anaikāntikam evāsādhāraṇatvāt sandigdhavyatirekitvād vā | yathā hīdaṃ kramākramanivṛttāv akṣaṇikān nivṛttaṃ, tathā sāpekṣatvānapekṣatvayor ekatvānekatvayor api vyāpakayor nivṛttau kṣaṇikād api | 
	\pend
      

	  \pstart tathā hi upasarpaṇapratyayena devadattakarapallavādinā sahacaro bījakṣaṇaḥ pūrvasmād eva puñjāt samartho jāto 'napekṣa ādyātiśayasya janaka iṣyate | 
	\pend
      

	  \pstart tatra ca samānakuśūlajanmasu bahuṣu bījasantāneṣu kasmāt kiñcid eva bījaṃ paramparayāṅkurotpādānuguṇam upajanayati bījakṣaṇaṃ, nānye bījakṣaṇā bhinnasantānāntaḥpātinaḥ | na hy upasarpaṇapratyayāt prāg eva teṣāṃ samānāsamānasantānavartināṃ bījakṣaṇānāṃ kaścit paramparātiśayaḥ | 
	\pend
      

	  \pstart athopasarpaṇapratyayāt prāṅ na tatsantānavartino 'pi janayanti, paramparayāpy aṅkurotpādānuguṇaṃ bījakṣaṇaṃ bījamātrajananāt teṣām | kasyacid eva bījakṣaṇasyopasarpaṇapratyayasahabhuva ādyātiśayotpādaḥ | hanta tarhi tadabhāve saty utpanno 'pi janayed eva | 
	\pend
      

	  \pstart tathā kevalānāṃ vyabhicārasambhavād ādyātiśayotpādakam aṅkuraṃ vā prati kṣityādīnāṃ parasparāpekṣāṇām evotpādakatvam akāmenāpi svīkartavyam | 
	\pend
      

	  \pstart ato na tāvad anapekṣā kṣaṇikasya sambhavinī | nāpy apekṣā yujyate, samasamayakṣaṇayoḥ savyetaragobiṣāṇayor ivopakāryopakārakabhāvāyogād iti nāsiddhaḥ prathamo vyāpakābhāvaḥ |
	\pend
      

	  \pstart api cāntyo bījakṣaṇo 'napekṣo 'ṅkurādikaṃ kurvan yadi yenaiva rūpeṇāṅkuraṃ karoti tenaiva kṣityādikaṃ, tadā kṣityādīnām apy aṅkurasvābhāvyāpattir abhinnakāraṇatvād iti na tāvad ekatvasambhavaḥ ||
	\pend
      

	  \pstart nanu rūpāntareṇa karoti | tathā hi bījasyāṅkuraṃ praty upādānatvam | kṣityādikaṃ tu prati sahakāritvam | yady evaṃ, sahakāritvopādānatve kim ekaṃ tattvaṃ nānā vā | ekaṃ cet, kathaṃ rūpāntareṇa janakam | nānātve tv anayor bījād bhedo 'bhedo vā | bhede kathaṃ bījasya janakatvaṃ tābhyām evāṅkurādīnām utpatteḥ | abhede vā kathaṃ bījasya na nānātvaṃ bhinnatādātmyāt, etayor vaikatvam ekatādātmyāt |
	\pend
      

	  \pstart yady ucyeta kṣityādau janayitavye tadupādānaṃ pūrvam eva kṣityādi bījasya rūpāntaram iti | na tarhi bījaṃ tadanapekṣaṃ kṣityādīnāṃ janakam | tadanapekṣatve teṣām aṅkurād bhedānupapatteḥ | na cānupakārakāṇy apekṣanta iti tvayaivotkam | na ca kṣaṇasyopakāra sambhavo 'nyatra jananāt, tasyābhedyatvād ity anekatvam api nāstīti dvitīyo 'pi vyāpakābhāvo nāsiddhaḥ | tasmād asādhāraṇānaikāntikatvaṃ gandhavattvavad iti |
	\pend
      

	  \pstart yadi manyetānupakārakā api bhavanti sahakāriṇo 'pekṣaṇīyāś ca kāryeṇānuvihitabhāvābhāvāc ca sahakāryakaraṇāc ca |
	\pend
      

	  \pstart nanv anena krameṇākṣaṇiko 'pi bhāvo 'nupakārakān api sahakāriṇaḥ kramavataḥ kramavat kāryeṇānukṛtānvayavyatirekān apekṣiṣyate | kariṣyate ca kramavatsahakārivaśaḥ krameṇa kāryāṇīti vyāpakānupalabdher asiddheḥ sandigdhavyatirekam anaikāntikaṃ sattvaṃ kṣaṇikatvasiddhāv iti |
	\pend
      

	  \pstart atra brūmaḥ | kīdṛśaṃ punar apekṣārtham ādāya kṣaṇike sāpekṣānapekṣatvanivṛttir ucyate | kiṃ sahakāriṇam apekṣata iti sahakāriṇāsyopakāraḥ karttavyaḥ | atha pūrvāvasthitasyaiva bījādeḥ sahakāriṇā saha sambhūyakaraṇam | yadvā pūrvāvasthitasyety anapekṣya militāvasthasya karaṇamātram apekṣārthaḥ | atra prathamapakṣasyāsambhavād anapekṣaiva kṣaṇikasya, katham ubhayavyāvṛttiḥ | 
	\pend
      

	  \pstart yady anapekṣaḥ kṣaṇikaḥ , kimity upasarpaṇapratyayābhāve 'pi na karoti | karoty eva yadi syāt | svayam asambhavī tu kathaṃ karotu | atha tad vā tādṛg vāsīd iti na kaścid viśeṣaḥ | tatas tādṛk svabhāvasambhave 'py akaraṇaṃ sahakāriṇi nirapekṣān na kṣamata iti cet |
	\pend
      

	  \pstart asambaddham etat | varṇasaṃsthānasāmye 'py akartus tatsvabhāvatāyā virahāt | sa cādyātiśayajanakatvalakṣaṇaḥ svabhāvaviśeṣo na samānāsamānasantānavartiṣu bījakṣaṇeṣu sarveṣv eva sambhavī | kiṃ tu keṣucid eva karmakarakarapallavasahacareṣu |
	\pend
      

	  \pstart nanv ekatra kṣetre niṣpattilavanādipūrvakam ānīyaikatra kuśūle kṣiptāni sarvāṇy eva bījāni sādhāraṇarūpāṇy eva pratīyante | tat kutastyo 'yam ekabījasambhavī viśeṣo 'nyeṣāṃ iti cet | 
	\pend
      

	  \pstart ucyate | kāraṇam khalu sarvatra kārye dvividham | dṛṣṭam adṛṣṭaṃ ceti | sarvāstikaprasiddham etat | tataḥ pratyakṣaparokṣasahakāripratyayasākalyam asarvavidā pratyakṣato na śakyaṃ pratipattum | tato bhaved api kāraṇasāmagrīśaktibhedāt tādṛśaḥ svabhāvabhedaḥ keṣāñcid eva bījakṣaṇānāṃ yena ta eva bījakṣaṇā ādyātiśayam aṅkuraṃ vā paramparayā janayeyuḥ | nānye ca bījakṣaṇāḥ |
	\pend
      

	  \pstart nanu yeṣūpasarpaṇapratyayasahacareṣu svakāraṇaśaktibhedād ādyātiśaya janakatvalakṣaṇo viśeṣaḥ sambhāvyate sa tatrāvaśyam astīti kuto labhyam iti cet | 
	\pend
      

	  \pstart aṅkurotpādād anumitād ādyātiśayalakṣaṇāt kāryād iti brūmaḥ | kāraṇānupalabdhes tarhi tadabhāva eva bhaviṣyatīti cet | na | dṛśyādṛśyasamudāyasya kāraṇasyādarśane 'py abhāvāsiddheḥ kāraṇānupalabdheḥ sandigdhāsiddhatvāt |
	\pend
      

	  \pstart tad ayam arthaḥ 
	\pend
      

	  \pstart pāṇisparśavataḥ kṣaṇasya na bhidā bhinnānyakālakṣaṇād bhedo veti matadvaye mitibalaṃ yasyāsty asau jitvaraḥ |
	\pend
      

	  \pstart tatraikasya balaṃ nimittavirahaḥ kāryāṅgam anyasya vā sāmagrī tu na sarvathekṣaṇasahā kāryaṃ tu mānānugam || 
	\pend
      

	  \pstart iti |
	\pend
      

	  \pstart tad evaṃ nopakāro 'pekṣārtha ity anapekṣaiva kṣaṇikasya sahakāriṣu nobhayavyāvṛttiḥ ||
	\pend
      

	  \pstart atha sambhūyakaraṇam apekṣārthaḥ, tadā yadi pūrvasthitasyeti viśeṣaṇāpekṣā tadā kṣaṇikasya naivaṃ kadācid ity anapekṣaivākṣīṇā |
	\pend
      

	  \pstart atha pūrvasthitasyety anapekṣya militāvasthitasyaiva karaṇam apekṣārthas tadā sāpekṣataiva, nānapekṣā | tathā ca nobhayavyāvṛttir ity asiddhaḥ prathamo vyāpakānupalambhaḥ | 
	\pend
      

	  \pstart tathaikatvānekatvayor api vyāpakayoḥ kṣaṇikād vyāvṛttir asiddhā | tattadvyāvṛttibhedam āśrityopādānatvādi kālpanikasvabhāvabhede 'pi paramārthata ekenaiva svarūpeṇānekakāryaniṣpādanād ubhayavyāvṛtter abhāvāt |
	\pend
      

	  \pstart yac ca bījasyaikenaiva svabhāvena kārakatve kṣityādīnām aṅkurasvābhāvyāpattir anyathā kāraṇābhede 'pi kāryabhede 'pi kāryasyāhetukatvaprasaṅgād ity uktam tad asaṅgatam | kāraṇaikatvasya kāryabhedasya ca paṭunendriyapratyakṣeṇa prasādhanāt | ekakāraṇajanyatvaikatvayor vyāpteḥ pratihatatvāt | prasaṅgasyānupadatvāt |
	\pend
      

	  \pstart yac ca kāraṇābhede kāryābheda ity uktaṃ tatra sāmagrīsvarūpaṃ kāraṇam abhipretam | sāmagrīsajātīyatve na kāryavijātīyatety arthaḥ | na punaḥ sāmagrīmadhyagatenaikenānekaṃ kāryaṃ na kartavyaṃ nāma, ekasmād anekotpatteḥ pratyakṣasiddhatvāt | na caivaṃ pratyabhijñānāt kālabhede 'py abhedasiddhir ity uktaprāyam | na cendriyapratyakṣaṃ bhinnadeśaṃ sapratighaṃ dṛśyam arthadvayam ekam evopalambhayatīti kvacid upalabdham | yena tatrāpi bhedaśaṅkā syāt | śaṅkāyāṃ vā paṭupratyakṣasyāpy apalāpe sarvapramāṇocchedaprasaṅgād |
	\pend
      

	  \pstart nāpi sattvahetoḥ sandigdhavyatirekitvam , kṣityāder dravyāntarasya bījasvabhāvatvenāsmābhir asvīkṛtatvāt | anupakāriṇy apekṣāyāḥ pratyākhyātatvāt vyāpakānupalambhasyāsiddhatvāyogāt |
	\pend
      

	  \pstart tad etau dvāv api vyāpakānupalambhāv asiddhau na kṣaṇikāt sattvaṃ nivartayata iti nāyam asādhāraṇo hetuḥ ||
	\pend
      

	  \pstart api ca vidyamāno bhāvaḥ sādhyetarayor aniścitānvayavyatireko gandhavattādivad asādhāraṇo yuktaḥ | prakṛtavyāpakānupalambhāc ca sarvathārthakriyaivāsatī ubhābhyāṃ vādibhyām ubhayasmād vinivartitatvena nirāśrayatvāt | tat katham asādhāraṇānaikāntiko bhaviṣyatīty alaṃ pralāpini nirbandhena |
	\pend
      

	  \pstart tad evaṃ śaktasya kṣepāyogāt samarthavyavahāragocaratvaṃ jananena vyāptam iti prasaṅgaviparyayayoḥ sattve hetor api nānaikāntikatvam | ataḥ kṣaṇabhaṅgasiddhir iti sthitam |
	\pend
      
	    
	    \stanza[\smallbreak]
	\label{thakur75-82.15} iti sādharmyadṛṣṭānte 'nvayarūpavyāptyā kṣaṇabhaṅgasiddhiḥ samāptā ||&\label{thakur75-82.17} kṛtir iyaṃ mahāpaṇḍitaratnakīrtipādānām iti ||\&[\smallbreak]


	
	    
	    \endnumbering% ending numbering from div
	    \endgroup
	    
	  
	  
	% new div opening: depth here is 0
	
	    
	    \begingroup
	    \beginnumbering% beginning numbering from div depth=0
	    
	  
\chapter[{Kṣaṇabhaṅgasiddhiḥ Vyatirekātmikā}]{Kṣaṇabhaṅgasiddhiḥ Vyatirekātmikā}\label{Kṣaṇabhaṅgasiddhiḥ_Vyatirekātmikā}

	  \pstart namas tārāyai 
	    \pend
	  
	    
	    \stanza[\smallbreak]
	\label{thakur75-83.6}\edlabel{thakur75-83.6}\flagstanza{\tiny\textenglish{...5-83.6}}vyatirekātmikā vyāptir ākṣiptānvayarūpiṇī |&vaidharmyavati dṛṣṭānte sattvahetor ihocyate ||\&[\smallbreak]


	
	    \pstart
	   \edlabel{thakur75-83.8}\label{thakur75-83.8} yat sat tat kṣaṇikam | yathā ghaṭaḥ | santaś cāmī vivādāspadībhūtāḥ padārthā iti svabhāvahetuḥ | \edlabel{thakur75-83.10}\label{thakur75-83.10} na tāvad asyāsiddhiḥ sambhavati, yathāyogaṃ pratyakṣānumānapramāṇapratīte dharmiṇi sattvaśabdenābhipretasyārthakriyākāritvalakṣaṇasya sādhanasya pramāṇasamadhigatatvāt | \edlabel{thakur75-83.12}\label{thakur75-83.12} na ca viruddhānaikāntikate, vyāpakānupalambhātmanā viparyaye bādhakapramāṇena vyāpteḥ prasādhanāt | \edlabel{thakur75-83.13}\label{thakur75-83.13} yasya kramākramau na vidyete na tasyārthakriyāsāmarthyam | yathā śaśaviṣāṇasya | na vidyete cākṣaṇikasya kramākramāv iti vyāpakānupalambhaḥ | \edlabel{thakur75-83.14}\label{thakur75-83.14} na tāvad ayam asiddho hetuḥ, akṣaṇike dharmiṇi kramākramasadbhāvāyogāt | tathā hi prāptāparakālayor ekatve nityatvam | tasya kramākramayoge kṣaṇadvaye 'py avaśyaṃ bhedaḥ | bhedābhedayoś ca parasparavirodhāt kuto 'kṣaṇike kramākramasambhavaḥ | kṣaṇadvaye 'pi bhede kramākramayogaḥ | abhede hi prathama eva kṣaṇe śaktatvād bhāvino 'pi kāryasya karaṇaprasaṅge kathaṃ kāryāntarakaraṇe kramāntarāvakāśaḥ | na cākṣaṇikasyākrameṇaiva sakalasvakāryaṃ kṛtvā svāsthyam | kṣaṇāntare 'pi śaktatvāt punas tatkāryakaraṇaprasaṅgāt | \edlabel{thakur75-83.21}\label{thakur75-83.21} tasmād akṣaṇikam iti pūrvāparakālayor abhedaḥ | kramākramayoga iti pūrvāparakālayor bhedaḥ | anayoś ca parasparaparihārasthitilakṣaṇo virodhaḥ | \edlabel{thakur75-83.23}\label{thakur75-83.23} tad ayam akṣaṇike dharmiṇi kramākramābhāvalakṣaṇo hetur nāsiddho vaktavyaḥ | kramākramayogitvākṣaṇikatvayor virodhād eva | \edlabel{thakur75-84.1}\label{thakur75-84.1} nāpi viruddhaḥ, sapakṣe bhāvāt | \edlabel{thakur75-84.2}\label{thakur75-84.2} na cānaikāntikaḥ, kramākramābhāvasyārthakriyāsāmarthyābhāvena vyāptatvāt | \edlabel{thakur75-84.3}\label{thakur75-84.3} yenaiva hi pratyakṣātmanā pramāṇenāparaprakārābhāvād vidhibhūtābhyāṃ kramākramābhyāṃ vidhibhūtasyārthakriyāsāmarthyasya vyāptiḥ prasādhitā, tenaivārthakriyāsāmarthyābhāvena kramākramābhāvasya vyāptiḥ prasādhiteti svīkartavyam | na hi dahanādinā dhūmāder vyāptisādhakapramāṇād aparaṃ dhūmādyabhāvena dahanādyabhāvasya vyāptisādhakaṃ kiñcit pramāṇaṃ śaraṇabhūtam asti | tasmād vidhyor eva vyāptisādhakaṃ pramāṇam abhāvayor api vyāptisādhakam iti nyāyasya duratikramatvāt sattvābhāvena kramākramābhāvo vyāpta eveti nānaikāntika ity anavadyo vyāpakānupalambhaḥ | tad ayam akṣaṇikād vinivartamāṇaḥ svavyāpyaṃ sattvaṃ nivartya kṣaṇike viśrāmayatīti sattvahetoḥ kṣaṇabhaṅgasiddhir apy anavadyā | \edlabel{thakur75-85.1}\label{thakur75-85.1} nanu vyāpakānupalambhataḥ sattvasya kathaṃ svasādhyapratibandhasiddhiḥ, asyāpy anekadoṣaduṣṭatvāt. tathā hi – na tāvad ayaṃ prasaṅgahetuḥ, sādhyadharmiṇi pramāṇasiddhatvāt, parābhyupagamasiddhatvābhāvāt, viparyayaparyavasānābhāvāc ca. atha svatantraḥ, tadāśrayāsiddhaḥ, akṣaṇikasyāśrayasyāsambhavād apratītatvād vā. pratītir hi2 [a] pratyakṣeṇa [b] anumānena [c] vikalpamātreṇa vā syāt | \edlabel{thakur75-85.6}\label{thakur75-85.6} [a] [b] prathamapakṣadvaye sākṣāt pāramparyeṇa vā svapratītilakṣaṇārthakāritve maulaḥ sādhāraṇo hetuḥ vyāpakānupalambhaś ca svarūpāsiddhaḥ syāt, arthakriyākāritve kramākramayor anyatarasyāvaśyambhāvāt | \edlabel{thakur75-85.8}\label{thakur75-85.8} [c] antimapakṣe tu na kaścid dhetur anāśrayaḥ syāt, vikalpamātrasiddhasya dharmiṇaḥ sarvatra sulabhatvāt. \edlabel{thakur75-85.10}\label{thakur75-85.10} api ca – tat kalpanājñānaṃ [c1] pratyakṣapṛṣṭhabhāvi vā syāt, [c2] liṅgajanma vā, [c3] saṃskārajaṃ vā, [c4] sandigdhavastukaṃ vā, [c5] avastukaṃ vā. \edlabel{thakur75-85.12}\label{thakur75-85.12} tatra [c1][c2] ādyapakṣadvaye 'kṣaṇikasya sattaivāvyāhatā, kathaṃ bādhakāvatāraḥ. \edlabel{thakur75-85.12a}\label{thakur75-85.12a} [c3] tṛtīye tu na sarvadākṣaṇikasattāniṣedhaḥ, tadarpitasaṃskārābhāve tatsmaraṇāyogāt | \edlabel{thakur75-85.13}\label{thakur75-85.13} [c4] caturthe tu sandigdhāśrayatvaṃ hetudoṣaḥ | \edlabel{thakur75-85.14}\label{thakur75-85.14} [c5] pañcame ca tadviṣayasyābhāvo na tāvat pratyakṣataḥ sidhyati, akṣaṇikātmanaḥ sarvadaiva tvanmate 'pratyakṣatvāt | na cānumānatas tadabhāvas tatpratibaddhaliṅgānupalambhād ity āśrayāsiddhis tāvad uddhatā | evaṃ dṛṣṭānto 'pi pratihantavyaḥ | \edlabel{thakur75-85.18}\label{thakur75-85.18} svarūpāsiddho 'py ayaṃ hetuḥ, sthirasyāpi kramākramisahakāryapekṣayā kramākramābhyām arthakriyopapatteḥ | nāpi kramayaugapadyapakṣoktadoṣaprasaṅgaḥ | tathā hi kramisahakāryapekṣayā kramikāryakāritvaṃ tāvad aviruddham | \edlabel{thakur75-85.21}\label{thakur75-85.21} tathā ca Śaṅkarasya saṃkṣipto 'yam abhiprāyaḥ | sahakārisākalyaṃ hi sāmarthyam | tadvaikalyaṃ cāsāmarthyam | na ca tayor āvirbhāvatirobhāvābhyāṃ tadvataḥ kācit kṣatiḥ, tasya tābhyām anyatvāt | tat kathaṃ sahakāriṇo 'napekṣya kāryakaraṇaprasaṅga iti | \edlabel{thakur75-85.25}\label{thakur75-85.25} \persName{trilocanasyā}py ayaṃ saṃkṣiptārthaḥ | kāryam eva hi sahakāriṇam apekṣate | na kāryotpattihetuḥ | yasmāt dvividhaṃ sāmarthyaṃ nijam āgantukaṃ ca sahakāryantaram, tato 'kṣaṇikasyāpi kramavatsahakārinānātvād api kramavatkāryanānātvopapatter aśakyaṃ bhāvānāṃ pratikṣaṇam anyānyatvam upapādayitum iti | \edlabel{thakur75-85.29}\label{thakur75-85.29} Nyāyabhūṣaṇo 'pi lapati | prathamakāryotpādanakāle hi uttarakāryotpādanasvabhāvaḥ | ataḥ prathamakāla evāśeṣāṇi kāryāṇi kuryād iti cet | \edlabel{thakur75-85.30}\label{thakur75-85.30} tad idaṃ mātā me bandhyetyādivat svavacanavirodhād ayuktaṃ | yo hi uttarakāryajananasvabhāvaḥ sa katham ādau tat kāryaṃ kuryāt | atha kuryāt na tarhi tatkāryakaraṇasvabhāvaḥ | na hi nīlotpādanasvabhāvaḥ pītādikam api karotīti | \edlabel{thakur75-86.3}\label{thakur75-86.3} Vācaspatir api paṭhati | nanv ayam akṣaṇikaḥ svarūpeṇa kāryaṃ janayati | tac cāsya svarūpaṃ tṛtīyādiṣv iva kṣaṇeṣu dvitīye 'pi kṣaṇe sad iti tadāpi janayet | akurvan vā tṛtīyādiṣv api na kurvīta, tasya tādavasthyāt | atādavasthye vā tad evāsya kṣaṇikatvam || \edlabel{thakur75-86.7}\label{thakur75-86.7} atrocyate | satyaṃ svarūpeṇa kāryaṃ janayati na tu tenaiva | sahakārisahitād eva tataḥ kāryotpattidarśanāt | tasmād vyāptivat kāryakāraṇabhāvo 'py ekatrānyayogavyavacchedena | anyatrāyogavyavacchedenāvaboddhavyaḥ | tathaiva laukikaparīkṣakāṇāṃ saṃpratipatter iti na kramikāryakāritvapakṣoktadoṣāvasaraḥ || \edlabel{thakur75-86.11}\label{thakur75-86.11} nāpy akṣaṇike yaugapadyapakṣoktadoṣāvakāśaḥ | ye hi kāryam utpāditavanto dravyaviśeṣās teṣāṃ vyāpārasya niyatakāryotpādanasamarthasya niṣpādite kārye 'nuvartamāneṣv api teṣu dravyeṣu nivṛttārthādūnā sāmagrī jāyate | tat kathaṃ niṣpāditaṃ niṣpādayiṣyati | na hi daṇḍādayaḥ svabhāvenaiva kartāro yenāmī niṣpatter ārabhya kāryaṃ vidadhyuḥ | kiṃ tarhi vyāpārāveśinaḥ | na ceyatā svarūpeṇa na kartāraḥ, svarūpakārakatvanirvāhaparatayā vyāpārasamāveśād iti || \edlabel{thakur75-86.17}\label{thakur75-86.17} kiṃ ca kramākramābhāvaś ca bhaviṣyati na ca sattvābhāva iti sandigdhavyatireko 'py ayaṃ vyāpakānupalambhaḥ | na hi kramākramābhyām anyasya prakārasyābhāvaḥ siddhaḥ, viśeṣaniṣedhasya śeṣābhyanujñāviṣayatvāt | \edlabel{thakur75-86.20}\label{thakur75-86.20} kiṃ ca prakārāntarasya dṛśyatve nātyantaniṣedhaḥ | adṛśyatve tu nāsattāniścayo viprakarṣiṇām iti na kramākramābhyām arthakriyāsāmarthyasya vyāptisidhiḥ | ataḥ sandigdhavyatireko 'pi vyāpakānupalambhaḥ | \edlabel{thakur75-86.23}\label{thakur75-86.23} kiṃ ca dṛśyādṛśyasahakāripratyayasākalyavataḥ kramayaugapadyasyātyantaparokṣatvāt tena vyāptaṃ sattvam api parokṣam eveti na tāvat pratibandhaḥ pratyakṣataḥ sidhyati | nāpy anumānataḥ tatpratibaddhaliṅgābhāvād iti | \edlabel{thakur75-86.26}\label{thakur75-86.26} api ca kramākramābhyām arthakriyākāritvaṃ vyāptam ity atisubhāṣitam | yadi krameṇa vyāptaṃ katham akrameṇa | athākrameṇa na tarhi krameṇa | kramākramābhyām vyāptam iti tu bruvatā vyāpter evābhāvaḥ pradarśito bhavati | na hi bhavati dhūmo vahnibhāvābhāvābhyāṃ vyāpta iti | ato vyāpter anaikāntikatvam | \edlabel{thakur75-86.30}\label{thakur75-86.30} capi ca kim idaṃ bādhakam akṣaṇikānām asattāṃ sādhayati, utasvid akṣaṇikāt sattvasya vyatirekam, atha sattvakṣaṇikatvayoḥ pratibandham. \edlabel{thakur75-86.31}\label{thakur75-86.31} na pūrvo vikalpaḥ, uktakrameṇa hetor āśrayāsiddhatvāt | \edlabel{thakur75-87.1}\label{thakur75-87.1} na ca dvitīyaḥ. yato vyāpakanivṛttisahitā vyāpyanivṛttir vyatirekaśabdasyārthaḥ. sā ca yadi pratyakṣeṇa pratīyate tadā taddhetuḥ syād iti sattvam anaikāntikam. vyāpakānupalambhaḥ svarūpāsiddhaḥ. atha sā vikalpyate tadā pūrvoktakrameṇa pañcadhā vikalpya vikalpo dūṣaṇīyaḥ. \edlabel{thakur75-87.4}\label{thakur75-87.4} ata eva na tṛtīyo 'pi vikalpaḥ vyatirekāsiddhau sambandhāsiddheḥ | \edlabel{thakur75-87.6}\label{thakur75-87.6} kiṃ ca na bhūtalavad atrākṣaṇiko dharmī dṛśyate | na ca svabhāvānupalambhe vyāpakānupalambhaḥ kasyacit dṛśyasya pratipattim antareṇāntarbhāvayituṃ śakyata iti | \edlabel{thakur75-87.9}\label{thakur75-87.9} kiṃ cāsyābhāvadharmatve āśrayāsiddhatvam itaretarāśrayatvaṃ ca | bhāvadharmatve viruddhatvaṃ ca | ubhayadharmatve cānaikāntikatvam iti na trayīṃ doṣajātim atipatati | \edlabel{thakur75-87.11}\label{thakur75-87.11} yat punar uktam akṣaṇikatve kramayaugapadyābhyām arthakriyāvirodhād iti | dtatra virodhasiddhim anusaratā virodhy api pratipattavyaḥ | tatpratītināntarīyakatvād virodhasiddheḥ | yathā tuhinadahanayoḥ sāpekṣadhruvabhāvayoś ca | \edlabel{thakur75-87.13}\label{thakur75-87.13} pratiyogī cākṣaṇikaḥ pratīyamānaḥ pratītikāritvāt sann eva syāt, ajanakasyāprameyatvāt | \edlabel{thakur75-87.15}\label{thakur75-87.15} saṃvṛtisiddhenākṣaṇikatvena virodhasiddhir iti cet | saṃvṛtisiddham api vāstavaṃ kālpanikaṃ vā syāt | \edlabel{thakur75-87.17}\label{thakur75-87.17} yadi vāstavaṃ kathaṃ tasyāsattvam | kathaṃ cārthakriyākāritvavirodhaḥ | arthakriyāṃ kurvad dhi vāstavam ucyate | \edlabel{thakur75-87.19}\label{thakur75-87.19} atha kālpanikam | tatra kiṃ virodho vāstavaḥ, kālpaniko vā | na tāvad vāstavaḥ, kalpitavirodhivirodhatvāt, bandhyāputravirodhavat | atha virodho 'pi kālpanikaḥ na tarhi sattvasya vyatirekaḥ pāramārthika iti kṣaṇabhaṅgo dattajalāñjalir iti | \edlabel{thakur75-87.23}\label{thakur75-87.23} ayam eva codyaprabandho 'smadgurubhiḥ saṅgṛhītaḥ | enityaṃ nāsti na vā pratītiviṣayaṃ3 tenāśrayāsiddhatā hetoḥ svānubhavasya ca kṣatir ataḥ kṣiptaḥ sapakṣo 'pi ca | śūnyaś ca dvitayena sidhyati na cāsattāpi sattā yathā no nityena virodhasiddhir asatā śakyā kramāder api || J 89,16-19; cf. R 94,21-24 iti | \edlabel{thakur75-87.28}\label{thakur75-87.28} atrocyate – iha vastuny api dharmidharmavyavahāro dṛṣṭaḥ, yathā gavi gotvam, paṭe śuklatvam, turage gamanam ityādi. avastuny api dharmidharmavyavahāro dṛṣṭaḥ, yathā śaśaviṣāṇe tīkṣṇatvābhāvaḥ, bandhyāputre vaktṛtvābhāvaḥ, gaganāravinde gandhābhāva ityādi. tatrāvastuni dharmitvaṃ nāstīti kiṃ vastudharmeṇa dharmitvaṃ nāsti, āhosvid avastudharmeṇāpi | \edlabel{thakur75-88.3}\label{thakur75-88.3} prathamapakṣe siddhasādhanam. dvitīyapakṣe tu svavacanavirodhaḥ. yad āhur guravaḥ – fdharmasya kasyacid avastuni mānasiddhā bādhāvidhivyavahṛtiḥ kim ihāsti no vā | kvāpy asti cet katham iyanti na dūṣaṇāni nāsty eva cet svavacanapratirodhasiddhiḥ || J 89,21-24; cf. R 94,26-28 \edlabel{thakur75-88.8}\label{thakur75-88.8} avastuno dharmitvasvīkārapūrvakatvasya vyāpakasyābhāvād āśrayāsiddhidūṣaṇasyānupanyāsaprasaṅga ity arthaḥ | yenaiva hi vacanenāvastuno dharmitvaṃ pratiṣidhyate, tenaivāvastuno dharmitvābhāvena dharmeṇa dharmitvam abhyupagatam | paran tu pratiṣidhyata iti vyaktam idam īśvaraceṣṭitam | tathā hy avastuno dharmitvaṃ nāstīti vacanena dharmitvābhāvaḥ kim avastuni vidhīyate, anyatra vā, na vā kvacid apīti trayaḥ pakṣāḥ | \edlabel{thakur75-88.13}\label{thakur75-88.13} prathamapakṣe 'vastuno na dharmitvaniṣedhaḥ dharmitvābhāvasya dharmasya tatraiva vidhānāt | \edlabel{thakur75-88.14}\label{thakur75-88.14} dvitīye 'vastuni kim āyātam anyatra dharmitvābhāvavidhānāt | \edlabel{thakur75-88.15}\label{thakur75-88.15} tṛtīyas tu pakṣo vyartha eva nirāśrayatvād iti katham avastuno dharmitvaniṣedhaḥ | tasmād yathā pramāṇopanyāsaḥ prameyasvīkārapūrvakatvena vyāptaḥ vācakaśabdopanyāso vā vācyasvīkārapūrvakatvena vyāptas tathāvastuno dharmitvaṃ nāstīti vacanopanyāso 'vastuno dharmitvasvīkārapūrvakatvena vyāptaḥ | anyathā tadvacanopanyāsasya vyarthatvāt | \edlabel{thakur75-88.19}\label{thakur75-88.19} tad yadi vacanopanyāso vyāpyadharmas tadā 'vastuno dharmitvasvīkāro 'pi vyāpakadharmo durvāraḥ | atha na vyāpakadharmaḥ tadā vyāpyasyāpi vacanopanyāsasyāsambhava iti mūkataivātra balād āyāteti kathaṃ na svavacanapratirodhasiddhiḥ | \edlabel{thakur75-88.22}\label{thakur75-88.22} yad āhācāryaḥ: na hy abruvan paraṃ bodhayitum īśaḥ | bruvan vā doṣam imaṃ parihartum iti mahati saṃkaṭe praveśaḥ | \edlabel{thakur75-88.24}\label{thakur75-88.24} avastuprastāve sahṛdayānāṃ mūkataiva yujyata iti cet | aho mahadvaidagdhyam | avastuprastāve svayam eva yathāśakti valgitvā bhagno mūkataiva nyāyaprāpteti paribhāṣayā niḥsartum icchati | na cāvastuprastāvo rājadaṇḍena vinā caraṇamardanādināniṣṭimātreṇa vā pratiṣeddhaṃ śakyate | tataś cātrāpi kramākramabhāvasya sādhanatve sattvābhāvasya ca sādhyatve sandigdhavastubhāvasyāvastvātmano vā kṣaṇikasya dharmitvaṃ kena pratiṣidhyate | \edlabel{thakur75-89.1}\label{thakur75-89.1} trividho hi dharmo dṛṣṭaḥ | kaścit vastuniyato nīlādiḥ | kaścid avastuniyato yathā sarvopākhyāvirahaḥ | kaścid ubhayasādhāraṇo yathā 'nupalabdhimātram | tatra vastudharmeṇāvastuno dharmitvaniṣedha iti yuktam | na tv avastudharmeṇa vastvavastudharmeṇa vā, svavacanasyānupanyāsaprasaṅgād ity akṣaṇikasyābhāve sandehe 'pi vā vastudharmeṇa dharmitvam avyāhatam iti nāyam āśrayāsiddho vyāpakānupalambhaḥ | \edlabel{thakur75-89.6}\label{thakur75-89.6} akṣaṇikāpratītāv āśrayāsiddho hetur iti yuktam uktam, tadapratītau tadvyavahārāyogāt | kevalam asau vyavahārāṅgabhūtā pratītir vastvavastunor ekarūpā na bhavati | sākṣāt pāramparyeṇa vastusāmarthyabhāvinī hi vastupratītiḥ | yathā pratyakṣam anumānaṃ pratyakṣapṛṣṭhabhāvī ca vikalpaḥ | avastunas tu sāmarthyābhāvād vikalpamātram eva pratītiḥ | vastuno hi vastubalabhāvinī pratītir yathā sākṣāt pratyakṣam, paramparayā tatpṛṣṭhabhāvī vikalpo 'numānaṃ ca | avastunas tu na vastubalabhāvinī pratītis tatkārakatvenāvastutvahāniprasaṅgāt | tasmād vikalpamātram evāvastunaḥ pratītiḥ | \edlabel{thakur75-89.12}\label{thakur75-89.12} na hy abhāvaḥ kaścid vigrahavān yaḥ sākṣāt kartavyo 'pi tu vyavahartavyaḥ | sa ca vyavahāro vikalpād api sidhyaty eva anyathā sarvajanaprasiddho 'vastuvyavahāro na syāt | iṣyate ca taddharmitvapratiṣedhānubandhād ity akāmakenāpi vikalpamātrasiddho 'kṣaṇikaḥ svīkartavya iti nāyam apratītatvād apy āśrayāsiddho hetur vaktavyaḥ | tataś cākṣaṇikasya vikalpamātrasiddhatve yad uktam | \edlabel{thakur75-89.17}\label{thakur75-89.17} na kaścid dhetur anāśrayaḥ vikalpamātrasiddhasya dharmiṇaḥ sarvatra sulabhatvād iti tad asaṅgatam | vikalpamātrasiddhasya dharmiṇaḥ sarvatra sambhave 'pi vastudharmeṇa dharmitvāyogāt | vastudharmahetutvāpekṣayā āśrayāsiddhasyāpi hetoḥ sambhavāt | \edlabel{thakur75-89.20}\label{thakur75-89.20} yathātmano vibhutvasādhanārtham upanyastaṃ sarvatropalabhyamānaguṇatvād iti sādhanam | vikalpaś cāyaṃ hetūpanyāsāt pūrvaṃ sandigdhavastukaḥ | samarthite tu hetāv avastuka iti brūmaḥ | \edlabel{thakur75-89.23}\label{thakur75-89.23} na cātra sandigdhāśrayatvaṃ nāma hetudośaḥ | āstāṃ tāvat | sandigdhasyāvastuno 'pi vikalpamātrasiddhasyāvastudharmāpekṣayā dharmitvaprasādhanāt | vastudharmahetvapekṣayaiva sandigdhāśrayasya hetvābhāsasya vyavasthāpanāt | yatheha nikuñje mayūraḥ kekāyitād iti | avastukavikalpaviṣayasyāsattvaṃ tu vyāpakānupalambhād eva prasādhitam | evaṃ dṛṣṭāntasyāpi vyomotpalāder dharmitvaṃ vikalpamātreṇa pratītiś cāvagantavyā | tad evam avastudharmāpekṣāyāvastuno dharmitvasya vikalpamātreṇa pratīteś cāpahnotum aśakyatvān nāyam āśrayāsiddho hetuḥ | na ca dṛṣṭāntakṣatiḥ | \edlabel{thakur75-89.30}\label{thakur75-89.30} na caiṣa svarūpāsiddhaḥ, akṣaṇike dharmiṇi kramākramayor vyāpakayor ayogāt | tathā hi yadi tasya prathame kṣaṇe dvitīyādikṣaṇabhāvikāryakaraṇasāmarthyam asti tadā prathamakṣaṇabhāvikāryavat dvitīyādikṣaṇabhāvy api kāryaṃ kuryāt, samarthasya kṣepāyogāt | \edlabel{thakur75-90.2}\label{thakur75-90.2} atha tadā sahakārisākalyalakṣaṇasāmarthyaṃ nāsti, tadvaikalyalakṣaṇasyāsāmarthyasya sambhavāt | na hi bhāvaḥ svarūpeṇa karotīti svarūpeṇaiva karoti, sahakārisahitād eva tataḥ kāryotpattidarśanād iti cet | \edlabel{thakur75-90.4}\label{thakur75-90.4} yadā tāvad amī militāḥ santaḥ kāryaṃ kurvate | tadaikārthakaraṇalakṣaṇaṃ sahakāritvam eṣām astu, ko niṣeddhā | militair eva tu tatkāryaṃ kartavyam iti kuto labhyate | pūrvāparakālayor ekasvabhāvatvād bhāvasya sarvadā jananājananayor anyataraniyamaprasaṅgasya durvāratvāt | tasmāt sāmagrī janikā, naikaṃ janakam iti sthiravādināṃ manorajyasyāpy aviṣayaḥ | \edlabel{thakur75-90.9}\label{thakur75-90.9} kiṃ kurmo dṛśyate tāvad evam iti cet | dṛśyatāṃ, kiṃ tu pūrvasthitād eva paścāt sāmagrīmadhyapraviṣṭād bhāvāt kāryotpattir anyasmād eva viśiṣṭasāmagrīsamutpannāt kṣaṇād iti vivādapadam etat | tatra prāg api sambhave sarvadaiva kāryotpattir na vā kadācid apīti virodham asamādhāya tata eva kāryotpattir iti sādhyānuvādamātrapravṛttaḥ kṛpām arhati | \edlabel{thakur75-90.14}\label{thakur75-90.14} na ca pratyabhijñānād evaikatvasiddhiḥ, tatpauruṣasya lūnapunarjātakeśakuśakadalīstambādau nirdalanāt | vistareṇa ca pratyabhijñādūṣaṇam asmābhiḥ sthirasiddhidūṣaṇe pratipāditam iti tata evāvadhāryam | \edlabel{thakur75-90.17}\label{thakur75-90.17} nanu kāryam eva sahakāriṇam apekṣate | na tu kāryotpattihetuḥ | yasmād dvividhaṃ sāmarthyaṃ nijam āgantukaṃ ca sahakāryantaram tato akṣaṇikasyāpi kramavatsahakārinānātvād api kramavatkāryanānātvam iti cet | \edlabel{thakur75-90.19}\label{thakur75-90.19} bhavatu tāvat nijāgantukabhedena dvividhaṃ sāmarthyam | tathāpi tat prātisvikaṃ vastusvalakṣaṇaṃ sadyaḥ kriyādharmakam avaśyābhyupagantavyam | tad yadi prāg api, prāg api kāryaprasaṅgaḥ | atha paścād eva, na tadā sthiro bhāvaḥ | \edlabel{thakur75-90.23}\label{thakur75-90.23} na ca kāryaṃ sahakariṇo 'pekṣata iti yuktam, tasyāsattvāt | hetuś ca sann api yadi svakāryaṃ na karoti, tadā tatkāryam eva tan na syāt, svātantryāt | \edlabel{thakur75-90.25}\label{thakur75-90.25} yac coktam – yo hi uttarakāryajananasvabhāvaḥ sa katham ādau kāryaṃ kuryāt, atha kuryāt na tarhi tatkāryakaraṇasvabhāvaḥ, na hi nīlotpādanasvabhāvaḥ pītādikam api karotīti tad asaṅgatam | sthirasvabhāvatve bhāvasyottarakālam evedaṃ na pūrvakālam iti kuta etat | tadabhāvāc ca kāraṇam apy uttarakāryasvabhāvam ity api kutaḥ | \edlabel{thakur75-90.29}\label{thakur75-90.29} kiṃ kurmaḥ, uttarakālam eva tasya janmeti cet | sthiratve tadanupapadyamānam asthiratām ādiśatu | sthiratve 'py eṣa eva svabhāvas tasya yad uttarakṣaṇa eva kāryaṃ karotīti cet | na | pramāṇabādhite svabhāvābhyupagamāyogād iti na tāvad akṣaṇikasya kramikāryakāritvam asti | nāpy akramikāryakāritvasambhavaḥ, dvitīye 'pi kṣaṇe kārakasvarūpasadbhāve punar api kāryakaraṇaprasaṅgāt | \edlabel{thakur75-91.4}\label{thakur75-91.4} kārye niṣpanne tadviṣayavyāpārābhāvād ūnā sāmagrī na niṣpāditaṃ niṣpādayed iti cet | na | sāmagrīsambhavāsambhavayor api sadyaḥ kriyākārakasvarūpasambhave janakatvam avāryam iti prāg eva pratipādanāt | kāryasya hi niṣpāditatvāt punaḥ kartum aśakyatvam eva kāraṇam asamartham āvedayati | \edlabel{thakur75-91.7}\label{thakur75-91.7} tad ayam akṣaṇike kramākramikāryakāritvābhāvo na siddhaḥ | na ca kramākramābhyām aparaprakārasambhavo yena tābhyām avyāptau sandigdhavyatireko hetuḥ syāt | prakārāntaraśaṅkāyāṃ tasyāpi dṛśyatvādṛśyatvaprakāradvayadūṣaṇe 'pi svapakṣe 'py anāśvāsaprasaṅgāt | tasmād anyonyavyavacchedasthitayor nāparaḥ prakāraḥ sambhavati | svarūpāpraviṣṭasya vastuno 'vastuno vātatsvabhāvatvāt | prakārāntarasyāpi kramasvarūpāpraviṣṭatvāt | tathātīndriyasya sahakāriṇo 'dṛśyatve 'py ayogavyavacchedena dṛśyasahakārisahitasya dṛśyasyaiva sattvasya dṛśyakramākramābhyāṃ vyāptiḥ pratyakṣād eva sidhyati | evaṃ kramākramābhyām arthakriyākāritvaṃ vyāptam iti kramākramayor anyonyavyavacchedena sthitatvād etatprakāradvayaparihāreṇārthakriyākāritvam anyatra na gatam ity arthaḥ | ata evaitayor vinivṛttau nivartate || \edlabel{thakur75-91.17}\label{thakur75-91.17} \persName{trilocanasyā}pi vikalpatraye prathamadūṣaṇam āśrayāsiddhidoṣaparihārato nirastam | \edlabel{thakur75-91.18}\label{thakur75-91.18} dvitīyaṃ cāsaṅgatam, vikalpajñānena vyatirekasya pratītatvāt | na hy abhāvaḥ kaścidvigrahavān yaḥ sākṣātkartavyaḥ, api tu vikalpād eva vyavahartavyaḥ | na hy abhāvasya vikalpād anyā pratipattir apratipattir vā sarvathā | ubhayathāpi tadvyavahārahāniprasaṅgāt | evaṃ vaidharmyadṛṣṭāntasya hetuvyatirekasya ca vikalpād eva pratipattiḥ | \edlabel{thakur75-91.22}\label{thakur75-91.22} tṛtīyam api dūṣaṇam asaṅgatam | vyāpakānupalambhena nirdoṣeṇa sattvasya kṣaṇikatvena vyāpter avyāhatatvāt | \edlabel{thakur75-91.23}\label{thakur75-91.23} tad ayaṃ vyāpakānupalambho 'kṣaṇikasyāsattvam sattvasya tato vyatirekaṃ kṣaṇikatvena vyāptiṃ ca sādhayaty ekavyāpārātmakatvād iti sthitam || \edlabel{thakur75-91.25}\label{thakur75-91.25} nanu vyāpakānupalabdhir iti yady anupalabdhimātraṃ tadā na tasya sādhyabuddhijanakatvam avastutvāt | na cānyopalabdhir vyāpakānupalabdhir abhidhātuṃ śakyā bhūtalādivad anyasya kasyacid anupalabdher iti cet | \edlabel{thakur75-91.27}\label{thakur75-91.27} tad asaṅgatam | dharmyupalabdher evānyatrānupalbdhitayā vyavasthāpanāt | yathā hi neha śiṃśapā vṛkṣābhāvād ity atra vṛkṣāpekṣayā kevalapradeśasya dharmiṇa upalabdhir vṛkṣānupalabdhiḥ | śiśapāpekṣayā ca kevalapradeśasya dharmiṇa upalabdhir eva śiṃśapāyā abhāvopalabdhir iti svabhāvahetuparyavasāyivyāpāro vyāpakānupalambhaḥ | tathā nityasya dharmiṇo vikalpabuddhyavasitasya kramikāritvākramikāritvāpekṣayā kevalagrahaṇād eva kramikāritvākramikāritvānupalambhaḥ | arthakriyāpekṣayā ca kevalapratītir evārthakriyāyogapratītir iti vyāpakānupalambhāntarād asya na kaścid viśeṣaḥ ||
	\pend
      

	  \pstart adhyavasāyāpekṣayā ca bāhye 'kṣaṇike vastuni vyāpakābhāvād vyāpyābhāvasiddhivyavahāraḥ | adhyavasāyaś ca samanantarapratyayabalāyātākāraviśeṣayogād agṛhīte 'pi pravartanaśaktir boddhavyaḥ | īdṛśaś cādhyavasāyo 'smaccitrādvaitasiddhau nirvāhitaḥ | sa cāvisaṃvādī vyavahāraḥ parihartum aśakyaḥ | yad vyāpakaśūnyaṃ tadvyāpyaśūnyam iti | etasyaivārthasyānenāpi krameṇa pratipādanāt | ayaṃ ca nyāyo yathā vastubhūte dharmiṇi tathāvastubhūte 'pīti ko viśeṣaḥ | 
	\pend
      

	  \pstart tathā hy ekajñānasaṃsargy atra vikalpya eva | yathā ca hariṇaśirasi tenaikajñānasaṃsargi śṛṅgam upalabdhaṃ śaśaśirasy api tena sahaikajñānasaṃsargitvasambhāvanayaiva śṛṅgaṃ niṣidhyate, tathā nīlādāv apariniṣṭhitanityānityabhāve kramākramau svadharmiṇā sārdham ekajñānasaṃsargiṇau dṛṣṭau, yadi nitye bhavataḥ, nityagrāhijñāne svadharmiṇā nityena sahaiva gṛhye yātām iti sambhāvanayā ekajñānasaṃsargadvārakam eva pratiṣidhyate | kathaṃ punar etasminn ity ajñāne kramākramayor asphuraṇam iti yāvatā kramākramakroḍīkṛtam eva nityaṃ vikalpayām iti cet | ata eva bādhakāvatāro viparītāropam antareṇa tasya vaiyarthyāt | \edlabel{thakur75-92.17}\label{thakur75-92.17} kālāntare 'py ekarūpatayā nityatvam | kramākramau ca kṣaṇadvaye bhinnarūpatayā | tato nityatvasya kramākramikāryaśakteś ca parasparaparihārasthitilakṣaṇatayā durvāro virodha iti kathaṃ nitye kramākramayor antarbhāvaḥ | anantarbhāvāc ca śuddhanityavikalpena dūrīkṛtakramākramasamāropeṇa katham ullekhaḥ | tataś ca pratiyogini nitye 'pi vikalpyamāna ekajñānasaṃsargilakṣaṇaprāpte nityopalabdhir eva nityaviruddhasyānupalabhyamānasya kramākramasyānupalabdhiḥ | tata eva cārthakriyāśakter anupalabdhiḥ | tasmād vyāpakavivekidharmyupalabdhitayā na vyāpakānupalambhāntarād asya viśeṣaḥ || \edlabel{thakur75-92.25}\label{thakur75-92.25} na tv etad avastu dharmitvopayogivastvadhiṣṭhānatvāt pramāṇavyavasthāyā iti cet | kim idaṃ vastvadhiṣṭhānatvaṃ nāma | kiṃ pamparayāpi vastunaḥ sakāśād āgatatvam, atha vastuni kenacid ākāreṇa vyavahārakāraṇatvam, vastubhūtadharmipratibaddhatvaṃ vā | \edlabel{thakur75-93.1}\label{thakur75-93.1} yady ādyaḥ pakṣas tadā kramākramasyārthakriyāyāś ca vyāptigrahaṇagocaravastupratibaddhatvam asyāpi na kṣīṇam | \edlabel{thakur75-93.2}\label{thakur75-93.2} na dvitīye 'pi pakṣe doṣaḥ sambhavati, kṣaṇabhaṅgivastusādhanopāyatvād asya | \edlabel{thakur75-93.3}\label{thakur75-93.3} na cāntimo 'pi vikalpaḥ kalpyate, tasyaiva nityavikalpasya vastuno dharmibhūtasya kramākramavad bāhyanityopādānaśūnyatvenārthakriyāvad bāhyanityopādānaśūnyatve prasādhanāt | paryudāsavṛttyā buddhisvabhāvabhūtākṣaṇikākāre vastubhūte dharmiṇi pratibaddhatvasambhavāt || \edlabel{thakur75-93.7}\label{thakur75-93.7} ayam eva nyāyo na vaktā bandhyāsutaś caitanyābhāvād ityādau yojyaḥ | etena yathā vṛkṣābhāvādir antarbhāvayituṃ śakyate na tathāyam iti \persName{trilocano} 'pi nirastaḥ || \edlabel{thakur75-93.10}\label{thakur75-93.10} na ca kramādyabhāvastrayīṃ doṣajātiṃ nātikrāmati, abhāvadharmatve 'pi āśrayāsiddhidoṣaparihārāt | \edlabel{thakur75-93.11}\label{thakur75-93.11} yat tv anena pramāṇāntarān nityānām asattvasiddhau kramādivirahasyābhāvadharmatā sidhyatīty uktam, tadbālasyāpi durabhidhānam | nityo hi dharmī | asattvaṃ sādhyam | kramikāryakāritvākramikāryakāritvaviraho hetuḥ | asya cābhāvadharmatvaṃ nāmāsattvalakṣaṇasvasādhyāvinābhāvitvam ucyate | tac ca kramākrameṇa sattvasya vyāptisiddhau sattvasya vyāpyasyābhāvena kramākramasya vyāpakasya viraho vyāptaḥ sidhyatīty abhāvadharmatvaṃ prāg eva vidhyor vyāptisādhanāt pratyakṣād anumānād ekasmād vā pramāṇāntarāt siddham iti netaretarāśrayadoṣaḥ | \edlabel{thakur75-93.18}\label{thakur75-93.18} na ca sattāyām ivāsattāyām api tulyaḥ prasaṅgo bhinnanyāyatvāt | vastubhūtaṃ hi tatra sādhyaṃ sādhanaṃ ca | tayor dharmy api vastv eva yujyate | \edlabel{thakur75-93.19}\label{thakur75-93.19} vastunas tu pratyakṣānumānābhyām eva siddhiḥ | tayor abhāve niyamenāśrayāsiddhir iti yuktam | asattāsādhane tv avastudharmo hetur avastuvikalpamātrasiddhe dharmiṇi nāśrayāsiddhidoṣeṇa dūṣayituṃ śakyaḥ | tathākṣaṇikasya kramayaugapadyābhyām arthakriyāvirodhaḥ sidhyaty eva | \edlabel{thakur75-93.22}\label{thakur75-93.22} tathā vikalpād evākṣaṇiko virodhī siddhaḥ | vikalpollikhitaś cāsya svabhāvo nāpara ity api vyavahartavyam | anyathā tadanuvādena kramākramādirahitatvādiniṣedhādikam ayuktam, tatsvarūpasyānullekhād anyasyollekhād ity akṣaṇikaśaśaviṣāṇādiśabdānuccāraṇaprasaṅgaḥ | asti ca | ato yathā pramāṇābhāve 'pi vikalpasiddhasya bandhyāsutādeḥ saundaryādiniṣedho 'nurūpas tathā vikalpopanītasyaivākṣaṇikarūpasya tata eva pratyanīkākāreṇa saha virodhavyavasthāyāṃ kīdṛśo doṣaḥ syāt | yadi cākṣaṇikānubhavābhāvād virodhapratiṣedhas tarhi bandhyāputrādyanubhavābhāvād eva saundaryādiniṣedho 'pi mā bhūt || \edlabel{thakur75-94.5}\label{thakur75-94.5} nanv evaṃ virodhasyāpāramārthikatvam | taddvāreṇa kṣaṇabhaṅgasiddhir apy apāramārthikī syād iti cet | \edlabel{thakur75-94.6}\label{thakur75-94.6} na hi virodho nāma vastvantaraṃ kiñcid ubhayakoṭidattapādasambandhābhidhānam iṣyate 'smābhir upapadyate vā yenaikasambandhino vastutvābhāve 'pāramārthikaṃ syāt | yathā tv iṣyate tathā pāramārthika eva | viruddhābhimatayor anyonyasvarūpaparihāramātraṃ virodhārthaḥ | sa ca bhāvābhāvayoḥ pāramārthika eva | na bhāvo 'bhāvarūpam āviśati, nāpy abhāvo bhāvarūpaṃ praviśatīti yo 'yam anayor asaṃkaraniyamaḥ sa eva pāramārthiko virodhaḥ | kālāntaraikarūpatayā hi nityatvam | kramākramau kṣaṇadvaye 'pi bhinnarūpatayā | tato nityatvakramākramikāryakāritvayor bhāvābhāvavad virodho 'sty eva || \edlabel{thakur75-94.14}\label{thakur75-94.14} nanu nityatvaṃ kramayaugapadyavattvaṃ ca viruddhau dharmau vidhūya nāparo virodho nāma, kasya vāstavatvam iti cet | \edlabel{thakur75-94.15}\label{thakur75-94.15} na | na hi dharmāntarasya sambhavena virodhasya pāramārthikatvaṃ brūmaḥ | kiṃ tu viruddhayor dharmayoḥ sadbhāve | anyathā virodhanāmadharmāntarasambhave 'pi yadi na viruddhau dharmau kva pāramārthikavirodhasambhavaḥ | viruddhau ced dharmau tāvataiva tāttviko virodhavyavahāraḥ kim apareṇa pratijñāmātrasiddhena virodhanāmnā vastvantareṇa | \edlabel{thakur75-94.20}\label{thakur75-94.20} tad ayaṃ pūrvapakṣasaṃkṣepaḥ gnityaṃ nāsti na vā pratītiviṣayas tenāśrayāsiddhatā hetoḥ svānubhavasya ca kṣatir ataḥ kṣiptaḥ sapakṣo 'pi ca | śūnyaś ca dvitayena sidhyati na cāsattā 'pi sattā yathā no nityena virodhasiddhir asatā śakyā kramāder api || J 89,16-19; cf. R 87,24-27 iti | \edlabel{thakur75-94.25}\label{thakur75-94.25} atra siddhāntasaṃkṣepaḥ hdharmasya kasyacid avastuni mānasiddhā bādhāvidhivyavahṛtiḥ kim ihāsti no vā | kvāpy asti cet katham iyanti na dūṣaṇāni nāsty eva cet svavacanapratirodhasiddhiḥ || J 89,21-24; cf. R 88,4-7 \edlabel{thakur75-95.1}\label{thakur75-95.1} tad evaṃ nityaṃ na kramikāryakāritvākramikāryakāritvayogīti paramārthaḥ | tataś ca sattāyuktam api naiveti paramārthaḥ | tataś ca kṣaṇikākṣaṇikaparihāreṇa rāśyantarābhāvād akṣaṇikān nivartamānam idaṃ sattvaṃ kṣaṇika eva viśrāmyat tena vyāptaṃ sidhyatīti sattvāt kṣaṇikatvasiddhir avirodhinī ||
	\pend
      
	    
	    \stanza[\smallbreak]
	prakṛtiḥ sarvadharmāṇāṃ yad bodhān muktir iṣyate |&sa eva tīrthyanirmāthī kṣaṇabhaṅgaḥ prasādhitaḥ ||\&[\smallbreak]


	

	  \pstart iti kṛtir ayaṃ Ratnakīrteḥ ||
	\pend
      
	    
	    \endnumbering% ending numbering from div
	    \endgroup
	    
	  \label{Pramāṇāntarbhāvaprakaraṇam}
	  
	% new div opening: depth here is 0
	
	    
	    \begingroup
	    \beginnumbering% beginning numbering from div depth=0
	    
	  
\chapter[{Pramāṇāntarbhāvaprakaraṇam}]{Pramāṇāntarbhāvaprakaraṇam}\label{Pramāṇāntarbhāvaprakaraṇam}
	    
	    \stanza[\smallbreak]
	\label{thakur75-96.4}\flagstanza{\tiny\textenglish{...5-96.4}}pramāṇadvitayād\edtext{}{\lemma{pramāṇadvitayād}\Bfootnote{Chapter starts in .}} anyapramāṇagaṇadūṣaṇam |&nāpūrvam ucyate tat tu prayogeṇātra mudryate ||\&[\smallbreak]


	

	  \pstart iha khalu pramāṇamātre na kecid vipratipadyante | antataś \name{cārvākasyā}pi saṃpratipatteḥ | \persName{pramāṇamātrocchedavādī} ca tattadāṅśakya pratividhānād \persName{asmadgurubhir} avajñātaḥ
	\pend
      
	    
	    \stanza[\smallbreak]
	pramāṇam apramāṇaṃ ced vicārāvasaro hataḥ |&bruvatā niyataṃ kiñcit sādhyaṃ vā bādhyam eva vā ||&tatrāyuktiṃ bruvāṇasya ślāghā sadasi kīdṛśī |&nānumāyāḥ parā yuktiḥ kiṃ siddhaṃ tadanādare ||&svīkṛtā tena sety asmāt tanmatyā bādhanaṃ yadi |&abādhane 'syāḥ svīkārāt tadbhiyā bādha\leavevmode\ledsidenote{\textenglish{pb in\cite{RNAms}}}\label{RNAms_51b}naṃ katham ||&sādhyaṃ na kiñcid iti cet bādhāyā api sādhyatā |&sāpi neti vaco vyarthaṃ praśnamātre 'pi kiṃ phalam ||&phalaṃ yadi giraḥ kvāpi nānyat tac cāvabodhanāt |&vācaḥ pratyāyane śaktā nākṣadhūmādi sundaram ||&saṃvṛtau mānam iṣṭaṃ ced vicāro 'py eṣa saṃvṛtiḥ |&saṃvṛtāv api neṣṭaṃ ced vadan jetā yathā tathā || \edtext{}{\lemma{||}\Bfootnote{(JNA 363f.)}}&saṃvṛtiś ca vinā mānaṃ vāṅmātreṇa na sidhyati |&mānato yadi durvāraḥ pramāṇasya parigrahaḥ ||\&[\smallbreak]


	

	  \pstart \persName{ācāryo} 'py āha—
	\pend
      
	    
	    \stanza[\smallbreak]
	aniṣṭeś cet pramāṇaṃ hi sarveṣṭīnāṃ nibandhanam |&bhāvābhāvavyavasthāṃ kaḥ kartuṃ tena vinā prabhuḥ ||\edtext{\textsuperscript{*}}{\lemma{*}\Bfootnote{(PV IV 215)}}\&[\smallbreak]


	

	  \pstart iti |
	\pend
      

	  \pstart tad evaṃ pramāṇamātrāpratikṣepe pratyakṣaṃ tāvad ādau gaṇanīyam, tanmūlatvād aparapramāṇopapatteḥ | na ca cārvāko 'py anumānam anavasthāpya sthātuṃ prabhavati, vyāpāratrayakaraṇāt | 
	\pend
      

	  \pstart tac chāstre hi pratyakṣetarasāmānyayoḥ pramāṇetaravidhānaṃ lakṣaṇapraṇayanato vidhātavyam | tac ca lakṣaṇaṃ pratyakṣe dharmiṇi lakṣye prāmāṇye pratyetavye svabhāvo hetuḥ | parabuddhipratipattau ca kāryādivyāpāraḥ kāryahetuḥ | paralokapratiṣedhe ca dṛśyānupalambho 'ṅgīkartavya iti katham anumānāpalāpaḥ | yad ācāryaḥ 
	\pend
      
	    
	    \stanza[\smallbreak]
	pramāṇetarasāmānyasthiter anyadhiyo gateḥ |&pramāṇāntarasadbhāvapratiṣedhāc ca kasyacit ||\edtext{\textsuperscript{*}}{\lemma{*}\Bfootnote{\edlabel{quoted-pvin1-kārika-2}\textit{pramāṇetarasāmānyasthiter anyadhiyo gateḥ /} \textit{pramāṇāntarasadbhāvaḥ pratiṣedhāc ca kasyacit //}  [[(PVin I 2)]]}}\&[\smallbreak]


	

	  \pstart api ca
	\pend
      
	    
	    \stanza[\smallbreak]
	arthasyāsambhave 'bhāvāt pratyakṣe 'pi pramāṇatā |&pratibaddhasvabhāvasya taddhetutve samaṃ dvayam ||\edtext{\textsuperscript{*}}{\lemma{*}\Bfootnote{\edlabel{quoted-pvin1-kārika-3}\textit{arthasyāsambhave 'bhāvāt pratyakṣe 'pi pramāṇatā /} \textit{pratibaddhasvabhāvasya taddhetutve samaṃ dvayam //}  (PVin I 3)}}\&[\smallbreak]


	

	  \pstart ity anumānam api pramāṇam | prāmāṇyaṃ ca pramāṇāntarāgṛhītaniścitapravṛttiviṣayārthatayā \edlabel{ratnakīrtinibandhāvali__36r1PF7IMT3EQQQQAHQ1C6G66VT}\label{ratnakīrtinibandhāvali__36r1PF7IMT3EQQQQAHQ1C6G66VT}\edtext{}{\lemma{tatprāpaṇe}\xxref{ratnakīrtinibandhāvali__36r1PF7IMT3EQQQQAHQ1C6G66VT}{ratnakīrtinibandhāvali__36r1PF7IMT1V4QEHAA34ET9TA1F}\Afootnote{tatprāpaṇe \cite{Thakur} ; tatpra\add{ā}paṇe\footnote{The ā-marker was written above the letter.} \cite{RNAms}   {\rmlatinfont [App type: corr]}}}tatprāpaṇe\edlabel{ratnakīrtinibandhāvali__36r1PF7IMT1V4QEHAA34ET9TA1F}\label{ratnakīrtinibandhāvali__36r1PF7IMT1V4QEHAA34ET9TA1F} śaktiḥ ||
	\pend
      

	  \pstart nanv astu prāpaṇe śaktiḥ prāmāṇyam, paramasaunārthād utpatteḥ, api tv arthadarśanād iti cet | kim idam arthadarśanam | arthasya dharmo dṛśyatvam | jñānasya dharmo draṣṭṛtvam | prathamapakṣe nīlatvavad dṛśyatvasyāpi sādhāraṇatvād ekagocaro 'rthaḥ sarvagocaraḥ syāt | na hi pratipuruṣam arthānāṃ bhedo nairātmyaprasaṅgāt | dvitīyapakṣe tu katham anyasmin jñānasvabhāve draṣṭṛtve saty anyasyāsambaddhasyārthasya pratyāśā syāt | draṣṭṛatvaṃ dṛśyatvam antareṇānupapadyamānaṃ tadākṣipatīti cet | nanu jñānārthayor utpattisārūpyabalato draṣṭḥdṛśyatvavyavasthāpanam etat | anabhyupagame draṣṭṛtvaṃ dṛśyatvaṃ ca na sambhavatīti kiṃ kenākṣipyatām | bhavatu vā prakārāntareṇāpi draṣṭṛdṛśyabhāvas tathāpi bhede saty avyabhicā\edlabel{ratnakīrtinibandhāvali__36r1PF7IMT0CNEF991ZUWMJP2TL}\label{ratnakīrtinibandhāvali__36r1PF7IMT0CNEF991ZUWMJP2TL}\edtext{}{\lemma{ra du}\xxref{ratnakīrtinibandhāvali__36r1PF7IMT0CNEF991ZUWMJP2TL}{ratnakīrtinibandhāvali__36r1PF7IMSYSOSGMXQFQBZSPM0O}\Afootnote{ra\deletion{\unclear{ḥ}\gap{}\add{\unclear{}}}du \cite{RNAms} ; ras ta \cite{thakur75}   {\rmlatinfont [App type: var]}}}ra du\edlabel{ratnakīrtinibandhāvali__36r1PF7IMSYSOSGMXQFQBZSPM0O}\label{ratnakīrtinibandhāvali__36r1PF7IMSYSOSGMXQFQBZSPM0O}tpattir eva prāptinimittam | sā ca prāpaṇaśaktiḥ pratyakṣānumānayor aviśiṣṭeti pramāṇe eva | \edlabel{thakur75-97.23}\label{thakur75-97.23} nanv anyad api śābdopamānādikaṃ pramāṇam asti | tathā hi śabdāc codanārūpād asannikṛṣṭe 'rthe svargādau yaj jñānam utpadyate tad api śābdaṃ jñānaṃ pramāṇam eva | pratyayitoditavākyaprasūtaṃ ca jñānaṃ pramāṇam | yad āha Kumārilaḥ
	\pend
      
	    
	    \stanza[\smallbreak]
	tac cākartṛkato vākyād anyād vā pratyayito[?]ditāt |\edtext{\textsuperscript{*}}{\lemma{*}\Bfootnote{Find this! (Not in e-text of śv.)}}\&[\smallbreak]


	

	  \pstart iti |
	\pend
      

	  \pstart tatra yadā śabdasamutthaṃ jñānaṃ pramāṇaṃ tadopādānādibuddhiḥ phalam | yadā tu śabdas tadā tadālambanaṃ jñānaṃ phalam iti Naiyāyikasya punaḥ: āptopadeśaḥ śabdaḥ \edtext{}{\lemma{śabdaḥ}\Bfootnote{(NSū 1.1.7)}}, iti śabdapramāṇalakṣaṇasūtram | tatra śabda iti lakṣyapadam | āptopadeśa iti lakṣaṇapadam | asyāyaṃ saṃkṣepārthaḥ | āptopadiṣṭaḥ śabdaḥ pramāṇam iti | āptaś ca sākṣātkṛtaheyopādeyatattvo yathādṛṣṭasya cārthasyācikhyāsayā prayukta upadeṣṭā abhidhīyate | pramāṇaphalavyavasthā ca pūrvavad draṣṭavyeti |
	\pend
      

	  \pstart tathā \name{Mīmāṃsakānām} upamānaṃ pramāṇam | yad uktaṃ \persName{Śabarasvāminā} \edlabel{kāśikā1_start}\label{kāśikā1_start}upamānam api sādṛśyam \edlabel{ratnakīrtinibandhāvali__36r1PF7IMSX4WQH5NSCHU1A53RC}\label{ratnakīrtinibandhāvali__36r1PF7IMSX4WQH5NSCHU1A53RC}\edtext{}{\lemma{asannikṛṣṭe 'rthe}\xxref{ratnakīrtinibandhāvali__36r1PF7IMSX4WQH5NSCHU1A53RC}{ratnakīrtinibandhāvali__36r1PF7IMSVM9L5FKHNUR7OH8NM}\Afootnote{ \cite{thakur75,śabara_bhāṣya} °asannikṛṣṭatve \cite{RNAms}   {\rmlatinfont [App type: emendation]}}}asannikṛṣṭe 'rthe\edlabel{ratnakīrtinibandhāvali__36r1PF7IMSVM9L5FKHNUR7OH8NM}\label{ratnakīrtinibandhāvali__36r1PF7IMSVM9L5FKHNUR7OH8NM} 'rthe buddhim utpādayati | yathā gavayadarśanaṃ goḥ smaraṇasyeti |\edtext{}{\lemma{|}\Bfootnote{Cf. \cite{śabara_bhāṣya}: upamānam api sādṛśyam asaṃnikṛṣṭe 'rthe buddhim utpādayati, yathā gavayadarśanaṃ gosmaraṇasya.}}
	\pend
      

	  \pstart asyāyam arthaḥ | ekatra dṛśyamānaṃ sādṛśyaṃ kartṛ | pratiyogyantare dṛśyamānapratiyogisādṛśyaviśiṣṭatayaitatsādṛśyaviśiṣṭo 'sau ity asannikṛṣṭe 'rthe yāṃ buddhim utpādayati tadupamānaṃ pramāṇam iti yat tadoradhyāhāra iti |\edlabel{kāśikā1_end}\label{kāśikā1_end}\edtext{}{\lemma{---}\Afootnote{upamānam api sādṛśyam asannikṛṣṭe 'rthe buddhim utpādayati. yathā gavayadarśanaṃ gosmaraṇasyeti bhāṣyam. asyāyaṃ tātparyārthaḥ -- upamānam api na parīkṣaṇīyam, evaṃ lakṣaṇakatvenāvyabhicārād iti. avayavārthas tv ekatra dṛśyamānaṃ sādṛśyaṃ pratiyogyantare dṛśyamānapratiyogisādṛśyaviśiṣṭatayāsannikṛṣṭe 'rthe yāṃ buddhim utpādayati etatsādṛśyaviśiṣṭo 'sāv iti, sopamānam iti yattadoradhyāhāraḥ. na ca vācyaṃ viṣayaviśeṣānupādānāt kathaṃ sādṛśyaviśiṣṭaviṣayā buddhir avagamyata iti, prasiddhapramāṇānuvādena hy atrāparīkṣā pratipādyate. loke ca sādṛśyaviśiṣṭaviṣayaiva \cite{sucarita}   {\rmlatinfont [App type: parallel]}}} tasmāt samaratīti smaraṇaṃ puruṣaḥ | tenāyam arthaḥ - yathā gavaye dṛśyamānaṃ sādṛśyaṃ gāṃ smarato manuṣyasya etatsādṛśyaviśiṣṭo 'sau gaur iti buddhim utpādayatīti |
	\pend
      

	  \pstart na cedam upamānaṃ smaraṇaṃ kartavyam, gavayasādṛśyaviśiṣṭasya gor goviśiṣṭasya ca sādṛśyasya prameyatvāt | gosādṛśyayor viśeṣaṇaviśeṣyabhāvasyopamānapramāṇaviṣayasya gogrāhiṇā gavayagrāhiṇā vā pratyakṣeṇa kenacid agrahaṇāt | yad āha Bhaṭṭaḥ
	\pend
      
	    
	    \stanza[\smallbreak]
	\&[\smallbreak]


	

	  \pstart na ca grahaṇam antareṇa smaraṇam asti | tasmān nopamānaṃ smaraṇam ataḥ pramaṇam iti | Naiyāyikādīnāṃ tūpamānasūtram,
	\pend
      
	    
	    \stanza[\smallbreak]
	prasiddhasādharmyāt sādhyasādhanam upamānam iti |\edtext{\textsuperscript{*}}{\lemma{*}\Bfootnote{(NSū 1.1.6)}}\&[\smallbreak]


	

	  \pstart asyāyam arthaḥ | prasiddhaṃ sādharmyaṃ yasya tasmād gavayādeḥ sādhyasya saṃjñāsaṃjñisambandhasya sādhanaṃ siddhis tadupamānaphalam | samākhyāsambandhapratipattihetur upamānam ity arthaḥ | ayam asya prapañcaḥ | yaḥ pratipattā gāṃ jānāti na gavayam, ādiṣṭaś ca svāminā gacchāraṇyaṃ gavayamānayāsmād iti, gavayaśabdavācyam artham ajānāno vanecaram anyaṃ vā tajjñaṃ pṛṣṭavān, bhrātaḥ kīdṛśo gavaya iti | tena cādiṣṭaṃ yathā gaus tathā gavaya iti | tasya śrutātideśavākyasya kasyāñcid araṇyānyām upagatasyātideśavākyārthsmaraṇasahakāri yad gavayasārūpyajñānaṃ tatprathamata evāsau gavayaśabdavācyo 'rtha iti pratipattiṃ prastuvānam upamānaṃ pramāṇam iti |
	\pend
      

	  \pstart tathārthāpattisaṃjñaṃ pramāṇaṃ mīmāṃsakasya | arthāpattir api dṛṣṭaḥ śruto vārtho 'nyathā nopadyamāno yad arthāntaraṃ parikalpayati sārthāpattiḥ | yathā jīvati devadatte gṛhābhāvadarśanena bahirbhāvasyārthasya parikalpanā | asyāyam arthaḥ | pratyakṣādibhiḥ ṣaḍbhiḥ pramāṇaiḥ prasiddho yo 'rthaḥ sa yena vinā na yujyate tasyārthasya kalpanam arthāpattir iti | sā ca ṣaṭpramāṇapūrvikā ṣaṭprakāraiveti ||
	\pend
      

	  \pstart pratyakṣānumānādipramāṇapañcakābhāvasvabhāvam abhāvākhyaṃ pramāṇam | prameyaṃ ghaṭādyabhāvaḥ | nāstīha ghaṭādīti jñānaṃ ghaṭādyabhāvālambanaṃ phalam | yadāha Kumārilaḥ
	\pend
      
	    
	    \stanza[\smallbreak]
	\&[\smallbreak]


	\footnote{\begin{english}(ŚV XIII 11; 1)\end{english}}

	  \pstart iti | 
	\pend
      

	  \pstart etāni ṣaṭ pramāṇāni pratyakṣādīny asaṃkīrṇasvasvalakṣaṇayogitvād anyāpraviṣṭasvabhāvāni pratyetavyānīti ||
	\pend
      

	  \pstart atrocyate | codanāyās tāvad bāhye 'rthe pratibandhābhāvān na prāmāṇyam | prayogaḥ - yasya yatra pratibandho nāsti na tasya tatra prāmāṇyam | yathā dahane 'pratibaddhasya rāsabhasya | apratibaddhāś ca bahirarthe vaidikāḥ śabdāḥ iti vyāpakānupalabdhiḥ | na tāvad ayam asiddho hetuḥ | śabdānāṃ vastutaḥ pratibandhābhāvāt | pratibaddhasvabhāvatā hi pratibandhaḥ | na ca sā nirnibandhanā, sarveṣāṃ sarvatra pratibaddhasvabhāvatāprasaṅgāt | nibandhanaṃ cāsyās tādātmyatadutpattibhyām anyan nopalabhyate, atatsvabhāvasyātadutpatteś ca tatrāpratibaddhasvabhāvatvāt | na hi śabdānāṃ bahirarthasvabhāvatāsti bhinnapratibhāsāvabodhaviṣayatvāt | nāpi śabdā bahirarthād upajāyante, artham antareṇāpi puruṣasyecchāpratibaddhavṛtteḥ śabdasyotpādadarśanāt |
	\pend
      

	  \pstart nanu yogyatayaiva kiñcit pratibaddhasvabhāvam upalabhyate | yathā cakṣur indriyaṃ rūpe | cakṣuḥ khalu vyāpāryamāṇaṃ rūpam evopalabhyati | tathaivaite vaidikāḥ śabdās tādātmyatadutpattiviyuktā api yogyatāmātreṇātīndriyam arthaṃ bodhayiṣyanti tat kathaṃ tādātmyatadutpattivirahamātreṇāpratibandho yenaivaṃ vyāpakānupalabdhiḥ sidhyatīti | naiṣa doṣaḥ | yataś cakṣur indriyam api rasādiparihāreṇa rūpa eva prakāśakatvena pratiniyataṃ tatkāryatvāt | rūpaṃ hi cakṣur upakaroti | na sattāmātreṇa cakṣū rūpaṃ prakāśayati, vyavahitasyāpi rūpopalabdhiprasaṅgāt | tasmād rūpād yogyadeśasannihitāt tajjñānajananayogyatām āsādya cakṣū rūpajñānam utpādayattatkāryam iti vyaktam avasīyate | anyathā tadupakārānapekṣasya tasyāpi tatprakāśananiyamo nopapadyate | na hy anupakāryatvāviśeṣe cakṣū rūpasyaiva prakāśakam, na rasāder iti ghaṭām upaiti niyamaḥ | ayam eva tarhi niyamaḥ kuto yad rūpeṇaiva cakṣur upakartavyam, na rasādineti | yadi vastuvaśād eva rūpam upakaroti na rasādikam, hanta tarhi yathopakāryatvaṃ prati niyamaś cakṣuṣo rūpeṇa, tathā śabdānām api svābhāvika evāstu bahirarthapratyāyananiyama iti |
	\pend
      

	  \pstart atrocyate | na cakṣuṣaḥ svābhāviko rūpopakāryatāniyamaḥ, kasyacid vastunaḥ svābhāvikatvānupapatteḥ | tathā hi svābhāvikatvaṃ vastudharmasyānujānānaḥ praṣṭavyaḥ - kiṃ svābhāvika iti svato bhavati, āhosvit parataḥ, athāhetutaḥ | yadi svato bhavati, tad asaṅgatam, svātmani kriyāvirodhāt | athāhetutaḥ, tad ayuktam, ahetor deśādiniyamāyogāt | tasmān na svābhāviko rūpopakāryatāpratiniyamaś cakṣuṣaḥ | kiṃnibandhanas tarhi svahetupratibaddha iti, brūmaḥ - cakṣuḥ khalu svahetunā janyamānaṃ tādṛśam eva janitam yadrūpopakartavyam eva bhavati | rūpam api tādṛśam eva svahetunā janitaṃ yat tad upakārakasvabhāvam |
	\pend
      

	  \pstart śabdānām api sa svabhāvaḥ svahetupratibaddho yenaite bāhyārthāvyabhicāriṇa iti cet | na śakyam evam abhidhātum, nityatvābhyupagamād vedavākyānām | athānityatvam abhyupagamyāyam ākṣepaḥ parihartum iṣyate, tad api duṣkaram, doṣāntaraprasaṅgāt | yadi svahetunaiva te niyamārthopadarśanaśaktimanto janitāḥ, tadāvyutpannasamayasyāpi svārtham avabodhayeyuḥ | yathā cakṣuḥ svaheto rūpaprakāśakam utpannaṃ sat prakāśayaty eva rūpam asaṅketavido 'pi, na ca śabdād uccaritāt prāgapratītasamayasyāpi viśeṣāvagamaḥ samasti | tasmān na svahetupratibaddhaś cakṣurāder iva śabdānām arthapratipādananiyama iti niścayaḥ || 
	\pend
      

	  \pstart atha svahetubhir evāyam īdṛśas teṣāṃ svabhāvo datto yena te saṃketaviśeṣasahāyā eva kam apy artham avabodhayanti | na tarhi saṅketaparāvṛttau padārthāntaravṛttayo bhaveyuḥ | yadi hy ayam agnihotraśabdaḥ saṃketāpekṣo yāgaviśeṣapratipādakaḥ, kathaṃ saṅketānyatvenārthāntaraṃ pratipādayati | na hi kṣityādyapekṣeṇa bījena svahetor aṅkurajananasvabhāvenotpannena rāsabhaḥ śakyo janayitum, tathā śabdo 'pi yad arthapratipādananiyatas tam eva prakāśayet || 
	\pend
      

	  \pstart atha tattatsaṅketāpekṣas tattadarthapratyāyanayogya evāyaṃ jāta ity ucyate | tad api na prasutopayogi | na hy evam asya prāmāṇyam avatiṣṭhate | yadā hi saṅketenāpuruṣārthapratipādanam api sambhāvyata eva, tadā na śakyam upakalpayituṃ kim ayam abhimatasyaivārthasya dyotako na veti | tarhi vācyavācakalakṣaṇaḥ śabdārthayoḥ sambandho bhaviṣyati | tathā cāha
	\pend
      
	    
	    \stanza[\smallbreak]
	vācyavācakasambandhāḥ santi yady api vāstavāḥ |&saṅketair anabhivyaktā na te 'rthavyaktihetavaḥ ||\&[\smallbreak]


	

	  \pstart iti cet | nanu tasya vāstavatve 'saṅketavido 'py arthapratipattir bhaved ity uktam, saṅketāpekṣāyāṃ cārthāntare na pravartetetyādyabhihitam | ataḥ pūrvam evāyaṃ pratyākhyāto vācyavācakalakṣaṇaḥ sambandhaḥ | tasmān na bahirarthe pratibandhaḥ śabdānām iti nirṇayaḥ ||
	\pend
      

	  \pstart tataś ca nāsiddho hetuḥ ||
	\pend
      

	  \pstart nāpy viruddhaḥ, viparyayavyāptyabhāvāt | tadabhāvaś ca sapakṣe vṛttyupadarśanāt | na hi viruddhasya sādharmyavati dharmiṇi sadbhāvo yuktaḥ, sādhyaviparyayasya tatrābhāvāt | na ca vyāpakam antareṇa vyāpyasya sambhavaḥ, tatpracyutiprasaṅgāt || 
	\pend
      

	  \pstart nāpy anaikāntiko hetuh, viparyaye bādhakapramāṇasambhavāt | prāmāṇyapratiṣedhe hi sādhye prāmāṇyam eva vipakṣaḥ | na ca tasmin pratibandhābhāvalakṣaṇo hetur asti, svaviruddhena pratibandhena vyāptatvāt | na khalv ayaṃ prādeśikaḥ pramāṇaśabdo jñāneṣu nirnibandhana eva, sarvajñāneṣu prāmāṇyavyapadeśaprasaṅgāt | nibandhanaṃ ca svaviṣayapratibandhād anyan nopapadyate | tasmāt pramāṇasya pramāṇavyapadeśaviṣayatvaṃ svaviṣayapratibandhena vyāptam | ataḥ pramāṇe dharmiṇi vipakṣe prāmāṇyasya viruddhavyāptasyopalambhena vipakṣe vyavacchedasiddher nānaikāntiko hetuḥ |
	\pend
      

	  \pstart na cānyo doṣaḥ sambhavī | tasmān nirastāśeṣadoṣeṇa hetunā yat prasiddhaṃ tad upādeyam eva satām iti paṇḍitaśrījitāripādair eva vedāprāmāṇye darśitam | 
	\pend
      

	  \pstart evaṃ ca vaidikaśabdānāṃ prāmāṇye niraste tadutthaṃ jñānam apy apramāṇam eva | āptapraṇītasya punar vacanasyārthāvyabhicāre tajjanmano jñānasyāvyabhicārasambhave 'pi na prāmāṇyam upagantuṃ śakyate, paracittavṛttīnām aśakyaniścayatvenāptatvāparijñānāt vacanasyāpi tatpraṇītatvāpratipatteḥ | prayogaś cātra -
	\pend
      

	  \pstart yad yena rūpeṇa na niścitaṃ na tat tena rūpeṇa vyavahriyate | yathā rathyāpuruṣaḥ sarvajñatvena | na pratīyate cābhimatapuruṣa āptatveneti vyāpakānupalabdhiḥ || 
	\pend
      

	  \pstart nāyam asiddhiḥ, āptābhimatasya tathātvāniścayāt | tathā hi paracittavṛttayo 'tīndriyatvān na pratyakṣasamadhigamyā iti kāyavāgvyavahārato 'numātavyāḥ | tau ca kāyavāgvyavahārau buddhir pūrvam anyathāpi kartuṃ śakyate | tatas tatpratibaddhatvenāniścayāt kathaṃ kāyavāgvyavahārato viśiṣṭaparacittavṛttyanumānam ||
	\pend
      

	  \pstart nāpi viruddhaḥ, sapakṣe sadbhāvasambhavāt ||
	\pend
      

	  \pstart nāpy anaikāntikaḥ, prāmāṇikatadrūpavyavahartavyatvaniścitatvayor vyāpyavyāpakabhūtayor vidhibhūtayor vṛkṣatvaśiṃśapātvayor iva pratyakṣānupalambhābhyāṃ sarvopasaṃhāreṇa vyāpteḥ siddhatvāt | tad ataḥ sādhanād doṣatrayarahitāt sādhyaṃ siddhyad avācyam eva | tad evam āptatvasya durbodhatvena tatpraṇītatvāniścayād ekaprahāranihatam āptavacasaḥ prāmāṇyam |
	\pend
      

	  \pstart ato yad etasya prāmāṇyaprasiddhyarthaṃ vācaspatiprabhṛtīnāṃ valgitaṃ tadaprāptāvasaram eva | evaṃ pratyayoditam api bhaṭṭābhimataṃ śābdaṃ prāmāṇyaṃ vyastam iti boddhavyam | tasmāt sthitam etat na śābdaṃ bahirarthe pramāṇam astīti | buddhyākāre tu tatkāryaprasūtatvāt tadanumānam eveti |
	\pend
      

	  \pstart mīmāṃsakoktaṃ tāvad upamānaṃ mānam eva na bhavati, nirviṣayatvād asya | ihāpi prayogaḥ - yasya na viṣayavattvaṃ na tasya prāmāṇyam | yathā keśoṇḍukajñānasya | na siddhaṃ ca viṣayavattvam upamānajñānasyeti vyāpakānupalambhaḥ |
	\pend
      

	  \pstart nāyam asiddho hetuḥ, nirviṣayatvād upamānasya | tathā hi sādṛśyaviśiṣṭaḥ piṇḍaḥ piṇḍaviśiṣṭaṃ vā sādṛśyam upamānasya viṣayo varṇyate | na sadṛśavastuvyatiriktaṃ sādṛśyaṃ vyavasthāpayituṃ śakyate, pramāṇenāpratītatvāt | \edlabel{thakur75-102.14}\label{thakur75-102.14} nanu sādṛśyaṃ vastu durvāram eva | yadāha
	\pend
      
	    
	    \stanza[\smallbreak]
	sādṛśyasya ca vastutvaṃ na śakyam apabādhitum | &bhūyo 'vayavasāmānyayogo jātyantarasya tat || \edtext{}{\lemma{||}\Bfootnote{(ŚV XIII 18)}}\&[\smallbreak]


	

	  \pstart iti |
	\pend
      

	  \pstart atrocyate | yadi sadṛśātiriktaṃ sādṛśyaṃ vastu dṛśyaṃ syāt, tadā dṛśyānupalambhagrastam eva, śāstrānāhitasaṃskāreṇāpi kenacit tasyādarśanāt | tasya cāstitve sarvaṃ sarvatrāstīty apravṛttinivṛttikaṃ jagadāpadyeta | athādṛśyaṃ tatsādṛśyam upeyate, tathāpi tatra prasiddhaliṅgābhāvād asiddham eva | siddhena ca tena viṣayavattopamānasya sidhyeta | sādṛśyapratyayas tu svahetos tathotpannena sadṛśavastunāpi kriyamāṇo ghaṭata eva iti na sādṛśyam upsthāpayituṃ prabhavati | upamānapramāṇabalād eva sādṛśyasiddhir iti cet | na | pramāṇāntarasiddhayor eva sādṛśyapiṇḍayor viśeṣaṇaviśeṣyabhāvasyopamānaviṣayatvāt kathaṃ sādṛśyamātrasyopamānāt siddhiḥ | tataś ca sādṛśyasyāsiddher na tadviśiṣṭaḥ piṇḍaḥ piṇḍaviśiṣṭaṃ vā sādṛśyam upamānasya viṣayaḥ | tad evam upamānasya nirviṣayatvaṃ siddham iti nāsiddho hetuḥ | nāpi viruddhaḥ, sapakṣe bhāvāt |
	\pend
      

	  \pstart na cānaikāntikaḥ | tathā hi prāmāṇyānbhāve sādhye prāmāṇyam eva vipakṣaḥ | tac ca viṣayavattayā vyāptam, nirnimittatve sarvajñānaprāmāṇyaprasaṅgāt | tad yaṃ viruddhavyāptopalabdhyā vipakṣān nivartamāno viṣayavattvābhāvalakṣaṇo hetuḥ prāmāṇyābhāvalakṣaṇa eva viśrāmyatīti vyāptisiddhiḥ | ato nopamānaṃ pramāṇam iti |
	\pend
      

	  \pstart naiyāyikaparikalpitopamānanirākaraṇārtham apy ayam eva prayogo draṣṭavyaḥ, tasyāpi nirviṣayatvāt | tathā hi samākhyāsambandhas tasya viṣayo varṇyate | sa ca paramārthato nāsti | sa hi sambandhaḥ sambandhibhyāṃ bhinno 'bhinno vā | yadi bhinnas tadā tayor iti kutaḥ | na ca sambandhāntarād iti vaktavyam, tad api kathaṃ teṣām iti cintāyām anavasthāprasaṅgaḥ | na ca yathā pradīpaḥ prakāśāntaram antareṇa prakāśate tathā sambandho 'pi sambandhāntareṇa sambaddho bhaviṣyatīti vaktum ucitam | pramāṇasiddhe hi vasturūpe 'yam asya svabhāva iti varṇyate | yathā pradīpasyaiva | sambandhas tu na pramāṇapratītaḥ | tat ka evaṃ jānātv ayam asya svabhāva iti, yad vā nāsty evāyam iti | ayam anayoḥ sambandhaḥ sambaddhāv etāv iti tu buddhiḥ svahetubalāt sambaddhavastudvayād api sambhāvyamānā na sambandham ākṣeptuṃ prabhavati | tasmān na bhinnasambandhasiddhiḥ | athābhinnaḥ tadā sambandhināv eva kevalāv iti na samākhyāsambandho nāma, yaḥ kaścid upamānasya viṣayaḥ syāt | nanu sambandhabuddhijanakatvaṃ sambaddhapadārthād bhinnam abhinnaṃ vā | bhede ca sa eva sambandhaḥ nāmni paraṃ vivādaḥ | athābhinnam, tadā yathā sambaddhapadārthasya svabhāvaḥ sarvapadārthasādhāraṇas tathā tad api rūpaṃ tadavyatibhinnaṃ sarvapadārthasādhāraṇam iti sa padārtho 'bhimatapadārtheneva parair api padārthaiḥ saha sambaddhaḥ syāt |
	\pend
      

	  \pstart na caivam, tasmād bhinnaṃ tatsambandhabuddhijanakatvaṃ sambaddhapadārthād eṣṭavyam iti cet | nanv etad āśaṅkya Rājakulapādaiḥ parihṛtam eva | tathā hi
	\pend
      
	    
	    \stanza[\smallbreak]
	sambaddhaṃ svayam eva cen nanu yathā taṃ tasya sambandhinaṃ pratyātmā jagatīm api prati tathā tat kena yogo 'sya na |&sambandhe parato 'pi tulyam akhilaṃ tenaiva cet saṃyamo hetuḥ kiṃ na niyāmakaḥ sa ca kathaṃ yogaḥ kvacin nāpare ||\&[\smallbreak]


	

	  \pstart iti | tasmāt sambandhābhāvāt pūrvoktena nyāyena sārūpyābhāvāc cāsiddhaṃ naiyāyikasyāpi nirviṣayam upamānaṃ pramāṇam ato 'nantareṇaiva vyāpakānupalambhena nirākṛtam |
	\pend
      

	  \pstart arthāpattir api | yad etat sāmānyalakṣaṇaṃ pratyakṣādipratīto yo 'rthaḥ sa yena vinā nopapadyate tasyārthasya parikalpanam athāpattir ity atra vicāryate | yasyārthasya darśanād yo 'rthaḥ parikalpyate tayor yadi pratibandho 'sti tadārthāpattir anumānam eva | athāpattir iti nāmāntarakaraṇe nāsmākaṃ kācid vipratipattiḥ | tathā hi pramāṇaparidṛṣṭo 'rthaḥ kenacid vinā nopapadyata iti kuto labhyate, yadi paridṛśyamānaparikalpyamānayoḥ kaścit sambandhaḥ syāt | anyathā tena vinā nopapadyata ity ahrīkād anyo na brūyāt, ghaṭapaṭavat | sa ca sambandhaḥ kvacit pūrvam avaśyaṃ pratyakṣānupalambhataḥ, kvacid adṛśyatve 'pi viparyayabādhakapramāṇabalād vā niścetavyaḥ | anyathā tena vinānupapattijñānasyaivānupapatteḥ | sati caivam, ekaṃ sambadhinaṃ dṛṣṭvā yatrasthena vinā tatrasthaṃ nopapadyate, tasya dvitīyasya sambandhinaḥ kalpanam anumānam eva | tatra svabhāvapratibandhe svabhāvahetujaiva sārthāpattiḥ | tadutpattipratibandhe kāryaliṅgajaiva | tad uktam: anyathānupapannatvam anvayavyatirekiṇy arthe bhavati yat, tasmān nārthāpattiḥ, pramāṇāntaram iti | tasmāt paridṛśyamānaparikalpyamānayoḥ sati pratibandhe nārthāpattiḥ pramāṇāntaram iti | atha tayor na pratibandhaḥ, tadārthāpattiḥ pramāṇam eva na bhavatīti mantavyam, sākṣāt pāramparyeṇa ca sambandhābhāvāt | yasya yatra pratibandho nāsti na tasya tatra prāmāṇyam ityādir vedanirākaraṇārthaṃ yaḥ pūrvam upanyastaḥ sa evāsyā api prāmāṇyanirākaraṇāya draṣṭavyaḥ | sāmānyenaivārthāpattau nirākṛtāyāṃ pratyakṣādir pūrvakatvalakṣaṇas tatprapañco nirasto bhavaty eveti tadarthaṃ na prabandho 'bhidhīyate, gavi nirākṛte śāvaleyanirākṛtivat | tasmān nārthāpattiḥ pramāṇāntaram iti | 
	\pend
      

	  \pstart tathā abhāvapramāṇasyāpi prāmāṇyaṃ nopapadyate, tasyāpi nirviṣayatvāt | tataś ca Mīmāṃsakopavalgitopamānanirākaraṇārtham upanyasto yo viṣayavattvābhāvalakṣaṇo 'nupalambhaḥ sa evāsyāpi nirāsārtham upanyasitavyaḥ | nanu cātrāsiddho hetuḥ |
	\pend
      

	  \pstart tathā hi yadi ghaṭābhāvo vāstavaḥ prameyabhūto na syāt, tadā nāstīha ghaṭa iti pratyayaḥ katham utpadyata iti cet | kevalapradeśagrāhipratyakṣād iti brūmaḥ | nanu yadi kaivalyaṃ pradeśasvarūpaṃ tat tarhi saghaṭe 'pi pradeśe vidyata iti tatrāpi tasya pratyayasya sadbhāvaprasaṅgaḥ | athātiriktaḥ, mukhāntareṇābhāva evābhyupagato bhavatīti cet, na |
	\pend
      

	  \pstart kaivalyaṃ tadviviktatvam asaṅkīrṇatvam ityādibhiḥ padaiḥ pradeśasya ghaṭaṃ pratyanāpannādhārabhāvasya svahetuta utpannasya ghaṭapradeśād anya evātmābhidhīyate | sa eva cābhāvapratyayaṃ janayatīti kim apareṇābhāvena kartavyam |
	\pend
      

	  \pstart nanu ghaṭaṃ pratyanāpannādhārabhāvasya pradeśasyeti ghaṭābhāvayuktasya pradeśasyety uktaṃ bhavatīti cet | tarhi ghaṭābhāvo 'pi ghaṭaṃ pratyanāpannādhārabhāvaḥ kim abhāvāntareṇa svarūpeṇaiva vā | prathamapakṣe 'navasthā | atha tadabhāvarūpatvād abhāvāntaram antareṇaiva ghaṭābhāvo ghaṭaṃ pratyanāpannādhārabhāvaḥ | yady evam asahāyaḥ pradeśaviśeṣo 'pi paryudāsavṛttyā ghaṭābhāvarūpatvād abhāvaṃ vinaiva ghaṭaṃ pratyanāpannādhārabhāvo yukta iti kim akāṇḍam āhopuruṣikayā mithyāpralāpenābodhaviklavaṃ śiṣyapudgalam ākulayasi | tasmād bhūtalātiriktasyābhāvasyāsiddhatvān nāyaṃ viṣayavattābhāvalakṣaṇo hetur asiddhaḥ | pramāṇapañcakābhāvād eva tu prameyābhāvasiddhipratyāśāpi na yujyate, vipratipattiviṣayatvād asyānenaiva prameyābhāvasiddher ayogāt |
	\pend
      

	  \pstart viruddhānaikāntikatve ca pūrvam eva hetoḥ parihṛte | tad ataḥ siddham abhāvapramāṇābhimatasyāprāmāṇyam iti |
	\pend
      

	  \pstart atha vābhāvapramāṇasvarūpam eva nirūpyatām | kaḥ punaḥ pramāṇābhāvātmābhimato bhavatām, kiṃ prasajyavṛttyā pramāṇānutpattimātram, atha vā paryudāsavṛttyā bhāvāntaram | vastvantaram api jaḍarūpaṃ jñānarūpaṃ vā | jñānarūpam api jñānamātrakam ekajñānasaṃsargivastujñānaṃ veti ṣaḍ vikalpāḥ |
	\pend
      

	  \pstart tatra na tāvan nivṛttirūpo 'bhāvo yujyate | sa khalu nikhilaśaktivikalatayā na kiñcit | yac ca na kiñcit tat katham abhāvaṃ paricchindyāt, tadviṣayaṃ vā jñānaṃ janayet, pratītaṃ vā tat katham iti sarvam andhakāranartanam | yad āhuḥ: na hy abhāvaḥ kasyacit pratipattiḥ pratipattihetur vā tasyāpi kathaṃ pratipattir iti \edtext{}{\lemma{iti}\Bfootnote{(HB 25,12-14)}} | nāpi vastvantaratāpakṣe jaḍarūpo 'bhāvaḥ saṅgacchate, tasyābhāvalakṣaṇaprameyaparicchedābhāvāt, paricchedasya jñānadharmatvāt | nāpi jñānamātrasvabhāvo 'bhāvo vaktavyaḥ, deśakālasvabhāvaviprakṛṣṭasyāpi tato 'bhāvaprasaṅgāt, tadapekṣayāpi jñānamātratvāt tasya | athaikajñānasaṃsargivastujñānasvabhāvo 'numanyate tadāstam abhāvapramāṇapratyāśayā, pratyakṣaviśeṣasyaivābhāvanāmakaraṇāt | tasya cāsmābhir dṛśyānupalambhākhyasādhanatvena svīkṛtatvāt | ato na kācid vipratipattir nāma | tasmād abhāvapramāṇasvarūpam api nirūpyamāṇaṃ viśīryata eva | yad apy asya lakṣaṇam uktam
	\pend
      
	    
	    \stanza[\smallbreak]
	pratyakṣāder anutpattiḥ pramāṇābhāva ucyate |\&[\smallbreak]


	\footnote{\begin{english}(ŚV, abhāva, 11ab)\end{english}}

	  \pstart ityādi, tad api yācitakam aṇḍanam | tasmāt sthitam etat, pramāṇasya sato 'traivāntarbhāvāt pramāṇa eva |
	\pend
      

	  \pstart || pramāṇāntararbhāvaprakaraṇaṃ samāptam || 
	\pend
      
	    
	    \endnumbering% ending numbering from div
	    \endgroup
	    
	  
	  
	% new div opening: depth here is 0
	
	    
	    \begingroup
	    \beginnumbering% beginning numbering from div depth=0
	    
	  
\chapter[{Vyāptinirṇayaḥ}]{Vyāptinirṇayaḥ}\label{Vyāptinirṇayaḥ}

	  \pstart iha dahanādinā dhūmāder arthāntarasya vyāptis tadutpattilakṣaṇā | sā ca viśiṣṭānvayavyatirekagrahaṇapravaṇaviśiṣṭapratyakṣānupalambhasādhaneti nyāyaḥ | atra ca bhaṭṭaprabhṛtayo vipratipadyante | tathā hi te 'gnimati pradeśe dhūmasya bhūyodarśanaṃ tadvyukte ca tathaivādarśanam ity anvayavyatirekitvaṃ kalpayām babhūvuḥ | 
	\pend
      

	  \pstart nanu bhūyasāpi pravṛtte darśanādarśane ghaṭakulaṭādāv upalabdho vyabhicāra iti cet | kim etāvatā tatrāpy tatrāpy anumānam astu, tadvad vā dhūmādāv api mā bhūt | prathamapakṣas tāvad vyabhicārād eva nirastaḥ | dvitīyo 'pi vyabhicārād eva | na hy anyasya vyabhicāre dhūmasya kiñcit | tasmād agnidhūmayor avyabhicārasyāsambhave śaktam api tadupapattayaḥ tatprasādhakaviśiṣṭapratyakṣānupalambhā vā nānumānopayoginaḥ | sambhave vā kiṃ tadutpattyā tadupayoginā viśiṣṭapratyakṣānupalambhena, darśanādarśanābhyām evāvyabhicārasiddheḥ | tathā ca Kāśikākāraḥ: prācīnānekadarśanajanitasaṃskārasahāyena carameṇa cetasā dhūmasyāgniniyatatvaṃ gṛhyata iti ||
	\pend
      

	  \pstart \persName{trilocanas} tv āha: pratyakṣānupalambhayor viśeṣaviṣayatvāt kathaṃ tābhyāṃ sāmānyayoḥ sambandhapratipattiḥ | athānagnivyāvṛttenādhūmavyāvṛttasya sambandhaḥ pratīyata eveti | nanu so 'pi kasya pramāṇasya viṣayaḥ | na tāvat pratyakṣasya, svalakṣaṇaviṣayatvāt tasya | nāpy anumānasya, tasyāpi tatpūrvakatvāt | na ca vyāvṛttyoḥ \edtext{}{\lemma{vyāvṛttyoḥ}\Bfootnote{(J2 vyāvṛttaḥ)}} kaścit sambandhaḥ | atha pratyakṣapṛṣṭhabhāvī vikalpo dṛṣṭe bhede 'bhedam adhyavasyati, tad eva sāmānyam | evam api vikalpānāṃ na vastv eva viṣayaḥ | api tu grāhyākāraḥ | sa ca na vastu | vastu tu teṣāṃ parokṣam eveti, kathaṃ tenāpi sambandhagrahaḥ | asmākaṃ tu bhūyodarśanasahāyena manasā tajjātīyānāṃ sambandho gṛhīto bhavati | ato dhūmo nāgniṃ vyabhicarati | tadvyabhicāre dhūma upādhirahitaṃ sambandham atikrāmed iti hetor vipakṣaśaṅkānivartakaṃ pramāṇam upalabdhilakṣaṇaprāptopādhivirahahetur anupalambhākhyaṃ pratyakṣam eva | tataḥ siddhaḥ svābhāvikaḥ sambandhaḥ ||\edtext{}{\lemma{||}\Bfootnote{Cf. \cref{thakur75-46.27}.}}
	\pend
      

	  \pstart Vācaspates tu prapañcaḥ | tathā hi dhūmādīnāṃ vahnyādibhiḥ svābhāvikaḥ sambandhaḥ | na tu vahnyādīnāṃ dhūmādibhiḥ | te hi vināpi dhūmādibhir upalabhyante | vahnyādayas tu yadārdrendhanasambandham anubhavanti tadā dhūmādibhiḥ sambadhyante | vahnyādīnāṃ tu sphuṭamārdrendhanādyupādhikṛtaḥ sambandho na tu svābhāvikaḥ | tato 'niyataḥ | svābhāvikas tu dhūmādīnāṃ vahnyādibhiḥ sambandhaḥ, tadupādher anupalabhyamānatvāt | kvacid vyabhicārasyādarśanād anupalabhyamānasyāpi kalpanānupapatteḥ | na cādṛśyamāno 'pi darśanānarhatayā sādhakabādhakapramāṇābhāvena sagdihyamāna upādhiḥ sambandhasya svābhāvikatvaṃ pratibadhnātīti yuktam |
	\pend
      

	  \pstart avaśyaṃ śaṅkayā bhāvyaṃ niyāmakam apaśyatām \edtext{}{\lemma{apaśyatām}\Bfootnote{(PV I 324cd)}}
	\pend
      

	  \pstart iti tu dattāvakāśā laukikamaryādātikrameṇa śaṅkāpiśācī labdhaprasarā na kvacin nāstīti nāyaṃ kvacit pravarteta | sarvatraiva kasyacid anarthasya kathañcic chaṅkāspadatvāt | anarthaśaṅkāyāś ca prekṣāvatāṃ nivṛttyaṅgatvāt | antataḥ snigdhānnapānopayoge 'pi maraṇadarśanāt | tasmāt prāmāṇikalokayātrām anupālayatā yathādarśanam eva śaṅkanīyam | na tv adṛṣṭapūrvam api | viśeṣasmṛtyapekṣa eva hi saṃśayo nāsmṛter bhavati | na ca smṛtir ananubhūtacare bhavitum arhati | tad uktaṃ Mīmāṃsāvārtikakṛtā: nāśaṅkā niḥpramāṇikā iti | tasmād upādhiṃ prayatnenānviṣyanto 'nupalabhamānā nāstīty avagamya svābhāvikatvaṃ sambandhasya niścinumaḥ || \edlabel{thakur75-107.16}\label{thakur75-107.16} syād etat | anyasyānyena sahākāraṇena cet svābhāvikaḥ sambandho bhavet, sarvaṃ sarveṇa svabhāvataḥ sambadhyeta | sarvaṃ sarvasmād gamyeta | athānyasya ced anyat kāryaṃ kasmāt sarvaṃ sarvasmān na bhavati, anyatvāviśeṣāt | tataś ca sa evātiprasaṅgaḥ | yady ucyeta na bhāvasvabhāvāḥ paryanuyojyāḥ, tasmād anyatvāviśeṣe 'pi kiñcid eva kāraṇaṃ kāryaṃ ca kiñcid iti | nanv eṣa svabhāvānām anuyogo bhinnānāṃ akāryakāraṇabhūtānām api svabhāvapratibandhe tulya eva | tasmād yatkiñcid etad api | kena punaḥ pramāṇenaiṣa svābhāvikaḥ sambandho gṛhyate | pratyakṣasambandhiṣu pratyakṣeṇa tathā hi abhijātamaṇibhedatattvavad bhūyodarśanajanitasaṃskārasahāyam indriyam eva dhūmādīnāṃ vahnyādibhiḥ svābhāvikasambandhagrāhīti yuktam utpaśyāmaḥ | evaṃ mānāntaraviditasambandheṣu mānāntarāṇy eva yathāsvaṃ bhūyodarśanasahāyāni svābhāvikasambandhagrahaṇe pramāṇāny unnetavyāni | svabhāvataś ca pratibaddhā hetavaḥ svasādhyena yadi sādhyam antareṇa bhaveyuḥ, svabhāvād eva pracyaverann iti tarkasahāyā nirastasādhyavyatir ekavṛttisandehā yatra dṛṣṭās tatra svasādhyam upasthāpayanty eveti || \edlabel{thakur75-108.3}\label{thakur75-108.3} atrocyate | iha khalu bhede tadutpattir eva vyāptiḥ | na cāsāvanyo vā svata evāvinābhāvalakṣaṇaḥ svābhāvikaḥ sambandho bhūyodarśanamātrataḥ sidhyati | tathā hi, kiṃ yatra bhūyodarśanapravṛttis tatra niyatatvavyavasthā, yatra vā niyatatvam asti tatraiva bhūyodarśanapravṛttiḥ | prathamapakṣe ghaṭād api kulaṭā, pārthivatvād api lohalekhyatvaṃ sidhyet, bhūyodarśanasambhave 'pi niyatatvasambhavāt | \edlabel{thakur75-108.8}\label{thakur75-108.8} vyabhicāradarśanān naivam iti cet | kasya punarvyabhicāradarśanam yasya kasyacit śāstrakārasya, pratipattur vā | prathamapakṣe pratipattuḥ kim āyātaṃ yato nānumānam ayaṃ kuryāt | anyathānyasya tadviṣayapratyakṣīkāreṇaiva so 'pi kṛtārtha iti kim avaśyam anumānam anveṣate | na cāptavacanād avyabhicāradarśanād anumānam | āptasya niścetum aśakyatvād ity anyatra prasādhanāt | śāstrakāraṃ ca pṛṣṭvā dṛṣṭasambandho 'pi dhūmād agnim anumāsyata ity alaukikam | pratipattus tu nāvaśyaṃ sann api vyabhicāro gocarībhavati | na hi yatra vyabhicāras tatraiva tāvati kāle deśe vāvaśyaṃ pratītim avatarati | apratīyamānaś ca nāsty eveti na niyamaḥ | saty api vyabhicāre darśanasāmagryabhāvāt tasyādarśanāt | aticirakālavyavadhāne 'pi darśanāt brāhmaṇyādivyabhicāravat || \edlabel{thakur75-108.18}\label{thakur75-108.18} ghaṭapārthivādau pratipattaiva pravṛttaḥ | tadaiva krameṇa vā vyabhicāraṃ paśyed iti cet | yadi tāvad asau kathañcit pravartate, pravṛtto 'pi vā sāmagryabhāvāvyabhicāraṃ na paśyet | vajraṃ vā lohena vyāpārayet | vyaktaṃ tasya tāvat tad apy amānam āpannam iti mahat pāṇḍityam | tasmād yadi vyabhicāradarśanād anumānaṃ tadādṛṣṭavyabhicārasya pratipattur ghaṭapārthivatvād apy asti | tathā adarśanamātreṇa vyabhicārābhāvo na sidhyati, yogyānupalabdher eva sarvatrābhāvasādhane 'dhikārāt | tato bahulaṃ sahacāramātreṇa na vyabhicārī na vyāvyabhicārī niścita iti śaṅkāvakāśaḥ || \edlabel{thakur75-108.25}\label{thakur75-108.25} yady evam adṛṣṭavyabhicārād api dhūmād anumānaṃ mā bhūt | na | īdṛśasya śaṅkāvakāśasya sarvatra tadutpattirahite sambhavād iti | atha kadācit pratipattā pravṛtto vyabhicāraṃ paśyati | na tarhi yatra bhūyodarśanam, tatra niyatatvasthitiḥ | tatra kuto dhūme pratibandhasiddhiḥ | bhūyodarśanasyānyatra niyatatvopasthāpakatvakṣatau malinapauruṣatvena sarvatrānāśvāsāt || \edlabel{thakur75-108.30}\label{thakur75-108.30} yady evaṃ dvicandrādau cakṣurādipratyakṣaṃ malinapauruṣam upalabdham iti ghaṭādikam api nopasthāpayed iti cet | na | indriyaviṣayakāryaṃ hi pratyakṣam | na dvicandrādijñānam īdṛśam arthakāryatvābhāvāt | tato bhinnalakṣaṇasya pratyakṣābhāsattve 'pi ghaṭajñānaṃ pratyakṣam eva | na caiva dhūmādau pārthivatvādau ca vyāptigrāhakasya bhūyodarśanasya lakṣaṇabhedo yenaikatrāśvāsaḥ syāt || \edlabel{thakur75-109.4}\label{thakur75-109.4} ete evārthakāryatvākāryatve lakṣaṇabheda iti cet | na | ghaṭādijñānasya hy arthakāryatvavivāde pramāṇāntarato 'rthakriyālābhato vā niścayaḥ, na pratijñāmātreṇa | na cātra dhūmasyāgnisahacāraḥ sadātano 'yam atha suhṛddvayasyeva sātyayo gṛhīta iti saṃśaye sadātanasahacāraprasādhakapramāṇāntarasaṅgatir asti, tatkāryaṃ vā kiñcid upalabhyate | tarhi bādhyamānatvābādhyamānatvalakṣaṇo lakṣaṇabhedo bhaviṣyatīty api na vaktavyam, avyabhicāragrahākasya bhūyodarśanasya bādhitatvāsiddheḥ | abādhamātraṃ hi prasajyapratiṣedho 'pramāṇam | pramāṇāntarasaṅgatir arthakriyālābho vā prayudāsaś cāsiddha iti na tāvat prathamaḥ pakṣaḥ | nāpi dvitīyaḥ | niyatatvābhāve 'pi pārthivatvādau bhūyodarśanasambhavād iti na bhūyodarśanagamyā vyāptiḥ || \edlabel{thakur75-109.13}\label{thakur75-109.13} \persName{trilocana}codye 'pi brūmaḥ | yadi pratyakṣaṃ svalakṣaṇaviṣayam ity ayogavyavacchedenocyate tadā siddhasādhanam | anyayogavyavacchedas tv asiddhaḥ, pratyakṣānumānādisarvajñānānāṃ grāhyāvaseyabhedena viṣayadvaividhyānatikramāt | yad dhi yatra jñāne pratibhāsate tad grāhyam | yatra tu tat pravarte tad adhyavaseyam | tatra pratyakṣasya svalakṣaṇaṃ grāhyam | adhyavaseyaṃ tu sāmānyam, atadrūpaparāvṛttasvalakṣaṇamātrātmakam | anumānasya tu viparyayaḥ | tataś ca sāṃvyavahārikapramāṇāpekṣayā rūparasagandhasparśasamudāyātmakasya ghaṭasya rūpabhedamātragrahaṇe 'pi pratyakṣataḥ samudāyasiddhivyavasthā | tathaikasyātadrūpaparāvṛttasya grahaṇe 'pi sādhyasādhanasāmānyayor atadrūpaparāvṛttavastumātrātmanor ayogavyavacchedena viṣayabhūtayor vyāptigraho yukta eva | ata eva vikalpānām avastv eva viṣayaḥ, vastu tu teṣāṃ parokṣam evety api durjñānam, sarvavikalpānām adhyavaseyāpekṣayā vastuviṣayatvāt | śāstre 'pi tathaiva pratipādanāt | na ca manasā tajjātīyānāṃ vyāptigrahaḥ śakyaḥ, manaso bahir asvātantryāt | anyathā andhabadhir ādyabhāvaprasaṅgāt | na ca vahnivyabhicāre dhūma upādhirahitaṃ sambandham atikrāmed iti vaktum ucitam, svakapolakalpitasvābhāvikasambandhasya yācitakamaṇḍanatvād iti || \edlabel{thakur75-109.27}\label{thakur75-109.27} yad api vācaspatijalpitam, yo yatropādhinā niyatas tatra tasya svābhāvikaḥ sambandhaḥ | yathā dahane dhūmasya | tadupādher dṛśyasyānupalabhyamānatvāt kvacid vyabhicārasyādarśanād ity atredaṃ vicāryate | yasyādarśanataḥ svābhāvikaḥ sambandho vavasthāpanīyaḥ, sa khalu dhūmasvarūpād arthāntaram upādhir vaktavyo yathā dahanād indhanam | arthāntaraṃ ca kiñcid dṛśyam adṛśyaṃ ca kiñcit, na tu sarvam eva dṛśyatāniyatam | tataś ca dhūmasyāpi hutāśane syād upādhiḥ, na copalabhyate ity upādhimātrānupalabdhir anaikāntikī | tat katham adarśanamātrān nāsty evopādhiḥ, yataḥ svābhāvikasambandhasiddhiḥ syāt | dṛśyopadhyabhāvasādhane tu siddhasādhanam | paramadṛśyopādhiśaṅkāsambhave svābhāvikatvapratirodhas tadavstha eva | kvacid vyabhicārādarśanād ity asambaddham eva, upādhivat vyabhicārasyāpy adarśanamātrād abhāvāsiddheḥ | vyabhicārasya sarvadeśakālayoḥ sambhave 'pi sarvadā sarvatra sarveṇa sāmagryabhāvād api niścetum aśakyatvāt | brāhmaṇyādivyabhicāravad evāhatyādarśane 'pi deśakālāntare taddarśanasya niṣeddham aśakyatvāt | \edlabel{thakur75-110.7}\label{thakur75-110.7} nanu yadi dhūmasyāpekṣaṇīyam arthāntaram upādhiḥ syāt kathaṃ dhūma ity eva pāvakasattāniyama iti cet | nanv idam eva cintyate kiṃ dhūme saty avaśyam agniḥ sambhavī na veti | kadācid arthāntaram upādhim apekṣya dhūmo 'pi syān nāgnir iti kim atra niṣṭabdhaṃ kāraṇam | tasmāt pāvakaparādhīnodayo dhūmaḥ pariniṣṭhitaḥ kathaṃ tadabhāve bhāvaṃ svīkuryād ity eva sādhu | \edlabel{thakur75-110.12}\label{thakur75-110.12} atha vyaktau jātau vā vahnivyabhicāro na dṛṣṭaḥ, kathaṃ tatra śaṅkyata iti cet | tat kiṃ sthāṇuvyaktau jātau vā puruṣatvaṃ dṛṣṭaṃ yena sthāṇau śaṅkyate | anyatrordhvatāliṅgite dṛṣṭam iti cet | ihāpy anyatra bhūyaḥ sahacāriṇi pārthivatvādau dṛṣṭa eva vyabhicāraḥ | yatraiva tu yat saṃśayate tatraiva tasya darśanam apekṣyata ity alaukikam | yadi dhūmavyaktau vyabhicāro dṛṣṭas tadā dhūmasāmānyaṃ vyāptau bahirbhūtam eva, kathaṃ saṃśayaḥ | atha jātau dṛṣṭas tadāpi vyabhicāraniścaya eva, kathaṃ saṃśayaḥ | ato dhūmajātāv adṛśyamāno 'pi vyabhicāra upādhir vā darśanāyogyatayā niṣeddhum aśakya iti saṃśayo durvāraprasaraḥ | sa cedānīm upādher vyabhicārasya vā saṃśayaḥ svābhāvikatvasaṃśayasvabhāvaḥ svābhāvikatvaniścayaṃ tāvad avaśyaṃ pratibadhnāti | tasmāt svābhāvikatvaniścayapratibandha evārthataḥ, niścayam antareṇa gamakasya svayam akiñcitkaratvāt | tad evam upādhyanupalabdhir vyabhicārasyānupalabdhir vā 'naikāntikī na tayor abhāvaṃ sādhayati, yataḥ sambandhasya svābhāvikatvasiddhiḥ syāt | asiddhā ceyam upādhyanupalabdhiḥ | yathā dahano nendhanena vinā dhūmena sambadhyate tathā dhūmo 'pi na vināgninā sambadhyata iti samānam upādhitvam indhanasyobhayatra | \edlabel{thakur75-110.26}\label{thakur75-110.26} atha siddhasyāgner indhanasāhityena dhūmalābha ity upādhivyavasthā, asiddhasya tu dhūmasya tannimittātmalābhatayāvyabhicārāt svābhāvikaḥ sambandha iti vyavasthāpyata iti cet | evam api saiva tadutpattir āyātā | saiva svābhāvikaḥ sambandhaḥ | na punaḥ pratijñāsiddhaḥ sahacāramātrātmakaḥ | kiṃ ca svābhāvikatvād avyabhicāraḥ sarvatra, sarvatrāvyabhicārāc ca svābhāvikatvam atītaretarāśrayatvam anivāryam | yasya tu sakṛttadutpattipratītir eva sarvatrāvyabhicārapratītis tasya nāyaṃ prasaṅgaḥ | \edlabel{thakur75-110.32}\label{thakur75-110.32} yady evaṃ mamāpi bhūyodarśanād avyabhicārasiddhir iti cet | na | bhūya ity apariniṣṭhitavārasaṃkhyatvāt kiyatā darśanena lakṣaṇānusārī nirvṛtim āsādayet | asmākaṃ tu pratyakṣānupalabdhau parigaṇitasaṃkhyāv eva | yad āhuḥ
	\pend
      
	    
	    \stanza[\smallbreak]
	prāg adṛṣṭau kramāt paśyan veti hetuphalasthitim |&dṛṣṭau vā kramaśo 'paśyann anyathā tv anavasthitiḥ ||\&[\smallbreak]


	

	  \pstart iti ||
	\pend
      

	  \pstart yat tv anupalabhyamānasyāpi kalpanānupapatter iti vilapitam, tadbālasyāpy asāmpratam | anupalabhyamāne 'rthe ca kalpanāvakāśāt | na hi dṛśyamāno ghaṭaḥ kalpita ucyate | na ca sandihyamāna upādhiḥ sambandhasya svābhāvikatvaṃ pratibadhnātīti yuktam, sādhakabādhakābhāva eva saṃśayasya nyāyaprāptatvāt | ata eva na sarvatra śaṅkāpiśācāvakāśaḥ | tat kathaṃ nāyaṃ pravarteta | \edlabel{thakur75-111.9}\label{thakur75-111.9} pramāṇaviṣaye 'pi śaṅkā kartuṃ śakyata iti cet | na | svīkṛtapramāṇasya hi niścayaphalatvāt pramāṇasyāvipratipannapramāṇaviṣaye niścayasvīkāranāntarīyaka eva tatsvīkāraḥ | na ca śaṅkety eva na pravṛttiḥ, arthasaṃśayenāpi pravṛtter anivāryatvāt snigdhānnapānopayogavat | tadupayoge kadācin maraṇadarśane 'pi koṭiśo jīvitadarśanāt | na ca prāmāṇikalokayātrākṣatiḥ, prāmāṇikair eva pramāṇābhāve saṃśayasya vihitatvāt | yathādarśanam āśaṅkanīyam ityādy api siddhasādhanam, anyatra dṛṣṭasyaivopādher vyabhicārasya vā śaṅkitatvāt | kiṃ ca bādhakādarśane 'pi sādhakābhāvād api śaṅkā syād eva | \edlabel{thakur75-111.17}\label{thakur75-111.17} yad api syād etad iti valgitaṃ tad api niḥsāram | pramāṇasiddhe hi rūpe svābhāvāvalambanam | na tu svabhāvāvalambanenaiva vastusvarūpavyavasthā | tad yadi niyataviṣayānvayavyatirekagrāhakapratyakṣānupalambhapramāṇasiddhe hetuphalabhāve svabhāvavādas tat kim āyātaṃ svābhāvikasambandhe | yatra tadutpattisāmagrīṃ hṛdayena dūrīkṛtyānyataḥ sahacaritadvayād viśeṣeṇa pratītau pratyupāya eva davīyān | tatsāmagryapakṣaṇe ca tadutpattir eva sā | kim āhopuruṣikayā nāmāntarakaraṇena | kena punaḥ pramāṇena eṣa svābhāvikaḥ sambandho gṛhyata ityādis tadgrahaṇaprakāraḥ pūrvam eva nirākṛtaḥ | tathā svābhāvikatvāsiddhau svabhāvataś ca pratibaddhā hetava ityādy upasaṃhāro 'pi manorājyamātram | tasmād arthāntare gamye kāryahetus tadbhāvasiddhiś ca pratyakṣānupalambhād iti sthitam | tad evaṃ svābhāvikavādena hṛdayānulepanam aśucin eva parihāryaṃ dūrata iti |
	\pend
      

	  \pstart || vyāptinirṇyaḥ samāpto ratnakīrtipādānām || 
	\pend
      
	    
	    \endnumbering% ending numbering from div
	    \endgroup
	    
	  
	  
	% new div opening: depth here is 0
	
	    
	    \begingroup
	    \beginnumbering% beginning numbering from div depth=0
	    
	  
\chapter[{Sthirasiddhiduṣaṇam}]{Sthirasiddhiduṣaṇam}\label{Sthirasiddhiduṣaṇam}

	  \pstart namas tārāyai ||
	\pend
      
	    
	    \stanza[\smallbreak]
	yadyogād andhavad viśvaṃ saṃsāre bhramad iṣyate |&sā kṛpāvaśagaiḥ pāpā sthirasiddhir apāsyate ||\&[\smallbreak]


	

	  \pstart iha pare sakalapadārthasthairyaprasādhanārthaṃ pratyakṣam anumānam arthāpattiṃ [ca] pramāṇāny ācakṣate | \edlabel{thakur75-112.7}\label{thakur75-112.7} tathā hi | sa evāyaṃ ghaṭasphaṭikādir iti pratyabhijñākhyaṃ pratyakṣam udīyamānaṃ sthairyam utthāpayati | na cedam apramāṇam abhidhātavyam | aprāmāṇyaṃ hi bhavad aprāmāṇyakāraṇopapattyā vā bhavet, prāmāṇyalakṣaṇavirahād vā | \edlabel{thakur75-112.9}\label{thakur75-112.9} yady ādyaḥ pakṣaḥ | kiṃ aprāmāṇyakāraṇam, mithyātvam ajñānaṃ saṃśayo vā | \edlabel{thakur75-112.10}\label{thakur75-112.10} na tāvad atra mithyātvam | mithyātvaṃ hi tadviṣaye bādhakapratyayād vā hetūktadoṣato vā sambhāvyeta | \edlabel{thakur75-112.11}\label{thakur75-112.11} na tāvad bādhagandho 'pi sambhavati | deśakālanarāntareṣv apy asambhavāt | na cānavagatāpi bādhā kadācid api bhaviṣyatīti śaṅkā yuktimatī | nirbījaśaṅkānupapatteḥ |
	\pend
      

	  \pstart avaśayaṃ śaṅkayā bhāvyaṃ niyāmakam apaśyatām | \edtext{}{\lemma{|}\Bfootnote{(PV I 324cd)}}
	\pend
      

	  \pstart iti dattāvakāśā saṃśayapiśācī labdhaprasarā na kvacin nāstīti nāyaṃ kvacit pravarteta | antataḥ snigdhānnapānopayoge 'pi maraṇadarśanena sarvatra śaṅkānivṛtteḥ | tasmāt prāmāṇikalokayātrām anupālayatā yathā darśanam eva śaṅkanīyaṃ nādṛṣṭapūrvam api | \edlabel{thakur75-112.18}\label{thakur75-112.18} yad uktaṃ Kārikāyāṃ nāśaṅkā niṣpramāṇikā \edtext{}{\lemma{niṣpramāṇikā}\Bfootnote{(ŚV II 60d)}} | iti | Bṛhaṭṭīkāyām api
	\pend
      
	    
	    \stanza[\smallbreak]
	utprekṣeta hi yo mohād ajñātam api bādhakam |&sa sarvavyavahāreṣu saṃśayātmā kṣayaṃ vrajet ||\edtext{}{\lemma{||}\Bfootnote{(=TS 2871)}}\&[\smallbreak]


	

	  \pstart iti |
	\pend
      

	  \pstart kṣaṇabhaṅgasādhanaṃ bādhakam asyeti cet | na | anumānasya paramparayāpi pratyakṣapūrvatvāt pratyakṣam pradhānam | prādhānyāc cānumānasya bādhakam | na tv anumānam asya | pratyakṣāntaraṃ tu bādhakaṃ bhavati | yathā sarpādipratyayasya rajjvādipratyakṣam | tac cātra na sambhavati | \edlabel{thakur75-112.28}\label{thakur75-112.28} nanu pratyakṣe 'pi bādhake kasmān na bhavati parasparapratibhandhena dvayor apy apratyakṣatā | \edlabel{thakur75-112.28a}\label{thakur75-112.28a} na, arthakriyāsamarthavastuviṣayāviṣayatvena samānatvābhāvād ekasya pratyakṣābhāsatvād iti na sadviṣayatvabādhakapratyayān mithyātvam | \edlabel{thakur75-113.1}\label{thakur75-113.1} nāpi hetūktadoṣataḥ | deśakālanarāntareṣv avisaṃvādāt | \edlabel{thakur75-113.2}\label{thakur75-113.2} nāpy ajñānam aprāmāṇyakāraṇam atrāsti | pratyabhijñānasaṃvedanasambhavāt | \edlabel{thakur75-113.3}\label{thakur75-113.3} na ca saṃśayaḥ | na hi tad evedaṃ syād vā na veti sphaṭikādiṣūdayati matiḥ | kiṃ tu tad evedaṃ sphaṭikādikam iti nirastā vibhramāśaṅkā | tan nāprāmāṇyakāraṇopapattyā pratyabhijñānasyāprāmāṇyam | \edlabel{thakur75-113.5}\label{thakur75-113.5} nāpi lakṣaṇakṣayāt | yad eva hi utpannam asandigdham aduṣṭakāraṇajanyaṃ deśakālanarāntareṣv abādhitaṃ ca tad eva pramāṇam iti naḥ siddhāntaḥ | tad uktam |
	\pend
      
	    
	    \stanza[\smallbreak]
	tasmād dṛḍhaṃ yad utpannaṃ na visaṃvādam ṛcchati |&jñānāntareṇa vijñānaṃ tat pramāṇaṃ pratīyatām ||\edtext{}{\lemma{||}\Bfootnote{(ŚV II 80; =TS 2904)}}\&[\smallbreak]


	

	  \pstart tathā Bṛhaṭṭīkāpi
	\pend
      
	    
	    \stanza[\smallbreak]
	\label{ratnakīrtinibandhāvali__36r1PD1OQIW6MGHHK6UP56XMABT}\flagstanza{\tiny\textenglish{...6XMABT}}tatrāpūrvārthavijñānaṃ niścitaṃ bādhavarjitam |&aduṣṭakāraṇārabdhaṃ pramāṇaṃ lokasammatam ||\edtext{}{\lemma{||}\Bfootnote{See also PVA 21,17f = PVAO 53,4f; TBV 13,24f, 318,25f, 394,16f; TR 126,21, ; Ravigupta, D304b1-2 (vol 9) = Q151a1:; cf. Mimaki 1976: 88f and 284f.}}\&[\smallbreak]


	

	  \pstart iti | etac ca lakṣaṇam uktanyāyena pratyabhijñāne 'pi sambhavatīti pramāṇam evedam | \edlabel{thakur75-113.14}\label{thakur75-113.14} nanv idam ekam eva na bhavati kāraṇabhedāt, viṣayabhedāt, svabhāvavirodhāc ca | \edlabel{thakur75-113.14a}\label{thakur75-113.14a} tathā hi | sa iti saṃskārakāryam | ayam iti cendriyakāryam | na ca kāraṇabhede 'pi kāryābhedo viśvavaicitryāhetukatvaprasaṅgāt | \edlabel{thakur75-113.16}\label{thakur75-113.16} tathā saty api sphaṭikaḥ sphaṭika iti vyapadeśābhede pūrvadeśakālasambandhāparadeśakālasambandhābhyāṃ viruddhadharmābhyāṃ yogāt sphaṭikaḥ pūrvāparakālayor bhidyata iti viṣayabhedo vaktavyaḥ | \edlabel{thakur75-113.18}\label{thakur75-113.18} tathā sa iti parokṣam | ayam iti sākṣātkāraḥ | na cānayoḥ svabhāvaviruddhayor dahanatuhinayor iva śakyā śakreṇāpy ekatā āpādayitum | trailokasyaikyaprasaṅgāt | \edlabel{thakur75-113.20}\label{thakur75-113.20} na cāsya prāmāṇyam, vikalpatvenāvastunirbhāsitvāt, smārtād aviśeṣāc ca | tasmāt pratyabhijñā ekatvaṃ sthāpayati bhāvānām iti manorathamātram | \edlabel{thakur75-113.23}\label{thakur75-113.23} atrocyate | ekam evedaṃ pratyabhijñānaṃ samākhyātam, \edlabel{thakur75-113.23a}\label{thakur75-113.23a} yady apīndriyaṃ kevalam asamartham, yady api saṃskāramātram, saṃskārasadhrīcīnaṃ tu indriyaṃ bhāvayiṣyati pratyabhijñām | tadbhāvābhāvānuvidhānāt pratyabhijñābhāvābhāvayoḥ | na hi nājījanad bījamātram aṅkuram iti mṛdādisahitam api na janayati | \edlabel{thakur75-113.26}\label{thakur75-113.26} atha bhavatu deśakālayos tatsaṃsargayor vā parasparanānātvam | na tadavacchinnasya padmarāgasya | tasya tābhyāṃ tatsaṃsargābhyāṃ cānyatvāt | \edlabel{thakur75-113.29}\label{thakur75-113.29} tato 'nyatve tatsaṃsargayoḥ kutas tadīyatvam iti cet | svabhāvād eveti saṃsargaparīkṣāyāṃ nipuṇataram upapādayiṣyate | \edlabel{thakur75-113.30}\label{thakur75-113.30} na ca svabhāvavirodhaḥ, anumānasyāpy anekatvaprasaṅgāt | tad api hi pratyakṣam apratyakṣaṃ ca | avikalpo vikalpaś ca | asamāropaḥ samāropaś ca | \edlabel{thakur75-113.32}\label{thakur75-113.32} svānubhavāvasthāpitābhedasya svarūpatadgrāhyabhedāpekṣayā pratyakṣādīnām avirodha iti cet | na, ihāpi sāmyāt | na khalv etad api vijñānaṃ tattedantādhikaraṇam ekam ābhyām anuraktaṃ sphaṭikaṃ gocarayad abhinnaṃ nānubhūyate nāvasīyate vā | ekatve 'pi ca vastunas tadanurañjakatattedantābhedāpekṣayā pratyakṣatāparokṣate na virotsyete, sahasambhavāt | vijñānaikatvasya ca pramāṇasiddhatvāt | \edlabel{thakur75-114.3}\label{thakur75-114.3} na ca sa iti pūrvadeśakālasaṃsargo 'yam iti ca sannihitadeśakālasaṃsarga ekasya virudhyate | yato yuktaṃ yat padmarāgasya svarūpe paricchidyamāne tadabhāvo vyavacchidyata iti tadavyavacchede tatsvarūpāparicchedāt, svapracyutivyavacchedyasvabhāvatvāt padmarāgabhāvasya tadanavacchede tatparicchedānupapatteḥ | \edlabel{thakur75-114.8}\label{thakur75-114.8} kasmāt punas tadanye puṣparāgādayo vyavacchidyante | tadabhāvāvinābhāvād iti cet, sa eva kutaḥ | pratyakṣeṇa kadācid api puṣparāgapadmarāgayos tādātmyānupalambhād iti cet | yatra tarhi tatas tādātmyapratītiḥ, tatra tadavinābhāvaḥ | samasti ca so 'yaṃ padmarāga iti deśakalāvasthānugatam ekaṃ padmarāgam avabhāsayantī sākṣātkāravatī pratītiḥ | \edlabel{thakur75-114.12}\label{thakur75-114.12} na vikalparūpatayāsyā aprāmāṇyam | abhilāpasaṃsargapratibhāsatvaprāmāṇyayor avirodhāt | \edlabel{thakur75-114.13}\label{thakur75-114.13} na cedaṃ smārtam | adeśakālāvasthāvato 'sya deśakālāvasthānugatatvenādhikyād iti | \edlabel{thakur75-114.15}\label{thakur75-114.15} atha keśakuśakadalīstambādau saty api bhede pratyabhijñānam utpannam iti cet | utpadyatāṃ ko doṣaḥ | kim anena pratipāditaṃ bhavati | kiṃ pratyabhijñāyāḥ sādhāraṇānaikāntikatvam, atha śabdasāmyād ubhayor apy aprāmāṇyam, uta saṃśayāpādanamātram | \edlabel{thakur75-114.18}\label{thakur75-114.18} prathamaḥ pakṣo 'nabhyupagamād eva nirastaḥ | na hīyam anumānatvenopanyastā | anumānatve 'py abādhitatvād iti viśeṣaṇe na doṣa iti pratipādayiṣyāmaḥ | \edlabel{thakur75-114.19}\label{thakur75-114.19} nāpi dvitīyaḥ pakṣaḥ | dṛṣṭāntamātrataḥ sādhyasiddher ayogāt | keśoṇḍukādiviṣayasya cakṣurvijñānasyāpy aprāmāṇye ghaṭādipratyakṣasyāprāmāṇyaprasaṅgāt | \edlabel{thakur75-114.21}\label{thakur75-114.21} saṃśayamātraṃ tu vyavahārocchedakatvān nāśraṇīyam eveti pratipāditam iti na tṛtīyo 'pi pakṣaḥ | \edlabel{thakur75-114.23}\label{thakur75-114.23} kiṃ ca keśādau yadi pratyabhijñā vyabhicāriṇī, kāryakāraṇapratītiḥ kiṃ na vyabhicāriṇī | yā vyavicāriṇī sā kāryakāraṇapratītir eva na bhavatīti cet | yady evaṃ yā visaṃvādinī sā pratyabhijñaiva na bhavati tadābhāsatvād iti samānam | pratyabhijñānasya ca sati prāmāṇye 'numānādiṣv anantarbhāve pratyakṣaiva | saṃskārasahāyendriyānvayavyatirekānuvidhāyitvāc ca | satsaṃprayoge satīndriyāṇāṃ bhāvāc ca | tad iyaṃ pratyabhijñā 'nekadeśakālāvasthāsambaddham ekaṃ sphaṭikādikaṃ gocarayantī sthairyaṃ vyavasthāpayati | \edlabel{thakur75-114.30}\label{thakur75-114.30} tathānumānato 'pi sthiratāsiddhiḥ | prayogaḥ | vivādādhyāsitaḥ sa evāyaṃ sphaṭika ityādi pratyabhijñāpratyayo yathārthaḥ | abādhitapratyayatvāt | yāvān abādhitapratyayaḥ sa sarvo yathārtha upalabdhaḥ | yathā svasaṃvedanapratyayaḥ | abādhitaś cāyam | tasmāt tatheti | abādhitañ ca parodbhāvitakṣaṇikatvasādhanabādhakoddhārān niśceyam | \edlabel{thakur75-115.1}\label{thakur75-115.1} athāparaḥ prayogaḥ | vivādādhyāsitā bhāvāḥ pūrvāparakālayor ekasvabhāvāḥ abādhitapratyabhijñayā pratyabhijñāyamānatvāt | yad yad abādhitapratyabhijñayā pratyabhijñāyate tat sarvam abhinnam, yathā yas tvayā dṛṣṭo nīlo 'rthaḥ sa eva mayā dṛṣta iti nīlo 'rthaḥ pratyabhijñāyate | tathā caite bhāvāḥ | tasmāt tatheti | pūrvaṃ pratyayasya dharmitā | adhunā bhāvānām iti viśeṣaḥ | \edlabel{thakur75-115.6}\label{thakur75-115.6} kiṃ ca sahetukatvād vināśasya sthairyaṃ siddham | prayogaḥ | vivādāspadībhūtā bhāvā yathāsvaṃ vināśahetusannidheḥ prāṅ na vināśinaḥ | sahetukavināśatvāt | yad yaddhetukaṃ tat tadasannidhau na bhavati | yathā vahnyādyabhāve dhūmādiḥ | sahetukavināśāś cāṃī bhāvāḥ | tasmāt tatheti | \edlabel{thakur75-115.9}\label{thakur75-115.9}sahetukavināśatvaṃ ca ghaṭasyāgnidhūmayor iva pratyakṣānupalambhato mudgaravināśayor api kāryakāraṇabhāvasiddhau siddham | na ca vināśahetor asāmarthyavaiyarthyābhidhānam ucitam | aṅkurādihetor api tathātvaprasaṅgāt | śakyaṃ hi vaktum arthasya bhaviṣṇutāyāṃ asamartho janmahetuḥ | bhaviṣṇutāyāṃ vyartha iti | \edlabel{thakur75-115.13}\label{thakur75-115.13} api ca akṣaṇikāḥ santaḥ | kāraṇavattvāt | yat kāraṇavat tad akṣaṇikam | yathā bhāvavināśaḥ | kāraṇavantaś ceme santaḥ | tasmād akṣaṇikā iti | \edlabel{thakur75-115.15}\label{thakur75-115.15} kāraṇavattvasya sādhyaviparyaye vṛttiśaṅkā vināśasya sahetukatvam eva nivartayatīti prasiddhavyāptikāt kāraṇavattvād akṣaṇikatvasiddhir iti | \edlabel{thakur75-115.17}\label{thakur75-115.17} tathā Śaṅkaraḥ Sthirasiddhau prāha | notpattyanantaravināśī bhāvaḥ prameyatvāt | vastuvyāvṛttivad iti | \edlabel{thakur75-115.18}\label{thakur75-115.18} avidyamānavipakṣatvād anvayy eva hetuḥ | prameyatvasya kṣaṇikatvena virodhābhāvāt sandigdhavyatirekitvam iti cet | \edlabel{thakur75-115.19}\label{thakur75-115.19} na khalu kṣaṇikatve kasyacit prameyatvaṃ sidhyati | kṣaṇasthitidharmaṇaḥ pramāṇakāle 'pātāt | atītasya ca prameyatve 'tiprasaṅgād iti | \edlabel{thakur75-115.22}\label{thakur75-115.22} evam eva prayogam upastuvan \persName{trilocano} 'py āha | akṣaṇikāḥ sarvabhāvāḥ | prameyatvāt | yat pramīyate tad akṣaṇikam | yathā bhāvavināśaḥ | prameyāś ca sarvabhāvāḥ | tasmād akṣaṇikā iti | \edlabel{thakur75-115.25}\label{thakur75-115.25} asiddho dṛṣṭāntadharmīti cet | na svakāraṇakalāpād utpattimato bhāvasyāntareṇa nivṛttiprasavaṃ sarvadāvasthānaprasaṅgāt | tadaiva bhāvo 'sti na pūrvaṃ na paścād ity api śabdaḥ kṣaṇikaparyāyatveneṣyamāṇaḥ kṣaṇād ūrdhvaṃ sattāvicchedopajananam antareṇa nārthavān devair api śakyaḥ parikalpayitum | vināśakālāpekṣayā hi kṣaṇo 'lpīyān kālaḥ | tena so 'syāstīti kṣaṇiko vaktavyaḥ | itarathā janmavināśayor ekasmin kāle bhavatoḥ tulyahetukatvenaikatvaprasaṅgaḥ | ekatve tu dvayor ekataraḥ prahātavyaḥ | tatra janmaprahāṇe bhāvā niḥsvabhāvāḥ prasajyeran | nivṛttipratiyāge ca janmino bhāvā nityā iti durnivāraḥ prasaṅgaḥ | tat siddho dṛṣṭāntaḥ | \edlabel{thakur75-116.1}\label{thakur75-116.1} nanu prameyatvakṣaṇikatvayor virodhāsiddheḥ sandigdhavipakṣavyāvṛttikaṃ prameyatvam iti cet | \edlabel{thakur75-116.2}\label{thakur75-116.2} naitad asti | yasmād arthaṃ kiñcit prāpayat pratyakṣaṃ tena pratyāsannatvāt prāpayati | pratyāsattiś ca tadutpattir evāvakalpate | na tādātmyam | sākāranirākāravādayor aprakṛtatvāt | anyatra nirākṛtatvāc ca | sā ca niyatavastupratibhāsākṣiptā kāryakāraṇabhāvalakṣaṇā pratyāsattis tulyakālaṃ pramāṇaprameyayor anupapannā, sevyetaraviṣāṇayor iva | tataḥ pramāṇam arthasattāṃ bodhayat tadadhīnotpādatayā bodhayati | kāraṇabhāvamātrānubandhitvāc ca tasya pūrvakālasattyā bhavitavyam | ataḥ pūrvakālasattvena vyāptaṃ prameyatvam | pūrvakālasattvaṃ ca kṣaṇikatve 'nupapannam iti vyāpakānupalabdhyā vipakṣāt kṣaṇikatvād vyāvartamānaṃ prameyatvam akṣaṇikatvena vyāpyata iti asandigdho vyatirekaḥ | \edlabel{thakur75-116.10}\label{thakur75-116.10} tad evam anumānapramāṇasiddho 'kṣaṇika iti || \edlabel{thakur75-116.11}\label{thakur75-116.11} evam arthāpattir apy asya sādhikā | tathā hi kāryakāraṇabhāvagrahaṇaṃ kramayaugapadyagrahaṇaṃ smaraṇam abhilāṣaḥ svayaṃnihitapratyanumārgaṇaṃ dṛṣṭārthakutūhalaviramaṇaṃ karmaphalasambandhaḥ saṃśayapūrvakanirṇayaḥ bandhamokṣaḥ mokṣaprayatnaḥ śubhādike karmaṇi pravṛttiḥ pratyabhijñā kāryakāraṇabhāvaḥ | upādānopādeyabhāvaprabhṛtayaḥ sthirasattām antareṇānupapadyamānāḥ sthairyaṃ sādhayanti | pratikṣaṇaṃ bhede saty anubhavitur vinaṣṭatve 'nyasya kāryakāraṇabhāvagrahaṇādyanupapatter iti kathaṃ kṣaṇabhaṅgaśaṅkā 'pi || \edlabel{thakur75-116.17}\label{thakur75-116.17} atrābhidhīyate | apramāṇam evāyaṃ pratyabhijñākhyo vikalpo mithyātvaṃ ca sadviṣayatvabādhakapratyayāt | \edlabel{thakur75-116.18}\label{thakur75-116.18} nanv asya bādhakaṃ pratyakṣam asambhavi | anumānaṃ cāsamartham āveditam iti cet | nanv asya pratyabhijñānasya svārthāvinābhāvadārḍhye pratyakṣasahasreṇāpi kim | saṃvādaśaithilye tu bādhakapratyakṣavad anumānam api prāptāvakāśam | pramāṇasyaiva siddhibādhyor adhikārāt | tathā hi māyākāraḥ śirasi nimajjitaṃ golakam āsyena niḥsārayatīti pratyabhijñā śirasi cchidraprasaṅgasaṅgatenānumānena bādhyamānā pratītaiva | bādhyamānā na pratyabhijñeti prastute 'py astu | \edlabel{thakur75-116.23}\label{thakur75-116.23} yathā 'vanatākāśapratibhāsaḥ sarvasaṃpratipattāv api bādhya eva tadvad ekatāgrahaḥ sarvasaṃpratipattāv api bādhyo 'stu | tasmād asyāḥ pratyakṣatākīrtanaṃ yācitakamaṇḍanamātram atrāṇam | katham ataḥ sthairyasthitir astu | \edlabel{thakur75-116.26}\label{thakur75-116.26} tataś cānumānatvam apy asyā dhvastam | uktakrameṇābādhitatvaviśeṣaṇaviruddhabādhyamānatāyāḥ prasādhanād iti viśeṣaṇāsiddho hetuḥ | \edlabel{thakur75-116.27}\label{thakur75-116.27} yadāpi kṣaṇabhaṅgasādhakaṃ bādhakaṃ nocyate asyās tadāpīyam apramāṇam eva | lūnapunarjātakeśādau vyabhicāropalambhāt | \edlabel{thakur75-1}\label{thakur75-1} nanūktaṃ yā vyabhicāriṇī sā na pratyabhijñetyādi | \edlabel{thakur75-116.30}\label{thakur75-116.30} yuktam etat | yadi kāryakāraṇabhāvapratītival lakṣaṇabhedaḥ pratipādayituṃ śakyeta | yathā hy anvayavyatirekagrahaṇapravaṇapratyakṣānupalambhād upapanno niścayaḥ kāryakāraṇabhāvapratītir anyas tadābhāsapratītir ity anayor lakṣaṇabhedaḥ, tathā yadi pratyabhijñe 'pi lakṣaṇabhedo darśitaḥ syāt, darśayituṃ vā śakyo vyabhicārāvyabhicāropayogī, tadā bhavatu pratyabhijñātadābhāsayor vivekaḥ | na tv evam asti | sarvatrātyantasadṛśe vastuni pṛthagjanapratyabhijñāyā ekarasatvāt | \edlabel{thakur75-117.3}\label{thakur75-117.3} saṃvāditvāsaṃvāditve lakṣaṇabheda iti cet | na | aliṅgasya hi vikalpasya saṃvādo nāma pramāṇāntarasaṅgatir athakriyāprāptir vā | \edlabel{thakur75-117.4}\label{thakur75-117.4} tatra na tāvad ādyaḥ pakṣaḥ | paścād api sa evāyam iti svatantraikādhyavasāyamātrād aparasya pramāṇagandhasyāpy abhāvāt | \edlabel{thakur75-117.6}\label{thakur75-117.6} nāpi dvitīyaḥ pakṣaḥ saṅgacchate | na hi pūrvāparakālayor ekavastupratibaddhā siddhā kācid arthakriyā | bhinnenāpi tatsamānaśaktinā tādṛgarthakriyāyāḥ karaṇāvirodhāt | tathā hi yathaiko ghaṭo vāri dhārayatīti tatkālabhāvino 'py anyasya deśāntaravartino na vāridhāraṇavāraṇam, tathā dvitīyādikṣaṇo 'py anyo yadi vāri dhārayati, kīdṛśo doṣaḥ syāt | visadṛśakriyāyāṃ tu cintaiva nāsti | tat kathaṃ pratyabhijñānasya saṃvādasambhavaḥ | \edlabel{thakur75-117.12}\label{thakur75-117.12} nanu yady ekam pratyabhijñānaṃ visaṃvādi dṛṣṭam iti sarvam eva pratyabhijñānaṃ visaṃvādi śaṃkyate, tadaikam indriyajñānaṃ keśoṇḍukadvicandrādau visaṃvādyupalabdham iti ghaṭādiṣv api sarvam eva pratyakṣaṃ visaṃvādi sambhāvyatām | indriyajanyatvasyaikalakṣaṇasya sarvatra sambhavād iti cet | \edlabel{thakur75-117.15}\label{thakur75-117.15} na, tatrāpi lakṣaṇabhedasya sadbhāvāt | tathā hi bahirarthasthitāv indriyārthakāryatayā sākṣād arthākārānukāritvaṃ pratyakṣatvam | tac cābhyāsaviśeṣāsāditapaṭimnā pratyakṣeṇa niścīyate | kvacit tv arthakriyāprāptijñānād iti pratyakṣatvaṃ anavadyam eva | dvicandrādau tv arthavinākṛtena timirādiviplutacakṣurmātreṇa tajjñānaṃ janitam iti pratyakṣābhāsam eva | dvicandrādyarthābhāvas tu deśakālanarāntarair dvicandrāder arthasya bādhitatvād avyāhata iti pratyakṣābhāsapariihāre 'pi pratyakṣeṣu ka āśvāsavirodhaḥ | \edlabel{thakur75-117.21}\label{thakur75-117.21} pratyabhijñāne 'pi sarvam idam astīti na yuktam | yathā hi pūrvaṃ pāvakādau pākādikriyā pratibaddhā siddhā paścād anubhūyamānā dahanajñānasya saṃvādam āvedayati | anyathā bāhyārthocchedān nirīhaṃ jagaj jāyate | na tathā prathamacaramakālayor ekībhāvapratibaddhā kācid arthakriyā upalabdhigocarā pūrvāparakālayor ekatvam antareṇa vā pravṛttyādikṣatir yenaikatāvagraho 'pi saṃvādī syāt | \edlabel{thakur75-117.26}\label{thakur75-117.26} tad iyam anumānabādhitatvād vyabhicāraśaṅkākalaṅkitatvāc ca na pratyakṣam anumānaṃ veti | katham ataḥ sthairyasiddhir anumānapratihatir vā | \edlabel{thakur75-117.28}\label{thakur75-117.28} yat punar Vācaspatir uvāca | saṃskārendriyayor militayor eva pratyabhijñānaṃ prati kāraṇatvam iti, tad ayuktam | bhinnasāmagrīprasūtatvād anayor jñānayoḥ | tathā hi nimīlite cakṣuṣi sa ity atrendriyavinākṛtasyaiva saṃskārasya sāmarthyam upalabdham | prathamadarśane tv ayam ity atra saṃskārarahitasyaivendriyasya sāmarthyaṃ dṛṣṭam | tasmāt sāmagrīdvayapratibaddhaṃ jñānadvayam idam avadhāritam | katham ubhābhyāṃ militvaikam eva pratyabhijñānam utpāditam ity udghuṣyate | bījakṣityādyos tu pṛthak sāmarthyaṃ na dṛṣṭam ity ekaiva sāmagrīty aṅkuro 'py eka evāstu | \edlabel{thakur75-118.3}\label{thakur75-118.3} tathā pūrvadeśakālāparadeśakālābhyāṃ tatsambaddhābhyām anyatvāt padmarāgasyābheda ity apy asaṅgatam | viruddhayor dharmayoḥ padmarāgād anyatve 'pi viruddhadharmayogāt padmarāgasya bhedaḥ katham apahnūyate | trailokaikatvaprasaṅgasya durvāratvāt | na hi dharmadharmiṇor anyatve 'pi brāhmaṇatvacaṇḍālatve ekādhāre bhavitum arhata iti padmarāgasya bhedo duratikramaḥ | \edlabel{thakur75-118.7}\label{thakur75-118.7} tathā ca na svabhāvavirodho 'numānasyāpy anekatvaprasaṅgāt | tad api pratyakṣam apratyakṣaṃ cāvikalpo vikalpaś cāsamāropaḥ samāropaś cety apy ayuktam | anumānasya hi paramārthataḥ svasaṃvedanapratyakṣātmano 'vikalpasyāsamāropasvabhāvasyāpartyakṣatvavikalpatvasamāropatvādeḥ parāpekṣayā prajñaptatvād viruddhadharmādhyāsābhāvāt kathaṃ bhedasiddhiḥ | sa evāyam iti tu pratyabhijñānasya sa ity aspaṣṭākārayogitvam, ayam iti spaṣṭākārayogitvam iti viruddhadharmadvayaṃ bhedakam | \edlabel{thakur75-118.13}\label{thakur75-118.13} nacaivaṃ vaktavyam | tattedantāpekṣayā pratyabhijñānasyāpy ekasyaiva pārokṣyāpārokṣyam aviruddham iti | na hīdam ekākāratayā vyavasthitam, yenānumānavad asyāpi pārokṣyāpārokṣyavyavasthāmātraṃ syāt | yāvad atītārthākārānukāro vartamānārthānukāraś ca svadharmo na bhavati tāvat tadarthagocarataiva nāsti | kutaḥ pārokṣyāpārokṣyavyavahāro bhaviṣyati | tasmāt spaṣṭāspaṣṭākāradvayaviruddhadharmādhyāsāt pratyabhijñānaṃ pratyayadvayam etad iti sthitam || \edlabel{thakur75-118.19}\label{thakur75-118.19} tathā sahetukavināśatvād ayam apy asiddho hetuḥ | yat punar atroktam | sahetukavināśatvaṃ ghaṭasyāgnidhūmayor iva pratyakṣānupalambhato mudgraghaṭavināśayor api kāryakāraṇabhāvasiddhau siddham iti | tad asaṅgatam | agnidhūmayor api dṛśyatvāt, pratyakṣānupalambhato dhūmasya vahnikāryatā sidhyatu | vināśaśabdavācyas tv artho na kaścid idantayā dṛṣṭaḥ | karparam eva ghaṭamudgarābhyām utpadyamānam upalabdham | \edlabel{thakur75-118.23}\label{thakur75-118.23} yad āhur guravaḥ |
	\pend
      

	  \pstart dṛṣṭas tāvad ayaṃ ghaṭo 'tra ca patan dṛṣṭas tathā mudgaro dṛṣṭā karparasaṃhatiḥ paramato nāśo na dṛṣṭaḥ paraḥ | tenābhāva iti śrutiḥ kva nihitā kiṃ vātra tatkāraṇaṃ svādhīnā palighasya kevalam iyaṃ dṛṣṭā kapālāvaliḥ || \edtext{}{\lemma{||}\Bfootnote{(JNA 107,13ff.)}}
	\pend
      

	  \pstart tad ayam abhāvo dṛśyānupalabdhibādhitaḥ kathaṃ pratyakṣato mudgarādikāryam avadhāryaḥ | \edlabel{thakur75-118.29}\label{thakur75-118.29} yat punar asminn adṛśyamāne 'pi dṛśyata iti bāgjālaṃ sā bhaṇḍavidyā | tadvacanād gṛhṇann api paśur eva | tatha hi
	\pend
      

	  \pstart kasyacit pratibhāsena sādhyate 'pratibhāsi yat | pratibhāso 'sya nāsyeti nopapattes tu gocaraḥ || iti | \edlabel{thakur75-119.1}\label{thakur75-119.1} athaivaṃ vaktavyam | kim anyena dhvaṃsena, karparam eva ghaṭadhvaṃso 'stu | tathā ca sati mudgarādyabhāve karparābhāvāt ghaṭasthairyam avyāhatam iti \edlabel{thakur75-119.2}\label{thakur75-119.2} durāśā khalv eṣā | tathā hi yathā nāśaśabdena karparam ucyate tathā yady abhāvaśabdenāpi karparam evocyate tadaikatra pradeśe ghaṭam ekam apanīya ghaṭāntaranyāse tatrāpanītaghaṭasyābhāvavyavahāro na syāt | tatpradhvaṃsakapālayos tatrānutpādāt | tasmād yathāpanītaghaṭasya pracyutimātrāpekṣayā nyastaghaṭe 'bhāvavyavahāras tathā mudgarādikāraṇābhāvāt pradhvaṃsakarparayor anupāde 'pi pracyutimātrāpekṣayaiva pratikṣaṇam anyānyatvavyavahāro ghaṭasya sidhyatīti kutaḥ sthairyasiddhiḥ | tasmāt pradhvaṃsakarparābhāve 'pi pracyutimātrātmakabhāvāpekṣayāpy asmanmatam avyāhatam | \edlabel{thakur75-119.9}\label{thakur75-119.9} yad āhur guravaḥ |
	\pend
      

	  \pstart āstāṃ karparapaṃktir eva kalaśadhvaṃso na ceyaṃ purā tena sthairyam api prasidhyatu tato bhinnena nāśena kim |
	\pend
      

	  \pstart atrottaram,
	\pend
      

	  \pstart nāsaḥ saiva yathocyate yadi tathābhāvo 'pi kumbhāntaranyāse 'bhāvavacaḥ kathaṃ matam ataḥ sidhyaty abhāve 'pi naḥ || iti | \edtext{}{\lemma{|}\Bfootnote{(JNA 108,4ff.)}} \edlabel{thakur75-119.16}\label{thakur75-119.16} nanu yadi svahetujanito nāśo nāsti, kathaṃ kvacid eva deśe kāle ghaṭo naṣṭa iti pratītiniyamaḥ | na ca mudgarād anyo nāśasya hetur vaktavyaḥ | prāg api nāśasambhave naṣṭaghaṭabuddhisambhavaprasaṅgāt | yad āhuḥ |
	\pend
      

	  \pstart nāśo nāsti yadi svahetuniyataḥ kiṃ desakāle kvacit kumbho naṣta iti pratītiniyamas tenāsti kāryaś ca saḥ | nāpy anayat kila kāraṇaṃ rayavato daṇḍāt purāpy anyathā nāśotthānakṛtā vinaṣṭaghaṭadhīḥ kenoddhurā vāryate || \edtext{}{\lemma{||}\Bfootnote{(JNA 108,21ff.)}}
	\pend
      

	  \pstart iti cet | \edlabel{thakur75-119.23}\label{thakur75-119.23} tarhīdānīm arthāpattyā pradhvaṃsaṃ prasādhya mudgarādhīnatvam asya sādhayitum ārabdham | tathā ca sati dhūmāgnivat pratyakṣataḥ pradhvaṃsasya mudgarādikāryatvaṃ siddham ity utphullagallam ullapitaṃ vyāluptam | \edlabel{thakur75-119.26}\label{thakur75-119.26} na cārthāpattito 'pi tatsiddhiḥ sampadyate, ghaṭo naṣṭa iti pratīter anyathāpy upapadyamānatvāt | vināśaṃ vināpi hi ghaṭadarśanavato mudgarakṛtakapālānubhava eva naṣṭaghaṭāvasāyasādhanaḥ, kim apareṇa nāśena kartavyam | ghaṭo naṣṭa iti buddher ghaṭaniścayapūrvakamudgarakṛtakapālānubhavamātrānvayavyatirekānuvidhānadarśanāt | \edlabel{thakur75-119.29}\label{thakur75-119.29} na ceyaṃ sāmagrī pūrvam apy asti | mudgarābhāve karparapaṃkter evābhāvāt kathaṃ prāg api naṣṭaghaṭabuddhiprasaṅgaḥ saṅgato nāma | \edlabel{thakur75-119.31}\label{thakur75-119.31} yad āhur guravaḥ |
	\pend
      

	  \pstart dṛṣte 'mbhobhṛti mudgarādijanitāṃ dṛṣtvā kapālāvalīṃ saṅketānugamād vinaṣṭaghaṭadhīs tāvat samutpādyate | sāmagryām iha nāśanāma na kim apy aṅgaṃ na cāsyām api syād eṣā na kadāpi nāpi ca purāpy eṣā samagrā sthitiḥ || arthāpattir ato gatā kṣayam iyaṃ na dhvaṃsasiddhau prabhuḥ | iti | \edtext{}{\lemma{|}\Bfootnote{(JNA 109,4ff; 23)}} \edlabel{thakur75-120.4}\label{thakur75-120.4} yadi nāśānubhavo nāsti kapālānubhavāt kapālakalpanaiva syāt | na naṣṭaghaṭabuddhir iti cet | \edlabel{thakur75-120.5}\label{thakur75-120.5} tad etad atisāhasam | ghaṭaniścayapūrvakakapālavalayadarśanād eva naṣṭaghaṭabuddheḥ sākṣād evānubhūyamānatvāt | tadapalāpe dhūmādīnām api dahanādipūrvakatvaniścayo na syād ity atiprasaṅgaḥ | \edlabel{thakur75-120.8}\label{thakur75-120.8} nanu ghaṭo naṣṭa iti buddhir viśeṣyabuddhiḥ | sā ca vināśaṃ viśeṣaṇam ākṣipatīti cet | \edlabel{thakur75-120.9}\label{thakur75-120.9} tad asat, yataḥ |
	\pend
      

	  \pstart svabuddhyā rajyate yena viśeṣyaṃ tad viśeṣaṇam | \edtext{}{\lemma{|}\Bfootnote{(JNA 110,1)}}
	\pend
      

	  \pstart ucyate | na cāvidyamānam adṛśyaṃ vā svabuddhyā kiñcid rañjyati | \edlabel{thakur75-120.11}\label{thakur75-120.11} prayogo 'tra | yasya na svarūpanirbhāsas tan na kasyacit svānuraktapratītinimittam | yathā karikeśaraḥ | nāsti ca svarūpanirbhāso dhvaṃsasyeti vyāpakānupalabdhiḥ | nāsyā asiddhiḥ | abhāvasya svarūpeṇaivedantayā nirbhāsābhāvāt | na ca viruddhatā, sapakṣe bhāvāt | nāpy anaikāntikatvam | pratibhāsābhāve 'pi svānuraktapratītihetutve śaśaviṣāṇāder api tathātvaṃ syād ity atiprasaṅgaḥ | \edlabel{thakur75-120.17}\label{thakur75-120.17} nanu
	\pend
      

	  \pstart na dhvaṃsena vinā vinaśyati jagad bhāvena sārdhaṃ sa cet sac cāsac ca kim astu vastu niyataṃ bhāvānujo 'sau tataḥ| bhāvāt tena tu bhinnakāraṇatayā tatkāraṇāsambhave 'bhāvāt tena kṛtānyatāpi galitā bhaṅgaḥ koto 'nukṣaṇaṃ || \edtext{}{\lemma{||}\Bfootnote{(JNA 117,23ff.)}} \edlabel{thakur75-120.22}\label{thakur75-120.22} atrocyate | kāraṇāntarād utpadyamāno dhvaṃso 'bhinno bhinno vā | \edlabel{thakur75-120.22a}\label{thakur75-120.22a} nādyaḥ pakṣaḥ | bhinnakāraṇatvāt, tair anabhyupagatatvāc ca | atha dvitīyaḥ pakṣaḥ | tadā kaḥ punar bhāvasya pradveṣo yena pradhvaṃsākhye vastuni svahetor utpanne nivartate nāma | \edlabel{thakur75-120.25}\label{thakur75-120.25} yat punar etad ucyate | nābhāvasyotpāde bhāvasya parā nivṛttiḥ | kiṃ tv abhāvotpattir eva tannivṛttir iti | katham anyasyotpāde 'nyasya nivṛttiḥ | atra svabhāvabhedair uttaraṃ vācyam ye parasparaparihārasthitayaḥ svahetubhyo jāyante, na hi svato 'nyasyāṅkurasya vahnir na kāraṇam ity anyatvāviśeṣād bhasmano 'pi na kāraṇam | svabhāvabhedena tu kāryakāraṇabhāvasamarthanaṃ parasparaparihārasthitiniyame 'pi tulyam | yathā cotpādasya purastād akhilasāmarthyarahitasyāṅkuraprāgabhāvasyāpakāraṃ kiñcid akurvanto 'pi bījādayo 'ṅkuram ārabhamāṇāḥ prāgabhāvaṃ nivartayanti | tadutpādasyaiva tatprāgabhāvanivṛttirūpatvāt | evaṃ tadabhāvahetavo 'pi bhāvarūpe 'kiñcitkarā api tadabhāvam ādadhānās tan nivartayanti | abhāvotpādasyaiva bhāvanivṛttirūpatvāt | tena pūrvavan nārthakriyākaraṇaprasaṅga iti | \edlabel{thakur75-121.2}\label{thakur75-121.2} tad ucitaṃ syād yadi kāryakāraṇayor evāsyāpy ātmā pramāṇapratītaḥ syāt | kevalaṃ dṛśyānupalambhagraste 'py etasminn upalabhyata iti pralāpo vyaktam iyaṃ bhaṇḍavidyety uktam | \edlabel{thakur75-121.4}\label{thakur75-121.4} arthāpattir api kṣīṇety api prāgabhāvasya ca dṛṣṭāntatvenopanyāso bhaṇḍālekhyanyāyaḥ | \edlabel{thakur75-121.6}\label{thakur75-121.6} kiñ ca kaḥ punar atra virodhaḥ |
	\pend
      

	  \pstart sahasthānābhāvo yadi tava virodho 'rthavipadoḥ sahasthānāsaṅgaḥ kṣaṇam api yathā śītaśikhinoḥ | sa ca dhvaṃso dhvaṃsāntaram upanayan saṃprati bhaved virodhī so 'py anyaṃ kṣayam iti na nāśaḥ katham api || \edtext{}{\lemma{||}\Bfootnote{(JNA 115,16ff.)}} \edlabel{thakur75-121.11}\label{thakur75-121.11} anyathā siddhasattāmātreṇa virodhitve sarvaṃ sarveṇa viruddhaṃ prasajyeta | svabhāvālambhanam apy adarśanād eva nirastam iti |
	\pend
      
	    
	    \stanza[\smallbreak]
	athānyonyābhāvaprakṛtikatayārthe sati tadā kṣayasyaivābhāvaḥ saha bhavatu vā hetubalataḥ |&anena dhvaṃse ca prakṛtahatir asya tv anudaye balīyān evārthaḥ svayam apacaye 'nyena kim iha || \edtext{}{\lemma{||}\Bfootnote{(JNA 119,20ff.)}}\&[\smallbreak]


	

	  \pstart sac cāsac ca kim astu vastv iti tu prasaṅgas \persName{trilocana}prastāve nirākaraṇīyaḥ | ata evātra prastāve bhuvanaikagurūn bhagavataḥ Kīrtipādān avamanyamānaḥ êaṅkaraḥ paśor api paśur iti kṛpāpātram evaiṣa jālmaḥ |
	\pend
      

	  \pstart yad apy āha \persName{Trilocanaḥ} | bhāvavyatiriktāṃ nivṛttim anicchadbhir aśakyā svarūpanivṛttir avasthāpayitum | yā hi tasya prāktanī kācid avasthā bhavadbhir arthakriyānirvartanayogyā dṛṣṭā saiva yady uttarakālam apy anuvartate tarhi svarūpeṇaiva nivṛtto bhāvaḥ katham avasthāpyate | tadānīm ayaṃ naṣṭo nāma yadi svahetupratilabdhasvarūpavyatirekinī tasya kācid avasthotpādyata, utpattau saiva tasyātmāntaraṃ jātam ity atādavasthyam evāsya vināśaṃ brūmaḥ | tādavasthyatādātmye ca svarūpeṇa nivṛtto bhāva ity asya śabdasya satyam arthaṃ na vidmaḥ |
	\pend
      

	  \pstart svarūpanivṛttiḥ khalv iyaṃ bhavantī bhāva eva syāt, bhāvād anyā vā | tattve svakāraṇebhyo niṣpannasyārthasyānyathānupapattāv utpatter ārabhya sattvān nityatvaṃ prasajyeta | anyatve ca tad eva nivṛtter anyatvanirvṛtir iti priyam anuṣṭhitaṃ priyeṇa | tasmād utsṛjya vibhramaṃ nāśotpattir eva naṣṭatvam abhyupagantavyam iti | \edlabel{thakur75-122.1}\label{thakur75-122.1} tad etad ajñānaphalam | tathā hi
	\pend
      
	    
	    \stanza[\smallbreak]
	svakāraṇād eva yathānyadeśavicchinnarūpaḥ samudeti bhāvaḥ |&vicchinnabhinnakṣaṇavṛttir evaṃ svakāraṇād eva na jāyate kim ||&abhāvato 'rthāntararūpabādhe tatrāpy abhāvāntaram īkṣaṇīyam |&pradīpadṛṣṭāntamataṃ na kāntaṃ svarūpasandarśanaviprayogāt ||\edtext{}{\lemma{||}\Bfootnote{JNA 140,4ff.}}\&[\smallbreak]


	

	  \pstart yathā hi deśāntaraparāvṛttam anīlādiparāvṛttaṃ ca svahetor utpannaṃ vastu tathā dvitīyakṣaṇātaraparāvṛttaṃ api | yathā cānyadeśānavasthāyitvaṃ taddeśāvasthāyitvenāviruddham, viruddhaṃ ca deśāntarāvasthāyitvenaiva | tathā dvitīyakṣaṇānavasthāyitvaṃ prathamakṣaṇāvasthāyitvenāviruddham | viruddhaṃ punar dvitīyakṣaṇāvasthāyitvenaiva | kevalaṃ deśāntaradvitīyakṣaṇayos tatpracyutimātraṃ vyavahriyate | tad anyonyābhāvapradhvaṃsābhāvayoḥ padārthayoḥ sadbhāve 'py avāryam | abhāvāntarāsvīkāre 'pi bhāvābhāvayor apy amiśratvāsvīkāre tādātmyaprasaṅgāt | tasmād abhāvābhāvayos tādātmyam iti | \edlabel{thakur75-122.16}\label{thakur75-122.16} yathārthakriyākāritvasya taddeśavartitvanīlatvādibhinnavirodhas tathā dvitīyakṣaṇānavasthāyitvenāpīti vivakṣitam | paramārthatas tu dharmidharmayos tādātmyaṃ vyāvṛttikṛto bhedavyavahāra iti \name{apohasiddhau} prasādhitam | etac coktakrameṇāviruddham āpāditam | evāvati tu tattve vākchalamātrapravṛttā dveṣaviṣajvalitātmānaḥ kṣudrāḥ pralapantīti kim atra brūmaḥ | \edlabel{thakur75-122.21}\label{thakur75-122.21} tataś ca vyatiriktanivṛttyutpattim antareṇa svarūpanivṛtter upapatteḥ kathaṃ kṣaṇād ūrdhvaṃ prāktanasattāvasthitiḥ | tasmād utsṛṣṭavibhramaṃ naṣṭavyavahāramātram astu | na tv asyānyat kiñcij jāyeta | \edlabel{thakur75-122.23}\label{thakur75-122.23} bhāvasya tādavarthyaprasaṅgāt | abhāvaḥ kathaṃ niṣidhyata iti cet | \edlabel{thakur75-122.24}\label{thakur75-122.24} na, tadanutpattimātraviṣayasya vācāniścayena ca paścād abhāvavyavahāramātrapravartanasyeṣṭatvād vastūtpatter eva niṣiddhatvāt | \edlabel{thakur75-122.26}\label{thakur75-122.26} nanu keyaṃ vācoyuktiḥ, abhāvavyavahāramātram iṣyate paścān nābhāva iti | evaṃ sati visaṃvāditāprasaṅgo abhāvavyavahārasya | abhāvaś ca mithyeti bhāva eva pratiṣeddhavyaḥ syāt | sa cābhāvaḥ paścād bhavatīti sphuṭataram asya kādācitkatvam ātmahetukatvam, vastutvaṃ ceti | \edlabel{thakur75-122.29}\label{thakur75-122.29} asad etat | abhāvākhyavastvantarāsvīkāre 'pi pracyutimātrāpekṣayāpi vyavahārasya caritārthatvapratipādanāt | yat tu tadviviktabhūtalāder viṣayatvam āśaṅkyoktam, na bhūtalāder vastvantaratvāt | na ca vastvantare pratipādite pratīte vā ghaṭādi vastubhūtam iti pratipāditaṃ vā bhavati | \edlabel{thakur75-123.1}\label{thakur75-123.1} evaṃ vastvantaram eva nāśa iti | asmin mate yad dūṣaṇam uktaṃ tat svayam eva parihṛtaṃ syād iti, tad apy asambaddhaṃ, kevalaṃ hi bhūtalam asya viṣaya iti kathaṃ na ghaṭāder abhūtatvabodhaḥ | yaiva hi ghaṭādyapekṣayā kaivalyāvasthā pradeśasya sa eva ghaṭavirahaḥ | vacanādināpy evaṃ kevalapradeśapratipādane katham iva na prakṛtaghaṭādyabhāvapratipādanam | kaivalyaṃ cāsahāyapraseśād avyatibhinnam eva | \edlabel{thakur75-123.6}\label{thakur75-123.6} na ceha ghaṭo nāstīti pratyayasya ghaṭavaty api pradeśe prasaṅgaḥ | svahetos tathotpannasya saghaṭapradeśasya kevalapradeśād anyatvāt | \edlabel{thakur75-123.7}\label{thakur75-123.7} na ca pratyabhijñānataḥ saghaṭāghaṭapradeśayor ekatvaṃ pūrvam asya nirākaraṇāt | \edlabel{thakur75-123.8}\label{thakur75-123.8} na ca vināśahetor asāmarthyavaiyarthyābhidhāne 'ṅkurādihetor api tathābhidhātum ucitam | asiddhe hi kārye hetor āśrayaṇam avāryam | siddhe ceyaṃ cintā, yadi hetor nityo 'nityo vā 'rtho jātaḥ kiṃ nāśakāraṇeneti hetupuraskāreṇaiva pravṛtteḥ | na caivam asiddhe 'ṅkurādau kārye śakyam abhidhātum | svarūpasyaivābhāvāt | taddharmakatvā[tad]dharmakatvādiparyanuyogasya nirviṣayatvāt | \edlabel{thakur75-123.13}\label{thakur75-123.13} nanu tvayāpi bhāvābhāvayor lakṣaṇabhedo 'bhihitaḥ | tat katham ekatvaṃ sarvārthānām | lakṣaṇabhedād eva bhedavyavasthā | tato 'pi cen na bhedavyavasthitiḥ, na kasyacit kutaścid bhedavyavasthitir ity advaitaprasaṅga iti cet | \edlabel{thakur75-123.15}\label{thakur75-123.15} na | yo hi naśvarasvabhāvaḥ sa eva nāśo naśyatīti bahulādhikārāt kartari ghañaḥ prasādhanāt taṃ nāśaṃ bhāvasvabhāvam icchāmaḥ | naśanaṃ nāśa iti prasajyātmā dvidhā kartavyaḥ | tattvatas tāvad vastutvavirahāt tattvānyatvavirahita evāsau bhāvo na bhavatīti tadbhāvaniṣedhamātram āyātaṃ tu bhavati | kharaśṛṅgādivat | saṃvṛtau tu yathā kālabhedena vikalpyamānaḥ kādācitka iva pratibhāti tathā sarvopākhyāviraharūpatayā bhāvād bhinna iva pratibhātīti nāvastutvopalakṣaṇabhedākhyānavirodhaḥ | evaṃ ca sati saṃvṛttyā lakṣaṇabhede bhāvābhāvayor bhedasyeṣṭatvāt | tattvena ca lakṣaṇaikatāvirahe bhāvasya tenaikyaniṣedhāt katham advaitaprasaṅgopālambhaḥ | \edlabel{thakur75-123.24}\label{thakur75-123.24} syād etat | na ca vivekāpratītau tadviviktagrahaṇaṃ bhavati | tadvivekaś ca na bhūtalādisvarūpam eva viśeṣaṇatvād iti | \edlabel{thakur75-123.25}\label{thakur75-123.25} tad etan nyāyabahiṣkṛtam | viśeṣaṇaviśeṣyabhāvo hi saṅkalpārūḍhe rūpe bāhyārthasparśe vikalpaśabdaliṅgāntarāṇāṃ vaiyarthyaprasaṅgād iti śāstre vistareṇa pratipādanāt | sa ca saṅkalpo 'bhinnam api bhāvaṃ bhinnam ivākalayati | yathā śilāputrakasya śarīram, śarīre karaṇādayaḥ | lambakarṇo Devadatta ityādi | tasmāt kalpanādhīno viśeṣaṇaviśeṣyabhāvaḥ | abhinne 'pi bhāve bhedavivakṣāpekṣo bhedavyavahāraḥ kathaṃ bhedaniyatam ātmānam ātanotu | \edlabel{thakur75-123.31}\label{thakur75-123.31} skhaladgatir ayaṃ rāhoḥ śira ity ādinirdeśa itic cet | \edlabel{thakur75-123.31a}\label{thakur75-123.31a} yadi satyam etat, tadā śiro 'tiriktasya rāhor iva kṣmātalāder atiriktasya vivektasya dṛśyānupalambhabādhitatvād ayam api nirdeśaḥ skhaladgatir eva, tathāpi neti koṣapānaṃ pramāṇam | tasmāt saghaṭāt pradeśāntarāt pradeśa evāyam anyo ghaṭaviviktaḥ svahetor utpanno na tu ghaṭavivekena viśeṣitaḥ | svahetor utpannasya viviktasyābhāve vivekasyābhāvāt | \edlabel{thakur75-124.3}\label{thakur75-124.3} kiṃ ca
	\pend
      

	  \pstart vyāptaṃ bhidā yadi viśeṣyaviśeṣaṇatvaṃ bhedātyayān nanu tadā tadabhāva eva | deśo viśiṣṭa iti nāsti yathā tathedam apy asti dṛśyamatabhedadṛg asti neti || \edtext{}{\lemma{||}\Bfootnote{(JNA 150,24ff.)}} \edlabel{thakur75-124.8}\label{thakur75-124.8} tasmān nābhāvo nāma kaścid yatra kāraṇavyāpāraḥ | tad evaṃ sahetukavināśatvād iti hetuḥ svarūpāsiddha iti sthitam || \edlabel{thakur75-124.10}\label{thakur75-124.10} satām akṣaṇikatvaṃ kāraṇavattvād ity apy asambaddham eva | kṣaṇikatvakāraṇavattvayor virodhābhāvād akṣaṇikatvena kāraṇavattvasya vyāpter asiddheḥ | sandigdhavyatirekatvāt | na cāsya viparyaye vṛttiśaṅkā nāśasya sahetukatvam eva nivartayati | uktakrameṇa nāśasyaivābhāvād iti || \edlabel{thakur75-124.14}\label{thakur75-124.14} tathā prameyatvād api sthirasiddhir manorathamātram | sākāravedanodayapakṣasthitau hi dvitīyakṣaṇānuvṛttāv apy arthasya vyavahitatvāt, prakāśānupapatter viṣayasvarūpavedanam eva jñānasya viṣayavedanam | evaṃ ca vartamānānurodhaḥ, atīte 'pi tatpratyāsatter apracyuteḥ | na cātiprasaṅgaḥ | anantarātītād anyena kṣaṇena sārūpyāsamarpaṇāt | tataś ca kāraṇatvād yadi nāma prameyatvasya pūrvakālasattvena vyāptis tathāpi prameyatvavat pūrvakālasattvam api kṣaṇike 'viruddham iti prameyatvākṣaṇikatvayor vyāptisādhano vyāpakānupalambho 'siddhaḥ | jñānākārārpakatvaṃ hi hetutvam, prameyatvaṃ prāmāṇikapratītam | tac cānantarātīta eva kṣaṇe samupapadyate | \edlabel{thakur75-124.22}\label{thakur75-124.22} jñānasattāsamaye 'rthānuvṛtter abhāvān nirviṣayateti cet | \edlabel{thakur75-124.22a}\label{thakur75-124.22a} nanv ananuvṛttāv api tadarpitākārasvarūpasaṃvedanam eva tadvedanam | tad eva ca saviṣayatvam | iyaṃ ca pratyāsattir anantarātīte 'pi kṣaṇe 'kṣīneti na dvitīyakṣaṇānuvṛtter anurodha ity uktam | ataḥ sandigdhavyatirekitvād anaikāntikam eva prameyatvam | \edlabel{thakur75-124.26}\label{thakur75-124.26} atha sākāravādavidveṣād anākārajñānagrāhyatvaṃ prameyatvam abhipretaṃ tadā 'siddhatā 'sya hetoḥ | \edlabel{thakur75-124.27}\label{thakur75-124.27} indriyārthasannikarṣāder jñānam utpadyatāṃ nāma | \edlabel{thakur75-124.27a}\label{thakur75-124.27a} tac cānubhavaikarasatvena sarvatrārthe sadṛśākāratvāt kasya grāhakam astu, \edlabel{thakur75-124.28}\label{thakur75-124.28} yenābhisambaddham iti cet | \edlabel{thakur75-124.28a}\label{thakur75-124.28a} ātamamanaḥsaṃyogādīnām api grahaṇaṃ syāt | \edlabel{thakur75-124.29}\label{thakur75-124.29} janakasya grahaṇam iti cet | \edlabel{thakur75-124.29a}\label{thakur75-124.29a} tathāpy ātmādīnāṃ grahaṇaprasaṅgaḥ | viṣayatvena janakasya grahaṇam ity apy asādhu | viṣayatvasyādyāpy aniścayāt | \edlabel{thakur75-124.30}\label{thakur75-124.30} idaṃ dṛṣṭam śrutaṃ vedam ity adhyavasāyo yatrārthe sa viṣaya iti cet | \edlabel{thakur75-124.31}\label{thakur75-124.31} nanv asty eva pratiniyato vyavahāraḥ | kaḥ punar atra pratyāsattiniyama iti pṛcchāmaḥ | sa ced upavarṇayituṃ na śakyate, vyavahāro 'pi tvanmate niyato na syād iti brūmaḥ | \edlabel{thakur75-124.33}\label{thakur75-124.33} asti tāvad iti cet | \edlabel{thakur75-124.34}\label{thakur75-124.34} ata evārthasārūpyam asādhāraṇaṃ pratyāsattinimittam astu | nirnimitte niyamāyogāt | \edlabel{thakur75-125.1}\label{thakur75-125.1} nanu sārūpyam apy arthādarśane katham avadhāryate | tac ca kim ekadeśena, sarvātmanā vā | ādye pakṣe sarvaṃ sarvasya vedanaṃ syāt | dvitīye tu jñānam ajñānatāṃ vrajet | kiṃ ca sārūpyād arthavedane 'nantaraṃ jñānaṃ tulyaviṣayaṃ viśayaḥ syād iti cet | \edlabel{thakur75-125.4}\label{thakur75-125.4} mā bhūd arthasya darśanam | ākāraviśeṣabalād adhyavasitārthasyārthakriyāprāpter evārtho 'pīdṛśa iti sārūpyavyavahāro 'viruddhaḥ | ata eva sthūlagataṃ paramāṇugataṃ vā sārūpyaṃ na cintyate | jñānākārasya sthūlatve 'py ekasāmagrīpratibaddhapuñjaviśeṣād apy abhīṣṭakriyākaraṇāt puruṣārthasiddheḥ | \edlabel{thakur75-125.7}\label{thakur75-125.7} sārūpyaṃ caikadeśenaiva | na cātra sarvavedanaprasaṅgaḥ | sarveṣāṃ jñānaṃ praty ajanakatvāt | janakānāṃ ca svavyapadeśanimittāsādhāraṇaikadeśārpakatvena grāhyatvāt | \edlabel{thakur75-125.9}\label{thakur75-125.9} nāpi tulyaviṣayānantarajñānagrahaṇaprasaṅgaḥ, tasya svasaṃvedanād eva pramāṇāt siddhatvāt | pramāṇāntarasya tatra vaiyarthyāt | jaḍatve saty ākārārpakasya vastuno grāhyatvād ity asyārthasyābhīṣṭatvāc ca | bāhyārthasthitau ceyaṃ cinteti sarvam anavadyam | \edlabel{thakur75-125.12}\label{thakur75-125.12} tad evam ayaṃ prameyatvād iti hetuḥ sākāravādapakṣe sandigdhavyatirekaḥ | nirākārapakṣe cāsiddha iti sthitam || \edlabel{thakur75-125.14}\label{thakur75-125.14} na cārthāpattir api sthirātmasādhanī | kāryakāraṇabhāvagrahaṇādīnām anyathopapatteḥ | \edlabel{thakur75-125.15}\label{thakur75-125.15} tathā hi upādānopādheyabhāvasthitacittasantatim apy āśrityeyaṃ vyavasthā sustheti katham ātmānaṃ pratyujjīvayatu | tatra kāryakāraṇabhāvapratītis tāvad anākulā | tathāpi prāgbhāvivastuniścayajñānasyopādeyabhūtena tadarpitasaṃskāragarbheṇa paścādbhāvivastujñānenāsmin satīdaṃ bhavatīti niścayo janyate | tathā prāgbhāvivastvapekṣayā kevalabhūtalaniścayakajñānopādeyabhūtena tadarpitasaṃskāragarbheṇa paścādbhāvivastvapekṣayā kevalabhūtalaniścāyakajñānenāsmin asatīdaṃ na bhavatīti vyatirekaniścayo janyate | yathoktam |
	\pend
      
	    
	    \stanza[\smallbreak]
	ekāvasāyasamantarajātam anyavijñānam anvayavimarśam upādadhāti |&evaṃ tadekavirahānubhavodbhavānyavyāvṛttidhīḥ prathayati vyatirekabuddhim ||\&[\smallbreak]


	

	  \pstart ata eva devadattenāgnau pratīte yajñadattena ca dhūme pratīte na kāryakāraṇabhāvagrahaṇaṃ tajjñānayor upādānopādeyabhāvābhāvāt | yatra tv ekasantāne jñānakṣaṇayor upādānopādeyabhāvas tatra kāryādigrahaḥ sugrahaḥ | anyathā saty api nityātmani pratisandhātari kāryakāraṇabhāvādīnām apratītir eva syāt | \edlabel{thakur75-125.30}\label{thakur75-125.30} tathā hi ātmanaḥ sakāśāt pratisandheyabuddhīnām abhedo bhedo vā bhedābhedo vā | \edlabel{thakur75-125.31}\label{thakur75-125.31} prathamapakṣe ātmaiva syāt pratisandhātā | buddhaya eva vā syuḥ pratisandheyā iti kaḥ pratisandhārthaḥ | \edlabel{thakur75-125.32}\label{thakur75-125.32} bhedapakṣe 'pi buddhibhyo bhidyamānasya jaḍasyātmanaḥ kaḥ pratisandhānārtha iti na vidmaḥ | \edlabel{thakur75-125.33}\label{thakur75-125.33} buddhiyogād draṣṭṛtvavat pratisandhātṛtvam iti cet | \edlabel{thakur75-126.1}\label{thakur75-126.1} buddhir eva tarhi draṣṭrī pratisandhātrī ceti niyamasvīkāre tadyogād asya tathātvam iti kim anena yācitakamaṇḍanena | \edlabel{thakur75-126.2}\label{thakur75-126.2} buddhīnāṃ kartṛtvābhāvād iti cet | \edlabel{thakur75-126.2a}\label{thakur75-126.2a} taddvāreṇāpi tarhi tasyātmano draṣṭṛtvādivyavahārānupapattiḥ | yadi hi buddhir hetoḥ phalasya vā draṣṭṛī syāt tadānantaryapratiniyamasya cānusandhātrī kalpitā | tadyogād draṣṭṛtvaṃ pratisandhātṛtvaṃ cocyata iti syād api prativiṣayam alabdhaviśeṣāyāṃ ca buddhau sambandho 'pi na viśeṣaṃ vyavahārayitum īśaḥ | adhunā nibandhanādhigantā | adhunā phalasya | idānīṃ pratisandhāteti | tathāpi ca buddhiyutaviśeṣasvīkāre tu kim apareṇātmanā kartavyam | tāvataiva paryāptatvād vyavahārasya | \edlabel{thakur75-126.9}\label{thakur75-126.9} sthirātmānam antareṇa saiva buddhir na syād iti cet | \edlabel{thakur75-126.9a}\label{thakur75-126.9a} kenaivaṃ pratārito 'si | aho mohamāhātmyaṃ yad īdṛśān api paravaśīkaroti | tathā hi nedam idam antareṇa yad ucyate tat khalv anyatra pratyakṣānupalambhābhyāṃ sāmarthyāvadhāraṇe sati yujyate vahner iva dhūme | cakṣurādivad vā dṛṣṭakāraṇāntarasāmagyā kāryādarśane paścād darśane ca kiñcid anyad apekṣaṇīyam astīti sāmānyākāreṇa | \edlabel{thakur75-126.14}\label{thakur75-126.14} ādyaḥ pakṣas tāvan nāstīti vyaktam | dvitīyo 'pi na sambhavī | na hi kāraṇabuddhisamanantaraṃ kāryabuddhau satyāṃ niścayapravṛttasyedam asyānantaraṃ dṛṣṭam mayeti pratisandhānam adṛṣṭapūrvaṃ kadācit | yato 'nyasya sāmarthyaparikalpanaṃ syād ity udasya vyāmoham uktakrameṇaiva kārykāraṇagrahaṇavyavasthā svīkartavyā | \edlabel{thakur75-126.18}\label{thakur75-126.18} bhedābhedapakṣas tu dhakkāra eva | tasyaiva tadapekṣayā bhedābhedaviruddhadharmādhyāsād ekatvānupapatteḥ | tataś ca yad bhinnaṃ bhinnam evābhinnṃ cābhinnam iti naikasya bhedābhedau | tathapy abheda viśvam ekam iti yugapadutpādasthitipralayaprasaṅgaḥ | \edlabel{thakur75-126.20}\label{thakur75-126.20} evaṃ kramivastugrāhakaiḥ kramijñānair upādānopādeyabhūtaiḥ sākṣāt pāramparyeṇa krameṇāmī jāyanta iti niścayo janyate | ekakālikānekavastugrāhakair eva tajjñānair ekopādānatvāt sakṛd imāni jātānīti vikalpaḥ kriyata iti kramākramagrahaṇam apy anavadyam | \edlabel{thakur75-126.24}\label{thakur75-126.24} katham anekajñānād ekavikalpa iti cet | \edlabel{thakur75-126.24a}\label{thakur75-126.24a} ko doṣaḥ |
	\pend
      
	    
	    \stanza[\smallbreak]
	bhavantu bhinnā matayas tathāpi tā dadhaty upādānatayaikakalpanam |&na bhinnasaṃkhyā phalahetubādhanī na cānyasantānabhavā ivākṣamāḥ ||\&[\smallbreak]


	

	  \pstart yad apy uktaṃ \persName{Śaṅkareṇa}: atha pūrvottarakṣaṇayoḥ saṃvittī | tābhyāṃ vāsanā, tayā hetuphalabhāvādhyavasāyī vikalpa iti cet | \edlabel{thakur75-126.30}\label{thakur75-126.30} tat kim idānīṃ yat kiñcid āśaṅkitena | vaktavyam ity evaṃ vidhir anuṣṭhīyate bhavatā | vikalpo hy agṛhītānusandhānam atadrūpasamāropo vā syāt | \edlabel{thakur75-126.32}\label{thakur75-126.32} na tāvat pūrvaḥ pakṣaḥ | adṛṣṭānvayavyatirekasya puruṣasya hetuphalabhāvāgrahe 'nusandhānapratyayahetor vāsanāviśeṣasyaivānupapatteḥ | agṛhītasya cānusandhāne 'tiprasaṅgād iti | \edlabel{thakur75-127.1}\label{thakur75-127.1} tad etan na samyag ālocitam | yato hetuphalabhūtayoḥ pūrvottarakṣaṇayor ekaikena jñānenānanubhave 'py upādānopādheyabhūtābhyāṃ kramijñānābhyāṃ hetuphalatve gṛhīte eva | kevalaṃ hetukāle phalābhāvāt tadviṣayasāmarthyagrahaṇe 'pi phalādarśanāt tadavasāya evāpravṛttaḥ kāryadarśanena pravartyate | tathā phalāvalokane 'pi tatkāryatā gṛhītaiva vikalpenānusandhīyata iti gṛhītānusandhānarūpa evāyaṃ vikalpa iti yat kiñcid etat | \edlabel{thakur75-127.7}\label{thakur75-127.7} yad āha Mahābhāṣyālaṅkāraḥ |
	\pend
      
	    
	    \stanza[\smallbreak]
	\label{ratnakīrtinibandhāvali__lg__yadi_nāmaikam2}\flagstanza{\tiny\textenglish{...aikam2}}yadi nāmaikam adhyakṣam na pūrvāparavittimat |&adhyakṣadvayasadbhāve prākparāvedanaṃ katham || iti |\&[\smallbreak]


	

	  \pstart tathā smaraṇam abhilāṣaḥ, svayaṃnihitapratyanumārgaṇaṃ, dṛṣṭārthakutūhalaviramaṇaṃ, karmaphalasambandhaḥ, saṃśayapūrvakanirṇayaś ca pūrvapūrvārthānubhavair upādānakāraṇaiḥ samarpitasaṃskāragarbhair uttarottarārthānubhavair evopādeyabhūtair janyamāno yujyata iti kim adhikenātmānā parikalpitena | \edlabel{thakur75-127.13}\label{thakur75-127.13} upādānopādeyabhāvaniyamād eva ca na santānāntare smaraṇādiprasaṅgaḥ saṅgataḥ | kim idam upādānam iti cet | \edlabel{thakur75-127.14}\label{thakur75-127.14} ucyate | yatsantānanivṛttyā yad utpadyate tat tasyopādānakāraṇam | yathā mṛtsantānanivṛttyotpadyamānasya kumbhasya mṛd upādānam iti śāstre prapañcitam | na cātra paralokakṣatiḥ | \edlabel{thakur75-127.17}\label{thakur75-127.17} yad apy uktam | cittaśarīrayoḥ kiyatkālasthitinibandhanasya dṛṣṭasya nivṛttau cittasyāpi nivṛttiprasaṅgaḥ | maraṇavedanayā hi cittaṃ vikalam | tato 'vikalā cittāntarajananāvasthā na sambhavati | tasmād upasthite maraṇaduḥkhe sarvasaṃskāravirodhini cittam apy ucchidyeteti nāstikyam āyātam iti | \edlabel{thakur75-127.20}\label{thakur75-127.20} tad ayuktam | yato maraṇaduḥkhaṃ cittaviśeṣa eva, tasya cittāntarajananasāmarthyasvabhāvasya svabhāvād avāryaiva jñānotpattir iti | \edlabel{thakur75-127.22}\label{thakur75-127.22} bandhān mokṣo 'pi saṃsāricittaprabandhād anāśravacittaprabandho yaḥ | \edlabel{thakur75-127.23}\label{thakur75-127.23} śubhādimokṣayor api pravṛttir avāryā | yataḥ saty apy ātmany aham eva mukto bhaviṣyāmi sukhī cety ātmagrahalakṣaṇād adhyavasāyāt pravartate | na punar ātmanā galahastitaḥ | sa cānādyavidyāparamparāyātaḥ pūrvāparayor ekatvāropako mithyāsaṅkalpo bādhite 'py ātmany avyāhataprasara iti katham apravṛttiḥ | \edlabel{thakur75-127.26}\label{thakur75-127.26} nanu
	\pend
      
	    
	    \stanza[\smallbreak]
	nairātmyavādapakṣe [tu] pūrvam evāvabudhyate |&madvināśāt phalaṃ na syān matto 'nyasyāthavā bhaved ||\&[\smallbreak]


	

	  \pstart iti | apravṛttir evāstv iti cet | \edlabel{thakur75-127.29}\label{thakur75-127.29} astu ko doṣaḥ | yady ayam ātmagraho nirviṣayo 'pi pravṛttim anākṣipya kṣaṇam api sthātuṃ [na] prabhavati | yathā hi jātasyāvaśyaṃ mṛtyur iti jātavato 'py apratikriyaputrādimaraṇe sorastāḍam ākrando maraṇādau ca yatnaḥ śokodrekāt | evam avidyodrekād eva nairātmyaṃ jānann api pravartate | na sukham āsta iti kim atra kriyatām | avidyāyāḥ pravartanaśakter avāryatvāt | \edlabel{thakur75-128.3}\label{thakur75-128.3} pratyabhijñā ca pūrvam eva dhvastā | kāryakāraṇabhāvaniyatā paścādbhāvipūrvabhāvitā | sā ca kṣaṇike 'py aviruddhā | upādānopādeyatā ca kramisvasaṃvedanajñānadvayena sākṣātkṛta tatpṛṣṭhabhāvinā niścīyata iti, \edlabel{thakur75-128.6}\label{thakur75-128.6} asaty apy ātmani pratisandhātari kāryakāraṇagrahaṇādaya upapadyamānā nātmānam upasthāpayituṃ prabhavanti | ato 'rthāpattir api na kṣameti bhāgyahīnamanorājyam iva sthirasiddhir viśīryata eva | \edlabel{thakur75-121.8}\label{thakur75-121.8} tathā ca kṣaṇabhaṅgasandehe sattvādyanumānaṃ prāptāvasaram ||
	\pend
      

	  \pstart Sthirasiddhidūṣaṇaṃ samāptam || 
	\pend
      
	    
	    \endnumbering% ending numbering from div
	    \endgroup
	    
	  
	  
	% new div opening: depth here is 0
	
	    
	    \begingroup
	    \beginnumbering% beginning numbering from div depth=0
	    
	  
\chapter[{Citrādvaitaprakāśavādaḥ}]{Citrādvaitaprakāśavādaḥ}\label{Citrādvaitaprakāśavādaḥ}

	  \pstart || namas tārāyai || 
	\pend
      
	    
	    \stanza[\smallbreak]
	dig eṣā svaparāśeṣaprativādiprasādhanī |&citrādvaitamatābodhadhvāntastomakadarthinī ||\&[\smallbreak]


	

	  \pstart iha khalu sakalajaḍapadārtharāśau pratyākhyāte nirākṛte ca nirākāravijñānavāde pratihate cālīkākārayogini pāramārthikaprakāśamātre samyagunmūlite ca sākāravijñānālīkatvasamārope pratisantānaṃ ca svapnavad abādhitadehabhogapratiṣṭhādyākāraprakāśamātrātmake jagati vyavasthite yasya yadā yāvad ākāracakrapratibhāsaṃ yadvijñānaṃ parisphurati tasya tadā tāvad ākāracakraparikaritaṃ tadvijñānaṃ citrādvaitam iti sthitiḥ | tad evaṃ citram advaitaṃ vijñānam iti padatrayam iha pratyupasthitam || 
	\pend
      

	  \pstart atra ca vipratipattir nāma kiṃ citratāyām advaite vijñānatve sarvatraiveti vikalpāḥ || 
	\pend
      

	  \pstart na tāvad asau citrasvarūpānusāriṇī bhavitum arhati, tanmātrasya \edlabel{ratnakīrtinibandhāvali__36r1PF7IMSTQ7OTJHL7VDAR91JN}\label{ratnakīrtinibandhāvali__36r1PF7IMSTQ7OTJHL7VDAR91JN}\edtext{}{\lemma{sarvajanānu}\xxref{ratnakīrtinibandhāvali__36r1PF7IMSTQ7OTJHL7VDAR91JN}{ratnakīrtinibandhāvali__36r1PF7IMSRS80XSPP70EAGRVE0}\Afootnote{sarvajanānu \cite{} ; sarvajanāu \cite{} ; sarvajñānu}}sarvajanānu\edlabel{ratnakīrtinibandhāvali__36r1PF7IMSRS80XSPP70EAGRVE0}\label{ratnakīrtinibandhāvali__36r1PF7IMSRS80XSPP70EAGRVE0}bhavasiddhatvāt, anyathā śaśaviṣāṇādāv iva jaḍam idam alīkaṃ vijñānaṃ veti vipratipattīnām anavakāśaprasaṅgāt |
	\pend
      

	  \pstart nāpi vijñānatve vivādaḥ kartum ucitaḥ,
	\pend
      
	    
	    \stanza[\smallbreak]
	sahopalambhaniyamād\edtext{\textsuperscript{*}}{\lemma{*}\Bfootnote{Cf. PVin 1.54a.}}\&[\smallbreak]


	

	  \pstart ityādinā pūrvam eva nīlādīnāṃ sākāravijñānatvaprasādhanāt | ata eva sarvatrāpi vimatir asaṅgatā, sākāravijñānasiddhāv eva citrādvaitavādāvatārāt | tasmāc citrateyam advaitavirodhinīti vyāmohād ekatva eva [ {\corr vipratipatir}] iti tatra prasādhanaṃ sādhanam idam ucyate ||
	\pend
      

	  \pstart yat prakāśate tad ekam | yathā citrākāracakramadhyavartī nīlākāraḥ | prakāśate cedaṃ gauragāndhāramadhurasurabhisukumārasātetarādivicitrākārakadambakam iti svabhāvahetuḥ | \edlabel{thakur75-129.25}\label{thakur75-129.25} na tāvad asyāsiddhir abhidhātuṃ śakyate, pratyakṣapramāṇaprasiddhasadbhāve vijñānātmakanīlādyākāracakre dharmiṇi prakāśamānatāyāḥ pratyakṣasiddhatvāt | na cāsya \leavevmode\ledsidenote{\textenglish{pb in}}\label{thakur75-130} hetor viruddhatā sambhavati, vicitrākāramadhyavartini nīlākāre dṛṣṭāntadharmiṇi prakāśamānatālakṣaṇasya sādhanasya dṛṣṭatvāt | nanu caikatve sādhye yad aparam ekatvādhikaraṇaṃ tad iha dṛṣṭāntīkartum ucitam | na cāsya nīlākārasya ekatā vidyate, viruddhadharmādhyāsaprasiddhasyānekatvasya sambhavāt | deśakālākārabhedo hi viruddhadharmādhyāsaḥ | tataś ca yathā citratākāracakrasyākārabhedato bhedas tathā nīlākārasyāpi deśabhedato bhedaḥ | tad ayaṃ sādhyaśūnyo dṛṣṭānto hetuś ca vipakṣe paridṛśyamāno | yadi tatraiva niyatas tadā viruddhaḥ \edlabel{thakur75-130.8}\label{thakur75-130.8} tatrāpi sambhave 'naikānta iti cet ||
	\pend
      

	  \pstart atrocyate | yadi deśabhedato vijñānātmakasthūlanīlākārasya bhedas tadāsya pratiparamāṇudeśabhede bhedasambhavāt paramāṇupracayamātrātmako vijñānātmakasthūlanīlākāraḥ syāt | tathā ca sati sarveṣāṃ vijñānātmakanīlaparamāṇūnāṃ svasvarūpanimagnatvena saṃtamasanimagnānekapuruṣavad vyativedanābhāvāt sthūlanīlākhaṇḍalakapratibhāsābhāvaprasaṅgaḥ |
	\pend
      

	  \pstart na ca svasvarūpanimagnatvenāpy anyenānyasya vedanaṃ yujyate, yena sthūlapratibhāsaḥ saṅgataḥ syāt, grāhyagrāhakalakṣaṇayoḥ purastād apakartavyatvāt |
	\pend
      

	  \pstart na caivaṃ vaktavyam paramāṇūnāṃ [ {\corr sva}]svarūpanimagnatve 'py ekopādānatayā puñjātmaiva sthūlaḥ sthūlam ātmanaṃ jñāsyatīti, saty apy ekopādānatve svasvarūpanimagnatvād eva sthūlavyavasthāpakasya bhinnasyātmano 'nyonyam vā grāhyāgrāhakabhāvasyāyogāt | tādātmyena vyativedanasya cānabhyupagamāt |
	\pend
      
	    
	    \stanza[\smallbreak]
	vargo vargaṃ veti\&[\smallbreak]


	

	  \pstart ity asyānupadatvāt | na ca yathā bāhyārthavāde sthūlaikākārajñānapratibhāsa eva bāhyaparamāṇupracayapratibhāsavyavasthā gatyantarābhāvāt, tathā jñānaparamāṇuvyavasthā[ {\corr [nne]}]sthūlaikākārayogivijñānāntarasyānabhyupagamāt | abhyupagame vā tasyaiva dṛṣṭāntatvāt | tasmād yāvad yāvat pratibhāsas tāvat tāvat sthūlatayaiva vyāptaḥ | asthūle paramāṇau sthūlanivṛttimātre ca pratibhāsasya dṛśyānupalambhabādhitatvāt | yathā prasiddhānumāne sattvaṃ kṣaṇikatvena vyāptaṃ kramākramkāritvenāpi, kṣaṇikatvābhāvāc ca kramākramanivṛttau nivartamānaṃ kṣaṇikatve niyataṃ sidhyati, tathātrāpi prakāśamānatvaṃ sādhanam ekatvenāpi sthaulyenāpi, ekatvābhāvāc ca vipakṣāt paramāṇupuñjātmana ekatvanivṛttimātrātmanaś ca svaviruddhopalambhāt sthaulyasya vyāpakasya nivṛttau nivartamānam [ {\corr ekatve}] niyataṃ sidhyati | tataś ca yathā bahirvyāptipakṣe ghaṭe dṛṣṭāntadharmiṇi viparyayabādhakapramāṇabalāt sattvaṃ kṣaṇikatvaniyatam avadhārya\edlabel{ratnakīrtinibandhāvali__36r1PF7IMSPU9OV2Y47VRORIBMX}\label{ratnakīrtinibandhāvali__36r1PF7IMSPU9OV2Y47VRORIBMX}\edtext{}{\lemma{}\xxref{ratnakīrtinibandhāvali__36r1PF7IMSPU9OV2Y47VRORIBMX}{ratnakīrtinibandhāvali__36r1PF7IMSNVGCLLFHWVZ63NZNW}\Afootnote{\label{RNA-tc-0}dhārya \cite{CAPV,130.28} ; dhāryamāṇaṃ \cite{RNAms?citedRange=70a3} }}dhārya\edlabel{ratnakīrtinibandhāvali__36r1PF7IMSNVGCLLFHWVZ63NZNW}\label{ratnakīrtinibandhāvali__36r1PF7IMSNVGCLLFHWVZ63NZNW} sattvāt pakṣe kṣaṇikabhaṅgasiddhiḥ, tathātrāpi nīlākāre dṛṣṭāntadharmiṇi viparyayabādhakapramāṇabalād eva prakāśamānatvam ekatvaniyatam avagamya prakāśamānatvād vicitrākāracakrasādhyadharmiṇy ekatvasiddhir iti na dṛṣṭāntasya sādhyaśūnyatvam | nāpi hetor viruddhatā | na cānaikāntikatā || \edlabel{thakur75-130.33}\label{thakur75-130.33} nanv ekatve sādhye tatpracyutir dvitvaṃ ca vipakṣaḥ, tasmāc ca vipakṣād dhetuvyatirekapratipattyavasare kiṃ vipakṣātmā prakāśate na vā | pratibhāsapakṣe prakāśamānatvasya hetoḥ sādhāraṇānaikāntikatā, vipakṣe 'pi dṛṣṭatvāt | atha na prakāśate tadā sandigdhavyatirekitvam, kuto vyatireka ity avadher evāprakāśamānaśarīratvāt katham ataḥ sādhyasiddhipratyāśā | \edlabel{thakur75-131.4}\label{thakur75-131.4} atrocyate | iha dvividho vijñānānāṃ viṣayaḥ grāhyo 'dhyavaseyaś ca | pratibhāsamāno grāhyaḥ | agṛhīto 'pi pravṛttiviṣayo 'dhyavaseyaḥ | tatrāsarvajñe 'numātari sakalavipakṣapratibhāsābhāvān na grāhyatayā vipakṣo viṣayo vaktavyaḥ, sarvānumānocchedaprasaṅgāt, sarvatra sakalavipakṣapratibhāsābhāvāt tato vyatirekāsiddheḥ | pratibhāse ca deśakālasvabhāvāntaritasakalavipakṣasākṣātkāre sādhyātmāpi virākaḥ sutarāṃ pratīyata ity anumānavaiyarthyam | tasmād apratibhāse 'py adhyavasāyasiddhād eva vipakṣād dhūmāder vyatireko niścitaḥ | tat kim artham atra vipakṣapratibhāsaḥ prārthyate | yadi punar asyādhyavasāyo 'pi na syāt tadā vyatire\edlabel{capv-np-4a-start}\label{capv-np-4a-start}\edtext{}{\lemma{vyatire}\Bfootnote{This is where \cite{capv-np} starts. The verso of this folio is numbered as 4 in the left margin.}}ko na niścīyata iti yuktam, pratiniyataviṣayavyavahārābhāvāt ||
	\pend
      

	  \pstart nanv asminmate vastvavastvātmakasakalavipakṣapratipattisambhavāt tato hetuvyatirekaḥ saṃpratyetuṃ śakyata eva | na ca pratibhāsamātreṇa sattvaprasaṅgaḥ, arthakriyākāritvalakṣaṇatvāt sattvasya | tvanmate tu prakāśa eva vastutvam | ato vipakṣayor ekatvapracyu\edlabel{ratnakīrtinibandhāvali__36r1PF7IMSLXLPJA0LN21NP81M7}\label{ratnakīrtinibandhāvali__36r1PF7IMSLXLPJA0LN21NP81M7}\edtext{}{\lemma{}\xxref{ratnakīrtinibandhāvali__36r1PF7IMSLXLPJA0LN21NP81M7}{ratnakīrtinibandhāvali__36r1PF7IMSJYGHJZ8T01KRVE74E}\Afootnote{\label{RNA-tc-1}tidvi \cite{RNAms?citedRange=70b1} ; tir dvi \cite{CAPV?citedRange=131.16} }}tidvi\edlabel{ratnakīrtinibandhāvali__36r1PF7IMSJYGHJZ8T01KRVE74E}\label{ratnakīrtinibandhāvali__36r1PF7IMSJYGHJZ8T01KRVE74E}tvayoḥ pratibhāse prakāśamānatvasādhanasya vipakṣasādhāraṇatā | apratibhāse ca sandigdhavyatirekitvam iti codyaṃ duruddharam eveti cet | tad etad asaṅgatam | tathā hi dhūmādir avahnyāder vipakṣād vyāvṛtto vahnyādiniyataḥ sidhyati[ {\corr  | }] tasya ca vastvavastvātmakasakalavipakṣapadārtharāśeḥ svarūpanirbhāsa iti kiṃ nirvikalpajñāne kalpanāyāṃ vā | nirvikalpe cet | pratibhāsa iti ca ko 'rthaḥ | kiṃ nirākāre jñāne sakalavipakṣādisvarūpasya sākṣāt sphuraṇam, yadi vā tadarpitabuddhisvabhāvabhūtasadṛśākāraprakāśaḥ, atha samanantarapratyayabalāyātabuddhigatabāhyasadṛśākārapratibhāsaḥ, āhosvid buddher ātmabhūtavipakṣasadṛśālīkākāraparisphūrtiḥ | \edlabel{thakur75-131.24}\label{thakur75-131.24} na tāvad ādyaḥ pakṣo yuktaḥ, deśakā\edlabel{capv-np-4a-end}\label{capv-np-4a-end}\edlabel{capv-np-4b-start}\label{capv-np-4b-start}lasvabhāvaviprakṛṣṭānāṃ padārthānām arvācīne jane nirākāre ca jñāne sphuraṇāyogād ity asyārthasya śāstre eva vistareṇa prasādhānāt | sphuraṇe [ {\corr vā}]sādhyasyāpi prakāśanaprasaṅge 'numānavaiyarthyasya pratipādanāt | \edlabel{thakur75-131.27}\label{thakur75-131.27} nāpi dvitīyaḥ pakṣaḥ, deśādiviprakṛṣṭatvād eva sākṣātsvākārasamarpaṇasāmarthyābhāvāt | \edlabel{thakur75-131.29}\label{thakur75-131.29} na ca tṛtīyaḥ saṅgataḥ, sādṛśyasambhave 'pi samanantarabalād evāyātasya bāhyena saha pratyāsatter abhāvāt | \edlabel{thakur75-131.31}\label{thakur75-131.31} na caturtho 'pi prakāraḥ sambhavati, asatprakāśayor virodhāt, sphurato 'līkatvāyogāt | tathā hy asatprakāśa iti kim asadīśvarādeḥ khyātiḥ, bhāsamāno vā ākāro 'san, san vā na kaścit khyātīti vivakṣitam | tatra yasya padārthasya svarūpaparinirbhāsaḥ sa katham asann iti prāṇadhāribhir abhidhātavyaḥ | sphurataḥ keśoṇḍukākārasya bāhyarūpatayā bādhyatve 'pi jñānarūpatayārthatvasya ācāryeṇa pratipāditatvāt grāhakābhimatanirākāraprakāśasyāpy asattvābhidhānaprasaṅgāt || \edlabel{thakur75-132.3}\label{thakur75-132.3} pratibhāse 'pi bādhanād asatyatvam iti cet | kiṃ tad bādhakam, pratyakṣam anumānaṃ vā | yady ekatra svarūpasā\edlabel{capv-np-4b-end}\label{capv-np-4b-end}kṣātkāriṇi pratyakṣe 'viśvāsaḥ katham anyatra bādhake svarūpāntaraprakāśa eva nirvṛttis tatpūrvakam anumānaṃ ca sutarām aviśvāsabhājanam iti na bādhakavārtāpi | yad āhur guravaḥ
	\pend
      
	    
	    \stanza[\smallbreak]
	yasya svarūpanirbhāsas tad evāsāt kathaṃ bhavet |&bādhāto yadi sāpy ekā pratyakṣānumayor nanu ||&pratyakṣe yady aviśvāsa ekatrānyatra kā gatiḥ |&tatpūrvam anumānaṃ ca katham āśvāsagocaraḥ || iti | \edtext{}{\lemma{|}\Bfootnote{(JNA 391,1ff.)}}\&[\smallbreak]


	

	  \pstart nanu
	\pend
      
	    
	    \stanza[\smallbreak]
	dṛṣṭam eva dvicandrādipratibhāse 'pi bādhitam |&na dṛṣṭe 'nupapannatvaṃ tajjñātam api bādhyate ||\edtext{}{\lemma{||}\Bfootnote{(JNA 391,13f.)}}\&[\smallbreak]


	

	  \pstart iti cet | na | bādhyasyāpratibhāsanāt | pratibhāsinaś cābādhyatvāt | tathā hi
	\pend
      
	    
	    \stanza[\smallbreak]
	buddhyākārasya nirbhāso bādhā bāhyasya vastunaḥ |&sphūrtāv apy aviśvāse kva viśvāsa iti kīrtitam ||\edtext{}{\lemma{||}\Bfootnote{(JNA 391,16f.)}}\&[\smallbreak]


	

	  \pstart etena bhāsamāno vākāro 'sann iti dvitīyo 'pi pakṣaḥ pratikṣiptaḥ, pratibhāsād eva sattāsiddher bādhakāvakāśābhāvāt | 
	\pend
      

	  \pstart tathā san vā kaścin na khyātīti tṛtīyasaṅkalpo 'pi vyākulaḥ, prakāśavyāptatvāt sattāyāḥ | aprakāśasyāsattayā grastatvāt || 
	\pend
      

	  \pstart nanu prakāśo nāma vastunaḥ sattāsādhakaṃ pramāṇam | na ca pramāṇanivṛttāv arthābhāvaḥ | arthakriyāśaktis tu sattvam | tac cāprakāśasyāpi na virudhyata iti cet | satyam etat | bahirarthavāde 'prakāśasyāpi sāmarthyābhyupagamāt | keśoṇḍukādipratibhāse 'dhyavasitasyārthakriyāśaktiviyogād evābhāvasiddheḥ | sarvathā bahirabhāve tu jñānasya prakāśāvyabhicārāt tāvataiva sattve kim arthakriyayā | 
	\pend
      
	    
	    \stanza[\smallbreak]
	katham anyahṛdaḥ sattvaṃ prakāśād eva nāsya cet |&nārthakriyāpi sarvasmai kvacic ced bhāsanaṃ na kim ||\edtext{}{\edlabel{RNA-n-1}\lemma{||}\Bfootnote{(JNA 399,3f.)}}\&[\smallbreak]


	

	  \pstart iti | nirvikalpe tāvat svasaṃvedanasiddhasvākāram antareṇa vipakṣādayo na parisphuranti | athāmī vikalpe pratibhāsanta iti dvitīyaḥ saṅkalpo 'bhyupagamyate, asminn api pakṣe pratibhāsamāna ākāro 'sādhāraṇo 'śabdasaṃsṛṣṭatayā svasaṃvedanatādātmye praviṣṭatvād vastusann eva | \edlabel{thakur75-132.32}\label{thakur75-132.32} adhyavseyatā vipakṣādayo gṛhyanta iti cet | tadāpi teṣāṃ svarūpasya nirbhāso 'sti na vā | nirbhāse pratyakṣasiddhataiva, nāsatkhyātiḥ | śāstre 'pi
	\pend
      
	    
	    \stanza[\smallbreak]
	svarūpasākṣātkāritvam eva pratyakṣatvam\&[\smallbreak]


	

	  \pstart uktam | tasya cetarapratyakṣeṣv iva vikalpe 'pi svīkāre viruddhavyāptopalambhena vikalpabhrāntatvayor dūram apāstatvād vikalpe 'pi tvanmate pratyakṣatvam akṣatam | tat kathaṃ tatsiddhasya pratyakṣāntarānumānābhyāṃ bādhābhidhānam, tayor api svarūpāntaraprakāśapauruṣatvāt || \edlabel{thakur75-133.5}\label{thakur75-133.5} atha vikalpabhrāntatvayor vyāpakaviruddhayoḥ sambhavāt vikalpe pratyakṣatvam evāsambhavi | nanv asya pratyakṣatvam asambhavīti svarūpasākṣātkāritvam asambhavīty uktam | atha vipakṣādir artho 'smin prakāśata iti vācā svarūpasākṣātkāritvaṃ kathitam iti mātā me bandhyeti vṛttāntaḥ | iṣyate ca tvayā vipakṣādisvarūpasākṣātkāritvaṃ vikalpasyeti pratyakṣatānatikramaḥ, apratyakṣatve vastusvarūpasphuraṇāyogāt | tataś ca tatpratibhāsino 'pi rūpasya sata eva khyātir nāsatkhyātiḥ | na ca tad eva vikalpe parisphuradrūpam asatām īśvarādīnāṃ svarūpam, asattvasyaivābhāvaprasaṅgāt | svarūpasphuraṇe 'py asattve 'nyatrāpi prakāśiny anāśvāsāt | tato yat sākāravāde jalpitam
	\pend
      
	    
	    \stanza[\smallbreak]
	nityādayaḥ santa eva syuḥ\&[\smallbreak]


	

	  \pstart iti tadātmana eva patitam |\edtext{}{\lemma{|}\Bfootnote{tato yat --- patitam Ce'e JNA 392,15f. (has evāpatitam).}} yad āhur \name{guruvaḥ}
	\pend
      
	    
	    \stanza[\smallbreak]
	svarūpasākṣātkaraṇād adhyakṣatvaṃ na cāparam |&vikalpabhramabhūmitvam ata eva hi bādhitam ||\edtext{}{\lemma{||}\Bfootnote{Ce' JNA 391,5f. For ab also cf. JNA 563,5.}}&yadi nādhyakṣatā tasya rūpanirbhāsa eva na |&tatas tadasadīśādi pratibhātīty asaṅgatam ||&yadi tu pratibhāseta rūpam asya sad eva tat |&tad asat pratibhātīti tac ca bhāty asad eva vaḥ || \edtext{}{\lemma{||}\Bfootnote{(JNA 391,7ff.)}}\&[\smallbreak]


	

	  \pstart athādhyavasāye 'dhyavaseyasvarūpasya pratibhāso nāstīty ucyate | na tadā kasyacid adhyavasāyaḥ | katham ataḥ pratiniyatavastuvyavasthāsiddhiḥ | kiṃ ca ko 'yam adhyavasāyo nāma | kiṃ vyāvṛttibhedaparikalpitasya prakāśāṃśasya, svākārāṃśasya, alīkākārasya, bāhyavastuno 'vastuno vā sphuraṇam adhyavasāyārthaḥ | yadi vā svākāre bāhyāropaḥ, bāhye vā svākārāropaḥ, svākārabāhyayor yojanā, tayor ekīkaraṇam ekapratipattir abhedena pratipattiḥ, bhedāgraho 'dhyavasāyārtha iti vikalpāḥ | \edlabel{thakur75-133.28}\label{thakur75-133.28} tatra na tāvad ādimau pakṣau kalpanām arthaḥ | svarūpe sarvasyaiva sphuraṇasya nirvikalpatvād avasāyānupapattiḥ | itarathā nirvikalpakajñānābhāvaprasaṅgāt | \edlabel{thakur75-133.30}\label{thakur75-133.30} alīkasphuraṇaṃ tu prāk pratyākhyātam | saty api sphuraṇe 'sphuṭatvān nirvikalpakam etat | dvicandrādijñānavat | astu svagrāhye tannirvikalpakam, bāhye tu adhyavaseye adhyavasāya iti cet | na | tatsambandhābhāvāta, tadapratibhāsāc ca | anyathātiprasaṅgād ity uktaprāyam | bāhyavastusvarūpasphuraṇe tu pratyakṣapratipattir evāsāv iti ko 'dhyavasāyaḥ | avastusphuraṇaṃ punas tridhā vikalpya prāg eva pratyākhyātam | \edlabel{thakur75-134.4}\label{thakur75-134.4} svākāre tu bāhyāropo na sambhavaty eva | tathā hi jñānaṃ kenacid ākāreṇa satyenālīkena vopajātaṃ nāma | bāhyāropas tu tadākāre tatkṛto 'nyakṛto vā syāt | tatkṛtatve na tāvat tatkāla eva vyāpārāntaram anubhūyata iti kutas tadāropaḥ | kālāntare ca svayam evāsat kasya vyāpāraḥ syāt |
	\pend
      

	  \pstart dvitīyapakṣe jñānāntaram api nākārāroparāgasaṅginīm utpattim antareṇa vyāpārāntareṇa kvacit kiñcitkaraṃ nāma | tad etad arvācīnajñānasadṛśākāragocarīkaraṇe 'pi na bāhyāropavyāpāram aparaṃ spṛśati tadākāraleśānukāram apahāya | na ca śabdāmukhīkaraṇam atirikto vyāpāraḥ, śabdākārasyāpi svarūpa evāntarbhāvād iti nākārād anyo jñānavyāpāraḥ | āropyamāṇaś cāsāv artho bāhyaḥ | tatra buddhau yadi svarūpeṇa sphurati satyapratītir evāsau, ka āropaḥ | atha na parisphurati tathāpi ka āropaḥ | sphuraṇe vādhikaraṇabhūtasvākārātiriktasyāropyamāṇākārasyāpi pratibhāsaprasaṅgaḥ |\edtext{}{\lemma{|}\Bfootnote{Parallel to 388.7--18}}
	\pend
      

	  \pstart tadākārasphuraṇam eva tasya sphuraṇam iti cet | na | tasyāropaviṣayatvāt | na hi marīcisphuraṇam eva jalasphuraṇam iti na svākāre bāhyāropaḥ |
	\pend
      

	  \pstart ata eva bāhye svākārāropo nāsti, āropaviṣayasya bāhyasyāsphuraṇāt |
	\pend
      

	  \pstart tata eva svākārabāhyayor yojanāpy asambhavinī, yogyayor apratibhāsāt |
	\pend
      

	  \pstart na caikīkaraṇam adhyavasāyaḥ | ko 'yam ekīkaraṇārthaḥ | yady ekatāpatau prayojakatvaṃ tadāropyāropaviṣayayoḥ kadācid ekībhāvābhāvād asambhava eva | na hi śaśaviṣāṇe kāraṇaṃ kiñcit | na ca pūrvam anekam ekatām etīti kṣaṇikavādinaḥ sāṃpratam | arthāntarotpattimātraṃ tu syāt | na ca tadupalabdhigocaro 'nyatrāropaviṣayāt svākārāt | na ca tāvatāpy arthasya kiñcid iti katham ekīkaraṇam |
	\pend
      

	  \pstart athaikapratītir adhyavasāyaḥ | tathāpi na dvayor ekapratipattir adhyavaseyānubhavābhāvāt | na ca dvayoḥ pratītir ity evādhyavasāyaḥ nīlapītavat | \edlabel{thakur75-134.27}\label{thakur75-134.27}
	\pend
      

	  \pstart na cābhedena pratītir adhyavasāyaḥ | yataḥ paryudāsapakṣe aikyapratītir uktā bhavati | sā ca prayuktā, adhyavaseyapratyabhāvāt | bhedena pratītiniṣedhamātre 'pi na bāhyasya pratītir ukteti kutas tadadhyavasāyaḥ | yadi hi bāhyaṃ prakāśeta ekatvenānekatvena vā satā asatā vā pratītir iti yuktam | 
	\pend
      

	  \pstart sarvākāratatsvarūpatiraskāreṇa sā pratītir ity ekapratītir iti cet | tatsvarūpatiraskāre tarhi tadapratibhāsanam eva | kasyacid aṃśasya pratibhāsanād iti cet | na | niraṃśatvād vastunaḥ sarvātmanā pratibhāso 'pratibhāso veti śāstram evātra vistareṇa parīkṣyate | \edlabel{thakur75-135.1}\label{thakur75-135.1} na ca bhedāgraho 'dhyavasāyo vaktavyaḥ | tathā hi kiṃ bāhye gṛhyamāṇe 'grahyamāṇe vā | na ca prathamaḥ pakṣaḥ, bāhyagrahaṇasya pratikṣiptatvāt | grahaṇe vādhyavasāyasya pratyakṣatāprasaṅgāt | agṛhyamāṇe tu bāhye pravṛttiniyamo na syāt, anyeṣām api tadānīm agrahaṇād anyatrāpi pravṛttiprasaṅgāt |
	\pend
      

	  \pstart \persName{trilocano} 'pīttham adhyavasāyaṃ dūṣayati | ko 'yam adhyavasāyaḥ | kiṃ grahaṇam, ahosvit karaṇam, uta yojanā, atha samāropaḥ | tatra svābhāsam anartham arthaṃ kathaṃ gṛhṇīyāt, kuryād vā vikalpaḥ | na hi nīlaṃ pītaṃ śakyaṃ grahītuṃ kartuṃ vā śilpakuśalenāpi | nāpy agṛhītena svalakṣaṇena svākāraṃ yojayitum arhati vikalpaḥ | na ca svalakṣaṇaṃ vikalpagrahaṇagocaraḥ | na ca svākāram anartham artham āropayati | na tāvad agṛhītasvākāraḥ śakya āropayitum iti tadgrahaṇam eṣitavayam | tatra kiṃ gṛhītvā āropayati, atha yadaiva svākāraṃ gṛhṇati tadaivāropayati | nādyaḥ | na hi kṣaṇikaṃ vikalpavijñānaṃ kramavantau grahaṇasamāropau kartum arhati | uttarasmiṃs tu kalpe 'vikalpasvasaṃvedanapratyakṣād vikalpākārād ahaṅkārāspadād \edlabel{ratnakīrtinibandhāvali__36r1PF7IMSHYRSCDBXH9LGA21E4}\label{ratnakīrtinibandhāvali__36r1PF7IMSHYRSCDBXH9LGA21E4}\edtext{}{\lemma{}\xxref{ratnakīrtinibandhāvali__36r1PF7IMSHYRSCDBXH9LGA21E4}{ratnakīrtinibandhāvali__36r1PF7IMSFWSK9HRW3N8HTPE53}\Afootnote{\label{RNA-tc-2}anahaṅkārāspadaṃ \cite{NVTṬ?citedRange=441.19} ; anahaṅkārāspadaḥ \cite{CAPV?citedRange=130.28,RNAms?citedRange=72b3} }}anahaṅkārāspadaṃ\edlabel{ratnakīrtinibandhāvali__36r1PF7IMSFWSK9HRW3N8HTPE53}\label{ratnakīrtinibandhāvali__36r1PF7IMSFWSK9HRW3N8HTPE53} samāropyamāṇo vikalpena svagocaro na śakyo 'bhinnaḥ pratipattum | nāpi bāhya\edlabel{ratnakīrtinibandhāvali__36r1PF7IMSDWIZ4SK58R7BCRQH3}\label{ratnakīrtinibandhāvali__36r1PF7IMSDWIZ4SK58R7BCRQH3}\edtext{}{\lemma{}\xxref{ratnakīrtinibandhāvali__36r1PF7IMSDWIZ4SK58R7BCRQH3}{ratnakīrtinibandhāvali__36r1PF7IMSBV2F426HMP90OT53H}\Afootnote{\label{RNA-tc-3}svalakṣaṇaikatvena \cite{RNAms?citedRange=72b3} ; svalakṣaṇakatvena \cite{CAPV?citedRange=135.14} }}svalakṣaṇaikatvena\edlabel{ratnakīrtinibandhāvali__36r1PF7IMSBV2F426HMP90OT53H}\label{ratnakīrtinibandhāvali__36r1PF7IMSBV2F426HMP90OT53H} śakyaḥ pratipattum, vikalpākāre svalakṣaṇasya bāhyasyāpratibhāsanād iti |
	\pend
      

	  \pstart \persName{vācaspatir} apy adhyavasāyaṃ pratikṣipati | \edlabel{quote-nk-adhyavasāya-start}\label{quote-nk-adhyavasāya-start} anarthaṃ svābhāsam artham adhyavasyatīti nirvacanīyam etat | nanv ayam āropayatīti kiṃ vikalpasya svarūpānubhava evāropaḥ, uta vyāpārāntaraṃ svarūpānubhavāt | na tāvat pūrvaḥ kalpaḥ, anubhavasamāropayor vikalpāvikalparūpatayā dravakaṭhinavat tādātmyānupapatteḥ | vyāpārāntaratve tu kramaḥ samānakālatā vā | na tāvat kramaḥ, kṣaṇikasya vijñānasya kramavadvyāpārāyogāt | akṣaṇikavādinām api buddhikarmaṇor viramya vyāpārānupapatteḥ na kramavadvyāpārasambhavaḥ | anubhavasamāropau samānakālāv iti cet | bhavatu samānakālatvaṃ kevalam | ātmā svabhāvasthita eva vedyaḥ, parabhāvena vedane svarūpavedanānupapatteḥ | tathā cātmā jñānasya grāhyagrāhakākāro 'nubhūto 'rthaś ca samāropitaḥ | na tv ātmā vedyamānaḥ samāropito nārthaḥ samāropyamāṇaḥ pratyakṣavedyaḥ | sa ca samāropaḥ sato 'sato vā grahaṇam eva | na ca jñānātiriktasya grahaṇaṃ sambhavatīty upapāditam |\edlabel{thakur75-135.27}\label{thakur75-135.27} svapratibhāsasya bāhyād bhedāgraho bāhyasamāropas tato bāhye vṛttir iti cet | sa kiṃ gṛhyamāṇe bāhye na vā | na tāvad gṛhyamāṇe | uktaṃ hy etan na \edlabel{ratnakīrtinibandhāvali__36r1PF7IMS9V2FE679H8QDDC4HL}\label{ratnakīrtinibandhāvali__36r1PF7IMS9V2FE679H8QDDC4HL}\edtext{}{\lemma{}\xxref{ratnakīrtinibandhāvali__36r1PF7IMS9V2FE679H8QDDC4HL}{ratnakīrtinibandhāvali__36r1PF7IMS7TSIJTA6L3F390OH3}\Afootnote{tadagrahaṇaṃ \cite{CAPV} ; tadgrahaḥ \cite{NK1} }}tadagrahaṇaṃ\edlabel{ratnakīrtinibandhāvali__36r1PF7IMS7TSIJTA6L3F390OH3}\label{ratnakīrtinibandhāvali__36r1PF7IMS7TSIJTA6L3F390OH3} sambhavatīti | agṛhyamāṇe tu bhedāgrahe na pravṛttiniyamaḥ syāt, anyeṣām api tadānīm agrahād anyatrāpi pravṛttiprasaṅgād iti \edlabel{quote-nk-adhyavasāya-end}\label{quote-nk-adhyavasāya-end}|\edtext{}{\lemma{|}\Bfootnote{\href{quote-nk-adhyavasāya-start}{quote-nk-adhyavasāya-start} to \href{quote-nk-adhyavasāya-end}{quote-nk-adhyavasāya-end} is a quote from \cite{NK1}.}} tasmād yathā yathāyam adhyavasāyaś cintyate tathā tathā viśīryata eva | tathā vikalpāropābhimānagrahaniścayādayo 'py adhyavasāyavat svākāraparyavasitā eva sphuranto bāhyasya vārtāmātram api na jānantīty adhyavasāyasvabhāvā eva \edlabel{ratnakīrtinibandhāvali__36r1PF7IMS5TO6HY4RVSXOVCLO1}\label{ratnakīrtinibandhāvali__36r1PF7IMS5TO6HY4RVSXOVCLO1}\edtext{}{\lemma{}\xxref{ratnakīrtinibandhāvali__36r1PF7IMS5TO6HY4RVSXOVCLO1}{ratnakīrtinibandhāvali__36r1PF7IMS3RQL7PXXLRXDJR361}\Afootnote{śabdapravṛttinimittabhede \cite{RNAms?citedRange=73a1} ; śabdapravṛttimittabhede \cite{CAPV?citedRange=136.1}   {\rmlatinfont [App type: misprint]}}}śabdapravṛttinimittabhede\edlabel{ratnakīrtinibandhāvali__36r1PF7IMS3RQL7PXXLRXDJR361}\label{ratnakīrtinibandhāvali__36r1PF7IMS3RQL7PXXLRXDJR361} 'pi, tat kathaṃ yuktyāgamābahirbhūto\edtext{}{\lemma{yuktyāgamābahirbhūto}\Bfootnote{\cite{thakur75} suggests an emendation to yuktyāgamabahirbhūto, but that seems unnecessary.}} 'nātmāsphuraṇam ācakṣīta | \edlabel{thakur75-136.3}\label{thakur75-136.3} nanv evaṃ vikalpādīnām asambhave sambhave 'py anātmaprakāśakatvānabhyupagame sarvajanaprasiddhavidhipratiṣedhavyavahārocchedaprasaṅga iti lokavirodhaḥ | vikalpa ity adhyavasāya ity āropa ity abhimāna iti graha iti niścaya ityādikaṃ śāstre pratipadaṃ pratipāditam, tatsiddhaṃ ca bahirarthādikam abhyupagatam ity ācāryavirodhaḥ, nyāyavirodhaś ca | tathā hi sarvair eva prakārair\edtext{}{\lemma{prakārair}\Bfootnote{``prakāśair" acc. to PPU and SāSiŚā.}} aviparītasvarūpasaṃvedanād bhrānter atyantam abhāvaḥ syāt | tataś ca sarvasattvāḥ sadaiva samyaksambuddhā bhaveyuḥ | \edlabel{thakur75-136.9}\label{thakur75-136.9} vikalpikā buddhir brāntiḥ, svapratibhāse 'narthe 'rthādhyavasāyād iti cet | katham avasīyamānas tayā so 'rtho na prakāśate | prakāśamāno vā katham asau tasyāṃ na prakāśate | atha prakāśata eva, tadārthasya tādātmyaprasaṅgaḥ | asati cārthe sārasyāt abhūn māndhātā, bhaviṣyati śaṅkho 'styātmā, nityaḥ śabda iti sarvātmanā ca niścayaḥ syāt | gaur iti spaṣṭena ca svena lakṣaṇena prakāśeta | svalakṣaṇe ca saṅketāyogāt vikalpikaiva sā buddhir na syāt | tasmād aśeṣagovyaktisādhāraṇena gotvena gobuddhir alīkena sābhilāpena viplavāt prakhyātīti tathā prakāśanam asyā gavārthāvasāya ity eṣṭavyam | evaṃ hy ete doṣā na syuḥ, apratibhāsamānasyāpi svalakṣaṇasya bhrāntyāvasāyād iti || \edlabel{thakur75-136.18}\label{thakur75-136.18} atrābhidhīyate | na tāval lokaśāstraviro\edlabel{capv-np-10a-start}\label{capv-np-10a-start}dhau, agṛhīte 'pi bāhye pravṛttinivṛttyādisamarthanāt svaparavādiduratikramādhyavasāyasvarūpanirvacanāt | nyāyavirodhasya tu gandho 'pi nāsti | \edlabel{thakur75-136.21}\label{thakur75-136.21} tathā hi kā punar ayaṃ bhrāntir asatkhyātiratasmiṃs tadgraho vā yadabhāvādidānīm eva muktir āsajyate | \edlabel{thakur75-136.23}\label{thakur75-136.23} na tāvad ādyaḥ pakṣaḥ, asatkhyāteḥ pratyākhyānāt | yad āhur guruvaḥ
	\pend
      
	    
	    \stanza[\smallbreak]
	yasya svarūpanirbhāso bādhakād yadi tan na sat |&bādhake 'pi ka āśvāsaḥ svarūpāntarabhāsini ||&anyasvarūpopanayāt tatsvarūpanivāraṇam |&tatrāpi saṃśayo jātaḥ pūrvabādhopalabdhitaḥ ||&iyam evāgrahe bādhā nādyajasyāparā yadi |&asya pūrvaiva bhavatu rūpanirbhāsanaṃ samam || &nyāyā ca bhāvinīty atra pramāṇaṃ kiñcid asti vaḥ |&api svarūpanirbhāse yadā bādhakasambhavaḥ ||&anirbhāse svarūpasya hetuśodhanaviplave |&bādhaśaṅkāvinirbhāse 'py evaṃ ced viplavo mahān || iti ||\edtext{}{\edlabel{RNA-n-2}\lemma{||}\Bfootnote{JNA 392,19-393,3}}\&[\smallbreak]


	

	  \pstart śāstre ca atasmiṃs tadgrahāt svapratibhāse 'narthe 'rthādhyavasāyād dṛśyavikalpyayor ekīkaraṇād bhrāntir uktā | tām ayaṃ samarthayitum asamarthaḥ svātantryeṇālīkasphuraṇaṃ bhrāntir iti kāvyaṃ viracayya vistārayati || 
	\pend
      

	  \pstart nanv atasmiṃs tadgraho 'pi bhramaḥ svā\edlabel{capv-np-10a-end}\label{capv-np-10a-end}\edlabel{capv-np-10b-start}\label{capv-np-10b-start}kāraparyavasitajñānād atirikto bahubhir bahudhā vicārya pratyākhyātaḥ | tat kathaṃ tasminn api pakṣe na bhrāntikṣatir yenedānīm eva muktiprasaṅgo na syād iti cet | tad etad bhagavato \persName{bhāṣyakārasya} matavidveṣaviṣavyākulavikrośitam atikātarayati kṛpāparavaśadhiyaḥ | tathā hi samanantarapratyayabalāyātasvapratibhāsaviśeṣavedanamātrād agṛhīte 'pi paratra pravṛttyākṣepo 'dhyavasāyaḥ | na cāsau pūrvoktavāgjālaiḥ pratihantuṃ śakyaḥ, sarvaprāṇabhṛtāṃ pratyātmaviditatvāt, kaiścid apy anudbhinnatvāt | ayam eva ca saṃsāras tatkṣayo mokṣa iti kvedānīm eva tadvārtāpi | tathā hi vicitrānādivāsanāvaśāt prabodhakapratyayaviśeṣāpekṣayā vikalpaḥ kenacid ākāreṇopajāyamāna eva bahir mukhapravṛttyanukūlam arthakriyāsmaraṇābhilāṣādiprabandham ādhatte | tataḥ puruṣārthakriyārthino bahirarthānurūpāṇi pravṛttinivṛttyavadhāraṇāni bhavanti | pṛthagjanasantānajñāna\edlabel{ratnakīrtinibandhāvali__36r1PF7IMS1N1FWQDPUYFL9R0TC}\label{ratnakīrtinibandhāvali__36r1PF7IMS1N1FWQDPUYFL9R0TC}\edtext{}{\lemma{}\xxref{ratnakīrtinibandhāvali__36r1PF7IMS1N1FWQDPUYFL9R0TC}{ratnakīrtinibandhāvali__36r1PF7IMRZL0VOIHDHNC5NEYNP}\Afootnote{kṣaṇānāṃ \cite{} ; lakṣaṇānāṃ \cite{} }}kṣaṇānāṃ\edlabel{ratnakīrtinibandhāvali__36r1PF7IMRZL0VOIHDHNC5NEYNP}\label{ratnakīrtinibandhāvali__36r1PF7IMRZL0VOIHDHNC5NEYNP} tādṛśo hetuphalabhāvasya niyatatvāt | aniścitārthasambandhavikalpakāle 'pi sada\edlabel{capv-np-10b-end}\label{capv-np-10b-end}sattānirṇayādipravṛttiprasavaḥ | tatra yadubhayathā pravṛttisādhanasāmarthyam asya svahetubalāyātam ayam eva pravṛttiviṣayatvāropo 'dhyavasāyāparanāmā | yathā candrādijñānasya bhrāntasyābhrāntasya vā taddarśanāvasāyajananam eva grahaṇavyāpāraḥ |
	\pend
      
	    
	    \stanza[\smallbreak]
	svavid apīyam arthavid eva kāryato draṣṭavyeti \edtext{\textsuperscript{*}}{\lemma{*}\Bfootnote{Cf. \textbackslash cite[349]\{pv3.320toend\}: yathā niviśate so 'rtho yataḥ sā prathate tathā | arthasthites tadātmatvāt svavid apy arthavin matā || }}\&[\smallbreak]


	

	  \pstart nyāyāt | tathā vikalpasyāpy agnir atretyādinākāreṇotpadyamānasya pravṛttyākṣepakatvam eva bāhyāvasānaṃ nāma | yathā ca nirvikalpadvicandrādyākārataiva tathāvasāyasādhanī, evam avasāyasyāpi tādṛśākārataiva viṣayāntaravimukhapravṛttisādhanī || \edlabel{thakur75-137.25}\label{thakur75-137.25} nanu tathā ca tac ca tena pratipādyate na ca tajjñāne tat prakāśata iti śapathenāpi na saṃpratyaya iti cet | asambaddham etat | na hy adhyavasāyād bāhyasya paṭāder vastuno bādhakāvatārāt pūrvasandigdhavastubhāvasya kṣaṇikāder avastuno vā śaśaviṣāṇāder asphuraṇe 'pi siddhipratibandho brahmaṇāpi pratividhātuṃ śakyaḥ | dvividho hi viṣayavyavahāraḥ, pratibhāsād adhyavasāyāc ca | tad iha pratibhāsābhāve 'pi parāpoḍhasvalakṣaṇāder adhyavasāyamātreṇa viṣayatvam uktam, sarvathā nirviṣayatve pravṛttinivṛttyādisakalavyavahārocchedaprasaṅgāt | tataś ca tena ca tat pratipādyate na ca jñāne tatprakāśa iti saṅgatir asty eva, prakāśyaprakāśakabhāvābhāve 'py \edlabel{ratnakīrtinibandhāvali__36r1PF7IMRXJYE54SUH8HG2DO14}\label{ratnakīrtinibandhāvali__36r1PF7IMRXJYE54SUH8HG2DO14}\edtext{}{\lemma{adhyavaseya}\xxref{ratnakīrtinibandhāvali__36r1PF7IMRXJYE54SUH8HG2DO14}{ratnakīrtinibandhāvali__36r1PF7IMRVGV16KLF4NAL7T8DL}\Afootnote{\label{RNA-tc-4} \cite{RNAms?citedRange=74a2} adhyavasāyā \cite{CAPV?citedRange=137.32} }}adhya\edlabel{ratnakīrtinibandhāvali__36r1PF7IMRVGV16KLF4NAL7T8DL}\label{ratnakīrtinibandhāvali__36r1PF7IMRVGV16KLF4NAL7T8DL}dhyavasāyakabhāvenāpi viṣayaviṣayibhāvopapatteḥ | \edlabel{thakur75-138.1}\label{thakur75-138.1} nanu yadi nādhyavaseyapratītis tadāgṛhīte 'pi svalakṣaṇādau pravṛttir iti sarvatrāviśeṣeṇa prasajyeta, sarvatrāgṛhītatvena viśeṣābhāvāt, tataś ca prāptir api nābhimatasya niyamenety anumānam api viplutam | atra brūmaḥ | yady adhyavaseyam agṛhītaṃ viśvam apy agṛhītam, tathāpi niyataviṣayaiva pravṛttir na sarvatra, tathābhūtasamanantarapratyayabalāyātaniyatākāratayā niyataśaktitvād vikalapasya | niyataśaktayo bhāvā hi pramāṇapariniṣṭhitasvabhāvāḥ, na śaktisāṅkaryaparyanuyogabhājaḥ, asadutpattivat | sarvatrāsattve 'pi hi bījād aṅkurasyaivotpattiḥ, tatraiva tasya śakteḥ pramāṇena nirūpaṇāt | tathehāpi hutavahākārasya vikalpasya dāhapākādyarthakriyārthinas tatsmaraṇavato hutavahaviṣayāyām eva pravṛttau sāmarthyaṃ pramāṇapratītaṃ katham atiprasaṅgabhāgi | pratyāsatticintāyāṃ ca tāttvikasyāpi vahner jvaladbhāsvarākā\edlabel{capv-np-13a-start}\label{capv-np-13a-start}ratvaṃ vikalpollikhitasyāpīti, tāvatā tatraiva pravartanaśaktir jvalanavikalpasya na jalādau ||
	\pend
      

	  \pstart nanu ca sādṛśyāropeṇa kiṃ svākārasya bāhye svākāre vā bāhyasyāropaḥ | ubhayathāpy asaṅgatiḥ, āropyāropaviṣayayoḥ svākārabāhyayor dvayor grahaṇāsambhavād iti cet | na vayam āropeṇa pravṛttiṃ brūmaḥ | kiṃ tarhi, svavāsanāparipākavaśād upajāyamānaiva sā buddhir apaśyanty api bāhyaṃ bāhye pravṛttimātanotīti viplutaiva saṃsārātmikā ca | yat śāstraṃ 
	\pend
      
	    
	    \stanza[\smallbreak]
	na jñāne tulyam utpattito dhiyaḥ |&tathāvidhāyāḥ \&[\smallbreak]


	

	  \pstart iti | tasmān na rūpyādivad āropadvāreṇa pravṛttir api tu tathāvidhākārotpattipratibaddhaśaktiniyamāt | na ca vicārakasya vastvadarśananiścayād apravṛttiḥ saṅgacchate | darśane 'pi hi pravṛttir arthakriyārthitayā | arthakriyāprāptiś ca vastusattāniyame | sa ca niyamo yathā darśanād vastupratibandhakṛtaḥ, tathā vikalpaviśeṣād api pāramparyeṇa vastuprativastupratibandhakṛta ity adarśane 'pi adhyavasāyāt pravṛttir yujyata iti nānumānam anavasthitam | etena tac ca na pratīyate, tena cābhedābhāsanam ity upālambho 'sambhavī\edlabel{ratnakīrtinibandhāvali__36r1PF7IMRTEOQOZSS4FDJI58TN}\label{ratnakīrtinibandhāvali__36r1PF7IMRTEOQOZSS4FDJI58TN}\edtext{}{\lemma{ty upadarśitam}\xxref{ratnakīrtinibandhāvali__36r1PF7IMRTEOQOZSS4FDJI58TN}{ratnakīrtinibandhāvali__36r1PF7IMRRBQQ43WJS6VRW8BE9}\Afootnote{ti darśitaḥ \cite{capv-np} }}ty upadarśitam\edlabel{ratnakīrtinibandhāvali__36r1PF7IMRRBQQ43WJS6VRW8BE9}\label{ratnakīrtinibandhāvali__36r1PF7IMRRBQQ43WJS6VRW8BE9}\edlabel{capv-np-13a-end}\label{capv-np-13a-end}\edlabel{capv-np-13b-start}\label{capv-np-13b-start}, apratibhāse 'pi pravṛttiviṣayīkaraṇam ity abhedādiniṣṭhāyā darśitatvāt | tasmād avicāraramaṇīyo 'tasmiṃs tatgraha eva bhrāntir āropāparanāmā, tatkṣayaś ca mokṣa iti yuktam |
	\pend
      

	  \pstart yad āhur guruvaḥ
	\pend
      
	    
	    \stanza[\smallbreak]
	tasmāt pravṛtter ākṣepe vikalpākārajanmani |&mato jalādyāropo 'pi satyāsatyasamaś ca saḥ ||&tato yady api tattvena nāropo nāma kasyacit |&vyavahārakṛtas tv eṣa pratiṣeddhuṃ na śakyate ||&marīcau jalavad yāvad anātmany ātmakalpanam |&bhrama eva hi saṃsāro nirvāṇaṃ tattvasaṃsthitiḥ ||&tataś ca yāvan na vicārasambhavo bhavo 'yam anyaḥ śama ity ayaṃ nayaḥ |&vicāralīlālalite tu mānase bhavaḥ śamo vā ka iheti kathyatām ||\edtext{}{\lemma{||}\Bfootnote{JNA 554,17-25}}\&[\smallbreak]


	

	  \pstart tathā Āryamaitreyanāthapādā api 
	\pend
      
	    
	    \stanza[\smallbreak]
	na cāntaraṃ kiṃcana vidyate 'nayoḥ sadarthavṛttyā śamajanmanor iha |&tathāpi janmakṣayato vidhīyate śamasya lābhaḥ śubhakarmakāriṇām ||\&[\smallbreak]


	

	  \pstart Āryanāgārjunapādāś ca 
	\pend
      
	    
	    \stanza[\smallbreak]
	nirvāṇaṃ ca bhavaś caiva dvayam eva na vidyate |&parijñānaṃ bhavasyaiva nirvāṇam iti kathyate ||\&[\smallbreak]


	

	  \pstart iti sarvair eva prakāśair aviparītasvarūpasaṃvedane 'pi bhrāntivyavasthāsambhavād asti saṃsāraḥ || 
	\pend
      

	  \pstart yad apy uktaṃ vikalpasyāviṣayaś ca bāhyam grahaṇaṃ cāsya śabdena saṃyojyeti vikalpatvam api duryojam, ā\edlabel{capv-np-13b-end}\label{capv-np-13b-end}\edtext{}{\lemma{ā}\Bfootnote{This is where folio 13 of \cite{capv-np} ends.}}tmani ca śabdayojanā nāstīti vikalpo nāma nāsty eva, tat kasya vikalpacinteti | atrābhidhīyate | ihāgnir atrety adhyavasāyo yathā kāyikīṃ vṛttiṃ prasūte tathāgnir mayā pratīyata iti vācikīm api prasūte, etadākārānuvyavasāyarūpāṃ mānasīm api prasavati | evaṃ ca sati yathā vikalpenāyam artho gṛhīta iti niścayaḥ, tathā śabdena saṃyojya gṛhīta ity api, arthākāraleśavac chabdākārasyāpi sphuraṇāt | tasmād arthagrahābhimānavān mānavastāvad abhidhānasaṃyuktagrahaṇābhimānavān apīty avasāyānurodhād eva vikalpavyavasthā na tattvataḥ | yad āhur guravaḥ
	\pend
      
	    
	    \stanza[\smallbreak]
	na śabdaiḥ saṃsargaḥ kvacid api bahir vā manasi vākṣarākārākīrṇaḥ sphurati punar arthākṛtilavaḥ | &ubhāv apy ākārau yad api dhiya evādhyavasitir vidhatte tau bāhye vacasi ca vikalpasthitir ataḥ ||\edtext{}{\edlabel{RNA-n-3}\lemma{||}\Bfootnote{JNA 227,6ff.}}&abhāne pratibhāne vā na cāropo 'pi kasyācit |&pratītyotpādabhedena vyavasthāmātram\label{ratnakīrtinibandhāvali__36r1PF7IMRP9MIIAM76HTLHOH8T}\edtext{}{\lemma{īdṛśaḥ}\xxref{ratnakīrtinibandhāvali__36r1PF7IMRP9MIIAM76HTLHOH8T}{ratnakīrtinibandhāvali__36r1PF7IMRN5V3LF550RHCON51E}\Afootnote{\label{RNA-tc-5} \cite{RNAms?citedRange=74b6} īdṛśam \cite{CAPV?citedRange=139.29} }}īdṛśaḥ\label{ratnakīrtinibandhāvali__36r1PF7IMRN5V3LF550RHCON51E} ||&nirvikalpād vikalpasya bhāve leśānukāriṇaḥ |&saṅketakārivacanād buddhyākāre viśeṣiṇi ||&saṅketaḥ kṛta ityāsthā tādṛk śabdaśrutau punaḥ |&pravṛttyākṣepabuddhyātmabhāve vācyavyavasthitiḥ || iti | \edtext{}{\edlabel{RNA-n-4}\lemma{|}\Bfootnote{JNA 554,11-16}}\&[\smallbreak]


	

	  \pstart tasmād vastu vā ghaṭapaṭādi sandigdhavastu vā sādhakabādhakātikrāntam, avastu vātmadikkālākṣaṇikādikam adhyavasitam iti, apratibhāse 'pi pravṛttiviṣayīkṛtam ity arthaḥ | ayam eva cāropaikīkaraṇādhyavasāyābhedagrahādīnām arthaḥ sarvatra śāstre boddhavyaḥ | tasmād adhyavasāyasyākāraviśeṣayogād agṛhīte 'pi pravartanayogyatā nāma yo dharmas tayā bāhyādhyavasāyayor grāhyagrāhakabhāvaś cet savṛttyā duṣpariharaḥ, tadā viṣayiviṣayabhāvo 'pi labdha ity adhyavasāyamātreṇa viṣayaviṣayitvam uktam iti yuktam | yad āha \persName{Alaṅkārakāraḥ}
	\pend
      
	    
	    \stanza[\smallbreak]
	kathaṃ tadviṣayatvaṃ tatra pravartanād iti | \&[\smallbreak]


	

	  \pstart etena yad uktam, katham avasīyamānas \edlabel{ratnakīrtinibandhāvali__36r1PF7IMRL3AMLSE72I8R6WZC0}\label{ratnakīrtinibandhāvali__36r1PF7IMRL3AMLSE72I8R6WZC0}\edtext{}{\lemma{tayā so 'rtho}\xxref{ratnakīrtinibandhāvali__36r1PF7IMRL3AMLSE72I8R6WZC0}{ratnakīrtinibandhāvali__36r1PF7IMRIYIIGI4TNLZD3AB1C}\Afootnote{\label{RNA-tc-6}tayāsortho \cite{RNAms?citedRange=75a3} ; tayātmārtho \cite{CAPV?citedRange=140.12} ; tayā so 'rtho \cite{SāSiŚā?citedRange=387.11} }}tayā so 'rtho\edlabel{ratnakīrtinibandhāvali__36r1PF7IMRIYIIGI4TNLZD3AB1C}\label{ratnakīrtinibandhāvali__36r1PF7IMRIYIIGI4TNLZD3AB1C} na prakāśyata ityādi, tan nirastam, tadaprakāśe 'pi tadadhyavasāyasya vyavasthāpitatvāt | asati cārthe sā na syād ity apy ayuktam, ātmāder adhyavaseyasya pratibhāsapratikṣepe buddhyā saha tādātmyābhāvāt | na ca sarvākāraniścayaprasaṅgadoṣaḥ saṅgataḥ | sarvākāraniścayo hi sarveṣv ākāreṣu pravṛttikārakatvātmā niruktaḥ, na caikārollekhino vikalpasyākārāntare pravartanaśaktir anubhavaviṣaya iti kutaḥ śabdapramāṇāntarānapekṣeti yuktam | tatra nirvikalpakaṃ spaṣṭapratibhāsatvād grāhakaṃ vyavasthāpyate | vikalpas tu spaṣṭaikavyāvṛttyullekhād āropakādivyavahārabhājanam | yathā ca bāhye sati kvacid bhramavyavasthā tathāntarnaye 'pi sarvatra | kevalaṃ bahirmukhapravṛtyapekṣayā kriyamāṇo nātmani kaścid bhrama ity uktaṃ bhavati | na ca gosvalakṣaṇaprakāśāvakāśaḥ, svākārasyaiva sphuraṇāt, svalakṣaṇe ca saṃketāyogāt | vikalpikaiva na syād iti tu svarūpāpekṣayā siddhasādhanam | bāhyāpekṣayā tv adhyavasāyavad vikalpikaiva sā buddhis tathā | tasmād aśeṣagovyaktisādhāraṇena gotvena gobuddhir alīkena sābhilāpena viplavāt prakhyātīti tathā prakhyānam asyā gavāvasāya ity eṣṭavyam ity api neṣṭavyam eva, caraṇam ardanādinā pratyavasthāne 'pi yuktiśāstravahirbhūtatvād etadabhāve 'pi kathitadoṣapradhvaṃsāt | na hi vikalpabuddhāv alīkākārasphuraṇam eva bāhyasyādhyavasāya iti kācid arthasaṅgatiḥ, arthasyeti sambandhānupapatteḥ \edlabel{ratnakīrtinibandhāvali__36r1PF7IMRGUOJ6VH79IC2IAM5I}\label{ratnakīrtinibandhāvali__36r1PF7IMRGUOJ6VH79IC2IAM5I}\edtext{}{\lemma{bodhe ca bhramābhāvāt}\xxref{ratnakīrtinibandhāvali__36r1PF7IMRGUOJ6VH79IC2IAM5I}{ratnakīrtinibandhāvali__36r1PF7IMREQOM5KKKSKN74E0AR}\Afootnote{\label{RNA-tc-7} \cite{RNAms?citedRange=75a6--75b1,SāSiŚā?citedRange=369.23} | buddher atra kramābhāvāt \cite{CAPV?citedRange=140.28} }}bodhe ca bhramābhāvāt\edlabel{ratnakīrtinibandhāvali__36r1PF7IMREQOM5KKKSKN74E0AR}\label{ratnakīrtinibandhāvali__36r1PF7IMREQOM5KKKSKN74E0AR} pratyakṣataiva, katham adhyavasāyaḥ | apratibhāsamānasyāpi svalakṣaṇasya bhrāntyāvasāyād iti tu na budhyāmahe | avasāyena hi tadvittisparśe pratibhāsaḥ ko 'paraḥ | tadvittāv apy aspaṣṭatvād adhyavasāya ity apy ayuktam, tadrūpavittāv aspaṣṭatvasyaivābhāvāt |
	\pend
      
	    
	    \stanza[\smallbreak]
	jāto nāmāśrayo 'nyonyaś cetasāṃ tasya vastunaḥ |&ekasyaiva kuto rūpaṃ bhinnākārāvabhāsi yat ||\&[\smallbreak]


	

	  \pstart ity ācāryaḥ smaryatām | na ca tadāsau bhrāntir bhavitum arhati, vastusvarūpasyaiva nirbhāsāt || \edlabel{thakur75-141.3}\label{thakur75-141.3} alīkavṛtter iti cet | saivāstu | bāhyasyāsphurato 'dhyavasāyaḥ katham | saiva sa iti cet | alīkam idam iti viduṣo bāhyādhyavasāyavyasthābhāvāt, bāhyāsphuraṇāt tadapratibaddhatvāc ca | pratibandhe 'pi tasyeti syāt, na punas tadadhyavasāyaḥ, tadasphuraṇasphuraṇayor api tadayogād ity ala[ {\corr mi}]tinirbandhena | tad evam apratibhāsino 'pi vipakṣād adhyavasāyamātrasiddhād eva vyāvṛtto doṣatrayanirmuktaḥ prakāśamānatātmako hetur yāvat prakāśāvadhijñānātmakacitrākāracakrasyaikatvaṃ sādhayaty eva || yad āhur guravaḥ
	\pend
      
	    
	    \stanza[\smallbreak]
	bhāsate yat tad ekaṃ tad yathā citre sitākṛtiḥ |&bhāsate cākhilaṃ citraṃ pītaśītasukhādikam || \edtext{}{\lemma{||}\Bfootnote{(JNA 569,13f.)}}&nātrāsiddhiḥ prakāśasya citre dharmiṇi darśanāt |&na ca sādhyaviyuktatvaṃ dṛṣṭāntasyāpi dṛśyate ||&ekaikāṇunimagnatvāt saṃvittir na parasparam |&na caikāṇuprakāśo 'sti sthūlam eva sphuraty ataḥ ||&bāhyāṇūnāṃ pratībhāso buddhir ekā sthavīyasī |&jñānāṇūnāṃ ka ekas tu pratibhāso bhaviṣyati || \edtext{}{\lemma{||}\Bfootnote{(JNA 569,19-22)}}&tasmāt sthūlatayā vyāpto nirbhāsas tannivṛttitaḥ |&nivartamāno 'nekasmād ekatve viniyamyate ||&yathā sajātīyamatād bhāgād bhedanirākriyā |&anābhāsaprasaṅgena vijātīyamatāt tathā ||&tan nāstu sādhyo dṛṣṭānto na ca śaṅkāviparyaye |&ato nirdoṣato hetoś citrādvaitavyavasthitiḥ || \edtext{}{\lemma{||}\Bfootnote{(JNA 570,3-8)}}\&[\smallbreak]


	

	  \pstart saṅgrahaślokaś ca 
	\pend
      
	    
	    \stanza[\smallbreak]
	ekatvena yathāptimān abhimato bhāsas tathā vyāpyate sthaulyenāpy aṇuśo na hi kvacid idaṃ svapne 'pi nirbhāsanam |&tena pratyaṇubhedanety uparataṃ tadvyāpakasyātyayād ekatvena parītam ākṛticayaś cāyaṃ vinirbhāsate\&[\smallbreak]


	

	  \pstart || iti ||
	\pend
      

	  \pstart nanu cātra dṛṣṭāntadārṣṭāntikayor ubhayatrāpy ekatvaṃ pratyakṣato 'numānāc ca viruddhadharmādhyāsalakṣaṇāt pratihatam, tat katham anumānād ekatvasiddhir iti cet | ucyate | yad etat pratyakṣaṃ bhedasādhakam upanīyate, tat kiṃ nīlādīnām anātmabhūtam ātambhūtaṃ vā | prathamapakṣe, āstāṃ tāvad eṣām ato bhedasiddhiḥ, \leavevmode\ledsidenote{\textenglish{pb in}}\label{RNAms_76a} sattāmātram api na sidhyet | sa hi nīlādiko 'rtho jaḍo vijñānāntarātmālīkasvabhāvo vā svīkartavyaḥ | triṣv api pakṣeṣu prakāśyaprakāśakabhāvābhāvaḥ | tathā hi jñānasya prakāśakatvaṃ nāma kiṃ vidyamānatvaṃ vyāpārāveśo vā | prathamapakṣe sarvasarvadarśitvaprasaṅgaḥ, sarvapuruṣajñānavidyamānatāyāḥ sarvaṃ pratyaviśiṣṭatvāt | tathā nīlādibhir api jñānasya grahaṇaprasaṅgaḥ, teṣām api vidyamānatvalakṣaṇagrāhakatvasambhavāt ||
	\pend
      

	  \pstart atha jñānatve sati vidyamānatvam iti saviśeṣaṇaṃ lakṣaṇam ucyate | tat kiṃ nīlādīnām ajñānatve kośapānam āyuṣmatā kartavyam, yena sattāmātreṇa samasamayaṃ sphurator vijñānanīlādyoḥ pratijñāmātrād ekasya jaḍatvālīkatvabādhyatvāprakāśatvādi vyavasthāpyate |
	\pend
      

	  \pstart atha dvitīyas tadā sa kiṃ vyāpāraḥ pratyakṣasyātmā jñānāntaram, arthasyātmārthāntaraṃ vā syāt | prathamavikalpe svātmani kāritravirodhaḥ |\edtext{}{\lemma{|}\Bfootnote{kāritra actually in ms, not kāritva}} dvitīyapakṣe jñānāntaraṃ yady anyaviṣayam arthasya na kiñcit | tadviṣayatvaṃ cādyāpi na siddham, tatpratyāsatter eva cintyamānatvāt ||
	\pend
      

	  \pstart tṛtīye punaḥ saṅkalpe nīlādikaṃ kṛtam eva syāt, na prakāśitam, tailavartyādibhir iva pradīpaḥ | prakāśas tu svayam eva | tathā ca jñānāntaratvāt santānāntaravad apratibhāsaprasaṅgaḥ |
	\pend
      

	  \pstart caturthe tu vikalpe arthāntare kṛte nīlādikaṃ tadavastham eva | na cānātmaprakāśanasāmarthyaṃ jñānasya svīkartum ucitam, vyāpāravat prakāśanasyāpy evaṃ nirākartavyatvāt | na cāgnidhūmayoḥ kāryakāraṇabhāva iva jñānajñeyayor api svābhāviko grāhyagrāhakabhāvo vaktavyaḥ, pramāṇasiddhakāryakāraṇabhāvavad grāhyagrāhakasvarūpayor adyāpi nirvaktum aśakyatvād iti kva nīlādivārtāpi yadbhedasiddhipratyāśā pratyakṣataḥ sampadyate ||
	\pend
      

	  \pstart athātmabhūtaṃ tat pratyakṣam iti dvītyaḥ pakṣaḥ, tadātmasvasaṃvedanam eva bhedasādhakam abhyupagataṃ bhavet | tac ca yadi pratyākāraṃ bhinnaṃ tadā sarveṣāṃ svasvarūpanimagnatvāc citraprakāśapraṇāśaprasaṅga ity uktam |
	\pend
      

	  \pstart athaitad doṣabhayāt sarveṣām ākārāṇām ekatvam eva svabhāvabhūtaṃ svasaṃvedanam iṣyate, tadaitad eva citrādvaitaṃ vijñānam ucyate, yad anekābhimatānāṃ sahopalabdhānāṃ nīlasukhādyākārāṇāṃ svabhāvabhūtākhaṇḍasvasaṃvedanapratyakṣaṃ nāma | yad āhur guruvaḥ
	\pend
      
	    
	    \stanza[\smallbreak]
	bhramābhramākalpanakalpanāni śātāsitādīny akhilākṣajāni |&jñānāny abhinnāni sahopalabdheḥ pūrvāparatvaṃ tu na vedyam eva ||\edtext{\textsuperscript{*}}{\lemma{*}\Bfootnote{(JNA 458,14-17)}}\&[\smallbreak]


	

	  \pstart iti |
	\pend
      

	  \pstart tad evaṃ dṛṣṭāntadārṣṭāntikayor ubhayatrāpi svasaṃvedanapratyakṣasiddham ekatvam avidyāvaśād vipratipattau satyām anumānataḥ sādhyate | ata eva svasaṃvedanapratyakṣād anumānāc ca ekatvasiddhau na pratyakṣāntaram | nāpi viruddhadharmādhyāsalakṣaṇam anumānaṃ bhedasādhanāya prāptāvasaram, bhedagrāhakasya bhinnasya pratyakṣasyoktakrameṇāprāmāṇyāt, pakṣasya pratyakṣādibādhitatvāt | \edlabel{thakur75-143.6}\label{thakur75-143.6} nanu brūyān nāma kiñcit, tathāpi pratibhāsabhedād bheda eva, na hi dṛṣṭe 'nupapannaṃ nāmeti cet | hanta pratibhāsaśabdena kim abhipretam, kim ākāracakraṃ sphuraṇaṃ vā | tatra yadi prathamaḥ pakṣaḥ, tadā bāhye 'rthe pratyetavye buddhyākāraḥ pramāṇam | tathācākārabhedo vyavahartavya eva | anyathā bāhyabhedo na sidhyet | yadā punar ākāracakram eva prameyam svasaṃvedanaṃ ca pramāṇaṃ tadā tenaiva nīlādīnāṃ svabhāvabhūtenākhaṇḍātmanā ekīkṛtānāṃ katham apramādī bhedam ācakṣīta | \edlabel{thakur75-143.12}\label{thakur75-143.12} dvitīyapakṣe tu sphuraṇaṃ svabhāvabhūtākhaṇḍasvasaṃvedanam evoktam iti | tathāpi kathaṃ bhedas tasmād yathordhvam indriyapratyakṣataḥ kṣaṇabhede pratīte 'py avidyāvaśād ekatvādhyavasāyaḥ tathā tiryaksvasaṃvedanapratyakṣeṇākārābhede 'dhigate 'py avidyāvaśād eva bhedāvasāyaḥ || \edlabel{thakur75-143.15}\label{thakur75-143.15} yady evaṃ viruddhadharmādhyāsato vijñānākāracakravad vyāpto 'pi na bhidyeteti cet | na, bāhye dharmiṇy anekatvasya sādhyasya pratyakṣādyabādhitatvāt | buddhyākārakadambake tūktakrameṇa svasaṃvedanādisiddhaikatve 'nekatvasya pratyākhyānād bādhakāvatāra eva nāsti | tasmād vijñānatve satīti hetuviśeṣaṇaṃ kartavyam yena bāhyasyaiva bhedaḥ sidhyati || \edlabel{thakur75-143.20}\label{thakur75-143.20} nanu yadi vijñānātmakaṃ vicitrākāracakram ekaṃ tadā nīlākāra eva pītādyākāravṛndaṃ praviśet | tathā prakāśākāracakrayor abhedo vyaktisāmānyavat prakāśa eva, ākāracakram eva vā syād iti cet | asad etat | tathā hi dvayor apy anayoḥ [ {\corr prasaṅgayor viparyayo}] bhedaḥ, sa ca bāhyārthavāda eva yujyate, tatra bhedagrāhakasyendriyapratyakṣasyeṣṭatvāt | vijñānavāde tv anātmaprakāśābhāvāt svasaṃvedanam evaikaṃ pramāṇam | tato 'pi viparyayasya [ {\corr bhedasyāsiddheḥ}] prasaṅgo 'py asaṅgataḥ ity advaitam eva | \edlabel{thakur75-143.26}\label{thakur75-143.26} kiṃ ca evaṃ sthūlanīlādyākāro 'pi paramāṇumātre praviśed ity apratibhāsaṃ jagad āpadyeta | asti ca pratibhāsaḥ | tasmād yahtāvasthitānām evākārāṇām akhaṇḍasvasaṃvedanātmataivaikatvam, na bhedo na saṃkocaḥ svīkartavyo 'pratibhāsaprasaṅgāt | tathā kṛtakatvasyānityatvavastutvādibhir abhede kṛtakatvam evānityatvam eva vā syād ity api prasaṅgo vaktavya āpadyeta, sāmānyavyaktyor iva tayor vastuto 'bhedo 'khaṇḍātmatvāt || \edlabel{thakur75-143.31}\label{thakur75-143.31} vyāvṛttibheda eva param iti cet | yady evaṃ prakāśanīlādyor apy ayam eva kramo jāgartīty ekāvaśeṣaprasaṅgo bālapralāpaḥ | tad evaṃ
	\pend
      

	  \pstart bāhyaṃ na naśyati bhidāṇutayāpi sattvād arthakriyāvirahasaṃkaratātmabhede | buddhis tu naśyati bhidaiva vidaiva sattvāc citrāpy ato na bhidam eti kim atra kurmaḥ || \edtext{}{\lemma{||}\Bfootnote{(JNA 573,21-24)}} nanu deśavitānāptir nātmāntaraviyoginaḥ | deśavitānahānau na bhāsa ity api śakyate ||
	\pend
      

	  \pstart iti cet |
	\pend
      

	  \pstart na svātmāntaram anyātmā sa bāhyasyaiva yujyate | buddheḥ svavittiniṣṭhāyā yaḥ paras tasya kā gatiḥ || \edtext{}{\lemma{||}\Bfootnote{(JNA 572,3f.)}}
	\pend
      

	  \pstart hanta tathāpi
	\pend
      

	  \pstart nīlādivat tad ekaṃ ca katham etat sametu cet | nīlam aṃśāntaraṃ caikaṃ kathaṃ tadbhāti saṅgatam || neṣṭaṃ tad api cet tarhi kvāṇvantarbhidi bhāsanam | na parīkṣākṣamaṃ cāṇuḥ kutas tasya tadā bhidā || mā bhūd avastubhāvāc cet so 'py ekatvahatau bhavet | nirbhāsād ekatāsiddhau svavitter vastutā sthitā || \edtext{}{\lemma{||}\Bfootnote{(JNA 571,19-24)}} na pratītyasamutpādo 'nutpādo vāsya bādhakaḥ | ekānekaviyoge 'pi sphūrtimātreṇa sattvataḥ || kiṃ ca pūrvāparajñānam advaite yan na vidyate | pratītyotpannatā tasmād asiddher apy asādhanam || \edtext{}{\lemma{||}\Bfootnote{(JNA 577,22)}} anutpādo 'py anekānto 'kāryakāraṇarūpakam | hāne 'pi hetuphalayoḥ sphuradrūpaṃ kva gacchatu || \edtext{}{\lemma{||}\Bfootnote{(JNA 578,2)}} ekānekatayā vastuvyāptiḥ siddhā yadi kvacit | sarvaśūnyatvasamaye hetur iṣṭavighātakṛt || atha lokaprasiddhau ca na sarvalokakalpitam | vastuvyavasthā śaraṇaṃ kiṃ tu mānena saṅgatam || na cādhyakṣānumānābhyām anaṅgaṃ kvacid īkṣitam | yasya rāśir anekaṃ syān nāpi vastu ca kiñcana || \edtext{}{\lemma{||}\Bfootnote{(JNA 574,8-11)}} yasya caikataratvābhyāṃ sattvavyāptiḥ sa hanyatām | abhrāntavittimātreṇa sattāvādī tu jitvaraḥ || \edtext{}{\lemma{||}\Bfootnote{(JNA 574, 16f.)}}
	\pend
      

	  \pstart ||samāptaś citrādvaitaprakāśavādo 'yam ||
	\pend
      
	    
	    \stanza[\smallbreak]
	grāhyaṃ na tasya grahaṇaṃ na tena jñānāntaragrāhyatayāpi śūnyaḥ |&tathāpi ca jñānamayaḥ prakāśaḥ pratyakṣapakṣas tu tavāvirāsīt ||\&[\smallbreak]


	
	    
	    \endnumbering% ending numbering from div
	    \endgroup
	    
	  
	  
	% new div opening: depth here is 0
	
	    
	    \begingroup
	    \beginnumbering% beginning numbering from div depth=0
	    
	  
\chapter[{Santānāntaradūṣaṇam}]{Santānāntaradūṣaṇam}\label{Santānāntaradūṣaṇam}

	  \pstart atheha prakāśasahopalambhādisādhanabalena jaḍapadārtharāśāvapāste nīlapītādyaśeṣapadārthajāte ca svacittapratibhāsātmani svapnamāyādivad advayarūpe siddhe santānāntarasadasattānirūpaṇārtham idam ārabhyate | \edlabel{thakur75-145.6}\label{thakur75-145.6} evaṃ hi kecid āhuḥ | asty eva santānāntaram anumānapratītam | tathā hīcchācittasamanantaravyāhāravyavahārābhāsasya darśanāt tadabhāve cādarśanād upalambhānupalambhasādhanam anvayavyatirekaśarīram icchācittena saha vyāhārādyābhāsasya kāryakāraṇabhāvam ātmasantāne 'vadhāryecchācittasyāpratisaṃvedanasamaye 'pi vicchinnavyāhārādyābhāsadarśanāt tatkāraṇabhūtam icchācittam anumīyamānaṃ santānāntaram eva vyavatiṣṭhata iti | \edlabel{thakur75-145.12}\label{thakur75-145.12} atredam ālocyate | tadicchācittaṃ vyāhārādyābhāsasya kāraṇatayā vyavasthāpyamānam anumātur darśanayogyam atha dṛśyādṛśyaviśeṣaṇānapekṣam icchāmātram | yadi tāvad ādyo vikalpas tadānumātur darśanayogyatvād icchācittasyānumānakāle 'nupalabdhir abhāvam eva gamayatīty anupambhākhyapratyakṣabādhitatvāt kvānumānāvakāśas tasya | yadi punar icchācittam anumānakāle 'py anubhūyeta, tadā kim asyānumānena | athaivam agnidhūmayos tadutpattisiddhyanantaraṃ naganikuñje dhūmam upalabhamāno nāgnim apy anuminuyāt, tatrāpy agner anupalabdhibādhitatvāt, upalambhe cānumānavaiphalyāt | naivam, anumānasamaye deśaviprakarṣavato vahner darśanāyogyatvena dṛśyānupalabdhivirahāt, adṛśyānupalambhasya cābhāvasādhanatvavirodhāt | icchācittasya tu nāsti deśaviprakarṣaḥ | icchācittaṃ hi svasambaddham evānumātur darśanayogyam, tasya ca deśādiviprakarṣa ity alaukikam etat | \edlabel{thakur75-145.23}\label{thakur75-145.23} atha dvitīyo vikalpaḥ | tathā hīcchācittamātraṃ svaparasantānasādhāraṇadṛśyādṛśyaviśeṣaṇānapekṣaṃ vyāhārādyābhāsaṃ prati kāraṇatayāvadhāryate | tadavadhāraṇaṃ kena pramāṇena | vyāhārādyābhāsasya hīcchāmātrābhāve 'bhāvaṃ pratītya tadutpattisiddhigaveṣaṇā | na cecchāmātrasya svaparasantānasādhāraṇasya svasaṃvedanenānyena vābhāvaḥ śakyāvagamaḥ | yathā hi vahnimātrasya deśakālavyavahitasyāpi dhūmotpādadeśakālayor yadi syād upalabhyetaiva mayeti sambhāvitasyānumātṛpuruṣendriyapratyakṣeṇa dhūmotpādāt prāgabhāvo 'vadhāryamāṇas tadutpattisiddhim adhyāsayatīti vyavahitadeśakālasyāpi vahner dhūmamātraṃ prati kāraṇatvāvadhāraṇam, svabhāvaviprakṛṣṭasya tu jaṭharabhavādisādhāraṇasya sarvathānumātṛpuruṣāśakyābhāvapratītikasya vyāptibahirbhāva eva | tathātrāpīcchācittaṃ parasantānasādhāraṇam api yāvad yadīha syād upalabhyetaiva mayeti yadi sambhāvayituṃ śakyeta tadā tadvyatirekasiddhidvāreṇa kāraṇatayāvadhāryate | kevalaṃ svabhāvaviprakṛṣṭe cittamātre 'stamiteyaṃ katheti || \edlabel{thakur75-146.7}\label{thakur75-146.7} na ca pracittaṃ kālaviprakṛṣṭaṃ varamānatvād asya, atītānāgatayor eva kālaviprakṛṣṭatvena vyavahārāt | \edlabel{thakur75-146.9}\label{thakur75-146.9} nāpi deśaviprakṛṣṭam, yasminn eva hi śuklaśaṅkhādideśe svacittaṃ śuklākārapratibhāsi svasaṃvedanena vedyate taddeśavarty eva pītākārapratibhāsi parasantānabhāvi cittaṃ na vedyate | tat katham eṣa deśaviprakarṣaḥ || \edlabel{thakur75-146.12}\label{thakur75-146.12} athecchācittamātraṃ svasaṃvedanamātrāpekṣayā na svabhāvaviprakṛṣṭam | na hy agnir apy eko yenaivendriyavijñānena dṛśyate tenaivānyo 'pi dṛśyam | tatra yathā cakṣurvijñānamātrāpekṣayā agnimātraṃ dṛśyam iti vyavasthāpyate tathātrāpi svasaṃvedanamātrāpekṣayā icchācittamātraṃ svaparasantānasādhāraṇam api dṛśyam eveti | \edlabel{thakur75-146.16}\label{thakur75-146.16} atrocyate | kim atra mātraśabdenānumātṛpuruṣasambandhāsambandhābhyām aviśeṣitaṃ yasya kasyacit puruṣasyendriyajñānaṃ vastuviṣayīkurvāṇam asya dṛśyatāsambhave 'pi nānimittam abhimatam | yady evaṃ piśācādir api dṛśyaḥ syāt | so 'pi hi kasyacit puṃso yogyādeḥ svajātīyasya vā piśācāntarasya bhavaty evendriyajñānagocara iti na kaścit svabhāvaviprakṛṣṭaḥ syāt | tasmād anumātṛpuruṣasambandhitvam anapāsya vijñānasya svalakṣaṇādibhedanirāsapara eva mātraśabdo yuktaḥ | etad evāśaṅkya Dharmottareṇābhihitam -
	\pend
      

	  \pstart ekapratipattrapekṣaṃ cedaṃ pratyakṣalakṣaṇam | \edtext{}{\lemma{|}\Bfootnote{(NBṬ 104,5f.)}}
	\pend
      

	  \pstart ityādi | tenaivaṃ dṛśyatāsambhāvanā yadīha deśe kāle vā syād ghaṭādir niyamenopalabhyeta, madīyasya cakṣurvijñānamātrasya viṣayībhaved iti | paricitte tu na śakyam evam | yadīha paricittaṃ syāt niyamena madīyasya svasaṃvedanamātrasya viṣayi syād iti || \edlabel{thakur75-146.26}\label{thakur75-146.26} yadi cecchācittamātraṃ tadutpattigrhaṇasamaye dṛśyatayā sambhāvayitavyam, tadānumānakāle 'pi dṛśyatayā sambhāvya tadanupalambhenābhāvasādhane katham anumānaṃ pravartayitum idam ārabdham, pratyakṣeṇaiva pakṣabādhāt | na ca kālabhedena svabhāvaviprakarṣetarāv iti yatkiñcid etat | tasmād icchācittamātrasya svaparasantānasādhāraṇasya dṛśyatayā sambhāvayitum aśakyatvāt vyahārādyutpādāt prāg anupalambhe 'py abhāvasiddhau na tadabhāvaprayukto vyāhārādyabhāvaḥ pratīyata iti kathaṃ kāraṇatvasiddhir yataḥ kāryahetudvāreṇānumīyeta | icchācittaviśeṣas tu svasantānabhāvī na bhavaty evānumātur dṛśyaḥ | kiṃ tu tasya dṛśyānupalambhāj jijñāsitaviśeṣe dharmiṇi bādhitasya katham anumānam ity uktam eva || \edlabel{thakur75-147.1}\label{thakur75-147.1} tad evam icchācittaviśeṣe svasantānabhāvini sādhye pakṣasya pratyakṣabādhaḥ, icchācittamātre 'pi svaparasantānasādhāraṇe sādhye yady anupalambhamātreṇa dṛśyaviśeṣaṇānapekṣeṇa pratibandhasiddhisamaye tasyābhāvaḥ pratīyate, tadā pakṣīkṛte dharmiṇi tatheti sa eva doṣaḥ | atha na pratīyate tadā sandigdhavyatireko hetvābhāso vyāhārādir iti sthitam | \edlabel{thakur75-147.6}\label{thakur75-147.6} evaṃ tarhi santānāntarasādhakasyābhāvād bādhakasyāpi kasyacid adarśanād bhavatu tatra sandeha eveti kecit | tair idaṃ bādhakam abhidhīyamānam avadhīyatām | yadi hi santānāntaraṃ sambhavet tadā tato bhedena svasantānasyāvaśyaṃ bhavitavyam | anyathā svasantānād api prakāśamānāt tasya parasantānābhimatasya bhedo na syāt | na cābhedas tayor iti svasantānād bhedābhedābhyām abādhyasya parasantānasya sāmānyaśaśaviṣāṇādivad abhāva evāyāta iti kathaṃ sandehaḥ | tasmāt parasantānāpekṣayā svasantānasya bhedo 'py avaśyambhāvyah | sa ca bhedaḥ santānasya svabhāvaḥ svasantāne pratibhāsamāne niyamena pratibhāseta | katham aparathā pratibhānāpratibhānalakṣaṇaviruddhadharmādhyāse 'pi svasantānasya parasantānād bhedaḥ svabhāvatām āsādayet || \edlabel{thakur75-147.15}\label{thakur75-147.15} na cāsau bhedaḥ pratibhāsate | bhedapratibhāse hi upagamyamāne tadavadhibhūtasyāpi parasantānasya pratibhāso durapahnavaḥ syāt |
	\pend
      

	  \pstart asmād bhinnam itīdaṃ cet svarūpaṃ svasya cetasaḥ | sāvadher asya bhāsaḥ syān na vā grāhyaṃ tadātmanā || \edtext{}{\lemma{||}\Bfootnote{(JNA 570,15f.)}}
	\pend
      

	  \pstart bhede 'nyaleśam api naiti kuto bhinnaḥ | evam ādikam aśeṣam iha pravacanapradīpaśrīsākārasaṅgrahādivacanam anusmryatām | \edlabel{thakur75-147.21}\label{thakur75-147.21} yathā hi svasantānamātre parisphurati śaśaviṣāṇād asphurato na bhedaḥ pratibhāti tathā parasantānād api sphuraṇavirahiṇo na bhāty eva bhedaḥ | na hi parasantānāpekṣayā kaścid viśeṣaleśaḥ svasantānasya parisphurati yo nāsti śaśaviṣāṇāpekṣayā | na ca śaśaviṣāṇaparasantānāv apekṣya samāne svasantānapratibhāse śaśaviṣāṇāpekṣayā na bhedo nāpy abhedaḥ pratibhāti | parasantānāpekṣayā tu bheda eva bhātīty evam avasthāpayituṃ śakyam | \edlabel{thakur75-147.27}\label{thakur75-147.27} bhedābhedayor abhāvaparihāreṇa hi yathā bhedo vyavasthitaḥ tadvad bhedapratibhāso 'pi bhedābhedābhāvapratibhāsavilakṣaṇa evocito bhavitum, na ca tathānubhūyate | tathāpi bhedaḥ pratibhātīti vacanaracanam etat | bhāṣyakāranyāyo 'py atra bhedapratibhāsadūṣaṇe vistarato 'vagantavyaḥ || \edlabel{thakur75-148.1}\label{thakur75-148.1} yadi cāvadhipratibhāsavirahe 'pi bhedapratibhānam idaṃ paracittānukampayā kṣamitavyaṃ tarhi bahirarthasyāpi katham abhāvaḥ sidhyati | śakyaṃ hi tatrāpi sandeham avatārayitum, na bahirarthaḥ kasyacid ābhāsate, parasantānas tu parasya pratibhāsata eva, tataś cātraiva sandeho na bahirartha iti cet | etad api sakalaṃ sandigdham eva | na hy avaśyaṃ parasantānaḥ parasyābhāsate, kadācid asau nāsaty eva na cāsāv avabhāsata ity api vaktuṃ śakteḥ | \edlabel{thakur75-148.7}\label{thakur75-148.7} kiṃ ca mā nāma bhāsiṣṭa bahirarthaḥ kasyacid api tathāpi kathaṃ tadabhāvasiddhir bhedapratibhāsābhyupagamavādina itīyanmātram iha vivakṣitam | na cātra kaścid doṣaḥ | tasmād bahirarthena sādhāraṇaṃ santānāntaram iti kathaṃ vijñāptivādinām api saṃmataṃ bhaviṣyati | kiṃ ca kāryakāraṇabhāvo 'pi vijñānadvayasya bhedapratibhāsavādinā bādhitum aśakyaḥ | pūrvabhāvinī hi saṃvittiḥ parasaṃvittyapekṣayā bhedaṃ pūrvatvaṃ cātmano gṛhṇāty evāvadhipratibhāsavigame 'pi || \edlabel{thakur75-148.13}\label{thakur75-148.13} parabhāviny api saṃvittiḥ pūrvasaṃvittyapekṣayā bhedaṃ paratvaṃ cātmano 'dhigacchaty eva santānāntaravad iti niyatapūrvāparabhāvalakṣaṇe kāryakāraṇabhāve 'vabhāsamāne 'vasīyamāne ca nīlādicitrākāravat katham
	\pend
      

	  \pstart saṃvṛttyāstu yathā tathā \edtext{}{\lemma{tathā}\Bfootnote{(PV III 4d)}}
	\pend
      

	  \pstart iti bhagavato Vārtikakārasya vacanena phalitam atra mate | api ca citrākāracakre dharmiṇy advaitasādhanārtham upanyastasya prakāśamānatvādihetor bhedagrāhakapratyakṣāpahṛtaviṣayatvam udbhāvayataḥ prativādino bhedagrahaṇam anumanyamānena santānāntarasandehaṃ ca vinā katham uttaritavyaṃ bhavatā | \edlabel{thakur75-148.21}\label{thakur75-148.21} nanv evam api santānāntarābhāvaḥ kena pramāṇena siddhaḥ | na tāvat pratyakṣeṇa, tasya vidhiviṣayasya pratiṣedhasādhanānadhikārāt | nāpy anumānena, tasya dṛśyābhāvasādhananiyatasyātīndriyaparacittābhāvasādhane 'navatārād iti cet | atra brūmaḥ | santānāntarasambhave niyatabhāvaḥ tato bhedaḥ svacittasya | abhede svasantānāt parasantāna eva syāt | yathā ca yad upalabhyamānaṃ yena rūpeṇa na bhāsate na tat tena rūpeṇa sadvyavahārayogyaṃ yathā nīlaṃ pītarūpeṇa | nopalabhyate ca svacittam upalabhyamānaṃ parasantānād bhinnena rūpeṇeti bhedasya svacittatādātmyaniṣedhe dṛśyaviśeṣaṇaprayogānapekṣā svabhāvānupalabdhir iyam || \edlabel{thakur75-148.28}\label{thakur75-148.28} nāpy asiddhiḥ, bhedapratibhāse tadavadher api pratibhāsaprāpteḥ | avadhyapratibhāse tu bhedapratibhāsābhāvaḥ śaśaviṣāṇabhedapratibhāsābhāvavat siddha eva | evam anena pramāṇena santānāntarasya svacittāpekṣayā bhede pratikṣipte abhede ca svayam evāsambhavini bhedābhedābhyām avācyatvaṃ siddham | sāmānyādivad vastutāpahatir iti, kathaṃ bādhakābhāvāt santānāntare sandeho 'bhidhīyate | etac ca śāstrīyaprameyasmāraṇamātraphalaṃ kiñcil likhitam iti | param iha svayam anusandheyam | \edlabel{thakur75-149.3}\label{thakur75-149.3} api ca santānātare tāvad arvāgdṛśāṃ sandeho bhavadbhir anumanyate | bhagavatas tu kim avasthāpyatām | saṃdehāvasthāpane kathaṃ sarvajñatā | vidyamānam eva kadācit santānāntaraṃ bhagavatā nāvadhāryate tathāpy asau sarvajña iti katham etat | anumānaṃ ca santānāntaraviṣayaṃ prāg eva cintitam | na cānumānena pratītāv api sarvajñatā bhavitam arhati | pratyakṣeṇa paracittapratītau grāhyagrāhakabhāvo 'pi paracittasya bhagavaccittena sahāyāta iti bahirarthavāda eva mukhāntareṇopagataḥ syāt, katham ayaṃ vañcayati vādaḥ || \edlabel{thakur75-149.9}\label{thakur75-149.9} asmadīyam etena tu paracittaṃ nāsty eveti tadavadhāraṇakṛto [na] bhagavataḥ sarvajñatākṣatidoṣaḥ | yāvac ca bhedagrahaṇābhimānarūpā saṃvṛsttitāvat santānāntare sandehāt tadavabodhanārthaṃ vacanādir api pravartata iti svavacanavirodho 'pi na sambhavaty eva | na khalu santānāntaraviṣayaḥ sarvathā sandeho nāsty evety abhimatam asmākam, api tu paramārthagatir iyam upadarśitā | idam hi santānāntarābhāvasādhanam advayasādhanena sādhāraṇam iti naikaniyataḥ svavacanādivirodhas tatparihāro vā | citrākārasambhavamātreṇāpi ca vedāntadhvāntāpasāro Bhāṣyakāreṇa darśitaḥ | tathā ca
	\pend
      

	  \pstart ātmā sa tasyānubhavaḥ sa ca nānyasya kasyacit \edtext{}{\lemma{kasyacit}\Bfootnote{(PVA III 326ab)}}
	\pend
      

	  \pstart ityādivārtikavyākhyānabhāṣyam | \edlabel{thakur75-149.18}\label{thakur75-149.18} ātmavādas tarhi prasakta iti cet | na citrākārasaṃvedanāt \edtext{}{\lemma{citrākārasaṃvedanāt}\Bfootnote{(PVA 352,26)}} ityādi dveṣacikaluṣāśeṣā eva tuṣākāro 'pi vedāntasiddhānta ity alakṣita tadgranthānutthāpayantī santānāntarāpekṣayā paṭhitavatīty avasthā (?) sarvā saṃvṛtisatyāntaḥpātinī hy evāpaitīti sakalam anākulam iti ||
	\pend
      

	  \pstart || santānāntaradūṣaṇaṃ samāptam ||
	\pend
      
	    
	    \endnumbering% ending numbering from div
	    \endgroup
	    
	  \backmatter 
       \chapter{Bibliographical Hacks}
       \begin{minted}[fontfamily=rmfamily,fontsize=\footnotesize]{xml}
     <listBibl xmlns="http://www.tei-c.org/ns/1.0" xmlns:xi="http://www.w3.org/2001/XInclude"
          xml:lang="en">
   <head>References</head>
   <biblStruct xml:id="Frauwallner37">
      <analytic>
         <author>Erich Frauwallner</author>
         <title level="a">Beiträge zur Apohalehre II: Dharmottara</title>
      </analytic>
      <monogr>
         <title level="j">Wiener Zeitschrift für die Kunde des Morgenlandes</title>
         <imprint>
            <date>1937</date>
            <biblScope unit="pp">233--287</biblScope>
            <biblScope unit="vol">44</biblScope>
         </imprint>
      </monogr>
   </biblStruct>
   <biblStruct xml:id="buehnemann80">
      <monogr>
         <author>Gudrun Bühnemann</author>
         <title>Der Allwissende Buddha: Ein Beweis und seine Probleme: Ratnakīrti’s Sarvajñasiddhi</title>
         <imprint>
            <date>1980</date>
            <publisher>Arbeitskreis für Tibetische und Buddhistische Studien</publisher>
            <pubPlace>Wien</pubPlace>
         </imprint>
      </monogr>
      <series>
	        <title level="s">Tibetan Sanskrit Works Series</title>
	        <biblScope unit="vol">3</biblScope>
	     </series>
   </biblStruct>
   <biblStruct xml:id="moriyama11_transl_capv_1">
      <analytic>
         <author>Shinya Moriyama</author>
         <title>An Annotated Japanese Translation of Ratnakīrti's
	    Citrādvaitaprakāśavāda (1)</title>
      </analytic>
      <monogr>
         <title level="j">South Asian Classical Studies</title>
         <imprint>
            <date>2011</date>
            <biblScope unit="pp">51–93</biblScope>
            <biblScope unit="vol">6</biblScope>
         </imprint>
      </monogr>
   </biblStruct>
   <biblStruct xml:id="mccrea_patil06">
      <analytic>
         <author>Lawrence J. McCrea</author>
         <author>Parimal G. Patil</author>
         <title>Traditionalism and Innovation: Philosophy, Exegesis, and
	    Intellectual History in Jñānaśrīmitra’s Apohaprakaraṇa</title>
      </analytic>
      <monogr>
         <title level="j">Journal of Indian Philosophy</title>
         <imprint>
            <date>2006</date>
            <biblScope unit="pp">303--366</biblScope>
            <biblScope unit="vol">34.4</biblScope>
         </imprint>
      </monogr>
   </biblStruct>
   <bibl xml:id="CAPV">
	  Citrādvaitaprakāśavāda, in <ref target="#thakur75">Ratnakīrtinibandhāvaliḥ (Buddhist Nyāya Works of
	  Ratnakīrti)</ref>, pp. 129--144.
	</bibl>
   <bibl xml:id="NVTṬ">
	  Vācaspatimiśra. “Nyāyavārttikatātparyaṭīkā”. In:
	  Nyāyavārttikatātparyaṭīkā of Vācaspatimiśra. Ed. by Anantalal
	  Thakur.  New Delhi: Indian Council of Philosophical Research,
	  1996.
	</bibl>
   <bibl xml:id="NK1">
	  Vācaspatimiśra. “Nyāyakaṇikā”. In: Vidhiviveka of Śrī Maṇḍana
	  Miśra With the Commentary Nyāyakaṇikā of Vācaspati
	  Miśra. Ed. by Mahaprabhu Lal Goswami. Varanasi: Tara Printing
	  Works, 1984.
	</bibl>
   <bibl xml:id="SāSiŚā">
	  Jñānaśrīmitra. “Sākārasiddhiśāstram”. In: Jñānaśrīmitranibandhāvali.
	  Ed. by Anantalal Thakur. 2nd ed. Tibetan Sanskrit Works Series
	  5. Patna: Kashi Prasad Jayaswal Research Institute, 1987, 367–513.
	</bibl>
   <bibl xml:id="śabara_bhāṣya">
	     <title>Jaimini: Mimamsasutra, with Sabara's Bhasya, Adhyayas
	  1-7</title> Based on six editions (details see below). Input by
	  <editor>Andreas Pohlus</editor>
	     <ref target="http://gretil.sub.uni-goettingen.de/gretil/1_sanskr/6_sastra/3_phil/mimamsa/msbh1-7u.htm"/>
	  </bibl>
   <bibl xml:id="sucarita">
	     <title>Mīmāṃsāślokavarttikakāśikā</title>
	     <author>Sucaritamiśra</author>
	     <note>E-text of Trivandrum Sanskrit Series, 90, 99, 150 <date>1926,
	  1929, 1943</date>
      </note>  
	  </bibl>
   <msDesc xml:id="capv-np">
      <msIdentifier>
         <settlement>Nepal</settlement>
         <idno>5-137/ vi. mīm. 4 (reel b21/31); NGMCP id: 33642</idno>
         <msName>[Citrādvaitaprakāśavāda (incompl.)]</msName>
      </msIdentifier>
      <msContents>
         <p>Incomplete manuscript of the Citrādvaitaprakāśavāda.</p>
         <p>Identified by (and received from) <persName key="name person hi">Harunaga Isaacson</persName>.</p>
         <p>See also <ref target="http://catalogue.ngmcp.uni-hamburg.de/wiki/B_21-31_Khy%C4%81tiv%C4%81dagrantha">http://catalogue.ngmcp.uni-hamburg.de/wiki/B_21-31_Khyātivādagrantha</ref>.</p>
      </msContents>
   </msDesc>
   <msDesc xml:id="RNAms">
      <msIdentifier>
         <settlement>Beijing</settlement>
         <idno>Pek.-L., Nr. 52--58</idno>
         <msName>Ratnakīrtinibandhāvalī</msName>
      </msIdentifier>
      <msContents>
         <p>Please refer to the introduction to <ref target="#thakur75"/>, and to the description pp. 58 ff. in <bibl>Bandurski,
	    Frank. “Übersicht über die Göttinger Sammlungen der von Rāhula
	    Sāṅkṛtyāyana in Tibet aufgefundenen buddhistischen
	    Sanskrit-Texte (Funde buddhistischer Sanskrit-Handschriften,
	    III)”. In: Untersuchungen zur buddhistischen Literatur. Ed. by
	    Frank Bandurski et al. Sanskrit-Wörterbuch der buddhistischen
	    Texte aus den Turfan Funden Beiheft 5. Göttingen: Vandenhoeck
	    &amp; Ruprecht, 1994, 9– 126.</bibl>
	        </p>
         <p>The original ms could not be consulted. Instead, copies of
	    catalogue entry Xc 14/26 in the ``Sammlung des Seminars für
	    Indologie und Buddhismuskunde in Göttingen" (Collection of the
	    Seminar for Indology and Buddhist studies in Göttingen) were
	    used.</p>
      </msContents>
   </msDesc>
   <bibl xml:id="TBh-GOS">
	  Mokṣākaragupta. “Tarkabhāṣā”. In: Tarkabhāṣā of
	  Mokṣākara Gupta. Ed. by Embar
	  Krishnamacharya. Gaekwad’s Oriental Series 94. Baroda:
	  Oriental Institute, 1942, 1–39
	</bibl>
   <bibl xml:id="TBh-Mysore">
	  Mokṣākaragupta. “Tarkabhāṣa”. In: Tarkabhāṣa and
	  Vādasthāna of Mokṣākaragupta and Jitāripāda. Ed. by
	  H.R. Rangaswami Iyengar. 2nd ed. Mysore: The Hindusthan
	  Press, 1952, 1–71.
	</bibl>
   <bibl xml:id="krasser02_zaGkar_Izvar_texts">
	  Krasser, H. (2002). Śaṅkaranandanas Īśvarāpākaraṇasaṅkṣepa. 1: Texte. Wien: Verl. der Österr. Akad. der Wiss.
	</bibl>
   <bibl xml:id="krasser02_zaGkar_Izvar_studie">
	  Krasser, H. (2002). Śaṅkaranandanas Īśvarāpākaraṇasaṅkṣepa. 2: Annotierte Übersetzungen und Studie zur Auseinandersetzung über die Existenz Gottes. Wien: Verl. der Österr. Akad. der Wiss.
	</bibl>
   <bibl xml:id="SVR">
	  Vādidevasūri. Syādvādaratnākara. In: Śrīmad Vādidevasūriviracitaḥ
	  Pramāṇanayatattvālokālaṅkāraḥ Tadvyākhyā ca Syādvādaratnākaraḥ,
	  ed. by Motīlāl Lādhājī. Puṇyapattana: Lakṣmaṇ Bhāurāv Kokāṭe, 1926--30
	</bibl>
</listBibl>
       \end{minted}
     
\chapter[{Critical Annotations}]{Critical Annotations}                                                                                                           % running endDocumentHook
     
	 \chapter{The TEI Header}
	 \begin{minted}[fontfamily=rmfamily,fontsize=\footnotesize,breaklines=true]{xml}
       <teiHeader xmlns="http://www.tei-c.org/ns/1.0" xmlns:xi="http://www.w3.org/2001/XInclude"
           xml:lang="en">
   <fileDesc>
      <titleStmt>
         <title type="main">Ratnakīrtinibandhāvali</title>
         <title type="sub">A SARIT edition</title>
         <author>Ratnakīrti</author>
         <respStmt>
            <persName key="name person jw">Jeson Woo
	  </persName>
            <resp>Creation of e-text from the Ratnakīrtinibandhāvali's
	  second edition (1975, see <ref target="#thakur75"/>).</resp>
         </respStmt>
         <respStmt>
            <persName key="name person pma">Patrick Mc Allister</persName>
            <resp>Conversion to TEI xml file, various corrections.
	  </resp>
            <resp>Maintenance of file for SARIT.
	  </resp>
         </respStmt>
      </titleStmt>
      <editionStmt>
         <p>The following remarks were at the beginning of the original word file:</p>
         <p>
	  Explanatory Remarks

	  <list>
               <item>1.  This is a database of Ratnakīrti’s works.  It
	    includes the whole work in the
	    Ratnakīrtinibandhāvaliḥ.</item>
               <item>2.  The list of the works is as follows:
	    <list>
                     <item>1) Sarvajñasiddhiḥ</item>
                     <item>2) Īśvarasādhanadūṣaṇam</item>
                     <item>3) Apohasiddhiḥ</item>
                     <item>4) Kṣaṇabhaṅgasiddhiḥ-Anvayātmikā</item>
                     <item>5) Kṣaṇabhaṅgasiddhiḥ-Vyatirekātmikā</item>
                     <item>6) Pramāṇāntarbhāvaprakaraṇam</item>
                     <item>7) Vyāptinirṇayaḥ</item>
                     <item>8) Sthirasiddhidūṣaṇam</item>
                     <item>9) Citrādvaitaprakāśavādaḥ</item>
                     <item>10) Santānāntaradūṣaṇam</item>
                  </list>
	              </item>
               <item>3.  The texts used for this database are as follows:
	    <p>
	      1), 2), 3), 4), 6), 7), 9) and 10): Ratnakīrtinibandhāvaliḥ, ed. A. Thakur, 
              Patna: Kashi Prasad Jayaswal Research Institute, 2nd ed. 1975.
	    </p>
	                 <p>
	      5): An Eleventh-Century Buddhist Logic of Exists, A. C. Senape Mcdermott, 
              Dordrecht-Holland: D. Reidel Publishing Company, 1967.
	    </p>
	    8): La Refutation Bouddhique de la Permanence des Choses
	    (Sthirasiddhidūṣaṇa) et la Preuve de la Momentanite des
	    Choses (Kṣaṇabhaṅgasiddhi), K. Mimaki, Paris: Instititut de
	    Civilization Indienne, 1976.</item>
               <item>4.  I give the page and the line numbers in two different ways.
	    <list>
                     <item>4.1 The numbers in each individual database but 5)
	      and 8) correspond to the page and the line numbers in
	      Thakur’s second edition.  For instance, [30.10] indicates
	      the page 30 and the line 10 in the edition.  The numbers
	      in 5) and 8) respectively correspond to those which appear
	      in Macdermott’s and Mimaki’s editions.  Therefore, their
	      numbers indicate the page and the line numbers in Thakur’s
	      first edition.</item>
                     <item>4.2 The whole number in the database of
	      Ratnakīrtinibandhāvaliḥ corresponds to the page and the
	      line numbers in Thakur’s second edition.</item>
                  </list>
               </item>
               <item>5.  I have made a critical edition of the
	    Kṣaṇabhaṅgasiddhi-Anvayātmikā on the basis of three previous
	    editions and the manuscript from the Nepal National
	    Library. I have also improved its some parts with the Pathna
	    manuscript, Jñānaśrīmitra’s Kṣaṇabhaṅgādhyāya and other
	    Naiyāika’s works, such as the Nyāyabhūṣaṇa and the
	    Tātparyaṭīkā.  However, I have made the database of other
	    works without a thorough investigation of them.  I have
	    intended to use it as a reference for reading the
	    Anvayātmikā.  Thus, I must admit that there are lots of
	    errors and misspellings in this version.  I would appreciate
	    it if the user would point out any mistake in this database
	    so that I can improve it.</item>
            </list>

	  Woo, Jeson 
	  Penn and Hiroshima U.
	</p>
         <p>
	  bearbeitet für die WORD-Benutzer von ONO, Dezember 1997. 
	  Ratnakīrtinibandhāvaliḥ,
	  ed. A. Thakur, Patna:
	  Kashi Prasad Jayaswal Research Institute, 2nd ed. 1975
	  _______________________________________________________
	</p>
      </editionStmt>
      <publicationStmt>
         <publisher>
            <ref target="http://sarit.indology.info">SARIT (http://sarit.indology.info)</ref>
         </publisher>
         <availability>
            <p>
	    This work is licensed under a Creative Commons Attribution-ShareAlike 3.0 Unported License.
	  </p>
         </availability>
         <date>2011</date>
         <idno type="gitBlob">$Id: 05a25e962ddfa8b45c17d31e3db3f2dd13628992 $</idno>
      </publicationStmt>
      <sourceDesc>
         <bibl xml:id="thakur75">
	           <title>Ratnakīrtinibandhāvaliḥ (Buddhist Nyāya Works of Ratnakīrti)</title>
	           <author>Ratnakīrti</author>
	           <editor xml:id="Thakur">Anantalal Thakur</editor>
	           <publisher>Kashi Prasad Jayaswal Research Institute</publisher>
	           <address>
               <name>Patna</name>
            </address>
	           <date>1975</date>
	           <edition>Second Revised Edition</edition>
	           <series>
	              <title level="s">Tibetan Sanskrit Works Series</title>
	              <biblScope unit="vol">3</biblScope>
	           </series>
	        </bibl>
         <bibl xml:id="kāśikā">
	           <title>The Mīmā[ṃ]sāślokavārtika with the commentary Kāśikā of Sucaritamiśra</title>
	           <editor>K. Sābaśiva Śāstrī</editor>
	           <publisher>Printed by the Superintendent, Government Press</publisher>
	           <date>1926--1943</date>
	           <address>
               <name>Trivandrum</name>
            </address>
	           <series>
	              <title level="s">Trivandrum Sanskrit Series</title>
	              <biblScope unit="vol">90,99,150</biblScope>
	           </series>
	        </bibl>
      </sourceDesc>
   </fileDesc>
   <encodingDesc>
      <p/>
      <!-- ... --></encodingDesc>
   <profileDesc><!-- ... --></profileDesc>
   <revisionDesc>
      <change who="#pma">Please see <ref target="https://github.com/paddymcall/SARIT/commits/master/ratnakIrti-nibandhAvali.xml">https://github.com/paddymcall/SARIT/commits/master/ratnakIrti-nibandhAvali.xml</ref> for a list of changes.</change>
      <change when="2011-07-20" who="#pma">
         <persName>Patrick Mc Allister</persName>: continuing work on the CAPV.
      </change>
      <change when="2009-03" who="#pma">
         <persName>Patrick Mc Allister</persName>: replaced all &lt; with « and
      all &gt; with ».
      </change>
      <change> Converted from source file to TEI XML by <persName>Patrick Mc
      Allister</persName> 
         <date>2009-03-10</date>
      </change>
   </revisionDesc>
</teiHeader>
	 \end{minted}
       
      \clearpage
      \begin{english}
      \printshorthands
      \printbibliography
      \end{english}
    
\end{document}
