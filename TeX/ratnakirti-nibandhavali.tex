\documentclass[article,a4paper]{memoir}
  \usepackage{euler}
  \usepackage{xltxtra}
  \usepackage{polyglossia}
  \setdefaultlanguage{sanskrit}
  % english should be available, notes and stuff
  \setotherlanguage{english}
  \setotherlanguage[numerals=arabic]{tibetan}
  \usepackage{fontspec}
  \usepackage{xunicode}
  \catcode`⃥=\active \def⃥{\textbackslash}
  \catcode`❴=\active \def❴{\{}
  \catcode`〔=\active \def〔{{[}}% translate 〔OPENING TORTOISE SHELL BRACKET
  \catcode`〕=\active \def〕{{]}}% translate 〕CLOSING TORTOISE SHELL BRACKET
  \catcode`❴=\active \def❴{\{}
  \catcode`❵=\active \def❵{\}}
  %% show a lot of tolerance
  \tolerance=9000
  \def\textJapanese{\fontspec{Kochi Mincho}}
  \def\textChinese{\fontspec{HAN NOM A}}
  \def\textKorean{\fontspec{Baekmuk Gulim} }
  % make sure English font is there
  \newfontfamily\englishfont[Mapping=tex-text]{TeX Gyre Schola}
    % set up a devanagari font
  \newfontfamily\devanagarifont{TeX Gyre Pagella}
	\newfontfamily\rmlatinfont[Mapping=tex-text]{TeX Gyre Pagella}
	\newfontfamily\tibetanfont[Script=Tibetan,Scale=1.2]{Tibetan Machine Uni}
  \newcommand\bo\tibetanfont
  
    \defaultfontfeatures{Scale=MatchLowercase,Mapping=tex-text}
	\setmainfont{TeX Gyre Pagella}
    \setsansfont{TeX Gyre Bonum}
  
  \setmonofont{DejaVu Sans Mono}
	    % numbering depth
	    \maxtocdepth{section}
	    \setsecnumdepth{all}
	    \newenvironment{docImprint}{\vskip 6pt}{\ifvmode\par\fi }
	    \newenvironment{docDate}{}{\ifvmode\par\fi }
	    \newenvironment{docAuthor}{\ifvmode\vskip4pt\fontsize{16pt}{18pt}\selectfont\fi\itshape}{\ifvmode\par\fi }
	    % \newenvironment{docTitle}{\vskip6pt\bfseries\fontsize{18pt}{22pt}\selectfont}{\par }
	    \newcommand{\docTitle}[1]{#1}
	    \newenvironment{titlePart}{ }{ }
	    \newenvironment{byline}{\vskip6pt\itshape\fontsize{16pt}{18pt}\selectfont}{\par }
	    % setup title page; see CTAN /info/latex-samples/TitlePages/, and memoir
	  \newcommand*{\plogo}{\fbox{$\mathcal{SARIT}$}}
	  \newcommand*{\makeCustomTitle}{\begin{english}\begingroup% from example titleTH, T&H Typography
	  \thispagestyle{empty}
	  \raggedleft
	  \vspace*{\baselineskip}
	  
	      % author(s)
	    {\Large Ratnakī\-rti}\\[0.167\textheight]
	    % maintitle
	    {\Huge Ratnakī\-rtinibandhā\-vali}\\[\baselineskip]
	    % titlesubtitle
	    {\small  — A SARIT edition}\\[\baselineskip]
	    {\Large SARIT}\\\vspace*{\baselineskip}\plogo\par
	  \vspace*{3\baselineskip}
	  \endgroup
	  \end{english}}
	  \newcommand{\gap}[1]{}
	  \newcommand{\corr}[1]{($^{x}$#1)}
	  \newcommand{\sic}[1]{($^{!}$#1)}
	  \newcommand{\reg}[1]{#1}
	  \newcommand{\orig}[1]{#1}
	  \newcommand{\abbr}[1]{#1}
	  \newcommand{\expan}[1]{#1}
	  \newcommand{\unclear}[1]{($^{?}$#1)}
	  \newcommand{\add}[1]{($^{+}$#1)}
	  \newcommand{\deletion}[1]{($^{-}$#1)}
	  \newcommand{\pratIka}[1]{\textcolor{cyan}{#1}}
	  \newcommand{\name}[1]{#1}
	  \newcommand{\persName}[1]{#1}
	  \newcommand{\placeName}[1]{#1}
	  % running latexPackages template
     \usepackage{xcolor}
     \definecolor{shadecolor}{gray}{0.95}
     \usepackage{longtable}
     \usepackage{ctable}
     \usepackage{rotating}
     \usepackage{lscape}
     \usepackage{ragged2e}
     
	 \usepackage{titling}
	 \usepackage{marginnote}
	 \renewcommand*{\marginfont}{\itshape\footnotesize}
	 \setlength\marginparwidth{.75in}
	 \usepackage{graphicx}
	 \usepackage{csquotes}
       
	 \def\Gin@extensions{.pdf,.png,.jpg,.mps,.tif}
       %% biblatex stuff start
	 \usepackage[backend=biber,citestyle=authoryear,bibstyle=authoryear]{biblatex}
	 
	       \addbibresource{sarit.bib}
	     
	 \renewcommand*{\citesetup}{%
	 \rmlatinfont
	 \biburlsetup
	 \frenchspacing}
	 \renewcommand{\bibfont}{\rmlatinfont}
	 \DeclareFieldFormat{postnote}{:#1}
	 \renewcommand{\postnotedelim}{}
	 %% biblatex stuff end
	 
	 \setcounter{errorcontextlines}{400}
       
	 \usepackage{lscape}
	 \usepackage{minted}
       
	   % pagestyles
	   \pagestyle{ruled}
	   
	   \makeoddfoot{ruled}{{\tiny\rmlatinfont \textit{Compiled: \today}}}{}{\rmlatinfont\thepage}
	   \makeevenfoot{ruled}{\rmlatinfont\thepage}{}{{\tiny\rmlatinfont \textit{Compiled: \today}}}
	   
	 
	 \usepackage[noend,series={A}]{reledmac}
	 
       % simplify what ledmac does with fonts, because it breaks. From the documentation of ledmac:
       % The notes are actually given seven parameters: the page, line, and sub-line num-
       % ber for the start of the lemma; the same three numbers for the end of the lemma;
       % and the font specifier for the lemma. 
       \makeatletter
       \def\select@lemmafont#1|#2|#3|#4|#5|#6|#7|%
       {}
       \makeatother
       \setlength{\stanzaindentbase}{20pt}
     \setstanzaindents{8,2,2,2,2,2,2,2,2,2,2,2,2,2,2,2,2,2,2,2,2,2,2,2,2,2,2,2,2,}
     % \setstanzapenalties{1,5000,10500}
     \lineation{page}
     % \linenummargin{inner}
     \linenumberstyle{arabic}
     \firstlinenum{
    5}
    \linenumincrement{5}
    \addtolength{\skip\Afootins}{1.5mm}
    \Xnotenumfont{\bfseries\footnotesize}
    \sidenotemargin{outer}
    \linenummargin{inner}
       
       \usepackage[pdftitle={Ratnakī\-rtinibandhā\-vali; A SARIT edition // Ratnakī\-rti},
       pdfauthor={SARIT (http://sarit.indology.info)}]{hyperref}
       \hyperbaseurl{}
       \usepackage[english]{cleveref}% clashes with eledmac < 1.10.1!
       % \newcommand{\cref}{\href}
     
\begin{document}
    
     \makeCustomTitle
     \let\tabcellsep&
	\frontmatter
	\tableofcontents
	% \listoffigures
	% \listoftables
	\cleardoublepage
        
	  
	% new div opening: depth here is 0
	
	    
	    \begingroup
	    \beginnumbering% beginning numbering from div depth=0
	    
	  
\chapter[{Ratnakī\-rtinibandhā\-vali}]{Ratnakī\-rtinibandhā\-vali}
	  
	% new div opening: depth here is 1
	
\section[{Sarvajñasiddhiḥ}]{Sarvajñasiddhiḥ}\edlabel{Sarvajñasiddhiḥ}\label{Sarvajñasiddhiḥ}

	  \pstart namas tā\-rā\-yai 
	\pend
      
	    
	    \stanza[\smallbreak]
yasminn avajñā\- narakaprasū\-tir bhaktiś ca sarvā\-bhimatapradā\-yinī\- |&avyā\-hataṃ yo jagadekabandhuḥ sa jñā\-yate sarvavid atra nirmalam ||\&[\smallbreak]


	

	  \pstart iha hi dharmajñā\-d aparam an- avaśeṣajñam anicchann api \persName{kumā\-rilo} dharmajña eva kevale pratiṣiddhe vedam upā\-deyam abhimanyamā\-naḥ paṭhati
	\pend
      
	    
	    \stanza[\smallbreak]
dharmajñatvaniṣedhas tu kevalo 'tropayujyate | &sarvam anyad vijā\-naṃs tu puruṣaḥ kena vā\-ryate || iti | \&[\smallbreak]


	

	  \pstart tad ayam ā\-cā\-ryo 'pi sarvasarvajñacaraṇareṇusanā\-thaṃ yā\-vad ā\-kā\-śaṃ jagadicchann api tribhuvanacū\-ḍā\-maṇī\-bhū\-tasaparikaraheyopā\-deyatattvajñapuruṣapuṇḍarī\-ka-prasā\-dhanā\-d apy apramā\-ṇakajaḍavaidikaśabdarā\-śipramukhasakaladurmatipravā\-dapratihatir ity antarnayann ā\-ha – 
	\pend
      
	    
	    \stanza[\smallbreak]
heyopā\-deyatattvasya sā\-bhyupā\-yasya vedakaḥ |&yaḥ pramā\-ṇam asā\-viṣṭo na tu sarvasya vedakaḥ ||\footnote{\begin{english}PV\end{english}}\&[\smallbreak]


	

	  \pstart ityā\-di || tad idā\-nī\-m upayuktasarvajñam eva tā\-vat prasā\-dhayā\-maḥ | paryante tu sarvasarvajñadohadam apy apaneṣyā\-maḥ | svā\-sthyam ā\-sthī\-yatā\-m | 
	\pend
      

	  \pstart yo yaḥ sā\-daranirantaradī\-rghakā\-lā\-bhyā\-sasahitacetoguṇaḥ sa sarvaḥ sphuṭī\-bhā\-vayogyaḥ | 
	\pend
      

	  \pstart yathā\- yuvatyā\-kā\-raḥ kā\-minaḥ puruṣasya | yathoktā\-bhyā\-sasahitacetoguṇā\-ś cā\-mī\- caturā\-ryasatyaviṣayā\- ā\-kā\-rā\- iti svabhā\-vo hetuḥ | 
	\pend
      

	  \pstart tatra na tā\-vā\-d ā\-śrayadvā\-reṇa hetudvā\-reṇa vā\-siddhisambhā\-vanā\- | saṃkalparū\-ḍhā\-nā\-ṃ caturā\-ryasatyā\-kā\-rā\-ṇā\-m \edtext{cetoguṇa}{\Afootnote{ \cite{} \cite{}cetoguṇatva \cite{}}}mā\-trasya ca hetoḥ pratyā\-tmavedyatvā\-t | nā\-pi sā\-daranirantaradī\-rghakā\-lā\-bhyā\-salakṣaṇaṃ hetuviśeṣaṇam asambhā\-vanī\-yam | tathā\- hi saṃsā\-rasvabhā\-vaṃ duḥkhā\-tiśayam apanetum iyaṃ saṃkalpā\-rū\-ḍhā\- caturā\-ryasatyā\-kā\-rabhā\-vanā\- prā\-rabdhā\- | asyā\-ś cā\-sambhā\-vanā\- nā\-ma kiṃ (1) bhā\-vyasya saṃkalpā\-rū\-ḍhatvā\-sambhavā\-t (2) anarthitvā\-t (3) heyarū\-pā\-niścayā\-t (4) heyasya nityatvā\-t (5) tasyā\-hetutvā\-t (6) taddhetor nityatvā\-t (7) heyahetvaparijñā\-nā\-t (8) tadbā\-dhakā\-bhā\-vā\-t (9) bā\-dhakā\-parijñā\-nā\-t (10) cittasya doṣā\-tmakatvā\-t (11) tasya vyavasthitaguṇatvā\-t (12) bhavā\-ntarā\-bhā\-vā\-t (13) dhvastadoṣapunarudbhavā\-d veti trayodaśa vikalpā\-ḥ ||
	\pend
      

	  \pstart tatra na tā\-vad ā\-dyaḥ pakṣaḥ | saparikaraheyopā\-deyā\-tmakasya caturā\-dyasatyā\-kā\-rasya bhā\-vyasya vikalpā\-rū\-ḍhasya pratyā\-tmavedyatvā\-t || 
	\pend
      

	  \pstart \edlabel{thakur75-2.9}\label{thakur75-2.9} nā\-pi dvitī\-yaḥ | duḥkhamā\-trasyā\-pi parityā\-gā\-rthitvena vyā\-pteḥ sarvajanā\-nubhavasiddhatvā\-t ||
	\pend
      

	  \pstart nā\-pi tṛtī\-yaḥ | saṃsā\-rā\-tmano duḥkhasvarū\-pasya pratī\-teḥ | katham asya duḥkhā\-tmakatvam iti cet | saṃkṣepataḥ kathitaṃ 
	\pend
      
	    
	    \stanza[\smallbreak]
sā\-kṣā\-d duḥkhaprakṛti narakaṃ pretatiryakkharū\-paṃ martye śama kvacana tad api grastam evā\-sukhena |&devā\-nā\-ṃ ca kṣayam upagate puṇyapā\-theyapiṇḍe caṇḍajvā\-lā\-vyatikaramuco hanta bhogā\-sta eva || \&[\smallbreak]


	

	  \pstart iti || 
	\pend
      

	  \pstart na ca caturthaḥ | vā\-rtamā\-nikapañcaskandhā\-tmakasya duḥkhasyotpā\-dadarśanā\-t || 
	\pend
      

	  \pstart na ca pañcamaḥ | duḥkhasya kā\-dā\-citkatvā\-t || 
	\pend
      

	  \pstart nā\-pi ṣaṣṭhaḥ | kā\-ryakā\-dā\-citkatvasya anityahetukatvena vyā\-ptatvā\-t || 
	\pend
      

	  \pstart nā\-pi saptamaḥ | duḥkhe viparyā\-satṛṣṇā\-pravṛttiśaktikarmabhiḥ sahitasyā\-tmadṛṣṭilakṣaṇasya hetoḥ sā\-ṃsā\-rikapañcaskandhalakṣaṇakā\-ryā\-nyathā\-nupapattito niścayā\-t | yad ā\-huḥ 
	\pend
      
	    
	    \stanza[\smallbreak]
ahaṃkā\-ras tā\-vat tadanu mamakā\-ras tadubhayaprasū\-to rā\-gā\-dis tadahitamater dveṣadahanaḥ |&tataḥ śeṣaḥ kleśas tata udayinaḥ karmavisarā\-dvisā\-rī\- saṃsā\-raḥ śaraṇarahito dā\-ruṇataraḥ ||\&[\smallbreak]


	
	    
	    \stanza[\smallbreak]
tasmā\-t tṛṣṇā\-viparyā\-sā\-v ā\-tmadṛṣṭipuraḥsarau |&aṃsā\-riskandhajanakau nirṇī\-tau kā\-ryahetutaḥ ||\&[\smallbreak]


	

	  \pstart ā\-tmadarśanasya cā\-vidyā\-tvam ā\-tmapratikṣepato draṣṭavyam | tadabhā\-ve 'pi kṣaṇabhaṅgaprastā\-ve paralokā\-dikam anā\-kulam avasthā\-pitam || 
	\pend
      

	  \pstart na cā\-ṣṭamaḥ | ā\-tmadṛṣṭirū\-pā\-yā\- avidyā\-yā\-ḥ pratipakṣabhū\-tasya nairā\-tmyadarśanasya sambhavā\-t || 
	\pend
      

	  \pstart nā\-pi navamaḥ | nairā\-tmyadarśanasya mā\-rgaśabdavā\-cyasya pramā\-ṇato niścitatvā\-t || 
	\pend
      

	  \pstart daśamo 'py asambhavī\- | doṣā\-vasthā\-yā\-ṃ cittasya saṃskā\-rā\-pekṣatvā\-t | yo hi yatsvabhā\-vas tasmin svabhā\-ve vyavasthito na saṃskā\-ram apekṣate | yathā\- doṣam apanī\-ya tapanī\-yam akṣayadaśā\-yā\-m avasthitam | apekṣate ca cittam avidyā\-vasthā\-yā\-ṃ saṃskā\-ram iti vyā\-pakaviruddhopalabdhiḥ | pratiṣedhyasya tatsvabhā\-vatvasya yadvyā\-pakaṃ saṃskā\-ranirapekṣatvaṃ tadviruddhaṃ tadapekṣatvam iti cittasya doṣā\-tmakatvakṣatiḥ || 
	\pend
      

	  \pstart ekā\-daśo 'py ayuktaḥ | cetasas tattatsamskā\-rā\-tiśaye prajñā\-tiśayadarśanā\-t || 
	\pend
      

	  \pstart na ca dvā\-daśaḥ | paralokaprasā\-dhanā\-t | tathā\- hi, yac cittaṃ tat cittā\-ntaraṃ pratisandhatte | yathedā\-nī\-ntanaṃ cittam | cittaṃ ca maraṇakā\-labhā\-vī\-ti svabhā\-vahetuḥ | 
	\pend
      

	  \pstart na cā\-rhaccaramacittena vyabhicā\-raḥ | tasyā\-gamamā\-trataḥ pratī\-tatvā\-t | niḥkleśacittā\-ntarajananā\-d vā\- | hetor vā\- kleśe satī\-ti viśeṣaṇā\-d ity anā\-gatabhavasiddhiḥ | evaṃ yac cittaṃ tac cittā\-ntarapū\-rvakaṃ yathedā\-nī\-ntanaṃ cittaṃ | cittaṃ ca janmasamayabhā\-vī\-ty arthataḥ kā\-ryahetur ity atī\-tabhavasiddhiḥ || 
	\pend
      

	  \pstart na ca trayodaśaḥ | doṣakā\-raṇasyā\-tmadarśanasya yadviruddhaṃ nairā\-tmyadarśanaṃ tasya nirupadravatvā\-t | bhū\-tā\-rthatvā\-t | svabhā\-vatvā\-c ca | sarvadā\-vasthiteḥ | tan nā\-yaṃ viśeṣaṇā\-siddho 'pi hetuḥ | tathā\-pī\-dṛśo 'bhyā\-so na kasyacid dṛśyata iti cet | na dṛśyatā\-m | sambhā\-vanā\- tā\-vad aśakyapratiṣedhā\- | idā\-nī\-ntanajanapravṛttiś cā\-vyā\-hateti nā\-paraṃ gamyate | ata evedaṃ sambhā\-vanā\-numā\-nam ucyate || 
	\pend
      

	  \pstart na caiṣa viruddho hetuḥ | sapakṣe kā\-miny ā\-kā\-re sambhavā\-t | 
	\pend
      

	  \pstart na cā\-naikā\-ntikaḥ | abhyā\-sasahitacetoguṇasphuṭapratibhā\-sayoḥ kā\-ryakā\-raṇayor ghaṭakumbhakā\-rayor iva sarvopasaṃhā\-reṇa pratyakṣā\-nupalambhataḥ kā\-ryakā\-raṇabhā\-vasiddhā\-v abhyā\-sasahitacetoguṇatvasya sā\-dhanasya sphuṭapratibhā\-sakaraṇayogyatayā\- vyā\-ptisiddheḥ | tathā\- hi vyā\-ptyadhikaraṇe kā\-mā\-tur avartini yuvatyā\-kā\-re sā\-daranirantaradī\-rghakā\-lā\-bhyā\-sasahitacetoguṇā\-t pū\-rvaṃ anupalabdhiḥ sphuṭā\-bhasya | paścā\-d abhyā\-sasaṃvedanaṃ sphuṭā\-bhasaṃvedanam iti | trividhapratyakṣā\-nupalambhasā\-dhyaḥ kā\-ryakā\-raṇabhā\-vaḥ sphuṭapratibhā\-sā\-bhyā\-sasacivacittā\-kā\-rayor iyam upapannā\- sarvopasaṃhā\-ravatī\- vyā\-ptiḥ | ato 'naikā\-ntikatā\-py asambhavinī\-ty anavadyo hetuḥ || 
	\pend
      

	  \pstart nanu katham anumā\-nataḥ sarvajñasiddhipratyā\-śā\- | tasya parokṣatvena tatpratibaddhaliṅgā\-niścayā\-t | kiṃ ca sarvajñasattā\-sā\-dhane sarvo hetur na trayī\-ṃ doṣajā\-tim atipatati | sarvajñe hi dharmiṇy asiddhatvam | asarvajñe hi viruddhatvam | ubhayā\-tmake 'py anaikā\-ntikatvam iti || 
	\pend
      

	  \pstart api ca abhyā\-sā\-t kā\-raṇā\-t kā\-ryasya sphuṭā\-bhasya pratī\-tau nā\-vaśyaṃ kā\-raṇā\-ni kā\-ryavanti bhavantī\-ty anaikā\-ntikatā\- | atha sphuṭī\-bhā\-vayogyatā\-numī\-yate | sā\-pi śaktir ucyate | sā\- ca kā\-rye 'nantarā\- sā\-ntarā\- vā\- | atrā\-dyā\- kā\-ryasamadhigamyā\- | na cā\-dhigatakā\-ryasya tayā\- kaścid upayogaḥ | dvitī\-yā\- tu kā\-ryā\-vasā\-yam aikā\-ntikaṃ na sā\-dhayet || 
	\pend
      

	  \pstart na ca kā\-ryā\-pratī\-tau yogyatā\-niścayaḥ sambhavī\- | nā\-pi yogyatā\-mā\-trasā\-dhane kṛtā\-rthaḥ sā\-dhanavā\-dī\- | sarvajñajñā\-ne kā\-rye vivā\-dasya tā\-davasthyā\-d | bhavatu sphuṭī\-bhā\-vasya siddhiḥ | tathā\-pi kaḥ prastā\-vaḥ sarvajñavivā\-de sā\-dhanam ā\-rabdhavataḥ sphuṭatvaṃ cetasaḥ sā\-dhayitum || 
	\pend
      

	  \pstart kiṃ ca prasiddhā\-numā\-ne bhū\-talasya dharmiṇi kumbhakā\-raghaṭayor api dharmayoḥ pratī\-tatvā\-t kā\-ryakā\-raṇabhā\-vo grahī\-tuṃ śakyata eva | prastute tu kā\-mā\-tur asantā\-navartino yuvatyā\-kā\-rasya dharmiṇas tatpragatā\-bhyā\-sasphuṭatvayor api dharmayoḥ parokṣatvā\-t | kathaṃ kā\-ryakā\-raṇagṛhī\-tiḥ | yathā\- ca naiyā\-yikaṃ prati yuṣmā\-bhir ucyate pratyakṣato na kā\-ryamā\-traṃ puruṣavyā\-ptaṃ sidhyati | kiṃ tv avā\-ntaram eva ghaṭajā\-tī\-yaṃ kā\-ryam iti tathā\- nā\-kā\-ramā\-tram abhyā\-sapū\-rvakaṃ sidhyati | kiṃ tv avā\-ntaram eva yuvatyā\-kā\-rasā\-mā\-nyam iti vyaktam eva | na cā\-bhyā\-sakā\-ryaḥ sphuṭī\-bhā\-vaḥ | tadabhā\-ve 'pi svapne darśanā\-t || 
	\pend
      

	  \pstart kiṃ ca sarvavido 'pi yadi caturā\-ryasatyaparijñā\-nataḥ sarvajñatā\-sthitiḥ, tarhi ghaṭā\-dikatipayavastujñā\-ne 'pi sarvajñeti sā\-dhvī\- śuddhiḥ | api ca 
	\pend
      

	  \pstart jñā\-navā\-n mṛgyate kaścit taduktapratipattaye | 
	\pend
      
	    
	    \stanza[\smallbreak]
ajñopadeśakaraṇe vipralambhanaśaṅkibhiḥ ||\footnote{\begin{english}(PV II 30)\end{english}}\&[\smallbreak]


	

	  \pstart iti yuṣmā\-bhir evocyate | na ca sarvajñā\-navā\-n viśeṣaniṣṭhatayā\-dhigantuṃ śakyate | na cā\-sya sattā\-mā\-trasiddhau kaścid upayogaḥ, pravṛtter anaṅgatvā\-d iti sarvam asamañjasam || 
	\pend
      

	  \pstart atrocyate | na vayaṃ sā\-kṣā\-tsarvajñasattā\-pratijñā\-yā\-ṃ hetuvyā\-pā\-ram anumanyā\-mahe | bhū\-dharā\-dhī\-navahnisattā\-vat | kiṃ tu caturā\-ryasatyā\-kā\-rasvarū\-pe dharmiṇi sphuṭā\-bhatvasya sā\-dhyasyā\-yogavyavacchedā\-rthaṃ parvate 'gnimā\-trā\-yogavyavacchedavat | sphuṭā\-bhatvaṃ tu kā\-miny ā\-kā\-rā\-didṛṣṭā\-nte dṛṣṭam eva | tac ca parvatī\-yā\-gnivat | pakṣadharmatā\-balataḥ satyacatuṣṭayā\-dhikaraṇaṃ sidhyat sarvajñatā\-m ā\-cakṣmahe | yathoktam 
	\pend
      
	    
	    \stanza[\smallbreak]
ityabhyā\-sabalā\-t parisphuṭadaśā\-koṭiḥ sphurat sambhavī\- heyā\-deyatadaṅgalakṣaṇaguṇaḥ sarvajñatā\- saiva naḥ || \&[\smallbreak]


	

	  \pstart iti | 
	\pend
      

	  \pstart tad atrā\-bhyā\-sasahitacaturā\-ryasatyā\-kā\-raḥ samagro dharmī\- sā\-magryam abhyā\-saviśiṣṭacetoguṇatvamā\-traṃ hetuḥ sphuṭī\-bhā\-vayogyatā\-sā\-dhyam | yathā\- sā\-gnitvā\-nagnitvasandehe parvatā\-tmā\- pramā\-ṇapratī\-to dharmī\- | tathā\-trā\-pi sarvajñatvā\-sarvajñatvavivā\-de 'pi pratyā\-tmaviditaḥ satyacatuṣṭayā\-kā\-ro dharmī\- | tasmā\-t sphuṭā\-bhatvena sā\-dhyena dṛṣṭā\-nte vyā\-ptisiddher asty eva tatpratibaddhaliṅganiścayaḥ | sā\-dhyasandehe 'pi dharmiṇaś caturā\-ryasatyā\-kā\-rasya siddher na trividhadoṣajā\-ter avasaraḥ | yogyatā\-yā\-ḥ prasā\-dhanena ca kā\-raṇā\-t kā\-ryapratī\-tā\-v anaikā\-ntikatvam ity apy anabhyupagamapratihatam | yogyatā\- ca sā\-ntaraiva sā\-dhyate | iyaṃ ca na gamayatu nā\-maikā\-ntataḥ kā\-ryasattvam | anupapadyamā\-naṃ punar asya sambhavam ā\-kṣipaty eva | tadā\- bhā\-vini kā\-rye sandehe 'pi kā\-raṇayogyatā\- niścī\-yata eva | brī\-hyā\-dau bhā\-viphalā\-niścaye 'pi yogyatā\-niścayena pravṛtteḥ | anyathā\- śilā\-śakalā\-der apy upā\-dā\-naprasaṅgaḥ | 
	\pend
      

	  \pstart tajjā\-tī\-yasya śarā\-vasthapaṅkoptasya sā\-marthyam upalabdham iti cet | atrā\-pi kā\-miny ā\-kā\-re bhā\-vanā\-jā\-tī\-yasya sphuṭī\-bhā\-vakaraṇayogyatā\- dṛṣṭeti samā\-nam | 
	\pend
      

	  \pstart evaṃ yogyatā\-mā\-trasā\-dhanenaiva kṛtā\-rthaḥ sā\-dhanavā\-dī\- | sarvajñakā\-raṇabhā\-vā\-t tadabhā\-vavā\-dinā\-ṃ nirdalanā\-t | kā\-ryasya ca traikā\-likasya sambhā\-vanā\-prasā\-dhanā\-t | muttkyarthinā\-ṃ ca pravṛtter avirodhā\-t | vā\-dino 'pi tanmā\-trasā\-dhanasyā\-bhipretatvā\-t | ata eva kaḥ prastā\-vaḥ sarvajñasattā\-vivā\-de sphuṭī\-bhā\-vasā\-dhanasyetyā\-dy apy anavakā\-śam | sarvajñaśabdena sphuṭī\-bhā\-vayogyatā\-yā\- vivakṣitatvā\-t | tathā\- kā\-ryakā\-raṇapratī\-tir api sambhavaty eva | tathā\- hi kā\-miny abhyā\-sasantatisahacā\-ri sambhramkā\-ryavacodarśanam eva kā\-miny ā\-kā\-rasya tadbhā\-vanā\-yā\-ś ca darśanam | tathā\-bhū\-takā\-yavaco 'darśanam eva bhā\-vanā\-yā\- adarśanam | evaṃ sphuṭapratibhā\-sasantatisahacā\-riviśiṣṭakā\-yavacodarśanaṃ sphuṭapratibhā\-sadarśanam | tathā\-vasthitakā\-yavaco 'darśanam eva sphuṭapratibhā\-sā\-darśanam ity asaty eva prastute 'pi pratyakṣā\-nupalambhataḥ kā\-ryakā\-raṇabhā\-vapratī\-tiḥ | iyaṃ ca tathā\-vasthakā\-mā\-tur aśarī\-ravacanagrahaṇe tadekadeśabhū\-tayuvatyā\-kā\-rā\-bhyā\-sasphuṭapratibhā\-sagrahaṇavyavasthā\- vyā\-vahā\-rikeṇā\-vaśyaṃ svī\-kartavyā\- | anyathā\- \edtext{}{\lemma{---}\Afootnote{citta \cite{}; citya \cite{}}}caityarū\-parasagandhasparśaparamā\-ṇupuñjā\-dyā\-tmakasya kumbhakā\-raghaṭapradeśā\-der api rū\-paikadeśagrā\-hakaṃ cakṣuḥpratyakṣaṃ na samudā\-yavyavasthā\-pakam iti sarvavyā\-vahā\-rikapramā\-ṇocchedaprasaṅgaḥ | tathā\- bā\-hyaghaṭakā\-m ityā\-dī\-nā\-ṃ śaktikṛtasya mahato jā\-tibhedasya sambhavā\-d anyajā\-tī\-yavyā\-ptigrahe 'nyajā\-tī\-yā\-d buddhimadanumā\-nam ayuktam | saṃkalpā\-rū\-ḍhā\-nā\-ṃ tu jalajvalanayuvatyā\-kā\-rā\-dī\-nā\-ṃ bā\-hyatvenā\-dhyastā\-nā\-m api vijñā\-naikasvarū\-patayaikajā\-tī\-yatvam astī\-ti bhā\-vanā\-sahitā\-kā\-ramā\-treṇaiva vaiśadyavyā\-ptir astu ||
	\pend
      

	  \pstart na ca svapne sphuṭatā\-vyabhicā\-raḥ | bhā\-vanā\-siddhalakṣaṇayor hetvor jā\-tibhede tatkā\-ryayor ekatvā\-bhimā\-ne 'pi jā\-tibhedasyā\-vaśyaṃ svī\-kartavyatvā\-t | dṛśyate hi siddhasā\-dhyā\- vaiśadyajā\-tir anapekṣya viparī\-tabhā\-vanā\-ṃ nidrā\-vicchede vicchidyamā\-nā\- | bhā\-vanā\-bhā\-vinī\- tu na vinā\- vipakṣā\-bhyā\-saṃ jā\-grato 'pi | yad ā\-huḥ 
	\pend
      
	    
	    \stanza[\smallbreak]
svapne 'pi sphuṭatā\- tathaiva na tathā\-py ekatvam evā\-nayor na prā\-kā\-rasamatvam eva samatā\-ṃ jā\-teḥ samā\-maṅgati |&anyanniddhanirodhabā\-dhyam itaradbā\-dhyaṃ pratyatnaiḥ punar vaiśadyaṃ viparī\-tabhā\-vanabalā\-n nairghṛṇyabhede yathā\- ||\&[\smallbreak]


	

	  \pstart iti ||
	\pend
      

	  \pstart yad api ghaṭā\-dikatipayajñā\-ne 'pi sarvajñaḥ syā\-d ity uktam | tatrā\-pi 
	\pend
      
	    
	    \stanza[\smallbreak]
ghaṭā\-diprakṛtā\-śeṣavedane 'pi bhayaṃ bhavā\-d dheyata yadi ko doṣaḥ so 'pi sarvajñatā\-ṃ vrajet |&saṃsā\-raduḥkhamokṣā\-ya spṛhayanto vayaṃ punar bhajema tadupā\-yajñaṃ sthā\-tuṃ tadgī\-tavartamani ||\&[\smallbreak]


	

	  \pstart ity uttaraṃ draṣṭavyam | tathā\- sattā\-mā\-tre vipratipannā\-n prati sattaiva kevalā\- prasā\-dhitā\- | viśeṣajijñā\-sā\-yā\-ṃ tu pramā\-ṇopapannakṣaṇikanairā\-tmyavā\-dina eva sugatasya bhagavataḥ sarvajñatā\- | ata etad api nirastaṃ yad ā\-ha Bhaṭṭaḥ 
	\pend
      
	    
	    \stanza[\smallbreak]
\edlabel{rna-ts-3149}\flagstanza{\tiny\textenglish{...s-3149}}sugato yadi sarvajñaḥ kapilo neti kā\- pramā\- |&athobhā\-v api sarvajñau matabhedaḥ kathaṃ tayoḥ || iti | \footnote{\begin{english}(=TS 3149)\end{english}}\&[\smallbreak]


	

	  \pstart tasmā\-t 
	\pend
      
	    
	    \stanza[\smallbreak]
uktakrameṇa munirā\-janaye pramā\-yā\-ḥ śaktir vyanakti gatim apramitā\-ṃ kṛpā\-ṃ ca | &anyatra tu dvayam udastam ado 'stamā\-ne tenaika eva śaraṇaṃ sa nirā\-tmavā\-dī\- || \&[\smallbreak]


	

	  \pstart iti viśeṣasiddhir apy anavadeyeti sarvam anā\-kulam ā\-kulā\-dhayaḥ pare na pratipadyante | sā\-dhane 'sminn avadye 'pi durnī\-tidahanadagdhabuddhayaḥ punar apy etad ā\-cakaṣate | bā\-dhakapramā\-ṇasadbhā\-vā\-t sarvajñasyā\-sadvyavahā\-ro yuktaḥ sadvyavahā\-rapratiṣedho vā\- prasā\-dhakapramā\-ṇā\-bhā\-vā\-d veti || 
	\pend
      

	  \pstart atra vicā\-ryate kiṃ punar asya bhagavato bā\-dhakaṃ pramā\-ṇaṃ pratyakṣam anumā\-naṃ śabdā\-dikaṃ veti vikalpā\-ḥ || 
	\pend
      

	  \pstart na tā\-vat pratyakṣaṃ | pratyakṣaṃ hi kevalapradeśā\-dau pravartamā\-naṃ svapravṛttiyogyam eva tatra vastu pratiṣedhati | na vastumā\-tram | na ca sarvajñasya pratyakṣapravṛttiyogyatā\-sti | svabhā\-vaviprakṛṣṭatvā\-t tasya || 
	\pend
      

	  \pstart syā\-d etat | na vayaṃ pratyakṣaṃ pravartamā\-nam abhā\-vaṃ sā\-dhayatī\-ti brū\-maḥ | kiṃ tarhi | nivartamā\-nam | tathā\- hi yatra vastuni pratyakṣasya nivṛttis tasyā\-sadbhā\-vaḥ | yathā\- śaśaviṣā\-ṇā\-deḥ | yatra tu pratyakṣasya pravṛttis tasya sadbhā\-vo yathā\- ghaṭā\-deḥ | asti ca sarvajñe pratyakṣanivṛttiḥ | tad asyā\-py abhā\-vaḥ kena nivā\-ryata iti || 
	\pend
      

	  \pstart ucyate | nivartamā\-naṃ pratyakṣam abhā\-vaṃ sā\-dhyatī\-ti ko 'rthaḥ | kiṃ pratyakṣasya yā\- nivṛttis tato 'bhā\-vasiddhiḥ, nivṛttisahitā\-d vā\- pratyakṣā\-t, nivṛttā\-d vā\- pratyakṣā\-d iti | 
	\pend
      

	  \pstart nā\-dyaḥ pakṣaḥ | saty api vastuni pratyakṣanivṛtter upalabhyamā\-nā\-yā\- vastvabhā\-vaniyatatvā\-siddheḥ || 
	\pend
      

	  \pstart nā\-pi dvitī\-yaḥ | svā\-bhā\-vena saha kasyacit sā\-hityā\-nupapatteḥ | anyathā\- tannivṛttatvā\-nupapatteḥ || 
	\pend
      

	  \pstart na ca tṛtī\-yaḥ | tathā\- hi nivṛttā\-t pratyakṣā\-d abhā\-vasiddhir ity asataḥ pratyakṣā\-d ity uktaṃ bhavati | na cā\-sato hetubhā\-vaḥ sambhavati | sarvasamarthyavirahalakṣaṇtvā\-t tasya | na hi tac ca nā\-sti tena ca pratipattir iti nyā\-yam | ato na tā\-vat pratyakṣaṃ sarvajñabā\-dhakam || 
	\pend
      

	  \pstart nā\-py anumā\-nam | tad dhi trividhaliṅgajatvena trividham | tatra kā\-ryasvabhā\-vayor vidhisā\-dhanatvā\-t, pratiṣedhe sā\-dhye 'navasaraḥ | na ca dṛśyā\-nupalambhaḥ tatprabhedo vā\- kā\-ryā\-nupalabdhyā\-dir yogyā\-nupalambho vā\- parā\-bhimato 'tra pramā\-ṇam | sarvajñatā\-yā\-ḥ svabhā\-vaviprakṛṣṭatvenā\-dṛśyatvā\-t || 
	\pend
      

	  \pstart nanu kā\-raṇā\-nupalambhā\-d eva sarvajñatā\-pratiṣedhaḥ sidhyati | tathā\- hi tatkā\-raṇam indriyavijñā\-naṃ vā\- mā\-nasaṃ vā\- bhā\-vanā\-balajaṃ vā\- | bhā\-vanā\-balajam api cā\-kṣuṣaṃ vā\-, mā\-nasaṃ veti vikalpā\-ḥ | 
	\pend
      

	  \pstart tatra na tā\-vac cakṣurindriyavijñā\-nam aśeṣā\-rthagrā\-hi | tasya pratiniyatā\-rthaviṣayatvā\-t | deśā\-ntare kā\-lā\-ntare ca tathaiva pratiniyamaḥ | anyathā\- hetuphalabhā\-vā\-bhā\-vaprasaṅgā\-t | anekendriyavaiyarthyaprasaṅgā\-c ca | tathā\- ca kā\-rikā\- 
	\pend
      

	  \pstart ekendriyapramā\-ṇena sarvajño yena kalpyate | 
	\pend
      

	  \pstart nū\-naṃ sa cakṣuṣā\- sarvā\-n rasā\-dī\-n pratipadyate || 
	\pend
      

	  \pstart yajjā\-tī\-yaiḥ pramā\-ṇaiś ca yajjā\-tī\-yā\-rthadarśanam | 
	\pend
      

	  \pstart bhaved idā\-nī\-ṃ lokasya tathā\- kā\-lā\-ntare 'py abhū\-t || iti | \footnote{\begin{english}(ŚV II 112-113; =TS 3158-3159)\end{english}}
	\pend
      

	  \pstart tataś caivaṃ prayogaḥ kartavyaḥ | buddhacakṣurnā\-tī\-tā\-diviṣayam | cakṣustvā\-t | asmadā\-dicakṣurvat | acakṣur vā\- | 
	\pend
      

	  \pstart atī\-tā\-diviṣayatvā\-t | śabdavat | iti sarvam etat śrotrā\-dā\-v api draṣṭavyam | na cakṣurā\-diprakarṣaḥ svā\-rtham atikramya dṛṣṭaḥ | \name{Kā\-rikā\-}
	\pend
      
	    
	    \stanza[\smallbreak]
yatrā\-py atiśayo dṛṣṭaḥ sa svā\-rthā\-natilaṅghanā\-t | &dū\-rasū\-kṣmā\-divṛttau syā\-n na rū\-pe śrotravṛttitaḥ || \footnote{\begin{english}(ŚV II 114)\end{english}}\&[\smallbreak]


	

	  \pstart \name{Bṛhaṭṭī\-kā\-} ca
	\pend
      
	    
	    \stanza[\smallbreak]
śrotragamyeṣu śabdeṣu dū\-rasū\-kṣmopalabdhitaḥ | &puruṣā\-tiśayo dṛṣṭo na rū\-pā\-dyupalambhanā\-t || &cakṣuṣā\-pi ca dū\-rasthasū\-kṣmarū\-popalambhanam | &kriyate 'tiśayaprā\-ptyā\- na tu śabdā\-didarśanam || \footnote{\begin{english}(=TS 3162-63)\end{english}}\&[\smallbreak]


	

	  \pstart na caitad vaktavyam | yadi nā\-maikaikenendriyeṇa tajjñā\-nena vā\- sarvasyā\-grhaṇaṃ tathā\-pi pañcabhir indriyais tajjñā\-nair vā\- svasvaviṣayapravṛttair evā\-tiśayaprā\-ptair bhaviṣyatī\-ti | ekaikasyā\-pi niḥśeṣasvaviṣayagrahaṇā\-darśanā\-t | paracittā\-dyatī\-ndriyā\-ṇā\-ṃ grahaṇā\-bhā\-vā\-c ca | tad evam indriyavijñā\-naṃ vā\- nā\-śeṣagrā\-hī\-ti na prathamaḥ pakṣaḥ || 
	\pend
      

	  \pstart nā\-pi dvitī\-yaḥ | tathā\- hi yady api tanmā\-nasaṃ sarvā\-rthaviṣayaṃ tathā\-pi na tasya svā\-tantryeṇā\-rthagrahaṇe vyā\-pā\-ro 'sti | manaso bahirasvā\-tantryā\-t | anyathā\-ndhavadhirā\-dyabhā\-vaprasaṅgaḥ | teṣā\-m api manaso bhā\-vā\-t | pā\-ratantrye cetndriyajñā\-naparigṛhī\-tā\-rthaviṣayatvā\-d atī\-tā\-nā\-gatadū\-rasū\-kṣmavyavahitaparacittā\-der arthasyendriyaparijñā\-nā\-gocarasya manasā\- paricchedo na prā\-pnotī\-ti kathaṃ sarvajñatā\- || 
	\pend
      

	  \pstart na ca bhā\-vanā\-balajaṃ sarvā\-rthagrā\-hī\-ti tṛtī\-yaḥ pakṣaḥ | tathā\- hi tadbhā\-vanā\-balajam api yadī\-ndriyā\-śritam iti caturthaḥ pakṣaḥ, tadā\- so 'saṅgataḥ | indriyasya tajjñā\-nasya ca niyataviṣayaviṣayatvapratipā\-danā\-t || 
	\pend
      

	  \pstart atha bhā\-vanā\-balena tathā\-vidham utpannaṃ manovijñā\-naṃ sarvā\-rthagrā\-hī\-ti pañcamaḥ pakṣaḥ | tadā\-nvarthatvā\-t pratyakṣaśabdasya tasya ca bhā\-vanā\-balā\-valambino 'py anakṣajatvā\-t nā\-rthasā\-kṣā\-tkā\-ritvam astī\-ti pratipā\-danī\-yam | kiṃ ca svaviṣayasī\-mā\-nam anatipatyaiva prakarṣo 'pi dṛśyate | na tu sarvaviṣayatveneti | kathaṃ tenā\-pi sakalā\-rthajā\-tā\-divedanam | yato na kasyacid abhyā\-se 'py atī\-ndriyā\-rthadarśitvam upalabdham || 
	\pend
      

	  \pstart Bṛhaṭṭī\-kā\- 
	\pend
      

	  \pstart ye 'pi sā\-tiśayā\- dṛṣṭā\-ḥ prajñā\-medhā\-balair narā\-ḥ | 
	\pend
      

	  \pstart stokastokā\-ntaratvena na te 'tī\-ndriyadarśanā\-ḥ || 
	\pend
      

	  \pstart prā\-jño 'pi ca naraḥ sū\-kṣmā\-n athā\-n draṣṭuṃ kṣamo 'pi san | 
	\pend
      

	  \pstart sajā\-tī\-r anatikrā\-man nā\-tiśete parā\-n api || \footnote{\begin{english}(=TS 3160-61)\end{english}}
	\pend
      

	  \pstart ekā\-vavarakasthasya pratyakṣaṃ yat pravartate | 
	\pend
      

	  \pstart śaktis tatraiva tasya syā\-n naivā\-vavarakā\-ntare || 
	\pend
      

	  \pstart ye cā\-rthā\- dū\-ravicchinnā\- deśaparvatasā\-garaiḥ | 
	\pend
      

	  \pstart varṣadvī\-pā\-ntarair ye ca kas tā\-n paśyed ihaiva san || \footnote{\begin{english}(=TS 3170-71)\end{english}}
	\pend
      

	  \pstart atra varṣaḥ kā\-laviśeṣaḥ | 
	\pend
      

	  \pstart evaṃ śā\-stravicā\-reṣu dṛśyate 'tiśayo mahā\-n | 
	\pend
      

	  \pstart na tu śā\-strā\-ntarajñā\-naṃ tanmā\-treṇaiva sidhyati || 
	\pend
      

	  \pstart jñā\-tvā\- vyā\-karaṇaṃ dū\-raṃ buddhiḥ śabdā\-paśabdayoḥ | 
	\pend
      

	  \pstart ā\-kṛṣyate na nakṣatratithigrahaṇanirṇaye || 
	\pend
      

	  \pstart jyotirvic ca prakṛṣṭo 'pi candrā\-rkagrahaṇā\-diṣu | 
	\pend
      

	  \pstart na bhavatyā\-diśabdā\-nā\-ṃ sā\-dhutvaṃ jñā\-tum arhati || 
	\pend
      

	  \pstart tathā\- vedetihā\-sā\-dijñā\-nā\-tiśayavā\-n api | 
	\pend
      

	  \pstart na svargadevatā\-pū\-rvapratyakṣī\-karaṇe kṣamaḥ || 
	\pend
      

	  \pstart daśahastā\-ntaraṃ vyomno ye nā\-motplutya gacchati | 
	\pend
      

	  \pstart na yojanam asau gantuṃ śakto 'bhyā\-saśatair api | 
	\pend
      

	  \pstart tasmā\-d atiśayajñā\-nair atidū\-ragatair api | 
	\pend
      

	  \pstart kiñcid evā\-dhikaṃ jñā\-tuṃ śakyate na tv atī\-ndriyam || iti | \footnote{\begin{english}(=TS 3164-69)\end{english}}
	\pend
      

	  \pstart pratyakṣasū\-tre tu kā\-śikā\-kā\-raḥ paramatam ā\-śaṅkyā\-ha, tan na, avagataviṣayatvā\-d bhā\-vanā\-yā\-ḥ | na cā\-kasmā\-d avagater utpattiḥ sambhavati | sarvotpattimatā\-ṃ kā\-raṇavattvā\-t | atha pramā\-ṇā\-ntarā\-vagataṃ bhā\-vyate | kiṃ bhā\-vanayā\- | tata eva tatsiddheḥ | kiṃ ca tatpramā\-ṇam | na tā\-vad anumā\-naṃ dharmā\-dharmayoḥ pū\-rvam agrahaṇena tadvyā\-ptaliṅgasaṃvedanā\-sambhavā\-t | jagadvaividhyā\-rthā\-patter api hi kim api kā\-raṇam astī\-ti etā\-vad unnī\-yate | na tu kaścid viśeṣaḥ | na cā\-nirdiṣṭaviśeṣaviṣayā\- bhā\-vanā\- bhavati | yogaśā\-streṣv api hi viśeṣā\- eva dhyeyatayopadiśyante | 
	\pend
      

	  \pstart dhyeya ā\-tmā\- prabhuryo 'sau hṛdi dī\-pa iva sthitaḥ | (Maitrī\- Up. 6,30) 
	\pend
      

	  \pstart ityā\-dibhiḥ | ā\-gamamā\-nā\-t tarhi avagataṃ bhā\-vayiṣyate | yadi pramā\-ṇā\-t tadā\- tata evā\-vagateḥ | kiṃ bhā\-vanayā\- | hā\-nopā\-dā\-nā\-rthaṃ hi vastu jijñā\-syate | te ca tata eva siddhe iti vyarthā\- bhā\-vanā\- | kā\-ruṇiko 'pi hi dharmā\-gamā\-n eva śiṣyebhyo vyā\-cakṣī\-ta | na bhā\-vanā\-bhedam anubhavet | 
	\pend
      

	  \pstart atha vipralambhabhū\-yiṣṭhatvā\-d ā\-gā\-mā\-nā\-ṃ pramā\-ṇam ā\-gamo na veti vicikitsamā\-no bhā\-vanayā\- jijñā\-sate | tan na | tato 'pi tadasiddheḥ | bhā\-vanā\-balapriniṣpannam api jñā\-nam anā\-śvsanī\-yā\-rtham eva | abhū\-tasyā\-pi bhā\-vyamā\-nasyā\-parokṣā\-rthavat prakā\-śanā\-t | yathā\- hi tair evoktam 
	\pend
      
	    
	    \stanza[\smallbreak]
tasmā\-d bhū\-tam abhū\-taṃ vā\- yad yad evā\-bhibhā\-vyate |&bhā\-vanā\-pariniṣpattau tat sphuṭā\- kalpadhī\-ḥ phalam ||\footnote{\begin{english}PV III 285; PVin I 30.\end{english}}\&[\smallbreak]


	

	  \pstart api ca bhā\-vanā\-balajam apramā\-ṇam | gṛhī\-tagrahaṇā\-t | yā\-vad eva hi gṛhī\-taṃ tā\-vad eva bhā\-vanayā\- viṣayī\-kriyate | mā\-trayā\-py adhikaṃ na bhā\-vanā\- gocarayati | yogā\-bhyā\-sā\-hitasaṃskā\-rapā\-ṭavanimittā\- hi smṛtir eva bhā\-vaneti gī\-yate | sā\- ca pramā\-ṇam iti sthitam eva | na ca taduttarakā\-laṃ sā\-kṣā\-tkā\-rijñā\-nam udetī\-ti pramā\-ṇam asti | indriyasannikarṣam antareṇā\-rthasā\-kṣā\-tkā\-rasya kvacid adarśanā\-t | yoginā\-ṃ dharmā\-dharmayor aparokṣapratibhā\-saṃ jñā\-naṃ nā\-sti, indriyasannikarṣā\-bhā\-vā\-d asmadā\-divat || 
	\pend
      

	  \pstart Vā\-capatis tu Kaṇikā\-yā\-m ā\-ha | satyaṃ śrutā\-numā\-nagocaracā\-riṇī\- bhā\-vanā\- viśadā\-bhajñā\-nahetur iti nā\-vajā\-nī\-mahe | kin tu yadviṣayajā\-taṃ tad eva viśadapratipattigocaraḥ | na jā\-tu rū\-pabhā\-vanā\-prakarṣo rasaviṣayavijñā\-navaiśadyā\-ya kalpate | 
	\pend
      

	  \pstart nanu na viṣayā\-ntaravaiśadyahetubhā\-vaṃ bhā\-vanā\-yā\-ḥ saṅgirā\-mahe | kintu śrutā\-numā\-naviṣayavaiśadyahetutā\-m eva | tadviṣayaś ca samastavastunairā\-tmyam iti tadbhā\-vanā\-prakarṣaḥ samastavastunairā\-tmyaṃ viśadayan samastavastuviśadatā\-m antareṇa tadupapatteḥ samastavastuvaiśadyam ā\-vahatī\-ty uktam | 
	\pend
      

	  \pstart satyam uktam | ayuktaṃ tu tat | tathā\- hi nā\-gamā\-numā\-nagocaratvaṃ nirā\-tmanā\-ṃ vastubhedā\-nā\-ṃ paramā\-rthasatā\-m | na hi te eteṣā\-m anyanivṛttimā\-trā\-vagā\-hinī\- paramā\-rthasatsvalakṣaṇaṃ gocarayitum arhataḥ | nā\-pi tadviṣayā\- bhā\-vanā\- | tadagrā\-hyam api svalakṣaṇaṃ tadadhyavaseyatayā\- tadviṣaya iti tadyonir api bhā\-vanā\- tadviṣayeti tatprakarṣas tadvaiśadyahetur iti cet | na | tadadhyavaseyasyā\-pi paramā\-rthasattvā\-bhā\-vā\-t | tathā\- hi yad anumā\-nena gṛhyate yac cā\-dhyavasī\-yate te dve apy anyanivṛttī\-, na vastunī\- |\edlabel{note-2objects-neither-real}\footnote{\label{note-2objects-neither-real}  \begin{english}Cf. also \cref{Frauwallner37}, \cref{buehnemann80}, \cref{mccrea_patil06}.\end{english}} svalakṣaṇā\-vagā\-hitve 'bhilā\-pasaṃsargayogyapratibhā\-sā\-nupapatteḥ ||
	\pend
      

	  \pstart mā\- bhū\-t tayoḥ svalakṣaṇaṃ viṣayaḥ | tatprabhavabhā\-vanā\-prakarṣaparyantajanmanas tu viśadā\-bhasya cetaso bhaviṣyati | kā\-mī\-nī\-vikalpaprabhavabhā\-vanā\-prakarṣā\-d iva kā\-mā\-tur asya kā\-minī\-svalakṣaṇasā\-kṣā\-tkā\-raḥ | karikumbhakaṭhorakucakalaśahā\-riṇi hariṇaśā\-valolalocane campakadalā\-vadā\-tagā\-tralate lā\-vaṇyasarasi nirantaralagnalalitadoḥkandalī\-mū\-lamā\-liṅganam aṅgane preyasitare prayaccha | sañjī\-vaya jī\-viteṣvari, patito 'smi tava caraṇanalinayor iti vacanakā\-yaceṣṭayor upalabdheḥ | asti ca vikalpā\-vikalpayoḥ kathañcit samā\-naviṣayateti nā\-tiprasaṅga iti cet | satyam | sambhavaty ayam anubhavo na punar asyā\-rthe prā\-mā\-ṇyasambhavaḥ | atadutpatter atadā\-tmanas tadavyabhicā\-raniyamā\-yogā\-t | atā\-dā\-tmyaṃ cā\-rthasya vijñā\-nā\-d atirekā\-t | anatireke 'pi ca vijñā\-nā\-nā\-m anyonyasya bhedā\-d atā\-dā\-tmyā\-t | ekasya vijñā\-nasyetaravijñā\-navedanā\-nupapatteḥ | vijñā\-nasvalakṣaṇaikatvā\-bhyupagame ca tannityam ekam advitī\-yaṃ brahmā\-bhyasanī\-yam iti kṣaṇikanairā\-tmyā\-bhyā\-sā\-bhyupagamo dattajalā\-ñjaliḥ prasajyeta | tan na tā\-dā\-tmyā\-t tasyā\-vyabhicā\-raḥ | nā\-pi tatkā\-ryatvā\-t | bhā\-vanā\-prakarṣakā\-ryaṃ khalv evan na viṣayakā\-ryam | yady ucyeta pā\-ramparyeṇa tatkā\-ryam anumā\-navat | yathā\- hi vahnisvalakṣaṇā\-d dhū\-masvalakṣaṇam | tato dhū\-mā\-nubhavas tato dahanavikalpaḥ, tataś cā\-numā\-nam utpannam iti pā\-ramparyeṇa vahnipratibandhā\-t prā\-pakaṃ ca vahner dā\-hapā\-kakā\-riṇaḥ tathedam api anumā\-najanitabhā\-vanā\-prakarṣaparyantajaṃ pā\-ramparyeṇā\-rthaprasū\-tatayā\- tadavyabhicā\-raniyamā\-t tatra pramā\-ṇam iti | tat kim anumā\-nena vahniṃ vyavasthā\-pya bhā\-vayato yad vahniviṣayamativiśadavijñā\-naṃ tat pramā\-ṇam iti | om iti brubā\-ṇasya parvatanitambā\-rohaṇe satī\-ndriyasannikarṣajanmano dahanavijñā\-nasya bhā\-vanā\-dhipatyaviśadā\-bhavijñā\-nena saha saṃvā\-daniyamaprasaṅgaḥ | visaṃvā\-daś ca bahulam upalabhyate | lakṣaṇayogini ca vyabhicā\-rasambhave tallakṣaṇam eva bā\-dhitam iti viśadā\-bham api prā\-tibham iva saṃśayā\-krā\-ntam apramā\-ṇam | tadbhā\-vanā\-yā\- bhū\-tā\-rthatvaṃ na tajjaviśadā\-bhavijñā\-naprā\-mā\-ṇyahetuḥ, vyabhicā\-rā\-t | etañ ca prā\-sarpakasyeva saktukarkarī\-prā\-ptimū\-lalā\-bhamanorathaparamparā\-hito draviṇasambhā\-rasā\-kṣā\-tkā\-ras tathā\-gatasya nirā\-tmakasamastavastusā\-kṣā\-tkā\-ra ity ā\-patitam | sarvā\-rthavastubhā\-vanā\-parikarmitacittasantā\-navartivijñā\-naṃ pratyā\-lambanapratyayatvam arthamā\-trasya | 
	\pend
      

	  \pstart tathā\- ca tadutpatteḥ tadavyabhicā\-raniyama iti cet | na | arthasya hy ā\-lambanapratyayatvavijñā\-naṃ pratī\-ndriyā\-pekṣatvena vyā\-ptam | tac cā\-smā\-t svaviruddhopalabdhyā\- vyā\-vartamā\-nam ā\-lambanapratyayatā\-m apy arthasya nivartayati | na khalv indhanaviśeṣo dhū\-mahetur iti vinā\-pi dahanaṃ sastreṇā\-pi saṃskā\-rair dhū\-mam ā\-dhatte | tadā\-dhā\-ne vā\- samastakā\-ryahetvanumā\-nocchedaprasaṅgaḥ | bhā\-vanā\-yā\-ś ca bhū\-tā\-rthā\-yā\- arthā\-napekṣā\-yā\- eva viśadavijñā\-najananasā\-marthyam upalabdhaṃ kā\-mā\-turā\-divartinyā\- iti bhū\-tā\-rthā\-pi tannirapekṣaiva samartheti nā\-rthasyā\-lambanapratyayatvaṃ śakyā\-vagamam | api ca ā\-lambanapratyayā\-pi ta evā\-sya kṣaṇā\- yujyante, ye tasya purastā\-t tanā\- avyavadhā\-nā\-s tathā\- ca ta evā\-sya grā\-hyā\- na punaḥ pū\-rvatarā\-ḥ | tatkā\-lā\- anā\-gatā\-ś ceti na sarvaviṣayatā\- | atha dṛśyamā\-nā\- dhā\-tutrayaparyā\-pannā\-ḥ prā\-ṇabhṛto janmā\-ntaraparivartopā\-ttā\-tī\-tā\-nā\-gataskandhakadambakopā\-dā\-nopā\-deyā\-tmā\-na iti taddarśanaṃ dṛśyamā\-natā\-dā\-tmyena tadviśeṣaṇatayā\-tī\-tā\-nā\-gatam api gocarayati | na cā\-smadā\-didarśanasyā\-pi tathā\-tvaprasaṅgaḥ, rā\-gā\-dimalā\-vṛtatvā\-t | tasya ca bhagavato nirmṛṣṭanikhilakleśopakleśamalaṃ vijñā\-namanā\-varaṇaṃ paritaḥ pradyotamā\-nam ā\-lambanapratyayaṃ sarvā\-kā\-raṃ gocarayet | tasya ca sā\-kṣā\-t paramparayā\- ca kathañcit sarveṇa sambandhā\-d deśakā\-laviprakī\-rṇavastumā\-traviśiṣṭasvabhā\-vatayā\- tathaiva gocarayet | na caitat sarvagrahaṇam antareṇeti sarvaviṣayam asya vijñā\-nam anā\-varaṇaṃ siddham | 
	\pend
      

	  \pstart tad anupapannam | vicā\-rā\-sahatvā\-t | tathā\- hī\-yam ā\-lambanapratyayasya sarvaviśiṣṭā\-tmatā\- bhā\-vikī\- na vā\- | bhā\-vikī\- cet | na tā\-vat sarvasminn ā\-lambanapratyaye caikā\- sambhavati | ekasyā\-nekavṛttitvā\-nupapatteḥ | nā\-nā\- cet | ā\-lambanapratyayā\-ś ca sarve ceti tattvam | tathā\- ca na sambandha iti na tadgrahaṇe sarvagrahaṇam | vikalpā\-ropitatayā\- tv avikalpakaṃ samastavastuviṣayaṃ sarvatra pratī\-yata iti subhā\-ṣitam | svā\-lambanapratyayamā\-tragocaram evā\-vikalpakaṃ samastavastuviśiṣṭā\-lambanā\-dhyavasā\-yajananam tenā\-dhyavasā\-yā\-nugatavyā\-pā\-ram avikalpakam api samastavastuviṣayaṃ bhavati | yad ā\-ha 
	\pend
      
	    
	    \stanza[\smallbreak]
vyavasyantī\-kṣaṇā\-d eva sarvā\-kā\-rā\-n mahā\-dhiyaḥ | \footnote{\begin{english}(PV III 107)\end{english}}\&[\smallbreak]


	

	  \pstart iti cet | atha katipayavastvā\-lambanā\-nubhavasya kutastya eṣa mahimā\- yataḥ samastavastvavasā\-ya iti | rā\-gā\-dyā\-varaṇavigamā\-d iti cet | tarhi yathā\-vad vastū\-ni paśyet | na punar asmā\-d apā\-rthatvam asyeti | tad ayuktaṃ vikalpanirmā\-ṇakauśalam asya yujyeta | tattvā\-varakatā\- hi sulabhamalā\-nā\-ṃ kleṣā\-dī\-nā\-ṃ na punarvikalpanirmā\-ṇapratibandhatā\- | tasmā\-d bhā\-vanā\-prakarṣamā\-trajatvā\-t, arthā\-vyabhicā\-raniyamā\-bhā\-vā\-t, viśadā\-bham api saṃśayā\-krā\-ntatvā\-d apramā\-ṇam apratyakṣaṃ ceti sā\-mpratam || 
	\pend
      

	  \pstart yad api sadarthaprakā\-śanaṃ buddheḥ svabhā\-vo 'sadarthatvaṃ cā\-gantukam iti, asati bā\-dhake sadarthatvam eveti, tad ayuktam | anumitabhā\-vitavahniviṣayaviśadā\-bhajñā\-naprā\-mā\-ṇyaprasaṅgā\-t tadvidhasya kvacid bā\-dhadarśanā\-d aprā\-mā\-ṇyam ihā\-pi samā\-nam | anyatrā\-bhiniveśā\-t | tad iha yadi viśadā\-bhavijñā\-nahetutvaṃ bhā\-vanā\-yā\- viśeṣaṇatrayayogena sā\-dhyate, tataḥ siddhasā\-dhanam | bhavatu tathā\-gatas tathā\-bhū\-tavijñā\-navā\-n | na tv etad vijñā\-nam asya pratyakṣam apramā\-ṇatvā\-t | tathā\- cā\-pakṣadharmatayā\- hetor asiddhatā\- | prasiddhadharmaṇo dharmiṇo 'jijñā\-sitaviśeṣatayā\- anumeyatvā\-bhā\-vā\-t | atha pratyakṣavijñā\-nahetutā\- bhā\-vanā\-yā\-ḥ paraṃ pratyasiddhā\- sā\-dhyate, tathā\- ca sati sā\-dhyaviparyayavyā\-pter viruddhatā\- hetoḥ, viśeṣaṇatrayavatyā\-pi bhā\-vanā\-yā\- viśadā\-bhabhrā\-ntavijñā\-najanakatvā\-t | dṛṣṭā\-ntasya ca sā\-dhyahī\-natvā\-t | yadā\- ca bhū\-tā\-rthabhā\-vanā\-janitatve 'pi nā\-sya prā\-mā\-ṇyam abhū\-tā\-rthatvā\-t, tadā\- yad ucyate, 
	\pend
      
	    
	    \stanza[\smallbreak]
nirupadravabhū\-tā\-rthasvabhā\-vasya viparyayaiḥ | &na bā\-dhā\- yatnavattve 'pi buddhes tatpakṣapā\-tataḥ || \footnote{\begin{english}(PV I 223; II 210)\end{english}}\&[\smallbreak]


	

	  \pstart iti | tad anupapannam | bhū\-tā\-rthatve 'pi hi buddheḥ tatpakṣapā\-titā\- bhū\-tā\-rthaiḥ pratipakṣair bā\-dho na bhavet | abhū\-tā\-rthā\- tv iyaṃ sā\-tmī\-bhā\-vam ā\-pannā\-py ā\-tmā\-tmī\-yadṛṣṭir iva sambhavadbā\-dhā\- | tasmā\-t pratipakṣavivṛddhimā\-tram | na tv ā\-tyantikī\- vivṛddhiḥ sambhavati | yayā\- samū\-lakā\-ṣaṃ kaṣitā\- doṣā\- na punar udbhaviṣyanti | ata evā\-sthirā\-śrayatve 'pi apunaryatnā\-pekṣatve 'pi asya nā\-tyantikī\- niṣṭhā\- sambhavati | ā\-tmā\-tmī\-yadṛśa iva virodhipratyayasambhavā\-t | tatsambhavaś cā\-bhū\-tā\-rthatvā\-t | śrutā\-numitaviṣayaṃ tu pratyakṣaṃ na sambhavaty eva | tayoḥ parokṣarū\-pā\-vagā\-hitvā\-t | pratyakṣasya ca tadviparī\-tatvā\-t | tadgatabhū\-tā\-bhū\-tā\-rthā\-nuvidhā\-yitvena svaviṣaye śrutā\-numā\-najñā\-nā\-pekṣayā\- prā\-mā\-ṇyā\-nupapatteś ca || 
	\pend
      

	  \pstart tat siddham etat bhū\-tā\-rthabhā\-vanā\-prakarṣaparyantajavijñā\-nam apratyakṣam arthe 'prā\-mā\-ṇyā\-t | yad apramā\-ṇaṃ tad apratyakṣam arthe | yathā\- kā\-mā\-tur asya kā\-minī\-vijñā\-nam | apramā\-ṇaṃ ca tat | nitā\-ntaviśadā\-bhatve sati bhā\-vanā\-prakarṣajatvā\-t | yan nitā\-ntaviśadā\-bhatve sati bhā\-vanā\-prakarṣajaṃ vijñā\-naṃ tad apramā\-ṇam | 
	\pend
      

	  \pstart yathā\-numitabhā\-vitavahniviśadavijñā\-nam iti | samā\-nahetujatvaṃ samā\-narū\-patayā\- vyā\-ptam | yad ā\-ha 
	\pend
      
	    
	    \stanza[\smallbreak]
tadatadrū\-piṇo bhā\-vā\-s tadatadrū\-pahetujā\-ḥ \footnote{\begin{english}(PV III 251ab)\end{english}}\&[\smallbreak]


	

	  \pstart iti | tad asya prā\-mā\-ṇyaṃ nivartamā\-naṃ tulyahetujatvam api nivartayati | na caiṣa bhū\-tā\-rthabhā\-vanā\-prakarṣaparyantajo 'nindriyasannikṛṣṭā\-numitabhā\-vitavahnivaiśadye ca nirā\-tmakasamastavastuvaiśadye ca viśiṣyate | na ca rā\-gā\-dyā\-varaṇaviraho viśeṣaḥ | na khalv ete kambalā\-divad ā\-varakā\- vijñā\-nasya | kiṃ tu tadā\-kṣiptamanā\- vividhaviṣayabhedatṛṣṇā\-diparipluto na śaknoti bhā\-vayitum iti bhā\-vanā\-daramā\-tra eva tadvirahopayogaḥ | asti cehā\-pi śiśirabharasambhṛtajaḍimamantharatarakā\-yakā\-ṇḍasyā\-numitavahnibhā\-vanā\-bhiyoga iti na hetubhedataḥ pratibandhasiddhiḥ | na caikapā\-rthivā\-ṇusamavā\-yikā\-raṇajanmabhir abhinnauṣṇyā\-pekṣaikavahnisaṃyogā\-samavā\-yikā\-raṇair gandharasarū\-pasparśair nā\-nā\-svabhā\-vair vyabhicā\-raḥ | sā\-marthyavaicitryā\-d ekatve 'pi pā\-rthivasya paramā\-ṇoḥ | tadvaicitryaṃ ca kā\-ryavaicitryopalambhā\-t | tac ca nityasamavetaṃ nityam, kā\-raṇasā\-marthyaprakrameṇa ca pā\-rthivā\-vayavini kā\-rye jā\-yata iti avadā\-tam | pariśiṣṭaṃ tu granthavyā\-khyā\-nasamaye vyā\-khyā\-syā\-maḥ | tadā\-stā\-ṃ tā\-vat || 
	\pend
      

	  \pstart \persName{trilocanas} tu \name{nyā\-yaprakī\-rṇake} prā\-ha | iha kila duḥkhasamudayanirodhamā\-rgā\-khyā\-nyā\-ryā\-ṇā\-ṃ satyā\-ni catvā\-ri | teṣā\-m satyā\-nā\-ṃ svarū\-pasā\-kṣā\-tkā\-rijñā\-naṃ yogipratyakṣaṃ | tatra duḥkhaṃ phalabhū\-tā\-ḥ pañcopā\-dā\-naskandhā\-ḥ | tac ca svarū\-pato jñā\-tavyam | ta eva hetubhū\-tā\-ḥ samudayaḥ | sa ca prahā\-tavyaḥ | niḥkleśā\-vasthā\- cittasya nirodhaḥ | sa ca sā\-kṣā\-tkartavayaḥ | tadavasthā\-prā\-ptihetur nairā\-tmyakṣaṇikatvā\-dyā\-kā\-raś cittaviśeṣo mā\-rgaḥ | sa ca bhā\-vayitavya iti saugatamatam |
	\pend
      

	  \pstart atrocyate | mā\-rgas tā\-vat pramā\-ṇapariśuddho na bhavatī\-ty uktaṃ prā\-k | ato 'bhū\-taviṣayasya vikalpasyā\-bhyā\-sā\-d asatyā\-rthavijñā\-naṃ syā\-n na saṃvā\-di | api ca pramā\-ṇapariśuddhamā\-rgavā\-dī\- śā\-kyaḥ pramā\-ṇaṃ pṛṣṭaḥ san sattvā\-khyaliṅgajaṃ vikalpaṃ brū\-yā\-t | tato yā\-vad vikalpena darśitarū\-paṃ tat sarvam asat | śabdasaṃsṛṣṭatvā\-t | tasmiṃs ca bhā\-vyamā\-ne sattve bhā\-vakasya vikalpakasya bhā\-vanopahite viśadā\-bhatve śabdasamsṛṣṭagrā\-hyanimittaṃ vikalpakatvaṃ nivartate | tadvyā\-vṛttau grā\-hyam api śabdasaṃsṛṣṭaṃ nivartate | ato nirvikalpakam api yogijñā\-naṃ nirviṣayaṃ prasaktam | yat tu pā\-ramā\-rthikaṃ vastvā\-tmakaṃ na tatpramā\-ṇapariśuddham | śuddhau vā\- bhā\-vanayā\- | bhā\-vyasya sā\-kṣā\-dvijñā\-tatvā\-t | na cā\-nyasmin śabdasaṃsṛṣṭe bhā\-vyamā\-ne sphuṭam anyad rū\-paṃ bhavati | śokā\-tur asyā\-pi niruddhendriyavyā\-pā\-rasya tanayabhā\-vanā\-yā\-ṃ mitrā\-dipratibhā\-saprasaṅgā\-t | 
	\pend
      

	  \pstart kṣaṇikatve bhā\-vye samā\-ropite vā\-stavaṃ kṣaṇikatvam eva yogivijñā\-napratibhā\-sī\-ti cet | na | satyā\-satyayor ekatvā\-bhā\-vā\-tmake hi bhede 'satyabhā\-vane 'pi yadi satyapratibhā\-saḥ, tarhi satyatanayā\-bhyā\-se 'pi śabdasā\-myā\-d abhedinas tanayasaṃjñakasya kasyacid aparasya svarū\-papratibhā\-saprasaṅgaḥ | tasmā\-d abhū\-taviṣayā\-bhyā\-saṃ nirvikalpakam api saṃvā\-dā\-n na pramā\-ṇam iti na sarvajñasiddhiḥ | 
	\pend
      

	  \pstart api ca bhā\-vyasya vastunaḥ punaḥ punaś cetasi niveśanam abhyā\-saḥ | sa ca brahmacaryeṇa tapasā\- sā\-daraṃ dī\-rghakā\-laṃ nirantaramā\-sevito dṛḍhabhū\-mir asphuṭā\-kā\-rasya vikalpasya sphuṭā\-bhatvajanana iṣṭaḥ | sa kṣaṇikatvanairā\-tmyavā\-dinā\- draḍhayitum aśakyaḥ | tathā\- hi bhā\-vyagrā\-hī\- yā\-dṛśo vikalpa utpannas tā\-dṛśa eva niranvayaṃ nirudhyate | tasmiṃś ca niruddhe punaḥ punar utpadyamā\-naḥ pratyayas tā\-dṛśa evā\-pū\-rva utpadyate | tad anena paryā\-yeṇa kalpasahasre 'py apū\-rvotpatter aviśeṣā\-n na tajjanyaḥ saṃskā\-ro 'bhyā\-sa utpadyate | etena viśiṣṭavijñā\-notpā\-do 'bhyā\-so vyā\-khyā\-taḥ | niranvayaniruddhaṃ hi pū\-rvapū\-rvavijñā\-naṃ katham uttarā\-vasthā\-ntaraṃ viśiṣtaṃ janayet | sarvathā\- kramabhā\-vibhiḥ pratyayair avasthitam eva rū\-paṃ śakyaṃ saṃskartum | anavasthitaṃ tu svotpā\-davyayayogimā\-tram ity aviśiṣṭaṃ syā\-t | tasmā\-t pratyā\-vṛttibhā\-vyavastupratyayajaḥ saṃskā\-ro vyutthā\-napratyayasaṃskā\-ravirodhī\- yasyā\-sti tasyaivā\-tmanaḥ prakṛṣṭo 'pi bhā\-vyasā\-kṣā\-tkā\-ripratyayahetur iti yuktaṃ paśyā\-maḥ | kiṃ ca cittam ekā\-graṃ vyavasthā\-payituṃ vikṣepatyā\-gā\-rtham abhyā\-so 'nuṣṭhī\-yate | na ca kṣaṇikavā\-dinā\-ṃ vikṣiptaṃ cittam asti | pratyarthaniyatatayā\- sarvasya vittaikā\-gratvā\-t | tathā\- hi yadi sā\-kā\-raṃ vikalpavijñā\-naṃ svapratibhā\-saniyatatvā\-t ekā\-gram eva tat kathaṃ vikṣipyate | atha nirā\-kā\-raṃ tathā\-pi vikalpakaṃ prati vikalpyaṃ bhinnam eva | na tu sarvavikalpā\-nā\-ṃ vikalpyam asti | tato nirā\-kā\-ram api vijñā\-naṃ niyatā\-lambanatvā\-d ekā\-gram eva, na vikṣiptam | sarvathā\- nā\-sti kṣaṇikavā\-dinā\-m ekam anekā\-rtham avasthitaṃ cittaṃ yad ekā\-graṃ kartum iṣyate | tad evam abhyā\-sā\-nupapatter asarvajñavatyā\-ṃ cittasantatau na ca vijñā\-naviśeṣaḥ sarvajñaḥ sidhyatī\-ti || 
	\pend
      

	  \pstart nyā\-yabhū\-ṣaṇakā\-ras tv ā\-ha | sarvajñā\-nā\-nā\-ṃ nirā\-lambanatve saṃvedamā\-tratve ca yogī\-tarapratyayayoḥ ko viśeṣaḥ | śuddhā\-śuddhatvam iti cet | bhavatu nā\-maivam | tathā\-pi caturā\-ryasatyā\-diviṣayatvam ayuktam | na hi svā\-tmamā\-travedanena caturā\-ryasatyā\-dikam sā\-kṣā\-tkṛtam iti yuktam, atiprasaṅgā\-t. 
	\pend
      

	  \pstart tadā\-kā\-ratvena tadviṣayatvam iti cet, tat kim idā\-nī\-ṃ sautrā\-ntikamatam abhyupagataṃ satyam | tathā\-py atī\-tā\-nā\-gataviṣayatvaṃ katham | na hy asataḥ kaścid ā\-kā\-ro 'sti | dṛṣṭaśrutā\-numitā\-kā\-raś ca yadi bhā\-vanā\-balataḥ spaṣṭa evā\-vabhā\-ti, tathā\- ca sati bhrā\-ntam eva yogipratyakṣaṃ syā\-t | avidyamā\-nasya vidyamā\-nā\-kā\-ratayā\- pratibhā\-sanā\-t, svapnavat | tathā\- 'visaṃvā\-ditvā\-n na bhrā\-ntam | na | anumā\-najñā\-nasya bhrā\-ntatve 'pi avisaṃvā\-ditvā\-bhyupagamā\-t | 
	\pend
      

	  \pstart atha bhrā\-ntasyā\-pi saṃvā\-ditvena prā\-mā\-ṇyam | tathā\-pi pratyakṣalakṣaṇasyā\-bhrā\-ntatvaviśeṣaṇaṃ virudhyate | na cā\-visaṃvā\-ditvam api tvanmate yuktam | yataḥ prā\-pyā\-rthadarśakatvaṃ vā\-, pravṛttiviṣayopadarśakatvaṃ vā\-, avabhā\-tā\-d arthakriyā\-niṣpattir vā\- bhavatā\-m avisaṃvā\-ditvam abhipretam | na caitad atī\-tā\-dyarthajñā\-ne sambhavati | vartamā\-nā\-rthajñā\-nasyā\-pi kṣaṇikatvapakṣe nopapadyata eva | tasmā\-t saugatā\-nā\-ṃ yogipratyakṣopavarṇanam ayuktam eveti || 
	\pend
      

	  \pstart kiṃ cedam api vaktum ucitam | yady anumā\-napū\-rvakam artheṣu bhā\-vanā\-balajajñā\-nam ā\-śvā\-sabhā\-janaṃ, tadā\-stā\-ṃ tā\-vad anumā\-napauruṣapratyā\-śā\- | pratyakṣeṇā\-pi cakṣurdahanā\-dikaṃ gṛhī\-tvā\- bhā\-vanā\-prakarṣaparyante jā\-taṃ sthirataraṃ tadā\-kā\-ravijñā\-naṃ syā\-t, yā\-van na viparī\-tabhā\-vanā\-bhiyogaparyantaḥ | astaṃ gataś ca tadviṣayo 'vasthā\-nataraprā\-pto veti kathaṃ pramā\-ṇopanī\-tavastugocaratve 'pi saṃvā\-dā\-śvā\-saḥ | api ca yadā\- hā\-lika eva havyā\-śanam anumā\-ya bhā\-vanayā\- sphuṭayet, tadā\- na tadyogijñā\-naṃ paramā\-rthaviṣayā\-bhā\-vā\-d iti pratyakṣā\-ntaraprasaṅgaḥ | 
	\pend
      

	  \pstart kiṃ ca tadyogijñā\-nam indriyajñā\-nā\-d bhinnam abhinnaṃ vā\- | abhedapakṣe na yogijñā\-naṃ nā\-ma pratyakṣeṇa bhinnam indriyajñā\-nenaiva saṅgrahā\-t | na ca bhā\-vanopaskṛtasantā\-nasya tathodayā\-d bhedavyavasthā\- | rasā\-yanā\-disaṃskā\-rā\-pekṣayā\-pi pratyakṣā\-ntaravyavasthā\-prasaṅgā\-t | bhedapakṣe ca bhā\-vanā\-sambhavaṃ jñā\-naṃ kṣaṇikasā\-kṣā\-tkā\-ri | indriyajñā\-naṃ ca syairyagrā\-hī\-ti sā\-dhvī\- siddhiḥ | indriyajñā\-nasyā\-pi tadavasthā\-yā\-m asthairyagrhaṇe kṛtaṃ yogijñā\-nena | na ca tasyā\-kasmikaḥ kṣaṇikatvā\-vabodhaḥ | bhā\-vanodbhū\-tavaiśadyasya hi tadbodhaḥ | na cendriyajñā\-nasya bhā\-vanā\- | api tu manovijñā\-ne | tā\-m antareṇā\-pi sā\-kṣā\-t kriyā\-lā\-bhe ca bhā\-vanā\-vaiyarthyam iti kā\-raṇā\-bhā\-vā\-d eva sarvajñapratihatiḥ || 
	\pend
      

	  \pstart atrā\-bhidhī\-yate | yat tā\-vat sarvapadā\-rthasaṃvedanasya kā\-raṇaṃ kim indriyajñā\-nam ityā\-di valgitaṃ tatra bhā\-vanā\-balajaṃ manovijñā\-nam eva sarvapadā\-rthagrā\-hī\-ti pañcama evā\-smā\-kaṃ pakṣaḥ | ataḥ pakṣā\-ntarabhā\-vino doṣā\- anubhyupagamapratihatā\-ḥ | yac cā\-smadabhyupagate pañcame pakṣe dū\-ṣaṇam uktam, anarthatvā\-t pratyakṣaśabdasya, tasya ca bhā\-vanā\-balā\-valambino 'py anakṣajatvā\-n nā\-rthasā\-kṣā\-tkā\-ritvam astī\-ti, tad asaṅgatam | tathā\- hi pratyakṣaśabdasya tā\-vad akṣā\-śritatvaṃ vyutpattinimittam arthasā\-kṣā\-tkā\-ritvaṃ tu pravṛttinimittam iti pratipā\-ditam | na ca bhā\-vanā\-balā\-valambino manovijñā\-nasyā\-nakṣā\-śritatve 'py arthasā\-kṣā\-rkaraṇe kaścid asti śaktipratighā\-taḥ | yathā\- hi cakṣurindriyaṃ svasā\-marthyā\-n atikrameṇa yogyadeśastham artham apekṣya svavijñā\-najanane pravartate, tathā\- sarvā\-vidyā\-paripanthibhū\-tā\-rthabhā\-vanā\-sahitaṃ mana indriyam api yogyadeśastham arthaṃ prā\-pya svavijñā\-najanane pravartiṣyate | aprā\-pyakā\-ritā\-yā\- ubhayoḥ sā\-dhā\-raṇatvā\-t | arthavattā\-yā\-ś ca manaso 'pi tadā\-nī\-m iṣṭatvā\-t | pṛthagjanasya tu na tā\-dṛśī\- śaktiḥ, yato netraśrotravanmano 'pi tā\-dṛṅmaryā\-dayā\- yogyadeśastham arthasahakā\-riṇam ā\-sā\-dya vedanam utpā\-dayet, sarvā\-vidyonmū\-lakasya bhā\-vanā\-viśeṣasya sahakā\-riṇo 'bhā\-vā\-d iti nā\-tiprasaṅgaḥ | tadavasthā\-yā\-ṃ tu śrutinayanayor iva manaso 'pi kiyaddū\-reṇa viṣayasannidhivyavasthitika eva pramā\-tuṃ kṣamaḥ | kevalam etā\-vad ucyate | yā\-vat tena śakyam adhigantuṃ svā\-kā\-rā\-rpaṇasamarthaṃ sahakā\-ri vastu tā\-vad itarajanā\-sā\-dhā\-raṇaṃ truṭyadrū\-patayā\- tasya gocarī\-bhavatī\-ti | ata evā\-rthā\-kā\-ro vastuto na bhā\-vanā\-mā\-trajanita iti na visaṃvā\-daśaṅkā\-pi | bhā\-vanayā\- punas tadī\-yasantā\-ne netra ivā\-ñjanaviśeṣeṇa śaktir atiśayavatī\- kā\-cid arpitā\- yatparajanā\-sā\-dhā\-raṇadarśanam asya | tasmā\-d anakṣajatve 'pi amnovijñā\-nasyā\-rthasā\-kṣā\-tkā\-ritvaṃ sambhavati | 
	\pend
      

	  \pstart nanu manaso bahirasvā\-tantryam | anyathā\-ndhabadhirā\-dyabhā\-vaprasaṅgā\-t | uktaṃ ca yoginā\-ṃ dharmā\-dharmayor aparokṣapratibhā\-saṃ jñā\-naṃ nā\-sti | indriyasannikarṣā\-bhā\-vā\-d asmadā\-divad iti | 
	\pend
      

	  \pstart api ca arthasya hy ā\-lambanapratyayatvam indriyā\-pekṣatvena vyā\-ptam | tac cā\-smā\-t svaviruddhopalabdhyā\- vyā\-vartamanam ā\-lambanapratyayatā\-m api tasya nivartayati | na khalv indhanaviśeṣo dhū\-mahetur iti vinā\-pi dahanaṃ sahasreṇā\-pi saṃskā\-rair dhū\-mam ā\-dhatte | tadā\-dhā\-ne samastakā\-ryahetukā\-numā\-nocchedaprasaṅgaḥ | na ca bhā\-vanā\-balena kasyacid atī\-ndriyadarśitvaṃ sarvajñatvaṃ vā\- dṛṣṭam iti cet | 
	\pend
      

	  \pstart atrocyate | manaḥśabdena tā\-vad asmā\-kam anakṣajaṃ vijñā\-nam evā\-bhipretam | na cā\-sminn andhabadhirā\-dyabhā\-vaprasaṅgaḥ | sarvā\-vidyā\-pratipakṣabhū\-tā\-rthabhā\-vanā\-lakṣaṇasya sahakā\-riviśeṣasyā\-ndhā\-dī\-nā\-m abhā\-vā\-t | indriyasannikarṣā\-bhā\-vā\-d iti tv arthasā\-kṣā\-tkā\-ritvamā\-trā\-pekṣayā\- sandigdhavyatirekitve anaikā\-ntikī\- kā\-raṇā\-nupalabdhiḥ | asmadvidhā\-rthasā\-kṣā\-tkā\-ritvā\-pekṣayā\- punaḥ siddhasā\-dhanam || 
	\pend
      

	  \pstart asmadā\-diviśeṣaṇaśū\-nyasyā\-rthasā\-kṣā\-tkā\-ritvamā\-trasyaivendriyā\-dhī\-natva-darśanā\-d anaikā\-ntikatvam asambhavī\-ti cet | yady evam arthasā\-kṣā\-tkā\-ritvamā\-trasyendiryavadā\-lokā\-dhī\-natvam upalabdham iti na santamase paśyeyur ulū\-kā\-dayaḥ | atha vyabhicā\-radarśanā\-d ā\-lokasyā\-vyā\-pakatvam, vyabhicā\-raśaṅkayā\- tarhī\-ndriyasyā\-py avyā\-pakatvam | vyā\-ptyā\- śaṅkā\- khaṇḍyata iti cet | śaṅkā\-sambhavā\-d vyā\-ptir evā\-sambhavinī\- yadi prathamata eva vyā\-ptiḥ, vyabhicā\-ro 'pi na dṛśyeta | 
	\pend
      

	  \pstart tasmā\-d vyabhicā\-radarśanaṃ vyā\-ptiśaithilyā\-d eva | sati ca vyā\-ptiśaithilye śaṅkā\-pi nyā\-yā\-d ā\-patantī\- kena pratihanyate | ulū\-kā\-dī\-nā\-ṃ bhinnajā\-tī\-yatvā\-d ā\-lokā\-bhā\-ve 'py arthasā\-kṣā\-tkā\-ritvam astv iti cet | tarhi bhagavato 'pi bhū\-tā\-rthabhā\-vanā\-prakarṣaparyantamahā\-pralayavā\-yunā\- nirastā\-nā\-dyā\-vipakṣasya saṃsā\-rakū\-papatitebhyaḥ prā\-ṇibhyo 'sty evā\-dbhū\-tavaijā\-tyam iti yuktam asyā\-vidyā\-pratipakṣabhā\-vanā\-tiśayasahitā\-tmakā\-ntarapratyayā\-d ā\-lambanapratyayā\-c ca sā\-kṣā\-dutpannasyendriyam antareṇā\-rthasā\-kṣā\-tkā\-ritvam | ataḥ kā\-raṇā\-nupalabdhiḥ kā\-śikā\-kā\-rasya vyā\-pakaviruddhopalabdhiś ca vā\-caspateḥ sandigdhavyatirekitvā\-d anaikā\-ntikī\- | sandigdhavyatirekitvaṃ tu dū\-ṣaṇam asmadī\-śvaradū\-ṣaṇe prasā\-dhitam || 
	\pend
      

	  \pstart tasmā\-t sā\-dhā\-raṇakarmanirjā\-tā\-nā\-m asmadā\-dī\-nā\-m arthasā\-kṣā\-tkā\-ritvam indiryā\-pekṣatvena vyā\-ptam iti siddhasā\-dhanam | prasiddhā\-numā\-nasya ca na kṣatir dṛśyatvopā\-dher dhū\-mā\-deḥ pratyakṣā\-nupalambhato vyā\-ptigrahaṇā\-virodhā\-t | sā\-ṃsā\-rikā\-gocarā\-rthasā\-kṣā\-tkā\-ritvamā\-trā\-pekṣayā\- tu sandigdhavyatirekitvam | adṛśyasya pratyakṣā\-nupalambhā\-bhyā\-ṃ kenacid vyā\-ptigrahaṇā\-yogā\-t | viparyaye bā\-dhakapramā\-ṇasya cā\-sambhavā\-d iti | na cā\-tī\-ndriyadarśitvaṃ sarvajñatvaṃ vā\-darśane 'pi niṣeddhuṃ śakyate, adṛśyā\-nupalambhato niṣedhā\-yogā\-t | kā\-raṇā\-nupalambatas tanniśedha iti cet | kā\-raṇā\-bhā\-vo 'pi adarśanamā\-trato na sidhyatī\-ti tadavasthaḥ paribhavaḥ || 
	\pend
      

	  \pstart yad api kā\-śikā\-kā\-reṇā\-bhihitam, atha pramā\-ṇā\-ntarā\-vagataṃ bhā\-vyate, kiṃ bhā\-vanayā\-, tata eva tatsiddher iti | tad apy asaṅgataṃ | pramā\-ṇā\-ntaraṃ hy anumā\-nam | na ca caturā\-ryasatyasvarū\-pe vastutattve niścite sā\-kṣā\-tkā\-ram antareṇa kleśajñeyā\-varaṇakṣatir iti svā\-rtham api tā\-vad bhā\-vanā\- yuktimatī\- | tattvasā\-kṣā\-tkā\-riṇi ca cittasantā\-ne sati śakyasā\-kṣā\-tkriyam idam ity anye 'pi niścayā\-nantaraṃ sā\-kṣā\-tkriyā\-yai pravartyante, tadupadiṣṭasvargasā\-dhanaṃ cā\-rthabhā\-vanayā\-nusarantī\-ti svargā\-pavargalakṣaṇaparā\-rthasiddhaye ca bhā\-vanā\- saphaleti | anyathā\- tattvā\-sā\-kṣā\-tkā\-riṇo lokā\-natikrā\-ntasya vacanam anā\-deyam eva syā\-d iti kva parā\-rthavā\-rtā\-pi | yac ca kiṃ ca tatpramā\-ṇam ityā\-dy ā\-rambhya tasmā\-d bhū\-tam abhū\-taṃ vety etatparyantena dharmā\-dharmayor anumā\-nā\-pravartanam uktam, tatra dharmā\-dharmaśabdena kim abhipretam | yadi kṣaṇikanirā\-tmakavastu tattvam, tadā\- tasya pratyakṣeṇā\-niścaye 'pi yathā\- viparyaye bā\-dhakapramā\-ṇabalena vyā\-ptisaṃvedanaṃ tathā\- kṣaṇabhaṅgasā\-dhanā\-vasare vyavasthā\-pitam | atha vastū\-nā\-ṃ svargā\-disā\-dhanatvam abhipretam, tadā\- tadviṣayaparijñā\-nā\-prasā\-dhane 'pi nā\-smā\-kaṃ kā\-cit kṣatiḥ | saparikarasaṃsā\-ranirvā\-ṇaparijñā\-nenaivopayuktasarvajñaprasā\-dhanā\-t | yad ā\-huḥ: heyopadeyatattvasyetyā\-di (PV I 217a) | 
	\pend
      

	  \pstart yad api, api ca bhā\-vanā\-balajaṃ gṛhī\-tagrahaṇā\-d apramā\-ṇam ity uktam, tatra gṛhī\-taṃ nā\-ma pratyakṣeṇā\-numā\-nena vā\- | pramā\-ṇā\-ntarasyā\-bhā\-vā\-t | na tā\-vat pratyakṣaṃ kṣaṇikatvā\-dā\-v arvā\-cī\-nasya kasyacid asti | anumā\-nena caikavyā\-vṛttiviśiṣṭe vastutattve 'vasite 'pi sarvā\-tmanā\- spaṣṭavastutattvasā\-kṣā\-tkā\-ri pratyakṣaṃ na gṛhī\-tagrā\-hi, anumā\-nena vastutattvā\-sparśanā\-t | na ca taduttarakā\-lam ityā\-di tu kā\-raṇā\-nupalabdhidū\-ṣaṇaprastā\-ve prativyū\-ḍham iti | 
	\pend
      

	  \pstart yad api \persName{vā\-caspatinā\-} satyam ityā\-dinā\- punaḥ punar uttarottaram ā\-śaṅkya tat kim anumā\-nena vahniṃ vyavasthā\-pyetyā\-dinā\- bhā\-vanā\-balajasyā\-numā\-napū\-rvakatve visaṃvā\-dam upadarśyopasaṃhṛtam, tan na bhā\-vanā\-yā\- bhū\-tā\-rthatvaṃ tajjaviśadavijñā\-naprā\-mā\-ṇyahetuḥ, vyabhicā\-rā\-d iti | tad asaṅgatam | tathā\- hy ayaṃ vahniviṣaye 'numā\-napū\-rvakabhā\-vanā\-balataḥ spaṣṭavahnipratyayaḥ kiṃ vahner apy utpannaḥ, tathā\-bhū\-tabhā\-vanā\-mā\-trā\-d eva vā\- |
	\pend
      

	  \pstart parathampakṣe visaṃvā\-daś ca bahulam upalabhyate iti yad uktaṃ tad durbhā\-ṣitam | sā\-kṣā\-d arthā\-d utpannasyā\-pi visaṃvā\-dasambhave 'nyasyā\-pi pratyakṣasya hastakatyā\-gaprasaṅgā\-t | 
	\pend
      

	  \pstart dvitī\-yapakṣe tu bhā\-vanā\-prakarṣamā\-trajasyā\-rthā\-d anutpannasya bahulaṃ visaṃvā\-dopalambhe 'pi bhā\-vanā\-rthā\-bhyā\-ṃ sā\-kṣā\-d utpannasya yogipratyakṣasyā\-pi visaṃvā\-dasambhava iti sthavī\-yasī\- bhrā\-ntiḥ | 
	\pend
      

	  \pstart nanu yadī\-ndriyaṃ vinā\-pi bhā\-vanā\-rthā\-bhyā\-ṃ yogijñā\-nam utpadyate, tarhi parvate bhā\-vanā\-vahnibhyā\-ṃ vahnijñā\-nam utpadyatā\-m avisaṃvā\-di | visaṃvā\-daś ca bahulam upalabhyata iti cet | na | sā\-kṣā\-d vahner utpā\-de sati visaṃvā\-dā\-bhā\-vā\-t | kevalam utpā\-da eva durā\-paḥ | na hi vayaṃ pramā\-ṇadṛṣṭavastubhā\-vanā\-sahitaṃ mana indiryam arthasvarū\-pagrā\-hijñā\-naṃ janayatī\-ti brū\-maḥ, api tv asaddṛṣṭilakṣaṇā\-vidyā\-paripanthikṣaṇikanairā\-tmyalakṣaṇasarvavastutattvabhā\-vanā\-sahitam | na ca vahnitvaṃ sarvavastutattvam, kiṃ tu kṣaṇikanairā\-tmyam eveti kṣaṇabhaṅgaprasā\-dhanataḥ pratipā\-ditam iti | kiṃ ca svamanī\-ṣā\-parikalpitaḥ khalv ayam anumitabhā\-vitavahniviṣayaviśadaḥ pratyayaḥ | na punar asya loke sambhavaḥ | tathā\- hi niṣprayojanam anunmatto na kaścid bhā\-vayati | prayojanaṃ ca śiśirabharamanthakā\-yakā\-ṇḍasyā\-pi dā\-hā\-dimā\-tram eva, tac cā\-numitenaiva vahninā\- taddeśopasarpaṇā\-t sidhyati | anupasarpaṇe bhā\-vā\-nā\-vaiyarthyam | purastā\-t tu bhā\-vite parisphurati tadathā\-pekṣayā\- bhrā\-ntiḥ prā\-sarpakasyevetyā\-dy upahā\-syam apy asya kṣatā\-tmano durnī\-tipū\-tigavī\-bhakṣaṇā\-dhmā\-tajaradgomā\-yor udgā\-ra iva satā\-m asahyaḥ | 
	\pend
      

	  \pstart yad api tato 'nantaramā\-śaṅkyā\-rthasyā\-lambanapratyayatvam indriyā\-pekṣitvena vyā\-ptam iti prasā\-dhitam, tatpū\-rvam eva pratyuktam | tathā\- bhā\-vanayā\-s cetyā\-dyā\-śaṅkyā\-rthasyā\-lambanapratyayatvam aśakyā\-vagamam iti yad uktaṃ tad apy asambaddham | 
	\pend
      

	  \pstart cakṣurindriyasyā\-py artham antareṇa dvicandrakeśoṇḍukā\-dau viśadabhrā\-ntajñanajananasā\-marthyam upalabdham ity arthasahitam api kevalam eva samartham | ato ghaṭā\-der apy ā\-lambanapratyayatvam aśakyā\-vagamam iti indriyapratyakṣam api pratihataṃ syā\-d iti | tathā\-pi cā\-lambanapratyayā\-pi ta eva yujyanta ityā\-dir na punar vikalpanirmā\-ṇapratibandhateti paryanto vyarthaḥ | asmā\-bhir evaṃvidhasya prastute 'nabhyupagatatvā\-t | ata eva tasmā\-d bhā\-vanā\-prakarśamā\-trajatvā\-t, arthā\-vyabhicā\-raniyamā\-bhā\-vā\-t, viśadā\-bham api saṃśayā\-krā\-ntatvā\-t, apramā\-ṇam apratyakṣaṃ ceti sā\-mpratam ity upasaṃhā\-ro 'pi dhikkā\-raḥ | sarveṣā\-m eva hetū\-nā\-m asiddhatvā\-t | bhā\-vanā\-balajasyā\-rthā\-d apy utpatter indriyapratyakṣavat | sadarthaprakā\-śanaṃ buddheḥ svabhā\-va ityā\-dy asmā\-kam api manoharam | bhā\-vanā\-yā\-ś ca sā\-mā\-nyena sphuṭā\-bhajñā\-nahetutvaṃ sā\-dhyate | pramā\-ṇopannacaturā\-ryasatyaviṣayaniṣṭhā\-yā\-ṃ tu sā\-marthyā\-t pratyakṣapramā\-ṇahetutā\-pi sā\-dhyate | ata eva kā\-minī\-pratibhā\-sasyā\-pramā\-ṇatve 'py apratyakṣatve 'pi sphuṭā\-bhatvasya sā\-dhyadharmasā\-mā\-nyasya sambhavā\-t na viruddho hetuḥ | nā\-pi dṛṣṭā\-ntasya sā\-dhyaśū\-nyateti | na ca nairā\-tmyadṛṣṭiḥ sambhavadbā\-dhā\-, arthā\-d utpatter abhū\-tā\-rthatvā\-bhā\-vā\-t | 
	\pend
      

	  \pstart śrutā\-numitaviṣayaṃ pratyakṣaṃ na sambhavatī\-ty apy ayuktam | ā\-gamā\-numā\-nayor dvividho viṣayaḥ grā\-hyo 'dhyavaseyaś ca | tatra grā\-hyaḥ svā\-kā\-raḥ, adhyavaseyas tu pā\-ramā\-rthikavastusvalakṣaṇā\-tmā\- | asya ca parokṣatve 'numā\-nasā\-magrī\-sambhave 'numā\-naviṣayatvam, pratyakṣasā\-magrī\-sambhave ca krameṇa pratyakṣaviṣayatvaṃ dṛṣṭam eva | tat siddham ityā\-dyupasaṃhā\-ro 'pi paryā\-kula eva | apramā\-ṇatvā\-d iti hetuś ca prathamo 'siddhaḥ | bhā\-vanā\-balajasyā\-rthā\-d apy utpatteḥ, pramā\-ṇaśaktisambhavā\-t, indriyapratyakṣavat | bhā\-vanā\-balajatvā\-d iti dvitī\-yas tu sandighavyatirekitvā\-d anaikā\-ntikaḥ | tathā\- yathā\-numitabhā\-vitavahniviṣayaviśadajñā\-nam iti dṛṣṭā\-nto 'py asambhavī\-ti pratipā\-ditam | bhavatu vā\-, tathā\-pi yogijñā\-nasya tena saha tulyahetutvam asiddham | tad dhi pramā\-ṇadṛṣṭavastubhā\-vanā\-mā\-trajam | yogijñā\-naṃ tv avidyā\-pratipakṣasarvavastutattvabhā\-vanā\-viṣayā\-bhyā\-m utpannam iti mahā\-ntam api viśeṣam asau durmatiprapā\-tapatito nā\-vagā\-hata ity upekṣaṇī\-yaḥ || 
	\pend
      

	  \pstart nyā\-yaprakī\-rṇe tu mā\-rgas tā\-vat pramā\-ṇapariśuddho na bhavatī\-ty uktaṃ yat, tat tatprasā\-dhakapramā\-ṇenaiva prayuktam | 
	\pend
      

	  \pstart yac cā\-pi cetyā\-dy ā\-rabhya yogijñā\-naṃ nirviṣayaṃ prasaktam ity uktam tatra keyaṃ nirviṣayatā\- nā\-ma | kiṃ vikalpā\-kā\-ranivṛttau nirā\-kā\-ratā\-, arthā\-kā\-rā\-d visadṛśā\-kā\-ratā\-, atha tadā\-kā\-ratve 'pi tadvastusaṃsparśitā\- | 
	\pend
      

	  \pstart na tā\-vat prathamaḥ pakṣaḥ kṣamaḥ | jñā\-nasya nirā\-kā\-ratā\-nupapatteḥ | 
	\pend
      

	  \pstart nā\-pi dvitī\-yaḥ | kā\-minyā\-dibhā\-vanā\-yā\-s tadā\-kā\-rasyaiva viśadasya darśanā\-t | 
	\pend
      

	  \pstart na ca tṛtī\-yaḥ | arthasamarpitā\-kā\-rasaṃsparśam apā\-syā\-nyasyā\-rthasaṃsparsasyā\-yogā\-t | 
	\pend
      

	  \pstart tathā\- coktam:
	\pend
      
	    
	    \stanza[\smallbreak]
arthena ghaṭayatyenā\-m | \footnote{\begin{english}(PV III 305a)\end{english}}\&[\smallbreak]


	

	  \pstart ityā\-di
	\pend
      

	  \pstart tayoś caikatvenā\-dhyavasā\-yā\-d bā\-hya eva pravṛttinivṛttī\-, vyā\-vahā\-rikasya sphuṭī\-bhā\-vo 'pi bahirabhimatasya paryante vikalpopā\-deyakṣaṇasyaiva sphuṭasyodayaḥ | tā\-vataiva sa viṣayas tena sā\-kṣā\-tkṛta iti vyavahā\-raḥ kevalam arthā\-d apy utpattau | anyathā\- vyabhicā\-rā\-d aprā\-mā\-ṇyam | na ca vikalpopadarśitam api rū\-pam avastu jñā\-nā\-tmakatvā\-t | anā\-tmakatve prakā\-śā\-yogā\-t | tadbhā\-vanaiva cā\-rthabhā\-vanā\-, tatsphuṭī\-bhā\-va eva bā\-hyasphuṭī\-bhā\-vaḥ, prakā\-rā\-ntareṇa bā\-hyasparśā\-yogā\-t | etena yat pā\-ramā\-rthikam ityā\-di na sarvajñasiddhir itiparyantaṃ prayuktam | 
	\pend
      

	  \pstart yac cā\-pi cetyā\-di na yuktaṃ paśyā\-ma itiparyantena dū\-ṣaṇam uktam, tad apy asaṅgatam | tathā\- hi yā\-dṛśa eva bhā\-vyagrā\-hī\- pratyayaḥ prathamo niranvayo niruddhas tā\-dṛśa evā\-para utpadyata iti niyamaniścayakā\-raṇaṃ na kiñcid asti caṇḍadevatā\-sparśā\-d anyat, kṣaṇikatvā\-d iti cet | nanu kṣaṇikatvaṃ sthā\-yitayā\- virudhyate na visadṛśotpā\-dena, tad dhi prā\-cī\-naṃ niranvayanirodhe yathā\- sadṛśakṣaṇā\-ntaram ā\-rabhate tathā\- svahetugatasā\-marthyayogā\-t kā\-ryotpā\-dā\-numeyā\-d yadi viśeṣaleśaviśiṣṭaṃ kṣaṇā\-ntaram utpā\-dayati, tadā\- na kā\-cit kṣatiḥ | na hi bhavata iva bhā\-vasyā\-pi kṣaṇikatā\-yā\-ṃ pradveṣo nā\-ma | tasmā\-n na kṣaṇikatvottaraviśiṣṭakṣaṇajanakatvayor virodha iti nā\-pā\-rthako 'bhyā\-saḥ | 
	\pend
      

	  \pstart yac cedaṃ kiñcetyā\-dinā\- kṣaṇikatve cittam avikṣiptam ā\-veditam, tad apy asā\-dhu | nairā\-tmyā\-ditattvaparā\-ṅmukhasya sarvasyaiva vikṣiptatvā\-t | bhā\-vanā\-balena tattvasā\-kṣā\-tkā\-riṇaḥ samā\-hitatvā\-t | atha ca tattvasā\-kṣā\-tkriyā\-lā\-bhā\-t grā\-hakā\-kā\-rā\-vagrahasambhavā\-t ca vyā\-vahā\-rikam api vikṣiptam asti cittam | yato mamaiva doṣakṣayo bhā\-vī\-ti mā\-rgā\-myā\-sapravṛttir abhyā\-hateti | paramā\-rthataḥ prā\-pyā\-dī\-nā\-m abhā\-ve 'pi tatsaṃkalpasyaivā\-nā\-dyavidyā\-prabhā\-vitasya sarvatra pravartakatvā\-t | ata eva mā\-rgasatyā\-bhyā\-sā\-t siddhaḥ sarvajñaḥ | 
	\pend
      

	  \pstart nyā\-yabhū\-ṣaṇasyā\-pi yogā\-cā\-rā\-pekṣayā\- dū\-ṣaṇam aprastutam | bahirarthā\-bhyupagamenaiva sā\-dhanaprakramā\-t | yac coktam tathā\-py atī\-tā\-nā\-gataviṣayatvaṃ katham, na hy asataḥ kaścid ā\-kā\-ro 'stī\-ti, tad etat prastā\-vā\-n avagā\-hanaphalam | upayuktasarvajñā\-dhikā\-reṇa hi sarvakṣaṇikanirā\-tmakavastubhā\-vanopakṣepaḥ, na sarvasarvajñā\-pekṣayā\- | tato 'tī\-tā\-nā\-gatam apratī\-yamā\-nam api na bā\-dhakam | tā\-vataiva duḥkhanirodhasiddheḥ | parasmai ca kṣaṇikatvā\-diniṣṭhakasya deśanā\-vatā\-rā\-t | na ca sarvasarvajñahastakatyā\-gaḥ | tathā\- hi caturā\-ryasatyasā\-kṣā\-tkā\-raprā\-ptau nirā\-varaṇā\-ntaḥkaraṇasya kā\-ruṇyā\-tiśayā\-t sarvā\-kā\-raparā\-rthaparatayā\- sakalagocaracā\-riṇi cetasi ciravirū\-ḍhotsā\-hasya tā\-dṛgupā\-yaviśeṣā\-dhigamo bhavaṣyati, yam anutiṣṭhatas tadutpattim antareṇā\-pi devatā\-dhipatyā\-t satyasvapnavat | pratiparamā\-ṇusarvaviṣayaṃ yathā\- deśakā\-lā\-kā\-rapratyavasthā\-nukā\-ri sphuṭataraṃ jñā\-nam udiyā\-t, tadā\- na tā\-vad vastuvyabhicā\-rakṛtaṃ visaṃvā\-ditvam, vastū\-nā\-m eva pratibhā\-sanā\-t | utpattisā\-rū\-pyā\-bhyā\-ṃ vedyasthitir iti tu pṛthagjanā\-pekṣayā\- | yoginas tu sā\-rū\-pyamā\-treṇaiva grahaṇam iti nyā\-yaḥ | 
	\pend
      

	  \pstart yad Vā\-rttikam 
	\pend
      
	    
	    \stanza[\smallbreak]
aviśuddhadhiyaḥ prati |&grā\-hyagrā\-hakacinteyam acintyā\- yoginā\-ṃ gatiḥ || iti | \footnote{\begin{english}(PV III 532)\end{english}}\&[\smallbreak]


	

	  \pstart tad evaṃ bhā\-vibhū\-tayor ajanakayor api yogijñā\-ne sphuraṇam abā\-dhyam | bhā\-vibhū\-tayos tarhi yadi svarū\-pasya sphuraṇam, vartamā\-nataiva syā\-t | atha svarū\-pam asannihitaṃ jñā\-nam eva tadā\-kā\-ram iti nirā\-lambanaṃ niyamena | tad api nā\-sti | yasmā\-d asannihite 'py arthe bhā\-vanā\-balā\-t taddeśakā\-lā\-kā\-rā\-nukā\-ri vijñā\-naṃ katham anā\-lambanam | tathā\-tvenā\-dhyavasā\-yā\-c ca, adhyavasitakā\-laviśiṣṭasyaiva satyasvapnavat tasya prā\-pteḥ | 
	\pend
      

	  \pstart yad \name{Bhā\-ṣyam}
	\pend
      
	    
	    \stanza[\smallbreak]
yathā\- sa dṛṣṭaḥ śaradā\-dikā\-layuktas tathā\- tasya na bā\-dhitatvam | &tatkā\-layuktas tu na tena dṛṣṭas tathā\-pratī\-tā\-v api nā\-sti doṣaḥ || \footnote{\begin{english}(PVA II 615)\end{english}}\&[\smallbreak]


	

	  \pstart jñā\-namā\-trasya tu tattvataḥ sphuraṇā\-c ca na vartamā\-natā\-prasaṅgaḥ saṅgataḥ | tathā\- kṣaṇikatvapakṣe 'pi ekatvā\-dhyā\-ropasā\-marthyā\-n na vyavahā\-rikaṃ prati pramā\-ṇasya kā\-cit kṣatir iti śā\-stre prapañcitam | 
	\pend
      

	  \pstart yad api kiñ cedam api vaktum ucitam ityā\-dy ā\-rabhya bhā\-vanā\-balajasyā\-numā\-napū\-rvakatve 'pi pratyakṣapū\-rvakatve 'pi vyabhicā\-rā\-bhidhā\-nam, tadarthā\-d api bhā\-vanā\-balajasya sā\-kṣā\-dutpattisvī\-kā\-rā\-d apahastitam | yathendriyajasyā\-pi dvicandrā\-dijñā\-nasyā\-rthā\-d anutpatter aprā\-mā\-ṇyam, arthendriyā\-bhyā\-m utpattau tu prā\-mā\-ṇyam evaṃ pramā\-ṇapū\-rvakasyā\-pi bhā\-vanā\-mā\-trā\-d utpannasyā\-prā\-mā\-ṇyam, bhā\-vanā\-rthā\-bhyā\-m utpannasya tu prā\-mā\-ṇyam | 
	\pend
      

	  \pstart yadi yogijñā\-nasyā\-rthā\-d utpattiḥ, pramā\-ṇapū\-rvakatvā\-pekṣayā\- na kiñcit prayojanam iti cet | na | deśakā\-lavastuviśeṣam apā\-sya sā\-mā\-neyana sarvadikkā\-lavartivastumā\-traṃ kṣaṇikanirā\-tmakam ity aniścaye mahā\-prayā\-sasā\-dhyapuruṣā\-yuṣavyā\-pinyā\-ṃ bhā\-vanā\-yā\-m eva pravṛtter abhā\-vā\-t | na ca hā\-liko havyā\-śanam anumā\-ya sphuṭī\-karoti yena pratyakṣā\-ntaratvaprasaṅgaḥ | asā\-marthyavaiyarthyā\-bhyā\-ṃ tadasambhavapratipā\-danā\-t | 
	\pend
      

	  \pstart yad apy uktaṃ yogino jñā\-nam indriyajñanā\-d abhinnaṃ bhinnaṃ vā\- | tatra prathamapakṣe tā\-van na vastudoṣaḥ | tā\-dṛkpuruṣaviśeṣasya siddhatvā\-t | vyavasthā\-dū\-ṣaṇam api nā\-sti | sā\-dhyatayaiva tā\-dṛgdaśā\-viśeṣasya lokā\-tikrā\-ntā\-tiśayasya paramapuruṣā\-rtharū\-pasya sā\-dhanaviśeṣapratipā\-danā\-ya pṛthagjanasā\-dhā\-raṇendriyajñā\-nā\-d bhedena nirdeśā\-t | paramapuruṣā\-rthaviṣayatvā\-bhā\-vā\-d eva ca rasā\-yanā\-disaṃskā\-rajasyā\-pi jñā\-nasya na pratyakṣā\-ntaratā\- | bhedapakṣe 'pi na tā\-vat sthairyetarasphuraṇakṛtopā\-lambhasambhavaḥ | indriyajñā\-nenā\-pi vastu sarvā\-tmanā\- gṛhṇatā\- truṭyadrū\-pasyaiva grahaṇā\-t | adhyavasā\-yo hi pū\-rvaṃ durllabhaḥ idā\-nī\-ṃ tu bhā\-vanā\-balanirdalitā\-vidye cittasantā\-ne so 'pī\-ndriyajñā\-nena janyata iti viśeṣaḥ | 
	\pend
      

	  \pstart nanu yogino manovijñā\-nendriyajñā\-nā\-bhyā\-ṃ paśyata ā\-kā\-radvayasphuraṇaprasaṅga iti cet | satyam | satyajñā\-nā\-kā\-ras tā\-vad vastuno na bhinnadeśo 'nyatarabhrā\-ntiprasaṅgā\-t | atas tā\-v ā\-kā\-rā\-v apratimau kayā\- gatyā\- sphurata iti ko nirṇetuṃ kṣamaḥ | yad ā\-ha: acintyā\- yoginā\-ṃ gatir iti |\footnote{\begin{english}(PV III 530d)\end{english}}
	\pend
      

	  \pstart sarvathā\- tu na yogijñā\-nasya kṣatir iti siddham | tad evaṃ kā\-raṇā\-nupalambhā\-d api na sarvajñatā\-bhā\-vaḥ | 
	\pend
      

	  \pstart nanu yadi nā\-ma yuṣmadabhimatasyā\-numā\-nasya na bā\-dhakam, tathā\-py asaty evā\-numā\-naṃ bā\-dhakam | tathā\- hi śakyam idam abhidhā\-tum 
	\pend
      

	  \pstart sugato 'sarvajñaḥ | jñeyatvā\-t, prameyatvā\-t, sattvā\-t, puruṣatvā\-t, vakṛtvā\-t, idriyā\-dimattvā\-d ityā\-di | rathyā\-puruṣavat | 
	\pend
      

	  \pstart tathā\- ca Bṛhaṭṭī\-kā\- 
	\pend
      

	  \pstart yasya jñeyaprameyatvavastusattvā\-dilakṣaṇā\-ḥ | 
	\pend
      

	  \pstart nihantuṃ hetavaḥ śaktā\-ḥ ko nu taṃ kalpayiṣyati || \footnote{\begin{english}(=TS 3157)\end{english}}
	\pend
      

	  \pstart Kā\-rikā\-pi 
	\pend
      

	  \pstart pratyakṣā\-dyavisaṃvā\-di prameyatvā\-di yasya ca | 
	\pend
      

	  \pstart sadbhā\-vavā\-raṇe śaktaṃ ko nu taṃ kalpayiṣyati | \footnote{\begin{english}(ŚV II 132)\end{english}}
	\pend
      

	  \pstart atrocyate | kim ete jñeyatvā\-dayaḥ sarvajñatvena sā\-kṣā\-d viruddhā\-ḥ paramparayā\- vā\- | aviruddhavidhā\-ne pratiṣedhā\-yogā\-t | sa ca sā\-kṣā\-d virodhaḥ parasparaparihā\-rasthitilakṣaṇo vā\-, bhā\-vā\-bhā\-vavat, sahā\-navasthā\-nalakṣaṇo vā\-, dahanatuhinavad iti | 
	\pend
      

	  \pstart na tā\-vad ā\-dyaḥ pakṣaḥ | yad vyavacchedanā\-ntarī\-yako yasya paricchedas tayor eva parasparaparihā\-rasthitilakṣaṇo virodhaḥ | na ca jñeyatvā\-di sarvajñatvavyavacchedena sthitam | kiṃ tarhi | ajñeyatvā\-divyacacchedena | tathā\- sarvajñatvam asarvajñatvavyavacchedena, na tu jñeyatvavyavacchedena | 
	\pend
      

	  \pstart nā\-pi dvitī\-yo virodhaḥ | yasya hy avikalakā\-raṇasya bhavato yat sannidhā\-nā\-d abhā\-vas tayor eva sahā\-navasthā\-nalakṣaṇo virodhaḥ | na ca sarvajñatvaṃ prā\-k pravṛttam avikalakā\-raṇaṃ dṛṣṭaṃ yena paścā\-j jñeyatvā\-disadbhā\-ve nirvartata iti syā\-t | tathā\-tve sati deśā\-diniṣedha eva bhaven na tu sarvathoccheda iti | 
	\pend
      

	  \pstart na ca paramparayā\- virodhaḥ | sa hi bhavan niṣedhyasya sarvajñatvasya vyā\-pakaviruddhatvā\-t, kā\-raṇaviruddhatvā\-t, kā\-ryaviruddhatvā\-t, svabhā\-vaviruddhakā\-ryatvā\-t, vyā\-pakaviruddhakā\-ryatvā\-t, kā\-raṇaviruddhakā\-ryatvā\-t, kā\-ryaviruddhakā\-ryatvā\-t, svabhā\-vaviruddhavyā\-ptatvā\-t, vyā\-pakaviruddhavyā\-ptatvā\-t, kā\-raṇaviruddhavyā\-ptatvā\-t, kā\-ryaviruddhavyā\-ptatvā\-d vā\- bhavet | tatra sarvajñatvasyā\-sattvā\-t, vyā\-pakakā\-raṇakā\-ryā\-ṇā\-m asiddhes tadviruddhakā\-ryavyā\-pyā\-bhā\-vā\-t na prameyatvā\-dayaḥ sarvajñatvena paramparayā\-pi viruddhā\-ḥ | 
	\pend
      

	  \pstart nanu vaktṛtvaṃ virudhyata eva sarvaviṣayanirvikalpajñā\-naviruddhavikalpakā\-ryatvā\-d vaktṛtvasya | naitad yuktam | savikalpā\-vikalpayor yugapadavṛtter vikalpatvena sarvajñasyā\-virodhā\-t | 
	\pend
      

	  \pstart kas tarhi pṛthagjanā\-d asya bheda iti cet | ucyate | yathā\- mā\-yā\-kā\-ro nirmitā\-śvā\-diviṣayaṃ vijñā\-naṃ nirviṣayatvena niścinvannabhrā\-ntaḥ, tadanyasmā\-c ca śreṣṭhaḥ, tathā\- bhagavā\-n api śuddhalaukikavikalpasammukhī\-bhā\-ve 'pi na bhrā\-nto nā\-pi pṛthagjanasamā\-na iti | tataś ca nirvikalpakasarvajñajñā\-navikalpayor virodhā\-bhā\-vā\-d vaktṛtvaṃ sarvajñatvena sahā\-viruddham eva || 
	\pend
      

	  \pstart etenaid api nirastam yad ā\-ha kā\-śikā\-kā\-raḥ, samā\-dher vyutthā\-yopadekṣyata iti cet | na | vyutthitasya hy abhilā\-pinī\- pratī\-tir bhrā\-ntabhā\-ṣitam apramā\-ṇaṃ bhaved iti || 
	\pend
      

	  \pstart yad apy uktaṃ Bṛhaṭṭī\-kā\-yā\-m 
	\pend
      

	  \pstart yadā\- copadiśedekaṃ kiñcit sā\-mā\-nyavaktṛvat | 
	\pend
      

	  \pstart ekadeśajñagī\-taṃ tan na syā\-t sarvajñabhā\-ṣitam ||\footnote{\begin{english}(=TS 3240)\end{english}}
	\pend
      

	  \pstart tad api nirastam, vikalpenaikasya kasyacid ā\-mukhī\-kṛtvopadeśe 'pi nirvikalpena sarvam avabudhyamā\-nasya vacanā\-nā\-ṃ sarvajñabhā\-ṣitatvā\-d eva || 
	\pend
      

	  \pstart yat punaḥ Kā\-rikā\-yā\-m uktam 
	\pend
      

	  \pstart sā\-nnidhyamā\-tratas tasya puṃsaś cintā\-maṇer iva | 
	\pend
      

	  \pstart niścaranti yathā\-kā\-mā\-ṃ kuḍyā\-dibhyo 'pi deśanā\-ḥ || 
	\pend
      

	  \pstart evam ā\-dyucyamā\-naṃ hi śraddadhā\-nasya śobhate | 
	\pend
      

	  \pstart kuḍyā\-diniḥsṛtatvā\-t tu nā\-śvā\-so deśanā\-su naḥ || 
	\pend
      

	  \pstart kin nu buddhapraṇī\-tā\-ḥ syuḥ kiṃ vā\- kaiścid durā\-tmabhiḥ | 
	\pend
      

	  \pstart adṛśyair vipralambhā\-rthaṃ piśā\-cā\-dibhir ī\-ritā\-ḥ ||\footnote{\begin{english}(ŚV II 138-140)\end{english}}
	\pend
      

	  \pstart Bṛhaṭṭī\-kā\-yā\-m api 
	\pend
      

	  \pstart tasmin dhyā\-nasamā\-dhisthe cintā\-ratnavadā\-sthite | 
	\pend
      

	  \pstart niścaranti yathā\-kā\-maṃ kuḍyā\-dibhyo 'pi deśanā\-ḥ || 
	\pend
      

	  \pstart tā\-bhir jijñā\-sitā\-n arthā\-n sarvā\-n jā\-nanti mā\-navā\-ḥ | 
	\pend
      

	  \pstart hitā\-ni ca yathā\-yogaṃ kṣipramā\-sā\-dayanti te || 
	\pend
      

	  \pstart ityā\-di kī\-rtamā\-naṃ tu śraddadhā\-nasya śobhate | 
	\pend
      

	  \pstart vayam aśraddadhā\-nā\-s tu ye yuktī\-r arthayā\-mahe || 
	\pend
      

	  \pstart kuḍyā\-diniḥsṛtā\-nā\-ṃ ca na syā\-d ā\-ptopadiṣṭatā\- | 
	\pend
      

	  \pstart viśvā\-saś ca na tā\-su syā\-t kenaitā\-ḥ kī\-rtitā\- iti || 
	\pend
      

	  \pstart kin nu buddhapraṇī\-tā\-ḥ syuḥ kiṃ vā\- brā\-hmaṇavañcakaiḥ | 
	\pend
      

	  \pstart krī\-ḍadbhir upadiṣṭā\-ḥ syur dū\-rasthapratiśabdakaiḥ || 
	\pend
      

	  \pstart kiṃ vā\- kṣudrapiśā\-cā\-dyair adṛṣṭaiḥ parikalpitā\-ḥ | 
	\pend
      

	  \pstart tasamā\-n na tā\-su viśvā\-saḥ kartavyaḥ prā\-jñamā\-nibhiḥ ||\footnote{\begin{english}(=TS 3241-46)\end{english}}
	\pend
      

	  \pstart etad apy anabhyupagamenaiva nirastam | śuddhalaukikavikalpasaṃmukhī\-bhā\-venaiva tasya deśakatvā\-bhyupagamā\-d iti || 
	\pend
      

	  \pstart atha vā\- yathā\- cakrasyoparate 'pi daṇḍapreraṇā\-vyā\-pā\-re pū\-rvā\-vegavaśā\-d bhramaṇam | evaṃ bhagavati pratyastamitasamastavikalpajā\-le 'pi sthite yadi pū\-rvapraṇidhā\-nā\-hitasatatā\-nā\-bhogavā\-hinī\- deśanā\- syā\-t tadā\- ko virodhaḥ | vivakṣā\-bhā\-ve kathaṃ vacanapravṛttir iti na vaktavyam | tadabhā\-ve 'pi nidrā\-ṇasya tattatpravyaktavacanasandarśanā\-t | vacanamā\-trasya vivakṣayā\- vyā\-pter abhā\-vā\-t | tasmā\-d yathā\- pū\-rvā\-bhyā\-sato jhaṭiti prabodhitasyā\-riṇā\- prahā\-rā\-didā\-nenā\-nurū\-pa eva prakramaḥ śastroddharaṇā\-dikaḥ, tathā\- sarvavedino 'pi sakalā\-ḥ kalā\-ḥ ity anā\-kulam | 
	\pend
      

	  \pstart yad ā\-hā\-laṅkā\-raḥ 
	\pend
      

	  \pstart śatrusā\-nnidhyamā\-treṇa pravartante 'vikalpataḥ | 
	\pend
      

	  \pstart prā\-g eva tannirā\-kā\-riprakramā\-ḥ kopanirmitā\-ḥ || \footnote{\begin{english}(PVA III 275)\end{english}}
	\pend
      

	  \pstart yat punar uktam: piśā\-cā\-dikṛtaśaṅkayā\- nā\-trā\-śvā\-saḥ satā\-ṃ yukta iti | 
	\pend
      

	  \pstart tad asaṅgatam, yataḥ 
	\pend
      

	  \pstart sambhinnā\-lā\-pahiṃsā\-dikutsitā\-rthopadarśanam | 
	\pend
      

	  \pstart krī\-ḍā\-śī\-lapiśā\-cā\-deḥ kā\-ryaṃ tā\-su na vidyate || 
	\pend
      

	  \pstart pramā\-ṇadvayasaṃvā\-di mataṃ tadviṣaye 'khile | 
	\pend
      

	  \pstart yasya bā\-dhā\- pramā\-ṇā\-bhyā\-maṇī\-yasy api nekṣate || 
	\pend
      

	  \pstart yathā\-tyantarokṣe 'pi na pū\-rvā\-parabā\-dhitam | 
	\pend
      

	  \pstart karuṇā\-diguṇotpatteḥ sarvapuṃsā\-ṃ pravartakam || 
	\pend
      

	  \pstart sarvā\-nuśayasaṃdohapratipakṣā\-bhidhā\-yakam | 
	\pend
      

	  \pstart nirvā\-ṇagaradvā\-rakapā\-ṭapuṭabhedam || 
	\pend
      

	  \pstart tac cet krī\-ḍanaśī\-lā\-nā\-ṃ rakṣasā\-ṃ vā\- vaco bhavet | 
	\pend
      

	  \pstart ta eva santu sambuddhā\-ḥ sarvatallakṣaṇasthiteḥ || \footnote{\begin{english}(=TS 3613-18)\end{english}}
	\pend
      

	  \pstart na ca nā\-mni vivā\-daḥ | na ca nā\-manivṛttau vastu nirvartate | pratyuta vedasyaiva krī\-ḍanaśī\-lapiśā\-cā\-dipraṇī\-tatvaṃ yuktaṃ sambhā\-vayitum | yena gośavā\-diṣu yogeṣv agamyā\-gamanā\-dayo 'satyasamudā\-cā\-rā\-ḥ saṃprakā\-śitā\-ḥ | lokaprasiddhiś ca | trayo vedasya kartā\-ro munibhaṇḍaniśā\-carā\-ḥ | iti alam atinirbandhena || 
	\pend
      

	  \pstart nanu sarvajñatvaṃ vī\-tarā\-gā\-ditvena vyā\-ptam iṣyate | tadviruddhaṃ ca rā\-gā\-diyogitvam, tatkā\-ryaṃ ca vacanam | tad etad vyā\-pakaviruddhakā\-ryabhū\-taṃ vacanaṃ sarvajñā\-bhā\-vaṃ sā\-dhayati paramparayā\- viruddhatvā\-d iti cet | na | rā\-gā\-dī\-nā\-ṃ vacasaś ca kā\-ryakā\-raṇabhā\-vā\-siddheḥ | tathā\- hi vacanaviśeṣo rā\-gā\-dikā\-ryam, yo rā\-geṇaiva janitaḥ, vacanamā\-traṃ vā\- | 
	\pend
      

	  \pstart tatra na tā\-vat prathamaḥ pakṣaḥ | tā\-dṛśasya vacanasya niścayopā\-yā\-sambhavā\-t | asabhyamaithunā\-cā\-raprakā\-śakaṃ vacanaṃ tatkā\-ryam iti cet | na | abhiprā\-yasya durlakṣyatvā\-t | virakto 'pi raktavac ceṣṭate, rakto 'pi viraktavad ity abhiprā\-yo durbodaḥ | tataś ca viśiṣṭavyavahā\-rasya sā\-ṃkaryeṇa na tatraikā\-ntena rā\-gā\-numā\-naṃ yujyate | nā\-pi vacanamā\-traṃ rā\-gā\-dikā\-ryam | asaṃmukhī\-bhū\-tarā\-gā\-dayo 'pi hi svā\-bhimatadevatā\-stutividhā\-ne mā\-trā\-digurujanasambhā\-ṣaṇā\-dau ca vacanamā\-tram uccā\-rayantaḥ samupalabhyante | na ca yad yadabhā\-ve bhavati tasya tatkā\-ryatocyate, atiprasaṅgā\-t | rā\-gā\-diyogyatā\- tarhi vacasaḥ kā\-raṇam, tayā\- vinopalakhaṇḍalā\-dau vacanasyā\-darśanā\-d iti cen | na | karaṇaguṇavaktukā\-mate hi vacanasya hetuḥ | tadabhā\-vā\-d evopalakhaṇḍalā\-dau nivartate, na rā\-gā\-diyogyatā\-yā\- abhā\-vā\-t | yadi kā\-raṇaguṇā\-disakalatadanyakā\-raṇabhā\-ve 'pi rā\-gā\-diyogyatā\-bhā\-vā\-n notpadyate vacanam iti sidhyet tasyā\-ḥ kā\-raṇatvam | upalakhaṇḍalā\-dau tu vaktukā\-matā\- nā\-sti | tat kathaṃ tatkā\-raṇatvaṃ vacasā\-m iti | evaṃ tarhi vaktukā\-mataiva rā\-go 'stu | iṣṭatvā\-n na kiñcid bā\-dhitaṃ syā\-t, nā\-mni vivā\-dā\-bhā\-vā\-t | paramā\-rthataḥ punar nityasukhā\-tmā\-tmī\-yadarśanā\-kṣiptaṃ sā\-śravaviṣayaṃ cetaso 'bhiṣvaṅgaṃ rā\-gam ā\-huḥ | 
	\pend
      

	  \pstart niṣpannasarvasampatter vivakṣā\-pi na yujyata iti cet | adoṣo 'yam, parā\-rthatvā\-divivakṣā\-yā\-ḥ | vī\-tarā\-ge 'rthā\-saṅgā\-bhā\-vā\-t kathṃ parā\-rthā\-pi pravṛttir iti cet | na | ā\-saṅgam antareṇa karuṇayā\-pi pravṛtteḥ | 
	\pend
      

	  \pstart saiva rā\-ga iti cet | iṣṭatvā\-d adoṣaḥ | rā\-gasya tu svarū\-pam uktam | kā\-ruṇikasyā\-pi niṣphalā\-rambho na yukta iti cet | na | parā\-rthasyaiva phalatvā\-t | iṣṭalakṣaṇatvā\-t phalasyeti yat kiñcid etat | 
	\pend
      

	  \pstart nanu nirvikalpasya bhagavataḥ kathaṃ tasyā\-m avasthā\-yā\-ṃ karuṇā\-sambhavaḥ | duḥkhavikalpaprabhavā\- hi karuṇety anvayavyatirekā\-bhyā\-m anyatvena niścitam | 
	\pend
      

	  \pstart tataś ca kā\-raṇā\-bhā\-vā\-t kathaṃ kā\-ryasambhava iti cet | na | yathā\- kumbhakā\-ranivṛttā\-v api svasantā\-namā\-trabhā\-vinī\- ghaṭā\-disthitis tathotthā\-pakavikalpā\-bhā\-ve 'pi samanantarapratyayabalā\-d anā\-lambanakaruṇā\-pravṛtter avā\-ryatvā\-t | yad ā\-hur guruvaḥ 
	\pend
      

	  \pstart sattā\-ropakṛto 'pi bhā\-vanavaśā\-t kā\-ṭhinyam ā\-pat tathā\- śaithilye 'pi yathā\-sya duḥkhahataye sā\-ndras tathaiva śramaḥ | 
	\pend
      

	  \pstart utpā\-de tu phalasya hetuniyamo no tu prabandhasthitau tasmā\-d duḥkhadṛśaḥ kṣaye 'pi vilasanmaitryā\-daye 'smai namaḥ || 
	\pend
      

	  \pstart etenaitad api nirastaṃ yad ā\-ha Kā\-rikā\-yā\-m 
	\pend
      
	    
	    \stanza[\smallbreak]
rā\-gā\-dirahite cā\-samin nirvyā\-pā\-re vyavasthite | &deśanā\-nyapraṇī\-taiva syā\-d ṛte pratyavekṣaṇā\-t || \footnote{\begin{english}(ŚV II 137)\end{english}}\&[\smallbreak]


	

	  \pstart nanu yadi nā\-maiva vaktṛtvaṃ sarvajñatvena sahā\-viruddhaṃ dehendriyabuddhyā\-diyogitvaṃ tu viruddham eva | sarvajñatā\-vyā\-pakavī\-tarā\-gatvaviruddharā\-gā\-dikā\-raṇatvā\-d dehā\-dī\-nā\-m | 
	\pend
      

	  \pstart tataś ca pratiṣedhyavyā\-pakaviruddhakā\-raṇopalambhā\-t sarvajñā\-bhā\-va iti cet | ucyate | dehā\-dī\-nā\-ṃ hetutve 'pi naiṣā\-ṃ kevalā\-nā\-ṃ sahakā\-rimā\-trā\-ṇā\-m ā\-tmā\-bhiniveśalakṣaṇopā\-dā\-nakā\-raṇavikalā\-nā\-ṃ rā\-gā\-dijanakatvam ity agamakā\- eva dehā\-dayaḥ sarvajñā\-bhā\-vasya | tasmā\-j jñeyatvā\-dī\-nā\-m apy asā\-marthyā\-n na paraparikalpitā\-numā\-nato 'pi sarvajñā\-bhā\-vaḥ | 
	\pend
      

	  \pstart nā\-pi svavikalpitaṃ śā\-bdā\-dikaṃ bhagavato bā\-dhakam | tathā\- hi yady api teṣā\-ṃ sati prā\-mā\-ṇye 'numā\-na evā\-ntarbhā\-vaḥ, anantarbhā\-ve cā\-prā\-mā\-ṇyam eveti sthū\-laṃ dū\-ṣaṇam asti, tathā\-pi tatprā\-mā\-ṇyam abhyupagamyā\-pi brū\-maḥ | yat tā\-vat pauruṣeyavacanaṃ tadapramā\-ṇam eva bhavatā\-m | na ca vaidikaṃ kiñcid vacanaṃ sarvanarā\-sarvajñatvapratipā\-dakam upalabhyate | pratyuta nimittanā\-mni śā\-khā\-ntare sphuṭataram eva sarvajñaḥ pratipā\-ditaḥ | 
	\pend
      

	  \pstart tathā\- hi: sa vetti viśvaṃ na ca tasya vettā\- ityā\-dinā\- ca sarvajño vede pratipā\-ditaḥ || 
	\pend
      

	  \pstart nā\-py upamā\-nā\-t tadabhā\-vaḥ sidhyati | tathā\- hi smaryamā\-ṇam eva gavā\-divastu purovartigavayā\-disā\-dṛśyopā\-dhi gavā\-dyupā\-dhi vā\- sā\-dṛśyam upamā\-nena pratī\-yata iti sthitiḥ | na ca sarvajñasantā\-navartī\-ni cetā\-ṃsi kenacit sarvajñenā\-nubhū\-tā\-ni yataḥ smaraṇena viṣayī\-kriyeran, paracittavitter ayogā\-t || 
	\pend
      

	  \pstart yat punar uktaṃ Kumā\-rilena 
	\pend
      

	  \pstart narā\-n dṛṣṭvā\- tv asravajñā\-n sarvā\-n evā\-dhunā\-tanā\-n | 
	\pend
      

	  \pstart tatsā\-dṛśyopamā\-nena śeṣā\-sarvajñaniścayaḥ || \footnote{\begin{english}(=TS 3215)\end{english}}
	\pend
      

	  \pstart tad apy ayuktam, adhunā\-tanasarvajñatvā\-niścayā\-t | niścaye cā\-tmany eva sarvajñatvā\-bhyupagamaprasaṅgā\-t | 
	\pend
      

	  \pstart nā\-py arthā\-pattir bā\-dhikā\- | yato dṛṣṭaḥ śruto vā\-rtho 'nyathā\- nopapadyata iti adṛṣṭā\-rthaparikalpanam arthā\-pattir ucyate | na cā\-sarvajñatvam antareṇa sarvanareṣu kaścid artho dṛṣṭaḥ śruto vā\- nopapadyate yatas tadarthā\-pattyā\- parikalpyeta | nanu saṃsā\-rasya tā\-vad anā\-ditvaṃ pramā\-ṇena pratī\-tam | tac ca na sarvajñena jñā\-yate, tajjñā\-nā\-vadheḥ parastā\-d asattve 'nā\-ditā\-kṣatiprasaṅgā\-t, tadanyathā\-nupapadyamā\-naṃ sarvabhā\-vā\-nā\-m anā\-ditvaṃ sarvajñā\-bhā\-vaṃ sā\-dhayatī\-ti cet | 
	\pend
      

	  \pstart ucyate | upayuktasarvajñā\-pekṣayā\- tā\-vad idam adū\-ṣaṇam | tasyā\-nā\-ditvā\-jñā\-ne 'pi upayuktasarvajñatvā\-vyā\-hateḥ | sarvasarvajñasyā\-py abhā\-ve sā\-dhye 'samartheyam arthā\-pattiḥ | tathā\- hi yathā\- saṃsā\-rasyā\-nā\-ditve pū\-rvapū\-rvavastusattā\-yā\- anavadhitvaṃ tathā\- sarvajñajñā\-nasyā\-pi pū\-rvapū\-rvavastusattā\-vyā\-pakatvenā\-navadhiprasaratā\- iti | ajñā\-tasyaikasyā\-pi vastuno 'navasthiteḥ | saty api sarvajñe 'nā\-ditvam upapadyamā\-naṃ na sarvajñā\-bhā\-vam ā\-kṣipati | tataś cā\-rthā\-pattir api na sarvajñasya bā\-dhikā\- | 
	\pend
      

	  \pstart na cā\-bhā\-vapramā\-ṇabā\-dhyaḥ sarvajñaḥ | pramā\-ṇapañcakanivṛttir 
	\pend
      

	  \pstart abhā\-vapramā\-ṇam iṣyate | tatra nivṛttir iti prasajyavṛttyā\- 
	\pend
      

	  \pstart pramā\-ṇā\-nutpattimā\-tram abhipretam, atha vā\- paryudā\-savṛttyā\- 
	\pend
      

	  \pstart vastvantaram, vastvantaram api jaḍarū\-paṃ jñā\-narū\-paṃ vā\-, jñā\-nam api 
	\pend
      

	  \pstart jñā\-namā\-tram, ekajñā\-nasaṃsargivastujñā\-naṃ veti vikalpā\-ḥ | 
	\pend
      

	  \pstart tatra na tā\-van nivṛttimā\-tram abhā\-vapramā\-ṇam upapadyate | tat 
	\pend
      

	  \pstart khalu nikhilaśaktivikalatayā\- na kiñcit | yac ca na kiñcit tat kathaṃ 
	\pend
      

	  \pstart prameyaṃ paricchindyā\-t, tadviṣayaṃ vā\- vijñā\-naṃ janayet, pratī\-taṃ vā\- 
	\pend
      

	  \pstart tat katham iti sarvam andhakā\-ranartanam | yathoktam: na hy abhā\-vaḥ 
	\pend
      

	  \pstart kasyacit pratipattiḥ pratipattihetur vā\- | tasyā\-pi vā\- kathaṃ 
	\pend
      

	  \pstart pratipattir \footnote{\begin{english}(HB 25,12-14)\end{english}} iti |
	\pend
      

	  \pstart nā\-pi vastvantaratā\-pakṣe jaḍarū\-paḥ pramā\-ṇā\-bhā\-vaḥ saṅgacchate, tasya prameyaparicchedā\-yogā\-t | paricchedasya jñā\-nadharmatvā\-t | nā\-pi jñā\-namā\-trasvabhā\-vo 'bhā\-vaḥ | deśakā\-lasvabhavaviprakṛṣṭasyā\-pi tato 'bhā\-vaprasaṅgā\-t | tadapekṣayā\-pi vijñā\-namā\-tratvā\-t tasya | athaikajñā\-nasaṃsargisvabhā\-vo 'numanyate, tadā\- kṣatam abhā\-vapramā\-ṇapratyā\-śayā\-, adhyakṣaviśeṣasyaivā\-bhā\-vapramā\-ṇanā\-makaraṇā\-t | tasya cā\-smā\-bhir dṛśyā\-nupalambhā\-khyasā\-dhanatvena svī\-kṛtatvā\-t | dṛśyā\-nupalambhaś ca bhagavadabhā\-vasā\-dhane 'samartha iti pū\-rvam evā\-veditam | 
	\pend
      

	  \pstart kiṃ ca, kaḥ punar ayaṃ pramā\-ṇā\-bhā\-vo 'bhimato bhavatā\-m | svapramā\-ṇagaṇanivṛttir atha sarvaprā\-ṇigaṇapramā\-ṇanivṛttiḥ | tatra svapramā\-ṇagaṇanivṛttir vyabhicā\-riṇī\-, tasyā\-ṃ satyā\-m api vyavahitasyā\-rthasyā\-napahnavatvā\-t | parapramā\-ṇanivṛttis tv asarvavido 'siddhā\- | yad ā\-ha 
	\pend
      

	  \pstart sarvā\-dṛṣṭiś ca sandigdhā\- svā\-dṛṣṭir vyabhicā\-riṇī\- | 
	\pend
      

	  \pstart vindhyā\-drirandhradū\-rvā\-der adṛṣṭā\-v api sattvataḥ || iti || \footnote{\begin{english}(=TS 122)\end{english}}
	\pend
      

	  \pstart tad evaṃ nā\-bhā\-vapramā\-ṇato 'pi sarvajñaniṣedha iti sthitam || 
	\pend
      

	  \pstart nanu tathā\-pi sadvyavahā\-rā\-rthaṃ sā\-dhakam apy asya na vidyate | tathā\- hi sarvavido 'tī\-ndriyatvā\-t na tā\-vad asmadā\-dipratyakṣam asya sā\-dhakam | yathā\- cā\-smā\-bhir asau nopalabhyate tathā\-smajjā\-tī\-yair apy apratyakṣasvabhā\-vaniyamā\-t | na cā\-yaṃ kā\-lā\-ntare 'bhū\-d iti ca kalpanā\- yujyate | yathā\- hi kā\-latvā\-didā\-nī\-ntanakā\-lavad iti anenā\-numā\-nena nirā\-kartuṃ śakyate, na tathā\- sā\-dhayitum | Kā\-rikā\- 
	\pend
      
	    
	    \stanza[\smallbreak]
sarvajñakalpanā\- tv anyair vede vā\-pauruṣeyatā\- | &tulyavat kalpyate yena tenedaṃ saṃpradhā\-ryate || &sarvajño dṛśyate tā\-van nedā\-nī\-m asmadā\-dibhiḥ | &nirā\-karaṇavac chakyā\- na cā\-sī\-d iti kalpanā\- ||\footnote{\begin{english}(ŚV II 116-117)\end{english}}\&[\smallbreak]


	

	  \pstart iti ||
	\pend
      
	    
	    \stanza[\smallbreak]
nā\-py anumā\-nataḥ sarvajñasiddhiḥ | tatpratibaddhaliṅgā\-niścayā\-t | \&[\smallbreak]


	

	  \pstart kiṃ ca sarvajñasattā\-sā\-dhane sarvo hetuḥ trayī\-ṃ doṣajā\-tiṃ nā\-tivartate asiddhatvaṃ viruddhatvam anaikā\-ntika-tvaṃ ceti | tathā\- hi sarvajñe dharmaṇi kriyamā\-ṇe na taddharmo hetuḥ siddhaḥ | tasyaiva dharmiṇaḥ sā\-dhyatvenā\-siddhatvā\-t | siddhau vā\- vaiyarthyaprasaṅgā\-t | asarvajñe dharmiṇi na sarvajñasiddhiḥ | hetoḥ sarvajñaviparī\-tasā\-dhanatvena viruddhatvā\-t | nā\-pi sarvajñā\-sarvajñadharmo hetuḥ | tasyā\-naikā\-ntikatvā\-t | tasmā\-n nā\-numā\-nato 'pi sarvajñasiddhiḥ | 
	\pend
      

	  \pstart Kā\-rikā\-
	\pend
      
	    
	    \stanza[\smallbreak]
dṛṣṭo na caikadeśo 'sti liṅgaṃ yo vā\-numā\-payet |\footnote{\begin{english}(=TS 3125cd)\end{english}}\&[\smallbreak]


	

	  \pstart iti ||
	\pend
      

	  \pstart nā\-py ā\-gamagamyaḥ | ā\-gamo hi dvividhaḥ pauruṣeyo nityaś ca | tatra pauruṣeyo 'py ā\-gamaḥ tadī\-yo vā\- tatra pramā\-ṇam, narā\-ntarapraṇī\-to vā\- | na tā\-vat tadī\-yaḥ | anyonyasaṃśrayā\-patteḥ | tathā\- hy ā\-gamasya sarvajñoktatve prā\-mā\-ṇyam | asya ca prā\-mā\-ṇye satyasmā\-t sarvajñasiddhir iti | narā\-ntarapraṇī\-tas tu pramā\-ṇatvenā\-nabhimata evety ato 'pi na sarvajñasiddhiḥ || 
	\pend
      

	  \pstart kiṃ ca sarvajñapraṇī\-tā\-d vacanā\-t sarvajñasiddhau kim aparā\-ddhaṃ svavacanena yenā\-to 'py asau na gamyeta | nā\-pi nityā\-gamagamyaḥ sarvajñaḥ, tathā\-vidhasya sarvajñapratipā\-dakasya nityā\-gamasyā\-bhā\-vā\-t | yac copaniṣadā\-dau sarvajñapratipā\-dakavā\-kyaṃ tasyā\-nyā\-rthatvaṃ draṣṭavyam | na ca nityavā\-kyasyā\-nityasarvajñatvapratipā\-dakatvam, nirviṣayatvaprasaṅgā\-t | 
	\pend
      

	  \pstart kiṃ ca yady aṅgī\-kṛto nityā\-gamaḥ, kiṃ sarvajñakalpanayā\-, nitya evā\-gamo dharme pramā\-ṇaṃ bhaviṣyati | 
	\pend
      

	  \pstart Kā\-rikā\- 
	\pend
      
	    
	    \stanza[\smallbreak]
na cā\-gamena sarvajñas tadī\-ye 'nyonyasaṃśrayā\-t | &narā\-ntarapraṇī\-tasya prā\-mā\-ṇyaṃ gamyate katham || &na cā\-py evaṃ paro nityaḥ śakyo labdhum ihā\-gamaḥ | &dṛṣṭaś ced arthavā\-datvaṃ tatpare syā\-d anityatā\- || &ā\-gamasya ca nityatve siddhe tatkalpanā\- vṛthā\- | &yatas taṃ pratipatsyante dharmam eva tato narā\-ḥ || \&[\smallbreak]


	[[(ŚV II 118-120)]]

	  \pstart Bṛhaṭṭī\-kā\-pi 
	\pend
      
	    
	    \stanza[\smallbreak]
na cā\-gamavidhiḥ kaścin nityaḥ sarvajñabodhakaḥ | \footnote{\begin{english}(=TS 3186ab)\end{english}}\&[\smallbreak]


	

	  \pstart ityā\-di saptacatvā\-riṃśat ślokā\-ḥ saprapañcam etam arthaṃ pratipā\-dayanti | tad evam ā\-gamato 'pi na sarvajñasiddhiḥ | 
	\pend
      

	  \pstart nā\-py upamā\-napramā\-ṇasamadhigamyaḥ | upamā\-naṃ hi sadṛśagrahaṇanā\-ntarī\-yakapravṛttikam asannikṛṣṭā\-rthagocaram | yathā\- gavanagrahaṇadvā\-reṇa goḥ smaraṇam | na ca sarvajñasadṛśaḥ kaścid asti | Kā\-rikā\- 
	\pend
      

	  \pstart sarvajñasadṛśaṃ kañcid yadi paśyema samprati | 
	\pend
      

	  \pstart upamā\-nena sarvajñaṃ jā\-nī\-yā\-mas tato vayam || \footnote{\begin{english}(=TS 3215)\end{english}}
	\pend
      

	  \pstart nā\-py arthā\-pattitaḥ sarvajñasiddhiḥ | dṛṣṭaḥ śruto vā\-rtho 'nyathā\- nopapadyata ity adṛṣṭā\-rthaparikalpanam arthā\-pattilakṣaṇam | na cā\-tra pramā\-ṇapratī\-taṃ kiñcid vastv asti yat sarvajñam anatareṇā\-nupapadyamā\-naṃ tat sattā\-m upanayet | tan nā\-rthā\-pattir api sarvajñasā\-dhanī\- | 
	\pend
      

	  \pstart na ca pramā\-ṇapañcakā\-bhā\-vasvabhā\-vā\-d abhā\-vapramā\-ṇā\-d asya siddhiḥ, vastvabhā\-vasā\-dha\leavevmode\textsuperscript{\rmlatinfont\tiny [17a/RNAms]}\label{RNAms_17a}natvā\-d asya | pratyutā\-yam evā\-syā\-bhā\-vaṃ sā\-dhayatī\-ti pratipā\-ditam | yad apī\-daṃ kā\-rikā\-bṛhaṭtī\-kayor ekaṣaṣṭyā\- ślokaiḥ sarvajñasiddhaye bauddhasya sā\-dhanam ā\-śaṅkya dū\-ṣitaṃ tad api ghṛṇā\-karam iti granthavistarabhayā\-n na likhitam |
	\pend
      

	  \pstart tathā\- hy etā\-ni kila saugataiḥ sarvajñasā\-dhanā\-ya sā\-dhanā\-ny abhidhī\-yante | sarvajño 'stī\-ti satyam, sarvajñoktatvā\-t, dharmā\-bhyupadeśakatvā\-t, buddhaḥ sarvajña iti cirapravṛttadṛḍhasmṛteḥ, prathamataram aśeṣaśiṣyajanavargasyā\-nekavidhacittacaittā\-diparijñā\-nā\-t, sakalapadā\-rtharā\-śitattvopadeśā\-d iti || 
	\pend
      

	  \pstart tasmā\-t sthitam etat nā\-tī\-ndriyadarśī\- sā\-kṣā\-d asti, api tu nityavacanadvā\-reṇaiva tasya darśanam iti | tad evaṃ sarvathā\- sarvajñasā\-dhakapramā\-ṇā\-sabhavā\-d ayukto bauddhā\-nā\-ṃ sarvajñe sadvyavahā\-ra iti || 
	\pend
      

	  \pstart atrocyate | anumā\-nā\-d anyato 'siddhau siddhasā\-dhanam | anumā\-nā\-d apī\-ty asiddham, anumā\-nasya pū\-rvam uktatvā\-t | tatpratibaddhaliṅgā\-niścayā\-d ityā\-didū\-ṣaṇaprabandho 'pi prativyū\-ḍha ity upayuktasarvajñas tā\-vat trailokyā\-lokaḥ siddhaḥ | 
	\pend
      

	  \pstart sarvasarvjñapakṣe 'pī\-daṃ sā\-dhanam | 
	\pend
      

	  \pstart yat pramā\-ṇasaṃvā\-diniścitā\-rthavacanaṃ tat sā\-kṣā\-t paramparayā\- vā\- tadarthasā\-kṣā\-tkā\-rijñā\-napū\-rvakam | yathā\- dahano dā\-haka iti vacanam | pramā\-ṇasaṃvā\-di niścitā\-rthavacanaṃ cedam | kṣaṇikā\-ḥ sarvajñasaṃskā\-rā\- ity arthataḥ kā\-ryahetuḥ | nā\-syā\-siddhiḥ, sarvabhā\-vakṣaṇabhaṅgaprasā\-dhanā\-d asya vacanasya satyā\-rthatvā\-t | nā\-pi virodhaḥ, sapakṣe bhā\-vā\-t | na cā\-naikā\-ntikaḥ, vacanamā\-trasya saṃśayaviparyā\-sapū\-rvakatve 'pi pramā\-ṇaniścitā\-rthavacanasya sā\-kṣā\-tpā\-ramparyeṇa tadarthasā\-kṣā\-tkā\-rijñā\-napū\-rvakatvā\-t | anyathā\- niyamena pramā\-ṇasaṃvā\-dā\-yogā\-t || 
	\pend
      

	  \pstart ayaṃ ca bhā\-ṣyakā\-rī\-yaḥ sarvasarvajñaprasā\-dhakaprayogaḥ paṇḍitajitā\-ribhiḥ prapañcita iti tata eva pracayato 'avadhā\-rya iti | 
	\pend
      
	    
	    \stanza[\smallbreak]
durvā\-raprativā\-divikramam anā\-dṛtya pramā\-prauḍhitaḥ sarvajño jagadekacakṣurudagā\-d eṣa prabhā\-vo 'tra ca | &sambuddhasthitimedinī\-kulagirer asmadguroḥ kin tv ayaṃ saṃkṣepo mama ratnakī\-rtikṛtinas tadvistaratrā\-sinaḥ || \&[\smallbreak]


	
	    
	    \stanza[\smallbreak]
viśvam astu śubhā\-d asmā\-d yathecchaṃ ratimanmataḥ | &mañjuvajraś ca paryante tatpā\-daṃ satphalapradam || &ahañ ca mañjuvajraḥ syā\-ṃ mañjughoṣo 'tha mañjuvā\-k | &mañjuśrī\-r vā\-dirā\-ṇmamañjukumā\-ro jinadhū\-rdharaḥ | \&[\smallbreak]


	

	  \pstart || sarvjñasiddhiḥ samā\-ptā\- || 
	\pend
      \label{īśvarasādhanadūṣaṇam}\edlabel{īśvarasādhanadūṣaṇam}
	  
	% new div opening: depth here is 1
	
\section[{Īśvarasā\-dhanadū\-ṣaṇam}]{Īśvarasā\-dhanadū\-ṣaṇam}\label{īśvarasādhanadūṣaṇam}\leavevmode\textsuperscript{\rmlatinfont\tiny [pb in]}\label{RNAms_18b}

	  \pstart \footnote{Mikogami\textunderscore Ms 18b1} \edlabel{thakur75-32.4}\label{thakur75-32.4} oṃ namas tā\-rā\-yai |
	\pend
      
	    
	    \stanza[\smallbreak]
sū\-ktaratnā\-śrayatvena jitaratnā\-karā\-d idam |&guror vā\-gambudheḥ smartuṃ kiñcid ā\-kṛṣya likhyate ||\&[\smallbreak]


	
	    
	    \stanza[\smallbreak]
rī\-tiḥ sudhā\-nidhir iyaṃ sattame madhyavartini |&vidveṣiṇi viṣajvā\-lā\- kiñcij jñe tu na kiñcana ||\&[\smallbreak]


	

	  \pstart ihaite naiyā\-yikā\-dayo vivā\-dapadasya kṣitidharā\-deḥ svarū\-popā\-dā\-nopakaraṇasaṃpradā\-naprayojanavibhā\-gapravī\-ṇaṃ sarvajñatā\-diguṇaviśiṣṭaṃ puruṣaviśeṣam icchanti | yad ā\-huḥ
	\pend
      
	    
	    \stanza[\smallbreak]
\edlabel{ratnakīrtinibandhāvali__īśvarakārikā}\flagstanza{\tiny\textenglish{...kā\-rikā\-}}eko vibhuḥ sarvavidekabuddhisamā\-śrayaḥ śā\-śvata ī\-śvarā\-khyaḥ |&pramā\-ṇam iṣto jagato vidhā\-tā\- svargā\-pavargā\-rthibhir arthanī\-yaḥ ||\&[\smallbreak]


	

	  \pstart iti |
	\pend
      

	  \pstart sa ca kathaṃ sidhyatī\-ti paryanuyuktā\-ḥ sā\-dhanam idam ā\-cakṣate |
	\pend
      

	  \pstart vivā\-dā\-dhyā\-sitaṃ buddhimaddhetukam |
	\pend
      

	  \pstart kā\-ryatvā\-t |
	\pend
      

	  \pstart yat kā\-ryaṃ tadbuddhimadhetukam | yathā\- ghaṭaḥ |
	\pend
      

	  \pstart kā\-ryaṃ cedam |
	\pend
      

	  \pstart tasmā\-d buddhimadhetukam iti |
	\pend
      

	  \pstart hetoḥ parokṣā\-rthapratipā\-dakatvam anubhū\-teṣu hetvā\-bhā\-seṣu na śakyam ā\-vedayitum | hetvā\-bhā\-sā\-ś ca pañca | yathoktam
	\pend
      
	    
	    \stanza[\smallbreak]
savyabhicā\-raviruddhaprakaraṇasamasā\-dhyasamā\-tī\-takā\-lā\- iti |\&[\smallbreak]


	

	  \pstart tatra na tā\-vad ayaṃ sā\-dhyasamo hetuḥ | asiddho hi sā\-dhyasamaḥ kathyate | sa ca saṃkṣepato vibhajyamā\-no dvidhā\- vyavatiṣṭhate | ā\-śrayā\-siddhatvā\-d vā\-siddho yathā\- surabhi gaganā\-ravindamaravindatvā\-d iti | saty api cā\-śraye pramā\-ṇena sambandhā\-siddher asiddho yathā\- anityaḥ śabdaḥ sā\-vayavatvā\-d iti | na cā\-bhyā\-ṃ prakā\-rā\-bhyā\-ṃ prastutasya hetor asiddhir asti | \edlabel{ratnakīrtinibandhāvali__36r1N54T2H6NLF3JHCUF8B8TWUQ}\label{ratnakīrtinibandhāvali__36r1N54T2H6NLF3JHCUF8B8TWUQ}kṣmā\-ruhā\-dau dharmiṇi pramā\-ṇasamadhigate kā\-ryatvasya sā\-dhanasya pramā\-ṇpratī\-tatvā\-t | cirotpannaparvatā\-dau ca dharmiṇi kā\-ryatvaṃ sā\-vayavatvena hetunā\- boddhavyam | tad yathā\-: vivā\-dapadaṃ kā\-ryam | sā\-vayavatvā\-t | yat sā\-vayavaṃ tat kā\-ryam | yathā\- vastram | tathā\- cedam | tasmā\-t kā\-ryam iti |
	\pend
      

	  \pstart nanu sā\-vayavatvena hetunā\- dravyā\-ṇā\-m eva kā\-ryatvaṃ sidhyati | na tu tatsamavetā\-nā\-ṃ guṇakarmā\-dī\-nā\-m | teṣā\-m avayavasambandhā\-bhā\-vā\-d iti cet | satyam | teṣā\-ṃ kā\-ryaguṇā\-ditvena hetvantareṇa kā\-ryatvam adhigantavyam | tathā\- hi; 
	\pend
      

	  \pstart janmabhā\-jo vivā\-dā\-dhyā\-sitanityetarasamavā\-yino guṇā\-dayaḥ |
	\pend
      

	  \pstart kā\-ryaguṇā\-ditvā\-t |
	\pend
      

	  \pstart yo yaḥ kā\-ryaguṇā\-diḥ sa sarvas tathā\-, yathā\- ghaṭā\-dirū\-pā\-diḥ |
	\pend
      

	  \pstart tathā\- caite |
	\pend
      

	  \pstart tasmā\-j janmabhā\-jaḥ | iti |
	\pend
      

	  \pstart \edtext{}{\lemma{---}\Afootnote{kā\-ryañca \cite{}; kā\-ryatvaṃ ca \cite{}}} na svakā\-raṇasamavā\-yaḥ, sā\-mā\-nyaviśeṣo vā\- \edtext{}{\lemma{---}\Afootnote{boddhavyaṃ\deletion{ḥ} \cite{}; boddhavyaḥ \cite{}}}, yenā\-sya pradhvaṃsā\-vyā\-pakatvā\-d bhā\-gā\-siddhatā\- syā\-t, kiṃ tu kā\-raṇā\-dhī\-nasvarū\-pamā\-tram | tac ca śabdā\-diṣv iva pradhvaṃsā\-dā\-v api pratyakṣeṇā\-dhigatam iti na tā\-vad ayam asiddho hetuḥ |
	\pend
      

	  \pstart nā\-pi viruddhaḥ | tathā\- \leavevmode\textsuperscript{\rmlatinfont\tiny [19a/RNAms]}\label{RNAms_19a}hi yo vipakṣa eva vartate sa khalu sā\-dhyaviparyayavyā\-pteḥ sā\-dhyaviruddhaṃ sā\-dhayan viruddho 'bhidhī\-yate | yathā\- nityaḥ śabdaḥ kṛtakatvā\-d iti | na cā\-yaṃ tathā\-, prasiddhakartṛkeṣu ghaṭā\-diṣu sapakṣeṣu sadbhā\-vadarśanā\-t |
	\pend
      

	  \pstart \edlabel{thakur75-33.24}\label{thakur75-33.24} nanu \edlabel{sarit__ratnakīrtinibandhāvali__84822}\label{sarit__ratnakīrtinibandhāvali__84822}\edtext{}{\lemma{buddhimatpūrvakatve … tadviśeṣaṇatvānupapatteḥ |}\xxref{sarit__ratnakīrtinibandhāvali__84822}{sarit__ratnakīrtinibandhāvali__86003}\Afootnote{\label{sarit__ratnakīrtinibandhāvali__395787}\textenglish{---\textsc{Note} NK 150--151.}\textenglish{---\textsc{Note} Sanskrit and translation in \cref{krasser02_zaGkar_Izvar_studie}.}  {\rmlatinfont [App type: parallel]}}}buddhimatpū\-rvakatve sā\-dhye siddhasā\-dhanam | abhimataṃ hi pareṣā\-m api karmajatvaṃ kā\-ryajā\-tasya, karmaṇaś ca cetanā\-tmakatvā\-t, cetanā\-hetukatvā\-d vā\- | taddhetukatvaṃ ca jagataḥ | \edlabel{ratnakīrtinibandhāvali__36r1N6ESBB6GEXJBT5XVAMTHRS5}\label{ratnakīrtinibandhāvali__36r1N6ESBB6GEXJBT5XVAMTHRS5}sarvajñapū\-rvakatve tu sā\-dhye vyā\-ptiḥ svapne 'pi \edlabel{ratnakīrtinibandhāvali__36r1N6F0R3CZ6K2E77O6CXTJDTY}\label{ratnakīrtinibandhāvali__36r1N6F0R3CZ6K2E77O6CXTJDTY}\edtext{}{\lemma{nopalabdhā |}\xxref{ratnakīrtinibandhāvali__36r1N6F0R3CZ6K2E77O6CXTJDTY}{ratnakīrtinibandhāvali__36r1N6F0R3EBPFS8O3U3VB2EVLU}\Afootnote{\label{ratnakīrtinibandhāvali__36r1N6F0S0SBC9X1LL8VAVGOA73}nopalabdhā\- | \cite{}; nopalabddhā\- | \cite{}  {\rmlatinfont [App type: typo]}}}nopalabdhā\- |\edlabel{ratnakīrtinibandhāvali__36r1N6F0R3EBPFS8O3U3VB2EVLU}\label{ratnakīrtinibandhāvali__36r1N6F0R3EBPFS8O3U3VB2EVLU} dṛṣṭā\-ntaś ca sā\-dhyahī\-naḥ, kulā\-lā\-dī\-nā\-m asarvajñatvā\-t | viruddhatā\- ca hetor asarvajñapū\-rvakatvenaiva kumbhā\-dau kā\-ryatvasya vyā\-pter upalabdheḥ | na copalabdhimatpū\-rvakatvamā\-traṃ sā\-dhanaviṣayaḥ, tadviśeṣasya tu sarvajñapū\-rvakatvasyā\-tadviṣayasyā\-pi tataḥ siddhir iti sā\-mpratam | tathā\-  hi yady asau viśeṣo na sā\-dhanaviṣayaḥ katham atas tatsiddhiḥ, \edlabel{ratnakīrtinibandhāvali__36r1N6H3ZXWODAWI2R2XNTCQWH5}\label{ratnakīrtinibandhāvali__36r1N6H3ZXWODAWI2R2XNTCQWH5}\edtext{}{\lemma{siddhyan vā}\xxref{ratnakīrtinibandhāvali__36r1N6H3ZXWODAWI2R2XNTCQWH5}{ratnakīrtinibandhāvali__36r1N6H3ZYQH06DWRZ87HBU0W3H}\Afootnote{\label{ratnakīrtinibandhāvali__36r1N6H411992B9T0CM06YGJ2RD}sidhyanvā\- \cite{}; siddhaṃ vā\- \cite{}  {\rmlatinfont [App type: var]}}}sidhyan vā\-\edlabel{ratnakīrtinibandhāvali__36r1N6H3ZYQH06DWRZ87HBU0W3H}\label{ratnakīrtinibandhāvali__36r1N6H3ZYQH06DWRZ87HBU0W3H} katham aviṣayaḥ, viṣayaś cet katham ananvayadoṣaṃ na spṛśed iti cet |
	\pend
      

	  \pstart \edlabel{thakur75-33.32}\label{thakur75-33.32} ucyate | sā\-mā\-nyamā\-travyā\-ptā\-v apy antarbhā\-vitaviśeṣasya sā\-mā\-nyasya pakṣadharmatā\-vaśena sā\-dhyadharmiṇy anumā\-nā\-t viśeṣaviṣayam anumā\-naṃ bhavaty eva | itarathā\- sarvā\-numā\-nocchedaprasaṅgā\-t | tathā\- hi vahnyanumā\-nam api na sā\-mā\-nyamā\-traviṣayam, tasya prā\-g eva siddhatvā\-t | nā\-pi tadviśiṣṭagirigocaram vahnitvasā\-mā\-nyasya tatsambandhā\-bhā\-vena tadviśeṣaṇatvā\-nupapatteḥ |\edlabel{sarit__ratnakīrtinibandhāvali__86003}\label{sarit__ratnakīrtinibandhāvali__86003} itarathā\- gotvasamavā\-yā\-d iva gā\-vaḥ śā\-baleyā\-dayaḥ parvato 'pi vahnitvasamavā\-yā\-d vahniḥ prasajyeta | asty eva girer vahnitvena saṃyuktasamavā\-yaḥ sambandha iti cet | tarhi nā\-pratipadya parvatasaṃyuktaṃ vahniviśeṣam asau śakyapratipattir iti vahniviśeṣasyā\-py ananumā\-nam | tathā\- cā\-nanvayadoṣaprasaṅgaḥ | indriyā\-numā\-ne 'py ayam eva nyā\-yo draṣṭavyaḥ, yathendriyalakṣaṇakaraṇaviśeṣasiddhiḥ | tathā\- hi tatrā\-pi nendriyakaraṇikā\- kā\-cit kriyopalabdhā\- | na khalu \edlabel{ratnakīrtinibandhāvali__36r1NBNCTLXSLLK5Q0BPDLIM95A}\label{ratnakīrtinibandhāvali__36r1NBNCTLXSLLK5Q0BPDLIM95A}\edtext{}{\lemma{cchidādyaḥ}\xxref{ratnakīrtinibandhāvali__36r1NBNCTLXSLLK5Q0BPDLIM95A}{ratnakīrtinibandhāvali__36r1NBNCTLZ1SMUV2GW43CFNXI9}\Afootnote{\label{ratnakīrtinibandhāvali__36r1NBNCUJSJ7ITQZY3I4WX64RH}cchidā\-dyaḥ \cite{} \cite{} \cite{}; cchidā\-dyā\-ḥ \cite{}  {\rmlatinfont [App type: var]}}}cchidā\-dyaḥ\edlabel{ratnakīrtinibandhāvali__36r1NBNCTLZ1SMUV2GW43CFNXI9}\label{ratnakīrtinibandhāvali__36r1NBNCTLZ1SMUV2GW43CFNXI9} kriyā\- \edlabel{ratnakīrtinibandhāvali__36r1NBNDW1G2X6VFZF0NJ3EWJDN}\label{ratnakīrtinibandhāvali__36r1NBNDW1G2X6VFZF0NJ3EWJDN}\edtext{}{\lemma{indriyasādhanā vraścanādīnām}\xxref{ratnakīrtinibandhāvali__36r1NBNDW1G2X6VFZF0NJ3EWJDN}{ratnakīrtinibandhāvali__36r1NBNDW1H5HREMMTOXP3XHCJD}\Afootnote{\label{ratnakīrtinibandhāvali__36r1NBNDWK0VLOSL6051B56H1PP}indriyasā\-dhanā\-vraścanā\-dī\-nā\-m \cite{}; indriyasā\-dhanā\-ḥ, vraścanā\-dī\-nā\-m \cite{}; indriyasā\-dhanā\-ḥ| vraścanā\-dī\-nā\-m \cite{}; indriyasā\-dhanā\-, vraścanā\-dī\-nā\-m \cite{}  {\rmlatinfont [App type: var]}}}indriyasā\-dhanā\-ḥ, \edlabel{ratnakīrtinibandhāvali__36r1NBNGZ65MJHSVBYOUHFQKGUA}\label{ratnakīrtinibandhāvali__36r1NBNGZ65MJHSVBYOUHFQKGUA}\edtext{}{\lemma{vraścanādīnām anindriyatvāt |}\xxref{ratnakīrtinibandhāvali__36r1NBNGZ65MJHSVBYOUHFQKGUA}{ratnakīrtinibandhāvali__36r1NBNGZ66X54IW5MBNP1T2YNQ}\Afootnote{\label{ratnakīrtinibandhāvali__36r1NBNGZ69U89855WFDDRMM450}\textenglish{---\textsc{Note} \cref{SVR}: \begin{sanskrit}paraśvadhā\-dī\-nā\-m anindriyatvā\-t |\end{sanskrit}}  {\rmlatinfont [App type: parallel]}}}vraścanā\-dī\-nā\-m\edlabel{ratnakīrtinibandhāvali__36r1NBNDW1H5HREMMTOXP3XHCJD}\label{ratnakīrtinibandhāvali__36r1NBNDW1H5HREMMTOXP3XHCJD} anindriyatvā\-t |\edlabel{ratnakīrtinibandhāvali__36r1NBNGZ66X54IW5MBNP1T2YNQ}\label{ratnakīrtinibandhāvali__36r1NBNGZ66X54IW5MBNP1T2YNQ} \edlabel{ratnakīrtinibandhāvali__36r1NBNFAY1WZTRAN8OW7C97F9D}\label{ratnakīrtinibandhāvali__36r1NBNFAY1WZTRAN8OW7C97F9D}\edtext{}{\lemma{na … kriyā |}\xxref{ratnakīrtinibandhāvali__36r1NBNFAY1WZTRAN8OW7C97F9D}{ratnakīrtinibandhāvali__36r1NBNFAY37HO66IDDUSZLKZSL}\Afootnote{\label{ratnakīrtinibandhāvali__36r1NBNFAY6REOBTPZAUSZEQ0TV}\textenglish{---\textsc{Note}  has plural: \begin{english}na ca vraścanā\-disā\-dhanā\-ḥ sambhavanti rū\-pā\-diparicchittilakṣaṇā\-ḥ kriyā\-ḥ |\end{english}. But \cref{SVR}: \begin{english}na ca paraśvadhā\-disā\-dhanā\- rū\-pā\-diparicchittirū\-pā\- kriyā\- sambhavati |\end{english}.}  {\rmlatinfont [App type: parallel]}}}na ca vraścanā\-disā\-dhanā\- sambhavati rū\-pā\-diparicchittilakṣaṇā\- kriyā\- |\edlabel{ratnakīrtinibandhāvali__36r1NBNFAY37HO66IDDUSZLKZSL}\label{ratnakīrtinibandhāvali__36r1NBNFAY37HO66IDDUSZLKZSL} tasmā\-d yathā\- kriyā\-tvasā\-mā\-nyasya karaṇamā\-trā\-dhī\-natvavyā\-ptatve pakṣadharmatā\-vaśā\-d indriyalakṣaṇakaraṇaviśeṣasiddhis \edlabel{sarit__ratnakīrtinibandhāvali__86683}\label{sarit__ratnakīrtinibandhāvali__86683}\edtext{}{\lemma{tathehāpi … āpatati |}\xxref{sarit__ratnakīrtinibandhāvali__86683}{sarit__ratnakīrtinibandhāvali__87284}\Afootnote{\label{sarit__ratnakīrtinibandhāvali__396860}---\textsc{Note} Cf. ---\textsc{Note} Sanskrit and translation in \href{krasser02_zaGkar_Izvar_studie}{krasser02\textunderscore zaGkar\textunderscore Izvar\textunderscore studie}  {\rmlatinfont [App type: parallel]}}}tathehā\-pi saty api kā\-ryatvasyopā\-dā\-nopakaraṇasaṃpradā\-naprayojanajñakartṛmā\-travyā\-ptatve 'pi vivā\-dā\-dhyā\-siteṣu pakṣadharmatā\-vaśā\-\leavevmode\textsuperscript{\rmlatinfont\tiny [19b/RNAms]}\label{RNAms_19b}d upā\-dā\-nā\-dyabhijñasā\-mā\-nyasyā\-kṣiptaviśeṣasyaiva siddhiḥ | anyathā\- sā\-mā\-nyasyā\-pi vyā\-pakā\-bhimatasya na siddhiḥ syā\-t, \edlabel{ratnakīrtinibandhāvali__36r1NBN3IACWENPEYV4Q9H7K18R}\label{ratnakīrtinibandhāvali__36r1NBN3IACWENPEYV4Q9H7K18R}\edtext{}{\lemma{nirviśeṣasyāsambhavādviśeṣasya … tasyānupapatteḥ |}\xxref{ratnakīrtinibandhāvali__36r1NBN3IACWENPEYV4Q9H7K18R}{ratnakīrtinibandhāvali__36r1NBN3IBWUTMDSV5RRK1CHVSV}\Afootnote{\label{ratnakīrtinibandhāvali__36r1NBN3IC09L08K3LEKT2N1R0A}nirviśeṣasyā\-sambhavadviśeṣasya vā\- tasyā\-nupapatteḥ | \cite{}\textenglish{---\textsc{Note} Cf. : \begin{english}nirviśeṣasyā\-sambhavā\-t, viśeṣasya cā\-nyasyā\-nupapatteḥ\end{english}.}\textenglish{---\textsc{Note} Cf. \cref{SVR}: \begin{english}nirviśeṣasya tasyā\-nupapatteḥ\end{english}.}  {\rmlatinfont [App type: parallel]}}}nirviśeṣasyā\-sambhavā\-dviśeṣasya vā\- tasyā\-nupapatteḥ |\edlabel{ratnakīrtinibandhāvali__36r1NBN3IBWUTMDSV5RRK1CHVSV}\label{ratnakīrtinibandhāvali__36r1NBN3IBWUTMDSV5RRK1CHVSV} asarvajñasya cā\-trā\-dṛṣṭā\-dibhedavijñā\-nasahitasyā\-dhiṣṭhā\-tṛbhā\-vā\-sambhavā\-t sarvajñā\-tmaka eva viśeṣo balā\-d ā\-patati |\edlabel{sarit__ratnakīrtinibandhāvali__87284}\label{sarit__ratnakīrtinibandhāvali__87284}
	\pend
      

	  \pstart \edlabel{thakur75-34.17}\label{thakur75-34.17}nanū\-pā\-dā\-nā\-dyabhijñakartṛmā\-treṇevā\-sarvajñatvadehitvā\-dibhir api vyā\-ptir aśakyaparihā\-rā\-, vyabhicā\-rā\-darśanasya samā\-natvā\-d iti cet | na | sarvajñatvā\-sarvajñatvayor dehitvā\-dehitvayor vā\- kā\-ryotpattā\-v anupayogā\-t | na hi sā\-rvajñyaṃ kartṛṇā\-m yogyatā\-m upasthā\-payati, asarvajñebhyaḥ kumbhakā\-rā\-dibhyaḥ kumbhā\-dī\-nā\-m aprasavaprasaṅgā\-t | nā\-py asā\-rvajñyaṃ kumbhakā\-rā\-d eva keyū\-rā\-dī\-nā\-m apy utpattiprasaṅgā\-t | tathā\- na dehitvaṃ kā\-ryotpattā\-v upayogi kumbhakā\-rā\-d eva keyū\-rā\-dī\-nā\-m utpattiprasaṅgā\-t | nā\-dehitvaṃ kumbhakā\-rā\-d ghaṭā\-dī\-nā\-m anutpā\-daprasaṅgā\-t | tataś copā\-dā\-nā\-dyabhijñapuruṣapū\-rvakatvam eva kā\-ryatvasya vyā\-pakam | tad eva ca buddhimatpuruṣapū\-rvakatvaśabdavā\-cyam | \edlabel{sarit__ratnakīrtinibandhāvali__88081}\label{sarit__ratnakīrtinibandhāvali__88081}\edtext{}{\lemma{tena … °pratikṣepahetutvāt ||}\xxref{sarit__ratnakīrtinibandhāvali__88081}{sarit__ratnakīrtinibandhāvali__88447}\Afootnote{\label{sarit__ratnakīrtinibandhāvali__397298}---\textsc{Note} Cf. ---\textsc{Note} Sanskrit and translation in \cref{krasser02_zaGkar_Izvar_studie}  {\rmlatinfont [App type: parallel]}}}tena yady api buddhimatpū\-rvakatvamā\-traṃ vyā\-ptiviṣayas tathā\-pi tadviśeṣasya sarvajñatvasya pakṣadharmatā\-balā\-t pratilambha iti viśeṣaviṣayam anumā\-nam | na coktadoṣaprasaṅgaḥ, tasya sā\-dhyadṛṣṭā\-ntayor dharmavikalpā\-d utkarṣā\-pakarṣalakṣaṇaparyanuyogasya sarvā\-numā\-nasā\-dhā\-raṇyenā\-numā\-namā\-traprā\-mā\-ṇyapratikṣepahetutvā\-t ||\edlabel{sarit__ratnakīrtinibandhāvali__88447}\label{sarit__ratnakīrtinibandhāvali__88447}
	\pend
      

	  \pstart etena yad uktaṃ kaṇikā\-yā\-ṃ \edlabel{nk-150.9}\label{nk-150.9}yadi kulā\-lā\-dī\-nā\-ṃ katipayopakaraṇā\-dijñā\-nam, na samastopakaraṇā\-dijñatā\-, tarhi tenaiva nidarśanena ī\-śvarasyā\-pi \edlabel{ratnakīrtinibandhāvali__36r1NLOHWTQ70VHBU3NJZWC5S9G}\label{ratnakīrtinibandhāvali__36r1NLOHWTQ70VHBU3NJZWC5S9G}\edtext{}{\lemma{tadupakaraṇādimātrajñānam | … sarvajñatāsiddhiḥ |}\xxref{ratnakīrtinibandhāvali__36r1NLOHWTQ70VHBU3NJZWC5S9G}{ratnakīrtinibandhāvali__36r1NLOHWTQZXFC6755VVL7IPZO}\Afootnote{\label{ratnakīrtinibandhāvali__36r1NLOI13GSGSHMPIZGYJ8Z7E2}tadupakaraṇā\-dimā\-trajñā\-nam | tanmā\-trajñā\-ne na sarvajñatā\-siddhiḥ \cite{}; tadupakaraṇā\-dimā\-trajñā\-naṃ tanmā\-trajñā\-ne ca na sarvajñatā\-siddhiḥ \cite{}; tadupakaraṇā\-dimā\-trajñā\-ne na sarvajñatā\-siddhiḥ \cite{}  {\rmlatinfont [App type: var]}}}tadupakaraṇā\-dimā\-trajñā\-nam | tanmā\-trajñā\-ne na sarvajñatā\-siddhiḥ |\edlabel{ratnakīrtinibandhāvali__36r1NLOHWTQZXFC6755VVL7IPZO}\label{ratnakīrtinibandhāvali__36r1NLOHWTQZXFC6755VVL7IPZO} katipayajño hi tathā\- sati syā\-t |
	\pend
      

	  \pstart na vā\- tanmā\-trajñā\-nam apī\-śvarasya \edtext{bālādivad ity āha | bālonmattādīnāṃ}{\Afootnote{bā\-lonmattā\-dī\-nā\-ṃ \cite{}; bā\-lā\-divad ity ā\-ha | bā\-lonmattā\-dī\-nā\-ṃ \cite{}; bā\-lā\-divad ity aha---bā\-lonmattā\-dayaśca \cite{}  {\rmlatinfont [App type: em]}}} svakā\-ryaprayojanā\-parijñā\-ne 'pi nirabhiprā\-yā\-ṇā\-ṃ tatra tatra pravṛttidarśanā\-t | na ca kulā\-lā\-dayo nidarśanaṃ na bā\-lā\-daya ity atra niyamahetur astī\-\edlabel{nk-150.14}\label{nk-150.14}ti tan nirastam ||\edtext{}{\lemma{---}\Afootnote{ \cite{}---\textsc{Note} Cf.   {\rmlatinfont [App type: parallel]}}}
	\pend
      

	  \pstart ī\-śvarasya hi katipayā\-tī\-ndriyopakaraṇā\-dijñā\-ne tatkā\-raṇasya sarvatra samā\-natvā\-d aśeṣopakaraṇā\-dijñatā\-yā\- durvā\-ratvā\-t | kā\-raṇam ca tajjñā\-ne sattā\-m antareṇa nā\-nyat, dharmā\-dharmā\-dī\-nā\-ṃ laukikapratyā\-sattihetū\-nā\-ṃ tatrā\-sambhavā\-t | kā\-raṇā\-bhede ca kā\-ryā\-bhedaḥ | anyathā\- katipayā\-tī\-ndriyajñā\-nam api na syā\-t | yathā\- hi \edtext{kulālādis}{\Afootnote{kulā\-lā\-dis \cite{}; ku\leavevmode\textsuperscript{\rmlatinfont\tiny [20a/RNAms]}\label{RNAms_20a}lā\-lā\-das \cite{}  {\rmlatinfont [App type: em]}}} tulyadarśanasā\-magrī\-keṣu nā\-kiñcij\edtext{jñaḥ}{\Afootnote{jñaḥ \cite{}; jñā\-ḥ \cite{}  {\rmlatinfont [App type: var]}}} tathā\-tī\-ndriyopakaraṇā\-diṣv apī\-śvaraḥ, sā\-marthyasyā\-viśeṣā\-t | na ca bā\-lonmattā\-dinidarśanena katipayopakaraṇajñatā\-niṣedho yuktaḥ, bī\-jadṛṣṭā\-ntena buddhimanmā\-trasyā\-pi niṣedhā\-bhidhā\-naprasaṅgā\-t | tasmā\-d yathopā\-dā\-nā\-dyabhijñasyā\-pi sambhavā\-d bī\-jā\-dibhir na vyabhicā\-rā\-bhidhā\-nam, tathā\- bā\-lonmattā\-dibhir apī\-ti kulā\-lā\-dī\-nā\-m eva dṛṣṭā\-ntatā\- yuktimatī\-, upā\-dā\-nā\-dyabhijñabuddhivanmā\-trakā\-ryatvayoḥ sā\-dhyasā\-dhanayos tatra prasiddhatvā\-t | tathā\- jñā\-navad ī\-śvarasya cikī\-rṣā\-prayatnau nityā\-v ity atrā\-pi |
	\pend
      

	  \pstart yad abhihitam: nityau cet kim ī\-śvarasya jñā\-nena cikī\-rṣā\-prayatnopayoginā\-, tayor nityatvā\-t, svotpā\-dopayogā\-napekṣaṇā\-dityā\-di | tad apy asā\-ram | ajñā\-takartṛtvā\-nupaptteḥ | jñā\-naṃ hi yatra cikī\-rṣā\-pratyatnā\-v anityau tatra tā\-v upasthā\-payadupakaraṇā\-dikam upadarśayati | yatra tu tau nityau tatropakaraṇā\-dikam upadarśayad api saphalam | tasmā\-t saty api cikī\-rṣā\-pratyatnayor nityatve saphalam ī\-śvarajñā\-naṃ sā\-kṣā\-tkā\-ryopattā\-v anupayogy api | ata eva ca so 'yam ī\-dṛśo viśeṣo vicā\-rā\-sahaḥ kathaṃ pakṣadharmatā\-balā\-d api sā\-dhyadharmiṇy upasaṃhriyata ityā\-dir api pralā\-pa eva | ī\-śvarajñā\-nasyā\-vyā\-hatau sarvajñatā\-viśeṣasya durvā\-ratvā\-t |
	\pend
      

	  \pstart yad abhihitam: \edlabel{nk_3522_start}\label{nk_3522_start}prekṣā\-vatā\-ṃ pravṛttiḥ prayojanavattayā\- vyā\-ptā\- | na ceśvarasya prekṣā\-vato jagannirmā\-ṇe prayojanam utpaśyā\-maḥ, prā\-ptanikhilaprā\-paṇī\-yasya prā\-ptavyā\-bhā\-vā\-t |\edlabel{nk_3522_end}\label{nk_3522_end}\edtext{}{\lemma{---}\Afootnote{---\textsc{Note} Corresponds to NK   {\rmlatinfont [App type: ce2]}}} tad api sā\-vadyam, tadabhiprā\-yasya durbodhatvā\-t, prayojanā\-bhā\-vā\-siddheḥ, vyā\-pakā\-nupalabdheḥ, sandigdhatvā\-t | vicitrā\- hi puruṣamā\-trasya cetovṛttiḥ prā\-g eva viśvasya kartuḥ | prā\-ptanikhilaprā\-paṇī\-yasyā\-pi karuṇayā\-pi parā\-rtha\edtext{pravṛtteḥ}{\Afootnote{pravṛtteḥ \cite{}; pravṛttaḥ \cite{}  {\rmlatinfont [App type: var]}}} sambhā\-vyamā\-natvā\-t | na cā\-sya narakā\-dinirmā\-ṇapravṛttiḥ kā\-ruṇikatā\-m upahanti, pratyuta pituḥ putragaṇḍapā\-ṭanavṛttir ivā\-lpaduḥkhadā\-nena prabhū\-tadā\-ruṇaduḥkhā\-panayanā\-t karuṇā\-tiśayam eva gamayati | prekṣā\-vatā\-m ivā\-syā\-pi niyatasthirapravṛttisiddheḥ prayojanā\-numitir eva nyā\-yaprā\-ptā\- ||
	\pend
      

	  \pstart yac cedam udī\-ritam: yadi hi sarvakā\-ryā\-ṇā\-m ekaḥ kartā\- syā\-t tato 'jñasya tattvā\-nupapatteḥ sarvajñatā\- syā\-t | \edtext{adya}{\Afootnote{adya \cite{}; atha \cite{}  {\rmlatinfont [App type: var]}}} punar ekaikaṃ kā\-ryam ekaikena kartrā\- \leavevmode\textsuperscript{\rmlatinfont\tiny [20b/RNAms]}\label{RNAms_20b} janyata iti yo yaj janayati sa tatkā\-raṇamā\-trajña eva na tu sarvajña iti |
	\pend
      

	  \pstart atrocyate | \edlabel{ratnakīrtinibandhāvali__36r1NSAQKYMBR7UQH8UZ9ZCFB5G}\label{ratnakīrtinibandhāvali__36r1NSAQKYMBR7UQH8UZ9ZCFB5G}kā\-ryaliṅgā\-viśeṣā\-d ekaḥ kartā\- \edlabel{ratnakīrtinibandhāvali__36r1NSAS7AJHGG1XWSGMMV9343A}\label{ratnakīrtinibandhāvali__36r1NSAS7AJHGG1XWSGMMV9343A}sad iti jñā\-nā\-viśeṣā\-t sattaikatvavat | kutaścil liṅgā\-d anumitasya vastuno nā\-nā\-tvasya liṅgā\-ntarā\-numeyatvā\-t, nā\-nā\-tvam upapā\-dayituṃ pramā\-ṇā\-ntaraṃ vaktavyam | yathā\-tmanā\-nā\-tvam avasthā\-payadbhiḥ \edtext{sukhādibhir nānātvavyavasthāpanam}{\Afootnote{sukhā\-divyavasthā\-panam \cite{}; sukhā\-dibhir nā\-nā\-tvavyavasthā\-panam \cite{}---\textsc{Note} It could be that there was a correction in the ms. The upper margin has some signs here (around three akṣaras), but they are completely illegible. Also, there is a dot in the middle above di and vya, which could have been a mark to insert something here.  {\rmlatinfont [App type: var]}}} ucyate | na ceha kartur anekatvā\-dhigame pramā\-ṇā\-ntaram asti | \edlabel{ratnakīrtinibandhāvali__36r1NSAZCVOOMC2YM2ZB4UM8USC}\label{ratnakīrtinibandhāvali__36r1NSAZCVOOMC2YM2ZB4UM8USC}ekatve tu na pramā\-ṇā\-taram anveṣṭavyam, ekasya kartur abhā\-ve \edlabel{ratnakīrtinibandhāvali__36r1NMMCGE4QHCX673V3O13PAR2}\label{ratnakīrtinibandhāvali__36r1NMMCGE4QHCX673V3O13PAR2}\edtext{}{\lemma{bahūnāṃ … syād}\xxref{ratnakīrtinibandhāvali__36r1NMMCGE4QHCX673V3O13PAR2}{ratnakīrtinibandhāvali__36r1NMMCGE62YQPZ3MBLV5PA87V}\Afootnote{\label{ratnakīrtinibandhāvali__36r1NMMCGE84C0Q7R96K5FVFGL5}\textenglish{---\textsc{Note} Corresponds to 269.}  {\rmlatinfont [App type: parallel]}}}bahū\-nā\-ṃ vyā\-hatamanasā\-ṃ \edtext{svātantryeṇa}{\Afootnote{svā\-tantryeṇa \cite{}; svā\-tantreṇa \cite{}  {\rmlatinfont [App type: var]}}} parasparavirodhena mithaḥ svā\-nukū\-lā\-bhiprā\-yā\-navabodhena yugapatkā\-ryā\-nutpattiḥ, utpannasya vā\- vilopā\-diprasaṅgaḥ syā\-d\edlabel{ratnakīrtinibandhāvali__36r1NMMCGE62YQPZ3MBLV5PA87V}\label{ratnakīrtinibandhāvali__36r1NMMCGE62YQPZ3MBLV5PA87V} iti | ekatve tu siddhe sarvajñatā\-siddhir avirodhinī\- | na ceśvarasya sakalakṣetrajñasamavā\-yidharmā\-dharmajñā\-nakā\-raṇā\-bhā\-vena tadajñā\-nam, tatsamavetā\-nā\-ṃ jñā\-nacikī\-rṣā\-pratyatnā\-nā\-ṃ nityatvā\-t | na ca buddhitannityatvayoḥ kaścit virodhaḥ | na ca buddher anityatā\-yā\-s tatra tatropalabdher ī\-śvarabuddher api tathā\-tvaṃ yuktam, rū\-pā\-dī\-nā\-m apy anityā\-nā\-ṃ tatra tatropalabdhes toyā\-diparamā\-ṇusamavetā\-nā\-m api rū\-pā\-dī\-nā\-m anityatvaprasaṅgā\-t | parapuruṣasamavetadharmā\-dharmā\-dhiṣṭhā\-nam apy asya yuktam eva, saṃyuktasaṃyogisamavā\-yasya sambandhasya sadbhā\-vā\-t | saṃyuktā\-ḥ khalv ī\-śvareṇa paramā\-ṇavaḥ, taiś ca kṣetrajñā\-ḥ, tatsamavetau ca dharmā\-dharmā\-v iti ||
	\pend
      

	  \pstart tad evaṃ kaṇikā\-yā\-ṃ vā\-caspater ī\-śvaradū\-ṣaṇaṃ yathā\-sā\-ram utthā\-pya vyudastam asmā\-bhiḥ | aparaṃ ca busaprā\-yam anabhyupagamaprasiddhasiddhā\-ntagrastam iha granthavistarabhayā\-n na likhitam | tad evam abhimatasyaiva sarvajñatā\-lakṣaṇasya viśeṣasya siddher naiṣa viśeṣaviruddho hetuḥ | nā\-pi karmabhiḥ siddhasā\-dhanam iti sthitam || 
	\pend
      

	  \pstart \edlabel{thakur75-36.21}\label{thakur75-36.21} na cā\-naikā\-ntikaḥ | sa hi bhavann asā\-dhā\-raṇo vā\- syā\-t, yathā\- nityā\- pṛthvī\- gandhavattvā\-d iti, anupasaṃhā\-ryo vā\-, yathā\- sarvaṃ nityaṃ prameyatvā\-d iti, sā\-dhā\-raṇo vā\- yathā\- nityaḥ śabdaḥ, asparśavattvā\-d iti |
	\pend
      

	  \pstart tatra na tā\-vad ā\-dimau pakṣau, sapakṣasadbhā\-vadarśanena pratikṣiptatvā\-t | nā\-py antimaḥ, adhigatakartṛnivṛtter vyomā\-der vipakṣā\-d vyā\-vṛtter upalabdheḥ | 
	\pend
      

	  \pstart nanu puruṣavyā\-pā\-ram antareṇa tṛṇā\-dī\-n udayamā\-nā\-navalokayan lokaḥ kā\-ryamā\-traṃ puruṣapū\-rvakam iti vyā\-ptim eva na pratipadyata iti cet | evaṃ tarhi prasiddhā\-numā\-nasthitir api dattajalā\-ñjaliḥ | tatrā\-pi hi vyā\-ptipratī\-tikā\-la eva vyā\-ghrā\-\leavevmode\textsuperscript{\rmlatinfont\tiny [21a/RNAms]}\label{RNAms_21a}diparyā\-kulā\-tidurgapradeśe vahnivyā\-pā\-ram antareṇa dhū\-maṃ puruṣavyā\-pā\-raṃ vinā\- pū\-rvaṃ siddhaṃ ghaṭaṃ vā\- vilokayan loko dhū\-mamā\-traṃ vahnipū\-rvakaṃ ghaṭamā\-traṃ vā\- puruṣapū\-rvakam iti vyā\-ptim eva na pratipadyata iti vaktuṃ śakyatvā\-t |
	\pend
      

	  \pstart tatra vahnipuruṣayor deśakā\-laviprakṛṣṭatvā\-d apratikṣepa iti cet | yady evaṃ tṛṇā\-dā\-v api puruṣasya svabhā\-vaviprakṛṣṭatvā\-d apratikṣepa iti sarvaṃ samā\-nam anyatrā\-bhiniveśā\-t | puruṣavyā\-pā\-rapū\-rvakatā\- tā\-van na pratī\-yate tṛṇā\-dī\-nā\-m | sā\- ca puruṣasyā\-dṛśyatvā\-d asattvā\-d vā\- na pratī\-yatā\-m, kim anena vicā\-ritena | sarvathā\- kiñcitkā\-ryam apū\-rvapuruṣapū\-rvakam apaśyan na vyā\-ptiṃ kā\-ryamā\-trasya puruṣeṇa kaścit cetanā\-vā\-n avagacchatī\-ti cet | yady evaṃ vahnimā\-trapū\-rvakatā\- tā\-van na pratī\-yate dhū\-masya, puruṣamā\-trapū\-rvakatā\- ca ghaṭasya | sā\- ca vahner deśaviprakṛṣṭatvā\-d asattvā\-d vā\- puruṣasya kā\-laviprakṛṣṭatvā\-d asattvā\-d vā\- na pratī\-yatā\-m, kim anena vicā\-ritena | sarvathā\- dhū\-mamā\-traṃ vahnivyā\-pā\-rapū\-rvakam apaśyan ghaṭamā\-traṃ vā\- puruṣapū\-rvakam apaśyann avyā\-ptim eva dhū\-masya vahnimā\-treṇa ghaṭasya puruṣamā\-treṇa vā\- kaścic cetanā\-vā\-n adhigacchatī\-ty apy ucyamā\-naṃ na vaktraṃ vakrī\-karoti | tat kim anena prasiddhā\-numā\-nā\-palā\-pinā\- jā\-tyuttareṇa ||
	\pend
      

	  \pstart \edlabel{rnā__96541}\label{rnā__96541}\edtext{}{\lemma{syād etat |  ... prabodhāśrayāyattatāsiddheḥ |}\xxref{rnā__96541}{rnā__97385}\Afootnote{\label{rnā__394381}See JNĀ 235. \cite{}  {\rmlatinfont [App type: parallel]}}}syā\-d etat | na sapakṣā\-sapakṣayor darśanā\-darśanamā\-treṇā\-vyabhicā\-raniścayaḥ, atadā\-tmano 'tadutpatteś cā\-vyabhicā\-raniyamā\-bhā\-vā\-t | tad idaṃ kā\-ryatvaṃ sandigdhavipakṣavyā\-vṛttikatvenā\-sā\-dhanam |
	\pend
      

	  \pstart atrocyate | nā\-sti vipakṣā\-d dhetor vyā\-vṛttisandehaḥ, dhū\-mā\-nalayor iva kā\-ryabuddhimator upalambhā\-nupalambhasā\-dhanasya \edtext{kāryakāraṇabhāvasya siddha}{\Afootnote{ \cite{} \cite{}kā\-ryakā\-raṇasiddha \cite{}}}tvā\-t |
	\pend
      

	  \pstart kā\-ryaviśeṣasyaiva tadutpā\-dasiddhir na kā\-\gap{}ryasā\-mā\-nyasya, yathā\- dhū\-mā\-divartino vastutvā\-der nā\-nalā\-dijanyatvaniścaya iti cet | na | viśeṣahetvabhā\-vā\-t | upalambhā\-nupalambhayos tadutpattisā\-dhanatveneṣṭayor aviśeṣā\-t kā\-ryaviśeṣasyeva kā\-ryasā\-mā\-nyasya prabodhā\-śrayā\-yattatā\-siddheḥ |\edlabel{rnā__97385}\label{rnā__97385} yathā\- hi kā\-ryaṃ vastrā\-dyupā\-dā\-navad \edtext{dṛṣṭam iti}{\Afootnote{ \cite{}dṛṣṭaṃ \cite{}---\textsc{Note} This is also parallel to the tathā\- part.  {\rmlatinfont [App type: emendation]}}} kā\-ryā\-ntaram apy adṛṣṭopā\-dā\-nam upā\-dā\-navat \edtext{kāryatvād vyavasthāpyate}{\Afootnote{ \cite{}kā\-ryatvā\-dy upasthā\-pyate \cite{}}}, tathā\- tad eva kā\-ryaṃ vastrā\-di dṛṣṭakartṛkam ity adṛṣṭakartṛkam api kā\-ryatvā\-t kartṛmad vyavasthā\-pyate | upā\-dā\-nasyeva kartur api kā\-ryeṇā\-nukṛtā\-nvyavyatirekatvā\-t | tanmā\-tranibandhanatvā\-c ca sarvatra kā\-ryakā\-raṇavyavahā\-rayoḥ | tasmā\-d yathā\- kā\-rya\leavevmode\textsuperscript{\rmlatinfont\tiny [21b/RNAms]}\label{RNAms_21b}ṃ ca syā\-n nirupā\-dā\-naṃ ceti na śakyam ā\-śaṅkitum, kā\-ryamā\-trasyopā\-dā\-namā\-trā\-d utpā\-dasiddheḥ tathā\- kā\-ryaṃ ca bhaved akartṛkaṃ ceti nā\-śaṅkanī\-yam, kā\-ryamā\-trasya kartṛmā\-trā\-d utpā\-dasiddher aviśeṣā\-t ||\edlabel{thakur-75-37.26}\label{thakur-75-37.26}\footnote{\cref{thakur75-37.12} to \cref{thakur-75-37.26} corresponds to . The passage is introduced by ‘Vittokas tv ā\-ha’.}
	\pend
      

	  \pstart nanu brū\-yā\- nā\-ma kiñcit | tathā\-pi na kā\-ryamā\-trā\-d buddhimadanumā\-nam, api tu kā\-ryaviśeṣā\-d eva | yaddarśanā\-d akriyā\-darśino 'pi kṛtabuddhiḥ syā\-t | na cā\-napekṣitatattvā\-nugamā\-c chabdamā\-trasā\-myā\-t sā\-dhyasiddhir yuktā\- | gośabdavā\-cyatā\-mā\-treṇa vā\-gā\-dī\-nā\-ṃ viṣā\-ṇitvā\-numitiprasaṅgā\-d iti cet | tad etat svasthottaram anuttarā\-rham, kā\-ryasā\-mā\-nyasyaiva vyā\-ptiprasā\-dhanā\-t | api ca kā\- punar iyaṃ kṛtabuddhiḥ, kim apekṣitaparavyā\-pā\-rā\-vasā\-yo 'tha puruṣakṛtam etad iti pauruṣeyatvaniścaya iti |
	\pend
      

	  \pstart yady ā\-dyaḥ pakṣaḥ, sa kathaṃ kṣityā\-diṣu nā\-sti, kā\-raṇavyā\-pā\-rā\-tmalā\-bhalakṣaṇasya kā\-ryatvasya kumbhā\-divat kṣityā\-diṣv aviśeṣā\-t | atha puruṣeṇa kṛtam iti pauruṣeyatvaniścayaḥ kṛtabuddhir abhimatā\-, tadā\-pi tā\-dṛśī\- kṛtabuddhiḥ kasya nā\-stī\-ti vaktavyam | kiṃ kā\-ryatvā\-d iti hetor avinā\-bhā\-vavedina ā\-hosvit tadviparī\-tasya | nā\-dyaḥ pakṣaḥ | avinā\-bhā\-vavedinaḥ sā\-dhyā\-pratipatter ayogā\-t | atha tadviparī\-tasya sā\-dhyabuddhir na bhavatī\-ti kṛtabuddhihetukatvam avanitanumahī\-ruhā\-diṣu nā\-stī\-ti buddhimato 'numā\-naṃ pratikṣipyate |
	\pend
      

	  \pstart nanv evaṃ sati sarvā\-numā\-nocchedaḥ syā\-t | sarvahetū\-nā\-m agṛhī\-tā\-vinā\-bhā\-vaṃ praty \edtext{agamakatvāt}{\Afootnote{agama\add{ka}tvā\-t \cite{}; agamakatvā\-t \cite{}}} | tasmā\-n na \edlabel{rnā__99759}\label{rnā__99759}\edtext{}{\lemma{kṛtabuddhihetutvaviśeṣaḥ}\xxref{rnā__99759}{rnā__99814}\Afootnote{\label{rnā__393999}kṛtabuddhihetutvaviśeṣaḥ \cite{}; kṛtabuddhihetutvaṃ viśeṣaḥ \cite{}---\textsc{Note} Emend to kṛtabuddhihetu-ka-tvaviśeṣaḥ?  {\rmlatinfont [App type: var]}}}kṛtabuddhihetutvaviśeṣaḥ\edlabel{rnā__99814}\label{rnā__99814} | bhavatu vā\- kaścid anirū\-pitarū\-po viśeṣas tathā\-pi kim anena | kā\-ryamā\-trasyaiva dhū\-mamā\-trasyeva vyā\-ptipratī\-teḥ | na ca kā\-ryatvena hetunā\- saha mṛdvikā\-rasya samakakṣatā\- | tasya svasā\-dhyena dṛśyakumbhakā\-reṇa saha vyabhicā\-rasya śataśo darśanā\-t | kā\-ryatvasya tu dṛśyā\-dṛśyasā\-dhā\-raṇena buddhimanmā\-treṇa tadyogā\-d iti nā\-yam anaikā\-ntikaḥ |
	\pend
      

	  \pstart nā\-pi prakaraṇasamaḥ, apratipakṣatvā\-t | na hy asya pratipakṣopasthā\-pakaṃ dharmā\-ntaram asti | \edlabel{RNAms-add-1-start}\label{RNAms-add-1-start}yathā\- nityaḥ śabdo vastutve saty anupalabhyamā\-\edtext{nānityadharmatvād}{\Afootnote{ \cite{}nanityadharmatvā\-d \cite{}---\textsc{Note} Cf. \href{thakur97:_gautam_with_bhAsy_vAtsy}{thakur97:\textunderscore gautam\textunderscore with\textunderscore bhAsy\textunderscore vAtsy} for a similar thought.  {\rmlatinfont [App type: em]}}} ity asya, anityaḥ śabdo vastutve saty anupalabhyamā\-nanityadharmatvā\-d iti pratipakṣakṛtaṃ dharmā\-ntaram asti |\edlabel{RNAms-add-1-end}\label{RNAms-add-1-end} na cedaṃ bā\-dhakaṃ vaktavyam | neśvarakartṛkaṃ jagat | vastutvasattvā\-d ityā\-di | ī\-śvarakartṛkatvasya vastutvā\-d iti virodhā\-bhā\-vā\-t | iti nā\-yaṃ prakaraṇasamo 'pi |
	\pend
      

	  \pstart na ca kā\-lā\-tyayā\-padiṣṭaḥ pratyakṣā\-numā\-nā\-gamair bā\-dhitaviṣayasya tathā\-bhā\-vā\-t | asya ca tair avirodhā\-t | tatra pratyakṣaviruddhaḥ, anuṣṇas tejo'vayavī\- kṛtakatvā\-t | anumā\-navi\leavevmode\textsuperscript{\rmlatinfont\tiny [22a/RNAms]}\label{RNAms_22a}ruddhaḥ, sā\-vayavā\-ḥ paramā\-ṇavo mū\-rtatvā\-t | ā\-gamaviruddhaḥ, śucina〔ra〕śiraḥkapā\-laṃ prā\-ṇyaṅgatvā\-d iti | tatra na tā\-vad ayaṃ pratyakṣaviruddhaḥ, sā\-dhyaviparyayasya pratyakṣā\-viṣayatvā\-t | nā\-py anumā\-naviruddhaḥ, dharmigrā\-hiṇā\-numā\-nenā\-bā\-dhitaviṣayatvā\-t | na cā\-gamaviruddhaḥ, ā\-gamena sā\-dhyaviparyayasyā\-paricchedā\-t | saugatā\-dyā\-gamair viparī\-taparicchedā\-d iti cet | na, teṣā\-ṃ kṣaṇikatvā\-dyarthavisaṃvā\-dopalambhena prā\-mā\-ṇyā\-bhā\-vā\-t | vedā\-gamo 'pi bā\-dhakatvena nā\-śaṅkanī\-yaḥ,
	\pend
      
	    
	    \stanza[\smallbreak]
sahasraśī\-rṣā\- puruṣaḥ \&[\smallbreak]


	

	  \pstart ityā\-dinā\- tatra kartur eva pratipā\-danā\-t | tathā\-bhū\-tapuruṣā\-tiśayapū\-rvakatvā\-bhā\-ve satyaprā\-mā\-ṇyā\-c ceti nā\-yam atikrā\-ntakā\-lo hetuḥ | tad evam apanī\-tahetvabhā\-savibhramā\-d ataḥ sā\-dhanā\-d upā\-dā\-nā\-dyabhijño buddhimā\-n abhimataḥ kartā\- sidhyati | sa  eva bhagavā\-n asmā\-kam ī\-śvara iti sthitam ||
	\pend
      

	  \pstart \edlabel{sarit__ratnakīrtinibandhāvali__102493}\label{sarit__ratnakīrtinibandhāvali__102493}\edtext{}{\lemma{tathāsya … tatheti |}\xxref{sarit__ratnakīrtinibandhāvali__102493}{sarit__ratnakīrtinibandhāvali__103091}\Afootnote{\label{sarit__ratnakīrtinibandhāvali__397794}---\textsc{Note} Sanskrit and translation in \cref{krasser02_zaGkar_Izvar_studie}  {\rmlatinfont [App type: parallel]}}}tathā\-sya siddhaye śaṅkaraḥ sā\-dhanam idam abhipraiti—
	\pend
      

	  \pstart jagad etat prabodhā\-śrayā\-yattaprasavam \edlabel{rnā__102330}\label{rnā__102330}\edtext{}{\lemma{abhilāpa}\xxref{rnā__102330}{rnā__102368}\Afootnote{\label{rnā__394697}abhilā\-pa \cite{}; abhilā\-ṣa \cite{}  {\rmlatinfont [App type: var]}}}abhilā\-ṣa\edlabel{rnā__102368}\label{rnā__102368}prī\-tiparamā\-ṇumū\-rtyā\-dhā\-raparatvā\-paratvā\-numeyasā\-mā\-nyasamavā\-yā\-ntyaviśeṣatadekā\-rthasamavetaparimā\-ṇaikatvapṛthaktvagurutvasnehā\-pā\-rthivarū\-parasasparśā\-pyadravatvā\-mū\-rtasaṃyogataditaretarā\-bhā\-vā\-nutpattirū\-pā\-rū\-pam asmadā\-divinirmitetarat |
	\pend
      

	  \pstart acetanopā\-dā\-natvā\-t |
	\pend
      

	  \pstart yad itthaṃ tat tathā\-, yathā\- kalasaḥ |
	\pend
      

	  \pstart tathā\- cedam |
	\pend
      

	  \pstart tasmā\-d idam api tatheti |\edlabel{sarit__ratnakīrtinibandhāvali__103091}\label{sarit__ratnakīrtinibandhāvali__103091}
	\pend
      

	  \pstart asyā\-yam arthaḥ | jagad iti dharmī\- | prabodhā\-śrayā\-yattaprasavam iti sā\-dhyam | \edlabel{rnā__102859}\label{rnā__102859}\edtext{}{\lemma{abhilāpe}\xxref{rnā__102859}{rnā__102897}\Afootnote{\label{rnā__395000}abhilā\-pe \cite{}; abhilā\-ṣe \cite{}  {\rmlatinfont [App type: var]}}}abhilā\-ṣe\edlabel{rnā__102897}\label{rnā__102897}tyā\-dy anutpattirū\-pā\-rū\-paparyantena dharmiviśeṣeṇā\-kā\-śā\-dinityavargaparihā\-raḥ | asmadā\-divinirmitetarad ity anenā\-pi dharmiviśeṣeṇa prasiddhakartṛkaghaṭā\-diparihā\-raḥ | \edlabel{rnā__103110}\label{rnā__103110}\edtext{}{\lemma{abhilāpaś}\xxref{rnā__103110}{rnā__103149}\Afootnote{\label{rnā__395215}abhilā\-paś \cite{}; abhilā\-ṣaś \cite{}; \texttibetan{not checked} \cite{}  {\rmlatinfont [App type: var]}}}abhilā\-paś\edlabel{rnā__103149}\label{rnā__103149} ca prī\-tiś ca paramā\-ṇumū\-rtiś ca | \edtext{āsām ādhāraḥ |}{\Afootnote{ \cite{}ā\-sā\-madhā\-ra  \cite{}}}ā\-kā\-śa ā\-tmā\- paramā\-ṇuḥ | paratvā\-paratvā\-numeyau dikkā\-lau | sā\-mā\-nyā\-dayas tu yathā\-prasiddhā\- grahī\-tavyā\-ḥ |
	\pend
      

	  \pstart \edlabel{sarit__ratnakīrtinibandhāvali__103807}\label{sarit__ratnakīrtinibandhāvali__103807}\edtext{}{\lemma{tathā … tatheti |}\xxref{sarit__ratnakīrtinibandhāvali__103807}{sarit__ratnakīrtinibandhāvali__104267}\Afootnote{\label{sarit__ratnakīrtinibandhāvali__398247}---\textsc{Note} Sanskrit and translation in \cref{krasser02_zaGkar_Izvar_studie}  {\rmlatinfont [App type: parallel]}}}tathā\- narasiṃhaḥ prā\-ha—
	\pend
      

	  \pstart vijñā\-nā\-dhā\-rā\-dhī\-na\edtext{janmājanmā}{\Afootnote{jan\add{mā\-}janmā\- \cite{}}}vacchinnā\-tmobhayavā\-dyavivā\-dā\-spadapuruṣapū\-rvakavyatireki bhā\-vā\-nubhā\-vi prameyajā\-tam |
	\pend
      

	  \pstart utpattimattvā\-t |
	\pend
      

	  \pstart yad yad ā\-khyā\-tasā\-dhanasambandhi tat tad uktasā\-dhyadharmā\-dhikaraṇam | yathā\- vā\-saḥ |
	\pend
      

	  \pstart tathā\- cedam |
	\pend
      

	  \pstart tasmā\-d idam api tatheti |\edlabel{sarit__ratnakīrtinibandhāvali__104267}\label{sarit__ratnakīrtinibandhāvali__104267}
	\pend
      

	  \pstart asyā\-yam arthaḥ | prameyajā\-taṃ dharmi | vijñā\-nā\-dhā\-rā\-dhī\-najanmeti sā\-dhyam | ajanmā\-vacchinnā\-tmeti dharmiviśeṣaṇam | etenā\-kā\-śā\-dinityavargaparihā\-raḥ | ubhayavā\-dyavivā\-dā\-spadapuruṣapū\-rvakavya\leavevmode\textsuperscript{\rmlatinfont\tiny [22b/RNAms]}\label{RNAms_22b}tirekī\-ty anenā\-pi prasiddhakartṛkaghaṭā\-diparihā\-raḥ | bhā\-vā\-nubhā\-vī\-ti vasturū\-pam | etena pradhvaṃsā\-diparihā\-raḥ | yad yadā\-khyā\-tasā\-dhanasambandhī\-ti vyā\-ptivacanaṃ yaddharmirū\-pam kathitasā\-dhanayogī\-ty arthaḥ |
	\pend
      

	  \pstart \label{sarit__ratnakīrtinibandhāvali__104795}\persName{trilocanas} tu vyatirekiṇam imaṃ prayogam ā\-ha —
	\pend
      

	  \pstart sarvaṃ kā\-ryaṃ prabodhavaddhetukam |
	\pend
      

	  \pstart utpattidharmakatvā\-t |
	\pend
      

	  \pstart yan nityaṃ dṛṣṭam abodhavaddhetukaṃ tasyā\-kā\-śā\-des tathotpattir nā\-stī\-ti dṛṣṭam |
	\pend
      

	  \pstart utpattidharmakaṃ ca pakṣī\-kṛtam asmadā\-divinirmitetarat |
	\pend
      

	  \pstart tasmā\-d bodhavaddhetukam iti |\edlabel{sarit__ratnakīrtinibandhāvali__105213}\label{sarit__ratnakīrtinibandhāvali__105213}
	\pend
      

	  \pstart punar dvyaṇukeśvarasiddhau \persName{trilocana} eva prā\-ha—
	\pend
      

	  \pstart vivā\-dā\-spadī\-bhū\-taṃ dvitvam ā\-tmotpattau kasyacid ekaikaviṣayā\-ṃ buddhim apekṣate |
	\pend
      

	  \pstart dvitvasaṃkhyā\-tvā\-t |
	\pend
      

	  \pstart yad yad dvitvaṃ tat tathā\- | yathā\- dve dravye |
	\pend
      

	  \pstart tathā\- cedaṃ dvyaṇukagataṃ dvitvam |
	\pend
      

	  \pstart tasmā\-t tatheti |
	\pend
      

	  \pstart yasya cā\-tra buddhir apekṣyate sa bhagavā\-n ī\-śvaraḥ ||
	\pend
      

	  \pstart tathā\- ca Vā\-caspatiḥ pramā\-ṇayati—
	\pend
      

	  \pstart vivā\-dā\-dhyā\-sitatanutarugirisā\-garā\-dayaḥ upā\-dā\-nā\-dyabhijñakartṛkā\-ḥ |
	\pend
      

	  \pstart kā\-ryatvā\-t |
	\pend
      

	  \pstart yad yat kā\-ryaṃ tat tad upā\-dā\-nā\-dyabhijñakartṛkam | yathā\- prā\-sā\-dā\-di |
	\pend
      

	  \pstart tathā\- ca vivā\-dā\-dhyā\-sitā\-s tanvā\-dayaḥ |
	\pend
      

	  \pstart tasmā\-t tatheti |
	\pend
      

	  \pstart evaṃ sthitvā\- sthitvā\- pravṛttidharmakatvā\-t, sanniveśavattvā\-t, arthakriyā\-kā\-ritvā\-d ityā\-dayo hetavaḥ kathitapañcā\-vayavakrameṇa boddhavyā\- iti |
	\pend
      \label{īsd-uttarapakṣa}\edlabel{īsd-uttarapakṣa}
	  
	% new div opening: depth here is 2
	\label{īsd-vyāptigrahaṇa}\edlabel{īsd-vyāptigrahaṇa}
	  
	% new div opening: depth here is 3
	

	  \pstart tad etad durmativispanditaṃ jagadandhī\-karaṇaṃ na satā\-m upekṣitum ucitam iti kiñcid ucyate | iha khalu buddhimatkā\-ryamā\-trayoḥ sā\-dhyasā\-dhanayoḥ sarvopasaṃhā\-ravatī\- vyā\-ptis tā\-vad avaśyaṃ grahī\-tavyā\- | anyathā\- gamyagamakabhā\-vā\-yogā\-t | sā\- ca gṛhyamā\-ṇā\- kiṃ kā\-raṇakā\-ryamā\-trayor iva viparyayabā\-dhakapramā\-ṇabalā\-t grā\-hyā\- | yad vā\- 'gnidhū\-mayor iva viśiṣṭā\-nvayavyatirekagrahaṇapravaṇaviśiṣṭapratyakṣā\-nupalambhā\-bhyā\-ṃ boddhavyā\- | uta svavyavasthayā\- sapakṣā\-sapakṣayor bhū\-yor darśanā\-darśanā\-bhyā\-ṃ pratyetavyā\- | ā\-hosvit sapakṣā\-sapakṣayoḥ sakṛddarśanā\-bhyā\-ṃ jñā\-tavyeti catvā\-ro vikalpā\-ḥ |
	\pend
      
	  
	% new div opening: depth here is 4
	

	  \pstart na tā\-vad ā\-dyaḥ pakṣaḥ, sā\-dhyaviparyaye buddhimadabhā\-ve kā\-ryatvasā\-mā\-nyasya sā\-dhanasya bā\-dhakapramā\-ṇā\-bhā\-vā\-t. nanu bā\-dhakapramā\-ṇā\-\edtext{bhāvo 'siddhaḥ}{\Afootnote{ \cite{}bhā\-vosiddhaḥ \cite{}}}. tathā\- hī\-daṃ kā\-ryatvaṃ yathā\- buddhimatā\- vyā\-ptam iṣyate tathā\- deśakā\-lasvabhā\-vaniyatatvenā\-pi, kadā\-cikakā\-raṇasannidhimattayā\-pi, sā\-magrī\-kā\-ryatvenā\-pi vyā\-ptam upalabdham | sa ca deśā\-diniyamaḥ kā\-dā\-citkakā\-raṇasanndhiḥ sā\-magrī\- vā\- buddhimatpū\-rvikā\- siddhā\- | yadi punar acetanā\-ni cetanā\-nadhiṣṭhatā\-ni kā\-ryaṃ kuryuḥ tato yatra kvacanā\-vasthitā\-ni janayeyur iti na deśakā\-lasvabhā\-vaniyataprasavaṃ kā\-ryam upalabhyeta |
	\pend
      

	  \pstart hetusamavadhā\-najanmatayā\- na kā\-ryaṃ pratyekaṃ kā\-raṇair janyata iti cet | samavadhā\-nam eva tu kā\-raṇā\-nā\-ṃ kutaḥ | kā\-dā\-citkaparipā\-kā\-dadṛṣṭaviśeṣā\-d iti cet | nanv ayam acetanaḥ kathaṃ yathā\-vat kā\-raṇā\-ni sannidhā\-payet | no khalu kvacid avasthitā\-ni daṇḍā\-dī\-ni vinā\- kumbhakā\-raprayatnam adṛṣṭaviśeṣavaśā\-d eva parasparaṃ sannidhī\-yante | sannihitā\-ni vā\- kā\-ryā\-ya prabhavantī\-ti buddhimatā\- deśakā\-lasvabhā\-vaniyamasya kā\-dā\-citkakā\-raṇasannidheḥ sā\-magryā\-ś ca vyā\-ptisiddhiḥ | buddhimadabhā\-ve caiṣā\-ṃ vyā\-pakā\-nā\-ṃ nivṛttau nivartamā\-naṃ kā\-ryatvaṃ buddhimatpū\-rvakatvena vyā\-pyata iti pratibandhasiddhaye vyā\-pakā\-nupalambhatrayam upanyastam | \edtext{tathā na}{\Afootnote{ \cite{}tathā\- ca na \cite{}}} kā\-ryaṃ buddhimatparityā\-gā\-d ahetukam eva bhavatī\-ti sambhā\-vyam, deśakā\-lasvabhavaniyamā\-bhā\-vaprasaṅgā\-t | nā\-pi buddhimato 'nyasmā\-d eva bhavatī\-ti śaṅkanī\-yam, sakṛd apy utpā\-dā\-bhā\-vaprasaṅgā\-t | na cā\-nyasmā\-d asmā\-d api bhavatī\-ti sambhā\-vyam, aniyatahetutve 'hetutvaprasaṅgā\-t | tathā\- buddhimantam antareṇā\-ce tanena karaṇe sarvadā\- kriyā\-yā\- avirā\-maprasaṅgaś cety api viparyayabā\-dhakam atiprasaṅgacatuṣṭayaṃ vyā\-ptiprasā\-dhakam iti | kā\-ryatvasya hetupū\-rvakatvam iva buddhimatpū\-rvakatvam apy avā\-ryam iti cet |
	\pend
      

	  \pstart atrocyate | sidhyaty evedaṃ manorā\-jyaṃ yadi deśakā\-lasvabhā\-vaniyamasya kā\-dā\-citkakā\-raṇasannidheḥ samagryā\-ś ca buddhimatpū\-rvakatvena vyā\-ptiḥ sidhyati | kevalam etad eva durā\-pam | buddhimadabhā\-ve 'pi hi svahetubalasamutpannasannidheḥ pratiniyatadeśakā\-laśaktinā\-cetanenā\-pi sā\-magrī\-lakṣaṇakā\-raṇaviśeṣeṇa kriyamā\-ṇā\-ni deśakā\-lasvabhā\-vaniyamakā\-dcitkakā\-raṇasannidhisā\-magrī\-kā\-ryatvā\-ni yujyanta iti sandigdhā\-siddhā\- vyā\-pakā\-nupalabdhayaḥ ||
	\pend
      

	  \pstart buddhimadabhā\-ve samavadhā\-nam eva kuta iti cet | tad api cetanā\-nadhiṣṭhitayathoktā\-cetanasā\-magrī\-viśeṣā\-d eva | so 'pi tā\-dṛśā\-d ity anā\-dyacetanasā\-magrī\-paramparā\-to 'pi deśā\-diniyamasambhā\-vanā\-yā\-ṃ nā\-vaśyaṃ buddhimadapekṣā\- | ghaṭā\-der deśakā\-lasvabhā\-vaniyamaḥ kā\-dā\-citkakā\-raṇasannidhiś ca, sā\-magrī\- ca buddhimatpū\-rvikā\- dṛṣṭā\- ity aparopi deśakā\-lasvabhā\-vaniyamā\-dis tathaiveti cet | yady evaṃ ghaṭā\-dikam api kā\-ryaṃ bahuśo buddhimatpū\-rvakam upalabdham iti sarvam eva kā\-ryaṃ tathā\-stu, kim anena vyā\-pakā\-nupalambhopanyā\-sadurvyasanena | ghaṭā\-der bahuśo buddhimatpū\-rvakatvadarśane 'pi na sarvatra kā\-ryamā\-trasya tathā\-bhā\-vaniścayaś cet | deśā\-diniyamā\-dī\-nā\-m apī\-daṃ samā\-nam iti katham atrā\-pi śaṅkā\-vyudā\-saḥ ||
	\pend
      

	  \pstart astu tadā\- pratyakṣam eva sarvatra vyā\-ptigrā\-hakam iti cet | na tarhi viparyayabā\-dhakapramā\-ṇabalā\-d vyā\-ptigrahanirvā\-haḥ | pratyakṣaṃ ca tatrā\-śaktam iti dvitī\-yavikalpā\-vasare nivedayiṣyate | \edtext{tathāsiddhe}{\Afootnote{ \cite{}tathā\- siddhe \cite{}}} kā\-ryakā\-raṇabhā\-ve dhū\-masyā\-hetukotpattā\-v anyasmā\-d evotpattā\-v anyasmā\-d apy utpattau sambhā\-vyamā\-nā\-yā\-ṃ \edtext{deśādiniyamābhāvasakṛdutpādābhāvāhetutvaprasaṅgāḥ}{\Afootnote{deśā\-diniyamā\-bhā\-va\add{\unclear{sā\-kṛ}dutpā\-dā\-hetutva}prasaṅgā\-ḥ \cite{}; deśā\-diniyamā\-bhā\-vakḷptahetutyā\-gā\-nyahetutvaprasaṅgā\-ḥ \cite{}---\textsc{Note} Emended according to the options given in \cref{RNA-ms-23a-3} to \cref{ĪSD-41-9}.}} saṅgacchante | prastute tu buddhimatkā\-ryamā\-trayoḥ kā\-ryakā\-raṇabhā\-vo nā\-dyā\-pi siddhaḥ | sā\-dhayituṃ vā\- śakyaḥ | na cā\-cetanasya kartṛtve kriyā\-yā\- \leavevmode\textsuperscript{\rmlatinfont\tiny [42/thakur75]} avirā\-maprasaṅgaḥ saṅgataḥ | na hy acetanam ity eva sarvadā\- sā\-marthyayogi, tasyā\-pi svahetuparamparā\-pratibaddhasā\-marthyatvā\-d ity acetanakā\-raṇaviśeṣaparamparā\-sambhā\-vanā\-yā\-ṃ nā\-vaśyaṃ buddhimadā\-kṣepa iti svamatavyā\-lopaviklavavikrośitamā\-tram evedaṃ na punar atra nyā\-yagandho 'pi |
	\pend
      

	  \pstart tad evaṃ vyā\-ptisā\-dhanā\-rtham upanyastaṃ vyā\-pakā\-nupalambhatrayaṃ sandigdhā\-siddham atiprasaṅgatucaṣṭayaṃ ca buddhimatkā\-ryamā\-trayor vyā\-ptyasiddhā\-v asaṅgatam | ataḥ kā\-ryatvaṃ sā\-dhanaṃ sandigdhavipakṣavyā\-vṛttikatvā\-d anaikā\-ntikam || 
	\pend
      

	  \pstart atra Vā\-caspatiḥ prā\-ha: sandigdhavipakṣavyā\-vṛttikatvaṃ nā\-ma hetudoṣa eva na bhavati | tat kathaṃ nirasyate | tathā\- hi ya eva vipakṣe dṛṣṭo hetuḥ sa eva prameyatvā\-divad abhimataṃ na sā\-dhyet | yas tu mahatā\-pi prayatnena mṛgyamā\-ṇo 'sapakṣe nopalakṣitaḥ sa kathaṃ sā\-dhyaṃ na sā\-dhayet | 
	    \pend
	  
	    
	    \stanza[\smallbreak]
\edlabel{sarit__ratnakīrtinibandhāvali__111722}\label{sarit__ratnakīrtinibandhāvali__111722}\edtext{}{\lemma{avaśyaṃ ... apaśyatām |}\xxref{sarit__ratnakīrtinibandhāvali__111722}{sarit__ratnakīrtinibandhāvali__111823}\Afootnote{\label{sarit__ratnakīrtinibandhāvali__402429}\textenglish{---\textsc{Note} Quoted from .}  {\rmlatinfont [App type: parallel]}}}avaśyaṃ śaṅkayā\- bhā\-vyaṃ niyā\-makam apaśyatā\-m |\edlabel{sarit__ratnakīrtinibandhāvali__111823}\label{sarit__ratnakīrtinibandhāvali__111823}\&[\smallbreak]


	
	    \pstart
	   iti tu dattā\-vakā\-śā\- laukikam aryā\-dā\-tikrameṇa saṃśayapiśā\-cī\- labdhaprasarā\- na kvacin nā\-stī\-ti nā\-yaṃ kvacit pravarteta | sarvasyaivā\-rthasya kathañcic chaṅkā\-spada\edtext{tvadarśanāt}{\Afootnote{ \cite{}tvā\-darśanā\-t \cite{}}} | anarthaśaṅkā\-yā\-ś ca prekṣā\-vatā\-ṃ nivṛttyaṅgatvā\-t | antataḥ snigdhā\-nnapā\-nopayoge 'pi maraṇadarśanā\-t | tasmā\-t prā\-mā\-ṇikalokayā\-trā\-m anupā\-layatā\- yathā\-darśanaṃ śaṅkanī\-yam, na tv adṛṣṭam api | viśeṣasmṛtyapekṣo hi saṃśayo nā\-smṛter bhavati | na ca smṛtir ananubhū\-tacare bhavati |
	\pend
      

	  \pstart tad uktaṃ mī\-mā\-ṃsā\-vā\-rttikakṛ\leavevmode\textsuperscript{\rmlatinfont\tiny [24a/RNAms]}tā\- adhyuṣṭasahasrikā\-yā\-m: 
	    \pend
	  
	    
	    \stanza[\smallbreak]
nā\-śaṅkā\- niḥpramā\-ṇiketi |\&[\smallbreak]


	
	    \pstart
	  \footnote{\begin{english}(ŚV II 60cd)\end{english}}
	\pend
      

	  \pstart tathā\- tenaiva Bṛhaṭṭī\-kā\-yā\-m:
	\pend
      
	    
	    \stanza[\smallbreak]
utprekṣeta hi yo mohā\-d ajā\-tam api bā\-dhakam |&sa sarvavyavahā\-reṣu saṃśayā\-tmā\- kṣayaṃ vrajet || iti | \footnote{\begin{english}(=TS 2871) \href{../../kamalaśīla/texts/tsp_ges.xml\#ts-2872}{../../kamalaśī\-la/texts/tsp\textunderscore ges.xml\#ts-2872}\end{english}}\&[\smallbreak]


	

	  \pstart tad etat pralā\-pamā\-tram | na hi mahatā\-pi prayatnena vipakṣe mṛgyamā\-ṇasya hetor adarśanamā\-treṇa vyatirekaḥ sidhyati | tathā\- hi vipakṣe hetur nopalabhyata ity anena tadupalambhakapramā\-ṇanivṛttir ucyate |  pramā\-ṇaṃ ca prameyasya kā\-ryam, nā\-kā\-raṇaṃ viṣaya iti nyā\-yā\-t | na ca kā\-ryanivṛttau kā\-raṇanivṛttir upalabdhā\-, nirdhū\-masyā\-pi vahner upalambhā\-t | yadi punaḥ pramā\-ṇasattayā\- prameyasattā\- vyā\-ptā\- syā\-t, tadā\- yuktam etat | kevalam iyam eva vyā\-ptir asambhavinī\-, sarvasya sarvadarśitvaprasaṅgā\-t | tan nā\-darśanamā\-treṇa vyatirekasiddhiḥ | yathoktam: 
	    \pend
	  
	    
	    \stanza[\smallbreak]
sarvā\-dṛṣṭiś ca sandigdhā\- svā\-dṛṣṭir vyabhicā\-riṇī\- |&vindhyā\-drirandhradū\-rvā\-der adṛṣṭā\-v api sattvataḥ || iti \footnote{\begin{english}(=TS 122)\end{english}}\&[\smallbreak]


	
	    \pstart
	   sakalavipakṣasyā\-rvā\-cī\-naṃ praty adṛśyatvā\-t ||
	\pend
      

	  \pstart yac coktam: saṃśayapiśā\-cī\- labdhaprasarā\- na kvacin nā\-stī\-ti na kvacit pravarteteti | tad asaṅgatam | arthasaṃśayasyā\-pi prekṣā\-vatā\-ṃ pravṛttyaṅgatvā\-t pravṛttir avirodhiny eva | anarthasandehaḥ sarvatra kartuṃ śakyate | antataḥ snigdhā\-nnapā\-nopayoge 'pi maraṇadarśanā\-d apravṛttir iti cet | durjñā\-nam etat | tathā\- hy arthasandeho 'narthasandeho veti nā\-yaṃ ṣaṣṭhī\-samā\-saḥ | kin tv arthonmukhaḥ sandeho 'rthasandehaḥ, anarthonmukhaḥ sandeho 'narthasandeha iti śā\-kapā\-rthivā\-divanmadhyapadalopī\- samā\-saḥ | evaṃ sati snigdhā\-nnapā\-nā\-dā\-v arthasandeha eva, tajjā\-tī\-yasya svaparasantā\-ne dṛṣṭipuṣṭyā\-dyarthasya koṭiśaḥ karaṇadarśanā\-t, maraṇā\-der anarthsya kvacit kadā\-cid darśanā\-t | \edtext{etadviparīto}{\Afootnote{ \cite{}etadviviparī\-to \cite{}}} 'narthasandeho draṣṭavyaḥ | tasmā\-t pramā\-ṇā\-divā\-rthasaṃśayā\-d api prekṣā\-vatā\-ṃ tatra tatra pravṛttir durvā\-raiva ||
	\pend
      

	  \pstart yad apī\-daṃ lapitaṃ yathā\-darśanaṃ śaṅkanī\-yaṃ nā\-dṛṣṭapū\-rvam api viśeṣasmṛtyapekṣo hi saṃśaya ityā\-di | tad asambaddhaṃ | sā\-dhakabā\-dhakapramā\-ṇā\-bhā\-vā\-d eva paryudā\-savṛttyā\- \edtext{vastvantara}{\Afootnote{ \cite{}vasvantara \cite{}}}rū\-pā\-t sarvatra saṃśayotpatteḥ | kiṃ ca viśeṣasmṛtyapekṣa evā\-yaṃ saṃśayaḥ | tathā\- hi lakṣaṇayogitvā\-yogitvā\-bhyā\-m eva tajjā\-tī\-yā\-tajjā\-tī\-ye vaktavye | anyathā\- lakṣaṇapraṇayanam anarthakaṃ syā\-t | evaṃ ca sati tā\-dā\-tmyatadutpattilakṣaṇapratibandhaviyogitvena sā\-dhā\-raṇena dharmeṇa prameyatvadhū\-matvakā\-rya\leavevmode\textsuperscript{\rmlatinfont\tiny [24b/RNAms]}tvā\-dī\-nā\-ṃ tvanmatena sajā\-tī\-yatvā\-t prameyatvavyabhicā\-radarśanam eva śaṅkā\-m upasthā\-payatī\-ti yathā\-darśanam evedam ā\-śaṅkitam |
	\pend
      

	  \pstart yaś ca Kumā\-rilasya sā\-kṣitvenopanyā\-saḥ sa khalu 
	    \pend
	  
	    
	    \stanza[\smallbreak]
dadhibhā\-ṇḍe viḍā\-laḥ sā\-kṣī\-ti\&[\smallbreak]


	
	    \pstart
	   pravā\-daṃ nā\-tipatatī\-ti kim atra vaktavyam | tad evaṃ vipakṣe 'darśanamā\-treṇa hetor vyatirekā\-siddheḥ sandigdhavipakṣavyā\-vṛttikatvaṃ nā\-ma hetudū\-ṣaṇaṃ durvā\-ram eva | ata evā\-syopanyā\-so 'doṣodbhā\-vanaṃ nā\-ma nigrahasthā\-nam iti yad anenā\-veditaṃ tad api sā\-vadyam | pratyutā\-smin heto\gap{}ḥ saddū\-ṣaṇe parihartavye nā\-yaṃ hetudoṣo 'to na parihartavyo 'sya copanyā\-so 'doṣodbhā\-vanaṃ nā\-ma nigrahasthā\-nam iti bruvann ayam eva tapasvī\- svamatena niranuyojyā\-nuyogalakṣaṇena nigrahasthā\-nena nigṛhyata iti kṛpā\-m arhati | tad evaṃ viparyayabā\-dhakapramā\-ṇā\-bhā\-vā\-d avyā\-pter asiddheḥ sandigdhavipakṣavyā\-vṛttikatvā\-d anaikā\-ntikaḥ kā\-ryatvalakṣaṇo hetuḥ ||
	\pend
      
	  
	% new div opening: depth here is 4
	

	  \pstart athā\-gnidhū\-mayor iva viśiṣṭā\-nvayavyatirekagrahaṇapravaṇaviśiṣṭapratyakṣā\-nupalambhā\-bhyā\-ṃ vyaptir niścī\-yata iti dvitī\-yaḥ pakṣaḥ | atrocyate | kiṃ dṛśyaśarī\-ropā\-dhinā\- buddhimanmā\-treṇa vyā\-ptigṛhyate, ā\-hosvit dṛśyaśarī\-ropā\-dhividhureṇa dṛśyā\-dṛśyasā\-dhā\-raṇeneti vikalpau | yady ā\-dyaḥ pakṣah, tadā\- tathā\-bhū\-tasā\-dhyam antareṇā\-py utpadyamā\-ne viṭapā\-dau kā\-ryatvadarśanā\-t prameyatvā\-divat sā\-dhā\-raṇā\-naikā\-ntiko hetuḥ |
	\pend
      

	  \pstart \edlabel{thakur75-44.2}\label{thakur75-44.2} nanu vṛkṣā\-dayaḥ pakṣī\-kṛtā\-ḥ | kathaṃ tair vyabhicā\-raḥ | trividho hi bhā\-varā\-śiḥ | sandigdhakartṛko yathā\- vṛkṣā\-diḥ | prasiddhakartṛko yathā\- ghaṭā\-diḥ | akartṛko yathā\- ā\-kā\-śā\-diḥ | tatra prasiddhakartṛke ghaṭā\-dau pratyakṣā\-nupalambhā\-bhyā\-ṃ vyā\-ptim ā\-dā\-ya sandehapade kṣmā\-ruhā\-dau kā\-ryatvam upasaṃhṛtya buddhimā\-n anumī\-yate | na punar asu vyabhicā\-raviṣayo bhavitum arhati | \edlabel{thakur75-44.8}\label{thakur75-44.8} yad ā\-ha: na sā\-dhyenaiva vyabhicā\-ra iti | ayuktam etat | na hi vyabhicā\-raviṣaya eva pakṣe bhavitum arhati:
	\pend
      
	    
	    \stanza[\smallbreak]
sandigdhe hetuvacanā\-d vyasto hetor anā\-śrayaḥ\footnote{\begin{english}(PV IV 91)\end{english}}\&[\smallbreak]


	

	  \pstart iti nyā\-yā\-t | vyabhicā\-raviṣayatā\- ca dṛśyaśarī\-ropā\-dher buddhimanmā\-trasya tṛṇā\-dyutpattau dṛśyā\-nupalambhena pratikṣiptatvā\-t | tataś ca kṣmā\-dharā\-dir eva sandigdhakartṛkaḥ pakṣī\-kartum ucitaḥ kṣmā\-ruhā\-dis tv acetanakartṛka iti caturtho bhavarā\-śir neṣṭavyaḥ | atha vyabhicā\-racamat\edtext{kārāttri}{\Afootnote{ \cite{}kā\-rā\-stri \cite{}}}vidhabhā\-varā\-śivyavasthā\-panā\-rthaṃ ca viṭapā\-dau pratyakṣā\-pratikṣiptena dṛśyā\-dṛśyasā\-dhā\-raṇena buddhimanmā\-treṇa vyā\-ptir avagamyata iti dvitī\-yaḥ saṅkalpaḥ | tadā\- viṭapā\-dau buddhimanmā\-trasya sambhā\-vyamā\-natvā\-d na sā\-dhā\-raṇā\-naikā\-ntikatā\-ṃ brū\-maḥ | kiṃ tarhi vyā\-ptigrahaṇakā\-le dṛśyā\-dṛśyasā\-dhā\-raṇasya buddhimanmā\-trasya sā\-dhyasyā\-dṛśyatayā\- dṛśyā\-nupalambhena vyatirekā\-siddher vyā\-pter abhā\-vat sandigdhavyā\-vṛttikatvam ā\-cakṣmahe | tathā\- hi | yadā\- kumbhakā\-ravyā\-pā\-rā\-t pū\-rvaṃ kumbhasya vyatirekaḥ pratyetavyas tadā\- na sā\-dhyā\-bhā\-vakṛto ghaṭavyatirekaḥ pratyetuṃ śakyaḥ | yathā\- hi viṭapā\-dijanmasamaye buddhimanmā\-trā\-syā\-dṛśyatvena niṣeddhum aśakyatvā\-t sattā\-sambhā\-vanā\- tathā\- ghaṭā\-dā\-v api vyatirekaniścayakā\-le buddhimanmā\-trasyā\-dṛśyatvā\-t sattvasambhā\-vanā\-yā\-ṃ sā\-dhyā\-bhā\-vaprayuktasya sā\-dhanā\-bhā\-vasyā\-siddhatvena vyā\-pter abhā\-vā\-t kathaṃ na sandigdhavyatireko hetuḥ |
	\pend
      

	  \pstart \edtext{yaccoktannacaṃ}{\Afootnote{yathoktam—na ca \cite{}}} yathā\- kā\-ryaṃ ca syā\-n nirupā\-dā\-nā\-ṃ ceti nā\-śaṅkanī\-yam, tathā\- kā\-ryaṃ ca bhaved akartṛkaṃ ceti nā\-śaṅkanī\-yam iti, tatrā\-pi kā\-ryaṃ ca syā\-n nirupā\-dā\-naṃ ca bhaved iti na vaktavyam iti kenaivaṃ pratā\-rito 'si | yadi hy atra pratyakṣā\-nupalambhā\-bhyā\-ṃ vyā\-ptir gṛhyate tadā\- katham upā\-dā\-napū\-rvakaṃ kā\-ryamā\-traṃ sidhyati | vyā\-ptigrahaṇaprakā\-rā\-ntaraṃ ca tvayā\-pi nopanyastam | dṛśyā\-dṛśyasā\-dhā\-raṇayor upā\-dā\-nakā\-ryamā\-trayor dṛśyaviṣayā\-bhyā\-ṃ pratyakṣā\-nupalambhā\-bhyā\-ṃ vyā\-pter \edtext{abhyūhitum}{\Afootnote{ \cite{}\unclear{angra}hī\-tum \cite{}}} aśakyatvā\-t | svamatavyā\-lopaprasaṅgas tu pramā\-ṇacintā\-vasare 'prā\-ptā\-vakā\-śaḥ | viparyayabā\-dhakapramā\-ṇabalā\-d vā\-tra vyā\-ptisiddhiḥ | tathā\- hi yathā\-ṅkurā\-dikaṃ kā\-ryaṃ niyatadeśakā\-lasvabhā\-vatvena vyā\-ptaṃ tathā\- śā\-litvā\-dinā\-pi jā\-tibhedena vyā\-ptam upalabdham | tataś cā\-nupā\-dā\-napū\-rvakatvā\-d vipakṣā\-tmanaḥ śā\-litvā\-dijā\-tibhedasya vyā\-pakasya nivṛttau nivartamā\-naṃ kā\-ryatvam upā\-dā\-napū\-rvakatve viśrā\-myat tena vyā\-ptaṃ sidhyati | na cā\-nupā\-dā\-nenā\-pi kriyamā\-ṇaḥ śā\-litvā\-dijā\-tibhedo yujyate, upā\-dā\-naṃ vinā\- \edtext{kṛtād anu}{\Afootnote{ \cite{}kṛtā\-nu \cite{}}}pā\-dā\-nā\-d eva kevalā\-d ekajā\-tī\-yakā\-raṇā\-t tadatajjā\-tī\-yakā\-ryotpattau kā\-ryabhedasyā\-hetukatvaprasaṅgā\-t | tad uktam: 
	    \pend
	  
	    
	    \stanza[\smallbreak]
tadatadrū\-piṇo bhā\-vā\-s tadatadrū\-pahetujā\-ḥ ||\&[\smallbreak]


	
	    \pstart
	   iti | \footnote{\begin{english}(PV III 251ab)\end{english}}
	\pend
      

	  \pstart anyathā\-nupā\-dā\-nā\-d eva kṣityā\-der aṅkurā\-dikam utpadyetety aṅkurā\-rthino bī\-jaṃ nā\-nusareyuḥ | tasmā\-d viparyayabā\-dhakapramā\-ṇabalā\-d eva kā\-ryatvasya hetumā\-trapū\-rvakatvenevopā\-dā\-napū\-rvakatvenā\-pi vyā\-ptisiddhir iti nyā\-yaḥ | na caivaṃ kā\-ryamā\-trakartṛtvamā\-trayor api vyā\-ptiprasā\-dhakaṃ viparyaye bā\-dhakaṃ pramā\-ṇam asti, pū\-rvoktasya vyā\-pakā\-nupalambhatrayasyā\-tiprasaṅgacatuṣṭayasya ca prā\-g eva pratyā\-khyā\-tatvā\-t | tasmā\-t kā\-ryaṃ ca syā\-t na ca dhī\-matkartṛpū\-rvakam iti śaṅkā\-ṃ kurvā\-ṇaḥ prativā\-dī\- vinā\- caraṇamardanā\-dinā\- niṣeddhum aśakyaḥ ||
	\pend
      

	  \pstart nanu yadi dṛśyā\-gnidhū\-masā\-mā\-nyayor iva \edtext{dṛśyātmanor}{\Afootnote{}} eva kā\-ryakā\-raṇasā\-mā\-nyayoḥ pratyakṣā\-nupalambhato vyā\-ptis tadā\- paracittā\-numā\-nakṣatiḥ | svaparasantā\-nasā\-dhā\-ra\edtext{ṇenādṛśyena}{\Afootnote{ \cite{}ṇena dṛśyā\-dṛśyena \cite{}\textenglish{---\textsc{Note} Thakur says ```dṛśyā\-' later addition." No trace of it in \cref{RNAms}, so probably his own emendation.}}} cinmā\-treṇa pratyakṣato dṛśyaviṣayā\-d \edtext{kampasya}{\Afootnote{ \cite{} \cite{}}}vyā\-ptigrahaṇā\-yogā\-d ity api na vā\-cyam | bā\-hyā\-rthasthitau hi svaparasantā\-nasā\-dhā\-raṇasya cinmā\-trasya svarū\-peṇā\-dṛśyatve 'pi dṛśyaśarī\-reṇa sahaikasā\-magrī\-pratibandhā\-d avinirbhā\-gavartitvam asty eva | tato yathā\- ghaṭaviṣayaṃ pratyakṣaṃ rū\-paikadeśapravṛttam apy avyabhicā\-rā\-t samudā\-yopasthā\-pakam tathā\- dehagrā\-hakam eva pratyakṣaṃ dehā\-vinirbhā\-gavarti svaparasantā\-nasā\-dhā\-raṇaṃ cinmā\-traṃ kampā\-der vyā\-pakam adhigacchanti | tad evaṃ dṛśyā\-tmano dṛśyā\-vinirbhā\-gavartino vā\- padā\-rthasya vyā\-vahā\-rikapaṭupratyakṣataḥ siddhir vyā\-ptigrahaś ca, na tu tathā\-tvavinā\-kṛtā\-dṛśyasā\-dhā\-raṇacinmā\-trasyeti santā\-nā\-ntarā\-numā\-nam ucitam | tasmā\-d yadi pratyakṣā\-nupalabhā\-bhyā\-ṃ vyā\-ptigrahas tadā\- dṛśyenaiva dṛśyasyeti nyā\-yaḥ | \edlabel{thakur75-45.24}\label{thakur75-45.24} tad ayaṃ saṃkṣepā\-rthaḥ:
	\pend
      
	    
	    \stanza[\smallbreak]
kā\-ryatvasya vipakṣavṛttihataye sambhā\-vyate 'tī\-ndriyaḥ kartā\- ced vyatirekasiddhividhurā\- vyā\-ptiḥ kathaṃ sidhyati |&dṛśyo 'tha vyatirekasiddhimanasā\- kartā\- samā\-śrī\-yate tattyā\-ge 'pi tadā\- tṛṇā\-dikam iti vyaktaṃ vipakṣe kṣaṇam ||\&[\smallbreak]


	[[(JNA 285,7-10)]]

	  \pstart ato na pratyakṣā\-nupalmbhā\-bhyā\-m api vyā\-ptisiddhiḥ ||
	\pend
      
	  
	% new div opening: depth here is 4
	

	  \pstart nanu bhū\-yodarśanā\-darśanā\-bhyā\-ṃ pratibandhaḥ pratī\-yata iti tṛtī\-ya evā\-samā\-kaṃ pakṣaḥ | kevalaṃ sa pratibandho na tadutpattilakṣaṇo grahī\-tavyaḥ | kin tu svā\-bhā\-vikaḥ | sa eva darśanā\-darśanā\-bhyā\-ṃ pratī\-yate | tathā\- caitam evā\-rthaṃ Vā\-caspatiḥ prā\-ha:\footnote{The following collects material from \href{NVTṬ\#nvtṭ-nsū_1-1.5}{NVTṬ\#nvtṭ-nsū\-\textunderscore 1-1.5}, pp. 135--136.} na sapakṣā\-sapakṣayor \edtext{darśanādarśanābhyāṃ}{\Afootnote{ \cite{}darśanā\-bhyā\-ṃ \cite{}}} kā\-ryatvasya gamakatvam api tu svā\-bhā\-vikapratibandhabalā\-d iti brū\-maḥ | sa eva tu sapakṣā\-sapakṣayor darśanā\-darśanā\-bhyā\-ṃ vakṣyamā\-ṇena krameṇa pratī\-yata iti tadupakṣepo 'pi yuktaḥ | \edlabel{sarit__ratnakīrtinibandhāvali__122689}\label{sarit__ratnakīrtinibandhāvali__122689}\edtext{}{\lemma{tena … api ||}\xxref{sarit__ratnakīrtinibandhāvali__122689}{sarit__ratnakīrtinibandhāvali__124304}\Afootnote{\label{sarit__ratnakīrtinibandhāvali__399142}---\textsc{Note} Mentioned in the context of NVTṬ and VyN in \cref{krasser02_zaGkar_Izvar_studie}.  {\rmlatinfont [App type: parallel]}}}tena yasyā\-sau \edtext{svābhāvikaḥ}{\Afootnote{ \cite{}svā\-bhā\-vika \cite{}}}pratibandho niyataḥ siddhaḥ sa eva gamako gamyaś cetaraḥ sambandhī\-ti yujyate | tathā\- hi dhū\-mā\-dī\-nā\-ṃ vahnyā\-dibhiḥ saha sambandhaḥ svā\-bhā\-viko na tu vahnyā\-dī\-nā\-ṃ dhū\-mā\-dibhiḥ | te hi vinā\- dhū\-mā\-dibhir upalambhyante | yadā\- tv \edtext{ārdrendhanādisambandham}{\Afootnote{ā\-rdrendhanā\-disambandham \cite{}; ā\-rdrendhanasambandham \cite{} \cite{}}} anubhavanti tadā\- dhū\-mā\-dibhiḥ sambadhyante | tasmā\-d vahnyā\-dī\-nā\-m ā\-rdredhanā\-dyupā\-dhikṛtaḥ sambandho na tu svā\-bhā\-vikas tato na niyataḥ | svā\-bhā\-vikas tu dhū\-mā\-dī\-nā\-ṃ vahnyā\-dibhiḥ sambandhaḥ, tadupā\-dher anupalabhyamā\-natvā\-t | kvacid vyabhicā\-rasyā\-darśanā\-t | anupalabhyamā\-nasyā\-pi kalpanā\-nupapatteḥ | na cā\-nupalabhyamā\-no darśanā\-narhatayā\- sā\-dhakabā\-dhakapramā\-ṇā\-bhā\-vena sandihyamā\-na upā\-dhiḥ sambandhasya svā\-bhā\-vikatvaṃ pratibadhnā\-tī\-ti yuktam | yathoktaṃ prā\-k seyaṃ saṃśayapiśā\-cī\-tyā\-di | tasmā\-d upā\-dhiṃ prayatnenā\-nviṣyanto 'nupalabhyamā\-nā\- nā\-stī\-ty avagamya svā\-bhā\-vikatvaṃ niścinumaḥ ||
	\pend
      

	  \pstart syā\-d etat | anyasyā\-nyena sahakā\-raṇena cet svā\-bhā\-vikaḥ sambandho bhavet, sarvaṃ sarveṇa sambadhyeta | tathā\- ca sarvaṃ sarvasmā\-d gamyeta | athā\-nyac ced anyasya kā\-ryaṃ kasmā\-t sarvaṃ sarvasmā\-n na bhavati, anyatvā\-viśeṣā\-t | tataś ca sa evā\-tiprasaṅgaḥ | yady ucyeta svabhā\-vā\- na paryanuyojyā\-ḥ | tasmā\-d anyatvā\-viśeṣe 'pi kiñcid eva kā\-raṇaṃ kā\-ryaṃ ca kiñcid iti | nanv eṣa svabhā\-vā\-nanuyogo 'kā\-ryakā\-raṇabhū\-tā\-nā\-m api svabhā\-vapratibandhe tulya eva | tasmā\-d yat kiñcid etad api ||\edlabel{sarit__ratnakīrtinibandhāvali__124304}\label{sarit__ratnakīrtinibandhāvali__124304}
	\pend
      

	  \pstart kim asya sambandhasya vyā\-ptigrā\-hakaṃ pramā\-ṇam iti cet | ucyate
	\pend
      
	    
	    \stanza[\smallbreak]
bhū\-yodarśanagamyā\- hi vyā\-ptiḥ sā\-mā\-nyadharmayoḥ | \footnote{\begin{english}(ŚV, anumā\-na, 12)\end{english}}\&[\smallbreak]


	

	  \pstart iti prasiddham eva | asyā\-yam arthaḥ kā\-śikā\-kā\-reṇa vyā\-khyā\-taḥ—prā\-cī\-nā\-nekadarśanajanitasaṃskā\-rasahā\-ye carame \edtext{cetasi}{\Afootnote{ \cite{}darśane cetasi \cite{} \cite{}}} cakā\-sti dhū\-masyā\-gniniyatasvabhā\-vatvam, ratnatattvam iva parī\-kṣakasya, śabdatattvam iva vyā\-karaṇasmṛtisaṃskṛtasya, brā\-hmaṇatvam iva mā\-tā\-pitṛsambandhasmaraṇasacivasyetyā\-di | na hy etat sarvam ā\-pā\-tato na pratibhā\-tam iti purastā\-d api pratibhā\-samā\-nam anyathā\- bhavatī\-ti ||
	\pend
      

	  \pstart \persName{trilocanena} punar ayam arthaḥ kathitah – \edlabel{sarit__ratnakīrtinibandhāvali__125156}\label{sarit__ratnakīrtinibandhāvali__125156}\edtext{}{\lemma{bhūyodarśanena … iti ||}\xxref{sarit__ratnakīrtinibandhāvali__125156}{sarit__ratnakīrtinibandhāvali__125583}\Afootnote{\label{sarit__ratnakīrtinibandhāvali__399616}---\textsc{Note} Cf. \href{thakur75-106.16}{thakur75-106.16}.---\textsc{Note} Sanskrit, translation, and discussion of parallels in \cref{krasser02_zaGkar_Izvar_studie}.  {\rmlatinfont [App type: parallel]}}}bhū\-yodarśanena bhū\-yodarśanasahā\-yena manasā\- tajjā\-tī\-yā\-nā\-ṃ sambandho gṛhī\-to bhavati | ato dhū\-mo 'gniṃ na vyabhicarati | tadvyabhicā\-re 'py upā\-dhirahitaṃ sambandham atikrā\-met | hetor vipakṣaśaṅkā\-nivartakaṃ pramā\-ṇam upalabdhilakṣaṇaprā\-ptopā\-dhivirahaniścayahetur anupalambhā\-khyaṃ pratyakṣam eva | tataḥ siddhaḥ svā\-bhā\-vikaḥ sambandhaḥ | tathehā\-pī\-ti svamataṃ vyavasthā\-pitam iti ||\edlabel{sarit__ratnakīrtinibandhāvali__125583}\label{sarit__ratnakīrtinibandhāvali__125583}
	\pend
      

	  \pstart Vā\-caspatinā\-pī\-dam uktam – abhijā\-tamaṇibhedatattvavad bhū\-yodarśanajanitasaṃskā\-rasahā\-yam indriyam eva dhū\-mā\-dī\-nā\-ṃ vahnyā\-dibhiḥ svā\-bhā\-vikasambandhagrā\-hī\-ti yuktam iti ||
	\pend
      

	  \pstart atrocyate | \edtext{'bhede}{\Afootnote{bhede \cite{}}} sati tadutpatter anyaḥ svā\-bhā\-vikaḥ sambandhaḥ śabdā\-sphā\-lanamā\-tram evedam | na khalu nirū\-pyamā\-ṇaḥ prā\-pyate | tathā\- hi svā\-bhā\-vikas tu dhū\-mā\-dī\-nā\-ṃ vahnyā\-dibhiḥ sambandhaḥ tadupā\-dher anupalabhyamā\-natvā\-t | kvacid vyabhicā\-rasyā\-darśanā\-d iti tvayaivā\-sya lakṣaṇam uktam | etac cā\-siddham | yataḥ, upā\-dhiśabdena svato 'rthā\-ntaram evā\-pekṣaṇī\-yam abhidhā\-tavyam | na cā\-rthā\-ntaraṃ dṛśyatā\-niyatam, adṛśyasyā\-pi deśakā\-lasvabhā\-vaviprakṛṣṭasya sambhavā\-t | tataś ca dhū\-masyā\-pi hutā\-śena saha sambandhe syā\-d upā\-dhiḥ, na copalakṣyata iti katham adarśanā\-n nā\-sty eva yataḥ svā\-bhā\-vikasambandhasiddhiḥ ||
	\pend
      

	  \pstart atha \edtext{yady arthā}{\Afootnote{ \cite{}yadyathā\- \cite{}}}nataram apekṣaṇī\-yaṃ syā\-t | \edlabel{sarit__ratnakīrtinibandhāvali__126671}\label{sarit__ratnakīrtinibandhāvali__126671}\edtext{}{\lemma{kathaṃ … veti}\xxref{sarit__ratnakīrtinibandhāvali__126671}{sarit__ratnakīrtinibandhāvali__126898}\Afootnote{\label{sarit__ratnakīrtinibandhāvali__400122}---\textsc{Note} Translation, and parallels in \cref{krasser02_zaGkar_Izvar_studie}.  {\rmlatinfont [App type: parallel]}}}kathaṃ dhū\-ma ity eva pā\-vakasattā\-niyama iti cet | nanv idam eva cintyate | tadutpatter asvī\-kā\-re sahasraśo darśane 'pi kiṃ sarvatra dhū\-me saty avaśyam agniḥ sambhavī\- na veti\edlabel{sarit__ratnakīrtinibandhāvali__126898}\label{sarit__ratnakīrtinibandhāvali__126898} kadā\-cid arthā\-ntaram upā\-dhim apekṣya dhū\-mo 'pi syā\-n nā\-gnir iti kim atra niṣṭaṅkakā\-raṇam | tadupā\-dher anupalabhyamā\-natvā\-t | kvacid vyabhicā\-rasyā\-darśanā\-d iti tu yad uktaṃ tat pratyuktam eva | adṛśyasyā\-py upā\-dheḥ sambhā\-vyamā\-natvā\-t | vyabhicā\-rasya ca pratyayā\-ntaravaikalyenā\-hatyā\-darśane 'pi niṣeddham aśakyatvā\-t | ata eva tayor bā\-dhakā\-bhā\-ve 'pi sā\-dhakabā\-dhakapramā\-ṇā\-bhā\-vā\-t śaṅkā\- sambhavaty eva | na punas tavā\-munā\- viklavavikrośitamā\-treṇa vyā\-vartate | na caitavatā\- prā\-mā\-ṇikalokayā\-trā\-tikramaḥ | prā\-mā\-ṇikair eva sā\-dhakabā\-dhakapramā\-ṇā\-bhā\-ve nyā\-yaprā\-ptasya saṃśayasya vihitatvā\-t | na ca sarvatrā\-pravṛttiprasaṅgaḥ, pamā\-ṇā\-d arthasaṃśayā\-c ca pravṛtter upapatteḥ | na cā\-narthasandehaḥ sarvatra kartuṃ śakyate, kvacid arthonmukhatā\-yā\- eva darśanā\-t ||
	\pend
      

	  \pstart yac cā\-nyatvā\-viśeṣe 'pi kiñcid eva kā\-raṇaṃ kā\-ryaṃ ca kiñcid iti svabhā\-vo yathā\- na paryanuyojyas tathaiṣa svabhā\-vā\-nanuyogo 'kā\-ryakā\-raṇabhū\-tā\-\leavevmode\textsuperscript{\rmlatinfont\tiny [27a/RNAms]}nā\-m api svabhā\-va pratibandhe tulya eveti grā\-myajanadhandhī\-karaṇaṃ \edtext{vandī}{\Afootnote{ \cite{}prativandī\- \cite{}}}karaṇam atilā\-ghavam ā\-viskaroti vā\-caspateḥ | tathā\- hi vastutvā\-viśeṣe 'py agnir dahati nā\-kā\-śam ity atra yathā\- nā\-tiprasaṅgaḥ saṅgataḥ pramā\-ṇasiddhatvā\-d asyā\-rthasya, tathā\- bhedā\-viśeṣe 'pi kiñcid eva kasyacit kā\-raṇaṃ kā\-ryaṃ ca kiñcid ity atrā\-pi nā\-tiprasaṅgā\-vatā\-raḥ | bhede sati tadanvayavyatirekā\-nuvidhā\-nalakṣaṇasya kā\-ryakā\-raṇabhā\-vasya pramā\-ṇasiddhatvā\-d eva | na caivaṃ svā\-bhā\-vikasambandhaśabdavā\-cyo 'rthaḥ pramā\-ṇasiddhaḥ kaścid asti, tallakṣaṇasyā\-siddhatvā\-d uktatvā\-t | na ca pratijñā\-siddhe vastuny atiprasaṅgo nā\-bhaidhā\-tavyaḥ, sarveṣā\-ṃ sarvatra tadrū\-pā\-bhyupagamamā\-treṇa vijetṛtvaprasaṅgā\-t | yad ā\-hā\-laṅkā\-rakā\-raḥ:
	\pend
      
	    
	    \stanza[\smallbreak]
yat kiñcid ā\-tmā\-bhimataṃ vidhā\-ya niruttaras tatra kṛtaḥ pareṇa |&vastusvabhā\-vair iti vā\-cyam itthaṃ tathottaraṃ syā\-d vijayī\- samastaḥ ||\&[\smallbreak]


	

	  \pstart iti ||
	\pend
      

	  \pstart kiṃ ca svā\-bhā\-vikasambandha iti ko 'rthaḥ | kiṃ svato bhū\-taḥ svahetuto bhū\-to 'hetuko veti trayaḥ pakṣā\-h | na tā\-vad ā\-dyaḥ pakṣaḥ, svā\-tmani kā\-ritravirodhā\-t | dvitī\-yapakṣe tu tadutpattir eva sambandho mukhā\-ntareṇa svī\-kṛta iti na kaścid vivā\-daḥ |ahetukatve tu deśakā\-lasvabhā\-vaniyamā\-bhā\-vaprasaṅgā\-d ity asaṅgataḥ svā\-bhā\-vikaḥ sambandhaḥ ||
	\pend
      

	  \pstart etena yad uktam: na sapakṣā\-sapakṣayor darśanā\-darśanā\-bhyā\-ṃ kā\-ryatvasya gamakatvam api tu svā\-bhā\-vikasambandhabalā\-d iti brū\-maḥ, sa eva tu sapakṣā\-sapakṣayor darśanā\-darśanā\-bhyā\-ṃ vakṣyamā\-ṇena krameṇa pratī\-yata iti, tadiṣṭakā\-matā\-mā\-trā\-viṣkaraṇam iti mantavyam | svā\-bhā\-vikasambandhasya hy upā\-dhinirapekṣaniyatatvaṃ lakṣaṇam uktam | tasya coktanyā\-yenā\-siddhau bhū\-yodarśanajanitasaṃskā\-rasahā\-ye carame cetasi manasi vā\- tathā\-bhū\-taṃ niyatatvaṃ parisphuratī\-ti \edtext{sahṛdayena}{\Afootnote{sa\deletion{ha}\add{hṛ}dayena \cite{}; sadayena \cite{}  {\rmlatinfont [App type: var]}}} vaktum aśakyatvā\-t |
	\pend
      

	  \pstart yac ca śabdatattvam iva brā\-hmaṇatvam iveti dṛṣṭā\-ntī\-kṛtaṃ tad dvayam apy asmā\-n pratyasiddham iti dṛṣṭā\-ntayitum anucitam | abhijā\-tamaṇibhedatattvaṃ tu parisphuratī\-ti yuktam | tasya hy upadeśaparamparā\-to mā\-ṇikyavattenā\-pi kaṣṭenendradhanurā\-kā\-rajyotirā\-dikaṃ lakṣaṇaṃ niścitam | na caivaṃ svā\-bhā\-vikasambandhalakṣaṇaṃ tvayā\- svakapolaracitam api pramā\-ṇena niścitam | yenā\-syā\-pi tā\-dṛśī\- vyavasthā\- syā\-d iti yat kiñcid etat ||
	\pend
      

	  \pstart kiṃ ca bhavatu tā\-vad ayam anavadhā\-ritarū\-paḥ svā\-bhā\-vikaḥ sambandhaḥ, tathā\-pi darśanā\-darśanā\-bhyā\-m asya grahaṇam atidurlabham | tathā\- hi yadi prā\-cī\-nā\-nekadarśanajanitasaṃskā\-rasahā\-yena caramacetasā\- dhū\-masyā\-gniniyatatvaṃ grā\-hyaṃ tadā\- sapakṣā\-sapakṣayoḥ koṭiśaḥ pravṛttadarśanā\-darśanajanitasaṃskā\-rasahā\-yena caramacetasā\- pā\-rthivatvasyā\-pi lohalekhyatvaniyatatvaṃ gṛhyata iti pā\-rthivatvā\-d api lohalekhyatvasiddhir astu | atha pā\-rthivatvasya lohalekhyatvaniyatatvam eva nā\-sti vajre vyabhicā\-radarśanā\-t | tat kathaṃ pratyakṣeṇa niyatatvagrahaḥ | tarhi dhū\-masya vahniniyatatvam eva nā\-sti, vyabhicā\-rā\-bhā\-vasya darśayitum aśakyatvā\-t | tat kathaṃ caramacittena niyamagraha ity apy tulyam |
	\pend
      

	  \pstart vyabhicā\-rā\-darśanā\-d avyabhicā\-ra iti cet | nanu vyabhicā\-rā\-darśanā\-d avyabhicā\-ra iti ko 'rthaḥ | kiṃ vyabhicā\-rā\-darśanā\-d avyabhicā\-raḥ, vyabhicā\-rā\-bhā\-vā\-d vā\- | prathame pakṣe vyabhicā\-ro bhavatu mā\- vā\- vyabhicā\-rā\-darśanā\-d evā\-vyabhicā\-ra iti niṣṇā\-taṃ pā\-ṇḍityam | atha dvitī\-yaḥ pakṣaḥ | tadā\- vyabhicā\-rā\-bhā\-vaḥ kuto jñā\-taḥ | adarśanā\-d iti cet | tat kim adarśanamā\-traṃ dṛśyā\-darśanaṃ vā\- | prathamam aśaktam | na hy adarśane 'pi vyabhicā\-ro nā\-stī\-ty abhidhā\-tuṃ śakyate, cirakā\-lanaṣṭabrā\-hmaṇī\-vyabhicā\-ravat | ā\-hatyā\-darśane 'py aticirakā\-lavyavadhā\-nena vyabhicā\-radarśanā\-t | dvitī\-yaṃ cā\-sambhavi, kvacit kadā\-cit kenacid vyabhicā\-radarśanasā\-magryā\-ṃ satyā\-ṃ vyabhicā\-radarśanā\-t | darśana\edtext{sāmagrībhāve}{\Afootnote{°sā\-magrī\-bhā\-ve \cite{}; sā\-magryabhā\-ve \cite{}  {\rmlatinfont [App type: var]}}} tu pratyayā\-ntaravaikalyā\-t deśakā\-lā\-ntaravartitvā\-d vā\- vyabhicā\-rasya \edtext{sarvaṃ pratyupalabdhi}{\Afootnote{sa\unclear{rvaṃ pratyu}palabdhi° \cite{}; sarvaṃ pratyupalabdhi \cite{}  {\rmlatinfont [App type: var]}}}lakṣaṇaprā\-ptatvā\-bhā\-vā\-t | tasmā\-t saty api vyabhicā\-re tadupalambhasā\-magryabhā\-vā\-d vyabhicā\-rā\-nupalambhaḥ | prakā\-rā\-ntareṇa vā\- tadutpattilakṣaṇenā\-vyabhicā\-re vyabhicā\-rā\-nupalambha ity ubhayathā\-pi vyabhicā\-ropalambhanivṛttir astu | tvayā\- tu yad avyabhicā\-rapratipattinibandhanaṃ darśanā\-darśanam upavarṇitaṃ tatpā\-rthivatvā\-dau vyabhicā\-rā\-d dhū\-me 'pi nā\-vyabhicā\-ranibandhanam iti dhū\-mo 'pi tvanmate nā\-śvā\-sabhā\-jam iti prasaktam |
	\pend
      

	  \pstart asmanmate tu pratyakṣā\-nupalambhā\-bhyā\-m ekatra kā\-ryakā\-raṇabhā\-vasiddhau na vyabhicā\-raśaṅkā\-sambhavaḥ | tadabhā\-ve tu: hetumattā\-ṃ vilaṅghayed \footnote{\begin{english}(PV I 34d)\end{english}} iti nyā\-yā\-t na saṃśayapiśā\-cā\-vasaraḥ | tad evaṃ bhū\-yodarśanā\-darśanā\-bhyā\-m api na vyā\-ptisiddhiḥ |
	\pend
      
	  
	% new div opening: depth here is 4
	

	  \pstart tarhi sakṛt sapakṣā\-sapakṣayor darśanā\-darśanā\-bhyā\-m vyā\-pter niścaya iti caturtha eva pakṣo 'stu | tathā\- hi kā\-ryatvasya buddhimanmā\-trapū\-rvakatvenā\-nvayo ghaṭā\-dau dṛṣtaḥ, ā\-kā\-śā\-dau buddhimatkā\-raṇanivṛttau kā\-ryatvasya vyatirekaḥ | tataś ca sakṛdanvayavyatirekasiddhau vyā\-pteḥ siddhatvā\-t kuto 'naikā\-ntikatā\- |
	\pend
      

	  \pstart atrā\-bhidhī\-yate | yadi buddhimatkā\-raṇakā\-ryatvayor ekatra pratibandhaḥ pramā\-ṇapratī\-taḥ syā\-t tadā\-kā\-śā\-dau buddhimannnivṛttau kā\-ryatvasya nivṛttir iti yuktam | sa ca pratibandhaḥ tā\-dā\-tmyaṃ tadutpattiḥ svā\-bhā\-viko 'nyo vā\- na sidhyati sā\-dhakapramā\-ṇā\-bhā\-vā\-d ity anantaram evā\-veditam | tataś cā\-kā\-śā\-dau buddhimannivṛttir api syā\-t | na ca kā\-ryatvasya nivṛttir iti sandigdhavipakṣavyā\-vṛttikatvā\-d anaikā\-ntikaṃ kā\-ryatvam |
	\pend
      

	  \pstart nanv ā\-kā\-śasyā\-samanmate nityatvaṃ tvanmate cā\-sattvam | tat katham ataḥ kā\-ryatvavyatirekaḥ sandigdha iti cet | ucyate | na hy ā\-kā\-śe \edtext{kāryatvavyā}{\Afootnote{kā\-ryatvavyā\-° \cite{}; kā\-ryavyā\-° \cite{}  {\rmlatinfont [App type: var]}}}vṛttimā\-traṃ vyatirekaḥ | kin tu sā\-dhyā\-bhā\-vaprayuktaḥ sā\-dhanā\-bhā\-vo vyatirekaḥ | sa cā\-kā\-śe grahī\-tum aśakyaḥ | yathā\- tatra buddhimatkā\-raṇanivṛttis tathā\- \edtext{'cetanasyāpi kāraṇasya nivṛttiḥ |}{\Afootnote{kā\-raṇa'cetanasyā\-\deletion{pi}kā\-raṇanivṛttiḥ | \cite{}; kā\-raṇamā\-trasyā\-pi nivṛttiḥ | \cite{}; acetanasyā\-pi kā\-raṇasya nivṛttir \cite{}  {\rmlatinfont [App type: em]}}} tat kasyā\-bhā\-vaprayuktaḥ kā\-ryā\-bhā\-vaḥ pratī\-yatā\-ṃ yena vyatirekaḥ sidhyati ||
	\pend
      

	  \pstart nanu satyam evaitat | yathā\-kā\-śe buddhimatkā\-raṇanivṛttis tathā\- kā\-raṇamā\-trasyā\-pi tatra nivṛttir na buddhimatkā\-raṇavyatirekā\-nuvidhā\-yitvaṃ kā\-ryatvasya niścetuṃ śakyate | tathā\-pi ghatā\-dau kā\-ryatvasya buddhimatā\-nvayadarśanā\-kā\-śe 'pi buddhimadabhā\-vaprayuktaḥ kā\-ryatvā\-bhā\-vaḥ pratī\-yate | tat kathaṃ vyatirekā\-siddhir iti cet | hanta ghaṭā\-dā\-v api na kā\-ryatvasya sattā\-mā\-tram anvayaḥ | kiṃ tu sā\-dhyasadbhā\-vaprayuktaḥ sā\-dhanasadbhā\-vaḥ | sa ca ghaṭe grahī\-tum aśakyaḥ | yathā\- hi tatra buddhimadbhā\-vas tathā\- kaṭakuḍyā\-dibhā\-vo 'pi | tat ka evaṃ jā\-nā\-tu kiṃ buddhimadbhā\-ve kā\-ryatvasya bhā\-vo yad vā\- kaṭakuḍyā\-dibhā\-ve bhā\-va iti | tasmā\-d atra viśiṣṭā\-nvayavyatirekagrahaṇapravaṇaviśiṣṭapratyakṣā\-nupalambhā\-v anusarta\edtext{vyau yad dṛśyayor eva kāryakāraṇayos}{\Afootnote{vyau | ya\unclear{}\add{\unclear{dṛśyayor eva kā\-ryakā\-raṇayo}}s \cite{}; vyau yad dṛśyayor eva kā\-ryakā\-raṇayos \cite{}  {\rmlatinfont [App type: var]}}} tadutpattisiddhā\-v anvayavyatirekau sidhyataḥ ||
	\pend
      

	  \pstart na ca pratibandhasā\-dhakaṃ pramā\-ṇaṃ svapne 'py astī\-ti caturtho 'pi pakṣaḥ kṣataḥ |
	\pend
      
	  
	% new div opening: depth here is 4
	

	  \pstart tad evaṃ buddhimatkā\-ryamā\-trayor vyā\-pter asiddhā\-v adhikaraṇasiddhā\-nta\footnote{A separate hand adds \begin{sanskrit}yasminna〔rthe〕 sidhyanti tadanuyā\-yī\-nya〔rthā\-〕ntarā\-〔ṇi〕 sidhyanti so 'dhikaraṇasiddhā\-ntaḥ |\end{sanskrit} Cf. \href{NVTṬ\#nvtṭ-ad-nsū_1-1.30}{NVTṬ\#nvtṭ-ad-nsū\-\textunderscore 1-1.30}.}\edtext{nyāyād upā}{\Afootnote{nyā\-yā\-dupā\- \cite{}; nyā\-yā\-dyupā\- \cite{}  {\rmlatinfont [App type: var]}}}dā\-nā\-dyabhijñaḥ sarvajñaḥ puruṣaviśeṣaḥ sidhyatī\-ti pratyā\-śā\- durā\-śaiva ||
	\pend
      
	  
	% new div opening: depth here is 4
	

	  \pstart yac ca kriyā\-sā\-mā\-nyasya pakṣadharmatā\-vaśā\-c cakṣurlakṣaṇakaraṇaviśeṣasiddhir iti dṛṣṭā\-nto darśitaḥ so 'pi sā\-dhyā\-bhinnaḥ | tatra hi rū\-pajñā\-nā\-nyathā\-nupapattyā\- siddhasya kā\-raṇā\-ntarasyaiva cakṣur indriyam iti nā\-makaraṇā\-t | rū\-pajñā\-najanakatvā\-tiriktasya cakṣurlakṣaṇaviśeṣasyā\-siddhatvā\-t | atha rū\-pajñā\-najanakatvam eva cakṣuṣṭvam ucyate | bhavatu ko doṣaḥ | etad evā\-smā\-bhiḥ kā\-raṇā\-ntaram ucyate | tathaiva yadi tvayā\-pi buddhimatsā\-mā\-nyā\-śrayamā\-trasya puruṣaviśeṣa iti nā\-ma kriyate, tadā\- nā\-smā\-kaṃ kā\-dacid vipratipattiḥ | paramā\-rthato buddhimatsā\-mā\-nyā\-śraye sarvajñatvā\-diviśeṣaś cakṣurā\-diviśeṣavat sidhyatī\-ti tatra vivadā\-mahe | ubhayor api dṛṣṭā\-ntadā\-rṣṭā\-ntikayor viśeṣasā\-dhanasā\-marthyā\-bhā\-vā\-t ||
	\pend
      
	  
	% new div opening: depth here is 4
	

	  \pstart tad ayaṃ saṃkṣepā\-rthaḥ:
	\pend
      
	    
	    \stanza[\smallbreak]
dṛśye tu sā\-dhye vyabhicā\-ra eva dṛśyaṃ na cen na \edtext{vyatireka}{\Afootnote{vyatireka \cite{}; vyabhicā\-ra \cite{}  {\rmlatinfont [App type: var]}}}siddhiḥ |&sā\-dhā\-raṇatvā\-d atha vā\- vipakṣasandehataḥ sā\-dhyam ato na sidhyati ||\footnote{Cf. \href{JNA\#jnā-sādhye-ca-dṛśye}{JNA\#jnā\--sā\-dhye-ca-dṛśye}.}\&[\smallbreak]


	

	  \pstart itī\-śvaro dattā\-jalā\-ñjaliḥ ||
	\pend
      \label{īsd-sādhanasvarūpa}\edlabel{īsd-sādhanasvarūpa}
	  
	% new div opening: depth here is 3
	

	  \pstart idā\-nī\-ṃ sā\-dhanasvarū\-paṃ nirū\-pyate | yad etan merumandaramedinī\-ghaṭapaṭā\-disā\-dhā\-raṇaṃ kā\-ryamā\-traṃ sā\-dhanam upanyastam yā\-vad asya buddhimadanvayavyatirekā\-nuvidhā\-nam ekatra nā\-vadhā\-ryate tā\-vad gamakatvam ayuktam | na ca tat svapne 'pi pratyetuṃ śakyam | tathā\- hi kumbhakā\-ravyā\-pā\-re sati mṛtpiṇḍā\-d ghaṭalakṣaṇaṃ kā\-ryam upalabhyatā\-ṃ nā\-ma | na tu vyā\-pā\-rā\-t pū\-rvaṃ ghaṭavatkā\-ryamā\-trasya vyatirekaḥ pratyetuṃ śakyaḥ, kumbhakā\-ravyatireke 'pi śoṣabhaṅgā\-dilakṣaṇasya kā\-ryasya mṛtpiṇḍe darśanā\-t | na ca yad vinā\-bhū\-taṃ yad upalabhyate tat tasya kā\-ryam atiprasaṅgā\-t | tṛṇā\-divanmṛtpiṇḍasya śoṣabhaṅgā\-dikā\-ryamā\-tram api pakṣī\-kṛtam iti cet | kriyatā\-ṃ buddhimadvyatireke kā\-ryamā\-travyatirekas tv ektrā\-pi pratipā\-dyatā\-ṃ yena vyā\-ptisiddhau \edtext{tṛṇādir}{\Afootnote{tṛṇā\-di\deletion{|}\gap{}r \cite{}; tṛṇā\-dir \cite{}  {\rmlatinfont [App type: var]}}} iva śoṣabhaṅgā\-der api buddhimadanumā\-naṃ syā\-t | ā\-kā\-śā\-divaidharmyadṛṣṭā\-ntas tu pū\-rva pratihataḥ, buddhimatpū\-rvakatvasyeva kā\-raṇmā\-trapū\-rvakatvasyā\-pi tatra sambhavā\-t kiṃprayuktaḥ kā\-ryatvā\-bhā\-va ity aparijñā\-nā\-t ||
	\pend
      

	  \pstart etena yad uktam - na vyabhicā\-ropalambhā\-t prā\-tisvikaviśeṣaparityā\-gena ghaṭā\-dī\-nā\-m abhū\-tvā\-bhavanā\-d anyarū\-paṃ viśeṣam upalakṣayā\-mo yanniṣṭhaṃ puruṣapū\-rvakatvaṃ vyavasthā\-payā\-ma iti tad api prativyū\-ḍham | kumbhakā\-rā\-dyabhā\-ve 'pi mṛtpiṇḍā\-dau śoṣabhaṅgā\-dikā\-ryadarśanā\-d abhū\-tvā\- \leavevmode\textsuperscript{\rmlatinfont\tiny [29a/RNAms]}\label{RNAms-29a} bhā\-valakṣaṇasya kā\-ryamā\-trasya vyatirekā\-siddher vyā\-pter abhā\-vā\-t ||
	\pend
      

	  \pstart nanu yadi kā\-ryatvamā\-trasya na buddhimatā\- pratyakṣato vyā\-ptigrahaḥ vyatirekā\-bhā\-vā\-t, tvayā\-pi tarhi kathaṃ kṛtakatvasyā\-nityatvena vyā\-ptir avadhā\-rayta iti cet | anapekṣā\-lakṣaṇaviparyayabā\-dhakapramā\-ṇabalā\-d iti brū\-maḥ | tac cā\-tadrū\-paparā\-vṛttasyaiva kṛtakatvasya vipakṣā\-d vyatirekaṃ sā\-dhayati | na ca tvayā\- viparyayabā\-dhakapramā\-ṇam abhidhā\-tuṃ śakyata iti prā\-g eva pratipā\-ditam | sandigdhavipakṣavyā\-vṛttikatvā\-d anaikā\-ntikam idaṃ kā\-ryatvamā\-tram ||
	\pend
      

	  \pstart etena yad etat naiyā\-yikā\-nā\-m ā\-kṣepaparihā\-raviḍambanam | iha khalu dve kā\-ryatve | kā\-ryamā\-tram | viśiṣṭaṃ ca | tatrā\-dyasya pratibandhā\-siddher anaikā\-ntikatvam | viśiṣṭasya bhū\-dharā\-diṣv asambhavā\-d asiddhatvam iti | tad asaṅgatam | kā\-ryatvamā\-trasyaiva pratibandhopapā\-danā\-t ||
	\pend
      

	  \pstart yac coktaṃ viśiṣṭaṃ kā\-ryatvam iti | kī\-dṛśaṃ punas tad iti vaktavyam | atha yat kā\-ryaṃ puruṣā\-nvayavyatirekā\-nuvidhā\-yitayā\- tatpū\-rvakam upalabdham | yaddṛṣṭer akriyā\-darśino 'pi kṛtabuddhir utpadyate tat kā\-ryaṃ sakalaprā\-sā\-dā\-dyanugataṃ bhū\-dharā\-divyā\-vṛttaṃ viśiṣṭam ity abhidhī\-yate | tad asundaram | vikalpā\-nupapatteḥ ||
	\pend
      

	  \pstart \edlabel{sarit__ratnakīrtinibandhāvali__139880}\label{sarit__ratnakīrtinibandhāvali__139880}\edtext{}{\lemma{tathā … iti ||}\xxref{sarit__ratnakīrtinibandhāvali__139880}{sarit__ratnakīrtinibandhāvali__140214}\Afootnote{\label{sarit__ratnakīrtinibandhāvali__400579}---\textsc{Note} Sanskrit and translation in \cref{krasser02_zaGkar_Izvar_studie}.  {\rmlatinfont [App type: parallel]}}}tathā\- cā\-ha śaṅkaraḥ—kṛtabuddhiḥ kiṃ sā\-dhyabuddhiḥ kiṃ vā\- sā\-dhanabuddhiḥ | sā\-dhyabuddhir api yadi gṛhī\-tavyā\-ptikasya, sā\- bhavaty eva | athā\-gṛhī\-tavyā\-ptikasya, kim anyatrā\-pi sā\- bhavantī\- dṛṣṭā\- | atha sā\-dhanabuddhiḥ | tarhi svopagamavirodhaḥ, sarvasya bhā\-vasya kṛtakatvopagamā\-d iti ||\edlabel{sarit__ratnakīrtinibandhāvali__140214}\label{sarit__ratnakīrtinibandhāvali__140214}
	\pend
      

	  \pstart vā\-caspatiḥ punar atrā\-ha - idam atra nipuṇataram nirū\-payatu bhavā\-n kiṃ buddhimadanvayavyatirekā\-nuvidhā\-naṃ viśeṣaḥ | ā\-hosvit tad darśanaṃ yat parvatā\-diṣu nastī\-ty abhidhī\-yate | yadi pū\-rvakaḥ kalpaḥ, sa buddhimaddhetukatvaṃ tanubhuvanā\-dī\-nā\-m ā\-tiṣṭhamā\-nair abhyupeyata eva | na hi kā\-raṇaṃ kā\-ryā\-nanuvihitabhā\-vā\-bhā\-vam anyo vaktyahrī\-kā\-t | atha taddarśanam iti caramaḥ kalpaḥ | na tarhi akriyā\-darśinaḥ kṛtabuddhisambhavaḥ | ya eva hi ghaṭo 'nena buddhimadanvayavyatirekā\-nuvidhā\-yī\- dṛṣṭaḥ, sa eva kā\-ryo na tu vipaṇivartī\- | tajjā\-tī\-yasya tadanvayavytirekā\-nuvidhā\-nadarśanā\-d adṛṣṭā\-nvayavyatirekā\-nuvidhā\-nam api tajjā\-tī\-yaṃ tatheti cet | hantotpattimadghaṭā\-di buddhimadanvayavyatirekā\-nuvidhā\-yī\-ti anyad api tanubhuvanā\-dikaṃ tathā\- bhavan na daṇḍena parā\-ṇudya\leavevmode\textsuperscript{\rmlatinfont\tiny [29b/RNAms]}\label{RNAms-29b}\edtext{te | ghaṭa}{\Afootnote{\unclear{ghaṭa} \cite{}; te | ghaṭa \cite{}  {\rmlatinfont [App type: var]}}}jā\-tī\-yam utpattimadbuddhimatpū\-rvakam iti cet | nanu prā\-sā\-dā\-di taddhetukaṃ na bhavet | aghaṭajā\-tī\-yatvā\-t | atha yajjā\-tī\-yam anvayavyatirekā\-nuvidhā\-yi dṛṣṭam,tajjā\-tī\-yam evā\-dṛṣṭā\-nvayavyatirekam api taddhetukam | tat kiṃ kā\-ryajā\-tī\-yaṃ prā\-sā\-dā\-di buddhimaddhetukaṃ na dṛṣṭam yenotpattimattanubhuvanā\-di tathā\- na syā\-t | na khalu tajjā\-tī\-yatve kaścid viśeṣa iti ||
	\pend
      

	  \pstart \edlabel{sarit__ratnakīrtinibandhāvali__141640}\label{sarit__ratnakīrtinibandhāvali__141640}\edtext{}{\lemma{vittokas … etad}\xxref{sarit__ratnakīrtinibandhāvali__141640}{sarit__ratnakīrtinibandhāvali__143038}\Afootnote{\label{sarit__ratnakīrtinibandhāvali__401027}---\textsc{Note} Sanskrit and translation in \cref{krasser02_zaGkar_Izvar_studie}.  {\rmlatinfont [App type: parallel]}}}vittokas tv ā\-ha—bhavatu vā\- kaścid anirū\-pitarū\-po viśeṣaḥ | kiṃ punar anena viśeṣaṃ pratipā\-dayatā\-bhipretam | kiṃ kā\-ryatvasā\-mā\-nyasyā\-siddhatvam | atha kā\-ryaviśeṣasya | atha kā\-ryamā\-trasya buddhimatkartṛvyabhicā\-raḥ | atha sā\-dhyadṛṣṭā\-ntayor vaidharmyamā\-tram | kiṃ cā\-taḥ | yadi tā\-vat kā\-ryasā\-mā\-nyasyā\-siddhatvam | tan nā\-sti | viśvambharā\-diṣv api kā\-raṇavyā\-pā\-rajanyatvasyobhayasiddhatvā\-t | atha kā\-ryaviśeṣasya kumbhā\-divartinaḥ pakṣe 'siddhir abhidhī\-yate | tadā\- na kā\-cid atra kṣatir viśeṣasya hetutvenā\-nupā\-dā\-nā\-t | yadi kā\-ryasā\-mā\-nyasya kartṛvyabhicā\-raḥ pratipā\-dayitum iṣṭaḥ | sa na śakyo vipakṣe 'darśanā\-t | tṛṇā\-deś ca pakṣī\-kṛtatvā\-t | śaṅkā\-mā\-trasya \edtext{sarvathā'ni}{\Afootnote{sarvathā\-'ni \cite{}; sarvathā\-ni \cite{}  {\rmlatinfont [App type: var]}}}ṣiddhatvā\-t | sandigdhavyatirekitvaṃ naiyā\-yikā\-nā\-ṃ niranuyojyā\-nuyogo bauddhā\-nā\-m adoṣodbhā\-vanaṃ nigrahasthā\-nam iti tu pratipā\-ditam | tathā\-pi bā\-dhakapramā\-ṇā\-ny abhiditā\-ny eva |
	\pend
      

	  \pstart tasmā\-n na pratibandhā\-siddheḥ sarvatra vyabhicā\-rā\-śaṅkā\- | atha sā\-dhyadṛṣṭā\-ntayor vaidharmyodbhā\-vanam | tan na | tasya sarvatra sulabhatvā\-t | yadi sā\-dhyadṛṣṭā\-ntayor vaidharmyamā\-trā\-t sā\-dhyā\-siddhiḥ nivṛttedā\-nī\-m anumā\-navā\-rtā\-pi nikuñjamahā\-nasayor api dhū\-mavattve 'pi kathañcid vaidharmyopapatter iti sakalaṃ yat kiñcid etad\edlabel{sarit__ratnakīrtinibandhāvali__143038}\label{sarit__ratnakīrtinibandhāvali__143038} iti |
	\pend
      

	  \pstart tad ayam atra saṃkṣepā\-rthaḥ | yat tā\-vat kā\-ryatvamā\-traṃ tadevoktena krameṇa pratibandhasiddher bhū\-dharā\-diṣu dṛṣṭaṃ puruṣam anumā\-payatī\-ty asmā\-kam abhimata\edtext{sādhyasiddhir}{\Afootnote{sā\-dhyasiddhir \cite{}; sā\-dhyam asiddhir \cite{}  {\rmlatinfont [App type: var]}}} upapannaiveti | kim asmā\-kam adhikacintayety aṅgī\-kṛtyā\-py uktaṃ viśiṣṭakā\-ryatvam | tad eva tu nā\-stī\-ti punar vistareṇa pratipā\-ditam iti tad api sarvam anavadheyam eva | tathā\- hi kā\-ryatvamā\-trasya tā\-vad uktena krameṇa vyā\-pter asiddhatvā\-d anaikā\-ntikatvam anirvā\-yam | yac ca viśiṣṭakā\-ryatvaṃ vikalpya dū\-ṣitaṃ tasyā\-smā\-bhir anabhyupagatatvā\-t taddū\-ṣaṇā\-ya prabandhaḥ prayā\-saikaphalaḥ | na hi kā\-ryatvaṃ dvividham abhimatam | ekaṃ sarvakā\-ryā\-nugatam, aparaṃ parvatā\-divyā\-vṛttaṃ ghaṭapaṭaprā\-sā\-dā\-dyanuyā\-yī\-ti | kiṃ tu kā\-ryam anekajā\-tī\-yakam | tatra yadi nā\-ma paṭasya prā\-sā\-dā\-dibhiḥ saha vastutvasaṃsthā\-naviśeṣayogitvakā\-ryatvā\-dibhir dharmaiḥ sajā\-tī\-yatvam asti tathā\-pi na tā\-n dharmā\-n buddhimatpū\-rvakā\-nadhigacchati vyā\-vahā\-rikaṃ pratyakṣaṃ, kā\-ryatvā\-dī\-nā\-ṃ buddhimadvyatirekā\-nuvidhā\-nā\-bhā\-vā\-t | tat kathaṃ prā\-sā\-daparvatā\-diṣu kā\-ryatvā\-didarśanā\-d buddhimadanumā\-nam astu | kiṃ tu yasyaiva ghaṭajā\-tī\-yakā\-ryacakrasya vyatirekasiddhis tasya buddhimadvyā\-ptatvaṃ pratyakṣataḥ sidhyatī\-ty uktam | tena deśakā\-lā\-ntare ghaṭajā\-tī\-yā\-d eva buddhimadanumā\-nam | yadā\- tu prā\-sā\-dajā\-tī\-yakam api buddhimaddhetukam ekatra pṛthag avadhā\-ryate tadā\- tajjā\-tī\-yā\-d api buddhimatsiddhiḥ | evaṃ tattajjā\-tī\-yasarā\-vodañcanaśakaṭapaṭakeyū\-raprabhṛtteḥ kā\-ryacakrā\-d buddhimatpū\-rvakatvena pṛthak pṛthag avadhā\-ritā\-d buddhimadanumā\-nam anavadyam |
	\pend
      

	  \pstart amum evā\-rtham abhisandhā\-yā\-cā\-ryapā\-dair abhihitam:
	\pend
      
	    
	    \stanza[\smallbreak]
siddhaṃ yā\-dṛg adhiṣṭhā\-tṛbhā\-vā\-bhā\-vā\-nuvṛttimat |&sanniveśā\-di tad yuktaṃ tasmā\-d yad anumī\-yate || \footnote{\begin{english}(PV II 11)\end{english}}\&[\smallbreak]


	

	  \pstart iti | evaṃ ghaṭapaṭaparvatadī\-nā\-ṃ kā\-ryatvavastutvā\-dibhir dharmaiḥ sajā\-tī\-yatve 'py avā\-ntaraṃ ghaṭapaṭaparvatatvā\-dijā\-tibhedam ā\-dā\-ya lokasya vyā\-ptigrā\-hakaṃ pratyakṣaṃ pravartata iti darśayituṃ saṃvyavahā\-rapragalbhapuruṣabuddhyapekṣayā\- yaddarśanā\-d akriyā\-darśino 'pi kṛtabuddhir bhavatī\-ty uktam | na tu śā\-straparavaśabuddhipuruṣā\-pekṣayā\- | tathā\- hi śā\-strasaṃskā\-rarahitasya vyavahā\-rapragalbhasya puruṣasya devakulajā\-tī\-yakaṃ puruṣapū\-rvakatayā\-vadhā\-ritavato nagarā\-d vanaṃ praviṣṭasya parvatadevakulayor darśane tayor dvayor apy akriyā\-darśino 'pi devakule kṛtabuddhir bhavati na parvate | tad anayor devakulaparvatayoḥ kā\-ryatvā\-dinā\- ekajā\-titve 'pi kṛtabuddhibhā\-vā\-bhā\-vau na tayoḥ parvatadevakulatvalakṣaṇā\-vā\-ntarajā\-tibhedam anavasthā\-pya sthā\-tuṃ prabhavataḥ | jā\-tibhede ca siddhe devakulajā\-tī\-ye vyā\-pter grahaṇā\-t na parvatajā\-tī\-yasya, na ca prā\-sā\-dajā\-tī\-yasya vyā\-ptisiddhir iti na tato buddhimadanumā\-nam | yadā\- tu prā\-sā\-dasyā\-pi pṛthag vyā\-ptigrahaḥ tadā\- tajjā\-tī\-yā\-d api buddhimadanumā\-nam astu | na kṣitidharā\-dijā\-tī\-yasya svapne 'pi vyā\-ptigrahaḥ | krī\-ḍā\-parvatā\-der nā\-mamā\-trā\-\leavevmode\textsuperscript{\rmlatinfont\tiny [30b/RNAms]}\label{RNAms_30b}bhede 'pi parvatā\-dibhir ekā\-ntato bhinnasvarū\-patvā\-t | \edlabel{sarit__ratnakīrtinibandhāvali__146084}\label{sarit__ratnakīrtinibandhāvali__146084}\edtext{}{\lemma{yac … brūmaḥ |}\xxref{sarit__ratnakīrtinibandhāvali__146084}{sarit__ratnakīrtinibandhāvali__146582}\Afootnote{\label{sarit__ratnakīrtinibandhāvali__401478}---\textsc{Note} Sanskrit quoted in \cref{krasser02_zaGkar_Izvar_studie}.  {\rmlatinfont [App type: parallel]}}}yac ca pṛṣṭaṃ keyaṃ kṛtabuddhir ityā\-di | tatra kā\-maṃ sā\-dhyabuddhir eveti brū\-maḥ | yac cā\-troktaṃ sā\-dhyabuddhir api yadi gṛhī\-tavyā\-ptikasya sā\- bhavaty eva | athā\-gṛhī\-tavyā\-ptikasya kim anyatrā\-pi sā\- bhavatī\- dṛṣṭeti ||
	\pend
      

	  \pstart atrocyate | gṛhī\-tavyā\-ptikasyā\-numā\-naṃ bhavati, agṛhī\-tavyā\-ptikasya na bhavatī\-ty atrā\-smā\-kaṃ na kā\-cid vipratipattiḥ | kevalaṃ gṛhī\-tavyā\-ptiko 'smin viṣaye na sambhavatī\-ti brū\-maḥ |\edlabel{sarit__ratnakīrtinibandhāvali__146582}\label{sarit__ratnakīrtinibandhāvali__146582} uktakrameṇa vyatirekā\-siddher vyā\-vahā\-rikapratyakṣeṇa kā\-ryatvasya vyā\-ptatvā\-niścayā\-t | tasmā\-d avā\-ntarajā\-tibhedaprasiddhyarthaṃ vyā\-vahā\-rikapuruṣā\-pekṣayaivā\-syā\- buddher bhā\-vā\-bhā\-vā\-v uktau | jā\-tibhede ca prayojanaṃ pū\-rvam eva pratipā\-ditam |
	\pend
      

	  \pstart yad apy atra \edlabel{rnā__144871}\label{rnā__144871}\edtext{}{\lemma{nipuṇamanyena}\xxref{rnā__144871}{rnā__144915}\Afootnote{\label{rnā__390487}nipu\unclear{ṇaṃ}manyena \cite{}; nipuṇammanyena \cite{}  {\rmlatinfont [App type: var]}}}nipuṇamanyena\edlabel{rnā__144915}\label{rnā__144915} vā\-caspattinā\- kathitaṃ tat kiṃ kā\-ryajā\-tī\-yaṃ prā\-sā\-dā\-di buddhimaddhetukaṃ na dṛṣṭaṃ yenotpattimattanubhuvanā\-di tathā\- na syā\-t, na khalu tajjā\-tī\-yakatve kascidviśeṣa iti | tad asaṅgatam | tathā\- hi bhavatu prā\-sā\-daparvatā\-dī\-nā\-ṃ kā\-ryatvā\-dinā\- sajā\-tī\-\gap{}yatvam | tat tu na vyā\-vahā\-rikapratyakṣeṇa buddhimadvyā\-ptaṃ pratyetuṃ śakyam, vyā\-ptigrahaṇasamaye dṛṣṭā\-nte buddhimadabhā\-vaprayuktasya kā\-ryamā\-travyatirekasya darśayitum aśakyatvā\-t |
	\pend
      

	  \pstart tad ayaṃ saṃkṣepā\-rthaḥ | kā\-ryatvamā\-trasyā\-vyatirekā\-d avyā\-ptasyā\-gamakatvam | avā\-ntaraṃ tu ghaṭaprā\-sā\-dā\-disā\-dhā\-raṇaṃ kā\-ryatvamā\-tram asmā\-bhir api na svī\-kṛtam eva | yathā\- tu ghaṭatvapaṭatvā\-diprā\-tisvikā\-nekajā\-tipuraskā\-reṇa prasiddhā\-numā\-navyavasthā\- sā\- cā\-navadyam avasthā\-piteti |
	\pend
      \label{rnā__īsd__sādhya}\edlabel{rnā__īsd__sādhya}
	  
	% new div opening: depth here is 3
	

	  \pstart saṃprati sā\-dhyā\-tmā\- vicā\-ryate | nanu vā\-dinā\- sā\-dhane samupanyaste taddū\-ṣaṇopanyā\-sam apā\-sya sā\-dhyasvarū\-pavikalpanaṃ nā\-ma naiyā\-yikamate niranuyojyā\-nuyogaḥ, saugatamate tv adoṣodbhā\-vanaṃ nigrahasthā\-nam iti cet | tad etaj jā\-lmajalpitam | tathā\- hi sā\-dhyasvarū\-pe 'pariniṣṭhite tadanusā\-riṇī\- pakṣasapakṣavipakṣavyavasthā\- kutaḥ | tadasiddhau cā\-siddhatā\-dayo doṣā\-ḥ pakṣadharmatā\-dayaś ca guṇā\- na vyavasthitā\- ity uktam | nedā\-nī\-ṃ hetor doṣaguṇakatheti mū\-kena prativā\-dinā\- sthā\-tavyam | tasmā\-d dhetudoṣopanyā\-saiveyaṃ sā\-dhyaniruktir ity ayam eva vā\-dī\- svamate niranuyojyā\-nuyogadū\-ṣaṇena nigrahasthā\-nena nigṛhyata iti kim atra nirbandhena | 
	\pend
      

	  \pstart yad etat kā\-rya\leavevmode\textsuperscript{\rmlatinfont\tiny [31a/RNAms]}\label{RNAms_31a}tvaṃ sā\-dhanaṃ kim anena viśvasya buddhimanmā\-trapū\-rvakatvaṃ sā\-dhyate | ā\-hosvid ekatvavibhutvasarvajñatvanityatvā\-diguṇaviśiṣṭabuddhimatpū\-rvakatvam | prathamapakṣe siddhasā\-dhanam | dvitī\-ye tu vyā\-pter abhā\-vā\-d anaikā\-ntikatā\- |
	\pend
      

	  \pstart nanu sā\-mā\-nyena vyā\-ptau pratī\-tā\-yā\-m api pakṣadharmatā\-balā\-d viśeṣasiddhiḥ | yathā\-gneḥ parvatā\-yogavyavacchedā\-disiddhiḥ | anyathā\- sarvā\-numā\-nocchedaḥ | anumā\-nadveṣī\- hy evaṃ jalpati:
	\pend
      
	    
	    \stanza[\smallbreak]
anumā\-nabhaṅgapaṅke 'smin nimagnā\- vā\-didantinaḥ |&viśeṣe 'nugamā\-bhā\-vaḥ sā\-mā\-nye siddhasā\-dhyatā\- ||\&[\smallbreak]


	

	  \pstart atrocyate | \edlabel{sarit__ratnakīrtinibandhāvali__149196}\label{sarit__ratnakīrtinibandhāvali__149196}\edtext{}{\lemma{sidhyaty … parvatavartitvādiviśeṣo}\xxref{sarit__ratnakīrtinibandhāvali__149196}{sarit__ratnakīrtinibandhāvali__149477}\Afootnote{\label{sarit__ratnakīrtinibandhāvali__401919}---\textsc{Note} Sanskrit, translation, and relation to Īśvarā\-pā\-karaṇasaṅkṣepa specified in \cref{krasser02_zaGkar_Izvar_studie}.  {\rmlatinfont [App type: parallel]}}}sidhyaty eva pakṣadharmatā\-balato viśeṣaḥ | na tu sarvaḥ | yena hi vinā\- \edlabel{rnā__147220}\label{rnā__147220}\edtext{}{\lemma{pakṣasthaṃ sādhanaṃ}\xxref{rnā__147220}{rnā__147269}\Afootnote{\label{rnā__390951}pakṣasthaṃ sā\-dhanaṃ \cite{} \cite{}\textenglish{---\textsc{Note} Thakur's note 2 is wrong, it is clealy °kṣastha° in the ms.}  {\rmlatinfont [App type: comment]}}}pakṣasthaṃ sā\-dhanaṃ\edlabel{rnā__147269}\label{rnā__147269} nopapadyate sa viśeṣaḥ sidhyatu | yathā\- vahner eva parvatavartitvā\-diviśeṣo\edlabel{sarit__ratnakīrtinibandhāvali__149477}\label{sarit__ratnakīrtinibandhāvali__149477} na pañcavarṇaśikhā\-kalā\-pakamanī\-yaḥ | na ca girī\-ṇā\-ṃ tarū\-ṇā\-ṃ kā\-ryatvaṃ kartur ekatvavibhutvasarvajñatvā\-dikam antareṇa nopapadyate, taditareṣv api darśanā\-t | tasmā\-t
	\pend
      
	    
	    \stanza[\smallbreak]
\edlabel{ratnakīrtinibandhāvali__36r1NFGMVPDL94KA62XUYBRHVL8}\flagstanza{\tiny\textenglish{...BRHVL8}}pakṣā\-yogavyavaccheda\edlabel{rnā__147570}\label{rnā__147570}\edtext{}{\lemma{bhedamātre na}\xxref{rnā__147570}{rnā__147613}\Afootnote{\label{rnā__391297}bhedamā\-tre\deletion{bhede}na \cite{}; bhedamā\-tre na \cite{}  {\rmlatinfont [App type: orth]}}}bhedamā\-tre na\edlabel{rnā__147613}\label{rnā__147613} dū\-ṣaṇam |&iṣṭasiddhyanvayā\-bhā\-vā\-d atirikte tu dū\-ṣaṇam ||\footnote{\begin{english}(JNA 268,19)\end{english}}\&[\smallbreak]


	

	  \pstart yady evaṃ \edlabel{rnā__147773}\label{rnā__147773}\edtext{}{\lemma{svasvarūpopā°}\xxref{rnā__147773}{rnā__147815}\Afootnote{\label{rnā__391587}sva\add{svarū\-po3}pā\- \cite{}; svasvarū\-popā\- \cite{}---\textsc{Note} As pointed out in \cref{thakur75}. Perhaps the intended correction was \begin{sanskrit}svarū\-po\end{sanskrit}, however.  {\rmlatinfont [App type: orth]}}}svasvarū\-popā\-\edlabel{rnā__147815}\label{rnā__147815}dā\-nopakaraṇasaṃpradā\-naprayojanā\-bhijña eva kartā\- \edlabel{rnā__147893}\label{rnā__147893}\edtext{}{\lemma{sādhyate | svarūpam iha ca dvyaṇukaṃ kāryam | upādānam}\xxref{rnā__147893}{rnā__147977}\Afootnote{\label{rnā__391975} \cite{}sā\-dhyate |\add{svarū\-pamihacadvyaṇukaṃkā\-ryaṃ3}upā\-dā\-nam \cite{}  {\rmlatinfont [App type: orth]}}}sā\-dhyate | svarū\-pam iha ca dvyaṇukaṃ kā\-ryam | upā\-dā\-nam\edlabel{rnā__147977}\label{rnā__147977} iha paramā\-ṇujā\-ticatuṣṭayam | upakaraṇaṃ samastakṣetrajñasamavā\-yidharmā\-dharmau | sampradanaṃ kṣetrajñā\-ḥ, yā\-nayaṃ bhagavā\-n svakarma\edlabel{rnā__148159}\label{rnā__148159}\edtext{}{\lemma{bhir abhipraiti |}\xxref{rnā__148159}{rnā__148206}\Afootnote{\label{rnā__392347} \cite{}bhirapraiti | \cite{}\textenglish{---\textsc{Note} Possible that something is added in the top margin. Only the bottom of two akṣā\-ras are visible there, because the folio is overlapped by the one on top on the photo.}  {\rmlatinfont [App type: em]}}}bhir abhipraiti |\edlabel{rnā__148206}\label{rnā__148206} prayojanaṃ sukhaduḥkhopabhogaḥ kṣetrajñā\-nā\-m | \edlabel{ratnakīrtinibandhāvali__36r1NJ252MXEUNK6K4TYO5KZWY7}\label{ratnakīrtinibandhāvali__36r1NJ252MXEUNK6K4TYO5KZWY7}\edtext{}{\lemma{evaṃbhūte}\xxref{ratnakīrtinibandhāvali__36r1NJ252MXEUNK6K4TYO5KZWY7}{ratnakīrtinibandhāvali__36r1NJ252MZ3SBLPSHXDBDAC4WY}\Afootnote{\label{ratnakīrtinibandhāvali__36r1NJ253DNHUHRNHC0HVCCV4OH}evaṃ bhū\-te \cite{}  {\rmlatinfont [App type: punctuation]}}}evaṃbhū\-te\edlabel{ratnakīrtinibandhāvali__36r1NJ252MZ3SBLPSHXDBDAC4WY}\label{ratnakīrtinibandhāvali__36r1NJ252MZ3SBLPSHXDBDAC4WY} buddhimati sā\-dhye kutaḥ siddhasā\-dhanam | na cā\-vyā\-ptiḥ | \edlabel{rnā__148350}\label{rnā__148350}\edtext{}{\lemma{kulāladṛṣṭānte upādānā}\xxref{rnā__148350}{rnā__148404}\Afootnote{\label{rnā__392792}kulā\-ladṛṣṭā\-ntena upā\-dā\-nā\- \cite{} \cite{}  {\rmlatinfont [App type: emendation]}}}kulā\-ladṛṣṭā\-nte upā\-dā\-nā\-\edlabel{rnā__148404}\label{rnā__148404}dyabhijñatvasya sambhavā\-t |
	\pend
      

	  \pstart tathā\- ca vā\-caspatiḥ pramā\-ṇyati:\edlabel{ratnakīrtinibandhāvali__36r1NJ1JDU74SXAW1GSGGXN9PN1}\label{ratnakīrtinibandhāvali__36r1NJ1JDU74SXAW1GSGGXN9PN1}\edtext{}{\lemma{vivādādhyāsitās … tatheti}\xxref{ratnakīrtinibandhāvali__36r1NJ1JDU74SXAW1GSGGXN9PN1}{ratnakīrtinibandhāvali__36r1NJ1JLUB0U2KHDR5ES91PSZB}\Afootnote{\label{ratnakīrtinibandhāvali__36r1NJ1JLUFISBCCCGEOBBH3N1M}\textenglish{---\textsc{Note} Similar to: \href{file://../../../nyAyA/vAcaspatimizra/texts/nyAyavArttikatAtparyaTIka.xml\#nvtṭ__36r1NJ1JB8C6NHNHPFGA5WCCXW4}{file://../../../nyAyA/vAcaspatimizra/texts/nyAyavArttikatAtparyaTIka.xml\#nvtṭ\textunderscore \textunderscore 36r1NJ1JB8C6NHNHPFGA5WCCXW4}.}  {\rmlatinfont [App type: parallel]}}} vivā\-dā\-dhyā\-sitā\-s tanugirisā\-garā\-dayaḥ upā\-dā\-nā\-dyabhijñakartṛkā\-ḥ | kā\-ryatvā\-t | yad yat kā\-ryaṃ tat tad upā\-dā\-nā\-dyabhijñakartṛkam | yathā\- prā\-sā\-dā\-di | tathā\- ca vivā\-dā\-dhyā\-sitā\-s tanvā\-dayaḥ | tasmā\-t tatheti\edlabel{ratnakīrtinibandhāvali__36r1NJ1JLUB0U2KHDR5ES91PSZB}\label{ratnakīrtinibandhāvali__36r1NJ1JLUB0U2KHDR5ES91PSZB} |
	\pend
      

	  \pstart evam ataḥ sā\-dhanā\-d upā\-dā\-nā\-dyabhijñakartṛmā\-traṃ prasā\-dhya tasya sarvajñatvasā\-dhanā\-ya vā\-caspatir eva punar apī\-dam ā\-ha: \edlabel{ratnakīrtinibandhāvali__36r1NJ1GUNF6HVSRTM28VAP3DL9}\label{ratnakīrtinibandhāvali__36r1NJ1GUNF6HVSRTM28VAP3DL9}\edtext{}{\lemma{bhavatu … tādṛgīśvarād}\xxref{ratnakīrtinibandhāvali__36r1NJ1GUNF6HVSRTM28VAP3DL9}{ratnakīrtinibandhāvali__36r1NJ1HHMVH5RR7U2GTVWCF63V}\Afootnote{\label{ratnakīrtinibandhāvali__36r1NJ1HHN0UIZL59XVOFV2QXCK}\textenglish{---\textsc{Note} Cf. \href{file://../../../nyAyA/vAcaspatimizra/texts/nyAyavArttikatAtparyaTIka.xml\#nvtṭ__36r1NJ1GUNHV7JE0L7ACW6RSRE0}{file://../../../nyAyA/vAcaspatimizra/texts/nyAyavArttikatAtparyaTIka.xml\#nvtṭ\textunderscore \textunderscore 36r1NJ1GUNHV7JE0L7ACW6RSRE0}.}  {\rmlatinfont [App type: parallel]}}}bhavatu tā\-vad upā\-dā\-nā\-dyabhijñakartṛmā\-trasiddhiḥ | pā\-riśeṣyā\-t tu vyatirekidvitī\-yanā\-mno 'numā\-nā\-d viśeṣasiddhiḥ | tathā\- hi: tanubhuvanā\-dyupā\-dā\-nā\-dyabhijñaḥ kartā\- nā\-nityā\-sarvaviṣayabuddhimā\-n | tatkartus tadupā\-dā\-nā\-dyanabhijñatvaprasaṅgā\-t | na hy evaṃvidhas tadupā\-dā\-nā\-dyabhijño \edlabel{ratnakīrtinibandhāvali__36r1NJAZS8ND3NTZPZDACR3GERA}\label{ratnakīrtinibandhāvali__36r1NJAZS8ND3NTZPZDACR3GERA}\edtext{}{\lemma{yathāsmadādiḥ | tadupādānādyabhijñaś}\xxref{ratnakīrtinibandhāvali__36r1NJAZS8ND3NTZPZDACR3GERA}{ratnakīrtinibandhāvali__36r1NJAZS90JKBU2BVD5ZRNFMIB}\Afootnote{\label{ratnakīrtinibandhāvali__36r1NJAZSYJ31M8C7Y763COK6UY}yathā\-smadā\-diḥ | tadupā\-dā\-nā\-dyabhijñaś \cite{}; yathā\-smadā\-diḥ | \deletion{tadupā\-dā\-nā\-dyanabhijñatvaprasaṅgā\-t} tadupā\-dā\-nā\-dyabhijñaś \cite{}  {\rmlatinfont [App type: correction]}}}yathā\-smadā\-diḥ | tadupā\-dā\-nā\-dyabhijñaś\edlabel{ratnakīrtinibandhāvali__36r1NJAZS90JKBU2BVD5ZRNFMIB}\label{ratnakīrtinibandhāvali__36r1NJAZS90JKBU2BVD5ZRNFMIB} cā\-yam | ta\leavevmode\textsuperscript{\rmlatinfont\tiny [31b/RNAms]}\label{RNAms_31b}smā\-t tatheti |
	\pend
      

	  \pstart no khalu paramā\-ṇubhedā\-n kṣetrajñasamavā\-yinaś ca karmā\-śayabhedā\-n aparimeyā\-n anyaḥ śakto jñā\-tum ṛte tā\-dṛgī\-śvarā\-d\edlabel{ratnakīrtinibandhāvali__36r1NJ1HHMVH5RR7U2GTVWCF63V}\label{ratnakīrtinibandhāvali__36r1NJ1HHMVH5RR7U2GTVWCF63V} iti |
	\pend
      

	  \pstart atrocyate | yā\-vanti dvyaṇukā\-ni bhinnadeśakā\-lasvabhā\-vā\-ni kā\-ryā\-ṇi santi teṣu sarveṣv eva kim eka eva buddhimā\-n vyā\-priyate | aneko vā\- | yad vā\- svasvaviṣayamā\-tropā\-dā\-nā\-divedinaḥ parasparavyā\-pā\-rā\-nabhijñā\- bhinnadeśakā\-lasvabhā\-vā\-ḥ pratidvyaṇukam anya eva buddhimanto vyā\-priyante iti trayaḥ pakṣā\-ḥ |
	\pend
      

	  \pstart na tā\-vat \edlabel{ratnakīrtinibandhāvali__36r1NM7W6RLH3UQ01ZEG4Q53FWC}\label{ratnakīrtinibandhāvali__36r1NM7W6RLH3UQ01ZEG4Q53FWC}\edtext{}{\lemma{prathamaḥ pakṣaḥ |}\xxref{ratnakīrtinibandhāvali__36r1NM7W6RLH3UQ01ZEG4Q53FWC}{ratnakīrtinibandhāvali__36r1NM7W6SG03AESJNSHQ1FK1L5}\Afootnote{\label{ratnakīrtinibandhāvali__36r1NM7W7SUFEKSHR9YQ6XSG9WS}prathamaḥ pakṣaḥ | \cite{}; prathamapakṣaḥ | \cite{}  {\rmlatinfont [App type: var]}}}prathamaḥ pakṣaḥ |\edlabel{ratnakīrtinibandhāvali__36r1NM7W6SG03AESJNSHQ1FK1L5}\label{ratnakīrtinibandhāvali__36r1NM7W6SG03AESJNSHQ1FK1L5} de\gap{}śakā\-lasvabhā\-vabhinnā\-nā\-ṃ sarveṣā\-ṃ dvyaṇukā\-nā\-ṃ kartur ekatvā\-siddheḥ | yac caikatvasā\-dhanā\-ya ‘kā\-ryaliṅgā\-viśeṣā\-d ityā\-dy’ api sā\-dhanam upanyastaṃ tad asaṅgatam | dhū\-maliṅgā\-viśeṣe 'pi hy agner anekatvavat tatrā\-pi tacchaṅkā\-sambhavā\-t | ‘\edlabel{ratnakīrtinibandhāvali__36r1NSAWNTV0PU093F550ENXSD2}\label{ratnakīrtinibandhāvali__36r1NSAWNTV0PU093F550ENXSD2}\edtext{}{\lemma{sad iti liṅgāviśeṣād}\xxref{ratnakīrtinibandhāvali__36r1NSAWNTV0PU093F550ENXSD2}{ratnakīrtinibandhāvali__36r1NSAWNTWQ48AKDPN5RIVJIC3}\Afootnote{\label{ratnakīrtinibandhāvali__36r1NSAWNU07VM3DNQQNL6L15SZ}\textenglish{---\textsc{Note} \begin{sanskrit}sad iti jñā\-nā\-viśeṣā\-t\end{sanskrit} \cref{ratnakīrtinibandhāvali__36r1NSAS7AJHGG1XWSGMMV9343A}.}  {\rmlatinfont [App type: parallel]}}}sad iti liṅgā\-viśeṣā\-d\edlabel{ratnakīrtinibandhāvali__36r1NSAWNTWQ48AKDPN5RIVJIC3}\label{ratnakīrtinibandhāvali__36r1NSAWNTWQ48AKDPN5RIVJIC3} iti’ tu dṛṣṭā\-nto 'smā\-n pratyasiddha eva | tasmā\-d yathā\- mayā\- nā\-nā\-tvasā\-dhanā\-ya pramā\-ṇaṃ vaktavyaṃ tathā\- tvayā\-py ekatvasā\-dhanā\-ya sā\-dhanam abhidhā\-nī\-yam |
	\pend
      

	  \pstart atha manyate anekatvasā\-dhanā\-bhā\-vā\-d ekatvasiddhir iti | yady evam ekatvasā\-dhanā\-bhā\-vad anekatvam eva kiṃ nā\-vagacchasi |
	\pend
      

	  \pstart yad apy uktam: \edlabel{ratnakīrtinibandhāvali__36r1NSAZOF4PYRJJ9T4KL5GNBA4}\label{ratnakīrtinibandhāvali__36r1NSAZOF4PYRJJ9T4KL5GNBA4}\edtext{}{\lemma{ekatve … ityādi}\xxref{ratnakīrtinibandhāvali__36r1NSAZOF4PYRJJ9T4KL5GNBA4}{ratnakīrtinibandhāvali__36r1NSAZOF6VPMT1M1P47SSA79H}\Afootnote{\label{ratnakīrtinibandhāvali__36r1NSAZP9YLE8DI58KMDSHDTF3}\textenglish{---\textsc{Note} Quote of \cref{ratnakīrtinibandhāvali__36r1NSAZCVOOMC2YM2ZB4UM8USC}.}  {\rmlatinfont [App type: parallel]}}}ekatve tu na pramā\-ṇā\-ntaram anveṣṭavyam ekasya kartur abhā\-ve bahū\-nā\-ṃ vyā\-hatamanasā\-m ityā\-di\edlabel{ratnakīrtinibandhāvali__36r1NSAZOF6VPMT1M1P47SSA79H}\label{ratnakīrtinibandhāvali__36r1NSAZOF6VPMT1M1P47SSA79H} | \edlabel{ratnakīrtinibandhāvali__36r1NMN5X0PC291R8Q4FO5BKCDN}\label{ratnakīrtinibandhāvali__36r1NMN5X0PC291R8Q4FO5BKCDN}tad api cintyatā\-m | bahubhiḥ karaṇe yugapat kā\-ryā\-nutpattir iti kiṃ bhinnadeśakā\-lā\-nā\-ṃ kā\-ryā\-ṇā\-m anutpattir vivakṣitā\- | ekasyaiva vā\- mahā\-vayavinaḥ kṣitighaṭā\-dirū\-pasya | tatra ekasminn api kā\-rye bahubhiḥ karaṇe utpattivirodhinaṃ na paśyā\-maḥ | bahū\-nā\-ṃ parasparaṃ vaimatyaniyamā\-bhā\-vā\-t | parasparā\-vyā\-ghā\-tapuruṣatvayor dvividhasyā\-pi virodhasyā\-sambhavā\-t | puruṣatvaṃ hi apuruṣatvena viruddham | na tu parasparā\-vyā\-ghā\-tena |
	\pend
      

	  \pstart ye tv anantadeśakā\-lasvabhā\-vabhedabhinnā\-steṣu sutarā\-m evā\-nekavyā\-pā\-raniṣedho 'sambhavī\-ti dvitī\-yo 'pi pakṣo vyudastaḥ | na ca kartur ekatvena dṛṣṭā\- vyā\-ptisiddhiḥ |  anekenā\-pi svatantreṇa svasvaprayojanā\-rthinā\- grā\-mapraviṣṭahariṇā\-dimā\-raṇaikakā\-ryadarśanā\-t | tasyā\-pi pakṣī\-karaṇe ekakartṛpū\-rvakā\-bhimatasyā\-pi pakṣī\-karaṇe ā\-tmakartṛpū\-rvakatvam astu | tad evaṃ na sarvadvyaṇukā\-nā\-ṃ kartur ekatvasiddhiḥ | \edlabel{ratnakīrtinibandhāvali__36r1NMMFX7IAGR5V3DIY9Q16BLF}\label{ratnakīrtinibandhāvali__36r1NMMFX7IAGR5V3DIY9Q16BLF}\edtext{}{\lemma{tathā coktam ... āśrayam |}\xxref{ratnakīrtinibandhāvali__36r1NMMFX7IAGR5V3DIY9Q16BLF}{ratnakīrtinibandhāvali__36r1NMFLNZF26QG3OB18MFYZVHQ}\Afootnote{\label{ratnakīrtinibandhāvali__36r1NMFLNZHO0BSBWPYS4CX4X7Q}\textenglish{---\textsc{Note} Cf. : \begin{sanskrit}एककर्तुरसिद्धौ च सर्वज्ञत्वं किमा\-श्रयम् ।\end{sanskrit}. \cref{thakur75} notes that this is a ``marginal edition, separate hand".}  {\rmlatinfont [App type: parallel]}}}tathā\- coktam
	\pend
      
	    
	    \stanza[\smallbreak]
\edlabel{ratnakīrtinibandhāvali__36r1NMFLFX785HFFPE2O9XY3QCP}\flagstanza{\tiny\textenglish{...XY3QCP}}ekakartur na siddhau tu sarvajñatvaṃ kim ā\-śrayam |\edlabel{ratnakīrtinibandhāvali__36r1NMFLNZF26QG3OB18MFYZVHQ}\label{ratnakīrtinibandhāvali__36r1NMFLNZF26QG3OB18MFYZVHQ}\&[\smallbreak]


	

	  \pstart ata eva dvitī\-yo 'pi pakṣaḥ kṣī\-ṇaḥ | saveṣu dvyaṇukeṣv ekasyā\-pi kartur apravṛttau bahū\-nā\-ṃ sutarā\-m apravṛtteḥ |
	\pend
      

	  \pstart tṛ\leavevmode\textsuperscript{\rmlatinfont\tiny [pb in]}\label{RNAms_32a}tī\-yas tu pakṣo yadi bhavet tadā\- svasvavyā\-pā\-raviṣayamā\-tropā\-dā\-nā\-dyabhijñatve 'pi naikaḥ kaścit sarvajñaḥ sidhyati | na ca \edlabel{rnā__151710}\label{rnā__151710}\edtext{}{\lemma{jñānasattāmātreṇa}\xxref{rnā__151710}{rnā__151759}\Afootnote{\label{rnā__393154}jñā\-nasattā\-mā\-treṇa \cite{}; jñā\-naṃ sattā\-mā\-treṇa \cite{}  {\rmlatinfont [App type: var]}}}jñā\-nasattā\-mā\-treṇa\edlabel{rnā__151759}\label{rnā__151759} katipayā\-tī\-ndriyadarśanavat sarvā\-rthagrahaṇaṃ yena tadabhedā\-t prastutaparamā\-ṇuvat sarvasyaivā\-viśeṣeṇa grahaṇā\-t sarvajñatā\- syā\-t | anumā\-nato hi katipayā\-tī\-ndriyadarśane siddhe 'pī\-śvarasya tatkā\-raṇayogitvaṃ niścī\-yate | na tu jñā\-nasā\-ttā\-mā\-treṇa prakā\-rā\-ntareṇeti niścaya iti kutaḥ sarvajñatā\- |
	\pend
      

	  \pstart nanv atī\-ndriyaṃ paramā\-ṇvā\-dikaṃ jā\-nato na kathaṃ sā\-rvajñyam iti cet | tat kim idā\-nī\-m asarvadarśitveṣv atī\-ndriyadarśanamā\-treṇa sarvajñatā\-pratyayā\-śā\- | evam eveti cet | hanta yadi nā\-ma nyā\-yavihastena tvayā\- ī\-dṛśo hastasamā\-racitaḥ sarvajñaḥ paribhā\-vitas tathā\-py anyeṣā\-m apā\-radū\-radeśakā\-lavartinā\-ṃ dvyaṇukā\-dī\-nā\-m upā\-dā\-nā\-diṣu januṣā\-ndhaprakhyasya paramapuruṣā\-rthā\-vedino vā\- lokaiḥ prā\-mā\-ṇikaiś ca nā\-sya sā\-rvajñyam anumanyate ||
	\pend
      

	  \pstart asmā\-kan tu nā\-tī\-ndriyadarśimā\-tre pradveṣaḥ | evaṃ ca kartur ekatvā\-siddhau vyatireky api hetur asamarthaḥ viśveṣā\-m ekasya kartur asiddhau tadupā\-dā\-nā\-dyabhijñabhā\-vasyā\-siddhatvā\-t | yaś ca yanmā\-trakā\-raḥ sa tanmā\-tropā\-dā\-nā\-dyabhijño bhavan na sarvajñaḥ | anekā\-śrayeṇā\-pi upā\-dā\-nā\-dyabhijñasā\-mā\-nyasya caritā\-rthatvā\-t | tad evam upā\-dā\-nā\-dyabhijñapuruṣamā\-trasiddhā\-v api naikatvasarvajñatvā\-diviśiṣṭapuruṣaviśeṣasiddhiḥ | puruṣamā\-tre ca siddhasā\-dhanam uktam | buddhimanmā\-trapū\-rvakatā\-m icchatā\-m upā\-dā\-nā\-dyabhijñabuddhimatpū\-rvakatve sā\-dhye kathaṃ \edlabel{ratnakīrtinibandhāvali__36r1NMNKPK8I8K079OIC2E8BVU5}\label{ratnakīrtinibandhāvali__36r1NMNKPK8I8K079OIC2E8BVU5}\edtext{}{\lemma{siddhasādhanam}\xxref{ratnakīrtinibandhāvali__36r1NMNKPK8I8K079OIC2E8BVU5}{ratnakīrtinibandhāvali__36r1NMNKPKAMZXOQ29O4V14D9F6}\Afootnote{\label{ratnakīrtinibandhāvali__36r1NMNKQ1PS32JKPAF5U8FE58P}siddhasā\-dhanam \cite{}; siddhisā\-dhanam \cite{}  {\rmlatinfont [App type: var]}}}siddhasā\-dhanam\edlabel{ratnakīrtinibandhāvali__36r1NMNKPKAMZXOQ29O4V14D9F6}\label{ratnakīrtinibandhāvali__36r1NMNKPKAMZXOQ29O4V14D9F6} iti cet | na tadapekṣayā\- siddhasā\-dhyatā\-yā\- janitatvā\-t kevalam asiddhoddhā\-re 'bhimate viśeṣe siddhe 'pi naiyā\-yikasyā\-pi nā\-bhimatasiddhir iti brū\-maḥ ||
	\pend
      

	  \pstart saugatasya tā\-vad aniṣṭasiddhir iti cet, na, svā\-bhimatasā\-dhyasā\-dhanenaiva hi parasyā\-niṣṭam api sā\-dhanī\-yam | anyathā\- mā\-tṛśokasmaraṇā\-dinā\-pi tadaniṣṭasiddhiḥ syā\-d iti | asya saṅgrahaḥ
	\pend
      
	    
	    \stanza[\smallbreak]
pareṣṭasiddhir napareṣṭabā\-dhakaṃ prasā\-dhane vedanayatnamā\-trayoḥ |&ananvayo 'bhī\-ṣṭaviśeṣasā\-dhane vipakṣasandehasahantu sā\-dhanam ||\&[\smallbreak]


	\label{īsd-sādhyacintā}\edlabel{īsd-sādhyacintā}
	  
	% new div opening: depth here is 3
	

	  \pstart sā\-dhyacintā\-dhikā\-ras tṛtī\-yaḥ ||
	\pend
      

	  \pstart evam anye 'pi hetavo yathā\-yogam abhyū\-hya dū\-ṣaṇī\-yā\-ḥ | tad evaṃ tā\-vad ī\-śvarasya sadvyavahā\-ro niṣiddhaḥ | asadvyavahā\-rā\-rthan tu tallakṣaṇavilakṣaṇakṣaṇabhaṅgasā\-dhakaṃ sattā\-disā\-dhanam eva draṣṭavyam iti ||\leavevmode\textsuperscript{\rmlatinfont\tiny [pb in]}\label{RNAms_32B1}
	\pend
      
	    
	    \stanza[\smallbreak]
\edlabel{thakur75-57.14}\flagstanza{\tiny\textenglish{...-57.14}}ity abodhajanakartṛvikalpa vyā\-pi mohatimirapratirodhi |&ratnakī\-rtir acanā\-malaramya jyotir astu ciramapratirodhi ||\&[\smallbreak]


	
	  
	% new div opening: depth here is 1
	
\section[{Apohasiddhiḥ}]{Apohasiddhiḥ}\edlabel{Apohasiddhiḥ}\label{Apohasiddhiḥ}

	  \pstart || namas tā\-rā\-yai || \edlabel{thakur75-58.5}\label{thakur75-58.5} apohaḥ śabdā\-rtho nirucyate | nanu ko 'yam apoho nā\-ma | kim idam anyasmā\-d apohyate | asmā\-d vā\-nyad apohyate | asmin vā\-nyad apohyata iti vyutpattyā\- vijā\-tivyā\-vṛttaṃ bā\-hyam eva vivakṣitam | buddhyā\-kā\-ro vā\- | yadi vā\- apohanam apoha ity anyavyā\-vṛttimā\-tram iti trayaḥ pakṣā\-ḥ | \edlabel{thakur75-58.9}\label{thakur75-58.9} na tā\-vad ā\-dimau pakṣau apohanā\-mnā\- vidher eva vivakṣitatvā\-t | antimo 'py asaṅgataḥ, pratī\-tibā\-dhitatvā\-t | tathā\- hi parvatoddeśe vahnir astī\-ti śā\-bdī\- pratī\-tir vidhirū\-pam evollikhantī\- lakṣyate | nā\-nagnir na bhavatī\-ti nitrṛttimā\-tram ā\-mukhayantī\- | yac ca pratyakṣabā\-dhitaṃ na tatra sā\-dhanā\-ntarā\-vakā\-śa ity atiprasiddham ||
	\pend
      

	  \pstart atha yady api nivṛttim ahaṃ pratyemī\-ti na vikalpaḥ tathā\-pi nivṛttapadā\-rthollekha eva nivṛttyullekhaḥ | na hy anantrbhā\-vitaviśeṣaṇapratī\-tir viśiṣṭapratī\-tiḥ | tato yathā\- sā\-mā\-nyam ahaṃ pratyemī\-ti vikalpā\-bhā\-ve 'pi sā\-dhā\-raṇā\-kā\-raparisphuraṇā\-d vikalpabuddhiḥ sā\-mā\-nyabuddhiḥ pareṣā\-m, tathā\- nivṛttapratyayā\-kṣiptā\- nivṛttibuddhir apohapratī\-tivyavahā\-ramā\-tanotī\-ti cet |
	\pend
      

	  \pstart nanu sā\-dhā\-raṇā\-kā\-raparisphuraṇe vidhirū\-patayā\- yadi sā\-mā\-nyabodhavyavasthā\-, tat kim ā\-yā\-tam asphuradabhā\-vā\-kā\-re cetasi nivṛttipratī\-tivyavasthā\-yā\-ḥ | tato nivṛttim ahaṃ pratyemī\-ty evam ā\-kā\-rā\-bhā\-ve 'pi nivṛttyā\-kā\-rasphuraṇaṃ yadi syā\-t ko nā\-ma nivṛttipratī\-tisthitim apalapet | anyathā\- asati pratibhā\-se tatpratī\-tivyavahṛtir iti gavā\-kā\-re 'pi cetasi turagabodha ity astu ||
	\pend
      

	  \pstart atha viśeṣaṇtayā\- antarbhū\-tā\- nivṛttipratī\-tir ity uktam | tathā\-pi yady agavā\-poḍha itī\-dṛśā\-kā\-ro vikalpas tadā\- viśeṣaṇatayā\- tadanupraveśo bhavatu kiṃ tu gaur iti pratī\-tiḥ | tadā\- ca sato 'pi nivṛttilakṣaṇasya viśeṣaṇasya tatrā\-nutkalanā\-t kathaṃ tatpratī\-tivyavasthā\- |
	\pend
      

	  \pstart athaivaṃ matiḥ: yad vidhirū\-paṃ sphurati tasya parā\-poho 'py astī\-ti tatpratī\-tir ucyate | tadā\-pi sambandhamā\-tram apohasya | vidhir eva sā\-kṣā\-n nirbhā\-sī\- | api caivam adhyakṣasyā\-py apohaviṣayatvam anivā\-ryam viśeṣato vikalpā\-d ekavyā\-vṛttollekhino 'khilā\-nyavyā\-vṛttam ī\-kṣamā\-ṇasya | tasmā\-d vidhyā\-kā\-rā\-vagrahā\-d adhyakṣavad vikalpasyā\-pi vidhiviṣayatvam eva nā\-nyā\-pohaviṣayatvam iti katham apohaḥ śabdā\-rtho ghuṣyate | 
	\pend
      

	  \pstart atrā\-bhidhī\-yate | nā\-smā\-bhir apohaśabdena vidhir eva kevalo 'bhipretaḥ | nā\-py anyavyā\-vṛttimā\-tram | kin tv anyā\-pohaviśiṣṭo vidhiḥ śabdā\-nā\-m arthaḥ | tataś ca na pratyekapakṣopanipā\-tidoṣā\-vakā\-śaḥ || \edlabel{thakur75-59.7}\label{thakur75-59.7} yat tu goḥ pratī\-tau na tadā\-tmā\-parā\-tmeti sā\-marthyā\-d apohaḥ paścā\-n niścī\-yata iti vidhivā\-dinā\-ṃ matam, anyā\-pohapratī\-tau vā\- sā\-marthyā\-d anyā\-poḍho 'vadhā\-ryate iti pratiṣedhavā\-dinā\-ṃ matam | tad asundaram | prā\-thamikasyā\-pi pratipattikramā\-darśanā\-t | na hi vidhiṃ pratipadya kaścid arthā\-pattitaḥ paścā\-d apoham avagacchati | apohaṃ vā\- pratipadyā\-nyā\-poḍham | tasmā\-d goḥ pratipattir ity anyā\-poḍhapratipattir ucyate | yady api cā\-nyā\-poḍhaśabdā\-nullekha uktas tathā\-pi nā\-pratipattir eva viśeṣaṇabhū\-tasyā\-pohasya | agavā\-poḍha eva gośabdasya niveśitatvā\-t | yathā\- nī\-lotpale niveśitā\-d indī\-varaśabdā\-n nī\-lotpalapratī\-tau tatkā\-la eva nī\-limasphuraṇam anivā\-ryaṃ tathā\- gośabdā\-d apy agavā\-poḍhe niveśitā\-d gopratī\-tau tulyakā\-lam eva viśeṣaṇtvā\-d ago 'pohasphuraṇam anivā\-ryam | yathā\- pratyakṣasya prasajyarū\-pā\-bhā\-vā\-grahaṇam abhā\-vavikalpotpā\-danaśaktir eva tathā\- vidhivikalpā\-nā\-m api tadanurū\-pā\-nuṣṭhā\-nadā\-naśaktir evā\-bhā\-vagrahaṇam abhidhī\-yate | paryudā\-sarū\-pā\-bhā\-vagrahaṇaṃ tu niyatasvarū\-pasaṃvedanam ubhayor aviśiṣṭam | anyathā\- yadi śabdā\-d arthapratipattikā\-le kalito na parā\-pohaḥ katham anyaparihā\-reṇa pravṛttiḥ | tato gā\-ṃ badhā\-neti codito 'śvā\-dī\-n api badhnī\-yā\-t || \edlabel{thakur75-59.21}\label{thakur75-59.21} yad apy avocad Vā\-caspatiḥ jā\-timatyo vyaktayo vikalpā\-nā\-ṃ śabdā\-nā\-ṃ ca gocaraḥ | tā\-sā\-ṃ ca tadvatī\-nā\-ṃ rū\-pam atajjā\-tī\-yaparā\-vṛttim ity atas tadavagater na gā\-ṃ badhā\-neti codito 'śvā\-dī\-n badhnā\-ti | tad apy anenaiva nirastam | yato jā\-ter adhikā\-yā\-ḥ prakṣepe 'pi vyaktī\-nā\-ṃ rū\-pam atajjā\-tī\-yaparā\-vṛttam eva cet, tadā\- tenaiva rū\-peṇa śabdavikalpayor viṣayī\-bhavantī\-nā\-ṃ katham atadvyā\-vṛttiparihā\-raḥ || \edlabel{thakur75-59.26}\label{thakur75-59.26} atha na vijā\-tī\-yavyā\-vṛttaṃ vyaktirū\-paṃ tathā\-pratī\-taṃ vā\- tadā\- jā\-tiprasā\-da eṣa iti katham arthato 'pi tadavagatir ity uktaprā\-yam | \edlabel{thakur75-59.28}\label{thakur75-59.28} atha jā\-tibalā\-d evā\-nyato 'vyā\-vṛttam | bhavatu jā\-tibalā\-t svahetuparamparā\-balā\-d vā\-nyavyā\-vṛttam | ubhayathā\-pi vyā\-vṛttapratipattau vyā\-vṛttipratipattir asty eva | \edlabel{thakur75-60.1}\label{thakur75-60.1} na cā\-gavā\-poḍhe gośabdasaṅketavidhā\-v anyonyā\-śrayadoṣaḥ | sā\-mā\-nye tadvati vā\- saṃkete 'pi taddoṣā\-v akā\-śā\-t | na hi sā\-mā\-nyaṃ nā\-ma sā\-mā\-nyamā\-tram abhipretam, turage 'pi gośabdasaṃketaprasaṅgā\-t | kiṃ tu gotvam | tā\-vatā\- ca sa eva doṣaḥ | gavā\-diparijñā\-ne gotvasā\-mā\-nyā\-parijñā\-nā\-t | gotvasā\-mā\-nyā\-parijñā\-ne gośabdavā\-cyā\-parijñā\-nā\-t | \edlabel{thakur75-60.6}\label{thakur75-60.6} tasmā\-d ekapiṇḍadarśanapū\-rvako yaḥ sarvavyaktisā\-dhā\-raṇa iva bahiradhyasto vikalpabuddhyā\-kā\-raḥ tatrā\-yaṃ gaur iti saṃketakaraṇe netaretarā\-śrayadoṣaḥ | \edlabel{thakur75-60.8}\label{thakur75-60.8} abhimate ca gośabdapravṛttā\-v agośabdena śeṣasyā\-py abhidhā\-nam ucitam | na cā\-nyā\-poḍhā\-nyā\-pohayor virodho viśeṣyaviśeṣaṇabhā\-vakṣatir vā\-, parasparavyavacchedā\-bhā\-vā\-t | sā\-mā\-nā\-dhikaraṇyasadbhā\-vā\-t | bhū\-talaghaṭā\-bhā\-vavat | svā\-bhā\-vena hi virodho na parā\-bhā\-venety ā\-bā\-laprasiddham | eṣa panthā\-ḥ śrudhnam upatiṣṭhata ity atrā\-py apoho gamyata eva | aprakṛtapathā\-ntarā\-pekṣayā\- eṣa eva śrudhnapratyanī\-kā\-niṣṭasthā\-nā\-pekṣayā\- śrudhnam eva | araṇyamā\-rgavad vicchedā\-bhā\-vā\-d upatiṣṭhata eva | sā\-rthadū\-tā\-divyavacchedena panthā\- eveti pratipadaṃ vyavacchedasya sulabhatvā\-t | tasmā\-d apohadharmaṇo vidhirū\-pasya śabdā\-d avagatiḥ puṇḍarī\-kaśabdā\-d iva śvetim aviśiṣṭasya padmasya || \edlabel{thakur75-60.16}\label{thakur75-60.16} yady evaṃ vidhir eva śabdā\-rtho vaktum ucitaḥ, katham apoho gī\-yata iti cet | uktamatrā\-pohaśabdenā\-nyā\-pohaviśiṣṭo vidhir ucyate | tatra vidhau pratī\-yamā\-ne viśeṣaṇatayā\- tulyakā\-lam anyā\-pohapratī\-tir iti | \edlabel{thakur75-60.19}\label{thakur75-60.19} na caivaṃ pratyakṣasyā\-py apohaviṣayatvavyavasthā\- kartum ucitā\- | tasya śā\-bdapratyayasyeva vastuviṣayatve vivā\-dā\-bhā\-vā\-t | vidhiśabdena ca yathā\-dhyavasā\-yam atadrū\-paparā\-vṛtto bā\-hyo 'rtho 'bhimataḥ, yathā\-pratibhā\-saṃ buddhyā\-kā\-raś ca | tatra bā\-hyo 'rtho 'dhyavasā\-yā\-d eva śabdavā\-cyo vyavasthā\-pyate | na svalakṣaṇaparisphū\-rtyā\- | pratyakṣavad deśakā\-lā\-vasthā\-niyatapravyaktasvalakṣaṇā\-sphuraṇā\-t | yac chā\-stram
	\pend
      

	  \pstart śabdenā\-vyā\-pṛtā\-kṣasya buddhā\-v apratibhā\-sanā\-t | arthasya dṛṣṭā\-v iva \footnote{\begin{english}(PVin I 15)\end{english}}
	\pend
      

	  \pstart iti | indriyaśabdasvabhā\-vopā\-yabhedā\-d ekasyaivā\-rthasya pratibhā\-sabheda iti cet | atrā\-py uktam:
	\pend
      

	  \pstart jā\-to nā\-mā\-śrayo 'nyā\-nyaḥ cetasā\-ṃ tasya vastutaḥ | ekasyaiva kuto rū\-paṃ bhinnā\-kā\-rā\-vabhā\-si tat || \footnote{\begin{english}(PV III 235)\end{english}} \edlabel{thakur75-60.29}\label{thakur75-60.29} na hi spaṣṭā\-spaṣṭe dve rū\-pe parasparaviruddhe ekasya vastunaḥ staḥ | yata ekenendriyabuddhau pratibhā\-setā\-nyena vikalpe | tathā\- sati vastuna eva bhedaprā\-pteḥ | na hi svarū\-pabhedā\-d aparo vastubhedaḥ | na ca pratibhā\-sabhedā\-d aparaḥ svarū\-pabhedaḥ | anyathā\- trailokyam ekam eva vastu syā\-t || \edlabel{thakur75-61.3}\label{thakur75-61.3} dū\-rā\-sannadeśavartinoḥ puruṣayor ekatra śā\-khini spaṣṭā\-spaṣṭapratibhā\-sabhede 'pi na śā\-khibheda iti cet | na brū\-maḥ pratibhā\-sabhedo bhinnavastuniyataḥ, kiṃ tv ekaviṣayatvā\-bhā\-vaniyata iti | tato yatrā\-rthakriyā\-bhedā\-disacivaḥ pratibhā\-sabhedas tatra vastubhedaḥ, ghaṭavat | anyatra punarniyamenaikaviṣayatā\-ṃ pariharatī\-ty ekapratibhā\-so bhrā\-ntaḥ || \edlabel{thakur75-61.7}\label{thakur75-61.7} etena yad ā\-ha Vā\-caspatiḥ: na ca śabdapratyakṣayor vastugocaratve pratyayā\-bhedaḥ kā\-raṇabhedena pā\-rokṣyā\-pā\-rokṣyabhedopapatter iti, tannopayogi | parokṣapratyayasya vastugocaratvā\-samarthatā\-t | parokṣatā\-śrayas tu kā\-raṇabheda indriyagocaragrahaṇaviraheṇaiva kṛtā\-rthaḥ | tan na | śā\-bde pratyaye svalakṣaṇaṃ parisphurati | kiṃ ca svalakṣaṇā\-tmani vastuni vā\-cye sarvā\-tmanā\- pratipatteḥ vidhiniṣedhayor ayogaḥ | tasya hi sadbhā\-ve 'stī\-ti vyartham, nā\-stī\-ty asamartham | asadbhā\-ve tu nā\-stī\-ti vyartham, astī\-ty asamartham | asti cā\-styā\-dipadaprayogaḥ | tasmā\-t śā\-bdapratibhā\-sasya bā\-hyā\-rthabhā\-vā\-bhā\-vasā\-dhā\-raṇyaṃ na tadviṣayatā\-ṃ kṣamate || \edlabel{thakur75-61.15}\label{thakur75-61.15} yac ca Vā\-caspatinā\- jā\-timadvyaktivā\-cyatā\-ṃ svavā\-caiva prastutyā\-ntaram eva na ca śabdā\-rthasya jā\-ter bhā\-vā\-bhā\-vasā\-dhā\-raṇyaṃ nopapadyate | sā\- hi svarū\-pato nityā\-pi deśakā\-laviprakī\-rṇā\-nekavyaktyā\-śrayatayā\- bhā\-vā\-bhā\-vasā\-dhā\-raṇī\-bhavanty astinā\-stisambandhayogyā\- | vartamā\-navyaktisambandhitā\- hi jā\-ter astitā\- | atī\-tā\-nā\-gatavyaktisambandhitā\- ca nā\-stiteti sandigdhavyatirekitvā\-d anaikā\-ntikaṃ bhā\-vā\-bhā\-vasā\-dhā\-raṇyam, anyathā\-siddhaṃ veti vikalpitam | tad aprastutam | tā\-vatā\- tā\-van na prakṛtakṣatiḥ | jā\-tau bharaṃ nyasyatā\- svalakṣaṇavā\-cyatvasya svayaṃ svī\-kā\-rā\-t | kiṃ ca sarvatra padā\-rthaya svalakṣaṇasvarū\-peṇaivā\-stitvā\-dikaṃ cintyate | jā\-tes tu vartamā\-nā\-divyaktisambadhī\- 'stitvā\-dikam iti tu bā\-lapratā\-raṇam | evaṃ jā\-timadvyaktivacane 'pi doṣaḥ | vyakteś cet pratī\-tisiddhiḥ jā\-tir adhikā\- pratī\-yatā\-ṃ mā\- vā\-, na tu vyaktipratī\-tidoṣā\-nmuktiḥ | \edlabel{thakur75-61.25}\label{thakur75-61.25} etena yad ucyate Kaumā\-rilaiḥ sabhā\-gatvā\-d eva vastuno na sā\-dhā\-raṇyadoṣaḥ | vṛkṣatvaṃ hy anirdhā\-ritabhā\-vā\-bhā\-vaṃ śabdā\-d avagamyate | tayor anyatareṇa śabdā\-ntarā\-vagatena sambadhyata iti | tad apy asaṅgatam | sā\-mā\-nyasya nityasya pratipattā\-v anirdhā\-ritabhā\-vā\-bhā\-vatvā\-yogā\-t | \edlabel{thakur75-62.1}\label{thakur75-62.1} yac cedam - na ca pratyakṣasyeva śabdā\-nā\-m arthapratyā\-yanaprakā\-ro yena taddṛṣṭa ivā\-styā\-diśabdā\-pekṣā\- na syā\-t, vicitraśaktitvā\-t pramā\-ṇā\-nā\-m iti | tad apy aindriyakaśā\-bdapratibhā\-sayor ekasvarū\-pagrā\-hitve bhinnā\-vabhā\-sadū\-ṣaṇena dū\-ṣitam | vicitraśaktitvaṃ ca pramā\-ṇā\-nā\-ṃ sā\-kṣā\-tkā\-rā\-dhyavasā\-yā\-bhyā\-m api caritā\-rtham | tato yadi pratyakṣā\-rthapratipā\-danaṃ śā\-bdena tadvad evā\-vabhā\-saḥ syā\-t | abhavaṃś ca na tadviṣayakhyā\-panaṃ kṣamate || \edlabel{thakur75-62.6}\label{thakur75-62.6} nanu vṛkṣaśabdena vṛkṣatvā\-ṃśo codite sattvā\-dyaṃśaniścayanā\-rtham astyā\-dipadaprayoga iti cet | \edlabel{thakur75-62.8}\label{thakur75-62.8} niraṃśatvena pratyakṣasamadhigatasya svalakṣaṇasya ko 'vakā\-śaḥ padā\-ntareṇa | dharmā\-ntaravidhiniṣedhayoḥ pramā\-ṇā\-ntareṇa vā\- | pratyakṣe 'pi pramā\-ṇā\-ntarā\-pekṣā\- dṛṣṭeti cet | bhavatu tasyā\-niścayā\-tmakatvā\-d anabhyastasvarū\-paviṣaye | vikalpas tu svayaṃ niścayā\-tmako yatra grā\-hī\- tatra kim apareṇa | asti ca śabdaliṅgā\-ntarā\-pekṣā\- | tato na vastusvarū\-pagrahaḥ || \edlabel{thakur75-62.13}\label{thakur75-62.13} nanu bhinnā\- jā\-tyā\-dayo dharmā\-ḥ parasparaṃ dharmiṇaś ceti jā\-tilakṣaṇaikadharmadvā\-reṇa pratī\-te 'pi śā\-khini dharmā\-ntaravattayā\- na pratī\-tir iti kiṃ na bhinnā\-bhidhā\-nā\-dhī\-no dharmā\-ntarasya nī\-lacaloccais taratvā\-der avabodhaḥ | tad etad asaṅgatam | akhaṇḍā\-tmanaḥ svalakṣaṇasya pratyakṣe 'pi pratibhā\-sā\-t | dṛśyasya dharmadharmibhedasya pratyakṣapratikṣitpatatvā\-t | anyathā\- sarvaṃ sarvatra syā\-d ity atiprasaṅgaḥ | kā\-lpanikabhedā\-śrayas tu dharmadharmivyavahā\-ra iti prasā\-dhitaṃ śā\-stre \footnote{\begin{english}(PVin?)\end{english}} |
	\pend
      

	  \pstart bhavatu vā\- pā\-ramā\-rthiko 'pi dharmadharmibhedaḥ | tathā\-py anayoḥ samavā\-yā\-der dū\-ṣitatvā\-d upakā\-ralakṣaṇaiva pratyā\-sattir eṣitavyā\- | evaṃ ca yathendriyapratyā\-sattyā\- pratyakṣeṇa dharmipratipattau sakalataddharmapratipattis tathā\- śabdaliṅgā\-bhyā\-m api vā\-cyavā\-cakā\-disambandhapratibaddhā\-bhyā\-ṃ dharmipratipatau niravaśeṣataddharmapratipattir bhavet | pratyā\-sattimā\-trasyā\-viśeṣā\-t ||
	\pend
      

	  \pstart yac ca Vā\-caspatiḥ, na caikopā\-dhinā\- sattvena viśiṣṭe tasmin gṛhī\-te upā\-dhyantaraviśiṣṭas tadgrahaḥ | svabhā\-vo hi dravyasyopā\-dhibhir viśiṣyate | na tū\-pā\-dhayo vā\- viśeṣyatvaṃ vā\- tasya svabhā\-va iti | tad api plavata eva | na hy abhedā\-d upā\-dhyantaragrahaṇam ā\-sañjitam | bhedaṃ punas kṛtyaivopakā\-rakagrahaṇe upakā\-ryagrahaṇaprasañjanā\-t | na cā\-gnidhū\-mayoḥ kā\-ryakā\-raṇabhā\-va iva svabhā\-vata eva dharmadharmiṇoḥ pratipattiniyamakalpanam ucitam | tayor api pramā\-ṇā\-siddhatvā\-t | pramā\-ṇsiddhe ca svabhā\-vopavarṇanam iti nyā\-yaḥ || \edlabel{thakur75-63.3}\label{thakur75-63.3} yac cā\-tra Nyā\-yabhū\-ṣaṇena sū\-ryā\-digrahaṇe tadupakā\-ryā\-śeṣavasturā\-śigrahaṇaprasañjanam uktam, tadabhiprā\-yā\-navagā\-hanaphalam | tathā\- hi tvanmate dharmadharmiṇor bhedaḥ, upakā\-ralakṣaṇaiva ca pratyā\-sattis tadopakā\-rakagrahaṇe samā\-nadeśasyaiva dharmarū\-pasyaiva copakā\-ryasya grahaṇam ā\-sañjitam | tat kathaṃ sū\-ryopakā\-ryasya bhinnadeśasya dravyā\-ntarasya vā\- dṛṣṭavyabhicā\-rasya grahaṇaprasaṅgaḥ saṅgataḥ | tasmā\-d ekadharmadvā\-reṇā\-pi vastusvarū\-papratipattau sarvā\-tmapratī\-teḥ kva śabdā\-ntareṇa vidhiniṣedhā\-vakā\-śaḥ | asti ca | tasmā\-n na svalakṣaṇsya śabdavikalpaliṅgapratibhā\-sitvam iti sthitam || \edlabel{thakur75-63.10}\label{thakur75-63.10} nā\-pi sā\-mā\-nyaṃ śā\-bdapratyayapratibhā\-si | saritaḥ pā\-re gā\-vaś carantī\-ti gavā\-diśabdā\-t sā\-snā\-śṛṅgalā\-ṅgū\-lā\-dayo 'kṣarā\-kā\-raparikaritā\-ḥ sajā\-tī\-yabhedā\-parā\-marśanā\-t sampiṇḍitaprā\-yā\-ḥ pratibhā\-sante | na ca tad eva sā\-mā\-nyam |
	\pend
      
	    
	    \stanza[\smallbreak]
varṇā\-kṛtyakṣarā\-kā\-raśū\-nyaṃ gotvaṃ hi kathyate | \footnote{\begin{english}(PV III 147)\end{english}}\&[\smallbreak]


	

	  \pstart tad eva ca sā\-snā\-śṛṅgā\-dimā\-tram akhilavyaktā\-v atyantavilakṣaṇam api svalakṣaṇenaikī\-kriyamā\-ṇaṃ sā\-mā\-nyam ity ucyate tā\-dṛśasya bā\-hyasyā\-prā\-pter bhrā\-ntir evā\-sau keśapratibhā\-savat | tasmā\-d vā\-sanā\-vaśā\-d buddher eva tadā\-tmanā\- vivarto 'yam astu | asad eva vā\- tadrū\-paṃ khyā\-tu | vyaktaya eva vā\- svajā\-tī\-yabhedatiraskā\-reṇā\-nyathā\- bhā\-santā\-m anubhavavyavadhā\-nā\-t smṛtipramoṣo vā\-bhidhī\-yatā\-m | sarvathā\- nirviṣayaḥ khalv ayaṃ sā\-mā\-nyapratyayaḥ | kva sā\-mā\-nyavā\-rtā\- | 
	\pend
      

	  \pstart yat punaḥ sā\-mā\-nyā\-bhā\-ve sā\-mā\-nyapratyayasyā\-kasmikatvam uktaṃ tad ayuktam | yataḥ pū\-rvapiṇḍadarśanasmaraṇasahakā\-riṇā\-tiricyamā\-naviśeṣapratyayajanikā\- sā\-magrī\- nirviṣayaṃ sā\-mā\-nyavikalpam utpā\-dayati | tad evaṃ na śā\-bde pratyaye jā\-tiḥ pratibhā\-ti | nā\-pi pratyakṣe | na cā\-numā\-nato 'pi siddhiḥ | adṛśyatve pratibaddhaliṅgā\-d adarśanā\-t | nā\-pī\-ndriyavad asyā\-ḥ siddhiḥ jñā\-nakā\-ryataḥ kā\-dā\-citkasyaiva nimittā\-ntarasya siddheḥ | yadā\- piṇḍā\-ntare antarā\-le vā\- gobuddher abhā\-vaṃ darśayet tadā\- śā\-valeyā\-disakalagopiṇḍā\-nā\-m evā\-bhā\-vā\-d abhā\-vo gobuddher upapadyamā\-naḥ katham arthā\-ntaram ā\-kṣipet | atha gotvā\-d eva gopiṇḍaḥ | anyathā\- turago 'pi gopiṇḍaḥ syā\-t | yady evaṃ gopiṇḍā\-d eva gotvam anyathā\- turagatvam api gotvaṃ syā\-t | tasmā\-t kā\-raṇaparamparā\-ta eva gopiṇḍo gotvaṃ tu bhavatu mā\- vā\- | nanu sā\-mā\-nyapratyayajananasā\-marthyaṃ yady ekasmā\-t piṇḍā\-d abhinnaṃ tadā\- vijā\-tī\-yavyā\-vṛttaṃ piṇḍā\-ntaram asamartham | atha bhinnam, tadā\- tad eva sā\-mā\-nyam, nā\-mni paraṃ vivā\-da iti cet | abhinnaiva sā\- śaktiḥ prativastu | yathā\- tv ekaḥ śaktasvabhā\-vo bhā\-vas tathā\-nyo 'pi bhavan kī\-dṛśaṃ doṣam ā\-vahati | yathā\- bhavatā\-ṃ jā\-tir ekā\-pi samā\-nadhvaniprasavahetuḥ, anyā\-pi svarū\-peṇaiva jā\-tyantaranirapekṣā\-, tathā\-smā\-kaṃ vyaktir api jā\-tinirapekṣā\- svarū\-peṇaiva bhinnā\- hetuḥ || \edlabel{thakur75-64.7}\label{thakur75-64.7} yat tu \persName{trilocanaḥ}: aśvatvagotvā\-dī\-nā\-ṃ sā\-mā\-nyaviśeṣā\-ṇā\-ṃ svā\-śraye samavā\-yaḥ sā\-mā\-nyaṃ sā\-mā\-nyam ity abhidhā\-napratyayor nimittam iti | yady evaṃ vyaktiṣv apy ayam eva tathā\-bhidhā\-napratyayahetus tu, kiṃ sā\-mā\-nyasvī\-kā\-rapramā\-dena | na ca samavā\-yaḥ sambhavī\- |
	\pend
      

	  \pstart iheti buddheḥ samavā\-yasiddhir iheti dhī\-ś ca dvayadarśanena | na ca kvacit tadviṣaye tv apekṣā\- svakalpanā\-mā\-tramato 'bhyupā\-yaḥ || \edlabel{thakur75-64.15}\label{thakur75-64.15} etena seyaṃ pratyayā\-nuvṛttir anuvṛttavastvanuyā\-yinī\- katham atyantabhedinī\-ṣu vyaktiṣu vyā\-vṛttaviṣayapratyayabhā\-vā\-nupā\-tinī\-ṣu bhavitum arhatī\-ty ū\-hā\-pravartanam asya pratyā\-khyā\-tam | jā\-tiṣv eva parasparavyā\-vṛttatayā\- vyaktī\-yamā\-nā\-sv anuvṛttapratyayena vyabhicā\-rā\-t | \edlabel{thakur75-64.19}\label{thakur75-64.19} yat punar anena viparyaye bā\-dhakam uktam, abhidhā\-napratyayā\-nuvṛttiḥ kutaścin nivṛttya kvacid eva bhavantī\- nimittavatī\-, na cā\-nyannimittam ityā\-di | tan na samyak | anuvṛttam anyatreṇā\-py abhidhā\-napratyayā\-nuvṛtter atadrū\-paparā\-vṛttasvarū\-paviśeṣā\-d avaśyaṃ svī\-kā\-rasya sā\-dhitatvā\-t | tasmā\-t
	\pend
      
	    
	    \stanza[\smallbreak]
tulye bhede yayā\- jā\-tiḥ pratyā\-sattyā\- prasarpati |&kvacin nā\-nyatra saivā\-stu śabdajñā\-nanibandhanam ||\footnote{\begin{english}PV I 162\end{english}}\&[\smallbreak]


	

	  \pstart yat punar atra Nyā\-yabhū\-ṣaṇoktam: na hy evaṃ bhavati, yayā\- pratyā\-sattyā\- daṇḍasū\-trā\-dikaṃ prasarpati kvacin nā\-nyatra saiva pratyā\-sattiḥ puruṣasphaṭikā\-diṣu daṇḍisū\-tritvā\-divyavahā\-ranibandhanam astu, kiṃ daṇḍasū\-trā\-dineti | tad asaṅgatam | daṇḍasū\-trayor hi puruṣasphaṭikapratyā\-sannyoḥ dṛṣṭayoḥ daṇḍisū\-tritvapratyayahetutvaṃ nā\-palapyate | sā\-mā\-nyaṃ tu svapne 'pi na dṛṣṭam | tad yadī\-daṃ parikalpanī\-yaṃ tadā\- varaṃ pratyā\-sattir eva sā\-mā\-nyapratyayahetuḥ parikalpyatā\-m, kiṃ gurvyā\- parikalpanayety abhiprā\-yā\-parijñā\-nā\-t |
	\pend
      

	  \pstart \edlabel{thakur75-65.1}\label{thakur75-65.1} athedaṃ jā\-tiprasā\-dhakam anumā\-nam abhidhī\-yate | yad viśiṣṭajñā\-naṃ tadviśeṣaṇagrahaṇanā\-ntarī\-yakam | yathā\- daṇḍijñā\-nam | viśiṣṭajñā\-naṃ cedaṃ gaurayam ity arthataḥ kā\-ryahetuḥ | viśeṣaṇā\-nubhavakā\-ryaṃ hi dṛṣṭā\-nte viśiṣṭabuddhiḥ siddheti | atrā\-nuyogaḥ | viśiṣṭabuddher bhinnaviśeṣaṇagrahaṇanā\-ntarī\-yakatvaṃ vā\- sā\-dhyaṃ viśeṣaṇamā\-trā\-nubhavanā\-ntarī\-yakatvaṃ vā\- |
	\pend
      

	  \pstart \edlabel{thakur75-65.6}\label{thakur75-65.6} prathamapakṣe pakṣasya pratyakṣabā\-dhā\- sā\-dhanā\-vadhā\-nam anavakā\-śayati, vastugrā\-hiṇaḥ pratyakṣasyobhayapratibhā\-sā\-bhā\-vā\-t | viśiṣṭabuddhitvaṃ ca sā\-mā\-nyahetur anaikā\-ntikaḥ, bhinnaviśeṣaṇagrahaṇam antareṇā\-pi darśanā\-t | yathā\- svarū\-pavā\-n ghaṭaḥ, gotvaṃ sā\-mā\-nyam iti vā\- |
	\pend
      

	  \pstart \edlabel{thakur75-65.10}\label{thakur75-65.10} dvitī\-yapakṣe tu siddhasā\-dhanam | svarū\-pavā\-n ghaṭa ityā\-divat gotvajā\-timā\-n piṇḍa iti parikalpitaṃ bhedam upā\-dā\-ya viśeṣaṇaviśeṣyabhā\-vasyeṣṭatvā\-d agovyā\-vṛttā\-nubhavabhā\-vitvā\-d gaurayam iti vyavahā\-rasya | tad evaṃ na sā\-mā\-nyasiddhiḥ | bā\-dhakaṃ ca sā\-mā\-nyaguṇakarmā\-dyupā\-dhicakrasya kevalavyaktigrā\-hakaṃ paṭupratyakṣaṃ dṛśyā\-nulambho vā\- prasiddhaḥ |
	\pend
      

	  \pstart \edlabel{thakur75-65.15}\label{thakur75-65.15} tad evaṃ vidhir eva śabdā\-rthaḥ | sa ca bā\-hyo 'rtho buddhyā\-kā\-raś ca vivakṣitaḥ | tatra na buddhyā\-kā\-rasya tattvataḥ saṃvṛtyā\- vā\- vidhiniṣedhau, svasaṃvedanapratyakṣagamyatvā\-t | anadhyavasā\-yā\-c ca | nā\-pi tattvato bā\-hyasyā\-pi vidhiniṣedhau, tasya śā\-bde pratyaye 'pratibhā\-sanā\-t | ata eva sarvadharmā\-ṇā\-ṃ tattvato 'nabhilā\-pyatvaṃ pratibhā\-sā\-dhyavasā\-yā\-bhā\-vā\-t | tasmā\-d bā\-hyasyaiva sā\-ṃvṛttau vidhiniṣedhau | anyathā\- saṃvyavahā\-rahā\-niprasaṅgā\-t | tad evaṃ
	\pend
      
	    
	    \stanza[\smallbreak]
\edlabel{thakur75-65.21}\flagstanza{\tiny\textenglish{...-65.21}}nā\-kā\-rasya na bā\-hyasya tattvato vidhisā\-dhanam |&bahir eva hi saṃvṛtyā\- samvṛtyā\-pi tu nā\-kṛteḥ ||\footnote{\begin{english}Corresponds to AP 229.3–4, Sā\-SiŚā\- 443.13–14.\end{english}}\&[\smallbreak]


	

	  \pstart etena yad \persName{Dharmottaraḥ} ā\-ropitasya bā\-hyatvasya vidhiniṣedhā\-v ity alaukikam anā\-gamamatā\-rkikī\-yaṃ kathayati, tad apy apahastitam | \edlabel{thakur75-65.26}\label{thakur75-65.26} nanv adhyavasā\-ye yady adhyavaseyaṃ vastu na sphurati tadā\- tad adhyavasitam iti ko 'rthaḥ | apratibhā\-se 'pi pravṛttiviṣayī\-kṛtam iti yo 'rthaḥ | apratibhā\-sā\-viśeṣe viṣayā\-ntaraparihā\-reṇa kathaṃ niyataviṣayā\- pravṛttir iti cet | ucyate | yady api viśvam agṛhī\-taṃ tathā\-pi vikalpasya niyatasā\-magrī\-prasū\-tatvena niyatā\-kā\-ratayā\-, niyataśaktitvā\-t niyataiva jalā\-dau pravṛttiḥ | dhū\-masya parokṣā\-gnijñā\-najananavat | \edlabel{thakur75-66.1}\label{thakur75-66.1} niyataviṣayā\- hi bhā\-vā\-ḥ pramā\-ṇapariniṣṭhitasvabhā\-vā\- na śaktisā\-ṃkaryaparyanuyogabhā\-jaḥ | tasmā\-t tadadhyavasā\-yitvam ā\-kā\-raviśeṣayogā\-t tatpravṛttijanakatvam | na ca sā\-dṛśyā\-d ā\-ropeṇa pravṛttiṃ brū\-maḥ, yenā\-kā\-re bā\-hyasya bā\-hye vā\-kā\-rasyā\-ropadvā\-reṇa dū\-ṣaṇā\-vakā\-śaḥ | kiṃ tarhi svavā\-sanā\-vipā\-kavaśā\-d upajā\-yamā\-naiva buddhir apaśyanty api bā\-hyaṃ bā\-hye pravṛttim ā\-tanotī\-ti viplutaiva | tad evam anyā\-bhā\-vaviśiṣṭo vijā\-tivyā\-vṛtto 'rtho vidhiḥ | sa eva cā\-pohaśabdavā\-cyaḥ śabdā\-nā\-m arthaḥ pravṛttinivṛttiviṣayaś ceti sthitam | \edlabel{thakur75-66.8}\label{thakur75-66.8} atra prayogaḥ | yad vā\-cakaṃ tat sarvam adhyavasitā\-tadrū\-paparā\-vṛttavastumā\-tragocaram | yatheha kū\-pe jalam iti vacanam | vā\-cakaṃ cedaṃ gavā\-diśabdarū\-pam iti svabhā\-vahetuḥ | nā\-yam asiddhaḥ | pū\-rvoktena nyā\-yena pā\-ramā\-rthikavā\-cyavā\-cakabhā\-vasyā\-bhā\-ve 'pi adhyavasā\-yakṛtasyaiva sarvavyavahā\-ribhir avaśyaṃ svī\-karttavyatvā\-t |anyathā\- sarvavyavahā\-rocchedaprasaṅgā\-t | nā\-pi viruddhaḥ | sapakṣe bhā\-vā\-t | na cā\-naikā\-ntikaḥ | tathā\- hi śabdā\-nā\-m adhyavasitavijā\-tivyā\-vṛttavastumā\-traviṣayatvam anicchadbhiḥ paraiḥ paramā\-rthato
	\pend
      

	  \pstart vā\-cyaṃ svalakṣaṇam upā\-dhir upā\-dhiyogaḥ sopā\-dhir astu yadi vā\- kṛtir astu buddhaḥ |
	\pend
      

	  \pstart gatyantarā\-bhā\-vā\-t | aviṣayatve ca vā\-cakatvā\-yogā\-t | tatra
	\pend
      

	  \pstart ā\-dyantayor na samayaḥ phalaśaktihā\-ner madhye 'py upā\-dhivirahā\-t tritayena yuktaḥ || \edlabel{thakur75-66.19}\label{thakur75-66.19} tad evaṃ vā\-cyā\-ntarasyā\-bhā\-vā\-t viṣayavattvalakṣaṇasya vyā\-pakasya nivṛttau vipakṣato nivarttamā\-naṃ vā\-cakatvam adhyavasitabā\-hyaviṣayatvena vyā\-pyata iti vyā\-ptisiddhiḥ |
	\pend
      

	  \pstart mahā\-paṇḍitaratnakī\-rtipā\-daviracitam apohaprakaraṇaṃ samā\-ptam || 
	\pend
      
	  
	% new div opening: depth here is 1
	
\section[{Kṣaṇabhaṅgasiddhiḥ Anvayā\-tmikā\-}]{Kṣaṇabhaṅgasiddhiḥ Anvayā\-tmikā\-}\edlabel{Kṣaṇabhaṅgasiddhiḥ_Anvayātmikā}\label{Kṣaṇabhaṅgasiddhiḥ_Anvayātmikā}

	  \pstart namas tā\-rā\-yai ||
	\pend
      
	    
	    \stanza[\smallbreak]
ā\-kṣiptavyatirekā\- yā\- vyā\-ptir anvayarū\-piṇī\- |&sā\-dharmyavati dṛṣṭā\-nte sattvahetor ihocyate ||\&[\smallbreak]


	

	  \pstart yat sat tat kṣaṇikam, yathā\- ghaṭaḥ, santaś cā\-mī\- vivā\-dā\-spadī\-bhū\-tā\-ḥ padā\-rthā\- iti |
	\pend
      

	  \pstart hetoḥ parokṣā\-rtha pratipā\-dakatvaṃ hetvā\-bhā\-satvaśaṅkā\-nirā\-karaṇam antareṇa na śakyate pratipā\-dayitum | hetvā\-bhā\-sā\-ś ca asiddhaviruddhā\-naikā\-ntikabhedena trividhā\-ḥ |
	\pend
      

	  \pstart tatra na tā\-vad ayam asiddho hetuḥ | 
	\pend
      

	  \pstart yadi nā\-ma darśane darśane nā\-nā\-prakā\-raṃ sattvalakṣaṇam uktam ā\-ste, arthakriyā\-kā\-ritvaṃ, sattā\-samavā\-yaḥ, svarū\-pasattvam, utpā\-davyayadhrauvyayogitvaṃ, pramā\-ṇaviṣayatvaṃ, sad upalambhaka pramā\-ṇagocaratvaṃ, vyapadeśaviṣayatvam ityā\-di, tathā\-pi kim anenā\-prastutenedā\-nī\-m eva niṣṭaṅkitena | yad eva hi pramā\-ṇato nirū\-pyamā\-ṇaṃ padā\-rthā\-nā\-ṃ sattvam upapannaṃ bhaviṣyati tad eva vayam api svī\-kariṣyā\-maḥ | 
	\pend
      

	  \pstart kevalaṃ tad etad arthakriyā\-kā\-ritvaṃ sarvajanaprasiddham ā\-ste
	\pend
      

	  \pstart tat khalv atra sattvaśabdenā\-bhisandhā\-ya sā\-dhanatvenopā\-ttam | tac ca
	\pend
      

	  \pstart yathā\-yogaṃ pratyakṣā\-numā\-napramā\-ṇaprasiddhasadbhā\-veṣu bhā\-veṣu
	\pend
      

	  \pstart pakṣī\-kṛteṣu pratyakṣā\-dinā\- pramā\-ṇena pratī\-tam iti na
	\pend
      

	  \pstart svarū\-peṇā\-śrayadvā\-reṇa vā\-siddhi sambhā\-vanā\-pi ||
	\pend
      

	  \pstart nā\-pi viruddhatā\-, sapakṣī\-kṛte ghaṭe sadbhā\-vā\-t |
	\pend
      

	  \pstart nanu katham asya sapakṣatvam, pakṣavad atrā\-pi kṣaṇabhaṅgā\-siddheḥ | na hy asya pratyakṣataḥ kṣaṇabhaṅgasiddhiḥ, tathā\-tvenā\-niścayā\-t | nā\-pi sattvā\-numā\-nataḥ, punarnidarśanā\-ntarā\-pekṣā\-yā\-m anavasthā\- prasaṅgā\-t | na cā\-nyad anumā\-nam asti | sambhave vā\- tenaiva pakṣe 'pi kṣaṇabhaṅgasiddher alaṃ sattvā\-numā\-neneti cet |
	\pend
      

	  \pstart ucyate | anumā\-nā\-ntaram eva prasaṅgaprasaṅgaviparyayā\-tmakaṃ ghaṭe kṣaṇabhaṅgaprasā\-dhakaṃ pramā\-ṇā\-ntaram asti | 
	\pend
      

	  \pstart tathā\- hi ghaṭo vartamā\-nakṣaṇe tā\-vad ekā\-m arthakriyā\-ṃ karoti | atī\-tā\-nā\-gatakṣaṇayor api kiṃ tā\-m evā\-rthakriyā\-ṃ kuryā\-t, anyā\-ṃ vā\-, na vā\- kā\-m api kriyā\-m iti trayaḥ pakṣā\-ḥ | 
	\pend
      

	  \pstart nā\-tra prathamaḥ pakṣo yuktaḥ, kṛtasya karaṇā\-yogā\-t |
	\pend
      

	  \pstart atha dvitī\-yo 'bhyupagamyate, tad idam atra vicā\-ryatā\-m | yadā\- ghaṭo vartamā\-nakṣaṇabhā\-vi kā\-ryaṃ karoti tadā\- kim atī\-tā\-nā\-gatakṣaṇabhā\-viny api kā\-rye śakto 'śakto vā\- |
	\pend
      

	  \pstart yadi śaktas tadā\- vartamā\-nakṣaṇabhā\-vikā\-ryavad atī\-tā\-nā\-gatakṣaṇabhā\-vy api kā\-ryaṃ tadaiva kuryā\-t | tatrā\-pi śaktatvā\-t | śaktasya ca kṣepā\-yogā\-t, anyathā\- varttamā\-nakṣaṇabhā\-vino 'pi kā\-ryasyā\-karaṇaprasaṅgā\-t pū\-rvā\-parakā\-layor api śaktatvenā\-viśeṣā\-t | samarthasya ca sahakā\-ryapekṣā\-yā\- ayogā\-t | 
	\pend
      

	  \pstart athā\-śaktaḥ, tadaikatra kā\-rye śaktā\-śaktatvaviruddhadharmā\-dhyā\-sā\-t kṣaṇavidhvaṅso ghaṭasya durvā\-raprasaraḥ syā\-t |
	\pend
      

	  \pstart nā\-pi tṛtī\-yaḥ pakṣaḥ saṅgacchate , śaktasvabhā\-vā\-nuvṛtter eva | yadā\- hi śaktasya padā\-rthasya vilambo 'py asahyas tadā\- dū\-rotsā\-ritam akaraṇam | anyathā\- vā\-rtamā\-nikasyā\-pi kā\-ryasyā\-karaṇaṃ syā\-d ity uktam | 
	\pend
      

	  \pstart tasmā\-d yad yadā\- yajjananavyavahā\-rapā\-traṃ tat tadā\- tat kuryā\-t | akurvac ca na jananavyavahā\-rabhā\-janam | tad evam ekatra kā\-rye samarthetarasvabhā\-vatayā\- pratikṣaṇaṃ bhedā\-d ghaṭasya sapakṣatvam akṣatam |
	\pend
      

	  \pstart atra prayogaḥ | yad yadā\- yajjananavyavahā\-rayogyaṃ tat tadā\- taj janayaty eva | yathā\- 'ntyā\- kā\-raṇasā\-magrī\- svakā\-ryam | atī\-tā\-nā\-gatakṣaṇabhā\-vikā\-ryajananavyavahā\-rayogyaś cā\-yaṃ ghaṭo vartamā\-nakṣaṇabhā\-vikā\-ryakaraṇakā\-le sakalakriyā\-tikramakā\-le 'pī\-ti svabhā\-vahetuprasaṅgaḥ |
	\pend
      

	  \pstart asya ca dvitī\-yā\-dikṣaṇabhā\-vikā\-ryakaraṇavyavahā\-ragocaratvasya prasaṅgasā\-dhanasya vā\-rtamā\-nikakā\-ryakaraṇakā\-le sakalakriyā\-tikramakā\-le ca ghaṭe dharmiṇi parā\-bhyupagamamā\-trataḥ siddhatvā\-d asiddhis tā\-vad asambhavinī\- |
	\pend
      

	  \pstart nā\-pi viruddhatā\-, sapakṣe 'ntya kā\-raṇasā\-magryā\-ṃ sadbhā\-vasambhavā\-t| 
	\pend
      

	  \pstart nanv ayaṃ sā\-dhā\-raṇā\-naikā\-ntiko hetuḥ | sā\-kṣā\-dajanake 'pi kuśū\-lā\-dyavasthitabī\-jā\-dau vipakṣe samarthavyavahā\-ragocaratvasya sā\-dhanasya darśanā\-d iti cet | 
	\pend
      

	  \pstart na | dvividho hi samarthavyavahā\-raḥ pā\-ramā\-rthika aupacā\-rikaś ca | tatra yat pā\-ramā\-rthikaṃ jananaprayuktaṃ jananavyavahā\-ragocaratvaṃ tad iha sā\-dhanatvenopā\-ttam | tasya ca kuśū\-lā\-dyavasthitabī\-jā\-dau kā\-raṇakā\-raṇatvā\-d aupacā\-rikajananavyavahā\-raviṣayabhū\-te sambhavā\-bhā\-vā\-t kutaḥ sā\-dhā\-raṇā\-naikā\-ntikatā\- | 
	\pend
      

	  \pstart na cā\-sya sandigdhavyatirekitā\-, viparyaye bā\-dhakapramā\-ṇasadbhā\-vat | 
	\pend
      

	  \pstart tathā\- hī\-daṃ jananavyavahā\-ragocaratvaṃ niyataviṣayatvena vyā\-ptam iti sarvajanā\-nubhavaprasiddham | na cedaṃ nirnimittam, deśakā\-lasvabhā\-vaniyamā\-bhā\-vaprasaṅgā\-t | na ca jananā\-d anyan nimittam upalabhyate, tadanvayavyatirekā\-nuvidhā\-nadarśanā\-t | yadi ca jananam antareṇā\-pi jananavyavahā\-ragocaratvaṃ syā\-t tadā\- sarvasya sarvatra jananavyavahā\-ra ity aniyamaḥ syā\-t | niyataś cā\-yaṃ pratī\-taḥ | tato jananā\-bhā\-ve vipakṣe niyataviṣayatvasya vyā\-pakasya nivṛttau nivartamā\-naṃ jananavyavahā\-ragocaratvaṃ janana eva viśrā\-myatī\-ti vyā\-ptisiddher anavadyo hetuḥ |
	\pend
      

	  \pstart na caiṣa ghaṭo varttamā\-nakā\-ryakaraṇakṣaṇe sakalakriyā\-tikramakā\-le cā\-tī\-tā\-nā\-gatakṣaṇabhā\-vikā\-ryaṃ janayati | tato na jananavyavahā\-rayogyaḥ, sarvaḥ prasaṅgaḥ prasaṅgaviparyayaniṣṭha iti nyā\-yā\-t |
	\pend
      

	  \pstart atrā\-pi prayogaḥ | yad yadā\- yan na karoti na tat tadā\- tatra samarthavyavahā\-rayogyam | yathā\- śā\-lyaṅkuram akurvan kodravaḥ śā\-lyaṅkure | na karoti caiṣa ghaṭo vartamā\-nakṣaṇabhā\-vikā\-ryakaraṇakā\-le sakalakriyā\-tikramakā\-le cā\-tī\-tā\-nā\-gatakṣaṇabhā\-vikā\-ryam iti vyā\-pakā\-nupalabdhir bhinatti samarthakṣaṇā\-d asamarthakṣaṇam |
	\pend
      

	  \pstart atrā\-py asiddhir nā\-sti, vartamā\-nakṣaṇabhā\-vikā\-ryakaraṇakā\-le sakalakriyā\-tikramakā\-le cā\-tī\-tā\-nā\-gatakṣaṇabhā\-vikā\-ryakaraṇasyā\-yogā\-t |
	\pend
      

	  \pstart nā\-pi virodhaḥ, sapakṣe bhā\-vā\-t |
	\pend
      

	  \pstart na cā\-naikā\-ntikatā\-, pū\-rvoktena nyā\-yena samarthavyavahā\-ragocaratvajanakatvayor vidhibhū\-tayoḥ sarvopasaṃhā\-ravatyā\- vyā\-pteḥ prasā\-dhanā\-t ||
	\pend
      

	  \pstart yat punar atroktam yad yadā\- yan na karoti na tat tadā\- tatra samartham ity atra kaḥ karotyarthaḥ | kiṃ kā\-raṇatvam | uta kā\-ryotpā\-dā\-nuguṇasahakā\-risā\-kalyam | ahosvit kā\-ryā\-vyabhicā\-raḥ | kā\-ryasambandho veti | tatra kā\-raṇatvam eva karotyarthaḥ | tataḥ pakṣā\-ntarabhā\-vino doṣā\- anabhyupagamapratihatā\-ḥ |
	\pend
      

	  \pstart na cā\-tra pakṣe kā\-raṇatvasā\-marthyayoḥ paryā\-yatvena vyā\-pakā\-nupalambhasya sā\-dhyā\-viśiṣṭatvam abhidhā\-tum ucitam, samarthavyavahā\-ragocaratvā\-bhā\-vasya sā\-dhyatvā\-t | kā\-raṇatvasamarthavyavahā\-ragocaratvayoś ca vṛkṣaśiṃśapayor iva vyā\-vṛttibhedo 'stī\-ty anavasara evaivaṃvidhasya kṣudrapralā\-pasya |
	\pend
      

	  \pstart tad evaṃ prasaṅgaprasaṅgaviparyayahetudvayabalato ghaṭe dṛṣṭā\-nte kṣaṇabhaṅgaḥ siddhaḥ | tat kathaṃ sattvā\-d anyad anumā\-nam dṛṣṭā\-nte kṣaṇabhaṅgasā\-dhakaṃ nā\-stī\-ty ucyate | na caivaṃ sattvahetor vaiyarthyam, dṛṣṭā\-ntamā\-tra eva prasaṅgaprasaṅgaviparyayā\-bhyā\-ṃ kṣaṇabhaṅgaprasā\-dhanā\-t ||
	\pend
      

	  \pstart nanv ā\-bhyā\-m eva pakṣe 'pi kṣaṇabhaṅgasiddhir astv iti cet | 
	\pend
      

	  \pstart astu, ko doṣaḥ | yo hi pratipattā\- prativastu yad yadā\- yajjananavyavahā\-rayojyaṃ tat tadā\- taj janayatī\-tyā\-dikam upanyasitum analasas tasya tata eva kṣaṇabhaṅgasiddhiḥ | yas tu prativastu tannyā\-yopanyā\-saprayā\-sabhī\-ruḥ sa khalv ekatra dharmiṇi yad yadā\- yajjananavyavahā\-rayogyaṃ tat tadā\- taj janayatī\-tyā\-dinyā\-yena sattvamā\-tram asthairyavyā\-ptam avadhā\-rya sattvā\-d evā\-nyatra kṣaṇikatvam avagacchayatī\-i, katham apramatto vaiyarthyam asyā\-cakṣī\-ta |
	\pend
      

	  \pstart tad evam ekakā\-ryakā\-riṇo ghaṭasya dvitī\-yā\-dikṣaṇabhā\-vikā\-ryā\-pekṣayā\- samarthetarasvabhā\-vaviruddhadharmā\-dhyā\-sā\-d bheda eveti kṣaṇabhaṅgitayā\- sapakṣatā\-m ā\-vahati ghaṭe sattvahetur upalabhyamā\-no na viruddhaḥ | 
	\pend
      

	  \pstart na cā\-yam anaikā\-ntikaḥ, atraiva sā\-dharmyavati dṛṣṭā\-nte sarvopasaṃhā\-ravatyā\- vyā\-pteḥ prasā\-dhanā\-t |
	\pend
      

	  \pstart nanu viparyayabā\-dhakapramā\-ṇabalā\-d vyā\-ptisiddhiḥ | tasya copanyā\-savā\-rtā\-pi nā\-sti | tat kathaṃ vyā\-ptiḥ prasā\-dhiteti cet | 
	\pend
      

	  \pstart tad etat taralabuddhivilasitam | tathā\- hi uktam etad vartamā\-nakṣaṇabhā\-vikā\-ryakaraṇakā\-le 'tī\-tā\-nā\-gatakṣaṇabhā\-vikā\-rye 'pi ghaṭasya śaktisambhave tadā\-nī\-m eva tatkaraṇam , akaraṇe ca śaktā\-śaktasvabhā\-vatayā\- pratikṣaṇaṃ bheda iti kṣaṇikatvena vyā\-ptaiva sā\- arthakriyā\-śaktiḥ || 
	\pend
      

	  \pstart nanv evam anvayamā\-tram astu | vipakṣā\-t punar ekā\-ntena vyā\-vṛttir iti kuto labhyata iti cet | 
	\pend
      

	  \pstart vyā\-ptisiddher eva |
	\pend
      

	  \pstart vyatirekasandehe vyā\-ptisiddhir eva katham iti cet | 
	\pend
      

	  \pstart na | dvividhā\- hi vyā\-ptisiddhiḥ | anvayarū\-pā\- ca kartṛdharmaḥ sā\-dhanadharmavati dharmiṇi sā\-dhyadharmasyā\-vaśyambhā\-vo yaḥ, vyatirekarū\-pā\- ca karmadharmaḥ sā\-dhyā\-bhā\-ve sā\-dhanasyā\-vaśyamabhā\-vo yaḥ | enayoś caikatarapratī\-tir niyamena dvī\-tyapratī\-tim ā\-kṣipati, anyathaikasyā\- evā\-siddheḥ |
	\pend
      

	  \pstart tasmā\-d yathā\- viparyaye bā\-dhakapramā\-ṇabalā\-t niyamavati vyatireke siddhe 'nvayaviṣayaḥ saṃśayaḥ pū\-rvaṃ sthito 'pi paścā\-t parigalati tato 'nvayaprasā\-dhā\-rthaṃ na pṛthak sā\-dhanam ucyate tathā\- prasaṅgatadviparyayahetudvayabalato niyamavaty anvaye siddhe vyatirekaviṣaye pū\-rvaṃ sthito 'pi sandehaḥ paścā\-t parigalaty eva | na ca vyatirekaprasā\-dhakam anyat pramā\-ṇaṃ vaktavyam | tataś ca sā\-dhyā\-bhā\-ve sā\-dhanasyaikā\-ntiko vyatirekaḥ, sā\-dhane sati
	\pend
      

	  \pstart sā\-dhyasyā\-vaśyam anvayo veti na kaścid arthabhedaḥ |
	\pend
      

	  \pstart tad evaṃ viparyayabā\-dhakapramā\-ṇam antareṇā\-pi prasaṅgaprasaṅgaviparyayahetudvayabalā\-d anvayarū\-pavyā\-ptisiddhau sattvahetor anaikā\-ntikatvasyā\-bhā\-vā\-d ataḥ sā\-dhanā\-t kṣaṇabhaṅgasiddhir anavadyeti ||
	\pend
      

	  \pstart nanu ca sā\-dhanam idam asiddham | na hi kā\-raṇabuddhyā\- kā\-ryaṃ gṛhyate, tasya bhā\-vitvā\-t | na ca kā\-ryabuddhyā\- \leavevmode\textsuperscript{\rmlatinfont\tiny [\cite[39b]{RNAms}]} kā\-raṇam, tasyā\-tī\-tatvā\-t | na ca vartamā\-nagrā\-hiṇā\- jñā\-nenā\-tī\-tā\-nā\-gatayor grahaṇaṃ atiprasaṅgā\-t |
	\pend
      

	  \pstart na ca pū\-rvā\-parayoḥ kā\-layor ekaḥ pratisandhā\-tā\- asti, kṣaṇabhaṅgabhaṅgaprasaṅgā\-t | kā\-raṇā\-bhā\-ve tu kā\-ryā\-bhā\-vapratī\-tiḥ svasaṃvedanavā\-dino manorathasyā\-py aviṣayaḥ |
	\pend
      

	  \pstart nanu ca pū\-rvottarakā\-layoḥ saṃvittī\-, tā\-bhyā\-ṃ vā\-sanā\-, tayā\- ca \edtext{hetu}{\Afootnote{hetū\-}}phalā\-vasā\-yī\- vikalpa iti cet tad ayuktam | sa hi vikalpo gṛhī\-tā\-nusandhā\-yako 'tadrū\-pasamā\-ropako vā\- |
	\pend
      

	  \pstart na prathamaḥ pakṣaḥ | ekasya pratisandhā\-tur abhā\-ve pū\-rvā\-paragrahaṇayor ayogā\-t, vikalpavā\-sanā\-yā\- evā\-bhā\-vā\-t |
	\pend
      

	  \pstart nā\-pi dvitī\-yaḥ | marī\-cikā\-yā\-m api jalavijñā\-nasya prā\-mā\-ṇyaprasaṅgā\-t |
	\pend
      

	  \pstart tad evam anvayavatirekyor apratipatter arthakriyā\-lakṣaṇaṃ sattvam asiddham iti || 
	\pend
      

	  \pstart kiṃ ca prakā\-rā\-ntarā\-d apī\-daṃ sā\-dhanam asiddham | tathā\- hi bī\-jā\-dī\-nā\-ṃ sā\-marthyaṃ bī\-jā\-dijñā\-nā\-t tatkā\-ryā\-d aṅkurā\-der vā\- niścetavyam |
	\pend
      

	  \pstart kā\-ryatvaṃ ca vastutvasiddhau sidhyati | vastutvaṃ ca kā\-ryā\-ntarā\-t | kā\-ryā\-ntarasyā\-pi kā\-ryatvaṃ vastutvasiddhau | tadvastutvaṃ ca tadaparakā\-ryā\-ntarā\-d ity anavasthā\- |
	\pend
      

	  \pstart athā\-navasthā\-bhayā\-t paryante kā\-ryā\-ntaraṃ nā\-pekṣate tadā\- tenaiva pū\-rveṣā\-m asattvaprasaṅgā\-n naikasyā\-py arthakriyā\-sā\-marthyaṃ sidhyati |
	\pend
      

	  \pstart nanu kā\-ryatvasattvayor bhinnavyā\-vṛttikatvā\-t \edtext{sattvāsiddhāv}{\Afootnote{satvā\-siddhĀv°; sattā\-siddhā\-v\textenglish{---\textsc{Note} tva and tta are very similar in this script. Corroboration will need more examples.}}} api kā\-ryatvasiddhau kā\- kṣatir iti cet |
	\pend
      

	  \pstart tad asaṅgatam | saty api kā\-ryatvasattvayor vyā\-vṛttibhede \edtext{sattvāsiddhāu}{\Afootnote{satvā\-siddhau; sattā\-siddhā\-u}} kutaḥ kā\-ryatvasiddhiḥ | kā\-ryatvaṃ hy abhū\-tvā\-bhā\-vitvaṃ | bhavanaṃ ca sattā\- | sattā\- ca saugatā\-nā\-ṃ sā\-marthyam eva | tataś ca sā\-marthyasandehe bhavatī\-ty eva vaktum aśakyam | katham abhū\-tvā\-bhā\-vitvaṃ kā\-ryatvaṃ setsyati |
	\pend
      

	  \pstart apekṣitaparavyā\-pā\-ratvaṃ kā\-ryatvam ity api nā\-sato dharmaḥ | sattvaṃ ca sā\-marthyam | tac ca sandigdham iti kutaḥ kā\-ryatvasiddhiḥ | tadasiddhau pū\-rvasya sā\-marthyaṃ na sidhyatī\-ti sandigdhā\-siddho hetuḥ || 
	\pend
      

	  \pstart tathā\- viruddho 'py ayam | tathā\- hi kṣaṇikatve sati na tā\-vad ajā\-tasyā\-nanvayaniruddhasya vā\- kā\-ryā\-rambhakatvaṃ sambhavati | na ca niṣpannasya tā\-vā\-n kṣaṇo 'sti yam upā\-dā\-ya kasmaicit kā\-ryā\-ya vyā\-pā\-ryeta | ataḥ kṣaṇikapakṣa evā\-rthakriyā\-nupapatter viruddhatā\- | 
	\pend
      

	  \pstart athavā\- vikalpena yad upanī\-yate tat sarvam avastu | tataś ca vastvā\-tmake kṣaṇikatve sā\-dhye 'vastū\-pasthā\-payann anumā\-navikalpo viruddhaḥ | 
	\pend
      

	  \pstart yadvā\- sarvasyaiva hetoḥ kṣaṇikatve sā\-dhye viruddhatvaṃ | deśakā\-lā\-ntarā\-nanugame sā\-dhyasā\-dhanabhā\-vā\-bhā\-vā\-t | anugame ca nā\-nā\-kā\-lam ekam akṣaṇikaṃ kṣaṇikatvena virudhyata iti ||
	\pend
      

	  \pstart anaikā\-ntiko 'py ayam, sattvasthairyayor virodhā\-bhā\-vā\-d iti |
	\pend
      

	  \pstart atrocyate | yat tā\-vad uktaṃ sā\-marthyaṃ na pratī\-yata iti, tat kiṃ sarvathaiva na pratī\-yate kṣaṇabhaṅgapakṣe vā\- |
	\pend
      

	  \pstart prathamapakṣe sakalakā\-rakajñā\-pakahetucakrocchedā\-n mukhaspandanamā\-trasyā\-py akaraṇaprasaṅgaḥ | anyathā\- yenaiva vacanena sā\-marthyaṃ nā\-stī\-ti pratipā\-dyate tasyaiva tatpratipā\-danasā\-marthyam avyā\-hatam ā\-yā\-tam | tasmā\-t paramapuruṣā\-rthasamī\-hayā\- vastutattvanirū\-paṇapravṛttasya śaktisvī\-kā\-rapū\-rvakaiva pravṛttiḥ | tadasvī\-kā\-re tu na kaścit kvacit pravarteteti nirī\-haṃ jagaj jā\-yeta |
	\pend
      

	  \pstart atha dvitī\-yaḥ pakṣaḥ, tadā\-sti tā\-vat sā\-marthyapratī\-tiḥ | sā\- ca kṣaṇikatve yadi nopapadyate tadā\- viruddhaṃ vaktum ucitam | asiddham iti tu nyā\-yabhū\-ṣaṇī\-yaḥ prā\-yo vilā\-paḥ | 
	\pend
      

	  \pstart na ca saty api kṣaṇikatve sā\-marthyapratī\-tivyā\-ghā\-taḥ | tathā\- hi kā\-raṇagrā\-hijñā\-nopā\-deyabhū\-tena kā\-ryagrā\-hiṇā\- jñā\-nena tadarpitasaṃskā\-ragarbheṇa asya bhā\-ve asya bhā\-va ity anvayaniścayo janyate | tathā\- kā\-raṇā\-pekṣayā\- bhū\-talakaivalyagrā\-hijñā\-nopā\-deyabhū\-tena kā\-ryā\-pekṣayā\- bhū\-talakaivalyagrā\-hiṇā\- jñā\-nena tadarpitasaṃskā\-ragarbheṇa asyā\-bhā\-ve asyā\-bhā\-va iti vyatirekaniścayo janyate | 
	\pend
      

	  \pstart yad ā\-hur guravaḥ 
	\pend
      
	    
	    \stanza[\smallbreak]
ekā\-vasā\-yasamanantarajā\-tam anyavijñā\-nam anvayavimarśam upā\-dadhā\-ti |&evaṃ tadekavirahā\-nubhavodbhavā\-nyavyā\-vṛttidhī\-ḥ prathayati vyatirekabuddhim ||\&[\smallbreak]


	

	  \pstart evaṃ sati gṛhī\-tā\-nusandhā\-yaka evā\-yaṃ vikalpaḥ | upā\-dā\-nopā\-deyabhū\-takramipratyakṣadvayagṛhī\-tā\-nusandhā\-nā\-t | 
	\pend
      

	  \pstart yad ā\-hā\-laṅkā\-raḥ 
	\pend
      
	    
	    \stanza[\smallbreak]
yadi nā\-maikam adhyakṣaṃ na pū\-rvā\-paravittimat |&adhyakṣadvayasadbhā\-ve prā\-kparā\-vedanaṃ katham ||\footnote{\begin{english}(PVA)\end{english}}\&[\smallbreak]


	

	  \pstart iti ||
	\pend
      

	  \pstart nā\-pi dvitī\-yo 'siddhaprabhedaḥ | sā\-marthyaṃ hi sattvam iti saugatā\-nā\-ṃ sthitir eṣā\- | na caitatprasā\-dhanā\-rtham asmā\-kam idā\-nī\-m eva prā\-rambhaḥ | kiṃ tu yatra pramā\-ṇapratī\-te bī\-jā\-dau vastubhū\-te dharmiṇi pramā\-ṇapratī\-taṃ sā\-marthyaṃ tatra kṣaṇabhaṅgaprasā\-dhanā\-ya | \edlabel{thakur75-72.26}\label{thakur75-72.26} tataś cā\-ṅkurā\-dī\-nā\-ṃ kā\-ryā\-darśanā\-d ā\-hatya sā\-marthyasandehe 'pi paṭupratyakṣaprasiddham sanmā\-tratvam avadhā\-ryam eva | anyathā\- na kvacid api vastumā\-trasyā\-pi pratipattiḥ syā\-t | \edlabel{thakur75-72.29}\label{thakur75-72.29} tasmā\-c chā\-strī\-yasattvalakṣaṇasandehe 'pi paṭupratyakṣabalā\-valambitavastubhā\-ve 'ṅkurā\-dau kā\-ryatvam upalabhyamā\-naṃ bī\-jā\-deḥ sā\-marthyam upasthā\-payatī\-ti nā\-siddhidoṣā\-vakā\-śaḥ ||
	\pend
      

	  \pstart nā\-pi kṣaṇikatve sā\-marthyakṣatiḥ, yato viruddhatā\- syā\-t, kṣaṇikatvaniyataprā\-gbhā\-vitvalakṣaṇakā\-raṇatvayor virodhā\-bhā\-vā\-t, kṣaṇamā\-trasthā\-yiny api sā\-marthyasambhavā\-d iti nā\-dimo virodhaḥ | \edlabel{thakur75-73.5}\label{thakur75-73.5} nā\-pi dvitī\-yo virodhaprabhedaḥ | avastuno vastuno vā\- svā\-kā\-rasya grā\-hyatve 'pi adhyavaseyavastvapekṣayaiva sarvatra prā\-mā\-ṇyapratipā\-danā\-t vastusvabhā\-vasyaiva kṣaṇikatvasya siddhir iti kva virodhaḥ |
	\pend
      

	  \pstart yac ca gṛhyate yac cā\-dhyavasī\-yate te dve 'py anyanivṛttī\- na vastunī\- svalakṣaṇā\-vagā\-hitve 'bhilā\-pasaṃsargā\-nupapatter iti cet |\edlabel{RNA-n-0}\footnote{\label{RNA-n-0}  \begin{english}Cf. \ref{note-2objects-neither-real}.\end{english}} \edlabel{thakur75-73.9}\label{thakur75-73.9} na | adhyavasā\-yasvarū\-pā\-parijñā\-nā\-t | agṛhī\-te 'pi vastuni [ {\corr mā\-nasā\-di}]pravṛttikā\-rakatvaṃ vikalpasyā\-dhyavasā\-yitvam | apratibhā\-se 'pi pravṛttiviṣayī\-kṛtatvam adhyavaseyatvam | etac cā\-dhyavaseyatvaṃ svalakṣaṇasyaiva yujyate, nā\-nyasya, arthakriyā\-rthitvā\-d arthipravṛtteḥ | \edlabel{thakur75-73.12}\label{thakur75-73.12} evaṃ cā\-dhyavasā\-ye svalakṣaṇasyā\-sphuraṇam eva | na ca tasyā\-sphuraṇe 'pi sarvatrā\-viśeṣeṇa pravṛttyā\-kṣepaprasaṅgaḥ, pratiniyatasā\-magrī\-prasū\-tā\-t pratiniyata svā\-kā\-rā\-t pratiniyataśaktiyogā\-t, pratiniyata evā\-tadrū\-paparā\-vṛtte 'pratī\-te 'pi pravṛttisā\-marthyadarśanā\-t | yathā\- sarvasyā\-sattve 'pi bī\-jā\-d aṅkurasyaivotpattiḥ, dṛṣṭasya niyatahetuphalabhā\-vasya pratikṣeptum aśakyatvā\-t | paraṃ bā\-hyenā\-rthena sati pratibandhe prā\-mā\-ṇyam | anyathā\- tv aprā\-mā\-ṇyam iti viśeṣaḥ ||
	\pend
      

	  \pstart tathā\- tṛtī\-yo 'pi pakṣaḥ prayā\-saphalaḥ | nā\-nā\-kā\-lasyaikasya vastuno vastuto 'sambhave 'pi sarvadeśakā\-lavartinor atadrū\-paparā\-vṛttayor eva sā\-dhyasā\-dhanayoḥ pratyakṣeṇa vyā\-ptigrahaṇā\-t | \edlabel{thakur75-73.20}\label{thakur75-73.20} dvividho hi pratyakṣasya viṣayaḥ, grā\-hyo 'dhyavaseyaś ca | sakalā\-tadrū\-paparā\-vṛttavastumā\-traṃ sā\-kṣā\-d asphuraṇā\-t pratyakṣasya grā\-hyo viṣayo mā\- bhū\-t | tadekadeśagrahaṇe tu tanmā\-trayor vyā\-ptiniścā\-yakavikalpajananā\-d adhyavaseyo viṣayo bhavaty eva | kṣaṇagrahaṇe santā\-naniścayavat, rū\-pamā\-tragrahaṇe rū\-parasagandhasparśā\-tmakaghaṭaniścayavac ca | anyathā\- sarvā\-numā\-nocchedaprasaṅgā\-t ||
	\pend
      

	  \pstart tathā\- hi vyā\-ptigrahaḥ sā\-mā\-nyayoḥ, viśeṣayoḥ, sā\-mā\-nyaviśiṣṭaviśeṣayoḥ viśeṣaviśiṣṭasā\-mā\-nyayor veti vikalpā\-ḥ |
	\pend
      

	  \pstart nā\-dyo vikalpaḥ, sā\-mā\-nyasya bā\-dhyatvā\-t | abā\-dhyatve 'py adṛśyatvā\-t | dṛśyatve 'pi puruṣā\-rthā\-nupayogitayā\- tasyā\-numeyatvā\-yogā\-t | nā\-py anumitā\-t sā\-mā\-nyā\-d viśeṣā\-numā\-nam | sā\-mā\-nyasarvaviśeṣayor vakṣyamā\-ṇa nyā\-yena pratibandhapratipatter ayogā\-t |
	\pend
      

	  \pstart nā\-pi dvitī\-yaḥ | viśeṣasyā\-nanugā\-mitvā\-t |
	\pend
      

	  \pstart antime tu vikalpadvaye sā\-mā\-nyā\-dhā\-ratayā\- dṛṣṭa eva viśeṣaḥ sā\-mā\-nyasya viśeṣyo viśeṣaṇaṃ vā\- kartavyaḥ | adṛṣṭa eva vā\- deśakā\-lā\-ntaravartī\- | yadvā\- dṛṣṭā\-dṛṣṭā\-tmako atadrū\-paparā\-vṛttaḥ sarvo viśeṣaḥ | 
	\pend
      

	  \pstart na prathamaḥ pakṣo 'nanugā\-mitvā\-t | nā\-pi dvitī\-yaḥ, adṛṣṭatvā\-t | na ca tṛtī\-yaḥ, prastutaikaviśeṣadarśane 'pi deśakā\-lā\-ntaravartinā\-ṃ viśeṣā\-ṇā\-m adarśanā\-t |
	\pend
      

	  \pstart atha teṣā\-ṃ sarveṣā\-m eva viśeṣā\-ṇā\-ṃ sadṛśatvā\-t sadṛśasā\-magrī\-prasū\-tatvā\-t sadṛśakā\-ryakā\-ritvā\-d iti pratyā\-sattyā\- ekaviśeṣagrā\-hakaṃ pratyakṣam atadrū\-paparā\-vṛttamā\-tre niścayaṃ janayad atadrū\-paparā\-vṛttaviśeṣamā\-trasya vyavasthā\-pakam | \edlabel{thakur75-74.9}\label{thakur75-74.9} yathaikasā\-magrī\-pratibaddharū\-pamā\-tragrā\-hakaṃ pratyakṣaṃ ghaṭe niścayaṃ janayad ghaṭagrā\-hakaṃ vyavasthā\-pyate | anyathā\- ghaṭo 'pi ghaṭasantā\-no 'pi pratyakṣato na sidhyet, sarvā\-tmanā\- grahaṇā\-bhā\-vā\-t | \edlabel{thakur75-74.12}\label{thakur75-74.12} tadekadeśagrahaṇaṃ tv atadrū\-paparā\-vṛtte 'py aviśiṣṭam | yady evam anenaiva krameṇa sarvasya viśeṣasya viśeṣaṇaviśeṣyabhā\-vavad vyā\-ptipratipattir apy astu | \edlabel{thakur75-74.13}\label{thakur75-74.13} tat kimarthaṃ nā\-nā\-kā\-lam ekam akṣaṇikam abhyupagantavyaṃ, yena kṣaṇikatvasā\-dhanasya viruddhatvaṃ syā\-d iti na kaścid virodhaprabhedaprasaṅgaḥ ||
	\pend
      

	  \pstart na cā\-yam anaikā\-ntiko 'pi hetuḥ, pū\-rvoktakrameṇa sā\-dharmyadṛṣṭā\-nte prasaṅgaviparyayahetubhyā\-m anvayarū\-pavyā\-pteḥ prasā\-dhanā\-t | \edlabel{thakur75-74.17}\label{thakur75-74.17} nanu yadi prasaṅgaviparyayahetudvayabalato ghaṭe dṛṣṭā\-nte kṣaṇabhaṅgaḥ sidhyet tadā\- sattvasya niyamena kṣaṇikatvena vyā\-ptisiddher anaikā\-ntikatvaṃ na syā\-d iti yuktam | kevalam idam evā\-sambhavi | tathā\- hi śakto 'pi ghaṭaḥ krami sahakā\-ryapekṣayā\- kramikā\-ryaṃ kariṣyati |
	\pend
      

	  \pstart na caitad vaktavyam, samartho 'rthaḥ svarū\-peṇa karoti, svarū\-paṃ ca sarvadā\-stī\-ty anupakā\-riṇi sahakā\-riṇy apekṣā\- na yujyata iti | saty api svarū\-peṇa kā\-rakatve sā\-marthyā\-bhā\-vā\-t kathaṃ karotu | sahakā\-risā\-kalyaṃ hi sā\-marthyam, tadvaikalyaṃ cā\-sā\-marthyam | na ca tayor ā\-virbhā\-vatirobhā\-vā\-bhyā\-ṃ tadvataḥ kā\-cit kṣatiḥ, tasya tā\-bhyā\-m anyatvā\-t | tasmā\-d arthaḥ samartho 'pi syā\-t, na ca karotī\-ti sandigdhavyatirekaḥ prasaṅgahetuḥ ||
	\pend
      

	  \pstart atrocyate | bhavatu tā\-vat sahakā\-risā\-kalyam eva sā\-marthyam | tathā\-pi so 'pi tā\-vad bhā\-vaḥ svarū\-peṇa kā\-rakaḥ | \edlabel{thakur75-74.28}\label{thakur75-74.28} tasya ca yā\-dṛśaś caramakṣaṇe 'kṣepakriyā\-dharmā\- svabhā\-vas tā\-dṛśa eva cet | \edlabel{thakur75-74.29}\label{thakur75-74.29} prathamakṣaṇe tadā\- tadā\-pi prasahya kurvā\-ṇo brahmaṇā\-py anivā\-ryaḥ | na ca so 'py akṣepakriyā\-dharmā\- svabhā\-vaḥ sā\-kalye sati jā\-to bhā\-vā\-d bhinna evā\-bhidhā\-tuṃ śakyaḥ, bhā\-vasyā\-kartṛtvaprasaṅgā\-t | evaṃ yā\-vad yā\-vad dharmā\-ntaraparikalpas tā\-vat tā\-vad udā\-sī\-no bhā\-vaḥ | \edlabel{thakur75-75.2}\label{thakur75-75.2} tasmā\-d yadrū\-pam ā\-dā\-ya svarū\-peṇā\-pi janayatī\-ty ucyate tasya prā\-g api bhā\-ve katham ajaniḥ kadā\-cit | akṣepakriyā\-pratyanī\-kasvabhā\-vasya vā\- prā\-cyasya paścā\-d anuvṛttau kathaṃ kadā\-cid api kā\-ryasambhavaḥ ||
	\pend
      

	  \pstart nanu yadi sa evaikaḥ kartā\- syā\-d yuktam etat | kiṃtu sā\-magrī\- janikā\- | tataḥ sahakā\-ryantaravirahavelā\-yā\-ṃ balī\-yaso 'pi na kā\-ryaprasava iti kim atra viruddham | na hi bhā\-vaḥ svarū\-peṇa karotī\-ti svarū\-peṇaiva karoti , sahakā\-risahitā\-d eva tataḥ kā\-ryotpattidarśanā\-t | tasmā\-d vyā\-ptivat kā\-ryakā\-raṇabhā\-vo 'py ekatrā\-nyayogavyavacchedenā\-nyatrā\-yogavyavacchedenā\-vaboddhavyaḥ, tathaiva laukikaparī\-kṣakā\-ṇā\-ṃ saṃpratipatter iti ||
	\pend
      

	  \pstart atrocyate | yadā\- militā\-ḥ santaḥ kā\-ryaṃ kurvate tadaikā\-rthakaraṇalakṣaṇaṃ sahakā\-ritvam eṣā\-m astu | ko niṣeddhā\- | militair eva tu tatkā\-ryaṃ kartavyam iti kuto labhyate | pū\-rvā\-parayor ekasvabhā\-vatvā\-d bhā\-vasya sarvadā\- jananā\-jananayor anyataraniyama prasaṅgasya durvā\-ratvā\-t | tasmā\-t sā\-magrī\- janikā\-, naikaṃ janakam iti sthiravā\-dinā\-ṃ manorathasyā\-py aviṣayaḥ |
	\pend
      

	  \pstart dṛśyate tā\-vad evam iti cet | dṛśyatā\-m | kiṃ tu pū\-rvasthitā\-d eva sā\-magrī\-madhyapraviṣṭā\-d bhā\-vā\-t kā\-ryotpattir anyasmā\-d eva vā\- viśiṣṭā\-d bhā\-vā\-d utpannā\-d iti vivā\-dapadam | tatra prā\-g api sambhave sarvadaiva kā\-ryotpattir na vā\- kadā\-cid apī\-ti virodham asamā\-dhā\-ya cakṣuṣī\- nimī\-lya tata eva kā\-ryotpattidarśanā\-d iti sā\-dhyā\-nuvā\-damā\-trapravṛttaḥ kṛpā\-m arhatī\-ti |
	\pend
      

	  \pstart na ca pratyabhijñā\- balā\-d ekatvasiddhiḥ | tatpauruṣasya lū\-napunarjā\-takeśanakhā\-dā\-v apy upalambhato nirdalanā\-t | lakṣaṇabhedasya ca darśayitum aśakyatvā\-t | sthirasiddhi dū\-ṣaṇe cā\-smā\-bhiḥ prapañcato nirastatvā\-t | \edlabel{thakur75-75.22}\label{thakur75-75.22} tasmā\-t sā\-kṣā\-t kā\-ryakā\-raṇabhā\-vā\-pekṣayobhayatrā\-py anyayogavyavacchedaḥ | vyā\-ptau tu sā\-kṣā\-t paramparayā\- kā\-raṇamā\-trā\-pekṣayā\- kā\-raṇe vyā\-pake 'yogavyavacchedaḥ | kā\-rye vyā\-pye 'nyayogavyavacchedaḥ | tathā\- tad atatsvabhā\-ve vyā\-pake 'yogavyavacchedaḥ | tatsvabhā\-ve ca vyā\-pye 'nyayogavyavacchedaḥ | vikalpā\-rū\-ḍharū\-pā\-pekṣayā\- vyā\-ptau dvividham avadhā\-raṇam |
	\pend
      

	  \pstart nanu yadi pū\-rvā\-parakā\-layor ekasvabhā\-vo bhā\-vaḥ sarvadā\- janakatvenā\-janakatvena vā\- vyā\-pta upalabdhaḥ syā\-t, tadā\-yaṃ prasaṅgaḥ saṅgacchate | na ca kṣaṇabhaṅgavā\-dinā\- pū\-rvā\-parakā\-layor ekaḥ kaścid upalabdha iti cet | \edlabel{thakur75-76.1}\label{thakur75-76.1} tad etad atigrā\-myam | tathā\- hi pū\-rvā\-parakā\-layor ekasvabhā\-vatve satī\-ty asyā\-yam arthaḥ, parakā\-labhā\-vī\- janako yaḥ svabhā\-vo bhā\-vasya sa eva yadi pū\-rvakā\-labhā\-vī\-, pū\-rvakā\-labhā\-vī\- vā\- yo 'janakaḥ svabhā\-vaḥ sa eva yadi parakā\-labhā\-vī\-, tadopalabdham eva jananam ajananaṃ vā\- syā\-t | tathā\- ca sati siddhayor eva svabhā\-vayor ekatvā\-rope siddham eva jananam ajananaṃ vā\-sajyata iti |
	\pend
      

	  \pstart nanu kā\-ryam eva sahakā\-riṇam apekṣate , na tu kā\-ryotpattihetuḥ | yasmā\-d dvividhaṃ sā\-marthyaṃ nijam ā\-gantukaṃ ca sahakā\-ryantaram | tato 'kṣaṇikasyā\-pi kramavatsahakā\-rinā\-nā\-tvā\-d api kramavatkā\-ryanā\-nā\-tvopapatter aśakyaṃ bhā\-vā\-nā\-ṃ pratikṣaṇam anyatvam upapā\-dayitum iti cet | \edlabel{thakur75-76.9}\label{thakur75-76.9} ucyate | bhavatu tā\-van nijā\-gantukabhedena dvividhaṃ sā\-marthyam | tathā\-pi tat prā\-tisvikaṃ vastusvalakṣaṇam arthakriyā\-dharmakam avaśyam abhyupagantavyam | tat kiṃ prā\-g api paścā\-d eva veti vikalpya yad dū\-ṣaṇam udī\-ritaṃ tatra kim uktam aneneti na pratī\-maḥ | \edlabel{thakur75-76.12}\label{thakur75-76.12} yat tu kā\-ryeṇaiva sahakā\-riṇo 'pekṣyanta ity upaskṛtaṃ tad api nirupayogam. yadi hi kā\-ryam eva svajanmani svatantraṃ syā\-d yuktam etat | kevalam evaṃ sati sahakā\-risā\-kalyasā\-marthyakalpanam aphalam | svā\-tantryā\-d eva hi kā\-ryaṃ kā\-dā\-citkaṃ bhaviṣyati | tathā\- ca sati santo hetavaḥ sarvathā\- 'samarthā\-ḥ | asat tu kā\-ryaṃ svatantram iti viśuddhā\- buddhiḥ |
	\pend
      

	  \pstart atha kā\-ryasyaivā\-yam aparā\-dho yad idaṃ samarthe kā\-raṇe saty api kadā\-cin nopapadyata iti cet | na tat tarhi tatkā\-ryaṃ, svā\-tantryā\-t | \edlabel{thakur75-76.18}\label{thakur75-76.18} yad bhā\-ṣyam,
	\pend
      
	    
	    \stanza[\smallbreak]
sarvā\-vasthā\-samā\-ne 'pi kā\-raṇe yady akā\-ryatā\- |&svatantraṃ kā\-ryam evaṃ syā\-n na tatkā\-ryaṃ tathā\- sati || \footnote{\begin{english}(PV II 396)\end{english}}\&[\smallbreak]


	

	  \pstart atha na tadbhā\-ve bhavatī\-ti tatkā\-ryam ucyate, kiṃtu tadabhā\-ve na bhavaty eveti vyatirekaprā\-dhā\-nyā\-d iti cet |
	\pend
      

	  \pstart na | yadi hi svayaṃ bhavan bhā\-vayed eva hetuḥ svakā\-ryam , tadā\- tadabhā\-vaprayukto 'syā\-bhā\-va iti pratī\-tiḥ syā\-t | no cet, yathā\- kā\-raṇe saty api kā\-ryaṃ svā\-tantryā\-n na bhavati, tathā\- tadabhā\-ve 'pi svā\-tantryā\-d eva na bhū\-tam iti śaṅkā\- kena nivā\-ryeta |
	\pend
      

	  \pstart yad Bhā\-ṣyam
	\pend
      
	    
	    \stanza[\smallbreak]
tadbhā\-ve 'pi na bhā\-vaś ced abhā\-ve 'bhā\-vitā\- kutaḥ |&tadabhā\-vaprayukto 'sya so 'bhā\-va iti tat kutaḥ || \footnote{\begin{english}(PVA II 411)\end{english}}\&[\smallbreak]


	

	  \pstart tasmā\-d yathaiva tadabhā\-ve niyamena na bhavati tathaiva tadbhā\-ve niyamena bhaved eva | abhavac ca na tatkā\-raṇatā\-m ā\-tmanaḥ kṣamate |
	\pend
      

	  \pstart yac coktaṃ prathamakā\-ryotpā\-danakā\-le hi uttarakā\-ryotpā\-danasvabhā\-vaḥ, ataḥ prathamakā\-la evā\-śeṣā\-ṇi kā\-ryā\-ṇi kuryā\-d iti, tad idaṃ mā\-tā\- me bandhyetyā\-divat svavacanavirodhā\-d ayuktam | yo hi uttarakā\-ryajananasvabhā\-vaḥ sa katham ā\-dau kā\-ryaṃ kuryā\-t | na tarhi tatkā\-ryakaraṇasvabhā\-vaḥ | na hi nī\-lotpā\-danasvabhā\-vaḥ pī\-tā\-dikam api karotī\-ti |
	\pend
      

	  \pstart artocyate | sthirasvabhā\-vatve hi bhā\-vasyottarakā\-lam evedaṃ kā\-ryaṃ na pū\-rvakā\-lam iti kuta etat | tadabhā\-vā\-c ca kā\-raṇam apy uttarakā\-ryakaraṇasvabhā\-vam ity api kutaḥ | 
	\pend
      

	  \pstart kiṃ kurmaḥ | uttarakā\-lam eva tasya janmeti cet | astu, sthiratve tad anupapadyamā\-nam, asthiratā\-m ā\-diśatu |
	\pend
      

	  \pstart sthiratve 'py eṣa eva svabhā\-vas tasya yad uttarakṣaṇa eva karotī\-ti cet | hatedā\-nī\-ṃ pramā\-ṇapratyā\-śā\- | dhū\-mā\-d atrā\-gnir ity atrā\-pi svabhā\-va evā\-sya yad idā\-nī\-m atra niragnir api dhū\-ma iti vaktuṃ śakyatvā\-t | tasmā\-t pramā\-ṇasiddhe svabhā\-vā\-valambanam | na tu svabhā\-vā\-valambanena pramā\-ṇavyā\-lopaḥ | 
	\pend
      

	  \pstart tasmā\-d yadi kā\-raṇasyottarakā\-ryakā\-rakatvam abhyupagamya kā\-ryasya prathamakṣaṇabhā\-vitvam ā\-sajyate, syā\-t svavacanavirodhaḥ | yadā\- tu kā\-raṇasya sthiratve kā\-ryasyottarakā\-latvam evā\-saṅgatam ataḥ kā\-raṇasyā\-py uttarakā\-ryajanakatvaṃ vastuto 'sambhavi tadā\- prasaṅgasā\-dhanam idam | jananavyavahā\-ragocaratvaṃ hi jananena vyā\-ptam iti prasā\-dhitam | uttarakā\-ryajananavyavahā\-ragocaratvaṃ ca tvad abhyupagamā\-t prathamakā\-ryakaraṇakā\-la eva ghaṭe dharmiṇi siddhaṃ | atas tanmā\-trā\-nubandhina uttarā\-bhimatasya kā\-ryasya prathame kṣaṇe 'sambhavā\-d eva prasaṅgaḥ kriyate | 
	\pend
      

	  \pstart na hi nī\-lakā\-rake 'pi pī\-takā\-rakatvā\-rope pī\-tasambhavaprasaṅgaḥ svavacanavirodho nā\-ma | 
	\pend
      

	  \pstart tad evaṃ śaktaḥ sahakā\-ryanapekṣitatvā\-d jananena vyā\-ptaḥ | ajanayaṃś ca śaktā\-śaktatvaviruddhadharmā\-dhyā\-sā\-d bhinna eva ||
	\pend
      

	  \pstart nanu bhavatu prasaṅgaviparyayabalā\-d ekakā\-ryaṃ prati śaktā\-śaktatvalakṣaṇaviruddhadharmā\-dhyā\-saḥ | tathā\-pi na tato bhedaḥ sidhyati | 
	\pend
      

	  \pstart tathā\- hi bī\-jam aṅkurā\-dikaṃ kurvad yadi yenaiva svabhā\-venā\-ṅkuraṃ karoti tenaiva kṣityā\-dikaṃ, tadā\- kṣityā\-dī\-nā\-m apy aṅkurasvā\-bhā\-vyā\-pattiḥ | nā\-nā\-svabhā\-vatvena tu kā\-rakatve svabhā\-vā\-nā\-m anyonyā\-bhā\-vā\-vyabhicā\-ritvā\-d ekatra bhā\-vā\-bhā\-vau parasparaviruddhau syā\-tā\-m ity ekam api bī\-jaṃ bhidyeta |
	\pend
      

	  \pstart evaṃ pradī\-po 'pi tailakṣayavarti dā\-hā\-dikam | 
	\pend
      

	  \pstart tathā\- pū\-rvarū\-pam apy uttararū\-parasagandhā\-dikam anekaiḥ svabhā\-vaiḥ parikaritaṃ karoti | 
	\pend
      

	  \pstart teṣā\-ṃ ca svabhā\-vā\-nā\-m anyonyā\-bhā\-vā\-vyabhicā\-rā\-d viruddhā\-nā\-ṃ yoge pradī\-pā\-dikaṃ bhidyeta | na ca bhidyate | tan na viruddhadharmā\-dhyā\-so bhedakaḥ | 
	\pend
      

	  \pstart tathā\- bī\-jasyā\-ṅkuraṃ prati kā\-rakatvaṃ gardabhā\-dikaṃ praty akā\-rakatvam iti kā\-rakatvā\-kā\-rakatve 'pi viruddhau dharmau | na ca tadyoge 'pi bī\-jabhedaḥ | 
	\pend
      

	  \pstart tad evaṃ ekatra bī\-je pradī\-pe rū\-pe ca vipakṣe paridṛśyamā\-naḥ śaktā\-śaktatvā\-dir viruddhadharmā\-dhyā\-so na ghaṭā\-der bhedaka iti |
	\pend
      

	  \pstart atra brū\-maḥ | bhavatu tā\-vad bī\-jā\-dī\-nā\-m anekakā\-ryakā\-ritvā\-d dharmabhū\-tā\-nekasvabhā\-vabhedaḥ, tathā\-pi kaḥ prastā\-vo viruddhadharmā\-dhyā\-sasya | svabhā\-vā\-nā\-ṃ hy anyonyā\-bhā\-vā\-vyabhicā\-re bhedaḥ prā\-ptā\-vasaro na virodhaḥ | virodhas tu yadvidhā\-ne yanniṣedho yanniṣedhe ca yadvidhā\-naṃ tayor ekatra dharmiṇi parasparaparihā\-rasthitatayā\- syā\-t | tad atraikaḥ svabhā\-vaḥ svā\-bhā\-vena viruddho yukto bhā\-vā\-bhā\-vavat | na tu svabhā\-vā\-ntareṇa ghaṭatvavastutvavat | 
	\pend
      

	  \pstart evam aṅkurā\-dikā\-ritvaṃ tadakā\-ritvena viruddhaṃ, na punar vastvantarakā\-ritvena | pratyakṣavyā\-pā\-raś cā\-tra yathā\- nā\-nā\-dharmair adhyā\-sitaṃ bhā\-vam abhinnaṃ vyavasthā\-payati tathā\- tatkā\-ryakā\-riṇaṃ kā\-ryā\-ntarā\-kā\-riṇaṃ ca | 
	\pend
      

	  \pstart tad yadi pratiyogitvā\-bhā\-vā\-d anyonyā\-bhā\-vā\-vyabhicā\-riṇā\-v api svabhā\-vā\-v aviruddhau tatkā\-rakatvā\-nyā\-kā\-rakatve vā\- viṣayabhedā\-d aviruddhe tat kim ā\-yā\-tam, ekakā\-ryaṃ prati śaktā\-śaktatvayoḥ parasparapratiyoginor viruddhayor dharmayoḥ | etayor api punar avirodhe virodho nā\-ma dattajalā\-ñ–jaliḥ ||
	\pend
      

	  \pstart bhavatu tarhy ekakā\-ryā\-pekṣayaiva sā\-marthyā\-sā\-marthyayor virodhaḥ | kevalaṃ yathā\- tad eva kā\-ryaṃ prati kvacid deśe śaktir deśā\-ntare cā\-śaktir iti deśabhedā\-d aviruddhe śaktyaśaktī\- tathaikatraiva kā\-rye kā\-labhedā\-d apy aviruddhe | yathā\- pū\-rvaṃ niṣkriyaḥ sphaṭikaḥ sa eva paścā\-t sakriya iti cet |
	\pend
      

	  \pstart ucyate | na hi vayaṃ paribhā\-ṣā\-mā\-trā\-d ekatra kā\-rye deśabhedā\-d aviruddhe śaktyaśaktī\- brū\-maḥ, kiṃ tu virodhā\-bhā\-vā\-t | taddeśakā\-ryakā\-ritvaṃ hi taddeśakā\-ryā\-kā\-ritvena viruddham, na punar deśā\-ntare tatkā\-ryā\-kā\-ritvenā\-nyakā\-ryakā\-ritvena vā\- ||
	\pend
      

	  \pstart yady evaṃ tatkā\-lakā\-ryakā\-ritvaṃ tatkā\-lakā\-ryā\-kā\-ritvena viruddham | na punaḥ kā\-lā\-ntare tatkā\-ryā\-kā\-ritvenā\-nyakā\-ryakā\-ritvena vā\- | tat kathaṃ kā\-labhede 'pi virodha iti cet |
	\pend
      

	  \pstart ucyate | dvayor hi dharmayor ekatra dharmiṇy anavasthitiniyamaḥ parasparaparihā\-rasthiti lakṣaṇo virodhaḥ | sa ca sā\-kṣā\-tparasparapratyanī\-katayā\- bhā\-vā\-bhā\-vavad vā\- bhavet, ekasya vā\- niyamena pramā\-ṇā\-ntareṇa bā\-dhanā\-n nityatvasattvavad vā\- bhaved iti na kaścid arthabhedaḥ | tad atraikadharmiṇi tatkā\-lakā\-ryakā\-ritvā\-dhā\-re kā\-lā\-ntare tatkā\-ryā\-kā\-ritvasyā\- nyakā\-ryakā\-ritvasya vā\- niyamena pramā\-ṇā\-ntareṇa bā\-dhanā\-d virodhaḥ | 
	\pend
      

	  \pstart tathā\- hi yatraiva dharmiṇi tatkā\-lakā\-ryakā\-ritvam upalabdhaṃ na tatraiva kā\-lā\-ntare tatkā\-ryā\-kā\-ritvam anyakā\-ryakā\-ritvaṃ vā\- brahmaṇā\-py upasaṃhartuṃ śakyate , yenā\-nayor avirodhaḥ syā\-t | kṣaṇā\-ntare
	\pend
      

	  \pstart kathitaprasaṅgaviparyayahetubhyā\-m avaśyambhā\-vena dharmibhedaprasā\-dhanā\-t ||
	\pend
      

	  \pstart na ca pratyabhijñā\-nā\-d ekatvasiddhiḥ, tatpauruṣasya nirmū\-litatvā\-t | ata eva vajro 'pi pakṣakukṣau nikṣiptaḥ | katham asau sphaṭiko varā\-kaḥ kā\-labhedenā\-bhedaprasā\-dhanā\-ya dṛṣṭā\-ntī\-bhavitum arhati | 
	\pend
      

	  \pstart na caivaṃ samā\-nakā\-lakā\-ryā\-ṇā\-ṃ deśabhede 'pi dharmibhedo yukto bhedaprasā\-dhaka pramā\-ṇā\-bhā\-vā\-t indriyapratyakṣeṇa nirastavibhramā\-śaṅkenā\-bhedaprasā\-dhanā\-c ceti na kā\-labhede 'pi śaktyaśaktyor virodhaḥ svasamayamā\-trā\-d apahastayituṃ śakyaḥ, samayapramā\-ṇayor apravṛtter iti |
	\pend
      

	  \pstart tasmā\-t sarvatra viruddhadharmā\-dhyā\-sasiddhir eva bhedasiddhiḥ | vipratipannaṃ prati tu viruddhadharmā\-dhyā\-sā\-d bhedavyavahā\-raḥ sā\-dhyate ||
	\pend
      

	  \pstart nanu tathā\-pi sattvam idam anaikā\-ntikam evā\-sā\-dhā\-raṇatvā\-t sandigdhavyatirekitvā\-d vā\- | yathā\- hī\-daṃ kramā\-kramanivṛttā\-v akṣaṇikā\-n nivṛttaṃ, tathā\- sā\-pekṣatvā\-napekṣatvayor ekatvā\-nekatvayor api vyā\-pakayor nivṛttau kṣaṇikā\-d api | 
	\pend
      

	  \pstart tathā\- hi upasarpaṇapratyayena devadattakarapallavā\-dinā\- sahacaro bī\-jakṣaṇaḥ pū\-rvasmā\-d eva puñjā\-t samartho jā\-to 'napekṣa ā\-dyā\-tiśayasya janaka iṣyate | 
	\pend
      

	  \pstart tatra ca samā\-nakuśū\-lajanmasu bahuṣu bī\-jasantā\-neṣu kasmā\-t kiñcid eva bī\-jaṃ paramparayā\-ṅkurotpā\-dā\-nuguṇam upajanayati bī\-jakṣaṇaṃ, nā\-nye bī\-jakṣaṇā\- bhinnasantā\-nā\-ntaḥpā\-tinaḥ | na hy upasarpaṇapratyayā\-t prā\-g eva teṣā\-ṃ samā\-nā\-samā\-nasantā\-navartinā\-ṃ bī\-jakṣaṇā\-nā\-ṃ kaścit paramparā\-tiśayaḥ | 
	\pend
      

	  \pstart athopasarpaṇapratyayā\-t prā\-ṅ na tatsantā\-navartino 'pi janayanti, paramparayā\-py aṅkurotpā\-dā\-nuguṇaṃ bī\-jakṣaṇaṃ bī\-jamā\-trajananā\-t teṣā\-m | kasyacid eva bī\-jakṣaṇasyopasarpaṇapratyayasahabhuva ā\-dyā\-tiśayotpā\-daḥ | hanta tarhi tadabhā\-ve saty utpanno 'pi janayed eva | 
	\pend
      

	  \pstart tathā\- kevalā\-nā\-ṃ vyabhicā\-rasambhavā\-d ā\-dyā\-tiśayotpā\-dakam aṅkuraṃ vā\- prati kṣityā\-dī\-nā\-ṃ parasparā\-pekṣā\-ṇā\-m evotpā\-dakatvam akā\-menā\-pi svī\-kartavyam | 
	\pend
      

	  \pstart ato na tā\-vad anapekṣā\- kṣaṇikasya sambhavinī\- | nā\-py apekṣā\- yujyate, samasamayakṣaṇayoḥ savyetaragobiṣā\-ṇayor ivopakā\-ryopakā\-rakabhā\-vā\-yogā\-d iti nā\-siddhaḥ prathamo vyā\-pakā\-bhā\-vaḥ |
	\pend
      

	  \pstart api cā\-ntyo bī\-jakṣaṇo 'napekṣo 'ṅkurā\-dikaṃ kurvan yadi yenaiva rū\-peṇā\-ṅkuraṃ karoti tenaiva kṣityā\-dikaṃ, tadā\- kṣityā\-dī\-nā\-m apy aṅkurasvā\-bhā\-vyā\-pattir abhinnakā\-raṇatvā\-d iti na tā\-vad ekatvasambhavaḥ ||
	\pend
      

	  \pstart nanu rū\-pā\-ntareṇa karoti | tathā\- hi bī\-jasyā\-ṅkuraṃ praty upā\-dā\-natvam | kṣityā\-dikaṃ tu prati sahakā\-ritvam | yady evaṃ, sahakā\-ritvopā\-dā\-natve kim ekaṃ tattvaṃ nā\-nā\- vā\- | ekaṃ cet, kathaṃ rū\-pā\-ntareṇa janakam | nā\-nā\-tve tv anayor bī\-jā\-d bhedo 'bhedo vā\- | bhede kathaṃ bī\-jasya janakatvaṃ tā\-bhyā\-m evā\-ṅkurā\-dī\-nā\-m utpatteḥ | abhede vā\- kathaṃ bī\-jasya na nā\-nā\-tvaṃ bhinnatā\-dā\-tmyā\-t, etayor vaikatvam ekatā\-dā\-tmyā\-t |
	\pend
      

	  \pstart yady ucyeta kṣityā\-dau janayitavye tadupā\-dā\-naṃ pū\-rvam eva kṣityā\-di bī\-jasya rū\-pā\-ntaram iti | na tarhi bī\-jaṃ tadanapekṣaṃ kṣityā\-dī\-nā\-ṃ janakam | tadanapekṣatve teṣā\-m aṅkurā\-d bhedā\-nupapatteḥ | na cā\-nupakā\-rakā\-ṇy apekṣanta iti tvayaivotkam | na ca kṣaṇasyopakā\-ra sambhavo 'nyatra jananā\-t, tasyā\-bhedyatvā\-d ity anekatvam api nā\-stī\-ti dvitī\-yo 'pi vyā\-pakā\-bhā\-vo nā\-siddhaḥ | tasmā\-d asā\-dhā\-raṇā\-naikā\-ntikatvaṃ gandhavattvavad iti |
	\pend
      

	  \pstart yadi manyetā\-nupakā\-rakā\- api bhavanti sahakā\-riṇo 'pekṣaṇī\-yā\-ś ca kā\-ryeṇā\-nuvihitabhā\-vā\-bhā\-vā\-c ca sahakā\-ryakaraṇā\-c ca |
	\pend
      

	  \pstart nanv anena krameṇā\-kṣaṇiko 'pi bhā\-vo 'nupakā\-rakā\-n api sahakā\-riṇaḥ kramavataḥ kramavat kā\-ryeṇā\-nukṛtā\-nvayavyatirekā\-n apekṣiṣyate | kariṣyate ca kramavatsahakā\-rivaśaḥ krameṇa kā\-ryā\-ṇī\-ti vyā\-pakā\-nupalabdher asiddheḥ sandigdhavyatirekam anaikā\-ntikaṃ sattvaṃ kṣaṇikatvasiddhā\-v iti |
	\pend
      

	  \pstart atra brū\-maḥ | kī\-dṛśaṃ punar apekṣā\-rtham ā\-dā\-ya kṣaṇike sā\-pekṣā\-napekṣatvanivṛttir ucyate | kiṃ sahakā\-riṇam apekṣata iti sahakā\-riṇā\-syopakā\-raḥ karttavyaḥ | atha pū\-rvā\-vasthitasyaiva bī\-jā\-deḥ sahakā\-riṇā\- saha sambhū\-yakaraṇam | yadvā\- pū\-rvā\-vasthitasyety anapekṣya militā\-vasthasya karaṇamā\-tram apekṣā\-rthaḥ | atra prathamapakṣasyā\-sambhavā\-d anapekṣaiva kṣaṇikasya, katham ubhayavyā\-vṛttiḥ | 
	\pend
      

	  \pstart yady anapekṣaḥ kṣaṇikaḥ , kimity upasarpaṇapratyayā\-bhā\-ve 'pi na karoti | karoty eva yadi syā\-t | svayam asambhavī\- tu kathaṃ karotu | atha tad vā\- tā\-dṛg vā\-sī\-d iti na kaścid viśeṣaḥ | tatas tā\-dṛk svabhā\-vasambhave 'py akaraṇaṃ sahakā\-riṇi nirapekṣā\-n na kṣamata iti cet |
	\pend
      

	  \pstart asambaddham etat | varṇasaṃsthā\-nasā\-mye 'py akartus tatsvabhā\-vatā\-yā\- virahā\-t | sa cā\-dyā\-tiśayajanakatvalakṣaṇaḥ svabhā\-vaviśeṣo na samā\-nā\-samā\-nasantā\-navartiṣu bī\-jakṣaṇeṣu sarveṣv eva sambhavī\- | kiṃ tu keṣucid eva karmakarakarapallavasahacareṣu |
	\pend
      

	  \pstart nanv ekatra kṣetre niṣpattilavanā\-dipū\-rvakam ā\-nī\-yaikatra kuśū\-le kṣiptā\-ni sarvā\-ṇy eva bī\-jā\-ni sā\-dhā\-raṇarū\-pā\-ṇy eva pratī\-yante | tat kutastyo 'yam ekabī\-jasambhavī\- viśeṣo 'nyeṣā\-ṃ iti cet | 
	\pend
      

	  \pstart ucyate | kā\-raṇam khalu sarvatra kā\-rye dvividham | dṛṣṭam adṛṣṭaṃ ceti | sarvā\-stikaprasiddham etat | tataḥ pratyakṣaparokṣasahakā\-ripratyayasā\-kalyam asarvavidā\- pratyakṣato na śakyaṃ pratipattum | tato bhaved api kā\-raṇasā\-magrī\-śaktibhedā\-t tā\-dṛśaḥ svabhā\-vabhedaḥ keṣā\-ñcid eva bī\-jakṣaṇā\-nā\-ṃ yena ta eva bī\-jakṣaṇā\- ā\-dyā\-tiśayam aṅkuraṃ vā\- paramparayā\- janayeyuḥ | nā\-nye ca bī\-jakṣaṇā\-ḥ |
	\pend
      

	  \pstart nanu yeṣū\-pasarpaṇapratyayasahacareṣu svakā\-raṇaśaktibhedā\-d ā\-dyā\-tiśaya janakatvalakṣaṇo viśeṣaḥ sambhā\-vyate sa tatrā\-vaśyam astī\-ti kuto labhyam iti cet | 
	\pend
      

	  \pstart aṅkurotpā\-dā\-d anumitā\-d ā\-dyā\-tiśayalakṣaṇā\-t kā\-ryā\-d iti brū\-maḥ | kā\-raṇā\-nupalabdhes tarhi tadabhā\-va eva bhaviṣyatī\-ti cet | na | dṛśyā\-dṛśyasamudā\-yasya kā\-raṇasyā\-darśane 'py abhā\-vā\-siddheḥ kā\-raṇā\-nupalabdheḥ sandigdhā\-siddhatvā\-t |
	\pend
      

	  \pstart tad ayam arthaḥ 
	\pend
      

	  \pstart pā\-ṇisparśavataḥ kṣaṇasya na bhidā\- bhinnā\-nyakā\-lakṣaṇā\-d bhedo veti matadvaye mitibalaṃ yasyā\-sty asau jitvaraḥ |
	\pend
      

	  \pstart tatraikasya balaṃ nimittavirahaḥ kā\-ryā\-ṅgam anyasya vā\- sā\-magrī\- tu na sarvathekṣaṇasahā\- kā\-ryaṃ tu mā\-nā\-nugam || 
	\pend
      

	  \pstart iti |
	\pend
      

	  \pstart tad evaṃ nopakā\-ro 'pekṣā\-rtha ity anapekṣaiva kṣaṇikasya sahakā\-riṣu nobhayavyā\-vṛttiḥ ||
	\pend
      

	  \pstart atha sambhū\-yakaraṇam apekṣā\-rthaḥ, tadā\- yadi pū\-rvasthitasyeti viśeṣaṇā\-pekṣā\- tadā\- kṣaṇikasya naivaṃ kadā\-cid ity anapekṣaivā\-kṣī\-ṇā\- |
	\pend
      

	  \pstart atha pū\-rvasthitasyety anapekṣya militā\-vasthitasyaiva karaṇam apekṣā\-rthas tadā\- sā\-pekṣataiva, nā\-napekṣā\- | tathā\- ca nobhayavyā\-vṛttir ity asiddhaḥ prathamo vyā\-pakā\-nupalambhaḥ | 
	\pend
      

	  \pstart tathaikatvā\-nekatvayor api vyā\-pakayoḥ kṣaṇikā\-d vyā\-vṛttir asiddhā\- | tattadvyā\-vṛttibhedam ā\-śrityopā\-dā\-natvā\-di kā\-lpanikasvabhā\-vabhede 'pi paramā\-rthata ekenaiva svarū\-peṇā\-nekakā\-ryaniṣpā\-danā\-d ubhayavyā\-vṛtter abhā\-vā\-t |
	\pend
      

	  \pstart yac ca bī\-jasyaikenaiva svabhā\-vena kā\-rakatve kṣityā\-dī\-nā\-m aṅkurasvā\-bhā\-vyā\-pattir anyathā\- kā\-raṇā\-bhede 'pi kā\-ryabhede 'pi kā\-ryasyā\-hetukatvaprasaṅgā\-d ity uktam tad asaṅgatam | kā\-raṇaikatvasya kā\-ryabhedasya ca paṭunendriyapratyakṣeṇa prasā\-dhanā\-t | ekakā\-raṇajanyatvaikatvayor vyā\-pteḥ pratihatatvā\-t | prasaṅgasyā\-nupadatvā\-t |
	\pend
      

	  \pstart yac ca kā\-raṇā\-bhede kā\-ryā\-bheda ity uktaṃ tatra sā\-magrī\-svarū\-paṃ kā\-raṇam abhipretam | sā\-magrī\-sajā\-tī\-yatve na kā\-ryavijā\-tī\-yatety arthaḥ | na punaḥ sā\-magrī\-madhyagatenaikenā\-nekaṃ kā\-ryaṃ na kartavyaṃ nā\-ma, ekasmā\-d anekotpatteḥ pratyakṣasiddhatvā\-t | na caivaṃ pratyabhijñā\-nā\-t kā\-labhede 'py abhedasiddhir ity uktaprā\-yam | na cendriyapratyakṣaṃ bhinnadeśaṃ sapratighaṃ dṛśyam arthadvayam ekam evopalambhayatī\-ti kvacid upalabdham | yena tatrā\-pi bhedaśaṅkā\- syā\-t | śaṅkā\-yā\-ṃ vā\- paṭupratyakṣasyā\-py apalā\-pe sarvapramā\-ṇocchedaprasaṅgā\-d |
	\pend
      

	  \pstart nā\-pi sattvahetoḥ sandigdhavyatirekitvam , kṣityā\-der dravyā\-ntarasya bī\-jasvabhā\-vatvenā\-smā\-bhir asvī\-kṛtatvā\-t | anupakā\-riṇy apekṣā\-yā\-ḥ pratyā\-khyā\-tatvā\-t vyā\-pakā\-nupalambhasyā\-siddhatvā\-yogā\-t |
	\pend
      

	  \pstart tad etau dvā\-v api vyā\-pakā\-nupalambhā\-v asiddhau na kṣaṇikā\-t sattvaṃ nivartayata iti nā\-yam asā\-dhā\-raṇo hetuḥ ||
	\pend
      

	  \pstart api ca vidyamā\-no bhā\-vaḥ sā\-dhyetarayor aniścitā\-nvayavyatireko gandhavattā\-divad asā\-dhā\-raṇo yuktaḥ | prakṛtavyā\-pakā\-nupalambhā\-c ca sarvathā\-rthakriyaivā\-satī\- ubhā\-bhyā\-ṃ vā\-dibhyā\-m ubhayasmā\-d vinivartitatvena nirā\-śrayatvā\-t | tat katham asā\-dhā\-raṇā\-naikā\-ntiko bhaviṣyatī\-ty alaṃ pralā\-pini nirbandhena |
	\pend
      

	  \pstart tad evaṃ śaktasya kṣepā\-yogā\-t samarthavyavahā\-ragocaratvaṃ jananena vyā\-ptam iti prasaṅgaviparyayayoḥ sattve hetor api nā\-naikā\-ntikatvam | ataḥ kṣaṇabhaṅgasiddhir iti sthitam |
	\pend
      
	    
	    \stanza[\smallbreak]
\edlabel{thakur75-82.15}\label{thakur75-82.15} iti sā\-dharmyadṛṣṭā\-nte 'nvayarū\-pavyā\-ptyā\- kṣaṇabhaṅgasiddhiḥ samā\-ptā\- ||&\edlabel{thakur75-82.17}\label{thakur75-82.17} kṛtir iyaṃ mahā\-paṇḍitaratnakī\-rtipā\-dā\-nā\-m iti ||\&[\smallbreak]


	
	  
	% new div opening: depth here is 1
	
\section[{Kṣaṇabhaṅgasiddhiḥ Vyatirekā\-tmikā\-}]{Kṣaṇabhaṅgasiddhiḥ Vyatirekā\-tmikā\-}\edlabel{Kṣaṇabhaṅgasiddhiḥ_Vyatirekātmikā}\label{Kṣaṇabhaṅgasiddhiḥ_Vyatirekātmikā}

	  \pstart namas tā\-rā\-yai \edlabel{thakur75-83.6}\label{thakur75-83.6} vyatirekā\-tmikā\- vyā\-ptir ā\-kṣiptā\-nvayarū\-piṇī\- | vaidharmyavati dṛṣṭā\-nte sattvahetor ihocyate || \edlabel{thakur75-83.8}\label{thakur75-83.8} yat sat tat kṣaṇikam | yathā\- ghaṭaḥ | santaś cā\-mī\- vivā\-dā\-spadī\-bhū\-tā\-ḥ padā\-rthā\- iti svabhā\-vahetuḥ | \edlabel{thakur75-83.10}\label{thakur75-83.10} na tā\-vad asyā\-siddhiḥ sambhavati, yathā\-yogaṃ pratyakṣā\-numā\-napramā\-ṇapratī\-te dharmiṇi sattvaśabdenā\-bhipretasyā\-rthakriyā\-kā\-ritvalakṣaṇasya sā\-dhanasya pramā\-ṇasamadhigatatvā\-t | \edlabel{thakur75-83.12}\label{thakur75-83.12} na ca viruddhā\-naikā\-ntikate, vyā\-pakā\-nupalambhā\-tmanā\- viparyaye bā\-dhakapramā\-ṇena vyā\-pteḥ prasā\-dhanā\-t | \edlabel{thakur75-83.13}\label{thakur75-83.13} yasya kramā\-kramau na vidyete na tasyā\-rthakriyā\-sā\-marthyam | yathā\- śaśaviṣā\-ṇasya | na vidyete cā\-kṣaṇikasya kramā\-kramā\-v iti vyā\-pakā\-nupalambhaḥ | \edlabel{thakur75-83.14}\label{thakur75-83.14} na tā\-vad ayam asiddho hetuḥ, akṣaṇike dharmiṇi kramā\-kramasadbhā\-vā\-yogā\-t | tathā\- hi prā\-ptā\-parakā\-layor ekatve nityatvam | tasya kramā\-kramayoge kṣaṇadvaye 'py avaśyaṃ bhedaḥ | bhedā\-bhedayoś ca parasparavirodhā\-t kuto 'kṣaṇike kramā\-kramasambhavaḥ | kṣaṇadvaye 'pi bhede kramā\-kramayogaḥ | abhede hi prathama eva kṣaṇe śaktatvā\-d bhā\-vino 'pi kā\-ryasya karaṇaprasaṅge kathaṃ kā\-ryā\-ntarakaraṇe kramā\-ntarā\-vakā\-śaḥ | na cā\-kṣaṇikasyā\-krameṇaiva sakalasvakā\-ryaṃ kṛtvā\- svā\-sthyam | kṣaṇā\-ntare 'pi śaktatvā\-t punas tatkā\-ryakaraṇaprasaṅgā\-t | \edlabel{thakur75-83.21}\label{thakur75-83.21} tasmā\-d akṣaṇikam iti pū\-rvā\-parakā\-layor abhedaḥ | kramā\-kramayoga iti pū\-rvā\-parakā\-layor bhedaḥ | anayoś ca parasparaparihā\-rasthitilakṣaṇo virodhaḥ | \edlabel{thakur75-83.23}\label{thakur75-83.23} tad ayam akṣaṇike dharmiṇi kramā\-kramā\-bhā\-valakṣaṇo hetur nā\-siddho vaktavyaḥ | kramā\-kramayogitvā\-kṣaṇikatvayor virodhā\-d eva | \edlabel{thakur75-84.1}\label{thakur75-84.1} nā\-pi viruddhaḥ, sapakṣe bhā\-vā\-t | \edlabel{thakur75-84.2}\label{thakur75-84.2} na cā\-naikā\-ntikaḥ, kramā\-kramā\-bhā\-vasyā\-rthakriyā\-sā\-marthyā\-bhā\-vena vyā\-ptatvā\-t | \edlabel{thakur75-84.3}\label{thakur75-84.3} yenaiva hi pratyakṣā\-tmanā\- pramā\-ṇenā\-paraprakā\-rā\-bhā\-vā\-d vidhibhū\-tā\-bhyā\-ṃ kramā\-kramā\-bhyā\-ṃ vidhibhū\-tasyā\-rthakriyā\-sā\-marthyasya vyā\-ptiḥ prasā\-dhitā\-, tenaivā\-rthakriyā\-sā\-marthyā\-bhā\-vena kramā\-kramā\-bhā\-vasya vyā\-ptiḥ prasā\-dhiteti svī\-kartavyam | na hi dahanā\-dinā\- dhū\-mā\-der vyā\-ptisā\-dhakapramā\-ṇā\-d aparaṃ dhū\-mā\-dyabhā\-vena dahanā\-dyabhā\-vasya vyā\-ptisā\-dhakaṃ kiñcit pramā\-ṇaṃ śaraṇabhū\-tam asti | tasmā\-d vidhyor eva vyā\-ptisā\-dhakaṃ pramā\-ṇam abhā\-vayor api vyā\-ptisā\-dhakam iti nyā\-yasya duratikramatvā\-t sattvā\-bhā\-vena kramā\-kramā\-bhā\-vo vyā\-pta eveti nā\-naikā\-ntika ity anavadyo vyā\-pakā\-nupalambhaḥ | tad ayam akṣaṇikā\-d vinivartamā\-ṇaḥ svavyā\-pyaṃ sattvaṃ nivartya kṣaṇike viśrā\-mayatī\-ti sattvahetoḥ kṣaṇabhaṅgasiddhir apy anavadyā\- | \edlabel{thakur75-85.1}\label{thakur75-85.1} nanu vyā\-pakā\-nupalambhataḥ sattvasya kathaṃ svasā\-dhyapratibandhasiddhiḥ, asyā\-py anekadoṣaduṣṭatvā\-t. tathā\- hi – na tā\-vad ayaṃ prasaṅgahetuḥ, sā\-dhyadharmiṇi pramā\-ṇasiddhatvā\-t, parā\-bhyupagamasiddhatvā\-bhā\-vā\-t, viparyayaparyavasā\-nā\-bhā\-vā\-c ca. atha svatantraḥ, tadā\-śrayā\-siddhaḥ, akṣaṇikasyā\-śrayasyā\-sambhavā\-d apratī\-tatvā\-d vā\-. pratī\-tir hi2 〔a〕 pratyakṣeṇa 〔b〕 anumā\-nena 〔c〕 vikalpamā\-treṇa vā\- syā\-t | \edlabel{thakur75-85.6}\label{thakur75-85.6} 〔a〕 〔b〕 prathamapakṣadvaye sā\-kṣā\-t pā\-ramparyeṇa vā\- svapratī\-tilakṣaṇā\-rthakā\-ritve maulaḥ sā\-dhā\-raṇo hetuḥ vyā\-pakā\-nupalambhaś ca svarū\-pā\-siddhaḥ syā\-t, arthakriyā\-kā\-ritve kramā\-kramayor anyatarasyā\-vaśyambhā\-vā\-t | \edlabel{thakur75-85.8}\label{thakur75-85.8} 〔c〕 antimapakṣe tu na kaścid dhetur anā\-śrayaḥ syā\-t, vikalpamā\-trasiddhasya dharmiṇaḥ sarvatra sulabhatvā\-t. \edlabel{thakur75-85.10}\label{thakur75-85.10} api ca – tat kalpanā\-jñā\-naṃ 〔c1〕 pratyakṣapṛṣṭhabhā\-vi vā\- syā\-t, 〔c2〕 liṅgajanma vā\-, 〔c3〕 saṃskā\-rajaṃ vā\-, 〔c4〕 sandigdhavastukaṃ vā\-, 〔c5〕 avastukaṃ vā\-. \edlabel{thakur75-85.12}\label{thakur75-85.12} tatra 〔c1〕〔c2〕 ā\-dyapakṣadvaye 'kṣaṇikasya sattaivā\-vyā\-hatā\-, kathaṃ bā\-dhakā\-vatā\-raḥ. \edlabel{thakur75-85.12a}\label{thakur75-85.12a} 〔c3〕 tṛtī\-ye tu na sarvadā\-kṣaṇikasattā\-niṣedhaḥ, tadarpitasaṃskā\-rā\-bhā\-ve tatsmaraṇā\-yogā\-t | \edlabel{thakur75-85.13}\label{thakur75-85.13} 〔c4〕 caturthe tu sandigdhā\-śrayatvaṃ hetudoṣaḥ | \edlabel{thakur75-85.14}\label{thakur75-85.14} 〔c5〕 pañcame ca tadviṣayasyā\-bhā\-vo na tā\-vat pratyakṣataḥ sidhyati, akṣaṇikā\-tmanaḥ sarvadaiva tvanmate 'pratyakṣatvā\-t | na cā\-numā\-natas tadabhā\-vas tatpratibaddhaliṅgā\-nupalambhā\-d ity ā\-śrayā\-siddhis tā\-vad uddhatā\- | evaṃ dṛṣṭā\-nto 'pi pratihantavyaḥ | \edlabel{thakur75-85.18}\label{thakur75-85.18} svarū\-pā\-siddho 'py ayaṃ hetuḥ, sthirasyā\-pi kramā\-kramisahakā\-ryapekṣayā\- kramā\-kramā\-bhyā\-m arthakriyopapatteḥ | nā\-pi kramayaugapadyapakṣoktadoṣaprasaṅgaḥ | tathā\- hi kramisahakā\-ryapekṣayā\- kramikā\-ryakā\-ritvaṃ tā\-vad aviruddham | \edlabel{thakur75-85.21}\label{thakur75-85.21} tathā\- ca Śaṅkarasya saṃkṣipto 'yam abhiprā\-yaḥ | sahakā\-risā\-kalyaṃ hi sā\-marthyam | tadvaikalyaṃ cā\-sā\-marthyam | na ca tayor ā\-virbhā\-vatirobhā\-vā\-bhyā\-ṃ tadvataḥ kā\-cit kṣatiḥ, tasya tā\-bhyā\-m anyatvā\-t | tat kathaṃ sahakā\-riṇo 'napekṣya kā\-ryakaraṇaprasaṅga iti | \edlabel{thakur75-85.25}\label{thakur75-85.25} \persName{trilocanasyā\-}py ayaṃ saṃkṣiptā\-rthaḥ | kā\-ryam eva hi sahakā\-riṇam apekṣate | na kā\-ryotpattihetuḥ | yasmā\-t dvividhaṃ sā\-marthyaṃ nijam ā\-gantukaṃ ca sahakā\-ryantaram, tato 'kṣaṇikasyā\-pi kramavatsahakā\-rinā\-nā\-tvā\-d api kramavatkā\-ryanā\-nā\-tvopapatter aśakyaṃ bhā\-vā\-nā\-ṃ pratikṣaṇam anyā\-nyatvam upapā\-dayitum iti | \edlabel{thakur75-85.29}\label{thakur75-85.29} Nyā\-yabhū\-ṣaṇo 'pi lapati | prathamakā\-ryotpā\-danakā\-le hi uttarakā\-ryotpā\-danasvabhā\-vaḥ | ataḥ prathamakā\-la evā\-śeṣā\-ṇi kā\-ryā\-ṇi kuryā\-d iti cet | \edlabel{thakur75-85.30}\label{thakur75-85.30} tad idaṃ mā\-tā\- me bandhyetyā\-divat svavacanavirodhā\-d ayuktaṃ | yo hi uttarakā\-ryajananasvabhā\-vaḥ sa katham ā\-dau tat kā\-ryaṃ kuryā\-t | atha kuryā\-t na tarhi tatkā\-ryakaraṇasvabhā\-vaḥ | na hi nī\-lotpā\-danasvabhā\-vaḥ pī\-tā\-dikam api karotī\-ti | \edlabel{thakur75-86.3}\label{thakur75-86.3} Vā\-caspatir api paṭhati | nanv ayam akṣaṇikaḥ svarū\-peṇa kā\-ryaṃ janayati | tac cā\-sya svarū\-paṃ tṛtī\-yā\-diṣv iva kṣaṇeṣu dvitī\-ye 'pi kṣaṇe sad iti tadā\-pi janayet | akurvan vā\- tṛtī\-yā\-diṣv api na kurvī\-ta, tasya tā\-davasthyā\-t | atā\-davasthye vā\- tad evā\-sya kṣaṇikatvam || \edlabel{thakur75-86.7}\label{thakur75-86.7} atrocyate | satyaṃ svarū\-peṇa kā\-ryaṃ janayati na tu tenaiva | sahakā\-risahitā\-d eva tataḥ kā\-ryotpattidarśanā\-t | tasmā\-d vyā\-ptivat kā\-ryakā\-raṇabhā\-vo 'py ekatrā\-nyayogavyavacchedena | anyatrā\-yogavyavacchedenā\-vaboddhavyaḥ | tathaiva laukikaparī\-kṣakā\-ṇā\-ṃ saṃpratipatter iti na kramikā\-ryakā\-ritvapakṣoktadoṣā\-vasaraḥ || \edlabel{thakur75-86.11}\label{thakur75-86.11} nā\-py akṣaṇike yaugapadyapakṣoktadoṣā\-vakā\-śaḥ | ye hi kā\-ryam utpā\-ditavanto dravyaviśeṣā\-s teṣā\-ṃ vyā\-pā\-rasya niyatakā\-ryotpā\-danasamarthasya niṣpā\-dite kā\-rye 'nuvartamā\-neṣv api teṣu dravyeṣu nivṛttā\-rthā\-dū\-nā\- sā\-magrī\- jā\-yate | tat kathaṃ niṣpā\-ditaṃ niṣpā\-dayiṣyati | na hi daṇḍā\-dayaḥ svabhā\-venaiva kartā\-ro yenā\-mī\- niṣpatter ā\-rabhya kā\-ryaṃ vidadhyuḥ | kiṃ tarhi vyā\-pā\-rā\-veśinaḥ | na ceyatā\- svarū\-peṇa na kartā\-raḥ, svarū\-pakā\-rakatvanirvā\-haparatayā\- vyā\-pā\-rasamā\-veśā\-d iti || \edlabel{thakur75-86.17}\label{thakur75-86.17} kiṃ ca kramā\-kramā\-bhā\-vaś ca bhaviṣyati na ca sattvā\-bhā\-va iti sandigdhavyatireko 'py ayaṃ vyā\-pakā\-nupalambhaḥ | na hi kramā\-kramā\-bhyā\-m anyasya prakā\-rasyā\-bhā\-vaḥ siddhaḥ, viśeṣaniṣedhasya śeṣā\-bhyanujñā\-viṣayatvā\-t | \edlabel{thakur75-86.20}\label{thakur75-86.20} kiṃ ca prakā\-rā\-ntarasya dṛśyatve nā\-tyantaniṣedhaḥ | adṛśyatve tu nā\-sattā\-niścayo viprakarṣiṇā\-m iti na kramā\-kramā\-bhyā\-m arthakriyā\-sā\-marthyasya vyā\-ptisidhiḥ | ataḥ sandigdhavyatireko 'pi vyā\-pakā\-nupalambhaḥ | \edlabel{thakur75-86.23}\label{thakur75-86.23} kiṃ ca dṛśyā\-dṛśyasahakā\-ripratyayasā\-kalyavataḥ kramayaugapadyasyā\-tyantaparokṣatvā\-t tena vyā\-ptaṃ sattvam api parokṣam eveti na tā\-vat pratibandhaḥ pratyakṣataḥ sidhyati | nā\-py anumā\-nataḥ tatpratibaddhaliṅgā\-bhā\-vā\-d iti | \edlabel{thakur75-86.26}\label{thakur75-86.26} api ca kramā\-kramā\-bhyā\-m arthakriyā\-kā\-ritvaṃ vyā\-ptam ity atisubhā\-ṣitam | yadi krameṇa vyā\-ptaṃ katham akrameṇa | athā\-krameṇa na tarhi krameṇa | kramā\-kramā\-bhyā\-m vyā\-ptam iti tu bruvatā\- vyā\-pter evā\-bhā\-vaḥ pradarśito bhavati | na hi bhavati dhū\-mo vahnibhā\-vā\-bhā\-vā\-bhyā\-ṃ vyā\-pta iti | ato vyā\-pter anaikā\-ntikatvam | \edlabel{thakur75-86.30}\label{thakur75-86.30} capi ca kim idaṃ bā\-dhakam akṣaṇikā\-nā\-m asattā\-ṃ sā\-dhayati, utasvid akṣaṇikā\-t sattvasya vyatirekam, atha sattvakṣaṇikatvayoḥ pratibandham. \edlabel{thakur75-86.31}\label{thakur75-86.31} na pū\-rvo vikalpaḥ, uktakrameṇa hetor ā\-śrayā\-siddhatvā\-t | \edlabel{thakur75-87.1}\label{thakur75-87.1} na ca dvitī\-yaḥ. yato vyā\-pakanivṛttisahitā\- vyā\-pyanivṛttir vyatirekaśabdasyā\-rthaḥ. sā\- ca yadi pratyakṣeṇa pratī\-yate tadā\- taddhetuḥ syā\-d iti sattvam anaikā\-ntikam. vyā\-pakā\-nupalambhaḥ svarū\-pā\-siddhaḥ. atha sā\- vikalpyate tadā\- pū\-rvoktakrameṇa pañcadhā\- vikalpya vikalpo dū\-ṣaṇī\-yaḥ. \edlabel{thakur75-87.4}\label{thakur75-87.4} ata eva na tṛtī\-yo 'pi vikalpaḥ vyatirekā\-siddhau sambandhā\-siddheḥ | \edlabel{thakur75-87.6}\label{thakur75-87.6} kiṃ ca na bhū\-talavad atrā\-kṣaṇiko dharmī\- dṛśyate | na ca svabhā\-vā\-nupalambhe vyā\-pakā\-nupalambhaḥ kasyacit dṛśyasya pratipattim antareṇā\-ntarbhā\-vayituṃ śakyata iti | \edlabel{thakur75-87.9}\label{thakur75-87.9} kiṃ cā\-syā\-bhā\-vadharmatve ā\-śrayā\-siddhatvam itaretarā\-śrayatvaṃ ca | bhā\-vadharmatve viruddhatvaṃ ca | ubhayadharmatve cā\-naikā\-ntikatvam iti na trayī\-ṃ doṣajā\-tim atipatati | \edlabel{thakur75-87.11}\label{thakur75-87.11} yat punar uktam akṣaṇikatve kramayaugapadyā\-bhyā\-m arthakriyā\-virodhā\-d iti | dtatra virodhasiddhim anusaratā\- virodhy api pratipattavyaḥ | tatpratī\-tinā\-ntarī\-yakatvā\-d virodhasiddheḥ | yathā\- tuhinadahanayoḥ sā\-pekṣadhruvabhā\-vayoś ca | \edlabel{thakur75-87.13}\label{thakur75-87.13} pratiyogī\- cā\-kṣaṇikaḥ pratī\-yamā\-naḥ pratī\-tikā\-ritvā\-t sann eva syā\-t, ajanakasyā\-prameyatvā\-t | \edlabel{thakur75-87.15}\label{thakur75-87.15} saṃvṛtisiddhenā\-kṣaṇikatvena virodhasiddhir iti cet | saṃvṛtisiddham api vā\-stavaṃ kā\-lpanikaṃ vā\- syā\-t | \edlabel{thakur75-87.17}\label{thakur75-87.17} yadi vā\-stavaṃ kathaṃ tasyā\-sattvam | kathaṃ cā\-rthakriyā\-kā\-ritvavirodhaḥ | arthakriyā\-ṃ kurvad dhi vā\-stavam ucyate | \edlabel{thakur75-87.19}\label{thakur75-87.19} atha kā\-lpanikam | tatra kiṃ virodho vā\-stavaḥ, kā\-lpaniko vā\- | na tā\-vad vā\-stavaḥ, kalpitavirodhivirodhatvā\-t, bandhyā\-putravirodhavat | atha virodho 'pi kā\-lpanikaḥ na tarhi sattvasya vyatirekaḥ pā\-ramā\-rthika iti kṣaṇabhaṅgo dattajalā\-ñjalir iti | \edlabel{thakur75-87.23}\label{thakur75-87.23} ayam eva codyaprabandho 'smadgurubhiḥ saṅgṛhī\-taḥ | enityaṃ nā\-sti na vā\- pratī\-tiviṣayaṃ3 tenā\-śrayā\-siddhatā\- hetoḥ svā\-nubhavasya ca kṣatir ataḥ kṣiptaḥ sapakṣo 'pi ca | śū\-nyaś ca dvitayena sidhyati na cā\-sattā\-pi sattā\- yathā\- no nityena virodhasiddhir asatā\- śakyā\- kramā\-der api || J 89,16-19; cf. R 94,21-24 iti | \edlabel{thakur75-87.28}\label{thakur75-87.28} atrocyate – iha vastuny api dharmidharmavyavahā\-ro dṛṣṭaḥ, yathā\- gavi gotvam, paṭe śuklatvam, turage gamanam ityā\-di. avastuny api dharmidharmavyavahā\-ro dṛṣṭaḥ, yathā\- śaśaviṣā\-ṇe tī\-kṣṇatvā\-bhā\-vaḥ, bandhyā\-putre vaktṛtvā\-bhā\-vaḥ, gaganā\-ravinde gandhā\-bhā\-va ityā\-di. tatrā\-vastuni dharmitvaṃ nā\-stī\-ti kiṃ vastudharmeṇa dharmitvaṃ nā\-sti, ā\-hosvid avastudharmeṇā\-pi | \edlabel{thakur75-88.3}\label{thakur75-88.3} prathamapakṣe siddhasā\-dhanam. dvitī\-yapakṣe tu svavacanavirodhaḥ. yad ā\-hur guravaḥ – fdharmasya kasyacid avastuni mā\-nasiddhā\- bā\-dhā\-vidhivyavahṛtiḥ kim ihā\-sti no vā\- | kvā\-py asti cet katham iyanti na dū\-ṣaṇā\-ni nā\-sty eva cet svavacanapratirodhasiddhiḥ || J 89,21-24; cf. R 94,26-28 \edlabel{thakur75-88.8}\label{thakur75-88.8} avastuno dharmitvasvī\-kā\-rapū\-rvakatvasya vyā\-pakasyā\-bhā\-vā\-d ā\-śrayā\-siddhidū\-ṣaṇasyā\-nupanyā\-saprasaṅga ity arthaḥ | yenaiva hi vacanenā\-vastuno dharmitvaṃ pratiṣidhyate, tenaivā\-vastuno dharmitvā\-bhā\-vena dharmeṇa dharmitvam abhyupagatam | paran tu pratiṣidhyata iti vyaktam idam ī\-śvaraceṣṭitam | tathā\- hy avastuno dharmitvaṃ nā\-stī\-ti vacanena dharmitvā\-bhā\-vaḥ kim avastuni vidhī\-yate, anyatra vā\-, na vā\- kvacid apī\-ti trayaḥ pakṣā\-ḥ | \edlabel{thakur75-88.13}\label{thakur75-88.13} prathamapakṣe 'vastuno na dharmitvaniṣedhaḥ dharmitvā\-bhā\-vasya dharmasya tatraiva vidhā\-nā\-t | \edlabel{thakur75-88.14}\label{thakur75-88.14} dvitī\-ye 'vastuni kim ā\-yā\-tam anyatra dharmitvā\-bhā\-vavidhā\-nā\-t | \edlabel{thakur75-88.15}\label{thakur75-88.15} tṛtī\-yas tu pakṣo vyartha eva nirā\-śrayatvā\-d iti katham avastuno dharmitvaniṣedhaḥ | tasmā\-d yathā\- pramā\-ṇopanyā\-saḥ prameyasvī\-kā\-rapū\-rvakatvena vyā\-ptaḥ vā\-cakaśabdopanyā\-so vā\- vā\-cyasvī\-kā\-rapū\-rvakatvena vyā\-ptas tathā\-vastuno dharmitvaṃ nā\-stī\-ti vacanopanyā\-so 'vastuno dharmitvasvī\-kā\-rapū\-rvakatvena vyā\-ptaḥ | anyathā\- tadvacanopanyā\-sasya vyarthatvā\-t | \edlabel{thakur75-88.19}\label{thakur75-88.19} tad yadi vacanopanyā\-so vyā\-pyadharmas tadā\- 'vastuno dharmitvasvī\-kā\-ro 'pi vyā\-pakadharmo durvā\-raḥ | atha na vyā\-pakadharmaḥ tadā\- vyā\-pyasyā\-pi vacanopanyā\-sasyā\-sambhava iti mū\-kataivā\-tra balā\-d ā\-yā\-teti kathaṃ na svavacanapratirodhasiddhiḥ | \edlabel{thakur75-88.22}\label{thakur75-88.22} yad ā\-hā\-cā\-ryaḥ: na hy abruvan paraṃ bodhayitum ī\-śaḥ | bruvan vā\- doṣam imaṃ parihartum iti mahati saṃkaṭe praveśaḥ | \edlabel{thakur75-88.24}\label{thakur75-88.24} avastuprastā\-ve sahṛdayā\-nā\-ṃ mū\-kataiva yujyata iti cet | aho mahadvaidagdhyam | avastuprastā\-ve svayam eva yathā\-śakti valgitvā\- bhagno mū\-kataiva nyā\-yaprā\-pteti paribhā\-ṣayā\- niḥsartum icchati | na cā\-vastuprastā\-vo rā\-jadaṇḍena vinā\- caraṇamardanā\-dinā\-niṣṭimā\-treṇa vā\- pratiṣeddhaṃ śakyate | tataś cā\-trā\-pi kramā\-kramabhā\-vasya sā\-dhanatve sattvā\-bhā\-vasya ca sā\-dhyatve sandigdhavastubhā\-vasyā\-vastvā\-tmano vā\- kṣaṇikasya dharmitvaṃ kena pratiṣidhyate | \edlabel{thakur75-89.1}\label{thakur75-89.1} trividho hi dharmo dṛṣṭaḥ | kaścit vastuniyato nī\-lā\-diḥ | kaścid avastuniyato yathā\- sarvopā\-khyā\-virahaḥ | kaścid ubhayasā\-dhā\-raṇo yathā\- 'nupalabdhimā\-tram | tatra vastudharmeṇā\-vastuno dharmitvaniṣedha iti yuktam | na tv avastudharmeṇa vastvavastudharmeṇa vā\-, svavacanasyā\-nupanyā\-saprasaṅgā\-d ity akṣaṇikasyā\-bhā\-ve sandehe 'pi vā\- vastudharmeṇa dharmitvam avyā\-hatam iti nā\-yam ā\-śrayā\-siddho vyā\-pakā\-nupalambhaḥ | \edlabel{thakur75-89.6}\label{thakur75-89.6} akṣaṇikā\-pratī\-tā\-v ā\-śrayā\-siddho hetur iti yuktam uktam, tadapratī\-tau tadvyavahā\-rā\-yogā\-t | kevalam asau vyavahā\-rā\-ṅgabhū\-tā\- pratī\-tir vastvavastunor ekarū\-pā\- na bhavati | sā\-kṣā\-t pā\-ramparyeṇa vastusā\-marthyabhā\-vinī\- hi vastupratī\-tiḥ | yathā\- pratyakṣam anumā\-naṃ pratyakṣapṛṣṭhabhā\-vī\- ca vikalpaḥ | avastunas tu sā\-marthyā\-bhā\-vā\-d vikalpamā\-tram eva pratī\-tiḥ | vastuno hi vastubalabhā\-vinī\- pratī\-tir yathā\- sā\-kṣā\-t pratyakṣam, paramparayā\- tatpṛṣṭhabhā\-vī\- vikalpo 'numā\-naṃ ca | avastunas tu na vastubalabhā\-vinī\- pratī\-tis tatkā\-rakatvenā\-vastutvahā\-niprasaṅgā\-t | tasmā\-d vikalpamā\-tram evā\-vastunaḥ pratī\-tiḥ | \edlabel{thakur75-89.12}\label{thakur75-89.12} na hy abhā\-vaḥ kaścid vigrahavā\-n yaḥ sā\-kṣā\-t kartavyo 'pi tu vyavahartavyaḥ | sa ca vyavahā\-ro vikalpā\-d api sidhyaty eva anyathā\- sarvajanaprasiddho 'vastuvyavahā\-ro na syā\-t | iṣyate ca taddharmitvapratiṣedhā\-nubandhā\-d ity akā\-makenā\-pi vikalpamā\-trasiddho 'kṣaṇikaḥ svī\-kartavya iti nā\-yam apratī\-tatvā\-d apy ā\-śrayā\-siddho hetur vaktavyaḥ | tataś cā\-kṣaṇikasya vikalpamā\-trasiddhatve yad uktam | \edlabel{thakur75-89.17}\label{thakur75-89.17} na kaścid dhetur anā\-śrayaḥ vikalpamā\-trasiddhasya dharmiṇaḥ sarvatra sulabhatvā\-d iti tad asaṅgatam | vikalpamā\-trasiddhasya dharmiṇaḥ sarvatra sambhave 'pi vastudharmeṇa dharmitvā\-yogā\-t | vastudharmahetutvā\-pekṣayā\- ā\-śrayā\-siddhasyā\-pi hetoḥ sambhavā\-t | \edlabel{thakur75-89.20}\label{thakur75-89.20} yathā\-tmano vibhutvasā\-dhanā\-rtham upanyastaṃ sarvatropalabhyamā\-naguṇatvā\-d iti sā\-dhanam | vikalpaś cā\-yaṃ hetū\-panyā\-sā\-t pū\-rvaṃ sandigdhavastukaḥ | samarthite tu hetā\-v avastuka iti brū\-maḥ | \edlabel{thakur75-89.23}\label{thakur75-89.23} na cā\-tra sandigdhā\-śrayatvaṃ nā\-ma hetudośaḥ | ā\-stā\-ṃ tā\-vat | sandigdhasyā\-vastuno 'pi vikalpamā\-trasiddhasyā\-vastudharmā\-pekṣayā\- dharmitvaprasā\-dhanā\-t | vastudharmahetvapekṣayaiva sandigdhā\-śrayasya hetvā\-bhā\-sasya vyavasthā\-panā\-t | yatheha nikuñje mayū\-raḥ kekā\-yitā\-d iti | avastukavikalpaviṣayasyā\-sattvaṃ tu vyā\-pakā\-nupalambhā\-d eva prasā\-dhitam | evaṃ dṛṣṭā\-ntasyā\-pi vyomotpalā\-der dharmitvaṃ vikalpamā\-treṇa pratī\-tiś cā\-vagantavyā\- | tad evam avastudharmā\-pekṣā\-yā\-vastuno dharmitvasya vikalpamā\-treṇa pratī\-teś cā\-pahnotum aśakyatvā\-n nā\-yam ā\-śrayā\-siddho hetuḥ | na ca dṛṣṭā\-ntakṣatiḥ | \edlabel{thakur75-89.30}\label{thakur75-89.30} na caiṣa svarū\-pā\-siddhaḥ, akṣaṇike dharmiṇi kramā\-kramayor vyā\-pakayor ayogā\-t | tathā\- hi yadi tasya prathame kṣaṇe dvitī\-yā\-dikṣaṇabhā\-vikā\-ryakaraṇasā\-marthyam asti tadā\- prathamakṣaṇabhā\-vikā\-ryavat dvitī\-yā\-dikṣaṇabhā\-vy api kā\-ryaṃ kuryā\-t, samarthasya kṣepā\-yogā\-t | \edlabel{thakur75-90.2}\label{thakur75-90.2} atha tadā\- sahakā\-risā\-kalyalakṣaṇasā\-marthyaṃ nā\-sti, tadvaikalyalakṣaṇasyā\-sā\-marthyasya sambhavā\-t | na hi bhā\-vaḥ svarū\-peṇa karotī\-ti svarū\-peṇaiva karoti, sahakā\-risahitā\-d eva tataḥ kā\-ryotpattidarśanā\-d iti cet | \edlabel{thakur75-90.4}\label{thakur75-90.4} yadā\- tā\-vad amī\- militā\-ḥ santaḥ kā\-ryaṃ kurvate | tadaikā\-rthakaraṇalakṣaṇaṃ sahakā\-ritvam eṣā\-m astu, ko niṣeddhā\- | militair eva tu tatkā\-ryaṃ kartavyam iti kuto labhyate | pū\-rvā\-parakā\-layor ekasvabhā\-vatvā\-d bhā\-vasya sarvadā\- jananā\-jananayor anyataraniyamaprasaṅgasya durvā\-ratvā\-t | tasmā\-t sā\-magrī\- janikā\-, naikaṃ janakam iti sthiravā\-dinā\-ṃ manorajyasyā\-py aviṣayaḥ | \edlabel{thakur75-90.9}\label{thakur75-90.9} kiṃ kurmo dṛśyate tā\-vad evam iti cet | dṛśyatā\-ṃ, kiṃ tu pū\-rvasthitā\-d eva paścā\-t sā\-magrī\-madhyapraviṣṭā\-d bhā\-vā\-t kā\-ryotpattir anyasmā\-d eva viśiṣṭasā\-magrī\-samutpannā\-t kṣaṇā\-d iti vivā\-dapadam etat | tatra prā\-g api sambhave sarvadaiva kā\-ryotpattir na vā\- kadā\-cid apī\-ti virodham asamā\-dhā\-ya tata eva kā\-ryotpattir iti sā\-dhyā\-nuvā\-damā\-trapravṛttaḥ kṛpā\-m arhati | \edlabel{thakur75-90.14}\label{thakur75-90.14} na ca pratyabhijñā\-nā\-d evaikatvasiddhiḥ, tatpauruṣasya lū\-napunarjā\-takeśakuśakadalī\-stambā\-dau nirdalanā\-t | vistareṇa ca pratyabhijñā\-dū\-ṣaṇam asmā\-bhiḥ sthirasiddhidū\-ṣaṇe pratipā\-ditam iti tata evā\-vadhā\-ryam | \edlabel{thakur75-90.17}\label{thakur75-90.17} nanu kā\-ryam eva sahakā\-riṇam apekṣate | na tu kā\-ryotpattihetuḥ | yasmā\-d dvividhaṃ sā\-marthyaṃ nijam ā\-gantukaṃ ca sahakā\-ryantaram tato akṣaṇikasyā\-pi kramavatsahakā\-rinā\-nā\-tvā\-d api kramavatkā\-ryanā\-nā\-tvam iti cet | \edlabel{thakur75-90.19}\label{thakur75-90.19} bhavatu tā\-vat nijā\-gantukabhedena dvividhaṃ sā\-marthyam | tathā\-pi tat prā\-tisvikaṃ vastusvalakṣaṇaṃ sadyaḥ kriyā\-dharmakam avaśyā\-bhyupagantavyam | tad yadi prā\-g api, prā\-g api kā\-ryaprasaṅgaḥ | atha paścā\-d eva, na tadā\- sthiro bhā\-vaḥ | \edlabel{thakur75-90.23}\label{thakur75-90.23} na ca kā\-ryaṃ sahakariṇo 'pekṣata iti yuktam, tasyā\-sattvā\-t | hetuś ca sann api yadi svakā\-ryaṃ na karoti, tadā\- tatkā\-ryam eva tan na syā\-t, svā\-tantryā\-t | \edlabel{thakur75-90.25}\label{thakur75-90.25} yac coktam – yo hi uttarakā\-ryajananasvabhā\-vaḥ sa katham ā\-dau kā\-ryaṃ kuryā\-t, atha kuryā\-t na tarhi tatkā\-ryakaraṇasvabhā\-vaḥ, na hi nī\-lotpā\-danasvabhā\-vaḥ pī\-tā\-dikam api karotī\-ti tad asaṅgatam | sthirasvabhā\-vatve bhā\-vasyottarakā\-lam evedaṃ na pū\-rvakā\-lam iti kuta etat | tadabhā\-vā\-c ca kā\-raṇam apy uttarakā\-ryasvabhā\-vam ity api kutaḥ | \edlabel{thakur75-90.29}\label{thakur75-90.29} kiṃ kurmaḥ, uttarakā\-lam eva tasya janmeti cet | sthiratve tadanupapadyamā\-nam asthiratā\-m ā\-diśatu | sthiratve 'py eṣa eva svabhā\-vas tasya yad uttarakṣaṇa eva kā\-ryaṃ karotī\-ti cet | na | pramā\-ṇabā\-dhite svabhā\-vā\-bhyupagamā\-yogā\-d iti na tā\-vad akṣaṇikasya kramikā\-ryakā\-ritvam asti | nā\-py akramikā\-ryakā\-ritvasambhavaḥ, dvitī\-ye 'pi kṣaṇe kā\-rakasvarū\-pasadbhā\-ve punar api kā\-ryakaraṇaprasaṅgā\-t | \edlabel{thakur75-91.4}\label{thakur75-91.4} kā\-rye niṣpanne tadviṣayavyā\-pā\-rā\-bhā\-vā\-d ū\-nā\- sā\-magrī\- na niṣpā\-ditaṃ niṣpā\-dayed iti cet | na | sā\-magrī\-sambhavā\-sambhavayor api sadyaḥ kriyā\-kā\-rakasvarū\-pasambhave janakatvam avā\-ryam iti prā\-g eva pratipā\-danā\-t | kā\-ryasya hi niṣpā\-ditatvā\-t punaḥ kartum aśakyatvam eva kā\-raṇam asamartham ā\-vedayati | \edlabel{thakur75-91.7}\label{thakur75-91.7} tad ayam akṣaṇike kramā\-kramikā\-ryakā\-ritvā\-bhā\-vo na siddhaḥ | na ca kramā\-kramā\-bhyā\-m aparaprakā\-rasambhavo yena tā\-bhyā\-m avyā\-ptau sandigdhavyatireko hetuḥ syā\-t | prakā\-rā\-ntaraśaṅkā\-yā\-ṃ tasyā\-pi dṛśyatvā\-dṛśyatvaprakā\-radvayadū\-ṣaṇe 'pi svapakṣe 'py anā\-śvā\-saprasaṅgā\-t | tasmā\-d anyonyavyavacchedasthitayor nā\-paraḥ prakā\-raḥ sambhavati | svarū\-pā\-praviṣṭasya vastuno 'vastuno vā\-tatsvabhā\-vatvā\-t | prakā\-rā\-ntarasyā\-pi kramasvarū\-pā\-praviṣṭatvā\-t | tathā\-tī\-ndriyasya sahakā\-riṇo 'dṛśyatve 'py ayogavyavacchedena dṛśyasahakā\-risahitasya dṛśyasyaiva sattvasya dṛśyakramā\-kramā\-bhyā\-ṃ vyā\-ptiḥ pratyakṣā\-d eva sidhyati | evaṃ kramā\-kramā\-bhyā\-m arthakriyā\-kā\-ritvaṃ vyā\-ptam iti kramā\-kramayor anyonyavyavacchedena sthitatvā\-d etatprakā\-radvayaparihā\-reṇā\-rthakriyā\-kā\-ritvam anyatra na gatam ity arthaḥ | ata evaitayor vinivṛttau nivartate || \edlabel{thakur75-91.17}\label{thakur75-91.17} \persName{trilocanasyā\-}pi vikalpatraye prathamadū\-ṣaṇam ā\-śrayā\-siddhidoṣaparihā\-rato nirastam | \edlabel{thakur75-91.18}\label{thakur75-91.18} dvitī\-yaṃ cā\-saṅgatam, vikalpajñā\-nena vyatirekasya pratī\-tatvā\-t | na hy abhā\-vaḥ kaścidvigrahavā\-n yaḥ sā\-kṣā\-tkartavyaḥ, api tu vikalpā\-d eva vyavahartavyaḥ | na hy abhā\-vasya vikalpā\-d anyā\- pratipattir apratipattir vā\- sarvathā\- | ubhayathā\-pi tadvyavahā\-rahā\-niprasaṅgā\-t | evaṃ vaidharmyadṛṣṭā\-ntasya hetuvyatirekasya ca vikalpā\-d eva pratipattiḥ | \edlabel{thakur75-91.22}\label{thakur75-91.22} tṛtī\-yam api dū\-ṣaṇam asaṅgatam | vyā\-pakā\-nupalambhena nirdoṣeṇa sattvasya kṣaṇikatvena vyā\-pter avyā\-hatatvā\-t | \edlabel{thakur75-91.23}\label{thakur75-91.23} tad ayaṃ vyā\-pakā\-nupalambho 'kṣaṇikasyā\-sattvam sattvasya tato vyatirekaṃ kṣaṇikatvena vyā\-ptiṃ ca sā\-dhayaty ekavyā\-pā\-rā\-tmakatvā\-d iti sthitam || \edlabel{thakur75-91.25}\label{thakur75-91.25} nanu vyā\-pakā\-nupalabdhir iti yady anupalabdhimā\-traṃ tadā\- na tasya sā\-dhyabuddhijanakatvam avastutvā\-t | na cā\-nyopalabdhir vyā\-pakā\-nupalabdhir abhidhā\-tuṃ śakyā\- bhū\-talā\-divad anyasya kasyacid anupalabdher iti cet | \edlabel{thakur75-91.27}\label{thakur75-91.27} tad asaṅgatam | dharmyupalabdher evā\-nyatrā\-nupalbdhitayā\- vyavasthā\-panā\-t | yathā\- hi neha śiṃśapā\- vṛkṣā\-bhā\-vā\-d ity atra vṛkṣā\-pekṣayā\- kevalapradeśasya dharmiṇa upalabdhir vṛkṣā\-nupalabdhiḥ | śiśapā\-pekṣayā\- ca kevalapradeśasya dharmiṇa upalabdhir eva śiṃśapā\-yā\- abhā\-vopalabdhir iti svabhā\-vahetuparyavasā\-yivyā\-pā\-ro vyā\-pakā\-nupalambhaḥ | tathā\- nityasya dharmiṇo vikalpabuddhyavasitasya kramikā\-ritvā\-kramikā\-ritvā\-pekṣayā\- kevalagrahaṇā\-d eva kramikā\-ritvā\-kramikā\-ritvā\-nupalambhaḥ | arthakriyā\-pekṣayā\- ca kevalapratī\-tir evā\-rthakriyā\-yogapratī\-tir iti vyā\-pakā\-nupalambhā\-ntarā\-d asya na kaścid viśeṣaḥ ||
	\pend
      

	  \pstart adhyavasā\-yā\-pekṣayā\- ca bā\-hye 'kṣaṇike vastuni vyā\-pakā\-bhā\-vā\-d vyā\-pyā\-bhā\-vasiddhivyavahā\-raḥ | adhyavasā\-yaś ca samanantarapratyayabalā\-yā\-tā\-kā\-raviśeṣayogā\-d agṛhī\-te 'pi pravartanaśaktir boddhavyaḥ | ī\-dṛśaś cā\-dhyavasā\-yo 'smaccitrā\-dvaitasiddhau nirvā\-hitaḥ | sa cā\-visaṃvā\-dī\- vyavahā\-raḥ parihartum aśakyaḥ | yad vyā\-pakaśū\-nyaṃ tadvyā\-pyaśū\-nyam iti | etasyaivā\-rthasyā\-nenā\-pi krameṇa pratipā\-danā\-t | ayaṃ ca nyā\-yo yathā\- vastubhū\-te dharmiṇi tathā\-vastubhū\-te 'pī\-ti ko viśeṣaḥ | 
	\pend
      

	  \pstart tathā\- hy ekajñā\-nasaṃsargy atra vikalpya eva | yathā\- ca hariṇaśirasi tenaikajñā\-nasaṃsargi śṛṅgam upalabdhaṃ śaśaśirasy api tena sahaikajñā\-nasaṃsargitvasambhā\-vanayaiva śṛṅgaṃ niṣidhyate, tathā\- nī\-lā\-dā\-v apariniṣṭhitanityā\-nityabhā\-ve kramā\-kramau svadharmiṇā\- sā\-rdham ekajñā\-nasaṃsargiṇau dṛṣṭau, yadi nitye bhavataḥ, nityagrā\-hijñā\-ne svadharmiṇā\- nityena sahaiva gṛhye yā\-tā\-m iti sambhā\-vanayā\- ekajñā\-nasaṃsargadvā\-rakam eva pratiṣidhyate | kathaṃ punar etasminn ity ajñā\-ne kramā\-kramayor asphuraṇam iti yā\-vatā\- kramā\-kramakroḍī\-kṛtam eva nityaṃ vikalpayā\-m iti cet | ata eva bā\-dhakā\-vatā\-ro viparī\-tā\-ropam antareṇa tasya vaiyarthyā\-t | \edlabel{thakur75-92.17}\label{thakur75-92.17} kā\-lā\-ntare 'py ekarū\-patayā\- nityatvam | kramā\-kramau ca kṣaṇadvaye bhinnarū\-patayā\- | tato nityatvasya kramā\-kramikā\-ryaśakteś ca parasparaparihā\-rasthitilakṣaṇatayā\- durvā\-ro virodha iti kathaṃ nitye kramā\-kramayor antarbhā\-vaḥ | anantarbhā\-vā\-c ca śuddhanityavikalpena dū\-rī\-kṛtakramā\-kramasamā\-ropeṇa katham ullekhaḥ | tataś ca pratiyogini nitye 'pi vikalpyamā\-na ekajñā\-nasaṃsargilakṣaṇaprā\-pte nityopalabdhir eva nityaviruddhasyā\-nupalabhyamā\-nasya kramā\-kramasyā\-nupalabdhiḥ | tata eva cā\-rthakriyā\-śakter anupalabdhiḥ | tasmā\-d vyā\-pakavivekidharmyupalabdhitayā\- na vyā\-pakā\-nupalambhā\-ntarā\-d asya viśeṣaḥ || \edlabel{thakur75-92.25}\label{thakur75-92.25} na tv etad avastu dharmitvopayogivastvadhiṣṭhā\-natvā\-t pramā\-ṇavyavasthā\-yā\- iti cet | kim idaṃ vastvadhiṣṭhā\-natvaṃ nā\-ma | kiṃ pamparayā\-pi vastunaḥ sakā\-śā\-d ā\-gatatvam, atha vastuni kenacid ā\-kā\-reṇa vyavahā\-rakā\-raṇatvam, vastubhū\-tadharmipratibaddhatvaṃ vā\- | \edlabel{thakur75-93.1}\label{thakur75-93.1} yady ā\-dyaḥ pakṣas tadā\- kramā\-kramasyā\-rthakriyā\-yā\-ś ca vyā\-ptigrahaṇagocaravastupratibaddhatvam asyā\-pi na kṣī\-ṇam | \edlabel{thakur75-93.2}\label{thakur75-93.2} na dvitī\-ye 'pi pakṣe doṣaḥ sambhavati, kṣaṇabhaṅgivastusā\-dhanopā\-yatvā\-d asya | \edlabel{thakur75-93.3}\label{thakur75-93.3} na cā\-ntimo 'pi vikalpaḥ kalpyate, tasyaiva nityavikalpasya vastuno dharmibhū\-tasya kramā\-kramavad bā\-hyanityopā\-dā\-naśū\-nyatvenā\-rthakriyā\-vad bā\-hyanityopā\-dā\-naśū\-nyatve prasā\-dhanā\-t | paryudā\-savṛttyā\- buddhisvabhā\-vabhū\-tā\-kṣaṇikā\-kā\-re vastubhū\-te dharmiṇi pratibaddhatvasambhavā\-t || \edlabel{thakur75-93.7}\label{thakur75-93.7} ayam eva nyā\-yo na vaktā\- bandhyā\-sutaś caitanyā\-bhā\-vā\-d ityā\-dau yojyaḥ | etena yathā\- vṛkṣā\-bhā\-vā\-dir antarbhā\-vayituṃ śakyate na tathā\-yam iti \persName{trilocano} 'pi nirastaḥ || \edlabel{thakur75-93.10}\label{thakur75-93.10} na ca kramā\-dyabhā\-vastrayī\-ṃ doṣajā\-tiṃ nā\-tikrā\-mati, abhā\-vadharmatve 'pi ā\-śrayā\-siddhidoṣaparihā\-rā\-t | \edlabel{thakur75-93.11}\label{thakur75-93.11} yat tv anena pramā\-ṇā\-ntarā\-n nityā\-nā\-m asattvasiddhau kramā\-divirahasyā\-bhā\-vadharmatā\- sidhyatī\-ty uktam, tadbā\-lasyā\-pi durabhidhā\-nam | nityo hi dharmī\- | asattvaṃ sā\-dhyam | kramikā\-ryakā\-ritvā\-kramikā\-ryakā\-ritvaviraho hetuḥ | asya cā\-bhā\-vadharmatvaṃ nā\-mā\-sattvalakṣaṇasvasā\-dhyā\-vinā\-bhā\-vitvam ucyate | tac ca kramā\-krameṇa sattvasya vyā\-ptisiddhau sattvasya vyā\-pyasyā\-bhā\-vena kramā\-kramasya vyā\-pakasya viraho vyā\-ptaḥ sidhyatī\-ty abhā\-vadharmatvaṃ prā\-g eva vidhyor vyā\-ptisā\-dhanā\-t pratyakṣā\-d anumā\-nā\-d ekasmā\-d vā\- pramā\-ṇā\-ntarā\-t siddham iti netaretarā\-śrayadoṣaḥ | \edlabel{thakur75-93.18}\label{thakur75-93.18} na ca sattā\-yā\-m ivā\-sattā\-yā\-m api tulyaḥ prasaṅgo bhinnanyā\-yatvā\-t | vastubhū\-taṃ hi tatra sā\-dhyaṃ sā\-dhanaṃ ca | tayor dharmy api vastv eva yujyate | \edlabel{thakur75-93.19}\label{thakur75-93.19} vastunas tu pratyakṣā\-numā\-nā\-bhyā\-m eva siddhiḥ | tayor abhā\-ve niyamenā\-śrayā\-siddhir iti yuktam | asattā\-sā\-dhane tv avastudharmo hetur avastuvikalpamā\-trasiddhe dharmiṇi nā\-śrayā\-siddhidoṣeṇa dū\-ṣayituṃ śakyaḥ | tathā\-kṣaṇikasya kramayaugapadyā\-bhyā\-m arthakriyā\-virodhaḥ sidhyaty eva | \edlabel{thakur75-93.22}\label{thakur75-93.22} tathā\- vikalpā\-d evā\-kṣaṇiko virodhī\- siddhaḥ | vikalpollikhitaś cā\-sya svabhā\-vo nā\-para ity api vyavahartavyam | anyathā\- tadanuvā\-dena kramā\-kramā\-dirahitatvā\-diniṣedhā\-dikam ayuktam, tatsvarū\-pasyā\-nullekhā\-d anyasyollekhā\-d ity akṣaṇikaśaśaviṣā\-ṇā\-diśabdā\-nuccā\-raṇaprasaṅgaḥ | asti ca | ato yathā\- pramā\-ṇā\-bhā\-ve 'pi vikalpasiddhasya bandhyā\-sutā\-deḥ saundaryā\-diniṣedho 'nurū\-pas tathā\- vikalpopanī\-tasyaivā\-kṣaṇikarū\-pasya tata eva pratyanī\-kā\-kā\-reṇa saha virodhavyavasthā\-yā\-ṃ kī\-dṛśo doṣaḥ syā\-t | yadi cā\-kṣaṇikā\-nubhavā\-bhā\-vā\-d virodhapratiṣedhas tarhi bandhyā\-putrā\-dyanubhavā\-bhā\-vā\-d eva saundaryā\-diniṣedho 'pi mā\- bhū\-t || \edlabel{thakur75-94.5}\label{thakur75-94.5} nanv evaṃ virodhasyā\-pā\-ramā\-rthikatvam | taddvā\-reṇa kṣaṇabhaṅgasiddhir apy apā\-ramā\-rthikī\- syā\-d iti cet | \edlabel{thakur75-94.6}\label{thakur75-94.6} na hi virodho nā\-ma vastvantaraṃ kiñcid ubhayakoṭidattapā\-dasambandhā\-bhidhā\-nam iṣyate 'smā\-bhir upapadyate vā\- yenaikasambandhino vastutvā\-bhā\-ve 'pā\-ramā\-rthikaṃ syā\-t | yathā\- tv iṣyate tathā\- pā\-ramā\-rthika eva | viruddhā\-bhimatayor anyonyasvarū\-paparihā\-ramā\-traṃ virodhā\-rthaḥ | sa ca bhā\-vā\-bhā\-vayoḥ pā\-ramā\-rthika eva | na bhā\-vo 'bhā\-varū\-pam ā\-viśati, nā\-py abhā\-vo bhā\-varū\-paṃ praviśatī\-ti yo 'yam anayor asaṃkaraniyamaḥ sa eva pā\-ramā\-rthiko virodhaḥ | kā\-lā\-ntaraikarū\-patayā\- hi nityatvam | kramā\-kramau kṣaṇadvaye 'pi bhinnarū\-patayā\- | tato nityatvakramā\-kramikā\-ryakā\-ritvayor bhā\-vā\-bhā\-vavad virodho 'sty eva || \edlabel{thakur75-94.14}\label{thakur75-94.14} nanu nityatvaṃ kramayaugapadyavattvaṃ ca viruddhau dharmau vidhū\-ya nā\-paro virodho nā\-ma, kasya vā\-stavatvam iti cet | \edlabel{thakur75-94.15}\label{thakur75-94.15} na | na hi dharmā\-ntarasya sambhavena virodhasya pā\-ramā\-rthikatvaṃ brū\-maḥ | kiṃ tu viruddhayor dharmayoḥ sadbhā\-ve | anyathā\- virodhanā\-madharmā\-ntarasambhave 'pi yadi na viruddhau dharmau kva pā\-ramā\-rthikavirodhasambhavaḥ | viruddhau ced dharmau tā\-vataiva tā\-ttviko virodhavyavahā\-raḥ kim apareṇa pratijñā\-mā\-trasiddhena virodhanā\-mnā\- vastvantareṇa | \edlabel{thakur75-94.20}\label{thakur75-94.20} tad ayaṃ pū\-rvapakṣasaṃkṣepaḥ gnityaṃ nā\-sti na vā\- pratī\-tiviṣayas tenā\-śrayā\-siddhatā\- hetoḥ svā\-nubhavasya ca kṣatir ataḥ kṣiptaḥ sapakṣo 'pi ca | śū\-nyaś ca dvitayena sidhyati na cā\-sattā\- 'pi sattā\- yathā\- no nityena virodhasiddhir asatā\- śakyā\- kramā\-der api || J 89,16-19; cf. R 87,24-27 iti | \edlabel{thakur75-94.25}\label{thakur75-94.25} atra siddhā\-ntasaṃkṣepaḥ hdharmasya kasyacid avastuni mā\-nasiddhā\- bā\-dhā\-vidhivyavahṛtiḥ kim ihā\-sti no vā\- | kvā\-py asti cet katham iyanti na dū\-ṣaṇā\-ni nā\-sty eva cet svavacanapratirodhasiddhiḥ || J 89,21-24; cf. R 88,4-7 \edlabel{thakur75-95.1}\label{thakur75-95.1} tad evaṃ nityaṃ na kramikā\-ryakā\-ritvā\-kramikā\-ryakā\-ritvayogī\-ti paramā\-rthaḥ | tataś ca sattā\-yuktam api naiveti paramā\-rthaḥ | tataś ca kṣaṇikā\-kṣaṇikaparihā\-reṇa rā\-śyantarā\-bhā\-vā\-d akṣaṇikā\-n nivartamā\-nam idaṃ sattvaṃ kṣaṇika eva viśrā\-myat tena vyā\-ptaṃ sidhyatī\-ti sattvā\-t kṣaṇikatvasiddhir avirodhinī\- ||
	\pend
      
	    
	    \stanza[\smallbreak]
prakṛtiḥ sarvadharmā\-ṇā\-ṃ yad bodhā\-n muktir iṣyate |&sa eva tī\-rthyanirmā\-thī\- kṣaṇabhaṅgaḥ prasā\-dhitaḥ ||\&[\smallbreak]


	

	  \pstart iti kṛtir ayaṃ Ratnakī\-rteḥ ||
	\pend
      \label{Pramāṇāntarbhāvaprakaraṇam}\edlabel{Pramāṇāntarbhāvaprakaraṇam}
	  
	% new div opening: depth here is 1
	
\section[{Pramā\-ṇā\-ntarbhā\-vaprakaraṇam}]{Pramā\-ṇā\-ntarbhā\-vaprakaraṇam}\label{Pramāṇāntarbhāvaprakaraṇam}
	    
	    \stanza[\smallbreak]
\edlabel{thakur75-96.4}\flagstanza{\tiny\textenglish{...5-96.4}}pramā\-ṇadvitayā\-d anyapramā\-ṇagaṇadū\-ṣaṇam |&nā\-pū\-rvam ucyate tat tu prayogeṇā\-tra mudryate ||\&[\smallbreak]


	[[Chapter starts in \cref{RNAms}.]]

	  \pstart iha khalu pramā\-ṇamā\-tre na kecid vipratipadyante | antataś \name{cā\-rvā\-kasyā\-}pi saṃpratipatteḥ | \persName{pramā\-ṇamā\-trocchedavā\-dī\-} ca tattadā\-ṅśakya pratividhā\-nā\-d \persName{asmadgurubhir} avajñā\-taḥ
	\pend
      
	    
	    \stanza[\smallbreak]
pramā\-ṇam apramā\-ṇaṃ ced vicā\-rā\-vasaro hataḥ |&bruvatā\- niyataṃ kiñcit sā\-dhyaṃ vā\- bā\-dhyam eva vā\- ||&tatrā\-yuktiṃ bruvā\-ṇasya ślā\-ghā\- sadasi kī\-dṛśī\- |&nā\-numā\-yā\-ḥ parā\- yuktiḥ kiṃ siddhaṃ tadanā\-dare ||&svī\-kṛtā\- tena sety asmā\-t tanmatyā\- bā\-dhanaṃ yadi |&abā\-dhane 'syā\-ḥ svī\-kā\-rā\-t tadbhiyā\- bā\-dha\leavevmode\textsuperscript{\rmlatinfont\tiny [pb in\cite{RNAms}]}\label{RNAms_51b}naṃ katham ||&sā\-dhyaṃ na kiñcid iti cet bā\-dhā\-yā\- api sā\-dhyatā\- |&sā\-pi neti vaco vyarthaṃ praśnamā\-tre 'pi kiṃ phalam ||&phalaṃ yadi giraḥ kvā\-pi nā\-nyat tac cā\-vabodhanā\-t |&vā\-caḥ pratyā\-yane śaktā\- nā\-kṣadhū\-mā\-di sundaram ||&saṃvṛtau mā\-nam iṣṭaṃ ced vicā\-ro 'py eṣa saṃvṛtiḥ |&saṃvṛtā\-v api neṣṭaṃ ced vadan jetā\- yathā\- tathā\- || \footnote{\begin{english}(JNA 363f.)\end{english}}&saṃvṛtiś ca vinā\- mā\-naṃ vā\-ṅmā\-treṇa na sidhyati |&mā\-nato yadi durvā\-raḥ pramā\-ṇasya parigrahaḥ ||\&[\smallbreak]


	

	  \pstart \persName{ā\-cā\-ryo} 'py ā\-ha—
	\pend
      
	    
	    \stanza[\smallbreak]
aniṣṭeś cet pramā\-ṇaṃ hi sarveṣṭī\-nā\-ṃ nibandhanam |&bhā\-vā\-bhā\-vavyavasthā\-ṃ kaḥ kartuṃ tena vinā\- prabhuḥ ||\footnote{\begin{english}(PV IV 215)\end{english}}\&[\smallbreak]


	

	  \pstart iti |
	\pend
      

	  \pstart tad evaṃ pramā\-ṇamā\-trā\-pratikṣepe pratyakṣaṃ tā\-vad ā\-dau gaṇanī\-yam, tanmū\-latvā\-d aparapramā\-ṇopapatteḥ | na ca cā\-rvā\-ko 'py anumā\-nam anavasthā\-pya sthā\-tuṃ prabhavati, vyā\-pā\-ratrayakaraṇā\-t | 
	\pend
      

	  \pstart tac chā\-stre hi pratyakṣetarasā\-mā\-nyayoḥ pramā\-ṇetaravidhā\-naṃ lakṣaṇapraṇayanato vidhā\-tavyam | tac ca lakṣaṇaṃ pratyakṣe dharmiṇi lakṣye prā\-mā\-ṇye pratyetavye svabhā\-vo hetuḥ | parabuddhipratipattau ca kā\-ryā\-divyā\-pā\-raḥ kā\-ryahetuḥ | paralokapratiṣedhe ca dṛśyā\-nupalambho 'ṅgī\-kartavya iti katham anumā\-nā\-palā\-paḥ | yad ā\-cā\-ryaḥ 
	\pend
      
	    
	    \stanza[\smallbreak]
pramā\-ṇetarasā\-mā\-nyasthiter anyadhiyo gateḥ |&pramā\-ṇā\-ntarasadbhā\-vapratiṣedhā\-c ca kasyacit ||\footnote{\begin{english}
	    \begin{verse}
	  \textit{pramā\-ṇetarasā\-mā\-nyasthiter anyadhiyo gateḥ /}\\
	    \textit{pramā\-ṇā\-ntarasadbhā\-vaḥ pratiṣedhā\-c ca kasyacit //}\\
	    
	    \end{verse}
	   [[(PVin I 2)]]\end{english}}\&[\smallbreak]


	

	  \pstart api ca
	\pend
      
	    
	    \stanza[\smallbreak]
arthasyā\-sambhave 'bhā\-vā\-t pratyakṣe 'pi pramā\-ṇatā\- |&pratibaddhasvabhā\-vasya taddhetutve samaṃ dvayam ||\footnote{\begin{english}
	    \begin{verse}
	  \textit{arthasyā\-sambhave 'bhā\-vā\-t pratyakṣe 'pi pramā\-ṇatā\- /}\\
	    \textit{pratibaddhasvabhā\-vasya taddhetutve samaṃ dvayam //}\\
	    
	    \end{verse}
	   (PVin I 3)\end{english}}\&[\smallbreak]


	

	  \pstart ity anumā\-nam api pramā\-ṇam | prā\-mā\-ṇyaṃ ca pramā\-ṇā\-ntarā\-gṛhī\-taniścitapravṛttiviṣayā\-rthatayā\- \edtext{tatprāpaṇe}{\Afootnote{tatprā\-paṇe \cite{}; tatpra\add{ā\-}paṇe\footnote{The ā\--marker was written above the letter.} \cite{}  {\rmlatinfont [App type: corr]}}} śaktiḥ ||
	\pend
      

	  \pstart nanv astu prā\-paṇe śaktiḥ prā\-mā\-ṇyam, paramasaunā\-rthā\-d utpatteḥ, api tv arthadarśanā\-d iti cet | kim idam arthadarśanam | arthasya dharmo dṛśyatvam | jñā\-nasya dharmo draṣṭṛtvam | prathamapakṣe nī\-latvavad dṛśyatvasyā\-pi sā\-dhā\-raṇatvā\-d ekagocaro 'rthaḥ sarvagocaraḥ syā\-t | na hi pratipuruṣam arthā\-nā\-ṃ bhedo nairā\-tmyaprasaṅgā\-t | dvitī\-yapakṣe tu katham anyasmin jñā\-nasvabhā\-ve draṣṭṛtve saty anyasyā\-sambaddhasyā\-rthasya pratyā\-śā\- syā\-t | draṣṭṛatvaṃ dṛśyatvam antareṇā\-nupapadyamā\-naṃ tadā\-kṣipatī\-ti cet | nanu jñā\-nā\-rthayor utpattisā\-rū\-pyabalato draṣṭḥdṛśyatvavyavasthā\-panam etat | anabhyupagame draṣṭṛtvaṃ dṛśyatvaṃ ca na sambhavatī\-ti kiṃ kenā\-kṣipyatā\-m | bhavatu vā\- prakā\-rā\-ntareṇā\-pi draṣṭṛdṛśyabhā\-vas tathā\-pi bhede saty avyabhicā\-\edtext{ra du}{\Afootnote{ra\deletion{\unclear{ḥ}\gap{}\add{\unclear{}}}du \cite{}; ras ta \cite{}  {\rmlatinfont [App type: var]}}}tpattir eva prā\-ptinimittam | sā\- ca prā\-paṇaśaktiḥ pratyakṣā\-numā\-nayor aviśiṣṭeti pramā\-ṇe eva | \edlabel{thakur75-97.23}\label{thakur75-97.23} nanv anyad api śā\-bdopamā\-nā\-dikaṃ pramā\-ṇam asti | tathā\- hi śabdā\-c codanā\-rū\-pā\-d asannikṛṣṭe 'rthe svargā\-dau yaj jñā\-nam utpadyate tad api śā\-bdaṃ jñā\-naṃ pramā\-ṇam eva | pratyayitoditavā\-kyaprasū\-taṃ ca jñā\-naṃ pramā\-ṇam | yad ā\-ha Kumā\-rilaḥ
	\pend
      
	    
	    \stanza[\smallbreak]
tac cā\-kartṛkato vā\-kyā\-d anyā\-d vā\- pratyayito〔?〕ditā\-t |\footnote{Find this! (Not in e-text of śv.)}\&[\smallbreak]


	

	  \pstart iti |
	\pend
      

	  \pstart tatra yadā\- śabdasamutthaṃ jñā\-naṃ pramā\-ṇaṃ tadopā\-dā\-nā\-dibuddhiḥ phalam | yadā\- tu śabdas tadā\- tadā\-lambanaṃ jñā\-naṃ phalam iti Naiyā\-yikasya punaḥ: ā\-ptopadeśaḥ śabdaḥ \footnote{\begin{english}(NSū\- 1.1.7)\end{english}}, iti śabdapramā\-ṇalakṣaṇasū\-tram | tatra śabda iti lakṣyapadam | ā\-ptopadeśa iti lakṣaṇapadam | asyā\-yaṃ saṃkṣepā\-rthaḥ | ā\-ptopadiṣṭaḥ śabdaḥ pramā\-ṇam iti | ā\-ptaś ca sā\-kṣā\-tkṛtaheyopā\-deyatattvo yathā\-dṛṣṭasya cā\-rthasyā\-cikhyā\-sayā\- prayukta upadeṣṭā\- abhidhī\-yate | pramā\-ṇaphalavyavasthā\- ca pū\-rvavad draṣṭavyeti |
	\pend
      

	  \pstart tathā\- \name{Mī\-mā\-ṃsakā\-nā\-m} upamā\-naṃ pramā\-ṇam | yad uktaṃ \persName{Śabarasvā\-minā\-} \edlabel{kāśikā1_start}\label{kāśikā1_start}upamā\-nam api sā\-dṛśyam \edtext{asannikṛṣṭe 'rthe}{\Afootnote{ \cite{} \cite{}°asannikṛṣṭatve \cite{}  {\rmlatinfont [App type: emendation]}}} 'rthe buddhim utpā\-dayati | yathā\- gavayadarśanaṃ goḥ smaraṇasyeti |\footnote{Cf. \cref{śabara_bhāṣya}: upamā\-nam api sā\-dṛśyam asaṃnikṛṣṭe 'rthe buddhim utpā\-dayati, yathā\- gavayadarśanaṃ gosmaraṇasya.}
	\pend
      

	  \pstart asyā\-yam arthaḥ | ekatra dṛśyamā\-naṃ sā\-dṛśyaṃ kartṛ | pratiyogyantare dṛśyamā\-napratiyogisā\-dṛśyaviśiṣṭatayaitatsā\-dṛśyaviśiṣṭo 'sau ity asannikṛṣṭe 'rthe yā\-ṃ buddhim utpā\-dayati tadupamā\-naṃ pramā\-ṇam iti yat tadoradhyā\-hā\-ra iti |\edlabel{kāśikā1_end}\label{kāśikā1_end}\edtext{}{\lemma{---}\Afootnote{upamā\-nam api sā\-dṛśyam asannikṛṣṭe 'rthe buddhim utpā\-dayati. yathā\- gavayadarśanaṃ gosmaraṇasyeti bhā\-ṣyam. asyā\-yaṃ tā\-tparyā\-rthaḥ -- upamā\-nam api na parī\-kṣaṇī\-yam, evaṃ lakṣaṇakatvenā\-vyabhicā\-rā\-d iti. avayavā\-rthas tv ekatra dṛśyamā\-naṃ sā\-dṛśyaṃ pratiyogyantare dṛśyamā\-napratiyogisā\-dṛśyaviśiṣṭatayā\-sannikṛṣṭe 'rthe yā\-ṃ buddhim utpā\-dayati etatsā\-dṛśyaviśiṣṭo 'sā\-v iti, sopamā\-nam iti yattadoradhyā\-hā\-raḥ. na ca vā\-cyaṃ viṣayaviśeṣā\-nupā\-dā\-nā\-t kathaṃ sā\-dṛśyaviśiṣṭaviṣayā\- buddhir avagamyata iti, prasiddhapramā\-ṇā\-nuvā\-dena hy atrā\-parī\-kṣā\- pratipā\-dyate. loke ca sā\-dṛśyaviśiṣṭaviṣayaiva \cite{}  {\rmlatinfont [App type: parallel]}}} tasmā\-t samaratī\-ti smaraṇaṃ puruṣaḥ | tenā\-yam arthaḥ - yathā\- gavaye dṛśyamā\-naṃ sā\-dṛśyaṃ gā\-ṃ smarato manuṣyasya etatsā\-dṛśyaviśiṣṭo 'sau gaur iti buddhim utpā\-dayatī\-ti |
	\pend
      

	  \pstart na cedam upamā\-naṃ smaraṇaṃ kartavyam, gavayasā\-dṛśyaviśiṣṭasya gor goviśiṣṭasya ca sā\-dṛśyasya prameyatvā\-t | gosā\-dṛśyayor viśeṣaṇaviśeṣyabhā\-vasyopamā\-napramā\-ṇaviṣayasya gogrā\-hiṇā\- gavayagrā\-hiṇā\- vā\- pratyakṣeṇa kenacid agrahaṇā\-t | yad ā\-ha Bhaṭṭaḥ
	\pend
      
	    
	    \stanza[\smallbreak]
\&[\smallbreak]


	

	  \pstart na ca grahaṇam antareṇa smaraṇam asti | tasmā\-n nopamā\-naṃ smaraṇam ataḥ pramaṇam iti | Naiyā\-yikā\-dī\-nā\-ṃ tū\-pamā\-nasū\-tram,
	\pend
      
	    
	    \stanza[\smallbreak]
prasiddhasā\-dharmyā\-t sā\-dhyasā\-dhanam upamā\-nam iti |\footnote{\begin{english}(NSū\- 1.1.6)\end{english}}\&[\smallbreak]


	

	  \pstart asyā\-yam arthaḥ | prasiddhaṃ sā\-dharmyaṃ yasya tasmā\-d gavayā\-deḥ sā\-dhyasya saṃjñā\-saṃjñisambandhasya sā\-dhanaṃ siddhis tadupamā\-naphalam | samā\-khyā\-sambandhapratipattihetur upamā\-nam ity arthaḥ | ayam asya prapañcaḥ | yaḥ pratipattā\- gā\-ṃ jā\-nā\-ti na gavayam, ā\-diṣṭaś ca svā\-minā\- gacchā\-raṇyaṃ gavayamā\-nayā\-smā\-d iti, gavayaśabdavā\-cyam artham ajā\-nā\-no vanecaram anyaṃ vā\- tajjñaṃ pṛṣṭavā\-n, bhrā\-taḥ kī\-dṛśo gavaya iti | tena cā\-diṣṭaṃ yathā\- gaus tathā\- gavaya iti | tasya śrutā\-tideśavā\-kyasya kasyā\-ñcid araṇyā\-nyā\-m upagatasyā\-tideśavā\-kyā\-rthsmaraṇasahakā\-ri yad gavayasā\-rū\-pyajñā\-naṃ tatprathamata evā\-sau gavayaśabdavā\-cyo 'rtha iti pratipattiṃ prastuvā\-nam upamā\-naṃ pramā\-ṇam iti |
	\pend
      

	  \pstart tathā\-rthā\-pattisaṃjñaṃ pramā\-ṇaṃ mī\-mā\-ṃsakasya | arthā\-pattir api dṛṣṭaḥ śruto vā\-rtho 'nyathā\- nopadyamā\-no yad arthā\-ntaraṃ parikalpayati sā\-rthā\-pattiḥ | yathā\- jī\-vati devadatte gṛhā\-bhā\-vadarśanena bahirbhā\-vasyā\-rthasya parikalpanā\- | asyā\-yam arthaḥ | pratyakṣā\-dibhiḥ ṣaḍbhiḥ pramā\-ṇaiḥ prasiddho yo 'rthaḥ sa yena vinā\- na yujyate tasyā\-rthasya kalpanam arthā\-pattir iti | sā\- ca ṣaṭpramā\-ṇapū\-rvikā\- ṣaṭprakā\-raiveti ||
	\pend
      

	  \pstart pratyakṣā\-numā\-nā\-dipramā\-ṇapañcakā\-bhā\-vasvabhā\-vam abhā\-vā\-khyaṃ pramā\-ṇam | prameyaṃ ghaṭā\-dyabhā\-vaḥ | nā\-stī\-ha ghaṭā\-dī\-ti jñā\-naṃ ghaṭā\-dyabhā\-vā\-lambanaṃ phalam | yadā\-ha Kumā\-rilaḥ
	\pend
      
	    
	    \stanza[\smallbreak]
\&[\smallbreak]


	[[(ŚV XIII 11; 1)]]

	  \pstart iti | 
	\pend
      

	  \pstart etā\-ni ṣaṭ pramā\-ṇā\-ni pratyakṣā\-dī\-ny asaṃkī\-rṇasvasvalakṣaṇayogitvā\-d anyā\-praviṣṭasvabhā\-vā\-ni pratyetavyā\-nī\-ti ||
	\pend
      

	  \pstart atrocyate | codanā\-yā\-s tā\-vad bā\-hye 'rthe pratibandhā\-bhā\-vā\-n na prā\-mā\-ṇyam | prayogaḥ - yasya yatra pratibandho nā\-sti na tasya tatra prā\-mā\-ṇyam | yathā\- dahane 'pratibaddhasya rā\-sabhasya | apratibaddhā\-ś ca bahirarthe vaidikā\-ḥ śabdā\-ḥ iti vyā\-pakā\-nupalabdhiḥ | na tā\-vad ayam asiddho hetuḥ | śabdā\-nā\-ṃ vastutaḥ pratibandhā\-bhā\-vā\-t | pratibaddhasvabhā\-vatā\- hi pratibandhaḥ | na ca sā\- nirnibandhanā\-, sarveṣā\-ṃ sarvatra pratibaddhasvabhā\-vatā\-prasaṅgā\-t | nibandhanaṃ cā\-syā\-s tā\-dā\-tmyatadutpattibhyā\-m anyan nopalabhyate, atatsvabhā\-vasyā\-tadutpatteś ca tatrā\-pratibaddhasvabhā\-vatvā\-t | na hi śabdā\-nā\-ṃ bahirarthasvabhā\-vatā\-sti bhinnapratibhā\-sā\-vabodhaviṣayatvā\-t | nā\-pi śabdā\- bahirarthā\-d upajā\-yante, artham antareṇā\-pi puruṣasyecchā\-pratibaddhavṛtteḥ śabdasyotpā\-dadarśanā\-t |
	\pend
      

	  \pstart nanu yogyatayaiva kiñcit pratibaddhasvabhā\-vam upalabhyate | yathā\- cakṣur indriyaṃ rū\-pe | cakṣuḥ khalu vyā\-pā\-ryamā\-ṇaṃ rū\-pam evopalabhyati | tathaivaite vaidikā\-ḥ śabdā\-s tā\-dā\-tmyatadutpattiviyuktā\- api yogyatā\-mā\-treṇā\-tī\-ndriyam arthaṃ bodhayiṣyanti tat kathaṃ tā\-dā\-tmyatadutpattivirahamā\-treṇā\-pratibandho yenaivaṃ vyā\-pakā\-nupalabdhiḥ sidhyatī\-ti | naiṣa doṣaḥ | yataś cakṣur indriyam api rasā\-diparihā\-reṇa rū\-pa eva prakā\-śakatvena pratiniyataṃ tatkā\-ryatvā\-t | rū\-paṃ hi cakṣur upakaroti | na sattā\-mā\-treṇa cakṣū\- rū\-paṃ prakā\-śayati, vyavahitasyā\-pi rū\-popalabdhiprasaṅgā\-t | tasmā\-d rū\-pā\-d yogyadeśasannihitā\-t tajjñā\-najananayogyatā\-m ā\-sā\-dya cakṣū\- rū\-pajñā\-nam utpā\-dayattatkā\-ryam iti vyaktam avasī\-yate | anyathā\- tadupakā\-rā\-napekṣasya tasyā\-pi tatprakā\-śananiyamo nopapadyate | na hy anupakā\-ryatvā\-viśeṣe cakṣū\- rū\-pasyaiva prakā\-śakam, na rasā\-der iti ghaṭā\-m upaiti niyamaḥ | ayam eva tarhi niyamaḥ kuto yad rū\-peṇaiva cakṣur upakartavyam, na rasā\-dineti | yadi vastuvaśā\-d eva rū\-pam upakaroti na rasā\-dikam, hanta tarhi yathopakā\-ryatvaṃ prati niyamaś cakṣuṣo rū\-peṇa, tathā\- śabdā\-nā\-m api svā\-bhā\-vika evā\-stu bahirarthapratyā\-yananiyama iti |
	\pend
      

	  \pstart atrocyate | na cakṣuṣaḥ svā\-bhā\-viko rū\-popakā\-ryatā\-niyamaḥ, kasyacid vastunaḥ svā\-bhā\-vikatvā\-nupapatteḥ | tathā\- hi svā\-bhā\-vikatvaṃ vastudharmasyā\-nujā\-nā\-naḥ praṣṭavyaḥ - kiṃ svā\-bhā\-vika iti svato bhavati, ā\-hosvit parataḥ, athā\-hetutaḥ | yadi svato bhavati, tad asaṅgatam, svā\-tmani kriyā\-virodhā\-t | athā\-hetutaḥ, tad ayuktam, ahetor deśā\-diniyamā\-yogā\-t | tasmā\-n na svā\-bhā\-viko rū\-popakā\-ryatā\-pratiniyamaś cakṣuṣaḥ | kiṃnibandhanas tarhi svahetupratibaddha iti, brū\-maḥ - cakṣuḥ khalu svahetunā\- janyamā\-naṃ tā\-dṛśam eva janitam yadrū\-popakartavyam eva bhavati | rū\-pam api tā\-dṛśam eva svahetunā\- janitaṃ yat tad upakā\-rakasvabhā\-vam |
	\pend
      

	  \pstart śabdā\-nā\-m api sa svabhā\-vaḥ svahetupratibaddho yenaite bā\-hyā\-rthā\-vyabhicā\-riṇa iti cet | na śakyam evam abhidhā\-tum, nityatvā\-bhyupagamā\-d vedavā\-kyā\-nā\-m | athā\-nityatvam abhyupagamyā\-yam ā\-kṣepaḥ parihartum iṣyate, tad api duṣkaram, doṣā\-ntaraprasaṅgā\-t | yadi svahetunaiva te niyamā\-rthopadarśanaśaktimanto janitā\-ḥ, tadā\-vyutpannasamayasyā\-pi svā\-rtham avabodhayeyuḥ | yathā\- cakṣuḥ svaheto rū\-paprakā\-śakam utpannaṃ sat prakā\-śayaty eva rū\-pam asaṅketavido 'pi, na ca śabdā\-d uccaritā\-t prā\-gapratī\-tasamayasyā\-pi viśeṣā\-vagamaḥ samasti | tasmā\-n na svahetupratibaddhaś cakṣurā\-der iva śabdā\-nā\-m arthapratipā\-dananiyama iti niścayaḥ || 
	\pend
      

	  \pstart atha svahetubhir evā\-yam ī\-dṛśas teṣā\-ṃ svabhā\-vo datto yena te saṃketaviśeṣasahā\-yā\- eva kam apy artham avabodhayanti | na tarhi saṅketaparā\-vṛttau padā\-rthā\-ntaravṛttayo bhaveyuḥ | yadi hy ayam agnihotraśabdaḥ saṃketā\-pekṣo yā\-gaviśeṣapratipā\-dakaḥ, kathaṃ saṅketā\-nyatvenā\-rthā\-ntaraṃ pratipā\-dayati | na hi kṣityā\-dyapekṣeṇa bī\-jena svahetor aṅkurajananasvabhā\-venotpannena rā\-sabhaḥ śakyo janayitum, tathā\- śabdo 'pi yad arthapratipā\-dananiyatas tam eva prakā\-śayet || 
	\pend
      

	  \pstart atha tattatsaṅketā\-pekṣas tattadarthapratyā\-yanayogya evā\-yaṃ jā\-ta ity ucyate | tad api na prasutopayogi | na hy evam asya prā\-mā\-ṇyam avatiṣṭhate | yadā\- hi saṅketenā\-puruṣā\-rthapratipā\-danam api sambhā\-vyata eva, tadā\- na śakyam upakalpayituṃ kim ayam abhimatasyaivā\-rthasya dyotako na veti | tarhi vā\-cyavā\-cakalakṣaṇaḥ śabdā\-rthayoḥ sambandho bhaviṣyati | tathā\- cā\-ha
	\pend
      
	    
	    \stanza[\smallbreak]
vā\-cyavā\-cakasambandhā\-ḥ santi yady api vā\-stavā\-ḥ |&saṅketair anabhivyaktā\- na te 'rthavyaktihetavaḥ ||\&[\smallbreak]


	

	  \pstart iti cet | nanu tasya vā\-stavatve 'saṅketavido 'py arthapratipattir bhaved ity uktam, saṅketā\-pekṣā\-yā\-ṃ cā\-rthā\-ntare na pravartetetyā\-dyabhihitam | ataḥ pū\-rvam evā\-yaṃ pratyā\-khyā\-to vā\-cyavā\-cakalakṣaṇaḥ sambandhaḥ | tasmā\-n na bahirarthe pratibandhaḥ śabdā\-nā\-m iti nirṇayaḥ ||
	\pend
      

	  \pstart tataś ca nā\-siddho hetuḥ ||
	\pend
      

	  \pstart nā\-py viruddhaḥ, viparyayavyā\-ptyabhā\-vā\-t | tadabhā\-vaś ca sapakṣe vṛttyupadarśanā\-t | na hi viruddhasya sā\-dharmyavati dharmiṇi sadbhā\-vo yuktaḥ, sā\-dhyaviparyayasya tatrā\-bhā\-vā\-t | na ca vyā\-pakam antareṇa vyā\-pyasya sambhavaḥ, tatpracyutiprasaṅgā\-t || 
	\pend
      

	  \pstart nā\-py anaikā\-ntiko hetuh, viparyaye bā\-dhakapramā\-ṇasambhavā\-t | prā\-mā\-ṇyapratiṣedhe hi sā\-dhye prā\-mā\-ṇyam eva vipakṣaḥ | na ca tasmin pratibandhā\-bhā\-valakṣaṇo hetur asti, svaviruddhena pratibandhena vyā\-ptatvā\-t | na khalv ayaṃ prā\-deśikaḥ pramā\-ṇaśabdo jñā\-neṣu nirnibandhana eva, sarvajñā\-neṣu prā\-mā\-ṇyavyapadeśaprasaṅgā\-t | nibandhanaṃ ca svaviṣayapratibandhā\-d anyan nopapadyate | tasmā\-t pramā\-ṇasya pramā\-ṇavyapadeśaviṣayatvaṃ svaviṣayapratibandhena vyā\-ptam | ataḥ pramā\-ṇe dharmiṇi vipakṣe prā\-mā\-ṇyasya viruddhavyā\-ptasyopalambhena vipakṣe vyavacchedasiddher nā\-naikā\-ntiko hetuḥ |
	\pend
      

	  \pstart na cā\-nyo doṣaḥ sambhavī\- | tasmā\-n nirastā\-śeṣadoṣeṇa hetunā\- yat prasiddhaṃ tad upā\-deyam eva satā\-m iti paṇḍitaśrī\-jitā\-ripā\-dair eva vedā\-prā\-mā\-ṇye darśitam | 
	\pend
      

	  \pstart evaṃ ca vaidikaśabdā\-nā\-ṃ prā\-mā\-ṇye niraste tadutthaṃ jñā\-nam apy apramā\-ṇam eva | ā\-ptapraṇī\-tasya punar vacanasyā\-rthā\-vyabhicā\-re tajjanmano jñā\-nasyā\-vyabhicā\-rasambhave 'pi na prā\-mā\-ṇyam upagantuṃ śakyate, paracittavṛttī\-nā\-m aśakyaniścayatvenā\-ptatvā\-parijñā\-nā\-t vacanasyā\-pi tatpraṇī\-tatvā\-pratipatteḥ | prayogaś cā\-tra -
	\pend
      

	  \pstart yad yena rū\-peṇa na niścitaṃ na tat tena rū\-peṇa vyavahriyate | yathā\- rathyā\-puruṣaḥ sarvajñatvena | na pratī\-yate cā\-bhimatapuruṣa ā\-ptatveneti vyā\-pakā\-nupalabdhiḥ || 
	\pend
      

	  \pstart nā\-yam asiddhiḥ, ā\-ptā\-bhimatasya tathā\-tvā\-niścayā\-t | tathā\- hi paracittavṛttayo 'tī\-ndriyatvā\-n na pratyakṣasamadhigamyā\- iti kā\-yavā\-gvyavahā\-rato 'numā\-tavyā\-ḥ | tau ca kā\-yavā\-gvyavahā\-rau buddhir pū\-rvam anyathā\-pi kartuṃ śakyate | tatas tatpratibaddhatvenā\-niścayā\-t kathaṃ kā\-yavā\-gvyavahā\-rato viśiṣṭaparacittavṛttyanumā\-nam ||
	\pend
      

	  \pstart nā\-pi viruddhaḥ, sapakṣe sadbhā\-vasambhavā\-t ||
	\pend
      

	  \pstart nā\-py anaikā\-ntikaḥ, prā\-mā\-ṇikatadrū\-pavyavahartavyatvaniścitatvayor vyā\-pyavyā\-pakabhū\-tayor vidhibhū\-tayor vṛkṣatvaśiṃśapā\-tvayor iva pratyakṣā\-nupalambhā\-bhyā\-ṃ sarvopasaṃhā\-reṇa vyā\-pteḥ siddhatvā\-t | tad ataḥ sā\-dhanā\-d doṣatrayarahitā\-t sā\-dhyaṃ siddhyad avā\-cyam eva | tad evam ā\-ptatvasya durbodhatvena tatpraṇī\-tatvā\-niścayā\-d ekaprahā\-ranihatam ā\-ptavacasaḥ prā\-mā\-ṇyam |
	\pend
      

	  \pstart ato yad etasya prā\-mā\-ṇyaprasiddhyarthaṃ vā\-caspatiprabhṛtī\-nā\-ṃ valgitaṃ tadaprā\-ptā\-vasaram eva | evaṃ pratyayoditam api bhaṭṭā\-bhimataṃ śā\-bdaṃ prā\-mā\-ṇyaṃ vyastam iti boddhavyam | tasmā\-t sthitam etat na śā\-bdaṃ bahirarthe pramā\-ṇam astī\-ti | buddhyā\-kā\-re tu tatkā\-ryaprasū\-tatvā\-t tadanumā\-nam eveti |
	\pend
      

	  \pstart mī\-mā\-ṃsakoktaṃ tā\-vad upamā\-naṃ mā\-nam eva na bhavati, nirviṣayatvā\-d asya | ihā\-pi prayogaḥ - yasya na viṣayavattvaṃ na tasya prā\-mā\-ṇyam | yathā\- keśoṇḍukajñā\-nasya | na siddhaṃ ca viṣayavattvam upamā\-najñā\-nasyeti vyā\-pakā\-nupalambhaḥ |
	\pend
      

	  \pstart nā\-yam asiddho hetuḥ, nirviṣayatvā\-d upamā\-nasya | tathā\- hi sā\-dṛśyaviśiṣṭaḥ piṇḍaḥ piṇḍaviśiṣṭaṃ vā\- sā\-dṛśyam upamā\-nasya viṣayo varṇyate | na sadṛśavastuvyatiriktaṃ sā\-dṛśyaṃ vyavasthā\-payituṃ śakyate, pramā\-ṇenā\-pratī\-tatvā\-t | \edlabel{thakur75-102.14}\label{thakur75-102.14} nanu sā\-dṛśyaṃ vastu durvā\-ram eva | yadā\-ha
	\pend
      
	    
	    \stanza[\smallbreak]
sā\-dṛśyasya ca vastutvaṃ na śakyam apabā\-dhitum | &bhū\-yo 'vayavasā\-mā\-nyayogo jā\-tyantarasya tat || \footnote{\begin{english}(ŚV XIII 18)\end{english}}\&[\smallbreak]


	

	  \pstart iti |
	\pend
      

	  \pstart atrocyate | yadi sadṛśā\-tiriktaṃ sā\-dṛśyaṃ vastu dṛśyaṃ syā\-t, tadā\- dṛśyā\-nupalambhagrastam eva, śā\-strā\-nā\-hitasaṃskā\-reṇā\-pi kenacit tasyā\-darśanā\-t | tasya cā\-stitve sarvaṃ sarvatrā\-stī\-ty apravṛttinivṛttikaṃ jagadā\-padyeta | athā\-dṛśyaṃ tatsā\-dṛśyam upeyate, tathā\-pi tatra prasiddhaliṅgā\-bhā\-vā\-d asiddham eva | siddhena ca tena viṣayavattopamā\-nasya sidhyeta | sā\-dṛśyapratyayas tu svahetos tathotpannena sadṛśavastunā\-pi kriyamā\-ṇo ghaṭata eva iti na sā\-dṛśyam upsthā\-payituṃ prabhavati | upamā\-napramā\-ṇabalā\-d eva sā\-dṛśyasiddhir iti cet | na | pramā\-ṇā\-ntarasiddhayor eva sā\-dṛśyapiṇḍayor viśeṣaṇaviśeṣyabhā\-vasyopamā\-naviṣayatvā\-t kathaṃ sā\-dṛśyamā\-trasyopamā\-nā\-t siddhiḥ | tataś ca sā\-dṛśyasyā\-siddher na tadviśiṣṭaḥ piṇḍaḥ piṇḍaviśiṣṭaṃ vā\- sā\-dṛśyam upamā\-nasya viṣayaḥ | tad evam upamā\-nasya nirviṣayatvaṃ siddham iti nā\-siddho hetuḥ | nā\-pi viruddhaḥ, sapakṣe bhā\-vā\-t |
	\pend
      

	  \pstart na cā\-naikā\-ntikaḥ | tathā\- hi prā\-mā\-ṇyā\-nbhā\-ve sā\-dhye prā\-mā\-ṇyam eva vipakṣaḥ | tac ca viṣayavattayā\- vyā\-ptam, nirnimittatve sarvajñā\-naprā\-mā\-ṇyaprasaṅgā\-t | tad yaṃ viruddhavyā\-ptopalabdhyā\- vipakṣā\-n nivartamā\-no viṣayavattvā\-bhā\-valakṣaṇo hetuḥ prā\-mā\-ṇyā\-bhā\-valakṣaṇa eva viśrā\-myatī\-ti vyā\-ptisiddhiḥ | ato nopamā\-naṃ pramā\-ṇam iti |
	\pend
      

	  \pstart naiyā\-yikaparikalpitopamā\-nanirā\-karaṇā\-rtham apy ayam eva prayogo draṣṭavyaḥ, tasyā\-pi nirviṣayatvā\-t | tathā\- hi samā\-khyā\-sambandhas tasya viṣayo varṇyate | sa ca paramā\-rthato nā\-sti | sa hi sambandhaḥ sambandhibhyā\-ṃ bhinno 'bhinno vā\- | yadi bhinnas tadā\- tayor iti kutaḥ | na ca sambandhā\-ntarā\-d iti vaktavyam, tad api kathaṃ teṣā\-m iti cintā\-yā\-m anavasthā\-prasaṅgaḥ | na ca yathā\- pradī\-paḥ prakā\-śā\-ntaram antareṇa prakā\-śate tathā\- sambandho 'pi sambandhā\-ntareṇa sambaddho bhaviṣyatī\-ti vaktum ucitam | pramā\-ṇasiddhe hi vasturū\-pe 'yam asya svabhā\-va iti varṇyate | yathā\- pradī\-pasyaiva | sambandhas tu na pramā\-ṇapratī\-taḥ | tat ka evaṃ jā\-nā\-tv ayam asya svabhā\-va iti, yad vā\- nā\-sty evā\-yam iti | ayam anayoḥ sambandhaḥ sambaddhā\-v etā\-v iti tu buddhiḥ svahetubalā\-t sambaddhavastudvayā\-d api sambhā\-vyamā\-nā\- na sambandham ā\-kṣeptuṃ prabhavati | tasmā\-n na bhinnasambandhasiddhiḥ | athā\-bhinnaḥ tadā\- sambandhinā\-v eva kevalā\-v iti na samā\-khyā\-sambandho nā\-ma, yaḥ kaścid upamā\-nasya viṣayaḥ syā\-t | nanu sambandhabuddhijanakatvaṃ sambaddhapadā\-rthā\-d bhinnam abhinnaṃ vā\- | bhede ca sa eva sambandhaḥ nā\-mni paraṃ vivā\-daḥ | athā\-bhinnam, tadā\- yathā\- sambaddhapadā\-rthasya svabhā\-vaḥ sarvapadā\-rthasā\-dhā\-raṇas tathā\- tad api rū\-paṃ tadavyatibhinnaṃ sarvapadā\-rthasā\-dhā\-raṇam iti sa padā\-rtho 'bhimatapadā\-rtheneva parair api padā\-rthaiḥ saha sambaddhaḥ syā\-t |
	\pend
      

	  \pstart na caivam, tasmā\-d bhinnaṃ tatsambandhabuddhijanakatvaṃ sambaddhapadā\-rthā\-d eṣṭavyam iti cet | nanv etad ā\-śaṅkya Rā\-jakulapā\-daiḥ parihṛtam eva | tathā\- hi
	\pend
      
	    
	    \stanza[\smallbreak]
sambaddhaṃ svayam eva cen nanu yathā\- taṃ tasya sambandhinaṃ pratyā\-tmā\- jagatī\-m api prati tathā\- tat kena yogo 'sya na |&sambandhe parato 'pi tulyam akhilaṃ tenaiva cet saṃyamo hetuḥ kiṃ na niyā\-makaḥ sa ca kathaṃ yogaḥ kvacin nā\-pare ||\&[\smallbreak]


	

	  \pstart iti | tasmā\-t sambandhā\-bhā\-vā\-t pū\-rvoktena nyā\-yena sā\-rū\-pyā\-bhā\-vā\-c cā\-siddhaṃ naiyā\-yikasyā\-pi nirviṣayam upamā\-naṃ pramā\-ṇam ato 'nantareṇaiva vyā\-pakā\-nupalambhena nirā\-kṛtam |
	\pend
      

	  \pstart arthā\-pattir api | yad etat sā\-mā\-nyalakṣaṇaṃ pratyakṣā\-dipratī\-to yo 'rthaḥ sa yena vinā\- nopapadyate tasyā\-rthasya parikalpanam athā\-pattir ity atra vicā\-ryate | yasyā\-rthasya darśanā\-d yo 'rthaḥ parikalpyate tayor yadi pratibandho 'sti tadā\-rthā\-pattir anumā\-nam eva | athā\-pattir iti nā\-mā\-ntarakaraṇe nā\-smā\-kaṃ kā\-cid vipratipattiḥ | tathā\- hi pramā\-ṇaparidṛṣṭo 'rthaḥ kenacid vinā\- nopapadyata iti kuto labhyate, yadi paridṛśyamā\-naparikalpyamā\-nayoḥ kaścit sambandhaḥ syā\-t | anyathā\- tena vinā\- nopapadyata ity ahrī\-kā\-d anyo na brū\-yā\-t, ghaṭapaṭavat | sa ca sambandhaḥ kvacit pū\-rvam avaśyaṃ pratyakṣā\-nupalambhataḥ, kvacid adṛśyatve 'pi viparyayabā\-dhakapramā\-ṇabalā\-d vā\- niścetavyaḥ | anyathā\- tena vinā\-nupapattijñā\-nasyaivā\-nupapatteḥ | sati caivam, ekaṃ sambadhinaṃ dṛṣṭvā\- yatrasthena vinā\- tatrasthaṃ nopapadyate, tasya dvitī\-yasya sambandhinaḥ kalpanam anumā\-nam eva | tatra svabhā\-vapratibandhe svabhā\-vahetujaiva sā\-rthā\-pattiḥ | tadutpattipratibandhe kā\-ryaliṅgajaiva | tad uktam: anyathā\-nupapannatvam anvayavyatirekiṇy arthe bhavati yat, tasmā\-n nā\-rthā\-pattiḥ, pramā\-ṇā\-ntaram iti | tasmā\-t paridṛśyamā\-naparikalpyamā\-nayoḥ sati pratibandhe nā\-rthā\-pattiḥ pramā\-ṇā\-ntaram iti | atha tayor na pratibandhaḥ, tadā\-rthā\-pattiḥ pramā\-ṇam eva na bhavatī\-ti mantavyam, sā\-kṣā\-t pā\-ramparyeṇa ca sambandhā\-bhā\-vā\-t | yasya yatra pratibandho nā\-sti na tasya tatra prā\-mā\-ṇyam ityā\-dir vedanirā\-karaṇā\-rthaṃ yaḥ pū\-rvam upanyastaḥ sa evā\-syā\- api prā\-mā\-ṇyanirā\-karaṇā\-ya draṣṭavyaḥ | sā\-mā\-nyenaivā\-rthā\-pattau nirā\-kṛtā\-yā\-ṃ pratyakṣā\-dir pū\-rvakatvalakṣaṇas tatprapañco nirasto bhavaty eveti tadarthaṃ na prabandho 'bhidhī\-yate, gavi nirā\-kṛte śā\-valeyanirā\-kṛtivat | tasmā\-n nā\-rthā\-pattiḥ pramā\-ṇā\-ntaram iti | 
	\pend
      

	  \pstart tathā\- abhā\-vapramā\-ṇasyā\-pi prā\-mā\-ṇyaṃ nopapadyate, tasyā\-pi nirviṣayatvā\-t | tataś ca Mī\-mā\-ṃsakopavalgitopamā\-nanirā\-karaṇā\-rtham upanyasto yo viṣayavattvā\-bhā\-valakṣaṇo 'nupalambhaḥ sa evā\-syā\-pi nirā\-sā\-rtham upanyasitavyaḥ | nanu cā\-trā\-siddho hetuḥ |
	\pend
      

	  \pstart tathā\- hi yadi ghaṭā\-bhā\-vo vā\-stavaḥ prameyabhū\-to na syā\-t, tadā\- nā\-stī\-ha ghaṭa iti pratyayaḥ katham utpadyata iti cet | kevalapradeśagrā\-hipratyakṣā\-d iti brū\-maḥ | nanu yadi kaivalyaṃ pradeśasvarū\-paṃ tat tarhi saghaṭe 'pi pradeśe vidyata iti tatrā\-pi tasya pratyayasya sadbhā\-vaprasaṅgaḥ | athā\-tiriktaḥ, mukhā\-ntareṇā\-bhā\-va evā\-bhyupagato bhavatī\-ti cet, na |
	\pend
      

	  \pstart kaivalyaṃ tadviviktatvam asaṅkī\-rṇatvam ityā\-dibhiḥ padaiḥ pradeśasya ghaṭaṃ pratyanā\-pannā\-dhā\-rabhā\-vasya svahetuta utpannasya ghaṭapradeśā\-d anya evā\-tmā\-bhidhī\-yate | sa eva cā\-bhā\-vapratyayaṃ janayatī\-ti kim apareṇā\-bhā\-vena kartavyam |
	\pend
      

	  \pstart nanu ghaṭaṃ pratyanā\-pannā\-dhā\-rabhā\-vasya pradeśasyeti ghaṭā\-bhā\-vayuktasya pradeśasyety uktaṃ bhavatī\-ti cet | tarhi ghaṭā\-bhā\-vo 'pi ghaṭaṃ pratyanā\-pannā\-dhā\-rabhā\-vaḥ kim abhā\-vā\-ntareṇa svarū\-peṇaiva vā\- | prathamapakṣe 'navasthā\- | atha tadabhā\-varū\-patvā\-d abhā\-vā\-ntaram antareṇaiva ghaṭā\-bhā\-vo ghaṭaṃ pratyanā\-pannā\-dhā\-rabhā\-vaḥ | yady evam asahā\-yaḥ pradeśaviśeṣo 'pi paryudā\-savṛttyā\- ghaṭā\-bhā\-varū\-patvā\-d abhā\-vaṃ vinaiva ghaṭaṃ pratyanā\-pannā\-dhā\-rabhā\-vo yukta iti kim akā\-ṇḍam ā\-hopuruṣikayā\- mithyā\-pralā\-penā\-bodhaviklavaṃ śiṣyapudgalam ā\-kulayasi | tasmā\-d bhū\-talā\-tiriktasyā\-bhā\-vasyā\-siddhatvā\-n nā\-yaṃ viṣayavattā\-bhā\-valakṣaṇo hetur asiddhaḥ | pramā\-ṇapañcakā\-bhā\-vā\-d eva tu prameyā\-bhā\-vasiddhipratyā\-śā\-pi na yujyate, vipratipattiviṣayatvā\-d asyā\-nenaiva prameyā\-bhā\-vasiddher ayogā\-t |
	\pend
      

	  \pstart viruddhā\-naikā\-ntikatve ca pū\-rvam eva hetoḥ parihṛte | tad ataḥ siddham abhā\-vapramā\-ṇā\-bhimatasyā\-prā\-mā\-ṇyam iti |
	\pend
      

	  \pstart atha vā\-bhā\-vapramā\-ṇasvarū\-pam eva nirū\-pyatā\-m | kaḥ punaḥ pramā\-ṇā\-bhā\-vā\-tmā\-bhimato bhavatā\-m, kiṃ prasajyavṛttyā\- pramā\-ṇā\-nutpattimā\-tram, atha vā\- paryudā\-savṛttyā\- bhā\-vā\-ntaram | vastvantaram api jaḍarū\-paṃ jñā\-narū\-paṃ vā\- | jñā\-narū\-pam api jñā\-namā\-trakam ekajñā\-nasaṃsargivastujñā\-naṃ veti ṣaḍ vikalpā\-ḥ |
	\pend
      

	  \pstart tatra na tā\-van nivṛttirū\-po 'bhā\-vo yujyate | sa khalu nikhilaśaktivikalatayā\- na kiñcit | yac ca na kiñcit tat katham abhā\-vaṃ paricchindyā\-t, tadviṣayaṃ vā\- jñā\-naṃ janayet, pratī\-taṃ vā\- tat katham iti sarvam andhakā\-ranartanam | yad ā\-huḥ: na hy abhā\-vaḥ kasyacit pratipattiḥ pratipattihetur vā\- tasyā\-pi kathaṃ pratipattir iti \footnote{\begin{english}(HB 25,12-14)\end{english}} | nā\-pi vastvantaratā\-pakṣe jaḍarū\-po 'bhā\-vaḥ saṅgacchate, tasyā\-bhā\-valakṣaṇaprameyaparicchedā\-bhā\-vā\-t, paricchedasya jñā\-nadharmatvā\-t | nā\-pi jñā\-namā\-trasvabhā\-vo 'bhā\-vo vaktavyaḥ, deśakā\-lasvabhā\-vaviprakṛṣṭasyā\-pi tato 'bhā\-vaprasaṅgā\-t, tadapekṣayā\-pi jñā\-namā\-tratvā\-t tasya | athaikajñā\-nasaṃsargivastujñā\-nasvabhā\-vo 'numanyate tadā\-stam abhā\-vapramā\-ṇapratyā\-śayā\-, pratyakṣaviśeṣasyaivā\-bhā\-vanā\-makaraṇā\-t | tasya cā\-smā\-bhir dṛśyā\-nupalambhā\-khyasā\-dhanatvena svī\-kṛtatvā\-t | ato na kā\-cid vipratipattir nā\-ma | tasmā\-d abhā\-vapramā\-ṇasvarū\-pam api nirū\-pyamā\-ṇaṃ viśī\-ryata eva | yad apy asya lakṣaṇam uktam
	\pend
      
	    
	    \stanza[\smallbreak]
pratyakṣā\-der anutpattiḥ pramā\-ṇā\-bhā\-va ucyate |\&[\smallbreak]


	[[(ŚV, abhā\-va, 11ab)]]

	  \pstart ityā\-di, tad api yā\-citakam aṇḍanam | tasmā\-t sthitam etat, pramā\-ṇasya sato 'traivā\-ntarbhā\-vā\-t pramā\-ṇa eva |
	\pend
      

	  \pstart || pramā\-ṇā\-ntararbhā\-vaprakaraṇaṃ samā\-ptam || 
	\pend
      
	  
	% new div opening: depth here is 1
	
\section[{Vyā\-ptinirṇayaḥ}]{Vyā\-ptinirṇayaḥ}\edlabel{Vyāptinirṇayaḥ}\label{Vyāptinirṇayaḥ}

	  \pstart iha dahanā\-dinā\- dhū\-mā\-der arthā\-ntarasya vyā\-ptis tadutpattilakṣaṇā\- | sā\- ca viśiṣṭā\-nvayavyatirekagrahaṇapravaṇaviśiṣṭapratyakṣā\-nupalambhasā\-dhaneti nyā\-yaḥ | atra ca bhaṭṭaprabhṛtayo vipratipadyante | tathā\- hi te 'gnimati pradeśe dhū\-masya bhū\-yodarśanaṃ tadvyukte ca tathaivā\-darśanam ity anvayavyatirekitvaṃ kalpayā\-m babhū\-vuḥ | 
	\pend
      

	  \pstart nanu bhū\-yasā\-pi pravṛtte darśanā\-darśane ghaṭakulaṭā\-dā\-v upalabdho vyabhicā\-ra iti cet | kim etā\-vatā\- tatrā\-py tatrā\-py anumā\-nam astu, tadvad vā\- dhū\-mā\-dā\-v api mā\- bhū\-t | prathamapakṣas tā\-vad vyabhicā\-rā\-d eva nirastaḥ | dvitī\-yo 'pi vyabhicā\-rā\-d eva | na hy anyasya vyabhicā\-re dhū\-masya kiñcit | tasmā\-d agnidhū\-mayor avyabhicā\-rasyā\-sambhave śaktam api tadupapattayaḥ tatprasā\-dhakaviśiṣṭapratyakṣā\-nupalambhā\- vā\- nā\-numā\-nopayoginaḥ | sambhave vā\- kiṃ tadutpattyā\- tadupayoginā\- viśiṣṭapratyakṣā\-nupalambhena, darśanā\-darśanā\-bhyā\-m evā\-vyabhicā\-rasiddheḥ | tathā\- ca Kā\-śikā\-kā\-raḥ: prā\-cī\-nā\-nekadarśanajanitasaṃskā\-rasahā\-yena carameṇa cetasā\- dhū\-masyā\-gniniyatatvaṃ gṛhyata iti ||
	\pend
      

	  \pstart \persName{trilocanas} tv ā\-ha: pratyakṣā\-nupalambhayor viśeṣaviṣayatvā\-t kathaṃ tā\-bhyā\-ṃ sā\-mā\-nyayoḥ sambandhapratipattiḥ | athā\-nagnivyā\-vṛttenā\-dhū\-mavyā\-vṛttasya sambandhaḥ pratī\-yata eveti | nanu so 'pi kasya pramā\-ṇasya viṣayaḥ | na tā\-vat pratyakṣasya, svalakṣaṇaviṣayatvā\-t tasya | nā\-py anumā\-nasya, tasyā\-pi tatpū\-rvakatvā\-t | na ca vyā\-vṛttyoḥ \footnote{\begin{english}(J2 vyā\-vṛttaḥ)\end{english}} kaścit sambandhaḥ | atha pratyakṣapṛṣṭhabhā\-vī\- vikalpo dṛṣṭe bhede 'bhedam adhyavasyati, tad eva sā\-mā\-nyam | evam api vikalpā\-nā\-ṃ na vastv eva viṣayaḥ | api tu grā\-hyā\-kā\-raḥ | sa ca na vastu | vastu tu teṣā\-ṃ parokṣam eveti, kathaṃ tenā\-pi sambandhagrahaḥ | asmā\-kaṃ tu bhū\-yodarśanasahā\-yena manasā\- tajjā\-tī\-yā\-nā\-ṃ sambandho gṛhī\-to bhavati | ato dhū\-mo nā\-gniṃ vyabhicarati | tadvyabhicā\-re dhū\-ma upā\-dhirahitaṃ sambandham atikrā\-med iti hetor vipakṣaśaṅkā\-nivartakaṃ pramā\-ṇam upalabdhilakṣaṇaprā\-ptopā\-dhivirahahetur anupalambhā\-khyaṃ pratyakṣam eva | tataḥ siddhaḥ svā\-bhā\-vikaḥ sambandhaḥ ||\footnote{Cf. \cref{thakur75-46.27}.}
	\pend
      

	  \pstart Vā\-caspates tu prapañcaḥ | tathā\- hi dhū\-mā\-dī\-nā\-ṃ vahnyā\-dibhiḥ svā\-bhā\-vikaḥ sambandhaḥ | na tu vahnyā\-dī\-nā\-ṃ dhū\-mā\-dibhiḥ | te hi vinā\-pi dhū\-mā\-dibhir upalabhyante | vahnyā\-dayas tu yadā\-rdrendhanasambandham anubhavanti tadā\- dhū\-mā\-dibhiḥ sambadhyante | vahnyā\-dī\-nā\-ṃ tu sphuṭamā\-rdrendhanā\-dyupā\-dhikṛtaḥ sambandho na tu svā\-bhā\-vikaḥ | tato 'niyataḥ | svā\-bhā\-vikas tu dhū\-mā\-dī\-nā\-ṃ vahnyā\-dibhiḥ sambandhaḥ, tadupā\-dher anupalabhyamā\-natvā\-t | kvacid vyabhicā\-rasyā\-darśanā\-d anupalabhyamā\-nasyā\-pi kalpanā\-nupapatteḥ | na cā\-dṛśyamā\-no 'pi darśanā\-narhatayā\- sā\-dhakabā\-dhakapramā\-ṇā\-bhā\-vena sagdihyamā\-na upā\-dhiḥ sambandhasya svā\-bhā\-vikatvaṃ pratibadhnā\-tī\-ti yuktam |
	\pend
      

	  \pstart avaśyaṃ śaṅkayā\- bhā\-vyaṃ niyā\-makam apaśyatā\-m \footnote{\begin{english}(PV I 324cd)\end{english}}
	\pend
      

	  \pstart iti tu dattā\-vakā\-śā\- laukikamaryā\-dā\-tikrameṇa śaṅkā\-piśā\-cī\- labdhaprasarā\- na kvacin nā\-stī\-ti nā\-yaṃ kvacit pravarteta | sarvatraiva kasyacid anarthasya kathañcic chaṅkā\-spadatvā\-t | anarthaśaṅkā\-yā\-ś ca prekṣā\-vatā\-ṃ nivṛttyaṅgatvā\-t | antataḥ snigdhā\-nnapā\-nopayoge 'pi maraṇadarśanā\-t | tasmā\-t prā\-mā\-ṇikalokayā\-trā\-m anupā\-layatā\- yathā\-darśanam eva śaṅkanī\-yam | na tv adṛṣṭapū\-rvam api | viśeṣasmṛtyapekṣa eva hi saṃśayo nā\-smṛter bhavati | na ca smṛtir ananubhū\-tacare bhavitum arhati | tad uktaṃ Mī\-mā\-ṃsā\-vā\-rtikakṛtā\-: nā\-śaṅkā\- niḥpramā\-ṇikā\- iti | tasmā\-d upā\-dhiṃ prayatnenā\-nviṣyanto 'nupalabhamā\-nā\- nā\-stī\-ty avagamya svā\-bhā\-vikatvaṃ sambandhasya niścinumaḥ || \edlabel{thakur75-107.16}\label{thakur75-107.16} syā\-d etat | anyasyā\-nyena sahā\-kā\-raṇena cet svā\-bhā\-vikaḥ sambandho bhavet, sarvaṃ sarveṇa svabhā\-vataḥ sambadhyeta | sarvaṃ sarvasmā\-d gamyeta | athā\-nyasya ced anyat kā\-ryaṃ kasmā\-t sarvaṃ sarvasmā\-n na bhavati, anyatvā\-viśeṣā\-t | tataś ca sa evā\-tiprasaṅgaḥ | yady ucyeta na bhā\-vasvabhā\-vā\-ḥ paryanuyojyā\-ḥ, tasmā\-d anyatvā\-viśeṣe 'pi kiñcid eva kā\-raṇaṃ kā\-ryaṃ ca kiñcid iti | nanv eṣa svabhā\-vā\-nā\-m anuyogo bhinnā\-nā\-ṃ akā\-ryakā\-raṇabhū\-tā\-nā\-m api svabhā\-vapratibandhe tulya eva | tasmā\-d yatkiñcid etad api | kena punaḥ pramā\-ṇenaiṣa svā\-bhā\-vikaḥ sambandho gṛhyate | pratyakṣasambandhiṣu pratyakṣeṇa tathā\- hi abhijā\-tamaṇibhedatattvavad bhū\-yodarśanajanitasaṃskā\-rasahā\-yam indriyam eva dhū\-mā\-dī\-nā\-ṃ vahnyā\-dibhiḥ svā\-bhā\-vikasambandhagrā\-hī\-ti yuktam utpaśyā\-maḥ | evaṃ mā\-nā\-ntaraviditasambandheṣu mā\-nā\-ntarā\-ṇy eva yathā\-svaṃ bhū\-yodarśanasahā\-yā\-ni svā\-bhā\-vikasambandhagrahaṇe pramā\-ṇā\-ny unnetavyā\-ni | svabhā\-vataś ca pratibaddhā\- hetavaḥ svasā\-dhyena yadi sā\-dhyam antareṇa bhaveyuḥ, svabhā\-vā\-d eva pracyaverann iti tarkasahā\-yā\- nirastasā\-dhyavyatir ekavṛttisandehā\- yatra dṛṣṭā\-s tatra svasā\-dhyam upasthā\-payanty eveti || \edlabel{thakur75-108.3}\label{thakur75-108.3} atrocyate | iha khalu bhede tadutpattir eva vyā\-ptiḥ | na cā\-sā\-vanyo vā\- svata evā\-vinā\-bhā\-valakṣaṇaḥ svā\-bhā\-vikaḥ sambandho bhū\-yodarśanamā\-trataḥ sidhyati | tathā\- hi, kiṃ yatra bhū\-yodarśanapravṛttis tatra niyatatvavyavasthā\-, yatra vā\- niyatatvam asti tatraiva bhū\-yodarśanapravṛttiḥ | prathamapakṣe ghaṭā\-d api kulaṭā\-, pā\-rthivatvā\-d api lohalekhyatvaṃ sidhyet, bhū\-yodarśanasambhave 'pi niyatatvasambhavā\-t | \edlabel{thakur75-108.8}\label{thakur75-108.8} vyabhicā\-radarśanā\-n naivam iti cet | kasya punarvyabhicā\-radarśanam yasya kasyacit śā\-strakā\-rasya, pratipattur vā\- | prathamapakṣe pratipattuḥ kim ā\-yā\-taṃ yato nā\-numā\-nam ayaṃ kuryā\-t | anyathā\-nyasya tadviṣayapratyakṣī\-kā\-reṇaiva so 'pi kṛtā\-rtha iti kim avaśyam anumā\-nam anveṣate | na cā\-ptavacanā\-d avyabhicā\-radarśanā\-d anumā\-nam | ā\-ptasya niścetum aśakyatvā\-d ity anyatra prasā\-dhanā\-t | śā\-strakā\-raṃ ca pṛṣṭvā\- dṛṣṭasambandho 'pi dhū\-mā\-d agnim anumā\-syata ity alaukikam | pratipattus tu nā\-vaśyaṃ sann api vyabhicā\-ro gocarī\-bhavati | na hi yatra vyabhicā\-ras tatraiva tā\-vati kā\-le deśe vā\-vaśyaṃ pratī\-tim avatarati | apratī\-yamā\-naś ca nā\-sty eveti na niyamaḥ | saty api vyabhicā\-re darśanasā\-magryabhā\-vā\-t tasyā\-darśanā\-t | aticirakā\-lavyavadhā\-ne 'pi darśanā\-t brā\-hmaṇyā\-divyabhicā\-ravat || \edlabel{thakur75-108.18}\label{thakur75-108.18} ghaṭapā\-rthivā\-dau pratipattaiva pravṛttaḥ | tadaiva krameṇa vā\- vyabhicā\-raṃ paśyed iti cet | yadi tā\-vad asau kathañcit pravartate, pravṛtto 'pi vā\- sā\-magryabhā\-vā\-vyabhicā\-raṃ na paśyet | vajraṃ vā\- lohena vyā\-pā\-rayet | vyaktaṃ tasya tā\-vat tad apy amā\-nam ā\-pannam iti mahat pā\-ṇḍityam | tasmā\-d yadi vyabhicā\-radarśanā\-d anumā\-naṃ tadā\-dṛṣṭavyabhicā\-rasya pratipattur ghaṭapā\-rthivatvā\-d apy asti | tathā\- adarśanamā\-treṇa vyabhicā\-rā\-bhā\-vo na sidhyati, yogyā\-nupalabdher eva sarvatrā\-bhā\-vasā\-dhane 'dhikā\-rā\-t | tato bahulaṃ sahacā\-ramā\-treṇa na vyabhicā\-rī\- na vyā\-vyabhicā\-rī\- niścita iti śaṅkā\-vakā\-śaḥ || \edlabel{thakur75-108.25}\label{thakur75-108.25} yady evam adṛṣṭavyabhicā\-rā\-d api dhū\-mā\-d anumā\-naṃ mā\- bhū\-t | na | ī\-dṛśasya śaṅkā\-vakā\-śasya sarvatra tadutpattirahite sambhavā\-d iti | atha kadā\-cit pratipattā\- pravṛtto vyabhicā\-raṃ paśyati | na tarhi yatra bhū\-yodarśanam, tatra niyatatvasthitiḥ | tatra kuto dhū\-me pratibandhasiddhiḥ | bhū\-yodarśanasyā\-nyatra niyatatvopasthā\-pakatvakṣatau malinapauruṣatvena sarvatrā\-nā\-śvā\-sā\-t || \edlabel{thakur75-108.30}\label{thakur75-108.30} yady evaṃ dvicandrā\-dau cakṣurā\-dipratyakṣaṃ malinapauruṣam upalabdham iti ghaṭā\-dikam api nopasthā\-payed iti cet | na | indriyaviṣayakā\-ryaṃ hi pratyakṣam | na dvicandrā\-dijñā\-nam ī\-dṛśam arthakā\-ryatvā\-bhā\-vā\-t | tato bhinnalakṣaṇasya pratyakṣā\-bhā\-sattve 'pi ghaṭajñā\-naṃ pratyakṣam eva | na caiva dhū\-mā\-dau pā\-rthivatvā\-dau ca vyā\-ptigrā\-hakasya bhū\-yodarśanasya lakṣaṇabhedo yenaikatrā\-śvā\-saḥ syā\-t || \edlabel{thakur75-109.4}\label{thakur75-109.4} ete evā\-rthakā\-ryatvā\-kā\-ryatve lakṣaṇabheda iti cet | na | ghaṭā\-dijñā\-nasya hy arthakā\-ryatvavivā\-de pramā\-ṇā\-ntarato 'rthakriyā\-lā\-bhato vā\- niścayaḥ, na pratijñā\-mā\-treṇa | na cā\-tra dhū\-masyā\-gnisahacā\-raḥ sadā\-tano 'yam atha suhṛddvayasyeva sā\-tyayo gṛhī\-ta iti saṃśaye sadā\-tanasahacā\-raprasā\-dhakapramā\-ṇā\-ntarasaṅgatir asti, tatkā\-ryaṃ vā\- kiñcid upalabhyate | tarhi bā\-dhyamā\-natvā\-bā\-dhyamā\-natvalakṣaṇo lakṣaṇabhedo bhaviṣyatī\-ty api na vaktavyam, avyabhicā\-ragrahā\-kasya bhū\-yodarśanasya bā\-dhitatvā\-siddheḥ | abā\-dhamā\-traṃ hi prasajyapratiṣedho 'pramā\-ṇam | pramā\-ṇā\-ntarasaṅgatir arthakriyā\-lā\-bho vā\- prayudā\-saś cā\-siddha iti na tā\-vat prathamaḥ pakṣaḥ | nā\-pi dvitī\-yaḥ | niyatatvā\-bhā\-ve 'pi pā\-rthivatvā\-dau bhū\-yodarśanasambhavā\-d iti na bhū\-yodarśanagamyā\- vyā\-ptiḥ || \edlabel{thakur75-109.13}\label{thakur75-109.13} \persName{trilocana}codye 'pi brū\-maḥ | yadi pratyakṣaṃ svalakṣaṇaviṣayam ity ayogavyavacchedenocyate tadā\- siddhasā\-dhanam | anyayogavyavacchedas tv asiddhaḥ, pratyakṣā\-numā\-nā\-disarvajñā\-nā\-nā\-ṃ grā\-hyā\-vaseyabhedena viṣayadvaividhyā\-natikramā\-t | yad dhi yatra jñā\-ne pratibhā\-sate tad grā\-hyam | yatra tu tat pravarte tad adhyavaseyam | tatra pratyakṣasya svalakṣaṇaṃ grā\-hyam | adhyavaseyaṃ tu sā\-mā\-nyam, atadrū\-paparā\-vṛttasvalakṣaṇamā\-trā\-tmakam | anumā\-nasya tu viparyayaḥ | tataś ca sā\-ṃvyavahā\-rikapramā\-ṇā\-pekṣayā\- rū\-parasagandhasparśasamudā\-yā\-tmakasya ghaṭasya rū\-pabhedamā\-tragrahaṇe 'pi pratyakṣataḥ samudā\-yasiddhivyavasthā\- | tathaikasyā\-tadrū\-paparā\-vṛttasya grahaṇe 'pi sā\-dhyasā\-dhanasā\-mā\-nyayor atadrū\-paparā\-vṛttavastumā\-trā\-tmanor ayogavyavacchedena viṣayabhū\-tayor vyā\-ptigraho yukta eva | ata eva vikalpā\-nā\-m avastv eva viṣayaḥ, vastu tu teṣā\-ṃ parokṣam evety api durjñā\-nam, sarvavikalpā\-nā\-m adhyavaseyā\-pekṣayā\- vastuviṣayatvā\-t | śā\-stre 'pi tathaiva pratipā\-danā\-t | na ca manasā\- tajjā\-tī\-yā\-nā\-ṃ vyā\-ptigrahaḥ śakyaḥ, manaso bahir asvā\-tantryā\-t | anyathā\- andhabadhir ā\-dyabhā\-vaprasaṅgā\-t | na ca vahnivyabhicā\-re dhū\-ma upā\-dhirahitaṃ sambandham atikrā\-med iti vaktum ucitam, svakapolakalpitasvā\-bhā\-vikasambandhasya yā\-citakamaṇḍanatvā\-d iti || \edlabel{thakur75-109.27}\label{thakur75-109.27} yad api vā\-caspatijalpitam, yo yatropā\-dhinā\- niyatas tatra tasya svā\-bhā\-vikaḥ sambandhaḥ | yathā\- dahane dhū\-masya | tadupā\-dher dṛśyasyā\-nupalabhyamā\-natvā\-t kvacid vyabhicā\-rasyā\-darśanā\-d ity atredaṃ vicā\-ryate | yasyā\-darśanataḥ svā\-bhā\-vikaḥ sambandho vavasthā\-panī\-yaḥ, sa khalu dhū\-masvarū\-pā\-d arthā\-ntaram upā\-dhir vaktavyo yathā\- dahanā\-d indhanam | arthā\-ntaraṃ ca kiñcid dṛśyam adṛśyaṃ ca kiñcit, na tu sarvam eva dṛśyatā\-niyatam | tataś ca dhū\-masyā\-pi hutā\-śane syā\-d upā\-dhiḥ, na copalabhyate ity upā\-dhimā\-trā\-nupalabdhir anaikā\-ntikī\- | tat katham adarśanamā\-trā\-n nā\-sty evopā\-dhiḥ, yataḥ svā\-bhā\-vikasambandhasiddhiḥ syā\-t | dṛśyopadhyabhā\-vasā\-dhane tu siddhasā\-dhanam | paramadṛśyopā\-dhiśaṅkā\-sambhave svā\-bhā\-vikatvapratirodhas tadavstha eva | kvacid vyabhicā\-rā\-darśanā\-d ity asambaddham eva, upā\-dhivat vyabhicā\-rasyā\-py adarśanamā\-trā\-d abhā\-vā\-siddheḥ | vyabhicā\-rasya sarvadeśakā\-layoḥ sambhave 'pi sarvadā\- sarvatra sarveṇa sā\-magryabhā\-vā\-d api niścetum aśakyatvā\-t | brā\-hmaṇyā\-divyabhicā\-ravad evā\-hatyā\-darśane 'pi deśakā\-lā\-ntare taddarśanasya niṣeddham aśakyatvā\-t | \edlabel{thakur75-110.7}\label{thakur75-110.7} nanu yadi dhū\-masyā\-pekṣaṇī\-yam arthā\-ntaram upā\-dhiḥ syā\-t kathaṃ dhū\-ma ity eva pā\-vakasattā\-niyama iti cet | nanv idam eva cintyate kiṃ dhū\-me saty avaśyam agniḥ sambhavī\- na veti | kadā\-cid arthā\-ntaram upā\-dhim apekṣya dhū\-mo 'pi syā\-n nā\-gnir iti kim atra niṣṭabdhaṃ kā\-raṇam | tasmā\-t pā\-vakaparā\-dhī\-nodayo dhū\-maḥ pariniṣṭhitaḥ kathaṃ tadabhā\-ve bhā\-vaṃ svī\-kuryā\-d ity eva sā\-dhu | \edlabel{thakur75-110.12}\label{thakur75-110.12} atha vyaktau jā\-tau vā\- vahnivyabhicā\-ro na dṛṣṭaḥ, kathaṃ tatra śaṅkyata iti cet | tat kiṃ sthā\-ṇuvyaktau jā\-tau vā\- puruṣatvaṃ dṛṣṭaṃ yena sthā\-ṇau śaṅkyate | anyatrordhvatā\-liṅgite dṛṣṭam iti cet | ihā\-py anyatra bhū\-yaḥ sahacā\-riṇi pā\-rthivatvā\-dau dṛṣṭa eva vyabhicā\-raḥ | yatraiva tu yat saṃśayate tatraiva tasya darśanam apekṣyata ity alaukikam | yadi dhū\-mavyaktau vyabhicā\-ro dṛṣṭas tadā\- dhū\-masā\-mā\-nyaṃ vyā\-ptau bahirbhū\-tam eva, kathaṃ saṃśayaḥ | atha jā\-tau dṛṣṭas tadā\-pi vyabhicā\-raniścaya eva, kathaṃ saṃśayaḥ | ato dhū\-majā\-tā\-v adṛśyamā\-no 'pi vyabhicā\-ra upā\-dhir vā\- darśanā\-yogyatayā\- niṣeddhum aśakya iti saṃśayo durvā\-raprasaraḥ | sa cedā\-nī\-m upā\-dher vyabhicā\-rasya vā\- saṃśayaḥ svā\-bhā\-vikatvasaṃśayasvabhā\-vaḥ svā\-bhā\-vikatvaniścayaṃ tā\-vad avaśyaṃ pratibadhnā\-ti | tasmā\-t svā\-bhā\-vikatvaniścayapratibandha evā\-rthataḥ, niścayam antareṇa gamakasya svayam akiñcitkaratvā\-t | tad evam upā\-dhyanupalabdhir vyabhicā\-rasyā\-nupalabdhir vā\- 'naikā\-ntikī\- na tayor abhā\-vaṃ sā\-dhayati, yataḥ sambandhasya svā\-bhā\-vikatvasiddhiḥ syā\-t | asiddhā\- ceyam upā\-dhyanupalabdhiḥ | yathā\- dahano nendhanena vinā\- dhū\-mena sambadhyate tathā\- dhū\-mo 'pi na vinā\-gninā\- sambadhyata iti samā\-nam upā\-dhitvam indhanasyobhayatra | \edlabel{thakur75-110.26}\label{thakur75-110.26} atha siddhasyā\-gner indhanasā\-hityena dhū\-malā\-bha ity upā\-dhivyavasthā\-, asiddhasya tu dhū\-masya tannimittā\-tmalā\-bhatayā\-vyabhicā\-rā\-t svā\-bhā\-vikaḥ sambandha iti vyavasthā\-pyata iti cet | evam api saiva tadutpattir ā\-yā\-tā\- | saiva svā\-bhā\-vikaḥ sambandhaḥ | na punaḥ pratijñā\-siddhaḥ sahacā\-ramā\-trā\-tmakaḥ | kiṃ ca svā\-bhā\-vikatvā\-d avyabhicā\-raḥ sarvatra, sarvatrā\-vyabhicā\-rā\-c ca svā\-bhā\-vikatvam atī\-taretarā\-śrayatvam anivā\-ryam | yasya tu sakṛttadutpattipratī\-tir eva sarvatrā\-vyabhicā\-rapratī\-tis tasya nā\-yaṃ prasaṅgaḥ | \edlabel{thakur75-110.32}\label{thakur75-110.32} yady evaṃ mamā\-pi bhū\-yodarśanā\-d avyabhicā\-rasiddhir iti cet | na | bhū\-ya ity apariniṣṭhitavā\-rasaṃkhyatvā\-t kiyatā\- darśanena lakṣaṇā\-nusā\-rī\- nirvṛtim ā\-sā\-dayet | asmā\-kaṃ tu pratyakṣā\-nupalabdhau parigaṇitasaṃkhyā\-v eva | yad ā\-huḥ
	\pend
      
	    
	    \stanza[\smallbreak]
prā\-g adṛṣṭau kramā\-t paśyan veti hetuphalasthitim |&dṛṣṭau vā\- kramaśo 'paśyann anyathā\- tv anavasthitiḥ ||\&[\smallbreak]


	

	  \pstart iti ||
	\pend
      

	  \pstart yat tv anupalabhyamā\-nasyā\-pi kalpanā\-nupapatter iti vilapitam, tadbā\-lasyā\-py asā\-mpratam | anupalabhyamā\-ne 'rthe ca kalpanā\-vakā\-śā\-t | na hi dṛśyamā\-no ghaṭaḥ kalpita ucyate | na ca sandihyamā\-na upā\-dhiḥ sambandhasya svā\-bhā\-vikatvaṃ pratibadhnā\-tī\-ti yuktam, sā\-dhakabā\-dhakā\-bhā\-va eva saṃśayasya nyā\-yaprā\-ptatvā\-t | ata eva na sarvatra śaṅkā\-piśā\-cā\-vakā\-śaḥ | tat kathaṃ nā\-yaṃ pravarteta | \edlabel{thakur75-111.9}\label{thakur75-111.9} pramā\-ṇaviṣaye 'pi śaṅkā\- kartuṃ śakyata iti cet | na | svī\-kṛtapramā\-ṇasya hi niścayaphalatvā\-t pramā\-ṇasyā\-vipratipannapramā\-ṇaviṣaye niścayasvī\-kā\-ranā\-ntarī\-yaka eva tatsvī\-kā\-raḥ | na ca śaṅkety eva na pravṛttiḥ, arthasaṃśayenā\-pi pravṛtter anivā\-ryatvā\-t snigdhā\-nnapā\-nopayogavat | tadupayoge kadā\-cin maraṇadarśane 'pi koṭiśo jī\-vitadarśanā\-t | na ca prā\-mā\-ṇikalokayā\-trā\-kṣatiḥ, prā\-mā\-ṇikair eva pramā\-ṇā\-bhā\-ve saṃśayasya vihitatvā\-t | yathā\-darśanam ā\-śaṅkanī\-yam ityā\-dy api siddhasā\-dhanam, anyatra dṛṣṭasyaivopā\-dher vyabhicā\-rasya vā\- śaṅkitatvā\-t | kiṃ ca bā\-dhakā\-darśane 'pi sā\-dhakā\-bhā\-vā\-d api śaṅkā\- syā\-d eva | \edlabel{thakur75-111.17}\label{thakur75-111.17} yad api syā\-d etad iti valgitaṃ tad api niḥsā\-ram | pramā\-ṇasiddhe hi rū\-pe svā\-bhā\-vā\-valambanam | na tu svabhā\-vā\-valambanenaiva vastusvarū\-pavyavasthā\- | tad yadi niyataviṣayā\-nvayavyatirekagrā\-hakapratyakṣā\-nupalambhapramā\-ṇasiddhe hetuphalabhā\-ve svabhā\-vavā\-das tat kim ā\-yā\-taṃ svā\-bhā\-vikasambandhe | yatra tadutpattisā\-magrī\-ṃ hṛdayena dū\-rī\-kṛtyā\-nyataḥ sahacaritadvayā\-d viśeṣeṇa pratī\-tau pratyupā\-ya eva davī\-yā\-n | tatsā\-magryapakṣaṇe ca tadutpattir eva sā\- | kim ā\-hopuruṣikayā\- nā\-mā\-ntarakaraṇena | kena punaḥ pramā\-ṇena eṣa svā\-bhā\-vikaḥ sambandho gṛhyata ityā\-dis tadgrahaṇaprakā\-raḥ pū\-rvam eva nirā\-kṛtaḥ | tathā\- svā\-bhā\-vikatvā\-siddhau svabhā\-vataś ca pratibaddhā\- hetava ityā\-dy upasaṃhā\-ro 'pi manorā\-jyamā\-tram | tasmā\-d arthā\-ntare gamye kā\-ryahetus tadbhā\-vasiddhiś ca pratyakṣā\-nupalambhā\-d iti sthitam | tad evaṃ svā\-bhā\-vikavā\-dena hṛdayā\-nulepanam aśucin eva parihā\-ryaṃ dū\-rata iti |
	\pend
      

	  \pstart || vyā\-ptinirṇyaḥ samā\-pto ratnakī\-rtipā\-dā\-nā\-m || 
	\pend
      
	  
	% new div opening: depth here is 1
	
\section[{Sthirasiddhiduṣaṇam}]{Sthirasiddhiduṣaṇam}\edlabel{Sthirasiddhiduṣaṇam}\label{Sthirasiddhiduṣaṇam}

	  \pstart namas tā\-rā\-yai ||
	\pend
      
	    
	    \stanza[\smallbreak]
yadyogā\-d andhavad viśvaṃ saṃsā\-re bhramad iṣyate |&sā\- kṛpā\-vaśagaiḥ pā\-pā\- sthirasiddhir apā\-syate ||\&[\smallbreak]


	

	  \pstart iha pare sakalapadā\-rthasthairyaprasā\-dhanā\-rthaṃ pratyakṣam anumā\-nam arthā\-pattiṃ 〔ca〕 pramā\-ṇā\-ny ā\-cakṣate | \edlabel{thakur75-112.7}\label{thakur75-112.7} tathā\- hi | sa evā\-yaṃ ghaṭasphaṭikā\-dir iti pratyabhijñā\-khyaṃ pratyakṣam udī\-yamā\-naṃ sthairyam utthā\-payati | na cedam apramā\-ṇam abhidhā\-tavyam | aprā\-mā\-ṇyaṃ hi bhavad aprā\-mā\-ṇyakā\-raṇopapattyā\- vā\- bhavet, prā\-mā\-ṇyalakṣaṇavirahā\-d vā\- | \edlabel{thakur75-112.9}\label{thakur75-112.9} yady ā\-dyaḥ pakṣaḥ | kiṃ aprā\-mā\-ṇyakā\-raṇam, mithyā\-tvam ajñā\-naṃ saṃśayo vā\- | \edlabel{thakur75-112.10}\label{thakur75-112.10} na tā\-vad atra mithyā\-tvam | mithyā\-tvaṃ hi tadviṣaye bā\-dhakapratyayā\-d vā\- hetū\-ktadoṣato vā\- sambhā\-vyeta | \edlabel{thakur75-112.11}\label{thakur75-112.11} na tā\-vad bā\-dhagandho 'pi sambhavati | deśakā\-lanarā\-ntareṣv apy asambhavā\-t | na cā\-navagatā\-pi bā\-dhā\- kadā\-cid api bhaviṣyatī\-ti śaṅkā\- yuktimatī\- | nirbī\-jaśaṅkā\-nupapatteḥ |
	\pend
      

	  \pstart avaśayaṃ śaṅkayā\- bhā\-vyaṃ niyā\-makam apaśyatā\-m | \footnote{\begin{english}(PV I 324cd)\end{english}}
	\pend
      

	  \pstart iti dattā\-vakā\-śā\- saṃśayapiśā\-cī\- labdhaprasarā\- na kvacin nā\-stī\-ti nā\-yaṃ kvacit pravarteta | antataḥ snigdhā\-nnapā\-nopayoge 'pi maraṇadarśanena sarvatra śaṅkā\-nivṛtteḥ | tasmā\-t prā\-mā\-ṇikalokayā\-trā\-m anupā\-layatā\- yathā\- darśanam eva śaṅkanī\-yaṃ nā\-dṛṣṭapū\-rvam api | \edlabel{thakur75-112.18}\label{thakur75-112.18} yad uktaṃ Kā\-rikā\-yā\-ṃ nā\-śaṅkā\- niṣpramā\-ṇikā\- \footnote{\begin{english}(ŚV II 60d)\end{english}} | iti | Bṛhaṭṭī\-kā\-yā\-m api
	\pend
      
	    
	    \stanza[\smallbreak]
utprekṣeta hi yo mohā\-d ajñā\-tam api bā\-dhakam |&sa sarvavyavahā\-reṣu saṃśayā\-tmā\- kṣayaṃ vrajet ||\&[\smallbreak]


	[[(=TS 2871)]]

	  \pstart iti |
	\pend
      

	  \pstart kṣaṇabhaṅgasā\-dhanaṃ bā\-dhakam asyeti cet | na | anumā\-nasya paramparayā\-pi pratyakṣapū\-rvatvā\-t pratyakṣam pradhā\-nam | prā\-dhā\-nyā\-c cā\-numā\-nasya bā\-dhakam | na tv anumā\-nam asya | pratyakṣā\-ntaraṃ tu bā\-dhakaṃ bhavati | yathā\- sarpā\-dipratyayasya rajjvā\-dipratyakṣam | tac cā\-tra na sambhavati | \edlabel{thakur75-112.28}\label{thakur75-112.28} nanu pratyakṣe 'pi bā\-dhake kasmā\-n na bhavati parasparapratibhandhena dvayor apy apratyakṣatā\- | \edlabel{thakur75-112.28a}\label{thakur75-112.28a} na, arthakriyā\-samarthavastuviṣayā\-viṣayatvena samā\-natvā\-bhā\-vā\-d ekasya pratyakṣā\-bhā\-satvā\-d iti na sadviṣayatvabā\-dhakapratyayā\-n mithyā\-tvam | \edlabel{thakur75-113.1}\label{thakur75-113.1} nā\-pi hetū\-ktadoṣataḥ | deśakā\-lanarā\-ntareṣv avisaṃvā\-dā\-t | \edlabel{thakur75-113.2}\label{thakur75-113.2} nā\-py ajñā\-nam aprā\-mā\-ṇyakā\-raṇam atrā\-sti | pratyabhijñā\-nasaṃvedanasambhavā\-t | \edlabel{thakur75-113.3}\label{thakur75-113.3} na ca saṃśayaḥ | na hi tad evedaṃ syā\-d vā\- na veti sphaṭikā\-diṣū\-dayati matiḥ | kiṃ tu tad evedaṃ sphaṭikā\-dikam iti nirastā\- vibhramā\-śaṅkā\- | tan nā\-prā\-mā\-ṇyakā\-raṇopapattyā\- pratyabhijñā\-nasyā\-prā\-mā\-ṇyam | \edlabel{thakur75-113.5}\label{thakur75-113.5} nā\-pi lakṣaṇakṣayā\-t | yad eva hi utpannam asandigdham aduṣṭakā\-raṇajanyaṃ deśakā\-lanarā\-ntareṣv abā\-dhitaṃ ca tad eva pramā\-ṇam iti naḥ siddhā\-ntaḥ | tad uktam |
	\pend
      
	    
	    \stanza[\smallbreak]
tasmā\-d dṛḍhaṃ yad utpannaṃ na visaṃvā\-dam ṛcchati |&jñā\-nā\-ntareṇa vijñā\-naṃ tat pramā\-ṇaṃ pratī\-yatā\-m ||\&[\smallbreak]


	[[(ŚV II 80; =TS 2904)]]

	  \pstart tathā\- Bṛhaṭṭī\-kā\-pi
	\pend
      
	    
	    \stanza[\smallbreak]
tatrā\-pū\-rvā\-rthavijñā\-naṃ niścitaṃ bā\-dhavarjitam |&aduṣṭakā\-raṇā\-rabdhaṃ pramā\-ṇaṃ lokasammatam ||\&[\smallbreak]


	[[(auch PVA 21,17f = PVAO 53,4f; TBV 13,24f, 318,25f, 394,16f; TR 126,21, ; Ravigupta, D304b1-2 (vol 9) = Q151a1:; cf. Mimaki 1976: 88f und 284f)]]

	  \pstart iti | etac ca lakṣaṇam uktanyā\-yena pratyabhijñā\-ne 'pi sambhavatī\-ti pramā\-ṇam evedam | \edlabel{thakur75-113.14}\label{thakur75-113.14} nanv idam ekam eva na bhavati kā\-raṇabhedā\-t, viṣayabhedā\-t, svabhā\-vavirodhā\-c ca | \edlabel{thakur75-113.14a}\label{thakur75-113.14a} tathā\- hi | sa iti saṃskā\-rakā\-ryam | ayam iti cendriyakā\-ryam | na ca kā\-raṇabhede 'pi kā\-ryā\-bhedo viśvavaicitryā\-hetukatvaprasaṅgā\-t | \edlabel{thakur75-113.16}\label{thakur75-113.16} tathā\- saty api sphaṭikaḥ sphaṭika iti vyapadeśā\-bhede pū\-rvadeśakā\-lasambandhā\-paradeśakā\-lasambandhā\-bhyā\-ṃ viruddhadharmā\-bhyā\-ṃ yogā\-t sphaṭikaḥ pū\-rvā\-parakā\-layor bhidyata iti viṣayabhedo vaktavyaḥ | \edlabel{thakur75-113.18}\label{thakur75-113.18} tathā\- sa iti parokṣam | ayam iti sā\-kṣā\-tkā\-raḥ | na cā\-nayoḥ svabhā\-vaviruddhayor dahanatuhinayor iva śakyā\- śakreṇā\-py ekatā\- ā\-pā\-dayitum | trailokasyaikyaprasaṅgā\-t | \edlabel{thakur75-113.20}\label{thakur75-113.20} na cā\-sya prā\-mā\-ṇyam, vikalpatvenā\-vastunirbhā\-sitvā\-t, smā\-rtā\-d aviśeṣā\-c ca | tasmā\-t pratyabhijñā\- ekatvaṃ sthā\-payati bhā\-vā\-nā\-m iti manorathamā\-tram | \edlabel{thakur75-113.23}\label{thakur75-113.23} atrocyate | ekam evedaṃ pratyabhijñā\-naṃ samā\-khyā\-tam, \edlabel{thakur75-113.23a}\label{thakur75-113.23a} yady apī\-ndriyaṃ kevalam asamartham, yady api saṃskā\-ramā\-tram, saṃskā\-rasadhrī\-cī\-naṃ tu indriyaṃ bhā\-vayiṣyati pratyabhijñā\-m | tadbhā\-vā\-bhā\-vā\-nuvidhā\-nā\-t pratyabhijñā\-bhā\-vā\-bhā\-vayoḥ | na hi nā\-jī\-janad bī\-jamā\-tram aṅkuram iti mṛdā\-disahitam api na janayati | \edlabel{thakur75-113.26}\label{thakur75-113.26} atha bhavatu deśakā\-layos tatsaṃsargayor vā\- parasparanā\-nā\-tvam | na tadavacchinnasya padmarā\-gasya | tasya tā\-bhyā\-ṃ tatsaṃsargā\-bhyā\-ṃ cā\-nyatvā\-t | \edlabel{thakur75-113.29}\label{thakur75-113.29} tato 'nyatve tatsaṃsargayoḥ kutas tadī\-yatvam iti cet | svabhā\-vā\-d eveti saṃsargaparī\-kṣā\-yā\-ṃ nipuṇataram upapā\-dayiṣyate | \edlabel{thakur75-113.30}\label{thakur75-113.30} na ca svabhā\-vavirodhaḥ, anumā\-nasyā\-py anekatvaprasaṅgā\-t | tad api hi pratyakṣam apratyakṣaṃ ca | avikalpo vikalpaś ca | asamā\-ropaḥ samā\-ropaś ca | \edlabel{thakur75-113.32}\label{thakur75-113.32} svā\-nubhavā\-vasthā\-pitā\-bhedasya svarū\-patadgrā\-hyabhedā\-pekṣayā\- pratyakṣā\-dī\-nā\-m avirodha iti cet | na, ihā\-pi sā\-myā\-t | na khalv etad api vijñā\-naṃ tattedantā\-dhikaraṇam ekam ā\-bhyā\-m anuraktaṃ sphaṭikaṃ gocarayad abhinnaṃ nā\-nubhū\-yate nā\-vasī\-yate vā\- | ekatve 'pi ca vastunas tadanurañjakatattedantā\-bhedā\-pekṣayā\- pratyakṣatā\-parokṣate na virotsyete, sahasambhavā\-t | vijñā\-naikatvasya ca pramā\-ṇasiddhatvā\-t | \edlabel{thakur75-114.3}\label{thakur75-114.3} na ca sa iti pū\-rvadeśakā\-lasaṃsargo 'yam iti ca sannihitadeśakā\-lasaṃsarga ekasya virudhyate | yato yuktaṃ yat padmarā\-gasya svarū\-pe paricchidyamā\-ne tadabhā\-vo vyavacchidyata iti tadavyavacchede tatsvarū\-pā\-paricchedā\-t, svapracyutivyavacchedyasvabhā\-vatvā\-t padmarā\-gabhā\-vasya tadanavacchede tatparicchedā\-nupapatteḥ | \edlabel{thakur75-114.8}\label{thakur75-114.8} kasmā\-t punas tadanye puṣparā\-gā\-dayo vyavacchidyante | tadabhā\-vā\-vinā\-bhā\-vā\-d iti cet, sa eva kutaḥ | pratyakṣeṇa kadā\-cid api puṣparā\-gapadmarā\-gayos tā\-dā\-tmyā\-nupalambhā\-d iti cet | yatra tarhi tatas tā\-dā\-tmyapratī\-tiḥ, tatra tadavinā\-bhā\-vaḥ | samasti ca so 'yaṃ padmarā\-ga iti deśakalā\-vasthā\-nugatam ekaṃ padmarā\-gam avabhā\-sayantī\- sā\-kṣā\-tkā\-ravatī\- pratī\-tiḥ | \edlabel{thakur75-114.12}\label{thakur75-114.12} na vikalparū\-patayā\-syā\- aprā\-mā\-ṇyam | abhilā\-pasaṃsargapratibhā\-satvaprā\-mā\-ṇyayor avirodhā\-t | \edlabel{thakur75-114.13}\label{thakur75-114.13} na cedaṃ smā\-rtam | adeśakā\-lā\-vasthā\-vato 'sya deśakā\-lā\-vasthā\-nugatatvenā\-dhikyā\-d iti | \edlabel{thakur75-114.15}\label{thakur75-114.15} atha keśakuśakadalī\-stambā\-dau saty api bhede pratyabhijñā\-nam utpannam iti cet | utpadyatā\-ṃ ko doṣaḥ | kim anena pratipā\-ditaṃ bhavati | kiṃ pratyabhijñā\-yā\-ḥ sā\-dhā\-raṇā\-naikā\-ntikatvam, atha śabdasā\-myā\-d ubhayor apy aprā\-mā\-ṇyam, uta saṃśayā\-pā\-danamā\-tram | \edlabel{thakur75-114.18}\label{thakur75-114.18} prathamaḥ pakṣo 'nabhyupagamā\-d eva nirastaḥ | na hī\-yam anumā\-natvenopanyastā\- | anumā\-natve 'py abā\-dhitatvā\-d iti viśeṣaṇe na doṣa iti pratipā\-dayiṣyā\-maḥ | \edlabel{thakur75-114.19}\label{thakur75-114.19} nā\-pi dvitī\-yaḥ pakṣaḥ | dṛṣṭā\-ntamā\-trataḥ sā\-dhyasiddher ayogā\-t | keśoṇḍukā\-diviṣayasya cakṣurvijñā\-nasyā\-py aprā\-mā\-ṇye ghaṭā\-dipratyakṣasyā\-prā\-mā\-ṇyaprasaṅgā\-t | \edlabel{thakur75-114.21}\label{thakur75-114.21} saṃśayamā\-traṃ tu vyavahā\-rocchedakatvā\-n nā\-śraṇī\-yam eveti pratipā\-ditam iti na tṛtī\-yo 'pi pakṣaḥ | \edlabel{thakur75-114.23}\label{thakur75-114.23} kiṃ ca keśā\-dau yadi pratyabhijñā\- vyabhicā\-riṇī\-, kā\-ryakā\-raṇapratī\-tiḥ kiṃ na vyabhicā\-riṇī\- | yā\- vyavicā\-riṇī\- sā\- kā\-ryakā\-raṇapratī\-tir eva na bhavatī\-ti cet | yady evaṃ yā\- visaṃvā\-dinī\- sā\- pratyabhijñaiva na bhavati tadā\-bhā\-satvā\-d iti samā\-nam | pratyabhijñā\-nasya ca sati prā\-mā\-ṇye 'numā\-nā\-diṣv anantarbhā\-ve pratyakṣaiva | saṃskā\-rasahā\-yendriyā\-nvayavyatirekā\-nuvidhā\-yitvā\-c ca | satsaṃprayoge satī\-ndriyā\-ṇā\-ṃ bhā\-vā\-c ca | tad iyaṃ pratyabhijñā\- 'nekadeśakā\-lā\-vasthā\-sambaddham ekaṃ sphaṭikā\-dikaṃ gocarayantī\- sthairyaṃ vyavasthā\-payati | \edlabel{thakur75-114.30}\label{thakur75-114.30} tathā\-numā\-nato 'pi sthiratā\-siddhiḥ | prayogaḥ | vivā\-dā\-dhyā\-sitaḥ sa evā\-yaṃ sphaṭika ityā\-di pratyabhijñā\-pratyayo yathā\-rthaḥ | abā\-dhitapratyayatvā\-t | yā\-vā\-n abā\-dhitapratyayaḥ sa sarvo yathā\-rtha upalabdhaḥ | yathā\- svasaṃvedanapratyayaḥ | abā\-dhitaś cā\-yam | tasmā\-t tatheti | abā\-dhitañ ca parodbhā\-vitakṣaṇikatvasā\-dhanabā\-dhakoddhā\-rā\-n niśceyam | \edlabel{thakur75-115.1}\label{thakur75-115.1} athā\-paraḥ prayogaḥ | vivā\-dā\-dhyā\-sitā\- bhā\-vā\-ḥ pū\-rvā\-parakā\-layor ekasvabhā\-vā\-ḥ abā\-dhitapratyabhijñayā\- pratyabhijñā\-yamā\-natvā\-t | yad yad abā\-dhitapratyabhijñayā\- pratyabhijñā\-yate tat sarvam abhinnam, yathā\- yas tvayā\- dṛṣṭo nī\-lo 'rthaḥ sa eva mayā\- dṛṣta iti nī\-lo 'rthaḥ pratyabhijñā\-yate | tathā\- caite bhā\-vā\-ḥ | tasmā\-t tatheti | pū\-rvaṃ pratyayasya dharmitā\- | adhunā\- bhā\-vā\-nā\-m iti viśeṣaḥ | \edlabel{thakur75-115.6}\label{thakur75-115.6} kiṃ ca sahetukatvā\-d vinā\-śasya sthairyaṃ siddham | prayogaḥ | vivā\-dā\-spadī\-bhū\-tā\- bhā\-vā\- yathā\-svaṃ vinā\-śahetusannidheḥ prā\-ṅ na vinā\-śinaḥ | sahetukavinā\-śatvā\-t | yad yaddhetukaṃ tat tadasannidhau na bhavati | yathā\- vahnyā\-dyabhā\-ve dhū\-mā\-diḥ | sahetukavinā\-śā\-ś cā\-ṃī\- bhā\-vā\-ḥ | tasmā\-t tatheti | \edlabel{thakur75-115.9}\label{thakur75-115.9}sahetukavinā\-śatvaṃ ca ghaṭasyā\-gnidhū\-mayor iva pratyakṣā\-nupalambhato mudgaravinā\-śayor api kā\-ryakā\-raṇabhā\-vasiddhau siddham | na ca vinā\-śahetor asā\-marthyavaiyarthyā\-bhidhā\-nam ucitam | aṅkurā\-dihetor api tathā\-tvaprasaṅgā\-t | śakyaṃ hi vaktum arthasya bhaviṣṇutā\-yā\-ṃ asamartho janmahetuḥ | bhaviṣṇutā\-yā\-ṃ vyartha iti | \edlabel{thakur75-115.13}\label{thakur75-115.13} api ca akṣaṇikā\-ḥ santaḥ | kā\-raṇavattvā\-t | yat kā\-raṇavat tad akṣaṇikam | yathā\- bhā\-vavinā\-śaḥ | kā\-raṇavantaś ceme santaḥ | tasmā\-d akṣaṇikā\- iti | \edlabel{thakur75-115.15}\label{thakur75-115.15} kā\-raṇavattvasya sā\-dhyaviparyaye vṛttiśaṅkā\- vinā\-śasya sahetukatvam eva nivartayatī\-ti prasiddhavyā\-ptikā\-t kā\-raṇavattvā\-d akṣaṇikatvasiddhir iti | \edlabel{thakur75-115.17}\label{thakur75-115.17} tathā\- Śaṅkaraḥ Sthirasiddhau prā\-ha | notpattyanantaravinā\-śī\- bhā\-vaḥ prameyatvā\-t | vastuvyā\-vṛttivad iti | \edlabel{thakur75-115.18}\label{thakur75-115.18} avidyamā\-navipakṣatvā\-d anvayy eva hetuḥ | prameyatvasya kṣaṇikatvena virodhā\-bhā\-vā\-t sandigdhavyatirekitvam iti cet | \edlabel{thakur75-115.19}\label{thakur75-115.19} na khalu kṣaṇikatve kasyacit prameyatvaṃ sidhyati | kṣaṇasthitidharmaṇaḥ pramā\-ṇakā\-le 'pā\-tā\-t | atī\-tasya ca prameyatve 'tiprasaṅgā\-d iti | \edlabel{thakur75-115.22}\label{thakur75-115.22} evam eva prayogam upastuvan \persName{trilocano} 'py ā\-ha | akṣaṇikā\-ḥ sarvabhā\-vā\-ḥ | prameyatvā\-t | yat pramī\-yate tad akṣaṇikam | yathā\- bhā\-vavinā\-śaḥ | prameyā\-ś ca sarvabhā\-vā\-ḥ | tasmā\-d akṣaṇikā\- iti | \edlabel{thakur75-115.25}\label{thakur75-115.25} asiddho dṛṣṭā\-ntadharmī\-ti cet | na svakā\-raṇakalā\-pā\-d utpattimato bhā\-vasyā\-ntareṇa nivṛttiprasavaṃ sarvadā\-vasthā\-naprasaṅgā\-t | tadaiva bhā\-vo 'sti na pū\-rvaṃ na paścā\-d ity api śabdaḥ kṣaṇikaparyā\-yatveneṣyamā\-ṇaḥ kṣaṇā\-d ū\-rdhvaṃ sattā\-vicchedopajananam antareṇa nā\-rthavā\-n devair api śakyaḥ parikalpayitum | vinā\-śakā\-lā\-pekṣayā\- hi kṣaṇo 'lpī\-yā\-n kā\-laḥ | tena so 'syā\-stī\-ti kṣaṇiko vaktavyaḥ | itarathā\- janmavinā\-śayor ekasmin kā\-le bhavatoḥ tulyahetukatvenaikatvaprasaṅgaḥ | ekatve tu dvayor ekataraḥ prahā\-tavyaḥ | tatra janmaprahā\-ṇe bhā\-vā\- niḥsvabhā\-vā\-ḥ prasajyeran | nivṛttipratiyā\-ge ca janmino bhā\-vā\- nityā\- iti durnivā\-raḥ prasaṅgaḥ | tat siddho dṛṣṭā\-ntaḥ | \edlabel{thakur75-116.1}\label{thakur75-116.1} nanu prameyatvakṣaṇikatvayor virodhā\-siddheḥ sandigdhavipakṣavyā\-vṛttikaṃ prameyatvam iti cet | \edlabel{thakur75-116.2}\label{thakur75-116.2} naitad asti | yasmā\-d arthaṃ kiñcit prā\-payat pratyakṣaṃ tena pratyā\-sannatvā\-t prā\-payati | pratyā\-sattiś ca tadutpattir evā\-vakalpate | na tā\-dā\-tmyam | sā\-kā\-ranirā\-kā\-ravā\-dayor aprakṛtatvā\-t | anyatra nirā\-kṛtatvā\-c ca | sā\- ca niyatavastupratibhā\-sā\-kṣiptā\- kā\-ryakā\-raṇabhā\-valakṣaṇā\- pratyā\-sattis tulyakā\-laṃ pramā\-ṇaprameyayor anupapannā\-, sevyetaraviṣā\-ṇayor iva | tataḥ pramā\-ṇam arthasattā\-ṃ bodhayat tadadhī\-notpā\-datayā\- bodhayati | kā\-raṇabhā\-vamā\-trā\-nubandhitvā\-c ca tasya pū\-rvakā\-lasattyā\- bhavitavyam | ataḥ pū\-rvakā\-lasattvena vyā\-ptaṃ prameyatvam | pū\-rvakā\-lasattvaṃ ca kṣaṇikatve 'nupapannam iti vyā\-pakā\-nupalabdhyā\- vipakṣā\-t kṣaṇikatvā\-d vyā\-vartamā\-naṃ prameyatvam akṣaṇikatvena vyā\-pyata iti asandigdho vyatirekaḥ | \edlabel{thakur75-116.10}\label{thakur75-116.10} tad evam anumā\-napramā\-ṇasiddho 'kṣaṇika iti || \edlabel{thakur75-116.11}\label{thakur75-116.11} evam arthā\-pattir apy asya sā\-dhikā\- | tathā\- hi kā\-ryakā\-raṇabhā\-vagrahaṇaṃ kramayaugapadyagrahaṇaṃ smaraṇam abhilā\-ṣaḥ svayaṃnihitapratyanumā\-rgaṇaṃ dṛṣṭā\-rthakutū\-halaviramaṇaṃ karmaphalasambandhaḥ saṃśayapū\-rvakanirṇayaḥ bandhamokṣaḥ mokṣaprayatnaḥ śubhā\-dike karmaṇi pravṛttiḥ pratyabhijñā\- kā\-ryakā\-raṇabhā\-vaḥ | upā\-dā\-nopā\-deyabhā\-vaprabhṛtayaḥ sthirasattā\-m antareṇā\-nupapadyamā\-nā\-ḥ sthairyaṃ sā\-dhayanti | pratikṣaṇaṃ bhede saty anubhavitur vinaṣṭatve 'nyasya kā\-ryakā\-raṇabhā\-vagrahaṇā\-dyanupapatter iti kathaṃ kṣaṇabhaṅgaśaṅkā\- 'pi || \edlabel{thakur75-116.17}\label{thakur75-116.17} atrā\-bhidhī\-yate | apramā\-ṇam evā\-yaṃ pratyabhijñā\-khyo vikalpo mithyā\-tvaṃ ca sadviṣayatvabā\-dhakapratyayā\-t | \edlabel{thakur75-116.18}\label{thakur75-116.18} nanv asya bā\-dhakaṃ pratyakṣam asambhavi | anumā\-naṃ cā\-samartham ā\-veditam iti cet | nanv asya pratyabhijñā\-nasya svā\-rthā\-vinā\-bhā\-vadā\-rḍhye pratyakṣasahasreṇā\-pi kim | saṃvā\-daśaithilye tu bā\-dhakapratyakṣavad anumā\-nam api prā\-ptā\-vakā\-śam | pramā\-ṇasyaiva siddhibā\-dhyor adhikā\-rā\-t | tathā\- hi mā\-yā\-kā\-raḥ śirasi nimajjitaṃ golakam ā\-syena niḥsā\-rayatī\-ti pratyabhijñā\- śirasi cchidraprasaṅgasaṅgatenā\-numā\-nena bā\-dhyamā\-nā\- pratī\-taiva | bā\-dhyamā\-nā\- na pratyabhijñeti prastute 'py astu | \edlabel{thakur75-116.23}\label{thakur75-116.23} yathā\- 'vanatā\-kā\-śapratibhā\-saḥ sarvasaṃpratipattā\-v api bā\-dhya eva tadvad ekatā\-grahaḥ sarvasaṃpratipattā\-v api bā\-dhyo 'stu | tasmā\-d asyā\-ḥ pratyakṣatā\-kī\-rtanaṃ yā\-citakamaṇḍanamā\-tram atrā\-ṇam | katham ataḥ sthairyasthitir astu | \edlabel{thakur75-116.26}\label{thakur75-116.26} tataś cā\-numā\-natvam apy asyā\- dhvastam | uktakrameṇā\-bā\-dhitatvaviśeṣaṇaviruddhabā\-dhyamā\-natā\-yā\-ḥ prasā\-dhanā\-d iti viśeṣaṇā\-siddho hetuḥ | \edlabel{thakur75-116.27}\label{thakur75-116.27} yadā\-pi kṣaṇabhaṅgasā\-dhakaṃ bā\-dhakaṃ nocyate asyā\-s tadā\-pī\-yam apramā\-ṇam eva | lū\-napunarjā\-takeśā\-dau vyabhicā\-ropalambhā\-t | \edlabel{thakur75-1}\label{thakur75-1} nanū\-ktaṃ yā\- vyabhicā\-riṇī\- sā\- na pratyabhijñetyā\-di | \edlabel{thakur75-116.30}\label{thakur75-116.30} yuktam etat | yadi kā\-ryakā\-raṇabhā\-vapratī\-tival lakṣaṇabhedaḥ pratipā\-dayituṃ śakyeta | yathā\- hy anvayavyatirekagrahaṇapravaṇapratyakṣā\-nupalambhā\-d upapanno niścayaḥ kā\-ryakā\-raṇabhā\-vapratī\-tir anyas tadā\-bhā\-sapratī\-tir ity anayor lakṣaṇabhedaḥ, tathā\- yadi pratyabhijñe 'pi lakṣaṇabhedo darśitaḥ syā\-t, darśayituṃ vā\- śakyo vyabhicā\-rā\-vyabhicā\-ropayogī\-, tadā\- bhavatu pratyabhijñā\-tadā\-bhā\-sayor vivekaḥ | na tv evam asti | sarvatrā\-tyantasadṛśe vastuni pṛthagjanapratyabhijñā\-yā\- ekarasatvā\-t | \edlabel{thakur75-117.3}\label{thakur75-117.3} saṃvā\-ditvā\-saṃvā\-ditve lakṣaṇabheda iti cet | na | aliṅgasya hi vikalpasya saṃvā\-do nā\-ma pramā\-ṇā\-ntarasaṅgatir athakriyā\-prā\-ptir vā\- | \edlabel{thakur75-117.4}\label{thakur75-117.4} tatra na tā\-vad ā\-dyaḥ pakṣaḥ | paścā\-d api sa evā\-yam iti svatantraikā\-dhyavasā\-yamā\-trā\-d aparasya pramā\-ṇagandhasyā\-py abhā\-vā\-t | \edlabel{thakur75-117.6}\label{thakur75-117.6} nā\-pi dvitī\-yaḥ pakṣaḥ saṅgacchate | na hi pū\-rvā\-parakā\-layor ekavastupratibaddhā\- siddhā\- kā\-cid arthakriyā\- | bhinnenā\-pi tatsamā\-naśaktinā\- tā\-dṛgarthakriyā\-yā\-ḥ karaṇā\-virodhā\-t | tathā\- hi yathaiko ghaṭo vā\-ri dhā\-rayatī\-ti tatkā\-labhā\-vino 'py anyasya deśā\-ntaravartino na vā\-ridhā\-raṇavā\-raṇam, tathā\- dvitī\-yā\-dikṣaṇo 'py anyo yadi vā\-ri dhā\-rayati, kī\-dṛśo doṣaḥ syā\-t | visadṛśakriyā\-yā\-ṃ tu cintaiva nā\-sti | tat kathaṃ pratyabhijñā\-nasya saṃvā\-dasambhavaḥ | \edlabel{thakur75-117.12}\label{thakur75-117.12} nanu yady ekam pratyabhijñā\-naṃ visaṃvā\-di dṛṣṭam iti sarvam eva pratyabhijñā\-naṃ visaṃvā\-di śaṃkyate, tadaikam indriyajñā\-naṃ keśoṇḍukadvicandrā\-dau visaṃvā\-dyupalabdham iti ghaṭā\-diṣv api sarvam eva pratyakṣaṃ visaṃvā\-di sambhā\-vyatā\-m | indriyajanyatvasyaikalakṣaṇasya sarvatra sambhavā\-d iti cet | \edlabel{thakur75-117.15}\label{thakur75-117.15} na, tatrā\-pi lakṣaṇabhedasya sadbhā\-vā\-t | tathā\- hi bahirarthasthitā\-v indriyā\-rthakā\-ryatayā\- sā\-kṣā\-d arthā\-kā\-rā\-nukā\-ritvaṃ pratyakṣatvam | tac cā\-bhyā\-saviśeṣā\-sā\-ditapaṭimnā\- pratyakṣeṇa niścī\-yate | kvacit tv arthakriyā\-prā\-ptijñā\-nā\-d iti pratyakṣatvaṃ anavadyam eva | dvicandrā\-dau tv arthavinā\-kṛtena timirā\-diviplutacakṣurmā\-treṇa tajjñā\-naṃ janitam iti pratyakṣā\-bhā\-sam eva | dvicandrā\-dyarthā\-bhā\-vas tu deśakā\-lanarā\-ntarair dvicandrā\-der arthasya bā\-dhitatvā\-d avyā\-hata iti pratyakṣā\-bhā\-sapariihā\-re 'pi pratyakṣeṣu ka ā\-śvā\-savirodhaḥ | \edlabel{thakur75-117.21}\label{thakur75-117.21} pratyabhijñā\-ne 'pi sarvam idam astī\-ti na yuktam | yathā\- hi pū\-rvaṃ pā\-vakā\-dau pā\-kā\-dikriyā\- pratibaddhā\- siddhā\- paścā\-d anubhū\-yamā\-nā\- dahanajñā\-nasya saṃvā\-dam ā\-vedayati | anyathā\- bā\-hyā\-rthocchedā\-n nirī\-haṃ jagaj jā\-yate | na tathā\- prathamacaramakā\-layor ekī\-bhā\-vapratibaddhā\- kā\-cid arthakriyā\- upalabdhigocarā\- pū\-rvā\-parakā\-layor ekatvam antareṇa vā\- pravṛttyā\-dikṣatir yenaikatā\-vagraho 'pi saṃvā\-dī\- syā\-t | \edlabel{thakur75-117.26}\label{thakur75-117.26} tad iyam anumā\-nabā\-dhitatvā\-d vyabhicā\-raśaṅkā\-kalaṅkitatvā\-c ca na pratyakṣam anumā\-naṃ veti | katham ataḥ sthairyasiddhir anumā\-napratihatir vā\- | \edlabel{thakur75-117.28}\label{thakur75-117.28} yat punar Vā\-caspatir uvā\-ca | saṃskā\-rendriyayor militayor eva pratyabhijñā\-naṃ prati kā\-raṇatvam iti, tad ayuktam | bhinnasā\-magrī\-prasū\-tatvā\-d anayor jñā\-nayoḥ | tathā\- hi nimī\-lite cakṣuṣi sa ity atrendriyavinā\-kṛtasyaiva saṃskā\-rasya sā\-marthyam upalabdham | prathamadarśane tv ayam ity atra saṃskā\-rarahitasyaivendriyasya sā\-marthyaṃ dṛṣṭam | tasmā\-t sā\-magrī\-dvayapratibaddhaṃ jñā\-nadvayam idam avadhā\-ritam | katham ubhā\-bhyā\-ṃ militvaikam eva pratyabhijñā\-nam utpā\-ditam ity udghuṣyate | bī\-jakṣityā\-dyos tu pṛthak sā\-marthyaṃ na dṛṣṭam ity ekaiva sā\-magrī\-ty aṅkuro 'py eka evā\-stu | \edlabel{thakur75-118.3}\label{thakur75-118.3} tathā\- pū\-rvadeśakā\-lā\-paradeśakā\-lā\-bhyā\-ṃ tatsambaddhā\-bhyā\-m anyatvā\-t padmarā\-gasyā\-bheda ity apy asaṅgatam | viruddhayor dharmayoḥ padmarā\-gā\-d anyatve 'pi viruddhadharmayogā\-t padmarā\-gasya bhedaḥ katham apahnū\-yate | trailokaikatvaprasaṅgasya durvā\-ratvā\-t | na hi dharmadharmiṇor anyatve 'pi brā\-hmaṇatvacaṇḍā\-latve ekā\-dhā\-re bhavitum arhata iti padmarā\-gasya bhedo duratikramaḥ | \edlabel{thakur75-118.7}\label{thakur75-118.7} tathā\- ca na svabhā\-vavirodho 'numā\-nasyā\-py anekatvaprasaṅgā\-t | tad api pratyakṣam apratyakṣaṃ cā\-vikalpo vikalpaś cā\-samā\-ropaḥ samā\-ropaś cety apy ayuktam | anumā\-nasya hi paramā\-rthataḥ svasaṃvedanapratyakṣā\-tmano 'vikalpasyā\-samā\-ropasvabhā\-vasyā\-partyakṣatvavikalpatvasamā\-ropatvā\-deḥ parā\-pekṣayā\- prajñaptatvā\-d viruddhadharmā\-dhyā\-sā\-bhā\-vā\-t kathaṃ bhedasiddhiḥ | sa evā\-yam iti tu pratyabhijñā\-nasya sa ity aspaṣṭā\-kā\-rayogitvam, ayam iti spaṣṭā\-kā\-rayogitvam iti viruddhadharmadvayaṃ bhedakam | \edlabel{thakur75-118.13}\label{thakur75-118.13} nacaivaṃ vaktavyam | tattedantā\-pekṣayā\- pratyabhijñā\-nasyā\-py ekasyaiva pā\-rokṣyā\-pā\-rokṣyam aviruddham iti | na hī\-dam ekā\-kā\-ratayā\- vyavasthitam, yenā\-numā\-navad asyā\-pi pā\-rokṣyā\-pā\-rokṣyavyavasthā\-mā\-traṃ syā\-t | yā\-vad atī\-tā\-rthā\-kā\-rā\-nukā\-ro vartamā\-nā\-rthā\-nukā\-raś ca svadharmo na bhavati tā\-vat tadarthagocarataiva nā\-sti | kutaḥ pā\-rokṣyā\-pā\-rokṣyavyavahā\-ro bhaviṣyati | tasmā\-t spaṣṭā\-spaṣṭā\-kā\-radvayaviruddhadharmā\-dhyā\-sā\-t pratyabhijñā\-naṃ pratyayadvayam etad iti sthitam || \edlabel{thakur75-118.19}\label{thakur75-118.19} tathā\- sahetukavinā\-śatvā\-d ayam apy asiddho hetuḥ | yat punar atroktam | sahetukavinā\-śatvaṃ ghaṭasyā\-gnidhū\-mayor iva pratyakṣā\-nupalambhato mudgraghaṭavinā\-śayor api kā\-ryakā\-raṇabhā\-vasiddhau siddham iti | tad asaṅgatam | agnidhū\-mayor api dṛśyatvā\-t, pratyakṣā\-nupalambhato dhū\-masya vahnikā\-ryatā\- sidhyatu | vinā\-śaśabdavā\-cyas tv artho na kaścid idantayā\- dṛṣṭaḥ | karparam eva ghaṭamudgarā\-bhyā\-m utpadyamā\-nam upalabdham | \edlabel{thakur75-118.23}\label{thakur75-118.23} yad ā\-hur guravaḥ |
	\pend
      

	  \pstart dṛṣṭas tā\-vad ayaṃ ghaṭo 'tra ca patan dṛṣṭas tathā\- mudgaro dṛṣṭā\- karparasaṃhatiḥ paramato nā\-śo na dṛṣṭaḥ paraḥ | tenā\-bhā\-va iti śrutiḥ kva nihitā\- kiṃ vā\-tra tatkā\-raṇaṃ svā\-dhī\-nā\- palighasya kevalam iyaṃ dṛṣṭā\- kapā\-lā\-valiḥ || \footnote{\begin{english}(JNA 107,13ff.)\end{english}}
	\pend
      

	  \pstart tad ayam abhā\-vo dṛśyā\-nupalabdhibā\-dhitaḥ kathaṃ pratyakṣato mudgarā\-dikā\-ryam avadhā\-ryaḥ | \edlabel{thakur75-118.29}\label{thakur75-118.29} yat punar asminn adṛśyamā\-ne 'pi dṛśyata iti bā\-gjā\-laṃ sā\- bhaṇḍavidyā\- | tadvacanā\-d gṛhṇann api paśur eva | tatha hi
	\pend
      

	  \pstart kasyacit pratibhā\-sena sā\-dhyate 'pratibhā\-si yat | pratibhā\-so 'sya nā\-syeti nopapattes tu gocaraḥ || iti | \edlabel{thakur75-119.1}\label{thakur75-119.1} athaivaṃ vaktavyam | kim anyena dhvaṃsena, karparam eva ghaṭadhvaṃso 'stu | tathā\- ca sati mudgarā\-dyabhā\-ve karparā\-bhā\-vā\-t ghaṭasthairyam avyā\-hatam iti \edlabel{thakur75-119.2}\label{thakur75-119.2} durā\-śā\- khalv eṣā\- | tathā\- hi yathā\- nā\-śaśabdena karparam ucyate tathā\- yady abhā\-vaśabdenā\-pi karparam evocyate tadaikatra pradeśe ghaṭam ekam apanī\-ya ghaṭā\-ntaranyā\-se tatrā\-panī\-taghaṭasyā\-bhā\-vavyavahā\-ro na syā\-t | tatpradhvaṃsakapā\-layos tatrā\-nutpā\-dā\-t | tasmā\-d yathā\-panī\-taghaṭasya pracyutimā\-trā\-pekṣayā\- nyastaghaṭe 'bhā\-vavyavahā\-ras tathā\- mudgarā\-dikā\-raṇā\-bhā\-vā\-t pradhvaṃsakarparayor anupā\-de 'pi pracyutimā\-trā\-pekṣayaiva pratikṣaṇam anyā\-nyatvavyavahā\-ro ghaṭasya sidhyatī\-ti kutaḥ sthairyasiddhiḥ | tasmā\-t pradhvaṃsakarparā\-bhā\-ve 'pi pracyutimā\-trā\-tmakabhā\-vā\-pekṣayā\-py asmanmatam avyā\-hatam | \edlabel{thakur75-119.9}\label{thakur75-119.9} yad ā\-hur guravaḥ |
	\pend
      

	  \pstart ā\-stā\-ṃ karparapaṃktir eva kalaśadhvaṃso na ceyaṃ purā\- tena sthairyam api prasidhyatu tato bhinnena nā\-śena kim |
	\pend
      

	  \pstart atrottaram,
	\pend
      

	  \pstart nā\-saḥ saiva yathocyate yadi tathā\-bhā\-vo 'pi kumbhā\-ntaranyā\-se 'bhā\-vavacaḥ kathaṃ matam ataḥ sidhyaty abhā\-ve 'pi naḥ || iti | \footnote{\begin{english}(JNA 108,4ff.)\end{english}} \edlabel{thakur75-119.16}\label{thakur75-119.16} nanu yadi svahetujanito nā\-śo nā\-sti, kathaṃ kvacid eva deśe kā\-le ghaṭo naṣṭa iti pratī\-tiniyamaḥ | na ca mudgarā\-d anyo nā\-śasya hetur vaktavyaḥ | prā\-g api nā\-śasambhave naṣṭaghaṭabuddhisambhavaprasaṅgā\-t | yad ā\-huḥ |
	\pend
      

	  \pstart nā\-śo nā\-sti yadi svahetuniyataḥ kiṃ desakā\-le kvacit kumbho naṣta iti pratī\-tiniyamas tenā\-sti kā\-ryaś ca saḥ | nā\-py anayat kila kā\-raṇaṃ rayavato daṇḍā\-t purā\-py anyathā\- nā\-śotthā\-nakṛtā\- vinaṣṭaghaṭadhī\-ḥ kenoddhurā\- vā\-ryate || \footnote{\begin{english}(JNA 108,21ff.)\end{english}}
	\pend
      

	  \pstart iti cet | \edlabel{thakur75-119.23}\label{thakur75-119.23} tarhī\-dā\-nī\-m arthā\-pattyā\- pradhvaṃsaṃ prasā\-dhya mudgarā\-dhī\-natvam asya sā\-dhayitum ā\-rabdham | tathā\- ca sati dhū\-mā\-gnivat pratyakṣataḥ pradhvaṃsasya mudgarā\-dikā\-ryatvaṃ siddham ity utphullagallam ullapitaṃ vyā\-luptam | \edlabel{thakur75-119.26}\label{thakur75-119.26} na cā\-rthā\-pattito 'pi tatsiddhiḥ sampadyate, ghaṭo naṣṭa iti pratī\-ter anyathā\-py upapadyamā\-natvā\-t | vinā\-śaṃ vinā\-pi hi ghaṭadarśanavato mudgarakṛtakapā\-lā\-nubhava eva naṣṭaghaṭā\-vasā\-yasā\-dhanaḥ, kim apareṇa nā\-śena kartavyam | ghaṭo naṣṭa iti buddher ghaṭaniścayapū\-rvakamudgarakṛtakapā\-lā\-nubhavamā\-trā\-nvayavyatirekā\-nuvidhā\-nadarśanā\-t | \edlabel{thakur75-119.29}\label{thakur75-119.29} na ceyaṃ sā\-magrī\- pū\-rvam apy asti | mudgarā\-bhā\-ve karparapaṃkter evā\-bhā\-vā\-t kathaṃ prā\-g api naṣṭaghaṭabuddhiprasaṅgaḥ saṅgato nā\-ma | \edlabel{thakur75-119.31}\label{thakur75-119.31} yad ā\-hur guravaḥ |
	\pend
      

	  \pstart dṛṣte 'mbhobhṛti mudgarā\-dijanitā\-ṃ dṛṣtvā\- kapā\-lā\-valī\-ṃ saṅketā\-nugamā\-d vinaṣṭaghaṭadhī\-s tā\-vat samutpā\-dyate | sā\-magryā\-m iha nā\-śanā\-ma na kim apy aṅgaṃ na cā\-syā\-m api syā\-d eṣā\- na kadā\-pi nā\-pi ca purā\-py eṣā\- samagrā\- sthitiḥ || arthā\-pattir ato gatā\- kṣayam iyaṃ na dhvaṃsasiddhau prabhuḥ | iti | \footnote{\begin{english}(JNA 109,4ff; 23)\end{english}} \edlabel{thakur75-120.4}\label{thakur75-120.4} yadi nā\-śā\-nubhavo nā\-sti kapā\-lā\-nubhavā\-t kapā\-lakalpanaiva syā\-t | na naṣṭaghaṭabuddhir iti cet | \edlabel{thakur75-120.5}\label{thakur75-120.5} tad etad atisā\-hasam | ghaṭaniścayapū\-rvakakapā\-lavalayadarśanā\-d eva naṣṭaghaṭabuddheḥ sā\-kṣā\-d evā\-nubhū\-yamā\-natvā\-t | tadapalā\-pe dhū\-mā\-dī\-nā\-m api dahanā\-dipū\-rvakatvaniścayo na syā\-d ity atiprasaṅgaḥ | \edlabel{thakur75-120.8}\label{thakur75-120.8} nanu ghaṭo naṣṭa iti buddhir viśeṣyabuddhiḥ | sā\- ca vinā\-śaṃ viśeṣaṇam ā\-kṣipatī\-ti cet | \edlabel{thakur75-120.9}\label{thakur75-120.9} tad asat, yataḥ |
	\pend
      

	  \pstart svabuddhyā\- rajyate yena viśeṣyaṃ tad viśeṣaṇam | \footnote{\begin{english}(JNA 110,1)\end{english}}
	\pend
      

	  \pstart ucyate | na cā\-vidyamā\-nam adṛśyaṃ vā\- svabuddhyā\- kiñcid rañjyati | \edlabel{thakur75-120.11}\label{thakur75-120.11} prayogo 'tra | yasya na svarū\-panirbhā\-sas tan na kasyacit svā\-nuraktapratī\-tinimittam | yathā\- karikeśaraḥ | nā\-sti ca svarū\-panirbhā\-so dhvaṃsasyeti vyā\-pakā\-nupalabdhiḥ | nā\-syā\- asiddhiḥ | abhā\-vasya svarū\-peṇaivedantayā\- nirbhā\-sā\-bhā\-vā\-t | na ca viruddhatā\-, sapakṣe bhā\-vā\-t | nā\-py anaikā\-ntikatvam | pratibhā\-sā\-bhā\-ve 'pi svā\-nuraktapratī\-tihetutve śaśaviṣā\-ṇā\-der api tathā\-tvaṃ syā\-d ity atiprasaṅgaḥ | \edlabel{thakur75-120.17}\label{thakur75-120.17} nanu
	\pend
      

	  \pstart na dhvaṃsena vinā\- vinaśyati jagad bhā\-vena sā\-rdhaṃ sa cet sac cā\-sac ca kim astu vastu niyataṃ bhā\-vā\-nujo 'sau tataḥ| bhā\-vā\-t tena tu bhinnakā\-raṇatayā\- tatkā\-raṇā\-sambhave 'bhā\-vā\-t tena kṛtā\-nyatā\-pi galitā\- bhaṅgaḥ koto 'nukṣaṇaṃ || \footnote{\begin{english}(JNA 117,23ff.)\end{english}} \edlabel{thakur75-120.22}\label{thakur75-120.22} atrocyate | kā\-raṇā\-ntarā\-d utpadyamā\-no dhvaṃso 'bhinno bhinno vā\- | \edlabel{thakur75-120.22a}\label{thakur75-120.22a} nā\-dyaḥ pakṣaḥ | bhinnakā\-raṇatvā\-t, tair anabhyupagatatvā\-c ca | atha dvitī\-yaḥ pakṣaḥ | tadā\- kaḥ punar bhā\-vasya pradveṣo yena pradhvaṃsā\-khye vastuni svahetor utpanne nivartate nā\-ma | \edlabel{thakur75-120.25}\label{thakur75-120.25} yat punar etad ucyate | nā\-bhā\-vasyotpā\-de bhā\-vasya parā\- nivṛttiḥ | kiṃ tv abhā\-votpattir eva tannivṛttir iti | katham anyasyotpā\-de 'nyasya nivṛttiḥ | atra svabhā\-vabhedair uttaraṃ vā\-cyam ye parasparaparihā\-rasthitayaḥ svahetubhyo jā\-yante, na hi svato 'nyasyā\-ṅkurasya vahnir na kā\-raṇam ity anyatvā\-viśeṣā\-d bhasmano 'pi na kā\-raṇam | svabhā\-vabhedena tu kā\-ryakā\-raṇabhā\-vasamarthanaṃ parasparaparihā\-rasthitiniyame 'pi tulyam | yathā\- cotpā\-dasya purastā\-d akhilasā\-marthyarahitasyā\-ṅkuraprā\-gabhā\-vasyā\-pakā\-raṃ kiñcid akurvanto 'pi bī\-jā\-dayo 'ṅkuram ā\-rabhamā\-ṇā\-ḥ prā\-gabhā\-vaṃ nivartayanti | tadutpā\-dasyaiva tatprā\-gabhā\-vanivṛttirū\-patvā\-t | evaṃ tadabhā\-vahetavo 'pi bhā\-varū\-pe 'kiñcitkarā\- api tadabhā\-vam ā\-dadhā\-nā\-s tan nivartayanti | abhā\-votpā\-dasyaiva bhā\-vanivṛttirū\-patvā\-t | tena pū\-rvavan nā\-rthakriyā\-karaṇaprasaṅga iti | \edlabel{thakur75-121.2}\label{thakur75-121.2} tad ucitaṃ syā\-d yadi kā\-ryakā\-raṇayor evā\-syā\-py ā\-tmā\- pramā\-ṇapratī\-taḥ syā\-t | kevalaṃ dṛśyā\-nupalambhagraste 'py etasminn upalabhyata iti pralā\-po vyaktam iyaṃ bhaṇḍavidyety uktam | \edlabel{thakur75-121.4}\label{thakur75-121.4} arthā\-pattir api kṣī\-ṇety api prā\-gabhā\-vasya ca dṛṣṭā\-ntatvenopanyā\-so bhaṇḍā\-lekhyanyā\-yaḥ | \edlabel{thakur75-121.6}\label{thakur75-121.6} kiñ ca kaḥ punar atra virodhaḥ |
	\pend
      

	  \pstart sahasthā\-nā\-bhā\-vo yadi tava virodho 'rthavipadoḥ sahasthā\-nā\-saṅgaḥ kṣaṇam api yathā\- śī\-taśikhinoḥ | sa ca dhvaṃso dhvaṃsā\-ntaram upanayan saṃprati bhaved virodhī\- so 'py anyaṃ kṣayam iti na nā\-śaḥ katham api || \footnote{\begin{english}(JNA 115,16ff.)\end{english}} \edlabel{thakur75-121.11}\label{thakur75-121.11} anyathā\- siddhasattā\-mā\-treṇa virodhitve sarvaṃ sarveṇa viruddhaṃ prasajyeta | svabhā\-vā\-lambhanam apy adarśanā\-d eva nirastam iti |
	\pend
      
	    
	    \stanza[\smallbreak]
athā\-nyonyā\-bhā\-vaprakṛtikatayā\-rthe sati tadā\- kṣayasyaivā\-bhā\-vaḥ saha bhavatu vā\- hetubalataḥ |&anena dhvaṃse ca prakṛtahatir asya tv anudaye balī\-yā\-n evā\-rthaḥ svayam apacaye 'nyena kim iha || \footnote{\begin{english}(JNA 119,20ff.)\end{english}}\&[\smallbreak]


	

	  \pstart sac cā\-sac ca kim astu vastv iti tu prasaṅgas \persName{trilocana}prastā\-ve nirā\-karaṇī\-yaḥ | ata evā\-tra prastā\-ve bhuvanaikagurū\-n bhagavataḥ Kī\-rtipā\-dā\-n avamanyamā\-naḥ êaṅkaraḥ paśor api paśur iti kṛpā\-pā\-tram evaiṣa jā\-lmaḥ |
	\pend
      

	  \pstart yad apy ā\-ha \persName{Trilocanaḥ} | bhā\-vavyatiriktā\-ṃ nivṛttim anicchadbhir aśakyā\- svarū\-panivṛttir avasthā\-payitum | yā\- hi tasya prā\-ktanī\- kā\-cid avasthā\- bhavadbhir arthakriyā\-nirvartanayogyā\- dṛṣṭā\- saiva yady uttarakā\-lam apy anuvartate tarhi svarū\-peṇaiva nivṛtto bhā\-vaḥ katham avasthā\-pyate | tadā\-nī\-m ayaṃ naṣṭo nā\-ma yadi svahetupratilabdhasvarū\-pavyatirekinī\- tasya kā\-cid avasthotpā\-dyata, utpattau saiva tasyā\-tmā\-ntaraṃ jā\-tam ity atā\-davasthyam evā\-sya vinā\-śaṃ brū\-maḥ | tā\-davasthyatā\-dā\-tmye ca svarū\-peṇa nivṛtto bhā\-va ity asya śabdasya satyam arthaṃ na vidmaḥ |
	\pend
      

	  \pstart svarū\-panivṛttiḥ khalv iyaṃ bhavantī\- bhā\-va eva syā\-t, bhā\-vā\-d anyā\- vā\- | tattve svakā\-raṇebhyo niṣpannasyā\-rthasyā\-nyathā\-nupapattā\-v utpatter ā\-rabhya sattvā\-n nityatvaṃ prasajyeta | anyatve ca tad eva nivṛtter anyatvanirvṛtir iti priyam anuṣṭhitaṃ priyeṇa | tasmā\-d utsṛjya vibhramaṃ nā\-śotpattir eva naṣṭatvam abhyupagantavyam iti | \edlabel{thakur75-122.1}\label{thakur75-122.1} tad etad ajñā\-naphalam | tathā\- hi
	\pend
      
	    
	    \stanza[\smallbreak]
svakā\-raṇā\-d eva yathā\-nyadeśavicchinnarū\-paḥ samudeti bhā\-vaḥ |&vicchinnabhinnakṣaṇavṛttir evaṃ svakā\-raṇā\-d eva na jā\-yate kim ||&abhā\-vato 'rthā\-ntararū\-pabā\-dhe tatrā\-py abhā\-vā\-ntaram ī\-kṣaṇī\-yam |&pradī\-padṛṣṭā\-ntamataṃ na kā\-ntaṃ svarū\-pasandarśanaviprayogā\-t ||\&[\smallbreak]


	[[JNA 140,4ff.]]

	  \pstart yathā\- hi deśā\-ntaraparā\-vṛttam anī\-lā\-diparā\-vṛttaṃ ca svahetor utpannaṃ vastu tathā\- dvitī\-yakṣaṇā\-taraparā\-vṛttaṃ api | yathā\- cā\-nyadeśā\-navasthā\-yitvaṃ taddeśā\-vasthā\-yitvenā\-viruddham, viruddhaṃ ca deśā\-ntarā\-vasthā\-yitvenaiva | tathā\- dvitī\-yakṣaṇā\-navasthā\-yitvaṃ prathamakṣaṇā\-vasthā\-yitvenā\-viruddham | viruddhaṃ punar dvitī\-yakṣaṇā\-vasthā\-yitvenaiva | kevalaṃ deśā\-ntaradvitī\-yakṣaṇayos tatpracyutimā\-traṃ vyavahriyate | tad anyonyā\-bhā\-vapradhvaṃsā\-bhā\-vayoḥ padā\-rthayoḥ sadbhā\-ve 'py avā\-ryam | abhā\-vā\-ntarā\-svī\-kā\-re 'pi bhā\-vā\-bhā\-vayor apy amiśratvā\-svī\-kā\-re tā\-dā\-tmyaprasaṅgā\-t | tasmā\-d abhā\-vā\-bhā\-vayos tā\-dā\-tmyam iti | \edlabel{thakur75-122.16}\label{thakur75-122.16} yathā\-rthakriyā\-kā\-ritvasya taddeśavartitvanī\-latvā\-dibhinnavirodhas tathā\- dvitī\-yakṣaṇā\-navasthā\-yitvenā\-pī\-ti vivakṣitam | paramā\-rthatas tu dharmidharmayos tā\-dā\-tmyaṃ vyā\-vṛttikṛto bhedavyavahā\-ra iti \name{apohasiddhau} prasā\-dhitam | etac coktakrameṇā\-viruddham ā\-pā\-ditam | evā\-vati tu tattve vā\-kchalamā\-trapravṛttā\- dveṣaviṣajvalitā\-tmā\-naḥ kṣudrā\-ḥ pralapantī\-ti kim atra brū\-maḥ | \edlabel{thakur75-122.21}\label{thakur75-122.21} tataś ca vyatiriktanivṛttyutpattim antareṇa svarū\-panivṛtter upapatteḥ kathaṃ kṣaṇā\-d ū\-rdhvaṃ prā\-ktanasattā\-vasthitiḥ | tasmā\-d utsṛṣṭavibhramaṃ naṣṭavyavahā\-ramā\-tram astu | na tv asyā\-nyat kiñcij jā\-yeta | \edlabel{thakur75-122.23}\label{thakur75-122.23} bhā\-vasya tā\-davarthyaprasaṅgā\-t | abhā\-vaḥ kathaṃ niṣidhyata iti cet | \edlabel{thakur75-122.24}\label{thakur75-122.24} na, tadanutpattimā\-traviṣayasya vā\-cā\-niścayena ca paścā\-d abhā\-vavyavahā\-ramā\-trapravartanasyeṣṭatvā\-d vastū\-tpatter eva niṣiddhatvā\-t | \edlabel{thakur75-122.26}\label{thakur75-122.26} nanu keyaṃ vā\-coyuktiḥ, abhā\-vavyavahā\-ramā\-tram iṣyate paścā\-n nā\-bhā\-va iti | evaṃ sati visaṃvā\-ditā\-prasaṅgo abhā\-vavyavahā\-rasya | abhā\-vaś ca mithyeti bhā\-va eva pratiṣeddhavyaḥ syā\-t | sa cā\-bhā\-vaḥ paścā\-d bhavatī\-ti sphuṭataram asya kā\-dā\-citkatvam ā\-tmahetukatvam, vastutvaṃ ceti | \edlabel{thakur75-122.29}\label{thakur75-122.29} asad etat | abhā\-vā\-khyavastvantarā\-svī\-kā\-re 'pi pracyutimā\-trā\-pekṣayā\-pi vyavahā\-rasya caritā\-rthatvapratipā\-danā\-t | yat tu tadviviktabhū\-talā\-der viṣayatvam ā\-śaṅkyoktam, na bhū\-talā\-der vastvantaratvā\-t | na ca vastvantare pratipā\-dite pratī\-te vā\- ghaṭā\-di vastubhū\-tam iti pratipā\-ditaṃ vā\- bhavati | \edlabel{thakur75-123.1}\label{thakur75-123.1} evaṃ vastvantaram eva nā\-śa iti | asmin mate yad dū\-ṣaṇam uktaṃ tat svayam eva parihṛtaṃ syā\-d iti, tad apy asambaddhaṃ, kevalaṃ hi bhū\-talam asya viṣaya iti kathaṃ na ghaṭā\-der abhū\-tatvabodhaḥ | yaiva hi ghaṭā\-dyapekṣayā\- kaivalyā\-vasthā\- pradeśasya sa eva ghaṭavirahaḥ | vacanā\-dinā\-py evaṃ kevalapradeśapratipā\-dane katham iva na prakṛtaghaṭā\-dyabhā\-vapratipā\-danam | kaivalyaṃ cā\-sahā\-yapraseśā\-d avyatibhinnam eva | \edlabel{thakur75-123.6}\label{thakur75-123.6} na ceha ghaṭo nā\-stī\-ti pratyayasya ghaṭavaty api pradeśe prasaṅgaḥ | svahetos tathotpannasya saghaṭapradeśasya kevalapradeśā\-d anyatvā\-t | \edlabel{thakur75-123.7}\label{thakur75-123.7} na ca pratyabhijñā\-nataḥ saghaṭā\-ghaṭapradeśayor ekatvaṃ pū\-rvam asya nirā\-karaṇā\-t | \edlabel{thakur75-123.8}\label{thakur75-123.8} na ca vinā\-śahetor asā\-marthyavaiyarthyā\-bhidhā\-ne 'ṅkurā\-dihetor api tathā\-bhidhā\-tum ucitam | asiddhe hi kā\-rye hetor ā\-śrayaṇam avā\-ryam | siddhe ceyaṃ cintā\-, yadi hetor nityo 'nityo vā\- 'rtho jā\-taḥ kiṃ nā\-śakā\-raṇeneti hetupuraskā\-reṇaiva pravṛtteḥ | na caivam asiddhe 'ṅkurā\-dau kā\-rye śakyam abhidhā\-tum | svarū\-pasyaivā\-bhā\-vā\-t | taddharmakatvā\-〔tad〕dharmakatvā\-diparyanuyogasya nirviṣayatvā\-t | \edlabel{thakur75-123.13}\label{thakur75-123.13} nanu tvayā\-pi bhā\-vā\-bhā\-vayor lakṣaṇabhedo 'bhihitaḥ | tat katham ekatvaṃ sarvā\-rthā\-nā\-m | lakṣaṇabhedā\-d eva bhedavyavasthā\- | tato 'pi cen na bhedavyavasthitiḥ, na kasyacit kutaścid bhedavyavasthitir ity advaitaprasaṅga iti cet | \edlabel{thakur75-123.15}\label{thakur75-123.15} na | yo hi naśvarasvabhā\-vaḥ sa eva nā\-śo naśyatī\-ti bahulā\-dhikā\-rā\-t kartari ghañaḥ prasā\-dhanā\-t taṃ nā\-śaṃ bhā\-vasvabhā\-vam icchā\-maḥ | naśanaṃ nā\-śa iti prasajyā\-tmā\- dvidhā\- kartavyaḥ | tattvatas tā\-vad vastutvavirahā\-t tattvā\-nyatvavirahita evā\-sau bhā\-vo na bhavatī\-ti tadbhā\-vaniṣedhamā\-tram ā\-yā\-taṃ tu bhavati | kharaśṛṅgā\-divat | saṃvṛtau tu yathā\- kā\-labhedena vikalpyamā\-naḥ kā\-dā\-citka iva pratibhā\-ti tathā\- sarvopā\-khyā\-viraharū\-patayā\- bhā\-vā\-d bhinna iva pratibhā\-tī\-ti nā\-vastutvopalakṣaṇabhedā\-khyā\-navirodhaḥ | evaṃ ca sati saṃvṛttyā\- lakṣaṇabhede bhā\-vā\-bhā\-vayor bhedasyeṣṭatvā\-t | tattvena ca lakṣaṇaikatā\-virahe bhā\-vasya tenaikyaniṣedhā\-t katham advaitaprasaṅgopā\-lambhaḥ | \edlabel{thakur75-123.24}\label{thakur75-123.24} syā\-d etat | na ca vivekā\-pratī\-tau tadviviktagrahaṇaṃ bhavati | tadvivekaś ca na bhū\-talā\-disvarū\-pam eva viśeṣaṇatvā\-d iti | \edlabel{thakur75-123.25}\label{thakur75-123.25} tad etan nyā\-yabahiṣkṛtam | viśeṣaṇaviśeṣyabhā\-vo hi saṅkalpā\-rū\-ḍhe rū\-pe bā\-hyā\-rthasparśe vikalpaśabdaliṅgā\-ntarā\-ṇā\-ṃ vaiyarthyaprasaṅgā\-d iti śā\-stre vistareṇa pratipā\-danā\-t | sa ca saṅkalpo 'bhinnam api bhā\-vaṃ bhinnam ivā\-kalayati | yathā\- śilā\-putrakasya śarī\-ram, śarī\-re karaṇā\-dayaḥ | lambakarṇo Devadatta ityā\-di | tasmā\-t kalpanā\-dhī\-no viśeṣaṇaviśeṣyabhā\-vaḥ | abhinne 'pi bhā\-ve bhedavivakṣā\-pekṣo bhedavyavahā\-raḥ kathaṃ bhedaniyatam ā\-tmā\-nam ā\-tanotu | \edlabel{thakur75-123.31}\label{thakur75-123.31} skhaladgatir ayaṃ rā\-hoḥ śira ity ā\-dinirdeśa itic cet | \edlabel{thakur75-123.31a}\label{thakur75-123.31a} yadi satyam etat, tadā\- śiro 'tiriktasya rā\-hor iva kṣmā\-talā\-der atiriktasya vivektasya dṛśyā\-nupalambhabā\-dhitatvā\-d ayam api nirdeśaḥ skhaladgatir eva, tathā\-pi neti koṣapā\-naṃ pramā\-ṇam | tasmā\-t saghaṭā\-t pradeśā\-ntarā\-t pradeśa evā\-yam anyo ghaṭaviviktaḥ svahetor utpanno na tu ghaṭavivekena viśeṣitaḥ | svahetor utpannasya viviktasyā\-bhā\-ve vivekasyā\-bhā\-vā\-t | \edlabel{thakur75-124.3}\label{thakur75-124.3} kiṃ ca
	\pend
      

	  \pstart vyā\-ptaṃ bhidā\- yadi viśeṣyaviśeṣaṇatvaṃ bhedā\-tyayā\-n nanu tadā\- tadabhā\-va eva | deśo viśiṣṭa iti nā\-sti yathā\- tathedam apy asti dṛśyamatabhedadṛg asti neti || \footnote{\begin{english}(JNA 150,24ff.)\end{english}} \edlabel{thakur75-124.8}\label{thakur75-124.8} tasmā\-n nā\-bhā\-vo nā\-ma kaścid yatra kā\-raṇavyā\-pā\-raḥ | tad evaṃ sahetukavinā\-śatvā\-d iti hetuḥ svarū\-pā\-siddha iti sthitam || \edlabel{thakur75-124.10}\label{thakur75-124.10} satā\-m akṣaṇikatvaṃ kā\-raṇavattvā\-d ity apy asambaddham eva | kṣaṇikatvakā\-raṇavattvayor virodhā\-bhā\-vā\-d akṣaṇikatvena kā\-raṇavattvasya vyā\-pter asiddheḥ | sandigdhavyatirekatvā\-t | na cā\-sya viparyaye vṛttiśaṅkā\- nā\-śasya sahetukatvam eva nivartayati | uktakrameṇa nā\-śasyaivā\-bhā\-vā\-d iti || \edlabel{thakur75-124.14}\label{thakur75-124.14} tathā\- prameyatvā\-d api sthirasiddhir manorathamā\-tram | sā\-kā\-ravedanodayapakṣasthitau hi dvitī\-yakṣaṇā\-nuvṛttā\-v apy arthasya vyavahitatvā\-t, prakā\-śā\-nupapatter viṣayasvarū\-pavedanam eva jñā\-nasya viṣayavedanam | evaṃ ca vartamā\-nā\-nurodhaḥ, atī\-te 'pi tatpratyā\-satter apracyuteḥ | na cā\-tiprasaṅgaḥ | anantarā\-tī\-tā\-d anyena kṣaṇena sā\-rū\-pyā\-samarpaṇā\-t | tataś ca kā\-raṇatvā\-d yadi nā\-ma prameyatvasya pū\-rvakā\-lasattvena vyā\-ptis tathā\-pi prameyatvavat pū\-rvakā\-lasattvam api kṣaṇike 'viruddham iti prameyatvā\-kṣaṇikatvayor vyā\-ptisā\-dhano vyā\-pakā\-nupalambho 'siddhaḥ | jñā\-nā\-kā\-rā\-rpakatvaṃ hi hetutvam, prameyatvaṃ prā\-mā\-ṇikapratī\-tam | tac cā\-nantarā\-tī\-ta eva kṣaṇe samupapadyate | \edlabel{thakur75-124.22}\label{thakur75-124.22} jñā\-nasattā\-samaye 'rthā\-nuvṛtter abhā\-vā\-n nirviṣayateti cet | \edlabel{thakur75-124.22a}\label{thakur75-124.22a} nanv ananuvṛttā\-v api tadarpitā\-kā\-rasvarū\-pasaṃvedanam eva tadvedanam | tad eva ca saviṣayatvam | iyaṃ ca pratyā\-sattir anantarā\-tī\-te 'pi kṣaṇe 'kṣī\-neti na dvitī\-yakṣaṇā\-nuvṛtter anurodha ity uktam | ataḥ sandigdhavyatirekitvā\-d anaikā\-ntikam eva prameyatvam | \edlabel{thakur75-124.26}\label{thakur75-124.26} atha sā\-kā\-ravā\-davidveṣā\-d anā\-kā\-rajñā\-nagrā\-hyatvaṃ prameyatvam abhipretaṃ tadā\- 'siddhatā\- 'sya hetoḥ | \edlabel{thakur75-124.27}\label{thakur75-124.27} indriyā\-rthasannikarṣā\-der jñā\-nam utpadyatā\-ṃ nā\-ma | \edlabel{thakur75-124.27a}\label{thakur75-124.27a} tac cā\-nubhavaikarasatvena sarvatrā\-rthe sadṛśā\-kā\-ratvā\-t kasya grā\-hakam astu, \edlabel{thakur75-124.28}\label{thakur75-124.28} yenā\-bhisambaddham iti cet | \edlabel{thakur75-124.28a}\label{thakur75-124.28a} ā\-tamamanaḥsaṃyogā\-dī\-nā\-m api grahaṇaṃ syā\-t | \edlabel{thakur75-124.29}\label{thakur75-124.29} janakasya grahaṇam iti cet | \edlabel{thakur75-124.29a}\label{thakur75-124.29a} tathā\-py ā\-tmā\-dī\-nā\-ṃ grahaṇaprasaṅgaḥ | viṣayatvena janakasya grahaṇam ity apy asā\-dhu | viṣayatvasyā\-dyā\-py aniścayā\-t | \edlabel{thakur75-124.30}\label{thakur75-124.30} idaṃ dṛṣṭam śrutaṃ vedam ity adhyavasā\-yo yatrā\-rthe sa viṣaya iti cet | \edlabel{thakur75-124.31}\label{thakur75-124.31} nanv asty eva pratiniyato vyavahā\-raḥ | kaḥ punar atra pratyā\-sattiniyama iti pṛcchā\-maḥ | sa ced upavarṇayituṃ na śakyate, vyavahā\-ro 'pi tvanmate niyato na syā\-d iti brū\-maḥ | \edlabel{thakur75-124.33}\label{thakur75-124.33} asti tā\-vad iti cet | \edlabel{thakur75-124.34}\label{thakur75-124.34} ata evā\-rthasā\-rū\-pyam asā\-dhā\-raṇaṃ pratyā\-sattinimittam astu | nirnimitte niyamā\-yogā\-t | \edlabel{thakur75-125.1}\label{thakur75-125.1} nanu sā\-rū\-pyam apy arthā\-darśane katham avadhā\-ryate | tac ca kim ekadeśena, sarvā\-tmanā\- vā\- | ā\-dye pakṣe sarvaṃ sarvasya vedanaṃ syā\-t | dvitī\-ye tu jñā\-nam ajñā\-natā\-ṃ vrajet | kiṃ ca sā\-rū\-pyā\-d arthavedane 'nantaraṃ jñā\-naṃ tulyaviṣayaṃ viśayaḥ syā\-d iti cet | \edlabel{thakur75-125.4}\label{thakur75-125.4} mā\- bhū\-d arthasya darśanam | ā\-kā\-raviśeṣabalā\-d adhyavasitā\-rthasyā\-rthakriyā\-prā\-pter evā\-rtho 'pī\-dṛśa iti sā\-rū\-pyavyavahā\-ro 'viruddhaḥ | ata eva sthū\-lagataṃ paramā\-ṇugataṃ vā\- sā\-rū\-pyaṃ na cintyate | jñā\-nā\-kā\-rasya sthū\-latve 'py ekasā\-magrī\-pratibaddhapuñjaviśeṣā\-d apy abhī\-ṣṭakriyā\-karaṇā\-t puruṣā\-rthasiddheḥ | \edlabel{thakur75-125.7}\label{thakur75-125.7} sā\-rū\-pyaṃ caikadeśenaiva | na cā\-tra sarvavedanaprasaṅgaḥ | sarveṣā\-ṃ jñā\-naṃ praty ajanakatvā\-t | janakā\-nā\-ṃ ca svavyapadeśanimittā\-sā\-dhā\-raṇaikadeśā\-rpakatvena grā\-hyatvā\-t | \edlabel{thakur75-125.9}\label{thakur75-125.9} nā\-pi tulyaviṣayā\-nantarajñā\-nagrahaṇaprasaṅgaḥ, tasya svasaṃvedanā\-d eva pramā\-ṇā\-t siddhatvā\-t | pramā\-ṇā\-ntarasya tatra vaiyarthyā\-t | jaḍatve saty ā\-kā\-rā\-rpakasya vastuno grā\-hyatvā\-d ity asyā\-rthasyā\-bhī\-ṣṭatvā\-c ca | bā\-hyā\-rthasthitau ceyaṃ cinteti sarvam anavadyam | \edlabel{thakur75-125.12}\label{thakur75-125.12} tad evam ayaṃ prameyatvā\-d iti hetuḥ sā\-kā\-ravā\-dapakṣe sandigdhavyatirekaḥ | nirā\-kā\-rapakṣe cā\-siddha iti sthitam || \edlabel{thakur75-125.14}\label{thakur75-125.14} na cā\-rthā\-pattir api sthirā\-tmasā\-dhanī\- | kā\-ryakā\-raṇabhā\-vagrahaṇā\-dī\-nā\-m anyathopapatteḥ | \edlabel{thakur75-125.15}\label{thakur75-125.15} tathā\- hi upā\-dā\-nopā\-dheyabhā\-vasthitacittasantatim apy ā\-śrityeyaṃ vyavasthā\- sustheti katham ā\-tmā\-naṃ pratyujjī\-vayatu | tatra kā\-ryakā\-raṇabhā\-vapratī\-tis tā\-vad anā\-kulā\- | tathā\-pi prā\-gbhā\-vivastuniścayajñā\-nasyopā\-deyabhū\-tena tadarpitasaṃskā\-ragarbheṇa paścā\-dbhā\-vivastujñā\-nenā\-smin satī\-daṃ bhavatī\-ti niścayo janyate | tathā\- prā\-gbhā\-vivastvapekṣayā\- kevalabhū\-talaniścayakajñā\-nopā\-deyabhū\-tena tadarpitasaṃskā\-ragarbheṇa paścā\-dbhā\-vivastvapekṣayā\- kevalabhū\-talaniścā\-yakajñā\-nenā\-smin asatī\-daṃ na bhavatī\-ti vyatirekaniścayo janyate | yathoktam |
	\pend
      

	  \pstart ekā\-vasā\-yasamantarajā\-tam anyavijñā\-nam anvayavimarśam upā\-dadhā\-ti | evaṃ tadekavirahā\-nubhavodbhavā\-nyavyā\-vṛttidhī\-ḥ prathayati vyatirekabuddhim || \edlabel{thakur75-125.26}\label{thakur75-125.26} ata eva devadattenā\-gnau pratī\-te yajñadattena ca dhū\-me pratī\-te na kā\-ryakā\-raṇabhā\-vagrahaṇaṃ tajjñā\-nayor upā\-dā\-nopā\-deyabhā\-vā\-bhā\-vā\-t | yatra tv ekasantā\-ne jñā\-nakṣaṇayor upā\-dā\-nopā\-deyabhā\-vas tatra kā\-ryā\-digrahaḥ sugrahaḥ | anyathā\- saty api nityā\-tmani pratisandhā\-tari kā\-ryakā\-raṇabhā\-vā\-dī\-nā\-m apratī\-tir eva syā\-t | \edlabel{thakur75-125.30}\label{thakur75-125.30} tathā\- hi ā\-tmanaḥ sakā\-śā\-t pratisandheyabuddhī\-nā\-m abhedo bhedo vā\- bhedā\-bhedo vā\- | \edlabel{thakur75-125.31}\label{thakur75-125.31} prathamapakṣe ā\-tmaiva syā\-t pratisandhā\-tā\- | buddhaya eva vā\- syuḥ pratisandheyā\- iti kaḥ pratisandhā\-rthaḥ | \edlabel{thakur75-125.32}\label{thakur75-125.32} bhedapakṣe 'pi buddhibhyo bhidyamā\-nasya jaḍasyā\-tmanaḥ kaḥ pratisandhā\-nā\-rtha iti na vidmaḥ | \edlabel{thakur75-125.33}\label{thakur75-125.33} buddhiyogā\-d draṣṭṛtvavat pratisandhā\-tṛtvam iti cet | \edlabel{thakur75-126.1}\label{thakur75-126.1} buddhir eva tarhi draṣṭrī\- pratisandhā\-trī\- ceti niyamasvī\-kā\-re tadyogā\-d asya tathā\-tvam iti kim anena yā\-citakamaṇḍanena | \edlabel{thakur75-126.2}\label{thakur75-126.2} buddhī\-nā\-ṃ kartṛtvā\-bhā\-vā\-d iti cet | \edlabel{thakur75-126.2a}\label{thakur75-126.2a} taddvā\-reṇā\-pi tarhi tasyā\-tmano draṣṭṛtvā\-divyavahā\-rā\-nupapattiḥ | yadi hi buddhir hetoḥ phalasya vā\- draṣṭṛī\- syā\-t tadā\-nantaryapratiniyamasya cā\-nusandhā\-trī\- kalpitā\- | tadyogā\-d draṣṭṛtvaṃ pratisandhā\-tṛtvaṃ cocyata iti syā\-d api prativiṣayam alabdhaviśeṣā\-yā\-ṃ ca buddhau sambandho 'pi na viśeṣaṃ vyavahā\-rayitum ī\-śaḥ | adhunā\- nibandhanā\-dhigantā\- | adhunā\- phalasya | idā\-nī\-ṃ pratisandhā\-teti | tathā\-pi ca buddhiyutaviśeṣasvī\-kā\-re tu kim apareṇā\-tmanā\- kartavyam | tā\-vataiva paryā\-ptatvā\-d vyavahā\-rasya | \edlabel{thakur75-126.9}\label{thakur75-126.9} sthirā\-tmā\-nam antareṇa saiva buddhir na syā\-d iti cet | \edlabel{thakur75-126.9a}\label{thakur75-126.9a} kenaivaṃ pratā\-rito 'si | aho mohamā\-hā\-tmyaṃ yad ī\-dṛśā\-n api paravaśī\-karoti | tathā\- hi nedam idam antareṇa yad ucyate tat khalv anyatra pratyakṣā\-nupalambhā\-bhyā\-ṃ sā\-marthyā\-vadhā\-raṇe sati yujyate vahner iva dhū\-me | cakṣurā\-divad vā\- dṛṣṭakā\-raṇā\-ntarasā\-magyā\- kā\-ryā\-darśane paścā\-d darśane ca kiñcid anyad apekṣaṇī\-yam astī\-ti sā\-mā\-nyā\-kā\-reṇa | \edlabel{thakur75-126.14}\label{thakur75-126.14} ā\-dyaḥ pakṣas tā\-van nā\-stī\-ti vyaktam | dvitī\-yo 'pi na sambhavī\- | na hi kā\-raṇabuddhisamanantaraṃ kā\-ryabuddhau satyā\-ṃ niścayapravṛttasyedam asyā\-nantaraṃ dṛṣṭam mayeti pratisandhā\-nam adṛṣṭapū\-rvaṃ kadā\-cit | yato 'nyasya sā\-marthyaparikalpanaṃ syā\-d ity udasya vyā\-moham uktakrameṇaiva kā\-rykā\-raṇagrahaṇavyavasthā\- svī\-kartavyā\- | \edlabel{thakur75-126.18}\label{thakur75-126.18} bhedā\-bhedapakṣas tu dhakkā\-ra eva | tasyaiva tadapekṣayā\- bhedā\-bhedaviruddhadharmā\-dhyā\-sā\-d ekatvā\-nupapatteḥ | tataś ca yad bhinnaṃ bhinnam evā\-bhinnṃ cā\-bhinnam iti naikasya bhedā\-bhedau | tathapy abheda viśvam ekam iti yugapadutpā\-dasthitipralayaprasaṅgaḥ | \edlabel{thakur75-126.20}\label{thakur75-126.20} evaṃ kramivastugrā\-hakaiḥ kramijñā\-nair upā\-dā\-nopā\-deyabhū\-taiḥ sā\-kṣā\-t pā\-ramparyeṇa krameṇā\-mī\- jā\-yanta iti niścayo janyate | ekakā\-likā\-nekavastugrā\-hakair eva tajjñā\-nair ekopā\-dā\-natvā\-t sakṛd imā\-ni jā\-tā\-nī\-ti vikalpaḥ kriyata iti kramā\-kramagrahaṇam apy anavadyam | \edlabel{thakur75-126.24}\label{thakur75-126.24} katham anekajñā\-nā\-d ekavikalpa iti cet | \edlabel{thakur75-126.24a}\label{thakur75-126.24a} ko doṣaḥ |
	\pend
      

	  \pstart bhavantu bhinnā\- matayas tathā\-pi tā\- dadhaty upā\-dā\-natayaikakalpanam | na bhinnasaṃkhyā\- phalahetubā\-dhanī\- na cā\-nyasantā\-nabhavā\- ivā\-kṣamā\-ḥ || \edlabel{thakur75-126.29}\label{thakur75-126.29} yad apy uktaṃ \persName{Śaṅkareṇa}: atha pū\-rvottarakṣaṇayoḥ saṃvittī\- | tā\-bhyā\-ṃ vā\-sanā\-, tayā\- hetuphalabhā\-vā\-dhyavasā\-yī\- vikalpa iti cet | \edlabel{thakur75-126.30}\label{thakur75-126.30} tat kim idā\-nī\-ṃ yat kiñcid ā\-śaṅkitena | vaktavyam ity evaṃ vidhir anuṣṭhī\-yate bhavatā\- | vikalpo hy agṛhī\-tā\-nusandhā\-nam atadrū\-pasamā\-ropo vā\- syā\-t | \edlabel{thakur75-126.32}\label{thakur75-126.32} na tā\-vat pū\-rvaḥ pakṣaḥ | adṛṣṭā\-nvayavyatirekasya puruṣasya hetuphalabhā\-vā\-grahe 'nusandhā\-napratyayahetor vā\-sanā\-viśeṣasyaivā\-nupapatteḥ | agṛhī\-tasya cā\-nusandhā\-ne 'tiprasaṅgā\-d iti | \edlabel{thakur75-127.1}\label{thakur75-127.1} tad etan na samyag ā\-locitam | yato hetuphalabhū\-tayoḥ pū\-rvottarakṣaṇayor ekaikena jñā\-nenā\-nanubhave 'py upā\-dā\-nopā\-dheyabhū\-tā\-bhyā\-ṃ kramijñā\-nā\-bhyā\-ṃ hetuphalatve gṛhī\-te eva | kevalaṃ hetukā\-le phalā\-bhā\-vā\-t tadviṣayasā\-marthyagrahaṇe 'pi phalā\-darśanā\-t tadavasā\-ya evā\-pravṛttaḥ kā\-ryadarśanena pravartyate | tathā\- phalā\-valokane 'pi tatkā\-ryatā\- gṛhī\-taiva vikalpenā\-nusandhī\-yata iti gṛhī\-tā\-nusandhā\-narū\-pa evā\-yaṃ vikapa iti yat kiñcid etat | \edlabel{thakur75-127.7}\label{thakur75-127.7} yad ā\-ha Mahā\-bhā\-ṣyā\-laṅkā\-raḥ |
	\pend
      
	    
	    \stanza[\smallbreak]
yadi nā\-maikam adhyakṣam na pū\-rvā\-paravittimat |&adhyakṣadvayasadbhā\-ve prā\-kparā\-vedanaṃ katham || iti |\&[\smallbreak]


	

	  \pstart tathā\- smaraṇam abhilā\-ṣaḥ, svayaṃnihitapratyanumā\-rgaṇaṃ, dṛṣṭā\-rthakutū\-halaviramaṇaṃ, karmaphalasambandhaḥ, saṃśayapū\-rvakanirṇayaś ca pū\-rvapū\-rvā\-rthā\-nubhavair upā\-dā\-nakā\-raṇaiḥ samarpitasaṃskā\-ragarbhair uttarottarā\-rthā\-nubhavair evopā\-deyabhū\-tair janyamā\-no yujyata iti kim adhikenā\-tmā\-nā\- parikalpitena | \edlabel{thakur75-127.13}\label{thakur75-127.13} upā\-dā\-nopā\-deyabhā\-vaniyamā\-d eva ca na santā\-nā\-ntare smaraṇā\-diprasaṅgaḥ saṅgataḥ | kim idam upā\-dā\-nam iti cet | \edlabel{thakur75-127.14}\label{thakur75-127.14} ucyate | yatsantā\-nanivṛttyā\- yad utpadyate tat tasyopā\-dā\-nakā\-raṇam | yathā\- mṛtsantā\-nanivṛttyotpadyamā\-nasya kumbhasya mṛd upā\-dā\-nam iti śā\-stre prapañcitam | na cā\-tra paralokakṣatiḥ | \edlabel{thakur75-127.17}\label{thakur75-127.17} yad apy uktam | cittaśarī\-rayoḥ kiyatkā\-lasthitinibandhanasya dṛṣṭasya nivṛttau cittasyā\-pi nivṛttiprasaṅgaḥ | maraṇavedanayā\- hi cittaṃ vikalam | tato 'vikalā\- cittā\-ntarajananā\-vasthā\- na sambhavati | tasmā\-d upasthite maraṇaduḥkhe sarvasaṃskā\-ravirodhini cittam apy ucchidyeteti nā\-stikyam ā\-yā\-tam iti | \edlabel{thakur75-127.20}\label{thakur75-127.20} tad ayuktam | yato maraṇaduḥkhaṃ cittaviśeṣa eva, tasya cittā\-ntarajananasā\-marthyasvabhā\-vasya svabhā\-vā\-d avā\-ryaiva jñā\-notpattir iti | \edlabel{thakur75-127.22}\label{thakur75-127.22} bandhā\-n mokṣo 'pi saṃsā\-ricittaprabandhā\-d anā\-śravacittaprabandho yaḥ | \edlabel{thakur75-127.23}\label{thakur75-127.23} śubhā\-dimokṣayor api pravṛttir avā\-ryā\- | yataḥ saty apy ā\-tmany aham eva mukto bhaviṣyā\-mi sukhī\- cety ā\-tmagrahalakṣaṇā\-d adhyavasā\-yā\-t pravartate | na punar ā\-tmanā\- galahastitaḥ | sa cā\-nā\-dyavidyā\-paramparā\-yā\-taḥ pū\-rvā\-parayor ekatvā\-ropako mithyā\-saṅkalpo bā\-dhite 'py ā\-tmany avyā\-hataprasara iti katham apravṛttiḥ | \edlabel{thakur75-127.26}\label{thakur75-127.26} nanu
	\pend
      
	    
	    \stanza[\smallbreak]
nairā\-tmyavā\-dapakṣe 〔tu〕 pū\-rvam evā\-vabudhyate |&madvinā\-śā\-t phalaṃ na syā\-n matto 'nyasyā\-thavā\- bhaved ||\&[\smallbreak]


	

	  \pstart iti | apravṛttir evā\-stv iti cet | \edlabel{thakur75-127.29}\label{thakur75-127.29} astu ko doṣaḥ | yady ayam ā\-tmagraho nirviṣayo 'pi pravṛttim anā\-kṣipya kṣaṇam api sthā\-tuṃ 〔na〕 prabhavati | yathā\- hi jā\-tasyā\-vaśyaṃ mṛtyur iti jā\-tavato 'py apratikriyaputrā\-dimaraṇe sorastā\-ḍam ā\-krando maraṇā\-dau ca yatnaḥ śokodrekā\-t | evam avidyodrekā\-d eva nairā\-tmyaṃ jā\-nann api pravartate | na sukham ā\-sta iti kim atra kriyatā\-m | avidyā\-yā\-ḥ pravartanaśakter avā\-ryatvā\-t | \edlabel{thakur75-128.3}\label{thakur75-128.3} pratyabhijñā\- ca pū\-rvam eva dhvastā\- | kā\-ryakā\-raṇabhā\-vaniyatā\- paścā\-dbhā\-vipū\-rvabhā\-vitā\- | sā\- ca kṣaṇike 'py aviruddhā\- | upā\-dā\-nopā\-deyatā\- ca kramisvasaṃvedanajñā\-nadvayena sā\-kṣā\-tkṛta tatpṛṣṭhabhā\-vinā\- niścī\-yata iti, \edlabel{thakur75-128.6}\label{thakur75-128.6} asaty apy ā\-tmani pratisandhā\-tari kā\-ryakā\-raṇagrahaṇā\-daya upapadyamā\-nā\- nā\-tmā\-nam upasthā\-payituṃ prabhavanti | ato 'rthā\-pattir api na kṣameti bhā\-gyahī\-namanorā\-jyam iva sthirasiddhir viśī\-ryata eva | \edlabel{thakur75-121.8}\label{thakur75-121.8} tathā\- ca kṣaṇabhaṅgasandehe sattvā\-dyanumā\-naṃ prā\-ptā\-vasaram ||
	\pend
      

	  \pstart Sthirasiddhidū\-ṣaṇaṃ samā\-ptam || 
	\pend
      
	  
	% new div opening: depth here is 1
	
\section[{Citrā\-dvaitaprakā\-śavā\-daḥ}]{Citrā\-dvaitaprakā\-śavā\-daḥ}\edlabel{Citrādvaitaprakāśavādaḥ}\label{Citrādvaitaprakāśavādaḥ}

	  \pstart || namas tā\-rā\-yai || 
	\pend
      
	    
	    \stanza[\smallbreak]
dig eṣā\- svaparā\-śeṣaprativā\-diprasā\-dhanī\- |&citrā\-dvaitamatā\-bodhadhvā\-ntastomakadarthinī\- ||\&[\smallbreak]


	

	  \pstart iha khalu sakalajaḍapadā\-rtharā\-śau pratyā\-khyā\-te nirā\-kṛte ca nirā\-kā\-ravijñā\-navā\-de pratihate cā\-lī\-kā\-kā\-rayogini pā\-ramā\-rthikaprakā\-śamā\-tre samyagunmū\-lite ca sā\-kā\-ravijñā\-nā\-lī\-katvasamā\-rope pratisantā\-naṃ ca svapnavad abā\-dhitadehabhogapratiṣṭhā\-dyā\-kā\-raprakā\-śamā\-trā\-tmake jagati vyavasthite yasya yadā\- yā\-vad ā\-kā\-racakrapratibhā\-saṃ yadvijñā\-naṃ parisphurati tasya tadā\- tā\-vad ā\-kā\-racakraparikaritaṃ tadvijñā\-naṃ citrā\-dvaitam iti sthitiḥ | tad evaṃ citram advaitaṃ vijñā\-nam iti padatrayam iha pratyupasthitam || 
	\pend
      

	  \pstart atra ca vipratipattir nā\-ma kiṃ citratā\-yā\-m advaite vijñā\-natve sarvatraiveti vikalpā\-ḥ || 
	\pend
      

	  \pstart na tā\-vad asau citrasvarū\-pā\-nusā\-riṇī\- bhavitum arhati, tanmā\-trasya \edtext{sarvajanānu}{\Afootnote{sarvajanā\-nu \cite{}; sarvajanā\-u \cite{}; sarvajñā\-nu}}bhavasiddhatvā\-t, anyathā\- śaśaviṣā\-ṇā\-dā\-v iva jaḍam idam alī\-kaṃ vijñā\-naṃ veti vipratipattī\-nā\-m anavakā\-śaprasaṅgā\-t |
	\pend
      

	  \pstart nā\-pi vijñā\-natve vivā\-daḥ kartum ucitaḥ,
	\pend
      
	    
	    \stanza[\smallbreak]
sahopalambhaniyamā\-d\footnote{Cf. PVin 1.54a.}\&[\smallbreak]


	

	  \pstart ityā\-dinā\- pū\-rvam eva nī\-lā\-dī\-nā\-ṃ sā\-kā\-ravijñā\-natvaprasā\-dhanā\-t | ata eva sarvatrā\-pi vimatir asaṅgatā\-, sā\-kā\-ravijñā\-nasiddhā\-v eva citrā\-dvaitavā\-dā\-vatā\-rā\-t | tasmā\-c citrateyam advaitavirodhinī\-ti vyā\-mohā\-d ekatva eva [ {\corr vipratipatir}] iti tatra prasā\-dhanaṃ sā\-dhanam idam ucyate ||
	\pend
      

	  \pstart yat prakā\-śate tad ekam | yathā\- citrā\-kā\-racakramadhyavartī\- nī\-lā\-kā\-raḥ | prakā\-śate cedaṃ gauragā\-ndhā\-ramadhurasurabhisukumā\-rasā\-tetarā\-divicitrā\-kā\-rakadambakam iti svabhā\-vahetuḥ | \edlabel{thakur75-129.25}\label{thakur75-129.25} na tā\-vad asyā\-siddhir abhidhā\-tuṃ śakyate, pratyakṣapramā\-ṇaprasiddhasadbhā\-ve vijñā\-nā\-tmakanī\-lā\-dyā\-kā\-racakre dharmiṇi prakā\-śamā\-natā\-yā\-ḥ pratyakṣasiddhatvā\-t | na cā\-sya \leavevmode\textsuperscript{\rmlatinfont\tiny [pb in]}\label{thakur75-130} hetor viruddhatā\- sambhavati, vicitrā\-kā\-ramadhyavartini nī\-lā\-kā\-re dṛṣṭā\-ntadharmiṇi prakā\-śamā\-natā\-lakṣaṇasya sā\-dhanasya dṛṣṭatvā\-t | nanu caikatve sā\-dhye yad aparam ekatvā\-dhikaraṇaṃ tad iha dṛṣṭā\-ntī\-kartum ucitam | na cā\-sya nī\-lā\-kā\-rasya ekatā\- vidyate, viruddhadharmā\-dhyā\-saprasiddhasyā\-nekatvasya sambhavā\-t | deśakā\-lā\-kā\-rabhedo hi viruddhadharmā\-dhyā\-saḥ | tataś ca yathā\- citratā\-kā\-racakrasyā\-kā\-rabhedato bhedas tathā\- nī\-lā\-kā\-rasyā\-pi deśabhedato bhedaḥ | tad ayaṃ sā\-dhyaśū\-nyo dṛṣṭā\-nto hetuś ca vipakṣe paridṛśyamā\-no | yadi tatraiva niyatas tadā\- viruddhaḥ \edlabel{thakur75-130.8}\label{thakur75-130.8} tatrā\-pi sambhave 'naikā\-nta iti cet ||
	\pend
      

	  \pstart atrocyate | yadi deśabhedato vijñā\-nā\-tmakasthū\-lanī\-lā\-kā\-rasya bhedas tadā\-sya pratiparamā\-ṇudeśabhede bhedasambhavā\-t paramā\-ṇupracayamā\-trā\-tmako vijñā\-nā\-tmakasthū\-lanī\-lā\-kā\-raḥ syā\-t | tathā\- ca sati sarveṣā\-ṃ vijñā\-nā\-tmakanī\-laparamā\-ṇū\-nā\-ṃ svasvarū\-panimagnatvena saṃtamasanimagnā\-nekapuruṣavad vyativedanā\-bhā\-vā\-t sthū\-lanī\-lā\-khaṇḍalakapratibhā\-sā\-bhā\-vaprasaṅgaḥ |
	\pend
      

	  \pstart na ca svasvarū\-panimagnatvenā\-py anyenā\-nyasya vedanaṃ yujyate, yena sthū\-lapratibhā\-saḥ saṅgataḥ syā\-t, grā\-hyagrā\-hakalakṣaṇayoḥ purastā\-d apakartavyatvā\-t |
	\pend
      

	  \pstart na caivaṃ vaktavyam paramā\-ṇū\-nā\-ṃ [ {\corr sva}]svarū\-panimagnatve 'py ekopā\-dā\-natayā\- puñjā\-tmaiva sthū\-laḥ sthū\-lam ā\-tmanaṃ jñā\-syatī\-ti, saty apy ekopā\-dā\-natve svasvarū\-panimagnatvā\-d eva sthū\-lavyavasthā\-pakasya bhinnasyā\-tmano 'nyonyam vā\- grā\-hyā\-grā\-hakabhā\-vasyā\-yogā\-t | tā\-dā\-tmyena vyativedanasya cā\-nabhyupagamā\-t |
	\pend
      
	    
	    \stanza[\smallbreak]
vargo vargaṃ veti\&[\smallbreak]


	

	  \pstart ity asyā\-nupadatvā\-t | na ca yathā\- bā\-hyā\-rthavā\-de sthū\-laikā\-kā\-rajñā\-napratibhā\-sa eva bā\-hyaparamā\-ṇupracayapratibhā\-savyavasthā\- gatyantarā\-bhā\-vā\-t, tathā\- jñā\-naparamā\-ṇuvyavasthā\-[ {\corr 〔nne〕}]sthū\-laikā\-kā\-rayogivijñā\-nā\-ntarasyā\-nabhyupagamā\-t | abhyupagame vā\- tasyaiva dṛṣṭā\-ntatvā\-t | tasmā\-d yā\-vad yā\-vat pratibhā\-sas tā\-vat tā\-vat sthū\-latayaiva vyā\-ptaḥ | asthū\-le paramā\-ṇau sthū\-lanivṛttimā\-tre ca pratibhā\-sasya dṛśyā\-nupalambhabā\-dhitatvā\-t | yathā\- prasiddhā\-numā\-ne sattvaṃ kṣaṇikatvena vyā\-ptaṃ kramā\-kramkā\-ritvenā\-pi, kṣaṇikatvā\-bhā\-vā\-c ca kramā\-kramanivṛttau nivartamā\-naṃ kṣaṇikatve niyataṃ sidhyati, tathā\-trā\-pi prakā\-śamā\-natvaṃ sā\-dhanam ekatvenā\-pi sthaulyenā\-pi, ekatvā\-bhā\-vā\-c ca vipakṣā\-t paramā\-ṇupuñjā\-tmana ekatvanivṛttimā\-trā\-tmanaś ca svaviruddhopalambhā\-t sthaulyasya vyā\-pakasya nivṛttau nivartamā\-nam [ {\corr ekatve}] niyataṃ sidhyati | tataś ca yathā\- bahirvyā\-ptipakṣe ghaṭe dṛṣṭā\-ntadharmiṇi viparyayabā\-dhakapramā\-ṇabalā\-t sattvaṃ kṣaṇikatvaniyatam avadhā\-rya\edtext{}{\lemma{---}\Afootnote{\label{RNA-tc-0}dhā\-rya \cite{}; dhā\-ryamā\-ṇaṃ \cite{}}} sattvā\-t pakṣe kṣaṇikabhaṅgasiddhiḥ, tathā\-trā\-pi nī\-lā\-kā\-re dṛṣṭā\-ntadharmiṇi viparyayabā\-dhakapramā\-ṇabalā\-d eva prakā\-śamā\-natvam ekatvaniyatam avagamya prakā\-śamā\-natvā\-d vicitrā\-kā\-racakrasā\-dhyadharmiṇy ekatvasiddhir iti na dṛṣṭā\-ntasya sā\-dhyaśū\-nyatvam | nā\-pi hetor viruddhatā\- | na cā\-naikā\-ntikatā\- || \edlabel{thakur75-130.33}\label{thakur75-130.33} nanv ekatve sā\-dhye tatpracyutir dvitvaṃ ca vipakṣaḥ, tasmā\-c ca vipakṣā\-d dhetuvyatirekapratipattyavasare kiṃ vipakṣā\-tmā\- prakā\-śate na vā\- | pratibhā\-sapakṣe prakā\-śamā\-natvasya hetoḥ sā\-dhā\-raṇā\-naikā\-ntikatā\-, vipakṣe 'pi dṛṣṭatvā\-t | atha na prakā\-śate tadā\- sandigdhavyatirekitvam, kuto vyatireka ity avadher evā\-prakā\-śamā\-naśarī\-ratvā\-t katham ataḥ sā\-dhyasiddhipratyā\-śā\- | \edlabel{thakur75-131.4}\label{thakur75-131.4} atrocyate | iha dvividho vijñā\-nā\-nā\-ṃ viṣayaḥ grā\-hyo 'dhyavaseyaś ca | pratibhā\-samā\-no grā\-hyaḥ | agṛhī\-to 'pi pravṛttiviṣayo 'dhyavaseyaḥ | tatrā\-sarvajñe 'numā\-tari sakalavipakṣapratibhā\-sā\-bhā\-vā\-n na grā\-hyatayā\- vipakṣo viṣayo vaktavyaḥ, sarvā\-numā\-nocchedaprasaṅgā\-t, sarvatra sakalavipakṣapratibhā\-sā\-bhā\-vā\-t tato vyatirekā\-siddheḥ | pratibhā\-se ca deśakā\-lasvabhā\-vā\-ntaritasakalavipakṣasā\-kṣā\-tkā\-re sā\-dhyā\-tmā\-pi virā\-kaḥ sutarā\-ṃ pratī\-yata ity anumā\-navaiyarthyam | tasmā\-d apratibhā\-se 'py adhyavasā\-yasiddhā\-d eva vipakṣā\-d dhū\-mā\-der vyatireko niścitaḥ | tat kim artham atra vipakṣapratibhā\-saḥ prā\-rthyate | yadi punar asyā\-dhyavasā\-yo 'pi na syā\-t tadā\- vyatire\edlabel{capv-np-4a-start}\label{capv-np-4a-start}\footnote{\begin{english}This is where \cref{capv-np} starts. The verso of this folio is numbered as 4 in the left margin.\end{english}}ko na niścī\-yata iti yuktam, pratiniyataviṣayavyavahā\-rā\-bhā\-vā\-t || \edlabel{thakur75-131.13}\label{thakur75-131.13} nanv asminmate vastvavastvā\-tmakasakalavipakṣapratipattisambhavā\-t tato hetuvyatirekaḥ saṃpratyetuṃ śakyata eva | na ca pratibhā\-samā\-treṇa sattvaprasaṅgaḥ, arthakriyā\-kā\-ritvalakṣaṇatvā\-t sattvasya | tvanmate tu prakā\-śa eva vastutvam | ato vipakṣayor ekatvapracyu\edtext{}{\lemma{---}\Afootnote{\label{RNA-tc-1}tidvi \cite{}; tir dvi \cite{}}}tvayoḥ pratibhā\-se prakā\-śamā\-natvasā\-dhanasya vipakṣasā\-dhā\-raṇatā\- | apratibhā\-se ca sandigdhavyatirekitvam iti codyaṃ duruddharam eveti cet | tad etad asaṅgatam | tathā\- hi dhū\-mā\-dir avahnyā\-der vipakṣā\-d vyā\-vṛtto vahnyā\-diniyataḥ sidhyati[ {\corr  | }] tasya ca vastvavastvā\-tmakasakalavipakṣapadā\-rtharā\-śeḥ svarū\-panirbhā\-sa iti kiṃ nirvikalpajñā\-ne kalpanā\-yā\-ṃ vā\- | nirvikalpe cet | pratibhā\-sa iti ca ko 'rthaḥ | kiṃ nirā\-kā\-re jñā\-ne sakalavipakṣā\-disvarū\-pasya sā\-kṣā\-t sphuraṇam, yadi vā\- tadarpitabuddhisvabhā\-vabhū\-tasadṛśā\-kā\-raprakā\-śaḥ, atha samanantarapratyayabalā\-yā\-tabuddhigatabā\-hyasadṛśā\-kā\-rapratibhā\-saḥ, ā\-hosvid buddher ā\-tmabhū\-tavipakṣasadṛśā\-lī\-kā\-kā\-raparisphū\-rtiḥ | \edlabel{thakur75-131.24}\label{thakur75-131.24} na tā\-vad ā\-dyaḥ pakṣo yuktaḥ, deśakā\-\edlabel{capv-np-4a-end}\label{capv-np-4a-end}\edlabel{capv-np-4b-start}\label{capv-np-4b-start}lasvabhā\-vaviprakṛṣṭā\-nā\-ṃ padā\-rthā\-nā\-m arvā\-cī\-ne jane nirā\-kā\-re ca jñā\-ne sphuraṇā\-yogā\-d ity asyā\-rthasya śā\-stre eva vistareṇa prasā\-dhā\-nā\-t | sphuraṇe [ {\corr vā\-}]sā\-dhyasyā\-pi prakā\-śanaprasaṅge 'numā\-navaiyarthyasya pratipā\-danā\-t | \edlabel{thakur75-131.27}\label{thakur75-131.27} nā\-pi dvitī\-yaḥ pakṣaḥ, deśā\-diviprakṛṣṭatvā\-d eva sā\-kṣā\-tsvā\-kā\-rasamarpaṇasā\-marthyā\-bhā\-vā\-t | \edlabel{thakur75-131.29}\label{thakur75-131.29} na ca tṛtī\-yaḥ saṅgataḥ, sā\-dṛśyasambhave 'pi samanantarabalā\-d evā\-yā\-tasya bā\-hyena saha pratyā\-satter abhā\-vā\-t | \edlabel{thakur75-131.31}\label{thakur75-131.31} na caturtho 'pi prakā\-raḥ sambhavati, asatprakā\-śayor virodhā\-t, sphurato 'lī\-katvā\-yogā\-t | tathā\- hy asatprakā\-śa iti kim asadī\-śvarā\-deḥ khyā\-tiḥ, bhā\-samā\-no vā\- ā\-kā\-ro 'san, san vā\- na kaścit khyā\-tī\-ti vivakṣitam | tatra yasya padā\-rthasya svarū\-paparinirbhā\-saḥ sa katham asann iti prā\-ṇadhā\-ribhir abhidhā\-tavyaḥ | sphurataḥ keśoṇḍukā\-kā\-rasya bā\-hyarū\-patayā\- bā\-dhyatve 'pi jñā\-narū\-patayā\-rthatvasya ā\-cā\-ryeṇa pratipā\-ditatvā\-t grā\-hakā\-bhimatanirā\-kā\-raprakā\-śasyā\-py asattvā\-bhidhā\-naprasaṅgā\-t || \edlabel{thakur75-132.3}\label{thakur75-132.3} pratibhā\-se 'pi bā\-dhanā\-d asatyatvam iti cet | kiṃ tad bā\-dhakam, pratyakṣam anumā\-naṃ vā\- | yady ekatra svarū\-pasā\-\edlabel{capv-np-4b-end}\label{capv-np-4b-end}kṣā\-tkā\-riṇi pratyakṣe 'viśvā\-saḥ katham anyatra bā\-dhake svarū\-pā\-ntaraprakā\-śa eva nirvṛttis tatpū\-rvakam anumā\-naṃ ca sutarā\-m aviśvā\-sabhā\-janam iti na bā\-dhakavā\-rtā\-pi | yad ā\-hur guravaḥ
	\pend
      
	    
	    \stanza[\smallbreak]
yasya svarū\-panirbhā\-sas tad evā\-sā\-t kathaṃ bhavet |&bā\-dhā\-to yadi sā\-py ekā\- pratyakṣā\-numayor nanu ||&pratyakṣe yady aviśvā\-sa ekatrā\-nyatra kā\- gatiḥ |&tatpū\-rvam anumā\-naṃ ca katham ā\-śvā\-sagocaraḥ || iti | \footnote{\begin{english}(JNA 391,1ff.)\end{english}}\&[\smallbreak]


	

	  \pstart nanu
	\pend
      
	    
	    \stanza[\smallbreak]
dṛṣṭam eva dvicandrā\-dipratibhā\-se 'pi bā\-dhitam |&na dṛṣṭe 'nupapannatvaṃ tajjñā\-tam api bā\-dhyate ||\&[\smallbreak]


	[[(JNA 391,13f.)]]

	  \pstart iti cet | na | bā\-dhyasyā\-pratibhā\-sanā\-t | pratibhā\-sinaś cā\-bā\-dhyatvā\-t | tathā\- hi
	\pend
      
	    
	    \stanza[\smallbreak]
buddhyā\-kā\-rasya nirbhā\-so bā\-dhā\- bā\-hyasya vastunaḥ |&sphū\-rtā\-v apy aviśvā\-se kva viśvā\-sa iti kī\-rtitam ||\&[\smallbreak]


	[[(JNA 391,16f.)]]

	  \pstart etena bhā\-samā\-no vā\-kā\-ro 'sann iti dvitī\-yo 'pi pakṣaḥ pratikṣiptaḥ, pratibhā\-sā\-d eva sattā\-siddher bā\-dhakā\-vakā\-śā\-bhā\-vā\-t | 
	\pend
      

	  \pstart tathā\- san vā\- kaścin na khyā\-tī\-ti tṛtī\-yasaṅkalpo 'pi vyā\-kulaḥ, prakā\-śavyā\-ptatvā\-t sattā\-yā\-ḥ | aprakā\-śasyā\-sattayā\- grastatvā\-t || 
	\pend
      

	  \pstart nanu prakā\-śo nā\-ma vastunaḥ sattā\-sā\-dhakaṃ pramā\-ṇam | na ca pramā\-ṇanivṛttā\-v arthā\-bhā\-vaḥ | arthakriyā\-śaktis tu sattvam | tac cā\-prakā\-śasyā\-pi na virudhyata iti cet | satyam etat | bahirarthavā\-de 'prakā\-śasyā\-pi sā\-marthyā\-bhyupagamā\-t | keśoṇḍukā\-dipratibhā\-se 'dhyavasitasyā\-rthakriyā\-śaktiviyogā\-d evā\-bhā\-vasiddheḥ | sarvathā\- bahirabhā\-ve tu jñā\-nasya prakā\-śā\-vyabhicā\-rā\-t tā\-vataiva sattve kim arthakriyayā\- | 
	\pend
      
	    
	    \stanza[\smallbreak]
katham anyahṛdaḥ sattvaṃ prakā\-śā\-d eva nā\-sya cet |&nā\-rthakriyā\-pi sarvasmai kvacic ced bhā\-sanaṃ na kim ||\edlabel{RNA-n-1}\footnote{\label{RNA-n-1}  \begin{english}(JNA 399,3f.)\end{english}}\&[\smallbreak]


	

	  \pstart iti | nirvikalpe tā\-vat svasaṃvedanasiddhasvā\-kā\-ram antareṇa vipakṣā\-dayo na parisphuranti | athā\-mī\- vikalpe pratibhā\-santa iti dvitī\-yaḥ saṅkalpo 'bhyupagamyate, asminn api pakṣe pratibhā\-samā\-na ā\-kā\-ro 'sā\-dhā\-raṇo 'śabdasaṃsṛṣṭatayā\- svasaṃvedanatā\-dā\-tmye praviṣṭatvā\-d vastusann eva | \edlabel{thakur75-132.32}\label{thakur75-132.32} adhyavseyatā\- vipakṣā\-dayo gṛhyanta iti cet | tadā\-pi teṣā\-ṃ svarū\-pasya nirbhā\-so 'sti na vā\- | nirbhā\-se pratyakṣasiddhataiva, nā\-satkhyā\-tiḥ | śā\-stre 'pi
	\pend
      
	    
	    \stanza[\smallbreak]
svarū\-pasā\-kṣā\-tkā\-ritvam eva pratyakṣatvam\&[\smallbreak]


	

	  \pstart uktam | tasya cetarapratyakṣeṣv iva vikalpe 'pi svī\-kā\-re viruddhavyā\-ptopalambhena vikalpabhrā\-ntatvayor dū\-ram apā\-statvā\-d vikalpe 'pi tvanmate pratyakṣatvam akṣatam | tat kathaṃ tatsiddhasya pratyakṣā\-ntarā\-numā\-nā\-bhyā\-ṃ bā\-dhā\-bhidhā\-nam, tayor api svarū\-pā\-ntaraprakā\-śapauruṣatvā\-t || \edlabel{thakur75-133.5}\label{thakur75-133.5} atha vikalpabhrā\-ntatvayor vyā\-pakaviruddhayoḥ sambhavā\-t vikalpe pratyakṣatvam evā\-sambhavi | nanv asya pratyakṣatvam asambhavī\-ti svarū\-pasā\-kṣā\-tkā\-ritvam asambhavī\-ty uktam | atha vipakṣā\-dir artho 'smin prakā\-śata iti vā\-cā\- svarū\-pasā\-kṣā\-tkā\-ritvaṃ kathitam iti mā\-tā\- me bandhyeti vṛttā\-ntaḥ | iṣyate ca tvayā\- vipakṣā\-disvarū\-pasā\-kṣā\-tkā\-ritvaṃ vikalpasyeti pratyakṣatā\-natikramaḥ, apratyakṣatve vastusvarū\-pasphuraṇā\-yogā\-t | tataś ca tatpratibhā\-sino 'pi rū\-pasya sata eva khyā\-tir nā\-satkhyā\-tiḥ | na ca tad eva vikalpe parisphuradrū\-pam asatā\-m ī\-śvarā\-dī\-nā\-ṃ svarū\-pam, asattvasyaivā\-bhā\-vaprasaṅgā\-t | svarū\-pasphuraṇe 'py asattve 'nyatrā\-pi prakā\-śiny anā\-śvā\-sā\-t | tato yat sā\-kā\-ravā\-de jalpitam
	\pend
      
	    
	    \stanza[\smallbreak]
nityā\-dayaḥ santa eva syuḥ\&[\smallbreak]


	

	  \pstart iti tadā\-tmana eva patitam |\footnote{tato yat --- patitam Ce'e JNA 392,15f. (has evā\-patitam).} yad ā\-hur \name{guruvaḥ}
	\pend
      
	    
	    \stanza[\smallbreak]
svarū\-pasā\-kṣā\-tkaraṇā\-d adhyakṣatvaṃ na cā\-param |&vikalpabhramabhū\-mitvam ata eva hi bā\-dhitam ||\footnote{Ce' JNA 391,5f. For ab also cf. JNA 563,5.}&yadi nā\-dhyakṣatā\- tasya rū\-panirbhā\-sa eva na |&tatas tadasadī\-śā\-di pratibhā\-tī\-ty asaṅgatam ||&yadi tu pratibhā\-seta rū\-pam asya sad eva tat |&tad asat pratibhā\-tī\-ti tac ca bhā\-ty asad eva vaḥ || \footnote{\begin{english}(JNA 391,7ff.)\end{english}}\&[\smallbreak]


	

	  \pstart athā\-dhyavasā\-ye 'dhyavaseyasvarū\-pasya pratibhā\-so nā\-stī\-ty ucyate | na tadā\- kasyacid adhyavasā\-yaḥ | katham ataḥ pratiniyatavastuvyavasthā\-siddhiḥ | kiṃ ca ko 'yam adhyavasā\-yo nā\-ma | kiṃ vyā\-vṛttibhedaparikalpitasya prakā\-śā\-ṃśasya, svā\-kā\-rā\-ṃśasya, alī\-kā\-kā\-rasya, bā\-hyavastuno 'vastuno vā\- sphuraṇam adhyavasā\-yā\-rthaḥ | yadi vā\- svā\-kā\-re bā\-hyā\-ropaḥ, bā\-hye vā\- svā\-kā\-rā\-ropaḥ, svā\-kā\-rabā\-hyayor yojanā\-, tayor ekī\-karaṇam ekapratipattir abhedena pratipattiḥ, bhedā\-graho 'dhyavasā\-yā\-rtha iti vikalpā\-ḥ | \edlabel{thakur75-133.28}\label{thakur75-133.28} tatra na tā\-vad ā\-dimau pakṣau kalpanā\-m arthaḥ | svarū\-pe sarvasyaiva sphuraṇasya nirvikalpatvā\-d avasā\-yā\-nupapattiḥ | itarathā\- nirvikalpakajñā\-nā\-bhā\-vaprasaṅgā\-t | \edlabel{thakur75-133.30}\label{thakur75-133.30} alī\-kasphuraṇaṃ tu prā\-k pratyā\-khyā\-tam | saty api sphuraṇe 'sphuṭatvā\-n nirvikalpakam etat | dvicandrā\-dijñā\-navat | astu svagrā\-hye tannirvikalpakam, bā\-hye tu adhyavaseye adhyavasā\-ya iti cet | na | tatsambandhā\-bhā\-vā\-ta, tadapratibhā\-sā\-c ca | anyathā\-tiprasaṅgā\-d ity uktaprā\-yam | bā\-hyavastusvarū\-pasphuraṇe tu pratyakṣapratipattir evā\-sā\-v iti ko 'dhyavasā\-yaḥ | avastusphuraṇaṃ punas tridhā\- vikalpya prā\-g eva pratyā\-khyā\-tam | \edlabel{thakur75-134.4}\label{thakur75-134.4} svā\-kā\-re tu bā\-hyā\-ropo na sambhavaty eva | tathā\- hi jñā\-naṃ kenacid ā\-kā\-reṇa satyenā\-lī\-kena vopajā\-taṃ nā\-ma | bā\-hyā\-ropas tu tadā\-kā\-re tatkṛto 'nyakṛto vā\- syā\-t | tatkṛtatve na tā\-vat tatkā\-la eva vyā\-pā\-rā\-ntaram anubhū\-yata iti kutas tadā\-ropaḥ | kā\-lā\-ntare ca svayam evā\-sat kasya vyā\-pā\-raḥ syā\-t |
	\pend
      

	  \pstart dvitī\-yapakṣe jñā\-nā\-ntaram api nā\-kā\-rā\-roparā\-gasaṅginī\-m utpattim antareṇa vyā\-pā\-rā\-ntareṇa kvacit kiñcitkaraṃ nā\-ma | tad etad arvā\-cī\-najñā\-nasadṛśā\-kā\-ragocarī\-karaṇe 'pi na bā\-hyā\-ropavyā\-pā\-ram aparaṃ spṛśati tadā\-kā\-raleśā\-nukā\-ram apahā\-ya | na ca śabdā\-mukhī\-karaṇam atirikto vyā\-pā\-raḥ, śabdā\-kā\-rasyā\-pi svarū\-pa evā\-ntarbhā\-vā\-d iti nā\-kā\-rā\-d anyo jñā\-navyā\-pā\-raḥ | ā\-ropyamā\-ṇaś cā\-sā\-v artho bā\-hyaḥ | tatra buddhau yadi svarū\-peṇa sphurati satyapratī\-tir evā\-sau, ka ā\-ropaḥ | atha na parisphurati tathā\-pi ka ā\-ropaḥ | sphuraṇe vā\-dhikaraṇabhū\-tasvā\-kā\-rā\-tiriktasyā\-ropyamā\-ṇā\-kā\-rasyā\-pi pratibhā\-saprasaṅgaḥ |\footnote{Parallel to 388.7--18}
	\pend
      

	  \pstart tadā\-kā\-rasphuraṇam eva tasya sphuraṇam iti cet | na | tasyā\-ropaviṣayatvā\-t | na hi marī\-cisphuraṇam eva jalasphuraṇam iti na svā\-kā\-re bā\-hyā\-ropaḥ |
	\pend
      

	  \pstart ata eva bā\-hye svā\-kā\-rā\-ropo nā\-sti, ā\-ropaviṣayasya bā\-hyasyā\-sphuraṇā\-t |
	\pend
      

	  \pstart tata eva svā\-kā\-rabā\-hyayor yojanā\-py asambhavinī\-, yogyayor apratibhā\-sā\-t |
	\pend
      

	  \pstart na caikī\-karaṇam adhyavasā\-yaḥ | ko 'yam ekī\-karaṇā\-rthaḥ | yady ekatā\-patau prayojakatvaṃ tadā\-ropyā\-ropaviṣayayoḥ kadā\-cid ekī\-bhā\-vā\-bhā\-vā\-d asambhava eva | na hi śaśaviṣā\-ṇe kā\-raṇaṃ kiñcit | na ca pū\-rvam anekam ekatā\-m etī\-ti kṣaṇikavā\-dinaḥ sā\-ṃpratam | arthā\-ntarotpattimā\-traṃ tu syā\-t | na ca tadupalabdhigocaro 'nyatrā\-ropaviṣayā\-t svā\-kā\-rā\-t | na ca tā\-vatā\-py arthasya kiñcid iti katham ekī\-karaṇam |
	\pend
      

	  \pstart athaikapratī\-tir adhyavasā\-yaḥ | tathā\-pi na dvayor ekapratipattir adhyavaseyā\-nubhavā\-bhā\-vā\-t | na ca dvayoḥ pratī\-tir ity evā\-dhyavasā\-yaḥ nī\-lapī\-tavat | \edlabel{thakur75-134.27}\label{thakur75-134.27}
	\pend
      

	  \pstart na cā\-bhedena pratī\-tir adhyavasā\-yaḥ | yataḥ paryudā\-sapakṣe aikyapratī\-tir uktā\- bhavati | sā\- ca prayuktā\-, adhyavaseyapratyabhā\-vā\-t | bhedena pratī\-tiniṣedhamā\-tre 'pi na bā\-hyasya pratī\-tir ukteti kutas tadadhyavasā\-yaḥ | yadi hi bā\-hyaṃ prakā\-śeta ekatvenā\-nekatvena vā\- satā\- asatā\- vā\- pratī\-tir iti yuktam | 
	\pend
      

	  \pstart sarvā\-kā\-ratatsvarū\-patiraskā\-reṇa sā\- pratī\-tir ity ekapratī\-tir iti cet | tatsvarū\-patiraskā\-re tarhi tadapratibhā\-sanam eva | kasyacid aṃśasya pratibhā\-sanā\-d iti cet | na | niraṃśatvā\-d vastunaḥ sarvā\-tmanā\- pratibhā\-so 'pratibhā\-so veti śā\-stram evā\-tra vistareṇa parī\-kṣyate | \edlabel{thakur75-135.1}\label{thakur75-135.1} na ca bhedā\-graho 'dhyavasā\-yo vaktavyaḥ | tathā\- hi kiṃ bā\-hye gṛhyamā\-ṇe 'grahyamā\-ṇe vā\- | na ca prathamaḥ pakṣaḥ, bā\-hyagrahaṇasya pratikṣiptatvā\-t | grahaṇe vā\-dhyavasā\-yasya pratyakṣatā\-prasaṅgā\-t | agṛhyamā\-ṇe tu bā\-hye pravṛttiniyamo na syā\-t, anyeṣā\-m api tadā\-nī\-m agrahaṇā\-d anyatrā\-pi pravṛttiprasaṅgā\-t |
	\pend
      

	  \pstart \persName{trilocano} 'pī\-ttham adhyavasā\-yaṃ dū\-ṣayati | ko 'yam adhyavasā\-yaḥ | kiṃ grahaṇam, ahosvit karaṇam, uta yojanā\-, atha samā\-ropaḥ | tatra svā\-bhā\-sam anartham arthaṃ kathaṃ gṛhṇī\-yā\-t, kuryā\-d vā\- vikalpaḥ | na hi nī\-laṃ pī\-taṃ śakyaṃ grahī\-tuṃ kartuṃ vā\- śilpakuśalenā\-pi | nā\-py agṛhī\-tena svalakṣaṇena svā\-kā\-raṃ yojayitum arhati vikalpaḥ | na ca svalakṣaṇaṃ vikalpagrahaṇagocaraḥ | na ca svā\-kā\-ram anartham artham ā\-ropayati | na tā\-vad agṛhī\-tasvā\-kā\-raḥ śakya ā\-ropayitum iti tadgrahaṇam eṣitavayam | tatra kiṃ gṛhī\-tvā\- ā\-ropayati, atha yadaiva svā\-kā\-raṃ gṛhṇati tadaivā\-ropayati | nā\-dyaḥ | na hi kṣaṇikaṃ vikalpavijñā\-naṃ kramavantau grahaṇasamā\-ropau kartum arhati | uttarasmiṃs tu kalpe 'vikalpasvasaṃvedanapratyakṣā\-d vikalpā\-kā\-rā\-d ahaṅkā\-rā\-spadā\-d \edtext{}{\lemma{---}\Afootnote{\label{RNA-tc-2}anahaṅkā\-rā\-spadaṃ \cite{}; anahaṅkā\-rā\-spadaḥ \cite{} \cite{}}} samā\-ropyamā\-ṇo vikalpena svagocaro na śakyo 'bhinnaḥ pratipattum | nā\-pi bā\-hya\edtext{}{\lemma{---}\Afootnote{\label{RNA-tc-3}svalakṣaṇaikatvena \cite{}; svalakṣaṇakatvena \cite{}}} śakyaḥ pratipattum, vikalpā\-kā\-re svalakṣaṇasya bā\-hyasyā\-pratibhā\-sanā\-d iti |
	\pend
      

	  \pstart \persName{vā\-caspatir} apy adhyavasā\-yaṃ pratikṣipati | \edlabel{quote-nk-adhyavasāya-start}\label{quote-nk-adhyavasāya-start} anarthaṃ svā\-bhā\-sam artham adhyavasyatī\-ti nirvacanī\-yam etat | nanv ayam ā\-ropayatī\-ti kiṃ vikalpasya svarū\-pā\-nubhava evā\-ropaḥ, uta vyā\-pā\-rā\-ntaraṃ svarū\-pā\-nubhavā\-t | na tā\-vat pū\-rvaḥ kalpaḥ, anubhavasamā\-ropayor vikalpā\-vikalparū\-patayā\- dravakaṭhinavat tā\-dā\-tmyā\-nupapatteḥ | vyā\-pā\-rā\-ntaratve tu kramaḥ samā\-nakā\-latā\- vā\- | na tā\-vat kramaḥ, kṣaṇikasya vijñā\-nasya kramavadvyā\-pā\-rā\-yogā\-t | akṣaṇikavā\-dinā\-m api buddhikarmaṇor viramya vyā\-pā\-rā\-nupapatteḥ na kramavadvyā\-pā\-rasambhavaḥ | anubhavasamā\-ropau samā\-nakā\-lā\-v iti cet | bhavatu samā\-nakā\-latvaṃ kevalam | ā\-tmā\- svabhā\-vasthita eva vedyaḥ, parabhā\-vena vedane svarū\-pavedanā\-nupapatteḥ | tathā\- cā\-tmā\- jñā\-nasya grā\-hyagrā\-hakā\-kā\-ro 'nubhū\-to 'rthaś ca samā\-ropitaḥ | na tv ā\-tmā\- vedyamā\-naḥ samā\-ropito nā\-rthaḥ samā\-ropyamā\-ṇaḥ pratyakṣavedyaḥ | sa ca samā\-ropaḥ sato 'sato vā\- grahaṇam eva | na ca jñā\-nā\-tiriktasya grahaṇaṃ sambhavatī\-ty upapā\-ditam |\edlabel{thakur75-135.27}\label{thakur75-135.27} svapratibhā\-sasya bā\-hyā\-d bhedā\-graho bā\-hyasamā\-ropas tato bā\-hye vṛttir iti cet | sa kiṃ gṛhyamā\-ṇe bā\-hye na vā\- | na tā\-vad gṛhyamā\-ṇe | uktaṃ hy etan na \edtext{}{\lemma{---}\Afootnote{tadagrahaṇaṃ \cite{}; tadgrahaḥ \cite{}}} sambhavatī\-ti | agṛhyamā\-ṇe tu bhedā\-grahe na pravṛttiniyamaḥ syā\-t, anyeṣā\-m api tadā\-nī\-m agrahā\-d anyatrā\-pi pravṛttiprasaṅgā\-d iti \edlabel{quote-nk-adhyavasāya-end}\label{quote-nk-adhyavasāya-end}|\footnote{\begin{english}\href{quote-nk-adhyavasāya-start}{quote-nk-adhyavasā\-ya-start} to \href{quote-nk-adhyavasāya-end}{quote-nk-adhyavasā\-ya-end} is a quote from \cref{NK1}.\end{english}} tasmā\-d yathā\- yathā\-yam adhyavasā\-yaś cintyate tathā\- tathā\- viśī\-ryata eva | tathā\- vikalpā\-ropā\-bhimā\-nagrahaniścayā\-dayo 'py adhyavasā\-yavat svā\-kā\-raparyavasitā\- eva sphuranto bā\-hyasya vā\-rtā\-mā\-tram api na jā\-nantī\-ty adhyavasā\-yasvabhā\-vā\- eva \edtext{}{\lemma{---}\Afootnote{śabdapravṛttinimittabhede \cite{}; śabdapravṛttimittabhede \cite{}  {\rmlatinfont [App type: misprint]}}} 'pi, tat kathaṃ yuktyā\-gamā\-bahirbhū\-to\footnote{\cref{thakur75} suggests an emendation to yuktyā\-gamabahirbhū\-to, but that seems unnecessary.} 'nā\-tmā\-sphuraṇam ā\-cakṣī\-ta | \edlabel{thakur75-136.3}\label{thakur75-136.3} nanv evaṃ vikalpā\-dī\-nā\-m asambhave sambhave 'py anā\-tmaprakā\-śakatvā\-nabhyupagame sarvajanaprasiddhavidhipratiṣedhavyavahā\-rocchedaprasaṅga iti lokavirodhaḥ | vikalpa ity adhyavasā\-ya ity ā\-ropa ity abhimā\-na iti graha iti niścaya ityā\-dikaṃ śā\-stre pratipadaṃ pratipā\-ditam, tatsiddhaṃ ca bahirarthā\-dikam abhyupagatam ity ā\-cā\-ryavirodhaḥ, nyā\-yavirodhaś ca | tathā\- hi sarvair eva prakā\-rair\footnote{\begin{english}``prakā\-śair" acc. to PPU and Sā\-SiŚā\-.\end{english}} aviparī\-tasvarū\-pasaṃvedanā\-d bhrā\-nter atyantam abhā\-vaḥ syā\-t | tataś ca sarvasattvā\-ḥ sadaiva samyaksambuddhā\- bhaveyuḥ | \edlabel{thakur75-136.9}\label{thakur75-136.9} vikalpikā\- buddhir brā\-ntiḥ, svapratibhā\-se 'narthe 'rthā\-dhyavasā\-yā\-d iti cet | katham avasī\-yamā\-nas tayā\- so 'rtho na prakā\-śate | prakā\-śamā\-no vā\- katham asau tasyā\-ṃ na prakā\-śate | atha prakā\-śata eva, tadā\-rthasya tā\-dā\-tmyaprasaṅgaḥ | asati cā\-rthe sā\-rasyā\-t abhū\-n mā\-ndhā\-tā\-, bhaviṣyati śaṅkho 'styā\-tmā\-, nityaḥ śabda iti sarvā\-tmanā\- ca niścayaḥ syā\-t | gaur iti spaṣṭena ca svena lakṣaṇena prakā\-śeta | svalakṣaṇe ca saṅketā\-yogā\-t vikalpikaiva sā\- buddhir na syā\-t | tasmā\-d aśeṣagovyaktisā\-dhā\-raṇena gotvena gobuddhir alī\-kena sā\-bhilā\-pena viplavā\-t prakhyā\-tī\-ti tathā\- prakā\-śanam asyā\- gavā\-rthā\-vasā\-ya ity eṣṭavyam | evaṃ hy ete doṣā\- na syuḥ, apratibhā\-samā\-nasyā\-pi svalakṣaṇasya bhrā\-ntyā\-vasā\-yā\-d iti || \edlabel{thakur75-136.18}\label{thakur75-136.18} atrā\-bhidhī\-yate | na tā\-val lokaśā\-straviro\edlabel{capv-np-10a-start}\label{capv-np-10a-start}dhau, agṛhī\-te 'pi bā\-hye pravṛttinivṛttyā\-disamarthanā\-t svaparavā\-diduratikramā\-dhyavasā\-yasvarū\-panirvacanā\-t | nyā\-yavirodhasya tu gandho 'pi nā\-sti | \edlabel{thakur75-136.21}\label{thakur75-136.21} tathā\- hi kā\- punar ayaṃ bhrā\-ntir asatkhyā\-tiratasmiṃs tadgraho vā\- yadabhā\-vā\-didā\-nī\-m eva muktir ā\-sajyate | \edlabel{thakur75-136.23}\label{thakur75-136.23} na tā\-vad ā\-dyaḥ pakṣaḥ, asatkhyā\-teḥ pratyā\-khyā\-nā\-t | yad ā\-hur guruvaḥ
	\pend
      
	    
	    \stanza[\smallbreak]
yasya svarū\-panirbhā\-so bā\-dhakā\-d yadi tan na sat |&bā\-dhake 'pi ka ā\-śvā\-saḥ svarū\-pā\-ntarabhā\-sini ||&anyasvarū\-popanayā\-t tatsvarū\-panivā\-raṇam |&tatrā\-pi saṃśayo jā\-taḥ pū\-rvabā\-dhopalabdhitaḥ ||&iyam evā\-grahe bā\-dhā\- nā\-dyajasyā\-parā\- yadi |&asya pū\-rvaiva bhavatu rū\-panirbhā\-sanaṃ samam || &nyā\-yā\- ca bhā\-vinī\-ty atra pramā\-ṇaṃ kiñcid asti vaḥ |&api svarū\-panirbhā\-se yadā\- bā\-dhakasambhavaḥ ||&anirbhā\-se svarū\-pasya hetuśodhanaviplave |&bā\-dhaśaṅkā\-vinirbhā\-se 'py evaṃ ced viplavo mahā\-n || iti ||\edlabel{RNA-n-2}\footnote{\label{RNA-n-2}  \begin{english}JNA 392,19-393,3\end{english}}\&[\smallbreak]


	

	  \pstart śā\-stre ca atasmiṃs tadgrahā\-t svapratibhā\-se 'narthe 'rthā\-dhyavasā\-yā\-d dṛśyavikalpyayor ekī\-karaṇā\-d bhrā\-ntir uktā\- | tā\-m ayaṃ samarthayitum asamarthaḥ svā\-tantryeṇā\-lī\-kasphuraṇaṃ bhrā\-ntir iti kā\-vyaṃ viracayya vistā\-rayati || 
	\pend
      

	  \pstart nanv atasmiṃs tadgraho 'pi bhramaḥ svā\-\edlabel{capv-np-10a-end}\label{capv-np-10a-end}\edlabel{capv-np-10b-start}\label{capv-np-10b-start}kā\-raparyavasitajñā\-nā\-d atirikto bahubhir bahudhā\- vicā\-rya pratyā\-khyā\-taḥ | tat kathaṃ tasminn api pakṣe na bhrā\-ntikṣatir yenedā\-nī\-m eva muktiprasaṅgo na syā\-d iti cet | tad etad bhagavato \persName{bhā\-ṣyakā\-rasya} matavidveṣaviṣavyā\-kulavikrośitam atikā\-tarayati kṛpā\-paravaśadhiyaḥ | tathā\- hi samanantarapratyayabalā\-yā\-tasvapratibhā\-saviśeṣavedanamā\-trā\-d agṛhī\-te 'pi paratra pravṛttyā\-kṣepo 'dhyavasā\-yaḥ | na cā\-sau pū\-rvoktavā\-gjā\-laiḥ pratihantuṃ śakyaḥ, sarvaprā\-ṇabhṛtā\-ṃ pratyā\-tmaviditatvā\-t, kaiścid apy anudbhinnatvā\-t | ayam eva ca saṃsā\-ras tatkṣayo mokṣa iti kvedā\-nī\-m eva tadvā\-rtā\-pi | tathā\- hi vicitrā\-nā\-divā\-sanā\-vaśā\-t prabodhakapratyayaviśeṣā\-pekṣayā\- vikalpaḥ kenacid ā\-kā\-reṇopajā\-yamā\-na eva bahir mukhapravṛttyanukū\-lam arthakriyā\-smaraṇā\-bhilā\-ṣā\-diprabandham ā\-dhatte | tataḥ puruṣā\-rthakriyā\-rthino bahirarthā\-nurū\-pā\-ṇi pravṛttinivṛttyavadhā\-raṇā\-ni bhavanti | pṛthagjanasantā\-najñā\-na\edtext{}{\lemma{---}\Afootnote{kṣaṇā\-nā\-ṃ \cite{} \cite{}; lakṣaṇā\-nā\-ṃ \cite{}}} tā\-dṛśo hetuphalabhā\-vasya niyatatvā\-t | aniścitā\-rthasambandhavikalpakā\-le 'pi sada\edlabel{capv-np-10b-end}\label{capv-np-10b-end}sattā\-nirṇayā\-dipravṛttiprasavaḥ | tatra yadubhayathā\- pravṛttisā\-dhanasā\-marthyam asya svahetubalā\-yā\-tam ayam eva pravṛttiviṣayatvā\-ropo 'dhyavasā\-yā\-paranā\-mā\- | yathā\- candrā\-dijñā\-nasya bhrā\-ntasyā\-bhrā\-ntasya vā\- taddarśanā\-vasā\-yajananam eva grahaṇavyā\-pā\-raḥ |
	\pend
      
	    
	    \stanza[\smallbreak]
svavid apī\-yam arthavid eva kā\-ryato draṣṭavyeti \footnote{Cf. \textbackslash cite〔349〕\{pv3.320toend\}: 
	    \begin{verse}
	  yathā\- niviśate so 'rtho yataḥ sā\- prathate tathā\- |\\
	    arthasthites tadā\-tmatvā\-t svavid apy arthavin matā\- ||\\
	    
	    \end{verse}
	  }\&[\smallbreak]


	

	  \pstart nyā\-yā\-t | tathā\- vikalpasyā\-py agnir atretyā\-dinā\-kā\-reṇotpadyamā\-nasya pravṛttyā\-kṣepakatvam eva bā\-hyā\-vasā\-naṃ nā\-ma | yathā\- ca nirvikalpadvicandrā\-dyā\-kā\-rataiva tathā\-vasā\-yasā\-dhanī\-, evam avasā\-yasyā\-pi tā\-dṛśā\-kā\-rataiva viṣayā\-ntaravimukhapravṛttisā\-dhanī\- || \edlabel{thakur75-137.25}\label{thakur75-137.25} nanu tathā\- ca tac ca tena pratipā\-dyate na ca tajjñā\-ne tat prakā\-śata iti śapathenā\-pi na saṃpratyaya iti cet | asambaddham etat | na hy adhyavasā\-yā\-d bā\-hyasya paṭā\-der vastuno bā\-dhakā\-vatā\-rā\-t pū\-rvasandigdhavastubhā\-vasya kṣaṇikā\-der avastuno vā\- śaśaviṣā\-ṇā\-der asphuraṇe 'pi siddhipratibandho brahmaṇā\-pi pratividhā\-tuṃ śakyaḥ | dvividho hi viṣayavyavahā\-raḥ, pratibhā\-sā\-d adhyavasā\-yā\-c ca | tad iha pratibhā\-sā\-bhā\-ve 'pi parā\-poḍhasvalakṣaṇā\-der adhyavasā\-yamā\-treṇa viṣayatvam uktam, sarvathā\- nirviṣayatve pravṛttinivṛttyā\-disakalavyavahā\-rocchedaprasaṅgā\-t | tataś ca tena ca tat pratipā\-dyate na ca jñā\-ne tatprakā\-śa iti saṅgatir asty eva, prakā\-śyaprakā\-śakabhā\-vā\-bhā\-ve 'py \edtext{adhyavaseya}{\Afootnote{\label{RNA-tc-4} \cite{}adhyavasā\-yā\- \cite{}}}dhyavasā\-yakabhā\-venā\-pi viṣayaviṣayibhā\-vopapatteḥ | \edlabel{thakur75-138.1}\label{thakur75-138.1} nanu yadi nā\-dhyavaseyapratī\-tis tadā\-gṛhī\-te 'pi svalakṣaṇā\-dau pravṛttir iti sarvatrā\-viśeṣeṇa prasajyeta, sarvatrā\-gṛhī\-tatvena viśeṣā\-bhā\-vā\-t, tataś ca prā\-ptir api nā\-bhimatasya niyamenety anumā\-nam api viplutam | atra brū\-maḥ | yady adhyavaseyam agṛhī\-taṃ viśvam apy agṛhī\-tam, tathā\-pi niyataviṣayaiva pravṛttir na sarvatra, tathā\-bhū\-tasamanantarapratyayabalā\-yā\-taniyatā\-kā\-ratayā\- niyataśaktitvā\-d vikalapasya | niyataśaktayo bhā\-vā\- hi pramā\-ṇapariniṣṭhitasvabhā\-vā\-ḥ, na śaktisā\-ṅkaryaparyanuyogabhā\-jaḥ, asadutpattivat | sarvatrā\-sattve 'pi hi bī\-jā\-d aṅkurasyaivotpattiḥ, tatraiva tasya śakteḥ pramā\-ṇena nirū\-paṇā\-t | tathehā\-pi hutavahā\-kā\-rasya vikalpasya dā\-hapā\-kā\-dyarthakriyā\-rthinas tatsmaraṇavato hutavahaviṣayā\-yā\-m eva pravṛttau sā\-marthyaṃ pramā\-ṇapratī\-taṃ katham atiprasaṅgabhā\-gi | pratyā\-satticintā\-yā\-ṃ ca tā\-ttvikasyā\-pi vahner jvaladbhā\-svarā\-kā\-\edlabel{capv-np-13a-start}\label{capv-np-13a-start}ratvaṃ vikalpollikhitasyā\-pī\-ti, tā\-vatā\- tatraiva pravartanaśaktir jvalanavikalpasya na jalā\-dau ||
	\pend
      

	  \pstart nanu ca sā\-dṛśyā\-ropeṇa kiṃ svā\-kā\-rasya bā\-hye svā\-kā\-re vā\- bā\-hyasyā\-ropaḥ | ubhayathā\-py asaṅgatiḥ, ā\-ropyā\-ropaviṣayayoḥ svā\-kā\-rabā\-hyayor dvayor grahaṇā\-sambhavā\-d iti cet | na vayam ā\-ropeṇa pravṛttiṃ brū\-maḥ | kiṃ tarhi, svavā\-sanā\-paripā\-kavaśā\-d upajā\-yamā\-naiva sā\- buddhir apaśyanty api bā\-hyaṃ bā\-hye pravṛttimā\-tanotī\-ti viplutaiva saṃsā\-rā\-tmikā\- ca | yat śā\-straṃ 
	\pend
      
	    
	    \stanza[\smallbreak]
na jñā\-ne tulyam utpattito dhiyaḥ |&tathā\-vidhā\-yā\-ḥ \&[\smallbreak]


	

	  \pstart iti | tasmā\-n na rū\-pyā\-divad ā\-ropadvā\-reṇa pravṛttir api tu tathā\-vidhā\-kā\-rotpattipratibaddhaśaktiniyamā\-t | na ca vicā\-rakasya vastvadarśananiścayā\-d apravṛttiḥ saṅgacchate | darśane 'pi hi pravṛttir arthakriyā\-rthitayā\- | arthakriyā\-prā\-ptiś ca vastusattā\-niyame | sa ca niyamo yathā\- darśanā\-d vastupratibandhakṛtaḥ, tathā\- vikalpaviśeṣā\-d api pā\-ramparyeṇa vastuprativastupratibandhakṛta ity adarśane 'pi adhyavasā\-yā\-t pravṛttir yujyata iti nā\-numā\-nam anavasthitam | etena tac ca na pratī\-yate, tena cā\-bhedā\-bhā\-sanam ity upā\-lambho 'sambhavī\-\edtext{ty upadarśitam}{\Afootnote{ti darśitaḥ \cite{}}}\edlabel{capv-np-13a-end}\label{capv-np-13a-end}\edlabel{capv-np-13b-start}\label{capv-np-13b-start}, apratibhā\-se 'pi pravṛttiviṣayī\-karaṇam ity abhedā\-diniṣṭhā\-yā\- darśitatvā\-t | tasmā\-d avicā\-raramaṇī\-yo 'tasmiṃs tatgraha eva bhrā\-ntir ā\-ropā\-paranā\-mā\-, tatkṣayaś ca mokṣa iti yuktam |
	\pend
      

	  \pstart yad ā\-hur guruvaḥ
	\pend
      
	    
	    \stanza[\smallbreak]
tasmā\-t pravṛtter ā\-kṣepe vikalpā\-kā\-rajanmani |&mato jalā\-dyā\-ropo 'pi satyā\-satyasamaś ca saḥ ||&tato yady api tattvena nā\-ropo nā\-ma kasyacit |&vyavahā\-rakṛtas tv eṣa pratiṣeddhuṃ na śakyate ||&marī\-cau jalavad yā\-vad anā\-tmany ā\-tmakalpanam |&bhrama eva hi saṃsā\-ro nirvā\-ṇaṃ tattvasaṃsthitiḥ ||&tataś ca yā\-van na vicā\-rasambhavo bhavo 'yam anyaḥ śama ity ayaṃ nayaḥ |&vicā\-ralī\-lā\-lalite tu mā\-nase bhavaḥ śamo vā\- ka iheti kathyatā\-m ||\footnote{JNA 554,17-25}\&[\smallbreak]


	

	  \pstart tathā\- Āryamaitreyanā\-thapā\-dā\- api 
	\pend
      
	    
	    \stanza[\smallbreak]
na cā\-ntaraṃ kiṃcana vidyate 'nayoḥ sadarthavṛttyā\- śamajanmanor iha |&tathā\-pi janmakṣayato vidhī\-yate śamasya lā\-bhaḥ śubhakarmakā\-riṇā\-m ||\&[\smallbreak]


	

	  \pstart Āryanā\-gā\-rjunapā\-dā\-ś ca 
	\pend
      
	    
	    \stanza[\smallbreak]
nirvā\-ṇaṃ ca bhavaś caiva dvayam eva na vidyate |&parijñā\-naṃ bhavasyaiva nirvā\-ṇam iti kathyate ||\&[\smallbreak]


	

	  \pstart iti sarvair eva prakā\-śair aviparī\-tasvarū\-pasaṃvedane 'pi bhrā\-ntivyavasthā\-sambhavā\-d asti saṃsā\-raḥ || 
	\pend
      

	  \pstart yad apy uktaṃ vikalpasyā\-viṣayaś ca bā\-hyam grahaṇaṃ cā\-sya śabdena saṃyojyeti vikalpatvam api duryojam, ā\-\edlabel{capv-np-13b-end}\label{capv-np-13b-end}\footnote{\begin{english}This is where folio 13 of \cref{capv-np} ends.\end{english}}tmani ca śabdayojanā\- nā\-stī\-ti vikalpo nā\-ma nā\-sty eva, tat kasya vikalpacinteti | atrā\-bhidhī\-yate | ihā\-gnir atrety adhyavasā\-yo yathā\- kā\-yikī\-ṃ vṛttiṃ prasū\-te tathā\-gnir mayā\- pratī\-yata iti vā\-cikī\-m api prasū\-te, etadā\-kā\-rā\-nuvyavasā\-yarū\-pā\-ṃ mā\-nasī\-m api prasavati | evaṃ ca sati yathā\- vikalpenā\-yam artho gṛhī\-ta iti niścayaḥ, tathā\- śabdena saṃyojya gṛhī\-ta ity api, arthā\-kā\-raleśavac chabdā\-kā\-rasyā\-pi sphuraṇā\-t | tasmā\-d arthagrahā\-bhimā\-navā\-n mā\-navastā\-vad abhidhā\-nasaṃyuktagrahaṇā\-bhimā\-navā\-n apī\-ty avasā\-yā\-nurodhā\-d eva vikalpavyavasthā\- na tattvataḥ | yad ā\-hur guravaḥ
	\pend
      
	    
	    \stanza[\smallbreak]
na śabdaiḥ saṃsargaḥ kvacid api bahir vā\- manasi vā\-kṣarā\-kā\-rā\-kī\-rṇaḥ sphurati punar arthā\-kṛtilavaḥ | &ubhā\-v apy ā\-kā\-rau yad api dhiya evā\-dhyavasitir vidhatte tau bā\-hye vacasi ca vikalpasthitir ataḥ ||\edlabel{RNA-n-3}\footnote{\label{RNA-n-3}  \begin{english}JNA 227,6ff.\end{english}}&abhā\-ne pratibhā\-ne vā\- na cā\-ropo 'pi kasyā\-cit |&pratī\-tyotpā\-dabhedena vyavasthā\-mā\-tram\edtext{īdṛśaḥ}{\Afootnote{\label{RNA-tc-5} \cite{}ī\-dṛśam \cite{}}} ||&nirvikalpā\-d vikalpasya bhā\-ve leśā\-nukā\-riṇaḥ |&saṅketakā\-rivacanā\-d buddhyā\-kā\-re viśeṣiṇi ||&saṅketaḥ kṛta ityā\-sthā\- tā\-dṛk śabdaśrutau punaḥ |&pravṛttyā\-kṣepabuddhyā\-tmabhā\-ve vā\-cyavyavasthitiḥ || iti | \edlabel{RNA-n-4}\footnote{\label{RNA-n-4}  \begin{english}JNA 554,11-16\end{english}}\&[\smallbreak]


	

	  \pstart tasmā\-d vastu vā\- ghaṭapaṭā\-di sandigdhavastu vā\- sā\-dhakabā\-dhakā\-tikrā\-ntam, avastu vā\-tmadikkā\-lā\-kṣaṇikā\-dikam adhyavasitam iti, apratibhā\-se 'pi pravṛttiviṣayī\-kṛtam ity arthaḥ | ayam eva cā\-ropaikī\-karaṇā\-dhyavasā\-yā\-bhedagrahā\-dī\-nā\-m arthaḥ sarvatra śā\-stre boddhavyaḥ | tasmā\-d adhyavasā\-yasyā\-kā\-raviśeṣayogā\-d agṛhī\-te 'pi pravartanayogyatā\- nā\-ma yo dharmas tayā\- bā\-hyā\-dhyavasā\-yayor grā\-hyagrā\-hakabhā\-vaś cet savṛttyā\- duṣpariharaḥ, tadā\- viṣayiviṣayabhā\-vo 'pi labdha ity adhyavasā\-yamā\-treṇa viṣayaviṣayitvam uktam iti yuktam | yad ā\-ha \persName{Alaṅkā\-rakā\-raḥ}
	\pend
      
	    
	    \stanza[\smallbreak]
kathaṃ tadviṣayatvaṃ tatra pravartanā\-d iti | \&[\smallbreak]


	

	  \pstart etena yad uktam, katham avasī\-yamā\-nas \edtext{tayā so 'rtho}{\Afootnote{\label{RNA-tc-6}tayā\-sortho \cite{}; tayā\-tmā\-rtho \cite{}; tayā\- so 'rtho \cite{}}} na prakā\-śyata ityā\-di, tan nirastam, tadaprakā\-śe 'pi tadadhyavasā\-yasya vyavasthā\-pitatvā\-t | asati cā\-rthe sā\- na syā\-d ity apy ayuktam, ā\-tmā\-der adhyavaseyasya pratibhā\-sapratikṣepe buddhyā\- saha tā\-dā\-tmyā\-bhā\-vā\-t | na ca sarvā\-kā\-raniścayaprasaṅgadoṣaḥ saṅgataḥ | sarvā\-kā\-raniścayo hi sarveṣv ā\-kā\-reṣu pravṛttikā\-rakatvā\-tmā\- niruktaḥ, na caikā\-rollekhino vikalpasyā\-kā\-rā\-ntare pravartanaśaktir anubhavaviṣaya iti kutaḥ śabdapramā\-ṇā\-ntarā\-napekṣeti yuktam | tatra nirvikalpakaṃ spaṣṭapratibhā\-satvā\-d grā\-hakaṃ vyavasthā\-pyate | vikalpas tu spaṣṭaikavyā\-vṛttyullekhā\-d ā\-ropakā\-divyavahā\-rabhā\-janam | yathā\- ca bā\-hye sati kvacid bhramavyavasthā\- tathā\-ntarnaye 'pi sarvatra | kevalaṃ bahirmukhapravṛtyapekṣayā\- kriyamā\-ṇo nā\-tmani kaścid bhrama ity uktaṃ bhavati | na ca gosvalakṣaṇaprakā\-śā\-vakā\-śaḥ, svā\-kā\-rasyaiva sphuraṇā\-t, svalakṣaṇe ca saṃketā\-yogā\-t | vikalpikaiva na syā\-d iti tu svarū\-pā\-pekṣayā\- siddhasā\-dhanam | bā\-hyā\-pekṣayā\- tv adhyavasā\-yavad vikalpikaiva sā\- buddhis tathā\- | tasmā\-d aśeṣagovyaktisā\-dhā\-raṇena gotvena gobuddhir alī\-kena sā\-bhilā\-pena viplavā\-t prakhyā\-tī\-ti tathā\- prakhyā\-nam asyā\- gavā\-vasā\-ya ity eṣṭavyam ity api neṣṭavyam eva, caraṇam ardanā\-dinā\- pratyavasthā\-ne 'pi yuktiśā\-stravahirbhū\-tatvā\-d etadabhā\-ve 'pi kathitadoṣapradhvaṃsā\-t | na hi vikalpabuddhā\-v alī\-kā\-kā\-rasphuraṇam eva bā\-hyasyā\-dhyavasā\-ya iti kā\-cid arthasaṅgatiḥ, arthasyeti sambandhā\-nupapatteḥ \edtext{bodhe ca bhramābhāvāt}{\Afootnote{\label{RNA-tc-7} \cite{} \cite{}| buddher atra kramā\-bhā\-vā\-t \cite{}}} pratyakṣataiva, katham adhyavasā\-yaḥ | apratibhā\-samā\-nasyā\-pi svalakṣaṇasya bhrā\-ntyā\-vasā\-yā\-d iti tu na budhyā\-mahe | avasā\-yena hi tadvittisparśe pratibhā\-saḥ ko 'paraḥ | tadvittā\-v apy aspaṣṭatvā\-d adhyavasā\-ya ity apy ayuktam, tadrū\-pavittā\-v aspaṣṭatvasyaivā\-bhā\-vā\-t |
	\pend
      
	    
	    \stanza[\smallbreak]
jā\-to nā\-mā\-śrayo 'nyonyaś cetasā\-ṃ tasya vastunaḥ |&ekasyaiva kuto rū\-paṃ bhinnā\-kā\-rā\-vabhā\-si yat ||\&[\smallbreak]


	

	  \pstart ity ā\-cā\-ryaḥ smaryatā\-m | na ca tadā\-sau bhrā\-ntir bhavitum arhati, vastusvarū\-pasyaiva nirbhā\-sā\-t || \edlabel{thakur75-141.3}\label{thakur75-141.3} alī\-kavṛtter iti cet | saivā\-stu | bā\-hyasyā\-sphurato 'dhyavasā\-yaḥ katham | saiva sa iti cet | alī\-kam idam iti viduṣo bā\-hyā\-dhyavasā\-yavyasthā\-bhā\-vā\-t, bā\-hyā\-sphuraṇā\-t tadapratibaddhatvā\-c ca | pratibandhe 'pi tasyeti syā\-t, na punas tadadhyavasā\-yaḥ, tadasphuraṇasphuraṇayor api tadayogā\-d ity ala[ {\corr mi}]tinirbandhena | tad evam apratibhā\-sino 'pi vipakṣā\-d adhyavasā\-yamā\-trasiddhā\-d eva vyā\-vṛtto doṣatrayanirmuktaḥ prakā\-śamā\-natā\-tmako hetur yā\-vat prakā\-śā\-vadhijñā\-nā\-tmakacitrā\-kā\-racakrasyaikatvaṃ sā\-dhayaty eva || yad ā\-hur guravaḥ
	\pend
      
	    
	    \stanza[\smallbreak]
bhā\-sate yat tad ekaṃ tad yathā\- citre sitā\-kṛtiḥ |&bhā\-sate cā\-khilaṃ citraṃ pī\-taśī\-tasukhā\-dikam || \footnote{\begin{english}(JNA 569,13f.)\end{english}}&nā\-trā\-siddhiḥ prakā\-śasya citre dharmiṇi darśanā\-t |&na ca sā\-dhyaviyuktatvaṃ dṛṣṭā\-ntasyā\-pi dṛśyate ||&ekaikā\-ṇunimagnatvā\-t saṃvittir na parasparam |&na caikā\-ṇuprakā\-śo 'sti sthū\-lam eva sphuraty ataḥ ||&bā\-hyā\-ṇū\-nā\-ṃ pratī\-bhā\-so buddhir ekā\- sthavī\-yasī\- |&jñā\-nā\-ṇū\-nā\-ṃ ka ekas tu pratibhā\-so bhaviṣyati || \footnote{\begin{english}(JNA 569,19-22)\end{english}}&tasmā\-t sthū\-latayā\- vyā\-pto nirbhā\-sas tannivṛttitaḥ |&nivartamā\-no 'nekasmā\-d ekatve viniyamyate ||&yathā\- sajā\-tī\-yamatā\-d bhā\-gā\-d bhedanirā\-kriyā\- |&anā\-bhā\-saprasaṅgena vijā\-tī\-yamatā\-t tathā\- ||&tan nā\-stu sā\-dhyo dṛṣṭā\-nto na ca śaṅkā\-viparyaye |&ato nirdoṣato hetoś citrā\-dvaitavyavasthitiḥ || \footnote{\begin{english}(JNA 570,3-8)\end{english}}\&[\smallbreak]


	

	  \pstart saṅgrahaślokaś ca 
	\pend
      
	    
	    \stanza[\smallbreak]
ekatvena yathā\-ptimā\-n abhimato bhā\-sas tathā\- vyā\-pyate sthaulyenā\-py aṇuśo na hi kvacid idaṃ svapne 'pi nirbhā\-sanam |&tena pratyaṇubhedanety uparataṃ tadvyā\-pakasyā\-tyayā\-d ekatvena parī\-tam ā\-kṛticayaś cā\-yaṃ vinirbhā\-sate\&[\smallbreak]


	

	  \pstart || iti ||
	\pend
      

	  \pstart nanu cā\-tra dṛṣṭā\-ntadā\-rṣṭā\-ntikayor ubhayatrā\-py ekatvaṃ pratyakṣato 'numā\-nā\-c ca viruddhadharmā\-dhyā\-salakṣaṇā\-t pratihatam, tat katham anumā\-nā\-d ekatvasiddhir iti cet | ucyate | yad etat pratyakṣaṃ bhedasā\-dhakam upanī\-yate, tat kiṃ nī\-lā\-dī\-nā\-m anā\-tmabhū\-tam ā\-tambhū\-taṃ vā\- | prathamapakṣe, ā\-stā\-ṃ tā\-vad eṣā\-m ato bhedasiddhiḥ, \leavevmode\textsuperscript{\rmlatinfont\tiny [pb in]}\label{RNAms_76a} sattā\-mā\-tram api na sidhyet | sa hi nī\-lā\-diko 'rtho jaḍo vijñā\-nā\-ntarā\-tmā\-lī\-kasvabhā\-vo vā\- svī\-kartavyaḥ | triṣv api pakṣeṣu prakā\-śyaprakā\-śakabhā\-vā\-bhā\-vaḥ | tathā\- hi jñā\-nasya prakā\-śakatvaṃ nā\-ma kiṃ vidyamā\-natvaṃ vyā\-pā\-rā\-veśo vā\- | prathamapakṣe sarvasarvadarśitvaprasaṅgaḥ, sarvapuruṣajñā\-navidyamā\-natā\-yā\-ḥ sarvaṃ pratyaviśiṣṭatvā\-t | tathā\- nī\-lā\-dibhir api jñā\-nasya grahaṇaprasaṅgaḥ, teṣā\-m api vidyamā\-natvalakṣaṇagrā\-hakatvasambhavā\-t ||
	\pend
      

	  \pstart atha jñā\-natve sati vidyamā\-natvam iti saviśeṣaṇaṃ lakṣaṇam ucyate | tat kiṃ nī\-lā\-dī\-nā\-m ajñā\-natve kośapā\-nam ā\-yuṣmatā\- kartavyam, yena sattā\-mā\-treṇa samasamayaṃ sphurator vijñā\-nanī\-lā\-dyoḥ pratijñā\-mā\-trā\-d ekasya jaḍatvā\-lī\-katvabā\-dhyatvā\-prakā\-śatvā\-di vyavasthā\-pyate |
	\pend
      

	  \pstart atha dvitī\-yas tadā\- sa kiṃ vyā\-pā\-raḥ pratyakṣasyā\-tmā\- jñā\-nā\-ntaram, arthasyā\-tmā\-rthā\-ntaraṃ vā\- syā\-t | prathamavikalpe svā\-tmani kā\-ritravirodhaḥ |\footnote{kā\-ritra actually in ms, not kā\-ritva} dvitī\-yapakṣe jñā\-nā\-ntaraṃ yady anyaviṣayam arthasya na kiñcit | tadviṣayatvaṃ cā\-dyā\-pi na siddham, tatpratyā\-satter eva cintyamā\-natvā\-t ||
	\pend
      

	  \pstart tṛtī\-ye punaḥ saṅkalpe nī\-lā\-dikaṃ kṛtam eva syā\-t, na prakā\-śitam, tailavartyā\-dibhir iva pradī\-paḥ | prakā\-śas tu svayam eva | tathā\- ca jñā\-nā\-ntaratvā\-t santā\-nā\-ntaravad apratibhā\-saprasaṅgaḥ |
	\pend
      

	  \pstart caturthe tu vikalpe arthā\-ntare kṛte nī\-lā\-dikaṃ tadavastham eva | na cā\-nā\-tmaprakā\-śanasā\-marthyaṃ jñā\-nasya svī\-kartum ucitam, vyā\-pā\-ravat prakā\-śanasyā\-py evaṃ nirā\-kartavyatvā\-t | na cā\-gnidhū\-mayoḥ kā\-ryakā\-raṇabhā\-va iva jñā\-najñeyayor api svā\-bhā\-viko grā\-hyagrā\-hakabhā\-vo vaktavyaḥ, pramā\-ṇasiddhakā\-ryakā\-raṇabhā\-vavad grā\-hyagrā\-hakasvarū\-payor adyā\-pi nirvaktum aśakyatvā\-d iti kva nī\-lā\-divā\-rtā\-pi yadbhedasiddhipratyā\-śā\- pratyakṣataḥ sampadyate ||
	\pend
      

	  \pstart athā\-tmabhū\-taṃ tat pratyakṣam iti dvī\-tyaḥ pakṣaḥ, tadā\-tmasvasaṃvedanam eva bhedasā\-dhakam abhyupagataṃ bhavet | tac ca yadi pratyā\-kā\-raṃ bhinnaṃ tadā\- sarveṣā\-ṃ svasvarū\-panimagnatvā\-c citraprakā\-śapraṇā\-śaprasaṅga ity uktam |
	\pend
      

	  \pstart athaitad doṣabhayā\-t sarveṣā\-m ā\-kā\-rā\-ṇā\-m ekatvam eva svabhā\-vabhū\-taṃ svasaṃvedanam iṣyate, tadaitad eva citrā\-dvaitaṃ vijñā\-nam ucyate, yad anekā\-bhimatā\-nā\-ṃ sahopalabdhā\-nā\-ṃ nī\-lasukhā\-dyā\-kā\-rā\-ṇā\-ṃ svabhā\-vabhū\-tā\-khaṇḍasvasaṃvedanapratyakṣaṃ nā\-ma | yad ā\-hur guruvaḥ
	\pend
      
	    
	    \stanza[\smallbreak]
bhramā\-bhramā\-kalpanakalpanā\-ni śā\-tā\-sitā\-dī\-ny akhilā\-kṣajā\-ni |&jñā\-nā\-ny abhinnā\-ni sahopalabdheḥ pū\-rvā\-paratvaṃ tu na vedyam eva ||\footnote{\begin{english}(JNA 458,14-17)\end{english}}\&[\smallbreak]


	

	  \pstart iti |
	\pend
      

	  \pstart tad evaṃ dṛṣṭā\-ntadā\-rṣṭā\-ntikayor ubhayatrā\-pi svasaṃvedanapratyakṣasiddham ekatvam avidyā\-vaśā\-d vipratipattau satyā\-m anumā\-nataḥ sā\-dhyate | ata eva svasaṃvedanapratyakṣā\-d anumā\-nā\-c ca ekatvasiddhau na pratyakṣā\-ntaram | nā\-pi viruddhadharmā\-dhyā\-salakṣaṇam anumā\-naṃ bhedasā\-dhanā\-ya prā\-ptā\-vasaram, bhedagrā\-hakasya bhinnasya pratyakṣasyoktakrameṇā\-prā\-mā\-ṇyā\-t, pakṣasya pratyakṣā\-dibā\-dhitatvā\-t | \edlabel{thakur75-143.6}\label{thakur75-143.6} nanu brū\-yā\-n nā\-ma kiñcit, tathā\-pi pratibhā\-sabhedā\-d bheda eva, na hi dṛṣṭe 'nupapannaṃ nā\-meti cet | hanta pratibhā\-saśabdena kim abhipretam, kim ā\-kā\-racakraṃ sphuraṇaṃ vā\- | tatra yadi prathamaḥ pakṣaḥ, tadā\- bā\-hye 'rthe pratyetavye buddhyā\-kā\-raḥ pramā\-ṇam | tathā\-cā\-kā\-rabhedo vyavahartavya eva | anyathā\- bā\-hyabhedo na sidhyet | yadā\- punar ā\-kā\-racakram eva prameyam svasaṃvedanaṃ ca pramā\-ṇaṃ tadā\- tenaiva nī\-lā\-dī\-nā\-ṃ svabhā\-vabhū\-tenā\-khaṇḍā\-tmanā\- ekī\-kṛtā\-nā\-ṃ katham apramā\-dī\- bhedam ā\-cakṣī\-ta | \edlabel{thakur75-143.12}\label{thakur75-143.12} dvitī\-yapakṣe tu sphuraṇaṃ svabhā\-vabhū\-tā\-khaṇḍasvasaṃvedanam evoktam iti | tathā\-pi kathaṃ bhedas tasmā\-d yathordhvam indriyapratyakṣataḥ kṣaṇabhede pratī\-te 'py avidyā\-vaśā\-d ekatvā\-dhyavasā\-yaḥ tathā\- tiryaksvasaṃvedanapratyakṣeṇā\-kā\-rā\-bhede 'dhigate 'py avidyā\-vaśā\-d eva bhedā\-vasā\-yaḥ || \edlabel{thakur75-143.15}\label{thakur75-143.15} yady evaṃ viruddhadharmā\-dhyā\-sato vijñā\-nā\-kā\-racakravad vyā\-pto 'pi na bhidyeteti cet | na, bā\-hye dharmiṇy anekatvasya sā\-dhyasya pratyakṣā\-dyabā\-dhitatvā\-t | buddhyā\-kā\-rakadambake tū\-ktakrameṇa svasaṃvedanā\-disiddhaikatve 'nekatvasya pratyā\-khyā\-nā\-d bā\-dhakā\-vatā\-ra eva nā\-sti | tasmā\-d vijñā\-natve satī\-ti hetuviśeṣaṇaṃ kartavyam yena bā\-hyasyaiva bhedaḥ sidhyati || \edlabel{thakur75-143.20}\label{thakur75-143.20} nanu yadi vijñā\-nā\-tmakaṃ vicitrā\-kā\-racakram ekaṃ tadā\- nī\-lā\-kā\-ra eva pī\-tā\-dyā\-kā\-ravṛndaṃ praviśet | tathā\- prakā\-śā\-kā\-racakrayor abhedo vyaktisā\-mā\-nyavat prakā\-śa eva, ā\-kā\-racakram eva vā\- syā\-d iti cet | asad etat | tathā\- hi dvayor apy anayoḥ [ {\corr prasaṅgayor viparyayo}] bhedaḥ, sa ca bā\-hyā\-rthavā\-da eva yujyate, tatra bhedagrā\-hakasyendriyapratyakṣasyeṣṭatvā\-t | vijñā\-navā\-de tv anā\-tmaprakā\-śā\-bhā\-vā\-t svasaṃvedanam evaikaṃ pramā\-ṇam | tato 'pi viparyayasya [ {\corr bhedasyā\-siddheḥ}] prasaṅgo 'py asaṅgataḥ ity advaitam eva | \edlabel{thakur75-143.26}\label{thakur75-143.26} kiṃ ca evaṃ sthū\-lanī\-lā\-dyā\-kā\-ro 'pi paramā\-ṇumā\-tre praviśed ity apratibhā\-saṃ jagad ā\-padyeta | asti ca pratibhā\-saḥ | tasmā\-d yahtā\-vasthitā\-nā\-m evā\-kā\-rā\-ṇā\-m akhaṇḍasvasaṃvedanā\-tmataivaikatvam, na bhedo na saṃkocaḥ svī\-kartavyo 'pratibhā\-saprasaṅgā\-t | tathā\- kṛtakatvasyā\-nityatvavastutvā\-dibhir abhede kṛtakatvam evā\-nityatvam eva vā\- syā\-d ity api prasaṅgo vaktavya ā\-padyeta, sā\-mā\-nyavyaktyor iva tayor vastuto 'bhedo 'khaṇḍā\-tmatvā\-t || \edlabel{thakur75-143.31}\label{thakur75-143.31} vyā\-vṛttibheda eva param iti cet | yady evaṃ prakā\-śanī\-lā\-dyor apy ayam eva kramo jā\-gartī\-ty ekā\-vaśeṣaprasaṅgo bā\-lapralā\-paḥ | tad evaṃ
	\pend
      

	  \pstart bā\-hyaṃ na naśyati bhidā\-ṇutayā\-pi sattvā\-d arthakriyā\-virahasaṃkaratā\-tmabhede | buddhis tu naśyati bhidaiva vidaiva sattvā\-c citrā\-py ato na bhidam eti kim atra kurmaḥ || \footnote{\begin{english}(JNA 573,21-24)\end{english}} nanu deśavitā\-nā\-ptir nā\-tmā\-ntaraviyoginaḥ | deśavitā\-nahā\-nau na bhā\-sa ity api śakyate ||
	\pend
      

	  \pstart iti cet |
	\pend
      

	  \pstart na svā\-tmā\-ntaram anyā\-tmā\- sa bā\-hyasyaiva yujyate | buddheḥ svavittiniṣṭhā\-yā\- yaḥ paras tasya kā\- gatiḥ || \footnote{\begin{english}(JNA 572,3f.)\end{english}}
	\pend
      

	  \pstart hanta tathā\-pi
	\pend
      

	  \pstart nī\-lā\-divat tad ekaṃ ca katham etat sametu cet | nī\-lam aṃśā\-ntaraṃ caikaṃ kathaṃ tadbhā\-ti saṅgatam || neṣṭaṃ tad api cet tarhi kvā\-ṇvantarbhidi bhā\-sanam | na parī\-kṣā\-kṣamaṃ cā\-ṇuḥ kutas tasya tadā\- bhidā\- || mā\- bhū\-d avastubhā\-vā\-c cet so 'py ekatvahatau bhavet | nirbhā\-sā\-d ekatā\-siddhau svavitter vastutā\- sthitā\- || \footnote{\begin{english}(JNA 571,19-24)\end{english}} na pratī\-tyasamutpā\-do 'nutpā\-do vā\-sya bā\-dhakaḥ | ekā\-nekaviyoge 'pi sphū\-rtimā\-treṇa sattvataḥ || kiṃ ca pū\-rvā\-parajñā\-nam advaite yan na vidyate | pratī\-tyotpannatā\- tasmā\-d asiddher apy asā\-dhanam || \footnote{\begin{english}(JNA 577,22)\end{english}} anutpā\-do 'py anekā\-nto 'kā\-ryakā\-raṇarū\-pakam | hā\-ne 'pi hetuphalayoḥ sphuradrū\-paṃ kva gacchatu || \footnote{\begin{english}(JNA 578,2)\end{english}} ekā\-nekatayā\- vastuvyā\-ptiḥ siddhā\- yadi kvacit | sarvaśū\-nyatvasamaye hetur iṣṭavighā\-takṛt || atha lokaprasiddhau ca na sarvalokakalpitam | vastuvyavasthā\- śaraṇaṃ kiṃ tu mā\-nena saṅgatam || na cā\-dhyakṣā\-numā\-nā\-bhyā\-m anaṅgaṃ kvacid ī\-kṣitam | yasya rā\-śir anekaṃ syā\-n nā\-pi vastu ca kiñcana || \footnote{\begin{english}(JNA 574,8-11)\end{english}} yasya caikataratvā\-bhyā\-ṃ sattvavyā\-ptiḥ sa hanyatā\-m | abhrā\-ntavittimā\-treṇa sattā\-vā\-dī\- tu jitvaraḥ || \footnote{\begin{english}(JNA 574, 16f.)\end{english}}
	\pend
      

	  \pstart ||samā\-ptaś citrā\-dvaitaprakā\-śavā\-do 'yam ||
	\pend
      
	    
	    \stanza[\smallbreak]
grā\-hyaṃ na tasya grahaṇaṃ na tena jñā\-nā\-ntaragrā\-hyatayā\-pi śū\-nyaḥ |&tathā\-pi ca jñā\-namayaḥ prakā\-śaḥ pratyakṣapakṣas tu tavā\-virā\-sī\-t ||\&[\smallbreak]


	
	  
	% new div opening: depth here is 1
	
\section[{Santā\-nā\-ntaradū\-ṣaṇam}]{Santā\-nā\-ntaradū\-ṣaṇam}\edlabel{Santānāntaradūṣaṇam}\label{Santānāntaradūṣaṇam}

	  \pstart atheha prakā\-śasahopalambhā\-disā\-dhanabalena jaḍapadā\-rtharā\-śā\-vapā\-ste nī\-lapī\-tā\-dyaśeṣapadā\-rthajā\-te ca svacittapratibhā\-sā\-tmani svapnamā\-yā\-divad advayarū\-pe siddhe santā\-nā\-ntarasadasattā\-nirū\-paṇā\-rtham idam ā\-rabhyate | \edlabel{thakur75-145.6}\label{thakur75-145.6} evaṃ hi kecid ā\-huḥ | asty eva santā\-nā\-ntaram anumā\-napratī\-tam | tathā\- hī\-cchā\-cittasamanantaravyā\-hā\-ravyavahā\-rā\-bhā\-sasya darśanā\-t tadabhā\-ve cā\-darśanā\-d upalambhā\-nupalambhasā\-dhanam anvayavyatirekaśarī\-ram icchā\-cittena saha vyā\-hā\-rā\-dyā\-bhā\-sasya kā\-ryakā\-raṇabhā\-vam ā\-tmasantā\-ne 'vadhā\-ryecchā\-cittasyā\-pratisaṃvedanasamaye 'pi vicchinnavyā\-hā\-rā\-dyā\-bhā\-sadarśanā\-t tatkā\-raṇabhū\-tam icchā\-cittam anumī\-yamā\-naṃ santā\-nā\-ntaram eva vyavatiṣṭhata iti | \edlabel{thakur75-145.12}\label{thakur75-145.12} atredam ā\-locyate | tadicchā\-cittaṃ vyā\-hā\-rā\-dyā\-bhā\-sasya kā\-raṇatayā\- vyavasthā\-pyamā\-nam anumā\-tur darśanayogyam atha dṛśyā\-dṛśyaviśeṣaṇā\-napekṣam icchā\-mā\-tram | yadi tā\-vad ā\-dyo vikalpas tadā\-numā\-tur darśanayogyatvā\-d icchā\-cittasyā\-numā\-nakā\-le 'nupalabdhir abhā\-vam eva gamayatī\-ty anupambhā\-khyapratyakṣabā\-dhitatvā\-t kvā\-numā\-nā\-vakā\-śas tasya | yadi punar icchā\-cittam anumā\-nakā\-le 'py anubhū\-yeta, tadā\- kim asyā\-numā\-nena | athaivam agnidhū\-mayos tadutpattisiddhyanantaraṃ naganikuñje dhū\-mam upalabhamā\-no nā\-gnim apy anuminuyā\-t, tatrā\-py agner anupalabdhibā\-dhitatvā\-t, upalambhe cā\-numā\-navaiphalyā\-t | naivam, anumā\-nasamaye deśaviprakarṣavato vahner darśanā\-yogyatvena dṛśyā\-nupalabdhivirahā\-t, adṛśyā\-nupalambhasya cā\-bhā\-vasā\-dhanatvavirodhā\-t | icchā\-cittasya tu nā\-sti deśaviprakarṣaḥ | icchā\-cittaṃ hi svasambaddham evā\-numā\-tur darśanayogyam, tasya ca deśā\-diviprakarṣa ity alaukikam etat | \edlabel{thakur75-145.23}\label{thakur75-145.23} atha dvitī\-yo vikalpaḥ | tathā\- hī\-cchā\-cittamā\-traṃ svaparasantā\-nasā\-dhā\-raṇadṛśyā\-dṛśyaviśeṣaṇā\-napekṣaṃ vyā\-hā\-rā\-dyā\-bhā\-saṃ prati kā\-raṇatayā\-vadhā\-ryate | tadavadhā\-raṇaṃ kena pramā\-ṇena | vyā\-hā\-rā\-dyā\-bhā\-sasya hī\-cchā\-mā\-trā\-bhā\-ve 'bhā\-vaṃ pratī\-tya tadutpattisiddhigaveṣaṇā\- | na cecchā\-mā\-trasya svaparasantā\-nasā\-dhā\-raṇasya svasaṃvedanenā\-nyena vā\-bhā\-vaḥ śakyā\-vagamaḥ | yathā\- hi vahnimā\-trasya deśakā\-lavyavahitasyā\-pi dhū\-motpā\-dadeśakā\-layor yadi syā\-d upalabhyetaiva mayeti sambhā\-vitasyā\-numā\-tṛpuruṣendriyapratyakṣeṇa dhū\-motpā\-dā\-t prā\-gabhā\-vo 'vadhā\-ryamā\-ṇas tadutpattisiddhim adhyā\-sayatī\-ti vyavahitadeśakā\-lasyā\-pi vahner dhū\-mamā\-traṃ prati kā\-raṇatvā\-vadhā\-raṇam, svabhā\-vaviprakṛṣṭasya tu jaṭharabhavā\-disā\-dhā\-raṇasya sarvathā\-numā\-tṛpuruṣā\-śakyā\-bhā\-vapratī\-tikasya vyā\-ptibahirbhā\-va eva | tathā\-trā\-pī\-cchā\-cittaṃ parasantā\-nasā\-dhā\-raṇam api yā\-vad yadī\-ha syā\-d upalabhyetaiva mayeti yadi sambhā\-vayituṃ śakyeta tadā\- tadvyatirekasiddhidvā\-reṇa kā\-raṇatayā\-vadhā\-ryate | kevalaṃ svabhā\-vaviprakṛṣṭe cittamā\-tre 'stamiteyaṃ katheti || \edlabel{thakur75-146.7}\label{thakur75-146.7} na ca pracittaṃ kā\-laviprakṛṣṭaṃ varamā\-natvā\-d asya, atī\-tā\-nā\-gatayor eva kā\-laviprakṛṣṭatvena vyavahā\-rā\-t | \edlabel{thakur75-146.9}\label{thakur75-146.9} nā\-pi deśaviprakṛṣṭam, yasminn eva hi śuklaśaṅkhā\-dideśe svacittaṃ śuklā\-kā\-rapratibhā\-si svasaṃvedanena vedyate taddeśavarty eva pī\-tā\-kā\-rapratibhā\-si parasantā\-nabhā\-vi cittaṃ na vedyate | tat katham eṣa deśaviprakarṣaḥ || \edlabel{thakur75-146.12}\label{thakur75-146.12} athecchā\-cittamā\-traṃ svasaṃvedanamā\-trā\-pekṣayā\- na svabhā\-vaviprakṛṣṭam | na hy agnir apy eko yenaivendriyavijñā\-nena dṛśyate tenaivā\-nyo 'pi dṛśyam | tatra yathā\- cakṣurvijñā\-namā\-trā\-pekṣayā\- agnimā\-traṃ dṛśyam iti vyavasthā\-pyate tathā\-trā\-pi svasaṃvedanamā\-trā\-pekṣayā\- icchā\-cittamā\-traṃ svaparasantā\-nasā\-dhā\-raṇam api dṛśyam eveti | \edlabel{thakur75-146.16}\label{thakur75-146.16} atrocyate | kim atra mā\-traśabdenā\-numā\-tṛpuruṣasambandhā\-sambandhā\-bhyā\-m aviśeṣitaṃ yasya kasyacit puruṣasyendriyajñā\-naṃ vastuviṣayī\-kurvā\-ṇam asya dṛśyatā\-sambhave 'pi nā\-nimittam abhimatam | yady evaṃ piśā\-cā\-dir api dṛśyaḥ syā\-t | so 'pi hi kasyacit puṃso yogyā\-deḥ svajā\-tī\-yasya vā\- piśā\-cā\-ntarasya bhavaty evendriyajñā\-nagocara iti na kaścit svabhā\-vaviprakṛṣṭaḥ syā\-t | tasmā\-d anumā\-tṛpuruṣasambandhitvam anapā\-sya vijñā\-nasya svalakṣaṇā\-dibhedanirā\-sapara eva mā\-traśabdo yuktaḥ | etad evā\-śaṅkya Dharmottareṇā\-bhihitam -
	\pend
      

	  \pstart ekapratipattrapekṣaṃ cedaṃ pratyakṣalakṣaṇam | \footnote{\begin{english}(NBṬ 104,5f.)\end{english}}
	\pend
      

	  \pstart ityā\-di | tenaivaṃ dṛśyatā\-sambhā\-vanā\- yadī\-ha deśe kā\-le vā\- syā\-d ghaṭā\-dir niyamenopalabhyeta, madī\-yasya cakṣurvijñā\-namā\-trasya viṣayī\-bhaved iti | paricitte tu na śakyam evam | yadī\-ha paricittaṃ syā\-t niyamena madī\-yasya svasaṃvedanamā\-trasya viṣayi syā\-d iti || \edlabel{thakur75-146.26}\label{thakur75-146.26} yadi cecchā\-cittamā\-traṃ tadutpattigrhaṇasamaye dṛśyatayā\- sambhā\-vayitavyam, tadā\-numā\-nakā\-le 'pi dṛśyatayā\- sambhā\-vya tadanupalambhenā\-bhā\-vasā\-dhane katham anumā\-naṃ pravartayitum idam ā\-rabdham, pratyakṣeṇaiva pakṣabā\-dhā\-t | na ca kā\-labhedena svabhā\-vaviprakarṣetarā\-v iti yatkiñcid etat | tasmā\-d icchā\-cittamā\-trasya svaparasantā\-nasā\-dhā\-raṇasya dṛśyatayā\- sambhā\-vayitum aśakyatvā\-t vyahā\-rā\-dyutpā\-dā\-t prā\-g anupalambhe 'py abhā\-vasiddhau na tadabhā\-vaprayukto vyā\-hā\-rā\-dyabhā\-vaḥ pratī\-yata iti kathaṃ kā\-raṇatvasiddhir yataḥ kā\-ryahetudvā\-reṇā\-numī\-yeta | icchā\-cittaviśeṣas tu svasantā\-nabhā\-vī\- na bhavaty evā\-numā\-tur dṛśyaḥ | kiṃ tu tasya dṛśyā\-nupalambhā\-j jijñā\-sitaviśeṣe dharmiṇi bā\-dhitasya katham anumā\-nam ity uktam eva || \edlabel{thakur75-147.1}\label{thakur75-147.1} tad evam icchā\-cittaviśeṣe svasantā\-nabhā\-vini sā\-dhye pakṣasya pratyakṣabā\-dhaḥ, icchā\-cittamā\-tre 'pi svaparasantā\-nasā\-dhā\-raṇe sā\-dhye yady anupalambhamā\-treṇa dṛśyaviśeṣaṇā\-napekṣeṇa pratibandhasiddhisamaye tasyā\-bhā\-vaḥ pratī\-yate, tadā\- pakṣī\-kṛte dharmiṇi tatheti sa eva doṣaḥ | atha na pratī\-yate tadā\- sandigdhavyatireko hetvā\-bhā\-so vyā\-hā\-rā\-dir iti sthitam | \edlabel{thakur75-147.6}\label{thakur75-147.6} evaṃ tarhi santā\-nā\-ntarasā\-dhakasyā\-bhā\-vā\-d bā\-dhakasyā\-pi kasyacid adarśanā\-d bhavatu tatra sandeha eveti kecit | tair idaṃ bā\-dhakam abhidhī\-yamā\-nam avadhī\-yatā\-m | yadi hi santā\-nā\-ntaraṃ sambhavet tadā\- tato bhedena svasantā\-nasyā\-vaśyaṃ bhavitavyam | anyathā\- svasantā\-nā\-d api prakā\-śamā\-nā\-t tasya parasantā\-nā\-bhimatasya bhedo na syā\-t | na cā\-bhedas tayor iti svasantā\-nā\-d bhedā\-bhedā\-bhyā\-m abā\-dhyasya parasantā\-nasya sā\-mā\-nyaśaśaviṣā\-ṇā\-divad abhā\-va evā\-yā\-ta iti kathaṃ sandehaḥ | tasmā\-t parasantā\-nā\-pekṣayā\- svasantā\-nasya bhedo 'py avaśyambhā\-vyah | sa ca bhedaḥ santā\-nasya svabhā\-vaḥ svasantā\-ne pratibhā\-samā\-ne niyamena pratibhā\-seta | katham aparathā\- pratibhā\-nā\-pratibhā\-nalakṣaṇaviruddhadharmā\-dhyā\-se 'pi svasantā\-nasya parasantā\-nā\-d bhedaḥ svabhā\-vatā\-m ā\-sā\-dayet || \edlabel{thakur75-147.15}\label{thakur75-147.15} na cā\-sau bhedaḥ pratibhā\-sate | bhedapratibhā\-se hi upagamyamā\-ne tadavadhibhū\-tasyā\-pi parasantā\-nasya pratibhā\-so durapahnavaḥ syā\-t |
	\pend
      

	  \pstart asmā\-d bhinnam itī\-daṃ cet svarū\-paṃ svasya cetasaḥ | sā\-vadher asya bhā\-saḥ syā\-n na vā\- grā\-hyaṃ tadā\-tmanā\- || \footnote{\begin{english}(JNA 570,15f.)\end{english}}
	\pend
      

	  \pstart bhede 'nyaleśam api naiti kuto bhinnaḥ | evam ā\-dikam aśeṣam iha pravacanapradī\-paśrī\-sā\-kā\-rasaṅgrahā\-divacanam anusmryatā\-m | \edlabel{thakur75-147.21}\label{thakur75-147.21} yathā\- hi svasantā\-namā\-tre parisphurati śaśaviṣā\-ṇā\-d asphurato na bhedaḥ pratibhā\-ti tathā\- parasantā\-nā\-d api sphuraṇavirahiṇo na bhā\-ty eva bhedaḥ | na hi parasantā\-nā\-pekṣayā\- kaścid viśeṣaleśaḥ svasantā\-nasya parisphurati yo nā\-sti śaśaviṣā\-ṇā\-pekṣayā\- | na ca śaśaviṣā\-ṇaparasantā\-nā\-v apekṣya samā\-ne svasantā\-napratibhā\-se śaśaviṣā\-ṇā\-pekṣayā\- na bhedo nā\-py abhedaḥ pratibhā\-ti | parasantā\-nā\-pekṣayā\- tu bheda eva bhā\-tī\-ty evam avasthā\-payituṃ śakyam | \edlabel{thakur75-147.27}\label{thakur75-147.27} bhedā\-bhedayor abhā\-vaparihā\-reṇa hi yathā\- bhedo vyavasthitaḥ tadvad bhedapratibhā\-so 'pi bhedā\-bhedā\-bhā\-vapratibhā\-savilakṣaṇa evocito bhavitum, na ca tathā\-nubhū\-yate | tathā\-pi bhedaḥ pratibhā\-tī\-ti vacanaracanam etat | bhā\-ṣyakā\-ranyā\-yo 'py atra bhedapratibhā\-sadū\-ṣaṇe vistarato 'vagantavyaḥ || \edlabel{thakur75-148.1}\label{thakur75-148.1} yadi cā\-vadhipratibhā\-savirahe 'pi bhedapratibhā\-nam idaṃ paracittā\-nukampayā\- kṣamitavyaṃ tarhi bahirarthasyā\-pi katham abhā\-vaḥ sidhyati | śakyaṃ hi tatrā\-pi sandeham avatā\-rayitum, na bahirarthaḥ kasyacid ā\-bhā\-sate, parasantā\-nas tu parasya pratibhā\-sata eva, tataś cā\-traiva sandeho na bahirartha iti cet | etad api sakalaṃ sandigdham eva | na hy avaśyaṃ parasantā\-naḥ parasyā\-bhā\-sate, kadā\-cid asau nā\-saty eva na cā\-sā\-v avabhā\-sata ity api vaktuṃ śakteḥ | \edlabel{thakur75-148.7}\label{thakur75-148.7} kiṃ ca mā\- nā\-ma bhā\-siṣṭa bahirarthaḥ kasyacid api tathā\-pi kathaṃ tadabhā\-vasiddhir bhedapratibhā\-sā\-bhyupagamavā\-dina itī\-yanmā\-tram iha vivakṣitam | na cā\-tra kaścid doṣaḥ | tasmā\-d bahirarthena sā\-dhā\-raṇaṃ santā\-nā\-ntaram iti kathaṃ vijñā\-ptivā\-dinā\-m api saṃmataṃ bhaviṣyati | kiṃ ca kā\-ryakā\-raṇabhā\-vo 'pi vijñā\-nadvayasya bhedapratibhā\-savā\-dinā\- bā\-dhitum aśakyaḥ | pū\-rvabhā\-vinī\- hi saṃvittiḥ parasaṃvittyapekṣayā\- bhedaṃ pū\-rvatvaṃ cā\-tmano gṛhṇā\-ty evā\-vadhipratibhā\-savigame 'pi || \edlabel{thakur75-148.13}\label{thakur75-148.13} parabhā\-viny api saṃvittiḥ pū\-rvasaṃvittyapekṣayā\- bhedaṃ paratvaṃ cā\-tmano 'dhigacchaty eva santā\-nā\-ntaravad iti niyatapū\-rvā\-parabhā\-valakṣaṇe kā\-ryakā\-raṇabhā\-ve 'vabhā\-samā\-ne 'vasī\-yamā\-ne ca nī\-lā\-dicitrā\-kā\-ravat katham
	\pend
      

	  \pstart saṃvṛttyā\-stu yathā\- tathā\- \footnote{\begin{english}(PV III 4d)\end{english}}
	\pend
      

	  \pstart iti bhagavato Vā\-rtikakā\-rasya vacanena phalitam atra mate | api ca citrā\-kā\-racakre dharmiṇy advaitasā\-dhanā\-rtham upanyastasya prakā\-śamā\-natvā\-dihetor bhedagrā\-hakapratyakṣā\-pahṛtaviṣayatvam udbhā\-vayataḥ prativā\-dino bhedagrahaṇam anumanyamā\-nena santā\-nā\-ntarasandehaṃ ca vinā\- katham uttaritavyaṃ bhavatā\- | \edlabel{thakur75-148.21}\label{thakur75-148.21} nanv evam api santā\-nā\-ntarā\-bhā\-vaḥ kena pramā\-ṇena siddhaḥ | na tā\-vat pratyakṣeṇa, tasya vidhiviṣayasya pratiṣedhasā\-dhanā\-nadhikā\-rā\-t | nā\-py anumā\-nena, tasya dṛśyā\-bhā\-vasā\-dhananiyatasyā\-tī\-ndriyaparacittā\-bhā\-vasā\-dhane 'navatā\-rā\-d iti cet | atra brū\-maḥ | santā\-nā\-ntarasambhave niyatabhā\-vaḥ tato bhedaḥ svacittasya | abhede svasantā\-nā\-t parasantā\-na eva syā\-t | yathā\- ca yad upalabhyamā\-naṃ yena rū\-peṇa na bhā\-sate na tat tena rū\-peṇa sadvyavahā\-rayogyaṃ yathā\- nī\-laṃ pī\-tarū\-peṇa | nopalabhyate ca svacittam upalabhyamā\-naṃ parasantā\-nā\-d bhinnena rū\-peṇeti bhedasya svacittatā\-dā\-tmyaniṣedhe dṛśyaviśeṣaṇaprayogā\-napekṣā\- svabhā\-vā\-nupalabdhir iyam || \edlabel{thakur75-148.28}\label{thakur75-148.28} nā\-py asiddhiḥ, bhedapratibhā\-se tadavadher api pratibhā\-saprā\-pteḥ | avadhyapratibhā\-se tu bhedapratibhā\-sā\-bhā\-vaḥ śaśaviṣā\-ṇabhedapratibhā\-sā\-bhā\-vavat siddha eva | evam anena pramā\-ṇena santā\-nā\-ntarasya svacittā\-pekṣayā\- bhede pratikṣipte abhede ca svayam evā\-sambhavini bhedā\-bhedā\-bhyā\-m avā\-cyatvaṃ siddham | sā\-mā\-nyā\-divad vastutā\-pahatir iti, kathaṃ bā\-dhakā\-bhā\-vā\-t santā\-nā\-ntare sandeho 'bhidhī\-yate | etac ca śā\-strī\-yaprameyasmā\-raṇamā\-traphalaṃ kiñcil likhitam iti | param iha svayam anusandheyam | \edlabel{thakur75-149.3}\label{thakur75-149.3} api ca santā\-nā\-tare tā\-vad arvā\-gdṛśā\-ṃ sandeho bhavadbhir anumanyate | bhagavatas tu kim avasthā\-pyatā\-m | saṃdehā\-vasthā\-pane kathaṃ sarvajñatā\- | vidyamā\-nam eva kadā\-cit santā\-nā\-ntaraṃ bhagavatā\- nā\-vadhā\-ryate tathā\-py asau sarvajña iti katham etat | anumā\-naṃ ca santā\-nā\-ntaraviṣayaṃ prā\-g eva cintitam | na cā\-numā\-nena pratī\-tā\-v api sarvajñatā\- bhavitam arhati | pratyakṣeṇa paracittapratī\-tau grā\-hyagrā\-hakabhā\-vo 'pi paracittasya bhagavaccittena sahā\-yā\-ta iti bahirarthavā\-da eva mukhā\-ntareṇopagataḥ syā\-t, katham ayaṃ vañcayati vā\-daḥ || \edlabel{thakur75-149.9}\label{thakur75-149.9} asmadī\-yam etena tu paracittaṃ nā\-sty eveti tadavadhā\-raṇakṛto 〔na〕 bhagavataḥ sarvajñatā\-kṣatidoṣaḥ | yā\-vac ca bhedagrahaṇā\-bhimā\-narū\-pā\- saṃvṛsttitā\-vat santā\-nā\-ntare sandehā\-t tadavabodhanā\-rthaṃ vacanā\-dir api pravartata iti svavacanavirodho 'pi na sambhavaty eva | na khalu santā\-nā\-ntaraviṣayaḥ sarvathā\- sandeho nā\-sty evety abhimatam asmā\-kam, api tu paramā\-rthagatir iyam upadarśitā\- | idam hi santā\-nā\-ntarā\-bhā\-vasā\-dhanam advayasā\-dhanena sā\-dhā\-raṇam iti naikaniyataḥ svavacanā\-divirodhas tatparihā\-ro vā\- | citrā\-kā\-rasambhavamā\-treṇā\-pi ca vedā\-ntadhvā\-ntā\-pasā\-ro Bhā\-ṣyakā\-reṇa darśitaḥ | tathā\- ca
	\pend
      

	  \pstart ā\-tmā\- sa tasyā\-nubhavaḥ sa ca nā\-nyasya kasyacit \footnote{\begin{english}(PVA III 326ab)\end{english}}
	\pend
      

	  \pstart ityā\-divā\-rtikavyā\-khyā\-nabhā\-ṣyam | \edlabel{thakur75-149.18}\label{thakur75-149.18} ā\-tmavā\-das tarhi prasakta iti cet | na citrā\-kā\-rasaṃvedanā\-t \footnote{\begin{english}(PVA 352,26)\end{english}} ityā\-di dveṣacikaluṣā\-śeṣā\- eva tuṣā\-kā\-ro 'pi vedā\-ntasiddhā\-nta ity alakṣita tadgranthā\-nutthā\-payantī\- santā\-nā\-ntarā\-pekṣayā\- paṭhitavatī\-ty avasthā\- (?) sarvā\- saṃvṛtisatyā\-ntaḥpā\-tinī\- hy evā\-paitī\-ti sakalam anā\-kulam iti ||
	\pend
      

	  \pstart || santā\-nā\-ntaradū\-ṣaṇaṃ samā\-ptam || \textunderscore \textunderscore \textunderscore \textunderscore \textunderscore \textunderscore \textunderscore \textunderscore \textunderscore \textunderscore \textunderscore \textunderscore \textunderscore \textunderscore \textunderscore \textunderscore \textunderscore \textunderscore \textunderscore \textunderscore \textunderscore \textunderscore \textunderscore \textunderscore \textunderscore \textunderscore \textunderscore \textunderscore \textunderscore \textunderscore \textunderscore \textunderscore \textunderscore \textunderscore \textunderscore \textunderscore \textunderscore \textunderscore \textunderscore \textunderscore \textunderscore \textunderscore \textunderscore \textunderscore \textunderscore \textunderscore \textunderscore \textunderscore \textunderscore \textunderscore \textunderscore \textunderscore \textunderscore \textunderscore \textunderscore \textunderscore \textunderscore \textunderscore \textunderscore 
	\pend
      

	  \pstart The end of Ratnakī\-rtinibandhā\-valiḥ 
	\pend
      
	    
	    \endnumbering% ending numbering from div
	    \endgroup
	    
	  \backmatter 
       \chapter{Bibliographical Hacks}
       \begin{minted}[fontfamily=rmfamily,fontsize=\footnotesize]{xml}
     <listBibl xmlns="http://www.tei-c.org/ns/1.0" xmlns:xi="http://www.w3.org/2001/XInclude"
          xml:lang="en">
   <head>References</head>
   <biblStruct xml:id="Frauwallner37">
      <analytic>
         <author>Erich Frauwallner</author>
         <title level="a">Beiträge zur Apohalehre II: Dharmottara</title>
      </analytic>
      <monogr>
         <title level="j">Wiener Zeitschrift für die Kunde des Morgenlandes</title>
         <imprint>
            <date>1937</date>
            <biblScope unit="pp">233--287</biblScope>
            <biblScope unit="vol">44</biblScope>
         </imprint>
      </monogr>
   </biblStruct>
   <biblStruct xml:id="buehnemann80">
      <monogr>
         <author>Gudrun Bühnemann</author>
         <title>Der Allwissende Buddha: Ein Beweis und seine Probleme: Ratnakīrti’s Sarvajñasiddhi</title>
         <imprint>
            <date>1980</date>
            <publisher>Arbeitskreis für Tibetische und Buddhistische Studien</publisher>
            <pubPlace>Wien</pubPlace>
         </imprint>
      </monogr>
      <series>
         <title level="s">Tibetan Sanskrit Works Series</title>
         <biblScope unit="vol">3</biblScope>
      </series>
   </biblStruct>
   <biblStruct xml:id="moriyama11_transl_capv_1">
      <analytic>
         <author>Shinya Moriyama</author>
         <title>An Annotated Japanese Translation of Ratnakīrti's
      Citrādvaitaprakāśavāda (1)</title>
      </analytic>
      <monogr>
         <title level="j">South Asian Classical Studies</title>
         <imprint>
            <date>2011</date>
            <biblScope unit="pp">51–93</biblScope>
            <biblScope unit="vol">6</biblScope>
         </imprint>
      </monogr>
   </biblStruct>
   <biblStruct xml:id="mccrea_patil06">
      <analytic>
         <author>Lawrence J. McCrea</author>
         <author>Parimal G. Patil</author>
         <title>Traditionalism and Innovation: Philosophy, Exegesis, and
      Intellectual History in Jñānaśrīmitra’s Apohaprakaraṇa</title>
      </analytic>
      <monogr>
         <title level="j">Journal of Indian Philosophy</title>
         <imprint>
            <date>2006</date>
            <biblScope unit="pp">303--366</biblScope>
            <biblScope unit="vol">34.4</biblScope>
         </imprint>
      </monogr>
   </biblStruct>
   <bibl xml:id="CAPV">
    Citrādvaitaprakāśavāda, in <ref target="#thakur75">Ratnakīrtinibandhāvaliḥ (Buddhist Nyāya Works of
    Ratnakīrti)</ref>, pp. 129--144.
  </bibl>
   <bibl xml:id="NVTṬ">
    Vācaspatimiśra. “Nyāyavārttikatātparyaṭīkā”. In:
    Nyāyavārttikatātparyaṭīkā of Vācaspatimiśra. Ed. by Anantalal
    Thakur.  New Delhi: Indian Council of Philosophical Research,
    1996.
  </bibl>
   <bibl xml:id="NK1">
    Vācaspatimiśra. “Nyāyakaṇikā”. In: Vidhiviveka of Śrī Maṇḍana
    Miśra With the Commentary Nyāyakaṇikā of Vācaspati
    Miśra. Ed. by Mahaprabhu Lal Goswami. Varanasi: Tara Printing
    Works, 1984.
  </bibl>
   <bibl xml:id="SāSiŚā">
    Jñānaśrīmitra. “Sākārasiddhiśāstram”. In: Jñānaśrīmitranibandhāvali.
    Ed. by Anantalal Thakur. 2nd ed. Tibetan Sanskrit Works Series
    5. Patna: Kashi Prasad Jayaswal Research Institute, 1987, 367–513.
  </bibl>
   <bibl xml:id="śabara_bhāṣya">
      <title>Jaimini: Mimamsasutra, with Sabara's Bhasya, Adhyayas
    1-7</title> Based on six editions (details see below). Input by
    <editor>Andreas Pohlus</editor>
      <ref target="http://gretil.sub.uni-goettingen.de/gretil/1_sanskr/6_sastra/3_phil/mimamsa/msbh1-7u.htm"/>
  </bibl>
   <bibl xml:id="sucarita">
      <title>Mīmāṃsāślokavarttikakāśikā</title>
      <author>Sucaritamiśra</author>
      <note>E-text of Trivandrum Sanskrit Series, 90, 99, 150 <date>1926,
    1929, 1943</date>
      </note>  
  </bibl>
   <msDesc xml:id="capv-np">
      <msIdentifier>
         <settlement>Nepal</settlement>
         <idno>5-137/ vi. mīm. 4 (reel b21/31); NGMCP id: 33642</idno>
         <msName>[Citrādvaitaprakāśavāda (incompl.)]</msName>
      </msIdentifier>
      <msContents>
         <p>Incomplete manuscript of the Citrādvaitaprakāśavāda.</p>
         <p>Identified by (and received from) <persName key="name person hi">Harunaga Isaacson</persName>.</p>
         <p>See also <ref target="http://catalogue.ngmcp.uni-hamburg.de/wiki/B_21-31_Khy%C4%81tiv%C4%81dagrantha">http://catalogue.ngmcp.uni-hamburg.de/wiki/B_21-31_Khyātivādagrantha</ref>.</p>
      </msContents>
   </msDesc>
   <msDesc xml:id="RNAms">
      <msIdentifier>
         <settlement>Beijing</settlement>
         <idno>Pek.-L., Nr. 52--58</idno>
         <msName>Ratnakīrtinibandhāvalī</msName>
      </msIdentifier>
      <msContents>
         <p>Please refer to the introduction to <ref target="#thakur75"/>, and to the description pp. 58 ff. in <bibl>Bandurski,
      Frank. “Übersicht über die Göttinger Sammlungen der von Rāhula
      Sāṅkṛtyāyana in Tibet aufgefundenen buddhistischen
      Sanskrit-Texte (Funde buddhistischer Sanskrit-Handschriften,
      III)”. In: Untersuchungen zur buddhistischen Literatur. Ed. by
      Frank Bandurski et al. Sanskrit-Wörterbuch der buddhistischen
      Texte aus den Turfan Funden Beiheft 5. Göttingen: Vandenhoeck
      &amp; Ruprecht, 1994, 9– 126.</bibl>
         </p>
         <p>The original ms could not be consulted. Instead, copies of
    catalogue entry Xc 14/26 in the ``Sammlung des Seminars für
    Indologie und Buddhismuskunde in Göttingen" (Collection of the
    Seminar for Indology and Buddhist studies in Göttingen) were
    used.</p>
      </msContents>
   </msDesc>
   <bibl xml:id="TBh-GOS">
	       Mokṣākaragupta. “Tarkabhāṣā”. In: Tarkabhāṣā of
	       Mokṣākara Gupta. Ed. by Embar
	       Krishnamacharya. Gaekwad’s Oriental Series 94. Baroda:
	       Oriental Institute, 1942, 1–39
	     </bibl>
   <bibl xml:id="TBh-Mysore">
	       Mokṣākaragupta. “Tarkabhāṣa”. In: Tarkabhāṣa and
	       Vādasthāna of Mokṣākaragupta and Jitāripāda. Ed. by
	       H.R. Rangaswami Iyengar. 2nd ed. Mysore: The Hindusthan
	       Press, 1952, 1–71.
	     </bibl>
   <bibl xml:id="krasser02_zaGkar_Izvar_texts">
  Krasser, H. (2002). Śaṅkaranandanas Īśvarāpākaraṇasaṅkṣepa. 1: Texte. Wien: Verl. der Österr. Akad. der Wiss.
</bibl>
   <bibl xml:id="krasser02_zaGkar_Izvar_studie">
  Krasser, H. (2002). Śaṅkaranandanas Īśvarāpākaraṇasaṅkṣepa. 2: Annotierte Übersetzungen und Studie zur Auseinandersetzung über die Existenz Gottes. Wien: Verl. der Österr. Akad. der Wiss.
</bibl>
   <bibl xml:id="SVR">
  Vādidevasūri. Syādvādaratnākara. In: Śrīmad Vādidevasūriviracitaḥ
  Pramāṇanayatattvālokālaṅkāraḥ Tadvyākhyā ca Syādvādaratnākaraḥ,
  ed. by Motīlāl Lādhājī. Puṇyapattana: Lakṣmaṇ Bhāurāv Kokāṭe, 1926--30
</bibl>
</listBibl>
       \end{minted}
     
\chapter[{Critical Annotations}]{Critical Annotations}                                      % running endDocumentHook
     
	 \chapter{The TEI Header}
	 \begin{landscape}
	 \begin{minted}[fontfamily=rmfamily,fontsize=\footnotesize]{xml}
       <teiHeader xmlns="http://www.tei-c.org/ns/1.0" xmlns:xi="http://www.w3.org/2001/XInclude"
           xml:lang="en">
   <fileDesc>
      <titleStmt>
         <title type="main">Ratnakīrtinibandhāvali</title>
         <title type="sub">A SARIT edition</title>
         <author>Ratnakīrti</author>
         <respStmt>
            <persName key="name person jw">Jeson Woo
	  </persName>
            <resp>Creation of e-text from the Ratnakīrtinibandhāvali's
	  second edition (1975, see <ref target="#thakur75"/>).</resp>
         </respStmt>
         <respStmt>
            <persName key="name person pma">Patrick Mc Allister</persName>
            <resp>Conversion to TEI xml file, various corrections.
	  </resp>
            <resp>Maintenance of file for SARIT.
	  </resp>
         </respStmt>
      </titleStmt>
      <editionStmt>
         <p>The following remarks were at the beginning of the original word file:</p>
         <p>
	Explanatory Remarks

	<list>
               <item>1.  This is a database of Ratnakīrti’s works.  It
	  includes the whole work in the
	  Ratnakīrtinibandhāvaliḥ.</item>
               <item>2.  The list of the works is as follows:
	  <list>
                     <item>1) Sarvajñasiddhiḥ</item>
                     <item>2) Īśvarasādhanadūṣaṇam</item>
                     <item>3) Apohasiddhiḥ</item>
                     <item>4) Kṣaṇabhaṅgasiddhiḥ-Anvayātmikā</item>
                     <item>5) Kṣaṇabhaṅgasiddhiḥ-Vyatirekātmikā</item>
                     <item>6) Pramāṇāntarbhāvaprakaraṇam</item>
                     <item>7) Vyāptinirṇayaḥ</item>
                     <item>8) Sthirasiddhidūṣaṇam</item>
                     <item>9) Citrādvaitaprakāśavādaḥ</item>
                     <item>10) Santānāntaradūṣaṇam</item>
                  </list>
	              </item>
               <item>3.  The texts used for this database are as follows:
	  <p>
	    1), 2), 3), 4), 6), 7), 9) and 10): Ratnakīrtinibandhāvaliḥ, ed. A. Thakur, 
            Patna: Kashi Prasad Jayaswal Research Institute, 2nd ed. 1975.
	  </p>
	                 <p>
	    5): An Eleventh-Century Buddhist Logic of Exists, A. C. Senape Mcdermott, 
            Dordrecht-Holland: D. Reidel Publishing Company, 1967.
	  </p>
	  8): La Refutation Bouddhique de la Permanence des Choses
	  (Sthirasiddhidūṣaṇa) et la Preuve de la Momentanite des
	  Choses (Kṣaṇabhaṅgasiddhi), K. Mimaki, Paris: Instititut de
	  Civilization Indienne, 1976.</item>
               <item>4.  I give the page and the line numbers in two different ways.
	  <list>
                     <item>4.1 The numbers in each individual database but 5)
	    and 8) correspond to the page and the line numbers in
	    Thakur’s second edition.  For instance, [30.10] indicates
	    the page 30 and the line 10 in the edition.  The numbers
	    in 5) and 8) respectively correspond to those which appear
	    in Macdermott’s and Mimaki’s editions.  Therefore, their
	    numbers indicate the page and the line numbers in Thakur’s
	    first edition.</item>
                     <item>4.2 The whole number in the database of
	    Ratnakīrtinibandhāvaliḥ corresponds to the page and the
	    line numbers in Thakur’s second edition.</item>
                  </list>
               </item>
               <item>5.  I have made a critical edition of the
	  Kṣaṇabhaṅgasiddhi-Anvayātmikā on the basis of three previous
	  editions and the manuscript from the Nepal National
	  Library. I have also improved its some parts with the Pathna
	  manuscript, Jñānaśrīmitra’s Kṣaṇabhaṅgādhyāya and other
	  Naiyāika’s works, such as the Nyāyabhūṣaṇa and the
	  Tātparyaṭīkā.  However, I have made the database of other
	  works without a thorough investigation of them.  I have
	  intended to use it as a reference for reading the
	  Anvayātmikā.  Thus, I must admit that there are lots of
	  errors and misspellings in this version.  I would appreciate
	  it if the user would point out any mistake in this database
	  so that I can improve it.</item>
            </list>

Woo, Jeson 
Penn and Hiroshima U.
	</p>
         <p>
	  bearbeitet für die WORD-Benutzer von ONO, Dezember 1997. 
	  Ratnakīrtinibandhāvaliḥ,
	  ed. A. Thakur, Patna:
	  Kashi Prasad Jayaswal Research Institute, 2nd ed. 1975
	  _______________________________________________________
	</p>
      </editionStmt>
      <publicationStmt>
         <publisher>
            <ref target="http://sarit.indology.info">SARIT (http://sarit.indology.info)</ref>
         </publisher>
         <availability>
            <p>
	    This work is licensed under a Creative Commons Attribution-ShareAlike 3.0 Unported License.
	  </p>
         </availability>
         <date>2011</date>
         <idno type="gitBlob">$Id: 042f10eb98c18188e3482d374b7227747bbdbe7f $</idno>
      </publicationStmt>
      <sourceDesc>
         <bibl xml:id="thakur75">
	           <title>Ratnakīrtinibandhāvaliḥ (Buddhist Nyāya Works of Ratnakīrti)</title>
	           <author>Ratnakīrti</author>
	           <editor xml:id="Thakur">Anantalal Thakur</editor>
	           <publisher>Kashi Prasad Jayaswal Research Institute</publisher>
	           <address>
               <name>Patna</name>
            </address>
	           <date>1975</date>
	           <edition>Second Revised Edition</edition>
	           <series>
	              <title level="s">Tibetan Sanskrit Works Series</title>
	              <biblScope unit="vol">3</biblScope>
	           </series>
	        </bibl>
         <bibl xml:id="kāśikā">
	           <title>The Mīmā[ṃ]sāślokavārtika with the commentary Kāśikā of Sucaritamiśra</title>
	           <editor>K. Sābaśiva Śāstrī</editor>
	           <publisher>Printed by the Superintendent, Government Press</publisher>
	           <date>1926--1943</date>
	           <address>
               <name>Trivandrum</name>
            </address>
	           <series>
	              <title level="s">Trivandrum Sanskrit Series</title>
	              <biblScope unit="vol">90,99,150</biblScope>
	           </series>
	        </bibl>
      </sourceDesc>
   </fileDesc>
   <encodingDesc>
      <p/>
      <!-- ... --></encodingDesc>
   <profileDesc><!-- ... --></profileDesc>
   <revisionDesc>
      <change who="#pma">Please see <ref target="https://github.com/paddymcall/SARIT/commits/master/ratnakIrti-nibandhAvali.xml">https://github.com/paddymcall/SARIT/commits/master/ratnakIrti-nibandhAvali.xml</ref> for a list of changes.</change>
      <change when="2011-07-20" who="#pma">
         <persName>Patrick Mc Allister</persName>: continuing work on the CAPV.
</change>
      <change when="2009-03" who="#pma">
         <persName>Patrick Mc Allister</persName>: replaced all &lt; with « and
all &gt; with ».
</change>
      <change> Converted from source file to TEI XML by <persName>Patrick Mc
      Allister</persName> 
         <date>2009-03-10</date>
      </change>
   </revisionDesc>
</teiHeader>
	 \end{minted}
	 \end{landscape}
       
      \clearpage
      \begin{english}
      \printshorthands
      \printbibliography
      \end{english}
    
\end{document}
