\documentclass[article,12pt,a4paper]{memoir}
  \usepackage{euler}
  \usepackage{xltxtra}
  \usepackage{polyglossia}
  \PolyglossiaSetup{sanskrit}{
  hyphenmins={2,3},% default is {1,3}
  }
  \setdefaultlanguage{sanskrit}
  % english should be available, notes and stuff
  \setotherlanguage{english}
  \setotherlanguage[numerals=arabic]{tibetan}
  \usepackage{fontspec}
  \usepackage{xunicode}
  \catcode`⃥=\active \def⃥{\textbackslash}
  \catcode`❴=\active \def❴{\{}
  \catcode`〔=\active \def〔{{[}}% translate 〔OPENING TORTOISE SHELL BRACKET
  \catcode`〕=\active \def〕{{]}}% translate 〕CLOSING TORTOISE SHELL BRACKET
  \catcode`❴=\active \def❴{\{}
  \catcode`❵=\active \def❵{\}}
  \catcode` =\active \def {\,}
  %% show a lot of tolerance
  \tolerance=9000
  \def\textJapanese{\fontspec{Kochi Mincho}}
  \def\textChinese{\fontspec{HAN NOM A}}
  \def\textKorean{\fontspec{Baekmuk Gulim} }
  % make sure English font is there
  \newfontfamily\englishfont[Mapping=tex-text]{TeX Gyre Schola}
    % set up a devanagari font
  \newfontfamily\devanagarifont[Script=Devanagari,Mapping=devanagarinumerals]{Chandas}
	\newfontfamily\rmlatinfont[Mapping=tex-text]{TeX Gyre Pagella}
	\newfontfamily\tibetanfont[Script=Tibetan,Scale=1.2]{Tibetan Machine Uni}
  \newcommand\bo\tibetanfont
  
    \defaultfontfeatures{Scale=MatchLowercase,Mapping=tex-text}
	\setmainfont{Chandas}
    \setsansfont{TeX Gyre Bonum}
  
  \setmonofont{DejaVu Sans Mono}
	    % numbering depth
	    \maxtocdepth{section}
	    \setsecnumdepth{all}
	    \newenvironment{docImprint}{\vskip 6pt}{\ifvmode\par\fi }
	    \newenvironment{docDate}{}{\ifvmode\par\fi }
	    \newenvironment{docAuthor}{\ifvmode\vskip4pt\fontsize{16pt}{18pt}\selectfont\fi\itshape}{\ifvmode\par\fi }
	    % \newenvironment{docTitle}{\vskip6pt\bfseries\fontsize{18pt}{22pt}\selectfont}{\par }
	    \newcommand{\docTitle}[1]{#1}
	    \newenvironment{titlePart}{ }{ }
	    \newenvironment{byline}{\vskip6pt\itshape\fontsize{16pt}{18pt}\selectfont}{\par }
	    % setup title page; see CTAN /info/latex-samples/TitlePages/, and memoir
	  \newcommand*{\plogo}{\fbox{$\mathcal{SARIT}$}}
	  \newcommand*{\makeCustomTitle}{\begin{english}\begingroup% from example titleTH, T&H Typography
	  \thispagestyle{empty}
	  \raggedleft
	  \vspace*{\baselineskip}
	  
	      % author(s)
	    {\Large Dharmakīrti, Dharmottara and Durveka Miśra}\\[0.167\textheight]
	    % maintitle
	    {\Huge Nyāyabindu}\\[\baselineskip]
	    % titlesubtitle
	    {\small  — Nyāyabinduṭīkā — Dharmottarapradīpa}\\[\baselineskip]
	    {\Large SARIT}\\\vspace*{\baselineskip}\plogo\par
	  \vspace*{3\baselineskip}
	  \endgroup
	  \end{english}}
	  \newcommand{\gap}[1]{}
	  \newcommand{\corr}[1]{($^{x}$#1)}
	  \newcommand{\sic}[1]{($^{!}$#1)}
	  \newcommand{\reg}[1]{#1}
	  \newcommand{\orig}[1]{#1}
	  \newcommand{\abbr}[1]{#1}
	  \newcommand{\expan}[1]{#1}
	  \newcommand{\unclear}[1]{($^{?}$#1)}
	  \newcommand{\add}[1]{($^{+}$#1)}
	  \newcommand{\deletion}[1]{($^{-}$#1)}
	  \newcommand{\pratIka}[1]{\textcolor{cyan}{#1}}
	  \newcommand{\name}[1]{#1}
	  \newcommand{\persName}[1]{#1}
	  \newcommand{\placeName}[1]{#1}
	  % running latexPackages template
     \usepackage[x11names]{xcolor}
     \definecolor{shadecolor}{gray}{0.95}
     \usepackage{longtable}
     \usepackage{ctable}
     \usepackage{rotating}
     \usepackage{lscape}
     \usepackage{ragged2e}
     
	 \usepackage{titling}
	 \usepackage{marginnote}
	 \renewcommand*{\marginfont}{\color{black}\rmlatinfont\scriptsize}
	 \setlength\marginparwidth{.75in}
	 \usepackage{graphicx}
	 \usepackage{csquotes}
       
	 \def\Gin@extensions{.pdf,.png,.jpg,.mps,.tif}
       %% biblatex stuff start
	 \usepackage[backend=biber,citestyle=authoryear,bibstyle=authoryear]{biblatex}
	 
		 \addbibresource[location=remote]{https://raw.githubusercontent.com/paddymcall/Stylesheets/HEAD/profiles/sarit/latex/bib/sarit.bib}
	 \renewcommand*{\citesetup}{%
	 \rmlatinfont
	 \biburlsetup
	 \frenchspacing}
	 \renewcommand{\bibfont}{\rmlatinfont}
	 \DeclareFieldFormat{postnote}{:#1}
	 \renewcommand{\postnotedelim}{}
	 %% biblatex stuff end
	 
	 \setcounter{errorcontextlines}{400}
       
	 \usepackage{lscape}
	 \usepackage{minted}
       
	   % pagestyles
	   \pagestyle{ruled}
	   
	   \makeoddfoot{ruled}{{\tiny\rmlatinfont \textit{Compiled: \today}}}{}{\rmlatinfont\thepage}
	   \makeevenfoot{ruled}{\rmlatinfont\thepage}{}{{\tiny\rmlatinfont \textit{Compiled: \today}}}
	   
	 
	 \usepackage[noend,series={A}]{reledmac}
	 
       % simplify what ledmac does with fonts, because it breaks. From the documentation of ledmac:
       % The notes are actually given seven parameters: the page, line, and sub-line num-
       % ber for the start of the lemma; the same three numbers for the end of the lemma;
       % and the font specifier for the lemma. 
       \makeatletter
       \def\select@lemmafont#1|#2|#3|#4|#5|#6|#7|%
       {}
       \makeatother
       \setlength{\stanzaindentbase}{20pt}
     \setstanzaindents{8,2,2,2,2,2,2,2,2,2,2,2,2,2,2,}
     % \setstanzapenalties{1,5000,10500}
     \lineation{page}
     % \linenummargin{inner}
     \linenumberstyle{arabic}
     \firstlinenum{5}
    \linenumincrement{5}
    \addtolength{\skip\Afootins}{1.5mm}
    \Xnotenumfont{\bfseries\footnotesize}
    \sidenotemargin{outer}
    \linenummargin{inner}
       
       \usepackage[destlabel=true,% use labels as destination names; ; see dvipdfmx.cfg, option 0x0010, if using xelatex
       pdftitle={Nyāyabindu, Nyāyabinduṭīkā, and Dharmottarapradīpa},
       pdfauthor={SARIT: Search and Retrieval of Indic Texts. DFG/NEH Project (NEH-No.
	HG5004113), 2013-2016 }]{hyperref}
       \hyperbaseurl{}
       \usepackage[english]{cleveref}% clashes with eledmac < 1.10.1!
       % \newcommand{\cref}{\href}
     
\begin{document}
    
     \makeCustomTitle
     \let\tabcellsep&
	\frontmatter
	\tableofcontents
	% \listoffigures
	% \listoftables
	\cleardoublepage
        \begin{english} {\docTitle  \begin{titlePart} Nyāyabindu, Nyāyabinduṭīkā, and Dharmottarapradīpa\end{titlePart}}\textit{Dharmakīrti, Dharmottara, and Durveka Miśra}\end{english}\mainmatter 
\paragraph*{आचार्यदुर्वेकमिश्रविरचितो: धर्मोत्तरप्रदीपः ।}
	  
	% new div opening: depth here is 0
	
	    
	    \begingroup
	    \beginnumbering% beginning numbering from div depth=0
	    
	  
\chapter[{प्रथमः प्रत्यक्षपरिच्छेदः}]{प्रथमः प्रत्यक्षपरिच्छेदः}“

	  \begin{center}%% label @type='head'
	\textbf{आचार्यधर्मोत्तरविरचिता}
	\end{center}
	 

	  \begin{center}%% label @type='head'
	\textbf{न्यायबिन्दुटीका}
	\end{center}
	 “

	  \begin{center}%% label @type='head'
	\textbf{आचार्य्धर्मकीर्त्तिप्रणीतो न्यायबिन्दुः}
	\end{center}
	 

	  \begin{center}%% label @type='head'
	\textbf{प्रथमः प्रत्यक्षपरिच्छेदः ।}
	\end{center}
	 

	  \begin{center}%% label @type='head'
	\textbf{ऊँ नमो वीत्तरागाय[[ऊँ नमः सर्वज्ञाय--\cite{dp-msB} \cite{dp-edE} \cite{dp-edH} \cite{dp-edN} नास्ति \cite{dp-msA} \cite{dp-edP}]] ।}
	\end{center}
	 

	  \pstart सम्यग्ज्ञानपूर्विका सर्वपुरुषार्थसिद्धिरिति तद् व्युत्पाद्यते ॥१॥
	\pend
      ” 
	    
	    \stanza[\smallbreak]
	जयन्ति जातिव्यसनप्रबन्धप्रसूतिहेतोर्जगतो विजेतुः ।&रागाद्यरातेः सुगतस्य वाचो मनस्तमतानबवमादधानाः ॥\&[\smallbreak]


	”

	  \pstart \leavevmode\marginnote{\textenglish{1b/ms}}\add{\footnote{पत्रमत्र त्रुटितम्--सं०}सूत्राभिधर्म}विनयाः सुगतस्य वाचो बाधा विनाद्भुततया सततं जयन्ति । आभान्ति दर्पण इव प्रतिबिम्बितानि तत्त्वानि यत्र सकलस्य पदार्थराशेः ॥ \textbf{श्रीधर्मकीर्त्त्य}भिमुखानि जगद्धितानि \textbf{धर्मोत्तरस्य} विपुलानि पदान्यमूनि । सम्यक् प्रकाश्य सुकृतं समुपार्जयामि तेनापि दुःखजलधेर्जगदुद्धरामि ॥
	\pend
      
	    
	    \stanza[\smallbreak]
	मात्सर्यादिमलोपेतैः \footnote{पत्रमत्र त्रुटितम्--सं०}\add{शक्यं न गुणदर्शनम् ।}&यतो हे\add{मपरीक्षायाः सन्तो हि निकषोपलाः ॥}\&[\smallbreak]


	

	  \pstart \textbf{न्यायबिन्दुटीकां} चिकीर्षुरेष \textbf{धर्मोत्तरो} भगवत्पूजापुण्ययोः साध्यसाधनभावमतीन्द्रियमीदृशि विषये गत्यन्तरविरहादभ्युपेयप्रामाण्यादागमादवगम्य, पुण्यस्य चापुण्यविरोधित्वादपुण्यफलविध्नाद्यभावो व्यापकविरुद्धविधिना सम्भवी, तेनाविध्नेन ग्रन्थकरण\footnote{पत्रमत्र त्रुटितम्--सं०}\add{तत्समाप्तौ  \leavevmode\marginnote{\textenglish{2/dm}} तदध्ययने} च भगवद्गुणगणश्रवणजनितातिशयविशेषासादितपुण्यस्य तथैवाविध्नेन जिज्ञासितशास्त्रार्थतत्त्वावगमो भविष्यति, दृष्टशिष्टाचारोप्यनुपालितो भविष्यतीत्यभिसन्धाय वाग्विजयाख्यानद्वारां भगवतः पूजां स्तुतिमयीमारभते ।
	\pend
      

	  \pstart स्यादेतत्--पूजायाः पुण्योपजननमात्रप्रयोज\footnote{पत्रमत्र त्रुटितम्--सं०}\add{नस्य सम्पादकत्}वादारिप्सितारम्भात्प्राक् कायमयीमेवेष्टदेवतापूजामारचय्य किमिति न तदारभ्यते ? अथोच्यते--तत्पूर्विकायामपि प्रवृत्तौ “स्तुतिपूजैव प्रवृत्तिपुरःसरी किन्न कृता” इति चोद्यमापद्येत । तथा चाशोकवनिकाचोद्यसदृशमिदमिति नानुवाद्यमपि विदुषामिति । असदेतत् । एवं हि\footnote{पत्रमत्र त्रुटितम्--सं०}\add{कायपूजाया आरचने स्तुति}पदानि प्रयुञ्जानस्याऽतद्व्याख्यानभूतस्यास्य श्लोकस्याप्रकृतस्य करणं नापद्येत । कायपूजा तु सुखासनोपदेशनादीतिकर्त्तव्यतास्थानीयत्वान्नाप्रकृतकरणचोद्येनोपद्रूयत इति ।
	\pend
      

	  \pstart अत्रोच्यते--स्यादेवैतद् यदि स्वार्थमुद्दिश्य स्तुतिरीदृशी पूजा विधीयते । किं तर्हि ? श्रोतृजना\footnote{पत्रमत्र त्रुटितम्--सं०}\add{र्थमुद्दिश्यापि} भगवतो गुणकीर्त्तने कृते श्रोतृभिरन्ततः काव्यगुणजिज्ञासयापि श्लोकोऽवश्यं ज्ञातव्यः । तज्ज्ञानात् तथागतगुणास्तावत्कालं तावच्छोतृसन्तानमध्यासते । तत्र ये तावद् भगवति प्रागतिप्रसन्नमनसस्तेषामेवंविधगुणातिशयशालिनि “स्थान एवास्माकं मनः प्रसन्नम्”इति निश्चिन्वतां स्थे\footnote{पत्रमत्र त्रुटितम्--सं०}\add{मानमासादयति चित्तम् ।}ये च मध्यस्थास्तेषाम् “एवंभूतगुणरत्नालङ्कृते महतो महीयसि चित्तमावर्जयितुमुचितं स्वहितावहितैः, तद्वयमियन्तं कालं प्रमाद्यन्त एवोदास्महे स्म” इति निर्विद्यमानानां चित्तमतिमत्तं प्रतिष्ठते । येऽप्यनतिप्रसन्नास्तेषामपि—“एवंविधगुणनिकेतने किमस्माभिरकस्माद्विद्विष्यते” इति मननान्मनागावर्जनं माध्यस्थ्यं वा \footnote{पत्रमत्र त्रुटितम्--सं०}\add{स्यादित्यतिप्रसन्नमध्यस्थ}योः पुरुषयोश्चित्तप्रसादस्थैर्य-मनःप्रसादोपजननाभ्यां पुण्यातिशयो जायते । तृतीयस्यापि मनागावर्जनेपि पुण्यप्रसवः । माध्यस्थ्ये तु भगवद्विद्वेषोपचयोपनेयनरकेष्वनतिपतनं विद्वेषोपशमाद् भवति । असत्यां तु स्तुतिमयपूजायामित्थं त्रिविधश्रोतृजनप्रयोजनं यत् तन्न कृतं स्यात्--इति स्वपरार्थो\leavevmode\marginnote{\textenglish{2a/ms}}द्देशेन स्तुतिमयी पूजा विधीयत इति स्थितम् ।
	\pend
      

	  \pstart तत्रानेन श्लोकेन भगवान् स्वार्थसम्पदा परार्थसम्पदुपायसम्पदा परार्थसम्पदा च स्तूयते । तासां तिसृणामपि सम्पदामवश्यवक्तव्यत्वात् । तथाहि चिरकालाभ्याससात्म्यीकृतमहाकृपस्य भगवतः परार्थसम्पत् प्रधानं प्रयोजनम्, इतरदप्रधानम्, आनुषङ्गिकं तथागतत्वमपीति सा तावदवश्याभिधेया ।
	\pend
      

	  \pstart तदाह \textbf{भट्टवराहस्वामी}--
	\pend
      

	  \pstart “साक्षात्कृताशेषजगत्स्वभावं प्रासङ्गिकं यस्य तथागतत्वम् । इति ।”
	\pend
      

	  \pstart सा चोपायसम्पदमन्तरेणासम्भविनीति तदभिधानमप्यावश्यकम् । इयञ्चानधिगतस्वाथसम्पदो न सिध्यतीति तदभिधानमपि नियतमापतितम् । तदाह स एव--
	\pend
      “
	    
	    \stanza[\smallbreak]
	तीर्णः स्वयं याति जगत्समग्रं मार्गोपदेशेऽधिकृतो हि नाथः । इति ।\&[\smallbreak]


	”

	  \pstart तत्र स्वार्थसम्पन्नस्य परार्थसम्पादनोपायसम्पत् तत्साध्या च परार्थसम्पादनसम्पदिति प्रथमं \textbf{जाती}त्यादिना \textbf{सुगतस्ये}त्यन्तेन स्वार्थसम्पत्तिरुक्ता । अनुपायस्य परार्थसम्पत्तिर्न सम्पद्यत \leavevmode\marginnote{\textenglish{3/dm}} इति तदनु \textbf{वाच} इत्यनेन परार्थोपायसम्पदुक्ता । “ईदृशं वस्तुनो रूपम्, इदञ्चानुष्ठेयम्” इत्यादिरूपेण धर्मदेशनैव हि भगवतो जगद्धितावगमनलक्षणपरार्थसम्पादनोपायसम्पद् वैद्यवरस्येव व्याधिस्वरूपप्रतीतिशक्तिभैषज्योपदेशो रोगोपशमसम्पादनोपायसम्पत् । तदनन्तरं तदुपायसम्पत्तिसाध्या परार्थसम्पदन्येन पदेनोक्ता ।
	\pend
      

	  \pstart तत्र \textbf{वाचो जयन्ति}--इति सम्बन्धः । अविवक्षितकर्मत्वादकर्मत्वम् । \textbf{वाचः} सूत्राभिधर्मविनयलक्षणाः । \textbf{जयन्ति} उत्कृष्यन्ते प्रकृष्टा भवन्ति । उत्कर्षश्च सजातीयापेक्षयेति सामर्थ्यादीश्वरादिवचनेभ्यः प्रकृष्टा भवन्तीत्यर्थः । यद्वा \textbf{जयन्ति} अभिभवन्ति । अभिभवश्च प्रतियोगिगोचर इति । अर्थात् तीर्थिकशास्त्राभिभवं कुर्वन्तीत्यर्थः । प्रमाणोपपन्नार्थप्रतिपादकत्वादासाम् । तेषां तु तद्वैपरीत्यादिति बुद्धिसिद्धं कृत्वा केवलमेतदत्रोक्तम् । \textbf{विनिश्चयटीकायां} पुनरनेन “युक्तिप्रभावे”त्यादि हेतुभावेन विशेषणमुपात्तम् ।
	\pend
      

	  \pstart कस्य ता इत्याकाङ्क्षायामाह--\textbf{सुगतस्य} इति । \textbf{सु}शब्दोऽयमर्थत्रयवृत्तिर्द्रष्टव्यः । प्रशस्तं गतः सुगतः । प्रशस्तं यथा भवति तथा गतः--संसारात् प्रक्रान्तः । कथं तथा गमनं तस्य ? नैरात्म्यदर्शनेन संसारातिक्रमात् । तस्य च प्रज्ञानिश्रयात् । अथवा गत्यर्थानां ज्ञानार्थत्वादपि प्रशस्तं गतः ज्ञातवानित्यर्थः । प्रशस्तञ्च तत् तत्त्वं नैरात्म्यलक्षणम् । तत्त्वरूपत्वञ्च तस्य प्रमाणोपपन्नत्वात् । दृष्टश्चायं सुशब्दः प्रशस्तार्थवृत्तिः । सुरूपा रूपाजीवेति यथा । अपुनरावृत्त्या वा गतः सुगतः । गतः प्रयातः संसारात् । पुनरावृत्तिशब्दवाच्ययोर्जन्मदोषयोरात्मदर्शनबीजोपघातेन भगवता समूलघातं निहतत्वात् । एतदर्थेपि सुशब्दो दृष्टः । सुनयो ज्वर इति यथा । निःशेषं वा गतः सुगतः । निःशेषं यावद् गन्तव्यं तावद् गतः प्राप्तः । अक्लेशनिर्जरकायवाग्बुद्धिवैगुण्यलक्षणशेषलक्षणप्रहाणेन मुनेस्तत्पदप्राप्तेः । एवंवृत्तिरपि सुशब्दो दृश्यते । सुपूर्णो घट इति यथा ।
	\pend
      

	  \pstart तस्य सु\leavevmode\marginnote{\textenglish{2b/ms}}गतस्य किम्भूतस्येत्याह--\textbf{विजेतुः} अभिभवितुः । कस्यासौ विजेतेत्याह--\textbf{जगतो} जीवलोकस्य । विजयश्च जगदपेक्षया परमपदप्राप्त्या तस्य प्रकृष्टत्वं द्रष्टव्यम् । अत एव जगदभिभूतं भवति । तस्य तद्वैपरीत्यात् । न पुना राजविजय इव राजान्तरापेक्षया कायादितिरस्कारोऽभिभवोऽवसेयः ।
	\pend
      

	  \pstart जगतः कीदृशस्य ? जायते संसरत्यनयेति \textbf{जातिः} तृष्णा । व्यस्यते विविधेन प्रकारेणेतस्ततो अस्यते\footnote{तोऽस्यते}क्षिप्यते अनेनेति, अस्यतीति वा \textbf{व्यसनम्} । जातिरेव व्यसनमित्यन्तर्नीतनियमः समासः कर्त्तव्यः । तृष्यन्नेव हि प्राणी तत्तदाचरति येन संसारे संसरति । ततस्तयैवासौ इतस्तत उत्पादद्वारेण व्यस्तः क्षिप्तो भवतीति सैव व्यसनं युक्तम् । अथवा \textbf{जातौ} उत्पत्तौ निकायविशिष्टायां \textbf{व्यसनम्} आसक्तिः “विद्याधरोऽहं भूयासं”, “देवोऽहं भूयासम्” इत्याद्याकारोऽभिनिवेशः । यद्वा \textbf{जात्या}श्रितानि \textbf{व्यसनानि} दुःखानि\footnote{प्रतौ “दुष्खानि” इत्येवं वर्त्तते--सं०} रोगशोकादीनि । प्राक्तनव्याख्याने तस्य, अन्तिमव्याख्याने तेषां \textbf{प्रबन्धः} प्रवाहः । तदेव वा प्रकृष्टो \textbf{बन्धः} । प्रवाहपक्षे \textbf{प्रबन्धे}न । बन्धपक्षे \textbf{प्रबन्ध}स्य । \textbf{प्रसूतेः} कारणस्य । एवञ्च \textbf{प्रबन्धप्रसूति-} श्रुतिसहितं \textbf{हेतु}पदमुपाददानः परमतनिराकरणञ्चाभिप्रैति । तथा ह्येवमभिधाने सत्ययं \leavevmode\marginnote{\textenglish{4/dm}} तात्पर्याथः--जगदेवानन्यसत्त्वनेयं स्वयमेव तथा तत्तदाचरति येन तथा संसारे संसरति । न तु परप्रेरितं यथाऽन्ये मन्यन्ते--
	\pend
      “
	    
	    \stanza[\smallbreak]
	अज्ञो जन्तुरनीशोऽयमात्मनः सुखदुःखयोः ।&ईश्वरप्रेरितो गच्छेत् स्वर्गं वा स्व\footnote{श्व}भ्रमेव वा ॥ \href{http://http://sarit.indology.info/?cref=MBh.3.30.28}{महाभा० वन० ३०. २८}इति ।\&[\smallbreak]


	”

	  \pstart इतरथा जातिव्यसनाश्रयस्येति तदन्यविशेषणसहितमेव ब्रूयादिति ।
	\pend
      

	  \pstart किम्भूतस्य सुगतस्य ? \textbf{रागा}दीनां क्लेशानाम् \textbf{अरातेः} शत्रोः सत्कायदृष्टिविगमेन, तेन तेषां स्वसन्ताने समूलमुन्मूलितत्वात् । यत एव भगवान् रागाद्यरातिः, जगच्च जातिव्यसनप्रबन्धप्रसूतिहेतुः, अत एवासौ ततः प्रकृष्टो भवति \textbf{विजेता} तस्य । अथवा \textbf{रागाद्यराते}रिति \textbf{जगतो} विशेषणम् । रागादयोऽरातयः प्रतिपक्षा नित्यमुपघातका यस्येति कृत्वा । तदा तु तद्विजेता भगवान् सुगतत्वादेव बोद्धव्यः । जगतो ये रागादयस्तेषामरातेरित्यादिव्याख्यानं तु क्लिष्टत्वात् नोक्तम् ।
	\pend
      

	  \pstart किंविशिष्टा वाच इत्याह--\textbf{मन} इत्यादि । सदसदर्थविवेकविबन्धकत्वात् तम इव \textbf{तनो}ऽज्ञानमविद्या । तच्च मनसो विकल्पविज्ञानस्य कयाचिद्व्यपेक्षया धर्म इव भिन्नः कथ्यते । वस्तुतस्तु क्लिष्टमेव ज्ञानमविद्या । यद्वा क्लिष्टं \textbf{मन} एव तम इव \textbf{तमः} पूर्ववत् ।
	\pend
      

	  \pstart केचित्तु \textbf{मनस्तम} इति शब्दसमुदायमज्ञानार्थवाचकमाचक्षते ।
	\pend
      

	  \pstart तस्य \textbf{तानवं} तनुत्वं मन्दीभवनमित्यर्थात् । तद् \textbf{आदधानाः} कुर्वाणाः । तानवग्रहणेन चेदमाचष्टे--न भगवतो वाचः सर्वथा जगदज्ञानमुपध्नन्ति, तच्छ्रवणमात्रेण मोहहान्या मुक्तत्वात् मार्गभावनावैयर्थ्यप्रसङ्गात् । किन्तु कियन्तं कालमालोच्यमानाः समुदाचारतो मोहस्य मान्द्यापादनेन तं तनूकुर्वन्तीति ।
	\pend
      

	  \pstart अथवा अन्यथा व्याख्यायते--ता \textbf{वाचस्तमो जयन्ति} अभिभवन्ति\footnote{लेखोऽत्र घृष्टः ।}\add{... ... ... ... ...}\leavevmode\marginnote{\textenglish{3a/ms}}\add{... ... ...}मोहप्रचारम् \textbf{आदधानाः} । शेषं समानं पूर्वेण ।
	\pend
      

	  \pstart यद्वा मनसि तमो यस्य स \textbf{मनस्तमा} मूढ उक्तः । तस्य भावो \textbf{मनस्तमस्ता} । तस्या अनवो निष्ठा । तथाहि--नूतिर्नवः । तद्विरुद्धेन\footnote{लेखोऽत्र घृष्टः ।}\add{... ... ...}\textbf{तमादधानाः} मूढत्व\footnote{लेखोऽत्र घृष्टः ।}\add{... ... ...}प्रवर्त्तनात् । शेषं समानं पूर्वेणेति व्याख्यातः श्लोकः ।
	\pend
      

	  \pstart केवलमिदमालोच्यताम्--\textbf{तथागत}मभिस्तुवता \textbf{धर्मोत्तरे}णास्यैव विजयो मुख्यवृत्त्या किं न कथितः ? किं पुनर्धर्मिविशेषणत्वेनानुषङ्गतः प्रतिपादित इति ? अत्र समाधीयते । आचायश्री\textbf{धर्मकीर्त्तिना} भगवत्प्रवचनार्थसमर्थन\footnote{लेखोऽत्र घृष्टः ।}\add{... ... ...} \add{... ...}प्रकृतत्वात् विजयो मुख्यतः प्रतिपाद्यते ।
	\pend
      

	  \pstart यद्वा यस्य वाच एव तथा जयन्ति तस्य विजयो दण्डाज्जयन्यायेनातिशयेन प्रतिपादितः । सूचितश्चासौ \textbf{विजेतृ}पदेन ।
	\pend
      \leavevmode\marginnote{\textenglish{5/dm}}“

	  \pstart सम्यग्ज्ञानपूर्विके\footnote{०पूर्विका सर्वेत्यादिना--\cite{dp-msA} \cite{dp-edP} \cite{dp-edE}}त्यादिनाऽस्य प्रकरणस्याभिधेयप्रयोजनमुच्यते ।
	\pend
      ”

	  \pstart अथवा\textbf{ऽऽगमिकानां} मतेन निरुपधिशेषे निर्वाणधातौ परिनिर्वृतो भगवान् । परिनिर्वृतस्यास्य प्रवचनमयमेव वपुर्विद्यत इति \textbf{आगमिको} वाग्विजयमेव प्रतिपादयते स्मेति ।
	\pend
      

	  \pstart \textbf{सम्यग्ज्ञाने}त्यादिना प्रकरणस्य यत् प्रयोजनं सम्यग्ज्ञानव्युत्पत्तिः, तस्या यत् प्रयोजनं पुरुषार्थसिद्धिरूपं तदुच्यते । न च सम्यग्ज्ञानव्युत्पत्तेः सम्यग्ज्ञानपरिज्ञानं प्रयोजनं न पुरुषार्थसिद्धिरिति शक्यमभिधातुम् । विप्रतिपत्तिनिराकरणेन स्वरूपप्रतिपत्तिरि\footnote{रे}व हि सम्यग्ज्ञानस्य व्युत्पत्तिः । सा कथमात्मन एव प्रयोजनं भवेदित्यभिप्रायेण व्याख्यातवन्तौ \textbf{विनीतदेव-शान्तभद्रौ} । तद्व्याख्यानमवमन्यमानोऽ\textbf{भिधेयप्रयोजनमुच्यते} इति व्याचष्टे । अवज्ञाने चायमाशयः--अस्येदं प्रयोजनमिति खल्वन्वय-व्यतिरेकाभ्यामवधार्यते । नान्यथा । इयञ्च-स्वरूपा प्रसिद्धिः \footnote{पुरुषार्थसिद्धिः} सम्यग्ज्ञानव्युत्पत्तिमन्तरेणापि गोपालाङ्गनादीनां भवन्ती सति सम्यग्ज्ञानि\footnote{ने}, सत्यामपि तद्व्युत्पत्तौ असति सम्यग्ज्ञाने मनीषिणामभवन्ती न तद्व्युत्पत्तेरन्वयव्य-तिरेकावनुविधत्ते । किन्तर्हि ? सम्यग्ज्ञानस्येति तस्यैव प्रयोजनं भवितुमर्हतीति ।
	\pend
      

	  \pstart \textbf{सम्यग्ज्ञानेत्यादिना} वाक्येन करणेन \textbf{प्रकरणाभिधेयस्य} सम्यग्ज्ञानलक्षणस्य \textbf{प्रयोजनं} दृष्टं पुरुषार्थसिद्धिलक्षण\textbf{मुच्यते वार्त्तिक}कृता कर्त्रेत्यर्थात् ।
	\pend
      

	  \pstart ननु च यत्राभिधाव्यापारः समाप्यते स वाक्यार्थः । न चासौ पुरुषार्थसिद्धौ विश्राम्यति । किं तर्हि ? श्रोतुः सम्यग्ज्ञानव्युत्पत्तौ । तत् कथं सा वाक्यार्थत्वेनोच्यते ? उच्यते । यद्यपि शब्दाभिधाव्यापारापेक्षया तत् सम्यग्ज्ञानं व्युत्पाद्यते शिष्य इति शिष्यसम्यग्ज्ञानविषया व्युत्पत्तिक्रिया प्राधान्याद्वाक्यार्थः, तथाप्यस्य तात्पर्यार्थसम्भवे तन्निरूपणेनेदमुच्यते । तथा हि तद्व्युत्पत्तिमेवासौ किमति कार्यते ? यतः सम्यग्ज्ञानपूर्विका सर्वपुरुषार्थसिद्धिः\footnote{लेखोऽत्र घृष्टः ।}\add{... ... ...} \add{... ... ...}मुच्यते । यत्र पुनरभिधाविषय एवार्थः सम्भवी न तु तात्पर्यार्थो \leavevmode\marginnote{\textenglish{3b/ms}}\footnote{एकोऽक्षरोत्र वर्त्तते किन्तु स सम्यङ्न पठ्यते ।}...स्तत्र स एव वाक्यार्थः कल्प्यते । द्वितयसम्भवे तु यत्परं वाक्यं तथा च तस्यार्थ इत्यार्थेन न्यायेनाभिधेयस्य प्रयोजनं पुरुषाथसिद्धिर्वाक्यार्थत्वेनोच्यत इति व्याख्यायत इति ।
	\pend
      

	  \pstart केचित् पुनरिदं \textbf{धर्मोत्तरीयं} वाक्यमन्यथा व्याचक्षते । नात्र प्रयोजनशब्देन फलमभिप्रेतं किन्तु प्रयुज्यते प्रवर्त्त्यतेऽनेनेति, प्रयोजयतीति वा प्रयोजनम् । तच्च पुरुषार्थसिद्धिहेतुत्वम् । अभिधेयस्य हि सम्यग्ज्ञानस्य पुरुषार्थसिद्धिहेतुत्वेन प्रयुक्तः पुरुषः प्रवर्त्तत इति । उत्तरत्रापि प्रयोजनमिदमेव विवक्षितम् । अत एव--\textbf{अत्र चे}त्यादिवाक्ये \textbf{सर्वपुरुषार्थसिद्धिहेतुत्वं प्रयोजनं} प्रवर्त्तक\textbf{मुक्त}मिति स्पष्टीकरणं घटत इति । तच्च नातिश्लिष्टमुत्पश्यामः । तथा हि \textbf{अत्र च प्रकरणाभिधेयस्य सम्यग्ज्ञानस्य पुरुषार्थसिद्धिहेतुत्वं प्रयोजनमुक्त}मिति वक्ष्यमाणेन व्यक्तीकृतत्वादिहापि \textbf{अभिधेय}स्य \textbf{प्रयोजनम्} इति षष्ठीतत्पुरुषोऽवश्यकार्यः । तथा चायमसमर्थः पदविधिर्भवेत् । तद्धि पुरुषार्थसिद्धिहेतुत्वं नाभिधेयस्य सम्यग्ज्ञानस्य प्रवर्त्तकमपि तु पुरुषस्य । तत् कथमभिधेयपदेन समस्येत प्रयोजनपदम् ? न च केनचिद्रूपेणाभिधेयसम्बन्धि  \leavevmode\marginnote{\textenglish{6/dm}} त्वऽस्य समर्थयोरेकार्थीभावो भवतीति । न हि यज्ञदत्तपुत्रो भृत्यत्वादिना रूपेण देवदत्तसम्बन्धी भवन्देवदत्तपुत्र इति षष्ठीसमासस्य विषयो भवितुमर्हति ।
	\pend
      

	  \pstart अथ भावप्रधानत्वान्निर्देशस्य प्रयोजनं प्रयोजकत्वमित्यर्थः । तच्चाश्रितस्य प्रयोजकतया भवति सम्बन्धीति समर्थविधिः कल्प्यते ।
	\pend
      

	  \pstart ननु यदि सम्यग्ज्ञानगतं पुरुषार्थसिद्धिहेतुत्वं प्रयोजनं प्रयोजकत्वं तर्हि सम्यग्ज्ञानं प्रयोजनमित्यापन्नम् । यतो न प्रयोजनत्वमेव प्रयोजनं भवितुमर्हति । न च सम्यग्ज्ञानेन प्रयुक्तः पु षः सम्यग्ज्ञाने प्रवर्त्तते, किन्तु पुरुषार्थसिद्धिहेतुत्वेन । तत् कथं प्रयोजनं भवितुमर्हति ? किञ्च वृद्धव्यवहारो हि शब्दार्थव्यवहारभूमिः । न च वृद्धैः प्रयोजकः सत्यपि प्रयोजयितृत्वे प्रयोजनमुच्यते । अन्यथा सम्यग्ज्ञानं व्युत्पद्यमानानामात्मानं व्युत्पादकं कर्त्तुं प्रवर्त्तमान आचार्यः प्रयोजक इति सम्यग्ज्ञानव्युत्पत्तेः प्रयोजनमुच्येत । योऽपि कटं कुर्वन्तं कर्त्तुं प्रयुङ्क्ते सोऽपि तत्प्रयोजनमुच्येतेत्येवंवादी न लौकिको न परीक्षक इत्युपेक्षणीय एव ।
	\pend
      

	  \pstart अपि च किमेतदन्यथा नोपपद्यत एव येनैवं मृत्वा शीर्त्वोपपाद्येत । न चैवम्, अन्यथापि सूपपादत्वात् । प्रयोजकत्वमिति निर्द्देशे च \textbf{धर्मोत्तरः} किं गौरवं पश्यति येनैवमवाचकमाचक्षीत ? कथञ्च \textbf{प्रकरणस्येति} दुरुपपादं प्रयुञ्जीत ? किञ्च यदीप्सञ्जिहासन् वा पुरुषः प्रवर्त्तते तदुपादानपरित्यागाभ्यां प्रवर्त्तितो भवति । यथाह \textbf{अक्षपादः}--“यमर्थमधिकृत्य प्रवर्त्तते तत् प्रयोजनम्” \href{http://http://sarit.indology.info/?cref=ns.1.1.24}{न्यायसू० १. १. २४} इति । अधिकृत्य उद्दिश्येत्यर्थः । न च सम्यग्ज्ञानस्य भिन्नं सदपि पुरुषार्थसिद्धिहेतुत्वम् ईप्सन् जिहासन् वा प्रवर्त्तते । किन्तर्हि ? हिताहितप्राप्तिपरिहारावुद्दिश्येति तावेव प्रयोजने युक्ते । किञ्चिदजिहासोरनुपादित्सोरर्थनिरीह\leavevmode\marginnote{\textenglish{4a/ms}}स्य सत्यपि पुरुषार्थसिद्धिहेतुत्वे अप्रवर्त्तनात् । स्मरणादभिलाषेण व्यवहारः प्रवर्त्तते इत्यलं शब्दमात्रसमर्थनदृष्टेरर्थतत्त्वानवगाहिनो वचनेऽन्धादरेण ।
	\pend
      

	  \pstart अथाभिधेयप्रयोजनं प्रवर्त्तकमिति समस्यते । ततोऽयमदोष इति चेत् । तदवद्यम् । न हि पुरुषाथसिद्धिहेतुत्वेन प्रयुक्तः पुरुषः सम्यग्ज्ञानलक्षणे अभिधेये प्रवर्त्तते किन्तु प्रकरणे प्रवर्त्तते--ग्रन्थश्रवणलक्षणां प्रवृत्तिमनुतिष्ठति ।
	\pend
      

	  \pstart अथ ग्रन्थस्य शब्दार्थस्वभावत्वाद् । एवमप्यसौ शब्देऽभिधेये च समुदाये प्रवृत्तो भवति । तद् ग्रन्थस्य प्रकरणस्येति षष्ठी कथम् ? अथ पुनरयमर्थोऽस्य प्रकरणस्याभिधेयेऽर्थादभिधानाभिधेयसमुदाये प्रयोजनं प्रवर्त्तकमिति । तथा च प्रकरणस्य प्रकरणे प्रवर्त्तकमित्यसङ्गतमुक्तं स्यात् । यदि चोक्तया व्युत्पत्त्या प्रयोजनशब्देन प्रयोजकं तथाभूतमस्य बुबोधयिषितं भवेत् तदा \textbf{सम्यग्ज्ञानपूर्विके}त्यादिना अभिधेयप्रयोजनमुच्यत इत्युक्तं स्यात् । मुख्यश्च सम्बन्धी पुरुषः पुरुषस्येति दर्शितः स्यात् । न चैवम्, तस्मात् प्रयोजनशब्देन फलमेवास्याभिप्रेतमत्रोत्तरत्रापि ।
	\pend
      

	  \pstart अथोच्यते । फलार्थी चेत् प्रतिपत्ता फल एव किं न प्रवर्त्तते ? कि श्रमः सम्यग्ज्ञान इति । हेयोपादेययोर्हानोपादानलक्षणफलार्थितैव तन्निबन्धनं ज्ञानं मृगयते । अन्यथा विसंवादनमात्मीयमाशङ्कमान इति का क्षतिः ? फलपक्षे तु पुरुषप्रवृत्त्युपयोगि चास्याभिधेयमिति तद् दर्शितं भवति ।
	\pend
      \leavevmode\marginnote{\textenglish{7/dm}}“

	  \pstart द्विविधं हि प्रकरणशरीरम्--शब्दः, अर्थश्च\footnote{श्चेति--\cite{dp-edE} \cite{dp-edH} \cite{dp-edN} \cite{dp-edP}} ।
	\pend
       

	  \pstart तत्र शब्दस्य\footnote{शब्दस्याभिधे \cite{dp-msB} \cite{dp-msC} \cite{dp-msD}} स्वाभिधेयप्रतिपादनमेव प्रयोजनम् । नान्यत् । अतस्तन्न\footnote{तन्निरूप्यते A} निरूप्यते ।
	\pend
       

	  \pstart \footnote{अथाभिधेयस्यापि किं प्रयोजनाभिधानेनेत्याह--\cite{dp-msD-n}}अभिधेयं\footnote{अभिधेये तु निष्प्रयोजने तत्प्रति--\cite{dp-msB} \cite{dp-msD}} तु यदि निष्प्रयोजनं\footnote{प्रयोजनं तदा--C} स्यात् तदा \footnote{अभिधेयप्रतिपत्तये--\cite{dp-msD-n}}तत्प्रतिपत्तये शब्दसन्दर्भोऽपि\footnote{अपिशब्दादर्थसन्दर्भे[[र्भो]]ऽपि--\cite{dp-msD-n}} नारम्भणीयः स्यात् ।
	\pend
      ”

	  \pstart ननु च नायं प्रत्यस्तमितावयवार्थः संज्ञाशब्दः । किं तर्हि ? प्रयुज्यतेऽनेन इति, प्रयोजयतीति वा व्युत्पत्त्या फलेऽपि वर्त्तते । तत् कथमेतद् व्याख्यायत इति चेत् । सत्यमेतत् । केवलं न पुरुषार्थसिद्धिहेतुत्वं सम्यग्ज्ञानस्यात्मन एव प्रयोजकं किन्तु पुरुषस्य । तत्र चोक्ता दोषमात्रा । अभिधेये प्रयोजनमिति विग्रहे च भूयान् दोषो दर्शितः । तस्मात् तथाभूतव्युत्पत्तिनापि प्रयोजनशब्देनात्र न तथाभूतं प्रयोजकं वाच्यम्, अपि तु फलमेवेति सर्वमवदातम् ।
	\pend
      

	  \pstart ननु च प्रकरणे श्रोतृन् प्रविवर्त्तयिषुरयमभिधेयप्रयोजनमभिधत्ते तदनेनास्यैव तदाख्यातुमुचितम् । तच्च यथास्वमभिधेयप्रत्यायनलक्षणमित्याशङ्क्याह--\textbf{द्विविधं हि} इत्यादि । हिर्यस्माद् द्विप्रकारं प्रकरणस्य शरीरं स्वभावः । तदुक्तं \textbf{काव्यालङ्कारे}--“शब्दार्थौ सहितौ काव्यम्”\footnote{काव्यालङ्कारवृत्तौ--“काव्यशब्दोऽयं गुणालङ्कारसंस्कृतयोः शब्दार्थयोर्वर्त्तते” \href{http://http://sarit.indology.info/?cref=kāv.1.1}{ १. १.} इति ।} \href{http://http://sarit.indology.info/?cref=kā.1.16}{का० १. १६} इति । तस्मादभिधेयप्रयोजनमुच्यत इति ।
	\pend
      

	  \pstart कथं द्वैविध्यमित्याह--\textbf{शब्द} इति । \textbf{चः} शब्देन सहार्थं प्रकरणशरीरत्वेन समुच्चिनोति । एतदुक्तं भवति--शब्दार्थयोरवच्छेद्यावच्छेदकत्वेन स्थितयोः प्रकरणत्वं नान्यथेत्युभयस्वभावत्वात् प्रकरणस्याभिधेयप्रयोजनाभिधाने प्रकरणस्यैवाभिहितं भवतीति ।
	\pend
      

	  \pstart यद्येवं शरीरत्वाविशेषात् अभिधेयस्येवाभिधानात्मनोपि तत् किन्नोच्यते ? किञ्च, न निष्कृष्टरूपेऽभिधेये पुरुषः प्रवर्त्तते किन्तु ग्रन्थ एव श्रवणलक्षणां प्रवृत्तिमाचरति अभिधेयज्ञानाय । तदस्यैव प्रयोजनं वाच्यमिति पूर्वपक्षद्वितयं पश्यन्नाह--\textbf{तत्रेति} निर्धा\leavevmode\marginnote{\textenglish{4b/ms}}रणे चैतत् । \textbf{नान्यदिति} पुरुषप्रवृत्त्युपयोगि ।
	\pend
      

	  \pstart अयमस्याशयः--न हि शब्दस्य स्वार्थप्रत्यायनलक्षणं फलमस्तीति शब्दसन्दर्भ आरभ्यते श्रूयते वा, तस्य काकदन्तपरीक्षासाधारणत्वेनागण्यमानत्वात् । किन्तर्हि ? तदर्थस्य सप्रयोजनत्वेन । अतः किमनेनोक्तेनापीति ? तर्हि अभिधेयस्यापि तत् किमुच्यत इत्याह—\textbf{अभिधेयं तु} इत्यादि । \textbf{तु}रभिधानादभिधेयं भेदवद्दर्शयति । \textbf{अपि}शब्दो \textbf{नारम्भणीय} इत्यस्मात्परो द्रष्टव्यः । तदयमर्थः--तत्प्रतिपत्तये शब्दानां सन्दर्भो नारम्भणीयोऽपि स्यात्, किं पुनः श्रवणीय इति ।
	\pend
      \leavevmode\marginnote{\textenglish{8/dm}}“

	  \pstart यथा काकदन्तप्रयोजनाभावात् न तत्परीक्षा आरम्भणीया प्रेक्षावता ।
	\pend
       

	  \pstart तस्मादस्य प्रकरणस्यारम्भणीयत्वं दर्शयता अधियप्रयोजनमनेनोच्यते । यस्मात् सम्यग्ज्ञानपूर्विका सर्वपुरुषार्थसिद्धिः, तस्मात् तत्प्रतिपत्तये\footnote{तत्प्रतिपत्त्यर्थमिदम् \cite{dp-msA} \cite{dp-edE} \cite{dp-edN} \cite{dp-edH} \cite{dp-edP}} इदमारभ्यत इत्ययमत्र वाक्यार्थः ।
	\pend
       

	  \pstart \footnote{“अत्र च” एवंप्रकारवाक्ये--\cite{dp-msD-n}}अत्र च प्रकरणाभिधेयस्य सम्यग्ज्ञानस्य सर्वपुरुषार्थसिद्धिहेतुत्वं प्रयोजनमुक्तम् ।
	\pend
      ”

	  \pstart \textbf{यथे}ति सामान्येनोक्तस्यार्थस्य विषयोपदर्शने । यथैतद्दर्शितं तद्वत् सर्वं द्रष्टव्यमिति \textbf{यथा}शब्दार्थः ।
	\pend
      

	  \pstart \textbf{तत्परीक्षे}तीक्षेत्थमित्थञ्चेति कर्मोपदेशः । पदसंहतिरिति च बुद्धिस्थम् । परीक्षणं वा \textbf{परीक्षा} विमृष्यावधारणं तादर्थ्यात् शास्त्रमपि तथा ।
	\pend
      

	  \pstart ननु निष्प्रयोजनाभिधेयं मारम्भि, वचनेन त्वनेन किं क्रियत इत्याह--\textbf{तस्माद्} इत्यादि । यस्मान्निष्प्रयोजनाभिधेयं नारभ्यते \textbf{तस्मादारम्भणीयत्वं दर्शयता} आरम्भयोग्यमेवेदं मयारभ्यत इति प्रकाशयता, आरम्भणीयत्वे च दर्शिते श्रवणीयत्वमपि निमित्तसाम्याद् दर्शितं भवतीति ते प्रवर्त्तिता भवन्ति ।
	\pend
      

	  \pstart ननु च \textbf{सम्यग्ज्ञानपूर्विका सर्वपुरुषार्थसिद्धि}रित्यनेनैकदेशेनाभिधेयप्रयोजनमुवतम्, न समुदितेन । तत् कथम् अनेन “वाक्येनोच्यते” इत्युच्यते ? उच्यते । वाक्यं हि नामैकस्मिन्नर्थे प्राधान्येन प्रतिपाद्ये गुणगुणिभावमनुभवतामर्थद्वारेणान्योन्यापेक्षिणां सम्बद्धानां पदानां समूह उच्यते । तत्र यदि सर्वेषां पदार्थानां प्राधान्यं स्यात् तदा परस्परानुपकारात् सम्बन्ध एव पदानां न स्यादिति वाक्यरूपतैव हीयेतेति सूक्तमनेनेति । अभिधेयप्रयोजनाभिधानमेवास्य \textbf{सम्यग्ज्ञाने}त्यादिपदसमूहात्मकस्य यथा तथा \textbf{यस्मादि}त्यादिना कण्ठोक्तं करोति । \textbf{तस्य} सम्यग्ज्ञानस्य \textbf{प्रतिपत्तये} शिष्यस्येत्यर्थात् । एतच्च \textbf{तद् व्युत्पाद्यत} इत्यस्य सामर्थ्यावस्थितार्थ कथनं द्रष्टव्यम् । \textbf{इति}ना वाक्यार्थस्य स्वरूपम्, इदमा च बुद्धिसिद्धत्वेन तदेवाङ्गुलीव्यपदेशयोग्यमिव दर्शयति । अनेन च पुरुषार्थसिद्धेरुपेयत्वात् प्राधान्यम्, प्रतिपाद्यमानस्य च सम्यग्ज्ञानस्य तदुपायत्वादप्राधान्यम् । अत एव चार्थेन न्यायेन पुरुषार्थसिद्धेर्वाक्यार्थत्वम् । शाब्द्या तु वृत्त्या शिष्यसम्यग्ज्ञानविषया व्युत्पत्तिक्रिया वाक्यार्थ इति दर्शितम् ।
	\pend
      

	  \pstart \textbf{ननु सम्यग्ज्ञाने}त्यादिना वाक्येन सम्यग्ज्ञानस्य पुरुषार्थसिद्धौ हेतुभावः परं प्रदर्शितो न पुनरिदं तस्य प्रयोजनमिति दर्शितम् । तस्यैव प्रयोजनमनेनोच्यत इति चोक्तम् । तत् कथं युज्यते ? अथ न\footnote{चै}तावतापि न ज्ञायते कस्य किं तदभिधेयं किञ्च तस्य प्रयोजनमित्याशङ्क्याह--\textbf{अत्र चे}त्यादि ।
	\pend
      

	  \pstart अथ सम्यग्ज्ञानस्य पुरुषार्थसिद्धिहेतुत्वं प्रयोजनमित्यवद्यम् । मूलविरोधाद् युक्तिविरोधाच्च । तथा हि पुरुषार्थसिद्धिमेवाचार्यीयं \textbf{सम्यग्ज्ञानेत्या}दिवचनं प्रयोजनमनुजानाति न तु पुरुषार्थसि\leavevmode\marginnote{\textenglish{5a/ms}}द्धिहेतुत्वम् । अवयवार्थव्याख्याने तस्य तामेव प्रयोजनतया व्यक्तीकरिष्यति । न च हेतुत्वं हतोर्भिन्नमिष्यते युज्यते वा । तत् कथमस्यैव सम्यग्ज्ञानस्य प्रयोजनं  \leavevmode\marginnote{\textenglish{9/dm}} “
	  
	\footnote{ननु अभिधेयप्रयोजनाभिधानेपि यावच्छास्त्रस्य सम्बन्धादीनि नोक्तानि तावत् प्रेक्षावन्तो न तत्र प्रवर्त्तन्ते इत्याह--त[[अ]]त्रेत्यादि । ननु तथाप्यस्मदुक्तस्य किमुत्तर मित्याह--अस्मिंश्चार्थे--\cite{dp-msD-n}}मस्मिंश्चार्थ उच्यमाने सम्बन्धप्रयोजनाभिधेयानि\footnote{अभिधेयार्थेऽकथितान्यपि सामर्थ्यादुक्तानि भवन्तीत्यर्थः--\cite{dp-msD-n}} उक्तानि\footnote{अभिधेयप्रत्यागते--} भन्वित\add{}” भविष्यति । न चात्र प्रयोजनं प्रयोजकं वाच्यम्, उक्तन्यायात् । इहापि प्रवृत्तिसम्बन्धिन उपादानप्रसङ्गाच्च । न च राजशासनं सामान्यं किञ्चिदस्ति येन भावप्रधानः प्रयोजनशब्दः कल्प्येत । तदिष्टौ पूर्वोक्तं दूषणं परावर्त्तेतेति ।
	\pend
      

	  \pstart सत्यम् । न पुरुषार्थसिद्धिहेतुत्वं सम्यग्ज्ञानस्य प्रयोजनमभिप्रेतम् । किन्तर्हि ? पुरुषार्थसिद्धिरेव ।
	\pend
      

	  \pstart अशाब्दिकाऽसौ कथमुच्यतामिति चेत् । उच्यते । चशब्दोऽत्र यस्मादर्थे । ततो यस्मात् प्रकरणाभिधेयस्य सम्यग्ज्ञानस्य पुरुषार्थसिद्धिहेतुत्वमुक्तम्, तस्मात् प्रयोजनमुक्तमिति द्विरावर्तनीयम् । तच्चात्रार्थात् पुरुषार्थसिद्धिरेवावतिष्ठते । सम्यग्ज्ञानस्य तां प्रति हेतुत्वोक्तौ च सा प्रयोजनमुक्तेति किं साक्षाद्वाचकेन पदेन क्रियत इति भावः । एतावतैव तावदिदं समाधीयते । बहवः पुनरत्रायास्यन्ति ।
	\pend
      

	  \pstart सर्वा पुरुषार्थसिद्धिर्यतो हेतुत्वात् तत् सर्वपुरुषार्थसिद्धि तथाविधं हेतुत्वं यत्र प्रयोजनं तत् तथा । तच्च पुरुषार्थसिद्धिरेवेति । अनया व्युत्पत्त्या अनेन शब्देन सैवोक्ता ।
	\pend
      

	  \pstart कस्य तादृशमित्यपेक्षायामिदमुक्तम्--\textbf{प्रकरणाभिधेयस्य} सम्यग्ज्ञानस्येति केचित् प्रत्तिपद्यन्ते । अन्ये त्वर्ष \footnote{र्श} \footnote{पाणिनि \cite{dp-msD-n} ५. २. १२७}आदित्वे प्रयोजनमस्यास्तीति मत्वर्थीयमतं विधाय प्रधानवन्निर्देशं विवक्षित्वा प्रयोजनवत्त्वमिति प्रतिजानते । एवं चार्थं समर्थयन्ति--यत एव पुरुषार्थसिद्धिः सम्यग्ज्ञानसाध्या तत एव सा प्रयोजनम् । तस्यापि यदेव तां प्रति हेतुत्वं तदेव प्रयोजनवत्त्वमिति ।
	\pend
      

	  \pstart एके तु यतः सम्यग्ज्ञाने सति पुरुषार्थसिद्धिहेतुत्वं व्यवस्थाप्यते, ततः प्रमाण-फलवद् व्यवस्थाप्य-व्यवस्थापनभावमाश्रित्येदमुक्तं ततो न किञ्चिदवद्यमिति मन्यन्ते ।
	\pend
      

	  \pstart अपरे तु \textbf{च}शब्दं भिन्नक्रमं कृत्वा पुरुषार्थसिद्धिहेतुत्वमुक्तम्, प्रयोजनं चोक्तमिति योजयन्ति ।
	\pend
      

	  \pstart इतरे तु पुरुषार्थसिद्धिहेतुशब्देन कारणे कार्यमुपचर्य पुरुषार्थसिद्धिमेवाभिदधति । अनन्योपायसाध्यतादर्शनार्थं चोपचारकरणं समादधते ।
	\pend
      

	  \pstart केचित्तु निबन्धकृतः कथञ्चिदपीदं समर्थयितुमनीशानाः सहितशब्दपातात् प्रमादपाठ एवेति वर्णयन्ति ।
	\pend
      

	  \pstart अत्र सारासारं सन्त एव विवेचयिष्यन्ति ।
	\pend
      

	  \pstart स्यादेतत् । यथा प्रवृत्त्यङ्गतया अभिधेयप्रयोजनमुच्यते, तद्वत् सम्बन्धादिकमपि किं नोच्यत इत्याह--\textbf{अस्मिंश्चेति । चो} यस्मादर्थे ।  \leavevmode\marginnote{\textenglish{10/dm}} “
	  
	\footnote{सर्वपुरुषार्थसिद्धिहेतुः सम्यग्ज्ञानम् । अस्मिन्नर्थे उच्यमाने कथं सम्बन्धादीन्युक्तानि-\cite{dp-msD-n}}तथाहि--पुरुषार्थोपयोगि सम्यग्ज्ञानं व्युत्पादयितव्यमनेन प्रकरणेनेति ब्रुवता सम्यग्ज्ञानमस्य शब्दसन्दर्भस्याभिधेयम्, तद्व्युत्पादनं प्रयोजनम्, प्रकरणं चेदमपायो व्युत्पादनस्येत्युक्तं भवति । 
	  
	तस्मादभिधेयभागप्र योजनाभिधानसामर्थ्थात् सम्बन्धादीनि उक्तानि भवन्ति । \footnote{नन्वादिवाक्ये यथाऽभिधेयप्रयोजनमभिधत्ते एवं सम्बन्धादि किमिति न वक्ति ? इक्याह--\cite{dp-msD-n}}न त्विदमेकं धाक्यं सम्बन्धम्, अभिधेयम्, प्रयोजनं च वक्तुं साक्षात् समर्थम् ।\footnote{यदि साक्षान्न वक्ति कथं तर्हि समर्थं तद्दर्शने इत्याह--\cite{dp-msD-n}} एकं तु वदत् त्रयं सामर्थ्यात् दर्शयति । \footnote{\textbf{अथ वार्त्तिकेन} सामर्थ्यलब्धमभिधेयं प्रयोजनं चाह--\cite{dp-msD-n}}तत्र--” कथं पुनरन्यस्योक्तावन्यदुक्तं भवतीत्याह--\textbf{तथा हि} इति । निपातसमुदायश्चायं यस्मादित्यस्यार्थे सर्वत्र वर्त्तते ।
	\pend
      

	  \pstart \textbf{पुरुषार्थो} हेयोपादेयहानोपादानलक्षण \textbf{उपयो}गो व्यापारः । सोऽस्यास्तीति तथा ।
	\pend
      

	  \pstart एवं सति किं सिद्धमित्याह--\textbf{तस्माद्} इति । \textbf{तच्छ}ब्देनानन्तरोक्तो वाक्यार्थः प्रत्यवद्र\footnote{म्र}ष्टव्यः ।
	\pend
      

	  \pstart \textbf{अभिधेयभागोऽङ्ग} एकदेशः । सम्बन्धादिभागापेक्षया ।
	\pend
      

	  \pstart \textbf{सम्बन्धादी}त्यादिग्रहणेनाभिधेयप्रयोजनयोः संग्रहः । \leavevmode\marginnote{\textenglish{5b/ms}} \textbf{उक्तानीति} उक्तानीव \textbf{उक्तानि प्र}काशितानि । न तु तत्राऽभिधाऽस्य सम्भविनी ।
	\pend
      

	  \pstart अथाभिधेयप्रयोजनं यथा वाक्येन साक्षादुच्यते, तथा सम्बन्धादीन्यपि किन्नोच्यन्त इत्याह--\textbf{न त्विति} । तुरवधारयति विशिनष्टि वा ।
	\pend
      

	  \pstart अयमाशयः--यदि सर्वेपदार्थाः प्राधान्यमश्नुवीरन्, वाक्यमेव तदा विशीर्येतान्योन्याऽव्यपेक्षाभावेनैकार्थाप्रत्यायनात् । ततो वाक्यमेकार्थमभिधेयतया उपादातुं कल्प्यते नानेकम् ।
	\pend
      

	  \pstart यदि साक्षान्न समर्थम्, कथं नासमर्थम् ? अथासमर्थमेव । न साक्षात् समर्थमिति तर्हि न वाच्यमिति । आह--\textbf{एकं तु}--इति । \textbf{तु}रत्रापिवद् ग्राह्यः । साक्षाद्वाच्यत्वेन एकं व्यवच्छिनत्ति । \textbf{दर्शयति} प्रकाशयति, तत्र वाक्यस्याभिधाव्यापारासंभवात् । अत एवोक्तानीवोक्तानीति तथाऽस्माभिर्व्याख्यातम्, अन्यथा तदनेन विरुध्येत ।
	\pend
      

	  \pstart ननु यदि नामाभिधेयादिकं वाक्यस्य प्रकाश्यम्, नाभिधेयम् तथापि पदार्थेन तेनावश्यं भाव्यम् । तथा च कस्य किं वाचकं पदमित्याशङ् क्याह--\textbf{तत्र} इत्यादि--तेषु अभिधेयादिषु ।  \leavevmode\marginnote{\textenglish{11/dm}} “
	  
	तद् इति अभिधेयपदम् । \footnote{व्युत्पाद्यत इति णिच्निर्द्देशात् प्रयोक्तृप्रयोज्यविषयं प्रयोजनं वाच्यमित्याह--\cite{dp-msD-n}}व्युत्पाद्यत इति प्रयोजनपदम्\footnote{प्रयोजनमिदम्--\cite{dp-msA} \cite{dp-edP} \cite{dp-edH}} । प्रयोजनं चात्र वक्तुः\footnote{शास्त्रकर्त्तुः--\cite{dp-msD-n}} प्रकरणकरणव्यापारस्य चिन्त्यते, श्रोतुश्च श्रवणव्यापारस्य । 
	  
	तथा हि--सर्वे प्रेक्षावन्तः प्रवृत्तिप्रयोजनमन्विष्य प्रवर्त्तन्ते । ततश्चाचार्येण\footnote{तत आचा० \cite{dp-msA} \cite{dp-edP}} प्रकरणं किमर्थं कृतम्, श्रोतृभिश्च किमर्थं श्रूयत इति संशये\footnote{संशयव्यु० \cite{dp-edP} \cite{dp-edH}} व्युत्पादनं प्रयोजनमभिधीयते । सम्यग्ज्ञानं \footnote{व्युत्पाद्यमा० \cite{dp-msB} \cite{dp-edE} \cite{dp-edH} \cite{dp-edN} विबुध्यमानानाम्--\cite{dp-msD-n}}व्युत्पद्यमानानामात्मानं व्युत्पादकं कर्त्तुं प्रकरणमिदं कृतम् । शिष्यैश्चाचार्यप्रयुक्तामात्मनो व्युत्पत्तिमिच्छदिभः प्रकरणमिदं\footnote{प्रकरणं श्रूय० \cite{dp-msC} \cite{dp-msD} \cite{dp-msB}} श्रूयत इति प्रकरणकरण-श्रवणयोः प्रयोजनं\footnote{प्रयोजनव्यु० \cite{dp-edH} \cite{dp-edN}} व्युत्पादनम् ।” तदिति द्वितीयान्तमेतत् । यतो \textbf{व्युत्पाद्यत} इत्यत्र लकारः प्रयोज्यकर्मणि विहितो न प्रयोज्यक्रियाकर्मणीति । वाक्येऽपि लोके पद्यते गम्यतेऽनेनाऽर्थ इति व्युत्पत्त्या पदप्रयोगो दृश्यते । ततो \textbf{व्युत्पाद्यत इति प्रयोजनपद}मित्याह । अन्यथा पदवृन्दमिदं न पदम् । यद्वा \textbf{व्युत्पाद्यत} इत्यत्रेति द्रष्टव्यम् ।
	\pend
      

	  \pstart ननूक्तम्--“प्रयोजनं पुरुषार्थसिद्धिः” । तत् किं पुनश्चिन्त्यते । कथञ्च तदन्यथोच्यत इत्याह--\textbf{प्रयोजनं चात्र} इति । \textbf{चो} वक्तव्यान्तरसमुच्चये । तदयमर्थः--नोक्तं प्रयोजनमिति चिन्त्यते किन्तु गुणभूतार्थप्रयोजकपदस्य व्युत्पाद्यत इत्यस्य प्रयोजनम् । तद्व्युत्पत्तिः क्रियत इति णिचः किं प्रयोजनमिति प्रश्ने तस्य प्रयोजनं चिन्त्यत इति यावत् ।
	\pend
      

	  \pstart \textbf{अत्र}--इति गुणभूतार्थनिरूपणे प्रकरणकरणमेव व्यापारः ।
	\pend
      

	  \pstart ननु कस्मादियमाशङ्का--“वक्त्रा किमर्थं क्रियते, श्रोतृभिश्च किमर्थ श्रूयते” इति । तौ खल्वेवमेव प्रवर्त्तेयातामित्याशङ्क्याह--तथा हि इति यस्मात् ।
	\pend
      

	  \pstart अथ कथमुभयोर्व्युत्पादनं प्रयोजनमुच्यते ? यद्यतो व्युत्पद्यते तस्य व्युत्पत्तिः प्रयोजनम् । यस्तु व्युत्पादयति तस्य व्युत्पादनमिति । नैतदस्ति । प्रयोजकव्यापारवती \textbf{हि} व्युत्पत्तिः प्रयोजनमिष्टेत्युभयोर्व्युत्पादनमेव प्रयोजनम् । केवलमेकोऽन्यथा प्रवर्त्ततेऽन्यश्चान्यथा । कथं नाम मद्व्यापारवशेन श्रोतृसन्तानवर्त्तिनी व्युत्पत्तिर्भूयादिति आचार्यः प्रवर्त्तते । अत एव स्वप्रयोजकव्यापाराधिष्ठिता श्रोतृसन्तानवर्त्तिनी व्युत्पत्तिः कर्त्तुः प्रयोजनम् । एवमेवासौ व्युत्पादको भवति । श्रोता तु कथं नामैतदाचार्यव्यापारवसे\footnote{शे}न मत्संतानवर्तिनी व्युत्पत्तिर्भूयादिति प्रवर्तते ।
	\pend
      

	  \pstart अत एवाचार्यव्यापाराधिष्ठिता व्युत्पत्तिः श्रोतुरपि प्रयोजनम् । एवमेवासौ तद्व्युत्पाद्यो  \leavevmode\marginnote{\textenglish{12/dm}} “
	  
	\footnote{ननु यथाऽभिधेयप्रयोजने दर्शिते एवं सम्बन्धादि दर्श्यतामित्याह--\cite{dp-msD-n}}सम्बन्धप्रदर्शनपदं तु न विद्यते । सामर्थ्यादेव तु स प्रतिपत्तव्यः । 
	  
	प्रेक्षावता हि सम्यग्ज्ञानवव्युत्पादनाय प्रकरणमिदमारब्धवता “अयमेवोपायो नान्यः” इति दर्शित एवोपायोपेयभावः प्रकरणप्रयोजनयोः सम्बन्ध इति ।” भवति । ततो व्युत्पादनं प्रयोजकव्यापारवशवर्त्तिनी व्युत्पत्तिर्द्वयोरपि प्रयोजनमिति इतिशब्दं हेतुपदं कृत्वोपसंहरन्नाह--\textbf{प्रकरणकरणश्रवणयोः प्रयोजनं व्युत्पादनमिति ।}
	\pend
      

	  \pstart ननु च णिचि कृते व्युत्पादनं कथमुभयोः प्रयोजनम्, इत्थमिति \leavevmode\marginnote{\textenglish{6a/ms}} प्रश्नविसर्जने स्याताम् । तेनैव तावता दर्शितेन किं प्रयोजनम् ? सम्यग्ज्ञानव्युत्पत्तौ कथञ्चिन्निमित्तमात्रात् दधिभोजनादेराचार्यस्यासाधारणकारणताप्रतिपत्तिः प्रयोजनम् । अयमर्थः--अस्ति प्रयोजकव्यापारप्रदर्शने सम्यग्ज्ञानव्युत्पत्तावुपयोगित्वमात्रेण स्वास्थ्यादिना साम्यमाचार्यस्य दर्शितं स्यात् । अस्ति चास्य तद्व्युत्पत्तावुपदेशलक्षणोऽसाधारणो व्यापारः । स कथं नाम प्रतीयेतेति णिचा निर्देशः कृत इति ।
	\pend
      

	  \pstart अथ भोजनादेस्तद्व्युत्पत्तौ साक्षाद्व्यापाराऽसम्भवाद् आचार्यव्यापारः प्रतिपत्स्यत इति चेत् । नैतत् । एवं हि व्याख्यातृणामिदं प्रतिपादनकौशलं स्यान्न कर्त्तुरिति न्याय्यो णिचा निर्देशः ।
	\pend
      

	  \pstart ननूक्तम्--अभिधानस्य प्रयोजनं न निरूप्यत इति । तत् किमिदानीं तदेव व्याजान्तरेण निरूप्यते ? सत्यम् । केवलं वाक्यार्थत्वेन न निरूप्यत इत्यभिसन्धिना “तन्न निरूप्यते” इत्युक्तम् । न तु पदार्थत्वेनापीति किं विरोधः ?
	\pend
      

	  \pstart सम्यग्ज्ञानं व्युत्पद्यमानानाम् इत्यसाधुरयं शब्दः--“व्युत्पत्त्यर्थस्य पदेरकर्मत्वाद्” इति \textbf{भागवृत्तिकारः} । तन्नातिश्लिष्टम् ज्ञानार्थस्य पदेः सकर्मकस्य “व्युत्पन्नः सङ्केतः” इत्यादौ बहुशः प्रयोगदर्शनात् । यथाऽवादि \textbf{प्रमाणवार्त्तिके वार्त्तिककृता}--
	\pend
      

	  \pstart “मनोऽव्यु\footnote{मनो व्यु}त्पन्नसंकेतमस्ति तेन स चेन्मतः” \href{http://http://sarit.indology.info/?cref=pv.2.143}{प्रमाणवा० २. १४३} इति ।
	\pend
      

	  \pstart यद्यभिधेयादयः पदार्थाः, सम्बन्धस्तर्हि कस्य पदस्यार्थ इत्याह--\textbf{सम्बन्धे}त्यादि ।
	\pend
      

	  \pstart \textbf{सामर्थ्याद्} इति नाभिधाव्यापारेण । \textbf{एव}कारेण पदस्याप्यभिधामपोहति । \textbf{तु}शब्दः केवलमित्यस्यार्थे ।
	\pend
      

	  \pstart कथं सामर्थ्यादित्याह--\textbf{प्रेक्षावता} इति । \textbf{उपायः} कारणम् । \textbf{उपेया} सम्यग्ज्ञानव्युत्पत्तिः साध्या । तयो\textbf{र्भावः} । यद्वशेनोपायोपेयज्ञानाभिधाने भवतः । \textbf{एव}कार \textbf{उपाय}शब्दात् परो द्रष्टव्यः । प्रतियोगिनोपायायोगस्यैव शङ्कितत्वात् । तदन्ययोगस्य च शङ्किष्यमाणत्वात् । अनर्थाशङ्कोपस्थापितान्ययोगनिरासे तु कर्त्तव्ये सामर्थ्यगम्यमवधारणान्तरं कार्यम् । न त्विदमेव वाच्यम् । नान्य इति च नान्य एवेत्यवसेयम् ।
	\pend
      \leavevmode\marginnote{\textenglish{13/dm}}“

	  \pstart ननु च प्रकरणश्रवणात् प्राग् उक्तान्यपि\footnote{उक्तान्यभिधे० \cite{dp-msB}} अभिधेयादीनि प्रमाणाभावात् प्रेक्षावद्भिर्न गृह्यन्ते । तत् किमेतैरारम्भप्रदेशे उक्तैः ?
	\pend
       

	  \pstart सत्यम् । अश्रुते प्रकरणे कथितान्यपि न निश्चीयन्ते । उक्तेषु त्वप्रमाणकेष्वप्यभिधेयादिषु संशय उत्पद्यते । संशयाच्च प्रवर्त्तन्ते । अर्थसंशयोऽपि हि प्रवृत्त्यङ्गं प्रेक्षावताम् ।
	\pend
      ”

	  \pstart एवंरूपस्य सम्बन्धो व्युत्पाद्यत इति प्रयोजनाभिधानादेव दर्शितम् । तदुक्तम्--
	\pend
      “
	    
	    \stanza[\smallbreak]
	शास्त्रं प्रयोजनं चैव सम्बन्धस्याश्रयाद्गतौ ।\footnote{श्रयावुभौ--श्लोकवा०}&तदुक्त्यन्तर्गतस्तस्माभिन्नो नोक्तः प्रयोजनाद् ॥ \href{http://http://sarit.indology.info/?cref=śv.1.18}{श्लोकवा० १. १८} इति\&[\smallbreak]


	”

	  \pstart अभिधेयादिप्रकाशनद्वारेणादिवाक्यं प्रवर्त्तकमिति असहमान आह--ननु च इति । निपातसमुदायश्चायं “चोदयामि”, अभिमुखो भव इत्यस्यार्थस्य द्योतकः । श्रवणानन्तरं तेषां प्रतीयमानतया प्रमाणाभावासिद्धेः \textbf{श्रवणात्प्राग्} इत्याह । अयं पूर्वपक्षवादिनोऽभिप्रायः--श्रवणात् प्राक् शब्दाश्रयेण प्रवर्त्तमानमनुमानमन्यद्वा शब्दस्यार्थनान्तरीयतायां प्रवर्त्तयितुमर्हति । असम्बन्धनिबन्धनायाः परोक्षार्थप्रतिपत्तेः प्रामाण्यायोगात् । सा च शब्दस्यासम्भविनी । आचार्यस्य चाप्तभावो दुर्बोधो येनाप्तोपदेशतया शब्दोऽर्थतथात्वं प्रत्याययेदिति । आरभ्यत \textbf{इत्यारम्भः} प्रकरणम् । तस्य \textbf{प्रदेशः} एकदेशः । सामर्थ्यात् तदादिः ।
	\pend
      

	  \pstart सच्चोद्ये \textbf{सत्यम्} इत्याह । यदि सत्यमिदं किमर्थं तर्हि \leavevmode\marginnote{\textenglish{6b/ms}} ते कथ्यन्त इति ? \textbf{उक्तेषु} इति । तुरनुक्तपक्षादुक्तपक्षस्य विशेषं दर्शयति । अनेनैतदाह--साधकबाधकप्रमाणाभावे येनर्ते ब्याहारात् प्रतिनियताधिकरणो यौक्तः संशयो भवितुमर्हतीति । \textbf{अपि} न्यायतः सम्भावनां दर्शयति, अतिशयं वा । अन्यथा किं सप्रमाणकेष्वपि संशयो जायते येनापिश्रुतिः संगच्छेत् ।
	\pend
      

	  \pstart ननु प्रवृत्तिफलमादिवाक्यं, प्रवृत्तिश्चेन्नास्ति किं संशयेनोत्पादितेनापीत्याह--\textbf{संशयाच्च} इति । \textbf{चो} यस्मादर्थे ।
	\pend
      

	  \pstart अथ ये तावन्मदाः श्रद्धया वाऽऽचार्यमुपसन्नास्ते तद्वचनादर्थं निश्चित्यैव प्रवर्त्तमाना न संशयात्प्रवर्त्तन्ते । येऽपि तद्विपरीताः संशेरते \textbf{ते}षामपि न संशयात्प्रवृत्तिः । प्रवृत्तौ वा प्रेक्षावत्त्वहानिरित्याशङ्क्याह--\textbf{अर्थसंशयोपीति} । न केवलमर्थनिश्चय इत्यपिशब्देनाह । शक्यनिश्चये हि निश्चयमन्तरेण प्रवर्त्तमानाः प्रेक्षावत्ताया हीयेरन् । यत्र त्वर्थसंशयेनार्थितया प्रवर्त्तमानाः, नोपालम्भमर्हन्तीति भावः । यदि च युक्ता द्वितीयाकारानुप्रवेशेन संशयानस्य न क्वचित्प्रवृत्तिः, तर्हि न कस्यचिद् वचनात् क्वचित् प्रवर्त्तितव्यमिति भूयान् व्यवहारो विलुप्येत ।
	\pend
      

	  \pstart ननु निवृत्तिरनर्थनिश्चयनिबन्धना । अनर्थनिश्चयश्च प्रमाणात् । तच्चेह नास्ति । निवृत्तीतरस्तु प्रवृत्तिव्यातिरेकी प्रकारो नास्ति । ततोऽथत्नसिद्धा प्रवृत्तिः । तत्किं तदर्थेनादि \leavevmode\marginnote{\textenglish{14/dm}} “
	  
	अनर्थं संशयोऽपि\footnote{संशयो निवृ० \cite{dp-msA} \cite{dp-edP} \cite{dp-edH} \cite{dp-edE} \cite{dp-edN}} निवृत्त्यङ्गम् । अत एव शास्त्रकारेणैव\footnote{अत्रवशब्दो व्याख्यातृणामभिधेयादिप्रकाशने कदाचित् क्रीडाद्यर्थमपि प्रवृत्तिर्भवतीति--\cite{dp-msD-n}} पूर्व सम्बन्धादीनि युज्यन्ते वक्तुम् । 
	  
	\footnote{आख्यातृणां हि \cite{dp-edE} \cite{dp-edN} आख्यातृणां टीकाकाराणां हि \cite{dp-msB}}व्याख्यातृणां हि वचनं\footnote{क्रीडार्थं \cite{dp-msA} \cite{dp-edP} \cite{dp-edE}} क्रीडाद्यर्थमन्यथापि सम्भाव्यते । शास्त्रकृतां तु प्रकरणप्रारम्भे न विपरीताभिधेयाद्यभिधाने प्रयोजनमुत्पश्यामो नापि प्रवृत्तिम् । अतस्तेषु संशयो युक्तः । अनुक्तेषु तु प्रतिपत्तृभिर्निष्प्रयोजनमभिधेयं संभाव्येताऽस्य प्रकरणस्य काकदन्तपरीक्षाया इव\footnote{इव १--इत्यादिरूपेण अनर्थसम्भावनाया अनन्तरं संख्याङ्का निर्दिष्टाः B प्रतौ D प्रतौ च--सं०}, अशक्यानुष्ठानं वा ज्वरहरतक्षकचूडारत्नालङ्कारोपदेशवद्, अनभिमतं वा प्रयोजनं मातृविवाहक्रमोपदेशवद्, अतो वा प्रकरणात् लघुतर उपायः प्रयोजनस्य, अनुपाय एव वा प्रकरणं सम्भाव्येत ।” वाक्येनेत्याह--\textbf{अनर्थे}त्यादि । न केवलमनर्थनिश्चयोऽपीत्यपि शब्दः ।
	\pend
      

	  \pstart यद्यवश्यं वक्तव्यान्यभिघेयादीनि तर्हि व्याख्यातार एव तानि वक्ष्यन्ति । तत्किं शास्त्रकृतो व्याहारेणेत्याह--\textbf{शास्त्रकारेणैव} न व्याख्यातृभिः ।
	\pend
      

	  \pstart अथोच्यते--प्रवृत्त्यङ्गत्वाच्च व्याचक्षते । अभिधेयाद्यनभिधाने तु प्रवृत्तेरेवासम्भवात्कथं सम्भविनो व्याख्यातारो येनाशङ्क्य तेषां व्याहारमयं निरस्यतीति । अत्रोच्यते । न सर्वथाऽभिधेयादिप्रकाशनं केनचिन्न कर्त्तव्यमेवेति चोद्यं प्रवृत्तम्, किन्त्वादिवाक्यं तदर्थं न कर्त्तव्यमित्यभिसन्धिना । तत्र ये तावत् साक्षादाचार्योपसन्नारतेषां तद्वचनादेव तत्प्रतीतौ प्रवृत्तिज्ञानयोः सम्भवात् सम्भवि व्याख्यातृत्वम् । तेभ्योऽपि तदुपसन्नानामिति किमवद्यम् ?
	\pend
      

	  \pstart ननु च न शास्त्रकृद्वचनमपि प्रसह्य प्रवर्त्तयति, किं तर्हि प्रयोजनाद्यभिधानेन प्रवृत्तिविषयोपदर्शनात् । तच्च व्याख्यातृवचनेऽपि सम्भवतीति किमुच्यते शास्त्रकारेणैवेत्यत आह--\textbf{व्याख्यातॄणां हि} इति । हिर्यस्मादर्थे । शास्त्रकारेष्वपीदं समानमित्याह--\textbf{शास्त्रकृता}मिति । तुना व्याख्यातृभ्यः शास्त्रकृतां वैधर्म्यमाह । \textbf{विपरीतं} च तद\textbf{भिधेयाद्यभिधानं} चेति विग्रहः । अथवा \textbf{विपरीतं} च तदसत्यत्वादभिधेयञ्चाभिधेयतया प्रकाशनात् । तस्या\textbf{भिधान}मिति विग्रहीतव्यम् । \textbf{उत्पश्याम} उत्प्रेक्षामहे । अनेनैतदाह--तेषामपि तथाभिधाने किमस्माकं बाधकं प्रमाणम् ? केवलमेषां महीयसाऽऽशयेन शास्त्रं प्रणेतुमिच्छतां परार्थप्रवृत्तानां नैतत्सम्भावयाम इति ।
	\pend
      

	  \pstart प्रकरणस्य \textbf{प्रारम्भ} आदौ । एतच्च प्रयोजनाद्यभिधाननियतसन्निधेर्यथाभूतार्थचिन्ताविधानविषयस्य कालस्य निर्देशो न तु प्रकरणस्य \leavevmode\marginnote{\textenglish{7a/ms}} मध्येऽवसाने वा तत् सम्भवति सप्रयोजनं चेति ।
	\pend
      [[उपायोऽस्ति प्रयो० C]]\leavevmode\marginnote{\textenglish{15/dm}}“

	  \pstart एतासु\footnote{भवतु नामैतावद्द्वषणसंभावना को दोष इति चेदाह--\cite{dp-msD-n}} चानर्थसम्भावनास्वेकस्यामप्यनर्थसम्भावनायां न प्रेक्षावन्तः प्रवर्तन्ते ।
	\pend
      ”

	  \pstart अथयद्येवंविधा प्रवृत्तिः शास्त्रकृतां दृश्यते त्वयाऽप्यस्य किं न सम्भाव्यत इत्याह—\textbf{नापि} इति । \textbf{अपि}रप्रयोजनापेक्षया समं प्रवृत्तिदर्शनं समुच्चिनोति । तदेवोत्\textbf{पश्याम} इति सम्बध्यमानमिह पश्याम इत्यास्यार्थे सम्बद्ध\footnote{न्ध}व्यम् । पश्याम इति वाऽध्याहार्यम् । अनेनैतदाह--तथापि सम्भावयामो यदि तेषामीदृशी पश्यामो न त्वेवमिति । \textbf{अतो} हेतोस्तेष्वभिधेयादिषु यद्वा \textbf{तेषु} शास्त्रकारेषु वक्तृषु सत्सु तदुक्तेष्वभिधेयादिष्विति प्रकरणात्संशयोऽर्थोन्मुख इति द्रष्टव्यम् ।
	\pend
      

	  \pstart ननु सत्यप्यर्थसंशयेऽर्थित्वाभावे प्रवृत्तिर्नास्ति । तन्निमित्ता तु सा संशयमात्रादपि भवति । तन्मात्रञ्च साधकबाधकविरहादप्यर्थे सिद्धम् । तत् किं तदुत्पादार्थेन वाक्येनेत्याह--\textbf{अनुक्तेषु}--इति । तुरुक्तावस्थाया अनुक्तावस्थां भेदवतीं दर्शयति ।
	\pend
      

	  \pstart निर्गतं \textbf{प्रयोजनं} यस्मान्निष्कान्तं वा प्रयोजनात् । काकदन्ता \textbf{ह्यभिधेयाः} । ते च न क्वचित् पुरुषार्थ उपयुज्यन्ते । सप्रयोजनं वाभिधेयमशक्यानुष्ठानं सम्भाव्येत । किंवदित्याह—\textbf{ज्व}रेत्यादि । ज्वरं हरतीति \textbf{ज्वरहरः । तक्ष}काख्यस्य नागराजस्य \textbf{चूडा} शिखा । तस्या रत्नं रतिं तनोतीति विशिष्टं वस्तु । तेनालङ्करणम् \textbf{अलङ्कारः} । ज्वरहरश्चासावेवंभूतश्चेति विग्रहः । स उपदिश्यते येन ग्रन्थेन तद्वत् ।
	\pend
      

	  \pstart सतोर्वा सप्रयोजनशक्यानुष्ठानत्वयोरस्य प्रयोजन\textbf{मनभिमतमे}वास्तिकानां \textbf{सम्भाव्येत । मातुर्विवा}हस्य \textbf{क्रमः} परिपाटिरुपदिश्यते येन \textbf{पारसीकशास्त्रेण तद्वत्} । पारसीकशास्त्रेण हि मृते पितरि माता प्रथममग्रजेन पुत्रेण परिणेतव्या, तदनु तदनुजेनेत्युपदिश्यते ।
	\pend
      

	  \pstart \textbf{अतो वा प्रकरणाल्लघुर}ल्पग्रन्थः \textbf{प्रयोजनस्य} सम्यग्ज्ञानव्युत्पत्तिलक्षणस्य । \textbf{अनुपाय एव} अनिमित्तमेव ।
	\pend
      

	  \pstart चत्वारोऽपि \textbf{वा}शब्दाः पूर्वपूर्वापेक्षया पक्षान्तरमवद्योतयन्ति । सम्भावना च सर्वा वर्णिताऽनर्थोन्मुखा प्रतिपत्तिः युक्त्या द्वितीयाकारानुप्रवेशाच्च संशयपदाभिलाप्या प्रत्येतव्या, सर्वत्रेवाकारान्तरस्य प्रतियोगिनः समुदाचारतोऽदर्शनात् । तथाविधाऽनर्थसम्भावनायाश्च बीजमिदमस्त्यादिवाक्ये-यद्यस्य प्रयोजनादिकं भवेद् बहुभिरन्यैरिवानेनाप्यादावुक्तं भवेत् । न चानेन किञ्चिदवादीति मतिः ।
	\pend
      

	  \pstart कामममूरनर्थसम्भावना भवन्तु का नो हानिरित्याह--\textbf{एतास्विति} । न केवलं सर्वा इत्यपिशब्देनाह । प्रयोजनाद्यनभिधान इव यद्यभिधानेऽपि भवन्ति तास्तदा किमभिधेयादिभिरभिहितैरपीत्याह--\textbf{अभिधे}यादिषु इति । \textbf{तुरुक्तेषु} इत्यस्यानन्तरं द्रष्टव्यः, अनुक्तपक्षाच्च विशेषस्य द्योतकः ।
	\pend
      

	  \pstart अथ कथमनयोरर्थानर्थसम्भावनयोर्विरोधो यतः सम्भावना संशय उच्यते । स चोभयां \leavevmode\marginnote{\textenglish{16/dm}} “
	  
	अभिधेयादिषु\footnote{उक्तेषु त्वाभिधेयादिष्वर्थसं०--\cite{dp-edE} अभिधेयादिष्वर्थसं० \cite{dp-edH} \cite{dp-edP} अभिधेयादिषु त्त[[तू]]क्तेष्वर्थसं० \cite{dp-msB}} तूक्तेष्वर्थसंभावना\footnote{नन्वभिधेयाभिधानेऽप्यनर्थसम्भावनया न प्रवृत्तिर्भविष्यतीत्याह--\cite{dp-msD-n}} अनर्थसम्भावनाविरुद्धा उत्पद्यते । तया\footnote{तया तु प्रेक्षा०--\cite{dp-msA} \cite{dp-edP} \cite{dp-edH} \cite{dp-edN}} प्रेक्षावन्तः प्रवर्तन्ते इति\footnote{इति तस्मादर्थे--\cite{dp-msD-n}} प्रेक्षावतां प्रवृत्तन्यङ्गमर्थसम्भावनां कर्तुं सम्बन्धादीन्यभिधीयन्त इति स्थितम् ।” शावलम्बन इत्यन्योन्यस्यामन्याकारोऽस्तीति कुतः सहानवस्थानम् । उच्यते । एकस्यामनर्थांकार उद्रिक्तोऽनुभूयते, युक्तेस्तु द्वितीयाकारानुप्रवेशः । इतस्यां नू \footnote{तू} द्रिक्तोऽ\leavevmode\marginnote{\textenglish{7b/ms}}र्भाकारोऽनुभूयते युक्तेस्त्वनर्थाकारानुप्रवेश इत्युभयोरप्यन्योन्यवैपरीत्येनानुभवाद् कथमविरोधः ।
	\pend
      

	  \pstart भवत्वेवं तथापि कथमस्याभिधानेऽनर्थाशङ्का निरस्तेति चेत् । उच्यते । पुरुषार्थसिद्धिरूपाभिधेयप्रयोजनाभिधानेनाऽऽदिमाऽनर्थसम्भावना निरस्ता । अभिधेयतत्प्रयोजनयो रूपनिरूपणेन च द्वितीयतृतीये निरस्ते । अन्तिमाऽपि सामर्थ्यगतिसम्बन्धप्रतिपादनान्निरस्ता । तुरीयां तु परमविवेकशालिन आचार्यस्य प्रकरणारम्भसामर्थ्यान्निरस्यते । तथाहि न समानेऽप्युपायान्तरे सम्भवति महतो महानयं प्रेक्षापूर्वकारी प्रकरणमीदृशमारभते, किमङ्ग पुनर्लघूनीति ।
	\pend
      

	  \pstart अथ स्यात्--समसमयसम्भवी देशान्तरवर्ती च कश्चिदेतदग्रतो लघूपायान्तरकाकः \footnote{न्तरग्रन्थः ?} स्यात् । तत्कथं तत्प्रणीतलघूपायान्तरनिरासः ? अत्रापि तत्प्रेक्षावत्तैव महती निबन्धनम् । प्रेक्षापूर्वकारित्वादेव तथाभूतात्तेनास्मिन्नार्यावर्तेऽन्यत्र वा लोके नेदानीन्तनेनैतदर्थं प्रकरणं प्रणीतमिति बहुधाऽनुसर्त्तव्यम् । अनुसृत्य दृष्ट्वा तल्लघुरुपाय इति च निश्चित्येदं प्रणेतुमुचितं नान्यथेति । सर्वत्रैव प्रकरणे सत्यादिवाक्ये लघूपायान्तरनिरासे गतिरियमेव । नहि अन्यत्राप्यभिधेयतत्प्रयोजनादिप्रकाशनमन्तरेण लघूपायान्तरनिरासाभिधानमस्तीति किं नानुमन्यते ?
	\pend
      

	  \pstart यद्येवमितरासामप्यनर्थसम्भावनानामेवमेवास्तु निरासस्तत्किमादिवाक्येन ? नैतत् । नहि सूचीप्रवेशैत्येव मूष\footnote{मूस} लप्रवेशः । तथाहि--असत्यादिवाक्ये प्रेक्षापूर्वकारिप्रयुक्तत्वमेव प्रकरणस्य न शक्यते कल्पयितुम, प्रत्युताप्रेक्षापूर्वकारिप्रयुक्तत्वमेव शक्यकल्पनम् । दृश्यन्ते हि प्रेक्षापूर्वकारिणः प्रकरणादौ सर्वत्र प्रयोजनाद्यभिधायकमादिवाक्यं प्रणयन्ते\footnote{न्तः} । न चानेनादिवाक्यं तदर्थमकारि । तस्मान्नायं प्रकरणकारः प्रेक्षापूर्वकारीतिसङ्कल्पादु\footnote{न्नो}पाददीति \footnote{त} ।
	\pend
      

	  \pstart ननु च प्रवृत्त्यर्थोऽयं प्रयासः । सा चेन्नास्ति किं तयोत्पन्नयापीत्याह \textbf{तया}--इति । यद्यर्थसम्भावना प्रकरणे पुरुषस्य प्रवर्त्तयित्री तर्हि किमभिधेयाद्यभिधीयत इत्याह--\textbf{इति}—इति । यस्मादर्थसम्भावनया प्रवर्त्तन्ते \textbf{इति}स्तस्मादर्थसम्भावनां कर्त्तुम् । किम्भूतां ? \textbf{प्रवृतेरङ्गं} निमित्तम् । अङ्गादिशब्दानामसति बहुव्रीहौ परलिङ्गाग्रहणात्स्वलिङ्गेन निर्देशः । \textbf{इति}रेवमर्थे \textbf{स्थितं} निश्चितम् ।
	\pend
      \leavevmode\marginnote{\textenglish{17/dm}}“

	  \pstart अविसंवादकं ज्ञानं सम्यग्ज्ञानम् ।
	\pend
       

	  \pstart \footnote{अथाविसंवादकमिति कः शब्दार्थ इत्याह--\cite{dp-msD-n}}लोके च पूर्वमुपदर्शितमर्थं प्रापयन् संवादक उच्यते । तद्वज्ज्ञानमपि स्वयं \footnote{०मपि प्रद० \cite{dp-msA} \cite{dp-msC} \cite{dp-msD} \cite{dp-edP} \cite{dp-edE}}प्रदर्शितमर्थं प्रापयत् संवादकमुच्यते । \footnote{ननु प्रदर्शक-प्रवर्त्तक-प्रापकाणि विभिन्नान्येव प्रमाणानि । तत् कथं स्वयं प्रदर्शितमर्थं ज्ञानं प्रापयदिति सामानाधिकरण्यमित्याशङ्क्याह ॥ यद्वा प्रदर्शक-प्रवर्तक-प्रापकाणि विभिन्नान्येव प्रमाणानि अभ्युपगम्यन्ते कैश्चिदिति तन्निराकर्तुं प्रवर्त्तक-प्रापकयोस्तावदैवयं प्रदर्शयन्नाह--\cite{dp-msD-n}}प्रदर्शिते चार्थे प्रवर्त्तकत्वमेव प्रापकत्वम्, नान्यत् । तथा हि--न
	\pend
      ”

	  \pstart एवमनेन प्रवन्धेन \textbf{सम्यग्ज्ञाने}त्यादिवाक्यस्य समुदायार्थं व्याख्यायावयवार्थमिदानीम् \textbf{अविसंवादकम्} इत्यादिना व्याचष्टे ।
	\pend
      

	  \pstart अत्रायं पूर्वपक्षः । किमिदं सम्यक्त्वं ज्ञानस्याभिप्रेतम् ? यद्योगात्सम्यग्ज्ञानमुच्यते । यद्येवं वस्तुतत्त्वग्रहणं सम्यक्त्वम्, अथापि गृहीतवस्तुप्रापणम् ? उभयथाऽपि अनुमानमवस्तुग्रहणादसम्यग्ज्ञानम् । अगृहीतप्रापणाद् वा सम्यग्ज्ञानत्वे जलज्ञानमप्युपदर्शितमरीचिकाः प्रापयतीति न किञ्चित्सम्यग्ज्ञानं न स्यादिति ।
	\pend
      

	  \pstart सिद्धान्तवाद्यप्यमीषां पक्षाणामनभ्युपगमेन निरासं मन्यमानः--अविसंवादकत्वं सम्यक्त्वं विवक्षितमिति \leavevmode\marginnote{\textenglish{8a/ms}} दर्शयति । \textbf{अविसंवा}दकं संवादकमुच्यते । विशब्दो हि संवादकप्रतिषेधे वर्त्तते । तत्प्रतिषेधस्य विधिरूपत्वात् संवादक एवावतिष्ठत इति संवादकार्थोऽविसंवादकशब्दः । तदयमर्थः--अविसंवादकं प्रवृत्तिविषयवस्तुप्रापकं सम्यग्ज्ञानमिति ।
	\pend
      

	  \pstart स्यादेतत्--वृद्धव्यवहारो हि शब्दार्थनिश्चयभूमिः । तत्र च संवादकशब्दो नोपदर्शितार्थप्रापके वर्त्तते । किं तर्हि ? सत्यवादिनि । न च ज्ञानस्य ताद्रूप्यमस्ति । तत् कुतस्तत्र संवादकशब्द इत्याह--\textbf{लोके च}--इति । \textbf{चो} यस्मादर्थे अपिशब्दार्थे वा । \textbf{लोके} व्यवहर्त्तरि जने । अयमाशयो यथा लोके सत्यवादिशब्दप्रवृत्तिनिमित्तस्योपदर्शितार्थप्रापणस्य पुरुषे सम्भवात्संवादकशब्दः प्रवर्त्तते, तथा ज्ञानेऽपि तत्सम्भवादिति ।
	\pend
      

	  \pstart अथोच्यते नोपदर्शितार्थप्रापणनिमित्तकः पुरुषे संवाद\add{क}शब्दः किन्तु प्रतिज्ञातार्थप्रापणनिमित्तकः । तत् कथमिह निमित्तसम्भव इति ? तदवद्यम् । तत्रापि प्रतिज्ञयोपदर्शनस्योप लक्षणात् । तदेव तूपदर्शनं क्वचिद् वचनेन, क्वचिदध्यवसायेना\footnote{पाठोऽत्र पश्चाद् वर्धितो दृश्यते किन्तु सूक्ष्मत्वात् न पठ्यते ।}\add{... ...}पि क्वचिद् वस्तुप्रतिभासपूर्वकेण क्वचिदन्यथा वृत्तेनेति विशेषः । संवाद\add{क}शब्दःप्र\footnote{ब्दप्र}वृत्तिनिमित्तं तु सर्वत्र समानमिति ।
	\pend
      

	  \pstart \textbf{प्रापयन्} इति \footnote{पाणिनि ३. २. १२६ ।}लक्षणहेत्वोरिति हेतौ शतुर्विधानात् प्रापणादित्यर्थः । एवमुत्तरत्राप्यवसेयम् ।
	\pend
      \leavevmode\marginnote{\textenglish{18/dm}}“

	  \pstart ज्ञानं जनयदर्थं प्रापयति, अपि त्वर्थे पुरुषं प्रवर्त्तयत् प्रापयत्यर्थम् । प्रवर्त्तकत्वमपि प्रवृत्तिविषयप्रदर्शकत्वमेव । न हि पुरुषं हठात् प्रवर्त्तयितु शक्नोति \footnote{ज्ञानम्--\cite{dp-msB} \cite{dp-edN}}विज्ञानम् ।
	\pend
      ”

	  \pstart ननूपदर्शितार्थप्रापकं संवादकमिति ब्रुवता ज्ञानस्यैकस्योपदर्शकत्वप्रापकत्वे प्रतिज्ञाते । प्रवृत्तिमन्तरेण प्राप्तेरनुपपत्तेरर्थतः प्रवर्त्तकत्वमपि । न चैकस्यैते व्यापाराः सम्भवन्ति । यतोऽन्यत् प्रदर्शति येन जानीते, अन्यत्प्रवर्त्तयति यदनन्तरं प्रवृत्तिमाचरति, अन्यच्च प्रापयति यतः प्राप्त्याभिसम्बद्ध्यते पुरुष इत्याशङ्क्याह--\textbf{प्रदर्शिते च}--इति । \textbf{चो} यस्मादर्थे ।
	\pend
      

	  \pstart प्रवर्त्तकत्वमेव प्रापकत्वं ब्रुवतोऽयमभिप्रायः--यद्यप्यर्थं साक्षात्कृत्यानुरूपं निश्चयं जनयत्प्रदर्शकं किञ्चित्, अपरं प्रदर्शयद् वा बाह्यायाः प्रवृत्तेः कारणं भवत्प्रवर्त्तकम्, इतरत्प्रवर्त्तनद्वारेण बाह्यायाः प्राप्तेर्निमित्तं भवत् प्रापकं व्यपदिश्यते, तथापि स एव बाह्यप्रवृत्तिकारणभावः प्रवर्तयितृत्वादिप्रयोजकव्यापाररूपो ज्ञानस्य पुरुषप्रेरणेनार्थजननेन च प्रवर्त्तनप्रापणयोरसम्भवेन प्रदर्शनादन्यो नोपयुज्यत इति वस्तुतः सर्वस्यैव ज्ञानस्य निश्चयानुगतस्याधिगमान्नापरौ प्रवर्त्तन-प्रापणव्यापारौ । अत एव तत्राद्यमेव ज्ञानं प्रमाणं व्यवस्थाप्यते । ततो वस्तुतः प्रदर्शकत्वादीनामभेदः, व्यावृत्तिनिबन्धनस्तु भेदोऽस्त्येव । अत एवैते प्रदर्शक-प्रवर्त्तक-प्रापकशब्दाः कृतकत्वानित्यत्वादिवन्न पर्याया इति ।
	\pend
      

	  \pstart स्यान्मतम्--किं पुनः प्रयोजनं येन प्रर्वत्तनात् नाऽपरः प्रापणव्यापारो ज्ञानस्य, अधिगमाच्च नान्यत्प्रवर्त्तन प्रयत्नेन साध्यते ? यद्युत्तरज्ञानस्य प्रामाण्यनिषेधार्थम्, तदा गृहीतग्राहितैव तन्निषेत्स्य\leavevmode\marginnote{\textenglish{8b/ms}}तीति किं तदर्थेन प्रयासेनेति ? अत्रोच्यते । यदि प्रवर्त्तयितृत्वं प्रापयितृत्वं च प्रदर्शकत्वात्परमार्थतोऽन्यत् स्यात् तदा गृहीतग्राहितैव न शक्यते प्रतिपादयितुमिति केन प्रामाण्यं निषेध्येत ? तथा हि न तज्ज्ञानं गृहीतं गृह्णाति, अपि तु गृहीते प्रवर्त्तयति । अपरं तु गृहीतं प्रापयति । तथाकारिणोश्च भिन्नोपयोगत्वात्प्रामाण्यं कथमपाक्रियते ? यदा तु प्रवर्त्तनान्नापरः प्रापणव्यापारो ज्ञानस्य प्रदर्शनाच्च नान्यत्प्रवर्त्तनम्, प्रथमेनैव च प्रत्यक्षानुमानक्षणेनार्थक्रियासमर्थो वस्तुसन्तानः प्रवृत्तिविषयीकर्त्तुं निश्चयात् शक्यते, तदोत्तरेषां तत्सन्तानभाविनामभिन्नयोगक्षेमतया प्रामाण्यमपास्यत इति ।
	\pend
      

	  \pstart एतेन तदपि प्रत्युक्तं यत् केनचिदभ्यधायि \textbf{धर्मोत्तरे}--“यद्युपदर्शकत्वमेव प्रापकत्वं तर्हि यदुक्तं उपदर्शितमर्थं प्रापयत् संवादकमिति तस्यायमर्थः स्यात्--उपदर्शितमुपदर्शयदिति । न चैतद् युक्तमर्थभेदाभावात् । तथा हि यदि परमार्थतोऽर्थाभेद उच्यते तदा न किञ्चिदवद्यम् । अथ तदापि नैवं वक्तव्यः । अत्यल्पमिदमुच्यते । कृतकत्वमनित्यत्वं प्रतिपादयतीत्यपि न वक्तव्यम्, अर्थाभेदादित्यपि किं नोच्यते ? अथ शब्दप्रवृत्त्यपेक्षयाऽर्थाभेद उच्यते तदाऽसावसिद्धो व्यावृत्तिभेदस्य दर्शितत्वात् । तेनायमर्थः--उपदर्शितं साक्षात्कृत्य जनितानुरूपनिश्चयमर्थं प्रवर्त्तनद्वारेण बाह्यायाः प्राप्तेः साक्षाद् योग्यतया वा निमित्ततां गच्छत् संवादकमिति । वास्तवस्तु प्रवर्त्तकप्रापकयोः प्रमेयाधिगतिलक्षणात्प्रदर्शनव्यापारादन्यो व्यापारो नास्तीति न प्रवर्त्तयितृत्वं प्रापयितृत्वं च प्रयोजक-व्यापारोऽनयोर्भिन्नः” इति ।
	\pend
      \leavevmode\marginnote{\textenglish{19/dm}}“

	  \pstart \footnote{यत एव प्रवर्त्तकत्वमेव प्रापकत्व प्रवर्तकत्वमपि प्रवृत्तिविषयप्रदर्शकत्वम्--\cite{dp-msD-n}}अत एव \footnote{अत एवार्था० \cite{dp-msB}}चार्थाधिगतिरेव प्रमाणफलम् । अधिगते चार्थे प्रवर्तितः पुरुषः प्रापितश्चार्थः । तथा च सत्यर्थाधिगमात् समाप्तः प्रमाणव्यापारः । \footnote{यत एव समाप्ता प्रमाणव्यापृतिः--\cite{dp-msD-n}}अत एव \footnote{अत एवानधि \cite{dp-msA} \cite{dp-edP} \cite{dp-edH} \cite{dp-edE} \cite{dp-edN}}चानधिगतविषयं प्रमाणम् । येनैव हि ज्ञानन प्रथममधिगतोऽर्थः, तेनैव प्रवर्त्तितः पुरुषः, प्रापितश्चार्थः । तत्रैव चाथ\footnote{तत्रैवार्थे \cite{dp-msA} \cite{dp-edP} \cite{dp-edE} \cite{dp-edH} \cite{dp-edN}} किमन्येन ज्ञानेनाधिकं कार्यम् ? \footnote{ततो \cite{dp-msA} \cite{dp-msB} \cite{dp-msC} \cite{dp-msD} \cite{dp-edP} \cite{dp-edH} \cite{dp-edE} \cite{dp-edN}}अतोऽधिगतविषयमप्रमाणम् ।
	\pend
      ”

	  \pstart स्यान्मतम्--यद्यधिगतिरेवोत्तरेषामपि फलं स्यात् तदोपयोगान्तराभावाद् भवेदप्रामाण्यं यावता प्रवर्त्तकस्य प्रवृत्तिः फलम्, प्रापकस्य प्राप्तिरिति फलभेदनिष्पत्तेर्भिन्नो व्यापार इत्याह—\textbf{अत एव}--इति । यतो नार्थजननद्वारेणार्थप्रापणं ज्ञानस्य, यतश्च न प्रसह्य प्रेरणेन प्रवर्त्तनम्, \textbf{अत एवा}स्मादेव कारणादर्थस्याधिगतिः परिच्छित्तिः \textbf{फलम्,} न प्रवृत्त्यादि ।
	\pend
      

	  \pstart ननु तत् प्रमाणस्य फलं व्यवस्थाप्यते, यस्मिन् सति तद्व्यापारः परिसमाप्यते । न चाधिगतावपि प्रवृत्ति-प्राप्त्योरभावे स परिसमाप्यत इत्याह--\textbf{अधिगते च}--इति । \textbf{चो} यस्मादर्थे । यस्माद् येनार्थः सम्यग्ज्ञानेन दर्शितस्तत्र तेनाप्रवर्त्तितोऽपि पुरुषः प्रवृत्तियोग्योपदर्शनात्, तद्गतस्य च व्यापारान्तरस्याभावात्\textbf{प्रवर्त्तित} इत्युच्यते । सत्यर्थित्वे प्रवर्त्तनमेव । ज्ञानेन तावत्प्रवृत्तियोग्यः कृत इति यावत् ।
	\pend
      

	  \pstart अस्तु पुरुषस्तथा प्रवर्त्तितोऽर्थस्तु न प्रापितः, तथा च व्यापारान्तरमन्याधीनमस्तीत्याह\textbf{प्रापित} इति । चः पूर्ववत् । अर्थोऽप्यसावप्राप्तोऽपि शक्यप्राप्तिको दर्शित इति \textbf{प्रापित} उच्यते । अत एव प्रापणशक्तिरेव ज्ञानस्य प्रामाण्यम् । सा च प्राप्यादर्थादात्मलाभनिमित्तेति, यतो \leavevmode\marginnote{\textenglish{9a/ms}} येन प्रवर्त्तते तदपि प्रापणयोग्यमेव । शक्तिनिश्चयस्त्वर्थक्रियानिर्भासस्य सर्वस्यानुमानस्य च स्वत एव । प्रवर्त्तकाध्यक्षस्य च कस्यचित्स्वत एव यदभ्यासेन परितो निरस्तविभ्रमाशङ्कम्, यन्निद्राद्यनुपप्लुतं सद् आसन्नदेशमनाशङ्क्यव्यञ्जकाधीनाऽन्यथाभिव्यवित च वस्तु गृह्णाति । तद्रूपसंवेदनादेव सत्यार्थं निश्चीयते । कस्यचित्तु परतोऽर्थक्रियानिर्भासात्मकात् स्वतः प्रमाणादन्यतो वा यतः कुतश्चिन्नान्तरीयकार्थदर्शनान्मध्यकालवर्त्तिभ्रान्तिशङ्कापनोदेन निश्चीयत इति । \textbf{प्रापितश्चार्थमि}ति क्वचित्पाठः । स तु युक्तरूपः प्रवृत्तियोग्यीकरणात्प्रवर्त्तितः प्राप्यार्थोपदर्शनात्प्रापितोऽर्थमित्येकवाक्यतयोपदर्शनात् । एवमुत्तरत्राऽप्येष एव पाठोऽवदात इति । \textbf{तथा च सति} तस्मिंश्च प्रकारे सति । \textbf{समाप्तः} पर्यवसानं गतः । तस्मादधिगमस्य फलत्वं युक्तमिति भावः । यतोऽधिगमादन्यत्फलं नोपपद्यते, जाते च तस्मिन् समाप्यते व्यापारः, \textbf{अतो}ऽस्मात्कारणाद् \textbf{अनधिग}तो ज्ञानान्तरेणापरिच्छिन्नो \textbf{विषयो}ऽर्थो यस्य तत् प्रमाणं भवति ।
	\pend
      

	  \pstart ननु अधिगतविषयमुदीचीनं ज्ञानं तत्रार्थे किञ्चिदधिकमादधानं प्रमाणं भविष्यति । न हि विषयभेदादेव प्रमाणभेदोऽपि तूपयोगभेदादपीत्याह--\textbf{येनैव}--इति । \textbf{हि}र्यरमात् । पूर्वाद् योग्यतया प्रवर्त्तितः प्रापित इति चोक्तं तत्रैव पूर्वज्ञानादधिगते \textbf{किमधिकम}तिरिक्तं \textbf{कार्यं} कर्त्तव्यम् ?  \leavevmode\marginnote{\textenglish{20/dm}} “
	  
	\footnote{एवं सामान्येनाविसंवादकं सम्यग्ज्ञानं प्रतिपाद्य विशेषेण प्रत्यक्षानुमाने स्वव्यापारं कुर्वतां सम्यग्ज्ञानं भवत इति दर्शयन्नाह--\cite{dp-msD-n}}तत्र यो\unclear{र्थो} दृष्टत्वेन ज्ञातः स प्रत्यक्षेण प्रवृत्तिविषयीकृतः\footnote{विषयः कृतः \cite{dp-msB} \cite{dp-msC} \cite{dp-msD}} । यस्माद् यस्मिन्नर्थे प्रत्यक्षस्य साक्षात्कारित्वव्यापारो विकल्पेनानुगम्यते\footnote{अनुभूयते--\cite{dp-msD-n}} तस्य प्रदर्शकं प्रत्यक्षम्; तस्माद् दृष्टतया ज्ञातः\footnote{निश्चितः--\cite{dp-msD-n}} प्रत्यक्षदर्शितः । अनुमानं तु \footnote{लिङ्गदर्शनं लिङ्गजातम् । तच्च वह्न्यव्यभिचारि धूमनिश्चयं ज[[नयति]]सामान्येन साध्याविनाभावित्वस्मरणजातम्--यथा धूमं प्रत्यक्षेण गृहीत्वा सर्वत्रायं वह्निज इति स्मरणं तस्मात्--\cite{dp-msD-n}}लिगदर्शनान्निंश्चिन्वत्\footnote{विकल्पयत्--\cite{dp-msD-n}} प्रवृत्तिविषयं दर्शयति ।” न किञ्चित् । आद्येनैव कर्त्तव्यस्य कृतत्वादिति भावः । \textbf{अतो}ऽस्माद् हेतोरि\textbf{धगतषियं} तदधिगन्तृसजातीयं विजातीयं वा न प्रमाणम् । तेन प्रमाणसम्प्लवो नाम नास्त्येवेति प्रकाशितम् ।
	\pend
      

	  \pstart ननु च प्रबन्धेनानेन प्रवृत्तिविषयोपदर्शकं सम्थग्ज्ञानमिति दर्शितम् । वक्ष्यमाणया नीत्या प्रत्यक्षानुमाननामनी द्वे सम्यग्ज्ञाने । तत्र यदि समानमनयोः प्रवृत्तिविषयोपदर्शनं तदिदं प्रत्यक्षं सत्प्रमाणम्, इदमनुमानं सदिति भेदो न स्यात् । अतस्तदनयोरसमानविषयत्वमेव कथितव्यम् । एवं च कस्य कथं तदिति वक्तव्यमित्याह--\textbf{तत्र} इति । तयोः प्रत्यक्षानुमानयोर्मध्ये \textbf{दृष्टत्वेन ज्ञातो} निश्चितः । यदि ज्ञात इत्येव क्रियते तदाऽनुमेयोऽपि निश्चितः सन् प्रत्यक्षेण प्रवृत्तिविषयीकृतः प्रसज्येतेति दृष्टत्वेनेति कृतम् । अथ यो दृष्ट इति किं नोच्यते ? नोच्यते । क्षणिकत्वादेरपि दृष्टत्वेन प्रत्यक्षविषयत्वादनुमानावताराभावप्रसङ्गादिति ।
	\pend
      

	  \pstart ननु न प्रत्यक्षस्य निश्चयनाद् ग्रहणमपि तु प्रतिभासात् । तत्किमुच्यते ज्ञातो निश्चित इत्याशङ्क्याह--\textbf{यस्माद्} इति । \textbf{विकल्पेनानुगम्यते}ऽनुस्त्रियतेऽध्यवसीयते पश्यामीत्याकारेण । तस्माद् \textbf{दृष्टत्वेन ज्ञात} इत्युच्यते । \textbf{विकल्पेनेति} तत्पृष्ठभाविनाऽनुरूपेणेति द्रष्टव्यम् । अननुरूपविकल्पानुगतव्यापारस्य तत्राप्रामाण्यात् क्षणिकत्व इव । एवं ब्रुवतश्चायमभिप्रायः—सांव्यवहारिकस्य प्रमाणस्येदं लक्ष\leavevmode\marginnote{\textenglish{9b/ms}}णमुच्यते । ततो वस्तुवृत्त्या प्रकाशमानमप्यनुरूप विकल्पेनाविषयीकृतं सदप्रतिभासमानं नातिशेते, व्यवहारायोग्यत्वात् । एवं तद्ग्राहकमपि तथाविधविकल्पेनाननुगम्यमानव्यापारं व्यवहारयितुमपर्याप्तं सत् तृणस्यापि कुब्जीकरणेऽसमर्थमग्राहकं नातिवर्त्तते । तेन यदुक्तं प्रदर्शकत्वमेव प्रवर्त्तकत्वादीति, यच्चोक्तम्--अधिगतिरेव फलमिति तदनुरूपनिश्चयानुगतव्यापारमनुरूपनिश्चयानुगताविति द्रष्टव्यम् । एवं यत्र यत्रोच्यते प्रत्यक्षं वस्तूपदर्शकं वस्तुग्राहकमित्यादिना शब्देन तत्र सर्वत्रानुरूपनिश्चयानुगतव्यापारमेव बोद्धव्यम् ।
	\pend
      

	  \pstart अथैवं सति विकल्पस्यापि प्रामाण्यं प्रसज्येतेति चेत् । एतत्स्वयमेव \textbf{धर्मोत्तरेणा}शङ्क्य निराकरिष्यत इति नेहोच्यते । यदि प्रत्यक्षमेवं प्रवृत्तिविषयमुपदर्शयति अनुमानमप्येवं तदा  \footnote{तत्पृष्ठभाविना विकल्पेनावसीयते । एतदुक्तं भवति--प्रतिभासमानार्थाध्यवसायं कुर्वत् प्रत्यक्षप्रमाणं संवादकमित्यर्थः--\cite{dp-msD-n}}  \leavevmode\marginnote{\textenglish{21/dm}} “
	  
	तथा च प्रत्यक्षं प्रतिभासनं नियतमर्थं दर्शयति । अनुमानं च लिङ्गसम्बद्धं नियतमर्थं दर्शयति । अत एते \footnote{नियतार्थस्य \cite{dp-msB}}नियतस्यार्थस्य प्रदर्शके । तेन ते प्रमाणे । नान्यद्विज्ञानम् ।\footnote{शब्दोपमानादिकम्--\cite{dp-msD-n}} 
	  
	प्राप्तुं शक्यमर्थमादर्शयत्\footnote{०र्थमुपदर्श० \cite{dp-msC} \cite{dp-msD} \cite{dp-msB}} प्रापकम् । प्रापकत्वाच्च प्रमाणम् ।” कथं भेदव्यवस्थेति ? आह--\textbf{अनुमानं तु}--इति । \textbf{तुः} प्रत्यक्षादनुमानस्य वैधर्म्यमाह । प्रत्यक्षं न स्वयं निश्चिन्वत् प्रवृत्तिविषयं दर्शयति किन्तु निश्चाययत् । अनुमानं तु स्वयमेव निश्चिन्वदिति ।
	\pend
      

	  \pstart कथमप्रतिभासमानं निश्चेतुमीष्टे तदित्याह--\textbf{लिङ्गदर्शनाद्} इति । लिङ्गस्य साध्याविनाभूतस्य धूमादेर्दर्शनात् । दर्शनं च स्वरूपग्रहणपूर्वकं “सर्वत्रेदं साध्याविनाभूतमिति” ज्ञानम्, “सर्वत्र साध्याविनाभूतमिति” स्मरणपुरःसरं वा क्वचित् स्वरूपग्रहणमिह विवक्षितम् ।
	\pend
      

	  \pstart ननु च प्रत्यक्षानुमानज्ञानवदन्यस्यापि ज्ञानस्य यथा प्रवृत्तिविषयप्रदर्शनं तथा किं न प्रदर्श्यते ? अथाप्रामाण्यान्नोपदर्श्यते । कथं पुनरप्रामाण्यमन्यस्य ? अनियतप्रवृत्तिविषयप्रदर्शकत्वादिति चेत् । तर्हि प्रत्यक्षानुमानयोरपि तथात्वेन प्रामाण्यं न स्यादित्यागूर्य ज्ञानान्तराद भेदमनयोर्दर्शयन्नाह--\textbf{तथा} च--इति । तस्मिश्च प्रत्यक्षस्य स्वव्यापारानुसारिविकल्पोपजननेन निश्चयनात्, प्रवृत्तिविषयप्रदर्शनप्रकारे\footnote{सूक्ष्मत्वात् न पठ्यते ।}\add{... ... ...}मन्यदा लिङ्गदर्शनात्स्वयं निश्चयेन प्रवृत्ति—विषयप्रदर्शनप्रकारे । \textbf{प्रत्यक्षं} ज्ञानं \textbf{प्रतिभासमानं} स्वरूपेण प्रकाशमानं \textbf{निय}त\textbf{म}र्थक्रियाक्षमे भावरूप एव व्यवस्थितं \textbf{दर्शयति} । तेन नानर्थं नाप्यनियतं दर्शयतीत्याकूतम् । \textbf{अनुमानं च नियतमर्थं दर्शयति । चः} प्रत्यक्षेण सममनुमानं नियतप्रदर्शकत्वेन समुच्चिनोति । \textbf{लिङ्गसम्बद्धम्} इति हेतुभावेन विशेषणम् । तदयमर्थः--यस्माल्लिङ्गं विजातीयव्यावृत्तेऽर्थेक्रियाकारिणि तादात्म्येन तदुत्पत्त्या वा \textbf{सम्बद्ध}मायत्तं तस्मात् तत्प्रभवमप्यनुमानं नियतं दर्शयति अध्यवस्यतीति । तेनानुमानमपि नानर्थं नाप्यनियतं भावात्मन्यभावात्मनि वा दर्शयतीति प्रकाशितम् । इयांस्तु विशेषोऽनुमानं सम्बन्धग्रहणकालदृष्टसाधारणं रूपमाश्रित्योदयमानं स्वलक्षणमध्यवस्यदपि न सन्तानान्तरासाधारणमध्यवस्यतीति प्रवृत्तिविषयापेक्षमपि सामान्यविषयमेव । प्रत्यक्षं तु प्रवृत्तिविषयापेक्षमप्यसाधारणविषयमेव । सन्तानान्तरासाधा\leavevmode\marginnote{\textenglish{10a/ms}}रणेनैव रूपेण विषयस्य निश्चायनादिति ।
	\pend
      

	  \pstart इदानीं यस्यायमाशयः--“अस्तु प्रत्यक्षं नियतार्थदर्शकम्, अर्थस्य साक्षात्करणात्; अनुमानं तु परोक्षार्थस्यासाक्षात्करणात्कथं नियतं दर्शयति ? तस्मान्न द्वयोर्नियतार्थप्रदर्शकत्वम्, अपि तु एकस्यैव” इति । तं सामस्त्यनिषेधवादिनं प्रत्युपसंहरन्नाह--\textbf{अत} इति । यस्माद् वस्तुप्रकाशात् प्रत्यक्षं नियतं दर्शयति । परोक्षस्यान्तरा साक्षादनवभासेऽपि तत्सम्बद्धतया चानुमानं कार्यस्वभावजमेकान्तनियतं भावम्, अनुपलम्भजमेकान्तनियतमभावं तथाभूतं दर्शयति । \textbf{अतो}ऽस्माद् हेतोरेते द्वे अपि तथारूपे । ततोऽनेन सामस्त्यनिषेधस्य निषेध उपसंहृतः । अन्यथा व्याख्यायमाने तु यतोऽस्तु\footnote{यतस्तु} नियतमर्थं प्रदर्शयतोऽत \textbf{एते} नियतार्थोपदर्शके इत्यर्थः स्यात् । एतच्च परिस्फुटस्यैव स्कुटीकरणमनर्थकमापद्येतेति ।
	\pend
      \leavevmode\marginnote{\textenglish{22/dm}}“

	  \pstart आभ्यां प्रमाणाभ्यामन्येन\footnote{शब्दादिना--\cite{dp-msD-n}} च\footnote{०न्येन ज्ञानेन \cite{dp-edE} \cite{dp-edN}} ज्ञानेन \footnote{प्रदर्शितो \cite{dp-msA} \cite{dp-msC} \cite{dp-edP} \cite{dp-edN}}दर्शितोऽर्थः कश्चिदत्यन्तविपर्यस्तः । यथा मरीचिकासु जलम् । \footnote{अर्थः--\cite{dp-msD-n}}स चासत्त्वात् प्राप्तुमशक्यः । कश्चिदनियतो भावाभावयोः\footnote{०योः तद्व[[च्च]] यथा \cite{dp-msC}} । यथा संशयार्थः । न च भावाभावाभ्यां युक्तोऽर्थो जगत्यस्ति । ततः प्राप्तुमशक्यस्तादृशः ।\footnote{अनियतोऽर्थः--\cite{dp-msD-n}}
	\pend
       

	  \pstart सर्वेण \footnote{मानसविकल्प०--\cite{dp-msD-n}}चालिङ्गेन\footnote{चालिङ्गजेन \cite{dp-msC} \cite{dp-msD} \cite{dp-msA} \cite{dp-msB} \cite{dp-edP} \cite{dp-edH} \cite{dp-edE} \cite{dp-edN}} विकल्पेन \footnote{लिङ्गम्--\cite{dp-msD-n}}नियामकमदृष्ट्वा प्रवृत्तेन \footnote{सत्त्वासत्त्वयोः--\cite{dp-msD-n}}भावाभावयोरनियत
	\pend
      ”

	  \pstart भवतामेते नियतार्थोपदर्शके, प्रमाणे तु कथमित्याह--\textbf{तेन}--इति । \textbf{तेन} नियतार्थप्रदर्शकत्वेन । नियतार्थप्रदर्शकत्वाभावात् किलाप्रामाण्यमनयोः शङिकतम् । सत्यपि तस्मिन् कथं तत् स्यादिति भावः । अनयोः प्रवृत्तिविषयप्रदर्शनप्रकारोपदर्शने तदन्यस्य ज्ञानस्य च तदप्रदर्शने यद् बुद्धिस्थमासीत्, तदिदानीं \textbf{नान्यदि}त्यादिनाऽभिव्यनक्ति । \textbf{नान्यद्} विज्ञानं प्रमाणमिति वचनविपरिणामेन सम्बन्धः ।
	\pend
      

	  \pstart यद्येवं ज्ञानत्वाविशेषादमू अपि प्रमाणे मा भूतामित्याह--\textbf{प्राप्तुम्} इति । \textbf{आदर्शयदिति} हेतौ शतुर्विधानाद्धेतुपदमेतत् । तेनायमर्थः--यतः प्राप्तुं शक्यमर्थमादर्शयति तेन प्रत्यक्षादिकं प्रापकमिति । भवतु प्रापकम्, प्रमाणं तु कस्मादित्याह--\textbf{प्रापकत्वाच्च}--इति । \textbf{चो}ऽवधारणे ।
	\pend
      

	  \pstart ननु आभ्यामन्येनापि दर्शितोऽर्थः शक्यप्रापण एव ततस्तदुपदर्शकमपि किं न प्रापकं प्रापकत्वाच्च किं न प्रमाणमित्याह--प्र\textbf{माणाभ्याम्} इति । \textbf{चो} यस्मात् । \textbf{अत्यन्तग्रहणे}न संशयज्ञानविषयाद् विशेषं दर्शयति । तथाविधोऽपि किं न प्राप्येत इत्याह--\textbf{स च} इति । \textbf{चो} यस्मादर्थे । नियतोपदर्शकत्वेऽप्यनर्थोपदर्शकत्वादप्रामाण्यमस्य दर्शितम् । यदि कश्चिदीदृशस्तदन्यज्ञानविषयोऽन्यादृशो भविष्यति । तदुपदर्शकं च प्रमाणं भविष्यतीत्यत आह—\textbf{कश्चिद्} इति । \textbf{संशयार्थः} संशयालम्बनः “स्थाणुर्वा पुरुषो वा”इति हि प्रत्ययः स्थाणुमुल्लिख्य पुरुषो वेत्यालम्बयंस्तदभावमुल्लिखति । ततः स्थाण्वभावाव्यभिचारिणं पुरुषं पुरुष\footnote{\textbf{षा}}भावाव्यभिचारिणं च स्थाणुमवस्यन्न भावे नाप्यभाव\footnote{वे} नियतं स्थाणं पुरुषं वा दर्शयतीति भावाभावयोरनियतं दर्शयति । यत एवायमेकान्तनियतं दर्शयितुमनीशानो दोलायते, तत एव संशय इत्युपपद्यत इति ।
	\pend
      

	  \pstart अथ विपर्ययार्थोऽसत्त्वान्न प्राप्यताम्, अयं तु कस्मान्न प्राप्यत इत्याह--\textbf{न च} इति । \textbf{चो} यस्मादर्थे ।
	\pend
      

	  \pstart ननु भवतु संशयविपर्यययोरप्रामाण्यं\leavevmode\marginnote{\textenglish{10b/ms}} किं नश्छिन्नम् ? एतदतिरिक्तं प्रत्यक्षानुमानाभ्यामन्यत्प्रमाणं भविष्यतीत्याशङ्क्याह--\textbf{सर्वेण} च--इत्यादि । \textbf{चो} यस्मात् । सर्वान्त  \leavevmode\marginnote{\textenglish{23/dm}} “
	  
	एवार्थो दर्शयितव्यः । स\footnote{अनियतोर्थः--\cite{dp-msD-n}} च प्राप्तुमशक्यः । तस्मादशक्यप्रापणम्--अत्यन्तविपरीतम्, भावाभावानियतं चार्थं दर्शयद् अप्रमाणमन्यज्ज्ञानम् । अर्थक्रियार्थिभिश्चार्थक्रियासमर्थ\footnote{०“समर्थार्थप्रा”० \cite{dp-msA} \cite{dp-edH} \cite{dp-edP} \cite{dp-edN} ०समर्थप्रा० \cite{dp-msB} \cite{dp-edE}}वस्तुप्राप्तिनिमितं ज्ञानं मृग्यते । यच्च तैर्मृग्यते तदेव\footnote{तदेव तेन शास्त्रे \cite{dp-msB} \cite{dp-edH}} शास्त्रे विचार्यते । ततोऽर्थक्रियासमर्थवस्तुप्रदर्शकं सम्यग्ज्ञानम्\footnote{इति तद भवति सम्यग्ज्ञानमिति शेषः--\cite{dp-msD-n}} ।” र्गतत्वादनुमानस्यापि तथात्वं स्यादित्याह--\textbf{अलिङ्गेन}--इति । नास्य लिङ्गमुत्पादकत्वेन विद्यत इत्यलिङ्गम् । यद्यलिङ्गेन सर्वेण तत्करणीयं तदा प्रत्यक्षपृष्ठभाविनोऽपि विकल्पस्य तदाऽऽयातमित्याह--\textbf{नियामकम्} इति । विषयाधीनो हि नियतार्थपरिग्रहो ज्ञानस्येत्यर्थो नियामकः । तमदृष्ट्वा ।
	\pend
      

	  \pstart ननु किं प्रत्यक्षपृष्ठभाव्यपि नियामकमर्थं दृष्ट्वा प्रवर्त्तते, येन ताद्रूप्यविरहादन्येषां विकल्पानां तथात्वमाशङ्क्यते । उच्यते । \textbf{प्रवर्त्तेने}त्यत्रान्तर्भूतो णिजर्थो द्रष्टव्यः । ततोऽयमर्थःनियामकमदृष्ट्वा प्रवर्त्तितेनेति । तथा च सति प्रत्यक्षपृष्ठभावी विकल्पो नियामकं दृष्ट्वैव प्रत्यक्षेण प्रत्ययेन प्रवर्त्त्यत इति तथात्वेन परिहृतो भवति । इतरे त्वलिङ्गविकल्पा येन प्रवर्त्त्यन्ते जायन्ते न तेन \textbf{साक्षा}न्नियामको दृष्ट इति तेषां तथात्वानुषङ्गः ।
	\pend
      

	  \pstart अन्ये तु अनुमाननिवृत्त्यर्थमलिङ्गेति योजयित्वा कथं पुनरवसायात्मकेनाप्यलिङ्गेन तेनैवं करणीयमित्याशङ्क्य नियामकमदृष्ट्वेत्यस्य तु हेतुभावेन विशेषणत्वान्नियामकमदृष्ट्वा प्रवृत्तत्वादित्यर्थः, इति व्याख्याय तात्पर्यार्थमपि दर्शयन्ति । यद्यपि शब्दादिजन्मानो विकल्पा नोभयपक्षसंस्पर्शेन दोलायन्ते, तथापि ते नियामकानाश्रयेण प्रवर्त्तमाना न यौक्ति\footnote{क्त}संशयरूपतामतिवर्त्तन्त इति । पूर्वव्याख्यानेऽप्ययमेवाशयः । सर्वस्य तस्य समुदाचरतोर्विरुद्धयोराकारयोरभावेऽपि युक्त्या द्वितीयाकारानुप्रवेशात्संशयरूपत्वेन भावाभावानियतार्थोपदर्शकत्वमिति ।
	\pend
      

	  \pstart एके तु नियामकं विकल्पयितव्यं वस्तुनान्तरीयकं वस्त्वदृष्ट्वा प्रवृत्तेनेति योजयन्ति । \textbf{अलिङ्गे}नेति चास्यैवार्थस्य हेतुभावेन विशेषणमाहुः । उभयेऽपि तु प्रत्यक्षपृष्ठभाविविकल्पेनातिप्रसङ्गमनधिगतार्थाधिगन्तुर्विकल्प्य\footnote{ल्प}स्य प्रामाण्यचिन्ताऽधिकारेण निराकुर्वते । तेनायमर्थः—सर्वेण तेनानधिगतत्वा\footnote{गतार्था}धिगन्तृतयाऽभिप्रेतेनैवमवश्यकरणीयमिति ।
	\pend
      

	  \pstart अनियतार्थप्रदर्शकमपि तत्प्रापकं स्यादित्याह--\textbf{स च} इति । \textbf{चो} यस्मादर्थे । \textbf{अशक्य} इत्यस्यानन्तरं स अवधारणार्थो द्रष्टव्यः । \textbf{सो}ऽनियतोऽर्थो न प्राप्तुं \textbf{शक्यो}ऽशक्यत्वादिति भावः ।
	\pend
      

	  \pstart ननु च तादृशं सर्वं मा भूत्प्रापकम्, प्रमाणं तु कस्मान्न भवतीत्याशङ्क्य पूर्वोक्तमेवोपसंहरति \textbf{तस्माद्} इति । यस्मात्प्रापकत्वादेव प्रमाणं \textbf{तस्मात्} । अर्थस्याशक्यप्रापणत्वेऽत्यन्तविपरीतत्वानित्य\footnote{त्वानियत}त्वे हेतू हेतुभावेनानयोर्विशेषणत्वात् । \textbf{चो}ऽप्राप्तिकार्थसमुच्चये । कश्चिदत्यन्तविपर्यस्तः, कश्चिदनियत इत्यादेर्यथाक्रममुपसंहारः--\textbf{अन्यज्ज्ञान}मिति \textbf{नान्यज्ज्ञानमित्य}स्य ।
	\pend
      \leavevmode\marginnote{\textenglish{24/dm}}“

	  \pstart यच्च तेन\footnote{ज्ञानेन--\cite{dp-msD-n}} प्रदर्शितं तदेव\footnote{तदेव तेन प्रा० \cite{dp-msC} \cite{dp-msD}} प्रापणीयम् । \footnote{ननु च यदि उपदर्शितार्थप्रापणात् प्रमाणं सम्यग्ज्ञानं शुक्लशङ्खे पीतज्ञानं कुञ्चिकाविवरमणिप्रमायां मणिज्ञानम्, अर्धरात्रे मध्याह्नकालग्राहिस्वप्नज्ञानं च प्रमाणमाप्नोति । एभिरुपदर्शितार्थस्यार्थमात्रस्य प्राप्तेस्ततश्च सम्यग्ज्ञानमित्याशङ्क्याह--\cite{dp-msD-n}}अर्थाधिगमात्मकं\footnote{अर्थाधिगमात्मकत्वं हि प्रापकत्वमि० \cite{dp-msC} \cite{dp-msD} अर्थक्रियासमर्थवस्त्वधिगमात्मकत्वम्--\cite{dp-msB}} हि \footnote{प्रापणमि० \cite{dp-edN}}प्रापकमित्युक्तम् ।
	\pend
      ”

	  \pstart स्यादेतत्--शक्यप्राप्तिकार्थोपदर्शकमेव सम्यग्ज्ञानं\leavevmode\marginnote{\textenglish{11a/ms}} \textbf{नान्यदि}ति निर्निमित्तमिदम् । तथा चान्यदपि किं न परीक्ष्यत इत्याह--\textbf{अर्थक्रियार्थिभिः}--इति । \textbf{चो} यस्मात् । \textbf{अर्थ}स्य दाहादेः, \textbf{क्रिया}निष्पत्तिः । तामर्थयन्ताति तथा । एतैर\textbf{र्थक्रिया}यां शक्तस्य वस्तुनः \textbf{प्राप्तेर्यन्निमित्तं तन्मृग्यते}ऽन्विष्यते । अर्थक्रियार्थित्वादेव च तदन्वेषणमेषामवसेयम् । तैश्च प्रेक्षावद्भिरिति प्रकरणाद् द्रष्टव्यम् । इतरथार्थक्रियार्थिभिरप्यप्रेक्षावद्भिर्मिथ्याज्ञानमपि निभाल्यत इति तत्परीक्षाऽप्यापतेत् ।
	\pend
      

	  \pstart यदि नाम तैस्तन्मृग्यं तथापि तदितरदपि ज्ञानं किं न परीक्ष्यत इत्याह \textbf{यच्च}--इति । \textbf{चो}ऽवधारणे । \textbf{विचार्यते} परीक्ष्यते । ईदृशं त\textbf{त्सम्यग्ज्ञानं} प्रवर्त्तकं यदनुसरन्ति भवन्त इति विभज्य प्रतिपाद्यते । अयमस्याशयः--न व्यसनितयाऽऽ\textbf{चार्येण} ज्ञानं विचार्यते । किन्तर्हि ? अर्थक्रियार्थिजनानुरोधेन । ते तथाभूतमेव ज्ञानं मृगयन्ते नान्यदिति \textbf{तदेव} विचार्यत इति । \textbf{तदेवे}ति च मुख्यवृत्त्यभिप्रायेणोक्तम् न तु मिथ्याज्ञानं शास्त्रेऽस्मिन् न विचार्यते एव । प्रसङ्गात्तु तस्यापि कल्पनाप्रभृतौ विचारसम्भवादिति ।
	\pend
      

	  \pstart ननु शक्यप्रापणार्थोपदर्शकमेव सम्यग्ज्ञानं कुतो व्यवस्थाप्यते नान्यदिति प्रश्ने किमिदमप्रकृतमुच्यत इत्याह--\textbf{तत} इति । यतोऽर्थक्रियार्थिनां प्रेक्षावतां नान्यन्मृगयणीयम्, तदन्वेषणीयमेव विचारणीयम्, व्यसनितया विचारासम्भवात् । ततोऽर्थक्रियासमर्थस्य वस्तुनः प्रदर्शकं सम्यग्ज्ञानं नान्यदिति अर्थात् ।
	\pend
      

	  \pstart अस्तु सम्यग्ज्ञानमीदृशमेव । तत्पुनरन्यदुपदर्श्य अन्यत् प्रापयदपि किं न सम्यग्ज्ञानं सत् प्रमाणं व्यवह्रियते ? तथा च पीतसंख्या\footnote{शङ्खा}दिज्ञानमपि प्रामाण्यान्नापैति । आह--\textbf{यच्च} इति । \textbf{चो}ऽवधारणे । अनेन पीतसं\footnote{शङ्}खादिज्ञानमप्यंशे संवादात् प्रमाणमिति यत्कैश्चिदिष्टिं तदपि सम्यग्ज्ञानव्यवच्छेद्यमिति दर्शयति ।
	\pend
      

	  \pstart ननु चानुक्तसममिदं पीतशङ्खादिज्ञानेनानुपदर्शितस्यापि शुक्लशङ्खस्य, जलज्ञानेनानुपदर्शितस्यापि मरीचिकानिचयस्य प्रापणदर्शनात् । नैतदस्ति । यतो न तज्ज्ञानपूर्विका सा प्राप्तिः, तस्याऽज्ञातवस्तुविषयत्वात् । न च तेन शुक्लशङ्खवस्तु मरीचिवस्तु वा ज्ञातम् । यथा तु ज्ञानान्तरादेव तथाविधार्थप्राप्तिर्न ततस्तथा अनेनैव \textbf{प्रामाण्यपरीक्षायां} निर्लोठितमिति नेहोच्यते ।
	\pend
      \leavevmode\marginnote{\textenglish{25/dm}}“

	  \pstart तत्र प्रदर्शितादन्यद्वस्तु भिन्नाकारम्, भिन्नदेशम्, भिन्नकालं च । विरुद्धधर्मसंसर्गाद्धि अन्यद् वस्तु । देशकालाकारभेदश्च विरुद्धधर्मसंसर्गः ।
	\pend
       

	  \pstart तस्माद् अन्याकारवद्वस्तुग्राहि\footnote{०न्याकारवस्तुग्राहि--\cite{dp-msB} \cite{dp-msC} \cite{dp-msD}} नाकारान्तरवति वस्तुनि प्रमाणम् । यथा पीतशङ्खग्राहि शुक्ले शङ्खे । देशान्तरस्थग्राहि च न देशान्तरस्थे प्रमाणम् । यथा कुञ्चिकाविवरदेशस्थायां मणिप्रभायां मणिग्राहि ज्ञानं नापवरकस्थे\footnote{०वरकदेशस्थे--\cite{dp-msA} \cite{dp-msB} \cite{dp-msC} \cite{dp-msD} \cite{dp-edP} \cite{dp-edE} \cite{dp-edH} \cite{dp-edN}} मणौ । कालान्तरयुक्तग्राहि च न कालान्तरवति वस्तुनि प्रमाणम् । यथार्द्धरात्रे मध्याह्नकालवस्तुग्राहि स्वप्नज्ञानं नार्धरात्रकाले\footnote{०अर्धरात्रकालवस्तुनः \cite{dp-edN} अर्धरात्रः कालो यस्य मध्याह्नकालवस्तुनः \cite{dp-msD-n} ।} वस्तुनि प्रमाणम् ।
	\pend
      ”

	  \pstart तदेव तेन प्रापणीयमिति कुत एतदित्याह--\textbf{अर्थाधिगमे}ति । \textbf{हि}र्यस्मात् । \textbf{उक्त}मिति प्रदर्शनादीनां वस्तुतोऽभेदं प्रतिपाद्य \textbf{अत एवार्थाधिगतिरेव प्रमाणफलमिति} अनेन प्रकाशितत्वात्, न तु साक्षादभिहितम् । यद्यन्यदधिगम्यान्यत् प्रापयेत्, तदा प्रदर्शनप्रापणयोर्भेद एव स्यात्, अभेदश्च प्रतिपादित इति भावः ।
	\pend
      

	  \pstart कथं पुनर\textbf{धिगतिरेव फल}मित्यनेन तथात्वमस्योक्तम् ? यतः प्रापकत्वं प्रामाण्यम्, फलं च तदव्यतिरिक्तमिति । अथवा \textbf{प्राप्तुं शक्यमर्थमादर्शयत्प्रापकमित्य}नेनैवमुक्तम् । तथाहि तादृशमर्थमा\textbf{दर्शयदिति} परिच्छिन्ददित्यर्थः । परिच्छेदश्च प्रापकत्वान्न भिद्यत इति ।
	\pend
      

	  \pstart भवतूपदर्शितार्थप्रापकत्वमेव सम्यग्ज्ञानम् । यत्तु तदेव प्राप्य वस्तु शङ्खादिकं पीतरूपतया प्रदर्शयति तत्कथमसम्यग्ज्ञानमित्याह--\textbf{तत्र} इति । \textbf{तत्रे}ति वाक्योपक्षेपे चैतत् । \textbf{प्रदर्शिताद्} रूपाद् \textbf{भिन्नाकारादि} सद् \textbf{वस्त्वन्यदित्य}न्यत्वं विधेयम् । आकारादिभिन्नमपि न ततोऽन्यत्, तदेवेदमित्यध्यवसायादित्याह--\textbf{विरुद्धे}ति । हिर्यस्मात् । यदि विरुद्धधर्माध्यासादन्यत्वम्, न तर्हि देशादिभेदादित्याह--\textbf{देशे}ति । \textbf{चो} यस्मादर्थे । भवत्वेवं तथाप्याकारादिभिन्नग्राहिज्ञानमाकारान्तरादियोगिनि वस्तुनि किं न प्रमाणमित्याह--\textbf{तस्माद्} इति । यस्मात्सम्यग्ज्ञानेन यदेव हि दर्शितं तदेव प्रापणीयम्, आकारादिभिन्नं च ततोऽन्यत् \textbf{तस्मात्} । किमिव क्व न प्रमाणमित्याह--\textbf{यथेति} । भिन्नदेशग्राहिणः का वार्तेत्याह--\textbf{देशान्तरे}ति । किंवन्न प्रमाणमित्याह--\textbf{यथे}ति । \textbf{अव\footnote{प}वरक}शब्देन देशविशेषो लयनोदरादिशब्दवाच्य उच्यते । अपवरकशब्दस्याप्येतदर्थस्य भावात् । \textbf{अपवरकदेशस्थ} इति पाठेऽपि न दोषः ।
	\pend
      

	  \pstart भिन्नकालग्राहिणः का व्यवस्था इत्याह--\textbf{कालान्तरे}ति । उपदर्शनकालादन्येन \textbf{कालेन} व्यावहारिकेण \textbf{युक्तं} सम्बद्धं तद्\textbf{ग्राहि} भिन्नकालवस्तुग्राहीति यावत् । \textbf{चः} पूर्ववत् पूर्वेण सहेदमप्रामाण्येन समुच्चिनोति । तदुदाहरन्नाह--\textbf{यथे}ति । अयमस्यार्थः--अधरात्रे निद्राणस्य मध्याह्नकालप्रतिभासमनुभवतो वणिज्यागतसुतवस्तुदर्शनहृषितरोम्ण एव झटिति बोधे काकतालीयन्यायेन च सत्यपि पुत्रोपलम्भे तज्ज्ञानं न प्रमाणम् । तत्खलु परमार्थेनार्धरात्रकालसम्बद्धं मध्याह्नकालाकलितं च प्रतीतमिति मध्याह्नकालत्वेनावभासनादेव च मध्याह्नकालं तद्वस्तूक्तम् ।  \leavevmode\marginnote{\textenglish{26/dm}} “
	  
	ननु \footnote{ननु देश \cite{dp-msA} \cite{dp-edP} \cite{dp-edE}}च देशनियतम्, आकारनियतं च प्रापयितुं शक्यम् । यत्कालं तु परिच्छिन्नं तत्कालं न शक्यं प्रापयितुम् । नोच्यते--यस्मिन्नेवकाले परिच्छिद्यते तस्मिन्नेव काले प्रापयितव्यमिति । अन्यो हि दर्शनकालः, अन्यश्च प्राप्तिकालः । किन्तु यत्कालं परिच्छिन्नं तदेव तेन\footnote{तदेव प्रा० \cite{dp-msD} \cite{dp-msA} \cite{dp-edP} \cite{dp-edE}} प्रापणीयम् । अभेदाध्यवसायाच्च सन्तानगतमेकत्वं द्रष्टव्यमिति ।” \textbf{अर्धरात्रः काल} आगमनकालो यस्य तस्मिन् \textbf{न प्रमाणम्} । अथवा योऽर्धरात्रे सुप्तोऽह्नो मध्यमुद्गते सवितरि पुत्रमागतं दृष्टवा प्रभातायां रात्रौ सम्पन्ने मध्यन्दिने तस्मिन्नेव च देशे तमेव सुत\footnote{तं} प्रागतं प्रबुद्धोऽपि पश्यति । तस्य तज्ज्ञानं संवादमात्रभागपि न प्रमाणम्, यतोऽर्धरात्रे मध्याह्नम्, तदानीमनागतमपि पुत्रवस्त्वागतम्, तेनावगतम् । न चार्धरात्रे देशकालनियतस्य पुत्रस्य, तस्य च कालस्यास्ति सद्भावः । तदा त्वर्धरात्रः परमार्थतः प्रतिभासकालो यस्य मध्याह्नकालयुक्तवस्तुन इति योज्यम् ।
	\pend
      

	  \pstart एतेन सुप्तस्य केनचित् पठ्यमानं ग्रन्थं शृण्वतो ज्ञानमसत्यार्थं व्याख्यातमवसेयम् । यस्माद्देशकालभिन्नात्मानं श्रोतारं ग्रन्थं च श्रोतव्यं पश्यति निद्रोपहतः । न च तद्देशकालसम्बद्धौ स्तः । यद्देशकालौ च स्तस्तथा न गृह्णातीति ।
	\pend
      

	  \pstart अनयैव च दिशा वाजिस्वप्नोऽपि व्याख्यातो द्रष्टव्यः । यतस्तत्रापि घोटकस्वप्ने यत्कालदर्शनवि\leavevmode\marginnote{\textenglish{12a/ms}}षया भावा दृश्यन्ते न तथा सन्ति, यत्कालाश्च सन्ति तत्काला न दृश्यन्ते । अर्धरात्रादिषु हि स्वप्नदर्शनम्, सूर्योदयादिसम्बद्धाश्च ते भावा दृश्यन्ते । तस्मात्स्वप्ने केषाञ्चिदनुभूतानामत्यन्तमभावादर्थक्रियाया नास्ति सत्त्वम् । केचित्तु अर्थक्रियाकारितया अभिमताः सत्यस्वप्ने विषया दृष्टकालभेदव्यभिचारिणो न सन्त्येवेति प्रकरणार्थः ।
	\pend
      

	  \pstart इदमीदृग्विशिष्टं स्वप्नज्ञानमुदाहरतो \textbf{धर्मोत्तर}स्यायमाशयः--एवंविधस्य स्वप्नज्ञानस्य सदर्थत्वाभिमानः केषाञ्चित्सत्यस्वप्नवादिनामस्तीति तदभिमानशमनायेदं मयोदाहरणीकृतम् । न त्वस्मादन्यस्य स्वप्नज्ञानस्य कश्चिद् विशेषः । सर्वस्यैव स्वप्नज्ञानस्य निरालम्बनतया मिथ्याज्ञानत्वादिति ।
	\pend
      

	  \pstart इदानीं “परिच्छिद्यमानस्य यत्कालं परिच्छेदनं तत्कालमेव प्राप्यमाणस्य प्रापणमभिमतम्”--एवं ब्रुवत इति मत्वाऽस्यार्थस्यानुपपत्तिं चोदयति--\textbf{ननु च}--इति । देशे नियतमाकारे नियतं च शक्यं प्रापयितुमित्यभिदधानोऽनयोरविप्रतिपत्तिमाह । इदं तु न सम्भाव्यत इत्याह--\textbf{यत्कालं तु}--इति । \textbf{तु}नाऽनन्तरोक्ताभ्यां विधाभ्यां वैधर्म्यमाह । यः कालोस्य परिच्छेदनस्य तद् यथा भवति तथा परिच्छिन्नम् । स कालो यस्य प्रापणस्य तद् यथा भवति तथा न शक्यं प्रापयितुम् । यस्मिन् काले परिच्छिन्नं तस्मिन् काले न शक्यते प्रापयितुमित्यर्थः । एवं च ब्रुवताऽनेन \textbf{कालान्तरग्राहीति} यदुक्तं तदयुक्तमिति दर्शितम् । सर्वस्यैव ज्ञानस्य एवंशीलत्वादिति ।
	\pend
      

	  \pstart \textbf{नोच्यत} इत्यादिना सिद्धान्तवादी चोदकस्यानुक्तोपालम्भमाह । अनेनापि न किञ्चि \leavevmode\marginnote{\textenglish{27/dm}} “
	  
	सम्यग्ज्ञानं पूर्वं कारणं यस्याः सा तथोक्ता । कार्यात् \footnote{कथं पूर्वशब्दः कारणे वर्त्तते इत्याह--\cite{dp-msD-n}}पूर्वं भवत् कारणं पूर्वमुक्तम् । कारणशब्दोपादाने\footnote{शब्दापादाने \cite{dp-msA} \cite{dp-msB}} तु पुरुषार्थसिद्धेः\footnote{अव्यवहितम्--\cite{dp-msD-n}} साक्षात्कारणं \footnote{मन्यते \cite{dp-msA} \cite{dp-edH}}गम्येत । पूर्वशब्दे तु पूर्वमात्रम् ।” त्कारणमुक्तम् । तत् किं त्वयैवं नोच्यत इति पार्श्वस्थं प्रत्यस्यैवाभिप्रायं प्रकाशयति--अन्यो \textbf{हि}--इति । \textbf{ही}ति यस्मात् । यद्येवं नोच्यते तर्हि किं नामोच्यत इत्याह--\textbf{किन्तु}--इति । निपातानिपातसमुदायोऽयं केवलमित्यस्यार्थे ।
	\pend
      

	  \pstart ननु असङ्गतमिदं वाक्यम् । न हि यदेव परिच्छिन्नमित्यस्ति येनैवमुच्येत । यत्कालं तु परिच्छिन्नं तत्कालमिति तु युक्तं वक्तुम्, न तदेवेति । सत्यमेतत् । केवलं बोधे यत्नः करणीयः । \textbf{यत्कालमि}त्यनेन हि तत्कालमिति प्राप्तम्, तदेवेति तच्छब्देन वस्तुविषयो यच्छब्द आकृष्यते । ततोऽयमर्थः--यत्कालं परिच्छिन्नं यद् वस्तु तत्कालं तदेव वस्तु प्रापणीयमिति । अहो गडुप्रवेशेऽक्षितारानिर्गमो जातः । एवं खलु वाक्यं स्यात् समर्थितम्, पूर्वपक्षात्पुनरस्याविशेषः प्राप्तः । अस्ति विशेषो महान्, केवलं भवता न समीचीनं निरूपितः । तथाहि पूर्वपक्षावस्थायां यत्कालं तत्कालं इति चात्र बहुव्रीहिणा परिच्छेदनलक्षणा प्रापणलक्षणा च क्रियाऽभिधीयते । इदानीं पुनर्वस्तुतो नायमर्थः । यद् वस्तु येन कालेन सम्बद्धं परिच्छिन्नं तदेव तेन कालेन सम्बद्धं स्वरूपेण प्रापणीयं तदाऽन्यदा वा । परिच्छेदस्य यादृशः कालस्तस्मिन् काले यद् विद्यमानं तदेव प्रापणीयमिति यावत् । ततश्च परिच्छेदका\leavevmode\marginnote{\textenglish{12b/ms}}लाऽसतो यद् ग्राहकं तन्न प्रमाणमित्यवतिष्ठते । \textbf{नोच्यते यस्मिन् काल} इत्यादिना च यदेतदुक्तं तद् बाह्यप्रापणाभिप्रायेण द्रष्टव्यम् । परमार्थतस्तु ज्ञानस्य प्रदर्शनादन्यः प्रापणव्यापारो नास्तीति यस्मिन्नेव काले परिच्छिद्यतेऽर्थस्तस्मिन्नेव काले प्राप्यत इति । एतच्चानन्तरमेवानेनैव विस्तरेण प्रतिपादितमिति स्ववचनव्याघातोऽन्यथाऽस्य स्यादिति ।
	\pend
      

	  \pstart नन्वेवमपि परिच्छेदकालवर्त्तिनः प्रापणं न सम्भवत्येव । सर्वस्यैव विषयस्य क्षणिकत्वात् । तथा चोपदर्शितार्थप्रापकत्वं नाम कस्यचिदपि ज्ञानस्य नास्तीत्यसम्भवितैव सम्यग्ज्ञानत्वलक्षणस्य स्यादित्याशङ्क्याह--\textbf{अभेदेति । अभेदेनै}करूपत्वेन तदेवेदमित्याकारेणा\textbf{ध्यवसायात्} । उपादानोपादेयकृतक्षणप्रबन्धः सन्तानस्तद्गतस्तदाश्रितः ।
	\pend
      

	  \pstart अयमस्य भावः--सांव्यवहारिकस्येह प्रमाणस्य लक्षणमुच्यते । तेन नैकान्तेन वस्तुस्थितिरपेक्ष्यते । तत्र यद्यपि वस्तुस्थित्या परिच्छिन्नप्राप्ययोर्नानात्वं तथापि व्यवहर्त्तारो निरन्तरापरापरोत्पत्तेरविद्यावशाच्च हेतुफलरूपं क्षणप्रचयं तदेवेदमित्येकत्वेनाधिमुञ्चन्ति ततः परिच्छेदकालभाविनः प्रापणं सम्भवत्येव । \textbf{इतिः} सम्यग्ज्ञानपदव्याख्यानपरिसमाप्तौ ।
	\pend
      

	  \pstart तदनेन प्रबन्धेन सम्यग्ज्ञानपदं व्याख्यायाधुना \textbf{पूर्व}शब्दं व्याचिख्यासुस्तेन सार्धमस्य विग्रहमाह--\textbf{सम्यग्ज्ञानम्} इति ।
	\pend
      \leavevmode\marginnote{\textenglish{28/dm}}“

	  \pstart \footnote{किमनन्तरकारणं ज्ञानं भवति यदपेक्षया व्यवहितस्य ग्रहणार्थं पूर्वग्रहणं कृतमित्याह--\cite{dp-msD-n}}द्विविधं च सम्यग्ज्ञानं--अर्थक्रियानिर्भासम्, अर्थक्रियासमर्थे च प्रवर्त्तकम् । \footnote{तयोर्यत्--\cite{dp-msA} \cite{dp-edP} \cite{dp-edH} \cite{dp-edE} \cite{dp-edN}}तयोर्मध्ये यत् प्रवर्तकं तदिह परीक्ष्यते । तच्च पूर्वमात्रम् । न तु साक्षात्कारणम् । सम्यग्ज्ञाने हि सति पूर्वदृष्टस्मरणम् । स्मरणादभिलाषः । अभिलाषात् प्रवृत्तिः । प्रवृत्तेश्च प्राप्तिः । ततो न साक्षाद्धेतुः ।
	\pend
      ”

	  \pstart ननु पुरुषार्थसिद्धिः सम्यग्ज्ञानस्य प्रयोजनं कार्यत्वेन । तथा च सम्यग्ज्ञानकारणिकेति वक्तुमुचितं तत्किमेवमुक्तमित्याह--\textbf{कार्याद्} इति । अनेन व्यवस्थावाची सन्नेव पूर्वशब्दः कारणे वर्त्तत इति दर्शयति ।
	\pend
      

	  \pstart \textbf{भवदिति} “लक्षणहेत्वोः क्रियायाः” \href{http://http://sarit.indology.info/?cref=Pā.3.2.126}{पाणिनि. ३. २. १२६} इति हेतौ शतुर्विधानात् हेतुपदमेतत् । यद्येवं कारणशब्द एव क्रियतामित्याह--\textbf{कारणे}ति । अयं भावः--करोतीति कारणमुच्यते, न त्वकुर्वद्रूपम् । अकुर्वति पुनः कारणव्यपदेशः कारणकारणत्वादीपचारिकः । न च मुख्ये सम्भवति अमुख्ये प्रत्ययो युज्यते । ततो यदेव ज्ञानमव्यवधानेन पुरुषार्थसिद्धेर्निबन्धनं तदेव कारणशब्देन प्रतीयेतेति । यदि पूर्वग्रहणेऽप्येवं प्रत्ययस्तदा को विशेष इत्याह--\textbf{पूर्वशब्द} इति । \textbf{तुः} कारणशब्दाद् विशेषमस्याह । \textbf{पूर्वमात्रमिति} साक्षात्कारणमितरच्च ।
	\pend
      

	  \pstart ननूभयोः सम्भवे साक्षात्कारणपरिहारेणेतरग्रहणाय पूर्वशब्दः शोभते । न चैवमस्ति, सर्वस्यैव पुरुषार्थसिद्धिं प्रति साक्षात्कारणत्वादित्याह--\textbf{द्विविधं च}--इति । \textbf{चो} यस्मादर्थे । कथं द्वैविध्यमित्याह--\textbf{अर्थक्रिये}ति । निर्भासतेऽस्मिन्निति \textbf{निर्भासः} । अर्थक्रियाया निर्भासो यस्मात् तत् तथा । यतः साधनज्ञानादनन्तरं दाहादिप्रतिभासज्ञानमुत्पद्यते तदेवमुच्यते ।
	\pend
      

	  \pstart केचित्तु क्षणेन कार्यकारणव्यवहारस्यार्वाग्दर्शनेन कर्त्तुमशक्यत्वाद् दाहादिप्रतिभासमेव ज्ञानमेवं ब्रुवते । अर्थक्रियाया \leavevmode\marginnote{\textenglish{13a/ms}} निर्भासोऽस्मिन्निति कृत्वा । अपरम् \textbf{अर्थक्रियासमर्थे च प्रवर्त्तकम्} । यत् प्रवृत्तिसमधिगम्यार्थक्रियागोचरं तदेवमुक्तम् । \textbf{चो}ऽर्थक्रियानिर्भासापेक्षया समुच्चये । अनयोः किं परीक्ष्यत इत्याह \textbf{तयोः}--इति । प्रवर्त्तकमपि साक्षात्कारणं तत् कोऽन्यत्र विद्वेषो येनेदमेव विचार्यते ? किं वा पूर्वशब्देनेत्याह--\textbf{तच्च}--इति । \textbf{चो} यस्मात् ।
	\pend
      

	  \pstart अयमाशयः--यतः प्रवर्त्तन्ते तत् सर्वमर्थप्राप्तेर्व्यवहितं कारणम् । अतश्च सम्भवत्प्रतिबन्धं तत् प्रवृत्त्यङ्गत्वात्पूर्वमात्रम्, न त्वनन्तरं कारणमिति । पूर्वमात्राभिधानेन सान्तरम् । अमुमेव द्रढयन्नाह--\textbf{न तु}--इति । \textbf{तु}र्विशेषार्थः । कुत एतदित्याह--\textbf{सम्यग्ज्ञान} इति । \textbf{हि}र्यस्मात् । तस्मिन् सम्यग्ज्ञाने प्रकरणात् साधननिर्भासे \textbf{सति} । स्मृतिबीजोपजननयोग्यज्ञानज्ञातस्य दाहादेः \textbf{स्मरणम्} । ततोऽ\textbf{भिलाषः}--तत्प्राप्तीच्छा । ततः \textbf{प्रवृत्तिः} प्रवर्त्तकज्ञानोपदर्शितमर्थं प्राप्तुकामा व्यापारसहाया बुद्धिः । तस्याश्च प्राप्तिरुपादित्सितलाभः । यस्मात् स्मरणादिना व्यवधानं \textbf{ततः} तस्मात् \textbf{न साक्षात् हेतुः प्राप्ते}रुपायः पुरुषार्थसिद्धेरिति प्रकरणात् ।
	\pend
      \leavevmode\marginnote{\textenglish{29/dm}}“

	  \pstart अर्थक्रियानिर्भासं\footnote{०निर्भासे तु--\cite{dp-msA} \cite{dp-msC} \cite{dp-edP} \cite{dp-edH} \cite{dp-edN} निर्भासात्तु \cite{dp-msB} \cite{dp-edE} \cite{dp-msD}} तु यद्यपि साक्षात्\footnote{प्रवृत्तिस्तथापि--\cite{dp-msA} \cite{dp-edP} \cite{dp-edH} ०प्राप्तिहेतुः तथापि C} प्राप्तिः, तथापि तन्न परीक्षणीयम् । यत्रैव हि प्रेक्षावन्तोऽर्थिनः \footnote{ससन्देहाः--\cite{dp-msD-n}}साशङ्काः, तत् परीक्ष्यते । अर्थक्रियानिर्भासे च ज्ञाते\footnote{ज्ञाने \cite{dp-msB} \cite{dp-edP} \cite{dp-edH}} सति सिद्धः पुरुषार्थः । तेन तत्र न साशङ्का\footnote{साशङ्का अर्थे ज्ञाते \cite{dp-msA} \cite{dp-edP} \cite{dp-edH}} अर्थिनः । अतस्तन्न परीक्षणीयम् । तस्मात् परीक्षार्हमसाक्षात् कारणं सम्यग्ज्ञानमादर्शयितुं कारणशब्दं परित्यज्य पूर्वग्रहणं कृतम् ।
	\pend
      ”

	  \pstart यदि परम्परयाऽपि प्राप्तिहेतुः परीक्ष्यते तर्हि साक्षात् हेतुरतितरां परीक्षणीय इत्याह—\textbf{अर्थक्रियानिर्भासमिति । तुः} प्रवर्त्तकाध्यक्षादर्थक्रियानिर्भासं भेदवद् दर्शयति ।
	\pend
      

	  \pstart यदा \textbf{प्राप्तिहेतु}रिति पाठस्तदा अर्थक्रियानिर्भासमित्यस्यान्त्यव्याख्यानपक्षे भिन्ना सन्तोषादिप्राप्तिर्लोकाध्यवसायसिद्धा या तस्या हेतुः । \textbf{प्राप्ति}पाठे तु तद्व्याख्यानानवद्यता । पूर्वव्याख्याने तूपचारात् प्राप्तिहेतुः प्राप्तिशब्देन वक्तव्यः । करणसाधनो वा प्राप्तिशब्दो द्रष्टव्यः ।
	\pend
      

	  \pstart कस्मात् तत् न परीक्ष्यत इत्याह--यत्रैव हि--इति । हिर्यस्मात् । साशङ्काः ससन्देहाः । आशङ्काग्रहणस्योपलक्षणत्वात् सविपर्यासा इत्यपि द्रष्टव्यम् ।
	\pend
      

	  \pstart एवं ब्रुवतश्चायमाशयः--न व्यसनमेतच्छास्त्रकृतो येन ज्ञानमपरीक्षमाणः स्वास्थ्यमलभमानः परीक्षते । किन्तर्हि ? व्युत्पाद्यजनप्रयोजनोद्देशेनायमस्यारम्भः । संशयविपर्यासापसारणं च व्युत्पाद्यजनप्रयोजनम् । ततो यत्रैव ते तथाप्रवृत्तयस्तदेव परीक्ष्यत इति । अर्थक्रियानिर्भासेऽपि ते तथावृत्तय इत्याह--\textbf{अर्थक्रियेति । च}शब्दस्तुशब्दस्यार्थे ।
	\pend
      

	  \pstart कुतो न तथावृत्तयः ? येन साक्षात्कारणेऽस्मिन् सति सिद्धः पुरुषार्थः, तेनानन्तरमेव फलस्यानुभूयमानत्वात्, तदैव सन्तोषादिगमनाद् वा । व्याख्यानद्वयेपि अर्थक्रियानिर्भासशब्दवाच्य द्वये स्थिते द्वयमेतत् शङ्कायाः कारणम्--अनन्तरफलादर्शनम्, अध्यवसायसिद्धभिन्नप्राप्त्यभावो वा । तदभावे तु कथं शङ्केरन्निति भावः ।
	\pend
      

	  \pstart मा भूवंस्तत्र साशङ्कास्तत् किं सिद्धमित्याह--\textbf{अत} इति ।
	\pend
      

	  \pstart यदि व्यवहितं प्रवर्त्तकं पुरुषार्थसिद्धेर्न कारणं तर्हि कथमुक्तम्--\textbf{“कार्यात् पूर्वं भवत् कारणं पूर्वमुक्तम्”} इति ? असाक्षात् कारणे चाकारणे कारण\leavevmode\marginnote{\textenglish{13b/ms}}विशेषणं साक्षादिति निरर्थकमित्याशङ्क्याह--\textbf{तस्माद्} इति । यस्मात् साक्षात् कारणमाशङ्कानास्पदं सदपरीक्षणीयं \textbf{तस्माद्} यद\textbf{साक्षात्कारणमत} एव च \textbf{परीक्षार्हं} परीक्षायोग्यं प्रवर्त्तकं \textbf{सम्यग्ज्ञानमादर्शयितु}मादर्शयिष्यामि--इति \textbf{कारणशब्दं} विहाय \textbf{पूर्वग्रहणं कृतमा}चार्येणेत्यर्थात् ।
	\pend
      

	  \pstart एतदुक्तं भवति--योऽयं कारणशब्दो व्यवहिते कारणकारणे वर्त्तते नायं तत्राभिधायकत्वेन वर्त्तते करोतीति कारणमिति व्युत्पत्तेः । किन्तर्हि ? तादर्थ्यादुपचारत इति ।
	\pend
      \leavevmode\marginnote{\textenglish{30/dm}}“

	  \pstart पुरुषस्यार्थः\footnote{नास्तीदं पदं--\cite{dp-msA} \cite{dp-msB} \cite{dp-msD} \cite{dp-edP} \cite{dp-edH} \cite{dp-edE} \cite{dp-edN}} पुरुषार्थः । अर्थ्यत इत्यर्थः काम्यत इति यावत् । हयोऽर्थः, उपादेयो वा । हेयो ह्यर्थो हातुमिष्यते, उपादेयोऽपि उपादातुम् । न च हेयोपादेयाभ्यामन्यो राशिरस्ति । उपेक्षणीयो\footnote{०णीयोऽनु० \cite{dp-msA} ०णीयोप्यनु \cite{dp-msB} \cite{dp-edH} \cite{dp-edN} ०णीयोपि ह्यनु० \cite{dp-msC} \cite{dp-msD}} ह्यनुपादेयत्वात् हेय एव । तस्य सिद्धिः--हानम्, उपादानं च । हेतुनिबन्धना\footnote{अर्थहेतुनिबन्धना--\cite{dp-msD-n} ।} हि सिद्धिरुत्पत्तिरुच्यते । ज्ञाननिबन्धना तु सिद्धिरनुष्ठानम् । हेयस्य च\footnote{हेयस्य हा०--\cite{dp-msA} \cite{dp-msB} \cite{dp-edP} \cite{dp-edH} \cite{dp-edE} \cite{dp-edN}} हानमनुष्ठानम् उपादेयस्य चोपादानम् । ततो हेयोपादेययोर्हानोपादानलक्षणानुष्ठितिः \footnote{सिद्धिरुच्यते \cite{dp-msB}}सिद्धिरित्युच्यते ।
	\pend
      ”

	  \pstart स्यादेतत्--सत्यपि कारणग्रहणे विचारार्हमेव सम्यग्ज्ञानं प्रतिपत्स्यते । यतस्तद्व्युत्पाद्यत इत्यर्थः, साध्यत्वात् । प्रातिपदिकार्थस्तु कारकत्वाद् गुणः प्रधानानुयायी । तेन प्रधानानुरोधात् तच्छब्दो वर्णितया नीत्या मुख्यस्य कारणस्य सम्यग्ज्ञानस्य व्युत्पत्तिकर्मताऽनुपपत्तेस्तत्परित्यागेन लक्षणया तादर्थ्यभूतया प्रत्यासत्त्याऽर्थक्रियासमर्थप्रवर्त्तक एव प्रवर्त्तिष्यते । यथा मञ्चाः क्रोशन्तीत्यत्र प्रधानानुरोधात् मञ्चशब्दः क्रोशनक्रियाकर्त्तृत्वानुपपत्तेर्मुख्यमर्थं त्यक्त्वा लक्षणया तात्स्थ्यभूतया प्रत्यासत्त्या पुरुषेषु वर्त्तत इति । सत्यमेतत् । केवलमेवं सति व्याख्यातृणामिदं कौशलं स्यान्न शास्त्रकृत इति सर्वमनवद्यम् ।
	\pend
      

	  \pstart इदानीं \textbf{पुरुषार्थसिद्धि}पदं विवरिषुः \textbf{पुरुष}शब्देन सार्ध\textbf{मर्थ}शब्दस्य विग्रहम्, \textbf{अर्थस्य च} स्वरूपं \textbf{पुरुषस्ये}त्यादिनाचष्टे । \textbf{अर्थ्यत} इत्याचक्षाणो \footnote{\textbf{अर्थ उपयाच्ञायाम्}--धातुपाठ १०. ३७३ ।}“अर्थ याच्ञायाम्” इत्यतो णिजन्तात् कर्मण्यचं दर्शयति । \textbf{अर्थ्यत} इत्यस्यार्थं स्पष्टयति--\textbf{काम्य}त इष्यत इत्यनेन यावानेवार्थ उक्तस्तावाने\textbf{वार्थ्यत} इत्यनेनापीति \textbf{इति यावदि}त्यस्यार्थः ।
	\pend
      

	  \pstart कोऽसावर्थ इत्याह--\textbf{हेय} इति । \textbf{वा}शब्दश्चशब्दस्यार्थे । अर्थ्यमान इष्यमाणोऽर्थः । न तर्हि हेयोऽर्थ इत्याह--\textbf{हेय} इति । \textbf{हि}र्यस्मात् । \textbf{हेयो हातुं} त्यक्तुमिष्यते तस्मादर्थः । अयमाशयः--इष्यमाणः खल्वर्थः । अत्र यद्यपि स्वीकरणेच्छा नास्ति तथापि परिहारेच्छा तावदस्तीति । यदि हातुमिष्यमाणोऽर्थः कथमुपादेयोऽर्थ इत्याह--\textbf{उपादेयो}ऽपीष्यते केवलमुपादातुम् । न केवलं हेय इष्यते इत्यपिशब्दः ।
	\pend
      

	  \pstart ननु च न हेयोपादेयावर्थोऽपि तु अन्योपि । ततः सोऽपि कस्मान्न प्रदर्श्यत इत्याह—\textbf{न चेति । चो}वधारणे यस्मादर्थे वा ।
	\pend
      

	  \pstart ननूपेक्षणीयोऽपि राशिरस्ति । यत्र न प्रवृत्तिर्यतश्च न निवृत्तिः । तत् कथमनुभवसिद्धस्यापह्नव इत्याह--\textbf{उपेक्षणीय} इति । हिर्यस्मात् ।
	\pend
      

	  \pstart अत्र कश्चिदाह--“यथाऽसावनुपादेयस्तथाऽहयोपि । तत्र यद्यनुपादेयत्वाद्धेयस्तदाऽहेयत्वादुपादेयः किं न भवति” इति । साधूक्तं तेन \textbf{भदन्तेन} केवलं हेयशब्दार्थविचारे मनो न  \leavevmode\marginnote{\textenglish{31/dm}} “
	  
	सर्वा चासौ पुरुषार्थसिद्धिश्चेति । सर्वशब्द इह द्रव्यकार्त्सन्ये\footnote{प्रवृत्तः--\cite{dp-msD}} वृत्तो न \footnote{न च प्र० \cite{dp-msB} \cite{dp-edH} न प्रकार \cite{dp-msC} \cite{dp-edP} \cite{dp-edE}}तु प्रकारकार्त्स्न्ये । ततो नायमर्थः--द्विप्रकारापि सिद्धिः सम्यग्ज्ञाननिबन्धनैवेति ।\footnote{निबन्धनेति \cite{dp-msA} \cite{dp-msB} \cite{dp-msD} \cite{dp-edP} \cite{dp-edH} \cite{dp-edE}} अपि त्वयमर्थः--या काचित् सिद्धिः सा सर्वा कृत्स्नैवासौ सम्यग्ज्ञाननिबन्धनेति ।\footnote{निबन्धनैवेति \cite{dp-msA} \cite{dp-edP} \cite{dp-edH} \cite{dp-edE}} मिथ्याज्ञानाद्धि काकतालीयाऽपि नास्त्यर्थसिद्धिः । तथा हि--यदि प्रदर्शितमर्थं प्रापयत्येवं\footnote{एवमिति सम्यग्ज्ञानवत्--\cite{dp-msD-n}} ततो\footnote{ततो मिथ्याज्ञानात्--\cite{dp-msD-n}} भवत्यर्थसिद्धिः ।” प्रणिहितम् । तथाहि--हीयते, त्यज्यते, न स्वीक्रियते इति हेयः । हानञ्चास्वीकरणम् । न तु गृहीत्वा परित्यागः । तेन अहिविषकण्टकादीनामपि हानं तत्रा\leavevmode\marginnote{\textenglish{14a/ms}}ऽप्रवृत्तिरेव । सा चोपेक्षणीयेऽप्यस्ति । तथा च यदि तस्य स्वीकारो भवेत्, तदाऽहेयत्वं सिद्धिमध्यासीत । न च तत् स्वीक्रियत इति व्यक्तमयं हेतुरसिद्धस्तस्य । न त्वस्माकमसिद्धः । अनुपादेयत्वस्य अस्वीकर्त्तव्यत्वस्य सिद्धत्वात् ।
	\pend
      

	  \pstart अथानुभवसिद्धस्य तृतीयराशेरुपेक्षणीयस्यास्वीकरणमात्राद्धेयेन सार्धमैक्यप्रतिपादनमयुक्तमिति चेत् । प्रियमुक्तं प्रियेण । न हि वयमपि प्रसिद्धयोर्हेयोपेक्षणीययोरर्थयोरेकस्वभावतामातिष्ठामहे । किन्तु क्रियानिमित्तेन हेयशब्देनैकाभिधानम् तथाऽस्वीकरणार्थेन हानशब्देनोपेक्षाया अभिधानमिति किमवद्यम् ? एतच्च हेयोपादेयत्वमर्थस्यैकपुरुषैककालापेक्षया प्रत्येयम् ।
	\pend
      

	  \pstart अधुना \textbf{सिद्धि}शब्दस्यार्थपदेन विग्रहं ब्रुवन्नर्थमाह--\textbf{तस्य}--इति ।
	\pend
      

	  \pstart ननु लोके सिद्धिर्निष्पत्तिरुच्यते, यथौदनस्य सिद्धिरिति । तत्कथमेवं वर्ण्यत इत्याह \textbf{हेतुनिबन्धने}ति । \textbf{हेतुः} कारणं निबद्ध्यतेऽस्मिन्निति \textbf{निबन्धनं} प्रतिबन्धविषयः । हेतुर्निबन्धनमस्या इति विगृह्य हेतुप्रतिबद्धेत्यर्थो वक्तव्यः ।
	\pend
      

	  \pstart यदि जनकनिबन्धना सिद्धिरीदृशी तर्हि ज्ञाननिबन्धना कीदृशी भविष्यतीत्याह--\textbf{ज्ञाने}ति । \textbf{तुः} पूर्वस्याः सिद्धेरस्या भेदं दर्शयति । एवं च व्याचक्षाण \textbf{अर्थ}पदव्याख्याने “अर्थः प्रयोजनं दाहादि”, सिद्धिपदव्याख्याने च “तस्य दाहादेर्निष्पत्तिः” इति यद् \textbf{विनीतदेवशान्तभद्रौ} व्याचक्षातां तद् द्वयमप्यपास्यति । यतः स्वहेतोरेव वह्न्यादेर्दाहादेर्निष्पत्तिर्न तु ज्ञानात् तस्य तदकारकत्वादिति ।
	\pend
      

	  \pstart ननु यदि सिद्धिरनुष्ठानं तर्हि हानमुपादानं च कथं सिद्धिरित्याह--\textbf{हेयस्य च}--इति । \textbf{चो} हेतौ, अवधारणे वा \textbf{हान}मित्यस्मात्परो द्रष्टव्यः । द्वितीयश्चकारः समुच्चयार्थः । किमेवं सति सिद्धमित्याह--\textbf{तत} इति । यतो ज्ञाननिबन्धना सिद्धिरीदृशी \textbf{तत}स्तस्मात्कारणात् ।
	\pend
      

	  \pstart \textbf{पुरुषार्थसिद्धि}शब्दानां विग्रहमर्थं च व्याख्याय सम्प्रति \textbf{सर्व}शब्दं व्याख्यातुं विग्रहमाह—\textbf{सर्वा च}--इति । एवं च विगृह्णन् यद् \textbf{विनीतदेवेन} व्याख्यातं “सर्वश्चासौ लौकिको लोकोत्तरश्चासन्नदेशो दूरदेशश्च पुरुषार्थश्चेति, तथा तस्य सिद्धिः” इतिः तद् दूषयति । एवं हि  \leavevmode\marginnote{\textenglish{32/dm}} “
	  
	प्रदर्शितं च प्रापयत् सम्यग्ज्ञानमेव । प्रदर्शितं चाप्रापयत् मिथ्याज्ञानम् । अप्रापकं च कथमर्थसिद्धिनिबन्धनं स्यात्? तस्माद् यन्मिथ्याज्ञानं न ततोऽर्थसिद्धिः । यतश्चार्थसिद्धिस्तत् सम्यग्ज्ञानमेव । अत एव सम्यग्ज्ञानं यत्नतो व्युत्पादनीयम् । यतस्तदेव पुरुषार्थसिद्धि\footnote{सिद्धेः B}—निबन्धनम् ।” व्याख्यायमाने \textbf{सर्व}शब्देन \textbf{पुरुषार्थसिद्धे}रविशेषणाद् मिथ्याज्ञानात् काकतालीयार्थसिद्धिरनिवारिता स्यात् । न च सा सम्भविनी । यथा च सा न सम्भवति, तथाऽनन्तरमेव प्रतिपादयिष्यते ।
	\pend
      

	  \pstart योऽपि \textbf{शान्तभद्रः} “सर्वश्चासौ पुरुषार्थश्च, सर्वेषां वा पुरुषाणामर्थस्तस्य सिद्धिः” इति व्याचष्टे सोऽप्यनयैव द्वारा निराकृतः ।
	\pend
      

	  \pstart ननु सर्वशब्दस्य प्रकारकार्त्स्न्यवृत्तेरपि दर्शनान्न ज्ञायते किंवृत्तिरत्राभिप्रेत इत्याह—\textbf{सर्वशब्द} इति । \textbf{द्रव्यस्य} निःशेषतायां \textbf{वृत्तः} प्रवृत्तो वाचकभावेन । यथा “सर्वोत्पत्तिमतामीश्वरो निमित्तकारणम्” इत्य\leavevmode\marginnote{\textenglish{14b/ms}}त्र । द्रव्यशब्देन चेदं तदिति व्यपदेशयोग्यं ग्रहीतव्यम् । न तु \textbf{वैशेषिक}सिद्धान्तप्रसिद्धं पृथिव्यादि । तदुक्तम्--
	\pend
      “
	    
	    \stanza[\smallbreak]
	वस्तूपलक्षणं यत्र सर्वनाम प्रसज्यते ।&द्रव्यमित्युच्यते सोऽर्थो भेद्यत्वेन विवक्षितः ॥ इति ।\&[\smallbreak]


	”

	  \pstart पुरुषार्थसिद्धिरपीदं तदिति व्यपदेशयोग्या । तेन साऽपि द्रव्यम् । तथावृत्तेरस्य ग्रहणादन्यवृत्तितया प्रसिद्धस्यापि प्रतिषेधं कण्ठोक्तं करोति । \textbf{वृत्त} इति वर्त्तते । यथा पूर्वे व्याचक्षते “सर्वरसभोक्ताऽयं भिक्षाकः” इति । यथाऽयं सर्वशब्दः प्रकारकार्त्स्न्यवाची तद्वदत्रापीति । प्रकारकार्त्स्न्यवृत्तेरग्रहणे को गुण इत्याह--\textbf{तत} इति । ततः प्रकारकार्त्स्न्यवृत्तस्याग्रहणत् । तथापि कथमयमर्थो भवतीति चेत् । भवति हि प्रकारकार्त्स्न्यवचने सर्वशब्दे सत्ययमर्थः--न सोऽस्ति पुरुषार्थसिद्धेः प्रकारः फललक्षण उपादानलक्षणो वा यो न सम्यग्ज्ञाननिबन्धन इति । सति चैवं द्विप्रकाराऽपि सिद्धिः सम्यग्ज्ञानपूर्विकेत्ययमर्थो भवति ।
	\pend
      

	  \pstart ननु चास्मिन्नप्यर्थे प्रकारद्वितयस्य संगृहीतत्वात् किमसंगृहीतं नाम येनायमर्थो यत्नेन महता हीयत इति? कथं न हीयतां कस्याश्चित् तत्प्रकारान्तःपतिताया हानव्यक्ते\footnote{सम्यक् न पठ्यते ।}\add{... ... ...}। न हि षडपि रसप्रकारान् भुञ्जानः कश्चिन्मधुराम्लादिरसव्यक्तीः सर्वा एव भुङ्क्ते । न च तथाऽकुर्वन् सर्वरसभोक्ता न भवतीति ।
	\pend
      

	  \pstart यद्ययं नार्थः कस्तर्हीत्याह--\textbf{अपि} तु--इति । निपातसमुदायोऽयं किन्त्वित्यस्यार्थे सर्वत्र वर्त्तते ।
	\pend
      

	  \pstart ननु \textbf{कृत्स्नैवासौ सम्यग्ज्ञाननिबन्धने}त्युच्यमाने मिथ्याज्ञानात् काकतालीयाऽप्यर्थ\textbf{सिद्धि}र्नास्तीति दर्शितं स्यात् । न चैतद् युज्यते । यतो यद्यपि मिथ्याज्ञानान्नियमवती नास्त्यर्थसिद्धिः, तथापि कादाचित्की विद्यत एव । यथा कश्चित् दहनोपरिवर्त्तिं मशकवर्त्तिं धूममवसायाग्निमनुमाय यदि प्रवृत्त्याऽग्निमासादयति । अत एव च मिथ्याज्ञानस्य कादाचित्कार्थ \leavevmode\marginnote{\textenglish{33/dm}} सिद्धिनिबन्धनत्वात्, सम्यग्ज्ञानस्य तु नियमेनार्थसिद्धिनिबन्धनहेतुत्वात्सम्यग्ज्ञानमेव यत्नतो व्युत्पाद्यते । न त्वेतावता मिथ्याज्ञानान्नास्त्येवार्थसिद्धिः । प्रकारकार्त्स्न्यवृत्तग्रहणे तु नायं दोषः । एवं हि सम्यग्ज्ञानासाध्यः परप्रकारो निराक्रियते । न तु तत्प्रकारान्तःपातिन्याः कस्याश्चित् सिद्धिव्यक्तेर्मिथ्याज्ञानसाध्यत्वम् । न च हानोपादानव्यक्तयः प्रत्येकं प्रकारशब्दवाच्याः प्रकारानन्त्यप्रसङ्गात् । तस्मात् सर्वशब्दः प्रकारकार्त्स्न्यवृत्तिरेव ग्रहीतव्यः--इति \textbf{पूर्वेषां} मतमाशङ्क्याह--\textbf{मिथ्येति} । काकतालयोः संयोग इवाकस्मिकी \textbf{काकतालीया} । कुतो नास्तीत्याह--\textbf{तथा ही}ति । \textbf{तत} इति मिथ्याज्ञानात् । तदपि प्रापयत्येवेत्याह—\textbf{प्रदर्शितम्} इति । \textbf{चो} यस्मात् ।
	\pend
      

	  \pstart प्रदर्शितं प्रापयदपि मिथ्याज्ञानं भविष्यति । ततः सम्यग्ज्ञानमेवेत्यवधारणमयुक्तमित्याह--\textbf{प्रदर्शितम्} इति । \textbf{चो}ऽवधारणे \textbf{अप्रापय}दित्यतः परो द्रष्टव्यः । “लक्षणहेत्वोः”\href{http://http://sarit.indology.info/?cref=Pā.3.2.126}{पाणिनि—३. २. १२६} इति हेतौ शतुर्विधानात् । प्रदर्शितार्थाप्रापणादेव मिथ्याज्ञान \leavevmode\marginnote{\textenglish{15a/ms}} मित्यर्थः । मिथ्याज्ञानस्य हीदमेव तत्त्वं यत् प्रदर्शिताप्रापकत्वं नाम । प्रापके तु मिथ्याज्ञानव्यपदेशः समारोपितः स्यादिति भावः । मा भूत् प्रापकम्, तदर्थसिद्धेस्तु निबन्धनं किं न स्यादित्याह—\textbf{अप्रापक}मिति । \textbf{चो} यस्मादर्थे ।
	\pend
      

	  \pstart \textbf{तस्मादि}त्यादिनोक्तार्थोपसंहारं करोति । यस्मादुपदर्शिताप्रापकमेव मिथ्याज्ञानं \textbf{तस्माद्} यत्किञ्चिन्मिथ्याज्ञानं ततः सर्वत एव नार्थसिद्धिः । यः पुनरकस्माद् वह्नेरुपरि मशकवर्त्तिं धूममवसाय वह्निमवस्यति, सोऽप्यनुमानकालात्पूर्वं यावन्मशकवर्त्त्यादिभ्यो व्यावृत्तं वह्निकार्यं नियतं धूमरूपं निरूप्य विवेचितं तदानीमनुमानकाले नानुसरति, तदनुसरणे भ्रान्त्ययोगात्, तावदनिवर्त्तिंते संशयहेतौ, तिष्ठतु तावन्मशकवर्त्तिदर्शनम्, सत्यधूमदर्शनेऽपि सन्दिग्धवह्निकार्यत्वात्कथं निःशङ्को नाम सचेतनः । तस्मान्निश्चितनियतरूपादेव धूमाद् वह्निनिश्चयः । अन्यस्मात्त्वेकांशावग्रहोऽप्यनिवर्तितशङ्काहेतुः संशय एव । संशयश्च भावाभावानियतस्यार्थस्याशक्यप्रापणस्य दर्शको नार्थसिद्धेर्निबन्धनम् । ज्ञानान्तरादेव तु सा प्राप्तिरित्यस्याभिप्रायः । एवमन्यत्रापि मिथ्याज्ञाने यथा प्रामाण्यं न युज्यते, यथा च ज्ञानान्तरादेव सा प्राप्तिस्तथा \textbf{धर्मोत्तरेणैव विनिश्चयटीकायां} विपञ्चितमिति नेह प्रतन्यते ।
	\pend
      

	  \pstart कुतश्चिदर्थसिद्धिरपि स्यादित्याह--\textbf{यतश्च}--इति । \textbf{चोऽ}न्यस्माद् भेदवद्रूपमुपदर्शयति । अर्थसिद्धिनिबन्धनं मिथ्याज्ञानमा\textbf{चार्य}स्यापि नाभिमतमिति सामर्थ्याद् दर्शयति । यदाऽऽह—\textbf{अत ए}वेति । यत एव मिथ्याज्ञानादर्थसिद्धिर्नास्ति, सम्यग्ज्ञानाच्चास्त्येव । \textbf{अतो}ऽस्मादेव हेतोः । यत्नग्रहणमनुषङ्गेण मिथ्याज्ञानव्युत्पादनं दर्शयति । मिथ्याज्ञानस्यापि कस्यचिद् व्युत्पादनादित्युक्तप्रायम् । \textbf{अत एवे}त्यनेन यदेव हेतुत्वेनापेक्षितं तदेव \textbf{यत} इत्यादिना सुखप्रतिपत्त्यर्थं कण्ठोक्तं करोति । अयमाशयः--यद्या\textbf{चार्यो} मिथ्याज्ञानादप्यर्थसिद्धिं भवित्रीमभिप्रेयात्, तदा न सम्यग्ज्ञानमेव प्रस्तारेण व्युत्पादयेत् । तदेव च तथा व्युत्पादितवान् । अतोऽवसीयते नैव तस्येदमभिप्रेतमिति ।
	\pend
      

	  \pstart ननु सम्यग्ज्ञानपूर्विकैवेत्यवधारणे सति मिथ्याज्ञानादर्थसिद्धिर्नास्तीति लभ्यते । न चोपात्तमवधारणम् तत्कथमयमर्थो लभ्यत इत्याह--\textbf{तत} इति । यतः सर्वशब्दे द्रव्यकार्त्स्न्य- \leavevmode\marginnote{\textenglish{34/dm}} “
	  
	ततो यावद् ब्रूयात् \footnote{ब्रूयात् पुरु० \cite{dp-msB} \cite{dp-msC} \cite{dp-edP} \cite{dp-edH} \cite{dp-edE}}या काचित् पुरुषार्थसिद्धिः \footnote{सिद्धिः सम्य० \cite{dp-msA} \cite{dp-msC} \cite{dp-edP} \cite{dp-edH} \cite{dp-edE}}सा सम्यग्ज्ञाननिबन्धनैवेति तावदुक्तं सर्वा सा\footnote{सर्वा सम्य० \cite{dp-msD}} सम्यग्ज्ञानपूर्विकेति । इतिशब्दस्तस्मादित्यस्मिन्नर्थे । यत्तदोश्च नित्यमभिसम्बन्धः । तदयमर्थः--\footnote{तस्मात् \cite{dp-edP}}यस्मात् सम्यग्ज्ञानपूर्विका सर्वपुरुषार्थसिद्धिः, तस्मात् तत्\footnote{तद्व्युत्पा० \cite{dp-msA} \cite{dp-edP} \cite{dp-edH} \cite{dp-edE}} सम्यग्ज्ञानं व्युत्पाद्यते । 
	  
	\footnote{ननु बहुव्रीहिणा सर्वपुरुषार्थसिद्धिरुच्यते । ततः प्राधान्यात् तच्छब्देन तत्सम्बन्धो युक्तो न तु ज्ञानस्येत्याह--\cite{dp-msD-n}}यद्यपि च\footnote{यद्यपि समा० \cite{dp-msC}} समासे गुणीभूतं सम्यज्ञानं तथापीह प्रकरणे व्युत्पादयितव्यत्वात् प्रधानम् । ततस्तस्यैव तच्छब्देन सम्बन्धः । 
	  
	\footnote{एवं मन्यते--विप्रतिपत्तिनिराकरणद्वारेण सम्यग् ज्ञाप्यतेऽनेन प्रकरणेनेति वक्ष्यमाणलक्षणयुक्तं सम्यग्ज्ञानं नान्यलक्षणयुक्तमिति स्वरूपकथनम् । तामेव विप्रतिपत्तिं क्रमेण दर्शयति--\cite{dp-msD-n}}व्युत्पाद्यते इति विप्रतिपत्तिनिराकरणेन प्रतिपाद्यते\footnote{प्रतिपाद्यते व्युत्पाद्यत इति \cite{dp-edE} \cite{dp-edP}} इति ॥” वृत्तौ सति सर्वा कृत्स्ना पुरुषार्थसिद्धिः सम्यग्ज्ञानपूर्विका, न तु काचिद् व्यक्तिरस्ति या न तत्पूर्विकेत्ययमर्थो लभ्यते, ततः कारणात् \textbf{पुरुषार्थसिद्धिः सम्यज्ञाननिबन्धनैवेति} सावधारणं वाक्यम् । यत्परिमाणमभिधेयं ब्रूयाद् वक्तुं शक्नोति, \textbf{तावत्प}रिमाणमुक्तम् \textbf{सर्वा सम्यज्ञानपूर्विके}त्यनेन । वर्त्तता\textbf{मितिशब्दस्तस्मादित्यस्यार्थे} । क्व पुनर्यत् शब्दोऽस्ति येन वक्ष्यमाणार्थसङ्गतिरित्याह--\textbf{यद्} इति । \textbf{चो} \leavevmode\marginnote{\textenglish{15b/ms}} व्यक्तमेतदित्यस्मिन्नर्थे । शब्दयोः साक्षादन्योन्यापेक्षाभावात् यत्तदर्थयोरिति द्रष्टव्यम् । \textbf{अभिसम्बन्धो} व्यपेक्षा । एवं सति कीदृशोऽर्थो व्यवतिष्ठत इत्याह--\textbf{तद्} इति ।
	\pend
      

	  \pstart ननु बहुव्रीहौ गुणीभूतं सम्यग्ज्ञानं तत्कथं तस्याप्रधानस्य प्रधानप्रत्यवमर्शिना तच्छब्दे परामर्शः स्यादित्याह--\textbf{यद्यपि चेति} । यद्यपि चेति निपातसमुदायो विशेषाभिधानार्थाभ्युपगमे बर्त्तते । \textbf{महाभाष्ये} चायं प्रायेण दृश्यते । \textbf{समास} इति समासार्थ इत्यर्थः । \textbf{व्युत्पादयितव्यत्वात्प्रधानमि}त्यभिदधतोऽयमाशयः--द्वेधा हि प्राधान्यं शब्दतोऽर्थतश्च । तत्र शाब्देन न्यायेनासत्यपि प्राधान्ये सम्यग्ज्ञानस्य व्युत्पादयितव्यतया बुद्ध्यन्तरेणोपस्थापितस्य स्वतन्त्रस्याऽऽर्थेन न्यायेन प्राधान्यं कोऽपहस्तयेदिति \textbf{तच्छब्देन} तस्य परामर्शो न विरुद्ध्यत इत्येवं व्याचक्षाणो \textbf{विनीतदेव}व्याख्यां तिरस्करोति । एवं ह्यसौ व्याचष्टे--“तदिति नपुंसकलिङ्गेन निर्देशात् सम्यग्ज्ञानं परामृश्यत” इति । सम्यग्ज्ञानस्याप्राधान्येन परामर्शानुपपत्तौ तल्लिङ्गग्रहणस्यैवायोगात् कथमसिद्धेन साध्यतामित्याशयः । ततः प्राधान्यात्तस्यैव सम्यग्ज्ञानस्य  \leavevmode\marginnote{\textenglish{35/dm}} “
	  
	चतुर्विधा चात्र विप्रतिपत्तिः संख्या-लक्षण-गोचर-फलविषया । तत्र संख्याविप्रतिप्रत्तिं निराकर्त्तुमाह-- “
	  
	द्विविधं सम्यग्ज्ञानम् ॥ २ ॥” 
	  
	द्विविधम्\footnote{“द्विविधम्--इति” इति नास्ति \cite{dp-edH}} इति--द्वौ\footnote{द्वे विधे \cite{dp-msB}} विधौ प्रकारावस्येति द्विविधम् । संख्याप्रदर्शनद्वारेण च व्यक्तिभेदो दर्शितो भवति ।” तच्छब्देन सम्बन्धनं \textbf{सम्बन्धः} स्वीकार इति यावत् । ततोऽयमर्थः--तत् सम्यग्ज्ञानं कर्मभूतं \textbf{विप्रतिपत्तिनिराकरणेन प्रतिपाद्यते} बोध्यते शिष्यजन इत्यर्थात् । अत एव तदिति द्वितीयान्तमेतदिति सोपपत्तिकमाह ।\footnote{“स्मार्यते समयं परः” \href{http://http://sarit.indology.info/?cref=pv.4.267}{प्रमाणवा० ४. २६७}--सं० ।} “समयं स्म\footnote{स्मा}र्यते परः” \href{http://http://sarit.indology.info/?cref=pv.4.267}{प्रमाणवा०--४. २६७} इति यथा । \textbf{विप्रतिपत्तिनिराकरणेन प्रतिपाद्यत} इति ब्रुवता लक्षणं नं विरुध्यते । “प्रसिद्धानि प्रमाणानि” \href{http://http://sarit.indology.info/?cref=nā.2}{न्याया० २} इत्यादि तदपहस्तितं वेदितव्यम् ।
	\pend
      

	  \pstart ननु \textbf{वार्त्तिका}दिनैव सम्यग्ज्ञानस्य व्युत्पादनात् कथमस्य न वैयर्थ्यमिति चेत् संक्षिप्तरुचीन् प्राज्ञानधिकृत्येदं प्रकरणं प्रणीतमित्यदोषः ।
	\pend
      

	  \pstart अथ काऽत्र विप्रतिपत्तिः यन्निराकरणेनेदं प्रतिपाद्यत इत्याह--\textbf{चतुर्विधे}ति । \textbf{चो} यस्मात् । \textbf{चतुर्विधा} चतुःप्रकारा । \textbf{अत्र} सम्यग्ज्ञाने विरुद्धा प्रतिपत्ति\textbf{र्विप्रतिपत्तिः} । किंविषया सेत्याह--\textbf{संख्ये}ति । अनेन विषयभेदाच्चातुर्विध्यं विप्रतिपत्तेर्दर्शितम् । तथाहि संख्याविप्रतिपत्तिस्तावत् प्रत्यक्षमेवैकं प्रमाणमिति \textbf{लोकायतिकानाम्} । प्रत्यक्षानुमानाप्तवचनानि त्रीण्येव प्रमाणानीति \textbf{सांख्यानाम्} । आर्षस्य ज्ञानस्य प्रतिभापरनाम्नः कदाचिदिह लौकिकानामुत्पद्यमानस्य प्रत्यक्षानुमानयोरेकत्राप्यन्तर्भावाप्रदर्शनात्प्रत्यक्षानुंमानार्षाण्येवेति \textbf{चिरन्तनवैशेषिकाणाम्} । प्रत्यक्षानुमानोपमानशब्दा एवेति \textbf{नैयायिकानाम्} । प्रत्यक्षानुमानोपमानशब्दार्थापत्तय एवेति \textbf{प्राभाकराणामिति} । प्रत्यक्षानुमानोपमानशब्दार्थापत्त्यभावा एवेति \textbf{कौमारिलानाम् ।}
	\pend
      

	  \pstart लक्षणविप्रतिपत्तिरपि--सविकल्पकमेव प्रत्यक्षमिति \textbf{वैयाकरणबार्हस्पत्यादीनाम्} । सविकल्पकं निर्विकल्पकं चेति \textbf{नैयायिक} \leavevmode\marginnote{\textenglish{16a/ms}} \textbf{मीमांसकादीनाम्} । पीतशङ्खादिज्ञानं भ्रान्तमपि प्रत्यक्षमित्यंशसंवादवादिनाम् । एकलक्षणहेतुजमनुमानमित्य\textbf{ह्रीकाणाम्} । षड्लक्षणहेतुजमिति \textbf{पूर्वयौगानाम्} । पञ्चलक्षणहेतुजं चतुर्लक्षणहेतुजं चे\footnote{वे}ति \textbf{नैयायिकानाम्} ।
	\pend
      

	  \pstart विषयविप्रतिपत्तिरपि--सामान्यविषये प्रत्यक्षानुमाने इति \textbf{नैयायिकमीमांसकादीनाम्} ।
	\pend
      

	  \pstart फलविप्रतिपत्तिरपि--सर्वेषां प्रमाणा\footnote{ण}व्यतिरिक्तमेव प्रमाणरू\footnote{प्रमारू}पं फलमिति ।
	\pend
      

	  \pstart तत्र तासु मध्ये संत्ख्यात्वजात्या निर्धार्यते । ततो निर्धारणे युक्तं \textbf{द्वौ विधौ प्रकारावस्येति} विगृह्णन् विधशब्दोऽप्यस्ति प्रकारवाचीति दर्शयति । तथा हि \textbf{प्रकीर्णवृत्ति}कृद्\textbf{धर्मपालेनापि}  \leavevmode\marginnote{\textenglish{36/dm}} “
	  
	द्वे\footnote{द्वे सम्य० \cite{dp-msA} \cite{dp-edP} \cite{dp-edE}} एव सम्यग्ज्ञानव्यक्ती इति । \footnote{ननु सम्यग्ज्ञानलक्षणे कथिते सति पश्चात् तद्भेदः प्रदर्शयितुं युज्यत इत्याह--\cite{dp-msD-n}}व्यक्तिभेदे च\footnote{व्यक्तिभेदे प्रद० \cite{dp-msA} \cite{dp-edP} \cite{dp-edH} \cite{dp-edE} \cite{dp-edN} भेदे प्रद० \cite{dp-msB} व्यक्तिभेदे च दर्शिते \cite{dp-msD}} प्रदर्शिते प्रतिव्यक्तिनियतं सम्यग्ज्ञानलक्षणमाख्यातुं शक्यम् । अप्रदर्शिते तु व्यक्तिभेदे सकलव्यक्तत्यनुयायि सम्यग्ज्ञानलक्षणमेकं न शक्यं वक्तुम् । ततो लक्षण\footnote{लक्षणभेदकथ० \cite{dp-msA} \cite{dp-msB} \cite{dp-msD} \cite{dp-edP} \cite{dp-edH} \cite{dp-edE} \cite{dp-edN}} कथनाङ्गमेव संख्याभेदकथनम् । अप्रदर्शिते \footnote{च \cite{dp-msC} अप्रदर्शिते व्य० \cite{dp-msA} \cite{dp-edE}}तु व्यक्तिभेदात्मके संख्याभेदे लक्षणभेदस्य दर्शयितुमशक्यत्वात् । लक्षणनिर्देशाङ्गत्वादेव च प्रथमं संख्याभेदकथनम् ॥” विधशब्दः प्रकारवाची प्रदर्शितः । न पुनरस्यायमभिप्रायः--विधा\footnote{ध}शब्दो जातिवाचित्वात्प्रकारवाची न भवतीति । अनेकार्थत्वात्तस्य प्रकारवाचिनोऽपि प्रयोगस्य “चतसृषु चैवंविधासु तत्त्वं परिसमाप्यते” \href{http://http://sarit.indology.info/?cref=nbh.1.1.1_p3}{न्यायभा० पृ० २} इत्यादावनेन प्रायशो दृष्टत्वात् ।
	\pend
      

	  \pstart ननु कथं नाम प्रमाणादेव प्रवर्त्तेय नाप्रमाणादिति प्रवर्त्तितुमनाः प्रमाणस्य लक्षणमेव जिज्ञासते न सङ्ख्याम् । ततो लक्षणमेव व्युत्पाद्यं न सङ्ख्या । तच्च लक्षणं यद्येकस्यैवास्ति तदेकमेव प्रमाणम् । अथ बहूनां तदा प्रमाणबाहुल्यमनिवार्यमेव । अथ तल्लक्षणं नैकस्यैवास्ति, नापि बहूनाम् । तर्हि लक्षणव्युत्पत्तौ सामर्थ्यात्संख्याविप्रतिपत्तिर्निराकृता भवतीति किं पृथक् संख्याविप्रतिपत्तिनिराकरणेनेत्याह--\textbf{संख्ये}ति । \textbf{चो} यस्मादर्थे । \textbf{द्वार}मुपायः ।
	\pend
      

	  \pstart केनाकारेण दर्शितो भवतीत्याह--द्वे इति । \textbf{इति}ना व्यक्तिभेदस्य स्वरूपमाह । व्यक्तिभेदेनैव दर्शितेन किं प्रयोजनमित्याह--\textbf{व्यक्तिभेदे चे}ति । \textbf{चो} यस्मादर्थे । यद्यदर्शितेऽपि व्यक्तिभेदे लक्षणाख्यानं शक्यं तथापि किं दर्शितेनेत्यन्वयमात्रादप्रतिपत्तेर्व्यतिरेकमपि दर्शयितुमाह—\textbf{अप्रदर्शित} इति । \textbf{तुः} प्रदर्शनपक्षादप्रदर्शनपक्षस्य भेदमाह । \textbf{सकलव्यक्त्यनुयायी}ति ब्रुवतोऽयं भावः--व्यक्तिभेदानुपदर्शने प्रतिव्यक्तिनियतस्य लक्षणस्याख्यातुमशक्यत्वात् । लक्षणमुच्यमानं सकलव्यक्त्यनुयायि तदेकं वक्तव्यम् । न च तद् वक्तुं शक्यमसम्भवादेवेति ।
	\pend
      

	  \pstart ननु किमुच्यते सकलव्यक्त्यनुयायि वक्तुमशक्यमिति यावताऽविसंवादिज्ञानं प्रमाणमित्यस्ति लक्षणमेकं सुवचमपीति । सत्यम् । केवलं ज्ञानानां यत्प्रातिस्विकं रूपं प्रवृत्तिकामानां प्रवृत्त्युपयोगि, तदुपलक्षणं नास्तीत्यभिप्रायाददोषः । यद्वा विप्रतिपत्तिनिराकरणप्रवणं यत्साधारणं लक्षणं तन्नास्ति । यच्चाविसंवादित्वं लक्षणं न तेन विप्रतिपत्तिर्निराकृता भवति । अन्यत्रापि परैरविसंवादित्वस्येष्टत्वादित्यनेनाभिप्रायेणोक्तम्--\textbf{न शक्यमेकं वक्तुमिति}विशेषप्रतिषेधसामर्थ्यात् शेषविधिसिद्धौ च तथाभूतस्य साधारणस्य सम्भवादेव \textbf{सकल}शब्दोऽयमुपयुक्तकार्त्स्न्ये प्रवर्त्त\leavevmode\marginnote{\textenglish{16b/ms}}नीयः । तेनायमर्थः--प्रदर्शितव्यक्तिभेदात्मकचतुर्विधप्रत्यक्षानुयायि कल्पनाऽपोढाभ्रान्तत्वम्, प्रदर्शितव्यक्तिभेदात्मकद्विविधानुमानानुयायि त्रिरूपलिङ्गजत्वं शक्यं वक्तुमिति । यदि नामैवं ततः किमित्याह--\textbf{तत} इति । यतोऽप्रदर्शिते न शक्यमेकं तथाविधं दर्शयितुं \textbf{तत}स्तस्मात् । \textbf{लक्षणकथनाङ्ग}मिति प्रतिव्यक्तिनियतलक्षणकथनाङ्गमित्यवसेयम् ।
	\pend
      \leavevmode\marginnote{\textenglish{37/dm}}“

	  \pstart किं पुनस्तद् द्वैविध्यमित्याह--
	\pend
       “

	  \pstart प्रत्यक्षमनुमानञ्चेति\footnote{नास्ति “इति” पदं \cite{dp-msB} \cite{dp-edP} \cite{dp-edH} \cite{dp-edE} \cite{dp-edN} प्रतिषु किन्तु विद्यते तत्पदं \cite{dp-msC} \cite{dp-msD} प्रतयोः ।} ॥ ३ ॥
	\pend
      ””

	  \pstart अथ तथाविधं लक्षणमेकं वक्तुं न शक्यतां नाम । किमतः ? केवलमदर्शितेऽपि संख्याभेदे लक्षणभेदो दर्शयितुं शक्यताम् । तत् कस्मात्तत्कथनाङ्गं संख्याभेदकथनमित्याशङ्क्यान्वयमुखेनोक्तमर्थं द्रढयितुं व्यतिरेकमुखेणाह--\textbf{लक्षणेति । लक्षणभेदस्य} प्रतिव्यक्तिभिन्नस्य दृष्टस्य । कदा दर्शयितुमशक्यत्वं यत एवं भवतीत्याकाङ्क्षायाम्--\textbf{अप्रदर्शित} इति पश्चाद् योजनीयम् । तुरवधारयति, प्रदर्शिताद् वा भिनत्ति ।
	\pend
      

	  \pstart ननु च न संख्याभेद एव व्यक्तिभेदः । तत् कथं \textbf{व्यक्तिभेदात्मक} इत्युच्यत इति चेत् । वस्तुतः संख्येयादन्यस्याः संख्याया वास्तव्या अभावेन संख्यासंख्येययोरेकत्वविवक्षया चैवमुच्यते । पूर्वं तु कल्पनानिर्मितात्मनां\footnote{ना}भिन्नामवलम्ब्य \textbf{संख्याप्रदर्शनद्वारेणे}त्युक्तम् । संख्यासंख्येययोरेकत्वविवक्षयैव च \textbf{संख्याविप्रतिपत्तिं निराकर्त्तुमाहे}त्युक्तमवसेयम् । परमार्थतस्तु \textbf{प्रत्यक्षमनुमानं} चेत्युक्तेः संख्यासंख्येयविप्रतिपत्तिरेव निराकृतेति । अथवा वस्तु\footnote{व्यक्ति} भेदस्तदात्मा तथाविधः कल्पनाशिल्पिनिर्मितो यस्य संख्याभेदस्य स तथोक्तः । परमार्थतो भिन्ने हि वस्तुनि संख्याभेदः कल्प्यत इति वास्तवरूपानुवादमिदमुक्तमिति किमवद्यम् ? सर्वेण चानेन नै\footnote{चानेनै}तदुक्तम्--प्रवृत्तिकामानामुपयोगित्वाल्लक्षणमेव वक्तुं युक्तं तदेतत्त्वन्यथा न शक्यमाख्यातुमिति संख्याभेदप्रदर्शनमिति ।
	\pend
      

	  \pstart नन्वसत्यपि \textbf{द्विविध}मिति संख्याभेदप्रदर्शने तत् सम्यग्ज्ञानं \textbf{प्रत्यक्षमनुमानं} चेत्युक्तेऽपि व्यक्तिभेदो दर्शितो भवत्येव । सति चैवं लक्षणभेदाख्यानमपि सुकरमिति कृतं \textbf{द्विविध}शब्देनेति चेत् । सिद्धे सतीदं वचनं द्विविधमेवेति नियमार्थमिति ब्रूमः । इतरथेह तावदेतावद्व्युत्पाद्यतया प्रस्तुतमन्यत्र पुनरन्यदप्यस्ति व्युत्पाद्यं सम्यग्ज्ञानमित्याशङ्काहत्य न निराकृता स्यादिति । यत्पुनरुक्तम्--तर्हि लक्षणव्युत्पत्त्यैव संख्याविप्रतिपत्तिर्निराकृता भवतीति तदत्यन्तमसङ्गतम् । यतो यदि विशेषलक्षणमभिप्रेत्येदमुच्यते; तदा व्यक्तिभेदानुपदर्शने प्रतिव्यक्तिनियतं तदेवाशक्यं\footnote{य}कथनमित्युक्तम् । न चोक्तमिति तथा कर्त्तुमीष्टे । यतस्तदीदृशं प्रत्यक्षमनुमानं चेत्येतत्कर्त्तुं शक्नोति । न त्वाभ्यामन्यन्न सम्यग्ज्ञानमिति । अथ सामान्यलक्षणम् । तदपि तथाविधं साधारणं वक्तुमशक्यमिति केन तथा क्रियताम् ? न च तेनोक्तेनापि तादृशी विप्रतिपत्तिर्निराक्रियत इत्युक्तम् । अथापि स्यात्--द्विविधमित्युक्तेऽपि कथ \leavevmode\marginnote{\textenglish{17a/ms}} माभ्यामन्यस्यासम्यग्ज्ञानत्ववि\footnote{त्वाधि}गमः? उच्यते । ये तावदाचार्यप्रमाणका व्युत्पित्सवस्ते तद्वचनमात्रेणैव विवक्षितमर्थं बोध्यन्ते । ये ते युक्त्यनुसारिणस्तेभ्योऽपि प्रकरणान्तरेषूपपत्तिरुदाहृता । प्राज्ञजनाधिकारेण चास्य प्रकरणस्य प्रारम्भात्स्वयमेव तैरुपपत्तिरुक्तेति न संख्यावचनमनुपादेयमिति । अस्तु लक्षणनिर्देशाङ्गम्, आदौ तु कस्मादुच्यत इत्याह--\textbf{लक्षण} इति । चस्तुल्योपायत्वं समुच्चिनोति ।
	\pend
      

	  \pstart यथाकथञ्चिद् द्वैविध्यसम्भवे पृच्छति--\textbf{किं पुनरि}ति । \textbf{किमि}ति सामान्यतः, \textbf{पुन}रिति \leavevmode\marginnote{\textenglish{38/dm}} “
	  
	प्रत्यक्षमिति । प्रतिगतमाश्रितम् अक्षम् । “अत्यादयः क्रान्ताद्यर्थे द्वितीयया” \href{http://http://sarit.indology.info/?cref=htu.90}{वा० २. २. १८.} इति समासः । प्राप्तापन्नालङ्गति\footnote{उपसर्ग--\cite{dp-msD-n}} समासेषु परवल्लिङ्गप्रतिषेधाद् अभिधेयवल्लिङ्गे सति सर्वलिङ्गः प्रत्यक्षशब्दः सिद्धः । \footnote{ननु चास्यां व्युत्पत्तौ इन्द्रियज्ञानस्यैव प्रत्यक्षशब्दवाच्यता स्याद् न योगिज्ञानादेरित्याह--\cite{dp-msD-n}}अक्षाश्रितत्वं च व्युत्पत्तिर्निमित्तं शब्दस्य । न तु प्रवृत्तिनिमित्तम् । \footnote{इन्द्रियज्ञानेऽक्षाश्रितत्वमर्थसाक्षात्कारित्वं च समवेतम्--\cite{dp-msD-n}}अनेन \footnote{अनेन त्वलक्षाश्रि० \cite{dp-msA}}त्वक्षाश्रितत्वेनैका \footnote{एकस्मिन्निन्द्रियज्ञाने--\cite{dp-msD-n}}र्थसमवेत \footnote{एकार्थसम्बद्धम्--\cite{dp-msD-n}}मर्थसाक्षात्कारित्वं \footnote{लभ्यते--\cite{dp-msB} \cite{dp-edN}}लक्ष्यते । तदेव शब्दस्य प्रवृत्तिनिमित्तम् । ततश्चा यत्किञ्चिदर्थस्य साक्षात्कारि ज्ञानं तत् प्रत्यक्षमुच्यते ।” विशेषतः । अभिमतमाह--\textbf{प्रत्यक्षम्} इति । \textbf{प्रत्यक्ष}मित्यत्र कः समासः, केन च सूत्रेणेत्याशङ्क्य सुखप्रतिपत्त्यर्थं विप्रतिपत्तिनिराकरणार्थं चाह \textbf{प्रतिगतम्} इत्यादि । \textbf{अक्ष}मिन्द्रियं \textbf{गत}माश्रितं जन्यतया, न त्वाधेयभावेन । एवं च विगृह्णन् अक्षमक्षं प्रति वर्त्तत इति व्युत्पत्तिप्रकारेणाव्ययीभावं निरस्यति । \textbf{समास} एकार्थीभावः । स चायं समासस्तत्पुरुषसंज्ञको ज्ञातव्यः । केन सूत्रेणायमित्याह--\textbf{अत्यादय} इति । आहोपुरुषिकयाऽयं प्रकारस्त्वया दर्शितो न त्वस्य कश्चिद् गुणोऽस्तीत्याह--\textbf{सर्वलिङ्ग} इति । अव्ययीभावपक्षे प्रत्यक्षो वृक्षः प्रत्यक्षा मृगाश्चेति न स्यात् । प्रत्यक्षस्येति श्रुतिश्च न स्यादित्यभिप्रायः । कथं पुनः सर्वलिङ्गताऽस्येत्याह--\textbf{अभिधेयवदि}ति । अभिधेयस्येवाभिधेयवत् । अभिधेयवल्लिङ्गं यस्य तत्तथा तस्मिंश्च सत्ययं प्रत्यक्षशब्दः सर्वलिङ्गः ।
	\pend
      

	  \pstart ननु च “परवल्लिङ्गं द्वन्द्वतत्पुरुषयोः” \href{http://http://sarit.indology.info/?cref=Pā.2.4.26}{पाणिनि--२. ४. २६} इति समस्तेन प्रत्यक्षशब्देन प्रतिशब्दो\footnote{ब्दा}पेक्षया परस्थानस्थितस्याक्षशब्दस्य लिङ्गपरिग्रहात्कथमस्याभिधेयवल्लिङ्गतेत्याह--\textbf{प्राप्तेति} ।
	\pend
      

	  \pstart ननु केनास्य परवल्लिङ्गनिषेधः ? न तावत् प्राप्तापन्नालङ्गतिसमासेन तेषां स्वरूपग्रहणात्तत्र । अत्र च तदभावात् । गतिसमासादिति चेत् । तदप्यसत्, “उपसर्गाः क्रियायोगे” \href{http://http://sarit.indology.info/?cref=Pā.1.4.59}{पाणिनि--१. ४. ५९} इत्यतः क्रियायोगे वर्त्तमाने “गतिश्च” \href{http://http://sarit.indology.info/?cref=Pā.1.4.60}{पाणिनि--१. ४. ६०} इत्यनेन क्रियायोग एव गतिसंज्ञाविधानात् । न चायं प्रतिशब्दः क्रियायोगी । न च गमनक्रियायोगोऽत्रास्तीति वाच्यम् । प्रतिशब्दस्यैव तत्रार्थे वर्त्तमाना\footnote{नत्वा}त् नायं दोषः, गतिग्रहणेन तत्र येषां गतिसंज्ञा दृष्टा तेषां ग्रहणात् । प्रत्यादीनां सा दृष्टेति तेषामपि तथात्वेन सङ्ग्रह इति ।
	\pend
      

	  \pstart यद्येवं समासस्तर्हि कथ\textbf{माचार्यदिग्नागेन} “अक्षमक्षं प्रति वर्त्तत इति प्रत्यक्षम्” \add{न्यायमुख} इत्युक्तम् ? तदर्थमात्रं कथितमित्यदोषः ।
	\pend
      \leavevmode\marginnote{\textenglish{39/dm}}“

	  \pstart यदि त्वक्षाश्रितत्वमेव प्रवृत्तिनिमित्तं स्याद् इन्द्रियविज्ञानमेव\footnote{इन्द्रियज्ञान० \cite{dp-msA} \cite{dp-edP} \cite{dp-edH} \cite{dp-edE}} प्रत्यक्षमुच्येत, न मानसादि । यथा “गच्छतीति गौः” इति गमनक्रियायां व्युत्पादितोऽपि गोशब्दो गमनक्रियोपलक्षितमेकार्थसमवेतं गोत्वं\footnote{जातिः--\cite{dp-msD-n}} प्रवृत्तिनिमित्तीकरोति । तथा च गच्छत्यगच्छति च गवि गोशब्दः सिद्धो भवति ।
	\pend
       

	  \pstart मीयतेऽनेनेति मानम् । करणसाधनेन मानशब्देन\footnote{सामान्यलक्षण०--\cite{dp-msD-n} अर्थसारूप्य०--\cite{dp-msD-n}} सारूप्यलक्षणं प्रमाणमभिधीयते । लिङ्गग्रहण सम्बन्धस्मरणस्य पश्चात् मानम् अनुमानम् । गृहीते\footnote{गृहीतपक्ष० \cite{dp-msB}} पक्षधर्मे स्मृते च साध्यसाधनसम्बन्धेऽनुमानं प्रवर्त्तत इति पश्चात्कालभाव्युच्यते ।
	\pend
      ”

	  \pstart अथ प्रतिगतमाश्रितमक्षमित्यस्यामपि व्युत्पत्तौ मानस-स्वसंवेदन-योगिप्रत्यक्षाणां न स्यात्प्रत्यक्षशब्दवाच्यतेत्याह--\textbf{अक्षाश्रितत्वम्} इति । \textbf{चो} यस्मात् । प्रकृत्यादिविभागेन शब्दस्य निष्पत्ति\textbf{र्व्युत्पत्तिः । प्रवृत्ति}रर्थाभिधानम् \textbf{शब्दस्ये}ति प्रकृतस्य प्रत्यक्षशब्दस्य । तुरवधारणे । किं तर्हि प्रवृत्तिनिमित्तमित्याह--\textbf{अनेन}--इति । तुना व्युत्पत्तिनिमित्तादस्य भेदं द\leavevmode\marginnote{\textenglish{17b/ms}}र्शयति । यत्तदोर्नित्यमभिसम्बन्धात् यदक्षाश्रितत्वेनार्थसाक्षात्कारित्वमर्थापरोक्षीकरणमुपलक्ष्यते तदेव तु प्रत्यक्षशब्दस्य प्रवृत्तिनिमित्तम् । कथं पुनरन्येनासम्बद्धेनान्यस्योपलक्षणम् ? मा भूदि\footnote{द}तिप्रसक्तिरित्याह--\textbf{एकार्थसमवेतम्} इति हेतुभावेन विशेषणात् । यत एकार्थसमवेतं ततो लक्ष्यत इत्यर्थः । एकस्मिन्नर्थे ज्ञानलक्षणेऽर्थसाक्षात्कारित्वमक्षाश्रितत्वेन समं समवेतम् । यद्यपि परमार्थतः समवायसमवायिनौ न स्तस्तथापि तद्व्यावृत्तिनिबन्धनयोस्तथाभूतवस्त्वाश्रयेण कल्पितयोरक्षाश्रितत्वार्थसाक्षात्कारित्वयोरेकार्थसमवायस्य च कल्पितस्य सम्भवादेवमभिधाने को विरोधः ? परप्रसिद्ध्या वा एवमुक्तमिति का क्षतिः ? एवमपि कथं पूर्वदोषातिवृत्तिरित्याह--\textbf{ततश्चे}ति । साक्षात्कारित्वस्य प्रवृत्तिनिमित्तत्वात् । \textbf{च}शब्द एव तस्मादिदमुच्यत इत्यर्थं द्योतयति ।
	\pend
      

	  \pstart अक्षाश्रितत्वे प्रवृत्तिनिमित्ते को दोष इति पार्श्वस्थस्याश्रुतचोदकवाक्यस्य वचनमाशङ्क्याह--\textbf{यदि}--इति । \textbf{तु}ना भेदवदेतद्दर्शयति ।
	\pend
      

	  \pstart अथाभिधीयते--यदर्थसाक्षात्करणमक्षाश्रितत्वेन समवेतं तदेवानेनोपलक्षणीयमिति तदवस्थे\footnote{स्थो}मानसादेरसङ्ग्रह इति । न । अक्षजत्वस्योपलक्षणत्वेनाप्राधान्यादर्थसाक्षात्कारित्वमुपलक्ष्यमाणं प्रधानमुपलक्ष्यैव निवृत्तेरदोष एव । यथा काकेभ्यो दधि रक्ष्यतामित्यत्र । अन्यथाऽत्रापि काकत्वेनैकार्थसमवेतस्योपधातकत्वस्योपलक्षणात्, श्वादिभ्यो दधिरक्षा न स्यात् । अथ यदेवोपधातकं\footnote{कत्वं} काकेषु समवेतं तदेवान्यत्रापि । तच्चोपलक्षितमिति तदन्यतोऽपि दधिरक्षोच्यते । अर्थसाक्षात्कारित्वेऽपि समानमिदमिति कथं न मानसादेः प्रत्यक्षशब्दाभिलाप्यत्वमिति । किं दृष्टमिदं यदन्यत्र व्युत्पादितोऽपि शब्दोऽन्यत्र तत्प्रवृत्तिविषयीकरोतीत्याह--\textbf{यथे}ति । सुगममेतत्, केवल\textbf{मपि}रवधारणे \textbf{गोत्वम} \footnote{मित्य} स्मात्परो द्रष्टव्यः ।
	\pend
      

	  \pstart प्रत्यक्षमूलत्वादनुमानस्यादौ प्रत्यक्षमुपात्तं प्रत्येयम् । यत्र हि निमित्तान्तरं दर्शयितु  \leavevmode\marginnote{\textenglish{40/dm}} “
	  
	चकारः प्रत्यक्षानुमानयोस्तुल्यबलत्वं समुच्चिनोति । यथार्थाविनाभावित्वादर्थं प्रापयत् प्रत्यक्षं प्रमाणम्, तद्वदर्थाविनाभावित्वाद् अनुमानमपि परिच्छिन्नमर्थं प्रायपत् प्रमाणमिति ॥ “
	  
	तत्र प्रत्यक्षं कल्पनाऽपोढमभ्रान्तम् ॥ ४ ॥” 
	  
	तत्रेति सप्तम्यर्थे वर्त्तमानो निर्धारणे वर्त्तते । ततोयं वाक्यार्थः--तत्र तयोः प्रत्यक्षानमानयोरिति समुदायनिर्देशः । प्रत्यक्षम् इत्येकदेशनिर्देशः\footnote{इत्येकदेशः \cite{dp-msA} \cite{dp-msC} \cite{dp-edP} \cite{dp-edE} \cite{dp-edH}} । तत्र समुदायात् प्रत्यक्षत्वजात्यैकदेशस्य पृथक्करणं निर्धारणम् । तत्र प्रत्यक्षमनूद्य\footnote{प्रत्यक्षत्वमनू० \cite{dp-msB} \cite{dp-msD} \cite{dp-edP} \cite{dp-edE} \cite{dp-edH} \cite{dp-edN}} कल्पनाऽपोढत्वम्, अभ्रान्तत्वं च\footnote{त्वं विधी० \cite{dp-msB}} विधीयते । यत्\footnote{ननु प्रत्यक्षस्याद्याप्यसिद्धत्वात् कथमनूद्यत्वं सम्भवतीत्याह--\cite{dp-msD-n}} तत् भवताम् अस्माकं चार्थेषु साक्षात्कारि ज्ञानं प्रसिद्धं तत् कल्पनाऽपोढाभ्रान्तत्वयुक्तं द्रष्टव्यम् ।” मशक्यं तत्र क्रमप्रवृत्तित्वाद् वाचः, क्रमस्य न्यायप्राप्तत्वादित्युच्यते, न तु सत्यपि निमित्तान्तरे यथोद्देशमेव व्याख्यानं युक्तमिति ।
	\pend
      

	  \pstart प्रत्यक्षपदं व्याख्यायाधुनाऽनुमानपदं व्याचिख्यासुः \textbf{मान}शब्दस्य तावदर्थमाचष्टे \textbf{मीयत} इति । एवं व्युत्पादितेनानेन शब्देन किं प्रतिपाद्यत इति ? \textbf{करणे}ति । \textbf{करण}स्यानुमानलक्षणक्रियासिद्धौ साधकतमस्य \textbf{साधनेन} वाचकेन । \textbf{सारूप्य\footnote{प्यं}लक्षणं} स्वभावो यस्य तत्तथा । हेतुभावेन चैतद् विशेषणम् ।
	\pend
      

	  \pstart यद्येवं कथं तर्हि क्वचित् “त्रिरूपाल्लिङ्गाद् यदनुमेये ज्ञानं तदनुमानम्” इति ? सारूप्योपलक्षितं ज्ञानमेव तथाऽभिधास्यत इति को विरोधः ? एवं तावन्मानम् । अनुमानं तु कथमित्याह--\textbf{पश्चाद्} इति । अनेन पश्चादर्थवृत्तेनानुशब्दस्य मानशब्देन सह “गतिप्रादय” इत्यनेन तत्पुरुषसमासं दर्शयति । न च मानस्य पश्चादिति \leavevmode\marginnote{\textenglish{18a/ms}} विवक्षितम्, येनानुरथादिवदव्ययीभावो भवेत् । अत्र हि मानमेव पक्षधर्मग्रहणादेः पश्चाद्भावि विवक्षितम् । न मानस्य पश्चाद्भावि किञ्चिदिति । अव्ययीभावपक्षे तु न केवलं विवक्षितार्थक्षतिः, किन्त्वनु मानस्येति षष्ठी च न श्रूयेत । कस्य पश्चादित्याह--\textbf{लिङ्गेति} । “षष्ठ्यतसर्थप्रत्ययेन” \href{http://http://sarit.indology.info/?cref=Pā.2.3.30}{पाणिनि--२. ३. ३०} इत्यनेन पश्चात् शब्दयोगे षष्ठीयम् । कथमेतद् द्वयस्य पश्चाद् भाव्यनुमानमित्याह--\textbf{गृहीत} इति । \textbf{तुल्यबलत्वं} तुल्यसामर्थ्यम् । प्रामाण्यासादनायेति प्रकरणवशात् । \textbf{यथे}त्यादिनैतदेव समर्थयते । अनुमानस्य त्वर्थाविनाभावित्वं पारम्पर्येण द्रष्टव्यम् । न चैवं प्रापणे प्रामाण्यावहे कश्चिद् विशेषः । \textbf{परिच्छिन्नमि}त्यध्यवसितम् । एवं ब्रुवतः प्रत्यक्षमप्यध्यवसितमेव सन्तानं प्रापयत् प्रमाणम् । इदमपि तथेति कथमस्य न तथात्वमिति भावः । तुल्यबलप्रदर्शनेऽप्ययं भावः । प्रत्यक्षस्यापि यदर्थाविनाभावित्वं तत् तदुत्पत्तिनिमित्तकमेव । तच्चानुमानस्यापि समानमिति अर्थाव्यभिचारेणापि निमित्तिना समानेन भाव्यमिति ।
	\pend
      \leavevmode\marginnote{\textenglish{41/dm}}“

	  \pstart न\footnote{ननु च तयोरप्रसिद्धत्वात् प्रत्यक्षस्याप्यप्रसिद्धिरेव, प्रत्यक्षस्यैतत्स्वभावत्वादित्याह--\cite{dp-msD-n}} चैतन्मन्तव्यम्--कल्पनाऽपोढाऽभ्रान्तत्वं चेदप्रसिद्धम्, किमन्यत् प्रत्यक्षस्य ज्ञानस्य रूपमवशिष्यते यत् प्रत्यक्षशब्दवाच्यं सद् अनूद्येतेति । यस्मादिन्द्रियान्वयव्यतिरेकानुविधाय्यर्थेषु साक्षात्कारिज्ञानं प्रत्यक्षशब्दवाच्यं सर्वेषां\footnote{०षां सिद्धं \cite{dp-msA} \cite{dp-msC} \cite{dp-edP} \cite{dp-edH} \cite{dp-edE}} प्रसिद्धम्, तदनुवादेन कल्पनाऽपोढाभ्रान्त-\footnote{पोढत्वाभ्रा० \cite{dp-msC} \cite{dp-msD}} त्वविधिः ।
	\pend
       

	  \pstart कल्पनाया अपोढम् अपेतं कल्पनापोढम् । कल्पना\footnote{सर्वासु प्रतिषु “कल्पनास्वभावरहितमित्यर्थः” इति पाठस्य सत्त्वेऽपि प्रदीपानुसारी पाठोऽत्र गृहीतः ।--सं० ।} स्वभावेन रहितमित्यर्थः । अभ्रान्त-
	\pend
      ”

	  \pstart अनेन “प्रत्यक्षमेकं प्रमाणम्” इति ब्रुवाणश्\textbf{चार्वाकः,} “अनुमानादर्थनिश्चयो दुर्लभः । कनिष्ठं च तत्प्रमाणम्” इत्याचक्षाणो \textbf{मीमांसकश्च} निरस्तः । सर्वत्रायम् \textbf{इति}र्वाक्यार्थपरिसमाप्तौ । यत्र त्वर्थविशेषे वर्त्तते स कथ्यत एव ।
	\pend
      

	  \pstart \textbf{तत्रे}त्यादि \textbf{निर्धारणमि}त्यन्तं सुगमम् । केवलमिन्द्रियज्ञानादिप्रत्यक्षानुयायित्वात्प्रत्यक्षत्वाख्या जातिर्या व्यवहारसिद्धा तयेति द्रष्टव्यम् ।
	\pend
      

	  \pstart तत्रेत्ययमेवं व्यवस्थिते सतीति वाक्योपन्यासे । एवं चानुवादविधिक्रमेण यदन्यदन्येन व्याख्यातम् “प्रत्यक्षमिति संज्ञा कल्पनापोढत्वादि संज्ञ्येव । तेन संज्ञासंज्ञिसम्बन्धः प्रतिपाद्यते” इति तद् विप्रतिपत्तिनिराकरणे प्रस्तुतेऽप्रस्तुतम् । ततोऽपव्याख्यानमिति प्रकाशयति । यदप्यपरेण व्याख्यातं “प्रदेशान्तरप्रसिद्धयोः कल्पनापोढत्वाभ्रान्तत्वयोरनुवादेन प्रत्यक्षत्वं विधीयते” इति तदपि न चतुरस्रमिति प्रतिपादयति । यतः प्रसिद्धे लक्ष्यलक्षणभावे लक्षणानुवादेन लक्ष्यं विधेयम्, अप्रसिद्धे तु लक्षणवाक्येन लक्षणमेव विधेयम्, न लक्ष्यमिति न्याय इति । अनयोश्च पक्षयोर्यावान् समर्थनदूषणप्रकारस्तावाननेनैव \textbf{विनिश्चयटीकायां} स्वयं विवेचित इति नेहोच्यते । कथं प्रत्यक्षत्वानुवादेन कल्पनापोढत्वादिविधानमित्याह--\textbf{प्रसिद्धमिति} । अनेन पूर्वं प्रसिद्धस्य पश्चाच्छब्देनाभिधानमनुवाद इति दर्शयति । तदेवं \textbf{द्रष्टव्यमि}ति विदधानोऽज्ञातस्य शब्देन ज्ञापनं विधिरिति दर्शयति ।
	\pend
      

	  \pstart \textbf{न चैत}दित्यादीतिशब्दान्तं सुबोधम् ।
	\pend
      

	  \pstart \textbf{इन्द्रियान्वयव्यतिरेकानुविधायि}ग्रहणं मानसाद्यसाधारणतयाऽनुपन्यासार्हमपि प्रत्यक्षशब्दप्रवृत्तिनिमित्तस्यार्थसाक्षात्कारित्वस्य यदुपलक्षकं तदुपदर्शनार्थमुपन्यस्तमिति ज्ञातव्यम् । परिपाट्या प्रसिद्धत्वप्रदर्शनार्थं वा । तथा हि प्रथमं यत् तद् इन्द्रियान्वयव्यतिरेकानुविधायितया प्रत्यक्षं प्रसिद्धमिति प्रतिपाद्यते । तदनु यदीदमिन्द्रियाश्रितं ज्ञानं भवतां प्रत्यक्षं प्रसिद्धं तत्रापि प्रत्यक्षश \leavevmode\marginnote{\textenglish{18b/ms}}ब्दप्रवृत्तावर्थसाक्षात्कारित्वमेव निमित्तं जानीते इति प्रतिपाद्यते । \textbf{तदनुवादेनेति} सुगमम् ।
	\pend
      

	  \pstart यत् कल्पनयाऽपोढं रहितं तत् कल्पनाया अपोढमपगतं भवतीत्यर्थाभेदं मत्वा \textbf{कल्पनाया अपोढमिति} कर्त्तरि निष्ठामाह । न तु कल्पनापोढमिति कर्त्तरि निष्ठैव । एवं हि \textbf{तया रहित}मित्या\textbf{चार्य}विवरणविरोधः स्यात् ।
	\pend
      \leavevmode\marginnote{\textenglish{42/dm}}“

	  \pstart मर्थक्रियाक्षमे वस्तुरूपेऽविपर्यस्तमुच्यते । अर्थक्रियाक्षमं च\footnote{०क्षमं वस्तु० \cite{dp-msA} \cite{dp-edP} \cite{dp-edE}} वस्तुरूपं सन्निवेशोपाधिवर्णात्मकम्\footnote{०धिधर्मात्म० \cite{dp-edP} \cite{dp-edH}} । तत्र यन्न भ्राम्यति तदभ्रान्तम् ।
	\pend
      ”

	  \pstart ननु किमेकस्मिन् ज्ञाने ज्ञानान्तरसम्भवोऽस्ति येनायं प्रतिषेधः शोभेत ? कल्पनाज्ञानेऽपि च कल्पनान्तरं नास्तीति तदपि कल्पनापोढं प्रसज्येतेति वचनावकाशं पश्यन् \textbf{कल्पनास्वभावे}ने\footnote{धर्मोत्तरे सर्वत्र प्रतिषु “कल्पनास्वभावरहितम्” इत्येव पाठो लभ्यते न तु “कल्पनास्वभावेन रहितम्” इति ।} त्यादिनाऽस्य विवक्षितमर्थं स्फुटयति ।
	\pend
      

	  \pstart एवञ्च ब्रुवन् कल्पनाऽपोढमित्यत्रा\textbf{चार्योये} लक्षणवाक्ये भावप्रधानोऽयं कल्पनाशब्दः, तेन कल्पनात्वेन रहितमित्यर्थ इति दर्शयति । धर्मिणा वा कल्पनाख्येन धर्मस्याभिलापसंसर्गयोग्यप्रतिभासत्त्वाख्यस्य निर्देशं दर्शयति । उभयथाऽपि “कल्पनास्वभावरहितमित्यर्थः” इत्यस्य स्फुटीकरणस्य घटनात् । कल्पना च वक्ष्यमाणलक्षणा ग्राह्या । एवं च शब्दार्थं स्फुटयताऽनेन यद् \textbf{आचार्यदिग्नागीय}प्रत्यक्षलक्षणदूषणावसरे--“कल्पना नामजात्यादियोजना । तेन यन्नाम्ना नाभिधीयते, जात्यादिना च न व्यपदिश्यते विषयभेदानुविधायि यज् ज्ञानं तत्प्रत्यक्षम्” \href{http://http://sarit.indology.info/?cref=nv.1.1.p44.1}{न्यायवा० पृ० ४१} इत्यर्थं परिकल्प्य “अथ प्रत्यक्षशब्देन कोऽर्थोऽभिधीयते । यदि प्रत्यक्षं कथमवाच्यम् ? अथ न प्रत्यक्षमवाचकस्तर्हि प्रत्यक्षशब्दः । प्रत्यक्षत्वसामान्याभिधानेऽपि यदि तद्व्यतिरेकि; न प्रत्यक्षमुक्तम् । अथाव्यतिरेकि । कथं नोक्तम् ? कल्पेनापोढशब्देनापि यदि तदुच्यते कथं न व्याघातः ? अथ नोच्यते तस्योच्चारणवैयर्थ्यम् ? प्रत्यक्षं कल्पनाऽपोढमिति च व्यपदिश्यते, न चाभिधेयमिति कोऽन्यो \textbf{भदन्ताद्} वक्तुमर्हति” \href{http://http://sarit.indology.info/?cref=nv.1.1.p44-6-12}{न्यायवा० पृ० ४१} इत्यादि यदवादी\textbf{दुदद्योतकर}स्तत्सर्वं कल्पनापोढशब्दार्थाऽपरिज्ञानविलसितं तस्य तपस्विन इति सूचितम् ।
	\pend
      

	  \pstart “इहाविसंवादकत्वमभ्रान्तत्वमभिप्रेतम् । तेन द्विचन्द्रादिज्ञानं व्यवच्छिन्नम्, \textbf{योगाचार}मतमपि सगृहीतं भवती”ति पूर्वव्याख्यानमवमन्यमानोऽभ्रान्तशब्दस्यार्थमाह--\textbf{अभ्रान्तम्} इति । \textbf{अभ्रान्तमर्थक्रियाक्षमेऽविपर्यस्तं यत् तदुच्यते} । न त्वविसंवादकमिह ग्रहीतव्यमिति बुद्धिस्थं पश्चाद् व्यक्तीकर्त्तव्यम् । अनर्थोऽपि वस्तुतयाऽध्यवसीयत इत्य\textbf{र्थक्रियाक्षम} इति विशेषणम् । तर्ह्यर्थक्रियाक्षम इत्येवास्तु । न । अर्थक्रियाक्षमस्यैव वस्तुत्वज्ञापनार्थत्वात् । किं तदर्थक्रियाक्षमम् ? किमवयवि ? अथान्यदेवेत्याह--\textbf{अर्थक्रियेति । सन्निवेशश्च}तुरस्रत्वादिः प्रतिभासधर्मः । स \textbf{उपाधि}र्विशेषणं यस्य \textbf{वर्ण}स्य वस्तुशब्दवाच्यस्य शुक्लादिपरमाणुसंघातस्य तथोत्पन्नस्य स तथा । सन्निवेशविशिष्टस्यैव वर्णस्यान्वयव्यतिरेकाभ्यामर्थक्रियायामुपयोगदर्शनादेतदाह । स एवा\textbf{त्मा} स्वभावो यस्येति तत्तथा । एतच्च चाक्षुषज्ञानविषयाभिप्रायेणोक्तं द्रष्टव्यम् । अन्यथा बह्वसमञ्जसं स्यादिति । अनेन परमाणुप्रचयमात्रस्यैवार्थक्रि\leavevmode\marginnote{\textenglish{19a/ms}}याकारित्वं नावयविनस्तस्यासत्त्वादिति सूचितम् । \textbf{न भ्राम्य}ति न विपर्यस्यति--अन्यथाग्राहि न भवति ।
	\pend
      

	  \pstart स्यादेतत्--परमाण्वर्थ एव भवन्मते बाह्यं वस्तु । सर्वं च विज्ञानं तेषु परमसूक्ष्मेषु  \leavevmode\marginnote{\textenglish{43/dm}} स्थूलाभासमाजायते । तत्कथं किञ्चिदभ्रान्तं नामेति ? अत्रोच्यते । एकसामग्रीजन्मनां परमाणूनां भिन्नदेशस्वभावानां तद्धेत्वभावतच्छायालोकपरमाणुस्वभावेनान्तरेण रहितत्वान्निरन्तरत्वेन प्रतिभास एव देशवितानावभासात्मा स्थौल्यं नापरं किञ्चित् । तत्र तथाभूतपरमाणुसमुदायनिष्ठं निर्विकल्पकं विज्ञानं कथं भ्रान्तं स्यात्? यद्येकैकं परमाणुमनेकदेशावष्टम्भेन गृह्णीयान्न पुनरनेकमनेकदेशावष्टम्भेन गृह्णत् । इतोऽपि विपर्यस्येद् यदि भिन्नदेशान् परमाणून् एकदेशान् गृह्णीयात् । न चैतदस्ति, अणुमात्रकपिण्डप्रतिभासाभावात् । एकदेशग्रहणे हि पिण्डो भासेत अणुमात्रको न तु विततदेशः । न चानेकग्रहो भ्रमः । अतस्मिंस्तदिति प्रत्ययस्य तादात्म्यात् । तदयमर्थः--एकज्ञानग्राह्यास्तथाविधा बहवः परमाणवः स्थूल इति । एकोऽयं स्थूल इति तु तथाभूतप्रतिभासाश्रयेण व्यवस्थाप्यमानत्वात् प्रतिभासधर्म इत्युच्यते । न वस्तुधर्मः, प्रत्येकमपरिसमाप्तेरित्यलमिह विस्तरेण ।
	\pend
      

	  \pstart ननु चैवमप्यन्योन्यमसंसृष्टस्वभावान् परमाणून् संसृष्टान् गृह्णद्विज्ञानं कथमिवविपर्यस्तं नामेति । अत्राप्युच्यते । किमिदमसंसृष्टत्वमिष्टं भवता, यद् विपर्ययग्रहणाद् भ्रान्तं ज्ञानमुपवर्ण्यते ? किं नानारूपत्वम्, अथ नानादेशत्वम्, उत रूपेणैव विजातीयेन व्यवहितत्वम्, आहोस्विदिन्द्रियान्तरग्राह्येणार्थेन व्यवकीर्णत्वम् ? तत्र यदि नानारूपत्वमसंसृष्टत्वमिष्टं तदा न कश्चित्संसृष्टग्रहो नाम सम्भवति यतोऽसंसृष्टा एव परमाणवः सर्वदा गृह्यन्ते । विततदेशस्वभावानामेव तेषामवभासनात् । यदि ह्येकरूपा भासेरन, अणुमात्रकः पिण्डो भासेत । न तु विततदेशभासनं स्यात् ।
	\pend
      

	  \pstart अथ नानादेशत्वमसंसृष्टत्वमभिप्रेतं तदपि नतरामसंसृष्टग्रहो यतो नानादेशा नीला परमाणवो नानादेशा एव च गृह्यन्ते । एकदेशत्वभासने हि पिण्डो भासेताणुमात्रक इत्युक्तम् ।
	\pend
      

	  \pstart अथ रूपेणैव विजातीयेन व्यवहितत्वमसंसृष्टत्वं विवक्षितम्; तदा तु तदसम्भवादेव न तद्विपरीतग्रहः । यतो रूपान्तरव्यवधानरहिता एव निरन्तरा नीलाः परमाणवः, भासन्ते च तथाभूता इति कथं विभ्रमः । मध्यवर्त्तिनो विजातीयस्यालोकादिपरमाणोरनुत्पत्तेरप्रतिभासनाच्च । अथ च्छायालोकपरमाणुरुत्पद्यमानः केन प्रतिबद्धो यतो नोत्पद्यते । न च शक्यं वक्तुम्--मध्ये परमाण्वोर्नास्ति परमाण्वन्तरस्यावकाश इति । यतो निरवयवः परमाणुः सर्वत्र सावकाश इति । सत्यमेतत् । केवलं नावकाशाभावात् तदनुत्पत्तिरपि तु हेत्वभावात् । कस्माद् हेतुर्न भवति ? स्वहे\leavevmode\marginnote{\textenglish{19b/ms}}त्वभावादित्यपर्यनुयोग एव ।
	\pend
      

	  \pstart अथ भिन्नेन्द्रियग्राह्यस्पर्शादिव्यवकीर्णत्वमसंसृष्टत्वमभिमतम् । तदा संसृष्टान् परमाणून् गृह्णाति विज्ञानमिति इन्द्रियान्तरग्राह्यशून्यान् गृह्णातीत्युक्तं भवति । तथा च न किञ्चिदनिष्टम् । तथाहि यदि नामेन्द्रियान्तरग्राह्यस्पर्शादिर्न गृह्यते तथापि नीलरूपं तावत् स्वदेशस्वभावस्थितं गृह्यत एव । न च भिन्नेन्द्रियग्राह्यशून्यानां स्वरूपं गृह्यमाणं विपरीतं गृहीतं भवति । देशकालाकाराणामेकस्याप्यविपर्यासात् । न चाग्रहो भ्रम इति । ननु च परमाणूनामन्तराण्याकाशात्मकानि सन्ति । न च ते सान्तराः प्रतिभासन्ते । तत्कथमविपर्यास इति । अथ किमिदमाकाशं नाम । यदि रूपान्तरात्मकं तन्नास्तीत्युक्तम् । अथापि स्पर्शाद्यात्मकं तत्राप्युक्तम् । अथ सप्रतिघद्रव्याभावः । एवमप्यवस्त्वाकाशम् । ततश्चाकाशमन्तरमित्यन्यवस्त्वन्तरं न किञ्चिद- \leavevmode\marginnote{\textenglish{44/dm}} “
	  
	एतच्च \footnote{कल्पनाभ्रान्त०--\cite{dp-msD-n}}लक्षणद्वयं विप्रतिपत्तिनिरासार्थम्,\footnote{निराकरणार्थम् \cite{dp-msA} \cite{dp-msC} \cite{dp-edP} \cite{dp-edE} \cite{dp-edH} \cite{dp-edN}} न त्वनुमाननिवृत्त्यर्थम् । यतः” न्तरमित्युक्तं स्यात् । तथा निरन्तराः परमाणव इत्युक्तं भवति । ततो निरन्तराश्च परमाणवो निरन्तरा एव भासन्ते । तत् किमुच्यतेऽन्तरमाकाशम्, न च तत्प्रतिभासत इति ? यत्खल्वत्यन्तमसत् शशविषाणप्रख्यं तत्कथं भासेत ?
	\pend
      

	  \pstart नन्वाकाशात्मनोऽप्यन्तरस्याभावे रूपसंसर्गः परमाणुनां प्रसज्येत । नैष दोषः । नास्माभिरुच्यते रूपमेकं परमाणूनां देशो नैक इति । अपि तु भिन्नरूपदेशा उत्पन्ना मध्यवर्त्तिविजातीयरूपरहितास्तथैव भासन्त इति तत् कथं रूपसंसर्गप्रसङ्गः ?
	\pend
      

	  \pstart ननु च रसादिदेशे नीलरूपं प्रतिभासते । ततश्चातद्देशं तद्देशतया गृह्णद् विज्ञानं कथमभ्रान्तं नामेति ? तत्राप्युच्यते । यदा देशः प्रतिभासते तदा तस्मिन् देशे प्रतिभासमाने यः प्रतिभासतेऽर्थः स देशविशिष्ट उच्यते । यदि च रसादिश्चक्षुर्विज्ञाने प्रतिभासेत तदा तद्देशव्यापिनि नीले गृह्यमाणे स्याद् भ्रान्तं विज्ञानम् । न च तत्र रसादिः प्रतिभासते, इन्द्रियान्तरग्राह्यस्येन्द्रियान्तरज्ञाने प्रतिभासायोगात् । तत् कुतस्तद्देशनीलग्रहणम् ? नीलमेव हि भासमानं देशः नापरो देशः कश्चिदाभासते । इन्द्रियान्तरग्राह्याप्रतिभासे च शुद्धरूपप्रतिभासः । शुद्धरूपप्रतिभास एव च निरन्तरप्रतिभासः । ततो निरन्तरा नीलाः परमाणवो गृह्यन्ते । तस्मात् स्वदेशस्थायिनो नीलपरमाणवः स्वरूपेणैव गृह्यन्ते । ततो देशकालाकाराणामेकस्याप्यपरित्यागान्नीलाभासं विज्ञानमभ्रान्तमेव । सर्वं चैतद् ग्राह्यतत्त्वं \textbf{विनिश्चये धर्मोत्तरेणैव} विस्तरेण निरूपितमिति नेह प्रतन्यते ।
	\pend
      

	  \pstart नन्वेवमभ्रान्तत्वे \textbf{योगाचारमत}मसङ्गृहीतं स्यात् । ग्राह्यग्राहकाकारतया प्रवृत्तेः सर्वस्यैवासर्वज्ञाविज्ञानस्यालम्बने भ्रान्तत्वात् । तत्कथं पूर्वव्याख्यानावज्ञा न क्रियत इति चेत् । उच्यते । न \textbf{योगाचारनये} लक्षणमिदम्, किन्तु \textbf{सौत्रान्तिकनय} एव । न च सर्वं विज्ञानवादे योजयि\leavevmode\marginnote{\textenglish{20a/ms}}तुं शक्यम् । “तस्य विषयः स्वलक्षणम्” इत्यादेरशक्ययोजनत्वात् । तस्मिन् किं प्रत्यक्षलक्षणमिति चेत् । कल्पनापोढत्वमेव । द्विचन्द्रादिनिरासः कथमिति चेत् सम्यग्ज्ञानं सदेवमित्यभिप्रायाददोषः । \textbf{सौत्रान्तिकनये}ऽपि किं नैवं ? ततश्चाभ्रान्तग्रहणमतिरिच्यत इति चेत् । सत्यमेतत् । केवलं विप्रतिपत्तिनिरासार्थं वर्णयिष्यत इति । “इहाभ्रान्तपदं तैमिरिकादिज्ञानव्यवच्छेदार्थम् । कल्पनाऽपोढग्रहणं तु अनुमाननिरासार्थमिति” यत्पूर्वकैर्व्याख्यातं तद् व्यक्तमेवापहस्तयन्नाह--\textbf{एतच्च}--इति । \textbf{चो}ऽवधारणे । \textbf{विप्रतिपत्तिनिरासार्थ}मित्यतः परो द्रष्टव्यः । पूर्वेषामभिप्रेतं प्रतिषेधति । \textbf{न त्वि}ति भेदार्थः । ननु किमुच्यते \textbf{न त्वनुमाननिवृत्त्यर्थ}मिति ? यावतैकैकेनानुमाननिवर्त्तनादिति चेत् । उच्यते । \textbf{लक्षणद्वय}मिति द्वयोरुपादानं विप्रतिपत्तिनिराकरणार्थम्, एकैकेनानुमानव्यवच्छेदसिद्धेरिति समुदितफलमेतत् । ततो \textbf{नानुमाननिवृत्त्यर्थ}मिति नानुमाननिवृत्त्यर्थमेवेति सावधारणं निषिध्यते । यदि तु विप्रतिपत्तिनिराचिकीर्षया क्रियमाणमनुमानं व्यवच्छिनत्ति न तदर्थमेव द्वयमिति ।
	\pend
      

	  \pstart ननु नेदं लक्षणद्वयमनुमानव्यवच्छेदार्थं पूर्वकैर्व्याख्यातम् । किन्तु कल्पनाऽपोढग्रहणमेव । अभ्रान्तग्रहणं तु द्विचन्द्रादिज्ञानव्यवच्छेदार्थम् । तत् कथं \textbf{“न त्वनुमाननिवृत्त्यर्थम्”} तथा  \leavevmode\marginnote{\textenglish{45/dm}} “
	  
	कल्पनाऽपोढग्रहणेनैवानुमानं निवर्तितम् । तत्रासत्यभ्रान्तग्रहणे गच्छद्वृक्षदर्शनादि प्रत्यक्षं कल्पनाऽपोढत्वात् स्यात् । ततो हि प्रवृत्तेन\footnote{प्रमात्रा--\cite{dp-msD-n}} वृक्षमात्रमवाप्यते इति संवादकत्वात् सम्यग्ज्ञानम्, कल्पनाऽपोढत्वाच्च प्रत्यक्षमिति स्यादाशङ्का । तन्निवृत्त्यर्थम् अभ्रान्तग्रहणम् । तद्धि\footnote{गच्छदवृक्ष\add{दर्शनम्}--\cite{dp-msD-n}} भ्रान्तत्वात् न प्रत्यक्षम् । त्रिरूपलिङ्गजत्वाभावाच्च नानुमानम् । न च प्रमाणान्तरमस्ति । अतो गच्छद्वृक्षदर्शनादि मिथ्याज्ञानमित्युक्तं भवति ।” \textbf{“यतः कल्पनापोढग्रहणेनैवानुमानं निवर्त्तितम्”} इत्युच्यत इति चेत् । सत्यम् । केवलं यद्यभ्रान्तग्रहणं व्यवच्छेदार्थमेव तदानुमानव्यवच्छेदार्थमेव युज्यते । न तु द्विचन्द्रादिज्ञाननिरासार्थम्, तस्य सम्यग्ज्ञानाधिकारादेव व्यवच्छेदसिद्धेः । तथा हि द्विविधं सम्यग्ज्ञानमिति प्रस्तुत्य लक्षणमिदं विधीयमानं तदधिकारेणैव विहितं भवति ।
	\pend
      

	  \pstart नन्वेवं सति कल्पनापोढग्रहणेनैव निवृत्तेरभ्रान्तग्रहणमतिरिच्यते । अयमपरस्तेषां दोषोऽस्ति । न तु द्विचन्द्रादिज्ञाननिवृत्त्यर्थमिदं युज्यत इति \textbf{धर्मोत्तर}स्याशयः । यद्येतदर्थमभ्रान्तग्रहणं न भवति तर्हि किमनेनेत्याह--तत्र--इति ।
	\pend
      

	  \pstart ननु गच्छद्वृक्षदर्शनादेः कल्पनापोढस्यापि विसंवादकत्वेन सम्यग्ज्ञानत्वाभावादेव व्यवच्छेदे सिद्धे किमेतदर्थेनाभ्रान्तपदेनेत्याह--\textbf{ततो हि}--इति । वृक्षमात्रमिति गमनेनाऽनवच्छिन्नम् । \textbf{इति} हेतौ । अस्तु सम्यग्ज्ञानम्, प्रत्यक्षसम्भावना त्वस्य कथम् ? न हि यदेव सम्यग्ज्ञानं तदेव प्रत्यक्षमित्याशङ्क्य पूर्वोक्तमेव प्रसङ्गेनाह--\textbf{कल्पना}--इति । \textbf{चः} संवादक—त्वापेक्षया समुच्चयार्थः । \textbf{इति}करणेनाशङ्कायाः स्वरूपं दर्शयति । आशङ्का चेयमीदृशी यौक्ती द्रष्टव्या । यद् वा \textbf{चो} यस्मादर्थे ।\add{... ... ...}असत्यभ्रान्तग्रहणे स्यादियमाशङ्का । यदि सत्यपि स्यात्तदा किमनेनेत्याह--\textbf{तन्निवृत्त्यर्थ}मिति शङ्कानिवृत्त्यर्थमात्रम् । अनेन एतद् दर्शयति--यद्यपि परमार्थतः सम्यग्ज्ञानाधिकारादेवमादिज्ञानं व्यवच्छिद्यते । तथाप्यंशसंवादवादिनामाहत्य विप्रतिपत्तिनिराकरणार्थं क\leavevmode\marginnote{\textenglish{20b/ms}}र्त्तव्यमेवाभ्रान्तग्रहणमिति ।
	\pend
      

	  \pstart स्यादेतत्--किमभ्रान्तग्रहणेनाधिकं कृतम् । येनैतद् विप्रतिपत्तिनिराकरणार्थं भवतीत्याशङ्कायां दूरस्थितमपि \textbf{गच्छ}द्वृ\textbf{क्षदर्शनादि मिथ्याज्ञानमित्युक्तं भवतीति} योजनीयम् । \textbf{इतिना} उक्तेराकारं कथयति । \textbf{उक्तं} प्रकाशितं \textbf{भव}त्यभ्रान्तग्रहणेनेति प्रकरणात् । कुतो मिथ्याज्ञानं तदुक्तमित्यपेक्षायां प्रथममुक्तं \textbf{तदिति} योजनीयम् । हिर्यस्मात् ततोऽयमर्थः—यस्मात् तद् गच्छद्वृक्षदर्शनादि भ्रान्तत्वात् न प्रत्यक्षम्, अतो मिथ्याज्ञानमित्यभ्रान्तग्रहणेनोक्तं भवति । भवतु भ्रान्तत्वादप्रत्यक्षम्, अनुमानरूपत्वे त्वनिवर्त्तिते कथमस्य तथात्वमित्याशङ्कायां योज्यम् । \textbf{चो} यस्मान्नानुमानम् । कथं नानुमानम् ? \textbf{त्रिरूपलिङ्गजवाभावात्} । यद्येवमन्यत्प्रमाणं भविष्यति, तथापि कथं मिथ्याज्ञानमित्याशङ्कायां वाच्यम् । \textbf{चो} यस्मान्ना\textbf{भ्यां प्रमाणान्तरमस्ति । अतः} प्रत्यक्षत्वादिस्वाभावत्वाभावात् । यद् वाऽतोऽभ्रान्तग्रहणादित्यर्थः ।
	\pend
      

	  \pstart नन्वनुमानादिरूपनिराकरणे यद्यभ्रान्तग्रहणस्य व्यापारो भवेत्, भवेदेव मिथ्याज्ञानत्वाभिधाने सामर्थ्यं यावतो\footnote{ता}यथा स्वयमुपपत्त्यैवानुमानादिभावो व्युदस्तस्तत् कथमभ्रान्त \leavevmode\marginnote{\textenglish{46/dm}} “
	  
	यदि मिथ्याज्ञानम् कथं ततो वृक्षावाप्तिरिति चेत्, न ततो वृक्षावाप्तिः । नानादेशगामी हि\footnote{०गामी वृक्षः--\cite{dp-msC}} वृक्षस्तेन परिच्छिन्नः । एकदेशनियतश्च वृक्षोऽवाप्यते । ततो यद्देशो\footnote{यो देशो यस्य--\cite{dp-msD-n}} गच्छद्वृक्षो दृष्टः, तद्देशो नावाप्यते । यद्देशश्चावाप्यते\footnote{०प्यते न स दृष्ट इति--\cite{dp-msC}} स न दृष्ट इति न तस्मात् कश्चिदर्थोऽवाप्यते । ज्ञानान्तरादेव तु\footnote{देव च वृ० \cite{dp-msB} \cite{dp-msD}} वृक्षादिरर्थोऽवाप्यते । इत्येवमभ्रान्तग्रहणं विप्रतिपत्तिनिरासार्थम् । 
	  
	तथा\footnote{तथेत्यारभ्य विप्रतिपत्तिनिराकरणार्थमित्यन्तः पाठो नास्ति--\cite{dp-msA} \cite{dp-edP} \cite{dp-edE} \cite{dp-edH}} अभ्रान्तग्रहणेनाप्यनुमाने निवर्तिते कल्पनापोढग्रहणं \footnote{सविकल्पकज्ञानमपि प्रत्यक्षमित्युक्तं \textbf{मीमांसकैः} यथा “अस्ति ह्यालोचना” \href{http://http://sarit.indology.info/?cref=śv-pratyakṣa.112}{मीमांसाश्लो० प्रत्यक्षसूत्र--श्लो० ११२} इत्यादि । एतस्य विप्रतिपत्तेः--\cite{dp-msD-n}}विप्रतिपत्तिनिरा-” ग्रहणेन तन्मिथ्याज्ञानमित्युक्तं भवतीत्युच्यते । सत्यम् । किन्त्वनुमानादिरूपतानिराकरणहेतुना दत्तसाहायकेन सताऽभ्रान्तग्रहणेन तन्मिथ्याज्ञानमित्युक्तं भवतीत्युक्तमिति बोद्धव्यम् ।
	\pend
      

	  \pstart ननु प्रत्यक्षलक्षणशून्यस्याप्रत्यक्षतैव दर्शयितव्या, तत्किमनुमानादिरूपतानिराकरणमप्रकृतं कृतमिति चेत् । अप्रत्यक्षत्वप्रदर्शने प्रसङ्गेन कृतमिति का क्षतिः ?
	\pend
      

	  \pstart \textbf{यदीत्यादि} सुगमम् । केवलं \textbf{तत} इति गच्छद्वृक्षदर्शनरूपान्मिथ्याज्ञानात्प\textbf{रिच्छिन्नो} दृष्टः । \textbf{वृक्ष} इति वृक्षत्वेन प्रथनादुच्यते न त्वसौ तथाऽवभासमानो वृक्ष एव । \textbf{एकदेशनियत} इ ति प्राप्यमाणपरमार्थवृक्षाभिप्रायेणोक्तम् । \textbf{चो} हेतौ । ततस्तस्मादनन्तरोक्तात् कारणात् । \textbf{चो}ऽप्राप्यमाणाद् भेदमस्य दर्शयति । \textbf{इतिः} हेतौ । \textbf{तस्मा}न्मिथ्याज्ञानात् ।
	\pend
      

	  \pstart ननु ततोऽपि प्रवृत्तस्यास्ति च्छायाद्यर्थक्रियाकारिरूपस्य पादपस्य प्राप्तिस्तत्कथमपह्नूयत इति ? आह--\textbf{ज्ञानान्तराद्}--इति । तद्देशोपसर्पणजन्मनः स्थितवृक्षप्रतिभासात्मनो \textbf{ज्ञानान्तरात्} । तत्र मिथ्याज्ञानमर्थिनः प्रवृत्तिमात्रहेतुर्न तु वृक्षप्रापकमिति संक्षेपार्थः । यदि मिथ्याज्ञानत्वप्रतिपादनार्थमभ्रान्तग्रहणं न तर्हि विप्रतिपत्तिनिराकरणार्थमित्याह--\textbf{एवम्} इति । \textbf{एव}मित्यनेनानन्तरोक्तस्योपपत्तिप्रकारस्याकारो दर्शितः । ततोऽयमर्थः--एवमनन्तरोक्तेन युक्तिप्रकारेण । एतादृशमपि प्रत्यक्षमिति विरुद्धायाः प्रतिपत्तेर्निरासार्थमभ्रान्तग्रहणमिति । यद्येवमभ्रान्तग्रहणमेव विप्रतिपत्तिनिराकरणार्थम् । न तु कल्पनापोढग्रहणम् । उक्तं च द्वयमेतत् तन्निरासार्थमित्याशङ्क्याह--\textbf{तथे}ति । यथाऽभ्रान्तपदं तथेदमित्यर्थः ।
	\pend
      

	  \pstart \textbf{विप्रतिपतिनिराकरणार्थमि}ति प्रत्यक्षपृष्ठभाविनोऽपि विकल्पस्य व्यावहारिकलोकाध्यवसायेनाभ्रान्तस्य यत्प्रत्यक्षत्वं कैश्चिदिष्टम्, तन्नि\leavevmode\marginnote{\textenglish{21a/ms}}राकरणार्थमिति द्रष्टव्यम् । तत्रासति कल्पनापोढग्रहणे घटोऽयमित्यादिज्ञानं प्रत्यक्षमभ्रान्तत्वात् स्यात् । ततो हि प्रवृत्तेन घटादिरर्थः प्राप्यत इति संवादकत्वान्सम्यग्ज्ञानमभ्रान्तत्वाच्च प्रत्यक्षं स्यादा \add{दित्या} शङ्का । तन्निवृत्त्यर्थम् । कल्पनात्मकत्वान्न प्रत्यक्षम्, त्रिरूपलिङ्गजत्वाभावाच्च नानुमानम् ।  \leavevmode\marginnote{\textenglish{47/dm}} “
	  
	करणार्थम्\footnote{विप्रतिपत्तिनिरासार्थम्--\cite{dp-msB} \cite{dp-msC} \cite{dp-msD} \cite{dp-edN}} । भ्रान्तं हि अनुमानं स्वप्रतिभासेऽनर्थे\footnote{सामान्ये--\cite{dp-msD-n}}ऽर्थाध्यवसायेन प्रवृत्तत्वात् । प्रत्यक्षं तु ग्राह्ये रूपे न विपर्यस्तम् । 
	  
	न तु अविसंवादकमभ्रान्तमिह ग्रहीतव्यम् । यतः सम्यग्ज्ञानमेव प्रत्यक्षम्, नान्यत् । तत्र सम्यग्ज्ञानत्वादेवाऽविसंवादकत्वे लब्धे पुनरविसंवादक\footnote{०वादग्रह० \cite{dp-edE} ०वादकत्वग्रह \cite{dp-msC}}ग्रहणं निष्प्रयोजनमेव । एवं हि वाक्यार्थः स्यात्--प्रत्यक्षाख्यं यदविसंवादकं ज्ञानं तत् कल्पनापोढमविसंवादकं चेति । न चानेन द्विरविसंवादकग्रहणेन किञ्चित्\footnote{न किञ्चित् प्रयोजनम्--\cite{dp-msD-n}} । तस्माद् ग्राह्येऽर्थक्रियाक्षमे वस्तुरूपे यदविपर्यस्तं तदभ्रान्तमिह वेदितव्यम् ॥ 
	  
	कीदृशी पुनः कल्पनेह गृह्यत इत्याह-- “
	  
	अभिलापसंसर्गयोग्यप्रतिभासा\footnote{०भासप्र० \cite{dp-msB} \cite{dp-edP} \cite{dp-edE} \cite{dp-edH} \cite{dp-edN}} प्रतीतिः कल्पना ॥ ५ ॥” 
	  
	\footnote{अभिलापेति--\cite{dp-edE} \cite{dp-edP} “अभिलापेत्यादि” इति नास्ति--\cite{dp-msA} \cite{dp-msB} \cite{dp-edH} \cite{dp-edN}}अभिलापेत्यादि । \footnote{ननु न[[च]]यद्यपि तस्मिन् ज्ञाने आकारयोर्मिलनं तथापि शब्दार्थयोः संसर्गो नास्तीत्याह--\cite{dp-msD-n}}अभिल\footnote{अभिलाप्यते--\cite{dp-msA} \cite{dp-edP} \cite{dp-edH} \cite{dp-edE}}प्यतेऽनेनेति अभिलापः \textbf{वाचकः शब्दः । अभिलापेन संसर्गः}--\footnote{नास्ति--“अभिलापसंसर्गः” इति--\cite{dp-msA} \cite{dp-edP} \cite{dp-edE}}अभिलापसंसर्गः--\textbf{एकस्मिन् ज्ञानेऽभिधेयाकारस्याभिधानाकारेण सह\footnote{वाच्यरूपत्वेन--\cite{dp-msD-n}} ग्राह्याकारतया}” न च प्रमाणान्तरमस्ति । अतो घटोऽयमित्यादिज्ञानमसम्यग्ज्ञानं भवतीति पूर्ववद् वचनीयं योजनीयं च ।
	\pend
      

	  \pstart ननु च संवादिनोऽस्य कथमसम्यग्ज्ञानत्वम् ? यदि नाम विसंवादकत्वलक्षणम् असम्यग्ज्ञानत्वं नास्ति तथापि गृहीतार्थग्राहिणोऽस्यापूर्वाधिगमाभावात् करणार्थाभावरूपमसम्यग्ज्ञानत्वं स्मरणादेरिव किं नानुमन्यते ? जात्यादिविशिष्टवस्तुग्राहिणोऽस्यापूर्वार्थाधिगमोऽस्तीति चेत् । इदमन्यत्र विस्तरेण निरस्तमित्यास्तां तावदिहेति । \textbf{स्वप्रतिभास} इत्यादेर्ग्रन्थस्य तु सत्यमर्थं विषयविप्रतिपत्तिनिराकरणवाक्यविवरणं विवेचयिष्यन्तो विवेचयिष्यामः । सम्प्रति यद् दोषदर्शनात्पूर्वेषां व्याख्यानमवमन्यान्यथाऽयमभ्रान्तार्थं व्याचष्टे तं कण्ठोक्तं कर्त्तुं तेषामभिमतमभ्रान्तार्थं पूर्वं सामर्थ्यान्निषिद्धमपि साक्षान्निषेधन्नाह--\textbf{न तु}--इति । \textbf{तु}रतिशये । \textbf{यत} इत्यादि \textbf{वेदितव्यमि}त्यन्तं सुबोधम् ।
	\pend
      

	  \pstart यदि जात्यादियोजनात्मिका कल्पना । सा जात्याद्यभावादेव न सम्भवति । अथ ग्राह्यग्राहकभावेन प्रवर्त्तमानं ज्ञानं कल्पना तदा सर्वमसर्वज्ञज्ञानं तथा प्रवृत्तमिति किमवशिष्यते यदविकल्पकं स्यादित्यभिप्रेत्य कल्पनायाः स्वरूपं पृच्छति--\textbf{कीदृशीति} सामान्यतः पृच्छति । \textbf{पुन}रिति विशेषतः । \textbf{इहे}ति प्रत्यक्षलक्षणे ।
	\pend
      

	  \pstart अभिलप्यते उच्यतेऽ\textbf{नेनेत्यभिलापो वाचकः शब्दः} । स च सङ्केत-\textbf{मिलनम्} ।
	\pend
      \leavevmode\marginnote{\textenglish{48/dm}}“

	  \pstart ततो यदैकस्मिञ्ज्ञानेऽभिधेया\footnote{०ज्ञानेऽभिधानाभिधेययोः--\cite{dp-msB}}भिधानयोराकारौ\footnote{०कारौ निविष्टौ \cite{dp-msB}} संनिविष्टौ भवतस्तदा संसृष्टे अभिधानाभिधेये भवतः । अभिलापसंसर्गाय योग्योऽभिधेयाकार\footnote{रा} भासो\footnote{अभिधेयाभासो \cite{dp-msA} \cite{dp-edP} \cite{dp-edH} \cite{dp-edE} अभिधेयाकाराभासो--\cite{dp-edN}} यस्यां प्रतीतौ सा तथोक्ता ।
	\pend
       

	  \pstart तत्र काचित् प्रतीतिरभिलाप\footnote{०लापेन संसृ० \cite{dp-msA} \cite{dp-edP} \cite{dp-edH} \cite{dp-edE} \cite{dp-edN}} संसृष्टाभासा भवति । यथा व्युत्पन्नसङ्केतस्य घटार्थकल्पना घटशब्दसंसृष्टार्थावभासा \footnote{“भवति” इति नास्ति \cite{dp-msB} \cite{dp-msD}}भवति । काचित्त्वभिलापेनासंसृष्टापि अभिलापसंसर्गयोग्याभासा भवति । यथा बालकस्याव्युत्पन्नसङ्केतस्य कल्पना । तत्र “अभिलापसंसृष्टाभासा\footnote{०सृष्टप्रतिभासा \cite{dp-msD} संसृष्टप्र\add{ति}भासा \cite{dp-msB}} कल्पना” इत्युक्तावव्युत्पन्नसङ्केतस्य\footnote{नास्ति “कल्पना” इति \cite{dp-msA} \cite{dp-msB} \cite{dp-edP} \cite{dp-edH} \cite{dp-edE} \cite{dp-edN}} कल्पना न संगृह्येत\footnote{संगृह्यते \cite{dp-msC}} । योग्यग्रहणे\footnote{०ग्रहणेन तु \cite{dp-msC}} तु\footnote{बालकल्पना--\cite{dp-msD-n}}सापि संगृह्यते । यद्यपि अभिलापसंसृष्टाभासा\footnote{०सृष्टप्रतिभासा \cite{dp-msC} \cite{dp-msA}} न भवति तदहर्जातस्य\footnote{नास्ति “बालकस्य” इति \cite{dp-msA} \cite{dp-edP} \cite{dp-edE}} बालकस्य कल्पना, अभिलापसंसर्गयोग्यप्रतिभासा तु भवत्येव । या चाभिलापसंसृष्टा सापि योग्या । तत उभयोरपि योग्यग्रहणेन संग्रहः ।
	\pend
      ”

	  \pstart कालेन दृष्टेन रूपेणैक्यमापादितोऽन्तर्जल्पाकारस्तथा प्रतिभासमानो वाच्यः । \textbf{संसर्ग}पदेन सार्धमस्य विग्रहं प्राह--\textbf{अभिलापेन}--इति । \textbf{अभिलापेन} वाचकेन \textbf{संसर्गः} सम्बन्धः । “तृतीये”\href{http://http://sarit.indology.info/?cref=Pā.2.1.30}{पाणिनि--२. १. ३०}ति योगविभागात्समासः । अथवाऽर्थकथनमिदं कृतम्, समासस्तु षष्ठीतत्पुरुषः कार्यः । कथं पुनर्वाच्यवाचकयोः संसर्गः सम्भवतीत्याह--\textbf{एकस्मिन्} इति । \textbf{ग्राह्याकारतया मिलनमि}त्येकज्ञानग्राह्याकारतयाऽवस्थानमिति वाच्यम् । अर्थाकारश्चेद् वाचकशब्दाकारेण सहैकज्ञानारूढो भवति तदा वाच्यवाचकयोः संसर्ग इति यावत् । भवत्वभिधानाभिधेयाकारयोरेकज्ञानारोहः । किमेतावताऽभिधानाभिधेये वाच्यवाचकतया सम्बद्धे भवतः, येन तद्ग्राहि विज्ञानं सविकल्पकं स्यादित्याशङ्क्योपसंहारव्याजेनाह--\textbf{तत} इति । \textbf{यदाकारौ संन्निविष्टौ} प्रतिभासितौ \textbf{भवतस्तदा संसृष्टे} तथातया सम्बद्धे \textbf{भवतः} । एकस्मिन् ज्ञाने विशेष्यविशेषणत्वेनाकारप्रतिभास एवानयोर्वाच्यवाचकत्वग्रहणमिति भावः । \textbf{अभिलापसंसर्गाय योग्य} इत्यर्थकथनमिदं \leavevmode\marginnote{\textenglish{21b/ms}} कृतम् । समासस्त्वभिलापसंसर्गस्य योग्य इति कर्त्तव्यः । तादर्थ्यचतुर्थ्याः प्रकृतिविकार एव समासात् ।
	\pend
      

	  \pstart यद्येवमभिलापसंसृष्टप्रतिभासेत्येवास्तु किं योग्यग्रहणेनेत्युपालम्भं पश्यन् यथाऽस्य साफल्यं तथा दर्शयितुमुपक्रमते--तत्रेति । वाक्योपन्यासे चैतत् । अभिलापसंसृष्ट आभासो यस्याः सा तथा । कस्येवेत्याह--\textbf{यथेति} । “इदमस्य वाच्यमिदमस्य वाचकम्” इत्युभयांशाव  \leavevmode\marginnote{\textenglish{49/dm}} “
	  
	असत्यभिलापसंसर्गे कुतो योग्यतावसितिरिति\footnote{योग्यत्वावसितिः \cite{dp-msC} \cite{dp-msD} \cite{dp-msB}} चेत् । अनियतप्रतिभासत्वात् । \footnote{ननु च विकल्पज्ञानस्यानियतप्रतिभासित्वमेव न सिद्धम्--\cite{dp-msD-n}}अनियतप्रतिभासत्वं च प्रतिभासनियमहेतोरभावात् । ग्राह्यो ह्यर्थो विज्ञानं जनयन्नियतप्रतिभासं कुर्यात् । यथा रूपं चक्षुर्विज्ञानं जनयन्नियतप्रतिभासं जनयति । विकल्पविज्ञानं त्वर्थान्नोत्पद्यते । ततः प्रतिभासनियमहेतोरभावादनियतप्रतिभासम् ।” लम्बिज्ञानं सङ्केतः, शब्दार्थप्रयोगप्रतिपत्त्योः कार्यकारणभावरूपो वा । व्युत्पन्नो ज्ञातः संकेतो येन स \textbf{व्युत्पन्नसंकेतः} । तस्य यदि सर्वैव तादृशी तदा काचिदिति न वाच्यम्, योग्यग्रहणेन च न किञ्चिदित्याह--\textbf{काचिद्} इति । \textbf{तुः} व्युत्पन्नसंकेतप्रतीतेर्बालप्रतीतिं भेदवतीमाह । कस्य सदृशी सम्भवतीत्याह--\textbf{यथे}ति । बालक इत्यल्पार्थे कन् । यद्यभिलापसंसृष्टाभासेति कृतेऽपि तथाभूतायाः प्रतीतेः सङ्ग्रहः, कृतं तर्हि योग्यग्रहणेनेत्याह--\textbf{तत्रे}ति । तयोर्मध्येऽ\textbf{व्युत्पन्नसंकेतस्य} या सा न सङ्गृह्यते । कदा तु सङ्गृह्यतेत्याह--\textbf{अभिलापे}ति । अर्थप्रदर्शनत्वादस्या\textbf{भिलापसंसृष्टाभासा} प्रतीतिः \textbf{कल्पनेत्युक्तौ} वचने सतीत्यर्थः । इतिनोक्तेराकारः कथितः । असति तावदयं दोषः । यदि सत्यपि योग्यग्रहणे तदसंग्रहस्तथापि किं तेनेत्याह—\textbf{योग्ये}ति । \textbf{तु}रकरणावस्थाया विशेषमाह । न केवलं तत्संसृष्टाभासेत्यापेशब्दः । तथाविधबालस्य च कल्पनाऽनुमानसिद्धा । तेन साऽवश्यं ग्राह्येति भावः । कथं पुनर्योग्यग्रहणेन तस्याः सङ्ग्रह इत्याह--\textbf{यद्यपी}ति । निपातसमुदायोऽयं विशेषाभिधानार्थाभ्युपगमे सर्वत्र । \textbf{अह्नि जातं} जन्मास्येति तथा । अह्नि वा जात उत्पन्न इति तथा । अहर्जातस्यैतदहर्जातस्येति द्रष्टव्यम् । यद्येवं व्युत्पन्नसङ्केतस्य सा कथं सङ्गृह्यत इत्याह--\textbf{या चे}ति । \textbf{च} पूर्वापेक्षया समुच्चये । यदि सा तत्संसर्गयोग्या न स्यात्कथं तत्संसर्गमनुभवेदिति भावः । यत एवं तस्मात् ।
	\pend
      

	  \pstart ननु प्रतीतप्रतिभासस्य तत्संसर्गानुभवान्यथानुपपत्त्या योग्यता कल्पनीया । न च बालकल्पनाया अभिलापसंसर्गोऽस्ति, संकेताव्युत्पत्तेः । तत्कथं योग्यता कल्प्यतामित्यभिसन्धिना \textbf{असतीत्या}दिना पूर्वपक्षमुत्थापयति । एवं चोदयित्वाऽनुमानतः सा सिद्ध्यतीति मन्वानः साधनमाह--\textbf{अनियतेति} । प्रतिनियताकारत्वाभावादित्यर्थः । प्रकरणापन्ने साध्ये हेतौ चाभिहिते सुकरः साधनप्रयोग इति प्रयोगं न जगौ । एवमुत्तरत्रापि हेतुमात्राभिधानेऽस्यायमभिप्रायः प्रत्येतव्यः । प्रयोगस्त्वेवमिह कर्त्तव्यः--योऽनियतप्रतिभासो विकल्पः, स शब्दसंसर्गयोग्यो यथाच्छात्रविकल्पः । अनियतप्रतिभासश्च बालविकल्प इति भवत्येव प्रयोगः । केवलमिदमत्र निरूप्यताम् । बालस्य विकल्प एव कुतः सिद्धो येनेदमुत्तराणि चैतदङ्गानि साधनानि ना\leavevmode\marginnote{\textenglish{22a/ms}}श्रयासिद्धानि स्युरिति ? नैष दोषः । अनुमानसिद्धं विकल्पं धर्मिणं कृत्वाऽमीषामुपादानस्याभिप्रेतत्वात् । तत्पुनरनुमानमनेन सुज्ञातत्वान्नोपन्यस्तमिति वेदितव्यम् । एवं तद् द्रष्टव्यम्--या नियमवती प्रवृत्तिः क्वचित्प्राणिनः, सा विकल्पपूर्विका । यथा व्युत्पन्नसङ्केतव्यवहारस्यान्नादिविषया प्रवृत्तिः । नियमवती च तदितरपरिहारेण स्तनादौ प्रवृत्तिर्बालकस्येति कार्यहेतुः । सा च तादृशी प्रवृत्तिः प्रत्यक्षाधिगतेति स्वसिद्धौ न प्रमाणान्तरं  \leavevmode\marginnote{\textenglish{50/dm}} “
	  
	कुतः पुनरेतद्विकल्पोऽर्थान्नोत्पद्यत इति? \footnote{उत्तरमाह--\cite{dp-msD-n}}अर्थसन्निधिनिरपेक्षत्वात् । बालोऽपि हि यावद् दृश्यमानं स्तनं “स एवायम्” इति पूर्वदृष्टत्वेन न प्रत्यवमृशति तावन्नोपरतरुदितो” प्रयोजयति । न पुनरनेन \textbf{बालोऽपि ही}त्यादिना विकल्पसाधनमुपन्यस्तमिति मन्तव्यम्, यथाऽन्यैर्व्याख्यातम्, अर्थसन्निधिनिरपेक्षत्वसिद्धिप्रदर्शनोन्मुखत्वात्तस्य ग्रन्थस्येति ।
	\pend
      

	  \pstart ननु च परोक्षत्वात्तस्यानियतप्रतिभासत्वसन्देहे सन्दिग्धासिद्धोऽयं हेतुरिति । आह—\textbf{अनियतप्रतिभासत्वं च}--इति । \textbf{चो} यस्मादित्यस्मिन्नर्थे । \textbf{प्रतिभासस्य} ज्ञानाकारस्य \textbf{नियमः}—अर्थाकार एवायं शब्दाकार एव वा, रूपाकार एवायं रसाकार एव वेत्यात्मको वा तस्य हेतुर्जनकः तस्माद् । \textbf{अभावा}दनुत्पादात् । भवनं भाव उत्पादस्तन्निषेधाद् अभावोऽनुत्पाद एव । प्रतिभासनियमहेतोरभावाद्--अविद्यमानत्वादिति व्याख्याने तु यथाश्रुति व्यधिकरणासिद्धो हेतुः स्यात् । अन्यथा हेतुवचने तु क्रियमाणे तत्रार्थे हेतुरनभिहित एव स्यात् । साधनस्वरूपमात्रकथने चात्र \textbf{घर्मोत्तरस्य} शैली लक्ष्यते । \textbf{कुतः पुनरेतद् विकल्पोऽर्थान्नोत्पद्यत} इति च पूर्वपक्षः सूत्थानो न स्यात् । मृत्वा शीर्त्वा यथा कथञ्चित् सर्वस्यास्य समर्थने च वक्तुरकौशलं स्यादिति ।
	\pend
      

	  \pstart ननु प्रतिभासनियमहेतुर्यः कश्चन स च तत्र भविष्यति । अथ विशिष्टः । वक्तव्यस्तदाऽसौ यतः सकाशादनुत्पादात्तथात्वं ज्ञातव्यमित्याह--\textbf{ग्राह्य} इति । यद्यर्थ इत्येव क्रियते तदेन्द्रियमप्यर्थो ज्ञानं जनयतीति तस्याप्याकारनियामकत्वं स्यात् । न चापोद्धारद्वारेण तस्याकारविशेषहेतुत्वं व्यवस्थापितम्, अपि तु विषयग्रहणप्रतिनियमहेतुत्वमिति \textbf{ग्राह्य}ग्रहणम् । \textbf{ग्राह्य} आलम्बनः । यद्येवं ग्राह्य इत्येवास्तु । अपोहस्यापि ग्राह्यताऽस्ति । न चाऽसौ प्रतिभासनियमहेतुः । अतस्तन्निवृत्त्यर्थमर्थोऽर्थक्रियासमर्थ इति कृतम् । \textbf{कुर्यात्} कर्तुं शक्नोति । अपोद्धारेण तत्रैव तच्छक्तेरवधृतत्वात् । अत्र निदर्शनं \textbf{यथे}ति । \textbf{नियतप्रतिभास}मिति शब्दाकारपरिहारेणार्थाकारधार्येव रसाकारपरिहारेण रूपाकारधार्येव चेति । यद्येवं बालविकल्पोऽपि अर्थादेवोत्पत्स्यते, ततो नियतप्रतिभासो भविष्यतीत्याशङ्क्य पूर्वोक्तमेव प्रसङ्गेन स्मारयति--\textbf{विकल्पे}ति । \textbf{तु}रुदाहृताद् विज्ञान भिनत्ति । एवं तु प्रयोगः कार्यः--यदर्थान्नोत्पद्यते ज्ञानं तन्नियतप्रतिभासं न भवति । यथा व्युत्पन्नव्यवहारस्यातीतादिस्मरणम्, रूपरसादिसङ्कलनाज्ञानं वा । नोत्पद्यते चा\leavevmode\marginnote{\textenglish{22b/ms}}र्थाद् बालकस्य विकल्प इति व्यापकानुपलब्धिः ।
	\pend
      

	  \pstart ननु बालविकल्पस्य परोक्षत्वात् तत उत्पत्तिरनुत्पत्तिर्वा न शक्यते निश्चेतुम् । अतोऽयं सन्दिग्धासिद्धो हेतुरित्यभिप्रेत्य चोदयति \textbf{कुत} इति सामान्यहेतोः प्रश्नः । \textbf{पुन}रिति विशेषस्य । विदर्भ्योक्त्या चायं क्षेपे किमः प्रयोगः । तेन न कुतश्चिद्धेतोरिदमित्यर्थः । हेतुमाह--\textbf{अर्थे}ति । एवं तु प्रयोगः करणीयः--यज्ज्ञानं स्वोत्पत्तावर्थसन्निधिनिरपेक्षं तदर्थान्नोत्पद्यते । यथा व्युत्पन्नसंकेतस्य चिरनष्टवस्तुविषयं विज्ञानम् । अर्थसन्निधिनिरपेक्षं च बालविकल्पविज्ञानमिति विरुद्धव्याप्तोपलब्धिः । असिद्धिमस्याः परिहत्तु भूमिकां रचयन्नाह--\textbf{बालोऽइ हि}--इत्यादि । \leavevmode\marginnote{\textenglish{51/dm}} “
	  
	मुखमर्पयति स्तने । पूर्वदृष्टापरदृष्टं चार्थमेकीकुर्वद् विज्ञानमसन्निहितविषयम्, पूर्वदृष्टत्व\footnote{धर्मोत्तरे सर्वप्रतिषु “पूर्वदृष्टस्य” इति पाठः । किन्त्वत्र प्रदीपानुसारी पाठो गृहीतः ।--सं०}स्यासन्निहितत्वात् । असन्निहितविषयं चार्थनिरपेक्षम् । अनपेक्षं च प्रतिभासनियमहेतोरभावादनियतप्रतिभासम् । तादृशं चाभिलापसंसर्गयोग्यम् ।” न केवलं व्युत्पन्न इत्यपिशब्दः । हिर्यस्मात् । \textbf{यावदि}ति निपातोऽवधौ । \textbf{पूर्वदृष्टत्वं} पूर्वदर्शनविषयत्वम् । तेन यो मया पूर्वं क्षुत्प्रतिघातहेतुत्वेन प्रतिपन्नः स एवाऽयमिति प्रत्यवमर्षस्य रूपमाचष्टे । \textbf{न प्रत्यवमृशति} प्रत्यभिजानाति । \textbf{तावदि}त्यप्यवधौ । \textbf{उपरतं रुदितं} यस्मात् स तथा \textbf{सन्नायर्पति} । सर्वैरेव स्वसन्ततावेवमादिव्यवहारस्य दृष्टदृश्यमानयोरेकीकरणकारणत्वेनावगतत्वादेवमुच्यते । तस्यैव चान्यत्र लोष्टादौ ढौकितेऽपि तथाऽदर्शनाच्च ।
	\pend
      

	  \pstart ननु बालस्य करणानामपाटवात् संकेताग्रहणाच्च नान्तर्जल्पाकारोऽपि शब्दः सम्भवति । सम्भवे वा योग्यग्रहणानर्थक्यम् । तत्कथमेवं \textbf{यावन्न प्रत्यवमृशति}--इत्युच्यते ? उच्यते । अनयैव हि भङ्ग्या अस्ति सा काचिद् दृष्टदृश्यमानयोरेकीकरणावस्था या तत्र निमित्तमिति प्रतिपाद्यते । सा च शब्देन प्रतिपाद्यमाना अभ्यस्तेनामुना शब्देन प्रतिपाद्यते । न पुनरेवमेवासौ प्रत्यवमृशतीत्युच्यते ।
	\pend
      

	  \pstart ननु द्वितीयादिदर्शनकाले भवतु पूर्वदृष्टापरदृष्टयोरेकीकरणेन मुखार्पणम्, भूमिपातानन्तरं तु कथम् ? न खलु स्तनमद्राक्षीद् असौ येन दृष्टेन दृश्यमानमेकीकृत्यार्पयेत् । तदा तु जन्मान्तरानुभवादिति ब्रूमः । तत्रापि जन्मनि जन्मान्तरानुभवबलात् । न चादिमान् संसार इति का क्षतिः ? शब्दाकारोऽपि तत्रास्त्येव । तत् किमनेन योग्यग्रहणसफलीकरणप्रयासेनेति चेत् । अस्तु तत्र जन्मान्तराभ्यासात् मूर्च्छितः शब्दाकारः, मा च भूत् सर्वथा पश्चाद्व्युत्पत्स्यमानेन विशिष्टेन शब्देन संसृष्टार्थप्रतिभासः, न भवति तस्य विकल्पप्रत्यय इत्युच्यते । “स एवायम्” इति पूर्वदृष्टत्वेन स्तनं प्रत्यवमृशतु । किमत इत्याह--\textbf{पूर्वदृष्टे}ति । \textbf{पूर्वदृष्टं चापरदृष्टं} चेति द्वन्द्वैकवद्भावः । \textbf{चो}ऽवधारणे कुर्वदित्यस्मात्परो द्रष्टव्यः । हेतौ वा शतृप्रत्ययः ।
	\pend
      

	  \pstart तेनायमर्थः--यस्मात्पूर्वदष्टापरदृष्टमेकीकरोत्येव तस्मा\textbf{दसन्निहितविषयमिति} । तथा कुर्वदपि कथमसन्निहितविषयमित्याह--\textbf{पूर्वदृष्टत्वस्य}\footnote{धर्मोत्तरे सर्वप्रतिषु “पूर्वदृष्टस्य” इति पाठः । किन्त्वत्र प्रदीपानुसारी पाठोगृहीतः ।--सं०}--इति । अयमाशयः--पूर्वदर्शनविषयत्वं पूर्वदृष्टत्वमुच्यते । निवृत्ते च पूर्वदर्शने पूर्वदर्शनविषयत्वं वस्तुनो नास्ति । तदुत्तरकालभाविना ज्ञानेन पूर्वज्ञानविष\leavevmode\marginnote{\textenglish{23a/ms}}यत्वमसन्निहितमेव वस्तुनो दृश्यत इति । भवत्वसन्निहितविषयम्, अर्थसन्निधिनिरपेक्षं तु कथं सिद्धमित्याह--\textbf{असन्निहितेति । चो} यस्मात् । \textbf{असन्निहितविषय}मर्थसन्निधिनिरपेक्षम् । अनेनार्थसन्निधिनिरपेक्षत्वेऽर्थसन्निधिनिरपेक्षत्वं हेतुमाह । ईदृशस्तु प्रयोगो ज्ञातव्यः--यदसन्निहितविषयं ज्ञानं तदर्थसन्निधिनिरपेक्षम्, यथाऽतीतानागतविषयोऽस्मदादिविकल्पः । पूर्वदृष्टत्वेन प्रत्यभिज्ञानाच्च बालविज्ञानमसन्निहितविषयमिति स्वभावः ।
	\pend
      

	  \pstart सम्प्रति शिष्याणां सुखप्रतिपत्त्यर्थं यथोत्तरोत्तरस्य हेतोः सिद्धौ पूर्वपूर्वस्य हेतोः सिद्धिर्भवति तथा दर्शयति--\textbf{अनपेक्षं च}--इति ।
	\pend
      \leavevmode\marginnote{\textenglish{52/dm}}“

	  \pstart इन्द्रियविज्ञानं तु सन्निहितार्थ\footnote{सन्निहितमात्रग्रा० \cite{dp-msA} \cite{dp-edP} \cite{dp-edH} \cite{dp-edE} \cite{dp-edN}}मात्रग्राहित्वादर्थसापेक्षम् । अर्थस्य च प्रतिभासनिवमहेतुत्वान्नियतप्रतिभासम् । ततो नाभिलापसंसर्गयोग्यम् ।\footnote{इन्द्रियविज्ञानम्--\cite{dp-msD-n}}
	\pend
       

	  \pstart अत एव स्वलक्षणस्यापि वाच्यवाचकभावमभ्युपगम्यै\footnote{एतस्य इन्द्रियग्राहिणो ज्ञानस्य--\cite{dp-msD-n}}तदविकल्पकत्वमुच्यते । यद्यपि हि स्वलक्षणमेव वाच्यं वाचकं च भवेत् तथापि अभिलापसंसृष्टार्थ\footnote{वाच्यवाचकभावसद्भावेऽपि अभिलापसंसृष्टार्थं सद् विज्ञानं सविकल्पकम्, अन्यथा निर्विकल्पकमित्यर्थः--\cite{dp-msD-n}} विज्ञानं सविकल्पकम् । न चेन्द्रियविज्ञानम् अर्थेन नियमितप्रतिभासत्वाद् अभिलापसंसर्गयोग्यप्रतिभासं भवतीति निर्विकल्पक्रम् ।
	\pend
       

	  \pstart श्रोत्रज्ञानं\footnote{श्रोत्रविज्ञानं \cite{dp-msB} \cite{dp-msD} \cite{dp-edN}} तर्हि \footnote{तर्हि स्व० \cite{dp-msB}}शब्दस्वलक्षणग्राहि \footnote{शब्दादिकं घटादिकं वा--\cite{dp-msD-n}}शब्दस्वलक्षणं च\footnote{नास्ति “च” \cite{dp-msA} \cite{dp-edP} \cite{dp-edH} \cite{dp-edE}} किञ्चिद्वाच्यं किञ्चिद्वाचकम्--इत्यभिलापसंसर्गयोग्यप्रतिभासं स्यात् । तथा च सविकल्पकं स्यात् ।
	\pend
       

	  \pstart नैष दोषः । सत्यपि स्वलक्षणस्य वाच्यवाचकभावे संकेतकालदृष्टत्वेन गृह्यमाणं
	\pend
      ”

	  \pstart ननु किमनियतप्रतिभासत्वे साध्येऽर्थनिरपेक्षत्वं हेतुर्येन तथा सदनियतप्रतिभासमित्युच्यते । \textbf{प्रतिभासनियमहेतोरिति च} मध्यवर्त्ती च ग्रन्थः कथं नेयः ? उच्यते । नाय\textbf{मनपेक्षमि}त्यादिरेकवाक्यतया नेयः । किं तर्हि ? वाक्यभेदोऽत्र कर्त्तव्यः । तत्र \textbf{चो} यस्मात् । अनपेक्षमर्थसन्निधिनिरपेक्षं सत्प्रतिभासनियमहेतोरनुत्पन्नमिति प्रतिभासनियमहेतोरभावादित्यस्य सामर्थ्याद् वाक्यभेदं कृत्वा । ततः प्रतिभासनियमहेतोरभावादनुत्पादादनियतप्रतिभासमिति योजनीयम् । \textbf{तादृशमि}त्यनियतप्रतिभासम् । \textbf{चो} वक्तव्यमित्येतदित्यस्मिन्नर्थे ।
	\pend
      

	  \pstart सर्वमेतदिन्द्रियज्ञानेऽपि परः कदाचिदाशङ्कयेदिति तन्निरासार्थमाह--\textbf{इन्द्रियेति} । तुर्विकल्पज्ञानं विशिनष्टि । \textbf{ततो} नियतप्रतिभासत्वात् ।
	\pend
      

	  \pstart इह पूर्वव्याख्यातृभिः “असामर्थ्यवैयर्थ्याभ्यां स्वलक्षणस्य संकेतयितुमशक्यत्वादवाच्यवाचकत्वम् । अवाच्यावाचकस्वलक्षणग्राहित्वाच्चेन्द्रियज्ञानमविकल्पकमिति” व्याख्यातम् । तच्चावद्यम्, अन्यथाऽप्यविकल्पत्वस्य सुसाधत्वात् किमन्यापोहानयनेनाप्रकृतेनेति मन्यमान आह--\textbf{अत एव}--इति । यस्मादिन्द्रियज्ञानं नियतप्रतिभासम्--\textbf{अत एव} अस्मादेव हेतोः । \textbf{स्वलक्षणं} वक्ष्यमाणलक्षणम् । \textbf{अपि}रवधारणे । वाच्यग्रहणे कथमविकल्पकमित्याह--\textbf{यद्यपि हि}--इति । \textbf{यद्यपि ही}ति निपातसमुदायो “यदि नाम”शब्दस्यार्थे वर्त्तते । हिर्वाक्यालङ्कारो वा । अभिलापसंसृष्टार्थं सद् विज्ञानं विकल्पकं भवति । एवं ब्रुवतोऽयं भावः--कोष्ठशुद्ध्या वाच्यमस्तु स्वलक्षणं तथापि तदिन्द्रियज्ञानं केवलमेव स्वलक्षणमात्मन्यादर्शयति, न तु वाचकमपीति कथमभिलापसंसृष्टार्थप्रतिभासत्वं विकल्पकत्वमात्मसात् कुर्यादिति ?
	\pend
      \leavevmode\marginnote{\textenglish{53/dm}}“

	  \pstart स्वलक्षणं वाच्यं वाचकं च गृहीतं स्यात् । न च संकेतकालभावि दर्शनविषयत्वं वस्तुनः सम्प्रत्यस्ति । यथा हि संकेतकालभावि दर्शनमद्य\footnote{ग्रहणकाले--\cite{dp-msD-n}} निरुद्धम्, तद्वत् तद्विषयत्वमपि\footnote{स्मर्यमाणसंकेतस्य--\cite{dp-msD-n}} अर्थस्याद्य नास्ति । ततः पूर्वकालदृष्टत्वमपश्यच्छोत्रज्ञानं\footnote{श्रोत्रविज्ञानं--\cite{dp-msB} \cite{dp-msD} \cite{dp-edN}} न वाच्यवाचकभावग्राहि ।
	\pend
       

	  \pstart अनेनैव न्यायेन योगिज्ञानमपि सकलशब्दार्थावभासित्वेऽपि संकेतकालदृष्टत्वाग्रहणान्निर्विकल्पकम् ॥
	\pend
      ”

	  \pstart भवतु रूपाद्यालम्बनमिन्द्रियज्ञानम्, वाचकस्याप्रतिभासाद् वाच्यस्यैव प्रतिभासादविकल्पकम् । यत्पुनरेतद्द्वयप्रतिभासीन्द्रियज्ञानं तद् द्वयप्रतिभासाद् विकल्पकं प्राप्तमित्यभिप्रेत्य चोदयति--\textbf{श्रोत्रे}ति । \textbf{तर्हि}रक्षमायाम् । सर्वमिन्द्रियज्ञानमविकल्पकमिति न क्षम्यत एतदित्यर्थः । \textbf{श्रोत्रज्ञानमभिलापसंसर्गयोग्यप्रतिभासं स्यादि}ति सम्बन्धः । किम्भूतं ? \textbf{शब्दस्वलक्षणग्राहि} । हेतुभावेनास्य विशेषणत्वात् शब्दस्वलक्षणग्राहित्वादित्यर्थः । शब्दग्राहिणोऽपि वाच्याग्रहणे कथं तथात्वमित्याह--\textbf{शब्देति}\leavevmode\marginnote{\textenglish{23b/ms}} । \textbf{चो} यस्मात् । \textbf{किञ्चिद् वाचकं किञ्चिद् वाच्यम्} । यथा “तरप्तमपौ घः” \href{http://http://sarit.indology.info/?cref=Pā.1.1.22}{पाणिनि--१. १. २२}इत्यादि बुद्धिस्थम् । \textbf{इति}स्तस्मात् । अस्तु तथाप्रतिभासं किमत इत्याह--\textbf{तथा चे}ति । \textbf{तथा च} सति तस्मिश्चोक्तप्रकारे सति । एवं चैतद् द्रष्टव्यम्—यदैकेन तरप्तमपौ संज्ञिनावुच्चार्येते तदैव च कथञ्चिदन्येन “घः” इति संज्ञोच्चार्यते । तदा तद्द्वितयमेकेन श्रोत्रज्ञानेन प्रतियतः पुंसः श्रोत्रधीर्विकल्पिका प्रसज्येतेति ।
	\pend
      

	  \pstart कथं तर्हि \textbf{नैष दोष} इत्याह--\textbf{सत्यपी}ति । \textbf{चो} यस्मात् तदेवेदं यन्मया संकेतकालत्वेन गृह्यमाणं तथा गृहीतं भवति । यद्येवमस्तु सकेतकालदृष्टत्वेन ग्रहणमित्याह--\textbf{न चे}ति । \textbf{चो}ऽवधारणे हेतौ वा । \textbf{दर्शनविषयत्वमिति} ब्रुवन् पूर्वदृष्टत्वशब्दस्यार्थमाह । कथं नास्तीत्याह—\textbf{यथे}ति । \textbf{हि}र्यस्मादर्थे ।
	\pend
      

	  \pstart \textbf{अर्थस्ये}ति सम्प्रति दृश्यमानस्य । तद्भावे निगडाकर्षणन्यायेन प्राक्तनदर्शनस्यापि स्थितिः प्रसज्येतेति भावः । मा ग्रहीत् संकेतकालदृष्टत्वं तयोर्गृ ह्यमाणयोः संज्ञासंज्ञिभूतयोः शब्दयोस्तथापि कथं न वाच्यवाचकभावग्राहि तज्ज्ञानमित्याशङ्क्योपसंहरन्नाह--\textbf{तत} इति । यतः संकेतकालत्वेन गृह्यमाणं वाच्यं वाचकं च गृहीतं भवति \textbf{ततः} कारणात् श्रोत्रविज्ञानं न वाच्यवाचकभावग्राहि । भवतु तथागृह्यमाणस्य स्वलक्षणस्य तथात्वम् । किमतः ? एतत्पुनरेनं तथैव ग्रहीष्यतीत्याह—\textbf{पूर्वे}ति । हेतौ शतुर्विधानात् पूर्वकालदृष्टत्वाग्रहणादित्यर्थः ।
	\pend
      

	  \pstart स्यादेतत्--संकेतदृष्टत्वेनापि न ग्रहीष्यतेऽर्थः शब्दो वा । अथ च विशिष्टवाचकवाचकत्वेन विशिष्टार्थवाचकत्वेन च ग्रहीष्यत इति । असदेतत् । एवं हि ब्रुवतेदमभिप्रेतम्—येन ज्ञानेन योऽर्थः संकेतकालोपलब्धो यच्छब्दसंसर्गिणा रूपेण नैकीक्रियते, न स तेन तच्छब्दवाच्यो गृह्यते । यथा गोज्ञानेन संकेतकालोपलब्धाश्वशब्दसंसर्गिणाऽश्वरूपेण सहैकत्वेनाप्रतीयमानो गौर्नाश्वशब्दवाच्यो गृह्यते । श्रोत्रज्ञानेन संकेतकालोपलब्ध“घ”शब्दसंसर्गिणा च रूपेण नैकीक्रियेते च तदा तरप्तमपाविति व्यापकानुपलब्धिः । तथा येन ज्ञानेन यः शब्दः  \leavevmode\marginnote{\textenglish{54/dm}} ““
	  
	तया रहितं तिमिराशुभ्रमणनौयानसंक्षोभाद्यनाहितविभ्रमं ज्ञानं प्रत्यक्षम् ॥ ६ ॥” 
	  
	तया कल्पनया कल्पनास्वभावेन रहितं शून्यं सज्ज्ञानं यदभ्रान्तं तत् प्रत्यक्षम् इति \footnote{सूत्रेण--\cite{dp-msD-n}; परेण सूत्रेण सं० \cite{dp-msB}}परेण सम्बन्धः । कल्पनापोढत्वाभ्रान्तत्वे परस्परसापेक्षे प्रत्यक्षलक्षणम्, न प्रत्येकमिति दर्शयितुं तया रहितं यदभ्रान्तं तत् प्रत्यक्षमिति लक्षणयोः परस्परसापेक्षयोः प्रत्यक्षविषत्यवं दर्शितमिति\footnote{द्रष्टव्यमिति अध्याहारः--\cite{dp-msD-n}} ॥” संकेतकालदृष्टयदर्थसंसर्गिणा रूपेण नैकीक्रियते, न तेनासौ तदर्थवाचको गृह्यते । यथा गोशब्दज्ञानेन संकेतकालोपलब्धाश्वार्थसंसर्गिणाश्वशब्देन सहैकत्वेनाप्रतीयमानो गोशब्दो नाश्वार्थवाचको गृह्यते । श्रोत्रज्ञानेन संकेतदृष्टतरप्तमबर्थसंसर्गिणा रूपेण नैकीक्रियते च तदा “ध”शब्द इति सैव असिद्धिरनुभवेन निराकृता । वास्तवसम्बन्धनिराकरणाच्च व्याप्तिः सिद्धेति । यदा तु पूर्वदृष्टत्वेन ग्रहोऽस्ति तदा तज्ज्ञानमभिलापसंसृष्टार्थप्रतिभासं सद् विकल्परूपमेवेति सर्वमवदातम् ।
	\pend
      

	  \pstart श्रोत्रज्ञाने यथाऽभिलापसंसर्गयोग्यप्रतिभासत्वं चोदितं तथा योगिज्ञानेऽपि चोदयितुं शक्यमिति तत्राप्यमुमेव परिहारमतिदिशन्नाह--\textbf{अनेनेति} । यज्ज्ञानं संकेतकालविषयत्वं गृह्यमाणस्य न गृह्णाति, न तद् वाच्यवाचकभावग्राही\textbf{त्यनेन न्यायेन} युक्त्या \textbf{योगिज्ञानं} वक्ष्यमाण\textbf{लक्षणं सकलशब्दार्थावभासित्वे} सत्यपि \textbf{निर्विकल्पकम् ।}
	\pend
      

	  \pstart ननु चान्यस्य वाच्यवाचकभावग्राहिणो ज्ञानस्यास्तु निर्विकल्पकत्वम्, अस्य तु सर्वशब्दार्थग्राहिणः कथं तथात्वमित्याह--\textbf{संकेते}ति । तथाऽग्रहणमुक्तेन न्यायेन योगिनं प्रत्यपि पूर्वदर्शनविषयत्वस्यासत्त्वादिति भावः ।
	\pend
      

	  \pstart \textbf{तया रहित}मित्याचार्यीयं विवरणं व्याच\leavevmode\marginnote{\textenglish{24a/ms}}ष्टे--\textbf{तये}ति ।
	\pend
      

	  \pstart ननु विकल्पेऽपि विकल्पान्तरं नास्ति । ततस्तस्यापि कल्पनापोढत्वं किं न भवति ? अथ कल्पनया रहितमिति यन्न विकल्प्यते इत्युच्यते, तदपि कल्पनापोढमित्यनेनैवाकारेण विकल्प्यत इति कथमेतदुच्यत इति पश्यन् \textbf{कल्पनया रहित}मित्यस्य विवक्षितमर्थमाह--\textbf{कल्पनास्वभावेने}ति । कल्पनायाः स्वभावोऽभिलापसंसर्गयोग्यप्रतिभासत्वम्, तेन । यथा विषाणी ककुद्मान् प्रान्तेवालधिरिति विषाणादिमता\footnote{तो}विषाणादीन्येव गोत्वलिङ्गानि व्यपदिश्यन्ते । तद्वत् कल्पनाख्येन धर्मिणा धर्मोऽभिलापसंसर्गयोग्यप्रतिभासत्वाख्यो निर्द्दिष्ट इति दर्शयति । अत एव \textbf{चाचार्यदिग्नागीय}प्रत्यक्षलक्षणे कल्पनापोढत्वाख्ये \textbf{यदुद्द्योतकर}चोद्यमिहैव पूर्वमुपदर्शितं तत्खलु \textbf{धर्मकीर्त्ति}चोद्यसदृशमिति स्थितम् । \textbf{परेणे}ति परदेशस्थितेन \textbf{तदनाहितविभ्रमं ज्ञानं प्रत्यक्षमि}त्यनेन । अभ्रान्तग्रहणस्यैव तद् विवरणमिति अभ्रान्तग्रहणमिह । दूरस्थितेन सम्बन्धकरणे किम्प्रयोजनमित्याह--\textbf{कल्पनापोढे}ति । मिलितयोरनयोस्तल्लक्षणत्वात् \textbf{प्रत्यक्षलक्षणमि}त्येकवचनेन निर्देशः । \textbf{इति}ना दर्शनीयस्य रूपमाह । कथं परस्परसापेक्षतेत्याह--\textbf{तया रहितं यदभ्रान्तं तत्प्रत्यक्षं} व्यवहर्त्तव्यमिति शेषः । \textbf{इती}त्यादिनैव चोपसंहरति । यस्मादेवमितिस्तस्मात् । \textbf{प्रत्यक्षविषयत्वं दर्शितम}न्ते प्रत्यक्षग्रहणसामर्थ्यात् ।
	\pend
      \leavevmode\marginnote{\textenglish{55/dm}}“

	  \pstart तिमिरम् अक्ष्णोर्विप्लवः । इन्द्रियगतमिदं \footnote{०मिदं भ्रम० \cite{dp-msB}}विभ्रमकारणम् । आशुभ्रमणम् अलातादेः । मन्दं हि\footnote{मन्दं भ्रम्य A; मन्दं हि भ्राम्य० \cite{dp-msC} \cite{dp-msD} \cite{dp-msB} \cite{dp-edN}} भ्रम्यमाणेऽलातादौ न चक्रभ्रान्तिरुत्पद्यते । तदर्थम् आशुग्रहणेन विशेष्यते भ्रमणम् । एतच्च विषयगतं विभ्रमकारणम् । नावा गमनं नौयानम् । गच्छन्त्यां नावि स्थितस्य गच्छद्वृक्षादिभ्रान्तिरुत्पद्यते इति यानग्रहणम् । एतच्च बाह्याश्रयस्थितंर विभ्रमकारणम् । संक्षोभो वातपित्तश्लेष्मणाम् । वातादिषु हि क्षोभं गतेषु ज्वलितस्त\footnote{ज्वलितरूपस्त० \cite{dp-msB} \cite{dp-msD}}म्भादिभ्रान्तिरुत्पद्यते एतच्चाध्यात्मगतं\footnote{०ध्यात्मिकं भ्रान्तिका० \cite{dp-msB} \cite{dp-msD}} विभ्रमकारणम् ।
	\pend
       

	  \pstart सर्वैरेव च विभ्रमकारणैरिन्द्रियविषयबाह्याध्यात्मिकाश्रयगतैरिन्द्रियमेव विकर्त्तव्यम् । अविकृते इन्द्रिये इन्द्रिय\footnote{ज्ञानस्य--\cite{dp-msD-n}; अविकृते इन्द्रिये भ्रान्त्ययोगात्--\cite{dp-msB}}भ्रान्त्ययोगात् । एते संक्षोभपर्यन्ता आदयो येषां ते तथोक्ताः । आदिग्रहणेन काचकामलादय इन्द्रियस्था गृह्यन्ते । आशुनयनानयनादयो विषयस्थाः । आशुनयनानयने हि\footnote{नयने कार्य \cite{dp-msB} \cite{dp-msD} \cite{dp-edN}} कार्यमाणेऽलाते\footnote{०लातादौ--\cite{dp-msA} \cite{dp-edP} \cite{dp-edH} \cite{dp-edN}}ऽन्निवर्णदण्डाभासा भ्रान्तिर्भवति । हस्तियानादयो बाह्याश्रयस्थाः, गाढमर्मप्रहारादय आध्यात्मिकाश्रयस्था विभ्रमहेतवो गृह्यन्ते ।
	\pend
       

	  \pstart \footnote{एतैर० \cite{dp-msB} \cite{dp-msD}}तैरनाहितो विभ्रमो यस्मिंस्तथाविधं ज्ञानं प्रत्यक्षम् ॥
	\pend
      ”

	  \pstart \textbf{तिमिर}मक्ष्णोर्विप्लवहेतुत्वाद् \textbf{विप्लवः । इन्द्रियगतमि}न्द्रियाश्रितम् । \textbf{विभ्रम}स्य भ्रान्तत्वस्य \textbf{कारणं} साक्षात् कारणत्वेन । \textbf{अलातं} विदग्धकाष्ठम् । आशुग्रहणस्य फलमाह—\textbf{मन्दम्} इति । यद्यपि मन्दं भ्रम्यमाणस्यालातस्य विभ्रमकारणत्वादर्शनादाशु भ्रमणं लभ्यते, तथापि शास्त्रकृता स्वकीयवचनाकौशलपरिहारार्थ\textbf{माशु}ग्रहणं कृतमित्यवसेयम् । पूर्वस्मादेतद्भेदेन कथमुक्तमित्याह--\textbf{एतद्} इति । \textbf{चो} यस्मादर्थे । \textbf{आशु}ग्रहणेन नौयानमपि विशेषणीयम् । मन्दं हि गच्छन्त्यां नावि न गच्छद्वृक्षादिदर्शनं भवतीत्यनुभवसिद्धमेतत् । \textbf{इति}र्हेतौ । एतत् कस्मात् पृथगुक्तमित्याह \textbf{एतच्चे}ति । \textbf{चो} यस्मात् । \textbf{संक्षोभ} उपचयः । प्रकोप इति यावत् । स केषामित्याकाङ्क्षायामाह--\textbf{वाते}ति । नासौ विभ्रमकारणमित्याशङ्कामपनुदन्नाह--\textbf{वातादिषु}--इति । हिर्यस्मात् । \textbf{क्षोभं} प्रकोपं \textbf{गतेषु} संक्षुब्धेष्विति यावत् ।
	\pend
      

	  \pstart स्यादेतत्--अस्तु तिमिरस्येन्द्रियोपघातहेतुकत्वाद् भ्रान्तज्ञानहेतुत्वम् । नौयानादीनां तु कथं सत्स्वपि तेषु तदवस्थत्वादिन्द्रियस्येत्याह--\textbf{सर्वैरेवेति} । इन्द्रियगतस्य तिमिरादेरिन्द्रियविकारत्वेन सम्मतस्याप्युपादानं दृष्टान्तार्थम् । तेनायमर्थः--यथेन्द्रियगतं तिमिरादीन्द्रिय विकुर्वद् विभ्रमहेतुः, तथा अन्यैरपि तद्विकुर्वद्भिरेव विभ्रमहेतुभिर्भाव्यमिति ।
	\pend
      

	  \pstart कस्मात्पुनरमीभिरिन्द्रियविकारोऽवश्यकर्त्तव्य इत्याह--\textbf{अविकृत} इत्यादि । एतदुक्तं भवति । तद्विकारविकारित्वादाश्रयाश्चक्षुरादय इति । प्रयोग\leavevmode\marginnote{\textenglish{24b/ms}}स्त्वेवं कर्त्तव्यः । यदि  \leavevmode\marginnote{\textenglish{56/dm}} “
	  
	तदेवं लक्षणमाख्याय \footnote{वैभाषिकैः--\cite{dp-msD-n}}यैरिन्द्रियमेव द्रष्टृ कल्पितं मानसप्रत्यक्षलक्षणे च दोष उद्भावितः, स्वसंवेदनं च नाभ्युपगतम्, योगिज्ञानं च; तेषां विप्रतिपत्तिनिराकरणार्थं\footnote{निरासार्थम्--\cite{dp-msB} \cite{dp-msC} \cite{dp-msD}} प्रकारभेदं प्रत्यक्षस्य दर्शयन्नाह-- “
	  
	तत् चतुर्विधम् ॥ ७ ॥” 
	  
	तत् चतुर्विधमिति ॥ “
	  
	इन्द्रियज्ञानम् ॥ ८ ॥” 
	  
	इन्द्रियस्य ज्ञानम् इन्द्रियज्ञानम् । इन्द्रियाश्रितं यत् तत् प्रत्यक्षम् ।” न्द्रियजभ्रमकारणं तदिन्द्रियं विकरोत्येव यथा तिमिरादि । इन्द्रियजभ्रमकारणं च नौयानादीति स्वभावहेतुः । एवंविधस्य ज्ञानस्य भ्रमरूपत्वं बाधवशादवसितमिन्द्रियान्वयव्यतिरेकानुविधानाच्चेन्द्रियजत्वम् । तथाभूतविभ्रमकारणत्वं च नौयानादेः स्वान्वयव्यतिरेकानुविधानात् सिद्धम् । व्याप्तिरपि तिमिरादौ दर्शिता । अयमेव प्रयोगो दर्पणादिष्वपीन्द्रियजभ्रमनिमित्तेषु द्रष्टव्यः । उपयुक्तकार्त्स्न्ये वाऽयं \textbf{सर्व}शब्दः प्रवर्त्तनीयः । \textbf{एत} इत्यादि \textbf{विभ्रमहेतवो गृह्यन्त} इत्यन्तं सुबोधम् । केवलं \textbf{पर्यन्तो}ऽन्तो वाच्यः । \textbf{काचो}ऽक्षिविकारहेतू रोगविशेषः । \textbf{कामलोच} \footnote{लोऽत्र} नयनवदनादिविकारकारणं व्याधिविशेष एवेति द्रष्टव्यम् ।
	\pend
      

	  \pstart \textbf{तैरनाहितो विभ्रमो यस्मिन्नि}ति विगृह्णंस्त्रिपदं बहुव्रीहिं दर्शयति । अनाहितो विभ्रमो यस्मिंस्तत्तथेति प्रसाध्य तेषामनाहितविभ्रम इति षष्ठीतत्पुरुषः कार्यः । इदन्त्वर्थकथनमिति बोद्धव्यम् । यो हि तैरनाहितविभ्रमः स तेषां भवतीति । ज्ञानाधिकारेण लक्षणविधानाल्लब्धं ज्ञानम्, तेन \textbf{ज्ञान}मित्युक्तम् ॥
	\pend
      

	  \pstart सम्प्रत्याचार्यस्यावान्तरप्रत्यक्षभेदव्युत्पादने निमित्तं दर्शयन्नाह--\textbf{तदेव}मिति । निपातसमुदायश्चायमुक्तेन प्रकारेणेत्यस्यार्थे वर्त्तते । \textbf{यै}रिति प्रत्येकं योजनीयम् । \textbf{यैः} स्वयूथ्यै\textbf{र्वैभाषिकैः} । यदि ज्ञानं द्रष्टृ स्यात्, तस्याप्रतिघत्वात् व्यवहितमपि गृह्णीयात् । ततः “चक्षुः पश्यति रूपाणि” \href{http://http://sarit.indology.info/?cref=ak.1.42}{अभिध० १. ४२} इत्यादिवादिभिरिन्द्रियमेव द्रष्टृ कल्पितम् । \textbf{यैर्मीमांसकैराचार्यदिग्नागीये} मानसप्रत्यक्षलक्षणे गृहीतग्राहित्वलक्षणोऽप्रामाण्यनिमित्तं दोष उद्भावितः । लक्षणग्रहणस्योपलक्षणत्वात् मानसप्रत्यक्षाभ्युपगमेऽपि यो दोषोऽन्धबधिराद्यभावलक्षणः सोऽपि द्रष्टव्यः । \textbf{यैश्च मीमांसकैः कुमारिल}मतानुसारिभि\textbf{र्नैयायिकवैशेषिकैश्च} स्वात्मनि क्रियाविरोधेन स्वसंवेदनं नाभ्युपगतम् । \textbf{यै}श्च \textbf{चार्वाकमीमांसकैर्यो}गिन एव न सम्भवन्ति, कुतस्तेषां ज्ञानमिति \textbf{वादिभिर्योगिज्ञानं च} नाभ्युपगतमिति वर्त्तते । त्रयोऽपि चकारा वक्तव्यान्तरं समुच्चिन्वन्ति । \textbf{तेषाममीषां} तत्र तत्र या विमतिस्त\textbf{न्निराकरणार्थं} प्रत्यक्षस्योक्तलक्षणस्य \textbf{प्रकारो} विशेषस्तस्यं \textbf{भेदं} नानात्वम्, अवान्तरजातिभेदमिति यावत् ।
	\pend
      

	  \pstart तदिति प्रत्यक्षं \textbf{चतुर्विधं} चतुष्प्रकारम् ॥
	\pend
      

	  \pstart तत्रैकं तावदिन्द्रियज्ञानमुक्तं व्याचक्षाण आह--\textbf{इन्द्रियस्ये}ति । ज्ञायतेऽनेनेति \textbf{ज्ञानम्,}  \leavevmode\marginnote{\textenglish{57/dm}} “
	  
	मानस\footnote{मानसे च प्रत्य० \cite{dp-msB} \cite{dp-msD}} प्रत्यक्षे परैर्यो\footnote{परैर्दोष० \cite{dp-msD}} दोष उद्भावितस्तं निराकर्त्तुं मानसप्रत्यक्षलक्षणमाह-- “
	  
	स्वविषयानन्तरविषयसहकारिणेन्द्रियज्ञानेन समनन्तरप्रत्ययेन जनितं तन्मनोविज्ञानम् ॥ ९ ॥”” इन्द्रियं च तज्ज्ञानं चेति कृत्वा शक्यमिन्द्रियज्ञानमिति वक्तुम्, तत्कथं विप्रतिपत्तिर्निराकृतत्याशङ्कामपाकर्त्तुम् \textbf{इन्द्रियज्ञान}मित्यत्राचार्यस्ये\textbf{इन्द्रियस्य ज्ञानमिति} षष्ठीतत्पुरुषो विवक्षित इति दर्शयति-\textbf{इन्द्रियस्य ज्ञानमि}ति । यदिन्द्रियेण जन्यते तदिन्द्रियस्य भवतीति भावः ।
	\pend
      

	  \pstart ननु चेन्द्रियानुमानमपि कार्यप्रभवं भवतीन्द्रियस्य ज्ञानम् । ततस्तस्यापि प्रत्यक्षत्वं प्राप्तमित्याह \textbf{इन्द्रियाश्रितमि}ति । आश्रितं जन्यतया तस्मिन् सत्येव भावात् । न चेन्द्रियानु\leavevmode\marginnote{\textenglish{25a/ms}}मानमिन्द्रियेण साक्षाज्जन्यते, येन तस्य प्रत्यक्षत्वं स्यादिति भावः । अनेकहेतुत्वेऽपि ज्ञानस्येन्द्रियेण व्यपदेशोऽसाधारणत्वात्तस्येति द्रष्टव्यम् । इन्द्रियस्य ज्ञानं प्रत्यक्षमिति कथयता नेन्द्रियस्य द्रष्टृत्वमिति दर्शितम् । ततो विप्रतिपत्तिर्निराकृता ।
	\pend
      

	  \pstart ननु इन्द्रियज्ञानस्य द्रष्टृत्वे तस्याप्रतिघत्वाद् व्यवहितस्यापि ग्रहणं प्रसन्येतेति तैरुक्तो दोषः । स कथं परिहृतो येन विप्रतिपत्तिर्निराकृतेति चेत् । अयमाशयः--न ज्ञानं गत्वा गत्वाऽर्थं गृह्णाति, येन गमनविबन्धाभावात् व्यवहितमपि गृह्णीयात् । किन्तर्हि ? यदाकारमुत्पद्यते तद् गृह्णातीत्युच्यते । न चायोग्यदेशस्थोऽर्थस्तत्स्वरूपकोऽन्वयव्यतिरेकाभ्या विज्ञातः । तत्कथं तस्य तेन ग्रह इति सुज्ञानमेतदिति किमत्रोपपत्त्याभिहितयेति ? इन्द्रियज्ञान प्रत्यक्षमिन्द्रियादिसामग्रीजन्यज्ञानं प्रत्यक्षम्, एकस्य जनकत्वविरोधात् । नाज्ञानमित्यथात् ।
	\pend
      

	  \pstart एवं च दर्शयता \textbf{वार्त्तिककारेण} “यद्वेन्द्रियं प्रमाणं स्यात् तस्य वार्थेन सङ्गतिः” \href{http://http://sarit.indology.info/?cref=śv.4.60}{शलोकवा०
	    ४--६०} इत्यादि वचनात्तस्यापि या प्रत्यक्षप्रमाणता \textbf{मीमांसका}दिभिरभ्युपगता सा निरस्ता वेदितव्या । तथाहि--प्रापकं प्रमाणं भवेत् । नान्यत् । प्रापकत्वं च प्रवर्त्तकत्वेना\add{व्यवधानेन नान्यथा} । प्रवर्त्तकत्वं चाधिगमात्मकत्वेन नान्यथा । न च तस्य तथात्वमस्ति । अधिगमोपगमेन प्रवर्त्तकत्वे च नैव व्यवधीयते । यतश्चाव्यवधानेन प्रवृत्तिः स एव प्रवर्त्तकत्वात् प्रमाणं युक्तम् । तच्च ज्ञानमेव । तदुक्तम्, “
	    \pend
	  
	    
	    \stanza[\smallbreak]
	धीप्रमाणतां&प्रवृत्तेस्तत्प्रधानत्वाद्धेयोपादेयवस्तुनि\&[\smallbreak]


	
	    \pstart
	  ” \href{http://http://sarit.indology.info/?cref=pv.1.5}{प्रमाणवा० १. ५} इति ।
	\pend
      

	  \pstart यद्वा प्रमायाः करणात् प्रमाणम् । प्रमा च नीलस्येयं न पीतस्येति विशिष्टेनैवाधिकरणेनावच्छिन्ना प्रतीयत इति तत्करणं भवितुमर्हति । यत इयमव्यवधाना ताद्रूप्येण व्यवस्थां लभते । इन्द्रियार्थसन्निकर्षादिश्च नीलपीतादि\footnote{अस्पष्टम्--सं०}\add{... ... ...}स्य भेदव्यवस्थाङ्गम् । स्वभेदेनं क्रियाभेदनिबन्धनं च करणं ततो ज्ञानात्मकमेव सारूप्यमधिगतिक्रियाभेदव्यवस्थाङ्गं प्रमाणं युज्यत इति । द्वयी चेयं विधाऽन्यत्रावार्येण विपञ्चितेति नेहोच्यते, प्राज्ञजनाधिकारेणास्य प्रकरणस्य प्रारम्भात् । दिङ्मत्रिं तूक्तमिति ।
	\pend
      \leavevmode\marginnote{\textenglish{58/dm}}“

	  \pstart स्व आत्मीयो विषय इन्द्रिय\footnote{इन्द्रियविज्ञान० \cite{dp-msB}} ज्ञानस्य तस्य अनन्तरः--न विद्यतेऽन्तरमस्येति\footnote{०स्येति अनन्तरः \cite{dp-msB} \cite{dp-msD}} । अन्तरं \footnote{“च” नास्ति--\cite{dp-msC}}च व्यवधानं विशेषश्चोच्यते । ततश्चान्तरे प्रतिषिद्धे समानजातीयो द्वितीयक्षणभाव्युपादेयक्षण इन्द्रियविज्ञानविषयस्य गृह्यते । तथा च सति इन्द्रिय\footnote{इन्द्रियविज्ञान०--\cite{dp-msC}} ज्ञानविषयक्षणादुत्तरक्षण एकसन्तानान्तर्भूतो गृहीतः । स सहकारी यस्येन्द्रियज्ञानस्य\footnote{इन्द्रियविज्ञान० \cite{dp-msA} \cite{dp-msB} \cite{dp-msD} \cite{dp-edP} \cite{dp-edH} \cite{dp-edE} \cite{dp-edN}} तत् तथोक्तम् । द्विविधश्च सहकारी—परस्परोपकारी एककार्यकारी च । इह च क्षणिके वस्तुन्यतिशयाधानाऽयोगादेककार्यकारित्वेन\footnote{अत्र प्रदीपसम्मतः--“एकार्थक्रियाकारित्वेन” इति पाठः ।--सं०} सहकारी गृह्यते ।
	\pend
      ”

	  \pstart ननु चतुर्विधप्रत्यक्षाभिधाने निमित्तमुक्तमेव धर्मोत्तरेण तत्कथं तदेव \textbf{मानसे}त्यादिना पुनराचष्टे । अथाचार्यस्य चतुष्टयाभिधाने निमित्तं यदासीत् बुद्धिस्थं तत्प्रागुक्तम्, तदेव तु यथायोगं प्रत्यक्षव्यक्तिविशेषाभिधानेऽभिधानीयमिति । तर्हि यैरिन्द्रियं द्रष्टृ कल्पितं तन्निराकरणार्थमाह--इन्द्रियंज्ञानमिति तथाऽन्यत्रापि स्वसंवेदनादौ तत्तन्निमित्तमनेन किन्नोक्तम् ? तदत्रापि मानसं व्याख्यातुमाहेति वक्तुं युक्तमिति । सत्यमेतत् । केवलं मानसस्य प्रत्यक्षस्याशक्यनिश्चयत्वेनाव्यवहाराङ्गत्वात्परोक्तदोषनिराकरणमात्रमेवास्य लक्षणाख्याने निमित्तं नान्यदिति प्रतिपादयितुमुक्तमेव निमित्तमनूदितमिति न किञ्चिदवद्यम् । अथेन्द्रियज्ञानस्यापि लक्षण\leavevmode\marginnote{\textenglish{25b/ms}}मन्येषामिव किं नोक्तम् ? नोक्तम् । तल्लक्षणविप्रतिपत्त्यभावात् । इन्द्रियज्ञानमिति निगदेनैव व्याख्यातत्वाच्चेति ।
	\pend
      

	  \pstart स्यान्मतम् । अभिमतं चतुर्विधं प्रत्यक्षं दर्शयितव्यम् । तच्चैकदा वक्तुमशक्यत्वात् क्रमेण वचनीयम् । क्रमश्चैवंविधोऽस्त्वन्यादृशो वा किमत्रादरेणेति इन्द्रियज्ञानादिरीदृशः क्रमः कृतः, अथान्यदप्यस्य निमित्तमस्तीति ? अन्यदप्यस्तीत्युच्यते । तथाहि सर्वस्य सर्वव्यवहाराङ्गमिन्द्रियज्ञानमिति तदादितोऽभिहितम् । तेनैवेन्द्रियज्ञानेन तथाभूतेन मानसं जन्यत इति तदनु मानसं व्याख्यातम् । तस्य पश्चात्प्रत्यात्मवेद्यतयाऽस्तिवेन शक्यनिश्चयतया च स्वसंवेदनमुदितम् । ततः सर्वज्ञज्ञानस्य सामग्रीमात्रप्रदर्शनतयऽन्तिमपुरुषार्थतया चान्ते योगिज्ञानमुपदर्शितमिति ।
	\pend
      

	  \pstart \textbf{स्व} इत्यादिना \textbf{स्वविषये}त्यादिलक्षणं व्याचष्टे । तच्च \textbf{तत्तथोक्तमि}त्येतदन्तं सुगमम् ।
	\pend
      

	  \pstart नन्वनुपकारकस्य सहकारित्वेऽतिप्रसङ्गाद् उपकारिणा सहकारिणा भाव्यम् । समसमययोश्चोपकार्योपकारकभावाभावात् कथमसौ विषयक्षणस्तस्य सहकारीत्याशङ्क्याह--\textbf{द्विविधश्चे}ति । \textbf{चो} यस्मात् । \textbf{द्विविधो} द्विप्रकारः सहकरणशीलो भावः । कथं द्वैविध्यमित्याह—\textbf{परस्परे}त्यादि । परस्परोपकारितया च सहकारित्वमेकत्वाध्यवसायाधीनं सन्तानगतममुख्यम् । मुख्यं तु एकार्थक्रियाकारित्वेन । यदि द्वैविध्येऽपि प्राक्तनस्य ग्रहणं तदा तदवस्थो दोष इत्याह—\textbf{इहे}ति । \textbf{इह} मानसक्रियायाम् । \textbf{चो}ऽवधारणे । \textbf{एकार्थक्रियाकारित्वेने}त्यस्मात् परो द्रष्टव्यः । \textbf{सहकारी हि} गृह्यते विषयक्षण इति प्रकरणादवसेयम् । पूर्वं कस्मान् न गृह्यत  \leavevmode\marginnote{\textenglish{59/dm}} “
	  
	विषयविज्ञानाभ्यां \footnote{भ्यां मनो \cite{dp-msD} \cite{dp-msB}}हि मनोविज्ञानमेकं क्रियते यतस्तदनयोः\footnote{०नयोर्न परस्प० \cite{dp-msA} \cite{dp-msC} \cite{dp-edP} \cite{dp-edH} \cite{dp-edE} \cite{dp-edN}} परस्पर \footnote{परस्परस्य सह \cite{dp-msB} \cite{dp-msD}}सहकारित्वम् । 
	  
	\footnote{स्वविषयानन्तरेत्यादिना--\cite{dp-msD-n}}ईदृशेनेद्रिन्यविज्ञानेनाल\footnote{आलम्बनप्रत्ययभूतेनापि \cite{dp-msA} \cite{dp-msB} \cite{dp-msC} \cite{dp-msD} \cite{dp-edP} \cite{dp-edH} \cite{dp-edE} \cite{dp-edN} यदा योगिना परस्यैवंविधज्ञानमालम्ब्यते तदालम्बनभूतेन तेन च योगिज्ञानं जन्यते इति--\cite{dp-msD-n}}म्बनभूतेनापि\footnote{अपिशब्दो भिन्नक्रमे--योगिज्ञानमपि इति--\cite{dp-msD-n}} योगिज्ञानं जन्यते । तन्निरासार्थं \footnote{बौद्धानां मते समनन्तरप्रत्ययेति उपादानकारणमुच्यते--\cite{dp-msD-n}}समनन्तरप्रत्ययग्रहणं \footnote{“कृतम्” इति नास्ति \cite{dp-msD} \cite{dp-msB}}कृतम् । समश्चासौ ज्ञानत्वेन, अनन्तरश्चा\footnote{श्चाव्यव \cite{dp-msD} \cite{dp-msB}}सौ अव्यवहितत्वेन, स चासौ प्रत्ययश्च हेतुत्वात् समनन्तरप्रत्ययः, तेन जनितम् ।” इत्याह--\textbf{क्षणिक} इति । हेतुभावेन विशेषणम् । अथ यदि वस्तुनः क्षणिकत्वेनातिशयाधानासम्भवात् परस्परोपकारित्वेन न सहकारी गृह्यते \textbf{द्विविधश्च सहकारी}त्यनन्तरमुदितमनेन व्याहन्येत । \textbf{हेतुबिन्दुश्च} विरुध्येतेति सर्वमसमञ्जसम् । न । अभिप्रायापरिज्ञानात् । तथा हि \textbf{क्षणिके वस्तुनी}ति धारयाऽप्रवाहिणि सजातीयाप्रसवधर्मिण्यन्तिमे मानसोत्पादकतयाऽभिमत इन्द्रियज्ञाननाम्नि निरोधोन्मुख इत्यभिप्रेतम् । न त्वक्षणिकव्यावृत्त्या क्षणिकपदाभिलाप्येऽवस्तुमात्रव्यावृत्त्या च वस्तुमात्र इति ।
	\pend
      

	  \pstart एतदुक्तं भवति । यदीदं मानसोपजननोन्मुखमिन्द्रियज्ञानमेकजातीयप्रवाहवाहि स्यात्तदा सन्तानोपकारद्वारकातिशयाधायकत्वे विषयोऽस्य सहकारीति कल्प्येत । केवलमिदमपेतसजातीयप्रसवशक्तिकं सन्तानवर्त्तीति कथमस्य विषयस्तथा सहकारी कल्प्येतेति । ननु किं तदेकं यत्कुर्वतोरेतयोस्तथासहकारित्वमुच्यत इत्याह \textbf{विषये}ति । \textbf{हि}रवधारणे । \textbf{यतो} यस्मादाभ्यामेव \textbf{तदेकं क्रियते,} तस्मात् । \textbf{तदि}त्ययं तस्मादित्यस्मिन्नर्थे । अन्योन्यस्य सहकारित्वमेकार्थक्रियाकारित्वेनेति प्रकरणात् ।
	\pend
      

	  \pstart ईदृशेनेत्यादिना समनन्तरप्रत्ययग्रहणस्य व्यवच्छेदं दर्शयति । \textbf{ईदृशेन} स्वविषयानन्तरविषयग्रह\leavevmode\marginnote{\textenglish{26a/ms}}णसहकारिणा । आलम्ब्यत इति \textbf{आलम्बनभूतं} रूपं यस्य तेन ग्राह्यस्वभावेन सता जन्यत इत्यर्थः । तस्य योगिज्ञानस्य \textbf{निरासर्थं} मानसत्वनिराकरणार्थम् । एवञ्च व्याचक्षाणेन “समनन्तरप्रत्ययग्रहणमन्धबधिराद्यभावप्रसङ्गनिराकरणार्थमिति” यदन्येन व्याख्यातम्--तदपहस्तितम् । इन्द्रियज्ञानग्रहणेनैव तत्प्रसङ्गस्य निराकृतत्वादिति ।
	\pend
      

	  \pstart \textbf{समश्चे}त्यादिना समनन्तरप्रत्ययशब्दस्य विग्रहमर्थं चाह । हेत्वर्थः प्रत्ययार्थः । शकन्ध्वादिपाठाच्च न दीर्घत्वम् । यदि नामैवं व्युत्पत्तिस्तथापि प्राक्तनमेव विज्ञानमुपादानभूतमागमे तथा रूढं नान्यत् । तदुक्त\textbf{मभिधर्मकोशे}--“चित्तचैत्ता अचरमा उत्पन्नाः समनन्तरः” \href{http://http://sarit.indology.info/?cref=ak.2.62}{२. ६२} इति । सिद्धान्तप्रसिद्धया चामुया संज्ञया व्यवहरताऽऽचार्येण सिद्धान्तप्रसिद्धत्वमस्य दर्शितम् ।
	\pend
      \leavevmode\marginnote{\textenglish{60/dm}}“

	  \pstart तदनेनैक\unclear{स}न्तानान्तर्भूतयोरेवेन्द्रिय\footnote{०न्द्रियविज्ञान--\cite{dp-msC}}ज्ञान-मनोवि\footnote{मनोज्ञान \cite{dp-msA} \cite{dp-msB} \cite{dp-edP} \cite{dp-edH} \cite{dp-edE}}ज्ञानयोर्जन्यजनकभावे मनोविज्ञानं \unclear{प्रत्यक्षमि}\footnote{०क्षमुक्तं B--}त्युक्तं भवति । ततो योगिज्ञानं परसन्तानवर्त्ति निरस्तम् ।
	\pend
      ”

	  \pstart ननु किं कृतं समनन्तरप्रत्ययग्रहणेन येन योगिज्ञानस्य तथात्वेन निरास इत्याह--तदिति । यस्मात्समनन्तरप्रत्ययरूपेणेन्द्रियज्ञानेन यज्जनितं तन्मानसं तत्तस्मात्कारणादनेन समनन्तर\unclear{प्रत्ययग्रहणेनैकसन्तानान्तर्भू} तयोरेकसन्ततिपतितयोरेव \textbf{जन्यजनकभावे} सति \textbf{मनोविज्ञानमि}ति । \unclear{साक्षादिन्द्रियाजन्यतया} मनोमात्राश्रितत्वान्मानसमेवोक्तम् । आगमेऽपि मनःशब्देनैतदेवोच्यत इति \unclear{प्रदर्शनार्थं} चैवमभिधानं मन्तव्यम् । तथापि कथं योगिज्ञानं तथात्वेन निरस्तमित्याह—\unclear{तत्र} इति । यस्मादु\unclear{पदानो}पादेयभूतयोर्ग्रहणं भिन्नसन्तानवर्त्तिनोश्चोपादानोपादेयभावाभावः, \unclear{तस्माद् योगिज्ञानं निरस्तं मान}सत्वेन प्रतिक्षिप्तम् । यदि पुनरिन्द्रियज्ञानेन जनितमित्येतावदुच्येत तदाऽऽलम्बनभूतेनापि तेनादौ जन्यत इति निरस्तं न स्यादिति भावः ।
	\pend
      

	  \pstart \unclear{स्यादेतत्--यद्ये}तदर्थं समनन्तरप्रत्ययग्रहणं तर्हि नितरां न कर्त्तव्यम् । स्वविषयानन्तर\unclear{विषयक्षणसहकारिणेत्यनेनैव} योगिज्ञानस्य निरस्तत्वात् । न हि तदिन्द्रियज्ञानं योगिज्ञाने कर्त्तव्ये \unclear{स्वविषयानन्तरवि}षयक्षणसहकारीति । अथोच्यते--यदेन्द्रियज्ञानं योगिज्ञानं जनयति तदा \unclear{स्वो}पादेयमपि ज्ञानं जनयत्येव । ततः स्वसन्ततिपतितजन्यक्षणापेक्षया तद् भवति स्वविषयानन्तरविषयक्षणमहकारीत्यस्ति तज्जनितस्य तस्य मानसत्वलक्षणमिति । अहो शब्दार्थव्यवस्थापनकौशलम्? यदेव हि जन्यतया प्रकृतं तदपेक्षमेव सहकारिणः सहकारित्वं चिन्तनीयम् । न तु यत्किञ्चिदपेक्षम् । न च तदेव योगिज्ञानं प्रति तत् सहकारि तदिन्द्रियज्ञानं येन तथात्वमस्योच्यमानं शोभेत । अथ तत् सहकारि तद् वस्तुवृत्त्या तदिन्द्रियज्ञानम् । \unclear{तत् किम्} तदपेक्षया चिन्तयेति चेत् । यद्येवं कृतेऽपि समनन्तरप्रत्ययग्रहणे योगिज्ञानं निरसितु\unclear{मशक्यं} तथापि यथा तदिन्द्रियज्ञानं स्वोपादेयज्ञानापेक्षं स्वविषयानन्तरविषयसहकारि तथा तदेवापेक्ष्य समनन्तरप्रत्ययो\unclear{ऽ}प्यदो भ\leavevmode\marginnote{\textenglish{26b/ms}}वत्येव । ततस्तथाभूतेन तेन जन्यमानस्य योगिज्ञानस्य कथं तथात्वेन निरासः । ननु न तदपेक्षं तस्य समनन्तरप्रत्ययत्वम् । हन्त तत्सहकारित्वमपि किमस्य तदपेक्षमस्ति ? अथ समनन्तरप्रत्ययत्वं प्रकृतजन्यापेक्षं ग्राह्यम्, \unclear{न तु वास्तवं} रूपमुपादेयं फलाभावेन तस्यानुवाद्यताऽनुपपत्तेः । अन्यथा समस्तवास्तवरूपानु\unclear{वाद्यताप्रसं}\add{गा}दिति चेत् । समानमिदं स्वविषययानन्तरविषयक्षणसहकारिग्रहणेऽपि । न च \unclear{राजशासनस}मानं किञ्चिदुत्पश्यामो येन समनन्तरप्रत्ययजन्यत्वमेव प्रकृतजन्यापेक्षमिह ग्राह्यम्, न तु स्वविषयानन्तरविषयक्षणसहकारित्वमिति कल्पनीयमिति ।
	\pend
      

	  \pstart अत्रोच्यते । यदा--योगी इन्द्रियज्ञानं तत्सहभुवं चार्थक्षणं च कुतश्चित्कारणाद्दिदृक्षमाणो योगिज्ञानेन तद्द्वयमपि पश्यति तदा तेनेन्द्रियज्ञानेन स्वविषयानन्तरविषयक्षणसहकारिणा तद् योगिज्ञानं जन्यत इति असति समनन्तरप्रत्ययग्रहणे मानसप्रत्यक्षं समासज्येत । ततस्तन्निवृत्त्यर्थं कर्त्तव्यमेव समनन्तरप्रत्ययग्रहणम् । एवंविधमेव च योगिज्ञानमभिप्रत्येत्येदृशेनेत्येभिहितं \textbf{धर्मोत्तरेणे}ति सर्वमवदातम् ।
	\pend
      \leavevmode\marginnote{\textenglish{61/dm}}“

	  \pstart \footnote{अत्र परस्यायमाशयः--मनोविज्ञानमिन्द्रियसव्यपेक्षम् । अथ चेन्द्रियविज्ञानं[[ना]]दन्यो विषयस्तस्य । ततोऽन्धबधिराद्यभावो व्यवहितस्य नीलादेर्ग्रहणं च प्राप्नोतीत्याह \textbf{यदे}त्यादि । अयमर्थः--यस्मादिन्द्रियविज्ञानविषयोपादेयभूतः क्षण एको गृहीतक्षणो न भवत्यस्य विषयः, अन्धबधिरादेश्च नेन्द्रियज्ञानविषयसहकारि विद्यते तेन तेषां न मनोविज्ञानं भवतीत्यर्थः--\cite{dp-msD-n}}यदा चेन्द्रियज्ञानविषयादन्यो विषयो मनोविज्ञानस्य तदा गृहीतग्रहणादासञ्जितोऽ\footnote{कल्पितः--\cite{dp-msD-n}}प्रामाण्यदोषो निरस्तः ।
	\pend
       

	  \pstart यदा चेन्द्रिय\footnote{०न्द्रियज्ञान० \cite{dp-msA} \cite{dp-msB} \cite{dp-msC} \cite{dp-edP} \cite{dp-edH} \cite{dp-edE} \cite{dp-edN}} विज्ञानविषयोपादेयभूतः\footnote{०भूतः क्षण एको गृहीतः \cite{dp-msD}} क्षणो गृहीतः, तदेन्द्रियज्ञानेनागृहीतस्य विषयान्तरस्य ग्रहणादन्धबधिराद्यभावदोषप्रसङ्गो\footnote{प्रसङ्गोऽपि निर० \cite{dp-msD}}निरस्तः ।
	\pend
      ”

	  \pstart सम्प्रति यथा गृहीतग्राहित्वद्वारको दोषः परस्य परिह्रियते तथा दर्शयन्नाह--\textbf{यदा चे}ति । \textbf{चो} व्यक्तमेतदित्यस्मिन्नर्थे । \textbf{यदैवं} तदा निरस्तः । इन्द्रियज्ञानविषयादन्योऽस्य विषय इत्येतदेव कुतो येनासौ निरस्तो भवति इत्याशङ्क्याह--\textbf{यदा चे}ति । \textbf{चोऽ}वधारणे । \textbf{इन्द्रियविज्ञानविषयोपादेयभूत} इत्यस्मात्परो द्रष्टव्यः । इन्द्रियज्ञानस्य विषयो यस्तस्योपादेयभूतो यः क्षणः स \textbf{गृहीत} इन्द्रियज्ञानस्य मानसोपजनने सहकारिकारणत्वेन स्वीकृतस्तदेतीहैवच्छेदः कर्त्तव्यः । एवं ब्रुवतेदमाकूतम्--जनकविशेषस्य विषयत्वान्मानसप्रत्यक्षस्य विषयेण भवता जनकेनावश्यभाव्यम् । जनकश्चेन्द्रियज्ञानसहभूः क्षणस्तस्येति । सोऽस्येन्द्रियज्ञानविषयात्प्राक्तनार्थक्षणादन्यो विषय इति । अनेन च स्वविषयानन्तरविषयक्षणसहकारिग्रहणस्योपयोगो दर्शितः ।
	\pend
      

	  \pstart अथेत्थं गृहीतग्राहित्वादप्रामाण्यप्रसङ्गो निराक्रियताम् । अन्धबधिराद्यभावदोषप्रसङ्गस्तु कथङ्कारं निराक्रियेतेत्याशङ्क्याह--\textbf{इन्द्रियज्ञानेने}ति । अमुनाऽ\textbf{गृहीतस्य विषयान्तरस्य ग्रहणा}न्मानसेनेति प्रकरणात् । तस्माद् हेतोरिन्द्रियज्ञानेनागृहीतं मानसेन गृह्यत इति दर्शयता चानेन यस्यैवेन्द्रियमस्ति तस्यैव तेनेन्द्रियज्ञानेन सहभूविषयक्षणसहकारिणा जनितत्वान्मानसं प्रत्यक्षम् । तेनागृहीतं\add{त}त्सहभुवं विषयक्षणं गृह्णाति । यस्य पुनः पुंसोऽर्थविषयमिन्द्रियज्ञानमेव नास्ति तस्य कथं मानसमुत्पद्येत, गृह्णीयाद् वा तं विषयक्षणमिति दर्शितम् । अतोऽन्धबधिराद्यभावदोषप्रसङ्गो निरस्त इति । अनेनेन्द्रियज्ञानग्रहणस्य स्वविषयानन्तरविषयक्षणसहकारिसचिवस्योपयोगो दर्शितः । केवलेऽपि ज्ञानेनेति ग्रहणे\add{स्व}\leavevmode\marginnote{\textenglish{27a/ms}}विषयानन्तरविषयक्षणसहकारिणेतिविशेषणवलाद् यदि नामेन्द्रियज्ञानेनेति लभ्यते तथापि प्रतिपत्तिगौरवपरिहारार्थमिन्द्रियग्रहणं कृतमित्यवसेयम् । \textbf{अन्धबधिरादी}त्यत्रादिग्रहणात्काप्ति\footnote{कुष्ठि}रोगिणो ग्रहणम् तस्यापि मानसेन स्पर्शज्ञानात् ।
	\pend
      

	  \pstart इह \textbf{शान्तभद्रेण सौत्रान्तिकानां} मतं दर्शयता “पूर्वं चक्षूरूपे चक्षर्विज्ञानं ततस्तेनेन्द्रियविज्ञानेन सहजक्षणसहकारिणा तृतीयस्मिन् क्षणे मानसप्रत्यक्षं जन्यते” इति व्याख्यातम् ।  \leavevmode\marginnote{\textenglish{62/dm}} “
	  
	एतच्च मनोविज्ञानमुपरतव्यापारे चक्षुषि प्रत्यक्षमिव्यते । व्यापारवति तु चक्षुषि यद्रूपज्ञानं तत्सर्वं चक्षुराश्रितमेव । इतरथा चक्षुराश्रितत्वानुपपत्तिः कस्यचिदपि\footnote{०पि ज्ञानस्य \cite{dp-msB} \cite{dp-msD}} विज्ञानस्य ।” तदवमन्यमान आह--\textbf{एतच्चेति} । चोऽवधारणे । \textbf{उपरतव्यापार} इत्यस्यानन्तरं द्रष्टव्यः । चक्षुषो व्याप्रियमाणावस्था प्रणिधानमेव । एतन्मनोविज्ञानं प्रमाणं प्रमाणत्वेनोपगतमुपरतव्यापार एव चक्षुषि भवतीतीष्यत इति समुदायार्थो द्रष्टव्यः । यदि प्रणिहिते नयने मनोज्ञानं न जन्यते तर्हि किं नाम जायत इत्याह--\textbf{व्यापारे}ति । तुर्निर्व्यापारावस्थाया व्यापारवतीमवस्थां भिनत्ति । \textbf{रूपज्ञानमि}ति वास्तवानुवादो न पुनरन्यज्ञानसम्भवेन तद्व्यवच्छेदार्थम् । यद्वा \textbf{रूप}शब्दः स्वभाववचनस्तेन यत्स्वभावज्ञानमित्यर्थः । \textbf{सर्व}ग्रहणेन व्याप्तिमाह । \textbf{चक्षुराश्रि}तं चक्षुर्जन्यं चक्षर्विज्ञानमिति यावत् । उपपत्तिमाह--\textbf{इतरथे}ति ।
	\pend
      

	  \pstart ननु व्यापारवत्यपि चक्षुषि मानसोत्पत्तौ न चक्षुराश्रितत्वानुपपत्तिरादिमस्यैव ज्ञानस्य चक्षुराश्रितत्वोपपत्तेः । सत्यम् । केवलम\add{यम}भिप्रायः--यदपि तदाद्यं चक्षुर्विज्ञानं तदपि चक्षुरन्वयव्यतिरेकेऽ\footnote{का}नुविधानादेव तदाश्रितं व्यवस्थापनीयम् । सत्यपि तदन्वयव्यतिरेकानुविधाने यदि कस्यचिन्मानसत्वं तदा चक्षुरन्वयव्यतिरेकौ चक्षुराश्रितत्वव्यवस्थाया अनङ्गमित्याद्येऽपि तद्व्यवस्था न स्यादिति साधीयान् प्रसङ्गः ।
	\pend
      

	  \pstart स्यादेतत्--किमेतन्मानसं चक्षुर्विज्ञानेनैव रूपविषये जन्यते किं वा रसनादिज्ञानेनापि रसादिविषय इति ? किं चातः ? यदि रूप एव तदा \footnote{द्रष्टव्योऽभिधर्मकोषः--“पञ्च बाह्या द्विविज्ञेयाः”--\href{http://http://sarit.indology.info/?cref=ak.1.48}{ १. ४८}.}“द्विविधज्ञानविज्ञेयाः पञ्च बाह्यविषयाः” इत्यागमो विरुद्ध्येत । अथ समस्तरूपादिविषयमिष्यते तदा शब्दविषयत्वेन तस्यासम्भवः । नहि रसादिरिव ताल्वादिजन्मा शब्दः प्रवाहवाही, येन श्रोत्रेन्द्रियज्ञानेन स्वाविषयानन्तरविषयक्षणसहकारिणा तत्र मानसं जन्येतेति । उच्यते--रूपादिपञ्चकविषयमेतत्, न तु रूपमात्रविषयम् । तदाहाचार्यः--इन्द्रियज्ञानेन, न तु चक्षुर्विज्ञानेनेति । यत्पुनरवादीद् भगवान्—“द्वाभ्यां भिक्षवो रूपं गृह्यते, चक्षुर्विज्ञानेन तदाकृष्टेन च मनसेति” तद्रूपस्य ग्रहणप्रसङ्गेनोक्तम्, न तु रूपमेव द्विधा गृह्यत इति विवक्षितम् ।
	\pend
      

	  \pstart ननूक्तं न तस्य शब्दविषयत्वेन सम्भवस्तत्कथमेवमभ्युपगमः ? सत्यमेतत् । केवलं यदाऽभिहन्यमानं घण्टाटि रणति तदा तदुक्तः\footnote{त्थः}शब्दः कियन्तं कालं सन्तानेन प्रवहतीति श्रोत्रज्ञानेन स्वविषयानन्तरविषयक्षणसहकारिणा तद्विषयं मानसं जन्यते । एतावताऽपि द्विविधज्ञानवेद्यत्ववचनं चरितार्थम् । तदुक्तमभिधर्मकोशे--“रूपादिजात्यभिसम्बन्धिवचनाद्” इति । अस्यार्थः--रूपादिजात्यभिसन्धिना तदुक्तं भगवता, न तु \leavevmode\marginnote{\textenglish{27b/ms}} सर्वरूपादिव्यक्त्यभिप्रायेणेति ।
	\pend
      

	  \pstart इह \textbf{पूर्वैः}--“बाह्यार्थालम्बनमेवंविधं मनोविज्ञानमस्तीति कुतोऽवसेयम्” इत्याशङ्क्य “तदभावे तद्बलोत्पन्नानां विकल्पानामभावाद् रूपादौ विषये व्यवहाराभावप्रसङ्गः स्याद्” इत्युक्तम् । “चक्षुरादिविज्ञानेनानुभूतत्वान्न विकल्पाभाव” इति चाशङ्क्याभिहितम्--“देवदत्तेनापि  \leavevmode\marginnote{\textenglish{63/dm}} “
	  
	एतच्च सिद्धान्तप्रसिद्धं मानसं प्रत्यक्षम् । न त्वस्य प्रसाधकमस्ति प्रमाणम् । एवंजातीयकं तद् यदि स्यात् न कश्चिद् दोषः स्यादिति वक्तुं लक्षणमाख्यातमस्येति ॥” दृष्टे यज्ञदत्तस्यापि विकल्पप्रसङ्गः ।” “सन्तानभेदा\add{न्न}भविष्यति” इति च पुनराशङ्क्य “अत्रापि सन्तानभेदादेव विकल्पो न प्राप्नोति, यत इहापीन्द्रियाश्रयभेदादेव सन्तानभेदो युगपत्प्रवृत्तेश्च । दीर्धशष्कुलीभक्षणादौ हि युगपज्ज्ञानप्रवृत्तिर्दृश्यते । न च सन्तानस्यैक्ये युगपत्प्रवृत्तिर्न्याय्या । तस्माद् रूपादिविकल्पाभावो मा भूदित्यवश्यमविकल्पकं मनोविज्ञानमभ्युपेयम् । एतेन निश्च\add{य}स्मरणाभावप्रसङ्गोऽपि ढौकनीयः । निर्विकल्पकं मनोविज्ञानं यदि नास्त्येव तदा योगिज्ञानाभावप्रसङ्गः । अस्त्येव निर्विकल्पकं मनोविज्ञानं किन्त्विन्द्रियज्ञानपृष्ठभावि नास्तीति चेत् । सति सम्भवे तस्याप्यस्तित्वे को विरोधः । न ह्यत्र बाधकं प्रमाणं दृश्यते, येन तन्नास्तीति स्यात् । अस्तित्वे चोक्तं प्रमाणम् । त\add{स्मा}दस्तीन्द्रियज्ञानपृष्ठभावि मनोज्ञानं निर्विकल्पकम्” इत्येवमाद्यभिहितम् । तदेतत्तदीयं कदलीगर्भनिःसारं मन्यमानः प्राह--\textbf{एतच्चेति । चोऽ}वधारणे । \textbf{सिद्धान्तप्रसिद्धमि}त्यतः परो द्रष्टव्यः । एतदेव व्यतिरेकमुखेण द्रढयति \textbf{न त्वस्ये}ति । \textbf{तुर}वधारयति भिनत्ति वा । प्रवाहानारम्भकस्यास्य\add{ज्ञा}नात्मतया स्वसंवेदनरूपत्वेऽप्यसंविदितकल्पत्वादनुभवगम्य\add{मिदं}यथा चक्षुरादिज्ञानमिति दर्शयितुमशक्यत्वा\textbf{न्नास्य प्रसाधकं} निश्चायकं प्रमाणमस्ति ।
	\pend
      

	  \pstart ननु तैर्दर्शितमेव विकल्पाभावप्रसङ्गः स्यादित्यादिना प्रसङ्गमुखेन प्रमाणं विकल्पोदयादिति तत्किमुच्यते \textbf{न त्वस्य प्रसाधकमस्ति प्र\add{मा}णमि}ति । अयमस्याशयः--सत्यमुक्तमतार्किकीयं तु तत् । तथाहि यत् तावदुक्तमिन्द्रियाश्रयभेदाद् युगपत्प्रवृत्तेश्च सन्तानभेदोऽस्ति । न च सन्तानभेदेऽन्यानुभूतेऽन्यविकल्पो युक्त इति । तदवाच्यम् । यदीन्द्रियाश्रयभेदाद् युगपत्प्रवृत्तेश्च सन्तानभेदस्तद्भेदे च न कार्यकारणभावः, तदा स्वापादुत्थितमात्रस्य पुंसश्चक्षुर्विज्ञानं \footnote{पङ्क्तिबाह्यं लिखितः पाठो न पठ्यतेऽस्पष्टत्वात्--सं०}\add{... ... ...}नोत्पद्येत । तत् खलूपजायमानं मनोज्ञानाद् वा प्राचीनादुत्पद्येत, इन्द्रियज्ञानाद् वा । न तावदिन्द्रियज्ञानात् तस्य पूर्वमभावात्, नापि जागरावस्थाभावीन्द्रियज्ञानात् । तस्य चिरनिरुद्धत्वात् । न च चिरातीतं कारणमिष्यते । अथैवं भिन्नसन्तानैरेव विकल्पितैर्विकल्पोदयोऽस्त्येव न तु निर्विकल्पाद् वि\add{क}ल्पोदय इति चेत् । सन्तानभेदेऽपि जन्यजनकभावे निर्विकल्पादपि विकल्पोदयस्य को निषेद्धा । न चेन्द्रियाश्रयभेदाद् युगपत् प्रवृत्तेश्च सन्तानभेदो युज्यते । परस्परपरोक्षतादिप्रसङ्गात् । तस्मात्प्रभूतमल्पं वा सदृशादेव कारणाज्जायते । न तु प्रभूतस्योदयमात्रेण सन्तानभेदोऽभ्युपेतव्यः । यथाऽग्निकणिकायाः \leavevmode\marginnote{\textenglish{28a/ms}}प्रभाप्रतानवती प्रदीपशिखा जायमाना न सन्तानभेदमात्मन्यावहति । यत्पुनर्योगिज्ञानस्य तथात्वे-, ऽपीदानीं मनोज्ञानं निर्विकल्पकं नाभ्युपेयते तद् यादृश्याः सामग्र्यास्तदुद्भवो यदाकारञ्च तत् तत्सामग्र्यभावात्तदाकारस्य च मनोविज्ञानस्य निर्विकल्पस्यासंवेदनादिति किमत्रादरेणेति ।
	\pend
      

	  \pstart ननु च यद्यस्य प्रसाधकं प्रमाणं नास्ति किमर्थं तर्हि प्रत्यक्षप्रकरण उपन्यास इत्याशङ्क्याह—\textbf{एवमि}ति । \textbf{एवंजाती}यकमेवम्प्रकारवत् । एवम्प्रकारलक्षणकथनस्यैव किम्प्रयोजनमिति चेत् । \textbf{सूत्रकारस्य} सिद्धान्तप्रसिद्धमानसाभ्युपगमे प्रसक्तचोद्यनिराकरणम्--यद्यस्ति मानसं प्रत्यक्षमेवं तस्य लक्षणमिति ॥
	\pend
      \leavevmode\marginnote{\textenglish{64/dm}}“

	  \pstart स्वसंवेदनमाख्यातुमाह--
	\pend
       “

	  \pstart सर्वचित्तचैत्तानामात्मसंवेदनम् ॥ १० ॥
	\pend
      ” 

	  \pstart \footnote{“सर्वचित्तेत्यादि” नास्ति \cite{dp-msA} \cite{dp-msC}}सर्वचित्तेत्यादि । चित्तम् अर्थमात्र\footnote{मात्रावग्राहि \cite{dp-msC}}ग्राहि । चैत्ता विशेषावस्थाग्राहिणः सुखादयः । सर्वे च ते चित्तचैत्ताश्च सर्वचित्तचैत्ताः । सुखादय एव स्फुटानुभवत्वात् स्वसंविदिताः, नान्या चित्तावस्थेत्येतदाशङ्कानिवृत्त्यर्थं सर्वग्रहणं कृतम् । नास्ति सा काचित् चित्तावस्था यस्यामात्मनः\footnote{०मात्मसंवे० \cite{dp-msD}} संवेदनं न प्रत्यक्षं स्यात् ।
	\pend
       

	  \pstart येन हि रूपेणात्मा वेद्यते तद्रूपमात्मसंवेदनं प्रत्यक्षम् ।
	\pend
      ”

	  \pstart \textbf{वैभाषिकप्र}क्रियया यदाचार्येण चित्तचैत्तौ भेदेनोक्तौ तयोरर्थमाह--\textbf{चित्तमर्थमात्रग्राहि} वस्तुमात्रग्राहि । \textbf{“तत्रार्थदृष्टिर्विज्ञानम्” इति} वचनात् । \textbf{चैत्ता} विशेषावस्थाग्राहिणो विशेषावस्थास्वीकर्त्तारो विशेषावस्थाकारा इति यावत् । तद्विशेषे तु चैतसा इति वचनात् । क एवंरूपा इत्याह--\textbf{सुखादय} इति । \textbf{सर्वे चे}ति विगृह्णन् पूर्वं चित्तानि च चैत्ताश्चेति समस्य पश्चात्सर्वशब्देन समास इति दर्शयति । \textbf{सुखे}त्यादिना \textbf{सर्व}ग्रहणस्य फलमाह--\textbf{स्फुटो} व्यक्तो\textbf{ऽनुभवः} प्रकाशो यस्य तस्य भावस्तस्मात् । \textbf{सर्व}ग्रहणे सति कीदृशोऽर्थो भवति, येनाशङ्का निवर्त्त्यत इत्याह--\textbf{नास्तीति । काचिदिति} भ्रान्ता वाऽभ्रान्ता वेत्यर्थः ।
	\pend
      

	  \pstart ननु ज्ञानस्य संवेद्यात्मनः किमन्यद् रूपान्तरं यत्स्वसंवेदनं प्रत्यक्षं स्यादित्याशङ्क्याह—\textbf{येन ही}ति । हिर्यस्मादर्थे ।
	\pend
      

	  \pstart अथ यदनुभूयते तदन्येनैव यथा घटादि । तत्कथमात्मनैवात्मनः संवेदनमिति चेत् । न । योग्यतायास्तथाव्यवहारात् । अस्ति ज्ञानस्य सा योग्यता जडव्यावृत्तता ज्योतीरूपता यया स्वप्रकाशे प्रकाशान्तरं नापेक्षते । यथा प्रदीपः प्रकाशस्वाभाव्यादात्मानं स्वयमेव प्रकाशयति, न तु प्रदीपान्तरमपेक्षत इति ।
	\pend
      

	  \pstart ननु किं घटादिदृष्टान्तबलात्प्रकाशान्तरप्रकाश्यता ज्ञानस्यास्ताम्, आहोस्वित् प्रदीपदृष्टान्तसामर्थ्यात् प्रकाशान्तरनिरपेक्षतया स्वप्रकाशता ? अत्रोच्यते । ज्ञानस्य स्वप्रकाशरूपत्वाभावे घटादेः स्वरूपप्रकाशनानुपपत्तेः स्वप्रकाशतैवेष्टव्या । अत्र \textbf{च} प्रयोगः--यदव्यक्तव्यक्तिकं न तद् व्यक्तम् यथा किञ्चित्कदाचित् कथञ्चिदव्यक्तव्यक्तिकम् । अव्यक्तव्यक्तिकश्चायं घटादिः ज्ञानपरोक्षत्व इति व्यापकानुपलब्धिप्रसङ्गः । ज्ञानस्य ज्ञानान्तरेण व्यक्तौ हेतुरयमसिद्ध इति चेन्न, नीलज्ञानोदयकालेऽसिद्धत्वाद् हेतोर्नीलस्य परोक्षत्वप्रसञ्जनात् । न च भवतामपि मतै सर्वं विज्ञानमेकार्थसमवायिना ज्ञानेन ज्ञायते । बुभुत्साऽभावे तदभावात्, यथोपेक्षणीयविषया संवित् । तत उपेक्षणीयमेव तावद\add{व्यक्त}व्यक्तिकत्वादव्यक्तं प्रसज्येत । न चेयम  \leavevmode\marginnote{\textenglish{65/dm}} “
	  
	\footnote{मीमांसकान् प्रति विज्ञानं स्वसंवेदनप्रत्यक्षमुक्तम् । यश्च सांख्योऽपि “बाह्यरूपाः सुखादयः” इति मन्यते तं प्रत्याह--\cite{dp-msD-n} । इह रूपादौ \cite{dp-msD}}इह च रूपादौ वस्तुनि दृश्यमाने\footnote{०मानेऽन्तरः \cite{dp-msA} \cite{dp-msB} \cite{dp-msC} \cite{dp-msD} \cite{dp-edP} \cite{dp-edH} \cite{dp-edE}}आन्तरः सुखाद्याकारस्तुल्यकालं \footnote{०कालं वेद्यते \cite{dp-msC}}संवेद्यते । न च—गृह्यमाणाकारो नीलादिः \footnote{सातादिरू० \cite{dp-edP} \cite{dp-edH} \cite{dp-edE} \cite{dp-edN}}सातरूपो वेद्यते इति \footnote{इति वक्तुं शक्यम्--\cite{dp-msA} \cite{dp-edP} \cite{dp-edH} \cite{dp-edE}}शक्यं वक्तुम् । यतो नीलादिः \footnote{सातादिरूपेण \cite{dp-msA} \cite{dp-msC} \cite{dp-msD} \cite{dp-edP} \cite{dp-edH} \cite{dp-edE} \cite{dp-edN} सातानुरूपेण--\cite{dp-msB}}सातरूपेणानुभूयत इति न निश्चीयते । 
	  
	यदि हि “सातरूपोऽयं\footnote{सातादिरूपोऽयं \cite{dp-msA} \cite{dp-msC} \cite{dp-edP} \cite{dp-edE} \cite{dp-edN}} नीलादिरनुभूयते” इति निश्चीयेत, स्यात्\footnote{स्यात् तस्य \cite{dp-msA} \cite{dp-msC} \cite{dp-edP} \cite{dp-edE}} तदा तस्य सातादिरूपत्वम् । यस्मिन् रूपे प्रत्यक्षस्य साक्षात्कारित्वव्यापारो विकल्पेनानुगम्यते तत् प्रत्यक्षम् ।” नुपलब्धिः सन्दिग्धविपक्षव्यावृत्त्याऽनैकान्तिकी कीर्त्तनीया । \leavevmode\marginnote{\textenglish{28b/ms}} तथा हि यद्यव्यक्तव्यक्तिकमपि व्यक्तव्यवहारविषयस्तदा पुरुषान्तर्वर्त्तिज्ञानव्यवितकमपि वस्तु स्वज्ञानोदयकालवत् तथैव व्यक्तं व्यवह्रियेत । अस्वसंवेदनात्मतया स्वपरसन्तानवर्त्तिनोर्ज्ञानयोर्विशेषाभावात् । तत्सन्ताने ज्ञानस्याभावात् कथं वस्तुनस्तथा व्यवहार इति चेत् । भावेऽपि तदप्रकाशे कथं तथा व्यवहारः ? न हि तेनास्य किञ्चित् क्रियते । \footnote{पङ्क्तिबाह्यं लिखितं सम्यक् न पठ्यते--सं०}\add{... ...}तदा तदपि पुरुषान्तरस्य किन्न तथा व्यवहारगोचरः ? तदयं व्यक्तव्यवहारो व्यक्तव्यक्तिकत्वेन व्याप्तः । सिद्धे च व्याप्यव्यापकभावे व्यापकानुपलब्धिर्नानैकान्तिकीति ।
	\pend
      

	  \pstart ननु किं तद्रूपान्तरम्, येनोच्यते--येनात्मा संवेद्यते, तदात्मसंवेदनं प्रत्यक्षमिति ? केवलमर्थशून्यमेतदुच्यते इत्याशङ्क्याह--इहेत्यादि । अन्तरे भव \textbf{आन्तरो}ऽध्यात्मपरिस्पन्दी । कोऽसावीदृश इत्याह--\textbf{सुखे}ति ।
	\pend
      

	  \pstart ननु गृह्यमाण एव रूपादिः सुखाद्याकारोऽनुभूयते । न तु ततोऽन्यत्सुखादिरूपं येन तस्य वेदनरूपता व्यवस्थाप्येतेत्याह--\textbf{न चे}ति । \textbf{चो}ऽवधारणे, यस्मादर्थे वा । गृह्यमाण आकारो यस्य ग्राह्यस्वभाव इत्यर्थः । \textbf{सातरूपः} सुखस्वभावः । सातग्रहणस्योपलक्षणत्वाद् दुःखरूप इत्यपि द्रष्टव्यम् । \textbf{इति}ना वचनस्याकारं दर्शयति । कुत एवं वक्तुं न शक्यत इत्याह--\textbf{यत} इति । \textbf{इति}र्निश्चयस्य स्वरूपमाह । किं ताद्रूप्येण निश्चयेऽपि ताद्रूप्यानुभवः सिद्ध्यति, येन तदा\footnote{द}भावान्नैवं शक्यते वक्तुमित्युच्यते इत्याह--\textbf{यदी}ति । हि\add{र्य}स्माद् इति । \textbf{इ}तिकरणेन निश्चयस्वरूपमुक्तम् । अनेन यदेवानुरूपविकल्पेन यथात्वेन निश्चीयते तदेव \textbf{तथात्व}व्यवहारगोचरो यथा नीलादिरित्याकूतम् । ननु च तत्प्रतिभासादनुभवः प्रमाणम् ।  \leavevmode\marginnote{\textenglish{66/dm}} “
	  
	न च नीलस्य \footnote{सातदिरूप० \cite{dp-msC}}सातरूपत्वमनुगम्यते । तस्मादसातान्नीलाद्यर्थादन्यदेव\footnote{नीलादर्थाद० \cite{dp-msC} \cite{dp-msD}} सातमनुभूयते\footnote{सातरूपत्वमनु० \cite{dp-msD}} नीलानुभवकाले । तच्च ज्ञानमेव । ततोऽस्ति\footnote{०स्ति विज्ञा० \cite{dp-msD}} ज्ञानानुभवः । 
	  
	तच्च \footnote{ज्ञानरूपं वेदनं \cite{dp-msA} \cite{dp-msB} \cite{dp-edP} \cite{dp-edH} \cite{dp-edE} \cite{dp-edN} ज्ञानस्वरूपवेदनं \cite{dp-msC} \cite{dp-msD}}ज्ञानरूपवेदनमात्मनः साक्षात्कारि निर्विकल्पकमभ्रान्तं च । \footnote{तस्मात्--\cite{dp-msA} \cite{dp-msB} \cite{dp-msC} \cite{dp-edP} \cite{dp-edE} \cite{dp-edH} \cite{dp-edN}}ततः प्रत्यक्षम् ॥” निश्चयो भवतु, मा वा भूत् । तत्किमेवमुच्यत इत्याशङ्क्य पूर्वोक्तमेव स्मारयति--\textbf{यस्मिन्नि}ति । रूपे स्वभावे । शेषं प्रागेव कृतव्याख्यानम् । गृहीतं नीलस्य सातरूपत्वं विकल्पेनानुगम्यत एवेत्याह--\textbf{न चे}ति । चः पूर्ववत् । \textbf{अनुगम्य}ते विकल्पेनेति वर्त्तते । \textbf{तस्मादित्या}दिनोक्तमुपसंहरति । यस्माद् गृह्यमाणाकारो नीलादिर्न तथा निश्चीयते, अस्ति च हर्षाद्याकारसंवेदनम्, तस्माद् ।
	\pend
      

	  \pstart भवतु साताकारोऽनुभूतः । केवलमसावज्ञानात्मा भविष्यति । तथा च कथमज्ञानेन ज्ञानात्मसंवेदनम् ? कथं चात्मवेदनं प्रत्यक्षमित्याह--\textbf{तच्चे}ति । \textbf{चो}ऽज्ञात्मन एवं भिनत्ति । \textbf{ज्ञानमेवेत्यवधा}रयतः प्रकाशात्मन एवं ज्ञानत्वमन्यथा प्रकाशायोगादित्यभिप्रायः । यत एवं \textbf{ततोऽस्ति ज्ञानानुभवः} । इत्यन्तर्मुखस्य सुखाद्याकारस्य ग्राहकाकाराख्यस्येत्यर्थः । न च ग्राह्याकारादन्य दनुभूयमानं सातं ग्राहकाकारादप्यन्यदुपपद्यते । ग्राह्यं वा प्रकाशेत, ग्राहकं वा । न च तद् ग्राह्यमतो ग्राहकमेव । अथ ज्ञानमस्तु तथानुभूतम् । तत्पुनरात्मसंवेदनं प्रत्यक्षं कुत इत्याह--\textbf{तच्चे}ति । \textbf{चो} यस्माद् । वेद्यतेऽनेनेति \textbf{वेदनम्} । ज्ञानरूपस्य \textbf{वेदनमि}ति विग्रहः\leavevmode\marginnote{\textenglish{29a/ms}} । यद्वा ज्ञानरूपं च तद् वेदनञ्चेति तथा । \textbf{साक्षात्कारि} अपरोक्षताकारि, स्कुटावभासमिति यावत्, हेतुभावेनास्य विशेषणत्वात् । अत एव च निर्विकल्पकम् । विकल्पानुबद्धस्य स्पष्टार्थप्रतिभासित्वायोगात् । भवतु निर्विकल्पकं द्विचन्द्रादिज्ञानवद् भविष्यतीत्याह--\textbf{अभ्रान्तं चे}ति । \textbf{च}स्तुल्योपायत्वं समुच्चिनोति । नहि तत् स्वात्मनि अतस्मिंस्तदिति प्रवृत्तम्, येन तत्र भ्रान्तिर्भविष्यतीति भावः । भवतां निर्विकल्पकत्वाभ्रान्तत्वे ततः किं सिद्धमित्याह--तत इति । यत एतद् रूपद्वययोगि ततस्तस्मात् । एतावन्मात्रलक्षणत्वात्प्रत्यक्षस्येति भावः ।
	\pend
      

	  \pstart तत्र चेयं व्यवस्था । आत्मात्मयोग्यता प्रमाणम्, आत्मसंवित् फलमिति द्रष्टव्यम् । स्यान्मतम्--इत्यं तस्य वेदनस्य प्रत्यक्षत्वे विकल्पात्मवेदनस्यापि तत्त्वं स्यात् । न च विकल्पात्मा प्रत्यक्षं युज्यते । युज्यते, स्वात्मनि अविकल्पनादभ्रान्तत्वाच्च । विकल्पो हि बाह्यं विकल्पयति न त्वात्मानम् । भ्राम्यति च ब्राह्ये, नात्मनि । ततः किं न प्रत्यक्षम् ? प्रयोगः—यदभ्रान्तत्वे सत्यविकल्पं तत् प्रत्यक्षम् । यथेन्द्रियज्ञानस्य बाह्यसंवेदनम् सारूप्याख्यम् । अभ्रान्तत्वे सत्यविकल्पकं चात्मनि विकल्परूपवेदनमिति स्वभावः । यथा च ज्ञानात्मनि समयासम्भवो यथा वा समितः शब्दसंसृष्टो न गृह्यते तथाऽन्यत्र प्रपञ्चितमिति नेहोच्यत इति ॥
	\pend
      \leavevmode\marginnote{\textenglish{67/dm}}“

	  \pstart योगिप्रत्यक्षं व्याख्यातुमाह\footnote{०ख्यातुकाम आह \cite{dp-msC} \cite{dp-msD}}
	\pend
      ”“

	  \pstart भूतार्थभावनाप्रकर्षपर्यन्तजं योगिज्ञानं चेति ॥ ११ ॥
	\pend
      ”

	  \pstart भूतः \footnote{यथावस्थितः--\cite{dp-msD-n}}सद्भूतोऽर्थः । प्रमाणेन दृष्टश्च सद्भूतः ।
	\pend
      

	  \pstart यथा \footnote{दुःखसमुदयमार्गनिरोधाः । तत्र दुःखं संसारिणः स्कन्धाः । समुदयो रागादिगणः । मार्गः क्षणिकत्वभावना । निरोधो मोक्षः ।--\cite{dp-msD-n}}चत्वार्यार्यसत्यानि ।
	\pend
      “

	  \pstart भूतस्य भावना पुनः पुनश्चेतसि विनिवेशनम् । भावनायाः प्रकर्षो भाव्यमानार्थाभासस्य\footnote{०र्थावभासस्य \cite{dp-msB} \cite{dp-edN}} ज्ञानस्य स्फुटाभत्वारम्भः । प्रकर्षस्य पर्यन्तो यदा स्फुटाभत्वमीषदसम्पूर्णं भवति । यावद्धि स्फुटाभत्वमपरिपूर्णं तावत् तस्य\footnote{स्फुटाभत्वस्य--\cite{dp-msD-n}} प्रकर्ष\footnote{प्रकर्षगतिः \cite{dp-msA} \cite{dp-edP} \cite{dp-edH} \cite{dp-edE} \cite{dp-edN}}गमनम् ।
	\pend
      ”

	  \pstart \textbf{भूत}शब्दस्य विवक्षितमर्थमाह--\textbf{सद्भूत} इति । \textbf{अर्थ} इति ब्रुवाणो भूतश्चासावर्थश्चेति कर्मधारयं दर्शयति । ननु सुखादिमयत्वमप्यर्थस्य \textbf{सांख्य}परिकल्पितं सद्भूतमित्याह—\textbf{प्रमाणेने}ति । \textbf{दृष्टो} निश्चितः । \textbf{च}कारः स्फुटमेतदित्यर्थं द्योतयति ।
	\pend
      

	  \pstart कः पुनरीदृशोऽर्थो विवक्षित इत्याह--\textbf{यथे}ति । अनेन \textbf{भूतार्थ}शब्देनात्र सत्यचतुष्टयं विवक्षितमिति दर्शितम् । यथा चतुरार्यसत्यभावनाबलजं ज्ञानं योगिनः प्रत्यक्षम्, तथाऽन्यसद्भूतार्थविषयमपि द्रष्टव्यमिति यथाशब्दार्थोऽप्यस्ति । य\textbf{द्विनिश्चयः} “तुरार्यसत्यदर्शनवदिति ।” आरात्पापकेभ्यो धर्मेभ्यो याता इत्या\textbf{र्याः} । अत एव तानि सत्यतया मन्यन्त इति तेषां सत्यानि । चतुष्ट्वाच्च तेषां चत्वारीत्युक्तम् ।
	\pend
      

	  \pstart फलभूताः पञ्च सक्लेशस्कन्धा दुःखाख्यं सत्यमेकम् । त एव हेतुभूतास्तृष्णासहायाः समुदयाख्यं सत्यं द्वितीयम् । चित्तस्य निष्क्लेशावस्था निरोधाख्यं सत्यं तृतीयम् । तदवस्थाप्राप्तिहेतुनैरात्म्याद्याकारश्चित्तविशेषो मार्गाख्यं सत्यं चतुर्थमिति ।
	\pend
      

	  \pstart भावनाशब्देन विग्रहं तस्य चार्थमाचष्टे \textbf{भूतस्ये}ति । भूतार्थस्येति द्रष्टव्यम् । लक्ष्यते च भूतशब्दसान्निध्याल्लेखकेन प्रथमपुस्तके भूशब्दः प्रक्षिप्तः । \textbf{तस्ये}ति तु वचनं संक्षेपेण विग्रहं दर्शयतो \textbf{धर्मोत्तरस्य} पाठोऽन्यथा यथाभूतं विग्रहं दर्शयितुकामेनार्थपदोपादाने किमक्षरगौरवं दृष्टम्, येन केवलभूतशब्दोपादाने प्रतिपत्तिगौरवं लिखनाकौशलञ्चाविष्कृतमिति । भावनार्थमाह—\textbf{पुनरि}ति । \textbf{पुनरि}त्यप्रथमतः । द्विर्वचनेनाप्रथमप्रचारस्य प्राचुर्यं दर्शयति । तथानिवेशनञ्च विजा\leavevmode\marginnote{\textenglish{29b/ms}}तीयाव्यवधानेन द्रष्टव्यम् । सत्यचतुष्टयविषयो विजातीयाव्यवहितः सदृशचित्तप्रवाहो भावनेति यावत् ।
	\pend
      \leavevmode\marginnote{\textenglish{68/dm}}“

	  \pstart सम्पूर्ण तु यदा, तदा नास्ति प्रकर्षगतिः । ततः सम्पूर्णावस्थायाः प्राक्तन्यवस्था स्फुटा\footnote{त्वं प्रक० \cite{dp-edE}}भत्वप्रकर्षपर्यन्त उच्यते । तस्मात् पर्यन्ताद् यज्जातं\footnote{जातं ज्ञानं भाव्य० \cite{dp-msC} \cite{dp-msD}} भाव्यमानस्यार्थस्य\footnote{भाव्यमानस्य सं० \cite{dp-msA} \cite{dp-msB} \cite{dp-edP} \cite{dp-edH} \cite{dp-edE}} सन्निहितस्येव स्फुटतराकारग्राहि ज्ञानं योगिनः प्रत्यक्षम् ।
	\pend
       

	  \pstart तदिह स्फुटाभत्वारम्भावस्था भावनाप्रकर्षः । अभ्रकव्यवहितमिव यदा भाव्यमानं
	\pend
      ”

	  \pstart यादृशो योगिनां भावनाक्रमो \textbf{विनिश्चये} “श्रुतमये”त्यादिनाभिहितो यथा भावनाप्रकर्षविशदाभत्वयोः कार्यकारणभावस्तत्रैव “कामशोके”त्यादिना दर्शितस्तथेहापि द्रष्टव्यः । यादृशश्चाकारस्तेषां सत्यानामनित्यत्वादिके भावनीयो यावत्कालावधिका च भावनाऽनेकजन्मपरम्परानुयाता, यच्च निबन्धनं भावनायाः करुणा बोधिसत्त्वानाम्, तदन्येषां संसारोद्वेगस्तदपि सर्व यथा \textbf{प्रमाणवार्त्तिके} निर्णीतं तथेहाप्यनुगन्तव्यम् । इह तु योगिज्ञानस्य स्वरूपमात्रमुपदर्शयितुमुपक्रान्तमिति ।
	\pend
      

	  \pstart \textbf{प्रकर्ष}शब्देन सह विग्रहं तस्य चार्थं विवक्षितमाह--\textbf{भावनाया} इति । \textbf{स्फुटाभत्वस्या}रम्भ उपक्रमः । स च यत्स्फुटत्वतदधिकस्फुटत्वादिना रूपेण तज्ज्ञानस्योदय एव । पर्यन्तशब्देन विग्रहं तस्य चार्थमाचष्टे \textbf{प्रकर्षस्येति । पर्यन्तो}ऽवसानम् । कदा च तस्यावसानमित्याह--\textbf{यदे}ति । यस्मिन् काले स्फुटाभत्वं भावनार्थं विषयस्य ज्ञानस्येति प्रकरणात् । इदं लेशतोऽसम्पूर्णं भवति यद\add{न}न्तरं योगिप्रत्यक्षेण भवितव्यं तस्मिन् काले प्रकर्षस्य पर्यन्तोऽवसानं ज्ञातव्यः । तत्कालोपलक्षितं तथाभूतं ज्ञानं पर्यन्त इत्यर्थः ।
	\pend
      

	  \pstart ननु प्रकर्षस्य पर्यन्तः स युज्यते यस्मिन् सति प्रकर्षो निवर्त्तते । तच्च सम्पूर्वं\footnote{र्ण}मेव स्कुटाभत्वम्, तत्कथमुच्यते कथञ्च \textbf{पर्यन्तजं योगिज्ञानम्,} न तु तदेव पर्यन्त इत्याशङ्क्याह--\textbf{यावदि}ति । हिर्यस्मात् ।
	\pend
      

	  \pstart सम्पूर्णे तु प्रकर्षगमनं नास्तीति दर्शयति--\textbf{सम्पूर्ण}मिति । \textbf{तु}रिमामवस्थां भेदवतीमाह । \textbf{प्रकर्ष}स्य गतिर्गमनम् । एवं ब्रुवतोऽयमाशयः । प्रकर्षः प्रकृष्यमाणता सातिशयं रूपमुच्यते । पर्यन्तश्च गत्यर्थादामर्द्धातोस्तत्प्रत्ययेन \footnote{?} परिसमन्तादन्त इति प्रादिसमासेन निःशेषगमनमेवोच्यते । ततः स पर्यन्त उच्यते यदनन्तरं प्रकर्ष\footnote{कृष्य}माणेन न भवितव्यम् । न तु यदर्थं येषां प्रकर्षवतामुदय इति । एतदेवोपसंहरति \textbf{तत} इति । \textbf{प्राक्तनी} व्यवधानशून्या यदनन्तरं स्फुटतरज्ञानोदयः । \textbf{सम्पूर्णावस्थायाः} स्फुटत्वसम्पूर्णावस्थायाः स्फुटतराकारग्रहणावस्थाया इति यावत् । \textbf{सन्निहित}स्येति । यथाऽन्यस्याभावितस्य निकटस्थस्य घटघ\footnote{प}टाद\footnote{दे}रन्यज्ञानं स्फुटतराकारग्राहि प्रत्यक्षं तद्वद् भाव्यमानार्थस्फुटतराकारग्राहि \textbf{यज्ज्ञानं तद्योगिनः प्रत्यक्षम्} ।
	\pend
      

	  \pstart उक्तमेव भावनाप्रकर्षार्थं तत्पर्यन्तार्थं तज्ज्ञानं चोपसंहारव्याजेन सुखप्रतिपत्त्यर्थं पुनर्दर्शयति \textbf{तदि}ति । \textbf{तत्त}स्मात् । \textbf{इहे}ति योगिप्रत्यक्षलक्षणप्रतीतिकाले । \textbf{अवस्था} भावनाज्ञान  \leavevmode\marginnote{\textenglish{69/dm}} “
	  
	वस्तु पश्यति सा प्रकर्षपर्यन्तावस्था । करतलामलकवद् भाव्यमानस्यार्थस्य यद् दर्शनं तद् योगिनः प्रत्यक्षम् । तद्धि स्फुटाभम् । 
	  
	स्फुटाभत्वादेव च निर्विकल्पकम् । विकल्पविज्ञानं हि सङ्केतकालदृष्टत्वेन वस्तु गृह्णच्छब्दसंसर्गयोग्यं गृह्णीयात् । सङ्केतकालदृष्टत्वं च सङ्केतकालोत्पन्नज्ञानविषयत्वम् । यथा च पूर्वोत्पन्नं विनष्टं ज्ञानं सम्प्रत्यसत्, तद्वत् पूर्वविनष्टज्ञानविषयत्वमपि सम्प्रति नास्ति वस्तुनः । तदसद्रूपं वस्तुनो गृह्णद् असन्निहि\footnote{०हितग्राहि० \cite{dp-msB} \cite{dp-msC}} तार्थग्राहित्वाद् अस्फुटाभम्\footnote{०ग्राहित्वादस्फुटाभं सविकल्पकम् \cite{dp-msC} अस्फुटाभम् । अस्फुटाभत्वात् सविकल्पकं--\cite{dp-msA} \cite{dp-edP} अस्फुटाभम् । अस्फुटाभत्वादेव च सवि० \cite{dp-msD} \cite{dp-msB} \cite{dp-edH} \cite{dp-edE} \cite{dp-edN}} विकल्पकम् । ततः स्फुटाभत्वान्निर्विकल्पकम् ।” स्येति प्रकरणात् । \textbf{अभ्रके}णातिस्वच्छतया \textbf{पाश}\footnote{र्श्व}तो द्विधाकर्त्तुमशक्येनेति प्रस्तावाद् \textbf{व्यवहित}मावृतं तदिव । करस्वरूपं \textbf{करतलं} सन्निहितस्येवेत्यनेन यत्पूर्वमुक्तं\leavevmode\marginnote{\textenglish{30a/ms}} तस्यायमुपसंहारः ।
	\pend
      

	  \pstart अस्तु तथाविधं योगिज्ञानम्, तत्पुनः कथं प्रत्यक्षमित्याशङ्कामपाकर्त्तु प्रत्यक्षलक्षणेन योगमस्य दर्शयन्नाह \textbf{तद्धी}ति । \textbf{हि}र्यस्मात् ।
	\pend
      

	  \pstart स्फुटाभं भवतु, निर्विकल्पकं तु कथं येन प्रत्यक्षं स्यादित्याह--\textbf{स्फुटाभत्वादेवेति । चो निर्विकल्पकमि}त्यतः परः स्फुटाभत्वापेक्षैकविषयत्वं निर्विकल्पकत्वस्य समुच्चिनोति । अयमस्याशयः--शब्दाकारसंसर्गो हि स्फुटाभत्वविरोधीति यद्यसावभविष्यत् तदा स्फुटाभमेव नाभविष्यदिति । अभिलापसंसर्गयोग्यप्रतिभासमेव स्फुटाभं भविष्यति कोऽनयोर्विरोध इत्याशङ्क्य \textbf{विकल्पेत्यादिना} विरोधमेव दर्शयितुमुपक्रमते । \textbf{हि}र्यस्मात् । \textbf{वस्त्वि}ति दृश्यमानं यथा चैतत्तथा पूर्वमेव विवेचितं \textbf{गृह्णीयाद्} ग्रहीतुमर्हति । सङ्केतकालदृष्टत्वेन च वस्तुग्राहित्वेऽसन्निहितविषयं स्यादिति दर्शयितुमाह--\textbf{सङ्केतेत्या}दि । \textbf{चो} हेतौ । भवतु सङ्केतकालदृष्टत्वस्य तथात्वम्, किमत इत्याह--\textbf{यथा चे}ति । \textbf{चो} वक्तव्यमेतदित्यस्यार्थे ।
	\pend
      

	  \pstart भवत्वेवं तथापि वाचकाकारसंसर्गिणोऽपि कथं न स्फुटाभत्वमित्याह--\textbf{तदि}ति । यत एवं \textbf{तत्त}स्मात् । विकल्पयतीति \textbf{विकल्पकं} विज्ञानम् । अस्फुटाभं तु कुतस्तदित्याह—\textbf{असन्निहितार्थग्राहित्वादि}ति । सन्निहिततया हि भासमानो विशदो भवेत् । पूर्ववृत्तदर्शनविषयतया च भूतपूर्वो भावो गृह्यमाणो न सन्निहितरूपो भासते । तेनाविशदाभो विकल्पः । तेन शब्दाकारसंसर्गो विकल्पस्य विशदाभत्वविरोधीत्यस्याभिप्रायः । असन्निहितविषयत्वमेव तस्य कुतोऽवसीयत इत्याह--\textbf{असद्रूपमिति} । यत्पूर्वदृष्टत्वं सम्प्रत्यसद्, असद्रूपं वस्तुनो गृह्यमाणस्य \textbf{गृह्णदिति} हेतौ शतुर्विधानादसद्रूपग्रहणादित्यर्थः ।
	\pend
      

	  \pstart \textbf{स्फुटाभत्वादेवे}ति यदवोचत्, तदेवोपसंहरति \textbf{तत} इति । यतो विकल्पस्य स्पष्टाभत्वं न युज्यते, अनन्तरोक्तेन क्रमेण \textbf{तत}स्तस्मा\textbf{त्स्फुटाभत्वाद्} विशदाभत्वा\textbf{न्निर्विकल्पकं} योगिज्ञानमिति प्रकरणात् । प्रयोगः--यत् सङ्केतकालदृष्टतया वस्तुसंस्पर्शि ज्ञानं न तत् स्फुटाभम् । यथा चिर \leavevmode\marginnote{\textenglish{70/dm}} “
	  
	प्रमाणशुद्धार्थग्राहित्वाच्च संवादकम् । अतः प्रत्यक्षम् । इतरप्रत्यक्षवत् । 
	  
	योगः \footnote{शुभचित्तैकाग्र्यम्--\cite{dp-msD-n}}समाधिः । स यस्यास्ति\footnote{यस्यास्तीति स \cite{dp-msC}} स योगी । तस्य ज्ञानं प्रत्यक्षम् । इतिशब्दः परिसमाप्ति\footnote{परिसमाप्त्यर्थः \cite{dp-msA} \cite{dp-edP} \cite{dp-edH} \cite{dp-edE} \cite{dp-edN}}वचनः । इयदेव प्रत्यक्षमिति ॥ 
	  
	तदेवं प्रत्यक्षस्य कल्पनापोढत्वाभ्रान्तत्वयुक्तस्य प्रकारभेदं प्रतिपाद्य विषयविप्रतिपत्तिं निराकर्त्तुमाह-- “
	  
	\textbf{तस्य विषयः स्वलक्षणम् ॥ १२ ॥}” 
	  
	तस्येत्यादि । तस्य चतु\footnote{तस्य प्रत्यक्षस्य \cite{dp-msC} तस्य चतुर्विधप्रत्य० \cite{dp-msA} \cite{dp-edP} \cite{dp-edH} \cite{dp-edE} \cite{dp-edN}}र्विधस्य प्रत्यक्षस्य विषयो बोद्धव्यः\footnote{पर्यायग्राह्य इत्यर्थः--\cite{dp-msD-n}} स्वलक्षणम् । स्वम्असाधारणं लक्षणं तत्त्वं\footnote{अर्थक्रियाकारित्वम्--\cite{dp-msD-n}} स्वलक्षणम् । वस्तुनो ह्यासधारणं च तत्त्वमस्ति सामान्यं च । \footnote{“तत्र” नास्ति--\cite{dp-msA} \cite{dp-msB} \cite{dp-msD} \cite{dp-edP} \cite{dp-edH} \cite{dp-edE}}तत्र यदसाधारणं तत् प्रत्यक्षस्य\footnote{प्रत्यक्षग्राह्यम् \cite{dp-msA} \cite{dp-msB} \cite{dp-edP} \cite{dp-edH} \cite{dp-edE}} ग्राह्यम् ।” दृष्टनष्टवस्तुविकल्पः । सङ्केतकालदृष्टतया च दृश्यमानवस्तुसंस्पर्शी विकल्पः । स्फुटाभासत्वं नाम सन्निहितरूपभासनम् । असन्निहितरूपभासनेन च वस्तुनः पूर्वदृष्टत्वग्रहणं व्याप्तम् । न हि पूर्वज्ञानविषयत्वं पूर्वस्मिन् ज्ञाने निवृत्ते वस्तुनोऽस्ति । तदुत्तरकालभाविना ज्ञानेन पूर्वज्ञानविषयत्वमसन्निहितमेव वस्तुनः स्पृश्यते । पूर्वज्ञानविषयत्वग्रहणमेव च सङ्केतकालदृष्टशब्दविशिष्टत्वग्रहणम् । तदिदं विरुद्धं\footnote{द्ध}व्याप्तोपलब्धेर्भवतु निर्विकल्पकम् ।
	\pend
      

	  \pstart असत्यभ्रान्तत्वे तु कथं प्रत्यक्षमित्याह--\textbf{प्रमाणेति} । प्रमाणाधिगतोऽर्थः \textbf{प्रमाणशुद्ध} उच्यते । सत्यचतुष्टयं चैवमात्मकम् । तदेव शुद्धत्वेन विवक्षितम् । \textbf{चः} \leavevmode\marginnote{\textenglish{30b/ms}} \textbf{संवादक}मित्यतः परो निर्विकल्पत्वेन सह संवादकत्वेन संवादकत्वं समुच्चिनोति, तत्रस्थ ए\add{व}वा पूर्वहेत्वपेक्षया हेत्वन्तरसमुच्चयार्थः । \textbf{अतो}ऽस्मान्निर्विकल्पकत्वादभ्रान्तत्वाच्च ।
	\pend
      

	  \pstart योगिशब्दस्य व्युत्पत्तिमाह--\textbf{योग} इति । \textbf{समाधि}श्चित्तैकाग्रता । इह \textbf{धर्मोत्तरेण} लोकप्रसिद्धिराश्रिता । \textbf{विनिश्चयटीकायां} तु शास्त्रस्थितिस्तेनाविरोधः । यद्वा \textbf{समाधि}ग्रहणस्योपलक्षणत्वात् प्रज्ञा च विवेककरणशक्तिर्द्रष्टव्या । \textbf{स यस्यास्ति स} नित्यसमाहितो विवेककरणतत्परश्च \textbf{योगी} । परिसमाप्तेराकारं दर्शयति--\textbf{इयदिति । इयदेव} चतुःसंख्यावच्छिन्नमेव ॥
	\pend
      

	  \pstart \textbf{तदेव}मित्या\textbf{द्याहेत्ये}तदन्तं सुबोधम् । \textbf{विषयो बोद्धव्यः} । क इत्याकाङ्क्षायामाह—\textbf{स्वलक्षणमिति} । स्वशब्दस्य लक्षणशब्दस्य चार्थमाचक्षाणो विग्रहमुपलक्षयति--\textbf{स्वमि}त्यादिना । \textbf{स्वमा}त्मीयमुच्यते । यस्य यत् स्वं तत् तस्यैव नान्यस्येति लक्षणया स्वशब्देनासाधारणमुक्तम् । \textbf{लक्षण}शब्देन च \textbf{तत्त्वं} स्वरूपं विवक्षितम् । \textbf{स्व}शब्द\textbf{लक्षण}शब्दयोरर्थमभिधाय तयोः समस्तं पदमाह--\textbf{स्वलक्षणमिति} । अनेन स्वमसाधारणं च तल्लक्षणं स्वरूपं चेति कर्मधारयो दर्शितः ।
	\pend
      \leavevmode\marginnote{\textenglish{71/dm}}“

	  \pstart द्विविधो हि विषयः \footnote{हि प्रमाणस्य विषयः \cite{dp-msA} \cite{dp-edP} \cite{dp-edH} \cite{dp-edN}}प्रमाणस्य--ग्राह्यश्च यदाकारमुत्यद्यते, प्रापणीयश्च यमध्यवस्यति । अन्यो हि ग्राह्योऽन्यश्चाध्यवसेयः । प्रत्यक्षस्य हि क्षण एको ग्राह्यः । अध्यवसेयस्तु प्रत्यक्षबलोत्पन्नेन निश्चयेन संतान एव । सन्तान एव च प्रत्यक्षस्य प्रापणीयः । क्षणस्य प्रापयितुमशक्यत्वात् ।
	\pend
       

	  \pstart तथानुमानमपि \footnote{सामान्ये--\cite{dp-msD-n}}स्वप्रतिभासेऽनर्थेऽर्थाव्यवसायेन\footnote{ऽनर्थेऽनर्थाध्य० \cite{dp-msA} \cite{dp-edP} \cite{dp-edH}} प्रवृत्तेरनर्थग्राहि\footnote{सामान्यग्राहि--\cite{dp-msD-n}} ।
	\pend
      ”

	  \pstart ननु सम्भवे व्यभिचारे च विशेषणमर्थवत् । अत्र च सर्वस्यैव स्वरूपस्यासाधारणत्वात् सम्भव एव, न व्यभिचार इति किं स्वशब्देन ? एवं तु वक्तव्यम्--वस्तुरूपं तस्य विषय इत्याशङ्कयाह--\textbf{वस्तुन} इति । हिर्यस्मात् । \textbf{चो} वक्ष्यमाणापेक्ष्यः\add{क्षः} समुच्चये । \textbf{सामान्यं} साधारणं रूपं संवृतिज्ञानघटितम् । \textbf{चः} पूर्वापेक्षः समुच्चये । सति चैवं द्वैरूप्ये किं पूर्वं रूपम्, अथ परं प्रत्यक्षस्य विषय इति सन्देहे--\textbf{यदसाधारणं तत्प्रत्यक्षस्य ग्राह्य}मुच्यत इति शेषः । ग्राह्यमिति ब्रुवन् विषयशब्देनाचार्यस्य ग्राह्यो विषयोऽभिप्रेत इति दर्शयति ।
	\pend
      

	  \pstart स्यादेतत्--प्रवृत्तिविषय एव प्रमाणस्य विषयस्ततः प्रवृत्तिविषयस्तस्य विषय इति वक्तव्यम् । तत्किं \textbf{ग्राह्य}मित्युच्यत इत्याह--\textbf{द्विविध} इति । अनेन भेदः प्रतिज्ञातः । \textbf{हि}र्यस्मात् द्विप्रकारो \textbf{विषयः प्रमाणस्येति} । जातिविवक्षयैकवचनम् । एको ग्राह्योऽन्यः प्रापणीयः । भेदमुपपादयति--\textbf{यदाकारं} यत्प्रतिभासं ज्ञानमुत्पद्यते । सोऽपि द्विविधः--परमार्थ आरोपितश्च । द्वयोरपि स्वज्ञाने प्रकाशनमस्त्येवेति द्रष्टव्यम् । \textbf{प्रापणीयो यमध्यवस्यति} । ततो ज्ञानाद् यत्र प्रवर्त्तत इति यावत् । \textbf{च}कारौ पूर्वापरापेक्षया समुच्चयार्थौ ।
	\pend
      

	  \pstart ननु ग्राह्याध्यवसेयशब्दयोरेव भेदो न त्वर्थस्य । यतो यदेव प्रकाशते तदेवाध्यवसीयते, तत् किंमेवमुच्यत इत्याह--\textbf{अन्यो ही}ति । हिर्यस्मादर्थे । \textbf{चो}ऽवधारणे । कस्य कीदृशो ग्राह्य इतरो वेत्याह--\textbf{प्रत्यक्षस्ये}ति । \textbf{हि}रवधारणे एक इत्यस्मात्परो द्रष्टव्यः । कस्तर्ह्यवसेय अध्यवसेय इति \leavevmode\marginnote{\textenglish{31a/ms}} । तुर्ग्राह्यादध्यवसेयं भिनत्ति । अथ किं प्रत्यक्षमवसायात्मकं येन तस्यासाववसेय इत्याह--\textbf{प्रत्यक्षे}ति । प्रत्यक्षपृष्ठभाविनो निश्चयस्य प्रत्यक्षगृहीत एव प्रवृत्ततयाऽनतिशयाधानेन यत् तेनाध्यवसितं तत्प्रत्यक्षेणैवावसितमिति भावः ।
	\pend
      

	  \pstart ननु ग्राह्यादन्यः प्रापणीयो विषयः प्रत्यक्षस्योक्तः । इदानीं पुनरवसेयः । तदयं तृतीयो विषयः प्राप्त इत्याह--\textbf{सन्तान} एवेति । \textbf{चो} यस्मादर्थे । उपादानोपादेयभावेन व्यवस्थितः क्षणप्रबन्ध एकत्वेनाधिमुक्तः \textbf{सन्तानः} । ग्राह्य एव कस्मान्न तेन प्राप्यत इत्याह--\textbf{लक्षणस्ये}ति \footnote{इत्याह--\textbf{क्षणस्ये}ति} । यदाकारं प्रत्यक्षमुत्पद्यते तस्य क्षणस्येति प्रस्तावात् । एवञ्च विवृण्वतोऽस्य \textbf{प्रमाणस्ये}त्यत्र प्रत्यक्षाभिप्रायेण प्रमाणशब्द इन्द्रियजप्रत्यक्षविवक्षया द्रष्टव्यः । स्वसंवेदनादीनां विषयद्वैविध्यासम्भवादिति ।
	\pend
      

	  \pstart अथ भवत्वेकं प्रत्यक्षं विषयद्वैविध्यवद्, अनुमानं तु कथं तथाविधम् ? \textbf{प्रमाणस्ये}ति च  \leavevmode\marginnote{\textenglish{72/dm}} “
	  
	स पुनरारोपितोऽर्थो गृह्यमाणः स्वलक्षणत्वेनावसीयते\footnote{नाध्यवसी० \cite{dp-msC} \cite{dp-msD} \cite{dp-edN}} यतः,\footnote{०सीयते ततश्च स्व० \cite{dp-msD}} ततः स्वलक्षणमवसितं\footnote{०मध्यवसितम् \cite{dp-msA} \cite{dp-msB} \cite{dp-msD} \cite{dp-edP} \cite{dp-edH} \cite{dp-edE} \cite{dp-edN}} प्रवृत्तिविषयोऽनुमानस्य\footnote{०षयोऽस्यानुमा० \cite{dp-msC} \cite{dp-msD}} । अनर्थस्तु ग्राह्यः । तदत्र प्रमाणस्य ग्राह्यं विषयं दर्शयता प्रत्यक्षस्य स्वलक्षणं विषय उक्तः ।” ब्रुवता तदपि विषयद्वैविध्यवदभ्युपगतमेव इत्याशङ्क्याह--\textbf{तथे}ति । येन प्रतिभासाध्यवसायलक्षणेन प्रकारेण प्रत्यक्षमन्यग्राह्यन्याध्यवसायि तथा तेन प्रकारेणेति \textbf{तथा}शब्दार्थः । न केवलं प्रत्यक्षमन्यद् गृह्णाति, अन्यदध्यवस्यति, किन्त्वनुमानमप्यन्यग्राह्यन्याध्यवसायीत्यपिशब्देनाह । इहैवच्छेदः कर्त्तव्योऽन्यथा व्याख्यानमसमञ्जसं स्यात् । किं गृह्णातीत्याह--\textbf{अनर्थग्राही}ति । अनुमानमिति प्रकृतत्वात् । कथमुपपद्यत इत्याशङ्क्योपपत्तिमाह--\textbf{स्वप्रतिभास} इति । स्वस्य प्रतिभास इव प्रतिभासः । शक्तिद्वययोगात्तथारोप्यमाणं रूपम् । तस्मिन्ननर्थेऽबाह्यरूप्ये\footnote{रूपे}\textbf{ऽर्थाध्यवसायेन} बाह्याध्यवसायेन तद्भेदानवभासनात्मकाभेदाध्यवसानलक्षणेन \textbf{प्रवृत्तेः} प्रवर्त्तनात् ।
	\pend
      

	  \pstart अथानर्थे स्वप्रतिभासेऽर्थाध्यवसायेनानुमानविकल्पोऽन्यो वा प्रवर्त्तक इति किमुक्तं भवति ? स्वप्रतिभासस्यारोप्यमाणस्य चार्थस्यावसीयमानस्य विवेकं न प्रतिपद्यत इत्युक्तं भवति । न हि तेनैव विवेकप्रतिपत्तिः, तेन स्वप्रतिभासस्य तथात्वेनाविकल्पनात् । अत्र चानुभवः प्रमाणम् । बाह्यवहन्यादावप्रवृत्त्यादिप्रसङ्गश्च । नापि विकल्पान्तरेण । तस्यापि तथा प्रवृत्तितया पूर्वविकल्पप्रतिभासासंस्पर्शादित्यलमिह विस्तरेण । यद्येवं कोऽस्याध्यवसीयमानः प्रापणीयो विषय इत्याह--\textbf{स} इति । \textbf{पुन}रिति सर्वतो विशिनष्टि । \textbf{आरोपित} इति स्वरूपानुवादः । \textbf{अर्थ} इति अर्थ इवार्थो गृह्यमाणः प्रतीयमानस्तस्मिन् ज्ञाने प्रकाशमान इति यावत् । अनेन स्वलक्षणमध्यवस्यतीदमिति प्रकाशितम् । तदनेनानुमानस्य ग्राह्याध्यवसेयौ भेदेन सामर्थ्यादभिधाय सुखप्रतिपत्त्यर्थं तावेव कण्ठोक्तौ करोति--\textbf{तत} इति । यतः स्वलक्षणत्वेन तस्यावसाय\textbf{स्ततः}\leavevmode\marginnote{\textenglish{31b/ms}}स्वलक्षणमवसितं प्रवृत्तिविषयोऽनुमानस्य । \textbf{स्वलक्षणमवसितमि}त्येतदप्यभिमानादभिधीयते । न पुनः स्वलक्षणमवसायस्य गोचरः । तद्विषयत्वे तस्य निरंशत्वात् क्षणिकत्वादेरप्यवसितावनुमानानवतारप्रसङ्गात् । एतच्च किञ्चिदिहैवोपरिष्टात्स्पष्टयिष्यामः । परार्थानुमाने तु यथाऽवसरं विस्तरेण निर्णेष्यामः ।
	\pend
      

	  \pstart प्रवृत्तिविषयस्यैव प्रापणीयत्वात् प्रापणीय एव प्रवृत्तिविषयशब्देनोक्तः । \textbf{तुः} प्रापणीयाद् विषयाद् ग्राह्यं भेदवन्तं दर्शयति । तस्मिन् ज्ञाने प्रकाशमाने प्रकाशनादनर्थो ग्राह्य उक्तः ।
	\pend
      

	  \pstart स्यादेतत्--यद्यवसितं स्वलक्षणं प्रवृत्तिविषयोऽनुमानस्य तदा कथमेष \textbf{धर्मोत्तरो विनिश्चयटीकाया}मवादीत् “अवसितश्चाकारो विकल्पानां ग्राह्यः” इति । एवं हि स्वलक्षणमेव ग्राह्यं समर्थितं स्यान्नानर्थ इति । ततो व्यक्तो व्याघातः । परमार्थदृष्ट्या तत्र तथाभिधानान्न दोषः । तथाहि परमार्थतो विकल्पानामारोपितमेव रूपमवसेयं ग्राह्यम् तथापि व्यवहर्त्तार आरोपितमेव रूपमवृक्षव्यावृत्तमवस्यन्तो बाह्यस्यापि तथात्वाद् बाह्यं वृक्षमवस्याम इत्यभिमन्यन्ते, तयोर्विवे  \leavevmode\marginnote{\textenglish{73/dm}} काप्रतिपत्तेः । तदनुरोधात्स्वलक्षणमवसितमित्युच्यते । तस्यैव त्वारोपितस्य परमार्थतोऽनुमेयस्य प्रतिभासमानं केवलमाकारमाश्रित्य ग्राह्यत्वमुच्यते । ततस्तत्र वास्तवावसितरूपाभिप्रायेण तथाभिधानात् को विरोधः ?
	\pend
      

	  \pstart तदत्रेत्यादिनोपसंहरति । यस्मात्प्रमाणस्य द्विविधो विषयो ग्राह्याध्यवसायभेदेन तत् तस्मात् । \textbf{अत्र} विषयविप्रतिपत्तिनिराकरणकाले \textbf{ग्राह्यं विषयं दर्शयते}ति ब्रुवन् विषयशब्देनाचार्यस्य ग्राह्मो विषयोऽभिप्रेत इति स्फुटयति ।
	\pend
      

	  \pstart स्यादेतत्--न वै ग्राह्यविषयापेक्षं प्रत्यक्षस्य प्रामाण्यमुपपद्यते । यत एकः क्षणस्तस्य ग्राह्यः । न च तत्प्रापणं सम्भवति । नाप्यनुमानस्य ग्राह्यापेक्षं प्रामाण्यम् । अनर्थो हि तस्य ग्राह्यः । न च तत्प्राप्तिः सम्भविनी । न च येन यस्याप्रापणं तस्य तत्र प्रामाण्यमभ्युपेयते, अतिप्रसङ्गापत्तेः । अथैकनीलक्षणाकारतयोत्पत्तिरेव प्रत्यक्षस्य तत्प्रापणम्, अनुमानस्यापि वह्न्याद्यध्यवसायितया तथोत्पत्तिरेव तत्प्रापणमुच्यते । तर्हि न किञ्चिदवशेषितं स्यात् । कस्य नाम ज्ञानस्य सविकल्पस्य निर्विकल्पकस्य वा तत्तदाकारतयोत्पत्तिर्नास्ति येन तस्य प्रामाण्यं न स्यात् । तस्मान्न द्विविधो विषयः प्रमाणस्याभिधानीयोऽपि त्वेक एव प्रवृत्तिविषयाख्यो विषयः ख्यापनीय इति ।
	\pend
      

	  \pstart अत्र च समाधीयते । ज्ञानानां तावद् ग्राह्याध्यवसा\footnote{से}यभेदेन द्विविधो विषयोऽवश्यैषितव्योऽनुभवसिद्धत्वात् । तत्र प्रामाण्यं प्रवृत्तिविषयापेक्षं व्यवस्थाप्यते । ज्ञानत्वं तूभयापेक्षमेव । अज्ञानस्य च प्रामाण्यासम्भवेन ज्ञानान्तर्भूतं प्रमाणं विषयद्वैविध्यवदेव भवति । केवलं न ग्राह्यापे\leavevmode\marginnote{\textenglish{32a/ms}}क्षं प्रामाण्यमपि तु प्रवृत्तिविषयापेक्षमिति प्रतिपाद्यते । ग्राह्यापेक्षया तु प्रमाणस्येति वचनं स्वरूपानुवादकम् । यद्वा प्रमाणशब्देन ज्ञानमेवात्र विवक्षितम् । यथाऽयं \textbf{विनिश्चयटीकायां} स्वार्थानुमानव्याख्यानावसरे व्यक्तमाह--“द्विविधो ज्ञानानां विषयो ग्राह्यश्चाध्यवसेयश्च” इत्यादि । ततो न किञ्चिदवद्यम् ।
	\pend
      

	  \pstart अथापि स्यात् । यदि ग्राह्यो विषयो न कदाचिदपि प्रत्यक्षस्य प्रापणीयस्तर्हि तेन दर्शितेन किम्प्रयोजनं येनोच्यते \textbf{तदत्र ग्राह्यं विषयं दर्शयता प्रत्यक्षस्य स्वलक्षणं विषय उक्त} इति । नैष दोषः । एवमेवास्य प्रवृत्तिविषयप्रदर्शनात् । यदि हि प्रत्यक्षं क्षणमेकं गृह्णाति तत्रानीलव्यावृत्तिनिश्चयोपजने सन्तानं निश्चाययत् प्रवृत्तिविषयं प्रदर्शयेत् तत्सन्तानभाविनाञ्च ज्ञानानामप्रामाण्यमापादयेत्, न तु किञ्चिदगृह्णत् । तथा हि सन्तानोल्लेखेन निश्चयाभावेऽप्यनीलव्यावृत्तिस्तावदनागतसर्वक्षणसाधारणी प्रत्यक्षेण गृहीतनिश्चिता । तन्निश्चय एव सर्वानीलव्यावृत्तोपादानोपादेयभावस्थितनीलक्षणनिश्चयः, तदभिन्नयोगक्षेमत्वादुत्तरप्रबन्धस्य । तथानिश्चय एव च सन्ताने निश्चय उच्यते । अत एव च तदाद्यं ज्ञानं तथाऽनुष्ठानं तत्सन्तानभावीनि पराञ्चि ज्ञानान्यनीलव्यावृत्तं रूपं गृहीतमेव गृह्णन्त्यनीलव्यावृत्तिनिश्चयं च कृतमेव कुर्वन्त्यनतिशयाधायीनि प्रचुराण्यपि प्रामाण्यात्प्रच्यावयत् प्रामाण्यमात्मसात्करोतीति न्याय्यं ग्राह्यप्रदर्शनमिति सर्वमवदातम् ॥
	\pend
      

	  \pstart इहाचार्यस्य--स्वमसाधारणं सन्तानान्तरसाधारणं यन्न भवति, यदर्थक्रियाक्षममेव सर्वतो व्यावृत्तं तत्त्वं तदेव स्वलक्षणम्, न चैतद्विपरीतमनुमा विषयोऽपीत्यभिप्रेतम् । \textbf{स्वमसाधारणं} \leavevmode\marginnote{\textenglish{74/dm}} “
	  
	कः पुनरसौ विषयो ज्ञानस्य यः स्वलक्षणं प्रतिपत्तव्य इत्याह-- “
	  
	यस्यार्थस्य संनिधानासंनिधानाभ्यां ज्ञानप्रतिभासभेदस्तत् स्वलक्षणम् ॥ १३ ॥” 
	  
	यस्यार्थस्येत्यादि । अर्थशब्दो विषयपर्यायः । यस्य ज्ञानविषयस्य । संनिधानं निकटदेशावस्थानम् । असंनिधानं दूरदेशावस्थानम् । तस्मात् संनिधानादसंनिधानाच्च ज्ञानप्रतिभासस्य ग्राह्याकारस्य भेदः स्फुटत्वास्फुटत्वाभ्याम् । यो हि ज्ञान\footnote{ज्ञानस्य विषयः \cite{dp-msA} \cite{dp-edP} \cite{dp-edH} \cite{dp-edE} \cite{dp-edN}}विषयः संनिहितः सन्\footnote{स स्फुटा० \cite{dp-msB} \cite{dp-msC} \cite{dp-msD}} स्फुटाभासं ज्ञानस्य करोति, असंनिहितस्तु योग्यदेशस्थ\footnote{देशावस्थित \cite{dp-msA} \cite{dp-msB} \cite{dp-msC} \cite{dp-edP} \cite{dp-edH} \cite{dp-edE} \cite{dp-edN}} एवास्फुटं करोति, तत् स्वलक्षणम् । सर्वाण्येव हि वस्तूनि दूरादस्फुटानि दृश्यन्ते, समीपे स्फुटानि । तान्येव\footnote{तान्येव हि स्व० \cite{dp-msC}} स्वलक्षणानि ॥” \textbf{लक्षणं तत्त्वं स्वलक्षणमि}ति विवृण्वता च \textbf{धर्मोत्तरेण} तदेव दर्शितम् । केवलं न व्यक्तीकृतमतस्तस्य \textbf{विषयः स्वलक्षणमि}ति शब्दमात्रश्राविणोऽविदिताचार्याभिप्रायस्याविभावितधर्मोत्तरविवरणार्थस्य यद्यसाधारणं पररूपेणामिश्रं रूपं स्वलक्षणं तदाऽग्नित्वमपि गोत्वादिसामान्येनामिश्रं स्वलक्षणं प्रसज्येतेति व्याप्तिं मन्यमानस्यायं \textbf{कः पुन}रित्यादिप्रश्नः । \textbf{क} इत्यनेन सामान्याकारेण विषयं पृच्छति । \textbf{पुनरि}त्यनेन विशेषाकारेण । स्वलक्षणशब्दस्यासति बहुव्रीहावजहल्लिङ्गत्वात्\textbf{स्वलक्षणमि}ति ।
	\pend
      

	  \pstart चोदकेन विषयं सम्पृष्ट आचार्यः कस्मादर्थमुत्तरीकरोतीत्याशङ्क्याह--\textbf{अर्थेति} । ज्ञानस्य प्रत्यासन्नत्वाज्ज्ञानविषयो लब्ध इत्यभिप्रेत्य \textbf{यस्य ज्ञानविषय}स्येति विवृणोति । \textbf{सन्निधानासन्निधान}शब्दौ निकटदूरावस्थानार्थौ व्याचक्षाणो यदन्यैराख्यातं--“असन्निधानं योग्यदेशे सर्वथा वस्तुनोऽभावः” इति तदपाकरोति । अयञ्चास्याभिप्रायो यत्र वस्तु नास्ति तत्र ज्ञानमेव न जायते । न तु तत्प्रतिभासस्य भेदो नानात्वमस्तीति । \textbf{तस्मादि}त्यादिना \textbf{स}\leavevmode\marginnote{\textenglish{32b/ms}}\textbf{न्निधानासन्निधानाभ्यामिति} पञ्चमीद्विवचनान्तमेतदिति दर्शयति । \textbf{स्फुटत्वास्फुटत्वाभ्यामिति अर्थक्रिया}सामर्थ्यानुगताभ्यामिति द्रष्टव्यम् । \footnote{पाणिनि २. ३. २१.}इत्थम्भूतलक्षणा चेयं तृतीया ।
	\pend
      

	  \pstart पदार्थं व्याख्याय समुदायार्थं व्याचष्टे । द्वयी चेयं शैली व्याख्यातृणाम् । क्वचित्समुदायार्थ व्याख्याय पश्चात्पदार्थम् विवृण्वते । क्वचित्पदार्थं विवृत्य पश्चात्समुदायार्थं व्याचक्षत इति । \textbf{हि}रवधारणे । \textbf{योग्यदेशस्थ एव}त्यनेन दूरदेशस्थं निरस्यति । दूरस्थो हि ज्ञानमेव \textbf{न जनय}ति किमङ्ग पुनर्भिन्द्यादित्यभिप्रायः ।
	\pend
      

	  \pstart स्वलक्षणलक्षणस्याव्यापित्वासम्भवित्वशङ्कामपाकुर्वन्नाह--सर्वा\textbf{णीति} । । \textbf{सर्व}शब्दो व्याप्तिप्रदर्शनार्थः । अनेनाव्यापित्वं निराकृतम् । \textbf{हि}र्यस्मादर्थे । \textbf{दृश्यन्ते} प्रतीयन्ते । अनेनासम्भवित्वं निरस्तम् । अतिव्यापित्वमपहस्तयन्नाह--\textbf{तान्येवेति} । यान्यमूनि प्रत्यक्षविषयस्तान्येव । नानुमानविषयोऽपीति प्रकरणात् । \textbf{दूरादस्फुटानि दृश्यन्त} इति वदतश्चायमाशयः--दूरे हि वस्तु गृह्यमाणं प्रचुररजोनीहारादिसंसृष्टं गृह्यते । ततोऽस्पष्टं गृह्यते । न तु  \leavevmode\marginnote{\textenglish{75/dm}} “
	  
	कस्मात् पुनः प्रत्यक्षविषय एव स्वलक्षणम् ? तथा हि विकल्पविषयोऽपि वह्निर्दृश्यात्मक\footnote{०क इवाव \cite{dp-msD}} एवावसीयत इत्याह-- “
	  
	तदेव परमार्थसत् ॥ १४ ॥” 
	  
	तदेव परमार्थसदिति । \footnote{परमार्थोऽकृ० \cite{dp-msA} \cite{dp-msB} \cite{dp-msC} \cite{dp-edP} \cite{dp-edH} \cite{dp-edE} \cite{dp-edN}}परमोऽर्थोऽकृत्रिममनारोपितं रूपम् । तेनास्तीति परमार्थसत् । य एवार्थः सन्निधानासन्निधानाभ्यां स्फुटमस्फुटं च प्रतिभासं करोति परमार्थसन् स एव । स\footnote{स एव च \cite{dp-msB} \cite{dp-edH} \cite{dp-edE} \cite{dp-edN}} च प्रत्यक्षस्य\footnote{प्रत्यक्षविषयो \cite{dp-msA} \cite{dp-msB} \cite{dp-msC} \cite{dp-msD} \cite{dp-edP} \cite{dp-edH} \cite{dp-edE} \cite{dp-edN}} विषयो यतः, तस्मात् तदेव स्वलक्षणम् ॥” न गृह्यत एव । तथागृहीतस्यापि वृक्षस्य च्छायाद्यर्थक्रियाकारित्वात् । न चाधिकग्रहणं भ्रम इति ।
	\pend
      

	  \pstart इदं पुनरत्र निरूप्यते । यदि य एवार्थः सन्निधानादसन्निधानाच्च ज्ञानप्रतिभासं स्फुटत्वास्फुटत्वाभ्यां भिनत्ति स एव स्वलक्षणं तर्हि स्पर्शरसौ स्वलक्षणे न स्याताम् । तौ खलु असन्निहितौ ज्ञानमेव न जनयतः । किं पुनर्ज्ञानप्रतिभासं स्फुटत्वास्फुटत्वाभ्यां भेत्स्यतः ? किञ्चैतस्मिन् स्वलक्षणलक्षणे विज्ञानमस्वलक्षणं स्यात् । तस्याऽऽस्तां तावदसन्निहितस्यास्फुटज्ञानजनकत्वं सन्निहितस्यापि स्फुटज्ञानजनकत्वं नास्ति । न च तस्य दूरान्तिकवर्त्तित्वमस्त्यदेशत्वात् । तदीदृशीं महतीमव्यापितामनालोच्येदृशं स्वलक्षणं प्रणयन्ना\textbf{चार्यः, धर्मोत्तरो}ऽप्येवं प्रसभं व्याचक्षाणः कथं न प्रमाद्यतीति ? न । अभिप्रायापरिज्ञानात् । यदि हीदं लक्षणं यथाश्रुति व्यवतिष्ठेत, स्यादेवैतत् । केवलं सन्निधानासन्निधानाभ्यां ज्ञानप्रतिभासभेदकत्वेन यदेकार्थसमवेतमसाधारण्यं व्यक्त्यन्तरासनुयायित्वं तदुपलक्षितम् । यतो \textbf{हेतुबिन्दुः} “तत्र तदाद्यमसाधारणविषयम्” इति \href{http://http://sarit.indology.info/?cref=hbṭ.1.28}{पृ० ५३} । अत एव--“असाधारणविषयं स्वलक्षणविषयम्” इति \href{http://http://sarit.indology.info/?cref=hbṭ.1.14}{हेतु० टी० पृ० २५} \textbf{भट्टार्चटो} व्याचष्टे । अत एवानुमानस्यैतद्विपर्ययेण साधारणं रूप विषयो दर्शयिष्यते । तेन नाव्याप्तिर्न चान्यो लक्षणदोषः । \textbf{तस्य विषयः स्वलक्षणमि}त्यभिहितेऽपि स्वलक्षणशब्दस्यान्यथापि निर्वचनसम्भवात्, नायमभिमतोऽर्थो ज्ञायेत प्रतिपत्तृभिरिति तदभिमतार्थ इत्थमुपलक्षित इति च द्रष्टव्यम् ॥
	\pend
      

	  \pstart \textbf{तस्य विषयः स्वलक्षण}मित्यत्र यदि तस्यैव विषयः स्वलक्षणमित्यवधार्यते तदा यत एवकारकरणं ततोऽन्यत्रावधारणमिति स्वलक्षणस्य प्रत्यक्षे नियमात् प्रत्यक्षमन्यविषयमपि स्यात् । अथ स्वलक्षणमेवेत्यवधार्यते । तदाऽपि प्रत्यक्षस्य स्वलक्षणे नियतत्वाद् \leavevmode\marginnote{\textenglish{33a/ms}} अनियतं स्वलक्षणमनुमानस्यापि विषयः स्यादित्युभयावधारणं कार्यम् । तस्यैव विषयः स्वलक्षणमेवेति । एतदसहमानः पूर्वपक्षवाद्याह--\textbf{कस्मादिति । कस्मादिति} सामान्येन कारणं पृच्छति । \textbf{पुनरिति} विशेषतः । यद्वा \textbf{यस्यार्थस्ये}त्यत्र तत्स्वलक्षणमिति । \textbf{तदेव} प्रत्यक्षविषयः स्वलक्षणं नानुमानविषय इत्यभिप्रेतं तदसहमान एवमाह । \textbf{कस्मात्प्रत्यक्षविषय एवेति} ब्रुवतोऽनुमानस्यापि विषयः किं न स्वलक्षणमित्यभिप्रायः । स एव किं न तथेत्याशङ्क्य पूर्वपक्षवाद्येवोपपत्ति  \leavevmode\marginnote{\textenglish{76/dm}} “
	  
	कस्मात् पुनस्तदेव परमार्थसदित्याह-- “
	  
	अर्थक्रियासामर्थ्यलक्षणत्वाद्वस्तुनः ॥ १५ ॥” 
	  
	अर्थ्यत इत्यर्थः । हेय उपादेयश्च । हेयो हि हातुमिष्यते उपादेयश्चोपादातुम् । अर्थस्य प्रयोजनस्य क्रिया निष्पत्तिः । तस्यां सामर्थ्यं शक्तिः । तदेव लक्षणं रूपं यस्य वस्तुनः” माह--तथा हीति । विकल्पशब्दः प्रमाणविषयचिन्तनादनुमानविकल्पो द्रष्टव्यः । \textbf{दृश्यात्मक एव} स्वलक्षणात्मक एव ।
	\pend
      

	  \pstart ननु “यस्यार्थस्य सन्निधानासन्निधानाभ्यां ज्ञानप्रतिभासभेदस्तत्स्वलक्षणम्” इत्युक्ते कुतोऽस्य प्रश्नस्यावकाशः ? न हि विकल्पविषयस्य सन्निधानासन्निधानाच्च ज्ञानप्रतिभासभेदोऽस्ति यतोऽस्योत्थानं स्यात् । सत्यम् । केवलमयमस्याशयः--यदेतद् भवद्भिर्लक्षणं स्वलक्षणस्य प्रतीतं तदस्य लक्षणमेव न भवति किन्तु यदेव दृश्यतयाऽध्यवसीयते तदेव स्व\add{लक्षण}मितीदं तस्य लक्षणम् । सति चैवमनुमानविकल्पविषयस्यापि तथात्वमनिवारितमेवेति सूत्थानः प्रश्नः । \textbf{एतत्प्रश्नविसर्जनमाचार्यीयं} दर्शयन्नाह--\textbf{तदेवेति ।}
	\pend
      

	  \pstart अथ स्यात् यदि\footnote{पङिक्तबाह्यं लिखितं न पठ्यते--सं०}\add{... ... ...}त्युच्यमानं शोभेत यावतेदमेव न सिद्धं तत्कुतोऽयं प्रश्नविसर्जनप्रकार इति ? उच्यते । स्वलक्षणमिति स्वमसाधारणं पारमार्थिकः स्वभाव उच्यते । स एव च पारमार्थिकः स्वभाव उच्यते, य एवार्थक्रियाक्षमः । स एव च परमार्थसन्निति युक्तमिदमुत्तरं \textbf{तदेवेति । परमोऽर्थ} इति दर्शयन् कर्मधारयं दर्शयति । \textbf{अकृत्रिम}मित्यादि परमशब्दस्य व्याख्यानम् । \textbf{तेनास्ती}ति सदित्यस्यार्थकथनम् । तेन रूपेण सद् विद्यमानमिति विग्रहः । शत्रन्तश्चायमसिः । “कर्त्तृकरणे कृता बहुलम्” \href{http://http://sarit.indology.info/?cref=Pā.2.1.32}{पाणिनि २. १. ३२}इति च समासः । परमार्थसदित्यस्यार्थमाख्याय तदेवेत्येतद् विवृणोति \textbf{य एवेति । प्रतिभासं} ज्ञानस्येति प्रकरणात् । अर्थक्रियासमर्थोऽर्थः । स्वलक्षणं चैवमात्मकमित्यभिप्रायेण \textbf{य एवार्थः स एवेत्याह} । अस्तु तादृशः परमार्थसन् । स तु न तस्य विषयोऽन्यस्यापि वा विषयो भविष्यति । तथा च विवक्षितार्थासिद्धिरित्याह \textbf{स चे}ति । \textbf{चो}ऽवधारणे \textbf{प्रत्यक्षस्ये}त्यतः परो द्रष्टव्यः । \textbf{तदेवेति} प्रत्यक्षविषयत्वेन स्थितं \textbf{वस्त्वेव} न त्बनुमानविषयोऽपीति । अनेन \textbf{कस्मात्पुनः प्रत्यक्षविषय एव स्वलक्षण}मित्येतत्प्रश्नविषर्जनस्योपसंहारः कृतो वेदितव्यः ॥
	\pend
      

	  \pstart इदानीं स्वलक्षणस्यैव परमार्थसत्त्वमसहमान आह--\textbf{कस्मादि}ति । अयमस्याशयः—स एव खलु परमार्थसन्नर्थो य एवार्थत्वेनाध्यवसीयते । अनुमानविषयोऽपि च वह्निस्तथाऽध्यवसीयत इति सोऽपि कस्मान्न परमा\leavevmode\marginnote{\textenglish{33b/ms}}र्थसन्निति ।
	\pend
      

	  \pstart अत्र--\textbf{अर्थे}त्यादि यदुत्तरं तद् \textbf{अर्थ्यत} इत्यादिना व्याचष्टे । \textbf{च}स्तुल्यबलत्वं समुच्चिनोति । हेयः कथमर्थ्यत इत्याह--हेय इति । हिर्यस्मात् । यदि हातुमर्थ्यमानोऽर्थो न तर्ह्युपादेयोऽर्थ इत्याह--\textbf{उपादेय} इति । \textbf{चो} हेयापेक्षयोपादेयस्यार्थ्यमानत्वं समुच्चिनोति । \textbf{अर्थ}स्य प्रयोजनस्य दाहादेः । एवं च व्याचक्षणः साध्यो दाहादिरेव मुख्यवृत्त्योपादेया हेयो वा । \leavevmode\marginnote{\textenglish{77/dm}} “
	  
	तद् अर्थक्रियासामर्थ्यलक्षणम् । तस्य भावः, तस्मात् । वस्तुशब्दः परमार्थपर्यायः । तदयमर्थः--यस्मादर्थक्रियासमर्थं परमार्थसदुच्यते, सन्निधानासन्निधानाभ्यां च ज्ञानप्रतिभासस्य भेदकोऽर्थोऽर्थक्रियासमर्थः, तस्मात् स एव परमार्थसन् । तत एव हि प्रत्यक्षविषयादर्थक्रिया प्राप्यते न विकल्पविषयात्\footnote{सामान्यात्--\cite{dp-msD-n}} । अत एव यद्यपि विकल्पविषयो दृश्य इवावसीयते तथापि न\footnote{न दृश्य \cite{dp-msA} \cite{dp-msB} \cite{dp-edP} \cite{dp-edH} \cite{dp-edE} \cite{dp-edN}} स दृश्य एव \footnote{विकल्पविषयात्--\cite{dp-msD-n}}ततोऽर्थक्रियाया\footnote{०क्रियाभावात्--\cite{dp-msA} \cite{dp-msB} \cite{dp-edP} \cite{dp-edH} \cite{dp-edE} \cite{dp-edN}} अभावात्, दृश्याच्च भावात् । अतस्तदेव स्वलक्षणं न विकल्पविषयः\footnote{विषयम् \cite{dp-msA} \cite{dp-msB} \cite{dp-edP} \cite{dp-edH} \cite{dp-edE} \cite{dp-edN}} ॥ “
	  
	अन्यत् सामान्यलक्षणम् ॥ १६ ॥” 
	  
	\footnote{“अन्यदित्यादि”--नास्ति \cite{dp-msA} \cite{dp-edP} \cite{dp-edH} \cite{dp-edE} \cite{dp-edN}}अन्यदित्यादि । एतस्मात् स्वलक्षणाद् यद् अन्यत्--स्वलक्षणं यो न भवति ज्ञानविषयः--तत्\footnote{तस्मात् सामा० \cite{dp-msB}} सामान्यलक्षणम्, विकल्पज्ञानेनावसीयमानो ह्यर्थः सन्निधानासन्निधानाभ्यां ज्ञानप्रतिभासं न भिनत्ति । तथाहि आरोप्यमाणो वह्निरारोपादस्ति । आरोपाच्च दूरस्थो निकटस्थश्च । तस्य समारोपितस्य सन्निधानासन्निधानाच्च ज्ञानप्रतिभासस्य न भेदः स्फुटत्वेनास्फुटत्वेन वा । ततः स्वलक्षणादन्य उच्यते । सामान्येन लक्षणं सामान्यलक्षणम् । साधारणं रूपमित्यर्थः । 
	  
	समारोप्यमाणं हि रूपं सकलवह्निसाधारणम् । ततः\footnote{ततस्तस्मात् सामा० \cite{dp-msB}} तत् सामान्यलक्षणम् ॥” तमेवोपादातुं हातुं वा तत्साधनस्योपादानं हानं वाऽन्यथा न शक्यत इति तत्साधनभूतो वह्न्यादिरुपादेयादिरुपपद्यत इति दर्शयति । \textbf{तस्यामित्या}दि \textbf{तस्मादि}त्यन्तं सुबोधम् । परमार्थसतस्तत्त्वे प्रतिपाद्ये किमति वस्तुनस्ताद्रूप्यमाचार्येण दर्शितमित्याशङ्कामपाकुर्वन्नाह--\textbf{वस्तुशब्द} इति । \textbf{तदयमर्थ} इत्यादिर्न \textbf{विकल्पविषय} इत्यन्तो ग्रन्थस्तूक्तार्थोपसंहारः । स च सुज्ञानः ॥
	\pend
      

	  \pstart इह प्रस्तुतस्वलक्षणापेक्षमन्यत्वमाचार्यस्याभिप्रेतम् । स्वलक्षणशब्दस्य चान्यशब्दयोगात्पञ्चम्या भवितव्यमित्यभिप्रेत्य \textbf{एतस्मा}दित्याह । अन्यत्वमभिव्यनक्ति--\textbf{स्वलक्षणं यो न भवती}ति । शशविषाणादिव्यवच्छेदार्थमाह--\textbf{ज्ञानविषय} इति । प्रमाणविषयचिन्तायाः प्रस्तुतत्वाद् विज्ञानविषय इति लब्धम् । असिद्धमस्य ततोऽन्यत्वम्, अस्यापि स्वलक्षणकार्यकारित्वादित्याह--\textbf{विकल्पेत्यादि । हिर्य}स्मादर्थे ।
	\pend
      

	  \pstart प्रतिभासाभेदकत्वं तस्य कुतोऽवसीयते सर्वदा तस्य सन्निहितरूपत्वादसन्निहितरूपत्वादेव वेत्याह--\textbf{तथा हीति । आरोप्यमा}णस्तथा प्रतीयमानः । \textbf{आरोपात्} तथोत्पादलक्षणादध्यवसायाद् \textbf{अस्ति} वह्निरूपेण । आस्तामारोपात् तथाभूतस्तथापि तस्य सन्निधानासन्निधान\footnote{ने} कुत इत्याह--\textbf{आरोपादि}ति । \textbf{चो} व्यक्तमेतदित्यस्यार्थे । द्वितीयश्चकारो दूरस्थत्वापेक्षया निकटस्थत्वस्यैकविषयत्वं समुच्चिनोति ।
	\pend
      

	  \pstart तस्येत्याद्युच्यत इत्यन्तं स्पष्टार्थम् ।
	\pend
      \leavevmode\marginnote{\textenglish{78/dm}}“

	  \pstart तच्चानुमानस्य ग्राह्यं दर्शयितुमाह--
	\pend
       “

	  \pstart सोऽनुमानस्य विषयः ॥ १७ ॥
	\pend
      ” 

	  \pstart सोऽनुमानस्य विषयो ग्राह्यरूपः\footnote{बहुव्रीहि--\cite{dp-msD-n}} । सर्वनाम्नोऽभिधेयवल्लिङ्गपरिग्रहः ।
	\pend
       

	  \pstart सामान्यलक्षणमनुमानस्य विषयं व्याख्यातुकामेनायं स्वलक्षणस्वरूपाख्यानग्रन्थ आवर्त्तनीयः स्यात् । ततो लाघवार्थं प्रत्यक्षपरिच्छेद एवानुमानविषय उक्तः ॥
	\pend
      ”

	  \pstart सामान्यशब्देन लक्षणशब्दस्य विग्रहम ह--\textbf{सामान्येने}ति । \textbf{सामान्येन, \add{न}} विशेर्षेण सन्तानान्तरसाधारणं स्वरूपम् । साधनं कृतेति समासः । तस्य पदस्यार्थं स्पष्टयति—\textbf{साधारणमि}ति ।
	\pend
      

	  \pstart विकल्पविषयस्याप्यसाधारणत्वादसिद्धं व्यक्त्यन्तरसाधारणत्वमित्याह--\textbf{समारोप्यमाणमि}ति । \textbf{हि}र्यस्मात् । \textbf{समारोप्यमाणं} विकल्पेन तथा प्रतीयमानम् । \textbf{सकलश्चासौ} तार्णपार्णादिभेदभिन्नो \textbf{वह्नि}श्चेति तथा । तत्\textbf{साधारणं} तथाविधलिङ्गबलेनानग्निव्यावृत्तवस्तुमात्रप्रतिभासनादिति भावः ।
	\pend
      

	  \pstart ननु च तथाविधं सामान्यं विकल्पगोचरोऽवस्तु । तद्विषयत्वेऽनुमानस्य कथं बाह्ये प्रवर्त्तकत्वं तत्प्रापकत्वञ्च, यतः प्रामाण्यमस्य स्यादिति चेद् । उच्यते । विकल्पाः खल्वेतेऽनाद्यविद्यावशात्स्वप्रतिभासमनग्निव्यावृत्तमवस्यन्तो बाह्योऽप्यनग्निव्यावृत्त इति तदध्यवसानमेव बाह्यो वह्निरध्यवसित इति मन्यन्ते । अनग्निव्यावृत्ततया बाह्यसदृशवह्न्यध्यवसाय एव बाह्यवह्न्यध्यवसायः । तयोर्विवेकाप्रतिपत्तेः । अत एव ते विकल्पा दृश्यविकल्प्यावर्थावेकीकृत्य बाह्ये लोकं प्रवर्त्तयन्ति । दृश्यविकल्प्यैकीकरणमपि तेषां तथा प्रवृत्तिहेतुतयोत्पत्तेरेव द्रष्टव्यम् । \leavevmode\marginnote{\textenglish{34a/ms}} बाह्यसम्बद्धसम्बद्धत्वाच्चानुमानविकल्पः संवादकः, अध्यवसेयापेक्षया \textbf{च} प्रमाणम् । तदाह \textbf{न्यायवादी}--“भ्रान्तिरपि सम्बन्धतः प्रमा” इति । एष चार्थः \textbf{स्वप्रतिभासेऽनर्थेऽर्थाध्यवसायेन प्रवृत्ते}रित्यत्राप्यपेक्षणीयः । समासतस्तु तत्रास्माभिः किञ्चिदवादीति । यत एवं तत्तस्मादित्युपसंहारः । \textbf{तदि}ति तदारोप्यमाणम् ॥
	\pend
      

	  \pstart \textbf{तच्चे}त्यादिना सोऽनुमानेत्यस्यावतारं दर्शयति । \textbf{चो}ऽवधारणे । \textbf{ग्राह्यं रूपं} स्वभावोऽस्येति विग्रहः । एवं व्याचक्षाणो विषयशब्देन ग्राह्यो विषयोऽभिप्रेतोऽत्र प्रमाणव्यापारविषयोऽध्यवसेयः स्वलक्षणस्यैव तदध्यवसेयत्वेन तथा व्यवस्थापनादिति दर्शयति । तस्य तु तत्र स्वरूपेणाप्रतिभासनादग्राह्यत्वम् । प्रकाशमानं च रूपमाश्रित्य ग्राह्यत्वमुच्यत इत्युक्तप्रायम् ।
	\pend
      

	  \pstart अनुमानस्यापि ग्राह्यविषयदर्शनेऽयमभिप्रायो यदीदमनुमानमनर्थमनग्निव्यावृत्तिमात्रं सामान्यरूपं गृहणाति, तत्र तत्सामान्यरूपं किञ्चिदवभासेत, तदा तत्स्वलक्षणत्वेनावस्यत्स्वलक्षणं प्रवृत्तिविषयीकुर्यात् । तथा कुर्वच्च वस्तुविषयं प्रामाण्यमश्नुवीत नान्यथेति । स \leavevmode\marginnote{\textenglish{79/dm}} “
	  
	विषयविप्रतिपत्तिं निराकृत्य फलविप्रतिपर्त्तिं निराकर्त्तुमाह-- “
	  
	तदेव\footnote{तदेव प्रत्य० \cite{dp-msC}} च प्रत्यक्षं ज्ञानं प्रमाणफलम् ॥ १८ ॥” 
	  
	तदेवेति । यदेवानन्तरमुक्तं प्रत्यक्षं \footnote{प्रत्यक्षं तदेव \cite{dp-msA} \cite{dp-msB} \cite{dp-edP} \cite{dp-edH} \cite{dp-edE}}ज्ञानं तदेव प्रमाणस्य फलम् ॥ 
	  
	कथं प्रमाणफलमित्याह-- “
	  
	अर्थप्रतीतिरूपत्वात् ॥ १९ ॥” 
	  
	अर्थस्य प्रतीतिरवगमः, सैव रूपं यस्य प्रत्यक्ष\footnote{प्रत्यक्षस्य ज्ञान० \cite{dp-msD}}ज्ञानस्य तदर्थप्रतीतिरूपम् । तस्य भावः, तस्मात् । 
	  
	एतदुक्तं भवति--प्रापकं ज्ञानं प्रमाणम् । प्रापणशक्तिश्च न केवलादर्थाविनाभावित्वाद् भवति । बीजाद्यविनाभाविनोऽप्यङ्कुरादेरप्रापकत्वात् । तस्मात् \footnote{तस्मादर्थादुत्प० \cite{dp-msA} \cite{dp-edP} \cite{dp-edH} \cite{dp-edE} तस्माद् ग्राह्यादर्थादुत्प० \cite{dp-edN}}प्राप्यादर्थादुत्पत्तावप्यस्य ज्ञानस्याऽस्ति कश्चिदवश्यकर्त्तव्यः प्रापकव्यापारोयेन कृतेनार्थः प्रापितो भवति । 
	  
	स एव च प्रमाणफलम्, यदनुष्ठानात् प्रापकं भवति ज्ञानम् । उक्तं च पुरस्तात् “प्रवृत्तिविषयप्रदर्शनमेव प्रापकस्य प्रापकव्यापारो नाम” । 
	  
	तदेव च प्रत्यक्षम् अर्थप्रतीतिरूपम् \footnote{अर्थदर्शन \cite{dp-msA} \cite{dp-edP} \cite{dp-edH} \cite{dp-edE} \cite{dp-edN}}अर्थप्रदर्शनरूपम् । अतस्तदेव प्रमाणफलम् ॥” इति पुंल्लिङ्गनिर्देशसमर्थनार्थमाह--\textbf{सर्वे}ति । इवार्थे \textbf{वति} । तद्वल्लिङ्गस्य परिग्रह \add{इति षष्ठ्यन्तेन} विग्रहः । अत्रापि सामान्यलक्षणमेवानुमानस्य विषय इत्यवधारणीयं न त्वनुमानस्यैवेति प्रत्यक्षपृष्ठभाविनोऽपि विकल्पविशेषस्य तत्विषयत्वादन्यथा लिङ्गनिश्चयायोगादिति ।
	\pend
      

	  \pstart सम्प्रति प्रत्यक्षस्य विषयविप्रतिपत्तिनिराकरणे प्रकृते किमप्रकृतमनुमानस्य विषयविप्रतिपत्तिनिराकरणमाचरितमित्याशङ्कामपाकर्त्तुं \textbf{सामान्येत्या}दिनोपक्रमते । एत\textbf{च्चोक्त} इत्येतदन्तं सुज्ञानम् ॥
	\pend
      

	  \pstart \textbf{विषये}त्यादि \textbf{फल}मित्येदन्तं सुबोधम् । फलप्रदर्शने चायं \textbf{वार्त्तिककार}स्याशयः—प्रमाकरणं खलु लोके प्रमाणमुच्यते । ततः करणसाध्या फलरूपा प्रमितिरवश्यदर्शयितव्येति ॥
	\pend
      

	  \pstart ननु यस्मिन् विषये क्रियासाधनं व्याप्रियते\footnote{अत्र पङ्क्तिबाह्यं किञ्चिल्लिखितमस्ति । सूक्ष्मत्वात् न पठ्यते--सं०}...कथमिति । किमन्यत्प्रमाणमिति पुनरिहोपेक्षितमनेन । अर्थेत्यादि प्रतिवचनं व्याचष्टे \textbf{अर्थस्ये}ति । एतच्च \textbf{तस्मादि}त्यन्तं सुगमम् ।
	\pend
      

	  \pstart ननु संवादकं ज्ञानं प्रमाणम् । संवादश्च तदुत्पत्तिमात्रात् । तत् किमर्थे प्रतीतिस्तत्फलं मृग्यत इत्याशङ्क्याह--\textbf{एतदुक्तं भवती}ति ।
	\pend
      \leavevmode\marginnote{\textenglish{80/dm}}

	  \pstart अथ प्रापकत्वं प्रापणशक्तियोगात् । सा च शक्तिस्तत उत्पत्तेरेव । तथा च कथं पूर्वपक्षातिक्रम इत्याह--\textbf{प्रापणशक्तिश्चे}ति । \textbf{चो} यस्मात् । \textbf{न केवला}देकाकिनोऽर्थप्रदर्शनविनाकृतात् । विवक्षितेन विना भवतीति विनाभावी ग्रहादित्वाण्णिनिः, व्यभिचारीत्यर्थः । अर्थेन विना भावीति “तृतीया” \href{http://http://sarit.indology.info/?cref=Pā.2.1.30}{पा० २. १. ३०} इति योगविभागात्समासः । तस्य भावस्तस्मात् । कथं पुनर्न केवलादित्याह--\textbf{बीजादिति} । यत एवं \textbf{तस्मात्} । यद्यपि सन्तान एव प्राप्यो न च तस्मात्तस्योत्पत्तिस्तथापि विषयेऽस्यैकत्वमध्यवसाय \textbf{प्राप्यादर्थादुत्त्पत्तावपी}त्युक्तम् । \textbf{प्रापकस्य ज्ञानस्यावश्यकर्त्तव्यः} प्रतीतिक्रियारूपः । प्रापितो भवतीति पूर्ववद् योग्यत\leavevmode\marginnote{\textenglish{34b/ms}}योच्यते । भवतु तस्य कर्त्तव्यमन्यत्, तथापि न तत्फलमित्याह--\textbf{स एवे}ति । \textbf{च} स्फुटमेतदित्यस्मिन्नर्थे ।
	\pend
      

	  \pstart भवतु अवश्यकर्त्तव्यः प्रापकव्यापारः । स पुनरज्ञानात्मको भविष्यति । तथा च कथं प्रतीतिः फलमित्याह \textbf{उक्त}मिति । \textbf{चो} यस्मादर्थे । \textbf{पुरस्ता}त्पूर्वस्मिन् । केनोक्तमेतदिति चेत् \textbf{प्रवर्त्तकत्वमपि प्रवृत्तिविषयोपदर्शकत्वमेवे}त्यादिना । आस्तामर्थप्रतीतिरवश्यकर्त्तव्या प्रमाणस्य तथापि कथं तदेव प्रत्यक्षज्ञानं फलं तस्यातद्रूपत्वादित्याशङ्क्य यदर्थप्रतीतिरूपत्वं प्रतिज्ञातं तदुपपादयन्नाह--\textbf{तदेवे}ति । \textbf{चो} यस्मादर्थे ।
	\pend
      

	  \pstart नन्वर्थप्रदर्शनं कर्त्तव्यतया प्रदर्शितम् । तत् कथमर्थप्रतीत्यामेकं तत् प्रदर्श्यत इत्याह—\textbf{अर्थप्रदर्शनरूपमि}ति । अनयोरर्थाभेद इत्यभिप्रायः । यत एव\textbf{मतो}ऽस्मा\textbf{त्तदेव} प्रत्यक्षं ज्ञानमेव नार्थान्तरमित्यर्थात् । प्रमाणादन्यज्ज्ञानमेव फलं यदि स्यात्, किंस्यात् ? येन यत्नेन गरीयसा प्रमाणादभेदोऽस्य साध्यत इति चेत् । उच्यते । “धीप्रमाणता, प्रवृत्तेस्तत्प्रधानत्वात्” \href{http://http://sarit.indology.info/?cref=pv.1.5}{प्रमाणवा०
	    १. ५.}इत्यत्राज्ञानात्मनस्तावत्प्रामाण्यमपहस्तितम् । ज्ञानात्मनश्च प्रमाणस्या\footnote{स्य} भिन्नेऽपि प्रमितिरूपे फलेऽवश्यमर्थपरिच्छेदात्मताऽभ्युपेतव्या । अन्यथा ज्ञानत्वमेव तस्य न स्यात् । प्रमाजनकत्वेन च ज्ञानत्वे चक्षुरादेरपि ज्ञानत्वं स्यात् । बोधरूपत्वं चास्वसंवेदनतया न नियामकम् । स्वसंवेदनरूपत्वे वा ग्राह्याकारसंवेदनमेवार्थवेदनमिति कथं भिन्नं फलम् ? प्रतिकर्मव्यवस्थाऽनुपपत्तेश्च न निराकारत्वं विज्ञानस्याभ्युपेयम् । ज्ञानमेवार्थस्य प्रकाश इति च सर्वतन्त्रसिद्धोऽयमर्थः । ततो न ज्ञानस्याधिगमरूपतायां विप्रतिपत्तव्यं केनचित् । सत्यां च तदात्मतायां तदेव फलं युज्यते, तावतैव प्रमाणव्यापारपरिसमाप्तेः ।
	\pend
      

	  \pstart अपि च यदि ज्ञानस्य स्वयमर्थरूपपरिच्छेदरूपत्वेनार्थपरिच्छेदकत्वं न स्यात्, किन्तु भिन्नपरिच्छित्तिजनकत्वेन; तदाऽर्थस्य परिच्छित्तेरपरोक्षतैव व्यवस्थापयितुं न शक्येत । तथाहि तदाद्यं ज्ञानमर्थं परिच्छिनत्त्यर्थमपरोक्षयतीति कोऽर्थोऽर्थविषयां परिच्छित्तिं जनयतीति । साऽपि यद्यर्थमपरोक्षयति तदाऽपरां प्रतीतिं जनयतीति स्यात् । एवमुत्तरत्राप्येवमेवेति परिच्छित्तीनामानन्त्यम्, न त्वर्थस्यापरोक्षतेत्यायातमान्ध्यमशेषस्य जगतः ।
	\pend
      

	  \pstart अथ तासामेका स्वयमर्थपरिच्छेदरूपा अर्थापरोक्षतारूपोपेयते; तदा न भिन्नपरिच्छित्तिजनकत्वं परिच्छेदकत्वम् । किन्तर्हि ? स्वयमर्थपरिच्छेदात्मत्वमिति आद्यस्यापि तथात्वमनिवारितम् । तथा च कथं प्रमाणाद् व्यतिरिक्तं फलमिति ? ।
	\pend
      

	  \pstart इह न क्रियैव करणं लोके तयोर्भेदेनावस्थितेः । या चेयं ज्ञानलक्षणा क्रिया सा चेत्फलं
	\pend
      \leavevmode\marginnote{\textenglish{81/dm}}“

	  \pstart यदि तर्हि ज्ञानं प्रमितिरूपत्वात् प्रमाणफलम्, किं तर्हि प्रमाणमित्याह--
	\pend
       “

	  \pstart अर्थसारूप्यमस्य प्रमाणम् ॥ २० ॥
	\pend
      ” 

	  \pstart अर्थेन सह यत् सारूप्यं\footnote{०रूप्यं यत् सादृ० \cite{dp-msB} \cite{dp-msD}} सादृश्यम् अस्य ज्ञानस्य तत् प्रमाणम् । इह यस्माद्विषयाद्\footnote{०याद् ज्ञान० \cite{dp-msA} \cite{dp-msC} \cite{dp-edP} \cite{dp-edH} \cite{dp-edE} \cite{dp-edN}} विज्ञानमुदेति तद्विषयसदृशं तद् भवति । यथा नीलादुत्पद्यमानं नीलसदृशम् । तच्च\footnote{तच्च सादृ० \cite{dp-msA} \cite{dp-edP} \cite{dp-edE}} सारूप्यं सादृश्यम् आकार इत्याभास इत्यपि व्यपदिश्यते ॥
	\pend
       

	  \pstart ननु च ज्ञानादव्यतिरिक्तं सादृश्यम् । तथा च सति तदेव ज्ञानं प्रमाणं तदेव च\footnote{“च” नास्ति \cite{dp-msA} \cite{dp-msB} \cite{dp-msC} \cite{dp-edP} \cite{dp-edH} \cite{dp-edE} \cite{dp-edN}} प्रमाणफलम् । न चैकं वस्तु साध्यं साधनं चोपपद्यते । तत् कथं सारूप्यं प्रमाणमित्याह--
	\pend
       “

	  \pstart तद्वशादर्थप्रतीतिसिद्धेरिति ॥ २१ ॥
	\pend
       
	    
	    \pstart
	    \begin{center}
	  ॥ [[प्रथमः परिच्छेदः \cite{dp-msB} \cite{dp-msC}]]प्रत्यक्षपरिच्छेदः ॥
	    \end{center}
	    \pend
	  
	  ””

	  \pstart किमन्यत्प्रमाणं भविष्यति इत्यागूर्य पूर्वपक्षवादी \textbf{यदीत्या}द्याह । \textbf{यदी}ति सम्भावयति । \textbf{तर्हि}शब्दोऽक्षमायाम् । \textbf{तदेव} प्रत्यक्षं ज्ञानमेवमि\footnote{मेवे}ति न क्षम्यत एतदित्यर्थः । \textbf{प्रमाण}स्य \textbf{फलं} साध्यम् । \textbf{तर्हि} तस्मिन् काले \textbf{किम्प्रमाणमिति} योज्यम् ।
	\pend
      

	  \pstart अत्रा\textbf{र्थसारूप्यमि}त्युत्तरं व्याचष्टे--अ\leavevmode\marginnote{\textenglish{35a/ms}}र्थेनेति । \textbf{अर्थेन} विषयेण । विषयसारूप्यं च \textbf{ज्ञानस्य} प्रत्यक्षाख्यस्य विषयसमानाकारतयोत्पादः । \textbf{इहे}त्यादिना सारूप्यमेवोपपादयति । एतच्च \textbf{नीलसदृशमित्यन्तं} सुबोधम् । केवलमेवं वदतोऽयमाशयः--अनेकप्राग्भावेनोदयमानमपि विज्ञानमर्थस्यैवाकारं बिभर्ति, नान्यस्येत्यनुभवसिद्धमपर्यनुयोज्यम्, सदृशत्वनिश्चयस्य ताद्रूप्येण सततमुदयात् । यदि हि तदन्याकारो विषयः स्यात् तदा तदितरस्याकारधारि विज्ञानं कदाचिज्जनयेत् । यथा शुक्तिः कथञ्चिद्रजताकारज्ञानप्रबन्धोदयेऽपि तद्देशोपसृष्टस्य स्वाकारानुकारि ज्ञानं जनयति । न चाभ्रान्तस्य नीलज्ञानस्य कदाचिदप्यन्याकारत्वमस्ति । तस्मादर्थोऽप्येवमाकार इति निश्चीयते । बाह्यार्थेऽर्थसारूप्यावगमे गतिरियमेवेति ।
	\pend
      

	  \pstart ननु चान्यत्र विषयाभासः प्रमाणमुक्तस्तथाविषयाकारः, इह त्वर्थसारूप्यम् । तत्कथं न व्याघात इत्याशङ्क्याह--\textbf{तच्चेति । चो} यस्मादवधारणे वा । \textbf{इत्यप्य}नेनापि शब्देन । अर्थसारूप्यमेव तेन तेन शब्देनाभिहितम् । ततो न व्याघात इत्यभिप्रायः ॥
	\pend
      

	  \pstart \textbf{किं तर्हि प्रमाणमिति} पृच्छता यच्चेतसि निहितमासीत् तदिदानीं ननु \textbf{चे}त्यादिना कण्ठोक्तं करोति । एतच्च \textbf{प्रमाणमित्येत}दन्तं सुबोधम् ।
	\pend
      \leavevmode\marginnote{\textenglish{82/dm}}“

	  \pstart तद्वशादिति । तदिति सारूप्यम्, तस्य वशात् सारूप्यसामर्थ्यात् । अर्थस्य प्रतीतिः अवबोधः । तस्याः सिद्धिः । \footnote{ततः सिद्धेः \cite{dp-msB}}तत्सिद्धेः कारणात् । अर्थस्य प्रतीतिरूपं प्रत्यक्षं विज्ञानं सारूप्यवशात् सिद्ध्यति प्रतीतं भवतीत्यर्थः । नीलनिर्भासं हि विज्ञानं यतः, तस्मात् नीलस्य प्रतीतिरवसीयते । \unclear{ये}भ्यो हि चक्षुरादिभ्यो\footnote{भ्यो ज्ञानं \cite{dp-msD} \cite{dp-msB}} विज्ञानमुत्पद्यते न तद्वशात् तज्ज्ञानं नीलस्य संवेदनं शक्यतेऽवस्थापयितुम्\footnote{साधारणत्वाच्चक्षुरादीनाम्--\cite{dp-msD-n}} । नीलसदृशं तु अनुभूयमानं नीलस्य संवेदनमवस्थाप्यते ।
	\pend
       

	  \pstart न चात्र जन्यजनकभावनिबन्धनः, साध्यसाधनभावः, येनैकस्मिन् वस्तुनि विरोधः स्यात् । अपि तु व्यवस्थाप्यव्यवस्थापनभावेन\footnote{स्थापकभा० \cite{dp-msA} \cite{dp-msB} \cite{dp-msC} \cite{dp-msD} \cite{dp-edP} \cite{dp-edE} \cite{dp-edH} \cite{dp-edN}} । तत एकस्य वस्तुनः किञ्चिद्रूपं प्रमाणं किञ्चित् प्रमाणफलं न विरुध्यते ।
	\pend
      ”

	  \pstart एतस्मिन् पूर्वपक्षे \textbf{तद्वशादि}त्याद्युत्तरमाचार्यीयं विवृणोति \textbf{तदि}ति । तच्छब्दार्थं \textbf{वश}शब्दार्थं \add{च} स्फुटयति \textbf{सारूप्येति । अर्थस्येति} नीलपीताद्यात्मना विशिष्टस्येति द्रष्टव्यम् । यतः प्रमितिरियं विशिष्टेनैव कर्मणाऽवच्छिन्ना प्रतीयते । अमुमेवार्थम्--\textbf{अर्थस्येत्या}दिना स्पष्टयति । \textbf{सिद्धि}शब्दस्यार्थमभिव्यनक्ति \textbf{प्रतीतमिति । प्रतीतं} तन्नीलस्येदं ज्ञानं न पीतस्येत्याद्याकारेण निश्चितं भवतीत्यर्थः । उक्तमर्थं \textbf{नीले}त्यादिनोपसंहरति । \textbf{हिर}वधारणे । यत एवं तत्तस्मात् नीलस्य प्रतीतिर्नीलस्यैवेदं ज्ञानमित्यवसीयते ।
	\pend
      

	  \pstart ननु जनकेभ्य एवेयं ज्ञानस्य व्यवस्था भविष्यति । ते हि ज्ञानं जनयितुं शक्ताः किमङ्ग निश्चाययितुं न शक्नुयुः ? अतश्च न सारूप्याधीनोऽयं प्रतीतिनियम इत्याशङ्क्याह—येभ्य इति । \textbf{हि}र्यस्मात् । एवं ब्रुवतोऽयं भावः--नेन्द्रियालोकौ नियामकौ । तयोः सर्वज्ञानसाधारणत्वात् । अर्थगतोऽपि विशेषो न नियामकः । ज्ञानादेव हि स प्रत्येतव्यः । तदविशेषे स कथं विशेषव्यवस्थाया अङ्गं भवेदिति । नीलसारूप्येऽपि यद्ययं न्यायस्तदा कथं तस्यापि नियामकत्वमित्याह--\textbf{नीलेति} । तुरिन्द्रियादिभ्यो जनकेभ्यः सारूप्यं भेदवद् दर्शयति । \textbf{नीलसदृशमनुभूयमानं} संवेदनं ज्ञानं \textbf{नीलस्य} नीलस्यैवेति व्यवस्थाप्यते ।
	\pend
      

	  \pstart अयमत्र प्रकरणार्थः--यदि ज्ञानमर्थसरूपं न स्यात् किन्तु निराकारं बोधैकरूपं तदाऽनुभवैकरूपतया तदविशिष्टं, सर्वत्र परिच्छेद्यतया कर्मस्थानप्राप्ते नीलपीतादाविति नीलस्यैवेदं संवेदनम्, इदं पीतस्यैवेत्यनुभवसिद्धः प्रतिकर्मविभागो हीयेत । \leavevmode\marginnote{\textenglish{35b/ms}}अर्थगतश्चाकारो ज्ञानाधीनप्रतिपत्तितया ज्ञानस्य विशिष्टरूपतासन्देहेन सन्दिग्धः । न च तेनैव सन्दिग्धरूपेण तदेव सन्दिग्धं रूपं निश्चेतुं शक्यम् । ज्ञानगतश्च विशेषोऽर्थकृतः सारूप्यादन्यो नोपपद्यते । सन्नपि न तावदिदंतयाऽसौ निर्देष्टुं शक्यः । न चानिरूपितेन तदात्मकः कर्मनियमनिश्चयः । यद्रूपश्च यश्च कर्मनियमनिश्चयो ज्ञानस्य तस्मिन्ननिरूपिते कीदृशी तद्रूपताव्यवस्था ? यथा  \leavevmode\marginnote{\textenglish{83/dm}} “
	  
	व्यवस्थापनहेतुर्हि सारूप्यं तस्य ज्ञानस्य । व्यवस्थाप्यं च नीलसंवेदनरूपम् । 
	  
	व्यवस्थाप्यव्यवस्थापकभावोऽपि कथमेकरय ज्ञानस्येति\footnote{ज्ञानस्य चेत् \cite{dp-msA} \cite{dp-msC} \cite{dp-edP}} चेत् । उच्यते । नीलसदृशमनुभूय\footnote{०ते । सदृशमनुभूयमानं तद्वि० \cite{dp-msA} \cite{dp-msB} \cite{dp-edP} \cite{dp-edH} \cite{dp-edE} \cite{dp-edN} ०ते । सदृशमनुभूय तद्वि० \cite{dp-msC} \cite{dp-msD}} तद्विज्ञानं यतो नीलस्य ग्राहकमवस्थाप्यते निश्चयप्रत्ययेन, तस्मात् सारूप्यमनुभूतं व्यवस्थापनहेतुः । निश्चयप्रत्ययेन च तज्ज्ञानं नीलसंवेदनमवस्थाप्यमानं व्यवस्थाप्यम् । 
	  
	तस्मादसारूप्यव्यावृत्त्या सारूप्यं ज्ञानस्य व्यवस्थापनहेतुः । अनीलबोधव्यावृत्त्या च नीलबोधरूपत्वं व्यवस्थाप्यम् ।” नीलादिरनिरूपिते क्षणिकत्वे तद्रूपो न निरूप्यत इति । तस्माद् यत इयमधिगतिरव्यवधानात् तत्त्वं प्रतिलभते तदेवान्येनाव्यवधीयमानव्यापारं स्वभेदेन भेदकं प्रमाकरणं प्रमाणम् । न पुनरनेनैव व्यवधीयमानव्यापारमिति । तच्चेदृशं सारूप्यमेवेति । एष च वादोऽस्माभिर्विस्तरेण \textbf{विशेषाख्यानेऽभि}हित इति ततोऽप्यपेक्षितव्य इति ।
	\pend
      

	  \pstart ननु चैवमप्येकस्यैव ज्ञानस्य साध्यत्वं साधनत्वञ्च कथमुपपद्यते ? नहि परशुरेव च्छिदा भवति । ततः प्रमाणफलयोरभेदो नाभ्युपैतव्य इत्याशङ्क्याह--\textbf{न चे}त्यादि । \textbf{चो} यस्मादर्थे । \textbf{अत्रे}ति प्रमाणफलचिन्तायां \textbf{विरोधोऽ}नुपपत्तिः स्यात् । यद्येवं न भवति कथं नामेत्याह--\textbf{अपि तु} किन्तु । \textbf{व्यवस्थाप्यं} विशेषरूपेण नियाम्यम् । व्यवस्थाप्यते विशिष्टेनात्मना नियम्यतेऽनेनेति व्यवस्थानिमित्तम् व्यवस्थापनमभिप्रेतम् । \textbf{व्यवस्थापनभावेने}त्ययं पाठो वक्ष्यमाणविरोधी । यदा तु \textbf{व्यवस्थापकभावेने}ति पाठो दृश्यते तदा करणे कर्त्तृभावविवक्षया तथा द्रष्टव्यम् । साध्वसिश्च्छिनत्तीति यथा । तयो\textbf{र्व्यवस्थाप्यव्यवस्थापनयोर्भा}वस्तथाज्ञानाभिधाननिमित्तं रूपं तेन । इत्थंभूतलक्षणा चेयं तृतीया । यत एवं \textbf{तत}स्तस्मा\textbf{देकस्य वस्तुनः} प्रत्यक्षलक्षणस्य ज्ञानस्य वस्तुनः \textbf{किञ्चिद्रूपं} व्यावृत्तिपरिकल्पितं कृतकत्वादिवत् । तत्रेयं ब्राह्यार्थे प्रमाणादिव्यवस्था । ग्राह्याकारोऽर्थसारूप्याख्य आत्मनः संविदमर्थसंविदमादर्शयन्नर्थे प्रमाणम् । अत एव बाह्योऽर्थः प्रमेयः । वस्तुतः स्वविदपीयमर्थसम्बन्धिनी व्यवसीयमानाऽर्थसंवित्तिः फलमिति ।
	\pend
      

	  \pstart ननु किमत्र व्यवस्थापननिमित्तम्, किञ्च व्यवस्थाप्यं येनैतत् स्यादित्याह--\textbf{व्यवस्थापनेति । व्यवस्थापनं} व्यवस्थाकारणम् । व्यवस्थायां प्रयोजकव्यापार इति यावत् । तस्य हेतुर्निमित्तम् । हिर्यस्मात् । \textbf{तस्य} प्रत्यक्षज्ञानस्य । किन्तर्हि व्यवस्थाप्यमिति ? \textbf{चो व्यवस्था}पनहेतोर्व्यवस्थाप्यं भिनत्ति । नीलस्येदं नान्यस्येत्यनेनाकारेण य\textbf{न्नीलसंवेदनं} त\textbf{द्रूपम् ।}
	\pend
      

	  \pstart एतदसहमानो \textbf{व्यवस्थाप्ये}त्याद्याह । न केवलमेकस्य जन्यजनकभावोऽनुपपन्न इत्यपिना दर्शयति । \textbf{उच्यत} इत्यादिना परिहरति । \textbf{यतो} यस्मा\textbf{न्नीलसदृशमनुभूये}ति वास्तवं रूपमनूदितम् । न तु नीलसदृशमनुभवामीति निश्चयोऽस्ति । अपि तु नीलमेवानुभवामीति \textbf{नीलस्य ग्राहकमवस्थाप्यते} ।
	\pend
      \leavevmode\marginnote{\textenglish{84/dm}}“

	  \pstart व्यवस्थापकश्च विकल्पप्रत्ययः प्रत्यक्षबलोत्पन्नो द्रष्टव्यः । न तु\footnote{न च निर्वि० \cite{dp-msC}} निर्विकल्पकत्वात् प्रत्यक्षमेव नीलबोधरूपत्वेना\footnote{रूपत्वं नात्मा० \cite{dp-msC}}त्मानमवस्थापयितुं शक्नोति । निश्चयप्रत्ययेनाव्यवस्थापितं सदपि नीलबोधरूपं विज्ञानमसत्कल्पमेव । तस्मान्निश्चयेन नीलबोधरूपं व्यवस्थापितं विज्ञानं नीलबोधात्मना सद् भवति ।
	\pend
       

	  \pstart तस्मादध्यवसायं कुर्वदेव प्रत्यक्षं प्रमाणं भवति । अकृते त्वध्यवसाये नीलबोधरूपत्वेनाव्यवस्थापितं भवति विज्ञानम् । तथा च प्रमाणफलमर्थाधिगम\footnote{०रूपत्वम् \cite{dp-msA} \cite{dp-msC} \cite{dp-msD} \cite{dp-edP} \cite{dp-edH} \cite{dp-edE}}रूपमनिष्पन्नम् । अतः साधकतमत्वाभावात् प्रमाणमेव न स्याज्ज्ञानम् ।
	\pend
      ”

	  \pstart ननु न तावता ज्ञानमात्मनैवात्मानं\footnote{“इतिकरणेनाभ्युपगमस्य स्वरूपं चेत्यनेनाभ्युपगमं दर्शयति” । इति पाठोऽष्टमपंक्तिसम्बद्धत्वेन पक्तिबाह्यभागे लिखितः प्रतौ दृश्यते किन्तु तस्य कुत्र निवेश इति न ज्ञायते--सं०} तथा\leavevmode\marginnote{\textenglish{36a/ms}}व्यवस्थापयति, अनिश्चयात्मकत्वात् । न च निश्चयोऽप्यात्मानं तथाऽवस्थापयितुं पर्यवाप्नोति, स्वात्मन्यविकल्पकत्वात् । तत्केन तथा व्यवस्थाप्यत इत्याह--\textbf{निश्चयप्रत्ययेने}ति । निश्चयात्मकज्ञानेनोत्तरकालभाविना । तेनाप्यनुरूपेणेति द्रष्टव्यम् । \textbf{तस्मात्सारूप्यमनुभूतं} स्वसंवेदनेन प्रतीतं \textbf{व्यवस्थापनहेतु}र्नीलस्येदं संवेदनम्, न पीतस्येति नियमकरणस्य हेतुर्निमित्तम् । इदं च व्यवस्थापनहेतुर्हि \textbf{सारूप्यं तस्य ज्ञान}स्येत्यस्य समर्थनम् । यदि ज्ञानस्य तस्य सारूप्यमेवं व्यवस्थापनहेतुस्तर्हि किं कीदृशं च सद् व्यवस्थाप्यमित्याह--\textbf{निश्चयेति । चो}ऽवधारणे । \textbf{तदि}त्यस्मात्परो द्रष्टव्यः । यस्यैव नीलज्ञानस्य सारूप्यं तथोक्तं तदेव ज्ञानं व्यवस्थाप्यं नीलस्यैवेत्याकारेण नियाम्यमित्यर्थः ।
	\pend
      

	  \pstart कीदृशं सत्तथा व्यवस्थाप्यमित्याह--\textbf{नीले}ति । पीतादिसंवेदनव्यावृत्त्या नीलस्येदं संवेदनं ज्ञानमि\textbf{त्यवस्थाप्यमानं} निश्चीयमानम् । केनावस्थाप्यमानमित्याकाङ्क्षायां \textbf{निश्चयप्रत्ययेने}ति योजनीयम् । ननु किं ज्ञानात् सारूप्यं व्यतिरिच्यते येन सारूप्यं तथोक्तम्, ज्ञानं त्वेवमुच्यत इत्याशङ्क्योपसंहारव्याजेनाह--\textbf{तस्मादि}ति । यस्मात् सारूप्यादन्यव्यवस्थापनहेतुर्न घटते, सारूप्यं च ज्ञानादन्यं नोपपद्यते \textbf{तस्मा}त्कारणात् । अयं च--\textbf{तत एकस्य वस्तुनः किञ्चिद्रूपमि}त्यादेरुपपत्त्या प्रसाधितस्योपसंहारः ।
	\pend
      

	  \pstart अयं प्रकरणार्थः । एकस्यैव ज्ञानस्य व्यावृत्तिकृतं भेदमाश्रित्याह व्यवस्थाप्यव्यस्थापनभावः । वास्तवं चाभेदमुपादाय प्रमाणफलयोरभेद उच्यते । न च काचित्क्षतिरिति ।
	\pend
      

	  \pstart सम्प्रति \textbf{निश्चयप्रत्यये}नेति ब्रुवता यादृशो निश्चयो विवक्षितस्तं \textbf{व्यवस्थापक} इत्यादिनोपसंहारव्याजेन स्पष्टयति । व्यवस्थापयतीति \textbf{व्यवस्थाप}कः । स च तद्बलोत्पन्नोऽप्यनुरूपो द्रष्टव्यः । अननुरूपविकल्पेन व्यवस्थापितयोरपि व्यस्वथाप्यव्यवस्थापनयोरनुपपत्तेः । यथा मरीचीर्दृष्ट्वा तद्बलोत्पन्नेन विकल्पेनावस्थाप्यमानयोर्जलसारूप्यज्ञानयोर्न तथाभावः ।  \leavevmode\marginnote{\textenglish{85/dm}} “
	  
	जनितेन त्वध्यवसायेन सारूप्यवशान्नीलबोधरूपे ज्ञाने \footnote{ज्ञानेऽवस्था० \cite{dp-msA} \cite{dp-edP} \cite{dp-edH} \cite{dp-edE} \cite{dp-edN}}व्यवस्थाप्यमाने सारूप्यं व्यवस्थापनहेतुत्वात् प्रमाणं सिद्धं भवति । 
	  
	यद्येवमध्यवसायसहितमेव प्रत्यक्षं प्रमाणं स्यात् न केवलमिति चेत् । नैतदेवम् । यस्मात् प्रत्यक्षबलोत्पन्नेनाध्यवसायेन \footnote{दृष्टत्वेना० \cite{dp-msA} \cite{dp-msB} \cite{dp-msC} \cite{dp-msD} \cite{dp-edP} \cite{dp-edH} \cite{dp-edE} \cite{dp-edN}}दृश्यत्वेनार्थोऽवसीयते नोत्प्रेक्षितत्वेन । दर्शनञ्चार्थसाक्षात्करणाख्यं प्रत्यक्षव्यापारः । उत्प्रेक्षणं तु\footnote{०क्षणं विक० \cite{dp-msA} \cite{dp-msC} \cite{dp-msD}} विकल्पव्यापारः । तथाहि परोक्षमर्थं” येनाभिप्रायेण निश्चयप्रत्ययस्य व्यवस्थापकत्वमुक्तं तमभिव्यनक्ति--\textbf{न त्वि}ति । \textbf{तु}र्निश्चयात्प्रत्यक्षं भेदवद् दर्शयति । अवस्थापनाऽशक्तौ हेतुमाह--\textbf{निर्विकल्पकत्वा}दिति ।
	\pend
      

	  \pstart ननु परमार्थतो यदि तज्ज्ञानं नीलमेव रूपतया सरूपं तदा तथानिश्चयो भवतु वा मा वा । स्वयमेवाविकल्पकत्वेऽपि तेन रूपेण सद्व्यवहारगोचरो भविष्यतीत्याशङ्क्याह—\textbf{निश्चयप्रत्ययेने}त्यादि । निश्चयप्रत्ययेनानुरूपेण । अननुरूपेणाक्षणिकविकल्पेनावस्थापितस्यापि क्षणिकबोधस्य सद्व्यवहारस्य योग्यत्वात् । तेना\textbf{व्यवस्थापितमसत्कल्पम}सत्तुल्यम् स्वविषये व्यवहारयितुमशक्तत्वा\leavevmode\marginnote{\textenglish{36b/ms}}त् । गच्छत्तृणज्ञानादिवत् ।
	\pend
      

	  \pstart \textbf{तस्मादि}त्यादिनाऽमुमर्थमुपसंहरति । यस्मान्निश्चयेन तथाऽव्यवस्थापितमसता सदृशं भवति \textbf{तस्मात् । नीलबोधात्मना} नीलस्यायं बोध इत्याकारेण \textbf{सद् भवति} सद्व्यवहारयोग्यं भवति । यस्मात्स्वसामर्थ्योत्पन्नेनानुरूपेण निश्चयप्रत्ययेन व्यवस्थापितं तथा भवति नान्यथा \textbf{तस्मा}त्कारणा\textbf{दध्यवसाय}मनुरूपं निश्चयं \textbf{कुर्वत् । एव}कारेणाऽकरणावस्थायाः प्रमाणव्यवहारं निरस्यति । अमुमेवार्थं व्यतिरेकमुखेन द्रढयन्नाह--\textbf{अकृते त्वि}ति । \textbf{तुः} करणावस्थां भेदवतीमाह ।
	\pend
      

	  \pstart भवतु तथाऽव्यवस्थापितं किमत इत्याह--\textbf{तथा चे}ति । तस्मिंश्च तेनात्मनाऽव्यवस्थापनप्रकारे सति । तदनिष्पत्तावपि किं न प्रमाणमित्याह--\textbf{अत} इति । \textbf{अतो}ऽधिगतिक्रियायाः फलभूताया अनिष्पत्तेः । \textbf{साधकतमत्वाभाव} इति वाक्यभेदः । तस्मात्सा\textbf{धकतमत्वाभावात्} । अर्थाधिगमलक्षणप्रमाणसिद्धौ हि साधकतमं प्रमाणमुच्यते । सा चेन्न निष्पन्ना तर्हि किमपेक्ष्य साधकतमत्वमात्मसात्कुर्यात्, येन तज्ज्ञानं प्रामाण्यमश्नुवीतेति समुदायार्थः ।
	\pend
      

	  \pstart ननु अवसायाभावे तावदियं गतिस्तद्भावेऽपि यद्येषैव गतिस्तदा न प्रत्यक्षं नाम प्रमाणमित्याशङ्क्यान्वयमुखेनेदानीमाह--\textbf{जनिते}नेति । \textbf{तु}रजननावस्थाया जननावस्थां भेदवतीं दर्शयति । तेन तथाव्यवस्थाप्यमाने व्यवस्थापकत्वे च निमित्तमाह--सा\textbf{रूप्यवशादिति । सिद्धं} प्रतीतं भवति ।
	\pend
      

	  \pstart यदि प्रत्यक्षं केवलमसहायं प्रवर्त्तयितुमनीशानं नियमेन निश्चयमपेक्षते तर्हि तत्सहितमेव प्रमाणं प्रसज्येतेत्यभिप्रायवान् प्राह--\textbf{यद्येव}मिति । \textbf{एव}मनन्तरोक्तं \textbf{यद्यभ्यु}पगम्यते तदैवं स्यात् । \textbf{चेदि}ति पराभ्युपगमं दर्शयति । नेति प्रतिषेधति । एवं सत्येतन्न भवति । हेतुमाह—  \leavevmode\marginnote{\textenglish{86/dm}} “
	  
	विकल्पयन्त उत्प्रेक्षामहे न तु पश्याम इति उत्प्रेक्षात्मकं विकल्पव्यापारमनुभवादध्यवस्यन्ति\footnote{भवादवस्यन्ति \cite{dp-msA} \cite{dp-msC} \cite{dp-edP} \cite{dp-edH} \cite{dp-edE} \cite{dp-edN}} । तस्मात् स्वव्यापारं तिरस्कृत्य प्रत्यक्षव्यापारमादर्शयति यत्रार्थे प्रत्यक्षपूर्वकोऽध्यवसायस्तत्र प्रत्यक्षं केवलमेव प्रमाणमिति ॥ 
	  
	॥ \footnote{प्रमाणमिति न्यायबिन्दुटीकायां प्रथमः परिच्छेदः समाप्तः ॥ मङ्गलमस्तु ॥ \cite{dp-msA} प्रमाणमिति आचार्यधर्मोत्तरविरचितायां न्यायबिन्दुटीकायां प्रथमः परिच्छेदः समाप्तः । \cite{dp-msC}}आचार्यधर्मोत्तरविरचितायां न्यायबिन्दुटीकायां प्रत्यक्षपरिच्छेदः प्रथमः ॥” \textbf{यस्मादि}ति । \textbf{दृश्यत्वेन} दर्शनविशिष्टत्वेन । प्रकारान्तरं निरस्यति \textbf{नेति} । उत्प्रेक्षितत्वं साक्षात्करणाभिमानशून्यज्ञातृत्वम् ।
	\pend
      

	  \pstart ननु विकल्पव्यापारो दर्शनम् । ततः स्वव्यापारविशिष्टार्थाध्यवसाने तस्य का क्षतिरित्याशङ्क्याह--\textbf{दर्शनञ्चे}ति । \textbf{चो} यस्मादर्थे । यद्येवं कस्तर्हि विकल्पव्यापार इत्याह—\textbf{उत्प्रेक्षण}मिति ।
	\pend
      

	  \pstart एतदेव \textbf{तथा}हीत्यादिनाऽ\textbf{वस्यन्तीत्यनेन\footnote{त्यन्तेन}} ग्रन्थेन समर्थयति ।
	\pend
      

	  \pstart भवत्वेवं तथापि कथं प्रत्यक्षस्य केवलस्य प्रामाण्यं न तु विकल्पस्यापीत्याशङ्क्योपसंहारापदेशेनाह--\textbf{तस्मादि}ति । एतच्च सुबोधम् ।
	\pend
      

	  \pstart एवमभिधाने चायमस्याशयो बोद्धव्यः । स्यात्खलु विकल्पस्यापि प्रामाण्यं यद्यसौ स्वव्यापारयुक्त एव प्रत्यक्षस्य साहाय्यं भजते । न चायं तथा व्याप्रियते । स्वीकृतप्रत्यक्षव्यापारस्यैव प्रवृत्तिदर्शनात् । स च दर्शनलक्षणो व्यापारः प्रत्यक्षेणैव सम्पादित इति कथमयमपि पिष्टपेषणकारी प्रामाण्यं प्रतिलभेत ।
	\pend
      

	  \pstart \textbf{अथ} प्रत्यक्षं तावत्स्वप्रामाण्यव्यव\leavevmode\marginnote{\textenglish{37a/ms}}हाराय तमपेक्षते । ततः सोऽपि प्रमाणमेवेति मतिस्तर्हि पिता पितृव्यवहारायावश्यं पुत्रमपेक्षत इति पुत्रोऽपि पिता प्रसज्यत इति कृतमतार्किकवचनविचारणयेति सर्वमवदातम् ॥
	\pend
      

	  \pstart ॥ पण्डित\textbf{दुर्वेक}विरचित\textbf{धर्मोत्तरनिबन्धस्य धर्मोत्तरप्रदीपसंज्ञितस्य} प्रथमः परिच्छेदः ॥
	\pend
      
	    
	    \endnumbering% ending numbering from div
	    \endgroup
	    
	  
	  
	% new div opening: depth here is 0
	
	    
	    \begingroup
	    \beginnumbering% beginning numbering from div depth=0
	    
	  
\chapter[{द्वितीयः स्वार्थानुमानपरिच्छेदः ।}]{द्वितीयः स्वार्थानुमानपरिच्छेदः ।}\leavevmode\marginnote{\textenglish{87/dm}}“

	  \pstart एवं प्रत्यक्षं व्याख्यायानुमानं व्याख्यातुकाम\footnote{व्याख्यातुमाह--\cite{dp-msA}} आह--
	\pend
       “

	  \pstart अनुमानं द्विधा ॥ १ ॥
	\pend
      ” 

	  \pstart अनुमानं द्विधा द्विप्रकारम् । अथानुमानलक्षणे वक्तव्ये किमकस्मात् प्रकारभेदः कथ्यते ? उच्यते । परार्थानुमानं शब्दात्मकम्, स्वार्थानुमानं तु ज्ञानात्मकम् । तयोरत्यन्तभेदात् नैकं लक्षणमस्ति । ततस्तयोः प्रतिनियतं लक्षणमाख्यातुं प्रकारभेदः कथ्यते । प्रकारभेदो हि व्यक्तिभेदः । व्यक्तिभेदे च कथिते प्रतिव्यक्तिनियतं लक्षणं शक्यते वक्तुम् । नान्यथा । ततो लक्षणनिर्देशाङ्ग\footnote{निर्देशार्थमेव \cite{dp-msA} \cite{dp-msC}}मेव प्रकारभेदकथनम् । अशक्यतां च प्रकारभेदकथनमन्तरेण लक्षणनिर्देशस्य ज्ञात्वा प्राक्\footnote{ज्ञात्वा प्रथमं प्रका० \cite{dp-msC} \cite{dp-msD}} प्रकारभेदः कथ्यत इति ॥
	\pend
      ”

	  \pstart प्रत्यक्षानुमानभेदेन द्वैधं प्रमाणमुद्दिष्टम् । तत्र व्याख्यातं प्रत्यक्षम् । यथोद्देशमधुनाऽनुमानं व्याख्यातुमवसरप्राप्तमित्यभिसन्धायाह--एवमिति । एवमनन्तरोक्तेन चतुर्विधविप्रतिपत्तिनिराकरणप्रकारेण । प्रकारे धाप्रत्ययोऽयमिति दर्शयन्नाह--\textbf{द्विप्रकारमिति ।}
	\pend
      

	  \pstart ननु चानुमानस्य लक्षणं वक्तुकामेनास्य लक्षणमेव वक्तव्यम् । तत्किमिदमप्रस्तुताभिधानमास्थीयत इति पूर्वपक्षम्--\textbf{अथे}त्यादिनोत्थापयति । \textbf{अथ}शब्दोऽत्र प्रश्ने । \textbf{अकस्मादि}ति निपातो निर्निमित्तवचनः । \textbf{उच्यत} इत्यादिना परिहरति । \textbf{तयो}र्ज्ञानाभिधानात्मनोः । एकमिति साधारणम् । यथा चतुर्णामपि प्रत्यक्षाणां ज्ञानरूपत्वादेकं कल्पनापोढत्वादिसाधारणं लक्षणं सम्भवति, तथा यदि स्यात् प्रत्यक्षवल्लक्षणमेव प्रथममुक्तं स्यादिति भावः । \textbf{प्रतिनियतं} प्रातिस्विकम् । \textbf{प्रका}रस्य \textbf{भेदो} नानात्वम् ।
	\pend
      

	  \pstart ननु प्रतिव्यक्तिनियतं लक्षणं व्यक्तिविशेषोपदर्शनं विना न शक्यते दर्शयितुमिति व्यक्तिभेद एव कथयितव्यः । तत्किं प्रकारभेदः कथ्यत इत्याह--\textbf{प्रका}रेति । \textbf{हि}र्यस्मात् । यदि तस्मिन् दर्शितेऽपि प्रतिनियतलक्षणाख्यानं न शक्यं तर्हि किं तेन कथितेनेत्याह--\textbf{व्यक्तीति । चो} यस्मादर्थे । \textbf{प्रति}शब्दोऽत्र नियतार्थवृत्तिः, तेन \textbf{प्रति} विशिष्टा \textbf{व्यक्ति}स्तत्र \textbf{नियत}मिति,  \leavevmode\marginnote{\textenglish{88/dm}} “
	  
	किं पुनस्तद् द्वैविध्यमित्याह-- “
	  
	स्वार्थं परार्थं च ॥ २ ॥” 
	  
	स्वस्मायिदं स्वार्थम् । येन स्वयं प्रतिपद्यते तत् स्वार्थम् । परस्मायिदं परार्थम् । येन परं प्रतिपादयति तत् परार्थम् ॥” सप्तमी\href{http://http://sarit.indology.info/?cref=Pā.2.1.40}{पाणिनि २-१-४०} इति योगविभागात्समासः । यद्वा नियतं विशि\textbf{ष्टं} लक्षणं न शक्यं वक्तुम् । क्व च नियतमित्याशङ्क्योक्तं--\textbf{प्रतिव्यक्ती}ति । व्यक्तौ व्यक्तावित्यव्ययीभावः । यस्मादन्यथा प्रतिनियतलक्षणाख्यानस्याशक्यत्वं \textbf{तत}स्तस्मात् । \textbf{लक्षणनिर्देशाङ्गमे}वेति लक्षणनिर्देशनिमित्तमेव ।
	\pend
      

	  \pstart एतेन यच्चोद्यते “लक्षणमात्रे कथिते विशिष्टलक्षणमनुमानमेकमनेकं वाऽस्तु । किं तस्य प्रकारभेदकथनेन” इति तत्परिहृतम् । यदि हि साधारणं लक्षणमभिप्रेत्येदमुच्यते तदा तन्नास्तीति किं कथ्येत । अथ विशिष्टं लक्षणं तदपि व्यक्तिभेदकथनमन्तरेण वक्तुं यदि शक्येत किं न कथ्येत । केवलमिदमेव नास्तीति । अत एवादावेव तदभिधानं न्याय्यम्, न तु पश्चादिति दर्शयितुमाह--\textbf{अशक्यताम्} इति । तद\textbf{न्तरेण लक्षणनिर्देशस्याशक्यतां ज्ञात्वा । चो}ऽवधारणे । \textbf{प्रा}ग्लक्षणकथनात्पूर्वम् ।
	\pend
      

	  \pstart स्यादेतत्--स्वार्थानुमानमेवंलक्षणं परार्थानुमानमेवंलक्षणमिति किं विशिष्टं लक्षणं न शक्यते वक्तुम् ? एवमपि किमनुमानद्वैतं नावेदितं भवति येन ससंख्येया संख्या--\textbf{अनुमानं द्विधा स्वार्थं परार्थं चे}त्युच्यत इति ? सत्यमेतत् । केवलं नियमार्थमेतद् विभागवचनमिति ब्रूमः । \textbf{अनुमानं द्विधा}--द्विधैवैवमात्मकमिति कथं नाम \leavevmode\marginnote{\textenglish{37b/ms}} प्रतीयेतेति । इतरथेह तावदेतावदेव व्युत्पाद्यतया प्रस्तुतम्, अन्यत्र पुनरन्यदप्यनुमानं व्युत्पाद्यमस्तीत्याशङ्का नाहत्य निराकृता स्यादिति ॥
	\pend
      

	  \pstart पूर्ववच्छेषवदादिरूपेण अन्यथाऽपि द्वैविध्यसम्भवात् संशयानः पृच्छति--\textbf{किं पुनरि}ति । \textbf{किमि}ति सामान्यात् पृच्छति । \textbf{पुन}रिति विशेषतः ।
	\pend
      

	  \pstart स्वार्थशब्दस्य विग्रहं दर्शयति--\textbf{स्वस्मायि}ति । \textbf{अर्थ}शब्देन नित्यसमासादस्य पदविग्रहमाह । \textbf{इद}मित्यनुमानन् । \textbf{स्वार्थमि}ति समस्तपदनिर्देश एषः । अस्य चात्मप्रतिपत्तिः प्रयोजनमित्यर्थः । अमुमेवार्यं स्फुटयन्नाह--\textbf{येने}ति । \textbf{येना}नुमानेन करणभूतेनानुमाता \textbf{स्वयं प्रतिपद्यते} परोक्षमर्थमिति शेषः, प्रकरणलभ्यं वा । तत्स्वार्थज्ञानमात्मप्रतिपत्तिप्रयोजनमिति यावत् ।
	\pend
      

	  \pstart अयमाशयः--त्रिरूपलिङ्गस्य ज्ञानं यस्य सन्तान उत्पद्यते तत्तदर्थमेव । तेनाऽन्यस्याप्रतिपत्तेः । ततः स्वार्थमुच्यते । न तु किञ्चिज्ज्ञानं क्वचित्पुंसि नियतमस्ति । यदपेक्षया स्वार्थमुच्येत । \textbf{येन स्वयं प्रतिपद्यत} इति ब्रुवतश्चायमभिप्रायः । यद्यपि प्रतिपत्तिरनुमानज्ञानात्मिका तथाप्येकस्यापि व्यवस्थाप्यव्यवस्थापनभावेन क्रियाकरणभेदो दर्शित इति सारूप्यभागः करणमनुमानम्, अधिगमरूपा फलावस्था प्रतिपत्तिरिति ।
	\pend
      \leavevmode\marginnote{\textenglish{89/dm}}“

	  \pstart तत्र\footnote{तत्र त्रिरू० \cite{dp-edE}} स्वार्थं त्रिरूपाल्लिङ्गाद् यदनुमेये\footnote{०मेयज्ञानं \cite{dp-msC}} ज्ञानं तदनुमानम् ॥ ३ ॥
	\pend
      ”“

	  \pstart तत्र तयोः स्वार्थपरार्थानुमानयोर्मध्ये स्वार्थं ज्ञानं किंविशिष्टमित्याह--त्रिरूपादिति । त्रीणि रूपाणि यस्य वक्ष्यमाण\footnote{वक्ष्यमाणानि \cite{dp-msD}}लक्षणानि तत् त्रिरूपम् । लिङ्ग्यते गम्यतेऽनेनाऽर्थ इति
	\pend
      ”

	  \pstart परार्थमित्यस्य विग्रहमाह--\textbf{परस्मायि}ति । पूर्ववदस्य पदविग्रहः । \textbf{परार्थमिति} समस्तं पदमुक्तम् । अस्य च परप्रतिपत्तिः प्रयोजनमित्यर्थः । अमुमर्थं \textbf{येने}त्यादिना स्पष्टयति । \textbf{येन} वाक्येन करणेन \textbf{परं} प्रति वाच्यं \textbf{प्रतिपादयति} परोक्षमर्थं बोधयति \textbf{तत्} त्रिरूपलिङ्गाख्यानं वाक्यं \textbf{परार्थम}नुमानम् ।
	\pend
      

	  \pstart अत्राप्ययमस्याभिप्रायः--यद्यपि अभिधानरूपमप्यनुमानं न नियतं पुंसि तथाऽपि तत्परार्थमेव । तथाहि यद् यदुद्दिश्य प्रवर्त्तते तत् तदर्थमुच्यते । परमुद्दिश्य प्रवर्त्तते च शब्दो नात्मानम् । अतो नानवस्थितपारार्थ्यः शब्दः । प्रयोक्तृसंमीहाविषयस्यार्थस्य पर एव प्रयोजको यस्मादिति । परोक्षार्थप्रतिपत्तिफलत्वेन पारम्पर्येणाविशिष्टविषयत्वेऽपि स्वार्थादस्य पृथग्वचनम्, साक्षादनयोर्व्यापारभेदादिति च द्रष्टव्यम् ।
	\pend
      

	  \pstart ननु च परार्थानुमानोत्पादकवाक्यवदस्ति किञ्चिद् वाक्यं यत्परप्रत्यक्षोपयोगि । यथा “एष कलभो धावति” इति वाक्यम् । अतः परार्थानुमानवत्परार्थं प्रत्यक्षं किं न व्युत्पाद्यत इति ? अत्रोच्यते--परोक्षार्थप्रतिपत्तेर्या सामग्री--लिङ्गस्य पक्षधर्मता साध्यव्याप्तिश्च--तदाख्यानाद् वाक्यमुपचारतः परार्थानुमानमुच्यते । न तु तत्र कथञ्चिदङ्गभावमात्रेण, स्वास्थ्यादेरपि तथा प्रसङ्गात् । इदं पुनः “अयं कलभः” इत्यादिवाक्यं न प्रत्यक्षोत्पत्तेर्या सामग्रीन्द्रियालोकादि तदभिधानात्तन्निमित्तं भवत्तथा व्यपदेशमश्नुते येन व्युत्पाद्यतामप्यश्नुवीत । किं तर्हि ? कस्यचिद् दिदृक्षामात्रजननेन । यथा कथञ्चित्परप्रत्यक्षोत्पत्तावङ्गभावमात्रेण ताद्रूप्ये नेत्रोत्सवे वस्तुनि सन्निहितेऽपि कथञ्चित्पराङ्मुखस्य परेण यदभिमुखीकरणं \leavevmode\marginnote{\textenglish{38a/ms}} शिरसस्तदपि वचनात्मकं परार्थप्रत्यक्षं व्युत्पादयितुर्व्युत्पाद्यमापद्येत । एतच्च कः स्वस्थात्मा मनसि निवेशयेत् । किञ्च भवतु तथाविधं वचनं परार्थं प्रत्यक्षम् । किं नश्छिन्नम् ? तस्यापि व्युत्पादनार्हस्याव्युत्पादनात्प्रमाद एव महती क्षतिरिति चेत् । न तथारूपस्य व्युत्पादनम्, अविप्रतिपत्तेः । विप्रतिपत्तिनिराकरणेन हि स्वरूपप्रतिपादनं व्युत्पादनम् । न तु केचित् तथाविधे वचने परार्थप्रत्यक्षोपयोगिनि विप्रतिपद्यन्ते । येन तदपि व्युत्पाद्येत । परार्थानुमाने\footnote{अस्पष्टम्--सं०}\add{... ... ...} वस्तु प्रतिपद्यमाना अपि तद्धर्मव्याप्तिव्यतिरेकाभ्यां निगदन्तो दृष्टाः, अविनाभावावचनात्, उपनयसाध्यतदावृत्तिवचनानाञ्च प्रयोगादिति तद् व्युत्पाद्यते । यदि तु तत्रापि न विप्रतिपद्येरन् परे तदा तदपि नैव व्युत्पादितं स्याद्, इत्यलमतिविस्तरेण ॥
	\pend
      

	  \pstart इह यथैव स्वयं प्रतिपन्नः परोक्षार्थस्तथैव परस्मै प्रतिपाद्यत इति स्वार्थानुमानपूर्वकत्वात्परार्थानुमानस्य प्रथमं स्वार्थानुमानमुक्तम् । यथोद्देशमेव च लक्षणं प्रणेयमिति स्वार्थानुमान  \leavevmode\marginnote{\textenglish{90/dm}} “
	  
	लिङ्गम् । तस्मात् त्रिरूपाल्लिङ्गात् यत् जातं ज्ञानम् इति । एतद् \footnote{त्रिरूपलिङ्गजं ज्ञानमित्यर्थः । हेतुः कारणम्--\cite{dp-msD-n} ।}हेतुद्वारेण विशेषणम् । तत् \footnote{०षणम् । त्रिरूपा० \cite{dp-msC} \cite{dp-msD}}त्रिरूपाच्च लिङ्गात् त्रिरूपलिङ्गालम्बनमप्युत्पद्यत इति विशिनष्टि--अनुमेय इति । एतच्च विषयद्वारेण विशे षणम् । 
	  
	त्रिरूपाल्लिङ्गाद्यदुत्पन्नमनुमेयालम्बनं ज्ञानं तत् \footnote{स्वार्थानु० \cite{dp-msD}}स्वार्थमनुमानमिति ॥ 
	  
	लक्षणविप्रतिपत्तिं निराकृत्य फलविप्रतिपत्तिं निराकर्त्तुमाह-- “
	  
	प्रमाणफलव्यवस्थाऽत्रापि प्रत्यक्षवत् ॥ ४ ॥” 
	  
	प्रमाणस्य\footnote{प्रमाणफलमिति । प्रमाणस्य यत्--\cite{dp-msB} \cite{dp-msC} \cite{dp-msD}} यत् फलं तस्य या व्यवस्था \footnote{०स्थाऽत्रा० \cite{dp-msA} \cite{dp-msB} \cite{dp-edP} \cite{dp-edH} \cite{dp-edE}}साऽत्रानुमानेऽपि \footnote{०पि प्रत्यक्षवत् प्रत्यक्ष इव \cite{dp-edP} \cite{dp-edH} \cite{dp-edE}}प्रत्यक्ष इव प्रत्यक्षवत् वेदितव्या ।” स्यैवं लक्षणं \textbf{तत्रे}त्यादिनाऽऽदित उपदिष्टमाचार्येण तद् व्याचष्टे \textbf{तत्रे}ति । स्वार्थपरार्थानुमानसमुदायात् स्वार्थानुमानं स्वार्थत्वज्ञात्या निर्धार्यते । \textbf{तस्मात्त्रिरूपाल्लिङ्गाद् यज्जातमिति} व्याचक्षाणो मूले \textbf{त्रिरूपाल्लिङ्गादि}ति या पञ्चमी सा गम्यमानजनिक्रियापेक्षया “जनिकर्त्तुः प्रकृतिः” \href{http://http://sarit.indology.info/?cref=Pā.1.4.30}{पाणिनि १. ४. ३०} इत्यनेन लब्धापादानसंज्ञकादपादान एवेति दर्शयति । \textbf{हेतुद्वारेण} जनकमुखेन । \textbf{त्रिरूपाल्लिङ्गादि}ति चाचक्षाणेनाचार्येणैकद्विपदव्युदासेन षट्पक्षीं प्रतिक्षिप्य सप्तमपक्षपरिग्रहेण लिङ्गस्य लक्षणमभिप्रेतं प्रकाशितमिति । यथा चैतत् तथा \textbf{भट्टार्चटनिबन्धनमर्चटालोकसंज्ञितं} विधास्यन्तो विस्तरेण स्पष्टयिष्यामः ।
	\pend
      

	  \pstart अनुमेयग्रहणस्य व्यावर्त्त्यं दर्शयति--\textbf{त्रिरूपाच्चे}ति । \textbf{चो} यस्मादर्थे । \textbf{इति}र्हेतौ । \textbf{त्रिरूपलिङ्गालम्बन}मिति धूमं दृष्ट्वा सर्वत्रायं वह्निनान्तरीयक इति ज्ञानं वाच्यम् । तद्धि परम्परया त्रिरूपाल्लिङ्गाज्जातमिति । \textbf{इति}ना विशेषणस्य स्वरूपमुक्तम् । विशेषितमेव ज्ञानम् । किम्पुनर्विशिष्यत इत्याह--\textbf{एतच्चे}ति । \textbf{चो} यस्मात् । \textbf{विषयद्वारेणा}वसीयमानविषय\textbf{द्वारेण विशेषणं} व्यवच्छेदकम् ।
	\pend
      

	  \pstart अवयवार्थं व्याख्याय समुदायार्थं \textbf{त्रिरूपे}त्यादिना व्याचष्टे । \textbf{अनुमेयो} धर्मधर्मिसमुदायः आलम्ब्यत इत्या\textbf{लम्बनं} यस्येति विग्रहः । \textbf{इति}र्वाक्यार्थपरिसमाप्तौ एवमर्थः सन्नापरेण सम्बद्ध्यते--एवमुक्तेन प्रकारेण \textbf{लक्षणविप्रतिपत्तिं निराकृत्ये}ति ॥
	\pend
      

	  \pstart ननु च \textbf{प्रमाणस्य फलमि}ति यद्या\textbf{चार्यस्य} विवक्षितं \textbf{धर्मोत्तरेण} चैवं व्याख्यायते तदा प्रमाणभागव्यवस्थायां किमुक्त\textbf{माचार्येण, धर्मोत्तरेणापि} “नीलसारूप्यं व्यवस्थापनहेतुः प्रमाणम्” इत्युपरिष्टात्\href{http://http://sarit.indology.info/?cref=p91}{पृ० ९१} किमिति दर्शयिष्यते इति चेत् । नैष दोषः । नहि \textbf{प्रमाणस्ये}त्यादिना  \leavevmode\marginnote{\textenglish{91/dm}} “
	  
	यथा हि \footnote{नीलस्वरूपं \cite{dp-msC} \cite{dp-msD}}नीलसरूपं प्रत्यक्षमनुभूयमानं नीलबोधरूपमवस्थाप्यते\footnote{०रूपमेवावस्था० \cite{dp-msC}}, तेन नीलसारूप्यं \footnote{०रूप्यं नीलव्यव० \cite{dp-msC}}व्यवस्थापनहेतुः प्रमाणम्, नीलबोधरूपं तु \footnote{०रूपं त्ववस्था० \cite{dp-msD} ०रूपत्वमवस्था० \cite{dp-msC}}व्यवस्थाप्यमानं प्रमाणफलम्; तद्वद् अनुमानं नीलाकारमुत्पद्यमानं नीलबोधरूपमवस्थाप्यते, तेन नीलसारूप्यमस्य\footnote{०प्यं प्रमा० \cite{dp-msC}} प्रमाणम्, नीलविकल्पनरूपं त्वस्य\footnote{०रूपं तु प्रमा० \cite{dp-msB} \cite{dp-msD}} प्रमाणफलम् । सारूप्यवशाद्धि तन्नीलप्रतीतिरूपं सिध्यति । नान्यथेति ॥ 
	  
	एवमिह संख्या-लक्षण-फलविप्रतिपत्तयः । प्रत्यक्षपरिच्छेदे तु गोचरविप्रतिपत्तिर्निराकृता । लक्षणनिर्देशप्रसङ्गेन तु त्रिरूपं लिङ्गं प्रस्तुतम् । तदेव व्याख्यातुमाह-- “
	  
	त्रैरूप्यं पुनर्लिङ्गस्यानुमेये सत्त्वमेव, सपक्ष एव सत्त्वम्, असपक्षे चासत्त्वमेव निश्चितम् ॥ ५ ॥” 
	  
	त्रैरूप्यमित्यादि । लिङ्गस्य यत् त्रैरूप्यं यानि त्रीणि रूपाणि तदिदमुच्यत इति शेषः । किं पुनस्तत्\footnote{पुनस्त्रैरूप्यम्--\cite{dp-msA}} त्रैरूप्यमित्याह--अनुमेयं वक्ष्यमाणलक्षणम् । तस्मिन् लिङ्गस्य सत्त्वमेव निश्चितम्--एकं रूपम् । यद्यपि चात्र निश्चितग्रहणं न कृतं तथापि अन्ते कृतं प्रक्रान्तयोर्द्वयोरपि रूपयोरपेक्षणीयम् । यतो न योग्यतया लिङ्गं परोक्ष\footnote{परोक्षं ज्ञानस्य--\cite{dp-msC}} ज्ञानस्य निमितम् । यथा” मौलस्य \textbf{प्रमाण}शब्दस्य \textbf{फल}शब्देन विग्रहो दर्शितः । किन्त्व\leavevmode\marginnote{\textenglish{38b/ms}}र्थप्रदर्शनं कृतम् । एतच्चोपलक्षणं तेन फलस्य साधनं च यत्तस्यापि या व्यवस्था साऽपि गृह्यते । मूले तु द्वन्द्वसमास एवाभिप्रेतो \textbf{वार्त्तिककार}स्य । पूर्वनिपातविधेश्चानित्यत्वात् न फलशब्दस्य पूर्वनिपातः । अत एव \textbf{विनिश्चयः}--“न प्रमाणफलयोर्विषयभेदः” इति ।\footnote{पङ्क्तिबाह्यं लिखितं न पठ्यते--सं०}...लक्षणगोचरफलविषय इति । \textbf{नीलसारूप्यम्} अस्पष्टनीलसारूप्यम्, अनुमानस्यापरोक्षीकरणाभावात्, विजातीयमात्रव्यावृत्तस्यानुमानेन प्रतीतेः । \textbf{नीलप्रतीतिरूपं} नीलविकल्पनरूपं \textbf{सिद्ध्य}ति निश्चीयते ॥
	\pend
      

	  \pstart ननु सङ्ख्यालक्षणफलविप्रतिपत्तय एवानुमानस्य निराकृता, न तु विषयविप्रतिपत्तिरित्याह--\textbf{एव}मिति । \textbf{एव}मनन्तरोक्तेन प्रकारेण \textbf{इहा}नुमानपरिच्छेदे \textbf{निराकृता} इति शेषः, वक्ष्यमाणं वा \textbf{निराकृते}ति पदं वचनविपरिणामेन सम्बन्धनीयम् ।
	\pend
      

	  \pstart \textbf{त्रैप्रूय}मित्यादिग्रन्थस्य यत उत्थानम्, तत् \textbf{लक्षणे}त्यादिना दर्शयति । \textbf{प्रसङ्गेन} प्रस्तावेन \textbf{यानि त्रीणि रूपाणि} तान्येव तथेत्यावेदयति । \textbf{तदिदं} त्रैरूप्यमिति प्रकृतत्वात् \textbf{शेषोऽ}ध्याहारः । अबाधितविषयत्वाद्यनेकरूपसम्भवे पृच्छति \textbf{किं पुन}रिति । \textbf{किमि}ति सामान्यतः पृच्छति । \textbf{पुनरि}ति विशेषतः । कस्मात् पुनस्तदपेक्षणीयमित्याह--\textbf{यत} इति । परोक्षो योऽर्थस्तस्य यज्ज्ञानं तस्य । \textbf{बीजं} वैधर्म्यदृष्टान्तः । कस्मान्न तथेत्याह--\textbf{अदृष्टादि}ति । \textbf{अप्रतिपत्तेः} परोक्षार्थस्येति प्रकरणात् । यद्यज्ञातं लिङ्गं न परोक्षज्ञाननिमित्तं तर्हि पक्षधर्मतया ज्ञातमेवास्तु परोक्षप्रकाशनं  \leavevmode\marginnote{\textenglish{92/dm}} “
	  
	बीजमङ्कुरस्य । अदृष्टाद् धूमाद् अग्नेरप्रतिपत्तेः । \footnote{ननु च यदा धूमस्वरूपमेव प्रत्यक्षेण ज्ञायते तदा परोक्षस्य निश्चायकं भवति । आह ।--\cite{dp-msD-n}}नापि स्वविषयज्ञानापेक्षं \footnote{परोक्षप्रका० \cite{dp-msC}}परोक्षार्थप्रकाशनं । यथा प्रदीपो घटादेः । \footnote{धूमात्--\cite{dp-msD-n}}दृष्टादप्यनिश्चितसम्बन्धादप्रतिपत्तेः\footnote{अग्नेः--\cite{dp-msD-n}} । तस्मात् \footnote{०क्षार्थाना० \cite{dp-edE}}परोक्षार्थनान्तरीयकतया निश्चयनमेव लिङ्गस्य परोक्षार्थप्रतिपादनव्यापारः । नापरः कश्चित् । अतोऽन्वयव्यतिरेकपक्षधर्मत्वनिश्चयो लिङ्गव्यापारात्मकत्वादवश्यकर्त्तव्य इति सर्वेषु रूपेषु निश्चितग्रहणमपेक्षणीयम् । 
	  
	तत्र सत्त्ववचनेनासिद्धं चाक्षुषत्वादि निरस्तम्\footnote{दि निषिद्धम् \cite{dp-msB} \cite{dp-msD}} । एवकारेण पक्षैकदेशासिद्धो निरस्तः\footnote{०सिद्धो यथा \cite{dp-msA} \cite{dp-edN} \cite{dp-edP} ०सिद्धो निरस्तो हेतुः यथा \cite{dp-msB} \cite{dp-msD} \cite{dp-edE} \cite{dp-edH}} । यथा चेतनास्तरवः स्वापादिति । पक्षीकृतेषु तरुषु पत्रसङ्कोचलक्षणः स्वाप एकदेशे न सिद्धः । न हि सर्वे वृक्षा रात्रौ पत्रसंकोचभाजः किन्तु केचिदेव । सत्त्ववचनस्य” तत् किमन्वयव्यतिरेकनिश्चयेन तस्य ? इत्याशङ्क्याह--\textbf{नापी}ति । \textbf{लिङ्गमि}ति सम्बध्यते । \textbf{प्रदीपोऽपि दृष्टा\add{द}निश्चिताद}पि । न केवलमदृष्टादित्यपिशब्दः । यत एवं \textbf{तस्माद्} हेतोः । \textbf{अन्तरं} व्यवधानम्, न तथा \textbf{नान्तरम्} । “न भ्राड्” \href{http://http://sarit.indology.info/?cref=Pā.6.3.75}{पाणिनि ६. ३. ७१}इत्यादिसूत्रे नेति योगविभागान्नलोपाभावः । नान्तरे भव इति गहादित्वाच्छ । ततः स्वार्थिकः कन् । परोक्षस्य वह्न्यादेर्नान्तरीयकोऽविनाभावी तस्य भावस्तत्ता तया निश्चयनम्--सर्वत्रायमेतदविनाभावीति विकल्पनम् । \textbf{लिङ्गस्य} गमकस्य व्यापृतं रूपं \textbf{व्यापारः} । तथानिश्चयारूढस्यैव रूपस्य लिङ्गत्वात् । \textbf{एव}कारेण व्यवच्छिन्नमेवान्यस्य तद् व्यापाररूपत्वं स्पष्टार्थं \textbf{नापर} इत्यनेनानूदितम् । यत्पुनरत्राप्ययमेतन्नान्तरीयक इति ज्ञानं तदनुमानज्ञानमिति ज्ञेयम् । पूर्वकं तु लिङ्गज्ञानमिति । \textbf{अतः} सत्त्वेनानिश्चिताल्लिङ्गादनिश्चितसम्बन्धाच्चाप्रतिपत्तेः कारणादन्वयादिनिश्चयो\textbf{ऽवश्यकर्त्तव्यः} । तथापि कथं कर्त्तव्य इत्याशङ्क्य व्यतिरेकमुखेणोक्तमेव हेतुमन्वयमुखेनापि दर्शयन्नाह--\textbf{लिङ्गेति । लिङ्गव्यापारस्य} परोक्षप्रतिपादने गमकव्यापारस्यात्मा स एव तथेति स्वार्थिकः कन् कर्त्तव्यः । तस्य भावस्तस्मादिति हेतुपदं कृत्वाऽयमुपसंहारः । \textbf{सर्वेष्व}नुमेयसत्त्वादिषु \textbf{रूपेषु} लक्षणेषु । \leavevmode\marginnote{\textenglish{39a/ms}}
	\pend
      

	  \pstart सम्प्रत्येकैकस्य रूपस्य यद् व्यावर्त्त्य तत्क्रमेण दर्शयितुमाह--\textbf{तत्रे}ति । \textbf{आदि}ग्रहणाद् व्यधिकरणासिद्धविशेषणासिद्धविशेष्यासिद्धानां सङ्ग्रहः । अमीषामपि स्वरूपासिद्ध एवान्तर्भावात् । यथा तु प्रभेदो यथा वान्तर्भावस्तथोपरिष्टाद् वक्ष्यामः । \textbf{केचिदेव} तिन्तिडिकाप्रभृतयः ।
	\pend
      

	  \pstart अथापि स्यात् स्वापवत् धूमोऽप्ययमेकदेशासिद्ध एव । तथाहि पर्वतादिरिह \textbf{पक्षस्तत्र} च क्वचिदेव देशे सिद्धो न सर्वत्र । न च पर्वतादिरेकोऽवयव्यभ्युपगतः । अथैवंविधोऽपि पक्षव्यापक उच्यते तर्हि स्वापेन किमपराद्धं येनासावेवैको न युक्त इति ।
	\pend
      \leavevmode\marginnote{\textenglish{93/dm}}“

	  \pstart पश्चात्कृतेनैव\footnote{अग्रे कृतेन--\cite{dp-msD-n}}कारेणासाधारणो धर्मो निरस्तः । यदि हि “अनुमेय एव सत्त्वम्” इति कुर्यात्\footnote{इति ब्रूयात्--\cite{dp-msB} \cite{dp-msD}} श्रावणत्वमेव हेतुः स्यात् । निश्चितग्रहणेन सन्दिग्धासिद्धः \footnote{असर्वज्ञः कश्चित् वक्तृत्वात्--\cite{dp-msD-n}}सर्वो निरस्तः ।
	\pend
       

	  \pstart सपक्षो वक्ष्यमाणलक्षणः । तस्मिन्नेव सत्त्वं निश्चितमिति द्वितीयं रूपम् । इहापि
	\pend
      ”

	  \pstart अत्र केचिदेवं प्रतिविदधति । जिज्ञासितधर्मविशेषवत्त्वेन हि रूपेण धर्मी पक्ष उच्यते । यश्चासौ पर्वतादिस्तत्र न वह्निर्जिज्ञासितः । किन्तर्हि ? यत्र तत्रोद्देशे । तत्रैकदेशस्थधूमदर्शनेऽप्येकदेशवह्निर्जिज्ञासितो ज्ञायत इति किं सर्वपर्वतव्यापिना धूमेन कार्यम् ? यदि पुनः सर्वत्रैव वह्निमत्त्वं जिज्ञासितं स्यात्, स्यादेवायं पक्षैकदेशासिद्धः । न चैतदेवम् । ततः कथमस्य तथात्वम् ? अत्र पुनः सर्वेषामेव तरूणां चैतन्यं जिज्ञासितमिति सकलपादपव्यापिनैव स्वापेन प्रयोजनम् । यथा वेदस्य सर्वस्यैव पौरुषेयत्वे साध्ये सकलाम्नायव्यापकेनैव वाक्यत्वेन प्रयोजनम् । न चायं सर्वान् व्याप्नोति । ततः पक्षैकदेशासिद्ध उच्यते ।
	\pend
      

	  \pstart एके तु--लोकाध्यवसायसिद्धं महीधरादेरेकत्वमवलम्ब्यायं व्यवहारः, न च सर्वेषां तरूणां तथैकत्वं लोकोऽध्यवस्यति, येन स्वापस्यापि तथा सिद्धिर्भवति, ततः किमवद्यं नामेति प्रतिपन्नाः ।
	\pend
      

	  \pstart अथाभिधीयते--यदि कश्चित्तिन्तिडिकाप्रभृतीनेव पादपान् पक्षयित्वा स्वापं हेतूकरोति, तदाऽयं न पक्षैकदेशासिद्ध इति किं न साधयेच्चैतन्यमिति ।
	\pend
      

	  \pstart असदेतत्--विकल्पानुपपत्तेः । यदि सर्वजनप्रसिद्ध इन्द्रियव्यापारविरोध्यवस्थाविशेषः स्वापो निद्रापरनामा हेतुरभिप्रेतस्तदाऽयं तेष्वपि तरुष्वसिद्ध इति कथं चैतन्यमनुमापयेत् ?
	\pend
      

	  \pstart अथ येन केनचिदुपाधिना स्वापशब्दमात्रवाच्योऽर्थो हेतुः; तथाविधस्य स्वापस्य चैतन्येन व्याप्त्यसिद्धेः सन्दिग्धविपक्षव्यावृत्तिकतयाऽनैकान्तिकः ।
	\pend
      

	  \pstart “षष्ठ्यतसर्थप्रत्येयन”\href{http://http://sarit.indology.info/?cref=Pā.2.3.30}{पाणिनि २. ३. ३०}इत्यनेन पश्चाच्छब्दयोगे \textbf{सत्त्ववचनात्ष}ष्ठी । \textbf{असाधारणः} सपक्षासपक्षसाधारणो यो न भवति । पक्षस्यैव यो धर्म इति यावत् । \textbf{निरस्तो} हेतुत्वेन प्रतिक्षिप्तः । पूर्वावधारणे तु नायं निराकृतः स्यादिति दर्शयति \textbf{यदी}ति । \textbf{हि}र्यस्मात् । अयमस्याशयः--यद्यनुमेय एव सत्त्वं यस्येति लक्षणं स्यात्तदा सपक्ष एव सत्त्वमिति वचनमतिरिच्यमानं लक्षणान्तरं भविष्यति । व्याघात एव वा भविष्यतीति । निश्चितग्रहणस्येहापेक्षितस्य व्यवच्छेद्यं दर्शयति--\textbf{निश्चि}तेति । सन्दिग्धश्चासावस्य हेतुत्वेन विशेषणत्वेन चासिद्धोऽनिश्चितश्चेति विग्रहः । सन्दिग्धविशेषणासिद्धः, सन्दिग्धविशेष्यासिद्धश्च सन्दिग्धासिद्ध एवान्त\leavevmode\marginnote{\textenglish{39b/ms}}र्भवतीति, सोऽप्यनेनैव निरस्तः । यथाऽनयोर्भेदो यथा चान्तर्भावस्तथापरस्तात्प्रदर्शयिष्यामः । इहाश्रितत्वादार्थेन न्यायेन हेतुसत्त्वस्य विशेषणरूपत्वाद् अनेन सहोदिति\footnote{त}एवकारो धर्मायोगस्य व्यवच्छेदको द्रष्टव्यः ।
	\pend
      \leavevmode\marginnote{\textenglish{94/dm}}“

	  \pstart सत्त्वग्रहणेन विरुद्धो निरस्तः । स हि नास्ति सपक्षे । एवकारेण साधारणानैकान्तिकः\footnote{अत्र \cite{dp-msD} प्रतौ पङ्क्तिबाह्यभागे “अनित्यः शब्दः प्रमेयत्वात्” इति टिप्पणं वर्त्तते । तच्च \cite{dp-msB} प्रतौ मूले निवेशितम् इति प्रतिभाति । \cite{dp-edH} \cite{dp-edN} प्रतावपि एवमेव--सं०} । स हि न सपक्ष एव वर्त्तत किन्तूभयत्रापि । सत्त्वग्रहणात् पूर्वावधारणवचनेन \footnote{सपक्षव्या० \cite{dp-msB} \cite{dp-edH}}सपक्षाव्यापिसत्ताकस्यापि प्रयत्नानन्तरीयकस्य हेतुत्वं कथितम् । पश्चादवधारणे\footnote{०णे हि अय० \cite{dp-msA} \cite{dp-msC}} त्वयमर्थः स्यात्—सपक्षे सत्त्वमेव यस्य स हेतुरिति प्रयत्नानन्तरीयकत्वं न हेतुः स्यात् । \footnote{निश्चयवच० \cite{dp-msA}}निश्चितवचनेन सन्दिग्धान्वयोऽनैकान्तिको निरस्तः । यथा सर्वज्ञः कश्चिद् वक्तृत्वात् । वक्तृत्वं हि सपक्षे सन्दिग्धम् ।
	\pend
       

	  \pstart असपक्षो वक्ष्यमाणलक्षणः । तस्मिन्नसत्त्वमेव निश्चितम्--तृतीयं रूपम् । तत्रासत्त्वग्रहणेन विरुद्धस्य निरासः\footnote{पूर्ववदत्रापि \cite{dp-msD} प्रतौ “नित्यः शब्दः कृतकत्वात् खवत्” इति पङ्क्तिबाह्यं टिप्पणत्वेन लिखितः पाठः \cite{dp-msB} प्रतौ मूलत्वेन संनिविष्ट इति भाति--\cite{dp-edN} प्रतौ अपि तथैव कृतम् । \cite{dp-edH} प्रतौ तु “विपक्षैकदेशवृत्तेर्निरासः” इत्यनन्तरं मुद्रितः--सं०} । विरुद्धो हि विपक्षेऽस्ति । एवकारेण साधारणस्य विपक्षैकदेशवृत्तेर्निरासः । \footnote{प्रयत्नानन्तरीयकः शब्दः, अनित्यत्वात् घटवत्--\cite{dp-msD-n}}प्रयत्नानन्तरीयकत्वे साध्ये ह्यनित्यत्वं विपक्षैकदेशे विद्युदादौ अस्ति, आकाशादौ नास्ति । ततो \footnote{नियमतोऽस्य \cite{dp-msC}}नियमेनास्य निरासः । \footnote{असत्त्ववचनात् पूर्व० \cite{dp-msA} \cite{dp-edP} \cite{dp-edH} \cite{dp-edE} \cite{dp-edN} ०शब्दात् पूर्व० \cite{dp-msB} \cite{dp-msC} \cite{dp-msD}}असत्त्वशब्दाद्धि पूर्वस्मिन्नवधारणेऽयमर्थः स्यात्--विपक्ष एव यो नास्ति स हेतुः । तथा च प्रयत्नानन्तरीयकत्वं सपक्षेऽपि \footnote{०पि नास्ति--\cite{dp-msB} \cite{dp-edE}}सर्वत्र
	\pend
      ”

	  \pstart \textbf{साधारण}श्चासौ सपक्षासपक्षवृत्तित्वादनैकान्तिकश्चैकस्मिन्नन्ते साध्ये विपर्यये वाऽव्यवस्थितश्चेति तथा निरस्त इति वर्त्तते । \textbf{सपक्षाव्यापिनी} सकलसपक्षावर्त्तिनी \textbf{सत्ता} यस्येति विग्रहः । प्रयत्नानन्तरीयकत्वमप्यनित्यत्वसिद्धौ समर्थो हेतुरिति दर्शयितुं सकलसपक्षाव्यापि प्रयत्नानन्तरीयकत्वमुदाहृतमिति द्रष्टव्यम् । वस्तुतस्तु सर्व एव कार्यहेतुर्धूमादिः सपक्षैकदेशवृत्तिर्बोद्धव्यः । \textbf{इति}रेवमर्थे । एवं सति प्रयत्नानन्तरीयकत्वमुपलक्षणत्वादस्य धूमादिकमपि न हेतुः स्यात् । \textbf{सन्दिग्धोऽन्वयो} यस्य स तथा । इहाश्रयत्वादार्थेन न्यायेन सपक्षलक्षणस्य धर्मिणो विशेष्यत्वादनेन सहोदितो निपातः पक्षे वृक्षे वृत्तौ लब्धायां समुच्चीयमानावधारणत्वादसपक्षलक्षणधर्म्यन्तरयोगस्य व्यवच्छेदको द्रष्टव्यः ।
	\pend
      

	  \pstart तृतीयं रूपं व्याख्यातुमाह--\textbf{असपक्ष} इति । \textbf{वक्ष्यमाणं लक्षण}मस्येति विग्रहः । कथमस्य निरास इत्याह--\textbf{विरुद्धो ही}ति । हिर्यस्मात् । \textbf{साधारणस्य} सपक्षासपक्षसाधारणस्य । कस्मिन् साध्ये किन्तदीदृशमित्याह--\textbf{प्रयत्ने}ति । \textbf{प्रयत्नः} पुरुषव्यापारः । \textbf{नियमेना}  \leavevmode\marginnote{\textenglish{95/dm}} “
	  
	नास्ति । ततो न हेतुः स्यात् । ततः पूर्वं न कृतम् । निश्चितग्रहणेन सन्दिग्धविपक्षव्यावृत्तिकोऽनैकान्तिको\footnote{असर्वज्ञः कश्चित् वक्तृत्वात्--\cite{dp-msD-n}} निरस्तः । 
	  
	ननु च सपक्ष एव सत्त्वमित्युक्ते विपक्षेऽसत्त्वमेवेति यम्यत एव । तत् किमर्थं \footnote{०र्थमुभ \cite{dp-msC} \cite{dp-msD}}पुनरुभयोरुपादानं कृतम् ? उच्यते \footnote{तदुच्यते--\cite{dp-msA} \cite{dp-edP} \cite{dp-edH} \cite{dp-edE}}। अन्वयो व्यतिरेको वा \add{प्रयुज्यमानः} नियमवानेव प्रयोक्तव्यो नान्यथेति दर्शयितुं\footnote{द्वयोरुपादानं--\cite{dp-msB} \cite{dp-msC} \cite{dp-msD}}द्वयोरप्युपादानं कृतम् । अनियते\footnote{अनियमे हि \cite{dp-msA} \cite{dp-edP} \cite{dp-edH} \cite{dp-edE} \cite{dp-edN}} हि द्वयोरपि प्रयोगेऽयमर्थः स्यात्—सपक्षे योऽस्ति विपक्षे च यो\footnote{च नास्ति स \cite{dp-msA} \cite{dp-msB} \cite{dp-msD} \cite{dp-edP} \cite{dp-edH} \cite{dp-edE} \cite{dp-edN}} नास्ति स हेतुरिति । तथा च सति स श्यामः\footnote{श्यामः त्वत्पुत्र० \cite{dp-msC}}तत्पुत्रत्वात् दृश्यमान-” वधारणेन एवकारणेति यावत् । असपक्ष एवेति किं नावधार्यत इत्याह--\textbf{असत्त्वे}ति । \textbf{हि}र्यस्मात् । भवत्वेवं का क्षतिरित्याह \textbf{तथा चे}ति । प्रयत्नानन्तरीयकग्रहणं पूर्ववदुपलक्षणम् । निश्चितग्रहणस्य व्यावृत्त्यर्थमाह--\textbf{निश्चिते}ति । विपक्षाद् व्यावृत्ति\textbf{र्विपक्षव्यावृत्तिः} । सन्दिग्धा विपक्षव्यावृत्तिर्यस्य स तथा ।
	\pend
      

	  \pstart ननु न सन्दिग्धविपक्षव्यावृत्तिकत्वं नागायं हेतुदोषः । तत्कथं निरस्यते ? तथाहि—य एव विपक्षे वीक्षितो हेतुः स एव प्रमेयत्वादिवदभिमतं न साधयेत् । यः पुनर्महताऽपि प्रयत्नेन मृग्यमाणोऽसपक्षे नोपलक्षितः स कथमङ्ग साध्यं न साधयेदिति ? तदेतदवद्यम् । यतो योऽपि विपक्षे वीक्षितो हेतुः सोऽपि इष्टो दुष्टः । कथं ? साध्यं विनाऽप्युपलब्धेरिति चेत् । ननु यदि नामासौ साध्यमन्तरेणान्यत्र दृष्टस्तथापि विवादाध्यासिते धर्मिणि साध्यं किं न साधयति ? नहि अयमत्रापि साध्यं विनैव वर्त्तत इति प्रसाधकं प्रमाणमस्ति । न चैकत्र येन विना यो दृष्टः सर्वत्रासौ तेन विनैव वर्त्तते इति सिद्धम् । अन्यथाऽपि बहुलं दर्शनात् । अथ साध्यमन्तरेण यो वृत्तः साध्ये सत्येवासौ वर्त्तत इति नियमाभावाद् विवादाध्यासिते सन्देहहेतुर्न सम्यग् हेतुर्भवितुमर्हति । हन्त तर्हि विपक्षे वीक्षणं तस्य साध्यधर्मिणि साध्यविनाकृतां वृत्तिं सम्भावयद् हेतुमनिश्चयहेतूकरोतीति आयातम् । सति चैवं साध्यविपर्यये हेतुसत्तावाधकप्रमाणादर्शनमपि--यद्ययं च धर्मो भविष्यति, न च ततः साध्यमित्येवंविधां वृत्तिमस्य—सम्भावयति तदा किमयं निश्चयहेतुर्भवितुमर्हति ? ततो वि\leavevmode\marginnote{\textenglish{40a/ms}}पक्षे दर्शनं वा तथा शङ्काबीजमस्तु विपर्यये वा बाधकप्रमाणादर्शनं वेति को विशेषः ? तथा चाह \textbf{वार्त्तिककारः}--“न तु सपक्षविपक्षयोः सत्त्वमसत्त्वं वा निश्चयापेक्षम् । निश्चयेऽपि सन्देहमुखेनैव दोषात् । सोऽनिश्चयेऽपि तुल्य इति तथाविधोद्भावनमप्यत्र दूषणमेव । तथा विपक्षप्रचाराशङ्काव्यवच्छेदेन लभ्यं गमकत्वं कथमात्मसात्कुर्याद्” इति ।
	\pend
      

	  \pstart \textbf{ननु चे}त्यादिना लक्षणे चोद्यमाशङ्कते । \textbf{विपक्षेऽसत्त्वमेवेति गम्यत इति} ब्रुवतोऽयं भावः । पक्षे वृत्तौ लब्धायां सपक्ष एव सत्त्वं निश्चितमिति खलु समुच्चीयमानावधारणोऽयं निर्णयः । अयं चासपक्षेऽसत्त्वेऽनिश्चिते सन्दिग्धे वा न घटत इति अवश्यमसपक्षेऽसत्त्वमेव    \leavevmode\marginnote{\textenglish{96/dm}} “
	  
	पुत्रवदिति तत्पुत्रत्वं हेतुः स्यात् । तस्मान्नियमवतोरेवान्वयव्यतिरेकयोः प्रयोगः कर्त्तव्यो येन प्रतिबन्धो गम्येत साधनस्य साध्येन । नियमवतोश्च प्रयोगेऽवश्यकर्त्तव्ये द्वयोरेक एव प्रयोक्तव्यो\footnote{एव कर्त्तव्यो न \cite{dp-msA}} न द्वाविति नियमवानेवान्वयो व्यतिरेको वा प्रयोक्तव्य इति शिक्षणार्थं द्वयोरुपादानमिति ॥ 
	  
	त्रैरूप्यकथनप्रसङ्गेनानुमेयः सपक्षो विपक्षश्चोक्तः । तेषां लक्षणं वक्तव्यम् । तत्र कोऽनुमेय इत्याह--” “
	  
	अनुमेयोऽत्र जिज्ञासितविशेषो धर्मी ॥ ६ ॥” निश्चितमाक्षिपतीति । एवमुपलक्षणत्वादस्यासपक्षे चासत्त्वमेव निश्चितमित्यनेन सपक्षे वृत्तिमात्रं लब्धमेव । तेनैव च प्रयोजनमिति किं सपक्ष एव सत्त्वमित्यनेनेति द्रष्टव्यम् ।
	\pend
      

	  \pstart \textbf{उच्यत} इति परिहारः । साध्येनान्वीयमानत्वं सत्येव साध्ये भवितृत्वं साधनस्य स्वगतो \textbf{धर्मोऽन्वयः} । साध्याभावे व्यतिरिच्यमानत्वम् अभवितृत्वं हेतोः स्वगतो धर्मो \textbf{व्यतिरेकः} । स चासश्च \footnote{“प्रयुज्यमानः” इति पाठः मूले नोपलभ्यते । प्रदीपानुरोधात् तत्र कोष्ठके स्थापितः ।}\textbf{प्रयुज्यमानः} शब्देन प्रतिपाद्यमानो \textbf{नियमवान}व्यभिचारवान् \textbf{प्रयोक्तव्यो}ऽन्यथा परोक्षप्रतीत्यङ्गं नोक्तं स्यात् । कथं च तथा प्रयुक्तो भवति ? यदाऽन्वयवाक्ये साध्ये नियतं साधनं व्यतिरेकवाक्ये च साधनाभावे नियतः साध्याभावः । साधनानुवादपूर्वसाध्यविधानेन साध्याभावानुवादपूर्वसाधनाभावविधानेन वीप्सापदयुक्तेन सर्वशब्दसंहितेन एवकारोपेतेन । अन्यथा न प्रतिपाद्यते ।
	\pend
      

	  \pstart अनियमेनापि प्रयोगे यद्यन्वयव्यतिरेकयोस्तादृशोः प्रतीतिरस्ति तदा किं नियमवत्प्रयोगेणेत्याह--\textbf{अनियते हीति} । हिर्यस्मादनियते विवक्षितनियमाख्यापके । आस्तां तावदेकस्य प्रयोगे द्वयोरप्ययमर्थः स्यादित्यपिशब्दः । \textbf{इति}रर्थस्याकारं दर्शयति । अस्त्वयमर्थः । का क्षतिरित्याह--\textbf{तथा चेति} । काकतालीयन्यायेन चाश्यामेषु निकटवर्त्तिषु अतत्पुत्रेषु एतद् द्रष्टव्यम् । अन्यथा ह्यनियतोऽपि व्यतिरेकोऽत्र नोक्त इति वचनावकाशः स्यात् । यत \textbf{एवं तस्माद्} येन नियमवत्प्रयोगेण प्रतिबन्धः प्रतिबद्धत्वमायत्तत्वम् । कस्येत्याकाङ्क्षायामाह—\textbf{साधनस्ये}ति । कुत्रेत्यपेक्षायामाह--\textbf{साध्य} इति ।
	\pend
      

	  \pstart यद्येवमेकस्मिन्नेव साधनवाक्ये नियमवानेवान्वयो व्यतिरेकश्च प्रयुज्यतामित्याह--\textbf{नियमवतोश्चे}ति । \textbf{चो} वक्तव्यान्तरसमुच्चये । अयमाशयः--अन्वयवाक्येनापि तथाप्रयुक्तेन सामर्थ्याद् व्यतिरेकस्य, व्यतिरेकवाक्येनापि सामर्थ्यादन्वयस्य प्रकाशनात् किं प्रतीतप्रत्यायकेन द्वितीयवाक्येन कार्यमिति ? यत एवमितिस्तस्माद् । द्वितीय \textbf{इतिः} शिक्षणस्य स्वरूपं दर्शयति । \textbf{शिक्षण}मज्ञमुद्दिश्यज्ञापनम् । तदर्थं तन्निमित्तम् । अनेन प्रयो\leavevmode\marginnote{\textenglish{40b/ms}}गसमास एव दर्शितो न रूपसमास इति दर्शितम् । प्रयोगदर्शनाभ्यासात्कश्चित्प्रयोगभङ्ग्यैव स्वयमपि परोक्षमर्थं प्रतिपद्यत इति स्वार्थेऽप्यनुमाने निर्णीतमिदं लक्षणकथनप्रसङ्गेन । ततो न दोष इति द्रष्टव्यम् । एतच्चोपरिष्टान्निवेदयिष्यते ॥
	\pend
      \leavevmode\marginnote{\textenglish{97/dm}}“

	  \pstart अनुमेयोऽत्रेत्यादि । अत्र हेतुलक्षणे निश्चेतव्ये\footnote{ज्ञातव्ये पक्षधर्मत्वे पक्षो धर्म्यभिधीयते । व्याप्तिकाले भवेद् धर्मः साध्यसिद्धौ पुनर्द्वयम् ॥--\cite{dp-msD-n}} धर्मो अनुमेयः । अन्यत्र तु \footnote{साध्ये प्रति०--\cite{dp-msC}}साध्यप्रतिपत्तिकाले समुदायोऽनुमेयः । व्याप्तिनिश्चयकाले तु धर्मोऽनुमेय इति दर्शयितुम् “अत्र” ग्रहणम् । जिज्ञासितो ज्ञातुमिष्टो विशेषो धर्मो यस्य धर्मिणः स तथोक्तः ॥
	\pend
       

	  \pstart कः सपक्षः ?
	\pend
       “

	  \pstart साध्यधर्मसामान्येन समानोऽर्थः सपक्षः ॥ ७ ॥
	\pend
      ” 

	  \pstart समानोऽर्थः सपक्षः । समानः सदृशो\footnote{सदृशोऽर्थो यः पक्षेण \cite{dp-msC}} योऽर्थः पक्षेण स\footnote{पक्षेण स सपक्ष \cite{dp-msB} \cite{dp-msC} \cite{dp-msD} \cite{dp-edP} \cite{dp-edH} \cite{dp-edE} \cite{dp-edN}} पक्ष उक्त उपचारात् समानशब्देन विशेष्यते । समानः पक्षः सपक्षः, समानस्य च सशब्दादेशः\footnote{सशब्द आदेशः \cite{dp-edE}} ।
	\pend
      ”

	  \pstart ननु यदि धर्मधर्मिसमुदायो मुख्योऽनुमेयोऽत्र गृह्यते तदा न कस्यचित्साधनस्य तद्धर्मत्वं ग्रहीतुं शक्येत । ग्रहणे वा व्यक्तमेव वैयर्थ्यमित्यभिप्रायवान् पृच्छति--तत्र क इति तेषु मध्ये ।
	\pend
      

	  \pstart अत्रेत्यस्य तात्पर्यार्थमत्रेत्यादिना निरूपयति । धर्मिधर्मयोश्चानुमेयत्वम् । तस्मिंस्तस्मिन् कालेऽनुमेयैकदेशत्वादुपचारतो द्रष्टव्यम् । तदुक्तम्--
	\pend
      “
	    
	    \stanza[\smallbreak]
	समुदायस्य साध्यत्वात् धर्ममात्रे च धर्मिणि ।&अमुख्येऽप्येकदेशत्वात्साध्यत्वमुपचर्यते ॥ इति ॥\&[\smallbreak]


	”

	  \pstart अथ यदि सह पक्षेण वर्त्तते इति सपक्षोऽभिप्रेतस्तदाऽर्थासङ्गतिः । अथापि समानः पक्षेणेति मतं तदा पक्षसमान इति प्राप्नोतीत्यभिप्रेत्य पृच्छति क इति । \textbf{समानोऽर्थ} इति सिद्धान्ती । समानशब्दस्यार्थमाह--\textbf{समानः सदृशोऽर्थ} इति । अर्थशब्दोऽत्र प्रतीयमानार्थोऽर्थ्यते गम्यत इति कृत्वा । न त्वर्थोऽर्थक्रियासमर्थो वाच्यः । क्रमाक्रमायोगेनाक्षणिकस्य सामर्थ्याभाव\footnote{वे} साध्येऽम्बरारविन्दादेरपि सपक्षत्वेनेष्टत्वात् ।
	\pend
      

	  \pstart ननु समानश्चासौ पक्षश्चेति किं नाभिप्रेतम् ? तत्र पक्ष एवासौ दृष्टान्तधर्मी कथं येन समानशब्देन विशिष्यत इत्याह--\textbf{पक्ष} इति । कथमन्यार्थेन तेन शब्देन स तथोच्यतामित्याह--\textbf{उपचारादि}ति । क्वचित्सपक्ष उक्त इति पाठः । तत्र सोऽर्थः पक्ष उक्तः इति योज्यम् । उपचारे च साध्यधर्मयोगो निमित्तम् । स उपचाराद् यः पक्ष उक्तः \textbf{विशि\footnote{शे}ष्यते} व्यवच्छेद्यते तदसमानात् । यदि समानशब्दो विशेषणमस्य तर्हि सशब्दश्रुतिः कथमित्याह--\textbf{समानस्येति । समानस्य} समानशब्दस्य स्थाने सशब्दादेशश्च “समानस्य” \href{http://http://sarit.indology.info/?cref=Pā.6.3.84}{पाणिनि ६. ३. ८४.}इति योग वेभागात् । समानः पक्षो यस्य स तथेति बहुव्रीहिः किमिति \textbf{धर्मोत्तरेण} नाश्रितो येनैवमात्मा प्रयासित इति \textbf{चे}त् । सत्यम् । केवलं \textbf{विनिश्चया}नुरोधादेव  \leavevmode\marginnote{\textenglish{98/dm}} “
	  
	स्यादेतत्--किं तत् पक्षसपक्षयोः \footnote{साम्यम् \cite{dp-msB}}सामान्यं येन समानः सपक्षः पक्षेणेत्याह--साध्यधर्मसामान्येनेति । साध्यश्चासौ असिद्धत्वात्, धर्मश्च पराश्रितत्वात् साध्यधर्मः । न च विशेषः साध्यः, अपि तु सामान्यम् । अत इह सामान्यं \footnote{पक्ष--\cite{dp-msD-n}}साध्यमुक्तम् । साध्यधर्मश्चासौ सामान्यं चेति साध्यधर्मसामान्येन समानः पक्षेण सपक्ष इत्यर्थः ॥ 
	  
	कोऽसपक्ष इत्याह-- “
	  
	\textbf{न सपक्षोऽसपक्षः ॥ ८ ॥}” 
	  
	न सपक्षोऽसपक्षः । सपक्षो यो न भवति सोऽसपक्षः ॥ 
	  
	कश्च सपक्षो न भवति ?” माचरितम् । \textbf{विनिश्चये} हि नवपक्षधर्मप्रवेदननिर्देशप्रकरणे “साध्यधर्मसामान्येन समानः पक्षः सपक्षस्तदभावोऽसपक्षः” इत्युक्तम् । अतो \textbf{वार्त्तिककारस्यैव} दृष्टान्तधर्मी पक्षोऽभिप्रेत उपचारादित्यत्राप्येवमयं व्याचष्ट इत्यदोषः ।
	\pend
      

	  \pstart \textbf{येन} सामान्येन \textbf{पक्षेण समानः सपक्ष इत्या}हेति योज्यम् । \textbf{साध्य}शब्देनोपचाराद् वह्न्यादिकमभिप्रेतम् । साध्यत्वे हेतुमाह--\textbf{असिद्धत्वा}त् तत्रानिश्चितत्वात् ।
	\pend
      

	  \pstart ननु साध्यधर्मश्चासौ सामान्यं चेति कर्मधारयगर्भः कर्मधारय इहाभिप्रेतः, विशेषश्च साध्यत इति कथं सामान्यशब्देन साध्यधर्मशब्दस्य समास इत्याशङ्क्याह--\textbf{न चे}त्यादि । \textbf{चो} यस्मादर्थे अवधारणे वा । सामान्यमतद्रूपव्यावृत्तवस्तुमात्रम् । तथाविधेनैव हेतोः \leavevmode\marginnote{\textenglish{41a/ms}} व्याप्यत्वादित्यभिप्रायः । यत एवमतोऽस्माद्धेतोरिह सपक्षलक्षणकाले । सामान्यस्य साध्यतामुक्त्वा साध्यधर्मशब्देन सह सामान्यशब्दस्य विग्रहं दर्शयति--\textbf{साध्येति । साध्यधर्म्ये\footnote{र्मे}}त्यादिनोपसंहरति । समान एवेत्यवधारणीयम् न तु सामान्येनैवेति, अन्येनापि वस्तुत्वादिना सादृश्यात् दृष्टान्तधर्मिणः सपक्षत्वाभावप्रसङ्गादिति ।
	\pend
      

	  \pstart यदि सपक्षादन्योऽसपक्षोऽन्यधर्मयोगाच्चान्यस्तदा सपक्ष एवान्यधर्मयोगादन्य इति तत्रासपक्षे वर्त्तमानो हेतुः सर्व एवानैकान्तिकत्वादहेतुः प्रसज्येत । अथ विरुद्धे नञो विधानात् सपक्षविरुद्धोऽसपक्षः सहानवस्थानलक्षणेन विरोधेन विरुद्धस्तदाऽग्निलक्षणो हेतुरौष्ण्यं न गमयेत् । अथाभावे नञिष्यते, तदाऽभावे कस्यचित्सत्त्वासम्भवात् साधारणानैकान्तिको न कश्चित्स्यादिति मनसि निवेश्य पृच्छति--\textbf{कोऽसपक्ष} इति । अनीदृशाशयस्याज्ञस्यैव वा प्रश्नः ।
	\pend
      

	  \pstart \textbf{न सपक्ष} इत्याद्युत्तरम् । \textbf{सपक्षो यो न} भवतीति साध्यधर्मवान् यो न भवतीत्यर्थः । एवञ्चाचक्षाणः प्रसज्यप्रतिषेधवृत्तिं नञं दर्शयति । यद्येवम्, समासः कथमिति चेत् । गमकत्वादसूर्य्यम्पश्यानीत्यादिवत् । अपुनर्ज्ञेयानि सामानि, अलवणभोजी, असूर्य्यम्पश्यानि मुखानि “सुडनपुंसकस्य” \href{http://http://sarit.indology.info/?cref=Pā.1.1.43}{पाणिनि १. १. ४३} इति । प्रयोगसंख्यानियमस्तु \textbf{भाष्यकारीयो यथाऽनुपपन्नस्तथा} चाचार्येणैव \textbf{विनिश्चये} “दुःखं बतायं तपस्वी” इत्याद्युपहासपूर्वकं--“यथा निकेतेन प्रतिपत्तेः” इत्यादिना प्रतिपादित इति नेहोच्यते ।
	\pend
      \leavevmode\marginnote{\textenglish{99/dm}}““

	  \pstart ततोऽन्यस्तद्विरुद्धस्तदभावश्चेति ॥ ९ ॥
	\pend
      ” 

	  \pstart ततः सपक्षाद् \footnote{इयं पृष्वी, गन्धवत्त्वात् । यत्तु पृथ्वी न भवति तद् गन्धवदपि न, यथा अबादिः । अत्र अबादिर्दृष्टान्तीकृतः साध्यादन्य इति--\cite{dp-msD-n}}अन्यः । तेन च\footnote{तेन विरु० \cite{dp-msB}} विरुद्धः\footnote{वह्निनिवृत्तौ धूमनिवृत्तेरास्पदं जलाशय इति साध्येन सह विरुद्धः--\cite{dp-msD-n}} । तस्य च सपक्षस्याभावः\footnote{यथा क्षणिकत्वनिवृत्तौ सत्त्वनिवृत्तेरास्पदं खरविषाणमिति साध्याभावमात्रम्--\cite{dp-msD-n}} । सपक्षादन्यत्वं तद्विरुद्धत्वं च न तावत् प्रत्येतुं शक्यं यावत् सपक्षस्वभावाभावो न विज्ञातः । तस्मादन्यत्वविरुद्धत्वप्रतीतिसामर्थ्यात् सपक्षाभावरूपौ प्रतीतावन्यविरुद्धौ ।
	\pend
      ”

	  \pstart एवं सत्यन्यविरुद्धयोरसपक्षत्वं न स्यादित्यसपक्षशब्देन तदभावतदन्यतद्विरुद्धनां त्रयाणामपि सङ्ग्रहं दर्शयितुं \textbf{कश्चे}त्यादिना प्रश्नपूर्वमुपक्र\add{म}ते । \textbf{चकारः} पुनःशब्दस्यार्थे ।
	\pend
      

	  \pstart \textbf{अन्य} इति विवक्षितधर्मानाधारः, अन्यविषयेऽपि नञि विभागेन नियोगवृत्तेः । न हि स एव ब्राह्मणस्तज्जातियोगाद्, अब्राह्मणश्च धर्मान्तरसमावेशाल्लोके प्रतीयत इति । \textbf{विरुद्ध} इति सहस्थितिलक्षणेनाऽन्योन्यात्मपरिहारस्थितिलक्षणेन च विरोधेन विरुद्धः । \textbf{चो}ऽन्यापेक्षया विरुद्धमसपक्षत्वेन समुच्चिनोति । \textbf{तस्य चाभाव}स्तुच्छरूपः प्रसज्यात्मा व्यवहर्त्तव्यैकस्वभावः ।
	\pend
      

	  \pstart अयमत्र प्रकरणार्थः--सपक्षाभावोऽसपक्षः । साध्यधर्मवान् यो न भवतीत्यर्थः । साध्यधर्माभावार्थत्वादसपक्षशब्दस्य । न चैवं निषेधमात्रमसपक्षः । किन्तर्हि ? सर्वः प्रतियोगी निषेधः पर्युदस्तश्च, अतत्त्वलक्षणत्वादसपक्षस्य । तद् विवक्षिते प्रतियोगिनि तुल्यम्, व्यतिरेकगतेः सर्वत्र तुल्यत्वात्, साक्षादर्थापत्त्या वेति । चः पूर्वापेक्षः समुच्चये । अन्यविरुद्धयोरपि वस्तुसतोः कल्पितयोश्चासपक्षत्वात् । तत्र वर्त्तमानस्य साधारणानैकान्तिकत्वान्न तदभावदोषः, अन्यविरुद्धयोश्च स्वरूपकथनेन तदसपक्षत्वपक्षोक्तो दोषो निरस्त इति सर्वमवदातम् ।
	\pend
      

	  \pstart ननु च \textbf{सपक्षो यो न भवतीति} वचनेन यस्य सपक्षाभावस्वभावत्वं तस्यैवासपक्षत्वं \leavevmode\marginnote{\textenglish{41b/ms}}प्रतिपाद्यते । न चान्यविरुद्धयोस्तदभावस्वभावता सम्भव\footnote{वि}नी, विधिरूपत्वात् । तत्कथं तयोस्तथात्वमित्याशङ्क्याह--\textbf{सपक्षादि}ति । \textbf{अन्यत्वं} ततो भिद्यमानत्वं पृथक्त्वमिति यावत् । तेन च विरुद्धत्वं तावन्न शक्यं ज्ञातुम्, यावत्सपक्षाभावस्वभावो न विज्ञातो भवति । सोऽन्यो विरुद्धश्चेत्यर्थात् ।
	\pend
      

	  \pstart अयमाशयः--यदि तस्यान्यत्वाभिमतस्य यतोऽन्यत्वं व्यवहर्त्तव्यं तद्रूपता चेत्तस्यासिद्धा सन्दिग्धा वा भवेत् तदाऽन्यत्वमेव न स्यात्, तदात्मवत्, धूमस्येव वा बाष्पादिभावेन सन्दिह्यमानस्य न बाष्पादेरन्यत्वनिश्चयः । येन च विरुद्धं यत् तद्रूपं \textbf{चे}त् सिद्धं सन्दिग्धं वा ताद्रूप्येण, तदा तेन सह तस्यावस्थितिः, तदात्मपरिहारेण वाऽवस्थानं कथम्, कथं च निश्चीयेत तदात्मवत् पूर्वोक्तधूमवत् ?
	\pend
      

	  \pstart यत एवं \textbf{तस्मात्} । तयोः साध्यधर्मवत्त्वाभावादित्यर्थः । यत एवं \textbf{ततो} हेतोः साक्षादव्यवधानेन । अनेन यद्येक एवासपक्षो वक्तव्यः तदा तदभाव एवेति सूचितम् । तुरभावा  \leavevmode\marginnote{\textenglish{100/dm}} “
	  
	ततोऽभावः साक्षात् सपक्षाभावरूपः प्रतीयते । अन्यविरुद्धौ तु सामर्थ्यादभावरूपौ प्रतीयेते । ततस्त्रयाणामप्यसपक्षत्वम् ॥ “
	  
	त्रिरूपाणि च त्रीण्येव लिङ्गानि ॥ १० ॥” 
	  
	उक्तेन त्रैरूप्येण त्रिरूपाणि च त्रीण्येव लिङ्गानि इति । चकारो वक्तव्यान्तरसमुच्चयार्थः । त्रैरूप्यमादौ पृष्टं त्रिरूपाणि च लिङ्गानि परेण । तत्र त्रैरूप्यमुक्तम् । त्रिरूपाणि चोच्यन्ते--त्रीण्येव त्रिरूपाणि लिङ्गानि । त्रयस्त्रिरूपलिङ्गप्रकारा इत्यर्थः ॥ 
	  
	कानि पुनस्तानीत्याह-- “
	  
	अनुपलब्धिः स्वभावः\footnote{०भावकार्यं \cite{dp-edE} ०भावकार्ये \cite{dp-msD} \cite{dp-msB} \cite{dp-edP} \cite{dp-edH} \cite{dp-edN}} कार्यं चेति ॥ ११ ॥” 
	  
	प्रतिषेध्यस्य साध्यस्यानुपलब्धिस्त्रिरूपा । विधेयस्य साध्यस्य स्वभावश्च\footnote{०भावस्त्रिरूपः \cite{dp-msA} \cite{dp-msB} \cite{dp-msD} \cite{dp-edP} \cite{dp-edH} \cite{dp-edE} \cite{dp-edN}} त्रिरूपः, कार्य च ॥” दन्यविरुद्धयोर्वैधर्म्यमाह । यतोऽन्यो विरुद्धश्चोक्तया नीत्या सपक्षाभावरूपौ ततस्तस्मात् । आस्तामेकस्य द्वयोर्वा त्रयाणामपीत्यपिशब्दः ।
	\pend
      

	  \pstart \textbf{त्रैरूप्येण} त्रिरूपेणेत्यर्थः । त्रीणि रूपाणि लक्षणानि येषामिति विग्रहः । \textbf{त्रीण्येव} त्रिसंख्यान्येव । \textbf{त्रिरूपाणी}त्यनेनाबाधितविषयत्वादिरूपान्तरयोगेन चतुर्लक्षणत्वं षड्लक्षणत्वं वा पराभिमतं हेतोः प्रतिषेधति, \textbf{त्रीण्येवे}त्यनेन संयोग्यादिभेदेन भूयिष्ठसंख्यत्वम् ।
	\pend
      

	  \pstart \textbf{पृष्ट} इति पाठे त्वाचार्य इति शेषः । यत् त्रैरूप्यं पृष्टो यद् वा यत्त्रैरूप्यं पृष्टं तत्त्रैरूप्यमुक्तमुक्तेन ग्रन्थेन । अथवा यद् यस्मात्त्रैरूप्यं पृष्ट आचार्यः, पृष्टं वा तत् तस्मात् त्रैरूप्यमुक्तमिति । \textbf{च}स्त्रैरूप्येण समं त्रिरूपाणामुक्तकर्मतां समुच्चिनोति ।
	\pend
      

	  \pstart ननु न तावत्परस्यैतद् द्वितयप्रश्नवाक्यं श्रुतम् । तत्कथं परस्य द्वेधा प्रश्नः संकीर्त्त्यत इति चेत् । उच्यते । त्रिरूपाल्लिङ्गादिति श्रुतवता पूर्वपक्षवादिनाऽवश्यं किं तत् त्रैरूप्यं कियच्च त्रिरूपं लिङ्गमित्याकाङ्क्षितव्यम्, तेन पृष्टमित्युच्यते । एतदेव कथमवसीयत इति चेत् । त्रैरूप्यं लिङ्गस्यैवमात्मकमित्यभिधानादाचार्यस्य पूर्वपक्षवादिन एवंरूप प्रश्नोऽवसीयते । त्रिरूपाणि च त्रीण्येवेत्यभिधानाच्च संख्याप्रश्नः । ततः साधूक्तं \textbf{त्रैरूप्यमादावि}त्यादि । \textbf{लिङ्गप्रकारा} लिङ्गस्वरूपाणि ॥
	\pend
      

	  \pstart संयोग्यादिभेदेन त्रित्वासम्भवात् पृच्छति--\textbf{कानीति} । सामान्यविशेषाकाराभ्यां प्रश्नः । \textbf{कार्यं} च विधेयस्येति प्रकृतम् । केवलं \textbf{विधेयस्ये}ति पूर्वं सामर्थ्यादनर्थान्तरस्य विधेयस्य, अधुना त्वर्थान्तरस्य विधेयस्येत्यवसेयम् । विधेयस्यार्थान्तरस्य कार्यं त्रिरूपमिति योजनीयम् । चकारौ पूर्वापेक्षया समुच्चयार्थौ ॥
	\pend
      \leavevmode\marginnote{\textenglish{101/dm}}“

	  \pstart अनुपलब्धिमुदाहर्त्तुमाह--
	\pend
       “

	  \pstart तत्रानुपलब्धिर्यथा--न प्रदेशविशेषे क्वचिद् घटः, उपलब्धिलक्षणप्राप्तस्यानुपलब्धेरिति ॥ १२ ॥
	\pend
      ” 

	  \pstart यथेत्यादि । यथेत्युपप्रदर्शनार्थम् । यथेयमनुपलब्धिस्तथान्यापि । न त्वियमेवेत्यर्थः । प्रदेश एकदेशः । विशिष्यत इति विशेषः प्रतिपत्तृप्रत्यक्षः । तादृशश्च न सर्वः प्रदेशः । तदाह--क्वचिद् इति । प्रतिपत्तृप्रत्यक्षे क्वचिदेव प्रदेश इति धर्मो । न घट इति साध्यम् । \footnote{उपलब्धिज्ञानम् । तस्य लक्षणं जनिका--\cite{dp-msA}}उपलब्धिर्ज्ञानम् । तस्या \footnote{कारणम्--\cite{dp-msD-n}}लक्षणं जनिका सामग्री । तया हि\footnote{ह्यनुपल० \cite{dp-edH}} उपलब्धिर्लक्ष्यते । तत्प्राप्तोऽर्थो\footnote{आलम्बनत्वेन--\cite{dp-msD-n}}जनकत्वेन सामग्र्यन्तर्भावात् । उपलब्धिलक्षणप्राप्तो दृश्य इत्यर्थः । तस्यानुपलब्धेः—इत्ययं हेतुः ।
	\pend
       

	  \pstart अथ यो यत्र नास्ति स कथं तत्र दृश्यः ? दृश्यत्वसमारोपादसन्नपि दृश्य उच्यते । \footnote{यश्चार्थ--\cite{dp-msD-n}}यश्चैवं सम्भाव्यते--यद्यसावत्र भवेद् दृश्य एव भवेदिति । स तत्र अविद्यमानोऽपि दृश्यः समारोप्यः । कश्चैवं सम्भाव्यः ? यस्य समग्राणि स्वालम्बनदर्शनकारणानि भवन्ति । कदा च तानि समग्राणि गम्यन्ते ? यदैकज्ञानसंसर्गिवस्त्वन्तरोपलम्भः । एकेन्द्रियज्ञानग्राह्यं लोचनादिप्रणिधानाभिमुखं \footnote{पट-भूतल--\cite{dp-msD-n}}वस्तुद्वयमन्योन्यापेक्षमेकज्ञानसंसर्गि \footnote{गम्यते \cite{dp-msB}}कथ्यते । तयोर्हि
	\pend
      ”

	  \pstart \textbf{न घट} इति घटाभावव्यवहारयोग्यतेति द्रष्टव्यम् । घटाभावस्य प्रत्यक्षसिद्धत्वात् । एतच्च परस्तादभिधास्यते । \textbf{तया लक्ष्यत} इति ब्रुवता लक्ष्यतेऽनेनेति लक्षणमिति व्यक्तीकृतस् । \textbf{तदि}त्युपलब्धिलक्षणम् । कथं तत्प्रान्त इत्याह--\textbf{जनकत्वे}न । किं तेनैकेन सा जन्यते ये\leavevmode\marginnote{\textenglish{42a/ms}}नैवमुच्यत इत्याह--\textbf{सामग्र्यन्तर्भावादि}ति ।
	\pend
      

	  \pstart समुदायार्थं स्फुटयति । \textbf{दृश्यो} दर्शनयोग्यः । इतिरेवमर्थे । एवमभिधेयो यस्योपलब्धिलक्षणप्राप्तशब्दस्य ।
	\pend
      

	  \pstart यद्यविद्यमानः समारोपाद् दृश्य उच्यते तदा भिन्नेन्द्रियग्राह्यमपि तत्रानेकमस्ति । तस्यापि तर्हि तथात्वमायातमित्याशङ्क्याह--यश्चैवमिति । \textbf{चो}ऽवधारणे । स तादृशो दृश्यतया समारोप्यो न सर्व इत्यर्थात् । \textbf{कश्चैवं सम्भाव्य} इति पृच्छति । पुनःशब्दार्थश्चकारः ।
	\pend
      

	  \pstart \textbf{यस्येति} सिद्धान्ती । यस्येति प्रतिषेध्यस्य । स्वमात्मानमालम्बत इत्याल\textbf{म्बनं । यस्य दर्शन}स्य । तस्य \textbf{कारणानि} । तच्चेन्नास्ति कथं तद्दर्शनकारणसामग्र्यावगतिरित्याशयः पृच्छति \textbf{कदे}ति । \textbf{च} पूर्ववत् । एकस्मिन् ज्ञाने संसर्गः प्रतिभासः स यस्यास्ति । तच्च \textbf{तद् वस्त्वन्तरं} चेति तथा तस्यो\textbf{पलम्भो} यदेत्युक्तम् ।
	\pend
      \leavevmode\marginnote{\textenglish{102/dm}}“

	  \pstart सतोर्नैकनियता भवति प्रतिपत्तिः । योग्यताया द्वयोरप्यविशिष्टत्वात् । तस्मादेकज्ञानसंसर्गिणि दृश्यमाने\footnote{भूतले--\cite{dp-msD-n}} सत्येकस्मिन्नितरत्\footnote{घटः--\cite{dp-msD-n}} समग्रदर्शनसामग्रीकं यदि भवेद् दृश्यमेव भवेदिति सम्भावितं \footnote{दृश्यमा० \cite{dp-msA} \cite{dp-edP} \cite{dp-edH} \cite{dp-edE} \cite{dp-edN}}दृश्यत्वमारोप्यते । तस्यानुपलम्भो दृश्यानुपलम्भः । तस्मात् स एव \footnote{घटादिविविक्त० \cite{dp-msB}}घटविविक्तप्रदेशस्तदालम्बनं च ज्ञानं दृश्यानुपलम्भनिश्चयहेतुत्वात् दृश्यानुपलम्भ उच्यते ।
	\pend
       

	  \pstart यावद्धि एकज्ञानसंसर्गि वस्तु\footnote{वस्तु तज्ज्ञानं चा[[वा]] न निश्चितम् न तावद् \cite{dp-msB} वस्तु तज्ज्ञानं वा न निश्चितम् न तावद् \cite{dp-msD}} न निश्चितम्, तज्ज्ञानं च, न तावद् दृश्यानुपलम्भनिश्चयः । ततो वस्तु अनुपलम्भ उच्यते, तज्ज्ञानं च । दर्शननिवृत्तिमात्रं तु स्वयमनिश्चितत्वादगमकम्\footnote{०गमकमेव \cite{dp-msC}} । \footnote{०कम् । तादृशघटरहितः । \cite{dp-msB}}ततो दृश्यघटरहितः प्रदेशः, तज्ज्ञानं च वचनसामर्थ्यादेव दृश्यानुपलम्भरूपमुक्तं द्रष्टव्यम् ॥
	\pend
      ”

	  \pstart तत्र यदि निराकारं विज्ञानमिति स्थितिस्तदैकशब्दो द्वयादिसंख्यानिरासार्थः । यदा तु साकारमिति स्थितिस्तदा पक्षद्वयम्--एकमेव वाऽनेकाकारमशक्यविवेचनं चित्रं ज्ञानम्, प्रतिवस्त्वनेकमेव वा तत्तदाकारानुकारि । पूर्वस्मिन् पक्षे पूर्ववदेकशब्दः । शेषपक्षे त्वेकचक्षुरायतनप्रभवत्वादेकरूपालम्बनत्वाच्चैकशब्दो गौणः ।
	\pend
      

	  \pstart ननु यदि तत् प्रतिषिध्यमानमनेन भूतलादिना सहैकस्मिन् ज्ञाने प्रतिभासेत तदपेक्षमिदमेकज्ञानसंसर्गि कथ्येत । यावतेदमेव नास्तीत्याशङ्क्य तथात्वमेव तयो\textbf{रेकेन्द्रियेत्या}दिना दर्शयति । \textbf{लोचनादीनां प्रणिधानं} स्वज्ञानोपजनने योग्यीभवनं तत्राभिमुखमनुगुणम् । \textbf{अन्योन्यापे}क्षमिति घटाद्यपेक्षं भूतलादि, तदपेक्षं च घटादीत्यर्थः ।
	\pend
      

	  \pstart यदि नाम द्वयं तत्रावस्थितं तथाप्येकमेव तत्रावभासिष्यत इत्याह--\textbf{तयोरिति । हि}र्यस्मात् । उपपत्तिमाह--\textbf{योग्यताया} इति । यत एव तयोर्घटादिभूतलाद्योरेकज्ञानसंसर्गः सम्भवी \textbf{तस्मात्} । एकस्मिन् भूतलादिके दृश्यमाने सतीतरद् घटादिकं परिपूर्णदर्शनसामग्रीकं ज्ञातव्यम् । तथासदविद्यमानमपि समारोपाद् दृश्यमुच्यत इति दर्शयति ।
	\pend
      

	  \pstart \textbf{भवेदिति}ना सम्भावनाया आकारः कथितः । \textbf{तस्यैव} सम्भावनार्हस्यानुपलम्भस्तद्विविक्तान्योपलम्भरूपः । यस्मात्तस्यैकज्ञानसंसर्गिणो भूतलादेर्दर्शनात् तदनुपलम्भो निश्चीयते \textbf{तस्मात्} ।
	\pend
      

	  \pstart ननु भूतलादिज्ञाननिश्चय एव तदनुपलम्भनिश्चयहेतुत्वादनुपलम्भोऽस्तु भूतलादिकं तु कथमित्याह--\textbf{यावदि}ति । \textbf{हि}र्यस्मात् । अनेन ज्ञानविशिष्टं भूतलादि, भूतलाद्यवच्छिन्नञ्च ज्ञानं दृश्यस्यानुपलम्भमुपलम्भाभावं व्यवहर्त्तव्यैकस्वभावं निश्चाययतीति दर्शितम् । वस्त्विति भूतलादि । ज्ञानमिति तद्ग्राहि । चस्तुल्योपायत्वं समुच्चिनोति ।
	\pend
      \leavevmode\marginnote{\textenglish{103/dm}}“

	  \pstart का पुनरुपलब्धिलक्षणप्राप्तिरित्याह--
	\pend
       “

	  \pstart उपलब्धिलक्षणप्राप्तिरुपलम्भप्रत्ययान्तरसाकल्यं स्वभावविशेषश्च ॥ १३ ॥
	\pend
      ” 

	  \pstart उपलब्धिलक्षणप्राप्तिः--उपलब्धिलक्षणप्राप्तत्वं घटस्य उपलम्भप्रत्ययान्तरसाकल्यमिति । ज्ञानस्य घटोऽपि जनकः, अन्ये च चक्षुरादयः । घटाद् दृश्यादन्ये हेतवः प्रत्ययान्तराणि । तेषां साकल्यं सन्निधिः । स्वभाव एव विशिष्यते तदन्यस्मादिति विशेषो विशिष्ट इत्यर्थः । तदयं विशिष्टः स्वभावः प्रत्ययान्तरसाकल्यं चैतद् द्वयमुपलब्धिलक्षणप्राप्तत्वं घटादेर्द्रष्टव्यम् ॥
	\pend
      ”

	  \pstart यद्येकस्मिन् ज्ञाने ययोः संसर्गोऽस्ति तयोरेकतरोपलम्भस्तदितरानुपलम्भनिश्चयहेतुत्वाद् अनुपलम्भस्तस्माच्च तस्याभावव्यहारस्तदा नीलज्ञानानुभवे पीतज्ञानाभावव्यवहारो न स्यात् तयोरेकज्ञा\leavevmode\marginnote{\textenglish{42b/ms}}नसंसर्गाभावात् । न हि भवन्मते ज्ञानं ज्ञानान्तरेण वेद्यते, स्वसंवेदनत्वाभावप्रसङ्गादिति चेत् । सत्यमेतत् । केवलमेकज्ञानसंसर्गिशब्देनान्योन्याव्यभिचरितोपलम्भत्वमिह \textbf{विवक्षितम्} । तच्च ज्ञानेऽप्यस्ति । यदि हि तज्ज्ञानं विद्यमानं स्यात् तदा नीलज्ञानवत्संविदितमेव भवेत् । न च संवेद्यते । तस्मान्नास्तीति व्यवह्रियत इति किमवद्यम् ।
	\pend
      

	  \pstart स्यादेतत् । किं पुनर्ज्ञातृज्ञेयधर्मोपलब्धिव्युदासेन पर्युदासवृत्तिना नञा ज्ञातृज्ञेयधर्मलक्षणा द्विविध पलब्धिः प्रतिपाद्यते ? न तूपलम्भाभावमात्रं प्रसज्यप्रतिषेधाश्रयेणोच्यते \textbf{यथेश्वरसेनो} मन्यत इत्याशङ्क्याह--\textbf{दर्शन}त्यादि । \textbf{दर्शनमु}पलब्धिस्तस्य \textbf{निवृत्ति}रभावस्तुच्छरूपः सैव तन्मात्रं वस्त्वन्तरसंसर्गविरहः । \textbf{तुः} पूर्वस्मादनुपलम्भाद् वैधर्म्यमस्य द्योतयति । \textbf{स्वयमनिश्चितत्वा}दिति ब्रुवताऽनुपलम्भात् तत्प्रतिपत्तावनवस्थादोषप्रसङ्गेन तस्य साधकाभावः सूचितः । अनिश्चितत्वादेवागमकः । एवञ्च व्याचक्षाणेन इदं सूचितम्--तथाविधानुपलब्धिः प्रमाणनिवृत्तावप्यर्थाभावाभावादभावव्यवहारे साध्येऽनैकान्तिकीति । \textbf{वचनसामर्थ्या}दित्युपलब्धिलक्षणप्राप्तस्येतिवचनसामर्थ्यात्, अन्यथैतदतिरिच्येतेति भावः ॥
	\pend
      

	  \pstart ननूपलब्धिलक्षणप्राप्तः स उच्यते यस्योपलब्धिलक्षणप्राप्तिरस्ति । यथा यस्याप्तिर्यथार्थदर्शनादिरूपाऽस्ति स आप्त इत्युच्यते । सा चोपलब्धिलक्षणप्राप्तिर्यद्यात्ममनःसन्निकर्षः, इन्द्रियमन संयोगः, इन्द्रियार्थसन्निकर्षः, विषयप्रकाशसंयोगः, अनेकद्रव्यवत्त्वम्, रूपं चोद्भूतं समाख्यातं तदा त्वन्मतेऽमीषामभावादुपलब्धिलक्षणप्राप्तिरसम्भविनीति मन्यमानः पृच्छति \textbf{का पुन}रिति । सामान्यविशेषाकाराभ्यामयं प्रश्नः । आचार्यस्यापि नामून्युपलब्धिलक्षणप्राप्तिशब्देन विवक्षितानि ।
	\pend
      

	  \pstart किं तर्हीयमित्यभिप्रायेण यदु\textbf{पलब्धिलक्षणे}त्यादिप्रतिवचनं तदुपलब्धीत्यादिना व्याख्यातुमुपक्रमते । व्याख्येयमेवो\textbf{पलब्धिलक्षणप्राप्ति}शब्दसमानार्थेनो\textbf{पलब्धिलक्षणप्राप्तत्व}शब्देनानुवदति । अयं चास्याशयः--यस्योपलब्धिलक्षणप्राप्तिस्तस्यावश्यमुपलब्धिलक्षणप्राप्तत्वमस्तीति । अत एवोपलब्धिलक्षणप्राप्तिरुपलब्धिलक्षणप्राप्तत्वमित्युक्तम् । कस्येत्याकाङ्क्षायामाह—\textbf{घटस्ये}ति ।
	\pend
      

	  \pstart ननु साकल्यं नामानेकधर्मः । न च ज्ञानस्य हेतवो बहवः । किञ्च यदि प्रतिषेध्योऽपि ज्ञानस्य हेतुः स्यात्तदा तस्मात्प्रत्ययादन्ये प्रत्ययाः प्रत्ययान्तराण्युच्यन्त इत्याशङ्क्याह--\textbf{ज्ञानस्ये}ति । \leavevmode\marginnote{\textenglish{104/dm}} “
	  
	कीदृशः स्वभावविशेष इत्याह-- “
	  
	\footnote{यः स्वभावः सत्स्वन्येषूपलम्भप्रत्ययेषु यत्प्रत्यक्ष एव भवति स स्वभावः \cite{dp-msB} \cite{dp-edP} \cite{dp-edH} सत्स्वन्येषूपलम्भप्रत्ययेषु यः प्रत्यक्ष एव भवति स स्वभावः \cite{dp-edE}}यः स्वभावः सत्स्वन्येषूपलम्भप्रत्ययेषु सन् प्रत्यक्ष एव भवति \footnote{“स स्वभावविशेषः” इति नास्ति \cite{dp-msC} \cite{dp-msD} प्रतयोः}स स्वभावविशेषः ॥ १४ ॥” 
	  
	सत्स्वित्यादि । उपलम्भस्य यानि घटाद् दृश्याद् प्रत्ययान्तराणि तेषु सत्सु विद्यमानेषु यः स्वभावः सन् प्रत्यक्ष एव भवति स स्वभावविशेषः । तदयमत्रार्थः--एकप्रतिपत्त्रपेक्षमिदं प्रत्यक्षलक्षणम् । तथा च सति द्रष्टुं प्रवृत्तस्यैकस्य द्रष्टुर्दृश्यमान उभयवान् भावः । अदृश्यमानास्तु देशकालस्वभावविप्रकृष्टाः स्वभावविशेषरहिताः प्रत्ययान्तरसाकल्यवन्तस्तु । यैर्हि प्रत्ययैः स द्रष्टा पश्यति ते सन्निहिताः । अतश्च सन्निहिता\footnote{०हिताय द्रष्टुं \cite{dp-msA} \cite{dp-edP} \cite{dp-edE} ०हिता यः द्रष्टुं \cite{dp-edH} ०हिता यैः द्रष्टुं \cite{dp-edN}} यद् द्रष्टुं \footnote{प्रवृत्ताः \cite{dp-msB}}प्रवृत्तः सः ।” न केवलं प्रत्ययान्तरसाकल्यमुपलब्धिलक्षणप्राप्तिः किन्त्वन्यदपीत्याह--\textbf{स्वभावे}ति । \textbf{च}स्तुल्यबलत्वं समुच्चिनोति । \textbf{तदन्यस्मात्} पिशाचादेर्वि\textbf{शिष्यते} । ज्ञानजननयोग्यतया विशेषणत्वेऽप्यस्य राजदन्तादिपाठाद् \textbf{विशेष}शब्दस्य पूर्वनिपाता\leavevmode\marginnote{\textenglish{43a/ms}}भावः । कर्मसाधनस्यैव \textbf{विशेष-} शब्दस्यार्थमाह--\textbf{विशिष्ट} इति ।
	\pend
      

	  \pstart द्वयमेतन्मिलितमेवोपलब्धिलक्षणप्राप्तिशब्दवाच्यमुपसंहारव्याजेन \textbf{तदित्यादिना} दर्शयति । यतः प्रत्ययान्तरसाकल्यं स्वभावविशेषश्चोपलब्धिलक्षणप्राप्तिर्विवक्षिता तत्तस्मादुपलब्धिलक्षणप्राप्तिशब्दवाच्यमुपलब्धिलक्षणप्राप्तत्वं घटादेः प्रतिषेध्यस्य ॥
	\pend
      

	  \pstart अथ किं स्थवीयान् स्वभावः स्वभावविशेष उतस्वित्पररूपामिश्रस्वलक्षणात्मक इत्यभिप्रेत्य पृच्छति \textbf{कीदृश} इति । सत्स्वित्याद्युत्तरमुपलम्भस्येत्यादिना व्याचष्टे ।
	\pend
      

	  \pstart ननु किमस्य सम्भवोऽस्ति यदुत प्रत्ययान्तरसाकल्ये सत्यपि स्वभावः प्रत्यक्ष एव भवतीति । तथा हि सत्यपि घटस्य तादृशे स्वभावे विदूरवर्त्तिनः पुरुषस्य लोचनादिप्रणिधानेऽपि नासौ प्रत्यक्षो भवतीत्याशङ्क्याह--\textbf{तदयमि}ति । यस्माद् द्वयमेतदुपलब्धिलक्षणप्राप्तिमवोचदाचार्यस्तत्तस्मादत्र प्रस्तावेऽ\textbf{यमर्थो} वाच्योऽभिमतः ।
	\pend
      

	  \pstart कोऽसावित्याह--\textbf{एके}ति । एकश्चासौ विवक्षितः प्रतिपत्ता चेति तथा तदपेक्षमिदं \textbf{प्रत्यक्षलक्षणम्} । यः स्वभावः सत्स्वन्येषूपलम्भप्रत्ययान्तरेषु सन् प्रत्यक्ष एव भवतीत्येवं रूपः । एकश्च प्रतिपत्ता स एव वाच्यो योऽव्यवधानादिदेशो द्रष्टुं प्रवृत्तश्च । तथाविधे च द्रष्टरि तथाविधोऽवश्यं प्रत्यक्ष एव भवतीति ।
	\pend
      

	  \pstart तथापि कथं पूर्वपक्षातिक्रम इत्याह--\textbf{तथा चेति} तस्मिश्च प्रकारे सति । \textbf{दृश्यमान} इति हेतुभावेन विशेषणम् । \textbf{उभयवान्} स्वभावविशेषवान् प्रत्ययान्तरसाकल्यवांश्च । यद्य  \leavevmode\marginnote{\textenglish{105/dm}} “
	  
	द्रष्टुमप्रवृत्तस्य तु योग्यदेशस्था अपि द्रष्टुं ते न शक्याः प्रत्ययान्तरवैकल्यवन्तः, स्वभावविशेषयुक्तास्तु । दूरदेशकालास्तु उभयविकलाः । तदेवं पश्यतः कस्यचिन्न प्रत्ययान्तरविकलो नाम, स्वभावविशेषविकलस्तु भवेत् । अपश्यतस्तु\footnote{०स्तु शक्यो द्रष्टुं योग्य० \cite{dp-msA} \cite{dp-edP} \cite{dp-edH} \cite{dp-edE} \cite{dp-edN}} द्रष्टुं शक्यो योग्यदेशस्थः प्रत्ययान्तरविकल । अन्ये तूभयविकला इति ॥” दृश्यमाना अपि स्वभावविशेषवन्तस्तदा किं स्वभावविशेषग्रहणेनेत्याह--\textbf{अदृश्यमाना इति} । तुर्दृश्यमानेभ्योऽदृश्यमानान् भिनत्ति । देशादिविप्रकृष्टत्वमदृश्यमानत्वे निबन्धनम्, हेतुभावेन विशेषणात् । \textbf{ते प्रत्ययान्तरसाकल्यवन्तः । तुः} स्वभावविशेषविरहात्प्रत्ययान्तरसाकल्यवत्त्वेन तान् विशिनष्टि । \textbf{द्रष्टुं प्रवृत्तस्ये}त्यस्यानुवृत्ताविदं द्रष्टव्यम् ।
	\pend
      

	  \pstart अयमत्र प्रकरणार्थः--एकप्रतिपत्त्रपेक्षया यस्तथाविधः स्वभावः सोऽपि यद्युपलब्धिलक्षणप्राप्तिलक्षणत्वेन नोपादीयते, तदा तेषामपि देशादिविप्रकर्षिणां प्रत्ययान्तरसाकल्यमुपलब्धिलक्षणप्राप्तिरस्तीत्युपलब्धिलक्षणप्राप्तानामनुपलम्भादभावव्यवहारः प्रवर्त्तनीयः स्यात् । न चैतद् युज्यते । तस्माद् विशिष्टप्रतिपत्त्रपेक्षमिदं स्वभावविशेषस्य लक्षणमित्युपलब्धिलक्षणप्राप्तिलक्षणं सूक्तमिति ।
	\pend
      

	  \pstart प्रत्ययान्तरसाकल्यवत्त्वमेव तेषां साधयन्नाह--यैरिति । हिर्यस्मादर्थे । एतदेव कुतः सिद्ध्यतीत्याह--\textbf{अत} इति ॥ \textbf{अत} इत्ययं निपातो वक्ष्यमाणहेत्वर्थः । \textbf{चो} वक्तव्यमेतदित्यस्मिन्नर्थे । \textbf{सन्निहितास्ते} प्रत्यया यद् यस्माद् \textbf{द्रष्टुं प्रवृत्तस्}तद् विविक्तं द्रष्टुं प्रवृत्तो यत इत्यर्थः । यद् वा तदेव निरीक्षितुं प्रवृत्तो यत इति । तदा तु\leavevmode\marginnote{\textenglish{43b/ms}}प्रेक्षापूर्वकारीति द्रष्टव्यम् । दर्शनप्रवृत्तपुरुषापेक्षया तावदर्थस्यैवंप्रकारवत्त्वम्, द्रष्टुमप्रवृत्तस्य तु स कीदृश इत्याह—\textbf{द्रष्टुमप्रवृत्तस्ये}ति । तुना द्रष्टुं प्रवृत्तादप्रवृत्तस्य भेदमाह । यावत्येव देशे सति तस्मिन् प्रत्ययान्तरे दृश्यन्तेऽर्थाः स एव \textbf{योग्यो देशः,} तत्र\textbf{स्थाः} ।
	\pend
      

	  \pstart तस्माद् दृश्यादन्ये ये चक्षुरादयो हेतवस्तानि \textbf{प्रत्ययान्तराणि} । तेषां \textbf{वैकल्यमभाव}स्तद्वन्तः, हेतुभावेन विशेषणाद् । अत एव द्रष्टुं ते न शक्याः । स्वभावविशेषस्तु तेषामस्तीति दर्शयन्नाह--\textbf{स्वभावे}ति । तुना प्रत्ययान्तरवैकल्यवत्त्वात्स्वभावविशेषयुक्तत्वेनतान् विशिनष्टि । तथाविधपुरुषापेक्षया देशकालविप्रकृष्टानां तु का वार्त्तेत्याह--दूरेति । दूरौ देशकालौ येषां ते तथोक्ताः । ये हि देशेन विप्रकृष्टास्ते दूरदेशाः, ये च कालेन ते दूरकाला भवन्तीति भावः । । \textbf{तुः} पूर्वेभ्य इमान् भिनत्ति । अत्रापि द्रष्टुमप्रवृत्तस्येत्यनुवर्त्तते । \textbf{तदेव}मित्यादिनोक्तमेवोपसंहरति ।
	\pend
      

	  \pstart अथवा \textbf{एकप्रतिपत्त्रपेक्षमिद}मित्यन्यथा व्याख्यायते--इहोपलब्धिलक्षणप्राप्तस्येत्यनेन देशकालस्वभावविप्रकृष्टतयाऽनुपलब्धिलक्षणप्राप्ताः किल व्यावर्त्तयितव्याः । न च तेऽप्यनुपलब्धिलक्षणप्राप्ताः शक्या वक्तुं यतो व्यवच्छिद्येरन्, तथापि पिशाचोऽपि सजातीयैरुपलभ्यते । एवं देशविप्रकृष्टोऽपि तद्देशीयैः । तथा कालविप्रकृष्टोऽपि तत्कालिकैरिति व्यावर्त्त्याभावादुप \leavevmode\marginnote{\textenglish{106/dm}} “
	  
	अनुपलब्धिमुदाहृत्य स्वभावमुदाहर्त्तुमाह-- “
	  
	स्वभावः \footnote{स्वभावः सत्ता० \cite{dp-msC} स्वभावः स्वसत्ताभावि \cite{dp-edE}}स्वसत्तामात्रभाविनि साध्यधर्मे हेतुः ॥ १५ ॥” 
	  
	\footnote{“स्वभाव इत्यादि” इति नास्ति \cite{dp-msA} \cite{dp-edP} \cite{dp-edH} \cite{dp-edE} \cite{dp-edN}}स्वभाव इत्यादि । स्वभावो हेतुरिति सम्बन्धः । कीदृशो हेतुः साध्यस्य\footnote{साध्यस्यैव स्व० \cite{dp-msA} \cite{dp-edP} \cite{dp-edH} \cite{dp-edE} \cite{dp-edN} साध्यस्य भाव \cite{dp-msB}} स्वभाव इत्याह--स्वस्य आत्मनः सत्ता । सैव केवला स्वसत्तामात्रम् । तस्मिन् सति भवितुं शीलं यस्येति । यो हेतोरात्मनः सत्तामपेक्ष्य विद्यमानो भवति, न तु हेतुसत्ताया व्यतिरिक्तं कञ्चिद्धेतुमपेक्षते स\footnote{“स” इति नास्ति \cite{dp-msA}} स्वसत्तामात्रभावी साध्यः । तस्मिन् साध्ये यो हेतुः स स्वभावः तस्य साध्यस्य \footnote{“नान्यः” इति नास्ति \cite{dp-msC}}नान्यः ॥ 
	  
	उदाहरणम्-- “
	  
	यथा वृक्षोऽयं शिंशपात्वादिति\footnote{त्वाद् । \cite{dp-msC}} ॥ १६ ॥” 
	  
	\footnote{“यथेति” इति नास्ति \cite{dp-edH} \cite{dp-edE} \cite{dp-edN}}यथेति । अयम् इति धर्मो । वृक्ष इति साध्यम् । शिंशपात्वादिति हेतुः । तदयमर्थः--वृक्षव्यवहारयोग्योऽयम्, शिंशपाव्यवहारयोग्यत्वादिति । यत्र प्रचुरशिंशपे\footnote{शिंशपदेशे--\cite{dp-msC}} देशेऽविदितशिंशपाव्यवहारो जडो \footnote{यथा \cite{dp-msB}}यदा केनचिदुच्चां \footnote{०मुपदर्श्य \cite{dp-msB}}शिंशपामुपादर्श्योच्यते “अयं वृक्षः” इति तदासौ जाड्याच्छिंशपाया उच्चत्वमपि \footnote{वृक्षत्वव्यवहारनिमि० \cite{dp-msC} ०व्यवहारनिमि० \cite{dp-msA} \cite{dp-edP} \cite{dp-edE} \cite{dp-edH} \cite{dp-edN}}वृक्षव्यवहारस्य निमित्तमधस्यति तदा यामेवानुच्चां \footnote{०च्चां शिंशपां पश्यति तामे० \cite{dp-msA} \cite{dp-edP} \cite{dp-edH} \cite{dp-edE} \cite{dp-edN}}पश्यति शिंशपां \footnote{०मेवावृक्षत्वम० \cite{dp-msA}}तामेवावृक्षमवस्यति । स मूढः शिंशपात्वमात्रनिमित्ते” लब्धिलक्षणप्राप्तस्येति विशेषणमनर्थकमित्याशङ्क्याह--\textbf{तदयमत्रार्थ} इति । तदा तु प्रत्यक्षशब्देनोपलब्धिलक्षणप्राप्तत्वं वाच्यम्, तस्य लक्षणमिदं पूर्वोवतमिति योज्यम् । \textbf{एकप्रतिपत्त्रपेक्षमिदं प्रत्यक्षलक्षण}मुपलब्धिलक्षणमित्यर्थः ।
	\pend
      

	  \pstart तथापि कथं चोद्यातिक्रम इत्याह--\textbf{तथा चे}ति । शेषं पूर्वमेव कृतव्याख्यानम् ॥
	\pend
      

	  \pstart सम्प्रति स्वभावहेतुं विवरितुम\textbf{नुपलब्धिमि}त्यादिनोपक्रमते । सामान्यवृत्तिरप्ययं \textbf{स्वभाव}शब्दः साध्यधर्मस्य श्रुतत्वात्तस्यैव स्वभावे वर्त्तते इत्यभिप्रायेण साध्यस्य \textbf{स्वभाव} इत्यभ्यवादीत् । हेतोः स्वरूपस्य चिन्तनात्स्वशब्देन तस्यैवात्मा विवक्षितः । एतदेव यो हेतोरित्यादिना स्फुटयति । तस्मिन् साध्ये यो हेतुर्गमकः ॥
	\pend
      \leavevmode\marginnote{\textenglish{107/dm}}“

	  \pstart वृक्षव्यवहारे प्रवर्त्त्यते । नोच्चत्वादि निमित्तान्तरमिह वृक्षव्यवहारस्य । अपि तु शिंशपात्वमात्रं निमित्तं--शिंशपागतशाखादिमत्त्वं निमित्तमित्यर्थः ॥
	\pend
       

	  \pstart कार्यमुदाहर्त्तुमाह--
	\pend
       “

	  \pstart कार्यं यथा\footnote{यथाग्निरत्र \cite{dp-msB} \cite{dp-msD} \cite{dp-edP} \cite{dp-edH} \cite{dp-edE} \cite{dp-edN}} वह्निरत्र धूमादिति ॥ १७ ॥
	\pend
      ” 

	  \pstart \footnote{अग्निरिति \cite{dp-msB} \cite{dp-msD} \cite{dp-edP} \cite{dp-edH} \cite{dp-edE} \cite{dp-edN}}वह्निरिति साध्यम् । अत्रेति धर्मी । धूमादिति हेतुः । कार्यकारणभावो लोके प्रत्यक्षानुपलम्भनिबन्धनः\footnote{०पलम्भः निबन्धनं प्र० \cite{dp-msB}} प्रतीत इति न स्वभावस्येव कार्यस्य लक्षणमुक्तम् ॥
	\pend
      ”

	  \pstart \textbf{उदाहरण}मस्यार्थस्येति प्रकरणात् । उदाह्रियते प्रदर्श्यते स्वभावहेतुरनेनेत्युदाहरणं स्वभावहेतुप्रतिपादकं वाक्यम् । इदं च स्वभावहेतोरर्थकथनं न तु तत्प्रयोगोपदर्शनम् । प्रयोगस्तु--यः शिंशपात्वव्यवहारयोग्यः, स वृक्षत्वव्यवहारयोग्यः । यथा प्रवर्त्तितवृक्षत्वव्यवहारा पूर्वाधिगता शिंशपा । शिंशपाव्यवहारयोग्यश्चायमिति ।
	\pend
      

	  \pstart ननु च यः शिंशपां पश्यति स वृक्षं जानात्येव । तत्कथमत्र साध्यसाधनभाव इत्याह—\textbf{तदयमि}ति । यस्मात् शिंशपा साधनत्वेनोपन्यस्ता, वृक्षः साध्यत्वेन । न चैतद् यथाश्रुति सङ्गच्छते \textbf{तत्} तस्मात्, वृक्षोऽयं--\textbf{वृक्षव्यवहारयो}\leavevmode\marginnote{\textenglish{44a/ms}}\textbf{ग्योऽयं} शिंशपात्वात् \textbf{शिंशपाव्यवहारयोग्यत्वादित्ययमर्थो वाच्यः--वृक्षोऽयं शिंशपात्वा}दित्यस्य वाक्यस्येति प्रकरणात् । \textbf{इति}र्वाक्यार्थंस्यैव स्वरूपं दर्शयति ।
	\pend
      

	  \pstart ननु यो विदितवृक्षव्यवहारः स स्वयं प्रत्यक्षेणैव तं व्यवहारं प्रवर्त्तयिष्यति तत्कथमस्यानुमानस्यावतार इत्याह--\textbf{तत्रे}ति वाक्योपक्षेपे । \textbf{निमित्तमि}त्यन्तं सुबोधम् ।
	\pend
      

	  \pstart अनेन च प्रबन्धेन मूढं प्रत्येतद् व्यवहारसाधनमनुमानमिति दर्शितम् । केवलमिदमत्र निरूपणीयम् । \add{यदा च} वृक्षत्वव्यहारव्युत्पत्तिं कार्यमाण एवारोपितोच्चत्वादिनिमित्तः, तदा केन दृष्टान्तेन बोधयितव्यः ? आदित एव तेन शाखादिमत्त्वमात्रं निमित्तं न गृहीतमिति । सत्यमेतत् । केवलं बोधे यत्नः करणीयः । \textbf{तदासौ जाड्यादुच्चत्वमपि निमित्तमवस्य}तीतीदमेतस्मिन् मूढे प्रतिपत्तरि योजनीयम् । यः प्रथमं तावत् शिंशपागतं शाखादिमत्त्वमात्रमेव निमित्तमवसाय वृक्षव्यवहारं प्रावर्त्तयत् पश्चाज्जाड्यवशात्तन्मात्रं निमित्तं विस्मृत्यान्यदेव वृक्षव्यवहारकाले उच्चत्वमपि निमित्तमासीदिति व्यामुह्य तदोच्चत्वमपि वृक्षव्यवहारनिमित्तमवकल्पयतीति । स चैवंभूतो जडः शिंशपाव्यवहारयोग्यत्वेन हेतुना प्रथमं प्रवर्त्तिततन्मात्रनिमित्तवृक्षव्यवहारया तदानीं तथास्मारितया शिंशपया दृष्टान्तेन वृक्षव्यवहारयोग्यतां बोधयितुं शक्यत एव । यः पुनरादित एवारोपितोच्चत्वादिनिमित्तस्तं प्रति हेतूपन्यास एव न युज्यते । किन्तर्हि ? वृक्षव्यवहारसमयमेवासौ ग्राहयितव्य इति सर्वमवदातम् ॥
	\pend
      \leavevmode\marginnote{\textenglish{108/dm}}“

	  \pstart ननु त्रिरूपत्वादेकमेव लिङ्गं\footnote{लिङ्गमयुक्तम् \cite{dp-msB} \cite{dp-msC} \cite{dp-msD} \cite{dp-edP} \cite{dp-edH} \cite{dp-edE} \cite{dp-edN}} युक्तम् । अथ प्रकारभेदाद्भेदः । एवं सति स्वभावहेतोरेकस्यानन्तप्रकारत्वात् त्रित्वमयुक्तमित्याह--
	\pend
       “

	  \pstart अत्र द्वौ वस्तुसाधनौ । एकः प्रतिषेधहेतुः ॥ १८ ॥
	\pend
      ” 

	  \pstart अत्र द्वौ इति । अत्रेत्येषु त्रिषु हेतुषु मध्ये द्वौ हेतू वस्तुसाधनौ--विधेः साधनौ गमकौ । एकः प्रतिषेधस्य हेतुर्गमकः । प्रतिषेध इति चाभावोऽभावव्यवहारश्चोक्तो द्रष्टव्यः ।
	\pend
      ”

	  \pstart इदानीं कार्यहेतुं विवरीतुमाह--कार्यमिति । हेतोः प्रकृतत्वात्कार्यमिति कार्यहेतुमित्य\textbf{वसे}यं सुखप्रतिपत्त्यर्थम् ।
	\pend
      

	  \pstart साध्यादिस्वरूपमाह \textbf{वह्निरि}ति । एतदपि हेतोरर्थकथनम् । न तु प्रयोगप्रदर्शनम् । व्याप्तेर्दर्शयितव्याया अप्रदर्शनात् । अनिर्देश्यायाश्च प्रतिज्ञाया निर्देशात् । व्याप्तिवेदिन्यपि पुंसि हेतुरनुवाद्येनैव रूपेण निर्दिश्यमान प्रथमान्त एव निर्देश्यः--अत्र धूम इति । न तु धूमादिति । न च तथाविधं प्रत्यपि प्रतिज्ञा प्रयोज्या अन्यथा क एनामसाधनाङ्गं ब्रूयात् । साधनाङ्गत्वे च शतमुखी बाधा \textbf{वादन्यायस्या}पद्येत ।
	\pend
      

	  \pstart ईदृशस्तु प्रयोगः करणीयः--यत्र धूमस्तत्र सर्वत्र वह्निर्यथा महानसे, धूमश्चात्रेति । स्वभावानुपलब्ध्योरिव कार्यहेतोरपि कस्माल्लक्षणमाचार्येण न प्रणीतमित्याशङ्कामपाकुर्वन्नाह—\textbf{कार्ये}ति । \textbf{लो}के व्यवहर्त्तरि जने । \textbf{प्रतीतः} प्रसिद्धः । \textbf{इति}र्हेतौ । \textbf{नोक्तमा}चार्येणेति शेषः ।
	\pend
      

	  \pstart अयमभिप्रायः--अनुपलब्धौ खलु बहवो विप्रतिपन्नाः । उपलब्ध्यभावमात्रमनुपलब्धि\textbf{मीश्वरसेनो} मन्यते । \textbf{कुमारिल}स्तु वस्त्वन्तरस्यैकज्ञानसंसर्गितामनपेक्ष्यैवान्यमात्रस्य ज्ञानमनुपलब्धिमभावप्रमाणतया वर्णयति । तथा, विवक्षितज्ञानानाधारतालक्षणमात्मनोऽपरिणामं स्वापादिसाधारणम\leavevmode\marginnote{\textenglish{44b/ms}}पि तथात्वेन वर्णयति । यदाह--“सात्मनोऽपरिणामो वा विज्ञानं वाऽन्यवस्तुनि” \href{http://http://sarit.indology.info/?cref=śv.abhāva.11}{श्लोकवा० अभाव० ११} इति ।
	\pend
      

	  \pstart तथा स्वभावेऽपि हेतौ बहवो विप्रतिपेदिरे । केचिदर्थान्तरापेक्षिण्यपि धर्मे स्वभावं हेतुमध्यवसिताः । केचित्तु वस्तुनो धर्मविशेषमाश्रितं स्वभावमिति ।
	\pend
      

	  \pstart तद्विप्रतिपत्तिनिराकरणार्थं तयोर्लक्षणमाख्यातम् । अत्र तु कार्यत्वरूपे न केचिद् विप्रतिपद्यन्त इति नास्य लक्षणमुक्तमिति । कार्यकारणभावेन गम्यगमकभावे सर्वथा गम्यगमकभावप्रसङ्ग इत्यादिकायां विप्रतिपत्तावपि न कार्यस्य लक्षणे विप्रतिपत्ति । किन्तर्हि ? तस्य गमकत्वे । सा चान्यत्र निराकृताऽत्रापि प्राज्ञैः स्वयमभ्यूह्या प्राज्ञजनाधिकारेणास्य प्रारम्भादिति ॥
	\pend
      

	  \pstart सम्प्रति त्रिरूपाणि च त्रीण्येवेत्यसहमानः प्राह--\textbf{नन्वि}ति । \textbf{ननु} प्रश्नः । \textbf{अथ}शब्दो \textbf{यदिशब्द}स्यार्थे । \textbf{प्रकारस्य} स्वरूपस्य \textbf{भेदा}द् विशेषात् \textbf{भेदो} नानात्वम् । एवमभ्युपगमे सति । \textbf{एकस्ये}त्यभिन्नस्य । अभिन्नत्वञ्चास्वभावहेतुत्वव्यावृत्तेः सर्वत्र भावात् ।
	\pend
      \leavevmode\marginnote{\textenglish{109/dm}}“

	  \pstart तदयमर्थः--हेतुः साध्यसिद्ध्यर्थत्वात् साध्याङ्गम्, साध्यं प्रधानम् । अतश्च साध्योपकरणस्य हेतोः प्रधानसाध्यभेदाद्भेदः न \footnote{स्वरूपात् \cite{dp-msC}}स्वरूपभेदात् । साध्यश्च कश्चिद्विधिः, कश्चित् प्रतिषेधः । विधिप्रतिषेधयोश्च \footnote{परस्परं परि० \cite{dp-msB} \cite{dp-msD}}परस्परपरिहारेणावस्थानात् तयोर्हेतू भिन्नौ । विधिरपि कश्चिद्धेतोर्भिन्नः, कश्चिदभिन्न । भेदाभेदयोरप्यन्योन्यत्यागेनात्मस्थितेर्भिन्नौ हेतू । ततः साध्यस्य परस्परविरोधात् हेतवो\footnote{हेतवोऽपि भिन्ना \cite{dp-msB}} भिन्नाः, न तु स्वत एवेति ॥
	\pend
       

	  \pstart कस्मात् पुनस्त्रयाणां हेतुत्वम्, कस्माच्चान्येषामहेतुत्वमित्याशङ्क्य यथा त्रयाणामेव हेतुत्वमन्येषां चाहेतुत्वं तदुभ्यं दर्शयितुमाह--
	\pend
      ”“

	  \pstart स्वभावप्रतिबन्धे हि सत्यर्थोऽर्थं गमयेत् ॥ १९ ॥
	\pend
      ”

	  \pstart \textbf{अनन्तप्रकारत्वा}दिति ब्रुवतोऽयं भावः--सविशेषणनिर्विशेषणव्यतिरिक्ताव्यतिरिक्तविशेषणत्वादिभेदेनानन्तस्वभावत्वादिति । \textbf{गमकावि}ति विवृण्वन् साधयत इति \textbf{साधनावि}ति कर्त्तरि ल्युटं दर्शयति । अत्र च वस्तुनः साधनावेवेत्यवधारणीयं न तु वस्तुन एवेति । इतरव्यवच्छेदस्यापि ताभ्यां साधनात् । प्रतिपत्त्रध्यवसायानुरोधात्तु विधिसाधनत्वमनयोरुच्यते । प्रतिषेधस्य हेतुरेवेत्यवधारणीयम्, न त्वयमेवेति, पूर्वाभ्यामपि सामर्थ्यात्प्रतिषेधस्य साधनात् ।
	\pend
      

	  \pstart ननु न दृश्यानुपलम्भेनाभावः साध्यते, तस्य प्रत्यक्षसिद्धत्वात् । किन्तु व्यवहारः । तत्कथं प्रतिषेधोऽनुपलम्भसाध्यः ? अथाभावव्यवहारः प्रतिषेध उच्यते । तर्हि व्यापकानुपलम्भादिना व्याप्याद्यभावे साध्ये केनाभावः साधितः ? ततः किमत्र प्रतिषेधशब्देन प्रतिपत्तव्यमित्याशङ्क्याह--\textbf{प्रतिषेध} इति । \textbf{इतिः} प्रतिषेधशब्दं प्रत्यवमृशति ।
	\pend
      

	  \pstart तेन \textbf{प्रतिषेध} इत्यनेन शब्दे\textbf{नाभावोऽभावव्यवहारश्चोक्तो द्रष्टव्य} इत्यर्थः । एकस्य प्रतिषेधेन इति मुख्यया वृत्त्या सङ्ग्रहोऽन्यस्य प्रतिषेधाश्रयतया गौण्या वृत्त्येति भावः । नास्तीति \textbf{ज्ञानं} नास्तीत्यभिधानं निःशङ्काऽत्र गमनागमनलक्षणा प्रवृत्ति\textbf{र्व्यवहारः} । स च हठात्प्रवर्त्तयितुं न शक्यत इति तद्योग्यतैव साध्येति द्रष्टव्यम् । एवं तु स्वभावहेतावन्तर्भावेऽप्यनुपलम्भस्य ततः पृथक्करणं प्रतिपत्त्रध्यवसायवशादित्यवसेयम् ।
	\pend
      

	  \pstart ननु विधिप्रतिषेधसाधनत्वेऽप्यमीषां त्रिरूपत्वमविशिष्टम् । तत्त्वादेव चाभेदश्चोदितः । तत्कथमिदमुत्तरं पूर्वपक्षमतिवर्त्ततामित्याशङ्क्याह--\textbf{तदयमि}ति । यस्मादिदमुत्तरीकृतमाचार्येण यथाश्रुति च पूर्वपक्षं नातिक्रामति \textbf{तत्तस्मादयमर्थो} वाक्यस्यायं तात्पर्यार्थ इत्यर्थः । एतमेवार्थं हेतुरित्यादिना \textbf{न तु स्वत एवेत्य}न्तेन ग्रन्थेन प्रतिपादयति ॥
	\pend
      

	  \pstart अथ कथमन्योऽर्थोऽन्यमर्थं न व्यभिचरति येनैते त्रयो हेतवः ? य\leavevmode\marginnote{\textenglish{45a/ms}}था चामीषां स्वसाध्यसाधनाद् गमकत्वं तथाऽन्येषामप्यकार्यस्वभावानुपलम्भात्मनां किं न भवतीति मन्वानः प्रश्नेनोपक्रमते--\textbf{कस्मादि}ति । \textbf{कस्मादिति} सामान्यतो हेतुं पृच्छति \textbf{पुनरि}ति विशेषतः । \textbf{त्रयाणाम}नुपलब्ध्यादीनाम् । \textbf{चः} पूर्वनिमित्तापेक्षया निमित्तान्तरसमुच्चयार्थः । \textbf{अन्येषा}मनीदृशात्मनां संयोग्यादीनाम् ।
	\pend
      \leavevmode\marginnote{\textenglish{110/dm}}“

	  \pstart स्वभावप्रतिबन्ध इति । स्वभावेन प्रतिबन्धः \footnote{“स्वभावप्रतिबन्धः” इति नास्ति \cite{dp-msC}}स्वभावप्रतिबन्धः । “साधनं कृता” \href{http://http://sarit.indology.info/?cref=vk-mbh.2.1.33}{व्या०
	    महा० २. १. ३३} इति समासः । स्वभावप्रतिबद्धत्वं प्रतिबद्धस्वभावत्वमित्यर्थः । कारणे \footnote{स्वभावे स्वोत्पत्तौ सत्यां प्रतिबन्धः स्वसत्तायाः प्रतिबन्धः ।--\cite{dp-msD-n}}स्वभावे च साध्ये स्वभावेन प्रतिबन्धः कार्यस्वभावयोरविशिष्ट इत्येकेन समासेन द्वयोरपि संग्रहः । हिर्यस्मादर्थे । यस्मात् स्वभावप्रतिबन्धे सति साधनार्थः साध्यार्थ गमयेत्, तस्मात् त्रयाणां गमकत्वम्, अन्येषामगमकत्वम् ॥
	\pend
       

	  \pstart कस्मात् पुनः स्वभावप्रतिबन्ध एव सति गम्यगमकभावो नान्यथेत्याह--
	\pend
       “

	  \pstart तदप्रतिबद्धस्य तदव्यभिचारनियमाभावात् ॥ २० ॥
	\pend
      ” 

	  \pstart तदप्रतिबद्धस्येति । “तद्” इति स्वभाव उक्तः । तेन स्वभावेन अप्रतिबद्धः--तदप्रतिबद्धः । यो यत्र स्वभावेन न प्रतिबद्धः तस्य\footnote{लिङ्गस्य--\cite{dp-msD-n}} \footnote{स्वभावेन अप्रतिबद्धस्य--\cite{dp-msD-n}}तदप्रतिबद्धस्य तदव्यभिचारनियमाभावात्\footnote{०नियमाभावः--\cite{dp-msA} \cite{dp-edP} \cite{dp-edH} \cite{dp-edE}} । \footnote{तस्याप्रतिबद्धविषयस्य \cite{dp-msA} \cite{dp-edP} \cite{dp-edH} तस्य प्रति० \cite{dp-edN}}तस्याप्रतिबन्धविषयस्याव्यभिचारः तदव्यभिचारः, तस्य नियमः तदव्यभिचारनियमः, तस्याभावात् ।
	\pend
       

	  \pstart \footnote{अयमर्थः \cite{dp-msA} \cite{dp-msB} \cite{dp-msC} \cite{dp-msD} \cite{dp-edP} \cite{dp-edH} \cite{dp-edE} \cite{dp-edN}}तदयमर्थः--न हि यो यत्र स्वभावेन\footnote{स्वभावेन प्रति \cite{dp-msB}} न प्रतिबद्धः, स \footnote{तमप्रतिबद्धविष० \cite{dp-msA} \cite{dp-edP} \cite{dp-edH}}तमप्रतिबन्धविषयमवश्यमेव न व्यभिचरतीति नास्ति तयोरव्यभिचारनियमः--अविनाभावनियमः । अव्यभिचार-
	\pend
      ”

	  \pstart \textbf{स्वभावेन} स्वरूपेण । “साधनं कृता” \href{http://http://sarit.indology.info/?cref=vk-mbh.2.1.33}{व्या० महा० २. १. ३३}इति \textbf{पाणिनीयभाष्यकारस्येदं} सूत्रम् । तेन “कर्त्तृकरणे कृता बहुलम्” \href{http://http://sarit.indology.info/?cref=Pā.2.1.32}{पाणिनि २. १. ३२} इति सूत्रमपनीय गलेचोपक इत्यादिसिद्ध्यर्थ “साधनं कृता” इति सूत्रं कृतम् । “वार्त्तिकसूत्रि”काणां तु “तृतीया” \href{http://http://sarit.indology.info/?cref=Pā.2.1.30}{पाणिनि
	    २. १. ३०} इति योगविभागात्समासोऽवसेयः । अनेन च तृतीयासमासेनैव कार्यस्वभावयोः संग्रहादावृत्त्या षष्ठीसप्तमीसमासाभ्यां कार्यस्वभावयोः संग्रह इति यत्पूर्वैर्व्याख्यातं तदपव्याख्यानमिति ख्यापितम् ।
	\pend
      

	  \pstart समस्तस्य पदस्यार्थमाह--\textbf{स्वभावेति} । अनेन प्रतिबन्धशब्देन \textbf{प्रतिबद्धत्व}मायत्तत्वमुच्यत इति दर्शयति । अस्यैवार्थं स्पष्टयति । \textbf{प्रतिबद्धे}ति । यः स्वरूपेण क्वचिदायत्तस्तस्य स्वभावस्तत्र प्रतिबद्ध आयत्त इत्यर्थाभेदेन \textbf{प्रतिबद्धस्वभावत्वमित्यर्थ} इति स्पष्टीकृतम् ।
	\pend
      

	  \pstart ननु पूर्वेषामभिमतसमासव्युदासेन तृतीयासमासं दर्शयता किंस्विदति यो लब्धः ? केवलमाहोपुरुषिका प्रकाशितेत्याशङ्क्य पूर्वं बुद्धिस्थमेव स्फुटयितुमाह--\textbf{कारण} इति । कस्यासौ
	\pend
      \leavevmode\marginnote{\textenglish{111/dm}}“

	  \pstart नियमाच्च गम्यगमकभावः । न हि योग्यतया प्रदीपवत् परोक्षार्थप्रतिपत्तिनिमित्तमिष्टं लिङ्गम् अपि त्वव्यभिचारित्वेन निश्चितम् । ततः स्वभावप्रतिबन्धे \footnote{सत्यविनाभावनिश्चयः । \cite{dp-msA} \cite{dp-edP} \cite{dp-edH} \cite{dp-edE} \cite{dp-edN}}सत्यविनाभावित्वनिश्चयः, ततो गम्यगमकभावः । तस्मात् स्वभावप्रतिबन्धे सत्यर्थोऽर्थं गमयेन्नान्यथेति स्थितम् ॥
	\pend
       

	  \pstart ननु च परायत्तस्य प्रतिबन्धोऽपरायत्ते । तदिह साध्यसाधनयोः कस्य क्व प्रतिबन्ध इत्याह--
	\pend
      ”“

	  \pstart स च प्रतिबन्धः साध्येऽर्थे लिङ्गस्य ॥ २१ ॥
	\pend
      ”

	  \pstart प्रतिबन्ध इत्य काङ्क्षायामाह--\textbf{कार्यस्वभावयो}स्तयोरेव प्रकृतत्वात् । \textbf{अविशिष्टः} साधारणः । \textbf{द्वयोरपि} कार्यस्वभावयोरतिशये \textbf{सङ्ग्रहः} स्वीकारः । अयमेवातिशय इति भावः । साधनलक्षणोऽर्थः साध्यलक्षणमर्थं \textbf{गमयेत्} बोधयितुं शवनोति । यत एवं \textbf{तस्माद् अन्येषां} तद्व्यतिरिक्तानाम् । तेषां स्वभावप्रतिबन्धाभावात् । तदभावश्च तेषां तादात्म्यतदुत्पत्त्यभावात् । तदन्यस्य च सम्बन्धस्याभावात् । तादात्म्यतदुत्पत्तिभावे च कार्यस्वभावयोरेवान्तर्भाव इति भावः ॥
	\pend
      

	  \pstart \textbf{तदि}त्यादिना समासं प्रदर्श्य \textbf{तदप्रतिबद्धस्ये}ति योजयता \textbf{धर्मोत्त}रेण मूले \textbf{तदप्रतिबद्धस्येति} पाठो दर्शितः । दृश्यते च बहुशस्तदप्रतिबद्धस्वभावस्येति पाठः । तत्रापि पाठे तदप्रतिबद्धः साध्याप्रतिबद्धः स्वभावः स्वरूपं यस्य लिङ्गस्येति विग्रहः कार्यः । \textbf{प्रतिबन्धः} प्रतिबद्धत्वमायत्तत्वं यत्साधनस्य, तस्य \textbf{विषयोऽव्यभिचार}स्तेन विनाऽभवितृत्वम् । तस्य \textbf{नियमो}ऽवस्य\footnote{श्य}न्ता ।
	\pend
      

	  \pstart ननु तदप्रतिबद्धत्वेऽपि अद्यतन आदित्योदयोऽस्तमयमप्रतिबन्धविषयं न व्यभिचरतीत्याशङ्क्याह--\textbf{तदयमर्थ} इति । यत एवमुक्तम् \textbf{तत्तस्मादयं} तात्पर्यार्थः । आदित्योदयोऽपि हि भविष्यति । न च तदहरस्तमयः, महर्षिणाऽन्येन वा महर्धिना केनचित्तस्यास्तमयविबन्धसम्भवात् । तस्मादादित्यस्य तस्मादस्तमययोःयतैव साध्या--अयमादित्योदयोऽस्तमययोग्य उदयत्वात् । श्वस्तनोदयवदिति । सति चैवं स्वभावहेतुत्वमस्यायातमिति भावः ।
	\pend
      

	  \pstart ननु चाव्यभिचारमात्रेण प्रयोजनम्, तत्किं नियमेनेत्याह--\textbf{अव्यभिचारेति । चो}ऽवधारणे हेतौ वा । एतदेव कुत इत्याह--\textbf{न ही}ति । \leavevmode\marginnote{\textenglish{45b/ms}} \textbf{हि}र्यस्मात् । प्रदीपो वैधर्म्यदृष्टान्तः । \textbf{अपि} तु किन्त्व\textbf{व्यभिचारित्वेन} साध्यस्य प्रकृतत्वात् साध्याविनाभावित्वेन \textbf{निश्चितम्} । नियमाभावे च कुतोऽव्यभिचारनिश्चय इत्यस्य तात्पर्यार्थः । निश्चीयतां तदव्यभिचारोऽन्येषामपि, प्रतिबन्धस्तु कस्मान्मृग्यत इत्याह--तत इति । यतोऽवश्यमव्यभिचारो निश्चेतव्यस्ततस्तस्मात् । अन्यथाऽव्यभिचार एव न शक्यते निश्चेतुमित्यर्थादनेन दर्शितम् । एतावताऽपि कथं गमकत्वमित्याह--\textbf{तत} इति । ततो निश्चितादव्यभिचाराल्लिङ्गस्य गमकत्वे मिद्धे साध्यस्यापि गमक\footnote{गम्य}त्वं सिद्ध्यतीति द्वयोरुपयासः । उक्तमर्थमुपसंहरन्नाह--\textbf{तस्मादि}ति । \textbf{अर्थो} लिङ्गलक्षणः, \textbf{अर्थं} लिङ्गिलक्षणम् ॥
	\pend
      

	  \pstart \textbf{तत्तस्मादिहानु}मानानुमेयचिन्तायां स्वभावप्रतिबन्धचिन्तायां वा ।
	\pend
      \leavevmode\marginnote{\textenglish{112/dm}}“

	  \pstart स चेति । स च स्वभावप्रतिबन्धो लिङ्गस्य साध्येऽर्थे । लिङ्गं परायत्तत्वात् प्रतिबद्धम् । साध्यस्त्वर्थोऽपरायत्तत्वात् प्रतिबन्धवि\unclear{यो} न \footnote{न प्रतिब० \cite{dp-msA} \cite{dp-msB} \cite{dp-msD} \cite{dp-edP} \cite{dp-edH} \cite{dp-edE} \cite{dp-edN}}तु प्रतिबद्ध इत्यर्थः । तत्रायमर्थः--तादात्म्याविशेषेऽपि यत् प्रतिबद्धं\footnote{लिङ्गम्--\cite{dp-msD-n}} तद् गमकम् । यत् प्रतिबन्धविषयः तद् गम्यम् । यस्य च धर्मस्य \footnote{यन्नियतः \cite{dp-msA} \cite{dp-msB} \cite{dp-msC} \cite{dp-msD} \cite{dp-edP} \cite{dp-edH} \cite{dp-edE} \cite{dp-edN}}यो नियतः स्वभावः स तत्प्रतिबद्धः । यथा प्रयत्नानन्तरीयकत्वाख्योऽनित्यत्वे । यस्य तु स चान्यश्च स्वभावः च प्रतिबन्धविषयः, न तु प्रतिबद्धः । यथाऽनित्यत्वाख्यः प्रयत्नानन्तरीयकत्वाख्ये । निश्चयापेक्षो हि गम्यगमकभावः । प्रयत्नान्तरीयकत्वमेव चानित्यत्वस्वभावं निश्चितम् । अतस्तदेव अनित्यत्वे प्रतिबद्धम् । तस्मान्नियतविषय एव गम्यगमकभावः नान्यथेति ।
	\pend
      ”

	  \pstart ननु भिन्नयोर्गम्यगमकभावे लिङ्गं तदुत्पत्त्या परायत्तत्वात्प्रतिबद्धम् । साध्यस्त्वपरायत्तत्वात्प्रतिबन्धविषयः । तदाश्रयश्च साध्यसाधनभावनियमः स्यात् । स्वभावयोस्तु तद्भावे हेतोस्तादात्म्यप्रतिबद्धत्वम् । तादात्म्यं चोभयोरविशिष्टम् । ततः प्रतिवद्धत्वं प्रतिबन्धविषयत्वं वा द्वयोरविशिष्टमायाति । तदाश्रयश्च नियतः साध्यसाधनभावः प्रसक्त इत्याह—\textbf{तत्रायमर्थ} इति । न केवलमर्थान्तरत्व इत्यपिशब्दः । \textbf{यत्प्रतिबद्ध}मिति यत्साध्यप्रतिबद्धतया तदायत्ततया निश्चितमिति द्रष्टव्यम् । \textbf{प्रतिबन्धविषयो}ऽपि तत्त्वेन निश्चितो द्रष्टव्यः ।
	\pend
      

	  \pstart ननु तादात्म्याविशेषादेकस्तत्र प्रतिबन्धविषयतया न तु प्रतिबद्धतयेत्ययमेव विभागः कुत इत्याह--\textbf{यस्येति} । यद्वा तादात्म्यानुभाविनि द्वये किन्तत्र प्रतिबद्धं यद् गमकं किञ्च प्रतिबन्धविषयो यद् गम्यमित्यजानन्तं प्रत्याह--\textbf{यस्ये}ति । \textbf{चो} हेतौ द्वितीयपक्षेऽवधारणे । \textbf{यस्य} धर्मस्य व्यावृत्तिकल्पितस्य \textbf{यो नियतः} प्रतिनियतः स एव स्वभावो न तदन्योऽपि । स इति \textbf{यस्ये}ति षष्ठ्यन्तेनोक्तः परामृष्टः । \textbf{तदि}ति \textbf{य} इति प्रथमान्तेनोक्तः प्रत्यवमृष्टस्तस्मिन् प्रतिबद्ध इति विग्रहः । निश्चीयत इति शेषः । प्रकरणलभ्यं वा । कः पुनरीदृश इत्याह--\textbf{यथे}ति । \textbf{प्रयत्नः} पुरुषव्यापारस्तस्या\textbf{नन्तरम}व्यवधानम् । तत्र भव इति दिगादित्वाद् यत् । ततः स्वार्थे कन् । तस्य भावस्तत्त्वम् । तदाख्या नाम यस्य स तथा । प्रयत्नानन्तर्यकत्वस्य ह्यनित्यत्वमेव स्वभावो न तु नित्यत्वमपि । ततोऽनित्यत्वे प्रतिबद्धं निश्चीयते ।
	\pend
      

	  \pstart ईदृशस्य तावदियं गतिरन्यस्य तु का वार्त्तेत्याह--\textbf{यस्ये}ति । धर्मस्येत्यनुवृत्तेर्यस्येति धर्मस्य । तुर्विशिष्टं धर्मं दर्शयति । \textbf{स च} सोऽपि प्रयत्नानन्तर्याख्योऽन्यश्चासश्च\footnote{चा}प्रयत्नानन्तरीयको वनकुसुमादिरपि । \textbf{स प्रतिबन्ध}स्य साधनगतप्रतिबद्धत्वस्य \textbf{विषयो} गोचरः । एतदेव व्यतिरेकमुखेण द्रढयति--\textbf{न त्वि}ति । पुनरर्थे \textbf{तु}शब्दः । कोऽसावीदृश इत्याह--\textbf{यथे}ति । अनित्यत्वाख्यः प्रयत्नानन्तर्यत्वाख्येन प्रतिबद्ध इति योज्यम् । एवञ्चार्थात्प्रतिबन्धविषये चेत्यव\leavevmode\marginnote{\textenglish{46a/ms}}तिष्ठते । एतच्चानित्यत्वस्य तदसत्स्वभावत्वेनानियतस्वभात्वमनित्यत्वसामान्याभिप्रायेणोक्तम् । अन्यथा घटादिगतानित्यत्वस्य प्रयत्नानन्तरीयकत्वमन्तरेण कुतोऽवस्थानम् । येनान्यस्वभावतयाऽस्यानियतत्वं स्यादिति ।
	\pend
      \leavevmode\marginnote{\textenglish{113/dm}}“

	  \pstart कस्मात् पुनः स्वभावप्रतिबन्धो \footnote{लिङ्गस्य न वस्तुन इत्याह--\cite{dp-msB} \cite{dp-msD} \cite{dp-edH} लिङ्गस्य साध्येनेत्याह \cite{dp-edN}}लिङ्गस्येत्याह--
	\pend
       “

	  \pstart \footnote{वस्तुनः \cite{dp-edP}}वस्तुतस्तादात्म्यात्\footnote{तादात्म्यात्साध्यार्थादुत्प० \cite{dp-msB} \cite{dp-edP} \cite{dp-edH} \cite{dp-edE} \cite{dp-edN}} तदुत्पत्तेश्च ॥ २२ ॥
	\pend
      ” 

	  \pstart वस्तुत इत्यादि । स साध्योऽर्थ\footnote{स साध्यः स्वभावो--\cite{dp-msB} ऽर्थस्वभाव आत्मा यस्य \cite{dp-msD}} आत्मा स्वभावो यस्य तत् तदात्मा\footnote{तदात्म--\cite{dp-msC} \cite{dp-msD}} । तस्य भावस्तादात्म्यम्\footnote{तादात्म्यं तत्स्वभावत्वम् तस्मा० \cite{dp-msB}} । तस्माद्धेतोः । यतः साध्यस्वभावं साधनं तस्मात्\footnote{तत् तत्र \cite{dp-msB} \cite{dp-edN} \cite{dp-edH}} तत्र स्वभावप्रतिबन्ध\footnote{प्रतिबद्धमित्यर्थः--\cite{dp-msA} \cite{dp-msB} \cite{dp-edP} \cite{dp-edH} \cite{dp-edE} \cite{dp-edN}} इत्यर्थः ।
	\pend
       

	  \pstart यदि साध्यस्वभावं साधनं साध्यसाधनयोरभेदात् प्रतिज्ञार्थैकदेशो हेतुः स्यादित्या वस्तुत इति । परमार्थसता रूपेणाऽभेदस्तयोः । विकल्पविषयस्तु यत् समारोपितं रूपम् । तदपेक्षः साध्यसाधनभेदः । \footnote{विकल्प०--\cite{dp-msD-n}}निश्चयापेक्ष\footnote{निश्चयापेक्षया--\cite{dp-msB} \cite{dp-msC} \cite{dp-msD}} एव हि गम्यगमकभावः । ततो निश्चयारूढरूपापेक्ष एव तयोर्भेदो युक्तः वास्तवस्त्वभेद इति । न केवलात् \footnote{केवलं तादा० \cite{dp-msA} \cite{dp-msB} \cite{dp-msD} \cite{dp-edP} \cite{dp-edH} \cite{dp-edE} \cite{dp-edN}}तादात्म्यादपि तु ततः साध्यादर्थाद् उत्पत्तिर्लिङ्गस्य--तदुत्पत्तेश्च साध्येऽर्थे स्वभावप्रतिबन्धो लिङ्गस्य ॥
	\pend
      ”

	  \pstart स्यादेतत्तादात्म्यं तावत्तयोरस्ति । तत्किं प्रतिबद्धत्वप्रतिबन्धविषयत्वनिश्चयेन गम्यगमकत्वव्यवस्थानिबन्धनीकृतेनेत्याह--\textbf{निश्चये}ति । हीति यस्मात् । तदपेक्षायामपि किमिति प्रयत्नानन्तरीयकत्वमेव गमकमित्याह--\textbf{प्रयत्नेति । चो} यस्मादर्थे । यतस्तत्स्वभावं निश्चितम् अतोऽस्मात्तदेवानित्यत्वे प्रतिबद्धमुच्यत इति शेषः । यतस्तादात्म्याविशेषेऽपि यत्प्रतिबद्धतया निश्चितं तदेव गमकमितरद् गम्यं \textbf{तस्मान्नियतः} प्रतिनियतः प्रयत्नान्तरीयकमेव गमकम्, अनित्यत्व च गम्यमेवेत्येवंरूपो \textbf{विषयो} यस्य गम्यगमकभावस्य स तथा । एतदेव व्यतिरेकमुखेण द्रढयति \textbf{नान्यथे}ति ॥
	\pend
      

	  \pstart ननु च य एकस्यान्यत्र प्रतिबन्धस्तदायत्तत्वं स तावन्नाहेतुकः । कश्चासौ हेतुरित्यभिप्रेत्य प्रश्नयति--\textbf{कस्मादि}ति । निमित्तप्रश्नश्चैषः । \textbf{तादात्म्यादिति} मौलमुत्तर व्याख्यातुमाह--\textbf{यत} इति । \textbf{तत्र} साध्ये \textbf{स्वभावेन प्रतिबन्धः} प्रतिबद्धत्वं लिङ्गस्येति शेषः ।
	\pend
      

	  \pstart \textbf{प्रतिज्ञा} साध्यनिर्देशः । तस्या अर्थो धर्मधर्मिसमुदायः । अत्र च साध्यसाधनयोरैकात्म्यस्य प्रस्तुतत्वात्साध्यलक्षणस्य प्रतिज्ञार्थस्य हेतुत्वमासक्तम् । ततश्च साध्यधर्मवत्साधनधर्मस्याप्यसिद्धिः । सिद्धौ वा हेतुवैयर्थ्यमिति भावः । यदि परमार्थतोऽभेदः, कथं तर्हि भेदनिबन्धनो गम्यगमकभाव इत्याह--\textbf{विकल्पे}ति । तुः पारमार्थिकादभेदाद् वैधर्म्यमाह । कोऽसौ विकल्पविषयः ? यदि बाह्यस्तदा तदवस्थो दोष इत्याह--\textbf{यदि}ति । तमपेक्षत  \leavevmode\marginnote{\textenglish{114/dm}} “
	  
	कस्मान्निमित्तद्वयात् स्वभावप्रतिबन्धो लिङ्गस्य नान्यस्मादित्याह-- “
	  
	अतत्स्वभावस्यातदुत्पत्तेश्च तत्राप्रतिबद्धस्वभावत्वात् ॥ २३ ॥” 
	  
	अतत्स्वभावस्येति । स स्वभावोऽस्य\footnote{०भावो यस्य सो \cite{dp-msD} ०भावो यस्येति सो \cite{dp-msC}} सोऽयं तत्स्वभावः । न तत्स्वभावोऽतत्स्वभावः । तस्मादुत्पत्तिरस्य सोऽयं तदुत्पत्तिः । न तथाऽतदुत्पत्तिः । यो यत्स्वभावो यदुत्पत्तिश्च न भवति तस्य अतत्स्वभावस्य, अतदुत्पत्तेश्च । तत्र \footnote{असाध्य[[ध्येऽ]]कारणे च--\cite{dp-msD-n}}अतत्स्वभावे अनुत्पादके चाप्रतिबद्धः स्वभावोऽस्येति \footnote{०ति अप्रति० \cite{dp-msD}}सोऽयमप्रतिबद्धस्वभावः । तस्य \footnote{तस्य भावस्तस्मात्--\cite{dp-msD} \cite{dp-msC}}भावोऽप्रतिबद्धस्वभावत्वम् । तस्मादप्रतिबद्धस्वभावत्वात् । यद्यतत्स्वभावेऽनुत्पादके च कश्चित् प्रतिबद्धस्वभावो भवेद्, भवेदन्यतोऽपि\footnote{संयोगसमवायादेः--\cite{dp-msD-n}} निमित्तात्\footnote{निमित्तत्वात् स्व० \cite{dp-msC}} स्वभावप्रतिबन्धः । प्रतिबद्धस्वभावत्वं हि स्वभावप्रतिबन्धः । न चान्यः\footnote{संयोग्यादिः--\cite{dp-msD-n}} कश्चिदायत्तस्वभावः । तस्मात् तादात्म्यतदुत्पत्तिभ्यामेव स्वभावप्रतिबन्धः ॥” इति \textbf{तदपेक्षः} । इदं साधनमिदं साध्यमिति \textbf{साध्यसाधन}रूपो \textbf{भेदो} नानात्वमित्यर्थः । यदि नाम कल्पनानिर्मितो भेदस्तथापि कथं गम्यगमकभाव इत्याह--\textbf{निश्चये}ति । \textbf{हि}र्यस्मात् । \textbf{निश्चयापेक्ष} इति निश्चयविषयीकृतरूपापेक्ष इत्यर्थः । यत एवं ततस्तस्मात्तयोः साध्यसाधनयोर्भेदो नानात्वं युक्त्या सङ्गतो \textbf{युक्तः} । वस्तुनोऽकृत्रिमाद् रूपादागतो \textbf{वास्तवः} ।
	\pend
      

	  \pstart स्यान्मतम्--भेदेन कल्पितयोर्न तादात्म्यं गम्यगमकभावनिबन्धनमस्ति । वास्तवेन च रूपेणैकत्वान्न गम्यगमकभाव इति कथं स्वभावहेतोर्गमकत्वम् ? नाहेतुत्वम् । यद्दर्शनद्वारायातावेतौ धर्मौ तथाप्रतीयमानौ तत् तावत्परमार्थतस्तदात्मकमित्येकस्य धर्मान्तराव्यभिचारः । वास्तवं तादात्म्यगतं च यस्य गमकत्वं स स्वभावहेतुरुच्यत इति को विरोधः ?
	\pend
      

	  \pstart अयं प्रकरणार्थः--न निश्चयस्थे समारोपिते रूपे समारोपितत्वेनाध्यवसीयमाने गम्यगमके किन्तु स्वलक्षणत्वेनाध्यवसीयमाने । तत्र तादात्म्यमस्ति । एतदुक्तं भवति--आरोप्यमाणं रूपमारोपितभेद\leavevmode\marginnote{\textenglish{46b/ms}}म् । आरोपितसदृशं च स्वलक्षणम् । तेनारोपितेन रूपेणानुगम्यमानं भिन्नमघ्यवसीयते । तदवध्यवसितभेदनिबन्धनो गम्यगमकभावस्तस्य स्वतश्च तादात्म्यमिति ।
	\pend
      

	  \pstart द्वितीयं प्रतिबन्धकारणं व्याख्यातुमाह--\textbf{न केवला}दिति । \textbf{चः} साधारणं निमित्तं समुच्चिनोति ॥
	\pend
      

	  \pstart ननु च समवायादितोऽपि निमित्तात्प्रतिबन्धो नासम्भवी । तत्कथं तादात्म्यात्तदुत्पत्तेरेव च स उच्यते इत्यभिप्रायवान् पृच्छति--\textbf{कस्मादि}ति । अन्यनिमित्तस्यानभिधाना\textbf{न्निमित्तद्वयादि}त्याह ।
	\pend
      

	  \pstart अतत्स्वभावस्येत्यादि ब्रुवतश्चाचार्यस्यायमाशयः--भवेदेवान्यतः सम्बन्धात्प्रतिबन्धो  \leavevmode\marginnote{\textenglish{115/dm}} यदि समवायादिरन्यः सम्बन्धः प्रमाणबाधितो न भवेत् । न चासौ न बाध्यते । तत्कुतोऽसावसन्नस्य निमित्तं भवेदिति ।
	\pend
      

	  \pstart समुदायार्थं व्याचष्टे--यो यत्स्वभाव इति । अप्रतिबद्धस्वभावत्वादिति मूलस्य भावप्रत्ययं त्यक्त्वा विग्रहमाह--\textbf{अप्रतिबद्ध} इति । \textbf{तस्य भावस्}तत्त्वम् । \textbf{तस्मादि}ति तु योज्यम् । अमुमेवार्थं \textbf{यदी}त्यादिना स्फुटयति । कस्मात्पुनरन्यतो निमित्तान्न भवतीत्याह--\textbf{प्रतिब}द्धेति । हिर्यस्मात् । प्रतिबद्धस्वभावत्वमेवान्यस्यान्यत्र भविष्यतीत्याह--\textbf{न चे}ति । \textbf{चो}ऽवधारणे हेतौ वा । अन्यस्य सम्बन्धस्याभावात्तादात्म्यतदुत्पत्त्यभावे चेति प्रकरणलभ्यं कृत्वा \textbf{न चान्यः कश्चिदायत्तस्वभाव} इत्युक्तम् ।
	\pend
      

	  \pstart ननु चासत्यपि तादात्म्ये तदुत्पत्तौ चान्यत्रास्वभावेऽनुत्पादके चान्यत्प्रतिबद्धं यथा—आतपो वृक्षच्छायायाम् । तुलाया अर्वाग्भागनमनावनमने परभागोन्नमनावनमनयोः । अर्वाग्भागः परभागे । रसो रूपे । पाणिः पादयोः । अपतज्जलमाधारे । बलाका सलिले । नदीपूर उपरिवृत्तायां वृष्टौ । चन्द्रोदयः समुद्रवृद्धौ कुमुदविकासे च । कृत्तिकोदयो रोहिण्युदये । पिपिलिकोत्सरणं मत्स्यविकारश्च वृष्टौ । शरदि जलप्रसादोऽगस्त्योदये । विशिष्टो मेघो\add{द}यो वर्षकर्मणि । अद्यादित्योदयोऽस्तमये श्वस्तनोदये च । कुष्माण्डगुडकोऽन्तःस्थितबीजे । परिव्राजको दण्डे । सन्त्रस्तो नकुलः सर्पे । कियद् वा शक्यते निदर्शयितुम् ? एतावत्तूच्यते—यद् येनाविनाभूतं दृश्यते तत्तत्र प्रतिबद्धम् । तस्य च लिङ्गम् । अत एव त्रीण्येव लिङ्गानीति संख्यानियमोऽप्ययुक्तः । केवलं लिङ्गस्य रूपाण्येव वक्तव्यानि । यद्दर्शनात् हेतुत्वमवसीयत इति ।
	\pend
      

	  \pstart नैष दोषः । अमीषां मध्ये येषां प्रतिबन्धोऽस्ति तेषां तादात्म्यतदुत्पत्त्योरन्यतरसम्भवाद्, येषु च तदभावस्तेषामप्रतिबन्धादगमकत्वात् । तथाहि वृक्षस्य छायायामेकसामग्र्यधीनतयैव प्रतिबन्धः । ततस्तत्प्रतिपत्तिः कार्यलिङ्गजैव । छाया हि प्रतिभासमानरूपसंस्थानवती शैत्याद्यर्थक्रियाकारिणी वस्त्वेव, न त्वेवालोकाभावः । सा च पूर्वस्मादालोकोपादानात् पूर्ववृक्षक्षणाद् वृक्षक्षणेन सार्धमुत्पद्यते । तथाऽर्वाग्भागनमनावनमने अपि तुलायाः परभागोन्नमनावनमनाभ्यामेव समं पुरुषप्रयत्नादेव तथाविधात्तदुपादानसह\leavevmode\marginnote{\textenglish{47a/ms}}कारिण उत्पद्येते । तथाऽर्वाग्भागपरभागयोरसरूपयोरप्येकसामग्र्यधीनतैव । पाणिस्त्वप्रतिबद्ध एव, व्यङ्गस्यापि सम्भवात् । अव्यभिचारे चैकसंसर्गाधीनतैव निबन्धनम् । तादृशं च जलमाधारस्य कार्यमेव, तादृशी च बलाका सलिलस्य । नदीपूरोऽपि तथाविध उपरिवृष्टेः । दृश्यादृश्यसमुदायश्च यथायोगं सर्वत्र धर्मी कर्त्तव्यः । नदीपूरे चान्योऽपि प्रकारो वक्तुं शक्यः । नदी धर्मिणी । उपरिवृष्टिमद्देशसबन्धित्वमस्याः साध्यम् । तथाविधपूरत्वमात्रं हेतुः । एवं चन्द्रोदयसमुद्रवृद्धिकुमुदविकाशा\footnote{सा}नामप्येकसामग्र्यधीनतैव । एकस्मादेव महाभूतविशेषात् कालव्यवहारविषयादेतदुत्पादापेक्षिणस्तेषामुत्पादात् । तथा य एव कृत्तिकोदयहेतुर्महाभूतविशेषः कालसंज्ञितः स एव कतिपयकालव्यवधानेन रोहिण्युदयहेतुरिति तद्दर्शनाद् हेतोस्तज्जननयोग्यताधर्मोऽनुमीयते एव । तथा पिपीलिकोत्सरणस्य मत्स्यविकारस्य च यो हेतुः स एव कतिपयकालव्यवधानेन वर्षकरणयोग्यस्ततः पूर्ववद् हेतुधर्मानुमानम्, रूपरसयोरिवैकसामग्र्यधीनतयैव वा तत्समकालिकवर्षणानुमानम् । तदा त्वश्रवणीयबहिःस्थितशब्दगर्भगृहादिव्यवस्थितोऽनुमाता प्रत्येतव्यः । तथा शरदादिजलप्रसादोऽगस्त्योदयस्य कार्यमेव । अथ जलप्रसादं दृष्ट्वोदेष्यतीत्यनुमीयते, तदा तस्मादेव महाभूतात्कालसंज्ञिताज्जलप्रसादः । स \leavevmode\marginnote{\textenglish{116/dm}} “
	  
	भवतु नाम तादात्म्यतदुत्पत्तिभ्यामेव स्वभावप्रतिबन्धः । कार्यस्वभावयोरेव तुगमकत्वं कथमित्याह-- “
	  
	ते च तादात्म्यतदुत्पत्ती स्वभावकार्ययोरेवेति ताभ्यामेव वस्तुसिद्धिः ॥ २४ ॥” 
	  
	ते चेति । इतिः तस्मादर्थे । यस्मात् स्वभावे कार्ये एव च तादात्म्यतदुत्पत्ती स्थिते, तन्निबन्धनश्च गम्यगमकभावस्तस्मात् ताभ्यामेव कार्यस्वभावाभ्यां वस्तुनो विधेः\footnote{साध्यस्य--\cite{dp-msD-n}} सिद्धिः ॥ 
	  
	अथ प्रतिषेधसिद्धिरदृश्यानुपलम्भादपि कस्मान्नेष्टेत्याह-- “
	  
	प्रतिषेधसिद्धिरपि\footnote{०सिद्धिर्यथोक्ता० \cite{dp-edE}} यथोक्ताया एवानुपलब्धेः ॥ २५ ॥” 
	  
	प्रतिषेधव्यवहारस्य सिद्धिर्यथोक्ता या दृश्यानुपलब्धिस्तत एव भवति यतस्तस्मादन्यतो\footnote{अदृश्यानुपलब्धेः--\cite{dp-msD-n}} नोक्ता ॥ 
	  
	ततस्तावत् कस्माद्भवतीत्याह-- “
	  
	सति वस्तुनि तस्या\footnote{तस्यासम्भवात् \cite{dp-msC}} असम्भवात् ॥ २६ ॥” 
	  
	सति तस्मिन् प्रतिषेध्ये वस्तुनि, यस्माद् दृश्यानुपलब्धिर्न सम्भवति तस्माद्--असम्भवात् ततः प्रतिषेधसिद्धिः ॥” च कतिपयकालव्यवधानेन तदुदयनिमित्तमिति पूर्ववद् हेतुधर्मानुमानम् । मेघस्यापि तथाविधस्यात्यन्तायोग्यताव्यावृत्त्या वृष्टिकरणयोग्यताऽनुमेया । न तु भाविवर्षं व्यभिचारसम्भवात् । सा च स्वभावभूतैवानुमीयत इति तादात्म्यमेव निबन्धनम् । आदित्योदयस्य तु प्रभावातिशयवता योग्यादिना विबन्धसम्भवात् नास्त्येवाविनाभावः । अन्यथाद्य गर्दभदर्शनस्याप्यस्तमयश्वस्तनोदययोस्तथात्वं स्यात् । एवं तु युक्तम्--अयमुदेता अस्तमयश्वस्तनोदययोग्य इति । तथा चोदयतथाविधयोग्यतयोस्तादात्म्यमेव निबन्धनम् । कुष्माण्डस्यापि बीजेनैकसामग्र्यधीनतैव । परिव्राजकनकुलौ दण्डसर्पयोरप्रतिबद्धावेव । अन्यथापि सम्भवात् । कियद् वा शक्यते परिहर्त्तुम् ? एतावदुच्यते--असति तादात्म्ये तदुत्पत्तौ वा कस्यचित्क्वचित्प्रतिबन्धे ताद्रूप्येण च गमकत्वे सर्वं सर्वत्र प्रतिबद्धं तद्गमकं प्रसज्येतेति ॥
	\pend
      

	  \pstart \textbf{कार्यस्वभावयोरेव तु गमकत्वं कथमि}ति ब्रुवतः पूर्वपक्षवादिनोऽयमाशयः--तादात्म्यतदुपत्ती एवान्यस्य भविष्यतः, ततश्च गमकत्वमिति । कार्यस्वभावयोरिति द्वयो\add{रु}च्चारणे चायं तस्य भावः--भवद्भिरेवानुपलम्भोऽनयोरन्तर्भावित इति ॥
	\pend
      

	  \pstart \footnote{अस्पष्टम्--सं०}\add{... ... ...}नुपलब्धिरित्युक्ते कुतोस्य पूर्वपक्षस्योत्थानम् ? सत्यमेतत्, केवलं तदेवा\footnote{अस्पष्टम्--सं०}\add{... ... ...}मित्यस्या अप्यनुपल\leavevmode\marginnote{\textenglish{47b/ms}}ब्धेः  \leavevmode\marginnote{\textenglish{117/dm}} “
	  
	अथ तत एव कस्मादित्याह-- “
	  
	अन्यथा चानुपलब्धिलक्षणप्राप्तेषु देशकालस्वभाववि\footnote{०प्रकृष्टेष्वात्मप्र०--\cite{dp-msB} \cite{dp-edP} \cite{dp-edH} \cite{dp-edE} \cite{dp-edN}}प्रकृष्टेष्वर्थेष्वात्मप्रत्यक्षनिवृत्तेरभावनिश्चयाभावात् ॥ २७ ॥” 
	  
	अन्यथा चेति । सति वस्तुनि तस्या अदृश्यानुपलब्धेः सम्भवादित्यन्यथाशब्दार्थः । एतस्मात् कारणात् नान्यस्या\footnote{अदृश्यानुपलब्धेः--\cite{dp-msD-n}} अनुपलब्धेः प्रतिषेधसिद्धिः । 
	  
	कुत एतत्--सत्यपि वस्तुनि तस्याः सम्भव इत्याह--अनुपलब्धिलक्षणप्राप्तेष्वित्यादि । इह प्रत्ययान्तरसाकल्यात् स्वभावविशेषाच्चोपलब्धिलक्षणप्राप्तोऽर्थ\footnote{प्राप्तार्थ उक्तः \cite{dp-msA} \cite{dp-msB} \cite{dp-edP} \cite{dp-edH} प्राप्तेऽर्थे \cite{dp-msC}} उक्तः । \footnote{द्वयोरेकस्याप्य० \cite{dp-msB} द्वयोरेकैकस्याभावे--\cite{dp-msC}}द्वयोरेकैकस्याप्यभावेऽनुपलब्धिलक्षणप्राप्तोऽर्थ\footnote{प्राप्तेऽर्थे उच्यते--\cite{dp-msC}} उच्यते । 
	  
	तदिहानुपलब्धिलक्षणप्राप्तेष्विति प्रत्ययान्तरवैकल्यवन्त उक्ताः । देशकालस्वभावविप्रकृष्टेष्विति \footnote{स्वभावविशेषविप्रकृष्टाः \cite{dp-msA} \cite{dp-edP} \cite{dp-edH} \cite{dp-edE} \cite{dp-edN} स्वभावविप्रकृष्टा--\cite{dp-msC}}स्वभावविशेषरहिता उक्ताः । देशश्च कालश्च स्वभावश्च तैर्विप्रकृष्टा इति विग्रहः । तेष्वभावनिश्चयस्याभावात् । सत्यपि वस्तुनि \footnote{अदृश्यानुपलब्धेः--\cite{dp-msD-n} । तस्याभावः \cite{dp-edP} \cite{dp-edH} \cite{dp-edE} \cite{dp-edN}}तस्या भाव इष्टः । 
	  
	कस्मान्निश्चयाभाव इत्याह--तेषु प्रतिपत्तुरात्मनो यत् प्रत्यक्षं तस्य निवृत्तेः कारणात् निश्चयाभावः । यस्मादनुपलब्धिलक्षणप्राप्तेष्वात्मप्रत्यक्षनिवृत्तेरभावनिश्चयाभावः, तस्मात् सत्यपि वस्तुनि आत्मप्रत्यक्षनिवृत्तिलक्षणाया अदृश्यानुपलब्धेः सम्भवः । ततो यथोक्ताया एव प्रतिषेधसिद्धिः ॥” प्रतिषेधसम्भवादित्यभिप्रायेण पूर्वपक्षप्रवृत्तेरदोष एषः । \textbf{प्रतिषे}धशब्देन व्यवहारोऽभिप्रेत इति \textbf{प्रतिषेधव्यवहार} इति विवृतम् । मूले त्वपिशब्दः साध्यान्तरसमुच्चये । काऽसौ यथोक्तेत्याह--येति ।
	\pend
      

	  \pstart मूलसामर्थ्यस्थितमभिव्यनवित--\textbf{तस्मादन्यतो}ऽदृश्यानुपलब्धे\textbf{र्नोवता प्रतिषेधसिद्धिरि}ति प्रकृतेन सम्बन्धः ॥
	\pend
      

	  \pstart यद्यदृश्यानुपलब्धेर्न भवति तदाऽनुपलब्धित्वाविशेषे विवक्षिताया अपि मा भूदित्यभिप्रेत्याह--\textbf{ततस्ताव}दिति । तस्मादसम्भवात् कारणा\textbf{तत} इति तत एव दृश्यानुपलब्धेरिति विवक्षितमितरथाऽन्यस्या अपि प्रतिषेधसिद्धिकथनप्रसङ्गात् ॥
	\pend
      

	  \pstart ननु ततोऽप्यस्त्यन्यतोऽपि । कथं पुनस्तत एवेति नियमो लभ्यते इत्यभिप्रायवानाह—अथेति । \textbf{अथ}शब्दः प्रश्ने । \textbf{अन्यथा चे}त्युत्तरं व्याचक्षाण इहैवच्छेदं दर्शयति चशब्दञ्च यस्मादर्थे । दृश्यानुपलब्धेरुक्तात्प्रकाराददृश्यानुपलब्धेरन्यप्रकारत्वमन्यथात्वं विवक्षितमाचार्य  \leavevmode\marginnote{\textenglish{118/dm}} “
	  
	अथेयं दृश्यानुपलब्धिः कस्मिन् काले प्रमाणम्, किंस्वभावा, किंव्यापारा चेत्याह-- “
	  
	अमूढस्मृतिसंस्कारस्यातीतस्य वर्त्तमानस्य च प्रतिपत्तृप्रत्यक्षस्य \footnote{निवृत्तिरनुपलब्धिरभाव० \cite{dp-msC} \cite{dp-msD}}निवृत्तिरभावव्यवहारप्रवर्त्तनी\footnote{०हारसाधनी \cite{dp-msB} \cite{dp-msC} \cite{dp-msD} \cite{dp-edP} \cite{dp-edH} \cite{dp-edE} \cite{dp-edN}} ॥ २८ ॥” 
	  
	\footnote{अमूढेति \cite{dp-msA} \cite{dp-edP} \cite{dp-edE} नास्ति \cite{dp-edH} \cite{dp-edN}}अमूढेत्यादि । प्रतिपत्तुः प्रत्यक्षो घटादिरर्थः, तस्य निवृत्तिरनुपलब्धिः तद्भावस्यभावेति यावत् । अत एवाभावो न साध्यः स्वभावानुपलब्धेः, सिद्धत्वात् । \footnote{यथा च प्रतिपत्तृप्रत्यक्षनिवृत्तिरनुपलब्धिः प्रदेशस्तज्ज्ञानं चोच्यते तथा अविद्यमानोऽपीत्यादिना दर्शयति--\cite{dp-msD-n}}अविद्यमानोऽपि \footnote{०मानोपि घटा० \cite{dp-msC}}च घटादिरेकज्ञानसंसर्गिणि \footnote{०र्गिर्णि भास० \cite{dp-msA} \cite{dp-edE} \cite{dp-edP}}भूतले भासमाने समग्रसामग्रीको ज्ञायमानो \footnote{दृश्यमानतया \cite{dp-msB} \cite{dp-msD} \cite{dp-edH} \cite{dp-edN}}दृश्तया सम्भावितत्वात् प्रत्यक्ष उक्तः । अत एकज्ञानसंसर्गो\footnote{संसर्गात् \cite{dp-msB}} दृश्यमानोऽर्थस्तज्ज्ञानं च प्रत्यक्षनिवृत्तिरुच्यते । ततो हि दृश्यमानादर्थात् तद्बुद्धेश्च समग्रदर्शनसामग्रीकत्वेन प्रत्यक्षतया सम्भावितस्य निवृत्तिरवसीयते । तस्मादर्थज्ञाने एव प्रत्यक्षस्य घटस्याभाव उच्यते । न तु निवृत्तिमात्रमिहाभावः, निवृत्तिमात्राद् दृश्यनिवृत्त्यनिश्चयात् ।” स्येति दर्शयन्नाह--\textbf{सतीति । एतस्मा}त्सति वस्तुनि तत्सम्भवात् । मूले त्व\textbf{न्यथा चानुपलब्धिलक्षणप्राप्ते}ष्वित्येकवाक्यतयैवार्थः सङ्गच्छते । ना यशब्दार्थव्याख्या, नाप्युत्तरपदव्याख्याने पूर्वपक्षवचना\footnote{न}प्रयासः कश्चित् । तथा तु न प्रक्रान्तं \textbf{धर्मोत्तरेणेति} किमत्र कुर्मः ? \textbf{कुत एतदिति} सामान्येनोक्त्वा विशेषनिष्ठं करोति सत्य\textbf{पीति} । एतच्च \textbf{विग्रह} इत्यन्तं सुगमम् । \textbf{तेष्व}नुपलब्धिलक्षणप्राप्तेष्वभावनिश्चयाभावात् । सत्यपि वस्तुनि तस्या अनुपलब्धेर्भाव इष्टः । अनेनैतदाह--नास्माकमत्र प्रमाणमस्ति यत्सत्येव वस्तुनि सा भवतीति । किन्तु तस्यां सत्यामपि यस्मात्प्रत्ययो दोलायते तस्मादेवमुच्यत इति ।
	\pend
      

	  \pstart सम्प्रति निश्चयाभावस्यावधिं पर्येषमाण आह--\textbf{कस्मा}त्सकाशादिति । परप्रत्यक्षनिवृत्तेरशक्यनिश्चयत्वे तन्निवृत्त्यर्थमात्मप्रत्यक्षग्रहणम् । ननु यद्ययमपादानप्रश्नो न तु हेतुप्रश्नस्तदा कथमिदमाह--\textbf{तस्य निवृत्तेः कारणान्निश्चयाभाव} इति चेत् । न । अन्यार्थत्वात् कारणशब्दस्य । निश्चयाभावस्य शाब्देन न्यायेन जायमानस्यैषा प्रकृतिः कारणम् । “जनिकर्त्तुः प्रकृतिः” \href{http://http://sarit.indology.info/?cref=Pā.1.4.30}{पाणिनि १. ४. ३०}इत्यनेन लब्धापादानसंज्ञकादस्मादित्यर्थस्य विवक्षितत्वात् । अन्यथा त्वसमञ्जसं स्यात् । \textbf{यस्मादि}त्यादिनोवतमर्थमुपसंहरति । एतच्च \textbf{प्रतिषेधसिद्धिरित्येत}दन्तं सुगमम् ॥
	\pend
      

	  \pstart सम्प्रत्यनुपलब्धेरनुमानज्ञानहेतुत्वात् प्रामाण्यं स्वभावविशेषो व्यापारश्चोक्तोऽपि कालपुरुषविशेषपरिग्रहेण वक्तुम्--अथेत्यादिना प्रश्नपूर्वमुपक्रमते । अथशब्द आरम्भे पूर्ववत् ।
	\pend
      \leavevmode\marginnote{\textenglish{119/dm}}“

	  \pstart ननु च दृश्यनिवृत्तिरवसीयते दृश्यानुपलम्भात् । सत्यमेवैतत् । केवलमेकज्ञानसंसर्गिणि दृश्यमाने घटो यदि भवेद् दृश्य एव भवेदिति दृश्यः सम्भावितः । \footnote{विविक्तप्रदेशज्ञानात्०--\cite{dp-msD-n}}ततो दृश्यानुपलब्धिर्निश्चिता । \footnote{दृश्यानुपलम्भनिश्च \cite{dp-msC} \cite{dp-msD}}दृश्यानुपलब्धिनिश्चयसामर्थ्यादेव च\footnote{०देव दृश्या० \cite{dp-edE}} दृश्याभावो निश्चितः । \footnote{अमुमेवार्थं व्यतिरेकमुखेन भावयति--\cite{dp-msD-n}}यदि हि दृश्यस्तत्र भवेद् दृश्यानुपलम्भो न भवेत् । अतो दृश्यानुपलम्भनिश्चयाद् दश्याभावः सामर्थ्यादवसितः, न\footnote{न तु व्यव० \cite{dp-msA} \cite{dp-edP} \cite{dp-edH} \cite{dp-edE} \cite{dp-edN}} व्यवहृत इति दृश्यानुपलम्भेन व्यवहर्त्तव्यः ।
	\pend
       

	  \pstart तस्मादर्थान्तरम्--एकज्ञानसंसर्गि दृश्यमानम्, तज्ज्ञानं च प्रत्यक्षनिवृत्तिनिश्चयहेतुत्वात् प्रत्यक्षनिवृत्तिरुक्तं द्रष्टव्यम् ।
	\pend
      ”

	  \pstart प्रत्यक्षपरिच्छेद्यत्वात् \textbf{प्रत्यक्षो घटादिः । निवृत्ति}शब्देनाचार्यस्यानुपलब्धिर्विवक्षितेति दर्शयति \textbf{तस्य निवृत्तिरनुपलब्धिरिति} । अनुपलब्धिशब्देनापि विवक्षितकर्त्तृकर्मधर्मोपलब्धिपर्युदासेनान्यदेकज्ञानसंसर्गि वस्तु तज्ज्ञानं च विवक्षितम् । \leavevmode\marginnote{\textenglish{48a/ms}}एतदेव स्पष्टयति \textbf{तदभावस्वभावेति यावदि}ति । तस्य प्रतिषेध्यस्य घटादेरभावो विशिष्टो भावस्तत्स्वभावा । तदभावस्वभावशब्देन यावानर्थ उक्त तदनुपलब्धिशब्देनापीति इति यावदित्यस्यार्थः । यतोऽन्योपलब्धिरेव तदनुपलब्धिः । सैव च तदभावो नान्योऽत एवास्मादेव कारणादभावो घटादेर्न \textbf{साध्यः} । कुतो न साध्य इत्याह--\textbf{स्वभावानुपलब्धे}र्लिङ्गात् । कुतो न साध्य इत्याह—\textbf{सिद्धत्वा}त् निश्चितत्वात् घटाभावस्येति प्रकरणात् ।
	\pend
      

	  \pstart एवं मन्यते--तदेकज्ञानसंसर्गि वस्तु तज्ज्ञानं च घटाद्यनुपलब्धिस्तदभावश्च । तच्चेन्द्रियजेन प्रत्यक्षेण स्वसंवेदनेन च सिद्धमिति न लिङ्गादभावः साध्यत इति । नन्वविद्यमानो घटादिः कथं प्रत्यक्षः ? अथ प्रत्यक्षः, कथं तदनुपलब्धिरुच्यत इत्याह--\textbf{अविद्यमानोऽपि चे}ति । न केवलं विद्यमानः प्रत्यक्ष उच्यते इत्यपिशब्दः । \add{\textbf{समग्रा} समस्ता \textbf{सामग्री}}कारणकलापो यस्येति विग्रहः । “शेषाद्विभाषा” \href{http://http://sarit.indology.info/?cref=Pā.5.4.154}{पाणिनि ५. ४. १५४} इति कप् । “न कपि” \href{http://http://sarit.indology.info/?cref=Pā.7.4.14}{पाणिनि ७. ४. १४} इति ह्रस्वत्वप्रतिषेधः । \textbf{ज्ञायमानो} निश्चीयमानो \add{घटादिरेकज्ञान}संसर्गिणि प्र\footnote{अस्पष्टम्--सं०}...तदुपलब्धेरेवेत्यवधारणीयम् । \textbf{प्रत्यक्षनिवृ}त्तिर्घटाद्यनुपलब्धिः । कस्मात्तु द्वयं तथोच्यत इत्याह--ततो हीति । हीति यस्मात् । यस्मादमू एव दृश्यघटादितुच्छरूपनिवृत्त्यवसेयहेतू \textbf{तस्मात्} कारणात् । \textbf{अर्थ} एकज्ञानसंसर्गिवस्त्वन्तरम्, ज्ञानं च तस्यैव । एतदेव व्यतिरेकमुखेण द्रढयति--न त्विति । \textbf{निवृत्तिमात्रमु}पलब्ध्यभावमात्रम् । तस्य तथात्वे को दोष इत्याह--\textbf{निवृत्तिमात्रादि}ति । \textbf{दृश्यनिवृत्त्यनिश्चयाद्} दृश्याभावानिश्चयात् । इह निवृत्तिमात्रस्य प्रसज्यप्रतिषेधात्मनो निश्चेतुमशक्यत्वान्न हेतुत्वं युज्यत इति \textbf{धर्मोत्तरस्या}शयो \textbf{निवृत्तिमात्राद् दृश्यनिवृत्त्यनिश्चया}दिति ब्रुवतः । पूर्वपक्षवादिना त्वेवं ज्ञातम्--निवृत्तिमात्रान्निर्विशेषणादयमेवं प्रतिषेधति । तदहं सविशेषणं निवृत्तिमात्रमेव दर्शयामीति प्रमोदमान आह--\textbf{ननु चे}ति । \textbf{दृश्यनिवृत्ति}र्दृश्याभावः । \textbf{दृश्यानुपलम्भादि}त्यत्रानुपलम्भशब्देनोपलम्भाभावमात्रं विवक्षितमितरथा पूर्वपक्षवादिनः प्रकृतं हीयेत । अनेन सविशेषणमेवोपलम्भाभावमात्रं प्रसज्यप्रति  \leavevmode\marginnote{\textenglish{120/dm}} “
	  
	यथा\footnote{प्रतिपत्तृप्रत्यक्षस्यैतद् व्याख्याय अतीतस्य वर्त्तमानस्यैतद् विशेषणद्वयं व्याचष्टे--\cite{dp-msD-n}} चैकज्ञानसंसर्गिणि प्रत्यक्षे घटस्य प्रत्यक्षत्वमारोपितम् असतोऽपि, तथा तस्मिन्नेकज्ञानसंसर्गिण्यतीते\footnote{०तीते वर्त्तमाने चामूढस्मृतिसंस्कारे च घट० \cite{dp-msA} \cite{dp-msB} \cite{dp-msC} \cite{dp-msD} \cite{dp-edP} \cite{dp-edH} \cite{dp-edN}} चामूढस्मृतिसंस्कारे, वर्त्तमाने च घटस्य तत्त\footnote{तद्रूप \cite{dp-msA} \cite{dp-msB} \cite{dp-edP} \cite{dp-edH} \cite{dp-edE} \cite{dp-edN}}द्रू\footnote{अतीतादि--\cite{dp-msD-n}} पमारोपितमसत इति द्रष्टव्यम् । अनेन च\footnote{अनेन दृश्या \cite{dp-msA} \cite{dp-msB} \cite{dp-edP} \cite{dp-edH} \cite{dp-edE} \cite{dp-edN}} दृश्यानुपलब्धिः प्रत्यक्षघटनिवृत्तिस्वभावोक्ता । सा च सिद्धा । तेन न\footnote{तेन घटा० \cite{dp-msB}} घटाभावः साध्यः, अपि तु अभावव्यवहार इत्युक्तम् ।” षेधरूपं लिङ्गमस्तु, न तु नञः पर्युदासवृत्त्या तदेकज्ञानसंसर्गि वस्तु, तज्ज्ञानं चेति पूर्वपक्षवादी दर्शयति । \textbf{सत्य}मित्यादिना प्रतिविधत्ते । किन्त्वेक\textbf{ज्ञानसंसर्गिणि} भूतलादौ \textbf{दृश्यमाने} सति \textbf{दृश्यः सम्भावित} आरोपितः स प्रतिषेध्य इति प्रकरणात् । \textbf{ततो} दृश्यत्वसमारोपात् \textbf{दृश्यानुपलब्धि}र्दृश्यज्ञानाभावस्तुच्छरूपो व्यवहर्त्तव्यमात्रं \textbf{निश्चिता} भवति । निश्चीयतामुपलब्ध्यभावो ज्ञेयाभावस्तावन्न निश्चित इति तन्निश्चयार्थं निवृत्तिमात्रं व्यापरिष्यत इत्याह--\textbf{दृश्येति} । दृश्यज्ञानाभावनिश्चयसामर्थ्यादन्यथाऽनु\leavevmode\marginnote{\textenglish{48b/ms}}पपत्तेः । \textbf{चो} हेतौ । \textbf{दृश्यस्य} ज्ञेयस्या\textbf{भावो} व्यवहर्त्तव्यैकरूपः । सामर्थ्यमेव \textbf{यदी}त्यादिना दर्शयति । हिर्यस्मादर्थे । \textbf{दृश्यानुपलम्भ} इति दृश्योपलम्भाव इत्यर्थः । \textbf{अतो}ऽस्माद् \textbf{दृश्यानुपलम्भनिश्चयात्} । दृश्यस्य \textbf{ज्ञेयस्याभाव} उक्ता\textbf{त्सामर्थ्याद}वसितः । दृश्योपलम्भाभावनिश्चयस्त्वेकज्ञानसंसर्गिवस्त्वन्तरोपलग्भेनेति द्रष्टव्यम् ।
	\pend
      

	  \pstart ननु यदि ज्ञेयाभावोऽप्यवसितस्तर्हि कथं लिङ्गेन साध्यत इत्याह--\textbf{न व्यवहृत} इति । अदृष्टानामपि सत्त्वशङ्कया न प्रत्यक्षं व्यवहारयितुं शक्नोतीति भावः । केन तर्हि व्यवह्रियत इत्याह--\textbf{दृश्यानुपलम्भेन} लिङ्गभूतेन ।
	\pend
      

	  \pstart \footnote{अस्पष्टम्--सं०}...ऽभावव्यवहार एव \footnote{अस्पष्टम्--सं०}\add{... ... ...}\textbf{ज्ञानं चे}ति । चकारस्तुल्यकक्षतां दर्शयति । द्वयोश्च निवृत्तिनिश्चयहेतुत्वम्; ज्ञानमन्तरेण--“यस्मादयं केवलः प्रदेशस्तस्मात् घटादिर्नास्ति”इत्यध्यवसातुमशक्यत्वात्; तथा विषयमन्तरेण--“यस्मात्केवलप्रदेशापरोक्षीकरणं तस्मात् तज्ज्ञानं नास्ति” इति निश्चेतुमशक्यत्वादिति द्रष्टव्यम् । प्रत्यक्षस्य घटादे\textbf{निवृत्तिनिश्चयहेतुत्वात्प्रत्यक्षनिवृत्ति\footnote{अस्पष्टम्--सं०}\add{रुक्तं द्रष्टव्यम्}} ।
	\pend
      

	  \pstart ननु च प्रतिषेध्यस्य घटादेरसतोऽपि तथाऽस्तु प्रत्यक्षत्वम् । यत्पुनस्तत्र नासीदेव तस्य कथमतीतत्वम्; यच्च तत्र नास्त्येव तस्य कथं वर्त्तमानत्वम्, कथं च तत्रानुपलब्धेर्व्यापार इत्याशङ्कामपाकुर्वन्नाह--यथेति । येन प्रकारेण घटो यदि भवेद् दृश्य एव भवेदित्येवं रूपेण । \textbf{चो} हेतौ । \textbf{तस्मिन्नेकज्ञानसंसर्गिण्यतीतेऽमूढस्मृतिसंस्कारे । चो} वक्तव्यान्तरसमुच्चये । \textbf{वर्त्तमाने च} । समुच्चये चकारः । \textbf{तथा} तेन प्रकारेण--यदि तत्र पूर्वं घटः स्थितो यदि स्यात्, उपलब्धः स्यात्, न चोपलभ्यते--इत्येवमात्मना \textbf{घटस्या}रोपात् प्रत्यक्षस्य तदानी\textbf{मसतस्तद्रूपम}तीतत्वं वर्त्तमानत्वञ्\textbf{चारोपितमा}रोपसिद्धम् \textbf{इतिरे}वं द्रष्टव्याकारं दर्शयति । \textbf{द्रष्टव्यं} ज्ञातव्यमिति योजनीयमिदम् ।
	\pend
      \leavevmode\marginnote{\textenglish{121/dm}}“

	  \pstart \footnote{पदानां प्रयोजनं प्रतिपाद्येदानीं सम्बन्धमर्थं च दर्शयति--\cite{dp-msD-n}}अमूढोऽभ्रष्टो दर्शनाहितः स्मृतिजननरूपः संस्कारो यस्मिन् घटादौ स तथोक्तः । तस्य अतीतस्य प्रतिपत्तृप्रत्यक्षस्येति सम्बन्धः । वर्त्तमानस्य च प्रतिपत्तृप्रत्यक्षस्येति सम्बन्धः । अमूढस्मृतिसंस्कारग्रहणं तु न वर्त्तमानविशेषणम् । यस्मादतीते घटविविक्तप्रदेशदर्शने स्मृतिसंस्कारो मूढो दृश्यघटानुपलम्भे दृश्ये च घटे मूढो भवति । वर्त्तमाने तु\footnote{वर्त्तमाने च घट० \cite{dp-msA} \cite{dp-msB} \cite{dp-edP} \cite{dp-edH} \cite{dp-edE} \cite{dp-edN}} घटरहितप्रदेशदर्शने न स्मृतिसंस्कारमोहः । अत एव \footnote{अतएव न घटानुपलम्भे नापि घटे मोहः \cite{dp-msA} \cite{dp-msC} \cite{dp-msD} \cite{dp-edP} \cite{dp-edH} \cite{dp-edE} \cite{dp-edN} अत एव घटानुपलम्भे नापि घटे मोहः--\cite{dp-msB}}न घटाभावे, नापि घटानुपलम्भे मोहः । तस्मान्न वर्त्तमाननिषेध्यविशेषणममूढस्मृतिसंस्कारग्रहणम्, \footnote{०हणम् व्यभि० \cite{dp-msC}}स्मृतिसंस्कारव्यभिचाराभावाद् वर्त्तमानस्यार्थस्य । अत एव वर्त्तमानस्य चेति चशब्दः कृतः, विशेषणरहितस्य वर्त्तमानस्य विशेषणवतातीतेन समुच्चयो यथा विज्ञायेतेति\footnote{विज्ञायेत । तदय० \cite{dp-msB}} ।
	\pend
       

	  \pstart तदयमर्थः--अतीतोऽनुपलम्भः स्फुटः\footnote{स्फुटं \cite{dp-msB} \cite{dp-edP} \cite{dp-edH} \cite{dp-edE} \cite{dp-edN}} स्मर्यमाणः प्रमाणम्, वर्त्तमानश्च । ततो “नासीदिह घटः, अनुपलब्धत्वात्”, “नास्ति अनुपलभ्यमानत्वात्” इति शक्यं ज्ञातुम् । न तु “न भविष्यत्यत्र घटः, \footnote{अनुपलभ्यमानत्वात्--\cite{dp-msA}}अनुपलप्स्यमानत्वात्” इति\footnote{इति ज्ञातुं शक्यम्--\cite{dp-msC}} शक्यं ज्ञातुम् । अनागताया अनुपलब्धेः सत्त्वसन्देहादिति कालविशेषोऽनुपलब्धेर्व्याख्यातः ।
	\pend
      ”

	  \pstart ननु दृश्यानुपलब्धिः कस्मिन् काले प्रमाणं किंस्वभावा किंव्यापारा चेति त्रितयं पृष्ट आचार्यस्तत्रानेन किमाख्यातमाचार्येणेत्याशङ्कामपाकुर्वन्नाह--\textbf{अनेने}ति । \textbf{अनेन} प्रतिपत्तृप्रत्यक्षस्य निवृत्तिरिति वचनेन । \textbf{चो}ऽवधारणे । स्व\textbf{भावस्ये}\footnote{भावे}त्यस्यानन्तरं द्रष्टव्यः । प्रत्यक्षस्य घटस्य । प्रसज्यप्रतिषेधलक्षणनिवृत्तिनिश्चयहेतुत्वात्प्रत्यक्षघटनिवृत्तिः प्रत्यक्षघटानुपलब्धिः । नञः पर्युदासवृत्त्या । तदेकज्ञानसंसर्गि वस्तु तज्ज्ञानं च द्वयं \textbf{स्वभावो} रूपं यस्या दृश्यानुपलब्धेः सा तथोवता व्याख्यातेति शेषः ।
	\pend
      

	  \pstart ननु दृश्यानुपलब्धिरूपं लिङ्गमस्तु पर्युदासरूपम्, साध्यस्तु प्रसज्यप्रतिषेधरूपः किं न भवतीत्याह--\textbf{सेति । चो} यस्मात् । \textbf{सा} तथाविधाऽनुपलब्धि\add{... ... ...}उक्तमाचार्येण साऽ\textbf{भावव्यवहारप्रवर्त्तनी}त्यनेन शब्देनेति भावः । \textbf{धर्मोत्तरेण} चात एव \textbf{अभावो न साध्य} इत्यनेन ।
	\pend
      

	  \pstart घटविविक्तप्र\leavevmode\marginnote{\textenglish{49a/ms}}देशदर्शनेनाहित आरोपितः संस्कारो विशिष्टशक्तियुक्तविज्ञानात्मिका वासना, न तु पराभिमतो भावनाख्यः । अनुरूपस्मरणं जनयितुमनीशानो \textbf{मूढ} इति व्यपदिश्यते । \textbf{दृश्यघटानुपलम्भे} दृश्यस्य घटस्योपलब्ध्यभावे तुच्छरूपे । अत एव च दृश्ये \textbf{घटे} । अमूढस्मृतिसंस्कारग्रहणं कस्मान्न वर्त्तमानविशेषणमित्याह--\textbf{वर्त्तमाने त्वि}ति । तुरतीताद्वर्त्तमानस्य वैधर्म्यमाह । \textbf{अतो} मोहाभावान्न \textbf{घटाभावे} व्यवहर्त्तव्यैकरूपे । \textbf{नापि घटानुपलम्भे} घटोपलब्ध्यभावे तुच्छरूपे । प्रत्यक्षवदारोपाद् \textbf{वर्त्तमानस्यार्थस्य} घटादेः । \textbf{अत एव} वर्त्तमानग्रहणस्य निर्विशेषणत्वादेव ।
	\pend
      [[अस्पष्टम्--सं०]]\leavevmode\marginnote{\textenglish{122/dm}}“

	  \pstart व्यापारं दर्शयति । अभावस्य व्यवहारः “नास्ति” इत्येवमाकारं ज्ञानम्, शब्दश्चैवमाकारः, निःशङ्कं\footnote{निःशङ्कगमागमलक्ष० \cite{dp-msD} \cite{dp-msB} निःशङ्का गमनागमनयो [[?]] लक्ष० \cite{dp-msC}} गमनागमनलक्षणा च प्रवृत्तिः कायिकोऽभावव्यवहारः । घटाभावे हि ज्ञाते निःशङ्कं गन्तुमागन्तुं च प्रवर्त्तते ।
	\pend
       

	  \pstart \footnote{तदेवमस्य \cite{dp-msC} \cite{dp-msD} तदेवमेतस्य \cite{dp-msA} \cite{dp-edP} \cite{dp-edH} \cite{dp-edE} \cite{dp-edN} तदेव तस्य--\cite{dp-msB}}तदेतस्य त्रिविधस्याप्य\footnote{प्यभावस्य व्यव० \cite{dp-msC}}भावव्यवहारस्य दृश्यानुपलब्धिः \footnote{“प्रवर्त्तनी” इति नास्ति \cite{dp-msA} \cite{dp-msB} \cite{dp-msC} \cite{dp-msD} \cite{dp-edP} \cite{dp-edH} \cite{dp-edE} \cite{dp-edN}}प्रवर्त्तनी साधनी प्रवर्त्तिका ।
	\pend
       

	  \pstart यद्यपि च “नास्ति घटः” इति ज्ञानमनुपलब्धेरेव भवति, अयमेव चाभावनिश्चयः, तथापि यस्मात् प्रत्यक्षेण केवलः प्रदेश उपलब्धस्तस्मात् “इह घटो नास्ति” इत्येवं \footnote{त्येवं प्रत्य० \cite{dp-msC}}च प्रत्यक्षव्यापारमनुसरत्यभावनिश्चयः; तस्मात् प्रत्यक्षस्य केवलप्रदेशग्रहणव्यापारानुसार्यभावनिश्चयः प्रत्यक्षकृतः ।
	\pend
      ”

	  \pstart ननु यथा विकल्पेन विषयीक्रियमाणोऽतीतोऽनुपलम्भः प्रमाणमुच्यते तथा विकल्पस्याव्याहतप्रसरत्वादनागतोऽप्यनुपलम्भो विकल्प्यमानः किं न तथाप्रमाणमित्याह--तदयमिति । यस्मात्केवलप्रदेशदर्शनाहितः संस्कारोऽतीते घटादावमूढो गृह्यते, वर्त्तमाने तु तस्मिन् स्मृतिसंस्कारे मोहो न सम्भवत्येव तत्तस्मादयं तात्पर्यार्थः । अनुपलम्भनिश्चयहेतु\textbf{त्वादनुपलम्भः} केवलप्रदेशादिः \textbf{स्फुटो} यथाऽसौ केवलोऽनुभूतस्तथा स्मृत्वा विषयीक्रियमाणः \textbf{प्रमाणम् । वर्त्तमानश्च} तादृगनुपलम्भः \textbf{स्फुटो}ऽभ्रान्तेन ज्ञानेन गृह्यमाण इति द्रष्टव्यम् । यत ईदृशोऽनुपलम्भः प्रमाणं \textbf{तत}स्तस्मात् । कुतो न शक्यते ज्ञातुमित्याह--\textbf{अनागताया} इति । तथात्वेन निश्चितो हि हेतुर्गमकोऽन्यथा सन्दिग्धासिद्धता हेतुदोषः स्यादित्यभिप्रायः । प्रत्यक्षनिवृत्तिशब्देन तावद् दृश्यानुपलब्धेः स्वभावो दर्शितोऽ\textbf{मूढेत्या}दिना तु किं दर्शितमित्याह--कालेति । \add{\textbf{इतिरेवमर्थे ते}}नैवं \textbf{कालविशेषोऽनुपलब्धेर्व्याख्यात} इति । \textbf{कालविशेषो}ऽतीतो वर्त्तमानश्च \textbf{व्याख्यातः} कथितोऽनेनेति शेषः । एतच्चातीते वर्त्तमाने च कालेऽनुपलब्धेः प्रामाण्याख्यानमस्योपलक्षणं द्रष्टव्यम्, कार्यस्वभावहेत्वोरपि तयोः कालयोः प्रामाण्यात् । तथा ह्यासीदत्र वह्निर्धूमस्योपलब्धत्वात् । अस्ति वह्निरिह धूमस्योपलभ्यमानत्वात् । तथाऽसीदिह पादपः शिंशपाया उपलब्धत्वात् । अस्तीह वृक्षः शिंशपाया उपलभ्यमानत्वाद् इत्यपि भवत्येव ।
	\pend
      

	  \pstart ननु दृश्यानुपलम्भे भवतु ज्ञानाभिधानलक्षणो व्यवहारः । कायिकस्तु कथमित्याह--\textbf{घटे}ति । हिर्यस्मात् । \textbf{प्रवर्त्तत} इति योग्यतयोच्यते । प्रवृत्तियोग्यस्तावद् भवतीति । अत एव चाभावयोग्यता साध्योच्यते ।
	\pend
      

	  \pstart \textbf{तदेतस्ये}ति लोकोक्तिरेषा । तच्चैतच्चेति विग्रहः कार्यः । यद्वा यतोऽभावनिश्चयोऽनुपलब्धिनिमित्तकस्तत्तस्मात् । \textbf{अपि}रतिशये । आस्तामेकस्य द्वयोर्वा प्रवर्त्तनी, त्रिविधस्य यावत्प्रवर्त्तनीत्यर्थः । \textbf{प्रवर्तनी}त्यस्य द्वयमेतद् विवरणं स्पष्टार्थं \textbf{साधनी प्रवर्त्तिके}\leavevmode\marginnote{\textenglish{49b/ms}}\textbf{ति} प्रवर्त्तयतीति प्रवर्त्तनीति योग्यतयोक्तम् । साभावत्रितयमभावव्यवहारं प्रवर्त्तयितुं योग्या तावन्मात्रनिमित्तकत्वात्तस्य । सत्यर्थित्वे पुरुषस्तान् व्यवहारानाचरतु मा वा । अत एवाभावव्यवहारयोग्यता प्रदेशादेः साध्यते दृश्यानुपलम्भेनेति परमार्थः ।
	\pend
      \leavevmode\marginnote{\textenglish{123/dm}}“

	  \pstart किञ्च । दृश्यानुपलम्भनिश्चयकरणसामर्थ्यादेव पूर्वोक्तया नीत्या प्रत्यक्षेणैवाभावो निश्चितः । केवलमदृष्टानामपि सत्त्वसम्भवात्, सत्त्वशङ्कया न शक्नोत्यसत्त्वं\footnote{शक्नोत्यभावं व्य० \cite{dp-msC}} व्यवहर्त्तुम् । अतोऽनुपलम्भोऽभावं\footnote{०लम्भो व्यव० \cite{dp-msA}} व्यवहारयति--“दृश्यो यतोऽनुपलब्धः, तस्मान्नास्ति” इति । अतो दृश्यानुपलम्भोऽभावज्ञानं कृतं प्रवर्त्तयति, न तु अकृतं करोती\footnote{करोत्यभाव० \cite{dp-msC}}त्यभावनिश्चयोऽनुपलम्भात् प्रवृत्तोऽपि प्रत्यक्षेण कृतोऽनुपलम्भेन प्रवर्त्तित उक्त इत्यभावव्यवहारप्र\footnote{०हारे प्रवर्त्त्यनुपलब्धिः \cite{dp-msC} प्रवर्तन्युपल \cite{dp-msA} प्रवर्तिन्युपल \cite{dp-edP} \cite{dp-edH} प्रवर्तिन्यनु० \cite{dp-msB}} वर्त्तन्यनुलब्धिः ॥
	\pend
       

	  \pstart कस्मात् पुनरतीते वर्त्तमाने चानुपलब्धिर्गमिकेत्याह--
	\pend
       “

	  \pstart तस्या एवाभावनिश्चयात् ॥ २९ ॥
	\pend
      ” 

	  \pstart तस्या एव यथोक्तकालाया अनुपलब्धेरभावनिश्चयात् । अनागता ह्यनुपलब्धिः स्वयमेव सन्दिग्धस्वभावा । तस्या असिद्धाया नाऽभावनिश्चयोऽपि त्वतीतवर्त्तमानाया इति ॥
	\pend
      ”

	  \pstart ननु यदि नास्तीत्येवमाकारं ज्ञानमनुपलब्धेर्लिङ्गाद् भवति, कथं तर्हि प्रत्यक्षावसितोऽभाव उक्त इत्याह--\textbf{यद्यपि चे}ति निपातसमुदायो यद्यपिशब्दवद् विशेषाभिधाननिमित्ताभ्युपगमे वर्त्तते । अभावनिश्चयशब्दसामानाधिकरण्यादयमेवेति निर्देशः । चशब्दो वक्तव्यमेतदित्यस्यार्थेऽत्र वर्त्तते । अभावनिश्चयस्य तदा स्थैर्यलाभादनुपलम्भादेवेत्युक्तं द्रष्टव्यम् । \textbf{तथापी}ति लौकिकोक्तिरियं निगदाभिधानारम्भे । \textbf{तस्मा}च्छब्देन यस्माच्छब्द आक्षिप्तः । तेनायमर्थः । यस्मात्केवलप्रदेशग्रहणलक्षणव्यापारानुसारी घटो नास्तीत्यभावनिश्चयः \textbf{तस्मात्प्रत्यक्षकृत} उच्यत इति शेषः । कस्य व्यापारानुसारीत्याकाङ्क्षायामुक्तम्--\textbf{प्रत्यक्षस्य} प्रमाणविशेषस्येति ।
	\pend
      

	  \pstart अथ स्याद्--यद्ययं प्रत्यक्षकृतस्तदा प्रत्यक्षप्रवर्त्तितोऽपि । तत्किं दृश्यानुपलम्भेन क्रियत इत्याशङ्क्याह--\textbf{किञ्चे}ति वक्तव्यान्तरसमुच्चये । तद्विविक्तप्रदेशादिग्राहिणा \textbf{प्रत्यक्षेणैवाभावो} निषेध्याभावः प्रसज्यप्रतिषेधात्मा \textbf{निश्चि}तः । कथं ? \textbf{दृश्यस्यानुपलम्भ} उपलम्भाभावस्तुच्छरूपस्त\textbf{न्निश्चयकरणसामर्थ्या}दन्यथाऽनुपपत्तेः । \textbf{पूर्वोक्तया नीत्या} युक्त्या यदि हि दृश्यस्तत्र भवेत्, दृश्यानुपलम्भो न भवेदित्येवमात्मिकया । यदि प्रत्यक्षमित्थं प्रतिषेध्याभावं निश्चाययति, व्यवहारयितुमपि शक्नोत्येवेत्याह--\textbf{केवलं} किन्तु \textbf{न शक्नो}ति \textbf{व्यवहर्त्तुमित्य}न्तर्भूतणिजर्थत्वान्निर्देशस्य व्यवहारयितुमित्यर्थः । कुतो \textbf{न} शक्नोतीत्याह--\textbf{सत्त्वशङ्कये}ति प्रतिषेध्य\textbf{सत्त्वस्य} स्वरूपस्य \textbf{सत्त्वशङ्कया} सन्देहेन हेतुना ।
	\pend
      

	  \pstart नन्वदर्शनेऽपि कथं सन्देह इत्या\textbf{ह--अदृष्टे}ति । तेन प्रत्यक्षेणादृष्टानामपि पिशाचादीनां \textbf{सत्वस्य} सद्भावस्य \textbf{सम्भवात्} सम्भाव्यमानत्वात् नित्यं शङ्क्यमानानुपलम्भव्यभिचारो ह्यभाव इति भावः । अतः प्रत्यक्षस्य तत्राशक्तत्वात् साऽपि कथं व्यवहारयतीत्याह—दृश्य इति । अतोऽस्मात्कारणात्कृतं प्रत्यक्षेणेति प्रकरणात् । एतदेव व्यतिरेकमुखेण द्रडयति--\textbf{न त्वि}ति । यस्मात् प्रत्यक्षव्यापारानुसार्यभावनिश्चय \textbf{इति}स्तस्मात् । \textbf{अभाव}  \leavevmode\marginnote{\textenglish{124/dm}} “
	  
	सम्प्रत्यनुपलब्धेः प्रकारभेदं दर्शयितुमाह-- “
	  
	सा च प्रयोगभेदादेकादशप्रकारा ॥ ३० ॥” 
	  
	सा च एषानुपलब्धिः \footnote{ब्धिरेकादश प्रकारा अस्या \cite{dp-msA} \cite{dp-edP} \cite{dp-edE}}एकादशप्रकारा--एकादश प्रकारा अस्या इत्येकादशप्रकारा । 
	  
	कुतः प्रकारभेदः ? प्रयोगभेदात् । प्रयोगः प्रयुक्तिः शब्दस्याभिधाव्यापार\footnote{भिधानव्या० \cite{dp-msA} \cite{dp-msB} \cite{dp-msC} \cite{dp-msD} \cite{dp-edP} \cite{dp-edH} \cite{dp-edE} \cite{dp-edN}} उच्यते । शब्दो हि साक्षात् क्वचिदर्थान्तराभिधायी\footnote{शीतादिविरुद्धवह्न्यभिधायी--\cite{dp-msD-n}}, क्वचित् \footnote{वृक्षाप्रद्य [[?]] भावप्रतिषेधाभिधायी--\cite{dp-msD-n}}प्रतिषेधान्तराभिधायी । सर्वत्रैव तु दृश्यानुपलब्धिरशब्दोपात्तापि गम्यत इति वाचकव्यापारभेदादनुपलम्भप्रकारभेदो न तु स्वरूपभेदादिति यावत् ॥ 
	  
	प्रकारभेदान् आह--” “
	  
	स्वभावानुपलब्धिर्यथा--नात्र धूम उपलब्धिलक्षणप्राप्तस्यानुपलब्धेरिति ॥ ३१ ॥” निश्चयोऽनुपलम्भात् प्रवृत्तोऽपि दृढीभूतोऽपि \textbf{प्रत्यक्षेण} केवलप्रदेशादिवेदिना सामर्थ्यात् कृत उत्पादितोऽ\textbf{नुपलम्भेन} दृश्यानुपलम्भेन \textbf{प्रवर्त्तितः} साधित \textbf{उक्त} आचार्येणा\textbf{भावव्यवहारप्रवर्त्तनीत्यनेन} शब्देनेति बुद्धिस्थम् । \textbf{इतीत्या}दिनोपसंहारः । \textbf{इति}रेवमुक्तेन क्रमेणा\textbf{नुपलब्धि}र्दृश्यानुपलब्धिस्तस्या एव प्रकृतत्वात् ।
	\pend
      

	  \pstart \textbf{कस्मादित्यादि वर्त्तमानाया} इत्यन्तं सुगमम् ।
	\pend
      

	  \pstart अनागताया अप्यनुपलब्धेरभावनिश्चयः कस्मान्न भवतीत्याह--\textbf{अनागतेति । हि}र्यस्मादर्थे । \textbf{तस्या} इति पञ्चम्यन्तमिदम् । अयमेव च सामर्थ्यावसितो मौलो\add{र्थोऽ\textbf{नागता हि}} इत्यादिना \textbf{धर्मोत्तरेण\footnote{अस्पष्टम्--सं०}}...काङ्क्षोपशमार्थं पुरस्तादुक्तः ।
	\pend
      

	  \pstart स्यादेतत्--\textbf{प्रतिषेधसिद्धिरि}त्यादिना \textbf{निश्चयाभावादि}त्यन्तेन \leavevmode\marginnote{\textenglish{50a/ms}} ग्रन्थेनास्यार्थस्य \textbf{गतत्वात्तस्या एवाभावनिश्चयादि}त्ययमाचार्यीयो ग्रन्थः पुनरुक्त इति । न पुनरुक्तः । यतो यथाऽतीतेऽपि काले घटादेस्तत्कालवर्त्तिदृश्यानुपलब्धेरतीतायाः स्मृत्यारूढाया स्व\footnote{अ}भावनिश्चयस्तवाऽनागतेऽपि काले घटादेरभावनिश्चयः किं न भवतीति केनापाकृतं येनायं पुनरुक्तः स्यादिति ॥
	\pend
      

	  \pstart सम्प्रति \add{अनुपलब्धेः} \textbf{प्रकार}स्य स्वरूपस्य भेदं नानात्वं\footnote{अस्पष्टम्--सं०}...प्रकारभेद\footnote{अस्पष्टम्--सं०}... द्रष्टव्यम् । \textbf{चो} वक्तव्यान्तरसमुच्चये । \textbf{सेति} मूलानुवादः । \textbf{एषे}ति तस्य व्याख्यानम् । एकादशग्रहणं चाचार्यस्योपलक्षणार्थं यथा \textbf{प्रमाणवार्त्ति}के\footnote{अस्पष्टम्--सं०}...ग्रहणं\footnote{अस्पष्टम्--सं०}...षोडशप्रकारेति तु द्रष्टव्यम् । एतच्च कारणविरुद्धकार्योपलब्धिव्याख्यानानन्तरं दर्शयिष्यामः । विवक्षितान्योपलम्भैकरूपत्वाद् दृश्यानुपलब्धेः, कथमयं भेद उपपद्येतेत्यभिप्रेत्य पृच्छति \textbf{कुत} इति ।  \leavevmode\marginnote{\textenglish{125/dm}} “
	  
	स्वभावेत्यादि । प्रतिषेध्यस्य यः स्वभावस्तस्यानुपलब्धिर्यथेति । अत्रेति धर्मी । न धूम इति साध्यम्, उपलब्धिलक्षणप्राप्तस्यानुपलब्धेरिति हेतुः, अयं च हेतुः पूर्ववद्व्याख्येयः ॥ 
	  
	प्रतिषेध्यस्य यत् कार्यं तस्यानुपलब्धिरुदाह्नियते-- “
	  
	कार्यानुपलब्धिर्यथा--नेहाप्रतिबद्धसामर्थ्यानि धूमकारणानि सन्ति, धूमाभावादिति\footnote{“इति” नास्ति । \cite{dp-msB} \cite{dp-edP} \cite{dp-edH} \cite{dp-edE} \cite{dp-edN}} ॥ ३२ ॥” 
	  
	यथेति । इहेति धर्मी । अप्रतिबद्धम् अनुपहतं धूमजननं प्रति सामर्थ्यं येषां तान्यप्रतिबद्धसामर्थ्यानि न सन्तीति साध्यम् । धूमाभावदिति हेतुः ।” \textbf{प्रयोगभेदा}दित्याचार्यीयमुत्तरमनूद्य प्रयोगशब्दं व्याचष्टे \textbf{प्रयो}ग इति । \textbf{प्रयुक्त्य}र्थमाह--\textbf{शब्दे}ति । शब्दशब्देन प्रकरणाद् वाचकः शब्दो गृह्यते । \textbf{अभिधा} अर्थप्रकाशनम् । तत्र \textbf{व्यापारो} व्यापृतिः प्रवृत्तिः, यद्वाऽभिधा अर्थप्रकाशनं तल्लक्षणो व्यापारस्तस्य प्रयोगस्य भेदाद् भिद्यमानत्वादिति तु सुबोधत्वात् \textbf{धर्मोत्तरेण} न व्याख्यातम् ।
	\pend
      

	  \pstart ननु शब्दस्यैवानुपलम्भवाचकस्यानुपलम्भ एव वाच्यस्तत्कथं प्रयोगभेदो येनानुपलम्भस्य प्रकारभेद उच्यत इत्याह--शब्दो \textbf{हीति । हि}र्यस्मात् । \textbf{साक्षा}दव्यवधानेन । \textbf{क्वचिद्} विरुद्धोपलम्भादौ प्रतिषेध्याच्छीतस्पर्शादेरर्थान्तरमग्न्याद्यभिधत्ते । क्वचिद् व्यापकानुपलब्ध्यादौ विवक्षितात् शिंशपादिप्रतिषेधात्प्रतिषेधान्तरं वृक्षादिप्रतिषेधमभिधत्ते । यद्यर्थान्तरविधिरर्थान्तरप्रतिषेधश्च क्रियते तर्हि\footnote{अस्पष्टम्--सं०}\add{... ... ...}द्यत इत्याह--\textbf{सर्वत्रे}ति । \textbf{तुर्वि}शेषार्थः । अशब्दोपात्ता स्ववाचकपदानुपात्ता । यथा चाशब्दोपात्ताऽपि सा प्रतीयते तथा पुरस्तादभिधास्यते । \textbf{अपिरव}धारणे अतिशये वा । \textbf{इति}स्तस्मादर्थे एवमर्थे\footnote{अस्पष्टम्--सं०}\add{... ... ...}अत एवार्थीं गतिमाश्रित्योक्तं न तु शाब्दीमिति द्रष्टव्यम् ॥
	\pend
      

	  \pstart अनुपलम्भस्य प्रकृतत्वात् \textbf{प्रकारभेदान्}--इति मन्तव्यम् । \textbf{तस्यानुपलब्धिः} पूर्वोक्तया नीत्या तद्विविक्तः प्रदेशस्तज्ज्ञानं चावसेयम् । एवं कार्यानुपलब्ध्यादिषु द्विरूपैवाऽनुपलब्धि\footnote{अस्पष्टम्--सं०}...ति निदर्शनेन ।
	\pend
      

	  \pstart ननु कीदृश्युपलब्धिलक्षणप्राप्तिः ? कथञ्चाविद्यमान\footnote{अस्पष्टम्--सं०}\add{... ... ...}\textbf{अयञ्चेति । पूर्व}स्मिन्ननुपलब्धिलक्षणाख्यान इव । \textbf{चो}\leavevmode\marginnote{\textenglish{50b/ms}}यस्मादर्थे, अवधारणार्थे वा \textbf{पूर्ववदि}त्यस्यानन्तरं द्रष्टव्यः । एतच्च स्वभावानुपलम्भस्यार्थकथनं कृतमाचार्येण न प्रयोगो दर्शितः, असाधनाङ्गस्य प्रतिज्ञाया उपादानात् साधनाङ्गस्य च व्याप्तेरप्रदर्शनात् हेतोश्चानुवाद्यरूपस्य प्रथमान्तस्यानिर्देशात् । एवमितरासु सर्वास्वेवानुपलब्धिषु बोद्धव्यम् ।
	\pend
      

	  \pstart प्रयोगः पुनरीदृशः--यद् यत्रोपलब्धिलक्षणप्राप्तं सन्नोपलभ्यते तत्सर्वं तत्रासद्व्यवहारयोग्यम्--यथा--तुरङ्गमोत्तमाङ्गे शृङ्गम् । नोपलभ्यते चात्रोपलब्धिलक्षणप्राप्तो धूम इति । अनेन सामान्यादिनिराकरणे दृश्यानुपलम्भः प्रयोक्तव्य इति दर्शितमाचार्येण तुल्यन्यायत्वादित्य वसेयम् ।
	\pend
      \leavevmode\marginnote{\textenglish{126/dm}}“

	  \pstart कारणानि च नावश्यं कार्यवन्ति भवन्तीति कार्यादर्शनादप्रतिबद्धसामर्थ्यानामेवाभावः साध्यः,\footnote{साध्यते न त्व० \cite{dp-msC}} न त्वन्येषाम् । अप्रतिबद्धशक्तीनि चान्त्यक्षणभावीन्येव, अन्येषां प्रतिबन्धसम्भवात् ।
	\pend
       

	  \pstart कार्यानुपलब्धिश्च यत्र कारणमदृश्यं तत्र प्रयुज्यते । दृश्ये तु कारणे दृश्यानुपलब्धिरेव गमिका ।
	\pend
       

	  \pstart तत्र \footnote{०गृहस्योपरि० \cite{dp-msB} \cite{dp-msC} \cite{dp-msD}}धवलगृहोपरिस्थितो गृहाङ्गणमपश्यन्नपि चर्तुषु पार्श्वेष्वङ्गणभित्तिपर्यन्तं पश्यति । भित्तिपर्यन्तसमं \footnote{आलोकतमसी \add{आकाश} इति बौद्धमतम्--\cite{dp-msD-n}}चालोकसंज्ञकमाकाशदेशं धूमविविक्तं पश्यति । तत्र धूमाभावनिश्चयाद् । यद्देशस्थेन वह्निना जन्यमानो धूमस्तद्देशः स्यात् । तस्य च\footnote{एवार्थः--\cite{dp-msD-n} । तस्य वह्ने० \cite{dp-msB}} वह्नेरप्रतिबद्धसामर्थ्यस्याभावः प्रतिपत्तव्यः । तद्गृहाङ्गणदेशेन\footnote{तद्गृहाङ्गणदेशस्थेन वह्निना--\cite{dp-msC} \cite{dp-msD} तद्गृहाङ्गणदेशेन भ[[च]] \cite{dp-msA} तद्गृहाङ्गणस्थेन च वह्निना--\cite{dp-msB} तद्गृहाङ्गणदेशेन वह्नि \cite{dp-edP} \cite{dp-edH} \cite{dp-edE}} च वह्निना जन्यमानो धूमस्तद्देशः स्यात् । तस्मात् तद्देशस्य \footnote{अप्रतिबद्धसामर्थ्यस्येत्यर्थः--\cite{dp-msD-n}}वह्नेरभावः प्रतिपत्तव्यः ।
	\pend
       

	  \pstart तद्गृहाङ्गणदेशं भित्तिपरिक्षिप्तं भित्तिपर्यन्तपरिक्षिप्तेन चालोकात्मना धूमविविक्तेनाकाशदेशेन सह धर्मिणं करोति ।
	\pend
      ”

	  \pstart इह--अनुपलम्भः केवलप्रदेशादिः, अभावव्यवहारयोग्यता च साध्या, दृश्यानुपलम्भस्य तादात्म्यलक्षण एव प्रतिबन्धो गम्यगमकभावनिबन्धनम्--इति केषाञ्चिन्मतम् । केचित्तु—अनुपलम्भाऽभावव्यवहारयोग्यतयोर्गम्यगमकभावे विपर्ययतादात्म्यलक्षणः प्रतिबन्धो निमित्तमिति प्रतिपेदिरे । अत एवानुपलम्भः कार्यस्वभावाभ्यां भेदेन निर्दिष्टः, स्वसाध्ये प्रतिबन्धानपेक्षणाद्, इतरयोश्च तद्वैपरीत्यादिति च । एवं चैतत् समादधतीति । तथा हि--उपलब्ध्यव्यभिचारात् तादृशी अनुपलब्धेरेव सत्ता । तत उपलब्धिलक्षणप्राप्तसत्त्वे तदुपलम्भयोस्तादात्म्यादेवानुपलम्भासद्व्यवहारयोर्गम्यगमकभावः । यदि तु तद्विविक्तप्रदेशाद्युपलम्भरूपस्यानुपलम्भस्य तदसद्व्यवहारयोग्तायाश्च यत्तादात्म्यं तन्निमित्तमुच्येत, तदा तस्य भूतलादेरनुपलब्धिलक्षणप्राप्ताऽसद्व्यवहारयोग्यताऽप्यात्मभूतैवेति पिशाचाद्यभावव्यवहारमप्यनुपलम्भः साधयेत् तत्प्रतिबन्धादिति ॥
	\pend
      

	  \pstart ननु कार्याभावात् कारणाभावः साध्यताम् । कि\textbf{मप्रतिबद्धसामर्थ्यस्ये}ति साध्यधर्मविशेषणेनेत्याह--\textbf{कारणानीति । चो} यस्माद् । \textbf{इति}स्तस्मात् । \textbf{साध्यः} साधयितुं युज्यत इत्यर्थः । एतदेव व्यतिरेकमुखेण द्रढयति \textbf{न त्विति । तु}र्वैधर्म्यार्थ एवकारार्थो वा । \textbf{नान्येषां} कारणकारणतयोपचरितकुर्वद्रूपाणाम् ।
	\pend
      

	  \pstart अथ परिपूर्णया अपि सामग्र्या विधारकसम्भवात् प्रतिबन्धः सम्भाव्यते । तदसम्भवीदं  \leavevmode\marginnote{\textenglish{127/dm}} “
	  
	तस्माद् \footnote{दृश्यमानाका० \cite{dp-msC} \cite{dp-msA}}दृश्यमानादृश्यमानाकाशदेशावयवः प्रत्यक्षाप्रत्यक्षसमुदायो वह्नयभावप्रतीतिसामर्थ्यायातो धर्मी, न दृश्यमान\footnote{दृश्यमान इव \cite{dp-msC}} एव । “इह”--इति तु प्रत्यक्षनिर्देशो दृश्यमानभागापेक्षः । 
	  
	न केवलमिहैव दृश्यादृश्यसमुदायो धर्मी, अपि त्वन्यत्रापि । शब्दस्य क्षणिकत्वे साध्ये कश्चिदेव शब्दः प्रत्यक्षोऽन्यस्तु परोक्षस्तद्वदिहापि । यथा चात्र धर्मी साध्यप्रतिपत्त्यधिकरणभूतो दृश्यादृश्यावयवो दर्शितस्तद्वदुत्तरेष्वपि प्रयोगेषु स्वयं प्रतिपत्तव्यः ॥” विशेषणमित्याह--\textbf{अप्रतिबद्धे}ति चशब्दस्तुशब्दस्यार्थे हेतौ वा । कुत एतदित्याह—\textbf{अन्येषामि}ति । अन्त्यक्षणप्राप्तानि च कारणानि योगिनाऽपि प्रतिबद्धु\footnote{न्धु}मशक्यानि, इतरथा ताद्रूप्यमेव हीयेत । अतस्तथाविधानि कारणानि कार्याव्यभिचारीणि कार्येण व्याप्यन्ते । ततः कार्यं निवर्तमानं कारणानि तादृशानि निवर्त्तयतीति द्रष्टव्यम् ।
	\pend
      

	  \pstart इह यद्यप्यनेनान्त्यक्षणेनेदं कार्यं कृतमिति नावगतं तथापि वस्तुनः काचिदवस्था कार्यं कुर्वती प्रतीतैव यदनन्तरं कार्यमुपलब्धम् । सैव चावस्थाऽत्रापि निषिध्यते । अत एवायं क्षणिकाक्षणिकसाधारणावस्थातिशयनिषेधः ।
	\pend
      

	  \pstart ननु कार्यस्य ता\leavevmode\marginnote{\textenglish{51a/ms}}वद् दृश्यस्यात्रानुपलम्भो हेतुरुपादातव्यः । तत्किमस्य सम्भवोऽस्ति यत्कार्यमेव दृश्यं न तु कारणं येन कार्यानुपलब्धिः प्रयुज्यत इत्याशङ्क्य विषयमस्या दर्शयितुमाह--\textbf{कार्ये}ति ।
	\pend
      

	  \pstart कोऽसावेवंविधो विषय इत्याह--\textbf{तत्रे}ति वाक्योपक्षेपे ।
	\pend
      

	  \pstart यदि पुनर्गृहाङ्गणं पश्येत् दृश्यानुपलब्धिरेव प्रयुक्ता स्यादित्यभिप्राथेणाह--\textbf{गृहाङ्गणमपश्यन्नपीति । अपि}रवधारणे । \textbf{अङ्गणभित्तेः पर्यन्तम}वसानं । \textbf{भित्तिपर्यन्तेन समं} तुल्यम् ।
	\pend
      

	  \pstart \textbf{सौत्रान्तिका}नामालोकतमः स्वभाव एवाक श इति । तन्मयत्या\footnote{तन्मत्या} आलोकसंज्ञाकमित्युक्तम् ।
	\pend
      

	  \pstart अत्राकाशे किं प्रत्येतव्यमित्याह--\textbf{यद्देशे}ति । यदा \textbf{यद्देशे चे}ति पाठस्तदा यो देशोऽस्येति विग्रहः । यदा तु \textbf{यद्देशस्थे}नेति तदा यश्चासौ देशश्च तत्र तिष्ठतीति \textbf{तद्देशस्थ} इति । स देशोऽस्येति विग्रहः ।
	\pend
      

	  \pstart एतावताऽपि न ज्ञायते कस्तेन प्रतिपत्त्रा धर्मी कृत इत्याह--\textbf{तद्गृहे}ति । स चासौ \textbf{गृहाङ्गणदेश}श्चेति समासः । अङ्गणस्य प्रकृतत्वाद् \textbf{भित्तिः} तस्यैव, तया \textbf{परिक्षिप्तं} परिच्छिन्नं \textbf{धर्मिणं करोतीति} सम्बन्धः । किमेतावदेवेत्याह \textbf{भित्ती}ति । \textbf{भित्ति}रङ्गणस्यैव । \textbf{भि}त्तेश्च प्रान्तो वाच्यस्तस्याः \textbf{पर्यन्तोऽव}सानं निष्ठा तेन \textbf{परिक्षिप्तं} परिच्छिन्नं तेन । \textbf{च}स्तुल्यबलत्वसमुच्चयार्थः ।
	\pend
      

	  \pstart \textbf{तस्मादि}त्यादिनोपसंहारव्याजेन लोकाध्यवसायसिद्धं धर्मिणं दर्शयति । \textbf{दृश्यादृश्यसमुदायो} लोकेनैकत्वेनाकलितः । तावान् देशो दृश्यमानः किमात्मकः ? \textbf{देशावयवो} भागो यस्य  \leavevmode\marginnote{\textenglish{128/dm}} “
	  
	प्रतिषेध्यस्य व्याप्यस्य यो व्यापको धर्मस्तस्यानुपलब्धिरुदाह्नियते--” स तथा । कथमेवंविधो धर्मीत्याह--\textbf{वहन्यभावेति} । वह्निरिति विशिष्टो धूमजननेऽव्यवधेयशक्तिः । तस्याप्युपलक्षणत्वादन्यस्य धूमजननेऽप्रतिबद्धस्याभावप्रतीतिर्ग्रहीतव्या ।
	\pend
      

	  \pstart अयमर्थः--लोकस्तावत्तथाविधदेशे धूममनुपलभमानस्तावति देशे तथाभूतवह्न्यभावं प्रत्येति । न चैतदेवंविधं धर्मिणमन्तरेण घटत इति \textbf{सामर्थ्यम}न्यथाऽनुपपत्तिस्तस्माद् \textbf{आयात} उपस्थितः । अत एवाचार्येणापि “
	    \pend
	  
	    
	    \stanza[\smallbreak]
	इष्टं विरुद्धकार्येऽपि देशकालाद्यपेक्षणम् ।\&[\smallbreak]


	
	    \pstart
	  ” \href{http://http://sarit.indology.info/?cref=pv.3.5}{प्रमाणवा० ३. ५} इति ब्रुवतैवं धर्मीष्ट एवेति भावः ।
	\pend
      

	  \pstart नन्वप्रत्यक्षस्यापि धर्मित्वे कथमिहेतीदमो हप्रत्ययान्तस्य निर्देश इत्याशङ्क्याह—इहेति । तुशब्दो यस्मादर्थे । \textbf{इति}रिहशब्दस्य निर्दिष्टस्याकारं प्रत्यवमृशति । प्रत्यक्षवस्तुप्रतिपादको निर्देशस्तथोक्तः ।
	\pend
      

	  \pstart ननु यदि दृश्यादृश्यसमुदायो धर्मीं घटते तदा कार्यस्वभावहेत्वोरपि किमयं न सम्भवतीत्याह--\textbf{न केवलमि}ति । \textbf{इहैव} कार्यानुपलम्भ एव । \textbf{अपि तु} किन्त्व\textbf{न्यत्रापि} कार्यविशेषे स्वभावविशेषे च तत्र तत्र कार्यहेतोर्धूम-वाक्यादग्नि-पौरुषेयत्वादिसिद्धौ तथाविधो धर्मी सुव्यक्त इति न तत्र दर्शितस्तुल्यत्वा\footnote{न्या}यतया वा \textbf{द्रष्टव्यः} ।
	\pend
      

	  \pstart कथं हेतावप्येवंविधस्य धर्मिणः सम्भव इत्थाह--\textbf{शब्दस्ये}ति । \textbf{कश्चिदेव} श्रूयमाणः \textbf{प्रत्यक्षोऽन्यस्त्व}श्रूयमाणः \textbf{परोक्षः} । अनेनावश्यं दृश्यादृश्यशब्दसमुदायोऽत्र धर्मीति दर्शितम् । दृश्यादृश्यात्मना च शब्देन धर्मिणा भाव्यमिति प्रतिपाद्यजनसंशयारोपाभ्यामायातम् । नहि प्र\leavevmode\marginnote{\textenglish{51b/ms}}तिपाद्यः शब्दस्यानित्यत्वे संशयानो विपर्यस्यति, अर्थात् घटशब्द एव, श्रूयमाण एव च सन्देग्धि, विपर्यस्यति वा । किन्त्विहे\footnote{ह}घटपटादिशब्दे श्रूयमाणे श्रुते श्रोष्यमाणे च । ततस्तदनुरोधात् प्रत्यक्षाप्रत्यक्षशब्दसमुदायोऽनित्यत्वे साध्ये धर्मीं प्रसह्य पतितः । \textbf{तद्वदिहापि कार्यानुपलम्भे} ।
	\pend
      

	  \pstart अमुमेव न्यायमन्यत्राप्यतिदिशन्नाह--\textbf{यथा चेति । चो}ऽवधारणे \textbf{साध्यप्रतिपत्त्यधिकरणभूत} इति विशेषणव्याजेन धर्मिणो लक्षणमुक्तम् ।
	\pend
      

	  \pstart कार्यानुपलब्धिप्रयोगस्त्वेवं कर्त्तव्यः । यत्र यस्य कार्यमुपलब्धिलक्षणप्राप्तं नोपलभ्यते तत्तु तज्जननाप्रतिबद्धसामर्थ्यं नास्ति । यथा क्वचिद् दृश्यमानेऽङ्कुरे तथाविधं बीजम् । नोपलभ्यते चात्रोपलब्धिलक्षणप्राप्तो धूम इति ।
	\pend
      

	  \pstart अनेन च कार्यानुपलम्भप्रदर्शनेन यज्जल्पितं \textbf{जल्पमहोदधिना} “निःशब्दे देशे शब्दमात्राभावे साध्ये कस्तदेकज्ञानसंसर्गिवस्त्वन्तरोपलम्भो येन शब्दाभावव्यवहारो बौद्धानां भवेद्” इति तत्प्रत्युक्तं द्रष्टव्यम् । तथाहि--नेहाप्रतिबद्धसामर्थ्यानि श्रोत्रज्ञानकारणानि सन्ति । श्रोत्रज्ञानाभावादिति कार्यानुपलब्धिः स्फुटैव । श्रोत्रज्ञानानुपलम्भश्च तदन्यज्ञानोपलम्भरूपः संविदितोऽस्त्येव । एकज्ञानसंसर्गित्वं चान्योन्याव्यभिचरितोपलम्भत्वमित्युक्तं पुरस्तात् । प्रयोगस्त्वनन्तरवद्विज्ञातव्य इति ॥
	\pend
      \leavevmode\marginnote{\textenglish{129/dm}}““

	  \pstart व्यापकानुपलब्धिर्यथा\footnote{नुपलब्धेर्य० \cite{dp-msC}}--नात्र शिंशपा, वृक्षाभावादिति\footnote{“इति” नास्ति \cite{dp-edE}} ॥ ३३ ॥
	\pend
      ” 

	  \pstart यथेति । अत्रेति\footnote{अत्र धर्मी--\cite{dp-msA} \cite{dp-msB} \cite{dp-edP} \cite{dp-edH} \cite{dp-edE}} धर्मी । न शिंशपेति शिंशपाऽभावः साध्यः । वृक्षस्य व्यापकस्याभावादिति हेतुः ।
	\pend
       

	  \pstart इयमप्यनुपलब्धिर्व्याप्यस्य शिंशपात्वस्या\footnote{शिंशपात्वदृश्यस्याभाव \cite{dp-msB} शिंशपात्वस्य दृश्याभावे \cite{dp-msA} \cite{dp-edP} \cite{dp-edH} \cite{dp-edE} \cite{dp-edN} शिंशपात्वस्य दृश्यस्याभावे--\cite{dp-msC} \cite{dp-msD}}ऽदृश्यस्याभावे\footnote{साध्ये \cite{dp-msD-n} ।} प्रयुज्यते । उपलब्धिलक्षणप्राप्ते तु व्याप्ये दृश्यानुपलब्धिर्गमिका । तत्र यदा पूर्वापरावुपश्लिप्टौ समुन्नतौ देशौ भवतः, तयोरेकस्तरुगहनोपेतो\footnote{०पेताऽपर \cite{dp-msC}}ऽपरश्चैकशिलाघटितो निर्वृक्षकक्षकः\footnote{तृणोत्करः--\cite{dp-msD-n} । ०कक्षः । द्र० \cite{dp-msA} \cite{dp-msB} \cite{dp-msC} \cite{dp-msD} \cite{dp-edP} \cite{dp-edH} \cite{dp-edE} \cite{dp-edN}} । द्रष्टापि तत्स्थान् वृक्षान् पश्यन्नपि शिंशपादिभेदं\footnote{०भेदं न यो विवे--\cite{dp-msA} \cite{dp-edP} \cite{dp-edH} \cite{dp-edE}} यो न विवेचयति, तस्य वृक्षत्वं प्रत्यक्षं अप्रत्यक्षं\footnote{०त्यक्षं शिंश० \cite{dp-msA} \cite{dp-msB} \cite{dp-msC} \cite{dp-edH} \cite{dp-edE} \cite{dp-edN}} तु शिंशपात्वम् । स हि निर्वृक्ष एकशिलाघटिते वृक्षाभावं दृश्यत्वाद् दृश्यानुपलम्भादवस्यति । शिंशपात्वाभावं तु व्यापकस्य वृक्षत्वस्याभावादिति । तादृशे\footnote{तादृशविष० \cite{dp-msA}} विषयेऽस्या\footnote{स्याः प्रयोगोऽभावसाधनाय \cite{dp-msC}} अभावसाधनाय प्रयोगः ॥
	\pend
       “

	  \pstart स्वभावविरुद्धोपलब्धिर्यथा--नात्र शीतस्पर्शो \footnote{अग्नेरिति--\cite{dp-msB} \cite{dp-edP} \cite{dp-edH} \cite{dp-edE} \cite{dp-edN}}वह्नेरिति ॥ ३४ ॥
	\pend
      ” 

	  \pstart प्रतिषेध्यस्य स्वभावेन विरुद्धस्योपलब्धिरुद्राह्रियते \footnote{०ह्रियते । अत्रेति--\cite{dp-msB} \cite{dp-msC} \cite{dp-msD}}यथेति । अत्रेति धर्मी । न शीतस्पर्श \footnote{“इति” नास्ति \cite{dp-msA}}इति शीतस्पर्शप्रतिषेधः साध्यः । वह्नेरिति हेतुः । इयं चानुपलब्धिस्तत्र प्रयोक्तव्या यत्र शीतस्पर्शोदृश्ऽयः, दृश्ये\footnote{दृश्ये तु दृश्या० \cite{dp-msB} \cite{dp-msC} \cite{dp-msD}} दृश्यानुपलब्धिप्रयोगात् ।
	\pend
      ”

	  \pstart व्यापकानुपलब्धिं व्याख्यातुमाह--\textbf{प्रतिषेध्यस्ये}ति । तादात्म्याविशेषेऽपि यथा कश्चिदेव धर्मो व्याप्य इतरो व्यापकश्च तथा प्रागेव \textbf{धर्मोत्तरेण} निर्णीतम् । \textbf{धर्म} इति च धर्म्यपेक्षया वृक्षत्वादि । \textbf{न शिंशपे}ति न शिंशपात्वमित्यर्थः । \textbf{वृक्षस्ये}ति च धर्मिणा धर्मस्य वृक्षत्वस्य निर्देशः ।
	\pend
      

	  \pstart ननूपलब्धिलक्षणप्राप्तस्य तावद् वृक्षत्वस्यानुपलब्धिः प्रयोक्तव्या । तथा च शिंशपात्वमपि दृश्यमेव निषेध्यमिति दृश्यानुपलब्धिरेव प्रयोगार्हेत्याह--\textbf{इयमपी}ति । न केवलं पूर्विका विशिष्टे विषये किन्त्वि\textbf{यमपी}त्यपिशब्दः । \textbf{शिंशपात्वस्यादृश्यस्येति}--यदि स्याद् दृश्यमेव स्यादिति सम्भावनामतिवृत्तस्येत्यर्थः । एवमुत्तरत्राप्यदृश्यत्वमीदृशमेव द्रष्टव्यम् । तस्याभावे साध्य इत्यध्याहारः ।
	\pend
      \leavevmode\marginnote{\textenglish{130/dm}}“

	  \pstart तस्माद् यत्र वर्णविशेषाद् वह्निर्दृश्यः, शीतस्पर्शो दूरस्थत्वात्\footnote{दूरत्वात् \cite{dp-msB} \cite{dp-msD}} सन्नप्यदृश्यः, तत्रास्याः प्रयोगः ॥
	\pend
       “

	  \pstart विरुद्धकार्योपलब्धिर्यथा--नात्र शीतस्पर्शो धूमादिति ॥ ३५ ॥
	\pend
      ” 

	  \pstart प्रतिषेध्येन यद् विरुद्धं तत्कार्यस्योपलब्धिर्गमिका--यथेति । अत्रेति धर्मी । न शीतस्पर्श इति शीतस्पर्शाभावः साध्यः । धूमादिति हेतुः । यत्र शीतस्पर्शः सन् दृश्यः
	\pend
      ”

	  \pstart कोऽसावेवंविधो विषय इत्याह--तत्रेति वाक्योपन्यासे । \textbf{उपश्लिष्टौ} प्रत्यासन्नौ \textbf{समुन्न}तावुच्चौ । तरूणां \textbf{गहनं} गह्वरं \textbf{तेनोपेतो} युक्तः । द्वितीय \textbf{एकया शिलया घटितो} निर्मित एकशिलारूपस्तूप इति यावत् । तत्त्वेनैव च \textbf{निर्वृक्षकक्षकः । कक्ष}स्तृणम् । निर्गतौ वृक्षकक्षौ यत इति विग्रहः । भवत्वेवं तथापि कथमस्याः प्रयोग इत्याह--\textbf{द्रष्टापीति । अपिर}वधारणे—\textbf{न विवेचयती}त्यस्यानन्तरं द्रष्टव्यम् \footnote{व्यः} । \textbf{तस्य} तादृशस्य द्रष्टुः ।
	\pend
      

	  \pstart भवतु द्रष्टु\add{स्}तावद् दूरदेशस्थायितया शिंशपाया अविवेकस्तथापि येनैव वृक्षाभावं प्रतिपद्यते तेनैव शिंशपाऽभावमपि किं न प्रतिपद्यत इत्याह--\textbf{स ही}ति । \textbf{हि}र्यस्मात् । कथं दृश्यानुपलम्भादवस्यतीत्याह--\textbf{दृश्यत्वाद्} वृक्षत्वस्येति प्रकरणात् । कुतस्तर्हि शिंशपा\leavevmode\marginnote{\textenglish{52a/ms}}त्वाभावमवैतीत्याह--\textbf{शिंश}पेति । \textbf{तु}शब्दो वैधर्म्ये । \textbf{इति}स्तस्मादर्थे । \textbf{अभाव}शब्देनाभावोऽभावव्यवहारश्चोक्तो द्रष्टव्यः । एवमुत्तरत्रापि प्रत्येयम् ।
	\pend
      

	  \pstart प्रयोगः पुनरीदृशः कार्यः--यत्र यस्य व्यापकं नास्ति न तत् तत्रास्ति । यथा--असति प्रमेयत्वे प्रामाण्यम् । नास्ति च वृक्षत्वं शिंशपात्वस्य व्यापकमिति । अनेन व्यापकानुपलम्भस्य व्याप्याभावे गमकत्वप्रतिपादनेन नित्यानामर्थक्रियाकारित्वाभावः क्रमयौगपद्ययोव्यपिकयोरभावादित्यादि दर्शितं द्रष्टव्यम् ॥
	\pend
      

	  \pstart \textbf{प्रतिषेध्येत्या}दिना स्वभावविरुद्धोपलब्धिं व्याचष्टे--मूले तूदाह्रियत इत्यध्याहार \textbf{इति} दर्शयन्नाह--\textbf{उदाह्रियत} इति ।
	\pend
      

	  \pstart सुखप्रतिपत्त्यर्थं मौल धर्म्यादिप्रविभागं दर्शयन्नाह--अत्रेति ।
	\pend
      

	  \pstart एतेनेश्वरेऽप्येकोपादानादिविकल्पे सम्प्रदानादिविकल्पाभावो विरुद्धोपलब्धिप्रसङ्गेन दर्शितः । विकल्पस्य विकल्पान्तरेण सहास्थितिलक्षणस्य विरोधस्य स्वसन्ताने सिद्धत्वात् । ततश्च तस्यानिरूप्यकर्त्तृत्वमायातम् । अन्यथा युगपद्दृष्टोत्पादानामनुत्पत्तिः प्रसज्येत । अनिरूप्यकर्त्तृत्वे चाऽऽधिपत्यमात्रेण कर्त्तृत्वं स्यात् । तथा च कर्मणा सिद्धसाधनत्वमीश्वरसाधनानां कार्यत्वादीनामित्यादि दर्शितम् ॥
	\pend
      

	  \pstart विरुद्धकार्योपलब्धिं व्याचक्षाण आह--\textbf{प्रतिषेध्येने}ति । मूलगमिका विवक्षिताभावप्रतिपादकेति विवक्षितमिति दर्शयितुमाह--\textbf{गमिके}ति । पूर्ववद् धर्म्यादिकथनम् ।
	\pend
      \leavevmode\marginnote{\textenglish{131/dm}}“

	  \pstart स्यात् तत्रदृश्यानुपलब्धिर्गमिका । यत्र विरुद्धो वह्निः प्रत्यक्षः, तत्र विरुद्धोपलब्धिर्गमिका\footnote{ब्धिः । द्वयो० \cite{dp-msA} \cite{dp-msB} \cite{dp-edP} \cite{dp-edH} \cite{dp-edE} \cite{dp-edN}} । द्वयोरपि तु परोक्षत्वे \footnote{विरोधका० \cite{dp-msA}}विरुद्वकार्योपलब्धिः प्रयुज्यते ।
	\pend
       

	  \pstart \footnote{यत्र \cite{dp-msA}}तत्र समस्तापवरकस्थं शीतं निवर्त्तयितुं समर्थस्याग्नेरनुमापकं यदा विशिष्टं धूमकलापं निर्यान्तमपवरकात् पश्यति, तदा विशिष्टाद्वह्नेरनुमितात् शीतस्पर्शनिवृत्तिमनुमिमीते\footnote{निवृत्तिरनुमीयते \cite{dp-msA} \cite{dp-msC}} । इह दृश्यमानद्वारप्रदेशसहितः \footnote{सर्वापवरका० \cite{dp-msA} \cite{dp-edP} \cite{dp-edH} \cite{dp-edE} \cite{dp-edN}}सर्वोऽपवरकाभ्यन्तरदेशो धर्मी साध्यप्रतिपत्त्यनुसरणात् पूर्ववद् द्रष्टव्य इति\footnote{“इति” नास्ति \cite{dp-msA} \cite{dp-msB} \cite{dp-msC} \cite{dp-edP} \cite{dp-edH} \cite{dp-edE} \cite{dp-edN}} ॥
	\pend
       “

	  \pstart विरुद्धव्याप्तोपलब्धिर्यथा--न ध्रुवभावी भूतस्यापि भावस्य विनाशः, हेत्वन्तरापेक्षणादिति\footnote{“इति” नास्ति \cite{dp-edE}} ॥ ३६ ॥
	\pend
      ” 

	  \pstart प्रतिषेध्यस्य यद् विरुद्धं तेन व्याप्तस्य धर्मान्तरस्य उपलब्धिरुदाहर्त्तव्या । यथेति । ध्रुवम् अवश्यं भवतीति\footnote{भवति ध्रु० \cite{dp-msB} \cite{dp-msC} \cite{dp-msD}} ध्रुवभावी नेति ध्रुवभावित्वनिषेधः\footnote{०त्वप्रतिषेधः--\cite{dp-msC}} साध्यः । विनाशो धर्मी ।
	\pend
      ”

	  \pstart कस्मिन्नियं प्रयोक्तव्येत्याह--यत्रेति । द्वयोर्विरुद्धशीतस्पर्शयोः । \textbf{अपि}रवधारणे । \textbf{तुः} पूर्वस्माद् वैधर्म्ये ।
	\pend
      

	  \pstart कः पुनरीदृशो विषय इत्याह--\textbf{तत्रेति} ।
	\pend
      

	  \pstart ननु च प्रदीपशिखाप्रभावे\footnote{भवे} धूमेऽपि न शीतस्पर्शाभावः, तत्कथमियं गमिकेत्याह—\textbf{समस्ते}ति । \footnote{प्रतौ सर्वत्र अववरकेत्यादि दृश्यते टीकायां तु अपवरकेति ।}अपवरकग्रहणं शीतस्थानोपलक्षणार्थम् । \textbf{अपवरकात् निर्यान्तं} निर्गच्छन्तम् । अन्यत्र च गम्यमानो धूमः कथमन्यत्र शीताभावं साधयतीत्याह--इहेति । \textbf{इह} विरुद्धकार्योपलब्धौ । \textbf{दृश्यमान}श्चासौ \textbf{द्वारदेश}श्च तेन \textbf{सहितः ।}
	\pend
      

	  \pstart उपपत्तिमाह--\textbf{साध्येति । साध्यस्य} शीतस्पर्शाभावस्य \textbf{प्रतिपत्ति}रवबोधस्तस्य प्रतिप\textbf{त्तेरनुसरणं} निरूपणं तस्मात् । \textbf{पूर्ववदि}ति यथापूर्वं कार्यानुपलम्भे वह्न्याद्यभावप्रतीतिसामर्थ्यायातस्तादृशो धर्मी तद्वत् ।
	\pend
      

	  \pstart अयमस्य भावः--शीतस्पर्शाभावप्रतीतिरेवेयं विमृश्यमाणाऽवश्यमेवंविधधर्मिणमाकर्षतीति ।
	\pend
      

	  \pstart प्रयोगः पुनरस्या एवं कर्त्तव्यः--यत्र धूमविशेषस्तत्र शीतस्पर्शाभावः । यथा महानसादौ । तथाविधश्चात्र धूम इति । एतच्चात्यन्ताभ्यासाज्झटिति धूमदर्शनाच्छीतस्पर्शाभावप्रतीत्युदये विरुद्धकार्योपलम्भजमेकमनुमानमाचार्येणोक्तमिति द्रष्टव्यम् । अनभ्यासदशया  \leavevmode\marginnote{\textenglish{132/dm}} “
	  
	भूतस्यीपि भावस्येति धर्मिविशेषणम् । भूतस्य जातस्यापि विनश्वरः स्वभावो नावश्यम्भावी, किमुताजातस्येति अपिशब्दार्थः । \footnote{जननाद् \cite{dp-msA} \cite{dp-edP} \cite{dp-edH} \cite{dp-edE}}जनकाद्धेतोरन्यो हेतुः हेत्वन्तरं मुद्गरादि\footnote{मुद्गरादिः \cite{dp-msC}} । तदपेक्षते विनश्वरः\footnote{विनश्वरस्यापेक्ष० \cite{dp-msA}} । तस्यापेक्षुणादिति हेतुः । हेत्वन्तरापेक्षणं \footnote{पेक्षणं नाध्रुव० \cite{dp-msB}}नामाध्रुवभावित्वेन व्याप्तं यथा वाससि रागस्य \footnote{राजकादि० \cite{dp-msB} \cite{dp-msC} \cite{dp-msD}}रञ्जमादिहेत्वन्तरापेक्षणमध्रुवभावित्वेन व्याप्तम्\footnote{व्याप्तं तद्वद् । ध्रुव० \cite{dp-msC}} । ध्रुवभावित्वविरुद्धं चाध्रुवभावित्वम् । विनाशश्च विनश्वरस्वभावात्मा हेत्वन्तरापेक्ष इष्टः । ततो विरुद्धव्याप्तहेत्वन्तरापेक्षणदर्शनाद् ध्रुवभावित्वनिषेधः । 
	  
	इह ध्रुवभावित्वं नित्यत्वम्, अध्रुवभावित्वं \footnote{“च” नास्ति \cite{dp-msB} \cite{dp-msD}}चानित्यत्वम् । नित्यत्वानित्यत्वयोश्च परस्परपरिहारेणावस्थानादेकत्र विरोधः । तथा\footnote{स्वभावानुपलब्धिरूपता स्यात् । स्वभावानुपलब्धिरूपा चाभ्युपेता पूर्वाचार्यैरित्याह--\cite{dp-msD-n}} च सति परस्परपरिहारवतोर्द्वयोर्यदैकं दृश्यते तत्र द्वितीयस्य तादात्म्यनिषेधः कार्यः । तादात्म्यनिषेधश्च \footnote{धश्च तयाभ्यु० \cite{dp-msA}}दृश्यतयाऽभ्युपगतस्य सम्भवति । \footnote{य एवं--\cite{dp-msA}}यत एवं तादात्म्यनिषेधः क्रियते--यद्ययं दृश्यमानो नित्यो भवेन्नित्यरूपो दृश्येत । न च नित्यरूपो दृश्यते । तस्मान्न नित्यः । एवं च प्रतिषेध्यस्य नित्यत्वस्य दृश्यमाना\footnote{दृश्यमानात्मत्वमभ्यु \cite{dp-msA} \cite{dp-edP} \cite{dp-edH} \cite{dp-edE} \cite{dp-edN}}त्मकत्वमभ्युपगम्य प्रतिषेधःकृतो भवति । \footnote{अथ न वस्त्वेकत्वविरोधोऽनयोः परं यो निषेधो ध्रुवभावित्वस्य विधीयते स । यद्यया [[?]] दृश्यत्वे सति पूर्वानुपलब्धिष्विव भवेत् तदा नास्यानुपलब्धेः । अथ भवतु नित्यत्वस्यावस्तुन एवं निषेधः, पिशाचादीनां तु सतां कथं निषेध इत्याह--\cite{dp-msD-n}}वस्तुनोऽप्यदृश्यस्य पिशाचादेर्यदि\footnote{यदैव \cite{dp-msB}}दृश्यघटात्मक\footnote{०त्मत्वं० \cite{dp-msA} \cite{dp-edP} \cite{dp-edH} \cite{dp-edE} \cite{dp-edN}}-” पुनरज्ञातेऽनुमाने कार्यलिङ्गजविरुद्धोपलम्भजे भवतः । तथाहि--यत्र धूमस्तत्र सर्वत्र वह्निर्यथाऽयस्कारकुट्याम्, धूमश्चात्रेति कार्यलिङ्गजमेकमत्र नियतप्राग्भावि, तदनु यत्र वह्निर्न तत्र शीतस्पर्शो यथा रसवतीप्रदेशे, वह्निश्चात्रेति विरुद्धोपलम्भजं द्वितीयमिति ॥
	\pend
      

	  \pstart \textbf{प्रतिषेध्यस्ये}त्यादिना विरुद्धव्याप्तोपलब्धिं\leavevmode\marginnote{\textenglish{52b/ms}} व्याचष्टे । पूर्ववत्साध्यादिप्रदर्शनम् ।
	\pend
      

	  \pstart \textbf{किमुते}ति निपातसमुदायः किम्पुनरित्यस्यार्थे वर्त्तते ।
	\pend
      

	  \pstart ननु किमजातस्यापि वस्तुनो नाशमवश्यं भाविनं केचिदिच्छन्ति येनापिशब्दः समुच्चये व्याख्यायत इति ? नैष दोषः । अजातस्य तावदनिष्टत्वादेव नावश्यम्भावी विनाशः, जातस्यापि नावश्यम्भावीतीत्थं मूलेऽपिशब्दः । केवलं \textbf{किमुताजातस्ये}ति व्याचक्षाणेन \textbf{धर्मोत्तरेणा}यमर्थो न व्यक्तीकृतः । तत्रापि किम्पुनरजातस्य यस्य विनाश एव नेष्ट इत्यभिप्रायेण  \leavevmode\marginnote{\textenglish{133/dm}} “
	  
	त्वनिषेधः क्रियते दृश्यात्मक\footnote{०त्मत्वं० \cite{dp-msA} \cite{dp-edP} \cite{dp-edH} \cite{dp-edE} \cite{dp-edN}} त्वमभ्युपगम्य कर्त्तव्यः । \footnote{यद्ययं दृश्य० \cite{dp-msA} \cite{dp-edP} \cite{dp-edH} \cite{dp-edE} \cite{dp-edN}}यद्ययं घटो दृश्यमानः पिशाचात्मा भवेत् पिशाचो दृष्टो भवेत् । न च दृष्टः । तस्मात् न पिशाच इति । दृश्यात्मत्वाभ्युपगमपूर्वको दृश्यमाने घटादौ\footnote{०माने वस्तुनि घटादौ \cite{dp-msC}} वस्तुनि वस्तुनोऽवस्तुनो वा दृश्यस्यादृश्यस्य च तादात्म्य\footnote{निषेधः \cite{dp-msA} \cite{dp-msB} \cite{dp-edP} \cite{dp-edH} \cite{dp-edE} \cite{dp-edN}}प्रतिषेधः । तथा च सति यथा घटस्य दृश्यत्वमभ्युपगम्य \footnote{निपेधः \cite{dp-msB}}प्रतिषेधो दृश्यानुपलम्भादेव तद्वत् सर्वस्य परस्परपरिहारवतोऽन्यत्र दृश्यमाने निषेधो दृश्यानुपलम्भादेव । तथा \footnote{तादात्म्यनिषेधसंसूचकस्य व्यापकानुपलब्धिप्रयोगस्य । स च एवं--नित्यस्य सत्ता स्थिरोपलम्भत्वेन व्याप्ता । तस्य स्थिरोपलम्भविषयत्वस्य तत्र घटादौ अनुपलब्ध्या नित्यसत्ताया व्याप्ता\add{याः}निषेधः--\cite{dp-msD-n}}चास्यैवंजातीयकस्य प्रयोगस्य स्वभावानुपलब्धावन्तर्भावः ॥” योजनीयः । यद्वा प्रागभावस्यानादेरजातस्यापि नाशमवश्यम्भाविनं केचिदिच्छन्तीति तदपेक्षया \textbf{अपिशब्दः} समुच्चये । \textbf{तन्मु}द्गरा\textbf{द्यपेक्षते विनश्वरो} विनंष्टुमिति शेषः । \textbf{तस्य} हेत्वन्तरस्य ।
	\pend
      

	  \pstart प्रयोगः पुनरीदृशः कर्त्तव्यः--यद्यदवस्थाप्राप्तौ हेत्वन्तरमपेक्षते न तदवश्यं तद्रूपं भवति । यथा वस्त्रं रक्तरूपतापत्तौ रागद्रव्यसंयोगापेक्षं नावश्यं रक्तं भवति । अपेक्षते च भावो विनंष्टुं हेत्वन्तरमिति विरुद्धव्याप्तोपलब्धिप्रसङ्ग एषः । अत एव मूलेऽपिशब्दः प्रसङ्गसाधनत्वप्रसङ्गार्थो लक्ष्यते । स्वतन्त्रसाधनं तु विरुद्धव्याप्तोपलम्भाख्यमेवं द्रष्टव्यम्--यो विरुद्धधर्मसंसर्गवान्नासावेको यथा द्रवकठिने । विरुद्धधर्मसंसर्गवांश्च सामान्यादिरिति ।
	\pend
      

	  \pstart ननु च कोऽर्थयोर्विरोधः, किञ्चास्य विरोधस्य साधकं प्रमाणमित्याशङ्कामपाकर्त्तुमाह--इहेति । \textbf{नित्यत्व}शब्देनावश्यम्भावित्वम्, \textbf{अनित्यत्व}शब्देनाऽनवश्यभ्भावित्वमुक्तं द्रष्टव्यम् । अन्यथा केन नित्यो विनाशोऽभ्युपेतो येनास्याऽनित्यताऽपो\footnote{ऽऽपा}द्येत । यच्च पूर्वं \textbf{ध्रुवभावित्व}शब्दं विवृण्वताऽनेन \textbf{ध्रुवमवश्यं भवती}ति विवृतं तच्च व्याहन्येत ।
	\pend
      

	  \pstart सम्प्रति विरोधमुपपादयति--\textbf{नित्यानित्ययोरिति । चो} हेतौ ।
	\pend
      

	  \pstart इदानीं परस्परपरिहारस्थितलक्षणविरोधव्यवस्थापकं दृश्यानुपलम्भं दर्शयितुमाह—\textbf{तथा चे}ति । कथं दृश्यतयाऽभ्युप\add{ग}तस्य निषेध इत्याह--\textbf{यत} इति । \textbf{एवं} वक्ष्यमाणेन प्रकारेण । तमेवाह--\textbf{यद्ययमि}ति । \textbf{न च} नैवं \textbf{नित्यरूपो}ऽवश्यम्भाविस्वरूपो \textbf{दृश्यते} प्रतीयते । यद्यप्येवं तथापि दृश्यात्मकाभ्युपगम इत्याह--\textbf{एवमि}ति \textbf{चो} यस्मात् । \textbf{एव}मनन्तरोक्तेन क्रमेण दृश्यमानस्यादृश्येनावस्तुनाऽन्योन्यपरिहारस्थितलक्षणविरोधव्यवस्थायां तावदेवं दृश्यानुपलम्भ उपायः । वस्तुनाऽप्यदृश्येन तथात्वव्यवस्थायामयमेवोपाय इति दर्शयितुं \textbf{वस्तुनोऽपी}त्यादिनोपक्रमते । न केवलं कल्पितस्यावस्तुन \textbf{इत्यपि}शब्दः । \textbf{इति}र्हेतावेवमर्थे वा । \textbf{दृश्यस्ये}ति वस्त्वपेक्षया ।
	\pend
      \leavevmode\marginnote{\textenglish{134/dm}}““

	  \pstart कार्यविरुद्धोपलब्धिर्यथा--नेहाप्रतिबद्धसामर्थ्यानि शीतकारणानि सन्ति, \footnote{अग्नेरिति--\cite{dp-msB} \cite{dp-msC} \cite{dp-edP} \cite{dp-edH} \cite{dp-edE} \cite{dp-edN}}वह्नेरिति ॥ ३७ ॥
	\pend
      ” 

	  \pstart प्रतिषेध्यस्य यत् कार्यं तस्य यद्विरुद्धं तस्योपलब्धेरुदाहरणम्--यथेति । इहेति धर्मी । अप्रतिबद्धं सामर्थ्य येषां शीतकारणानां शीतजननं प्रति\footnote{प्रति न तानि सन्ति इति--\cite{dp-msA} \cite{dp-edP} \cite{dp-edH} \cite{dp-edE} \cite{dp-edN}}, तानि न सन्ति इति साध्यम् । वह्नेरिति हेतुः ।
	\pend
       

	  \pstart यत्र शीतकारणानि अदृश्यानि, शीतस्पर्शोऽप्यदृश्यः, तत्रायं हेतुः प्रयोक्तव्यः । दृश्यत्वे तु शीतस्पर्शस्य तत्कारणानां वा कार्यानुपलब्धिर्दृश्यानुपलब्धिर्वा गमिका । तस्मादेषाप्यभावसाधनी । ततो यस्मिन् \footnote{देशे \cite{dp-msA} \cite{dp-msB} \cite{dp-msC} \cite{dp-msD} \cite{dp-edP} \cite{dp-edH} \cite{dp-edE} \cite{dp-edN}}उद्देशे सदपि शीतकारणमदृश्यं शीतस्पर्शश्च\footnote{अदृश्यः--\cite{dp-msD-n}} दूरस्थत्वात् प्रतिपत्तुर्वह्निर्भास्वरवर्णत्वाद् दूरादपि दृश्यस्तत्रायं प्रयोग इति\footnote{“इति” नास्ति \cite{dp-msA} \cite{dp-msB} \cite{dp-edP} \cite{dp-edH} \cite{dp-edE} \cite{dp-edN}} ॥
	\pend
      ”

	  \pstart वाशब्दार्थश्चकार इति केचित् ।
	\pend
      

	  \pstart अन्ये तु “अवस्तुनोऽदृश्यत्वस्य सिद्धत्वात् किं तदनुवादेन कार्यम् ? ततो द्वयमप्येतद् वस्त्वपेक्षया योज्यम् । \textbf{वस्तुनो दृश्यस्य} घटादेः, \textbf{अदृश्यस्य} पिशाचादेः । अन्यथा पिशाचादिवस्तुनस्तथात्वं नोक्तं स्यात् । प्रकृतं च तदेव” इति प्रतिपन्नाः ।
	\pend
      

	  \pstart भवत्वेवं ततः किं सिद्धमित्याह--तथा \textbf{च सती}ति । \textbf{तद्वद्} घटवत् । \textbf{अन्य}\leavevmode\marginnote{\textenglish{53a/ms}}\textbf{त्र} अन्यस्मिन् दृश्यमाने वस्तुनि ।
	\pend
      

	  \pstart ननु भवतु दृश्यानुपलम्भाद् दृश्यमानेङ्गुल्यादौ सर्वस्य सुमेर्वादेस्तादात्म्यनिषेधस्तथाप्युक्तास्वनुपलब्धिषु कुत्रायमन्तर्थवतीत्याशङ्कामपाकर्त्तुमुपसंहारव्याजेनातिदेशमप्याह--\textbf{एवमिति । एवंजातीयकस्यै}वम्प्रकारवतः । एवम्प्रकारस्येत्युक्ते वचनभं \footnote{वचनलभ्यं ?} स्यात् । तत्रान्तर्भावो दर्शित एवेति भावः ।
	\pend
      

	  \pstart अथ यदि दृश्यानुपलम्भादन्यत्रान्यस्य दृश्यस्यादृस्यश्य वा तादात्म्यनिषेधः कथं विरुद्धे\footnote{द्ध}व्याप्तोपलब्धेरवतार इति चेत् । न दोषः । विरोधप्रतिपत्तिकाले दृश्यानुपलम्भस्य व्यापारात् । तदवगतविरोधेन तु व्याप्तं यत्र दृश्यते विरुद्धव्याप्तोपलम्भादेव विवक्षिताभावप्रतीतिरिति किमवद्यम् ॥
	\pend
      

	  \pstart प्रतिषेध्यस्येत्यादिना कार्यविरुद्धोपलब्धिं विवृणोति । पूर्ववद् धर्म्यादिप्रदर्शनम् । \textbf{वह्नेरि}ति शीतनिवर्त्तनक्षमाद् विशिष्टादिति द्रष्टव्यम् । अन्यथा प्रति\footnote{दी}पाद्यात्मनः शीतऽनिवर्त्तकत्वेनानैकान्तिकतापत्तेः ।
	\pend
      \leavevmode\marginnote{\textenglish{135/dm}}““

	  \pstart व्यापकविरुद्धोपलब्धिर्यथा--नात्र तुषारस्पर्शो \footnote{अग्नेरिति \cite{dp-msB} \cite{dp-msC} \cite{dp-edP} \cite{dp-edH} \cite{dp-edN} \cite{dp-edE}}वह्नेरिति ॥ ३८ ॥
	\pend
      ” 

	  \pstart प्रतिषेध्यस्य यद् व्यापकं तेन यद् विरुद्धं तस्योपलब्धिरुदाहर्त्तव्या यथेति । अत्रेति धर्मी । तुषारस्पर्शो नेति साध्यम् । \footnote{अग्नेरिति \cite{dp-msD} \cite{dp-msB}}वह्नेरिति हेतुः ।
	\pend
       

	  \pstart यत्र \footnote{यत्र प्रतिषेध्यतुषार० \cite{dp-msC}}व्याप्यस्तुषारस्पर्शो व्यापकश्च\footnote{व्यापकं च शी० \cite{dp-msD}} शीतस्पर्शो न दृश्यस्तत्रायं हेतुः । तयोर्दृश्यत्वे\footnote{यथासंख्यम्--\cite{dp-msD-n}} स्वभावस्य व्यापकस्य चानुपलब्धिर्यतः प्रयोक्तव्या\footnote{प्रयोगश्चैवम्--नात्र तुषारस्पर्शः, उपलब्धिलक्षणप्राप्तस्यानुपलब्धेः । नात्र तुषारस्पर्शः, शीतस्पर्शाभावात् ।} । तथा च\footnote{“च” नास्ति \cite{dp-msB} \cite{dp-edP} \cite{dp-edE}} सत्यभावसाधनीयम् । दूरवर्त्तिनश्च प्रतिपत्तुस्तुषारस्पर्शः शीतस्पर्शविशेषः, शीतमात्रं \footnote{“च” नास्ति \cite{dp-msB}}च परोक्षम् । वह्निस्तु रूपविशेषाद् दूरस्थोऽपि प्रत्यक्षः । ततो वह्नेः शीतमात्राभावः । ततः शीतविशेषेतुषारस्पर्शाभावनिश्चयः । शीतविशेषस्य शीतसामान्येन व्याप्तत्वादिति \footnote{विशिष्टे विष० \cite{dp-msB}}विशिष्टविषयेऽस्याः प्रयोगः ॥
	\pend
       “

	  \pstart कारणानुपलब्धिर्यथा--नात्र धूमो \footnote{अग्न्यभावादिति । \cite{dp-msB} \cite{dp-msD} \cite{dp-edP} \cite{dp-edH} \cite{dp-edN} अग्न्यभावात् \cite{dp-edE}}वहायभावादिति ॥ ३९ ॥
	\pend
      ” 

	  \pstart प्रतिषेध्यस्य यत् कारणं तस्यानुपलब्धेरुदाहरणं यथेति । अत्रेति \textbf{धर्मी} । न धूम
	\pend
      ”

	  \pstart कीदृशि विषयेऽस्याः प्रयोग इत्याह--यत्रेति । न केवलं पूर्व इत्यपिशब्दः । अभावोऽभावव्यवहारश्चाभावशब्देनोक्तः ।
	\pend
      

	  \pstart स्यान्मतम् । कथं पुनः शीतस्पर्शशीतकारणेऽदृश्ये वह्निस्तु दृश्यः सम्भवति येनास्याः प्रयोगो घटत इत्याह--\textbf{यस्मिन्नि}ति । उद्देशे प्रदेशे \textbf{प्रतिपत्तुर्दूरस्थत्वादिति} शीतस्पर्शशीतकारणयोरदृश्यत्वे कारणम् । \textbf{भास्वरवर्णत्वादि}ति वह्नेर्दृश्यत्वे निबन्धनम् । \textbf{भास्वरो} भासनशीलो \textbf{वर्णो} यस्य तद्भावस्तस्मात् । न केवलं निकट इत्यपिशब्दः । तत्र तस्मिन् देशे ।
	\pend
      

	  \pstart प्रयोगः पुनरेवं कार्यः--यत्र विशिष्टो वह्निर्न तत्र शीतोपजननाप्रतिबद्धशक्तीनि शीतकारणानि । यथा क्वचिदनुभूते प्रदेशे । तथाभूतश्चात्र वह्निरिति ।
	\pend
      

	  \pstart एतच्चाभ्यासाज्झटिति वह्निदर्शनेन तथाभूतशीतकारणाभावप्रतीतिजन्मन्येकं कार्यविरुद्धोपलम्भजमनुमानमुक्तमाचार्येणेति द्रष्टव्यम् । अन्यथा तु विरुद्धोपलम्भकार्यानुपलम्भजे द्वे एते अनुमाने । तथा हि यत्र वह्निर्न तत्र शीतस्पर्श इति स्वभावविरुद्धोपलम्भजमेकमनुमानम् । यत्र च यत्कार्यं नास्ति, न तत्र तत्कारणं तज्जननाप्रतिबद्धसामर्थ्यमस्तीति कार्यानुपलम्भजं द्वितीयमिति ॥
	\pend
      \leavevmode\marginnote{\textenglish{136/dm}}“

	  \pstart इति साध्यम् । वह्न्यभावादिति हेतुः । यत्र कार्यं सदपि \footnote{सदपि न दृश्यं भवति \cite{dp-msC} सदपि दृश्यं न भवति \cite{dp-msA} \cite{dp-edP} \cite{dp-edH} \cite{dp-edE} \cite{dp-edN}}अदृश्यं भवति तत्रायं प्रयोगः । दृश्ये तु कार्ये दृश्यानुपलब्धिर्गमिका । ततोऽयमप्यभावसाधनः\footnote{०साधकः \cite{dp-msC}} । निष्कम्पायतसलिलपूरिते ह्रदे हेमन्तोचितबाष्पयोद्गमे विरले सन्ध्यातमसि सति सन्नपि तत्र धूमो न दृश्यत\footnote{दृश्य इति \cite{dp-msA} \cite{dp-edP} \cite{dp-edH} \cite{dp-edE} \cite{dp-edN}} इति कारणानुपलब्ध्या \footnote{प्रतिषिध्यते \cite{dp-msB}}प्रतिषेध्यते । वह्निस्तु यदि तस्याम्भस उपरि प्लवमानो भवेत्\footnote{भवेज्ज्वलितो रूप० \cite{dp-msA} \cite{dp-edP} \cite{dp-edH} \cite{dp-edN} भवेज्ज्वलितरूप० \cite{dp-edE}} प्रज्वलितो, रूपविशेषादेवोपलब्धो भवेत् । अज्वलितस्तु \footnote{तु वनमध्य० \cite{dp-msB}}इन्धनमध्यनिविष्टो भवेत् । तत्रापि दहनाधिकरणमिन्धनं प्रत्यक्षमिति स्वरूपेण, आधाररूपेण वा दृश्य\footnote{दृश्यमानरूप एव \cite{dp-msC}} एव वह्निरिति तत्रास्य\footnote{तत्रास्याः प्रयोगः \cite{dp-msA} \cite{dp-msB} \cite{dp-msC} \cite{dp-msD} \cite{dp-edP} \cite{dp-edH} \cite{dp-edE} \cite{dp-edN}} प्रयोग \footnote{“इति” नास्ति \cite{dp-msA} \cite{dp-msB} \cite{dp-edP} \cite{dp-edH} \cite{dp-edE} \cite{dp-edN}}इति ॥
	\pend
      ”

	  \pstart व्यापकविरुद्धोपलब्धिं व्याख्यातुमाह--\textbf{प्रतिषे}ध्येत्यादि । पूर्ववद् धर्म्यादिप्रदर्शनम् । अत्रापि विशिष्टाद् \textbf{वह्नेरि}ति द्रष्टव्यम् ।
	\pend
      

	  \pstart अस्यापि प्रयोगविषयमाह--\textbf{यत्रेति} । कथं तयोरदृश्यत्वम्, वह्नेश्च दृश्यत्वम्, कथं च न शीतस्पर्श एव तुषारस्पर्श इत्याशङ्कात्रितयमपाकुर्वन्नाह--\textbf{दूरेति । चो} हेतौ नियमे वा । तयोर्भेदमुपपादयति \textbf{तुषारे}ति । \textbf{शीतमात्रम}शीतव्यावृत्तिमात्रम् । \textbf{ततः} शीतमात्राऽभावात् । \textbf{शीतविशेषश्चासौ तुषारस्पर्श}श्चेति विग्रहः ।
	\pend
      

	  \pstart कथं तदभावनिश्चय इत्याह--\textbf{शीतविशेषस्ये}ति । एष च वास्तवो निवृत्तिक्रमः परामर्शदशायां दर्शितो न तु तत्प्रयोग\leavevmode\marginnote{\textenglish{53b/ms}}कालिकः । तथात्वे हि नैकमनुमानमिदं स्यात् । \textbf{इती}त्यादिनोपसंहरति । \textbf{इति}रेवमर्थे । तस्मादर्थे वा । एतेन यद् यत्र नियतसहोपलम्भं तत्ततो न भिद्यते । यथैकस्माच्चन्द्रमसो द्वितीयश्चन्द्रमा । नियतसहोपलम्भस्तु नीलादिर्ज्ञानेनेत्यादि दर्शितं द्रष्टव्यम् ॥
	\pend
      

	  \pstart कारणानुपलब्धिं विवरिषुराह--\textbf{प्रतिषेध्ये}ति ।
	\pend
      

	  \pstart ननु द्वयोरपि तुल्यस्वज्ञानजननयोग्यतारूपत्वात् तुल्यदृश्यत्वमिति कथमस्याः प्रयोग इत्याशङ्क्य विषयमस्या दर्शयितुं \textbf{यत्रे}त्यादिनोपक्रमते ।
	\pend
      

	  \pstart ननु मनोमोदकोपयोगमात्रमेतत् न पुनरीदृशो विषयोऽस्ति यत्राग्निरेव दृश्यो न धूम इति कथं पूर्वोक्तातिक्रम इत्याह--\textbf{निष्कम्पे}ति । ह्रदो जलाधारविशेषः । \textbf{निर्गतः कम्प}श्चलनं यस्मात्स तथा स चासा\textbf{वायतो} महानिति तथा । स चासौ सलिलपूरितश्चेत्येवं विग्रहः कार्यः । \textbf{आयत}ग्रहणेन ह्रदस्य महत्त्वाद् बाष्पे भूयस्त्वमत एव धूमस्य ततो भेदेनानुपलक्षणमिति दर्शयति ।
	\pend
      

	  \pstart पुनः किं विशिष्टे ? हेमन्ते हेमन्तसंज्ञके काले । उचितोऽधिकृतश्चासौ \textbf{बाष्प}श्चेति  \leavevmode\marginnote{\textenglish{137/dm}} ““
	  
	कारणविरुद्धोपलधिर्यथा--नास्य रोमहर्षादिविशेषाः, सन्निहितदहनविशेषत्वादिति ॥ ४० ॥” 
	  
	प्रतिषेध्यस्य यत् कारणं तस्य यद्विरुद्धं तस्योपलब्धेरुदाहरणं यथेति । अस्येति धर्मी । रोम्णां हर्ष उद्भेदः । स आदिर्येषां दन्तवीणादीनां शीतकृतानाम्, ते विशि \footnote{विशेष्यन्ते \cite{dp-msB}}ष्यन्ते तदन्येभ्यो भयश्रद्धादिकृतेभ्य इति रोमहर्षादिविशेषाः । ते न सन्तीति साध्यम् । दहन एव विशिष्यते\footnote{विशेष्यन्तेऽन्यस्मा० \cite{dp-msB} विशिष्यतेऽन्यस्मा० \cite{dp-msC} \cite{dp-msD}} तदन्यस्माद्दहनाच्छीतनिवर्त्तनसामर्थ्येनेति दहनविशेषः । कश्चिद् दहनः सन्नपि न शीतनिवर्त्तनक्षमो यथा प्रदीपः । तादृशनिवृत्तये विशेषग्रहणम् । सन्निहितो दहनविशेषो यस्य स तथोक्तः । तस्य भावस्तस्मादिति हेतुः । यत्र शीतस्पर्शः सन्नप्यदृश्यो रोमहर्षादिविशेषाश्चादृश्याः, तत्रायं प्रयोगः । रोमहर्षादिविशेषस्य दृश्यत्वे दृश्यानुपलब्धिः प्रयोक्तव्या । शीतस्पर्शस्य दृश्यत्वे कारणानुपलब्धिः । तस्मादभावसाधनोऽयम् । रूपविशेषाद्धि दूराद्” तथा तस्योद्गम ऊर्ध्वं गमनं यस्मिन् यस्माद् \textbf{वा} स तथा । कालविशेषेऽप्यस्याः प्रयोग इति दर्शयति विरले । सन्ध्याकालोचितं तमः \textbf{सन्ध्यातमः} तस्मिन् विरले मन्दप्रचारे । कुत्र ? ह्रदे तथाविधे च तमसि सति । \textbf{इति}स्तस्माद् । वह्नेरपि तत्रेयं गतिर्भविष्यतीत्याह—\textbf{वह्निस्त्विति} । तुः पूर्ववत् । \textbf{अम्भस} इति षष्ठी पुनः “षष्ठ्यतसर्थे” \href{http://http://sarit.indology.info/?cref=Pā.2.3.30}{पाणिनि २. ३. ३०}त्यादिना उपरिशब्दस्यातसर्थप्रत्ययान्तत्वात् । तथाहि ऊर्ध्वं\add{र्ध्व}शब्दाद् “उपर्युपरिष्टाद्” \href{http://http://sarit.indology.info/?cref=Pā.5.3.31}{पाणिनि ५. ३. ३१}इतिरित्प्रत्ययो निपातितः । तेनैव सूत्रेणोर्ध्वशब्दस्योपादेशोऽपि ।
	\pend
      

	  \pstart \textbf{प्लवमानो}ऽवतिष्ठमानः । अनेकार्थत्वाद् धातोर्गच्छन्निति वा । \textbf{स्वरूपेण} ज्वालारूपेणाधाररूपेणेन्धननिविष्टेन । \textbf{इति}र्हेतौ । \textbf{तत्र} तस्मिन् स्थानविशेषे । \textbf{अस्य} कारणानुपलम्भस्य ।
	\pend
      

	  \pstart तमिस्रायामेव तु रात्रौ निराधारके प्रदेशे कारणानुपलब्धेः प्रयोगः सुकरः, तत्र वह्नेर्दृश्यत्वाद् धूमस्य सतोऽप्यदृश्यत्वात् । अनेन पुनरेवंविधं विषयं परित्यज्यान्यं विषयमुपपादयता किमित्यात्माऽऽयासित इति न प्रतीमः ।
	\pend
      

	  \pstart एवं तु प्रयोगः कार्यः--यत्र यस्य कारणं नास्ति न तत्तत्रास्ति । यथा बीजाभावेऽ ङ्कुरः । नास्ति चात्र धूमस्य कारणं वह्निरिति ।
	\pend
      

	  \pstart एतेन यत्र यत्र विज्ञानस्य कारणं विज्ञानं नास्ति न तत्र विज्ञानमुपपद्यते । यथोपलशकले । नास्ति च प्राग्भवीयं विज्ञानं कललावस्थायामिति कारणानुपलब्धिप्रसङ्गः सूचितस्तुल्यन्यायार्थः ॥
	\pend
      

	  \pstart कारणविरुद्धोपलब्धिं व्याख्यातुमाह \textbf{प्रतिषेध्यस्ये}ति । पूर्ववद् धर्म्यादिप्रदर्शनम् । \textbf{उद्भेदः} पुलक इत्यर्थः । \textbf{दन्तवीणा}ऽधरोपरिस्थितदन्तपङ्क्तिभ्यां सत्वरमभिहन्यमानाभ्यां\add{... ... ...} कटकटकरणम् । \textbf{आदि}शब्देन शरीरकम्पस्य ग्रहणम् । \footnote{टीकायां नास्ति सुखादीति पाठः--सं०}\textbf{सुखादि}त्यादिग्रहेण हर्षवीररसयोर्ग्रहणम् ।
	\pend
      \leavevmode\marginnote{\textenglish{138/dm}}“

	  \pstart व्धनं पश्यति । शीतस्पर्शस्त्वदृश्यो रोमहर्षादिविशेषाश्च । तेषां कारणविरुद्धोपलब्ध्याऽभावं\footnote{०भावः प्रति \cite{dp-edP} \cite{dp-edH} \cite{dp-edE} \cite{dp-edN} भाव प्रति \cite{dp-msA}} प्रतिपद्यत इति तत्रास्य प्रयोग\footnote{तत्रास्याः प्रयोग०--\cite{dp-msA} \cite{dp-msB} \cite{dp-msC} \cite{dp-msD} \cite{dp-edP} \cite{dp-edH} \cite{dp-edE} \cite{dp-edN}} इति ॥
	\pend
       “

	  \pstart कारणविरुद्धकार्योपलब्धिर्यथा--न रोमहर्षादिविशेषयुक्तपुरुषवानयं प्रदेशः, धूमादिति ॥ ४१ ॥
	\pend
      ” 

	  \pstart प्रतिषेध्यस्य यत् कारणं तस्य यद् विरुद्धं तस्य यत् कार्यं तस्योपलब्धिरुदाहर्त्तव्या—यथेति अयं\footnote{अयं देश इति \cite{dp-msA} \cite{dp-edP} \cite{dp-edH} \cite{dp-edE} \cite{dp-edN}} प्रदेश इति धर्मी । योगो युक्तम् । रोमहर्षादिविशेषैर्युवतं \footnote{०हर्षविशेष० \cite{dp-msC} “रोमहर्षादिविशेषयुक्तम्” नास्ति--\cite{dp-msB}}रोमहर्षादिविशेषयुक्तम् । तस्य सम्बन्धी \footnote{“पुरुषो” नास्ति--\cite{dp-msC}}पुरुषो \footnote{विशेषगुणयुक्तः \cite{dp-msB}}रोमहर्षादिविशेषयुक्तपुरुषः । तद्वान् न भवतीति साध्यम् । धूमादिति हेतुः ।
	\pend
      ”

	  \pstart किं \leavevmode\marginnote{\textenglish{54a/ms}} शीतनिवर्त्तनाऽक्षमोऽप्यस्ति दहनो येन ततो विशिष्यत इत्याह--\textbf{कश्चिदिति । सन्निहितो दहनविशेषो} यस्येति विगृह्णन् “सन्निहितश्चासौ दहनविशेषश्चेति” यदन्येन व्याख्य तं तदपहस्तयति । तदा हि व्यधिकरणासिद्धो हेतुः स्यादिति ।
	\pend
      

	  \pstart क्व पुनरस्याः प्रयोग इत्याह--यत्रेति ।
	\pend
      

	  \pstart ननु शीतस्पर्शरोमहर्षविशेषाणामदृश्यत्वे कथं वह्नेर्दृश्यत्वमित्याह--\textbf{रूपेति । हि}र्यस्मात् । \textbf{इतिरे}वमनन्तरोक्तेन न्यायेन । तत्र विशेषेऽस्य कारणविरुद्धोपलम्भस्य । एष तु प्रयोगोऽभिधानीयःयत्र यत्कारणविरुद्धमस्ति न तत्तत्रास्ति । यथा श्लेष्मविरुद्धे पित्ते न श्लैष्मिको व्याधिः । अस्ति च रोमहर्षादिकारणविरुद्धो वह्निरत्रेति ।
	\pend
      

	  \pstart एतदप्यत्यन्ताभ्यासाज्झटिति सन्निहितदहनविशेषत्वावगममात्रे रोमहर्षादिविशेषाभावप्रतीत्युदये सत्येकमाचार्येणोक्तम् । \textbf{धर्मोत्तरेणापि} तथा व्याख्यायत इति द्रष्टव्यम् । अन्यथा तु विरुद्धोपलम्भकारणानुपलम्भसम्भवे द्वे इमे अनुमाने । तथा हि--यत्र वह्निर्न तत्र शीतस्पर्श इति विरुद्धोपलम्भजमेकमनुमानम् । यत्र शीतस्पर्शाभावो न तत्र तत्कार्यरोमहर्षादीति कारणानुपलम्भजं द्वितीयमिति ॥
	\pend
      

	  \pstart \textbf{प्रतिषेध्यस्ये}त्यादिना कारणविरुद्धकार्योपलब्धिं व्याचष्टे । पूर्ववद् धर्म्यादिप्रदर्शनम् ।
	\pend
      

	  \pstart रोमहर्षादिविशेषैर्युक्तश्चासौ पुरुषश्चेति कर्मधारयं कृत्वा स विद्यते यत्र स तद्वानिति \textbf{शान्तभद्रेण} व्याख्यातम् । तच्चावद्यम् । यतः “कर्मधारयमत्त्वर्थीयाद् बहुव्रीहिरेव लाघवेन” इति वचनाद् रोमहर्षादिविशेषयुक्तः पुरुषो यत्रेत्येवं विशिष्टे प्रदेशेऽवगते \textbf{किं} मत्वर्थीयेनेति मन्यमानो महावैयाकरणोऽयं \textbf{धर्मोत्तरः} प्राहः--\textbf{योगो युक्तमि}ति भावे निष्ठा । यद्येवं कृष्णः सर्पो  \leavevmode\marginnote{\textenglish{139/dm}} “
	  
	\footnote{रोमहर्षविशेष \cite{dp-msD} \cite{dp-msB}}रोमहर्षादिविशेषस्य प्रत्यक्षत्वे दृश्यानुपलब्धिः । कारणस्य शीतस्पर्शस्य प्रत्यक्षत्वे कारणानुपलब्धिः । वह्नेस्तु\footnote{वह्नेः प्रत्य० \cite{dp-msC} \cite{dp-msD}} प्रत्यक्षत्वे कारणविरुद्धोपलब्धिः प्रयोक्तव्या । त्रयाणामप्यदृश्यत्वेऽयं प्रयोगः । तस्मादभावसाधनोऽयम् । \footnote{यत्र \cite{dp-msA}}तत्र दूरस्थस्य प्रतिपत्तुर्दहनशीतस्पर्शरोमहर्षादिविशेषा अप्रत्यक्षाः सन्तोऽपि, धूमस्तु प्रत्यक्षो यत्र, तत्रैतत् प्रमाणम् । धूमस्तु \footnote{यादृशस्तस्मिन्देशे \cite{dp-msA} \cite{dp-msC} \cite{dp-edP} \cite{dp-edH} \cite{dp-edE} \cite{dp-edN}}यादृशस्तद्देशे स्थितं शीतं निवर्त्तयितुं समर्थस्य वह्नेरनुमापकः स इह ग्राह्यः । धूममात्रेण \footnote{“तु” नास्ति--\cite{dp-msA}}तु वह्निमात्रेऽनुमितेऽपि न शीतस्पर्शनिवृत्तिः, नापि \footnote{हर्षादिनिवृत्ति \cite{dp-msD} \cite{dp-msB}}रोमहर्षादिविशेषनिवृत्तिरवसातुं \footnote{शक्यत इति \cite{dp-msC} शक्येति धूम० \cite{dp-msA} शक्येते न धू० \cite{dp-edH}}शक्येति न धूममात्रं हेतुरिति द्रष्टव्यमिति ॥” यस्मिन् वल्मीके, लोहितः शालिर्यस्मिन् ग्रामे, गौरः खरो यस्मिन्नरण्य इति बहुव्रीहिणा भवितव्यम् । ततश्च कृष्णमर्पवान् वल्मीको, लोहितशालिमान् ग्रामः, गौरखरवदरण्यमिति न स्यात् । साधवश्चामी प्रयोगास्तत्कथमनेनैवं व्याख्यातमिति चेत् । नैष दोषः । यस्मात् “कर्मधारयमत्वर्थीयाद् बहुव्रीहिर्लाघवेन” इतीदं वचनं संज्ञाशब्दं वर्जयित्वा वेदितव्यम् । संज्ञाशब्दाश्चैते कृष्णसर्पलोहितशालिगौरखरशब्दा इति साधूक्तं \textbf{योगो युक्तमि}ति । \textbf{रोमहर्षादिविशेषयुक्तमि}ति रोमहर्षादिविशेषयोग इत्यर्थः । \textbf{तस्य} तद्युक्तस्य तद्योगस्य \textbf{सम्बन्धी । सम्बन्धीत्यने}न सम्बन्धषष्ठीयं समस्यत इति दर्शयति । यतोऽयं रोमहर्षादिविशेषयोगः स्वात्मना पुरुषं व्यवच्छिनत्ति । ततो व्यवच्छेदकः सन्पुरुषं स्वसम्बन्धिनमुपपादयति ।
	\pend
      

	  \pstart कदाऽयं प्रयोगो द्रष्टव्य इत्याह--\textbf{त्रयाणामि}ति वह्निशीतस्पर्शरोमहर्षादिविशेषाणाम् । \textbf{अपि}रवधारणे । कस्मादेवमित्याह--\textbf{रोमहर्षादिविशेषस्ये}त्यादि । हिशब्दार्थश्चात्रार्थाद् द्रष्टव्यः । यत एवं तस्मात् । \leavevmode\marginnote{\textenglish{54b/ms}} \textbf{अयमि}त्ययमपीति द्रष्टव्यम् ।
	\pend
      

	  \pstart क्व पुनस्त्रयाणामप्रत्यक्षत्वं धूमस्य तु प्रत्यक्षत्वमित्याह--तत्रेति वाक्योपक्षेपे । विद्यमाना अप्यप्रत्यक्षा यद्यभविष्यन्, नियतमुपालप्स्यन्तेति सम्भावनामतिवृत्ताः । \textbf{दूरस्थस्ये}ति हेतु\textbf{भावेन} विशेषणम् । ततोऽयमर्थः--दूरस्थत्वात् प्रतिपत्तुस्ते सन्तोऽप्यप्रत्यक्षा इति ।
	\pend
      

	  \pstart तत्र स्थाने । \textbf{एतत्} कारणविरुद्धकार्यरूपं साधनं \textbf{प्रमाणमि}त्यनुमानाख्यप्रमाणजनकवात् । अदूरस्थत्वे तु प्रतिपत्तुः प्राकारादिव्यवहितोद्देशेऽयं प्रयोगो द्रष्टव्यः । धूमशब्देन विशिष्टो धूमो विवक्षित इति दर्शयति--\textbf{धूमस्त्वि}ति । तुशब्दो विशेषार्थः ।
	\pend
      

	  \pstart एतदेव व्यतिरेकमुखेण द्रढयन्नाह--\textbf{धूममात्रेणे}ति । \textbf{तुः} पूर्वस्माद् वैधर्म्यमाह । \textbf{इति}र्हेतौ । द्वितीय इतिरेवमर्थः ।
	\pend
      \leavevmode\marginnote{\textenglish{140/dm}}“

	  \pstart यद्येकः प्रतिषेधहेतुरुक्तः कथमेकादशाऽभावहेतव इत्याह--
	\pend
       “

	  \pstart इसे सर्वे कार्यानुपलध्यादयो दशानुपलब्धिप्रयोगाः स्वभावानुपलब्धौ सङ्ग्रहमुपयान्ति ॥ ४२ ॥
	\pend
      ” 

	  \pstart \footnote{“इमे सर्वे इत्यादि” नास्ति--\cite{dp-edH} \cite{dp-edN} इम इत्यादि \cite{dp-msA} \cite{dp-msB} \cite{dp-msD} \cite{dp-edE} \cite{dp-edP}}इमे सर्वे इत्यादि । इमेऽनुपलब्धिप्रयोगाः । इदमानन्तरप्रक्रान्ता\footnote{०न्तरप्रयोगान्ता \cite{dp-edH} \cite{dp-edE} ०न्तरप्रयोक्तानां नि० \cite{dp-msA}} निर्दिष्टाः । तत्र कियतामपि ग्रहणे प्रसक्त आह--कार्यानुपलब्ध्यादय इति । कार्यानुपलब्ध्यादीनामपि त्रयाणां चतुर्णां वा ग्रहणे प्रसक्त\footnote{प्रसक्ते सत्या \cite{dp-edP} \cite{dp-edH} \cite{dp-edE} \cite{dp-edN} प्रसक्तेत्याह \cite{dp-msA}} आह--दशेति । \footnote{“तत्र” नास्ति--\cite{dp-msA} \cite{dp-msB} \cite{dp-edP} \cite{dp-edH} \cite{dp-edE} \cite{dp-edN}}तत्र दशानामप्युदाहृतमात्राणां ग्रहणप्रसङ्गे \footnote{ङ्गे आह--\cite{dp-msB}}सत्याह--सर्व इति ।
	\pend
       

	  \pstart एतदुक्तं भवति--अप्रयुक्ता\footnote{भवति । प्रयुक्ता \cite{dp-msC}} अपि प्रयुक्तोदाहरणसदृशाश्च सर्व एवेति । दशग्रहणमन्तरेण सर्वग्रहणे क्रियमाणे प्रयुक्तोदाहरणकार्त्स्न्यं \footnote{गम्यते \cite{dp-msA} \cite{dp-msB} \cite{dp-msC} \cite{dp-msD} \cite{dp-edP} \cite{dp-edH} \cite{dp-edE} \cite{dp-edN}}गम्येत । दशग्रहणात्\footnote{दशग्रहणोदाहरण० \cite{dp-msB}} \footnote{०दाहृतका० \cite{dp-msC}}तूदाहरणकार्त्स्न्येऽवगते सर्वग्रहणमतिरिच्यमानमुदाहृतसदृशकार्त्स्न्यावगतये\footnote{०याधिगतये \cite{dp-msC}} \footnote{जातम् \cite{dp-edE}}जायते ।
	\pend
      ”

	  \pstart प्रयोगः पुनरीदृशो वाच्यः--यत्र यत्कारणविरुद्धकार्यमस्ति तत्र तन्नास्ति । यथा रुदितविशेषे सति न स्मितविशेषः । रोमहर्षादिविशेषयुक्तपुरुषवत्त्वकारणशीतस्पर्शविरुद्धवह्निकार्यञ्चात्र धूम इति ।
	\pend
      

	  \pstart एतदप्यत्यन्ताभ्यासाज्झटिति धूमदर्शनेन रोमहर्षादियुक्तपुरुषवत्त्वाभावप्रतीत्युदये सति कारणविरुद्वकार्योपलब्धिजमेकमनुमानमुक्तमाचार्येणेति द्रष्टव्यम् । अन्यथा कार्यहेतुविरुद्धोपलम्भकारणानुपलम्भसम्भवानि त्रीण्यमून्यनुमानानि । तथा हि--तदेयं परिपाटिः--यत्र धूमस्तत्राग्निरिति कार्यहेतुजमेकमनुमानम् । यत्र वह्निर्न तत्र शीतस्पर्श इति विरुद्धोपलम्भजं द्वितीयम् । यत्र शीतस्पर्शाभावो न तत्र तत्कार्यरोमहर्षादिविशेषयुक्तपुरुषभाव इति कारणानुपलम्भजं तृतीयमिति ।
	\pend
      

	  \pstart एते च प्रकारा अनुपलब्धेरुपलक्षणं वेदितव्याः । अन्यासामपि विधानसम्भवात् । तथाहि व्यापकविरुद्धकार्योपलब्धिरप्यस्ति--यथा \unclear{ा}त्र तुषारस्पर्शो धूमादिति । कार्यविरुद्धकार्योपलब्धिरप्यस्ति । यथा--नेहाप्रतिबद्धसामर्थ्यानि शीतकारणानि सन्ति धूमादिति । व्यापकविरुद्धव्याप्तोपलब्धिरप्यस्ति--यथा नायं नित्यः कदाचित्कार्यकारित्वादिति । प्रतिषेध्यस्य नित्यत्वस्य व्यापकं निरतिशयत्वम् । तस्य विरुद्धं सातिशयत्वम् । तेन व्याप्तं कदाचित्कार्यकारित्वमिति । आसाञ्च यथास्वं यथायोगं प्रयोगाः स्वयमूह्याः ।
	\pend
      \leavevmode\marginnote{\textenglish{141/dm}}“

	  \pstart ते स्वभावानुपलब्धौ सङ्ग्रहं \footnote{तत्स्वाभाव्येन--\cite{dp-msD-n}}तादात्म्येन गच्छन्ति । स्वभावानुपलब्धिस्वभावा\footnote{स्वभावा आत्मोत्पादकाः । भावध्वनिरुत्पादकपर्यायः--\cite{dp-msD-n}} इत्यर्थः ॥
	\pend
       

	  \pstart ननु च स्वभावानुपलब्धिप्रयोगाद् भिद्यन्ते कार्यानुपलब्ध्यादयः । तत् \footnote{०मन्तर्भाव इत्याह--\cite{dp-msC}}कथमन्तर्भवन्ति ? इत्याह--
	\pend
       “

	  \pstart पारम्पर्येणार्थान्तरविधिप्रतिपेधाभ्यां प्रयोगभेदेऽपि ॥ ४३ ॥
	\pend
      ” 

	  \pstart प्रयोगभेदेऽपि--प्रयोगस्य शब्दव्यापारस्य भेदेऽपि अन्तर्भवन्ति । कथं प्रयोगभेद इत्याह--अर्थान्तरविधीति\footnote{विधीत्यादि \cite{dp-edP} \cite{dp-edH} \cite{dp-edE} \cite{dp-edN}} । \footnote{प्रतिषेध्यादर्थान्तरस्य \cite{dp-msB}}प्रतिषेध्यादर्थादर्थान्तरस्य विधिरुपलब्धिः स्वभावविरुद्धाद्युपलब्धिप्रयोगेषु । प्रतिषेधः कार्यानुपलब्ध्यादिषु प्रयोगेषु । अर्थान्तरविधिना, अर्थान्तर-
	\pend
      ”

	  \pstart केचित्तु नेहाप्रतिबद्धसामर्थ्यानि वह्निकारणानि सन्ति, तुषारस्पर्शादिति कार्यविरुद्धव्याप्तोपलब्धिमिच्छन्ति । नात्र धूमस्तुषारस्पर्शादिति कारणविरुद्धव्याप्तोपलब्धिमपीति ॥
	\pend
      

	  \pstart \textbf{सा च प्रयोगभेदादेकादश} प्रकारेति यदुक्तं तदसहमानश्चोदयति \textbf{यदी}ति । \footnote{न्यायबिन्दुः २. १७.}अत्र द्वौ \textbf{वस्तुसाधनावेकः प्रतिषेधहेतु}रित्यनेनैकः \textbf{प्रतिषेधहेतुरुक्त} इति चोदयितुराशयः । \textbf{कार्ये}त्यादिना दशग्रहणस्य तात्पर्यार्थं व्याचष्टे । अपिशब्दः शङ्कायाम् । \textbf{सर्व}ग्रहणस्यापि तात्पर्यार्थमाह—तत्रेति वाक्योपन्यासे । \textbf{अपिः} पूर्ववत् । \textbf{उदाहृत} एवो\textbf{दाहृतमात्राणि} तेषाम् । अनेन द्रव्यकार्त्स्न्यवृत्तः सर्वशब्दो गृहीत इति दर्शितम् ।
	\pend
      

	  \pstart तर्हि \leavevmode\marginnote{\textenglish{55a/ms}} \textbf{सर्व}ग्रहणमेवास्तु किं दशग्रहणेनेत्याह--दशेति । \textbf{प्रयुक्तोदाहरणकार्त्स्न्यं गम्येतेति} लोके “सर्वे पदातु\footnote{त}योऽत्र योद्धारः” इत्यादौ सर्वशब्दस्योपदर्शितकार्त्स्न्यवृत्तस्य दर्शनादुक्तम् । यद्येवं दशग्रहणेऽप्येवं किं न स्यादित्याह--\textbf{दशग्रहणादि}ति । तुर्दशग्रहणरहितपक्षाद् वैधर्म्यमाह । \textbf{अतिरिच्यमानम}धिकीभवद् गतार्थं सदिति यावत् उपदर्शिततुल्यावबोधाय सम्पद्यते ।
	\pend
      

	  \pstart स्यादेतत्--इमे सर्वे दशानुपलब्धिप्रयोगा इत्येतावतैव कियतां ग्रहणप्रसङ्गो निराकृत एवेति कथं कार्यानुपलब्ध्यादिग्रहण\textbf{माचार्यस्य} नातिरिच्यते, कथं च \textbf{धर्मोत्तरस्यैषा} तात्पर्यार्थव्याख्या--\textbf{तत्कियतामपि प्रसक्त आ}हेत्यपर्यालोचितव्याख्यानं \add{न} भवतीति चेत् । नैष दोषः । तथाहि--प्राक्तनैकादशग्रहणस्योपलक्षणार्थत्वेनान्येषामपि व्यापकविरुद्धव्याप्तोपलब्ध्यादिप्रयोगाणामभिमतत्वादाद्यान् कार्याऽनुपलब्ध्यादिप्रयोगान्विहाय दशसंख्यापूरणं कृतं भवत्येवेति तदाशङ्कानिवृत्त्यर्थं कार्यानुपलब्ध्यादिग्रहणं कृतमा\textbf{चार्येण । धर्मोत्तरेणा}ऽपि तथा व्याख्यातमिति । न तर्हि दशग्रहणं कर्त्तव्यमिति चेत् । न । अस्योपलक्षणार्थत्वाददोष एषः ।  \leavevmode\marginnote{\textenglish{142/dm}} “
	  
	प्रतिषेधेन च प्रयोगा भिद्यन्ते । यदि प्रयोगान्तरेष्वर्थान्तरविधिप्रतिषेधौ कथं \footnote{कथमन्तर्भ० \cite{dp-msC}}तर्हि अन्तर्भवन्ति ? इत्याह--पारम्पर्येणेति प्रणालिकयेत्यर्थः । 
	  
	एतदुक्तं भवति--न साक्षादेते प्रयोगा दृश्यानुपलब्धिमभिदधति, दृश्यानुपलब्ध्यव्यभिचारिणं त्वर्थान्तरस्य विधिं निषेधं वाऽभिदधति । ततः प्रणालिकयामीषां स्वभावानुपलब्धौ सङ्ग्रहो न साक्षादिति ॥ 
	  
	यदि प्रयोगभेदादेष\footnote{भेदेन भेदः \cite{dp-msA} \cite{dp-edP} \cite{dp-edH} \cite{dp-edE} \cite{dp-edN} प्रयोगभेदादेव भेदः \cite{dp-msB} \cite{dp-msC}} भेदः; परार्थानुमाने वक्तव्य एषः । शब्दभेदो हि प्रयोगभेदः । शब्दश्च\footnote{शब्दस्तु परा० \cite{dp-msB} \cite{dp-msC}} परार्थानुमानमित्याशङ्क्याह-- “
	  
	प्रयोगदर्शनाभ्यासात् स्वयमप्येवं व्यवच्छेदप्रतीतिर्भवतीति\footnote{प्रतीतिरिति स्वार्थानुमानेऽप्यस्याः प्रभेदनिर्देशः--\cite{dp-msC}} स्वार्थेऽप्यनुमानेऽस्याः प्रयोगनिर्देशः ॥ ४४ ॥” 
	  
	प्रयोगदर्शनेत्यादि । प्रयोगाणां \footnote{शास्त्रघटितानाम् \cite{dp-msA} \cite{dp-edP} \cite{dp-edH} \cite{dp-edE} \cite{dp-edN} शास्त्रपरिघटितानाम् \cite{dp-msB} \textbf{शास्त्रगदितानां}—पाठान्तरम्--\cite{dp-msD-n}}शास्त्रपरिपठितानां दर्शनमुपलम्भः । तस्याभ्यासः पुनः पुनरावर्त्तनम् । तस्मान्निमित्तात् । स्वयमपीति प्रतिपत्तुरात्मनोऽपि । एवम् इत्यनन्तरोक्तेन \footnote{प्रयोगदर्शनाभ्यासक्रमेण--\cite{dp-msD-n}}क्रमेण । व्यवच्छेदस्य प्रतिषेधस्य प्रतीतिर्भवतीति\footnote{भवति इतिश० \cite{dp-msC} \cite{dp-msB}} इति\textbf{शब्दस्तस्मादर्थे} ।” कथं \textbf{सङ्ग्रहम}न्तर्भावं \textbf{गच्छन्ती}त्याह--\textbf{तादात्म्येने}ति । तस्याः स्वभावानुपलब्धेरात्मा तदात्मा तस्य भावस्तेन स्वभावानुपलब्धित्वेन । \textbf{अनुपलब्धिस्वभावा} इत्यर्थः--इतीदं स्पष्टीकरणमपि स्वभावानुपलब्धिस्मारकत्वेन तत्स्वभावा इति द्रष्टव्यम् ॥
	\pend
      

	  \pstart \footnote{अस्पष्टम्--सं०}...रेव किं न शब्दाख्या यत इति चेत् न--\textbf{पारम्पर्य}ग्रहणव्याघातप्रसङ्गात्, \textbf{न साक्षादेत} इत्यादिवक्ष्यमाणधर्मोत्तरीयव्याख्यानव्याघातप्रसङ्गाच्चेति । \textbf{शब्द}स्य \textbf{व्यापारो}ऽभिधालक्षणः तस्य \textbf{भेदे} भिद्यमानत्वेऽ\textbf{पि । स्वभावविरुद्धादी}त्यादिग्रहणेन कारणविरुद्धादीनां ग्रहणम् । \textbf{कार्यानुपलब्ध्यादी}त्यादिग्रहणेन व्यापकानुपलब्ध्यादीनां ग्रहणम् । \textbf{अर्थान्तरविधिप्रतिषेधाभ्या}मिति मूले करणतृतीयाद्विवचनान्तमेतदिति दर्शयन्नाह--\textbf{अर्थान्तरे}ति । \textbf{च}स्तुल्यबलत्वं समुच्चिनोति । \textbf{भिद्यन्ते} नानारूपा भवन्ति । \textbf{प्रयोगान्तरेष्वि}त्यन्तरशब्दोऽन्यवचनः स्वभावानुपलब्ध्यपेक्षया । \textbf{परम्परा} परिपाटिः । सैव पारम्पर्यमिति स्वार्थिकः प्रत्ययः । एतदेव स्पष्टयति--\textbf{प्रणालिकये}ति ।
	\pend
      

	  \pstart ननु यद्यमीषां दृश्यानुपलब्धावन्तर्भावस्तदा साक्षात्तदभिधानं तथात्वे च कथं \textbf{पारम्पर्येणे}त्याशङ्क्याह--\textbf{एतदुक्तं भवति} । यदि नाभिदधति तदा--\textbf{न साक्षादि}ति न कर्त्तव्यम् ।  \leavevmode\marginnote{\textenglish{143/dm}} “
	  
	तदयमर्थः--यस्मात् स्वयमप्येवमनेनोपायेन\footnote{मनेन मेयेन--पाठः--\cite{dp-msD-n}} प्रतिपद्यते प्रयोगाभ्यासात्, तस्मात् स्वप्रतिपत्तावप्युपयुज्यमानस्यास्य प्रयोगभेदस्य स्वार्थानुमाने निर्देशः । \footnote{यत् पुनस्त्रिरूपं लिङ्गाख्यानम्--\cite{dp-msD-n}}यत् पुनः परप्रतिपत्तावेवोपयुज्यते तत् परार्थानुमान एव वक्तव्यमिति ॥” सङ्ग्रहश्च कथमित्याह--\textbf{दृश्ये}ति । \textbf{तु}र्विशेषार्थे यस्मादर्थे वा । \textbf{विधि}मग्न्यादेः । \textbf{निषेधं} व्यापकादेः । चकारो वाशब्दार्थे । ततस्तदव्यभिचारिविधिनिषेधाभिधानात् । \textbf{न साक्षात्} नाव्यवधानेन । अर्थान्तरविधिप्रतिषेधयोश्च दृश्यानुपलम्भाव्यभिचारित्वं कार्यकारणभावादिग्रहणकालप्रवृत्तदृश्यानुपलम्भस्मारकादि द्रष्टव्यम् ।
	\pend
      

	  \pstart \textbf{एष भे}\leavevmode\marginnote{\textenglish{55b/ms}}द इति स्वभावानुपलब्ध्यादिरूपः । कस्मात्तत्र वाच्य इत्याह \textbf{शब्दे}ति । \textbf{हि}र्यस्मात् । \textbf{शब्दभेद}स्त्रिरूपलिङ्गवाक्यनानात्वम् । यद्यप्येवं तथापि कथं तत्र वक्तव्य इत्याह--शब्दश्चेति । \textbf{चो} हेतौ ।
	\pend
      

	  \pstart \textbf{शास्त्रपरिपठितानामि}ति शास्त्रपरिपठितद्वारेण परिज्ञातानां स्वभावाद्यनुपलब्ध्यादिवाचकानां वाक्यानामिति द्रष्टव्यम् । उपलम्भो द्विविधो वाच्यरूपो वाचकरूपश्च । अत एवावर्त्तनमपि द्वेधा शब्दरूपावर्त्तनम्, अर्थावर्त्तनं च । तत्रार्थावर्त्तनं पुनः पुनश्चेतसि निवेशनम् । शब्दावर्त्तनं पुनः पुनरुच्चारणम् ।
	\pend
      

	  \pstart मूले \textbf{स्वयं}शब्द आत्मन इति षष्ठ्यर्थे वर्त्तमानो गृहीत इत्याशयेनाह--\textbf{प्रतिपत्तुरात्मन} इति । स्वार्थानुमानप्रस्तावात्प्रतिपत्तृशब्देन यस्त्रिरूपेण लिङ्गेन परोक्षमर्थं प्रतिपद्यते स गृह्यते । \textbf{अपि}शब्दात्परोऽपि तस्मात्प्रतिपद्यत इति सम्बन्धनीयम् । मूले तु न केवलं परस्येति योजनीयम् । \textbf{अनन्तरोक्तेन} परिपठितस्वभावानुपलब्ध्यादिसूचितेन स्वभावानुपलब्ध्यादिप्रयोगक्रमेण ।
	\pend
      

	  \pstart यत \textbf{इति}शब्दस्तस्मादर्थे तत्तस्माद् \textbf{अयं} वक्ष्यमाणोऽर्थः । \textbf{तस्माच्छ}ब्देन यस्मा च्छब्दस्यान्वयाद् \textbf{यस्मादि}त्युक्तम् । \textbf{अनेन} स्वभावानुपलब्ध्यादिप्रयोगलक्षणेनो\textbf{पायेन प्रतिपद्यत} इत्याशङ्क्य पूर्वमेव स्मरयति \textbf{प्रयोगाऽभ्यासादि}ति ।
	\pend
      

	  \pstart स्यान्मतम्--न स्वयमुच्चरितः शब्दस्तत्प्रतिपत्तेर्निमित्तम् । प्रतिपन्ने शब्दप्रयोगात् ॥ अन्यथा प्रतिनियतप्रयोगायोगात् । तत्कथं पिष्टपेषणकारी शब्द उपायत्वेनोच्यत इति नैतदस्ति । यतो लिङ्गदर्शनेनान्यतो वा निमित्तात्प्रबुद्धवासनो मन्दप्रचारार्थस्मरणोऽत्यन्ताभ्यस्तप्रयोगस्तथाविधप्रयोगमुच्चार्यैवं तत्त्वमवगाहमानस्तदर्थं प्रतिपद्यते--यथा कश्चिदभ्यासात्सति धर्मिणि धर्माणां लोके चिन्ता प्रवर्त्तत इत्युच्चार्यैवास्यार्थं प्रतिपद्यते, तद्वत् त्रिरूपाख्यानं वाक्यमुच्चार्यैव कश्चिदभ्यस्तप्रयोगः परोक्षमर्थं प्रतिपद्यते । ततोऽयमुपायो भवत्येव ।
	\pend
      

	  \pstart ततो परस्यापि प्रतीत्युदयात् परार्थानुमानमपि स्यादिति चेत् । भवतु । का क्षतिः ? स्वप्रतिपत्तिप्रयोजनं सत्स्वार्थानुमानं तदैव च तेनान्यः प्रतिपद्यत इति परप्रतिपत्तिप्रयोजनं सत् परार्थानुमानम् । अत एवमशब्दोऽपि कश्चिन्नियतोऽस्तीत्यवाचामेति ।
	\pend
      \leavevmode\marginnote{\textenglish{144/dm}}“

	  \pstart ननु च कार्यानुपलब्ध्यादिषु कारणादीनामदृश्यानामेव \footnote{०मेव प्रतिषेधः \cite{dp-msA} \cite{dp-edP} \cite{dp-edH} \cite{dp-edE} \cite{dp-edN}}निषेधः, दृश्यनिषेधे स्वभावा\footnote{०नुपलम्भप्र० \cite{dp-msA} \cite{dp-msB} \cite{dp-edP} \cite{dp-edH} \cite{dp-edE} \cite{dp-edN}}नुपलब्धिप्रयोगप्रसङ्गात् । तथा च सति\footnote{अदृश्यानां निषेधे सति--\cite{dp-msD-n}} न तेषां\footnote{कारणादीनाम्--\cite{dp-msD-n}} दृश्यानुपलब्धेर्निषेधः । तत् कथमेषां प्रयोगाणां दृश्यानुपलब्धावन्तर्भाव इत्याह--
	\pend
       “

	  \pstart सर्वत्र \footnote{चास्यामभावाभावव्यव \cite{dp-edE}}चास्यामभावव्यवहारसाधन्यामनुपलब्धौ येषां स्वभावविरुद्धादीनामुपलब्ध्या\footnote{विरुद्धानामुप० \cite{dp-msC}} कारणादीनामनुपलब्ध्या च प्रतिषेध उक्तस्तेषामुपलधिलक्षणप्राप्तानामेवोपलब्धिरनुपलब्धिश्च वेदितव्या ॥ ४५ ॥
	\pend
      ” 

	  \pstart \footnote{“सर्वत्र चेत्यादि” नास्ति \cite{dp-edH} \cite{dp-edN}}सर्वत्र चेत्यादि । अभावश्च \footnote{अभावश्च तस्य च व्यवहारोऽभावव्य० \cite{dp-msA} \cite{dp-edP} \cite{dp-edH} \cite{dp-edE}}तद्व्यवहारश्च अभावव्यवहारौ । रवभावानुपलब्धावभावव्यवहारः साध्य । शिष्टेष्वभावः । तयोः साधन्यामनुपलब्धौ । सर्वत्र चेति चशब्दो हिशब्दस्यार्थे । यस्मात् सर्वत्रानुपलब्धौ\footnote{लब्धौ सत्यां--\cite{dp-msB} \cite{dp-edH} L.} येषां प्रतिषेध उक्तस्तेषामुपलब्धिलक्षणप्राप्तानां \footnote{दृश्यमानानामेव \cite{dp-msB}}दृश्यानामेवं \footnote{मेव स प्रति० \cite{dp-msA} \cite{dp-edP} \cite{dp-edH} \cite{dp-edE} \cite{dp-edN}}प्रतिषेधस्तस्माद् दृश्यानुपलब्धावन्तर्भावः ।
	\pend
       

	  \pstart कुत एतद् दृश्यानामेवेत्याह--स्वभावेत्यादि । अत्रापि चकारो हेत्वर्थः । यस्मात् स्वभावविरुद्ध आदिर्येषां तेषामुपलब्ध्या, कारणमादिर्येषां तेषामनुपलब्ध्या प्रतिषेध उक्तस्तस्माद् दृश्यानामेव प्रतिषेध इत्यर्थः ।
	\pend
       

	  \pstart यदि नाम स्वभावविरुद्धाद्युपलब्ध्या कारणाद्यनुपलब्ध्या\footnote{कारणानुपलब्ध्या--\cite{dp-msC}} च प्रतिषेध उक्तस्तथापि कथं दृश्यानामेव प्रतिषेध इत्याह--उपलब्धिरित्यादि । अत्रापि चकारो हेत्वर्थः । यस्माद् ये विरोधिनः, व्याप्यव्यापकभूताः, कार्यकारणभूताश्च ज्ञातास्तेषामवश्यमेवोपलब्धिः, उपलब्धिपूर्वा चानुपलब्धिर्वेदितव्या \footnote{“ज्ञातव्या” नास्ति \cite{dp-msA} \cite{dp-msB} \cite{dp-edP} \cite{dp-edH} \cite{dp-edE} \cite{dp-edN}}ज्ञातव्या । उपलब्ध्यनुपलब्धी च द्वे येषां स्तस्ते दृश्या एव । तस्मात् स्वभावविरुद्धाद्युपलब्ध्या कारणाद्यनुपलब्ध्या चोपलब्ध्यनुपलब्धिमतां विरुद्धादीनां प्रतिषेधः क्रियमाणो दृश्यानामेव कृतो द्रष्टव्यः ।
	\pend
      ”

	  \pstart केचित्पुनरेवं व्याचक्षते--स्वयमित्यादिना ग्रन्थेन वार्त्तिककृतेदमुक्तम्--स्वभावादीनामनुपलब्ध्या विरुद्धादीनाञ्चोपलब्ध्या यथायोगमभावं तद्व्यवहारं च प्रयोगनिरपेक्ष एव प्रतिपत्ता प्रत्येति । न केवलं प्रयोगाभ्यासात् प्रतिपत्तिसमय एव प्रयोगमुच्चारयति । न तु ततोऽपूर्वमवगच्छतीतरथा प्रतिनियतशब्दोच्चारणं न भवेदिति । \textbf{अनेनोपायेने}ति \textbf{चोपाय} इहो\textbf{पाय}स्तथाशब्दोच्चारणक्रमस्तेनेति वर्णयन्ति । एतेन चानुपलब्धिप्रयोगसमर्थनन्यायेन ।
	\pend
      \leavevmode\marginnote{\textenglish{145/dm}}“

	  \pstart बहुषु चोद्येषु प्रक्रान्तेषु परिहारसमुच्चयार्थश्चकारो हेत्वर्थो भवति । यस्मादिदं चेदं च समाधानमस्ति तस्मात् तत्तच्चोद्यमयुक्तमिति चकारार्थः ॥
	\pend
       

	  \pstart कस्मात् पुनः प्रतिषेध्यानां विरुद्धादीनामुपलब्ध्यनुपलब्धी वेदितव्ये इत्याह--
	\pend
       “

	  \pstart अन्येषां विरोधकार्यकारणभावाभावासिद्धेः\footnote{०कारणभावासिद्धेः \cite{dp-edE} ०कारणभावासिद्धिः \cite{dp-msB} \cite{dp-edP} \cite{dp-edH}} ॥ ४६ ॥
	\pend
      ” 

	  \pstart अन्येषामिति । उपलब्ध्यनुपलब्धिमद्भ्योऽन्येऽनुपलब्धा एव ये तेषां विरोधश्च कार्यकारणभावश्च केनचित्सहाभावश्च व्याप्यस्य\footnote{व्याप्यस्येति व्याप्यरूपाणामिति व्याख्येयम् । बहुस्थानेषु पाठोऽपि न--\cite{dp-msD-n}} व्यापकस्याभावे\footnote{व्यापकाभावे \cite{dp-msC}} न सिध्यति यस्मात् ततो विरोध\footnote{विरोधिकार्य० \cite{dp-msA} \cite{dp-msB} \cite{dp-edP} \cite{dp-edE} \cite{dp-edH} \cite{dp-edN}} कार्यकारणभावाभावासिद्धेः कारणाद् उपलब्ध्यनुपलब्धिमन्त एवं विरुद्धादयो निषेध्याः । उभयवन्तश्च दृश्या एव । तस्माद् दृश्यानामेव प्रतिषेधः ।
	\pend
       

	  \pstart तदयमर्थः । विरोधश्च\footnote{विरोधः कार्य० \cite{dp-msA} \cite{dp-edP} \cite{dp-edH} \cite{dp-edE}} कार्यकारणभावश्च व्यापकाभावे व्याप्याभावश्च दृश्यानुपलब्धेरेवेति । \footnote{०संनिधाने परा० \cite{dp-msC}}एकसंनिधावपराभावप्रतीतौ ज्ञातो विरोधः । कारणाभिमताभावे च कार्याभिमताभावप्रत्ययेऽवसितः\footnote{०वसितकार्य० \cite{dp-edP} \cite{dp-edH} \cite{dp-msA} \cite{dp-msB}} कार्यकारणभावः । व्यापकाभिमताभावे च \footnote{व्याप्याभावे \cite{dp-msA} \cite{dp-msB} \cite{dp-msD} \cite{dp-edP} \cite{dp-edH} \cite{dp-edE} \cite{dp-edN}}व्याप्याभि-
	\pend
      ”

	  \pstart अन्ये पुनरन्यथा व्यवस्थिताः--\textbf{स्वयमपी}त्यादिकं नाविर्भूतप्रयोगमधिकृत्योक्तम्, किन्त्वन्तर्जल्पाकारप्रवृत्तं स्वप्रतिपत्तिकालभा\leavevmode\marginnote{\textenglish{56a/ms}}विनमिति ।
	\pend
      

	  \pstart अत्र च साध्वसाधु वा व्याख्यानं साधुभिरेव ज्ञातव्यमिति ।
	\pend
      

	  \pstart स्यादेतत्--यथा प्रयोगभेदः स्वार्थानुमाने कथ्यते तथा च न किञ्चिद् वाच्यं परार्थानुमाने स्यादित्याशङ्क्याह--\textbf{यत्पुनरिति । परप्रतिपत्तावेव,} न तु स्वप्रतिपत्तावपीत्यवधारणार्थः ॥
	\pend
      

	  \pstart सम्प्रति दृश्यानुपलब्धावन्तर्भावं सर्वानुपलब्धीनामसहमान आह--\textbf{ननु चेति । तथा च सति} कारणादीनामदृश्यानां निषेधप्रकारे सति ।
	\pend
      

	  \pstart \textbf{शिष्टेषु} परिशिष्टेषु \textbf{अभाव} इत्यभावोपीति द्रष्टव्यम् । न त्वभाव एव व्यवहारस्यापि साधनात् ।
	\pend
      

	  \pstart \textbf{कारणा\add{द्य}नुपलब्ध्या} च करणभूतया । कार्यकारणभावादिग्रहणकाले योपलब्धिरनुपलब्धिश्च पूर्वमासीत् तद्वतां \textbf{प्रतिषेधः क्रियमाणो दृश्यानामेव कृतो द्रष्टव्यो} ज्ञातव्यः । यथा अर्थविरोधादिग्रहणकालेऽवश्यंभाविनी दृश्यानुपलब्धिस्तथाऽनन्तरमेव \textbf{धर्मोत्तरेण} प्रसाधयिष्यते ।
	\pend
      

	  \pstart ननु च \textbf{कारणादीनां चे}त्यनेन प्र\footnote{च}कारेणावश्यं समुच्चयार्थेन भाव्यम्, तत्कथं हेत्वर्थे व्याख्यायत इत्याशङ्क्य पूर्वं बुद्धिस्थं स्पष्टयन्नाह--\textbf{बहुष्विति} ।
	\pend
      

	  \pstart एवम्मन्यते--समुच्चयार्थे वर्त्तमान एवायं हेत्वर्थे वर्त्तते । न त्वेवं समुच्चयार्थो निराक्रियते, हेतूनां परस्परसमुच्चयस्य प्रतीयमानत्वात् । तथा स न हेत्वर्थो भवति । \textbf{इतिरेवं चकार}स्यार्थः प्रयोजनम् ॥
	\pend
      \leavevmode\marginnote{\textenglish{146/dm}}“

	  \pstart मताभावे निश्चिते निश्चितो व्याप्यव्यापकभावः । तत्र\footnote{तत्--\cite{dp-msC}} व्याप्यव्यापकभावप्रतीतेर्निमित्तमभावः प्रतिपत्तव्यः । इह गृहीते वृक्षाभावे हि शिंशपात्वाभावप्रतीतौ \footnote{०भावप्रतीतौ व्या० \cite{dp-msB}}प्रतीतो व्याप्यव्यापकभावः । अभावप्रतिपत्तिश्च सर्वत्र दृश्यानुपलब्धेरेव । तस्माद्विरोधम् कार्यकारणभावम्, व्याप्यव्यापकभावं च स्मरता विरोध-कार्यकारणभाव-व्याप्यव्यापकभाविविषयाभावप्रतिपत्ति\footnote{देशकालस्वभावविप्रकृष्टाः पिशाचादययस्तेषां पिशाचादीनां विरोधश्च केनचिदग्निना सह न सिध्यतीति सम्बन्धः । तथा कार्यकारणभावश्च पिशाचादीनां केनचिद्धूमेन सार्धं न सिध्यति । \cite{dp-msD-n} ।} निबन्धनं दृश्यानुपलब्धिः स्मर्तव्या । दृश्यानुपलब्ध्यस्मरणे विरोधादीनामस्मरणम् । तथा च सति न विरुद्धादिविधिप्रतिषेधाभ्यामितराभावप्रतीतिः स्यात् । विरोधादिग्रहणकालभाविन्यां च दृश्यानुपलब्धाववश्यस्मर्तव्यायां तत एवाभावप्रतीतिः ।
	\pend
       

	  \pstart तत्र यद्यपि संप्रतितनी \footnote{संप्रति नास्ति--\cite{dp-msA} \cite{dp-edP} \cite{dp-edH} \cite{dp-edE} \cite{dp-edN}}नास्ति दृश्यानुपलब्धिर्विरोधादिग्रहणकाले त्वासीत् । या दृश्यानुपलब्धिः संप्रति स्मर्यमाणा सैवाभावप्रतिपत्तिनिबन्धनम् । ततः संप्रति नास्ति \footnote{दृश्योपलब्धि० \cite{dp-msA} \cite{dp-msB} \cite{dp-edP} \cite{dp-edH} \cite{dp-edE} \cite{dp-edN}}दृश्यानुपलब्धिरित्यभावसाधनत्वेन दृश्यानुपलब्धिप्रयोगाद् भिद्यन्ते कार्यानुपलब्ध्यादिप्रयोगाः ।
	\pend
      ”

	  \pstart \textbf{विरुद्ध}शब्देन प्रतिषेध्यस्य विरुद्धं ग्राह्यम् । \textbf{आदि}शब्देन विरुद्धकार्यादीनां ग्रहणम् । येषामेकदोपलब्धिस्तेभ्योऽन्येऽ\textbf{नुपलब्धा एव} । कदाचित्क्वचिदज्ञाता एव । \textbf{व्यापकस्याभावेऽभावश्च व्याप्यस्य न सिद्ध्यिति यस्मात्} ।
	\pend
      

	  \pstart अयमाशयः--यदि पूर्वं व्यापकाभिमतस्याभावे व्याप्याभिमताभावो निश्चितो भवेत् तदा व्याप्यव्यापकभावः सिद्ध्येत्, तदा च व्यापकानुपलब्धिर्गमिका स्यात्, नान्यदेति ।
	\pend
      

	  \pstart अयमत्र प्रकरणार्थः--प्रबन्धेन भवतो यद्भावे यस्याभावस्तस्य विरोधगतिर्यत्स्वभावश्च येनोपलभ्यते तेन सह कार्यकारणभावोऽपि पञ्चप्रत्यक्षानुपलम्भसमधिगम्यः, व्याप्यव्यापकभावोऽपि प्रत्यक्षानुपलम्भावसेय इति कथमदृश्यस्य सिद्ध्यन्तीति ।
	\pend
      

	  \pstart ननु भवन्तु तेऽन्यत्र दृश्याः, तत्र तावददृश्या एव वक्तव्याः । दृश्यत्वे दृश्यानुपलम्भप्रयोगात् । तत्कथं दृश्यानुपलब्धावितरासामनुपलब्धीनामन्तर्भाव इत्याशङ्क्य तथा तत्रान्तर्भावस्तथा दर्शयितुमाह--\textbf{तदिति} । यस्मादन्यत्र दृश्यत्वेऽपि विरुद्धादीनां दृश्यानुपलम्भेऽन्तर्भावो न घटते, स चाचार्येणोक्तस्तत्तस्मा\textbf{दयमर्थः--सर्वत्र चे}त्यादेर्वाक्यस्य तात्पर्यार्थः । कथं \textbf{दृश्यानुपलब्धेरि}त्यमुमर्थं तावत्प्रसाधयति--\textbf{एके}ति ।
	\pend
      

	  \pstart कार्यकारणभावे का वार्तेत्याह--\textbf{कारणे}ति ।
	\pend
      

	  \pstart यद्येवं व्याप्यव्यापकभावस्य का गतिरित्याह--\textbf{व्यापके}ति ।
	\pend
      

	  \pstart ननु \textbf{च} व्याप्यव्यापकभावनिश्चये तयोरेकत्वात् किमभावनिश्चयेनेत्याह--तत्र व्याप्यव्यापकभावनिश्चये कर्त्त\leavevmode\marginnote{\textenglish{56b/ms}}व्ये । कथमभावप्रतीतिर्व्याप्यव्यापकभावप्रतीतेर्निमित्तमित्याह--इहेति ।
	\pend
      \leavevmode\marginnote{\textenglish{147/dm}}“

	  \pstart विरुद्धविधिना, कारणादिनिषेधेन च यतो दृश्यानुपलब्धिराक्षिप्ता ततो दृश्यानुपलब्धेरेव\footnote{०लब्धिरेव \cite{dp-msC}} कालान्तरवृत्तायाः स्मृतिविषयभूताया अभावप्रतिपत्तिः । अमीषां च प्रयोगाणां दृश्यानुपलब्धावन्तर्भावः । तदनेन सर्वेण दृश्यानुपलब्धावन्तर्भावो दशानामनुपलब्धिप्रयोगाणां पारम्पर्येण दर्शित इति वेदितव्यम् ॥
	\pend
      ”

	  \pstart आस्तां सर्वत्र विरोधादावभावप्रतीतिः, दृश्यानुपलब्धिस्तु क्वोपयुज्यत इत्याह—\textbf{अभावे}ति । \textbf{चो} हेतौ ।
	\pend
      

	  \pstart ननु यदि नाम विरोधादिग्रहणकाले दृश्यानुपलब्धिरासीत्, तथापि न सा विरुद्धोप\textbf{लब्धि-व्यापकानु}पलब्ध्यादिप्रयोगविषये सम्प्रत्यनुवर्त्तते । तत्कथं विरुद्धोपलब्ध्यादीनां तत्रान्तर्भाव इत्याह--\textbf{तस्मादि}ति । यतोऽभावप्रतीतिमन्तरेण न विरोधादिसिद्धिः, अभावसिद्धिश्च दृश्यानुपलब्धे\textbf{स्तस्माद्} हेतो\textbf{र्विरोधा}दिकं \textbf{स्मरते}ति विरुद्धोपलम्भव्यापकानुपलम्भादिप्रयोगकाल इति द्रष्टव्यम् । तदस्मरणे हि यतो विरुद्धादिरिहास्ति, यतोऽयं व्यापकादिर्नास्ति, तस्मात्तत्तन्नास्तीत्यस्याः प्रतीतेरयोगात्--इत्यपि द्रष्टव्यम् । \textbf{स्मरते}ति च तदानीं विरोधादेर्ग्रहणाद् गृहीतस्यैव विकल्पनादुक्तम् । कुतः पुनरवश्यस्मर्त्तव्या सेत्याह--\textbf{दृश्ये}ति ।
	\pend
      

	  \pstart अथ स्यात्--प्राक्तनी दृश्यानुपलब्धिः सदा स्मर्यताम्, तथापि कथमसौ विवक्षिताभावसिद्धावुपयोगं भजते, येनात्मनीतरा अनुपलब्धौ\footnote{ब्धी}रन्तर्भावयतीत्याह--\textbf{विरोधे}ति । \textbf{चो} यस्मात्तस्यां नियतस्मरणायां सत्याम् ।
	\pend
      

	  \pstart ननु विरुद्धोपलब्धिकारणाद्यनुपलब्धिप्रयोगविषये सा नास्ति तत्कथमविद्यमाना सैवाभावप्रतीतेर्निबन्धनमित्याह--तत्रेति वाक्योपन्यासे । \textbf{सम्प्रती}दानीं \textbf{स्मर्यमाणा सैवाभावप्रतीतेर्निबन्धनं} विरुद्धाद्यभावज्ञानस्य करणम् ।
	\pend
      

	  \pstart कथं तर्हि दृश्यानुपलब्धेर्विरुद्धोपलब्ध्यादीनां भेद इत्याह--\textbf{तत} इति । \textbf{ततो} विरोधादिग्रहणकालप्रवृत्ताया दृश्यानुपलब्धेः स्मर्यमाणत्वात्, \textbf{सम्प्रति} सा \textbf{नास्ति । इति}स्तस्मादत्राभावः साध्यते तेना\textbf{भावसाधनत्वेन} ततो \textbf{भिद्यन्ते} विरुद्धोपलब्ध्यादिप्रयोगाः ।
	\pend
      

	  \pstart \textbf{विरुद्धविधिने}त्यादिनोक्तमर्थमुपसंहरति । \textbf{विरुद्धविधिने}त्युपलक्षणम । विरुद्धकार्यादिविधिनाऽपि दृश्यानुपलम्भाक्षेपात् । अन्यथा तासां तत्रान्तर्भावो न स्यात् ।
	\pend
      

	  \pstart ननु स्वातन्त्र्येण त्वया तस्या एव प्राक्प्रवृत्ताया अभावनिश्चयो दर्शितस्तत्र चानुपलब्धीनामन्तर्भावः, न त्वाचार्यस्यायमभिप्रेत इत्याशङ्क्याचार्यस्यैवायमभिप्रेतोऽर्थ इति दर्शयन्नाह—\textbf{तदनेनेति । अनेन “इमे सर्व”} \href{http://http://sarit.indology.info/?cref=2.42}{२. ४२} इत्यादिना\footnote{अत्र मूले प्रदीपानुसारी पाठो नोपलभ्यते । तत्र तु “विरोधकार्यकारणभावाभावासिद्धेः” २. ४६ इति लभ्यते ।} \textbf{व्यापकभावासिद्धे}रित्यन्तेन ।
	\pend
      

	  \pstart यदि साक्षात्तस्यामन्तर्भावस्तदा दश्यानुपलब्धिः स्यात्, न बिरुद्धोपलब्ध्यादिभेद इत्याह—\textbf{पारम्पर्येणे}ति । विरुद्धादिप्रयोगकाले न साऽस्ति केवलं प्राक्प्रवृत्ता सा स्मर्यत इति ।
	\pend
      

	  \pstart अयमत्र प्रकरणार्थः--दृश्यानुपलम्भस्यावक्तव्यत्वेन दशानामप्यनुपलब्धोनां तत्रान्तर्भावः, विरुद्धाद्यभावप्रतीतावनुपयोगश्चेति ।
	\pend
      \leavevmode\marginnote{\textenglish{148/dm}}“

	  \pstart उक्ता दृश्यानुपलब्धिरभावे, अभावव्यवहारे च साध्ये प्रमाणम् । अदृश्यानुपलब्धिस्तु\footnote{लब्धिः किं \cite{dp-msA} \cite{dp-edP} \cite{dp-edH} \cite{dp-edE}} किंस्वभावा, किंव्यापारा\footnote{पारेत्याह--\cite{dp-msB} \cite{dp-msD}} चेत्याह--
	\pend
       “

	  \pstart विप्रकृष्टविषया\footnote{विषयानुप०--\cite{dp-msB} \cite{dp-msC} \cite{dp-edH} \cite{dp-edE} \cite{dp-edN}} पुनरनुपलब्धिः प्रत्यक्षानुमाननिवृत्तिलक्षणा संशयहेतुः ॥ ४७ ॥
	\pend
      ” 

	  \pstart विप्रकृष्टेत्यादि । विप्रकृष्टस्त्रिभिर्देशकालस्वभावविप्रकर्षैर्यस्या विषयः सा विप्रकृष्टविषयेति संशयहेतुः । किंस्वभावा सेत्याह--प्रत्यक्षानुमाननिवृत्तिर्लक्षणं स्वभावो यस्याः सा प्रत्यक्षानुमाननिवृत्तिलक्षणा । न ज्ञानज्ञेयस्वभावेति यावत् ॥
	\pend
      ”

	  \pstart केचित्पुनरत्रैवं ब्रुवते--इहैकज्ञानसंसर्गिवस्त्वन्तरोपलम्भोऽनुपलम्भः । न च शीतस्य निषेधे साध्ये दूरत्वाद् वह्नेर्भास्वररूपोपलब्धिः शीतस्पर्शानुपलब्धिर्युज्यते । येनानुपलब्धिः सिद्ध्येत्, रूपस्पर्शयोरेकज्ञानसंसर्गित्वाभावात् । \leavevmode\marginnote{\textenglish{57a/ms}} न च विरोधग्रहणकालप्रवृत्तदृश्यानुपलम्भस्मारकत्वेन दृश्यानुपलम्भत्वं वाच्यम्, प्राक्प्रवृत्तप्रत्यक्षस्मारकत्वेनापि प्रत्यक्षत्वप्रसंङ्गात् । अत एव विरोधादिग्रहणकालप्रवृत्तदृश्यानुपलम्भस्मारकत्वेनानुपलब्धीनां तत्रान्तर्भावो न युज्यते । नापि स्मृतायास्तस्या एवाभावनिश्चयः, व्याप्तिग्राहकप्रमाणस्मारकत्वेन परार्थानुमानस्य तत्प्रमाणान्तर्भावप्रसङ्गात्, तत एव स्मर्यमाणात् प्रमाणाद् विवक्षितप्रतीतिप्रसङ्गाच्च । तस्मात्सर्वत्रैव सम्प्रतितनो दृश्यानुपलम्भो दर्शनीयस्तद्बलेनैवाभावनिश्चयो वाच्यः, न तु प्राक्प्रवृत्ताद् दृश्यानुपलम्भात् स्मृत्या विषयीकृतादभावनिश्चयः । नापि तत्स्मारकत्वेनानुपलब्धीनां तथात्वमिति ।
	\pend
      

	  \pstart यद्येवं कथङ्कारं स प्रदर्श्यतामिति चेत् । उच्यते । दूराद् वह्ने रूपविशेषं दृष्ट्वा यत्रैवंविधरूपविशेषस्तत्र तावद्देशव्यापकस्तुषारस्पर्शविशेषोऽस्ति । यथा महानसादौ तथाविधमेवरूपमित्यानुमानिकी विशिष्टोष्णस्पर्शप्रतीतिः । सैव च शीतस्पर्शानुपलब्धिरुष्णशीतस्पर्शयोरेकज्ञानसंसर्गित्वात् । विवक्षितोपलम्भादन्य उपलम्भोऽनुपलम्भः । स क्वचित्प्रत्यक्षात्मा क्वचिदनुमानात्मेति न शास्त्रविरोधो न युक्तिविरोधः । तत एव तद्दृश्यानुपलम्भाच्छीतस्पर्शाभावप्रतीतिः ।
	\pend
      

	  \pstart आहत्यन्तं \footnote{आहत्य} दृश्यानुपलब्धेरनुदयाद् दृश्यानुपलब्धेर्भेदेन निर्देशः । अत एव चानुमिताऽनुमानमेतत् । केवलमत्यन्ताभ्यासाञ्झटिति तथाप्रतीत्युदये सत्येकमनुमानमुक्तम् । वस्तुतस्त्वनेकमनुमानमेतत् । एवं व्यापकविरुद्धकार्योपलब्ध्यादावपि सर्वं द्रष्टव्यम् । तथा च व्यापकविरुद्धोपलब्ध्यादिष्वपरमनुमानमेकं प्लवमानमवसेयम् । ययोश्च परस्परपरिहारस्थितलक्षणो विरोधस्तत्र दृश्यानुपलब्धिः स्फुटैव । तेन विरुद्धव्याप्तोपलब्ध्यादिषु सम्प्रत्येव दृश्यानुपलब्धिरस्ति । भेदस्तु पारम्पर्येण तदुदयात् । एवं व्यापककारणानुपलब्ध्यादिष्वकारणव्यापकादेर्धर्मस्यैवानुपलम्भादभावे \footnote{सिद्धे तत एव सामर्थ्यात्कार्यादेरभावावसाय इतीदानीन्तनस्यैव दृश्यानुपलम्भस्योपयोगो} न प्राक्तनस्य स्मृत्यादिविषयीकृतस्य । दृश्यानुपलम्भस्य च साक्षात्कारणे व्यापारात् पारम्पर्येण च विरुद्धाद्यभावे भावे व्यापारात् दृश्यानुपलब्धेर्भेदेन कारणानुपलब्ध्यादीनां प्रयोग इति ॥
	\pend
      \leavevmode\marginnote{\textenglish{149/dm}}“

	  \pstart ननु च प्रमाणात् प्रमेयसत्ताव्यवस्था । ततः प्रमाणाभाबात् प्रमेयाभावप्रतित्तिर्युक्तेत्याह--
	\pend
       “

	  \pstart प्रमाणनिवृत्तावप्यर्थाभावासिद्धेरिति ॥ ४८ ॥ ॥ \footnote{॥ द्वितीयपरिच्छेदः ॥--\cite{dp-msD} \cite{dp-msB} ॥ न्यायबिन्दुप्रकरणे द्वितीयः परिच्छेदः समाप्तः ॥--\cite{dp-edE}}स्वार्थानुमानपरिच्छेदो द्वितीयः समाप्तः ॥
	\pend
      ” 

	  \pstart प्रमाणनिवृत्तावपीत्याह । कारणं व्यापकं च निवर्तमानं कार्यं व्याप्यं च निवर्तयेत् । न च प्रमाणं प्रमेयस्य कारणं नापि व्यापकम् । अतः प्रमाणयोर्निवृत्तावपि अर्थस्य प्रमेयस्य निवृत्तिर्न सिध्यति । ततोऽसिद्धेः संशयहेतुरदृश्यानुपलब्धिः, न निश्चयहेतुः । यत् पुनः प्रमाणसत्तया प्रमेयसत्ता सिध्यति तद् युक्तम् । प्रमेयकार्य हि प्रमाणम् । न च कारणमन्तरेण कार्यमस्ति । न \footnote{न च--\cite{dp-msB} \cite{dp-msD}}तु कारणान्यवश्यं कार्यवन्ति भवन्ति । तस्मात् प्रमाणात् प्रमेयसत्ता व्यवस्थाप्या, न प्रमाणामावात् प्रमेयाभावव्यवस्थेति ॥
	\pend
       

	  \pstart ॥ \footnote{॥ न्यायबिन्दुटीकायां द्वितीयः परिच्छेदः समाप्तः ॥ \cite{dp-msA} \cite{dp-msB} \cite{dp-edP} \cite{dp-edH} \cite{dp-edE} ॥ न्यायबिन्दुटीकायां द्वितीयः परिच्छेदः ॥ \cite{dp-msD}}आचार्यधर्मोत्तरकृतायां न्यायबिन्दुटीकायां स्वार्थानुमानो द्वितीयः परिच्छेदः ।
	\pend
      ”

	  \pstart सम्प्रत्यदृश्यानुपलब्धिमधिकृत्योक्तं व्याचक्षाण आह--\textbf{उक्तेति ।}
	\pend
      

	  \pstart ननु यदि प्रत्यक्षाऽनुमाननिवृत्तिमात्ररूपाऽदृश्यानुपलब्धिरभावे साध्ये संशयहेतुरनैकान्तिकी तर्हि सा ततश्च हेत्वाभासावसर एव वक्तव्या । तत् किमिहोच्चयत इति चेत् । नैतम\footnote{द}स्ति । यतो न सा वचनव्यक्त्याऽनुपलब्धिक्षणप्राप्तस्यानुपलब्धिरसद्व्यवहारे साध्ये न प्रमाणम्, ऐकान्तिकसद्व्यवहारनिषेधे तु प्रमाणमिति प्रदर्शनात् ।
	\pend
      

	  \pstart \textbf{न ज्ञानज्ञेयस्वभावेति यावदि}त्यार्थं न्यायमाश्रित्योक्तं न शाब्दमिति द्रष्टव्यम् ॥
	\pend
      

	  \pstart यदि प्रमाणनिवृत्त्या\leavevmode\marginnote{\textenglish{57b/ms}}प्रमेयानिवृत्तिस्तर्हि तत्सत्तयाऽपि न प्रमेयसत्ता सिद्ध्येदित्याशङ्क्य तत्रोपपत्तिं दर्शयन्नाह--\textbf{यत्पुनरि}ति । ननु कारणमप्यवश्यं कार्यवद् भवति ततश्च सति ज्ञेये ज्ञानेनाप्यवश्यभाव्यम् । तच्चेन्नास्ति ज्ञेयमपि नास्त्येव । ततः सिद्ध्यत्येवाऽदृश्यस्याप्यभाव इत्याशङ्क्याह--\textbf{न त्वि}ति । \textbf{तु}ना कार्यधर्मात् कारणधर्मस्य वैधर्म्यमाह । एतच्च कारणमात्राभिप्रायेणोक्तम्, तथा चैत्तथा प्रागेव निर्णीतम् । \textbf{तस्मादि}त्यादिनाऽस्यैव प्रकृतस्योपसंहार इति ॥
	\pend
      

	  \pstart ॥ पण्डित\textbf{जितारि}शिष्यदु\textbf{र्वेकमिश्र}विरचित\textbf{धर्मोत्तरनिबन्ध}स्य द्वितीयः परिच्छेदः ॥
	\pend
      
	    
	    \endnumbering% ending numbering from div
	    \endgroup
	    
	  
	  
	% new div opening: depth here is 0
	
	    
	    \begingroup
	    \beginnumbering% beginning numbering from div depth=0
	    
	  
\chapter[{तृतीयः परार्थानुमानपरिच्छेदः ।}]{तृतीयः परार्थानुमानपरिच्छेदः ।}\leavevmode\marginnote{\textenglish{150/dm}}“

	  \pstart स्वार्थ-परार्थानुमानयोः स्वार्थं व्याख्याय परार्थ व्याख्यातुकाम आह--
	\pend
       “

	  \pstart त्रिरूपलिङ्गाख्यानं \footnote{परार्थानुमा० \cite{dp-msB} \cite{dp-edP} \cite{dp-edH} \cite{dp-edE} \cite{dp-edN}}परार्थमनुमानम् ॥ १ ॥
	\pend
      ” 

	  \pstart त्रिरूपलिङ्गाख्यानमिति । त्रीणि रूपाणि--अन्वय-व्यतिरेक-पक्षधर्मत्वसंज्ञकानि यस्य तत् त्रिरूपम् । त्रिरूपं च \footnote{च लिङ्ग चं \cite{dp-msA}}तल्लिङ्गं च तस्याख्यानम् । आख्यायते प्रकाश्यतेऽनेनेति—त्रिरूपं \footnote{त्रिरूपलि० \cite{dp-edE}}लिङ्गमिति आख्यानम् । किं पुनस्तत् ? वचनम् । वचनेन हि त्रिरूपं\footnote{त्रिरूपलि० \cite{dp-msC}} लिङ्गमाख्यायते । परस्मायिदं \footnote{परस्मायिति परा० \cite{dp-edE}}परार्थम् ॥
	\pend
       

	  \pstart ननु \footnote{ननु सम्य० \cite{dp-msA}}च सम्यग्ज्ञानात्मकमनुमानमुक्तम् । तत् किमर्थं संप्रति वचनात्मकमनुमानमुच्यत इत्याह--
	\pend
       “

	  \pstart कारणे कार्योपचारात् ॥ २ ॥
	\pend
      ” 

	  \pstart कारणे कार्योपचारादिति । \footnote{त्रिरूपलिङ्गालम्बना स्मृतिः \cite{dp-msC} \cite{dp-msD} \cite{dp-msB}}त्रिरूपलिङ्गाभिधानात् त्रिरूपलिङ्गस्मृतिरुत्पद्यते\footnote{श्रोतुः--\cite{dp-msD-n}} । स्मृतेश्चानुमानम् । \footnote{तस्यानुमानस्य--\cite{dp-msA} \cite{dp-msB} \cite{dp-edP} \cite{dp-edH} \cite{dp-edE} \cite{dp-edN}}तस्माद् अनुमानस्य परम्परया त्रिरूपलिङ्गाभिधानं कारणम् । तस्मिन् कारणे वचने कार्यस्य अनुमानस्योपचारः समारोपः क्रियते । ततः समारोपात् कारणं वचन-
	\pend
      ”

	  \pstart त्रिरूपं लिङ्गं ज्ञातमपि वक्तुमविदुषो बालस्य व्युत्पादनार्थं त्रिरूपलिङ्गाख्यानलक्षणं यत्परार्थमनुमानमुक्तं तद् व्याख्यातुं \textbf{स्वार्थेत्यादि}ना प्रस्तौति । द्वयो रूपयोरभिधानादेकस्य गम्यमानत्वा\textbf{दाख्यायते प्रकाश्यतेऽनेनेति त्रिरूपं लिङ्गमि}ति विवृतं न त्वभिधीयतेऽनेनेति । अभिधेयस्य गम्यमानस्य च प्रकाश्यत्वं तुल्यमिति प्रकाश्यते इत्यनेन द्वयोः सङ्ग्रहः । येनार्थक्रमेणात्मनः परोक्षार्थज्ञानमुत्पन्नं तेनैव क्रमेण परसन्ताने लिङ्गिज्ञानोत्पिपादयिषया त्रिरूपस्य लिङ्गस्य ख्यापकं यद् वचनं तत्परार्थमनुमानमिति द्रष्टव्यम् ॥
	\pend
      

	  \pstart कारणे वचने कार्यस्य ज्ञानस्योपचारात् समारोपात् ।
	\pend
      

	  \pstart कथं पुनर्वचनस्यानुमानहेतुत्वमित्याह--\textbf{त्रिरूपेति} ।
	\pend
      \leavevmode\marginnote{\textenglish{151/dm}}“

	  \pstart मनुमानशब्देनोच्यते । औपचारिकं\footnote{औपचारकम्--\cite{dp-msA}} वचनमनुमानम्, न मुख्यमित्यर्थः । न यावत्\footnote{न च यावत् \cite{dp-msA} \cite{dp-msB} \cite{dp-msD} \cite{dp-edP} \cite{dp-edH} \cite{dp-edE} \cite{dp-edN}} किंचिदुपचारादनुमानशब्देन वक्तुं शक्यं तावत् सर्वं व्याख्येयम् । किन्त्वनुमानं व्याख्यातुकामेनानुमानस्वरूपस्य\footnote{स्वरूपस्यैव व्या० \cite{dp-msC}} व्याख्येयत्वान्निमित्तं व्याख्येयम् । निमित्तं च त्रिरूपं लिङ्गम् । तच्च स्वयं वा प्रतीतमनुमानस्य निमित्तं भवति, परेण वा प्रतिपादितं भवति\footnote{“भवति” नास्ति \cite{dp-msA} \cite{dp-msC} \cite{dp-edP} \cite{dp-edE} \cite{dp-edN}} । तस्माल्लिङ्गस्य स्वरूपं \footnote{“च” नास्ति \cite{dp-msA} \cite{dp-edP} \cite{dp-edH} \cite{dp-edE} \cite{dp-edN}}च व्याख्येयम्, तत्प्रतिपादकश्च शब्दः । तत्र स्वरूपं स्वार्थानुमाने व्याख्यातम् । प्रतिपादकश्च\footnote{०दकः शब्द \cite{dp-msA} \cite{dp-msB} \cite{dp-edP} \cite{dp-edH} \cite{dp-edE} \cite{dp-edN}} शब्द इह व्याख्येयः । ततः प्रतिपादकं शब्दमवश्यं वक्तव्यं दर्शयन् अनुमानशब्देनोक्तवानाचार्य इति परमार्थः ॥
	\pend
       

	  \pstart परार्थानुमानस्य प्रकारभेदं दर्शयितुमाह--
	\pend
      ”“

	  \pstart तद् द्विविधम् ॥ ३ ॥
	\pend
      ”

	  \pstart ननु च त्रिरूपलिङ्गाभिधानादवगते सति धर्मिणि लिङ्गं ज्ञायते । तस्य तु व्याप्तिः स्मर्यते । तत्कथं “त्रिरूपलिङ्गवचनात् तत्स्मृतिरुत्पद्यते” इत्युच्यत इति चेत् । उच्यते । गृह्यमाणमपि धूमादिवस्तु न तावल्लिङ्गं यावद् वह्न्यादिसाध्याविनाभूततया न ज्ञायते । तथात्वं च तस्य न तदा ग्राह्यमपि तु पूर्वगृहीतमेव स्मर्त्तव्यमिति सूक्तं \textbf{त्रिरूपलिङ्गस्मृतिरुत्पद्यत} इति । \textbf{स्मृते}रिति पक्षधर्मग्रहणसहिताया इति द्रष्टव्यम् ।
	\pend
      

	  \pstart अयमर्थः--वचनमपि त्रिरूपं लिङ्गं स्मरयत् परोक्षार्थज्ञानस्य परम्परया कारणं भवदुपचारादनुमानमुच्यत इति ।
	\pend
      

	  \pstart अथाबाधितत्वाद्यपि लिङ्गस्य लक्षणमित्याचक्षते केचिदिति विप्रतिपत्तिदर्शनात् तद्व्युत्पादनं युक्तम्, न तु तद्वचनम्, तस्य विप्रतिपत्त्यभावादिति चेत् । न अत्राप्यव्याप्तिव्यतिरेकाभ्यां निगदन्तो विप्रतिपन्ना इत्यस्यापि व्युत्पादनं न्याय्यम् ।
	\pend
      

	  \pstart अथाऽपि स्यात्, यदि परम्परयाऽनुमानहेतुत्वेन वचनमुपचारादनुमानमिति व्युत्पाद्यते तर्हि जिज्ञासास्वास्थ्यादिकमपि परम्परयाऽनुमानहेतुत्वादनुमानशब्देन वक्तुं शक्यमिति तदपि किं नोच्यत इत्याह--\textbf{न यावदि}ति ।
	\pend
      

	  \pstart ननु स्वास्थ्यादिकमपि निमित्तमिति तदवस्थो दोषः । न । \textbf{निमित्तं व्याख्पेय}मित्यव्यहितमसाधारणं निमित्तमाख्येयमित्यर्थः ।
	\pend
      

	  \pstart ननु स्वयं प्रतीतं लिङ्गमनुमानस्य निमित्तम् । तत्किं तद्वचनेन व्याख्यातेनेत्याह—\textbf{तच्चे}ति । \textbf{चो} यस्मादर्थे । वाशब्दो विकल्पार्थः । यतः परेण प्रतिपादितमपि तल्लिङ्गमनुमानस्य निमित्तं \textbf{तत}स्तस्माद\textbf{वश्यं वक्तव्यं प्र}\leavevmode\marginnote{\textenglish{58a/ms}}\textbf{तिपादकं} लिङ्गप्रतिपादकं वचनं \textbf{दर्शयन्ननुमानशब्देनोक्तवानाचार्यः} ।
	\pend
      \leavevmode\marginnote{\textenglish{152/dm}}“

	  \pstart तद् द्विविधमिति । तदिति परार्थानुमानम् । द्वौ विधौ प्रकारौ यस्य तद् द्विविधम् ॥ कुतो द्विविधमित्याह--
	\pend
       “

	  \pstart प्रयोगभेदात् ॥ ४ ॥
	\pend
      ” 

	  \pstart प्रयोगस्य शब्दव्यापारस्य भेदात् । प्रयुक्तिः प्रयोगोर्थाभिधानम् । शब्दस्यार्थाभिधानव्यापारभेदाद् द्विविधमनुमानम् ॥
	\pend
       

	  \pstart तदेवाभिधानव्यापारनिबन्धनं\footnote{अभिधानस्य व्यापारो निबन्धनं यस्य--\cite{dp-msD-n}} द्वैविध्यं दर्शयितुमाह--
	\pend
       “

	  \pstart माधर्म्यवद्वैधर्म्यवच्चेति\footnote{वच्च ॥ \cite{dp-msC}} ॥ ५ ॥
	\pend
      ” 

	  \pstart साधर्म्यवद्वैधर्म्मवच्चेति । समानो धर्मोऽस्य\footnote{धर्मो यस्य \cite{dp-msC} \cite{dp-msA} \cite{dp-edP} \cite{dp-edH} \cite{dp-edE} \cite{dp-edN}} सोऽयं सधर्मा । तस्य भावः साधर्म्यम् । विसदृशो धर्मोऽस्य विधर्मा । विधर्मणो भावो वैधर्म्यम् । दृष्टान्तधर्मिणा सह साध्यधर्मिणः सादृश्यं हेतुकृतं साधर्म्यमुच्यते । असादृश्यं च हेतुकृतं वैधर्म्यंमुच्यते । \footnote{तयोः साधर्म्यवैधर्म्ययोः--\cite{dp-msD-n}}तत्र यस्य साधनवाक्यस्य साधर्म्यमभिधेयं तत् साधर्म्यवत् । यथा--यत कृतकं तदनित्यं यथा घटः, तथा च कृतकः शब्द इत्यत्र कृतकत्वकृतं दृष्टान्तसाध्यधर्मिणोः सादृश्यमभिधेयम् । यस्य तु वैधर्म्यमभिघेयं तद् वैधर्म्यवत् । यथा--यन्नित्यं तदकृतकं दृष्टं यथाकाशम् । शब्दस्तु कृतक इति \footnote{इति अकृतकत्वकृतम्--\cite{dp-msC}}कृतकत्वाऽकृतकत्वकृतं शब्दाकाशयोः साध्यदृष्टान्तधर्मिणोरसादृश्यमिहाभिधेयम् ॥
	\pend
       

	  \pstart यद्यनयोः प्रयोगयोरभिधेयं भिन्नं कथं तर्हि त्रिरूपं लिङ्गमभिन्नं प्रकाश्यमित्याह--
	\pend
       “

	  \pstart नानयोरर्थतः कश्चिद्भेदः ॥ ६ ॥
	\pend
      ” 

	  \pstart नानयोरर्थत इति अर्थः प्रयोजनम्\footnote{०जनम् प्रकाशयितव्यं वस्तु यदुद्दिश्यानु० \cite{dp-msA} \cite{dp-edP} \cite{dp-edH} \cite{dp-edE} \cite{dp-edN}} । यत् प्रयोजनं प्रकाशयितव्यं वस्तु उद्दिश्यानुमाने प्रयुज्येते, ततः \footnote{प्रयोजनान्नानयोर्भेदः--\cite{dp-msB} \cite{dp-msD}}प्रयोजनादनयोर्न भेदः कश्चित् । त्रिरूपं हि लिङ्गं प्रकाशयितव्यम् । तदुद्दिश्य द्वे अप्येते प्रयुज्येते । द्वाभ्यामपि त्रिरूपं लिङ्गं प्रकाश्यत एव । ततः प्रकाशयितव्यं प्रयोजनमनयोरभिन्नम् । तथा च न ततो भेदः कश्चित् ॥
	\pend
      ”

	  \pstart अनेनैतदाह-न स्वास्थ्यादि प्रतिपन्नमपि परोक्षार्थप्रतिपादकं येन तदुच्येत । तद्वचनमवश्यं दर्शयितव्य\textbf{मनुमानशब्देना}भिलप्येत । स्वयमशक्तमपि तु हेतुवचनं परोक्षप्रकाशनवस्तुसूचकत्वादनुमानशब्देनोक्तमिति । \textbf{इति परमार्थ} एवमस्योपचारस्य परमः प्रकृष्टोऽर्थः प्रयोजनम् ॥
	\pend
      

	  \pstart विधशब्देन च विगृह्णतोऽभिप्रायः प्रागेव प्रदर्शितः ॥
	\pend
      

	  \pstart अभिधानमर्थप्रकाशनम् ॥
	\pend
      \leavevmode\marginnote{\textenglish{153/dm}}“

	  \pstart अभिधेयभेदोऽपि तर्हि न स्यादित्याह--
	\pend
       “

	  \pstart अन्यत्र प्रयोगभेदात् ॥ ७ ॥
	\pend
      ” 

	  \pstart अन्यत्र प्रयोगभेदादिति । प्रयोगोऽभिधानं वाचकत्वम् । वाचकत्वभेदादन्यो भेदः प्रयोजनकृतो नास्तीत्यर्थः ।
	\pend
       

	  \pstart एतदुक्तं भवति । अन्यदभिधेयमन्यत् प्रकाश्यं प्रयोजनम् । तत्राभिघेयापेक्षया वाचकत्वं भिद्यते । प्रकाश्यं त्वभिन्नम् । अन्वये हि कथिते वक्ष्यमाणेन न्यायेन व्यतिरेकगतिर्भवति । व्यतिरेके चान्वयगतिः । ततस्त्रिरूपं लिङ्गं प्रकाश्यमभिन्नम् । न च यत्राभिधेयभेदस्तत्र सामर्थ्यगम्योऽप्यर्थो\footnote{प्रयोजनम्--\cite{dp-msD-n}} भिद्यते । यस्मात् “पीनो देवदत्तो दिवा न भुङ्क्ते” “पीनो देवदत्तो रात्रौ भुङ्क्ते” इत्यनयोर्वाक्ययोरभिधेयभेदेपि गम्यमानमेकमेव\footnote{गम्यं दिवा भोजनाभावविशेषं पीनत्वं रात्रिभोजनकार्यमेकमेव--\cite{dp-msD-n}} तद्वदिहाभिधेयभेदेपि गम्यसानं वस्त्वेकमेव ॥
	\pend
      ”

	  \pstart केन कस्य किं कृतं च साधर्म्यं वैधर्म्यं चेत्याह--\textbf{दृष्टान्ते}ति । \textbf{सादृश्यं हेतुकृतमि}ति हेतुसद्भावद्वारकम् । \textbf{असादृश्यं च हेतुकृतमि}ति हेतुसद्भावासद्भावद्वारकं द्रष्टव्यम् । वतुबर्थं प्रयोजयितुमाह--\textbf{तत्रेति} वाक्योपन्यासे । \textbf{यस्य वाक्यस्य साधर्म्य} सादृश्यम\textbf{भिधेयम}स्ति । एतदेवोदाहरणेन दर्शयन्नाह--\textbf{यथे}ति । \textbf{यत्कृतकमि}ति । यद् यद् कृतकमिति वीप्सार्थों विवक्षितः, \textbf{तदि}त्यत्रापि । \textbf{तथा च कृतकः शब्द} इति पक्षधर्मताकथनमिदम् । न त्वेवं पक्षधर्मो दर्शनीयः । “कृतकश्च शब्दः” इत्येतावतैव गतार्थत्वेन तथाशब्दस्य वैयर्थ्यात् । ततः “कृतकश्च शब्दः” इत्येव दर्शनीयः । इतरथा परेषामिवोपनयप्रयोगः स्यात् । स चायुक्त इति । \textbf{यन्नित्यमि}ति साध्याभावानुवाद\textbf{स्तदकृतकमि}ति साधनाभावविधिः ॥
	\pend
      

	  \pstart \textbf{अभिधेयं भिन्नमि}ति ब्रुवतोऽयमाशयः--साधर्म्यवत्प्रयोगस्यान्वयः पक्षधर्मता चाभिधेया । \textbf{अभिन्नं} साधारणं \textbf{प्रकाश्यं} द्वयोः प्रयोगयोः ।
	\pend
      

	  \pstart \textbf{अर्थः} प्रयोजनवाच्याचार्येणोक्तो नाभिधेयवाचीति दर्शयति \textbf{यदुद्दिश्ये}ति स्पष्टयन् प्रयुज्यतेऽनेनेति \textbf{प्रयोजनं} साधनवाक्यस्य प्रवर्त्तकं लिङ्गवस्तूक्तं न फलं प्रयोजनमिति दर्शयति । \textbf{प्रकाशयितव्यं} रूपत्रययोगिलिङ्गं त\textbf{च्चाभिन्नं} साधारणं द्वयोरपि प्रयोगयोः, द्वाभ्यामपि तस्यैव प्रतिपादनात् । अनुमानहेतुत्वादनुमाने साधर्म्यवैधर्म्यवती वाक्ये कथिते । \textbf{तत} इति \textbf{प्रयोजन-} समानाधिकरणं न हेतुपदमेतत् ॥
	\pend
      

	  \pstart नन्वभिधेयमेव प्रकाश्यं तत्कथं प्रकाश्याभेद इत्याशङ्क्याह--\textbf{एतदुक्तं भवती}ति । \textbf{अन्यत्प्रकाश्यमि}ति सामर्थ्यगम्यं प्रकाश्यम् । न तु तत्राभिधाव्यापार इत्याकूतम् । \textbf{तत्रे}ति वाक्योपक्षेपे ।
	\pend
      

	  \pstart यदि साधर्म्यवद्वाक्येऽन्वयोऽभिधेयस्तर्हि कथं व्यतिरेकः प्रकाश्यतां गतः ? बैधर्म्यवाक्ये  \leavevmode\marginnote{\textenglish{154/dm}} “
	  
	तत्र साधर्म्यवत्प्रोगः\footnote{०र्म्यवत् यदुप०--\cite{dp-msD} \cite{dp-msB} \cite{dp-edP} \cite{dp-edH} \cite{dp-edE} \cite{dp-edN}}--यदुपलब्धिलक्षणप्राप्तं सन्नोपलभ्यते सोऽसद्व्यवहारविषयः सिद्धः । यथाऽन्यः कश्रिद् दृष्टः शशविषाणादिः । नोपलभ्यते च क्वचित् प्रदेशविशेष\footnote{विषये उप० \cite{dp-msC}} उपलब्धिलक्षणप्राप्तो घट\footnote{घट इति ॥ \cite{dp-msD} \cite{dp-msB} \cite{dp-edP} \cite{dp-edH} \cite{dp-edE} \cite{dp-edN}} इत्यनुपलब्धिप्रयोगः ॥ ८ ॥” “
	  
	तत्रेति तयोः साधर्म्यवैधर्म्यवतोरनुमानयोः साधर्म्यवत् \footnote{तावदुहारणमुदाहर्त्तुमनुप० \cite{dp-msC} \cite{dp-msB}}तावदुहारन्ननुपलब्धिमाह—यदित्यादिना । यदुपलब्धिलक्षणप्राप्तं--यद् दृश्यं सन्नोपलभ्यते इत्यनेन दृश्यानुपलम्भोऽनूद्यते । सोऽसद्व्यवहारविषयः\footnote{विषयस्तद० \cite{dp-msB} \cite{dp-msC} \cite{dp-msD}} सिद्धः--तदसदिति व्यवहर्तव्यमित्यर्थः । अनेनासद्व्यवहारयोग्यत्वस्य\footnote{योग्यत्वे विधिः \cite{dp-msB}} विधिः कृतः । ततश्चासद्व्यवहारयोग्यत्वे\footnote{०हारस्य योग्य० \cite{dp-msA} \cite{dp-msB} \cite{dp-edP} \cite{dp-edH} \cite{dp-edE} \cite{dp-edN}} दृश्यानुपलम्भो नियतः कथितः । दृश्यमनुपलब्धमसद्व्यवहारयोग्यमेवेत्यर्थः । साधनस्य च साध्येऽर्थे नियतत्वकथनं व्याप्ति-” च यदि व्यतिरेकोऽभिधेयः, कथमन्वयः प्रकाश्यतामापन्न इत्याह--\textbf{अन्वय} इति । हिर्यस्मात् । \textbf{वक्ष्यमाणेन “साधर्म्येणापि ही”} त्यादिना प्रतिपादयिष्यमाणेन न्यायेन युक्ता ।
	\pend
      

	  \pstart स्यान्मतम्--ययोरभिधेयं भिन्नं तयोः सामर्थ्यगम्यमप्यवश्यं भिद्यते--यथा गङ्गायां घोषः, कूपे गर्गकुलमित्याशङ्क्याह--\textbf{न चे}ति । \textbf{चो} यस्मादर्थे । \textbf{अर्थः} प्रयोजनं \textbf{यदुद्दिश्य प्रवृत्तं वाक्यम्} ।
	\pend
      

	  \pstart उपपत्तिमाह--\textbf{यस्मादि}ति । एकस्य वाक्यस्य दिवा भोजनाभावोऽभिधेयोऽन्यस्य रात्रिभोजनमित्यनयोर्वाक्य\leavevmode\marginnote{\textenglish{58b/ms}}योरभिधेयभेदोऽस्ति । तस्मिन् सत्यपि यथा भोजनविशिष्टं देवदत्तस्य पीनत्वं प्रतिपाद्यं न भिद्यते \textbf{तद्वदत्राभिधेयस्या}न्वयपक्षधर्मतालक्षणस्य व्यतिरेकपक्षधर्मतालक्षणस्य \textbf{भेदेऽपि गम्यमानमेकम}भिन्नम् ।
	\pend
      

	  \pstart अथ गम्ययानं सामर्थ्यात् प्रतीयमानमुच्यते । तच्च दृष्टान्तदार्ष्टान्तिकयोर्भिद्यत एव । तथाहि दिवा भोजननिषेधवाक्यस्य गम्यमानं रात्रिभोजनं रात्रिभोजनविधानवाक्यस्य तु गम्यमानं दिवा भोजननिषेधनम्, तथा दार्ष्टान्तिकेपि \textbf{साधन} \footnote{साधर्म्य} वद् वाक्ये व्यतिरेको गम्यमानो वैधर्म्यवद्वाक्ये चान्वयो गम्यमानः । यदेकस्याभिधेयं तदेकस्य गम्यमानम्, यदन्यस्य गम्यमानं तदितरस्याभिधेयमिति संक्षेपः । ततः कथमुच्यते वाक्ययोर्गम्यमानमेकमिति । किन्नोच्यते ? गम्यमानशब्दस्येहान्यार्थस्य विवक्षितत्वात् । तथाहि गम्यमानशब्देनात्राभिधेयं सामर्थ्यप्रकाश्यञ्च । यत्तु द्वयं प्रतीयमानतामात्रेणोपाधिना विवक्षितं तच्च दृष्टान्तवाक्ययोर्भोजननिमित्तपीनत्वलक्षणं \textbf{दा}र्ष्टान्तिकवाक्ययोश्चान्वयतिरेकपक्षधर्मतात्मकरूपत्रययोगलिङ्गलक्षणमेकमभिन्नमित्यनवद्यमेतत् ॥
	\pend
      

	  \pstart साधर्म्यमभिधेयं यस्य विद्यते तदुदाहरणेन दर्शय\textbf{न्ननुपलब्धिमाह । तावच्छब्दः क्रमे} ।  \leavevmode\marginnote{\textenglish{155/dm}} “
	  
	कथनम् । यथोक्तम् “व्याप्ति\footnote{व्याप्तिव्या० \cite{dp-msA}} र्व्यापकस्य तत्र भाव एव, व्याप्यस्य \footnote{स्य च \cite{dp-msB} \cite{dp-msC} \cite{dp-msD} \cite{dp-edP} \cite{dp-edH} \cite{dp-edE} \cite{dp-edN}}वा तत्रैव भावः” इति । \href{http://http://sarit.indology.info/?cref=hb.1.4}{हेतु० पृ० ५३} व्याप्तिसाधनस्य प्रमाणस्य विषयो दृष्टान्तः । तमेव दर्शयितुमाह--यथान्य इति । साध्यधर्मिणोऽन्यो दृष्टान्त इत्यर्थः । दृष्ट इति प्रमाणेन\footnote{प्रत्यक्षेण निश्चित इति न व्याकर्तव्यम्--\cite{dp-msD-n}} निश्चितः । शशविषाणं हि न चक्षुषा विषयीकृतम् अपि तु प्रमाणेन दृश्यानुपलम्भेनासद्व्यवहारयोग्यं विज्ञातम् । शशविषाणमादिर्यस्यासद्व्यवहारविषयस्य स तथोक्तः । शशविषाणादौ हि दृश्यानुपलम्भमात्रनिमित्तोऽसद्वयवहारः\footnote{निमित्तान्तराभावोपदर्शनेन दृश्यानुपलब्धिव्यतिरिक्तासम्भविनिमित्तश्चासद्व्यवहारः--\cite{dp-msD-n}} प्रमाणेन सिद्धः । तत एव\footnote{ततः प्रमा० \cite{dp-msC} दृश्यानुपलम्भादेव--\cite{dp-msD-n}} प्रमाणादनेन\footnote{व्याप्तिसाधकेन--\cite{dp-msD-n}} वाक्येनाभिधीयमाना व्याप्तिर्ज्ञातव्या । 
	  
	संप्रति व्याप्तिं कथयित्वा दृश्यानुपलम्भस्य पक्षधर्मत्वं दर्शयितुमाह--नोपलभ्यते चेति ।” स च द्विरावृत्त्याऽनुपलब्धिमित्यत्रापि द्रष्टव्यः । तेनायमर्थः--साधर्म्यवत्तावदुदाहरन्ननुपलब्धिमाह । पश्चाद् वैधर्म्यवदुदाहरन् वक्ष्यति । तथाऽनुपलब्धिं तावदाह पश्चात्स्वभावकार्ये वक्ष्यतीति ।
	\pend
      

	  \pstart नन्वाचार्येणैव साधर्म्यवद् वैधर्म्यवच्चोदाहृतमत्रेति किमिति \textbf{धर्मोत्तरेण}--यथा यत्कृतकमित्यादिना साधर्म्यवद् वैधर्म्यवच्चोदाहृतं प्रागिति चेत् । नैष दोषः । दृष्टान्तसाध्यधर्मिणोः सादृश्याख्यं साधर्म्यम्, वैसदृश्याख्यं वैधर्म्यं च हेतुद्वारकमेव न तु सामान्येन, अन्यथा प्रतियोग्यपेक्षयाऽपि साधर्म्यं वैधर्म्यं च केनचिदाकारेणास्तीति न तन्निराकृतं स्यादिति दर्शयितुं तदुदाहृतम्, न त्वाचार्येण नोदाहृतमित्युदाहृतमिति । तेनात्राप्याचार्यीये निदर्शने हेतुकृतमेव तत्प्रत्येतव्यं व्यवस्थितम् ।
	\pend
      

	  \pstart \textbf{असद्व्यवहारः}--असदिति ज्ञानमसदित्यभिधानं निःशङ्का च गमागमादिका प्रवृत्तिः । यतो दृश्यानुपलम्भोऽभूदतोऽसद्व्यवहारयोग्यत्वं विहितम् । ततस्तस्मात् । \textbf{चो}ऽवधारणे ।
	\pend
      

	  \pstart अयमाशयः--यदनूद्यते तद् व्याप्यम् । यद् विधीयते तद् व्यापकम् । व्याप्यं च व्यापके नियतं भवतीति । एवमुत्तरत्राप्यनुवादविधिक्रमो द्रष्टव्यः ।
	\pend
      

	  \pstart अथ द्व्यवयवे साधनवाक्ये दर्शयितव्ये व्याप्तिः पक्षधर्मता च दर्शनीया । न चात्र व्याप्तिरुपदर्शिता, केवलमनुवादविधिक्रमो दर्शितः, पक्षधर्मता च दर्शयिष्यते । तत्कथं परिपूर्णं साधनवाक्यमिदं भविष्यतीत्याशङ्क्याह--\textbf{साधनस्ये}ति । \textbf{चो} हेतौ ।
	\pend
      

	  \pstart ननु परोक्षार्थप्रतिपत्तौ सर्वथाऽनुपयोगी दृष्टान्तस्तत्किं तेनाख्यातेनेत्याह \textbf{व्याप्ती}ति । व्याप्यव्यापकधर्मलक्षणा \leavevmode\marginnote{\textenglish{59a/ms}} व्याप्तिः साध्यते निश्चीयते येन प्रमाणेन तस्य \textbf{विषयो} यत्र  \leavevmode\marginnote{\textenglish{156/dm}} “
	  
	प्रदेश एकदेशः पृथिव्याः । स एव विशिष्यतेऽन्यस्मादिति विशेषः एकः । प्रदेशविशेष इत्येकस्मिन् प्रदेशे । क्वचिदिति । प्रतिपत्तुः प्रत्यक्ष\footnote{प्रत्यक्षे । एको० \cite{dp-msC} \cite{dp-msD} \cite{dp-edE} \cite{dp-edN}} एकोऽपि प्रदेशः । स एवाभावव्यवहाराधिकरणं यः प्रतिपत्तुः प्रत्यक्षो नान्यः । उपलब्धिलक्षणप्राप्त इति दृश्यः । यथा चासतोऽपि घटस्य समारोपितमुपलब्धिलक्षणप्राप्तत्वं तथा व्याख्यातम् ॥ 
	  
	स्वभावहेतोः साधर्म्यवन्तं प्रयोगं दर्शयितुमाह-- “
	  
	तथा स्वभावहेतोः प्रयोगः--यत् सत् तत् सर्वमनित्यम्, यथा घटादि रिति शुद्धस्य\footnote{शुद्धस्वभावस्य प्रयोगः \cite{dp-msC} ।} स्वभावहेतोः प्रयोगः ॥ ९ ॥” 
	  
	तथेति । यथाऽनुपलब्धेस्तथा स्वभावहेतोः साधर्म्यवान् प्रयोग इत्यर्थः । यत् सदिति” प्रवृत्तं प्रमाणं साध्यसाधनयोर्व्याप्तिमवस्यति । स च विषयो \textbf{दृष्टो} निश्चितः साध्यसाधनयोरन्तोऽवसानं यथायोगं नियतत्वनियमविषयत्वनिपुणो यस्मिन्निति व्युत्पत्त्या दृष्टान्तशब्दोऽभिलप्यः । तमेव ख्यापयितुमाहाचार्यः । अनेनैतदाकूतम्--व्याप्तिसाधकप्रमाणस्याधिकरणतां गच्छन् दृष्टान्तः साधनावयवस्य व्याप्तेः प्रतिपत्त्यङ्गम् । न तु साक्षात्साधनस्य । नापि साध्यसिद्धेः । तद्वचनमपि तत्स्मारकत्वेन साधनवाक्य उपयुज्यते । अत एव वचनसाधनवाक्यस्यावयवोऽथ च प्रयोक्तव्य इति । कुतोऽन्य इत्याह--\textbf{साध्यधर्मिण} इति । \textbf{शशवि}षाणादेश्च व्याप्तिसाधकप्रमाणाधिकरणत्वेन दृष्टान्तरूपत्वाद् \textbf{दृष्टान्त इत्यर्थ} इति स्पष्टयति ।
	\pend
      

	  \pstart ननु दृष्टश्चक्षुषा ज्ञात इति किं न व्याख्यायते ? किं पुनरेवं व्याख्यायत इत्याह—\textbf{शशेति । ही}ति यस्मात् । \textbf{विषयीकृतं} विज्ञातमिति चातीते निष्ठां प्रयुञ्जानः प्राग्भावि व्याप्तिग्रहणं दर्शयति । कथं पुनः शशविषाणादि दृष्टान्तो येन सा ख्याप्यत इत्याह--\textbf{शशेति} । \textbf{हि}र्यस्मात् । \textbf{दृश्यानुपलम्भ} एव \textbf{तन्मात्रं} तन्नि\textbf{मित्तं} यस्य स तथा । अनेन व्याप्तिसाधकप्रमाणाधिकरणत्वात्तस्य दृष्टान्तरूपतामाह । किं तत्प्रमाणं येन तत्र प्रवृत्तेन दृश्यानुपलम्भाभावव्यवहारयोग्यत्वयोर्व्याप्तिः साध्यत इति चेत् । उच्यते । \textbf{वादन्यायोक्तेन} न्यायेन बुद्धिव्यपदेशाभावादेरसद्व्यवहारानिमित्तत्वेन निमित्तान्तराभावे दृश्यानुपलम्भ एवान्यनिरपेक्षो निमित्तम् । यच्च यन्मात्रनिमित्तं तत्तस्मिन् सति भवति । यथा बीजादिसामग्रीमात्रनिमित्तोऽङ्कुरः सति तस्मिन् भवति । दृश्यानुपलम्भमात्रनिमित्तश्चासद्व्यवहार इत्यनुमानं तत्र प्रबृत्तं साध्यसाधनयोर्व्याप्तिमवस्यतीति ।
	\pend
      

	  \pstart \textbf{अनेन वाक्येन} यदुपलब्धिलक्षणप्राप्तमित्यादिनाऽ\textbf{भिधीयमाना} प्रकाश्यमाना । तत \textbf{एव} प्राक्प्रवृत्तादनन्तरोक्तादनुमानात्मनः प्रमाणाद् \textbf{व्याप्तिर्ज्ञातव्या} ।
	\pend
      

	  \pstart एतदुक्तं भवति तत्प्रमाणसिद्धैव व्याप्तिरनेन वाक्येन स्मर्यत इति ॥
	\pend
      

	  \pstart \textbf{स्वभावेत्यादि}ना स्वभावहेतोः साधर्म्यवत्प्रयोगं विवरितुमुपक्रमते । \textbf{सर्व}शब्दस्याऽशेषताऽर्थः । तयैव प्रतिपादितया साधनस्य साध्यायत्तताख्यो यो नियमः स प्रतिपादितो  \leavevmode\marginnote{\textenglish{157/dm}} सत्त्वमनूद्य तत् सर्वमनित्यमित्यनित्यत्वं विधीयते । सर्वग्रहणं च नियमार्थम् । सर्वमनित्यम् । न किञ्चिन्नानित्यम् । यत् सत् तदनित्यमेव । अनित्यत्वादन्यत्र नित्यत्वे सत्त्वं नास्तीत्येवं सत्त्वमनित्यत्वे साध्ये नियतं ख्यापितं भवति । तथा च सति व्याप्तिप्रदर्शनवाक्यमिदम् । यथा घटादिरिति \footnote{व्याप्तिसाधनस्य \cite{dp-msA} \cite{dp-msB} \cite{dp-msC} \cite{dp-msD} \cite{dp-edP} \cite{dp-edH} \cite{dp-edE} \cite{dp-edN} नित्यक्रमयौगपद्याभ्याम् अर्थक्रियाविरोधादिति विपर्ययबाधकं प्रमाणम्--\cite{dp-msD-n}}व्याप्तिसाधकस्य प्रमाणस्य विषयकथनमेतत् । शुद्धस्येति निर्विशेषणस्य स्वभावस्य \footnote{प्रयोगस्य विशेषणं दर्श० \cite{dp-msB}}प्रयोगः ।
	\pend
      

	  \pstart सविशेषणं दर्शयितुमाह--
	\pend
      “

	  \pstart यदुत्पत्तिमत् तदनित्यमिति स्वभावभूतधर्मभेदेन स्वभावस्य प्रयोगः ॥ १० ॥
	\pend
      ”“

	  \pstart यदुत्पत्तिमदिति ।\footnote{यदुत्पत्तिः \cite{dp-msC} यदुत्पत्तिमदिति उत्पत्तिमत्त्वमनू० \cite{dp-msA}} उत्पतिः स्वरूपलाभो\footnote{लाभःस यस्यास्ति--\cite{dp-msB}} यस्यास्ति तद् उत्पत्तिमत् । उत्पत्ति मित्त्वमनूद्य तदनित्यमित्यनित्यत्वविधिः\footnote{विधेः \cite{dp-msA} \cite{dp-edP} \cite{dp-edH}} । तथा च सत्युत्पत्तिमत्त्वमनित्यत्वे नियतमाख्यातम् ।
	\pend
       

	  \pstart स्वभावं\footnote{अग्रे स्वयं दुर्वेकेण “स्वभावभूतः स्वभावात्मकः” इत्यादिना “स्वभावभूतः” इत्येव धर्मोत्तरसंमतः पाठ इति गृहीतस्तथापि अत्र तेनैव “स्वभावं भूतः” इत्येवंरूपेण गृहीतोऽस्ति इति व्याख्यानुरोधाद् भाति-सं० । स्वभावभूतः-\cite{dp-msA} \cite{dp-msB} \cite{dp-msC} \cite{dp-msD} \cite{dp-edP} \cite{dp-edH} \cite{dp-edE} \cite{dp-edN}} भूतः \footnote{स्वभावात्मको \cite{dp-msA} \cite{dp-msB} \cite{dp-msC} \cite{dp-msD} \cite{dp-edP} \cite{dp-edH} \cite{dp-edE} \cite{dp-edN}}तदात्मको धर्मः । तस्य भेदेन । भेदं हेतूकृत्य प्रयोगः । \footnote{ननु स्वभावभूतस्य कथं भद इत्याह--\cite{dp-msD-n}}अनुत्पन्नेभ्यो हि व्यावृत्तिमाश्रित्योत्पन्नो भाव\footnote{भाव उच्यते \cite{dp-msA} \cite{dp-msB} \cite{dp-msD} \cite{dp-edP} \cite{dp-edH} \cite{dp-edE} \cite{dp-edN}} इत्युच्यते । सैव व्यावृत्तिर्यदा व्यावृत्त्यन्तरनिरपेक्षा वक्तुमिष्यते तदा व्यतिरेकिणीव निर्दिश्यते--भावस्य उत्पत्तिरिति । तया च व्यतिरिक्तयेवोत्पत्त्या विशिष्टं\footnote{विशिष्टं च वस्तु--\cite{dp-msC}} वस्तु उत्पत्तिमदुक्तम् । तेन स्वभावभूतेन धर्मेण कल्पितभेदेन
	\pend
      ”

	  \pstart भवतीति \textbf{नियमख्यापनार्थं सर्वग्रहणं} भवति । अन्यथा निःशेषत्वानुपपत्तेरिति । \textbf{सर्वमि}त्याद्यस्यैव स्पस्टीकरणम् । \textbf{तथा च सति} नियतत्वनियमविषयत्वख्यापनप्रकारे सति । \textbf{इदं वाक्यं} यत्सत्तदनित्यमित्यात्म\textbf{कम् । व्याप्तिसाधकस्य प्रमाणस्येति} यस्य क्रमाक्रमाऽयोगो न तस्य क्वचित्सामर्थ्यं यथाऽऽकाशकुशेशयस्य । अस्ति चाक्षणिके स इति व्यापकानुपलम्भसम्भवस्यानुमानस्येति द्रष्टव्यम् । एतच्च बहुवाच्यमन्यत्र विपञ्चितं नेहाप्रकृतत्वात्प्रतन्यते ।
	\pend
      

	  \pstart कथं पुनरुत्पत्तिर्भावस्य विशेषणमित्याह--\textbf{स्वभावमि}ति । \textbf{स्वभावं भूतः} प्राप्त इति कर्त्तरि निष्ठा “द्वितीया”\href{http://http://sarit.indology.info/?cref=Pā.2.1.24}{पाणिनि २. १. २४.}इति योगविभागा\leavevmode\marginnote{\textenglish{59b/ms}}त्समासः । अस्यैव स्पष्टीकरणं \textbf{तदात्मक} इति । यदि स्वभावः कथं विशेषणम्, भेदेन तस्य दर्शनाद् इत्याह—  \leavevmode\marginnote{\textenglish{158/dm}} “
	  
	विशिष्टः स्वभावः प्रयुक्तो द्रष्टव्यः ॥ “
	  
	यत् कृतकं तदनित्यमित्युपाधिभेदेन ॥ ११ ॥” 
	  
	यत् कृतकमिति कृतकत्वमनूद्य अनित्यत्वं विधीयत इति \footnote{अनियतत्वे--\cite{dp-msA}}अनित्यत्वे नियतं कृतकत्वमुक्तम् । अतो व्याप्तिरनित्यत्वेन कृतकत्वस्य दर्शिता । उपाधिभेदेन स्वभावस्य प्रयोग इति संबन्धः । उपाधिर्विशेषणम् । तस्य भेदेन भिन्नेनोपाधिना विशिष्टः स्वभावः प्रयुक्त इत्यर्थः । 
	  
	\footnote{परार्थानुमाने--\cite{dp-msD-n}}इह कदाचिच्छुद्ध एवार्थ उच्यते, कदाचिदव्यतिरिक्तेन विशेषणेन विशिष्टः कदाचिद्व्यतिरिक्तेन । देवदत्त इति शुद्धः, लम्बकर्ण इत्यभिन्नकर्णद्वयविशिष्टः, चित्रगुरिति व्यतिरिक्तचित्रगवीविशिष्टः । तद्वत् सत्त्वं शुद्धम्, उत्पतिमत्त्वमव्यतिरिक्तविशेषणम्, कृतकत्वं व्यतिरिक्तविशेषणम् ॥” तस्येति । \textbf{भेदेन} विशेष्याव्यतिरिक्ततया विशेषकत्वलक्षणेन विकल्पसन्दर्शितेन । स्वभावभूतः स्वभावात्मको धर्म इति च परमार्थाभिप्रायेणोक्तम् । \textbf{भेदेने}तीयं तृतीया हेताविति दर्शयन्नाह—\textbf{भेदमि}ति । व्यवहारसिद्धं \textbf{भेद}मुत्पत्तुः सकाशादन्यत्वं \textbf{हेतूकृत्य} निबन्धनीकृत्य \textbf{प्रयोगः} सवि\textbf{शेषणस्य} स्वभावहेतोरिति प्रकरणात् ।
	\pend
      

	  \pstart कल्पनयाऽपि कथं भेदो येनोत्पत्त्या विशिष्टमुत्पत्तिमदुच्यत इत्याह--\textbf{अनुत्पन्नेभ्य} इति । हिर्यस्मात् । \textbf{अनुत्पन्नेभ्य} आकाशादिभ्यो \textbf{व्यावृत्तिं} व्यवच्छेद\textbf{माश्रित्य} परिकल्प्य । यदि व्यावृत्त्याश्रयेणोत्पन्नो भाव उच्यते तर्हि कथमुत्पत्तिरस्येति प्रयोग इत्याह--\textbf{सैवेति । व्यावृत्त्यन्तरं} महत्त्वादि \textbf{तन्निरपेक्षा वक्तुमिष्यते} यदा \textbf{तदा । तेन} परमार्थः स्वभावभूतेनोत्पत्त्याख्येन \textbf{धर्मेण कल्पितः} समारोपितो \textbf{भेदो}ऽर्थान्तरत्वं यस्य ये\footnote{ते}न अव्यतिरिक्तेन विशेषणेन विशिष्टस्य स्वभावहेतोः प्रयोग इत्यर्थः ।
	\pend
      

	  \pstart पूर्वमव्यतिरिक्तविशेषणविशिष्टस्य स्वभावस्य प्रयोगः । अधुना तु भिन्नविशेषणविशिष्टस्येति भेदस्तदाह--\textbf{भिन्नेने}ति । यद्वा \textbf{भिन्नेन} पूर्वस्मादन्यादृशेन सङ्केतवशादन्तर्भावितेन, न तु विद्यमानस्ववाचकेन । अत एवायमन्यतो भिद्यते प्रयोगः ।
	\pend
      

	  \pstart \textbf{इहेति} परप्रतिपादनार्थे शब्दप्रयोगे । \textbf{शुद्धो} निर्विशेषणः । अथ किमेकोऽर्थः शुद्धः कदाचिदव्यतिरिक्तोपाधिना विशिष्टः; कदाचिद् व्यतिरिक्तविशेषणविशिष्टो दृष्टः शिष्टैः प्रयुज्यमानो येनैवमुच्यमानं परभागं पुष्णातीत्याह--\textbf{देवदत्त} इति । वक्ष्यमाणतद्वत्शब्दात् यद्वत्शब्दोऽत्र द्रष्टव्यः । \textbf{चि}त्रा चासौ गौश्चेति “गोरतद्धितलुकि” \href{http://http://sarit.indology.info/?cref=Pā.5.4.92}{पाणिनि ५. ४. ६२} इति टच्, टित्वाङ्ङीप् तया विशिष्टः । यथाक्रममेव दृष्टान्तदार्ष्टान्तिकयोजना कार्या ।
	\pend
      

	  \pstart \textbf{चित्रगु}शब्देन कृतकशब्दस्य साम्यं नास्तीति मन्वानः पर आह--\textbf{ननु} चेति । \textbf{कारणा नां व्यापारो} नियतप्राग्भाव \unclear{स्त}दतिरेकिणो व्यापारस्याभावात् । ननु च कृतकशब्दे न विशेषणवाचिशब्दोऽस्तीति यदुक्तं तत्तदवदस्थमेवेत्याह--\textbf{यद्यपीति । अन्तर्भावितं} प्रकाशितम् । कथं  \leavevmode\marginnote{\textenglish{159/dm}} “
	  
	ननु च चित्रगुशब्दे व्यतिरिक्तस्य विशेषणस्य वाचकश्चित्रशब्दो गोशब्दश्चास्ति । कृतकशब्दे तु निर्विशेषणवाचिनः शब्दस्य प्रयोगोस्तीत्याशङ्क्याह-- “
	  
	अपेक्षितपरव्यापारो हि भावः स्वभावनिष्पत्तौ कृतक इति ॥ १२ ॥” 
	  
	अपेक्षितेति । परेषां कारणानां व्यापारः स्वभावस्य निष्पत्तौ--निष्पत्त्यर्थमपेक्षितः परव्यापारो यन स तथोक्तः । हीति यस्मादर्थे । यस्मादपेक्षितपरव्यापारः कृतक उच्यते तस्माद् व्यतिरिक्तेन विशेषणेन विशिष्टः स्वभाव उच्यते । यद्यपि व्यतिरिक्तं विशेषणपदं न\footnote{न च प्रयु० \cite{dp-msB}} प्रयुक्तं तथापि कृतकशब्देनैव व्यतिरिक्तं \footnote{विशेषणमन्त० \cite{dp-msA} \cite{dp-edP} \cite{dp-edH} \cite{dp-edN}}विशेषणपदमन्तर्भावितम् । अत एव संज्ञाप्रकारोऽयं कृतकशब्दो यस्मात् संज्ञायामयं कन्प्रत्ययो विहितः । यत्र च विशेषणमन्तर्भाव्यते तत्र विशेषणपदं न प्रयुज्यते । 
	  
	क्वचित्तु\footnote{क्वचित् प्र० \cite{dp-msA} \cite{dp-msB} \cite{dp-edP} \cite{dp-edH} \cite{dp-edE} \cite{dp-edN}} प्रतीयमानं विशेषणं यथा कृत इत्युक्ते हतुभिरित्यतत् प्रतायते । तत्र\footnote{तत्र हेतु \cite{dp-msB}} च हेतुशब्दः प्रयुज्यते, कदाचिन्न वा प्रयुज्यते ॥ “
	  
	एवं प्रत्ययभेदभेदित्वादयोऽपि\footnote{०दयो द्र० \cite{dp-msB} \cite{dp-edP} \cite{dp-edH} \cite{dp-edE} \cite{dp-edN}} द्रष्टव्याः ॥ १३ ॥” 
	  
	प्रयुज्यमानस्व\footnote{स्वशब्दो विशेषणशब्दः--\cite{dp-msD-n}} शब्दश्च यथा प्रत्ययभेदभेदिशब्दे\footnote{प्रत्ययभेदशब्दे \cite{dp-msB}} प्रत्ययभेदशब्दः\footnote{प्रत्ययभेदः \cite{dp-msA} \cite{dp-msB} \cite{dp-edP} \cite{dp-edH} \cite{dp-edE} \cite{dp-edN}} । यथा च\footnote{यथा कृत० \cite{dp-msC}} कृतकशब्दो भिन्नविशेषणस्वभावाभिधायी एवं प्रत्ययभेदभेदित्वमादिर्येषां प्रयत्नान्तरीयकत्वादीनां तेऽपि स्वभावहेतोः प्रयोगा भिन्नविशेषणस्वभावाभिधायिनो द्रष्टव्याः । 
	  
	प्रत्ययानां कारणानां भेदो विशेषस्तेन प्रत्ययकालाभेदेनं भेत्तुं शीलं यस्यस प्रत्ययभेदभेदी शब्दस्तस्य भावः प्रत्ययभेदभेदित्वम् । ततः प्रत्ययभेदभेदित्वाच्छब्दस्य कृतकत्वं साध्यते । प्रयत्नानन्तरीयकत्वादनित्यत्वम्\footnote{त्वं साध्यते । \cite{dp-msA} \cite{dp-edP} \cite{dp-edH} \cite{dp-edE} \cite{dp-edN}} । तत्र प्रत्ययभेदशब्दो व्यतिरिक्तविशेषणाभिधायी प्रत्ययभेदभेदिशब्दे प्रयुक्तः । प्रयत्नानन्तरीयकशब्दे च प्रयत्नशब्दः ।” \textbf{पुनः} कृतकशब्देनान्तर्भावितमित्याह--\textbf{यस्मादिति । संज्ञायां} नाम्नि कन्प्रत्ययो विहितस्तस्मादन्तर्भावितमिति । \textbf{अत एव} संज्ञाया कनो विधानादेवायं \textbf{कृतकशब्दः संज्ञाप्रकारः} संज्ञाविशेषः संज्ञाशब्द इति यावात् ।
	\pend
      

	  \pstart अन्तर्भावेऽपि कथं विशेषणपदाप्रयोग इत्याह--\textbf{यत्रेति । चो} यस्मादर्थे । अथावसितस्याप्यस्ति प्रयोगो यथा कृतक इत्युक्ते हेतुनेति प्रतीतावपि हेतुशब्दप्रयोग इत्याह--\textbf{क्वचिदि}ति । \textbf{तुः} पूर्वस्माद् वैधर्म्यं \footnote{र्म्ये} । \textbf{प्रतीयमानं} स्वत उत्पादायोगात् सामर्थ्यादवसीयमानम् ।  \leavevmode\marginnote{\textenglish{160/dm}} “
	  
	तदेवं त्रिविधः स्वभावहेतुप्रयोगो\footnote{०हेतुयोगो \cite{dp-msB}} दर्शितः शुद्धोऽव्यतिरिक्तविशेषणो व्यतिरिक्तविशेषणश्च । 
	  
	\footnote{एतदर्थम्--\cite{dp-msB} \cite{dp-msD}}एवमर्थं चैतदाख्यातम्--वाचकभेदान्मा भूत् कस्यचित्स्वभावहेतावपि प्रयुक्ते व्यामोह इति ॥ “
	  
	\footnote{सन्नोत्प० \cite{dp-edE}}सन्नुत्पत्तिमान् कृतको वा शब्द इति पक्षधर्मोपदर्शनम्\footnote{अस्मिन् सूत्रे न किंचित्केनचिद्व्याख्यातम्--सं०} ॥ १४ ॥” 
	  
	अथ किमेते स्वभावहेतवः सिद्धसम्बन्धे स्वभावे साध्ये प्रयोक्तव्या आहोस्विदसिद्धसम्बन्ध इत्याशङ्क्य सिद्धसम्बन्धे प्रयोक्तव्या इति दर्शयितुमाह-- “
	  
	सर्व एते साधनधर्मा यथास्वं प्रमाणौः सिद्धसाधनधर्ममात्रानुवन्ध एव साध्यधर्मेऽवगन्तव्याः ॥ १५ ॥” 
	  
	सर्व एत इति । गमकत्वात् साधनानि, पराश्रितत्वाच्च धर्माः, साधनधर्मा एव” अयमस्याशयः--न कृतकशब्देन प्रतिपादितस्यार्थस्य अन्यथानुपपत्त्या हेतुव्यापारोऽत्र प्रतीयते येनात्रापि विशेषण \leavevmode\marginnote{\textenglish{60a/ms}} पदस्य प्रयोगो वक्तुरिच्छातः स्याद् वा न वा किन्तु स्वोत्पत्तावपेक्षितपरव्यापारस्यैवार्थस्येदं नामेति कुतोऽनयोः साम्यमिति ।
	\pend
      

	  \pstart एवं तावत्कश्चिद् विशेषणभूतोऽर्थोऽर्थात्प्रतीयमानः स्वशब्देनोच्यते न वा वक्तुरिच्छावशादिति प्रतिपाद्य कश्चित्पुनर्विशेषणभूतोऽर्थः प्रतीयमानोऽप्यवश्यं स्वशब्देनोपादेय इति दर्शयितुमाह--\textbf{प्रयुज्यमाने}ति । \textbf{प्रयुज्यमान} उपादीयमानः । \textbf{स्वशब्दः} स्ववाचको यस्य विशेषणरूपस्यार्थस्य स तथोक्तः । कोऽसावीदृश इत्याह--\textbf{यथे}ति । \textbf{प्रत्ययभेदः} प्रत्ययभेदलक्षणोऽर्थो विशेषणात्माऽवश्यं स्ववाचकेन प्रत्ययभेदशब्देनाभिधीयत इति प्रकरणात् ।
	\pend
      

	  \pstart \textbf{भेत्तुं} भिदां गन्तुं । कस्मिन् साध्ये साधनमिदमित्याह--\textbf{शब्दस्ये}ति । एतच्च \textbf{मीमांसका}दीनां प्रति द्रष्टव्यम् ।
	\pend
      

	  \pstart \textbf{तदेवमि}त्यादिनोपसंहारः । स्वभावहेतोः साधर्म्यवत्प्रयोगमात्रे दर्शयितव्ये किमनेकस्य स्वभावहेतोः प्रयोगो दर्शित इत्याशङ्क्य फलमस्योपदर्शयन्नाह--\textbf{एवमर्थं चैतदि}ति । \textbf{एवं} वक्ष्यमाणकोऽर्थः प्रयोजनं यस्येति विग्रहः कार्यः । \textbf{चो}ऽवधारणे हेतौ वा । \textbf{एतदि}ति त्रैविध्यम् । \textbf{एवमर्थं चैत}त् \textbf{हेतुजातमि}ति क्वचित्पाठस्तत्र च \textbf{जातं} वृन्दं द्रष्टव्यम् । तमेवार्थं \textbf{वाचके}त्यादिना \textbf{दर्शयति । वाचके व्यामोहो} भ्रमस्तस्मात् । कस्यचित्प्रतिपत्तुः स्वभावहेतावपि प्रयुक्ते व्यामोहो “नायं स्वभावः” इति विपर्ययज्ञानम् ।
	\pend
      

	  \pstart एतदुक्तं भवति--यदि त्रयाणामन्यतम उपादीयेत तदा कदाचिदन्येनान्यथा प्रयुक्ते स्वभावहेतौ शास्त्रोक्तस्वभावहेतुवाचक\add{भिन्न}त्वात् “नायं स्वभाववाचकः” इति वाचके  \leavevmode\marginnote{\textenglish{161/dm}} “
	  
	साधनधर्ममात्रम् । मात्रशब्देनाधिकस्यापेक्षणीयस्य निरासः । तस्यानुबन्धोऽनुगमनमन्वयः । सिद्धः साधनधर्ममात्रानुबन्धो यस्य स तथोक्तः । केन सिद्ध इत्याह--यथास्वं प्रमाणैरिति\footnote{प्रमाणैर्यस्य साधनधर्मस्य यदात्मीयं \cite{dp-msC} \cite{dp-msD} प्रमाणैर्यस्य यदात्मीयं--\cite{dp-msB}} । यस्य साध्यधर्मस्य यदात्मीयं प्रमाणं तेनैव प्रमाणेन सिद्ध इत्यर्थः । स्वभावहेतूनां च बहुभेदत्वात् संबन्धसाधनान्यपि प्रमाणानि बहूनीति प्रमाणैरिति बहुवचननिर्देशः । गमयितव्यत्वात् साध्यः, पराश्रितत्वाच्च धर्मः साध्यधर्मः । 
	  
	तदयं परमार्थः--न हेतुः प्रदीपवद् योग्यतया गमकोऽपि तु नान्तरीयकतया विनिश्चितः\footnote{अथ नान्तरीयत्वानिश्चयेऽपि लिङ्गस्य परोक्षार्थप्रतिपादकत्वं स्यादित्याह--\cite{dp-msD-n}} । साध्याविनाभावित्वनिश्चयनमेव\footnote{०मेव हेतोः \cite{dp-msB}} हि हेतोः साध्यप्रतिपादनव्यापारो नान्यः कश्चित् ।” व्यामुह्य “नायं स्वभावः” इति साधनेऽपि व्यामुह्येत । स व्यामोहो मा भूदित्येतदर्थं त्रिविधः स्वभावहेतुरुक्तः ।
	\pend
      

	  \pstart अथेति सम्बोधने । \textbf{सिद्धो} निश्चितः \textbf{सम्बन्ध}स्तादात्म्यलक्षणो यस्य तस्मिन् । \textbf{यस्य साध्यधर्मस्य यदात्मीयमिति} येन प्रमाणेन यस्य साध्यस्य साधनव्याप\add{क}त्वं निश्चीयते । तदेव तस्येत्यभिप्रायेणोक्तम् । व्यक्तिभेदविवक्षया \textbf{बहूनी}त्युक्तम् ।
	\pend
      

	  \pstart साध्यसाधनयोः सम्बन्धो वास्तवोऽस्तु । किं तन्निश्चयेनेत्याह--\textbf{तदयमिति} । यत एवं तत्तस्मा\textbf{दयम}भिधास्यमानस्तात्पर्यार्थः \textbf{सर्व} इत्यादेर्वाक्यस्य । \textbf{प्रदीपो} वैधर्म्यदृष्टान्तः । \textbf{योग्यतया} तथाशक्यतया ।
	\pend
      

	  \pstart ननु परोक्षार्थप्रतिपादनव्यापारसमावेशाद् हेतुर्गमकः तत्किं साध्यनान्तरीयकत्वनिश्चयेनेत्याह--साध्येति । हिर्यस्मादर्थे । \textbf{साध्याविनाभावित्वनिश्चयनम}त्रापीदमेतत्स्वभावमिति निश्चयः । तदेव \textbf{हेतोः प्रतिपादनव्यापारः} परोक्षार्थप्रतिपादनलक्षणो व्यापारः परोक्षार्थप्रतिपादकत्वमित्यर्थः । यद्येवमनिश्चितसम्बन्धेऽपि साध्यधर्मे साधनधर्मस्तन्नान्तरीयकतया निश्चेतुं शक्य इति किं सिद्धसम्बन्धेन साध्येनानुसृतेनेत्याह--\textbf{प्रथममि}ति । \textbf{प्रथमं} हेतुप्रयोगात्प्राक्, \textbf{बाधकेन} साध्यविपर्ययो हेतो\leavevmode\marginnote{\textenglish{60b/ms}}\footnote{नैतत्पत्रं प्रतिबिम्बितम्--सं०} \add{... ... ... ... ... ... ... ... ... ... ... ...} ...\leavevmode\marginnote{\textenglish{61a/ms}} साध्यसाधनभाव इति चेत् । यद्दर्शनद्वारायातावेतौ कृतकत्वानित्यत्वविकल्पौ व्यावृत्तिनिष्ठौ परमार्थतस्तस्य तादात्म्यादित्युक्तप्रायमित्यदोषः । तादात्म्यावसायः कुत इति चेद् विपर्यये वाधकप्रमाणवशात् । तत एव तर्हि साध्यं सिद्धमिति किं साध्यत इति चेत् । न । ततो धर्म्यनवच्छेदेन प्लवमानाकारायाः प्रतीतेरप्रवृत्त्यङ्गस्योदयात् । यत्पुनरियं घर्मकालविशे  \leavevmode\marginnote{\textenglish{162/dm}} “
	  
	\footnote{प्रागनुमानात्--\cite{dp-msD-n}}प्रथमं बाधकेन प्रमाणेन साध्यप्रतिबन्धो निश्चेतव्यो हेतोः । पुनरनुमानकाले\footnote{०कालेन साधनं--\cite{dp-msA} \cite{dp-msB} \cite{dp-edP} \cite{dp-edH} \cite{dp-edE}} साधनं साध्य\footnote{साध्यानन्तरीयकं \cite{dp-msA} \cite{dp-msB} \cite{dp-edP} \cite{dp-edH} \cite{dp-edE}} नान्तरीयकं सामान्येन स्मर्तव्यम् । कृतकत्वं नामानित्य\footnote{०त्यत्वस्व० \cite{dp-msA} \cite{dp-msB} \cite{dp-edP} \cite{dp-edH} \cite{dp-edE} \cite{dp-edN}} स्वभावमिति सामान्येन स्मृतमर्थ\footnote{स्मृतमर्थाय--\cite{dp-msB}} पुनर्विशेषे योजयति--इदमपि कृतकत्वं शब्दे वर्तमानमनित्य\footnote{०त्यत्वस्व० \cite{dp-msC}} स्वभावमेवेति । 
	  
	तत्र सामान्यस्मरणं लिङ्गज्ञानम्\footnote{लिङ्गनिश्चायकं ज्ञानम्--\cite{dp-msD-n}} । विशिष्टस्य तु शब्दगतकृतकत्वस्या\footnote{कृतकस्या० \cite{dp-edH}} ऽनित्यत्वस्वभावस्य स्मरणमनुमानज्ञानम्\footnote{मानं ज्ञानं--\cite{dp-msC}} । तथा च सत्यविनाभावित्वज्ञानमेव परोक्षार्थप्रतिपादकत्वं नाम । तेन निश्चिततन्मात्रानुबन्धे साध्यधर्मे स्वभावहेतवः प्रयोक्तव्या नान्यत्रेत्युक्तम् ॥ 
	  
	यद्येवं सम्बन्धो निश्चेतव्यः साध्यस्य साधनेन सह । साधनधर्ममात्रानुबन्धस्तु साध्यस्य कस्मान्निश्चितो मृग्यत इत्याह-- “
	  
	\footnote{तस्यैव--इति नास्ति--\cite{dp-edP} \cite{dp-edH}}तस्यैव तत्स्वभावत्वात् ॥ १६ ॥” 
	  
	तस्यैवेति सिद्धसाधनधर्ममात्रानुबन्धस्य । तत्स्वभावत्वादिति साधनधर्मरवभावत्वात् । यो हि साध्यधर्मः साधनधर्ममात्रानुबन्धवात् स एव तस्य साधनधर्मस्य स्वभावो नान्यः ॥ 
	  
	भवतु ईदृश एव स्वभावः । स्वभाव एव तु साध्ये कस्माद्धेतुप्रयोगः ? “
	  
	स्वभावस्य च\footnote{०स्य हेतु० \cite{dp-msC}} हेतुत्वात् ॥ १७ ॥” 
	  
	स्वभावस्य च\footnote{०स्य हेतु० \cite{dp-msC}} हेतुत्वात् । स्वभाव एव\footnote{स्वभाव एव हेतु--\cite{dp-msB} स्वभाव इह \cite{dp-msA} \cite{dp-edP} \cite{dp-edH} \cite{dp-edN}} इह हेतुः प्रक्रान्तः । तस्मात् स एव साध्यः कर्तव्यः यः साधनस्य स्वभावः स्यात् । साधनधर्ममात्रानुबन्धवांश्च\footnote{०न्धश्च स्व० \cite{dp-msA} \cite{dp-edP} \cite{dp-edE} \cite{dp-edH} \cite{dp-edN}} स्वभावो नान्यः ॥ 
	  
	यदि साध्यधर्मः साधनस्य स्वभावः\footnote{स्वभावः प्रति० \cite{dp-msA} \cite{dp-msB} \cite{dp-msD} \cite{dp-edP} \cite{dp-edE} \cite{dp-edH} \cite{dp-edN}} स्यात् प्रतिज्ञार्थैकदेशस्तर्हि हेतुः स्यादित्याह-- “
	  
	वस्तुतस्तयोस्तादात्म्यम्\footnote{०त्म्यात् \cite{dp-msB} \cite{dp-msD} \cite{dp-edP} \cite{dp-edH} \cite{dp-edE} \cite{dp-edN}} ॥ १८ ॥” 
	  
	वस्तुत इति । वस्तुतः परमार्थतः साध्यसाधनयोस्तादात्म्यम् । समारोपितस्तु साध्यसाधनभेदः\footnote{०साधनयोर्भेदः \cite{dp-msA} \cite{dp-edP} \cite{dp-edH} \cite{dp-edE} \cite{dp-edN}} । साध्यसाधनभावो हि निश्चयारूढे रूपे । निश्चयारूढं च रूपं समारोपितेन” षानवच्छेदेन तदात्मताप्रतीतिः प्रवृत्त्यङ्गमियमस्मादेव लिङ्गादिति किमवद्यम् ? एवं सत्त्वहेतावपि द्रष्टव्यम् । एतच्चोक्तमपि स्वार्थानुमानेऽधिकाभिधानार्थं पुनरुक्तमिहेति द्रष्टव्यम् ॥
	\pend
      \leavevmode\marginnote{\textenglish{163/dm}}“

	  \pstart भेदेनेतरेतरव्यावृत्तिकृतेन भिन्नमिति अन्यत् साधनम्, अन्यत् साध्यम् । दूराद्धि शाखादिमानर्थो बृक्ष इति निश्चीयते न शिंशपेति । अथ च स एव वृक्षः सैव शिंशपा । तस्मादभिन्नमपि वस्तु निश्चयो भिन्नमादर्शयति व्यावृत्तिभेदेन । तस्मान्निश्चयारूढरूपापेक्षया अन्यत् साधनम अन्यत् साध्यम् । अतो न प्रतिज्ञार्थैकदेशो हेतुः । वास्तवं च तादात्म्यमिति ॥
	\pend
       

	  \pstart कस्मात् पुनः साधनधर्ममात्रानुबन्ध्येव\footnote{०बन्धे च साध्यः--\cite{dp-msB}} साध्यः स्वभावो नान्य इत्याह--
	\pend
       “

	  \pstart तन्निष्पत्तावनिष्पन्नस्य तत्स्वभावत्वाभावात्\footnote{तत्स्वभावात् \cite{dp-msC}} ॥ १९ ॥
	\pend
      ” 

	  \pstart तन्निष्पत्ताविति । यो हि यन्नानुबध्नाति स\footnote{स तन्निष्पत्तावनिष्पन्नस्य साधन A} तन्निष्पत्तावनिष्पन्नः । तस्य तन्निष्पत्तावनिष्पन्नस्य साधनस्वभावत्वमयुक्तम् । यतो निष्पत्त्यनिष्पत्ती भावाभावरूपे । भावाभावौ च परस्परपरिहारेण स्थितौ । यदि च पूर्वनिष्पन्नस्य, अनिष्पन्नस्य चैक्यं भवेदेकस्यैवार्थस्य भावाभावौ स्यातां युगपत् । न च विरुद्धयोर्भावाभावयोरैक्यं युज्यते, विरुद्धधर्मसंसर्गात्मकत्वादेकत्वाभावस्य ।\footnote{अयमेव भेदोभावनां विरुद्धधर्माध्यासः, कारणभेदश्च--\cite{dp-msD-n}}
	\pend
       

	  \pstart किञ्च पश्चादुत्पद्यमानं पूर्वनिष्पन्नाद्भिन्नहेतुकम् । हेतुभेदपूर्वकश्च कार्यभेदः । ततो निष्पन्नानिष्पन्नयोर्विरुद्धधर्मसंसर्गात्मको भेदो भेदहेतुश्च कारणभेद इति कुत एकत्वम् ? तस्मात् साधनधर्ममात्रानुबन्ध्येव साध्यः स्वभावो नान्यः ॥
	\pend
       

	  \pstart मा भूत् पश्चान्निष्पन्नः पूर्वजस्य स्वभावः । साध्यस्तु कस्मान्न भवतीत्याह--
	\pend
       “

	  \pstart व्यभिचारसंभवाच्च ॥ २० ॥
	\pend
      ” 

	  \pstart \footnote{“व्यभिचारेत्यादि”--नास्ति--\cite{dp-msA} \cite{dp-msB} \cite{dp-msD} \cite{dp-edP} \cite{dp-edH} \cite{dp-edE} \cite{dp-edN}}व्यभिचारेत्यादि । पूर्वजेन पश्चान्निष्पन्नस्य व्यभिचारः परित्यागो यस्तस्य संभवाच्च
	\pend
      ”

	  \pstart ननु किमस्य सम्भवोऽस्ति यदुतैकात्मन एवैका व्यावृत्तिर्निश्चीयते नेतरेत्याह--दूरादिति । हिर्यस्मात् । \textbf{अथ चे}ति निपातसमुदायः प्रतिपिपादयिषितपरमार्थद्योतकः । \textbf{वृक्षः शिंशपे}ति चोपलक्षणमेतत् ।
	\pend
      

	  \pstart \textbf{तस्मादि}त्यादिनोपसंहारः । \textbf{वास्तवं} वस्तुस्वरूपादागतमसमारोपितमित्यर्थः । तुशब्दार्थश्चकारः ॥
	\pend
      

	  \pstart \textbf{यः} साध्यधर्मो \textbf{यं} साधनधर्मं \textbf{नानुबध्नाति,} नानुगच्छति, तस्मिन् सति नियमेन नोपति ठत इति यावत् । कस्मात् तत्स्वभावत्वमयुक्तमित्याह--\textbf{यत} इति । भवतां भावाभावरूपे तथाऽपि किं तयोस्तत्स्वभावत्वमित्याह--\textbf{भावे}ति । चो यस्मादर्थे । \textbf{यदिच्छ}\footnote{यदि चे}ति चशब्दो वक्तव्यान्तरसमुच्चये ।
	\pend
      \leavevmode\marginnote{\textenglish{164/dm}}“

	  \pstart न पूर्वनिष्पन्नस्य पश्चान्निष्पन्नः साध्यः । तस्मात् साधनधर्ममात्रानुबन्ध्येव\footnote{०बन्ध्येव यः स्वभावः \cite{dp-msB} \cite{dp-msC} \cite{dp-msD} \cite{dp-edE}} स्वभावः । स एव च साध्यः । तथा च सिद्धसाधनधर्ममात्रानुबन्ध एव स्वभावे स्वभावहेतवः प्रयोक्तव्या इति स्थितम् ॥
	\pend
       “

	  \pstart कार्यहेतोः\footnote{कार्यहेतुप्रयोगः \cite{dp-msC} कार्यहेतोरपि प्रयोगः \cite{dp-msB} \cite{dp-msD} \cite{dp-edP} \cite{dp-edH} \cite{dp-edE} \cite{dp-edN}} प्रयोगः--यत्र धूमस्तत्राग्निः । यथा महानसादौ । अस्ति चेह धूम इति ॥ २१ ॥
	\pend
      ” 

	  \pstart कार्यहेतोः प्रयोगः । साधर्म्यवानिति प्रकरणादपेक्षणीयम् । यत्र धूम इति धूममनूद्य तत्राग्निरित्यग्ने\footnote{त्यग्निविधः--\cite{dp-msB}} र्विधिः । तथा च \footnote{व्याप्तिर्व्यापकस्य तत्र भाव एवेत्यादिकः--\cite{dp-msD-n}}नियमार्थः पूर्ववदवगन्तव्यः । तदनेन कार्यकारणभावनिमित्ता व्याप्तिर्दर्शिता ।
	\pend
       

	  \pstart व्याप्तिसाधनप्रमाणविषयं दर्शयितुमाह--यथा महानसादाविति । महानसादौ हि प्रत्यक्षानुपलम्भाभ्यां कार्यकारणभावात्माविनाभावो निश्चितः ।
	\pend
       

	  \pstart अस्ति चेहेति साध्यधर्मिणि पक्षधर्मोपसंहारः ॥
	\pend
       “

	  \pstart इहापि सिद्ध एव कार्यकारणभावे कारणे साध्ये \footnote{कार्यं हेतु० \cite{dp-msC} \cite{dp-msD}}कार्यहेतुर्वक्तव्यः ॥ २२ ॥
	\pend
      ” 

	  \pstart इहापीति । न केवलं स्वभावहेताविहापि कार्यहेतौ\footnote{कार्यहेतोः \cite{dp-msB}} । सिद्ध एवेति निश्चिते कार्यकारणत्वे । कार्यकारणभाव\footnote{कार्यकारणत्वनिश्चयो \cite{dp-msA} \cite{dp-edP} \cite{dp-edH} \cite{dp-edN} कार्यकारणनिश्चयो \cite{dp-msC}} निश्चयो ह्यवश्यं कर्तव्यः । यतो न योग्यतया हेतुर्गमकोऽपि तु नान्तरीयकत्वादित्युक्तम् ॥
	\pend
      ”

	  \pstart एवं विरुद्धधर्मसंसर्गात्मकभेदं प्रतिपाद्य तस्य हेतुं कारणभेदं प्रतिपादयितुमाह—\textbf{किञ्चे}ति निपातसमुदायो वक्तव्यान्तरसमुच्चये । पूर्वनिष्पन्नवस्तुहेतुकत्वे तदैवोत्पत्तिप्रसङ्गेन पश्चादुत्पादायोगादिति भावो \textbf{भिन्नहेतुकमिति} ब्रुवतः । भिन्नहेतुकत्वेऽपि कथं भेद इत्याह—\textbf{हेतुभेदे}ति । \textbf{चो} हेतौ । \textbf{तत} इत्यादिनोऽपसंहारः । \textbf{तस्मादि}त्यादिनाऽऽद्यस्योपसंहारः ॥
	\pend
      

	  \pstart पूर्वस्मिन् काले जातः \textbf{पूर्वजः । व्यभिचार}स्तदन्तरेणाऽपि केवलस्य स्थितिः । \textbf{चो}ऽस्वभावतापेक्षयाऽसाध्यतां समुच्चिनोति । \textbf{तस्मादि}त्यादिना साधनधर्ममात्रानुबन्धिनस्तत्स्वभावत्वं साधनस्वभावत्वञ्चोपसंहरति । चशब्दः \textbf{साध्य} इत्यस्यानन्तरं तत्स्वभावत्वापेक्षया साध्यत्वं समुच्चिनोति । \textbf{तथा च} तन्निष्पत्तावेव निष्पन्नस्य तादात्म्ये तन्मात्रानुबन्धिन एव च साध्यत्वे सति ॥
	\pend
      \leavevmode\marginnote{\textenglish{165/dm}}“

	  \pstart साधर्म्यवान्स्वभावकार्यानुपलम्भानां प्रयोगो दर्शितः । वैधर्म्यवन्तं दर्शयितुमाह--
	\pend
       “

	  \pstart वैधर्म्यवतः\footnote{वैधर्मवतः \cite{dp-edE}} प्रयोगः--यत् सदुपलब्धिलक्षणप्राप्तं तदुपलभ्यत एव । यथा नीलादिविशेषः । न चैवमिहोपलब्धिलक्षणप्राप्तस्य सत उपलब्धिर्घटस्येत्यनुपलब्धिप्रयोगः ॥ २३ ॥
	\pend
      ” 

	  \pstart वैधर्म्यवत \footnote{वैधर्म्यवतः यत् सदिति \cite{dp-msC}}इति यत् सदुपलब्धिलक्षणप्राप्तमिति यत् सत् दृश्यमित्यस्तित्वानुवादः । तदुपलभ्यत इत्युपलम्भविधिः । \footnote{तस्मात्--\cite{dp-msD-n}}तदनेन दृश्यस्य सत्त्वं दर्शनविषयत्वेन व्याप्तं कथितम्, असत्त्वनिवृत्तिश्च सत्त्वम । अनुपलम्भनिवृत्तिश्च उपलम्भः । तेन साध्यनिवृत्त्यनुवादेन साधननिवृत्तिर्विहिता । तथा च\footnote{सत्त्वम्--\cite{dp-msD-n}} साध्यनिवृत्तिः साधननिवृत्तौ नियतत्वात् साधननिवृत्त्या व्याप्ता कथिता । यदि च धर्मिणि साध्यधर्मो न भवेद् \footnote{हेतुरपि । हेत्व० \cite{dp-msB} \cite{dp-edP} \cite{dp-edH} हेतुरपि न स्यात्--\cite{dp-msD}}हेतुरपि न भवेत् । \footnote{उपलम्भः--\cite{dp-msD-n}}हेत्वभावेन \footnote{सत्त्वस्य--\cite{dp-msD-n}}साध्याभावस्य व्याप्तत्वात् । अस्ति च हेतुः\footnote{अनुपलभ्यमानः--\cite{dp-msD-n}} । अतो व्यापकस्य साधनाभा\footnote{उपलम्भः--\cite{dp-msD-n}} वस्याभावाद् व्याप्यस्य\footnote{सत्त्वस्य--\cite{dp-msD-n}} साध्याभावस्याभाव इति साध्यगति\footnote{साध्यनिश्चयो भवति \cite{dp-msA} \cite{dp-msB} \cite{dp-msC} \cite{dp-msD} \cite{dp-edP} \cite{dp-edH} \cite{dp-edE} \cite{dp-edN}} र्भवति । ततो वैधर्म्यप्रयोगे साधनाभावे साध्याभावो\footnote{नियमः \cite{dp-msB}} नियतो दर्शनीयः सर्वत्रिति न्यायः ॥
	\pend
      ”

	  \pstart \textbf{प्रकरणा}त्साधर्म्यवत्प्रयोगदर्शनप्रस्तावात् । \textbf{तथा चे}ति धूमानुवादेनाग्निविधाने सतीत्यर्थः । \textbf{नियमो}ऽव्यभिचारः, तल्लक्षणोऽर्थः प्रतिपाद्यतयाऽभिधेयः प्रयोजनं वाऽस्य--यत्र धूम इत्यादेः प्रयोगस्येति प्रस्तावात्, सोऽ\textbf{नुगन्तव्यः} प्रत्येतव्यः । \textbf{पूर्ववदनु}पलब्ध्यादिवत् ।
	\pend
      

	  \pstart एतदेव व्यनक्ति \textbf{तदि}ति यत एवं \textbf{तत्त}स्मात् । \textbf{व्याप्ति}रविनाभावः, साध्यनियतत्वं साधनस्येति यावत् । \textbf{दर्शिता} प्रदर्शिता । किंनिमित्ता सेत्याह--\textbf{कार्ये}ति ।
	\pend
      

	  \pstart अयमाशयः--व्याप्तिः खलुः प्रतिबन्धः साध्यायत्तत्वम् । तच्चार्थान्तरस्यार्थान्तरे प्रतिबद्धत्वं कार्यकारणभाववशादिति स एव \textbf{निमित्तं} तस्य, अनर्थान्तरस्य तु तादा\leavevmode\marginnote{\textenglish{61b/ms}}त्म्यम् ।
	\pend
      

	  \pstart \textbf{महानसः} सूपकारशाला । \textbf{आदि}शब्देनायस्कारकुट्यादेर्ग्रहणम् । किं तत्र व्याप्तिसाधकं प्रमाणं यदपेक्षया तस्य विषयत्वमित्याह--\textbf{प्रत्यक्षे}ति । हिर्यस्मादर्थे । प्रत्यक्षानुपलम्भानां प्रत्येकं जात्येकत्वविवक्षया \textbf{प्रत्यक्षानुपलम्भाभ्यामि}त्युक्तम् । परमार्थतस्तु त्रिभिरनुपलम्भैर्द्वाभ्यां प्रत्यक्षाभ्यामित्यवसेयम् । \textbf{कार्यकारणभावादात्मा} निश्चयारूढः स्वभावो यस्येति विग्रहः कार्योऽन्यथा युक्तिविरोधः स्ववचनविरोधश्चाऽस्य स्यात् । \textbf{अविनाभावो}ऽव्यभिचारः साध्यायत्तता साधनस्येति यावत् ।
	\pend
      

	  \pstart \textbf{पक्षधर्म}स्यो\textbf{पसंहारो} ढौकनं तत्र सत्त्वप्रदर्शनमित्यर्थः ॥
	\pend
      \leavevmode\marginnote{\textenglish{166/dm}}“

	  \pstart स्वभावहेतोर्वैधर्म्यप्रयोगमाह--
	\pend
       “

	  \pstart असत्यनित्यत्वे नास्त्येव\footnote{नास्ति सत्त्वम्--\cite{dp-msB} \cite{dp-msD} \cite{dp-edP} \cite{dp-edH} \cite{dp-edE} \cite{dp-edN}} सत्त्वमुत्पत्तिमत्त्वं कृतकत्वं वा ।\footnote{असंश्च \cite{dp-msB} \cite{dp-edP} \cite{dp-edH} \cite{dp-edN}} संश्च शब्द उत्पत्तिमान् कृतको वेति स्वभावहेतोः प्रयोगः ॥ २४ ॥
	\pend
      ” 

	  \pstart असत्यनित्यत्व इति । इहानित्यत्वस्य साध्यस्याभावो हेतोरभावे नियत\footnote{नियमः \cite{dp-msB}} उच्यते । तेन हेत्वभावेन\footnote{सत्त्वेन--\cite{dp-msD-n}} साध्याभावो व्याप्त\footnote{नित्यत्वम्--\cite{dp-msD-n}} उक्तस्त्रिष्वपि स्वभावहेतुषु । सन्नुत्पत्तिमान् कृतको वा शब्द इति त्रयाणामपि पक्षधर्मत्वप्रदर्शनम् । इह\footnote{शब्दे धर्मिणि--\cite{dp-msD-n}} च साधनाभावस्य व्यापकस्याभाव उक्तः । ततो व्याप्योऽपि साध्याभावो निवर्तत\footnote{निवृत्त इति \cite{dp-msA} \cite{dp-edP} \cite{dp-edH} \cite{dp-edE} \cite{dp-edN}} इति साध्यगतिः ॥
	\pend
       

	  \pstart कार्यहेतोर्वैध\footnote{वैधर्म्यप्रयो० \cite{dp-msA} \cite{dp-edP} \cite{dp-edH} \cite{dp-edE} \cite{dp-edN}} र्म्यवत्प्रयोगमाह--
	\pend
       “

	  \pstart असत्यग्नौ न भवत्येव धूमः । अत्र चास्ति धूम\footnote{अत्र चास्तीति कार्य--\cite{dp-msC} \cite{dp-msD}} इति कार्यहेतोः प्रयोगः ॥ २५ ॥
	\pend
      ” 

	  \pstart असत्यग्नाविति । इहापि\footnote{इहापि च \cite{dp-msC} \cite{dp-msD}} वह्न्यभावो धूमाभावेन व्याप्त उक्तः । \footnote{अत्र चास्ति धूम इति \cite{dp-msC} \cite{dp-msD}}अस्ति चात्र धूम इति व्यापकस्य धूमाभावस्याभाव उक्तः । ततो व्याप्यस्य वह्न्यभावस्याभावे साध्यगतिः ॥
	\pend
       

	  \pstart ननु च साधर्म्यवति\footnote{साधर्म्यव्यतिरेको \cite{dp-msB}} व्यतिरेको नोक्तः । वैधर्म्यवति चान्वयः । तत् कथमेतत् त्रिरूपलिङ्गाख्यानमित्याह--
	\pend
      ”

	  \pstart दृष्टान्तदार्ष्टान्तिकयोर्हेतुसद्भावासद्भावद्वारकं \textbf{वैधर्म्यं} विद्यते प्रतिपाद्यतया यस्य तं \textbf{दर्शयितुमाह वार्तिककारः ।}
	\pend
      

	  \pstart \textbf{तथा चेति} साध्यनिवृत्त्यनुवादेन साधननिवृत्तिप्रकारे सतीत्यर्थः । \textbf{साध्यनिवृत्तिरभावः साधन}स्य \textbf{निवृत्त्या}ऽभावेन \textbf{व्याप्ता} आत्मनियतीकृता \textbf{कथिता} प्रकाशिता । कुत इत्याह—\textbf{साधन}स्य \textbf{निवृत्ता}वभावे \textbf{नियतत्वाद}व्यभिचारित्वात्साध्यनिवृत्तेरिति प्रक्रमात् । यतो यत्र साध्याभावस्तन्नियतत्वमस्य । इमामेव व्याप्तिं व्यनक्ति \textbf{यदी}ति । \textbf{चो} हेतौ । \textbf{धर्मिणीत्यने}न धर्मिमात्रमुपदर्शयन् सर्वोपसंहारवतीं व्याप्तिमाह । कुतः पुनः साध्याभावे साधनाभाव इत्याह—\textbf{हेत्वभावेनेति । अस्ति हेतु}र्श्दृयानुपलम्भः । अनेन व्यापकस्य साधनाभावलक्षणस्याभावो दर्शितः । \textbf{अतो} व्यापकाभावात्साध्याभावः सद्व्यवहारयोग्यत्वलक्षणो व्याप्यो निवर्त्तते । यत एवम्
	\pend
      \leavevmode\marginnote{\textenglish{167/dm}}““

	  \pstart साधर्म्येणापि हि प्रयोगेऽर्थाद्वधर्म्यगतिरिति\footnote{गतिः ।--\cite{dp-msC}} ॥ २६ ॥
	\pend
      ” 

	  \pstart साधर्म्येणेति । साधर्म्येणापि अभिधेयेन युक्ते प्रयोगे क्रियमाणे अर्थात्\footnote{अर्थादिति--\cite{dp-msC}} सामर्थ्यात् वैधर्म्यस्य व्यतिरेकस्य गतिर्भवतीति\footnote{भवति । ही \cite{dp-msB} \cite{dp-msC} \cite{dp-msD}} । हीति यस्मात् । तस्मात् त्रिरूपलिङ्गाख्यानमेतत्\footnote{मेव तत् \cite{dp-msC}} ।
	\pend
       

	  \pstart यदि नाम व्यतिरेकोऽन्वयवता\footnote{वति नोक्तोऽन्वय० \cite{dp-msA} \cite{dp-edP} \cite{dp-edE} \cite{dp-msC}} नोक्तस्तथापि अन्वयवचनसामर्थ्यादेवावसीयते ॥ कथम् ?
	\pend
       “

	  \pstart असति तस्मिन् साध्येन हेतोरन्वयाभावात् ॥ २७ ॥
	\pend
      ” 

	  \pstart असति तस्मिन् व्यतिरेके\footnote{व्यतिरेकबु० \cite{dp-msA}} \footnote{बुद्ध्यध्यवसिते \cite{dp-msA} \cite{dp-msB} \cite{dp-edP} \cite{dp-edH} \cite{dp-edE} \cite{dp-edN}}बुद्ध्याध्यवसिते साध्येन हेतोरन्वयस्य \footnote{बुद्ध्यावसितस्य \cite{dp-msA} \cite{dp-edP} \cite{dp-edH} \cite{dp-edE} \cite{dp-edN} बुद्ध्यवसितस्य \cite{dp-msB} \cite{dp-msD}}बुद्ध्याध्यवसित-\footnote{०सितत्वाभावात् \cite{dp-msC}} स्याभावात् । \footnote{साध्ये नियतमित्यादिनाऽन्वयबा[[बो]]धसामर्थ्यात् व्यतिरेकं दर्शयति--\cite{dp-msD-n}}साध्ये नियतं साधनमन्वयवाक्यादवस्यता साध्याभावे साधनं नाशङ्कनीयम् ।
	\pend
      ”

	  \pstart इतिस्तस्मात् । \textbf{साध्य}स्यासद्व्यवहारयोग्यत्वस्य \textbf{गति}रवसायो भवति, \textbf{सर्वत्र} हेतुत्रयवैधर्म्यप्रयोगे ॥
	\pend
      

	  \pstart स्वभावहेतुमधिकृत्याह--\textbf{स्वभावे}ति । वैधर्म्यप्रतिपादकः प्रयोगस्तथोक्तः । कथं पुनरत्र साध्यनिश्चयो जायत इत्याह--इहेति स्वभावहेतुप्रयोगत्रये । \textbf{चो} यस्मात् \textbf{साधनाभावस्य} सत्त्वादिनिवृत्तेर\textbf{भावः} सत्त्वादिविधि\textbf{रुक्तः । तत}स्तस्मात् । \textbf{व्याप्योऽपि} क्षणिकत्वाभावोऽपि । \textbf{अपिः} साधनाभावनिवृत्त्यपेक्षया साध्याभावनिवृत्तिं समुच्चिनोति । यत एवमितिस्तस्मात् । \textbf{साध्यस्य} क्षणिकत्वस्य \textbf{गति}र्निश्चय इति ॥
	\pend
      

	  \pstart कार्यहेतुमुद्दिश्याह--\textbf{कार्येति} । न केवलं पूर्वयोरित्यपिशब्दः । \textbf{धूमाभावस्याभावो} धूमसत्तैव प्रतिषेधप्रतिषेधस्य विधिरूपत्वादेवं पूर्वत्रापि विज्ञेयम् । ततो व्यापकस्य घूमाभावस्याभावात् \textbf{व्याप्यस्य वह्र्यभावस्याभाव}\footnote{वे} न्यायसिद्धे सति ॥
	\pend
      

	  \pstart न केवलं व्यतिरेकेणाभिधेयेन युक्त इ\textbf{त्यपिश}ब्दः । \textbf{अभिधेयेन} साक्षादभिधाविश्रामविषयेण । \textbf{सामर्थ्याद}न्यथाऽनुपपत्तेः ।
	\pend
      

	  \pstart एतदिह ज्ञातव्यम्--अन्वयव्यतिरेकयोर्भेदस्य व्यावृत्तिनिबन्धनत्वाद् वस्तुतस्तादात्म्यात् स्वभावहेतुजानुमानबलादितरप्रतीतिर्न त्वन्यथाऽनुपपत्तिलक्षणार्थापत्तिरनेनोच्यत इति ।
	\pend
      \leavevmode\marginnote{\textenglish{168/dm}}“

	  \pstart इतरथा \footnote{साध्ये नियत० \cite{dp-msC}}साध्यनियतमेव न प्रतीतं स्यात् । साध्याभावे च साधनाभावगतिर्व्यतिरेकगतिः । अतः साध्यनियतस्य साधनस्याभिधानसामर्थ्यादन्वयवाक्येऽवसितो व्यतिरेकः ॥
	\pend
       “

	  \pstart तथा वैधर्म्येणाप्यन्वयगतिः ॥ २८ ॥
	\pend
      ” 

	  \pstart तथेति । यथाऽन्वयवाक्ये तथाऽर्थादेव वैधर्म्येण प्रयोगेऽन्वयस्यानभिधीयमानस्यापि गतिः ॥ कथम् ?
	\pend
       “

	  \pstart असति तस्मिन् साध्याभावे हेत्वभावस्यासिद्धेः ॥ २९ ॥
	\pend
      ” 

	  \pstart असति तस्मिन् अन्वये बुद्धिगृहीते \footnote{गृहीते ते साध्या--\cite{dp-msA} \cite{dp-edP} \cite{dp-edH}}साध्याभावे हेत्वभावस्यासिद्धेरनवसायात् । हेत्वभावे साध्याभावं नियतं व्यतिरेकवाक्यादवस्यता हेतुसंभवे साध्याभावो नाशङ्कनीयः । इतरथा हेत्वभावे\footnote{साध्याभारो[[वो]]ऽनग्निः--\cite{dp-msD-n}} नियतो\footnote{नियतः साध्याभावो न स्यात्--\cite{dp-msA} \cite{dp-msB} \cite{dp-msD} \cite{dp-edP} \cite{dp-edH} \cite{dp-edE} \cite{dp-edN}} न स्यात् प्रतीतः । हेतुसत्त्वे च साध्यसत्त्व\footnote{साध्यसत्त्वं गतिः--\cite{dp-msA}} गतिरन्वयगतिः । अतः साधनाभाव\footnote{साधनाभावे नियतस्य--\cite{dp-msC}} नियतस्य साध्याभावस्याभिधानसामर्थ्याद् व्यतिरेकवाक्येऽन्वयगतिः ॥
	\pend
      ”

	  \pstart एतदेव दर्शयितुमाह--\textbf{यदि नामे}ति । विशेषाभिधाननिमित्ताभ्युपगमे चायं निपातसमुदायः । \textbf{अन्वयवता}ऽन्वयेनाभिधेयेन युक्ते प्रयोगेणेति प्रस्तावात् । \textbf{अन्वयवचनसामर्थ्यादन्व}\leavevmode\marginnote{\textenglish{62a/ms}}याभिधानबलात् ॥
	\pend
      

	  \pstart अभिप्रायमजानानः पर आह--\textbf{कथमि}ति ।
	\pend
      

	  \pstart \textbf{असती}त्यादि सिद्धान्तवादी । अभिधेयतया स्थित इत्याशङ्कामपाकर्त्तुमाह--\textbf{बुद्ध्येति} । तादात्म्यतदुत्पत्तिनिबन्धने प्रतिबन्धेन प्रतिबद्धत्वात्साधनमिदं साध्याभावे न भवत्येवेति बुद्धयाऽ\textbf{ध्यवसिते} विषयीकृतेऽसति । \textbf{साध्येन हेतोरन्वयस्य}--अन्वीयमानत्वस्य--यत्र साधनं तत्र सर्वत्रावश्यं साध्यमित्येवंरूपस्य \textbf{बुद्ध्याऽध्यवसितस्ये}त्येतदन्वयवाक्योपस्थापितया बुद्ध्या गृहीतस्याभावादभावप्रसङ्गादित्यर्थः ।
	\pend
      

	  \pstart एतदुक्तं भवति । यदि साध्याभावेऽपि साधनं स्यात् तदा यत्रैवादः साध्याभावेऽपि वृत्तमिष्यते तत्रैव तत्साधनमप्यस्ति न च साध्यमिति कथं यत्र यत्र साधनधर्मस्तत्र तत्र साध्यधर्म इति सर्वोपसंहारेणान्वय उक्तः स्यादिति । तस्माद् हेतोरन्वयाभावाद् हेतोः \textbf{साध्ये नियतं} साध्याविनाभाविसाधन\textbf{मन्वयवाक्वादिवा}\footnote{क्यादव}\textbf{स्यता} प्रतियता \textbf{साध्याभावे साधनं नाशङ्कनीयं} न सन्देहनीयम् । आशङ्कानिषेधेन च विपर्ययोऽत्यन्तं निषिद्धः ।
	\pend
      

	  \pstart कथं पुनस्तेनैवं नाशङ्कनीयमित्याह--\textbf{इतरथे}ति--अतोऽन्येन प्रकारेण । अस्तु साध्याभावे साधनाभावावसायो व्यतिरेकस्तु कथं प्रतीयत इत्याह--\textbf{साध्येति । चो} यस्मादर्थे । यत  \leavevmode\marginnote{\textenglish{169/dm}} “
	  
	यदि नामाकाशादौ साध्याभावे साधनाभावस्तथापि किमिति हेतुसंभवे साध्यसंभव इत्याह-- “
	  
	न हि स्वभावप्रतिबन्धेऽसत्येकस्य निवृत्तावपरस्य नियमेन निवृत्तिः ॥ ३० ॥” 
	  
	नहीति । \footnote{भावः--उत्पादः, सत्ता वा--\cite{dp-msD-n}}स्वभावेन प्रतिबन्धो यस्तस्मिन्नसत्येकस्य साध्यस्य निवृत्त्या नापरस्य साधनस्य नियमेन युक्ता नियमवती निवृत्तिः ॥ “
	  
	स च द्विप्रकारः सर्वस्य । तादात्म्यलक्षणस्तदुत्पत्तिलक्षणश्चेत्युक्तम् ॥ ३१ ॥”” एव\textbf{मतो} हेतोरित्यादिनोपसंहारः । असति व्यतिरेके प्रतिबन्धानाक्षेपादन्वयस्यैवासत्त्वादसतश्च सत्त्वेन तत्प्रतिपादनायोगात्प्रेक्षावतामन्वयवचनमेव न प्रयुक्तं स्यादिति समुदायार्थः ॥
	\pend
      

	  \pstart \textbf{वैधर्म्येणा}भिधेयेन युक्त इत्यध्याहारः । \textbf{प्रयोगे} साधनवाचकशब्दसमूहे ॥
	\pend
      

	  \pstart \textbf{कथमिति} परः ।
	\pend
      

	  \pstart \textbf{असती}त्यादि सिद्धान्तवादी । \textbf{बुद्धिगृहीत} इति बुद्ध्यन्तरगृहीत इति ग्राह्यम् । एतदेव प्रतिपादयन्नाह--\textbf{हेत्वभाव} इति । \textbf{इतरथा} हेतुसद्भावे साध्याभावसम्भवप्रकारे सति \textbf{नियतो न स्यात्प्रतीतः} साध्याभाव इति शेषः ।
	\pend
      

	  \pstart मा भून्नियतोऽनुगतः किं नश्छिन्नमित्याह--\textbf{हेतुसत्त्वे} इति । \textbf{चः} समुच्चये ।
	\pend
      

	  \pstart अयमाशयः--सति साधनेऽवश्यं साध्यमित्येवंलक्षणोऽन्वयोऽस्त्येव । केवलं व्यतिरेकवाक्यान्न प्रतीयत इत्युच्यते पूर्वपक्षवादिना । यदा च हेत्वभावे न नियतः साध्याभावः सम्भाव्यत इति कुतो यत्र यत्र साधनं तत्र तत्र साध्यमित्येवंरूपोऽन्वयः सिद्ध्येत् । तत्रैवं सम्भावनाविषये हेतुभावेऽपि साध्याभावादिति ।
	\pend
      

	  \pstart यत एवमतोऽस्माद् हेतोरित्यादिनोपसंहारः । अत्राप्ययमाशयः--यदि यत्र साधनं तत्रावश्यं साध्यमिति न स्यात्तदा तत्रैव तावदसत्यपि साध्ये साधनं वृत्तमिति कुतः साध्याभावे साधनं न वर्त्तत इत्येवंरूपो व्यतिरेकः सिद्ध्येदिति ॥
	\pend
      

	  \pstart अत्राभिप्रायमपरिज्ञायमानः \footnote{रिजानानः} प्राह--\textbf{यदि नामेति} ।
	\pend
      

	  \pstart \textbf{नही}त्यादि प्रतिविधानमाचार्यीयं नहीत्यादिना व्याचष्टे । अयं च मौलो हिशब्दः पश्चाद् व्याख्यास्यते । \textbf{नियमेना}वश्यंतया । या चावश्यं \textbf{भाविनी निवृत्तिः} सा नियमेन युक्ता भवतीत्यर्थकथनमेतत् ॥
	\pend
      \leavevmode\marginnote{\textenglish{170/dm}}“

	  \pstart स च स्वभावप्रतिबन्धो द्विप्रकारः सर्वस्य \footnote{हेतोः--\cite{dp-msD-n} । “प्रतिबद्धस्य” इति नास्ति \cite{dp-msA} \cite{dp-edP} \cite{dp-edH}}प्रतिबद्धस्य । तादात्म्यं लक्षणं निमित्तं यस्य स तथोक्तः । तदुत्पत्तिर्लक्षणं निमित्तं यस्य स तथोक्तः । यो यत्र प्रतिबद्धस्तस्य स प्रतिबन्धविषयोऽर्थः स्वभावः कारणं वा स्यात् । अन्यस्मिन् प्रतिबद्धत्वानुपपत्तेः । तस्माद् \unclear{द्वि}प्रकारः स इत्युक्तम् । स च सार्ध्येऽर्थे लिङ्गस्य इत्यत्रान्तरेऽभिहितः ॥
	\pend
       “

	  \pstart तेनं हि निवृत्तिं कथयता प्रतिबन्धो दर्शनीयः । तस्मात् निवृत्तिवचनमाक्षिप्तप्रतिबन्धोपदर्शनभेव भवति । यच्च प्रतिबन्धोपदर्शनं \footnote{तदन्वय--\cite{dp-msC}}तदेवान्वयवचनमित्येकेनापि वाक्येनान्वयमुखेन व्यतिरेकमुखेन वा प्रयुक्तेन सपक्षासपक्षयोर्लिङ्गस्य सदसत्त्वख्यापनं \footnote{सदसत्त्वाख्यापनं--\cite{dp-msC}}कृतं भवतीति नावश्यं वाक्यद्वयप्रयोगः ॥ ३२ ॥
	\pend
      ” 

	  \pstart \footnote{पूर्वसूत्रोक्तः--\cite{dp-msD-n}}हिर्यस्मादर्थे । यस्मात् स्वभावप्रतिबन्धे निवर्त्यनिवर्तकभावस्तेन\footnote{तेनेत्युपसंहरति--\cite{dp-msD-n}} साध्यस्य निवृत्तौ साधनस्य\footnote{साधननिवृत्तिं \cite{dp-msC} \cite{dp-msD}} निवृत्तिं कथयता प्रतिबन्धो निवर्त्यनिवर्तकयोर्दर्शनीयः । यदि हि साधनं साध्ये प्रतिबद्धं भवेद् एवं साध्यनिवृत्तौ \footnote{निवृत्तौ नियमेन--\cite{dp-msB} \cite{dp-msC} \cite{dp-msD}}तन्नियमेन निवर्तेत । यतश्च तस्य प्रतिबन्धो दर्शनीयः तस्मात् साध्यनिवृत्तौ यत् साधननिवृत्तिवचनं\footnote{अन्तर्भावितम्--\cite{dp-msD-n} । तेनाक्षिप्त प्रतिबन्घोपदर्शनं तदेवान्वय--\cite{dp-msB}} तेनाक्षिप्तं प्रतिबन्धोपदर्शनम् । यच्च तदाक्षिप्तं\footnote{क्षिप्तप्रति \cite{dp-msA} \cite{dp-edP} \cite{dp-edH} \cite{dp-edE} \cite{dp-edN}} प्रतिबन्धोपदर्शनं तदेवान्वयवचनम् । प्रतिबन्धश्चेदवश्यं दर्शयितव्यो न वक्तव्यस्तर्ह्यन्वयः । यस्माद् दृष्टान्ते प्रमाणेन प्रतिबन्धो\footnote{वचनरूपो यत्र यत्र धूमस्तत्राग्निरित्येवं न वक्तव्यस्तर्ह्यन्वयः प्रतिबन्धशून्यः । साध्यहेत्वोस्तादात्म्यतदुत्पत्तिरूपप्रतिबन्धे स्थिते सिद्ध एवान्वय इति भावः--\cite{dp-msD-n} ।} दर्श्यमान एवान्वयो नापरः कश्चित्,
	\pend
      ”

	  \pstart अथ स्वभावप्रतिबन्धश्चेदेकनिवृत्तावपरनिवृत्तिनिबन्धनं तदा कार्यहेतोरेव व्यतिरेको न स्वभावहेतो\leavevmode\marginnote{\textenglish{62b/ms}}रिति\footnote{पाठोऽत्र घृष्टः ।}...\textbf{तस्या}सौ न तर्हि कार्यंहेतावित्याह--\textbf{स चे}ति । \textbf{चो} यस्मादर्थे । \textbf{स्वभावेन प्रतिबन्धः} प्रतिबद्धत्वं साध्यायत्तत्वम् । कस्यासावित्याशङ्कायामाह--\textbf{प्रतिबद्धस्य} साधनस्य । सर्वस्येत्यनेन व्याप्तिं दर्शयति । तत्र संयोगादिनिमित्तशङ्काव्युदासायाभिमतं द्वितं\footnote{द्वैतं}दर्शयन्नाह--तादात्म्यमित्यादि । लक्ष्यतेऽनेनेति \textbf{लक्षणम्} । अत एवाह \textbf{निमित्तमि}ति किम्पुनस्तेन स्वभावेन\footnote{पाठोऽत्र घृष्टः ।}...मित्याह--\textbf{अन्यस्मिन्न}स्वभावेऽकारणे च । संयोगसमवाययोः प्रमाणबाधितत्वेन निमित्तत्वानुपपत्तेरिति भावः । यत एवं \textbf{तस्माद्} हेतोः । \textbf{स} इति प्रतिबन्धः ॥
	\pend
      

	  \pstart पूर्वकं हिशब्दमिदानीं यथायोगं व्याचष्टे--हिरिति यस्मादर्थवृत्तिं हिशब्दम् । अस्य  \leavevmode\marginnote{\textenglish{171/dm}} “
	  
	तस्मान्निवर्त्यनिवर्तकयोः\footnote{०र्तकप्रतिब--\cite{dp-msA} \cite{dp-msD} \cite{dp-edP} \cite{dp-edH} \cite{dp-edE}} प्रतिबन्धो ज्ञातव्यः । तथा चान्वय एव ज्ञातो भवति । इतिशब्दो हेतौ । यस्मादन्वयेऽपि\footnote{अन्वये व्य० \cite{dp-msA} \cite{dp-edP} \cite{dp-edH} \cite{dp-edE} \cite{dp-edN}} व्यतिरेकगतिः व्यतिरेके चान्वयगतिः, तस्माद् एकेनापि सपक्षे चासपक्षे च सत्त्वासत्त्वयोः ख्यापनं कृतम् । 
	  
	अन्वयो मुखमुपायोऽभिधेयत्वाद् यस्य तद् अन्वयमुखं वाक्यम् । एवं व्यतिरेको मुखं \footnote{मुखमस्य--\cite{dp-msC}}यस्येति । इति\footnote{इतिकरणो हेतौ \cite{dp-msA}} हेतौ । यस्मादेकेनापि वाक्येन द्वयगतिस्तस्मादेकस्मिन् साधनवाक्ये द्वयोरन्वयव्यतिरेकवाक्ययोरवश्यमेव प्रयोगो न कर्त्तव्यः । 
	  
	अर्थगत्यर्थो हि शब्दप्रयोगः । अर्थश्चेदवगतः, किं शब्दप्रयोगेण ? \footnote{एकमेव त्वन्वय \cite{dp-msA} \cite{dp-edP} \cite{dp-edH} \cite{dp-edE} \cite{dp-edN}}एकमेवान्वयवाक्यं व्यतिरेकवाक्यं वा प्रयोक्तव्यम् ॥” “
	  
	अनुपलब्धावपि--यत् सद् उपर्लाब्धलक्षणप्राप्तं तद् उपलभ्यत एवेत्युक्ते--अनुपलभ्यमानं तादृशमसदिति प्रतीतेरन्वयसिद्धिः ॥ ३३ ॥” हिशब्दस्य अर्थं कृत्वा शब्दपदार्थकस्या\footnote{पाठोऽत्र घृष्टः ।}...एवमुक्ते वक्ष्यमाणे च सर्वत्र द्रष्टव्यम् । अमुमेव यस्मादर्थमपेक्ष्य तेनेति योजयितुमिति दर्शयितुमाह--\textbf{यस्मादिति} । एतच्च \textbf{नहीत्यादि}वाक्यस्य प्रकाश्यमर्थं गृहीत्वोक्तं न त्वभिधेयम्, निवृत्तिनिषेधस्यैव तत्राभिधेयत्वात् ।
	\pend
      

	  \pstart ननु प्रतिबन्धः प्रतिबद्धत्वम् । स च निवर्त्तमानस्यैव न निवर्त्तकस्य । \textbf{यदि ही}त्यादिना च\footnote{पाठोऽत्र घृष्टः ।}...दर्शयिष्यति । तत्कथमिह निवर्त्यनिवर्त्तकयोरित्युक्तम् । सत्यम् । केवलमत्र प्रतिबन्धशब्देन प्रतिबद्धत्वं प्रतिबन्धविषयत्वं च विवक्षितम् ।
	\pend
      

	  \pstart तेनायमर्थः । प्रतिबन्धविषये प्रतिबद्धत्वं दर्शनीयम् । तथा च न कश्चिद् दोषः ।
	\pend
      

	  \pstart कस्मात्पुनः प्रतिबन्धो दर्शनीय इत्याशङ्क्याह--\textbf{यदी}ति । हीति यस्मात् । \textbf{तेन} व्यतिरेकव\textbf{चनेनाक्षिप्तं} प्रकाशितं । प्रतिबन्धस्तादात्म्यतदुत्पत्तिनिबन्धनं दर्शयष्यिते प्रकाश्यतेऽनेनेति तथा । प्रतिबन्धोऽवश्यदर्शयितव्योऽन्यथा व्यतिरेकस्यैवासिद्धेरिति भावः । भवतु तत्तथा--\textbf{यदि}ति । \textbf{चोऽ}वधारणे । तदयमर्थः--यदेवाक्षिप्तप्रतिबन्धोपदर्शनं \textbf{तदेवान्वयवचनम}न्वयप्रकाशनम् ।
	\pend
      

	  \pstart ननूपदर्श्यतां प्रतिबन्धो\footnote{पाठोऽत्र घृष्टः ।}...स कथं तेनोक्तो भवतीत्याह--\textbf{प्रतिबन्ध} इति । \textbf{यद्यवश्यं दर्शयितव्यो} नियमेन ख्यापनीयस्त\textbf{र्हि न वक्तव्यः} प्रतिपादयितव्योऽन्वयः ।
	\pend
      

	  \pstart ननु च दृष्टान्तेन प्रतिबन्धसाधनकेन प्रमाणेन केवलं प्रतिबन्धः प्रदर्श्यते, न त्वन्वयः । तत्कथं वक्तव्यस्तेन वाक्येनेत्याह--\textbf{यस्मादि}ति । तस्य तादात्म्यनिबन्धनस्य तदुत्पत्तिनिबन्धनस्य वाऽन्वयात्मकत्वादिति भावः ॥
	\pend
      \leavevmode\marginnote{\textenglish{172/dm}}“

	  \pstart अनुपलब्धावपि व्यतिरेकेणो\footnote{केण युक्तेन \cite{dp-msB}} क्तेनान्वयगतिः । यत् सद् उपलब्धिलक्षणप्राप्तमिति साध्यस्य--असद्व्यवहारयोग्यत्वस्य निवृत्तिं दृश्यसत्त्व\footnote{सत्त्वस्वरूपामाह--\cite{dp-msC}} रूपामाह । तदुपलभ्यत एवेत्यनुपलम्भस्य निवृत्तिमुपलम्भरूपामाह । तदनेन साध्यनिवृत्तिः साधननिवृत्त्या व्याप्ता दर्शिता । यदि च साधनसंभवेपि \footnote{सत्त्वम्--\cite{dp-msD-n}}साध्यनिवृत्तिर्भवेत् न साधनाभावेन\footnote{उपलम्भेन--\cite{dp-msD-n}} व्याप्ता भवेत् । अतो व्याप्तिं\footnote{व्याप्तिप्रति \cite{dp-msA}} प्रतिपद्यमानेन साधनसंभवः साध्यसंभवेन व्याप्तः प्रतिपत्तव्यः । अत एवाह--अनुपलभ्यमानं तादृशमिति दृश्यमसदिति प्रतीतेः\footnote{“प्रतीतेः” इति नास्ति \cite{dp-msA}} संप्रत्ययाद् अन्वयसिद्धिरिति ॥
	\pend
       “

	  \pstart द्वयोरप्यनयोः प्रयोगयो\footnote{प्रयोगेऽवश्यं \cite{dp-msB} \cite{dp-edP} \cite{dp-edH} प्रयोगे नावश्यं--\cite{dp-edE} \cite{dp-edN}}र्नावश्यं पक्षनिर्देशः ॥ ३४ ॥
	\pend
      ” 

	  \pstart यतश्च साधनं साध्यधर्मप्रतिबद्धं तादात्म्य-ददुत्पत्तिभ्यां प्रतिपत्तव्यं द्वयोरपि प्रयोगयोः, तस्मात् पक्षोऽवश्यमेव न निर्देश्यः । यत् साधनं साध्यनियतं प्रतीतं तत एव साध्यधर्मिणि दृष्टात्\footnote{दृष्ट्वा \cite{dp-msA} \cite{dp-edP} \cite{dp-edH}} साध्यप्रतीतिः । अतो न किंचित् साध्यनिर्देशेनेति ॥
	\pend
      ”

	  \pstart \textbf{निवृत्तिवचनं} चैतदुपलक्षणं द्रष्टव्यम् । तेनान्वयवचनेपि सर्वं यथायोगं द्रष्टव्यम् ।
	\pend
      

	  \pstart यस्मादेवमनुवादविधिक्रमस्तत् तस्मादनेन वाक्येन \textbf{दर्शिता} प्रकाशिता । भवत्वभाक्योर्व्याप्यव्यापकभावस्तथाप्यन्वयः कथं सिद्ध्यतीत्याह--\textbf{यदी}ति । \textbf{चो} हेतौ । \textbf{न व्याप्ता भवेत् साध्यनिवृत्तिरि}ति प्रकृतत्वात् ॥
	\pend
      

	  \pstart \textbf{तस्मात्पक्षोवश्यमेव न} निर्देश्य इत्यनेन \textbf{नावश्यं पक्षनिर्देश} इत्यस्यार्थः कथितः ।
	\pend
      

	  \pstart एवञ्च व्याचक्षाणेन यत्कैश्चित्स्वयूथ्यैर्विद्वस्यमानैः “अवश्यं पक्षनिर्देशो न; किन्तर्हि ? कदाचिन्निर्देशः, कदाचिन्न” इति व्याख्यातं तदपहस्तितं द्रष्टव्यम् । कदाचिदपि तस्य प्रयोगार्हत्वे प्रतिज्ञायाः साधनाङ्गत्वप्रसङ्गात् । तथात्वे च \textbf{वादन्यायस्य}\leavevmode\marginnote{\textenglish{63a/ms}}\footnote{अस्मिन् पत्रे अधिकं घृष्टं वर्तते । अत एव सम्यक् न पठयते--सं०} \add{विरोधः स्यात्} ।
	\pend
      

	  \pstart नन्वसति साध्यनिर्देशे कुतस्तदवगतिर्येन तदनिर्देश इत्याशङ्क्याह--\textbf{यदि}ति । \textbf{साध्यनियतं} साध्यनान्तरीयकम् । प्रतिबन्धसाधकेन प्रमाणेनेति बुद्धिस्थम् । \textbf{तत एव} साधनात् \textbf{धर्मिणि} विवादास्पदीभूते \textbf{दृष्टात्} प्रमाणेनावगतात् \textbf{साध्य}स्य धर्मधर्मिसमुदायस्य \textbf{प्रतीति}र्भवति ।
	\pend
      

	  \pstart एतदुक्तं भवति...प्रतीतार्थप्रतिपादकेन कर्त्तव्यमिति । \textbf{शङ्कराचार्य}...ईश्वरकारणे...विशेषः प्रतीयेतेत्यादिना वाक्यप्रबन्धेन विरुद्धशङ्काव्यवच्छेदार्थं साध्यवचनमिति समाधानात् । तथाहि तस्य प्रबन्धस्यायमर्थः--असति साध्यवचने यत् कृतकं तद् सर्वमनित्यं  \leavevmode\marginnote{\textenglish{173/dm}} यथाः...कुतश्चिद् भ्रान्तिनिमित्तादीदृशं व्याप्तिवचनं संभाव्यते । अथ तथाविधाभिप्रायो वक्ता कृतकत्वं प्रयुञ्जानः तस्य नित्यत्वेन व्याप्तिं ब्रूयात् । तदयुक्तम् । यतस्तयोर्व्याप्तिं ब्रुवाणो...कथयेत्...चेत् साधन\add{... ... ...} साधनविकलः साध्यविकलो वा मा भूद् दृष्टान्त इति यत् कृतकं तदनित्यमिति प्रयोगोपि विरुद्धवाद्येव द्रष्टव्यः । ... तदेतद्भौ \add{ताख्यानं} किमत्र ब्रूमः तथा हि कः खलु प्रेक्षावान् साधनविकलं विहायसं साध्यविकलं च कुम्भमालोचयितुमीशानोऽभिप्रेतनित्यत्वविरुद्धेनानित्यत्वेन साध्ये \footnote{ध्य} विकलतया कुम्भसन्निभ एव घटे व्याप्तिं दर्शयेत् । भ्रान्त्या चेत् । साधनवैकल्यमाकाशस्य\add{... ...}वाऽभिप्रेतेन नित्यत्वेन कृतकत्वस्य व्याप्निं न प्रदर्शयेत् । तत्र साधनवैकल्यम्, कुम्भे च साध्यवैकल्य...कुम्भतुल्येपि घटे व्यामुह्यति, भ्रान्तेर्नियतनिमित्तत्वादिति चेत् । एवं तर्हि पक्षवचनेपि कथं...संभाव्यत्वात् । ...समानोपि विसंवादनाभिप्रायो न...। न चाप्युत्पत्तौ न चान्यदास्यान्यादृशं वचनमिदानीं तु द्रुतादिभेदभिन्नमित्यादिना प्रकारेण वचनविशेषेण ज्ञानेन कार्यभूतेनाविपरीतोऽभिप्रायोऽवधार्यते । तेन\add{... ...}तदनित्यमित्यभिधातरि...विशेषः । वचनसांकर्यान्नैवं तत्र निश्चय इति चेत् । एतत् पक्षवचनेपि समानम् । एवं निश्चेतुं शक्येति तस्य तत्र तथात्वावगमो भविष्यतीति चेत् सर्वं समानमन्यत्राऽविशेषात् । किञ्च स एवं वादी तपस्वी ...\leavevmode\marginnote{\textenglish{63b/ms}}स्वयमेव तावद् दुष्यति । न हि कश्चित्साधनवादी प्रतिज्ञाहेतूदाहरणान्येवाभिधायोपरमते । किन्तर्हि ? निगमनमप्युपादत्ते--तस्मादनित्यः शब्द इति । एवमुक्ते च कुतो नित्यत्वशङ्का, यतः कृतकत्वस्य विरुद्धता भवेत् ? ततश्च निगमनेनैव विरुद्धशङ्काव्यवच्छेदस्य कृतत्वात्पक्षवचनमपार्थकम् । यदा\textbf{हाक्षपादः}\footnote{अत्र अक्षपादवचनत्वेन यदुद्धृतं तन्नास्ति न्यायसूत्रे किन्तु वार्तिके--“साध्यविपरीतप्रसङ्गप्रतिषेधार्थं यत् पुनरभिधानं तत् निगमनम्” इति वर्तते--\href{http://http://sarit.indology.info/?cref=}{१. १. ३९.}}--“साध्ये विपरीतशङ्काव्यवच्छेदार्थ निगमनमिति” । तथा हेतुवचनस्याप्यनुत्थानमायातम् । निगमनेऽनाश्वास इति चेद् । हन्त प्रतिज्ञावचनेऽप्यनाश्वासस्तुल्यः । यत्तत्र समाधानं तन्निगमनेऽपि भविष्यति । तस्मात् तन्मतेऽपि निगमनादेवाभिप्रेतसाध्यप्रतीतेर्न विरुद्धाशङ्कानिरासार्थं पक्षवचनं करणीयम् । तदयं यथा नाम कश्चित् स्वाङ्गुलिज्वालया परं दिधक्षुः स परं दहेद्वा न वा, स्वाङ्गुलिदाहमेव तावदनुभवतीति वृत्तान्तो जातः ।
	\pend
      

	  \pstart यद्येवं निगमनमप्यपार्थकमापद्यत इति चेत् । अयमपरोऽस्तु दोषः । कथं नाम \textbf{ताथागता ज}यन्ति ? केवलं सति निगमने विरुद्धशङ्काव्यवच्छेदार्थं पक्षवचनं न कार्यम् । निगमनेनैव तदाशङ्काव्यवच्छेदस्य कृतत्वादित्युच्यत इति ।
	\pend
      

	  \pstart \textbf{त्रिलोचनः} पुन\textbf{र्न्यायभाष्यटीका}यामिदमवादीत्--“साध्यवचनमसाधनाङ्गवचनं न भवति, यतो विवादेषु परप्रतिपत्तिमधिकृत्य न प्रयोगनियमः शक्यः । पटुमन्दादिभावेन परप्रतिपत्तीनामनवस्थानात् । तथा हि हेतुवचनादेव कश्चित्प्रत्येति । कश्चित्पुनरन्तरेणापि हेतुवचनं वक्तृस्वरूपपरिशीलनात्प्रागेव शब्दनिष्पत्तेरोष्ठादिस्थानव्यापारोपलब्धेर्वक्तुरभिप्रेतमन्वेति । तस्मादनपेक्षितपरप्रतिपत्तिरेवायं ज्ञाता ज्ञानस्थमर्थं प्रतिपादयन्तं तस्य स्वप्रतिपत्त्याऽऽरूढस्यार्थस्य \leavevmode\marginnote{\textenglish{174/dm}} “
	  
	\footnote{एवमे० \cite{dp-msA} \cite{dp-msB} \cite{dp-edP} \cite{dp-edH} \cite{dp-edE} एतमे० \cite{dp-edN}}एनमेवार्थमनुपलब्धिप्रयोगे दर्शयति-- “
	  
	यस्मात् साधर्म्यवत्प्रयोगेऽपि--यदुपलब्धिलक्षणप्राप्तं सन्नोपलभ्यते सोऽसद्व्यवहारविषयः\footnote{विषयः सिद्धः \cite{dp-msC}} । नोपलभ्यते चात्रोपलब्धिलक्षणप्राप्तो घट इत्युक्ते सामर्थ्यादेव नेह घट इति भवति ॥ ३५ ॥” 
	  
	साधर्म्यवति प्रयोगेपि सामर्थ्यादेव नेह प्रदेशे\footnote{नेह घट० \cite{dp-msA} \cite{dp-msB} \cite{dp-edP} \cite{dp-edH} \cite{dp-edN}} घट इति भवति । 
	  
	किं पुनस्तत् सामर्थ्यमित्याह--यदुपलब्धिलक्षणप्राप्तं\footnote{प्राप्त मिति । अनु० \cite{dp-msA} \cite{dp-edP} \cite{dp-edH}} सन्नोपलभ्यते--इत्यनुपलम्भानुवादः । सोऽसद्व्यवहारविषयः--इत्यसद्व्यवहारयोग्यत्वविधिः । तथा च सति दृश्यानु-” वाचकं शब्दं प्रयोक्तुमर्हति । स्वप्रतिपत्तिश्च लिङ्गजा ज्ञापनीयधर्मविशिष्टं धर्णिमभिनिविशते । तस्मात्परस्य विवादयित्रा ज्ञानस्थमर्थं परो बोद्धव्य इति स एव परं प्रत्युपाय इति ।”
	\pend
      

	  \pstart तदेत\textbf{त्कार्पटिककर्णाट}रटितमश्रद्धेयं धीमताम् । तथा हि--सत्यम्, स्वप्रतिपत्त्याऽऽरूढ एवार्थः परस्मै प्रतिपाद्यते । केवलमिदमालोच्यताम्--किं पक्षधर्मवचनाद् व्याप्तिवचनसहितात्सोऽर्थः प्रतिपादितो भवति नवेति । प्रतिपादने किं प्रतीतप्रत्यायकेन तद्वचनेन कार्यम् ? तावतो वचनात्तत्प्रतिपत्तिमपह्नवानेव \footnote{नेन} तु नापहनुतं नाम किञ्चिद् । अवश्यं चैतदन्यथा स्वार्थानुमानकाले प्रतिज्ञावचनमन्तरेण कथं प्रतिपत्तिः स्यात् ? स्वप्रतिपत्तिकाले च यावतोऽ थात्साध्यप्रतीतिरासीत् परार्थानुमानकालेऽपि तावत एव वचनमुपादेयम् । तत्र च न प्रज्ञापनीयधर्मविशिष्टधर्मिदर्शनपूर्वकादिसाधनादिदर्शनात् साध्यप्रतीतिरासीत् । किन्तर्हि ? पक्षधर्मदर्शनात् तदविनाभावस्मरणसहितादिति तावत एव वचनं न्याय्यम् ।
	\pend
      

	  \pstart \textbf{अथो}क्तं पटुमन्दादि भावेन परप्रतीतीनामनवस्थानान्न शक्यते प्रयोगनियमः कर्त्तुमिति । सत्यमुक्तम्\leavevmode\marginnote{\textenglish{64a/ms}} केवलं स्ववधाय कृत्योत्थापनप्रायं तत् । यतः पटुमन्दादिभेदेन प्रतिपतृणामनेकप्रकारत्वात्स्यादपि कश्चिद् यः पञ्चावयवेऽपि वाक्ये प्रयुक्ते पूर्वं संशयजिज्ञासादिवचनमन्तरेण न बुद्ध्यते बोधयितव्यमिति तद्वचनस्याप्यवश्यप्रयोज्यत्वादवयवत्वादवगत\footnote{दपगतं} पञ्चावयवत्वं साधनवाक्यस्य । अभ्युपगमे च \textbf{गौड्.अकाश्मीर}पुरुषविधायो \footnote{विषयो}पाख्यानं कुतूहलास्पदमवतरते । प्रतिज्ञाहे\textbf{तू}दाहरणोपनयनिगमनान्यवावयवा इति शास्त्रस्थितेरपसिद्धान्तोऽपि दीप्ताज्ञः पार्थिव इव निगृहणाति ।
	\pend
      

	  \pstart अथ किमस्य सम्भवोऽस्ति यो निर्दिष्टे हि साध्ये साधने वाऽभिहिते निदर्शिते चोदाहरणे कृतेऽप्युपनये निगमिते च सर्वावयवव्यापारे साध्यं न बुध्यत इति ? ननु अस्यापि प्रतिपत्तुः किमस्ति सम्भवो यत्र धर्मिणि साधनं बोधितः, तस्य साध्याविनाभावितां स्मरितोऽपि यस्तत्र साध्यं नावबुध्यत इति ? सम्भवति बुद्धिमान्द्यादिति चेत् । सर्वं समानमिदमन्यत्राभिनिवेशादित्यलं विस्तरेण ।
	\pend
      \leavevmode\marginnote{\textenglish{175/dm}}“

	  \pstart पलम्भोऽसद्व्यवहारयोग्यत्वेन व्याप्तो दर्शितः । नोपलभ्यते च इत्यादिना\footnote{०भ्यते इत्या० \cite{dp-msA} \cite{dp-msB} \cite{dp-edP} \cite{dp-edH} \cite{dp-edN}} साध्यधर्मिणि सत्त्वं लिङ्गस्य दर्शितम् । यदि च साध्यधर्मस्तत्र साध्यधर्मिणि न भवेत् साधनर्मोऽपि न भवेत् । साध्यनियतत्वात् तस्य साधनधर्मस्येति सामर्थ्यम ॥
	\pend
       “

	  \pstart तथा वैधर्म्यवत्प्रयोगेऽपि--यः सद्व्यवहारविषय उपलब्धिलक्षणप्राप्तः, स उपलभ्यत एव । न\footnote{न च \cite{dp-msC}} तथाऽत्र तादृशो घट उपलभ्यत इत्युक्ते सामर्थ्यादेव नेह सद्व्यवहारविषय इति भवति ॥ ३६ ॥
	\pend
      ” 

	  \pstart यथा साधर्म्यवत्प्रयोगे तथा वैधर्म्यवत्प्रयोगेऽपि सामर्थ्यादेव नेह सद्व्यवहार\footnote{०व्यवहारस्य विष० \cite{dp-msC} \cite{dp-msD}}विषयोऽस्ति घट इति भवति ।
	\pend
       

	  \pstart सामर्थ्य दर्शयितुमाह--यः सद्व्यवहारविषय इति विद्यमानः । उपलब्धिलक्षणप्राप्त इति दृश्यः इत्येषा साध्यनिवृत्तिः । \footnote{“स” नास्ति \cite{dp-msA} \cite{dp-msB} \cite{dp-msC} \cite{dp-msD} \cite{dp-edP} \cite{dp-edH}}स उपलभ्यत एवेति साधननिवृत्तिरिति । अनेन च\footnote{“च” नास्ति \cite{dp-msC} \cite{dp-edE} अनेन न \cite{dp-edH}} साध्यनिवृत्तिः साधननिवृत्त्या व्याप्ता दर्शिता । न तथेति--यथाऽन्यो दृश्य उपलभ्यते न तथात्र प्रदेशे तादृश इति दृश्यो घट उपलभ्यत इति । अनेन साध्यनिवृत्तेर्व्यापिका निवृत्तिरसती साध्यधर्मिणि दर्शिता । यदि च\footnote{च न साध्य० \cite{dp-msA} \cite{dp-msB} \cite{dp-edP} \cite{dp-edH} \cite{dp-edN}} साध्यधर्मः साध्यधर्मिणि न स्यात् साधनधर्मोऽपि\footnote{०र्मिणि भवेत् साधन० \cite{dp-msA} \cite{dp-msB} \cite{dp-edP} \cite{dp-edH} \cite{dp-edE} \cite{dp-edN}} न भवेत् ।
	\pend
      ”

	  \pstart तदपि\footnote{यदपि}\textbf{न्यायभाष्यटीका-वातिंकयोर्विश्वरूपोद्योतकरावा}हतुः “पुरा विषयनिरूपणपूर्वकमेव हि करणव्यापरणं दृष्टम् । करणं च साधन\footnote{नं}व्यापारयितव्यम् । अतो विषयनिरूपणं साध्यवचनेन क्रियते, अन्यथा करणप्रवर्त्तनस्याशक्यत्वादिति ।”\footnote{नोपलभ्यते न्यायवार्तिके--सं०} तदपि न चतुरस्रम् । यतो यदि हेतुं प्रयुञ्जानेन विषयः सिसाधयिषितोऽर्थों निरूपयितव्यो बुद्धौ निवेशनीय इत्यभिमतम्, तदाऽभ्युपगम एवोत्तरम् । नहि कश्चित् साध्यमनिश्चित्यैव परप्रतिपत्तये साधनवाक्यमभिधत्ते । अथ वचनेन करणस्य हेतोः स विषयो दर्शयितव्य इति मतिः, तदा तेनैव तावद् दर्शितेन कोऽर्थः ? यदि परस्य प्रतीतिरन्यथा न स्यात्सर्वं शोभेतेत्युत्तरमिति किं क्षुण्णक्षोदीकरणेन ?
	\pend
      

	  \pstart \textbf{अध्ययनः} पुना \textbf{रुचिटीका}यामिदमवोचत् “धर्मविशिष्टस्य धर्मिणो निर्देशः क्रियते श्रोतुराश्वासनार्थम् । न त्वादौ धर्मविशिष्टस्य धर्मिणो निर्देशो युक्तः । अयुक्ततां \footnote{ता} तस्य प्रतिपत्तावदृष्टत्वात् । तत्र प्रदेशमात्रमुपलभते, तत्स्थं च धर्मम् । ततोऽविनाभावं स्मरति । तदनन्तरं तदेवेदमिति परामृशति । ततो विशिष्टतां प्रदेशस्य प्रतिपद्यते, न त्वादेव \footnote{त्वादावेव} । परामर्शस्य च स्वार्थपूर्वकत्वम् । न च स्वार्थे धर्मविशिष्टस्य धर्मिणो दर्शनमस्ति । तेन परप्रतिपत्तावपि न कार्यम् । आदौ तु क्रियते, प्रतिपाद्यस्यास्थोत्पादनार्थमिति ।” \leavevmode\marginnote{\textenglish{176/dm}} “
	  
	अस्ति च साधनधर्म इति सामर्थ्यम् ।\footnote{सामर्थ्यात् ततः \cite{dp-msA} \cite{dp-msB} \cite{dp-edP} \cite{dp-edH} \cite{dp-edN}} अतः सामर्थ्यात् नात्स्यत्र घट इति प्रतीतेर्न पक्षनिर्देशः । एवं कार्यस्वभावहेत्वोरपि सामर्थ्यात् संप्रत्यय इति न \footnote{पक्षो निर्देश्यः \cite{dp-edE}}पक्षनिर्देशः ॥ “
	  
	कीदृशः पुनः \footnote{पक्षः निर्देशः \cite{dp-msC}}पक्ष इति निर्देश्यः ? ॥ ३७ ॥” 
	  
	कीदृशः पुररर्थः पक्ष इति--अनेनशब्देन निर्देश्यो वक्तव्यः ? इत्याह-- “
	  
	स्वरूपेणैव स्वयमिष्टो\footnote{०मिष्टो निरा० \cite{dp-msB} \cite{dp-edP}}ऽनिराकृतः पक्ष इति\footnote{इति निर्देश्यः--\cite{dp-msC}} ॥ ३८ ॥” 
	  
	स्वरूपेणैवेति साध्यत्वेनैव । स्वयमिति वादिना । इष्ट इति--नोक्त एवापि त्विष्टोऽपीत्यर्थः । एवंभूतः सन् प्रत्यक्षादिभिः अनिराकृतो \footnote{०कृतोऽर्थो यः स \cite{dp-msB} \cite{dp-msC} \cite{dp-msD}}योऽर्थः स पक्ष इत्युच्यते । 
	  
	अथ यदि \footnote{अथ यदि न पक्षो \cite{dp-msA} \cite{dp-msB} \cite{dp-edP} \cite{dp-edH} \cite{dp-edE} \cite{dp-edN}}पक्षो न निर्देश्यः, कथमनिर्देश्यस्य लक्षणमुक्तम् ? न साधनवाक्यावयवत्वादस्य लक्षणमुक्तमपि त्वसाध्यं \footnote{किञ्चत्--\cite{dp-msB}}केचित् साध्यम्, साध्यं चासाध्यं \footnote{“केचित्” नास्ति--\cite{dp-msC} \cite{dp-msA} \cite{dp-edP} \cite{dp-edH} \cite{dp-edE} \cite{dp-edN}}केचित् प्रतिपन्नाः । तत् साध्यासाध्यविप्रतिपत्तिनिकारणार्थं पक्षलक्षणमुक्तम् ॥ 
	  
	स्वरूपेणेष्ट इत्यस्य विवरणम-- “
	  
	स्वरूपेणेति साध्यत्वेनेष्टः ॥ ३९ ॥” 
	  
	साध्यत्वेनेष्ट इति । पक्षस्य साध्यत्वान्नापरमस्ति रूपम् । अतः स्वरूपं साध्यत्वमिति ॥ 
	  
	एवशब्दं विवरितुमाह-- “
	  
	स्वरूपेणैवेति साध्यत्बेनैवेष्टो\footnote{०त्वेनेष्टो \cite{dp-msC} \cite{dp-msB} \cite{dp-edH} \cite{dp-edP} \cite{dp-edE} \cite{dp-edN}} न साधनत्वेनापि ॥ ४० ॥” 
	  
	स्वरूपेणैवेति । ननु चैवशब्दः केवल एव प्रत्यवमर्ष्टव्यस्तत्\footnote{तत्कथम्--\cite{dp-msB}} किमर्थं स्वरूपशब्देन” तेन तु तपस्विना बहूक्तं समञ्जसं । केवलं प्रतिपत्तुराश्वासेनैवोत्पादितेन किं प्रयोजनम् ? कथं चासौ सन्दिग्धार्थाभिधायिनः प्रतिज्ञावचनादास्थामुत्पादयतीति समीचीनं निरूपितम्! आस्था खलु इदमेव मन्येऽथेत्यभिसम्प्रत्ययः । सा कथं वचनमात्राज्जायेत ? जातौ वा साधनाद्यभिधानं न कथं वैयर्थ्यमश्नुवीतेत्यलं बहुना ॥
	\pend
      

	  \pstart अत्र सामर्थ्यात्स्वयं शब्दस्य वादिनेति विवृतिः कृता न तु स्वयंशब्दस्य वादिनेत्यर्थः । एतच्चानन्तरमेव दर्शयिष्यते ॥
	\pend
      \leavevmode\marginnote{\textenglish{177/dm}}“

	  \pstart सह प्रत्यवमृष्टः ? उच्यते । एवशब्दो निपातो द्योतकः । पदान्तराभिहितस्यार्थस्य विशेषं द्योतयति इति पदान्तरेण विशेष्यवाचिना सह निर्दिष्टः । न साधनत्वेनापीति । यत् साधनत्वेन निर्दिष्टं तत् साधनत्वेनेष्टम् । असिद्धत्वाच्च\footnote{०त्वात् साध्य० \cite{dp-msB}} साध्यत्वेनापीष्टम् । तस्य निवृत्त्यर्थ\footnote{०त्त्यर्थम्--\cite{dp-msD}} एवशब्दः ॥
	\pend
       

	  \pstart तदुदाहरति--
	\pend
       “

	  \pstart यथा शब्दस्यानित्यत्वे साध्ये चाक्षुषत्वं हेतुः, शब्देऽसिद्धत्वात् साध्यम् । न पुनस्तदिह साध्यत्वेनैवेष्टम्\footnote{साध्यत्वेनेष्टम् \cite{dp-msC} \cite{dp-msD} \cite{dp-edE}}, साधनत्वेनाभिधानात्\footnote{०त्वेनाप्यभिधानात् \cite{dp-msB} \cite{dp-edP} \cite{dp-edH} \cite{dp-edE} \cite{dp-edN}} ॥ ४१ ॥
	\pend
      ” 

	  \pstart \footnote{“यथेति” नास्ति \cite{dp-msA}}यथेति । शब्दस्यानित्यत्वे साध्ये चाक्षुषत्वं हेतुः शब्देऽसिद्धत्वात् साध्यम्--इत्यनेन साध्यत्वेनेष्टिमाह ।
	\pend
       

	  \pstart तद् इति चाक्षुषत्वम् । इहेति शब्दे । साध्यत्वेनैवेष्टम्--इति साध्यत्वेनेष्टिनियमाभावमाह । साधनत्वेनाभिधानाद् इति--यतः साधनत्वेनाभिहितम्, अतः साधनत्वेनापीष्टम् । न साध्यत्वेनैवेति ॥
	\pend
       

	  \pstart स्वयमित्यनेन स्वयंशब्दं व्याख्येयमुपक्षिप्य तस्यार्थमाह--
	\pend
       “

	  \pstart स्वयमिति वादिना ॥ ४२ ॥
	\pend
      ” 

	  \pstart वादिनेति । स्वयंशब्दो निपात आत्मन इति \footnote{नाशं स्वयमिच्छतीत्यादौ आत्मनो नाशमिच्छतीत्यर्थः--\cite{dp-msD-n}}षष्ठ्यन्तस्यात्मनेति च तृतीयान्तस्याथ\footnote{०स्यार्थेन युक्तः--\cite{dp-msB}} वर्त्तते । तदिह तृतीयान्तस्यात्मशब्दस्यार्थे वृत्तः स्वयंशब्दः । आत्मशब्दश्च सम्बन्धिशब्दः । वादी च प्रत्यासन्नः\footnote{प्रत्यासन्नभूतः यस्य \cite{dp-msA} \cite{dp-msB} \cite{dp-edP} \cite{dp-edH}} । ततो यस्य वादिन आत्मा तृतीयार्थयुक्तः\footnote{तृतीयार्थेन युक्तः \cite{dp-msC} \cite{dp-msD}} स एव\footnote{“एव” नास्ति--\cite{dp-msB}} तृतीयार्थयुक्तो निर्दिष्टो वादिनेति । न\footnote{ननु \cite{dp-msA} \cite{dp-msB} \cite{dp-edP} \cite{dp-edH}} तु स्वयंशब्दस्य वादिनेत्येष पर्यायः ॥
	\pend
      ”

	  \pstart पक्षस्यानुमेयस्य ॥
	\pend
      

	  \pstart स्वरूपशब्देनेति सहार्थे तृतीया ।
	\pend
      

	  \pstart \leavevmode\marginnote{\textenglish{64b/ms}} \textbf{उच्यत} इति सिद्धान्तवादी ।
	\pend
      

	  \pstart यत्साधनत्वेनेष्टं तत्कथं साध्यत्वेनापीष्टं भवतीत्याह--\textbf{असिद्ध\add{त्वा}दिति । चो} यस्मात् । साध्यत्वेनेष्टोऽपि यदा साधनत्वेनोक्तस्त दाऽपक्ष इत्येवमर्थ \textbf{एव}शब्द इति समुदायार्थः ॥
	\pend
      \leavevmode\marginnote{\textenglish{178/dm}}“

	  \pstart कः पुनरसौ वादीत्याह--
	\pend
       “

	  \pstart यस्तदा साधनमाह ॥ ४३ ॥
	\pend
      ” 

	  \pstart यस्तदा--इति वादकाले साधनमाह । अनेकवादिसम्भवेऽपि\footnote{०सम्भवे स्वयं० \cite{dp-msB} \cite{dp-msC} \cite{dp-msD}} स्वयंशब्दवाच्यस्य वादिनो विशेषणमेतत् ।
	\pend
       

	  \pstart यद्येवं\footnote{यद्येव--\cite{dp-msA} \cite{dp-msB} \cite{dp-edP} \cite{dp-edH}} वादिन इष्टः साध्यः--इत्युक्तम् । एतेन च किमुक्तेन ? अनेन\footnote{अनेन च \cite{dp-msC}} तदा वादकाले तेन वादिना स्वयं यो धर्मः साधयितुमिष्टः स एव साध्यो\footnote{साध्यो धर्मो नेतर इत्यु० \cite{dp-msC}} नेतरो \footnote{“धर्म” नास्ति \cite{dp-msB}}धर्म इत्युक्तं भवति । वादिनोऽनिष्टधर्मसाध्यत्वनिवर्त्तनमस्य वचनस्य फलमिति यावत् ॥
	\pend
       

	  \pstart अथ कस्मिन् सत्यन्यधर्मसाध्यत्वस्य\footnote{साध्यत्वसम्भ० \cite{dp-msA} \cite{dp-msC} \cite{dp-edP} \cite{dp-edH} \cite{dp-edE}} सम्भवो यन्निवृत्त्यर्थं \footnote{०त्त्यर्थं चेदं \cite{dp-msA} \cite{dp-msD} \cite{dp-edP} \cite{dp-edH} \cite{dp-edE} \cite{dp-edN} ०त्त्यर्थं चैतत्--\cite{dp-msB}}तद् वक्तव्यमित्याह--
	\pend
       “

	  \pstart एतेन यद्यपि क्वचिच्छास्त्रे स्थितः साधनमाह, तच्छास्त्रकारेण तस्मिन् धर्मिएयनेकधर्माभ्युपगमेऽपि यस्तदा तेन वादिना\footnote{तेन स्वयं वादिना धर्मः साध० \cite{dp-msC}} धर्मः स्वयं साधयितु-
	\pend
      ””

	  \pstart \textbf{तत्त}स्मादर्थद्वयवृत्तित्वात् । \textbf{इह} पक्षलक्षणे \textbf{स्वयंशब्दो} गृहीत इति शेषः । तृतीयाप्रतिपाद्योऽर्थोऽत्रैषणकर्त्तृत्वम् । \textbf{आत्मना} इष्ट इत्यत्र तृतीयायाः कर्त्तरि विधानात् ॥
	\pend
      

	  \pstart \textbf{अनेकवादिसम्भवेऽपि} शब्दगताकाशगुणत्वादिवादिभूयस्त्वेऽपि । वादित्वं च योग्यतया । न तु तदा स्वपरपक्षसिद्ध्यसिद्ध्यर्थवचनलक्षणवादप्रणेतारः । \textbf{विशेषणं} व्यवच्छेदक\textbf{मेतद् यस्तदासाधनमाहे}ति वचनम् ।
	\pend
      

	  \pstart \textbf{यद्येवमि}ति परः । अयं च निपातसमुदायोऽनिष्टापादनप्रारम्भे वर्त्तते । \textbf{इत्युक्तम}नेन वाक्येनेति शेषः ।
	\pend
      

	  \pstart उच्यतामेवं को दोष इत्याह--\textbf{एतेनेति । च}शब्दोऽपिशब्दस्यार्थे । शास्त्रकारेष्टमपि वादीष्टं भवति । तत्कोऽतिशयोऽनेन प्रतिपादित इति चोदयितुराशयः । \textbf{अनेने}ति सिद्धान्तवादी । \textbf{अनेन} यस्तदा साधनमाहेति विशेषणावच्छिन्नेन स्वयंशब्देन ।
	\pend
      

	  \pstart एतदुक्तं भवति । यच्छास्त्राभ्युपगमेनापि वादी क्वचित्साधनमभिधत्ते, तच्छास्त्रकारेण तत्र यावदिष्टं तावच्चेत्तस्य साध्यत्वेनेष्टं तदेष्टमित्येव कृतं स्यात्, न तु स्वयमिति । \textbf{नेतर इति} तच्छास्त्रकारेष्टोऽम्बरगुणत्वादिरिति बुद्धिस्थम् । \textbf{वादिन} इति आद्यस्यैव व्यक्तीकरणम् ॥
	\pend
      

	  \pstart अथेत्यामन्त्रणे ।
	\pend
      \leavevmode\marginnote{\textenglish{179/dm}}““

	  \pstart मिष्टः, स एव साध्यो नेतर इत्युक्तं भवति ॥ ४४ ॥
	\pend
      ” 

	  \pstart तच्छास्त्रकारेणेति । यच्छास्त्रं तेन वादिनाऽभ्युपगतं तच्छास्त्रकारेण तस्मिन् साध्यधर्मिणि अनेकस्य \footnote{अनेकधर्म० \cite{dp-msC}}धर्मस्याभ्युपगमे सति अन्यधर्मसाध्यत्वसम्भवः । तथाहि--शास्त्रं येनाभ्युपगतं \footnote{तस्मिन् सिद्धो \cite{dp-edE}}तत्सिद्धो धर्मः सर्व एव तेन साध्य इत्यस्ति विप्रतिपत्तिः । अनेनापास्यते । अनेकधर्माभ्युपगमेऽपि सति स एव साध्यो यो वादिन इष्टो नान्य इति ।
	\pend
       

	  \pstart ननु च शास्त्रानपेक्षं \footnote{तादात्म्यतदुत्पत्तेः--\cite{dp-msD-n}}वस्तुबलप्रवृत्तं लिङ्गम् । अतोऽनपेक्षणीयत्वान्न शास्त्रे स्थित्वा वादः कर्त्तव्यः । सत्यम् । आहोपुरुषिकया तु यद्यपि क्वचिच्छास्त्रे स्थित इति किञ्चिच्छास्त्रमभ्युपगतः साधनमाह, तथापि य एव तस्येष्टः स एव\footnote{स एव तस्य \cite{dp-msA} \cite{dp-edE}} साध्य\footnote{साध्यत इति--\cite{dp-msC}} इति ज्ञापनायेदमुक्तम् ॥
	\pend
       

	  \pstart इष्ट इतीष्टशब्दमुपक्षिप्य व्याचष्टे--
	\pend
       “

	  \pstart इष्ट इति यत्रार्थे विवादेन साधनमुपन्यस्तं तस्य सिद्धिमिच्छता सो \footnote{सोऽर्थोऽनु० \cite{dp-msC}}ऽनुक्तोऽपि वचनेन साध्यः ॥ ४५ ॥
	\pend
      ” 

	  \pstart यत्रार्थ आत्मनि विरुद्धो वादः प्रक्रान्तः--“नास्ति आत्मा”--इत्यात्मप्रतिषेधवाद आत्मसत्तावादविरुद्धः, विधिप्रतिषेधयोर्विरोधात् । तेन विवादेन हेतुना साधनमुपन्यस्तं तस्यात्मार्थस्य सिद्धिं निश्चयम् इच्छता वादिना सोऽर्थः साध्य इत्युक्तं भवति इष्टशब्देन । यत् तद् “इत्युक्तं भवति” इति ग्रहणमन्ते तदिहापेक्ष्य वाक्यं \footnote{वाक्यं परिसमा० \cite{dp-msA} \cite{dp-msB} \cite{dp-msC} \cite{dp-msD} \cite{dp-edP} \cite{dp-edH} \cite{dp-edE} \cite{dp-edN}}समापयितव्यम् ।
	\pend
       

	  \pstart यद्यपि परार्थानुमान उक्त एव साध्यो युक्तः, अनुक्तोपि \footnote{तुशब्दस्तथापीत्यर्थे--\cite{dp-msD-n}}तु वचनेन साध्यः, सामर्थ्योक्तत्वात् तस्य ॥
	\pend
      ”

	  \pstart इत्यस्ति विप्रतिपत्तिर्न्यायविरुद्धा प्रतिपत्तिः केषाञ्चित् । \textbf{अनेना}त्मविशेषणेनापास्यते । \textbf{अनेके}त्यादिनोपसंहरति । \textbf{अनेकधर्माभ्युपगमेऽपि} शास्त्रकारस्य तत्र धर्मिण्यनेकधर्मोपगमे सत्यपि । \textbf{वादिन} इत्यात्मन इति षष्ठ्यन्तस्यार्थे वृत्तं स्वयंशब्दमु\add{पा}दाय ।
	\pend
      

	  \pstart \textbf{आहोपुरुषिकयाऽभ्युपगत} इति कर्त्तरीयं निष्ठा ॥
	\pend
      

	  \pstart \textbf{विरुद्धः} परस्य चाभिप्रेतविपरीतार्थोपस्थापको वादः स्वपरपक्षयोः सिद्ध्यसिद्ध्यर्थं वचनम् । \textbf{प्रक्रान्तः} प्रवृत्तः । \textbf{इत्युक्तं भवती}ति नात्र श्रूयते तत्कथमेवं व्याख्यायत इत्याह—यत्तदिति । लोकोक्तिश्चैषा । इहेष्टपदविवरणे । \textbf{समापयितव्यं} सङ्गतार्थं कर्त्तव्यम् ।
	\pend
      \leavevmode\marginnote{\textenglish{180/dm}}“

	  \pstart कुत एतदित्याह--
	\pend
       “

	  \pstart तदधिकरणत्वाद्विवादस्य ॥ ४६ ॥
	\pend
      ” 

	  \pstart \footnote{तदित्यादि तदिति \cite{dp-msA} \cite{dp-msB} \cite{dp-msC} \cite{dp-msD} \cite{dp-edP} \cite{dp-edH} \cite{dp-edE} \cite{dp-edN}}तदिति सोऽर्थोऽधिकरणम् आश्रयो यस्य स तदधिकरणो विवादः । तस्य भावस्तत्त्वम् । तस्मादिति ।
	\pend
       

	  \pstart एतदुक्तं भवति--यस्माद्विवादं निराकर्त्तुमिच्छता वादिना साधनमुपन्यस्तं तस्माद् यद् अधिकरणं विवादस्य तदेव साध्यम् । यतो विरुद्धं वादमपनेतुं साधनमुपन्यस्तं तच्चेत् न साध्यं किमिदानीं \footnote{जगति नियतम्--\cite{dp-msB} \cite{dp-msC} \cite{dp-msD} \cite{dp-edP} \cite{dp-edH} \cite{dp-edE} \cite{dp-edN}}जातिनियतं किंचित् साध्यं स्यादिति ॥
	\pend
       

	  \pstart अनुक्तमपि परार्थानुमाने साध्यमिष्टम् । \footnote{०मिष्टमुदाहरति--\cite{dp-msB} साध्यं दृष्टमुदाह० \cite{dp-msC} \cite{dp-msD}}तदुदाहरति--
	\pend
       “

	  \pstart यथा परार्थाश्चक्षुरादयः संघातत्वाच्छयनासनाद्यङ्गवदिति । \footnote{०दिति । आत्मार्था इत्यनुत्था [[क्ता]] प्यात्मार्थता साध्यते तेन--\cite{dp-msC}}अत्रात्मार्था इत्यनुक्तावप्यात्मार्थता साध्या । \footnote{साध्या । अनेन \cite{dp-msB} \cite{dp-edP} \cite{dp-edH} \cite{dp-edE}}तेन नोक्तमात्रमेव साध्यम्--इत्युक्तं भवति ॥ ४७ ॥
	\pend
      ” 

	  \pstart परार्था इति । चक्षुरादिर्येषां श्रोत्रादीनां ते\footnote{“ते” नास्ति \cite{dp-msB}} चक्षुरादय इति धर्मी । परस्मायिमे परार्था इति साध्यं पारार्थ्यम् । सङ्घातत्वादिति हेतुः । व्याप्तिविषयप्रदर्शनं \footnote{“च” नास्ति--\cite{dp-msA} \cite{dp-msB} \cite{dp-msD} \cite{dp-edP} \cite{dp-edH} \cite{dp-edE} \cite{dp-edN}}च शयनासनाद्यङ्गवदिति । शयनमासनं च ते आदी\footnote{आदिर्यस्य \cite{dp-msD}} यस्य तच्छयनासनादि पुरुषोपभोगाङ्गं संघातरूपम् । तद्वदत्र \footnote{तद्वदत्र यत्प्रमाणे--\cite{dp-msA} \cite{dp-msB} \cite{dp-edP} \cite{dp-edH}}प्रमाणे यदप्यात्मार्थाश्चक्षुरादय इत्यात्मार्थता नोक्ता \footnote{नोक्ताप्या० \cite{dp-msB} अनुक्ताप्या--\cite{dp-msA} \cite{dp-msC} \cite{dp-msD} \cite{dp-edP} \cite{dp-edH} \cite{dp-edE} \cite{dp-edN}}अनुक्तावप्यात्मार्थता साध्या ।
	\pend
      ”

	  \pstart \textbf{अनुक्तोऽपि तु वचनेन । वचनेन} साक्षादभिधाव्यापारविषयत्वमनापादितोऽपि । \textbf{तुर्विशेषदीपने । साध्यः} साध्य एव । कुत इत्याह--\textbf{सामर्थ्योक्तत्वात्तस्य} बुद्धिस्थस्यात्मादेः ।
	\pend
      

	  \pstart एतदुक्तं भवति--परार्थानुमाने उक्तोऽर्थः साध्यः । उक्तश्च प्रकाशित उच्यते । प्रकाश्यमानता च साक्षादभिधाव्यापारविषयतया च सामर्थ्यगम्यतया च । उक्तम \footnote{उक्तता} तु प्रकाशितताख्या द्वयोरप्यविशिष्टेति ॥
	\pend
      

	  \pstart \textbf{कुत एतदि}ति सामर्थ्योक्तत्व\textbf{मिति} हेतो\textbf{राह वार्त्तिककारः ।}
	\pend
      \leavevmode\marginnote{\textenglish{181/dm}}“

	  \pstart तथा हि--सांख्येनोक्तम्--अस्ति आत्मा । तद्विरुद्धं बौद्धेनोक्तं--नास्त्यात्मेति । ततः सांख्येन स्ववादविरुद्धं बौद्धवाद हेतूकृत्य विरुद्धवादनिराकरणाय स्ववादप्रतिष्ठापनाय च साधनमुपन्यस्तम् । अतोऽनुक्तावप्यात्मार्थता\footnote{०नुक्ताप्या० \cite{dp-msA} \cite{dp-msB} \cite{dp-msC} \cite{dp-msD} \cite{dp-edP} \cite{dp-edH} \cite{dp-edE} \cite{dp-edN}} साध्या, \footnote{य आत्मप्रतिषेधवादोऽधिकरणं यस्य--\cite{dp-msD-n}}तदधिकरणत्वाद् विवादस्य ।\footnote{शयनादिषु \cite{dp-msA}} शयनासनादिषु हि पुरुषोपभोगाङ्गेष्वात्मार्थत्वेनान्वयो\footnote{०र्थत्बेन प्रसिद्धः \cite{dp-msA}} न प्रसिद्धः । सङ्घातत्वस्य \footnote{परार्थमा० \cite{dp-msA} \cite{dp-msB} \cite{dp-edP} \cite{dp-edH} परार्थत्वमा० \cite{dp-edN}}पारार्थ्यमात्रेण तु सिद्धः । ततः परार्था इत्युक्तम् ।
	\pend
       

	  \pstart चक्षुरादय इत्यत्रादिग्रहणाद्विज्ञानमपि\footnote{बौद्धानां मते परमाणुरूपं ज्ञानमतः सङ्घातरूपत्वम्--\cite{dp-msD-n}} परार्थं साधयितुमिष्टम् । विज्ञानाच्च पर आत्मैव स्यात् ।
	\pend
       

	  \pstart \footnote{“परस्य” नास्ति--\cite{dp-msB}}परस्यार्थकारि विज्ञानं सेत्स्यतीति सामर्थ्यादात्मार्थत्वं सिध्यति चक्षुरादीनामिति मत्वा परार्थग्रहणं कृतम् । तेनेष्ट\footnote{साध्यवच० \cite{dp-msA} \cite{dp-msB} \cite{dp-edP} \cite{dp-edH} E, \cite{dp-edN}} साध्यत्ववचनेन नोक्तमात्रम्, अपि तु प्रतिवादिनो विवादास्पदत्वाद् वादिनः\footnote{वादिना \cite{dp-edE}} साधयितुमिष्टम्-उक्तम्, अनुक्तं वा प्रकरणगम्यं साध्यमित्युक्तं भवति ॥
	\pend
      ”

	  \pstart यद्यपि तन्मूलो विवादस्तथापि अभिधाव्यापारविश्रामभूमिरेवार्थः साध्यः । न चात्मादिर्विवादाधिकरणम् । तथाभूतस्तत्कथं साध्य इत्याशङ् क्याह--\textbf{एतदुक्तं} भवतीति । \textbf{यदा}त्मादि \textbf{अधिकरणमा}स्पदं \textbf{विवादस्य}--अस्तीदं नास्तीदमित्यात्मकस्य परस्परविरुद्धस्य \textbf{वादस्य} । अत्रोपप\leavevmode\marginnote{\textenglish{65a/ms}} त्तिमाह--\textbf{यत} इति । विरुद्धं स्वपक्षप्रत्यनीकं नास्तीदमिति वादम् । \textbf{तद्} विरुद्धवादापनेतृहेतूपन्यासविषयं \textbf{चेद्} यदि \textbf{न साध्यं} न जिज्ञापयिषित\textbf{मिदानी}मेतस्मिन्नभ्युपगमे किं \textbf{जातिनियतं} साध्यत्वजातिनियतं जातिवशं \textbf{किञ्चिद्} वस्तु \textbf{साध्यं स्याद्} भवितुमर्हति । \textbf{क्षेपे किमः} प्रयोगान्न किञ्चिदित्यर्थोऽवतिष्ठते ॥
	\pend
      

	  \pstart \textbf{अनुक्तमपि} साक्षादभिधाव्यापाराविषयोऽपि । \textbf{संहा \footnote{धा}} तत्वादनेकरूपत्वात् । कालविशेषानपेक्षं चैतद् द्रष्टव्यम् । तेन क्रमेण युगपद् वा संहतं तदिति संहतरूपं विज्ञानमपि क्रमेणानेकरूपमतस्तत्रापि संहा \footnote{धा} तत्वं सिद्धिमिति तदप्यादिशब्देन सङ्गृहीतं धर्मि कर्त्तव्यम् । अत एवानन्तरम् \textbf{“अत्रादिशब्दाद् विज्ञानमपि”} इति वक्ष्यते । अन्यथा तु चक्षुरादीनां विज्ञानलक्षणपरार्थतासिद्धावा \footnote{व} पि नाभिप्रेतं \textbf{सांख्यस्य} सिद्ध्येत् । \textbf{अनुक्तावप्य}नभिधानेऽपि तस्येत्यर्थात् । \textbf{क्वचित्पुनरनुक्ताप्यात्मार्थते}ति पाठः । तत्रार्जवेनै \add{व} सम्बन्धः । साधनोपन्यासाश्रयत्वेन प्रकृतत्वात्तस्या इति भावः ।
	\pend
      

	  \pstart \textbf{तथा ही}त्या\textbf{दिनै}तदेव प्रतिपादयति । \textbf{हेतूकृत्य} निमित्तीकृत्य । \textbf{चः} फलसमुच्चये । \textbf{अनुक्तावपी}ति पूर्ववद्वाच्यम् । यद्ययं तस्याभिप्रायस्तदाऽऽत्मार्था इत्येव किं न ब्रवीतीत्याशङ्क्य येनाभिप्रायेणैवमवादीत्तं दर्शयितुमाह--\textbf{शयनेति । हि}र्यस्मादर्थे ।
	\pend
      \leavevmode\marginnote{\textenglish{182/dm}}“

	  \pstart अनिराकृत इति--एतल्लक्षणयोगेऽपि यः साधयितुमिष्टोऽप्यर्थः प्रत्यक्षानुमानप्रतीतिस्ववचनैर्निराक्रियते, न स पक्ष\footnote{पक्षः दर्श० \cite{dp-msC}} इति प्रदर्शनार्थम् ॥ ४८ ॥
	\pend
      ”“

	  \pstart अनिराकृत इति व्याख्येयम् । एतदिति--अनन्तरप्रक्रान्तं यत् पक्षलक्षणमुक्तं साध्यत्वेनेष्टेत्यादि\footnote{साध्यत्वेनेष्टत्वादि \cite{dp-msC} \cite{dp-msD}}--एतल्लक्षणेन योगेऽप्यर्थो न पक्ष इति प्रदर्शनार्थम् \footnote{प्रदर्शनाय \cite{dp-msA} \cite{dp-msC} \cite{dp-edP} \cite{dp-edH}}प्रतिपादनाय अनिराकृतग्रह्णं कृतम् ।
	\pend
       

	  \pstart कीदृशोऽर्थो न पक्षः साधयितुमिष्टोऽपीत्याह--यः साधयितुमिष्टोऽर्थः--प्रत्यक्षं चानुमानं च प्रतीतिश्च स्ववचनं च--\footnote{तैर्नि० \cite{dp-edE} \cite{dp-edN}}एतैर्निराक्रियते--विपरीतः साध्यते \footnote{साध्यते स न पक्ष \cite{dp-msB} \cite{dp-msD}}न स पक्ष इति ॥
	\pend
       “

	  \pstart तत्र प्रत्यक्षनिराकृतो यथा--अश्रावणः शब्द इति ॥ ४९ ॥
	\pend
      ” 

	  \pstart तत्रेति । तेषु चतुर्षु प्रत्यक्षादिनिराकृतेषु\footnote{०कृतेषु निरा० \cite{dp-msB}} प्रत्यक्षनिराकृतः कीदृशः ? यथेति । यथाऽयं प्रत्यक्षनिराकृतस्तथाऽन्येऽपि द्रटव्या इति यथाशब्दार्थः ।
	\pend
       

	  \pstart श्रवणेन ग्राह्यः श्रावणः । न श्रावणोऽश्रावणः । श्रोत्रेण न ग्राह्य इति प्रतिज्ञार्थः । श्रोत्राग्राह्यत्वं शब्दस्य प्रत्यक्षसिद्धेन श्रोत्र\footnote{श्रोत्रज्ञानग्राह्य०--\cite{dp-msD-n}} ग्राह्यत्वेन बाध्यते ॥
	\pend
      ”

	  \pstart ननु चक्षुरादीनां विज्ञानलक्षणपरार्थता सेत्स्यति । तत्कथमात्मार्थतासिद्धिरित्याह \textbf{चक्षुरादय इत्यत्रे}ति । तथापि कुतस्तत्सिद्धिरित्याह \textbf{विज्ञानाच्चे}ति । \textbf{चो} यस्मात् । अथ विज्ञानस्यापि धर्मित्वे कथमात्मार्थतासिद्धिरित्याशङ्क्य स्पष्टयितुमाह--\textbf{परस्ये}ति । \textbf{सामर्थ्यादात्मार्थत्वं सिद्ध्यति चक्षुरादीनामित्येवं मत्वा परार्थग्रहणं कृतं सांख्येने}ति प्रस्तावात्, अध्याहारे वा, तेषां विज्ञानार्थताया अपि सम्भाव्यत्वात् । तत् सिद्धैकं सामर्थ्यमित्याशङ्क्य \textbf{परस्ये}ति योज्यम् ।
	\pend
      

	  \pstart \textbf{अर्थकारि} प्रयोजनकारि \textbf{विज्ञानम}पीति द्रष्टव्यम् । \textbf{सेत्स्यती}ति ब्रुवतोऽयं भावः । \textbf{आदि}शब्दाद् विज्ञानस्यापि तथात्वे साध्ये विज्ञानं परार्थकारि सेत्स्यतीति । \textbf{इति}र्हेतौ । अनेन सामर्ध्यं चक्षुरादीनामात्मार्थतासिद्धौ दर्शितम् ।
	\pend
      

	  \pstart \textbf{तेने}त्यस्य मूलस्य व्याख्यान\textbf{मिष्टसाधन \footnote{साध्यत्व} वचनेनेति वादिनः साधयितु}मिच्छया विषयीकृतम् ।
	\pend
      

	  \pstart तच्च द्विविधमिति दर्शयन्नाह--\textbf{उक्त}मित्यादि । \textbf{उक्तं} साक्षादभिधाविषयीकृतम् । तद्विपरीत\textbf{मनुक्तम्} । कथं तर्हि तत्साध्यमित्याह--\textbf{प्रकरणे}ति । \textbf{प्रकरणे}न साध्योपन्यासाश्रयतया प्रकृतत्वेन \textbf{गम्यं} प्रकाश्यं साध्यं साध्यमेवे\textbf{त्युक्तं भवति} ॥
	\pend
      \leavevmode\marginnote{\textenglish{183/dm}}““

	  \pstart अनुमाननिराकृतो यथा--नित्यः शब्द इति ॥ ५० ॥
	\pend
      ” 

	  \pstart \footnote{श्रवणेन्द्रिय प्रभवज्ञानेनेत्यर्थः--\cite{dp-msD-n}}अनुमाननिराकृतो यथा\footnote{“यथा” नास्ति \cite{dp-msC} \cite{dp-msD} \cite{dp-edH} \cite{dp-edE} \cite{dp-edN}} नित्यः शब्द इति । शब्दस्य प्रतिज्ञातं नित्यत्वम् अनित्यत्वेनानुमानसिद्धेन निराक्रियते ॥
	\pend
       “

	  \pstart प्रतीतिनिराकृतो यथा--अचन्द्रः शशीति ॥ ५१ ॥
	\pend
      ” 

	  \pstart प्रतीत्या निराकृतः अचन्द्र इति चन्द्रशब्दवाच्यो न भवति शशीति प्रतिज्ञातार्थः ।\footnote{प्रतिज्ञार्थः--\cite{dp-msC} \cite{dp-msD}} अयं च प्रतीत्या निराकृतः । प्रतीतोऽर्थ उच्यते विकल्पविज्ञानविषयः । प्रतीतिः प्रतीतत्वं
	\pend
      ”

	  \pstart \textbf{एतल्लक्षणयोगेऽपी}त्यस्यार्थकथनमिद\textbf{मेतल्लक्षणेन योगेऽप्यर्थ} इति ।
	\pend
      

	  \pstart ननु च प्रत्यक्षादिभिर्निराक्रियतेऽपसार्यत इति किल मतम् । न चाक्षा\footnote{र्था}पसारणं प्रत्यक्षादिधर्मोऽपि तु वस्तुव्यवस्थापनमित्याह--\textbf{विपरीत} इति ॥
	\pend
      

	  \pstart श्रूयतेऽनेनेति श्र\textbf{वणं} श्रोत्रेन्द्रियं तेन \textbf{ग्राह्यः} उपलब्धः । तज्ज्ञानग्राह्यत्वाच्च तद्ग्राह्यत्वमुक्तम् । नञा समासमाह--\textbf{नेति} । समस्तस्यार्थमाह--श्रोत्रेणेति । अयमस्याशयः—अश्रावणः शब्दः--श्रोत्रजे\leavevmode\marginnote{\textenglish{65b/ms}}न ज्ञानेन नानुभूयत इति यः प्रतिजानीते तस्य सा प्रतिज्ञा श्रवणेन्द्रियजेन प्रत्यक्षेण शब्दालम्बनेन बाध्यते शब्दगताऽप्रतिभासनविपरीतस्य तत्प्रतिभासनस्य तेनोपस्थापनादिति ।
	\pend
      

	  \pstart एतेन तन्निराकृतम्, \textbf{यदुद्द्योतकरेणो}क्तम्\footnote{न्यायवार्त्तिकं द्रष्टव्यम्--१. १. ३३. पृ० ११३.}--“अश्रावणः शब्द इति प्रत्यक्षविरुद्धोदाहरणं वर्णयति । न प्रत्यक्षस्य विषयो ज्ञातो नानुमानस्य । किं कारणम् ? इन्द्रियवृत्तीनामतीन्द्रियत्वात् । श्रावणत्वञ्चेन्द्रियवृत्तिः । सा कथं प्रत्यक्षा भविष्यति ? तस्मादनुमानविरुद्धस्योदाहरणमिदम् । अनुष्णो वह्निः कृतकत्वादिति प्रत्यक्षविरुद्धस्येति ।”
	\pend
      

	  \pstart न ह्मश्रावणशब्देन शब्दाख्ये विषये ज्ञानोत्पत्तौ श्रोत्रेन्द्रियस्य वृत्तेरभावोऽभिमतो यस्य वादिनस्तं प्रति शब्दविषयश्रवणेन्द्रियवृत्तेः प्रत्यक्षविरुद्धत्वात्प्रत्यक्षनिराकृतमिदमाचार्येणोक्तम्, येनोच्यतेऽतीन्द्रियेन्द्रियवृत्तिः कथं प्रत्यक्षा येनेदमुदाहरणं संगच्छेतेति । किन्तर्हि ? यः कश्चिद् व्यामोहमाहात्म्याद् यदेतच्छ्रोत्रग्राह्यं रूपमद्वयं तन्नास्तीति प्रतीजानीते तं वादिविशेषं प्रतीति कथं न प्रत्यक्षविरुद्धोदाहरणमिदमिति ।
	\pend
      

	  \pstart एतदुक्तं भवति । शब्दो नास्तीत्येवमपि ब्रुवाणस्यास्ति प्रत्यक्षबाधा । केवलं विषयो निषेधोऽनेकमार्गः । शब्दो नास्ति व्यापितया नित्यतयेत्यादि । तत्रासति श्रावणशब्दे सर्वस्यैव निषेधे प्रत्यक्षबाधा शङ्क्येत । श्रावणशब्देन तु श्रुतिमात्रग्राह्यमेव यद्रूपं तन्निषेधे प्रत्यक्षबाधा न तु सामान्यधर्मनिषेध इति ख्याप्यत इति ॥
	\pend
      \leavevmode\marginnote{\textenglish{184/dm}}“

	  \pstart विकल्पविज्ञानविषयत्वमुच्यते । तेन विकल्पज्ञानेन\footnote{विकल्पविज्ञानविषयत्वेन प्र० \cite{dp-msB} \cite{dp-edP} \cite{dp-edH} \cite{dp-edE} विकल्पज्ञानविषयत्वेन \cite{dp-msC} \cite{dp-msD} विकल्पविज्ञानेन \cite{dp-msA}} प्रतीतिरूपेण शशिनश्चन्द्रशब्दवाच्यत्वं सिद्धमेव । तथा हि--यद्विकल्प\footnote{यद्विकल्पज्ञान० \cite{dp-msB} \cite{dp-edP} \cite{dp-edH} \cite{dp-edN} यज्ज्ञानग्राह्यं \cite{dp-msA}} विज्ञानग्राह्यं तच्छब्दाकारसंसर्गयोग्यम् । यच्छब्दाकारसंसर्गयोग्यं तत् साङ्केतिकेन शब्देन वक्तुं शक्यम् । अतः प्रतीतिरूपेण विकल्पविज्ञानविषयत्वेन सिद्धं चन्द्रशब्दवाच्यत्वमचन्द्रत्वस्य बाधकम् । स्वभावहेतुश्च प्रतीतिः । यस्माद्विकल्पविषयत्वमात्रानुबन्धिनी साङ्केतिकशब्दवाच्यता, ततः स्वभावहेतुसिद्धं चन्द्रशब्दवाच्यत्वमवाच्यत्वस्य बाधकं द्रष्टव्यम् ॥
	\pend
      ”

	  \pstart \textbf{प्रतीत्या} विकल्पविज्ञानरूपेण विषयिणा विषयस्य निर्देशात् । एतदेव दर्शयति \textbf{प्रतीत} इति । \textbf{प्रतीत उच्यते} व्यवह्रियते, साक्षात्कृतस्याप्यजाताध्यवसायस्य तथाव्यवहाराभावादिति भावः । विषयिणा विषयगतो धर्म उक्त इति स्फुटयति \textbf{प्रतीतिः} प्रतीतत्वरूपेण \textbf{प्रतीतत्वमि}ति । तेन विकल्पज्ञानेनेति विकल्पविज्ञानविषयत्वेनेति ज्ञेयं \textbf{प्रतीतिरूपेण । तथा} हीत्यनेनैतदेवोपपादयति ।
	\pend
      

	  \pstart भवतु शब्दाकारसंसर्गयोग्यत्वम् । तच्छब्दवाच्यता तु कथमित्याह--\textbf{यदिति । अत} इत्यादिना प्रतीतेर्बाधकत्वं दर्शयति । \textbf{अचन्द्रत्वस्या}चन्द्रशब्दवाच्यत्वस्य । किंसाधनसिद्धमिदं चन्द्रशब्दवाच्यत्वमित्याह--\textbf{स्वभावे}ति । स्वभावहेतुलक्षणं योजयन्नाह--\textbf{यस्मादि}ति । \textbf{तत} इत्यादिनोपसंहारः ।
	\pend
      

	  \pstart एवं तु प्रयोगो द्रष्टव्यः--योऽर्थो विकल्पविज्ञानविषयः स साङ्केतिकेन शब्देन वक्तुं शक्यः । यथा शाखादिमानर्थो वृक्षशब्देन । विकल्पविज्ञानविषयश्च शशीति ।
	\pend
      

	  \pstart इह केनचिच्छब्देन कस्यचिदभिधातुमशक्यत्वं वास्तवे प्रतिनियतार्थशब्दसम्बन्धे सति स्यात् । स चान्यत्र प्रतिषिद्धः । पारिशेष्याज्ज्ञानात्मन्यारूढस्यार्थस्य शब्दसम्बन्धः कर्त्तुं कस्य शक्यो यस्तेन शब्दाकारेण सह नैकस्मिन् ज्ञानेन संसृज्यते । अनियतार्थं च विज्ञानमिति तदारूढोऽर्थस्तदभिधानाकारसंसर्गयोग्य एव । तस्मात्तेन शब्देनाभिधातुमशक्यत्वमतदाकारसंसर्गयोग्यत्वेन व्याप्तम् । व्या\leavevmode\marginnote{\textenglish{66a/ms}}पकविरुद्धं च तदभिधानाकारसंसर्गयोग्यत्वम् । तेन च विकल्पविज्ञानविषयत्वं व्याप्तम् । तदेवं विकल्पविज्ञानविषयत्वं तद्व्यापकविरुद्धव्याप्तत्वात्तेनापि विरुद्ध्यते । ततश्च तद्विरुद्धेन शक्यत्वेन व्याप्त इति स्वभावहेतुः । तस्माद् विकल्पविज्ञानविषयत्वमेव प्रतीतिः प्रसिद्धिर्व्यवहारश्चोच्यते । अनया प्रतीत्या यत्साधितं शशिनश्चन्द्रशब्दवाच्यत्वं तत्स्वविरुद्धस्य तदनभिधेयत्वस्य बाधकं भवति । तेन प्रतीतेर्विकल्पविज्ञानलक्षणाया जातो धर्म इष्टशब्दाभिधेयत्वलक्षणस्तेनानभिधेयत्वस्य बाधनात्प्रतीतिबाधोच्यत इति परमार्थः ।
	\pend
      

	  \pstart यद्वा प्रतीतेस्तथारूपाया जात एवेष्टशब्दाभिधेयत्वलक्षणो धर्मः प्रतीतिशब्देनोक्तः,  \leavevmode\marginnote{\textenglish{185/dm}} ““
	  
	स्ववचननिराकृतो यथा-नानुमानं प्रमाणम् ॥ ५२ ॥” 
	  
	स्ववचनं प्रतिज्ञार्थस्यात्मीयो वाचकः शब्दः । तेन निराकृतः प्रतिज्ञार्थो न साध्यः । यथा नानुमानं प्रमाणम्--इत्यत्र\footnote{प्रमाणम् । अत्र \cite{dp-msA} \cite{dp-msB} \cite{dp-msD} \cite{dp-edP} \cite{dp-edH} \cite{dp-edE} \cite{dp-edN}} अनुमानस्य प्रामाण्यनिषेधः प्रतिज्ञार्थः । स\footnote{स चानुमा० \cite{dp-msC}} नानुमानं प्रमाणम्--इत्यनेन स्ववाचकेन वाक्येन बाध्यते । वाक्यं हि एतत् प्रयुज्यमानं वक्तुः \footnote{शाब्दस्य प्रत्य० \cite{dp-msA} \cite{dp-msB} \cite{dp-msC} \cite{dp-msD} \cite{dp-edP} \cite{dp-edH} \cite{dp-edE} \cite{dp-edN}}शाब्दप्रत्ययस्य सदर्थत्वमिष्टं सूचयति । तथाहि-मद्वाक्याद् योऽर्थसम्प्रत्ययस्तवोत्पद्यते सोऽसत्यार्थ इति दर्शयन् वाक्यमेव नोच्चारयेद्वक्ता, वचनार्थश्चेदसत्यः परेण ज्ञातव्यो वचनमपार्थकम् । योऽपि हि सर्वं मिथ्या ब्रवीमीति\footnote{ब्रवीति वक्ति \cite{dp-msB}} वक्ति सोऽप्यस्य वाक्यस्य सत्यार्थत्वमादर्शयन्नेव वाक्यमुच्चारयति । यद्येतद्वाक्यं सत्यार्थमादर्शितम्, एवं वाक्यान्तराण्यात्मीयान्यसत्यार्थानि\footnote{०न्यसत्यानि \cite{dp-msA}} दर्शितानि भवन्ति । \footnote{“एतदेव” इत्यादि “भवन्ति” इत्यन्तं सूत्रत्वेन \cite{dp-edH} प्रतौ मुद्रितम् । किन्तु नास्त्येतत् सूत्रम्--सं०}एतदेव तु यद्यसत्यार्थम्, अन्यान्यसत्यार्भानि न दर्शितानि भवन्ति । ततश्च न किञ्चिदुच्चारणस्य फलमिति नोच्चारयेत् । तस्माद्वाक्यप्रभवं वाक्यार्थालम्बनं विज्ञानं सत्यार्थं दर्शयन्नेव वक्ता \footnote{अनुमानप्रामाण्यनिषेधलक्षणम्--\cite{dp-msD-n}}वाक्यमुच्चारयति । तथा\footnote{यथा \cite{dp-msA}} च सति बाह्यवस्तुनान्तरीयकं शब्दं दर्शयता शब्दजं विज्ञानं सत्यार्थं दर्शयितव्यम् । ततो बाह्यार्थकार्याच्छब्दादुत्पन्नं विज्ञानं सत्यार्थमादर्शयता\footnote{आदर्शयिता--\cite{dp-msA}} कार्यलिङ्गजमनुमानं प्रमाणं शाब्दं दर्शितं भवति ।” प्रतीतिमात्रादेव सिद्धो योऽर्थः स इह बाधक इति दर्शयितुम् । एतच्च स्वभावहेतुत्वं कल्पितमिष्टम्, न वास्तवम्, शशिनो विकल्पविज्ञानविषयत्वस्याऽऽध्यवसानिकत्वात् । अन्यथाऽनुमाननिराकृतान्नास्य पृथङ्निर्देशः स्यादिति ॥
	\pend
      

	  \pstart स्वशब्देनात्मवचनेन प्रकृतत्वात्साध्यस्यात्मा गृह्यत इति अभिप्रायेणाह--प्रतिज्ञार्थस्यात्मीय इति । तेन \textbf{निराकृत} इति तदुपस्थापितेनानुमानप्रामाण्येन निराकृत इति द्रष्टव्यम् । कथं निराक्रियत इत्याह--\textbf{वाक्यमि}ति । यस्माद् \textbf{एतद् वाक्यं प्रयुज्यमानं} सद् \textbf{वक्तुः शाब्दप्रत्ययस्य} शब्दप्रभवस्य ज्ञानस्य \textbf{सदर्थत्वं} सत्यार्थत्व\textbf{मिष्टं सूचयति} प्रकाशयति ।
	\pend
      

	  \pstart प्रयुक्तमपि कथं तथाकरोतीत्याह--\textbf{तथा ही}ति । \textbf{नोच्चारये}दुच्चारयितुं नार्हति, अपार्थकत्वादिति बुद्धिस्थम् । \textbf{वचने}त्यादिना त्वेतदेव व्यनक्ति-न सर्वं वचनं प्रयुज्यमानं तथाकारि यथा सर्वं मिथ्या ब्रवीमीति वचनमित्याह--\textbf{योऽपी}ति । हिर्यस्मात् । कथं तथादर्शयन्नुच्चारयतीत्याह--यद्येतदित्यादि । भवतु वाक्यप्रभवं वाक्यार्थालम्बनं ज्ञानं सत्यार्थम्, तथापि कथमनुमानप्रामाण्यं वचनेनोपस्थाप्यते येनानुमानप्रामाण्यप्रतिषेधस्तद्वचनोपस्थापितानुमानप्रामाण्येन  \leavevmode\marginnote{\textenglish{186/dm}} “
	  
	तस्मात् “नानुमानं प्रमाणम्”--इति ब्रुवता शाब्दस्य प्रत्ययस्यासन्नर्थो\footnote{०स्यासन् ग्राह्य उक्तो \cite{dp-msA} \cite{dp-edP} \cite{dp-edH} \cite{dp-edN}} ग्राह्य उक्तः । असदर्थत्वमेव ह्यप्रामाण्यमुच्यते, नान्यत् । शब्दोच्चारणसामर्थ्याच्चार्थाविनाभावी स्वशब्दो दर्शितः । तथा च \footnote{सन्नार्थो--\cite{dp-edE}}सन्नर्थो दर्शितः । \footnote{“नानुमानं प्रमाणं” इत्यस्माच्छब्दाद्योऽर्थो बुध्यते तेन जनितो “नानुमान” इत्यादिकः शब्दः न प्रत्येष्यति नास्तिको व्यभिचारात्--\cite{dp-msD-n}}ततः \footnote{अध्यारोपित०--\cite{dp-msD-n}}कल्पितादर्थकार्याच्छब्दाच्छाब्द\footnote{०च्छब्दप्र० \cite{dp-msA}} प्रत्ययार्थस्यानुमितं सत्त्वं प्रतिज्ञायमानमसत्त्वं प्रतिबध्नाति । 
	  
	तदेवं स्ववचनानुमितेन सत्त्वेनासत्त्वं बाध्यमानं\footnote{वाच्यमानं \cite{dp-msA} \cite{dp-msB} \cite{dp-edP} \cite{dp-edH} \cite{dp-edE} \cite{dp-edN}} स्ववचनेन बाधितमुक्तमित्ययमत्रार्थः । 
	  
	अन्ये त्वाहुः--अभिप्रायकार्याच्छब्दाज्जातं ज्ञानमभिप्रायालम्बनम् । सदर्थमिच्छतः शब्दप्रयोगः । तेनाप्रामाण्यं प्रतिज्ञातं बाध्यत इति ।” निराक्रियमाणः स्ववचननिराकृतो भवतीत्याशङ्क्याह--तथा च सतीति--शब्दज्ञानस्य सत्यार्थत्वप्रतिपादनाभिप्रायेण वाक्योच्चारणप्रकारे सति । \textbf{बाह्यवस्तुनान्तरीयकं} तदविनाभाविनं दर्शयता सत्यार्थं दर्शयितव्यं दर्शयितुं युज्यते शक्यत इत्यर्थः । यत एवं ततस्तस्मात् । \textbf{बाह्यार्थकार्यादि}ति सति भेदे तदुत्पत्त्यैव नान्तरीयकत्वसम्भवादिति भावः । \textbf{शाब्दमि}ति प्रकृतत्वाज्ज्ञानं \textbf{कार्यलिङ्गजं} शब्दरूपकार्यलिङ्गजमिति हेतुभावेन विशेषण\textbf{मनुमानं प्रमाणं दर्शितं प्रकाशितं भवति} ।
	\pend
      

	  \pstart \textbf{तस्मादित्या}दिनोपसंहरति ।
	\pend
      

	  \pstart नन्वनुमानस्याऽसन् ग्राह्य इति युज्यते वक्तुम् । तत्किं शाब्दस्य प्रत्ययस्येत्युवतमिति चेत् । नैष दोषः । शाब्दस्यापि प्रत्ययस्योक्तेन न्यायेनानुमानत्वात् । सर्वानुमानप्रामाण्यप्रतिषेधे चास्यापि प्रतिषिद्धत्वात् । अस्यैव चानुमानस्यानुमानप्रामाण्यप्रतिषेधलक्षणप्रतिज्ञार्थबाधकत्वादुपन्यासो युक्तरूपः ।
	\pend
      

	  \pstart यद्येवमप्रामाण्यमुक्तम् तच्च बाध्य\leavevmode\marginnote{\textenglish{66b/ms}}त इति वक्तुं युज्यते । तत्किमेवमुक्तमित्याह\textbf{असदर्थत्वमि}ति । \textbf{ही}ति यस्मात् । आस्तामसन् ग्राह्योऽभिहितः किमत इत्याह--\textbf{शब्देति} । \textbf{चो} हेताववधारणे वा । शब्दोऽप्यतस्तथा दर्शितस्तथापि किमायातमित्याह--\textbf{तथा चे}ति शब्दस्य बाह्यार्थाविनाभाविप्रदर्शनप्रकारे सति \textbf{सन्नर्थो}ऽध्यवसेयो \textbf{दर्शितः} । सदर्थत्वं प्रामाण्यलक्षणं दर्शितमिति यावत् । तस्य सदर्थत्वं प्रदर्श्यतां बाधा तु कथमित्याह \textbf{तत} इति । \textbf{शाब्दप्रत्ययस्य} योऽर्थो ज्ञाप्योऽनुमानप्रामाण्यलक्षणस्तस्यानेनैव शब्दलिङ्गेनानुमाने\textbf{नानुमितं सत्त्वं} कर्त्तृ \textbf{प्रतिज्ञायमान}मसदर्थमप्रामाण्यलक्षणं कर्मभूतं \textbf{प्रतिबध्नाति} । निराकरोतीति वक्तव्ये \textbf{प्रतिबध्नाती}ति \textbf{ब्रुवाणः} परमार्थतोऽस्यावास्तवत्वान्न बाधा किन्त्वेतदुपस्थापितेतरयोः परस्परप्रतिबन्ध इति सूचयति ।
	\pend
      \leavevmode\marginnote{\textenglish{187/dm}}“

	  \pstart तदयुक्तम् । यत इह प्रतीतेः स्वभावहेतुत्वम्, स्ववचनस्य च कार्यहेतुत्वं कल्पितमिष्टम् । न वास्तवम्\footnote{वास्तवमिति \cite{dp-msC} \cite{dp-msD}} अभिप्रायकार्यत्वं च वास्तवमेव शब्दस्य । ततस्तदिह न गृह्यते ।
	\pend
       

	  \pstart किञ्च । यथा--अनुमानमनिच्छन्\footnote{अग्न्यव्य० \cite{dp-msD}} वह्न्यव्यभिचारित्वं धूमस्य न प्रत्येति, तथा शब्दस्याप्यभिप्रायाव्यभिचारित्वं न प्रत्येष्यति । बाह्यवस्तुप्रत्यायनाय च शब्दः प्रयुज्यते । तन्न शब्दस्याभिप्रायाविनाभावित्वाभ्युपगमपूर्वकः शब्दप्रयोगः । \footnote{“अपि च” इति यस्मादर्थेऽव्ययम्--\cite{dp-msD-n} । “अपि च” इत्यारभ्य “शब्दप्रयोगः” इत्यन्तः पाठो नास्ति--\cite{dp-msB}}अपि च, न स्वाभिप्राय निवेदनाय शब्द उच्चार्यते, अपि तु बाह्य \footnote{बाह्यवस्तुसत्त्व \cite{dp-msA} \cite{dp-msB} \cite{dp-msC} \cite{dp-msD} \cite{dp-edP} \cite{dp-edH} \cite{dp-edE} \cite{dp-edN}}सत्त्वप्रतिपादनाय, तस्माद् बाह्यवस्त्वविनाभावित्वाभ्युपगमपूर्वकः शब्दप्रयोगः । ततः पूर्वकमेव\footnote{पूर्वमेव \cite{dp-edE}} व्याख्यातमनवद्यम्\footnote{व्याख्यानम० \cite{dp-msA} \cite{dp-msB} \cite{dp-msC} \cite{dp-msD} \cite{dp-edP} \cite{dp-edH} \cite{dp-edE} \cite{dp-edN}} ॥
	\pend
       “

	  \pstart इति चत्वारः पक्षाभासा निराकृता भवन्ति ॥ ५३ ॥
	\pend
      ” 

	  \pstart एवं\footnote{एवं सति--\cite{dp-msA} \cite{dp-edP} \cite{dp-edN}}च सति--अनिराकृतग्रहणेनानन्तरोक्ताश्चत्वारः पक्षवदाभासन्त इति पक्षाभासा निरस्ता भवन्ति ॥
	\pend
       

	  \pstart सम्प्रति पक्षलक्षणपदानि येषां व्यवच्छेदकानि तेषां व्यवछेदेन यादृशः \footnote{पक्षो--\cite{dp-msC}}पक्षार्थो लभ्यते तं दर्शयितुं व्यवच्छेद्यान् संक्षिप्य दर्शयति--
	\pend
      ”

	  \pstart नन्वनुमानप्रामाण्यमसत्त्वं बाधते न तु स्ववचनं तत्कथं स्ववचननिराकृतोदाहरणमिदमुक्तमित्याशङ्क्य स्ववचननिराकृत इत्यत्र यादृशोऽर्थो विवक्षितस्तं दर्शयति \textbf{तदेवमि}ति । अत्रेति स्ववचननिराकृते । यथानुमानं न प्रमाणमित्यत्र ।
	\pend
      

	  \pstart स्वाभिप्रायेणैव व्याख्याय \textbf{अन्ये त्वि}त्यादिना पूर्वव्याख्यानं दूषयितुमाह । तुना स्वव्याख्यानाद् वैसदृश्यमाह ।
	\pend
      

	  \pstart \textbf{न वास्तवमि}ति ब्रुवतो यदीदं वास्तवमनुमानं स्यात्तदानुमाननिराकृतोदाहरणान्न पृथगुच्येतेति । \textbf{किञ्चे}ति वक्तव्यान्तराभ्युच्चये । \textbf{बाह्यवस्तुप्रत्यायनाय चे}ति \textbf{च}कारोऽपि वक्तव्यान्तरसमुच्चये । यत एवं \textbf{ततस्मा}त् । स्वस्योच्चारयितुमभिप्रायो वक्तुकामता, तदव्यभिचारित्वा\textbf{भ्युपगमः पूर्वो} यस्य तथा न भवति । पूर्वं \textbf{बाह्यवस्तुप्रत्यायनाये}त्यनेन सामर्थ्यान्न स्वाभिप्रायप्रत्यायनायेत्युक्तमपि स्फुटं दर्शयितुं \textbf{तन्न शब्दस्याभिप्रायाविनाभावित्वाभ्युपगमपूर्वकः शब्दप्रयोग} इत्यनेन सामर्थ्याद् बाह्मवस्त्वविनाभावित्वाभ्युपगमपूर्वक इति दर्शितमपि साक्षाद् दर्शयितु\textbf{मपि चेत्या}दिनोपक्रमते । \textbf{अपि चे}ति पूर्वोक्तस्य वक्तव्यान्तर द्योतकस्य स्पष्टीकरणम् । यत्राप्यसत्त्वप्रतिपादनाय शब्दः प्रयुज्यते तत्रापि विवक्षितसत्त्वविविक्तस्यान्यसत्त्वस्य प्रतिपादनाद् \textbf{बाह्यसत्त्वप्रतिपादनाये}त्युक्तम् । यद्वा \textbf{सत्त्व}ग्रहणस्योप  \leavevmode\marginnote{\textenglish{188/dm}} “
	  
	\footnote{“एवं” नास्ति \cite{dp-msB} \cite{dp-edP} \cite{dp-edH}}एवं सिद्धस्य, असिद्धस्यापि साधनत्वेनाभिमतस्य, स्वयं वादिना तदा साधयितुमनिष्टस्य, उक्तमात्रस्य निराकृतस्य च विपर्ययेण साध्यः । तेनैव स्वरूपेणाभिमतो वादिन इष्टोऽनिराकृतः पक्ष इति पक्षलक्षणमनवद्यं\footnote{लक्षणमवद्यं \cite{dp-msB} \cite{dp-edP}} दर्शितं भवति ॥ ५४ ॥” “
	  
	एवम्--इत्यनन्तरोक्तक्रमेण\footnote{०रोक्तेन क्रमेण \cite{dp-msB} \cite{dp-msC} \cite{dp-msD}} । सिद्धस्य विपर्ययेण विपरीतत्वेन हेतुना साध्यो द्रष्टव्यः । यस्मादर्थात् सिद्धोऽर्थो विपरीतः, स साध्य इत्यर्थः । सिद्धश्च विपरीतोऽसिद्धस्य । तस्माद् असिद्धः साध्यः । असिद्धोऽपि न सर्वोऽपि तु साधनत्वेनोक्तस्यासिद्धस्यापि विपर्ययेण । स्वर्य वादिना साधयितुमनिष्टस्य असिद्धस्य विपर्ययेण । तथा उक्तमात्रस्य असिद्धस्यापि विपर्ययेण । तथा निराकृतस्यासिद्धस्यापि विपर्ययेण साध्यः । 
	  
	यश्चायं पञ्चभिर्व्यवच्छेद्यै रहितोऽर्थोऽसिद्धो\footnote{“असिद्धो” इत्यादिपदानन्तरं \cite{dp-msA} प्रतौ \cite{dp-msD} प्रतौ च संख्याङ्काः दत्ता वर्त्तन्ते--सं०}ऽसाधनं वादिनः स्वयं साधयितुमिष्ट उक्तोऽनुक्तो वा प्रमाणैरनिराकृतः साध्यः, स एवासौ स्वरूपेणैव स्वयमिष्टोऽनिराकृत एतैः पदैरुक्त इत्यर्थः । यश्चायं साध्यः स पक्ष \footnote{“इति” नास्ति \cite{dp-msA} \cite{dp-msB} \cite{dp-msD} \cite{dp-edP} \cite{dp-edH} \cite{dp-edE} \cite{dp-edN}}इति उच्यते । इतिशब्द एवमर्थे । एवं पक्षलक्षणमनवद्यमिति । अविद्यमानमवद्यं दोषो यस्य तदनवद्यम् । दर्शितं कथितम् । 
	  
	त्रिरूपलिङ्गाख्यानं परिसमापय्य\footnote{समाप्य \cite{dp-edE}} प्रसङ्गागतं च पक्षलक्षणमभिधाय हेत्वाभासान् वक्तुकामस्तेषां प्रस्तावं रचयति त्रिरूपेत्यादिना-- “
	  
	त्रिरूपलिङ्गाख्यानं परार्थानुमानमित्युक्तम् । तत्र त्रयाणां रूपाणामेकस्यापि रूपस्यानुक्तौ साधनाभासः ॥ ५५ ॥” 
	  
	एतदुक्तं भवति--त्रिरूपलिङ्गं\footnote{लिङ्गाख्यानं वक्तु० \cite{dp-msA} \cite{dp-msB} \cite{dp-edP} \cite{dp-edH} \cite{dp-edE} \cite{dp-edN}} वक्तुकामेन स्फुंट तद्वक्तव्यम् । एवं च तत् स्फुटमुक्तं” लक्षणत्वाद् बाह्यसत्त्वप्रतिपादनायेत्यपि द्रष्टव्यम् । सर्वथा न स्वाभिप्रायस्य सत्त्वमसत्त्वं वा प्रतिपादयितुं शब्दप्रयोग इत्यने \add{न} हेतुम् । यत एवं \textbf{तत}स्तस्मात्\textbf{पूर्वकमेव} यन्मया व्याख्यात\textbf{मनवद्यम}पगतदोषम् ॥
	\pend
      

	  \pstart \textbf{इति चत्वारः पक्षाभासा निराकृता भवन्ती}ति मूलं व्याचक्षाण आह--\textbf{एवमि}ति । \textbf{निरस्ता भवन्ति} पक्षत्वेनेति प्रस्तावात् ॥
	\pend
      

	  \pstart \textbf{पक्ष इत्युच्यते} व्यक्तीक्रियत इति व्युत्पत्त्येति भावः ॥
	\pend
      

	  \pstart ननु त्रिरूपलिङ्गाख्यानं प्रकृतमुक्तमेव । तत् किं हेत्वाभासाख्यानमप्रकृतं क्रियत  \leavevmode\marginnote{\textenglish{189/dm}} “
	  
	भवति यदि तच्च, तत्प्रतिरूपकं\footnote{प्रतिरूपं बोध्यते \cite{dp-msA} \cite{dp-msB} \cite{dp-edP} \cite{dp-edH} लिङ्गाभासम्--\cite{dp-msD-n}} चोच्यते । हेयज्ञाने\footnote{हेयज्ञाते \cite{dp-edE}} हि तद्विविक्तमुपादेयं सुज्ञातं भवतीति । त्रिरूपलिङ्गाख्यानं \footnote{परार्थानु--\cite{dp-msA} \cite{dp-msB} \cite{dp-edP} \cite{dp-edH} \cite{dp-edE} \cite{dp-edN}}“परार्थमनुमानम्” \cref{nb.3.1} इति प्राग् उक्तम् । 
	  
	तत्रेति तस्मिन् सति । त्रिरूपलिङ्गाख्याने परार्थानुमाने\footnote{परार्थेऽनुमाने--\cite{dp-msC}} सतीत्यर्थः । त्रयाणां रूपाणां मध्य एकस्याप्यनुक्तौ । अपिशब्दाद् द्वयोरपि । साधनस्य आभासः सदृशं साधनस्य, न साधनमित्यर्थः । त्रयाणां रूपाणां न्यूनता नाम साधनदोषः ॥ “
	  
	उक्तावप्यसिद्धौ सन्देहे वा प्रतिपाद्यप्रतिपादकयोः ॥ ५६ ॥” 
	  
	न केवलमनुक्तावुक्तावप्यसिद्धौ सन्देहे वा । कस्येत्याह--प्रतिपाद्यस्य प्रतिवादिनः, प्रतिपादकस्य च वादिनो हेत्वाभासः ॥ 
	  
	अथ \footnote{कस्य रूप० \cite{dp-msA} \cite{dp-edP} \cite{dp-edH} \cite{dp-edE} \cite{dp-edN} हेत्वाभासोऽप्येकस्य रूप० \cite{dp-msC}}कस्यैकस्यारूपस्यासिद्धौ सन्देहे\footnote{सन्देहे वाक्यं सं० \cite{dp-msA}} वा किंसंज्ञको हेत्वाभास इत्याह--” “
	  
	एकस्य रूपस्य धर्मिसम्बन्धस्यासिद्धौ सन्देहे \footnote{सन्देहे चासि० \cite{dp-msB} \cite{dp-msC} \cite{dp-msD} \cite{dp-edP} \cite{dp-edH} \cite{dp-edE}}वाऽसिद्धो हेत्वाभासः ॥ ५७ ॥” इत्याशङ्क्याह--\textbf{एतदुक्तं भवती}ति । त्रिरूपलिङ्गाख्यानस्य स्फुटाभिधानार्थत्वं चोपलक्षणं तेन विप्रतिपत्तिनिराकरणार्थं चेति द्रष्टव्यम्, सन्दिग्धविपक्षव्यावृत्तिकत्वादोष \footnote{दिषु} विप्रतिपत्तिदर्शनात् ।
	\pend
      

	  \pstart सदृ\leavevmode\marginnote{\textenglish{67a/ms}} शं साधनस्येत्यर्थकथनमेतद् । व्युत्पत्तिस्त्वाभासन\textbf{माभासः} प्रतिभासः, साधनस्येवाभासः प्रतिभासोऽस्येति तथा । \textbf{न्यूनता} ऊनत्वमपरिपूर्णता । कस्येत्याकाङ्क्षायामाह—\textbf{त्रयाणामि}ति । यस्मादेकस्य द्वयोर्वाऽनुक्तौ त्रीणि परिपूर्णं प्रतिपादितानि न भवन्ति तस्मात्त्रयाणां न्यूनतेत्यर्थः ।
	\pend
      

	  \pstart ननु हीनाङ्गत्वं न साधनदोषः । विद्यमानेऽपि हि रूपत्रये द्वयोरेकस्य वा वक्त्राऽनभिधाने सति न्यनतायाः सम्भवात् । तत्कथं वक्तृदोषः साधनदोष इत्युच्यत इति चेत् । सत्यम्; केवलं नात्र साधनशब्देन लिङ्गमभिप्रेतम् । किं तर्हि ? तत्प्रतिपादकं वाक्यं तस्य चापरिपूर्णता दोषो भवत्येव । वक्तृदोषस्तु निमित्त\add{म}परिपूर्णताया इति किमवद्यम् ?
	\pend
      

	  \pstart एवमुपलक्षणार्थत्वादस्य यथाऽन्यतमेनापि रूपेण हीनत्वान्न्यूनता साधनदोषस्तथा हेतूदाहरणादप्याधिक्यं साधनदोषो \textbf{वार्त्तिक}कारस्याभिप्रेतो द्रष्टव्यः, उभयत्रापि त्वसाधनाङ्गवचनाद् वादिना निग्रहो विवक्षित इत्यपीति ॥
	\pend
      

	  \pstart \textbf{प्रतिपाद्यस्ये}त्यादिना प्रतिपाद्यप्रतिपादकशब्दयोरेवार्थं व्याचष्टे । आचार्यस्य तु प्रतिपाद्यस्य प्रतिपादकस्य च प्रतिपाद्यप्रतिपादकयोश्चेति व्यस्तसमस्तनिर्देशोऽभिप्रेतः ।
	\pend
      \leavevmode\marginnote{\textenglish{190/dm}}“

	  \pstart एकस्य रूपस्येति । धर्मिणा सह सम्बन्धः धर्मिसम्बन्धः । धर्मिणि सत्त्वं हेतोः । तस्य असिद्धौ सन्देहे वा असिद्धसंज्ञाको हेत्वाभासः । असिद्धत्वादेव च धर्मिण्यप्रतिपत्तिहेतुः । न साध्यस्य, न विरुद्धस्य, न संशयस्य हेतुरपि त्वप्रतिपत्तिहेतुः । न कस्यचिदतः प्रतिपत्तिरिति कृत्वा । अयं चार्थोऽसिद्धसंज्ञाकरणादेव प्रतिपत्तव्यः ॥
	\pend
       

	  \pstart उदाहरणमाह--
	\pend
       “

	  \pstart यथा--अनित्यः शब्द इति साध्ये चाक्षुषत्वमुभयासिद्धम् ॥ ५८ ॥
	\pend
      ” 

	  \pstart यथेत्यादि । \footnote{नित्यः \cite{dp-msB}}अनित्यः शब्द इत्यनित्यत्वविशिष्टे शब्दे साध्ये चाक्षुषत्वं चक्षुर्ग्राह्यत्वं शब्दे \footnote{द्वयोर्द्वयोरपि \cite{dp-msB}}द्वयोरपि वादिप्रतिवादिनोरसिद्धम् ॥
	\pend
       “

	  \pstart चेतनास्तरव इति साध्ये सर्वत्वगपहरणे मरणं\footnote{मरणादिति प्रति० \cite{dp-msC}} प्रतिवाद्यसिद्धम्, विज्ञानेन्द्रियायुर्निरोधलक्षणस्य\footnote{०स्यानेन मरणस्याभ्यु० \cite{dp-msC}} मरणस्यानेनाभ्युपगमात्, तस्य च तरुष्वसम्भवात् ॥ ५९ ॥
	\pend
      ” 

	  \pstart चेतनास्तरव इति तरूणां चैतन्ये साध्ये । सर्वा त्वक् सर्वत्वक् । तस्या अपहरणे सति मरणं दिगम्बरैरुपन्यस्तं प्रतिवादिनो बौद्धस्यासिद्धम् ।
	\pend
       

	  \pstart कस्मादसिद्धमित्याह--विज्ञानं चेन्द्रियं चायुश्चेति द्वन्द्वः\footnote{चायुश्च रूपा० \cite{dp-msA} \cite{dp-edP} \cite{dp-edH} चायुश्च तत्र \cite{dp-msB} \cite{dp-edN}} । तत्र विज्ञानं\footnote{विज्ञानचक्षु--\cite{dp-msB}} चक्षुरादिजनितम्\footnote{जयित--\cite{dp-msB} चक्षुरादिविज्ञानं रूपा० \cite{dp-edE} चक्षुरादिज्ञानं रूपा० \cite{dp-msC}} । रूपादिविज्ञानोत्पत्त्या यदनुमितं \footnote{कार्यान्त० \cite{dp-msB} \cite{dp-edP}}कायान्तर्भूतं चक्षुर्गोलकादिस्थितं\footnote{स्थितरूपम्--\cite{dp-msA} \cite{dp-msB} \cite{dp-edP} \cite{dp-edH} \cite{dp-edN}} रूपं
	\pend
      ”

	  \pstart अयं हेत्वाभासः कीदृशीं प्रतिपत्तिं प्रसूत इत्याह--\textbf{असिद्धत्वादेवेति} । तुशब्दार्थश्चकारः । कभमेतत्प्राप्यत इत्याह--\textbf{अयञ्चे}ति । \textbf{चो} यस्मादर्थे । कस्याश्चिदपि प्रतिपत्तेहेतुतया न सिद्ध इति \textbf{असिद्ध} उक्त इत्यभिप्रायः ॥
	\pend
      

	  \pstart \textbf{द्वयोरपि वादिप्रतिवादिनोरसिद्धम}निश्चितम् । अनेन व्यधिकरणासिद्धोऽप्युक्तो द्रष्टव्यः । यथा राज्ञोऽयं प्रासादः, काकस्य कार्ष्ण्यादिति । तस्यापि कार्ष्ण्यस्य प्रासादे धर्मिणि उभयोरसिद्धत्वात्, केवलं तत्रासिद्धोऽप्यर्थधर्मिगतत्वेनोपादानात् तथा व्यपदिश्यते ।
	\pend
      

	  \pstart ननु व्यधिकरणमपि लिङ्गं गमकं दृष्टम् । यथा प्रत्यग्रशरावदर्शनं भ्रान्तिम \add{त्}प्रत्यग्रशरावत्वस्य भ्रान्तिमच्चक्रवत्त्वे । अस्य च हेतुत्वाद् । देश एव हि धर्मी अविदूरकुलालसम्बन्धित्वं साध्यम् तस्य च धर्मिणः प्रत्यग्रशरावसम्बन्धित्वं भ्रान्तिमच्चक्रसम्बन्धित्वञ्च धर्मो भवत्येव । न त्वत्र कुलालस्य धर्मिणः सद्भावः साध्यते, येनैव तदुच्येतेति ।
	\pend
      

	  \pstart तथा, विशेषणासिद्धविशेष्यासिद्धावप्यनेनैवोक्तौ द्रष्टव्यौ । यथाऽनित्यः शब्दोऽनभि \leavevmode\marginnote{\textenglish{191/dm}} “
	  
	तदिन्द्रियम् । आयुरिति लोके प्राणा \footnote{उच्यते--\cite{dp-msB}}उच्यन्ते । न चागमसिद्धमिह युज्यते वक्तुम् । अतः \footnote{प्रमाणस्व० \cite{dp-edH}}प्राणस्वभावमायुरिह । तेषां निरोधो निवृत्तिः । स लक्षणं तत्त्वं यस्य तत् तथोक्तम् । तथाभूतस्य मरणस्य अनेन बौद्धेन प्रतिज्ञातत्वात् । 
	  
	यदि नामैवं तथापि कथमसिद्धमित्याह--तस्य च विज्ञानादिनिरोधात्मकस्य \footnote{तरुष्वभावात्--\cite{dp-msB}}तरुष्वसम्भवात् । सत्तापूर्वको निरोधः । ततश्च यो विज्ञाननिरोधं तरुष्विच्छेत् स कथं विज्ञानं नेच्छेत् । तस्माद् विज्ञानानिष्टेर्निरोधोऽपि नेष्टस्तरुषु । 
	  
	ननु च शोषोऽपि मरणमुच्यते । स च तरुषु सिद्धः । सत्यम् । केवलं विज्ञानसत्तया\footnote{सत्ताया \cite{dp-msA}} व्याप्तं यत् मरणं तदिह हेतुः । विज्ञाननिरोधश्च तत्सत्तया व्याप्तः, न शोषमात्रम् । ततो\footnote{तत्र \cite{dp-msA}} यन्मरणं\footnote{मरणहेतुः--\cite{dp-msA} \cite{dp-msB} \cite{dp-edP} \cite{dp-edH} \cite{dp-edN}} हेतुस्तत् तरुष्वसिद्धम् । यत्तु\footnote{यच्च \cite{dp-msD}} सिद्धं शोषात्मकं तदहेतुः । 
	  
	दिगम्बरस्तु साध्येन \footnote{विज्ञानेन्द्रियायुर्निरोधलक्षणम्--\cite{dp-msD-n}}व्याप्तमव्याप्तं\footnote{शोषलक्षणम् ।} वा मरणमविविच्य मरणमात्रं हेतुमाह । तदस्य वादिनो हेतुभूतं\footnote{हेतुज्ञात्ततं \cite{dp-msA} विज्ञानेन्द्रियायुर्निरोधलक्षणम्--\cite{dp-msD-n}} मरणं न ज्ञातम् । अज्ञानात् सिद्धं शोषरूपम्, शोषरूपस्य मरणस्य तरुषु दर्शनात् । प्रतिवादिनस्तु ज्ञातमतोऽसिद्धम् । यदा तु वादिनोऽपि ज्ञातं तदा वादिनोप्यसिद्धं स्यादिति न्यायः ॥” धेयत्वे सति प्रमेयत्वात् । अनित्यः शब्दः प्रमेयत्वे सति अनभिधेयत्वादिति विशेषणविशिष्टस्य रूपस्य तत्र धर्मिणि द्वयोरपि वादिप्रतिवादिनोरसिद्धत्वात् । केवलं तत्रासिद्धो विशेषणविशिष्टतयोपात्तस्य रूपस्य तथाऽसिद्धेस्तथा तथा व्यपदिश्यत इति ॥
	\pend
      

	  \pstart \textbf{चेतना} इति । चेतयन्त इति \textbf{चेतनाः} । दिश एवाम्बरं येषामिति व्युत्पत्त्या \textbf{दिगम्बराः} क्षपणका उच्यन्ते । तैः किम्प्रमाणसमधिगतमिन्द्रियमित्याह--\textbf{रूपादी}ति । सत्स्वन्येषु कारणेष्वव्यापृते चक्षुरादौ रूपादिज्ञानमनुत्पद्यमानं स्वोत्पत्तौ कारणा\leavevmode\marginnote{\textenglish{67b/ms}}न्तरमपेक्षणीयं सूचयति । प्रणिहिते तु चक्षुरादौ जायमानं तत्रस्थं तत् किमपि कारणमस्तीति ख्यापयति । अत एवाह--\textbf{कायान्तर्भूतं} कायाश्रितम् । सामान्येनोक्तं काया\textbf{श्रि}तत्वम् । विशिष्टज्ञानाभिप्रायेणोक्तं विशिष्टाश्रयाश्रितत्वं दर्शयति \textbf{चक्षुरि}ति । \textbf{आदि}शब्देन रसनादिपरिग्रहः । प्रसन्नार्थादेरपि सद्भावान्न गोलकादिरेवेन्द्रियमिति भावः । किमायुरित्याह--\textbf{आयुरिति} आयुःशब्देनेत्यर्थः । \textbf{लोके} व्यवहर्त्तरि जने प्राणोऽन्तःशरीरे रसमलधातूनां प्रेरणादिहेतुरेकः सन् क्रियाभेदोद \footnote{भेदाद} पानादिसंज्ञां लभत इति तत्तदवस्थाविवक्षया \textbf{प्राणा} इति बहुवचनम् ।
	\pend
      

	  \pstart नन्वागमे जीवितेन्द्रियमायुरित्युक्तम् । तत्किमेवं व्याख्यायत इत्याह--न चेति । \textbf{चो}ऽवधारणे हेतौ वा । यत एव\textbf{मतः} कारणात् । \textbf{इह} प्रमाणसिद्धवस्तूपदर्शनप्रस्तावे तथा  \leavevmode\marginnote{\textenglish{192/dm}} “
	  
	अचेतनाः सुखादय इति साध्य उत्पत्तिमत्वम् अनित्यत्वं\footnote{अनित्यं \cite{dp-msB} \cite{dp-edP} \cite{dp-edH} \cite{dp-edE} \cite{dp-edN}} वा सांख्यस्य स्वयं वादिनोऽसिद्धम् ॥ ६० ॥” “
	  
	अचेतनाः सुखादय इति--सुखमादिर्येषां दुःखादीनां ते सुखादयः । तेषामचैतन्ये साध्ये उत्पत्तिमत्त्वम्, अनित्यत्वं वा लिङ्गमुपन्यस्तम् । य उत्पत्तिमन्तोऽनित्या वा ते न चेतनाः । यथा रूपादयः । तथा चोत्पत्तिमन्तोऽनित्या वा सुखादयस्तस्मादचेतनाः । चैतन्यं तु पुरुषस्य \footnote{स्वं रूपम्--\cite{dp-msA} \cite{dp-msB} \cite{dp-edP} \cite{dp-edH} \cite{dp-edN}}स्वरूपम् । अत्र चोत्पत्तिमत्त्वमनित्यत्वं वा पर्यायेण हेतुर्न युगपत् । तच्च द्वयमपि सांख्यस्य वादिनो न सिद्धम् । परार्थो\footnote{परार्थादि हे० \cite{dp-msB}} हि हेतूपन्यासः । तेन यः परस्य सिद्धः स हेतुर्वक्तव्यः । परस्य चासत उत्पाद उत्पत्तिमत्त्वम्, सतश्च निरन्वयो विनाशोऽनित्यत्वं सिद्धम् । तादृशं च द्वयमपि सांख्यस्यासिद्धम् । इहाप्यनित्यत्वोत्पत्तिमत्त्वसाधनाद् वादिनोऽसिद्धम् । यदि त्वनित्यत्वोत्पत्तिमत्त्वयोः \footnote{प्रमाणं \cite{dp-msA} \cite{dp-msC} \cite{dp-edP} \cite{dp-edH} \cite{dp-edE} \cite{dp-edN}}प्रामाण्यं वादिनो \footnote{विज्ञानं \cite{dp-msC}}ज्ञातं स्याद् \footnote{“तदा” नास्ति \cite{dp-msA} \cite{dp-edP} \cite{dp-edH} \cite{dp-edE} \cite{dp-edN}}तदा वादिनोऽपि सिद्धं स्यात् । ततः प्रमाणापरिज्ञानादिदं वादिनोऽसिद्धम् ॥ 
	  
	संदिग्धासिद्धं दर्शयितुमाह-- “
	  
	तथा स्वयं तदाश्रयणस्य वा संदेहेऽसिद्धः ॥ ६१ ॥” 
	  
	स्वयमिति हेतोरात्मनः सन्देहेऽसिद्धः । तदाश्रयणस्य \footnote{०स्य चेति \cite{dp-msA} \cite{dp-msB} \cite{dp-msC} \cite{dp-msD} \cite{dp-edP} \cite{dp-edH} \cite{dp-edE}}वेति--तस्य हेतोराश्रयणम्—आश्रीयतेऽस्मिन् हेतुरित्याश्रयणं हेतोर्व्यतिरिक्त आश्रयभूतः\footnote{आश्रयभूतसा० \cite{dp-edE}} साध्यधर्मी कथ्यते । तत्र हि” तस्यैव रूपस्य । \textbf{विज्ञानसत्तया व्याप्तं यदि}ति यस्मिन् मरणेऽवश्यं प्रागासीद् विज्ञानम्, तद् विज्ञानसत्तया व्याप्तमुक्तम् । तच्च श्वासोष्मपरिस्पन्दादिविगमलक्षणम् । \textbf{दिगम्बरस्यापि कथं} सिद्धमित्याह--\textbf{अज्ञानादिति} । चैतन्याव्यभिचारिणो मरणस्याज्ञानात् । अनेनैतदाह—यदि तेन साध्यव्याप्तं मरणं मरणशब्दमात्रसमतां बिभ्रतः शोषमात्राद् भेदेन विवेचितं स्यात् केवलमज्ञानात्तत्सिद्धमुच्यत इति ।
	\pend
      

	  \pstart एतदेवाह--\textbf{यदा त्वि}ति । एवमुत्तरत्रापि द्रष्टव्यम् ।
	\pend
      

	  \pstart \textbf{सुखम}नुकूलवेदनीयम् । आदिशब्दादिच्छाद्वेषादिपरिग्रहः । पुरुषस्य सांख्यपरिकल्पितस्यात्मनः । “पुरुषश्चेतयते बुद्धिरध्यवस्यतीति” सिद्धान्तात् । \textbf{सांख्यस्य} संख्यया पञ्चविंशतितत्त्वानीत्यनया व्यवहरतीति \textbf{सांख्यो} योगरूढिश्चैषा, \textbf{कपिल} एव तथोच्यते ।
	\pend
      

	  \pstart ननूत्पत्तिमत्त्वं कृतकत्वम् वा स्वसिद्धमेव तेनोपन्यसनीयमुपन्यस्तं च । तत्कथं वाद्यसिद्धतेत्याह--\textbf{परार्थो} हीति । हिर्यस्मात् । \textbf{परार्थः} परप्रतिपत्तिप्रयोजनः । \textbf{निरन्वयः} सर्वथोच्छेदः ।  \leavevmode\marginnote{\textenglish{193/dm}} “
	  
	हेतुर्वर्त्तमानो गमकत्वेनाश्रीयते । तस्याश्रयणस्य सन्देहे सन्दिग्धः ॥ 
	  
	\footnote{स्वात्मना \cite{dp-msA} \cite{dp-msB} \cite{dp-msC} \cite{dp-msD} \cite{dp-edP} \cite{dp-edE} \cite{dp-edH} \cite{dp-edN}}आत्मना सन्दिह्यमानमुदाहर्त्तुमाह-- “
	  
	यथा बाष्पादिभावेन संदिह्यमानो भूतसङ्घातोऽग्निसिद्धावुपदिश्यमानः \footnote{“संदिग्धासिद्धः” नास्ति \cite{dp-msC}}संदिग्धासिद्धः ॥ ६२ ॥” 
	  
	यथेति । बाष्प आदिर्यस्य स बाष्पादिः । तद्भावेन बाष्पादित्वेन संदिह्यमानो भूतसंङ्घात इति भूतानां पृथिव्यादीनां संङ्घातः समूहः । अग्निसिद्धौ--अग्निसिद्ध्यर्थम् उपादीयमानोऽसिद्धः । 
	  
	एतदुक्तं भवति--यदा धूमोऽपि बाष्पादित्वेन संदिग्धो भवति तदाऽसिद्धः, गमकरूपानिश्चयात् । धूमतया निश्चितो \footnote{वह्निकार्यत्वाद्ग० \cite{dp-msC}}वह्निजन्यत्वाद् गमकः । यदा तु संदिग्धस्तदा न गमक इत्यसिद्धताख्यो दोषः ॥ 
	  
	आश्रयणासिद्धमुदाहरति-- “
	  
	यथेह निकुञ्जे\footnote{निगुञ्जे--\cite{dp-msC}} मयूरः केकायितादिति ॥ ६३ ॥” 
	  
	यथेति । इह निकुञ्ज इति धर्मी । पर्वतोपरिभागेन तिर्यङ्निर्गतेन प्रच्छादितो भूमागो निकुञ्जः । मयूर इति साध्यम् । केकायितादिति हेतुः । केकायितं--मयूरध्वनिः ॥” न तु विनष्टस्यापि सत्त्वरजस्तमोरूपेणानुगम इष्टः । न केवलं पूर्वमेवेत्यपिशब्दः । \textbf{अनित्यत्वोत्पत्तिमत्त्व}योर्यत्साधनं प्रमाणं तस्य \footnote{स्याऽ} ज्ञानादनिश्चयात् यदीत्यादिनैतदेव द्रढयति ॥
	\pend
      

	  \pstart \textbf{स्वय}मि\textbf{त्यात्मन} इति षष्ठ्यन्तस्यानुवर्त्तते । हेतोश्च प्रकृतत्वाद् \textbf{हेतो}रिति विवृणोति । आश्रीयते साधनत्वेनोपादीयते \textbf{अस्मिन्निति} ॥
	\pend
      

	  \pstart यस्यात्मनः सन्देहः स आत्मना सन्दिह्यमानो भवतीत्यभिप्रायेणाह--\textbf{आत्मना सन्दिह्यमानमि}ति । \textbf{आदि}शब्देन नीहारादिपरिग्रहः ।
	\pend
      

	  \pstart ननु यद्यसौ परमार्थतो धूमस्तदा सन्देहेऽपि किं न गमक इत्याह--\textbf{एतदुक्तं भवतीति} । किं तद् गमकं रूपं येनानिश्चित इत्याह--\textbf{वह्नी}ति । \textbf{यदा त्वि}त्यादिनोक्तमेव स्पष्टयति । \textbf{तदा न गमक} इति ब्रुवतश्चायमाशयो वह्निजन्यत्वस्यैव गमकत्वनिबन्धनस्य तदाऽनिश्चितत्वात् ।
	\pend
      

	  \pstart एतेन सन्दिग्धविशेषणासिद्धः सन्दिग्धविशेष्यासिद्धश्चोक्तो द्रष्टव्यः \leavevmode\marginnote{\textenglish{68a/ms}} यथा षड्जादिसत्त्वसन्देहे मयूरशब्दोऽयं षड्जादिमत्त्वे सति अवर्णात्मकत्वात् । अवर्णात्मकत्वे सति  \leavevmode\marginnote{\textenglish{194/dm}} “
	  
	\footnote{कथमाश्र० \cite{dp-msA} \cite{dp-msB} \cite{dp-edP} \cite{dp-edH} \cite{dp-edE}}कथमयमाश्रयणासिद्ध इत्याह-- “
	  
	तदापातदेशविभ्रमे ॥ ६४ ॥” 
	  
	तदापात\footnote{तदघात--\cite{dp-msA}} इति । तस्य केकायितस्यापात \footnote{आगमः \cite{dp-msC}}आगमनं तस्य देशः स उच्यते यस्माद वेशादागच्छति केकायितम् । तस्य विभ्रमे व्यामोहे सत्ययमाश्रयणासिद्धः । निरन्तरेषु वहुषु निकुञ्जेषु सत्सु यदा केकायितापातनिकुञ्जे\footnote{केकायितापातविभ्रमः \cite{dp-msA} \cite{dp-edP} \cite{dp-edH}} विभ्रमः--किमस्मान्निकुञ्जात् केकायितमागतम् । आहोस्विदन्यस्मादिति\footnote{आहोस्विदस्मादिति--\cite{dp-msA} \cite{dp-edP} \cite{dp-edH} \cite{dp-edN}}, \footnote{तादाश्रय० \cite{dp-msA} \cite{dp-msB} \cite{dp-edP} \cite{dp-edH} \cite{dp-edE} \cite{dp-edN}}तदायमाश्रयणासिद्ध इति ॥ 
	  
	धर्मिणोऽसिद्धावप्यसिद्धत्वमुदाहरति-- “
	  
	धर्म्यसिद्धावप्यसिद्धः--यथा सर्वगत आत्मेति साध्ये सर्वत्रोपलभ्यमानगुणत्वम् ॥ ६५ ॥” 
	  
	यथेति । सर्वस्मिन् गतः स्थितः सर्वगतो व्यापीति यावत् । व्यापित्व आत्मनः साध्ये सर्वत्रोपलभ्यमानगुणत्वं लिङ्गम् । सर्वत्र देश उपलभ्यमानाः सुखदुःखेच्छाद्वेषादयो गुणा यस्यात्मनस्तस्य भावस्तत्त्वम् । न गुणा गुणिनमन्तरेण वर्त्तन्ते । गुणानां गुणिनि समवायात् । निष्क्रियश्चात्मा । ततश्च यदि व्यापी न भवेत् कथं दाक्षिणापथ उपलब्धाः सुखादयो मध्यदेश उपलभ्येरन् । तस्मात् सर्वगत आत्मा । 
	  
	तदिह बौद्धस्यात्मैव न सिद्धः, किमुत सर्वत्रोपलभ्यमानगुणत्वं सिध्येत् तस्येत्यसिद्धौ\footnote{घर्मिति [[णि]] हेतोः सम्बन्धस्य सत्त्वस्येत्यसिद्धौ--\cite{dp-msD-n}} हेत्वाभासः । पूर्वमाश्रयणसंदेहेन धर्मिणि संदेह उक्तः । संप्रति त्वसिद्धो धर्म्युक्त इत्यननयोर्विशेषः ।” षड्जादिमत्त्वादिति । उभयत्रापि विशेषणविशिष्टस्य रूपस्य वादिप्रतिवादिनोर्द्वयोरप्यनिश्चितत्वात्केवलं विशेषणसन्देहेन च विशेषणविशिष्टेन रूपेणासिद्ध इति तथा व्यपदिश्यत इति ॥
	\pend
      

	  \pstart \textbf{पर्वतोपरिभागेन तिर्यग्निर्ग}तेनेति च भूभाग इति चोपलक्षणं द्रष्टव्यम् । न तु तथाविध एव निकुञ्जः, पर्वतगह्वरदेशस्यैव निकुञ्जशब्दाभिलप्य वात् ॥
	\pend
      

	  \pstart \textbf{यस्माद् देशादागच्छती}ति वचनव्यक्त्या चोत्पन्नः शब्दश्चतुर्दिवक शब्दसन्तानं जनयति, स च जलतरङ्गन्यायेन श्रोत्रदेशमागतो गृहीत इति दर्शयति ॥
	\pend
      

	  \pstart \textbf{द्वेषादी}त्य\textbf{त्रादि}ग्रहणेन बुद्धिप्रयत्नादीनां ग्रहणम् । “सामान्यवान् गुणः संयोगविभागयोरनपेक्षो न कारणम्” इति गुणलक्षणयोगाद् \textbf{गुणाः । समवाया}त्समवेतत्वात् । प्रतिषिद्धानां  \leavevmode\marginnote{\textenglish{195/dm}} “
	  
	तदेवमेकस्य\footnote{सर्वेष्वपि असिद्धेषु एकं रूपं पक्षधर्मत्वमसिद्धम्--\cite{dp-msD-n}} रूपस्य \footnote{धर्मिबद्ध० \cite{dp-msA} \cite{dp-edP} \cite{dp-msB} \cite{dp-edH}}धर्मिसम्बद्धस्यासिद्धावसिद्धो हेत्वाभासः ॥ “
	  
	तथैकस्य रूपस्यासपक्षेऽसत्त्वस्यासिद्धावनैकान्तिको हेत्वाभासः ॥ ६६ ॥” 
	  
	\footnote{तथा पर० \cite{dp-msA} \cite{dp-msB} \cite{dp-edP} \cite{dp-edH}}तथाऽपरस्यैकस्य रूपस्य--\footnote{विपक्षे--\cite{dp-msD-n}}असपक्षेऽसत्त्वा\textbf{ख्य}स्यासिद्धावनैकान्तिको हेत्वाभासः । 
	  
	एकोऽन्त एकान्तो निश्चयः । स प्रयोजनमस्येत्यैकान्तिकः\footnote{०मस्यैकान्तिकः \cite{dp-msC} \cite{dp-msD}} । नैकान्तिकोऽनैकान्तिकः । यस्मान्न साध्यस्य न विपर्ययस्य निश्चयोऽपि तु तद्विपरीतः संशयः । साध्येतरयोः संशयहेतुरनैकान्तिक उक्तः ॥” च न समवाय इति । \textbf{निष्क्रियत्वं} च क्रियाया मूर्त्तद्रव्यवृत्तित्वात्, आत्मनश्चामर्त्तत्वादिति सिद्धान्तस्थितेः ।
	\pend
      

	  \pstart \textbf{किमुते}ति निपातः किम्पुनरित्यस्यार्थे वर्त्तते ।
	\pend
      

	  \pstart आश्रयणासिद्धधर्म्यसिद्धयोः कियान् भेद इत्याशङ्क्य भेदमुपपादयन्नाह--पूर्वमिति । अयमर्थः--पूर्वं परमार्थतो विद्यमानोऽपि हेत्वाधाररूपतया सन्देहादनिश्चित इति तद् धर्म्यसिद्ध उक्तः । \textbf{सम्प्रति तु} सर्वथैवासौ धर्मित्वेनासिद्ध उच्यत इति । धर्म्यसिद्ध एवाश्रयासिद्ध उच्यत इति । तेन नाश्रयासिद्धो नाम अन्यः प्रभेदः ।
	\pend
      

	  \pstart अन्यथासिद्धस्त्वसिद्ध एव न भवतीति न तस्यान्तर्भावश्चिन्त्यते । तथा ह्यन्यथासिद्ध इति कोऽर्थः ? किमन्यथैव सिद्ध आहोस्वित् अन्यथाऽपि सिद्धः ? ननु यद्यन्यथैव, तदा जिज्ञापयिषितविपर्ययेणैव सिद्ध उपपन्नो नानेन प्रकारेणेति विरुद्ध एव । अथान्यथाऽपि सिद्धः, तदैतस्मादन्येनापि प्रकारेण सिद्धोऽयमित्ययमपि भविष्यति । न च साध्यमिति सन्दिग्धविपक्षव्यावृत्तिरनैकान्तिक एवेति ॥
	\pend
      

	  \pstart अनैकान्तिकशब्दस्य व्युत्पत्तिमाह--\textbf{एक} इति । \textbf{एक} इति ब्रुवन्नेकस्ता\footnote{कश्चा}सौ साध्यलक्षणैकार्थविषयत्वादन्तश्च कथावसानहेतुत्वादाकाङ्क्षोपशमहेतुत्वाद् वेति दर्शयति । समस्तं पदमर्थञ्चाह--\textbf{एकान्तो निश्चय} इति । साध्येतरयोरेकतरनिश्चयफल इत्यर्थः । तद्विरुद्धे चायं नञ् द्रष्टव्यः । \textbf{यस्मादि}त्यादिनैतदेव स्फुटयति । \textbf{यस्माद्}धेतोरित्यपादाने चेयं पञ्चमी । किन्तु \textbf{तद्विपरीतो} निश्चयविपरीतः यत्तदोर्नित्यमभिसम्बन्धात्स इति द्रष्टव्यम् । क्व संशय इत्याह--\textbf{साध्येति} ।
	\pend
      

	  \pstart यद्वा कथमनैकान्तिको भवतीत्याह--\textbf{यस्मादि}ति हेतुपञ्चमी । ऐकान्तिकप्रतिषेधेनान्यः प्रतिपत्तिहेतुरुक्तो नञेंत्यभिप्रायेणाह--\textbf{संशयहेतु}रिति । तेन नासिद्धस्य तथात्वप्रसङ्गः । अथवैकान्तनियतत्वादेक इत्यन्त इति च निश्चयः प्रोक्तः । तथाहि सर्वोऽयं पदार्थभेद  \leavevmode\marginnote{\textenglish{196/dm}} “
	  
	तमुदाहरति-- “
	  
	यथा शब्दस्यानित्यत्वादिके धर्मे साध्ये प्रमेयत्वादिको धर्मः सपक्षविपक्षयोः सर्वत्रैकदेशे वा वर्त्तमानः ॥ ६७ ॥” 
	  
	यथेत्यादिना । अनित्यत्वमादिर्यस्याऽसौ\footnote{०र्यस्य सोऽनि० \cite{dp-msA} \cite{dp-msB} \cite{dp-msD} \cite{dp-edP} \cite{dp-edH} \cite{dp-edE} \cite{dp-edN}} अनित्यत्वादिको धर्मः । आदिशब्दादप्रयत्नानन्तरीयकत्वं प्रयत्नानन्तरीयकत्वं \footnote{“नित्यत्वं” नास्ति \cite{dp-msB}}नित्यत्वं च परिगृह्यते । प्रमेयत्वम् आदिर्यस्य स प्रमेयत्वादिकः । आदिशब्दादनित्यत्वम्, पुनरनित्यत्वम्, अमूर्त्तत्वं च \footnote{गृह्यन्ते \cite{dp-msB}}गृह्यते । शब्दस्य धर्मिणोऽनित्यत्वादिके धर्मे साध्ये प्रमेयत्वादिको धर्मोऽनैकान्तिकः । चतुर्णामपि \footnote{नास्ति “हि”--\cite{dp-msA} \cite{dp-msB} \cite{dp-edP} \cite{dp-edH} \cite{dp-edN}}हि विपक्षेऽसत्त्वमसिद्धम् । 
	  
	तथाहि-अनित्यः शब्दः प्रमेयत्वात्\footnote{प्रमेयत्वात्, आकाशवत् घटवदिति--\cite{dp-msA} \cite{dp-msB} \cite{dp-msD} \cite{dp-edP} \cite{dp-edH} \cite{dp-edE} \cite{dp-edN}} घटवद्-आकाशवदिति प्रमेयत्वं सपक्षविपक्षव्यापि । 
	  
	अप्रयत्नानन्तरीयकः शब्दोऽनित्यत्वात्, विद्युदाकाशवद् \footnote{काशघटव० \cite{dp-msA}}घटवच्च--इत्यनित्यत्वं सपक्षैकदेशवृत्ति--विद्युदादावस्ति, नाकाशादौ; \footnote{इत आरम्भ “नाकाशादौ” पर्यन्तः पाठो नास्ति \cite{dp-msB}}विपक्षव्यापि--\footnote{०व्यापि सर्वत्र प्रयत्नानन्तरीयके भावात् \cite{dp-msC}}प्रयत्नानन्तरीयके सर्वत्र भावात्\footnote{अनित्यत्वस्य--\cite{dp-msD-n}} । 
	  
	अनित्यत्वात् प्रयत्नानन्तरीयकः शब्दो घटवद् विद्युदाकाशवच्च--इत्यनित्यत्वं विपक्षैकदेशवृत्ति--विद्युदादावस्ति नाकाशादौ । सपक्षव्यापि\footnote{व्यापि सपक्षे सर्वत्र \cite{dp-msC} \cite{dp-msD}} सर्वत्र प्रयत्नानन्तरीयके भावात्\footnote{साध्यैः सह क्रमात् योज्यते--\cite{dp-msD-n}} । 
	  
	नित्यः शब्दोऽमूर्त्तत्वाद् आकाशपरमाणुवत्,\footnote{नित्यत्वस्य--\cite{dp-msD-n}} कर्मघटवच्च । इत्यमूर्तत्वमुभयैकदेशवृत्ति--उभयोरेकदेश आकाशे कर्मणि च वर्तते । परमाणौ तु सपक्षैकदेशे घटादौ च विपक्षैकदेशे न वर्त्तते । मूर्त्तत्वात् घटपरमाणुप्रभृतीनाम् । 
	  
	नित्यास्तु परमाणवो \footnote{वैशेषिकैरप्यभ्युप० \cite{dp-msC}}वैशेषिकैरभ्युपगम्यन्ते । ततः सपक्षान्तर्गताः ।” एकस्मिन्ना\footnote{न्न}न्तेऽवतिष्ठेते--नित्यो वाऽनित्यो वेत्यादिरूपेण । \textbf{स प्रयोजनमस्य} स तथा । न तथाऽ\textbf{नैकान्तिकः} ॥
	\pend
      

	  \pstart \textbf{विपक्षेऽसत्त्वमसिद्धं} सर्वत्रेति द्रष्टव्यम् । \leavevmode\marginnote{\textenglish{68b/ms}} “सदकारणवन्नित्यम्” \href{http://http://sarit.indology.info/?cref=vsū.4.1.1}{वै० सू० ४.
	    १. १.} इति नित्यलक्षणयोगान्नित्या इष्टाः \textbf{परमाणवः} ॥
	\pend
      \leavevmode\marginnote{\textenglish{197/dm}}“

	  \pstart अस्य चतुर्विधस्य पक्षधर्मस्यासत्त्वमसिद्धं विपक्षे । ततोऽनैकान्तिकता ॥
	\pend
       “

	  \pstart तथा--अस्यैव रूपस्य संदेहेऽप्यनैकान्तिक एव ॥ ६८ ॥
	\pend
      ” 

	  \pstart यथा चास्य रूपस्यासिद्धावनैकान्तिकस्तथा अस्यैव विपक्षेऽसत्त्वाख्यस्य\footnote{०ख्यरूपस्य \cite{dp-msC}} रूपस्य संदेहेऽनैकान्तिकः ॥
	\pend
       

	  \pstart तमुदाहरति--
	\pend
       “

	  \pstart यथाऽसर्वज्ञः कश्चिद्विवक्षितः पुरुषो रागादिमान् वेति साध्ये वक्तृत्वादिको धर्मः सन्दिग्धविपक्षव्यावृत्तिकः ॥ ६९ ॥
	\pend
      ” 

	  \pstart यथेति । असर्वज्ञ इत्यसर्वज्ञत्वं साध्यम् । कश्चिद्विवक्षित इति वक्तुरभिप्रेतः पुरुषो धर्मीं । रागा आदिर्यस्य द्वेषादेः स रागादिः । स यस्यास्ति स रागादिमान् इति द्वितीयं साध्यम् । \footnote{वेति ग्रह० \cite{dp-msC} \cite{dp-msD}}वाग्रहणं रागादिमत्त्वस्य पृथक्साध्यत्वख्यापनार्थम् । ततोऽसर्वज्ञत्वे रागादिमत्त्वे वा\footnote{मत्त्वे च साध्ये \cite{dp-msC}} साध्ये प्रकृते वक्तृत्वं--वचनशक्तिस्तदादिर्यस्योन्मेषनिमेषादेः स वक्तृत्वादिको धर्मोऽनैकान्तिकः ।
	\pend
       

	  \pstart सन्दिग्धा\footnote{संदिग्धवि० \cite{dp-msB}} विपक्षा व्यावृत्तिर्यस्य स तथोक्तः । असर्वज्ञत्वे साध्ये सर्वज्ञत्वं विपक्षः । तत्र वचनादेः सत्त्वमसत्त्वं वा सन्दिग्धम् । अतो न ज्ञायते किं\footnote{“किं” नास्ति \cite{dp-msA} \cite{dp-msB} \cite{dp-msD} \cite{dp-edP} \cite{dp-edH} \cite{dp-edE} \cite{dp-edN}} वक्ता सर्वज्ञ उतासर्वज्ञ इत्यनैकान्तिकं वक्तृत्वम् ।
	\pend
       

	  \pstart ननु च सर्वज्ञो वक्ता नोपलभ्यते तत्कथं वचनं सर्वज्ञे सन्दिग्धम् ? अत एव
	\pend
       “

	  \pstart \footnote{सर्वत्रैकदेशे वा सर्वज्ञो \cite{dp-msB} \cite{dp-edP} \cite{dp-edH}}“सर्वज्ञो वक्ता नोपलभ्यते” इत्येवंप्रकारस्यानुपलभ्भस्यादृश्यात्मविषयत्वोन \footnote{संदेहे हेतु \cite{dp-msB} \cite{dp-edP} \cite{dp-edH} \cite{dp-edE}}संदेहहेतुत्वाद् ।\footnote{त्वादसर्व० \cite{dp-msB} \cite{dp-msD} \cite{dp-edP} \cite{dp-edH} \cite{dp-edE}} ततोऽसर्वज्ञविपर्ययाद्वक्तृत्वादेर्व्यावृत्तिः सन्दिग्धा ॥ ७० ॥
	\pend
      ” 

	  \pstart “सर्वज्ञो वक्ता नोपलभ्यते” इत्येवम्प्रकारस्य--\footnote{०जातीयकस्य--\cite{dp-msB}}एवंजातीयस्यानुपलम्भस्य संदेहहेतुत्वात् । कुत इत्याह--अदृश्य\footnote{अदृश्यात्मा \cite{dp-msA} \cite{dp-msB} \cite{dp-edP} \cite{dp-edH} \cite{dp-edE} \cite{dp-edN}} आत्मा विषयो यस्य तस्य भावोऽदृश्यात्मविषयत्वं तेन सन्देहहेतुत्वम् ।
	\pend
      ”

	  \pstart \textbf{द्वेषादे}रित्यादिग्रहणेन मोहादेर्ग्रहणम् ॥
	\pend
      

	  \pstart \textbf{अत एवा}नुपलम्भमात्रादेवेति सिद्धान्ती । अमुमेवार्थं मूलेन संस्यन्दयन्नाह--\textbf{सर्वज्ञ} इति । \textbf{तेना}दृश्यविषयत्वेन हेतुना \textbf{सन्देहहेतुत्वं} सन्देहहेतुत्वादिति । हेतुपञ्चमीमिदानीं व्याचष्टे—यत इति ॥
	\pend
      \leavevmode\marginnote{\textenglish{198/dm}}“

	  \pstart यतोऽदृश्यविषयोऽनुपलम्भः\footnote{संशयहे० \cite{dp-msA} \cite{dp-edP} \cite{dp-edH} \cite{dp-edE} \cite{dp-edN}} सन्देहहेतुर्न निश्चयहेतुस्ततोऽसर्वज्ञविपक्षात् सर्वज्ञाद् वक्तृत्वादेर्व्यावृत्तिः सन्दिग्धा ॥
	\pend
       

	  \pstart नानुपलम्भात् \footnote{०भात् संदिग्धे वक्तृ० \cite{dp-msB}}सर्वज्ञे वक्तृत्वमसद्ब्रूमः । अपि तु सर्वंज्ञत्वेन सह वक्तृत्वस्य विरोधात् । एतन्न\footnote{“एतन्न” नास्ति \cite{dp-edE}} ।
	\pend
       “

	  \pstart वक्तृत्वसर्वज्ञत्वयोर्विरोधाभावाच्च यः सर्वज्ञः स वक्ता न भवतीत्यदर्शनेपि व्यतिरेको न सिध्यति, संदेहात् ॥ ७१ ॥
	\pend
      ” 

	  \pstart सर्वज्ञत्ववक्तृत्वयोर्विरोधो नास्ति । विरोधाभावाच्च कारणाद् व्यतिरेको न सिध्यति—इति संबन्धः ।
	\pend
       

	  \pstart व्याप्तिमन्तं व्यतिरेकं दर्शयति--यः सर्वज्ञ इति । साध्याभावरूपं सर्वज्ञत्वमनूद्य \footnote{न स वक्ता भवति \cite{dp-msA} \cite{dp-edP} \cite{dp-edH} \cite{dp-edE} \cite{dp-edN}} “स वक्ता न भवति” इति साधनस्य वक्तृत्वस्याभावो विधीयते । तेन साध्याभावः साधनाभावे नियतत्वात् \footnote{साधर्म्यभावेन \cite{dp-msB}}साधनाभावेन व्याप्त उक्त इति । व्याप्तिमानीदृशो व्यतिरेको विरोधे सति वक्तृत्वसर्वज्ञत्वयोः सिध्येत् । न चास्ति विरोधः । तस्मान्न सिध्यतीति\footnote{सिध्यति । कुतः \cite{dp-msA} \cite{dp-edP} \cite{dp-edH} \cite{dp-edE} \cite{dp-edN}} । कुत इत्याह—संदेहात् । यतो विरोधाभावः, तस्मात् संदेहः । सन्देहाद् व्यतिरेकासिद्धिः ॥
	\pend
       

	  \pstart कथं विरोधाभावः ?
	\pend
       “

	  \pstart द्विविधो हि पदार्थानां विरोधः ॥ ७२ ॥
	\pend
      ” 

	  \pstart \footnote{हिर्यस्मादर्थेद्विवि० \cite{dp-msD} हिर्यस्मात्--\cite{dp-msB}}हीति यस्माद् द्विविध एव विरोधो नान्यः, तस्मान्न वक्तृत्वसर्वंज्ञत्वयोर्विंरोधः ॥
	\pend
       

	  \pstart कः पुनरसौ द्विविधो विरोध इत्याह--
	\pend
       “

	  \pstart अविकलकारणस्य भवतोऽन्यभावेऽभावाद्\footnote{०भावः । अभा० \cite{dp-msB} \cite{dp-edP} \cite{dp-edH}} विरोधगतिः \footnote{गतिरिति--\cite{dp-msC}}॥ ७३ ॥
	\pend
      ” 

	  \pstart अविकलकारणस्येति । अविकलानि समग्राणि कारणानि यस्य स तथोक्तः । यस्य कारणवैकल्यादभावो न तस्य केनचिदपि विरोधगतिः । तदर्थम् अविकलकारणग्रहणम् ।
	\pend
       

	  \pstart ननु च यस्यापि कारणसाकल्यं तस्यापि निवृत्तिरशक्या केनचिदपि कर्त्तुम् । तत् कुतो विरोधगतिः ? एवं तर्हि अविकलकारणस्यापि यत्कृतात् कारणवैकल्याद् अभावस्तेन विरोधगतिः ।
	\pend
      ”

	  \pstart \textbf{यस्ये}त्यादिनाऽविकलकारणस्य फलं वर्णयति । एवं तर्हीत्युत्तरम् । तर्हि तस्मिन् काले । \textbf{एवं} बोद्धव्यमित्यर्थः । \textbf{अपि} सम्भावनायाम् । न्यायबलादेवं सम्भावयाम इत्यर्थः । \textbf{विरोध}स्य गतिः प्रतिपत्तिः ।
	\pend
      \leavevmode\marginnote{\textenglish{199/dm}}“

	  \pstart तथा च सति यो यस्य विरुद्धः स तस्य किञ्चित्कर एव । तथा हि--शीतस्पर्शस्य जनको भूत्वा शीतस्पर्शान्तरजननशक्तिं प्रतिबध्नन् शीतस्पर्शस्य निवर्त्तको विरुद्धः । तस्माद्धेतुवैकल्यकारी विरुद्धो जनक एक निवर्त्त्यस्य । सहानवस्थानविरोधश्चायम् । ततो विरुद्धयोरेकस्मिन्नपि क्षणे सहावस्थानं परिहर्त्तव्यम् । दूरस्थयोर्विरोधाभावाच्च निकटस्थयोरेव निवर्त्त्यनिवर्त्तकभावः ।
	\pend
       

	  \pstart तस्माद्यो यस्य निवर्त्तकः स तं यदि परं तृतीये क्षणे निवर्त्तयति । प्रथमे क्षणे सन्निपतन्नसमर्थावस्थाधानयोग्यो\footnote{स्थाधाने योग्यो \cite{dp-msD} ०स्थानयोग्यो \cite{dp-msA} \cite{dp-msB} \cite{dp-edP} \cite{dp-edH} \cite{dp-edE}} भवति । द्वितीये विरुद्धमसमर्थं करोति । तृतीये त्वसमर्थे निवृत्ते तद्देशमाक्रामति ।
	\pend
       

	  \pstart तत्रालोको गतिधर्मा क्रमेण जलतरङ्गन्यायेन \footnote{तद्देशमा० \cite{dp-msC} \cite{dp-msD}}देशमाक्रामन्\footnote{०माक्रामयन् \cite{dp-msB} \cite{dp-edN}} यदाऽन्धकारनिरन्तर-\footnote{कारे निर० \cite{dp-msA} \cite{dp-msB} \cite{dp-edP} \cite{dp-edH} \cite{dp-edE} \cite{dp-edN}} मालोकक्षणं जनयति तदाऽऽलोकसमीपवर्त्तिनमन्धकारमसमर्थं जनयति । ततोऽसामर्थ्यं तस्प यस्य समीपवर्त्त्यालोकः । \footnote{असमर्थ्ये \cite{dp-msA} असामर्थ्ये \cite{dp-edP} \cite{dp-edH} \cite{dp-edE} \cite{dp-edN}}असमर्थे निवृत्ते \footnote{तादृशो \cite{dp-msA} \cite{dp-msB} \cite{dp-edP} \cite{dp-edH} \cite{dp-edN}}तद्देशो जायत आलोक इत्येवं क्रमेणाऽऽलोकेनान्धकारोऽपनेयः । तथोष्णस्पर्शेन शीतस्पर्शो निवर्त्तनीयः ।
	\pend
      ”

	  \pstart किमतः सिद्धमित्याह--तथा चेति कारणवैकल्यकारिणो विरोधावगमप्रकारे सति । किञ्चित्करत्वमेव तथा हीत्यादिना दर्शयति । यथा चास्य जनकत्वं तथाऽनन्तरमेव व्यक्तीकरिष्यते ।
	\pend
      

	  \pstart ननु किं कतिपयक्षणसहितयोः पश्चान्निवर्त्त्यनिवर्त्तकभावेन विरोधोऽथवाऽन्यथेत्याशङ्क्याह--सहेति । \textbf{चो} यस्मात् ततस्तस्मात् । न केवलं बहुषु क्षणेष्वित्यपि शब्दः । \textbf{सहावस्थानमे}कत्र स्थितिः । निकटावस्थानं तु न परिहर्त्तव्यमिति बुद्धिस्थम् । \textbf{परिहर्त्तव्यं} नाङ्गीकर्त्तव्यम् । तयोरेकस्मिन्नपि क्षणे सहस्थित्यभावात् कथमेवमङ्गीक्रियते ? अत एव । न सहस्थितयोः पश्चाद् विरोध इति वा कृतमनेन ।
	\pend
      

	  \pstart यद्येवं क्वचित्प्रदेशे वर्त्तमान आलोकस्त्रिलोकीव्यवस्थितानि तमांस्यनेनैव क्रमेणापनयेदिति न क्वचित् तमांस्यवतिष्ठेरन्नित्याह--\textbf{निकटस्थयोरि}ति ययोर्निवर्त्त्यनिवर्त्तकभावो दृष्टस्तयोर्निकटस्थयोरेव न तु निकटस्थयोरवश्यं निवर्त्त्यनिर्वर्त्तकभाव इत्यस्यार्थो द्रष्टव्यः । तयोरेव कथं तथाभाव इत्याशङ्कायां \textbf{दूरस्थायोरि}ति योज्यम् । \textbf{चो}ऽवधारणे । यतः किञ्चित्करस्यैव निवर्त्तकत्वं \textbf{तस्माद्} हेतोः \textbf{परं} प्रकृष्टं यथा भवति । एतदेवोपपादयन्नाह--\textbf{प्रथम} इति \textbf{सन्नियतन्नि}कटीभवन्निवर्त्तक इति प्रकरणात् । \textbf{असमर्था} चोपादेयक्षणनिर्माणे अशक्ता\textbf{वस्था} यस्यान्धकारक्षणस्य तस्याऽऽ\textbf{धानमु}त्पादनम्, तत्र \textbf{योग्यः} समर्थो भवति । \textbf{द्वितीये} क्षणे इत्यनुवर्त्तते । \textbf{विरुद्धम}न्धकार\textbf{मसमर्थं} सजातीयक्षणान्तरजननाक्षमं करोति । तृतीये क्षणेऽऽ\textbf{समर्थे} तस्मिन्  \leavevmode\marginnote{\textenglish{200/dm}} “
	  
	यदा त्वालोकस्तत्रैवान्धकारदेशे जन्यते तदा यतः क्षणादन्धकारदेशस्यालोकस्य जनकः\footnote{जनकक्षणः \cite{dp-msA} \cite{dp-msB} \cite{dp-edP} \cite{dp-edH} \cite{dp-edE} \cite{dp-edN}} क्षणः उत्पद्यते तत एवान्धकारोऽन्धकारान्तरजननासमर्थ\footnote{अन्धकारान्तरासमर्थः \cite{dp-msA} अन्धकारान्तराजननासमर्थः \cite{dp-msB} \cite{dp-msC}} उत्पन्नः । \footnote{ततोऽसामर्थ्याव० \cite{dp-msB}}ततोऽसमर्थावस्थाजनकत्वमेव निवर्त्तकत्वम् । 
	  
	अतश्च यस्मिन् क्षणे जनकस्ततस्तृतीये क्षणे निवृत्तो विरुद्धो यदि शीघ्रं निवर्त्तते । 
	  
	जन्यजनकभावच्च \footnote{सन्तनयोः \cite{dp-msA}}सन्तानयोर्विरोधो न क्षणयोः । यद्यपि च न सन्तानो नाम वस्तु तथापि सन्तानिनो वस्तुभूताः । ततोऽयं परमार्थः--न क्षणयोर्विरोधः । अपि तु बहूनां” वन्ध्यक्षणे \textbf{निवृत्ते} स्वरसतो निरुद्धे तद्देशं तस्यासमर्थक्षणस्य देशं स्थानमा\textbf{क्रामति,} तद्देशो भवति निवर्त्तक इत्यर्थात् ।
	\pend
      

	  \pstart इह कश्चिन्निवर्त्तक आलोको यामेव दिशमाक्रामति तद्दिग्वर्त्तिनमेव स्वविरुद्धं गतिक्रमेणैव निवर्त्तयति । कश्चित्पुनर्विरुद्धावष्टब्ध एव देशे समुत्पन्नमात्र एवानेकदिग्वर्त्तिनं विरुद्धं झटिति निवर्त्तयति । तत्र न ज्ञायते कस्य कथं किञ्चित्करतया निवर्त्तकत्वमित्याह--\textbf{तत्रेति} वाक्योपक्षेपे । देशमभिमतं स्थानमा\textbf{क्रामं}स्तद्देशो भवन्नालोक इत्यर्थः । \textbf{अन्धकारनिरन्तरम}न्धकाराव्यवहितम् । \textbf{आलोकसमीपवर्त्तिनमि}ति तज्जन्यमानालोकसमीपवर्त्तिनम् । \textbf{असमर्थ}मन्धकारान्तरजननाशक्तं जनयति । यत एवं ततस्तस्य जनकत्वम् । ततस्तस्मात्समीप\textbf{वर्त्त्यालोक} इति गतिधर्मेति द्रष्टव्यम् । \textbf{असामर्थ्यं} चो\leavevmode\marginnote{\textenglish{69a/ms}}पादेयक्षणोपजननं प्रतीति प्रस्तावादवसेयम् । \textbf{असमर्थे} तस्मिन्नन्धकारे \textbf{निवृत्ते} स्वरसतोवि\footnote{नि}रुद्धे सति । सोऽसमर्थान्धकारक्षणदेशो देशो यस्य स तथा जायते \textbf{आलोकः} । इतिस्तस्मात् । \textbf{एवम}नन्तरोक्तेन \textbf{क्रमेण} परिपाट्या \textbf{गतिधर्मालोकस्}तद्देशाक्रमणाय सन्निपतन्नसमर्थावस्थाऽऽधानयोग्यो भवति । द्वितीये क्षणेऽसमर्थं जनयति । तृतीये तद्देशो जायत इत्यनन्तरोक्तः क्रमो विभज्य योजनीयः ।
	\pend
      

	  \pstart अमुमेव क्रममन्यत्रादिशन्नाह--\textbf{तथे}ति । यथा--प्रालोकान्धकारयोर्निवर्त्त्यनिवर्त्तकभावस्तेन प्रकारेण \textbf{उष्णस्पर्शेन} गतिधर्मेण दृष्टान्तवशाद् द्रष्टव्यम् ।
	\pend
      

	  \pstart गतिधर्मणस्तावदालोकस्यायं क्रमः । विरुद्धाक्रान्तदेशमध्योत्पन्नस्य कीदृश इत्याह—यदेति । तुर्विशेषणार्थः । \textbf{यतः क्षणादालोकस्य जनकः क्षण उत्पद्यते} । कीदृशस्यालोकक्षणस्येत्याह--\textbf{अन्धकारे}ति । \textbf{अन्धकारदेशस्य} निवर्त्त्याऽन्धकारसम्बन्धी देशो यस्य स तथा \textbf{तत एव} तमोदेशालोकोत्पादक्षणयोरेकसामग्र्यधीनतामाह । यतोऽन्धकारदेशालोकहेतूत्पादकस्य क्षणस्य वन्ध्यान्धकाराधायकत्वतो हेतोरविद्यमानं सजातीयजन्मनि सामर्थ्यं यस्या अवस्थाया अन्धकारसम्बन्धिन्याः सा तथा । तज्जनकत्वमेवालोकस्येति प्रकरणात् ।
	\pend
      

	  \pstart अत्रापि तृतीये क्षणे परं निवर्त्तकत्वमिति दर्शयन्नाह--\textbf{अतश्चे}ति । \textbf{चो}ऽवधारणे । अत्रापि \textbf{प्रथमे क्षणे}ऽन्धकारदेशालोकहेतूत्पादकः क्षणः समुद्भवन्नेवान्धकारासमर्थावस्थातद्देशालोकहेतुजनन  \leavevmode\marginnote{\textenglish{201/dm}} “
	  
	क्षणानाम् । यतः सत्सु दहनक्षणेषु प्रवृत्ता अपि शीतक्षणा निवृत्तिधर्माणो भवन्तीति सन्तानयोर्निवर्त्त्यनिवर्त्तकत्वनिमित्ते च विरोधे स्थिते सर्वेषां परमाणूनां सत्यप्येकदेशावस्थानाभावे न विरोधः, इतरेतरसन्तानानिवर्त्तनात् तेषाम् । गतिधर्मा चालोको यां \footnote{दिशं क्राम० \cite{dp-msC}}दिशमाक्रामति \footnote{तद्विवर्त्तिनः \cite{dp-msA}}तद्दिग्वर्त्तिनो विरोधिसन्तानान् निवर्त्तयति । ततोऽपवरकैकदेशस्था प्रदीपप्रभाऽन्धकारनिकटवर्त्तिव्यपि नान्धकारं निवर्त्तयति, \footnote{अन्धकारायाक्रान्तायां \cite{dp-msA}}अन्धकाराक्रान्तायां दिश्यालोकक्षणान्तरजननासामर्थ्यात् । कारणासामर्थ्यहेतुत्वकृतं\footnote{०हेतुकृतं--\cite{dp-msA} \cite{dp-edP} \cite{dp-edH} \cite{dp-edE} \cite{dp-edN}} सन्ताननिष्ठमेव विरोधं दर्शयता \footnote{भवतेति \cite{dp-edP} \cite{dp-edH}}भवत इति कृतम् । भवतः प्रबन्धेन\footnote{०न्धेन वर्त्त० \cite{dp-msA} \cite{dp-msB} \cite{dp-msD} \cite{dp-edP} \cite{dp-edH} \cite{dp-edE} \cite{dp-edN}} प्रवर्त्तमानस्य\footnote{०स्य सन्तान० \cite{dp-msC}} शीतस्पर्शसन्तानस्याभावोऽन्यस्योष्णसन्तानस्य भावे सतीति ।” योग्यो भवति । द्वितीयेऽन्धकारदेशालोकोत्पादकक्षणविरुद्धानन्धकारानसमर्थान् जनयति । तृतीये त्वसमर्थेषु निवृत्तेषु तद्देश आलोको जायत इति प्रत्येतव्यम् । तथा शीताक्रान्तदेशमध्योत्पन्नेनोष्णस्पर्शेन स्थितधर्मणा तथैव शीतस्पर्शो निवर्त्तनीय इत्यपि द्रष्टव्यम् ।
	\pend
      

	  \pstart ननु च येनालोकक्षणेन सन्निपति \footnote{पत} तान्धकारक्षणोऽसमर्थो जन्यते न तेन तद्देश आक्रम्यते । येन चाक्रम्यते न तेनासमर्थो जन्यते । तथा योऽन्धकारस्तत्सन्निपतनकालभावी नासौ तद्विरुद्धः । यश्चासमर्थस्तज्जन्मा सोऽपि तज्जन्यत्वादविरोधी । ये चानुत्पत्तिधर्माणस्तेऽप्यसत्त्वात्कथं तैर्विरुद्धा इत्याशङ्क्याह--\textbf{जन्यजनकभावादि}ति । \textbf{चो}ऽवधारणे \textbf{सन्तानयो}रित्यस्यानन्तरं द्रष्टव्यः ।
	\pend
      

	  \pstart अयमाशयः--जन्यजनकभावविशेष एवायं निवर्त्त्यनिवर्त्तकभावः अर्वाग्दर्शी च न क्षणयोः कार्यकारणभावं विभावयितुं विभवति । अपि तु सन्तानयोस्ततोऽन्धकारक्षणप्रबन्धमेकत्वेनावसाय निवर्त्त्यं विरुद्धमध्यवस्यालोकक्षणप्रबन्धं चैकत्वेनाधिमच्य तद्विरोधिनमधिमुञ्चतीति ।
	\pend
      

	  \pstart परमार्थदृष्ट्या चेदं क्षणोल्लेखेनाख्यायते । न तु लोकस्थित्याश्रयेण । न तर्हि परमार्थतो विरोध इति चेत् । किं वै कार्यकारणभावविशेष एवैवंविधो न विद्यते, येनैवं वक्तुमध्यवसितो भवानिति ? एतच्चानन्तरमेव निरूपयिष्यते ।
	\pend
      

	  \pstart ननु न सन्तानिव्यतिरेकेण सन्तानो नामान्यः सम्भवी । तत्कथं\leavevmode\marginnote{\textenglish{69b/ms}}द्वयोः सन्तानयोर्विरोध उच्यत इत्याह \textbf{यद्यपी}त्यनुमतौ । यतः सन्तानिनो वस्तुभूताः सन्ति \textbf{ततो} हेतोरयं वक्ष्यमाणकः । उपपत्तिमाह--\textbf{यत} इति । यस्माद् ग\footnote{शी}तक्षणप्रबन्धस्याभावान्निवृत्तिधर्मकत्वम् । तथा यतः सत्स्वालोके\footnote{क}क्षणेषु प्रवृत्ता अप्यन्धकारक्षणान्नि\footnote{णा नि}वृत्तिधर्माणो भवन्तीति द्रष्टव्यम् । अन्धकारादिक्षणप्रबन्धस्य\footnote{कोष्ठकान्तर्गतः पाठः व्यर्थः--सं० ।} \footnote{आलोकादिक्षणप्रबन्धस्य} आलोकादिक्षणप्रबन्धेन सह विरोध इति प्रकरणार्थः ।
	\pend
      \leavevmode\marginnote{\textenglish{202/dm}}“

	  \pstart ये त्वाहुर्न विरोधो वास्तव इति त इदं वक्तव्याः--यथा न निष्पन्ने कार्ये कश्चिज्जन्य-
	\pend
      ”

	  \pstart यदि येन सह यस्यैकदेशास्थितिर्न भवति तेन तस्य सहानवस्थानलक्षणो विरोधस्तर्हि सर्व एव परमाणवः सप्रतिघत्वादन्योन्यदेशपरिहारेण वर्त्तन्त इति सर्वेषामेव परमाणूनामयं विरोधः किन्न व्यवस्थाप्यत इत्याशङ्क्याह--\textbf{सन्तानयोरि}ति । \textbf{चो}ऽवधारणे । हेतुमाह—\textbf{इतरेतरे}ति । यतः सत्स्वपि तेषु सर्व एव सन्तानेन प्रवहन्ति ततः सन्तानाऽनिवर्त्तनं तेषाम् ।
	\pend
      

	  \pstart ननु यद्यालोकान्धकारयोर्निवर्त्त्यनिवर्त्तकभावेन विरोधस्तर्हि प्रदीपमल्लिकातलवर्त्त्येव वरकात्माण निवर्त्ती\footnote{?} अन्धकारस्तत्समीपवर्त्तिनाऽऽलोकेन किं न निवर्त्त्यत इत्याशङ्क्याह—\textbf{गतिधर्मे}ति । \textbf{चो} यस्मादर्थे । \textbf{तद्दिग्वर्त्तिन} एवेत्यर्थाद् द्रष्टव्यः । यतो यद्दिगभिमुखगतिरालोकस्तदाक्रम्यमाणदिग्वर्त्तिन एव विरोधिसन्तानान्निवर्त्तयति । ततस्तस्मात्कारणात् ।
	\pend
      

	  \pstart कुतो न निवर्त्तयतीत्याह--\textbf{अन्धकारेति । अन्धकाराक्रान्ताया}मित्यनेन दिशोऽन्त\footnote{ऽन्ध}काराक्रान्तत्वमालोकक्षणान्तराजननासामर्थ्यकारणं नोक्तम् । किन्तर्हि ? वास्तवानुवादः कृतः । या सा दिगन्धकाराक्रान्ता दृश्यते तत्र तस्य तज्जननासामर्थ्यादित्यर्थः । अन्यथाऽन्धकाराक्रान्तत्वमेव तस्य न स्यात् । आलोकेन समीपवर्त्तिनाऽन्धकारापनयासम्भवादिति कथमेनं संगच्छेत ।
	\pend
      

	  \pstart अयमत्र परमार्थः--दृश्यते तावत्काचिदन्धकारमात्रा निकटस्थितेनाप्यालोकेनाऽनिवर्त्तिता । दृष्टश्चान्यस्याववरकवर्त्तिनोऽन्धकारप्रचयस्योच्छेदः । तस्मादालोकस्यालोकान्तरजननासामर्थ्यमन्यत्र तु सामर्थ्यं तत्त्वचिन्तकैरचिन्त्यत्वात्प्रतीत्यसमुत्पादस्य कल्प्यत इति । अत एव ययोर्जन्यजनकभावेन निवर्त्त्यनिवर्त्तकभावो नास्ति तयोः प्रदीपमल्लिकादितलवर्त्त्यन्धकारतदासन्नालोकयोर्नविरोधः । प्रायोवृत्त्या तु तौ विरोधेनावबुद्ध्येते । अत एव पूर्वं \textbf{दूरस्थयोर्विंरोधाभावाच्च निकटस्थयोरेव निवर्त्त्यंनिवर्त्तकभावः\textbf{}} इत्युक्तम्, न तु “निकटस्थयोर्निवर्त्त्यनिवर्त्तकभाव एव” इति । सति निवर्त्त्यनिवर्त्तकत्वे “निकटस्थयोरेव, न तु निकटस्थयोरवश्यं निवर्त्त्यनिवर्त्तकभावः” इति च व्याख्यातमेव ।
	\pend
      

	  \pstart सम्प्रति जन्यजनकभावनिबन्धनं सन्तानगतमेव च विरोधं स्वयं प्रतिपादितमाचार्यस्याप्यभिप्रेतमेतदिति दर्शयन्नाह--\textbf{कारणैरिति} \footnote{णेति} । \textbf{कारणस्य} निवर्त्तयितव्यस्य शीतस्पर्शादेर्यद\textbf{सामर्थ्यं} सजातीयक्षणनिर्माणेऽशक्तत्वं तत्र यद्धेतुत्वं निवर्त्तकस्य \textbf{तत्कृतं} तत्प्रयुक्तम् । अत एव \textbf{सन्ताननिष्ठं सन्ताने} क्षणप्रबन्धे \textbf{निष्ठा} व्यवस्थाप्यतया पर्यवसानं यस्य तं \textbf{दर्शयता} प्रकाश\leavevmode\marginnote{\textenglish{70a/ms}}यताऽऽचार्येणेत्यर्थात् ।
	\pend
      

	  \pstart ये पुनः \textbf{शान्तभद्रादयः}--“\textbf{न} तावदालोकादेरुत्पन्नेनान्धकारादिना विरोधः, तस्यातीतत्वेनासत्त्वात् । न चोत्पित्सुना सह, तस्याप्यनागततयाऽसत्त्वात् । नाऽपि वर्त्तमानेन, तस्यापि तज्जन्मतयाऽविरोधित्वात् । तस्मान्न विरोधो नाम द्विष्ठः सम्बन्धोऽस्ति । किन्तु काल्पनिक एव । अत एवाचार्येण \textbf{विरोधगतिरि}त्यभिधायि । न तु विरोध इति ।”--इति व्याख्यातवन्तस्तान् वचनभङ्ग्या निराचिकीर्षुराह--\textbf{ये त्वि}ति ।
	\pend
      \leavevmode\marginnote{\textenglish{203/dm}}“

	  \pstart जनकभावो नाम \footnote{दृष्टोऽस्ति--\cite{dp-msA} \cite{dp-msB} \cite{dp-edP} \cite{dp-edH} \cite{dp-edE} \cite{dp-edN}}द्विष्ठोऽस्ति । कारणपूर्विका तु कार्यंवृत्तिः\footnote{कार्यप्रवृत्तिः--\cite{dp-msA} \cite{dp-msB} \cite{dp-msC} \cite{dp-msD} \cite{dp-edP} \cite{dp-edH} \cite{dp-edE} \cite{dp-edN}} । अतो वास्तव एव । तद्वत् न निवृत्ते वस्तुनि कश्चित् \footnote{कश्चिदिष्टो \cite{dp-msA} \cite{dp-msB} \cite{dp-edP} \cite{dp-edH} \cite{dp-edN} कश्चिद् दृष्टो--\cite{dp-edE}}द्विष्ठो नाम विरोधोऽस्ति । दहननिमित्तं तु शीतस्पर्शस्य \footnote{क्षणान्तरासाम० \cite{dp-msA} \cite{dp-msB} \cite{dp-edP} \cite{dp-edH} \cite{dp-edE} \cite{dp-edN}}क्षणान्तरजननासामर्थ्यम् । अतो\footnote{ततो--\cite{dp-msC}} विरोधोऽपि वास्तव एव ॥
	\pend
       

	  \pstart उदाहरणमाह--
	\pend
       “

	  \pstart शीतोष्णस्पर्शवत् ॥ ७४ ॥
	\pend
      ” 

	  \pstart \footnote{शीतञ्चो० \cite{dp-msD}}शीतश्चोष्णश्च \textbf{तावेव} स्पर्शौ तयोरिव । शीतोष्णस्पर्शयोर्हिं पूर्ववद्विरोधो योजनीयः ॥
	\pend
       

	  \pstart द्वितीयमपि विरोधं दर्शयितुमाह--
	\pend
       “

	  \pstart परस्परपरिहारस्थितलक्षणतया \footnote{“वा” नास्ति--\cite{dp-msC}}वा \footnote{वा भाववत् \cite{dp-msB} \cite{dp-edP} \cite{dp-edH}}भावामाववत् ॥ ७५ ॥
	\pend
      ” 

	  \pstart परस्परस्य\footnote{परस्परं \cite{dp-msA} \cite{dp-msB} \cite{dp-edP} परस्परपरि० \cite{dp-edH} \cite{dp-edE}} परिहारः परित्यागस्तेन स्थितं लक्षणं रूपं ययोस्तद्भावः परस्परपरिहारस्थितलक्षणता\footnote{“तया” नास्ति--\cite{dp-msC}} तया ।
	\pend
       

	  \pstart इह यस्मिन् परिच्छिद्यमाने यद् व्यवच्छिद्यते तत् परिच्छिद्यमानमवच्छिद्यमानपरिहारेण स्थितरूपं द्रष्टव्यम् । नीले च परिच्छिद्यमाने ताद्रूप्यप्रच्युतिरवच्छिद्यते, तदव्यवच्छेदे नीलापरिच्छेदप्रसङ्गात् । तस्माद्वस्तुनो भावाभावौ परस्परपरिहारेण स्थितरूपौ । नीलात्तु यदन्यद्रूपं तन्नीलाभावाव्यभिचारि । नीलस्य दृश्यस्य पीतादावुपलभ्यमानेऽनुपलम्भाद-
	\pend
      ”

	  \pstart \textbf{कारणपूर्विका} कारणत्वेनाभिमतपदार्थसत्तापूर्विका कार्यंस्य कार्यत्वेनाभिमतस्य \textbf{वृत्तिः} प्रवृत्तिर्भाव इति यावत् । तुर्विशेषणार्थः । यत एव\textbf{मतो} हेतो\textbf{र्वास्तवः} पारमार्थिकः । \textbf{अन्यथा} कार्यकारणभावोऽप्यवास्तवोऽस्त्विति भावः ।
	\pend
      

	  \pstart ननु किं कार्यकारणभावोऽपि द्विष्ठः सम्बन्धः कश्चिदिष्टो येनैवमुच्यत इति चेत् । न । कारणपूर्विकायाः कार्यवृत्तेर्वास्तवत्वात् । इहापि तर्हि दहनादिनिमित्तं शीतस्पर्शादेर्जननासामर्थ्यं वास्तवमस्तु । न तु विरोधः सम्बन्ध इति चेत् । न । एतावतोऽन्यस्मात्कार्यकारणभावादस्य कार्यकारणभावस्य विशेषरूपत्वाभ्युपगमात् । अस्माभिरपीदृश एव कार्यकारणभावविशेषो विरोध इत्युच्यत इति कथमयमवास्तवः स्यादिति ॥
	\pend
      

	  \pstart पूर्वंवत्पूर्वोपदर्शितवत् ॥
	\pend
      \leavevmode\marginnote{\textenglish{204/dm}}“

	  \pstart भावनिश्चयात् । यथा च नीलं\footnote{नीलमभावं \cite{dp-msA} \cite{dp-msB} \cite{dp-msC} \cite{dp-edP} \cite{dp-edH}} स्वाभावं परिहरति, \footnote{तमिव अभाववत्--\cite{dp-msD-n}}तद्वद् अभावाव्यभिचारि पीतादिकमपीति\footnote{मपि । तथा \cite{dp-msA} \cite{dp-msB} \cite{dp-edP} \cite{dp-edH} \cite{dp-edE} \cite{dp-edN}} । तथा च भावाभावयोः साक्षाद्विरोधः,\footnote{विरोधौ \cite{dp-msA} क्षाद्विरोधः कः कस्य \cite{dp-msB}} वस्तुनोस्त्वन्योन्याभावाव्यभिचारित्वाद्विरोधः ।
	\pend
       

	  \pstart कस्य चान्यत्राभावावसायः ? यो नियताकारोऽर्थः,\footnote{र्थः, न तु \cite{dp-edE}} तस्य । न त्वनियताकारः,\footnote{०कारोऽर्थः क्ष० \cite{dp-msA} \cite{dp-msB} \cite{dp-edP} \cite{dp-edH} \cite{dp-edE} \cite{dp-edN}} क्षणिकत्वादिवत् । क्षणिकत्वं हि सर्वेषां नीलादीनां स्वरूपात्मकम् । अतो न नियताकारम् । \footnote{अतः \cite{dp-msD} \cite{dp-edE} “प्रत्यन्तरे “यतः” इति” इति \cite{dp-msD} प्रतौ टिप्पणं वर्त्तते ।}यतः क्षणिकत्वपरिहारेण न किञ्चिद् दृश्यते ।
	\pend
      ”

	  \pstart ननु सर्वमेव वस्तु सत्त्वरजस्तमोरूपेणैकमिति कथमन्योन्यरूपपरित्याग इत्याह--\textbf{इहे}ति । \textbf{यद् व्यवच्छिद्यते} यन्न परिच्छिद्यते । अपरिच्छेदस्यैव व्यवच्छेदरूपत्वात् । \textbf{अवच्छिद्यमानपरिहारेण} व्यवच्छिद्यमानपरिहारेण \textbf{स्थितं} व्यवस्थितं रूपं स्वरूपं यस्य तत्तथा । किम्पुनरिदं प्रसिद्धमित्याह--\textbf{नीलमि}ति । \textbf{चो} यस्मात् । तदेव रूपं तद्रूपम्, तद्रूपमेव ताद्रूप्यम् तस्य प्रच्युतिरभावो \textbf{व्य}वहर्त्तव्यैकरूपः प्रसज्यप्रतिषेधात्मा तुच्छरूपः । उपपत्तिमाह--\textbf{तदव्यवच्छेद} इति । यत एवं \textbf{तस्मात्} कारणात् । यदि भावाभावयोर्विरोधः, न तर्हि नीलपीतयोः स स्यादित्याह—\textbf{नीलादि}ति । तुर्विशेषद्योतकः । अभावाव्यभिचारित्वमेव साधयन्नाह--नीलस्येति । भवत्येवं नीलस्य पीतादावभावः, न तु तत्परिहारेण तद् व्यवस्थितमित्याह--यथेति । \textbf{चो} यस्मादर्थे । नीलं कर्त्तृ स्वाभावं स्वभावं \footnote{?} स च मानभावश्च \footnote{?} तं न व्यभिचरतीति तथा । एवं सति किं व्यवस्थितमित्याह--\textbf{तथा} चेति नीलस्य साक्षात्स्वाभावपरिहारप्रकारे तदव्यभिचारित्वादर्थान्तरपरिहारप्रकारे च सति ।
	\pend
      

	  \pstart नतु \footnote{नु} यद् यदभावाव्यभिचारि य \footnote{त} त्तत्तेन विरुद्ध्यते । तस्य न तदात्मकत्वेनाभावावसायस्तादात्म्याभाव\add{आ}वसायफलत्वादन्यस्य विरोधस्य । तर्हिक्षणिकत्वमपि नीलाभावाव्यभिचारित्वान्नीलेन विरुद्ध्यमानं न नीलात्मकं स्यात् । तदपि नीलाभाववदेव अन्यथा क्षणिकत्वं नीलात्मतैव स्यात् । तथा च यावत्क्षणिकं तावन्नीलमिति कृत्स्ना त्रिलोकी नीलैव स्यादिति मनसि निधायाह--\textbf{कस्य चेति} । तुशब्दार्थश्चकारः ।
	\pend
      

	  \pstart परमु\leavevmode\marginnote{\textenglish{70b/ms}}खेन प्रश्नं कृत्वा प्रश्नविसर्जनमाह--य इति । \textbf{नियत}स्य प्रतिनियतस्य \textbf{वस्तुन आकारः} स्वरूपमिति विग्रहीतव्यम् । एतदेब व्यतिरेकमुखेणाह--\textbf{न त्वि}ति । न पुनरनियतस्य सर्ववस्तुस्वरूपात्मकस्य । तदेव दर्शंयति--\textbf{क्षणिकत्वादिवदिति । आदि}शब्दात्परमाणुमयत्वादिपरिग्रहः । अनेनैतदाह--अभावाव्यभिचारित्वेऽपि नियताकारेण तेन सम\add{म}स्य विरोधो नानियताकारेणेति ।
	\pend
      

	  \pstart अनियताकारत्वमस्योपपादयन्नाह--\textbf{क्षणिकत्वं} हीति । हिर्यस्मादर्थे । सन्मात्रानुबन्धित्वात्क्षणिकस्येत्यभिप्रायः । यत एव्र\textbf{मतः} कारणात् । न त्वयमर्थः--\textbf{नियतः} प्रतिनियत  \leavevmode\marginnote{\textenglish{205/dm}} “
	  
	यद्येवमभावोऽपि न नियताकारः । \footnote{कथं न निय० \cite{dp-msA} \cite{dp-msB} \cite{dp-edP} \cite{dp-edH} \cite{dp-edN}}कथमनियताकारो नाम ? यावता वस्तुरूपविविक्ताकारः कल्पितोऽभावः । ततो दृष्टं कल्पितं वा नियतं रूपमन्यत्रासदित्यवसीयते\footnote{०त्राऽसदवसीयते \cite{dp-msA} \cite{dp-msB} \cite{dp-msD} \cite{dp-edP} \cite{dp-edH} \cite{dp-edE} \cite{dp-edN}} । नानियतम् । एवं \footnote{नित्यत्वे पिशा० \cite{dp-msA} \cite{dp-edP} \cite{dp-edH}}नित्यत्वपिशाचादिरपि नियताकारः कल्पितो द्रष्टव्यः । एकात्मत्व-\footnote{एकात्मकत्व \cite{dp-msA} \cite{dp-msB} \cite{dp-msC} \cite{dp-msD} \cite{dp-edP} \cite{dp-edH} \cite{dp-edE} \cite{dp-edN}} विरोधश्चायम् । ययोर्हि \footnote{परस्परेणाव० \cite{dp-msB}}परस्परपरिहारेणावस्थानं तयोरेकत्वाभावः ।” \textbf{आकारो} यस्येति । एवं हि क्षणिकत्वादेरपि नियताकारत्वं स्यात् । तथाहि परमसङ्कुचितकालवर्त्तिरूपत्वेन नियताकारत्वात् । \textbf{धर्मोत्तरो}ऽपि \textbf{क्षणिकत्वं} हि \textbf{सर्वेषां नीलादीनां स्वरूपात्मकमि}ति ब्रुवाणो \textbf{नियताकार} इत्यत्र षष्ठीतत्पुरुषमभिव्यनक्ति इतरथा “क्षणिकत्वस्य हि सर्वो नीलादिः स्वरूपम्” इत्यभिदध्यादिति । एवञ्चा \footnote{च} क्षणिकस्यापि न नीलेऽभावावसायस्तस्य सर्वनीलादिवस्त्वात्मकत्वेनानियताकारत्वात् । येन च क्षणिकं कल्पितं न तेन प्रतिनियतवस्त्वात्मक \footnote{कं} कल्पितमत एव नीलग्राहि प्रत्यक्षं क्षणिकत्वाक्षणिकत्वयोरुदासीनं नीलमात्रे प्रमाणम् । तथा च नीलस्याक्षणिकत्वपरिहारेणावस्थानं क्षणिकत्वसिद्धेः प्राङ् निश्चेतुमशक्यमिति न्यायबलात्प्राप्तम् ।
	\pend
      

	  \pstart एवञ्चाविरोधरूपविवेचके \textbf{धर्मोत्तरे} सत्त्यपि ये केचिद् द्विष्यमानजल्पमहोदधिप्रभृतयो विरोधचोद्यपरिजिहीर्षया परस्परपरिहारस्थितलक्षणं विरोधं परिहारीकुर्वन्ति तैरयं “कस्य \textbf{चान्यत्राभावावसा\add{यः ।}यो नियताकारो न त्वनियताकारः क्षणिकत्वादिरि}ति” धर्मोत्तरस्य ग्रन्थो न दृष्टो न चार्थस्य समीचीनिधिस \footnote{?} ज्ञात इति लक्ष्यते ।
	\pend
      

	  \pstart यद्ययमपरिहारः, कस्तत्र परिहार इति चेत् । यथैतत्परिह्रियते तथा \textbf{विशेषाख्यान} एवास्माभिरभ्यधायीति तत एवापेक्षितव्यम् । इह पुनरप्रकृतत्वान्नोच्यत इति ।
	\pend
      

	  \pstart यदि नियताकारं वस्तु परिहरति नानियताकारम्, तर्हि भावो नाभावं परिहृत्य तिष्ठेत् तस्यानियताकारत्वाद् इत्याशङ्कमान आह--\textbf{यद्येवमिति} । एवञ्चेदभ्युपगम्यते तदेत्यर्थात् । न केवलं क्षणिकत्वादीत्यपि शब्दः । कथमिति सिद्धान्ती ।
	\pend
      

	  \pstart \textbf{यावते}ति तृतीयान्तप्रतिनिरूपको निपातोऽत्र यस्मादित्यस्यार्थे वर्त्तते । \textbf{वस्तुरूपविविक्तो} दृश्यनीलादिस्वभावरहित \textbf{आकारो} यस्येति विग्रहः ।
	\pend
      

	  \pstart कल्पितग्रहणेनैतदाह--नाभावो नाम कश्चित् प्रमाणसिद्धोऽस्ति । केवलं कल्पिकया बुद्ध्या तथा समारोपित इति । यत एवं तस्माद् \textbf{दृष्टं} प्रमाणावगतं \textbf{कल्पितम्} आगमाश्रयेणान्यथा वा समारोपितम् । \textbf{अन्यत्र} ततोऽन्यस्मिन्, नीलादौ वाऽनियतम्, न नीलाद्यात्मकं सत् तत्रैवाऽस\add{दि} त्यवसीयते । अमुमेव न्यायमन्यत्रादिशन्ना\leavevmode\marginnote{\textenglish{71a/ms}}ह--एवमिति । यथाऽभावो नियताकार एव कल्पितो नानियताकार एवं \textbf{नित्यत्वमपि} सर्वकालावस्थायित्वलक्षणं \textbf{नियताकारमेव}  \leavevmode\marginnote{\textenglish{206/dm}} “
	  
	अत एव लाक्षणिकोऽयं विरोध उच्यते । लक्षणं रूपं वस्तूनां प्रयोजनमस्येति कृत्वा । विरोधेन ह्यनेन वस्तुतत्त्वं विभक्तं व्यवस्थाप्यते । अत एव दृश्यमाने रूपे यन्निषिध्यते तद् दृश्यमेवाभ्युपगम्य निषिध्यते । तथा हि--अभावोऽपि पिशाचोऽपि यदा पीते निषेद्धुमिष्यते तदा दृश्यात्मतया निषेध्य इति दृश्यत्वमभ्युपगम्य दृश्यानुपलब्धेरेव निषेधः । तथा च सति रूपे परिच्छिद्यमान एकस्मिंस्तदभावो दृश्यो व्यवच्छिद्यते । \footnote{अथ यत्तदभाववत् पा [[पी]] तादि तत् कथं व्यवच्छिद्यते इत्याह--\cite{dp-msD-n}}यच्च तदभाववन्नियताकारं” \textbf{कल्पितम्,} न तु सर्वस्य नीलादेः । अक्षणिकत्वं तु सर्वस्य नीलादेः स्वरूपात्मकं सर्वस्यैवानेकक्षणस्थायित्वात् । न चाक्षणिक एव नित्यः, सतोऽकारणस्याकाशादेः कियत एव तथात्वात् । तथा \textbf{पिशाचत्वमप्य}स्थिस्नायुमयसूचीवक्त्रादिरूपस्यैव स्वरूपं \textbf{कल्पितमि}ति । तस्यापि नीलाकारत्वान्नीलादिना \footnote{ता} । \textbf{अयं} विरोधः । यद्वा नियतः प्रतिनियत आकारः स्वभावो यस्य स तथा निषेधेना\textbf{नियताकारः} । तदा तु सर्वनीलाद्यनात्मकत्वसर्वनीलाद्यात्मकत्वे नियताकारत्वामियताकारत्वे वाच्ये । तेन न क्षणिकत्वादौ नियताकारत्वस्य प्रसङ्गः । \textbf{सर्देषां स्वरूपात्मकमिति} च विवरणमर्थाभेदेन नेयमिति ।
	\pend
      

	  \pstart ननु चानेनापि विरोधेन विरोधिनोः सहावस्थानं निषिध्यते । पूर्वेणापि परस्परपरिहारावस्थानं प्रतिपाद्यत इति कथमन्योन्यानन्तर्भाव इत्याशङ्क्याह--\textbf{एकात्मत्वेति । चो} यस्मादर्थे । विरुद्धयोरेकात्मनिषेधको विरोध \textbf{एकात्मविरोध} उक्तः । कथमस्य तथात्वमित्याह--\textbf{ययोरि}ति । हिर्यस्मादर्थे ।
	\pend
      

	  \pstart यत एतेन विरोधेन विरुद्धयोरेकात्मत्वं निषिध्यते \textbf{अत एवा}स्मादेव कारणात् । कथमीदृशो विरोधो भवता \textbf{लाक्षणि}कशब्देनाभिधीयते इत्याह--\textbf{लक्षण}मिति । विभक्तस्वरूपं \textbf{प्रयोजनं} व्यवस्थाप्यतया साध्यम्, प्रयुज्यते अनेन इति वा \textbf{प्रयोजनं} प्रयोजकस्य । \textbf{इति कृत्वा} एवं व्युत्पाद्य ईदृश्या व्युत्पत्त्येति यावत् ।
	\pend
      

	  \pstart कथमेतत्प्रयोजनमित्याह--\textbf{विरोधेने}ति । \textbf{ही}ति यस्मात् । \textbf{विभक्त}मन्येन विभक्तं यतोऽ\textbf{नेनेति विरोधेन} नीलादेर्विभक्तरूपव्यवस्थापनादन्येन सहैकात्म्यं निषिध्यते । \textbf{अत एवा}स्मादेव कारणात् । \textbf{दृश्यमाने रूपे} प्रतीयमाने वस्तुस्वरूपे \textbf{यन्निषिध्यते} दृश्यमानात्मकत्वेन प्रतिषिध्यते ।
	\pend
      

	  \pstart ननु दृश्ये वस्तुनि दृश्यान्तरस्य दृश्यत्वाभ्युपगमपूर्वको निषेधो युक्तो न त्वदृश्यस्येत्याशङ्क्याह--\textbf{तथा ही}ति । न केवलं भाव इत्यपिशब्दः । न केवलमभाव इत्यपिशब्दः । \textbf{दृश्यात्मतया} दृश्यपीतात्मतया \textbf{निषेध्यो} निषेधार्हः, नायं दृश्यमानः पीतः, अभावः, पिशाचो वा ताद्रूप्येणाप्रतिभासनादित्येवं निषेधादित्यभिप्रायः । इतिर्हेतौ । \textbf{दृश्यानुपलब्धेरे}वान्यस्य तादात्म्येनान्यस्मि\textbf{न्निषेधः} ।
	\pend
      

	  \pstart अथ स्यात्प्रत्यक्षमेवात्र नीलस्य पीतात्मताऽभावव्यवहारं करोति । तत् किमेवमुच्यते ? अथोक्तमेतददृष्टानामपि सत्त्वसंज्ञया न शक्तो व्यवहारयितुमिति चेत् । न । इह तादात्म्य \leavevmode\marginnote{\textenglish{207/dm}} “
	  
	रूपं तदपि दृश्यं व्यवच्छिद्यते । ततः स्वप्रच्युतिवत् प्रच्युतिमन्तोऽपि व्यवच्छिन्ना इति ये परस्परपरिहारस्थितरूपाः सर्वे तेऽनेन निषिद्धैकत्वा इति । सत्यपि चास्मिन् विरोधे सहावस्थानं स्यादपि । 
	  
	ततो भिन्नव्यापारौ विरोधौ । एकेन विरोधेन शीतोष्णस्पर्शयोरेकत्वं वार्यते । अन्येन सहावस्थानम् । भिन्नविषयौ\footnote{भिन्नप्रवृत्तिविषयौ \cite{dp-msA} \cite{dp-edP} \cite{dp-edH} \cite{dp-edN}} च । सकले वस्तुन्यवस्तुनि च परस्परपरिहारविरोधः । वस्तुन्येव कतिपये \footnote{सहावस्थान० \cite{dp-msB} \cite{dp-msC}}सहानवस्थानविरोधः । तस्माद्भिन्नव्यापारौ भिन्नविषयौ च । ततो नानयोरन्योन्यान्तर्भाव इति ॥” “
	  
	स च द्विविधोऽपि विरोधो वक्तृत्वसर्वज्ञत्वयोर्न सम्भवति ॥ ७६ ॥” निषेधात् । आधेयनिषेधे ह्ययं न्यायो न तु दृश्यमानात्मतानिषेध इति । सत्यमेतत् । केवलमत्यन्तमूढं प्रत्येतदुक्तमित्यदोषः । किमेवं सति सिद्धिमित्याह--\textbf{तथा च तति} ऐकात्म्यनिषेधे सर्वस्य दृश्यात्मतया निषेधप्रकारे सति । \textbf{तदभाव}स्तस्य परिच्छिद्यमानस्य स्वरूपस्य नीलादेरभावस्त\textbf{दभावो दृश्यो} दृश्यात्म\leavevmode\marginnote{\textenglish{71b/ms}}कः सन् \textbf{व्यवच्छिद्यते} तादात्म्येन निषिध्यते । अयं दृश्यमानो नीलो नाभावः तुच्छरूपेण अभावरूपेणाप्रतिभासनादिति कृत्वा दृश्यमानरूपात्मतया निषेधादिति भावः ।
	\pend
      

	  \pstart भवतु परिच्छिद्यमानाऽभावस्य दृश्यस्य व्यवच्छेदस्तदव्यभिचारिणस्तु निषेधे का वार्त्तेत्याह--\textbf{यच्चेति} । अपिशब्दार्थश्चकारः । \textbf{तदभावो} विद्यतेऽस्येति तथा ।
	\pend
      

	  \pstart यदि तदभाववांस्तादात्म्येन प्रतिषिध्यते तर्हि क्षणिकत्वमपि पूर्वोक्तेन न्यायेन नीलाभाववदिति तदपि तादात्म्यतया व्यवच्छेद्यं स्यादित्याह--\textbf{नियताकारमिति} । एतच्च पूर्वमेव कृतव्याख्यानम् । यतो द्वयोरप्यभावतद्वतोर्दृश्यमानात्मतया निषेधाद् द्श्ययोरेव निषेधस्ततस्तस्मात्\textbf{स्वप्रच्युतिरिव} स्वाभाव इव व्यवच्छिन्ना निषिद्धतादात्म्याः । इतिस्तस्मात् । \textbf{सर्व}ग्रहणं कार्त्स्न्यप्रतिपादनार्थम् । अनेनेति विरोधेन \textbf{निषिद्धमेकत्वं} येषामिति विग्रहः । \textbf{विरोधे} परस्परपरिहारस्थितात्मलक्षणे सहैकत्र लोकप्रतीतिसिद्धे देशेऽवस्थानं स्थितिः स्यात् । \textbf{अपिः} सम्भावनायाम् । यस्मादनेन विरोधेन नैकत्रावस्थानं निषिद्ध्यते किन्त्वेकात्मकत्वम् । पूर्वेण चैकत्रावस्थानं न त्वेकात्मकत्वम् ।
	\pend
      

	  \pstart \textbf{ततः} कारणाद् \textbf{भिन्नौ} नानाभूतौ \textbf{व्यापारौ} ययोस्तौ तथोक्तौ । भिन्नव्यापारत्वमेवानयोरेकेत्यादिना स्फुटयति । न केवलं व्यापारभेदादनयोर्भेदेनोपन्यासः । किन्तु विषयभेदादपीत्याह--\textbf{भिन्नविषयौ} चेति । न केवलं भिन्नव्यापारौ भिन्नविषयावपीत्यपिशब्दार्थश्चकारः । भिन्नविषयत्वमेव दर्शयन्नाह--\textbf{सकल} इति ॥
	\pend
      

	  \pstart भवतूक्तलक्षणो द्विविध एव विरोधः । तथाप्यनयोरन्यतर एव विरोधो वक्तृत्वसर्वज्ञत्वयोर्भविष्यतीत्याह--\textbf{स चे}ति । \textbf{चो} यस्मात्सोऽयमनन्तरोक्तो \textbf{द्विविधो} नास्ति । अपिरतिशये ।
	\pend
      \leavevmode\marginnote{\textenglish{208/dm}}“

	  \pstart स चायं द्विविधोऽपि विरोधो वक्तृत्वं च सर्वज्ञत्वं च तयोर्न सम्भवति । न ह्यविकलकारणस्य सर्वज्ञत्वस्य वक्तृत्वभावादभावगतिः\footnote{गतिरिति--\cite{dp-msD}} । सर्वज्ञत्वं ह्यदृश्यम् । अदृष्टस्य चाभावो नावसीयते । ततो नानेन विरोधगतिर्भवति ।
	\pend
       

	  \pstart न च वक्तृत्वपरिहारेण सर्वज्ञत्वमवस्थितम् । काष्ठादयो हि\footnote{०दयोऽपि वक्तृ० \cite{dp-msA} \cite{dp-msB} \cite{dp-edP} \cite{dp-edH} \cite{dp-edE} \cite{dp-edN} ०दयोऽपि हि वक्तृ \cite{dp-msC}} वक्तृत्वपरिहृताः । तेषामपि सर्वज्ञत्वप्रसङ्गात् । नापि सर्वज्ञत्वपरिहारेण वक्तृत्वम् । काष्ठादीनामपि वक्तृत्वप्रसङ्गात् । तत एवाविरोधाद् वक्तृत्व\footnote{विधाने न सर्व० \cite{dp-msA} \cite{dp-msB} \cite{dp-edP} \cite{dp-edH} \cite{dp-edE} \cite{dp-edN}} विधेर्न सर्वज्ञत्वनिषेधः ॥
	\pend
       

	  \pstart स्यादेतत्--यदि नास्त्येव विरोधो घटपटयोरिव स्यादपि तयोः सहावस्थितिदर्शनम्\footnote{०र्शनम् । अदर्शनात् तु--\cite{dp-msA} \cite{dp-msB} \cite{dp-msD} \cite{dp-edP} \cite{dp-edH} \cite{dp-edE} \cite{dp-edN}} । सहाव\unclear{वस्थित्यदर्शनात्तु विरोधगतिः । विरोधा\footnote{विरोधादभा \cite{dp-msB} \cite{dp-msD}}च्चाभावगतिरित्याशङ्क्याह--}
	\pend
      ”“

	  \pstart न\footnote{न च विरु० \cite{dp-msC}} चाविरुद्धविधेरनुपलब्धावप्यभावगतिः ॥ ७७ ॥
	\pend
      ”

	  \pstart तत्राद्यस्य तावदभावं \textbf{न ही}त्यादिना दर्शयति । हीति यस्मात् । कुतो नाभावगतिरित्याह--\textbf{सर्वज्ञत्वं} हीति । हिर्यस्मात् । अदृश्यस्यापि किं माभावगतिरित्याह--\textbf{अदृष्टस्}येति । \textbf{चो} यस्मादर्थे । यत एवं ततस्तस्मात् । अनेन वक्तृत्वेन विरोधगतिर्नास्ति तस्य सर्वज्ञत्वस्येत्यर्थात् ।
	\pend
      

	  \pstart द्वितीयस्य विरोधस्य \add{अभावं} प्रतिपादयन्नाह--\textbf{न चे}ति । \textbf{चः} प्रतिषेधसमुच्चये । उपपत्तिमाह--\textbf{काष्ठे}ति । हिर्यस्मात् । कुतो वक्तृत्वपरिहारेण सर्वज्ञत्वं नाबस्थितमित्याह—\textbf{तेषामपि} काष्ठादीनां सर्वज्ञत्वस्य प्रसङ्गात्प्रसक्तेः । कुतस्तेषां तथात्वप्रसङ्ग इत्याशङ्क्य योजनीयं \textbf{काष्ठादय} इति । हिर्यस्मात् । \textbf{काष्ठादयो वक्तृत्वेन} वचनशक्त्या \textbf{परिहृता}स्त्यक्ताः । वक्तृत्वमेव सर्वज्ञत्वपरिहारेण व्यवस्थितं भविष्यतीत्याह--\textbf{नापीति । अपिः} प्रतिषेधसमुच्चये । अत्रापि तामेवोपपत्तिमाह--काष्ठेति । इहापि काष्ठादयो हि सर्वज्ञत्वेन परिहृता इति द्रष्टव्यम् ।
	\pend
      

	  \pstart स्यादेतत्--वक्तृत्वसर्वज्ञत्वयोः परस्परपरिहारस्थितलक्षणताविरोधेऽपि का क्षतिर्येन तन्निषेधः कृतः । तथा ह्यन्योन्यपरिहारेणावस्थानेऽपि यद् वक्तृत्वं तत्सर्वज्ञत्वं मा भूत्, सर्वज्ञत्वं वा वक्तृत्वम् । तयोस्त्वेकत्रस्थितिरविरुद्धैव । तदनन्तरमेवोक्तं \textbf{धर्मोत्तरेण} \leavevmode\marginnote{\textenglish{72a/ms}} \textbf{सत्यपि चास्मिन् विरोधे सहावस्थानं स्यादपी}ति ।
	\pend
      

	  \pstart तदेतदसत् । येषां हि वक्तृत्वसर्वज्ञत्वलक्षणौ धर्मिणो भिन्नावेव धर्मौ तेषामिदं शोभते \textbf{न तु ताथागतानां} धर्मधर्मिणोर्वास्तवमभेदमिच्छताम् । तथा हि यदि वक्तृत्वं सर्वज्ञत्वपरि  \leavevmode\marginnote{\textenglish{209/dm}} “
	  
	न चाविरुद्धविधेरिति अनुपलब्धावपि नायं विरुद्धविधिः यद्यपि \footnote{यद्यपि च \cite{dp-msA} \cite{dp-msB} \cite{dp-edP} \cite{dp-edH} \cite{dp-edE} \cite{dp-edN}}सहावस्थानानुपलम्भस्तथापि न तयोर्विरोधः । यस्मान्न सहानुपलम्भमात्राद् विरोधोऽपि तु द्वयोरुपलभ्यमानयोर्निव\footnote{निर्वत्य \cite{dp-msA}} र्त्यनिवर्तकभावावसायात् । तस्मादनुपलब्धावपि न वक्तृत्वविधे र्विरुद्धविधिः ।\footnote{वक्तृत्वविरोधिविरुद्धविधिः \cite{dp-msA} \cite{dp-msB} \cite{dp-edP} \cite{dp-edH} \cite{dp-edN}} अतोऽस्मान्नान्यस्याभावगतिः ॥ 
	  
	तथा न वक्तृत्वाद् रागादिमत्त्वगतिः । यतो यदि वचनादि रागादीनां कार्यं स्याद्वचनादे रागादिगतिः स्यात् । रागादिनिवृत्तौ वचनादिनिवृत्तिः\footnote{वचननिवृ० \cite{dp-msB} \cite{dp-msC} \cite{dp-msD}} स्यात् । न च कार्यम् । कुतः-- “
	  
	रागादीनां वचनादेश्च कार्यकारणभावासिद्धेः ॥ ७८ ॥” 
	  
	रागादीनां वचनादेश्च कार्यकारणभावस्याऽसिद्धेः कारणान्न कार्यम् । अतोऽस्मान्न गतिः ॥ 
	  
	मा भूद्रागादिकार्यं वचनं सहचारि तु भवति । ततो रागादौ सहचारिणि निवृत्ते निवर्त्तंते\footnote{निवर्त्तिष्यते \cite{dp-msC}} वचनमित्याशङ्क्याह-- “
	  
	अर्थान्तरस्य \footnote{वा कारण० \cite{dp-msA} \cite{dp-msB} \cite{dp-edP} \cite{dp-edH} \cite{dp-edE} वाऽकारण० \cite{dp-edN}}चाकारणस्य निवृत्तौ न वचनादेर्निवृत्तिः ॥ ७९ ॥” 
	  
	अर्थान्तरस्य\footnote{वा कारण० \cite{dp-msA} \cite{dp-msB} \cite{dp-edP} \cite{dp-edH} \cite{dp-edE} वाऽकारण० \cite{dp-edN}} चाकारणस्य निवृत्तौ सहचारित्वदर्शनमात्रेण नान्यस्य वचनादेर्निवृत्तिः । अतो वक्तृत्वं भवेद्रागादिविरहश्च ॥ “
	  
	इति सन्दिग्ध\footnote{सन्दिग्धव्यावृत्ते [[त्ति]] काऽनै० \cite{dp-msC}} व्यतिरेकोऽनैकान्तिको वचनादिः ॥ ८० ॥”” हृतं स्यात् तदा वक्तृत्वस्य वक्तुरभेदात्सर्वज्ञत्वस्य च सर्वज्ञाद्, वक्तैव सर्वज्ञो न स्यात्, सर्वज्ञ एव वा वक्तेति युक्तमनयोः परस्परपरिहारस्थितलक्षणताऽभावप्रतिपादनमिति ।
	\pend
      

	  \pstart \textbf{यदि नास्त्येव विरोधः} सहानवस्थानलक्षण इति द्रष्टव्यम् । अन्यथा \textbf{घटपटयो}र्दृष्टान्तता न स्यात् । उपसंहारे चायमर्थो व्यक्तीकरिष्यते । उक्तं विरुद्धत्वे सहावस्थानादर्शनं कारणम्, तत्किमेवमुच्यत इत्याशङ्क्याह--\textbf{यद्यपीति । तथापि} तेनापि प्रकारेण । \textbf{अतो} हेतो\textbf{रस्मा}द् वक्तृत्वाद् \textbf{अन्यस्य} सर्वज्ञत्वस्य \textbf{नाभाव}प्रतिपत्तिः ॥
	\pend
      

	  \pstart यथा वक्तृत्वादसर्वज्ञत्वगतिर्नास्ति तथा वक्तृत्वाद् रागादिमत्त्वस्यापि गतिर्नास्तीति जिज्ञापयिषुराह--\textbf{तथे}ति । आदिग्रहणाद् द्वेषादिपरिग्रहः ।
	\pend
      

	  \pstart \textbf{रागा}दिशब्दसन्निधानाद्रागादि \textbf{सहचारी}ति बोद्धव्यम् । यथा सन्दिग्धविपक्षव्यावृत्तिकसाधनदोषस्तथा प्रागेवाभिहितमिति न पुनरुच्यते ॥
	\pend
      \leavevmode\marginnote{\textenglish{210/dm}}“

	  \pstart इति शब्दस्तस्मादर्थे । \footnote{तस्मादसर्वज्ञत्वावीतरागत्वविपर्ययात् विपक्षात् सर्वज्ञत्वा [[त्व]] वीतरागादिमत्त्वात् सन्दिग्धो--\cite{dp-msB} तस्मादसर्वज्ञरागादिमत्त्वविपर्ययात् विपक्षात् सर्वज्ञत्वादरागादिमत्त्वाच्च सन्दिग्धो \cite{dp-msC}}तस्मादसर्वज्ञत्वविपर्ययाद् विपक्षात्सर्वज्ञत्वाद्, रागादिमत्त्वविपर्ययादरागादिमत्त्वात् सन्दिग्धो व्यतिरेको वचनादेः ।\footnote{वचनादिः \cite{dp-msC}} अतोऽनैकान्तिको वचनादिः ॥
	\pend
       

	  \pstart एवमेकैकरूपादिसिद्धिसन्देहे हेतुदोषान् आख्याय द्वयोर्द्वयो रूपयोरसिद्धिसन्देहे हेतुदोषान् वक्तुकाम\footnote{कामा \cite{dp-msC}} आह--
	\pend
       “

	  \pstart द्वयो रूपयोर्विपर्ययसिद्धौ विरुद्धः ॥ ८१ ॥
	\pend
      ” 

	  \pstart द्वयोरिति\footnote{द्वयोः । \cite{dp-msD}} । द्वयो रूपयोर्विपर्ययसिद्धौ सत्त्यां विरुद्धः ॥
	\pend
       

	  \pstart त्रीणि च रूपाणि सन्ति । ततो विशेषज्ञाप\footnote{ज्ञानार्थ० \cite{dp-msC}}नार्थमाह--
	\pend
       “

	  \pstart कयोर्द्वयोः ? ॥ ८२ ॥
	\pend
      ” 

	  \pstart कयोर्द्वयोरिति ॥
	\pend
       

	  \pstart विशिष्टे रूपे दर्शयति--
	\pend
       “

	  \pstart सपक्षे सत्वस्य, असपक्षे चासत्वस्य । यथा कृतकत्वं\footnote{स्य च कृत० \cite{dp-msC}} प्रयत्नानन्तरीय कत्वं च नित्यत्वे साध्ये विरुद्धो हेत्वाभासः ॥ ८३ ॥
	\pend
      ” 

	  \pstart सपक्षे सत्त्वस्य, असपक्षे चासत्त्वस्य विपर्ययसिद्धाविति\footnote{सिद्धिरिति \cite{dp-msC}} सम्बन्धः । कृतकत्वमिति स्वभावहेतुः । प्रयत्नानन्तरीयकत्व मिति कार्यहेतुः\footnote{कार्यहेतोः \cite{dp-edH}} । प्रयत्नानन्तरीयक\footnote{०कत्वश० \cite{dp-msC}} शब्देन हि प्रयत्नानन्तरं जन्म ज्ञानं च प्रयत्नानन्तरीयकमुच्यते । जन्म जायमानस्य स्वभावः । ज्ञानं ज्ञेयस्य कार्यम् । तदिह प्रयत्नानन्तरं \footnote{प्रयत्नानन्तरीयकत्वग्राहकम्--\cite{dp-msD-n}}ज्ञानं गृह्यते । \footnote{ततः का० \cite{dp-msC} \cite{dp-msD}}तेन कार्यहेतुः ।
	\pend
      ”

	  \pstart सर्वानैकान्तिकप्रकारानुक्त्वा विरुद्धत्वाख्यं हेतुदोषमभिदधानो वार्तिककारस्याभिप्रायम् \textbf{एवमि}त्यादिना दर्शयति ।
	\pend
      

	  \pstart व्यक्तिभेदविवक्षया हेतुदोषानिति बहुवचनेनाह ॥
	\pend
      

	  \pstart \textbf{विशिष्टे} रूपे द्वे \textbf{दर्शयति} । यदि द्वावपि स्वभावहेतू तदा किं द्वयाभिधानेन ? एकेनापि स्वभावहेतुना विरुद्धत्वस्य दर्शितत्वादित्याशङ्क्य समर्थनमाह--\textbf{कृतकत्वमि}ति । ननु स्वभावहेतुत्वेन प्रयत्नानन्तरीयकत्वमन्यत्र दर्शितम् । तत्कथमनेनैवमुच्यत इत्याह--\textbf{प्रयत्नेति} । \textbf{हिर्य}स्मात् । प्रयत्नस्य पुरुषव्यापारस्या\textbf{नन्तरम}व्यवहितं \textbf{जन्म ज्ञानं च} तद्विषयमुच्यते ।
	\pend
      [[चेति \cite{dp-msC}]]\leavevmode\marginnote{\textenglish{211/dm}}“

	  \pstart एतौ हेतू नित्यत्वे साध्ये विरुद्धौ हेत्वाभासौ ॥
	\pend
       

	  \pstart कस्मात् पुनरेतौ विरुद्धावित्याह--
	\pend
       “

	  \pstart अनयोः सपक्षेऽसत्वम्, असपक्षे च सत्वमिति विपर्ययसिद्धिः\footnote{०द्धिरिति--\cite{dp-msC}} ॥ ८४ ॥
	\pend
      ” 

	  \pstart अनयोरिति । सपक्षे हि\footnote{“हि” नास्ति \cite{dp-msA} \cite{dp-msB} \cite{dp-edP} \cite{dp-edH} \cite{dp-edE} \cite{dp-edN}} नित्ये कतकत्वप्रयत्नानन्तरीयकत्वयोरसत्त्वमेव निश्चितम् । अनित्ये विपक्षे एव सत्त्वं निश्चितमिति विपर्ययसिद्धिः ॥
	\pend
       

	  \pstart कस्मात् पुनर्विपर्ययसिद्धावप्येतौ विरुद्धावित्याह--
	\pend
       “

	  \pstart एतौ च साध्यविपर्ययसाधनाद्विरुद्धौ ॥ ८५ ॥
	\pend
      ” 

	  \pstart एतौ च साध्यस्य नित्यत्वस्य विपर्ययम्--अनित्यत्वं साधयतः । ततः\footnote{“ततः” नास्ति--\cite{dp-msA} \cite{dp-msB} \cite{dp-edP} \cite{dp-edH} \cite{dp-edN}} साध्यविपर्ययसाधनाद्विरुद्धौ ॥
	\pend
       

	  \pstart यदि साध्यविपर्ययसाधनाद्विरुद्धावेतौ, उक्तं च परार्थानुमाने साध्यम्, न त्वनुक्तम्; इष्टं च अनुक्तम् । अतोऽन्य इष्टविघातकृदाभ्यामिति दर्शयन्नाह--
	\pend
       “

	  \pstart \footnote{तत्र च--\cite{dp-msB} \cite{dp-edP} \cite{dp-edH}}ननु च तृतीयोऽपीष्टविघातकृद् विरुद्धः ॥ ८६ ॥
	\pend
      ” 

	  \pstart ननु च तृतीयोऽपि विरुद्ध उक्तः\footnote{प्रमाणविनिश्चयादौ--\cite{dp-msD-n}} । उक्तविपर्ययसाधनौ द्वौ । तृतीयोऽयमिष्टस्य शब्देनानुपात्तस्य \footnote{विधानं--\cite{dp-msA} \cite{dp-msB} \cite{dp-edP} \cite{dp-edH}}विघातं करोति विपर्ययसाधनादिति इष्टविघातकृत् ॥
	\pend
       

	  \pstart तमुदाहरति--
	\pend
      ”“

	  \pstart यथा परार्थाश्चक्षुरादयः सङ्घातत्वाच्छयनासनाद्यङ्गवदिति ॥ ८७ ॥
	\pend
      ”

	  \pstart ननु \textbf{प्रयत्नानन्तरीय}क\add{त्व}शब्द उपात्तस्तत्किंप्रयत्नानन्तरीयकशब्दस्यार्थ उच्यत इत्याह--\textbf{प्रयत्नानन्तरीयकत्व}मिति प्रयत्नानन्तरीयकत्वशब्देनापि तदेवोक्तं वस्तुत इत्यर्थः । तेन शब्देन द्वयस्याभिधाने कदा स्वभावस्याभिधानं कदा कार्यस्येत्याशङ्क्य \textbf{जन्मे}त्यादिना विभजते । यतो ज्ञानस्याप्यभिधानं \textbf{तत्त}स्मा\textbf{दिह} विरुद्धोदाहरणप्रक्रमे ।
	\pend
      

	  \pstart अनुपलम्भस्त्वनयोरेवान्तर्भूतत्वान्न पृथगुपदर्शितः । एतयोरुदाहृतयोरेव सोऽपि सुज्ञान इति भाव उन्नेयः ।
	\pend
      

	  \pstart लक्ष्यते चाय\textbf{माचार्यं}स्याशयः--सपक्षावृत्तौ सत्यां व्याप्त्याऽव्याप्त्या वा विपक्षवृत्तो विरुद्ध एवेति दर्शयितुमुभयोपादानम् । अन्यथा कृतकत्वमात्रे प्रदर्शिते सपक्षावृत्तौ विपक्षव्यापक एव यः स विरुद्धो न तु प्रयत्नानन्तरीयकवद् यो विपक्षव्यापीति शङ्का स्यात् । तथा  \leavevmode\marginnote{\textenglish{212/dm}} “
	  
	यथेति । चक्षुरादय इति धर्मी । परोऽर्थः प्रयोजनं \footnote{परोऽर्थः प्रयोजनं सं०--\cite{dp-msA} \cite{dp-msB} \cite{dp-msC} \cite{dp-msD} \cite{dp-edP} \cite{dp-edH} \cite{dp-edE} \cite{dp-edN}}परार्थः प्रयोजकः संस्कार्य उपकर्त्तव्यो येषां ते परार्थाः-इति साध्यम् । संघातत्वात् सञ्चितरूपत्वादिति हेतुः । चक्षुरादयो हि परमाणुसञ्चिति\footnote{०संचयरू \cite{dp-msC} \cite{dp-msD}}रूपाः । ततः संघातरूपा उच्यन्ते । शयनमासनं चादिर्यस्य तच्छयनासनादि । तदेवाङ्गं पुरुषोपभोगाङ्गत्वात् । अयं व्याप्तिप्रदर्शनविषयो दृष्टान्तः । अत्र हि पारार्थ्येन संहतत्वं\footnote{संघातत्वं \cite{dp-msD}} व्याप्तम् । यतः \footnote{शयनादयः \cite{dp-msD}}शयनासनादयः संघातरूपाः पुरुषस्य भोगिनो भवन्त्युपकारका इति परार्था उच्यन्ते ॥ 
	  
	कथमयमिष्टविघातकृदित्याह-- “
	  
	तदिष्टासंहतपारार्थ्यविपर्ययसाधनाद्विरुद्धः ॥ ८८ ॥” 
	  
	तदिष्टासंहतपारार्थ्यविपर्ययसाधनादिति । असंहते विषये पारार्थ्यमसंहतपारार्थ्यम् ।\footnote{०र्थ्यम् । इष्टासंहतपारार्थ्यं तस्य सांख्यस्य वादिन इष्टासंहतपारार्थ्यम् \cite{dp-msC} \cite{dp-msD}} तस्य सांख्यस्य वादिन इष्टमसंहतपारार्थ्यं तदिष्टासंहतपाराथ्यंम् । तस्य विपर्ययः संहतपारार्थ्यं नाम । तस्य साधनाद् विरुद्धः । 
	  
	“आत्मा अस्ति” इति ब्रुवाणः सांख्यः “कुत एतद्” इति पर्यनुयुक्तो बौद्धेनेदमात्मनः सिद्धये प्रमाणमाह । तस्मादसंहतस्यात्मन उपकारकत्वं साध्यं चक्षुरादीनाम् । अयं तु हेतुर्विपर्ययव्याप्तः । यस्माद्यो यस्योपकारकः स तस्य जनकः । जन्यमानश्च युगपत्\footnote{०मानस्तु युग० \cite{dp-msD}} क्रमेण वा भवति\footnote{भवतीति \cite{dp-msD}} संहतः । तस्मात् परार्थाश्चक्षुरादयः\footnote{चक्षुरादय इति संहं० \cite{dp-msA} \cite{dp-msB} \cite{dp-edP} \cite{dp-edH} \cite{dp-edN}} संहतपरार्था इति सिद्धम् ॥” प्रयत्नानन्तरीयकत्वमात्रे प्रदर्शिते सपक्षावृत्तावव्याप्त्यैव यो विपक्षे वर्त्तते स एव विरुद्धो नान्य इति शङ्का स्यात्, एतावतापि \leavevmode\marginnote{\textenglish{72b/ms}} विपर्ययसिद्धेरुपपत्तेः । यदि तु कार्यहेतुरुदाहृत इत्युच्यते, तदानुपलब्धिरप्युदाहर्त्तव्या स्यात् । न च तस्यास्तत्रान्तर्भावान्नोदाहरणमिति युक्तम्, अन्वयादिप्रदर्शनेऽपि अनुदाहरणप्रसङ्गात् । एवं \textbf{धर्मोत्तरेण} कथं न व्याख्यातमिति न प्रतीमः ॥
	\pend
      

	  \pstart अर्थशब्दः प्रयोजने वृत्तस्तस्य च प्रयोजयतीति व्युत्पत्त्या प्रयोजकशब्दाभिलप्यत्वमित्यभिप्रेत्य \textbf{परार्थः प्रयोजक} इत्युक्तः । परस्मिन्नर्थः प्रयोजनं येषामिति गमकत्वाद् व्यधिकरणबहुव्रीहौ तु सर्वं समञ्जसं केवलमनेन तथा न व्याख्यातमिति न विद्मः । \textbf{अङ्गं} निमित्तम् । कस्य तदङ्गमित्युच्यत इत्याह--\textbf{पुरुषे}ति ॥
	\pend
      

	  \pstart क्रमेण युगपद् वाऽपि द्वेधाप्यसंहतं द्रष्टव्यम् ।
	\pend
      

	  \pstart कुतः पुनरसंहतविषयं पारार्थ्यमिष्टं \textbf{सांख्यस्या}ख्यायत इत्याशङ्क्याह \textbf{आत्मेति} । असंहतोपकारकत्वं तावच्चक्षुरादीनां तेनैषितव्यमन्यथाऽऽत्माऽसिद्धेः । यच्च तदसंहतरूपं स एवात्मेत्यभिप्रायेणोक्तम\textbf{संहतस्यात्मन} इति ॥
	\pend
      \leavevmode\marginnote{\textenglish{213/dm}}“

	  \pstart अयं च विरुद्ध आचार्यदिङ् नागेनोक्तः--
	\pend
       “

	  \pstart स \footnote{स कस्मा० \cite{dp-msC}}इह कस्मान्नोक्तः ॥ ८९ ॥
	\pend
      ” 

	  \pstart स कस्माद् वार्त्तिककारेण सता त्वया नोक्तः ? इतर\footnote{वार्त्तिंककारः--\cite{dp-msD-n}} आह--
	\pend
       “

	  \pstart अनयोरेवान्तर्भावात् ॥ ९० ॥
	\pend
      ” 

	  \pstart अनयोरेव साध्यविपर्ययसाधनयोरन्तर्भावात् ।
	\pend
       

	  \pstart ननु चोक्तविपर्यपं न साधयति । तत् कथमुक्तविपर्ययसाधनयोरेवान्तर्भाव इत्याह--
	\pend
       “

	  \pstart न ह्ययमाभ्यां साध्यविपर्ययसाधनत्वेन भिद्यते ॥ ९१ ॥
	\pend
      ” 

	  \pstart न ह्ययमिति । हीति यस्मादर्थे । यस्माद् अयमिष्टविधातकृदाभ्यां हेतुभ्यां साध्यविपर्ययस्य\footnote{विपर्ययसाध०--\cite{dp-msA} \cite{dp-msB} \cite{dp-edP} \cite{dp-edH} \cite{dp-edE} \cite{dp-edN}} साधनत्वेन न भिद्यते । यथा तौ साध्यविपर्ययसाधनौ तथाऽयमपीति । उक्तविपर्ययं तु साधयतु\footnote{ध्यतु मा वा \cite{dp-edE} \cite{dp-msD}} वा मा वा किमुक्तविपर्ययसाधनेन । तस्मादनयोरेवान्तर्भावः ॥
	\pend
       

	  \pstart ननु चोक्तमेव साध्यं\footnote{साध्ये \cite{dp-msA} \cite{dp-msB} \cite{dp-edP} \cite{dp-edH} \cite{dp-edN}} तत् कथं साध्यविपर्ययसाधनत्वेनाभेद इत्याह--
	\pend
       “

	  \pstart नहीष्टोक्तयोः साध्यत्वेन कश्चिद्विशेष इति ॥ ९२ ॥
	\pend
      ” 

	  \pstart न हीति यस्मादिष्टोक्तयोः \footnote{परस्परस्य \cite{dp-msD} \cite{dp-msB} \cite{dp-edP} \cite{dp-edH} \cite{dp-edE}}परस्परस्मात् साध्यत्वेन न कश्चिद्विशेषो भेद इति । तस्मादनयोरेवान्तर्भाव इत्युपसंहारः ।
	\pend
       

	  \pstart प्रतिवादिनो हि यज्जिज्ञासितं तत् प्रकरणापन्नम् । यच्च प्रकरणापन्नं तत् साधनेच्छया विषयीकृतं साध्यमिष्टमुक्तमनुक्तं वा, न तूक्तमात्रमेव साध्यम् । तेनाविशेष इति ॥
	\pend
       “

	  \pstart द्वयो रूपयोरेकस्यासिद्धावपरस्य च सन्देहेऽनैकान्तिकः ॥ ९३ ॥
	\pend
      ” 

	  \pstart द्वयो रूपयो\footnote{रूपयोरसिद्धौ विरुद्धः--\cite{dp-msA} \cite{dp-msB} \cite{dp-msC} \cite{dp-msD} \cite{dp-edP} \cite{dp-edH} \cite{dp-edN}} र्विपर्ययसिद्धौ विरुद्ध उक्तः । \footnote{अनयोस्तु द्वयो० \cite{dp-msD} अनयोर्द्वयोर्म० \cite{dp-msA} \cite{dp-msB} \cite{dp-edP} \cite{dp-edH} \cite{dp-edE} \cite{dp-edN}}तयोस्तु द्वयोर्मध्य एकस्यासिद्धौ, अपरस्य च सन्देहेऽनैकान्तिकः ॥
	\pend
      ”

	  \pstart \textbf{इतर} इति चोदकादन्यो \textbf{वार्तिककार} इत्यर्थात् ।
	\pend
      

	  \pstart नह्यक्तविपर्ययसाधकत्वेन विरुद्ध उच्यते । किन्तर्हि ? साध्यविपर्ययसाधकत्वेन । तस्मात् \textbf{किमुक्तविपर्ययसाधनेने}त्युक्तम् ॥
	\pend
      

	  \pstart तमेव साध्यत्वेनाभेदं साधयन्नाह--\textbf{प्रतिवादिनो ही}ति । हिर्यस्मादर्थे । \textbf{तेन} साध्येच्छया विषयीकृतमात्रस्य साध्यत्वेना\textbf{विशेषो}ऽभेदः ॥
	\pend
      \leavevmode\marginnote{\textenglish{214/dm}}“

	  \pstart कीदृशोऽसावित्याह--
	\pend
       “

	  \pstart यथा वीतरागः \footnote{“कश्चित्” नास्ति--\cite{dp-msC}}कश्चित् सर्वज्ञो वा, वक्तृत्वादिति\footnote{वक्तृत्वात् । व्य० \cite{dp-msC}} । व्यतिरेकोऽत्रासिद्धः । सन्दिग्धोऽन्वयः ॥ ९४ ॥
	\pend
      ” 

	  \pstart यथेति । विगतो रागो यस्य स वीतराग इत्येकं साध्यम् । सर्वज्ञो वेति द्वितीयम् । वक्तृत्वादिति हेतुः । व्यतिरेकोऽत्रासिद्ध इति । स्वात्मन्येव सरागे चासर्वज्ञे च विपक्षे वक्तृत्वं दृष्टम् । अतोऽसिद्धो व्यतिरेकः । सन्दिग्धोऽन्वयः ॥
	\pend
       

	  \pstart कुत इत्याह--
	\pend
       “

	  \pstart सर्वज्ञवीतरागयोर्विप्रकर्षाद्वचनादेस्तत्र सत्त्वमसत्त्वं वा सन्दिग्धम् ॥ ९५ ॥
	\pend
      ” 

	  \pstart सपक्षभूतयोः सर्वज्ञवीतरागयोर्विप्रकर्षादित्यतीन्द्रियत्वाद् वचनादेरिन्द्रियगम्यस्यापि तत्र अतीन्द्रिययोः सर्वज्ञत्ववी \footnote{ज्ञवी} तरागयोः सत्त्वमसत्त्वं वा सन्दिग्धम् । ततश्च न ज्ञायते किं बक्तृत्वात् सर्वंज्ञ उत नेत्यनैकान्तिकं इति ॥
	\pend
       

	  \pstart सम्प्रति द्वयोरेव सन्देहेऽनैकान्तिकं वक्तुमाह--
	\pend
       “

	  \pstart अनयोरेव द्वयो रूपयोः सन्देहेऽनैकान्तिकः ॥ ९६ ॥
	\pend
      ” 

	  \pstart अनयोरेव--अन्वय-व्यतिरेकरूपयोः सन्देहात् संशयहेतुः ।
	\pend
       

	  \pstart उदाहरणम्\footnote{उदाहरणं यथा \cite{dp-msD} उदाहरणं च \cite{dp-msC}}--
	\pend
       “

	  \pstart यथा\footnote{“यथा” नास्ति \cite{dp-msB} \cite{dp-msC} \cite{dp-msD} \cite{dp-edP} \cite{dp-edH}} सात्मकं जीवच्छरीरं प्राणादिमत्वादिति\footnote{०मत्वात् ।--\cite{dp-msC}} ॥ ९७ ॥
	\pend
      ” 

	  \pstart यथेति\footnote{“यथेति” नास्ति--\cite{dp-msA} \cite{dp-msB} \cite{dp-msC} \cite{dp-msD} \cite{dp-edP} \cite{dp-edH} \cite{dp-edN}} । सहात्मना वर्त्तंते सात्मकमिति साध्यम् । शरीरमिति धर्मी । जीवद्ग्रहणं धर्मिविशेषणम् । मृते ह्यात्मानं नेच्छति ।
	\pend
       

	  \pstart प्राणाः \footnote{प्राणा आश्वा० \cite{dp-msA} \cite{dp-msB} \cite{dp-edP} \cite{dp-edH} \cite{dp-edE} \cite{dp-edN}}श्वासादय आदिर्यस्योन्मेषनिमेषादेः प्राणिधर्मस्य स प्राणादिः । स यस्यास्ति तत् प्राणादिमत् जीवच्छरीरम् । तस्य भावस्तत्त्वम् । तस्मादित्येष हेतुः । अयमसाधारणः संशयहेतुरुपपादयितव्यः ॥
	\pend
      ”

	  \pstart \textbf{जीव}त्प्राणान् धारयच्च तच्छरीरं चेति विग्रहः । \textbf{उन्मेष}श्चक्षुर्विकाश आदिर्यस्य \textbf{निमेषा}देस्तस्य \textbf{प्राणिधर्मस्य} जीवधर्मस्य । \textbf{असाधारणः} सपक्षविपक्षावृत्तेः । विवादाध्यासितस्यैव धर्मिणो धर्म इत्यर्थः । असाधारणत्वादेव \textbf{संशयहेतुरि}ति हेतुभावेन विशेषणम् ॥
	\pend
      \leavevmode\marginnote{\textenglish{215/dm}}“

	  \pstart पक्षधर्मस्य च द्वाभ्यां कारणाभ्यां संशयहेतुत्वम् । संशयविषयौ यावाकारौ ताभ्यां सर्वस्य वस्तुनः संग्रहात् । तयोश्च व्यापकयोराकारयोरेकत्रापि वृत्त्यनिश्चयात् । \footnote{यकाभ्यां \cite{dp-msA} \cite{dp-msB} \cite{dp-msC} \cite{dp-msD} \cite{dp-edP} \cite{dp-edH} \cite{dp-edN}}याभ्यां ह्याकाराभ्यां सर्वं वस्तु न संगृह्यते तयोराकारयोर्न संशयः । प्रकारान्तरसम्भवे हि पक्षधर्मो धर्मिणमवियुक्तं द्वयोरेकेन धर्मेण दर्शयितुं न शक्नुयात् । अतो न संशयहेतुः स्यात् । द्वयोर्धर्मयोरनियतं भावं दर्शयन् संशयहेतुः । द्वयोस्त्वनियतमपि भावं दर्शयितुमशक्तो\footnote{शशविषाणादिः--\cite{dp-msD-n}}ऽप्रतिपत्तिहेतुः । नियतं तु\footnote{“तु” नास्ति \cite{dp-msA} \cite{dp-msB} \cite{dp-edP} \cite{dp-edH} \cite{dp-edE} \cite{dp-edN}} भावं दर्शयन् \footnote{“सम्यग्” नास्ति \cite{dp-msA} \cite{dp-msB} \cite{dp-edP} \cite{dp-edH} \cite{dp-edE} \cite{dp-edN}}सम्यग् हेतुर्विरुद्धो वा स्यात् । तस्माद् \footnote{यकाभ्यां \cite{dp-msB} \cite{dp-msC} \cite{dp-msD} \cite{dp-edP} \cite{dp-edH} \cite{dp-edN}}याभ्यां सर्वं वस्तु संगृह्यते तयोः\footnote{तयोराकारयोः सं० \cite{dp-msC}} संशयहेतुर्यदि तयोरेकत्रापि सद्भावनिश्चयो न स्यात् । सद्भावनिश्चये तु यद्येकत्र नियतसत्तानिश्चयो\footnote{०यो विरुद्धो हेतुर्वा स्यात् \cite{dp-msA} \cite{dp-msB} \cite{dp-msD} \cite{dp-edP} \cite{dp-edH} \cite{dp-edE} \cite{dp-edN}} हेतुर्विरुद्धो वा स्यात् । अनियतसत्तानिश्चये तु साधारणानैकान्तिकः, सन्दिग्धविपक्षव्यावृत्तिकः, सन्दिग्धान्वयोऽसिद्धव्यतिरेको वा स्यात् । एकत्रापि तु
	\pend
      ”

	  \pstart कथं पुनरयं संशयहेतुरुत्पादयितुं शक्यते यावता नास्मात्सात्मकत्वस्यानात्मकत्वस्य वा प्रतिपत्तिर्जायते । ततोऽप्रतिपत्तिरेवासाधारण इ\textbf{त्युद्द्योतकरम}तमाशङ्क्य यत्र \footnote{यत् न} हीत्युक्तं वार्तिककृता तदवतारयितुं भूमिकां रचयन्नाह--\textbf{पक्षधर्मस्ये}ति । \textbf{चो} यस्मादर्थे । पक्षस्य धर्मस्य सतो \textbf{द्वाभ्यां कारणाभ्यां} निमित्ताभ्याम् । \textbf{सर्वस्य} निःशेषस्य संग्रहाज्ज्ञापनात् । \textbf{तयोर्व्यापकयोराकारयोः} सात्मकत्वानात्मकत्वाख्ययोर्विषयभूतयोरेकत्रापि विषयेऽ\textbf{निश्चयात्त}स्य पक्षधर्मस्येत्यर्थात् ।
	\pend
      

	  \pstart ननु संशय्यमानयोराकारयोः सर्ववस्तुव्यापनेन किं ? येन तथात्वं तयोरुपवर्ण्यत इत्याह--\textbf{याभ्यामि}ति । हीति यस्मात् । कथं पुनस्तयोराकारयोर्न संशय इत्याह--\textbf{प्रकारान्तरे}ति । हिर्यस्मादर्थे । \textbf{पक्षधर्मः} सन् \textbf{धर्मिणं} तं पक्षं \textbf{द्वयोरेकेन धर्मेण} सात्मकत्वाख्येन अनात्मत्वाख्येन वा । \textbf{अवियुक्तं} युक्तं सम्बद्धमिति यावत् । \textbf{न शक्नोति दर्शयितुं} प्रकारान्तरेण सम्बन्धस्य तस्य सम्भवात् । यत एव\textbf{मतः} स पक्षधर्मो \textbf{न संशयहेतुः स्या}त् । \textbf{तयो}राकारयोरिति सामर्थ्यात् ।
	\pend
      

	  \pstart अथ कथं सात्मकत्वानात्मकत्वे द्वौ धर्मौ दर्शयन्नपि संशयहेतुरुच्यत इत्याह--\textbf{द्वयोरि}ति । दर्श\leavevmode\marginnote{\textenglish{73a/ms}}\textbf{यन्नि}ति हेतौ शतुर्विधानादनियतभावप्रदर्शनादित्यर्थो बोद्धव्यः ।
	\pend
      

	  \pstart ननु नियताभावमदर्शयन्नप्रतिपत्तिहेतुरेवायं युज्यत इत्याह--\textbf{द्वयोस्त्वि}ति । तुशब्दो यस्मादर्थे । नियतप्रदर्शकस्यापि किं न तथात्वमित्याह--\textbf{नियतत्वं\footnote{यतं} त्वि}ति । \textbf{तुः} पूर्ववद् विशेषणार्थो वा । यस्मादेवं \textbf{तस्माद्} हेतोः ।
	\pend
      

	  \pstart \textbf{नियतो}ऽन्यत्राननुगामी \textbf{सत्तानिश्चयो} यस्य स तथा । विरुद्धोऽपि विपर्यये सम्यघेतुरेवेत्येकत्र सत्तानिश्चये \textbf{विरुद्धो वा स्यादि}त्युक्तम् ।
	\pend
      \leavevmode\marginnote{\textenglish{216/dm}}“

	  \pstart वृत्त्यनिश्चयादसाधारणानैकान्तिको भवति । ततोऽसाधारणानैकान्तिकस्यानैकान्तिकत्वे हेतुद्वयं दर्शयितुमाह--
	\pend
       “

	  \pstart न हि \footnote{सात्मकानात्मका० \cite{dp-msC}}सात्मकनिरात्मकाभ्यामन्यो राशिरस्ति यत्रायं\footnote{यत्र प्राणा० \cite{dp-msD} \cite{dp-msB} \cite{dp-edP} \cite{dp-edH} \cite{dp-edE} \cite{dp-edN} प्राणादिर्वर्त्तेत \cite{dp-edN}} प्राणादिर्वर्त्तते \footnote{र्वर्तेत} ॥ ९८ ॥
	\pend
      ” 

	  \pstart नहीति । सहात्मना वर्त्तते सात्मकः । निष्क्रान्त आत्मा यस्मात् स निरात्मकः । \footnote{आभ्यां \cite{dp-msD}}ताभ्यां यस्मान्नान्यो राशिरस्ति । किंभूतः ? यत्रायं वस्तुधर्मः प्राणादिर्वर्त्तेत ? तस्मादयं \footnote{०दयं द्वयोर्भ० \cite{dp-msC}}तयोर्भवति संशयहेतुः ॥
	\pend
       

	  \pstart कस्मादन्यराश्यभाव इत्याह--
	\pend
       “

	  \pstart आत्मनो\footnote{आत्मवृत्ति० \cite{dp-msC}} वृति-व्यवच्छेदाभ्यां \footnote{सर्वस्य सं० \cite{dp-msC}}सर्वसंग्रहात् ॥ ९९ ॥
	\pend
      ” 

	  \pstart आत्मनो वृत्तिः सद्भावो व्यवच्छेदोऽभावः । ताभ्यां सर्वस्य वस्तुनः संग्रहात् क्रोडीकरणात् । यत्र ह्यात्मा अस्ति तत् सात्मकम् । \footnote{तदन्यन्नि० \cite{dp-msA} \cite{dp-msB} \cite{dp-edP} \cite{dp-edH} \cite{dp-edN}}अन्यन्निरात्मकम् । ततो नान्यो राशिरस्ति—\footnote{“इति” नास्ति \cite{dp-msA} \cite{dp-msB} \cite{dp-msC} \cite{dp-msD} \cite{dp-edP} \cite{dp-edH} \cite{dp-edN}}इति संशयहेतुत्वकारणम् ॥
	\pend
      ”

	  \pstart यदैकत्रैव सत्तानिश्चयो विरुद्धो वा स्यादित्युक्तम्, यदैकत्रैव सत्तानिश्चयो नास्ति तदा का वार्त्तेत्याह--\textbf{अनियतेति} । \textbf{तु}र्विशेषार्थः । \textbf{अनियतो}ऽत्रैवायं वर्त्तत इत्येवंरूपनियमशून्यो यः \textbf{सत्तानिश्चय}स्तस्मिन् सति । यदोभयत्र सत्तानिश्चयस्तदा सपक्षविपक्षसाधारणत्वात्साधारणः, यदा तु विपक्षवृत्तिसम्भावनायामनिश्चय\footnote{नियत}सत्तानिश्चयस्तदा सन्दिग्धविपक्षव्यावृत्तिकः । यदा पुनः सपक्षे वृत्तिसन्देहनानियतसत्तानिश्चयस्तदा सन्धिग्धान्वयासिद्धव्यतिरेकः । यदा तु सपक्षासपक्षयोरेकत्रापि सत्तानिश्चयो नास्ति तदा सपक्षासपक्षावृत्तेरसाधारणः । एतदेवाह—\textbf{एकत्रापीति} । यतः पक्षधर्मस्योक्ताभ्यां कारणाभ्यां संशयहेतुत्वं \textbf{तत}स्तस्मात् \textbf{दर्शयितुं} दर्शयिष्यामीति मत्त्वा ।
	\pend
      

	  \pstart राशिः प्रकारः । वस्तुधर्मत्वं च प्राणादेरवस्तुनि शशविषाणादाववृत्तेः । यतो राश्यन्तराभाव\textbf{स्तस्मा}त्कारणादयं प्राणादिमत्त्वाख्यो हेतुः । \textbf{तयोः} सात्मकनिरात्मकयोः ।
	\pend
      

	  \pstart साधारणस्य धर्मस्य संशयहेतुत्वे द्वे कारणे । तत्रामुना \textbf{नहीत्या}दिना मूलेन संशयविषयाभ्यामाकाराभ्यां सर्ववस्तुसंग्रह एकं कारणमुक्तम् । \textbf{नाप्यनयो}रित्यादिना तु तयोरेकत्रापि वृत्त्यनिश्चयो द्वितीयं कारणमुक्तमिति दर्शयितुमाह--\textbf{संशयेति} ॥
	\pend
      \leavevmode\marginnote{\textenglish{217/dm}}“

	  \pstart प्रकाराभ्यां \footnote{सर्वसंग्रहं \cite{dp-msA} \cite{dp-msB} \cite{dp-msC} \cite{dp-edP} \cite{dp-edH} \cite{dp-edE} \cite{dp-edN}}सर्ववस्तुसङ्ग्रहं प्रतिपाद्य द्वितीयमाह--
	\pend
       “

	  \pstart नाप्यनयोरेकत्र वृत्तिनिश्चयः ॥ १०० ॥
	\pend
      ” 

	  \pstart नाप्यनयोः सात्मकानात्मकयोर्मध्य एकत्र सात्मकेऽनात्मके\footnote{०के निरात्मके \cite{dp-msC}} वा वृत्तेः सद्भावस्य निश्चयोऽस्ति । द्वावपि राशी त्यक्त्वा न वर्त्तते प्राणादिः, वस्तुधर्मत्वात् । ततश्चानयोरेव वर्त्तते\footnote{वर्त्तते एताव० \cite{dp-msC}} इत्येतावदेव ज्ञातम् । विशेषे तु वृत्तिनिश्चयो नास्तीत्ययमर्थः\footnote{नास्तीत्यर्थः \cite{dp-msC} \cite{dp-msD}} ॥
	\pend
       

	  \pstart तदाह--
	\pend
       “

	  \pstart \footnote{सात्मकत्वेन निरात्मक० \cite{dp-msB} \cite{dp-edP} \cite{dp-edH} \cite{dp-edE} \cite{dp-edN}}सात्मकत्वेनाऽनात्मकत्वेन वा प्रसिद्धे प्राणादेरसिद्धेः\footnote{रसिद्धिस्ताभ्यां न व्यतिरिच्यते \cite{dp-edE}} ॥ १०१ ॥
	\pend
      ” 

	  \pstart \footnote{सात्मकत्वेन निरात्मक० \cite{dp-msD}}सात्मकत्वेनाऽनात्मकत्वेन वा विशेषेण युक्ते प्रसिद्धे निश्चिते वस्तुनि प्राणादेर्धर्मस्य \footnote{०र्धर्मस्यासिद्धेरनैकान्तिकोऽनिश्चि० \cite{dp-msA} \cite{dp-msB} \cite{dp-edP} \cite{dp-edH} \cite{dp-edE} \cite{dp-edN} ०र्धर्मस्यासिद्धेरनिश्चि० \cite{dp-msC}}सर्ववस्तुव्यापिनोः प्रकारयोरेकत्र नियतसद्भावस्यासिद्धेरनैकान्तिकः, अनिश्चितत्वात् ।
	\pend
       

	  \pstart तदेवमसाधारणस्य धर्मस्यानैकान्तिकत्वे कारणद्वयमभिहितम् ॥
	\pend
      ”

	  \pstart \textbf{प्रकाराभ्या}मात्मव्यवच्छेदरूपाभ्यामाकाराभ्याम् । \textbf{वृत्तिः} प्रवृत्तिरर्थात् भाव एवावतिष्ठत इत्यभिप्रायेणा\textbf{ह--वृत्तेः सद्भावस्ये}ति ।
	\pend
      

	  \pstart यद्येवं तयोर्न वर्त्तत इत्येव किं न स्यात् ? तथा च कथं संशयहेतुरित्याशङ्क्य यादृशोऽस्यार्थोऽभिप्रेतस्तं स्फुटयितुमाह--\textbf{द्वावपीति ।} कुतो न वर्त्तत इत्याह--\textbf{वस्तुधर्मत्वादि}ति । प्रायादे \footnote{प्राणादे} रिति विभक्तिविपरिणामेन सम्बन्धनीयम् । वस्तुना वाऽवश्यं सात्मकेनाऽनात्मकेन वा भाव्यमिति भावः ।
	\pend
      

	  \pstart ततो वस्तुसत्त्वेन सिद्धपरित्यागेनान्यत्रावृत्तेः कारणात् । \textbf{अनयोः} सात्मकानात्मकयोः । एवकारेणान्यत्रा \footnote{त्र} वृत्तिनिषेधं स्पष्टयति । इतिरेतावतः स्वरूपं दर्शयति । यदिदमनन्तरोक्तमेतत् परिमाणं यस्य प्रमेयस्य तद् \textbf{एतावद्} वस्तुतत्त्वं निश्चितम् । कुतस्तर्हि नास्य वृत्तिनिश्चय इत्याह--\textbf{विशेषे त्वि}ति । \textbf{विशेषे} विशिष्टे प्रकारे । \textbf{तु}रिमामवस्थां भेदवतीमाह । तेन सात्मकत्व \footnote{सात्मक} एवानात्मक एवेत्यर्थः । \textbf{वृत्तेः} स्वभावस्य प्राणादेरिति प्रकरणात् । \textbf{इति}रेवमर्थे । \textbf{अर्थो}ऽभिधेयो यस्य \textbf{“नाप्यनयोरेकत्र वृत्तिनिश्चयः”} इत्यस्य मौलस्य वाक्यस्येत्यर्थात् ॥
	\pend
      

	  \pstart यस्मादेवमेतद् वक्तुं युज्यते, नान्यथा तत्तस्मा \leavevmode\marginnote{\textenglish{73b/ms}} \textbf{दाह वार्त्तिककारः ।} किमाहेत्याह \textbf{सात्मेति । तदेवमि}त्यादिनोपसंहरति । एवं च व्याचक्षाणेन \add{न} मया स्वातन्त्र्येण \textbf{पक्षधर्मस्ये}  \leavevmode\marginnote{\textenglish{218/dm}} “
	  
	पक्षधर्मश्च भवन्\footnote{भवत् सर्वः \cite{dp-msA}} सर्वः साधारणोऽसाधारणो वा भवत्यनैकान्तिकः । तस्मादुपसंहारव्याजेन पक्षधर्मत्वं दर्शयति-- “
	  
	तस्माज्जीवच्छरीरसम्बन्धी प्राणादिः सात्मकादनात्मकाच्च सर्वस्माद् व्यावृत्तत्वोनासिद्धेस्ताभ्यां\footnote{०नासिद्धिः ॥ \cite{dp-edE} “ताभ्यामि” त्यादि नास्ति \cite{dp-edE}} न व्यतिरिच्यते ॥१०२॥” 
	  
	तस्मादित्यादिना । जीवच्छरीरस्य सम्बन्धी पक्षधर्म इत्यर्थः । यस्मात् तयोरेकत्रापि न निवृत्तिनिश्चयस्तस्मात् ताभ्यां न व्यतिरिच्यते । 
	  
	वस्तुधमे हि सर्ववस्तुव्यापिनोः प्रकारयोरेकत्र नियतसद्भावो निश्चितः प्रकारान्तरान्निवर्त्तेत ।\footnote{०रान्निवर्त्तते । तत एवाह--\cite{dp-msD} ०रान्निवर्त्तेत । तत एवाह \cite{dp-msC} \cite{dp-msA} \cite{dp-msB} \cite{dp-edP} \cite{dp-edH} \cite{dp-edE} \cite{dp-edN}} अत एवाह--सात्मकादनात्मकाच्च सर्वस्माद् वस्तुनो व्यावृत्तत्वेनासिद्धेरित । प्राणादिस्तावत् कुतश्चिद् घटादेर्निवृत्त एव । तत एतावदवसातुं शक्यम्--सात्मकादनात्मकाद्वा कियतो निवृत्तः । सर्वस्मात् \footnote{“तु” नास्ति--\cite{dp-msC}}तु निवृत्तो नावसीयते । ततो न कुतश्चिद् व्यतिरेकः ॥” दिनाऽसाधारणस्य संशयहेतुत्वनिमित्तद्वयमादितो दर्शितम् । किं तर्हि ? \textbf{वार्त्तिककारेणैवैतद}भिहितमिति दर्शितम् ॥
	\pend
      

	  \pstart ननु चासाधारणस्य प्राणादेरनैकान्तिकत्वकारणद्वयमनन्तरोक्तमभिधीयताम् । \textbf{तस्मादि}त्यादिना तु शरीरसम्बन्धित्वमस्य कस्मादाचार्यो दर्शयतीत्याशङ्कां निराचिकीर्षुः \textbf{पक्षधर्मश्चे}त्यादिनोपक्रमते । \textbf{चो} यस्मादर्थे । अपक्षधर्मस्त्वसिद्धत्वाख्यामन्यामेव दोषजातिमनुश्नुत इति भावः पक्षधर्मत्वं प्रदर्शयतो \textbf{वार्तिककार}स्योन्नेयः । \textbf{वा}शब्देनानियतप्रभेदेऽनास्\textbf{थां} दर्शयति । न त्वसाधारणत्वाख्यं पक्षान्तरम्, सन्दिग्धविपक्षव्यावृत्तिकादेरसङ्ग्रहप्रसङ्गात् । यस्मादपक्षधर्मो नानैकान्तिक उपवर्णितेनाभिप्रायेण \textbf{तस्मा}त्कारणात् । \textbf{दर्शयति} प्रकाशयति ।
	\pend
      

	  \pstart केन दर्शयतीत्याकाङ्क्षायामाह--\textbf{तस्मादित्यादिने}ति । पक्षधर्मत्वप्रदर्शनं तु \textbf{जीवच्छरीरसम्बन्धी}ति वचनं द्रष्टव्यम् । \textbf{तस्मादि}त्यनेन यस्मादित्याक्षिप्तं दर्शयन्नाह--\textbf{यस्मादि}ति । \textbf{एकत्रापि न वृत्तिनिश्चय}स्तस्य प्राणादेरित्यर्थात् । \textbf{तस्मा}त्कारणात् \textbf{ताभ्यां} सात्मकत्वानात्मकत्वाभ्यां \textbf{न व्यतिरिच्यते} न निवर्त्तते, तदसंस्पर्शी न भवतीति यावत् ।
	\pend
      

	  \pstart तयोरेकत्र वृत्त्यनिश्चयेऽपि कथं ताभ्यां न व्यतिरिच्यत इत्याह--\textbf{वस्त्विति । ही}ति यस्मादर्थे । \textbf{प्रकारयो}स्तद्वृत्तिव्यवच्छेदरूपयोः स्वरूपयोराकारयोरिति यावत् । तयोर्धर्म \textbf{एकत्र नियतः} “अत्रैवायं वर्त्तत इति नियमवान्” \textbf{सद्भावः} सत्त्वं यस्य स तथा । \textbf{प्रकारान्तरान्नि}यतसद्भावविषयादन्यस्मादाकारात् । \textbf{निवर्त्तेत} निवर्तितुमर्हति, तन्न संस्पृशेदिति यावत् । एकत्र वृत्त्यनिश्चयाच्च नायं तथेत्यभिप्रायः ।
	\pend
      

	  \pstart यत एवमेतद् भवति, न चायं प्राणादिस्तथा, अत \textbf{एवा}हाचार्यः । किमाहेत्याह—\textbf{सात्मकादि}त्यादि । \textbf{सर्वस्मादि}ति प्रत्येकं सम्बद्ध \footnote{न्द्ध} व्यम् ।
	\pend
      \leavevmode\marginnote{\textenglish{219/dm}}“

	  \pstart यद्येवमन्वयोऽस्तु तयोर्निश्चित इत्याह--
	\pend
       “

	  \pstart न\footnote{न च तत्रा० \cite{dp-msC} \cite{dp-msD}} तत्रान्वेति ॥ १०३ ॥
	\pend
      ” 

	  \pstart न\footnote{न च तत्र \cite{dp-msD}} तत्र सात्मकेऽनात्मके वाऽर्थेन्वेति--अन्वयवान् प्राणादिः ॥
	\pend
       

	  \pstart कुत इत्याह--
	\pend
       “

	  \pstart एकात्मन्यप्यसिद्धेः ॥ १०४ ॥
	\pend
      ” 

	  \pstart \footnote{“एकात्मन्यपीति”--नास्ति \cite{dp-msA} \cite{dp-msB} \cite{dp-msC} \cite{dp-msD} \cite{dp-edP} \cite{dp-edH} \cite{dp-edN}}एकात्मन्यपीति । एकात्मनि सात्मकेऽनात्मके वाऽसिद्धेः कारणात् । वस्तुधर्मतया तयोर्द्वयोरेकत्र\footnote{०र्द्वयोरप्येकत्र \cite{dp-msC} \cite{dp-msD} ०कत्र तावत् वर्त्तत इ० \cite{dp-msC}} वा वर्त्तत इत्यवसितः प्राणादिः । न तु सात्मक एव निरात्मक एव वा वर्त्तत इति कुतोऽन्वयनिश्चयः ॥
	\pend
       

	  \pstart ननु च प्रतिवादिनो न किञ्चत् सात्मकमस्ति । ततोऽस्य हेतोर्न सात्मकेऽन्वयो\footnote{अनुगमनं सद्भाव इत्यर्थः \cite{dp-msD-n}} न व्यतिरेक\footnote{व्यावृत्तिः--अभाव इत्यर्थः--\cite{dp-msD-n}} इत्यन्वयव्यतिरेकयोरभावनिश्चयः सात्मके, न तु सद्भावसंशय इत्याशङ्क्याह--
	\pend
       “

	  \pstart नापि \footnote{सात्मकान्निरात्मका० \cite{dp-msB} \cite{dp-msD} \cite{dp-edP} \cite{dp-edH} \cite{dp-edE} \cite{dp-edN}}सात्मकादनात्मकाच्च तस्यान्वयव्यतिरेकयोरभावनिश्चयः ॥ १०५ ॥
	\pend
      ””

	  \pstart ननु घटपटादेरनेकस्मात्प्राणादिर्निवर्त्तमानो दृष्टस्तत्कथं तस्य व्यावृत्तत्वेनासिद्धिरित्याह--\textbf{प्राणादिरिति । ततः} सर्वस्मात् सात्मकादनात्मकाच्च निवृत्त्यनवसायात् । कुतः ? सात्मकादनात्मकाच्च प्रतिबन्धासिद्धेरिति चात्र सर्वत्राभिप्रायः ।
	\pend
      

	  \pstart ननु किमुच्यते \textbf{न कुतश्चिदि}ति ? यावता निरात्मकादेव व्यतिरेकोऽस्यावसातुं शक्यः, बौद्धेन घटादेर्निरात्मकत्वेनेष्टत्वादिति चेत् । यद्येवं जीवच्छरीरमपि \textbf{बौद्धेन} तथात्वेनेष्टमिति तस्यापि तथात्वं किन्न भवेत् । अभ्युपगमेन च सात्मकानात्मके विभज्य हेतुं कथयता गमिकत्वमिति यत्किञ्चिदेतत् ॥
	\pend
      

	  \pstart तयोरिति विषयसप्तमी । तस्य प्राणादेरिति च शेषः ॥
	\pend
      

	  \pstart ननु वस्तुधर्मेण तेनावश्यं क्वापि नियतेन भाव्यम् । तत्कथमन्वयाभाव इत्याह—\textbf{वस्तुधर्मतये}ति । न चानियतवृत्तिनिश्चयोऽन्वयो नामेति भावः ।
	\pend
      

	  \pstart ननु चासाधारणत्वान्निरात्मकेऽन्वयनिश्चयो मा भूद् । व्यतिरेकनिश्चयस्त्वस्तु, निरात्मके घटादौ प्राणादेरदर्शनादिति चेत् । न । तस्यैव \leavevmode\marginnote{\textenglish{74a/ms}} शरीरस्य निरात्मकत्वसम्भावनायां सर्वस्मान्निरात्मकान्निवृत्तिनिश्चयाभावात् । न च तथानिश्चयनिमित्तं प्रतिबन्धनिश्चयोऽ ऽस्तीति ॥
	\pend
      \leavevmode\marginnote{\textenglish{220/dm}}“

	  \pstart नापि सात्मकाद् वस्तुनः तस्य प्राणादेरन्वयव्यतिरेकयोरभावनिश्चयः । नापि च निरात्मकात् । सात्मकादनात्मकादिति च पञ्चमी व्यतिरेकशब्दापेक्षया द्रष्टव्या ॥
	\pend
       

	  \pstart कथमन्वयव्यतिरेकयोर्नाभावनिश्चय इत्याह--
	\pend
       “

	  \pstart \footnote{एकस्याभा० \cite{dp-msC}}एकाभावनिश्चयस्यापर\footnote{०स्यापराभावनान्त० \cite{dp-msB} \cite{dp-edP} \cite{dp-edH} ०स्यापरभावनान्तरी० \cite{dp-msD} \cite{dp-edE}} भावनिश्चयनान्तरीयकत्वात् ॥ १०६ ॥
	\pend
      ” 

	  \pstart एकस्यान्वयस्य व्यतिरेकस्य वा योऽभावनिश्चयः \footnote{स एवापरस्य \cite{dp-edE} \cite{dp-edN}}सोऽपरस्य द्वितीयस्य \footnote{भावे निश्च० \cite{dp-msA} \cite{dp-msB} \cite{dp-edP} \cite{dp-edH}}भावनिश्चयनान्तरीयकः \footnote{०रीयकः भवति निश्च० \cite{dp-msA} \cite{dp-msB} \cite{dp-edP} \cite{dp-edH}}भावनिश्चयस्याव्यभिचारी । तस्य भावस्तत्त्वं तस्मात् । यत एकाभावनिश्चयोऽपरभावनिश्चय\footnote{“निश्चय” नास्ति \cite{dp-msA} \cite{dp-msC}} नान्तरीयकः, तस्मान्न द्वयोरेकत्राभावनिश्चयः ॥
	\pend
       

	  \pstart कस्मात् पुनरेकस्याभावनिश्चयोऽपरसद्भावनिश्चयाऽव्यभिचारीत्याह--
	\pend
       “

	  \pstart \footnote{अत एवान्वय० \cite{dp-msC}}अन्वय-व्यतिरेकयोरन्योन्यव्यवच्छेदरूपत्वात् । \footnote{अत एव \cite{dp-msD} \cite{dp-msB} \cite{dp-edP} \cite{dp-edH} \cite{dp-edE} \cite{dp-edN}}तत एवान्वयव्यतिरेकयोः सन्देहादनैकान्तिकः ॥ १०७ ॥
	\pend
      ” 

	  \pstart अन्वयव्यतिरेकयोरन्योन्यव्यवछेदरूपत्वादिति । अन्योन्यस्य व्यवच्छेदोऽभावः, स एव रूपं ययोस्तयोर्भावस्तत्त्वम् तस्मात् कारणात् ।
	\pend
       

	  \pstart अन्वयव्यतिरेकौ भावाभावौ । भावाभावौ च परस्परव्यवच्छेदरूपौ । यस्य व्यवच्छेदेन यत् परिच्छिद्यते तत् तत्परिहारेण व्यवस्थितम् । स्वाभावव्यवच्छेदेन च भावः परिच्छिद्यते ।
	\pend
       

	  \pstart तस्मात् स्वाभावव्यवच्छेदेन भावो व्यवस्थितः । अभावो हि नीरूपो यादृशो विकल्पेन दर्शितः । नीरूपतां च व्यवच्छिद्य रूपमाकारवत् परिच्छिद्यते । तथा च सत्यन्वयाभावो व्यतिरेकः, व्यतिरेकाभावश्चान्वयः ।
	\pend
       

	  \pstart ततोऽन्वयाभावे निश्चिते व्यतिरेको निश्चितो भवति । व्यतिरेकाभावे च निश्चितेऽन्वयो निश्चितो भवति ।
	\pend
      ”

	  \pstart \textbf{ननु चे}त्यादि \textbf{तस्मात्कारणा}दित्येतदन्तं स्पष्टार्थ तेन न व्याख्यायते ।
	\pend
      

	  \pstart अन्योन्यव्यवच्छेदरूपत्वमेवान्वयव्यतिरेकयोः कथमित्याशङक्याह--\textbf{अन्वयव्यतिरेकावि}ति । अभावरूपत्वञ्च व्यतिरेकस्य प्रतीतिसिद्धस्य बोद्धव्यम् । भवतां तौ तथारूपौ किमत इत्याह--\textbf{भावाभावावि}ति । \textbf{चो} यास्मादर्थे । भवत्वेवं तथापि कथं तयोरन्योन्यपरिहारेणावस्थानमित्याह--\textbf{यस्ये}ति ।
	\pend
      

	  \pstart नन्वत्र कस्य व्यवच्छेदेन किं परिच्छिद्यते येन तत्परिहारेहण तद् व्यवतिष्ठत इत्याह—\textbf{स्वाभावे}ति । हेत्वर्थश्चकारः । तावत्कासा \footnote{कोऽसा} वभावो नाम यद्व्यवच्छेदेन भावः  \leavevmode\marginnote{\textenglish{221/dm}} “
	  
	तस्माद् यदि नाम सात्मकमवस्तु निरात्मकं च वस्तु, तथापि \footnote{तथापि न तयोः \cite{dp-msA} \cite{dp-msB} \cite{dp-edP} \cite{dp-edH} \cite{dp-edE} \cite{dp-edN}}तयोर्न प्राणादेरन्वयव्यतिरेकयोरभावनिश्चयः । एकत्र\footnote{एकवस्तु० \cite{dp-msA} \cite{dp-msB} \cite{dp-edP} \cite{dp-edH} \cite{dp-edE} \cite{dp-edN}} वस्तुन्येकस्य\footnote{एकवस्तु० \cite{dp-msA} \cite{dp-msB} \cite{dp-edP} \cite{dp-edH} \cite{dp-edE} \cite{dp-edN}} वस्तुनो युगपद्भावाभावविरोधात् तयोरभावनिश्चयायोगात्\footnote{निश्चययोगात् \cite{dp-msB}} । 
	  
	न च प्रतिवाद्यनुरोधात् सात्मकानात्मके वस्तुनी सदसती । किन्तु प्रमाणानुरोधाद् । इत्युभे सन्दिग्धे । ततस्तयोः प्राणादिमत्त्वस्य सदसत्त्वसंशयः ।\footnote{सदसत्त्वानिश्चयः \cite{dp-msC}} 
	  
	यत एव क्वचिदन्वय-व्यतिरेकयोर्न भावनिश्चयो \footnote{नाभावनि० \cite{dp-msC}}नाप्यभावनिश्चयः, तत एवान्वयव्यतिरेकयोः सन्देहः । 
	  
	यदि तु क्वचिद\footnote{क्वचिदन्वय० \cite{dp-msC} \cite{dp-msD}} प्यन्वय-व्यतिरेकयोरेकस्याप्यभावनिश्चयः स्यात्, स एव द्वितीयस्य भावनिश्चय इत्यन्वयव्यतिरेकसन्देह एव न स्यात् । यतश्च\footnote{यतस्तु \cite{dp-msC} \cite{dp-msD}} न क्वचिद्भावाभावनिश्चयस्तत एवान्वयव्यतिरेकयोः सन्देहः । सन्देहाच्चानैकान्तिकः\footnote{०कान्तिक इत्याह । \cite{dp-edE}} ॥” परिच्छिद्यत इत्याह--\textbf{तथा चे}ति नीरूपताव्यवच्छेदेन रूपस्य प्रतिष्ठिताकारवतः परिच्छेदप्रकारे सति । \textbf{ततो}ऽन्योन्याभावरूपत्वादनयोः ।
	\pend
      

	  \pstart ननु \textbf{बौद्धानां} सात्मकं नाम नास्त्येवेत्यवस्तु । सन्मात्रं तु निरात्मकमतो वस्तु । तत्र वस्तु\add{नि}निरात्मको\footnote{के} हेतोरन्वयव्यतिरेकयोरभावनिश्चयो मा भूत्सात्मके त्ववस्तुनि स कथं न स्यादित्याशङ्क्योपसंहारव्याजेनाह \textbf{तस्मादिति} । यस्माद् विधिप्रतिषेधयोरेकप्रतिषेधोऽपरविधिनान्तरीयकस्तस्मात् । कथं न भावनिश्चयस्तयोरित्याशङक्योपपत्तिमाह--एकत्रेति । तयोरन्वयव्यतिरेकयोर्भावाभावात्मनो\textbf{रभावनिश्चय}स्या\textbf{योगा}दनुपपत्तिः । कथमयोग इत्याह—\textbf{एकस्ये}ति । कालभेदे किं न युज्यते इत्याह--\textbf{युगपदि}ति ।
	\pend
      

	  \pstart इदं च प्रतिवाद्यभ्युपगमबलात्सात्मकानात्मकयोः । सदसत्त्वमभ्युपगम्योक्तम् । तदेव तु न युज्यत इति दर्शयन्नाह--\textbf{न चे}ति । \textbf{चो} वक्तव्यान्तरसमुच्चये । प्रकरणादिह \textbf{प्रतिवादी बौ}द्धस्तदनुरोधवशादिष्ट्यनिष्टिवशा\textbf{त्सात्मकमसन्निरात्मकं सदि}ति यथायोगं योजनीयम् । एवं ह्यवास्तवमनुमानं स्यात् न वस्तुबलप्रवृत्तमिति भावः । यद्येवं ते सदसती न भवतः कथं नामेत्याह--\textbf{किन्त्विति} । प्रमाणं चेदं नियतं वर्त्तते । आत्मन्येव च विवादवृत्तेरन्याऽपि \footnote{न्यत्रापि ?} सात्मकत्वमनात्मकत्वेन \footnote{कत्वं वा न ?} व्यवतिष्ठत इति भावः । \textbf{इति}स्तस्मात् । उभे सात्मकत्वानात्मकत्वे । यत एवं \textbf{ततः} कारणात्सात्मकानात्मकयोः \textbf{सदसत्त्वयोः संशयः} । कस्येत्याकाङ्क्षायामुक्तम्--\textbf{प्राणादिमत्त्वस्येति ।}
	\pend
      \leavevmode\marginnote{\textenglish{222/dm}}“

	  \pstart कस्मादनैकान्तिक इत्याह\footnote{“इत्याह” नास्ति \cite{dp-msA} \cite{dp-msB} \cite{dp-msC} \cite{dp-msD} \cite{dp-edP} \cite{dp-edH} \cite{dp-edN}}--
	\pend
       “

	  \pstart साध्येतरयोरतो निश्चयाभावात् ॥१०८॥
	\pend
      ” 

	  \pstart साध्यस्य, इतरस्य च विरुद्धस्य अतः--सन्दिग्धान्वययतिरेकान्निश्चयाभावात् । सपक्षविपक्षयोर्हि सदसत्त्वसन्देहे न साध्यस्य न विरुद्धस्य सिद्धिः\footnote{असिद्धिः \cite{dp-msB}} । न च सात्मकानात्मकाभ्यां\footnote{०भ्यां च परः \cite{dp-msA} \cite{dp-msB} \cite{dp-edP} \cite{dp-edH} \cite{dp-edN}} परः प्रकारः संभवति । ततः प्राणादिमत्त्वाद् धर्मिणि जीवच्छरीरे संशय आत्मभावाभावयोरित्यनैकान्तिकः प्राणादिरिति ॥
	\pend
      ”

	  \pstart अस्तु सदसत्त्वसंशयोऽन्वयव्यतिरेकनिश्चयस्तु किन्न भवतीत्याह--\textbf{यत} इति । न \textbf{भावनिश्चयो नाभावनिश्चय} इत्येकत्र सात्मकेऽनात्मके वेति द्रष्टव्यम् । सात्मकेऽनात्मके वा प्राणादेः सदसत्त्वनिश्चयाभावादेवान्वयव्यतिरेकयोः सन्देहो नान्यथेति प्रतिपादयितुमाह--\textbf{यदि} त्विति । तुरिमामवस्थां भेदवतीमाह ।
	\pend
      

	  \pstart उक्तमेवोपसंहरन्नाह--यतश्चेति । \textbf{चो}ऽवधारणे । तस्मात् \textbf{सन्देहात् अनैकान्तिकः} प्राणादिमत्त्वाख्यो हेतुरिति प्रकरणात् ॥
	\pend
      

	  \pstart \textbf{सन्दिग्धावन्वयव्यतिरेकौ} यस्य तत्तथा, तस्मात्साध्यस्य विरुद्धस्य वा \textbf{निश्चयाभावात् । सपक्षे}त्यादि\leavevmode\marginnote{\textenglish{74b/ms}}नैतदेव समर्थयते । हिर्यस्मात् । \textbf{सपक्षविपक्षयो}र्विषयभूतयोर्हेतोः \textbf{सदसत्त्वसन्देहे न साध्यस्या}नुमित्सितस्य \textbf{विरुद्धस्य} विपर्ययस्य \textbf{सिद्धि}र्निश्चयः । विरुद्धोऽपि विपर्यये सम्यग्धेतुरित्यभिप्रायेणेदमुक्तम् ।
	\pend
      

	  \pstart \textbf{न चे}त्यादि \textbf{प्राणादिरि}त्यन्तं सुगमम् ।
	\pend
      

	  \pstart ईदृश एव चासाधारणो हेतुः कैश्चि\textbf{न्नैयायिकै}रनुपसंहार्य इत्युक्तम् । ततोऽनुपसंहार्योऽयं हेत्वाभास इति शब्दश्रवणार्थ...साहः करणीयः ।
	\pend
      

	  \pstart \textbf{उद्द्योयकर}स्तु श्रावणत्वाख्येऽसाधारणहेतावाचार्य\textbf{दिग्नागेन} दर्शित इदमवादीत्—“यद्येतच्छ्रावणत्वं नित्यानित्ययोर्दृष्टं स्याज्जनयेत्तयोः संशयमूर्ध्वत्वमिव स्थाणुपुरुषयोः । न च दृष्टम् । तस्मान्नायं संशयहेतुरपि त्वप्रतिपत्तिहेतुरेव । अथ श्रावणत्वं वस्तुधर्मः । वस्तुना नित्येन भाव्यमनित्येन वा प्रकारान्तराभावात् । न च तयोरेकत्रापि दृष्टम् । अतस्तयोः संशयं करोति । तर्हि वस्तुधर्मत्वात् संशयो न श्रावणत्वादिति ।” तुल्यन्यायतयाऽत्राप्यसाधारणे तदीयमिदमीदृशं चर्चितमासज्यत एवेति “कथमयं संशयहेतुरुपपादयितव्यः” इति साधूक्तं तेन । केवलं गमकरूपविवेचने समीचीनमनो न प्रहितम् । यतो वस्तुधर्मत्वं श्रावणत्वस्य नित्याकारसंस्पर्शिज्ञानजनने निबन्धनम् । न तु तस्मादेव वस्तुधर्मादुभयाकारसंस्पर्शी प्रत्ययो दोलायते न च यद्यस्य प्रतिपत्तिकारणे कारणम् तत एव सा प्रतिपत्तिः, न तु तस्मादिति शक्यते वक्तुम् । तदुत्पत्तेरग्निप्रतिपत्तिर्न तु धूमादित्यस्याभिधानप्रसङ्गात् ।
	\pend
      \leavevmode\marginnote{\textenglish{223/dm}}

	  \pstart किञ्चैवं प्रमेयत्वादेरपि न संशयः स्यात् । शक्यते हि तत्रापि वक्तुमुभयत्र दर्शनात् संशयो न प्रमेयत्वादिति । अथ तस्य तावदुभयत्र दर्शनं तेन तस्मादुच्यते । यद्येवं वस्तुधर्मत्वमपि श्रावणत्वस्यैवेति कथं न तस्मादसौ । अपि चोर्ध्वत्वमपि यद्यपि स्थाणुपुरुषयोरपि दृष्टं तथापि तावत्तत्रान्तरेणात्रैव भविष्यतीति पर्यनुयोगे सतीदमेव वाच्यम्--यदुतोर्ध्वत्वं नाम वस्तुधर्मः । वस्तुना चैवंविधिः--स्थाणुना पुरुषेण वाऽवश्यं भाव्यमिति । तथा च वस्तुधर्मादेव संशयो नोर्ध्वत्वादित्यनिष्टापादनं--केन निराक्रियेतेत्यलं विस्तरेण ।
	\pend
      

	  \pstart साधनस्य सिद्धेर्यन्नाङ्गमसिद्धो विरुद्धोऽनैकान्तिको हेत्वाभासः । तस्यापि वचनं वादिनो निग्रहस्थानमसमर्थोपादानात् । तस्मादेवंविधो हेत्वाभासः स्वयमप्रयोज्यः परप्रयुक्तश्चावश्यमुद्भावयितव्य इति हेत्वाभासव्युत्पादने \textbf{वार्त्तिककार}स्याभिप्रायो बोद्धव्यः ।
	\pend
      

	  \pstart स्यादेतत्--असमर्थंविशेषणोऽसमर्थविशेष्यश्चास्ति प्रभेदः । यथाऽनित्यः प्रमेयत्वे सति कृतकत्वात् । अत्र कृतकत्वं विशेष्यमेव साध्यसिद्धौ समर्थम्, न तु प्रमेयत्वं विशेषणमित्यसमर्थं विशेषणम्, यत्र विशेष्यमेव समर्थमिति कृत्वा भवत्यसमर्थविशेषणो हेतुः । यञ्च \footnote{च्चा} नित्यः शब्दः कृतकत्वे सति प्रमेयत्वादिति । अत्र हि कृतकत्वं विशेषणमेव साध्यसिद्धौ समर्थम्, न तु प्रमेयत्वं विशेष्यमित्यसमर्थं विशेष्यम् । यत्र हि विशेषणमे \leavevmode\marginnote{\textenglish{75a/ms}} व समर्थमिति कृत्वा भवत्ययमसमर्थविशेष्यो हेतुः । शेषमुभयीविधास्वन्तर्भाव्यताम् । न तावदसिद्धे, द्वयोरपि धर्मिणि सिद्धेः । न च विरुद्धे, विपर्ययव्याप्त्यभावात् । नाप्यनैकान्तिके कृतकत्वविशिष्टप्रमेयत्वस्य प्रमेयत्वविशिष्टकृतकत्वस्य च साध्याऽव्यभिचारात् । तस्मादसिद्धत्वादेरन्य एवायं हेतुदोषप्रकारः प्राप्त इति ।
	\pend
      

	  \pstart तदेतदवद्यम्, हेत्वदोषात् । यदि ह्येवमयं प्रयुक्तो हेतुर्द्विष्येत् \footnote{त} तदाऽस्यामीषु हेतुराशिष्वन्तर्भावश्चिन्त्येत, अन्यो वा हेतुदोषोऽभ्युपगम्येत । यावता नैवमयं प्रयुक्तोऽन्यथेति साध्यसाधनादिति । न तर्ह्येवं वादी निगृह्यत इति चेत् । किं न निगृह्यते, असाधनाङ्गवचनात् ? उभयत्रापि साध्यसिद्ध्यनङ्गस्य प्रमेयत्वस्यासमर्थस्याभिधानात् । यथा च साध्यसिद्ध्यनङ्गस्य वचने निग्रहोऽवश्यम्भावी, अनिग्रहे वा दोषः, तथा \textbf{वादन्यायेऽवादीन्न्यावादी}ति ततस्तदपेक्षितव्यः । ततोऽयमर्थो वक्तृदोष एव न हेतुदोषः । तेनानन्तर्भावेऽपि न हेत्वाभासानुषङ्ग इति ।
	\pend
      

	  \pstart भवतु तावदत्रेयं गतिः । सिद्धसाधने तु साधने किं भविष्यति ? न तावत् सिद्धसाधनं साधन\add{म}सिद्धत्वाद्यन्यतमदोषदूषितं साध्यसाधनसामर्थ्याप्रच्युतेरिति ।
	\pend
      

	  \pstart अत्रोच्यते--इह हेतुर्द्वेधा दुष्यति । कश्चिदसामर्थ्यात्, अपरो वैयर्थ्यात् । तत्रासामर्थ्य एव दोषो \textbf{वार्त्तिककारेणाऽ}नन्तरोक्तेन क्रमेण त्रिधा दर्शितः । न तु वैयर्थ्यलक्षणः । सिद्धसाधनं तु वैयर्थ्यलक्षणोऽन्य एवायं हेतोः स्वगतो दोष इति कस्मादस्यान्तर्भावश्चिन्तनीयः ? यदाहाचार्यः--“अन्यथानिष्ठं\footnote{ष्टं}भवेद् विफलमेव वा । तथा न साध्यत्वे वैकल्याद्” इत्यादीति ।
	\pend
      

	  \pstart वक्तृदोष एवैष इत्यपि वार्त्ता, यथायोगं परिपूर्णसाधनरूपाभिधानादनुपयुक्तानभिधानाच्च वक्तुरदुष्टत्वात् । वक्ताऽयं हेतुनिश्चितो \footnote{ता}ऽर्यप्रयुक्तो वैयर्थ्यमनुभवति । न तु \leavevmode\marginnote{\textenglish{224/dm}} “
	  
	त्रयाणां रूपाणामसिद्धौ\footnote{०मसिद्धिसंदेहे हेतु० \cite{dp-msC}} सन्देहे च\footnote{वा \cite{dp-msD}} हेतुदोषानुपपाद्योपसंहरन्नाह-- “
	  
	\footnote{एवं त्र० \cite{dp-msB} \cite{dp-msD} \cite{dp-edP} \cite{dp-edH} \cite{dp-edE} \cite{dp-edN}}एवमेषां त्रयाणां रूपाणामेकैकस्य र्द्वयोर्द्वयोर्वा रूपयोरसिद्धौ संदेहे वा\footnote{च \cite{dp-msB} \cite{dp-edP} \cite{dp-edH} \cite{dp-edE} \cite{dp-edN}} यथायोगमसिद्धविरुद्धानैकान्तिकास्रयो हेत्वाभासाः ॥ १०९ ॥” 
	  
	एवमित्यनन्तरोक्तेन क्रमेण । एषां मध्य एकैकं रूपं \footnote{यदसिद्धं \cite{dp-msA} \cite{dp-msB} \cite{dp-edP} \cite{dp-edH} \cite{dp-edN}}यदाऽसिद्धं सन्दिग्धं वा\footnote{“वा” नास्ति \cite{dp-msB}} भवति, द्वे द्वे वाऽसिद्धे संदिग्धे वा\footnote{“वा” नास्ति \cite{dp-msB}} भवतः, तदासिद्धश्च विरुद्धश्चानैकान्तिकश्च ते हेत्वाभासाः । यथायोगमिति । यस्यासिद्धौ संदेहे वा यो हेत्वाभासो युज्यते स तस्याऽसिद्धेः संदेहाच्च व्यवस्थाप्यत इति यस्य यस्य \footnote{“येन” नास्ति \cite{dp-msB}}येन येन योगो यथायोगमिति ॥ “
	  
	विरुद्धाव्यभिचार्यपि संशयहेतुरुक्तः । स इह कस्मान्नोक्तः ? ॥ ११० ॥” 
	  
	ननु चाऽऽचार्येण विरुद्धाव्यभिचार्यपि संशयहेतुरुक्तः । हेत्वन्तरसाधितस्य \footnote{०स्य यद्विरुद्धं तन्न \cite{dp-msB} \cite{dp-msD}}विरुद्धं यत् तन्न व्यभिचरतीति\footnote{०रति स विरु० \cite{dp-msA} \cite{dp-msB} \cite{dp-edP} \cite{dp-edH} \cite{dp-edE} \cite{dp-edN}} विरुद्धाव्यभिचारी । यदि वा विरुद्धश्चासौ साधनान्तरसिद्धस्य धर्मस्य विरुद्धसाधनात्, अव्यभिचारी च स्वसाध्याव्यभिचाराद्विरुद्धाव्यभिचारी ॥” स्वतोऽ\footnote{तो}दुष्टस्ततो वक्तृदोषो युज्यत एवेति चेत् । तर्हि विरुद्धत्वमपि वक्तृदोषोऽस्तु न्यायस्य समानत्वात् । शक्यते हि तत्राप्येवमभिधातुम्--वक्ताऽयमननुरूपे साध्ये प्रयुक्ते विपर्ययसाधानाद् विरुद्धतामनुभवति न त्वयं स्वतो दुष्टो नामेति । विवक्षितार्थसाधनासामर्थ्यं तावदस्य स्वतोऽस्ति तेना\footnote{न}हेतुदोष एवायमिति चेत् । इहापि निश्चितार्थनिश्चयनं तावदस्य स्वतोऽस्तीति कथं न वैयर्थ्यं तस्य दोष इति चिन्त्यतामिति ।
	\pend
      

	  \pstart निग्रहस्त्वेवंवादिनोऽसाधनाङ्गवचनाद् बोद्धव्यः । सिद्धि\add{ः}साधनं तदङ्गंम् धर्मो यस्यार्थस्य विवादाश्रयस्य वादप्रस्तावाद् हेतोः । स साधनाङ्गः । तथा यो न भवति तस्याप्रस्तुतस्याभिधानादिति कृत्वेति सर्वमेवावदातम् ।
	\pend
      

	  \pstart केचित्पुनरेवमसिद्धेऽन्तर्भावयितुं प्रयतन्ते । अन्ये तु विरुद्धे । यथा च तेषां प्रयतिर्यथा तदभजमानमभिधानं तथा \textbf{स्वयूथ्यविचार} एवाभिहित इति तत एवापेक्षितव्य इति ॥
	\pend
      

	  \pstart \textbf{त्रयाणामि}त्यादि \textbf{व्यवस्थाप्यत} इत्येतदन्तं सुबोधम् ।
	\pend
      

	  \pstart \textbf{यस्ये}ति हेत्वाभासस्य । सर्वहेत्वाभाससङ्ग्रहणार्थं \textbf{यस्ये}ति द्विरुक्तं \textbf{येन} \leavevmode\marginnote{\textenglish{75b/ms}} \textbf{येन} दोषेण \textbf{योगः} सम्बन्धः । एतच्चार्थकथनम् । योगानतिक्रमेणेति विग्रहः कार्यः ॥
	\pend
      

	  \pstart \textbf{ननु चे}त्यादीत्याहेत्येतदन्तं सुगमम् ।
	\pend
      \leavevmode\marginnote{\textenglish{225/dm}}“

	  \pstart सत्यम् । उक्त आचार्येण । मया त्विह नोक्तः । कस्मादित्याह--
	\pend
       “

	  \pstart अनुमानविषयेऽ\footnote{०विषये तस्यासं० \cite{dp-msC}}सम्भवात् ॥ १११ ॥
	\pend
      ” 

	  \pstart अनुमानस्य विषयः प्रमाणसिद्धं त्रैरूप्यम् । यतो हि अनुमानसम्भवः\footnote{अनुमानसद्भावः \cite{dp-msA} \cite{dp-msB} \cite{dp-msC} \cite{dp-edP} \cite{dp-edH} \cite{dp-edN} अनुमानस्य सम्भवः \cite{dp-edE}} सोऽनुमानस्य विषयः । प्रमाणसिद्धाच्च त्रैरूप्यादनुमानसम्भवः\footnote{अनुमानसद्भावः \cite{dp-msA} \cite{dp-msB} \cite{dp-edP} \cite{dp-edH} \cite{dp-edN}} । तस्मात् तदेवानुमानविषयः । तस्मिन् प्रक्रान्ते न विरुद्धाव्यभिचारिसम्भवः । प्रामाणसिद्धे हि त्रैरूप्ये प्रस्तुते स एव हेत्वाभासः सम्भबति यस्य प्रमाणसिद्धं रूपम् । न च विरुद्धाव्यभिचारिणः प्रमाणसिद्धमस्ति रूपम् । अतो न सम्भवः । ततोऽसम्भवात्\footnote{ततोऽसम्भवो नोक्तः \cite{dp-msA} \cite{dp-msB} \cite{dp-edP} \cite{dp-edH}} नोक्तः ॥
	\pend
       

	  \pstart कस्मादसंभव इत्याह--
	\pend
       “

	  \pstart न हि सम्भवोऽस्ति कार्यस्वभावयोरुक्तलक्षणयोरनुपलम्भस्य च\footnote{वा \cite{dp-msD}} विरुद्धतायाः ॥११२॥
	\pend
      ” 

	  \pstart न हीति । यस्मान्न सम्भवोऽस्ति विरुद्धतायाः । कार्य च स्वभावश्च तयोरुक्तलक्षणयोरिति ।
	\pend
      ”

	  \pstart \textbf{अनुमानस्य} त्रैरूप्याल्लिङ्गसम्बन्धिनोऽन्यस्मात् \textbf{त्रैरूप्यं विषयः} । यथा मत्स्यानां विषयो जलमिति । एतदेवाह--\textbf{यत} इति । हिर्यस्मादर्थे । \textbf{प्रक्रान्ते} प्रस्तुते अनुमानविषयो\footnote{ये} वै\footnote{त्रै}रूप्ये सति ।
	\pend
      

	  \pstart कस्मात्तत्र विरुद्धाव्यभिचारिणोऽसम्भव इत्याह--\textbf{प्रमाणे}ति । हीति यस्मात् । स एव \textbf{हेत्वाभासो} विरुद्धाव्यभिचार्याख्यः \textbf{सम्भवति यस्य प्रमाणेन सिद्धं रूपं} पक्षधर्मान्वयव्यतिरेकात्मकमिति विवक्षितम् ।
	\pend
      

	  \pstart अयमाशयः--यदि तदेकेन द्वाभ्यां वा रूपाभ्यां हीनं स्यात्तदेष्वेव हेत्वाभासेष्वन्तर्भवेत् । न त्वेतदतिरिक्तो विरुद्धाव्यभिचारी नाम हेत्वाभासो भवेत् । भवता त्वनेन प्रमाणसिद्धत्रैरूप्येणैव भाव्यमिति ।
	\pend
      

	  \pstart अस्तु तस्य तथात्वमित्याह--\textbf{न चे}ति । \textbf{चो}ऽवधारणे, व्यक्तमेतदित्यस्मिन्नर्थे वा ।
	\pend
      

	  \pstart एवं ब्रुवतश्चास्यायमभिप्रायः--वस्तुनः परस्परविरुद्धरूपद्वयासम्भवाद् अवश्यमनयोरेकमसम्पूर्णाङ्गमिति ॥
	\pend
      

	  \pstart कुतः पुनरनुमानविषयेऽस्यासम्भवोऽवसीयत इत्यभिप्रेत्य पृच्छति परः--\textbf{कस्मादि}ति  \leavevmode\marginnote{\textenglish{226/dm}} “
	  
	कार्यस्य कारणाज्जन्मलक्षणं तत्त्वम् । स्वमावस्य च साध्यव्याप्तत्वं तत्त्वम् । यत् कार्यम्, यश्च स्वभावः, स कथमात्मवारणं व्यापकं च स्वभावं परित्यज्य भवेद् येन विरुद्धः स्यात् । अनुपलम्भस्य च उक्तलक्षणस्येति । दृश्यानुपलम्भत्वं \footnote{०त्वमनुप० \cite{dp-msA} \cite{dp-msB} \cite{dp-msC} \cite{dp-edP} \cite{dp-edH} \cite{dp-edE} \cite{dp-edN}}चानुपलम्भलक्षणम् । तस्यापि\footnote{तस्यापि च स्वभावा० \cite{dp-msA} \cite{dp-msB} \cite{dp-edP} \cite{dp-edH}} वस्त्वभावाव्यभिचारित्वान्न विरुद्धत्वसम्भवः\footnote{०सम्भवः स्यात् । एते० \cite{dp-msA} \cite{dp-edP} \cite{dp-edH} ०सम्भवः स्यादेति [[?]] तत् एते० \cite{dp-msB}} ॥ 
	  
	स्यादेतत्--एतेभ्योऽन्यो भविष्यतीत्याह-- “
	  
	न चान्योऽव्यभिचारी ॥ ११३ ॥” 
	  
	न चान्य एतेभ्योऽव्यभिचारी त्रिभ्यः । अत \footnote{अत एव तेष्वेव \cite{dp-msA} \cite{dp-msB} \cite{dp-edP} \cite{dp-edH} \cite{dp-edE} \cite{dp-edN} अत एवैतेष्वेव \cite{dp-msC}}एवैष्वेव हेतुत्वम् ॥ 
	  
	क्व तर्ह्याचार्यदिग्नागेनायं हेतुदोप उक्त इत्याह-- “
	  
	तस्मादवस्तुदर्शनबलप्रवृत्तमागमाश्रयमनुमानमाश्रित्य तदर्थविचारेषु विरुद्वाव्यभिचारी साधनदोष उक्तः ॥ ११४ ॥” 
	  
	यस्माद् वस्तुबलप्रवृत्तेऽनुमाने न सम्भवति तस्माद् आगमाश्रयमनुमानमाश्रित्यव्यभिचार्युक्तः । आगमसिद्धं हि यस्यानुमानस्य लिङ्गत्रैरूप्यं तस्यागम आश्रयः । 
	  
	ननु चागमसिद्धमपि त्रैरूप्यं प्रमाणसिद्धमित्याह--अवस्तुदर्शनबलप्रवृत्तमिति । अवस्तुनो दर्शनं विकल्पमात्रम् तस्य बलं सामर्थ्यम् । ततः प्रवृत्तम्--अप्रमाणाद्विकल्पमात्राद् व्यवस्थितं त्रैरूप्यमागमसिद्धमनुमानस्य । न तु प्रमाणात् ।” \textbf{नही}त्यत्रस्थस्य हिशब्दस्यार्थो \textbf{यस्मादि}त्यनेनोक्तः । \textbf{विरुद्धतायाः} एकसाधनसाधितस्यार्थस्य प्रत्यनीकपक्षसाधनरूपतायाः ।
	\pend
      

	  \pstart एतदुक्तं मवति--स्वसाध्याव्यभिचारिणा हि भवता कार्येण स्वभावेन वा भाव्यम् । \textbf{न च} वस्तुनस्तदतत्स्वभावौ स्तो येन तदतत्स्वभावाव्यभिचारिणौ द्वौ हेतू सन्निपतन्तौ विरुद्धा\textbf{व्यभिचारिणौ} स्यातामिति । येन कार्यस्वभावयोः कारणव्यापकविधिना कृसतद्भावेन विरुद्धं तत्रैव धर्मिणि हेत्वन्तरसाधितार्थं विरुद्धसाधनं भवेत् ।
	\pend
      

	  \pstart यद्येवमनुपलम्भे तत्सम्भविष्यतीत्याह--\textbf{अनुपलम्भस्ये}ति । चः पूर्वापेक्षः समुच्चये ॥
	\pend
      

	  \pstart \textbf{यत} एतदतिरिक्तोऽव्यभिचारी \add{न} सम्मत \add{\textbf{अत}} एवास्मादेव कारणात् । \textbf{एष्वेव} कार्यस्वभावानुपलम्भेष्वेव ॥
	\pend
      

	  \pstart \textbf{कथमा}गमाश्रयत्वमनुमानस्येत्याह--\textbf{आगमसिद्धमि}ति । हिर्यस्मादर्थे । सति तस्मिन्नागमेऽनुमानस्य प्रवृत्तेरसावाश्रयस्तस्य ।
	\pend
      \leavevmode\marginnote{\textenglish{227/dm}}“

	  \pstart तत् तर्ह्यनुमान\footnote{नुमानेनागम० \cite{dp-msA} \cite{dp-msB} \cite{dp-edP} \cite{dp-edH}} मागमसिद्ध\footnote{सिद्धं त्रैरू० \cite{dp-msC}} त्रैरूप्यं क्वाधिकृतमित्याह--\footnote{०त्याह तस्यागम० \cite{dp-msB}}तदर्थेति । तस्यागमस्य योऽर्थोऽतीन्द्रियः प्रत्यक्षानुमानाभ्यामविषयीकृतः सामान्यादिस्तस्य विचारेषु प्रक्रान्तेषु आगमाश्रयमनुमानं सम्भवति । तदाश्रयो विरुद्धाव्यभिचार्युक्त आचार्येणेति ॥
	\pend
       

	  \pstart कस्मात् पुनरागमाश्रये\footnote{०श्रयोऽप्य \cite{dp-msA}} प्यनुमाने सम्भव इत्याह--
	\pend
       “

	  \pstart शास्त्रकाराणामर्थेषु भ्रान्त्या \footnote{विपरीतस्य स्वभा० \cite{dp-msB} \cite{dp-edP} \cite{dp-edH} \cite{dp-edE}}विपरीतस्वभावो\footnote{भावस्योपसं० \cite{dp-edN}} पसंहारसम्भवात् ॥ ११५ ॥
	\pend
      ” 

	  \pstart शास्त्रकृतां विपरीतस्य वस्तुविरुद्धस्य स्वभावस्य उपसंहारो ढौकनमर्थेषु । तस्य संभवाद् विरुद्धाव्यभिचारिसम्भवः । भ्रान्त्येति विपर्यासेन । विपर्यस्ता हि शास्त्रकाराः\footnote{०कारास्तं तमसन्तं स्वभा० \cite{dp-edE}} सन्तमसन्तं स्वभावमारोपयन्तीति ॥
	\pend
       

	  \pstart यदि शास्त्रकृतोऽपि भ्रान्ताः, अन्येष्वपि पुरुषेषु क आश्वास इत्याह--
	\pend
       “

	  \pstart न ह्यस्य सम्भवो\footnote{सम्भवोऽस्ति यथा० \cite{dp-msC}} यथावस्थितवस्तुस्थितिष्वात्मकार्यानुपलम्भेषु\footnote{०ष्वात्मकार्येषूपलम्भेषु--\cite{dp-msB} \cite{dp-edP} \cite{dp-edH}} ॥ ११६ ॥
	\pend
      ” 

	  \pstart नहीति । न हेतुषु कल्पनया हेतुत्वव्यवस्था । अपि तु वस्तुस्थित्या । ततो यथावस्थितवस्तुस्थितिष्वात्मकार्यानुपलम्भेष्वस्य सम्भवो नास्ति ।
	\pend
       

	  \pstart अवस्थितं परमार्थसद्वस्तु तदनतिक्रान्ता यथावस्थिता \footnote{०स्थितवस्तु० \cite{dp-msC}}वस्तुस्थितिर्व्यवस्था\footnote{०स्थितिव्यव० \cite{dp-msA} \cite{dp-msB} \cite{dp-edP} \cite{dp-edH} \cite{dp-edE}} येषां ते यथावस्थितवस्तुस्थितयः । ते हि यथा वस्तु स्थितं तथा स्थिताः\footnote{तथा स्थापिता न \cite{dp-msC} \cite{dp-msD}} । न कल्पनया । \footnote{अतः \cite{dp-msA} \cite{dp-msB} \cite{dp-edP} \cite{dp-edH} \cite{dp-edE} \cite{dp-edN}}ततस्तेषु न भ्रान्तेरवकाशोऽस्ति येन विरुद्धाव्यभिचारिसम्भवः स्यात् ॥
	\pend
       

	  \pstart तत्र विरुद्धाव्यभिचारिण्युदाहरणम्--
	\pend
      ”

	  \pstart \textbf{ननु चेत्यादि विपर्यस्ता ही}त्येतदन्तं सुगमम् । \textbf{शास्त्रकारा} इति तीर्थिकशास्त्रप्रणेतार इति द्रष्टव्यम्, तद्वचनस्यैव प्रमाणबाधितत्वेन तेषामेव विपर्यस्तत्वात् ॥
	\pend
      

	  \pstart \textbf{अन्येष्वपी}ति कार्यादिहेतुप्रयोक्तृषु ।
	\pend
      

	  \pstart \textbf{यथावस्थितवस्तुस्थितिष्विति}--अस्य तात्पर्यार्थमाह--\textbf{न हेतुष्वि}ति । ततः कल्पनया हेतुत्वाद्य\footnote{त्वव्य}वस्थायाः । अर्थक्रियासमर्थत्वं\footnote{न्तु} \textbf{परमार्थसत्} । कथं ते तथारूपा इत्याह--\textbf{ते ही}ति । \textbf{ते} कार्यादयो हिर्यस्मादर्थे । हेतुभावे चैतद्विशेषणम् । यतस्ते यथावस्थितयस्ततस्तेष्व\footnote{षु}सम्भवो नास्तीत्यर्थः । सत्यां स्थितौ किं न सम्भव इत्याह--\textbf{तत} इति ।    \leavevmode\marginnote{\textenglish{228/dm}} “
	  
	\footnote{अत्रोदा० \cite{dp-edE}}तत्रोदाहरणम्--यत् सर्वदेशावस्थितैः \footnote{०तैः सम्बन्धिभिः सम्ब० \cite{dp-msC} तैः स्वसम्वन्धिभिः सम्ब० \cite{dp-msB} \cite{dp-edP} \cite{dp-edH} \cite{dp-edN}}स्वसम्बन्धिभिर्युगपदभिसम्बध्यते तत् सर्वगतम् । यथाऽऽकाशम् ।\footnote{०काशमिति \cite{dp-msC}} अभिसम्बध्यते च\footnote{“च” नास्त्रि \cite{dp-msB} \cite{dp-edP} \cite{dp-edH} \cite{dp-edE}} सर्वादेशावस्थितैः स्वसम्बन्धिभिर्युगपत् सामान्यमिति ॥ ११७ ॥” “
	  
	यत् सर्वस्मिन् देशेऽवस्थितैः स्वसम्बन्धिभिर्युगपदभिसम्बध्यते \footnote{ध्यते तत्सर्वः \cite{dp-msA} \cite{dp-msB} \cite{dp-edP} \cite{dp-edH} \cite{dp-edE} \cite{dp-edN}}इति सर्वदेशावस्थितरभिसम्बध्यमानत्वं सामान्यस्य अनूद्य सर्वगतत्वं विधीयते । तेन युगपदभिसम्बध्यमानत्वं सर्वगतत्वे नियतं तेन व्याप्तं कथ्यते । 
	  
	इह सामान्यं कणादमहर्षिणा निष्क्रियं दृश्यमेकं\footnote{दृश्यमेवोक्तम् \cite{dp-msB}} चोक्तम् । युगपच्च सर्वैः स्वैः \footnote{सर्वैः स्वैः स्वैः सम्ब० \cite{dp-msA} \cite{dp-edP} \cite{dp-edH} \cite{dp-edE} \cite{dp-edN} सर्वैः स्वैः स्वैः स्वसम्ब० \cite{dp-msB}}सम्बन्धिभिः समवायेन संबद्धम् । तत्र पैलुकेन कणादशिष्येण व्यक्तिषु व्यक्तिरहितेषु च” यतस्ते कल्पनया नः स्थापितास्ततः कारणात् । \textbf{भ्रान्ते}र्विपर्यासस्यावसरोऽ\textbf{वकाशः । येन} भ्रान्त्यवकाशेन ॥
	\pend
      

	  \pstart कस्यानुवादेनात्र कस्य विधिरित्या\leavevmode\marginnote{\textenglish{76a/ms}}ह--\textbf{सर्वे}ति । \textbf{सर्वदेशावस्थितैः}--स्वसम्बन्धिभिरित्यर्थात् । यत एवमनुवादविधिक्रमस्तेन हेतुना ।
	\pend
      

	  \pstart ननु सर्वैः स्वसम्बन्धिभिर्युगपदभिसम्बन्धो नाम सामान्यस्य युगपत्सर्वसम्बन्धिसमवाय एव । सर्वगतत्वमपीदमेवास्येति ।
	\pend
      

	  \pstart कथमनयोर्व्यावृत्तितोऽपि भेदसम्बन्धभावतो गम्यगमकभाव इति चेत् । नैष दोषः । नानादेशस्थैः स्वसम्बन्धिभिः शावलेयादिभिर्युगपदभिसम्बन्धो हेतुः । सम्बन्धिदेशतदन्तरालव्यापित्वं तु साध्यमिति गम्यगमकभावो न विरुध्यते । सर्वसम्बन्धिभिर्युगपदभिसम्बन्धश्चागत्वाऽनागच्छद्भिरिति द्रष्टव्यम् ।
	\pend
      

	  \pstart अथ केन विरुद्धो\footnote{द्धा}व्यभिचारिप्रसवबीजं धर्मद्वययोः किमभ्युपगतं येन तयोः सन्निपाताद् विरुद्ध\footnote{द्धा}व्यभिचारिसम्भव इत्याह--\textbf{इहे}ति । \textbf{इह} सामान्यपदार्थविचारप्रक्रमे । कणमत्तीति \textbf{कणादः} । रूढिवशाच्चायं शब्दः \textbf{काश्यपे} मुनौ वर्त्तते । स चासौ महर्षिश्चेति । हेतुभावेनास्य विशेषणत्वात् कणादत्वादेव महर्षिः । एवं तस्य हि काष्ठागता निःस्पृहता यतोऽ न्यरवो\footnote{?}स्वभोज्यादिकमपि परित्यज्य कणमात्रं भुक्त्वा ध्यानादिकमाचरति । अथोऽसावन्येभ्यः सातिशयवान् भवतीति । \textbf{निष्क्रियं} क्रियाशून्यममूर्त्तत्वात् । \textbf{एक}मनानारूपम्, प्रत्येकं स्वाश्रयेषु लक्षणाविशेषाद्, विशेषलक्षणाभावाच्च । न तु समवायादेकं त्रिलोक्यां सामान्यम्, प्रत्ययभेदात् परस्परतोऽन्यत्वात् । गोत्वादीनाञ्च निष्क्रियत्वेन सहाऽस्यैकत्वं समुच्चिनोति ।  \leavevmode\marginnote{\textenglish{229/dm}} “
	  
	देशेषु सामान्यं स्थितं साधयितुं \footnote{प्रमाणमुप० \cite{dp-msC}}प्रमाणमिदमुपन्यस्तम् यथाकाशमिति--व्याप्तिप्रदर्शनविषयो दृष्टान्तः । आकाशमपि हि सर्वदेशावस्थितैर्वृक्षादिभिः स्वसंयोगिभिर्युगपदभिसम्बध्यमानं सर्वगतं च । अभिसम्बध्यते च\footnote{०ते वा सर्व० \cite{dp-msB}} सर्वदेशावस्थितैः स्वसम्बन्धिभिरिति हेतोः पक्षधर्मत्वप्रदर्शनम् ॥ 
	  
	अस्य स्वभावहेतुत्वं \footnote{प्रयोजयितुमाह--\cite{dp-msA} \cite{dp-msB} \cite{dp-edP} \cite{dp-edH} \cite{dp-edN} योजयन्नाह \cite{dp-edE}}योजयितुमाह-- “
	  
	तत्सम्बन्धिस्वभावमात्रानुबन्धिनी तद्देशसन्निहितस्वभावता ॥ ११८ ॥” 
	  
	तत्सम्बन्धीति । तेषां सर्वदेशावस्थितानां द्रव्याणां सम्बन्धी सामान्यस्य स्वभावः स एव तत्सम्बन्धिस्वभावमात्रम् । तदनुबध्नातीति तदनुबन्धिनी । 
	  
	कासावित्याह--तद्देशसंन्निहितस्वभावता । तेषां सम्बन्धिनां देशस्तद्देशः । तद्देशे सन्निहितः स्वभावो यस्य तत् तद्देशसंनिहितस्वभावम्\footnote{स्वभावः \cite{dp-msC} \cite{dp-msD} \cite{dp-edE}} । तस्य भावस्तत्ता । यस्य हि येषां सम्बन्धी स्वभावः तन्नियमेन तेषां देशे सन्निहितं भवति । ततस्तत्सम्बन्धित्वानुबन्धिनी तद्देशसंनिहितता सामान्यस्य ॥” \textbf{युगपदे}ककालम् । \textbf{चः} पूर्वापेक्षया समुच्यये \textbf{समवायेन} सम्बन्धेन सम्बद्धत्वम् । एवमभिहिते \textbf{कणादेन} तच्छिष्येण \textbf{पैलुकेन} । पीलवः परमाणवः । पीलुपाके चायं पीलुशब्द उपचारासतास्त्त्येन \footnote{उपचरितोऽस्ति । तेन} निमित्तेन व्यवहरतीति \textbf{पैलुकः} । तेनाविवक्षिताऽऽन्तरभेदस्ययुगपत्सर्वसम्बन्धमात्रहेतुत्वादाकाशस्य दृष्टान्तरूपता द्रष्टव्या । न तु सामान्यस्येवास्य सम्बन्धिभिः समवायेन सम्बन्धः । संयोगलक्षणेनास्य सम्बन्धेन सम्बन्धात् ।
	\pend
      

	  \pstart एतदेवाभिप्रेत्याह--\textbf{आकाशमपी}ति । हीति यस्मात् । \textbf{स्वसंयोगिभि}रिति वास्तवानुवादः । \textbf{न} त्येतत् प्रकृताङ्गम् । दार्ष्टान्तिकेऽस्यानुपपत्तेः । \textbf{चो}ऽभिसम्बन्ध\footnote{न्ध्य}मानत्वेन सह सर्वगतत्वस्यैकविषयतां समुच्चिनोति ॥
	\pend
      

	  \pstart \textbf{द्रव्याणां} गवादीनाम् । एतच्च गोत्वादिसामान्यविवक्षयोक्तम् । उपलक्षणं द्रव्यग्रहणं कर्त्तव्यम्, इतरथोत्क्षेपणत्वादिसामान्यस्यासङ्ग्रहः स्यात् ।
	\pend
      

	  \pstart एतच्च \textbf{तस्य भाव} इत्येदन्तं सुगमम् ।
	\pend
      

	  \pstart \footnote{०ते वा सर्व० \cite{dp-msB}}\textbf{तस्मादि}त्यनेनार्थागतं स्वभावहेतुत्वनिमित्तं दर्शयति । \textbf{न} तु तद्धितप्रत्ययान्ते पञ्चम्यस्ति यां व्याचक्षीत । अयं त्वस्यार्थः--यस्माद् युगपत् सर्वदेशावस्थितसम्बन्धिसम्बन्धः स्वसत्तामात्रानुबन्धिनि साध्ये हेतुः, तस्मात्स्वभाव हेतुत्वमस्येति ।
	\pend
      

	  \pstart ननु तत्सम्बन्धिनोऽपि तद्देशसन्निहितस्वभावतैव कुतो येनैवं भवतीत्याह--\textbf{यस्ये}ति । हिर्यस्मात् । यतोऽयं सामान्यन्यायः \textbf{तत}स्त\leavevmode\marginnote{\textenglish{76b/ms}}स्मात् । यद्वा सर्वसम्बन्धित्वेऽपि कस्मा  \footnote{अत्र मूले \textbf{“तस्य भावः । तस्मात्”} इति पाठः कल्प्यः--सं०} \leavevmode\marginnote{\textenglish{230/dm}} “
	  
	ननु च गवां सम्बन्धी स्वामी । न च \footnote{न च तद्देशे सन्नि० \cite{dp-msA} \cite{dp-msB} \cite{dp-edP} \cite{dp-edH} \cite{dp-edE} न च तद्देशसंनि० \cite{dp-edN}}गोदेशे सन्निहितस्वभावः । \footnote{०भावः स्वामी \cite{dp-msA} \cite{dp-msB} \cite{dp-edP} \cite{dp-edH} \cite{dp-edE} \cite{dp-edN}}तत् कर्थं \footnote{कथं संबं० \cite{dp-msA} \cite{dp-msB} \cite{dp-msC} \cite{dp-msD} \cite{dp-edP} \cite{dp-edH} \cite{dp-edN}}तत्सम्बन्धित्वात् तद्देशत्वमित्याह-- “
	  
	न हि यो यत्र नास्ति तद्देशमात्मना व्याप्नोतीति स्वभावहेतुप्रयोगः ॥ ११९ ॥” 
	  
	न हीति । यो यत्र देशे नास्ति स देशो यस्य स तद्देशः तं न व्याप्नोत्यात्मना स्वरूपेण । 
	  
	इह सामान्यस्य तद्वतां च समवायलक्षणः \footnote{लक्षणसम्ब० \cite{dp-msA}}सम्बन्धः । स चाभिन्नदेशयोरेव । \footnote{अनेन \cite{dp-msB}}तेन यत्र यत् समवेतं \footnote{कर्त्तृ--\cite{dp-msD-n}}तत् \footnote{कर्म--\cite{dp-msD-n}}तदात्मी न रूपेण क्रोडीकुर्वत् \footnote{समवायरूप० \cite{dp-msC}}समवायिरूपदेशे स्वात्मानं निवेशयति ।” त्तद्देशसन्निहितस्वभावतेत्याह--\textbf{तस्मादि}ति । \textbf{तस्मा}त्तत्सम्बन्धिमात्रानुबन्धिनीति । तद्देशसन्निहितस्वभावताऽऽकाशस्य दृष्टा \textbf{तस्मात्} कारणात् । \textbf{यस्य} वस्तुनस्तेषां \textbf{\footnote{नो येषां} सम्बन्धी स्वभावः । हि}रवधारणे \textbf{सम्ब}न्धीत्यस्मात्परो द्रष्टव्यः । तद् वस्तु \textbf{नियमेनाव}श्यंतया \textbf{तेषां} सम्बन्धीनां \textbf{देशे सन्निहि}तं \textbf{भवति} । यत एवं सामान्यन्याय\textbf{स्ततः । तस्मादि}ति पाठे भावगतिः । \footnote{अत्र मूले “तस्य भावः । कस्मात् ?” इति पाठः कल्प्यः--सं०}\textbf{कस्मादि}ति \textbf{तु} क्वचित्पुस्तके पाठः । स तु युक्तरूपः । तत्सम्बन्धिस्वभावमावमात्रानुबन्धिनी तद्देशसन्निहितस्वभावतेति व्याख्या । \textbf{कस्मादेतदि}ति कारणाकाड्क्षासम्बन्धात् । तदनन्तरं च \textbf{यस्य हीत्या}देः सामान्योत्तरस्य, \textbf{तत} इत्यादेश्चोपसंहारव्यपदेशेन विशेषोत्तरस्य \textbf{सुमो} \footnote{यो} ज्यत्वादिति ।
	\pend
      

	  \pstart यत्र तु \textbf{तस्य भावस्तत्ते}ति पाठः तत्र सर्वमवदातम् ॥
	\pend
      

	  \pstart ननु चैकदेशस्थमेव सामान्यं युगपत्सर्वैः सम्बन्धिभिरभिसम्भन्त्स्यते । तत्किं तस्य व्यापारासम्भवेनावेदितेनेत्याह--इहेति ।
	\pend
      

	  \pstart \textbf{स} इति समवायः । चकारः पुनः शब्दस्यार्थे । \textbf{अभिन्नदेशयो}रिति लोकप्रसिद्धदेशापेक्षयोक्तम् न तु शास्त्रप्रसिद्धदेशापेक्षयेति द्रष्टव्यम् । अन्यथा यदा \footnote{था पट} तन्तूनां समवायो न स्यात् । पटस्य तन्नथा \footnote{तन्तवो} देशः । तन्तूनां पुनरंशवः । सामान्यतद्वतोश्च न स्यात् । गोत्वसामान्यस्य गौर्देशः । गोश्च सास्नादयोऽवयवा इति ।
	\pend
      

	  \pstart अथवा सामान्यलक्षणयुगा \footnote{योगा} पेक्षया अभिन्नदेशत्वं विवक्षितम् । न तु प्रत्येकापेक्षम् । तेनायमर्थः--कुण्डबदरवद् यत्र द्वावपि सम्बन्धिनौ भिन्नदेशौ न तयोः समवायः । ययोस्त्वेकतरस्यान्यतरो देशस्तयो समवाय इति । एवञ्च पटतन्तूनां सामान्यतद्वतोश्च नासङ्ग्रह इति । येन कारणेनाभिन्नदेशयोरेव समवाय\textbf{स्तेन} प्रथमव्याख्याने \textbf{समवायिरूपस्य} समवायिस्वभावस्य देश इति । द्वितीयव्याख्याने \textbf{समवायिरूप}मेव समवायिस्वभाव एव \textbf{देश} इति विगृह्य तस्मिन्निति  \leavevmode\marginnote{\textenglish{231/dm}} “
	  
	\footnote{तद्देशे रूप० \cite{dp-msD}}तद्देशरूपनिवेशनमेव तत्क्रोडीकरणम् । ततस्तत्समवायः । 
	  
	तस्माद् यद् यत्र समवेतं तत् \footnote{तत् तत्र द्र० \cite{dp-edE}}तद्द्रव्यं व्याप्नुवदात्मना तद्देशे संन्निहितं भवति । 
	  
	तदयमर्थः--तद्देशस्थवस्तुव्यापनं तद्देशसत्तया व्याप्तम् । तद्देशसत्ताऽभावे\footnote{तद्देशसत्ताया अभावे \cite{dp-msB} \cite{dp-msC}} तद्व्यापनाभावाद् व्यापनलक्षण समवायसंम्बन्धो न स्यात् । अस्ति च व्यापनम् । अतस्तद्देशे सन्निहितत्वमिति । तदयं स्वभावहेतुः ॥ 
	  
	पैठरप्रयोगं दर्शयन्नाह-- “
	  
	द्वितीयोऽपि प्रयोगः--यदुपलब्धिलक्षणप्राप्तं सन्नोपलभ्यते न तत् तत्रास्ति । तद्यथा--क्कचिदविद्यमानो घटः ।\footnote{घट इति \cite{dp-msC}} नोपलभ्यते चोपलब्धिलक्षणप्राप्तं सामान्यं व्यक्तयन्तरालेष्विति । अयमनुपलम्भः\footnote{लम्भप्रयोगः स्व० \cite{dp-msD} \cite{dp-msB} \cite{dp-edP} \cite{dp-edH} \cite{dp-edE} \cite{dp-edN}} स्वभावश्च परस्परविरुद्धार्थसाधनादेकत्र संशयं जनयतः ॥ १२० ॥” 
	  
	द्वितीयोऽपि--इति । यदुपलब्धेर्लक्षणतां विषयतां प्राप्तं दृश्यमित्यर्थः । एतेन दृश्यानुपलब्धिमनूद्य “न \footnote{०नूद्य तत्तत्र \cite{dp-msB} नूद्य त [[न]] तत्तत्र \cite{dp-msA} नूद्य तत्तत्तत्र \cite{dp-edP} \cite{dp-edH}}तत् तत्रास्ति” इत्यसद्व्यवहार्यत्वं\footnote{व्यवहारविषयत्वं विहितम् \cite{dp-msD}} विहितम् । ततो \footnote{व्याप्यस्य दृ० \cite{dp-edE}}व्याप्यदृश्यानुपलब्धेर्व्यापकमसद्व्यवहार्यत्वं दर्शितम् । तद्यथेति क्वचिदसन् घटो दृष्टान्तः ।” योज्यम् । उभयत्रापि तु \textbf{समवाय} \footnote{यि} शब्देनाधारोऽभिप्रेतः । \textbf{स्वात्मानं निवेशयत्यु}पनयति ।
	\pend
      

	  \pstart ननु तद्व्यापनं तत्क्रोडीकरणमभिप्रेतम् । तत्कथं सम्बन्धिनि स्वात्मनि निवेशनं व्याख्यायत इत्याह--तद्देश इति । स चासौ देशश्च तत्र रूपस्य स्वरूपस्य \textbf{निवेशन}मुपनयनम् । \textbf{तत}स्तस्मात्तद्देशरूपनिवेशनात्तस्यः । सम्बन्धिनः समवायः । \textbf{तस्मादित्या}दिनोपसंहारः ।
	\pend
      

	  \pstart ननु तद्व्यापनमपि भविष्यति, न च तद्देशसन्निहितस्वभावतेत्याशङ्क्याह--\textbf{तदयमिति} । यत आत्मना तद्व्यापनलक्षणेन सम्बन्धेन तद्व्याप्यमानदेशसन्निधानमुक्तं \textbf{तत्त}स्मादयं तात्प\textbf{र्यार्थः} । स चासौ \textbf{देश}श्च तत्रस्थ\textbf{वस्तुव्यापनं} लोकप्रसिद्धदेशापेक्षया तस्य देश\textbf{स्तद्देश}स्तत्र या \textbf{सत्ता} विद्यमानता तया \textbf{व्याप्तम्} । अन्यथा तु स्वरूपेण व्यापनासम्भवादित्यभिप्रायः ।
	\pend
      

	  \pstart तदेव व्यतिरेकमुखेणोपपादयन्ना\leavevmode\marginnote{\textenglish{77a/ms}}ह--\textbf{तद्देशेति} । नास्त्येवायं सम्बन्ध इति \add{चेदाह--} \textbf{अस्ति चे}ति । \textbf{चो}ऽवधारणे । यत उक्तेन क्रमेण स्वभावलक्षणयोगोऽस्यास्ति । \textbf{तत्तस्मादयं} युगपत् सर्वसम्बन्ध्यभिसम्बन्धलक्षणो हेतुः \textbf{स्वभावः ॥}
	\pend
      \leavevmode\marginnote{\textenglish{232/dm}}“

	  \pstart पक्षधर्मत्वं दर्शयितुमाह--नोपलभ्यते चेति । \footnote{व्यक्तेरन्तरालं \cite{dp-msA} \cite{dp-msB} \cite{dp-edP} \cite{dp-edH}}व्यक्त्यन्तरालं--व्यक्त्यन्तरं च व्यक्तिशून्यं चाकाशम् । दृश्यमपि कस्यांचिद्व्यक्तौ गोसामान्यमश्वादिषु व्यक्त्यन्तरेषु व्यक्तिशून्ये\footnote{रेषु शून्ये चाका० \cite{dp-msA}} चाकाशे \footnote{चोपलभ्य० \cite{dp-msA} \cite{dp-msB} \cite{dp-edP} \cite{dp-edH}}नोपलभ्यते । तस्मान्न तेष्वस्तीति गम्यते ।
	\pend
       

	  \pstart अयमनुपलम्भः पूर्वोक्तश्च स्वभावः \footnote{परस्परविरु० \cite{dp-msA} \cite{dp-msB} \cite{dp-msC} \cite{dp-edP} \cite{dp-edH} \cite{dp-edE} \cite{dp-edN}}परस्परस्य विरुद्धौ यावर्थौ तयोः साधनात् तावेकस्मिन् धर्मिणि संशयं जनयतः । न ह्येकोऽर्थः परस्परविरुद्धस्वभावो भवितुमर्हति । \footnote{स्वभावेन--\cite{dp-msD-n}}एकेन चात्र \footnote{“व्यक्त्यन्तरेषु”--नास्ति \cite{dp-msB}}व्यक्त्यन्तरेषु व्यक्तिशून्ये चाकाशे सत्त्वम्, अपरेण चानुपलम्भेनासत्त्वं साध्यते । न
	\pend
      ”

	  \pstart \textbf{कणा}दस्यैवापरः शिष्यः \textbf{पैठर}स्तस्य \textbf{प्रयोगम्} । पिठरोऽवयविद्रव्यम् । पूववदुपचारात्पिठरशब्दस्तेन व्यवहरतीति तथोक्तः ।
	\pend
      

	  \pstart \textbf{असदव्यवहार्यत्व}मसदिति व्यवहारणीयत्वं \textbf{विहितम् । कस्याञ्चि}त्सास्नादिमत्यां व्याप्तौ \footnote{\textbf{व्यक्तौ}} व्यज्यते सामान्यमनयेति \textbf{व्यक्तिः} ।
	\pend
      

	  \pstart प्रागुक्तस्तावत्स्वभावः, अयं तु किंसंज्ञको हेतुरित्याह--\textbf{अयमिति} । प्रागुक्तं स्मारयति \textbf{पूर्वे}ति । तुशब्दश्चशब्दस्यार्थे । \textbf{तावे}तौ हेतू \textbf{धर्मिण्येकस्मिन्} सामान्याख्ये । \textbf{संशयं} प्रकृतयोः साध्ययोरित्यर्थात् कथं संशयं जनयत इत्याह । एकस्यैव तौ विवक्षितसर्वगतत्वासर्वंगतत्वलक्षणौ स्वभावौ भविष्यत इत्याह--\textbf{न हीति} । हिर्यस्मात् । \textbf{परस्परविरुद्धौ स्वभावौ} यस्येति तथा ।
	\pend
      

	  \pstart ननु चात्र विरुद्धावेव धर्मावेकस्य सामान्यस्य द्वाभ्यामेताभ्यां सिद्ध्येते इति । तत्किमेतदुच्यत इत्याह--\textbf{एकेनेति । चो} यस्मादर्थे । \textbf{एकेन}ति प्रागुक्तेन स्वभावेन । \textbf{अपरेणे}ति पश्चादुक्तेन । तमेवाह \textbf{अनुपलम्भेन} प्रकरणाद् दृश्यानुपलम्भेनेति नेयम् । साधयतां तर्हि व्यक्त्यन्तराले सामान्यस्य सत्त्वमसत्त्वं च एतौ हेतू का क्षतिरिह--इत्याह \textbf{न चेति । चो}ऽवधारणे । सत्त्वमसत्त्वं च द्वयोर्द्वाभ्यां साधने किमनुपपन्नमित्याशङ्क्याह--\textbf{एकस्यापि} । कालभेदे किन्नैवं सम्भवतीत्याशङ्क्याह--\textbf{एकदैवेति} । कस्याप्येकदैवाधिकरणभेदेऽप्येतदित्याशङ्क्याह—\textbf{एकत्रेति ।} कथमयुक्तमित्याह--\textbf{तयोरि}ति । \textbf{तयोः} सत्त्वासत्त्वयोः । \textbf{विरोधा}त्परस्परपरिहारव्यवस्थितरूपत्वात् ।
	\pend
      

	  \pstart कथं पुनरागमाश्रयानुमानाश्रयत्वं विरुद्धाव्यभिचारि\add{णी} त्याशङक्योपसंहारव्याजेन यथाऽनयोस्तथात्वं तथा दर्शयन्नाह--\textbf{तदि}ति । यत एवं तत्तद्वदेकं सामान्यं \textbf{कणादेन} उक्तम्, तद्रूपविचारे तच्छिष्याभ्यामेवं प्रक्रान्तं \textbf{तत्} तस्मादा\textbf{गमसि}द्धस्यागमप्रतिपादितस्यानुपलम्भेनापि व्यक्तत्यन्तरालासत्त्वप्रतिपादनद्वारम्\footnote{द्वाराऽ}सर्वगतत्वस्य साधनात् । सर्वगतत्वासर्वगतत्वयोः \textbf{साध्ययोरेता}वित्याह । एवावन्तं \footnote{?} तथोक्तौ हेतू ।
	\pend
      

	  \pstart विरुद्धाव्यभिचारित्वमेवानयोरुपदर्शयन्नाह--\textbf{यत} इति । \textbf{च}कारः पूर्वापेक्षया समुच्चये ।  \leavevmode\marginnote{\textenglish{233/dm}} “
	  
	चैकस्यैकदैकत्र सत्त्वमसत्त्वं च\footnote{वा \cite{dp-msD}} युक्तम्, तयोर्विरोधात् । \footnote{तस्मादागम० \cite{dp-msB}}तदागमसिद्धस्य सामान्यस्य सर्वगतत्वासर्वगतत्वयोः साध्ययोरेतौ विरुद्धाव्यभिचारिणौ जातौ । \footnote{अथ परस्परविरुद्धाव्यभिचारिहेतुदोषः कणादशिष्ययोरेवायम्, न कणादस्येत्याह--\cite{dp-msD-n}}यतः सामान्यस्यैकस्य युगपत् सर्वदेशावस्थितैरभिसम्बन्धित्वं\footnote{०भिसम्बद्धत्वम् \cite{dp-msB} भिसम्बन्धत्वम् \cite{dp-msD}} चाभ्युपगतम्, दृश्यत्वं च । ततः सर्वसम्बन्धित्वात् सर्वगतत्वम्, \footnote{दृश्यत्वादन्तरालानुपल० \cite{dp-msA} \cite{dp-msB} \cite{dp-msC} \cite{dp-msD} \cite{dp-edP} \cite{dp-edH} \cite{dp-edE} \cite{dp-edN}}दृश्यत्वादन्तरालादनुपलम्भादसर्वगतत्वम् । ततः \footnote{कणादद्वारेण--\cite{dp-msD-n}}शास्त्रकारेणैव विरुद्धव्याप्तत्वमपश्यता विरुद्धव्याप्तौ धर्मावुक्त्वा\footnote{धर्मावुक्तौ । इह विरुद्धा० \cite{dp-msB} धर्मावुक्तौ विरु० \cite{dp-msC} \cite{dp-msD}} विरुद्धाव्यभिचार्यवकाशो दत्त इति । न च\footnote{“च” नास्ति \cite{dp-msB} \cite{dp-msD}} वस्तुन्यस्य \footnote{०स्य हेतोः सम्भवः \cite{dp-msB} \cite{dp-msD} विरुद्धाव्यभिचारिणः--\cite{dp-msD-n}}सम्भवः । इत्युक्ता हेत्वाभासाः ॥ 
	  
	ननु च साधनावयवत्वाद् यथा हेतव उक्तास्तत्प्रसङ्गेन \footnote{“च” नास्ति--\cite{dp-msD}}च हेत्वाभासाः, तथा साधनावयवत्वाद् दृष्टान्ता वक्तव्यास्तत्प्रसङ्गेन च दृष्टान्ताभासाः । तत् कथं नोक्ता इत्याह--” ततोऽभ्युपगतात्सर्वसम्बन्धिनः सर्वसम्बन्धित्वात् । सर्वगतत्वमन्तरेण तदसम्भवादिति भावः । अनुपलम्भेऽपि कथमसर्वगतत्वं निश्चेतुं शक्यत इत्याह--दृश्यत्वात्तत इत्यनुवर्त्तते ।
	\pend
      

	  \pstart ननु चान्तरालेऽनुपलम्भादिति युज्यते वक्तुम् । तत्किमुक्त\textbf{मन्तरालादि}ति । सत्यमेतत् । केवल\textbf{मनुपलम्भादनु}पलम्भव्यवहारादित्यर्थविवक्षितत्वाददोषः । क्वचित्पु\textbf{नरन्तरालानुपलम्भादि}ति पाठः । तत्र च यथायोगं समासः ।
	\pend
      

	  \pstart ननु यदि शास्त्रात्मकाऽऽगमकारेणास्य परस्परविरुद्धस्वभावावहं किञ्चिद् द्वयमुक्तं भवेत्, विरुद्धाव्यभिचार्यवकाशः । यावतेदमेव नास्तीत्याशङ्कापनोदव्याजेनोपसंहरन्नाह--\textbf{तत इति} । \leavevmode\marginnote{\textenglish{77b/ms}} यत एवं शास्त्रकारेणैवेदञ्चेदञ्चाभ्युपगतं \textbf{तत}स्तस्मात् \textbf{शास्त्रकारेणैव} प्रकरणा\textbf{त्कणादेन विरुद्धाव्यभिचारिणोऽवकाशो दत्तः} । किं कृत्वा तेन तदवकाशो दत्त इत्याह--\textbf{विरुद्धेति} । परस्परसाध्यविरुद्धत्वव्याप्तिधर्मौ प्रकरणाद् युगपत्सर्वसम्बन्धिसम्बन्धदृश्यत्वाख्या \footnote{ख्यौ} वक्तुर\footnote{?}दृश्यत्वविषयस्यानुपलम्भस्य विरुद्धव्याप्तत्वाद् दृश्यत्वं विरुद्धव्याप्तमिति तु द्रष्टव्यम् ।
	\pend
      

	  \pstart कथं पुनस्तयोर्धर्मयोः परस्परविरुद्धार्थव्याप्तत्वं विदुषा तेनैवमुच्येत येन तदवकाशदानं तस्य कल्प्यत इत्याह--\textbf{विरुद्धव्याप्तत्वमि}ति । परस्परविरुद्धसर्वगतत्वासर्वगतत्वलक्षणार्थव्याप्तत्वं तयोर्धर्मयोरित्यर्थात् । \textbf{अपश्यता} अनालोचयता भ्रान्त्या तावदभिधायेति यावत् । इतिस्तस्मात् \textbf{न वस्तुनि} न वस्तुवलप्रवृत्तेऽनुमानेऽस्य विरुद्धाव्यभिचारिणो जातिविवक्षयैकवचनम् । \textbf{इति}रेवमर्थे । तेनायमर्थः--एवमुक्तेन प्रकारेणो\textbf{क्ताः} कथिता \textbf{हेत्वाभासाः,} कुतश्चित्साम्याद् हेतुवदाभासाः ॥
	\pend
      

	  \pstart साम्प्रतं हेतौ तदाभासे च कथिते तुल्यन्यायतयाऽपरमपि स्वाभाससहितं किं नोक्तमिति  \leavevmode\marginnote{\textenglish{234/dm}} “
	  
	त्रिरूपो\footnote{त्रिलक्षणो हेतु० \cite{dp-msC}} हेतुरुक्तः । तावता \footnote{तावतैवार्थ० \cite{dp-msB} \cite{dp-msD} \cite{dp-edP} \cite{dp-edH} \cite{dp-edE} \cite{dp-edN} तावता वा [[चा]]र्थप्रतीतिसिद्धेरिति \cite{dp-msC}}चार्थप्रतीतिरिति न पृथग्दृष्टान्तो नाम साधनावयवः कश्चित् । तेन नास्य लक्षणं पृथगुच्यते गतार्थत्वात् ॥ १२१ ॥” “
	  
	त्रिरूपो हेतुरुक्तः, तत् किं दृष्टान्तैः ? 
	  
	स्यादेतत्--तावता नार्थप्रतीतिरित्याह--तावता \footnote{तावता वे[[चे]]ति \cite{dp-msC} तावतैवेति \cite{dp-msB} \cite{dp-msD} \cite{dp-edP} \cite{dp-edH} \cite{dp-edE} \cite{dp-edN}}चेति । उक्तलक्षणेनैव हेतुना भवति साध्यप्रतीतिः । \footnote{ततः \cite{dp-msB}}अतः स एव गमकः । \footnote{“ततः” नास्ति \cite{dp-msA} \cite{dp-edP} \cite{dp-edH}}ततस्तद्वचनमेव साधनम् । न दृष्टान्तो नाम नाधनस्यावयवः । यतश्चायं नावयवः, तेन नास्य दृष्टान्तस्य लक्षणं हेतुलक्षणात् पृथगुच्यते । 
	  
	कथं तर्हि हेतोर्व्याप्तिनिश्चयो यद्यदृष्टान्तको हेतुरिति चेत् । नोच्यते हेतुरदृषटान्तक एव । अपि तु न हेतोः पृथग्दृष्टान्तो नाम हेत्वन्तर्भूत एव दृष्टान्तः । अत एवोक्तं नास्य लक्षणं पृथगुच्यत \footnote{“इति” नास्ति \cite{dp-msB} \cite{dp-msD}}इति । न त्वेवमुक्तम्--नास्य लक्षणमुच्यत इति ।\footnote{“इति” नास्ति \cite{dp-msB}}” चोदयन्नाह--\textbf{ननु चे}ति । अथ कथं साधनावयवत्वं हेतूनां येन \textbf{साधनावयत्वाद् यथा हेतव उक्ता} इत्युच्यते ? तथा हि साध्यते निश्चीयते साध्यमनेनेति \textbf{साधनम्} । पक्षधर्मान्वयव्यतिरेकवल्लिङ्गमुच्यते । तदाख्यानादेव च वाक्यमपि \textbf{साधनमु}च्यते । न तु वाक्यात्साध्यसिद्ध्युपयोगितया हेतुरपि तदेव तद्वचनमपीति । द्वेधाऽपि तं समुदायमपेक्ष्यास्यावयवत्वं येनैवमुच्यत इति । न । अभिप्रायापरिज्ञानात् । इहैवं पूर्वपक्षवादी मन्यते । परोक्षोऽर्थो निश्चीयमानो हेतुदृष्टान्ताभ्यां निश्चीयते । न तु हेतुनैव । अदृष्टान्तकस्य हेतोः साध्यसाधनाशक्तेः । ततश्च हेतुदृष्टान्तसमुदायः साध्यस्य साधनम् । तत्र यथा समुदायापेक्षया हेतुरूपोऽवयव उक्तस्तदाभासश्च तथा साधनावयवत्वाविशेषत्वाद् दृष्टान्तरूपोऽपरोऽयमवयवः सप्रतिपक्षः किन्नोक्त इति । एवञ्चोदयन्नयमेव साधनावयवत्वाद्यभिधानमिति । \textbf{तत्प्रसङ्गेन} तत्प्रस्तावेन ।
	\pend
      

	  \pstart \textbf{उक्तं} पक्षधर्मान्वयव्यतिरेकात्मकं \textbf{लक्षणं} यस्य तेन । \textbf{एव}कारेण दृष्टान्तस्य साक्षात् सिद्ध्युपयोगितां निरस्यति । यत एव\textbf{मतो} हेतोः, \textbf{स} इति हेतुः । नियमेन दृष्टान्तस्य गमकरूपविरहं द्रढयति । यतः साक्षात् सिद्धिरुपजायते, \textbf{स एव} गमयति प्रत्याययति साध्यमिति कृत्वा उच्यते, नान्य इति भावः । यस्माद् हेतोरेव गमकत्वं \textbf{तत}स्तस्मात् \textbf{तस्य} हेतो स्त्रिरूपस्य \textbf{वचनं साधनं} साधनाभिधानं न दृष्टान्तस्येत्यर्थात् । \textbf{तावता चार्थप्रतीति}रिति \textbf{मौलमितिशब्दमपेक्ष्य न पृथग्दृष्टान्तो नाम साधनावयव} इति मूलं व्याचष्टे \textbf{न दृष्टान्त} इति । अर्थरूपो \textbf{दृष्टान्त}स्तद्वचनं वा, \textbf{न साधनस्या}र्थरूपस्य वचनस्य वा । \textbf{नावयवो} नैकदेशः । हेत्वन्तरर्भूतत्वं च व्याप्तिग्राहकप्रमाणाधिकरणतया हेतावुपयोगात् । न तस्य स्वरू\leavevmode\marginnote{\textenglish{78a/ms}}पेण  \leavevmode\marginnote{\textenglish{235/dm}} “
	  
	यद्येवं हेतूपयोगिनोऽपि लक्षणं वक्तव्यमेवेत्याह--गतार्थत्वात्--गतोऽर्थः प्रयोजनमभिधेयं वा यस्य दृष्टान्तलक्षणस्य । \footnote{“तत्” नास्ति \cite{dp-edE} \cite{dp-edN}}तत् तथा । तस्य भावस्तत्त्वम् । तस्मात् । दृष्टान्तलक्षणं ह्यच्यते दृष्टान्तप्रतीतिर्यथा स्यात् । दृष्टान्तश्च हेतुलक्षणादेवावसितः । ततो दृष्टान्तलक्षणस्य यत् प्रयोजनम्--दृष्टान्तप्रतीतिस्तद् गतं निष्पन्नम् । अभिधेयं वा गतं ज्ञातं\footnote{ज्ञानं \cite{dp-msB} \cite{dp-edP} \cite{dp-edH} \cite{dp-edE} गतं दृष्टा० \cite{dp-msA}} दृष्टान्ताख्यम् ॥ 
	  
	कथं \footnote{गतार्थमि० \cite{dp-edE}}गतार्थत्वमित्याह-- “
	  
	हेतोः सपक्ष एव सत्त्वमसपक्षाच्च सर्वतो व्यावत्ती \footnote{व्यावृत्तरूप० \cite{dp-msB} \cite{dp-edP} व्यावृत्तो रूप० \cite{dp-edH}}रूपमुक्तमभेदेन । पुनर्विशेषेण \footnote{०स्वभावयोर्जन्म० \cite{dp-msB} \cite{dp-msD} \cite{dp-edP} \cite{dp-edH} \cite{dp-edE} \cite{dp-edN}}कार्यस्वभावयोरुक्तलक्षणयोर्जन्मतन्मात्रानुबन्धौ दर्शनीयावक्तौ । तच्च दर्शयता-यत्र धूमस्तत्राग्निः, असत्यग्नौ न क्वचिद् धूमो यथा महानसेतरयोः; यत्र कृतकत्वं तत्रानित्यत्वम्, अनित्यत्वाभावो कृतकत्वाऽसम्भवो\footnote{सम्भवोऽस्ति \cite{dp-msC}} यथा घटाकाशयोः--इति दर्शनीयम् । न हयन्यथा सपक्षविपक्षयोः सदसत्वे यथोक्तप्रकारे शक्ये दर्शयितुम् । तत्कार्यतानियमः\footnote{नियमं \cite{dp-msC}} कार्यलिङ्गस्य, स्वभावलिङ्गस्य \footnote{वा \cite{dp-msC}}च स्वभावेन\footnote{स्वभावव्याप्तिरिति \cite{dp-msC}} व्याप्तिः । \footnote{अस्मिन्नर्थे \cite{dp-msC}}अस्मिंश्चार्थे दर्शिते दर्शित एव दृष्टान्तो भवति । एतावन्मात्ररूपत्वात् तस्येति\footnote{“इति” नास्ति \cite{dp-msC}} ॥ १२२ ॥” 
	  
	हेतो \footnote{रूपमभेदेनोक्तं सा० \cite{dp-msA} \cite{dp-msD} \cite{dp-edP} \cite{dp-edH} \cite{dp-edE}}रूपमुक्तमभेदेन सामान्येन । साधारणं कार्यस्वभावानुपलम्भानामेतल्लक्षणमित्यर्थः । किं पुनस्तत् ? सपक्ष एव \footnote{“यत्” नास्ति \cite{dp-msB} \cite{dp-msC} \cite{dp-msD}}यत् सत्त्वम्, विपक्षाच्च सर्वस्मात् व्यावृत्तिर्या । रूपद्वयमेतदभेदेनोक्तम् । 
	  
	न च सामान्यमुक्तमपि शक्यं ज्ञातुम् । अतस्तदेव विशेषनिष्ठं वक्तव्यम् । अतः पुनरपि विशेषेण विशेषवन्तौ जन्मतन्मात्रानुबन्धौ दर्शनीयावुक्तौ । कार्यस्य जन्म ज्ञातव्यमुक्तम् । जन्मनि हि \footnote{विज्ञाते \cite{dp-msA} \cite{dp-edP} \cite{dp-edH} \cite{dp-edE}}ज्ञाते कार्यस्य सपक्ष एव सत्त्वं विपक्षाच्च सर्वस्माद् व्यावृत्तिर्ज्ञाता” हेतावन्तर्भावः, साक्षात्साध्यसिद्ध्युपयोगिताप्रसङ्गात् । न च साऽस्य सम्भविनीति । अनेकार्थत्वाद् वा हेतो \footnote{धातो} \textbf{र्गतं निष्पन्नमि}ति विवृतम् । गत्यर्थानां ज्ञानार्थत्वादपि \textbf{गतं ज्ञात}मित्यपि चोक्तम् ॥
	\pend
      

	  \pstart ननु च हेतुः सपक्षवृत्तिना विपक्ष\footnote{क्षा}वृत्तेनैव च रूपेण गमकः । तच्चेत् कथितं किं विशेषलक्षणकथनेनेत्याह--\textbf{न चे}ति । \textbf{चोऽ}वधारणे हेतौ वा । \textbf{सामान्यं} साधारणं प्रकरणा  \leavevmode\marginnote{\textenglish{236/dm}} “
	  
	भवति । स्वभावस्य तन्मात्रानुबन्धो दर्शनीय उक्तः । तदिति साधनम् । तदेव तन्मात्रम् साधनमात्रम्\footnote{मात्रस्यानु० \cite{dp-msC} \cite{dp-msD} साधनमात्रम् । साधनमात्रस्यानुब० \cite{dp-msB} \cite{dp-edN}} । तस्यानुबन्धोऽनुगमनम्--साधनमात्रस्य\footnote{मात्रभावे \cite{dp-msA} \cite{dp-msB} \cite{dp-msD} \cite{dp-edP} \cite{dp-edH} \cite{dp-edE} \cite{dp-edN}} भावे भावः साध्यस्य । \footnote{अथ साधनधर्ममात्रानुबन्धः साध्यस्य तथापि तादात्म्यं न स्यादित्याह--\cite{dp-msD-n}}तन्मात्रभावित्वमेव हि साध्यस्य तादात्म्यम् । साधनस्य\footnote{यदा साधनस्य च स्व० \cite{dp-msC}} यदा स्वभावो ज्ञातो भवति, तदा स्वभावहेतोः--सपक्ष एव सत्त्वम्, विपक्षाच्च सर्वस्माद् व्यावृत्तिर्ज्ञाता भवति । 
	  
	तदेवं\footnote{तदेव \cite{dp-edE}} सामान्यलक्षणं विशेषात्मकं ज्ञातव्यं नान्यथा । ततो विशेषलक्षणमुक्तम् । 
	  
	किमतो\footnote{विशेषलक्षणात्--\cite{dp-msD-n}} यदि नामैव\footnote{नामैवं तदित्याह \cite{dp-edE} प्रागुक्तन्यायेन--\cite{dp-msD-n}} मित्याह--तच्च\footnote{तत्र \cite{dp-msA} \cite{dp-msB} \cite{dp-edP} \cite{dp-edH} \cite{dp-edE}} सामान्यलक्षणं दर्शयितुकामेन विशेषलक्षणं दर्शयतैवं\footnote{०तैवं च दर्श० \cite{dp-msB}} दर्शनीयम्--इति सम्बन्धः । यत्र धूमस्तत्राग्निरिति कार्यहेतोव्याप्तिर्दशिता । व्याप्तिश्च कार्यकारणभावसाधनात्\footnote{प्रत्यक्षानुपलम्भाभ्याम्--\cite{dp-msD-n}} प्रमाणान्निश्चीयते । ततो यथा महानस इति दर्शनीयम् ।” ल्लक्षणं \textbf{न शक्यं ज्ञातुमि}ति प्रवृत्त्युपयोगितया न शक्यमिति मन्तव्यम् । न तु सामान्यलक्षणस्य वाक्यात्प्रतीतिर्न भवत्येव । \textbf{विशेषवन्तौ} च विशिष्टावित्यर्थः । एतदेव विभज्यमान आचार्ये\footnote{आह--\textbf{कार्यस्ये}} त्यादि ।
	\pend
      

	  \pstart ननूक्तेऽप्यस्मिन् यदि सामान्यलक्षणप्रतीतिर्नास्ति किमनेनोक्तेनापीति । आह—\textbf{जन्मे}ति । \textbf{हि}र्यस्मादर्थे । ननु स्वभावहेतौ साधनस्वभावता साध्यस्य दर्शयितुं युज्यते, तत्किं तन्मात्रानुबन्धो दर्शनीय उक्त इत्याह--\textbf{तन्मात्रे}ति । \textbf{हि}र्यस्मात् । अथ तन्मात्रानुवन्धे दर्शितेऽपि कथं सामान्यलक्षणप्रतिपत्तिरित्याह--\textbf{साधनस्ये}ति । \textbf{तदेवमि}त्यादिनोपसंहारः ।
	\pend
      

	  \pstart अनुपलब्धेश्चानयोरेवान्तर्भावान्न पृथग् विशिष्टलक्षणाभिधानमित्यवसेयम् ।
	\pend
      

	  \pstart अथ किं हेतोरन्वयव्यतिरेकावेव लक्षणं येनैतद् द्वयमेव लक्षणतया स्मर्यते । न चैतत्, पक्षधर्मताया अपि लक्षणत्वादभिहितत्वाच्चेति चेत् । सत्यम् । केवलं न हेतोः सर्वरूपमिह प्रक्रान्तम् । किन्तु यद्रूपप्रदर्शनेन दृष्टान्तप्रदर्शनं कृतं भवति तत्प्रकृतमिति ।
	\pend
      

	  \pstart यद्यप्ये\textbf{वं यदि नाम} सामान्यविशेषलक्षणाऽभिधानमतोऽभिधानात् \textbf{किं} भवति ? किमः क्षेपे प्रयोगात्, न किञ्चिद् भवतीत्यर्थः । \textbf{य \footnote{त} दित्युत्तर}म् । \textbf{चो} यस्मात् । व्याप्तिप्रदर्शनेऽपि कथं दृष्टान्ताख्यानमवतरतीत्याह--\textbf{व्याप्तिश्चेति} । हेत्वर्थश्चकारः । कार्यकारणभावसाधनात्प्रमाणान्निश्चीयतां व्याप्तिस्तथापि दृष्टान्तः कथं प्रकृतो भवतींत्याह--\textbf{तत} इति । दृष्टान्तमन्तरेण तदेव प्रमाणं न प्रवर्त्तेत । तदप्रवृत्तौ च सामान्यलक्षणमेव न ज्ञायेतेति भावः ।  \leavevmode\marginnote{\textenglish{237/dm}} “
	  
	असत्यग्नौ न भवत्येव धूम इति व्यतिरेको दर्शितः । स च यथेतरस्मिन्निति दर्शनीयः । वह्निनिवृत्तिर्हि धूमनिवृत्तौ नियता दर्शनीया । सा च महानसादितरत्रेति दर्शनीया । 
	  
	यत्र कृतकत्वं तत्रानित्यत्वमिति स्वभावहेतोर्व्याप्तिर्दर्शिता । अनित्यत्वाभावे न भवत्येव कृतकत्वमिति व्यतिरेको दर्शितः । व्याप्तेश्च साधकं प्रमाणं \footnote{साधर्म्यम् दृ० \cite{dp-msB}}साधर्म्यदृष्टान्ते दर्शनीयम् । प्रसिद्धव्याप्तिकस्य च हेतोः साध्यनिवृत्तौ निवृत्तिर्नियता\footnote{निवृत्तिर्दर्श० \cite{dp-msA} \cite{dp-msB} \cite{dp-msD} \cite{dp-edP} \cite{dp-edH} \cite{dp-edN} \cite{dp-edE}} दर्शनीया । तदवश्यं यथा घटे, यथा आकाशे चेति दर्शनीयम् । 
	  
	कस्मादेवमित्याह--न हीति । यस्मादन्यथा सामान्यलक्षणरूपे सपक्षविपक्षयोः सदसत्त्वे यथोक्तप्रकारे इति नियते--सपक्ष एव सत्त्वम्, विपक्षेऽसत्त्वमेवेति नियमो यथोक्तप्रकारः--ते न शक्ये दर्शयितुम् । विशेषलक्षणे हि दर्शिते यथोक्तप्रकारे\footnote{यथोक्तलक्षणे सद० \cite{dp-msC}} सदसत्त्वे दर्शिते भवतः । न च विशेषलक्षणमन्यथा शक्यं दर्शयितुम् । 
	  
	तस्य साध्यस्य कार्यम्--तत्कार्यं धूमः । तस्य भावस्तत्कार्यता । सैव नियमो” अन्वयप्रदर्शनेन साधर्म्यदृष्टान्ताख्यानमाख्याय व्यतिरेकोक्त्याऽपि वैधर्म्यदृष्टान्तौ \textbf{किं} \footnote{दृष्टान्तोक्तिं} दर्शयितुमाह--\textbf{असत्यग्नाविति} । व्यतिरेकप्रदर्शनेऽपि कथं दृष्टान्ताऽऽपतनमिति आशङ्क्याह--\textbf{स चेति । चो} यस्मादर्थे । \textbf{इतरस्मिन्न}ग्निमत्प्रदेशादन्यस्मिन् महाह्रदादौ । यत्र कयोश्चिद् व्याप्यव्यापकभावो दर्शयितव्यस्तत्रास्तु दृष्टान्तोपनिपातः, व्यतिरेकोपदर्शने किं तेनेत्याह--\textbf{वह्नीति । ही}ति यस्मात् । अत्रापि विध्योर्विपर्ययेण व्याप्यव्यापकभावो दर्शयितव्य इत्यर्थः । एतच्च व्यतिरेकाख्यानं दृष्टान्तोपदर्शनमुद्दिष्टविषयं प्रयोगमधिकृत्योक्तमित्यधिगन्तव्यम् ।
	\pend
      

	  \pstart \textbf{यत्रे}त्यादिना स्वभावहेतुमधिकृत्य दृष्टान्तस्य गतार्थत्वं दर्शयितुमुपक्रमते । एवं प्रदर्शयतां व्याप्तिः, दृष्टान्तावतारकस्तु\footnote{रस्तु}कथमित्याह--\textbf{व्याप्तेश्चे}ति । \textbf{चो} यस्मात् । प्रमाणाऽप्रदर्शने व्याप्तेरसिद्धेः । तच्च प्रमाणम्, अन्तरेणाधिकरणं न सम्भवतीति भावः । यद्येवं व्यतिरेक \leavevmode\marginnote{\textenglish{78b/ms}} प्रदर्शने कृतं दृष्टान्तेनेत्याह--\textbf{प्रसिद्धेति} । दर्श्यतामेव, तथापि दृष्टान्तोपनिपातः कथमित्याह--\textbf{तदि}ति । यस्मादेव दर्शनीयं तत्तस्मात् । सच्चासच्च सदसती तयोर्भावस्ते प्रदर्शनक्रियापेक्षया च द्वितीयाद्विवचनान्तमेतत् । यादृश उक्तो \textbf{यथोक्तः प्रकारः} स्वरूपं ययोस्ते तथोक्ते । अस्यैवार्थमाह--\textbf{नियते} इति । सपक्षासपक्षसदसत्त्वयोरुक्तमेव प्रकारं स्पष्टयन्नाह--\textbf{सपक्ष} इति । \textbf{इति}र्नियमस्याकारं दर्शयति । \textbf{ते} यथोक्तप्रकारे सदसत्त्वे । कथं ते न शक्ये दर्शंयितुमित्याह--\textbf{विशेषे}ति । \textbf{हि}र्यस्मादर्थे । दर्श्यतां तर्हि विशेषलक्षणम् । दृष्टान्तस्य तु किमायातमित्याह--\textbf{न चे}ति । \textbf{चो} यस्मादर्थेऽवधारणे वा । यद्विशेषलक्षणं दृष्टान्तमन्तरेणाशक्यप्रदर्शनं तत्स्वरूपाख्यानम् । मूलं व्याख्यातुमाह--\textbf{तस्ये}त्यादि । विशेष  \leavevmode\marginnote{\textenglish{238/dm}} “
	  
	यतस्तत्कार्यतया धूमो दहने नियतः । सोऽयं तत्कार्यतानियमो विशेषलक्षणरूपोऽन्यथ दर्शयितुमशक्यः । स्वभावलिङ्गस्य च स्वभावेन साध्येन व्याप्तिर्विशेषलक्षणरूपा न शक्या दर्शयितुम् । यस्मात् कार्यकारणभावस्तादात्म्यं च महानसे घटे च ज्ञातव्यम्, तस्माद् व्याप्तिसाधनं प्रमाणं दर्शयता \footnote{साध्यदृष्टा० \cite{dp-msB}}साधर्म्यदृष्टान्तो दर्शनीयः । वैधर्म्यदृष्टान्तस्तु प्रसिद्धे तत्कार्यत्वे कारणाभावे कार्याभावप्रतिपत्त्यर्थम्\footnote{०त्त्यर्थः \cite{dp-edE}} । तत एव नावश्यं वस्तु भवति । कारणाभावे कार्याभावो वस्तुन्यवस्तुनि वा भवति । ततो वस्त्ववस्तु वा वैधर्म्यदृष्टान्त इष्यते । 
	  
	तस्माद् दृष्टान्तमन्तरेण\footnote{दृष्टान्तव्यतिरेकेण हे० \cite{dp-msA} \cite{dp-edP} \cite{dp-edH} \cite{dp-edE} \cite{dp-edN}} न हेतोरन्वयो व्यतिरेको वा\footnote{वा न शक्यो \cite{dp-msA} \cite{dp-edP} \cite{dp-edH} \cite{dp-edE} \cite{dp-edN}} शक्यो दर्शयितुम् अतो हेतुरूपाख्यानादेव हेतोर्व्याप्तिसाधनस्य\footnote{०साधकस्य \cite{dp-edE}} प्रमाणस्य दर्शकः साधर्म्यदृष्टान्तः । 
	  
	प्रसिद्धव्याप्तिकस्य साध्याभावे हेत्वभाव\footnote{भावदर्श० \cite{dp-msC} \cite{dp-msD}} प्रदर्शनाद्वैधर्म्यदृष्टान्त उपादेय इति \footnote{“च” एवार्थे--\cite{dp-msD-n}}च दर्शितं भवति ।” लक्षणस्य स्वरूपमाख्यायाख्यान्यथाप्रदर्शनाशक्यत्वं दर्शयन्नाह--\textbf{सोऽयमि}ति ।
	\pend
      

	  \pstart वैधर्म्यदृष्टान्तस्तर्हि न प्रदर्शनीय इत्याह--\textbf{वैधर्म्येति । तुः} साधर्म्यदृष्टान्ताद् वैधर्म्यदृष्टान्तं भेदवन्तं दर्शयति । एतच्च यदा व्यतिरेकमुखेण प्रयोगः क्रियते, तत्कालाभिप्रायेणोच्यत इति द्रष्टव्यम् । न त्वेकस्मिन् प्रयोगे द्वयोपन्यासः सम्भवी । एतच्च प्रागेव निर्लोठितम् । यतः \textbf{प्रसिद्धे तत्कार्यत्वे कारणाभावे कार्याभावप्रतिपत्त्यथं} दर्शनीयो वैधर्म्यदृष्टान्तस्तत एव तस्मादेव कारणात् । \textbf{नावश्यं नियमे \add{न} वस्तु भवति} । वस्त्वप्यवस्त्वपि वैधर्म्यदृष्टान्तो भवतीत्यर्थः । यदि हि कस्यचिद् विधिना कस्यचिद् विधिर्दर्शयितव्यः स्यात्, तदा गम \footnote{गगना} दिरसन् कथं कस्यचिदाधारः स्यादिति, न सिद्ध्येत्तदुपदर्शनम् । यदा त्वेकस्य व्याप्यस्याभावे अपरस्य व्यापकस्याभावो दर्शयितव्यस्तदा व्यापकस्याऽभावेऽवस्तुनि सुष्ठु सम्भवति । इतरथाऽभावेऽपि अस्य भावो विहितो भवेत् । सोऽपि नेति चेत्, अयमेवाभाव इत्यभिप्रायः । वैधर्म्यदृष्टान्तग्रहणं चैतदेकधर्माभावेनापरधर्माभावोपदर्शनविषयोपलक्षणं द्रष्टव्यमन्यथा साधर्म्यदृष्टान्ते त्ववश्यं वस्तु स आश्रय इष्ट इति स्यात् । तथा च सति बह्वसमञ्जसं स्यादिति । यतः कारणाभावात्कार्याभावो वस्तुन्यवस्तुनि च भवति \textbf{ततः} कारणाद् \textbf{वस्तु} महाह्रदादि । \textbf{अवस्तु} आकाशादि । वाशब्दस्तुल्यबलत्वं समुच्चिनोति ।
	\pend
      

	  \pstart \textbf{तस्मादि}त्यादिना प्रकृतमुपसंहरति । व्याप्तिसाधकप्रमाणाधिकरणत्वाद् दर्शयतीति \textbf{दर्शकः} । यद्येवं साध्यसाधकव्याप्तिप्रसाधकप्रमाणप्रदर्शकत्वात्साधर्म्यदृष्टान्त एवोपादेयो न वैधर्म्यदृष्टान्त इत्याह--\textbf{प्रसिद्धे}तिं । \textbf{वैधर्म्यदृष्टान्त उपादेय} इति योजयित्वा कुत उपादेय इत्याशङ्कायां \textbf{प्रसिद्धेत्या}दिहेतुपदं योज्यम् । \textbf{साध्याभावे हेत्वभावप्रदर्शनात्} । तत्रेति बुद्धिस्थम् । यद्वा प्रदर्श्यतेऽस्मिन्निति प्रदर्शयतीति वा \textbf{प्रदर्शनो हेत्वभावस्य प्रदर्शन} इति तथा । भावप्रधानत्वान्निर्देशस्य हत्वभावप्रदर्शनत्वादित्यर्थः ।
	\pend
      \leavevmode\marginnote{\textenglish{239/dm}}“

	  \pstart अस्मिंश्चार्थे दर्शिते दर्शित एव दृष्टान्तो भवति । योऽयमर्थो व्याप्तिसाधनप्रमाणप्रदर्शनः\footnote{प्रमाणदर्शिनः \cite{dp-msA} \cite{dp-msB} \cite{dp-edP} \cite{dp-edH} \cite{dp-edN}} कश्चिदुपादेयो निवृत्तिप्रदर्शन\footnote{निवृत्तिप्रदर्शकश्च \cite{dp-msB}} श्च--इत्यस्मिन्नर्थे \footnote{प्रदर्शिते \cite{dp-msA} \cite{dp-edP} \cite{dp-edH} \cite{dp-edE} \cite{dp-edN}}दर्शिते दर्शितो दृष्टान्त \footnote{दृष्टान्तः । कस्मादित्याह \cite{dp-edE}}इत्याह--एतावन्मात्रं रूपं यस्य तस्य भावस्तत्त्वम्, तस्मादिति । एतावदेव हि रूपं दृष्टान्तस्य, यदुत व्याप्तिसाधनप्रमाण\footnote{०प्रमाणदर्शनत्वं \cite{dp-msA} \cite{dp-msB} \cite{dp-msD} \cite{dp-edP} \cite{dp-edH} \cite{dp-edE}} प्रदर्शकत्वं नाम साधर्म्यदृष्टान्तस्य, प्रसिद्धव्याप्तिकस्य च\footnote{वा \cite{dp-msA} \cite{dp-msB} \cite{dp-msD} \cite{dp-edP} \cite{dp-edH} \cite{dp-edE} \cite{dp-edN}} साध्यनिवृत्तौ साधननिवृत्तिप्रदर्शकत्वमित्येत\footnote{“वैधर्म्यदृष्टान्तस्य एतत्” नास्ति \cite{dp-msB} “वैधर्म्यदृष्टान्तस्य” नास्ति. \cite{dp-msD}} द्वैधर्म्यदृष्टान्तस्य । \footnote{तत् \cite{dp-msA} \cite{dp-msD} \cite{dp-edP} \cite{dp-edH} \cite{dp-edE} \cite{dp-edN} प्रागुक्तन्यायेन--\cite{dp-msD-n}}एतच्च हेतुरूपाख्यानादेवाख्यातमिति किं दृष्टान्तलक्षणेन ? ॥
	\pend
       “

	  \pstart एतेनैव दृष्टान्तदोषा अपि निरस्ता भवन्ति ॥ १२३ ॥
	\pend
      ” 

	  \pstart एतेनैव च हेतुरूपाख्यानाद् दृष्टान्तत्वप्रदर्शनेन दृष्टान्तदोषा\footnote{दृष्टान्तस्य दोषाः \cite{dp-msA} \cite{dp-edP} \cite{dp-edH} \cite{dp-edE} \cite{dp-edN}} दृष्टान्ताभासाः\footnote{निरसनद्वारा कथिताः--\cite{dp-msD-n}} कथिता भवन्ति । तथाहि--पूर्वोक्तसिद्धये य उपादीयमानोऽपि\footnote{०पि न समर्थः \cite{dp-msB}} दृष्टान्तो न समर्थः स्वकार्यं साधयितुं स दृष्टान्तदोष इति सामर्थ्यादुक्तं\footnote{०र्थ्यादित्येतदुक्तं \cite{dp-msB}} भवति ॥
	\pend
      ”

	  \pstart \textbf{अस्मिंश्चेत्या}दि मूलमनूद्य व्याचष्टे \textbf{योऽयमि}ति । पूर्ववत्प्रदर्शनशब्दस्य व्युत्पत्तिः, समासश्च कर्त्तव्यः । प्रसिद्धायां व्याप्तौ साध्या\leavevmode\marginnote{\textenglish{79a/ms}}भावे हेतोर्निवृत्तिप्रदर्शन इत्यर्थो द्रष्टव्यः । \textbf{एतावदेवै}तावन्मात्रम् । अन्यदप्यस्य रूपमस्तीत्याह--\textbf{एतावदेवे}ति । हिर्यस्मादेतत् परिमाणमस्येति तथा । \textbf{एव}कारेणान्यस्य ताद्रूप्यनिरासो दृढीकृतः । किं तद्रूपमित्याह--\textbf{यदुते}ति । निपातसमुदायश्चायं यदेतदित्यस्यार्थे ॥
	\pend
      

	  \pstart येषु दृष्टान्तत्वेनोपात्तेषु स्वकार्यका\footnote{क}रणासामर्थ्ये दोषः सम्भवति ते कुतश्चित्सामान्याद् दृष्टान्तवदाभासमाना दृष्टान्ताभासा भवन्तीति अर्थं न्यायमाश्रित्याह--\textbf{दृष्टान्तदोषा दृष्टान्ताभासाः कथिता भवन्तीति} । यस्मात्ते दृष्टान्ताभासत्वे\add{न} कथिता भवन्त्यत एव दृष्टान्तत्वेन \textbf{निरस्ता भवन्ती}त्यत एव मूलमर्थतो व्याख्यातमित्यवगन्तव्यम् । दृष्टान्ततत्त्वप्रदर्शनेन यथा दृष्टान्ताभासा\footnote{स}कथनं कृतं भवति तथा दर्शयितुं \textbf{तथा ही}त्यादिनोपक्रमते । \textbf{पूर्वोक्त}स्य जन्मतन्मात्रानुबन्धस्य \textbf{सिद्धये} निश्चयाय । \textbf{स्वकार्यं} साधनसम्भवे साध्यप्रदर्शनसम्भवलक्षणम्, सिद्धव्याप्तिकस्य च हेतोः साध्याभावे साधननिवृत्तिप्रदर्शनलक्षणं च \textbf{साधयितुं य उपादीयमानोऽपि न समर्थः, स दृष्टान्तदोषो} दुष्टो दृष्टान्तः, दृष्टान्ताभास इति यावत् । \textbf{सामर्थ्यात्} स्वकार्याकारणलक्षणात् ॥
	\pend
      \leavevmode\marginnote{\textenglish{240/dm}}“

	  \pstart दृष्टान्ताभासामुदाहरति--
	\pend
       “

	  \pstart यथा नित्यः शब्दोऽमूर्त्तत्वात् । कर्मवत् परमाणुवद् घटवदिति । एते दृष्टान्ताभासाः साध्यसाधनधर्मोभयविकलाः ॥ १२४ ॥
	\pend
      ” 

	  \pstart यथा \footnote{यथेति नित्यः \cite{dp-msA} \cite{dp-msB} \cite{dp-edP} \cite{dp-edH} \cite{dp-edE} \cite{dp-edN}}नित्यः शब्द इति \footnote{इति नित्यत्वे साध्ये शब्दस्यामू० \cite{dp-msD} \cite{dp-msB}}शब्दस्य नित्यत्वे साध्येऽमूर्तत्वादिति हेतुः । साधर्म्येण कर्मवत् परमाणुवद् घटवदित्येते दृष्टान्ता उपन्यस्ताः । एते च दृष्टान्तदोषाः । साध्यं च साधनं चोभयं चेति\footnote{च । तैर्वि० \cite{dp-msC}} । तैर्विकलाः । साध्यविकलं कर्म, तस्याऽनित्यत्वात् । साधनविकलः परमाणुः, मूर्त्तत्वात् परमाणूनाम । \footnote{असर्वगतं द्र० \cite{dp-msA} \cite{dp-msB} \cite{dp-edP} \cite{dp-edH} \cite{dp-edN}}असर्वगतद्रव्यपरिमाणं मूर्त्तिः । असर्वगताश्च द्रव्यरूपाश्च परमाणवः । नित्यास्तु वैशेषिकैरिष्यन्ते । ततो न साध्यविकलः\footnote{साध्यविकलाः \cite{dp-msC} \cite{dp-msD}} । घटस्तूभयविकलः, अनित्यत्वान्मूर्त्तत्वाच्च घटस्येति\footnote{घटस्य । \cite{dp-edE}} ॥
	\pend
       “

	  \pstart तथा सन्दिग्धसाध्यधर्मादयश्च । यथा रागादिमानयं वचनाद्रथ्यापुरुषवत् । मरणधर्माऽयं पुरुषो रागादिमत्वाद्रथ्यापुरुषवत् । असर्वज्ञोऽयं रागादिमत्वाद्रथ्यापुरुषवदिति ॥ १२५ ॥
	\pend
      ” 

	  \pstart तथा\footnote{तथेति \cite{dp-edE}} संदिग्धः साध्यधर्मो यस्मिन् स संदिग्धसाध्यधर्मः । स आदिर्येषां ते तथोक्ताः । संदिग्धसाध्यधर्मः । संदिग्धसाधनधर्मः संदिग्धोभयः\footnote{०भयधर्मः । \cite{dp-msC}} ।
	\pend
      ”

	  \pstart \textbf{दृष्टान्ताभासानि}त्यादि \textbf{परमाणूनामि}त्येतदन्तं सुगमम् ।
	\pend
      

	  \pstart कथं परमाणवः साधनविकला न साध्यविकला इत्याह--अ\textbf{सर्वे}ति । \textbf{परिमाणं} मानव्यवहारकारणम् । तच्च गुणत्वाद् द्रव्याश्रयीति । \textbf{द्रव्य}ग्रहणेन वास्तवं रूपमनूदितम् । तत्र यदि द्रव्यपरिमाणं मूर्त्तिरित्येव तावदुच्यते तदाऽऽकाशादेरपि द्रव्यस्य परममहत्त्वनामधेयं परिमाणमस्तीति मूर्त्तत्वं प्रसज्येत । अतस्तन्निवृत्त्यर्थं द्रव्ये विशेषणम\textbf{सर्वगत}ग्रहणम् । तत्र परमाणोः परिमाणं भवत्यसर्वगतस्य द्रव्यस्येति दर्शयन्नाह--\textbf{असर्वगताश्चे}ति । समवायिकारणं द्रव्यं गुणवद्वेति द्रव्यलक्षणयोगाद् द्रव्यरूपः । \textbf{च}कारौ पूर्वापेक्षया एकविषयत्वमनयोः समुच्चिनुतः । तत् पुनः परमाणोः परिमाणं पारिमाण्डलसंज्ञकं ज्ञातव्यम् । इयता साधनवैकल्यं दर्शितम् ।
	\pend
      

	  \pstart साध्यावैकल्यं दर्शयन्नाह--\textbf{नित्यास्त्वि}ति । तुनेनार्थमान्तरेण \footnote{?} विशिनष्टि । द्रव्यगुणकर्मसामान्यविशेषसमवायात्मकैः पदार्थविशेषैर्व्यवहरन्तीति \textbf{वैशेषिकाः,} रूढ\footnote{ढे} श्चाभ्युपगत\textbf{कणादशास्त्रा} एवोच्यन्ते । अथ\add{वा} षट्पदार्थीप्रतिपादकतया विशिष्यते तदन्यस्माच्छास्त्रादिति विशेषः \textbf{काणादं शास्त्रं} विवक्षितम् । \footnote{तुलना--“तदधीते तद्वेद” \href{http://http://sarit.indology.info/?cref=Pā.4.3.59}{पाणिनि ४. २. ५६}.}“तद् विदन्त्यधीयते वा” इति \textbf{वैशेषिकास्तैरिष्यन्त} इति वचनव्यक्त्या चेष्टिमात्रमेतन्न पुनरत्र प्रमाणमस्तीति सूचयति ।
	\pend
      [[दिति साध्य० \cite{dp-msB} \cite{dp-edP} \cite{dp-edH} \cite{dp-edE} \cite{dp-edN}]]\leavevmode\marginnote{\textenglish{241/dm}}“

	  \pstart उदाहरणम्--रागादिमानिति रागादिमत्त्वं साध्यम् । वचनाद् इति हेतुः । रथ्यापुरुषवदिति दृष्टान्ते\footnote{दृष्टान्तः रा० \cite{dp-msA} \cite{dp-msB} \cite{dp-msD} \cite{dp-edP} \cite{dp-edH} \cite{dp-edE} \cite{dp-edN}} रागादिमत्त्वं सन्दिग्धम् । मरणं धर्मोऽस्येति मरणधर्मा । तस्य भावो मरणधर्मत्वं साध्यम् । अयं पुरुष इति धर्मी । रागादिमत्त्वादिति हेतुः । रथ्यापुरुषे दृष्टान्ते सन्दिग्धं साधनम् । साध्यं तु निश्चितं मरणधर्मत्वमिति । असर्वज्ञ इति । असर्वज्ञत्वं साध्यम् । रागादिमत्त्वादिति हेतुः । तदुभयमपि रथ्यापुरुषे दृष्टान्ते सन्दिग्धम् । असर्वज्ञत्वं रागादिमत्त्वं चेति ॥
	\pend
       “

	  \pstart \footnote{“तथा” नास्ति \cite{dp-msB} \cite{dp-edP} \cite{dp-edH} \cite{dp-edE} \cite{dp-edN}}तथाऽनन्वयोऽप्रदर्शितान्वयश्च । यथा - यो वक्ता स रागादिमान्, इष्टपुरुषवत् । अनित्यः शब्दः कृतकत्वाद् घटवदिति ॥ १२६ ॥
	\pend
      ” 

	  \pstart तथाऽनन्वय इति । यस्मिन् दृष्टान्ते साध्यसाधनयोः सम्भवमात्रं दृश्यते, न तु साध्येन व्याप्तो हेतुः, सोऽनन्वयः । अप्रदर्शितान्वयश्च--यस्मिन् दृष्टान्ते विद्यमानोऽप्यन्वयो न प्रदर्शितो वक्त्रा सोऽप्रदर्शितान्वयः ।
	\pend
       

	  \pstart अनन्वयमुदाहरति--यथेति । यो वक्तेति वक्तृत्वमनूद्य स रागादिमानिति रागादिमत्त्वं\footnote{०मत्त्वे वि० \cite{dp-msA}} विहितम् । ततो वक्तृत्वस्य रागादिमत्त्वं \footnote{वे} प्रतिनियमः । तेन व्याप्तिरुक्ता । इष्टपुरुषवदिति । इष्टग्रहणेन प्रतिवाद्यपि \footnote{गृह्यते \cite{dp-msA} \cite{dp-edP} \cite{dp-edH} \cite{dp-edE} \cite{dp-edN}}संगृह्यते वाद्यपि । तेन वक्तृत्वरागादिमत्त्वयोः सत्त्वमात्रमिष्टे पुरुषे सिद्धम् । व्याप्तिस्तु न सिद्धा । तेनाऽनन्वयो दृष्टान्त इति ।
	\pend
       

	  \pstart अनित्यः शब्द इत्यनित्यत्वं साध्यम् । कृतकत्वादिति हेतुः । घटवदिति\footnote{०दित्यत्र दृ० \cite{dp-msA} \cite{dp-edP} \cite{dp-edH} \cite{dp-edE}} दृष्टान्ते न प्रदर्शितोऽन्वयः ।
	\pend
      ”

	  \pstart \textbf{तत} इत्यादि \textbf{मरणधर्मत्वमि}त्येतदन्तं सुज्ञानम् ।
	\pend
      

	  \pstart ननु मरणधर्मत्वमप्यस्य कथं निश्चितं येन हेतुरेव सन्दिग्ध उच्यत इति चेत् । सत्यम् । केवलं प्रसिद्धिसमाश्रयेणैवमुक्तमित्यवसेयम् । अन्वीयमानत्वं साधनस्य साध्येनान्वयः । स च प्रतिबन्धसाधकप्रमाणाक्षेपात् प्रसिद्ध्यति । यत्र तु तन्नास्ति केवलं सम्भवमात्रं साहचर्यमात्रं दृश्यत इत्यभिप्रायः ॥
	\pend
      

	  \pstart \textbf{वक्तृत्वस्य} हेतो \textbf{रागादिमत्त्वे} \leavevmode\marginnote{\textenglish{79b/ms}} साध्ये \textbf{प्रतिनियमः} प्रतिनियतत्वमुक्तमिति शेषः । \textbf{तेन} साध्यनियतत्वेन \textbf{व्याप्तिर}नयो\textbf{रुक्ता} प्रदर्शिता । एवंविधानुवादविध्युपदर्शने तथा प्रतीतेराहत्योदयाद् \textbf{व्याप्तिरुक्ते}त्युक्तम् । न त्वसावनयोर्वाक्यतोऽस्ति । यतो द्वयोरपि वादिप्रतिवादिनोदृष्टान्तत्वेन सङ्ग्रहः \textbf{तेन} हेतुना \textbf{सत्त्वमात्रं} सम्भवमात्रं साहचर्यमात्रमिति यावत् । अस्ति चेष्टिः सहमात्रोपदर्शनमदोषावहमित्याह--\textbf{व्याप्तिस्त्वि}ति । \textbf{तुः} सत्त्वमात्राद् व्याप्तिं भिनत्ति । \textbf{न सिद्धा न} प्रमाणनिश्चिता । अनयोर्व्याप्तिसाधकप्रमाणाभावादिति भावः ।
	\pend
      

	  \pstart अप्रदर्शितान्वयमुदाहरन्नाह--\textbf{अनित्य} इति ।
	\pend
      \leavevmode\marginnote{\textenglish{242/dm}}“

	  \pstart इह यद्यपि कृतकत्वेन घटसदृशः शब्दस्तथापि नानित्यत्वेनापि सदृशः प्रत्येतुं शक्यते\footnote{शक्योऽति० \cite{dp-msA} \cite{dp-edP} \cite{dp-edH} \cite{dp-edE} \cite{dp-edN}}ऽतिप्रसङ्गात्\footnote{पाक्योऽपि प्राप्तः--\cite{dp-msD-n}} । यदि तु कृतकत्वम् अनित्य\footnote{अनित्यत्व \cite{dp-msA} \cite{dp-msC} \cite{dp-edP} \cite{dp-edH} \cite{dp-edE} \cite{dp-edN}} स्वभावं\footnote{ज्ञातं \cite{dp-msC}} विज्ञातं भवत्येवं\footnote{०त्येव कृ० \cite{dp-msC}} कृतकत्वादनित्यत्वप्रतीतिः स्यात् । तस्माद् यत् कृतकं तदनित्यमिति कृतकत्वमनित्यत्वे\footnote{०त्यत्वनिय० \cite{dp-msA} \cite{dp-edP} \cite{dp-edH} \cite{dp-edE} \cite{dp-edN}} नियतमभिधाय नियमसाधनायान्वयवाक्यार्थप्रतिपत्तिविषयो दृष्टान्त उपादेयः । स च प्रदर्शितान्वय एव । अनेन त्वन्वयवाक्यमनुक्त्वैव दृष्टान्त उपात्तः । ईदृशश्च साधर्म्यमात्रेणैवोपयोगी । न च साधर्म्यात् साध्यसिद्धिः । अतोऽन्वयार्थो दृष्टान्तस्तदर्थश्चानेन नोपात्तः । साधर्म्यार्थश्चोपात्तो निरुपयोग
	\pend
      ”

	  \pstart ननु घटोपदर्शनेन कृतकत्वेन तावद् घटसदृशः शब्दो दर्शितः । तथा चानित्यत्वेनापि सदृशो दर्शितस्ततश्च व्याप्तिर्दर्शितैवेत्याह--इहेति । इहानित्यत्वसिद्धिप्रस्तावे । कुतस्तथा\textbf{प्रत्येतुमशक्य} इत्याह--\textbf{अतिप्रसङ्गादिति} । मूर्त्तत्वादि\footnote{दे}रपि सादृश्यागम \footnote{सादृश्यावगमात्} \textbf{प्रसक्ति \footnote{क्ते} रिष्टं धर्ममति}क्रान्तः \textbf{प्रसङ्गोऽतिप्रसङ्गः} तस्मात् । न तर्हि कृतकत्वादनित्यत्वं कदाचिदपि प्रत्येतव्यमित्याह--\textbf{यदि त्वि}ति । तुरिमामवस्थां विशेषवतीं दर्शयति । यदि कृतकत्वानुवादपूर्विका \footnote{पूर्वक} विधानेनार्थान्तरत्वे सति न नित्यत्वस्वभावं कृतकत्वं प्रतीतं \textbf{भवत्येवं} तत्स्वाभाव्येन प्रतिपत्तौ \textbf{कृतकत्वादनित्यत्वप्रतीतिः स्या}त् । यतः कृतकत्वस्यानित्यत्वस्वभावावगमात्ततस्तत्प्रतीतिर्नान्यथा \textbf{तस्माद्} हेतोः । \textbf{अन्वयवाक्यस्य} साध्यनियतत्वं साधनस्या\textbf{र्थो}ऽभिधेयः, \textbf{तत्प्रतिपत्तिविष}यस्तत्प्रदर्शन इत्यर्थः । अन्वयवाक्यार्थप्रतिपत्तिविषयोऽपि दृष्टान्तो यद्यप्रदर्शितान्वयस्तदा सोऽपि निरुपयोगः किमित्युपादेय इत्याह--\textbf{स चे}ति । \textbf{चो}ऽवधारणे यस्मादर्थे वा । अनेन कथं नामायमुपात्तो येनैवमभिधीयत इत्याह--\textbf{अनेनेति । अनेन} वादिना । तुर्विशेषार्थः । अन्वयप्रतिपादकं \textbf{वाक्यमन्वयवाक्यम्} । तदनुक्त्वैव । अथैवमुपात्तोऽपि यद्युपयुज्यते तदा का क्षतिरित्याह--\textbf{ईदृशश्चे}ति । \textbf{च}कारोऽस्येमामवस्थां भेदवतीमाह । समानः प्रकरणाद् घटेन सदृशो धर्मो यस्यासौ \textbf{सधर्मा} । तस्य भावः \textbf{साधर्म्यम्} । तदेव तन्\textbf{मात्रम् । मात्रग्र}हणेन विशिष्टं साधर्म्यमपाकरोति । तेनैवकारेण निराकृतनिरासमेव द्रढयति । साधर्म्यप्रदर्शनमात्रेणैवायमुपयोगवानित्यर्थः । अथैतत्पददर्शितात्साधर्म्यादपि यदि साध्यं सिद्ध्यति तदा कथमयमनुपयुक्त इत्याह--न \textbf{चे}ति । \textbf{चो}ऽवधारणे यस्मादर्थे वा ।
	\pend
      

	  \pstart एवं ब्रुवतोऽयं भावः--अनेन खलु साधर्म्यं प्रदर्शनीयम् । कृतकत्वेनैव च साधर्म्यं प्रदर्श्यायञ्चरितार्थो भविष्यति । न च कृतकत्वेन घटसाधर्म्ये शब्दस्यावगतेऽप्यनित्यत्वेनापि तत्साधर्म्यावगमोऽवश्यम्भावीति शक्यमभिधातुम् । मूर्त्तत्वादिनापि सादृश्यावगमेऽस्यानिवार्यत्वप्रसङ्गादिति । यतः साधर्म्यमात्रान्न साध्यसिद्धिरतः कारणात् । \textbf{अन्वयोऽन्वीयमानत्वं साधनस्य साध्येन सोऽर्थः} प्रयोज \leavevmode\marginnote{\textenglish{80a/ms}} नं यस्य स तथा न साधर्म्यमात्रप्रदर्शनार्थ इत्यर्थात् । अनेनाप्यन्वयार्थ एवायमुपात्त इत्याह--\textbf{तदर्थश्चे}ति । \textbf{चो} यस्मादर्थे । सोऽन्वयोऽर्थो यस्य स तथा । \textbf{अनेन} वादिना । किमर्थस्तर्ह्यनेनायमुपात्त इत्याह--\textbf{साधर्म्ये}ति । \textbf{चो} यस्मादर्थे ।  \leavevmode\marginnote{\textenglish{243/dm}} “
	  
	इति वक्तृदोषादयं दृष्टान्तदोषः । वक्त्रा ह्यत्र परः प्रतिपादयितव्यः । ततो यदि नाम न दुष्टं वस्तु तथापि वक्त्रा दुष्टं दर्शितमिति दुष्टमेव ॥ “
	  
	तथा विपरीतान्वयः--यदनित्यं तत् कृतकमिति ॥ १२७ ॥” 
	  
	तथा विपरीतोऽन्वयो यस्मिन् दृष्टान्ते स तथोक्तः । तमेवोदाहरति--यदनित्यं तत् कृतकमिति । कृतकत्वमनित्यत्वनियतं दृष्टान्ते\footnote{दृष्टान्तेन \cite{dp-edE}} दर्शनीयम् । एवं कृतकत्वादनित्यत्वगतिः स्यात् । अत्र त्वनित्यत्वं कृतकत्वे\footnote{०कत्वनियतं \cite{dp-msD}} नियतं दर्शितम् । कृतकत्वं \footnote{त्वनियतमेवोक्तमनित्यत्वे \cite{dp-msC} \cite{dp-msD}}त्वनियतमेवानित्यत्वे ।\footnote{अत्र “ततः कृतकत्वमनित्यत्वे नियतमेव” इत्यधिकः पाठोऽस्ति \cite{dp-edE} प्रतौ ।} ततो यादृशमिह कृतकत्वमनियतमनित्यत्वे \footnote{प्रदर्शितम् \cite{dp-msA} \cite{dp-msB} \cite{dp-edP} \cite{dp-edH} \cite{dp-edE} \cite{dp-edN}}दर्शितं तादृशान्नास्त्यनित्यत्वप्रतीतिः । तथाहि—यदनित्यमित्यनित्यत्वमनूद्य तत् कृतकमिति कृतकत्वं विहितम् । अतोऽनित्यत्वं नियतमुक्तं कृतकत्वे, न तु कृतकत्वमनित्यत्वे । ततो यथाऽनित्यत्वादनियतात् प्रयत्नानन्तरीयकत्वे न प्रयत्नानन्तरीयकत्वप्रतीतिः, तद्वत् कृतकत्वादनित्यत्वप्रतिपत्तिर्न स्याद्, अनित्यत्वेऽनियतत्वात् कृतकत्वस्य ।” \textbf{साधर्म्यार्थः} साधर्म्यप्रतिपादनप्रयोजन \textbf{उपात्त}स्तस्मा\textbf{न्निरुपयोगः} स । तस्मादर्थें वा । अयं चशब्दात् परो द्रष्टव्यः । ननु कृतकत्वानित्यत्वयोस्तावद् वस्तुतोऽन्वयोऽस्त्येव । तत्कथं विद्यमानेऽपि तस्मिन् तथाऽप्रदर्शनमात्रेणासौ दृष्टान्तो दुष्यतीत्याह--\textbf{वक्तृदोषा}दिति । \textbf{वक्ता ही}त्यादिनैतदेव समर्थयते । ननु वक्तैवासौ तथा प्रदर्शयन्नपराध्यतु, अन्यस्य दृष्टान्ततपस्विनः कोऽपराध इति चेत् । स्वकार्याऽकरणमेव दोषः । तदकरणं तस्य स्वत एव \footnote{एवाऽ}सामर्थ्याद्, अन्येनान्यथाप्रदर्शनाद् वाऽस्तु । किमेतावता तदकरणं तस्य नास्त्येवेति सूक्तं \textbf{वक्तुदोषादयं दृष्टान्तदोष} इति ॥
	\pend
      

	  \pstart \textbf{विपरीतोऽन्वय} इति वैपरीत्येन प्रदर्शनाद् विपरीत उक्तो न तु विपरीतोन्वयोऽस्त्येव । कीदृशोऽविपरीतोऽन्वयो यस्मादयं विपरीत इत्याह--\textbf{कृतकत्वमि}ति । एवं प्रदर्शने को गुण इत्याह--\textbf{एवमि}ति । एवं कृतकत्वस्यानित्यत्वे नियतत्वप्रदर्शने सति । अत्र पुनः कुत्र किं नियतमित्याह--\textbf{अत्रेति} । तुरिमामवस्थां विशेषवतीं दर्शयति । अनित्यत्वानुवादेन कृतकत्वस्य विधानाद् विधीयमानस्य व्यापकतया नियमविषयत्वादित्यभिसन्धिः कृतकत्वं पुनः कीदृशं दर्शितमित्याह--\add{कृ\textbf{तकत्वमि}ति} । तुरनित्यत्वात् कृतकत्वं भिनत्ति । \textbf{अनि}त्यत्वे\textbf{ऽनियतमि}ति कोऽर्थोऽनिप्यत्वेऽपि कृतकत्त्वं भवति, अन्तरेण कृतकत्वमिति । अनित्यत्वानियतादपि कृतकत्वादि \footnote{द} नित्यत्वं प्रतिपश्येत इत्याह--\textbf{तत} इति । यत एवमनुवादविधिक्रमे कृतकत्वमनित्यत्वानियतं दर्शितं भवति, ततस्तस्मादिह प्रयोगे । \textbf{यादृश}मिति विधीयमानम् । पूर्वमेवंवादिनाऽ\textbf{नित्यत्वं कृतकत्वे} नियतं \textbf{दर्शितम्,} न कृतकत्वमनित्यत्वे चेति प्रतिज्ञामात्रेणोक्तमधुना तु येन प्रकारेण तस्यैव प्रदर्शनमायातं \textbf{तथाही}त्यादिना तद् दर्शयति । अनित्यत्वं प्रयत्नानन्तरीयक  \leavevmode\marginnote{\textenglish{244/dm}} “
	  
	यद्यपि च कृतकत्वं वस्तुस्थित्याऽनित्यत्वे नियतं \footnote{“तथाप्यनियतं” नास्ति \cite{dp-msB}}तथाप्यनियतं वक्त्रा\footnote{प्रदर्शितम् \cite{dp-msC}} दर्शितम् । अतः \footnote{अतस्तत् स्वयं न दुष्ट० \cite{dp-msA} \cite{dp-msB} \cite{dp-edP} \cite{dp-edH} \cite{dp-edE} \cite{dp-edN}}स्वयमदुष्टमपि \footnote{वक्तुर्दोषात्--\cite{dp-msA} \cite{dp-edP} \cite{dp-edH} \cite{dp-edE}}वक्तृदोषाद् दुष्टम् । 
	  
	तस्माद् विपरीतान्वयोऽपि वक्तुरपराधात्, न वस्तुतः \footnote{नः} । परार्थानुमाने च\footnote{“च” नास्ति \cite{dp-msC} \cite{dp-msD}} वक्तुरपि \footnote{न केवलं हेतोः--\cite{dp-msD-n}}दोषश्चिन्त्यत इति ॥ “
	  
	साधर्म्येण दृष्टान्तदोषाः \footnote{“दृष्टान्तदोषाः” इति नास्ति--\cite{dp-msB} \cite{dp-edP} \cite{dp-edH} \cite{dp-edE} \cite{dp-edN}}॥ १२८ ॥” 
	  
	साधर्म्येण\footnote{०र्म्येण तद्दृ० \cite{dp-msB}} नव दृष्टान्तदोषा उक्ताः ॥ 
	  
	वैधर्म्येणापि\footnote{०पि नव दृ० \cite{dp-msA} \cite{dp-edP} \cite{dp-edH} \cite{dp-edE} \cite{dp-edN}} दृष्टान्तदोषान्\footnote{वक्तुकाम आह \cite{dp-msB} \cite{dp-msD}} वक्तुमाह-- “
	  
	वैधर्म्येणापि--परमाणुवत् कर्मवद् आकाशवदिति साध्याद्यव्यतिरेकिणः ॥ १२९ ॥” 
	  
	नित्यत्वे शब्दस्य साध्ये हेतावमूर्त्तत्वे \footnote{परमाणुवद्वध० \cite{dp-msA} \cite{dp-edP} \cite{dp-edH} \cite{dp-edE} परमाणुर्वैधर्म्येण \cite{dp-msC}}परमाणुर्वैधर्म्यदृष्टान्तः साध्याव्यतिरेकी । नित्यत्वात् परमाणूनाम् । कर्म साधनाव्यतिरेकि, अमूर्त्तत्वात् कर्मणः आकाशमुभयाव्यतिरेकि, नित्यत्वादमूर्त्तत्वाच्च ।” त्वमन्तरेणाचिरप्रभादौ दृश्यमानमनियतं तत्र । तत्र यथाऽनित्यत्वात्प्रयत्नानन्तरीयकत्वाप्रतीतिस्तद्वत् कृतकत्वादप्यनित्य \add{त्व} नियतान्नानित्यत्वप्रतीतिर्भवितुमर्हति ।
	\pend
      

	  \pstart ननु भवतु अनित्यत्वात्प्रयत्नानन्तरीयकत्वाप्रतीतिर्वस्तुतस्तस्य तत्रानियतत्वात्कृतकत्वं तु परमार्थतो नियतमनित्यत्वे । तत्कुतस्तस्मात्तस्याप्रतीतिरित्याह--\textbf{यद्यपित्या}दि । \textbf{एवं} ब्रुवतोऽयं भावः--वस्तुतश्छेदनस्वभावोऽपि परशुर्यदाच्छेदकेन भ्रान्त्याऽन्यथा वा उद्वर्त्त्य बाहुमुद्यम्य द्वैधीभावार्थं धवादौ निपात्यते, तदा तेन स्वयमदुष्टेनापि यथा छिदा न सम्पाद्यते, तद्वदनेनापि परोक्षार्थप्रतीतिर्न सम्पाद्यते, स्वयमदुष्टेनापीति ।
	\pend
      

	  \pstart यस्मादेवं त \leavevmode\marginnote{\textenglish{80b/ms}} स्माद् हेतोर्वक्तुरनुमापयितु\textbf{रपराधाद्} दोषाद् \textbf{विपरीतान्वयोऽपि} दृष्टान्तदोष इति दोषः । न केवलमप्रदर्शितान्वय इत्याह--\textbf{न वस्तुनः} कृतकत्वस्यापराधात् । तत्तावत्स्वतोऽदुष्टम् । तत्किं वक्तृदोषेण चिन्तितेनेत्याह--\textbf{परार्थे}ति । न केवलं साधनस्य दोषश्चिन्त्यत इत्यपि शब्दात् ॥
	\pend
      

	  \pstart यतोऽल्पीयो नास्ति स \textbf{परमाणुः} वैधर्म्यप्रतिपाद \add{न} विषयोपात्तत्वान् \textbf{वैधर्म्यदृष्टान्त उक्तः} । \textbf{साध्य}स्या\textbf{व्यतिरेको} निवृत्त्यभावः सोऽस्यास्तीति तथोक्तः ॥
	\pend
      \leavevmode\marginnote{\textenglish{245/dm}}“

	  \pstart साध्यमादिर्येषां तानि साध्यादीनि साध्यसाधनोभयानि । तेषामव्यतिरेको \footnote{०रेको वृत्त्य० \cite{dp-msA} \cite{dp-edP} \cite{dp-edH} \cite{dp-edN} ०रेको निवृत्ता[[त्त्य]] भावः \cite{dp-msB}}निवृत्त्यभावः । स येषामस्ति ते साध्याद्यव्यतिरेकिणः । ते चोदाहृताः ॥
	\pend
       

	  \pstart अपरानुदाहर्त्तुमाह--
	\pend
       “

	  \pstart तथा संदिग्धसाध्यव्यतिरेकादयः । यथाऽसर्वज्ञाः कपिलादयोऽनाप्ता वा अविद्यमानसर्वज्ञताप्ततालिङ्गभूतप्रमाणातिशयशासनत्वादिति । अत्र \footnote{वैधर्म्येणोदा० \cite{dp-msC}}वैधर्म्योदाहरणम्--यः सर्वज्ञ आप्तो वा स ज्योतिर्ज्ञानादिकमुपदिष्टवान् । \footnote{तद्यथा \cite{dp-msB} \cite{dp-msD} \cite{dp-edP} \cite{dp-edH} \cite{dp-edE} \cite{dp-edN}}यथा--\footnote{वृषभ \cite{dp-msC}}ऋषभवर्धमानादिरिति । \footnote{रिति वैधर्म्योदाहरणादसर्व० \cite{dp-msC}}तत्रासर्वज्ञतानाप्ततयोः साध्यधर्मयोः संदिग्धो व्यतिरेकः ॥१३०॥
	\pend
      ” 

	  \pstart तथेति । साध्यस्य व्यतिरेकः साध्यव्यतिरेकः । संदिग्धः साध्यव्यतिरेको यस्मिन् स संदिग्धसाध्यव्यतिरेकः । स आदिर्येषां ते तथोक्ताः ।
	\pend
       

	  \pstart संदिग्धसाध्यव्यतिरेकमुदाहर्त्तुमाह--यथेति । असर्वज्ञा इत्येकं साध्यम् । अनाप्ता अक्षीणदोषा इति द्वितीयम् । कपिलादय इति धर्मीं । अविद्यमानसर्वज्ञतेत्यादि हेतुः । सर्वज्ञता च आप्तता च तयोर्लिङ्गभूतः प्रमाणातिशयो लिङ्गात्मकः प्रमाणविशेषः । अविद्यमानः सर्वज्ञताप्ततालिङ्गभूतः प्रमाणातिशयो यस्मिन् तत् तथोक्तं शासनम् । तादृशं शासनं येषां ते\footnote{कपिलादयः--\cite{dp-msD-n}} तथोक्ताः । तेषां भावस्तत्त्वम् । तस्मात् । प्रमाणातिशयो ज्योतिर्ज्ञानोपदेश इहाभिप्रेतः । यदि हि कपिलादयः सर्वज्ञा आप्ता वा स्युस्तदा ज्योतिर्ज्ञानादिकं कस्मान्नोपदिष्टवन्तः ? न चोपदिष्टवन्तः । तस्मान्न सर्वज्ञा आप्ता वा ।
	\pend
       

	  \pstart अत्र प्रमाणे वैधर्म्योदाहरणम् । यः सर्वज्ञ आप्तो वा स ज्योतिर्ज्ञानादिकं सर्वज्ञताप्ततालिङ्गभूतमुपदिष्टवान् । यथा ऋषभो वर्धमानश्च तावादी यस्य स ऋषभ-वर्धमानादिर्दि-
	\pend
      ”

	  \pstart प्रकरणात्क्षीणदोषत्वमाप्तत्वलक्षणमस्य विवक्षितमिति तत्वा \textbf{अनाप्ता अक्षीणदोषा} इति व्याचष्टे । \textbf{कपिलः सांख्य}दर्शनकारो मुनिः । आदिशब्देन \textbf{गौतमा}देः सङ्ग्रहः । \textbf{शासनं} मार्गस्तत्प्रणीतशास्त्रमिति यावत् ।
	\pend
      

	  \pstart ननु तेषामपि शासने लिङ्गात्मकः प्रमाणातिशयोऽनेकोऽस्त्येव । तत्कथं हेतुरयमसिद्धो न भवतीत्याह--\textbf{प्रमाणातिशय} इति । \textbf{ज्योतिषां} ग्रहनक्षत्राणां \textbf{ज्ञानं} तस्यो\textbf{पदेशः} । ज्योतिर्ज्ञानस्योपलक्षणत्वाज्ज्योतिर्ज्ञानाद्युपदेश इत्यवगन्तव्यम् । तदयं समुदायार्थः--यस्मात्तैर्ज्योतिर्ज्ञानमन्यातीन्द्रियज्ञानं च स्वशासने नोपदिष्टं तस्मात्ते न तथारूपा इति । \textbf{यदि ही}त्यादिनैतदेव दर्शयति ।  \leavevmode\marginnote{\textenglish{246/dm}} “
	  
	गम्बराणां शास्ता\footnote{शास्ता साध्यव्यतिरेकः सर्व० \cite{dp-msC}} सर्वज्ञश्च\footnote{“च” नास्ति \cite{dp-msA} \cite{dp-msB} \cite{dp-edP} \cite{dp-edH} \cite{dp-edE} \cite{dp-edN}} आप्तश्चेति । तदिह वैधर्म्योदाहरणाद् ऋषभादेरसर्वज्ञत्वस्यानाप्ततायाश्च व्यतिरेको व्यावृत्तिः संदिग्धा । यतो ज्योतिर्ज्ञानं चोपदिशेद् असर्वज्ञाश्च भवेद् अनाप्ता वा । कोऽत्र विरोधः ? नैमित्तिकमेतज्ज्ञानं व्यभिचारि न सर्वज्ञत्वमनुमापयेत् ॥” आदिशब्देनान्यातीन्द्रियज्ञानं संगृह्यते । ऋषभवर्धमाननामधेयावह्नीकाणामिष्टदेवौ । आदिशब्देन \textbf{पार्श्वनाथारिष्टनेमिप्रभृ}तेः सङ्ग्रहः । \textbf{शास्ता} स्वदृष्टमन्योपदेष्टा । \textbf{सर्वज्ञश्चाप्तश्चे}त्यनेन द्वयस्यापि साध्यस्याभावं दर्शयति । यदाह--\textbf{अकलङ्कः}--यदि सूक्ष्मे व्यवहिते वा वस्तुनि बुद्धिरत्यन्तपरोक्षे न स्यात्कथं तर्हि ज्योतिर्ज्ञानाविसंवादः ? ज्योतिर्ज्ञानमपि हि सर्वज्ञ\textbf{प्रवर्त्तितमेव,} एतस्मादविसंवादिनो ज्योतिर्ज्ञानात्सर्वज्ञसिद्धिः । तदुक्तम्--
	\pend
      “
	    
	    \stanza[\smallbreak]
	धीरत्यन्तपरोक्षेऽर्थे न चेत्पुंसां कुतः पुनः ।&ज्योतिर्ज्ञानाविसंवादः श्रुतत्वाच्चेत्साधनान्तरम् ॥\&[\smallbreak]


	\href{http://http://sarit.indology.info/?cref=svi.p413}{सिद्धिवि०, पृ० ४१३} इति ।”

	  \pstart कथं पुनः सत्यपि ज्योतिर्ज्ञानाद्युपदेशे विपक्षादृषभादेरसर्वज्ञत्वादेर्व्यावृत्तिः सन्दिग्धेत्याह—\textbf{यत} इति । अत्र सर्वज्ञतायामनिष्टायां सत्यामपि ज्योतिर्ज्ञानाद्युपदेशे \textbf{को विरोधो}ऽनुपपत्तिः ? क्षेपे \textbf{किमः} प्रयोगान्न कश्चिदित्यर्थः ।
	\pend
      

	  \pstart अथासर्वज्ञत्वे तस्यैतस्मादिदं ग्रहोपरागादि भावि ततश्चैवं भावीति ज्ञानं कथं वृत्तं येन सम्वादि तथोपदिशेत् । तस्मात् सर्वज्ञ एवासाविति निश्चय इत्याशङ्क्याह--\textbf{नैभित्तिकमेत}दिति । एतज्ज्योतिर्ज्ञानादिकं निमित्तात्परम्परया कारणाद् । विषयेण च विषयिणो निर्देशान्निमित्तदर्शनादागतं \textbf{नैमित्तिकम्,} अत एव तदुपदेष्टुः सर्वज्ञता व्यभिचरतीति । तथा सर्वज्ञतामन्तरेणापि भवतीदानीन्तनज्योतिषिकाणामिवाऽतीन्द्रियोपरागादिज्ञानमित्यभिप्रायः । तथाभूतं \textbf{ज्ञानं न सर्वज्ञतामनुमापयेद}नुमापयितुं शक्नोति ।
	\pend
      

	  \pstart ननु च तथाभूतेन भाविवस्तुना सह कस्यचित्कार्यकारणभाव एव तेन कश्चिद् ज्ञातः । यद्य\leavevmode\marginnote{\textenglish{81a/ms}}सावसर्वज्ञो भवेद् असर्वज्ञश्च कथमुपदिशेदिति चेत् । न । एतदन्यतोऽपि ज्ञात्वा तदुपदेशसम्भवात् । तस्यान्यतस्तथाविधात् । न चादिमान् संसारः । येन स एवाद्यस्तथा ज्ञानी सर्वज्ञः, स चास्माकमृषभादिरित्यप्युच्येत । अथवा--यद्यसावसर्वज्ञस्तदा तस्य तथाभूतस्य साध्यसाधनभावस्याविदुष उपदेशादस्मदादीनामतीन्द्रियोपरागादिज्ञानं संवादि च कथं भवेदित्याशङ्क्याह--\textbf{नैमित्तिक}मिति ।
	\pend
      

	  \pstart अयमर्थः--कारणदर्शनस्वभावकार्यज्ञानमेतत्; एतच्च भाविवस्तुव्यतिरेकेण भवेदपि, नावश्यं कारणानि कार्यवन्ति भवन्तीति न्यायात् । एतच्च व्यभिचारि ज्ञानमुपदिश्यमानं नोपदेष्टुः सर्वज्ञतामनुमापयितुं कल्प्यते ।
	\pend
      

	  \pstart अथ तदेव तस्य तथाभूतस्य कारणवस्तुनस्तत्तद्भाविवस्तु प्रति कारणत्वं कथं जानीयात्कथं चोपदिशेद् यद्यसावसर्वज्ञ इति चेत् । अन्यतस्तद्धि ज्ञात्वा ज्ञातं \footnote{नं} चोपदेशश्च तस्योप  \leavevmode\marginnote{\textenglish{247/dm}} ““
	  
	सन्दिग्धसाधनव्यतिरेको यथा--न त्रयीविदा ब्राह्मणेन ग्राह्यवचनः कश्चिद् \footnote{कश्चित् पुरुषो \cite{dp-msB} \cite{dp-edP} \cite{dp-edH} \cite{dp-edE}}विवक्षितः पुरुषो रागादिमत्त्वादिति । अत्र वैधर्म्योदाहरणम्--ये ग्राह्यवचना न ते रागादिमन्तः । तद्यथा \footnote{गोतमा० \cite{dp-msD}}गौतमादयो धर्मशास्त्राणां\footnote{धर्मशास्त्रप्रणे० \cite{dp-msC}} प्रणेतार इति । गौतमादिभ्यो रागादिमत्तवस्य साधनधर्मस्य व्यावृत्तिः सन्दिग्धा ॥ १३१ ॥” 
	  
	सन्दिग्धः साधनव्यतिरेको यस्मिन् स तथोक्तः । तमुदाहरति--यथेति । ऋक्सामयजंषि त्रीणि त्रयी तां वेत्तीति\footnote{वेत्ति त्रयी० \cite{dp-msA} \cite{dp-msB} \cite{dp-msD} \cite{dp-edP} \cite{dp-edH} \cite{dp-edE} \cite{dp-edN}} त्रयीवित् । तेन न ग्राह्यं वचनं यस्येति साध्यम् । विवक्षित इति कपिलादिर्धर्मी । रागादिमत्त्वादिति हेतुः । 
	  
	अत्र प्रमाणे वैधर्म्योदाहरणम्--साध्याभावः साधनाभावेन \footnote{०भावेन व्याप्तो यत्र दर्श्यते--\cite{dp-msA} \cite{dp-edP} \cite{dp-edH} \cite{dp-edE}}यत्र व्याप्तो दर्श्यते तद् वैधर्म्योदाहरणम् । ग्राह्यं वचनं येषां ते ग्राह्यवचना इति साध्यनिवृत्तिमनूद्य न ते रागादिमन्त इति साधनाभावो विहितः । गौतम आदिर्येषां ते तथोक्ता मन्वादयो धर्मशास्त्राणि स्मृतयस्तेषां कर्त्तारः त्रयीविदा हि ब्राह्मणेन ग्राह्यवचना धर्मशास्त्रकृतो वीतरागाश्च त इतीह \footnote{इतीह धर्मिव्य० \cite{dp-msC} \cite{dp-msD} इति धर्मी \cite{dp-msA} \cite{dp-edP} \cite{dp-edH} \cite{dp-edN} इति धर्मिव्य० \cite{dp-edE}}धर्मी व्यतिरेकविषयो गौतमादय इति । गौतमादिभ्यो रागादिमत्त्वस्य साधनस्य निवृत्तिः सन्दिग्धा । यद्यपि ते ग्राह्यवचनास्त्रयीविदस्तथापि\footnote{०विदा तथापि \cite{dp-msA} \cite{dp-edP} \cite{dp-edH} \cite{dp-edE} \cite{dp-edN}} किं सरागा उत वीतरागा इति सन्देहः ॥ “
	  
	सन्दिग्धोभयव्यतिरेको यथा--अवीतरागाः कपिलादयः, परिग्रहाग्रह योगादिति । अत्र वैधर्म्येणोदाहरणम्--यो वीतरागो न तस्य परिग्रहाग्रहः ।”” पद्यते । तस्याप्यन्यस्माद् । अनादिश्च संसार इति कथमेतावन्मात्रेण वर्धमानाहेः सर्वज्ञत्वसिद्धिरिति ॥
	\pend
      

	  \pstart त्रयोऽवयवा यस्याः संहतेरिति \textbf{त्रयी}त्यदन्तात् ङीप् । किमत्र साध्या \add{भावा} नुवादेन साधनाभावो विहितो येनैतद् वैधर्म्योदाहरणं भवतीत्याशङ्क्याह--\textbf{साध्ये}ति । \textbf{गौतमोऽक्षपादा}परनामा \textbf{न्यायसूत्रस्या}पि प्रणेता मुनिः । \textbf{मनु}रिति स्मृतिकारो मुनिः । आदिशब्दाद् \textbf{विश्वरूपयाज्ञवल्क्यसंवृत्ता}देः सङ्ग्रहः । इतिस्तस्मात् । धर्मित्वमात्रजिज्ञापिषया \textbf{धर्मी व्यतिरेकविषय} इति अभिधाय तस्यैव विशेषनिष्ठप्रतिपादनेच्छया \textbf{गौतमादय} इत्युक्तम् । \footnote{सन्दिग्धम्--स०}इन्द्रियमनस्कार\add{ज}मेतद्राजा\footnote{मेतज्ज्ञा}नमित्यादिवत् तस्माद् वैधर्मी दृष्टान्तो गौतमादिरिति वाक्यार्थः । तथात्वं च तेषां तथात्वेनोपादानान्न तु परमार्थत इत्यवसेयम् । ग्राह्यवचनत्वेऽपि तेषां कथं ताद्रूप्यसन्देह इत्याह--\textbf{यद्यपी}ति । \textbf{उते}ति पक्षान्तरमुद्द्योतयति ॥
	\pend
      \leavevmode\marginnote{\textenglish{248/dm}}““

	  \pstart यथर्षभादेरिति\footnote{भादेः । ऋष० \cite{dp-msC}} । ऋषभादेरवीतरागत्वपरिग्रहग्रहयोगयोः साध्यसाधनधर्मयोः सन्दिग्धो व्यतिरेकः ॥१३२॥
	\pend
      ” 

	  \pstart सन्दिग्ध उभयोर्व्यतिरेको यस्मिन् स तथोक्तः । तमुदाहरति यथेति । अवीतरागा इति रागादिमत्त्वं साध्यम् । कपिलादय इति धर्मी । परिग्रहो लभ्यमानस्य स्वीकारः प्रथमः । स्वीकारादूर्ध्वं यद् गार्ध्यं मात्सर्यं स आग्रहः । परिग्रहश्च आग्रहश्च, ताभ्यां योगात् । कपिलादयो लभ्यमानं स्वीकुर्वन्ति स्वीकृतं न मुञ्चन्ति--इति ते रागादिमन्तो गम्यन्ते । अत्र प्रमाणे वैधर्म्योदाहरणम्--यत्र साध्याभावे साधनाभावो दर्शयितव्यः । यो वीतराग इति साध्याभावमनूद्य, न तस्य परिग्रहाग्रहाविति साधनाभावो विहितः । यथर्षभादेरिति दृष्टान्तः । एतस्मादृ षभादेर्दृष्टान्ताद् अवीतरागत्वस्य साध्यस्य परिग्रहाग्रहयोगस्य\footnote{योगत्वस्य \cite{dp-msA}} च साधनस्य \footnote{व्यावृत्तिः \cite{dp-msC} \cite{dp-edE}}निवृत्तिः सन्दिग्धा । ऋषभादीनां हि परिग्रहाग्रहयोगोऽपि सन्दिग्धो वीतरागत्वं च । यदि नाम तत्सिद्धान्ते वीतरागाश्च निष्परिग्रहाश्च \footnote{परिपठ्यन्ते \cite{dp-msD}}पठ्यन्ते तथापि सन्देह एव ॥
	\pend
       

	  \pstart अपरानपि \footnote{अपराण्यपि त्रीण्युदा० \cite{dp-msD}}त्रीनुदाहर्त्तुमाह--
	\pend
       “

	  \pstart अव्यतिरेको यथा--अवीतरागोऽयं\footnote{०रागो वक्तृ० \cite{dp-msD} \cite{dp-msB} \cite{dp-edP} \cite{dp-edH} \cite{dp-edE} \cite{dp-edN}} वक्तृत्वात् । वैधर्म्येणोदाहरणम्\footnote{वैधर्म्योदाहर० \cite{dp-msB} \cite{dp-edP} \cite{dp-edH} \cite{dp-edE} \cite{dp-edN} तवात् । यत्रावी० \cite{dp-msC}}--\footnote{यत्र वीत० \cite{dp-msB} \cite{dp-edP} \cite{dp-edH} \cite{dp-edE}}यत्राऽवीतरागत्वं नास्ति, \footnote{नास्ति स वक्ता \cite{dp-msB} \cite{dp-edP} \cite{dp-edH} \cite{dp-edE} नास्ति स न वक्ता \cite{dp-msC}}न स वक्ता । यथा--उपलखण्ड इति । यद्यप्युपलखण्डादुभयं व्यावृत्तं \footnote{व्यावृत्तया सर्वो--\cite{dp-msB} \cite{dp-edP} \cite{dp-edH} व्यावृत्तं यो सर्वो० \cite{dp-edE} ०वृत्तं तथा सर्वो० \cite{dp-msC}}तथापि सर्वो वीतरागो न वक्तेति व्याप्त्या व्यतिरेकासिद्धेरव्यतिरेकः ॥ १३३ ॥
	\pend
      ””

	  \pstart उभयश\footnote{उभश}ब्दस्य द्विवचनान्तस्य प्रयोगदर्शनादुभयोरित्युभशब्देनार्थमाह । लब्धमिदं वस्तु मत्तोऽन्यत्र नरामदि \footnote{मागादि}ति तु विशेषोऽत्र मात्सर्यमभिप्रेत \textbf{आग्रहः} । न मुञ्चति\footnote{न्ति} नान्यस्मै ददति । अनेनैव रूपेण वैधर्म्योदाहरणं भवति । नान्यथेति द्रढयितुमुक्तमपि स्मारयन्नाह--\textbf{यत्रे}ति । \textbf{ऋषभादीनामि}त्यनेन वैधर्म्योदाहरणाद् \textbf{ऋषभादेः} साध्यसाधनयोर्व्यावृत्तिसन्देहं दर्शयति ।
	\pend
      

	  \pstart नन्वस्मदागमे तद्गुणद्वययोगिनस्ते कथितास्तत्कथमनयोस्ततो व्यावृत्तिः सन्दिह्यत इत्याह--\textbf{यदि नामे}ति । \textbf{पठ्यन्त} इति च वचनव्यक्त्या च पाठमात्रेण तेषां तद्गुणयोगः सिद्धः न तु प्रमाणेनेति दर्शयति । अत एवाह--त\textbf{थापी}ति ॥
	\pend
      

	  \pstart \textbf{त्रीनि}ति दृष्टान्तदोषान् ।
	\pend
      \leavevmode\marginnote{\textenglish{249/dm}}“

	  \pstart अविद्यमानो व्यतिरेको यस्मिन् सोऽव्यतिरेकः । अवीतराग इति रागादिमत्त्वं साध्यम् । वक्तृत्वादिति हेतुः । इह व्यतिरेकमाह--यत्रावीतरागत्वं नास्तीति साध्याभावानुवादः । तत्र वक्तृत्वमपि नास्ति--इति साधनाभावविधिः । तेन साधनाभावेन साध्याभावो व्याप्त उक्तः । \footnote{अत्र दृष्टा० \cite{dp-msD}}दृष्टान्तो यथोपलखण्ड \footnote{०खण्डेति० \cite{dp-msA} \cite{dp-msB} \cite{dp-edP} \cite{dp-edH}}इति ।
	\pend
       

	  \pstart कथमव्यमव्यतिरेको यावतोपलखण्डादुभयं\footnote{०भयमपि नि० \cite{dp-msD}} निवृत्तम् ? किमतः ?\footnote{यद्युपल० \cite{dp-msA} \cite{dp-msB} \cite{dp-edP} \cite{dp-edH} \cite{dp-edE}} यद्यपि उपलखण्डादुभयं व्यावृत्तं सरागत्वं च वक्तृत्वं च\footnote{यत्र वीतरागत्वं तत्र वक्तृत्वं नास्ति--\cite{dp-msD-n}}, तथापि व्याप्त्या\footnote{व्याप्तो व्य० \cite{dp-msC}} व्यतिरेको यस्तस्याऽसिद्धेः कारणाद् अव्यतिरेकोऽयम् ।
	\pend
       

	  \pstart कीदृशी पुनर्व्याप्तिरित्याह--सर्वो वीतराग इति साध्याभावानुवादः । न वक्तेति साधनाभावविधिः । तेन साध्याभावः साधनाभाव\footnote{साधनाभावे नि० \cite{dp-msD} साधनाभावो नि० \cite{dp-msC}}नियतः \footnote{स्थापितो--\cite{dp-msA}}ख्यापितो भवतीति\footnote{भवति । ई० \cite{dp-msD} \cite{dp-msB}} । इदृशी व्याप्तिः । तया व्यतिरेको न सिद्धः । अस्य चार्थस्य प्रसिद्धये दृष्टान्तः । तत् स्वकार्या ऽकरणाद् दुष्टः ॥
	\pend
       “

	  \pstart अप्रदर्शितव्यतिरेको यथा--अनित्यः शब्दः, कृतकत्वादाकाशवदिति वधर्म्येण\footnote{नास्ति “वैधर्म्येण” \cite{dp-edE} वैधर्म्येणापि \cite{dp-msB} \cite{dp-msD} \cite{dp-edP} \cite{dp-edH} न मूलत्वेनापि तु टीकास्थं गृहीतं \cite{dp-edN} प्रतौ ।} ॥१३४॥
	\pend
      ””

	  \pstart येनायमनुवादविधिक्रमस्तेन हेतुना । \textbf{यावते}ति तृतीयान्तप्रतिरूपको यस्मादित्यस्यार्थे वर्त्तमानोऽत्र गृहीतः । उ\leavevmode\marginnote{\textenglish{81b/ms}}\textbf{पलखण्डा}च्छिलाशकलात् । \textbf{उभ}यं साध्यसाधनम् । \textbf{किमत} इति सिद्धान्ती । अत उभयनिवृत्ते किं भवति ? न किञ्चिदित्यर्थः । ननूपलखण्डात्तावद् वैधर्मीदृष्टान्तादुभयं निवृत्तं दर्शितं येन तत्किमेवमुच्यत इत्याह--\textbf{यद्यपी}ति । \textbf{व्याप्त्या} सर्वरागित्वजन्यतास्वीकारेण । तस्य व्याप्तिमतो व्यतिरेकस्याऽ\textbf{सिद्धे}रनिश्चयात् । \textbf{कीदृशी}ति सामान्यतः पृच्छति । \textbf{पुन}रिति विशेषतः । \textbf{इति}रनन्तरोक्तं शब्दं परामृशति । \textbf{तेनाने}न शब्देनेत्यर्थः । न वक्तव्येत्यत्रापीति पूर्ववत् । येनैवमनुवादविधिक्रम\textbf{स्तेन । इति}रीदृश्या व्याप्तेः स्वरूपं प्रकाशयति यस्मादर्थे वा । तत्प्रतीत्थंभूतलक्षणतयेयं तृतीया, साधनाभावेन साध्याभावन्यायेन लक्षणा व्याप्तिरीदृशीत्यर्थः । ईदृशं व्याप्तिमन्तं व्यतिरेकं प्रसाधयितुमसमर्थोऽयं कथमयं दृष्टान्तोऽत्रेष्ट इत्याह--\textbf{अस्य चे}ति । \textbf{अस्य} साधनाभावे साध्याऽभावनियतलक्षणस्या\textbf{र्थस्य । चो} यस्मादर्थेऽवधारणे वा । \textbf{प्रसिद्धये} निश्चयार्थं दृ\textbf{ष्टान्त} उपादीयत इति शेषः, प्रकरणलभ्यं वा, न चायं तथाप्रदर्शन इत्यमुमर्थं प्रकरणगम्यं कृत्वा । \textbf{तत्त}स्मात् \textbf{स्वकार्याकरणाद् दुष्ट} इत्युक्तम् । यद् वा भवत्वस्यार्थस्य प्रसिद्धये          \leavevmode\marginnote{\textenglish{250/dm}} “
	  
	अप्रदर्शितो व्यतिरेको यस्मिन् स तथोक्तः । अनित्यः शब्द इत्यनित्यत्वं साध्यम् । कृतकत्वादिति हेतुः । आकाशवदिति वैधर्म्येण दृष्टान्तः । 
	  
	इह परार्थानुमाने परस्मादर्थः प्रतिपत्तव्यः । स शुद्धोऽपि स्वतो यदि परेणाशुद्धः ख्याप्यते स तावद्यथा प्रकाशितस्तथा न युक्तः । यथा युक्तस्तथा न प्रकाशितः । प्रकाशितश्च हेतुः । अतो वक्तुरपराधादपि परार्थानुमाने हेतुर्दृष्टान्तो वा दुष्टः स्यादपि । न च सादृश्यावसादृश्याद्वा साध्यप्रतिपत्तिः, अपि तु साध्यनियताद्धेतोः । अतः साध्यनियतो हेतुरन्वयवाक्येन व्यतिरेकवाक्येन \footnote{च वक्त० \cite{dp-msA} \cite{dp-msB} \cite{dp-edP} \cite{dp-edH} \cite{dp-edE} \cite{dp-edN}}वा वक्तव्यः । अन्यथा गमको नोक्तः स्यात् । स तथोक्तो दृष्टान्तेन\footnote{दृष्टान्तेनासिद्धो \cite{dp-msB}}” दृष्टान्तस्तथाप्ययं कथं दुष्ट इत्याह--\textbf{तत्स्वकार्ये}ति । तच्च \textbf{तत्स्वकार्यं} च । \textbf{साधनाभावे} साध्याभावनियतख्यापनलक्षणं चेति । तथा तस्या\textbf{करणाद}सम्पादनाद् \textbf{दुष्ट} इति ॥
	\pend
      

	  \pstart अथ परमार्थतस्तावद् दृष्टान्ते नभसि साध्याभावोऽप्यस्ति, साधनाभावश्च । तत्कथमप्रदर्शितव्यतिरेको दृष्टान्तो दुष्ट इत्याह--\textbf{इहे}ति । \textbf{परस्मा}त्साधनवादिनः । \textbf{अर्थो} हेतुलक्षणः । प्रकरणात्साध्याभावे साधनाभावलक्षणश्च । \textbf{स स्वतः शुद्धो} वस्तुवृत्त्या परिशुद्धः । तथात्वेन विद्यमान इति यावत् । न केवलमशुद्धः--इत्यपिशब्दात् । \textbf{परेण} साधनप्रयोक्त्रा । \textbf{यथा प्रकाशितो} व्यतिरेकमात्रवान् प्रकाशितो व्याप्तिशून्यश्च प्रकाशितः । \textbf{तथा न युक्तो} नोपयुक्तः साध्यसिद्धौ । यदि हि साध्याभावानुवादेन साध्या\footnote{धना}भावो विधीयते दृष्टान्ते प्रदर्श्येतैवमसौ हेतुः साध्यसिद्ध्यङ्गव्यतिरेकवान् सिद्ध्येत् । एवमेव चाऽसौ व्याप्तिमद्व्यतिरेकः प्रसिद्ध्येत् । तत्प्रदर्शनश्च दृष्टान्तोऽदुष्टो भवेदित्यभिप्रायः ।
	\pend
      

	  \pstart यदि नाम तथा न प्रकाशितस्तथापि तदुपयोगी हेतुर्दृष्टान्तो वा तथा किं न प्रतिपद्यत इत्याह--\textbf{प्रकाशितश्चे}ति । \textbf{चो} यस्मादर्थे । \textbf{हेतुरि}त्युपलक्षणम् । तेन दृष्टान्तोऽपि द्रष्टव्यः । अ\footnote{य}त एवम् \textbf{अतो} अस्माद् हेतोः । न केवलं वचोव्यवस्थिताद् दोषादित्यपिशब्देनाह । यद्यपि दृष्टान्त एव प्रकृतस्तथापि हेतुरप्येवंविधः स्वतोऽदुष्टोऽपि वक्तृदोषादेव दुष्यतीति तुल्यन्यायतया प्रसङ्गन दर्शितम् । यद्वा यथैवंविधो हेतुर्वक्तृदोषाद् दुष्टो भवति तद्वद् दृष्टान्तोऽपीति दृष्टान्तार्थं हेतोः वक्त्रपराधेना\footnote{न} दुष्टत्वख्यापनं कृतमिति सर्वमवदातम् ।
	\pend
      

	  \pstart \leavevmode\marginnote{\textenglish{82a/ms}} ननु च यथा कृतकत्वेनाकाशविधर्मा शब्दः प्रतीयते तथाऽनित्यत्वेनापि तद्विधर्मा भविष्यति तत्कथमनुपयुक्त इत्याह--\textbf{न चे}ति । \textbf{चो}ऽवधारणे हेतौ वा । एवंवदतोऽयमाशयः—यद्येकेन धर्मेण वैधर्म्य\footnote{र्म्ये} प्रतीतेऽपरेणापि तद्वैधम्यप्रतीतिरवश्यम्भाविनी, तदा मूर्त्तत्वेनापि शब्दस्य तद्वैधर्म्यप्रतीतिः प्रसज्येतेति । तुल्यन्यायतयाऽन्वयवाक्यमधिकृत्य \textbf{सादृश्यादि}त्युक्तम् । \textbf{सादृश्यादसादृश्याद्} वेति साधर्म्यवैधर्म्यदृष्टान्तप्रतिपादितादिति प्रकरणात् ।
	\pend
      

	  \pstart व्यतिरेकवाक्येनापि साधनाभावेनापि नियमख्यापनद्वारा साध्य एव हेतोर्नियतत्वख्यापनाद् \textbf{व्यतिरेकवाक्येन चे}त्युक्तम् ।
	\pend
      \leavevmode\marginnote{\textenglish{251/dm}}“

	  \pstart सिद्धो दर्शयितव्यः । तस्माद् दृष्टान्तो नामान्वयव्यतिरेकवाक्यार्थप्रदर्शनः\footnote{प्रदर्शनार्थे । \cite{dp-msC} \cite{dp-msD}} । न चेह व्यतिरेकवाक्यं प्रयुक्तम् । अतो वैधर्म्यदृष्टान्त इहासादृश्यमात्रेण\footnote{सादृश्यभावेन साध० \cite{dp-msA} \cite{dp-msB} \cite{dp-edP} \cite{dp-edH} \cite{dp-edN}} साधक उपन्यस्तः । न च तथा साधकः । व्यतिरेकविषयत्वेन स साधकः । च न तथोपन्यस्त इति\footnote{इति । अतोऽप्र० \cite{dp-msA} \cite{dp-edP} \cite{dp-edH} \cite{dp-edE} \cite{dp-edN} इति । अप्र० \cite{dp-msC}} अयमप्रदर्शितव्यतिरेको वक्तुरपराधाद् दुष्टः ॥
	\pend
       “

	  \pstart विपरीतव्यतिरेको \footnote{“यथा” नास्ति \cite{dp-msD}}यथा--यदकृतकं तन्नित्यं भवतीति ॥१३५॥
	\pend
      ” 

	  \pstart \footnote{“विपरीतो” इत्यारभ्य “तमुदाहरति” पर्यन्तः पाठो दुर्वेकसमीपस्थैकस्मिन्नादर्शे नासीदिति व्याख्यानुरोधात् ज्ञायते--सं०}विपरीतो व्यतिरेको यस्मिन् वैधर्म्यदृष्टान्ते स तथोक्तः । तमुदाहरति--\footnote{“यथा” नास्ति \cite{dp-msA} \cite{dp-msB} \cite{dp-edP} \cite{dp-edH} \cite{dp-edE} \cite{dp-edN}}यथा यद-
	\pend
      ”

	  \pstart अन्वयवाक्ये साधनमनूद्य साध्यं विधातव्यम् । व्यतिरेकवाक्ये च साध्याभावमनूद्य साधनाभावो विधातव्यः । तथैव हेतोः साध्यनियतत्वाभिधानादित्यस्यार्थः ।
	\pend
      

	  \pstart कस्मादसौ वाक्यद्वयेनाप्येव वक्तव्य इत्याह--\textbf{अन्यथेति} । अस्मादन्येन प्रकारेण \textbf{गमकः} परोक्षार्थप्रकाशको \textbf{नोक्तः स्यात्} ।
	\pend
      

	  \pstart काममसावेवमुच्यताम् । दृष्टान्तस्तु कथमत्राधिक्रियत इत्याह--\textbf{स} इति । \textbf{स} हेतु\textbf{स्तथोक्तः} साध्यनियत उक्तः । \textbf{दृष्टान्तेन} साधर्म्यवता वैधर्म्यवता च करणेन \textbf{सिद्धो निश्चितो दर्शयितव्यः} ।
	\pend
      

	  \pstart नन्वेवमपि न ज्ञायते किंव्यापारो दृष्टान्त इहोपयुज्यते इत्याशङ्क्योपसंहारव्याजेनाह—\textbf{तस्मादि}ति । \textbf{नाम}शब्दः प्रसिद्धाविह । \textbf{अन्वयव्यतिरेकवाक्य}योरथा \footnote{र्थोऽ} भिधेयः--\textbf{उक्तेन} प्रकारेण हेतोः साध्यनियतत्वम्--प्रदर्श्यः । तं प्रदर्शयतीति तथा । यद्वा प्रदर्श्यतेऽनेनेति प्रदर्श्यतेऽस्मिन्निति \textbf{प्रदर्शनः} । तस्य \textbf{प्रदर्शन} इति विग्रहः ।
	\pend
      

	  \pstart यद्येवमयमपि वैधर्म्यदृष्टान्तस्तथाकार्येवात्रोपयोज्यत इत्याह--\textbf{न चेति । चो} यस्मादर्थे । \textbf{इह} प्रयोगे । \textbf{व्यतिरेक}ख्यापकं साध्याभावानुवादेन साधनाभावविधायकं \textbf{वाक्}यमित्यर्थः । अतस्तथाभूतवाक्यप्रयोगाद वैधर्म्यदृष्टान्त आकाशः । \textbf{इहा}नित्यत्वसाधनप्रयोगे साध्यधर्मिणोऽ\textbf{सादृश्यं} केवलं यत्तन्मात्रेण तन्मात्रप्रदर्शनेन, विषयेण विषयिणो निर्देशात् । \textbf{साधको} निश्चायको हेतोः साध्यनियतत्वस्येति प्रकरणात् ।
	\pend
      

	  \pstart यदि तन्मात्रेणापि साधकस्तदा का क्षतिरित्याह--\textbf{न चे}ति । \textbf{तथे}त्यसादृश्यमात्रेण । वाद्युक्तेन धर्मेण साध्यधर्मिणोऽसादृश्यावगमे धर्मान्तरेणाप्यसादृश्यावगमोऽवश्यम्भावीति युज्यतेऽतिप्रसङ्गादित्यभिप्रायः ।
	\pend
      \leavevmode\marginnote{\textenglish{252/dm}}“

	  \pstart कृतकमित्यादि । इहान्वय\footnote{व्यतिरेकवाक्याभ्यां \cite{dp-msA} \cite{dp-msB} \cite{dp-msD} \cite{dp-edP} \cite{dp-edH} \cite{dp-edE} \cite{dp-edN}} व्यतिरेकाभ्यां साध्यनियतो हेतुर्दर्शयितव्यः । यदा च साध्यनियतो हेतुर्दर्शयितव्यस्तदा व्यतिरेकवाक्ये साध्याभावः साधनाभावे नियतो दर्शयितव्यः । एवं हि हेतुः साध्यनियतो दर्शितः स्यात् । यदि तु साध्याभावः साधनाभावे नियतो नाख्यायते साधनसत्तायामपि साध्याभावः सम्भाव्येत\footnote{सम्भाव्यते \cite{dp-msB}} । तथा च साधनं साध्यनियतं न प्रतीयेत\footnote{प्रतीयते \cite{dp-msB} \cite{dp-msC}} । तस्मात् साध्याभावः साधनाभावे नियतो वक्तव्यः । विपरीतव्यतिरेके च साधनाभावः साध्याभावे नियत उच्यते । न साध्याभावः साधनाभावे । तथा हि--यदकृतकमिति साधनाभावमनूद्य तन्नित्यमिति साध्याभावविधिः ।
	\pend
       

	  \pstart ततोऽयमर्थः--अकृतको नित्य एव । तथा च सति अकृतकत्वं नित्यत्वे साध्याभावे नियतमुक्तम्, न नित्यत्वं साधनाभावे । ततो न साध्यनियतं हेतुं व्यतिरेकवाक्यमाह । तथा च विपरीतव्यतिरेकोऽपि वक्तुरपराधाद् दुष्टः ॥
	\pend
      ”

	  \pstart यद्येवमसाधकः कथं नाम साधक इत्याह--\textbf{व्यतिरेके}ति । \textbf{व्यतिरेकवि}षयत्वेनेति व्यतिरेकप्रतिपत्तिविषयत्वेन । \textbf{चो} यस्मादर्थे, व्यक्तमेतदित्यस्मिन्नथे वा । अनेनापि तथैवोपन्यस्त इत्याह--\textbf{न चे}ति । \textbf{चो}ऽवधारणे । व्यतिरेकवाक्यमनुक्त्वैव तस्योपादानादित्यभिप्रायः । \textbf{इति}स्तस्मादथे\unclear{} एवमर्थे\unclear{} वा ॥
	\pend
      

	  \pstart विपरीतव्यतिरेकं व्याचक्षाण आह--\textbf{यदा चे}ति । विपरीतान्वयश \leavevmode\marginnote{\textenglish{82b/ms}} ब्दस्य व्युत्पत्तौ दर्शितायां विपरीतव्यतिरेकशब्दस्यापि--\textbf{विपरीतो} वैपरीत्येन प्रदर्शनाद् \textbf{व्यतिरेको यस्मिन् दृष्टान्ते स तथोक्त} इति--व्युत्पत्तिर्दर्शिता भवत्येवेति चाभिप्रायेण नोक्ता । \textbf{तमुदाहरती}ति सुज्ञानत्वान्नोक्तमिति प्रतिपत्तव्यम् । यत्र तु पुस्तके \textbf{विपरीतव्यतिरेको यथे}त्यस्य मूलस्य व्याख्यानग्रन्थोऽस्ति तत्र सर्वमवदातम् । असतीयं गतिरस्माभिर्दर्शिता ।
	\pend
      

	  \pstart ननु किं नाम व्यतिरेकवाक्येन दर्शनीयम् ? यद्वैपरीत्येन दर्शनादयं विपरीतव्यतिरेक उच्यत इत्याह--\textbf{यदा} चेति । \textbf{चो}ऽवधारणे । \textbf{साध्यनियत} इत्यस्मात्परः प्रतिपत्तव्यः । \textbf{यदा} यस्मिन् काले प्रतिपादनकाल इत्यर्थात् । कालान्तरे तथाप्रदर्शनानुपपत्तेः । हेतोः साध्ये नियतत्वप्रदर्शनञ्चावश्यकार्यमन्यथा गमको नोक्तः स्यादित्यभिप्रायः । \textbf{तदा} तस्मिन् काले । साध्याभावानुवादेन साधनाभावोऽभिधातव्य इत्यस्यार्थः । कस्मात्पुनरेवं दर्शयितव्य इत्याह—\textbf{एवं ही}ति । \textbf{ही}ति यस्मात् । \textbf{एवं} साध्याभावस्य साधनाभावे नियतत्वप्रदर्शनप्रकारे सति ।
	\pend
      

	  \pstart अथान्यथाप्रदर्शनेऽपि यदि साधनं साध्यनियतं प्रतीयते तदा तथाप्रदर्शनेनैव किं प्रयोजनमित्याह--\textbf{यदि} त्विति । तुस्तथाऽनाख्यानावस्थां भेदवतीं दर्शयति । \textbf{तथा च} साधनसत्तायामपि साध्याभावसम्भावनाप्रकारे सति । यस्मादेवं \textbf{तस्मादि}त्युपसंहारः । अस्मिन् प्रयोगे किं नामोच्यत इत्याह--\textbf{विपरीते}ति । तुशब्दार्थश्चकारः । \textbf{तथा ही}त्यादिनैतदेव प्रतिपादयति । यत एवमनुवादविधिस्ततः । \textbf{तथा चा}कृतकस्य नित्यत्वोक्तिप्रकारे सति ।   \leavevmode\marginnote{\textenglish{253/dm}} “
	  
	दृष्टान्तदोषानुदाहृत्य दुष्टत्वनिबन्धनत्वं दर्शयितुमाह-- “
	  
	न ह्येभिर्दृष्टान्ताभासैर्हेतोः सामान्यलक्षणं सपक्ष एव सत्वं विपक्षे च सर्वत्रासत्वमेव निश्चयेन शक्यं दर्शयितुं विशेषलक्षणं वा । तदर्थापत्यैषां निरासो \footnote{निरासो वेदितव्यः \cite{dp-msB} \cite{dp-msD} \cite{dp-edP} \cite{dp-edH} \cite{dp-edE} \cite{dp-edN}}द्रष्टव्यः ॥१३६॥” 
	  
	नह्येभिरिति । साध्यनियतहेतुप्रदर्शनाय हि दृष्टान्ता वक्तव्याः । एभिश्च हेतोः सपक्ष एव सत्त्वं विपक्षे च सर्वत्रासत्त्वमेव यत् सामान्यलक्षणं तत् निश्चयेन न शक्यं दर्शयितुम् । 
	  
	ननु च सामान्यलक्षणं विशेषनिष्ठमेव प्रतिपत्तव्यं न स्वत एवेत्याह--विशेषलक्षणं वा । यदि विशेषलक्षणं प्रतिपादयितुं शक्येत स्यादेव सामान्यलक्षणप्रतिपत्तिः । विशेषलक्षणमेव तु न शक्यमेभिः प्रतिपादयितुम् । तस्मादर्थापत्त्या सामर्थ्येन\footnote{सामर्थ्येनेति न तेषां \cite{dp-msA} \cite{dp-msB} \cite{dp-edP} सामर्थ्येनेति तेषां \cite{dp-edE} \cite{dp-edH} सामर्थ्येन तेषां \cite{dp-msD}} एषां निराकरणं द्रष्टव्यम् । साध्यनियतसाधनप्रतीतये\footnote{प्रतिपत्तये उपा० \cite{dp-msB}} उपात्ताः । तदसमर्था दुष्टाः, \footnote{स्वकार्यकरणात् \cite{dp-msA} \cite{dp-msB} \cite{dp-edP} \cite{dp-edH}}स्वकार्याकरणादिति \footnote{“असामर्थ्यम्” इति संशोधितं \cite{dp-msC} \cite{dp-msD} प्रतयोः । इति सामर्थ्यम्--\cite{dp-msA} \cite{dp-msB} \cite{dp-edP} \cite{dp-edH} \cite{dp-edE} \cite{dp-edN}}असामर्थ्यम् । इयता साधनमुक्तम् ॥” \textbf{अकृतकत्वं} कृतकत्वस्य साधनस्याभावः । \textbf{नित्यत्वे}ऽनित्यत्यत्वलक्षणसाध्याभावे । \textbf{न नित्यत्वं} साध्याभावलक्षणं \textbf{साधनाभावे} कृतकत्वलक्षणसाधनाभावेऽकृतकत्व इत्यर्थात् । यत एवं \textbf{ततो} हेतो\textbf{र्व्यतिरेकवाक्यं} कर्त्तृ \textbf{हेतुं} कर्मभूतं \textbf{न साध्यनियतमाह} । उक्तया नीत्या साधनाभावः साध्याभावे नियतत्वात्तमन्तरेण न भवेत् । न तु यत्र साध्याभावस्तत्रावश्यं साधनाभाव इति साध्यमन्तरेणापि साधनं भवेत् । ततश्च साध्यानियतं साधनमित्यभिप्रायः । \textbf{तथा च} साध्यनियतहेत्वप्रदर्शनप्रकारे सति \textbf{वक्तुरे}वंवाक्यप्रयोक्तुः ॥
	\pend
      

	  \pstart \textbf{निश्चयेना}वश्यन्तया । \textbf{विशेषनिष्ठमेव प्रत्येतव्यमि}ति ब्रुवतोऽयं भावः--यदि नामामीभिः सपक्ष एव सत्त्वं विपक्षे सर्वत्रासत्त्वं निश्चयेन शक्यते दर्शयितुम्, तथाप्येते विशेषलक्षणं सामान्यलक्षणप्रतिपत्त्यङ्गं प्रतिपादयन्त उपयोक्ष्यन्त इति । अत्र \textbf{विशेषलक्षणश्चे}त्युत्तरं \textbf{यदीत्या}दिना व्याचष्टे । यतः सामान्यलक्षणं विशेषलक्षणं वा न शक्यमेभिर्दर्शयितुम् । \textbf{तस्मात्} कारणात् । \textbf{अर्थापत्त्ये}त्यस्य व्याख्यानम् \textbf{सामर्थ्येन} सामान्यविशेषलक्षणाप्रतिपादनलक्षणन । \textbf{एषां} दृष्टान्ताभासानां \textbf{निराकरणं} दृष्टान्तरूपत्वेनेत्यर्थात् ।
	\pend
      

	  \pstart कथममी दुष्टा येन तथात्वेन निराकरणमेषामित्याशङ्क्योपसंहरन्नाह--\textbf{साध्ये}ति । \textbf{तदसमर्था}स्तदप्र \footnote{तत्प्र} तीतिकारणाशक्ताः । असामर्थ्यमेव कथं येनासाम \leavevmode\marginnote{\textenglish{83a/ms}}र्थ्याद् दुष्टा उच्यन्त इत्याह--\textbf{स्वकार्य}स्य हेतोः साध्यनियतत्वप्रदर्शनलक्षणस्या\textbf{करणात्} । ननु तदकरणमेवासामर्थ्यमुक्तमिति चेत् । सत्यम् । केवलमसामर्थ्यव्यवहारापेक्षयैवमुक्तमित्यवसेयम् । इति स्तस्मादर्थे, एवमर्थे वा । \textbf{असामर्थ्यमेषा}मित्यर्थात् । साध्यादिविकलस्यानन्वयाप्रदर्शितान्वया    \leavevmode\marginnote{\textenglish{254/dm}} “
	  
	दूषणं वक्तुमाह-- “
	  
	दूषणा \footnote{न्यूनतायुक्तिः--\cite{dp-msB} \cite{dp-edP} \cite{dp-edH} दूषणानि न्यू० \cite{dp-edE}}न्यूनताद्युक्तिः ॥ १३७ ॥” 
	  
	\footnote{दूषणानि कानि द्रष्टव्यानि--\cite{dp-edE}}दूषणा का द्रष्टव्या ? न्यूनतादीनामुवितः । उच्यतेऽनयेत्युवितर्वचनम् न्यूनतादे\footnote{न्यूनतादिर्व \cite{dp-msA} \cite{dp-msB} \cite{dp-edP} \cite{dp-edH} न्यूनतादिवच० \cite{dp-edE} \cite{dp-edN}}र्वचनम् ॥ 
	  
	दूषणं विवरीतुमाह-- “
	  
	ये पूर्वं न्यूनतादायः साधनदोषा उक्तास्तेषामुद्भावनं दूषणम् । तेन परेष्टार्थसिद्धिप्रतिबन्धात् ॥१३८॥” 
	  
	ये पूर्वं न्यूनतादयोऽसिद्धविरुद्धानैकान्तिका उक्तास्तेषामुद्भावकं यद्वचनं तद् दूषणम् । 
	  
	ननु च न्यूनतादयो न विपर्ययसाधनाः । तत् कथं दूषणमित्याह--तेन न्यूनतादिवचनेन परेषामिष्टश्चासावर्थश्च तस्य सिद्धिः निश्चयस्तस्याः प्रतिबन्धात् । नावश्यं विपर्ययसाधनादेव दूषणं विरुद्धवत् । अपि तु परस्याभिप्रेतनिश्चय\footnote{०निश्चयनिबन्ध० \cite{dp-msA} \cite{dp-msB} \cite{dp-msC} \cite{dp-msD} \cite{dp-edP} \cite{dp-edH} \cite{dp-edN} निश्चयप्रतिबन्ध० \cite{dp-edE}}विबन्धान्निश्चयाभावो भवति निश्चयविपर्यय इत्यस्त्येव\footnote{इत्यस्ति विप० \cite{dp-msA} \cite{dp-edP} \cite{dp-edH} \cite{dp-edE}} विपर्ययसिद्धिरिति । \footnote{उक्तंदूषणम् \cite{dp-edE}}उक्ता \footnote{नास्ति “दूषणा” \cite{dp-msA} \cite{dp-msB} \cite{dp-edP} \cite{dp-edH}}दूषणा ॥” देरपि दृष्टान्ताभासस्यासाधनाङ्गवचनाद् वादिनो निग्रहोऽसामर्थ्योपादानान्न्यायप्राप्तः । तस्मादेवंविधाः स्ववाक्ये वर्जनीयाः । परोपात्ताश्च चोदनीया इति दृष्टान्ताभासव्युत्पादने \textbf{वार्त्तिककृतो}ऽभिप्रायः प्रत्येतव्य इति ।
	\pend
      

	  \pstart सम्प्रति सुखग्रहणार्थमुक्तप्रबन्धस्याचार्यीयस्य प्रतिपादितमवच्छिन्दन्नाह--इयतेति \textbf{त्रिरूपलिङ्गाख्}यानमित्यादिनैतदन्तेन, इदं परिमाणमस्ये\textbf{तीयत् तेनेयता} महावाक्येन \textbf{साधनमुक्त}माचार्येणेति शेषः । प्रसङ्गागतस्यानेकस्यापि तदभिधानमतिवृत्तम्, तदपि साधनप्रतिपादन एव साक्षात्परम्परया वा समुपयुक्तम् । साधनाभासाभिधानमपि तस्यैव स्फुटावगमार्थं तत्रैवोपयुक्तमिति मन्यमानेनोक्त\textbf{मियता साधनमुक्तमि}ति ।
	\pend
      

	  \pstart इति भूतदूषणोद्भावना \textbf{दूषणा} । दुषेर्णिजन्ताद् युचंकृत्वा टाप्कर्त्तव्यः । एतच्च \textbf{तददूषणमि}त्यन्तं सुबोधम् ।
	\pend
      

	  \pstart विपरीतसाधनस्यैव दूषणत्वात्कथं न्यूनताद्युक्ति \footnote{क्ते} र्दूषणत्वमित्यभिप्रेत्याह--\textbf{ननु चे}ति । अत्र \textbf{तेने}त्याद्युत्तरं व्याचक्षाण आह--\textbf{तेने}त्यादि ।
	\pend
      \leavevmode\marginnote{\textenglish{255/dm}}““

	  \pstart दूषणाभासास्तु जातयः ॥ १३९ ॥
	\pend
      ” 

	  \pstart दूषणाभासा इति । \footnote{दूषणावत् \cite{dp-msB}}दूषणवदाभासन्त इति दूषणाभासाः । के ते ? जातयः \footnote{“इति” नास्ति \cite{dp-msA} \cite{dp-msB} \cite{dp-edP} \cite{dp-edH} \cite{dp-edE} \cite{dp-edN}}इति । जातिशब्दः सादृश्यवचनः । उत्तरसदृशानि जात्युत्तराणि\footnote{०राणीति \cite{dp-msA} \cite{dp-edP} \cite{dp-edH} \cite{dp-edE}} । उत्तरस्थानप्रयुक्तत्वाद् उत्तरसदृशानि जात्युत्तराणि ॥
	\pend
       

	  \pstart तदेवोत्तरसादृश्यमुत्तरस्थानप्रयुक्त्वेन दर्शयितुमाह--
	\pend
       “

	  \pstart \footnote{अनुभूत \cite{dp-msB} \cite{dp-edP} \cite{dp-edH}}अभूतदोषोद्भावनानि जात्युत्तराणीति ॥ १४० ॥
	\pend
      ” 

	  \begin{center}%% label @type='head'
	\textbf{॥ [[“तृतीयपरिच्छेदः समाप्तः” इति नास्ति \cite{dp-msC} \cite{dp-msD}]]तृतीयपरिच्छेदः समाप्तः ॥}
	\end{center}
	 

	  \begin{center}%% label @type='head'
	\textbf{॥ न्यायबिन्दुः समाप्तः ॥ लघुधर्मोत्तरसूत्रं समाप्तमिति ॥}
	\end{center}
	”

	  \pstart ननूक्तं विपरीतसाधनं दूषणम् । तत्कथं परेष्टार्थसिद्धिप्रतिबन्धकस्यापि न्यूनतादिवचनस्य तथात्वमुच्यत इत्याशङ्कामपाकुर्वन्नाह--\textbf{नावश्यमिति} । किन्त्वपरस्य सिसाधयिषितार्थ\textbf{निश्चयविबन्धा}द् अप्यत्रार्थाद् द्रष्टव्यम् । अन्यथाऽ\textbf{वश्यं} ग्रहणमवधारणं \textbf{विरुद्धवदि}त्यपि दुर्योजं स्यात् । मृत्वा शीर्त्वा च तद्योजने वक्तुरकौशलं स्यादिति ।
	\pend
      

	  \pstart यदि त्ववश्यं विपर्ययसाधनत्वस्यैव दूषणत्वमिति निर्बन्धस्तदा तदप्यस्य न्यूनतादिवचनस्यास्तीति दर्शयन्नाह--\textbf{निश्चये}ति । वाशब्दः पक्षान्तरमवद्योतयति । इतिस्तस्मा\textbf{दस्त्येव} ॥
	\pend
      

	  \pstart सादृश्यार्थवृत्तेरपि जातिशब्दस्य दर्शना\textbf{ज्जातिशब्दः सादृश्यवचन} इत्याह ।
	\pend
      

	  \pstart ननु जातिशब्दः सादृश्यवचनत्वादस्त्वयमर्थः--जातयः सदृशा इति । कानि पुनस्तानि केन च सदृशानीति न ज्ञायत इत्याशङ्कामपाकुर्वन्नाह--\textbf{उत्तरे}ति । एतच्चाचार्येणैव विवरणे स्पष्टीकृतमिति मन्यते । तदयमर्थः-जातिशब्देन जात्युत्तरमेवात्र \textbf{वार्त्तिकका}रस्य विवक्षितमिति ।
	\pend
      

	  \pstart कथं पुनर्दूषणाभासानां जात्युत्तरशब्दवाच्यत्वमित्याशङ्क्याह--\textbf{उत्तरे}ति । लक्ष्यते चायमाचार्यस्याशयो यदुतोत्तरस्थाने जायत इति । विवरणेऽप्युत्तरस्थाने जायमानत्वाज्जायत उत्तरत्वेनाभासना\textbf{दुत्तराणी}ति । एवं तु कथमनेन \add{न} व्याख्यातमिति न प्रतीमः ॥
	\pend
      

	  \pstart \textbf{जात्युत्तर}शब्दस्य विग्रहं दर्शयन्नाह--\textbf{जात्येति । अभूतदोषोद्भावनानि जात्युत्तराणी}ति ब्रुवता \textbf{वार्त्तिककृता} भूतदोषोद्भावनं तु यद् दूषणाख्यं तदेवोत्तीर्यते अनिष्टपक्षादनेनोत्तारयति, निर्वाहयति वा स्वपक्षमिति । \leavevmode\marginnote{\textenglish{83b/ms}}...मिति सूचयति । एतच्च जात्युत्तरव्युत्पादनमाचार्यस्य हेत्वाभासवन्न प्रयोगार्थम् । यथा \textbf{नैयायिका} मन्यन्ते--अत्यन्तपराजीयमानावस्थायां     \leavevmode\marginnote{\textenglish{256/dm}} “
	  
	अभूतस्यासत्यस्य दोषस्य उद्भावनानि । उद्भाव्यत \footnote{एतैरुद्भा \cite{dp-msC}}एतैरित्युद्धावनानि वचनानि । तानि जात्युत्तराणि । जात्या सादृश्येनोत्तराणि जात्युत्तराणीति ॥ 
	  
	कतिपयपदवस्तुव्याख्यया यन्मयाप्तं कुशलममलमिन्दोरंशुवन्न्यायबिन्दोः । 
	  
	पदमजरमवाप्य ज्ञानधर्मोत्तरं यज् जगदुपकृतिमात्रव्यापृतिः\footnote{व्यापृतः \cite{dp-msC} \cite{dp-msD}} स्यामतोऽहम् ॥ 
	  
	आचार्यधर्मोत्तर\footnote{समाप्तेयं न्यायबिन्दुटीका कृतिराचार्यधर्मोत्तरस्य ॥ आ. ०रपादविरचितायां \cite{dp-msB} \cite{dp-msD}} विरचितायां न्यायबिन्दुटीकायां तृतीयः परिच्छेदः समाप्तः ॥\footnote{\cite{dp-msA} \cite{dp-msB} प्रतिषु--
	    \begin{verse}
	सहस्रमेकं श्लोकानां तथा शतचतुष्टयम् ।\\
	    सप्तसप्ततिसंयुक्तं निपुणं परिपिण्डितम् ॥\\
	    
	    \end{verse}
	   \cite{dp-msD} प्रतौ--० पिण्डितम् ॥ १४७ ॥ मंगलं महा श्री ॥ \cite{dp-msC} प्रतौ--समाप्तमिति ॥ संवत् १४९० वर्षे मार्गशिर सुदि ३ रवौ श्री खरतरगच्छे श्री जिनराजसूरिपट्टे, श्री श्री जिनभद्रसूरिराज्ये परीक्षगूर्जरसुतधरणाकेन लिखापितं ॥ शुभं भवतु ॥ कल्याणमस्तु ॥ न्यायबिन्दुसूत्रवृत्ति...पुरोहितहरीयाकेन लिखितम् ॥}” जातिर्हेत्वाभासश्च प्रयोक्तव्य इति । किञ्च दूषणस्वरूपस्य स्फुटार्थबोधनार्थम् । हेयज्ञाने हि तद्विविक्तमुपादेयं सुज्ञातं भवतीति जाति \footnote{तेः} हेत्वाभासानाञ्च स्फुटस्वरूपपरिज्ञानस्य प्रयोजनम् स्ववाक्ये परिवर्जनं परप्रयुक्तानामपि दोषोद्भावनमितरथा व्यामोहः स्यादित्युक्तप्रायम् ।
	\pend
      

	  \pstart तत्र जात्युत्तरस्योदाहरणं यथा--अनित्यः शब्दः कृतकत्वाद् घटवदित्युक्ते किमिदं कृतकत्वं शब्दगतं हेतुत्वेनोपनीतमाहोस्विद् घटगतम् । यदि शब्दगतं तस्य घटो दृष्टान्तोऽसम्भवादव्याप्तेरनैकान्तिको\add{... ... ...}प्रत्यवस्थानात्मिका जाति\add{... ... ...}साधर्म्यादिति नैयायिक\add{... ...} \add{... ... ...}\textbf{आचार्य}स्य त्वयमाशयो\add{... ... ...} \add{... ... ...}तथाहि \textbf{नैयायिका}\add{... ...}प्रतिज्ञापदयोर्विरोधमाहुस्तथा--प्रयत्नानन्तरीयकः शब्दः\add{... ... ...}प्रयत्नानन्तरीयकत्वादि\add{... ... ...}हेतुमाचक्षते । तयैव दूषणा\add{... ... ...} शब्दस्य धर्मित्वात् । न चेयं जाति\add{... ... ...}भवति । ततश्च यस्यैव प्रत्यवस्थानस्य\add{... ... ...}तत्प्रयो \add{... ... ...} \add{... ... ...}जात्युत्तराणीत्यत्रेतिशब्दो माहावाक्यपरिसमाप्तौ ॥
	\pend
      \leavevmode\marginnote{\textenglish{257/dm}}

	  \pstart \textbf{आचार्यश्रीधर्मकीर्त्ति}विरचितस्यास्य \textbf{न्यायबिन्दु}संज्ञकस्य प्रकरणस्य यथावदर्थप्रकाशिकां महापटीयसीमीदृशीं \add{व्याख्यां} \add{विरच}यता मया\add{... ...}किमपि पुण्यमुपार्जितं तदनेन तादृशीमवस्थां प्राप्य सकलसत्त्वोपकारं\add{... ...}मित्यध्याशयो मे\add{... ... ...}क्रियायोगात्सात्मीकृतपरार्थकरणोऽयं \textbf{धर्मो}त्तरः \textbf{कतिपये}त्यादिना\add{... ... ...}श्लोकमाह ।
	\pend
      

	  \pstart अस्यायं समुदायार्थः । \textbf{न्यायबिन्दोः} कियत्\add{... ... ...}कुशलमाप्तमतः कुशला\textbf{दजरं ज्ञानधर्मोत्तरं} च पदं तदवाप्य \textbf{जगदुपकृतिमात्र}\add{व्यापृतिः} स्यामिति ।\add{... ... ... ...}बुद्ध्यते । पद्यन्ते गम्यन्तेऽर्था एभिरिति पदानि वाक्यानि तेषां वस्तुप्रतिपाद्यतयास्ति तमभिधेयरूपं\add{... ...}शेषो ज्ञेयः । \textbf{आप्तं} प्राप्तं कुशलं सुकृतम् । किं \textbf{कुर्वता} ? \leavevmode\marginnote{\textenglish{84a/ms}}... ...भवति पुण्यमपि कुशलञ्च । यथा पर\add{... ...}परोद्भवापि । ततो व्यभिचारसंभवाद्विशेषणम् । किंवन्निर्मलम् ? \textbf{इन्दोरंशुवदि}ति । \textbf{इन्दो}श्चन्द्रमसोंऽ\textbf{शवः} किरणास्त इव । एवंविधविधानेन यत् पुण्यं जन्यते तदवश्यममलतयैतत्तुल्यं भवतीति भावः । \textbf{पदं} प्रतिष्ठामवस्थामिति यावत् । किं\add{... ...}द्यते...यत्र तत्तथा । जराग्रहणस्योपलक्षणत्वात् मृत्योरपि सङ्ग्रहो ज्ञातव्यः । तेनायमर्थः--अजरममर्त्यं चेति । अथवा जरानिर्देशेनैव दण्डापूपन्यायेन मृत्योः प्रतिषेधः कृत एवेत्यवसेयम् । पुनरपि तद्विशिनष्टि--\textbf{ज्ञाने}ति । ज्ञानं हेयोपादेयतत्त्वस्य साभ्युपायस्यावबोधो विवक्षितः । \textbf{धर्मं}श्च सर्वोपकरणनिवर्त्तकोऽदृष्टः । तावेवोत्तरावधिकौ यत्र पदे तत्तथा । ताभ्यां \textbf{चोत्त}रं श्रेष्ठम् । यत्तदोश्च नित्यमभिसम्बधेन तच्छब्दस्य लब्धत्वात् तद\textbf{वाप्ये}त्यर्थोऽवतिष्ठते । \textbf{जगतो} जीवलोकस्य \textbf{उपकृति}रुपकारः । सैव त\textbf{न्मात्रम्} । तद्\textbf{व्यापृति}र्व्यापारो व्याप्रियमाणता यस्य मम सोऽहं तथा ।
	\pend
      

	  \pstart नावाशंसाविषयेऽस्मिन्नाशीर्लिङ्गा भूयासमिति शब्दसिद्धे\add{र्नैवम}नेन वक्तव्यम् तत् किमेवमवादीदिति चेत् । न । आशंसाविषयत्वाभावात् । यत ऽएवंविधानुष्ठानजन्मना पुण्यातिशयेन एवम्भूतपदप्राप्तेस्ततोऽपि ममैवंविधक्रियस्य सम्भाव्यमानत्वादेव... ...यमेतदित्यभिप्रायात् सर्वमवदातमिति ॥
	\pend
      
	    
	    \stanza[\smallbreak]
	गुरो\textbf{र्जिता}रेरधिगम्य धीधनं मया हि टीका विवृता पटीयसी ।&कुतूहलेनापि तदत्र युज्यते निरीक्षणं साधु विवेचकानाम् ॥&अज्ञो जनस्त्यजति लब्धमपीह रत्नं काचेन तुल्यमिति चलायतेति\footnote{मिति चंचलमानसोऽपि} ।&एतावतैव तदलंङ्करणं न किं स्यात् किं वाऽऽदरेण तदुपाददते न धन्याः ॥&इमं निबन्धं विधिवद्विधाय \add{मया ह्य}वाप्तं सुकृतंथकञ्चित् ।&इहैव जन्मन्यथ तेन सत्त्वा अनन्तसंबोधिमवाप्नुवन्तु ॥&॥ पण्डितदुर्वैकमिश्रविरचितधर्मोत्तरप्रदीपो नाम निबन्धः समाप्तः ॥\&[\smallbreak]


	
	    
	    \endnumbering% ending numbering from div
	    \endgroup
	    
	  % running endDocumentHook
     \backmatter 
	 \chapter{The TEI Header}
	 \begin{minted}[fontfamily=rmfamily,fontsize=\footnotesize,breaklines=true]{xml}
       <teiHeader xmlns="http://www.tei-c.org/ns/1.0" xml:lang="en">
   <fileDesc>
      <titleStmt>
         <title type="main" subtype="basetext">Nyāyabindu</title>
         <title type="sub" subtype="commentary" n="1">Nyāyabinduṭīkā</title>
         <title type="sub" subtype="commentary" n="2">Dharmottarapradīpa</title>
         <author role="baseauthor">Dharmakīrti</author>
         <author role="commentator" n="1">Dharmottara</author>
         <author role="commentator" n="2">Durveka Miśra</author>
         <funder>Deutsche Forschungsgemeinschaft</funder>
         <funder>The National Endowment for the Humanities</funder>
         <principal>
	           <persName>Birgit Kellner</persName>
	        </principal>
         <respStmt>
            <resp>data entry by</resp>
            <name key="name aurorachana">Aurorachana, Auroville</name>
         </respStmt>
         <respStmt xml:id="sarit-encoder-dp">
            <resp>prepared for SARIT by</resp>
            <persName key="name person lo">Liudmila Olalde</persName>
         </respStmt>
      </titleStmt>
      <editionStmt>
         <p> </p>
      </editionStmt>
      <publicationStmt>
         <publisher>SARIT: Search and Retrieval of Indic Texts. DFG/NEH Project (NEH-No.
	HG5004113), 2013-2016 </publisher>
         <idno>2015-08-07</idno>
         <availability status="restricted">
            <p>Copyright Notice:</p>
            <p>Copyright 2015 SARIT</p>
            <licence> 
	              <p>Distributed under a <ref target="https://creativecommons.org/licenses/by-sa/4.0/">Creative Commons Attribution-ShareAlike 4.0 International licence.</ref> Under this licence, you are free to:</p>
	              <list>
                  <item>Share — copy and redistribute the material in any medium or format.</item>
                  <item>Adapt — remix, transform, and build upon the material for any purpose, even commercially.</item>
               </list>
	              <p>The licensor cannot revoke these freedoms as long as you follow the license terms.</p>
	              <p>Under the following terms:</p>
	              <list>
                  <item>Attribution — You must give appropriate credit, provide a link to the license, and indicate if changes were made. You may do so in any reasonable manner, but not in any way that suggests the licensor endorses you or your use.</item>
                  <item>ShareAlike — If you remix, transform, or build upon the material, you must distribute your contributions under the same license as the original.</item>
               </list>
	              <p>More information and fuller details of this license are given on the Creative Commons website.</p>
	           </licence>
            <p>SARIT assumes no responsibility for unauthorised use that infringes the rights of any copyright owners, known or unknown.</p>
         </availability>
         <date>2015</date>
      </publicationStmt>
      <sourceDesc>
         <bibl xml:id="dp-malvania-book">
	           <title type="main">Dharmottarapradīpa</title>
	           <title type="sub">Being a sub-commentary on Dharmottara's Nyāyabinduṭikā, a commentary on Dharmakīrti's Nyāyabindu</title>
	           <author>Dharmakīrti</author>
	           <author>Dharmottara</author>
	           <author>Durveka Miśra</author>
	           <editor xml:id="ed-dm">
               <forename>Dalsukhbhai</forename> 
               <surname>Malvania</surname>
            </editor>
	           <publisher>Kashiprasad Jayaswal Research Institute</publisher>
	           <pubPlace>Patna</pubPlace>
	           <date>1971</date>
	           <note>Revised second edition</note>
	        </bibl>
         <listWit>
            <head>List of manuscripts and published works utilised by Malvania. These descriptions are extracted from  Malvanias  introduction to his <ref target="#dp-malvania-book">edition</ref> pp. iii-viii).</head>
            <witness xml:id="dp-ms-dp">
	              <msDesc>
                  <msIdentifier>
                     <idno resp="#sarit-encoder-dp">Dharmottarapradīpa-MS</idno>
                     <altIdentifier>
                        <idno/>
                        <!-- is there a standard identifier? --></altIdentifier>
                  </msIdentifier>
                  <msContents>
                     <msItem>
                        <author>Durveka Miśra</author>
                        <title>Dharmottarapradīpa</title>
                     </msItem>
                  </msContents>
                  <physDesc>
                     <objectDesc>
                        <p>84 palm-leaves, written in Newari script. The first leaf is in a mutilated condition.</p>
                     </objectDesc>
                  </physDesc>
                  <history>
                     <p>Malvania had no access to the manuscript itself, but merely  to the photos procured by Rāhula Sāṅkṛtyāyana of the palm-leaf manuscripts of Sanskrit works in Tibet that were preserved in the Bihar Research Society, Patna, in thirteen albums. According to Malvania's description: "the 13th album contains the Dharmottarapradīpa in 28 plates. [...] Each of the plates bears the caption हे० बि० अ० suggesting that the work is Hetu-bindu-anuṭīkā. The work however is Nyāya-bindu-anuṭīkā, which has been called Dharmottara-pradīpa by the commentator himself. The original copy covers 84 leaves. It is written in Newari script. When the photocopy was made, the 60th leaf was not reversed. Consequently, 60A has been photographed twice, whereas there is no photo of the reverse, i. e., 60B. The MS is correct, but here and there it is indistinct. The first leaf is in a mutilated condition." (Malvania, "Introduction", <ref target="#dp-malvania-book">Dharmottarapradīpa</ref>, p. vii).</p>
                  </history>
               </msDesc>

	           </witness>
            <witness xml:id="dp-msA">
	              <msDesc>
                  <msIdentifier>
                     <idno>A</idno>
                     <altIdentifier>
                        <idno/>
                        <!-- is there a standard identifier? --></altIdentifier>
                  </msIdentifier>
                  <msContents>
                     <msItem>
                        <author>Dharmottara</author>
                        <title>Nyāyabinduṭīkā</title>
                     </msItem>
                  </msContents>
                  <physDesc>
                     <objectDesc>
                        <p>Palm-leaf manuscript.</p>
                     </objectDesc>
                  </physDesc>
                  <history>
                     <p>Palm-leaf manuscript belonging to the Śāntinātha Jaina Bhaṇḍāra Khambhāt. It was written in Vikrama Saṃvat 1229 (A.D. 1172). For a detailed description see <bibl>Peterson, A Third Report of Operaions in Search of Sanskrit MSS. in the Bombay Circle. 1887, No. 215.</bibl>(Malvania, "Introduction", <ref target="#dp-malvania-book">Dharmottarapradīpa</ref>, p. v.)</p>
                  </history>
               </msDesc>
	           </witness>
            <witness xml:id="dp-msB">
	              <msDesc resp="#ed-dm">
                  <msIdentifier>
                     <idno>B</idno>
                     <altIdentifier>
                        <idno/>
                        <!-- is there a standard identifier? --></altIdentifier>
                  </msIdentifier>
                  <msContents>
                     <msItem>
                        <author>Dharmakīrti</author>
                        <title>Nyāyabindu</title>
                        <author>Dharmottara</author>
                        <title>Nyāyabinduṭīkā</title>
                     </msItem>
                  </msContents>
                  <physDesc>
                     <objectDesc>
                        <p/>
                     </objectDesc>
                  </physDesc>
                  <history>
                     <p>This manuscript is included in the Bhau Daji collection of MSS. in the Bombay Branch of the Royal Asiatic Society. It is recorded as Laghu-dharmottara-sūtra. (Malvania, "Introduction", <ref target="#dp-malvania-book">Dharmottarapradīpa</ref>, p. v)</p>
                  </history>
               </msDesc>
	           </witness>
            <witness xml:id="dp-msC">
	              <msDesc>
                  <msIdentifier>
                     <idno>C</idno>
                     <altIdentifier>
                        <idno/>
                        <!-- is there a standard identifier? --></altIdentifier>
                  </msIdentifier>
                  <msContents>
                     <msItem>
                        <author>Dharmakīrti</author>
                        <title>Nyāyabindu</title>
                        <author>Dharmottara</author>
                        <title>Nyāyabinduṭīkā</title>
                     </msItem>
                  </msContents>
                  <physDesc>
                     <objectDesc>
                        <p>Palm-leaf manuscript</p>
                     </objectDesc>
                  </physDesc>
                  <history>
                     <p>Manuscript of the Jaina Jñāna Bhaṇḍāra, Jaisalmer. Palm-leaf manuscript no. 364. It was copied by Purohita Hariyāka at the instance of Śrāvaka Dharaṇāka at the time of Ācārya Śrī Jinabhadra Sūri, the pupil of Śrī Jinarāja Sūri, in V.S. 1490 (1433 A.D.). Muni Śrī Puṇyavijayaji noted down the variants of this MS.(Malvania, "Introduction", <ref target="#dp-malvania-book">Dharmottarapradīpa</ref>, pp. v-vi)</p>
                  </history>
               </msDesc>
	           </witness>
            <witness xml:id="dp-msD">
	              <msDesc>
                  <msIdentifier>
                     <idno>D</idno>
                     <altIdentifier>
                        <idno/>
                        <!-- is there a standard identifier? --></altIdentifier>
                  </msIdentifier>
                  <msContents>
                     <msItem>
                        <author>Dharmakīrti</author>
                        <title>Nyāyabindu</title>
                        <author>Dharmottara</author>
                        <title>Nyāyabinduṭīkā</title>
                     </msItem>
                  </msContents>
                  <physDesc>
                     <objectDesc>
                        <p>Palm-leaf manuscript.</p>
                     </objectDesc>
                  </physDesc>
                  <history>
                     <p>The manuscript belongs to the Bhaṇḍāra at Jaisalmer. It is the Palm-leaf manuscript no. 376. It appears to be older than <ref target="#dp-msC">C</ref>. It has been corrected by some reader. Muni Śrī Puṇyavijayaji conjectures that it belongs to the second half of the 13th century. Muni Śrī Puṇyavijayaji noted down the variants of this MS. (Malvania, "Introduction", <ref target="#dp-malvania-book">Dharmottarapradīpa</ref>, p. vi.)</p>
                  </history>
               </msDesc>
	           </witness>
            <witness xml:id="dp-msD-n">Marginal notes in manuscript <ref target="#dp-msD">D</ref>. Some of these notes  are included in the text of the manuscript <ref target="dp-msB">B</ref>. These notes were printed in Malvania's edition with the mark "टि०" in the footnotes.</witness>
            <witness xml:id="dp-edP">P = <ref target="http://east.uni-hd.de/bib/5186/">Peter Peterson's edition of the Nyāyabinduṭīka</ref>, published by the Asiatic Society of Bengal in the Bibliotheca Indica in 1889.The Nyāyabinduṭīkā was prepared on the basis of MSS. <ref target="#dp-msB">A</ref> and <ref target="#dp-msB">B</ref>, whereas the text of the Nyayabindu was finalised on that of the MS. <ref target="#dp-msB">B</ref> only. The second edition (1929) has no changes.
	  </witness>
            <witness xml:id="dp-edE">E = <ref target="http://east.uni-hd.de/bib/5509/">Stscherbatsky's edition of the Nyāyabindu and the Nyāyabinduṭīkā</ref>, published in the Bibliotheca Buddhica from Petrograd in 1918. Prepared on the basis of <ref target="#dp-edP">Peterson's edition</ref> and the MS of the Nyāyabinduṭīkā belonging to Denison Ross.
	  </witness>
            <witness xml:id="dp-edH">H = <ref target="http://east.uni-hd.de/bib/5196/">Chandra Shekhar Shāstrī's edition</ref> published in Banaras in the Haridas Sanskirt Series Volume No. 22, under the title of Nyāya-bindu in 1924. Republished in 1954 without any alteration.
	  </witness>
            <witness xml:id="dp-edN">N = <ref target="http://east.uni-hd.de/bib/5226/">Śrī P. I. Tarkas' edition of the Nyāyabindu and the Nyāyabinduṭīkā</ref>, published as volume 1 of the Nūtana Sanskrit Granthamāla of Akola in 1952. Chiefly based on <ref target="dp-edE">Stscherbatsky's edition</ref>.
	  </witness>
         </listWit>
      </sourceDesc>
   </fileDesc>
   <encodingDesc>
      <p>The texts is structured according to the pages of the printed edition. The different text layers (base-text, commentary 1 and commentary 2) were encoded as follows:
      <list>
            <item>The Dharmottarapradīpa is the uppermost level and was enclosed in the body-element. It is subdivided in 3 parts encoded as &lt;div type="chapter" ana="dp"&gt;.</item>
            <item>The Nyāyabindu was enclosed in &lt;quote type="basetext" ana="nb"&gt;.</item>
            <item>The Nyāyabinduṭīkā was enclosed in &lt;quote type="commentary1" ana="nbṭ"&gt;.</item>
         </list>
      </p>
      <p>Line breaks: In the source file, there were two types of line breaks: returns (and possible surrounding space) and hyphens+returns. These were replaced with lb-elements. The ed-attribute "dm" refers to Malvania's ed<ref sameAs="#dp-malvania-book"/>.</p>
      <p>The folio numbers of the Dharmottarapradīpa-manuscript were encoded as pb-elements with the attribute ed="ms", wich refers to the <ref target="#dp-ms-dp">manuscript</ref> used by Malvania, described above in the source description.</p>
      <p>The page breaks of <ref sameAs="#dp-malvania-book">Malvania's edition</ref> were encoded as pb-elements with the ed-attribute "dm".</p>
      <p>Round and square brackets were replaced by SARIT with the following TEI-elements:
      <list>
            <item>References to other works were enclosed in &lt;ref cRef=""&gt;. The attribute cert="unknown" indicates that cRef was not checked by the encoder; whereas  cert="high" indicates that the value of the cRef was checked by the encoder.</item>
            <item>Text in square brackets was enclosed in &lt;add&gt;, following Malvania's own explanation of the use of square brackets in his introduction (p. viii).</item>
            <item>Text in round brackets  was enclosed in &lt;note type="correction"&gt;, following Malvania's own explanation of the use of round brackets in his introduction (p. viii).</item>
            <item>All suspension points ("...") were enclosed in &lt;add type="gap"&gt;. Malvania used this mark to indicate "the portion that could not be read" ("Introduction", p. viii).</item>
         </list>
      </p>
      <p>Bold characters were enclosed in &lt;hi rend="bold"&gt;</p>
      <p>Double quotes were replaced by quote-elements.</p>
      <p>Single quotes were replaced by &lt;q&gt;.</p>
      <p>Characters that were not readable in the printed edition available to SARIT were enclosed &lt;unclear reason="illegible"&gt;.</p>
      <p>Abbreviations used in the cRef- and ana-attributes in this file: <!-- this is a provisory list and has to be replaced by a refsDecl -->
      <list>
            <item>ak = Vasubandhu's Abhidharmakośa</item>
            <item>dp = Durveka Miśra's, Dharmottarapradīpa</item>
            <item>hb = Dharmakīrti's Hetubindu. Chapter numbers in the cRef-attributes correspond to: <bibl>Hetubindu of Dharmakīti: a point on probans. A Sanskrit version translated with an introduction and notes by Pradeep P. Gokhale. Delhi: Sri Satguru Publications, 1997.<ref target="http://www.dsbcproject.org/node/7643"/>
               </bibl>
            </item>
            <item>hbṭ = Bhaṭṭa Arcaṭa's Hetubinduṭikā. Chapter numbers in the cRef-attributes correspond to: <bibl>Sanghavi, Sukhlalji, Muni Śrī Jambuvijayaji, eds. 1949. Hetubinduṭīkā of Bhaṭṭa Arcaṭa with the Sub-Commentary Entitled Āloka of Durveka Miśra. Baroda: Oriental Institute, pp. 1-229.<ref target="http://east.uni-hd.de/buddh/ind/15/41/494/ http://www.dsbcproject.org/node/7724"/>
               </bibl>
            </item>
            <item>htu = Jitari's <ref target="http://east.uni-hd.de/buddh/ind/26/74/559/ http://www.dsbcproject.org/node/7707">Hetutattvopadeśa</ref>.</item>
            <item>kā = Bhāmaha's Kāvyālaṃkāra </item>
            <item>kāv = Kāvyālaṃkāravṛtti </item>
            <item>MBh = Mahābhārata </item>
            <item>nā = Nyāyāvatāra </item>
            <item>nb = Dharmakīrti's Nyāyabindu</item>
            <item>nbṭ = Dharmottara's Nyāyabinduṭīkā</item>
            <item>ns = Nyāyasūtra; sūtra-numbers in the cRef-attributes correspond to the <ref target="https://www.worldcat.org/title/srigautamamunipranitanyayasutrani-vatsyanamunikrtabhasyavisvanathabhattacaryakrtavrttisametani/oclc/644135949">1922 edition.</ref>
            </item>
            <item>nbh = Vātsyāyana's Nyāyasūtrabhāṣya; page numbers in the cRef-attributes correspond to the <ref target="https://www.worldcat.org/title/srigautamamunipranitanyayasutrani-vatsyanamunikrtabhasyavisvanathabhattacaryakrtavrttisametani/oclc/644135949">1922 edition.</ref>
            </item>
            <item>nv = Uddyotakara's Nyāyavārttika. The page- and line-numbers in the cRef-attributes were take from <persName>Yasuhiro Okazaki</persName>'s <ref target="http://indology.info/etexts/archive/texts/uddyotakara-nyayavarttika.zip">index</ref> and refer to: <bibl>V. P. Dvivedin ed.: Nyāya-Vārttika, Bibliotheca Indica, Calcutta, 1887; rep. Delhi 1986.</bibl>
            </item>
            <item>Pā = Pāṇini's Aṣṭādhyāyī</item>
            <item>pv = Dharmakīrti's Pramāṇavārttika; the verse numbers correspond to <ref target="http://east.uni-hd.de/buddh/ind/36/120/193/">Sāṅkṛtyāyana's edition</ref>.</item>
            <item>pv-pandey = Dharmakīrti's Pramāṇavārttika; the verse numbers correspond to <ref target="http://east.uni-hd.de/buddh/ind/7/16/658/">Pandey's edition</ref>.</item>
            <item>vk-mbh = Vyākaraṇa-mahābhāṣya</item>
            <item>vsū = Kaṇāda's Vaiśeṣikasūtra; the verse numbers correspond to: <bibl>Jambūvijaya, ed. Vaiśeṣikasūtra of Kaṇāda: with the commentary of Candrānanda. Baroda: Oriental Institiute, 1982.</bibl>
            </item>
            <item>svi = Akalaṅka's Siddhiviniścaya. Page numbers might correspond to: <bibl>Mahendra Kumar Jain, ed. Siddhiviniścaya of Akalaṅka edited with the commentary Siddhivniścayaṭīkā of Anantavīrya. 2 vols. Delhi: Bhāratīya Jñānapīṭha Prakāśana (Jñānapīṭha Mūrtidevī Jaina Granthamālā 10), 1953-1957.</bibl>
            </item>
            <item>śv = Kumārila's Ślokavārttika:
	<list>
                  <item>śv-abhāva</item>
                  <item>śv-pratyakṣa</item>
               </list>
            </item>
         </list>
      </p>
      <p>The footnotes were encoded as note-elements with their corresponding n-attribute, which indicates the page and the note number. The note-elements are located were the footnote references appear in the printed edition. When several references point to the the same footnote, the note-element was duplicated and the number was extended with "a", "b", etc. This concerns the following notes: 
    	<list>
            <item>2-1</item>
            <item>4-1</item>
            <item>51-1</item>
            <item>116-5</item>
            <item>120-7</item>
            <item>124-5</item>
            <item>125-2</item>
            <item>162-11</item>
            <item>166-1</item>
            <item>170-11</item>
            <item>171-6</item>
            <item>209-6</item>
            <item/>
            <item/>
            <item>229-2</item>
         </list>
      </p>
      <p>For some footnotes there were no references in the corresponding pages. Since the footnotes appear in the book, they were encoded as note-elements with the type-attribute "noreference". The notes listed below were relocated; the cert-attribute indicates thee degree of certainty of the location assigned by the encoder.
      <list>
            <item>95-3</item>
            <item>148-3</item>
            <item>151-2</item>
            <item>170-3</item>
            <item>172-7</item>
            <item>175-1</item>
            <item>175-7</item>
            <item>180-4</item>
            <item>180-5</item>
            <item>190-4</item>
            <item>198-9</item>
            <item>209-2</item>
            <item>209-3</item>
            <item>209-4</item>
            <item>210-6</item>
            <item>212-6</item>
            <item>227-9</item>
            <item>244-5</item>
            <item>244-7</item>
            <item>248-7</item>
         </list>
      The following notes could not be assigned a position within the text flow and are found at the end of the page, where they appeared in the printed edition.
      <list>
            <item>14-6</item>
            <item>20-3</item>
            <item>121-9</item>
            <item>210-7</item>
            <item>229-5</item>
            <item>240-1</item>
         </list>
          </p>
   </encodingDesc>
   <revisionDesc>
      <change when="2015-09-05" who="pma">Resolved references to notes.</change>
      <change when="2015-09-13" who="#sarit-encoder-dp">Corrected footnote reference:
      <list>
            <item> 95-8 to 95-7</item>
            <item>Second occurence of 121-4 to 121-7</item>
            <item>First occurence of 182-5 to 182-3</item>
         </list>
      </change>
      <change when="2015-09-14" who="#sarit-encoder-dp">Fixed problems with notes for which there where no references.</change>
      <change when="2015-12-30" who="#sarit-encoder-dp">Added @xml:lang to the front-element.</change>
   </revisionDesc>
</teiHeader>
	 \end{minted}
       
      \clearpage
      \begin{english}
      \printshorthands
      \printbibliography
      \end{english}
    
\end{document}
