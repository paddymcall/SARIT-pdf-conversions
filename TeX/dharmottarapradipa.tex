\documentclass[article,12pt,broadsheetpaper]{memoir}%
    \usepackage{syntonly}%
    \syntaxonly%
    
	  \usepackage[normalem]{ulem}
	  \usepackage{eulervm}
	  \usepackage{xltxtra}
  \usepackage{polyglossia}
  \PolyglossiaSetup{sanskrit}{
  hyphenmins={2,3},% default is {1,3}
  }
  \setdefaultlanguage{sanskrit}
  % english should be available, notes and stuff
  \setotherlanguage{english}
  \setotherlanguage{german}
  \setotherlanguage[numerals=arabic]{tibetan}
  \usepackage{fontspec}
  %% redefine some chars (either changed by parsing, or not commonly in font)
  \catcode`⃥=\active \def⃥{\textbackslash}
  \catcode`‿=\active \def‿{\textunderscore}
  \catcode`❴=\active \def❴{\{}
  \catcode`❵=\active \def❵{\}}
  \catcode`〔=\active \def〔{{[}}% translate 〔OPENING TORTOISE SHELL BRACKET
  \catcode`〕=\active \def〕{{]}}% translate 〕CLOSING TORTOISE SHELL BRACKET
  \catcode` =\active \def {\,}
  \catcode`·=\active \def·{\textbullet}
  %% BREAK PERMITTED HERE: \discretionary{-}{}{}\nobreak\hspace{0pt}
  \catcode`‚=\active \def‚{\-}
  \catcode`ꣵ=\active \defꣵ{%
  म्\textsuperscript{cb}%for candrabindu
  }
  %% show a lot of tolerance
  \tolerance=9000
  \def\textJapanese{\fontspec{Kochi Mincho}}
  \def\textChinese{\fontspec{HAN NOM A}}
  \def\textKorean{\fontspec{Baekmuk Gulim} }
  % make sure English font is there
  \newfontfamily\englishfont[Mapping=tex-text]{TeX Gyre Schola}
    % set up a devanagari font
  \newfontfamily\devanagarifont[Script=Devanagari,Mapping=devanagarinumerals,AutoFakeBold=1.5,AutoFakeSlant=0.3]{Chandas}
	\newfontfamily\rmlatinfont[Mapping=tex-text]{TeX Gyre Pagella}
	\newfontfamily\tibetanfont[Script=Tibetan,Scale=1.2]{Tibetan Machine Uni}
  \newcommand\bo\tibetanfont
  
    \defaultfontfeatures{Scale=MatchLowercase,Mapping=tex-text}
	\setmainfont{Chandas}
    \setsansfont{TeX Gyre Bonum}
  
  \setmonofont{DejaVu Sans Mono}
	    \usepackage{savetrees}
	  
	  \sloppybottom
	
	    % numbering depth
	    \maxtocdepth{section}
	    % set up layout of toc
	    \setpnumwidth{4em}
	    \setrmarg{5em}
	    \setsecnumdepth{all}
	    \newenvironment{docImprint}{\vskip 6pt}{\ifvmode\par\fi }
	    \newenvironment{docDate}{}{\ifvmode\par\fi }
	    \newenvironment{docAuthor}{\ifvmode\vskip4pt\fontsize{16pt}{18pt}\selectfont\fi\itshape}{\ifvmode\par\fi }
	    % \newenvironment{docTitle}{\vskip6pt\bfseries\fontsize{18pt}{22pt}\selectfont}{\par }
	    \newcommand{\docTitle}[1]{#1}
	    \newenvironment{titlePart}{ }{ }
	    \newenvironment{byline}{\vskip6pt\itshape\fontsize{16pt}{18pt}\selectfont}{\par }
	    % setup title page; see CTAN /info/latex-samples/TitlePages/, and memoir
	  \newcommand*{\plogo}{\fbox{$\mathcal{SARIT}$}}
	  \newcommand*{\makeCustomTitle}{\begin{english}\begingroup% from example titleTH, T&H Typography
	  \thispagestyle{empty}
	  \raggedleft
	  \vspace*{\baselineskip}
	  
	      % author(s)
	    {\Large Dharmakīrti, Dharmottara and Durveka Miśra}\\[0.167\textheight]
	    % maintitle
	    {\Huge Nyāyabindu}\\[\baselineskip]
	    % titlesubtitle
	    {\small  — Nyāyabinduṭīkā — Dharmottarapradīpa}\\[\baselineskip]
	    {\Large SARIT}\\\vspace*{\baselineskip}\plogo\par
	  \vspace*{3\baselineskip}
	  \endgroup
	  \end{english}}
	  \newcommand{\gap}[1]{}
	  \newcommand{\corr}[1]{($^{x}$#1)}
	  \newcommand{\sic}[1]{($^{!}$#1)}
	  \newcommand{\reg}[1]{#1}
	  \newcommand{\orig}[1]{#1}
	  \newcommand{\abbr}[1]{#1}
	  \newcommand{\expan}[1]{#1}
	  \newcommand{\unclear}[1]{($^{?}$#1)}
	  \newcommand{\add}[1]{($^{+}$#1)}
	  \newcommand{\deletion}[1]{($^{-}$#1)}
	  \newcommand{\quotelemma}[1]{\textcolor{cyan}{#1}}
	  \newcommand{\name}[1]{#1}
	  \newcommand{\persName}[1]{#1}
	  \newcommand{\placeName}[1]{#1}
	  % running latexPackages template
     \usepackage[x11names]{xcolor}
     \definecolor{shadecolor}{gray}{0.95}
     \usepackage{longtable}
     \usepackage{ctable}
     \usepackage{rotating}
     \usepackage{lscape}
     \usepackage{ragged2e}
     
	 \usepackage{titling}
	 \usepackage{marginnote}
	 \renewcommand*{\marginfont}{\color{black}\rmlatinfont\scriptsize}
	 \setlength\marginparwidth{.75in}
	 \usepackage{graphicx}
	 \graphicspath{{images/}}
	 \usepackage{csquotes}
       
	 \def\Gin@extensions{.pdf,.png,.jpg,.mps,.tif}
       
      \usepackage[noend,series={A,B}]{reledmac}
       % simplify what ledmac does with fonts, because it breaks. From the documentation of ledmac:
       % The notes are actually given seven parameters: the page, line, and sub-line num-
       % ber for the start of the lemma; the same three numbers for the end of the lemma;
       % and the font specifier for the lemma. 
       \makeatletter
       \def\select@lemmafont#1|#2|#3|#4|#5|#6|#7|%
       {}
       \makeatother
       \setlength{\stanzaindentbase}{20pt}
     \setstanzaindents{3,2,2,2,2,2,2,2,2,2,2,2,2,2,2,}
     % \setstanzapenalties{1,5000,10500}
     \lineation{page}
     % \linenummargin{inner}
     \linenumberstyle{arabic}
     \firstlinenum{5}
    \linenumincrement{5}
    \renewcommand*{\numlabfont}{\normalfont\scriptsize\color{black}}
    \addtolength{\skip\Afootins}{1.5mm}
    \Xnotenumfont{\bfseries\footnotesize}
    \sidenotemargin{outer}
    \linenummargin{inner}
    \Xarrangement{twocol}
    \arrangementX{twocol}
    %% biblatex stuff start
	 \usepackage[backend=biber,%
	 citestyle=authoryear,%
	 bibstyle=authoryear,%
	 language=english,%
	 sortlocale=en_US,%
	 ]{biblatex}
	 
		 \addbibresource[location=remote]{https://raw.githubusercontent.com/paddymcall/Stylesheets/HEAD/profiles/sarit/latex/bib/sarit.bib}
	 \renewcommand*{\citesetup}{%
	 \rmlatinfont
	 \biburlsetup
	 \frenchspacing}
	 \renewcommand{\bibfont}{\rmlatinfont}
	 \DeclareFieldFormat{postnote}{:#1}
	 \renewcommand{\postnotedelim}{}
	 %% biblatex stuff end
	 
	 \setcounter{errorcontextlines}{400}
       
	 \usepackage{lscape}
	 \usepackage{minted}
       
	   % pagestyles
	   \pagestyle{ruled}
	   
	   \makeoddfoot{ruled}{{\tiny\rmlatinfont \textit{Compiled: \today}}}{}{\rmlatinfont\thepage}
	   \makeevenfoot{ruled}{\rmlatinfont\thepage}{}{{\tiny\rmlatinfont \textit{Compiled: \today}}}
	   
	 
	   \usepackage{perpage}
           \MakePerPage{footnote}
	 
       \usepackage[destlabel=true,% use labels as destination names; ; see dvipdfmx.cfg, option 0x0010, if using xelatex
       pdftitle={Nyāyabindu, Nyāyabinduṭīkā, and Dharmottarapradīpa},
       pdfauthor={SARIT: Search and Retrieval of Indic Texts. DFG/NEH Project (NEH-No.
	HG5004113), 2013-2016 }]{hyperref}
       \hyperbaseurl{}
       \renewcommand\UrlFont{\rmlatinfont}
       \usepackage[english]{cleveref}% clashes with eledmac < 1.10.1!
       % \newcommand{\cref}{\href}
     
\begin{document}
    
     \makeCustomTitle
     \let\tabcellsep&
	\frontmatter
	\tableofcontents
	% \listoffigures
	% \listoftables
	\cleardoublepage
        \begin{english} {\docTitle  \begin{titlePart} Nyāyabindu, Nyāyabinduṭīkā, and Dharmottarapradīpa\end{titlePart}}\textit{Dharmakīrti, Dharmottara, and Durveka Miśra}\end{english}\mainmatter 
\chapter*[{आचार्य‚दुर्वेक‚मिश्र‚विर‚चितो: ध‚र्मोत्त‚र‚प्र‚दीपः ।}]{आचार्य‚दुर्वेक‚मिश्र‚विर‚चितो: ध‚र्मोत्त‚र‚प्र‚दीपः ।}
	  
	% new div opening: depth here is 0
	
	    
	    \begingroup
	    \beginnumbering% beginning numbering from div depth=0
	    
	  
\chapter*[{प्र‚थ‚मः प्र‚त्य‚क्ष‚प‚रिच्छेदः}]{प्र‚थ‚मः प्र‚त्य‚क्ष‚प‚रिच्छेदः}\textsuperscript{\textenglish{1/dm}}
	  \bigskip
	  \begingroup
	

	  \begin{center}%% label @type='head'
	\textbf{आचार्य‚ध‚र्मोत्त‚र‚विर‚चिता}
	\end{center}
	‚{\tiny $_{lb}$}‚ 

	  \begin{center}%% label @type='head'
	\textbf{न्याय‚बिन्दुटीका}
	\end{center}
	‚{\tiny $_{lb}$}‚ 
	  \bigskip
	  \begingroup
	

	  \begin{center}%% label @type='head'
	\textbf{आचार्य्ध‚र्म‚कीर्त्तिप्र‚णीतो न्याय‚बिन्दुः}
	\end{center}
	‚{\tiny $_{lb}$}‚ 

	  \begin{center}%% label @type='head'
	\textbf{प्र‚थ‚मः प्र‚त्य‚क्ष‚प‚रिच्छेदः ।}
	\end{center}
	‚{\tiny $_{lb}$}‚ 

	  \begin{center}%% label @type='head'
	\textbf{ऊँ न‚मो वीत्त‚रागाय\footnote{ऊँ न‚मः स‚र्व‚ज्ञाय--\cite{dp-msB} \cite{dp-edE} \cite{dp-edH} \cite{dp-edN} नास्ति \cite{dp-msA} \cite{dp-edP}} ।}
	\end{center}
	‚{\tiny $_{lb}$}‚ 

	  \pstart \leavevmode% starting standard par
	स‚म्य‚ग्ज्ञान‚पूर्विका स‚र्व‚पुरुषार्थ‚सिद्धिरिति त‚द् व्युत्पाद्य‚ते ॥१॥
	\pend% ending standard par
      
	  \endgroup
	 ‚{\tiny $_{lb}$}‚ 
	    \begin{quote}
	  
	    
	    \stanza[\smallbreak]
	ज‚य‚न्ति जातिव्य‚स‚न‚प्र‚ब‚न्ध‚प्र‚सूतिहेतोर्ज‚ग‚तो विजेतुः ।&रागाद्य‚रातेः सुग‚त‚स्य वाचो म‚न‚स्त‚म‚तान‚ब‚व‚माद‚धानाः ॥\&[\smallbreak]


	
	    \end{quote}
	  
	  \endgroup
	‚{\tiny $_{lb}$}‚

	  \pstart \leavevmode% starting standard par
	\leavevmode\ledsidenote{\textenglish{1b/ms}}\add{\edtext{\textsuperscript{*}}{\lemma{*}\Bfootnote{प‚त्र‚म‚त्र त्रुटित‚म्--सं०}}सूत्राभिध‚र्म}विन‚याः सुग‚त‚स्य वाचो बाधा विनाद्भुत‚त‚या स‚त‚तं ज‚य‚न्ति ।‚{\tiny $_{lb}$}‚ आभान्ति द‚र्प‚ण इव प्र‚तिबिम्बितानि त‚त्त्वानि य‚त्र स‚क‚ल‚स्य प‚दार्थ‚राशेः ॥‚{\tiny $_{lb}$}‚ \textbf{श्रीध‚र्म‚कीर्त्त्य}भिमुखानि ज‚ग‚द्धितानि \textbf{ध‚र्मोत्त‚र‚स्य} विपुलानि प‚दान्य‚मूनि ।‚{\tiny $_{lb}$}‚ स‚म्य‚क् प्र‚काश्य सुकृतं स‚मुपार्ज‚यामि तेनापि दुःख‚ज‚ल‚धेर्ज‚ग‚दुद्ध‚रामि ॥
	\pend% ending standard par
      ‚{\tiny $_{lb}$}‚
	    
	    \stanza[\smallbreak]
	मात्स‚र्यादिम‚लोपेतैः \edtext{}{\lemma{लोपेतैः}\Bfootnote{प‚त्र‚म‚त्र त्रुटित‚म्--सं०}}\add{श‚क्यं न गुण‚द‚र्श‚न‚म् ।}&य‚तो हे\add{म‚प‚रीक्षायाः स‚न्तो हि निक‚षोप‚लाः ॥}\&[\smallbreak]


	‚{\tiny $_{lb}$}‚

	  \pstart \leavevmode% starting standard par
	\textbf{न्याय‚बिन्दुटीकां} चिकीर्षुरेष \textbf{ध‚र्मोत्त‚रो} भ‚ग‚व‚त्पूजापुण्य‚योः साध्य‚साध‚न‚भाव‚म‚तीन्द्रिय‚{\tiny $_{lb}$}‚मीदृशि विष‚ये ग‚त्य‚न्त‚र‚विर‚हाद‚भ्युपेय‚प्रामाण्यादाग‚माद‚व‚ग‚म्य, पुण्य‚स्य चापुण्य‚विरोधित्वाद‚{\tiny $_{lb}$}‚पुण्य‚फ‚ल‚विध्नाद्य‚भावो व्याप‚क‚विरुद्ध‚विधिना स‚म्भ‚वी, तेनाविध्नेन ग्र‚न्थ‚क‚र‚ण\edtext{}{\lemma{ण}\Bfootnote{प‚त्र‚म‚त्र त्रुटित‚म्--सं०}}\add{त‚त्स‚माप्तौ‚{\tiny $_{lb}$}‚ ‚{\tiny $_{lb}$}‚ \leavevmode\ledsidenote{\textenglish{2/dm}}‚{\tiny $_{lb}$}‚ त‚द‚ध्य‚य‚ने} च भ‚ग‚व‚द्गुण‚ग‚ण‚श्र‚व‚ण‚ज‚नितातिश‚य‚विशेषासादित‚पुण्य‚स्य त‚थैवाविध्नेन जिज्ञासित‚{\tiny $_{lb}$}‚शास्त्रार्थ‚त‚त्त्वाव‚ग‚मो भ‚विष्य‚ति, दृष्ट‚शिष्टाचारोप्य‚नुपालितो भ‚विष्य‚तीत्य‚भिस‚न्धाय वाग्विज‚या‚{\tiny $_{lb}$}‚ख्यान‚द्वारां भ‚ग‚व‚तः पूजां स्तुतिम‚यीमार‚भ‚ते ।
	\pend% ending standard par
      ‚{\tiny $_{lb}$}‚

	  \pstart \leavevmode% starting standard par
	स्यादेत‚त्--पूजायाः पुण्योप‚ज‚न‚न‚मात्र‚प्र‚योज\edtext{}{\lemma{योज}\Bfootnote{प‚त्र‚म‚त्र त्रुटित‚म्--सं०}}\add{न‚स्य स‚म्पाद‚क‚त्}वादारिप्सितार‚म्भात्प्राक्‚{\tiny $_{lb}$}‚ काय‚म‚यीमेवेष्ट‚देव‚तापूजामार‚च‚य्य किमिति न त‚दार‚भ्य‚ते ? अथोच्य‚ते--त‚त्पूर्विकायाम‚पि‚{\tiny $_{lb}$}‚ प्र‚वृत्तौ स्तुतिपूजैव प्र‚वृत्तिपुरःस‚री किन्न कृता इति चोद्य‚माप‚द्येत । त‚था चाशोक‚व‚निकाचोद्य‚{\tiny $_{lb}$}‚स‚दृश‚मिद‚मिति नानुवाद्य‚म‚पि विदुषामिति । अस‚देत‚त् । एवं हि\edtext{}{\lemma{हि}\Bfootnote{प‚त्र‚म‚त्र त्रुटित‚म्--सं०}}\add{काय‚पूजाया आर‚च‚ने स्तुति}‚{\tiny $_{lb}$}‚प‚दानि प्र‚युञ्जान‚स्याऽत‚द्व्याख्यान‚भूत‚स्यास्य श्लोक‚स्याप्र‚कृत‚स्य क‚र‚णं नाप‚द्येत । काय‚पूजा तु‚{\tiny $_{lb}$}‚ सुखास‚नोप‚देश‚नादीतिक‚र्त्त‚व्य‚तास्थानीय‚त्वान्नाप्र‚कृत‚क‚र‚ण‚चोद्येनोप‚द्रूय‚त इति ।
	\pend% ending standard par
      ‚{\tiny $_{lb}$}‚

	  \pstart \leavevmode% starting standard par
	अत्रोच्य‚ते--स्यादेवैत‚द् य‚दि स्वार्थ‚मुद्दिश्य स्तुतिरीदृशी पूजा विधीय‚ते । किं त‚र्हि ?‚{\tiny $_{lb}$}‚ श्रोतृज‚ना\edtext{}{\lemma{ना}\Bfootnote{प‚त्र‚म‚त्र त्रुटित‚म्--सं०}}\add{र्थ‚मुद्दिश्यापि} भ‚ग‚व‚तो गुण‚कीर्त्त‚ने कृते श्रोतृभिर‚न्त‚तः काव्य‚गुण‚जिज्ञास‚यापि‚{\tiny $_{lb}$}‚ श्लोकोऽव‚श्यं ज्ञात‚व्यः । त‚ज्ज्ञानात् त‚थाग‚त‚गुणास्ताव‚त्कालं ताव‚च्छोतृस‚न्तान‚म‚ध्यास‚ते । त‚त्र‚{\tiny $_{lb}$}‚ ये ताव‚द् भ‚ग‚व‚ति प्राग‚तिप्र‚स‚न्न‚म‚न‚स‚स्तेषामेवंविध‚गुणातिश‚य‚शालिनि स्थान एवास्माकं म‚नः‚{\tiny $_{lb}$}‚ प्र‚स‚न्न‚म्इति निश्चिन्व‚तां स्थे\edtext{}{\lemma{स्थे}\Bfootnote{प‚त्र‚म‚त्र त्रुटित‚म्--सं०}}\add{मान‚मासाद‚य‚ति चित्त‚म् ।}ये च म‚ध्य‚स्थास्तेषाम् एवंभूत‚गुण‚र‚त्ना‚{\tiny $_{lb}$}‚ल‚ङ्कृते म‚ह‚तो म‚हीय‚सि चित्त‚माव‚र्ज‚यितुमुचितं स्व‚हिताव‚हितैः, त‚द्व‚य‚मिय‚न्तं कालं प्र‚माद्य‚न्त‚{\tiny $_{lb}$}‚ एवोदास्म‚हे स्म इति निर्विद्य‚मानानां चित्त‚म‚तिम‚त्तं प्र‚तिष्ठ‚ते । येऽप्य‚न‚तिप्र‚स‚न्नास्तेषाम‚पि—‚{\tiny $_{lb}$}‚एवंविध‚गुण‚निकेत‚ने किम‚स्माभिर‚क‚स्माद्विद्विष्य‚ते इति म‚न‚नान्म‚नागाव‚र्ज‚नं माध्य‚स्थ्यं वा‚{\tiny $_{lb}$}‚ \edtext{\textsuperscript{*}}{\lemma{*}\Bfootnote{प‚त्र‚म‚त्र त्रुटित‚म्--सं०}}\add{स्यादित्य‚तिप्र‚स‚न्न‚म‚ध्य‚स्थ}योः पुरुष‚योश्चित्त‚प्र‚साद‚स्थैर्य-म‚नःप्र‚सादोप‚ज‚न‚नाभ्यां पुण्यातिश‚यो‚{\tiny $_{lb}$}‚ जाय‚ते । तृतीय‚स्यापि म‚नागाव‚र्ज‚नेपि पुण्य‚प्र‚स‚वः । माध्य‚स्थ्ये तु भ‚ग‚व‚द्विद्वेषोप‚च‚योप‚नेय‚{\tiny $_{lb}$}‚न‚र‚केष्व‚न‚तिप‚त‚नं विद्वेषोप‚श‚माद् भ‚व‚ति । अस‚त्यां तु स्तुतिम‚य‚पूजायामित्थं त्रिविध‚श्रोतृज‚न‚{\tiny $_{lb}$}‚प्र‚योज‚नं य‚त् त‚न्न कृतं स्यात्--इति स्व‚प‚रार्थो\leavevmode\ledsidenote{\textenglish{2a/ms}}द्देशेन स्तुतिम‚यी पूजा विधीय‚त इति‚{\tiny $_{lb}$}‚ स्थित‚म् ।
	\pend% ending standard par
      ‚{\tiny $_{lb}$}‚

	  \pstart \leavevmode% starting standard par
	त‚त्रानेन श्लोकेन भ‚ग‚वान् स्वार्थ‚स‚म्प‚दा प‚रार्थ‚स‚म्प‚दुपाय‚स‚म्प‚दा प‚रार्थ‚स‚म्प‚दा च स्तूय‚ते ।‚{\tiny $_{lb}$}‚ तासां तिसृणाम‚पि स‚म्प‚दाम‚व‚श्य‚व‚क्त‚व्य‚त्वात् । त‚थाहि चिर‚कालाभ्यास‚सात्म्यीकृत‚म‚हाकृप‚स्य‚{\tiny $_{lb}$}‚ भ‚ग‚व‚तः प‚रार्थ‚स‚म्प‚त् प्र‚धानं प्र‚योज‚न‚म्, इत‚र‚द‚प्र‚धान‚म्, आनुष‚ङ्गिकं त‚थाग‚त‚त्व‚म‚पीति सा‚{\tiny $_{lb}$}‚ ताव‚द‚व‚श्याभिधेया ।
	\pend% ending standard par
      ‚{\tiny $_{lb}$}‚

	  \pstart \leavevmode% starting standard par
	त‚दाह \textbf{भ‚ट्ट‚व‚राह‚स्वामी}--
	\pend% ending standard par
      ‚{\tiny $_{lb}$}‚

	  \pstart \leavevmode% starting standard par
	\hphantom{.}साक्षात्कृताशेष‚ज‚ग‚त्स्व‚भावं प्रास‚ङ्गिकं य‚स्य त‚थाग‚त‚त्व‚म् । इति ।
	\pend% ending standard par
      ‚{\tiny $_{lb}$}‚

	  \pstart \leavevmode% starting standard par
	सा चोपाय‚स‚म्प‚द‚म‚न्त‚रेणास‚म्भ‚विनीति त‚द‚भिधान‚म‚प्याव‚श्य‚क‚म् । इय‚ञ्चान‚धिग‚त‚{\tiny $_{lb}$}‚स्वाथ‚स‚म्प‚दो न सिध्य‚तीति त‚द‚भिधान‚म‚पि निय‚त‚माप‚तित‚म् । त‚दाह स एव--
	\pend% ending standard par
      ‚{\tiny $_{lb}$}‚
	  \bigskip
	  \begingroup
	
	    \begin{quote}
	  
	    
	    \stanza[\smallbreak]
	तीर्णः स्व‚यं याति ज‚ग‚त्स‚म‚ग्रं मार्गोप‚देशेऽधिकृतो हि नाथः । इति ।\&[\smallbreak]


	
	    \end{quote}
	  
	  \endgroup
	‚{\tiny $_{lb}$}‚

	  \pstart \leavevmode% starting standard par
	त‚त्र स्वार्थ‚स‚म्प‚न्न‚स्य प‚रार्थ‚स‚म्पाद‚नोपाय‚स‚म्प‚त् त‚त्साध्या च प‚रार्थ‚स‚म्पाद‚न‚स‚म्प‚दिति‚{\tiny $_{lb}$}‚ प्र‚थ‚मं \textbf{जाती}त्यादिना \textbf{सुग‚त‚स्ये}त्य‚न्तेन स्वार्थ‚स‚म्प‚त्तिरुक्ता । अनुपाय‚स्य प‚रार्थ‚स‚म्प‚त्तिर्न स‚म्प‚द्य‚त‚{\tiny $_{lb}$}‚ \leavevmode\ledsidenote{\textenglish{3/dm}}‚{\tiny $_{lb}$}‚ इति त‚द‚नु \textbf{वाच} इत्य‚नेन प‚रार्थोपाय‚स‚म्प‚दुक्ता । ईदृशं व‚स्तुनो रूप‚म्, इद‚ञ्चानुष्ठेय‚म् इत्यादि‚{\tiny $_{lb}$}‚रूपेण ध‚र्म‚देश‚नैव हि भ‚ग‚व‚तो ज‚ग‚द्धिताव‚ग‚म‚न‚ल‚क्ष‚ण‚प‚रार्थ‚स‚म्पाद‚नोपाय‚स‚म्प‚द् वैद्य‚व‚र‚स्येव‚{\tiny $_{lb}$}‚ व्याधिस्व‚रूप‚प्र‚तीतिश‚क्तिभैष‚ज्योप‚देशो रोगोप‚श‚म‚स‚म्पाद‚नोपाय‚स‚म्प‚त् । त‚द‚न‚न्त‚रं त‚दुपाय‚{\tiny $_{lb}$}‚स‚म्प‚त्तिसाध्या प‚रार्थ‚स‚म्प‚द‚न्येन प‚देनोक्ता ।
	\pend% ending standard par
      ‚{\tiny $_{lb}$}‚

	  \pstart \leavevmode% starting standard par
	त‚त्र \textbf{वाचो ज‚य‚न्ति}--इति स‚म्ब‚न्धः । अविव‚क्षित‚क‚र्म‚त्वाद‚क‚र्म‚त्व‚म् । \textbf{वाचः} सूत्राभि‚{\tiny $_{lb}$}‚ध‚र्म‚विन‚य‚ल‚क्ष‚णाः । \textbf{ज‚य‚न्ति} उत्कृष्य‚न्ते प्र‚कृष्टा भ‚व‚न्ति । उत्क‚र्ष‚श्च स‚जातीयापेक्ष‚येति‚{\tiny $_{lb}$}‚ साम‚र्थ्यादीश्व‚रादिव‚च‚नेभ्यः प्र‚कृष्टा भ‚व‚न्तीत्य‚र्थः । य‚द्वा \textbf{ज‚य‚न्ति} अभिभ‚व‚न्ति । अभिभ‚व‚श्च‚{\tiny $_{lb}$}‚ प्र‚तियोगिगोच‚र इति । अर्थात् तीर्थिक‚शास्त्राभिभ‚वं कुर्व‚न्तीत्य‚र्थः । प्र‚माणोप‚प‚न्नार्थ‚प्र‚तिपाद‚क‚{\tiny $_{lb}$}‚त्वादासाम् । तेषां तु त‚द्वैप‚रीत्यादिति बुद्धिसिद्धं कृत्वा केव‚ल‚मेत‚द‚त्रोक्त‚म् । \textbf{विनिश्च‚य‚{\tiny $_{lb}$}‚टीकायां} पुन‚र‚नेन युक्तिप्र‚भावेत्यादि हेतुभावेन विशेष‚ण‚मुपात्त‚म् ।
	\pend% ending standard par
      ‚{\tiny $_{lb}$}‚

	  \pstart \leavevmode% starting standard par
	क‚स्य ता इत्याकाङ्क्षायामाह--\textbf{सुग‚त‚स्य} इति । \textbf{सु}श‚ब्दोऽय‚म‚र्थ‚त्र‚य‚वृत्तिर्द्र‚ष्ट‚व्यः । प्र‚श‚स्तं‚{\tiny $_{lb}$}‚ ग‚तः सुग‚तः । प्र‚श‚स्तं य‚था भ‚व‚ति त‚था ग‚तः--संसारात् प्र‚क्रान्तः । क‚थं त‚था ग‚म‚नं त‚स्य ?‚{\tiny $_{lb}$}‚ नैरात्म्य‚द‚र्श‚नेन संसारातिक्र‚मात् । त‚स्य च प्र‚ज्ञानिश्र‚यात् । अथ‚वा ग‚त्य‚र्थानां ज्ञानार्थ‚त्वाद‚पि‚{\tiny $_{lb}$}‚ प्र‚श‚स्तं ग‚तः ज्ञात‚वानित्य‚र्थः । प्र‚श‚स्त‚ञ्च त‚त् त‚त्त्वं नैरात्म्य‚ल‚क्ष‚ण‚म् । त‚त्त्व‚रूप‚त्व‚ञ्च‚{\tiny $_{lb}$}‚ त‚स्य प्र‚माणोप‚प‚न्न‚त्वात् । दृष्ट‚श्चायं सुश‚ब्दः प्र‚श‚स्तार्थ‚वृत्तिः । सुरूपा रूपाजीवेति य‚था ।‚{\tiny $_{lb}$}‚ अपुन‚रावृत्त्या वा ग‚तः सुग‚तः । ग‚तः प्र‚यातः संसारात् । पुन‚रावृत्तिश‚ब्द‚वाच्य‚योर्ज‚न्म‚दोष‚यो‚{\tiny $_{lb}$}‚रात्म‚द‚र्श‚न‚बीजोप‚घातेन भ‚ग‚व‚ता स‚मूल‚घातं निह‚त‚त्वात् । एत‚द‚र्थेपि सुश‚ब्दो दृष्टः । सुन‚यो‚{\tiny $_{lb}$}‚ ज्व‚र इति य‚था । निःशेषं वा ग‚तः सुग‚तः । निःशेषं याव‚द् ग‚न्त‚व्यं ताव‚द् ग‚तः प्राप्तः ।‚{\tiny $_{lb}$}‚ अक्लेश‚निर्ज‚र‚काय‚वाग्बुद्धिवैगुण्य‚ल‚क्ष‚ण‚शेष‚ल‚क्ष‚ण‚प्र‚हाणेन मुनेस्त‚त्प‚द‚प्राप्तेः । एवंवृत्तिर‚पि‚{\tiny $_{lb}$}‚ सुश‚ब्दो दृश्य‚ते । सुपूर्णो घ‚ट इति य‚था ।
	\pend% ending standard par
      ‚{\tiny $_{lb}$}‚

	  \pstart \leavevmode% starting standard par
	त‚स्य सु\leavevmode\ledsidenote{\textenglish{2b/ms}}ग‚त‚स्य किम्भूत‚स्येत्याह--\textbf{विजेतुः} अभिभ‚वितुः । क‚स्यासौ विजेते‚{\tiny $_{lb}$}‚त्याह--\textbf{ज‚ग‚तो} जीव‚लोक‚स्य । विज‚य‚श्च ज‚ग‚द‚पेक्ष‚या प‚र‚म‚प‚द‚प्राप्त्या त‚स्य प्र‚कृष्ट‚त्वं द्र‚ष्ट‚व्य‚म् ।‚{\tiny $_{lb}$}‚ अत एव ज‚ग‚द‚भिभूतं भ‚व‚ति । त‚स्य त‚द्वैप‚रीत्यात् । न पुना राज‚विज‚य इव राजान्त‚रापेक्ष‚या‚{\tiny $_{lb}$}‚ कायादितिर‚स्कारोऽभिभ‚वोऽव‚सेयः ।
	\pend% ending standard par
      ‚{\tiny $_{lb}$}‚

	  \pstart \leavevmode% starting standard par
	ज‚ग‚तः कीदृश‚स्य ? जाय‚ते संस‚र‚त्य‚न‚येति \textbf{जातिः} तृष्णा । व्य‚स्य‚ते विविधेन प्र‚कारेणे‚{\tiny $_{lb}$}‚त‚स्त‚तो अस्य‚ते\edtext{}{\lemma{ते}\Bfootnote{तोऽस्य‚ते}}क्षिप्य‚ते अनेनेति, अस्य‚तीति वा \textbf{व्य‚स‚न‚म्} । जातिरेव व्य‚स‚न‚{\tiny $_{lb}$}‚मित्य‚न्त‚र्नीत‚निय‚मः स‚मासः क‚र्त्त‚व्यः । तृष्य‚न्नेव हि प्राणी त‚त्त‚दाच‚र‚ति येन संसारे संस‚र‚ति ।‚{\tiny $_{lb}$}‚ त‚त‚स्त‚यैवासौ इत‚स्त‚त उत्पाद‚द्वारेण व्य‚स्तः क्षिप्तो भ‚व‚तीति सैव व्य‚स‚नं युक्त‚म् । अथ‚वा‚{\tiny $_{lb}$}‚ \textbf{जातौ} उत्प‚त्तौ निकाय‚विशिष्टायां \textbf{व्य‚स‚न‚म्} आस‚क्तिः विद्याध‚रोऽहं भूयासं, देवोऽहं भूयास‚म्‚{\tiny $_{lb}$}‚ इत्याद्याकारोऽभिनिवेशः । य‚द्वा \textbf{जात्या}श्रितानि \textbf{व्य‚स‚नानि} दुःखानि\edtext{}{\lemma{दुःखानि}\Bfootnote{प्र‚तौ दुष्खानि इत्येवं व‚र्त्त‚ते--सं०}} रोग‚शोकादीनि ।‚{\tiny $_{lb}$}‚ प्राक्त‚न‚व्याख्याने त‚स्य, अन्तिम‚व्याख्याने तेषां \textbf{प्र‚ब‚न्धः} प्र‚वाहः । त‚देव वा प्र‚कृष्टो \textbf{ब‚न्धः} ।‚{\tiny $_{lb}$}‚ प्र‚वाह‚प‚क्षे \textbf{प्र‚ब‚न्धे}न । ब‚न्ध‚प‚क्षे \textbf{प्र‚ब‚न्ध}स्य । \textbf{प्र‚सूतेः} कार‚ण‚स्य । एव‚ञ्च \textbf{प्र‚ब‚न्ध‚प्र‚सूति-}‚{\tiny $_{lb}$}‚ श्रुतिस‚हितं \textbf{हेतु}प‚द‚मुपाद‚दानः प‚र‚म‚त‚निराक‚र‚ण‚ञ्चाभिप्रैति । त‚था ह्येव‚म‚भिधाने स‚त्य‚यं‚{\tiny $_{lb}$}‚ \leavevmode\ledsidenote{\textenglish{4/dm}}‚{\tiny $_{lb}$}‚ तात्प‚र्याथः--ज‚ग‚देवान‚न्य‚स‚त्त्व‚नेयं स्व‚य‚मेव त‚था त‚त्त‚दाच‚र‚ति येन त‚था संसारे संस‚र‚ति । न तु‚{\tiny $_{lb}$}‚ प‚र‚प्रेरितं य‚थाऽन्ये म‚न्य‚न्ते--
	\pend% ending standard par
      ‚{\tiny $_{lb}$}‚
	  \bigskip
	  \begingroup
	
	    \begin{quote}
	  
	    
	    \stanza[\smallbreak]
	अज्ञो ज‚न्तुर‚नीशोऽय‚मात्म‚नः सुख‚दुःख‚योः ।&ईश्व‚र‚प्रेरितो ग‚च्छेत् स्व‚र्गं वा स्व\edtext{}{\lemma{स्व}\Bfootnote{श्व}}भ्र‚मेव वा ॥ \href{http://sarit.indology.info/?cref=MBh.3.30.28}{म‚हाभा० व‚न० ३०. २८}इति ।\&[\smallbreak]


	
	    \end{quote}
	  
	  \endgroup
	‚{\tiny $_{lb}$}‚

	  \pstart \leavevmode% starting standard par
	इत‚र‚था जातिव्य‚स‚नाश्र‚य‚स्येति त‚द‚न्य‚विशेष‚ण‚स‚हित‚मेव ब्रूयादिति ।
	\pend% ending standard par
      ‚{\tiny $_{lb}$}‚

	  \pstart \leavevmode% starting standard par
	किम्भूत‚स्य सुग‚त‚स्य ? \textbf{रागा}दीनां क्लेशानाम् \textbf{अरातेः} श‚त्रोः स‚त्काय‚दृष्टिविग‚मेन,‚{\tiny $_{lb}$}‚ तेन तेषां स्व‚स‚न्ताने स‚मूल‚मुन्मूलित‚त्वात् । य‚त एव भ‚ग‚वान् रागाद्य‚रातिः, ज‚ग‚च्च जाति‚{\tiny $_{lb}$}‚व्य‚स‚न‚प्र‚ब‚न्ध‚प्र‚सूतिहेतुः, अत एवासौ त‚तः प्र‚कृष्टो भ‚व‚ति \textbf{विजेता} त‚स्य । अथ‚वा \textbf{रागाद्य‚राते}रिति‚{\tiny $_{lb}$}‚ \textbf{ज‚ग‚तो} विशेष‚ण‚म् । रागाद‚योऽरात‚यः प्र‚तिप‚क्षा नित्य‚मुप‚घात‚का य‚स्येति कृत्वा । त‚दा तु‚{\tiny $_{lb}$}‚ त‚द्विजेता भ‚ग‚वान् सुग‚त‚त्वादेव बोद्ध‚व्यः । ज‚ग‚तो ये रागाद‚य‚स्तेषाम‚रातेरित्यादिव्याख्यानं‚{\tiny $_{lb}$}‚ तु क्लिष्ट‚त्वात् नोक्त‚म् ।
	\pend% ending standard par
      ‚{\tiny $_{lb}$}‚

	  \pstart \leavevmode% starting standard par
	किंविशिष्टा वाच इत्याह--\textbf{म‚न} इत्यादि । स‚द‚स‚द‚र्थ‚विवेक‚विब‚न्ध‚क‚त्वात् त‚म इव‚{\tiny $_{lb}$}‚ \textbf{त‚नो}ऽज्ञान‚म‚विद्या । त‚च्च म‚न‚सो विक‚ल्प‚विज्ञान‚स्य क‚याचिद्व्य‚पेक्ष‚या ध‚र्म इव भिन्नः क‚थ्य‚ते ।‚{\tiny $_{lb}$}‚ व‚स्तुत‚स्तु क्लिष्ट‚मेव ज्ञान‚म‚विद्या । य‚द्वा क्लिष्टं \textbf{म‚न} एव त‚म इव \textbf{त‚मः} पूर्व‚व‚त् ।
	\pend% ending standard par
      ‚{\tiny $_{lb}$}‚

	  \pstart \leavevmode% starting standard par
	केचित्तु \textbf{म‚न‚स्त‚म} इति श‚ब्द‚स‚मुदाय‚म‚ज्ञानार्थ‚वाच‚क‚माच‚क्ष‚ते ।
	\pend% ending standard par
      ‚{\tiny $_{lb}$}‚

	  \pstart \leavevmode% starting standard par
	त‚स्य \textbf{तान‚वं} त‚नुत्वं म‚न्दीभ‚व‚न‚मित्य‚र्थात् । त‚द् \textbf{आद‚धानाः} कुर्वाणाः । तान‚व‚ग्र‚ह‚णेन‚{\tiny $_{lb}$}‚ चेद‚माच‚ष्टे--न भ‚ग‚व‚तो वाचः स‚र्व‚था ज‚ग‚द‚ज्ञान‚मुप‚ध्न‚न्ति, त‚च्छ्र‚व‚ण‚मात्रेण मोह‚हान्या मुक्त‚{\tiny $_{lb}$}‚त्वात् मार्ग‚भाव‚नावैय‚र्थ्य‚प्र‚स‚ङ्गात् । किन्तु किय‚न्तं काल‚मालोच्य‚मानाः स‚मुदाचार‚तो मोह‚स्य‚{\tiny $_{lb}$}‚ मान्द्यापाद‚नेन तं त‚नूकुर्व‚न्तीति ।
	\pend% ending standard par
      ‚{\tiny $_{lb}$}‚

	  \pstart \leavevmode% starting standard par
	अथ‚वा अन्य‚था व्याख्याय‚ते--ता \textbf{वाच‚स्त‚मो ज‚य‚न्ति} अभिभ‚व‚न्ति\edtext{}{\lemma{न्ति}\Bfootnote{लेखोऽत्र घृष्टः ।}}\add{... ... ...‚{\tiny $_{lb}$}‚ ... ...}\leavevmode\ledsidenote{\textenglish{3a/ms}}\add{... ... ...}मोह‚प्र‚चार‚म् \textbf{आद‚धानाः} । शेषं स‚मानं पूर्वेण ।
	\pend% ending standard par
      ‚{\tiny $_{lb}$}‚

	  \pstart \leavevmode% starting standard par
	य‚द्वा म‚न‚सि त‚मो य‚स्य स \textbf{म‚न‚स्त‚मा} मूढ उक्तः । त‚स्य भावो \textbf{म‚न‚स्त‚म‚स्ता} । त‚स्या‚{\tiny $_{lb}$}‚ अन‚वो निष्ठा । त‚थाहि--नूतिर्न‚वः । त‚द्विरुद्धेन\edtext{}{\lemma{द्विरुद्धेन}\Bfootnote{लेखोऽत्र घृष्टः ।}}\add{... ... ...}\textbf{त‚माद‚धानाः}‚{\tiny $_{lb}$}‚ मूढ‚त्व\edtext{}{\lemma{त्व}\Bfootnote{लेखोऽत्र घृष्टः ।}}\add{... ... ...}प्र‚व‚र्त्त‚नात् । शेषं स‚मानं पूर्वेणेति व्याख्यातः श्लोकः ।
	\pend% ending standard par
      ‚{\tiny $_{lb}$}‚

	  \pstart \leavevmode% starting standard par
	केव‚ल‚मिद‚मालोच्य‚ताम्--\textbf{त‚थाग‚त}म‚भिस्तुव‚ता \textbf{ध‚र्मोत्त‚रे}णास्यैव विज‚यो मुख्य‚वृत्त्या‚{\tiny $_{lb}$}‚ किं न क‚थितः ? किं पुन‚र्ध‚र्मिविशेष‚ण‚त्वेनानुष‚ङ्ग‚तः प्र‚तिपादित इति ? अत्र स‚माधीय‚ते ।‚{\tiny $_{lb}$}‚ आचाय‚श्री\textbf{ध‚र्म‚कीर्त्तिना} भ‚ग‚व‚त्प्र‚व‚च‚नार्थ‚स‚म‚र्थ‚न\edtext{}{\lemma{न}\Bfootnote{लेखोऽत्र घृष्टः ।}}\add{... ... ...}‚{\tiny $_{lb}$}‚ \add{... ...}प्र‚कृत‚त्वात् विज‚यो मुख्य‚तः प्र‚तिपाद्य‚ते ।
	\pend% ending standard par
      ‚{\tiny $_{lb}$}‚

	  \pstart \leavevmode% starting standard par
	य‚द्वा य‚स्य वाच एव त‚था ज‚य‚न्ति त‚स्य विज‚यो द‚ण्डाज्ज‚य‚न्यायेनातिश‚येन प्र‚तिपादितः ।‚{\tiny $_{lb}$}‚ सूचित‚श्चासौ \textbf{विजेतृ}प‚देन ।
	\pend% ending standard par
      ‚{\tiny $_{lb}$}‚\textsuperscript{\textenglish{5/dm}}‚{\tiny $_{lb}$}‚
	  \bigskip
	  \begingroup
	

	  \pstart \leavevmode% starting standard par
	स‚म्य‚ग्ज्ञान‚पूर्विके\edtext{}{\lemma{पूर्विके}\Bfootnote{०पूर्विका स‚र्वेत्यादिना--\cite{dp-msA} \cite{dp-edP} \cite{dp-edE}}}त्यादिनाऽस्य प्र‚क‚र‚ण‚स्याभिधेय‚प्र‚योज‚न‚मुच्य‚ते ।
	\pend% ending standard par
      
	  \endgroup
	‚{\tiny $_{lb}$}‚

	  \pstart \leavevmode% starting standard par
	अथ‚वा\textbf{ऽऽग‚मिकानां} म‚तेन निरुप‚धिशेषे निर्वाण‚धातौ प‚रिनिर्वृतो भ‚ग‚वान् । प‚रिनिर्वृत‚{\tiny $_{lb}$}‚स्यास्य प्र‚व‚च‚न‚म‚य‚मेव व‚पुर्विद्य‚त इति \textbf{आग‚मिको} वाग्विज‚य‚मेव प्र‚तिपाद‚य‚ते स्मेति ।
	\pend% ending standard par
      ‚{\tiny $_{lb}$}‚

	  \pstart \leavevmode% starting standard par
	\textbf{स‚म्य‚ग्ज्ञाने}त्यादिना प्र‚क‚र‚ण‚स्य य‚त् प्र‚योज‚नं स‚म्य‚ग्ज्ञान‚व्युत्प‚त्तिः, त‚स्या य‚त् प्र‚योज‚नं‚{\tiny $_{lb}$}‚ पुरुषार्थ‚सिद्धिरूपं त‚दुच्य‚ते । न च स‚म्य‚ग्ज्ञान‚व्युत्प‚त्तेः स‚म्य‚ग्ज्ञान‚प‚रिज्ञानं प्र‚योज‚नं न पुरुषार्थ‚{\tiny $_{lb}$}‚सिद्धिरिति श‚क्य‚म‚भिधातुम् । विप्र‚तिप‚त्तिनिराक‚र‚णेन स्व‚रूप‚प्र‚तिप‚त्तिरि\edtext{}{\lemma{त्तिरि}\Bfootnote{रे}}व हि‚{\tiny $_{lb}$}‚ स‚म्य‚ग्ज्ञान‚स्य व्युत्प‚त्तिः । सा क‚थ‚मात्म‚न एव प्र‚योज‚नं भ‚वेदित्य‚भिप्रायेण व्याख्यात‚व‚न्तौ‚{\tiny $_{lb}$}‚ \textbf{विनीत‚देव-शान्त‚भ‚द्रौ} । त‚द्व्याख्यान‚म‚व‚म‚न्य‚मानोऽ\textbf{भिधेय‚प्र‚योज‚न‚मुच्य‚ते} इति व्याच‚ष्टे ।‚{\tiny $_{lb}$}‚ अव‚ज्ञाने चाय‚माश‚यः--अस्येदं प्र‚योज‚न‚मिति ख‚ल्व‚न्व‚य-व्य‚तिरेकाभ्याम‚व‚धार्य‚ते । नान्य‚था ।‚{\tiny $_{lb}$}‚ इय‚ञ्च-स्व‚रूपा प्र‚सिद्धिः \edtext{}{\lemma{सिद्धिः}\Bfootnote{पुरुषार्थ‚सिद्धिः}} स‚म्य‚ग्ज्ञान‚व्युत्प‚त्तिम‚न्त‚रेणापि गोपालाङ्ग‚नादीनां‚{\tiny $_{lb}$}‚ भ‚व‚न्ती स‚ति स‚म्य‚ग्ज्ञानि\edtext{}{\lemma{ग्ज्ञानि}\Bfootnote{ने}}, स‚त्याम‚पि त‚द्व्युत्प‚त्तौ अस‚ति स‚म्य‚ग्ज्ञाने म‚नीषिणाम‚भ‚व‚न्ती‚{\tiny $_{lb}$}‚ न त‚द्व्युत्प‚त्तेर‚न्व‚य‚व्य-तिरेकाव‚नुविध‚त्ते । किन्त‚र्हि ? स‚म्य‚ग्ज्ञान‚स्येति त‚स्यैव‚{\tiny $_{lb}$}‚ प्र‚योज‚नं भ‚वितुम‚र्ह‚तीति ।
	\pend% ending standard par
      ‚{\tiny $_{lb}$}‚

	  \pstart \leavevmode% starting standard par
	\textbf{स‚म्य‚ग्ज्ञानेत्यादिना} वाक्येन क‚र‚णेन \textbf{प्र‚क‚र‚णाभिधेय‚स्य} स‚म्य‚ग्ज्ञान‚ल‚क्ष‚ण‚स्य \textbf{प्र‚योज‚नं} दृष्टं‚{\tiny $_{lb}$}‚ पुरुषार्थ‚सिद्धिल‚क्ष‚ण\textbf{मुच्य‚ते वार्त्तिक}कृता क‚र्त्रेत्य‚र्थात् ।
	\pend% ending standard par
      ‚{\tiny $_{lb}$}‚

	  \pstart \leavevmode% starting standard par
	न‚नु च य‚त्राभिधाव्यापारः स‚माप्य‚ते स वाक्यार्थः । न चासौ पुरुषार्थ‚सिद्धौ‚{\tiny $_{lb}$}‚ विश्राम्य‚ति । किं त‚र्हि ? श्रोतुः स‚म्य‚ग्ज्ञान‚व्युत्प‚त्तौ । त‚त् क‚थं सा वाक्यार्थ‚त्वेनोच्य‚ते ? उच्य‚ते ।‚{\tiny $_{lb}$}‚ य‚द्य‚पि श‚ब्दाभिधाव्यापारापेक्ष‚या त‚त् स‚म्य‚ग्ज्ञानं व्युत्पाद्य‚ते शिष्य इति शिष्य‚स‚म्य‚ग्ज्ञान‚विष‚या‚{\tiny $_{lb}$}‚ व्युत्प‚त्तिक्रिया प्राधान्याद्वाक्यार्थः, त‚थाप्य‚स्य तात्प‚र्यार्थ‚स‚म्भ‚वे त‚न्निरूप‚णेनेद‚मुच्य‚ते । त‚था हि‚{\tiny $_{lb}$}‚ त‚द्व्युत्प‚त्तिमेवासौ किम‚ति कार्य‚ते ? य‚तः स‚म्य‚ग्ज्ञान‚पूर्विका स‚र्व‚पुरुषार्थ‚सिद्धिः\edtext{}{\lemma{सिद्धिः}\Bfootnote{लेखोऽत्र घृष्टः ।}}\add{... ... ...}‚{\tiny $_{lb}$}‚ \add{... ... ...}मुच्य‚ते । य‚त्र पुन‚र‚भिधाविष‚य एवार्थः स‚म्भ‚वी न तु तात्प‚र्यार्थो \leavevmode\ledsidenote{\textenglish{3b/ms}}\edtext{}{\lemma{र्यार्थो}\Bfootnote{एकोऽक्ष‚रोत्र व‚र्त्त‚ते किन्तु स स‚म्य‚ङ्न प‚ठ्य‚ते ।}}...स्त‚त्र‚{\tiny $_{lb}$}‚ स एव वाक्यार्थः क‚ल्प्य‚ते । द्वित‚य‚स‚म्भ‚वे तु य‚त्प‚रं वाक्यं त‚था च त‚स्यार्थ इत्यार्थेन न्यायेना‚{\tiny $_{lb}$}‚भिधेय‚स्य प्र‚योज‚नं पुरुषाथ‚सिद्धिर्वाक्यार्थ‚त्वेनोच्य‚त इति व्याख्याय‚त इति ।
	\pend% ending standard par
      ‚{\tiny $_{lb}$}‚

	  \pstart \leavevmode% starting standard par
	केचित् पुन‚रिदं \textbf{ध‚र्मोत्त‚रीयं} वाक्य‚म‚न्य‚था व्याच‚क्ष‚ते । नात्र प्र‚योज‚न‚श‚ब्देन फ‚ल‚म‚भिप्रेतं‚{\tiny $_{lb}$}‚ किन्तु प्र‚युज्य‚ते प्र‚व‚र्त्त्य‚तेऽनेनेति, प्र‚योज‚य‚तीति वा प्र‚योज‚न‚म् । त‚च्च पुरुषार्थ‚सिद्धिहेतुत्व‚म् ।‚{\tiny $_{lb}$}‚ अभिधेय‚स्य हि स‚म्य‚ग्ज्ञान‚स्य पुरुषार्थ‚सिद्धिहेतुत्वेन प्र‚युक्तः पुरुषः प्र‚व‚र्त्त‚त इति ।‚{\tiny $_{lb}$}‚ उत्त‚र‚त्रापि प्र‚योज‚न‚मिद‚मेव विव‚क्षित‚म् । अत एव--\textbf{अत्र चे}त्यादिवाक्ये \textbf{स‚र्व‚पुरुषार्थ‚सिद्धिहेतुत्वं‚{\tiny $_{lb}$}‚ प्र‚योज‚नं} प्र‚व‚र्त्त‚क\textbf{मुक्त}मिति स्प‚ष्टीक‚र‚णं घ‚ट‚त इति । त‚च्च नातिश्लिष्ट‚मुत्प‚श्यामः । त‚था‚{\tiny $_{lb}$}‚ हि \textbf{अत्र च प्र‚क‚र‚णाभिधेय‚स्य स‚म्य‚ग्ज्ञान‚स्य पुरुषार्थ‚सिद्धिहेतुत्वं प्र‚योज‚न‚मुक्त}मिति‚{\tiny $_{lb}$}‚ व‚क्ष्य‚माणेन व्य‚क्तीकृत‚त्वादिहापि \textbf{अभिधेय}स्य \textbf{प्र‚योज‚न‚म्} इति ष‚ष्ठीत‚त्पुरुषोऽव‚श्य‚कार्यः । त‚था चाय‚{\tiny $_{lb}$}‚म‚स‚म‚र्थः प‚द‚विधिर्भ‚वेत् । त‚द्धि पुरुषार्थ‚सिद्धिहेतुत्वं नाभिधेय‚स्य स‚म्य‚ग्ज्ञान‚स्य प्र‚व‚र्त्त‚क‚म‚पि‚{\tiny $_{lb}$}‚ तु पुरुष‚स्य । त‚त् क‚थ‚म‚भिधेय‚प‚देन स‚म‚स्येत प्र‚योज‚न‚प‚द‚म् ? न च केन‚चिद्रूपेणाभिधेय‚स‚म्ब‚न्धि‚{\tiny $_{lb}$}‚‚{\tiny $_{lb}$}‚ ‚{\tiny $_{lb}$}‚ \leavevmode\ledsidenote{\textenglish{6/dm}}‚{\tiny $_{lb}$}‚ त्वऽस्य स‚म‚र्थ‚योरेकार्थीभावो भ‚व‚तीति । न हि य‚ज्ञ‚द‚त्त‚पुत्रो भृत्य‚त्वादिना रूपेण देव‚द‚त्त‚{\tiny $_{lb}$}‚स‚म्ब‚न्धी भ‚व‚न्देव‚द‚त्त‚पुत्र इति ष‚ष्ठीस‚मास‚स्य विष‚यो भ‚वितुम‚र्ह‚ति ।
	\pend% ending standard par
      ‚{\tiny $_{lb}$}‚

	  \pstart \leavevmode% starting standard par
	अथ भाव‚प्र‚धान‚त्वान्निर्देश‚स्य प्र‚योज‚नं प्र‚योज‚क‚त्व‚मित्य‚र्थः । त‚च्चाश्रित‚स्य प्र‚योज‚क‚त‚या‚{\tiny $_{lb}$}‚ भ‚व‚ति स‚म्ब‚न्धीति स‚म‚र्थ‚विधिः क‚ल्प्य‚ते ।
	\pend% ending standard par
      ‚{\tiny $_{lb}$}‚

	  \pstart \leavevmode% starting standard par
	न‚नु य‚दि स‚म्य‚ग्ज्ञान‚ग‚तं पुरुषार्थ‚सिद्धिहेतुत्वं प्र‚योज‚नं प्र‚योज‚क‚त्वं त‚र्हि स‚म्य‚ग्ज्ञानं प्र‚योज‚न‚{\tiny $_{lb}$}‚मित्याप‚न्न‚म् । य‚तो न प्र‚योज‚न‚त्व‚मेव प्र‚योज‚नं भ‚वितुम‚र्ह‚ति । न च स‚म्य‚ग्ज्ञानेन प्र‚युक्तः‚{\tiny $_{lb}$}‚ पु षः स‚म्य‚ग्ज्ञाने प्र‚व‚र्त्त‚ते, किन्तु पुरुषार्थ‚सिद्धिहेतुत्वेन । त‚त् क‚थं प्र‚योज‚नं भ‚वितुम‚र्ह‚ति ?‚{\tiny $_{lb}$}‚ किञ्च वृद्ध‚व्य‚व‚हारो हि श‚ब्दार्थ‚व्य‚व‚हार‚भूमिः । न च वृद्धैः प्र‚योज‚कः स‚त्य‚पि प्र‚योज‚यितृत्वे‚{\tiny $_{lb}$}‚ प्र‚योज‚न‚मुच्य‚ते । अन्य‚था स‚म्य‚ग्ज्ञानं व्युत्प‚द्य‚मानानामात्मानं व्युत्पाद‚कं क‚र्त्तुं प्र‚व‚र्त्त‚मान आचार्यः‚{\tiny $_{lb}$}‚ प्र‚योज‚क इति स‚म्य‚ग्ज्ञान‚व्युत्प‚त्तेः प्र‚योज‚न‚मुच्येत । योऽपि क‚टं कुर्व‚न्तं क‚र्त्तुं प्र‚युङ्क्ते सोऽपि‚{\tiny $_{lb}$}‚ त‚त्प्र‚योज‚न‚मुच्येतेत्येवंवादी न लौकिको न प‚रीक्ष‚क इत्युपेक्ष‚णीय एव ।
	\pend% ending standard par
      ‚{\tiny $_{lb}$}‚

	  \pstart \leavevmode% starting standard par
	अपि च किमेत‚द‚न्य‚था नोप‚प‚द्य‚त एव येनैवं मृत्वा शीर्त्वोप‚पाद्येत । न चैव‚म्, अन्य‚थापि‚{\tiny $_{lb}$}‚ सूप‚पाद‚त्वात् । प्र‚योज‚क‚त्व‚मिति निर्द्देशे च \textbf{ध‚र्मोत्त‚रः} किं गौर‚वं प‚श्य‚ति येनैव‚म‚वाच‚क‚माच‚क्षीत ?‚{\tiny $_{lb}$}‚ क‚थ‚ञ्च \textbf{प्र‚क‚र‚ण‚स्येति} दुरुप‚पादं प्र‚युञ्जीत ? किञ्च य‚दीप्स‚ञ्जिहास‚न् वा पुरुषः प्र‚व‚र्त्त‚ते‚{\tiny $_{lb}$}‚ त‚दुपादान‚प‚रित्यागाभ्यां प्र‚व‚र्त्तितो भ‚व‚ति । य‚थाह \textbf{अक्ष‚पादः}--य‚म‚र्थ‚म‚धिकृत्य प्र‚व‚र्त्त‚ते त‚त्‚{\tiny $_{lb}$}‚ प्र‚योज‚न‚म् \href{http://sarit.indology.info/?cref=nsū.1.1.24}{न्याय‚सू० १. १. २४} इति । अधिकृत्य उद्दिश्येत्य‚र्थः । न च स‚म्य‚ग्ज्ञान‚स्य भिन्नं‚{\tiny $_{lb}$}‚ स‚द‚पि पुरुषार्थ‚सिद्धिहेतुत्व‚म् ईप्स‚न् जिहास‚न् वा प्र‚व‚र्त्त‚ते । किन्त‚र्हि ? हिताहित‚प्राप्ति‚{\tiny $_{lb}$}‚प‚रिहारावुद्दिश्येति तावेव प्र‚योज‚ने युक्ते । किञ्चिद‚जिहासोर‚नुपादित्सोर‚र्थ‚निरीह\leavevmode\ledsidenote{\textenglish{4a/ms}}स्य‚{\tiny $_{lb}$}‚ स‚त्य‚पि पुरुषार्थ‚सिद्धिहेतुत्वे अप्र‚व‚र्त्त‚नात् । स्म‚र‚णाद‚भिलाषेण व्य‚व‚हारः प्र‚व‚र्त्त‚ते इत्य‚लं‚{\tiny $_{lb}$}‚ श‚ब्द‚मात्र‚स‚म‚र्थ‚न‚दृष्टेर‚र्थ‚त‚त्त्वान‚व‚गाहिनो व‚च‚नेऽन्धाद‚रेण ।
	\pend% ending standard par
      ‚{\tiny $_{lb}$}‚

	  \pstart \leavevmode% starting standard par
	अथाभिधेय‚प्र‚योज‚नं प्र‚व‚र्त्त‚क‚मिति स‚म‚स्य‚ते । त‚तोऽय‚म‚दोष इति चेत् । त‚द‚व‚द्य‚म् ।‚{\tiny $_{lb}$}‚ न हि पुरुषाथ‚सिद्धिहेतुत्वेन प्र‚युक्तः पुरुषः स‚म्य‚ग्ज्ञान‚ल‚क्ष‚णे अभिधेये प्र‚व‚र्त्त‚ते किन्तु प्र‚क‚र‚णे‚{\tiny $_{lb}$}‚ प्र‚व‚र्त्त‚ते--ग्र‚न्थ‚श्र‚व‚ण‚ल‚क्ष‚णां प्र‚वृत्तिम‚नुतिष्ठ‚ति ।
	\pend% ending standard par
      ‚{\tiny $_{lb}$}‚

	  \pstart \leavevmode% starting standard par
	अथ ग्र‚न्थ‚स्य श‚ब्दार्थ‚स्व‚भाव‚त्वाद् । एव‚म‚प्य‚सौ श‚ब्देऽभिधेये च स‚मुदाये प्र‚वृत्तो भ‚व‚ति ।‚{\tiny $_{lb}$}‚ त‚द् ग्र‚न्थ‚स्य प्र‚क‚र‚ण‚स्येति ष‚ष्ठी क‚थ‚म् ? अथ पुन‚र‚य‚म‚र्थोऽस्य प्र‚क‚र‚ण‚स्याभिधेयेऽर्थाद‚भिधाना‚{\tiny $_{lb}$}‚भिधेय‚स‚मुदाये प्र‚योज‚नं प्र‚व‚र्त्त‚क‚मिति । त‚था च प्र‚क‚र‚ण‚स्य प्र‚क‚र‚णे प्र‚व‚र्त्त‚क‚मित्य‚स‚ङ्ग‚त‚मुक्तं‚{\tiny $_{lb}$}‚ स्यात् । य‚दि चोक्त‚या व्युत्प‚त्त्या प्र‚योज‚न‚श‚ब्देन प्र‚योज‚कं त‚थाभूत‚म‚स्य बुबोध‚यिषितं भ‚वेत्‚{\tiny $_{lb}$}‚ त‚दा \textbf{स‚म्य‚ग्ज्ञान‚पूर्विके}त्यादिना अभिधेय‚प्र‚योज‚न‚मुच्य‚त इत्युक्तं स्यात् । मुख्य‚श्च स‚म्ब‚न्धी‚{\tiny $_{lb}$}‚ पुरुषः पुरुष‚स्येति द‚र्शितः स्यात् । न चैव‚म्, त‚स्मात् प्र‚योज‚न‚श‚ब्देन फ‚ल‚मेवास्याभिप्रेत‚म‚त्रो‚{\tiny $_{lb}$}‚त्त‚र‚त्रापि ।
	\pend% ending standard par
      ‚{\tiny $_{lb}$}‚

	  \pstart \leavevmode% starting standard par
	अथोच्य‚ते । फ‚लार्थी चेत् प्र‚तिप‚त्ता फ‚ल एव किं न प्र‚व‚र्त्त‚ते ? कि श्र‚मः स‚म्य‚ग्ज्ञान‚{\tiny $_{lb}$}‚ इति । हेयोपादेय‚योर्हानोपादान‚ल‚क्ष‚ण‚फ‚लार्थितैव त‚न्निब‚न्ध‚नं ज्ञानं मृग‚य‚ते । अन्य‚था विसंवाद‚न‚{\tiny $_{lb}$}‚मात्मीय‚माश‚ङ्क‚मान इति का क्ष‚तिः ? फ‚ल‚प‚क्षे तु पुरुष‚प्र‚वृत्त्युप‚योगि चास्याभिधेय‚मिति‚{\tiny $_{lb}$}‚ त‚द् द‚र्शितं भ‚व‚ति ।
	\pend% ending standard par
      \textsuperscript{\textenglish{7/dm}}‚{\tiny $_{lb}$}‚
	  \bigskip
	  \begingroup
	

	  \pstart \leavevmode% starting standard par
	द्विविधं हि प्र‚क‚र‚ण‚श‚रीर‚म्--श‚ब्दः, अर्थ‚श्च\edtext{}{\lemma{श्च}\Bfootnote{श्चेति--\cite{dp-edE} \cite{dp-edH} \cite{dp-edN} \cite{dp-edP}}} ।
	\pend% ending standard par
       ‚{\tiny $_{lb}$}‚ 

	  \pstart \leavevmode% starting standard par
	त‚त्र श‚ब्द‚स्य\edtext{}{\lemma{स्य}\Bfootnote{श‚ब्द‚स्याभिधे \cite{dp-msB} \cite{dp-msC} \cite{dp-msD}}} स्वाभिधेय‚प्र‚तिपाद‚न‚मेव प्र‚योज‚न‚म् । नान्य‚त् । अत‚स्त‚न्न\edtext{}{\lemma{न्न}\Bfootnote{त‚न्निरूप्य‚ते \cite{dp-msA}}} निरूप्य‚ते ।
	\pend% ending standard par
       ‚{\tiny $_{lb}$}‚ 

	  \pstart \leavevmode% starting standard par
	\edtext{\textsuperscript{*}}{\lemma{*}\Bfootnote{अथाभिधेय‚स्यापि किं प्र‚योज‚नाभिधानेनेत्याह--\cite{dp-msD-n}}}अभिधेयं\edtext{}{\lemma{अभिधेयं}\Bfootnote{अभिधेये तु निष्प्र‚योज‚ने त‚त्प्र‚ति--\cite{dp-msB} \cite{dp-msD}}} तु य‚दि निष्प्र‚योज‚नं\edtext{}{\lemma{नं}\Bfootnote{प्र‚योज‚नं त‚दा--C}} स्यात् त‚दा \edtext{}{\lemma{दा}\Bfootnote{अभिधेय‚प्र‚तिप‚त्त‚ये--\cite{dp-msD-n}}}त‚त्प्र‚तिप‚त्त‚ये श‚ब्द‚स‚न्द‚र्भोऽपि\edtext{}{\lemma{र्भोऽपि}\Bfootnote{अपिश‚ब्दाद‚र्थ‚स‚न्द‚र्भे[[र्भो]]ऽपि--\cite{dp-msD-n}}}‚{\tiny $_{lb}$}‚ नार‚म्भ‚णीयः स्यात् ।
	\pend% ending standard par
      
	  \endgroup
	‚{\tiny $_{lb}$}‚

	  \pstart \leavevmode% starting standard par
	न‚नु च नायं प्र‚त्य‚स्त‚मिताव‚य‚वार्थः संज्ञाश‚ब्दः । किं त‚र्हि ? प्र‚युज्य‚तेऽनेन इति,‚{\tiny $_{lb}$}‚ प्र‚योज‚य‚तीति वा व्युत्प‚त्त्या फ‚लेऽपि व‚र्त्त‚ते । त‚त् क‚थ‚मेत‚द् व्याख्याय‚त इति चेत् । स‚त्य‚{\tiny $_{lb}$}‚मेत‚त् । केव‚लं न पुरुषार्थ‚सिद्धिहेतुत्वं स‚म्य‚ग्ज्ञान‚स्यात्म‚न एव प्र‚योज‚कं किन्तु पुरुष‚स्य । त‚त्र‚{\tiny $_{lb}$}‚ चोक्ता दोष‚मात्रा । अभिधेये प्र‚योज‚न‚मिति विग्र‚हे च भूयान् दोषो द‚र्शितः । त‚स्मात् त‚थाभूत‚{\tiny $_{lb}$}‚व्युत्प‚त्तिनापि प्र‚योज‚न‚श‚ब्देनात्र न त‚थाभूतं प्र‚योज‚कं वाच्य‚म्, अपि तु फ‚ल‚मेवेति स‚र्व‚म‚व‚दात‚म् ।
	\pend% ending standard par
      ‚{\tiny $_{lb}$}‚

	  \pstart \leavevmode% starting standard par
	न‚नु च प्र‚क‚र‚णे श्रोतृन् प्र‚विव‚र्त्त‚यिषुर‚य‚म‚भिधेय‚प्र‚योज‚न‚म‚भिध‚त्ते त‚द‚नेनास्यैव त‚दाख्यातु‚{\tiny $_{lb}$}‚मुचित‚म् । त‚च्च य‚थास्व‚म‚भिधेय‚प्र‚त्याय‚न‚ल‚क्ष‚ण‚मित्याश‚ङ्क्याह--\textbf{द्विविधं हि} इत्यादि ।‚{\tiny $_{lb}$}‚ हिर्य‚स्माद् द्विप्र‚कारं प्र‚क‚र‚ण‚स्य श‚रीरं स्व‚भावः । त‚दुक्तं \textbf{काव्याल‚ङ्कारे}--श‚ब्दार्थौ स‚हितौ‚{\tiny $_{lb}$}‚ काव्य‚म्\edtext{\textsuperscript{*}}{\lemma{*}\Bfootnote{काव्याल‚ङ्कार‚वृत्तौ--काव्य‚श‚ब्दोऽयं गुणाल‚ङ्कार‚संस्कृत‚योः श‚ब्दार्थ‚योर्व‚र्त्त‚ते \href{http://sarit.indology.info/?cref=kāv.1.1}{ १. १.} इति ।}} \href{http://sarit.indology.info/?cref=kā.1.16}{का० १. १६} इति । त‚स्माद‚भिधेय‚प्र‚योज‚न‚मुच्य‚त इति ।
	\pend% ending standard par
      ‚{\tiny $_{lb}$}‚

	  \pstart \leavevmode% starting standard par
	क‚थं द्वैविध्य‚मित्याह--\textbf{श‚ब्द} इति । \textbf{चः} श‚ब्देन स‚हार्थं प्र‚क‚र‚ण‚श‚रीर‚त्वेन स‚मुच्चिनोति ।‚{\tiny $_{lb}$}‚ एत‚दुक्तं भ‚व‚ति--श‚ब्दार्थ‚योर‚व‚च्छेद्याव‚च्छेद‚क‚त्वेन स्थित‚योः प्र‚क‚र‚ण‚त्वं नान्य‚थेत्युभ‚य‚स्व‚भाव‚{\tiny $_{lb}$}‚त्वात् प्र‚क‚र‚ण‚स्याभिधेय‚प्र‚योज‚नाभिधाने प्र‚क‚र‚ण‚स्यैवाभिहितं भ‚व‚तीति ।
	\pend% ending standard par
      ‚{\tiny $_{lb}$}‚

	  \pstart \leavevmode% starting standard par
	य‚द्येवं श‚रीर‚त्वाविशेषात् अभिधेय‚स्येवाभिधानात्म‚नोपि त‚त् किन्नोच्य‚ते ? किञ्च, न‚{\tiny $_{lb}$}‚ निष्कृष्ट‚रूपेऽभिधेये पुरुषः प्र‚व‚र्त्त‚ते किन्तु ग्र‚न्थ एव श्र‚व‚ण‚ल‚क्ष‚णां प्र‚वृत्तिमाच‚र‚ति अभिधेय‚ज्ञानाय ।‚{\tiny $_{lb}$}‚ त‚द‚स्यैव प्र‚योज‚नं वाच्य‚मिति पूर्व‚प‚क्ष‚द्वित‚यं प‚श्य‚न्नाह--\textbf{त‚त्रेति} निर्धा\leavevmode\ledsidenote{\textenglish{4b/ms}}र‚णे चैत‚त् ।‚{\tiny $_{lb}$}‚ \textbf{नान्य‚दिति} पुरुष‚प्र‚वृत्त्युप‚योगि ।
	\pend% ending standard par
      ‚{\tiny $_{lb}$}‚

	  \pstart \leavevmode% starting standard par
	अय‚म‚स्याश‚यः--न हि श‚ब्द‚स्य स्वार्थ‚प्र‚त्याय‚न‚ल‚क्ष‚णं फ‚ल‚म‚स्तीति श‚ब्द‚स‚न्द‚र्भ आर‚भ्य‚ते‚{\tiny $_{lb}$}‚ श्रूय‚ते वा, त‚स्य काक‚द‚न्त‚प‚रीक्षासाधार‚ण‚त्वेनाग‚ण्य‚मान‚त्वात् । किन्त‚र्हि ? त‚द‚र्थ‚स्य‚{\tiny $_{lb}$}‚ स‚प्र‚योज‚न‚त्वेन । अतः किम‚नेनोक्तेनापीति ? त‚र्हि अभिधेय‚स्यापि त‚त् किमुच्य‚त इत्याह—‚{\tiny $_{lb}$}‚\textbf{अभिधेयं तु} इत्यादि । \textbf{तु}र‚भिधानाद‚भिधेयं भेद‚व‚द्द‚र्श‚य‚ति । \textbf{अपि}श‚ब्दो \textbf{नार‚म्भ‚णीय} इत्य‚स्मात्प‚रो‚{\tiny $_{lb}$}‚ द्र‚ष्ट‚व्यः । त‚द‚य‚म‚र्थः--त‚त्प्र‚तिप‚त्त‚ये श‚ब्दानां स‚न्द‚र्भो नार‚म्भ‚णीयोऽपि स्यात्, किं पुनः‚{\tiny $_{lb}$}‚ श्र‚व‚णीय इति ।
	\pend% ending standard par
      ‚{\tiny $_{lb}$}‚‚{\tiny $_{lb}$}‚\textsuperscript{\textenglish{8/dm}}‚{\tiny $_{lb}$}‚
	  \bigskip
	  \begingroup
	

	  \pstart \leavevmode% starting standard par
	य‚था काक‚द‚न्त‚प्र‚योज‚नाभावात् न त‚त्प‚रीक्षा आर‚म्भ‚णीया प्रेक्षाव‚ता ।
	\pend% ending standard par
       ‚{\tiny $_{lb}$}‚ 

	  \pstart \leavevmode% starting standard par
	त‚स्माद‚स्य प्र‚क‚र‚ण‚स्यार‚म्भ‚णीय‚त्वं द‚र्श‚य‚ता अधिय‚प्र‚योज‚न‚म‚नेनोच्य‚ते । य‚स्मात्‚{\tiny $_{lb}$}‚ स‚म्य‚ग्ज्ञान‚पूर्विका स‚र्व‚पुरुषार्थ‚सिद्धिः, त‚स्मात् त‚त्प्र‚तिप‚त्त‚ये\edtext{}{\lemma{ये}\Bfootnote{त‚त्प्र‚तिप‚त्त्य‚र्थ‚मिद‚म् \cite{dp-msA} \cite{dp-edE} \cite{dp-edN} \cite{dp-edH} \cite{dp-edP}}} इद‚मार‚भ्य‚त इत्य‚य‚म‚त्र वाक्यार्थः ।
	\pend% ending standard par
       ‚{\tiny $_{lb}$}‚ 

	  \pstart \leavevmode% starting standard par
	\edtext{\textsuperscript{*}}{\lemma{*}\Bfootnote{अत्र च एवंप्र‚कार‚वाक्ये--\cite{dp-msD-n}}}अत्र च प्र‚क‚र‚णाभिधेय‚स्य स‚म्य‚ग्ज्ञान‚स्य स‚र्व‚पुरुषार्थ‚सिद्धिहेतुत्वं प्र‚योज‚न‚मुक्त‚म् ।
	\pend% ending standard par
      
	  \endgroup
	‚{\tiny $_{lb}$}‚

	  \pstart \leavevmode% starting standard par
	\textbf{य‚थे}ति सामान्येनोक्त‚स्यार्थ‚स्य विष‚योप‚द‚र्श‚ने । य‚थैत‚द्द‚र्शितं त‚द्व‚त् स‚र्वं द्र‚ष्ट‚व्य‚मिति‚{\tiny $_{lb}$}‚ \textbf{य‚था}श‚ब्दार्थः ।
	\pend% ending standard par
      ‚{\tiny $_{lb}$}‚

	  \pstart \leavevmode% starting standard par
	\textbf{त‚त्प‚रीक्षे}तीक्षेत्थ‚मित्थ‚ञ्चेति क‚र्मोप‚देशः । प‚द‚संह‚तिरिति च बुद्धिस्थ‚म् । प‚रीक्ष‚णं‚{\tiny $_{lb}$}‚ वा \textbf{प‚रीक्षा} विमृष्याव‚धार‚णं ताद‚र्थ्यात् शास्त्र‚म‚पि त‚था ।
	\pend% ending standard par
      ‚{\tiny $_{lb}$}‚

	  \pstart \leavevmode% starting standard par
	न‚नु निष्प्र‚योज‚नाभिधेयं मार‚म्भि, व‚च‚नेन त्व‚नेन किं क्रिय‚त इत्याह--\textbf{त‚स्माद्} इत्यादि ।‚{\tiny $_{lb}$}‚ य‚स्मान्निष्प्र‚योज‚नाभिधेयं नार‚भ्य‚ते \textbf{त‚स्मादार‚म्भ‚णीय‚त्वं द‚र्श‚य‚ता} आर‚म्भ‚योग्य‚मेवेदं म‚यार‚भ्य‚त‚{\tiny $_{lb}$}‚ इति प्र‚काश‚य‚ता, आर‚म्भ‚णीय‚त्वे च द‚र्शिते श्र‚व‚णीय‚त्व‚म‚पि निमित्त‚साम्याद् द‚र्शितं भ‚व‚तीति ते‚{\tiny $_{lb}$}‚ प्र‚व‚र्त्तिता भ‚व‚न्ति ।
	\pend% ending standard par
      ‚{\tiny $_{lb}$}‚

	  \pstart \leavevmode% starting standard par
	न‚नु च \textbf{स‚म्य‚ग्ज्ञान‚पूर्विका स‚र्व‚पुरुषार्थ‚सिद्धि}रित्य‚नेनैक‚देशेनाभिधेय‚प्र‚योज‚न‚मुव‚त‚म्, न‚{\tiny $_{lb}$}‚ स‚मुदितेन । त‚त् क‚थ‚म् अनेन वाक्येनोच्य‚ते इत्युच्य‚ते ? उच्य‚ते । वाक्यं हि नामैक‚स्मिन्न‚र्थे‚{\tiny $_{lb}$}‚ प्राधान्येन प्र‚तिपाद्ये गुण‚गुणिभाव‚म‚नुभ‚व‚ताम‚र्थ‚द्वारेणान्योन्यापेक्षिणां स‚म्ब‚द्धानां प‚दानां स‚मूह‚{\tiny $_{lb}$}‚ उच्य‚ते । त‚त्र य‚दि स‚र्वेषां प‚दार्थानां प्राधान्यं स्यात् त‚दा प‚र‚स्प‚रानुप‚कारात् स‚म्ब‚न्ध एव‚{\tiny $_{lb}$}‚ प‚दानां न स्यादिति वाक्य‚रूप‚तैव हीयेतेति सूक्त‚म‚नेनेति । अभिधेय‚प्र‚योज‚नाभिधान‚मेवास्य‚{\tiny $_{lb}$}‚ \textbf{स‚म्य‚ग्ज्ञाने}त्यादिप‚द‚स‚मूहात्म‚क‚स्य य‚था त‚था \textbf{य‚स्मादि}त्यादिना क‚ण्ठोक्तं क‚रोति । \textbf{त‚स्य}‚{\tiny $_{lb}$}‚ स‚म्य‚ग्ज्ञान‚स्य \textbf{प्र‚तिप‚त्त‚ये} शिष्य‚स्येत्य‚र्थात् । एत‚च्च \textbf{त‚द् व्युत्पाद्य‚त} इत्य‚स्य साम‚र्थ्याव‚स्थितार्थ‚{\tiny $_{lb}$}‚ क‚थ‚नं द्र‚ष्ट‚व्य‚म् । \textbf{इति}ना वाक्यार्थ‚स्य स्व‚रूप‚म्, इद‚मा च बुद्धिसिद्ध‚त्वेन त‚देवाङ्गुली‚{\tiny $_{lb}$}‚व्य‚प‚देश‚योग्य‚मिव द‚र्श‚य‚ति । अनेन च पुरुषार्थ‚सिद्धेरुपेय‚त्वात् प्राधान्य‚म्, प्र‚तिपाद्य‚मान‚स्य च‚{\tiny $_{lb}$}‚ स‚म्य‚ग्ज्ञान‚स्य त‚दुपाय‚त्वाद‚प्राधान्य‚म् । अत एव चार्थेन न्यायेन पुरुषार्थ‚सिद्धेर्वाक्यार्थ‚त्व‚म् ।‚{\tiny $_{lb}$}‚ शाब्द्या तु वृत्त्या शिष्य‚स‚म्य‚ग्ज्ञान‚विष‚या व्युत्प‚त्तिक्रिया वाक्यार्थ इति द‚र्शित‚म् ।
	\pend% ending standard par
      ‚{\tiny $_{lb}$}‚

	  \pstart \leavevmode% starting standard par
	\textbf{न‚नु स‚म्य‚ग्ज्ञाने}त्यादिना वाक्येन स‚म्य‚ग्ज्ञान‚स्य पुरुषार्थ‚सिद्धौ हेतुभावः प‚रं प्र‚द‚र्शितो न‚{\tiny $_{lb}$}‚ पुन‚रिदं त‚स्य प्र‚योज‚न‚मिति द‚र्शित‚म् । त‚स्यैव प्र‚योज‚न‚म‚नेनोच्य‚त इति चोक्त‚म् । त‚त् क‚थं‚{\tiny $_{lb}$}‚ युज्य‚ते ? अथ न\edtext{}{\lemma{न}\Bfootnote{चै}}ताव‚तापि न ज्ञाय‚ते क‚स्य किं त‚द‚भिधेयं किञ्च त‚स्य प्र‚योज‚न‚मित्या‚{\tiny $_{lb}$}‚श‚ङ्क्याह--\textbf{अत्र चे}त्यादि ।
	\pend% ending standard par
      ‚{\tiny $_{lb}$}‚

	  \pstart \leavevmode% starting standard par
	अथ स‚म्य‚ग्ज्ञान‚स्य पुरुषार्थ‚सिद्धिहेतुत्वं प्र‚योज‚न‚मित्य‚व‚द्य‚म् । मूल‚विरोधाद् युक्ति‚{\tiny $_{lb}$}‚विरोधाच्च । त‚था हि पुरुषार्थ‚सिद्धिमेवाचार्यीयं \textbf{स‚म्य‚ग्ज्ञानेत्या}दिव‚च‚नं प्र‚योज‚न‚म‚नुजानाति न‚{\tiny $_{lb}$}‚ तु पुरुषार्थ‚सि\leavevmode\ledsidenote{\textenglish{5a/ms}}द्धिहेतुत्व‚म् । अव‚य‚वार्थ‚व्याख्याने त‚स्य तामेव प्र‚योज‚न‚त‚या व्य‚क्ती‚{\tiny $_{lb}$}‚क‚रिष्य‚ति । न च हेतुत्वं ह‚तोर्भिन्न‚मिष्य‚ते युज्य‚ते वा । त‚त् क‚थ‚म‚स्यैव स‚म्य‚ग्ज्ञान‚स्य प्र‚योज‚नं‚{\tiny $_{lb}$}‚ ‚{\tiny $_{lb}$}‚ \leavevmode\ledsidenote{\textenglish{9/dm}}‚{\tiny $_{lb}$}‚ 
	  
	\edtext{\textsuperscript{*}}{\lemma{*}\Bfootnote{न‚नु अभिधेय‚प्र‚योज‚नाभिधानेपि याव‚च्छास्त्र‚स्य स‚म्ब‚न्धादीनि नोक्तानि ताव‚त्‚{\tiny $_{lb}$}‚ प्रेक्षाव‚न्तो न त‚त्र प्र‚व‚र्त्त‚न्ते इत्याह--त[[अ]]त्रेत्यादि । न‚नु त‚थाप्य‚स्म‚दुक्त‚स्य किमुत्त‚र‚{\tiny $_{lb}$}‚ मित्याह--अस्मिंश्चार्थे--\cite{dp-msD-n}}}म‚स्मिंश्चार्थ उच्य‚माने स‚म्ब‚न्ध‚प्र‚योज‚नाभिधेयानि\edtext{}{\lemma{नाभिधेयानि}\Bfootnote{अभिधेयार्थेऽक‚थितान्य‚पि साम‚र्थ्यादुक्तानि भ‚व‚न्तीत्य‚र्थः--\cite{dp-msD-n}}} उक्तानि\edtext{}{\lemma{उक्तानि}\Bfootnote{अभिधेय‚प्र‚त्याग‚ते--}} भ‚न्वित\add{}‚{\tiny $_{lb}$}‚ भ‚विष्य‚ति । न चात्र प्र‚योज‚नं प्र‚योज‚कं वाच्य‚म्, उक्त‚न्यायात् । इहापि प्र‚वृत्तिस‚म्ब‚न्धिन‚{\tiny $_{lb}$}‚ उपादान‚प्र‚स‚ङ्गाच्च । न च राज‚शास‚नं सामान्यं किञ्चिद‚स्ति येन भाव‚प्र‚धानः प्र‚योज‚न‚श‚ब्दः‚{\tiny $_{lb}$}‚ क‚ल्प्येत । त‚दिष्टौ पूर्वोक्तं दूष‚णं प‚राव‚र्त्तेतेति ।
	\pend% ending standard par
      ‚{\tiny $_{lb}$}‚

	  \pstart \leavevmode% starting standard par
	स‚त्य‚म् । न पुरुषार्थ‚सिद्धिहेतुत्वं स‚म्य‚ग्ज्ञान‚स्य प्र‚योज‚न‚म‚भिप्रेत‚म् । किन्त‚र्हि ?‚{\tiny $_{lb}$}‚ पुरुषार्थ‚सिद्धिरेव ।
	\pend% ending standard par
      ‚{\tiny $_{lb}$}‚

	  \pstart \leavevmode% starting standard par
	अशाब्दिकाऽसौ क‚थ‚मुच्य‚तामिति चेत् । उच्य‚ते । च‚श‚ब्दोऽत्र य‚स्माद‚र्थे । त‚तो‚{\tiny $_{lb}$}‚ य‚स्मात् प्र‚क‚र‚णाभिधेय‚स्य स‚म्य‚ग्ज्ञान‚स्य पुरुषार्थ‚सिद्धिहेतुत्व‚मुक्त‚म्, त‚स्मात् प्र‚योज‚न‚मुक्त‚मिति‚{\tiny $_{lb}$}‚ द्विराव‚र्त‚नीय‚म् । त‚च्चात्रार्थात् पुरुषार्थ‚सिद्धिरेवाव‚तिष्ठ‚ते । स‚म्य‚ग्ज्ञान‚स्य तां प्र‚ति हेतुत्वोक्तौ‚{\tiny $_{lb}$}‚ च सा प्र‚योज‚न‚मुक्तेति किं साक्षाद्वाच‚केन प‚देन क्रिय‚त इति भावः । एताव‚तैव ताव‚दिदं‚{\tiny $_{lb}$}‚ स‚माधीय‚ते । ब‚ह‚वः पुन‚र‚त्रायास्य‚न्ति ।
	\pend% ending standard par
      ‚{\tiny $_{lb}$}‚

	  \pstart \leavevmode% starting standard par
	स‚र्वा पुरुषार्थ‚सिद्धिर्य‚तो हेतुत्वात् त‚त् स‚र्व‚पुरुषार्थ‚सिद्धि त‚थाविधं हेतुत्वं य‚त्र प्र‚योज‚नं‚{\tiny $_{lb}$}‚ त‚त् त‚था । त‚च्च पुरुषार्थ‚सिद्धिरेवेति । अन‚या व्युत्प‚त्त्या अनेन श‚ब्देन सैवोक्ता ।
	\pend% ending standard par
      ‚{\tiny $_{lb}$}‚

	  \pstart \leavevmode% starting standard par
	क‚स्य तादृश‚मित्य‚पेक्षायामिद‚मुक्त‚म्--\textbf{प्र‚क‚र‚णाभिधेय‚स्य} स‚म्य‚ग्ज्ञान‚स्येति केचित् प्र‚त्ति‚{\tiny $_{lb}$}‚प‚द्य‚न्ते । अन्ये त्व‚र्ष \edtext{}{\lemma{र्ष}\Bfootnote{र्श}} \edtext{\textsuperscript{*}}{\lemma{*}\Bfootnote{पाणिनि \cite{dp-msD-n} ५. २. १२७}}आदित्वे प्र‚योज‚न‚म‚स्यास्तीति म‚त्व‚र्थीय‚म‚तं विधाय प्र‚धान‚व‚न्निर्देशं‚{\tiny $_{lb}$}‚ विव‚क्षित्वा प्र‚योज‚न‚व‚त्त्व‚मिति प्र‚तिजान‚ते । एवं चार्थं स‚म‚र्थ‚य‚न्ति--य‚त एव पुरुषार्थ‚सिद्धिः‚{\tiny $_{lb}$}‚ स‚म्य‚ग्ज्ञान‚साध्या त‚त एव सा प्र‚योज‚न‚म् । त‚स्यापि य‚देव तां प्र‚ति हेतुत्वं त‚देव‚{\tiny $_{lb}$}‚ प्र‚योज‚न‚व‚त्त्व‚मिति ।
	\pend% ending standard par
      ‚{\tiny $_{lb}$}‚

	  \pstart \leavevmode% starting standard par
	एके तु य‚तः स‚म्य‚ग्ज्ञाने स‚ति पुरुषार्थ‚सिद्धिहेतुत्वं व्य‚व‚स्थाप्य‚ते, त‚तः प्र‚माण-फ‚ल‚व‚द्‚{\tiny $_{lb}$}‚ व्य‚व‚स्थाप्य-व्य‚व‚स्थाप‚न‚भाव‚माश्रित्येद‚मुक्तं त‚तो न किञ्चिद‚व‚द्य‚मिति म‚न्य‚न्ते ।
	\pend% ending standard par
      ‚{\tiny $_{lb}$}‚

	  \pstart \leavevmode% starting standard par
	अप‚रे तु \textbf{च}श‚ब्दं भिन्न‚क्र‚मं कृत्वा पुरुषार्थ‚सिद्धिहेतुत्व‚मुक्त‚म्, प्र‚योज‚नं चोक्त‚मिति‚{\tiny $_{lb}$}‚ योज‚य‚न्ति ।
	\pend% ending standard par
      ‚{\tiny $_{lb}$}‚

	  \pstart \leavevmode% starting standard par
	इत‚रे तु पुरुषार्थ‚सिद्धिहेतुश‚ब्देन कार‚णे कार्य‚मुप‚च‚र्य पुरुषार्थ‚सिद्धिमेवाभिद‚ध‚ति । अन‚न्यो‚{\tiny $_{lb}$}‚पाय‚साध्य‚ताद‚र्श‚नार्थं चोप‚चार‚क‚र‚णं स‚माद‚ध‚ते ।
	\pend% ending standard par
      ‚{\tiny $_{lb}$}‚

	  \pstart \leavevmode% starting standard par
	केचित्तु निब‚न्ध‚कृतः क‚थ‚ञ्चिद‚पीदं स‚म‚र्थ‚यितुम‚नीशानाः स‚हित‚श‚ब्द‚पातात् प्र‚माद‚पाठ‚{\tiny $_{lb}$}‚ एवेति व‚र्ण‚य‚न्ति ।
	\pend% ending standard par
      ‚{\tiny $_{lb}$}‚

	  \pstart \leavevmode% starting standard par
	अत्र सारासारं स‚न्त एव विवेच‚यिष्य‚न्ति ।
	\pend% ending standard par
      ‚{\tiny $_{lb}$}‚

	  \pstart \leavevmode% starting standard par
	स्यादेत‚त् । य‚था प्र‚वृत्त्य‚ङ्ग‚त‚या अभिधेय‚प्र‚योज‚न‚मुच्य‚ते, त‚द्व‚त् स‚म्ब‚न्धादिक‚म‚पि‚{\tiny $_{lb}$}‚ किं नोच्य‚त इत्याह--\textbf{अस्मिंश्चेति । चो} य‚स्माद‚र्थे ।‚{\tiny $_{lb}$}‚ ‚{\tiny $_{lb}$}‚ \leavevmode\ledsidenote{\textenglish{10/dm}}‚{\tiny $_{lb}$}‚ 
	  
	\edtext{\textsuperscript{*}}{\lemma{*}\Bfootnote{स‚र्व‚पुरुषार्थ‚सिद्धिहेतुः स‚म्य‚ग्ज्ञान‚म् । अस्मिन्न‚र्थे उच्य‚माने क‚थं स‚म्ब‚न्धादीन्युक्तानि-\cite{dp-msD-n}}}त‚थाहि--पुरुषार्थोप‚योगि स‚म्य‚ग्ज्ञानं व्युत्पाद‚यित‚व्य‚म‚नेन प्र‚क‚र‚णेनेति ब्रुव‚ता‚{\tiny $_{lb}$}‚ स‚म्य‚ग्ज्ञान‚म‚स्य श‚ब्द‚स‚न्द‚र्भ‚स्याभिधेय‚म्, त‚द्व्युत्पाद‚नं प्र‚योज‚न‚म्, प्र‚क‚र‚णं चेद‚म‚पायो व्युत्पाद‚न‚{\tiny $_{lb}$}‚स्येत्युक्तं भ‚व‚ति । ‚{\tiny $_{lb}$}‚ 
	  
	त‚स्माद‚भिधेय‚भाग‚प्र योज‚नाभिधान‚साम‚र्थ्थात् स‚म्ब‚न्धादीनि उक्तानि भ‚व‚न्ति । \edtext{\textsuperscript{*}}{\lemma{*}\Bfootnote{न‚न्वादिवाक्ये य‚थाऽभिधेय‚प्र‚योज‚न‚म‚भिध‚त्ते एवं स‚म्ब‚न्धादि किमिति न व‚क्ति ?‚{\tiny $_{lb}$}‚ इक्याह--\cite{dp-msD-n}}}न‚{\tiny $_{lb}$}‚ त्विद‚मेकं धाक्यं स‚म्ब‚न्ध‚म्, अभिधेय‚म्, प्र‚योज‚नं च व‚क्तुं साक्षात् स‚म‚र्थ‚म् ।\edtext{\textsuperscript{*}}{\lemma{*}\Bfootnote{य‚दि साक्षान्न व‚क्ति क‚थं त‚र्हि स‚म‚र्थं त‚द्द‚र्श‚ने इत्याह--\cite{dp-msD-n}}} एकं तु व‚द‚त् त्र‚यं‚{\tiny $_{lb}$}‚ साम‚र्थ्यात् द‚र्श‚य‚ति । \edtext{\textsuperscript{*}}{\lemma{*}\Bfootnote{\textbf{अथ वार्त्तिकेन} साम‚र्थ्य‚ल‚ब्ध‚म‚भिधेयं प्र‚योज‚नं चाह--\cite{dp-msD-n}}}त‚त्र--‚{\tiny $_{lb}$}‚ क‚थं पुन‚र‚न्य‚स्योक्ताव‚न्य‚दुक्तं भ‚व‚तीत्याह--\textbf{त‚था हि} इति । निपात‚स‚मुदाय‚श्चायं‚{\tiny $_{lb}$}‚ य‚स्मादित्य‚स्यार्थे स‚र्व‚त्र व‚र्त्त‚ते ।
	\pend% ending standard par
      ‚{\tiny $_{lb}$}‚

	  \pstart \leavevmode% starting standard par
	\textbf{पुरुषार्थो} हेयोपादेय‚हानोपादान‚ल‚क्ष‚ण \textbf{उप‚यो}गो व्यापारः । सोऽस्यास्तीति त‚था ।
	\pend% ending standard par
      ‚{\tiny $_{lb}$}‚

	  \pstart \leavevmode% starting standard par
	एवं स‚ति किं सिद्ध‚मित्याह--\textbf{त‚स्माद्} इति । \textbf{त‚च्छ}ब्देनान‚न्त‚रोक्तो वाक्यार्थः‚{\tiny $_{lb}$}‚ प्र‚त्य‚व‚द्र\edtext{}{\lemma{द्र}\Bfootnote{म्र}}ष्ट‚व्यः ।
	\pend% ending standard par
      ‚{\tiny $_{lb}$}‚

	  \pstart \leavevmode% starting standard par
	\textbf{अभिधेय‚भागोऽङ्ग} एक‚देशः । स‚म्ब‚न्धादिभागापेक्ष‚या ।
	\pend% ending standard par
      ‚{\tiny $_{lb}$}‚

	  \pstart \leavevmode% starting standard par
	\textbf{स‚म्ब‚न्धादी}त्यादिग्र‚ह‚णेनाभिधेय‚प्र‚योज‚न‚योः संग्र‚हः । \leavevmode\ledsidenote{\textenglish{5b/ms}} \textbf{उक्तानीति} उक्तानीव‚{\tiny $_{lb}$}‚ \textbf{उक्तानि प्र}काशितानि । न तु त‚त्राऽभिधाऽस्य स‚म्भ‚विनी ।
	\pend% ending standard par
      ‚{\tiny $_{lb}$}‚

	  \pstart \leavevmode% starting standard par
	अथाभिधेय‚प्र‚योज‚नं य‚था वाक्येन साक्षादुच्य‚ते, त‚था स‚म्ब‚न्धादीन्य‚पि किन्नोच्य‚न्त‚{\tiny $_{lb}$}‚ इत्याह--\textbf{न त्विति} । तुर‚व‚धार‚य‚ति विशिन‚ष्टि वा ।
	\pend% ending standard par
      ‚{\tiny $_{lb}$}‚

	  \pstart \leavevmode% starting standard par
	अय‚माश‚यः--य‚दि स‚र्वेप‚दार्थाः प्राधान्य‚म‚श्नुवीर‚न्, वाक्य‚मेव त‚दा विशीर्येतान्योन्याऽव्य‚{\tiny $_{lb}$}‚पेक्षाभावेनैकार्थाप्र‚त्याय‚नात् । त‚तो वाक्य‚मेकार्थ‚म‚भिधेय‚त‚या उपादातुं क‚ल्प्य‚ते नानेक‚म् ।
	\pend% ending standard par
      ‚{\tiny $_{lb}$}‚

	  \pstart \leavevmode% starting standard par
	य‚दि साक्षान्न स‚म‚र्थ‚म्, क‚थं नास‚म‚र्थ‚म् ? अथास‚म‚र्थ‚मेव । न साक्षात् स‚म‚र्थ‚मिति‚{\tiny $_{lb}$}‚ त‚र्हि न वाच्य‚मिति । आह--\textbf{एकं तु}--इति । \textbf{तु}र‚त्रापिव‚द् ग्राह्यः । साक्षाद्वाच्य‚त्वेन‚{\tiny $_{lb}$}‚ एकं व्य‚व‚च्छिन‚त्ति । \textbf{द‚र्श‚य‚ति} प्र‚काश‚य‚ति, त‚त्र वाक्य‚स्याभिधाव्यापारासंभ‚वात् । अत एवो‚{\tiny $_{lb}$}‚क्तानीवोक्तानीति त‚थाऽस्माभिर्व्याख्यात‚म्, अन्य‚था त‚द‚नेन विरुध्येत ।
	\pend% ending standard par
      ‚{\tiny $_{lb}$}‚

	  \pstart \leavevmode% starting standard par
	न‚नु य‚दि नामाभिधेयादिकं वाक्य‚स्य प्र‚काश्य‚म्, नाभिधेय‚म् त‚थापि प‚दार्थेन तेनाव‚श्यं‚{\tiny $_{lb}$}‚ भाव्य‚म् । त‚था च क‚स्य किं वाच‚कं प‚द‚मित्याश‚ङ् क्याह--\textbf{त‚त्र} इत्यादि--तेषु अभिधेयादिषु ।‚{\tiny $_{lb}$}‚ ‚{\tiny $_{lb}$}‚ \leavevmode\ledsidenote{\textenglish{11/dm}}‚{\tiny $_{lb}$}‚ 
	  
	त‚द् इति अभिधेय‚प‚द‚म् । \edtext{\textsuperscript{*}}{\lemma{*}\Bfootnote{व्युत्पाद्य‚त इति णिच्निर्द्देशात् प्र‚योक्तृप्र‚योज्य‚विष‚यं प्र‚योज‚नं वाच्य‚मित्याह--\cite{dp-msD-n}}}व्युत्पाद्य‚त इति प्र‚योज‚न‚प‚द‚म्\edtext{}{\lemma{म्}\Bfootnote{प्र‚योज‚न‚मिद‚म्--\cite{dp-msA} \cite{dp-edP} \cite{dp-edH}}} । प्र‚योज‚नं चात्र व‚क्तुः\edtext{}{\lemma{क्तुः}\Bfootnote{शास्त्र‚क‚र्त्तुः--\cite{dp-msD-n}}}‚{\tiny $_{lb}$}‚ प्र‚क‚र‚ण‚क‚र‚ण‚व्यापार‚स्य चिन्त्य‚ते, श्रोतुश्च श्र‚व‚ण‚व्यापार‚स्य । ‚{\tiny $_{lb}$}‚ 
	  
	त‚था हि--स‚र्वे प्रेक्षाव‚न्तः प्र‚वृत्तिप्र‚योज‚न‚म‚न्विष्य प्र‚व‚र्त्त‚न्ते । त‚त‚श्चाचार्येण\edtext{}{\lemma{श्चाचार्येण}\Bfootnote{त‚त आचा० \cite{dp-msA} \cite{dp-edP}}} प्र‚क‚र‚णं‚{\tiny $_{lb}$}‚ किम‚र्थं कृत‚म्, श्रोतृभिश्च किम‚र्थं श्रूय‚त इति संश‚ये\edtext{}{\lemma{ये}\Bfootnote{संश‚य‚व्यु० \cite{dp-edP} \cite{dp-edH}}} व्युत्पाद‚नं प्र‚योज‚न‚म‚भिधीय‚ते । स‚म्य‚ग्ज्ञानं‚{\tiny $_{lb}$}‚ \edtext{\textsuperscript{*}}{\lemma{*}\Bfootnote{व्युत्पाद्य‚मा० \cite{dp-msB} \cite{dp-edE} \cite{dp-edH} \cite{dp-edN} विबुध्य‚मानानाम्--\cite{dp-msD-n}}}व्युत्प‚द्य‚मानानामात्मानं व्युत्पाद‚कं क‚र्त्तुं प्र‚क‚र‚ण‚मिदं कृत‚म् । शिष्यैश्चाचार्य‚प्र‚युक्तामात्म‚नो‚{\tiny $_{lb}$}‚ व्युत्प‚त्तिमिच्छ‚दिभः प्र‚क‚र‚ण‚मिदं\edtext{}{\lemma{मिदं}\Bfootnote{प्र‚क‚र‚णं श्रूय० \cite{dp-msC} \cite{dp-msD} \cite{dp-msB}}} श्रूय‚त इति प्र‚क‚र‚ण‚क‚र‚ण-श्र‚व‚ण‚योः प्र‚योज‚नं\edtext{}{\lemma{नं}\Bfootnote{प्र‚योज‚न‚व्यु० \cite{dp-edH} \cite{dp-edN}}} व्युत्पाद‚न‚म् ।‚{\tiny $_{lb}$}‚ त‚दिति द्वितीयान्त‚मेत‚त् । य‚तो \textbf{व्युत्पाद्य‚त} इत्य‚त्र ल‚कारः प्र‚योज्य‚क‚र्म‚णि विहितो न प्र‚योज्य‚{\tiny $_{lb}$}‚क्रियाक‚र्म‚णीति । वाक्येऽपि लोके प‚द्य‚ते ग‚म्य‚तेऽनेनाऽर्थ इति व्युत्प‚त्त्या प‚द‚प्र‚योगो दृश्य‚ते ।‚{\tiny $_{lb}$}‚ त‚तो \textbf{व्युत्पाद्य‚त इति प्र‚योज‚न‚प‚द}मित्याह । अन्य‚था प‚द‚वृन्द‚मिदं न प‚द‚म् । य‚द्वा \textbf{व्युत्पाद्य‚त}‚{\tiny $_{lb}$}‚ इत्य‚त्रेति द्र‚ष्ट‚व्य‚म् ।
	\pend% ending standard par
      ‚{\tiny $_{lb}$}‚

	  \pstart \leavevmode% starting standard par
	\hphantom{.}न‚नूक्त‚म्--प्र‚योज‚नं पुरुषार्थ‚सिद्धिः । त‚त् किं पुन‚श्चिन्त्य‚ते । क‚थ‚ञ्च त‚द‚न्य‚थोच्य‚त‚{\tiny $_{lb}$}‚ इत्याह--\textbf{प्र‚योज‚नं चात्र} इति । \textbf{चो} व‚क्त‚व्यान्त‚र‚स‚मुच्च‚ये । त‚द‚य‚म‚र्थः--नोक्तं प्र‚योज‚न‚मिति‚{\tiny $_{lb}$}‚ चिन्त्य‚ते किन्तु गुण‚भूतार्थ‚प्र‚योज‚क‚प‚द‚स्य व्युत्पाद्य‚त इत्य‚स्य प्र‚योज‚न‚म् । त‚द्व्युत्प‚त्तिः क्रिय‚त‚{\tiny $_{lb}$}‚ इति णिचः किं प्र‚योज‚न‚मिति प्र‚श्ने त‚स्य प्र‚योज‚नं चिन्त्य‚त इति याव‚त् ।
	\pend% ending standard par
      ‚{\tiny $_{lb}$}‚

	  \pstart \leavevmode% starting standard par
	\textbf{अत्र}--इति गुण‚भूतार्थ‚निरूप‚णे प्र‚क‚र‚ण‚क‚र‚ण‚मेव व्यापारः ।
	\pend% ending standard par
      ‚{\tiny $_{lb}$}‚

	  \pstart \leavevmode% starting standard par
	\hphantom{.}न‚नु क‚स्मादिय‚माश‚ङ्का--व‚क्त्रा किम‚र्थं क्रिय‚ते, श्रोतृभिश्च किम‚र्थ श्रूय‚ते इति । तौ‚{\tiny $_{lb}$}‚ ख‚ल्वेव‚मेव प्र‚व‚र्त्तेयातामित्याश‚ङ्क्याह--त‚था हि इति य‚स्मात् ।
	\pend% ending standard par
      ‚{\tiny $_{lb}$}‚

	  \pstart \leavevmode% starting standard par
	अथ क‚थ‚मुभ‚योर्व्युत्पाद‚नं प्र‚योज‚न‚मुच्य‚ते ? य‚द्य‚तो व्युत्प‚द्य‚ते त‚स्य व्युत्प‚त्तिः प्र‚योज‚न‚म् ।‚{\tiny $_{lb}$}‚ य‚स्तु व्युत्पाद‚य‚ति त‚स्य व्युत्पाद‚न‚मिति । नैत‚द‚स्ति । प्र‚योज‚क‚व्यापार‚व‚ती \textbf{हि} व्युत्प‚त्तिः‚{\tiny $_{lb}$}‚ प्र‚योज‚न‚मिष्टेत्युभ‚योर्व्युत्पाद‚न‚मेव प्र‚योज‚न‚म् । केव‚ल‚मेकोऽन्य‚था प्र‚व‚र्त्त‚तेऽन्य‚श्चान्य‚था । क‚थं‚{\tiny $_{lb}$}‚ नाम म‚द्व्यापार‚व‚शेन श्रोतृस‚न्तान‚व‚र्त्तिनी व्युत्प‚त्तिर्भूयादिति आचार्यः प्र‚व‚र्त्त‚ते । अत एव‚{\tiny $_{lb}$}‚ स्व‚प्र‚योज‚क‚व्यापाराधिष्ठिता श्रोतृस‚न्तान‚व‚र्त्तिनी व्युत्प‚त्तिः क‚र्त्तुः प्र‚योज‚न‚म् । एव‚मेवासौ‚{\tiny $_{lb}$}‚ व्युत्पाद‚को भ‚व‚ति । श्रोता तु क‚थं नामैत‚दाचार्य‚व्यापार‚व‚से\edtext{}{\lemma{से}\Bfootnote{शे}}न म‚त्संतान‚व‚र्तिनी व्युत्प‚त्ति‚{\tiny $_{lb}$}‚र्भूयादिति प्र‚व‚र्त‚ते ।
	\pend% ending standard par
      ‚{\tiny $_{lb}$}‚

	  \pstart \leavevmode% starting standard par
	अत एवाचार्य‚व्यापाराधिष्ठिता व्युत्प‚त्तिः श्रोतुर‚पि प्र‚योज‚न‚म् । एव‚मेवासौ त‚द्व्युत्पाद्यो‚{\tiny $_{lb}$}‚ ‚{\tiny $_{lb}$}‚ \leavevmode\ledsidenote{\textenglish{12/dm}}‚{\tiny $_{lb}$}‚ 
	  
	\edtext{\textsuperscript{*}}{\lemma{*}\Bfootnote{न‚नु य‚थाऽभिधेय‚प्र‚योज‚ने द‚र्शिते एवं स‚म्ब‚न्धादि द‚र्श्य‚तामित्याह--\cite{dp-msD-n}}}स‚म्ब‚न्ध‚प्र‚द‚र्श‚न‚प‚दं तु न विद्य‚ते । साम‚र्थ्यादेव तु स प्र‚तिप‚त्त‚व्यः । ‚{\tiny $_{lb}$}‚ 
	  
	प्रेक्षाव‚ता हि स‚म्य‚ग्ज्ञान‚व‚व्युत्पाद‚नाय प्र‚क‚र‚ण‚मिद‚मार‚ब्ध‚व‚ता अय‚मेवोपायो नान्यः‚{\tiny $_{lb}$}‚ इति द‚र्शित एवोपायोपेय‚भावः प्र‚क‚र‚ण‚प्र‚योज‚न‚योः स‚म्ब‚न्ध इति ।‚{\tiny $_{lb}$}‚ भ‚व‚ति । त‚तो व्युत्पाद‚नं प्र‚योज‚क‚व्यापार‚व‚श‚व‚र्त्तिनी व्युत्प‚त्तिर्द्व‚योर‚पि प्र‚योज‚न‚मिति इतिश‚ब्दं‚{\tiny $_{lb}$}‚ हेतुप‚दं कृत्वोप‚संह‚र‚न्नाह--\textbf{प्र‚क‚र‚ण‚क‚र‚ण‚श्र‚व‚ण‚योः प्र‚योज‚नं व्युत्पाद‚न‚मिति ।}
	\pend% ending standard par
      ‚{\tiny $_{lb}$}‚

	  \pstart \leavevmode% starting standard par
	न‚नु च णिचि कृते व्युत्पाद‚नं क‚थ‚मुभ‚योः प्र‚योज‚न‚म्, इत्थ‚मिति \leavevmode\ledsidenote{\textenglish{6a/ms}} प्र‚श्न‚विस‚र्ज‚ने‚{\tiny $_{lb}$}‚ स्याताम् । तेनैव ताव‚ता द‚र्शितेन किं प्र‚योज‚न‚म् ? स‚म्य‚ग्ज्ञान‚व्युत्प‚त्तौ क‚थ‚ञ्चिन्निमित्त‚{\tiny $_{lb}$}‚मात्रात् द‚धिभोज‚नादेराचार्य‚स्यासाधार‚ण‚कार‚ण‚ताप्र‚तिप‚त्तिः प्र‚योज‚न‚म् । अय‚म‚र्थः--अस्ति‚{\tiny $_{lb}$}‚ प्र‚योज‚क‚व्यापार‚प्र‚द‚र्श‚ने स‚म्य‚ग्ज्ञान‚व्युत्प‚त्तावुप‚योगित्व‚मात्रेण स्वास्थ्यादिना साम्य‚माचार्य‚स्य‚{\tiny $_{lb}$}‚ द‚र्शितं स्यात् । अस्ति चास्य त‚द्व्युत्प‚त्तावुप‚देश‚ल‚क्ष‚णोऽसाधार‚णो व्यापारः । स क‚थं नाम‚{\tiny $_{lb}$}‚ प्र‚तीयेतेति णिचा निर्देशः कृत इति ।
	\pend% ending standard par
      ‚{\tiny $_{lb}$}‚

	  \pstart \leavevmode% starting standard par
	अथ भोज‚नादेस्त‚द्व्युत्प‚त्तौ साक्षाद्व्यापाराऽस‚म्भ‚वाद् आचार्य‚व्यापारः प्र‚तिप‚त्स्य‚त इति‚{\tiny $_{lb}$}‚ चेत् । नैत‚त् । एवं हि व्याख्यातृणामिदं प्र‚तिपाद‚न‚कौश‚लं स्यान्न क‚र्त्तुरिति न्याय्यो णिचा‚{\tiny $_{lb}$}‚ निर्देशः ।
	\pend% ending standard par
      ‚{\tiny $_{lb}$}‚

	  \pstart \leavevmode% starting standard par
	न‚नूक्त‚म्--अभिधान‚स्य प्र‚योज‚नं न निरूप्य‚त इति । त‚त् किमिदानीं त‚देव व्याजान्त‚रेण‚{\tiny $_{lb}$}‚ निरूप्य‚ते ? स‚त्य‚म् । केव‚लं वाक्यार्थ‚त्वेन न निरूप्य‚त इत्य‚भिस‚न्धिना त‚न्न निरूप्य‚ते‚{\tiny $_{lb}$}‚ इत्युक्त‚म् । न तु प‚दार्थ‚त्वेनापीति किं विरोधः ?
	\pend% ending standard par
      ‚{\tiny $_{lb}$}‚

	  \pstart \leavevmode% starting standard par
	\hphantom{.}स‚म्य‚ग्ज्ञानं व्युत्प‚द्य‚मानानाम् इत्य‚साधुर‚यं श‚ब्दः--व्युत्प‚त्त्य‚र्थ‚स्य प‚देर‚क‚र्म‚त्वाद् इति‚{\tiny $_{lb}$}‚ \textbf{भाग‚वृत्तिकारः} । त‚न्नातिश्लिष्ट‚म् ज्ञानार्थ‚स्य प‚देः स‚क‚र्म‚क‚स्य व्युत्प‚न्नः स‚ङ्केतः इत्यादौ‚{\tiny $_{lb}$}‚ ब‚हुशः प्र‚योग‚द‚र्श‚नात् । य‚थाऽवादि \textbf{प्र‚माण‚वार्त्तिके वार्त्तिक‚कृता}--
	\pend% ending standard par
      ‚{\tiny $_{lb}$}‚

	  \pstart \leavevmode% starting standard par
	\hphantom{.}म‚नोऽव्यु\edtext{}{\lemma{नोऽव्यु}\Bfootnote{म‚नो व्यु}}त्प‚न्न‚संकेत‚म‚स्ति तेन स चेन्म‚तः \href{http://sarit.indology.info/?cref=pv.2.143}{प्र‚माण‚वा० २. १४३} इति ।
	\pend% ending standard par
      ‚{\tiny $_{lb}$}‚

	  \pstart \leavevmode% starting standard par
	य‚द्य‚भिधेयाद‚यः प‚दार्थाः, स‚म्ब‚न्ध‚स्त‚र्हि क‚स्य प‚द‚स्यार्थ इत्याह--\textbf{स‚म्ब‚न्धे}त्यादि ।
	\pend% ending standard par
      ‚{\tiny $_{lb}$}‚

	  \pstart \leavevmode% starting standard par
	\textbf{साम‚र्थ्याद्} इति नाभिधाव्यापारेण । \textbf{एव}कारेण प‚द‚स्याप्य‚भिधाम‚पोह‚ति । \textbf{तु}श‚ब्दः‚{\tiny $_{lb}$}‚ केव‚ल‚मित्य‚स्यार्थे ।
	\pend% ending standard par
      ‚{\tiny $_{lb}$}‚

	  \pstart \leavevmode% starting standard par
	क‚थं साम‚र्थ्यादित्याह--\textbf{प्रेक्षाव‚ता} इति । \textbf{उपायः} कार‚ण‚म् । \textbf{उपेया} स‚म्य‚ग्ज्ञान‚व्युत्प‚त्तिः‚{\tiny $_{lb}$}‚ साध्या । त‚यो\textbf{र्भावः} । य‚द्व‚शेनोपायोपेय‚ज्ञानाभिधाने भ‚व‚तः । \textbf{एव}कार \textbf{उपाय}श‚ब्दात् प‚रो‚{\tiny $_{lb}$}‚ द्र‚ष्ट‚व्यः । प्र‚तियोगिनोपायायोग‚स्यैव श‚ङ्कित‚त्वात् । त‚द‚न्य‚योग‚स्य च श‚ङ्किष्य‚माण‚त्वात् ।‚{\tiny $_{lb}$}‚ अन‚र्थाश‚ङ्कोप‚स्थापितान्य‚योग‚निरासे तु क‚र्त्त‚व्ये साम‚र्थ्य‚ग‚म्य‚म‚व‚धार‚णान्त‚रं कार्य‚म् । न त्विद‚मेव‚{\tiny $_{lb}$}‚ वाच्य‚म् । नान्य इति च नान्य एवेत्य‚व‚सेय‚म् ।
	\pend% ending standard par
      ‚{\tiny $_{lb}$}‚\textsuperscript{\textenglish{13/dm}}‚{\tiny $_{lb}$}‚
	  \bigskip
	  \begingroup
	

	  \pstart \leavevmode% starting standard par
	न‚नु च प्र‚क‚र‚ण‚श्र‚व‚णात् प्राग् उक्तान्य‚पि\edtext{}{\lemma{पि}\Bfootnote{उक्तान्य‚भिधे० \cite{dp-msB}}} अभिधेयादीनि प्र‚माणाभावात् प्रेक्षाव‚द्भिर्न‚{\tiny $_{lb}$}‚ गृह्य‚न्ते । त‚त् किमेतैरार‚म्भ‚प्र‚देशे उक्तैः ?
	\pend% ending standard par
       ‚{\tiny $_{lb}$}‚ 

	  \pstart \leavevmode% starting standard par
	स‚त्य‚म् । अश्रुते प्र‚क‚र‚णे क‚थितान्य‚पि न निश्चीय‚न्ते । उक्तेषु त्व‚प्र‚माण‚केष्व‚प्य‚{\tiny $_{lb}$}‚भिधेयादिषु संश‚य उत्प‚द्य‚ते । संश‚याच्च प्र‚व‚र्त्त‚न्ते । अर्थ‚संश‚योऽपि हि प्र‚वृत्त्य‚ङ्गं प्रेक्षाव‚ताम् ।
	\pend% ending standard par
      
	  \endgroup
	‚{\tiny $_{lb}$}‚

	  \pstart \leavevmode% starting standard par
	एवंरूप‚स्य स‚म्ब‚न्धो व्युत्पाद्य‚त इति प्र‚योज‚नाभिधानादेव द‚र्शित‚म् । त‚दुक्त‚म्--
	\pend% ending standard par
      ‚{\tiny $_{lb}$}‚
	  \bigskip
	  \begingroup
	
	    \begin{quote}
	  
	    
	    \stanza[\smallbreak]
	शास्त्रं प्र‚योज‚नं चैव स‚म्ब‚न्ध‚स्याश्र‚याद्ग‚तौ ।\edtext{\textsuperscript{*}}{\lemma{*}\Bfootnote{श्र‚यावुभौ--श्लोक‚वा०}}&त‚दुक्त्य‚न्त‚र्ग‚त‚स्त‚स्माभिन्नो नोक्तः प्र‚योज‚नाद् ॥ \href{http://sarit.indology.info/?cref=śv.1.18}{श्लोक‚वा० १. १८} इति\&[\smallbreak]


	
	    \end{quote}
	  
	  \endgroup
	‚{\tiny $_{lb}$}‚

	  \pstart \leavevmode% starting standard par
	अभिधेयादिप्र‚काश‚न‚द्वारेणादिवाक्यं प्र‚व‚र्त्त‚क‚मिति अस‚ह‚मान आह--न‚नु च इति ।‚{\tiny $_{lb}$}‚ निपात‚स‚मुदाय‚श्चायं चोद‚यामि, अभिमुखो भ‚व इत्य‚स्यार्थ‚स्य द्योत‚कः । श्र‚व‚णान‚न्त‚रं तेषां‚{\tiny $_{lb}$}‚ प्र‚तीय‚मान‚त‚या प्र‚माणाभावासिद्धेः \textbf{श्र‚व‚णात्प्राग्} इत्याह । अयं पूर्व‚प‚क्ष‚वादिनोऽभिप्रायः--श्र‚व‚णात्‚{\tiny $_{lb}$}‚ प्राक् श‚ब्दाश्र‚येण प्र‚व‚र्त्त‚मान‚म‚नुमान‚म‚न्य‚द्वा श‚ब्द‚स्यार्थ‚नान्त‚रीय‚तायां प्र‚व‚र्त्त‚यितुम‚र्ह‚ति ।‚{\tiny $_{lb}$}‚ अस‚म्ब‚न्ध‚निब‚न्ध‚नायाः प‚रोक्षार्थ‚प्र‚तिप‚त्तेः प्रामाण्यायोगात् । सा च श‚ब्द‚स्यास‚म्भ‚विनी ।‚{\tiny $_{lb}$}‚ आचार्य‚स्य चाप्त‚भावो दुर्बोधो येनाप्तोप‚देश‚त‚या श‚ब्दोऽर्थ‚त‚थात्वं प्र‚त्याय‚येदिति । आर‚भ्य‚त‚{\tiny $_{lb}$}‚ \textbf{इत्यार‚म्भः} प्र‚क‚र‚ण‚म् । त‚स्य \textbf{प्र‚देशः} एक‚देशः । साम‚र्थ्यात् त‚दादिः ।
	\pend% ending standard par
      ‚{\tiny $_{lb}$}‚

	  \pstart \leavevmode% starting standard par
	स‚च्चोद्ये \textbf{स‚त्य‚म्} इत्याह । य‚दि स‚त्य‚मिदं किम‚र्थं त‚र्हि \leavevmode\ledsidenote{\textenglish{6b/ms}} ते क‚थ्य‚न्त इति ? \textbf{उक्तेषु}‚{\tiny $_{lb}$}‚ इति । तुर‚नुक्त‚प‚क्षादुक्त‚प‚क्ष‚स्य विशेषं द‚र्श‚य‚ति । अनेनैत‚दाह--साध‚क‚बाध‚क‚प्र‚माणाभावे येन‚र्ते‚{\tiny $_{lb}$}‚ ब्याहारात् प्र‚तिनिय‚ताधिक‚र‚णो यौक्तः संश‚यो भ‚वितुम‚र्ह‚तीति । \textbf{अपि} न्याय‚तः स‚म्भाव‚नां‚{\tiny $_{lb}$}‚ द‚र्श‚य‚ति, अतिश‚यं वा । अन्य‚था किं स‚प्र‚माण‚केष्व‚पि संश‚यो जाय‚ते येनापिश्रुतिः संग‚च्छेत् ।
	\pend% ending standard par
      ‚{\tiny $_{lb}$}‚

	  \pstart \leavevmode% starting standard par
	न‚नु प्र‚वृत्तिफ‚ल‚मादिवाक्यं, प्र‚वृत्तिश्चेन्नास्ति किं संश‚येनोत्पादितेनापीत्याह--\textbf{संश‚याच्च}‚{\tiny $_{lb}$}‚ इति । \textbf{चो} य‚स्माद‚र्थे ।
	\pend% ending standard par
      ‚{\tiny $_{lb}$}‚

	  \pstart \leavevmode% starting standard par
	अथ ये ताव‚न्म‚दाः श्र‚द्ध‚या वाऽऽचार्य‚मुप‚स‚न्नास्ते त‚द्व‚च‚नाद‚र्थं निश्चित्यैव प्र‚व‚र्त्त‚माना न‚{\tiny $_{lb}$}‚ संश‚यात्प्र‚व‚र्त्त‚न्ते । येऽपि त‚द्विप‚रीताः संशेर‚ते \textbf{ते}षाम‚पि न संश‚यात्प्र‚वृत्तिः । प्र‚वृत्तौ वा‚{\tiny $_{lb}$}‚ प्रेक्षाव‚त्त्व‚हानिरित्याश‚ङ्क्याह--\textbf{अर्थ‚संश‚योपीति} । न केव‚ल‚म‚र्थ‚निश्च‚य इत्य‚पिश‚ब्देनाह । श‚क्य‚{\tiny $_{lb}$}‚निश्च‚ये हि निश्च‚य‚म‚न्त‚रेण प्र‚व‚र्त्त‚मानाः प्रेक्षाव‚त्ताया हीयेर‚न् । य‚त्र त्व‚र्थ‚संश‚येनार्थित‚या प्र‚व‚र्त्त‚{\tiny $_{lb}$}‚मानाः, नोपाल‚म्भ‚म‚र्ह‚न्तीति भावः । य‚दि च युक्ता द्वितीयाकारानुप्र‚वेशेन संश‚यान‚स्य न‚{\tiny $_{lb}$}‚ क्व‚चित्प्र‚वृत्तिः, त‚र्हि न क‚स्य‚चिद् व‚च‚नात् क्व‚चित् प्र‚व‚र्त्तित‚व्य‚मिति भूयान् व्य‚व‚हारो‚{\tiny $_{lb}$}‚ विलुप्येत ।
	\pend% ending standard par
      ‚{\tiny $_{lb}$}‚

	  \pstart \leavevmode% starting standard par
	न‚नु निवृत्तिर‚न‚र्थ‚निश्च‚य‚निब‚न्ध‚ना । अन‚र्थ‚निश्च‚य‚श्च प्र‚माणात् । त‚च्चेह नास्ति ।‚{\tiny $_{lb}$}‚ निवृत्तीत‚र‚स्तु प्र‚वृत्तिव्यातिरेकी प्र‚कारो नास्ति । त‚तोऽथ‚त्न‚सिद्धा प्र‚वृत्तिः । त‚त्किं त‚द‚र्थेनादि‚{\tiny $_{lb}$}‚‚{\tiny $_{lb}$}‚ \leavevmode\ledsidenote{\textenglish{14/dm}}‚{\tiny $_{lb}$}‚ 
	  
	अन‚र्थं संश‚योऽपि\edtext{}{\lemma{योऽपि}\Bfootnote{संश‚यो निवृ० \cite{dp-msA} \cite{dp-edP} \cite{dp-edH} \cite{dp-edE} \cite{dp-edN}}} निवृत्त्य‚ङ्ग‚म् । अत एव शास्त्र‚कारेणैव\edtext{}{\lemma{कारेणैव}\Bfootnote{अत्र‚व‚श‚ब्दो व्याख्यातृणाम‚भिधेयादिप्र‚काश‚ने क‚दाचित् क्रीडाद्य‚र्थ‚म‚पि प्र‚वृत्ति‚{\tiny $_{lb}$}‚र्भ‚व‚तीति--\cite{dp-msD-n}}} पूर्व स‚म्ब‚न्धादीनि युज्य‚न्ते व‚क्तुम् । ‚{\tiny $_{lb}$}‚ 
	  
	\edtext{\textsuperscript{*}}{\lemma{*}\Bfootnote{आख्यातृणां हि \cite{dp-edE} \cite{dp-edN} आख्यातृणां टीकाकाराणां हि \cite{dp-msB}}}व्याख्यातृणां हि व‚च‚नं\edtext{}{\lemma{नं}\Bfootnote{क्रीडार्थं \cite{dp-msA} \cite{dp-edP} \cite{dp-edE}}} क्रीडाद्य‚र्थ‚म‚न्य‚थापि स‚म्भाव्य‚ते । शास्त्र‚कृतां तु प्र‚क‚र‚ण‚प्रार‚म्भे‚{\tiny $_{lb}$}‚ न विप‚रीताभिधेयाद्य‚भिधाने प्र‚योज‚न‚मुत्प‚श्यामो नापि प्र‚वृत्तिम् । अत‚स्तेषु संश‚यो युक्तः ।‚{\tiny $_{lb}$}‚ अनुक्तेषु तु प्र‚तिप‚त्तृभिर्निष्प्र‚योज‚न‚म‚भिधेयं संभाव्येताऽस्य प्र‚क‚र‚ण‚स्य काक‚द‚न्त‚प‚रीक्षाया इव\edtext{}{\lemma{इव}\Bfootnote{इव १--इत्यादिरूपेण अन‚र्थ‚स‚म्भाव‚नाया अन‚न्त‚रं संख्याङ्का निर्दिष्टाः \cite{dp-msB} प्र‚तौ‚{\tiny $_{lb}$}‚ \cite{dp-msD} प्र‚तौ च--सं०}},‚{\tiny $_{lb}$}‚ अश‚क्यानुष्ठानं वा ज्व‚र‚ह‚र‚त‚क्ष‚क‚चूडार‚त्नाल‚ङ्कारोप‚देश‚व‚द्, अन‚भिम‚तं वा प्र‚योज‚नं‚{\tiny $_{lb}$}‚ मातृविवाह‚क्र‚मोप‚देश‚व‚द्, अतो वा प्र‚क‚र‚णात् ल‚घुत‚र उपायः प्र‚योज‚न‚स्य, अनुपाय एव वा‚{\tiny $_{lb}$}‚ प्र‚क‚र‚णं स‚म्भाव्येत ।‚{\tiny $_{lb}$}‚ वाक्येनेत्याह--\textbf{अन‚र्थे}त्यादि । न केव‚ल‚म‚न‚र्थ‚निश्च‚योऽपीत्य‚पि श‚ब्दः ।
	\pend% ending standard par
      ‚{\tiny $_{lb}$}‚

	  \pstart \leavevmode% starting standard par
	य‚द्य‚व‚श्यं व‚क्त‚व्यान्य‚भिघेयादीनि त‚र्हि व्याख्यातार एव तानि व‚क्ष्य‚न्ति । त‚त्किं‚{\tiny $_{lb}$}‚ शास्त्र‚कृतो व्याहारेणेत्याह--\textbf{शास्त्र‚कारेणैव} न व्याख्यातृभिः ।
	\pend% ending standard par
      ‚{\tiny $_{lb}$}‚

	  \pstart \leavevmode% starting standard par
	अथोच्य‚ते--प्र‚वृत्त्य‚ङ्ग‚त्वाच्च व्याच‚क्ष‚ते । अभिधेयाद्य‚न‚भिधाने तु प्र‚वृत्तेरेवास‚म्भ‚वात्क‚थं‚{\tiny $_{lb}$}‚ स‚म्भ‚विनो व्याख्यातारो येनाश‚ङ्क्य तेषां व्याहार‚म‚यं निर‚स्य‚तीति । अत्रोच्य‚ते । न स‚र्व‚थाऽ‚{\tiny $_{lb}$}‚भिधेयादिप्र‚काश‚नं केन‚चिन्न क‚र्त्त‚व्य‚मेवेति चोद्यं प्र‚वृत्त‚म्, किन्त्वादिवाक्यं त‚द‚र्थं न‚{\tiny $_{lb}$}‚ क‚र्त्त‚व्य‚मित्य‚भिस‚न्धिना । त‚त्र ये ताव‚त् साक्षादाचार्योप‚स‚न्नार‚तेषां त‚द्व‚च‚नादेव त‚त्प्र‚तीतौ‚{\tiny $_{lb}$}‚ प्र‚वृत्तिज्ञान‚योः स‚म्भ‚वात् स‚म्भ‚वि व्याख्यातृत्व‚म् । तेभ्योऽपि त‚दुप‚स‚न्नानामिति किम‚व‚द्य‚म् ?
	\pend% ending standard par
      ‚{\tiny $_{lb}$}‚

	  \pstart \leavevmode% starting standard par
	न‚नु च न शास्त्र‚कृद्व‚च‚न‚म‚पि प्र‚स‚ह्य प्र‚व‚र्त्त‚य‚ति, किं त‚र्हि प्र‚योज‚नाद्य‚भिधानेन प्र‚वृत्ति‚{\tiny $_{lb}$}‚विष‚योप‚द‚र्श‚नात् । त‚च्च व्याख्यातृव‚च‚नेऽपि स‚म्भ‚व‚तीति किमुच्य‚ते शास्त्र‚कारेणैवेत्य‚त‚{\tiny $_{lb}$}‚ आह--\textbf{व्याख्यातॄणां हि} इति । हिर्य‚स्माद‚र्थे । शास्त्र‚कारेष्व‚पीदं स‚मान‚मित्याह--\textbf{शास्त्र‚{\tiny $_{lb}$}‚कृता}मिति । तुना व्याख्यातृभ्यः शास्त्र‚कृतां वैध‚र्म्य‚माह । \textbf{विप‚रीतं} च त‚द\textbf{भिधेयाद्य‚भिधानं}‚{\tiny $_{lb}$}‚ चेति विग्र‚हः । अथ‚वा \textbf{विप‚रीतं} च त‚द‚स‚त्य‚त्वाद‚भिधेय‚ञ्चाभिधेय‚त‚या प्र‚काश‚नात् । त‚स्या‚{\tiny $_{lb}$}‚\textbf{भिधान}मिति विग्र‚हीत‚व्य‚म् । \textbf{उत्प‚श्याम} उत्प्रेक्षाम‚हे । अनेनैत‚दाह--तेषाम‚पि त‚थाभिधाने‚{\tiny $_{lb}$}‚ किम‚स्माकं बाध‚कं प्र‚माण‚म् ? केव‚ल‚मेषां म‚हीय‚साऽऽश‚येन शास्त्रं प्र‚णेतुमिच्छ‚तां प‚रार्थ‚{\tiny $_{lb}$}‚प्र‚वृत्तानां नैत‚त्स‚म्भाव‚याम इति ।
	\pend% ending standard par
      ‚{\tiny $_{lb}$}‚

	  \pstart \leavevmode% starting standard par
	प्र‚क‚र‚ण‚स्य \textbf{प्रार‚म्भ} आदौ । एत‚च्च प्र‚योज‚नाद्य‚भिधान‚निय‚त‚स‚न्निधेर्य‚थाभूतार्थ‚चिन्ता‚{\tiny $_{lb}$}‚विधान‚विष‚य‚स्य काल‚स्य निर्देशो न तु प्र‚क‚र‚ण‚स्य \leavevmode\ledsidenote{\textenglish{7a/ms}} म‚ध्येऽव‚साने वा त‚त् स‚म्भ‚व‚ति‚{\tiny $_{lb}$}‚ स‚प्र‚योज‚नं चेति ।
	\pend% ending standard par
      ‚{\tiny $_{lb}$}‚\footnote{उपायोऽस्ति प्र‚यो० \cite{dp-msC}}‚{\tiny $_{lb}$}‚\textsuperscript{\textenglish{15/dm}}‚{\tiny $_{lb}$}‚
	  \bigskip
	  \begingroup
	

	  \pstart \leavevmode% starting standard par
	एतासु\edtext{}{\lemma{एतासु}\Bfootnote{भ‚व‚तु नामैताव‚द्द्व‚ष‚ण‚संभाव‚ना को दोष इति चेदाह--\cite{dp-msD-n}}} चान‚र्थ‚स‚म्भाव‚नास्वेक‚स्याम‚प्य‚न‚र्थ‚स‚म्भाव‚नायां न प्रेक्षाव‚न्तः प्र‚व‚र्त‚न्ते ।
	\pend% ending standard par
      
	  \endgroup
	‚{\tiny $_{lb}$}‚

	  \pstart \leavevmode% starting standard par
	अथ‚य‚द्येवंविधा प्र‚वृत्तिः शास्त्र‚कृतां दृश्य‚ते त्व‚याऽप्य‚स्य किं न स‚म्भाव्य‚त इत्याह—‚{\tiny $_{lb}$}‚\textbf{नापि} इति । \textbf{अपि}र‚प्र‚योज‚नापेक्ष‚या स‚मं प्र‚वृत्तिद‚र्श‚नं स‚मुच्चिनोति । त‚देवोत्\textbf{प‚श्याम} इति‚{\tiny $_{lb}$}‚ स‚म्ब‚ध्य‚मान‚मिह प‚श्याम इत्यास्यार्थे स‚म्ब‚द्ध\edtext{}{\lemma{द्ध}\Bfootnote{न्ध}}व्य‚म् । प‚श्याम इति वाऽध्याहार्य‚म् ।‚{\tiny $_{lb}$}‚ अनेनैत‚दाह--त‚थापि स‚म्भाव‚यामो य‚दि तेषामीदृशी प‚श्यामो न त्वेव‚मिति । \textbf{अतो} हेतोस्तेष्व‚भि‚{\tiny $_{lb}$}‚धेयादिषु य‚द्वा \textbf{तेषु} शास्त्र‚कारेषु व‚क्तृषु स‚त्सु त‚दुक्तेष्व‚भिधेयादिष्विति प्र‚क‚र‚णात्संश‚यो‚{\tiny $_{lb}$}‚ऽर्थोन्मुख इति द्र‚ष्ट‚व्य‚म् ।
	\pend% ending standard par
      ‚{\tiny $_{lb}$}‚

	  \pstart \leavevmode% starting standard par
	न‚नु स‚त्य‚प्य‚र्थ‚संश‚येऽर्थित्वाभावे प्र‚वृत्तिर्नास्ति । त‚न्निमित्ता तु सा संश‚य‚मात्राद‚पि‚{\tiny $_{lb}$}‚ भ‚व‚ति । त‚न्मात्र‚ञ्च साध‚क‚बाध‚क‚विर‚हाद‚प्य‚र्थे सिद्ध‚म् । त‚त् किं त‚दुत्पादार्थेन वाक्येने‚{\tiny $_{lb}$}‚त्याह--\textbf{अनुक्तेषु}--इति । तुरुक्ताव‚स्थाया अनुक्ताव‚स्थां भेद‚व‚तीं द‚र्श‚य‚ति ।
	\pend% ending standard par
      ‚{\tiny $_{lb}$}‚

	  \pstart \leavevmode% starting standard par
	निर्ग‚तं \textbf{प्र‚योज‚नं} य‚स्मान्निष्कान्तं वा प्र‚योज‚नात् । काक‚द‚न्ता \textbf{ह्य‚भिधेयाः} । ते च न‚{\tiny $_{lb}$}‚ क्व‚चित् पुरुषार्थ उप‚युज्य‚न्ते । स‚प्र‚योज‚नं वाभिधेय‚म‚श‚क्यानुष्ठानं स‚म्भाव्येत । किंव‚दित्याह—‚{\tiny $_{lb}$}‚\textbf{ज्व}रेत्यादि । ज्व‚रं ह‚र‚तीति \textbf{ज्व‚र‚ह‚रः । त‚क्ष}काख्य‚स्य नाग‚राज‚स्य \textbf{चूडा} शिखा । त‚स्या‚{\tiny $_{lb}$}‚ र‚त्नं र‚तिं त‚नोतीति विशिष्टं व‚स्तु । तेनाल‚ङ्क‚र‚ण‚म् \textbf{अल‚ङ्कारः} । ज्व‚र‚ह‚र‚श्चासावेवंभूत‚श्चेति‚{\tiny $_{lb}$}‚ विग्र‚हः । स उप‚दिश्य‚ते येन ग्र‚न्थेन त‚द्व‚त् ।
	\pend% ending standard par
      ‚{\tiny $_{lb}$}‚

	  \pstart \leavevmode% starting standard par
	स‚तोर्वा स‚प्र‚योज‚न‚श‚क्यानुष्ठान‚त्व‚योर‚स्य प्र‚योज‚न\textbf{म‚न‚भिम‚त‚मे}वास्तिकानां \textbf{स‚म्भाव्येत ।‚{\tiny $_{lb}$}‚ मातुर्विवा}ह‚स्य \textbf{क्र‚मः} प‚रिपाटिरुप‚दिश्य‚ते येन \textbf{पार‚सीक‚शास्त्रेण त‚द्व‚त्} । पार‚सीक‚शास्त्रेण हि‚{\tiny $_{lb}$}‚ मृते पित‚रि माता प्र‚थ‚म‚म‚ग्र‚जेन पुत्रेण प‚रिणेत‚व्या, त‚द‚नु त‚द‚नुजेनेत्युप‚दिश्य‚ते ।
	\pend% ending standard par
      ‚{\tiny $_{lb}$}‚

	  \pstart \leavevmode% starting standard par
	\textbf{अतो वा प्र‚क‚र‚णाल्ल‚घुर}ल्प‚ग्र‚न्थः \textbf{प्र‚योज‚न‚स्य} स‚म्य‚ग्ज्ञान‚व्युत्प‚त्तिल‚क्ष‚ण‚स्य । \textbf{अनुपाय‚{\tiny $_{lb}$}‚ एव} अनिमित्त‚मेव ।
	\pend% ending standard par
      ‚{\tiny $_{lb}$}‚

	  \pstart \leavevmode% starting standard par
	च‚त्वारोऽपि \textbf{वा}श‚ब्दाः पूर्व‚पूर्वापेक्ष‚या प‚क्षान्त‚र‚म‚व‚द्योत‚य‚न्ति । स‚म्भाव‚ना च स‚र्वा‚{\tiny $_{lb}$}‚ व‚र्णिताऽन‚र्थोन्मुखा प्र‚तिप‚त्तिः युक्त्या द्वितीयाकारानुप्र‚वेशाच्च संश‚य‚प‚दाभिलाप्या प्र‚त्येत‚व्या,‚{\tiny $_{lb}$}‚ स‚र्व‚त्रेवाकारान्त‚र‚स्य प्र‚तियोगिनः स‚मुदाचार‚तोऽद‚र्श‚नात् । त‚थाविधाऽन‚र्थ‚स‚म्भाव‚नायाश्च‚{\tiny $_{lb}$}‚ बीज‚मिद‚म‚स्त्यादिवाक्ये-य‚द्य‚स्य प्र‚योज‚नादिकं भ‚वेद् ब‚हुभिर‚न्यैरिवानेनाप्यादावुक्तं भ‚वेत् । न‚{\tiny $_{lb}$}‚ चानेन किञ्चिद‚वादीति म‚तिः ।
	\pend% ending standard par
      ‚{\tiny $_{lb}$}‚

	  \pstart \leavevmode% starting standard par
	काम‚म‚मूर‚न‚र्थ‚स‚म्भाव‚ना भ‚व‚न्तु का नो हानिरित्याह--\textbf{एतास्विति} । न केव‚लं स‚र्वा‚{\tiny $_{lb}$}‚ इत्य‚पिश‚ब्देनाह । प्र‚योज‚नाद्य‚न‚भिधान इव य‚द्य‚भिधानेऽपि भ‚व‚न्ति तास्त‚दा किम‚भिधेयादिभि‚{\tiny $_{lb}$}‚र‚भिहितैर‚पीत्याह--\textbf{अभिधे}यादिषु इति । \textbf{तुरुक्तेषु} इत्य‚स्यान‚न्त‚रं द्र‚ष्ट‚व्यः, अनुक्त‚प‚क्षाच्च‚{\tiny $_{lb}$}‚ विशेष‚स्य द्योत‚कः ।
	\pend% ending standard par
      ‚{\tiny $_{lb}$}‚

	  \pstart \leavevmode% starting standard par
	अथ क‚थ‚म‚न‚योर‚र्थान‚र्थ‚स‚म्भाव‚न‚योर्विरोधो य‚तः स‚म्भाव‚ना संश‚य उच्य‚ते । स चोभ‚यां‚{\tiny $_{lb}$}‚ \leavevmode\ledsidenote{\textenglish{16/dm}}‚{\tiny $_{lb}$}‚ 
	  
	अभिधेयादिषु\edtext{}{\lemma{अभिधेयादिषु}\Bfootnote{उक्तेषु त्वाभिधेयादिष्व‚र्थ‚सं०--\cite{dp-edE} अभिधेयादिष्व‚र्थ‚सं० \cite{dp-edH} \cite{dp-edP} अभिधेयादिषु‚{\tiny $_{lb}$}‚ त्त[[तू]]क्तेष्व‚र्थ‚सं० \cite{dp-msB}}} तूक्तेष्व‚र्थ‚संभाव‚ना\edtext{}{\lemma{ना}\Bfootnote{न‚न्व‚भिधेयाभिधानेऽप्य‚न‚र्थ‚स‚म्भाव‚न‚या न प्र‚वृत्तिर्भ‚विष्य‚तीत्याह--\cite{dp-msD-n}}} अन‚र्थ‚स‚म्भाव‚नाविरुद्धा उत्प‚द्य‚ते । त‚या\edtext{}{\lemma{या}\Bfootnote{त‚या तु प्रेक्षा०--\cite{dp-msA} \cite{dp-edP} \cite{dp-edH} \cite{dp-edN}}} प्रेक्षाव‚न्तः‚{\tiny $_{lb}$}‚ प्र‚व‚र्त‚न्ते इति\edtext{}{\lemma{इति}\Bfootnote{इति त‚स्माद‚र्थे--\cite{dp-msD-n}}} प्रेक्षाव‚तां प्र‚वृत्त‚न्य‚ङ्ग‚म‚र्थ‚स‚म्भाव‚नां क‚र्तुं स‚म्ब‚न्धादीन्य‚भिधीय‚न्त इति स्थित‚म् ।‚{\tiny $_{lb}$}‚ शाव‚ल‚म्ब‚न इत्य‚न्योन्य‚स्याम‚न्याकारोऽस्तीति कुतः स‚हान‚व‚स्थान‚म् । उच्य‚ते । एक‚स्या‚{\tiny $_{lb}$}‚म‚न‚र्थांकार उद्रिक्तोऽनुभूय‚ते, युक्तेस्तु द्वितीयाकारानुप्र‚वेशः । इत‚स्यां नू \edtext{}{\lemma{नू}\Bfootnote{तू}} द्रिक्तोऽ\leavevmode\ledsidenote{\textenglish{7b/ms}}‚{\tiny $_{lb}$}‚र्भाकारोऽनुभूय‚ते युक्तेस्त्व‚न‚र्थाकारानुप्र‚वेश इत्युभ‚योर‚प्य‚न्योन्य‚वैप‚रीत्येनानुभ‚वाद् क‚थ‚{\tiny $_{lb}$}‚म‚विरोधः ।
	\pend% ending standard par
      ‚{\tiny $_{lb}$}‚

	  \pstart \leavevmode% starting standard par
	भ‚व‚त्वेवं त‚थापि क‚थ‚म‚स्याभिधानेऽन‚र्थाश‚ङ्का निर‚स्तेति चेत् । उच्य‚ते । पुरुषार्थ‚{\tiny $_{lb}$}‚सिद्धिरूपाभिधेय‚प्र‚योज‚नाभिधानेनाऽऽदिमाऽन‚र्थ‚स‚म्भाव‚ना निर‚स्ता । अभिधेय‚त‚त्प्र‚योज‚न‚यो‚{\tiny $_{lb}$}‚ रूप‚निरूप‚णेन च द्वितीय‚तृतीये निर‚स्ते । अन्तिमाऽपि साम‚र्थ्य‚ग‚तिस‚म्ब‚न्ध‚प्र‚तिपाद‚ना‚{\tiny $_{lb}$}‚न्निर‚स्ता । तुरीयां तु प‚र‚म‚विवेक‚शालिन आचार्य‚स्य प्र‚क‚र‚णार‚म्भ‚साम‚र्थ्यान्निर‚स्य‚ते । त‚थाहि‚{\tiny $_{lb}$}‚ न स‚मानेऽप्युपायान्त‚रे स‚म्भ‚व‚ति म‚ह‚तो म‚हान‚यं प्रेक्षापूर्व‚कारी प्र‚क‚र‚ण‚मीदृश‚मार‚भ‚ते, किम‚ङ्ग‚{\tiny $_{lb}$}‚ पुन‚र्ल‚घूनीति ।
	\pend% ending standard par
      ‚{\tiny $_{lb}$}‚

	  \pstart \leavevmode% starting standard par
	अथ स्यात्--स‚म‚स‚म‚य‚स‚म्भ‚वी देशान्त‚र‚व‚र्ती च क‚श्चिदेत‚द‚ग्र‚तो ल‚घूपायान्त‚र‚काकः \edtext{}{\lemma{काकः}\Bfootnote{न्त‚र‚{\tiny $_{lb}$}‚ग्र‚न्थः ?}} स्यात् । त‚त्क‚थं त‚त्प्र‚णीत‚ल‚घूपायान्त‚र‚निरासः ? अत्रापि त‚त्प्रेक्षाव‚त्तैव म‚ह‚ती निब‚न्ध‚न‚म् ।‚{\tiny $_{lb}$}‚ प्रेक्षापूर्व‚कारित्वादेव त‚थाभूतात्तेनास्मिन्नार्याव‚र्तेऽन्य‚त्र वा लोके नेदानीन्त‚नेनैत‚द‚र्थं प्र‚क‚र‚णं‚{\tiny $_{lb}$}‚ प्र‚णीत‚मिति ब‚हुधाऽनुस‚र्त्त‚व्य‚म् । अनुसृत्य दृष्ट्वा त‚ल्ल‚घुरुपाय इति च निश्चित्येदं प्र‚णेतुमुचितं‚{\tiny $_{lb}$}‚ नान्य‚थेति । स‚र्व‚त्रैव प्र‚क‚र‚णे स‚त्यादिवाक्ये ल‚घूपायान्त‚र‚निरासे ग‚तिरिय‚मेव । न‚हि अन्य‚त्रा‚{\tiny $_{lb}$}‚प्य‚भिधेय‚त‚त्प्र‚योज‚नादिप्र‚काश‚न‚म‚न्त‚रेण ल‚घूपायान्त‚र‚निरासाभिधान‚म‚स्तीति किं नानुम‚न्य‚ते ?
	\pend% ending standard par
      ‚{\tiny $_{lb}$}‚

	  \pstart \leavevmode% starting standard par
	य‚द्येव‚मित‚रासाम‚प्य‚न‚र्थ‚स‚म्भाव‚नानामेव‚मेवास्तु निरास‚स्त‚त्किमादिवाक्येन ? नैत‚त् ।‚{\tiny $_{lb}$}‚ न‚हि सूचीप्र‚वेशैत्येव मूष\edtext{}{\lemma{मूष}\Bfootnote{मूस}} ल‚प्र‚वेशः । त‚थाहि--अस‚त्यादिवाक्ये प्रेक्षापूर्व‚कारिप्र‚युक्त‚त्व‚मेव‚{\tiny $_{lb}$}‚ प्र‚क‚र‚ण‚स्य न श‚क्य‚ते क‚ल्प‚यितुम, प्र‚त्युताप्रेक्षापूर्व‚कारिप्र‚युक्त‚त्व‚मेव श‚क्य‚क‚ल्प‚न‚म् । दृश्य‚न्ते‚{\tiny $_{lb}$}‚ हि प्रेक्षापूर्व‚कारिणः प्र‚क‚र‚णादौ स‚र्व‚त्र प्र‚योज‚नाद्य‚भिधाय‚क‚मादिवाक्यं प्र‚ण‚य‚न्ते\edtext{}{\lemma{न्ते}\Bfootnote{न्तः}} । न चानेना‚{\tiny $_{lb}$}‚दिवाक्यं त‚द‚र्थ‚म‚कारि । त‚स्मान्नायं प्र‚क‚र‚ण‚कारः प्रेक्षापूर्व‚कारीतिस‚ङ्क‚ल्पादु\edtext{}{\lemma{ल्पादु}\Bfootnote{न्नो}}पाद‚दीति \edtext{}{\lemma{दीति}\Bfootnote{त}} ।
	\pend% ending standard par
      ‚{\tiny $_{lb}$}‚

	  \pstart \leavevmode% starting standard par
	न‚नु च प्र‚वृत्त्य‚र्थोऽयं प्र‚यासः । सा चेन्नास्ति किं त‚योत्प‚न्न‚यापीत्याह \textbf{त‚या}--इति ।‚{\tiny $_{lb}$}‚ य‚द्य‚र्थ‚स‚म्भाव‚ना प्र‚क‚र‚णे पुरुष‚स्य प्र‚व‚र्त्त‚यित्री त‚र्हि किम‚भिधेयाद्य‚भिधीय‚त इत्याह--\textbf{इति}—‚{\tiny $_{lb}$}‚इति । य‚स्माद‚र्थ‚स‚म्भाव‚न‚या प्र‚व‚र्त्त‚न्ते \textbf{इति}स्त‚स्माद‚र्थ‚स‚म्भाव‚नां क‚र्त्तुम् । किम्भूतां ?‚{\tiny $_{lb}$}‚ \textbf{प्र‚वृतेर‚ङ्गं} निमित्त‚म् । अङ्गादिश‚ब्दानाम‚स‚ति ब‚हुव्रीहौ प‚र‚लिङ्गाग्र‚ह‚णात्स्व‚लिङ्गेन निर्देशः ।‚{\tiny $_{lb}$}‚ \textbf{इति}रेव‚म‚र्थे \textbf{स्थितं} निश्चित‚म् ।
	\pend% ending standard par
      ‚{\tiny $_{lb}$}‚‚{\tiny $_{lb}$}‚\textsuperscript{\textenglish{17/dm}}‚{\tiny $_{lb}$}‚
	  \bigskip
	  \begingroup
	

	  \pstart \leavevmode% starting standard par
	अविसंवाद‚कं ज्ञानं स‚म्य‚ग्ज्ञान‚म् ।
	\pend% ending standard par
       ‚{\tiny $_{lb}$}‚ 

	  \pstart \leavevmode% starting standard par
	\edtext{\textsuperscript{*}}{\lemma{*}\Bfootnote{अथाविसंवाद‚क‚मिति कः श‚ब्दार्थ इत्याह--\cite{dp-msD-n}}}लोके च पूर्व‚मुप‚द‚र्शित‚म‚र्थं प्राप‚य‚न् संवाद‚क उच्य‚ते । त‚द्व‚ज्ज्ञान‚म‚पि स्व‚यं \edtext{}{\lemma{यं}\Bfootnote{०म‚पि प्र‚द० \cite{dp-msA} \cite{dp-msC} \cite{dp-msD} \cite{dp-edP} \cite{dp-edE}}}प्र‚द‚र्शित‚म‚र्थं‚{\tiny $_{lb}$}‚ प्राप‚य‚त् संवाद‚क‚मुच्य‚ते । \edtext{\textsuperscript{*}}{\lemma{*}\Bfootnote{न‚नु प्र‚द‚र्श‚क-प्र‚व‚र्त्त‚क-प्राप‚काणि विभिन्नान्येव प्र‚माणानि । त‚त् क‚थं स्व‚यं प्र‚द‚र्शित‚{\tiny $_{lb}$}‚म‚र्थं ज्ञानं प्राप‚य‚दिति सामानाधिक‚र‚ण्य‚मित्याश‚ङ्क्याह ॥ य‚द्वा प्र‚द‚र्श‚क-प्र‚व‚र्त‚क-प्राप‚काणि‚{\tiny $_{lb}$}‚ विभिन्नान्येव प्र‚माणानि अभ्युप‚ग‚म्य‚न्ते कैश्चिदिति त‚न्निराक‚र्तुं प्र‚व‚र्त्त‚क-प्राप‚क‚योस्ताव‚दैव‚यं‚{\tiny $_{lb}$}‚ प्र‚द‚र्श‚य‚न्नाह--\cite{dp-msD-n}}}प्र‚द‚र्शिते चार्थे प्र‚व‚र्त्त‚क‚त्व‚मेव प्राप‚क‚त्व‚म्, नान्य‚त् । त‚था हि--न
	\pend% ending standard par
      
	  \endgroup
	‚{\tiny $_{lb}$}‚

	  \pstart \leavevmode% starting standard par
	एव‚म‚नेन प्र‚व‚न्धेन \textbf{स‚म्य‚ग्ज्ञाने}त्यादिवाक्य‚स्य स‚मुदायार्थं व्याख्यायाव‚य‚वार्थ‚मिदानीम्‚{\tiny $_{lb}$}‚ \textbf{अविसंवाद‚क‚म्} इत्यादिना व्याच‚ष्टे ।
	\pend% ending standard par
      ‚{\tiny $_{lb}$}‚

	  \pstart \leavevmode% starting standard par
	अत्रायं पूर्व‚प‚क्षः । किमिदं स‚म्य‚क्त्वं ज्ञान‚स्याभिप्रेत‚म् ? य‚द्योगात्स‚म्य‚ग्ज्ञान‚मुच्य‚ते ।‚{\tiny $_{lb}$}‚ य‚द्येवं व‚स्तुत‚त्त्व‚ग्र‚ह‚णं स‚म्य‚क्त्व‚म्, अथापि गृहीत‚व‚स्तुप्राप‚ण‚म् ? उभ‚य‚थाऽपि अनुमान‚म‚व‚स्तुग्र‚ह‚णा‚{\tiny $_{lb}$}‚द‚स‚म्य‚ग्ज्ञान‚म् । अगृहीत‚प्राप‚णाद् वा स‚म्य‚ग्ज्ञान‚त्वे ज‚ल‚ज्ञान‚म‚प्युप‚द‚र्शित‚म‚रीचिकाः प्राप‚य‚तीति‚{\tiny $_{lb}$}‚ न किञ्चित्स‚म्य‚ग्ज्ञानं न स्यादिति ।
	\pend% ending standard par
      ‚{\tiny $_{lb}$}‚

	  \pstart \leavevmode% starting standard par
	सिद्धान्त‚वाद्य‚प्य‚मीषां प‚क्षाणाम‚न‚भ्युप‚ग‚मेन निरासं म‚न्य‚मानः--अविसंवाद‚क‚त्वं स‚म्य‚क्त्वं‚{\tiny $_{lb}$}‚ विव‚क्षित‚मिति \leavevmode\ledsidenote{\textenglish{8a/ms}} द‚र्श‚य‚ति । \textbf{अविसंवा}द‚कं संवाद‚क‚मुच्य‚ते । विश‚ब्दो हि संवाद‚क‚प्र‚तिषेधे‚{\tiny $_{lb}$}‚ व‚र्त्त‚ते । त‚त्प्र‚तिषेध‚स्य विधिरूप‚त्वात् संवाद‚क एवाव‚तिष्ठ‚त इति संवाद‚कार्थोऽविसंवाद‚क‚श‚ब्दः ।‚{\tiny $_{lb}$}‚ त‚द‚य‚म‚र्थः--अविसंवाद‚कं प्र‚वृत्तिविष‚य‚व‚स्तुप्राप‚कं स‚म्य‚ग्ज्ञान‚मिति ।
	\pend% ending standard par
      ‚{\tiny $_{lb}$}‚

	  \pstart \leavevmode% starting standard par
	स्यादेत‚त्--वृद्ध‚व्य‚व‚हारो हि श‚ब्दार्थ‚निश्च‚य‚भूमिः । त‚त्र च संवाद‚क‚श‚ब्दो नोप‚द‚र्शितार्थ‚{\tiny $_{lb}$}‚प्राप‚के व‚र्त्त‚ते । किं त‚र्हि ? स‚त्य‚वादिनि । न च ज्ञान‚स्य ताद्रूप्य‚म‚स्ति । त‚त् कुत‚स्त‚त्र‚{\tiny $_{lb}$}‚ संवाद‚क‚श‚ब्द इत्याह--\textbf{लोके च}--इति । \textbf{चो} य‚स्माद‚र्थे अपिश‚ब्दार्थे वा । \textbf{लोके} व्य‚व‚ह‚र्त्त‚रि‚{\tiny $_{lb}$}‚ ज‚ने । अय‚माश‚यो य‚था लोके स‚त्य‚वादिश‚ब्द‚प्र‚वृत्तिनिमित्त‚स्योप‚द‚र्शितार्थ‚प्राप‚ण‚स्य पुरुषे‚{\tiny $_{lb}$}‚ स‚म्भ‚वात्संवाद‚क‚श‚ब्दः प्र‚व‚र्त्त‚ते, त‚था ज्ञानेऽपि त‚त्स‚म्भ‚वादिति ।
	\pend% ending standard par
      ‚{\tiny $_{lb}$}‚

	  \pstart \leavevmode% starting standard par
	अथोच्य‚ते नोप‚द‚र्शितार्थ‚प्राप‚ण‚निमित्त‚कः पुरुषे संवाद\add{क}श‚ब्दः किन्तु प्र‚तिज्ञातार्थ‚प्राप‚ण‚{\tiny $_{lb}$}‚निमित्त‚कः । त‚त् क‚थ‚मिह निमित्त‚स‚म्भ‚व इति ? त‚द‚व‚द्य‚म् । त‚त्रापि प्र‚तिज्ञ‚योप‚द‚र्श‚न‚स्योप‚{\tiny $_{lb}$}‚ ल‚क्ष‚णात् । त‚देव तूप‚द‚र्श‚नं क्व‚चिद् व‚च‚नेन, क्व‚चिद‚ध्य‚व‚सायेना\edtext{}{\lemma{सायेना}\Bfootnote{पाठोऽत्र प‚श्चाद् व‚र्धितो दृश्य‚ते किन्तु सूक्ष्म‚त्वात् न प‚ठ्य‚ते ।}}\add{... ...}पि क्व‚चिद् व‚स्तुप्र‚ति‚{\tiny $_{lb}$}‚भास‚पूर्व‚केण क्व‚चिद‚न्य‚था वृत्तेनेति विशेषः । संवाद\add{क}श‚ब्दःप्र\edtext{}{\lemma{ब्दःप्र}\Bfootnote{ब्द‚प्र}}वृत्तिनिमित्तं तु स‚र्व‚त्र‚{\tiny $_{lb}$}‚ स‚मान‚मिति ।
	\pend% ending standard par
      ‚{\tiny $_{lb}$}‚

	  \pstart \leavevmode% starting standard par
	\textbf{प्राप‚य‚न्} इति \edtext{}{\lemma{इति}\Bfootnote{पाणिनि ३. २. १२६ ।}}ल‚क्ष‚ण‚हेत्वोरिति हेतौ श‚तुर्विधानात् प्राप‚णादित्य‚र्थः । एव‚मुत्त‚र‚त्रा‚{\tiny $_{lb}$}‚प्य‚व‚सेय‚म् ।
	\pend% ending standard par
      ‚{\tiny $_{lb}$}‚‚{\tiny $_{lb}$}‚\textsuperscript{\textenglish{18/dm}}‚{\tiny $_{lb}$}‚
	  \bigskip
	  \begingroup
	

	  \pstart \leavevmode% starting standard par
	ज्ञानं ज‚न‚य‚द‚र्थं प्राप‚य‚ति, अपि त्व‚र्थे पुरुषं प्र‚व‚र्त्त‚य‚त् प्राप‚य‚त्य‚र्थ‚म् । प्र‚व‚र्त्त‚क‚त्व‚म‚पि प्र‚वृत्तिविष‚य‚{\tiny $_{lb}$}‚प्र‚द‚र्श‚क‚त्व‚मेव । न हि पुरुषं ह‚ठात् प्र‚व‚र्त्त‚यितु श‚क्नोति \edtext{}{\lemma{क्नोति}\Bfootnote{ज्ञान‚म्--\cite{dp-msB} \cite{dp-edN}}}विज्ञान‚म् ।
	\pend% ending standard par
      
	  \endgroup
	‚{\tiny $_{lb}$}‚

	  \pstart \leavevmode% starting standard par
	न‚नूप‚द‚र्शितार्थ‚प्राप‚कं संवाद‚क‚मिति ब्रुव‚ता ज्ञान‚स्यैक‚स्योप‚द‚र्श‚क‚त्व‚प्राप‚क‚त्वे प्र‚तिज्ञाते ।‚{\tiny $_{lb}$}‚ प्र‚वृत्तिम‚न्त‚रेण प्राप्तेर‚नुप‚प‚त्तेर‚र्थ‚तः प्र‚व‚र्त्त‚क‚त्व‚म‚पि । न चैक‚स्यैते व्यापाराः स‚म्भ‚व‚न्ति ।‚{\tiny $_{lb}$}‚ य‚तोऽन्य‚त् प्र‚द‚र्श‚ति येन जानीते, अन्य‚त्प्र‚व‚र्त्त‚य‚ति य‚द‚न‚न्त‚रं प्र‚वृत्तिमाच‚र‚ति, अन्य‚च्च‚{\tiny $_{lb}$}‚ प्राप‚य‚ति य‚तः प्राप्त्याभिस‚म्ब‚द्ध्य‚ते पुरुष इत्याश‚ङ्क्याह--\textbf{प्र‚द‚र्शिते च}--इति । \textbf{चो} य‚स्माद‚र्थे ।
	\pend% ending standard par
      ‚{\tiny $_{lb}$}‚

	  \pstart \leavevmode% starting standard par
	प्र‚व‚र्त्त‚क‚त्व‚मेव प्राप‚क‚त्वं ब्रुव‚तोऽय‚म‚भिप्रायः--य‚द्य‚प्य‚र्थं साक्षात्कृत्यानुरूपं निश्च‚यं‚{\tiny $_{lb}$}‚ ज‚न‚य‚त्प्र‚द‚र्श‚कं किञ्चित्, अप‚रं प्र‚द‚र्श‚य‚द् वा बाह्यायाः प्र‚वृत्तेः कार‚णं भ‚व‚त्प्र‚व‚र्त्त‚क‚म्, इत‚र‚त्प्र‚{\tiny $_{lb}$}‚व‚र्त्त‚न‚द्वारेण बाह्यायाः प्राप्तेर्निमित्तं भ‚व‚त् प्राप‚कं व्य‚प‚दिश्य‚ते, त‚थापि स एव बाह्य‚प्र‚वृत्ति‚{\tiny $_{lb}$}‚कार‚ण‚भावः प्र‚व‚र्त‚यितृत्वादिप्र‚योज‚क‚व्यापार‚रूपो ज्ञान‚स्य पुरुष‚प्रेर‚णेनार्थ‚ज‚न‚नेन च प्र‚व‚र्त्त‚न‚{\tiny $_{lb}$}‚प्राप‚ण‚योर‚स‚म्भ‚वेन प्र‚द‚र्श‚नाद‚न्यो नोप‚युज्य‚त इति व‚स्तुतः स‚र्व‚स्यैव ज्ञान‚स्य निश्च‚यानुग‚त‚स्याधि‚{\tiny $_{lb}$}‚ग‚मान्नाप‚रौ प्र‚व‚र्त्त‚न-प्राप‚ण‚व्यापारौ । अत एव त‚त्राद्य‚मेव ज्ञानं प्र‚माणं व्य‚व‚स्थाप्य‚ते । त‚तो व‚स्तुतः‚{\tiny $_{lb}$}‚ प्र‚द‚र्श‚क‚त्वादीनाम‚भेदः, व्यावृत्तिनिब‚न्ध‚न‚स्तु भेदोऽस्त्येव । अत एवैते प्र‚द‚र्श‚क-प्र‚व‚र्त्त‚क-प्राप‚क‚श‚ब्दाः‚{\tiny $_{lb}$}‚ कृत‚क‚त्वानित्य‚त्वादिव‚न्न प‚र्याया इति ।
	\pend% ending standard par
      ‚{\tiny $_{lb}$}‚

	  \pstart \leavevmode% starting standard par
	स्यान्म‚त‚म्--किं पुनः प्र‚योज‚नं येन प्र‚र्व‚त्त‚नात् नाऽप‚रः प्राप‚ण‚व्यापारो ज्ञान‚स्य, अधि‚{\tiny $_{lb}$}‚ग‚माच्च नान्य‚त्प्र‚व‚र्त्त‚न प्र‚य‚त्नेन साध्य‚ते ? य‚द्युत्त‚र‚ज्ञान‚स्य प्रामाण्य‚निषेधार्थ‚म्, त‚दा गृहीत‚{\tiny $_{lb}$}‚ग्राहितैव त‚न्निषेत्स्य\leavevmode\ledsidenote{\textenglish{8b/ms}}तीति किं त‚द‚र्थेन प्र‚यासेनेति ? अत्रोच्य‚ते । य‚दि प्र‚व‚र्त्त‚यितृत्वं प्राप‚यि‚{\tiny $_{lb}$}‚तृत्वं च प्र‚द‚र्श‚क‚त्वात्प‚र‚मार्थ‚तोऽन्य‚त् स्यात् त‚दा गृहीत‚ग्राहितैव न श‚क्य‚ते प्र‚तिपाद‚यितुमिति केन‚{\tiny $_{lb}$}‚ प्रामाण्यं निषेध्येत ? त‚था हि न त‚ज्ज्ञानं गृहीतं गृह्णाति, अपि तु गृहीते प्र‚व‚र्त्त‚य‚ति ।‚{\tiny $_{lb}$}‚ अप‚रं तु गृहीतं प्राप‚य‚ति । त‚थाकारिणोश्च भिन्नोप‚योग‚त्वात्प्रामाण्यं क‚थ‚म‚पाक्रिय‚ते ? य‚दा‚{\tiny $_{lb}$}‚ तु प्र‚व‚र्त्त‚नान्नाप‚रः प्राप‚ण‚व्यापारो ज्ञान‚स्य प्र‚द‚र्श‚नाच्च नान्य‚त्प्र‚व‚र्त्त‚न‚म्, प्र‚थ‚मेनैव च प्र‚त्य‚क्षा‚{\tiny $_{lb}$}‚नुमान‚क्ष‚णेनार्थ‚क्रियास‚म‚र्थो व‚स्तुस‚न्तानः प्र‚वृत्तिविष‚यीक‚र्त्तुं निश्च‚यात् श‚क्य‚ते, त‚दोत्त‚रेषां‚{\tiny $_{lb}$}‚ त‚त्स‚न्तान‚भाविनाम‚भिन्न‚योग‚क्षेम‚त‚या प्रामाण्य‚म‚पास्य‚त इति ।
	\pend% ending standard par
      ‚{\tiny $_{lb}$}‚

	  \pstart \leavevmode% starting standard par
	एतेन त‚द‚पि प्र‚त्युक्तं य‚त् केन‚चिद‚भ्य‚धायि \textbf{ध‚र्मोत्त‚रे}--य‚द्युप‚द‚र्श‚क‚त्व‚मेव प्राप‚क‚त्वं त‚र्हि‚{\tiny $_{lb}$}‚ य‚दुक्तं उप‚द‚र्शित‚म‚र्थं प्राप‚य‚त् संवाद‚क‚मिति त‚स्याय‚म‚र्थः स्यात्--उप‚द‚र्शित‚मुप‚द‚र्श‚य‚दिति । न‚{\tiny $_{lb}$}‚ चैत‚द् युक्त‚म‚र्थ‚भेदाभावात् । त‚था हि य‚दि प‚र‚मार्थ‚तोऽर्थाभेद उच्य‚ते त‚दा न किञ्चिद‚व‚द्य‚म् ।‚{\tiny $_{lb}$}‚ अथ त‚दापि नैवं व‚क्त‚व्यः । अत्य‚ल्प‚मिद‚मुच्य‚ते । कृत‚क‚त्व‚म‚नित्य‚त्वं प्र‚तिपाद‚य‚तीत्य‚पि न‚{\tiny $_{lb}$}‚ व‚क्त‚व्य‚म्, अर्थाभेदादित्य‚पि किं नोच्य‚ते ? अथ श‚ब्द‚प्र‚वृत्त्य‚पेक्ष‚याऽर्थाभेद उच्य‚ते त‚दाऽसाव‚सिद्धो‚{\tiny $_{lb}$}‚ व्यावृत्तिभेद‚स्य द‚र्शित‚त्वात् । तेनाय‚म‚र्थः--उप‚द‚र्शितं साक्षात्कृत्य ज‚नितानुरूप‚निश्च‚य‚म‚र्थं‚{\tiny $_{lb}$}‚ प्र‚व‚र्त्त‚न‚द्वारेण बाह्यायाः प्राप्तेः साक्षाद् योग्य‚त‚या वा निमित्त‚तां ग‚च्छ‚त् संवाद‚क‚मिति ।‚{\tiny $_{lb}$}‚ वास्त‚व‚स्तु प्र‚व‚र्त्त‚क‚प्राप‚क‚योः प्र‚मेयाधिग‚तिल‚क्ष‚णात्प्र‚द‚र्श‚न‚व्यापाराद‚न्यो व्यापारो नास्तीति‚{\tiny $_{lb}$}‚ न प्र‚व‚र्त्त‚यितृत्वं प्राप‚यितृत्वं च प्र‚योज‚क-व्यापारोऽन‚योर्भिन्नः इति ।
	\pend% ending standard par
      ‚{\tiny $_{lb}$}‚\textsuperscript{\textenglish{19/dm}}‚{\tiny $_{lb}$}‚
	  \bigskip
	  \begingroup
	

	  \pstart \leavevmode% starting standard par
	\edtext{\textsuperscript{*}}{\lemma{*}\Bfootnote{य‚त एव प्र‚व‚र्त्त‚क‚त्व‚मेव प्राप‚क‚त्व प्र‚व‚र्त‚क‚त्व‚म‚पि प्र‚वृत्तिविष‚य‚प्र‚द‚र्श‚क‚त्व‚म्--\cite{dp-msD-n}}}अत एव \edtext{}{\lemma{एव}\Bfootnote{अत एवार्था० \cite{dp-msB}}}चार्थाधिग‚तिरेव प्र‚माण‚फ‚ल‚म् । अधिग‚ते चार्थे प्र‚व‚र्तितः पुरुषः प्रापित‚{\tiny $_{lb}$}‚श्चार्थः । त‚था च स‚त्य‚र्थाधिग‚मात् स‚माप्तः प्र‚माण‚व्यापारः । \edtext{\textsuperscript{*}}{\lemma{*}\Bfootnote{य‚त एव स‚माप्ता प्र‚माण‚व्यापृतिः--\cite{dp-msD-n}}}अत एव \edtext{}{\lemma{एव}\Bfootnote{अत एवान‚धि \cite{dp-msA} \cite{dp-edP} \cite{dp-edH} \cite{dp-edE} \cite{dp-edN}}}चान‚धिग‚त‚विष‚यं‚{\tiny $_{lb}$}‚ प्र‚माण‚म् । येनैव हि ज्ञान‚न प्र‚थ‚म‚म‚धिग‚तोऽर्थः, तेनैव प्र‚व‚र्त्तितः पुरुषः, प्रापित‚श्चार्थः । त‚त्रैव‚{\tiny $_{lb}$}‚ चाथ\edtext{}{\lemma{चाथ}\Bfootnote{त‚त्रैवार्थे \cite{dp-msA} \cite{dp-edP} \cite{dp-edE} \cite{dp-edH} \cite{dp-edN}}} किम‚न्येन ज्ञानेनाधिकं कार्य‚म् ? \edtext{\textsuperscript{*}}{\lemma{*}\Bfootnote{त‚तो \cite{dp-msA} \cite{dp-msB} \cite{dp-msC} \cite{dp-msD} \cite{dp-edP} \cite{dp-edH} \cite{dp-edE} \cite{dp-edN}}}अतोऽधिग‚त‚विष‚य‚म‚प्र‚माण‚म् ।
	\pend% ending standard par
      
	  \endgroup
	‚{\tiny $_{lb}$}‚

	  \pstart \leavevmode% starting standard par
	स्यान्म‚त‚म्--य‚द्य‚धिग‚तिरेवोत्त‚रेषाम‚पि फ‚लं स्यात् त‚दोप‚योगान्त‚राभावाद् भ‚वेद‚प्रामाण्यं‚{\tiny $_{lb}$}‚ याव‚ता प्र‚व‚र्त्त‚क‚स्य प्र‚वृत्तिः फ‚ल‚म्, प्राप‚क‚स्य प्राप्तिरिति फ‚ल‚भेद‚निष्प‚त्तेर्भिन्नो व्यापार इत्याह—‚{\tiny $_{lb}$}‚\textbf{अत एव}--इति । य‚तो नार्थ‚ज‚न‚न‚द्वारेणार्थ‚प्राप‚णं ज्ञान‚स्य, य‚त‚श्च न प्र‚स‚ह्य प्रेर‚णेन प्र‚व‚र्त्त‚न‚म्,‚{\tiny $_{lb}$}‚ \textbf{अत एवा}स्मादेव कार‚णाद‚र्थ‚स्याधिग‚तिः प‚रिच्छित्तिः \textbf{फ‚ल‚म्,} न प्र‚वृत्त्यादि ।
	\pend% ending standard par
      ‚{\tiny $_{lb}$}‚

	  \pstart \leavevmode% starting standard par
	न‚नु त‚त् प्र‚माण‚स्य फ‚लं व्य‚व‚स्थाप्य‚ते, य‚स्मिन् स‚ति त‚द्व्यापारः प‚रिस‚माप्य‚ते । न‚{\tiny $_{lb}$}‚ चाधिग‚ताव‚पि प्र‚वृत्ति-प्राप्त्योर‚भावे स प‚रिस‚माप्य‚त इत्याह--\textbf{अधिग‚ते च}--इति । \textbf{चो}‚{\tiny $_{lb}$}‚ य‚स्माद‚र्थे । य‚स्माद् येनार्थः स‚म्य‚ग्ज्ञानेन द‚र्शित‚स्त‚त्र तेनाप्र‚व‚र्त्तितोऽपि पुरुषः प्र‚वृत्तियोग्योप‚{\tiny $_{lb}$}‚द‚र्श‚नात्, त‚द्ग‚त‚स्य च व्यापारान्त‚र‚स्याभावात्\textbf{प्र‚व‚र्त्तित} इत्युच्य‚ते । स‚त्य‚र्थित्वे प्र‚व‚र्त्त‚न‚मेव ।‚{\tiny $_{lb}$}‚ ज्ञानेन ताव‚त्प्र‚वृत्तियोग्यः कृत इति याव‚त् ।
	\pend% ending standard par
      ‚{\tiny $_{lb}$}‚

	  \pstart \leavevmode% starting standard par
	अस्तु पुरुष‚स्त‚था प्र‚व‚र्त्तितोऽर्थ‚स्तु न प्रापितः, त‚था च व्यापारान्त‚र‚म‚न्याधीन‚म‚स्तीत्याह‚{\tiny $_{lb}$}‚\textbf{प्रापित} इति । चः पूर्व‚व‚त् । अर्थोऽप्य‚साव‚प्राप्तोऽपि श‚क्य‚प्राप्तिको द‚र्शित इति \textbf{प्रापित}‚{\tiny $_{lb}$}‚ उच्य‚ते । अत एव प्राप‚ण‚श‚क्तिरेव ज्ञान‚स्य प्रामाण्य‚म् । सा च प्राप्याद‚र्थादात्म‚लाभ‚निमित्तेति,‚{\tiny $_{lb}$}‚ य‚तो \leavevmode\ledsidenote{\textenglish{9a/ms}} येन प्र‚व‚र्त्त‚ते त‚द‚पि प्राप‚ण‚योग्य‚मेव । श‚क्तिनिश्च‚य‚स्त्व‚र्थ‚क्रियानिर्भास‚स्य स‚र्व‚स्यानु‚{\tiny $_{lb}$}‚मान‚स्य च स्व‚त एव । प्र‚व‚र्त्त‚काध्य‚क्ष‚स्य च क‚स्य‚चित्स्व‚त एव य‚द‚भ्यासेन प‚रितो निर‚स्त‚विभ्र‚मा‚{\tiny $_{lb}$}‚श‚ङ्क‚म्, य‚न्निद्राद्य‚नुप‚प्लुतं स‚द् आस‚न्न‚देश‚म‚नाश‚ङ्क्य‚व्य‚ञ्ज‚काधीनाऽन्य‚थाभिव्य‚वित च व‚स्तु‚{\tiny $_{lb}$}‚ गृह्णाति । त‚द्रूप‚संवेद‚नादेव स‚त्यार्थं निश्चीय‚ते । क‚स्य‚चित्तु प‚र‚तोऽर्थ‚क्रियानिर्भासात्म‚कात्‚{\tiny $_{lb}$}‚ स्व‚तः प्र‚माणाद‚न्य‚तो वा य‚तः कुत‚श्चिन्नान्त‚रीय‚कार्थ‚द‚र्श‚नान्म‚ध्य‚काल‚व‚र्त्तिभ्रान्तिश‚ङ्काप‚नोदेन‚{\tiny $_{lb}$}‚ निश्चीय‚त इति । \textbf{प्रापित‚श्चार्थ‚मि}ति क्व‚चित्पाठः । स तु युक्त‚रूपः प्र‚वृत्तियोग्यीक‚र‚णात्प्र‚व‚र्त्तितः‚{\tiny $_{lb}$}‚ प्राप्यार्थोप‚द‚र्श‚नात्प्रापितोऽर्थ‚मित्येक‚वाक्य‚त‚योप‚द‚र्श‚नात् । एव‚मुत्त‚र‚त्राऽप्येष एव पाठोऽव‚दात‚{\tiny $_{lb}$}‚ इति । \textbf{त‚था च स‚ति} त‚स्मिंश्च प्र‚कारे स‚ति । \textbf{स‚माप्तः} प‚र्य‚व‚सानं ग‚तः । त‚स्माद‚धिग‚म‚स्य‚{\tiny $_{lb}$}‚ फ‚ल‚त्वं युक्त‚मिति भावः । य‚तोऽधिग‚माद‚न्य‚त्फ‚लं नोप‚प‚द्य‚ते, जाते च त‚स्मिन् स‚माप्य‚ते व्यापारः,‚{\tiny $_{lb}$}‚ \textbf{अतो}ऽस्मात्कार‚णाद् \textbf{अन‚धिग}तो ज्ञानान्त‚रेणाप‚रिच्छिन्नो \textbf{विष‚यो}ऽर्थो य‚स्य त‚त् प्र‚माणं भ‚व‚ति ।
	\pend% ending standard par
      ‚{\tiny $_{lb}$}‚

	  \pstart \leavevmode% starting standard par
	न‚नु अधिग‚त‚विष‚य‚मुदीचीनं ज्ञानं त‚त्रार्थे किञ्चिद‚धिक‚माद‚धानं प्र‚माणं भ‚विष्य‚ति । न हि‚{\tiny $_{lb}$}‚ विष‚य‚भेदादेव प्र‚माण‚भेदोऽपि तूप‚योग‚भेदाद‚पीत्याह--\textbf{येनैव}--इति । \textbf{हि}र्य‚र‚मात् । पूर्वाद् योग्य‚{\tiny $_{lb}$}‚त‚या प्र‚व‚र्त्तितः प्रापित इति चोक्तं त‚त्रैव पूर्व‚ज्ञानाद‚धिग‚ते \textbf{किम‚धिक‚म}तिरिक्तं \textbf{कार्यं} क‚र्त्त‚व्य‚म् ?‚{\tiny $_{lb}$}‚ ‚{\tiny $_{lb}$}‚ \leavevmode\ledsidenote{\textenglish{20/dm}}‚{\tiny $_{lb}$}‚ 
	  
	\edtext{\textsuperscript{*}}{\lemma{*}\Bfootnote{एवं सामान्येनाविसंवाद‚कं स‚म्य‚ग्ज्ञानं प्र‚तिपाद्य विशेषेण प्र‚त्य‚क्षानुमाने स्व‚व्यापारं‚{\tiny $_{lb}$}‚ कुर्व‚तां स‚म्य‚ग्ज्ञानं भ‚व‚त इति द‚र्श‚य‚न्नाह--\cite{dp-msD-n}}}त‚त्र यो\unclear{र्थो} दृष्ट‚त्वेन ज्ञातः स प्र‚त्य‚क्षेण प्र‚वृत्तिविष‚यीकृतः\edtext{}{\lemma{यीकृतः}\Bfootnote{विष‚यः कृतः \cite{dp-msB} \cite{dp-msC} \cite{dp-msD}}} । य‚स्माद् य‚स्मिन्न‚र्थे‚{\tiny $_{lb}$}‚ प्र‚त्य‚क्ष‚स्य साक्षात्कारित्व‚व्यापारो विक‚ल्पेनानुग‚म्य‚ते\edtext{}{\lemma{ते}\Bfootnote{अनुभूय‚ते--\cite{dp-msD-n}}} त‚स्य प्र‚द‚र्श‚कं प्र‚त्य‚क्ष‚म्; त‚स्माद् दृष्ट‚त‚या‚{\tiny $_{lb}$}‚ ज्ञातः\edtext{}{\lemma{ज्ञातः}\Bfootnote{निश्चितः--\cite{dp-msD-n}}} प्र‚त्य‚क्ष‚द‚र्शितः । अनुमानं तु \edtext{}{\lemma{तु}\Bfootnote{लिङ्ग‚द‚र्श‚नं लिङ्ग‚जात‚म् । त‚च्च व‚ह्न्य‚व्य‚भिचारि धूम‚निश्च‚यं ज[[न‚य‚ति]]सामान्येन‚{\tiny $_{lb}$}‚ साध्याविनाभावित्व‚स्म‚र‚ण‚जात‚म्--य‚था धूमं प्र‚त्य‚क्षेण गृहीत्वा स‚र्व‚त्रायं व‚ह्निज इति स्म‚र‚णं‚{\tiny $_{lb}$}‚ त‚स्मात्--\cite{dp-msD-n}}}लिग‚द‚र्श‚नान्निंश्चिन्व‚त्\edtext{}{\lemma{त्}\Bfootnote{विक‚ल्प‚य‚त्--\cite{dp-msD-n}}} प्र‚वृत्तिविष‚यं द‚र्श‚य‚ति ।‚{\tiny $_{lb}$}‚ न किञ्चित् । आद्येनैव क‚र्त्त‚व्य‚स्य कृत‚त्वादिति भावः । \textbf{अतो}ऽस्माद् हेतोरि\textbf{ध‚ग‚त‚षियं} त‚द‚धि‚{\tiny $_{lb}$}‚ग‚न्तृस‚जातीयं विजातीयं वा न प्र‚माण‚म् । तेन प्र‚माण‚स‚म्प्ल‚वो नाम नास्त्येवेति प्र‚काशित‚म् ।
	\pend% ending standard par
      ‚{\tiny $_{lb}$}‚

	  \pstart \leavevmode% starting standard par
	न‚नु च प्र‚ब‚न्धेनानेन प्र‚वृत्तिविष‚योप‚द‚र्श‚कं स‚म्थ‚ग्ज्ञान‚मिति द‚र्शित‚म् । व‚क्ष्य‚माण‚या नीत्या‚{\tiny $_{lb}$}‚ प्र‚त्य‚क्षानुमान‚नाम‚नी द्वे स‚म्य‚ग्ज्ञाने । त‚त्र य‚दि स‚मान‚म‚न‚योः प्र‚वृत्तिविष‚योप‚द‚र्श‚नं त‚दिदं‚{\tiny $_{lb}$}‚ प्र‚त्य‚क्षं स‚त्प्र‚माण‚म्, इद‚म‚नुमानं स‚दिति भेदो न स्यात् । अत‚स्त‚द‚न‚योर‚स‚मान‚विष‚य‚त्व‚मेव‚{\tiny $_{lb}$}‚ क‚थित‚व्य‚म् । एवं च क‚स्य क‚थं त‚दिति व‚क्त‚व्य‚मित्याह--\textbf{त‚त्र} इति । त‚योः प्र‚त्य‚क्षानुमान‚यो‚{\tiny $_{lb}$}‚र्म‚ध्ये \textbf{दृष्ट‚त्वेन ज्ञातो} निश्चितः । य‚दि ज्ञात इत्येव क्रिय‚ते त‚दाऽनुमेयोऽपि निश्चितः स‚न्‚{\tiny $_{lb}$}‚ प्र‚त्य‚क्षेण प्र‚वृत्तिविष‚यीकृतः प्र‚स‚ज्येतेति दृष्ट‚त्वेनेति कृत‚म् । अथ यो दृष्ट इति किं नोच्य‚ते ?‚{\tiny $_{lb}$}‚ नोच्य‚ते । क्ष‚णिक‚त्वादेर‚पि दृष्ट‚त्वेन प्र‚त्य‚क्ष‚विष‚य‚त्वाद‚नुमानाव‚ताराभाव‚प्र‚स‚ङ्गादिति ।
	\pend% ending standard par
      ‚{\tiny $_{lb}$}‚

	  \pstart \leavevmode% starting standard par
	न‚नु न प्र‚त्य‚क्ष‚स्य निश्च‚य‚नाद् ग्र‚ह‚ण‚म‚पि तु प्र‚तिभासात् । त‚त्किमुच्य‚ते ज्ञातो निश्चित‚{\tiny $_{lb}$}‚ इत्याश‚ङ्क्याह--\textbf{य‚स्माद्} इति । \textbf{विक‚ल्पेनानुग‚म्य‚ते}ऽनुस्त्रिय‚तेऽध्य‚व‚सीय‚ते प‚श्यामीत्याकारेण ।‚{\tiny $_{lb}$}‚ त‚स्माद् \textbf{दृष्ट‚त्वेन ज्ञात} इत्युच्य‚ते । \textbf{विक‚ल्पेनेति} त‚त्पृष्ठ‚भाविनाऽनुरूपेणेति द्र‚ष्ट‚व्य‚म् । अन‚नु‚{\tiny $_{lb}$}‚रूप‚विक‚ल्पानुग‚त‚व्यापार‚स्य त‚त्राप्रामाण्यात् क्ष‚णिक‚त्व इव । एवं ब्रुव‚त‚श्चाय‚म‚भिप्रायः—‚{\tiny $_{lb}$}‚सांव्य‚व‚हारिक‚स्य प्र‚माण‚स्येदं ल‚क्ष\leavevmode\ledsidenote{\textenglish{9b/ms}}ण‚मुच्य‚ते । त‚तो व‚स्तुवृत्त्या प्र‚काश‚मान‚म‚प्य‚नुरूप‚{\tiny $_{lb}$}‚ विक‚ल्पेनाविष‚यीकृतं स‚द‚प्र‚तिभास‚मानं नातिशेते, व्य‚व‚हारायोग्य‚त्वात् । एवं त‚द्ग्राह‚क‚म‚पि‚{\tiny $_{lb}$}‚ त‚थाविध‚विक‚ल्पेनान‚नुग‚म्य‚मान‚व्यापारं व्य‚व‚हार‚यितुम‚प‚र्याप्तं स‚त् तृण‚स्यापि कुब्जीक‚र‚णेऽस‚म‚र्थ‚{\tiny $_{lb}$}‚म‚ग्राह‚कं नातिव‚र्त्त‚ते । तेन य‚दुक्तं प्र‚द‚र्श‚क‚त्व‚मेव प्र‚व‚र्त्त‚क‚त्वादीति, य‚च्चोक्त‚म्--अधिग‚तिरेव‚{\tiny $_{lb}$}‚ फ‚ल‚मिति त‚द‚नुरूप‚निश्च‚यानुग‚त‚व्यापार‚म‚नुरूप‚निश्च‚यानुग‚ताविति द्र‚ष्ट‚व्य‚म् । एवं य‚त्र य‚त्रोच्य‚ते‚{\tiny $_{lb}$}‚ प्र‚त्य‚क्षं व‚स्तूप‚द‚र्श‚कं व‚स्तुग्राह‚क‚मित्यादिना श‚ब्देन त‚त्र स‚र्व‚त्रानुरूप‚निश्च‚यानुग‚त‚व्यापार‚मेव‚{\tiny $_{lb}$}‚ बोद्ध‚व्य‚म् ।
	\pend% ending standard par
      ‚{\tiny $_{lb}$}‚

	  \pstart \leavevmode% starting standard par
	अथैवं स‚ति विक‚ल्प‚स्यापि प्रामाण्यं प्र‚स‚ज्येतेति चेत् । एत‚त्स्व‚य‚मेव \textbf{ध‚र्मोत्त‚रेणा}श‚ङ्क्य‚{\tiny $_{lb}$}‚ निराक‚रिष्य‚त इति नेहोच्य‚ते । य‚दि प्र‚त्य‚क्ष‚मेवं प्र‚वृत्तिविष‚य‚मुप‚द‚र्श‚य‚ति अनुमान‚म‚प्येवं त‚दा‚{\tiny $_{lb}$}‚ ‚{\tiny $_{lb}$}‚ \edtext{\textsuperscript{*}}{\lemma{*}\Bfootnote{त‚त्पृष्ठ‚भाविना विक‚ल्पेनाव‚सीय‚ते । एत‚दुक्तं भ‚व‚ति--प्र‚तिभास‚मानार्थाध्य‚व‚सायं‚{\tiny $_{lb}$}‚ कुर्व‚त् प्र‚त्य‚क्ष‚प्र‚माणं संवाद‚क‚मित्य‚र्थः--\cite{dp-msD-n}}} ‚{\tiny $_{lb}$}‚ \leavevmode\ledsidenote{\textenglish{21/dm}}‚{\tiny $_{lb}$}‚ 
	  
	त‚था च प्र‚त्य‚क्षं प्र‚तिभास‚नं निय‚त‚म‚र्थं द‚र्श‚य‚ति । अनुमानं च लिङ्ग‚स‚म्ब‚द्धं निय‚त‚म‚र्थं‚{\tiny $_{lb}$}‚ द‚र्श‚य‚ति । अत एते \edtext{}{\lemma{एते}\Bfootnote{निय‚तार्थ‚स्य \cite{dp-msB}}}निय‚त‚स्यार्थ‚स्य प्र‚द‚र्श‚के । तेन ते प्र‚माणे । नान्य‚द्विज्ञान‚म् ।\edtext{\textsuperscript{*}}{\lemma{*}\Bfootnote{श‚ब्दोप‚मानादिक‚म्--\cite{dp-msD-n}}} ‚{\tiny $_{lb}$}‚ 
	  
	प्राप्तुं श‚क्य‚म‚र्थ‚माद‚र्श‚य‚त्\edtext{}{\lemma{त्}\Bfootnote{०र्थ‚मुप‚द‚र्श० \cite{dp-msC} \cite{dp-msD} \cite{dp-msB}}} प्राप‚क‚म् । प्राप‚क‚त्वाच्च प्र‚माण‚म् ।‚{\tiny $_{lb}$}‚ क‚थं भेद‚व्य‚व‚स्थेति ? आह--\textbf{अनुमानं तु}--इति । \textbf{तुः} प्र‚त्य‚क्षाद‚नुमान‚स्य वैध‚र्म्य‚माह । प्र‚त्य‚क्षं न‚{\tiny $_{lb}$}‚ स्व‚यं निश्चिन्व‚त् प्र‚वृत्तिविष‚यं द‚र्श‚य‚ति किन्तु निश्चाय‚य‚त् । अनुमानं तु स्व‚य‚मेव निश्चिन्व‚दिति ।
	\pend% ending standard par
      ‚{\tiny $_{lb}$}‚

	  \pstart \leavevmode% starting standard par
	क‚थ‚म‚प्र‚तिभास‚मानं निश्चेतुमीष्टे त‚दित्याह--\textbf{लिङ्ग‚द‚र्श‚नाद्} इति । लिङ्ग‚स्य साध्या‚{\tiny $_{lb}$}‚विनाभूत‚स्य धूमादेर्द‚र्श‚नात् । द‚र्श‚नं च स्व‚रूप‚ग्र‚ह‚ण‚पूर्व‚कं स‚र्व‚त्रेदं साध्याविनाभूत‚मिति ज्ञान‚म्,‚{\tiny $_{lb}$}‚ स‚र्व‚त्र साध्याविनाभूत‚मिति स्म‚र‚ण‚पुरःस‚रं वा क्व‚चित् स्व‚रूप‚ग्र‚ह‚ण‚मिह विव‚क्षित‚म् ।
	\pend% ending standard par
      ‚{\tiny $_{lb}$}‚

	  \pstart \leavevmode% starting standard par
	न‚नु च प्र‚त्य‚क्षानुमान‚ज्ञान‚व‚द‚न्य‚स्यापि ज्ञान‚स्य य‚था प्र‚वृत्तिविष‚य‚प्र‚द‚र्श‚नं त‚था किं न‚{\tiny $_{lb}$}‚ प्र‚द‚र्श्य‚ते ? अथाप्रामाण्यान्नोप‚द‚र्श्य‚ते । क‚थं पुन‚र‚प्रामाण्य‚म‚न्य‚स्य ? अनिय‚त‚प्र‚वृत्तिविष‚य‚प्र‚द‚र्श‚क‚{\tiny $_{lb}$}‚त्वादिति चेत् । त‚र्हि प्र‚त्य‚क्षानुमान‚योर‚पि त‚थात्वेन प्रामाण्यं न स्यादित्यागूर्य ज्ञानान्त‚राद‚{\tiny $_{lb}$}‚ भेद‚म‚न‚योर्द‚र्श‚य‚न्नाह--\textbf{त‚था} च--इति । त‚स्मिश्च प्र‚त्य‚क्ष‚स्य स्व‚व्यापारानुसारिविक‚ल्पोप‚ज‚न‚नेन‚{\tiny $_{lb}$}‚ निश्च‚य‚नात्, प्र‚वृत्तिविष‚य‚प्र‚द‚र्श‚न‚प्र‚कारे\edtext{}{\lemma{कारे}\Bfootnote{सूक्ष्म‚त्वात् न प‚ठ्य‚ते ।}}\add{... ... ...}म‚न्य‚दा लिङ्ग‚द‚र्श‚नात्स्व‚यं निश्च‚येन प्र‚वृत्ति—‚{\tiny $_{lb}$}‚विष‚य‚प्र‚द‚र्श‚न‚प्र‚कारे । \textbf{प्र‚त्य‚क्षं} ज्ञानं \textbf{प्र‚तिभास‚मानं} स्व‚रूपेण प्र‚काश‚मानं \textbf{निय}त\textbf{म}र्थ‚क्रियाक्ष‚मे भाव‚रूप‚{\tiny $_{lb}$}‚ एव व्य‚व‚स्थितं \textbf{द‚र्श‚य‚ति} । तेन नान‚र्थं नाप्य‚निय‚तं द‚र्श‚य‚तीत्याकूत‚म् । \textbf{अनुमानं च निय‚त‚म‚र्थं‚{\tiny $_{lb}$}‚ द‚र्श‚य‚ति । चः} प्र‚त्य‚क्षेण स‚म‚म‚नुमानं निय‚त‚प्र‚द‚र्श‚क‚त्वेन स‚मुच्चिनोति । \textbf{लिङ्ग‚स‚म्ब‚द्ध‚म्} इति‚{\tiny $_{lb}$}‚ हेतुभावेन विशेष‚ण‚म् । त‚द‚य‚म‚र्थः--य‚स्माल्लिङ्गं विजातीय‚व्यावृत्तेऽर्थेक्रियाकारिणि तादात्म्येन‚{\tiny $_{lb}$}‚ त‚दुत्प‚त्त्या वा \textbf{स‚म्ब‚द्ध}माय‚त्तं त‚स्मात् त‚त्प्र‚भ‚व‚म‚प्य‚नुमानं निय‚तं द‚र्श‚य‚ति अध्य‚व‚स्य‚तीति । तेनानु‚{\tiny $_{lb}$}‚मान‚म‚पि नान‚र्थं नाप्य‚निय‚तं भावात्म‚न्य‚भावात्म‚नि वा द‚र्श‚य‚तीति प्र‚काशित‚म् । इयांस्तु‚{\tiny $_{lb}$}‚ विशेषोऽनुमानं स‚म्ब‚न्ध‚ग्र‚ह‚ण‚काल‚दृष्ट‚साधार‚णं रूप‚माश्रित्योद‚य‚मानं स्व‚ल‚क्ष‚ण‚म‚ध्य‚व‚स्य‚द‚पि न‚{\tiny $_{lb}$}‚ स‚न्तानान्त‚रासाधार‚ण‚म‚ध्य‚व‚स्य‚तीति प्र‚वृत्तिविष‚यापेक्ष‚म‚पि सामान्य‚विष‚य‚मेव । प्र‚त्य‚क्षं तु‚{\tiny $_{lb}$}‚ प्र‚वृत्तिविष‚यापेक्ष‚म‚प्य‚साधार‚ण‚विष‚य‚मेव । स‚न्तानान्त‚रासाधा\leavevmode\ledsidenote{\textenglish{10a/ms}}र‚णेनैव रूपेण विष‚य‚स्य‚{\tiny $_{lb}$}‚ निश्चाय‚नादिति ।
	\pend% ending standard par
      ‚{\tiny $_{lb}$}‚

	  \pstart \leavevmode% starting standard par
	\hphantom{.}इदानीं य‚स्याय‚माश‚यः--अस्तु प्र‚त्य‚क्षं निय‚तार्थ‚द‚र्श‚क‚म्, अर्थ‚स्य साक्षात्क‚र‚णात्; अनुमानं‚{\tiny $_{lb}$}‚ तु प‚रोक्षार्थ‚स्यासाक्षात्क‚र‚णात्क‚थं निय‚तं द‚र्श‚य‚ति ? त‚स्मान्न द्व‚योर्निय‚तार्थ‚प्र‚द‚र्श‚क‚त्व‚म्, अपि तु‚{\tiny $_{lb}$}‚ एक‚स्यैव इति । तं साम‚स्त्य‚निषेध‚वादिनं प्र‚त्युप‚संह‚र‚न्नाह--\textbf{अत} इति । य‚स्माद् व‚स्तुप्र‚काशात्‚{\tiny $_{lb}$}‚ प्र‚त्य‚क्षं निय‚तं द‚र्श‚य‚ति । प‚रोक्ष‚स्यान्त‚रा साक्षाद‚न‚व‚भासेऽपि त‚त्स‚म्ब‚द्ध‚त‚या चानुमानं कार्य‚{\tiny $_{lb}$}‚स्व‚भाव‚ज‚मेकान्त‚निय‚तं भाव‚म्, अनुप‚ल‚म्भ‚ज‚मेकान्त‚निय‚त‚म‚भावं त‚थाभूतं द‚र्श‚य‚ति । \textbf{अतो}ऽस्माद्‚{\tiny $_{lb}$}‚ हेतोरेते द्वे अपि त‚थारूपे । त‚तोऽनेन साम‚स्त्य‚निषेध‚स्य निषेध उप‚संहृतः । अन्य‚था‚{\tiny $_{lb}$}‚ व्याख्याय‚माने तु य‚तोऽस्तु\edtext{}{\lemma{तोऽस्तु}\Bfootnote{य‚त‚स्तु}} निय‚त‚म‚र्थं प्र‚द‚र्श‚य‚तोऽत \textbf{एते} निय‚तार्थोप‚द‚र्श‚के इत्य‚र्थः स्यात् ।‚{\tiny $_{lb}$}‚ एत‚च्च प‚रिस्फुट‚स्यैव स्कुटीक‚र‚ण‚म‚न‚र्थ‚क‚माप‚द्येतेति ।
	\pend% ending standard par
      ‚{\tiny $_{lb}$}‚‚{\tiny $_{lb}$}‚\textsuperscript{\textenglish{22/dm}}‚{\tiny $_{lb}$}‚
	  \bigskip
	  \begingroup
	

	  \pstart \leavevmode% starting standard par
	आभ्यां प्र‚माणाभ्याम‚न्येन\edtext{}{\lemma{न्येन}\Bfootnote{श‚ब्दादिना--\cite{dp-msD-n}}} च\edtext{}{\lemma{च}\Bfootnote{०न्येन ज्ञानेन \cite{dp-edE} \cite{dp-edN}}} ज्ञानेन \edtext{}{\lemma{ज्ञानेन}\Bfootnote{प्र‚द‚र्शितो \cite{dp-msA} \cite{dp-msC} \cite{dp-edP} \cite{dp-edN}}}द‚र्शितोऽर्थः क‚श्चिद‚त्य‚न्त‚विप‚र्य‚स्तः । य‚था‚{\tiny $_{lb}$}‚ म‚रीचिकासु ज‚ल‚म् । \edtext{\textsuperscript{*}}{\lemma{*}\Bfootnote{अर्थः--\cite{dp-msD-n}}}स चास‚त्त्वात् प्राप्तुम‚श‚क्यः । क‚श्चिद‚निय‚तो भावाभाव‚योः\edtext{}{\lemma{योः}\Bfootnote{०योः त‚द्व[[च्च]] य‚था \cite{dp-msC}}} । य‚था‚{\tiny $_{lb}$}‚ संश‚यार्थः । न च भावाभावाभ्यां युक्तोऽर्थो ज‚ग‚त्य‚स्ति । त‚तः प्राप्तुम‚श‚क्य‚स्तादृशः ।\edtext{\textsuperscript{*}}{\lemma{*}\Bfootnote{अनिय‚तोऽर्थः--\cite{dp-msD-n}}}
	\pend% ending standard par
       ‚{\tiny $_{lb}$}‚ 

	  \pstart \leavevmode% starting standard par
	स‚र्वेण \edtext{}{\lemma{र्वेण}\Bfootnote{मान‚स‚विक‚ल्प०--\cite{dp-msD-n}}}चालिङ्गेन\edtext{}{\lemma{चालिङ्गेन}\Bfootnote{चालिङ्ग‚जेन \cite{dp-msC} \cite{dp-msD} \cite{dp-msA} \cite{dp-msB} \cite{dp-edP} \cite{dp-edH} \cite{dp-edE} \cite{dp-edN}}} विक‚ल्पेन \edtext{}{\lemma{ल्पेन}\Bfootnote{लिङ्ग‚म्--\cite{dp-msD-n}}}नियाम‚क‚म‚दृष्ट्वा प्र‚वृत्तेन \edtext{}{\lemma{वृत्तेन}\Bfootnote{स‚त्त्वास‚त्त्व‚योः--\cite{dp-msD-n}}}भावाभाव‚योर‚निय‚त
	\pend% ending standard par
      
	  \endgroup
	‚{\tiny $_{lb}$}‚

	  \pstart \leavevmode% starting standard par
	भ‚व‚तामेते निय‚तार्थोप‚द‚र्श‚के, प्र‚माणे तु क‚थ‚मित्याह--\textbf{तेन}--इति । \textbf{तेन} निय‚तार्थ‚प्र‚द‚र्श‚{\tiny $_{lb}$}‚क‚त्वेन । निय‚तार्थ‚प्र‚द‚र्श‚क‚त्वाभावात् किलाप्रामाण्य‚म‚न‚योः श‚ङिक‚त‚म् । स‚त्य‚पि त‚स्मिन्‚{\tiny $_{lb}$}‚ क‚थं त‚त् स्यादिति भावः । अन‚योः प्र‚वृत्तिविष‚य‚प्र‚द‚र्श‚न‚प्र‚कारोप‚द‚र्श‚ने त‚द‚न्य‚स्य ज्ञान‚स्य च‚{\tiny $_{lb}$}‚ त‚द‚प्र‚द‚र्श‚ने य‚द् बुद्धिस्थ‚मासीत्, त‚दिदानीं \textbf{नान्य‚दि}त्यादिनाऽभिव्य‚न‚क्ति । \textbf{नान्य‚द्} विज्ञानं‚{\tiny $_{lb}$}‚ प्र‚माण‚मिति व‚च‚न‚विप‚रिणामेन स‚म्ब‚न्धः ।
	\pend% ending standard par
      ‚{\tiny $_{lb}$}‚

	  \pstart \leavevmode% starting standard par
	य‚द्येवं ज्ञान‚त्वाविशेषाद‚मू अपि प्र‚माणे मा भूतामित्याह--\textbf{प्राप्तुम्} इति । \textbf{आद‚र्श‚य‚दिति}‚{\tiny $_{lb}$}‚ हेतौ श‚तुर्विधानाद्धेतुप‚द‚मेत‚त् । तेनाय‚म‚र्थः--य‚तः प्राप्तुं श‚क्य‚म‚र्थ‚माद‚र्श‚य‚ति तेन प्र‚त्य‚क्षादिकं‚{\tiny $_{lb}$}‚ प्राप‚क‚मिति । भ‚व‚तु प्राप‚क‚म्, प्र‚माणं तु क‚स्मादित्याह--\textbf{प्राप‚क‚त्वाच्च}--इति । \textbf{चो}ऽव‚धार‚णे ।
	\pend% ending standard par
      ‚{\tiny $_{lb}$}‚

	  \pstart \leavevmode% starting standard par
	न‚नु आभ्याम‚न्येनापि द‚र्शितोऽर्थः श‚क्य‚प्राप‚ण एव त‚त‚स्त‚दुप‚द‚र्श‚क‚म‚पि किं न प्राप‚कं‚{\tiny $_{lb}$}‚ प्राप‚क‚त्वाच्च किं न प्र‚माण‚मित्याह--प्र\textbf{माणाभ्याम्} इति । \textbf{चो} य‚स्मात् । \textbf{अत्य‚न्त‚ग्र‚ह‚णे}न‚{\tiny $_{lb}$}‚ संश‚य‚ज्ञान‚विष‚याद् विशेषं द‚र्श‚य‚ति । त‚थाविधोऽपि किं न प्राप्येत इत्याह--\textbf{स च} इति ।‚{\tiny $_{lb}$}‚ \textbf{चो} य‚स्माद‚र्थे । निय‚तोप‚द‚र्श‚क‚त्वेऽप्य‚न‚र्थोप‚द‚र्श‚क‚त्वाद‚प्रामाण्य‚म‚स्य द‚र्शित‚म् । य‚दि क‚श्चि‚{\tiny $_{lb}$}‚दीदृश‚स्त‚द‚न्य‚ज्ञान‚विष‚योऽन्यादृशो भ‚विष्य‚ति । त‚दुप‚द‚र्श‚कं च प्र‚माणं भ‚विष्य‚तीत्य‚त आह—‚{\tiny $_{lb}$}‚\textbf{क‚श्चिद्} इति । \textbf{संश‚यार्थः} संश‚याल‚म्ब‚नः स्थाणुर्वा पुरुषो वाइति हि प्र‚त्य‚यः स्थाणुमुल्लिख्य‚{\tiny $_{lb}$}‚ पुरुषो वेत्याल‚म्ब‚यंस्त‚द‚भाव‚मुल्लिख‚ति । त‚तः स्थाण्व‚भावाव्य‚भिचारिणं पुरुषं पुरुष\edtext{}{\lemma{पुरुष}\Bfootnote{\textbf{षा}}}‚{\tiny $_{lb}$}‚भावाव्य‚भिचारिणं च स्थाणुम‚व‚स्य‚न्न भावे नाप्य‚भाव\edtext{}{\lemma{भाव}\Bfootnote{वे}} निय‚तं स्थाणं पुरुषं वा द‚र्श‚य‚तीति‚{\tiny $_{lb}$}‚ भावाभाव‚योर‚निय‚तं द‚र्श‚य‚ति । य‚त एवाय‚मेकान्त‚निय‚तं द‚र्श‚यितुम‚नीशानो दोलाय‚ते, त‚त‚{\tiny $_{lb}$}‚ एव संश‚य इत्युप‚प‚द्य‚त इति ।
	\pend% ending standard par
      ‚{\tiny $_{lb}$}‚

	  \pstart \leavevmode% starting standard par
	अथ विप‚र्य‚यार्थोऽस‚त्त्वान्न प्राप्य‚ताम्, अयं तु क‚स्मान्न प्राप्य‚त इत्याह--\textbf{न च} इति ।‚{\tiny $_{lb}$}‚ \textbf{चो} य‚स्माद‚र्थे ।
	\pend% ending standard par
      ‚{\tiny $_{lb}$}‚

	  \pstart \leavevmode% starting standard par
	न‚नु भ‚व‚तु संश‚य‚विप‚र्य‚य‚योर‚प्रामाण्यं\leavevmode\ledsidenote{\textenglish{10b/ms}} किं न‚श्छिन्न‚म् ? एत‚द‚तिरिक्तं प्र‚त्य‚क्षानु‚{\tiny $_{lb}$}‚मानाभ्याम‚न्य‚त्प्र‚माणं भ‚विष्य‚तीत्याश‚ङ्क्याह--\textbf{स‚र्वेण} च--इत्यादि । \textbf{चो} य‚स्मात् । स‚र्वान्त‚{\tiny $_{lb}$}‚‚{\tiny $_{lb}$}‚ ‚{\tiny $_{lb}$}‚ \leavevmode\ledsidenote{\textenglish{23/dm}}‚{\tiny $_{lb}$}‚ 
	  
	एवार्थो द‚र्श‚यित‚व्यः । स\edtext{}{\lemma{स}\Bfootnote{अनिय‚तोर्थः--\cite{dp-msD-n}}} च प्राप्तुम‚श‚क्यः । त‚स्माद‚श‚क्य‚प्राप‚ण‚म्--अत्य‚न्त‚विप‚रीत‚म्,‚{\tiny $_{lb}$}‚ भावाभावानिय‚तं चार्थं द‚र्श‚य‚द् अप्र‚माण‚म‚न्य‚ज्ज्ञान‚म् । अर्थ‚क्रियार्थिभिश्चार्थ‚क्रियास‚म‚र्थ\edtext{}{\lemma{र्थ}\Bfootnote{०स‚म‚र्थार्थ‚प्रा० \cite{dp-msA} \cite{dp-edH} \cite{dp-edP} \cite{dp-edN} ०स‚म‚र्थ‚प्रा० \cite{dp-msB} \cite{dp-edE}}}‚{\tiny $_{lb}$}‚व‚स्तुप्राप्तिनिमितं ज्ञानं मृग्य‚ते । य‚च्च तैर्मृग्य‚ते त‚देव\edtext{}{\lemma{देव}\Bfootnote{त‚देव तेन शास्त्रे \cite{dp-msB} \cite{dp-edH}}} शास्त्रे विचार्य‚ते । त‚तोऽर्थ‚क्रिया‚{\tiny $_{lb}$}‚स‚म‚र्थ‚व‚स्तुप्र‚द‚र्श‚कं स‚म्य‚ग्ज्ञान‚म्\edtext{}{\lemma{म्}\Bfootnote{इति त‚द भ‚व‚ति स‚म्य‚ग्ज्ञान‚मिति शेषः--\cite{dp-msD-n}}} ।‚{\tiny $_{lb}$}‚ र्ग‚त‚त्वाद‚नुमान‚स्यापि त‚थात्वं स्यादित्याह--\textbf{अलिङ्गेन}--इति । नास्य लिङ्ग‚मुत्पाद‚क‚त्वेन‚{\tiny $_{lb}$}‚ विद्य‚त इत्य‚लिङ्ग‚म् । य‚द्य‚लिङ्गेन स‚र्वेण त‚त्क‚र‚णीयं त‚दा प्र‚त्य‚क्ष‚पृष्ठ‚भाविनोऽपि विक‚ल्प‚स्य‚{\tiny $_{lb}$}‚ त‚दाऽऽयात‚मित्याह--\textbf{नियाम‚क‚म्} इति । विष‚याधीनो हि निय‚तार्थ‚प‚रिग्र‚हो ज्ञान‚स्येत्य‚र्थो‚{\tiny $_{lb}$}‚ नियाम‚कः । त‚म‚दृष्ट्वा ।
	\pend% ending standard par
      ‚{\tiny $_{lb}$}‚

	  \pstart \leavevmode% starting standard par
	न‚नु किं प्र‚त्य‚क्ष‚पृष्ठ‚भाव्य‚पि नियाम‚क‚म‚र्थं दृष्ट्वा प्र‚व‚र्त्त‚ते, येन ताद्रूप्य‚विर‚हाद‚न्येषां‚{\tiny $_{lb}$}‚ विक‚ल्पानां त‚थात्व‚माश‚ङ्क्य‚ते । उच्य‚ते । \textbf{प्र‚व‚र्त्तेने}त्य‚त्रान्त‚र्भूतो णिज‚र्थो द्र‚ष्ट‚व्यः । त‚तोऽय‚म‚र्थः‚{\tiny $_{lb}$}‚नियाम‚क‚म‚दृष्ट्वा प्र‚व‚र्त्तितेनेति । त‚था च स‚ति प्र‚त्य‚क्ष‚पृष्ठ‚भावी विक‚ल्पो नियाम‚कं दृष्ट्वैव‚{\tiny $_{lb}$}‚ प्र‚त्य‚क्षेण प्र‚त्य‚येन प्र‚व‚र्त्त्य‚त इति त‚थात्वेन प‚रिहृतो भ‚व‚ति । इत‚रे त्व‚लिङ्ग‚विक‚ल्पा येन प्र‚व‚र्त्त्य‚न्ते‚{\tiny $_{lb}$}‚ जाय‚न्ते न तेन \textbf{साक्षा}न्नियाम‚को दृष्ट इति तेषां त‚थात्वानुष‚ङ्गः ।
	\pend% ending standard par
      ‚{\tiny $_{lb}$}‚

	  \pstart \leavevmode% starting standard par
	अन्ये तु अनुमान‚निवृत्त्य‚र्थ‚म‚लिङ्गेति योज‚यित्वा क‚थं पुन‚र‚व‚सायात्म‚केनाप्य‚लिङ्गेन‚{\tiny $_{lb}$}‚ तेनैवं क‚र‚णीय‚मित्याश‚ङ्क्य नियाम‚क‚म‚दृष्ट्वेत्य‚स्य तु हेतुभावेन विशेष‚ण‚त्वान्नियाम‚क‚म‚दृष्ट्वा‚{\tiny $_{lb}$}‚ प्र‚वृत्त‚त्वादित्य‚र्थः, इति व्याख्याय तात्प‚र्यार्थ‚म‚पि द‚र्श‚य‚न्ति । य‚द्य‚पि श‚ब्दादिज‚न्मानो विक‚ल्पा‚{\tiny $_{lb}$}‚ नोभ‚य‚प‚क्ष‚संस्प‚र्शेन दोलाय‚न्ते, त‚थापि ते नियाम‚कानाश्र‚येण प्र‚व‚र्त्त‚माना न यौक्ति\edtext{}{\lemma{यौक्ति}\Bfootnote{क्त}}संश‚य‚रूप‚{\tiny $_{lb}$}‚ताम‚तिव‚र्त्त‚न्त इति । पूर्व‚व्याख्यानेऽप्य‚य‚मेवाश‚यः । स‚र्व‚स्य त‚स्य स‚मुदाच‚र‚तोर्विरुद्ध‚योराकार‚{\tiny $_{lb}$}‚योर‚भावेऽपि युक्त्या द्वितीयाकारानुप्र‚वेशात्संश‚य‚रूप‚त्वेन भावाभावानिय‚तार्थोप‚द‚र्श‚क‚त्व‚मिति ।
	\pend% ending standard par
      ‚{\tiny $_{lb}$}‚

	  \pstart \leavevmode% starting standard par
	एके तु नियाम‚कं विक‚ल्प‚यित‚व्यं व‚स्तुनान्त‚रीय‚कं व‚स्त्व‚दृष्ट्वा प्र‚वृत्तेनेति योज‚य‚न्ति ।‚{\tiny $_{lb}$}‚ \textbf{अलिङ्गे}नेति चास्यैवार्थ‚स्य हेतुभावेन विशेष‚ण‚माहुः । उभ‚येऽपि तु प्र‚त्य‚क्ष‚पृष्ठ‚भाविविक‚ल्पेनाति‚{\tiny $_{lb}$}‚प्र‚स‚ङ्ग‚म‚न‚धिग‚तार्थाधिग‚न्तुर्विक‚ल्प्य\edtext{}{\lemma{ल्प्य}\Bfootnote{ल्प}}स्य प्रामाण्य‚चिन्ताऽधिकारेण निराकुर्व‚ते । तेनाय‚म‚र्थः—‚{\tiny $_{lb}$}‚स‚र्वेण तेनान‚धिग‚त‚त्वा\edtext{}{\lemma{त्वा}\Bfootnote{ग‚तार्था}}धिग‚न्तृत‚याऽभिप्रेतेनैव‚म‚व‚श्य‚क‚र‚णीय‚मिति ।
	\pend% ending standard par
      ‚{\tiny $_{lb}$}‚

	  \pstart \leavevmode% starting standard par
	अनिय‚तार्थ‚प्र‚द‚र्श‚क‚म‚पि त‚त्प्राप‚कं स्यादित्याह--\textbf{स च} इति । \textbf{चो} य‚स्माद‚र्थे । \textbf{अश‚क्य}‚{\tiny $_{lb}$}‚ इत्य‚स्यान‚न्त‚रं स अव‚धार‚णार्थो द्र‚ष्ट‚व्यः । \textbf{सो}ऽनिय‚तोऽर्थो न प्राप्तुं \textbf{श‚क्यो}ऽश‚क्य‚त्वादिति भावः ।
	\pend% ending standard par
      ‚{\tiny $_{lb}$}‚

	  \pstart \leavevmode% starting standard par
	न‚नु च तादृशं स‚र्वं मा भूत्प्राप‚क‚म्, प्र‚माणं तु क‚स्मान्न भ‚व‚तीत्याश‚ङ्क्य पूर्वोक्त‚मेवोप‚{\tiny $_{lb}$}‚संह‚र‚ति \textbf{त‚स्माद्} इति । य‚स्मात्प्राप‚क‚त्वादेव प्र‚माणं \textbf{त‚स्मात्} । अर्थ‚स्याश‚क्य‚प्राप‚ण‚त्वेऽत्य‚न्त‚{\tiny $_{lb}$}‚विप‚रीत‚त्वानित्य\edtext{}{\lemma{त्वानित्य}\Bfootnote{त्वानिय‚त}}त्वे हेतू हेतुभावेनान‚योर्विशेष‚ण‚त्वात् । \textbf{चो}ऽप्राप्तिकार्थ‚स‚मुच्च‚ये ।‚{\tiny $_{lb}$}‚ क‚श्चिद‚त्य‚न्त‚विप‚र्य‚स्तः, क‚श्चिद‚निय‚त इत्यादेर्य‚थाक्र‚म‚मुप‚संहारः--\textbf{अन्य‚ज्ज्ञान}मिति \textbf{नान्य‚ज्ज्ञान‚{\tiny $_{lb}$}‚मित्य}स्य ।
	\pend% ending standard par
      ‚{\tiny $_{lb}$}‚‚{\tiny $_{lb}$}‚\textsuperscript{\textenglish{24/dm}}‚{\tiny $_{lb}$}‚
	  \bigskip
	  \begingroup
	

	  \pstart \leavevmode% starting standard par
	य‚च्च तेन\edtext{}{\lemma{तेन}\Bfootnote{ज्ञानेन--\cite{dp-msD-n}}} प्र‚द‚र्शितं त‚देव\edtext{}{\lemma{देव}\Bfootnote{त‚देव तेन प्रा० \cite{dp-msC} \cite{dp-msD}}} प्राप‚णीय‚म् । \edtext{\textsuperscript{*}}{\lemma{*}\Bfootnote{न‚नु च य‚दि उप‚द‚र्शितार्थ‚प्राप‚णात् प्र‚माणं स‚म्य‚ग्ज्ञानं शुक्ल‚श‚ङ्खे पीत‚ज्ञानं कुञ्चिका‚{\tiny $_{lb}$}‚विव‚र‚म‚णिप्र‚मायां म‚णिज्ञान‚म्, अर्ध‚रात्रे म‚ध्याह्न‚काल‚ग्राहिस्व‚प्न‚ज्ञानं च प्र‚माण‚माप्नोति ।‚{\tiny $_{lb}$}‚ एभिरुप‚द‚र्शितार्थ‚स्यार्थ‚मात्र‚स्य प्राप्तेस्त‚त‚श्च स‚म्य‚ग्ज्ञान‚मित्याश‚ङ्क्याह--\cite{dp-msD-n}}}अर्थाधिग‚मात्म‚कं\edtext{}{\lemma{कं}\Bfootnote{अर्थाधिग‚मात्म‚क‚त्वं हि प्राप‚क‚त्व‚मि० \cite{dp-msC} \cite{dp-msD} अर्थ‚क्रियास‚म‚र्थ‚व‚स्त्व‚धिग‚मात्म‚क‚त्व‚म्--\cite{dp-msB}}} हि \edtext{}{\lemma{हि}\Bfootnote{प्राप‚ण‚मि० \cite{dp-edN}}}प्राप‚क‚मित्युक्त‚म् ।
	\pend% ending standard par
      
	  \endgroup
	‚{\tiny $_{lb}$}‚

	  \pstart \leavevmode% starting standard par
	स्यादेत‚त्--श‚क्य‚प्राप्तिकार्थोप‚द‚र्श‚क‚मेव स‚म्य‚ग्ज्ञानं\leavevmode\ledsidenote{\textenglish{11a/ms}} \textbf{नान्य‚दि}ति निर्निमित्त‚मिद‚म् ।‚{\tiny $_{lb}$}‚ त‚था चान्य‚द‚पि किं न प‚रीक्ष्य‚त इत्याह--\textbf{अर्थ‚क्रियार्थिभिः}--इति । \textbf{चो} य‚स्मात् । \textbf{अर्थ}स्य‚{\tiny $_{lb}$}‚ दाहादेः, \textbf{क्रिया}निष्प‚त्तिः । ताम‚र्थ‚य‚न्ताति त‚था । एतैर\textbf{र्थ‚क्रिया}यां श‚क्त‚स्य व‚स्तुनः \textbf{प्राप्ते‚{\tiny $_{lb}$}‚र्य‚न्निमित्तं त‚न्मृग्य‚ते}ऽन्विष्य‚ते । अर्थ‚क्रियार्थित्वादेव च त‚द‚न्वेष‚ण‚मेषाम‚व‚सेय‚म् । तैश्च‚{\tiny $_{lb}$}‚ प्रेक्षाव‚द्भिरिति प्र‚क‚र‚णाद् द्र‚ष्ट‚व्य‚म् । इत‚र‚थार्थ‚क्रियार्थिभिर‚प्य‚प्रेक्षाव‚द्भिर्मिथ्याज्ञान‚म‚पि‚{\tiny $_{lb}$}‚ निभाल्य‚त इति त‚त्प‚रीक्षाऽप्याप‚तेत् ।
	\pend% ending standard par
      ‚{\tiny $_{lb}$}‚

	  \pstart \leavevmode% starting standard par
	य‚दि नाम तैस्त‚न्मृग्यं त‚थापि त‚दित‚र‚द‚पि ज्ञानं किं न प‚रीक्ष्य‚त इत्याह \textbf{य‚च्च}--इति ।‚{\tiny $_{lb}$}‚ \textbf{चो}ऽव‚धार‚णे । \textbf{विचार्य‚ते} प‚रीक्ष्य‚ते । ईदृशं त\textbf{त्स‚म्य‚ग्ज्ञानं} प्र‚व‚र्त्त‚कं य‚द‚नुस‚र‚न्ति भ‚व‚न्त इति‚{\tiny $_{lb}$}‚ विभ‚ज्य प्र‚तिपाद्य‚ते । अय‚म‚स्याश‚यः--न व्य‚स‚नित‚याऽऽ\textbf{चार्येण} ज्ञानं विचार्य‚ते । किन्त‚र्हि ?‚{\tiny $_{lb}$}‚ अर्थ‚क्रियार्थिज‚नानुरोधेन । ते त‚थाभूत‚मेव ज्ञानं मृग‚य‚न्ते नान्य‚दिति \textbf{त‚देव} विचार्य‚त इति ।‚{\tiny $_{lb}$}‚ \textbf{त‚देवे}ति च मुख्य‚वृत्त्य‚भिप्रायेणोक्त‚म् न तु मिथ्याज्ञानं शास्त्रेऽस्मिन् न विचार्य‚ते एव ।‚{\tiny $_{lb}$}‚ प्र‚स‚ङ्गात्तु त‚स्यापि क‚ल्प‚नाप्र‚भृतौ विचार‚स‚म्भ‚वादिति ।
	\pend% ending standard par
      ‚{\tiny $_{lb}$}‚

	  \pstart \leavevmode% starting standard par
	न‚नु श‚क्य‚प्राप‚णार्थोप‚द‚र्श‚क‚मेव स‚म्य‚ग्ज्ञानं कुतो व्य‚व‚स्थाप्य‚ते नान्य‚दिति प्र‚श्ने किमिद‚म‚{\tiny $_{lb}$}‚प्र‚कृत‚मुच्य‚त इत्याह--\textbf{त‚त} इति । य‚तोऽर्थ‚क्रियार्थिनां प्रेक्षाव‚तां नान्य‚न्मृग‚य‚णीय‚म्, त‚द‚न्वेष‚णीय‚{\tiny $_{lb}$}‚मेव विचार‚णीय‚म्, व्य‚स‚नित‚या विचारास‚म्भ‚वात् । त‚तोऽर्थ‚क्रियास‚म‚र्थ‚स्य व‚स्तुनः प्र‚द‚र्श‚कं‚{\tiny $_{lb}$}‚ स‚म्य‚ग्ज्ञानं नान्य‚दिति अर्थात् ।
	\pend% ending standard par
      ‚{\tiny $_{lb}$}‚

	  \pstart \leavevmode% starting standard par
	अस्तु स‚म्य‚ग्ज्ञान‚मीदृश‚मेव । त‚त्पुन‚र‚न्य‚दुप‚द‚र्श्य अन्य‚त् प्राप‚य‚द‚पि किं न स‚म्य‚ग्ज्ञानं‚{\tiny $_{lb}$}‚ स‚त् प्र‚माणं व्य‚व‚ह्रिय‚ते ? त‚था च पीत‚संख्या\edtext{}{\lemma{संख्या}\Bfootnote{श‚ङ्खा}}दिज्ञान‚म‚पि प्रामाण्यान्नापैति । आह--\textbf{य‚च्च}‚{\tiny $_{lb}$}‚ इति । \textbf{चो}ऽव‚धार‚णे । अनेन पीत‚सं\edtext{}{\lemma{सं}\Bfootnote{श‚ङ्}}खादिज्ञान‚म‚प्यंशे संवादात् प्र‚माण‚मिति य‚त्कैश्चि‚{\tiny $_{lb}$}‚दिष्टिं त‚द‚पि स‚म्य‚ग्ज्ञान‚व्य‚व‚च्छेद्य‚मिति द‚र्श‚य‚ति ।
	\pend% ending standard par
      ‚{\tiny $_{lb}$}‚

	  \pstart \leavevmode% starting standard par
	न‚नु चानुक्त‚स‚म‚मिदं पीत‚श‚ङ्खादिज्ञानेनानुप‚द‚र्शित‚स्यापि शुक्ल‚श‚ङ्ख‚स्य, ज‚ल‚ज्ञानेनानुप‚{\tiny $_{lb}$}‚द‚र्शित‚स्यापि म‚रीचिकानिच‚य‚स्य प्राप‚ण‚द‚र्श‚नात् । नैत‚द‚स्ति । य‚तो न त‚ज्ज्ञान‚पूर्विका सा‚{\tiny $_{lb}$}‚ प्राप्तिः, त‚स्याऽज्ञात‚व‚स्तुविष‚य‚त्वात् । न च तेन शुक्ल‚श‚ङ्ख‚व‚स्तु म‚रीचिव‚स्तु वा ज्ञात‚म् । य‚था‚{\tiny $_{lb}$}‚ तु ज्ञानान्त‚रादेव त‚थाविधार्थ‚प्राप्तिर्न त‚त‚स्त‚था अनेनैव \textbf{प्रामाण्य‚प‚रीक्षायां} निर्लोठित‚मिति‚{\tiny $_{lb}$}‚ नेहोच्य‚ते ।
	\pend% ending standard par
      ‚{\tiny $_{lb}$}‚‚{\tiny $_{lb}$}‚\textsuperscript{\textenglish{25/dm}}‚{\tiny $_{lb}$}‚
	  \bigskip
	  \begingroup
	

	  \pstart \leavevmode% starting standard par
	त‚त्र प्र‚द‚र्शिताद‚न्य‚द्व‚स्तु भिन्नाकार‚म्, भिन्न‚देश‚म्, भिन्न‚कालं च । विरुद्ध‚ध‚र्म‚संस‚र्गाद्धि अन्य‚द्‚{\tiny $_{lb}$}‚ व‚स्तु । देश‚कालाकार‚भेद‚श्च विरुद्ध‚ध‚र्म‚संस‚र्गः ।
	\pend% ending standard par
       ‚{\tiny $_{lb}$}‚ 

	  \pstart \leavevmode% starting standard par
	त‚स्माद् अन्याकार‚व‚द्व‚स्तुग्राहि\edtext{}{\lemma{स्तुग्राहि}\Bfootnote{०न्याकार‚व‚स्तुग्राहि--\cite{dp-msB} \cite{dp-msC} \cite{dp-msD}}} नाकारान्त‚र‚व‚ति व‚स्तुनि प्र‚माण‚म् । य‚था पीत‚श‚ङ्ख‚ग्राहि‚{\tiny $_{lb}$}‚ शुक्ले श‚ङ्खे । देशान्त‚र‚स्थ‚ग्राहि च न देशान्त‚र‚स्थे प्र‚माण‚म् । य‚था कुञ्चिकाविव‚र‚देश‚स्थायां‚{\tiny $_{lb}$}‚ म‚णिप्र‚भायां म‚णिग्राहि ज्ञानं नाप‚व‚र‚क‚स्थे\edtext{}{\lemma{स्थे}\Bfootnote{०व‚र‚क‚देश‚स्थे--\cite{dp-msA} \cite{dp-msB} \cite{dp-msC} \cite{dp-msD} \cite{dp-edP} \cite{dp-edE} \cite{dp-edH} \cite{dp-edN}}} म‚णौ । कालान्त‚र‚युक्त‚ग्राहि च न कालान्त‚र‚व‚ति‚{\tiny $_{lb}$}‚ व‚स्तुनि प्र‚माण‚म् । य‚थार्द्ध‚रात्रे म‚ध्याह्न‚काल‚व‚स्तुग्राहि स्व‚प्न‚ज्ञानं नार्ध‚रात्र‚काले\edtext{}{\lemma{काले}\Bfootnote{०अर्ध‚रात्र‚काल‚व‚स्तुनः \cite{dp-edN} अर्ध‚रात्रः कालो य‚स्य म‚ध्याह्न‚काल‚व‚स्तुनः \cite{dp-msD-n} ।}} व‚स्तुनि‚{\tiny $_{lb}$}‚ प्र‚माण‚म् ।
	\pend% ending standard par
      
	  \endgroup
	‚{\tiny $_{lb}$}‚

	  \pstart \leavevmode% starting standard par
	त‚देव तेन प्राप‚णीय‚मिति कुत एत‚दित्याह--\textbf{अर्थाधिग‚मे}ति । \textbf{हि}र्य‚स्मात् । \textbf{उक्त}‚{\tiny $_{lb}$}‚मिति प्र‚द‚र्श‚नादीनां व‚स्तुतोऽभेदं प्र‚तिपाद्य \textbf{अत एवार्थाधिग‚तिरेव प्र‚माण‚फ‚ल‚मिति} अनेन‚{\tiny $_{lb}$}‚ प्र‚काशित‚त्वात्, न तु साक्षाद‚भिहित‚म् । य‚द्य‚न्य‚द‚धिग‚म्यान्य‚त् प्राप‚येत्, त‚दा प्र‚द‚र्श‚न‚प्राप‚ण‚यो‚{\tiny $_{lb}$}‚र्भेद एव स्यात्, अभेद‚श्च प्र‚तिपादित इति भावः ।
	\pend% ending standard par
      ‚{\tiny $_{lb}$}‚

	  \pstart \leavevmode% starting standard par
	क‚थं पुन‚र\textbf{धिग‚तिरेव फ‚ल}मित्य‚नेन त‚थात्व‚म‚स्योक्त‚म् ? य‚तः प्राप‚क‚त्वं प्रामाण्य‚म्, फ‚लं च‚{\tiny $_{lb}$}‚ त‚द‚व्य‚तिरिक्त‚मिति । अथ‚वा \textbf{प्राप्तुं श‚क्य‚म‚र्थ‚माद‚र्श‚य‚त्प्राप‚क‚मित्य}नेनैव‚मुक्त‚म् । त‚थाहि‚{\tiny $_{lb}$}‚ तादृश‚म‚र्थ‚मा\textbf{द‚र्श‚य‚दिति} प‚रिच्छिन्द‚दित्य‚र्थः । प‚रिच्छेद‚श्च प्राप‚क‚त्वान्न भिद्य‚त इति ।
	\pend% ending standard par
      ‚{\tiny $_{lb}$}‚

	  \pstart \leavevmode% starting standard par
	भ‚व‚तूप‚द‚र्शितार्थ‚प्राप‚क‚त्व‚मेव स‚म्य‚ग्ज्ञान‚म् । य‚त्तु त‚देव प्राप्य व‚स्तु श‚ङ्खादिकं पीत‚रूप‚त‚या‚{\tiny $_{lb}$}‚ प्र‚द‚र्श‚य‚ति त‚त्क‚थ‚म‚स‚म्य‚ग्ज्ञान‚मित्याह--\textbf{त‚त्र} इति । \textbf{त‚त्रे}ति वाक्योप‚क्षेपे चैत‚त् । \textbf{प्र‚द‚र्शिताद्} रूपाद्‚{\tiny $_{lb}$}‚ \textbf{भिन्नाकारादि} स‚द् \textbf{व‚स्त्व‚न्य‚दित्य}न्य‚त्वं विधेय‚म् । आकारादिभिन्न‚म‚पि न त‚तोऽन्य‚त्, त‚देवेद‚{\tiny $_{lb}$}‚मित्य‚ध्य‚व‚सायादित्याह--\textbf{विरुद्धे}ति । हिर्य‚स्मात् । य‚दि विरुद्ध‚ध‚र्माध्यासाद‚न्य‚त्व‚म्, न‚{\tiny $_{lb}$}‚ त‚र्हि देशादिभेदादित्याह--\textbf{देशे}ति । \textbf{चो} य‚स्माद‚र्थे । भ‚व‚त्वेवं त‚थाप्याकारादिभिन्न‚ग्राहि‚{\tiny $_{lb}$}‚ज्ञान‚माकारान्त‚रादियोगिनि व‚स्तुनि किं न प्र‚माण‚मित्याह--\textbf{त‚स्माद्} इति । य‚स्मात्स‚म्य‚ग्ज्ञानेन‚{\tiny $_{lb}$}‚ य‚देव हि द‚र्शितं त‚देव प्राप‚णीय‚म्, आकारादिभिन्नं च त‚तोऽन्य‚त् \textbf{त‚स्मात्} । किमिव क्व‚{\tiny $_{lb}$}‚ न प्र‚माण‚मित्याह--\textbf{य‚थेति} । भिन्न‚देश‚ग्राहिणः का वार्तेत्याह--\textbf{देशान्त‚रे}ति । किंव‚न्न‚{\tiny $_{lb}$}‚ प्र‚माण‚मित्याह--\textbf{य‚थे}ति । \textbf{अव\edtext{}{\lemma{अव}\Bfootnote{प}}व‚र‚क}श‚ब्देन देश‚विशेषो ल‚य‚नोद‚रादिश‚ब्द‚वाच्य उच्य‚ते ।‚{\tiny $_{lb}$}‚ अप‚व‚र‚क‚श‚ब्द‚स्याप्येत‚द‚र्थ‚स्य भावात् । \textbf{अप‚व‚र‚क‚देश‚स्थ} इति पाठेऽपि न दोषः ।
	\pend% ending standard par
      ‚{\tiny $_{lb}$}‚

	  \pstart \leavevmode% starting standard par
	भिन्न‚काल‚ग्राहिणः का व्य‚व‚स्था इत्याह--\textbf{कालान्त‚रे}ति । उप‚द‚र्श‚न‚कालाद‚न्येन \textbf{कालेन}‚{\tiny $_{lb}$}‚ व्याव‚हारिकेण \textbf{युक्तं} स‚म्ब‚द्धं त‚द्\textbf{ग्राहि} भिन्न‚काल‚व‚स्तुग्राहीति याव‚त् । \textbf{चः} पूर्व‚व‚त् पूर्वेण स‚हेद‚{\tiny $_{lb}$}‚म‚प्रामाण्येन स‚मुच्चिनोति । त‚दुदाह‚र‚न्नाह--\textbf{य‚थे}ति । अय‚म‚स्यार्थः--अध‚रात्रे निद्राण‚स्य म‚ध्याह्न‚{\tiny $_{lb}$}‚काल‚प्र‚तिभास‚म‚नुभ‚व‚तो व‚णिज्याग‚त‚सुत‚व‚स्तुद‚र्श‚न‚हृषित‚रोम्ण एव झ‚टिति बोधे काक‚तालीय‚{\tiny $_{lb}$}‚न्यायेन च स‚त्य‚पि पुत्रोप‚ल‚म्भे त‚ज्ज्ञानं न प्र‚माण‚म् । त‚त्ख‚लु प‚र‚मार्थेनार्ध‚रात्र‚काल‚स‚म्ब‚द्धं‚{\tiny $_{lb}$}‚ म‚ध्याह्न‚कालाक‚लितं च प्र‚तीत‚मिति म‚ध्याह्न‚काल‚त्वेनाव‚भास‚नादेव च म‚ध्याह्न‚कालं त‚द्व‚स्तूक्त‚म् ।‚{\tiny $_{lb}$}‚ ‚{\tiny $_{lb}$}‚ \leavevmode\ledsidenote{\textenglish{26/dm}}‚{\tiny $_{lb}$}‚ 
	  
	न‚नु \edtext{}{\lemma{नु}\Bfootnote{न‚नु देश \cite{dp-msA} \cite{dp-edP} \cite{dp-edE}}}च देश‚निय‚त‚म्, आकार‚निय‚तं च प्राप‚यितुं श‚क्य‚म् । य‚त्कालं तु प‚रिच्छिन्नं‚{\tiny $_{lb}$}‚ त‚त्कालं न श‚क्यं प्राप‚यितुम् । नोच्य‚ते--य‚स्मिन्नेव‚काले प‚रिच्छिद्य‚ते त‚स्मिन्नेव काले प्राप‚यि‚{\tiny $_{lb}$}‚त‚व्य‚मिति । अन्यो हि द‚र्श‚न‚कालः, अन्य‚श्च प्राप्तिकालः । किन्तु य‚त्कालं प‚रिच्छिन्नं त‚देव‚{\tiny $_{lb}$}‚ तेन\edtext{}{\lemma{तेन}\Bfootnote{त‚देव प्रा० \cite{dp-msD} \cite{dp-msA} \cite{dp-edP} \cite{dp-edE}}} प्राप‚णीय‚म् । अभेदाध्य‚व‚सायाच्च स‚न्तान‚ग‚त‚मेक‚त्वं द्र‚ष्ट‚व्य‚मिति ।‚{\tiny $_{lb}$}‚ \textbf{अर्ध‚रात्रः काल} आग‚म‚न‚कालो य‚स्य त‚स्मिन् \textbf{न प्र‚माण‚म्} । अथ‚वा योऽर्ध‚रात्रे सुप्तोऽह्नो म‚ध्य‚मुद्ग‚ते‚{\tiny $_{lb}$}‚ स‚वित‚रि पुत्र‚माग‚तं दृष्ट‚वा प्र‚भातायां रात्रौ स‚म्प‚न्ने म‚ध्य‚न्दिने त‚स्मिन्नेव च देशे त‚मेव सुत\edtext{}{\lemma{सुत}\Bfootnote{तं}}‚{\tiny $_{lb}$}‚ प्राग‚तं प्र‚बुद्धोऽपि प‚श्य‚ति । त‚स्य त‚ज्ज्ञानं संवाद‚मात्र‚भाग‚पि न प्र‚माण‚म्, य‚तोऽर्ध‚रात्रे‚{\tiny $_{lb}$}‚ म‚ध्याह्न‚म्, त‚दानीम‚नाग‚त‚म‚पि पुत्र‚व‚स्त्वाग‚त‚म्, तेनाव‚ग‚त‚म् । न चार्ध‚रात्रे देश‚काल‚निय‚त‚स्य‚{\tiny $_{lb}$}‚ पुत्र‚स्य, त‚स्य च काल‚स्यास्ति स‚द्भावः । त‚दा त्व‚र्ध‚रात्रः प‚र‚मार्थ‚तः प्र‚तिभास‚कालो य‚स्य‚{\tiny $_{lb}$}‚ म‚ध्याह्न‚काल‚युक्त‚व‚स्तुन इति योज्य‚म् ।
	\pend% ending standard par
      ‚{\tiny $_{lb}$}‚

	  \pstart \leavevmode% starting standard par
	एतेन सुप्त‚स्य केन‚चित् प‚ठ्य‚मानं ग्र‚न्थं शृण्व‚तो ज्ञान‚म‚स‚त्यार्थं व्याख्यात‚म‚व‚सेय‚म् ।‚{\tiny $_{lb}$}‚ य‚स्माद्देश‚काल‚भिन्नात्मानं श्रोतारं ग्र‚न्थं च श्रोत‚व्यं प‚श्य‚ति निद्रोप‚ह‚तः । न च त‚द्देश‚काल‚स‚म्ब‚द्धौ‚{\tiny $_{lb}$}‚ स्तः । य‚द्देश‚कालौ च स्त‚स्त‚था न गृह्णातीति ।
	\pend% ending standard par
      ‚{\tiny $_{lb}$}‚

	  \pstart \leavevmode% starting standard par
	अन‚यैव च दिशा वाजिस्व‚प्नोऽपि व्याख्यातो द्र‚ष्ट‚व्यः । य‚त‚स्त‚त्रापि घोट‚क‚स्व‚प्ने‚{\tiny $_{lb}$}‚ य‚त्काल‚द‚र्श‚न‚वि\leavevmode\ledsidenote{\textenglish{12a/ms}}ष‚या भावा दृश्य‚न्ते न त‚था स‚न्ति, य‚त्कालाश्च स‚न्ति त‚त्काला न दृश्य‚न्ते ।‚{\tiny $_{lb}$}‚ अर्ध‚रात्रादिषु हि स्व‚प्न‚द‚र्श‚न‚म्, सूर्योद‚यादिस‚म्ब‚द्धाश्च ते भावा दृश्य‚न्ते । त‚स्मात्स्व‚प्ने केषाञ्चि‚{\tiny $_{lb}$}‚द‚नुभूतानाम‚त्य‚न्त‚म‚भावाद‚र्थ‚क्रियाया नास्ति स‚त्त्व‚म् । केचित्तु अर्थ‚क्रियाकारित‚या‚{\tiny $_{lb}$}‚ अभिम‚ताः स‚त्य‚स्व‚प्ने विष‚या दृष्ट‚काल‚भेद‚व्य‚भिचारिणो न स‚न्त्येवेति प्र‚क‚र‚णार्थः ।
	\pend% ending standard par
      ‚{\tiny $_{lb}$}‚

	  \pstart \leavevmode% starting standard par
	इद‚मीदृग्विशिष्टं स्व‚प्न‚ज्ञान‚मुदाह‚र‚तो \textbf{ध‚र्मोत्त‚र}स्याय‚माश‚यः--एवंविध‚स्य स्व‚प्न‚ज्ञान‚स्य‚{\tiny $_{lb}$}‚ स‚द‚र्थ‚त्वाभिमानः केषाञ्चित्स‚त्य‚स्व‚प्न‚वादिनाम‚स्तीति त‚द‚भिमान‚श‚म‚नायेदं म‚योदाह‚र‚णीकृत‚म् ।‚{\tiny $_{lb}$}‚ न त्व‚स्माद‚न्य‚स्य स्व‚प्न‚ज्ञान‚स्य क‚श्चिद् विशेषः । स‚र्व‚स्यैव स्व‚प्न‚ज्ञान‚स्य निराल‚म्ब‚न‚त‚या‚{\tiny $_{lb}$}‚ मिथ्याज्ञान‚त्वादिति ।
	\pend% ending standard par
      ‚{\tiny $_{lb}$}‚

	  \pstart \leavevmode% starting standard par
	\hphantom{.}इदानीं प‚रिच्छिद्य‚मान‚स्य य‚त्कालं प‚रिच्छेद‚नं त‚त्काल‚मेव प्राप्य‚माण‚स्य प्राप‚ण‚म‚भि‚{\tiny $_{lb}$}‚म‚त‚म्--एवं ब्रुव‚त इति म‚त्वाऽस्यार्थ‚स्यानुप‚प‚त्तिं चोद‚य‚ति--\textbf{न‚नु च}--इति । देशे निय‚त‚{\tiny $_{lb}$}‚माकारे निय‚तं च श‚क्यं प्राप‚यितुमित्य‚भिद‚धानोऽन‚योर‚विप्र‚तिप‚त्तिमाह । इदं तु न स‚म्भाव्य‚त‚{\tiny $_{lb}$}‚ इत्याह--\textbf{य‚त्कालं तु}--इति । \textbf{तु}नाऽन‚न्त‚रोक्ताभ्यां विधाभ्यां वैध‚र्म्य‚माह । यः कालोस्य‚{\tiny $_{lb}$}‚ प‚रिच्छेद‚न‚स्य त‚द् य‚था भ‚व‚ति त‚था प‚रिच्छिन्न‚म् । स कालो य‚स्य प्राप‚ण‚स्य त‚द् य‚था भ‚व‚ति‚{\tiny $_{lb}$}‚ त‚था न श‚क्यं प्राप‚यितुम् । य‚स्मिन् काले प‚रिच्छिन्नं त‚स्मिन् काले न श‚क्य‚ते प्राप‚यितुमित्य‚र्थः ।‚{\tiny $_{lb}$}‚ एवं च ब्रुव‚ताऽनेन \textbf{कालान्त‚र‚ग्राहीति} य‚दुक्तं त‚द‚युक्त‚मिति द‚र्शित‚म् । स‚र्व‚स्यैव ज्ञान‚स्य‚{\tiny $_{lb}$}‚ एवंशील‚त्वादिति ।
	\pend% ending standard par
      ‚{\tiny $_{lb}$}‚

	  \pstart \leavevmode% starting standard par
	\textbf{नोच्य‚त} इत्यादिना सिद्धान्त‚वादी चोद‚क‚स्यानुक्तोपाल‚म्भ‚माह । अनेनापि न किञ्चि‚{\tiny $_{lb}$}‚‚{\tiny $_{lb}$}‚ \leavevmode\ledsidenote{\textenglish{27/dm}}‚{\tiny $_{lb}$}‚ 
	  
	स‚म्य‚ग्ज्ञानं पूर्वं कार‚णं य‚स्याः सा त‚थोक्ता । कार्यात् \edtext{}{\lemma{कार्यात्}\Bfootnote{क‚थं पूर्व‚श‚ब्दः कार‚णे व‚र्त्त‚ते इत्याह--\cite{dp-msD-n}}}पूर्वं भ‚व‚त् कार‚णं पूर्व‚मुक्त‚म् ।‚{\tiny $_{lb}$}‚ कार‚ण‚श‚ब्दोपादाने\edtext{}{\lemma{ब्दोपादाने}\Bfootnote{श‚ब्दापादाने \cite{dp-msA} \cite{dp-msB}}} तु पुरुषार्थ‚सिद्धेः\edtext{}{\lemma{सिद्धेः}\Bfootnote{अव्य‚व‚हित‚म्--\cite{dp-msD-n}}} साक्षात्कार‚णं \edtext{}{\lemma{णं}\Bfootnote{म‚न्य‚ते \cite{dp-msA} \cite{dp-edH}}}ग‚म्येत । पूर्व‚श‚ब्दे तु पूर्व‚मात्र‚म् ।‚{\tiny $_{lb}$}‚ त्कार‚ण‚मुक्त‚म् । त‚त् किं त्व‚यैवं नोच्य‚त इति पार्श्व‚स्थं प्र‚त्य‚स्यैवाभिप्रायं प्र‚काश‚य‚ति--अन्यो‚{\tiny $_{lb}$}‚ \textbf{हि}--इति । \textbf{ही}ति य‚स्मात् । य‚द्येवं नोच्य‚ते त‚र्हि किं नामोच्य‚त इत्याह--\textbf{किन्तु}--इति ।‚{\tiny $_{lb}$}‚ निपातानिपात‚स‚मुदायोऽयं केव‚ल‚मित्य‚स्यार्थे ।
	\pend% ending standard par
      ‚{\tiny $_{lb}$}‚

	  \pstart \leavevmode% starting standard par
	न‚नु अस‚ङ्ग‚त‚मिदं वाक्य‚म् । न हि य‚देव प‚रिच्छिन्न‚मित्य‚स्ति येनैव‚मुच्येत । य‚त्कालं‚{\tiny $_{lb}$}‚ तु प‚रिच्छिन्नं त‚त्काल‚मिति तु युक्तं व‚क्तुम्, न त‚देवेति । स‚त्य‚मेत‚त् । केव‚लं बोधे य‚त्नः‚{\tiny $_{lb}$}‚ क‚र‚णीयः । \textbf{य‚त्काल‚मि}त्य‚नेन हि त‚त्काल‚मिति प्राप्त‚म्, त‚देवेति त‚च्छ‚ब्देन व‚स्तुविष‚यो य‚च्छ‚ब्द‚{\tiny $_{lb}$}‚ आकृष्य‚ते । त‚तोऽय‚म‚र्थः--य‚त्कालं प‚रिच्छिन्नं य‚द् व‚स्तु त‚त्कालं त‚देव व‚स्तु प्राप‚णीय‚मिति ।‚{\tiny $_{lb}$}‚ अहो ग‚डुप्र‚वेशेऽक्षितारानिर्ग‚मो जातः । एवं ख‚लु वाक्यं स्यात् स‚म‚र्थित‚म्, पूर्व‚प‚क्षात्पुन‚र‚स्या‚{\tiny $_{lb}$}‚विशेषः प्राप्तः । अस्ति विशेषो म‚हान्, केव‚लं भ‚व‚ता न स‚मीचीनं निरूपितः । त‚थाहि‚{\tiny $_{lb}$}‚ पूर्व‚प‚क्षाव‚स्थायां य‚त्कालं त‚त्कालं इति चात्र ब‚हुव्रीहिणा प‚रिच्छेद‚न‚ल‚क्ष‚णा प्राप‚ण‚ल‚क्ष‚णा च‚{\tiny $_{lb}$}‚ क्रियाऽभिधीय‚ते । इदानीं पुन‚र्व‚स्तुतो नाय‚म‚र्थः । य‚द् व‚स्तु येन कालेन स‚म्ब‚द्धं प‚रिच्छिन्नं‚{\tiny $_{lb}$}‚ त‚देव तेन कालेन स‚म्ब‚द्धं स्व‚रूपेण प्राप‚णीयं त‚दाऽन्य‚दा वा । प‚रिच्छेद‚स्य यादृशः काल‚स्त‚{\tiny $_{lb}$}‚स्मिन् काले य‚द् विद्य‚मानं त‚देव प्राप‚णीय‚मिति याव‚त् । त‚त‚श्च प‚रिच्छेद‚का\leavevmode\ledsidenote{\textenglish{12b/ms}}लाऽस‚तो‚{\tiny $_{lb}$}‚ य‚द् ग्राह‚कं त‚न्न प्र‚माण‚मित्य‚व‚तिष्ठ‚ते । \textbf{नोच्य‚ते य‚स्मिन् काल} इत्यादिना च य‚देत‚दुक्तं त‚द्‚{\tiny $_{lb}$}‚ बाह्य‚प्राप‚णाभिप्रायेण द्र‚ष्ट‚व्य‚म् । प‚र‚मार्थ‚त‚स्तु ज्ञान‚स्य प्र‚द‚र्श‚नाद‚न्यः प्राप‚ण‚व्यापारो नास्तीति‚{\tiny $_{lb}$}‚ य‚स्मिन्नेव काले प‚रिच्छिद्य‚तेऽर्थ‚स्त‚स्मिन्नेव काले प्राप्य‚त इति । एत‚च्चान‚न्त‚र‚मेवानेनैव विस्त‚रेण‚{\tiny $_{lb}$}‚ प्र‚तिपादित‚मिति स्व‚व‚च‚न‚व्याघातोऽन्य‚थाऽस्य स्यादिति ।
	\pend% ending standard par
      ‚{\tiny $_{lb}$}‚

	  \pstart \leavevmode% starting standard par
	न‚न्वेव‚म‚पि प‚रिच्छेद‚काल‚व‚र्त्तिनः प्राप‚णं न स‚म्भ‚व‚त्येव । स‚र्व‚स्यैव विष‚य‚स्य‚{\tiny $_{lb}$}‚ क्ष‚णिक‚त्वात् । त‚था चोप‚द‚र्शितार्थ‚प्राप‚क‚त्वं नाम क‚स्य‚चिद‚पि ज्ञान‚स्य नास्तीत्य‚स‚म्भ‚वितैव स‚म्य‚ग्‚{\tiny $_{lb}$}‚ज्ञान‚त्व‚ल‚क्ष‚ण‚स्य स्यादित्याश‚ङ्क्याह--\textbf{अभेदेति । अभेदेनै}क‚रूप‚त्वेन त‚देवेद‚मित्याकारे‚{\tiny $_{lb}$}‚णा\textbf{ध्य‚व‚सायात्} । उपादानोपादेय‚कृत‚क्ष‚ण‚प्र‚ब‚न्धः स‚न्तान‚स्त‚द्ग‚त‚स्त‚दाश्रितः ।
	\pend% ending standard par
      ‚{\tiny $_{lb}$}‚

	  \pstart \leavevmode% starting standard par
	अय‚म‚स्य भावः--सांव्य‚व‚हारिक‚स्येह प्र‚माण‚स्य ल‚क्ष‚ण‚मुच्य‚ते । तेन नैकान्तेन व‚स्तु‚{\tiny $_{lb}$}‚स्थितिर‚पेक्ष्य‚ते । त‚त्र य‚द्य‚पि व‚स्तुस्थित्या प‚रिच्छिन्न‚प्राप्य‚योर्नानात्वं त‚थापि व्य‚व‚ह‚र्त्तारो‚{\tiny $_{lb}$}‚ निर‚न्त‚राप‚राप‚रोत्प‚त्तेर‚विद्याव‚शाच्च हेतुफ‚ल‚रूपं क्ष‚ण‚प्र‚च‚यं त‚देवेद‚मित्येक‚त्वेनाधिमुञ्च‚न्ति त‚तः‚{\tiny $_{lb}$}‚ प‚रिच्छेद‚काल‚भाविनः प्राप‚णं स‚म्भ‚व‚त्येव । \textbf{इतिः} स‚म्य‚ग्ज्ञान‚प‚द‚व्याख्यान‚प‚रिस‚माप्तौ ।
	\pend% ending standard par
      ‚{\tiny $_{lb}$}‚

	  \pstart \leavevmode% starting standard par
	त‚द‚नेन प्र‚ब‚न्धेन स‚म्य‚ग्ज्ञान‚प‚दं व्याख्यायाधुना \textbf{पूर्व}श‚ब्दं व्याचिख्यासुस्तेन सार्ध‚म‚स्य‚{\tiny $_{lb}$}‚ विग्र‚ह‚माह--\textbf{स‚म्य‚ग्ज्ञान‚म्} इति ।
	\pend% ending standard par
      ‚{\tiny $_{lb}$}‚‚{\tiny $_{lb}$}‚\textsuperscript{\textenglish{28/dm}}‚{\tiny $_{lb}$}‚
	  \bigskip
	  \begingroup
	

	  \pstart \leavevmode% starting standard par
	\edtext{\textsuperscript{*}}{\lemma{*}\Bfootnote{किम‚न‚न्त‚र‚कार‚णं ज्ञानं भ‚व‚ति य‚द‚पेक्ष‚या व्य‚व‚हित‚स्य ग्र‚ह‚णार्थं पूर्व‚ग्र‚ह‚णं‚{\tiny $_{lb}$}‚ कृत‚मित्याह--\cite{dp-msD-n}}}द्विविधं च स‚म्य‚ग्ज्ञानं--अर्थ‚क्रियानिर्भास‚म्, अर्थ‚क्रियास‚म‚र्थे च प्र‚व‚र्त्त‚क‚म् । \edtext{\textsuperscript{*}}{\lemma{*}\Bfootnote{त‚योर्य‚त्--\cite{dp-msA} \cite{dp-edP} \cite{dp-edH} \cite{dp-edE} \cite{dp-edN}}}त‚योर्म‚ध्ये‚{\tiny $_{lb}$}‚ य‚त् प्र‚व‚र्त‚कं त‚दिह प‚रीक्ष्य‚ते । त‚च्च पूर्व‚मात्र‚म् । न तु साक्षात्कार‚ण‚म् । स‚म्य‚ग्ज्ञाने हि स‚ति‚{\tiny $_{lb}$}‚ पूर्व‚दृष्ट‚स्म‚र‚ण‚म् । स्म‚र‚णाद‚भिलाषः । अभिलाषात् प्र‚वृत्तिः । प्र‚वृत्तेश्च प्राप्तिः । त‚तो‚{\tiny $_{lb}$}‚ न साक्षाद्धेतुः ।
	\pend% ending standard par
      
	  \endgroup
	‚{\tiny $_{lb}$}‚

	  \pstart \leavevmode% starting standard par
	न‚नु पुरुषार्थ‚सिद्धिः स‚म्य‚ग्ज्ञान‚स्य प्र‚योज‚नं कार्य‚त्वेन । त‚था च स‚म्य‚ग्ज्ञान‚कार‚णिकेति‚{\tiny $_{lb}$}‚ व‚क्तुमुचितं त‚त्किमेव‚मुक्त‚मित्याह--\textbf{कार्याद्} इति । अनेन व्य‚व‚स्थावाची स‚न्नेव पूर्व‚श‚ब्दः‚{\tiny $_{lb}$}‚ कार‚णे व‚र्त्त‚त इति द‚र्श‚य‚ति ।
	\pend% ending standard par
      ‚{\tiny $_{lb}$}‚

	  \pstart \leavevmode% starting standard par
	\textbf{भ‚व‚दिति} ल‚क्ष‚ण‚हेत्वोः क्रियायाः \href{http://sarit.indology.info/?cref=Pā.3.2.126}{पाणिनि. ३. २. १२६} इति हेतौ श‚तुर्विधानात्‚{\tiny $_{lb}$}‚ हेतुप‚द‚मेत‚त् । य‚द्येवं कार‚ण‚श‚ब्द एव क्रिय‚तामित्याह--\textbf{कार‚णे}ति । अयं भावः--क‚रोतीति‚{\tiny $_{lb}$}‚ कार‚ण‚मुच्य‚ते, न त्व‚कुर्व‚द्रूप‚म् । अकुर्व‚ति पुनः कार‚ण‚व्य‚प‚देशः कार‚ण‚कार‚ण‚त्वादीप‚चारिकः । न‚{\tiny $_{lb}$}‚ च मुख्ये स‚म्भ‚व‚ति अमुख्ये प्र‚त्य‚यो युज्य‚ते । त‚तो य‚देव ज्ञान‚म‚व्य‚व‚धानेन पुरुषार्थ‚सिद्धेर्निब‚न्ध‚नं‚{\tiny $_{lb}$}‚ त‚देव कार‚ण‚श‚ब्देन प्र‚तीयेतेति । य‚दि पूर्व‚ग्र‚ह‚णेऽप्येवं प्र‚त्य‚य‚स्त‚दा को विशेष इत्याह--\textbf{पूर्व‚श‚ब्द}‚{\tiny $_{lb}$}‚ इति । \textbf{तुः} कार‚ण‚श‚ब्दाद् विशेष‚म‚स्याह । \textbf{पूर्व‚मात्र‚मिति} साक्षात्कार‚ण‚मित‚र‚च्च ।
	\pend% ending standard par
      ‚{\tiny $_{lb}$}‚

	  \pstart \leavevmode% starting standard par
	न‚नूभ‚योः स‚म्भ‚वे साक्षात्कार‚ण‚प‚रिहारेणेत‚र‚ग्र‚ह‚णाय पूर्व‚श‚ब्दः शोभ‚ते । न चैव‚म‚स्ति,‚{\tiny $_{lb}$}‚ स‚र्व‚स्यैव पुरुषार्थ‚सिद्धिं प्र‚ति साक्षात्कार‚ण‚त्वादित्याह--\textbf{द्विविधं च}--इति । \textbf{चो} य‚स्माद‚र्थे ।‚{\tiny $_{lb}$}‚ क‚थं द्वैविध्य‚मित्याह--\textbf{अर्थ‚क्रिये}ति । निर्भास‚तेऽस्मिन्निति \textbf{निर्भासः} । अर्थ‚क्रियाया निर्भासो‚{\tiny $_{lb}$}‚ य‚स्मात् त‚त् त‚था । य‚तः साध‚न‚ज्ञानाद‚न‚न्त‚रं दाहादिप्र‚तिभास‚ज्ञान‚मुत्प‚द्य‚ते त‚देव‚मुच्य‚ते ।
	\pend% ending standard par
      ‚{\tiny $_{lb}$}‚

	  \pstart \leavevmode% starting standard par
	केचित्तु क्ष‚णेन कार्य‚कार‚ण‚व्य‚व‚हार‚स्यार्वाग्द‚र्श‚नेन क‚र्त्तुम‚श‚क्य‚त्वाद् दाहादिप्र‚तिभास‚मेव‚{\tiny $_{lb}$}‚ ज्ञान‚मेवं ब्रुव‚ते । अर्थ‚क्रियाया \leavevmode\ledsidenote{\textenglish{13a/ms}} निर्भासोऽस्मिन्निति कृत्वा । अप‚र‚म् \textbf{अर्थ‚क्रियास‚म‚र्थे च‚{\tiny $_{lb}$}‚ प्र‚व‚र्त्त‚क‚म्} । य‚त् प्र‚वृत्तिस‚म‚धिग‚म्यार्थ‚क्रियागोच‚रं त‚देव‚मुक्त‚म् । \textbf{चो}ऽर्थ‚क्रियानिर्भासापेक्ष‚या‚{\tiny $_{lb}$}‚ स‚मुच्च‚ये । अन‚योः किं प‚रीक्ष्य‚त इत्याह \textbf{त‚योः}--इति । प्र‚व‚र्त्त‚क‚म‚पि साक्षात्कार‚णं त‚त् कोऽन्य‚त्र‚{\tiny $_{lb}$}‚ विद्वेषो येनेद‚मेव विचार्य‚ते ? किं वा पूर्व‚श‚ब्देनेत्याह--\textbf{त‚च्च}--इति । \textbf{चो} य‚स्मात् ।
	\pend% ending standard par
      ‚{\tiny $_{lb}$}‚

	  \pstart \leavevmode% starting standard par
	अय‚माश‚यः--य‚तः प्र‚व‚र्त्त‚न्ते त‚त् स‚र्व‚म‚र्थ‚प्राप्तेर्व्य‚व‚हितं कार‚ण‚म् । अत‚श्च स‚म्भ‚व‚{\tiny $_{lb}$}‚त्प्र‚तिब‚न्धं त‚त् प्र‚वृत्त्य‚ङ्ग‚त्वात्पूर्व‚मात्र‚म्, न त्व‚न‚न्त‚रं कार‚ण‚मिति । पूर्व‚मात्राभिधानेन सान्त‚र‚म् ।‚{\tiny $_{lb}$}‚ अमुमेव द्र‚ढ‚य‚न्नाह--\textbf{न तु}--इति । \textbf{तु}र्विशेषार्थः । कुत एत‚दित्याह--\textbf{स‚म्य‚ग्ज्ञान} इति । \textbf{हि}‚{\tiny $_{lb}$}‚र्य‚स्मात् । त‚स्मिन् स‚म्य‚ग्ज्ञाने प्र‚क‚र‚णात् साध‚न‚निर्भासे \textbf{स‚ति} । स्मृतिबीजोप‚ज‚न‚न‚योग्य‚ज्ञान‚{\tiny $_{lb}$}‚ज्ञात‚स्य दाहादेः \textbf{स्म‚र‚ण‚म्} । त‚तोऽ\textbf{भिलाषः}--त‚त्प्राप्तीच्छा । त‚तः \textbf{प्र‚वृत्तिः} प्र‚व‚र्त्त‚क‚ज्ञानोप‚द‚र्शित‚म‚र्थं‚{\tiny $_{lb}$}‚ प्राप्तुकामा व्यापार‚स‚हाया बुद्धिः । त‚स्याश्च प्राप्तिरुपादित्सित‚लाभः । य‚स्मात् स्म‚र‚णादिना‚{\tiny $_{lb}$}‚ व्य‚व‚धानं \textbf{त‚तः} त‚स्मात् \textbf{न साक्षात् हेतुः प्राप्ते}रुपायः पुरुषार्थ‚सिद्धेरिति प्र‚क‚र‚णात् ।
	\pend% ending standard par
      ‚{\tiny $_{lb}$}‚‚{\tiny $_{lb}$}‚\textsuperscript{\textenglish{29/dm}}‚{\tiny $_{lb}$}‚
	  \bigskip
	  \begingroup
	

	  \pstart \leavevmode% starting standard par
	अर्थ‚क्रियानिर्भासं\edtext{}{\lemma{क्रियानिर्भासं}\Bfootnote{०निर्भासे तु--\cite{dp-msA} \cite{dp-msC} \cite{dp-edP} \cite{dp-edH} \cite{dp-edN} निर्भासात्तु \cite{dp-msB} \cite{dp-edE} \cite{dp-msD}}} तु य‚द्य‚पि साक्षात्\edtext{}{\lemma{साक्षात्}\Bfootnote{प्र‚वृत्तिस्त‚थापि--\cite{dp-msA} \cite{dp-edP} \cite{dp-edH} ०प्राप्तिहेतुः त‚थापि \cite{dp-msC}}} प्राप्तिः, त‚थापि त‚न्न प‚रीक्ष‚णीय‚म् । य‚त्रैव हि‚{\tiny $_{lb}$}‚ प्रेक्षाव‚न्तोऽर्थिनः \edtext{}{\lemma{न्तोऽर्थिनः}\Bfootnote{स‚स‚न्देहाः--\cite{dp-msD-n}}}साश‚ङ्काः, त‚त् प‚रीक्ष्य‚ते । अर्थ‚क्रियानिर्भासे च ज्ञाते\edtext{}{\lemma{ज्ञाते}\Bfootnote{ज्ञाने \cite{dp-msB} \cite{dp-edP} \cite{dp-edH}}} स‚ति सिद्धः‚{\tiny $_{lb}$}‚ पुरुषार्थः । तेन त‚त्र न साश‚ङ्का\edtext{}{\lemma{ङ्का}\Bfootnote{साश‚ङ्का अर्थे ज्ञाते \cite{dp-msA} \cite{dp-edP} \cite{dp-edH}}} अर्थिनः । अत‚स्त‚न्न प‚रीक्ष‚णीय‚म् । त‚स्मात् प‚रीक्षार्ह‚{\tiny $_{lb}$}‚म‚साक्षात् कार‚णं स‚म्य‚ग्ज्ञान‚माद‚र्श‚यितुं कार‚ण‚श‚ब्दं प‚रित्य‚ज्य पूर्व‚ग्र‚ह‚णं कृत‚म् ।
	\pend% ending standard par
      
	  \endgroup
	‚{\tiny $_{lb}$}‚

	  \pstart \leavevmode% starting standard par
	य‚दि प‚र‚म्प‚र‚याऽपि प्राप्तिहेतुः प‚रीक्ष्य‚ते त‚र्हि साक्षात् हेतुर‚तित‚रां प‚रीक्ष‚णीय इत्याह—‚{\tiny $_{lb}$}‚\textbf{अर्थ‚क्रियानिर्भास‚मिति । तुः} प्र‚व‚र्त्त‚काध्य‚क्षाद‚र्थ‚क्रियानिर्भासं भेद‚व‚द् द‚र्श‚य‚ति ।
	\pend% ending standard par
      ‚{\tiny $_{lb}$}‚

	  \pstart \leavevmode% starting standard par
	य‚दा \textbf{प्राप्तिहेतु}रिति पाठ‚स्त‚दा अर्थ‚क्रियानिर्भास‚मित्य‚स्यान्त्य‚व्याख्यान‚प‚क्षे भिन्ना‚{\tiny $_{lb}$}‚ स‚न्तोषादिप्राप्तिर्लोकाध्य‚व‚साय‚सिद्धा या त‚स्या हेतुः । \textbf{प्राप्ति}पाठे तु त‚द्व्याख्यानान‚व‚द्य‚ता ।‚{\tiny $_{lb}$}‚ पूर्व‚व्याख्याने तूप‚चारात् प्राप्तिहेतुः प्राप्तिश‚ब्देन व‚क्त‚व्यः । क‚र‚ण‚साध‚नो वा प्राप्तिश‚ब्दो‚{\tiny $_{lb}$}‚ द्र‚ष्ट‚व्यः ।
	\pend% ending standard par
      ‚{\tiny $_{lb}$}‚

	  \pstart \leavevmode% starting standard par
	क‚स्मात् त‚त् न प‚रीक्ष्य‚त इत्याह--य‚त्रैव हि--इति । हिर्य‚स्मात् । साश‚ङ्काः स‚स‚न्देहाः ।‚{\tiny $_{lb}$}‚ आश‚ङ्काग्र‚ह‚ण‚स्योप‚ल‚क्ष‚ण‚त्वात् स‚विप‚र्यासा इत्य‚पि द्र‚ष्ट‚व्य‚म् ।
	\pend% ending standard par
      ‚{\tiny $_{lb}$}‚

	  \pstart \leavevmode% starting standard par
	एवं ब्रुव‚त‚श्चाय‚माश‚यः--न व्य‚स‚न‚मेत‚च्छास्त्र‚कृतो येन ज्ञान‚म‚प‚रीक्ष‚माणः स्वास्थ्य‚म‚ल‚भ‚मानः‚{\tiny $_{lb}$}‚ प‚रीक्ष‚ते । किन्त‚र्हि ? व्युत्पाद्य‚ज‚न‚प्र‚योज‚नोद्देशेनाय‚म‚स्यार‚म्भः । संश‚य‚विप‚र्यासाप‚सार‚णं च‚{\tiny $_{lb}$}‚ व्युत्पाद्य‚ज‚न‚प्र‚योज‚न‚म् । त‚तो य‚त्रैव ते त‚थाप्र‚वृत्त‚य‚स्त‚देव प‚रीक्ष्य‚त इति । अर्थ‚क्रिया‚{\tiny $_{lb}$}‚निर्भासेऽपि ते त‚थावृत्त‚य इत्याह--\textbf{अर्थ‚क्रियेति । च}श‚ब्द‚स्तुश‚ब्द‚स्यार्थे ।
	\pend% ending standard par
      ‚{\tiny $_{lb}$}‚

	  \pstart \leavevmode% starting standard par
	कुतो न त‚थावृत्त‚यः ? येन साक्षात्कार‚णेऽस्मिन् स‚ति सिद्धः पुरुषार्थः, तेनान‚न्त‚र‚मेव‚{\tiny $_{lb}$}‚ फ‚ल‚स्यानुभूय‚मान‚त्वात्, त‚दैव स‚न्तोषादिग‚म‚नाद् वा । व्याख्यान‚द्व‚येपि अर्थ‚क्रियानिर्भास‚{\tiny $_{lb}$}‚श‚ब्द‚वाच्य द्व‚ये स्थिते द्व‚य‚मेत‚त् श‚ङ्कायाः कार‚ण‚म्--अन‚न्त‚र‚फ‚लाद‚र्श‚न‚म्, अध्य‚व‚साय‚सिद्ध‚{\tiny $_{lb}$}‚भिन्न‚प्राप्त्य‚भावो वा । त‚द‚भावे तु क‚थं श‚ङ्केर‚न्निति भावः ।
	\pend% ending standard par
      ‚{\tiny $_{lb}$}‚

	  \pstart \leavevmode% starting standard par
	मा भूवंस्त‚त्र साश‚ङ्कास्त‚त् किं सिद्ध‚मित्याह--\textbf{अत} इति ।
	\pend% ending standard par
      ‚{\tiny $_{lb}$}‚

	  \pstart \leavevmode% starting standard par
	य‚दि व्य‚व‚हितं प्र‚व‚र्त्त‚कं पुरुषार्थ‚सिद्धेर्न कार‚णं त‚र्हि क‚थ‚मुक्त‚म्--\textbf{कार्यात् पूर्वं भ‚व‚त् कार‚णं‚{\tiny $_{lb}$}‚ पूर्व‚मुक्त‚म्} इति ? असाक्षात् कार‚णे चाकार‚णे कार‚ण\leavevmode\ledsidenote{\textenglish{13b/ms}}विशेष‚णं साक्षादिति निर‚र्थ‚क‚{\tiny $_{lb}$}‚मित्याश‚ङ्क्याह--\textbf{त‚स्माद्} इति । य‚स्मात् साक्षात् कार‚ण‚माश‚ङ्कानास्प‚दं स‚द‚प‚रीक्ष‚णीयं \textbf{त‚स्माद्}‚{\tiny $_{lb}$}‚ य‚द\textbf{साक्षात्कार‚ण‚म‚त} एव च \textbf{प‚रीक्षार्हं} प‚रीक्षायोग्यं प्र‚व‚र्त्त‚कं \textbf{स‚म्य‚ग्ज्ञान‚माद‚र्श‚यितु}माद‚र्श‚{\tiny $_{lb}$}‚यिष्यामि--इति \textbf{कार‚ण‚श‚ब्दं} विहाय \textbf{पूर्व‚ग्र‚ह‚णं कृत‚मा}चार्येणेत्य‚र्थात् ।
	\pend% ending standard par
      ‚{\tiny $_{lb}$}‚

	  \pstart \leavevmode% starting standard par
	एत‚दुक्तं भ‚व‚ति--योऽयं कार‚ण‚श‚ब्दो व्य‚व‚हिते कार‚ण‚कार‚णे व‚र्त्त‚ते नायं त‚त्राभिधाय‚क‚त्वेन‚{\tiny $_{lb}$}‚ व‚र्त्त‚ते क‚रोतीति कार‚ण‚मिति व्युत्प‚त्तेः । किन्त‚र्हि ? ताद‚र्थ्यादुप‚चार‚त इति ।
	\pend% ending standard par
      ‚{\tiny $_{lb}$}‚‚{\tiny $_{lb}$}‚\textsuperscript{\textenglish{30/dm}}‚{\tiny $_{lb}$}‚
	  \bigskip
	  \begingroup
	

	  \pstart \leavevmode% starting standard par
	पुरुष‚स्यार्थः\edtext{}{\lemma{स्यार्थः}\Bfootnote{नास्तीदं प‚दं--\cite{dp-msA} \cite{dp-msB} \cite{dp-msD} \cite{dp-edP} \cite{dp-edH} \cite{dp-edE} \cite{dp-edN}}} पुरुषार्थः । अर्थ्य‚त इत्य‚र्थः काम्य‚त इति याव‚त् । ह‚योऽर्थः, उपादेयो‚{\tiny $_{lb}$}‚ वा । हेयो ह्य‚र्थो हातुमिष्य‚ते, उपादेयोऽपि उपादातुम् । न च हेयोपादेयाभ्याम‚न्यो राशिर‚स्ति ।‚{\tiny $_{lb}$}‚ उपेक्ष‚णीयो\edtext{}{\lemma{णीयो}\Bfootnote{०णीयोऽनु० \cite{dp-msA} ०णीयोप्य‚नु \cite{dp-msB} \cite{dp-edH} \cite{dp-edN} ०णीयोपि ह्य‚नु० \cite{dp-msC} \cite{dp-msD}}} ह्य‚नुपादेय‚त्वात् हेय एव । त‚स्य सिद्धिः--हान‚म्, उपादानं च । हेतुनिब‚न्ध‚ना\edtext{}{\lemma{ना}\Bfootnote{अर्थ‚हेतुनिब‚न्ध‚ना--\cite{dp-msD-n} ।}}‚{\tiny $_{lb}$}‚ हि सिद्धिरुत्प‚त्तिरुच्य‚ते । ज्ञान‚निब‚न्ध‚ना तु सिद्धिर‚नुष्ठान‚म् । हेय‚स्य च\edtext{}{\lemma{च}\Bfootnote{हेय‚स्य हा०--\cite{dp-msA} \cite{dp-msB} \cite{dp-edP} \cite{dp-edH} \cite{dp-edE} \cite{dp-edN}}} हान‚म‚नुष्ठान‚म्‚{\tiny $_{lb}$}‚ उपादेय‚स्य चोपादान‚म् । त‚तो हेयोपादेय‚योर्हानोपादान‚ल‚क्ष‚णानुष्ठितिः \edtext{}{\lemma{णानुष्ठितिः}\Bfootnote{सिद्धिरुच्य‚ते \cite{dp-msB}}}सिद्धिरित्युच्य‚ते ।
	\pend% ending standard par
      
	  \endgroup
	‚{\tiny $_{lb}$}‚

	  \pstart \leavevmode% starting standard par
	स्यादेत‚त्--स‚त्य‚पि कार‚ण‚ग्र‚ह‚णे विचारार्ह‚मेव स‚म्य‚ग्ज्ञानं प्र‚तिप‚त्स्य‚ते । य‚त‚स्त‚द्व्युत्पा‚{\tiny $_{lb}$}‚द्य‚त इत्य‚र्थः, साध्य‚त्वात् । प्रातिप‚दिकार्थ‚स्तु कार‚क‚त्वाद् गुणः प्र‚धानानुयायी । तेन प्र‚धाना‚{\tiny $_{lb}$}‚नुरोधात् त‚च्छ‚ब्दो व‚र्णित‚या नीत्या मुख्य‚स्य कार‚ण‚स्य स‚म्य‚ग्ज्ञान‚स्य व्युत्प‚त्तिक‚र्म‚ताऽनुप‚प‚त्तेस्त‚त्प‚रि‚{\tiny $_{lb}$}‚त्यागेन ल‚क्ष‚ण‚या ताद‚र्थ्य‚भूत‚या प्र‚त्यास‚त्त्याऽर्थ‚क्रियास‚म‚र्थ‚प्र‚व‚र्त्त‚क एव प्र‚व‚र्त्तिष्य‚ते । य‚था म‚ञ्चाः‚{\tiny $_{lb}$}‚ क्रोश‚न्तीत्य‚त्र प्र‚धानानुरोधात् म‚ञ्च‚श‚ब्दः क्रोश‚न‚क्रियाक‚र्त्तृत्वानुप‚प‚त्तेर्मुख्य‚म‚र्थं त्य‚क्त्वा ल‚क्ष‚ण‚या‚{\tiny $_{lb}$}‚ तात्स्थ्य‚भूत‚या प्र‚त्यास‚त्त्या पुरुषेषु व‚र्त्त‚त इति । स‚त्य‚मेत‚त् । केव‚ल‚मेवं स‚ति व्याख्यातृणामिदं‚{\tiny $_{lb}$}‚ कौश‚लं स्यान्न शास्त्र‚कृत इति स‚र्व‚म‚न‚व‚द्य‚म् ।
	\pend% ending standard par
      ‚{\tiny $_{lb}$}‚

	  \pstart \leavevmode% starting standard par
	इदानीं \textbf{पुरुषार्थ‚सिद्धि}प‚दं विव‚रिषुः \textbf{पुरुष}श‚ब्देन सार्ध\textbf{म‚र्थ}श‚ब्द‚स्य विग्र‚ह‚म्, \textbf{अर्थ‚स्य च}‚{\tiny $_{lb}$}‚ स्व‚रूपं \textbf{पुरुष‚स्ये}त्यादिनाच‚ष्टे । \textbf{अर्थ्य‚त} इत्याच‚क्षाणो \edtext{}{\lemma{क्षाणो}\Bfootnote{\textbf{अर्थ उप‚याच्ञायाम्}--धातुपाठ १०. ३७३ ।}}अर्थ याच्ञायाम् इत्य‚तो णिज‚न्तात्‚{\tiny $_{lb}$}‚ क‚र्म‚ण्य‚चं द‚र्श‚य‚ति । \textbf{अर्थ्य‚त} इत्य‚स्यार्थं स्प‚ष्ट‚य‚ति--\textbf{काम्य}त इष्य‚त इत्य‚नेन यावानेवार्थ‚{\tiny $_{lb}$}‚ उक्त‚स्तावाने\textbf{वार्थ्य‚त} इत्य‚नेनापीति \textbf{इति याव‚दि}त्य‚स्यार्थः ।
	\pend% ending standard par
      ‚{\tiny $_{lb}$}‚

	  \pstart \leavevmode% starting standard par
	कोऽसाव‚र्थ इत्याह--\textbf{हेय} इति । \textbf{वा}श‚ब्द‚श्च‚श‚ब्द‚स्यार्थे । अर्थ्य‚मान इष्य‚माणोऽर्थः ।‚{\tiny $_{lb}$}‚ न त‚र्हि हेयोऽर्थ इत्याह--\textbf{हेय} इति । \textbf{हि}र्य‚स्मात् । \textbf{हेयो हातुं} त्य‚क्तुमिष्य‚ते त‚स्माद‚र्थः ।‚{\tiny $_{lb}$}‚ अय‚माश‚यः--इष्य‚माणः ख‚ल्व‚र्थः । अत्र य‚द्य‚पि स्वीक‚र‚णेच्छा नास्ति त‚थापि प‚रिहारेच्छा ताव‚द‚{\tiny $_{lb}$}‚स्तीति । य‚दि हातुमिष्य‚माणोऽर्थः क‚थ‚मुपादेयोऽर्थ इत्याह--\textbf{उपादेयो}ऽपीष्य‚ते केव‚ल‚मुपादातुम् ।‚{\tiny $_{lb}$}‚ न केव‚लं हेय इष्य‚ते इत्य‚पिश‚ब्दः ।
	\pend% ending standard par
      ‚{\tiny $_{lb}$}‚

	  \pstart \leavevmode% starting standard par
	न‚नु च न हेयोपादेयाव‚र्थोऽपि तु अन्योपि । त‚तः सोऽपि क‚स्मान्न प्र‚द‚र्श्य‚त इत्याह—‚{\tiny $_{lb}$}‚\textbf{न चेति । चो}व‚धार‚णे य‚स्माद‚र्थे वा ।
	\pend% ending standard par
      ‚{\tiny $_{lb}$}‚

	  \pstart \leavevmode% starting standard par
	न‚नूपेक्ष‚णीयोऽपि राशिर‚स्ति । य‚त्र न प्र‚वृत्तिर्य‚त‚श्च न निवृत्तिः । त‚त् क‚थ‚म‚नुभ‚व‚{\tiny $_{lb}$}‚सिद्ध‚स्याप‚ह्न‚व इत्याह--\textbf{उपेक्ष‚णीय} इति । हिर्य‚स्मात् ।
	\pend% ending standard par
      ‚{\tiny $_{lb}$}‚

	  \pstart \leavevmode% starting standard par
	\hphantom{.}अत्र क‚श्चिदाह--य‚थाऽसाव‚नुपादेय‚स्त‚थाऽह‚योपि । त‚त्र य‚द्य‚नुपादेय‚त्वाद्धेय‚स्त‚दाऽहेय‚{\tiny $_{lb}$}‚त्वादुपादेयः किं न भ‚व‚ति इति । साधूक्तं तेन \textbf{भ‚द‚न्तेन} केव‚लं हेय‚श‚ब्दार्थ‚विचारे म‚नो न‚{\tiny $_{lb}$}‚ ‚{\tiny $_{lb}$}‚ \leavevmode\ledsidenote{\textenglish{31/dm}}‚{\tiny $_{lb}$}‚ 
	  
	स‚र्वा चासौ पुरुषार्थ‚सिद्धिश्चेति । स‚र्व‚श‚ब्द इह द्र‚व्य‚कार्त्स‚न्ये\edtext{}{\lemma{न्ये}\Bfootnote{प्र‚वृत्तः--\cite{dp-msD}}} वृत्तो न \edtext{}{\lemma{न}\Bfootnote{न च प्र० \cite{dp-msB} \cite{dp-edH} न प्र‚कार \cite{dp-msC} \cite{dp-edP} \cite{dp-edE}}}तु प्र‚कार‚{\tiny $_{lb}$}‚कार्त्स्न्ये । त‚तो नाय‚म‚र्थः--द्विप्र‚कारापि सिद्धिः स‚म्य‚ग्ज्ञान‚निब‚न्ध‚नैवेति ।\edtext{\textsuperscript{*}}{\lemma{*}\Bfootnote{निब‚न्ध‚नेति \cite{dp-msA} \cite{dp-msB} \cite{dp-msD} \cite{dp-edP} \cite{dp-edH} \cite{dp-edE}}} अपि त्व‚य‚म‚र्थः--या‚{\tiny $_{lb}$}‚ काचित् सिद्धिः सा स‚र्वा कृत्स्नैवासौ स‚म्य‚ग्ज्ञान‚निब‚न्ध‚नेति ।\edtext{\textsuperscript{*}}{\lemma{*}\Bfootnote{निब‚न्ध‚नैवेति \cite{dp-msA} \cite{dp-edP} \cite{dp-edH} \cite{dp-edE}}} मिथ्याज्ञानाद्धि काक‚तालीयाऽपि‚{\tiny $_{lb}$}‚ नास्त्य‚र्थ‚सिद्धिः । त‚था हि--य‚दि प्र‚द‚र्शित‚म‚र्थं प्राप‚य‚त्येवं\edtext{}{\lemma{त्येवं}\Bfootnote{एव‚मिति स‚म्य‚ग्ज्ञान‚व‚त्--\cite{dp-msD-n}}} त‚तो\edtext{}{\lemma{तो}\Bfootnote{त‚तो मिथ्याज्ञानात्--\cite{dp-msD-n}}} भ‚व‚त्य‚र्थ‚सिद्धिः ।‚{\tiny $_{lb}$}‚ प्र‚णिहित‚म् । त‚थाहि--हीय‚ते, त्य‚ज्य‚ते, न स्वीक्रिय‚ते इति हेयः । हान‚ञ्चास्वीक‚र‚ण‚म् ।‚{\tiny $_{lb}$}‚ न तु गृहीत्वा प‚रित्यागः । तेन अहिविष‚क‚ण्ट‚कादीनाम‚पि हानं त‚त्रा\leavevmode\ledsidenote{\textenglish{14a/ms}}ऽप्र‚वृत्तिरेव ।‚{\tiny $_{lb}$}‚ सा चोपेक्ष‚णीयेऽप्य‚स्ति । त‚था च य‚दि त‚स्य स्वीकारो भ‚वेत्, त‚दाऽहेय‚त्वं सिद्धिम‚ध्यासीत ।‚{\tiny $_{lb}$}‚ न च त‚त् स्वीक्रिय‚त इति व्य‚क्त‚म‚यं हेतुर‚सिद्ध‚स्त‚स्य । न त्व‚स्माक‚म‚सिद्धः । अनुपादेय‚त्व‚स्य‚{\tiny $_{lb}$}‚ अस्वीक‚र्त्त‚व्य‚त्व‚स्य सिद्ध‚त्वात् ।
	\pend% ending standard par
      ‚{\tiny $_{lb}$}‚

	  \pstart \leavevmode% starting standard par
	अथानुभ‚व‚सिद्ध‚स्य तृतीय‚राशेरुपेक्ष‚णीय‚स्यास्वीक‚र‚ण‚मात्राद्धेयेन सार्ध‚मैक्य‚प्र‚तिपाद‚न‚{\tiny $_{lb}$}‚म‚युक्त‚मिति चेत् । प्रिय‚मुक्तं प्रियेण । न हि व‚य‚म‚पि प्र‚सिद्ध‚योर्हेयोपेक्ष‚णीय‚योर‚र्थ‚योरेक‚स्व‚भाव‚ता‚{\tiny $_{lb}$}‚मातिष्ठाम‚हे । किन्तु क्रियानिमित्तेन हेय‚श‚ब्देनैकाभिधान‚म् त‚थाऽस्वीक‚र‚णार्थेन हान‚श‚ब्दे‚{\tiny $_{lb}$}‚नोपेक्षाया अभिधान‚मिति किम‚व‚द्य‚म् ? एत‚च्च हेयोपादेय‚त्व‚म‚र्थ‚स्यैक‚पुरुषैक‚कालापेक्ष‚या‚{\tiny $_{lb}$}‚ प्र‚त्येय‚म् ।
	\pend% ending standard par
      ‚{\tiny $_{lb}$}‚

	  \pstart \leavevmode% starting standard par
	अधुना \textbf{सिद्धि}श‚ब्द‚स्यार्थ‚प‚देन विग्र‚हं ब्रुव‚न्न‚र्थ‚माह--\textbf{त‚स्य}--इति ।
	\pend% ending standard par
      ‚{\tiny $_{lb}$}‚

	  \pstart \leavevmode% starting standard par
	न‚नु लोके सिद्धिर्निष्प‚त्तिरुच्य‚ते, य‚थौद‚न‚स्य सिद्धिरिति । त‚त्क‚थ‚मेवं व‚र्ण्य‚त इत्याह‚{\tiny $_{lb}$}‚ \textbf{हेतुनिब‚न्ध‚ने}ति । \textbf{हेतुः} कार‚णं निब‚द्ध्य‚तेऽस्मिन्निति \textbf{निब‚न्ध‚नं} प्र‚तिब‚न्ध‚विष‚यः । हेतुर्निब‚न्ध‚न‚{\tiny $_{lb}$}‚म‚स्या इति विगृह्य हेतुप्र‚तिब‚द्धेत्य‚र्थो व‚क्त‚व्यः ।
	\pend% ending standard par
      ‚{\tiny $_{lb}$}‚

	  \pstart \leavevmode% starting standard par
	य‚दि ज‚न‚क‚निब‚न्ध‚ना सिद्धिरीदृशी त‚र्हि ज्ञान‚निब‚न्ध‚ना कीदृशी भ‚विष्य‚तीत्याह--\textbf{ज्ञाने}ति ।‚{\tiny $_{lb}$}‚ \textbf{तुः} पूर्व‚स्याः सिद्धेर‚स्या भेदं द‚र्श‚य‚ति । एवं च व्याच‚क्षाण \textbf{अर्थ}प‚द‚व्याख्याने अर्थः प्र‚योज‚नं दाहादि,‚{\tiny $_{lb}$}‚ सिद्धिप‚द‚व्याख्याने च त‚स्य दाहादेर्निष्प‚त्तिः इति य‚द् \textbf{विनीत‚देव‚शान्त‚भ‚द्रौ} व्याच‚क्षातां त‚द्‚{\tiny $_{lb}$}‚ द्व‚य‚म‚प्य‚पास्य‚ति । य‚तः स्व‚हेतोरेव व‚ह्न्यादेर्दाहादेर्निष्प‚त्तिर्न तु ज्ञानात् त‚स्य त‚द‚कार‚क‚त्वादिति ।
	\pend% ending standard par
      ‚{\tiny $_{lb}$}‚

	  \pstart \leavevmode% starting standard par
	न‚नु य‚दि सिद्धिर‚नुष्ठानं त‚र्हि हान‚मुपादानं च क‚थं सिद्धिरित्याह--\textbf{हेय‚स्य च}--इति ।‚{\tiny $_{lb}$}‚ \textbf{चो} हेतौ, अव‚धार‚णे वा \textbf{हान}मित्य‚स्मात्प‚रो द्र‚ष्ट‚व्यः । द्वितीय‚श्च‚कारः स‚मुच्च‚यार्थः ।‚{\tiny $_{lb}$}‚ किमेवं स‚ति सिद्ध‚मित्याह--\textbf{त‚त} इति । य‚तो ज्ञान‚निब‚न्ध‚ना सिद्धिरीदृशी \textbf{त‚त}स्त‚स्मात्कार‚णात् ।
	\pend% ending standard par
      ‚{\tiny $_{lb}$}‚

	  \pstart \leavevmode% starting standard par
	\textbf{पुरुषार्थ‚सिद्धि}श‚ब्दानां विग्र‚ह‚म‚र्थं च व्याख्याय स‚म्प्र‚ति \textbf{स‚र्व}श‚ब्दं व्याख्यातुं विग्र‚ह‚माह—‚{\tiny $_{lb}$}‚\textbf{स‚र्वा च}--इति । एवं च विगृह्ण‚न् य‚द् \textbf{विनीत‚देवेन} व्याख्यातं स‚र्व‚श्चासौ लौकिको लोकोत्त‚र‚{\tiny $_{lb}$}‚श्चास‚न्न‚देशो दूर‚देश‚श्च पुरुषार्थ‚श्चेति, त‚था त‚स्य सिद्धिः इतिः त‚द् दूष‚य‚ति । एवं हि‚{\tiny $_{lb}$}‚ ‚{\tiny $_{lb}$}‚ \leavevmode\ledsidenote{\textenglish{32/dm}}‚{\tiny $_{lb}$}‚ 
	  
	प्र‚द‚र्शितं च प्राप‚य‚त् स‚म्य‚ग्ज्ञान‚मेव । प्र‚द‚र्शितं चाप्राप‚य‚त् मिथ्याज्ञान‚म् । अप्राप‚कं च‚{\tiny $_{lb}$}‚ क‚थ‚म‚र्थ‚सिद्धिनिब‚न्ध‚नं स्यात्? त‚स्माद् य‚न्मिथ्याज्ञानं न त‚तोऽर्थ‚सिद्धिः । य‚त‚श्चार्थ‚सिद्धिस्त‚त्‚{\tiny $_{lb}$}‚ स‚म्य‚ग्ज्ञान‚मेव । अत एव स‚म्य‚ग्ज्ञानं य‚त्न‚तो व्युत्पाद‚नीय‚म् । य‚त‚स्त‚देव पुरुषार्थ‚सिद्धि\edtext{}{\lemma{सिद्धि}\Bfootnote{सिद्धेः \cite{dp-msB}}}—‚{\tiny $_{lb}$}‚निब‚न्ध‚न‚म् ।‚{\tiny $_{lb}$}‚ व्याख्याय‚माने \textbf{स‚र्व}श‚ब्देन \textbf{पुरुषार्थ‚सिद्धे}र‚विशेष‚णाद् मिथ्याज्ञानात् काक‚तालीयार्थ‚सिद्धिर‚नि‚{\tiny $_{lb}$}‚वारिता स्यात् । न च सा स‚म्भ‚विनी । य‚था च सा न स‚म्भ‚व‚ति, त‚थाऽन‚न्त‚र‚मेव प्र‚तिपाद‚यिष्य‚ते ।
	\pend% ending standard par
      ‚{\tiny $_{lb}$}‚

	  \pstart \leavevmode% starting standard par
	योऽपि \textbf{शान्त‚भ‚द्रः} स‚र्व‚श्चासौ पुरुषार्थ‚श्च, स‚र्वेषां वा पुरुषाणाम‚र्थ‚स्त‚स्य सिद्धिः इति‚{\tiny $_{lb}$}‚ व्याच‚ष्टे सोऽप्य‚न‚यैव द्वारा निराकृतः ।
	\pend% ending standard par
      ‚{\tiny $_{lb}$}‚

	  \pstart \leavevmode% starting standard par
	न‚नु स‚र्व‚श‚ब्द‚स्य प्र‚कार‚कार्त्स्न्य‚वृत्तेर‚पि द‚र्श‚नान्न ज्ञाय‚ते किंवृत्तिर‚त्राभिप्रेत इत्याह—‚{\tiny $_{lb}$}‚\textbf{स‚र्व‚श‚ब्द} इति । \textbf{द्र‚व्य‚स्य} निःशेष‚तायां \textbf{वृत्तः} प्र‚वृत्तो वाच‚क‚भावेन । य‚था स‚र्वोत्प‚त्तिम‚ता‚{\tiny $_{lb}$}‚मीश्व‚रो निमित्त‚कार‚ण‚म् इत्य\leavevmode\ledsidenote{\textenglish{14b/ms}}त्र । द्र‚व्य‚श‚ब्देन चेदं त‚दिति व्य‚प‚देश‚योग्यं ग्र‚हीत‚व्य‚म् ।‚{\tiny $_{lb}$}‚ न तु \textbf{वैशेषिक}सिद्धान्त‚प्र‚सिद्धं पृथिव्यादि । त‚दुक्त‚म्--
	\pend% ending standard par
      ‚{\tiny $_{lb}$}‚
	  \bigskip
	  \begingroup
	
	    \begin{quote}
	  
	    
	    \stanza[\smallbreak]
	व‚स्तूप‚ल‚क्ष‚णं य‚त्र स‚र्व‚नाम प्र‚स‚ज्य‚ते ।&द्र‚व्य‚मित्युच्य‚ते सोऽर्थो भेद्य‚त्वेन विव‚क्षितः ॥ इति ।\&[\smallbreak]


	
	    \end{quote}
	  
	  \endgroup
	‚{\tiny $_{lb}$}‚

	  \pstart \leavevmode% starting standard par
	पुरुषार्थ‚सिद्धिर‚पीदं त‚दिति व्य‚प‚देश‚योग्या । तेन साऽपि द्र‚व्य‚म् । त‚थावृत्तेर‚स्य ग्र‚ह‚णाद‚न्य‚{\tiny $_{lb}$}‚वृत्तित‚या प्र‚सिद्ध‚स्यापि प्र‚तिषेधं क‚ण्ठोक्तं क‚रोति । \textbf{वृत्त} इति व‚र्त्त‚ते । य‚था पूर्वे व्याच‚क्ष‚ते‚{\tiny $_{lb}$}‚ स‚र्व‚र‚स‚भोक्ताऽयं भिक्षाकः इति । य‚थाऽयं स‚र्व‚श‚ब्दः प्र‚कार‚कार्त्स्न्य‚वाची त‚द्व‚द‚त्रापीति ।‚{\tiny $_{lb}$}‚ प्र‚कार‚कार्त्स्न्य‚वृत्तेर‚ग्र‚ह‚णे को गुण इत्याह--\textbf{त‚त} इति । त‚तः प्र‚कार‚कार्त्स्न्य‚वृत्त‚स्याग्र‚ह‚ण‚त् ।‚{\tiny $_{lb}$}‚ त‚थापि क‚थ‚म‚य‚म‚र्थो भ‚व‚तीति चेत् । भ‚व‚ति हि प्र‚कार‚कार्त्स्न्य‚व‚च‚ने स‚र्व‚श‚ब्दे स‚त्य‚य‚म‚र्थः--न‚{\tiny $_{lb}$}‚ सोऽस्ति पुरुषार्थ‚सिद्धेः प्र‚कारः फ‚ल‚ल‚क्ष‚ण उपादान‚ल‚क्ष‚णो वा यो न स‚म्य‚ग्ज्ञान‚निब‚न्ध‚न इति ।‚{\tiny $_{lb}$}‚ स‚ति चैवं द्विप्र‚काराऽपि सिद्धिः स‚म्य‚ग्ज्ञान‚पूर्विकेत्य‚य‚म‚र्थो भ‚व‚ति ।
	\pend% ending standard par
      ‚{\tiny $_{lb}$}‚

	  \pstart \leavevmode% starting standard par
	न‚नु चास्मिन्न‚प्य‚र्थे प्र‚कार‚द्वित‚य‚स्य संगृहीत‚त्वात् किम‚संगृहीतं नाम येनाय‚म‚र्थो य‚त्नेन‚{\tiny $_{lb}$}‚ म‚ह‚ता हीय‚त इति? क‚थं न हीय‚तां क‚स्याश्चित् त‚त्प्र‚कारान्तःप‚तिताया हान‚व्य‚क्ते\edtext{}{\lemma{क्ते}\Bfootnote{स‚म्य‚क् न प‚ठ्य‚ते ।}}\add{... ... ...}।‚{\tiny $_{lb}$}‚ न हि ष‚ड‚पि र‚स‚प्र‚कारान् भुञ्जानः क‚श्चिन्म‚धुराम्लादिर‚स‚व्य‚क्तीः स‚र्वा एव भुङ्क्ते । न च‚{\tiny $_{lb}$}‚ त‚थाऽकुर्व‚न् स‚र्व‚र‚स‚भोक्ता न भ‚व‚तीति ।
	\pend% ending standard par
      ‚{\tiny $_{lb}$}‚

	  \pstart \leavevmode% starting standard par
	य‚द्य‚यं नार्थः क‚स्त‚र्हीत्याह--\textbf{अपि} तु--इति । निपात‚स‚मुदायोऽयं किन्त्वित्य‚स्यार्थे‚{\tiny $_{lb}$}‚ स‚र्व‚त्र व‚र्त्त‚ते ।
	\pend% ending standard par
      ‚{\tiny $_{lb}$}‚

	  \pstart \leavevmode% starting standard par
	न‚नु \textbf{कृत्स्नैवासौ स‚म्य‚ग्ज्ञान‚निब‚न्ध‚ने}त्युच्य‚माने मिथ्याज्ञानात् काक‚तालीयाऽप्य‚र्थ\textbf{सिद्धि}‚{\tiny $_{lb}$}‚र्नास्तीति द‚र्शितं स्यात् । न चैत‚द् युज्य‚ते । य‚तो य‚द्य‚पि मिथ्याज्ञानान्निय‚म‚व‚ती नास्त्य‚र्थ‚{\tiny $_{lb}$}‚सिद्धिः, त‚थापि कादाचित्की विद्य‚त एव । य‚था क‚श्चित् द‚ह‚नोप‚रिव‚र्त्तिं म‚श‚क‚व‚र्त्तिं धूम‚{\tiny $_{lb}$}‚म‚व‚सायाग्निम‚नुमाय य‚दि प्र‚वृत्त्याऽग्निमासाद‚य‚ति । अत एव च मिथ्याज्ञान‚स्य कादाचित्कार्थ‚{\tiny $_{lb}$}‚‚{\tiny $_{lb}$}‚ \leavevmode\ledsidenote{\textenglish{33/dm}}‚{\tiny $_{lb}$}‚ सिद्धिनिब‚न्ध‚न‚त्वात्, स‚म्य‚ग्ज्ञान‚स्य तु निय‚मेनार्थ‚सिद्धिनिब‚न्ध‚न‚हेतुत्वात्स‚म्य‚ग्ज्ञान‚मेव य‚त्न‚तो‚{\tiny $_{lb}$}‚ व्युत्पाद्य‚ते । न त्वेताव‚ता मिथ्याज्ञानान्नास्त्येवार्थ‚सिद्धिः । प्र‚कार‚कार्त्स्न्य‚वृत्त‚ग्र‚ह‚णे तु नायं‚{\tiny $_{lb}$}‚ दोषः । एवं हि स‚म्य‚ग्ज्ञानासाध्यः प‚र‚प्र‚कारो निराक्रिय‚ते । न तु त‚त्प्र‚कारान्तःपातिन्याः‚{\tiny $_{lb}$}‚ क‚स्याश्चित् सिद्धिव्य‚क्तेर्मिथ्याज्ञान‚साध्य‚त्व‚म् । न च हानोपादान‚व्य‚क्त‚यः प्र‚त्येकं प्र‚कार‚श‚ब्द‚{\tiny $_{lb}$}‚वाच्याः प्र‚कारान‚न्त्य‚प्र‚स‚ङ्गात् । त‚स्मात् स‚र्व‚श‚ब्दः प्र‚कार‚कार्त्स्न्य‚वृत्तिरेव ग्र‚हीत‚व्यः--इति‚{\tiny $_{lb}$}‚ \textbf{पूर्वेषां} म‚त‚माश‚ङ्क्याह--\textbf{मिथ्येति} । काक‚ताल‚योः संयोग इवाक‚स्मिकी \textbf{काक‚तालीया} ।‚{\tiny $_{lb}$}‚ कुतो नास्तीत्याह--\textbf{त‚था ही}ति । \textbf{त‚त} इति मिथ्याज्ञानात् । त‚द‚पि प्राप‚य‚त्येवेत्याह—‚{\tiny $_{lb}$}‚\textbf{प्र‚द‚र्शित‚म्} इति । \textbf{चो} य‚स्मात् ।
	\pend% ending standard par
      ‚{\tiny $_{lb}$}‚

	  \pstart \leavevmode% starting standard par
	प्र‚द‚र्शितं प्राप‚य‚द‚पि मिथ्याज्ञानं भ‚विष्य‚ति । त‚तः स‚म्य‚ग्ज्ञान‚मेवेत्य‚व‚धार‚ण‚म‚युक्त‚{\tiny $_{lb}$}‚मित्याह--\textbf{प्र‚द‚र्शित‚म्} इति । \textbf{चो}ऽव‚धार‚णे \textbf{अप्राप‚य}दित्य‚तः प‚रो द्र‚ष्ट‚व्यः । ल‚क्ष‚ण‚हेत्वोः\href{http://sarit.indology.info/?cref=Pā.3.2.126}{पाणिनि—३. २. १२६} इति हेतौ श‚तुर्विधानात् । प्र‚द‚र्शितार्थाप्राप‚णादेव मिथ्याज्ञान \leavevmode\ledsidenote{\textenglish{15a/ms}} मित्य‚र्थः ।‚{\tiny $_{lb}$}‚ मिथ्याज्ञान‚स्य हीद‚मेव त‚त्त्वं य‚त् प्र‚द‚र्शिताप्राप‚क‚त्वं नाम । प्राप‚के तु मिथ्याज्ञान‚व्य‚प‚देशः स‚मा‚{\tiny $_{lb}$}‚रोपितः स्यादिति भावः । मा भूत् प्राप‚क‚म्, त‚द‚र्थ‚सिद्धेस्तु निब‚न्ध‚नं किं न स्यादित्याह—‚{\tiny $_{lb}$}‚\textbf{अप्राप‚क}मिति । \textbf{चो} य‚स्माद‚र्थे ।
	\pend% ending standard par
      ‚{\tiny $_{lb}$}‚

	  \pstart \leavevmode% starting standard par
	\textbf{त‚स्मादि}त्यादिनोक्तार्थोप‚संहारं क‚रोति । य‚स्मादुप‚द‚र्शिताप्राप‚क‚मेव मिथ्याज्ञानं‚{\tiny $_{lb}$}‚ \textbf{त‚स्माद्} य‚त्किञ्चिन्मिथ्याज्ञानं त‚तः स‚र्व‚त एव नार्थ‚सिद्धिः । यः पुन‚र‚क‚स्माद् व‚ह्नेरुप‚रि‚{\tiny $_{lb}$}‚ म‚श‚क‚व‚र्त्तिं धूम‚म‚व‚साय व‚ह्निम‚व‚स्य‚ति, सोऽप्य‚नुमान‚कालात्पूर्वं याव‚न्म‚श‚क‚व‚र्त्त्यादिभ्यो व्यावृत्तं‚{\tiny $_{lb}$}‚ व‚ह्निकार्यं निय‚तं धूम‚रूपं निरूप्य विवेचितं त‚दानीम‚नुमान‚काले नानुस‚र‚ति, त‚द‚नुस‚र‚णे‚{\tiny $_{lb}$}‚ भ्रान्त्य‚योगात्, ताव‚द‚निव‚र्त्तिंते संश‚य‚हेतौ, तिष्ठ‚तु ताव‚न्म‚श‚क‚व‚र्त्तिद‚र्श‚न‚म्, स‚त्य‚धूम‚द‚र्श‚नेऽपि‚{\tiny $_{lb}$}‚ स‚न्दिग्ध‚व‚ह्निकार्य‚त्वात्क‚थं निःश‚ङ्को नाम स‚चेत‚नः । त‚स्मान्निश्चित‚निय‚त‚रूपादेव धूमाद्‚{\tiny $_{lb}$}‚ व‚ह्निनिश्च‚यः । अन्य‚स्मात्त्वेकांशाव‚ग्र‚होऽप्य‚निव‚र्तित‚श‚ङ्काहेतुः संश‚य एव । संश‚य‚श्च भावाभावा‚{\tiny $_{lb}$}‚निय‚त‚स्यार्थ‚स्याश‚क्य‚प्राप‚ण‚स्य द‚र्श‚को नार्थ‚सिद्धेर्निब‚न्ध‚न‚म् । ज्ञानान्त‚रादेव तु सा प्राप्तिरित्य‚{\tiny $_{lb}$}‚स्याभिप्रायः । एव‚म‚न्य‚त्रापि मिथ्याज्ञाने य‚था प्रामाण्यं न युज्य‚ते, य‚था च ज्ञानान्त‚रादेव सा‚{\tiny $_{lb}$}‚ प्राप्तिस्त‚था \textbf{ध‚र्मोत्त‚रेणैव विनिश्च‚य‚टीकायां} विप‚ञ्चित‚मिति नेह प्र‚त‚न्य‚ते ।
	\pend% ending standard par
      ‚{\tiny $_{lb}$}‚

	  \pstart \leavevmode% starting standard par
	कुत‚श्चिद‚र्थ‚सिद्धिर‚पि स्यादित्याह--\textbf{य‚त‚श्च}--इति । \textbf{चोऽ}न्य‚स्माद् भेद‚व‚द्रूप‚मुप‚द‚र्श‚य‚ति ।‚{\tiny $_{lb}$}‚ अर्थ‚सिद्धिनिब‚न्ध‚नं मिथ्याज्ञान‚मा\textbf{चार्य}स्यापि नाभिम‚त‚मिति साम‚र्थ्याद् द‚र्श‚य‚ति । य‚दाऽऽह—‚{\tiny $_{lb}$}‚\textbf{अत ए}वेति । य‚त एव मिथ्याज्ञानाद‚र्थ‚सिद्धिर्नास्ति, स‚म्य‚ग्ज्ञानाच्चास्त्येव । \textbf{अतो}ऽस्मादेव हेतोः ।‚{\tiny $_{lb}$}‚ य‚त्न‚ग्र‚ह‚ण‚म‚नुष‚ङ्गेण मिथ्याज्ञान‚व्युत्पाद‚नं द‚र्श‚य‚ति । मिथ्याज्ञान‚स्यापि क‚स्य‚चिद् व्युत्पाद‚ना‚{\tiny $_{lb}$}‚दित्युक्त‚प्राय‚म् । \textbf{अत एवे}त्य‚नेन य‚देव हेतुत्वेनापेक्षितं त‚देव \textbf{य‚त} इत्यादिना सुख‚प्र‚तिप‚त्त्य‚र्थं‚{\tiny $_{lb}$}‚ क‚ण्ठोक्तं क‚रोति । अय‚माश‚यः--य‚द्या\textbf{चार्यो} मिथ्याज्ञानाद‚प्य‚र्थ‚सिद्धिं भ‚वित्रीम‚भिप्रेयात्, त‚दा‚{\tiny $_{lb}$}‚ न स‚म्य‚ग्ज्ञान‚मेव प्र‚स्तारेण व्युत्पाद‚येत् । त‚देव च त‚था व्युत्पादित‚वान् । अतोऽव‚सीय‚ते नैव‚{\tiny $_{lb}$}‚ त‚स्येद‚म‚भिप्रेत‚मिति ।
	\pend% ending standard par
      ‚{\tiny $_{lb}$}‚

	  \pstart \leavevmode% starting standard par
	न‚नु स‚म्य‚ग्ज्ञान‚पूर्विकैवेत्य‚व‚धार‚णे स‚ति मिथ्याज्ञानाद‚र्थ‚सिद्धिर्नास्तीति ल‚भ्य‚ते । न‚{\tiny $_{lb}$}‚ चोपात्त‚म‚व‚धार‚ण‚म् त‚त्क‚थ‚म‚य‚म‚र्थो ल‚भ्य‚त इत्याह--\textbf{त‚त} इति । य‚तः स‚र्व‚श‚ब्दे द्र‚व्य‚कार्त्स्न्य- \leavevmode\ledsidenote{\textenglish{34/dm}}‚{\tiny $_{lb}$}‚ 
	  
	त‚तो याव‚द् ब्रूयात् \edtext{}{\lemma{ब्रूयात्}\Bfootnote{ब्रूयात् पुरु० \cite{dp-msB} \cite{dp-msC} \cite{dp-edP} \cite{dp-edH} \cite{dp-edE}}}या काचित् पुरुषार्थ‚सिद्धिः \edtext{}{\lemma{सिद्धिः}\Bfootnote{सिद्धिः स‚म्य० \cite{dp-msA} \cite{dp-msC} \cite{dp-edP} \cite{dp-edH} \cite{dp-edE}}}सा स‚म्य‚ग्ज्ञान‚निब‚न्ध‚नैवेति ताव‚दुक्तं‚{\tiny $_{lb}$}‚ स‚र्वा सा\edtext{}{\lemma{सा}\Bfootnote{स‚र्वा स‚म्य० \cite{dp-msD}}} स‚म्य‚ग्ज्ञान‚पूर्विकेति । इतिश‚ब्द‚स्त‚स्मादित्य‚स्मिन्न‚र्थे । य‚त्त‚दोश्च नित्य‚म‚भि‚{\tiny $_{lb}$}‚स‚म्ब‚न्धः । त‚द‚य‚म‚र्थः--\edtext{\textsuperscript{*}}{\lemma{*}\Bfootnote{त‚स्मात् \cite{dp-edP}}}य‚स्मात् स‚म्य‚ग्ज्ञान‚पूर्विका स‚र्व‚पुरुषार्थ‚सिद्धिः, त‚स्मात् त‚त्\edtext{}{\lemma{त्}\Bfootnote{त‚द्व्युत्पा० \cite{dp-msA} \cite{dp-edP} \cite{dp-edH} \cite{dp-edE}}}‚{\tiny $_{lb}$}‚ स‚म्य‚ग्ज्ञानं व्युत्पाद्य‚ते । ‚{\tiny $_{lb}$}‚ 
	  
	\edtext{\textsuperscript{*}}{\lemma{*}\Bfootnote{न‚नु ब‚हुव्रीहिणा स‚र्व‚पुरुषार्थ‚सिद्धिरुच्य‚ते । त‚तः प्राधान्यात् त‚च्छ‚ब्देन त‚त्स‚म्ब‚न्धो‚{\tiny $_{lb}$}‚ युक्तो न तु ज्ञान‚स्येत्याह--\cite{dp-msD-n}}}य‚द्य‚पि च\edtext{}{\lemma{च}\Bfootnote{य‚द्य‚पि स‚मा० \cite{dp-msC}}} स‚मासे गुणीभूतं स‚म्य‚ज्ञानं त‚थापीह प्र‚क‚र‚णे व्युत्पाद‚यित‚व्य‚त्वात्‚{\tiny $_{lb}$}‚ प्र‚धान‚म् । त‚त‚स्त‚स्यैव त‚च्छ‚ब्देन स‚म्ब‚न्धः । ‚{\tiny $_{lb}$}‚ 
	  
	\edtext{\textsuperscript{*}}{\lemma{*}\Bfootnote{एवं म‚न्य‚ते--विप्र‚तिप‚त्तिनिराक‚र‚ण‚द्वारेण स‚म्य‚ग् ज्ञाप्य‚तेऽनेन प्र‚क‚र‚णेनेति व‚क्ष्य‚माण‚{\tiny $_{lb}$}‚ल‚क्ष‚ण‚युक्तं स‚म्य‚ग्ज्ञानं नान्य‚ल‚क्ष‚ण‚युक्त‚मिति स्व‚रूप‚क‚थ‚न‚म् । तामेव विप्र‚तिप‚त्तिं क्र‚मेण‚{\tiny $_{lb}$}‚ द‚र्श‚य‚ति--\cite{dp-msD-n}}}व्युत्पाद्य‚ते इति विप्र‚तिप‚त्तिनिराक‚र‚णेन प्र‚तिपाद्य‚ते\edtext{}{\lemma{ते}\Bfootnote{प्र‚तिपाद्य‚ते व्युत्पाद्य‚त इति \cite{dp-edE} \cite{dp-edP}}} इति ॥‚{\tiny $_{lb}$}‚ वृत्तौ स‚ति स‚र्वा कृत्स्ना पुरुषार्थ‚सिद्धिः स‚म्य‚ग्ज्ञान‚पूर्विका, न तु काचिद् व्य‚क्तिर‚स्ति या न‚{\tiny $_{lb}$}‚ त‚त्पूर्विकेत्य‚य‚म‚र्थो ल‚भ्य‚ते, त‚तः कार‚णात् \textbf{पुरुषार्थ‚सिद्धिः स‚म्य‚ज्ञान‚निब‚न्ध‚नैवेति} साव‚धार‚णं‚{\tiny $_{lb}$}‚ वाक्य‚म् । य‚त्प‚रिमाण‚म‚भिधेयं ब्रूयाद् व‚क्तुं श‚क्नोति, \textbf{ताव‚त्प}रिमाण‚मुक्त‚म् \textbf{स‚र्वा स‚म्य‚ज्ञान‚{\tiny $_{lb}$}‚पूर्विके}त्य‚नेन । व‚र्त्त‚ता\textbf{मितिश‚ब्द‚स्त‚स्मादित्य‚स्यार्थे} । क्व पुन‚र्य‚त् श‚ब्दोऽस्ति येन व‚क्ष्य‚{\tiny $_{lb}$}‚माणार्थ‚स‚ङ्ग‚तिरित्याह--\textbf{य‚द्} इति । \textbf{चो} \leavevmode\ledsidenote{\textenglish{15b/ms}} व्य‚क्त‚मेत‚दित्य‚स्मिन्न‚र्थे । श‚ब्द‚योः साक्षा‚{\tiny $_{lb}$}‚द‚न्योन्यापेक्षाभावात् य‚त्त‚द‚र्थ‚योरिति द्र‚ष्ट‚व्य‚म् । \textbf{अभिस‚म्ब‚न्धो} व्य‚पेक्षा । एवं स‚ति कीदृशोऽर्थो‚{\tiny $_{lb}$}‚ व्य‚व‚तिष्ठ‚त इत्याह--\textbf{त‚द्} इति ।
	\pend% ending standard par
      ‚{\tiny $_{lb}$}‚

	  \pstart \leavevmode% starting standard par
	न‚नु ब‚हुव्रीहौ गुणीभूतं स‚म्य‚ग्ज्ञानं त‚त्क‚थं त‚स्याप्र‚धान‚स्य प्र‚धान‚प्र‚त्य‚व‚म‚र्शिना त‚च्छ‚ब्दे‚{\tiny $_{lb}$}‚ प‚राम‚र्शः स्यादित्याह--\textbf{य‚द्य‚पि चेति} । य‚द्य‚पि चेति निपात‚स‚मुदायो विशेषाभिधानार्थाभ्युप‚ग‚मे‚{\tiny $_{lb}$}‚ ब‚र्त्त‚ते । \textbf{म‚हाभाष्ये} चायं प्रायेण दृश्य‚ते । \textbf{स‚मास} इति स‚मासार्थ इत्य‚र्थः । \textbf{व्युत्पाद‚यित‚{\tiny $_{lb}$}‚व्य‚त्वात्प्र‚धान‚मि}त्य‚भिद‚ध‚तोऽय‚माश‚यः--द्वेधा हि प्राधान्यं श‚ब्द‚तोऽर्थ‚त‚श्च । त‚त्र शाब्देन‚{\tiny $_{lb}$}‚ न्यायेनास‚त्य‚पि प्राधान्ये स‚म्य‚ग्ज्ञान‚स्य व्युत्पाद‚यित‚व्य‚त‚या बुद्ध्य‚न्त‚रेणोप‚स्थापित‚स्य स्व‚त‚न्त्र‚{\tiny $_{lb}$}‚स्याऽऽर्थेन न्यायेन प्राधान्यं कोऽप‚ह‚स्त‚येदिति \textbf{त‚च्छ‚ब्देन} त‚स्य प‚राम‚र्शो न विरुद्ध्य‚त इत्येवं‚{\tiny $_{lb}$}‚ व्याच‚क्षाणो \textbf{विनीत‚देव}व्याख्यां तिर‚स्क‚रोति । एवं ह्य‚सौ व्याच‚ष्टे--त‚दिति न‚पुंस‚क‚लिङ्गेन‚{\tiny $_{lb}$}‚ निर्देशात् स‚म्य‚ग्ज्ञानं प‚रामृश्य‚त इति । स‚म्य‚ग्ज्ञान‚स्याप्राधान्येन प‚राम‚र्शानुप‚प‚त्तौ त‚ल्लिङ्ग‚{\tiny $_{lb}$}‚ग्र‚ह‚ण‚स्यैवायोगात् क‚थ‚म‚सिद्धेन साध्य‚तामित्याश‚यः । त‚तः प्राधान्यात्त‚स्यैव स‚म्य‚ग्ज्ञान‚स्य‚{\tiny $_{lb}$}‚ ‚{\tiny $_{lb}$}‚ \leavevmode\ledsidenote{\textenglish{35/dm}}‚{\tiny $_{lb}$}‚ 
	  
	च‚तुर्विधा चात्र विप्र‚तिप‚त्तिः संख्या-ल‚क्ष‚ण-गोच‚र-फ‚ल‚विष‚या । त‚त्र संख्याविप्र‚तिप्र‚त्तिं‚{\tiny $_{lb}$}‚ निराक‚र्त्तुमाह-- ‚{\tiny $_{lb}$}‚ 
	  
	द्विविधं स‚म्य‚ग्ज्ञान‚म् ॥ २ ॥‚{\tiny $_{lb}$}‚ 
	  
	द्विविध‚म्\edtext{}{\lemma{म्}\Bfootnote{द्विविध‚म्--इति इति नास्ति \cite{dp-edH}}} इति--द्वौ\edtext{}{\lemma{द्वौ}\Bfootnote{द्वे विधे \cite{dp-msB}}} विधौ प्र‚काराव‚स्येति द्विविध‚म् । संख्याप्र‚द‚र्श‚न‚द्वारेण च‚{\tiny $_{lb}$}‚ व्य‚क्तिभेदो द‚र्शितो भ‚व‚ति ।‚{\tiny $_{lb}$}‚ त‚च्छ‚ब्देन स‚म्ब‚न्ध‚नं \textbf{स‚म्ब‚न्धः} स्वीकार इति याव‚त् । त‚तोऽय‚म‚र्थः--त‚त् स‚म्य‚ग्ज्ञानं क‚र्म‚भूतं‚{\tiny $_{lb}$}‚ \textbf{विप्र‚तिप‚त्तिनिराक‚र‚णेन प्र‚तिपाद्य‚ते} बोध्य‚ते शिष्य‚ज‚न इत्य‚र्थात् । अत एव त‚दिति द्वितीयान्त‚{\tiny $_{lb}$}‚मेत‚दिति सोप‚प‚त्तिक‚माह ।\edtext{\textsuperscript{*}}{\lemma{*}\Bfootnote{स्मार्य‚ते स‚म‚यं प‚रः \href{http://sarit.indology.info/?cref=pv.4.267}{प्र‚माण‚वा० ४. २६७}--सं० ।}} स‚म‚यं स्म\edtext{}{\lemma{स्म}\Bfootnote{स्मा}}र्य‚ते प‚रः \href{http://sarit.indology.info/?cref=pv.4.267}{प्र‚माण‚वा०--४. २६७} इति य‚था ।‚{\tiny $_{lb}$}‚ \textbf{विप्र‚तिप‚त्तिनिराक‚र‚णेन प्र‚तिपाद्य‚त} इति ब्रुव‚ता ल‚क्ष‚णं नं विरुध्य‚ते । प्र‚सिद्धानि प्र‚माणानि ‚{\tiny $_{lb}$}‚ \href{http://sarit.indology.info/?cref=nā.2}{न्याया० २} इत्यादि त‚द‚प‚ह‚स्तितं वेदित‚व्य‚म् ।
	\pend% ending standard par
      ‚{\tiny $_{lb}$}‚

	  \pstart \leavevmode% starting standard par
	न‚नु \textbf{वार्त्तिका}दिनैव स‚म्य‚ग्ज्ञान‚स्य व्युत्पाद‚नात् क‚थ‚म‚स्य न वैय‚र्थ्य‚मिति चेत् संक्षिप्त‚रुचीन्‚{\tiny $_{lb}$}‚ प्राज्ञान‚धिकृत्येदं प्र‚क‚र‚णं प्र‚णीत‚मित्य‚दोषः ।
	\pend% ending standard par
      ‚{\tiny $_{lb}$}‚

	  \pstart \leavevmode% starting standard par
	अथ काऽत्र विप्र‚तिप‚त्तिः य‚न्निराक‚र‚णेनेदं प्र‚तिपाद्य‚त इत्याह--\textbf{च‚तुर्विधे}ति । \textbf{चो}‚{\tiny $_{lb}$}‚ य‚स्मात् । \textbf{च‚तुर्विधा} च‚तुःप्र‚कारा । \textbf{अत्र} स‚म्य‚ग्ज्ञाने विरुद्धा प्र‚तिप‚त्ति\textbf{र्विप्र‚तिप‚त्तिः} । किं‚{\tiny $_{lb}$}‚विष‚या सेत्याह--\textbf{संख्ये}ति । अनेन विष‚य‚भेदाच्चातुर्विध्यं विप्र‚तिप‚त्तेर्द‚र्शित‚म् । त‚थाहि‚{\tiny $_{lb}$}‚ संख्याविप्र‚तिप‚त्तिस्ताव‚त् प्र‚त्य‚क्ष‚मेवैकं प्र‚माण‚मिति \textbf{लोकाय‚तिकानाम्} । प्र‚त्य‚क्षानुमानाप्त‚व‚च‚नानि‚{\tiny $_{lb}$}‚ त्रीण्येव प्र‚माणानीति \textbf{सांख्यानाम्} । आर्ष‚स्य ज्ञान‚स्य प्र‚तिभाप‚र‚नाम्नः क‚दाचिदिह लौकिकाना‚{\tiny $_{lb}$}‚मुत्प‚द्य‚मान‚स्य प्र‚त्य‚क्षानुमान‚योरेक‚त्राप्य‚न्त‚र्भावाप्र‚द‚र्श‚नात्प्र‚त्य‚क्षानुंमानार्षाण्येवेति \textbf{चिर‚न्त‚न‚वैशेषि‚{\tiny $_{lb}$}‚काणाम्} । प्र‚त्य‚क्षानुमानोप‚मान‚श‚ब्दा एवेति \textbf{नैयायिकानाम्} । प्र‚त्य‚क्षानुमानोप‚मान‚श‚ब्दार्थाप‚त्त‚य‚{\tiny $_{lb}$}‚ एवेति \textbf{प्राभाक‚राणामिति} । प्र‚त्य‚क्षानुमानोप‚मान‚श‚ब्दार्थाप‚त्त्य‚भावा एवेति \textbf{कौमारिलानाम् ।}
	\pend% ending standard par
      ‚{\tiny $_{lb}$}‚

	  \pstart \leavevmode% starting standard par
	ल‚क्ष‚ण‚विप्र‚तिप‚त्तिर‚पि--स‚विक‚ल्प‚क‚मेव प्र‚त्य‚क्ष‚मिति \textbf{वैयाक‚र‚ण‚बार्ह‚स्प‚त्यादीनाम्} ।‚{\tiny $_{lb}$}‚ स‚विक‚ल्प‚कं निर्विक‚ल्प‚कं चेति \textbf{नैयायिक} \leavevmode\ledsidenote{\textenglish{16a/ms}} \textbf{मीमांस‚कादीनाम्} । पीत‚श‚ङ्खादिज्ञानं भ्रान्त‚म‚पि‚{\tiny $_{lb}$}‚ प्र‚त्य‚क्ष‚मित्यंश‚संवाद‚वादिनाम् । एक‚ल‚क्ष‚ण‚हेतुज‚म‚नुमान‚मित्य\textbf{ह्रीकाणाम्} । ष‚ड्ल‚क्ष‚ण‚हेतुज‚मिति‚{\tiny $_{lb}$}‚ \textbf{पूर्व‚यौगानाम्} । प‚ञ्च‚ल‚क्ष‚ण‚हेतुजं च‚तुर्ल‚क्ष‚ण‚हेतुजं चे\edtext{}{\lemma{चे}\Bfootnote{वे}}ति \textbf{नैयायिकानाम्} ।
	\pend% ending standard par
      ‚{\tiny $_{lb}$}‚

	  \pstart \leavevmode% starting standard par
	विष‚य‚विप्र‚तिप‚त्तिर‚पि--सामान्य‚विष‚ये प्र‚त्य‚क्षानुमाने इति \textbf{नैयायिक‚मीमांस‚कादीनाम्} ।
	\pend% ending standard par
      ‚{\tiny $_{lb}$}‚

	  \pstart \leavevmode% starting standard par
	फ‚ल‚विप्र‚तिप‚त्तिर‚पि--स‚र्वेषां प्र‚माणा\edtext{}{\lemma{माणा}\Bfootnote{ण}}व्य‚तिरिक्त‚मेव प्र‚माण‚रू\edtext{}{\lemma{रू}\Bfootnote{प्र‚मारू}}पं फ‚ल‚मिति ।
	\pend% ending standard par
      ‚{\tiny $_{lb}$}‚

	  \pstart \leavevmode% starting standard par
	त‚त्र तासु म‚ध्ये संत्ख्यात्व‚जात्या निर्धार्य‚ते । त‚तो निर्धार‚णे युक्तं \textbf{द्वौ विधौ प्र‚काराव‚स्येति}‚{\tiny $_{lb}$}‚ विगृह्ण‚न् विध‚श‚ब्दोऽप्य‚स्ति प्र‚कार‚वाचीति द‚र्श‚य‚ति । त‚था हि \textbf{प्र‚कीर्ण‚वृत्ति}कृद्\textbf{ध‚र्म‚पालेनापि}‚{\tiny $_{lb}$}‚ ‚{\tiny $_{lb}$}‚ \leavevmode\ledsidenote{\textenglish{36/dm}}‚{\tiny $_{lb}$}‚ 
	  
	द्वे\edtext{}{\lemma{द्वे}\Bfootnote{द्वे स‚म्य० \cite{dp-msA} \cite{dp-edP} \cite{dp-edE}}} एव स‚म्य‚ग्ज्ञान‚व्य‚क्ती इति । \edtext{\textsuperscript{*}}{\lemma{*}\Bfootnote{न‚नु स‚म्य‚ग्ज्ञान‚ल‚क्ष‚णे क‚थिते स‚ति प‚श्चात् त‚द्भेदः प्र‚द‚र्श‚यितुं युज्य‚त इत्याह--\cite{dp-msD-n}}}व्य‚क्तिभेदे च\edtext{}{\lemma{च}\Bfootnote{व्य‚क्तिभेदे प्र‚द० \cite{dp-msA} \cite{dp-edP} \cite{dp-edH} \cite{dp-edE} \cite{dp-edN} भेदे प्र‚द० \cite{dp-msB} व्य‚क्तिभेदे च द‚र्शिते \cite{dp-msD}}} प्र‚द‚र्शिते प्र‚तिव्य‚क्तिनिय‚तं स‚म्य‚{\tiny $_{lb}$}‚ग्ज्ञान‚ल‚क्ष‚ण‚माख्यातुं श‚क्य‚म् । अप्र‚द‚र्शिते तु व्य‚क्तिभेदे स‚क‚ल‚व्य‚क्त‚त्य‚नुयायि स‚म्य‚ग्ज्ञान‚{\tiny $_{lb}$}‚ल‚क्ष‚ण‚मेकं न श‚क्यं व‚क्तुम् । त‚तो ल‚क्ष‚ण\edtext{}{\lemma{ण}\Bfootnote{ल‚क्ष‚ण‚भेद‚क‚थ० \cite{dp-msA} \cite{dp-msB} \cite{dp-msD} \cite{dp-edP} \cite{dp-edH} \cite{dp-edE} \cite{dp-edN}}} क‚थ‚नाङ्ग‚मेव संख्याभेद‚क‚थ‚न‚म् । अप्र‚द‚र्शिते \edtext{}{\lemma{र्शिते}\Bfootnote{च \cite{dp-msC} अप्र‚द‚र्शिते व्य० \cite{dp-msA} \cite{dp-edE}}}तु‚{\tiny $_{lb}$}‚ व्य‚क्तिभेदात्म‚के संख्याभेदे ल‚क्ष‚ण‚भेद‚स्य द‚र्श‚यितुम‚श‚क्य‚त्वात् । ल‚क्ष‚ण‚निर्देशाङ्ग‚त्वादेव च‚{\tiny $_{lb}$}‚ प्र‚थ‚मं संख्याभेद‚क‚थ‚न‚म् ॥‚{\tiny $_{lb}$}‚ विध‚श‚ब्दः प्र‚कार‚वाची प्र‚द‚र्शितः । न पुन‚र‚स्याय‚म‚भिप्रायः--विधा\edtext{}{\lemma{विधा}\Bfootnote{ध}}श‚ब्दो जातिवाचित्वात्प्र‚कार‚{\tiny $_{lb}$}‚वाची न भ‚व‚तीति । अनेकार्थ‚त्वात्त‚स्य प्र‚कार‚वाचिनोऽपि प्र‚योग‚स्य च‚त‚सृषु चैवंविधासु‚{\tiny $_{lb}$}‚ त‚त्त्वं प‚रिस‚माप्य‚ते \href{http://sarit.indology.info/?cref=nbh.1.1.1_p3}{न्याय‚भा० पृ० २} इत्यादाव‚नेन प्राय‚शो दृष्ट‚त्वात् ।
	\pend% ending standard par
      ‚{\tiny $_{lb}$}‚

	  \pstart \leavevmode% starting standard par
	न‚नु क‚थं नाम प्र‚माणादेव प्र‚व‚र्त्तेय नाप्र‚माणादिति प्र‚व‚र्त्तितुम‚नाः प्र‚माण‚स्य ल‚क्ष‚ण‚मेव‚{\tiny $_{lb}$}‚ जिज्ञास‚ते न स‚ङ्ख्याम् । त‚तो ल‚क्ष‚ण‚मेव व्युत्पाद्यं न स‚ङ्ख्या । त‚च्च ल‚क्ष‚णं य‚द्येक‚स्यैवास्ति‚{\tiny $_{lb}$}‚ त‚देक‚मेव प्र‚माण‚म् । अथ ब‚हूनां त‚दा प्र‚माण‚बाहुल्य‚म‚निवार्य‚मेव । अथ त‚ल्ल‚क्ष‚णं नैक‚स्यैवास्ति,‚{\tiny $_{lb}$}‚ नापि ब‚हूनाम् । त‚र्हि ल‚क्ष‚ण‚व्युत्प‚त्तौ साम‚र्थ्यात्संख्याविप्र‚तिप‚त्तिर्निराकृता भ‚व‚तीति किं पृथ‚क्‚{\tiny $_{lb}$}‚ संख्याविप्र‚तिप‚त्तिनिराक‚र‚णेनेत्याह--\textbf{संख्ये}ति । \textbf{चो} य‚स्माद‚र्थे । \textbf{द्वार}मुपायः ।
	\pend% ending standard par
      ‚{\tiny $_{lb}$}‚

	  \pstart \leavevmode% starting standard par
	केनाकारेण द‚र्शितो भ‚व‚तीत्याह--द्वे इति । \textbf{इति}ना व्य‚क्तिभेद‚स्य स्व‚रूप‚माह । व्य‚क्ति‚{\tiny $_{lb}$}‚भेदेनैव द‚र्शितेन किं प्र‚योज‚न‚मित्याह--\textbf{व्य‚क्तिभेदे चे}ति । \textbf{चो} य‚स्माद‚र्थे । य‚द्य‚द‚र्शितेऽपि व्य‚क्तिभेदे‚{\tiny $_{lb}$}‚ ल‚क्ष‚णाख्यानं श‚क्यं त‚थापि किं द‚र्शितेनेत्य‚न्व‚य‚मात्राद‚प्र‚तिप‚त्तेर्व्य‚तिरेक‚म‚पि द‚र्श‚यितुमाह—‚{\tiny $_{lb}$}‚\textbf{अप्र‚द‚र्शित} इति । \textbf{तुः} प्र‚द‚र्श‚न‚प‚क्षाद‚प्र‚द‚र्श‚न‚प‚क्ष‚स्य भेद‚माह । \textbf{स‚क‚ल‚व्य‚क्त्य‚नुयायी}ति ब्रुव‚तोऽयं‚{\tiny $_{lb}$}‚ भावः--व्य‚क्तिभेदानुप‚द‚र्श‚ने प्र‚तिव्य‚क्तिनिय‚त‚स्य ल‚क्ष‚ण‚स्याख्यातुम‚श‚क्य‚त्वात् । ल‚क्ष‚ण‚मुच्य‚मानं‚{\tiny $_{lb}$}‚ स‚क‚ल‚व्य‚क्त्य‚नुयायि त‚देकं व‚क्त‚व्य‚म् । न च त‚द् व‚क्तुं श‚क्य‚म‚स‚म्भ‚वादेवेति ।
	\pend% ending standard par
      ‚{\tiny $_{lb}$}‚

	  \pstart \leavevmode% starting standard par
	न‚नु किमुच्य‚ते स‚क‚ल‚व्य‚क्त्य‚नुयायि व‚क्तुम‚श‚क्य‚मिति याव‚ताऽविसंवादिज्ञानं प्र‚माण‚{\tiny $_{lb}$}‚मित्य‚स्ति ल‚क्ष‚ण‚मेकं सुव‚च‚म‚पीति । स‚त्य‚म् । केव‚लं ज्ञानानां य‚त्प्रातिस्विकं रूपं प्र‚वृत्तिकामानां‚{\tiny $_{lb}$}‚ प्र‚वृत्त्युप‚योगि, त‚दुप‚ल‚क्ष‚णं नास्तीत्य‚भिप्रायाद‚दोषः । य‚द्वा विप्र‚तिप‚त्तिनिराक‚र‚ण‚प्र‚व‚णं य‚त्सा‚{\tiny $_{lb}$}‚धार‚णं ल‚क्ष‚णं त‚न्नास्ति । य‚च्चाविसंवादित्वं ल‚क्ष‚णं न तेन विप्र‚तिप‚त्तिर्निराकृता भ‚व‚ति ।‚{\tiny $_{lb}$}‚ अन्य‚त्रापि प‚रैर‚विसंवादित्व‚स्येष्ट‚त्वादित्य‚नेनाभिप्रायेणोक्त‚म्--\textbf{न श‚क्य‚मेकं व‚क्तुमिति}विशेष‚प्र‚ति‚{\tiny $_{lb}$}‚षेध‚साम‚र्थ्यात् शेष‚विधिसिद्धौ च त‚थाभूत‚स्य साधार‚ण‚स्य स‚म्भ‚वादेव \textbf{स‚क‚ल}श‚ब्दोऽय‚मुप‚युक्त‚{\tiny $_{lb}$}‚कार्त्स्न्ये प्र‚व‚र्त्त\leavevmode\ledsidenote{\textenglish{16b/ms}}नीयः । तेनाय‚म‚र्थः--प्र‚द‚र्शित‚व्य‚क्तिभेदात्म‚क‚च‚तुर्विध‚प्र‚त्य‚क्षानुयायि क‚ल्प‚नाऽ‚{\tiny $_{lb}$}‚पोढाभ्रान्त‚त्व‚म्, प्र‚द‚र्शित‚व्य‚क्तिभेदात्म‚क‚द्विविधानुमानानुयायि त्रिरूप‚लिङ्ग‚ज‚त्वं श‚क्यं व‚क्तुमिति ।‚{\tiny $_{lb}$}‚ य‚दि नामैवं त‚तः किमित्याह--\textbf{त‚त} इति । य‚तोऽप्र‚द‚र्शिते न श‚क्य‚मेकं त‚थाविधं द‚र्श‚यितुं \textbf{त‚त}‚{\tiny $_{lb}$}‚स्त‚स्मात् । \textbf{ल‚क्ष‚ण‚क‚थ‚नाङ्ग}मिति प्र‚तिव्य‚क्तिनिय‚त‚ल‚क्ष‚ण‚क‚थ‚नाङ्ग‚मित्य‚व‚सेय‚म् ।
	\pend% ending standard par
      ‚{\tiny $_{lb}$}‚‚{\tiny $_{lb}$}‚\textsuperscript{\textenglish{37/dm}}‚{\tiny $_{lb}$}‚
	  \bigskip
	  \begingroup
	

	  \pstart \leavevmode% starting standard par
	किं पुन‚स्त‚द् द्वैविध्य‚मित्याह--
	\pend% ending standard par
       ‚{\tiny $_{lb}$}‚ 
	  \bigskip
	  \begingroup
	

	  \pstart \leavevmode% starting standard par
	प्र‚त्य‚क्ष‚म‚नुमान‚ञ्चेति\edtext{}{\lemma{ञ्चेति}\Bfootnote{नास्ति इति प‚दं \cite{dp-msB} \cite{dp-edP} \cite{dp-edH} \cite{dp-edE} \cite{dp-edN} प्र‚तिषु किन्तु विद्य‚ते त‚त्प‚दं \cite{dp-msC} \cite{dp-msD} प्र‚त‚योः ।}} ॥ ३ ॥
	\pend% ending standard par
      
	  \endgroup
	
	  \endgroup
	‚{\tiny $_{lb}$}‚

	  \pstart \leavevmode% starting standard par
	अथ त‚थाविधं ल‚क्ष‚ण‚मेकं व‚क्तुं न श‚क्य‚तां नाम । किम‚तः ? केव‚ल‚म‚द‚र्शितेऽपि संख्याभेदे‚{\tiny $_{lb}$}‚ ल‚क्ष‚ण‚भेदो द‚र्श‚यितुं श‚क्य‚ताम् । त‚त् क‚स्मात्त‚त्क‚थ‚नाङ्गं संख्याभेद‚क‚थ‚न‚मित्याश‚ङ्क्यान्व‚य‚मुखे‚{\tiny $_{lb}$}‚नोक्त‚म‚र्थं द्र‚ढ‚यितुं व्य‚तिरेक‚मुखेणाह--\textbf{ल‚क्ष‚णेति । ल‚क्ष‚ण‚भेद‚स्य} प्र‚तिव्य‚क्तिभिन्न‚स्य दृष्ट‚स्य ।‚{\tiny $_{lb}$}‚ क‚दा द‚र्श‚यितुम‚श‚क्य‚त्वं य‚त एवं भ‚व‚तीत्याकाङ्क्षायाम्--\textbf{अप्र‚द‚र्शित} इति प‚श्चाद् योज‚नीय‚म् ।‚{\tiny $_{lb}$}‚ तुर‚व‚धार‚य‚ति, प्र‚द‚र्शिताद् वा भिन‚त्ति ।
	\pend% ending standard par
      ‚{\tiny $_{lb}$}‚

	  \pstart \leavevmode% starting standard par
	न‚नु च न संख्याभेद एव व्य‚क्तिभेदः । त‚त् क‚थं \textbf{व्य‚क्तिभेदात्म‚क} इत्युच्य‚त इति चेत् ।‚{\tiny $_{lb}$}‚ व‚स्तुतः संख्येयाद‚न्य‚स्याः संख्याया वास्त‚व्या अभावेन संख्यासंख्येय‚योरेक‚त्व‚विव‚क्ष‚या चैव‚मुच्य‚ते ।‚{\tiny $_{lb}$}‚ पूर्वं तु क‚ल्प‚नानिर्मितात्म‚नां\edtext{}{\lemma{नां}\Bfootnote{ना}}भिन्नाम‚व‚ल‚म्ब्य \textbf{संख्याप्र‚द‚र्श‚न‚द्वारेणे}त्युक्त‚म् । संख्यासंख्येय‚यो‚{\tiny $_{lb}$}‚रेक‚त्व‚विव‚क्ष‚यैव च \textbf{संख्याविप्र‚तिप‚त्तिं निराक‚र्त्तुमाहे}त्युक्त‚म‚व‚सेय‚म् । प‚र‚मार्थ‚त‚स्तु \textbf{प्र‚त्य‚क्ष‚म‚नुमानं}‚{\tiny $_{lb}$}‚ चेत्युक्तेः संख्यासंख्येय‚विप्र‚तिप‚त्तिरेव निराकृतेति । अथ‚वा व‚स्तु\edtext{}{\lemma{स्तु}\Bfootnote{व्य‚क्ति}} भेद‚स्त‚दात्मा त‚था‚{\tiny $_{lb}$}‚विधः क‚ल्प‚नाशिल्पिनिर्मितो य‚स्य संख्याभेद‚स्य स त‚थोक्तः । प‚र‚मार्थ‚तो भिन्ने हि व‚स्तुनि‚{\tiny $_{lb}$}‚ संख्याभेदः क‚ल्प्य‚त इति वास्त‚व‚रूपानुवाद‚मिद‚मुक्त‚मिति किम‚व‚द्य‚म् ? स‚र्वेण चानेन नै\edtext{}{\lemma{नै}\Bfootnote{चानेनै}}‚{\tiny $_{lb}$}‚त‚दुक्त‚म्--प्र‚वृत्तिकामानामुप‚योगित्वाल्ल‚क्ष‚ण‚मेव व‚क्तुं युक्तं त‚देत‚त्त्व‚न्य‚था न श‚क्य‚माख्यातु‚{\tiny $_{lb}$}‚मिति संख्याभेद‚प्र‚द‚र्श‚न‚मिति ।
	\pend% ending standard par
      ‚{\tiny $_{lb}$}‚

	  \pstart \leavevmode% starting standard par
	न‚न्व‚स‚त्य‚पि \textbf{द्विविध}मिति संख्याभेद‚प्र‚द‚र्श‚ने त‚त् स‚म्य‚ग्ज्ञानं \textbf{प्र‚त्य‚क्ष‚म‚नुमानं} चेत्युक्तेऽपि‚{\tiny $_{lb}$}‚ व्य‚क्तिभेदो द‚र्शितो भ‚व‚त्येव । स‚ति चैवं ल‚क्ष‚ण‚भेदाख्यान‚म‚पि सुक‚र‚मिति कृतं \textbf{द्विविध}श‚ब्देनेति‚{\tiny $_{lb}$}‚ चेत् । सिद्धे स‚तीदं व‚च‚नं द्विविध‚मेवेति निय‚मार्थ‚मिति ब्रूमः । इत‚र‚थेह ताव‚देताव‚{\tiny $_{lb}$}‚द्व्युत्पाद्य‚त‚या प्र‚स्तुत‚म‚न्य‚त्र पुन‚र‚न्य‚द‚प्य‚स्ति व्युत्पाद्यं स‚म्य‚ग्ज्ञान‚मित्याश‚ङ्काह‚त्य न निराकृता‚{\tiny $_{lb}$}‚ स्यादिति । य‚त्पुन‚रुक्त‚म्--त‚र्हि ल‚क्ष‚ण‚व्युत्प‚त्त्यैव संख्याविप्र‚तिप‚त्तिर्निराकृता भ‚व‚तीति त‚द‚त्य‚न्त‚{\tiny $_{lb}$}‚म‚स‚ङ्ग‚त‚म् । य‚तो य‚दि विशेष‚ल‚क्ष‚ण‚म‚भिप्रेत्येद‚मुच्य‚ते; त‚दा व्य‚क्तिभेदानुप‚द‚र्श‚ने प्र‚तिव्य‚क्ति‚{\tiny $_{lb}$}‚निय‚तं त‚देवाश‚क्यं\edtext{}{\lemma{क्यं}\Bfootnote{य}}क‚थ‚न‚मित्युक्त‚म् । न चोक्त‚मिति त‚था क‚र्त्तुमीष्टे । य‚त‚स्त‚दीदृशं‚{\tiny $_{lb}$}‚ प्र‚त्य‚क्ष‚म‚नुमानं चेत्येत‚त्क‚र्त्तुं श‚क्नोति । न त्वाभ्याम‚न्य‚न्न स‚म्य‚ग्ज्ञान‚मिति । अथ सामान्य‚ल‚क्ष‚ण‚म् ।‚{\tiny $_{lb}$}‚ त‚द‚पि त‚थाविधं साधार‚णं व‚क्तुम‚श‚क्य‚मिति केन त‚था क्रिय‚ताम् ? न च तेनोक्तेनापि तादृशी‚{\tiny $_{lb}$}‚ विप्र‚तिप‚त्तिर्निराक्रिय‚त इत्युक्त‚म् । अथापि स्यात्--द्विविध‚मित्युक्तेऽपि क‚थ \leavevmode\ledsidenote{\textenglish{17a/ms}} माभ्याम‚न्य‚{\tiny $_{lb}$}‚स्यास‚म्य‚ग्ज्ञान‚त्व‚वि\edtext{}{\lemma{वि}\Bfootnote{त्वाधि}}ग‚मः? उच्य‚ते । ये ताव‚दाचार्य‚प्र‚माण‚का व्युत्पित्स‚व‚स्ते त‚द्व‚च‚न‚{\tiny $_{lb}$}‚मात्रेणैव विव‚क्षित‚म‚र्थं बोध्य‚न्ते । ये ते युक्त्य‚नुसारिण‚स्तेभ्योऽपि प्र‚क‚र‚णान्त‚रेषूप‚प‚त्ति‚{\tiny $_{lb}$}‚रुदाहृता । प्राज्ञ‚ज‚नाधिकारेण चास्य प्र‚क‚र‚ण‚स्य प्रार‚म्भात्स्व‚य‚मेव तैरुप‚प‚त्तिरुक्तेति न संख्या‚{\tiny $_{lb}$}‚व‚च‚न‚म‚नुपादेय‚मिति । अस्तु ल‚क्ष‚ण‚निर्देशाङ्ग‚म्, आदौ तु क‚स्मादुच्य‚त इत्याह--\textbf{ल‚क्ष‚ण} इति ।‚{\tiny $_{lb}$}‚ च‚स्तुल्योपाय‚त्वं स‚मुच्चिनोति ।
	\pend% ending standard par
      ‚{\tiny $_{lb}$}‚

	  \pstart \leavevmode% starting standard par
	य‚थाक‚थ‚ञ्चिद् द्वैविध्य‚स‚म्भ‚वे पृच्छ‚ति--\textbf{किं पुन‚रि}ति । \textbf{किमि}ति सामान्य‚तः, \textbf{पुन}रिति‚{\tiny $_{lb}$}‚ \leavevmode\ledsidenote{\textenglish{38/dm}}‚{\tiny $_{lb}$}‚ 
	  
	प्र‚त्य‚क्ष‚मिति । प्र‚तिग‚त‚माश्रित‚म् अक्ष‚म् । अत्याद‚यः क्रान्ताद्य‚र्थे द्वितीय‚या ‚{\tiny $_{lb}$}‚ \href{http://sarit.indology.info/?cref=htu.90}{वा० २. २. १८.} इति स‚मासः । प्राप्ताप‚न्नाल‚ङ्ग‚ति\edtext{}{\lemma{ति}\Bfootnote{उप‚स‚र्ग--\cite{dp-msD-n}}} स‚मासेषु प‚र‚व‚ल्लिङ्ग‚प्र‚तिषेधाद्‚{\tiny $_{lb}$}‚ अभिधेय‚व‚ल्लिङ्गे स‚ति स‚र्व‚लिङ्गः प्र‚त्य‚क्ष‚श‚ब्दः सिद्धः । \edtext{\textsuperscript{*}}{\lemma{*}\Bfootnote{न‚नु चास्यां व्युत्प‚त्तौ इन्द्रिय‚ज्ञान‚स्यैव प्र‚त्य‚क्ष‚श‚ब्द‚वाच्य‚ता स्याद् न योगिज्ञानादे‚{\tiny $_{lb}$}‚रित्याह--\cite{dp-msD-n}}}अक्षाश्रित‚त्वं च व्युत्प‚त्तिर्निमित्तं‚{\tiny $_{lb}$}‚ श‚ब्द‚स्य । न तु प्र‚वृत्तिनिमित्त‚म् । \edtext{\textsuperscript{*}}{\lemma{*}\Bfootnote{इन्द्रिय‚ज्ञानेऽक्षाश्रित‚त्व‚म‚र्थ‚साक्षात्कारित्वं च स‚म‚वेत‚म्--\cite{dp-msD-n}}}अनेन \edtext{}{\lemma{अनेन}\Bfootnote{अनेन त्व‚ल‚क्षाश्रि० \cite{dp-msA}}}त्व‚क्षाश्रित‚त्वेनैका \edtext{}{\lemma{त्वेनैका}\Bfootnote{एक‚स्मिन्निन्द्रिय‚ज्ञाने--\cite{dp-msD-n}}}र्थ‚स‚म‚वेत \edtext{}{\lemma{वेत}\Bfootnote{एकार्थ‚स‚म्ब‚द्ध‚म्--\cite{dp-msD-n}}}म‚र्थ‚साक्षा‚{\tiny $_{lb}$}‚त्कारित्वं \edtext{}{\lemma{त्कारित्वं}\Bfootnote{ल‚भ्य‚ते--\cite{dp-msB} \cite{dp-edN}}}ल‚क्ष्य‚ते । त‚देव श‚ब्द‚स्य प्र‚वृत्तिनिमित्त‚म् । त‚त‚श्चा य‚त्किञ्चिद‚र्थ‚स्य साक्षात्कारि‚{\tiny $_{lb}$}‚ ज्ञानं त‚त् प्र‚त्य‚क्ष‚मुच्य‚ते ।‚{\tiny $_{lb}$}‚ विशेष‚तः । अभिम‚त‚माह--\textbf{प्र‚त्य‚क्ष‚म्} इति । \textbf{प्र‚त्य‚क्ष}मित्य‚त्र कः स‚मासः, केन च सूत्रेणेत्याश‚ङ्क्य‚{\tiny $_{lb}$}‚ सुख‚प्र‚तिप‚त्त्य‚र्थं विप्र‚तिप‚त्तिनिराक‚र‚णार्थं चाह \textbf{प्र‚तिग‚त‚म्} इत्यादि । \textbf{अक्ष}मिन्द्रियं \textbf{ग‚त}माश्रितं‚{\tiny $_{lb}$}‚ ज‚न्य‚त‚या, न त्वाधेय‚भावेन । एवं च विगृह्ण‚न् अक्ष‚म‚क्षं प्र‚ति व‚र्त्त‚त इति व्युत्प‚त्तिप्र‚कारेणाव्य‚यी‚{\tiny $_{lb}$}‚भावं निर‚स्य‚ति । \textbf{स‚मास} एकार्थीभावः । स चायं स‚मास‚स्त‚त्पुरुष‚संज्ञ‚को ज्ञात‚व्यः । केन‚{\tiny $_{lb}$}‚ सूत्रेणाय‚मित्याह--\textbf{अत्याद‚य} इति । आहोपुरुषिक‚याऽयं प्र‚कार‚स्त्व‚या द‚र्शितो न त्व‚स्य क‚श्चिद्‚{\tiny $_{lb}$}‚ गुणोऽस्तीत्याह--\textbf{स‚र्व‚लिङ्ग} इति । अव्य‚यीभाव‚प‚क्षे प्र‚त्य‚क्षो वृक्षः प्र‚त्य‚क्षा मृगाश्चेति न स्यात् ।‚{\tiny $_{lb}$}‚ प्र‚त्य‚क्ष‚स्येति श्रुतिश्च न स्यादित्य‚भिप्रायः । क‚थं पुनः स‚र्व‚लिङ्ग‚ताऽस्येत्याह--\textbf{अभिधेय‚व‚दि}ति ।‚{\tiny $_{lb}$}‚ अभिधेय‚स्येवाभिधेय‚व‚त् । अभिधेय‚व‚ल्लिङ्गं य‚स्य त‚त्त‚था त‚स्मिंश्च स‚त्य‚यं प्र‚त्य‚क्ष‚श‚ब्दः‚{\tiny $_{lb}$}‚ स‚र्व‚लिङ्गः ।
	\pend% ending standard par
      ‚{\tiny $_{lb}$}‚

	  \pstart \leavevmode% starting standard par
	\hphantom{.}न‚नु च प‚र‚व‚ल्लिङ्गं द्व‚न्द्व‚त‚त्पुरुष‚योः \href{http://sarit.indology.info/?cref=Pā.2.4.26}{पाणिनि--२. ४. २६} इति स‚म‚स्तेन प्र‚त्य‚क्ष‚श‚ब्देन‚{\tiny $_{lb}$}‚ प्र‚तिश‚ब्दो\edtext{}{\lemma{ब्दो}\Bfootnote{ब्दा}}पेक्ष‚या प‚र‚स्थान‚स्थित‚स्याक्ष‚श‚ब्द‚स्य लिङ्ग‚प‚रिग्र‚हात्क‚थ‚म‚स्याभिधेय‚व‚ल्लिङ्ग‚ते‚{\tiny $_{lb}$}‚त्याह--\textbf{प्राप्तेति} ।
	\pend% ending standard par
      ‚{\tiny $_{lb}$}‚

	  \pstart \leavevmode% starting standard par
	न‚नु केनास्य प‚र‚व‚ल्लिङ्ग‚निषेधः ? न ताव‚त् प्राप्ताप‚न्नाल‚ङ्ग‚तिस‚मासेन तेषां‚{\tiny $_{lb}$}‚ स्व‚रूप‚ग्र‚ह‚णात्त‚त्र । अत्र च त‚द‚भावात् । ग‚तिस‚मासादिति चेत् । त‚द‚प्य‚स‚त्, उप‚स‚र्गाः क्रियायोगे ‚{\tiny $_{lb}$}‚ \href{http://sarit.indology.info/?cref=Pā.1.4.59}{पाणिनि--१. ४. ५९} इत्य‚तः क्रियायोगे व‚र्त्त‚माने ग‚तिश्च \href{http://sarit.indology.info/?cref=Pā.1.4.60}{पाणिनि--१. ४. ६०} इत्य‚नेन‚{\tiny $_{lb}$}‚ क्रियायोग एव ग‚तिसंज्ञाविधानात् । न चायं प्र‚तिश‚ब्दः क्रियायोगी । न च ग‚म‚न‚क्रिया‚{\tiny $_{lb}$}‚योगोऽत्रास्तीति वाच्य‚म् । प्र‚तिश‚ब्द‚स्यैव त‚त्रार्थे व‚र्त्त‚माना\edtext{}{\lemma{माना}\Bfootnote{न‚त्वा}}त् नायं दोषः,‚{\tiny $_{lb}$}‚ ग‚तिग्र‚ह‚णेन त‚त्र येषां ग‚तिसंज्ञा दृष्टा तेषां ग्र‚ह‚णात् । प्र‚त्यादीनां सा दृष्टेति तेषाम‚पि‚{\tiny $_{lb}$}‚ त‚थात्वेन स‚ङ्ग्र‚ह इति ।
	\pend% ending standard par
      ‚{\tiny $_{lb}$}‚

	  \pstart \leavevmode% starting standard par
	य‚द्येवं स‚मास‚स्त‚र्हि क‚थ\textbf{माचार्य‚दिग्नागेन} अक्ष‚म‚क्षं प्र‚ति व‚र्त्त‚त इति प्र‚त्य‚क्ष‚म् ‚{\tiny $_{lb}$}‚ \add{न्याय‚मुख} इत्युक्त‚म् ? त‚द‚र्थ‚मात्रं क‚थित‚मित्य‚दोषः ।
	\pend% ending standard par
      ‚{\tiny $_{lb}$}‚‚{\tiny $_{lb}$}‚\textsuperscript{\textenglish{39/dm}}‚{\tiny $_{lb}$}‚
	  \bigskip
	  \begingroup
	

	  \pstart \leavevmode% starting standard par
	य‚दि त्व‚क्षाश्रित‚त्व‚मेव प्र‚वृत्तिनिमित्तं स्याद् इन्द्रिय‚विज्ञान‚मेव\edtext{}{\lemma{मेव}\Bfootnote{इन्द्रिय‚ज्ञान० \cite{dp-msA} \cite{dp-edP} \cite{dp-edH} \cite{dp-edE}}} प्र‚त्य‚क्ष‚मुच्येत, न‚{\tiny $_{lb}$}‚ मान‚सादि । य‚था ग‚च्छ‚तीति गौः इति ग‚म‚न‚क्रियायां व्युत्पादितोऽपि गोश‚ब्दो ग‚म‚न‚क्रियो‚{\tiny $_{lb}$}‚प‚ल‚क्षित‚मेकार्थ‚स‚म‚वेतं गोत्वं\edtext{}{\lemma{गोत्वं}\Bfootnote{जातिः--\cite{dp-msD-n}}} प्र‚वृत्तिनिमित्तीक‚रोति । त‚था च ग‚च्छ‚त्य‚ग‚च्छ‚ति च ग‚वि‚{\tiny $_{lb}$}‚ गोश‚ब्दः सिद्धो भ‚व‚ति ।
	\pend% ending standard par
       ‚{\tiny $_{lb}$}‚ 

	  \pstart \leavevmode% starting standard par
	मीय‚तेऽनेनेति मान‚म् । क‚र‚ण‚साध‚नेन मान‚श‚ब्देन\edtext{}{\lemma{ब्देन}\Bfootnote{सामान्य‚ल‚क्ष‚ण०--\cite{dp-msD-n} अर्थ‚सारूप्य०--\cite{dp-msD-n}}} सारूप्य‚ल‚क्ष‚णं प्र‚माण‚म‚भिधीय‚ते ।‚{\tiny $_{lb}$}‚ लिङ्ग‚ग्र‚ह‚ण स‚म्ब‚न्ध‚स्म‚र‚ण‚स्य प‚श्चात् मान‚म् अनुमान‚म् । गृहीते\edtext{}{\lemma{गृहीते}\Bfootnote{गृहीत‚प‚क्ष० \cite{dp-msB}}} प‚क्ष‚ध‚र्मे स्मृते च‚{\tiny $_{lb}$}‚ साध्य‚साध‚न‚स‚म्ब‚न्धेऽनुमानं प्र‚व‚र्त्त‚त इति प‚श्चात्काल‚भाव्युच्य‚ते ।
	\pend% ending standard par
      
	  \endgroup
	‚{\tiny $_{lb}$}‚

	  \pstart \leavevmode% starting standard par
	अथ प्र‚तिग‚त‚माश्रित‚म‚क्ष‚मित्य‚स्याम‚पि व्युत्प‚त्तौ मान‚स-स्व‚संवेद‚न-योगिप्र‚त्य‚क्षाणां न‚{\tiny $_{lb}$}‚ स्यात्प्र‚त्य‚क्ष‚श‚ब्द‚वाच्य‚तेत्याह--\textbf{अक्षाश्रित‚त्व‚म्} इति । \textbf{चो} य‚स्मात् । प्र‚कृत्यादिविभागेन श‚ब्द‚स्य‚{\tiny $_{lb}$}‚ निष्प‚त्ति\textbf{र्व्युत्प‚त्तिः । प्र‚वृत्ति}र‚र्थाभिधान‚म् \textbf{श‚ब्द‚स्ये}ति प्र‚कृत‚स्य प्र‚त्य‚क्ष‚श‚ब्द‚स्य । तुर‚व‚धार‚णे । किं‚{\tiny $_{lb}$}‚ त‚र्हि प्र‚वृत्तिनिमित्त‚मित्याह--\textbf{अनेन}--इति । तुना व्युत्प‚त्तिनिमित्ताद‚स्य भेदं द\leavevmode\ledsidenote{\textenglish{17b/ms}}र्श‚य‚ति ।‚{\tiny $_{lb}$}‚ य‚त्त‚दोर्नित्य‚म‚भिस‚म्ब‚न्धात् य‚द‚क्षाश्रित‚त्वेनार्थ‚साक्षात्कारित्व‚म‚र्थाप‚रोक्षीक‚र‚ण‚मुप‚ल‚क्ष्य‚ते त‚देव तु‚{\tiny $_{lb}$}‚ प्र‚त्य‚क्ष‚श‚ब्द‚स्य प्र‚वृत्तिनिमित्त‚म् । क‚थं पुन‚र‚न्येनास‚म्ब‚द्धेनान्य‚स्योप‚ल‚क्ष‚ण‚म् ? मा भूदि\edtext{}{\lemma{भूदि}\Bfootnote{द}}तिप्र‚स‚क्ति‚{\tiny $_{lb}$}‚रित्याह--\textbf{एकार्थ‚स‚म‚वेत‚म्} इति हेतुभावेन विशेष‚णात् । य‚त एकार्थ‚स‚म‚वेतं त‚तो ल‚क्ष्य‚त‚{\tiny $_{lb}$}‚ इत्य‚र्थः । एक‚स्मिन्न‚र्थे ज्ञान‚ल‚क्ष‚णेऽर्थ‚साक्षात्कारित्व‚म‚क्षाश्रित‚त्वेन स‚मं स‚म‚वेत‚म् । य‚द्य‚पि‚{\tiny $_{lb}$}‚ प‚र‚मार्थ‚तः स‚म‚वाय‚स‚म‚वायिनौ न स्त‚स्त‚थापि त‚द्व्यावृत्तिनिब‚न्ध‚न‚योस्त‚थाभूत‚व‚स्त्वाश्र‚येण‚{\tiny $_{lb}$}‚ क‚ल्पित‚योर‚क्षाश्रित‚त्वार्थ‚साक्षात्कारित्व‚योरेकार्थ‚स‚म‚वाय‚स्य च क‚ल्पित‚स्य स‚म्भ‚वादेव‚म‚भिधाने‚{\tiny $_{lb}$}‚ को विरोधः ? प‚र‚प्र‚सिद्ध्या वा एव‚मुक्त‚मिति का क्ष‚तिः ? एव‚म‚पि क‚थं पूर्व‚दोषातिवृत्ति‚{\tiny $_{lb}$}‚रित्याह--\textbf{त‚त‚श्चे}ति । साक्षात्कारित्व‚स्य प्र‚वृत्तिनिमित्त‚त्वात् । \textbf{च}श‚ब्द एव त‚स्मादिद‚मुच्य‚त इत्य‚र्थं‚{\tiny $_{lb}$}‚ द्योत‚य‚ति ।
	\pend% ending standard par
      ‚{\tiny $_{lb}$}‚

	  \pstart \leavevmode% starting standard par
	अक्षाश्रित‚त्वे प्र‚वृत्तिनिमित्ते को दोष इति पार्श्व‚स्थ‚स्याश्रुत‚चोद‚क‚वाक्य‚स्य व‚च‚न‚{\tiny $_{lb}$}‚माश‚ङ्क्याह--\textbf{य‚दि}--इति । \textbf{तु}ना भेद‚व‚देत‚द्द‚र्श‚य‚ति ।
	\pend% ending standard par
      ‚{\tiny $_{lb}$}‚

	  \pstart \leavevmode% starting standard par
	अथाभिधीय‚ते--य‚द‚र्थ‚साक्षात्क‚र‚ण‚म‚क्षाश्रित‚त्वेन स‚म‚वेतं त‚देवानेनोप‚ल‚क्ष‚णीय‚मिति त‚द‚{\tiny $_{lb}$}‚व‚स्थे\edtext{}{\lemma{स्थे}\Bfootnote{स्थो}}मान‚सादेर‚स‚ङ्ग्र‚ह इति । न । अक्ष‚ज‚त्व‚स्योप‚ल‚क्ष‚ण‚त्वेनाप्राधान्याद‚र्थ‚साक्षात्कारित्व‚{\tiny $_{lb}$}‚मुप‚ल‚क्ष्य‚माणं प्र‚धान‚मुप‚ल‚क्ष्यैव निवृत्तेर‚दोष एव । य‚था काकेभ्यो द‚धि र‚क्ष्य‚तामित्य‚त्र ।‚{\tiny $_{lb}$}‚ अन्य‚थाऽत्रापि काक‚त्वेनैकार्थ‚स‚म‚वेत‚स्योप‚धात‚क‚त्व‚स्योप‚ल‚क्ष‚णात्, श्वादिभ्यो द‚धिर‚क्षा न स्यात् ।‚{\tiny $_{lb}$}‚ अथ य‚देवोप‚धात‚कं\edtext{}{\lemma{कं}\Bfootnote{क‚त्वं}} काकेषु स‚म‚वेतं त‚देवान्य‚त्रापि । त‚च्चोप‚ल‚क्षित‚मिति त‚द‚न्य‚तोऽपि‚{\tiny $_{lb}$}‚ द‚धिर‚क्षोच्य‚ते । अर्थ‚साक्षात्कारित्वेऽपि स‚मान‚मिद‚मिति क‚थं न मान‚सादेः प्र‚त्य‚क्ष‚श‚ब्दाभि‚{\tiny $_{lb}$}‚लाप्य‚त्व‚मिति । किं दृष्ट‚मिदं य‚द‚न्य‚त्र व्युत्पादितोऽपि श‚ब्दोऽन्य‚त्र त‚त्प्र‚वृत्तिविष‚यीक‚रोती‚{\tiny $_{lb}$}‚त्याह--\textbf{य‚थे}ति । सुग‚म‚मेत‚त्, केव‚ल\textbf{म‚पि}र‚व‚धार‚णे \textbf{गोत्व‚म} \edtext{\textsuperscript{*}}{\lemma{*}\Bfootnote{मित्य}} स्मात्प‚रो द्र‚ष्ट‚व्यः ।
	\pend% ending standard par
      ‚{\tiny $_{lb}$}‚

	  \pstart \leavevmode% starting standard par
	प्र‚त्य‚क्ष‚मूल‚त्वाद‚नुमान‚स्यादौ प्र‚त्य‚क्ष‚मुपात्तं प्र‚त्येय‚म् । य‚त्र हि निमित्तान्त‚रं द‚र्श‚यितु‚{\tiny $_{lb}$}‚ ‚{\tiny $_{lb}$}‚ \leavevmode\ledsidenote{\textenglish{40/dm}}‚{\tiny $_{lb}$}‚ 
	  
	च‚कारः प्र‚त्य‚क्षानुमान‚योस्तुल्य‚ब‚ल‚त्वं स‚मुच्चिनोति । य‚थार्थाविनाभावित्वाद‚र्थं प्राप‚य‚त्‚{\tiny $_{lb}$}‚ प्र‚त्य‚क्षं प्र‚माण‚म्, त‚द्व‚द‚र्थाविनाभावित्वाद् अनुमान‚म‚पि प‚रिच्छिन्न‚म‚र्थं प्राय‚प‚त् प्र‚माण‚मिति ॥ ‚{\tiny $_{lb}$}‚ 
	  
	त‚त्र प्र‚त्य‚क्षं क‚ल्प‚नाऽपोढ‚म‚भ्रान्त‚म् ॥ ४ ॥‚{\tiny $_{lb}$}‚ 
	  
	त‚त्रेति स‚प्त‚म्य‚र्थे व‚र्त्त‚मानो निर्धार‚णे व‚र्त्त‚ते । त‚तोयं वाक्यार्थः--त‚त्र त‚योः प्र‚त्य‚क्षा‚{\tiny $_{lb}$}‚न‚मान‚योरिति स‚मुदाय‚निर्देशः । प्र‚त्य‚क्ष‚म् इत्येक‚देश‚निर्देशः\edtext{}{\lemma{निर्देशः}\Bfootnote{इत्येक‚देशः \cite{dp-msA} \cite{dp-msC} \cite{dp-edP} \cite{dp-edE} \cite{dp-edH}}} । त‚त्र स‚मुदायात् प्र‚त्य‚क्ष‚त्व‚{\tiny $_{lb}$}‚जात्यैक‚देश‚स्य पृथ‚क्क‚र‚णं निर्धार‚ण‚म् । त‚त्र प्र‚त्य‚क्ष‚म‚नूद्य\edtext{}{\lemma{नूद्य}\Bfootnote{प्र‚त्य‚क्ष‚त्व‚म‚नू० \cite{dp-msB} \cite{dp-msD} \cite{dp-edP} \cite{dp-edE} \cite{dp-edH} \cite{dp-edN}}} क‚ल्प‚नाऽपोढ‚त्व‚म्, अभ्रान्त‚त्वं च\edtext{}{\lemma{च}\Bfootnote{त्वं विधी० \cite{dp-msB}}}‚{\tiny $_{lb}$}‚ विधीय‚ते । य‚त्\edtext{}{\lemma{त्}\Bfootnote{न‚नु प्र‚त्य‚क्ष‚स्याद्याप्य‚सिद्ध‚त्वात् क‚थ‚म‚नूद्य‚त्वं स‚म्भ‚व‚तीत्याह--\cite{dp-msD-n}}} त‚त् भ‚व‚ताम् अस्माकं चार्थेषु साक्षात्कारि ज्ञानं प्र‚सिद्धं त‚त् क‚ल्प‚नाऽपोढा‚{\tiny $_{lb}$}‚भ्रान्त‚त्व‚युक्तं द्र‚ष्ट‚व्य‚म् ।‚{\tiny $_{lb}$}‚ म‚श‚क्यं त‚त्र क्र‚म‚प्र‚वृत्तित्वाद् वाचः, क्र‚म‚स्य न्याय‚प्राप्त‚त्वादित्युच्य‚ते, न तु स‚त्य‚पि निमित्तान्त‚रे‚{\tiny $_{lb}$}‚ य‚थोद्देश‚मेव व्याख्यानं युक्त‚मिति ।
	\pend% ending standard par
      ‚{\tiny $_{lb}$}‚

	  \pstart \leavevmode% starting standard par
	प्र‚त्य‚क्ष‚प‚दं व्याख्यायाधुनाऽनुमान‚प‚दं व्याचिख्यासुः \textbf{मान}श‚ब्द‚स्य ताव‚द‚र्थ‚माच‚ष्टे \textbf{मीय‚त}‚{\tiny $_{lb}$}‚ इति । एवं व्युत्पादितेनानेन श‚ब्देन किं प्र‚तिपाद्य‚त इति ? \textbf{क‚र‚णे}ति । \textbf{क‚र‚ण}स्यानुमान‚{\tiny $_{lb}$}‚ल‚क्ष‚ण‚क्रियासिद्धौ साध‚क‚त‚म‚स्य \textbf{साध‚नेन} वाच‚केन । \textbf{सारूप्य\edtext{}{\lemma{सारूप्य}\Bfootnote{प्यं}}ल‚क्ष‚णं} स्व‚भावो य‚स्य त‚त्त‚था ।‚{\tiny $_{lb}$}‚ हेतुभावेन चैत‚द् विशेष‚ण‚म् ।
	\pend% ending standard par
      ‚{\tiny $_{lb}$}‚

	  \pstart \leavevmode% starting standard par
	\hphantom{.}य‚द्येवं क‚थं त‚र्हि क्व‚चित् त्रिरूपाल्लिङ्गाद् य‚द‚नुमेये ज्ञानं त‚द‚नुमान‚म् इति ? सारूप्यो‚{\tiny $_{lb}$}‚प‚ल‚क्षितं ज्ञान‚मेव त‚थाऽभिधास्य‚त इति को विरोधः ? एवं ताव‚न्मान‚म् । अनुमानं तु‚{\tiny $_{lb}$}‚ क‚थ‚मित्याह--\textbf{प‚श्चाद्} इति । अनेन प‚श्चाद‚र्थ‚वृत्तेनानुश‚ब्द‚स्य मान‚श‚ब्देन स‚ह ग‚तिप्राद‚य ‚{\tiny $_{lb}$}‚ इत्य‚नेन त‚त्पुरुष‚स‚मासं द‚र्श‚य‚ति । न च मान‚स्य प‚श्चादिति \leavevmode\ledsidenote{\textenglish{18a/ms}} विव‚क्षित‚म्, येनानुर‚थादि‚{\tiny $_{lb}$}‚व‚द‚व्य‚यीभावो भ‚वेत् । अत्र हि मान‚मेव प‚क्ष‚ध‚र्म‚ग्र‚ह‚णादेः प‚श्चाद्भावि विव‚क्षित‚म् । न मान‚स्य‚{\tiny $_{lb}$}‚ प‚श्चाद्भावि किञ्चिदिति । अव्य‚यीभाव‚प‚क्षे तु न केव‚लं विव‚क्षितार्थ‚क्ष‚तिः, किन्त्व‚नु मान‚स्येति‚{\tiny $_{lb}$}‚ ष‚ष्ठी च न श्रूयेत । क‚स्य प‚श्चादित्याह--\textbf{लिङ्गेति} । ष‚ष्ठ्य‚त‚स‚र्थ‚प्र‚त्य‚येन \href{http://sarit.indology.info/?cref=Pā.2.3.30}{पाणिनि--२. ३. ३०}‚{\tiny $_{lb}$}‚ इत्य‚नेन प‚श्चात् श‚ब्द‚योगे ष‚ष्ठीय‚म् । क‚थ‚मेत‚द् द्व‚य‚स्य प‚श्चाद् भाव्य‚नुमान‚मित्याह--\textbf{गृहीत}‚{\tiny $_{lb}$}‚ इति । \textbf{तुल्य‚ब‚ल‚त्वं} तुल्य‚साम‚र्थ्य‚म् । प्रामाण्यासाद‚नायेति प्र‚क‚र‚ण‚व‚शात् । \textbf{य‚थे}त्यादिनैत‚देव‚{\tiny $_{lb}$}‚ स‚म‚र्थ‚य‚ते । अनुमान‚स्य त्व‚र्थाविनाभावित्वं पार‚म्प‚र्येण द्र‚ष्ट‚व्य‚म् । न चैवं प्राप‚णे प्रामाण्याव‚हे‚{\tiny $_{lb}$}‚ क‚श्चिद् विशेषः । \textbf{प‚रिच्छिन्न‚मि}त्य‚ध्य‚व‚सित‚म् । एवं ब्रुव‚तः प्र‚त्य‚क्ष‚म‚प्य‚ध्य‚व‚सित‚मेव स‚न्तानं‚{\tiny $_{lb}$}‚ प्राप‚य‚त् प्र‚माण‚म् । इद‚म‚पि त‚थेति क‚थ‚म‚स्य न त‚थात्व‚मिति भावः । तुल्य‚ब‚ल‚प्र‚द‚र्श‚नेऽप्य‚यं‚{\tiny $_{lb}$}‚ भावः । प्र‚त्य‚क्ष‚स्यापि य‚द‚र्थाविनाभावित्वं त‚त् त‚दुत्प‚त्तिनिमित्त‚क‚मेव । त‚च्चानुमान‚स्यापि‚{\tiny $_{lb}$}‚ स‚मान‚मिति अर्थाव्य‚भिचारेणापि निमित्तिना स‚मानेन भाव्य‚मिति ।
	\pend% ending standard par
      ‚{\tiny $_{lb}$}‚‚{\tiny $_{lb}$}‚\textsuperscript{\textenglish{41/dm}}‚{\tiny $_{lb}$}‚
	  \bigskip
	  \begingroup
	

	  \pstart \leavevmode% starting standard par
	न\edtext{}{\lemma{न}\Bfootnote{न‚नु च त‚योर‚प्र‚सिद्ध‚त्वात् प्र‚त्य‚क्ष‚स्याप्य‚प्र‚सिद्धिरेव, प्र‚त्य‚क्ष‚स्यैत‚त्स्व‚भाव‚त्वादित्याह--\cite{dp-msD-n}}} चैत‚न्म‚न्त‚व्य‚म्--क‚ल्प‚नाऽपोढाऽभ्रान्त‚त्वं चेद‚प्र‚सिद्ध‚म्, किम‚न्य‚त् प्र‚त्य‚क्ष‚स्य ज्ञान‚स्य‚{\tiny $_{lb}$}‚ रूप‚म‚व‚शिष्य‚ते य‚त् प्र‚त्य‚क्ष‚श‚ब्द‚वाच्यं स‚द् अनूद्येतेति । य‚स्मादिन्द्रियान्व‚य‚व्य‚तिरेकानुविधा‚{\tiny $_{lb}$}‚य्य‚र्थेषु साक्षात्कारिज्ञानं प्र‚त्य‚क्ष‚श‚ब्द‚वाच्यं स‚र्वेषां\edtext{}{\lemma{र्वेषां}\Bfootnote{०षां सिद्धं \cite{dp-msA} \cite{dp-msC} \cite{dp-edP} \cite{dp-edH} \cite{dp-edE}}} प्र‚सिद्ध‚म्, त‚द‚नुवादेन क‚ल्प‚नाऽपोढाभ्रान्त-\edtext{\textsuperscript{*}}{\lemma{*}\Bfootnote{पोढ‚त्वाभ्रा० \cite{dp-msC} \cite{dp-msD}}}‚{\tiny $_{lb}$}‚ त्व‚विधिः ।
	\pend% ending standard par
       ‚{\tiny $_{lb}$}‚ 

	  \pstart \leavevmode% starting standard par
	क‚ल्प‚नाया अपोढ‚म् अपेतं क‚ल्प‚नापोढ‚म् । क‚ल्प‚ना\edtext{}{\lemma{ना}\Bfootnote{स‚र्वासु प्र‚तिषु क‚ल्प‚नास्व‚भाव‚र‚हित‚मित्य‚र्थः इति पाठ‚स्य स‚त्त्वेऽपि प्र‚दीपानुसारी‚{\tiny $_{lb}$}‚ पाठोऽत्र गृहीतः ।--सं० ।}} स्व‚भावेन र‚हित‚मित्य‚र्थः । अभ्रान्त-
	\pend% ending standard par
      
	  \endgroup
	‚{\tiny $_{lb}$}‚

	  \pstart \leavevmode% starting standard par
	\hphantom{.}अनेन प्र‚त्य‚क्ष‚मेकं प्र‚माण‚म् इति ब्रुवाण‚श्\textbf{चार्वाकः,} अनुमानाद‚र्थ‚निश्च‚यो दुर्ल‚भः ।‚{\tiny $_{lb}$}‚ क‚निष्ठं च त‚त्प्र‚माण‚म् इत्याच‚क्षाणो \textbf{मीमांस‚क‚श्च} निर‚स्तः । स‚र्व‚त्राय‚म् \textbf{इति}र्वाक्यार्थ‚प‚रि‚{\tiny $_{lb}$}‚स‚माप्तौ । य‚त्र त्व‚र्थ‚विशेषे व‚र्त्त‚ते स क‚थ्य‚त एव ।
	\pend% ending standard par
      ‚{\tiny $_{lb}$}‚

	  \pstart \leavevmode% starting standard par
	\textbf{त‚त्रे}त्यादि \textbf{निर्धार‚ण‚मि}त्य‚न्तं सुग‚म‚म् । केव‚ल‚मिन्द्रिय‚ज्ञानादिप्र‚त्य‚क्षानुयायित्वात्प्र‚त्य‚{\tiny $_{lb}$}‚क्ष‚त्वाख्या जातिर्या व्य‚व‚हार‚सिद्धा त‚येति द्र‚ष्ट‚व्य‚म् ।
	\pend% ending standard par
      ‚{\tiny $_{lb}$}‚

	  \pstart \leavevmode% starting standard par
	त‚त्रेत्य‚य‚मेवं व्य‚व‚स्थिते स‚तीति वाक्योप‚न्यासे । एवं चानुवाद‚विधिक्र‚मेण य‚द‚न्य‚द‚न्येन‚{\tiny $_{lb}$}‚ व्याख्यात‚म् प्र‚त्य‚क्ष‚मिति संज्ञा क‚ल्प‚नापोढ‚त्वादि संज्ञ्येव । तेन संज्ञासंज्ञिस‚म्ब‚न्धः प्र‚तिपाद्य‚ते ‚{\tiny $_{lb}$}‚ इति त‚द् विप्र‚तिप‚त्तिनिराक‚र‚णे प्र‚स्तुतेऽप्र‚स्तुत‚म् । त‚तोऽप‚व्याख्यान‚मिति प्र‚काश‚य‚ति । य‚द‚प्य‚{\tiny $_{lb}$}‚प‚रेण व्याख्यातं प्र‚देशान्त‚र‚प्र‚सिद्ध‚योः क‚ल्प‚नापोढ‚त्वाभ्रान्त‚त्व‚योर‚नुवादेन प्र‚त्य‚क्ष‚त्वं विधीय‚ते ‚{\tiny $_{lb}$}‚ इति त‚द‚पि न च‚तुर‚स्र‚मिति प्र‚तिपाद‚य‚ति । य‚तः प्र‚सिद्धे ल‚क्ष्य‚ल‚क्ष‚ण‚भावे ल‚क्ष‚णानुवादेन‚{\tiny $_{lb}$}‚ ल‚क्ष्यं विधेय‚म्, अप्र‚सिद्धे तु ल‚क्ष‚ण‚वाक्येन ल‚क्ष‚ण‚मेव विधेय‚म्, न ल‚क्ष्य‚मिति न्याय इति ।‚{\tiny $_{lb}$}‚ अन‚योश्च प‚क्ष‚योर्यावान् स‚म‚र्थ‚न‚दूष‚ण‚प्र‚कार‚स्तावान‚नेनैव \textbf{विनिश्च‚य‚टीकायां} स्व‚यं विवेचित‚{\tiny $_{lb}$}‚ इति नेहोच्य‚ते । क‚थं प्र‚त्य‚क्ष‚त्वानुवादेन क‚ल्प‚नापोढ‚त्वादिविधान‚मित्याह--\textbf{प्र‚सिद्ध‚मिति} ।‚{\tiny $_{lb}$}‚ अनेन पूर्वं प्र‚सिद्ध‚स्य प‚श्चाच्छ‚ब्देनाभिधान‚म‚नुवाद इति द‚र्श‚य‚ति । त‚देवं \textbf{द्र‚ष्ट‚व्य‚मि}ति विद‚धा‚{\tiny $_{lb}$}‚नोऽज्ञात‚स्य श‚ब्देन ज्ञाप‚नं विधिरिति द‚र्श‚य‚ति ।
	\pend% ending standard par
      ‚{\tiny $_{lb}$}‚

	  \pstart \leavevmode% starting standard par
	\textbf{न चैत}दित्यादीतिश‚ब्दान्तं सुबोध‚म् ।
	\pend% ending standard par
      ‚{\tiny $_{lb}$}‚

	  \pstart \leavevmode% starting standard par
	\textbf{इन्द्रियान्व‚य‚व्य‚तिरेकानुविधायि}ग्र‚ह‚णं मान‚साद्य‚साधार‚ण‚त‚याऽनुप‚न्यासार्ह‚म‚पि प्र‚त्य‚क्ष‚{\tiny $_{lb}$}‚श‚ब्द‚प्र‚वृत्तिनिमित्त‚स्यार्थ‚साक्षात्कारित्व‚स्य य‚दुप‚ल‚क्ष‚कं त‚दुप‚द‚र्श‚नार्थ‚मुप‚न्य‚स्त‚मिति ज्ञात‚व्य‚म् ।‚{\tiny $_{lb}$}‚ प‚रिपाट्या प्र‚सिद्ध‚त्व‚प्र‚द‚र्श‚नार्थं वा । त‚था हि प्र‚थ‚मं य‚त् त‚द् इन्द्रियान्व‚य‚व्य‚तिरेकानुविधायित‚या‚{\tiny $_{lb}$}‚ प्र‚त्य‚क्षं प्र‚सिद्ध‚मिति प्र‚तिपाद्य‚ते । त‚द‚नु य‚दीद‚मिन्द्रियाश्रितं ज्ञानं भ‚व‚तां प्र‚त्य‚क्षं प्र‚सिद्धं त‚त्रापि‚{\tiny $_{lb}$}‚ प्र‚त्य‚क्ष‚श \leavevmode\ledsidenote{\textenglish{18b/ms}}ब्द‚प्र‚वृत्ताव‚र्थ‚साक्षात्कारित्व‚मेव निमित्तं जानीते इति प्र‚तिपाद्य‚ते । \textbf{त‚द‚नु‚{\tiny $_{lb}$}‚वादेनेति} सुग‚म‚म् ।
	\pend% ending standard par
      ‚{\tiny $_{lb}$}‚

	  \pstart \leavevmode% starting standard par
	य‚त् क‚ल्प‚न‚याऽपोढं र‚हितं त‚त् क‚ल्प‚नाया अपोढ‚म‚प‚ग‚तं भ‚व‚तीत्य‚र्थाभेदं म‚त्वा \textbf{क‚ल्प‚नाया‚{\tiny $_{lb}$}‚ अपोढ‚मिति} क‚र्त्त‚रि निष्ठामाह । न तु क‚ल्प‚नापोढ‚मिति क‚र्त्त‚रि निष्ठैव । एवं हि \textbf{त‚या‚{\tiny $_{lb}$}‚ र‚हित}मित्या\textbf{चार्य}विव‚र‚ण‚विरोधः स्यात् ।
	\pend% ending standard par
      ‚{\tiny $_{lb}$}‚‚{\tiny $_{lb}$}‚\textsuperscript{\textenglish{42/dm}}‚{\tiny $_{lb}$}‚
	  \bigskip
	  \begingroup
	

	  \pstart \leavevmode% starting standard par
	म‚र्थ‚क्रियाक्ष‚मे व‚स्तुरूपेऽविप‚र्य‚स्त‚मुच्य‚ते । अर्थ‚क्रियाक्ष‚मं च\edtext{}{\lemma{च}\Bfootnote{०क्ष‚मं व‚स्तु० \cite{dp-msA} \cite{dp-edP} \cite{dp-edE}}} व‚स्तुरूपं स‚न्निवेशोपाधिव‚र्णात्म‚क‚म्\edtext{}{\lemma{म्}\Bfootnote{०धिध‚र्मात्म० \cite{dp-edP} \cite{dp-edH}}} ।‚{\tiny $_{lb}$}‚ त‚त्र य‚न्न भ्राम्य‚ति त‚द‚भ्रान्त‚म् ।
	\pend% ending standard par
      
	  \endgroup
	‚{\tiny $_{lb}$}‚

	  \pstart \leavevmode% starting standard par
	न‚नु किमेक‚स्मिन् ज्ञाने ज्ञानान्त‚र‚स‚म्भ‚वोऽस्ति येनायं प्र‚तिषेधः शोभेत ? क‚ल्प‚नाज्ञानेऽपि‚{\tiny $_{lb}$}‚ च क‚ल्प‚नान्त‚रं नास्तीति त‚द‚पि क‚ल्प‚नापोढं प्र‚स‚ज्येतेति व‚च‚नाव‚काशं प‚श्य‚न् \textbf{क‚ल्प‚नास्व‚भावे}‚{\tiny $_{lb}$}‚ने\edtext{}{\lemma{ने}\Bfootnote{ध‚र्मोत्त‚रे स‚र्व‚त्र प्र‚तिषु क‚ल्प‚नास्व‚भाव‚र‚हित‚म् इत्येव पाठो ल‚भ्य‚ते न तु क‚ल्प‚नास्व‚{\tiny $_{lb}$}‚भावेन र‚हित‚म् इति ।}} त्यादिनाऽस्य विव‚क्षित‚म‚र्थं स्फुट‚य‚ति ।
	\pend% ending standard par
      ‚{\tiny $_{lb}$}‚

	  \pstart \leavevmode% starting standard par
	एव‚ञ्च ब्रुव‚न् क‚ल्प‚नाऽपोढ‚मित्य‚त्रा\textbf{चार्योये} ल‚क्ष‚ण‚वाक्ये भाव‚प्र‚धानोऽयं क‚ल्प‚नाश‚ब्दः,‚{\tiny $_{lb}$}‚ तेन क‚ल्प‚नात्वेन र‚हित‚मित्य‚र्थ इति द‚र्श‚य‚ति । ध‚र्मिणा वा क‚ल्प‚नाख्येन ध‚र्म‚स्याभिलाप‚संस‚र्ग‚{\tiny $_{lb}$}‚योग्य‚प्र‚तिभास‚त्त्वाख्य‚स्य निर्देशं द‚र्श‚य‚ति । उभ‚य‚थाऽपि क‚ल्प‚नास्व‚भाव‚र‚हित‚मित्य‚र्थः इत्य‚स्य‚{\tiny $_{lb}$}‚ स्फुटीक‚र‚ण‚स्य घ‚ट‚नात् । क‚ल्प‚ना च व‚क्ष्य‚माण‚ल‚क्ष‚णा ग्राह्या । एवं च श‚ब्दार्थं स्फुट‚य‚ताऽनेन‚{\tiny $_{lb}$}‚ य‚द् \textbf{आचार्य‚दिग्नागीय}प्र‚त्य‚क्ष‚ल‚क्ष‚ण‚दूष‚णाव‚स‚रे--क‚ल्प‚ना नाम‚जात्यादियोज‚ना । तेन य‚न्नाम्ना‚{\tiny $_{lb}$}‚ नाभिधीय‚ते, जात्यादिना च न व्य‚प‚दिश्य‚ते विष‚य‚भेदानुविधायि य‚ज् ज्ञानं त‚त्प्र‚त्य‚क्ष‚म् ‚{\tiny $_{lb}$}‚ \href{http://sarit.indology.info/?cref=nv.1.1.p44.1}{न्याय‚वा० पृ० ४१} इत्य‚र्थं प‚रिक‚ल्प्य अथ प्र‚त्य‚क्ष‚श‚ब्देन कोऽर्थोऽभिधीय‚ते । य‚दि प्र‚त्य‚क्षं‚{\tiny $_{lb}$}‚ क‚थ‚म‚वाच्य‚म् ? अथ न प्र‚त्य‚क्ष‚म‚वाच‚क‚स्त‚र्हि प्र‚त्य‚क्ष‚श‚ब्दः । प्र‚त्य‚क्ष‚त्व‚सामान्याभिधानेऽपि य‚दि‚{\tiny $_{lb}$}‚ त‚द्व्य‚तिरेकि; न प्र‚त्य‚क्ष‚मुक्त‚म् । अथाव्य‚तिरेकि । क‚थं नोक्त‚म् ? क‚ल्पेनापोढ‚श‚ब्देनापि य‚दि‚{\tiny $_{lb}$}‚ त‚दुच्य‚ते क‚थं न व्याघातः ? अथ नोच्य‚ते त‚स्योच्चार‚ण‚वैय‚र्थ्य‚म् ? प्र‚त्य‚क्षं क‚ल्प‚नाऽपोढ‚मिति‚{\tiny $_{lb}$}‚ च व्य‚प‚दिश्य‚ते, न चाभिधेय‚मिति कोऽन्यो \textbf{भ‚द‚न्ताद्} व‚क्तुम‚र्ह‚ति \href{http://sarit.indology.info/?cref=nv.1.1.p44-6-12}{न्याय‚वा० पृ० ४१} इत्यादि‚{\tiny $_{lb}$}‚ य‚द‚वादी\textbf{दुद‚द्योत‚क‚र}स्त‚त्स‚र्वं क‚ल्प‚नापोढ‚श‚ब्दार्थाऽप‚रिज्ञान‚विल‚सितं त‚स्य त‚प‚स्विन इति सूचित‚म् ।
	\pend% ending standard par
      ‚{\tiny $_{lb}$}‚

	  \pstart \leavevmode% starting standard par
	\hphantom{.}इहाविसंवाद‚क‚त्व‚म‚भ्रान्त‚त्व‚म‚भिप्रेत‚म् । तेन द्विच‚न्द्रादिज्ञानं व्य‚व‚च्छिन्न‚म्, \textbf{योगाचार}‚{\tiny $_{lb}$}‚म‚त‚म‚पि स‚गृहीतं भ‚व‚तीति पूर्व‚व्याख्यान‚म‚व‚म‚न्य‚मानोऽभ्रान्त‚श‚ब्द‚स्यार्थ‚माह--\textbf{अभ्रान्त‚म्}‚{\tiny $_{lb}$}‚ इति । \textbf{अभ्रान्त‚म‚र्थ‚क्रियाक्ष‚मेऽविप‚र्य‚स्तं य‚त् त‚दुच्य‚ते} । न त्व‚विसंवाद‚क‚मिह ग्र‚हीत‚व्य‚मिति‚{\tiny $_{lb}$}‚ बुद्धिस्थं प‚श्चाद् व्य‚क्तीक‚र्त्त‚व्य‚म् । अन‚र्थोऽपि व‚स्तुत‚याऽध्य‚व‚सीय‚त इत्य\textbf{र्थ‚क्रियाक्ष‚म} इति‚{\tiny $_{lb}$}‚ विशेष‚ण‚म् । त‚र्ह्य‚र्थ‚क्रियाक्ष‚म इत्येवास्तु । न । अर्थ‚क्रियाक्ष‚म‚स्यैव व‚स्तुत्व‚ज्ञाप‚नार्थ‚त्वात् । किं‚{\tiny $_{lb}$}‚ त‚द‚र्थ‚क्रियाक्ष‚म‚म् ? किम‚व‚य‚वि ? अथान्य‚देवेत्याह--\textbf{अर्थ‚क्रियेति । स‚न्निवेश‚श्च}तुर‚स्र‚त्वादिः प्र‚ति‚{\tiny $_{lb}$}‚भास‚ध‚र्मः । स \textbf{उपाधि}र्विशेष‚णं य‚स्य \textbf{व‚र्ण}स्य व‚स्तुश‚ब्द‚वाच्य‚स्य शुक्लादिप‚र‚माणुसंघात‚स्य‚{\tiny $_{lb}$}‚ त‚थोत्प‚न्न‚स्य स त‚था । स‚न्निवेश‚विशिष्ट‚स्यैव व‚र्ण‚स्यान्व‚य‚व्य‚तिरेकाभ्याम‚र्थ‚क्रियायामुप‚योग‚{\tiny $_{lb}$}‚द‚र्श‚नादेत‚दाह । स एवा\textbf{त्मा} स्व‚भावो य‚स्येति त‚त्त‚था । एत‚च्च चाक्षुष‚ज्ञान‚विष‚याभिप्रायेणोक्तं‚{\tiny $_{lb}$}‚ द्र‚ष्ट‚व्य‚म् । अन्य‚था ब‚ह्व‚स‚म‚ञ्ज‚सं स्यादिति । अनेन प‚र‚माणुप्र‚च‚य‚मात्र‚स्यैवार्थ‚क्रि\leavevmode\ledsidenote{\textenglish{19a/ms}}या‚{\tiny $_{lb}$}‚कारित्वं नाव‚य‚विन‚स्त‚स्यास‚त्त्वादिति सूचित‚म् । \textbf{न भ्राम्य}ति न विप‚र्य‚स्य‚ति--अन्य‚थाग्राहि न‚{\tiny $_{lb}$}‚ भ‚व‚ति ।
	\pend% ending standard par
      ‚{\tiny $_{lb}$}‚

	  \pstart \leavevmode% starting standard par
	स्यादेत‚त्--प‚र‚माण्व‚र्थ एव भ‚व‚न्म‚ते बाह्यं व‚स्तु । स‚र्वं च विज्ञानं तेषु प‚र‚म‚सूक्ष्मेषु‚{\tiny $_{lb}$}‚ ‚{\tiny $_{lb}$}‚ \leavevmode\ledsidenote{\textenglish{43/dm}}‚{\tiny $_{lb}$}‚ स्थूलाभास‚माजाय‚ते । त‚त्क‚थं किञ्चिद‚भ्रान्तं नामेति ? अत्रोच्य‚ते । एक‚साम‚ग्रीज‚न्म‚नां‚{\tiny $_{lb}$}‚ प‚र‚माणूनां भिन्न‚देश‚स्व‚भावानां त‚द्धेत्व‚भाव‚त‚च्छायालोक‚प‚र‚माणुस्व‚भावेनान्त‚रेण र‚हित‚त्वान्नि‚{\tiny $_{lb}$}‚र‚न्त‚र‚त्वेन प्र‚तिभास एव देश‚वितानाव‚भासात्मा स्थौल्यं नाप‚रं किञ्चित् । त‚त्र त‚थाभूत‚प‚र‚{\tiny $_{lb}$}‚माणुस‚मुदाय‚निष्ठं निर्विक‚ल्प‚कं विज्ञानं क‚थं भ्रान्तं स्यात्? य‚द्येकैकं प‚र‚माणुम‚नेक‚देशाव‚ष्ट‚{\tiny $_{lb}$}‚म्भेन गृह्णीयान्न पुन‚र‚नेक‚म‚नेक‚देशाव‚ष्ट‚म्भेन गृह्ण‚त् । इतोऽपि विप‚र्य‚स्येद् य‚दि भिन्न‚देशान्‚{\tiny $_{lb}$}‚ प‚र‚माणून् एक‚देशान् गृह्णीयात् । न चैत‚द‚स्ति, अणुमात्र‚क‚पिण्ड‚प्र‚तिभासाभावात् । एक‚देश‚{\tiny $_{lb}$}‚ग्र‚ह‚णे हि पिण्डो भासेत अणुमात्र‚को न तु वित‚त‚देशः । न चानेक‚ग्र‚हो भ्र‚मः । अत‚स्मिं‚{\tiny $_{lb}$}‚स्त‚दिति प्र‚त्य‚य‚स्य तादात्म्यात् । त‚द‚य‚म‚र्थः--एक‚ज्ञान‚ग्राह्यास्त‚थाविधा ब‚ह‚वः प‚र‚माण‚वः स्थूल‚{\tiny $_{lb}$}‚ इति । एकोऽयं स्थूल इति तु त‚थाभूत‚प्र‚तिभासाश्र‚येण व्य‚व‚स्थाप्य‚मान‚त्वात् प्र‚तिभास‚ध‚र्म‚{\tiny $_{lb}$}‚ इत्युच्य‚ते । न व‚स्तुध‚र्मः, प्र‚त्येक‚म‚प‚रिस‚माप्तेरित्य‚ल‚मिह विस्त‚रेण ।
	\pend% ending standard par
      ‚{\tiny $_{lb}$}‚

	  \pstart \leavevmode% starting standard par
	न‚नु चैव‚म‚प्य‚न्योन्य‚म‚संसृष्ट‚स्व‚भावान् प‚र‚माणून् संसृष्टान् गृह्ण‚द्विज्ञानं क‚थ‚मिव‚{\tiny $_{lb}$}‚विप‚र्य‚स्तं नामेति । अत्राप्युच्य‚ते । किमिद‚म‚संसृष्ट‚त्व‚मिष्टं भ‚व‚ता, य‚द् विप‚र्य‚य‚ग्र‚ह‚णाद् भ्रान्तं‚{\tiny $_{lb}$}‚ ज्ञान‚मुप‚व‚र्ण्य‚ते ? किं नानारूप‚त्व‚म्, अथ नानादेश‚त्व‚म्, उत रूपेणैव विजातीयेन व्य‚व‚हित‚त्व‚म्,‚{\tiny $_{lb}$}‚ आहोस्विदिन्द्रियान्त‚र‚ग्राह्येणार्थेन व्य‚व‚कीर्ण‚त्व‚म् ? त‚त्र य‚दि नानारूप‚त्व‚म‚संसृष्ट‚त्व‚मिष्टं त‚दा‚{\tiny $_{lb}$}‚ न क‚श्चित्संसृष्ट‚ग्र‚हो नाम स‚म्भ‚व‚ति य‚तोऽसंसृष्टा एव प‚र‚माण‚वः स‚र्व‚दा गृह्य‚न्ते । वित‚त‚देश‚{\tiny $_{lb}$}‚स्व‚भावानामेव तेषाम‚व‚भास‚नात् । य‚दि ह्येक‚रूपा भासेर‚न, अणुमात्र‚कः पिण्डो भासेत । न‚{\tiny $_{lb}$}‚ तु वित‚त‚देश‚भास‚नं स्यात् ।
	\pend% ending standard par
      ‚{\tiny $_{lb}$}‚

	  \pstart \leavevmode% starting standard par
	अथ नानादेश‚त्व‚म‚संसृष्ट‚त्व‚म‚भिप्रेतं त‚द‚पि न‚त‚राम‚संसृष्ट‚ग्र‚हो य‚तो नानादेशा नीला‚{\tiny $_{lb}$}‚ प‚र‚माण‚वो नानादेशा एव च गृह्य‚न्ते । एक‚देश‚त्व‚भास‚ने हि पिण्डो भासेताणुमात्र‚क इत्युक्त‚म् ।
	\pend% ending standard par
      ‚{\tiny $_{lb}$}‚

	  \pstart \leavevmode% starting standard par
	अथ रूपेणैव विजातीयेन व्य‚व‚हित‚त्व‚म‚संसृष्ट‚त्वं विव‚क्षित‚म्; त‚दा तु त‚द‚स‚म्भ‚वादेव‚{\tiny $_{lb}$}‚ न त‚द्विप‚रीत‚ग्र‚हः । य‚तो रूपान्त‚र‚व्य‚व‚धान‚र‚हिता एव निर‚न्त‚रा नीलाः प‚र‚माण‚वः,‚{\tiny $_{lb}$}‚ भास‚न्ते च त‚थाभूता इति क‚थं विभ्र‚मः । म‚ध्य‚व‚र्त्तिनो विजातीय‚स्यालोकादिप‚र‚माणोर‚{\tiny $_{lb}$}‚नुत्प‚त्तेर‚प्र‚तिभास‚नाच्च । अथ च्छायालोक‚प‚र‚माणुरुत्प‚द्य‚मानः केन प्र‚तिब‚द्धो य‚तो नोत्प‚द्य‚ते ।‚{\tiny $_{lb}$}‚ न च श‚क्यं व‚क्तुम्--म‚ध्ये प‚र‚माण्वोर्नास्ति प‚र‚माण्व‚न्त‚र‚स्याव‚काश इति । य‚तो निर‚व‚य‚वः‚{\tiny $_{lb}$}‚ प‚र‚माणुः स‚र्व‚त्र साव‚काश इति । स‚त्य‚मेत‚त् । केव‚लं नाव‚काशाभावात् त‚द‚नुत्प‚त्तिर‚पि तु‚{\tiny $_{lb}$}‚ हेत्व‚भावात् । क‚स्माद् हेतुर्न भ‚व‚ति ? स्व‚हे\leavevmode\ledsidenote{\textenglish{19b/ms}}त्व‚भावादित्य‚प‚र्य‚नुयोग एव ।
	\pend% ending standard par
      ‚{\tiny $_{lb}$}‚

	  \pstart \leavevmode% starting standard par
	अथ भिन्नेन्द्रिय‚ग्राह्य‚स्प‚र्शादिव्य‚व‚कीर्ण‚त्व‚म‚संसृष्ट‚त्व‚म‚भिम‚त‚म् । त‚दा संसृष्टान् प‚र‚माणून्‚{\tiny $_{lb}$}‚ गृह्णाति विज्ञान‚मिति इन्द्रियान्त‚र‚ग्राह्य‚शून्यान् गृह्णातीत्युक्तं भ‚व‚ति । त‚था च न किञ्चिद‚नि‚{\tiny $_{lb}$}‚ष्ट‚म् । त‚थाहि य‚दि नामेन्द्रियान्त‚र‚ग्राह्य‚स्प‚र्शादिर्न गृह्य‚ते त‚थापि नील‚रूपं ताव‚त् स्व‚देश‚स्व‚भाव‚स्थितं‚{\tiny $_{lb}$}‚ गृह्य‚त एव । न च भिन्नेन्द्रिय‚ग्राह्य‚शून्यानां स्व‚रूपं गृह्य‚माणं विप‚रीतं गृहीतं भ‚व‚ति । देश‚काला‚{\tiny $_{lb}$}‚काराणामेक‚स्याप्य‚विप‚र्यासात् । न चाग्र‚हो भ्र‚म इति । न‚नु च प‚र‚माणूनाम‚न्त‚राण्याकाशात्म‚{\tiny $_{lb}$}‚कानि स‚न्ति । न च ते सान्त‚राः प्र‚तिभास‚न्ते । त‚त्क‚थ‚म‚विप‚र्यास इति । अथ किमिद‚माकाशं‚{\tiny $_{lb}$}‚ नाम । य‚दि रूपान्त‚रात्म‚कं त‚न्नास्तीत्युक्त‚म् । अथापि स्प‚र्शाद्यात्म‚कं त‚त्राप्युक्त‚म् । अथ‚{\tiny $_{lb}$}‚ स‚प्र‚तिघ‚द्र‚व्याभावः । एव‚म‚प्य‚व‚स्त्वाकाश‚म् । त‚त‚श्चाकाश‚म‚न्त‚र‚मित्य‚न्य‚व‚स्त्व‚न्त‚रं न किञ्चिद- \leavevmode\ledsidenote{\textenglish{44/dm}}‚{\tiny $_{lb}$}‚ 
	  
	एत‚च्च \edtext{}{\lemma{च्च}\Bfootnote{क‚ल्प‚नाभ्रान्त०--\cite{dp-msD-n}}}ल‚क्ष‚ण‚द्व‚यं विप्र‚तिप‚त्तिनिरासार्थ‚म्,\edtext{\textsuperscript{*}}{\lemma{*}\Bfootnote{निराक‚र‚णार्थ‚म् \cite{dp-msA} \cite{dp-msC} \cite{dp-edP} \cite{dp-edE} \cite{dp-edH} \cite{dp-edN}}} न त्व‚नुमान‚निवृत्त्य‚र्थ‚म् । य‚तः‚{\tiny $_{lb}$}‚ न्त‚र‚मित्युक्तं स्यात् । त‚था निर‚न्त‚राः प‚र‚माण‚व इत्युक्तं भ‚व‚ति । त‚तो निर‚न्त‚राश्च प‚र‚माण‚वो‚{\tiny $_{lb}$}‚ निर‚न्त‚रा एव भास‚न्ते । त‚त् किमुच्य‚तेऽन्त‚र‚माकाश‚म्, न च त‚त्प्र‚तिभास‚त इति ? य‚त्ख‚ल्व‚त्य‚न्त‚{\tiny $_{lb}$}‚म‚स‚त् श‚श‚विषाण‚प्र‚ख्यं त‚त्क‚थं भासेत ?
	\pend% ending standard par
      ‚{\tiny $_{lb}$}‚

	  \pstart \leavevmode% starting standard par
	न‚न्वाकाशात्म‚नोऽप्य‚न्त‚र‚स्याभावे रूप‚संस‚र्गः प‚र‚माणुनां प्र‚स‚ज्येत । नैष दोषः ।‚{\tiny $_{lb}$}‚ नास्माभिरुच्य‚ते रूप‚मेकं प‚र‚माणूनां देशो नैक इति । अपि तु भिन्न‚रूप‚देशा उत्प‚न्ना‚{\tiny $_{lb}$}‚ म‚ध्य‚व‚र्त्तिविजातीय‚रूप‚र‚हितास्त‚थैव भास‚न्त इति त‚त् क‚थं रूप‚संस‚र्ग‚प्र‚स‚ङ्गः ?
	\pend% ending standard par
      ‚{\tiny $_{lb}$}‚

	  \pstart \leavevmode% starting standard par
	न‚नु च र‚सादिदेशे नील‚रूपं प्र‚तिभास‚ते । त‚त‚श्चात‚द्देशं त‚द्देश‚त‚या गृह्ण‚द् विज्ञानं क‚थ‚{\tiny $_{lb}$}‚म‚भ्रान्तं नामेति ? त‚त्राप्युच्य‚ते । य‚दा देशः प्र‚तिभास‚ते त‚दा त‚स्मिन् देशे प्र‚तिभास‚माने यः प्र‚ति‚{\tiny $_{lb}$}‚भास‚तेऽर्थः स देश‚विशिष्ट उच्य‚ते । य‚दि च र‚सादिश्च‚क्षुर्विज्ञाने प्र‚तिभासेत त‚दा त‚द्देश‚व्यापिनि‚{\tiny $_{lb}$}‚ नीले गृह्य‚माणे स्याद् भ्रान्तं विज्ञान‚म् । न च त‚त्र र‚सादिः प्र‚तिभास‚ते, इन्द्रियान्त‚र‚ग्राह्य‚स्येन्द्रि‚{\tiny $_{lb}$}‚यान्त‚र‚ज्ञाने प्र‚तिभासायोगात् । त‚त् कुत‚स्त‚द्देश‚नील‚ग्र‚ह‚ण‚म् ? नील‚मेव हि भास‚मानं देशः‚{\tiny $_{lb}$}‚ नाप‚रो देशः क‚श्चिदाभास‚ते । इन्द्रियान्त‚र‚ग्राह्याप्र‚तिभासे च शुद्ध‚रूप‚प्र‚तिभासः । शुद्ध‚रूप‚{\tiny $_{lb}$}‚प्र‚तिभास एव च निर‚न्त‚र‚प्र‚तिभासः । त‚तो निर‚न्त‚रा नीलाः प‚र‚माण‚वो गृह्य‚न्ते । त‚स्मात्‚{\tiny $_{lb}$}‚ स्व‚देश‚स्थायिनो नील‚प‚र‚माण‚वः स्व‚रूपेणैव गृह्य‚न्ते । त‚तो देश‚कालाकाराणामेक‚स्याप्य‚प‚रित्यागा‚{\tiny $_{lb}$}‚न्नीलाभासं विज्ञान‚म‚भ्रान्त‚मेव । स‚र्वं चैत‚द् ग्राह्य‚त‚त्त्वं \textbf{विनिश्च‚ये ध‚र्मोत्त‚रेणैव} विस्त‚रेण‚{\tiny $_{lb}$}‚ निरूपित‚मिति नेह प्र‚त‚न्य‚ते ।
	\pend% ending standard par
      ‚{\tiny $_{lb}$}‚

	  \pstart \leavevmode% starting standard par
	न‚न्वेव‚म‚भ्रान्त‚त्वे \textbf{योगाचार‚म‚त}म‚स‚ङ्गृहीतं स्यात् । ग्राह्य‚ग्राह‚काकार‚त‚या प्र‚वृत्तेः‚{\tiny $_{lb}$}‚ स‚र्व‚स्यैवास‚र्व‚ज्ञाविज्ञान‚स्याल‚म्ब‚ने भ्रान्त‚त्वात् । त‚त्क‚थं पूर्व‚व्याख्यानाव‚ज्ञा न क्रिय‚त इति चेत् ।‚{\tiny $_{lb}$}‚ उच्य‚ते । न \textbf{योगाचार‚न‚ये} ल‚क्ष‚ण‚मिद‚म्, किन्तु \textbf{सौत्रान्तिक‚न‚य} एव । न च स‚र्वं विज्ञान‚वादे‚{\tiny $_{lb}$}‚ योज‚यि\leavevmode\ledsidenote{\textenglish{20a/ms}}तुं श‚क्य‚म् । त‚स्य विष‚यः स्व‚ल‚क्ष‚ण‚म् इत्यादेर‚श‚क्य‚योज‚न‚त्वात् । त‚स्मिन्‚{\tiny $_{lb}$}‚ किं प्र‚त्य‚क्ष‚ल‚क्ष‚ण‚मिति चेत् । क‚ल्प‚नापोढ‚त्व‚मेव । द्विच‚न्द्रादिनिरासः क‚थ‚मिति चेत् स‚म्य‚ग्ज्ञानं‚{\tiny $_{lb}$}‚ स‚देव‚मित्य‚भिप्रायाद‚दोषः । \textbf{सौत्रान्तिक‚न‚ये}ऽपि किं नैवं ? त‚त‚श्चाभ्रान्त‚ग्र‚ह‚ण‚म‚तिरिच्य‚त इति‚{\tiny $_{lb}$}‚ चेत् । स‚त्य‚मेत‚त् । केव‚लं विप्र‚तिप‚त्तिनिरासार्थं व‚र्ण‚यिष्य‚त इति । इहाभ्रान्त‚प‚दं तैमिरिकादि‚{\tiny $_{lb}$}‚ज्ञान‚व्य‚व‚च्छेदार्थ‚म् । क‚ल्प‚नाऽपोढ‚ग्र‚ह‚णं तु अनुमान‚निरासार्थ‚मिति य‚त्पूर्व‚कैर्व्याख्यातं‚{\tiny $_{lb}$}‚ त‚द् व्य‚क्त‚मेवाप‚ह‚स्त‚य‚न्नाह--\textbf{एत‚च्च}--इति । \textbf{चो}ऽव‚धार‚णे । \textbf{विप्र‚तिप‚त्तिनिरासार्थ}मित्य‚तः प‚रो‚{\tiny $_{lb}$}‚ द्र‚ष्ट‚व्यः । पूर्वेषाम‚भिप्रेतं प्र‚तिषेध‚ति । \textbf{न त्वि}ति भेदार्थः । न‚नु किमुच्य‚ते \textbf{न त्व‚नु‚{\tiny $_{lb}$}‚मान‚निवृत्त्य‚र्थ}मिति ? याव‚तैकैकेनानुमान‚निव‚र्त्त‚नादिति चेत् । उच्य‚ते । \textbf{ल‚क्ष‚ण‚द्व‚य}मिति द्व‚योरुपा‚{\tiny $_{lb}$}‚दानं विप्र‚तिप‚त्तिनिराक‚र‚णार्थ‚म्, एकैकेनानुमान‚व्य‚व‚च्छेद‚सिद्धेरिति स‚मुदित‚फ‚ल‚मेत‚त् । त‚तो‚{\tiny $_{lb}$}‚ \textbf{नानुमान‚निवृत्त्य‚र्थ}मिति नानुमान‚निवृत्त्य‚र्थ‚मेवेति साव‚धार‚णं निषिध्य‚ते । य‚दि तु विप्र‚तिप‚त्ति‚{\tiny $_{lb}$}‚निराचिकीर्ष‚या क्रिय‚माण‚म‚नुमानं व्य‚व‚च्छिन‚त्ति न त‚द‚र्थ‚मेव द्व‚य‚मिति ।
	\pend% ending standard par
      ‚{\tiny $_{lb}$}‚

	  \pstart \leavevmode% starting standard par
	न‚नु नेदं ल‚क्ष‚ण‚द्व‚य‚म‚नुमान‚व्य‚व‚च्छेदार्थं पूर्व‚कैर्व्याख्यात‚म् । किन्तु क‚ल्प‚नाऽपोढ‚ग्र‚ह‚ण‚मेव ।‚{\tiny $_{lb}$}‚ अभ्रान्त‚ग्र‚ह‚णं तु द्विच‚न्द्रादिज्ञान‚व्य‚व‚च्छेदार्थ‚म् । त‚त् क‚थं \textbf{न त्व‚नुमान‚निवृत्त्य‚र्थ‚म्} त‚था‚{\tiny $_{lb}$}‚ ‚{\tiny $_{lb}$}‚ \leavevmode\ledsidenote{\textenglish{45/dm}}‚{\tiny $_{lb}$}‚ 
	  
	क‚ल्प‚नाऽपोढ‚ग्र‚ह‚णेनैवानुमानं निव‚र्तित‚म् । त‚त्रास‚त्य‚भ्रान्त‚ग्र‚ह‚णे ग‚च्छ‚द्वृक्ष‚द‚र्श‚नादि प्र‚त्य‚क्षं‚{\tiny $_{lb}$}‚ क‚ल्प‚नाऽपोढ‚त्वात् स्यात् । त‚तो हि प्र‚वृत्तेन\edtext{}{\lemma{वृत्तेन}\Bfootnote{प्र‚मात्रा--\cite{dp-msD-n}}} वृक्ष‚मात्र‚म‚वाप्य‚ते इति संवाद‚क‚त्वात् स‚म्य‚ग्ज्ञान‚म्,‚{\tiny $_{lb}$}‚ क‚ल्प‚नाऽपोढ‚त्वाच्च प्र‚त्य‚क्ष‚मिति स्यादाश‚ङ्का । त‚न्निवृत्त्य‚र्थ‚म् अभ्रान्त‚ग्र‚ह‚ण‚म् । त‚द्धि\edtext{}{\lemma{द्धि}\Bfootnote{ग‚च्छ‚द‚वृक्ष\add{द‚र्श‚न‚म्}--\cite{dp-msD-n}}}‚{\tiny $_{lb}$}‚ भ्रान्त‚त्वात् न प्र‚त्य‚क्ष‚म् । त्रिरूप‚लिङ्ग‚ज‚त्वाभावाच्च नानुमान‚म् । न च प्र‚माणान्त‚र‚म‚स्ति ।‚{\tiny $_{lb}$}‚ अतो ग‚च्छ‚द्वृक्ष‚द‚र्श‚नादि मिथ्याज्ञान‚मित्युक्तं भ‚व‚ति ।‚{\tiny $_{lb}$}‚ \textbf{य‚तः क‚ल्प‚नापोढ‚ग्र‚ह‚णेनैवानुमानं निव‚र्त्तित‚म्} इत्युच्य‚त इति चेत् । स‚त्य‚म् । केव‚लं य‚द्य‚भ्रान्त‚{\tiny $_{lb}$}‚ग्र‚ह‚णं व्य‚व‚च्छेदार्थ‚मेव त‚दानुमान‚व्य‚व‚च्छेदार्थ‚मेव युज्य‚ते । न तु द्विच‚न्द्रादिज्ञान‚निरासार्थ‚म्,‚{\tiny $_{lb}$}‚ त‚स्य स‚म्य‚ग्ज्ञानाधिकारादेव व्य‚व‚च्छेद‚सिद्धेः । त‚था हि द्विविधं स‚म्य‚ग्ज्ञान‚मिति प्र‚स्तुत्य‚{\tiny $_{lb}$}‚ ल‚क्ष‚ण‚मिदं विधीय‚मानं त‚द‚धिकारेणैव विहितं भ‚व‚ति ।
	\pend% ending standard par
      ‚{\tiny $_{lb}$}‚

	  \pstart \leavevmode% starting standard par
	न‚न्वेवं स‚ति क‚ल्प‚नापोढ‚ग्र‚ह‚णेनैव निवृत्तेर‚भ्रान्त‚ग्र‚ह‚ण‚म‚तिरिच्य‚ते । अय‚म‚प‚र‚स्तेषां‚{\tiny $_{lb}$}‚ दोषोऽस्ति । न तु द्विच‚न्द्रादिज्ञान‚निवृत्त्य‚र्थ‚मिदं युज्य‚त इति \textbf{ध‚र्मोत्त‚र}स्याश‚यः । य‚द्येत‚द‚र्थ‚{\tiny $_{lb}$}‚म‚भ्रान्त‚ग्र‚ह‚णं न भ‚व‚ति त‚र्हि किम‚नेनेत्याह--त‚त्र--इति ।
	\pend% ending standard par
      ‚{\tiny $_{lb}$}‚

	  \pstart \leavevmode% starting standard par
	न‚नु ग‚च्छ‚द्वृक्ष‚द‚र्श‚नादेः क‚ल्प‚नापोढ‚स्यापि विसंवाद‚क‚त्वेन स‚म्य‚ग्ज्ञान‚त्वाभावादेव‚{\tiny $_{lb}$}‚ व्य‚व‚च्छेदे सिद्धे किमेत‚द‚र्थेनाभ्रान्त‚प‚देनेत्याह--\textbf{त‚तो हि}--इति । वृक्ष‚मात्र‚मिति ग‚म‚नेनाऽन‚{\tiny $_{lb}$}‚व‚च्छिन्न‚म् । \textbf{इति} हेतौ । अस्तु स‚म्य‚ग्ज्ञान‚म्, प्र‚त्य‚क्ष‚स‚म्भाव‚ना त्व‚स्य क‚थ‚म् ? न हि य‚देव‚{\tiny $_{lb}$}‚ स‚म्य‚ग्ज्ञानं त‚देव प्र‚त्य‚क्ष‚मित्याश‚ङ्क्य पूर्वोक्त‚मेव प्र‚स‚ङ्गेनाह--\textbf{क‚ल्प‚ना}--इति । \textbf{चः} संवाद‚क—‚{\tiny $_{lb}$}‚त्वापेक्ष‚या स‚मुच्च‚यार्थः । \textbf{इति}क‚र‚णेनाश‚ङ्कायाः स्व‚रूपं द‚र्श‚य‚ति । आश‚ङ्का चेय‚मीदृशी‚{\tiny $_{lb}$}‚ यौक्ती द्र‚ष्ट‚व्या । य‚द् वा \textbf{चो} य‚स्माद‚र्थे ।\add{... ... ...}अस‚त्य‚भ्रान्त‚ग्र‚ह‚णे स्यादिय‚माश‚ङ्का ।‚{\tiny $_{lb}$}‚ य‚दि स‚त्य‚पि स्यात्त‚दा किम‚नेनेत्याह--\textbf{त‚न्निवृत्त्य‚र्थ}मिति श‚ङ्कानिवृत्त्य‚र्थ‚मात्र‚म् । अनेन एत‚द्‚{\tiny $_{lb}$}‚ द‚र्श‚य‚ति--य‚द्य‚पि प‚र‚मार्थ‚तः स‚म्य‚ग्ज्ञानाधिकारादेव‚मादिज्ञानं व्य‚व‚च्छिद्य‚ते । त‚थाप्यंश‚संवाद‚{\tiny $_{lb}$}‚वादिनामाह‚त्य विप्र‚तिप‚त्तिनिराक‚र‚णार्थं क\leavevmode\ledsidenote{\textenglish{20b/ms}}र्त्त‚व्य‚मेवाभ्रान्त‚ग्र‚ह‚ण‚मिति ।
	\pend% ending standard par
      ‚{\tiny $_{lb}$}‚

	  \pstart \leavevmode% starting standard par
	स्यादेत‚त्--किम‚भ्रान्त‚ग्र‚ह‚णेनाधिकं कृत‚म् । येनैत‚द् विप्र‚तिप‚त्तिनिराक‚र‚णार्थं भ‚व‚ती‚{\tiny $_{lb}$}‚त्याश‚ङ्कायां दूर‚स्थित‚म‚पि \textbf{ग‚च्छ}द्वृ\textbf{क्ष‚द‚र्श‚नादि मिथ्याज्ञान‚मित्युक्तं भ‚व‚तीति} योज‚नीय‚म् ।‚{\tiny $_{lb}$}‚ \textbf{इतिना} उक्तेराकारं क‚थ‚य‚ति । \textbf{उक्तं} प्र‚काशितं \textbf{भ‚व}त्य‚भ्रान्त‚ग्र‚ह‚णेनेति प्र‚क‚र‚णात् । कुतो‚{\tiny $_{lb}$}‚ मिथ्याज्ञानं त‚दुक्त‚मित्य‚पेक्षायां प्र‚थ‚म‚मुक्तं \textbf{त‚दिति} योज‚नीय‚म् । हिर्य‚स्मात् त‚तोऽय‚म‚र्थः—‚{\tiny $_{lb}$}‚य‚स्मात् त‚द् ग‚च्छ‚द्वृक्ष‚द‚र्श‚नादि भ्रान्त‚त्वात् न प्र‚त्य‚क्ष‚म्, अतो मिथ्याज्ञान‚मित्य‚भ्रान्त‚ग्र‚ह‚णेनोक्तं‚{\tiny $_{lb}$}‚ भ‚व‚ति । भ‚व‚तु भ्रान्त‚त्वाद‚प्र‚त्य‚क्ष‚म्, अनुमान‚रूप‚त्वे त्व‚निव‚र्त्तिते क‚थ‚म‚स्य त‚थात्व‚मित्याश‚ङ्कायां‚{\tiny $_{lb}$}‚ योज्य‚म् । \textbf{चो} य‚स्मान्नानुमान‚म् । क‚थं नानुमान‚म् ? \textbf{त्रिरूप‚लिङ्ग‚ज‚वाभावात्} । य‚द्येव‚{\tiny $_{lb}$}‚म‚न्य‚त्प्र‚माणं भ‚विष्य‚ति, त‚थापि क‚थं मिथ्याज्ञान‚मित्याश‚ङ्कायां वाच्य‚म् । \textbf{चो} य‚स्मान्ना\textbf{भ्यां‚{\tiny $_{lb}$}‚ प्र‚माणान्त‚र‚म‚स्ति । अतः} प्र‚त्य‚क्ष‚त्वादिस्वाभाव‚त्वाभावात् । य‚द् वाऽतोऽभ्रान्त‚ग्र‚ह‚णादित्य‚र्थः ।
	\pend% ending standard par
      ‚{\tiny $_{lb}$}‚

	  \pstart \leavevmode% starting standard par
	न‚न्व‚नुमानादिरूप‚निराक‚र‚णे य‚द्य‚भ्रान्त‚ग्र‚ह‚ण‚स्य व्यापारो भ‚वेत्, भ‚वेदेव मिथ्याज्ञान‚त्वा‚{\tiny $_{lb}$}‚भिधाने साम‚र्थ्यं याव‚तो\edtext{}{\lemma{तो}\Bfootnote{ता}}य‚था स्व‚य‚मुप‚प‚त्त्यैवानुमानादिभावो व्युद‚स्त‚स्त‚त् क‚थ‚म‚भ्रान्त‚{\tiny $_{lb}$}‚‚{\tiny $_{lb}$}‚ \leavevmode\ledsidenote{\textenglish{46/dm}}‚{\tiny $_{lb}$}‚ 
	  
	य‚दि मिथ्याज्ञान‚म् क‚थं त‚तो वृक्षावाप्तिरिति चेत्, न त‚तो वृक्षावाप्तिः । नाना‚{\tiny $_{lb}$}‚देश‚गामी हि\edtext{}{\lemma{हि}\Bfootnote{०गामी वृक्षः--\cite{dp-msC}}} वृक्ष‚स्तेन प‚रिच्छिन्नः । एक‚देश‚निय‚त‚श्च वृक्षोऽवाप्य‚ते । त‚तो य‚द्देशो\edtext{}{\lemma{द्देशो}\Bfootnote{यो देशो य‚स्य--\cite{dp-msD-n}}}‚{\tiny $_{lb}$}‚ ग‚च्छ‚द्वृक्षो दृष्टः, त‚द्देशो नावाप्य‚ते । य‚द्देश‚श्चावाप्य‚ते\edtext{}{\lemma{ते}\Bfootnote{०प्य‚ते न स दृष्ट इति--\cite{dp-msC}}} स न दृष्ट इति न त‚स्मात् क‚श्चि‚{\tiny $_{lb}$}‚द‚र्थोऽवाप्य‚ते । ज्ञानान्त‚रादेव तु\edtext{}{\lemma{तु}\Bfootnote{देव च वृ० \cite{dp-msB} \cite{dp-msD}}} वृक्षादिर‚र्थोऽवाप्य‚ते । इत्येव‚म‚भ्रान्त‚ग्र‚ह‚णं विप्र‚तिप‚त्ति‚{\tiny $_{lb}$}‚निरासार्थ‚म् । ‚{\tiny $_{lb}$}‚ 
	  
	त‚था\edtext{}{\lemma{था}\Bfootnote{त‚थेत्यार‚भ्य विप्र‚तिप‚त्तिनिराक‚र‚णार्थ‚मित्य‚न्तः पाठो नास्ति--\cite{dp-msA} \cite{dp-edP} \cite{dp-edE} \cite{dp-edH}}} अभ्रान्त‚ग्र‚ह‚णेनाप्य‚नुमाने निव‚र्तिते क‚ल्प‚नापोढ‚ग्र‚ह‚णं \edtext{}{\lemma{णं}\Bfootnote{स‚विक‚ल्प‚क‚ज्ञान‚म‚पि प्र‚त्य‚क्ष‚मित्युक्तं \textbf{मीमांस‚कैः} य‚था अस्ति ह्यालोच‚ना \href{http://sarit.indology.info/?cref=śv-pratyakṣa.112}{मीमांसाश्लो० प्र‚त्य‚क्ष‚सूत्र--श्लो० ११२} इत्यादि । एत‚स्य विप्र‚तिप‚त्तेः--\cite{dp-msD-n}}}विप्र‚तिप‚त्तिनिरा-‚{\tiny $_{lb}$}‚ ग्र‚ह‚णेन त‚न्मिथ्याज्ञान‚मित्युक्तं भ‚व‚तीत्युच्य‚ते । स‚त्य‚म् । किन्त्व‚नुमानादिरूप‚तानिराक‚र‚ण‚{\tiny $_{lb}$}‚हेतुना द‚त्त‚साहाय‚केन स‚ताऽभ्रान्त‚ग्र‚ह‚णेन त‚न्मिथ्याज्ञान‚मित्युक्तं भ‚व‚तीत्युक्त‚मिति बोद्ध‚व्य‚म् ।
	\pend% ending standard par
      ‚{\tiny $_{lb}$}‚

	  \pstart \leavevmode% starting standard par
	न‚नु प्र‚त्य‚क्ष‚ल‚क्ष‚ण‚शून्य‚स्याप्र‚त्य‚क्ष‚तैव द‚र्श‚यित‚व्या, त‚त्किम‚नुमानादिरूप‚तानिराक‚र‚ण‚{\tiny $_{lb}$}‚म‚प्र‚कृतं कृत‚मिति चेत् । अप्र‚त्य‚क्ष‚त्व‚प्र‚द‚र्श‚ने प्र‚स‚ङ्गेन कृत‚मिति का क्ष‚तिः ?
	\pend% ending standard par
      ‚{\tiny $_{lb}$}‚

	  \pstart \leavevmode% starting standard par
	\textbf{य‚दीत्यादि} सुग‚म‚म् । केव‚लं \textbf{त‚त} इति ग‚च्छ‚द्वृक्ष‚द‚र्श‚न‚रूपान्मिथ्याज्ञानात्प\textbf{रिच्छिन्नो}‚{\tiny $_{lb}$}‚ दृष्टः । \textbf{वृक्ष} इति वृक्ष‚त्वेन प्र‚थ‚नादुच्य‚ते न त्व‚सौ त‚थाऽव‚भास‚मानो वृक्ष एव । \textbf{एक‚देश‚निय‚त}‚{\tiny $_{lb}$}‚ इ ति प्राप्य‚माण‚प‚र‚मार्थ‚वृक्षाभिप्रायेणोक्त‚म् । \textbf{चो} हेतौ । त‚त‚स्त‚स्माद‚न‚न्त‚रोक्तात् कार‚णात् ।‚{\tiny $_{lb}$}‚ \textbf{चो}ऽप्राप्य‚माणाद् भेद‚म‚स्य द‚र्श‚य‚ति । \textbf{इतिः} हेतौ । \textbf{त‚स्मा}न्मिथ्याज्ञानात् ।
	\pend% ending standard par
      ‚{\tiny $_{lb}$}‚

	  \pstart \leavevmode% starting standard par
	न‚नु त‚तोऽपि प्र‚वृत्त‚स्यास्ति च्छायाद्य‚र्थ‚क्रियाकारिरूप‚स्य पाद‚प‚स्य प्राप्तिस्त‚त्क‚थ‚म‚प‚ह्नूय‚त‚{\tiny $_{lb}$}‚ इति ? आह--\textbf{ज्ञानान्त‚राद्}--इति । त‚द्देशोप‚स‚र्प‚ण‚ज‚न्म‚नः स्थित‚वृक्ष‚प्र‚तिभासात्म‚नो \textbf{ज्ञानान्त‚रात्} ।‚{\tiny $_{lb}$}‚ त‚त्र मिथ्याज्ञान‚म‚र्थिनः प्र‚वृत्तिमात्र‚हेतुर्न तु वृक्ष‚प्राप‚क‚मिति संक्षेपार्थः । य‚दि मिथ्याज्ञान‚त्व‚प्र‚ति‚{\tiny $_{lb}$}‚पाद‚नार्थ‚म‚भ्रान्त‚ग्र‚ह‚णं न त‚र्हि विप्र‚तिप‚त्तिनिराक‚र‚णार्थ‚मित्याह--\textbf{एव‚म्} इति । \textbf{एव}मित्य‚नेना‚{\tiny $_{lb}$}‚न‚न्त‚रोक्त‚स्योप‚प‚त्तिप्र‚कार‚स्याकारो द‚र्शितः । त‚तोऽय‚म‚र्थः--एव‚म‚न‚न्त‚रोक्तेन युक्तिप्र‚कारेण ।‚{\tiny $_{lb}$}‚ एतादृश‚म‚पि प्र‚त्य‚क्ष‚मिति विरुद्धायाः प्र‚तिप‚त्तेर्निरासार्थ‚म‚भ्रान्त‚ग्र‚ह‚ण‚मिति । य‚द्येव‚म‚भ्रान्त‚{\tiny $_{lb}$}‚ग्र‚ह‚ण‚मेव विप्र‚तिप‚त्तिनिराक‚र‚णार्थ‚म् । न तु क‚ल्प‚नापोढ‚ग्र‚ह‚ण‚म् । उक्तं च द्व‚य‚मेत‚त् त‚न्निरासार्थ‚{\tiny $_{lb}$}‚मित्याश‚ङ्क्याह--\textbf{त‚थे}ति । य‚थाऽभ्रान्त‚प‚दं त‚थेद‚मित्य‚र्थः ।
	\pend% ending standard par
      ‚{\tiny $_{lb}$}‚

	  \pstart \leavevmode% starting standard par
	\textbf{विप्र‚तिप‚तिनिराक‚र‚णार्थ‚मि}ति प्र‚त्य‚क्ष‚पृष्ठ‚भाविनोऽपि विक‚ल्प‚स्य व्याव‚हारिक‚लोका‚{\tiny $_{lb}$}‚ध्य‚व‚सायेनाभ्रान्त‚स्य य‚त्प्र‚त्य‚क्ष‚त्वं कैश्चिदिष्ट‚म्, त‚न्नि\leavevmode\ledsidenote{\textenglish{21a/ms}}राक‚र‚णार्थ‚मिति द्र‚ष्ट‚व्य‚म् ।‚{\tiny $_{lb}$}‚ त‚त्रास‚ति क‚ल्प‚नापोढ‚ग्र‚ह‚णे घ‚टोऽय‚मित्यादिज्ञानं प्र‚त्य‚क्ष‚म‚भ्रान्त‚त्वात् स्यात् । त‚तो हि प्र‚वृत्तेन‚{\tiny $_{lb}$}‚ घ‚टादिर‚र्थः प्राप्य‚त इति संवाद‚क‚त्वान्स‚म्य‚ग्ज्ञान‚म‚भ्रान्त‚त्वाच्च प्र‚त्य‚क्षं स्यादा \add{दित्या} श‚ङ्का ।‚{\tiny $_{lb}$}‚ त‚न्निवृत्त्य‚र्थ‚म् । क‚ल्प‚नात्म‚क‚त्वान्न प्र‚त्य‚क्ष‚म्, त्रिरूप‚लिङ्ग‚ज‚त्वाभावाच्च नानुमान‚म् ।‚{\tiny $_{lb}$}‚ ‚{\tiny $_{lb}$}‚ \leavevmode\ledsidenote{\textenglish{47/dm}}‚{\tiny $_{lb}$}‚ 
	  
	क‚र‚णार्थ‚म्\edtext{}{\lemma{म्}\Bfootnote{विप्र‚तिप‚त्तिनिरासार्थ‚म्--\cite{dp-msB} \cite{dp-msC} \cite{dp-msD} \cite{dp-edN}}} । भ्रान्तं हि अनुमानं स्व‚प्र‚तिभासेऽन‚र्थे\edtext{}{\lemma{र्थे}\Bfootnote{सामान्ये--\cite{dp-msD-n}}}ऽर्थाध्य‚व‚सायेन प्र‚वृत्त‚त्वात् । प्र‚त्य‚क्षं‚{\tiny $_{lb}$}‚ तु ग्राह्ये रूपे न विप‚र्य‚स्त‚म् । ‚{\tiny $_{lb}$}‚ 
	  
	न तु अविसंवाद‚क‚म‚भ्रान्त‚मिह ग्र‚हीत‚व्य‚म् । य‚तः स‚म्य‚ग्ज्ञान‚मेव प्र‚त्य‚क्ष‚म्, नान्य‚त् ।‚{\tiny $_{lb}$}‚ त‚त्र स‚म्य‚ग्ज्ञान‚त्वादेवाऽविसंवाद‚क‚त्वे ल‚ब्धे पुन‚र‚विसंवाद‚क\edtext{}{\lemma{क}\Bfootnote{०वाद‚ग्र‚ह० \cite{dp-edE} ०वाद‚क‚त्व‚ग्र‚ह \cite{dp-msC}}}ग्र‚ह‚णं निष्प्र‚योज‚न‚मेव । एवं हि‚{\tiny $_{lb}$}‚ वाक्यार्थः स्यात्--प्र‚त्य‚क्षाख्यं य‚द‚विसंवाद‚कं ज्ञानं त‚त् क‚ल्प‚नापोढ‚म‚विसंवाद‚कं चेति । न‚{\tiny $_{lb}$}‚ चानेन द्विर‚विसंवाद‚क‚ग्र‚ह‚णेन किञ्चित्\edtext{}{\lemma{किञ्चित्}\Bfootnote{न किञ्चित् प्र‚योज‚न‚म्--\cite{dp-msD-n}}} । त‚स्माद् ग्राह्येऽर्थ‚क्रियाक्ष‚मे व‚स्तुरूपे य‚द‚विप‚र्य‚स्तं‚{\tiny $_{lb}$}‚ त‚द‚भ्रान्त‚मिह वेदित‚व्य‚म् ॥ ‚{\tiny $_{lb}$}‚ 
	  
	कीदृशी पुनः क‚ल्प‚नेह गृह्य‚त इत्याह-- ‚{\tiny $_{lb}$}‚ 
	  
	अभिलाप‚संस‚र्ग‚योग्य‚प्र‚तिभासा\edtext{}{\lemma{तिभासा}\Bfootnote{०भास‚प्र० \cite{dp-msB} \cite{dp-edP} \cite{dp-edE} \cite{dp-edH} \cite{dp-edN}}} प्र‚तीतिः क‚ल्प‚ना ॥ ५ ॥‚{\tiny $_{lb}$}‚ 
	  
	\edtext{\textsuperscript{*}}{\lemma{*}\Bfootnote{अभिलापेति--\cite{dp-edE} \cite{dp-edP} अभिलापेत्यादि इति नास्ति--\cite{dp-msA} \cite{dp-msB} \cite{dp-edH} \cite{dp-edN}}}अभिलापेत्यादि । \edtext{\textsuperscript{*}}{\lemma{*}\Bfootnote{न‚नु न[[च]]य‚द्य‚पि त‚स्मिन् ज्ञाने आकार‚योर्मिल‚नं त‚थापि श‚ब्दार्थ‚योः संस‚र्गो‚{\tiny $_{lb}$}‚ नास्तीत्याह--\cite{dp-msD-n}}}अभिल\edtext{}{\lemma{अभिल}\Bfootnote{अभिलाप्य‚ते--\cite{dp-msA} \cite{dp-edP} \cite{dp-edH} \cite{dp-edE}}}प्य‚तेऽनेनेति अभिलापः \textbf{वाच‚कः श‚ब्दः । अभिलापेन‚{\tiny $_{lb}$}‚ संस‚र्गः}--\edtext{\textsuperscript{*}}{\lemma{*}\Bfootnote{नास्ति--अभिलाप‚संस‚र्गः इति--\cite{dp-msA} \cite{dp-edP} \cite{dp-edE}}}अभिलाप‚संस‚र्गः--\textbf{एक‚स्मिन् ज्ञानेऽभिधेयाकार‚स्याभिधानाकारेण स‚ह\edtext{}{\lemma{ह}\Bfootnote{वाच्य‚रूप‚त्वेन--\cite{dp-msD-n}}} ग्राह्याकार‚त‚या}‚{\tiny $_{lb}$}‚ न च प्र‚माणान्त‚र‚म‚स्ति । अतो घ‚टोऽय‚मित्यादिज्ञान‚म‚स‚म्य‚ग्ज्ञानं भ‚व‚तीति पूर्व‚व‚द् व‚च‚नीयं‚{\tiny $_{lb}$}‚ योज‚नीयं च ।
	\pend% ending standard par
      ‚{\tiny $_{lb}$}‚

	  \pstart \leavevmode% starting standard par
	न‚नु च संवादिनोऽस्य क‚थ‚म‚स‚म्य‚ग्ज्ञान‚त्व‚म् ? य‚दि नाम विसंवाद‚क‚त्व‚ल‚क्ष‚ण‚म् अस‚म्य‚{\tiny $_{lb}$}‚ग्ज्ञान‚त्वं नास्ति त‚थापि गृहीतार्थ‚ग्राहिणोऽस्यापूर्वाधिग‚माभावात् क‚र‚णार्थाभाव‚रूप‚म‚स‚म्य‚ग्ज्ञान‚त्वं‚{\tiny $_{lb}$}‚ स्म‚र‚णादेरिव किं नानुम‚न्य‚ते ? जात्यादिविशिष्ट‚व‚स्तुग्राहिणोऽस्यापूर्वार्थाधिग‚मोऽस्तीति चेत् ।‚{\tiny $_{lb}$}‚ इद‚म‚न्य‚त्र विस्त‚रेण निर‚स्त‚मित्यास्तां ताव‚दिहेति । \textbf{स्व‚प्र‚तिभास} इत्यादेर्ग्र‚न्थ‚स्य तु स‚त्य‚म‚र्थं‚{\tiny $_{lb}$}‚ विष‚य‚विप्र‚तिप‚त्तिनिराक‚र‚ण‚वाक्य‚विव‚र‚णं विवेच‚यिष्य‚न्तो विवेच‚यिष्यामः । स‚म्प्र‚ति य‚द्‚{\tiny $_{lb}$}‚ दोष‚द‚र्श‚नात्पूर्वेषां व्याख्यान‚म‚व‚म‚न्यान्य‚थाऽय‚म‚भ्रान्तार्थं व्याच‚ष्टे तं क‚ण्ठोक्तं क‚र्त्तुं तेषाम‚भिम‚त‚{\tiny $_{lb}$}‚म‚भ्रान्तार्थं पूर्वं साम‚र्थ्यान्निषिद्ध‚म‚पि साक्षान्निषेध‚न्नाह--\textbf{न तु}--इति । \textbf{तु}र‚तिश‚ये । \textbf{य‚त}‚{\tiny $_{lb}$}‚ इत्यादि \textbf{वेदित‚व्य‚मि}त्य‚न्तं सुबोध‚म् ।
	\pend% ending standard par
      ‚{\tiny $_{lb}$}‚

	  \pstart \leavevmode% starting standard par
	य‚दि जात्यादियोज‚नात्मिका क‚ल्प‚ना । सा जात्याद्य‚भावादेव न स‚म्भ‚व‚ति । अथ ग्राह्य‚ग्राह‚क‚{\tiny $_{lb}$}‚भावेन प्र‚व‚र्त्त‚मानं ज्ञानं क‚ल्प‚ना त‚दा स‚र्व‚म‚स‚र्व‚ज्ञ‚ज्ञानं त‚था प्र‚वृत्त‚मिति किम‚व‚शिष्य‚ते य‚द‚विक‚ल्प‚कं‚{\tiny $_{lb}$}‚ स्यादित्य‚भिप्रेत्य क‚ल्प‚नायाः स्व‚रूपं पृच्छ‚ति--\textbf{कीदृशीति} सामान्य‚तः पृच्छ‚ति । \textbf{पुन}रिति विशेष‚तः ।‚{\tiny $_{lb}$}‚ \textbf{इहे}ति प्र‚त्य‚क्ष‚ल‚क्ष‚णे ।
	\pend% ending standard par
      ‚{\tiny $_{lb}$}‚

	  \pstart \leavevmode% starting standard par
	अभिल‚प्य‚ते उच्य‚तेऽ\textbf{नेनेत्य‚भिलापो वाच‚कः श‚ब्दः} । स च स‚ङ्केत-\textbf{मिल‚न‚म्} ।
	\pend% ending standard par
      ‚{\tiny $_{lb}$}‚‚{\tiny $_{lb}$}‚\textsuperscript{\textenglish{48/dm}}‚{\tiny $_{lb}$}‚
	  \bigskip
	  \begingroup
	

	  \pstart \leavevmode% starting standard par
	त‚तो य‚दैक‚स्मिञ्ज्ञानेऽभिधेया\edtext{}{\lemma{स्मिञ्ज्ञानेऽभिधेया}\Bfootnote{०ज्ञानेऽभिधानाभिधेय‚योः--\cite{dp-msB}}}भिधान‚योराकारौ\edtext{}{\lemma{योराकारौ}\Bfootnote{०कारौ निविष्टौ \cite{dp-msB}}} संनिविष्टौ भ‚व‚त‚स्त‚दा संसृष्टे अभिधा‚{\tiny $_{lb}$}‚नाभिधेये भ‚व‚तः । अभिलाप‚संस‚र्गाय योग्योऽभिधेयाकार\edtext{}{\lemma{योग्योऽभिधेयाकार}\Bfootnote{रा}} भासो\edtext{}{\lemma{भासो}\Bfootnote{अभिधेयाभासो \cite{dp-msA} \cite{dp-edP} \cite{dp-edH} \cite{dp-edE} अभिधेयाकाराभासो--\cite{dp-edN}}} य‚स्यां प्र‚तीतौ‚{\tiny $_{lb}$}‚ सा त‚थोक्ता ।
	\pend% ending standard par
       ‚{\tiny $_{lb}$}‚ 

	  \pstart \leavevmode% starting standard par
	त‚त्र काचित् प्र‚तीतिर‚भिलाप\edtext{}{\lemma{भिलाप}\Bfootnote{०लापेन संसृ० \cite{dp-msA} \cite{dp-edP} \cite{dp-edH} \cite{dp-edE} \cite{dp-edN}}} संसृष्टाभासा भ‚व‚ति । य‚था व्युत्प‚न्न‚स‚ङ्केत‚स्य घ‚टार्थ‚{\tiny $_{lb}$}‚क‚ल्प‚ना घ‚ट‚श‚ब्द‚संसृष्टार्थाव‚भासा \edtext{}{\lemma{भासा}\Bfootnote{भ‚व‚ति इति नास्ति \cite{dp-msB} \cite{dp-msD}}}भ‚व‚ति । काचित्त्व‚भिलापेनासंसृष्टापि अभिलाप‚संस‚र्ग‚{\tiny $_{lb}$}‚योग्याभासा भ‚व‚ति । य‚था बाल‚क‚स्याव्युत्प‚न्न‚स‚ङ्केत‚स्य क‚ल्प‚ना । त‚त्र अभिलाप‚संसृष्टाभासा\edtext{}{\lemma{संसृष्टाभासा}\Bfootnote{०सृष्ट‚प्र‚तिभासा \cite{dp-msD} संसृष्ट‚प्र\add{ति}भासा \cite{dp-msB}}}‚{\tiny $_{lb}$}‚ क‚ल्प‚ना इत्युक्ताव‚व्युत्प‚न्न‚स‚ङ्केत‚स्य\edtext{}{\lemma{स्य}\Bfootnote{नास्ति क‚ल्प‚ना इति \cite{dp-msA} \cite{dp-msB} \cite{dp-edP} \cite{dp-edH} \cite{dp-edE} \cite{dp-edN}}} क‚ल्प‚ना न संगृह्येत\edtext{}{\lemma{संगृह्येत}\Bfootnote{संगृह्य‚ते \cite{dp-msC}}} । योग्य‚ग्र‚ह‚णे\edtext{}{\lemma{णे}\Bfootnote{०ग्र‚ह‚णेन तु \cite{dp-msC}}} तु\edtext{}{\lemma{तु}\Bfootnote{बाल‚क‚ल्प‚ना--\cite{dp-msD-n}}}सापि संगृह्य‚ते ।‚{\tiny $_{lb}$}‚ य‚द्य‚पि अभिलाप‚संसृष्टाभासा\edtext{}{\lemma{संसृष्टाभासा}\Bfootnote{०सृष्ट‚प्र‚तिभासा \cite{dp-msC} \cite{dp-msA}}} न भ‚व‚ति त‚द‚ह‚र्जात‚स्य\edtext{}{\lemma{स्य}\Bfootnote{नास्ति बाल‚क‚स्य इति \cite{dp-msA} \cite{dp-edP} \cite{dp-edE}}} बाल‚क‚स्य क‚ल्प‚ना, अभिलाप‚संस‚र्ग‚{\tiny $_{lb}$}‚योग्य‚प्र‚तिभासा तु भ‚व‚त्येव । या चाभिलाप‚संसृष्टा सापि योग्या । त‚त उभ‚योर‚पि योग्य‚{\tiny $_{lb}$}‚ग्र‚ह‚णेन संग्र‚हः ।
	\pend% ending standard par
      
	  \endgroup
	‚{\tiny $_{lb}$}‚

	  \pstart \leavevmode% starting standard par
	कालेन दृष्टेन रूपेणैक्य‚मापादितोऽन्त‚र्ज‚ल्पाकार‚स्त‚था प्र‚तिभास‚मानो वाच्यः । \textbf{संस‚र्ग}‚{\tiny $_{lb}$}‚प‚देन सार्ध‚म‚स्य विग्र‚हं प्राह--\textbf{अभिलापेन}--इति । \textbf{अभिलापेन} वाच‚केन \textbf{संस‚र्गः} स‚म्ब‚न्धः ।‚{\tiny $_{lb}$}‚ तृतीये\href{http://sarit.indology.info/?cref=Pā.2.1.30}{पाणिनि--२. १. ३०}ति योग‚विभागात्स‚मासः । अथ‚वाऽर्थ‚क‚थ‚न‚मिदं कृत‚म्, स‚मास‚स्तु‚{\tiny $_{lb}$}‚ ष‚ष्ठीत‚त्पुरुषः कार्यः । क‚थं पुन‚र्वाच्य‚वाच‚क‚योः संस‚र्गः स‚म्भ‚व‚तीत्याह--\textbf{एक‚स्मिन्} इति ।‚{\tiny $_{lb}$}‚ \textbf{ग्राह्याकार‚त‚या मिल‚न‚मि}त्येक‚ज्ञान‚ग्राह्याकार‚त‚याऽव‚स्थान‚मिति वाच्य‚म् । अर्थाकार‚श्चेद् वाच‚क‚{\tiny $_{lb}$}‚श‚ब्दाकारेण स‚हैक‚ज्ञानारूढो भ‚व‚ति त‚दा वाच्य‚वाच‚क‚योः संस‚र्ग इति याव‚त् । भ‚व‚त्व‚भिधानाभि‚{\tiny $_{lb}$}‚धेयाकार‚योरेक‚ज्ञानारोहः । किमेताव‚ताऽभिधानाभिधेये वाच्य‚वाच‚क‚त‚या स‚म्ब‚द्धे भ‚व‚तः, येन‚{\tiny $_{lb}$}‚ त‚द्ग्राहि विज्ञानं स‚विक‚ल्प‚कं स्यादित्याश‚ङ्क्योप‚संहार‚व्याजेनाह--\textbf{त‚त} इति । \textbf{य‚दाकारौ‚{\tiny $_{lb}$}‚ संन्निविष्टौ} प्र‚तिभासितौ \textbf{भ‚व‚त‚स्त‚दा संसृष्टे} त‚थात‚या स‚म्ब‚द्धे \textbf{भ‚व‚तः} । एक‚स्मिन् ज्ञाने विशेष्य‚{\tiny $_{lb}$}‚विशेष‚ण‚त्वेनाकार‚प्र‚तिभास एवान‚योर्वाच्य‚वाच‚क‚त्व‚ग्र‚ह‚ण‚मिति भावः । \textbf{अभिलाप‚संस‚र्गाय योग्य}‚{\tiny $_{lb}$}‚ इत्य‚र्थ‚क‚थ‚न‚मिदं \leavevmode\ledsidenote{\textenglish{21b/ms}} कृत‚म् । स‚मास‚स्त्व‚भिलाप‚संस‚र्ग‚स्य योग्य इति क‚र्त्त‚व्यः । ताद‚र्थ्य‚{\tiny $_{lb}$}‚च‚तुर्थ्याः प्र‚कृतिविकार एव स‚मासात् ।
	\pend% ending standard par
      ‚{\tiny $_{lb}$}‚

	  \pstart \leavevmode% starting standard par
	य‚द्येव‚म‚भिलाप‚संसृष्ट‚प्र‚तिभासेत्येवास्तु किं योग्य‚ग्र‚ह‚णेनेत्युपाल‚म्भं प‚श्य‚न् य‚थाऽस्य‚{\tiny $_{lb}$}‚ साफ‚ल्यं त‚था द‚र्श‚यितुमुप‚क्र‚म‚ते--त‚त्रेति । वाक्योप‚न्यासे चैत‚त् । अभिलाप‚संसृष्ट आभासो‚{\tiny $_{lb}$}‚ य‚स्याः सा त‚था । क‚स्येवेत्याह--\textbf{य‚थेति} । इद‚म‚स्य वाच्य‚मिद‚म‚स्य वाच‚क‚म् इत्युभ‚यांशाव‚{\tiny $_{lb}$}‚‚{\tiny $_{lb}$}‚ ‚{\tiny $_{lb}$}‚ \leavevmode\ledsidenote{\textenglish{49/dm}}‚{\tiny $_{lb}$}‚ 
	  
	अस‚त्य‚भिलाप‚संस‚र्गे कुतो योग्य‚ताव‚सितिरिति\edtext{}{\lemma{सितिरिति}\Bfootnote{योग्य‚त्वाव‚सितिः \cite{dp-msC} \cite{dp-msD} \cite{dp-msB}}} चेत् । अनिय‚त‚प्र‚तिभास‚त्वात् ।‚{\tiny $_{lb}$}‚ \edtext{\textsuperscript{*}}{\lemma{*}\Bfootnote{न‚नु च विक‚ल्प‚ज्ञान‚स्यानिय‚त‚प्र‚तिभासित्व‚मेव न सिद्ध‚म्--\cite{dp-msD-n}}}अनिय‚त‚प्र‚तिभास‚त्वं च प्र‚तिभास‚निय‚म‚हेतोर‚भावात् । ग्राह्यो ह्य‚र्थो विज्ञानं ज‚न‚य‚न्निय‚त‚{\tiny $_{lb}$}‚प्र‚तिभासं कुर्यात् । य‚था रूपं च‚क्षुर्विज्ञानं ज‚न‚य‚न्निय‚त‚प्र‚तिभासं ज‚न‚य‚ति । विक‚ल्प‚विज्ञानं‚{\tiny $_{lb}$}‚ त्व‚र्थान्नोत्प‚द्य‚ते । त‚तः प्र‚तिभास‚निय‚म‚हेतोर‚भावाद‚निय‚त‚प्र‚तिभास‚म् ।‚{\tiny $_{lb}$}‚ ल‚म्बिज्ञानं स‚ङ्केतः, श‚ब्दार्थ‚प्र‚योग‚प्र‚तिप‚त्त्योः कार्य‚कार‚ण‚भाव‚रूपो वा । व्युत्प‚न्नो ज्ञातः‚{\tiny $_{lb}$}‚ संकेतो येन स \textbf{व्युत्प‚न्न‚संकेतः} । त‚स्य य‚दि स‚र्वैव तादृशी त‚दा काचिदिति न वाच्य‚म्,‚{\tiny $_{lb}$}‚ योग्य‚ग्र‚ह‚णेन च न किञ्चिदित्याह--\textbf{काचिद्} इति । \textbf{तुः} व्युत्प‚न्न‚संकेत‚प्र‚तीतेर्बाल‚प्र‚तीतिं भेद‚{\tiny $_{lb}$}‚व‚तीमाह । क‚स्य स‚दृशी स‚म्भ‚व‚तीत्याह--\textbf{य‚थे}ति । बाल‚क इत्य‚ल्पार्थे क‚न् । य‚द्य‚भिलाप‚{\tiny $_{lb}$}‚संसृष्टाभासेति कृतेऽपि त‚थाभूतायाः प्र‚तीतेः स‚ङ्ग्र‚हः, कृतं त‚र्हि योग्य‚ग्र‚ह‚णेनेत्याह--\textbf{त‚त्रे}ति ।‚{\tiny $_{lb}$}‚ त‚योर्म‚ध्येऽ\textbf{व्युत्प‚न्न‚संकेत‚स्य} या सा न स‚ङ्गृह्य‚ते । क‚दा तु स‚ङ्गृह्य‚तेत्याह--\textbf{अभिलापे}ति । अर्थ‚{\tiny $_{lb}$}‚प्र‚द‚र्श‚न‚त्वाद‚स्या\textbf{भिलाप‚संसृष्टाभासा} प्र‚तीतिः \textbf{क‚ल्प‚नेत्युक्तौ} व‚च‚ने स‚तीत्य‚र्थः । इतिनोक्तेराकारः‚{\tiny $_{lb}$}‚ क‚थितः । अस‚ति ताव‚द‚यं दोषः । य‚दि स‚त्य‚पि योग्य‚ग्र‚ह‚णे त‚द‚संग्र‚ह‚स्त‚थापि किं तेनेत्याह—‚{\tiny $_{lb}$}‚\textbf{योग्ये}ति । \textbf{तु}र‚क‚र‚णाव‚स्थाया विशेष‚माह । न केव‚लं त‚त्संसृष्टाभासेत्यापेश‚ब्दः । त‚था‚{\tiny $_{lb}$}‚विध‚बाल‚स्य च क‚ल्प‚नाऽनुमान‚सिद्धा । तेन साऽव‚श्यं ग्राह्येति भावः । क‚थं पुन‚र्योग्य‚ग्र‚ह‚णेन‚{\tiny $_{lb}$}‚ त‚स्याः स‚ङ्ग्र‚ह इत्याह--\textbf{य‚द्य‚पी}ति । निपात‚स‚मुदायोऽयं विशेषाभिधानार्थाभ्युप‚ग‚मे स‚र्व‚त्र ।‚{\tiny $_{lb}$}‚ \textbf{अह्नि जातं} ज‚न्मास्येति त‚था । अह्नि वा जात उत्प‚न्न इति त‚था । अह‚र्जात‚स्यैत‚द‚ह‚{\tiny $_{lb}$}‚र्जात‚स्येति द्र‚ष्ट‚व्य‚म् । य‚द्येवं व्युत्प‚न्न‚स‚ङ्केत‚स्य सा क‚थं स‚ङ्गृह्य‚त इत्याह--\textbf{या चे}ति । \textbf{च}‚{\tiny $_{lb}$}‚ पूर्वापेक्ष‚या स‚मुच्च‚ये । य‚दि सा त‚त्संस‚र्ग‚योग्या न स्यात्क‚थं त‚त्संस‚र्ग‚म‚नुभ‚वेदिति भावः । य‚त‚{\tiny $_{lb}$}‚ एवं त‚स्मात् ।
	\pend% ending standard par
      ‚{\tiny $_{lb}$}‚

	  \pstart \leavevmode% starting standard par
	न‚नु प्र‚तीत‚प्र‚तिभास‚स्य त‚त्संस‚र्गानुभ‚वान्य‚थानुप‚प‚त्त्या योग्य‚ता क‚ल्प‚नीया । न च‚{\tiny $_{lb}$}‚ बाल‚क‚ल्प‚नाया अभिलाप‚संस‚र्गोऽस्ति, संकेताव्युत्प‚त्तेः । त‚त्क‚थं योग्य‚ता क‚ल्प्य‚तामित्य‚भि‚{\tiny $_{lb}$}‚स‚न्धिना \textbf{अस‚तीत्या}दिना पूर्व‚प‚क्ष‚मुत्थाप‚य‚ति । एवं चोद‚यित्वाऽनुमान‚तः सा सिद्ध्य‚तीति‚{\tiny $_{lb}$}‚ म‚न्वानः साध‚न‚माह--\textbf{अनिय‚तेति} । प्र‚तिनिय‚ताकार‚त्वाभावादित्य‚र्थः । प्र‚क‚र‚णाप‚न्ने साध्ये‚{\tiny $_{lb}$}‚ हेतौ चाभिहिते सुक‚रः साध‚न‚प्र‚योग इति प्र‚योगं न ज‚गौ । एव‚मुत्त‚र‚त्रापि हेतुमात्राभिधानेऽ‚{\tiny $_{lb}$}‚स्याय‚म‚भिप्रायः प्र‚त्येत‚व्यः । प्र‚योग‚स्त्वेव‚मिह क‚र्त्त‚व्यः--योऽनिय‚त‚प्र‚तिभासो विक‚ल्पः, स‚{\tiny $_{lb}$}‚ श‚ब्द‚संस‚र्ग‚योग्यो य‚थाच्छात्र‚विक‚ल्पः । अनिय‚त‚प्र‚तिभास‚श्च बाल‚विक‚ल्प इति भ‚व‚त्येव प्र‚योगः ।‚{\tiny $_{lb}$}‚ केव‚ल‚मिद‚म‚त्र निरूप्य‚ताम् । बाल‚स्य विक‚ल्प एव कुतः सिद्धो येनेद‚मुत्त‚राणि चैत‚द‚ङ्गानि‚{\tiny $_{lb}$}‚ साध‚नानि ना\leavevmode\ledsidenote{\textenglish{22a/ms}}श्र‚यासिद्धानि स्युरिति ? नैष दोषः । अनुमान‚सिद्धं विक‚ल्पं ध‚र्मिणं कृत्वाऽ‚{\tiny $_{lb}$}‚मीषामुपादान‚स्याभिप्रेत‚त्वात् । त‚त्पुन‚र‚नुमान‚म‚नेन सुज्ञात‚त्वान्नोप‚न्य‚स्त‚मिति वेदित‚व्य‚म् ।‚{\tiny $_{lb}$}‚ एवं त‚द् द्र‚ष्ट‚व्य‚म्--या निय‚म‚व‚ती प्र‚वृत्तिः क्व‚चित्प्राणिनः, सा विक‚ल्प‚पूर्विका । य‚था‚{\tiny $_{lb}$}‚ व्युत्प‚न्न‚स‚ङ्केत‚व्य‚व‚हार‚स्यान्नादिविष‚या प्र‚वृत्तिः । निय‚म‚व‚ती च त‚दित‚र‚प‚रिहारेण स्त‚नादौ‚{\tiny $_{lb}$}‚ प्र‚वृत्तिर्बाल‚क‚स्येति कार्य‚हेतुः । सा च तादृशी प्र‚वृत्तिः प्र‚त्य‚क्षाधिग‚तेति स्व‚सिद्धौ न प्र‚माणान्त‚रं‚{\tiny $_{lb}$}‚ ‚{\tiny $_{lb}$}‚ \leavevmode\ledsidenote{\textenglish{50/dm}}‚{\tiny $_{lb}$}‚ 
	  
	कुतः पुन‚रेत‚द्विक‚ल्पोऽर्थान्नोत्प‚द्य‚त इति? \edtext{\textsuperscript{*}}{\lemma{*}\Bfootnote{उत्त‚र‚माह--\cite{dp-msD-n}}}अर्थ‚स‚न्निधिनिर‚पेक्ष‚त्वात् । बालोऽपि हि‚{\tiny $_{lb}$}‚ याव‚द् दृश्य‚मानं स्त‚नं स एवाय‚म् इति पूर्व‚दृष्ट‚त्वेन न प्र‚त्य‚व‚मृश‚ति ताव‚न्नोप‚र‚त‚रुदितो‚{\tiny $_{lb}$}‚ प्र‚योज‚य‚ति । न पुन‚र‚नेन \textbf{बालोऽपि ही}त्यादिना विक‚ल्प‚साध‚न‚मुप‚न्य‚स्त‚मिति म‚न्त‚व्य‚म्,‚{\tiny $_{lb}$}‚ य‚थाऽन्यैर्व्याख्यात‚म्, अर्थ‚स‚न्निधिनिर‚पेक्ष‚त्व‚सिद्धिप्र‚द‚र्श‚नोन्मुख‚त्वात्त‚स्य ग्र‚न्थ‚स्येति ।
	\pend% ending standard par
      ‚{\tiny $_{lb}$}‚

	  \pstart \leavevmode% starting standard par
	न‚नु च प‚रोक्ष‚त्वात्त‚स्यानिय‚त‚प्र‚तिभास‚त्व‚स‚न्देहे स‚न्दिग्धासिद्धोऽयं हेतुरिति । आह—‚{\tiny $_{lb}$}‚\textbf{अनिय‚त‚प्र‚तिभास‚त्वं च}--इति । \textbf{चो} य‚स्मादित्य‚स्मिन्न‚र्थे । \textbf{प्र‚तिभास‚स्य} ज्ञानाकार‚स्य \textbf{निय‚मः}—‚{\tiny $_{lb}$}‚अर्थाकार एवायं श‚ब्दाकार एव वा, रूपाकार एवायं र‚साकार एव वेत्यात्म‚को वा त‚स्य‚{\tiny $_{lb}$}‚ हेतुर्ज‚न‚कः त‚स्माद् । \textbf{अभावा}द‚नुत्पादात् । भ‚व‚नं भाव उत्पाद‚स्त‚न्निषेधाद् अभावोऽनुत्पाद‚{\tiny $_{lb}$}‚ एव । प्र‚तिभास‚निय‚म‚हेतोर‚भावाद्--अविद्य‚मान‚त्वादिति व्याख्याने तु य‚थाश्रुति व्य‚धिक‚र‚णा‚{\tiny $_{lb}$}‚सिद्धो हेतुः स्यात् । अन्य‚था हेतुव‚च‚ने तु क्रिय‚माणे त‚त्रार्थे हेतुर‚न‚भिहित एव स्यात् । साध‚न‚{\tiny $_{lb}$}‚स्व‚रूप‚मात्र‚क‚थ‚ने चात्र \textbf{घ‚र्मोत्त‚र‚स्य} शैली ल‚क्ष्य‚ते । \textbf{कुतः पुन‚रेत‚द् विक‚ल्पोऽर्थान्नोत्प‚द्य‚त} इति‚{\tiny $_{lb}$}‚ च पूर्व‚प‚क्षः सूत्थानो न स्यात् । मृत्वा शीर्त्वा य‚था क‚थ‚ञ्चित् स‚र्व‚स्यास्य स‚म‚र्थ‚ने च व‚क्तु‚{\tiny $_{lb}$}‚र‚कौश‚लं स्यादिति ।
	\pend% ending standard par
      ‚{\tiny $_{lb}$}‚

	  \pstart \leavevmode% starting standard par
	न‚नु प्र‚तिभास‚निय‚म‚हेतुर्यः क‚श्च‚न स च त‚त्र भ‚विष्य‚ति । अथ विशिष्टः ।‚{\tiny $_{lb}$}‚ व‚क्त‚व्य‚स्त‚दाऽसौ य‚तः स‚काशाद‚नुत्पादात्त‚थात्वं ज्ञात‚व्य‚मित्याह--\textbf{ग्राह्य} इति । य‚द्य‚र्थ इत्येव‚{\tiny $_{lb}$}‚ क्रिय‚ते त‚देन्द्रिय‚म‚प्य‚र्थो ज्ञानं ज‚न‚य‚तीति त‚स्याप्याकार‚नियाम‚क‚त्वं स्यात् । न चापोद्धार‚{\tiny $_{lb}$}‚द्वारेण त‚स्याकार‚विशेष‚हेतुत्वं व्य‚व‚स्थापित‚म्, अपि तु विष‚य‚ग्र‚ह‚ण‚प्र‚तिनिय‚म‚हेतुत्व‚मिति \textbf{ग्राह्य}‚{\tiny $_{lb}$}‚ग्र‚ह‚ण‚म् । \textbf{ग्राह्य} आल‚म्ब‚नः । य‚द्येवं ग्राह्य इत्येवास्तु । अपोह‚स्यापि ग्राह्य‚ताऽस्ति ।‚{\tiny $_{lb}$}‚ न चाऽसौ प्र‚तिभास‚निय‚म‚हेतुः । अत‚स्त‚न्निवृत्त्य‚र्थ‚म‚र्थोऽर्थ‚क्रियास‚म‚र्थ इति कृत‚म् । \textbf{कुर्यात्} क‚र्तुं‚{\tiny $_{lb}$}‚ श‚क्नोति । अपोद्धारेण त‚त्रैव त‚च्छ‚क्तेर‚व‚धृत‚त्वात् । अत्र निद‚र्श‚नं \textbf{य‚थे}ति । \textbf{निय‚त‚प्र‚ति‚{\tiny $_{lb}$}‚भास}मिति श‚ब्दाकार‚प‚रिहारेणार्थाकार‚धार्येव र‚साकार‚प‚रिहारेण रूपाकार‚धार्येव चेति । य‚द्येवं‚{\tiny $_{lb}$}‚ बाल‚विक‚ल्पोऽपि अर्थादेवोत्प‚त्स्य‚ते, त‚तो निय‚त‚प्र‚तिभासो भ‚विष्य‚तीत्याश‚ङ्क्य पूर्वोक्त‚मेव प्र‚स‚ङ्गेन‚{\tiny $_{lb}$}‚ स्मार‚य‚ति--\textbf{विक‚ल्पे}ति । \textbf{तु}रुदाहृताद् विज्ञान भिन‚त्ति । एवं तु प्र‚योगः कार्यः--य‚द‚र्था‚{\tiny $_{lb}$}‚न्नोत्प‚द्य‚ते ज्ञानं त‚न्निय‚त‚प्र‚तिभासं न भ‚व‚ति । य‚था व्युत्प‚न्न‚व्य‚व‚हार‚स्यातीतादिस्म‚र‚ण‚म्,‚{\tiny $_{lb}$}‚ रूप‚र‚सादिस‚ङ्क‚ल‚नाज्ञानं वा । नोत्प‚द्य‚ते चा\leavevmode\ledsidenote{\textenglish{22b/ms}}र्थाद् बाल‚क‚स्य विक‚ल्प इति व्याप‚कानुप‚{\tiny $_{lb}$}‚ल‚ब्धिः ।
	\pend% ending standard par
      ‚{\tiny $_{lb}$}‚

	  \pstart \leavevmode% starting standard par
	न‚नु बाल‚विक‚ल्प‚स्य प‚रोक्ष‚त्वात् त‚त उत्प‚त्तिर‚नुत्प‚त्तिर्वा न श‚क्य‚ते निश्चेतुम् । अतोऽयं‚{\tiny $_{lb}$}‚ स‚न्दिग्धासिद्धो हेतुरित्य‚भिप्रेत्य चोद‚य‚ति \textbf{कुत} इति सामान्य‚हेतोः प्र‚श्नः । \textbf{पुन}रिति विशेष‚स्य ।‚{\tiny $_{lb}$}‚ विद‚र्भ्योक्त्या चायं क्षेपे किमः प्र‚योगः । तेन न कुत‚श्चिद्धेतोरिद‚मित्य‚र्थः । हेतुमाह--\textbf{अर्थे}ति ।‚{\tiny $_{lb}$}‚ एवं तु प्र‚योगः क‚र‚णीयः--य‚ज्ज्ञानं स्वोत्प‚त्ताव‚र्थ‚स‚न्निधिनिर‚पेक्षं त‚द‚र्थान्नोत्प‚द्य‚ते । य‚था‚{\tiny $_{lb}$}‚ व्युत्प‚न्न‚संकेत‚स्य चिर‚न‚ष्ट‚व‚स्तुविष‚यं विज्ञान‚म् । अर्थ‚स‚न्निधिनिर‚पेक्षं च बाल‚विक‚ल्प‚विज्ञान‚मिति‚{\tiny $_{lb}$}‚ विरुद्ध‚व्याप्तोप‚ल‚ब्धिः । असिद्धिम‚स्याः प‚रिह‚त्तु भूमिकां र‚च‚य‚न्नाह--\textbf{बालोऽइ हि}--इत्यादि ।‚{\tiny $_{lb}$}‚ \leavevmode\ledsidenote{\textenglish{51/dm}}‚{\tiny $_{lb}$}‚ 
	  
	मुख‚म‚र्प‚य‚ति स्त‚ने । पूर्व‚दृष्टाप‚र‚दृष्टं चार्थ‚मेकीकुर्व‚द् विज्ञान‚म‚स‚न्निहित‚विष‚य‚म्, पूर्व‚दृष्ट‚त्व\edtext{}{\lemma{त्व}\Bfootnote{ध‚र्मोत्त‚रे स‚र्व‚प्र‚तिषु पूर्व‚दृष्ट‚स्य इति पाठः । किन्त्व‚त्र प्र‚दीपानुसारी पाठो‚{\tiny $_{lb}$}‚ गृहीतः ।--सं०}}स्या‚{\tiny $_{lb}$}‚स‚न्निहित‚त्वात् । अस‚न्निहित‚विष‚यं चार्थ‚निर‚पेक्ष‚म् । अन‚पेक्षं च प्र‚तिभास‚निय‚म‚हेतोर‚भावा‚{\tiny $_{lb}$}‚द‚निय‚त‚प्र‚तिभास‚म् । तादृशं चाभिलाप‚संस‚र्ग‚योग्य‚म् ।‚{\tiny $_{lb}$}‚ न केव‚लं व्युत्प‚न्न इत्य‚पिश‚ब्दः । हिर्य‚स्मात् । \textbf{याव‚दि}ति निपातोऽव‚धौ । \textbf{पूर्व‚दृष्ट‚त्वं} पूर्व‚{\tiny $_{lb}$}‚द‚र्श‚न‚विष‚य‚त्व‚म् । तेन यो म‚या पूर्वं क्षुत्प्र‚तिघात‚हेतुत्वेन प्र‚तिप‚न्नः स एवाऽय‚मिति प्र‚त्य‚व‚म‚र्ष‚स्य‚{\tiny $_{lb}$}‚ रूप‚माच‚ष्टे । \textbf{न प्र‚त्य‚व‚मृश‚ति} प्र‚त्य‚भिजानाति । \textbf{ताव‚दि}त्य‚प्य‚व‚धौ । \textbf{उप‚र‚तं रुदितं} य‚स्मात्‚{\tiny $_{lb}$}‚ स त‚था \textbf{स‚न्नाय‚र्प‚ति} । स‚र्वैरेव स्व‚स‚न्त‚तावेव‚मादिव्य‚व‚हार‚स्य दृष्ट‚दृश्य‚मान‚योरेकीक‚र‚ण‚कार‚ण‚{\tiny $_{lb}$}‚त्वेनाव‚ग‚त‚त्वादेव‚मुच्य‚ते । त‚स्यैव चान्य‚त्र लोष्टादौ ढौकितेऽपि त‚थाऽद‚र्श‚नाच्च ।
	\pend% ending standard par
      ‚{\tiny $_{lb}$}‚

	  \pstart \leavevmode% starting standard par
	न‚नु बाल‚स्य क‚र‚णानाम‚पाट‚वात् संकेताग्र‚ह‚णाच्च नान्त‚र्ज‚ल्पाकारोऽपि श‚ब्दः स‚म्भ‚व‚ति ।‚{\tiny $_{lb}$}‚ स‚म्भ‚वे वा योग्य‚ग्र‚ह‚णान‚र्थ‚क्य‚म् । त‚त्क‚थ‚मेवं \textbf{याव‚न्न प्र‚त्य‚व‚मृश‚ति}--इत्युच्य‚ते ? उच्य‚ते ।‚{\tiny $_{lb}$}‚ अन‚यैव हि भ‚ङ्ग्या अस्ति सा काचिद् दृष्ट‚दृश्य‚मान‚योरेकीक‚र‚णाव‚स्था या त‚त्र निमित्त‚मिति‚{\tiny $_{lb}$}‚ प्र‚तिपाद्य‚ते । सा च श‚ब्देन प्र‚तिपाद्य‚माना अभ्य‚स्तेनामुना श‚ब्देन प्र‚तिपाद्य‚ते । न पुन‚रेव‚{\tiny $_{lb}$}‚मेवासौ प्र‚त्य‚व‚मृश‚तीत्युच्य‚ते ।
	\pend% ending standard par
      ‚{\tiny $_{lb}$}‚

	  \pstart \leavevmode% starting standard par
	न‚नु द्वितीयादिद‚र्श‚न‚काले भ‚व‚तु पूर्व‚दृष्टाप‚र‚दृष्ट‚योरेकीक‚र‚णेन मुखार्प‚ण‚म्, भूमिपातान‚न्त‚रं‚{\tiny $_{lb}$}‚ तु क‚थ‚म् ? न ख‚लु स्त‚न‚म‚द्राक्षीद् असौ येन दृष्टेन दृश्य‚मान‚मेकीकृत्यार्प‚येत् । त‚दा तु‚{\tiny $_{lb}$}‚ ज‚न्मान्त‚रानुभ‚वादिति ब्रूमः । त‚त्रापि ज‚न्म‚नि ज‚न्मान्त‚रानुभ‚व‚ब‚लात् । न चादिमान् संसार‚{\tiny $_{lb}$}‚ इति का क्ष‚तिः ? श‚ब्दाकारोऽपि त‚त्रास्त्येव । त‚त् किम‚नेन योग्य‚ग्र‚ह‚ण‚स‚फ‚लीक‚र‚ण‚{\tiny $_{lb}$}‚प्र‚यासेनेति चेत् । अस्तु त‚त्र ज‚न्मान्त‚राभ्यासात् मूर्च्छितः श‚ब्दाकारः, मा च भूत्‚{\tiny $_{lb}$}‚ स‚र्व‚था प‚श्चाद्व्युत्प‚त्स्य‚मानेन विशिष्टेन श‚ब्देन संसृष्टार्थ‚प्र‚तिभासः, न भ‚व‚ति त‚स्य विक‚ल्प‚प्र‚त्य‚य‚{\tiny $_{lb}$}‚ इत्युच्य‚ते । स एवाय‚म् इति पूर्व‚दृष्ट‚त्वेन स्त‚नं प्र‚त्य‚व‚मृश‚तु । किम‚त इत्याह--\textbf{पूर्व‚दृष्टे}ति ।‚{\tiny $_{lb}$}‚ \textbf{पूर्व‚दृष्टं चाप‚र‚दृष्टं} चेति द्व‚न्द्वैक‚व‚द्भावः । \textbf{चो}ऽव‚धार‚णे कुर्व‚दित्य‚स्मात्प‚रो द्र‚ष्ट‚व्यः । हेतौ‚{\tiny $_{lb}$}‚ वा श‚तृप्र‚त्य‚यः ।
	\pend% ending standard par
      ‚{\tiny $_{lb}$}‚

	  \pstart \leavevmode% starting standard par
	तेनाय‚म‚र्थः--य‚स्मात्पूर्व‚द‚ष्टाप‚र‚दृष्ट‚मेकीक‚रोत्येव त‚स्मा\textbf{द‚स‚न्निहित‚विष‚य‚मिति} । त‚था‚{\tiny $_{lb}$}‚ कुर्व‚द‚पि क‚थ‚म‚स‚न्निहित‚विष‚य‚मित्याह--\textbf{पूर्व‚दृष्ट‚त्व‚स्य}\edtext{\textsuperscript{*}}{\lemma{*}\Bfootnote{ध‚र्मोत्त‚रे स‚र्व‚प्र‚तिषु पूर्व‚दृष्ट‚स्य इति पाठः । किन्त्व‚त्र प्र‚दीपानुसारी पाठो‚{\tiny $_{lb}$}‚गृहीतः ।--सं०}}--इति । अय‚माश‚यः--पूर्व‚द‚र्श‚न‚विष‚य‚त्वं‚{\tiny $_{lb}$}‚ पूर्व‚दृष्ट‚त्व‚मुच्य‚ते । निवृत्ते च पूर्व‚द‚र्श‚ने पूर्व‚द‚र्श‚न‚विष‚य‚त्वं व‚स्तुनो नास्ति । त‚दुत्त‚र‚काल‚भाविना‚{\tiny $_{lb}$}‚ ज्ञानेन पूर्व‚ज्ञान‚विष\leavevmode\ledsidenote{\textenglish{23a/ms}}य‚त्व‚म‚स‚न्निहित‚मेव व‚स्तुनो दृश्य‚त इति । भ‚व‚त्व‚स‚न्निहित‚विष‚य‚म्,‚{\tiny $_{lb}$}‚ अर्थ‚स‚न्निधिनिर‚पेक्षं तु क‚थं सिद्ध‚मित्याह--\textbf{अस‚न्निहितेति । चो} य‚स्मात् । \textbf{अस‚न्निहित‚विष‚य}‚{\tiny $_{lb}$}‚म‚र्थ‚स‚न्निधिनिर‚पेक्ष‚म् । अनेनार्थ‚स‚न्निधिनिर‚पेक्ष‚त्वेऽर्थ‚स‚न्निधिनिर‚पेक्ष‚त्वं हेतुमाह । ईदृश‚स्तु‚{\tiny $_{lb}$}‚ प्र‚योगो ज्ञात‚व्यः--य‚द‚स‚न्निहित‚विष‚यं ज्ञानं त‚द‚र्थ‚स‚न्निधिनिर‚पेक्ष‚म्, य‚थाऽतीतानाग‚त‚विष‚यो‚{\tiny $_{lb}$}‚ऽस्म‚दादिविक‚ल्पः । पूर्व‚दृष्ट‚त्वेन प्र‚त्य‚भिज्ञानाच्च बाल‚विज्ञान‚म‚स‚न्निहित‚विष‚य‚मिति स्व‚भावः ।
	\pend% ending standard par
      ‚{\tiny $_{lb}$}‚

	  \pstart \leavevmode% starting standard par
	स‚म्प्र‚ति शिष्याणां सुख‚प्र‚तिप‚त्त्य‚र्थं य‚थोत्त‚रोत्त‚र‚स्य हेतोः सिद्धौ पूर्व‚पूर्व‚स्य हेतोः सिद्धि‚{\tiny $_{lb}$}‚र्भ‚व‚ति त‚था द‚र्श‚य‚ति--\textbf{अन‚पेक्षं च}--इति ।
	\pend% ending standard par
      ‚{\tiny $_{lb}$}‚\textsuperscript{\textenglish{52/dm}}‚{\tiny $_{lb}$}‚
	  \bigskip
	  \begingroup
	

	  \pstart \leavevmode% starting standard par
	इन्द्रिय‚विज्ञानं तु स‚न्निहितार्थ\edtext{}{\lemma{न्निहितार्थ}\Bfootnote{स‚न्निहित‚मात्र‚ग्रा० \cite{dp-msA} \cite{dp-edP} \cite{dp-edH} \cite{dp-edE} \cite{dp-edN}}}मात्र‚ग्राहित्वाद‚र्थ‚सापेक्ष‚म् । अर्थ‚स्य च प्र‚तिभास‚{\tiny $_{lb}$}‚निव‚म‚हेतुत्वान्निय‚त‚प्र‚तिभास‚म् । त‚तो नाभिलाप‚संस‚र्ग‚योग्य‚म् ।\edtext{\textsuperscript{*}}{\lemma{*}\Bfootnote{इन्द्रिय‚विज्ञान‚म्--\cite{dp-msD-n}}}
	\pend% ending standard par
       ‚{\tiny $_{lb}$}‚ 

	  \pstart \leavevmode% starting standard par
	अत एव स्व‚ल‚क्ष‚ण‚स्यापि वाच्य‚वाच‚क‚भाव‚म‚भ्युप‚ग‚म्यै\edtext{}{\lemma{म्यै}\Bfootnote{एत‚स्य इन्द्रिय‚ग्राहिणो ज्ञान‚स्य--\cite{dp-msD-n}}}त‚द‚विक‚ल्प‚क‚त्व‚मुच्य‚ते । य‚द्य‚पि‚{\tiny $_{lb}$}‚ हि स्व‚ल‚क्ष‚ण‚मेव वाच्यं वाच‚कं च भ‚वेत् त‚थापि अभिलाप‚संसृष्टार्थ\edtext{}{\lemma{संसृष्टार्थ}\Bfootnote{वाच्य‚वाच‚क‚भाव‚स‚द्भावेऽपि अभिलाप‚संसृष्टार्थं स‚द् विज्ञानं स‚विक‚ल्प‚क‚म्, अन्य‚था‚{\tiny $_{lb}$}‚ निर्विक‚ल्प‚क‚मित्य‚र्थः--\cite{dp-msD-n}}} विज्ञानं स‚विक‚ल्प‚क‚म् ।‚{\tiny $_{lb}$}‚ न चेन्द्रिय‚विज्ञान‚म् अर्थेन निय‚मित‚प्र‚तिभास‚त्वाद् अभिलाप‚संस‚र्ग‚योग्य‚प्र‚तिभासं भ‚व‚तीति‚{\tiny $_{lb}$}‚ निर्विक‚ल्प‚क्र‚म् ।
	\pend% ending standard par
       ‚{\tiny $_{lb}$}‚ 

	  \pstart \leavevmode% starting standard par
	श्रोत्र‚ज्ञानं\edtext{}{\lemma{ज्ञानं}\Bfootnote{श्रोत्र‚विज्ञानं \cite{dp-msB} \cite{dp-msD} \cite{dp-edN}}} त‚र्हि \edtext{}{\lemma{र्हि}\Bfootnote{त‚र्हि स्व० \cite{dp-msB}}}श‚ब्द‚स्व‚ल‚क्ष‚ण‚ग्राहि \edtext{}{\lemma{ग्राहि}\Bfootnote{श‚ब्दादिकं घ‚टादिकं वा--\cite{dp-msD-n}}}श‚ब्द‚स्व‚ल‚क्ष‚णं च\edtext{}{\lemma{च}\Bfootnote{नास्ति च \cite{dp-msA} \cite{dp-edP} \cite{dp-edH} \cite{dp-edE}}} किञ्चिद्वाच्यं किञ्चिद्वाच‚{\tiny $_{lb}$}‚क‚म्--इत्य‚भिलाप‚संस‚र्ग‚योग्य‚प्र‚तिभासं स्यात् । त‚था च स‚विक‚ल्प‚कं स्यात् ।
	\pend% ending standard par
       ‚{\tiny $_{lb}$}‚ 

	  \pstart \leavevmode% starting standard par
	नैष दोषः । स‚त्य‚पि स्व‚ल‚क्ष‚ण‚स्य वाच्य‚वाच‚क‚भावे संकेत‚काल‚दृष्ट‚त्वेन गृह्य‚माणं
	\pend% ending standard par
      
	  \endgroup
	‚{\tiny $_{lb}$}‚

	  \pstart \leavevmode% starting standard par
	न‚नु किम‚निय‚त‚प्र‚तिभास‚त्वे साध्येऽर्थ‚निर‚पेक्ष‚त्वं हेतुर्येन त‚था स‚द‚निय‚त‚प्र‚तिभास‚मि‚{\tiny $_{lb}$}‚त्युच्य‚ते । \textbf{प्र‚तिभास‚निय‚म‚हेतोरिति च} म‚ध्य‚व‚र्त्ती च ग्र‚न्थः क‚थं नेयः ? उच्य‚ते । नाय‚{\tiny $_{lb}$}‚\textbf{म‚न‚पेक्ष‚मि}त्यादिरेक‚वाक्य‚त‚या नेयः । किं त‚र्हि ? वाक्य‚भेदोऽत्र क‚र्त्त‚व्यः । त‚त्र \textbf{चो} य‚स्मात् ।‚{\tiny $_{lb}$}‚ अन‚पेक्ष‚म‚र्थ‚स‚न्निधिनिर‚पेक्षं स‚त्प्र‚तिभास‚निय‚म‚हेतोर‚नुत्प‚न्न‚मिति प्र‚तिभास‚निय‚म‚हेतोर‚भावादित्य‚स्य‚{\tiny $_{lb}$}‚ साम‚र्थ्याद् वाक्य‚भेदं कृत्वा । त‚तः प्र‚तिभास‚निय‚म‚हेतोर‚भावाद‚नुत्पादाद‚निय‚त‚प्र‚तिभास‚मिति‚{\tiny $_{lb}$}‚ योज‚नीय‚म् । \textbf{तादृश‚मि}त्य‚निय‚त‚प्र‚तिभास‚म् । \textbf{चो} व‚क्त‚व्य‚मित्येत‚दित्य‚स्मिन्न‚र्थे ।
	\pend% ending standard par
      ‚{\tiny $_{lb}$}‚

	  \pstart \leavevmode% starting standard par
	स‚र्व‚मेत‚दिन्द्रिय‚ज्ञानेऽपि प‚रः क‚दाचिदाश‚ङ्क‚येदिति त‚न्निरासार्थ‚माह--\textbf{इन्द्रियेति} ।‚{\tiny $_{lb}$}‚ तुर्विक‚ल्प‚ज्ञानं विशिन‚ष्टि । \textbf{त‚तो} निय‚त‚प्र‚तिभास‚त्वात् ।
	\pend% ending standard par
      ‚{\tiny $_{lb}$}‚

	  \pstart \leavevmode% starting standard par
	\hphantom{.}इह पूर्व‚व्याख्यातृभिः असाम‚र्थ्य‚वैय‚र्थ्याभ्यां स्व‚ल‚क्ष‚ण‚स्य संकेत‚यितुम‚श‚क्य‚त्वाद‚वाच्य‚{\tiny $_{lb}$}‚वाच‚क‚त्व‚म् । अवाच्यावाच‚क‚स्व‚ल‚क्ष‚ण‚ग्राहित्वाच्चेन्द्रिय‚ज्ञान‚म‚विक‚ल्प‚क‚मिति व्याख्यात‚म् ।‚{\tiny $_{lb}$}‚ त‚च्चाव‚द्य‚म्, अन्य‚थाऽप्य‚विक‚ल्प‚त्व‚स्य सुसाध‚त्वात् किम‚न्यापोहान‚य‚नेनाप्र‚कृतेनेति म‚न्य‚मान‚{\tiny $_{lb}$}‚ आह--\textbf{अत एव}--इति । य‚स्मादिन्द्रिय‚ज्ञानं निय‚त‚प्र‚तिभास‚म्--\textbf{अत एव} अस्मादेव हेतोः । \textbf{स्व‚ल‚क्ष‚णं}‚{\tiny $_{lb}$}‚ व‚क्ष्य‚माण‚ल‚क्ष‚ण‚म् । \textbf{अपि}र‚व‚धार‚णे । वाच्य‚ग्र‚ह‚णे क‚थ‚म‚विक‚ल्प‚क‚मित्याह--\textbf{य‚द्य‚पि हि}--इति ।‚{\tiny $_{lb}$}‚ \textbf{य‚द्य‚पि ही}ति निपात‚स‚मुदायो य‚दि नामश‚ब्द‚स्यार्थे व‚र्त्त‚ते । हिर्वाक्याल‚ङ्कारो वा । अभि‚{\tiny $_{lb}$}‚लाप‚संसृष्टार्थं स‚द् विज्ञानं विक‚ल्प‚कं भ‚व‚ति । एवं ब्रुव‚तोऽयं भावः--कोष्ठ‚शुद्ध्या वाच्य‚म‚स्तु‚{\tiny $_{lb}$}‚ स्व‚ल‚क्ष‚णं त‚थापि त‚दिन्द्रिय‚ज्ञानं केव‚ल‚मेव स्व‚ल‚क्ष‚ण‚मात्म‚न्याद‚र्श‚य‚ति, न तु वाच‚क‚म‚पीति‚{\tiny $_{lb}$}‚ क‚थ‚म‚भिलाप‚संसृष्टार्थ‚प्र‚तिभास‚त्वं विक‚ल्प‚क‚त्व‚मात्म‚सात् कुर्यादिति ?
	\pend% ending standard par
      ‚{\tiny $_{lb}$}‚‚{\tiny $_{lb}$}‚\textsuperscript{\textenglish{53/dm}}‚{\tiny $_{lb}$}‚
	  \bigskip
	  \begingroup
	

	  \pstart \leavevmode% starting standard par
	स्व‚ल‚क्ष‚णं वाच्यं वाच‚कं च गृहीतं स्यात् । न च संकेत‚काल‚भावि द‚र्श‚न‚विष‚य‚त्वं व‚स्तुनः‚{\tiny $_{lb}$}‚ स‚म्प्र‚त्य‚स्ति । य‚था हि संकेत‚काल‚भावि द‚र्श‚न‚म‚द्य\edtext{}{\lemma{द्य}\Bfootnote{ग्र‚ह‚ण‚काले--\cite{dp-msD-n}}} निरुद्ध‚म्, त‚द्व‚त् त‚द्विष‚य‚त्व‚म‚पि\edtext{}{\lemma{पि}\Bfootnote{स्म‚र्य‚माण‚संकेत‚स्य--\cite{dp-msD-n}}} अर्थ‚स्याद्य‚{\tiny $_{lb}$}‚ नास्ति । त‚तः पूर्व‚काल‚दृष्ट‚त्व‚म‚प‚श्य‚च्छोत्र‚ज्ञानं\edtext{}{\lemma{ज्ञानं}\Bfootnote{श्रोत्र‚विज्ञानं--\cite{dp-msB} \cite{dp-msD} \cite{dp-edN}}} न वाच्य‚वाच‚क‚भाव‚ग्राहि ।
	\pend% ending standard par
       ‚{\tiny $_{lb}$}‚ 

	  \pstart \leavevmode% starting standard par
	अनेनैव न्यायेन योगिज्ञान‚म‚पि स‚क‚ल‚श‚ब्दार्थाव‚भासित्वेऽपि संकेत‚काल‚दृष्ट‚त्वाग्र‚ह‚णा‚{\tiny $_{lb}$}‚न्निर्विक‚ल्प‚क‚म् ॥
	\pend% ending standard par
      
	  \endgroup
	‚{\tiny $_{lb}$}‚

	  \pstart \leavevmode% starting standard par
	भ‚व‚तु रूपाद्याल‚म्ब‚न‚मिन्द्रिय‚ज्ञान‚म्, वाच‚क‚स्याप्र‚तिभासाद् वाच्य‚स्यैव प्र‚तिभासाद‚वि‚{\tiny $_{lb}$}‚क‚ल्प‚क‚म् । य‚त्पुन‚रेत‚द्द्व‚य‚प्र‚तिभासीन्द्रिय‚ज्ञानं त‚द् द्व‚य‚प्र‚तिभासाद् विक‚ल्प‚कं प्राप्त‚मित्य‚भिप्रेत्य‚{\tiny $_{lb}$}‚ चोद‚य‚ति--\textbf{श्रोत्रे}ति । \textbf{त‚र्हि}र‚क्ष‚मायाम् । स‚र्व‚मिन्द्रिय‚ज्ञान‚म‚विक‚ल्प‚क‚मिति न क्ष‚म्य‚त एत‚दित्य‚र्थः ।‚{\tiny $_{lb}$}‚ \textbf{श्रोत्र‚ज्ञान‚म‚भिलाप‚संस‚र्ग‚योग्य‚प्र‚तिभासं स्यादि}ति स‚म्ब‚न्धः । किम्भूतं ? \textbf{श‚ब्द‚स्व‚ल‚क्ष‚ण‚ग्राहि} ।‚{\tiny $_{lb}$}‚ हेतुभावेनास्य विशेष‚ण‚त्वात् श‚ब्द‚स्व‚ल‚क्ष‚ण‚ग्राहित्वादित्य‚र्थः । श‚ब्द‚ग्राहिणोऽपि वाच्याग्र‚ह‚णे‚{\tiny $_{lb}$}‚ क‚थं त‚थात्व‚मित्याह--\textbf{श‚ब्देति}\leavevmode\ledsidenote{\textenglish{23b/ms}} । \textbf{चो} य‚स्मात् । \textbf{किञ्चिद् वाच‚कं किञ्चिद् वाच्य‚म्} ।‚{\tiny $_{lb}$}‚ य‚था त‚र‚प्त‚म‚पौ घः \href{http://sarit.indology.info/?cref=Pā.1.1.22}{पाणिनि--१. १. २२}इत्यादि बुद्धिस्थ‚म् । \textbf{इति}स्त‚स्मात् । अस्तु त‚थाप्र‚तिभासं‚{\tiny $_{lb}$}‚ किम‚त इत्याह--\textbf{त‚था चे}ति । \textbf{त‚था च} स‚ति त‚स्मिश्चोक्त‚प्र‚कारे स‚ति । एवं चैत‚द् द्र‚ष्ट‚व्य‚म्—‚{\tiny $_{lb}$}‚य‚दैकेन त‚र‚प्त‚म‚पौ संज्ञिनावुच्चार्येते त‚दैव च क‚थ‚ञ्चिद‚न्येन घः इति संज्ञोच्चार्य‚ते । त‚दा‚{\tiny $_{lb}$}‚ त‚द्द्वित‚य‚मेकेन श्रोत्र‚ज्ञानेन प्र‚तिय‚तः पुंसः श्रोत्र‚धीर्विक‚ल्पिका प्र‚स‚ज्येतेति ।
	\pend% ending standard par
      ‚{\tiny $_{lb}$}‚

	  \pstart \leavevmode% starting standard par
	क‚थं त‚र्हि \textbf{नैष दोष} इत्याह--\textbf{स‚त्य‚पी}ति । \textbf{चो} य‚स्मात् त‚देवेदं य‚न्म‚या संकेत‚काल‚{\tiny $_{lb}$}‚त्वेन गृह्य‚माणं त‚था गृहीतं भ‚व‚ति । य‚द्येव‚म‚स्तु स‚केत‚काल‚दृष्ट‚त्वेन ग्र‚ह‚ण‚मित्याह--\textbf{न चे}ति ।‚{\tiny $_{lb}$}‚ \textbf{चो}ऽव‚धार‚णे हेतौ वा । \textbf{द‚र्श‚न‚विष‚य‚त्व‚मिति} ब्रुव‚न् पूर्व‚दृष्ट‚त्व‚श‚ब्द‚स्यार्थ‚माह । क‚थं नास्तीत्याह—‚{\tiny $_{lb}$}‚\textbf{य‚थे}ति । \textbf{हि}र्य‚स्माद‚र्थे ।
	\pend% ending standard par
      ‚{\tiny $_{lb}$}‚

	  \pstart \leavevmode% starting standard par
	\textbf{अर्थ‚स्ये}ति स‚म्प्र‚ति दृश्य‚मान‚स्य । त‚द्भावे निग‚डाक‚र्ष‚ण‚न्यायेन प्राक्त‚न‚द‚र्श‚न‚स्यापि स्थितिः‚{\tiny $_{lb}$}‚ प्र‚स‚ज्येतेति भावः । मा ग्र‚हीत् संकेत‚काल‚दृष्ट‚त्वं त‚योर्गृ ह्य‚माण‚योः संज्ञासंज्ञिभूत‚योः श‚ब्द‚योस्त‚थापि‚{\tiny $_{lb}$}‚ क‚थं न वाच्य‚वाच‚क‚भाव‚ग्राहि त‚ज्ज्ञान‚मित्याश‚ङ्क्योप‚संह‚र‚न्नाह--\textbf{त‚त} इति । य‚तः संकेत‚काल‚त्वेन‚{\tiny $_{lb}$}‚ गृह्य‚माणं वाच्यं वाच‚कं च गृहीतं भ‚व‚ति \textbf{त‚तः} कार‚णात् श्रोत्र‚विज्ञानं न वाच्य‚वाच‚क‚भाव‚ग्राहि ।‚{\tiny $_{lb}$}‚ भ‚व‚तु त‚थागृह्य‚माण‚स्य स्व‚ल‚क्ष‚ण‚स्य त‚थात्व‚म् । किम‚तः ? एत‚त्पुन‚रेनं त‚थैव ग्र‚हीष्य‚तीत्याह—‚{\tiny $_{lb}$}‚\textbf{पूर्वे}ति । हेतौ श‚तुर्विधानात् पूर्व‚काल‚दृष्ट‚त्वाग्र‚ह‚णादित्य‚र्थः ।
	\pend% ending standard par
      ‚{\tiny $_{lb}$}‚

	  \pstart \leavevmode% starting standard par
	स्यादेत‚त्--संकेत‚दृष्ट‚त्वेनापि न ग्र‚हीष्य‚तेऽर्थः श‚ब्दो वा । अथ च विशिष्ट‚वाच‚क‚{\tiny $_{lb}$}‚वाच‚क‚त्वेन विशिष्टार्थ‚वाच‚क‚त्वेन च ग्र‚हीष्य‚त इति । अस‚देत‚त् । एवं हि ब्रुव‚तेद‚म‚भिप्रेत‚म्—‚{\tiny $_{lb}$}‚येन ज्ञानेन योऽर्थः संकेत‚कालोप‚ल‚ब्धो य‚च्छ‚ब्द‚संस‚र्गिणा रूपेण नैकीक्रिय‚ते, न स तेन त‚च्छ‚ब्द‚{\tiny $_{lb}$}‚वाच्यो गृह्य‚ते । य‚था गोज्ञानेन संकेत‚कालोप‚ल‚ब्धाश्व‚श‚ब्द‚संस‚र्गिणाऽश्व‚रूपेण स‚हैक‚त्वेनाप्र‚तीय‚{\tiny $_{lb}$}‚मानो गौर्नाश्व‚श‚ब्द‚वाच्यो गृह्य‚ते । श्रोत्र‚ज्ञानेन संकेत‚कालोप‚ल‚ब्धघश‚ब्द‚संस‚र्गिणा च रूपेण‚{\tiny $_{lb}$}‚ नैकीक्रियेते च त‚दा त‚र‚प्त‚म‚पाविति व्याप‚कानुप‚ल‚ब्धिः । त‚था येन ज्ञानेन यः श‚ब्दः‚{\tiny $_{lb}$}‚ ‚{\tiny $_{lb}$}‚ \leavevmode\ledsidenote{\textenglish{54/dm}}‚{\tiny $_{lb}$}‚ 
	  
	त‚या र‚हितं तिमिराशुभ्र‚म‚ण‚नौयान‚संक्षोभाद्य‚नाहित‚विभ्र‚मं ज्ञानं प्र‚त्य‚क्ष‚म् ॥ ६ ॥‚{\tiny $_{lb}$}‚ 
	  
	त‚या क‚ल्प‚न‚या क‚ल्प‚नास्व‚भावेन र‚हितं शून्यं स‚ज्ज्ञानं य‚द‚भ्रान्तं त‚त् प्र‚त्य‚क्ष‚म् इति‚{\tiny $_{lb}$}‚ \edtext{\textsuperscript{*}}{\lemma{*}\Bfootnote{सूत्रेण--\cite{dp-msD-n}; प‚रेण सूत्रेण सं० \cite{dp-msB}}}प‚रेण स‚म्ब‚न्धः । क‚ल्प‚नापोढ‚त्वाभ्रान्त‚त्वे प‚र‚स्प‚र‚सापेक्षे प्र‚त्य‚क्ष‚ल‚क्ष‚ण‚म्, न प्र‚त्येक‚मिति‚{\tiny $_{lb}$}‚ द‚र्श‚यितुं त‚या र‚हितं य‚द‚भ्रान्तं त‚त् प्र‚त्य‚क्ष‚मिति ल‚क्ष‚ण‚योः प‚र‚स्प‚र‚सापेक्ष‚योः प्र‚त्य‚क्ष‚विष‚त्य‚वं‚{\tiny $_{lb}$}‚ द‚र्शित‚मिति\edtext{}{\lemma{मिति}\Bfootnote{द्र‚ष्ट‚व्य‚मिति अध्याहारः--\cite{dp-msD-n}}} ॥‚{\tiny $_{lb}$}‚ संकेत‚काल‚दृष्ट‚य‚द‚र्थ‚संस‚र्गिणा रूपेण नैकीक्रिय‚ते, न तेनासौ त‚द‚र्थ‚वाच‚को गृह्य‚ते । य‚था‚{\tiny $_{lb}$}‚ गोश‚ब्द‚ज्ञानेन संकेत‚कालोप‚ल‚ब्धाश्वार्थ‚संस‚र्गिणाश्व‚श‚ब्देन स‚हैक‚त्वेनाप्र‚तीय‚मानो गोश‚ब्दो नाश्वार्थ‚{\tiny $_{lb}$}‚वाच‚को गृह्य‚ते । श्रोत्र‚ज्ञानेन संकेत‚दृष्ट‚त‚र‚प्त‚म‚ब‚र्थ‚संस‚र्गिणा रूपेण नैकीक्रिय‚ते च त‚दा धश‚ब्द‚{\tiny $_{lb}$}‚ इति सैव असिद्धिर‚नुभ‚वेन निराकृता । वास्त‚व‚स‚म्ब‚न्ध‚निराक‚र‚णाच्च व्याप्तिः सिद्धेति । य‚दा‚{\tiny $_{lb}$}‚ तु पूर्व‚दृष्ट‚त्वेन ग्र‚होऽस्ति त‚दा त‚ज्ज्ञान‚म‚भिलाप‚संसृष्टार्थ‚प्र‚तिभासं स‚द् विक‚ल्प‚रूप‚मेवेति‚{\tiny $_{lb}$}‚ स‚र्व‚म‚व‚दात‚म् ।
	\pend% ending standard par
      ‚{\tiny $_{lb}$}‚

	  \pstart \leavevmode% starting standard par
	श्रोत्र‚ज्ञाने य‚थाऽभिलाप‚संस‚र्ग‚योग्य‚प्र‚तिभास‚त्वं चोदितं त‚था योगिज्ञानेऽपि चोद‚यितुं‚{\tiny $_{lb}$}‚ श‚क्य‚मिति त‚त्राप्य‚मुमेव प‚रिहार‚म‚तिदिश‚न्नाह--\textbf{अनेनेति} । य‚ज्ज्ञानं संकेत‚काल‚विष‚य‚त्वं‚{\tiny $_{lb}$}‚ गृह्य‚माण‚स्य न गृह्णाति, न त‚द् वाच्य‚वाच‚क‚भाव‚ग्राही\textbf{त्य‚नेन न्यायेन} युक्त्या \textbf{योगिज्ञानं} व‚क्ष्य‚माण‚{\tiny $_{lb}$}‚\textbf{ल‚क्ष‚णं स‚क‚ल‚श‚ब्दार्थाव‚भासित्वे} स‚त्य‚पि \textbf{निर्विक‚ल्प‚क‚म् ।}
	\pend% ending standard par
      ‚{\tiny $_{lb}$}‚

	  \pstart \leavevmode% starting standard par
	न‚नु चान्य‚स्य वाच्य‚वाच‚क‚भाव‚ग्राहिणो ज्ञान‚स्यास्तु निर्विक‚ल्प‚क‚त्व‚म्, अस्य तु स‚र्व‚श‚ब्दार्थ‚{\tiny $_{lb}$}‚ग्राहिणः क‚थं त‚थात्व‚मित्याह--\textbf{संकेते}ति । त‚थाऽग्र‚ह‚ण‚मुक्तेन न्यायेन योगिनं प्र‚त्य‚पि पूर्व‚द‚र्श‚न‚{\tiny $_{lb}$}‚विष‚य‚त्व‚स्यास‚त्त्वादिति भावः ।
	\pend% ending standard par
      ‚{\tiny $_{lb}$}‚

	  \pstart \leavevmode% starting standard par
	\textbf{त‚या र‚हित}मित्याचार्यीयं विव‚र‚णं व्याच\leavevmode\ledsidenote{\textenglish{24a/ms}}ष्टे--\textbf{त‚ये}ति ।
	\pend% ending standard par
      ‚{\tiny $_{lb}$}‚

	  \pstart \leavevmode% starting standard par
	न‚नु विक‚ल्पेऽपि विक‚ल्पान्त‚रं नास्ति । त‚त‚स्त‚स्यापि क‚ल्प‚नापोढ‚त्वं किं न भ‚व‚ति ? अथ‚{\tiny $_{lb}$}‚ क‚ल्प‚न‚या र‚हित‚मिति य‚न्न विक‚ल्प्य‚ते इत्युच्य‚ते, त‚द‚पि क‚ल्प‚नापोढ‚मित्य‚नेनैवाकारेण विक‚ल्प्य‚त‚{\tiny $_{lb}$}‚ इति क‚थ‚मेत‚दुच्य‚त इति प‚श्य‚न् \textbf{क‚ल्प‚न‚या र‚हित}मित्य‚स्य विव‚क्षित‚म‚र्थ‚माह--\textbf{क‚ल्प‚नास्व‚भावेने}ति ।‚{\tiny $_{lb}$}‚ क‚ल्प‚नायाः स्व‚भावोऽभिलाप‚संस‚र्ग‚योग्य‚प्र‚तिभास‚त्व‚म्, तेन । य‚था विषाणी क‚कुद्मान् प्रान्ते‚{\tiny $_{lb}$}‚वाल‚धिरिति विषाणादिम‚ता\edtext{}{\lemma{ता}\Bfootnote{तो}}विषाणादीन्येव गोत्व‚लिङ्गानि व्य‚प‚दिश्य‚न्ते । त‚द्व‚त् क‚ल्प‚नाख्येन‚{\tiny $_{lb}$}‚ ध‚र्मिणा ध‚र्मोऽभिलाप‚संस‚र्ग‚योग्य‚प्र‚तिभास‚त्वाख्यो निर्द्दिष्ट इति द‚र्श‚य‚ति । अत एव \textbf{चाचार्य‚{\tiny $_{lb}$}‚दिग्नागीय}प्र‚त्य‚क्ष‚ल‚क्ष‚णे क‚ल्प‚नापोढ‚त्वाख्ये \textbf{य‚दुद्द्योत‚क‚र}चोद्य‚मिहैव पूर्व‚मुप‚द‚र्शितं त‚त्ख‚लु \textbf{ध‚र्म‚{\tiny $_{lb}$}‚कीर्त्ति}चोद्य‚स‚दृश‚मिति स्थित‚म् । \textbf{प‚रेणे}ति प‚र‚देश‚स्थितेन \textbf{त‚द‚नाहित‚विभ्र‚मं ज्ञानं प्र‚त्य‚क्ष‚मि}त्य‚नेन ।‚{\tiny $_{lb}$}‚ अभ्रान्त‚ग्र‚ह‚ण‚स्यैव त‚द् विव‚र‚ण‚मिति अभ्रान्त‚ग्र‚ह‚ण‚मिह । दूर‚स्थितेन स‚म्ब‚न्ध‚क‚र‚णे किम्प्र‚यो‚{\tiny $_{lb}$}‚ज‚न‚मित्याह--\textbf{क‚ल्प‚नापोढे}ति । मिलित‚योर‚न‚योस्त‚ल्ल‚क्ष‚ण‚त्वात् \textbf{प्र‚त्य‚क्ष‚ल‚क्ष‚ण‚मि}त्येक‚व‚च‚नेन‚{\tiny $_{lb}$}‚ निर्देशः । \textbf{इति}ना द‚र्श‚नीय‚स्य रूप‚माह । क‚थं प‚र‚स्प‚र‚सापेक्ष‚तेत्याह--\textbf{त‚या र‚हितं य‚द‚भ्रान्तं‚{\tiny $_{lb}$}‚ त‚त्प्र‚त्य‚क्षं} व्य‚व‚ह‚र्त्त‚व्य‚मिति शेषः । \textbf{इती}त्यादिनैव चोप‚संह‚र‚ति । य‚स्मादेव‚मितिस्त‚स्मात् ।‚{\tiny $_{lb}$}‚ \textbf{प्र‚त्य‚क्ष‚विष‚य‚त्वं द‚र्शित‚म}न्ते प्र‚त्य‚क्ष‚ग्र‚ह‚ण‚साम‚र्थ्यात् ।
	\pend% ending standard par
      ‚{\tiny $_{lb}$}‚‚{\tiny $_{lb}$}‚\textsuperscript{\textenglish{55/dm}}‚{\tiny $_{lb}$}‚
	  \bigskip
	  \begingroup
	

	  \pstart \leavevmode% starting standard par
	तिमिर‚म् अक्ष्णोर्विप्ल‚वः । इन्द्रिय‚ग‚त‚मिदं \edtext{}{\lemma{मिदं}\Bfootnote{०मिदं भ्र‚म० \cite{dp-msB}}}विभ्र‚म‚कार‚ण‚म् । आशुभ्र‚म‚ण‚म् अलातादेः ।‚{\tiny $_{lb}$}‚ म‚न्दं हि\edtext{}{\lemma{हि}\Bfootnote{म‚न्दं भ्र‚म्य \cite{dp-msA}; म‚न्दं हि भ्राम्य० \cite{dp-msC} \cite{dp-msD} \cite{dp-msB} \cite{dp-edN}}} भ्र‚म्य‚माणेऽलातादौ न च‚क्र‚भ्रान्तिरुत्प‚द्य‚ते । त‚द‚र्थ‚म् आशुग्र‚ह‚णेन विशेष्य‚ते‚{\tiny $_{lb}$}‚ भ्र‚म‚ण‚म् । एत‚च्च विष‚य‚ग‚तं विभ्र‚म‚कार‚ण‚म् । नावा ग‚म‚नं नौयान‚म् । ग‚च्छ‚न्त्यां‚{\tiny $_{lb}$}‚ नावि स्थित‚स्य ग‚च्छ‚द्वृक्षादिभ्रान्तिरुत्प‚द्य‚ते इति यान‚ग्र‚ह‚ण‚म् । एत‚च्च बाह्याश्र‚य‚स्थितंर‚{\tiny $_{lb}$}‚ विभ्र‚म‚कार‚ण‚म् । संक्षोभो वात‚पित्त‚श्लेष्म‚णाम् । वातादिषु हि क्षोभं ग‚तेषु ज्व‚लित‚स्त\edtext{}{\lemma{स्त}\Bfootnote{ज्व‚लित‚रूप‚स्त० \cite{dp-msB} \cite{dp-msD}}}‚{\tiny $_{lb}$}‚म्भादिभ्रान्तिरुत्प‚द्य‚ते एत‚च्चाध्यात्म‚ग‚तं\edtext{}{\lemma{तं}\Bfootnote{०ध्यात्मिकं भ्रान्तिका० \cite{dp-msB} \cite{dp-msD}}} विभ्र‚म‚कार‚ण‚म् ।
	\pend% ending standard par
       ‚{\tiny $_{lb}$}‚ 

	  \pstart \leavevmode% starting standard par
	स‚र्वैरेव च विभ्र‚म‚कार‚णैरिन्द्रिय‚विष‚य‚बाह्याध्यात्मिकाश्र‚य‚ग‚तैरिन्द्रिय‚मेव विक‚र्त्त‚व्य‚म् ।‚{\tiny $_{lb}$}‚ अविकृते इन्द्रिये इन्द्रिय\edtext{}{\lemma{इन्द्रिय}\Bfootnote{ज्ञान‚स्य--\cite{dp-msD-n}; अविकृते इन्द्रिये भ्रान्त्य‚योगात्--\cite{dp-msB}}}भ्रान्त्य‚योगात् । एते संक्षोभ‚प‚र्य‚न्ता आद‚यो येषां ते त‚थोक्ताः ।‚{\tiny $_{lb}$}‚ आदिग्र‚ह‚णेन काच‚काम‚लाद‚य इन्द्रिय‚स्था गृह्य‚न्ते । आशुन‚य‚नान‚य‚नाद‚यो विष‚य‚स्थाः ।‚{\tiny $_{lb}$}‚ आशुन‚य‚नान‚य‚ने हि\edtext{}{\lemma{हि}\Bfootnote{न‚य‚ने कार्य \cite{dp-msB} \cite{dp-msD} \cite{dp-edN}}} कार्य‚माणेऽलाते\edtext{}{\lemma{माणेऽलाते}\Bfootnote{०लातादौ--\cite{dp-msA} \cite{dp-edP} \cite{dp-edH} \cite{dp-edN}}}ऽन्निव‚र्ण‚द‚ण्डाभासा भ्रान्तिर्भ‚व‚ति । ह‚स्तियानाद‚यो‚{\tiny $_{lb}$}‚ बाह्याश्र‚य‚स्थाः, गाढ‚म‚र्म‚प्र‚हाराद‚य आध्यात्मिकाश्र‚य‚स्था विभ्र‚म‚हेत‚वो गृह्य‚न्ते ।
	\pend% ending standard par
       ‚{\tiny $_{lb}$}‚ 

	  \pstart \leavevmode% starting standard par
	\edtext{\textsuperscript{*}}{\lemma{*}\Bfootnote{एतैर० \cite{dp-msB} \cite{dp-msD}}}तैर‚नाहितो विभ्र‚मो य‚स्मिंस्त‚थाविधं ज्ञानं प्र‚त्य‚क्ष‚म् ॥
	\pend% ending standard par
      
	  \endgroup
	‚{\tiny $_{lb}$}‚

	  \pstart \leavevmode% starting standard par
	\textbf{तिमिर}म‚क्ष्णोर्विप्ल‚व‚हेतुत्वाद् \textbf{विप्ल‚वः । इन्द्रिय‚ग‚त‚मि}न्द्रियाश्रित‚म् । \textbf{विभ्र‚म}स्य‚{\tiny $_{lb}$}‚ भ्रान्त‚त्व‚स्य \textbf{कार‚णं} साक्षात् कार‚ण‚त्वेन । \textbf{अलातं} विद‚ग्ध‚काष्ठ‚म् । आशुग्र‚ह‚ण‚स्य फ‚ल‚माह—‚{\tiny $_{lb}$}‚\textbf{म‚न्द‚म्} इति । य‚द्य‚पि म‚न्दं भ्र‚म्य‚माण‚स्यालात‚स्य विभ्र‚म‚कार‚ण‚त्वाद‚र्श‚नादाशु भ्र‚म‚णं ल‚भ्य‚ते,‚{\tiny $_{lb}$}‚ त‚थापि शास्त्र‚कृता स्व‚कीय‚व‚च‚नाकौश‚ल‚प‚रिहारार्थ\textbf{माशु}ग्र‚ह‚णं कृत‚मित्य‚व‚सेय‚म् । पूर्व‚स्मादेत‚द्‚{\tiny $_{lb}$}‚भेदेन क‚थ‚मुक्त‚मित्याह--\textbf{एत‚द्} इति । \textbf{चो} य‚स्माद‚र्थे । \textbf{आशु}ग्र‚ह‚णेन नौयान‚म‚पि विशेष‚{\tiny $_{lb}$}‚णीय‚म् । म‚न्दं हि ग‚च्छ‚न्त्यां नावि न ग‚च्छ‚द्वृक्षादिद‚र्श‚नं भ‚व‚तीत्य‚नुभ‚व‚सिद्ध‚मेत‚त् । \textbf{इति}‚{\tiny $_{lb}$}‚र्हेतौ । एत‚त् क‚स्मात् पृथ‚गुक्त‚मित्याह \textbf{एत‚च्चे}ति । \textbf{चो} य‚स्मात् । \textbf{संक्षोभ} उप‚च‚यः ।‚{\tiny $_{lb}$}‚ प्र‚कोप इति याव‚त् । स केषामित्याकाङ्क्षायामाह--\textbf{वाते}ति । नासौ विभ्र‚म‚कार‚ण‚मित्याश‚ङ्का‚{\tiny $_{lb}$}‚म‚प‚नुद‚न्नाह--\textbf{वातादिषु}--इति । हिर्य‚स्मात् । \textbf{क्षोभं} प्र‚कोपं \textbf{ग‚तेषु} संक्षुब्धेष्विति याव‚त् ।
	\pend% ending standard par
      ‚{\tiny $_{lb}$}‚

	  \pstart \leavevmode% starting standard par
	स्यादेत‚त्--अस्तु तिमिर‚स्येन्द्रियोप‚घात‚हेतुक‚त्वाद् भ्रान्त‚ज्ञान‚हेतुत्व‚म् । नौयानादीनां‚{\tiny $_{lb}$}‚ तु क‚थं स‚त्स्व‚पि तेषु त‚द‚व‚स्थ‚त्वादिन्द्रिय‚स्येत्याह--\textbf{स‚र्वैरेवेति} । इन्द्रिय‚ग‚त‚स्य तिमिरादेरिन्द्रिय‚{\tiny $_{lb}$}‚विकार‚त्वेन स‚म्म‚त‚स्याप्युपादानं दृष्टान्तार्थ‚म् । तेनाय‚म‚र्थः--य‚थेन्द्रिय‚ग‚तं तिमिरादीन्द्रिय‚{\tiny $_{lb}$}‚ विकुर्व‚द् विभ्र‚म‚हेतुः, त‚था अन्यैर‚पि त‚द्विकुर्व‚द्भिरेव विभ्र‚म‚हेतुभिर्भाव्य‚मिति ।
	\pend% ending standard par
      ‚{\tiny $_{lb}$}‚

	  \pstart \leavevmode% starting standard par
	क‚स्मात्पुन‚र‚मीभिरिन्द्रिय‚विकारोऽव‚श्य‚क‚र्त्त‚व्य इत्याह--\textbf{अविकृत} इत्यादि । एत‚दुक्तं‚{\tiny $_{lb}$}‚ भ‚व‚ति । त‚द्विकार‚विकारित्वादाश्र‚याश्च‚क्षुराद‚य इति । प्र‚योग\leavevmode\ledsidenote{\textenglish{24b/ms}}स्त्वेवं क‚र्त्त‚व्यः । य‚दि‚{\tiny $_{lb}$}‚‚{\tiny $_{lb}$}‚ ‚{\tiny $_{lb}$}‚ \leavevmode\ledsidenote{\textenglish{56/dm}}‚{\tiny $_{lb}$}‚ 
	  
	त‚देवं ल‚क्ष‚ण‚माख्याय \edtext{}{\lemma{माख्याय}\Bfootnote{वैभाषिकैः--\cite{dp-msD-n}}}यैरिन्द्रिय‚मेव द्र‚ष्टृ क‚ल्पितं मान‚स‚प्र‚त्य‚क्ष‚ल‚क्ष‚णे च दोष‚{\tiny $_{lb}$}‚ उद्भावितः, स्व‚संवेद‚नं च नाभ्युप‚ग‚त‚म्, योगिज्ञानं च; तेषां विप्र‚तिप‚त्तिनिराक‚र‚णार्थं\edtext{}{\lemma{णार्थं}\Bfootnote{निरासार्थ‚म्--\cite{dp-msB} \cite{dp-msC} \cite{dp-msD}}}‚{\tiny $_{lb}$}‚ प्र‚कार‚भेदं प्र‚त्य‚क्ष‚स्य द‚र्श‚य‚न्नाह-- ‚{\tiny $_{lb}$}‚ 
	  
	त‚त् च‚तुर्विध‚म् ॥ ७ ॥‚{\tiny $_{lb}$}‚ 
	  
	त‚त् च‚तुर्विध‚मिति ॥ ‚{\tiny $_{lb}$}‚ 
	  
	इन्द्रिय‚ज्ञान‚म् ॥ ८ ॥‚{\tiny $_{lb}$}‚ 
	  
	इन्द्रिय‚स्य ज्ञान‚म् इन्द्रिय‚ज्ञान‚म् । इन्द्रियाश्रितं य‚त् त‚त् प्र‚त्य‚क्ष‚म् ।‚{\tiny $_{lb}$}‚ न्द्रिय‚ज‚भ्र‚म‚कार‚णं त‚दिन्द्रियं विक‚रोत्येव य‚था तिमिरादि । इन्द्रिय‚ज‚भ्र‚म‚कार‚णं च नौयानादीति‚{\tiny $_{lb}$}‚ स्व‚भाव‚हेतुः । एवंविध‚स्य ज्ञान‚स्य भ्र‚म‚रूप‚त्वं बाध‚व‚शाद‚व‚सित‚मिन्द्रियान्व‚य‚व्य‚तिरेकानु‚{\tiny $_{lb}$}‚विधानाच्चेन्द्रिय‚ज‚त्व‚म् । त‚थाभूत‚विभ्र‚म‚कार‚ण‚त्वं च नौयानादेः स्वान्व‚य‚व्य‚तिरेकानुविधानात्‚{\tiny $_{lb}$}‚ सिद्ध‚म् । व्याप्तिर‚पि तिमिरादौ द‚र्शिता । अय‚मेव प्र‚योगो द‚र्प‚णादिष्व‚पीन्द्रिय‚ज‚भ्र‚म‚निमित्तेषु‚{\tiny $_{lb}$}‚ द्र‚ष्ट‚व्यः । उप‚युक्त‚कार्त्स्न्ये वाऽयं \textbf{स‚र्व}श‚ब्दः प्र‚व‚र्त्त‚नीयः । \textbf{एत} इत्यादि \textbf{विभ्र‚म‚हेत‚वो गृह्य‚न्त}‚{\tiny $_{lb}$}‚ इत्य‚न्तं सुबोध‚म् । केव‚लं \textbf{प‚र्य‚न्तो}ऽन्तो वाच्यः । \textbf{काचो}ऽक्षिविकार‚हेतू रोग‚विशेषः । \textbf{काम‚लो‚{\tiny $_{lb}$}‚च} \edtext{\textsuperscript{*}}{\lemma{*}\Bfootnote{लोऽत्र}} न‚य‚न‚व‚द‚नादिविकार‚कार‚णं व्याधिविशेष एवेति द्र‚ष्ट‚व्य‚म् ।
	\pend% ending standard par
      ‚{\tiny $_{lb}$}‚

	  \pstart \leavevmode% starting standard par
	\textbf{तैर‚नाहितो विभ्र‚मो य‚स्मिन्नि}ति विगृह्णंस्त्रिप‚दं ब‚हुव्रीहिं द‚र्श‚य‚ति । अनाहितो‚{\tiny $_{lb}$}‚ विभ्र‚मो य‚स्मिंस्त‚त्त‚थेति प्र‚साध्य तेषाम‚नाहित‚विभ्र‚म इति ष‚ष्ठीत‚त्पुरुषः कार्यः । इद‚न्त्व‚र्थ‚{\tiny $_{lb}$}‚क‚थ‚न‚मिति बोद्ध‚व्य‚म् । यो हि तैर‚नाहित‚विभ्र‚मः स तेषां भ‚व‚तीति । ज्ञानाधिकारेण‚{\tiny $_{lb}$}‚ ल‚क्ष‚ण‚विधानाल्ल‚ब्धं ज्ञान‚म्, तेन \textbf{ज्ञान}मित्युक्त‚म् ॥
	\pend% ending standard par
      ‚{\tiny $_{lb}$}‚

	  \pstart \leavevmode% starting standard par
	स‚म्प्र‚त्याचार्य‚स्यावान्त‚र‚प्र‚त्य‚क्ष‚भेद‚व्युत्पाद‚ने निमित्तं द‚र्श‚य‚न्नाह--\textbf{त‚देव}मिति । निपात‚{\tiny $_{lb}$}‚स‚मुदाय‚श्चाय‚मुक्तेन प्र‚कारेणेत्य‚स्यार्थे व‚र्त्त‚ते । \textbf{यै}रिति प्र‚त्येकं योज‚नीय‚म् । \textbf{यैः} स्व‚यूथ्यै\textbf{र्वैभाषिकैः} ।‚{\tiny $_{lb}$}‚ य‚दि ज्ञानं द्र‚ष्टृ स्यात्, त‚स्याप्र‚तिघ‚त्वात् व्य‚व‚हित‚म‚पि गृह्णीयात् । त‚तः च‚क्षुः प‚श्य‚ति‚{\tiny $_{lb}$}‚ रूपाणि \href{http://sarit.indology.info/?cref=ak.1.42}{अभिध० १. ४२} इत्यादिवादिभिरिन्द्रिय‚मेव द्र‚ष्टृ क‚ल्पित‚म् । \textbf{यैर्मीमांस‚कैराचार्य‚{\tiny $_{lb}$}‚दिग्नागीये} मान‚स‚प्र‚त्य‚क्ष‚ल‚क्ष‚णे गृहीत‚ग्राहित्व‚ल‚क्ष‚णोऽप्रामाण्य‚निमित्तं दोष उद्भावितः । ल‚क्ष‚ण‚{\tiny $_{lb}$}‚ग्र‚ह‚ण‚स्योप‚ल‚क्ष‚ण‚त्वात् मान‚स‚प्र‚त्य‚क्षाभ्युप‚ग‚मेऽपि यो दोषोऽन्ध‚ब‚धिराद्य‚भाव‚ल‚क्ष‚णः सोऽपि‚{\tiny $_{lb}$}‚ द्र‚ष्ट‚व्यः । \textbf{यैश्च मीमांस‚कैः कुमारिल}म‚तानुसारिभि\textbf{र्नैयायिक‚वैशेषिकैश्च} स्वात्म‚नि क्रियाविरोधेन‚{\tiny $_{lb}$}‚ स्व‚संवेद‚नं नाभ्युप‚ग‚त‚म् । \textbf{यै}श्च \textbf{चार्वाक‚मीमांस‚कैर्यो}गिन एव न स‚म्भ‚व‚न्ति, कुत‚स्तेषां ज्ञान‚मिति‚{\tiny $_{lb}$}‚ \textbf{वादिभिर्योगिज्ञानं च} नाभ्युप‚ग‚त‚मिति व‚र्त्त‚ते । त्र‚योऽपि च‚कारा व‚क्त‚व्यान्त‚रं स‚मुच्चिन्व‚न्ति ।‚{\tiny $_{lb}$}‚ \textbf{तेषाम‚मीषां} त‚त्र त‚त्र या विम‚तिस्त\textbf{न्निराक‚र‚णार्थं} प्र‚त्य‚क्ष‚स्योक्त‚ल‚क्ष‚ण‚स्य \textbf{प्र‚कारो} विशेष‚स्त‚स्यं \textbf{भेदं}‚{\tiny $_{lb}$}‚ नानात्व‚म्, अवान्त‚र‚जातिभेद‚मिति याव‚त् ।
	\pend% ending standard par
      ‚{\tiny $_{lb}$}‚

	  \pstart \leavevmode% starting standard par
	त‚दिति प्र‚त्य‚क्षं \textbf{च‚तुर्विधं} च‚तुष्प्र‚कार‚म् ॥
	\pend% ending standard par
      ‚{\tiny $_{lb}$}‚

	  \pstart \leavevmode% starting standard par
	त‚त्रैकं ताव‚दिन्द्रिय‚ज्ञान‚मुक्तं व्याच‚क्षाण आह--\textbf{इन्द्रिय‚स्ये}ति । ज्ञाय‚तेऽनेनेति \textbf{ज्ञान‚म्,}‚{\tiny $_{lb}$}‚ ‚{\tiny $_{lb}$}‚ \leavevmode\ledsidenote{\textenglish{57/dm}}‚{\tiny $_{lb}$}‚ 
	  
	मान‚स\edtext{}{\lemma{स}\Bfootnote{मान‚से च प्र‚त्य० \cite{dp-msB} \cite{dp-msD}}} प्र‚त्य‚क्षे प‚रैर्यो\edtext{}{\lemma{रैर्यो}\Bfootnote{प‚रैर्दोष० \cite{dp-msD}}} दोष उद्भावित‚स्तं निराक‚र्त्तुं मान‚स‚प्र‚त्य‚क्ष‚ल‚क्ष‚ण‚माह-- ‚{\tiny $_{lb}$}‚ 
	  
	स्व‚विष‚यान‚न्त‚र‚विष‚य‚स‚ह‚कारिणेन्द्रिय‚ज्ञानेन स‚म‚न‚न्त‚र‚प्र‚त्य‚येन ज‚नितं‚{\tiny $_{lb}$}‚ त‚न्म‚नोविज्ञान‚म् ॥ ९ ॥‚{\tiny $_{lb}$}‚ इन्द्रियं च त‚ज्ज्ञानं चेति कृत्वा श‚क्य‚मिन्द्रिय‚ज्ञान‚मिति व‚क्तुम्, त‚त्क‚थं विप्र‚तिप‚त्तिर्निराकृत‚त्या‚{\tiny $_{lb}$}‚श‚ङ्काम‚पाक‚र्त्तुम् \textbf{इन्द्रिय‚ज्ञान}मित्य‚त्राचार्य‚स्ये\textbf{इन्द्रिय‚स्य ज्ञान‚मिति} ष‚ष्ठीत‚त्पुरुषो विव‚क्षित इति‚{\tiny $_{lb}$}‚ द‚र्श‚य‚ति-\textbf{इन्द्रिय‚स्य ज्ञान‚मि}ति । य‚दिन्द्रियेण ज‚न्य‚ते त‚दिन्द्रिय‚स्य भ‚व‚तीति भावः ।
	\pend% ending standard par
      ‚{\tiny $_{lb}$}‚

	  \pstart \leavevmode% starting standard par
	न‚नु चेन्द्रियानुमान‚म‚पि कार्य‚प्र‚भ‚वं भ‚व‚तीन्द्रिय‚स्य ज्ञान‚म् । त‚त‚स्त‚स्यापि प्र‚त्य‚क्ष‚त्वं प्राप्त‚{\tiny $_{lb}$}‚मित्याह \textbf{इन्द्रियाश्रित‚मि}ति । आश्रितं ज‚न्य‚त‚या त‚स्मिन् स‚त्येव भावात् । न चेन्द्रियानु‚{\tiny $_{lb}$}‚\leavevmode\ledsidenote{\textenglish{25a/ms}}मान‚मिन्द्रियेण साक्षाज्ज‚न्य‚ते, येन त‚स्य प्र‚त्य‚क्ष‚त्वं स्यादिति भावः । अनेक‚हेतुत्वेऽपि‚{\tiny $_{lb}$}‚ ज्ञान‚स्येन्द्रियेण व्य‚प‚देशोऽसाधार‚ण‚त्वात्त‚स्येति द्र‚ष्ट‚व्य‚म् । इन्द्रिय‚स्य ज्ञानं प्र‚त्य‚क्ष‚मिति क‚थ‚य‚ता‚{\tiny $_{lb}$}‚ नेन्द्रिय‚स्य द्र‚ष्टृत्व‚मिति द‚र्शित‚म् । त‚तो विप्र‚तिप‚त्तिर्निराकृता ।
	\pend% ending standard par
      ‚{\tiny $_{lb}$}‚

	  \pstart \leavevmode% starting standard par
	न‚नु इन्द्रिय‚ज्ञान‚स्य द्र‚ष्टृत्वे त‚स्याप्र‚तिघ‚त्वाद् व्य‚व‚हित‚स्यापि ग्र‚ह‚णं प्र‚स‚न्येतेति तैरुक्तो‚{\tiny $_{lb}$}‚ दोषः । स क‚थं प‚रिहृतो येन विप्र‚तिप‚त्तिर्निराकृतेति चेत् । अय‚माश‚यः--न ज्ञानं ग‚त्वा‚{\tiny $_{lb}$}‚ ग‚त्वाऽर्थं गृह्णाति, येन ग‚म‚न‚विब‚न्धाभावात् व्य‚व‚हित‚म‚पि गृह्णीयात् । किन्त‚र्हि ? य‚दाकार‚{\tiny $_{lb}$}‚मुत्प‚द्य‚ते त‚द् गृह्णातीत्युच्य‚ते । न चायोग्य‚देश‚स्थोऽर्थ‚स्त‚त्स्व‚रूप‚कोऽन्व‚य‚व्य‚तिरेकाभ्या विज्ञातः ।‚{\tiny $_{lb}$}‚ त‚त्क‚थं त‚स्य तेन ग्र‚ह इति सुज्ञान‚मेत‚दिति किम‚त्रोप‚प‚त्त्याभिहित‚येति ? इन्द्रिय‚ज्ञान प्र‚त्य‚क्ष‚{\tiny $_{lb}$}‚मिन्द्रियादिसाम‚ग्रीज‚न्य‚ज्ञानं प्र‚त्य‚क्ष‚म्, एक‚स्य ज‚न‚क‚त्व‚विरोधात् । नाज्ञान‚मित्य‚थात् ।
	\pend% ending standard par
      ‚{\tiny $_{lb}$}‚

	  \pstart \leavevmode% starting standard par
	एवं च द‚र्श‚य‚ता \textbf{वार्त्तिक‚कारेण} य‚द्वेन्द्रियं प्र‚माणं स्यात् त‚स्य वार्थेन स‚ङ्ग‚तिः \href{http://sarit.indology.info/?cref=śv.4.60}{श‚लोक‚वा० 	४--६०} इत्यादि व‚च‚नात्त‚स्यापि या प्र‚त्य‚क्ष‚प्र‚माण‚ता \textbf{मीमांस‚का}दिभिर‚भ्युप‚ग‚ता सा निर‚स्ता‚{\tiny $_{lb}$}‚ वेदित‚व्या । त‚थाहि--प्राप‚कं प्र‚माणं भ‚वेत् । नान्य‚त् । प्राप‚क‚त्वं च प्र‚व‚र्त्त‚क‚त्वेना\add{व्य‚व‚धानेन‚{\tiny $_{lb}$}‚ नान्य‚था} । प्र‚व‚र्त्त‚क‚त्वं चाधिग‚मात्म‚क‚त्वेन नान्य‚था । न च त‚स्य त‚थात्व‚म‚स्ति ।‚{\tiny $_{lb}$}‚ अधिग‚मोप‚ग‚मेन प्र‚व‚र्त्त‚क‚त्वे च नैव व्य‚व‚धीय‚ते । य‚त‚श्चाव्य‚व‚धानेन प्र‚वृत्तिः स एव प्र‚व‚र्त्त‚क‚त्वात्‚{\tiny $_{lb}$}‚ प्र‚माणं युक्त‚म् । त‚च्च ज्ञान‚मेव । त‚दुक्त‚म्, 
	    \pend% close preceding par
	  
	    \begin{quote}
	  
	    
	    \stanza[\smallbreak]
	धीप्र‚माण‚तां&प्र‚वृत्तेस्त‚त्प्र‚धान‚त्वाद्धेयोपादेय‚{\tiny $_{lb}$}‚व‚स्तुनि\&[\smallbreak]


	
	    \end{quote}
	  
	    \pstart  \leavevmode% new par for following
	    \hphantom{.}
	   \href{http://sarit.indology.info/?cref=pv.1.5}{प्र‚माण‚वा० १. ५} इति ।
	\pend% ending standard par
      ‚{\tiny $_{lb}$}‚

	  \pstart \leavevmode% starting standard par
	य‚द्वा प्र‚मायाः क‚र‚णात् प्र‚माण‚म् । प्र‚मा च नील‚स्येयं न पीत‚स्येति विशिष्टेनैवाधि‚{\tiny $_{lb}$}‚क‚र‚णेनाव‚च्छिन्ना प्र‚तीय‚त इति त‚त्क‚र‚णं भ‚वितुम‚र्ह‚ति । य‚त इय‚म‚व्य‚व‚धाना ताद्रूप्येण व्य‚व‚स्थां‚{\tiny $_{lb}$}‚ ल‚भ‚ते । इन्द्रियार्थ‚स‚न्निक‚र्षादिश्च नील‚पीतादि\edtext{}{\lemma{पीतादि}\Bfootnote{अस्प‚ष्ट‚म्--सं०}}\add{... ... ...}स्य भेद‚व्य‚व‚स्थाङ्ग‚म् । स्व‚भेदेनं‚{\tiny $_{lb}$}‚ क्रियाभेद‚निब‚न्ध‚नं च क‚र‚णं त‚तो ज्ञानात्म‚क‚मेव सारूप्य‚म‚धिग‚तिक्रियाभेद‚व्य‚व‚स्थाङ्गं प्र‚माणं‚{\tiny $_{lb}$}‚ युज्य‚त इति । द्व‚यी चेयं विधाऽन्य‚त्रावार्येण विप‚ञ्चितेति नेहोच्य‚ते, प्राज्ञ‚ज‚नाधिकारेणास्य‚{\tiny $_{lb}$}‚ प्र‚क‚र‚ण‚स्य प्रार‚म्भात् । दिङ्म‚त्रिं तूक्त‚मिति ।
	\pend% ending standard par
      ‚{\tiny $_{lb}$}‚‚{\tiny $_{lb}$}‚\textsuperscript{\textenglish{58/dm}}‚{\tiny $_{lb}$}‚
	  \bigskip
	  \begingroup
	

	  \pstart \leavevmode% starting standard par
	स्व आत्मीयो विष‚य इन्द्रिय\edtext{}{\lemma{इन्द्रिय}\Bfootnote{इन्द्रिय‚विज्ञान० \cite{dp-msB}}} ज्ञान‚स्य त‚स्य अन‚न्त‚रः--न विद्य‚तेऽन्त‚र‚म‚स्येति\edtext{}{\lemma{स्येति}\Bfootnote{०स्येति अन‚न्त‚रः \cite{dp-msB} \cite{dp-msD}}} । अन्त‚रं‚{\tiny $_{lb}$}‚ \edtext{\textsuperscript{*}}{\lemma{*}\Bfootnote{च नास्ति--\cite{dp-msC}}}च व्य‚व‚धानं विशेष‚श्चोच्य‚ते । त‚त‚श्चान्त‚रे प्र‚तिषिद्धे स‚मान‚जातीयो द्वितीय‚क्ष‚ण‚भाव्युपादेय‚क्ष‚ण‚{\tiny $_{lb}$}‚ इन्द्रिय‚विज्ञान‚विष‚य‚स्य गृह्य‚ते । त‚था च स‚ति इन्द्रिय\edtext{}{\lemma{इन्द्रिय}\Bfootnote{इन्द्रिय‚विज्ञान०--\cite{dp-msC}}} ज्ञान‚विष‚य‚क्ष‚णादुत्त‚र‚क्ष‚ण एक‚स‚न्ता‚{\tiny $_{lb}$}‚नान्त‚र्भूतो गृहीतः । स स‚ह‚कारी य‚स्येन्द्रिय‚ज्ञान‚स्य\edtext{}{\lemma{स्य}\Bfootnote{इन्द्रिय‚विज्ञान० \cite{dp-msA} \cite{dp-msB} \cite{dp-msD} \cite{dp-edP} \cite{dp-edH} \cite{dp-edE} \cite{dp-edN}}} त‚त् त‚थोक्त‚म् । द्विविध‚श्च स‚ह‚कारी—‚{\tiny $_{lb}$}‚प‚र‚स्प‚रोप‚कारी एक‚कार्य‚कारी च । इह च क्ष‚णिके व‚स्तुन्य‚तिश‚याधानाऽयोगादेक‚कार्य‚कारित्वेन\edtext{}{\lemma{कारित्वेन}\Bfootnote{अत्र प्र‚दीप‚स‚म्म‚तः--एकार्थ‚क्रियाकारित्वेन इति पाठः ।--सं०}}‚{\tiny $_{lb}$}‚ स‚ह‚कारी गृह्य‚ते ।
	\pend% ending standard par
      
	  \endgroup
	‚{\tiny $_{lb}$}‚

	  \pstart \leavevmode% starting standard par
	न‚नु च‚तुर्विध‚प्र‚त्य‚क्षाभिधाने निमित्त‚मुक्त‚मेव ध‚र्मोत्त‚रेण त‚त्क‚थं त‚देव \textbf{मान‚से}त्यादिना‚{\tiny $_{lb}$}‚ पुन‚राच‚ष्टे । अथाचार्य‚स्य च‚तुष्ट‚याभिधाने निमित्तं य‚दासीत् बुद्धिस्थं त‚त्प्रागुक्त‚म्, त‚देव तु‚{\tiny $_{lb}$}‚ य‚थायोगं प्र‚त्य‚क्ष‚व्य‚क्तिविशेषाभिधानेऽभिधानीय‚मिति । त‚र्हि यैरिन्द्रियं द्र‚ष्टृ क‚ल्पितं त‚न्निरा‚{\tiny $_{lb}$}‚क‚र‚णार्थ‚माह--इन्द्रियंज्ञान‚मिति त‚थाऽन्य‚त्रापि स्व‚संवेद‚नादौ त‚त्त‚न्निमित्त‚म‚नेन किन्नोक्त‚म् ?‚{\tiny $_{lb}$}‚ त‚द‚त्रापि मान‚सं व्याख्यातुमाहेति व‚क्तुं युक्त‚मिति । स‚त्य‚मेत‚त् । केव‚लं मान‚स‚स्य प्र‚त्य‚क्ष‚स्या‚{\tiny $_{lb}$}‚श‚क्य‚निश्च‚य‚त्वेनाव्य‚व‚हाराङ्ग‚त्वात्प‚रोक्त‚दोष‚निराक‚र‚ण‚मात्र‚मेवास्य ल‚क्ष‚णाख्याने निमित्तं नान्य‚{\tiny $_{lb}$}‚दिति प्र‚तिपाद‚यितुमुक्त‚मेव निमित्त‚म‚नूदित‚मिति न किञ्चिद‚व‚द्य‚म् । अथेन्द्रिय‚ज्ञान‚स्यापि‚{\tiny $_{lb}$}‚ ल‚क्ष‚ण\leavevmode\ledsidenote{\textenglish{25b/ms}}म‚न्येषामिव किं नोक्त‚म् ? नोक्त‚म् । त‚ल्ल‚क्ष‚ण‚विप्र‚तिप‚त्त्य‚भावात् । इन्द्रिय‚{\tiny $_{lb}$}‚ज्ञान‚मिति निग‚देनैव व्याख्यात‚त्वाच्चेति ।
	\pend% ending standard par
      ‚{\tiny $_{lb}$}‚

	  \pstart \leavevmode% starting standard par
	स्यान्म‚त‚म् । अभिम‚तं च‚तुर्विधं प्र‚त्य‚क्षं द‚र्श‚यित‚व्य‚म् । त‚च्चैक‚दा व‚क्तुम‚श‚क्य‚त्वात्‚{\tiny $_{lb}$}‚ क्र‚मेण व‚च‚नीय‚म् । क्र‚म‚श्चैवंविधोऽस्त्व‚न्यादृशो वा किम‚त्राद‚रेणेति इन्द्रिय‚ज्ञानादिरीदृशः क्र‚मः‚{\tiny $_{lb}$}‚ कृतः, अथान्य‚द‚प्य‚स्य निमित्त‚म‚स्तीति ? अन्य‚द‚प्य‚स्तीत्युच्य‚ते । त‚थाहि स‚र्व‚स्य स‚र्व‚व्य‚व‚{\tiny $_{lb}$}‚हाराङ्ग‚मिन्द्रिय‚ज्ञान‚मिति त‚दादितोऽभिहित‚म् । तेनैवेन्द्रिय‚ज्ञानेन त‚थाभूतेन मान‚सं ज‚न्य‚त‚{\tiny $_{lb}$}‚ इति त‚द‚नु मान‚सं व्याख्यात‚म् । त‚स्य प‚श्चात्प्र‚त्यात्म‚वेद्य‚त‚याऽस्तिवेन श‚क्य‚निश्च‚य‚त‚या च‚{\tiny $_{lb}$}‚ स्व‚संवेद‚न‚मुदित‚म् । त‚तः स‚र्व‚ज्ञ‚ज्ञान‚स्य साम‚ग्रीमात्र‚प्र‚द‚र्श‚न‚त‚यऽन्तिम‚पुरुषार्थ‚त‚या चान्ते योगि‚{\tiny $_{lb}$}‚ज्ञान‚मुप‚द‚र्शित‚मिति ।
	\pend% ending standard par
      ‚{\tiny $_{lb}$}‚

	  \pstart \leavevmode% starting standard par
	\textbf{स्व} इत्यादिना \textbf{स्व‚विष‚ये}त्यादिल‚क्ष‚णं व्याच‚ष्टे । त‚च्च \textbf{त‚त्त‚थोक्त‚मि}त्येत‚द‚न्तं सुग‚म‚म् ।
	\pend% ending standard par
      ‚{\tiny $_{lb}$}‚

	  \pstart \leavevmode% starting standard par
	न‚न्व‚नुप‚कार‚क‚स्य स‚ह‚कारित्वेऽतिप्र‚स‚ङ्गाद् उप‚कारिणा स‚ह‚कारिणा भाव्य‚म् । स‚म‚स‚म‚य‚यो‚{\tiny $_{lb}$}‚श्चोप‚कार्योप‚कार‚क‚भावाभावात् क‚थ‚म‚सौ विष‚य‚क्ष‚ण‚स्त‚स्य स‚ह‚कारीत्याश‚ङ्क्याह--\textbf{द्विविध‚श्चे}ति ।‚{\tiny $_{lb}$}‚ \textbf{चो} य‚स्मात् । \textbf{द्विविधो} द्विप्र‚कारः स‚ह‚क‚र‚ण‚शीलो भावः । क‚थं द्वैविध्य‚मित्याह—‚{\tiny $_{lb}$}‚\textbf{प‚र‚स्प‚रे}त्यादि । प‚र‚स्प‚रोप‚कारित‚या च स‚ह‚कारित्व‚मेक‚त्वाध्य‚व‚सायाधीनं स‚न्तान‚ग‚त‚म‚मुख्य‚म् ।‚{\tiny $_{lb}$}‚ मुख्यं तु एकार्थ‚क्रियाकारित्वेन । य‚दि द्वैविध्येऽपि प्राक्त‚न‚स्य ग्र‚ह‚णं त‚दा त‚द‚व‚स्थो दोष इत्याह—‚{\tiny $_{lb}$}‚\textbf{इहे}ति । \textbf{इह} मान‚स‚क्रियायाम् । \textbf{चो}ऽव‚धार‚णे । \textbf{एकार्थ‚क्रियाकारित्वेने}त्य‚स्मात् प‚रो‚{\tiny $_{lb}$}‚ द्र‚ष्ट‚व्यः । \textbf{स‚ह‚कारी हि} गृह्य‚ते विष‚य‚क्ष‚ण इति प्र‚क‚र‚णाद‚व‚सेय‚म् । पूर्वं क‚स्मान् न गृह्य‚त‚{\tiny $_{lb}$}‚ ‚{\tiny $_{lb}$}‚ \leavevmode\ledsidenote{\textenglish{59/dm}}‚{\tiny $_{lb}$}‚ 
	  
	विष‚य‚विज्ञानाभ्यां \edtext{}{\lemma{विज्ञानाभ्यां}\Bfootnote{भ्यां म‚नो \cite{dp-msD} \cite{dp-msB}}}हि म‚नोविज्ञान‚मेकं क्रिय‚ते य‚त‚स्त‚द‚न‚योः\edtext{}{\lemma{योः}\Bfootnote{०न‚योर्न प‚र‚स्प० \cite{dp-msA} \cite{dp-msC} \cite{dp-edP} \cite{dp-edH} \cite{dp-edE} \cite{dp-edN}}} प‚र‚स्प‚र \edtext{}{\lemma{र}\Bfootnote{प‚र‚स्प‚र‚स्य स‚ह \cite{dp-msB} \cite{dp-msD}}}स‚ह‚कारित्व‚म् । ‚{\tiny $_{lb}$}‚ 
	  
	\edtext{\textsuperscript{*}}{\lemma{*}\Bfootnote{स्व‚विष‚यान‚न्त‚रेत्यादिना--\cite{dp-msD-n}}}ईदृशेनेद्रिन्य‚विज्ञानेनाल\edtext{}{\lemma{विज्ञानेनाल}\Bfootnote{आल‚म्ब‚न‚प्र‚त्य‚य‚भूतेनापि \cite{dp-msA} \cite{dp-msB} \cite{dp-msC} \cite{dp-msD} \cite{dp-edP} \cite{dp-edH} \cite{dp-edE} \cite{dp-edN} य‚दा योगिना प‚र‚स्यै‚{\tiny $_{lb}$}‚वंविध‚ज्ञान‚माल‚म्ब्य‚ते त‚दाल‚म्ब‚न‚भूतेन तेन च योगिज्ञानं ज‚न्य‚ते इति--\cite{dp-msD-n}}}म्ब‚न‚भूतेनापि\edtext{}{\lemma{भूतेनापि}\Bfootnote{अपिश‚ब्दो भिन्न‚क्र‚मे--योगिज्ञान‚म‚पि इति--\cite{dp-msD-n}}} योगिज्ञानं ज‚न्य‚ते । त‚न्निरासार्थं‚{\tiny $_{lb}$}‚ \edtext{\textsuperscript{*}}{\lemma{*}\Bfootnote{बौद्धानां म‚ते स‚म‚न‚न्त‚र‚प्र‚त्य‚येति उपादान‚कार‚ण‚मुच्य‚ते--\cite{dp-msD-n}}}स‚म‚न‚न्त‚र‚प्र‚त्य‚य‚ग्र‚ह‚णं \edtext{}{\lemma{णं}\Bfootnote{कृत‚म् इति नास्ति \cite{dp-msD} \cite{dp-msB}}}कृत‚म् । स‚म‚श्चासौ ज्ञान‚त्वेन, अन‚न्त‚र‚श्चा\edtext{}{\lemma{श्चा}\Bfootnote{श्चाव्य‚व \cite{dp-msD} \cite{dp-msB}}}सौ अव्य‚व‚हित‚त्वेन, स‚{\tiny $_{lb}$}‚ चासौ प्र‚त्य‚य‚श्च हेतुत्वात् स‚म‚न‚न्त‚र‚प्र‚त्य‚यः, तेन ज‚नित‚म् ।‚{\tiny $_{lb}$}‚ इत्याह--\textbf{क्ष‚णिक} इति । हेतुभावेन विशेष‚ण‚म् । अथ य‚दि व‚स्तुनः क्ष‚णिक‚त्वेनातिश‚याधाना‚{\tiny $_{lb}$}‚स‚म्भ‚वात् प‚र‚स्प‚रोप‚कारित्वेन न स‚ह‚कारी गृह्य‚ते \textbf{द्विविध‚श्च स‚ह‚कारी}त्य‚न‚न्त‚र‚मुदित‚म‚नेन‚{\tiny $_{lb}$}‚ व्याह‚न्येत । \textbf{हेतुबिन्दुश्च} विरुध्येतेति स‚र्व‚म‚स‚म‚ञ्ज‚स‚म् । न । अभिप्रायाप‚रिज्ञानात् ।‚{\tiny $_{lb}$}‚ त‚था हि \textbf{क्ष‚णिके व‚स्तुनी}ति धार‚याऽप्र‚वाहिणि स‚जातीयाप्र‚स‚व‚ध‚र्मिण्य‚न्तिमे मान‚सोत्पाद‚क‚त‚याऽ‚{\tiny $_{lb}$}‚भिम‚त इन्द्रिय‚ज्ञान‚नाम्नि निरोधोन्मुख इत्य‚भिप्रेत‚म् । न त्व‚क्ष‚णिक‚व्यावृत्त्या क्ष‚णिक‚प‚दाभि‚{\tiny $_{lb}$}‚लाप्येऽव‚स्तुमात्र‚व्यावृत्त्या च व‚स्तुमात्र इति ।
	\pend% ending standard par
      ‚{\tiny $_{lb}$}‚

	  \pstart \leavevmode% starting standard par
	एत‚दुक्तं भ‚व‚ति । य‚दीदं मान‚सोप‚ज‚न‚नोन्मुख‚मिन्द्रिय‚ज्ञान‚मेक‚जातीय‚प्र‚वाह‚वाहि स्यात्त‚दा‚{\tiny $_{lb}$}‚ स‚न्तानोप‚कार‚द्वार‚कातिश‚याधाय‚क‚त्वे विष‚योऽस्य स‚ह‚कारीति क‚ल्प्येत । केव‚ल‚मिद‚म‚पेत‚{\tiny $_{lb}$}‚स‚जातीय‚प्र‚स‚व‚श‚क्तिकं स‚न्तान‚व‚र्त्तीति क‚थ‚म‚स्य विष‚य‚स्त‚था स‚ह‚कारी क‚ल्प्येतेति । न‚नु किं‚{\tiny $_{lb}$}‚ त‚देकं य‚त्कुर्व‚तोरेत‚योस्त‚थास‚ह‚कारित्व‚मुच्य‚त इत्याह \textbf{विष‚ये}ति । \textbf{हि}र‚व‚धार‚णे । \textbf{य‚तो}‚{\tiny $_{lb}$}‚ य‚स्मादाभ्यामेव \textbf{त‚देकं क्रिय‚ते,} त‚स्मात् । \textbf{त‚दि}त्य‚यं त‚स्मादित्य‚स्मिन्न‚र्थे । अन्योन्य‚स्य स‚ह‚कारित्व‚{\tiny $_{lb}$}‚मेकार्थ‚क्रियाकारित्वेनेति प्र‚क‚र‚णात् ।
	\pend% ending standard par
      ‚{\tiny $_{lb}$}‚

	  \pstart \leavevmode% starting standard par
	ईदृशेनेत्यादिना स‚म‚न‚न्त‚र‚प्र‚त्य‚य‚ग्र‚ह‚ण‚स्य व्य‚व‚च्छेदं द‚र्श‚य‚ति । \textbf{ईदृशेन} स्व‚विष‚यान‚न्त‚र‚{\tiny $_{lb}$}‚विष‚य‚ग्र‚ह\leavevmode\ledsidenote{\textenglish{26a/ms}}ण‚स‚ह‚कारिणा । आल‚म्ब्य‚त इति \textbf{आल‚म्ब‚न‚भूतं} रूपं य‚स्य तेन‚{\tiny $_{lb}$}‚ ग्राह्य‚स्व‚भावेन स‚ता ज‚न्य‚त इत्य‚र्थः । त‚स्य योगिज्ञान‚स्य \textbf{निरास‚र्थं} मान‚स‚त्व‚निराक‚र‚णार्थ‚म् ।‚{\tiny $_{lb}$}‚ एव‚ञ्च व्याच‚क्षाणेन स‚म‚न‚न्त‚र‚प्र‚त्य‚य‚ग्र‚ह‚ण‚म‚न्ध‚ब‚धिराद्य‚भाव‚प्र‚स‚ङ्ग‚निराक‚र‚णार्थ‚मिति य‚द‚न्येन‚{\tiny $_{lb}$}‚ व्याख्यात‚म्--त‚द‚प‚ह‚स्तित‚म् । इन्द्रिय‚ज्ञान‚ग्र‚ह‚णेनैव त‚त्प्र‚स‚ङ्ग‚स्य निराकृत‚त्वादिति ।
	\pend% ending standard par
      ‚{\tiny $_{lb}$}‚

	  \pstart \leavevmode% starting standard par
	\textbf{स‚म‚श्चे}त्यादिना स‚म‚न‚न्त‚र‚प्र‚त्य‚य‚श‚ब्द‚स्य विग्र‚ह‚म‚र्थं चाह । हेत्व‚र्थः प्र‚त्य‚यार्थः । श‚क‚न्ध्वा‚{\tiny $_{lb}$}‚दिपाठाच्च न दीर्घ‚त्व‚म् । य‚दि नामैवं व्युत्प‚त्तिस्त‚थापि प्राक्त‚न‚मेव विज्ञान‚मुपादान‚भूत‚माग‚मे‚{\tiny $_{lb}$}‚ त‚था रूढं नान्य‚त् । त‚दुक्त\textbf{म‚भिध‚र्म‚कोशे}--चित्त‚चैत्ता अच‚र‚मा उत्प‚न्नाः स‚म‚न‚न्त‚रः \href{http://sarit.indology.info/?cref=ak.2.62}{२. ६२}‚{\tiny $_{lb}$}‚ इति । सिद्धान्त‚प्र‚सिद्ध‚या चामुया संज्ञ‚या व्य‚व‚ह‚र‚ताऽऽचार्येण सिद्धान्त‚प्र‚सिद्ध‚त्व‚म‚स्य द‚र्शित‚म् ।
	\pend% ending standard par
      ‚{\tiny $_{lb}$}‚‚{\tiny $_{lb}$}‚\textsuperscript{\textenglish{60/dm}}‚{\tiny $_{lb}$}‚
	  \bigskip
	  \begingroup
	

	  \pstart \leavevmode% starting standard par
	त‚द‚नेनैक\unclear{स}न्तानान्त‚र्भूत‚योरेवेन्द्रिय\edtext{}{\lemma{योरेवेन्द्रिय}\Bfootnote{०न्द्रिय‚विज्ञान--\cite{dp-msC}}}ज्ञान-म‚नोवि\edtext{}{\lemma{नोवि}\Bfootnote{म‚नोज्ञान \cite{dp-msA} \cite{dp-msB} \cite{dp-edP} \cite{dp-edH} \cite{dp-edE}}}ज्ञान‚योर्ज‚न्य‚ज‚न‚क‚भावे म‚नोविज्ञानं‚{\tiny $_{lb}$}‚ \unclear{प्र‚त्य‚क्ष‚मि}\edtext{\textsuperscript{*}}{\lemma{*}\Bfootnote{०क्ष‚मुक्तं \cite{dp-msB}--}}त्युक्तं भ‚व‚ति । त‚तो योगिज्ञानं प‚र‚स‚न्तान‚व‚र्त्ति निर‚स्त‚म् ।
	\pend% ending standard par
      
	  \endgroup
	‚{\tiny $_{lb}$}‚

	  \pstart \leavevmode% starting standard par
	न‚नु किं कृतं स‚म‚न‚न्त‚र‚प्र‚त्य‚य‚ग्र‚ह‚णेन येन योगिज्ञान‚स्य त‚थात्वेन निरास इत्याह--त‚दिति ।‚{\tiny $_{lb}$}‚ य‚स्मात्स‚म‚न‚न्त‚र‚प्र‚त्य‚य‚रूपेणेन्द्रिय‚ज्ञानेन य‚ज्ज‚नितं त‚न्मान‚सं त‚त्त‚स्मात्कार‚णाद‚नेन स‚म‚न‚न्त‚र‚{\tiny $_{lb}$}‚\unclear{प्र‚त्य‚य‚ग्र‚ह‚णेनैक‚स‚न्तानान्त‚र्भू} त‚योरेक‚स‚न्त‚तिप‚तित‚योरेव \textbf{ज‚न्य‚ज‚न‚क‚भावे} स‚ति \textbf{म‚नोविज्ञान‚मि}ति ।‚{\tiny $_{lb}$}‚ \unclear{साक्षादिन्द्रियाज‚न्य‚त‚या} म‚नोमात्राश्रित‚त्वान्मान‚स‚मेवोक्त‚म् । आग‚मेऽपि म‚नःश‚ब्देनैत‚देवोच्य‚त‚{\tiny $_{lb}$}‚ इति \unclear{प्र‚द‚र्श‚नार्थं} चैव‚म‚भिधानं म‚न्त‚व्य‚म् । त‚थापि क‚थं योगिज्ञानं त‚थात्वेन निर‚स्त‚मित्याह—‚{\tiny $_{lb}$}‚\unclear{त‚त्र} इति । य‚स्मादु\unclear{प‚दानो}पादेय‚भूत‚योर्ग्र‚ह‚णं भिन्न‚स‚न्तान‚व‚र्त्तिनोश्चोपादानोपादेय‚भावाभावः,‚{\tiny $_{lb}$}‚ \unclear{त‚स्माद् योगिज्ञानं निर‚स्तं मान}स‚त्वेन प्र‚तिक्षिप्त‚म् । य‚दि पुन‚रिन्द्रिय‚ज्ञानेन ज‚नित‚मित्येता‚{\tiny $_{lb}$}‚व‚दुच्येत त‚दाऽऽल‚म्ब‚न‚भूतेनापि तेनादौ ज‚न्य‚त इति निर‚स्तं न स्यादिति भावः ।
	\pend% ending standard par
      ‚{\tiny $_{lb}$}‚

	  \pstart \leavevmode% starting standard par
	\unclear{स्यादेत‚त्--य‚द्ये}त‚द‚र्थं स‚म‚न‚न्त‚र‚प्र‚त्य‚य‚ग्र‚ह‚णं त‚र्हि नित‚रां न क‚र्त्त‚व्य‚म् । स्व‚विष‚यान‚न्त‚र‚{\tiny $_{lb}$}‚\unclear{विष‚य‚क्ष‚ण‚स‚ह‚कारिणेत्य‚नेनैव} योगिज्ञान‚स्य निर‚स्त‚त्वात् । न हि त‚दिन्द्रिय‚ज्ञानं योगिज्ञाने क‚र्त्त‚व्ये‚{\tiny $_{lb}$}‚ \unclear{स्व‚विष‚यान‚न्त‚र‚वि}ष‚य‚क्ष‚ण‚स‚ह‚कारीति । अथोच्य‚ते--य‚देन्द्रिय‚ज्ञानं योगिज्ञानं ज‚न‚य‚ति त‚दा‚{\tiny $_{lb}$}‚ \unclear{स्वो}पादेय‚म‚पि ज्ञानं ज‚न‚य‚त्येव । त‚तः स्व‚स‚न्त‚तिप‚तित‚ज‚न्य‚क्ष‚णापेक्ष‚या त‚द् भ‚व‚ति स्व‚विष‚या‚{\tiny $_{lb}$}‚न‚न्त‚र‚विष‚य‚क्ष‚ण‚म‚ह‚कारीत्य‚स्ति त‚ज्ज‚नित‚स्य त‚स्य मान‚स‚त्व‚ल‚क्ष‚ण‚मिति । अहो श‚ब्दार्थ‚व्य‚{\tiny $_{lb}$}‚व‚स्थाप‚न‚कौश‚ल‚म्? य‚देव हि ज‚न्य‚त‚या प्र‚कृतं त‚द‚पेक्ष‚मेव स‚ह‚कारिणः स‚ह‚कारित्वं चिन्त‚नीय‚म् ।‚{\tiny $_{lb}$}‚ न तु य‚त्किञ्चिद‚पेक्ष‚म् । न च त‚देव योगिज्ञानं प्र‚ति त‚त् स‚ह‚कारि त‚दिन्द्रिय‚ज्ञानं येन‚{\tiny $_{lb}$}‚ त‚थात्व‚म‚स्योच्य‚मानं शोभेत । अथ त‚त् स‚ह‚कारि त‚द् व‚स्तुवृत्त्या त‚दिन्द्रिय‚ज्ञान‚म् ।‚{\tiny $_{lb}$}‚ \unclear{त‚त् किम्} त‚द‚पेक्ष‚या चिन्त‚येति चेत् । य‚द्येवं कृतेऽपि स‚म‚न‚न्त‚र‚प्र‚त्य‚य‚ग्र‚ह‚णे योगिज्ञानं निर‚सितु‚{\tiny $_{lb}$}‚\unclear{म‚श‚क्यं} त‚थापि य‚था त‚दिन्द्रिय‚ज्ञानं स्वोपादेय‚ज्ञानापेक्षं स्व‚विष‚यान‚न्त‚र‚विष‚य‚स‚ह‚कारि त‚था‚{\tiny $_{lb}$}‚ त‚देवापेक्ष्य स‚म‚न‚न्त‚र‚प्र‚त्य‚यो\unclear{ऽ}प्य‚दो भ\leavevmode\ledsidenote{\textenglish{26b/ms}}व‚त्येव । त‚त‚स्त‚थाभूतेन तेन ज‚न्य‚मान‚स्य‚{\tiny $_{lb}$}‚ योगिज्ञान‚स्य क‚थं त‚थात्वेन निरासः । न‚नु न त‚द‚पेक्षं त‚स्य स‚म‚न‚न्त‚र‚प्र‚त्य‚य‚त्व‚म् । ह‚न्त‚{\tiny $_{lb}$}‚ त‚त्स‚ह‚कारित्व‚म‚पि किम‚स्य त‚द‚पेक्ष‚म‚स्ति ? अथ स‚म‚न‚न्त‚र‚प्र‚त्य‚य‚त्वं प्र‚कृत‚ज‚न्यापेक्षं ग्राह्य‚म्,‚{\tiny $_{lb}$}‚ \unclear{न तु वास्त‚वं} रूप‚मुपादेयं फ‚लाभावेन त‚स्यानुवाद्य‚ताऽनुप‚प‚त्तेः । अन्य‚था स‚म‚स्त‚वास्त‚व‚रूपानु‚{\tiny $_{lb}$}‚\unclear{वाद्य‚ताप्र‚सं}\add{गा}दिति चेत् । स‚मान‚मिदं स्व‚विष‚य‚यान‚न्त‚र‚विष‚य‚क्ष‚ण‚स‚ह‚कारिग्र‚ह‚णेऽपि । न च‚{\tiny $_{lb}$}‚ \unclear{राज‚शास‚न‚स}मानं किञ्चिदुत्प‚श्यामो येन स‚म‚न‚न्त‚र‚प्र‚त्य‚य‚ज‚न्य‚त्व‚मेव प्र‚कृत‚ज‚न्यापेक्ष‚मिह ग्राह्य‚म्,‚{\tiny $_{lb}$}‚ न तु स्व‚विष‚यान‚न्त‚र‚विष‚य‚क्ष‚ण‚स‚ह‚कारित्व‚मिति क‚ल्प‚नीय‚मिति ।
	\pend% ending standard par
      ‚{\tiny $_{lb}$}‚

	  \pstart \leavevmode% starting standard par
	अत्रोच्य‚ते । य‚दा--योगी इन्द्रिय‚ज्ञानं त‚त्स‚ह‚भुवं चार्थ‚क्ष‚णं च कुत‚श्चित्कार‚णाद्दिदृक्ष‚माणो‚{\tiny $_{lb}$}‚ योगिज्ञानेन त‚द्द्व‚य‚म‚पि प‚श्य‚ति त‚दा तेनेन्द्रिय‚ज्ञानेन स्व‚विष‚यान‚न्त‚र‚विष‚य‚क्ष‚ण‚स‚ह‚कारिणा त‚द्‚{\tiny $_{lb}$}‚ योगिज्ञानं ज‚न्य‚त इति अस‚ति स‚म‚न‚न्त‚र‚प्र‚त्य‚य‚ग्र‚ह‚णे मान‚स‚प्र‚त्य‚क्षं स‚मास‚ज्येत । त‚त‚स्त‚न्नि‚{\tiny $_{lb}$}‚वृत्त्य‚र्थं क‚र्त्त‚व्य‚मेव स‚म‚न‚न्त‚र‚प्र‚त्य‚य‚ग्र‚ह‚ण‚म् । एवंविध‚मेव च योगिज्ञान‚म‚भिप्र‚त्येत्येदृशेनेत्येभिहितं‚{\tiny $_{lb}$}‚ \textbf{ध‚र्मोत्त‚रेणे}ति स‚र्व‚म‚व‚दात‚म् ।
	\pend% ending standard par
      ‚{\tiny $_{lb}$}‚‚{\tiny $_{lb}$}‚\textsuperscript{\textenglish{61/dm}}‚{\tiny $_{lb}$}‚
	  \bigskip
	  \begingroup
	

	  \pstart \leavevmode% starting standard par
	\edtext{\textsuperscript{*}}{\lemma{*}\Bfootnote{अत्र प‚र‚स्याय‚माश‚यः--म‚नोविज्ञान‚मिन्द्रिय‚स‚व्य‚पेक्ष‚म् । अथ चेन्द्रिय‚विज्ञानं[[ना]]‚{\tiny $_{lb}$}‚द‚न्यो विष‚य‚स्त‚स्य । त‚तोऽन्ध‚ब‚धिराद्य‚भावो व्य‚व‚हित‚स्य नीलादेर्ग्र‚ह‚णं च प्राप्नोतीत्याह‚{\tiny $_{lb}$}‚ \textbf{य‚दे}त्यादि । अय‚म‚र्थः--य‚स्मादिन्द्रिय‚विज्ञान‚विष‚योपादेय‚भूतः क्ष‚ण एको गृहीत‚क्ष‚णो न भ‚व‚त्य‚स्य‚{\tiny $_{lb}$}‚ विष‚यः, अन्ध‚ब‚धिरादेश्च नेन्द्रिय‚ज्ञान‚विष‚य‚स‚ह‚कारि विद्य‚ते तेन तेषां न म‚नोविज्ञानं‚{\tiny $_{lb}$}‚ भ‚व‚तीत्य‚र्थः--\cite{dp-msD-n}}}य‚दा चेन्द्रिय‚ज्ञान‚विष‚याद‚न्यो विष‚यो म‚नोविज्ञान‚स्य त‚दा गृहीत‚ग्र‚ह‚णादास‚ञ्जितोऽ\edtext{}{\lemma{ञ्जितोऽ}\Bfootnote{क‚ल्पितः--\cite{dp-msD-n}}}‚{\tiny $_{lb}$}‚प्रामाण्य‚दोषो निर‚स्तः ।
	\pend% ending standard par
       ‚{\tiny $_{lb}$}‚ 

	  \pstart \leavevmode% starting standard par
	य‚दा चेन्द्रिय\edtext{}{\lemma{चेन्द्रिय}\Bfootnote{०न्द्रिय‚ज्ञान० \cite{dp-msA} \cite{dp-msB} \cite{dp-msC} \cite{dp-edP} \cite{dp-edH} \cite{dp-edE} \cite{dp-edN}}} विज्ञान‚विष‚योपादेय‚भूतः\edtext{}{\lemma{भूतः}\Bfootnote{०भूतः क्ष‚ण एको गृहीतः \cite{dp-msD}}} क्ष‚णो गृहीतः, त‚देन्द्रिय‚ज्ञानेनागृहीत‚स्य विष‚या‚{\tiny $_{lb}$}‚न्त‚र‚स्य ग्र‚ह‚णाद‚न्ध‚ब‚धिराद्य‚भाव‚दोष‚प्र‚स‚ङ्गो\edtext{}{\lemma{ङ्गो}\Bfootnote{प्र‚स‚ङ्गोऽपि निर० \cite{dp-msD}}}निर‚स्तः ।
	\pend% ending standard par
      
	  \endgroup
	‚{\tiny $_{lb}$}‚

	  \pstart \leavevmode% starting standard par
	स‚म्प्र‚ति य‚था गृहीत‚ग्राहित्व‚द्वार‚को दोषः प‚र‚स्य प‚रिह्रिय‚ते त‚था द‚र्श‚य‚न्नाह--\textbf{य‚दा चे}ति ।‚{\tiny $_{lb}$}‚ \textbf{चो} व्य‚क्त‚मेत‚दित्य‚स्मिन्न‚र्थे । \textbf{य‚दैवं} त‚दा निर‚स्तः । इन्द्रिय‚ज्ञान‚विष‚याद‚न्योऽस्य विष‚य‚{\tiny $_{lb}$}‚ इत्येत‚देव कुतो येनासौ निर‚स्तो भ‚व‚ति इत्याश‚ङ्क्याह--\textbf{य‚दा चे}ति । \textbf{चोऽ}व‚धार‚णे । \textbf{इन्द्रिय‚{\tiny $_{lb}$}‚विज्ञान‚विष‚योपादेय‚भूत} इत्य‚स्मात्प‚रो द्र‚ष्ट‚व्यः । इन्द्रिय‚ज्ञान‚स्य विष‚यो य‚स्त‚स्योपादेय‚भूतो‚{\tiny $_{lb}$}‚ यः क्ष‚णः स \textbf{गृहीत} इन्द्रिय‚ज्ञान‚स्य मान‚सोप‚ज‚न‚ने स‚ह‚कारिकार‚ण‚त्वेन स्वीकृत‚स्त‚देतीहैव‚च्छेदः‚{\tiny $_{lb}$}‚ क‚र्त्त‚व्यः । एवं ब्रुव‚तेद‚माकूत‚म्--ज‚न‚क‚विशेष‚स्य विष‚य‚त्वान्मान‚स‚प्र‚त्य‚क्ष‚स्य विष‚येण भ‚व‚ता‚{\tiny $_{lb}$}‚ ज‚न‚केनाव‚श्य‚भाव्य‚म् । ज‚न‚क‚श्चेन्द्रिय‚ज्ञान‚स‚ह‚भूः क्ष‚ण‚स्त‚स्येति । सोऽस्येन्द्रिय‚ज्ञान‚{\tiny $_{lb}$}‚विष‚यात्प्राक्त‚नार्थ‚क्ष‚णाद‚न्यो विष‚य इति । अनेन च स्व‚विष‚यान‚न्त‚र‚विष‚य‚क्ष‚ण‚स‚ह‚कारिग्र‚ह‚ण‚स्यो‚{\tiny $_{lb}$}‚प‚योगो द‚र्शितः ।
	\pend% ending standard par
      ‚{\tiny $_{lb}$}‚

	  \pstart \leavevmode% starting standard par
	अथेत्थं गृहीत‚ग्राहित्वाद‚प्रामाण्य‚प्र‚स‚ङ्गो निराक्रिय‚ताम् । अन्ध‚ब‚धिराद्य‚भाव‚दोष‚प्र‚स‚ङ्ग‚स्तु‚{\tiny $_{lb}$}‚ क‚थ‚ङ्कारं निराक्रियेतेत्याश‚ङ्क्याह--\textbf{इन्द्रिय‚ज्ञानेने}ति । अमुनाऽ\textbf{गृहीत‚स्य विष‚यान्त‚र‚स्य‚{\tiny $_{lb}$}‚ ग्र‚ह‚णा}न्मान‚सेनेति प्र‚क‚र‚णात् । त‚स्माद् हेतोरिन्द्रिय‚ज्ञानेनागृहीतं मान‚सेन गृह्य‚त इति द‚र्श‚य‚ता‚{\tiny $_{lb}$}‚ चानेन य‚स्यैवेन्द्रिय‚म‚स्ति त‚स्यैव तेनेन्द्रिय‚ज्ञानेन स‚ह‚भूविष‚य‚क्ष‚ण‚स‚ह‚कारिणा ज‚नित‚त्वान्मान‚सं‚{\tiny $_{lb}$}‚ प्र‚त्य‚क्ष‚म् । तेनागृहीतं\add{त}त्स‚ह‚भुवं विष‚य‚क्ष‚णं गृह्णाति । य‚स्य पुनः पुंसोऽर्थ‚विष‚य‚मिन्द्रिय‚ज्ञान‚मेव‚{\tiny $_{lb}$}‚ नास्ति त‚स्य क‚थं मान‚स‚मुत्प‚द्येत, गृह्णीयाद् वा तं विष‚य‚क्ष‚ण‚मिति द‚र्शित‚म् । अतोऽन्ध‚ब‚धिरा‚{\tiny $_{lb}$}‚द्य‚भाव‚दोष‚प्र‚स‚ङ्गो निर‚स्त इति । अनेनेन्द्रिय‚ज्ञान‚ग्र‚ह‚ण‚स्य स्व‚विष‚यान‚न्त‚र‚विष‚य‚क्ष‚ण‚स‚ह‚कारि‚{\tiny $_{lb}$}‚स‚चिव‚स्योप‚योगो द‚र्शितः । केव‚लेऽपि ज्ञानेनेति ग्र‚ह‚णे\add{स्व}\leavevmode\ledsidenote{\textenglish{27a/ms}}विष‚यान‚न्त‚र‚विष‚य‚क्ष‚ण‚{\tiny $_{lb}$}‚स‚ह‚कारिणेतिविशेष‚ण‚व‚लाद् य‚दि नामेन्द्रिय‚ज्ञानेनेति ल‚भ्य‚ते त‚थापि प्र‚तिप‚त्तिगौर‚व‚प‚रिहारार्थ‚{\tiny $_{lb}$}‚मिन्द्रिय‚ग्र‚ह‚णं कृत‚मित्य‚व‚सेय‚म् । \textbf{अन्ध‚ब‚धिरादी}त्य‚त्रादिग्र‚ह‚णात्काप्ति\edtext{}{\lemma{णात्काप्ति}\Bfootnote{कुष्ठि}}रोगिणो ग्र‚ह‚ण‚म्‚{\tiny $_{lb}$}‚ त‚स्यापि मान‚सेन स्प‚र्श‚ज्ञानात् ।
	\pend% ending standard par
      ‚{\tiny $_{lb}$}‚

	  \pstart \leavevmode% starting standard par
	इह \textbf{शान्त‚भ‚द्रेण सौत्रान्तिकानां} म‚तं द‚र्श‚य‚ता पूर्वं च‚क्षूरूपे च‚क्ष‚र्विज्ञानं त‚त‚स्तेनेन्द्रिय‚{\tiny $_{lb}$}‚विज्ञानेन स‚ह‚ज‚क्ष‚ण‚स‚ह‚कारिणा तृतीय‚स्मिन् क्ष‚णे मान‚स‚प्र‚त्य‚क्षं ज‚न्य‚ते इति व्याख्यात‚म् ।‚{\tiny $_{lb}$}‚ ‚{\tiny $_{lb}$}‚ \leavevmode\ledsidenote{\textenglish{62/dm}}‚{\tiny $_{lb}$}‚ 
	  
	एत‚च्च म‚नोविज्ञान‚मुप‚र‚त‚व्यापारे च‚क्षुषि प्र‚त्य‚क्ष‚मिव्य‚ते । व्यापार‚व‚ति तु च‚क्षुषि‚{\tiny $_{lb}$}‚ य‚द्रूप‚ज्ञानं त‚त्स‚र्वं च‚क्षुराश्रित‚मेव । इत‚र‚था च‚क्षुराश्रित‚त्वानुप‚प‚त्तिः क‚स्य‚चिद‚पि\edtext{}{\lemma{पि}\Bfootnote{०पि ज्ञान‚स्य \cite{dp-msB} \cite{dp-msD}}} विज्ञान‚स्य ।‚{\tiny $_{lb}$}‚ त‚द‚व‚म‚न्य‚मान आह--\textbf{एत‚च्चेति} । चोऽव‚धार‚णे । \textbf{उप‚र‚त‚व्यापार} इत्य‚स्यान‚न्त‚रं द्र‚ष्ट‚व्यः ।‚{\tiny $_{lb}$}‚ च‚क्षुषो व्याप्रिय‚माणाव‚स्था प्र‚णिधान‚मेव । एत‚न्म‚नोविज्ञानं प्र‚माणं प्र‚माण‚त्वेनोप‚ग‚त‚मुप‚र‚त‚व्यापार‚{\tiny $_{lb}$}‚ एव च‚क्षुषि भ‚व‚तीतीष्य‚त इति स‚मुदायार्थो द्र‚ष्ट‚व्यः । य‚दि प्र‚णिहिते न‚य‚ने म‚नोज्ञानं न ज‚न्य‚ते‚{\tiny $_{lb}$}‚ त‚र्हि किं नाम जाय‚त इत्याह--\textbf{व्यापारे}ति । तुर्निर्व्यापाराव‚स्थाया व्यापार‚व‚तीम‚व‚स्थां भिन‚त्ति ।‚{\tiny $_{lb}$}‚ \textbf{रूप‚ज्ञान‚मि}ति वास्त‚वानुवादो न पुन‚र‚न्य‚ज्ञान‚स‚म्भ‚वेन त‚द्व्य‚व‚च्छेदार्थ‚म् । य‚द्वा \textbf{रूप}श‚ब्दः स्व‚{\tiny $_{lb}$}‚भाव‚व‚च‚न‚स्तेन य‚त्स्व‚भाव‚ज्ञान‚मित्य‚र्थः । \textbf{स‚र्व}ग्र‚ह‚णेन व्याप्तिमाह । \textbf{च‚क्षुराश्रि}तं च‚क्षुर्ज‚न्यं‚{\tiny $_{lb}$}‚ च‚क्ष‚र्विज्ञान‚मिति याव‚त् । उप‚प‚त्तिमाह--\textbf{इत‚र‚थे}ति ।
	\pend% ending standard par
      ‚{\tiny $_{lb}$}‚

	  \pstart \leavevmode% starting standard par
	न‚नु व्यापार‚व‚त्य‚पि च‚क्षुषि मान‚सोत्प‚त्तौ न च‚क्षुराश्रित‚त्वानुप‚प‚त्तिरादिम‚स्यैव ज्ञान‚स्य‚{\tiny $_{lb}$}‚ च‚क्षुराश्रित‚त्वोप‚प‚त्तेः । स‚त्य‚म् । केव‚ल‚म\add{य‚म}भिप्रायः--य‚द‚पि त‚दाद्यं च‚क्षुर्विज्ञानं त‚द‚पि‚{\tiny $_{lb}$}‚ च‚क्षुर‚न्व‚य‚व्य‚तिरेकेऽ\edtext{}{\lemma{तिरेकेऽ}\Bfootnote{का}}नुविधानादेव त‚दाश्रितं व्य‚व‚स्थाप‚नीय‚म् । स‚त्य‚पि त‚द‚न्व‚य‚व्य‚ति‚{\tiny $_{lb}$}‚रेकानुविधाने य‚दि क‚स्य‚चिन्मान‚स‚त्वं त‚दा च‚क्षुर‚न्व‚य‚व्य‚तिरेकौ च‚क्षुराश्रित‚त्व‚व्य‚व‚स्थाया अन‚ङ्ग‚{\tiny $_{lb}$}‚मित्याद्येऽपि त‚द्व्य‚व‚स्था न स्यादिति साधीयान् प्र‚स‚ङ्गः ।
	\pend% ending standard par
      ‚{\tiny $_{lb}$}‚

	  \pstart \leavevmode% starting standard par
	स्यादेत‚त्--किमेत‚न्मान‚सं च‚क्षुर्विज्ञानेनैव रूप‚विष‚ये ज‚न्य‚ते किं वा र‚स‚नादिज्ञानेनापि‚{\tiny $_{lb}$}‚ र‚सादिविष‚य इति ? किं चातः ? य‚दि रूप एव त‚दा \edtext{}{\lemma{दा}\Bfootnote{द्र‚ष्ट‚व्योऽभिध‚र्म‚कोषः--प‚ञ्च बाह्या द्विविज्ञेयाः--\href{http://sarit.indology.info/?cref=ak.1.48}{ १. ४८}.}}द्विविध‚ज्ञान‚विज्ञेयाः प‚ञ्च बाह्य‚विष‚याः‚{\tiny $_{lb}$}‚ इत्याग‚मो विरुद्ध्येत । अथ स‚म‚स्त‚रूपादिविष‚य‚मिष्य‚ते त‚दा श‚ब्द‚विष‚य‚त्वेन त‚स्यास‚म्भ‚वः ।‚{\tiny $_{lb}$}‚ न‚हि र‚सादिरिव ताल्वादिज‚न्मा श‚ब्दः प्र‚वाह‚वाही, येन श्रोत्रेन्द्रिय‚ज्ञानेन स्वाविष‚या‚{\tiny $_{lb}$}‚न‚न्त‚र‚विष‚य‚क्ष‚ण‚स‚ह‚कारिणा त‚त्र मान‚सं ज‚न्येतेति । उच्य‚ते--रूपादिप‚ञ्च‚क‚विष‚य‚मेत‚त्, न तु‚{\tiny $_{lb}$}‚ रूप‚मात्र‚विष‚य‚म् । त‚दाहाचार्यः--इन्द्रिय‚ज्ञानेन, न तु च‚क्षुर्विज्ञानेनेति । य‚त्पुन‚र‚वादीद् भ‚ग‚वान्—‚{\tiny $_{lb}$}‚द्वाभ्यां भिक्ष‚वो रूपं गृह्य‚ते, च‚क्षुर्विज्ञानेन त‚दाकृष्टेन च म‚न‚सेति त‚द्रूप‚स्य ग्र‚ह‚ण‚प्र‚स‚ङ्गेनो‚{\tiny $_{lb}$}‚क्त‚म्, न तु रूप‚मेव द्विधा गृह्य‚त इति विव‚क्षित‚म् ।
	\pend% ending standard par
      ‚{\tiny $_{lb}$}‚

	  \pstart \leavevmode% starting standard par
	न‚नूक्तं न त‚स्य श‚ब्द‚विष‚य‚त्वेन स‚म्भ‚व‚स्त‚त्क‚थ‚मेव‚म‚भ्युप‚ग‚मः ? स‚त्य‚मेत‚त् । केव‚लं‚{\tiny $_{lb}$}‚ य‚दाऽभिह‚न्य‚मानं घ‚ण्टाटि र‚ण‚ति त‚दा त‚दुक्तः\edtext{}{\lemma{दुक्तः}\Bfootnote{त्थः}}श‚ब्दः किय‚न्तं कालं स‚न्तानेन प्र‚व‚ह‚तीति‚{\tiny $_{lb}$}‚ श्रोत्र‚ज्ञानेन स्व‚विष‚यान‚न्त‚र‚विष‚य‚क्ष‚ण‚स‚ह‚कारिणा त‚द्विष‚यं मान‚सं ज‚न्य‚ते । एताव‚ताऽपि‚{\tiny $_{lb}$}‚ द्विविध‚ज्ञान‚वेद्य‚त्व‚व‚च‚नं च‚रितार्थ‚म् । त‚दुक्त‚म‚भिध‚र्म‚कोशे--रूपादिजात्य‚भिस‚म्ब‚न्धिव‚च‚नाद् ‚{\tiny $_{lb}$}‚ इति । अस्यार्थः--रूपादिजात्य‚भिस‚न्धिना त‚दुक्तं भ‚ग‚व‚ता, न तु \leavevmode\ledsidenote{\textenglish{27b/ms}} स‚र्व‚रूपादिव्य‚क्त्य‚{\tiny $_{lb}$}‚भिप्रायेणेति ।
	\pend% ending standard par
      ‚{\tiny $_{lb}$}‚

	  \pstart \leavevmode% starting standard par
	इह \textbf{पूर्वैः}--बाह्यार्थाल‚म्ब‚न‚मेवंविधं म‚नोविज्ञान‚म‚स्तीति कुतोऽव‚सेय‚म् इत्याश‚ङ्क्य‚{\tiny $_{lb}$}‚ त‚द‚भावे त‚द्ब‚लोत्प‚न्नानां विक‚ल्पानाम‚भावाद् रूपादौ विष‚ये व्य‚व‚हाराभाव‚प्र‚स‚ङ्गः स्याद् ‚{\tiny $_{lb}$}‚ इत्युक्त‚म् । च‚क्षुरादिविज्ञानेनानुभूत‚त्वान्न विक‚ल्पाभाव इति चाश‚ङ्क्याभिहित‚म्--देव‚द‚त्तेनापि‚{\tiny $_{lb}$}‚ ‚{\tiny $_{lb}$}‚ \leavevmode\ledsidenote{\textenglish{63/dm}}‚{\tiny $_{lb}$}‚ 
	  
	एत‚च्च सिद्धान्त‚प्र‚सिद्धं मान‚सं प्र‚त्य‚क्ष‚म् । न त्व‚स्य प्र‚साध‚क‚म‚स्ति प्र‚माण‚म् । एवं‚{\tiny $_{lb}$}‚जातीय‚कं त‚द् य‚दि स्यात् न क‚श्चिद् दोषः स्यादिति व‚क्तुं ल‚क्ष‚ण‚माख्यात‚म‚स्येति ॥‚{\tiny $_{lb}$}‚ दृष्टे य‚ज्ञ‚द‚त्त‚स्यापि विक‚ल्प‚प्र‚स‚ङ्गः । स‚न्तान‚भेदा\add{न्न}भ‚विष्य‚ति इति च पुन‚राश‚ङ्क्य अत्रापि‚{\tiny $_{lb}$}‚ स‚न्तान‚भेदादेव विक‚ल्पो न प्राप्नोति, य‚त इहापीन्द्रियाश्र‚य‚भेदादेव स‚न्तान‚भेदो युग‚प‚त्प्र‚वृत्तेश्च ।‚{\tiny $_{lb}$}‚ दीर्ध‚श‚ष्कुलीभ‚क्ष‚णादौ हि युग‚प‚ज्ज्ञान‚प्र‚वृत्तिर्दृश्य‚ते । न च स‚न्तान‚स्यैक्ये युग‚प‚त्प्र‚वृत्ति‚{\tiny $_{lb}$}‚र्न्याय्या । त‚स्माद् रूपादिविक‚ल्पाभावो मा भूदित्य‚व‚श्य‚म‚विक‚ल्प‚कं म‚नोविज्ञान‚म‚भ्युपेय‚म् ।‚{\tiny $_{lb}$}‚ एतेन निश्च\add{य}स्म‚र‚णाभाव‚प्र‚स‚ङ्गोऽपि ढौक‚नीयः । निर्विक‚ल्प‚कं म‚नोविज्ञानं य‚दि नास्त्येव त‚दा‚{\tiny $_{lb}$}‚ योगिज्ञानाभाव‚प्र‚स‚ङ्गः । अस्त्येव निर्विक‚ल्प‚कं म‚नोविज्ञानं किन्त्विन्द्रिय‚ज्ञान‚पृष्ठ‚भावि नास्तीति‚{\tiny $_{lb}$}‚ चेत् । स‚ति स‚म्भ‚वे त‚स्याप्य‚स्तित्वे को विरोधः । न ह्य‚त्र बाध‚कं प्र‚माणं दृश्य‚ते, येन त‚न्नास्तीति‚{\tiny $_{lb}$}‚ स्यात् । अस्तित्वे चोक्तं प्र‚माण‚म् । त\add{स्मा}द‚स्तीन्द्रिय‚ज्ञान‚पृष्ठ‚भावि म‚नोज्ञानं निर्विक‚ल्प‚क‚म् ‚{\tiny $_{lb}$}‚ इत्येव‚माद्य‚भिहित‚म् । त‚देत‚त्त‚दीयं क‚द‚लीग‚र्भ‚निःसारं म‚न्य‚मानः प्राह--\textbf{एत‚च्चेति । चोऽ}व‚धार‚णे ।‚{\tiny $_{lb}$}‚ \textbf{सिद्धान्त‚प्र‚सिद्ध‚मि}त्य‚तः प‚रो द्र‚ष्ट‚व्यः । एत‚देव व्य‚तिरेक‚मुखेण द्र‚ढ‚य‚ति \textbf{न त्व‚स्ये}ति । \textbf{तुर}व‚धार‚य‚ति‚{\tiny $_{lb}$}‚ भिन‚त्ति वा । प्र‚वाहानार‚म्भ‚क‚स्यास्य\add{ज्ञा}नात्म‚त‚या स्व‚संवेद‚न‚रूप‚त्वेऽप्य‚संविदित‚क‚ल्प‚त्वाद‚नुभ‚व‚{\tiny $_{lb}$}‚ग‚म्य\add{मिदं}य‚था च‚क्षुरादिज्ञान‚मिति द‚र्श‚यितुम‚श‚क्य‚त्वा\textbf{न्नास्य प्र‚साध‚कं} निश्चाय‚कं प्र‚माण‚म‚स्ति ।
	\pend% ending standard par
      ‚{\tiny $_{lb}$}‚

	  \pstart \leavevmode% starting standard par
	न‚नु तैर्द‚र्शित‚मेव विक‚ल्पाभाव‚प्र‚स‚ङ्गः स्यादित्यादिना प्र‚स‚ङ्ग‚मुखेन प्र‚माणं विक‚ल्पोद‚या‚{\tiny $_{lb}$}‚दिति त‚त्किमुच्य‚ते \textbf{न त्व‚स्य प्र‚साध‚क‚म‚स्ति प्र\add{मा}ण‚मि}ति । अय‚म‚स्याश‚यः--स‚त्य‚मुक्त‚म‚तार्किकीयं‚{\tiny $_{lb}$}‚ तु त‚त् । त‚थाहि य‚त् ताव‚दुक्त‚मिन्द्रियाश्र‚य‚भेदाद् युग‚प‚त्प्र‚वृत्तेश्च स‚न्तान‚भेदोऽस्ति । न च‚{\tiny $_{lb}$}‚ स‚न्तान‚भेदेऽन्यानुभूतेऽन्य‚विक‚ल्पो युक्त इति । त‚द‚वाच्य‚म् । य‚दीन्द्रियाश्र‚य‚भेदाद् युग‚प‚त्प्र‚वृत्तेश्च‚{\tiny $_{lb}$}‚ स‚न्तान‚भेद‚स्त‚द्भेदे च न कार्य‚कार‚ण‚भावः, त‚दा स्वापादुत्थित‚मात्र‚स्य पुंस‚श्च‚क्षुर्विज्ञानं‚{\tiny $_{lb}$}‚ \edtext{\textsuperscript{*}}{\lemma{*}\Bfootnote{प‚ङ्क्तिबाह्यं लिखितः पाठो न प‚ठ्य‚तेऽस्प‚ष्ट‚त्वात्--सं०}}\add{... ... ...}नोत्प‚द्येत । त‚त् ख‚लूप‚जाय‚मानं म‚नोज्ञानाद् वा प्राचीना‚{\tiny $_{lb}$}‚दुत्प‚द्येत, इन्द्रिय‚ज्ञानाद् वा । न ताव‚दिन्द्रिय‚ज्ञानात् त‚स्य पूर्व‚म‚भावात्, नापि जाग‚राव‚स्थाभावी‚{\tiny $_{lb}$}‚न्द्रिय‚ज्ञानात् । त‚स्य चिर‚निरुद्ध‚त्वात् । न च चिरातीतं कार‚ण‚मिष्य‚ते । अथैवं भिन्न‚स‚न्तानै‚{\tiny $_{lb}$}‚रेव विक‚ल्पितैर्विक‚ल्पोद‚योऽस्त्येव न तु निर्विक‚ल्पाद् वि\add{क}ल्पोद‚य इति चेत् । स‚न्तान‚भेदेऽपि‚{\tiny $_{lb}$}‚ ज‚न्य‚ज‚न‚क‚भावे निर्विक‚ल्पाद‚पि विक‚ल्पोद‚य‚स्य को निषेद्धा । न चेन्द्रियाश्र‚य‚भेदाद् युग‚प‚त् प्र‚वृत्तेश्च‚{\tiny $_{lb}$}‚ स‚न्तान‚भेदो युज्य‚ते । प‚र‚स्प‚र‚प‚रोक्ष‚तादिप्र‚स‚ङ्गात् । त‚स्मात्प्र‚भूत‚म‚ल्पं वा स‚दृशादेव कार‚णा‚{\tiny $_{lb}$}‚ज्जाय‚ते । न तु प्र‚भूत‚स्योद‚य‚मात्रेण स‚न्तान‚भेदोऽभ्युपेत‚व्यः । य‚थाऽग्निक‚णिकायाः \leavevmode\ledsidenote{\textenglish{28a/ms}}प्र‚भा‚{\tiny $_{lb}$}‚प्र‚तान‚व‚ती प्र‚दीप‚शिखा जाय‚माना न स‚न्तान‚भेद‚मात्म‚न्याव‚ह‚ति । य‚त्पुन‚र्योगिज्ञान‚स्य त‚थात्वे-,‚{\tiny $_{lb}$}‚ ऽपीदानीं म‚नोज्ञानं निर्विक‚ल्प‚कं नाभ्युपेय‚ते त‚द् यादृश्याः साम‚ग्र्यास्त‚दुद्भ‚वो य‚दाकार‚ञ्च त‚त्‚{\tiny $_{lb}$}‚ त‚त्साम‚ग्र्य‚भावात्त‚दाकार‚स्य च म‚नोविज्ञान‚स्य निर्विक‚ल्प‚स्यासंवेद‚नादिति किम‚त्राद‚रेणेति ।
	\pend% ending standard par
      ‚{\tiny $_{lb}$}‚

	  \pstart \leavevmode% starting standard par
	न‚नु च य‚द्य‚स्य प्र‚साध‚कं प्र‚माणं नास्ति किम‚र्थं त‚र्हि प्र‚त्य‚क्ष‚प्र‚क‚र‚ण उप‚न्यास इत्याश‚ङ्क्याह—‚{\tiny $_{lb}$}‚\textbf{एव‚मि}ति । \textbf{एवंजाती}य‚क‚मेव‚म्प्र‚कार‚व‚त् । एव‚म्प्र‚कार‚ल‚क्ष‚ण‚क‚थ‚न‚स्यैव किम्प्र‚योज‚न‚मिति‚{\tiny $_{lb}$}‚ चेत् । \textbf{सूत्र‚कार‚स्य} सिद्धान्त‚प्र‚सिद्ध‚मान‚साभ्युप‚ग‚मे प्र‚स‚क्त‚चोद्य‚निराक‚र‚ण‚म्--य‚द्य‚स्ति मान‚सं‚{\tiny $_{lb}$}‚ प्र‚त्य‚क्ष‚मेवं त‚स्य ल‚क्ष‚ण‚मिति ॥
	\pend% ending standard par
      ‚{\tiny $_{lb}$}‚\textsuperscript{\textenglish{64/dm}}‚{\tiny $_{lb}$}‚
	  \bigskip
	  \begingroup
	

	  \pstart \leavevmode% starting standard par
	स्व‚संवेद‚न‚माख्यातुमाह--
	\pend% ending standard par
       ‚{\tiny $_{lb}$}‚ 
	  \bigskip
	  \begingroup
	

	  \pstart \leavevmode% starting standard par
	स‚र्व‚चित्त‚चैत्तानामात्म‚संवेद‚न‚म् ॥ १० ॥
	\pend% ending standard par
      
	  \endgroup
	‚{\tiny $_{lb}$}‚ 

	  \pstart \leavevmode% starting standard par
	\edtext{\textsuperscript{*}}{\lemma{*}\Bfootnote{स‚र्व‚चित्तेत्यादि नास्ति \cite{dp-msA} \cite{dp-msC}}}स‚र्व‚चित्तेत्यादि । चित्त‚म् अर्थ‚मात्र\edtext{}{\lemma{मात्र}\Bfootnote{मात्राव‚ग्राहि \cite{dp-msC}}}ग्राहि । चैत्ता विशेषाव‚स्थाग्राहिणः सुखाद‚यः ।‚{\tiny $_{lb}$}‚ स‚र्वे च ते चित्त‚चैत्ताश्च स‚र्व‚चित्त‚चैत्ताः । सुखाद‚य एव स्फुटानुभ‚व‚त्वात् स्व‚संविदिताः, नान्या‚{\tiny $_{lb}$}‚ चित्ताव‚स्थेत्येत‚दाश‚ङ्कानिवृत्त्य‚र्थं स‚र्व‚ग्र‚ह‚णं कृत‚म् । नास्ति सा काचित् चित्ताव‚स्था य‚स्यामा‚{\tiny $_{lb}$}‚त्म‚नः\edtext{}{\lemma{नः}\Bfootnote{०मात्म‚संवे० \cite{dp-msD}}} संवेद‚नं न प्र‚त्य‚क्षं स्यात् ।
	\pend% ending standard par
       ‚{\tiny $_{lb}$}‚ 

	  \pstart \leavevmode% starting standard par
	येन हि रूपेणात्मा वेद्य‚ते त‚द्रूप‚मात्म‚संवेद‚नं प्र‚त्य‚क्ष‚म् ।
	\pend% ending standard par
      
	  \endgroup
	‚{\tiny $_{lb}$}‚

	  \pstart \leavevmode% starting standard par
	\textbf{वैभाषिक‚प्र}क्रिय‚या य‚दाचार्येण चित्त‚चैत्तौ भेदेनोक्तौ त‚योर‚र्थ‚माह--\textbf{चित्त‚म‚र्थ‚मात्र‚ग्राहि}‚{\tiny $_{lb}$}‚ व‚स्तुमात्र‚ग्राहि । \textbf{त‚त्रार्थ‚दृष्टिर्विज्ञान‚म् इति} व‚च‚नात् । \textbf{चैत्ता} विशेषाव‚स्थाग्राहिणो विशेषा‚{\tiny $_{lb}$}‚व‚स्थास्वीक‚र्त्तारो विशेषाव‚स्थाकारा इति याव‚त् । त‚द्विशेषे तु चैत‚सा इति व‚च‚नात् । क एवं‚{\tiny $_{lb}$}‚रूपा इत्याह--\textbf{सुखाद‚य} इति । \textbf{स‚र्वे चे}ति विगृह्ण‚न् पूर्वं चित्तानि च चैत्ताश्चेति स‚म‚स्य‚{\tiny $_{lb}$}‚ प‚श्चात्स‚र्व‚श‚ब्देन स‚मास इति द‚र्श‚य‚ति । \textbf{सुखे}त्यादिना \textbf{स‚र्व}ग्र‚ह‚ण‚स्य फ‚ल‚माह--\textbf{स्फुटो} व्य‚क्तो‚{\tiny $_{lb}$}‚\textbf{ऽनुभ‚वः} प्र‚काशो य‚स्य त‚स्य भाव‚स्त‚स्मात् । \textbf{स‚र्व}ग्र‚ह‚णे स‚ति कीदृशोऽर्थो भ‚व‚ति, येनाश‚ङ्का‚{\tiny $_{lb}$}‚ निव‚र्त्त्य‚त इत्याह--\textbf{नास्तीति । काचिदिति} भ्रान्ता वाऽभ्रान्ता वेत्य‚र्थः ।
	\pend% ending standard par
      ‚{\tiny $_{lb}$}‚

	  \pstart \leavevmode% starting standard par
	न‚नु ज्ञान‚स्य संवेद्यात्म‚नः किम‚न्य‚द् रूपान्त‚रं य‚त्स्व‚संवेद‚नं प्र‚त्य‚क्षं स्यादित्याश‚ङ्क्याह—‚{\tiny $_{lb}$}‚\textbf{येन ही}ति । हिर्य‚स्माद‚र्थे ।
	\pend% ending standard par
      ‚{\tiny $_{lb}$}‚

	  \pstart \leavevmode% starting standard par
	अथ य‚द‚नुभूय‚ते त‚द‚न्येनैव य‚था घ‚टादि । त‚त्क‚थ‚मात्म‚नैवात्म‚नः संवेद‚न‚मिति चेत् ।‚{\tiny $_{lb}$}‚ न । योग्य‚तायास्त‚थाव्य‚व‚हारात् । अस्ति ज्ञान‚स्य सा योग्य‚ता ज‚ड‚व्यावृत्त‚ता ज्योतीरूप‚ता य‚या‚{\tiny $_{lb}$}‚ स्व‚प्र‚काशे प्र‚काशान्त‚रं नापेक्ष‚ते । य‚था प्र‚दीपः प्र‚काश‚स्वाभाव्यादात्मानं स्व‚य‚मेव प्र‚काश‚य‚ति,‚{\tiny $_{lb}$}‚ न तु प्र‚दीपान्त‚र‚म‚पेक्ष‚त इति ।
	\pend% ending standard par
      ‚{\tiny $_{lb}$}‚

	  \pstart \leavevmode% starting standard par
	न‚नु किं घ‚टादिदृष्टान्त‚ब‚लात्प्र‚काशान्त‚र‚प्र‚काश्य‚ता ज्ञान‚स्यास्ताम्, आहोस्वित् प्र‚दीप‚{\tiny $_{lb}$}‚दृष्टान्त‚साम‚र्थ्यात् प्र‚काशान्त‚र‚निर‚पेक्ष‚त‚या स्व‚प्र‚काश‚ता ? अत्रोच्य‚ते । ज्ञान‚स्य स्व‚प्र‚काश‚रूप‚त्वा‚{\tiny $_{lb}$}‚भावे घ‚टादेः स्व‚रूप‚प्र‚काश‚नानुप‚प‚त्तेः स्व‚प्र‚काश‚तैवेष्ट‚व्या । अत्र \textbf{च} प्र‚योगः--य‚द‚व्य‚क्त‚व्य‚क्तिकं‚{\tiny $_{lb}$}‚ न त‚द् व्य‚क्त‚म् य‚था किञ्चित्क‚दाचित् क‚थ‚ञ्चिद‚व्य‚क्त‚व्य‚क्तिक‚म् । अव्य‚क्त‚व्य‚क्तिक‚श्चायं‚{\tiny $_{lb}$}‚ घ‚टादिः ज्ञान‚प‚रोक्ष‚त्व इति व्याप‚कानुप‚ल‚ब्धिप्र‚स‚ङ्गः । ज्ञान‚स्य ज्ञानान्त‚रेण व्य‚क्तौ हेतुर‚य‚म‚सिद्ध‚{\tiny $_{lb}$}‚ इति चेन्न, नील‚ज्ञानोद‚य‚कालेऽसिद्ध‚त्वाद् हेतोर्नील‚स्य प‚रोक्ष‚त्व‚प्र‚स‚ञ्ज‚नात् । न च भ‚व‚ताम‚पि‚{\tiny $_{lb}$}‚ म‚तै स‚र्वं विज्ञान‚मेकार्थ‚स‚म‚वायिना ज्ञानेन ज्ञाय‚ते । बुभुत्साऽभावे त‚द‚भावात्, य‚थोपेक्ष‚णीय‚{\tiny $_{lb}$}‚विष‚या संवित् । त‚त उपेक्ष‚णीय‚मेव ताव‚द\add{व्य‚क्त}व्य‚क्तिक‚त्वाद‚व्य‚क्तं प्र‚स‚ज्येत । न चेय‚म‚{\tiny $_{lb}$}‚‚{\tiny $_{lb}$}‚ ‚{\tiny $_{lb}$}‚ \leavevmode\ledsidenote{\textenglish{65/dm}}‚{\tiny $_{lb}$}‚ 
	  
	\edtext{\textsuperscript{*}}{\lemma{*}\Bfootnote{मीमांस‚कान् प्र‚ति विज्ञानं स्व‚संवेद‚न‚प्र‚त्य‚क्ष‚मुक्त‚म् । य‚श्च सांख्योऽपि बाह्य‚रूपाः‚{\tiny $_{lb}$}‚ सुखाद‚यः इति म‚न्य‚ते तं प्र‚त्याह--\cite{dp-msD-n} । इह रूपादौ \cite{dp-msD}}}इह च रूपादौ व‚स्तुनि दृश्य‚माने\edtext{}{\lemma{माने}\Bfootnote{०मानेऽन्त‚रः \cite{dp-msA} \cite{dp-msB} \cite{dp-msC} \cite{dp-msD} \cite{dp-edP} \cite{dp-edH} \cite{dp-edE}}}आन्त‚रः सुखाद्याकार‚स्तुल्य‚कालं \edtext{}{\lemma{कालं}\Bfootnote{०कालं वेद्य‚ते \cite{dp-msC}}}संवेद्य‚ते । न च—‚{\tiny $_{lb}$}‚गृह्य‚माणाकारो नीलादिः \edtext{}{\lemma{नीलादिः}\Bfootnote{सातादिरू० \cite{dp-edP} \cite{dp-edH} \cite{dp-edE} \cite{dp-edN}}}सात‚रूपो वेद्य‚ते इति \edtext{}{\lemma{इति}\Bfootnote{इति व‚क्तुं श‚क्य‚म्--\cite{dp-msA} \cite{dp-edP} \cite{dp-edH} \cite{dp-edE}}}श‚क्यं व‚क्तुम् । य‚तो नीलादिः \edtext{}{\lemma{नीलादिः}\Bfootnote{सातादिरूपेण \cite{dp-msA} \cite{dp-msC} \cite{dp-msD} \cite{dp-edP} \cite{dp-edH} \cite{dp-edE} \cite{dp-edN} सातानुरूपेण--\cite{dp-msB}}}सात‚{\tiny $_{lb}$}‚रूपेणानुभूय‚त इति न निश्चीय‚ते । ‚{\tiny $_{lb}$}‚ 
	  
	य‚दि हि सात‚रूपोऽयं\edtext{}{\lemma{रूपोऽयं}\Bfootnote{सातादिरूपोऽयं \cite{dp-msA} \cite{dp-msC} \cite{dp-edP} \cite{dp-edE} \cite{dp-edN}}} नीलादिर‚नुभूय‚ते इति निश्चीयेत, स्यात्\edtext{}{\lemma{स्यात्}\Bfootnote{स्यात् त‚स्य \cite{dp-msA} \cite{dp-msC} \cite{dp-edP} \cite{dp-edE}}} त‚दा त‚स्य‚{\tiny $_{lb}$}‚ सातादिरूप‚त्व‚म् । य‚स्मिन् रूपे प्र‚त्य‚क्ष‚स्य साक्षात्कारित्व‚व्यापारो विक‚ल्पेनानुग‚म्य‚ते त‚त्‚{\tiny $_{lb}$}‚ प्र‚त्य‚क्ष‚म् ।‚{\tiny $_{lb}$}‚ नुप‚ल‚ब्धिः स‚न्दिग्ध‚विप‚क्ष‚व्यावृत्त्याऽनैकान्तिकी कीर्त्त‚नीया । \leavevmode\ledsidenote{\textenglish{28b/ms}} त‚था हि य‚द्य‚व्य‚क्त‚{\tiny $_{lb}$}‚व्य‚क्तिक‚म‚पि व्य‚क्त‚व्य‚व‚हार‚विष‚य‚स्त‚दा पुरुषान्त‚र्व‚र्त्तिज्ञान‚व्य‚वित‚क‚म‚पि व‚स्तु स्व‚ज्ञानोद‚य‚काल‚व‚त्‚{\tiny $_{lb}$}‚ त‚थैव व्य‚क्तं व्य‚व‚ह्रियेत । अस्व‚संवेद‚नात्म‚त‚या स्व‚प‚र‚स‚न्तान‚व‚र्त्तिनोर्ज्ञान‚योर्विशेषाभावात् ।‚{\tiny $_{lb}$}‚ त‚त्स‚न्ताने ज्ञान‚स्याभावात् क‚थं व‚स्तुन‚स्त‚था व्य‚व‚हार इति चेत् । भावेऽपि त‚द‚प्र‚काशे क‚थं‚{\tiny $_{lb}$}‚ त‚था व्य‚व‚हारः ? न हि तेनास्य किञ्चित् क्रिय‚ते । \edtext{\textsuperscript{*}}{\lemma{*}\Bfootnote{प‚ङ्क्तिबाह्यं लिखितं स‚म्य‚क् न प‚ठ्य‚ते--सं०}}\add{... ...}त‚दा त‚द‚पि पुरुषान्त‚र‚स्य‚{\tiny $_{lb}$}‚ किन्न त‚था व्य‚व‚हार‚गोच‚रः ? त‚द‚यं व्य‚क्त‚व्य‚व‚हारो व्य‚क्त‚व्य‚क्तिक‚त्वेन व्याप्तः । सिद्धे‚{\tiny $_{lb}$}‚ च व्याप्य‚व्याप‚क‚भावे व्याप‚कानुप‚ल‚ब्धिर्नानैकान्तिकीति ।
	\pend% ending standard par
      ‚{\tiny $_{lb}$}‚

	  \pstart \leavevmode% starting standard par
	न‚नु किं त‚द्रूपान्त‚र‚म्, येनोच्य‚ते--येनात्मा संवेद्य‚ते, त‚दात्म‚संवेद‚नं प्र‚त्य‚क्ष‚मिति ?‚{\tiny $_{lb}$}‚ केव‚ल‚म‚र्थ‚शून्य‚मेत‚दुच्य‚ते इत्याश‚ङ्क्याह--इहेत्यादि । अन्त‚रे भ‚व \textbf{आन्त‚रो}ऽध्यात्म‚प‚रिस्प‚न्दी ।‚{\tiny $_{lb}$}‚ कोऽसावीदृश इत्याह--\textbf{सुखे}ति ।
	\pend% ending standard par
      ‚{\tiny $_{lb}$}‚

	  \pstart \leavevmode% starting standard par
	न‚नु गृह्य‚माण एव रूपादिः सुखाद्याकारोऽनुभूय‚ते । न तु त‚तोऽन्य‚त्सुखादिरूपं येन‚{\tiny $_{lb}$}‚ त‚स्य वेद‚न‚रूप‚ता व्य‚व‚स्थाप्येतेत्याह--\textbf{न चे}ति । \textbf{चो}ऽव‚धार‚णे, य‚स्माद‚र्थे वा । गृह्य‚माण‚{\tiny $_{lb}$}‚ आकारो य‚स्य ग्राह्य‚स्व‚भाव इत्य‚र्थः । \textbf{सात‚रूपः} सुख‚स्व‚भावः । सात‚ग्र‚ह‚ण‚स्योप‚ल‚क्ष‚ण‚त्वाद्‚{\tiny $_{lb}$}‚ दुःख‚रूप इत्य‚पि द्र‚ष्ट‚व्य‚म् । \textbf{इति}ना व‚च‚न‚स्याकारं द‚र्श‚य‚ति । कुत एवं व‚क्तुं न श‚क्य‚त‚{\tiny $_{lb}$}‚ इत्याह--\textbf{य‚त} इति । \textbf{इति}र्निश्च‚य‚स्य स्व‚रूप‚माह । किं ताद्रूप्येण निश्च‚येऽपि ताद्रूप्यानुभ‚वः‚{\tiny $_{lb}$}‚ सिद्ध्य‚ति, येन त‚दा\edtext{}{\lemma{दा}\Bfootnote{द}}भावान्नैवं श‚क्य‚ते व‚क्तुमित्युच्य‚ते इत्याह--\textbf{य‚दी}ति । हि\add{र्य}स्माद्‚{\tiny $_{lb}$}‚ इति । \textbf{इ}तिक‚र‚णेन निश्च‚य‚स्व‚रूप‚मुक्त‚म् । अनेन य‚देवानुरूप‚विक‚ल्पेन य‚थात्वेन निश्चीय‚ते‚{\tiny $_{lb}$}‚ त‚देव \textbf{त‚थात्व}व्य‚व‚हार‚गोच‚रो य‚था नीलादिरित्याकूत‚म् । न‚नु च त‚त्प्र‚तिभासाद‚नुभ‚वः प्र‚माण‚म् ।‚{\tiny $_{lb}$}‚ ‚{\tiny $_{lb}$}‚ \leavevmode\ledsidenote{\textenglish{66/dm}}‚{\tiny $_{lb}$}‚ 
	  
	न च नील‚स्य \edtext{}{\lemma{स्य}\Bfootnote{सात‚दिरूप० \cite{dp-msC}}}सात‚रूप‚त्व‚म‚नुग‚म्य‚ते । त‚स्माद‚सातान्नीलाद्य‚र्थाद‚न्य‚देव\edtext{}{\lemma{देव}\Bfootnote{नीलाद‚र्थाद० \cite{dp-msC} \cite{dp-msD}}} सात‚म‚नु‚{\tiny $_{lb}$}‚भूय‚ते\edtext{}{\lemma{ते}\Bfootnote{सात‚रूप‚त्व‚म‚नु० \cite{dp-msD}}} नीलानुभ‚व‚काले । त‚च्च ज्ञान‚मेव । त‚तोऽस्ति\edtext{}{\lemma{तोऽस्ति}\Bfootnote{०स्ति विज्ञा० \cite{dp-msD}}} ज्ञानानुभ‚वः । ‚{\tiny $_{lb}$}‚ 
	  
	त‚च्च \edtext{}{\lemma{च्च}\Bfootnote{ज्ञान‚रूपं वेद‚नं \cite{dp-msA} \cite{dp-msB} \cite{dp-edP} \cite{dp-edH} \cite{dp-edE} \cite{dp-edN} ज्ञान‚स्व‚रूप‚वेद‚नं \cite{dp-msC} \cite{dp-msD}}}ज्ञान‚रूप‚वेद‚न‚मात्म‚नः साक्षात्कारि निर्विक‚ल्प‚क‚म‚भ्रान्तं च । \edtext{\textsuperscript{*}}{\lemma{*}\Bfootnote{त‚स्मात्--\cite{dp-msA} \cite{dp-msB} \cite{dp-msC} \cite{dp-edP} \cite{dp-edE} \cite{dp-edH} \cite{dp-edN}}}त‚तः प्र‚त्य‚क्ष‚म् ॥‚{\tiny $_{lb}$}‚ निश्च‚यो भ‚व‚तु, मा वा भूत् । त‚त्किमेव‚मुच्य‚त इत्याश‚ङ्क्य पूर्वोक्त‚मेव स्मार‚य‚ति--\textbf{य‚स्मिन्नि}ति ।‚{\tiny $_{lb}$}‚ रूपे स्व‚भावे । शेषं प्रागेव कृत‚व्याख्यान‚म् । गृहीतं नील‚स्य सात‚रूप‚त्वं विक‚ल्पेनानुग‚म्य‚त‚{\tiny $_{lb}$}‚ एवेत्याह--\textbf{न चे}ति । चः पूर्व‚व‚त् । \textbf{अनुग‚म्य}ते विक‚ल्पेनेति व‚र्त्त‚ते । \textbf{त‚स्मादित्या}दिनोक्त‚{\tiny $_{lb}$}‚मुप‚संह‚र‚ति । य‚स्माद् गृह्य‚माणाकारो नीलादिर्न त‚था निश्चीय‚ते, अस्ति च ह‚र्षाद्याकार‚{\tiny $_{lb}$}‚संवेद‚न‚म्, त‚स्माद् ।
	\pend% ending standard par
      ‚{\tiny $_{lb}$}‚

	  \pstart \leavevmode% starting standard par
	भ‚व‚तु साताकारोऽनुभूतः । केव‚ल‚म‚साव‚ज्ञानात्मा भ‚विष्य‚ति । त‚था च क‚थ‚म‚ज्ञानेन‚{\tiny $_{lb}$}‚ ज्ञानात्म‚संवेद‚न‚म् ? क‚थं चात्म‚वेद‚नं प्र‚त्य‚क्ष‚मित्याह--\textbf{त‚च्चे}ति । \textbf{चो}ऽज्ञात्म‚न एवं भिन‚त्ति ।‚{\tiny $_{lb}$}‚ \textbf{ज्ञान‚मेवेत्य‚व‚धा}र‚य‚तः प्र‚काशात्म‚न एवं ज्ञान‚त्व‚म‚न्य‚था प्र‚काशायोगादित्य‚भिप्रायः । य‚त एवं‚{\tiny $_{lb}$}‚ \textbf{त‚तोऽस्ति ज्ञानानुभ‚वः} । इत्य‚न्त‚र्मुख‚स्य सुखाद्याकार‚स्य ग्राह‚काकाराख्य‚स्येत्य‚र्थः । न च‚{\tiny $_{lb}$}‚ ग्राह्याकाराद‚न्य द‚नुभूय‚मानं सातं ग्राह‚काकाराद‚प्य‚न्य‚दुप‚प‚द्य‚ते । ग्राह्यं वा प्र‚काशेत, ग्राह‚कं‚{\tiny $_{lb}$}‚ वा । न च त‚द् ग्राह्य‚म‚तो ग्राह‚क‚मेव । अथ ज्ञान‚म‚स्तु त‚थानुभूत‚म् । त‚त्पुन‚रात्म‚संवेद‚नं प्र‚त्य‚क्षं‚{\tiny $_{lb}$}‚ कुत इत्याह--\textbf{त‚च्चे}ति । \textbf{चो} य‚स्माद् । वेद्य‚तेऽनेनेति \textbf{वेद‚न‚म्} । ज्ञान‚रूप‚स्य \textbf{वेद‚न‚मि}ति‚{\tiny $_{lb}$}‚ विग्र‚हः\leavevmode\ledsidenote{\textenglish{29a/ms}} । य‚द्वा ज्ञान‚रूपं च त‚द् वेद‚न‚ञ्चेति त‚था । \textbf{साक्षात्कारि} अप‚रोक्ष‚ताकारि,‚{\tiny $_{lb}$}‚ स्कुटाव‚भास‚मिति याव‚त्, हेतुभावेनास्य विशेष‚ण‚त्वात् । अत एव च निर्विक‚ल्प‚क‚म् । विक‚ल्पा‚{\tiny $_{lb}$}‚नुब‚द्ध‚स्य स्प‚ष्टार्थ‚प्र‚तिभासित्वायोगात् । भ‚व‚तु निर्विक‚ल्प‚कं द्विच‚न्द्रादिज्ञान‚व‚द् भ‚विष्य‚ती‚{\tiny $_{lb}$}‚त्याह--\textbf{अभ्रान्तं चे}ति । \textbf{च}स्तुल्योपाय‚त्वं स‚मुच्चिनोति । न‚हि त‚त् स्वात्म‚नि अत‚स्मिंस्त‚दिति‚{\tiny $_{lb}$}‚ प्र‚वृत्त‚म्, येन त‚त्र भ्रान्तिर्भ‚विष्य‚तीति भावः । भ‚व‚तां निर्विक‚ल्प‚क‚त्वाभ्रान्त‚त्वे त‚तः किं‚{\tiny $_{lb}$}‚ सिद्ध‚मित्याह--त‚त इति । य‚त एत‚द् रूप‚द्व‚य‚योगि त‚त‚स्त‚स्मात् । एताव‚न्मात्र‚ल‚क्ष‚ण‚त्वा‚{\tiny $_{lb}$}‚त्प्र‚त्य‚क्ष‚स्येति भावः ।
	\pend% ending standard par
      ‚{\tiny $_{lb}$}‚

	  \pstart \leavevmode% starting standard par
	त‚त्र चेयं व्य‚व‚स्था । आत्मात्म‚योग्य‚ता प्र‚माण‚म्, आत्म‚संवित् फ‚ल‚मिति द्र‚ष्ट‚व्य‚म् ।‚{\tiny $_{lb}$}‚ स्यान्म‚त‚म्--इत्यं त‚स्य वेद‚न‚स्य प्र‚त्य‚क्ष‚त्वे विक‚ल्पात्म‚वेद‚न‚स्यापि त‚त्त्वं स्यात् । न च‚{\tiny $_{lb}$}‚ विक‚ल्पात्मा प्र‚त्य‚क्षं युज्य‚ते । युज्य‚ते, स्वात्म‚नि अविक‚ल्प‚नाद‚भ्रान्त‚त्वाच्च । विक‚ल्पो हि बाह्यं‚{\tiny $_{lb}$}‚ विक‚ल्प‚य‚ति न त्वात्मान‚म् । भ्राम्य‚ति च ब्राह्ये, नात्म‚नि । त‚तः किं न प्र‚त्य‚क्ष‚म् ? प्र‚योगः—‚{\tiny $_{lb}$}‚य‚द‚भ्रान्त‚त्वे स‚त्य‚विक‚ल्पं त‚त् प्र‚त्य‚क्ष‚म् । य‚थेन्द्रिय‚ज्ञान‚स्य बाह्य‚संवेद‚न‚म् सारूप्याख्य‚म् ।‚{\tiny $_{lb}$}‚ अभ्रान्त‚त्वे स‚त्य‚विक‚ल्प‚कं चात्म‚नि विक‚ल्प‚रूप‚वेद‚न‚मिति स्व‚भावः । य‚था च ज्ञानात्म‚नि‚{\tiny $_{lb}$}‚ स‚म‚यास‚म्भ‚वो य‚था वा स‚मितः श‚ब्द‚संसृष्टो न गृह्य‚ते त‚थाऽन्य‚त्र प्र‚प‚ञ्चित‚मिति नेहोच्य‚त इति ॥
	\pend% ending standard par
      ‚{\tiny $_{lb}$}‚‚{\tiny $_{lb}$}‚\textsuperscript{\textenglish{67/dm}}‚{\tiny $_{lb}$}‚
	  \bigskip
	  \begingroup
	

	  \pstart \leavevmode% starting standard par
	योगिप्र‚त्य‚क्षं व्याख्यातुमाह\edtext{}{\lemma{व्याख्यातुमाह}\Bfootnote{०ख्यातुकाम आह \cite{dp-msC} \cite{dp-msD}}}
	\pend% ending standard par
      
	  \endgroup
	‚{\tiny $_{lb}$}‚
	  \bigskip
	  \begingroup
	

	  \pstart \leavevmode% starting standard par
	भूतार्थ‚भाव‚नाप्र‚क‚र्ष‚प‚र्य‚न्त‚जं योगिज्ञानं चेति ॥ ११ ॥
	\pend% ending standard par
      
	  \endgroup
	‚{\tiny $_{lb}$}‚

	  \pstart \leavevmode% starting standard par
	भूतः \edtext{}{\lemma{भूतः}\Bfootnote{य‚थाव‚स्थितः--\cite{dp-msD-n}}}स‚द्भूतोऽर्थः । प्र‚माणेन दृष्ट‚श्च स‚द्भूतः ।
	\pend% ending standard par
      ‚{\tiny $_{lb}$}‚

	  \pstart \leavevmode% starting standard par
	य‚था \edtext{}{\lemma{था}\Bfootnote{दुःख‚स‚मुद‚य‚मार्ग‚निरोधाः । त‚त्र दुःखं संसारिणः स्क‚न्धाः । स‚मुद‚यो रागादिग‚णः ।‚{\tiny $_{lb}$}‚ मार्गः क्ष‚णिक‚त्व‚भाव‚ना । निरोधो मोक्षः ।--\cite{dp-msD-n}}}च‚त्वार्यार्य‚स‚त्यानि ।
	\pend% ending standard par
      ‚{\tiny $_{lb}$}‚
	  \bigskip
	  \begingroup
	

	  \pstart \leavevmode% starting standard par
	भूत‚स्य भाव‚ना पुनः पुन‚श्चेत‚सि विनिवेश‚न‚म् । भाव‚नायाः प्र‚क‚र्षो भाव्य‚मानार्थाभास‚स्य\edtext{}{\lemma{स्य}\Bfootnote{०र्थाव‚भास‚स्य \cite{dp-msB} \cite{dp-edN}}}‚{\tiny $_{lb}$}‚ ज्ञान‚स्य स्फुटाभ‚त्वार‚म्भः । प्र‚क‚र्ष‚स्य प‚र्य‚न्तो य‚दा स्फुटाभ‚त्व‚मीष‚द‚स‚म्पूर्णं भ‚व‚ति । याव‚द्धि‚{\tiny $_{lb}$}‚ स्फुटाभ‚त्व‚म‚प‚रिपूर्णं ताव‚त् त‚स्य\edtext{}{\lemma{स्य}\Bfootnote{स्फुटाभ‚त्व‚स्य--\cite{dp-msD-n}}} प्र‚क‚र्ष\edtext{}{\lemma{र्ष}\Bfootnote{प्र‚क‚र्ष‚ग‚तिः \cite{dp-msA} \cite{dp-edP} \cite{dp-edH} \cite{dp-edE} \cite{dp-edN}}}ग‚म‚न‚म् ।
	\pend% ending standard par
      
	  \endgroup
	‚{\tiny $_{lb}$}‚

	  \pstart \leavevmode% starting standard par
	\textbf{भूत}श‚ब्द‚स्य विव‚क्षित‚म‚र्थ‚माह--\textbf{स‚द्भूत} इति । \textbf{अर्थ} इति ब्रुवाणो भूत‚श्चासाव‚र्थ‚{\tiny $_{lb}$}‚श्चेति क‚र्म‚धार‚यं द‚र्श‚य‚ति । न‚नु सुखादिम‚य‚त्व‚म‚प्य‚र्थ‚स्य \textbf{सांख्य}प‚रिक‚ल्पितं स‚द्भूत‚मित्याह—‚{\tiny $_{lb}$}‚\textbf{प्र‚माणेने}ति । \textbf{दृष्टो} निश्चितः । \textbf{च}कारः स्फुट‚मेत‚दित्य‚र्थं द्योत‚य‚ति ।
	\pend% ending standard par
      ‚{\tiny $_{lb}$}‚

	  \pstart \leavevmode% starting standard par
	कः पुन‚रीदृशोऽर्थो विव‚क्षित इत्याह--\textbf{य‚थे}ति । अनेन \textbf{भूतार्थ}श‚ब्देनात्र स‚त्य‚च‚तुष्ट‚यं‚{\tiny $_{lb}$}‚ विव‚क्षित‚मिति द‚र्शित‚म् । य‚था च‚तुरार्य‚स‚त्य‚भाव‚नाब‚ल‚जं ज्ञानं योगिनः प्र‚त्य‚क्ष‚म्, त‚थाऽन्य‚स‚द्‚{\tiny $_{lb}$}‚भूतार्थ‚विष‚य‚म‚पि द्र‚ष्ट‚व्य‚मिति य‚थाश‚ब्दार्थोऽप्य‚स्ति । य\textbf{द्विनिश्च‚यः} तुरार्य‚स‚त्य‚द‚र्श‚न‚व‚दिति । ‚{\tiny $_{lb}$}‚ आरात्पाप‚केभ्यो ध‚र्मेभ्यो याता इत्या\textbf{र्याः} । अत एव तानि स‚त्य‚त‚या म‚न्य‚न्त इति तेषां स‚त्यानि ।‚{\tiny $_{lb}$}‚ च‚तुष्ट्वाच्च तेषां च‚त्वारीत्युक्त‚म् ।
	\pend% ending standard par
      ‚{\tiny $_{lb}$}‚

	  \pstart \leavevmode% starting standard par
	फ‚ल‚भूताः प‚ञ्च स‚क्लेश‚स्क‚न्धा दुःखाख्यं स‚त्य‚मेक‚म् । त एव हेतुभूतास्तृष्णास‚हायाः‚{\tiny $_{lb}$}‚ स‚मुद‚याख्यं स‚त्यं द्वितीय‚म् । चित्त‚स्य निष्क्लेशाव‚स्था निरोधाख्यं स‚त्यं तृतीय‚म् । त‚द‚{\tiny $_{lb}$}‚व‚स्थाप्राप्तिहेतुनैरात्म्याद्याकार‚श्चित्त‚विशेषो मार्गाख्यं स‚त्यं च‚तुर्थ‚मिति ।
	\pend% ending standard par
      ‚{\tiny $_{lb}$}‚

	  \pstart \leavevmode% starting standard par
	भाव‚नाश‚ब्देन विग्र‚हं त‚स्य चार्थ‚माच‚ष्टे \textbf{भूत‚स्ये}ति । भूतार्थ‚स्येति द्र‚ष्ट‚व्य‚म् । ल‚क्ष्य‚ते च‚{\tiny $_{lb}$}‚ भूत‚श‚ब्द‚सान्निध्याल्लेख‚केन प्र‚थ‚म‚पुस्त‚के भूश‚ब्दः प्र‚क्षिप्तः । \textbf{त‚स्ये}ति तु व‚च‚नं संक्षेपेण विग्र‚हं‚{\tiny $_{lb}$}‚ द‚र्श‚य‚तो \textbf{ध‚र्मोत्त‚र‚स्य} पाठोऽन्य‚था य‚थाभूतं विग्र‚हं द‚र्श‚यितुकामेनार्थ‚प‚दोपादाने किम‚क्ष‚र‚गौर‚वं दृष्ट‚म्,‚{\tiny $_{lb}$}‚ येन केव‚ल‚भूत‚श‚ब्दोपादाने प्र‚तिप‚त्तिगौर‚वं लिख‚नाकौश‚ल‚ञ्चाविष्कृत‚मिति । भाव‚नार्थ‚माह—‚{\tiny $_{lb}$}‚\textbf{पुन‚रि}ति । \textbf{पुन‚रि}त्य‚प्र‚थ‚म‚तः । द्विर्व‚च‚नेनाप्र‚थ‚म‚प्र‚चार‚स्य प्राचुर्यं द‚र्श‚य‚ति । त‚थानिवेश‚न‚ञ्च‚{\tiny $_{lb}$}‚ विजा\leavevmode\ledsidenote{\textenglish{29b/ms}}तीयाव्य‚व‚धानेन द्र‚ष्ट‚व्य‚म् । स‚त्य‚च‚तुष्ट‚य‚विष‚यो विजातीयाव्य‚व‚हितः‚{\tiny $_{lb}$}‚ स‚दृश‚चित्त‚प्र‚वाहो भाव‚नेति याव‚त् ।
	\pend% ending standard par
      ‚{\tiny $_{lb}$}‚‚{\tiny $_{lb}$}‚\textsuperscript{\textenglish{68/dm}}‚{\tiny $_{lb}$}‚
	  \bigskip
	  \begingroup
	

	  \pstart \leavevmode% starting standard par
	स‚म्पूर्ण तु य‚दा, त‚दा नास्ति प्र‚क‚र्ष‚ग‚तिः । त‚तः स‚म्पूर्णाव‚स्थायाः प्राक्त‚न्य‚व‚स्था स्फुटा‚{\tiny $_{lb}$}‚\edtext{}{\lemma{स्फुटा}\Bfootnote{त्वं प्र‚क० \cite{dp-edE}}}भ‚त्व‚प्र‚क‚र्ष‚प‚र्य‚न्त उच्य‚ते । त‚स्मात् प‚र्य‚न्ताद् य‚ज्जातं\edtext{}{\lemma{ज्जातं}\Bfootnote{जातं ज्ञानं भाव्य० \cite{dp-msC} \cite{dp-msD}}} भाव्य‚मान‚स्यार्थ‚स्य\edtext{}{\lemma{स्य}\Bfootnote{भाव्य‚मान‚स्य सं० \cite{dp-msA} \cite{dp-msB} \cite{dp-edP} \cite{dp-edH} \cite{dp-edE}}} स‚न्निहित‚स्येव‚{\tiny $_{lb}$}‚ स्फुट‚त‚राकार‚ग्राहि ज्ञानं योगिनः प्र‚त्य‚क्ष‚म् ।
	\pend% ending standard par
       ‚{\tiny $_{lb}$}‚ 

	  \pstart \leavevmode% starting standard par
	त‚दिह स्फुटाभ‚त्वार‚म्भाव‚स्था भाव‚नाप्र‚क‚र्षः । अभ्र‚क‚व्य‚व‚हित‚मिव य‚दा भाव्य‚मानं
	\pend% ending standard par
      
	  \endgroup
	‚{\tiny $_{lb}$}‚

	  \pstart \leavevmode% starting standard par
	यादृशो योगिनां भाव‚नाक्र‚मो \textbf{विनिश्च‚ये} श्रुत‚म‚येत्यादिनाभिहितो य‚था भाव‚ना‚{\tiny $_{lb}$}‚प्र‚क‚र्ष‚विश‚दाभ‚त्व‚योः कार्य‚कार‚ण‚भाव‚स्त‚त्रैव काम‚शोकेत्यादिना द‚र्शित‚स्त‚थेहापि द्र‚ष्ट‚व्यः ।‚{\tiny $_{lb}$}‚ यादृश‚श्चाकार‚स्तेषां स‚त्यानाम‚नित्य‚त्वादिके भाव‚नीयो याव‚त्कालाव‚धिका च भाव‚नाऽनेक‚ज‚न्म‚{\tiny $_{lb}$}‚प‚र‚म्प‚रानुयाता, य‚च्च निब‚न्ध‚नं भाव‚नायाः क‚रुणा बोधिस‚त्त्वानाम्, त‚द‚न्येषां संसारोद्वेग‚स्त‚द‚पि‚{\tiny $_{lb}$}‚ स‚र्व य‚था \textbf{प्र‚माण‚वार्त्तिके} निर्णीतं त‚थेहाप्य‚नुग‚न्त‚व्य‚म् । इह तु योगिज्ञान‚स्य स्व‚रूप‚मात्र‚मुप‚{\tiny $_{lb}$}‚द‚र्श‚यितुमुप‚क्रान्त‚मिति ।
	\pend% ending standard par
      ‚{\tiny $_{lb}$}‚

	  \pstart \leavevmode% starting standard par
	\textbf{प्र‚क‚र्ष}श‚ब्देन स‚ह विग्र‚हं त‚स्य चार्थं विव‚क्षित‚माह--\textbf{भाव‚नाया} इति । \textbf{स्फुटाभ‚त्व‚स्या}र‚म्भ‚{\tiny $_{lb}$}‚ उप‚क्र‚मः । स च य‚त्स्फुट‚त्व‚त‚द‚धिक‚स्फुट‚त्वादिना रूपेण त‚ज्ज्ञान‚स्योद‚य एव । प‚र्य‚न्त‚श‚ब्देन‚{\tiny $_{lb}$}‚ विग्र‚हं त‚स्य चार्थ‚माच‚ष्टे \textbf{प्र‚क‚र्ष‚स्येति । प‚र्य‚न्तो}ऽव‚सान‚म् । क‚दा च त‚स्याव‚सान‚मित्याह--\textbf{य‚दे}ति ।‚{\tiny $_{lb}$}‚ य‚स्मिन् काले स्फुटाभ‚त्वं भाव‚नार्थं विष‚य‚स्य ज्ञान‚स्येति प्र‚क‚र‚णात् । इदं लेश‚तोऽस‚म्पूर्णं भ‚व‚ति‚{\tiny $_{lb}$}‚ य‚द\add{न}न्त‚रं योगिप्र‚त्य‚क्षेण भ‚वित‚व्यं त‚स्मिन् काले प्र‚क‚र्ष‚स्य प‚र्य‚न्तोऽव‚सानं ज्ञात‚व्यः । त‚त्कालो‚{\tiny $_{lb}$}‚प‚ल‚क्षितं त‚थाभूतं ज्ञानं प‚र्य‚न्त इत्य‚र्थः ।
	\pend% ending standard par
      ‚{\tiny $_{lb}$}‚

	  \pstart \leavevmode% starting standard par
	न‚नु प्र‚क‚र्ष‚स्य प‚र्य‚न्तः स युज्य‚ते य‚स्मिन् स‚ति प्र‚क‚र्षो निव‚र्त्त‚ते । त‚च्च स‚म्पूर्वं‚{\tiny $_{lb}$}‚\edtext{}{\lemma{म्पूर्वं}\Bfootnote{र्ण}}मेव स्कुटाभ‚त्व‚म्, त‚त्क‚थ‚मुच्य‚ते क‚थ‚ञ्च \textbf{प‚र्य‚न्त‚जं योगिज्ञान‚म्,} न तु त‚देव प‚र्य‚न्त इत्या‚{\tiny $_{lb}$}‚श‚ङ्क्याह--\textbf{याव‚दि}ति । हिर्य‚स्मात् ।
	\pend% ending standard par
      ‚{\tiny $_{lb}$}‚

	  \pstart \leavevmode% starting standard par
	स‚म्पूर्णे तु प्र‚क‚र्ष‚ग‚म‚नं नास्तीति द‚र्श‚य‚ति--\textbf{स‚म्पूर्ण}मिति । \textbf{तु}रिमाम‚व‚स्थां भेद‚व‚ती‚{\tiny $_{lb}$}‚माह । \textbf{प्र‚क‚र्ष}स्य ग‚तिर्ग‚म‚न‚म् । एवं ब्रुव‚तोऽय‚माश‚यः । प्र‚क‚र्षः प्र‚कृष्य‚माण‚ता साति‚{\tiny $_{lb}$}‚श‚यं रूप‚मुच्य‚ते । प‚र्य‚न्त‚श्च ग‚त्य‚र्थादाम‚र्द्धातोस्त‚त्प्र‚त्य‚येन \edtext{}{\lemma{येन}\Bfootnote{?}} प‚रिस‚म‚न्ताद‚न्त इति प्रादिस‚मासेन‚{\tiny $_{lb}$}‚ निःशेष‚ग‚म‚न‚मेवोच्य‚ते । त‚तः स प‚र्य‚न्त उच्य‚ते य‚द‚न‚न्त‚रं प्र‚क‚र्ष\edtext{}{\lemma{र्ष}\Bfootnote{कृष्य}}माणेन न‚{\tiny $_{lb}$}‚ भ‚वित‚व्य‚म् । न तु य‚द‚र्थं येषां प्र‚क‚र्ष‚व‚तामुद‚य इति । एत‚देवोप‚संह‚र‚ति \textbf{त‚त} इति । \textbf{प्राक्त‚नी}‚{\tiny $_{lb}$}‚ व्य‚व‚धान‚शून्या य‚द‚न‚न्त‚रं स्फुट‚त‚र‚ज्ञानोद‚यः । \textbf{स‚म्पूर्णाव‚स्थायाः} स्फुट‚त्व‚स‚म्पूर्णाव‚स्थायाः‚{\tiny $_{lb}$}‚ स्फुट‚त‚राकार‚ग्र‚ह‚णाव‚स्थाया इति याव‚त् । \textbf{स‚न्निहित}स्येति । य‚थाऽन्य‚स्याभावित‚स्य निक‚ट‚स्थ‚स्य‚{\tiny $_{lb}$}‚ घ‚ट‚घ\edtext{}{\lemma{घ}\Bfootnote{प}}टाद\edtext{}{\lemma{टाद}\Bfootnote{दे}}र‚न्य‚ज्ञानं स्फुट‚त‚राकार‚ग्राहि प्र‚त्य‚क्षं त‚द्व‚द् भाव्य‚मानार्थ‚स्फुट‚त‚राकार‚ग्राहि‚{\tiny $_{lb}$}‚ \textbf{य‚ज्ज्ञानं त‚द्योगिनः प्र‚त्य‚क्ष‚म्} ।
	\pend% ending standard par
      ‚{\tiny $_{lb}$}‚

	  \pstart \leavevmode% starting standard par
	उक्त‚मेव भाव‚नाप्र‚क‚र्षार्थं त‚त्प‚र्य‚न्तार्थं त‚ज्ज्ञानं चोप‚संहार‚व्याजेन सुख‚प्र‚तिप‚त्त्य‚र्थं पुन‚{\tiny $_{lb}$}‚र्द‚र्श‚य‚ति \textbf{त‚दि}ति । \textbf{त‚त्त}स्मात् । \textbf{इहे}ति योगिप्र‚त्य‚क्ष‚ल‚क्ष‚ण‚प्र‚तीतिकाले । \textbf{अव‚स्था} भाव‚नाज्ञान‚{\tiny $_{lb}$}‚‚{\tiny $_{lb}$}‚ ‚{\tiny $_{lb}$}‚ \leavevmode\ledsidenote{\textenglish{69/dm}}‚{\tiny $_{lb}$}‚ 
	  
	व‚स्तु प‚श्य‚ति सा प्र‚क‚र्ष‚प‚र्य‚न्ताव‚स्था । क‚र‚त‚लाम‚ल‚क‚व‚द् भाव्य‚मान‚स्यार्थ‚स्य य‚द् द‚र्श‚नं त‚द्‚{\tiny $_{lb}$}‚ योगिनः प्र‚त्य‚क्ष‚म् । त‚द्धि स्फुटाभ‚म् । ‚{\tiny $_{lb}$}‚ 
	  
	स्फुटाभ‚त्वादेव च निर्विक‚ल्प‚क‚म् । विक‚ल्प‚विज्ञानं हि स‚ङ्केत‚काल‚दृष्ट‚त्वेन व‚स्तु गृह्ण‚{\tiny $_{lb}$}‚च्छ‚ब्द‚संस‚र्ग‚योग्यं गृह्णीयात् । स‚ङ्केत‚काल‚दृष्ट‚त्वं च स‚ङ्केत‚कालोत्प‚न्न‚ज्ञान‚विष‚य‚त्व‚म् । य‚था च‚{\tiny $_{lb}$}‚ पूर्वोत्प‚न्नं विन‚ष्टं ज्ञानं स‚म्प्र‚त्य‚स‚त्, त‚द्व‚त् पूर्व‚विन‚ष्ट‚ज्ञान‚विष‚य‚त्व‚म‚पि स‚म्प्र‚ति नास्ति व‚स्तुनः ।‚{\tiny $_{lb}$}‚ त‚द‚स‚द्रूपं व‚स्तुनो गृह्ण‚द् अस‚न्निहि\edtext{}{\lemma{न्निहि}\Bfootnote{०हित‚ग्राहि० \cite{dp-msB} \cite{dp-msC}}} तार्थ‚ग्राहित्वाद् अस्फुटाभ‚म्\edtext{}{\lemma{म्}\Bfootnote{०ग्राहित्वाद‚स्फुटाभं स‚विक‚ल्प‚क‚म् \cite{dp-msC} अस्फुटाभ‚म् । अस्फुटा‚{\tiny $_{lb}$}‚भ‚त्वात् स‚विक‚ल्प‚कं--\cite{dp-msA} \cite{dp-edP} अस्फुटाभ‚म् । अस्फुटाभ‚त्वादेव च स‚वि० \cite{dp-msD} \cite{dp-msB} \cite{dp-edH} \cite{dp-edE} \cite{dp-edN}}} विक‚ल्प‚क‚म् । त‚तः स्फुटाभ‚{\tiny $_{lb}$}‚त्वान्निर्विक‚ल्प‚क‚म् ।‚{\tiny $_{lb}$}‚ स्येति प्र‚क‚र‚णात् । \textbf{अभ्र‚के}णातिस्व‚च्छ‚त‚या \textbf{पाश}\edtext{}{\lemma{या}\Bfootnote{र्श्व}}तो द्विधाक‚र्त्तुम‚श‚क्येनेति प्र‚स्तावाद्‚{\tiny $_{lb}$}‚ \textbf{व्य‚व‚हित}मावृतं त‚दिव । क‚र‚स्व‚रूपं \textbf{क‚र‚त‚लं} स‚न्निहित‚स्येवेत्य‚नेन य‚त्पूर्व‚मुक्तं\leavevmode\ledsidenote{\textenglish{30a/ms}} त‚स्याय‚{\tiny $_{lb}$}‚मुप‚संहारः ।
	\pend% ending standard par
      ‚{\tiny $_{lb}$}‚

	  \pstart \leavevmode% starting standard par
	अस्तु त‚थाविधं योगिज्ञान‚म्, त‚त्पुनः क‚थं प्र‚त्य‚क्ष‚मित्याश‚ङ्काम‚पाक‚र्त्तु प्र‚त्य‚क्ष‚ल‚क्ष‚णेन‚{\tiny $_{lb}$}‚ योग‚म‚स्य द‚र्श‚य‚न्नाह \textbf{त‚द्धी}ति । \textbf{हि}र्य‚स्मात् ।
	\pend% ending standard par
      ‚{\tiny $_{lb}$}‚

	  \pstart \leavevmode% starting standard par
	स्फुटाभं भ‚व‚तु, निर्विक‚ल्प‚कं तु क‚थं येन प्र‚त्य‚क्षं स्यादित्याह--\textbf{स्फुटाभ‚त्वादेवेति ।‚{\tiny $_{lb}$}‚ चो निर्विक‚ल्प‚क‚मि}त्य‚तः प‚रः स्फुटाभ‚त्वापेक्षैक‚विष‚य‚त्वं निर्विक‚ल्प‚क‚त्व‚स्य स‚मुच्चिनोति ।‚{\tiny $_{lb}$}‚ अय‚म‚स्याश‚यः--श‚ब्दाकार‚संस‚र्गो हि स्फुटाभ‚त्व‚विरोधीति य‚द्य‚साव‚भ‚विष्य‚त् त‚दा स्फुटाभ‚मेव‚{\tiny $_{lb}$}‚ नाभ‚विष्य‚दिति । अभिलाप‚संस‚र्ग‚योग्य‚प्र‚तिभास‚मेव स्फुटाभं भ‚विष्य‚ति कोऽन‚योर्विरोध इत्याश‚{\tiny $_{lb}$}‚ङ्क्य \textbf{विक‚ल्पेत्यादिना} विरोध‚मेव द‚र्श‚यितुमुप‚क्र‚म‚ते । \textbf{हि}र्य‚स्मात् । \textbf{व‚स्त्वि}ति दृश्य‚मानं य‚था‚{\tiny $_{lb}$}‚ चैत‚त्त‚था पूर्व‚मेव विवेचितं \textbf{गृह्णीयाद्} ग्र‚हीतुम‚र्ह‚ति । स‚ङ्केत‚काल‚दृष्ट‚त्वेन च व‚स्तु‚{\tiny $_{lb}$}‚ग्राहित्वेऽस‚न्निहित‚विष‚यं स्यादिति द‚र्श‚यितुमाह--\textbf{स‚ङ्केतेत्या}दि । \textbf{चो} हेतौ । भ‚व‚तु स‚ङ्केत‚{\tiny $_{lb}$}‚काल‚दृष्ट‚त्व‚स्य त‚थात्व‚म्, किम‚त इत्याह--\textbf{य‚था चे}ति । \textbf{चो} व‚क्त‚व्य‚मेत‚दित्य‚स्यार्थे ।
	\pend% ending standard par
      ‚{\tiny $_{lb}$}‚

	  \pstart \leavevmode% starting standard par
	भ‚व‚त्वेवं त‚थापि वाच‚काकार‚संस‚र्गिणोऽपि क‚थं न स्फुटाभ‚त्व‚मित्याह--\textbf{त‚दि}ति । य‚त‚{\tiny $_{lb}$}‚ एवं \textbf{त‚त्त}स्मात् । विक‚ल्प‚य‚तीति \textbf{विक‚ल्प‚कं} विज्ञान‚म् । अस्फुटाभं तु कुत‚स्त‚दित्याह—‚{\tiny $_{lb}$}‚\textbf{अस‚न्निहितार्थ‚ग्राहित्वादि}ति । स‚न्निहित‚त‚या हि भास‚मानो विश‚दो भ‚वेत् । पूर्व‚वृत्त‚द‚र्श‚न‚{\tiny $_{lb}$}‚विष‚य‚त‚या च भूत‚पूर्वो भावो गृह्य‚माणो न स‚न्निहित‚रूपो भास‚ते । तेनाविश‚दाभो विक‚ल्पः ।‚{\tiny $_{lb}$}‚ तेन श‚ब्दाकार‚संस‚र्गो विक‚ल्प‚स्य विश‚दाभ‚त्व‚विरोधीत्य‚स्याभिप्रायः । अस‚न्निहित‚विष‚य‚त्व‚मेव‚{\tiny $_{lb}$}‚ त‚स्य कुतोऽव‚सीय‚त इत्याह--\textbf{अस‚द्रूप‚मिति} । य‚त्पूर्व‚दृष्ट‚त्वं स‚म्प्र‚त्य‚स‚द्, अस‚द्रूपं व‚स्तुनो‚{\tiny $_{lb}$}‚ गृह्य‚माण‚स्य \textbf{गृह्ण‚दिति} हेतौ श‚तुर्विधानाद‚स‚द्रूप‚ग्र‚ह‚णादित्य‚र्थः ।
	\pend% ending standard par
      ‚{\tiny $_{lb}$}‚

	  \pstart \leavevmode% starting standard par
	\textbf{स्फुटाभ‚त्वादेवे}ति य‚द‚वोच‚त्, त‚देवोप‚संह‚र‚ति \textbf{त‚त} इति । य‚तो विक‚ल्प‚स्य स्प‚ष्टाभ‚त्वं‚{\tiny $_{lb}$}‚ न युज्य‚ते, अन‚न्त‚रोक्तेन क्र‚मेण \textbf{त‚त}स्त‚स्मा\textbf{त्स्फुटाभ‚त्वाद्} विश‚दाभ‚त्वा\textbf{न्निर्विक‚ल्प‚कं} योगिज्ञान‚मिति‚{\tiny $_{lb}$}‚ प्र‚क‚र‚णात् । प्र‚योगः--य‚त् स‚ङ्केत‚काल‚दृष्ट‚त‚या व‚स्तुसंस्प‚र्शि ज्ञानं न त‚त् स्फुटाभ‚म् । य‚था चिर‚{\tiny $_{lb}$}‚‚{\tiny $_{lb}$}‚ \leavevmode\ledsidenote{\textenglish{70/dm}}‚{\tiny $_{lb}$}‚ 
	  
	प्र‚माण‚शुद्धार्थ‚ग्राहित्वाच्च संवाद‚क‚म् । अतः प्र‚त्य‚क्ष‚म् । इत‚र‚प्र‚त्य‚क्ष‚व‚त् । ‚{\tiny $_{lb}$}‚ 
	  
	योगः \edtext{}{\lemma{योगः}\Bfootnote{शुभ‚चित्तैकाग्र्य‚म्--\cite{dp-msD-n}}}स‚माधिः । स य‚स्यास्ति\edtext{}{\lemma{स्यास्ति}\Bfootnote{य‚स्यास्तीति स \cite{dp-msC}}} स योगी । त‚स्य ज्ञानं प्र‚त्य‚क्ष‚म् । इतिश‚ब्दः‚{\tiny $_{lb}$}‚ प‚रिस‚माप्ति\edtext{}{\lemma{माप्ति}\Bfootnote{प‚रिस‚माप्त्य‚र्थः \cite{dp-msA} \cite{dp-edP} \cite{dp-edH} \cite{dp-edE} \cite{dp-edN}}}व‚च‚नः । इय‚देव प्र‚त्य‚क्ष‚मिति ॥ ‚{\tiny $_{lb}$}‚ 
	  
	त‚देवं प्र‚त्य‚क्ष‚स्य क‚ल्प‚नापोढ‚त्वाभ्रान्त‚त्व‚युक्त‚स्य प्र‚कार‚भेदं प्र‚तिपाद्य विष‚य‚विप्र‚तिप‚त्तिं‚{\tiny $_{lb}$}‚ निराक‚र्त्तुमाह-- ‚{\tiny $_{lb}$}‚ 
	  
	\textbf{त‚स्य विष‚यः स्व‚ल‚क्ष‚ण‚म् ॥ १२ ॥}‚{\tiny $_{lb}$}‚ 
	  
	त‚स्येत्यादि । त‚स्य च‚तु\edtext{}{\lemma{तु}\Bfootnote{त‚स्य प्र‚त्य‚क्ष‚स्य \cite{dp-msC} त‚स्य च‚तुर्विध‚प्र‚त्य० \cite{dp-msA} \cite{dp-edP} \cite{dp-edH} \cite{dp-edE} \cite{dp-edN}}}र्विध‚स्य प्र‚त्य‚क्ष‚स्य विष‚यो बोद्ध‚व्यः\edtext{}{\lemma{व्यः}\Bfootnote{प‚र्याय‚ग्राह्य इत्य‚र्थः--\cite{dp-msD-n}}} स्व‚ल‚क्ष‚ण‚म् । स्व‚म्‚{\tiny $_{lb}$}‚असाधार‚णं ल‚क्ष‚णं त‚त्त्वं\edtext{}{\lemma{त्त्वं}\Bfootnote{अर्थ‚क्रियाकारित्व‚म्--\cite{dp-msD-n}}} स्व‚ल‚क्ष‚ण‚म् । व‚स्तुनो ह्यास‚धार‚णं च त‚त्त्व‚म‚स्ति सामान्यं च ।‚{\tiny $_{lb}$}‚ \edtext{\textsuperscript{*}}{\lemma{*}\Bfootnote{त‚त्र नास्ति--\cite{dp-msA} \cite{dp-msB} \cite{dp-msD} \cite{dp-edP} \cite{dp-edH} \cite{dp-edE}}}त‚त्र य‚द‚साधार‚णं त‚त् प्र‚त्य‚क्ष‚स्य\edtext{}{\lemma{स्य}\Bfootnote{प्र‚त्य‚क्ष‚ग्राह्य‚म् \cite{dp-msA} \cite{dp-msB} \cite{dp-edP} \cite{dp-edH} \cite{dp-edE}}} ग्राह्य‚म् ।‚{\tiny $_{lb}$}‚ दृष्ट‚न‚ष्ट‚व‚स्तुविक‚ल्पः । स‚ङ्केत‚काल‚दृष्ट‚त‚या च दृश्य‚मान‚व‚स्तुसंस्प‚र्शी विक‚ल्पः । स्फुटाभास‚त्वं‚{\tiny $_{lb}$}‚ नाम स‚न्निहित‚रूप‚भास‚न‚म् । अस‚न्निहित‚रूप‚भास‚नेन च व‚स्तुनः पूर्व‚दृष्ट‚त्व‚ग्र‚ह‚णं व्याप्त‚म् ।‚{\tiny $_{lb}$}‚ न हि पूर्व‚ज्ञान‚विष‚य‚त्वं पूर्व‚स्मिन् ज्ञाने निवृत्ते व‚स्तुनोऽस्ति । त‚दुत्त‚र‚काल‚भाविना ज्ञानेन पूर्व‚ज्ञान‚{\tiny $_{lb}$}‚विष‚य‚त्व‚म‚स‚न्निहित‚मेव व‚स्तुनः स्पृश्य‚ते । पूर्व‚ज्ञान‚विष‚य‚त्व‚ग्र‚ह‚ण‚मेव च स‚ङ्केत‚काल‚दृष्ट‚श‚ब्द‚{\tiny $_{lb}$}‚विशिष्ट‚त्व‚ग्र‚ह‚ण‚म् । त‚दिदं विरुद्धं\edtext{}{\lemma{विरुद्धं}\Bfootnote{द्ध}}व्याप्तोप‚ल‚ब्धेर्भ‚व‚तु निर्विक‚ल्प‚क‚म् ।
	\pend% ending standard par
      ‚{\tiny $_{lb}$}‚

	  \pstart \leavevmode% starting standard par
	अस‚त्य‚भ्रान्त‚त्वे तु क‚थं प्र‚त्य‚क्ष‚मित्याह--\textbf{प्र‚माणेति} । प्र‚माणाधिग‚तोऽर्थः \textbf{प्र‚माण‚शुद्ध}‚{\tiny $_{lb}$}‚ उच्य‚ते । स‚त्य‚च‚तुष्ट‚यं चैव‚मात्म‚क‚म् । त‚देव शुद्ध‚त्वेन विव‚क्षित‚म् । \textbf{चः} \leavevmode\ledsidenote{\textenglish{30b/ms}} \textbf{संवाद‚क}‚{\tiny $_{lb}$}‚मित्य‚तः प‚रो निर्विक‚ल्प‚त्वेन स‚ह संवाद‚क‚त्वेन संवाद‚क‚त्वं स‚मुच्चिनोति, त‚त्र‚स्थ ए\add{व}वा‚{\tiny $_{lb}$}‚ पूर्व‚हेत्व‚पेक्ष‚या हेत्व‚न्त‚र‚स‚मुच्च‚यार्थः । \textbf{अतो}ऽस्मान्निर्विक‚ल्प‚क‚त्वाद‚भ्रान्त‚त्वाच्च ।
	\pend% ending standard par
      ‚{\tiny $_{lb}$}‚

	  \pstart \leavevmode% starting standard par
	योगिश‚ब्द‚स्य व्युत्प‚त्तिमाह--\textbf{योग} इति । \textbf{स‚माधि}श्चित्तैकाग्र‚ता । इह \textbf{ध‚र्मोत्त‚रेण} लोक‚{\tiny $_{lb}$}‚प्र‚सिद्धिराश्रिता । \textbf{विनिश्च‚य‚टीकायां} तु शास्त्र‚स्थितिस्तेनाविरोधः । य‚द्वा \textbf{स‚माधि}ग्र‚ह‚ण‚स्योप‚{\tiny $_{lb}$}‚ल‚क्ष‚ण‚त्वात् प्र‚ज्ञा च विवेक‚क‚र‚ण‚श‚क्तिर्द्र‚ष्ट‚व्या । \textbf{स य‚स्यास्ति स} नित्य‚स‚माहितो विवेक‚क‚र‚ण‚{\tiny $_{lb}$}‚त‚त्प‚र‚श्च \textbf{योगी} । प‚रिस‚माप्तेराकारं द‚र्श‚य‚ति--\textbf{इय‚दिति । इय‚देव} च‚तुःसंख्याव‚च्छिन्न‚मेव ॥
	\pend% ending standard par
      ‚{\tiny $_{lb}$}‚

	  \pstart \leavevmode% starting standard par
	\textbf{त‚देव}मित्या\textbf{द्याहेत्ये}त‚द‚न्तं सुबोध‚म् । \textbf{विष‚यो बोद्ध‚व्यः} । क इत्याकाङ्क्षायामाह—‚{\tiny $_{lb}$}‚\textbf{स्व‚ल‚क्ष‚ण‚मिति} । स्व‚श‚ब्द‚स्य ल‚क्ष‚ण‚श‚ब्द‚स्य चार्थ‚माच‚क्षाणो विग्र‚ह‚मुप‚ल‚क्ष‚य‚ति--\textbf{स्व‚मि}त्यादिना ।‚{\tiny $_{lb}$}‚ \textbf{स्व‚मा}त्मीय‚मुच्य‚ते । य‚स्य य‚त् स्वं त‚त् त‚स्यैव नान्य‚स्येति ल‚क्ष‚ण‚या स्व‚श‚ब्देनासाधार‚ण‚मुक्त‚म् ।‚{\tiny $_{lb}$}‚ \textbf{ल‚क्ष‚ण}श‚ब्देन च \textbf{त‚त्त्वं} स्व‚रूपं विव‚क्षित‚म् । \textbf{स्व}श‚ब्द\textbf{ल‚क्ष‚ण}श‚ब्द‚योर‚र्थ‚म‚भिधाय त‚योः स‚म‚स्तं‚{\tiny $_{lb}$}‚ प‚द‚माह--\textbf{स्व‚ल‚क्ष‚ण‚मिति} । अनेन स्व‚म‚साधार‚णं च त‚ल्ल‚क्ष‚णं स्व‚रूपं चेति क‚र्म‚धार‚यो द‚र्शितः ।
	\pend% ending standard par
      ‚{\tiny $_{lb}$}‚‚{\tiny $_{lb}$}‚\textsuperscript{\textenglish{71/dm}}‚{\tiny $_{lb}$}‚
	  \bigskip
	  \begingroup
	

	  \pstart \leavevmode% starting standard par
	द्विविधो हि विष‚यः \edtext{}{\lemma{यः}\Bfootnote{हि प्र‚माण‚स्य विष‚यः \cite{dp-msA} \cite{dp-edP} \cite{dp-edH} \cite{dp-edN}}}प्र‚माण‚स्य--ग्राह्य‚श्च य‚दाकार‚मुत्य‚द्य‚ते, प्राप‚णीय‚श्च य‚म‚ध्य‚व‚स्य‚ति ।‚{\tiny $_{lb}$}‚ अन्यो हि ग्राह्योऽन्य‚श्चाध्य‚व‚सेयः । प्र‚त्य‚क्ष‚स्य हि क्ष‚ण एको ग्राह्यः । अध्य‚व‚सेय‚स्तु प्र‚त्य‚क्ष‚{\tiny $_{lb}$}‚ब‚लोत्प‚न्नेन निश्च‚येन संतान एव । स‚न्तान एव च प्र‚त्य‚क्ष‚स्य प्राप‚णीयः । क्ष‚ण‚स्य प्राप‚यितु‚{\tiny $_{lb}$}‚म‚श‚क्य‚त्वात् ।
	\pend% ending standard par
       ‚{\tiny $_{lb}$}‚ 

	  \pstart \leavevmode% starting standard par
	त‚थानुमान‚म‚पि \edtext{}{\lemma{पि}\Bfootnote{सामान्ये--\cite{dp-msD-n}}}स्व‚प्र‚तिभासेऽन‚र्थेऽर्थाव्य‚व‚सायेन\edtext{}{\lemma{सायेन}\Bfootnote{ऽन‚र्थेऽन‚र्थाध्य० \cite{dp-msA} \cite{dp-edP} \cite{dp-edH}}} प्र‚वृत्तेर‚न‚र्थ‚ग्राहि\edtext{}{\lemma{ग्राहि}\Bfootnote{सामान्य‚ग्राहि--\cite{dp-msD-n}}} ।
	\pend% ending standard par
      
	  \endgroup
	‚{\tiny $_{lb}$}‚

	  \pstart \leavevmode% starting standard par
	न‚नु स‚म्भ‚वे व्य‚भिचारे च विशेष‚ण‚म‚र्थ‚व‚त् । अत्र च स‚र्व‚स्यैव स्व‚रूप‚स्यासाधा‚{\tiny $_{lb}$}‚र‚ण‚त्वात् स‚म्भ‚व एव, न व्य‚भिचार इति किं स्व‚श‚ब्देन ? एवं तु व‚क्त‚व्य‚म्--व‚स्तुरूपं त‚स्य‚{\tiny $_{lb}$}‚ विष‚य इत्याश‚ङ्क‚याह--\textbf{व‚स्तुन} इति । हिर्य‚स्मात् । \textbf{चो} व‚क्ष्य‚माणापेक्ष्यः\add{क्षः} स‚मुच्च‚ये ।‚{\tiny $_{lb}$}‚ \textbf{सामान्यं} साधार‚णं रूपं संवृतिज्ञान‚घ‚टित‚म् । \textbf{चः} पूर्वापेक्षः स‚मुच्च‚ये । स‚ति चैवं द्वैरूप्ये किं‚{\tiny $_{lb}$}‚ पूर्वं रूप‚म्, अथ प‚रं प्र‚त्य‚क्ष‚स्य विष‚य इति स‚न्देहे--\textbf{य‚द‚साधार‚णं त‚त्प्र‚त्य‚क्ष‚स्य ग्राह्य}मुच्य‚त इति‚{\tiny $_{lb}$}‚ शेषः । ग्राह्य‚मिति ब्रुव‚न् विष‚य‚श‚ब्देनाचार्य‚स्य ग्राह्यो विष‚योऽभिप्रेत इति द‚र्श‚य‚ति ।
	\pend% ending standard par
      ‚{\tiny $_{lb}$}‚

	  \pstart \leavevmode% starting standard par
	स्यादेत‚त्--प्र‚वृत्तिविष‚य एव प्र‚माण‚स्य विष‚य‚स्त‚तः प्र‚वृत्तिविष‚य‚स्त‚स्य विष‚य इति‚{\tiny $_{lb}$}‚ व‚क्त‚व्य‚म् । त‚त्किं \textbf{ग्राह्य}मित्युच्य‚त इत्याह--\textbf{द्विविध} इति । अनेन भेदः प्र‚तिज्ञातः । \textbf{हि}र्य‚स्मात्‚{\tiny $_{lb}$}‚ द्विप्र‚कारो \textbf{विष‚यः प्र‚माण‚स्येति} । जातिविव‚क्ष‚यैक‚व‚च‚न‚म् । एको ग्राह्योऽन्यः प्राप‚णीयः । भेद‚मुप‚{\tiny $_{lb}$}‚पाद‚य‚ति--\textbf{य‚दाकारं} य‚त्प्र‚तिभासं ज्ञान‚मुत्प‚द्य‚ते । सोऽपि द्विविधः--प‚र‚मार्थ आरोपित‚श्च ।‚{\tiny $_{lb}$}‚ द्व‚योर‚पि स्व‚ज्ञाने प्र‚काश‚न‚म‚स्त्येवेति द्र‚ष्ट‚व्य‚म् । \textbf{प्राप‚णीयो य‚म‚ध्य‚व‚स्य‚ति} । त‚तो ज्ञानाद् य‚त्र‚{\tiny $_{lb}$}‚ प्र‚व‚र्त्त‚त इति याव‚त् । \textbf{च}कारौ पूर्वाप‚रापेक्ष‚या स‚मुच्च‚यार्थौ ।
	\pend% ending standard par
      ‚{\tiny $_{lb}$}‚

	  \pstart \leavevmode% starting standard par
	न‚नु ग्राह्याध्य‚व‚सेय‚श‚ब्द‚योरेव भेदो न त्व‚र्थ‚स्य । य‚तो य‚देव प्र‚काश‚ते त‚देवाध्य‚व‚सीय‚ते,‚{\tiny $_{lb}$}‚ त‚त् किंमेव‚मुच्य‚त इत्याह--\textbf{अन्यो ही}ति । हिर्य‚स्माद‚र्थे । \textbf{चो}ऽव‚धार‚णे । क‚स्य कीदृशो ग्राह्य‚{\tiny $_{lb}$}‚ इत‚रो वेत्याह--\textbf{प्र‚त्य‚क्ष‚स्ये}ति । \textbf{हि}र‚व‚धार‚णे एक इत्य‚स्मात्प‚रो द्र‚ष्ट‚व्यः । क‚स्त‚र्ह्य‚व‚सेय अध्य‚{\tiny $_{lb}$}‚व‚सेय इति \leavevmode\ledsidenote{\textenglish{31a/ms}} । तुर्ग्राह्याद‚ध्य‚व‚सेयं भिन‚त्ति । अथ किं प्र‚त्य‚क्ष‚म‚व‚सायात्म‚कं येन त‚स्या‚{\tiny $_{lb}$}‚साव‚व‚सेय इत्याह--\textbf{प्र‚त्य‚क्षे}ति । प्र‚त्य‚क्ष‚पृष्ठ‚भाविनो निश्च‚य‚स्य प्र‚त्य‚क्ष‚गृहीत एव प्र‚वृत्त‚{\tiny $_{lb}$}‚त‚याऽन‚तिश‚याधानेन य‚त् तेनाध्य‚व‚सितं त‚त्प्र‚त्य‚क्षेणैवाव‚सित‚मिति भावः ।
	\pend% ending standard par
      ‚{\tiny $_{lb}$}‚

	  \pstart \leavevmode% starting standard par
	न‚नु ग्राह्याद‚न्यः प्राप‚णीयो विष‚यः प्र‚त्य‚क्ष‚स्योक्तः । इदानीं पुन‚र‚व‚सेयः । त‚द‚यं तृतीयो‚{\tiny $_{lb}$}‚ विष‚यः प्राप्त इत्याह--\textbf{स‚न्तान} एवेति । \textbf{चो} य‚स्माद‚र्थे । उपादानोपादेय‚भावेन व्य‚व‚स्थितः‚{\tiny $_{lb}$}‚ क्ष‚ण‚प्र‚ब‚न्ध एक‚त्वेनाधिमुक्तः \textbf{स‚न्तानः} । ग्राह्य एव क‚स्मान्न तेन प्राप्य‚त इत्याह--\textbf{ल‚क्ष‚ण‚स्ये}ति‚{\tiny $_{lb}$}‚ \edtext{\textsuperscript{*}}{\lemma{*}\Bfootnote{इत्याह--\textbf{क्ष‚ण‚स्ये}ति}} । य‚दाकारं प्र‚त्य‚क्ष‚मुत्प‚द्य‚ते त‚स्य क्ष‚ण‚स्येति प्र‚स्तावात् । एव‚ञ्च‚{\tiny $_{lb}$}‚ विवृण्व‚तोऽस्य \textbf{प्र‚माण‚स्ये}त्य‚त्र प्र‚त्य‚क्षाभिप्रायेण प्र‚माण‚श‚ब्द इन्द्रिय‚ज‚प्र‚त्य‚क्ष‚विव‚क्ष‚या द्र‚ष्ट‚व्यः ।‚{\tiny $_{lb}$}‚ स्व‚संवेद‚नादीनां विष‚य‚द्वैविध्यास‚म्भ‚वादिति ।
	\pend% ending standard par
      ‚{\tiny $_{lb}$}‚

	  \pstart \leavevmode% starting standard par
	अथ भ‚व‚त्वेकं प्र‚त्य‚क्षं विष‚य‚द्वैविध्य‚व‚द्, अनुमानं तु क‚थं त‚थाविध‚म् ? \textbf{प्र‚माण‚स्ये}ति च‚{\tiny $_{lb}$}‚ ‚{\tiny $_{lb}$}‚ \leavevmode\ledsidenote{\textenglish{72/dm}}‚{\tiny $_{lb}$}‚ 
	  
	स पुन‚रारोपितोऽर्थो गृह्य‚माणः स्व‚ल‚क्ष‚ण‚त्वेनाव‚सीय‚ते\edtext{}{\lemma{ते}\Bfootnote{नाध्य‚व‚सी० \cite{dp-msC} \cite{dp-msD} \cite{dp-edN}}} य‚तः,\edtext{\textsuperscript{*}}{\lemma{*}\Bfootnote{०सीय‚ते त‚त‚श्च स्व० \cite{dp-msD}}} त‚तः स्व‚ल‚क्ष‚ण‚म‚व‚सितं\edtext{}{\lemma{सितं}\Bfootnote{०म‚ध्य‚व‚सित‚म् \cite{dp-msA} \cite{dp-msB} \cite{dp-msD} \cite{dp-edP} \cite{dp-edH} \cite{dp-edE} \cite{dp-edN}}}‚{\tiny $_{lb}$}‚ प्र‚वृत्तिविष‚योऽनुमान‚स्य\edtext{}{\lemma{स्य}\Bfootnote{०ष‚योऽस्यानुमा० \cite{dp-msC} \cite{dp-msD}}} । अन‚र्थ‚स्तु ग्राह्यः । त‚द‚त्र प्र‚माण‚स्य ग्राह्यं विष‚यं द‚र्श‚य‚ता‚{\tiny $_{lb}$}‚ प्र‚त्य‚क्ष‚स्य स्व‚ल‚क्ष‚णं विष‚य उक्तः ।‚{\tiny $_{lb}$}‚ ब्रुव‚ता त‚द‚पि विष‚य‚द्वैविध्य‚व‚द‚भ्युप‚ग‚त‚मेव इत्याश‚ङ्क्याह--\textbf{त‚थे}ति । येन प्र‚तिभासाध्य‚व‚साय‚{\tiny $_{lb}$}‚ल‚क्ष‚णेन प्र‚कारेण प्र‚त्य‚क्ष‚म‚न्य‚ग्राह्य‚न्याध्य‚व‚सायि त‚था तेन प्र‚कारेणेति \textbf{त‚था}श‚ब्दार्थः । न केव‚लं‚{\tiny $_{lb}$}‚ प्र‚त्य‚क्ष‚म‚न्य‚द् गृह्णाति, अन्य‚द‚ध्य‚व‚स्य‚ति, किन्त्व‚नुमान‚म‚प्य‚न्य‚ग्राह्य‚न्याध्य‚व‚सायीत्य‚पिश‚ब्देनाह ।‚{\tiny $_{lb}$}‚ इहैव‚च्छेदः क‚र्त्त‚व्योऽन्य‚था व्याख्यान‚म‚स‚म‚ञ्ज‚सं स्यात् । किं गृह्णातीत्याह--\textbf{अन‚र्थ‚ग्राही}ति ।‚{\tiny $_{lb}$}‚ अनुमान‚मिति प्र‚कृत‚त्वात् । क‚थ‚मुप‚प‚द्य‚त इत्याश‚ङ्क्योप‚प‚त्तिमाह--\textbf{स्व‚प्र‚तिभास} इति । स्व‚स्य‚{\tiny $_{lb}$}‚ प्र‚तिभास इव प्र‚तिभासः । श‚क्तिद्व‚य‚योगात्त‚थारोप्य‚माणं रूप‚म् । त‚स्मिन्न‚न‚र्थेऽबाह्य‚रूप्ये\edtext{}{\lemma{रूप्ये}\Bfootnote{रूपे}}‚{\tiny $_{lb}$}‚\textbf{ऽर्थाध्य‚व‚सायेन} बाह्याध्य‚व‚सायेन त‚द्भेदान‚व‚भास‚नात्म‚काभेदाध्य‚व‚सान‚ल‚क्ष‚णेन \textbf{प्र‚वृत्तेः} प्र‚व‚र्त्त‚नात् ।
	\pend% ending standard par
      ‚{\tiny $_{lb}$}‚

	  \pstart \leavevmode% starting standard par
	अथान‚र्थे स्व‚प्र‚तिभासेऽर्थाध्य‚व‚सायेनानुमान‚विक‚ल्पोऽन्यो वा प्र‚व‚र्त्त‚क इति किमुक्तं‚{\tiny $_{lb}$}‚ भ‚व‚ति ? स्व‚प्र‚तिभास‚स्यारोप्य‚माण‚स्य चार्थ‚स्याव‚सीय‚मान‚स्य विवेकं न प्र‚तिप‚द्य‚त इत्युक्तं‚{\tiny $_{lb}$}‚ भ‚व‚ति । न हि तेनैव विवेक‚प्र‚तिप‚त्तिः, तेन स्व‚प्र‚तिभास‚स्य त‚थात्वेनाविक‚ल्प‚नात् । अत्र‚{\tiny $_{lb}$}‚ चानुभ‚वः प्र‚माण‚म् । बाह्य‚व‚ह‚न्यादाव‚प्र‚वृत्त्यादिप्र‚स‚ङ्ग‚श्च । नापि विक‚ल्पान्त‚रेण । त‚स्यापि‚{\tiny $_{lb}$}‚ त‚था प्र‚वृत्तित‚या पूर्व‚विक‚ल्प‚प्र‚तिभासासंस्प‚र्शादित्य‚ल‚मिह विस्त‚रेण । य‚द्येवं कोऽस्या‚{\tiny $_{lb}$}‚ध्य‚व‚सीय‚मानः प्राप‚णीयो विष‚य इत्याह--\textbf{स} इति । \textbf{पुन}रिति स‚र्व‚तो विशिन‚ष्टि । \textbf{आरोपित}‚{\tiny $_{lb}$}‚ इति स्व‚रूपानुवादः । \textbf{अर्थ} इति अर्थ इवार्थो गृह्य‚माणः प्र‚तीय‚मान‚स्त‚स्मिन् ज्ञाने प्र‚काश‚मान‚{\tiny $_{lb}$}‚ इति याव‚त् । अनेन स्व‚ल‚क्ष‚ण‚म‚ध्य‚व‚स्य‚तीद‚मिति प्र‚काशित‚म् । त‚द‚नेनानुमान‚स्य ग्राह्याध्य‚{\tiny $_{lb}$}‚व‚सेयौ भेदेन साम‚र्थ्याद‚भिधाय सुख‚प्र‚तिप‚त्त्य‚र्थं तावेव क‚ण्ठोक्तौ क‚रोति--\textbf{त‚त} इति । य‚तः‚{\tiny $_{lb}$}‚ स्व‚ल‚क्ष‚ण‚त्वेन त‚स्याव‚साय\textbf{स्त‚तः}\leavevmode\ledsidenote{\textenglish{31b/ms}}स्व‚ल‚क्ष‚ण‚म‚व‚सितं प्र‚वृत्तिविष‚योऽनुमान‚स्य । \textbf{स्व‚ल‚क्ष‚ण‚{\tiny $_{lb}$}‚म‚व‚सित‚मि}त्येत‚द‚प्य‚भिमानाद‚भिधीय‚ते । न पुनः स्व‚ल‚क्ष‚ण‚म‚व‚साय‚स्य गोच‚रः । त‚द्विष‚य‚त्वे‚{\tiny $_{lb}$}‚ त‚स्य निरंश‚त्वात् क्ष‚णिक‚त्वादेर‚प्य‚व‚सिताव‚नुमानान‚व‚तार‚प्र‚स‚ङ्गात् । एत‚च्च किञ्चिदिहैवो‚{\tiny $_{lb}$}‚प‚रिष्टात्स्प‚ष्ट‚यिष्यामः । प‚रार्थानुमाने तु य‚थाऽव‚स‚रं विस्त‚रेण निर्णेष्यामः ।
	\pend% ending standard par
      ‚{\tiny $_{lb}$}‚

	  \pstart \leavevmode% starting standard par
	प्र‚वृत्तिविष‚य‚स्यैव प्राप‚णीय‚त्वात् प्राप‚णीय एव प्र‚वृत्तिविष‚य‚श‚ब्देनोक्तः । \textbf{तुः} प्राप‚णीयाद्‚{\tiny $_{lb}$}‚ विष‚याद् ग्राह्यं भेद‚व‚न्तं द‚र्श‚य‚ति । त‚स्मिन् ज्ञाने प्र‚काश‚माने प्र‚काश‚नाद‚न‚र्थो ग्राह्य उक्तः ।
	\pend% ending standard par
      ‚{\tiny $_{lb}$}‚

	  \pstart \leavevmode% starting standard par
	स्यादेत‚त्--य‚द्य‚व‚सितं स्व‚ल‚क्ष‚णं प्र‚वृत्तिविष‚योऽनुमान‚स्य त‚दा क‚थ‚मेष \textbf{ध‚र्मोत्त‚रो विनि‚{\tiny $_{lb}$}‚श्च‚य‚टीकाया}म‚वादीत् अव‚सित‚श्चाकारो विक‚ल्पानां ग्राह्यः इति । एवं हि स्व‚ल‚क्ष‚ण‚मेव ग्राह्यं‚{\tiny $_{lb}$}‚ स‚म‚र्थितं स्यान्नान‚र्थ इति । त‚तो व्य‚क्तो व्याघातः । प‚र‚मार्थ‚दृष्ट्या त‚त्र त‚थाभिधानान्न दोषः ।‚{\tiny $_{lb}$}‚ त‚थाहि प‚र‚मार्थ‚तो विक‚ल्पानामारोपित‚मेव रूप‚म‚व‚सेयं ग्राह्य‚म् त‚थापि व्य‚व‚ह‚र्त्तार आरोपित‚मेव‚{\tiny $_{lb}$}‚ रूप‚म‚वृक्ष‚व्यावृत्त‚म‚व‚स्य‚न्तो बाह्य‚स्यापि त‚थात्वाद् बाह्यं वृक्ष‚म‚व‚स्याम इत्य‚भिम‚न्य‚न्ते, त‚योर्विवे‚{\tiny $_{lb}$}‚‚{\tiny $_{lb}$}‚ ‚{\tiny $_{lb}$}‚ \leavevmode\ledsidenote{\textenglish{73/dm}}‚{\tiny $_{lb}$}‚ काप्र‚तिप‚त्तेः । त‚द‚नुरोधात्स्व‚ल‚क्ष‚ण‚म‚व‚सित‚मित्युच्य‚ते । त‚स्यैव त्वारोपित‚स्य प‚र‚मार्थ‚तोऽनुमेय‚स्य‚{\tiny $_{lb}$}‚ प्र‚तिभास‚मानं केव‚ल‚माकार‚माश्रित्य ग्राह्य‚त्व‚मुच्य‚ते । त‚त‚स्त‚त्र वास्त‚वाव‚सित‚रूपाभिप्रायेण‚{\tiny $_{lb}$}‚ त‚थाभिधानात् को विरोधः ?
	\pend% ending standard par
      ‚{\tiny $_{lb}$}‚

	  \pstart \leavevmode% starting standard par
	त‚द‚त्रेत्यादिनोप‚संह‚र‚ति । य‚स्मात्प्र‚माण‚स्य द्विविधो विष‚यो ग्राह्याध्य‚व‚साय‚भेदेन‚{\tiny $_{lb}$}‚ त‚त् त‚स्मात् । \textbf{अत्र} विष‚य‚विप्र‚तिप‚त्तिनिराक‚र‚ण‚काले \textbf{ग्राह्यं विष‚यं द‚र्श‚य‚ते}ति ब्रुव‚न् विष‚य‚श‚ब्देना‚{\tiny $_{lb}$}‚चार्य‚स्य ग्राह्मो विष‚योऽभिप्रेत इति स्फुट‚य‚ति ।
	\pend% ending standard par
      ‚{\tiny $_{lb}$}‚

	  \pstart \leavevmode% starting standard par
	स्यादेत‚त्--न वै ग्राह्य‚विष‚यापेक्षं प्र‚त्य‚क्ष‚स्य प्रामाण्य‚मुप‚प‚द्य‚ते । य‚त एकः क्ष‚ण‚स्त‚स्य‚{\tiny $_{lb}$}‚ ग्राह्यः । न च त‚त्प्राप‚णं स‚म्भ‚व‚ति । नाप्य‚नुमान‚स्य ग्राह्यापेक्षं प्रामाण्य‚म् । अन‚र्थो हि‚{\tiny $_{lb}$}‚ त‚स्य ग्राह्यः । न च त‚त्प्राप्तिः स‚म्भ‚विनी । न च येन य‚स्याप्राप‚णं त‚स्य त‚त्र प्रामाण्य‚म‚भ्युपेय‚ते,‚{\tiny $_{lb}$}‚ अतिप्र‚स‚ङ्गाप‚त्तेः । अथैक‚नील‚क्ष‚णाकार‚त‚योत्प‚त्तिरेव प्र‚त्य‚क्ष‚स्य त‚त्प्राप‚ण‚म्, अनुमान‚स्यापि‚{\tiny $_{lb}$}‚ व‚ह्न्याद्य‚ध्य‚व‚सायित‚या त‚थोत्प‚त्तिरेव त‚त्प्राप‚ण‚मुच्य‚ते । त‚र्हि न किञ्चिद‚व‚शेषितं स्यात् । क‚स्य‚{\tiny $_{lb}$}‚ नाम ज्ञान‚स्य स‚विक‚ल्प‚स्य निर्विक‚ल्प‚क‚स्य वा त‚त्त‚दाकार‚त‚योत्प‚त्तिर्नास्ति येन त‚स्य प्रामाण्यं न‚{\tiny $_{lb}$}‚ स्यात् । त‚स्मान्न द्विविधो विष‚यः प्र‚माण‚स्याभिधानीयोऽपि त्वेक एव प्र‚वृत्तिविष‚याख्यो विष‚यः‚{\tiny $_{lb}$}‚ ख्याप‚नीय इति ।
	\pend% ending standard par
      ‚{\tiny $_{lb}$}‚

	  \pstart \leavevmode% starting standard par
	अत्र च स‚माधीय‚ते । ज्ञानानां ताव‚द् ग्राह्याध्य‚व‚सा\edtext{}{\lemma{सा}\Bfootnote{से}}य‚भेदेन द्विविधो विष‚यो‚{\tiny $_{lb}$}‚ऽव‚श्यैषित‚व्योऽनुभ‚व‚सिद्ध‚त्वात् । त‚त्र प्रामाण्यं प्र‚वृत्तिविष‚यापेक्षं व्य‚व‚स्थाप्य‚ते । ज्ञान‚त्वं‚{\tiny $_{lb}$}‚ तूभ‚यापेक्ष‚मेव । अज्ञान‚स्य च प्रामाण्यास‚म्भ‚वेन ज्ञानान्त‚र्भूतं प्र‚माणं विष‚य‚द्वैविध्य‚व‚देव भ‚व‚ति ।‚{\tiny $_{lb}$}‚ केव‚लं न ग्राह्यापे\leavevmode\ledsidenote{\textenglish{32a/ms}}क्षं प्रामाण्य‚म‚पि तु प्र‚वृत्तिविष‚यापेक्ष‚मिति प्र‚तिपाद्य‚ते । ग्राह्यापेक्ष‚या‚{\tiny $_{lb}$}‚ तु प्र‚माण‚स्येति व‚च‚नं स्व‚रूपानुवाद‚क‚म् । य‚द्वा प्र‚माण‚श‚ब्देन ज्ञान‚मेवात्र विव‚क्षित‚म् । य‚थाऽयं‚{\tiny $_{lb}$}‚ \textbf{विनिश्च‚य‚टीकायां} स्वार्थानुमान‚व्याख्यानाव‚स‚रे व्य‚क्त‚माह--द्विविधो ज्ञानानां विष‚यो‚{\tiny $_{lb}$}‚ ग्राह्य‚श्चाध्य‚व‚सेय‚श्च इत्यादि । त‚तो न किञ्चिद‚व‚द्य‚म् ।
	\pend% ending standard par
      ‚{\tiny $_{lb}$}‚

	  \pstart \leavevmode% starting standard par
	अथापि स्यात् । य‚दि ग्राह्यो विष‚यो न क‚दाचिद‚पि प्र‚त्य‚क्ष‚स्य प्राप‚णीय‚स्त‚र्हि तेन द‚र्शितेन‚{\tiny $_{lb}$}‚ किम्प्र‚योज‚नं येनोच्य‚ते \textbf{त‚द‚त्र ग्राह्यं विष‚यं द‚र्श‚य‚ता प्र‚त्य‚क्ष‚स्य स्व‚ल‚क्ष‚णं विष‚य उक्त} इति ।‚{\tiny $_{lb}$}‚ नैष दोषः । एव‚मेवास्य प्र‚वृत्तिविष‚य‚प्र‚द‚र्श‚नात् । य‚दि हि प्र‚त्य‚क्षं क्ष‚ण‚मेकं गृह्णाति त‚त्रानील‚व्या‚{\tiny $_{lb}$}‚वृत्तिनिश्च‚योप‚ज‚ने स‚न्तानं निश्चाय‚य‚त् प्र‚वृत्तिविष‚यं प्र‚द‚र्श‚येत् त‚त्स‚न्तान‚भाविनाञ्च ज्ञानानाम‚{\tiny $_{lb}$}‚प्रामाण्य‚मापाद‚येत्, न तु किञ्चिद‚गृह्ण‚त् । त‚था हि स‚न्तानोल्लेखेन निश्च‚याभावेऽप्य‚नील‚व्या‚{\tiny $_{lb}$}‚वृत्तिस्ताव‚द‚नाग‚त‚स‚र्व‚क्ष‚ण‚साधार‚णी प्र‚त्य‚क्षेण गृहीत‚निश्चिता । त‚न्निश्च‚य एव स‚र्वानील‚व्यावृत्तो‚{\tiny $_{lb}$}‚पादानोपादेय‚भाव‚स्थित‚नील‚क्ष‚ण‚निश्च‚यः, त‚द‚भिन्न‚योग‚क्षेम‚त्वादुत्त‚र‚प्र‚ब‚न्ध‚स्य । त‚थानिश्च‚य एव च‚{\tiny $_{lb}$}‚ स‚न्ताने निश्च‚य उच्य‚ते । अत एव च त‚दाद्यं ज्ञानं त‚थाऽनुष्ठानं त‚त्स‚न्तान‚भावीनि प‚राञ्चि‚{\tiny $_{lb}$}‚ ज्ञानान्य‚नील‚व्यावृत्तं रूपं गृहीत‚मेव गृह्ण‚न्त्य‚नील‚व्यावृत्तिनिश्च‚यं च कृत‚मेव कुर्व‚न्त्य‚न‚तिश‚याधायीनि‚{\tiny $_{lb}$}‚ प्र‚चुराण्य‚पि प्रामाण्यात्प्र‚च्याव‚य‚त् प्रामाण्य‚मात्म‚सात्क‚रोतीति न्याय्यं ग्राह्य‚प्र‚द‚र्श‚न‚मिति स‚र्व‚{\tiny $_{lb}$}‚म‚व‚दात‚म् ॥
	\pend% ending standard par
      ‚{\tiny $_{lb}$}‚

	  \pstart \leavevmode% starting standard par
	इहाचार्य‚स्य--स्व‚म‚साधार‚णं स‚न्तानान्त‚र‚साधार‚णं य‚न्न भ‚व‚ति, य‚द‚र्थ‚क्रियाक्ष‚म‚मेव स‚र्व‚तो‚{\tiny $_{lb}$}‚ व्यावृत्तं त‚त्त्वं त‚देव स्व‚ल‚क्ष‚ण‚म्, न चैत‚द्विप‚रीत‚म‚नुमा विष‚योऽपीत्य‚भिप्रेत‚म् । \textbf{स्व‚म‚साधार‚णं} \leavevmode\ledsidenote{\textenglish{74/dm}}‚{\tiny $_{lb}$}‚ 
	  
	कः पुन‚र‚सौ विष‚यो ज्ञान‚स्य यः स्व‚ल‚क्ष‚णं प्र‚तिप‚त्त‚व्य इत्याह-- ‚{\tiny $_{lb}$}‚ 
	  
	य‚स्यार्थ‚स्य संनिधानासंनिधानाभ्यां ज्ञान‚प्र‚तिभास‚भेद‚स्त‚त् स्व‚ल‚क्ष‚ण‚म् ॥ १३ ॥‚{\tiny $_{lb}$}‚ 
	  
	य‚स्यार्थ‚स्येत्यादि । अर्थ‚श‚ब्दो विष‚य‚प‚र्यायः । य‚स्य ज्ञान‚विष‚य‚स्य । संनिधानं‚{\tiny $_{lb}$}‚ निक‚ट‚देशाव‚स्थान‚म् । असंनिधानं दूर‚देशाव‚स्थान‚म् । त‚स्मात् संनिधानाद‚संनिधानाच्च‚{\tiny $_{lb}$}‚ ज्ञान‚प्र‚तिभास‚स्य ग्राह्याकार‚स्य भेदः स्फुट‚त्वास्फुट‚त्वाभ्याम् । यो हि ज्ञान\edtext{}{\lemma{ज्ञान}\Bfootnote{ज्ञान‚स्य विष‚यः \cite{dp-msA} \cite{dp-edP} \cite{dp-edH} \cite{dp-edE} \cite{dp-edN}}}विष‚यः संनिहितः‚{\tiny $_{lb}$}‚ स‚न्\edtext{}{\lemma{न्}\Bfootnote{स स्फुटा० \cite{dp-msB} \cite{dp-msC} \cite{dp-msD}}} स्फुटाभासं ज्ञान‚स्य क‚रोति, असंनिहित‚स्तु योग्य‚देश‚स्थ\edtext{}{\lemma{स्थ}\Bfootnote{देशाव‚स्थित \cite{dp-msA} \cite{dp-msB} \cite{dp-msC} \cite{dp-edP} \cite{dp-edH} \cite{dp-edE} \cite{dp-edN}}} एवास्फुटं क‚रोति, त‚त्‚{\tiny $_{lb}$}‚ स्व‚ल‚क्ष‚ण‚म् । स‚र्वाण्येव हि व‚स्तूनि दूराद‚स्फुटानि दृश्य‚न्ते, स‚मीपे स्फुटानि । तान्येव\edtext{}{\lemma{तान्येव}\Bfootnote{तान्येव हि स्व० \cite{dp-msC}}}‚{\tiny $_{lb}$}‚ स्व‚ल‚क्ष‚णानि ॥‚{\tiny $_{lb}$}‚ \textbf{ल‚क्ष‚णं त‚त्त्वं स्व‚ल‚क्ष‚ण‚मि}ति विवृण्व‚ता च \textbf{ध‚र्मोत्त‚रेण} त‚देव द‚र्शित‚म् । केव‚लं न व्य‚क्तीकृत‚{\tiny $_{lb}$}‚म‚त‚स्त‚स्य \textbf{विष‚यः स्व‚ल‚क्ष‚ण‚मि}ति श‚ब्द‚मात्र‚श्राविणोऽविदिताचार्याभिप्राय‚स्याविभावित‚ध‚र्मोत्त‚र‚वि‚{\tiny $_{lb}$}‚व‚र‚णार्थ‚स्य य‚द्य‚साधार‚णं प‚र‚रूपेणामिश्रं रूपं स्व‚ल‚क्ष‚णं त‚दाऽग्नित्व‚म‚पि गोत्वादिसामान्येनामिश्रं‚{\tiny $_{lb}$}‚ स्व‚ल‚क्ष‚णं प्र‚स‚ज्येतेति व्याप्तिं म‚न्य‚मान‚स्यायं \textbf{कः पुन}रित्यादिप्र‚श्नः । \textbf{क} इत्य‚नेन सामान्याकारेण‚{\tiny $_{lb}$}‚ विष‚यं पृच्छ‚ति । \textbf{पुन‚रि}त्य‚नेन विशेषाकारेण । स्व‚ल‚क्ष‚ण‚श‚ब्द‚स्यास‚ति ब‚हुव्रीहाव‚ज‚ह‚ल्लिङ्ग‚त्वात्‚{\tiny $_{lb}$}‚\textbf{स्व‚ल‚क्ष‚ण‚मि}ति ।
	\pend% ending standard par
      ‚{\tiny $_{lb}$}‚

	  \pstart \leavevmode% starting standard par
	चोद‚केन विष‚यं स‚म्पृष्ट आचार्यः क‚स्माद‚र्थ‚मुत्त‚रीक‚रोतीत्याश‚ङ्क्याह--\textbf{अर्थेति} । ज्ञान‚स्य‚{\tiny $_{lb}$}‚ प्र‚त्यास‚न्न‚त्वाज्ज्ञान‚विष‚यो ल‚ब्ध इत्य‚भिप्रेत्य \textbf{य‚स्य ज्ञान‚विष‚य}स्येति विवृणोति । \textbf{स‚न्निधाना‚{\tiny $_{lb}$}‚स‚न्निधान}श‚ब्दौ निक‚ट‚दूराव‚स्थानार्थौ व्याच‚क्षाणो य‚द‚न्यैराख्यातं--अस‚न्निधानं योग्य‚देशे स‚र्व‚था‚{\tiny $_{lb}$}‚ व‚स्तुनोऽभावः इति त‚द‚पाक‚रोति । अय‚ञ्चास्याभिप्रायो य‚त्र व‚स्तु नास्ति त‚त्र ज्ञान‚मेव न‚{\tiny $_{lb}$}‚ जाय‚ते । न तु त‚त्प्र‚तिभास‚स्य भेदो नानात्व‚म‚स्तीति । \textbf{त‚स्मादि}त्यादिना \textbf{स}\leavevmode\ledsidenote{\textenglish{32b/ms}}‚{\tiny $_{lb}$}‚\textbf{न्निधानास‚न्निधानाभ्यामिति} प‚ञ्च‚मीद्विव‚च‚नान्त‚मेत‚दिति द‚र्श‚य‚ति । \textbf{स्फुट‚त्वास्फुट‚त्वाभ्यामिति‚{\tiny $_{lb}$}‚ अर्थ‚क्रिया}साम‚र्थ्यानुग‚ताभ्यामिति द्र‚ष्ट‚व्य‚म् । \edtext{\textsuperscript{*}}{\lemma{*}\Bfootnote{पाणिनि २. ३. २१.}}इत्थ‚म्भूत‚ल‚क्ष‚णा चेयं तृतीया ।
	\pend% ending standard par
      ‚{\tiny $_{lb}$}‚

	  \pstart \leavevmode% starting standard par
	प‚दार्थं व्याख्याय स‚मुदायार्थं व्याच‚ष्टे । द्व‚यी चेयं शैली व्याख्यातृणाम् । क्व‚चित्स‚{\tiny $_{lb}$}‚मुदायार्थ व्याख्याय प‚श्चात्प‚दार्थ‚म् विवृण्व‚ते । क्व‚चित्प‚दार्थं विवृत्य प‚श्चात्स‚मुदायार्थं व्याच‚क्ष‚त‚{\tiny $_{lb}$}‚ इति । \textbf{हि}र‚व‚धार‚णे । \textbf{योग्य‚देश‚स्थ एव}त्य‚नेन दूर‚देश‚स्थं निर‚स्य‚ति । दूर‚स्थो हि ज्ञान‚मेव‚{\tiny $_{lb}$}‚ \textbf{न ज‚न‚य}ति किम‚ङ्ग पुन‚र्भिन्द्यादित्य‚भिप्रायः ।
	\pend% ending standard par
      ‚{\tiny $_{lb}$}‚

	  \pstart \leavevmode% starting standard par
	स्व‚ल‚क्ष‚ण‚ल‚क्ष‚ण‚स्याव्यापित्वास‚म्भ‚वित्व‚श‚ङ्काम‚पाकुर्व‚न्नाह--स‚र्वा\textbf{णीति} । । \textbf{स‚र्व}श‚ब्दो‚{\tiny $_{lb}$}‚ व्याप्तिप्र‚द‚र्श‚नार्थः । अनेनाव्यापित्वं निराकृत‚म् । \textbf{हि}र्य‚स्माद‚र्थे । \textbf{दृश्य‚न्ते} प्र‚तीय‚न्ते ।‚{\tiny $_{lb}$}‚ अनेनास‚म्भ‚वित्वं निर‚स्त‚म् । अतिव्यापित्व‚म‚प‚ह‚स्त‚य‚न्नाह--\textbf{तान्येवेति} । यान्य‚मूनि प्र‚त्य‚क्ष‚{\tiny $_{lb}$}‚विष‚य‚स्तान्येव । नानुमान‚विष‚योऽपीति प्र‚क‚र‚णात् । \textbf{दूराद‚स्फुटानि दृश्य‚न्त} इति व‚द‚त‚श्चाय‚{\tiny $_{lb}$}‚माश‚यः--दूरे हि व‚स्तु गृह्य‚माणं प्र‚चुर‚र‚जोनीहारादिसंसृष्टं गृह्य‚ते । त‚तोऽस्प‚ष्टं गृह्य‚ते । न तु‚{\tiny $_{lb}$}‚ ‚{\tiny $_{lb}$}‚ \leavevmode\ledsidenote{\textenglish{75/dm}}‚{\tiny $_{lb}$}‚ 
	  
	क‚स्मात् पुनः प्र‚त्य‚क्ष‚विष‚य एव स्व‚ल‚क्ष‚ण‚म् ? त‚था हि विक‚ल्प‚विष‚योऽपि व‚ह्निर्दृश्यात्म‚क\edtext{}{\lemma{क}\Bfootnote{०क इवाव \cite{dp-msD}}}‚{\tiny $_{lb}$}‚ एवाव‚सीय‚त इत्याह-- ‚{\tiny $_{lb}$}‚ 
	  
	त‚देव प‚र‚मार्थ‚स‚त् ॥ १४ ॥‚{\tiny $_{lb}$}‚ 
	  
	त‚देव प‚र‚मार्थ‚स‚दिति । \edtext{\textsuperscript{*}}{\lemma{*}\Bfootnote{प‚र‚मार्थोऽकृ० \cite{dp-msA} \cite{dp-msB} \cite{dp-msC} \cite{dp-edP} \cite{dp-edH} \cite{dp-edE} \cite{dp-edN}}}प‚र‚मोऽर्थोऽकृत्रिम‚म‚नारोपितं रूप‚म् । तेनास्तीति प‚र‚मा‚{\tiny $_{lb}$}‚र्थ‚स‚त् । य एवार्थः स‚न्निधानास‚न्निधानाभ्यां स्फुट‚म‚स्फुटं च प्र‚तिभासं क‚रोति प‚र‚मार्थ‚स‚न्‚{\tiny $_{lb}$}‚ स एव । स\edtext{}{\lemma{स}\Bfootnote{स एव च \cite{dp-msB} \cite{dp-edH} \cite{dp-edE} \cite{dp-edN}}} च प्र‚त्य‚क्ष‚स्य\edtext{}{\lemma{स्य}\Bfootnote{प्र‚त्य‚क्ष‚विष‚यो \cite{dp-msA} \cite{dp-msB} \cite{dp-msC} \cite{dp-msD} \cite{dp-edP} \cite{dp-edH} \cite{dp-edE} \cite{dp-edN}}} विष‚यो य‚तः, त‚स्मात् त‚देव स्व‚ल‚क्ष‚ण‚म् ॥‚{\tiny $_{lb}$}‚ न गृह्य‚त एव । त‚थागृहीत‚स्यापि वृक्ष‚स्य च्छायाद्य‚र्थ‚क्रियाकारित्वात् । न चाधिक‚ग्र‚ह‚णं‚{\tiny $_{lb}$}‚ भ्र‚म इति ।
	\pend% ending standard par
      ‚{\tiny $_{lb}$}‚

	  \pstart \leavevmode% starting standard par
	इदं पुन‚र‚त्र निरूप्य‚ते । य‚दि य एवार्थः स‚न्निधानाद‚स‚न्निधानाच्च ज्ञान‚प्र‚तिभासं‚{\tiny $_{lb}$}‚ स्फुट‚त्वास्फुट‚त्वाभ्यां भिन‚त्ति स एव स्व‚ल‚क्ष‚णं त‚र्हि स्प‚र्श‚र‚सौ स्व‚ल‚क्ष‚णे न स्याताम् ।‚{\tiny $_{lb}$}‚ तौ ख‚लु अस‚न्निहितौ ज्ञान‚मेव न ज‚न‚य‚तः । किं पुन‚र्ज्ञान‚प्र‚तिभासं स्फुट‚त्वास्फुट‚{\tiny $_{lb}$}‚त्वाभ्यां भेत्स्य‚तः ? किञ्चैत‚स्मिन् स्व‚ल‚क्ष‚ण‚ल‚क्ष‚णे विज्ञान‚म‚स्व‚ल‚क्ष‚णं स्यात् । त‚स्याऽऽ‚{\tiny $_{lb}$}‚स्तां ताव‚द‚स‚न्निहित‚स्यास्फुट‚ज्ञान‚ज‚न‚क‚त्वं स‚न्निहित‚स्यापि स्फुट‚ज्ञान‚ज‚न‚क‚त्वं नास्ति । न च त‚स्य‚{\tiny $_{lb}$}‚ दूरान्तिक‚व‚र्त्तित्व‚म‚स्त्य‚देश‚त्वात् । त‚दीदृशीं म‚ह‚तीम‚व्यापिताम‚नालोच्येदृशं स्व‚ल‚क्ष‚णं प्र‚ण‚य‚न्ना‚{\tiny $_{lb}$}‚\textbf{चार्यः, ध‚र्मोत्त‚रो}ऽप्येवं प्र‚स‚भं व्याच‚क्षाणः क‚थं न प्र‚माद्य‚तीति ? न । अभिप्रायाप‚रिज्ञानात् ।‚{\tiny $_{lb}$}‚ य‚दि हीदं ल‚क्ष‚णं य‚थाश्रुति व्य‚व‚तिष्ठेत, स्यादेवैत‚त् । केव‚लं स‚न्निधानास‚न्निधानाभ्यां ज्ञान‚प्र‚ति‚{\tiny $_{lb}$}‚भास‚भेद‚क‚त्वेन य‚देकार्थ‚स‚म‚वेत‚म‚साधार‚ण्यं व्य‚क्त्य‚न्त‚रास‚नुयायित्वं त‚दुप‚ल‚क्षित‚म् । य‚तो‚{\tiny $_{lb}$}‚ \textbf{हेतुबिन्दुः} त‚त्र त‚दाद्य‚म‚साधार‚ण‚विष‚य‚म् इति \href{http://sarit.indology.info/?cref=hbṭ.1.28}{पृ० ५३} । अत एव--असाधार‚ण‚विष‚यं‚{\tiny $_{lb}$}‚ स्व‚ल‚क्ष‚ण‚विष‚य‚म् इति \href{http://sarit.indology.info/?cref=hbṭ.1.14}{हेतु० टी० पृ० २५} \textbf{भ‚ट्टार्च‚टो} व्याच‚ष्टे । अत एवानुमान‚स्यैत‚द्विप‚र्य‚येण‚{\tiny $_{lb}$}‚ साधार‚णं रूप विष‚यो द‚र्श‚यिष्य‚ते । तेन नाव्याप्तिर्न चान्यो ल‚क्ष‚ण‚दोषः । \textbf{त‚स्य विष‚यः‚{\tiny $_{lb}$}‚ स्व‚ल‚क्ष‚ण‚मि}त्य‚भिहितेऽपि स्व‚ल‚क्ष‚ण‚श‚ब्द‚स्यान्य‚थापि निर्व‚च‚न‚स‚म्भ‚वात्, नाय‚म‚भिम‚तोऽर्थो ज्ञायेत‚{\tiny $_{lb}$}‚ प्र‚तिप‚त्तृभिरिति त‚द‚भिम‚तार्थ इत्थ‚मुप‚ल‚क्षित इति च द्र‚ष्ट‚व्य‚म् ॥
	\pend% ending standard par
      ‚{\tiny $_{lb}$}‚

	  \pstart \leavevmode% starting standard par
	\textbf{त‚स्य विष‚यः स्व‚ल‚क्ष‚ण}मित्य‚त्र य‚दि त‚स्यैव विष‚यः स्व‚ल‚क्ष‚ण‚मित्य‚व‚धार्य‚ते त‚दा य‚त‚{\tiny $_{lb}$}‚ एव‚कार‚क‚र‚णं त‚तोऽन्य‚त्राव‚धार‚ण‚मिति स्व‚ल‚क्ष‚ण‚स्य प्र‚त्य‚क्षे निय‚मात् प्र‚त्य‚क्ष‚म‚न्य‚विष‚य‚म‚पि‚{\tiny $_{lb}$}‚ स्यात् । अथ स्व‚ल‚क्ष‚ण‚मेवेत्य‚व‚धार्य‚ते । त‚दाऽपि प्र‚त्य‚क्ष‚स्य स्व‚ल‚क्ष‚णे निय‚त‚त्वाद् \leavevmode\ledsidenote{\textenglish{33a/ms}}‚{\tiny $_{lb}$}‚ अनिय‚तं स्व‚ल‚क्ष‚ण‚म‚नुमान‚स्यापि विष‚यः स्यादित्युभ‚याव‚धार‚णं कार्य‚म् । त‚स्यैव विष‚यः स्व‚ल‚क्ष‚ण‚{\tiny $_{lb}$}‚मेवेति । एत‚द‚स‚ह‚मानः पूर्व‚प‚क्ष‚वाद्याह--\textbf{क‚स्मादिति । क‚स्मादिति} सामान्येन कार‚णं पृच्छ‚ति ।‚{\tiny $_{lb}$}‚ \textbf{पुन‚रिति} विशेष‚तः । य‚द्वा \textbf{य‚स्यार्थ‚स्ये}त्य‚त्र त‚त्स्व‚ल‚क्ष‚ण‚मिति । \textbf{त‚देव} प्र‚त्य‚क्ष‚विष‚यः स्व‚ल‚क्ष‚णं‚{\tiny $_{lb}$}‚ नानुमान‚विष‚य इत्य‚भिप्रेतं त‚द‚स‚ह‚मान एव‚माह । \textbf{क‚स्मात्प्र‚त्य‚क्ष‚विष‚य एवेति} ब्रुव‚तोऽनुमान‚{\tiny $_{lb}$}‚स्यापि विष‚यः किं न स्व‚ल‚क्ष‚ण‚मित्य‚भिप्रायः । स एव किं न त‚थेत्याश‚ङ्क्य पूर्व‚प‚क्ष‚वाद्येवोप‚प‚त्ति‚{\tiny $_{lb}$}‚‚{\tiny $_{lb}$}‚ ‚{\tiny $_{lb}$}‚ \leavevmode\ledsidenote{\textenglish{76/dm}}‚{\tiny $_{lb}$}‚ 
	  
	क‚स्मात् पुन‚स्त‚देव प‚र‚मार्थ‚स‚दित्याह-- ‚{\tiny $_{lb}$}‚ 
	  
	अर्थ‚क्रियासाम‚र्थ्य‚ल‚क्ष‚ण‚त्वाद्व‚स्तुनः ॥ १५ ॥‚{\tiny $_{lb}$}‚ 
	  
	अर्थ्य‚त इत्य‚र्थः । हेय उपादेय‚श्च । हेयो हि हातुमिष्य‚ते उपादेय‚श्चोपादातुम् । अर्थ‚स्य‚{\tiny $_{lb}$}‚ प्र‚योज‚न‚स्य क्रिया निष्प‚त्तिः । त‚स्यां साम‚र्थ्यं श‚क्तिः । त‚देव ल‚क्ष‚णं रूपं य‚स्य व‚स्तुनः‚{\tiny $_{lb}$}‚ माह--त‚था हीति । विक‚ल्प‚श‚ब्दः प्र‚माण‚विष‚य‚चिन्त‚नाद‚नुमान‚विक‚ल्पो द्र‚ष्ट‚व्यः । \textbf{दृश्यात्म‚क‚{\tiny $_{lb}$}‚ एव} स्व‚ल‚क्ष‚णात्म‚क एव ।
	\pend% ending standard par
      ‚{\tiny $_{lb}$}‚

	  \pstart \leavevmode% starting standard par
	\hphantom{.}न‚नु य‚स्यार्थ‚स्य स‚न्निधानास‚न्निधानाभ्यां ज्ञान‚प्र‚तिभास‚भेद‚स्त‚त्स्व‚ल‚क्ष‚ण‚म् इत्युक्ते कुतोऽस्य‚{\tiny $_{lb}$}‚ प्र‚श्न‚स्याव‚काशः ? न हि विक‚ल्प‚विष‚य‚स्य स‚न्निधानास‚न्निधानाच्च ज्ञान‚प्र‚तिभास‚भेदोऽस्ति‚{\tiny $_{lb}$}‚ य‚तोऽस्योत्थानं स्यात् । स‚त्य‚म् । केव‚ल‚म‚य‚म‚स्याश‚यः--य‚देत‚द् भ‚व‚द्भिर्ल‚क्ष‚णं स्व‚ल‚क्ष‚ण‚स्य प्र‚तीतं‚{\tiny $_{lb}$}‚ त‚द‚स्य ल‚क्ष‚ण‚मेव न भ‚व‚ति किन्तु य‚देव दृश्य‚त‚याऽध्य‚व‚सीय‚ते त‚देव स्व\add{ल‚क्ष‚ण}मितीदं त‚स्य‚{\tiny $_{lb}$}‚ ल‚क्ष‚ण‚म् । स‚ति चैव‚म‚नुमान‚विक‚ल्प‚विष‚य‚स्यापि त‚थात्व‚म‚निवारित‚मेवेति सूत्थानः प्र‚श्नः ।‚{\tiny $_{lb}$}‚ \textbf{एत‚त्प्र‚श्न‚विस‚र्ज‚न‚माचार्यीयं} द‚र्श‚य‚न्नाह--\textbf{त‚देवेति ।}
	\pend% ending standard par
      ‚{\tiny $_{lb}$}‚

	  \pstart \leavevmode% starting standard par
	अथ स्यात् य‚दि\edtext{}{\lemma{दि}\Bfootnote{प‚ङिक्त‚बाह्यं लिखितं न प‚ठ्य‚ते--सं०}}\add{... ... ...}त्युच्य‚मानं शोभेत याव‚तेद‚मेव न सिद्धं त‚त्कुतोऽयं प्र‚श्न‚{\tiny $_{lb}$}‚विस‚र्ज‚न‚प्र‚कार इति ? उच्य‚ते । स्व‚ल‚क्ष‚ण‚मिति स्व‚म‚साधार‚णं पार‚मार्थिकः स्व‚भाव उच्य‚ते ।‚{\tiny $_{lb}$}‚ स एव च पार‚मार्थिकः स्व‚भाव उच्य‚ते, य एवार्थ‚क्रियाक्ष‚मः । स एव च प‚र‚मार्थ‚स‚न्निति युक्त‚{\tiny $_{lb}$}‚मिद‚मुत्त‚रं \textbf{त‚देवेति । प‚र‚मोऽर्थ} इति द‚र्श‚य‚न् क‚र्म‚धार‚यं द‚र्श‚य‚ति । \textbf{अकृत्रिम}मित्यादि प‚र‚म‚श‚ब्द‚स्य‚{\tiny $_{lb}$}‚ व्याख्यान‚म् । \textbf{तेनास्ती}ति स‚दित्य‚स्यार्थ‚क‚थ‚न‚म् । तेन रूपेण स‚द् विद्य‚मान‚मिति विग्र‚हः । श‚त्र‚न्त‚{\tiny $_{lb}$}‚श्चाय‚म‚सिः । क‚र्त्तृक‚र‚णे कृता ब‚हुल‚म् \href{http://sarit.indology.info/?cref=Pā.2.1.32}{पाणिनि २. १. ३२}इति च स‚मासः । प‚र‚मार्थ‚स‚दित्य‚{\tiny $_{lb}$}‚स्यार्थ‚माख्याय त‚देवेत्येत‚द् विवृणोति \textbf{य एवेति । प्र‚तिभासं} ज्ञान‚स्येति प्र‚क‚र‚णात् । अर्थ‚क्रिया‚{\tiny $_{lb}$}‚स‚म‚र्थोऽर्थः । स्व‚ल‚क्ष‚णं चैव‚मात्म‚क‚मित्य‚भिप्रायेण \textbf{य एवार्थः स एवेत्याह} । अस्तु तादृशः प‚र‚मार्थ‚स‚न् ।‚{\tiny $_{lb}$}‚ स तु न त‚स्य विष‚योऽन्य‚स्यापि वा विष‚यो भ‚विष्य‚ति । त‚था च विव‚क्षितार्थासिद्धिरित्याह‚{\tiny $_{lb}$}‚ \textbf{स चे}ति । \textbf{चो}ऽव‚धार‚णे \textbf{प्र‚त्य‚क्ष‚स्ये}त्य‚तः प‚रो द्र‚ष्ट‚व्यः । \textbf{त‚देवेति} प्र‚त्य‚क्ष‚विष‚य‚त्वेन स्थितं \textbf{व‚स्त्वेव}‚{\tiny $_{lb}$}‚ न त्ब‚नुमान‚विष‚योऽपीति । अनेन \textbf{क‚स्मात्पुनः प्र‚त्य‚क्ष‚विष‚य एव स्व‚ल‚क्ष‚ण}मित्येत‚त्प्र‚श्न‚विष‚र्ज‚न‚{\tiny $_{lb}$}‚स्योप‚संहारः कृतो वेदित‚व्यः ॥
	\pend% ending standard par
      ‚{\tiny $_{lb}$}‚

	  \pstart \leavevmode% starting standard par
	इदानीं स्व‚ल‚क्ष‚ण‚स्यैव प‚र‚मार्थ‚स‚त्त्व‚म‚स‚ह‚मान आह--\textbf{क‚स्मादि}ति । अय‚म‚स्याश‚यः—‚{\tiny $_{lb}$}‚स एव ख‚लु प‚र‚मार्थ‚स‚न्न‚र्थो य एवार्थ‚त्वेनाध्य‚व‚सीय‚ते । अनुमान‚विष‚योऽपि च व‚ह्निस्त‚थाऽध्य‚{\tiny $_{lb}$}‚व‚सीय‚त इति सोऽपि क‚स्मान्न प‚र‚मा\leavevmode\ledsidenote{\textenglish{33b/ms}}र्थ‚स‚न्निति ।
	\pend% ending standard par
      ‚{\tiny $_{lb}$}‚

	  \pstart \leavevmode% starting standard par
	अत्र--\textbf{अर्थे}त्यादि य‚दुत्त‚रं त‚द् \textbf{अर्थ्य‚त} इत्यादिना व्याच‚ष्टे । \textbf{च}स्तुल्य‚ब‚ल‚त्वं स‚मु‚{\tiny $_{lb}$}‚च्चिनोति । हेयः क‚थ‚म‚र्थ्य‚त इत्याह--हेय इति । हिर्य‚स्मात् । य‚दि हातुम‚र्थ्य‚मानोऽर्थो न त‚र्ह्यु‚{\tiny $_{lb}$}‚पादेयोऽर्थ इत्याह--\textbf{उपादेय} इति । \textbf{चो} हेयापेक्ष‚योपादेय‚स्यार्थ्य‚मान‚त्वं स‚मुच्चिनोति । \textbf{अर्थ}स्य‚{\tiny $_{lb}$}‚ प्र‚योज‚न‚स्य दाहादेः । एवं च व्याच‚क्ष‚णः साध्यो दाहादिरेव मुख्य‚वृत्त्योपादेया हेयो वा ।‚{\tiny $_{lb}$}‚ \leavevmode\ledsidenote{\textenglish{77/dm}}‚{\tiny $_{lb}$}‚ 
	  
	त‚द् अर्थ‚क्रियासाम‚र्थ्य‚ल‚क्ष‚ण‚म् । त‚स्य भावः, त‚स्मात् । व‚स्तुश‚ब्दः प‚र‚मार्थ‚प‚र्यायः ।‚{\tiny $_{lb}$}‚ त‚द‚य‚म‚र्थः--य‚स्माद‚र्थ‚क्रियास‚म‚र्थं प‚र‚मार्थ‚स‚दुच्य‚ते, स‚न्निधानास‚न्निधानाभ्यां च ज्ञान‚प्र‚तिभास‚स्य‚{\tiny $_{lb}$}‚ भेद‚कोऽर्थोऽर्थ‚क्रियास‚म‚र्थः, त‚स्मात् स एव प‚र‚मार्थ‚स‚न् । त‚त एव हि प्र‚त्य‚क्ष‚विष‚याद‚र्थ‚क्रिया‚{\tiny $_{lb}$}‚ प्राप्य‚ते न विक‚ल्प‚विष‚यात्\edtext{}{\lemma{यात्}\Bfootnote{सामान्यात्--\cite{dp-msD-n}}} । अत एव य‚द्य‚पि विक‚ल्प‚विष‚यो दृश्य इवाव‚सीय‚ते त‚थापि‚{\tiny $_{lb}$}‚ न\edtext{}{\lemma{न}\Bfootnote{न दृश्य \cite{dp-msA} \cite{dp-msB} \cite{dp-edP} \cite{dp-edH} \cite{dp-edE} \cite{dp-edN}}} स दृश्य एव \edtext{}{\lemma{एव}\Bfootnote{विक‚ल्प‚विष‚यात्--\cite{dp-msD-n}}}त‚तोऽर्थ‚क्रियाया\edtext{}{\lemma{क्रियाया}\Bfootnote{०क्रियाभावात्--\cite{dp-msA} \cite{dp-msB} \cite{dp-edP} \cite{dp-edH} \cite{dp-edE} \cite{dp-edN}}} अभावात्, दृश्याच्च भावात् । अत‚स्त‚देव स्व‚ल‚क्ष‚णं‚{\tiny $_{lb}$}‚ न विक‚ल्प‚विष‚यः\edtext{}{\lemma{यः}\Bfootnote{विष‚य‚म् \cite{dp-msA} \cite{dp-msB} \cite{dp-edP} \cite{dp-edH} \cite{dp-edE} \cite{dp-edN}}} ॥ ‚{\tiny $_{lb}$}‚ 
	  
	अन्य‚त् सामान्य‚ल‚क्ष‚ण‚म् ॥ १६ ॥‚{\tiny $_{lb}$}‚ 
	  
	\edtext{\textsuperscript{*}}{\lemma{*}\Bfootnote{अन्य‚दित्यादि--नास्ति \cite{dp-msA} \cite{dp-edP} \cite{dp-edH} \cite{dp-edE} \cite{dp-edN}}}अन्य‚दित्यादि । एत‚स्मात् स्व‚ल‚क्ष‚णाद् य‚द् अन्य‚त्--स्व‚ल‚क्ष‚णं यो न भ‚व‚ति ज्ञान‚{\tiny $_{lb}$}‚विष‚यः--त‚त्\edtext{}{\lemma{त्}\Bfootnote{त‚स्मात् सामा० \cite{dp-msB}}} सामान्य‚ल‚क्ष‚ण‚म्, विक‚ल्प‚ज्ञानेनाव‚सीय‚मानो ह्य‚र्थः स‚न्निधानास‚न्निधानाभ्यां‚{\tiny $_{lb}$}‚ ज्ञान‚प्र‚तिभासं न भिन‚त्ति । त‚थाहि आरोप्य‚माणो व‚ह्निरारोपाद‚स्ति । आरोपाच्च दूर‚स्थो‚{\tiny $_{lb}$}‚ निक‚ट‚स्थ‚श्च । त‚स्य स‚मारोपित‚स्य स‚न्निधानास‚न्निधानाच्च ज्ञान‚प्र‚तिभास‚स्य न भेदः‚{\tiny $_{lb}$}‚ स्फुट‚त्वेनास्फुट‚त्वेन वा । त‚तः स्व‚ल‚क्ष‚णाद‚न्य उच्य‚ते । सामान्येन ल‚क्ष‚णं सामान्य‚{\tiny $_{lb}$}‚ल‚क्ष‚ण‚म् । साधार‚णं रूप‚मित्य‚र्थः । ‚{\tiny $_{lb}$}‚ 
	  
	स‚मारोप्य‚माणं हि रूपं स‚क‚ल‚व‚ह्निसाधार‚ण‚म् । त‚तः\edtext{}{\lemma{तः}\Bfootnote{त‚त‚स्त‚स्मात् सामा० \cite{dp-msB}}} त‚त् सामान्य‚ल‚क्ष‚ण‚म् ॥‚{\tiny $_{lb}$}‚ त‚मेवोपादातुं हातुं वा त‚त्साध‚न‚स्योपादानं हानं वाऽन्य‚था न श‚क्य‚त इति त‚त्साध‚न‚भूतो‚{\tiny $_{lb}$}‚ व‚ह्न्यादिरुपादेयादिरुप‚प‚द्य‚त इति द‚र्श‚य‚ति । \textbf{त‚स्यामित्या}दि \textbf{त‚स्मादि}त्य‚न्तं सुबोध‚म् । प‚र‚मा‚{\tiny $_{lb}$}‚र्थ‚स‚त‚स्त‚त्त्वे प्र‚तिपाद्ये किम‚ति व‚स्तुन‚स्ताद्रूप्य‚माचार्येण द‚र्शित‚मित्याश‚ङ्काम‚पाकुर्व‚न्नाह--\textbf{व‚स्तुश‚ब्द}‚{\tiny $_{lb}$}‚ इति । \textbf{त‚द‚य‚म‚र्थ} इत्यादिर्न \textbf{विक‚ल्प‚विष‚य} इत्य‚न्तो ग्र‚न्थ‚स्तूक्तार्थोप‚संहारः । स च सुज्ञानः ॥
	\pend% ending standard par
      ‚{\tiny $_{lb}$}‚

	  \pstart \leavevmode% starting standard par
	इह प्र‚स्तुत‚स्व‚ल‚क्ष‚णापेक्ष‚म‚न्य‚त्व‚माचार्य‚स्याभिप्रेत‚म् । स्व‚ल‚क्ष‚ण‚श‚ब्द‚स्य चान्य‚श‚ब्द‚{\tiny $_{lb}$}‚योगात्प‚ञ्च‚म्या भ‚वित‚व्य‚मित्य‚भिप्रेत्य \textbf{एत‚स्मा}दित्याह । अन्य‚त्व‚म‚भिव्य‚न‚क्ति--\textbf{स्व‚ल‚क्ष‚णं यो न‚{\tiny $_{lb}$}‚ भ‚व‚ती}ति । श‚श‚विषाणादिव्य‚व‚च्छेदार्थ‚माह--\textbf{ज्ञान‚विष‚य} इति । प्र‚माण‚विष‚य‚चिन्तायाः‚{\tiny $_{lb}$}‚ प्र‚स्तुत‚त्वाद् विज्ञान‚विष‚य इति ल‚ब्ध‚म् । असिद्ध‚म‚स्य त‚तोऽन्य‚त्व‚म्, अस्यापि स्व‚ल‚क्ष‚ण‚कार्य‚{\tiny $_{lb}$}‚कारित्वादित्याह--\textbf{विक‚ल्पेत्यादि । हिर्य}स्माद‚र्थे ।
	\pend% ending standard par
      ‚{\tiny $_{lb}$}‚

	  \pstart \leavevmode% starting standard par
	प्र‚तिभासाभेद‚क‚त्वं त‚स्य कुतोऽव‚सीय‚ते स‚र्व‚दा त‚स्य स‚न्निहित‚रूप‚त्वाद‚स‚न्निहित‚रूप‚त्वादेव‚{\tiny $_{lb}$}‚ वेत्याह--\textbf{त‚था हीति । आरोप्य‚मा}ण‚स्त‚था प्र‚तीय‚मानः । \textbf{आरोपात्} त‚थोत्पाद‚ल‚क्ष‚णाद‚ध्य‚{\tiny $_{lb}$}‚व‚सायाद् \textbf{अस्ति} व‚ह्निरूपेण । आस्तामारोपात् त‚थाभूत‚स्त‚थापि त‚स्य स‚न्निधानास‚न्निधान\edtext{}{\lemma{न्निधान}\Bfootnote{ने}}‚{\tiny $_{lb}$}‚ कुत इत्याह--\textbf{आरोपादि}ति । \textbf{चो} व्य‚क्त‚मेत‚दित्य‚स्यार्थे । द्वितीय‚श्च‚कारो दूर‚स्थ‚त्वापेक्ष‚या‚{\tiny $_{lb}$}‚ निक‚ट‚स्थ‚त्व‚स्यैक‚विष‚य‚त्वं स‚मुच्चिनोति ।
	\pend% ending standard par
      ‚{\tiny $_{lb}$}‚

	  \pstart \leavevmode% starting standard par
	त‚स्येत्याद्युच्य‚त इत्य‚न्तं स्प‚ष्टार्थ‚म् ।
	\pend% ending standard par
      ‚{\tiny $_{lb}$}‚‚{\tiny $_{lb}$}‚\textsuperscript{\textenglish{78/dm}}‚{\tiny $_{lb}$}‚
	  \bigskip
	  \begingroup
	

	  \pstart \leavevmode% starting standard par
	त‚च्चानुमान‚स्य ग्राह्यं द‚र्श‚यितुमाह--
	\pend% ending standard par
       ‚{\tiny $_{lb}$}‚ 
	  \bigskip
	  \begingroup
	

	  \pstart \leavevmode% starting standard par
	सोऽनुमान‚स्य विष‚यः ॥ १७ ॥
	\pend% ending standard par
      
	  \endgroup
	‚{\tiny $_{lb}$}‚ 

	  \pstart \leavevmode% starting standard par
	सोऽनुमान‚स्य विष‚यो ग्राह्य‚रूपः\edtext{}{\lemma{रूपः}\Bfootnote{ब‚हुव्रीहि--\cite{dp-msD-n}}} । स‚र्व‚नाम्नोऽभिधेय‚व‚ल्लिङ्ग‚प‚रिग्र‚हः ।
	\pend% ending standard par
       ‚{\tiny $_{lb}$}‚ 

	  \pstart \leavevmode% starting standard par
	सामान्य‚ल‚क्ष‚ण‚म‚नुमान‚स्य विष‚यं व्याख्यातुकामेनायं स्व‚ल‚क्ष‚ण‚स्व‚रूपाख्यान‚ग्र‚न्थ‚{\tiny $_{lb}$}‚ आव‚र्त्त‚नीयः स्यात् । त‚तो लाघ‚वार्थं प्र‚त्य‚क्ष‚प‚रिच्छेद एवानुमान‚विष‚य उक्तः ॥
	\pend% ending standard par
      
	  \endgroup
	‚{\tiny $_{lb}$}‚

	  \pstart \leavevmode% starting standard par
	सामान्य‚श‚ब्देन ल‚क्ष‚ण‚श‚ब्द‚स्य विग्र‚ह‚म ह--\textbf{सामान्येने}ति । \textbf{सामान्येन, \add{न}} विशेर्षेण‚{\tiny $_{lb}$}‚ स‚न्तानान्त‚र‚साधार‚णं स्व‚रूप‚म् । साध‚नं कृतेति स‚मासः । त‚स्य प‚द‚स्यार्थं स्प‚ष्ट‚य‚ति—‚{\tiny $_{lb}$}‚\textbf{साधार‚ण‚मि}ति ।
	\pend% ending standard par
      ‚{\tiny $_{lb}$}‚

	  \pstart \leavevmode% starting standard par
	विक‚ल्प‚विष‚य‚स्याप्य‚साधार‚ण‚त्वाद‚सिद्धं व्य‚क्त्य‚न्त‚र‚साधार‚ण‚त्व‚मित्याह--\textbf{स‚मारोप्य‚माण‚{\tiny $_{lb}$}‚मि}ति । \textbf{हि}र्य‚स्मात् । \textbf{स‚मारोप्य‚माणं} विक‚ल्पेन त‚था प्र‚तीय‚मान‚म् । \textbf{स‚क‚ल‚श्चासौ} तार्ण‚पार्णा‚{\tiny $_{lb}$}‚दिभेद‚भिन्नो \textbf{व‚ह्नि}श्चेति त‚था । त‚त्\textbf{साधार‚णं} त‚थाविध‚लिङ्ग‚ब‚लेनान‚ग्निव्यावृत्त‚व‚स्तुमात्र‚प्र‚ति‚{\tiny $_{lb}$}‚भास‚नादिति भावः ।
	\pend% ending standard par
      ‚{\tiny $_{lb}$}‚

	  \pstart \leavevmode% starting standard par
	न‚नु च त‚थाविधं सामान्यं विक‚ल्प‚गोच‚रोऽव‚स्तु । त‚द्विष‚य‚त्वेऽनुमान‚स्य क‚थं बाह्ये‚{\tiny $_{lb}$}‚ प्र‚व‚र्त्त‚क‚त्वं त‚त्प्राप‚क‚त्व‚ञ्च, य‚तः प्रामाण्य‚म‚स्य स्यादिति चेद् । उच्य‚ते । विक‚ल्पाः ख‚ल्वेतेऽनाद्य‚{\tiny $_{lb}$}‚विद्याव‚शात्स्व‚प्र‚तिभास‚म‚न‚ग्निव्यावृत्त‚म‚व‚स्य‚न्तो बाह्योऽप्य‚न‚ग्निव्यावृत्त इति त‚द‚ध्य‚व‚सान‚मेव‚{\tiny $_{lb}$}‚ बाह्यो व‚ह्निर‚ध्य‚व‚सित इति म‚न्य‚न्ते । अन‚ग्निव्यावृत्त‚त‚या बाह्य‚स‚दृश‚व‚ह्न्य‚ध्य‚व‚साय एव‚{\tiny $_{lb}$}‚ बाह्य‚व‚ह्न्य‚ध्य‚व‚सायः । त‚योर्विवेकाप्र‚तिप‚त्तेः । अत एव ते विक‚ल्पा दृश्य‚विक‚ल्प्याव‚र्थावे‚{\tiny $_{lb}$}‚कीकृत्य बाह्ये लोकं प्र‚व‚र्त्त‚य‚न्ति । दृश्य‚विक‚ल्प्यैकीक‚र‚ण‚म‚पि तेषां त‚था प्र‚वृत्तिहेतुत‚योत्प‚त्ते‚{\tiny $_{lb}$}‚रेव द्र‚ष्ट‚व्य‚म् । \leavevmode\ledsidenote{\textenglish{34a/ms}} बाह्य‚स‚म्ब‚द्ध‚स‚म्ब‚द्ध‚त्वाच्चानुमान‚विक‚ल्पः संवाद‚कः, अध्य‚व‚सेयापेक्ष‚या‚{\tiny $_{lb}$}‚ \textbf{च} प्र‚माण‚म् । त‚दाह \textbf{न्याय‚वादी}--भ्रान्तिर‚पि स‚म्ब‚न्ध‚तः प्र‚मा इति । एष चार्थः \textbf{स्व‚प्र‚ति‚{\tiny $_{lb}$}‚भासेऽन‚र्थेऽर्थाध्य‚व‚सायेन प्र‚वृत्ते}रित्य‚त्राप्य‚पेक्ष‚णीयः । स‚मास‚त‚स्तु त‚त्रास्माभिः किञ्चिद‚वादीति ।‚{\tiny $_{lb}$}‚ य‚त एवं त‚त्त‚स्मादित्युप‚संहारः । \textbf{त‚दि}ति त‚दारोप्य‚माण‚म् ॥
	\pend% ending standard par
      ‚{\tiny $_{lb}$}‚

	  \pstart \leavevmode% starting standard par
	\textbf{त‚च्चे}त्यादिना सोऽनुमानेत्य‚स्याव‚तारं द‚र्श‚य‚ति । \textbf{चो}ऽव‚धार‚णे । \textbf{ग्राह्यं रूपं} स्व‚भावो‚{\tiny $_{lb}$}‚ऽस्येति विग्र‚हः । एवं व्याच‚क्षाणो विष‚य‚श‚ब्देन ग्राह्यो विष‚योऽभिप्रेतोऽत्र प्र‚माण‚व्यापार‚{\tiny $_{lb}$}‚विष‚योऽध्य‚व‚सेयः स्व‚ल‚क्ष‚ण‚स्यैव त‚द‚ध्य‚व‚सेय‚त्वेन त‚था व्य‚व‚स्थाप‚नादिति द‚र्श‚य‚ति । त‚स्य तु‚{\tiny $_{lb}$}‚ त‚त्र स्व‚रूपेणाप्र‚तिभास‚नाद‚ग्राह्य‚त्व‚म् । प्र‚काश‚मानं च रूप‚माश्रित्य ग्राह्य‚त्व‚मुच्य‚त इत्युक्त‚{\tiny $_{lb}$}‚प्राय‚म् ।
	\pend% ending standard par
      ‚{\tiny $_{lb}$}‚

	  \pstart \leavevmode% starting standard par
	अनुमान‚स्यापि ग्राह्य‚विष‚य‚द‚र्श‚नेऽय‚म‚भिप्रायो य‚दीद‚म‚नुमान‚म‚न‚र्थ‚म‚न‚ग्निव्यावृत्तिमात्रं‚{\tiny $_{lb}$}‚ सामान्य‚रूपं गृह‚णाति, त‚त्र त‚त्सामान्य‚रूपं किञ्चिद‚व‚भासेत, त‚दा त‚त्स्व‚ल‚क्ष‚ण‚त्वेनाव‚स्य‚त्स्व‚{\tiny $_{lb}$}‚ल‚क्ष‚णं प्र‚वृत्तिविष‚यीकुर्यात् । त‚था कुर्व‚च्च व‚स्तुविष‚यं प्रामाण्य‚म‚श्नुवीत नान्य‚थेति । स‚{\tiny $_{lb}$}‚ \leavevmode\ledsidenote{\textenglish{79/dm}}‚{\tiny $_{lb}$}‚ 
	  
	विष‚य‚विप्र‚तिप‚त्तिं निराकृत्य फ‚ल‚विप्र‚तिप‚र्त्तिं निराक‚र्त्तुमाह-- ‚{\tiny $_{lb}$}‚ 
	  
	त‚देव\edtext{}{\lemma{देव}\Bfootnote{त‚देव प्र‚त्य० \cite{dp-msC}}} च प्र‚त्य‚क्षं ज्ञानं प्र‚माण‚फ‚ल‚म् ॥ १८ ॥‚{\tiny $_{lb}$}‚ 
	  
	त‚देवेति । य‚देवान‚न्त‚र‚मुक्तं प्र‚त्य‚क्षं \edtext{}{\lemma{क्षं}\Bfootnote{प्र‚त्य‚क्षं त‚देव \cite{dp-msA} \cite{dp-msB} \cite{dp-edP} \cite{dp-edH} \cite{dp-edE}}}ज्ञानं त‚देव प्र‚माण‚स्य फ‚ल‚म् ॥ ‚{\tiny $_{lb}$}‚ 
	  
	क‚थं प्र‚माण‚फ‚ल‚मित्याह-- ‚{\tiny $_{lb}$}‚ 
	  
	अर्थ‚प्र‚तीतिरूप‚त्वात् ॥ १९ ॥‚{\tiny $_{lb}$}‚ 
	  
	अर्थ‚स्य प्र‚तीतिर‚व‚ग‚मः, सैव रूपं य‚स्य प्र‚त्य‚क्ष\edtext{}{\lemma{क्ष}\Bfootnote{प्र‚त्य‚क्ष‚स्य ज्ञान० \cite{dp-msD}}}ज्ञान‚स्य त‚द‚र्थ‚प्र‚तीतिरूप‚म् । त‚स्य‚{\tiny $_{lb}$}‚ भावः, त‚स्मात् । ‚{\tiny $_{lb}$}‚ 
	  
	एत‚दुक्तं भ‚व‚ति--प्राप‚कं ज्ञानं प्र‚माण‚म् । प्राप‚ण‚श‚क्तिश्च न केव‚लाद‚र्थाविना‚{\tiny $_{lb}$}‚भावित्वाद् भ‚व‚ति । बीजाद्य‚विनाभाविनोऽप्य‚ङ्कुरादेर‚प्राप‚क‚त्वात् । त‚स्मात् \edtext{}{\lemma{स्मात्}\Bfootnote{त‚स्माद‚र्थादुत्प० \cite{dp-msA} \cite{dp-edP} \cite{dp-edH} \cite{dp-edE} त‚स्माद् ग्राह्याद‚र्थादुत्प० \cite{dp-edN}}}प्राप्याद‚र्थादुत्प‚{\tiny $_{lb}$}‚त्ताव‚प्य‚स्य ज्ञान‚स्याऽस्ति क‚श्चिद‚व‚श्य‚क‚र्त्त‚व्यः प्राप‚क‚व्यापारोयेन कृतेनार्थः प्रापितो भ‚व‚ति । ‚{\tiny $_{lb}$}‚ 
	  
	स एव च प्र‚माण‚फ‚ल‚म्, य‚द‚नुष्ठानात् प्राप‚कं भ‚व‚ति ज्ञान‚म् । उक्तं च पुर‚स्तात् प्र‚वृत्ति‚{\tiny $_{lb}$}‚विष‚य‚प्र‚द‚र्श‚न‚मेव प्राप‚क‚स्य प्राप‚क‚व्यापारो नाम । ‚{\tiny $_{lb}$}‚ 
	  
	त‚देव च प्र‚त्य‚क्ष‚म् अर्थ‚प्र‚तीतिरूप‚म् \edtext{}{\lemma{म्}\Bfootnote{अर्थ‚द‚र्श‚न \cite{dp-msA} \cite{dp-edP} \cite{dp-edH} \cite{dp-edE} \cite{dp-edN}}}अर्थ‚प्र‚द‚र्श‚न‚रूप‚म् । अत‚स्त‚देव प्र‚माण‚फ‚ल‚म् ॥‚{\tiny $_{lb}$}‚ इति पुंल्लिङ्ग‚निर्देश‚स‚म‚र्थ‚नार्थ‚माह--\textbf{स‚र्वे}ति । इवार्थे \textbf{व‚ति} । त‚द्व‚ल्लिङ्ग‚स्य प‚रिग्र‚ह \add{इति‚{\tiny $_{lb}$}‚ ष‚ष्ठ्य‚न्तेन} विग्र‚हः । अत्रापि सामान्य‚ल‚क्ष‚ण‚मेवानुमान‚स्य विष‚य इत्य‚व‚धार‚णीयं न त्व‚नुमान‚{\tiny $_{lb}$}‚स्यैवेति प्र‚त्य‚क्ष‚पृष्ठ‚भाविनोऽपि विक‚ल्प‚विशेष‚स्य त‚त्विष‚य‚त्वाद‚न्य‚था लिङ्ग‚निश्च‚यायोगादिति ।
	\pend% ending standard par
      ‚{\tiny $_{lb}$}‚

	  \pstart \leavevmode% starting standard par
	स‚म्प्र‚ति प्र‚त्य‚क्ष‚स्य विष‚य‚विप्र‚तिप‚त्तिनिराक‚र‚णे प्र‚कृते किम‚प्र‚कृत‚म‚नुमान‚स्य विष‚य‚{\tiny $_{lb}$}‚विप्र‚तिप‚त्तिनिराक‚र‚ण‚माच‚रित‚मित्याश‚ङ्काम‚पाक‚र्त्तुं \textbf{सामान्येत्या}दिनोप‚क्र‚म‚ते । एत\textbf{च्चोक्त}‚{\tiny $_{lb}$}‚ इत्येत‚द‚न्तं सुज्ञान‚म् ॥
	\pend% ending standard par
      ‚{\tiny $_{lb}$}‚

	  \pstart \leavevmode% starting standard par
	\textbf{विष‚ये}त्यादि \textbf{फ‚ल}मित्येद‚न्तं सुबोध‚म् । फ‚ल‚प्र‚द‚र्श‚ने चायं \textbf{वार्त्तिक‚कार}स्याश‚यः—‚{\tiny $_{lb}$}‚प्र‚माक‚र‚णं ख‚लु लोके प्र‚माण‚मुच्य‚ते । त‚तः क‚र‚ण‚साध्या फ‚ल‚रूपा प्र‚मितिर‚व‚श्य‚द‚र्श‚यित‚व्येति ॥
	\pend% ending standard par
      ‚{\tiny $_{lb}$}‚

	  \pstart \leavevmode% starting standard par
	न‚नु य‚स्मिन् विष‚ये क्रियासाध‚नं व्याप्रिय‚ते\edtext{}{\lemma{ते}\Bfootnote{अत्र प‚ङ्क्तिबाह्यं किञ्चिल्लिखित‚म‚स्ति । सूक्ष्म‚त्वात् न प‚ठ्य‚ते--सं०}}...क‚थ‚मिति । किम‚न्य‚त्प्र‚माण‚मिति‚{\tiny $_{lb}$}‚ पुन‚रिहोपेक्षित‚म‚नेन । अर्थेत्यादि प्र‚तिव‚च‚नं व्याच‚ष्टे \textbf{अर्थ‚स्ये}ति । एत‚च्च \textbf{त‚स्मादि}त्य‚न्तं सुग‚म‚म् ।
	\pend% ending standard par
      ‚{\tiny $_{lb}$}‚

	  \pstart \leavevmode% starting standard par
	न‚नु संवाद‚कं ज्ञानं प्र‚माण‚म् । संवाद‚श्च त‚दुत्प‚त्तिमात्रात् । त‚त् किम‚र्थे प्र‚तीति‚{\tiny $_{lb}$}‚स्त‚त्फ‚लं मृग्य‚त इत्याश‚ङ्क्याह--\textbf{एत‚दुक्तं भ‚व‚ती}ति ।
	\pend% ending standard par
      ‚{\tiny $_{lb}$}‚‚{\tiny $_{lb}$}‚\textsuperscript{\textenglish{80/dm}}‚{\tiny $_{lb}$}‚

	  \pstart \leavevmode% starting standard par
	अथ प्राप‚क‚त्वं प्राप‚ण‚श‚क्तियोगात् । सा च श‚क्तिस्त‚त उत्प‚त्तेरेव । त‚था च क‚थं‚{\tiny $_{lb}$}‚ पूर्व‚प‚क्षातिक्र‚म इत्याह--\textbf{प्राप‚ण‚श‚क्तिश्चे}ति । \textbf{चो} य‚स्मात् । \textbf{न केव‚ला}देकाकिनोऽर्थ‚प्र‚द‚र्श‚न‚{\tiny $_{lb}$}‚विनाकृतात् । विव‚क्षितेन विना भ‚व‚तीति विनाभावी ग्र‚हादित्वाण्णिनिः, व्य‚भिचारीत्य‚र्थः ।‚{\tiny $_{lb}$}‚ अर्थेन विना भावीति तृतीया \href{http://sarit.indology.info/?cref=Pā.2.1.30}{पा० २. १. ३०} इति योग‚विभागात्स‚मासः । त‚स्य भाव‚{\tiny $_{lb}$}‚स्त‚स्मात् । क‚थं पुन‚र्न केव‚लादित्याह--\textbf{बीजादिति} । य‚त एवं \textbf{त‚स्मात्} । य‚द्य‚पि स‚न्तान‚{\tiny $_{lb}$}‚ एव प्राप्यो न च त‚स्मात्त‚स्योत्प‚त्तिस्त‚थापि विष‚येऽस्यैक‚त्व‚म‚ध्य‚व‚साय \textbf{प्राप्याद‚र्थादुत्त्प‚त्ता‚{\tiny $_{lb}$}‚व‚पी}त्युक्त‚म् । \textbf{प्राप‚क‚स्य ज्ञान‚स्याव‚श्य‚क‚र्त्त‚व्यः} प्र‚तीतिक्रियारूपः । प्रापितो भ‚व‚तीति पूर्व‚व‚द्‚{\tiny $_{lb}$}‚ योग्य‚त\leavevmode\ledsidenote{\textenglish{34b/ms}}योच्य‚ते । भ‚व‚तु त‚स्य क‚र्त्त‚व्य‚म‚न्य‚त्, त‚थापि न त‚त्फ‚ल‚मित्याह--\textbf{स एवे}ति ।‚{\tiny $_{lb}$}‚ \textbf{च} स्फुट‚मेत‚दित्य‚स्मिन्न‚र्थे ।
	\pend% ending standard par
      ‚{\tiny $_{lb}$}‚

	  \pstart \leavevmode% starting standard par
	भ‚व‚तु अव‚श्य‚क‚र्त्त‚व्यः प्राप‚क‚व्यापारः । स पुन‚र‚ज्ञानात्म‚को भ‚विष्य‚ति । त‚था च क‚थं‚{\tiny $_{lb}$}‚ प्र‚तीतिः फ‚ल‚मित्याह \textbf{उक्त}मिति । \textbf{चो} य‚स्माद‚र्थे । \textbf{पुर‚स्ता}त्पूर्व‚स्मिन् । केनोक्त‚मेत‚दिति चेत्‚{\tiny $_{lb}$}‚ \textbf{प्र‚व‚र्त्त‚क‚त्व‚म‚पि प्र‚वृत्तिविष‚योप‚द‚र्श‚क‚त्व‚मेवे}त्यादिना । आस्ताम‚र्थ‚प्र‚तीतिर‚व‚श्य‚क‚र्त्त‚व्या प्र‚माण‚स्य‚{\tiny $_{lb}$}‚ त‚थापि क‚थं त‚देव प्र‚त्य‚क्ष‚ज्ञानं फ‚लं त‚स्यात‚द्रूप‚त्वादित्याश‚ङ्क्य य‚द‚र्थ‚प्र‚तीतिरूप‚त्वं प्र‚तिज्ञातं‚{\tiny $_{lb}$}‚ त‚दुप‚पाद‚य‚न्नाह--\textbf{त‚देवे}ति । \textbf{चो} य‚स्माद‚र्थे ।
	\pend% ending standard par
      ‚{\tiny $_{lb}$}‚

	  \pstart \leavevmode% starting standard par
	न‚न्व‚र्थ‚प्र‚द‚र्श‚नं क‚र्त्त‚व्य‚त‚या प्र‚द‚र्शित‚म् । त‚त् क‚थ‚म‚र्थ‚प्र‚तीत्यामेकं त‚त् प्र‚द‚र्श्य‚त इत्याह—‚{\tiny $_{lb}$}‚\textbf{अर्थ‚प्र‚द‚र्श‚न‚रूप‚मि}ति । अन‚योर‚र्थाभेद इत्य‚भिप्रायः । य‚त एव\textbf{म‚तो}ऽस्मा\textbf{त्त‚देव} प्र‚त्य‚क्षं ज्ञान‚मेव‚{\tiny $_{lb}$}‚ नार्थान्त‚र‚मित्य‚र्थात् । प्र‚माणाद‚न्य‚ज्ज्ञान‚मेव फ‚लं य‚दि स्यात्, किंस्यात् ? येन य‚त्नेन ग‚रीय‚सा‚{\tiny $_{lb}$}‚ प्र‚माणाद‚भेदोऽस्य साध्य‚त इति चेत् । उच्य‚ते । धीप्र‚माण‚ता, प्र‚वृत्तेस्त‚त्प्र‚धान‚त्वात् \href{http://sarit.indology.info/?cref=pv.1.5}{प्र‚माण‚वा० 	१. ५.}इत्य‚त्राज्ञानात्म‚न‚स्ताव‚त्प्रामाण्य‚म‚प‚ह‚स्तित‚म् । ज्ञानात्म‚न‚श्च प्र‚माण‚स्या\edtext{}{\lemma{स्या}\Bfootnote{स्य}} भिन्नेऽपि‚{\tiny $_{lb}$}‚ प्र‚मितिरूपे फ‚लेऽव‚श्य‚म‚र्थ‚प‚रिच्छेदात्म‚ताऽभ्युपेत‚व्या । अन्य‚था ज्ञान‚त्व‚मेव त‚स्य न स्यात् । प्र‚मा‚{\tiny $_{lb}$}‚ज‚न‚क‚त्वेन च ज्ञान‚त्वे च‚क्षुरादेर‚पि ज्ञान‚त्वं स्यात् । बोध‚रूप‚त्वं चास्व‚संवेद‚न‚त‚या न नियाम‚क‚म् ।‚{\tiny $_{lb}$}‚ स्व‚संवेद‚न‚रूप‚त्वे वा ग्राह्याकार‚संवेद‚न‚मेवार्थ‚वेद‚न‚मिति क‚थं भिन्नं फ‚ल‚म् ? प्र‚तिक‚र्म‚व्य‚व‚स्था‚{\tiny $_{lb}$}‚ऽनुप‚प‚त्तेश्च न निराकार‚त्वं विज्ञान‚स्याभ्युपेय‚म् । ज्ञान‚मेवार्थ‚स्य प्र‚काश इति च स‚र्व‚त‚न्त्र‚{\tiny $_{lb}$}‚सिद्धोऽय‚म‚र्थः । त‚तो न ज्ञान‚स्याधिग‚म‚रूप‚तायां विप्र‚तिप‚त्त‚व्यं केन‚चित् । स‚त्यां च त‚दात्म‚तायां‚{\tiny $_{lb}$}‚ त‚देव फ‚लं युज्य‚ते, ताव‚तैव प्र‚माण‚व्यापार‚प‚रिस‚माप्तेः ।
	\pend% ending standard par
      ‚{\tiny $_{lb}$}‚

	  \pstart \leavevmode% starting standard par
	अपि च य‚दि ज्ञान‚स्य स्व‚य‚म‚र्थ‚रूप‚प‚रिच्छेद‚रूप‚त्वेनार्थ‚प‚रिच्छेद‚क‚त्वं न स्यात्, किन्तु‚{\tiny $_{lb}$}‚ भिन्न‚प‚रिच्छित्तिज‚न‚क‚त्वेन; त‚दाऽर्थ‚स्य प‚रिच्छित्तेर‚प‚रोक्ष‚तैव व्य‚व‚स्थाप‚यितुं न श‚क्येत ।‚{\tiny $_{lb}$}‚ त‚थाहि त‚दाद्यं ज्ञान‚म‚र्थं प‚रिच्छिन‚त्त्य‚र्थ‚म‚प‚रोक्ष‚य‚तीति कोऽर्थोऽर्थ‚विष‚यां प‚रिच्छित्तिं ज‚न‚य‚तीति ।‚{\tiny $_{lb}$}‚ साऽपि य‚द्य‚र्थ‚म‚प‚रोक्ष‚य‚ति त‚दाऽप‚रां प्र‚तीतिं ज‚न‚य‚तीति स्यात् । एव‚मुत्त‚र‚त्राप्येव‚मेवेति‚{\tiny $_{lb}$}‚ प‚रिच्छित्तीनामान‚न्त्य‚म्, न त्व‚र्थ‚स्याप‚रोक्ष‚तेत्यायात‚मान्ध्य‚म‚शेष‚स्य ज‚ग‚तः ।
	\pend% ending standard par
      ‚{\tiny $_{lb}$}‚

	  \pstart \leavevmode% starting standard par
	अथ तासामेका स्व‚य‚म‚र्थ‚प‚रिच्छेद‚रूपा अर्थाप‚रोक्ष‚तारूपोपेय‚ते; त‚दा न भिन्न‚प‚रिच्छित्ति‚{\tiny $_{lb}$}‚ज‚न‚क‚त्वं प‚रिच्छेद‚क‚त्व‚म् । किन्त‚र्हि ? स्व‚य‚म‚र्थ‚प‚रिच्छेदात्म‚त्व‚मिति आद्य‚स्यापि त‚थात्व‚{\tiny $_{lb}$}‚म‚निवारित‚म् । त‚था च क‚थं प्र‚माणाद् व्य‚तिरिक्तं फ‚ल‚मिति ? ।
	\pend% ending standard par
      ‚{\tiny $_{lb}$}‚

	  \pstart \leavevmode% starting standard par
	इह न क्रियैव क‚र‚णं लोके त‚योर्भेदेनाव‚स्थितेः । या चेयं ज्ञान‚ल‚क्ष‚णा क्रिया सा चेत्फ‚लं
	\pend% ending standard par
      \textsuperscript{\textenglish{81/dm}}‚{\tiny $_{lb}$}‚
	  \bigskip
	  \begingroup
	

	  \pstart \leavevmode% starting standard par
	य‚दि त‚र्हि ज्ञानं प्र‚मितिरूप‚त्वात् प्र‚माण‚फ‚ल‚म्, किं त‚र्हि प्र‚माण‚मित्याह--
	\pend% ending standard par
       ‚{\tiny $_{lb}$}‚ 
	  \bigskip
	  \begingroup
	

	  \pstart \leavevmode% starting standard par
	अर्थ‚सारूप्य‚म‚स्य प्र‚माण‚म् ॥ २० ॥
	\pend% ending standard par
      
	  \endgroup
	‚{\tiny $_{lb}$}‚ 

	  \pstart \leavevmode% starting standard par
	अर्थेन स‚ह य‚त् सारूप्यं\edtext{}{\lemma{सारूप्यं}\Bfootnote{०रूप्यं य‚त् सादृ० \cite{dp-msB} \cite{dp-msD}}} सादृश्य‚म् अस्य ज्ञान‚स्य त‚त् प्र‚माण‚म् । इह य‚स्मा‚{\tiny $_{lb}$}‚द्विष‚याद्\edtext{}{\lemma{याद्}\Bfootnote{०याद् ज्ञान० \cite{dp-msA} \cite{dp-msC} \cite{dp-edP} \cite{dp-edH} \cite{dp-edE} \cite{dp-edN}}} विज्ञान‚मुदेति त‚द्विष‚य‚स‚दृशं त‚द् भ‚व‚ति । य‚था नीलादुत्प‚द्य‚मानं नील‚स‚दृश‚म् । त‚च्च\edtext{}{\lemma{च्च}\Bfootnote{त‚च्च सादृ० \cite{dp-msA} \cite{dp-edP} \cite{dp-edE}}}‚{\tiny $_{lb}$}‚ सारूप्यं सादृश्य‚म् आकार इत्याभास इत्य‚पि व्य‚प‚दिश्य‚ते ॥
	\pend% ending standard par
       ‚{\tiny $_{lb}$}‚ 

	  \pstart \leavevmode% starting standard par
	न‚नु च ज्ञानाद‚व्य‚तिरिक्तं सादृश्य‚म् । त‚था च स‚ति त‚देव ज्ञानं प्र‚माणं त‚देव च\edtext{}{\lemma{च}\Bfootnote{च नास्ति \cite{dp-msA} \cite{dp-msB} \cite{dp-msC} \cite{dp-edP} \cite{dp-edH} \cite{dp-edE} \cite{dp-edN}}}‚{\tiny $_{lb}$}‚ प्र‚माण‚फ‚ल‚म् । न चैकं व‚स्तु साध्यं साध‚नं चोप‚प‚द्य‚ते । त‚त् क‚थं सारूप्यं प्र‚माण‚मित्याह--
	\pend% ending standard par
       ‚{\tiny $_{lb}$}‚ 
	  \bigskip
	  \begingroup
	

	  \pstart \leavevmode% starting standard par
	त‚द्व‚शाद‚र्थ‚प्र‚तीतिसिद्धेरिति ॥ २१ ॥
	\pend% ending standard par
       ‚{\tiny $_{lb}$}‚ 
	    
	    \pstart
	    \begin{center}
	  ॥ \footnote{प्र‚थ‚मः प‚रिच्छेदः \cite{dp-msB} \cite{dp-msC}}‚प्र‚त्य‚क्ष‚प‚रिच्छेदः ॥
	    \end{center}
	    \pend
	  
	  
	  \endgroup
	
	  \endgroup
	‚{\tiny $_{lb}$}‚

	  \pstart \leavevmode% starting standard par
	किम‚न्य‚त्प्र‚माणं भ‚विष्य‚ति इत्यागूर्य पूर्व‚प‚क्ष‚वादी \textbf{य‚दीत्या}द्याह । \textbf{य‚दी}ति स‚म्भाव‚य‚ति । \textbf{त‚र्हि}‚{\tiny $_{lb}$}‚श‚ब्दोऽक्ष‚मायाम् । \textbf{त‚देव} प्र‚त्य‚क्षं ज्ञान‚मेव‚मि\edtext{}{\lemma{मि}\Bfootnote{मेवे}}ति न क्ष‚म्य‚त एत‚दित्य‚र्थः । \textbf{प्र‚माण}स्य \textbf{फ‚लं}‚{\tiny $_{lb}$}‚ साध्य‚म् । \textbf{त‚र्हि} त‚स्मिन् काले \textbf{किम्प्र‚माण‚मिति} योज्य‚म् ।
	\pend% ending standard par
      ‚{\tiny $_{lb}$}‚

	  \pstart \leavevmode% starting standard par
	अत्रा\textbf{र्थ‚सारूप्य‚मि}त्युत्त‚रं व्याच‚ष्टे--अ\leavevmode\ledsidenote{\textenglish{35a/ms}}र्थेनेति । \textbf{अर्थेन} विष‚येण । विष‚य‚सारूप्यं‚{\tiny $_{lb}$}‚ च \textbf{ज्ञान‚स्य} प्र‚त्य‚क्षाख्य‚स्य विष‚य‚स‚मानाकार‚त‚योत्पादः । \textbf{इहे}त्यादिना सारूप्य‚मेवोप‚पाद‚य‚ति ।‚{\tiny $_{lb}$}‚ एत‚च्च \textbf{नील‚स‚दृश‚मित्य‚न्तं} सुबोध‚म् । केव‚ल‚मेवं व‚द‚तोऽय‚माश‚यः--अनेक‚प्राग्भावेनोद‚य‚मान‚म‚पि‚{\tiny $_{lb}$}‚ विज्ञान‚म‚र्थ‚स्यैवाकारं बिभ‚र्ति, नान्य‚स्येत्य‚नुभ‚व‚सिद्ध‚म‚प‚र्य‚नुयोज्य‚म्, स‚दृश‚त्व‚निश्च‚य‚स्य‚{\tiny $_{lb}$}‚ ताद्रूप्येण स‚त‚त‚मुद‚यात् । य‚दि हि त‚द‚न्याकारो विष‚यः स्यात् त‚दा त‚दित‚र‚स्याकार‚धारि‚{\tiny $_{lb}$}‚ विज्ञानं क‚दाचिज्ज‚न‚येत् । य‚था शुक्तिः क‚थ‚ञ्चिद्र‚ज‚ताकार‚ज्ञान‚प्र‚ब‚न्धोद‚येऽपि त‚द्देशोप‚सृष्ट‚स्य‚{\tiny $_{lb}$}‚ स्वाकारानुकारि ज्ञानं ज‚न‚य‚ति । न चाभ्रान्त‚स्य नील‚ज्ञान‚स्य क‚दाचिद‚प्य‚न्याकार‚त्व‚म‚स्ति ।‚{\tiny $_{lb}$}‚ त‚स्माद‚र्थोऽप्येव‚माकार इति निश्चीय‚ते । बाह्यार्थेऽर्थ‚सारूप्याव‚ग‚मे ग‚तिरिय‚मेवेति ।
	\pend% ending standard par
      ‚{\tiny $_{lb}$}‚

	  \pstart \leavevmode% starting standard par
	न‚नु चान्य‚त्र विष‚याभासः प्र‚माण‚मुक्त‚स्त‚थाविष‚याकारः, इह त्व‚र्थ‚सारूप्य‚म् । त‚त्क‚थं न‚{\tiny $_{lb}$}‚ व्याघात इत्याश‚ङ्क्याह--\textbf{त‚च्चेति । चो} य‚स्माद‚व‚धार‚णे वा । \textbf{इत्य‚प्य}नेनापि श‚ब्देन ।‚{\tiny $_{lb}$}‚ अर्थ‚सारूप्य‚मेव तेन तेन श‚ब्देनाभिहित‚म् । त‚तो न व्याघात इत्य‚भिप्रायः ॥
	\pend% ending standard par
      ‚{\tiny $_{lb}$}‚

	  \pstart \leavevmode% starting standard par
	\textbf{किं त‚र्हि प्र‚माण‚मिति} पृच्छ‚ता य‚च्चेत‚सि निहित‚मासीत् त‚दिदानीं न‚नु \textbf{चे}त्यादिना‚{\tiny $_{lb}$}‚ क‚ण्ठोक्तं क‚रोति । एत‚च्च \textbf{प्र‚माण‚मित्येत}द‚न्तं सुबोध‚म् ।
	\pend% ending standard par
      ‚{\tiny $_{lb}$}‚‚{\tiny $_{lb}$}‚\textsuperscript{\textenglish{82/dm}}‚{\tiny $_{lb}$}‚
	  \bigskip
	  \begingroup
	

	  \pstart \leavevmode% starting standard par
	त‚द्व‚शादिति । त‚दिति सारूप्य‚म्, त‚स्य व‚शात् सारूप्य‚साम‚र्थ्यात् । अर्थ‚स्य‚{\tiny $_{lb}$}‚ प्र‚तीतिः अव‚बोधः । त‚स्याः सिद्धिः । \edtext{\textsuperscript{*}}{\lemma{*}\Bfootnote{त‚तः सिद्धेः \cite{dp-msB}}}त‚त्सिद्धेः कार‚णात् । अर्थ‚स्य प्र‚तीतिरूपं‚{\tiny $_{lb}$}‚ प्र‚त्य‚क्षं विज्ञानं सारूप्य‚व‚शात् सिद्ध्य‚ति प्र‚तीतं भ‚व‚तीत्य‚र्थः । नील‚निर्भासं हि विज्ञानं‚{\tiny $_{lb}$}‚ य‚तः, त‚स्मात् नील‚स्य प्र‚तीतिर‚व‚सीय‚ते । \unclear{ये}भ्यो हि च‚क्षुरादिभ्यो\edtext{}{\lemma{क्षुरादिभ्यो}\Bfootnote{भ्यो ज्ञानं \cite{dp-msD} \cite{dp-msB}}} विज्ञान‚मुत्प‚द्य‚ते न त‚द्व‚शात्‚{\tiny $_{lb}$}‚ त‚ज्ज्ञानं नील‚स्य संवेद‚नं श‚क्य‚तेऽव‚स्थाप‚यितुम्\edtext{}{\lemma{यितुम्}\Bfootnote{साधार‚ण‚त्वाच्च‚क्षुरादीनाम्--\cite{dp-msD-n}}} । नील‚स‚दृशं तु अनुभूय‚मानं नील‚स्य संवेद‚न‚{\tiny $_{lb}$}‚म‚व‚स्थाप्य‚ते ।
	\pend% ending standard par
       ‚{\tiny $_{lb}$}‚ 

	  \pstart \leavevmode% starting standard par
	न चात्र ज‚न्य‚ज‚न‚क‚भाव‚निब‚न्ध‚नः, साध्य‚साध‚न‚भावः, येनैक‚स्मिन् व‚स्तुनि विरोधः‚{\tiny $_{lb}$}‚ स्यात् । अपि तु व्य‚व‚स्थाप्य‚व्य‚व‚स्थाप‚न‚भावेन\edtext{}{\lemma{भावेन}\Bfootnote{स्थाप‚क‚भा० \cite{dp-msA} \cite{dp-msB} \cite{dp-msC} \cite{dp-msD} \cite{dp-edP} \cite{dp-edE} \cite{dp-edH} \cite{dp-edN}}} । त‚त एक‚स्य व‚स्तुनः किञ्चिद्रूपं प्र‚माणं‚{\tiny $_{lb}$}‚ किञ्चित् प्र‚माण‚फ‚लं न विरुध्य‚ते ।
	\pend% ending standard par
      
	  \endgroup
	‚{\tiny $_{lb}$}‚

	  \pstart \leavevmode% starting standard par
	एत‚स्मिन् पूर्व‚प‚क्षे \textbf{त‚द्व‚शादि}त्याद्युत्त‚र‚माचार्यीयं विवृणोति \textbf{त‚दि}ति । त‚च्छ‚ब्दार्थं‚{\tiny $_{lb}$}‚ \textbf{व‚श}श‚ब्दार्थं \add{च} स्फुट‚य‚ति \textbf{सारूप्येति । अर्थ‚स्येति} नील‚पीताद्यात्म‚ना विशिष्ट‚स्येति द्र‚ष्ट‚व्य‚म् ।‚{\tiny $_{lb}$}‚ य‚तः प्र‚मितिरियं विशिष्टेनैव क‚र्म‚णाऽव‚च्छिन्ना प्र‚तीय‚ते । अमुमेवार्थ‚म्--\textbf{अर्थ‚स्येत्या}दिना‚{\tiny $_{lb}$}‚ स्प‚ष्ट‚य‚ति । \textbf{सिद्धि}श‚ब्द‚स्यार्थ‚म‚भिव्य‚न‚क्ति \textbf{प्र‚तीत‚मिति । प्र‚तीतं} त‚न्नील‚स्येदं ज्ञानं न पीत‚स्येत्या‚{\tiny $_{lb}$}‚द्याकारेण निश्चितं भ‚व‚तीत्य‚र्थः । उक्त‚म‚र्थं \textbf{नीले}त्यादिनोप‚संह‚र‚ति । \textbf{हिर}व‚धार‚णे । य‚त‚{\tiny $_{lb}$}‚ एवं त‚त्त‚स्मात् नील‚स्य प्र‚तीतिर्नील‚स्यैवेदं ज्ञान‚मित्य‚व‚सीय‚ते ।
	\pend% ending standard par
      ‚{\tiny $_{lb}$}‚

	  \pstart \leavevmode% starting standard par
	न‚नु ज‚न‚केभ्य एवेयं ज्ञान‚स्य व्य‚व‚स्था भ‚विष्य‚ति । ते हि ज्ञानं ज‚न‚यितुं श‚क्ताः‚{\tiny $_{lb}$}‚ किम‚ङ्ग निश्चाय‚यितुं न श‚क्नुयुः ? अत‚श्च न सारूप्याधीनोऽयं प्र‚तीतिनिय‚म इत्याश‚ङ्क्याह—‚{\tiny $_{lb}$}‚येभ्य इति । \textbf{हि}र्य‚स्मात् । एवं ब्रुव‚तोऽयं भावः--नेन्द्रियालोकौ नियाम‚कौ । त‚योः स‚र्व‚ज्ञान‚{\tiny $_{lb}$}‚साधार‚ण‚त्वात् । अर्थ‚ग‚तोऽपि विशेषो न नियाम‚कः । ज्ञानादेव हि स प्र‚त्येत‚व्यः । त‚द‚विशेषे स‚{\tiny $_{lb}$}‚ क‚थं विशेष‚व्य‚व‚स्थाया अङ्गं भ‚वेदिति । नील‚सारूप्येऽपि य‚द्य‚यं न्याय‚स्त‚दा क‚थं त‚स्यापि निया‚{\tiny $_{lb}$}‚म‚क‚त्व‚मित्याह--\textbf{नीलेति} । तुरिन्द्रियादिभ्यो ज‚न‚केभ्यः सारूप्यं भेद‚व‚द् द‚र्श‚य‚ति । \textbf{नील‚{\tiny $_{lb}$}‚स‚दृश‚म‚नुभूय‚मानं} संवेद‚नं ज्ञानं \textbf{नील‚स्य} नील‚स्यैवेति व्य‚व‚स्थाप्य‚ते ।
	\pend% ending standard par
      ‚{\tiny $_{lb}$}‚

	  \pstart \leavevmode% starting standard par
	अय‚म‚त्र प्र‚क‚र‚णार्थः--य‚दि ज्ञान‚म‚र्थ‚स‚रूपं न स्यात् किन्तु निराकारं बोधैक‚रूपं त‚दाऽ‚{\tiny $_{lb}$}‚नुभ‚वैक‚रूप‚त‚या त‚द‚विशिष्टं, स‚र्व‚त्र प‚रिच्छेद्य‚त‚या क‚र्म‚स्थान‚प्राप्ते नील‚पीतादाविति नील‚{\tiny $_{lb}$}‚स्यैवेदं संवेद‚न‚म्, इदं पीत‚स्यैवेत्य‚नुभ‚व‚सिद्धः प्र‚तिक‚र्म‚विभागो हीयेत । \leavevmode\ledsidenote{\textenglish{35b/ms}}अर्थ‚ग‚त‚श्चाकारो‚{\tiny $_{lb}$}‚ ज्ञानाधीन‚प्र‚तिप‚त्तित‚या ज्ञान‚स्य विशिष्ट‚रूप‚तास‚न्देहेन स‚न्दिग्धः । न च तेनैव स‚न्दिग्ध‚रूपेण‚{\tiny $_{lb}$}‚ त‚देव स‚न्दिग्धं रूपं निश्चेतुं श‚क्य‚म् । ज्ञान‚ग‚त‚श्च विशेषोऽर्थ‚कृतः सारूप्याद‚न्यो नोप‚प‚द्य‚ते ।‚{\tiny $_{lb}$}‚ स‚न्न‚पि न ताव‚दिदंत‚याऽसौ निर्देष्टुं श‚क्यः । न चानिरूपितेन त‚दात्म‚कः क‚र्म‚निय‚म‚निश्च‚यः ।‚{\tiny $_{lb}$}‚ य‚द्रूप‚श्च य‚श्च क‚र्म‚निय‚म‚निश्च‚यो ज्ञान‚स्य त‚स्मिन्न‚निरूपिते कीदृशी त‚द्रूप‚ताव्य‚व‚स्था ? य‚था‚{\tiny $_{lb}$}‚ ‚{\tiny $_{lb}$}‚ \leavevmode\ledsidenote{\textenglish{83/dm}}‚{\tiny $_{lb}$}‚ 
	  
	व्य‚व‚स्थाप‚न‚हेतुर्हि सारूप्यं त‚स्य ज्ञान‚स्य । व्य‚व‚स्थाप्यं च नील‚संवेद‚न‚रूप‚म् । ‚{\tiny $_{lb}$}‚ 
	  
	व्य‚व‚स्थाप्य‚व्य‚व‚स्थाप‚क‚भावोऽपि क‚थ‚मेक‚र‚य ज्ञान‚स्येति\edtext{}{\lemma{स्येति}\Bfootnote{ज्ञान‚स्य चेत् \cite{dp-msA} \cite{dp-msC} \cite{dp-edP}}} चेत् । उच्य‚ते । नील‚स‚दृश‚{\tiny $_{lb}$}‚म‚नुभूय\edtext{}{\lemma{नुभूय}\Bfootnote{०ते । स‚दृश‚म‚नुभूय‚मानं त‚द्वि० \cite{dp-msA} \cite{dp-msB} \cite{dp-edP} \cite{dp-edH} \cite{dp-edE} \cite{dp-edN} ०ते । स‚दृश‚म‚नुभूय त‚द्वि० \cite{dp-msC} \cite{dp-msD}}} त‚द्विज्ञानं य‚तो नील‚स्य ग्राह‚क‚म‚व‚स्थाप्य‚ते निश्च‚य‚प्र‚त्य‚येन, त‚स्मात् सारूप्य‚म‚नुभूतं‚{\tiny $_{lb}$}‚ व्य‚व‚स्थाप‚न‚हेतुः । निश्च‚य‚प्र‚त्य‚येन च त‚ज्ज्ञानं नील‚संवेद‚न‚म‚व‚स्थाप्य‚मानं व्य‚व‚स्थाप्य‚म् । ‚{\tiny $_{lb}$}‚ 
	  
	त‚स्माद‚सारूप्य‚व्यावृत्त्या सारूप्यं ज्ञान‚स्य व्य‚व‚स्थाप‚न‚हेतुः । अनील‚बोध‚व्यावृत्त्या च‚{\tiny $_{lb}$}‚ नील‚बोध‚रूप‚त्वं व्य‚व‚स्थाप्य‚म् ।‚{\tiny $_{lb}$}‚ नीलादिर‚निरूपिते क्ष‚णिक‚त्वे त‚द्रूपो न निरूप्य‚त इति । त‚स्माद् य‚त इय‚म‚धिग‚तिर‚व्य‚व‚धानात्‚{\tiny $_{lb}$}‚ त‚त्त्वं प्र‚तिल‚भ‚ते त‚देवान्येनाव्य‚व‚धीय‚मान‚व्यापारं स्व‚भेदेन भेद‚कं प्र‚माक‚र‚णं प्र‚माण‚म् । न‚{\tiny $_{lb}$}‚ पुन‚र‚नेनैव व्य‚व‚धीय‚मान‚व्यापार‚मिति । त‚च्चेदृशं सारूप्य‚मेवेति । एष च वादोऽस्माभि‚{\tiny $_{lb}$}‚र्विस्त‚रेण \textbf{विशेषाख्यानेऽभि}हित इति त‚तोऽप्य‚पेक्षित‚व्य इति ।
	\pend% ending standard par
      ‚{\tiny $_{lb}$}‚

	  \pstart \leavevmode% starting standard par
	न‚नु चैव‚म‚प्येक‚स्यैव ज्ञान‚स्य साध्य‚त्वं साध‚न‚त्व‚ञ्च क‚थ‚मुप‚प‚द्य‚ते ? न‚हि प‚र‚शुरेव च्छिदा‚{\tiny $_{lb}$}‚ भ‚व‚ति । त‚तः प्र‚माण‚फ‚ल‚योर‚भेदो नाभ्युपैत‚व्य इत्याश‚ङ्क्याह--\textbf{न चे}त्यादि । \textbf{चो}‚{\tiny $_{lb}$}‚ य‚स्माद‚र्थे । \textbf{अत्रे}ति प्र‚माण‚फ‚ल‚चिन्तायां \textbf{विरोधोऽ}नुप‚प‚त्तिः स्यात् । य‚द्येवं न भ‚व‚ति क‚थं‚{\tiny $_{lb}$}‚ नामेत्याह--\textbf{अपि तु} किन्तु । \textbf{व्य‚व‚स्थाप्यं} विशेष‚रूपेण नियाम्य‚म् । व्य‚व‚स्थाप्य‚ते विशिष्टे‚{\tiny $_{lb}$}‚नात्म‚ना निय‚म्य‚तेऽनेनेति व्य‚व‚स्थानिमित्त‚म् व्य‚व‚स्थाप‚न‚म‚भिप्रेत‚म् । \textbf{व्य‚व‚स्थाप‚न‚भावेने}त्य‚यं‚{\tiny $_{lb}$}‚ पाठो व‚क्ष्य‚माण‚विरोधी । य‚दा तु \textbf{व्य‚व‚स्थाप‚क‚भावेने}ति पाठो दृश्य‚ते त‚दा क‚र‚णे क‚र्त्तृ‚{\tiny $_{lb}$}‚भाव‚विव‚क्ष‚या त‚था द्र‚ष्ट‚व्य‚म् । साध्व‚सिश्च्छिन‚त्तीति य‚था । त‚यो\textbf{र्व्य‚व‚स्थाप्य‚व्य‚व‚स्थाप‚न‚यो‚{\tiny $_{lb}$}‚र्भा}व‚स्त‚थाज्ञानाभिधान‚निमित्तं रूपं तेन । इत्थंभूत‚ल‚क्ष‚णा चेयं तृतीया । य‚त एवं \textbf{त‚त}स्त‚{\tiny $_{lb}$}‚स्मा\textbf{देक‚स्य व‚स्तुनः} प्र‚त्य‚क्ष‚ल‚क्ष‚ण‚स्य ज्ञान‚स्य व‚स्तुनः \textbf{किञ्चिद्रूपं} व्यावृत्तिप‚रिक‚ल्पितं कृत‚क‚त्वा‚{\tiny $_{lb}$}‚दिव‚त् । त‚त्रेयं ब्राह्यार्थे प्र‚माणादिव्य‚व‚स्था । ग्राह्याकारोऽर्थ‚सारूप्याख्य आत्म‚नः संविद‚म‚र्थ‚{\tiny $_{lb}$}‚संविद‚माद‚र्श‚य‚न्न‚र्थे प्र‚माण‚म् । अत एव बाह्योऽर्थः प्र‚मेयः । व‚स्तुतः स्व‚विद‚पीय‚म‚र्थ‚स‚म्ब‚न्धिनी‚{\tiny $_{lb}$}‚ व्य‚व‚सीय‚मानाऽर्थ‚संवित्तिः फ‚ल‚मिति ।
	\pend% ending standard par
      ‚{\tiny $_{lb}$}‚

	  \pstart \leavevmode% starting standard par
	न‚नु किम‚त्र व्य‚व‚स्थाप‚न‚निमित्त‚म्, किञ्च व्य‚व‚स्थाप्यं येनैत‚त् स्यादित्याह--\textbf{व्य‚व‚स्था‚{\tiny $_{lb}$}‚प‚नेति । व्य‚व‚स्थाप‚नं} व्य‚व‚स्थाकार‚ण‚म् । व्य‚व‚स्थायां प्र‚योज‚क‚व्यापार इति याव‚त् । त‚स्य‚{\tiny $_{lb}$}‚ हेतुर्निमित्त‚म् । हिर्य‚स्मात् । \textbf{त‚स्य} प्र‚त्य‚क्ष‚ज्ञान‚स्य । किन्त‚र्हि व्य‚व‚स्थाप्य‚मिति ? \textbf{चो‚{\tiny $_{lb}$}‚ व्य‚व‚स्था}प‚न‚हेतोर्व्य‚व‚स्थाप्यं भिन‚त्ति । नील‚स्येदं नान्य‚स्येत्य‚नेनाकारेण य\textbf{न्नील‚संवेद‚नं} त\textbf{द्रूप‚म् ।}
	\pend% ending standard par
      ‚{\tiny $_{lb}$}‚

	  \pstart \leavevmode% starting standard par
	एत‚द‚स‚ह‚मानो \textbf{व्य‚व‚स्थाप्ये}त्याद्याह । न केव‚ल‚मेक‚स्य ज‚न्य‚ज‚न‚क‚भावोऽनुप‚प‚न्न इत्य‚पिना‚{\tiny $_{lb}$}‚ द‚र्श‚य‚ति । \textbf{उच्य‚त} इत्यादिना प‚रिह‚र‚ति । \textbf{य‚तो} य‚स्मा\textbf{न्नील‚स‚दृश‚म‚नुभूये}ति वास्त‚वं रूप‚म‚नू‚{\tiny $_{lb}$}‚दित‚म् । न तु नील‚स‚दृश‚म‚नुभ‚वामीति निश्च‚योऽस्ति । अपि तु नील‚मेवानुभ‚वामीति \textbf{नील‚स्य‚{\tiny $_{lb}$}‚ ग्राह‚क‚म‚व‚स्थाप्य‚ते} ।
	\pend% ending standard par
      ‚{\tiny $_{lb}$}‚‚{\tiny $_{lb}$}‚\textsuperscript{\textenglish{84/dm}}‚{\tiny $_{lb}$}‚
	  \bigskip
	  \begingroup
	

	  \pstart \leavevmode% starting standard par
	व्य‚व‚स्थाप‚क‚श्च विक‚ल्प‚प्र‚त्य‚यः प्र‚त्य‚क्ष‚ब‚लोत्प‚न्नो द्र‚ष्ट‚व्यः । न तु\edtext{}{\lemma{तु}\Bfootnote{न च निर्वि० \cite{dp-msC}}} निर्विक‚ल्प‚क‚त्वात्‚{\tiny $_{lb}$}‚ प्र‚त्य‚क्ष‚मेव नील‚बोध‚रूप‚त्वेना\edtext{}{\lemma{त्वेना}\Bfootnote{रूप‚त्वं नात्मा० \cite{dp-msC}}}त्मान‚म‚व‚स्थाप‚यितुं श‚क्नोति । निश्च‚य‚प्र‚त्य‚येनाव्य‚व‚स्थापितं‚{\tiny $_{lb}$}‚ स‚द‚पि नील‚बोध‚रूपं विज्ञान‚म‚स‚त्क‚ल्प‚मेव । त‚स्मान्निश्च‚येन नील‚बोध‚रूपं व्य‚व‚स्थापितं विज्ञानं‚{\tiny $_{lb}$}‚ नील‚बोधात्म‚ना स‚द् भ‚व‚ति ।
	\pend% ending standard par
       ‚{\tiny $_{lb}$}‚ 

	  \pstart \leavevmode% starting standard par
	त‚स्माद‚ध्य‚व‚सायं कुर्व‚देव प्र‚त्य‚क्षं प्र‚माणं भ‚व‚ति । अकृते त्व‚ध्य‚व‚साये नील‚बोध‚रूप‚त्वे‚{\tiny $_{lb}$}‚नाव्य‚व‚स्थापितं भ‚व‚ति विज्ञान‚म् । त‚था च प्र‚माण‚फ‚ल‚म‚र्थाधिग‚म\edtext{}{\lemma{म}\Bfootnote{०रूप‚त्व‚म् \cite{dp-msA} \cite{dp-msC} \cite{dp-msD} \cite{dp-edP} \cite{dp-edH} \cite{dp-edE}}}रूप‚म‚निष्प‚न्न‚म् । अतः‚{\tiny $_{lb}$}‚ साध‚क‚त‚म‚त्वाभावात् प्र‚माण‚मेव न स्याज्ज्ञान‚म् ।
	\pend% ending standard par
      
	  \endgroup
	‚{\tiny $_{lb}$}‚

	  \pstart \leavevmode% starting standard par
	न‚नु न ताव‚ता ज्ञान‚मात्म‚नैवात्मानं\edtext{}{\lemma{नैवात्मानं}\Bfootnote{इतिक‚र‚णेनाभ्युप‚ग‚म‚स्य स्व‚रूपं चेत्य‚नेनाभ्युप‚ग‚मं द‚र्श‚य‚ति । इति पाठोऽष्ट‚म‚{\tiny $_{lb}$}‚पंक्तिस‚म्ब‚द्ध‚त्वेन प‚क्तिबाह्य‚भागे लिखितः प्र‚तौ दृश्य‚ते किन्तु त‚स्य कुत्र निवेश इति न‚{\tiny $_{lb}$}‚ ज्ञाय‚ते--सं०}} त‚था\leavevmode\ledsidenote{\textenglish{36a/ms}}व्य‚व‚स्थाप‚य‚ति, अनिश्च‚यात्म‚क‚त्वात् ।‚{\tiny $_{lb}$}‚ न च निश्च‚योऽप्यात्मानं त‚थाऽव‚स्थाप‚यितुं प‚र्य‚वाप्नोति, स्वात्म‚न्य‚विक‚ल्प‚क‚त्वात् । त‚त्केन‚{\tiny $_{lb}$}‚ त‚था व्य‚व‚स्थाप्य‚त इत्याह--\textbf{निश्च‚य‚प्र‚त्य‚येने}ति । निश्च‚यात्म‚क‚ज्ञानेनोत्त‚र‚काल‚भाविना ।‚{\tiny $_{lb}$}‚ तेनाप्य‚नुरूपेणेति द्र‚ष्ट‚व्य‚म् । \textbf{त‚स्मात्सारूप्य‚म‚नुभूतं} स्व‚संवेद‚नेन प्र‚तीतं \textbf{व्य‚व‚स्थाप‚न‚हेतु}र्नील‚स्येदं‚{\tiny $_{lb}$}‚ संवेद‚न‚म्, न पीत‚स्येति निय‚म‚क‚र‚ण‚स्य हेतुर्निमित्त‚म् । इदं च व्य‚व‚स्थाप‚न‚हेतुर्हि \textbf{सारूप्यं त‚स्य‚{\tiny $_{lb}$}‚ ज्ञान}स्येत्य‚स्य स‚म‚र्थ‚न‚म् । य‚दि ज्ञान‚स्य त‚स्य सारूप्य‚मेवं व्य‚व‚स्थाप‚न‚हेतुस्त‚र्हि किं कीदृशं च‚{\tiny $_{lb}$}‚ स‚द् व्य‚व‚स्थाप्य‚मित्याह--\textbf{निश्च‚येति । चो}ऽव‚धार‚णे । \textbf{त‚दि}त्य‚स्मात्प‚रो द्र‚ष्ट‚व्यः । य‚स्यैव‚{\tiny $_{lb}$}‚ नील‚ज्ञान‚स्य सारूप्यं त‚थोक्तं त‚देव ज्ञानं व्य‚व‚स्थाप्यं नील‚स्यैवेत्याकारेण नियाम्य‚मित्य‚र्थः ।
	\pend% ending standard par
      ‚{\tiny $_{lb}$}‚

	  \pstart \leavevmode% starting standard par
	कीदृशं स‚त्त‚था व्य‚व‚स्थाप्य‚मित्याह--\textbf{नीले}ति । पीतादिसंवेद‚न‚व्यावृत्त्या नील‚स्येदं संवेद‚नं‚{\tiny $_{lb}$}‚ ज्ञान‚मि\textbf{त्य‚व‚स्थाप्य‚मानं} निश्चीय‚मान‚म् । केनाव‚स्थाप्य‚मान‚मित्याकाङ्क्षायां \textbf{निश्च‚य‚प्र‚त्य‚येने}ति‚{\tiny $_{lb}$}‚ योज‚नीय‚म् । न‚नु किं ज्ञानात् सारूप्यं व्य‚तिरिच्य‚ते येन सारूप्यं त‚थोक्त‚म्, ज्ञानं त्वेव‚मुच्य‚त‚{\tiny $_{lb}$}‚ इत्याश‚ङ्क्योप‚संहार‚व्याजेनाह--\textbf{त‚स्मादि}ति । य‚स्मात् सारूप्याद‚न्य‚व्य‚व‚स्थाप‚न‚हेतुर्न घ‚ट‚ते,‚{\tiny $_{lb}$}‚ सारूप्यं च ज्ञानाद‚न्यं नोप‚प‚द्य‚ते \textbf{त‚स्मा}त्कार‚णात् । अयं च--\textbf{त‚त एक‚स्य व‚स्तुनः किञ्चिद्रूप‚{\tiny $_{lb}$}‚मि}त्यादेरुप‚प‚त्त्या प्र‚साधित‚स्योप‚संहारः ।
	\pend% ending standard par
      ‚{\tiny $_{lb}$}‚

	  \pstart \leavevmode% starting standard par
	अयं प्र‚क‚र‚णार्थः । एक‚स्यैव ज्ञान‚स्य व्यावृत्तिकृतं भेद‚माश्रित्याह व्य‚व‚स्थाप्य‚व्य‚स्था‚{\tiny $_{lb}$}‚प‚न‚भावः । वास्त‚वं चाभेद‚मुपादाय प्र‚माण‚फ‚ल‚योर‚भेद उच्य‚ते । न च काचित्क्ष‚तिरिति ।
	\pend% ending standard par
      ‚{\tiny $_{lb}$}‚

	  \pstart \leavevmode% starting standard par
	स‚म्प्र‚ति \textbf{निश्च‚य‚प्र‚त्य‚ये}नेति ब्रुव‚ता यादृशो निश्च‚यो विव‚क्षित‚स्तं \textbf{व्य‚व‚स्थाप‚क} इत्यादि‚{\tiny $_{lb}$}‚नोप‚संहार‚व्याजेन स्प‚ष्ट‚य‚ति । व्य‚व‚स्थाप‚य‚तीति \textbf{व्य‚व‚स्थाप}कः । स च त‚द्ब‚लोत्प‚न्नोऽप्य‚नुरूपो‚{\tiny $_{lb}$}‚ द्र‚ष्ट‚व्यः । अन‚नुरूप‚विक‚ल्पेन व्य‚व‚स्थापित‚योर‚पि व्य‚स्व‚थाप्य‚व्य‚व‚स्थाप‚न‚योर‚नुप‚प‚त्तेः । य‚था‚{\tiny $_{lb}$}‚ म‚रीचीर्दृष्ट्वा त‚द्ब‚लोत्प‚न्नेन विक‚ल्पेनाव‚स्थाप्य‚मान‚योर्ज‚ल‚सारूप्य‚ज्ञान‚योर्न त‚थाभावः ।‚{\tiny $_{lb}$}‚ ‚{\tiny $_{lb}$}‚ \leavevmode\ledsidenote{\textenglish{85/dm}}‚{\tiny $_{lb}$}‚ 
	  
	ज‚नितेन त्व‚ध्य‚व‚सायेन सारूप्य‚व‚शान्नील‚बोध‚रूपे ज्ञाने \edtext{}{\lemma{ज्ञाने}\Bfootnote{ज्ञानेऽव‚स्था० \cite{dp-msA} \cite{dp-edP} \cite{dp-edH} \cite{dp-edE} \cite{dp-edN}}}व्य‚व‚स्थाप्य‚माने सारूप्यं व्य‚व‚{\tiny $_{lb}$}‚स्थाप‚न‚हेतुत्वात् प्र‚माणं सिद्धं भ‚व‚ति । ‚{\tiny $_{lb}$}‚ 
	  
	य‚द्येव‚म‚ध्य‚व‚साय‚स‚हित‚मेव प्र‚त्य‚क्षं प्र‚माणं स्यात् न केव‚ल‚मिति चेत् । नैत‚देव‚म् ।‚{\tiny $_{lb}$}‚ य‚स्मात् प्र‚त्य‚क्ष‚ब‚लोत्प‚न्नेनाध्य‚व‚सायेन \edtext{}{\lemma{सायेन}\Bfootnote{दृष्ट‚त्वेना० \cite{dp-msA} \cite{dp-msB} \cite{dp-msC} \cite{dp-msD} \cite{dp-edP} \cite{dp-edH} \cite{dp-edE} \cite{dp-edN}}}दृश्य‚त्वेनार्थोऽव‚सीय‚ते नोत्प्रेक्षित‚त्वेन । द‚र्श‚न‚ञ्चार्थ‚{\tiny $_{lb}$}‚साक्षात्क‚र‚णाख्यं प्र‚त्य‚क्ष‚व्यापारः । उत्प्रेक्ष‚णं तु\edtext{}{\lemma{तु}\Bfootnote{०क्ष‚णं विक० \cite{dp-msA} \cite{dp-msC} \cite{dp-msD}}} विक‚ल्प‚व्यापारः । त‚थाहि प‚रोक्ष‚म‚र्थं‚{\tiny $_{lb}$}‚ येनाभिप्रायेण निश्च‚य‚प्र‚त्य‚य‚स्य व्य‚व‚स्थाप‚क‚त्व‚मुक्तं त‚म‚भिव्य‚न‚क्ति--\textbf{न त्वि}ति । \textbf{तु}र्निश्च‚या‚{\tiny $_{lb}$}‚त्प्र‚त्य‚क्षं भेद‚व‚द् द‚र्श‚य‚ति । अव‚स्थाप‚नाऽश‚क्तौ हेतुमाह--\textbf{निर्विक‚ल्प‚क‚त्वा}दिति ।
	\pend% ending standard par
      ‚{\tiny $_{lb}$}‚

	  \pstart \leavevmode% starting standard par
	न‚नु प‚र‚मार्थ‚तो य‚दि त‚ज्ज्ञानं नील‚मेव रूप‚त‚या स‚रूपं त‚दा त‚थानिश्च‚यो भ‚व‚तु वा मा‚{\tiny $_{lb}$}‚ वा । स्व‚य‚मेवाविक‚ल्प‚क‚त्वेऽपि तेन रूपेण स‚द्व्य‚व‚हार‚गोच‚रो भ‚विष्य‚तीत्याश‚ङ्क्याह—‚{\tiny $_{lb}$}‚\textbf{निश्च‚य‚प्र‚त्य‚येने}त्यादि । निश्च‚य‚प्र‚त्य‚येनानुरूपेण । अन‚नुरूपेणाक्ष‚णिक‚विक‚ल्पेनाव‚स्थापित‚स्यापि‚{\tiny $_{lb}$}‚ क्ष‚णिक‚बोध‚स्य स‚द्व्य‚व‚हार‚स्य योग्य‚त्वात् । तेना\textbf{व्य‚व‚स्थापित‚म‚स‚त्क‚ल्प‚म}स‚त्तुल्य‚म् स्व‚विष‚ये‚{\tiny $_{lb}$}‚ व्य‚व‚हार‚यितुम‚श‚क्त‚त्वा\leavevmode\ledsidenote{\textenglish{36b/ms}}त् । ग‚च्छ‚त्तृण‚ज्ञानादिव‚त् ।
	\pend% ending standard par
      ‚{\tiny $_{lb}$}‚

	  \pstart \leavevmode% starting standard par
	\textbf{त‚स्मादि}त्यादिनाऽमुम‚र्थ‚मुप‚संह‚र‚ति । य‚स्मान्निश्च‚येन त‚थाऽव्य‚व‚स्थापित‚म‚स‚ता स‚दृशं‚{\tiny $_{lb}$}‚ भ‚व‚ति \textbf{त‚स्मात् । नील‚बोधात्म‚ना} नील‚स्यायं बोध इत्याकारेण \textbf{स‚द् भ‚व‚ति} स‚द्व्य‚व‚हार‚योग्यं‚{\tiny $_{lb}$}‚ भ‚व‚ति । य‚स्मात्स्व‚साम‚र्थ्योत्प‚न्नेनानुरूपेण निश्च‚य‚प्र‚त्य‚येन व्य‚व‚स्थापितं त‚था भ‚व‚ति नान्य‚था‚{\tiny $_{lb}$}‚ \textbf{त‚स्मा}त्कार‚णा\textbf{द‚ध्य‚व‚साय}म‚नुरूपं निश्च‚यं \textbf{कुर्व‚त् । एव}कारेणाऽक‚र‚णाव‚स्थायाः प्र‚माण‚व्य‚व‚हारं‚{\tiny $_{lb}$}‚ निर‚स्य‚ति । अमुमेवार्थं व्य‚तिरेक‚मुखेन द्र‚ढ‚य‚न्नाह--\textbf{अकृते त्वि}ति । \textbf{तुः} क‚र‚णाव‚स्थां भेद‚व‚तीमाह ।
	\pend% ending standard par
      ‚{\tiny $_{lb}$}‚

	  \pstart \leavevmode% starting standard par
	भ‚व‚तु त‚थाऽव्य‚व‚स्थापितं किम‚त इत्याह--\textbf{त‚था चे}ति । त‚स्मिंश्च तेनात्म‚नाऽव्य‚व‚स्था‚{\tiny $_{lb}$}‚प‚न‚प्र‚कारे स‚ति । त‚द‚निष्प‚त्ताव‚पि किं न प्र‚माण‚मित्याह--\textbf{अत} इति । \textbf{अतो}ऽधिग‚तिक्रियायाः‚{\tiny $_{lb}$}‚ फ‚ल‚भूताया अनिष्प‚त्तेः । \textbf{साध‚क‚त‚म‚त्वाभाव} इति वाक्य‚भेदः । त‚स्मात्सा\textbf{ध‚क‚त‚म‚त्वाभावात्} ।‚{\tiny $_{lb}$}‚ अर्थाधिग‚म‚ल‚क्ष‚ण‚प्र‚माण‚सिद्धौ हि साध‚क‚त‚मं प्र‚माण‚मुच्य‚ते । सा चेन्न निष्प‚न्ना त‚र्हि किम‚पेक्ष्य‚{\tiny $_{lb}$}‚ साध‚क‚त‚म‚त्व‚मात्म‚सात्कुर्यात्, येन त‚ज्ज्ञानं प्रामाण्य‚म‚श्नुवीतेति स‚मुदायार्थः ।
	\pend% ending standard par
      ‚{\tiny $_{lb}$}‚

	  \pstart \leavevmode% starting standard par
	न‚नु अव‚सायाभावे ताव‚दियं ग‚तिस्त‚द्भावेऽपि य‚द्येषैव ग‚तिस्त‚दा न प्र‚त्य‚क्षं नाम प्र‚माण‚{\tiny $_{lb}$}‚मित्याश‚ङ्क्यान्व‚य‚मुखेनेदानीमाह--\textbf{ज‚निते}नेति । \textbf{तु}र‚ज‚न‚नाव‚स्थाया ज‚न‚नाव‚स्थां भेद‚व‚तीं‚{\tiny $_{lb}$}‚ द‚र्श‚य‚ति । तेन त‚थाव्य‚व‚स्थाप्य‚माने व्य‚व‚स्थाप‚क‚त्वे च निमित्त‚माह--सा\textbf{रूप्य‚व‚शादिति ।‚{\tiny $_{lb}$}‚ सिद्धं} प्र‚तीतं भ‚व‚ति ।
	\pend% ending standard par
      ‚{\tiny $_{lb}$}‚

	  \pstart \leavevmode% starting standard par
	य‚दि प्र‚त्य‚क्षं केव‚ल‚म‚स‚हायं प्र‚व‚र्त्त‚यितुम‚नीशानं निय‚मेन निश्च‚य‚म‚पेक्ष‚ते त‚र्हि त‚त्स‚हित‚मेव‚{\tiny $_{lb}$}‚ प्र‚माणं प्र‚स‚ज्येतेत्य‚भिप्राय‚वान् प्राह--\textbf{य‚द्येव}मिति । \textbf{एव}म‚न‚न्त‚रोक्तं \textbf{य‚द्य‚भ्यु}प‚ग‚म्य‚ते त‚दैवं स्यात् ।‚{\tiny $_{lb}$}‚ \textbf{चेदि}ति प‚राभ्युप‚ग‚मं द‚र्श‚य‚ति । नेति प्र‚तिषेध‚ति । एवं स‚त्येत‚न्न भ‚व‚ति । हेतुमाह—‚{\tiny $_{lb}$}‚‚{\tiny $_{lb}$}‚ ‚{\tiny $_{lb}$}‚ \leavevmode\ledsidenote{\textenglish{86/dm}}‚{\tiny $_{lb}$}‚ 
	  
	विक‚ल्प‚य‚न्त उत्प्रेक्षाम‚हे न तु प‚श्याम इति उत्प्रेक्षात्म‚कं विक‚ल्प‚व्यापार‚म‚नुभ‚वाद‚ध्य‚व‚स्य‚न्ति\edtext{}{\lemma{न्ति}\Bfootnote{भ‚वाद‚व‚स्य‚न्ति \cite{dp-msA} \cite{dp-msC} \cite{dp-edP} \cite{dp-edH} \cite{dp-edE} \cite{dp-edN}}} ।‚{\tiny $_{lb}$}‚ त‚स्मात् स्व‚व्यापारं तिर‚स्कृत्य प्र‚त्य‚क्ष‚व्यापार‚माद‚र्श‚य‚ति य‚त्रार्थे प्र‚त्य‚क्ष‚पूर्व‚कोऽध्य‚व‚साय‚स्त‚त्र‚{\tiny $_{lb}$}‚ प्र‚त्य‚क्षं केव‚ल‚मेव प्र‚माण‚मिति ॥ ‚{\tiny $_{lb}$}‚ 
	  
	॥ \edtext{\textsuperscript{*}}{\lemma{*}\Bfootnote{प्र‚माण‚मिति न्याय‚बिन्दुटीकायां प्र‚थ‚मः प‚रिच्छेदः स‚माप्तः ॥ म‚ङ्ग‚ल‚म‚स्तु ॥ \cite{dp-msA}‚{\tiny $_{lb}$}‚ प्र‚माण‚मिति आचार्य‚ध‚र्मोत्त‚र‚विर‚चितायां न्याय‚बिन्दुटीकायां प्र‚थ‚मः प‚रिच्छेदः स‚माप्तः । \cite{dp-msC}}}आचार्य‚ध‚र्मोत्त‚र‚विर‚चितायां न्याय‚बिन्दुटीकायां प्र‚त्य‚क्ष‚प‚रिच्छेदः प्र‚थ‚मः ॥‚{\tiny $_{lb}$}‚ \textbf{य‚स्मादि}ति । \textbf{दृश्य‚त्वेन} द‚र्श‚न‚विशिष्ट‚त्वेन । प्र‚कारान्त‚रं निर‚स्य‚ति \textbf{नेति} । उत्प्रेक्षित‚त्वं‚{\tiny $_{lb}$}‚ साक्षात्क‚र‚णाभिमान‚शून्य‚ज्ञातृत्व‚म् ।
	\pend% ending standard par
      ‚{\tiny $_{lb}$}‚

	  \pstart \leavevmode% starting standard par
	न‚नु विक‚ल्प‚व्यापारो द‚र्श‚न‚म् । त‚तः स्व‚व्यापार‚विशिष्टार्थाध्य‚व‚साने त‚स्य का‚{\tiny $_{lb}$}‚ क्ष‚तिरित्याश‚ङ्क्याह--\textbf{द‚र्श‚न‚ञ्चे}ति । \textbf{चो} य‚स्माद‚र्थे । य‚द्येवं क‚स्त‚र्हि विक‚ल्प‚व्यापार इत्याह—‚{\tiny $_{lb}$}‚\textbf{उत्प्रेक्ष‚ण}मिति ।
	\pend% ending standard par
      ‚{\tiny $_{lb}$}‚

	  \pstart \leavevmode% starting standard par
	एत‚देव \textbf{त‚था}हीत्यादिनाऽ\textbf{व‚स्य‚न्तीत्य‚नेन\edtext{}{\lemma{नेन}\Bfootnote{त्य‚न्तेन}}} ग्र‚न्थेन स‚म‚र्थ‚य‚ति ।
	\pend% ending standard par
      ‚{\tiny $_{lb}$}‚

	  \pstart \leavevmode% starting standard par
	भ‚व‚त्वेवं त‚थापि क‚थं प्र‚त्य‚क्ष‚स्य केव‚ल‚स्य प्रामाण्यं न तु विक‚ल्प‚स्यापीत्याश‚ङ्क्यो‚{\tiny $_{lb}$}‚प‚संहाराप‚देशेनाह--\textbf{त‚स्मादि}ति । एत‚च्च सुबोध‚म् ।
	\pend% ending standard par
      ‚{\tiny $_{lb}$}‚

	  \pstart \leavevmode% starting standard par
	एव‚म‚भिधाने चाय‚म‚स्याश‚यो बोद्ध‚व्यः । स्यात्ख‚लु विक‚ल्प‚स्यापि प्रामाण्यं य‚द्य‚सौ‚{\tiny $_{lb}$}‚ स्व‚व्यापार‚युक्त एव प्र‚त्य‚क्ष‚स्य साहाय्यं भ‚ज‚ते । न चायं त‚था व्याप्रिय‚ते । स्वीकृत‚प्र‚त्य‚क्ष‚{\tiny $_{lb}$}‚व्यापार‚स्यैव प्र‚वृत्तिद‚र्श‚नात् । स च द‚र्श‚न‚ल‚क्ष‚णो व्यापारः प्र‚त्य‚क्षेणैव स‚म्पादित इति‚{\tiny $_{lb}$}‚ क‚थ‚म‚य‚म‚पि पिष्ट‚पेष‚ण‚कारी प्रामाण्यं प्र‚तिल‚भेत ।
	\pend% ending standard par
      ‚{\tiny $_{lb}$}‚

	  \pstart \leavevmode% starting standard par
	\textbf{अथ} प्र‚त्य‚क्षं ताव‚त्स्व‚प्रामाण्य‚व्य‚व\leavevmode\ledsidenote{\textenglish{37a/ms}}हाराय त‚म‚पेक्ष‚ते । त‚तः सोऽपि प्र‚माण‚मेवेति‚{\tiny $_{lb}$}‚ म‚तिस्त‚र्हि पिता पितृव्य‚व‚हारायाव‚श्यं पुत्र‚म‚पेक्ष‚त इति पुत्रोऽपि पिता प्र‚स‚ज्य‚त इति‚{\tiny $_{lb}$}‚ कृत‚म‚तार्किक‚व‚च‚न‚विचार‚ण‚येति स‚र्व‚म‚व‚दात‚म् ॥
	\pend% ending standard par
      ‚{\tiny $_{lb}$}‚

	  \pstart \leavevmode% starting standard par
	॥ प‚ण्डित\textbf{दुर्वेक}विर‚चित\textbf{ध‚र्मोत्त‚र‚निब‚न्ध‚स्य ध‚र्मोत्त‚र‚प्र‚दीप‚संज्ञित‚स्य} प्र‚थ‚मः प‚रिच्छेदः ॥
	\pend% ending standard par
      
	    
	    \endnumbering% ending numbering from div
	    \endgroup
	    
	  
	  
	% new div opening: depth here is 0
	
	    
	    \begingroup
	    \beginnumbering% beginning numbering from div depth=0
	    
	  
\chapter*[{द्वितीयः स्वार्थानुमान‚प‚रिच्छेदः ।}]{द्वितीयः स्वार्थानुमान‚प‚रिच्छेदः ।}\textsuperscript{\textenglish{87/dm}}‚{\tiny $_{lb}$}‚
	  \bigskip
	  \begingroup
	

	  \pstart \leavevmode% starting standard par
	एवं प्र‚त्य‚क्षं व्याख्यायानुमानं व्याख्यातुकाम\edtext{}{\lemma{व्याख्यातुकाम}\Bfootnote{व्याख्यातुमाह--\cite{dp-msA}}} आह--
	\pend% ending standard par
       ‚{\tiny $_{lb}$}‚ 
	  \bigskip
	  \begingroup
	

	  \pstart \leavevmode% starting standard par
	अनुमानं द्विधा ॥ १ ॥
	\pend% ending standard par
      
	  \endgroup
	‚{\tiny $_{lb}$}‚ 

	  \pstart \leavevmode% starting standard par
	अनुमानं द्विधा द्विप्र‚कार‚म् । अथानुमान‚ल‚क्ष‚णे व‚क्त‚व्ये किम‚क‚स्मात् प्र‚कार‚भेदः क‚थ्य‚ते ?‚{\tiny $_{lb}$}‚ उच्य‚ते । प‚रार्थानुमानं श‚ब्दात्म‚क‚म्, स्वार्थानुमानं तु ज्ञानात्म‚क‚म् । त‚योर‚त्य‚न्त‚भेदात्‚{\tiny $_{lb}$}‚ नैकं ल‚क्ष‚ण‚म‚स्ति । त‚त‚स्त‚योः प्र‚तिनिय‚तं ल‚क्ष‚ण‚माख्यातुं प्र‚कार‚भेदः क‚थ्य‚ते । प्र‚कार‚भेदो‚{\tiny $_{lb}$}‚ हि व्य‚क्तिभेदः । व्य‚क्तिभेदे च क‚थिते प्र‚तिव्य‚क्तिनिय‚तं ल‚क्ष‚णं श‚क्य‚ते व‚क्तुम् । नान्य‚था ।‚{\tiny $_{lb}$}‚ त‚तो ल‚क्ष‚ण‚निर्देशाङ्ग\edtext{}{\lemma{निर्देशाङ्ग}\Bfootnote{निर्देशार्थ‚मेव \cite{dp-msA} \cite{dp-msC}}}मेव प्र‚कार‚भेद‚क‚थ‚न‚म् । अश‚क्य‚तां च प्र‚कार‚भेद‚क‚थ‚न‚म‚न्त‚रेण ल‚क्ष‚ण‚{\tiny $_{lb}$}‚निर्देश‚स्य ज्ञात्वा प्राक्\edtext{}{\lemma{प्राक्}\Bfootnote{ज्ञात्वा प्र‚थ‚मं प्र‚का० \cite{dp-msC} \cite{dp-msD}}} प्र‚कार‚भेदः क‚थ्य‚त इति ॥
	\pend% ending standard par
      
	  \endgroup
	‚{\tiny $_{lb}$}‚

	  \pstart \leavevmode% starting standard par
	प्र‚त्य‚क्षानुमान‚भेदेन द्वैधं प्र‚माण‚मुद्दिष्ट‚म् । त‚त्र व्याख्यातं प्र‚त्य‚क्ष‚म् । य‚थोद्देश‚म‚धुनाऽ‚{\tiny $_{lb}$}‚नुमानं व्याख्यातुम‚व‚स‚र‚प्राप्त‚मित्य‚भिस‚न्धायाह--एव‚मिति । एव‚म‚न‚न्त‚रोक्तेन च‚तुर्विध‚{\tiny $_{lb}$}‚विप्र‚तिप‚त्तिनिराक‚र‚ण‚प्र‚कारेण । प्र‚कारे धाप्र‚त्य‚योऽय‚मिति द‚र्श‚य‚न्नाह--\textbf{द्विप्र‚कार‚मिति ।}
	\pend% ending standard par
      ‚{\tiny $_{lb}$}‚

	  \pstart \leavevmode% starting standard par
	न‚नु चानुमान‚स्य ल‚क्ष‚णं व‚क्तुकामेनास्य ल‚क्ष‚ण‚मेव व‚क्त‚व्य‚म् । त‚त्किमिद‚म‚प्र‚स्तुता‚{\tiny $_{lb}$}‚भिधान‚मास्थीय‚त इति पूर्व‚प‚क्ष‚म्--\textbf{अथे}त्यादिनोत्थाप‚य‚ति । \textbf{अथ}श‚ब्दोऽत्र प्र‚श्ने । \textbf{अक‚स्मादि}ति‚{\tiny $_{lb}$}‚ निपातो निर्निमित्त‚व‚च‚नः । \textbf{उच्य‚त} इत्यादिना प‚रिह‚र‚ति । \textbf{त‚यो}र्ज्ञानाभिधानात्म‚नोः । एक‚मिति‚{\tiny $_{lb}$}‚ साधार‚ण‚म् । य‚था च‚तुर्णाम‚पि प्र‚त्य‚क्षाणां ज्ञान‚रूप‚त्वादेकं क‚ल्प‚नापोढ‚त्वादिसाधार‚णं ल‚क्ष‚णं‚{\tiny $_{lb}$}‚ स‚म्भ‚व‚ति, त‚था य‚दि स्यात् प्र‚त्य‚क्ष‚व‚ल्ल‚क्ष‚ण‚मेव प्र‚थ‚म‚मुक्तं स्यादिति भावः । \textbf{प्र‚तिनिय‚तं}‚{\tiny $_{lb}$}‚ प्रातिस्विक‚म् । \textbf{प्र‚का}र‚स्य \textbf{भेदो} नानात्व‚म् ।
	\pend% ending standard par
      ‚{\tiny $_{lb}$}‚

	  \pstart \leavevmode% starting standard par
	न‚नु प्र‚तिव्य‚क्तिनिय‚तं ल‚क्ष‚णं व्य‚क्तिविशेषोप‚द‚र्श‚नं विना न श‚क्य‚ते द‚र्श‚यितुमिति‚{\tiny $_{lb}$}‚ व्य‚क्तिभेद एव क‚थ‚यित‚व्यः । त‚त्किं प्र‚कार‚भेदः क‚थ्य‚त इत्याह--\textbf{प्र‚का}रेति । \textbf{हि}र्य‚स्मात् ।‚{\tiny $_{lb}$}‚ य‚दि त‚स्मिन् द‚र्शितेऽपि प्र‚तिनिय‚त‚ल‚क्ष‚णाख्यानं न श‚क्यं त‚र्हि किं तेन क‚थितेनेत्याह--\textbf{व्य‚क्तीति ।‚{\tiny $_{lb}$}‚ चो} य‚स्माद‚र्थे । \textbf{प्र‚ति}श‚ब्दोऽत्र निय‚तार्थ‚वृत्तिः, तेन \textbf{प्र‚ति} विशिष्टा \textbf{व्य‚क्ति}स्त‚त्र \textbf{निय‚त}मिति,‚{\tiny $_{lb}$}‚ ‚{\tiny $_{lb}$}‚ \leavevmode\ledsidenote{\textenglish{88/dm}}‚{\tiny $_{lb}$}‚ 
	  
	किं पुन‚स्त‚द् द्वैविध्य‚मित्याह-- ‚{\tiny $_{lb}$}‚ 
	  
	स्वार्थं प‚रार्थं च ॥ २ ॥‚{\tiny $_{lb}$}‚ 
	  
	स्व‚स्मायिदं स्वार्थ‚म् । येन स्व‚यं प्र‚तिप‚द्य‚ते त‚त् स्वार्थ‚म् । प‚र‚स्मायिदं प‚रार्थ‚म् ।‚{\tiny $_{lb}$}‚ येन प‚रं प्र‚तिपाद‚य‚ति त‚त् प‚रार्थ‚म् ॥‚{\tiny $_{lb}$}‚ स‚प्त‚मी\href{http://sarit.indology.info/?cref=Pā.2.1.40}{पाणिनि २-१-४०} इति योग‚विभागात्स‚मासः । य‚द्वा निय‚तं विशि\textbf{ष्टं} ल‚क्ष‚णं न‚{\tiny $_{lb}$}‚ श‚क्यं व‚क्तुम् । क्व च निय‚त‚मित्याश‚ङ्क्योक्तं--\textbf{प्र‚तिव्य‚क्ती}ति । व्य‚क्तौ व्य‚क्तावित्य‚{\tiny $_{lb}$}‚व्य‚यीभावः । य‚स्माद‚न्य‚था प्र‚तिनिय‚त‚ल‚क्ष‚णाख्यान‚स्याश‚क्य‚त्वं \textbf{त‚त}स्त‚स्मात् । \textbf{ल‚क्ष‚ण‚निर्देशा‚{\tiny $_{lb}$}‚ङ्ग‚मे}वेति ल‚क्ष‚ण‚निर्देश‚निमित्त‚मेव ।
	\pend% ending standard par
      ‚{\tiny $_{lb}$}‚

	  \pstart \leavevmode% starting standard par
	\hphantom{.}एतेन य‚च्चोद्य‚ते ल‚क्ष‚ण‚मात्रे क‚थिते विशिष्ट‚ल‚क्ष‚ण‚म‚नुमान‚मेक‚म‚नेकं वाऽस्तु । किं त‚स्य‚{\tiny $_{lb}$}‚ प्र‚कार‚भेद‚क‚थ‚नेन इति त‚त्प‚रिहृत‚म् । य‚दि हि साधार‚णं ल‚क्ष‚ण‚म‚भिप्रेत्येद‚मुच्य‚ते त‚दा त‚न्नास्तीति‚{\tiny $_{lb}$}‚ किं क‚थ्येत । अथ विशिष्टं ल‚क्ष‚णं त‚द‚पि व्य‚क्तिभेद‚क‚थ‚न‚म‚न्त‚रेण व‚क्तुं य‚दि श‚क्येत किं न‚{\tiny $_{lb}$}‚ क‚थ्येत । केव‚ल‚मिद‚मेव नास्तीति । अत एवादावेव त‚द‚भिधानं न्याय्य‚म्, न तु प‚श्चादिति‚{\tiny $_{lb}$}‚ द‚र्श‚यितुमाह--\textbf{अश‚क्य‚ताम्} इति । त‚द\textbf{न्त‚रेण ल‚क्ष‚ण‚निर्देश‚स्याश‚क्य‚तां ज्ञात्वा । चो}ऽव‚धार‚णे ।‚{\tiny $_{lb}$}‚ \textbf{प्रा}ग्ल‚क्ष‚ण‚क‚थ‚नात्पूर्व‚म् ।
	\pend% ending standard par
      ‚{\tiny $_{lb}$}‚

	  \pstart \leavevmode% starting standard par
	स्यादेत‚त्--स्वार्थानुमान‚मेवंल‚क्ष‚णं प‚रार्थानुमान‚मेवंल‚क्ष‚ण‚मिति किं विशिष्टं ल‚क्ष‚णं न‚{\tiny $_{lb}$}‚ श‚क्य‚ते व‚क्तुम् ? एव‚म‚पि किम‚नुमान‚द्वैतं नावेदितं भ‚व‚ति येन स‚संख्येया संख्या--\textbf{अनुमानं‚{\tiny $_{lb}$}‚ द्विधा स्वार्थं प‚रार्थं चे}त्युच्य‚त इति ? स‚त्य‚मेत‚त् । केव‚लं निय‚मार्थ‚मेत‚द् विभाग‚व‚च‚न‚मिति ब्रूमः ।‚{\tiny $_{lb}$}‚ \textbf{अनुमानं द्विधा}--द्विधैवैव‚मात्म‚क‚मिति क‚थं नाम \leavevmode\ledsidenote{\textenglish{37b/ms}} प्र‚तीयेतेति । इत‚र‚थेह ताव‚देताव‚देव‚{\tiny $_{lb}$}‚ व्युत्पाद्य‚त‚या प्र‚स्तुत‚म्, अन्य‚त्र पुन‚र‚न्य‚द‚प्य‚नुमानं व्युत्पाद्य‚म‚स्तीत्याश‚ङ्का नाह‚त्य निराकृता‚{\tiny $_{lb}$}‚ स्यादिति ॥
	\pend% ending standard par
      ‚{\tiny $_{lb}$}‚

	  \pstart \leavevmode% starting standard par
	पूर्व‚व‚च्छेष‚व‚दादिरूपेण अन्य‚थाऽपि द्वैविध्य‚स‚म्भ‚वात् संश‚यानः पृच्छ‚ति--\textbf{किं पुन‚रि}ति ।‚{\tiny $_{lb}$}‚ \textbf{किमि}ति सामान्यात् पृच्छ‚ति । \textbf{पुन}रिति विशेष‚तः ।
	\pend% ending standard par
      ‚{\tiny $_{lb}$}‚

	  \pstart \leavevmode% starting standard par
	स्वार्थ‚श‚ब्द‚स्य विग्र‚हं द‚र्श‚य‚ति--\textbf{स्व‚स्मायि}ति । \textbf{अर्थ}श‚ब्देन नित्य‚स‚मासाद‚स्य प‚द‚विग्र‚{\tiny $_{lb}$}‚ह‚माह । \textbf{इद}मित्य‚नुमान‚न् । \textbf{स्वार्थ‚मि}ति स‚म‚स्त‚प‚द‚निर्देश एषः । अस्य चात्म‚प्र‚तिप‚त्तिः प्र‚योज‚न‚{\tiny $_{lb}$}‚मित्य‚र्थः । अमुमेवार्यं स्फुट‚य‚न्नाह--\textbf{येने}ति । \textbf{येना}नुमानेन क‚र‚ण‚भूतेनानुमाता \textbf{स्व‚यं प्र‚तिप‚द्य‚ते}‚{\tiny $_{lb}$}‚ प‚रोक्ष‚म‚र्थ‚मिति शेषः, प्र‚क‚र‚ण‚ल‚भ्यं वा । त‚त्स्वार्थ‚ज्ञान‚मात्म‚प्र‚तिप‚त्तिप्र‚योज‚न‚मिति याव‚त् ।
	\pend% ending standard par
      ‚{\tiny $_{lb}$}‚

	  \pstart \leavevmode% starting standard par
	अय‚माश‚यः--त्रिरूप‚लिङ्ग‚स्य ज्ञानं य‚स्य स‚न्तान उत्प‚द्य‚ते त‚त्त‚द‚र्थ‚मेव । तेनाऽन्य‚स्या‚{\tiny $_{lb}$}‚प्र‚तिप‚त्तेः । त‚तः स्वार्थ‚मुच्य‚ते । न तु किञ्चिज्ज्ञानं क्व‚चित्पुंसि निय‚त‚म‚स्ति । य‚द‚पेक्ष‚या‚{\tiny $_{lb}$}‚ स्वार्थ‚मुच्येत । \textbf{येन स्व‚यं प्र‚तिप‚द्य‚त} इति ब्रुव‚त‚श्चाय‚म‚भिप्रायः । य‚द्य‚पि प्र‚तिप‚त्तिर‚नुमान‚ज्ञा‚{\tiny $_{lb}$}‚नात्मिका त‚थाप्येक‚स्यापि व्य‚व‚स्थाप्य‚व्य‚व‚स्थाप‚न‚भावेन क्रियाक‚र‚ण‚भेदो द‚र्शित इति सारूप्य‚भागः‚{\tiny $_{lb}$}‚ क‚र‚ण‚म‚नुमान‚म्, अधिग‚म‚रूपा फ‚लाव‚स्था प्र‚तिप‚त्तिरिति ।
	\pend% ending standard par
      \textsuperscript{\textenglish{89/dm}}‚{\tiny $_{lb}$}‚
	  \bigskip
	  \begingroup
	

	  \pstart \leavevmode% starting standard par
	त‚त्र\edtext{}{\lemma{त्र}\Bfootnote{त‚त्र त्रिरू० \cite{dp-edE}}} स्वार्थं त्रिरूपाल्लिङ्गाद् य‚द‚नुमेये\edtext{}{\lemma{नुमेये}\Bfootnote{०मेय‚ज्ञानं \cite{dp-msC}}} ज्ञानं त‚द‚नुमान‚म् ॥ ३ ॥
	\pend% ending standard par
      
	  \endgroup
	‚{\tiny $_{lb}$}‚
	  \bigskip
	  \begingroup
	

	  \pstart \leavevmode% starting standard par
	त‚त्र त‚योः स्वार्थ‚प‚रार्थानुमान‚योर्म‚ध्ये स्वार्थं ज्ञानं किंविशिष्ट‚मित्याह--त्रिरूपादिति ।‚{\tiny $_{lb}$}‚ त्रीणि रूपाणि य‚स्य व‚क्ष्य‚माण\edtext{}{\lemma{माण}\Bfootnote{व‚क्ष्य‚माणानि \cite{dp-msD}}}ल‚क्ष‚णानि त‚त् त्रिरूप‚म् । लिङ्ग्य‚ते ग‚म्य‚तेऽनेनाऽर्थ इति
	\pend% ending standard par
      
	  \endgroup
	‚{\tiny $_{lb}$}‚

	  \pstart \leavevmode% starting standard par
	प‚रार्थ‚मित्य‚स्य विग्र‚ह‚माह--\textbf{प‚र‚स्मायि}ति । पूर्व‚व‚द‚स्य प‚द‚विग्र‚हः । \textbf{प‚रार्थ‚मिति} स‚म‚स्तं‚{\tiny $_{lb}$}‚ प‚द‚मुक्त‚म् । अस्य च प‚र‚प्र‚तिप‚त्तिः प्र‚योज‚न‚मित्य‚र्थः । अमुम‚र्थं \textbf{येने}त्यादिना स्प‚ष्ट‚य‚ति । \textbf{येन}‚{\tiny $_{lb}$}‚ वाक्येन क‚र‚णेन \textbf{प‚रं} प्र‚ति वाच्यं \textbf{प्र‚तिपाद‚य‚ति} प‚रोक्ष‚म‚र्थं बोध‚य‚ति \textbf{त‚त्} त्रिरूप‚लिङ्गाख्यानं वाक्यं‚{\tiny $_{lb}$}‚ \textbf{प‚रार्थ‚म}नुमान‚म् ।
	\pend% ending standard par
      ‚{\tiny $_{lb}$}‚

	  \pstart \leavevmode% starting standard par
	अत्राप्य‚य‚म‚स्याभिप्रायः--य‚द्य‚पि अभिधान‚रूप‚म‚प्य‚नुमानं न निय‚तं पुंसि त‚थाऽपि त‚त्प‚रार्थ‚{\tiny $_{lb}$}‚मेव । त‚थाहि य‚द् य‚दुद्दिश्य प्र‚व‚र्त्त‚ते त‚त् त‚द‚र्थ‚मुच्य‚ते । प‚र‚मुद्दिश्य प्र‚व‚र्त्त‚ते च श‚ब्दो‚{\tiny $_{lb}$}‚ नात्मान‚म् । अतो नान‚व‚स्थित‚पारार्थ्यः श‚ब्दः । प्र‚योक्तृसंमीहाविष‚य‚स्यार्थ‚स्य प‚र एव‚{\tiny $_{lb}$}‚ प्र‚योज‚को य‚स्मादिति । प‚रोक्षार्थ‚प्र‚तिप‚त्तिफ‚ल‚त्वेन पार‚म्प‚र्येणाविशिष्ट‚विष‚य‚त्वेऽपि स्वार्थाद‚स्य‚{\tiny $_{lb}$}‚ पृथ‚ग्व‚च‚न‚म्, साक्षाद‚न‚योर्व्यापार‚भेदादिति च द्र‚ष्ट‚व्य‚म् ।
	\pend% ending standard par
      ‚{\tiny $_{lb}$}‚

	  \pstart \leavevmode% starting standard par
	\hphantom{.}न‚नु च प‚रार्थानुमानोत्पाद‚क‚वाक्य‚व‚द‚स्ति किञ्चिद् वाक्यं य‚त्प‚र‚प्र‚त्य‚क्षोप‚योगि । य‚था एष‚{\tiny $_{lb}$}‚ क‚ल‚भो धाव‚ति इति वाक्य‚म् । अतः प‚रार्थानुमान‚व‚त्प‚रार्थं प्र‚त्य‚क्षं किं न व्युत्पाद्य‚त इति ?‚{\tiny $_{lb}$}‚ अत्रोच्य‚ते--प‚रोक्षार्थ‚प्र‚तिप‚त्तेर्या साम‚ग्री--लिङ्ग‚स्य प‚क्ष‚ध‚र्म‚ता साध्य‚व्याप्तिश्च--त‚दाख्यानाद्‚{\tiny $_{lb}$}‚ वाक्य‚मुप‚चार‚तः प‚रार्थानुमान‚मुच्य‚ते । न तु त‚त्र क‚थ‚ञ्चिद‚ङ्ग‚भाव‚मात्रेण, स्वास्थ्यादेर‚पि त‚था‚{\tiny $_{lb}$}‚ प्र‚स‚ङ्गात् । इदं पुनः अयं क‚ल‚भः इत्यादिवाक्यं न प्र‚त्य‚क्षोत्प‚त्तेर्या साम‚ग्रीन्द्रियालोकादि‚{\tiny $_{lb}$}‚ त‚द‚भिधानात्त‚न्निमित्तं भ‚व‚त्त‚था व्य‚प‚देश‚म‚श्नुते येन व्युत्पाद्य‚ताम‚प्य‚श्नुवीत । किं त‚र्हि ? क‚स्य‚चिद्‚{\tiny $_{lb}$}‚ दिदृक्षामात्र‚ज‚न‚नेन । य‚था क‚थ‚ञ्चित्प‚र‚प्र‚त्य‚क्षोत्प‚त्ताव‚ङ्ग‚भाव‚मात्रेण ताद्रूप्ये नेत्रोत्स‚वे व‚स्तुनि‚{\tiny $_{lb}$}‚ स‚न्निहितेऽपि क‚थ‚ञ्चित्प‚राङ्मुख‚स्य प‚रेण य‚द‚भिमुखीक‚र‚णं \leavevmode\ledsidenote{\textenglish{38a/ms}} शिर‚स‚स्त‚द‚पि व‚च‚नात्म‚कं‚{\tiny $_{lb}$}‚ प‚रार्थ‚प्र‚त्य‚क्षं व्युत्पाद‚यितुर्व्युत्पाद्य‚माप‚द्येत । एत‚च्च कः स्व‚स्थात्मा म‚न‚सि निवेश‚येत् । किञ्च‚{\tiny $_{lb}$}‚ भ‚व‚तु त‚थाविधं व‚च‚नं प‚रार्थं प्र‚त्य‚क्ष‚म् । किं न‚श्छिन्न‚म् ? त‚स्यापि व्युत्पाद‚नार्ह‚स्याव्यु‚{\tiny $_{lb}$}‚त्पाद‚नात्प्र‚माद एव म‚ह‚ती क्ष‚तिरिति चेत् । न त‚थारूप‚स्य व्युत्पाद‚न‚म्, अविप्र‚तिप‚त्तेः ।‚{\tiny $_{lb}$}‚ विप्र‚तिप‚त्तिनिराक‚र‚णेन हि स्व‚रूप‚प्र‚तिपाद‚नं व्युत्पाद‚न‚म् । न तु केचित् त‚थाविधे व‚च‚ने‚{\tiny $_{lb}$}‚ प‚रार्थ‚प्र‚त्य‚क्षोप‚योगिनि विप्र‚तिप‚द्य‚न्ते । येन त‚द‚पि व्युत्पाद्येत । प‚रार्थानुमाने\edtext{}{\lemma{रार्थानुमाने}\Bfootnote{अस्प‚ष्ट‚म्--सं०}}\add{... ... ...}‚{\tiny $_{lb}$}‚ व‚स्तु प्र‚तिप‚द्य‚माना अपि त‚द्ध‚र्म‚व्याप्तिव्य‚तिरेकाभ्यां निग‚द‚न्तो दृष्टाः, अविनाभावाव‚च‚नात्,‚{\tiny $_{lb}$}‚ उप‚न‚य‚साध्य‚त‚दावृत्तिव‚च‚नानाञ्च प्र‚योगादिति त‚द् व्युत्पाद्य‚ते । य‚दि तु त‚त्रापि न विप्र‚ति‚{\tiny $_{lb}$}‚प‚द्येर‚न् प‚रे त‚दा त‚द‚पि नैव व्युत्पादितं स्याद्, इत्य‚ल‚म‚तिविस्त‚रेण ॥
	\pend% ending standard par
      ‚{\tiny $_{lb}$}‚

	  \pstart \leavevmode% starting standard par
	इह य‚थैव स्व‚यं प्र‚तिप‚न्नः प‚रोक्षार्थ‚स्त‚थैव प‚र‚स्मै प्र‚तिपाद्य‚त इति स्वार्थानुमान‚पूर्व‚क‚त्वा‚{\tiny $_{lb}$}‚त्प‚रार्थानुमान‚स्य प्र‚थ‚मं स्वार्थानुमान‚मुक्त‚म् । य‚थोद्देश‚मेव च ल‚क्ष‚णं प्र‚णेय‚मिति स्वार्थानुमान‚{\tiny $_{lb}$}‚‚{\tiny $_{lb}$}‚ ‚{\tiny $_{lb}$}‚ \leavevmode\ledsidenote{\textenglish{90/dm}}‚{\tiny $_{lb}$}‚ 
	  
	लिङ्ग‚म् । त‚स्मात् त्रिरूपाल्लिङ्गात् य‚त् जातं ज्ञान‚म् इति । एत‚द् \edtext{}{\lemma{द्}\Bfootnote{त्रिरूप‚लिङ्ग‚जं ज्ञान‚मित्य‚र्थः । हेतुः कार‚ण‚म्--\cite{dp-msD-n} ।}}हेतुद्वारेण विशेष‚ण‚म् ।‚{\tiny $_{lb}$}‚ त‚त् \edtext{}{\lemma{त्}\Bfootnote{०ष‚ण‚म् । त्रिरूपा० \cite{dp-msC} \cite{dp-msD}}}त्रिरूपाच्च लिङ्गात् त्रिरूप‚लिङ्गाल‚म्ब‚न‚म‚प्युत्प‚द्य‚त इति विशिन‚ष्टि--अनुमेय इति ।‚{\tiny $_{lb}$}‚ एत‚च्च विष‚य‚द्वारेण विशे ष‚ण‚म् । ‚{\tiny $_{lb}$}‚ 
	  
	त्रिरूपाल्लिङ्गाद्य‚दुत्प‚न्न‚म‚नुमेयाल‚म्ब‚नं ज्ञानं त‚त् \edtext{}{\lemma{त्}\Bfootnote{स्वार्थानु० \cite{dp-msD}}}स्वार्थ‚म‚नुमान‚मिति ॥ ‚{\tiny $_{lb}$}‚ 
	  
	ल‚क्ष‚ण‚विप्र‚तिप‚त्तिं निराकृत्य फ‚ल‚विप्र‚तिप‚त्तिं निराक‚र्त्तुमाह-- ‚{\tiny $_{lb}$}‚ 
	  
	प्र‚माण‚फ‚ल‚व्य‚व‚स्थाऽत्रापि प्र‚त्य‚क्ष‚व‚त् ॥ ४ ॥‚{\tiny $_{lb}$}‚ 
	  
	प्र‚माण‚स्य\edtext{}{\lemma{स्य}\Bfootnote{प्र‚माण‚फ‚ल‚मिति । प्र‚माण‚स्य य‚त्--\cite{dp-msB} \cite{dp-msC} \cite{dp-msD}}} य‚त् फ‚लं त‚स्य या व्य‚व‚स्था \edtext{}{\lemma{स्था}\Bfootnote{०स्थाऽत्रा० \cite{dp-msA} \cite{dp-msB} \cite{dp-edP} \cite{dp-edH} \cite{dp-edE}}}साऽत्रानुमानेऽपि \edtext{}{\lemma{साऽत्रानुमानेऽपि}\Bfootnote{०पि प्र‚त्य‚क्ष‚व‚त् प्र‚त्य‚क्ष इव \cite{dp-edP} \cite{dp-edH} \cite{dp-edE}}}प्र‚त्य‚क्ष इव प्र‚त्य‚क्ष‚व‚त्‚{\tiny $_{lb}$}‚ वेदित‚व्या ।‚{\tiny $_{lb}$}‚ स्यैवं ल‚क्ष‚णं \textbf{त‚त्रे}त्यादिनाऽऽदित उप‚दिष्ट‚माचार्येण त‚द् व्याच‚ष्टे \textbf{त‚त्रे}ति । स्वार्थ‚प‚रार्थानुमान‚{\tiny $_{lb}$}‚स‚मुदायात् स्वार्थानुमानं स्वार्थ‚त्व‚ज्ञात्या निर्धार्य‚ते । \textbf{त‚स्मात्त्रिरूपाल्लिङ्गाद् य‚ज्जात‚मिति}‚{\tiny $_{lb}$}‚ व्याच‚क्षाणो मूले \textbf{त्रिरूपाल्लिङ्गादि}ति या प‚ञ्च‚मी सा ग‚म्य‚मान‚ज‚निक्रियापेक्ष‚या ज‚निक‚र्त्तुः‚{\tiny $_{lb}$}‚ प्र‚कृतिः \href{http://sarit.indology.info/?cref=Pā.1.4.30}{पाणिनि १. ४. ३०} इत्य‚नेन ल‚ब्धापादान‚संज्ञ‚काद‚पादान एवेति द‚र्श‚य‚ति । \textbf{हेतुद्वारेण}‚{\tiny $_{lb}$}‚ ज‚न‚क‚मुखेन । \textbf{त्रिरूपाल्लिङ्गादि}ति चाच‚क्षाणेनाचार्येणैक‚द्विप‚द‚व्युदासेन ष‚ट्प‚क्षीं प्र‚तिक्षिप्य‚{\tiny $_{lb}$}‚ स‚प्त‚म‚प‚क्ष‚प‚रिग्र‚हेण लिङ्ग‚स्य ल‚क्ष‚ण‚म‚भिप्रेतं प्र‚काशित‚मिति । य‚था चैत‚त् त‚था \textbf{भ‚ट्टार्च‚ट‚{\tiny $_{lb}$}‚निब‚न्ध‚न‚म‚र्च‚टालोक‚संज्ञितं} विधास्य‚न्तो विस्त‚रेण स्प‚ष्ट‚यिष्यामः ।
	\pend% ending standard par
      ‚{\tiny $_{lb}$}‚

	  \pstart \leavevmode% starting standard par
	अनुमेय‚ग्र‚ह‚ण‚स्य व्याव‚र्त्त्यं द‚र्श‚य‚ति--\textbf{त्रिरूपाच्चे}ति । \textbf{चो} य‚स्माद‚र्थे । \textbf{इति}र्हेतौ ।‚{\tiny $_{lb}$}‚ \textbf{त्रिरूप‚लिङ्गाल‚म्ब‚न}मिति धूमं दृष्ट्वा स‚र्व‚त्रायं व‚ह्निनान्त‚रीय‚क इति ज्ञानं वाच्य‚म् । त‚द्धि‚{\tiny $_{lb}$}‚ प‚र‚म्प‚र‚या त्रिरूपाल्लिङ्गाज्जात‚मिति । \textbf{इति}ना विशेष‚ण‚स्य स्व‚रूप‚मुक्त‚म् । विशेषित‚मेव‚{\tiny $_{lb}$}‚ ज्ञान‚म् । किम्पुन‚र्विशिष्य‚त इत्याह--\textbf{एत‚च्चे}ति । \textbf{चो} य‚स्मात् । \textbf{विष‚य‚द्वारेणा}व‚सीय‚मान‚विष‚य‚{\tiny $_{lb}$}‚\textbf{द्वारेण विशेष‚णं} व्य‚व‚च्छेद‚क‚म् ।
	\pend% ending standard par
      ‚{\tiny $_{lb}$}‚

	  \pstart \leavevmode% starting standard par
	अव‚य‚वार्थं व्याख्याय स‚मुदायार्थं \textbf{त्रिरूपे}त्यादिना व्याच‚ष्टे । \textbf{अनुमेयो} ध‚र्म‚ध‚र्मिस‚मुदायः‚{\tiny $_{lb}$}‚ आल‚म्ब्य‚त इत्या\textbf{ल‚म्ब‚नं} य‚स्येति विग्र‚हः । \textbf{इति}र्वाक्यार्थ‚प‚रिस‚माप्तौ एव‚म‚र्थः स‚न्नाप‚रेण‚{\tiny $_{lb}$}‚ स‚म्ब‚द्ध्य‚ते--एव‚मुक्तेन प्र‚कारेण \textbf{ल‚क्ष‚ण‚विप्र‚तिप‚त्तिं निराकृत्ये}ति ॥
	\pend% ending standard par
      ‚{\tiny $_{lb}$}‚

	  \pstart \leavevmode% starting standard par
	न‚नु च \textbf{प्र‚माण‚स्य फ‚ल‚मि}ति य‚द्या\textbf{चार्य‚स्य} विव‚क्षितं \textbf{ध‚र्मोत्त‚रेण} चैवं व्याख्याय‚ते त‚दा‚{\tiny $_{lb}$}‚ प्र‚माण‚भाग‚व्य‚व‚स्थायां किमुक्त\textbf{माचार्येण, ध‚र्मोत्त‚रेणापि} नील‚सारूप्यं व्य‚व‚स्थाप‚न‚हेतुः प्र‚माण‚म्‚{\tiny $_{lb}$}‚ इत्युप‚रिष्टात्\href{http://sarit.indology.info/?cref=p91}{पृ० ९१} किमिति द‚र्श‚यिष्य‚ते इति चेत् । नैष दोषः । न‚हि \textbf{प्र‚माण‚स्ये}त्यादिना‚{\tiny $_{lb}$}‚ ‚{\tiny $_{lb}$}‚ \leavevmode\ledsidenote{\textenglish{91/dm}}‚{\tiny $_{lb}$}‚ 
	  
	य‚था हि \edtext{}{\lemma{हि}\Bfootnote{नील‚स्व‚रूपं \cite{dp-msC} \cite{dp-msD}}}नील‚स‚रूपं प्र‚त्य‚क्ष‚म‚नुभूय‚मानं नील‚बोध‚रूप‚म‚व‚स्थाप्य‚ते\edtext{}{\lemma{ते}\Bfootnote{०रूप‚मेवाव‚स्था० \cite{dp-msC}}}, तेन नील‚सारूप्यं‚{\tiny $_{lb}$}‚ \edtext{\textsuperscript{*}}{\lemma{*}\Bfootnote{०रूप्यं नील‚व्य‚व० \cite{dp-msC}}}व्य‚व‚स्थाप‚न‚हेतुः प्र‚माण‚म्, नील‚बोध‚रूपं तु \edtext{}{\lemma{तु}\Bfootnote{०रूपं त्व‚व‚स्था० \cite{dp-msD} ०रूप‚त्व‚म‚व‚स्था० \cite{dp-msC}}}व्य‚व‚स्थाप्य‚मानं प्र‚माण‚फ‚ल‚म्; त‚द्व‚द् अनुमानं‚{\tiny $_{lb}$}‚ नीलाकार‚मुत्प‚द्य‚मानं नील‚बोध‚रूप‚म‚व‚स्थाप्य‚ते, तेन नील‚सारूप्य‚म‚स्य\edtext{}{\lemma{स्य}\Bfootnote{०प्यं प्र‚मा० \cite{dp-msC}}} प्र‚माण‚म्, नील‚विक‚ल्प‚न‚{\tiny $_{lb}$}‚रूपं त्व‚स्य\edtext{}{\lemma{स्य}\Bfootnote{०रूपं तु प्र‚मा० \cite{dp-msB} \cite{dp-msD}}} प्र‚माण‚फ‚ल‚म् । सारूप्य‚व‚शाद्धि त‚न्नील‚प्र‚तीतिरूपं सिध्य‚ति । नान्य‚थेति ॥ ‚{\tiny $_{lb}$}‚ 
	  
	एव‚मिह संख्या-ल‚क्ष‚ण-फ‚ल‚विप्र‚तिप‚त्त‚यः । प्र‚त्य‚क्ष‚प‚रिच्छेदे तु गोच‚र‚विप्र‚तिप‚त्ति‚{\tiny $_{lb}$}‚र्निराकृता । ल‚क्ष‚ण‚निर्देश‚प्र‚स‚ङ्गेन तु त्रिरूपं लिङ्गं प्र‚स्तुत‚म् । त‚देव व्याख्यातुमाह-- ‚{\tiny $_{lb}$}‚ 
	  
	त्रैरूप्यं पुन‚र्लिङ्ग‚स्यानुमेये स‚त्त्व‚मेव, स‚प‚क्ष एव स‚त्त्व‚म्, अस‚प‚क्षे‚{\tiny $_{lb}$}‚ चास‚त्त्व‚मेव निश्चित‚म् ॥ ५ ॥‚{\tiny $_{lb}$}‚ 
	  
	त्रैरूप्य‚मित्यादि । लिङ्ग‚स्य य‚त् त्रैरूप्यं यानि त्रीणि रूपाणि त‚दिद‚मुच्य‚त इति शेषः ।‚{\tiny $_{lb}$}‚ किं पुन‚स्त‚त्\edtext{}{\lemma{त्}\Bfootnote{पुन‚स्त्रैरूप्य‚म्--\cite{dp-msA}}} त्रैरूप्य‚मित्याह--अनुमेयं व‚क्ष्य‚माण‚ल‚क्ष‚ण‚म् । त‚स्मिन् लिङ्ग‚स्य स‚त्त्व‚मेव‚{\tiny $_{lb}$}‚ निश्चित‚म्--एकं रूप‚म् । य‚द्य‚पि चात्र निश्चित‚ग्र‚ह‚णं न कृतं त‚थापि अन्ते कृतं प्र‚क्रान्त‚यो‚{\tiny $_{lb}$}‚र्द्व‚योर‚पि रूप‚योर‚पेक्ष‚णीय‚म् । य‚तो न योग्य‚त‚या लिङ्गं प‚रोक्ष\edtext{}{\lemma{रोक्ष}\Bfootnote{प‚रोक्षं ज्ञान‚स्य--\cite{dp-msC}}} ज्ञान‚स्य निमित‚म् । य‚था‚{\tiny $_{lb}$}‚ मौल‚स्य \textbf{प्र‚माण}श‚ब्द‚स्य \textbf{फ‚ल}श‚ब्देन विग्र‚हो द‚र्शितः । किन्त्व\leavevmode\ledsidenote{\textenglish{38b/ms}}र्थ‚प्र‚द‚र्श‚नं कृत‚म् ।‚{\tiny $_{lb}$}‚ एत‚च्चोप‚ल‚क्ष‚णं तेन फ‚ल‚स्य साध‚नं च य‚त्त‚स्यापि या व्य‚व‚स्था साऽपि गृह्य‚ते । मूले तु द्व‚न्द्व‚{\tiny $_{lb}$}‚स‚मास एवाभिप्रेतो \textbf{वार्त्तिक‚कार}स्य । पूर्व‚निपात‚विधेश्चानित्य‚त्वात् न फ‚ल‚श‚ब्द‚स्य पूर्व‚निपातः ।‚{\tiny $_{lb}$}‚ अत एव \textbf{विनिश्च‚यः}--न प्र‚माण‚फ‚ल‚योर्विष‚य‚भेदः इति ।\edtext{\textsuperscript{*}}{\lemma{*}\Bfootnote{प‚ङ्क्तिबाह्यं लिखितं न प‚ठ्य‚ते--सं०}}...ल‚क्ष‚ण‚गोच‚र‚फ‚ल‚विष‚य इति ।‚{\tiny $_{lb}$}‚ \textbf{नील‚सारूप्य‚म्} अस्प‚ष्ट‚नील‚सारूप्य‚म्, अनुमान‚स्याप‚रोक्षीक‚र‚णाभावात्, विजातीय‚मात्र‚व्या‚{\tiny $_{lb}$}‚वृत्त‚स्यानुमानेन प्र‚तीतेः । \textbf{नील‚प्र‚तीतिरूपं} नील‚विक‚ल्प‚न‚रूपं \textbf{सिद्ध्य}ति निश्चीय‚ते ॥
	\pend% ending standard par
      ‚{\tiny $_{lb}$}‚

	  \pstart \leavevmode% starting standard par
	न‚नु स‚ङ्ख्याल‚क्ष‚ण‚फ‚ल‚विप्र‚तिप‚त्त‚य एवानुमान‚स्य निराकृता, न तु विष‚य‚विप्र‚तिप‚त्ति‚{\tiny $_{lb}$}‚रित्याह--\textbf{एव}मिति । \textbf{एव}म‚न‚न्त‚रोक्तेन प्र‚कारेण \textbf{इहा}नुमान‚प‚रिच्छेदे \textbf{निराकृता} इति शेषः,‚{\tiny $_{lb}$}‚ व‚क्ष्य‚माणं वा \textbf{निराकृते}ति प‚दं व‚च‚न‚विप‚रिणामेन स‚म्ब‚न्ध‚नीय‚म् ।
	\pend% ending standard par
      ‚{\tiny $_{lb}$}‚

	  \pstart \leavevmode% starting standard par
	\textbf{त्रैप्रूय}मित्यादिग्र‚न्थ‚स्य य‚त उत्थान‚म्, त‚त् \textbf{ल‚क्ष‚णे}त्यादिना द‚र्श‚य‚ति । \textbf{प्र‚स‚ङ्गेन} प्र‚स्तावेन‚{\tiny $_{lb}$}‚ \textbf{यानि त्रीणि रूपाणि} तान्येव त‚थेत्यावेद‚य‚ति । \textbf{त‚दिदं} त्रैरूप्य‚मिति प्र‚कृत‚त्वात् \textbf{शेषोऽ}ध्याहारः ।‚{\tiny $_{lb}$}‚ अबाधित‚विष‚य‚त्वाद्य‚नेक‚रूप‚स‚म्भ‚वे पृच्छ‚ति \textbf{किं पुन}रिति । \textbf{किमि}ति सामान्य‚तः पृच्छ‚ति ।‚{\tiny $_{lb}$}‚ \textbf{पुन‚रि}ति विशेष‚तः । क‚स्मात् पुन‚स्त‚द‚पेक्ष‚णीय‚मित्याह--\textbf{य‚त} इति । प‚रोक्षो योऽर्थ‚स्त‚स्य य‚ज्ज्ञानं‚{\tiny $_{lb}$}‚ त‚स्य । \textbf{बीजं} वैध‚र्म्य‚दृष्टान्तः । क‚स्मान्न त‚थेत्याह--\textbf{अदृष्टादि}ति । \textbf{अप्र‚तिप‚त्तेः} प‚रोक्षार्थ‚स्येति‚{\tiny $_{lb}$}‚ प्र‚क‚र‚णात् । य‚द्य‚ज्ञातं लिङ्गं न प‚रोक्ष‚ज्ञान‚निमित्तं त‚र्हि प‚क्ष‚ध‚र्म‚त‚या ज्ञात‚मेवास्तु प‚रोक्ष‚प्र‚काश‚नं‚{\tiny $_{lb}$}‚ ‚{\tiny $_{lb}$}‚ \leavevmode\ledsidenote{\textenglish{92/dm}}‚{\tiny $_{lb}$}‚ 
	  
	बीज‚म‚ङ्कुर‚स्य । अदृष्टाद् धूमाद् अग्नेर‚प्र‚तिप‚त्तेः । \edtext{\textsuperscript{*}}{\lemma{*}\Bfootnote{न‚नु च य‚दा धूम‚स्व‚रूप‚मेव प्र‚त्य‚क्षेण ज्ञाय‚ते त‚दा प‚रोक्ष‚स्य निश्चाय‚कं भ‚व‚ति ।‚{\tiny $_{lb}$}‚ आह ।--\cite{dp-msD-n}}}नापि स्व‚विष‚य‚ज्ञानापेक्षं \edtext{}{\lemma{ज्ञानापेक्षं}\Bfootnote{प‚रोक्ष‚प्र‚का० \cite{dp-msC}}}प‚रोक्षार्थ‚{\tiny $_{lb}$}‚प्र‚काश‚नं । य‚था प्र‚दीपो घ‚टादेः । \edtext{\textsuperscript{*}}{\lemma{*}\Bfootnote{धूमात्--\cite{dp-msD-n}}}दृष्टाद‚प्य‚निश्चित‚स‚म्ब‚न्धाद‚प्र‚तिप‚त्तेः\edtext{}{\lemma{त्तेः}\Bfootnote{अग्नेः--\cite{dp-msD-n}}} । त‚स्मात्‚{\tiny $_{lb}$}‚ \edtext{\textsuperscript{*}}{\lemma{*}\Bfootnote{०क्षार्थाना० \cite{dp-edE}}}प‚रोक्षार्थ‚नान्त‚रीय‚क‚त‚या निश्च‚य‚न‚मेव लिङ्ग‚स्य प‚रोक्षार्थ‚प्र‚तिपाद‚न‚व्यापारः । नाप‚रः‚{\tiny $_{lb}$}‚ क‚श्चित् । अतोऽन्व‚य‚व्य‚तिरेक‚प‚क्ष‚ध‚र्म‚त्व‚निश्च‚यो लिङ्ग‚व्यापारात्म‚क‚त्वाद‚व‚श्य‚क‚र्त्त‚व्य इति‚{\tiny $_{lb}$}‚ स‚र्वेषु रूपेषु निश्चित‚ग्र‚ह‚ण‚म‚पेक्ष‚णीय‚म् । ‚{\tiny $_{lb}$}‚ 
	  
	त‚त्र स‚त्त्व‚व‚च‚नेनासिद्धं चाक्षुष‚त्वादि निर‚स्त‚म्\edtext{}{\lemma{म्}\Bfootnote{दि निषिद्ध‚म् \cite{dp-msB} \cite{dp-msD}}} । एव‚कारेण प‚क्षैक‚देशासिद्धो‚{\tiny $_{lb}$}‚ निर‚स्तः\edtext{}{\lemma{स्तः}\Bfootnote{०सिद्धो य‚था \cite{dp-msA} \cite{dp-edN} \cite{dp-edP} ०सिद्धो निर‚स्तो हेतुः य‚था \cite{dp-msB} \cite{dp-msD} \cite{dp-edE} \cite{dp-edH}}} । य‚था चेत‚नास्त‚र‚वः स्वापादिति । प‚क्षीकृतेषु त‚रुषु प‚त्र‚स‚ङ्कोच‚ल‚क्ष‚णः स्वाप‚{\tiny $_{lb}$}‚ एक‚देशे न सिद्धः । न हि स‚र्वे वृक्षा रात्रौ प‚त्र‚संकोच‚भाजः किन्तु केचिदेव । स‚त्त्व‚व‚च‚न‚स्य‚{\tiny $_{lb}$}‚ त‚त् किम‚न्व‚य‚व्य‚तिरेक‚निश्च‚येन त‚स्य ? इत्याश‚ङ्क्याह--\textbf{नापी}ति । \textbf{लिङ्ग‚मि}ति स‚म्ब‚ध्य‚ते ।‚{\tiny $_{lb}$}‚ \textbf{प्र‚दीपोऽपि दृष्टा\add{द}निश्चिताद}पि । न केव‚ल‚म‚दृष्टादित्य‚पिश‚ब्दः । य‚त एवं \textbf{त‚स्माद्} हेतोः ।‚{\tiny $_{lb}$}‚ \textbf{अन्त‚रं} व्य‚व‚धान‚म्, न त‚था \textbf{नान्त‚र‚म्} । न भ्राड् \href{http://sarit.indology.info/?cref=Pā.6.3.75}{पाणिनि ६. ३. ७१}इत्यादिसूत्रे नेति‚{\tiny $_{lb}$}‚ योग‚विभागान्न‚लोपाभावः । नान्त‚रे भ‚व इति ग‚हादित्वाच्छ । त‚तः स्वार्थिकः क‚न् । प‚रोक्ष‚स्य‚{\tiny $_{lb}$}‚ व‚ह्न्यादेर्नान्त‚रीय‚कोऽविनाभावी त‚स्य भाव‚स्त‚त्ता त‚या निश्च‚य‚न‚म्--स‚र्व‚त्राय‚मेत‚द‚विनाभावीति‚{\tiny $_{lb}$}‚ विक‚ल्प‚न‚म् । \textbf{लिङ्ग‚स्य} ग‚म‚क‚स्य व्यापृतं रूपं \textbf{व्यापारः} । त‚थानिश्च‚यारूढ‚स्यैव रूप‚स्य लिङ्ग‚{\tiny $_{lb}$}‚त्वात् । \textbf{एव}कारेण व्य‚व‚च्छिन्न‚मेवान्य‚स्य त‚द् व्यापार‚रूप‚त्वं स्प‚ष्टार्थं \textbf{नाप‚र} इत्य‚नेनानूदित‚म् ।‚{\tiny $_{lb}$}‚ य‚त्पुन‚र‚त्राप्य‚य‚मेत‚न्नान्त‚रीय‚क इति ज्ञानं त‚द‚नुमान‚ज्ञान‚मिति ज्ञेय‚म् । पूर्व‚कं तु लिङ्ग‚ज्ञान‚{\tiny $_{lb}$}‚मिति । \textbf{अतः} स‚त्त्वेनानिश्चिताल्लिङ्गाद‚निश्चित‚स‚म्ब‚न्धाच्चाप्र‚तिप‚त्तेः कार‚णाद‚न्व‚यादिनिश्च‚यो‚{\tiny $_{lb}$}‚\textbf{ऽव‚श्य‚क‚र्त्त‚व्यः} । त‚थापि क‚थं क‚र्त्त‚व्य इत्याश‚ङ्क्य व्य‚तिरेक‚मुखेणोक्त‚मेव हेतुम‚न्व‚य‚मुखेनापि‚{\tiny $_{lb}$}‚ द‚र्श‚य‚न्नाह--\textbf{लिङ्गेति । लिङ्ग‚व्यापार‚स्य} प‚रोक्ष‚प्र‚तिपाद‚ने ग‚म‚क‚व्यापार‚स्यात्मा स एव त‚थेति‚{\tiny $_{lb}$}‚ स्वार्थिकः क‚न् क‚र्त्त‚व्यः । त‚स्य भाव‚स्त‚स्मादिति हेतुप‚दं कृत्वाऽय‚मुप‚संहारः । \textbf{स‚र्वेष्व}नुमेय‚{\tiny $_{lb}$}‚स‚त्त्वादिषु \textbf{रूपेषु} ल‚क्ष‚णेषु । \leavevmode\ledsidenote{\textenglish{39a/ms}}
	\pend% ending standard par
      ‚{\tiny $_{lb}$}‚

	  \pstart \leavevmode% starting standard par
	स‚म्प्र‚त्येकैक‚स्य रूप‚स्य य‚द् व्याव‚र्त्त्य त‚त्क्र‚मेण द‚र्श‚यितुमाह--\textbf{त‚त्रे}ति । \textbf{आदि}ग्र‚ह‚णाद्‚{\tiny $_{lb}$}‚ व्य‚धिक‚र‚णासिद्ध‚विशेष‚णासिद्ध‚विशेष्यासिद्धानां स‚ङ्ग्र‚हः । अमीषाम‚पि स्व‚रूपासिद्ध एवान्त‚{\tiny $_{lb}$}‚र्भावात् । य‚था तु प्र‚भेदो य‚था वान्त‚र्भाव‚स्त‚थोप‚रिष्टाद् व‚क्ष्यामः । \textbf{केचिदेव} तिन्तिडिका‚{\tiny $_{lb}$}‚प्र‚भृत‚यः ।
	\pend% ending standard par
      ‚{\tiny $_{lb}$}‚

	  \pstart \leavevmode% starting standard par
	अथापि स्यात् स्वाप‚व‚त् धूमोऽप्य‚य‚मेक‚देशासिद्ध एव । त‚थाहि प‚र्व‚तादिरिह \textbf{प‚क्ष‚स्त‚त्र}‚{\tiny $_{lb}$}‚ च क्व‚चिदेव देशे सिद्धो न स‚र्व‚त्र । न च प‚र्व‚तादिरेकोऽव‚य‚व्य‚भ्युप‚ग‚तः । अथैवंविधोऽपि‚{\tiny $_{lb}$}‚ प‚क्ष‚व्याप‚क उच्य‚ते त‚र्हि स्वापेन किम‚प‚राद्धं येनासावेवैको न युक्त इति ।
	\pend% ending standard par
      ‚{\tiny $_{lb}$}‚‚{\tiny $_{lb}$}‚\textsuperscript{\textenglish{93/dm}}‚{\tiny $_{lb}$}‚
	  \bigskip
	  \begingroup
	

	  \pstart \leavevmode% starting standard par
	प‚श्चात्कृतेनैव\edtext{}{\lemma{श्चात्कृतेनैव}\Bfootnote{अग्रे कृतेन--\cite{dp-msD-n}}}कारेणासाधार‚णो ध‚र्मो निर‚स्तः । य‚दि हि अनुमेय एव स‚त्त्व‚म् इति कुर्यात्\edtext{}{\lemma{कुर्यात्}\Bfootnote{इति ब्रूयात्--\cite{dp-msB} \cite{dp-msD}}}‚{\tiny $_{lb}$}‚ श्राव‚ण‚त्व‚मेव हेतुः स्यात् । निश्चित‚ग्र‚ह‚णेन स‚न्दिग्धासिद्धः \edtext{}{\lemma{न्दिग्धासिद्धः}\Bfootnote{अस‚र्व‚ज्ञः क‚श्चित् व‚क्तृत्वात्--\cite{dp-msD-n}}}स‚र्वो निर‚स्तः ।
	\pend% ending standard par
       ‚{\tiny $_{lb}$}‚ 

	  \pstart \leavevmode% starting standard par
	स‚प‚क्षो व‚क्ष्य‚माण‚ल‚क्ष‚णः । त‚स्मिन्नेव स‚त्त्वं निश्चित‚मिति द्वितीयं रूप‚म् । इहापि
	\pend% ending standard par
      
	  \endgroup
	‚{\tiny $_{lb}$}‚

	  \pstart \leavevmode% starting standard par
	अत्र केचिदेवं प्र‚तिविद‚ध‚ति । जिज्ञासित‚ध‚र्म‚विशेष‚व‚त्त्वेन हि रूपेण ध‚र्मी प‚क्ष उच्य‚ते ।‚{\tiny $_{lb}$}‚ य‚श्चासौ प‚र्व‚तादिस्त‚त्र न व‚ह्निर्जिज्ञासितः । किन्त‚र्हि ? य‚त्र त‚त्रोद्देशे । त‚त्रैक‚देश‚स्थ‚धूम‚{\tiny $_{lb}$}‚द‚र्श‚नेऽप्येक‚देश‚व‚ह्निर्जिज्ञासितो ज्ञाय‚त इति किं स‚र्व‚प‚र्व‚त‚व्यापिना धूमेन कार्य‚म् ? य‚दि पुनः‚{\tiny $_{lb}$}‚ स‚र्व‚त्रैव व‚ह्निम‚त्त्वं जिज्ञासितं स्यात्, स्यादेवायं प‚क्षैक‚देशासिद्धः । न चैत‚देव‚म् । त‚तः‚{\tiny $_{lb}$}‚ क‚थ‚म‚स्य त‚थात्व‚म् ? अत्र पुनः स‚र्वेषामेव त‚रूणां चैत‚न्यं जिज्ञासित‚मिति स‚क‚ल‚पाद‚प‚व्यापिनैव‚{\tiny $_{lb}$}‚ स्वापेन प्र‚योज‚न‚म् । य‚था वेद‚स्य स‚र्व‚स्यैव पौरुषेय‚त्वे साध्ये स‚क‚लाम्नाय‚व्याप‚केनैव वाक्य‚त्वेन‚{\tiny $_{lb}$}‚ प्र‚योज‚न‚म् । न चायं स‚र्वान् व्याप्नोति । त‚तः प‚क्षैक‚देशासिद्ध उच्य‚ते ।
	\pend% ending standard par
      ‚{\tiny $_{lb}$}‚

	  \pstart \leavevmode% starting standard par
	एके तु--लोकाध्य‚व‚साय‚सिद्धं म‚हीध‚रादेरेक‚त्व‚म‚व‚ल‚म्ब्यायं व्य‚व‚हारः, न च स‚र्वेषां‚{\tiny $_{lb}$}‚ त‚रूणां त‚थैक‚त्वं लोकोऽध्य‚व‚स्य‚ति, येन स्वाप‚स्यापि त‚था सिद्धिर्भ‚व‚ति, त‚तः किम‚व‚द्यं‚{\tiny $_{lb}$}‚ नामेति प्र‚तिप‚न्नाः ।
	\pend% ending standard par
      ‚{\tiny $_{lb}$}‚

	  \pstart \leavevmode% starting standard par
	अथाभिधीय‚ते--य‚दि क‚श्चित्तिन्तिडिकाप्र‚भृतीनेव पाद‚पान् प‚क्ष‚यित्वा स्वापं हेतूक‚रोति,‚{\tiny $_{lb}$}‚ त‚दाऽयं न प‚क्षैक‚देशासिद्ध इति किं न साध‚येच्चैत‚न्य‚मिति ।
	\pend% ending standard par
      ‚{\tiny $_{lb}$}‚

	  \pstart \leavevmode% starting standard par
	अस‚देत‚त्--विक‚ल्पानुप‚प‚त्तेः । य‚दि स‚र्व‚ज‚न‚प्र‚सिद्ध इन्द्रिय‚व्यापार‚विरोध्य‚व‚स्थाविशेषः‚{\tiny $_{lb}$}‚ स्वापो निद्राप‚र‚नामा हेतुर‚भिप्रेत‚स्त‚दाऽयं तेष्व‚पि त‚रुष्व‚सिद्ध इति क‚थं चैत‚न्य‚म‚नुमाप‚येत् ?
	\pend% ending standard par
      ‚{\tiny $_{lb}$}‚

	  \pstart \leavevmode% starting standard par
	अथ येन केन‚चिदुपाधिना स्वाप‚श‚ब्द‚मात्र‚वाच्योऽर्थो हेतुः; त‚थाविध‚स्य स्वाप‚स्य चैत‚न्येन‚{\tiny $_{lb}$}‚ व्याप्त्य‚सिद्धेः स‚न्दिग्ध‚विप‚क्ष‚व्यावृत्तिक‚त‚याऽनैकान्तिकः ।
	\pend% ending standard par
      ‚{\tiny $_{lb}$}‚

	  \pstart \leavevmode% starting standard par
	\hphantom{.}ष‚ष्ठ्य‚त‚स‚र्थ‚प्र‚त्येय‚न\href{http://sarit.indology.info/?cref=Pā.2.3.30}{पाणिनि २. ३. ३०}इत्य‚नेन प‚श्चाच्छ‚ब्द‚योगे \textbf{स‚त्त्व‚व‚च‚नात्ष}ष्ठी ।‚{\tiny $_{lb}$}‚ \textbf{असाधार‚णः} स‚प‚क्षास‚प‚क्ष‚साधार‚णो यो न भ‚व‚ति । प‚क्ष‚स्यैव यो ध‚र्म इति याव‚त् । \textbf{निर‚स्तो}‚{\tiny $_{lb}$}‚ हेतुत्वेन प्र‚तिक्षिप्तः । पूर्वाव‚धार‚णे तु नायं निराकृतः स्यादिति द‚र्श‚य‚ति \textbf{य‚दी}ति । \textbf{हि}र्य‚स्मात् ।‚{\tiny $_{lb}$}‚ अय‚म‚स्याश‚यः--य‚द्य‚नुमेय एव स‚त्त्वं य‚स्येति ल‚क्ष‚णं स्यात्त‚दा स‚प‚क्ष एव स‚त्त्व‚मिति व‚च‚न‚म‚ति‚{\tiny $_{lb}$}‚रिच्य‚मानं ल‚क्ष‚णान्त‚रं भ‚विष्य‚ति । व्याघात एव वा भ‚विष्य‚तीति । निश्चित‚ग्र‚ह‚ण‚स्येहा‚{\tiny $_{lb}$}‚पेक्षित‚स्य व्य‚व‚च्छेद्यं द‚र्श‚य‚ति--\textbf{निश्चि}तेति । स‚न्दिग्ध‚श्चासाव‚स्य हेतुत्वेन विशेष‚ण‚त्वेन‚{\tiny $_{lb}$}‚ चासिद्धोऽनिश्चित‚श्चेति विग्र‚हः । स‚न्दिग्ध‚विशेष‚णासिद्धः, स‚न्दिग्ध‚विशेष्यासिद्ध‚श्च स‚न्दिग्धासिद्ध‚{\tiny $_{lb}$}‚ एवान्त\leavevmode\ledsidenote{\textenglish{39b/ms}}र्भ‚व‚तीति, सोऽप्य‚नेनैव निर‚स्तः । य‚थाऽन‚योर्भेदो य‚था चान्त‚र्भाव‚स्त‚था‚{\tiny $_{lb}$}‚प‚र‚स्तात्प्र‚द‚र्श‚यिष्यामः । इहाश्रित‚त्वादार्थेन न्यायेन हेतुस‚त्त्व‚स्य विशेष‚ण‚रूप‚त्वाद् अनेन‚{\tiny $_{lb}$}‚ स‚होदिति\edtext{}{\lemma{होदिति}\Bfootnote{त}}एव‚कारो ध‚र्मायोग‚स्य व्य‚व‚च्छेद‚को द्र‚ष्ट‚व्यः ।
	\pend% ending standard par
      ‚{\tiny $_{lb}$}‚‚{\tiny $_{lb}$}‚\textsuperscript{\textenglish{94/dm}}‚{\tiny $_{lb}$}‚
	  \bigskip
	  \begingroup
	

	  \pstart \leavevmode% starting standard par
	स‚त्त्व‚ग्र‚ह‚णेन विरुद्धो निर‚स्तः । स हि नास्ति स‚प‚क्षे । एव‚कारेण साधार‚णानैकान्तिकः\edtext{}{\lemma{णानैकान्तिकः}\Bfootnote{अत्र \cite{dp-msD} प्र‚तौ प‚ङ्क्तिबाह्य‚भागे अनित्यः श‚ब्दः प्र‚मेय‚त्वात् इति टिप्प‚णं व‚र्त्त‚ते ।‚{\tiny $_{lb}$}‚ त‚च्च \cite{dp-msB} प्र‚तौ मूले निवेशित‚म् इति प्र‚तिभाति । \cite{dp-edH} \cite{dp-edN} प्र‚ताव‚पि एव‚मेव--सं०}} ।‚{\tiny $_{lb}$}‚ स हि न स‚प‚क्ष एव व‚र्त्त‚त किन्तूभ‚य‚त्रापि । स‚त्त्व‚ग्र‚ह‚णात् पूर्वाव‚धार‚ण‚व‚च‚नेन \edtext{}{\lemma{नेन}\Bfootnote{स‚प‚क्ष‚व्या० \cite{dp-msB} \cite{dp-edH}}}स‚प‚क्षाव्यापि‚{\tiny $_{lb}$}‚स‚त्ताक‚स्यापि प्र‚य‚त्नान‚न्त‚रीय‚क‚स्य हेतुत्वं क‚थित‚म् । प‚श्चाद‚व‚धार‚णे\edtext{}{\lemma{णे}\Bfootnote{०णे हि अय० \cite{dp-msA} \cite{dp-msC}}} त्व‚य‚म‚र्थः स्यात्—‚{\tiny $_{lb}$}‚स‚प‚क्षे स‚त्त्व‚मेव य‚स्य स हेतुरिति प्र‚य‚त्नान‚न्त‚रीय‚क‚त्वं न हेतुः स्यात् । \edtext{\textsuperscript{*}}{\lemma{*}\Bfootnote{निश्च‚य‚व‚च० \cite{dp-msA}}}निश्चित‚व‚च‚नेन‚{\tiny $_{lb}$}‚ स‚न्दिग्धान्व‚योऽनैकान्तिको निर‚स्तः । य‚था स‚र्व‚ज्ञः क‚श्चिद् व‚क्तृत्वात् । व‚क्तृत्वं हि स‚प‚क्षे‚{\tiny $_{lb}$}‚ स‚न्दिग्ध‚म् ।
	\pend% ending standard par
       ‚{\tiny $_{lb}$}‚ 

	  \pstart \leavevmode% starting standard par
	अस‚प‚क्षो व‚क्ष्य‚माण‚ल‚क्ष‚णः । त‚स्मिन्न‚स‚त्त्व‚मेव निश्चित‚म्--तृतीयं रूप‚म् । त‚त्रा‚{\tiny $_{lb}$}‚स‚त्त्व‚ग्र‚ह‚णेन विरुद्ध‚स्य निरासः\edtext{}{\lemma{निरासः}\Bfootnote{पूर्व‚व‚द‚त्रापि \cite{dp-msD} प्र‚तौ नित्यः श‚ब्दः कृत‚क‚त्वात् ख‚व‚त् इति प‚ङ्क्तिबाह्यं टिप्प‚ण‚त्वेन‚{\tiny $_{lb}$}‚ लिखितः पाठः \cite{dp-msB} प्र‚तौ मूल‚त्वेन संनिविष्ट इति भाति--\cite{dp-edN} प्र‚तौ अपि त‚थैव कृत‚म् । \cite{dp-edH} प्र‚तौ‚{\tiny $_{lb}$}‚ तु विप‚क्षैक‚देश‚वृत्तेर्निरासः इत्य‚न‚न्त‚रं मुद्रितः--सं०}} । विरुद्धो हि विप‚क्षेऽस्ति । एव‚कारेण साधार‚ण‚स्य विप‚क्षैक‚{\tiny $_{lb}$}‚देश‚वृत्तेर्निरासः । \edtext{\textsuperscript{*}}{\lemma{*}\Bfootnote{प्र‚य‚त्नान‚न्त‚रीय‚कः श‚ब्दः, अनित्य‚त्वात् घ‚ट‚व‚त्--\cite{dp-msD-n}}}प्र‚य‚त्नान‚न्त‚रीय‚क‚त्वे साध्ये ह्य‚नित्य‚त्वं विप‚क्षैक‚देशे विद्युदादौ अस्ति,‚{\tiny $_{lb}$}‚ आकाशादौ नास्ति । त‚तो \edtext{}{\lemma{तो}\Bfootnote{निय‚म‚तोऽस्य \cite{dp-msC}}}निय‚मेनास्य निरासः । \edtext{\textsuperscript{*}}{\lemma{*}\Bfootnote{अस‚त्त्व‚व‚च‚नात् पूर्व० \cite{dp-msA} \cite{dp-edP} \cite{dp-edH} \cite{dp-edE} \cite{dp-edN} ०श‚ब्दात् पूर्व० \cite{dp-msB} \cite{dp-msC} \cite{dp-msD}}}अस‚त्त्व‚श‚ब्दाद्धि पूर्व‚स्मिन्न‚व‚धार‚णेऽय‚म‚र्थः‚{\tiny $_{lb}$}‚ स्यात्--विप‚क्ष एव यो नास्ति स हेतुः । त‚था च प्र‚य‚त्नान‚न्त‚रीय‚क‚त्वं स‚प‚क्षेऽपि \edtext{}{\lemma{क्षेऽपि}\Bfootnote{०पि नास्ति--\cite{dp-msB} \cite{dp-edE}}}स‚र्व‚त्र
	\pend% ending standard par
      
	  \endgroup
	‚{\tiny $_{lb}$}‚

	  \pstart \leavevmode% starting standard par
	\textbf{साधार‚ण}श्चासौ स‚प‚क्षास‚प‚क्ष‚वृत्तित्वाद‚नैकान्तिक‚श्चैक‚स्मिन्न‚न्ते साध्ये विप‚र्य‚ये वाऽव्य‚व‚{\tiny $_{lb}$}‚स्थित‚श्चेति त‚था निर‚स्त इति व‚र्त्त‚ते । \textbf{स‚प‚क्षाव्यापिनी} स‚क‚ल‚स‚प‚क्षाव‚र्त्तिनी \textbf{स‚त्ता} य‚स्येति‚{\tiny $_{lb}$}‚ विग्र‚हः । प्र‚य‚त्नान‚न्त‚रीय‚क‚त्व‚म‚प्य‚नित्य‚त्व‚सिद्धौ स‚म‚र्थो हेतुरिति द‚र्श‚यितुं स‚क‚ल‚स‚प‚क्षाव्यापि‚{\tiny $_{lb}$}‚ प्र‚य‚त्नान‚न्त‚रीय‚क‚त्व‚मुदाहृत‚मिति द्र‚ष्ट‚व्य‚म् । व‚स्तुत‚स्तु स‚र्व एव कार्य‚हेतुर्धूमादिः स‚प‚क्षैक‚{\tiny $_{lb}$}‚देश‚वृत्तिर्बोद्ध‚व्यः । \textbf{इति}रेव‚म‚र्थे । एवं स‚ति प्र‚य‚त्नान‚न्त‚रीय‚क‚त्व‚मुप‚ल‚क्ष‚ण‚त्वाद‚स्य धूमादिक‚{\tiny $_{lb}$}‚म‚पि न हेतुः स्यात् । \textbf{स‚न्दिग्धोऽन्व‚यो} य‚स्य स त‚था । इहाश्र‚य‚त्वादार्थेन न्यायेन स‚प‚क्ष‚{\tiny $_{lb}$}‚ल‚क्ष‚ण‚स्य ध‚र्मिणो विशेष्य‚त्वाद‚नेन स‚होदितो निपातः प‚क्षे वृक्षे वृत्तौ ल‚ब्धायां स‚मुच्चीय‚{\tiny $_{lb}$}‚मानाव‚धार‚ण‚त्वाद‚स‚प‚क्ष‚ल‚क्ष‚ण‚ध‚र्म्य‚न्त‚र‚योग‚स्य व्य‚व‚च्छेद‚को द्र‚ष्ट‚व्यः ।
	\pend% ending standard par
      ‚{\tiny $_{lb}$}‚

	  \pstart \leavevmode% starting standard par
	तृतीयं रूपं व्याख्यातुमाह--\textbf{अस‚प‚क्ष} इति । \textbf{व‚क्ष्य‚माणं ल‚क्ष‚ण}म‚स्येति विग्र‚हः ।‚{\tiny $_{lb}$}‚ क‚थ‚म‚स्य निरास इत्याह--\textbf{विरुद्धो ही}ति । हिर्य‚स्मात् । \textbf{साधार‚ण‚स्य} स‚प‚क्षास‚प‚क्ष‚{\tiny $_{lb}$}‚साधार‚ण‚स्य । क‚स्मिन् साध्ये किन्त‚दीदृश‚मित्याह--\textbf{प्र‚य‚त्ने}ति । \textbf{प्र‚य‚त्नः} पुरुष‚व्यापारः । \textbf{निय‚मेना}‚{\tiny $_{lb}$}‚‚{\tiny $_{lb}$}‚ ‚{\tiny $_{lb}$}‚ \leavevmode\ledsidenote{\textenglish{95/dm}}‚{\tiny $_{lb}$}‚ 
	  
	नास्ति । त‚तो न हेतुः स्यात् । त‚तः पूर्वं न कृत‚म् । निश्चित‚ग्र‚ह‚णेन स‚न्दिग्ध‚विप‚क्ष‚{\tiny $_{lb}$}‚व्यावृत्तिकोऽनैकान्तिको\edtext{}{\lemma{व्यावृत्तिकोऽनैकान्तिको}\Bfootnote{अस‚र्व‚ज्ञः क‚श्चित् व‚क्तृत्वात्--\cite{dp-msD-n}}} निर‚स्तः । ‚{\tiny $_{lb}$}‚ 
	  
	न‚नु च स‚प‚क्ष एव स‚त्त्व‚मित्युक्ते विप‚क्षेऽस‚त्त्व‚मेवेति य‚म्य‚त एव । त‚त् किम‚र्थं \edtext{}{\lemma{र्थं}\Bfootnote{०र्थ‚मुभ \cite{dp-msC} \cite{dp-msD}}}पुन‚रुभ‚यो‚{\tiny $_{lb}$}‚रुपादानं कृत‚म् ? उच्य‚ते \edtext{}{\lemma{ते}\Bfootnote{त‚दुच्य‚ते--\cite{dp-msA} \cite{dp-edP} \cite{dp-edH} \cite{dp-edE}}}। अन्व‚यो व्य‚तिरेको वा \add{प्र‚युज्य‚मानः} निय‚म‚वानेव प्र‚योक्त‚व्यो‚{\tiny $_{lb}$}‚ नान्य‚थेति द‚र्श‚यितुं\edtext{}{\lemma{यितुं}\Bfootnote{द्व‚योरुपादानं--\cite{dp-msB} \cite{dp-msC} \cite{dp-msD}}}द्व‚योर‚प्युपादानं कृत‚म् । अनिय‚ते\edtext{}{\lemma{ते}\Bfootnote{अनिय‚मे हि \cite{dp-msA} \cite{dp-edP} \cite{dp-edH} \cite{dp-edE} \cite{dp-edN}}} हि द्व‚योर‚पि प्र‚योगेऽय‚म‚र्थः स्यात्—‚{\tiny $_{lb}$}‚स‚प‚क्षे योऽस्ति विप‚क्षे च यो\edtext{}{\lemma{यो}\Bfootnote{च नास्ति स \cite{dp-msA} \cite{dp-msB} \cite{dp-msD} \cite{dp-edP} \cite{dp-edH} \cite{dp-edE} \cite{dp-edN}}} नास्ति स हेतुरिति । त‚था च स‚ति स श्यामः\edtext{}{\lemma{श्यामः}\Bfootnote{श्यामः त्व‚त्पुत्र० \cite{dp-msC}}}त‚त्पुत्र‚त्वात् दृश्य‚मान-‚{\tiny $_{lb}$}‚ व‚धार‚णेन एव‚कार‚णेति याव‚त् । अस‚प‚क्ष एवेति किं नाव‚धार्य‚त इत्याह--\textbf{अस‚त्त्वे}ति ।‚{\tiny $_{lb}$}‚ \textbf{हि}र्य‚स्मात् । भ‚व‚त्वेवं का क्ष‚तिरित्याह \textbf{त‚था चे}ति । प्र‚य‚त्नान‚न्त‚रीय‚क‚ग्र‚ह‚णं पूर्व‚व‚दुप‚{\tiny $_{lb}$}‚ल‚क्ष‚ण‚म् । निश्चित‚ग्र‚ह‚ण‚स्य व्यावृत्त्य‚र्थ‚माह--\textbf{निश्चिते}ति । विप‚क्षाद् व्यावृत्ति\textbf{र्विप‚क्ष‚व्यावृत्तिः} ।‚{\tiny $_{lb}$}‚ स‚न्दिग्धा विप‚क्ष‚व्यावृत्तिर्य‚स्य स त‚था ।
	\pend% ending standard par
      ‚{\tiny $_{lb}$}‚

	  \pstart \leavevmode% starting standard par
	न‚नु न स‚न्दिग्ध‚विप‚क्ष‚व्यावृत्तिक‚त्वं नागायं हेतुदोषः । त‚त्क‚थं निर‚स्य‚ते ? त‚थाहि—‚{\tiny $_{lb}$}‚य एव विप‚क्षे वीक्षितो हेतुः स एव प्र‚मेय‚त्वादिव‚द‚भिम‚तं न साध‚येत् । यः पुन‚र्म‚ह‚ताऽपि‚{\tiny $_{lb}$}‚ प्र‚य‚त्नेन मृग्य‚माणोऽस‚प‚क्षे नोप‚ल‚क्षितः स क‚थ‚म‚ङ्ग साध्यं न साध‚येदिति ? त‚देत‚द‚व‚द्य‚म् ।‚{\tiny $_{lb}$}‚ य‚तो योऽपि विप‚क्षे वीक्षितो हेतुः सोऽपि इष्टो दुष्टः । क‚थं ? साध्यं विनाऽप्युप‚ल‚ब्धेरिति‚{\tiny $_{lb}$}‚ चेत् । न‚नु य‚दि नामासौ साध्य‚म‚न्त‚रेणान्य‚त्र दृष्ट‚स्त‚थापि विवादाध्यासिते ध‚र्मिणि साध्यं‚{\tiny $_{lb}$}‚ किं न साध‚य‚ति ? न‚हि अय‚म‚त्रापि साध्यं विनैव व‚र्त्त‚त इति प्र‚साध‚कं प्र‚माण‚म‚स्ति । न चैक‚त्र‚{\tiny $_{lb}$}‚ येन विना यो दृष्टः स‚र्व‚त्रासौ तेन विनैव व‚र्त्त‚ते इति सिद्ध‚म् । अन्य‚थाऽपि ब‚हुलं द‚र्श‚नात् ।‚{\tiny $_{lb}$}‚ अथ साध्य‚म‚न्त‚रेण यो वृत्तः साध्ये स‚त्येवासौ व‚र्त्त‚त इति निय‚माभावाद् विवादाध्यासिते स‚न्देह‚{\tiny $_{lb}$}‚हेतुर्न स‚म्य‚ग् हेतुर्भ‚वितुम‚र्ह‚ति । ह‚न्त त‚र्हि विप‚क्षे वीक्ष‚णं त‚स्य साध्य‚ध‚र्मिणि साध्य‚विनाकृतां‚{\tiny $_{lb}$}‚ वृत्तिं स‚म्भाव‚य‚द् हेतुम‚निश्च‚य‚हेतूक‚रोतीति आयात‚म् । स‚ति चैवं साध्य‚विप‚र्य‚ये हेतुस‚त्ता‚{\tiny $_{lb}$}‚वाध‚क‚प्र‚माणाद‚र्श‚न‚म‚पि--य‚द्य‚यं च ध‚र्मो भ‚विष्य‚ति, न च त‚तः साध्य‚मित्येवंविधां वृत्तिम‚स्य—‚{\tiny $_{lb}$}‚स‚म्भाव‚य‚ति त‚दा किम‚यं निश्च‚य‚हेतुर्भ‚वितुम‚र्ह‚ति ? त‚तो वि\leavevmode\ledsidenote{\textenglish{40a/ms}}प‚क्षे द‚र्श‚नं वा त‚था श‚ङ्का‚{\tiny $_{lb}$}‚बीज‚म‚स्तु विप‚र्य‚ये वा बाध‚क‚प्र‚माणाद‚र्श‚नं वेति को विशेषः ? त‚था चाह \textbf{वार्त्तिक‚कारः}--न‚{\tiny $_{lb}$}‚ तु स‚प‚क्ष‚विप‚क्ष‚योः स‚त्त्व‚म‚स‚त्त्वं वा निश्च‚यापेक्ष‚म् । निश्च‚येऽपि स‚न्देह‚मुखेनैव दोषात् ।‚{\tiny $_{lb}$}‚ सोऽनिश्च‚येऽपि तुल्य इति त‚थाविधोद्भाव‚न‚म‚प्य‚त्र दूष‚ण‚मेव । त‚था विप‚क्ष‚प्र‚चाराश‚ङ्काव्य‚{\tiny $_{lb}$}‚व‚च्छेदेन ल‚भ्यं ग‚म‚क‚त्वं क‚थ‚मात्म‚सात्कुर्याद् इति ।
	\pend% ending standard par
      ‚{\tiny $_{lb}$}‚

	  \pstart \leavevmode% starting standard par
	\textbf{न‚नु चे}त्यादिना ल‚क्ष‚णे चोद्य‚माश‚ङ्क‚ते । \textbf{विप‚क्षेऽस‚त्त्व‚मेवेति ग‚म्य‚त इति} ब्रुव‚तोऽयं‚{\tiny $_{lb}$}‚ भावः । प‚क्षे वृत्तौ ल‚ब्धायां स‚प‚क्ष एव स‚त्त्वं निश्चित‚मिति ख‚लु स‚मुच्चीय‚मानाव‚धार‚णोऽयं‚{\tiny $_{lb}$}‚ निर्ण‚यः । अयं चास‚प‚क्षेऽस‚त्त्वेऽनिश्चिते स‚न्दिग्धे वा न घ‚ट‚त इति अव‚श्य‚म‚स‚प‚क्षेऽस‚त्त्व‚मेव‚{\tiny $_{lb}$}‚ ‚{\tiny $_{lb}$}‚ ‚{\tiny $_{lb}$}‚ ‚{\tiny $_{lb}$}‚ \leavevmode\ledsidenote{\textenglish{96/dm}}‚{\tiny $_{lb}$}‚ 
	  
	पुत्र‚व‚दिति त‚त्पुत्र‚त्वं हेतुः स्यात् । त‚स्मान्निय‚म‚व‚तोरेवान्व‚य‚व्य‚तिरेक‚योः प्र‚योगः क‚र्त्त‚व्यो येन‚{\tiny $_{lb}$}‚ प्र‚तिब‚न्धो ग‚म्येत साध‚न‚स्य साध्येन । निय‚म‚व‚तोश्च प्र‚योगेऽव‚श्य‚क‚र्त्त‚व्ये द्व‚योरेक एव प्र‚योक्त‚व्यो\edtext{}{\lemma{व्यो}\Bfootnote{एव क‚र्त्त‚व्यो न \cite{dp-msA}}}‚{\tiny $_{lb}$}‚ न द्वाविति निय‚म‚वानेवान्व‚यो व्य‚तिरेको वा प्र‚योक्त‚व्य इति शिक्ष‚णार्थं द्व‚योरुपादान‚मिति ॥ ‚{\tiny $_{lb}$}‚ 
	  
	त्रैरूप्य‚क‚थ‚न‚प्र‚स‚ङ्गेनानुमेयः स‚प‚क्षो विप‚क्ष‚श्चोक्तः । तेषां ल‚क्ष‚णं व‚क्त‚व्य‚म् । त‚त्र‚{\tiny $_{lb}$}‚ कोऽनुमेय इत्याह--‚{\tiny $_{lb}$}‚ 
	  
	अनुमेयोऽत्र जिज्ञासित‚विशेषो ध‚र्मी ॥ ६ ॥‚{\tiny $_{lb}$}‚ निश्चित‚माक्षिप‚तीति । एव‚मुप‚ल‚क्ष‚ण‚त्वाद‚स्यास‚प‚क्षे चास‚त्त्व‚मेव निश्चित‚मित्य‚नेन स‚प‚क्षे‚{\tiny $_{lb}$}‚ वृत्तिमात्रं ल‚ब्ध‚मेव । तेनैव च प्र‚योज‚न‚मिति किं स‚प‚क्ष एव स‚त्त्व‚मित्य‚नेनेति द्र‚ष्ट‚व्य‚म् ।
	\pend% ending standard par
      ‚{\tiny $_{lb}$}‚

	  \pstart \leavevmode% starting standard par
	\textbf{उच्य‚त} इति प‚रिहारः । साध्येनान्वीय‚मान‚त्वं स‚त्येव साध्ये भ‚वितृत्वं साध‚न‚स्य‚{\tiny $_{lb}$}‚ स्व‚ग‚तो \textbf{ध‚र्मोऽन्व‚यः} । साध्याभावे व्य‚तिरिच्य‚मान‚त्व‚म् अभ‚वितृत्वं हेतोः स्व‚ग‚तो ध‚र्मो‚{\tiny $_{lb}$}‚ \textbf{व्य‚तिरेकः} । स चास‚श्च \edtext{}{\lemma{श्च}\Bfootnote{प्र‚युज्य‚मानः इति पाठः मूले नोप‚ल‚भ्य‚ते । प्र‚दीपानुरोधात्‚{\tiny $_{lb}$}‚ त‚त्र कोष्ठ‚के स्थापितः ।}}\textbf{प्र‚युज्य‚मानः} श‚ब्देन प्र‚तिपाद्य‚मानो \textbf{निय‚म‚वान}व्य‚भिचार‚वान्‚{\tiny $_{lb}$}‚ \textbf{प्र‚योक्त‚व्यो}ऽन्य‚था प‚रोक्ष‚प्र‚तीत्य‚ङ्गं नोक्तं स्यात् । क‚थं च त‚था प्र‚युक्तो भ‚व‚ति ? य‚दाऽन्व‚य‚{\tiny $_{lb}$}‚वाक्ये साध्ये निय‚तं साध‚नं व्य‚तिरेक‚वाक्ये च साध‚नाभावे निय‚तः साध्याभावः । साध‚नानु‚{\tiny $_{lb}$}‚वाद‚पूर्व‚साध्य‚विधानेन साध्याभावानुवाद‚पूर्व‚साध‚नाभाव‚विधानेन वीप्साप‚द‚युक्तेन स‚र्व‚श‚ब्द‚{\tiny $_{lb}$}‚संहितेन एव‚कारोपेतेन । अन्य‚था न प्र‚तिपाद्य‚ते ।
	\pend% ending standard par
      ‚{\tiny $_{lb}$}‚

	  \pstart \leavevmode% starting standard par
	अनिय‚मेनापि प्र‚योगे य‚द्य‚न्व‚य‚व्य‚तिरेक‚योस्तादृशोः प्र‚तीतिर‚स्ति त‚दा किं निय‚म‚व‚त्प्र‚योगेणे‚{\tiny $_{lb}$}‚त्याह--\textbf{अनिय‚ते हीति} । हिर्य‚स्माद‚निय‚ते विव‚क्षित‚निय‚माख्याप‚के । आस्तां ताव‚देक‚स्य‚{\tiny $_{lb}$}‚ प्र‚योगे द्व‚योर‚प्य‚य‚म‚र्थः स्यादित्य‚पिश‚ब्दः । \textbf{इति}र‚र्थ‚स्याकारं द‚र्श‚य‚ति । अस्त्व‚य‚म‚र्थः । का‚{\tiny $_{lb}$}‚ क्ष‚तिरित्याह--\textbf{त‚था चेति} । काक‚तालीय‚न्यायेन चाश्यामेषु निक‚ट‚व‚र्त्तिषु अत‚त्पुत्रेषु एत‚द्‚{\tiny $_{lb}$}‚ द्र‚ष्ट‚व्य‚म् । अन्य‚था ह्य‚निय‚तोऽपि व्य‚तिरेकोऽत्र नोक्त इति व‚च‚नाव‚काशः स्यात् । य‚त‚{\tiny $_{lb}$}‚ \textbf{एवं त‚स्माद्} येन निय‚म‚व‚त्प्र‚योगेण प्र‚तिब‚न्धः प्र‚तिब‚द्ध‚त्व‚माय‚त्त‚त्व‚म् । क‚स्येत्याकाङ्क्षायामाह—‚{\tiny $_{lb}$}‚\textbf{साध‚न‚स्ये}ति । कुत्रेत्य‚पेक्षायामाह--\textbf{साध्य} इति ।
	\pend% ending standard par
      ‚{\tiny $_{lb}$}‚

	  \pstart \leavevmode% starting standard par
	य‚द्येव‚मेक‚स्मिन्नेव साध‚न‚वाक्ये निय‚म‚वानेवान्व‚यो व्य‚तिरेक‚श्च प्र‚युज्य‚तामित्याह--\textbf{निय‚म‚{\tiny $_{lb}$}‚व‚तोश्चे}ति । \textbf{चो} व‚क्त‚व्यान्त‚र‚स‚मुच्च‚ये । अय‚माश‚यः--अन्व‚य‚वाक्येनापि त‚थाप्र‚युक्तेन‚{\tiny $_{lb}$}‚ साम‚र्थ्याद् व्य‚तिरेक‚स्य, व्य‚तिरेक‚वाक्येनापि साम‚र्थ्याद‚न्व‚य‚स्य प्र‚काश‚नात् किं प्र‚तीत‚प्र‚त्याय‚केन‚{\tiny $_{lb}$}‚ द्वितीय‚वाक्येन कार्य‚मिति ? य‚त एव‚मितिस्त‚स्माद् । द्वितीय \textbf{इतिः} शिक्ष‚ण‚स्य स्व‚रूपं द‚र्श‚य‚ति ।‚{\tiny $_{lb}$}‚ \textbf{शिक्ष‚ण}म‚ज्ञ‚मुद्दिश्य‚ज्ञाप‚न‚म् । त‚द‚र्थं त‚न्निमित्त‚म् । अनेन प्र‚यो\leavevmode\ledsidenote{\textenglish{40b/ms}}ग‚स‚मास एव द‚र्शितो न‚{\tiny $_{lb}$}‚ रूप‚स‚मास इति द‚र्शित‚म् । प्र‚योग‚द‚र्श‚नाभ्यासात्क‚श्चित्प्र‚योग‚भ‚ङ्ग्यैव स्व‚य‚म‚पि प‚रोक्ष‚म‚र्थं‚{\tiny $_{lb}$}‚ प्र‚तिप‚द्य‚त इति स्वार्थेऽप्य‚नुमाने निर्णीत‚मिदं ल‚क्ष‚ण‚क‚थ‚न‚प्र‚स‚ङ्गेन । त‚तो न दोष इति द्र‚ष्ट‚व्य‚म् ।‚{\tiny $_{lb}$}‚ एत‚च्चोप‚रिष्टान्निवेद‚यिष्य‚ते ॥
	\pend% ending standard par
      ‚{\tiny $_{lb}$}‚‚{\tiny $_{lb}$}‚\textsuperscript{\textenglish{97/dm}}‚{\tiny $_{lb}$}‚
	  \bigskip
	  \begingroup
	

	  \pstart \leavevmode% starting standard par
	अनुमेयोऽत्रेत्यादि । अत्र हेतुल‚क्ष‚णे निश्चेत‚व्ये\edtext{}{\lemma{व्ये}\Bfootnote{ज्ञात‚व्ये प‚क्ष‚ध‚र्म‚त्वे प‚क्षो ध‚र्म्य‚भिधीय‚ते ।‚{\tiny $_{lb}$}‚ व्याप्तिकाले भ‚वेद् ध‚र्मः साध्य‚सिद्धौ पुन‚र्द्व‚य‚म् ॥--\cite{dp-msD-n}}} ध‚र्मो अनुमेयः । अन्य‚त्र तु \edtext{}{\lemma{तु}\Bfootnote{साध्ये प्र‚ति०--\cite{dp-msC}}}साध्य‚{\tiny $_{lb}$}‚प्र‚तिप‚त्तिकाले स‚मुदायोऽनुमेयः । व्याप्तिनिश्च‚य‚काले तु ध‚र्मोऽनुमेय इति द‚र्श‚यितुम् अत्र‚{\tiny $_{lb}$}‚ ग्र‚ह‚ण‚म् । जिज्ञासितो ज्ञातुमिष्टो विशेषो ध‚र्मो य‚स्य ध‚र्मिणः स त‚थोक्तः ॥
	\pend% ending standard par
       ‚{\tiny $_{lb}$}‚ 

	  \pstart \leavevmode% starting standard par
	कः स‚प‚क्षः ?
	\pend% ending standard par
       ‚{\tiny $_{lb}$}‚ 
	  \bigskip
	  \begingroup
	

	  \pstart \leavevmode% starting standard par
	साध्य‚ध‚र्म‚सामान्येन स‚मानोऽर्थः स‚प‚क्षः ॥ ७ ॥
	\pend% ending standard par
      
	  \endgroup
	‚{\tiny $_{lb}$}‚ 

	  \pstart \leavevmode% starting standard par
	स‚मानोऽर्थः स‚प‚क्षः । स‚मानः स‚दृशो\edtext{}{\lemma{दृशो}\Bfootnote{स‚दृशोऽर्थो यः प‚क्षेण \cite{dp-msC}}} योऽर्थः प‚क्षेण स\edtext{}{\lemma{स}\Bfootnote{प‚क्षेण स स‚प‚क्ष \cite{dp-msB} \cite{dp-msC} \cite{dp-msD} \cite{dp-edP} \cite{dp-edH} \cite{dp-edE} \cite{dp-edN}}} प‚क्ष उक्त उप‚चारात्‚{\tiny $_{lb}$}‚ स‚मान‚श‚ब्देन विशेष्य‚ते । स‚मानः प‚क्षः स‚प‚क्षः, स‚मान‚स्य च स‚श‚ब्दादेशः\edtext{}{\lemma{ब्दादेशः}\Bfootnote{स‚श‚ब्द आदेशः \cite{dp-edE}}} ।
	\pend% ending standard par
      
	  \endgroup
	‚{\tiny $_{lb}$}‚

	  \pstart \leavevmode% starting standard par
	न‚नु य‚दि ध‚र्म‚ध‚र्मिस‚मुदायो मुख्योऽनुमेयोऽत्र गृह्य‚ते त‚दा न क‚स्य‚चित्साध‚न‚स्य त‚द्ध‚र्म‚त्वं‚{\tiny $_{lb}$}‚ ग्र‚हीतुं श‚क्येत । ग्र‚ह‚णे वा व्य‚क्त‚मेव वैय‚र्थ्य‚मित्य‚भिप्राय‚वान् पृच्छ‚ति--त‚त्र क इति‚{\tiny $_{lb}$}‚ तेषु म‚ध्ये ।
	\pend% ending standard par
      ‚{\tiny $_{lb}$}‚

	  \pstart \leavevmode% starting standard par
	अत्रेत्य‚स्य तात्प‚र्यार्थ‚म‚त्रेत्यादिना निरूप‚य‚ति । ध‚र्मिध‚र्म‚योश्चानुमेय‚त्व‚म् । त‚स्मिं‚{\tiny $_{lb}$}‚स्त‚स्मिन् कालेऽनुमेयैक‚देश‚त्वादुप‚चार‚तो द्र‚ष्ट‚व्य‚म् । त‚दुक्त‚म्--
	\pend% ending standard par
      ‚{\tiny $_{lb}$}‚
	  \bigskip
	  \begingroup
	
	    \begin{quote}
	  
	    
	    \stanza[\smallbreak]
	स‚मुदाय‚स्य साध्य‚त्वात् ध‚र्म‚मात्रे च ध‚र्मिणि ।&अमुख्येऽप्येक‚देश‚त्वात्साध्य‚त्व‚मुप‚च‚र्य‚ते ॥ इति ॥\&[\smallbreak]


	
	    \end{quote}
	  
	  \endgroup
	‚{\tiny $_{lb}$}‚

	  \pstart \leavevmode% starting standard par
	अथ य‚दि स‚ह प‚क्षेण व‚र्त्त‚ते इति स‚प‚क्षोऽभिप्रेत‚स्त‚दाऽर्थास‚ङ्ग‚तिः । अथापि स‚मानः‚{\tiny $_{lb}$}‚ प‚क्षेणेति म‚तं त‚दा प‚क्ष‚स‚मान इति प्राप्नोतीत्य‚भिप्रेत्य पृच्छ‚ति क इति । \textbf{स‚मानोऽर्थ} इति‚{\tiny $_{lb}$}‚ सिद्धान्ती । स‚मान‚श‚ब्द‚स्यार्थ‚माह--\textbf{स‚मानः स‚दृशोऽर्थ} इति । अर्थ‚श‚ब्दोऽत्र प्र‚तीय‚मानार्थोऽर्थ्य‚ते‚{\tiny $_{lb}$}‚ ग‚म्य‚त इति कृत्वा । न त्व‚र्थोऽर्थ‚क्रियास‚म‚र्थो वाच्यः । क्र‚माक्र‚मायोगेनाक्ष‚णिक‚स्य साम‚र्थ्याभाव\edtext{}{\lemma{र्थ्याभाव}\Bfootnote{वे}}‚{\tiny $_{lb}$}‚ साध्येऽम्ब‚रार‚विन्दादेर‚पि स‚प‚क्ष‚त्वेनेष्ट‚त्वात् ।
	\pend% ending standard par
      ‚{\tiny $_{lb}$}‚

	  \pstart \leavevmode% starting standard par
	न‚नु स‚मान‚श्चासौ प‚क्ष‚श्चेति किं नाभिप्रेत‚म् ? त‚त्र प‚क्ष एवासौ दृष्टान्त‚ध‚र्मी क‚थं‚{\tiny $_{lb}$}‚ येन स‚मान‚श‚ब्देन विशिष्य‚त इत्याह--\textbf{प‚क्ष} इति । क‚थ‚म‚न्यार्थेन तेन श‚ब्देन स त‚थोच्य‚ता‚{\tiny $_{lb}$}‚मित्याह--\textbf{उप‚चारादि}ति । क्व‚चित्स‚प‚क्ष उक्त इति पाठः । त‚त्र सोऽर्थः प‚क्ष उक्तः‚{\tiny $_{lb}$}‚ इति योज्य‚म् । उप‚चारे च साध्य‚ध‚र्म‚योगो निमित्त‚म् । स उप‚चाराद् यः प‚क्ष उक्तः‚{\tiny $_{lb}$}‚ \textbf{विशि\edtext{}{\lemma{विशि}\Bfootnote{शे}}ष्य‚ते} व्य‚व‚च्छेद्य‚ते त‚द‚स‚मानात् । य‚दि स‚मान‚श‚ब्दो विशेष‚ण‚म‚स्य त‚र्हि स‚श‚ब्द‚श्रुतिः‚{\tiny $_{lb}$}‚ क‚थ‚मित्याह--\textbf{स‚मान‚स्येति । स‚मान‚स्य} स‚मान‚श‚ब्द‚स्य स्थाने स‚श‚ब्दादेश‚श्च स‚मान‚स्य ‚{\tiny $_{lb}$}‚ \href{http://sarit.indology.info/?cref=Pā.6.3.84}{पाणिनि ६. ३. ८४.}इति योग वेभागात् । स‚मानः प‚क्षो य‚स्य स त‚थेति ब‚हुव्रीहिः किमिति \textbf{ध‚र्मो‚{\tiny $_{lb}$}‚त्त‚रेण} नाश्रितो येनैव‚मात्मा प्र‚यासित इति \textbf{चे}त् । स‚त्य‚म् । केव‚लं \textbf{विनिश्च‚या}नुरोधादेव‚{\tiny $_{lb}$}‚‚{\tiny $_{lb}$}‚ ‚{\tiny $_{lb}$}‚ \leavevmode\ledsidenote{\textenglish{98/dm}}‚{\tiny $_{lb}$}‚ 
	  
	स्यादेत‚त्--किं त‚त् प‚क्ष‚स‚प‚क्ष‚योः \edtext{}{\lemma{योः}\Bfootnote{साम्य‚म् \cite{dp-msB}}}सामान्यं येन स‚मानः स‚प‚क्षः प‚क्षेणेत्याह--साध्य‚{\tiny $_{lb}$}‚ध‚र्म‚सामान्येनेति । साध्य‚श्चासौ असिद्ध‚त्वात्, ध‚र्म‚श्च प‚राश्रित‚त्वात् साध्य‚ध‚र्मः । न च‚{\tiny $_{lb}$}‚ विशेषः साध्यः, अपि तु सामान्य‚म् । अत इह सामान्यं \edtext{}{\lemma{सामान्यं}\Bfootnote{प‚क्ष--\cite{dp-msD-n}}}साध्य‚मुक्त‚म् । साध्य‚ध‚र्म‚श्चासौ‚{\tiny $_{lb}$}‚ सामान्यं चेति साध्य‚ध‚र्म‚सामान्येन स‚मानः प‚क्षेण स‚प‚क्ष इत्य‚र्थः ॥ ‚{\tiny $_{lb}$}‚ 
	  
	कोऽस‚प‚क्ष इत्याह-- ‚{\tiny $_{lb}$}‚ 
	  
	\textbf{न स‚प‚क्षोऽस‚प‚क्षः ॥ ८ ॥}‚{\tiny $_{lb}$}‚ 
	  
	न स‚प‚क्षोऽस‚प‚क्षः । स‚प‚क्षो यो न भ‚व‚ति सोऽस‚प‚क्षः ॥ ‚{\tiny $_{lb}$}‚ 
	  
	क‚श्च स‚प‚क्षो न भ‚व‚ति ?‚{\tiny $_{lb}$}‚ माच‚रित‚म् । \textbf{विनिश्च‚ये} हि न‚व‚प‚क्ष‚ध‚र्म‚प्र‚वेद‚न‚निर्देश‚प्र‚क‚र‚णे साध्य‚ध‚र्म‚सामान्येन स‚मानः प‚क्षः स‚प‚क्ष‚{\tiny $_{lb}$}‚स्त‚द‚भावोऽस‚प‚क्षः इत्युक्त‚म् । अतो \textbf{वार्त्तिक‚कार‚स्यैव} दृष्टान्त‚ध‚र्मी प‚क्षोऽभिप्रेत उप‚चारादित्य‚त्रा‚{\tiny $_{lb}$}‚प्येव‚म‚यं व्याच‚ष्ट इत्य‚दोषः ।
	\pend% ending standard par
      ‚{\tiny $_{lb}$}‚

	  \pstart \leavevmode% starting standard par
	\textbf{येन} सामान्येन \textbf{प‚क्षेण स‚मानः स‚प‚क्ष इत्या}हेति योज्य‚म् । \textbf{साध्य}श‚ब्देनोप‚चाराद्‚{\tiny $_{lb}$}‚ व‚ह्न्यादिक‚म‚भिप्रेत‚म् । साध्य‚त्वे हेतुमाह--\textbf{असिद्ध‚त्वा}त् त‚त्रानिश्चित‚त्वात् ।
	\pend% ending standard par
      ‚{\tiny $_{lb}$}‚

	  \pstart \leavevmode% starting standard par
	न‚नु साध्य‚ध‚र्म‚श्चासौ सामान्यं चेति क‚र्म‚धार‚य‚ग‚र्भः क‚र्म‚धार‚य इहाभिप्रेतः, विशेष‚श्च‚{\tiny $_{lb}$}‚ साध्य‚त इति क‚थं सामान्य‚श‚ब्देन साध्य‚ध‚र्म‚श‚ब्द‚स्य स‚मास इत्याश‚ङ्क्याह--\textbf{न चे}त्यादि । \textbf{चो}‚{\tiny $_{lb}$}‚ य‚स्माद‚र्थे अव‚धार‚णे वा । सामान्य‚म‚त‚द्रूप‚व्यावृत्त‚व‚स्तुमात्र‚म् । त‚थाविधेनैव हेतोः \leavevmode\ledsidenote{\textenglish{41a/ms}}‚{\tiny $_{lb}$}‚ व्याप्य‚त्वादित्य‚भिप्रायः । य‚त एव‚म‚तोऽस्माद्धेतोरिह स‚प‚क्ष‚ल‚क्ष‚ण‚काले । सामान्य‚स्य‚{\tiny $_{lb}$}‚ साध्य‚तामुक्त्वा साध्य‚ध‚र्म‚श‚ब्देन स‚ह सामान्य‚श‚ब्द‚स्य विग्र‚हं द‚र्श‚य‚ति--\textbf{साध्येति । साध्य‚ध‚र्म्ये‚{\tiny $_{lb}$}‚\edtext{}{\lemma{र्म्ये}\Bfootnote{र्मे}}}त्यादिनोप‚संह‚र‚ति । स‚मान एवेत्य‚व‚धार‚णीय‚म् न तु सामान्येनैवेति, अन्येनापि‚{\tiny $_{lb}$}‚ व‚स्तुत्वादिना सादृश्यात् दृष्टान्त‚ध‚र्मिणः स‚प‚क्ष‚त्वाभाव‚प्र‚स‚ङ्गादिति ।
	\pend% ending standard par
      ‚{\tiny $_{lb}$}‚

	  \pstart \leavevmode% starting standard par
	य‚दि स‚प‚क्षाद‚न्योऽस‚प‚क्षोऽन्य‚ध‚र्म‚योगाच्चान्य‚स्त‚दा स‚प‚क्ष एवान्य‚ध‚र्म‚योगाद‚न्य इति‚{\tiny $_{lb}$}‚ त‚त्रास‚प‚क्षे व‚र्त्त‚मानो हेतुः स‚र्व एवानैकान्तिक‚त्वाद‚हेतुः प्र‚स‚ज्येत । अथ विरुद्धे न‚ञो‚{\tiny $_{lb}$}‚ विधानात् स‚प‚क्ष‚विरुद्धोऽस‚प‚क्षः स‚हान‚व‚स्थान‚ल‚क्ष‚णेन विरोधेन विरुद्ध‚स्त‚दाऽग्निल‚क्ष‚णो‚{\tiny $_{lb}$}‚ हेतुरौष्ण्यं न ग‚म‚येत् । अथाभावे न‚ञिष्य‚ते, त‚दाऽभावे क‚स्य‚चित्स‚त्त्वास‚म्भ‚वात् साधार‚णा‚{\tiny $_{lb}$}‚नैकान्तिको न क‚श्चित्स्यादिति म‚न‚सि निवेश्य पृच्छ‚ति--\textbf{कोऽस‚प‚क्ष} इति । अनीदृशाश‚य‚स्या‚{\tiny $_{lb}$}‚ज्ञ‚स्यैव वा प्र‚श्नः ।
	\pend% ending standard par
      ‚{\tiny $_{lb}$}‚

	  \pstart \leavevmode% starting standard par
	\textbf{न स‚प‚क्ष} इत्याद्युत्त‚र‚म् । \textbf{स‚प‚क्षो यो न} भ‚व‚तीति साध्य‚ध‚र्म‚वान् यो न भ‚व‚तीत्य‚र्थः ।‚{\tiny $_{lb}$}‚ एव‚ञ्चाच‚क्षाणः प्र‚स‚ज्य‚प्र‚तिषेध‚वृत्तिं न‚ञं द‚र्श‚य‚ति । य‚द्येव‚म्, स‚मासः क‚थ‚मिति चेत् । ग‚म‚क‚{\tiny $_{lb}$}‚त्वाद‚सूर्य्य‚म्प‚श्यानीत्यादिव‚त् । अपुन‚र्ज्ञेयानि सामानि, अल‚व‚ण‚भोजी, असूर्य्य‚म्प‚श्यानि‚{\tiny $_{lb}$}‚ मुखानि सुड‚न‚पुंस‚क‚स्य \href{http://sarit.indology.info/?cref=Pā.1.1.43}{पाणिनि १. १. ४३} इति । प्र‚योग‚संख्यानिय‚म‚स्तु \textbf{भाष्य‚कारीयो‚{\tiny $_{lb}$}‚ य‚थाऽनुप‚प‚न्न‚स्त‚था} चाचार्येणैव \textbf{विनिश्च‚ये} दुःखं ब‚तायं त‚प‚स्वी इत्याद्युप‚हास‚पूर्व‚कं--य‚था‚{\tiny $_{lb}$}‚ निकेतेन प्र‚तिप‚त्तेः इत्यादिना प्र‚तिपादित इति नेहोच्य‚ते ।
	\pend% ending standard par
      ‚{\tiny $_{lb}$}‚‚{\tiny $_{lb}$}‚\textsuperscript{\textenglish{99/dm}}‚{\tiny $_{lb}$}‚
	  \bigskip
	  \begingroup
	
	  \bigskip
	  \begingroup
	

	  \pstart \leavevmode% starting standard par
	त‚तोऽन्य‚स्त‚द्विरुद्ध‚स्त‚द‚भाव‚श्चेति ॥ ९ ॥
	\pend% ending standard par
      
	  \endgroup
	‚{\tiny $_{lb}$}‚ 

	  \pstart \leavevmode% starting standard par
	त‚तः स‚प‚क्षाद् \edtext{}{\lemma{क्षाद्}\Bfootnote{इयं पृष्वी, ग‚न्ध‚व‚त्त्वात् । य‚त्तु पृथ्वी न भ‚व‚ति त‚द् ग‚न्ध‚व‚द‚पि न, य‚था अबादिः ।‚{\tiny $_{lb}$}‚ अत्र अबादिर्दृष्टान्तीकृतः साध्याद‚न्य इति--\cite{dp-msD-n}}}अन्यः । तेन च\edtext{}{\lemma{च}\Bfootnote{तेन विरु० \cite{dp-msB}}} विरुद्धः\edtext{}{\lemma{विरुद्धः}\Bfootnote{व‚ह्निनिवृत्तौ धूम‚निवृत्तेरास्प‚दं ज‚लाश‚य इति साध्येन स‚ह विरुद्धः--\cite{dp-msD-n}}} । त‚स्य च स‚प‚क्ष‚स्याभावः\edtext{}{\lemma{स्याभावः}\Bfootnote{य‚था क्ष‚णिक‚त्व‚निवृत्तौ स‚त्त्व‚निवृत्तेरास्प‚दं ख‚र‚विषाण‚मिति साध्याभाव‚मात्र‚म्--\cite{dp-msD-n}}} ।‚{\tiny $_{lb}$}‚ स‚प‚क्षाद‚न्य‚त्वं त‚द्विरुद्ध‚त्वं च न ताव‚त् प्र‚त्येतुं श‚क्यं याव‚त् स‚प‚क्ष‚स्व‚भावाभावो न विज्ञातः ।‚{\tiny $_{lb}$}‚ त‚स्माद‚न्य‚त्व‚विरुद्ध‚त्व‚प्र‚तीतिसाम‚र्थ्यात् स‚प‚क्षाभाव‚रूपौ प्र‚तीताव‚न्य‚विरुद्धौ ।
	\pend% ending standard par
      
	  \endgroup
	‚{\tiny $_{lb}$}‚

	  \pstart \leavevmode% starting standard par
	एवं स‚त्य‚न्य‚विरुद्ध‚योर‚स‚प‚क्ष‚त्वं न स्यादित्य‚स‚प‚क्ष‚श‚ब्देन त‚द‚भाव‚त‚द‚न्य‚त‚द्विरुद्ध‚नां त्र‚याणा‚{\tiny $_{lb}$}‚म‚पि स‚ङ्ग्र‚हं द‚र्श‚यितुं \textbf{क‚श्चे}त्यादिना प्र‚श्न‚पूर्व‚मुप‚क्र\add{म}ते । \textbf{च‚कारः} पुनःश‚ब्द‚स्यार्थे ।
	\pend% ending standard par
      ‚{\tiny $_{lb}$}‚

	  \pstart \leavevmode% starting standard par
	\textbf{अन्य} इति विव‚क्षित‚ध‚र्मानाधारः, अन्य‚विष‚येऽपि न‚ञि विभागेन नियोग‚वृत्तेः । न हि‚{\tiny $_{lb}$}‚ स एव ब्राह्म‚ण‚स्त‚ज्जातियोगाद्, अब्राह्म‚ण‚श्च ध‚र्मान्त‚र‚स‚मावेशाल्लोके प्र‚तीय‚त इति । \textbf{विरुद्ध}‚{\tiny $_{lb}$}‚ इति स‚ह‚स्थितिल‚क्ष‚णेनाऽन्योन्यात्म‚प‚रिहार‚स्थितिल‚क्ष‚णेन च विरोधेन विरुद्धः । \textbf{चो}ऽन्यापेक्ष‚या‚{\tiny $_{lb}$}‚ विरुद्ध‚म‚स‚प‚क्ष‚त्वेन स‚मुच्चिनोति । \textbf{त‚स्य चाभाव}स्तुच्छ‚रूपः प्र‚स‚ज्यात्मा व्य‚व‚ह‚र्त्त‚व्यैक‚स्व‚भावः ।
	\pend% ending standard par
      ‚{\tiny $_{lb}$}‚

	  \pstart \leavevmode% starting standard par
	अय‚म‚त्र प्र‚क‚र‚णार्थः--स‚प‚क्षाभावोऽस‚प‚क्षः । साध्य‚ध‚र्म‚वान् यो न भ‚व‚तीत्य‚र्थः ।‚{\tiny $_{lb}$}‚ साध्य‚ध‚र्माभावार्थ‚त्वाद‚स‚प‚क्ष‚श‚ब्द‚स्य । न चैवं निषेध‚मात्र‚म‚स‚प‚क्षः । किन्त‚र्हि ? स‚र्वः प्र‚तियोगी‚{\tiny $_{lb}$}‚ निषेधः प‚र्युद‚स्त‚श्च, अत‚त्त्व‚ल‚क्ष‚ण‚त्वाद‚स‚प‚क्ष‚स्य । त‚द् विव‚क्षिते प्र‚तियोगिनि तुल्य‚म्, व्य‚तिरेक‚ग‚तेः‚{\tiny $_{lb}$}‚ स‚र्व‚त्र तुल्य‚त्वात्, साक्षाद‚र्थाप‚त्त्या वेति । चः पूर्वापेक्षः स‚मुच्च‚ये । अन्य‚विरुद्ध‚योर‚पि व‚स्तु‚{\tiny $_{lb}$}‚स‚तोः क‚ल्पित‚योश्चास‚प‚क्ष‚त्वात् । त‚त्र व‚र्त्त‚मान‚स्य साधार‚णानैकान्तिक‚त्वान्न त‚द‚भाव‚दोषः, अन्य‚{\tiny $_{lb}$}‚विरुद्ध‚योश्च स्व‚रूप‚क‚थ‚नेन त‚द‚स‚प‚क्ष‚त्व‚प‚क्षोक्तो दोषो निर‚स्त इति स‚र्व‚म‚व‚दात‚म् ।
	\pend% ending standard par
      ‚{\tiny $_{lb}$}‚

	  \pstart \leavevmode% starting standard par
	न‚नु च \textbf{स‚प‚क्षो यो न भ‚व‚तीति} व‚च‚नेन य‚स्य स‚प‚क्षाभाव‚स्व‚भाव‚त्वं त‚स्यैवास‚प‚क्ष‚त्वं‚{\tiny $_{lb}$}‚ \leavevmode\ledsidenote{\textenglish{41b/ms}}प्र‚तिपाद्य‚ते । न चान्य‚विरुद्ध‚योस्त‚द‚भाव‚स्व‚भाव‚ता स‚म्भ‚व\edtext{}{\lemma{व}\Bfootnote{वि}}नी, विधिरूप‚त्वात् ।‚{\tiny $_{lb}$}‚ त‚त्क‚थं त‚योस्त‚थात्व‚मित्याश‚ङ्क्याह--\textbf{स‚प‚क्षादि}ति । \textbf{अन्य‚त्वं} त‚तो भिद्य‚मान‚त्वं पृथ‚क्त्व‚मिति‚{\tiny $_{lb}$}‚ याव‚त् । तेन च विरुद्ध‚त्वं ताव‚न्न श‚क्यं ज्ञातुम्, याव‚त्स‚प‚क्षाभाव‚स्व‚भावो न विज्ञातो भ‚व‚ति ।‚{\tiny $_{lb}$}‚ सोऽन्यो विरुद्ध‚श्चेत्य‚र्थात् ।
	\pend% ending standard par
      ‚{\tiny $_{lb}$}‚

	  \pstart \leavevmode% starting standard par
	अय‚माश‚यः--य‚दि त‚स्यान्य‚त्वाभिम‚त‚स्य य‚तोऽन्य‚त्वं व्य‚व‚ह‚र्त्त‚व्यं त‚द्रूप‚ता चेत्त‚स्यासिद्धा‚{\tiny $_{lb}$}‚ स‚न्दिग्धा वा भ‚वेत् त‚दाऽन्य‚त्व‚मेव न स्यात्, त‚दात्म‚व‚त्, धूम‚स्येव वा बाष्पादिभावेन स‚न्दिह्य‚{\tiny $_{lb}$}‚मान‚स्य न बाष्पादेर‚न्य‚त्व‚निश्च‚यः । येन च विरुद्धं य‚त् त‚द्रूपं \textbf{चे}त् सिद्धं स‚न्दिग्धं वा‚{\tiny $_{lb}$}‚ ताद्रूप्येण, त‚दा तेन स‚ह त‚स्याव‚स्थितिः, त‚दात्म‚प‚रिहारेण वाऽव‚स्थानं क‚थ‚म्, क‚थं च निश्चीयेत‚{\tiny $_{lb}$}‚ त‚दात्म‚व‚त् पूर्वोक्त‚धूम‚व‚त् ?
	\pend% ending standard par
      ‚{\tiny $_{lb}$}‚

	  \pstart \leavevmode% starting standard par
	य‚त एवं \textbf{त‚स्मात्} । त‚योः साध्य‚ध‚र्म‚व‚त्त्वाभावादित्य‚र्थः । य‚त एवं \textbf{त‚तो} हेतोः साक्षाद‚{\tiny $_{lb}$}‚व्य‚व‚धानेन । अनेन य‚द्येक एवास‚प‚क्षो व‚क्त‚व्यः त‚दा त‚द‚भाव एवेति सूचित‚म् । तुर‚भावा‚{\tiny $_{lb}$}‚‚{\tiny $_{lb}$}‚ ‚{\tiny $_{lb}$}‚ \leavevmode\ledsidenote{\textenglish{100/dm}}‚{\tiny $_{lb}$}‚ 
	  
	त‚तोऽभावः साक्षात् स‚प‚क्षाभाव‚रूपः प्र‚तीय‚ते । अन्य‚विरुद्धौ तु साम‚र्थ्याद‚भाव‚रूपौ‚{\tiny $_{lb}$}‚ प्र‚तीयेते । त‚त‚स्त्र‚याणाम‚प्य‚स‚प‚क्ष‚त्व‚म् ॥ ‚{\tiny $_{lb}$}‚ 
	  
	त्रिरूपाणि च त्रीण्येव लिङ्गानि ॥ १० ॥‚{\tiny $_{lb}$}‚ 
	  
	उक्तेन त्रैरूप्येण त्रिरूपाणि च त्रीण्येव लिङ्गानि इति । च‚कारो व‚क्त‚व्यान्त‚र‚{\tiny $_{lb}$}‚स‚मुच्च‚यार्थः । त्रैरूप्य‚मादौ पृष्टं त्रिरूपाणि च लिङ्गानि प‚रेण । त‚त्र त्रैरूप्य‚मुक्त‚म् ।‚{\tiny $_{lb}$}‚ त्रिरूपाणि चोच्य‚न्ते--त्रीण्येव त्रिरूपाणि लिङ्गानि । त्र‚य‚स्त्रिरूप‚लिङ्ग‚प्र‚कारा इत्य‚र्थः ॥ ‚{\tiny $_{lb}$}‚ 
	  
	कानि पुन‚स्तानीत्याह-- ‚{\tiny $_{lb}$}‚ 
	  
	अनुप‚ल‚ब्धिः स्व‚भावः\edtext{}{\lemma{भावः}\Bfootnote{०भाव‚कार्यं \cite{dp-edE} ०भाव‚कार्ये \cite{dp-msD} \cite{dp-msB} \cite{dp-edP} \cite{dp-edH} \cite{dp-edN}}} कार्यं चेति ॥ ११ ॥‚{\tiny $_{lb}$}‚ 
	  
	प्र‚तिषेध्य‚स्य साध्य‚स्यानुप‚ल‚ब्धिस्त्रिरूपा । विधेय‚स्य साध्य‚स्य स्व‚भाव‚श्च\edtext{}{\lemma{श्च}\Bfootnote{०भाव‚स्त्रिरूपः \cite{dp-msA} \cite{dp-msB} \cite{dp-msD} \cite{dp-edP} \cite{dp-edH} \cite{dp-edE} \cite{dp-edN}}} त्रिरूपः,‚{\tiny $_{lb}$}‚ कार्य च ॥‚{\tiny $_{lb}$}‚ द‚न्य‚विरुद्ध‚योर्वैध‚र्म्य‚माह । य‚तोऽन्यो विरुद्ध‚श्चोक्त‚या नीत्या स‚प‚क्षाभाव‚रूपौ त‚त‚स्त‚स्मात् ।‚{\tiny $_{lb}$}‚ आस्तामेक‚स्य द्व‚योर्वा त्र‚याणाम‚पीत्य‚पिश‚ब्दः ।
	\pend% ending standard par
      ‚{\tiny $_{lb}$}‚

	  \pstart \leavevmode% starting standard par
	\textbf{त्रैरूप्येण} त्रिरूपेणेत्य‚र्थः । त्रीणि रूपाणि ल‚क्ष‚णानि येषामिति विग्र‚हः । \textbf{त्रीण्येव}‚{\tiny $_{lb}$}‚ त्रिसंख्यान्येव । \textbf{त्रिरूपाणी}त्य‚नेनाबाधित‚विष‚य‚त्वादिरूपान्त‚र‚योगेन च‚तुर्ल‚क्ष‚ण‚त्वं ष‚ड्ल‚क्ष‚ण‚त्वं वा‚{\tiny $_{lb}$}‚ प‚राभिम‚तं हेतोः प्र‚तिषेध‚ति, \textbf{त्रीण्येवे}त्य‚नेन संयोग्यादिभेदेन भूयिष्ठ‚संख्य‚त्व‚म् ।
	\pend% ending standard par
      ‚{\tiny $_{lb}$}‚

	  \pstart \leavevmode% starting standard par
	\textbf{पृष्ट} इति पाठे त्वाचार्य इति शेषः । य‚त् त्रैरूप्यं पृष्टो य‚द् वा य‚त्त्रैरूप्यं पृष्टं‚{\tiny $_{lb}$}‚ त‚त्त्रैरूप्य‚मुक्त‚मुक्तेन ग्र‚न्थेन । अथ‚वा य‚द् य‚स्मात्त्रैरूप्यं पृष्ट आचार्यः, पृष्टं वा त‚त् त‚स्मात्‚{\tiny $_{lb}$}‚ त्रैरूप्य‚मुक्त‚मिति । \textbf{च}स्त्रैरूप्येण स‚मं त्रिरूपाणामुक्त‚क‚र्म‚तां स‚मुच्चिनोति ।
	\pend% ending standard par
      ‚{\tiny $_{lb}$}‚

	  \pstart \leavevmode% starting standard par
	न‚नु न ताव‚त्प‚र‚स्यैत‚द् द्वित‚य‚प्र‚श्न‚वाक्यं श्रुत‚म् । त‚त्क‚थं प‚र‚स्य द्वेधा प्र‚श्नः संकीर्त्त्य‚त‚{\tiny $_{lb}$}‚ इति चेत् । उच्य‚ते । त्रिरूपाल्लिङ्गादिति श्रुत‚व‚ता पूर्व‚प‚क्ष‚वादिनाऽव‚श्यं किं त‚त् त्रैरूप्यं‚{\tiny $_{lb}$}‚ किय‚च्च त्रिरूपं लिङ्ग‚मित्याकाङ्क्षित‚व्य‚म्, तेन पृष्ट‚मित्युच्य‚ते । एत‚देव क‚थ‚म‚व‚सीय‚त इति चेत् ।‚{\tiny $_{lb}$}‚ त्रैरूप्यं लिङ्ग‚स्यैव‚मात्म‚क‚मित्य‚भिधानादाचार्य‚स्य पूर्व‚प‚क्ष‚वादिन एवंरूप प्र‚श्नोऽव‚सीय‚ते ।‚{\tiny $_{lb}$}‚ त्रिरूपाणि च त्रीण्येवेत्य‚भिधानाच्च संख्याप्र‚श्नः । त‚तः साधूक्तं \textbf{त्रैरूप्य‚मादावि}त्यादि ।‚{\tiny $_{lb}$}‚ \textbf{लिङ्ग‚प्र‚कारा} लिङ्ग‚स्व‚रूपाणि ॥
	\pend% ending standard par
      ‚{\tiny $_{lb}$}‚

	  \pstart \leavevmode% starting standard par
	संयोग्यादिभेदेन त्रित्वास‚म्भ‚वात् पृच्छ‚ति--\textbf{कानीति} । सामान्य‚विशेषाकाराभ्यां‚{\tiny $_{lb}$}‚ प्र‚श्नः । \textbf{कार्यं} च विधेय‚स्येति प्र‚कृत‚म् । केव‚लं \textbf{विधेय‚स्ये}ति पूर्वं साम‚र्थ्याद‚न‚र्थान्त‚र‚स्य‚{\tiny $_{lb}$}‚ विधेय‚स्य, अधुना त्व‚र्थान्त‚र‚स्य विधेय‚स्येत्य‚व‚सेय‚म् । विधेय‚स्यार्थान्त‚र‚स्य कार्यं त्रिरूप‚मिति‚{\tiny $_{lb}$}‚ योज‚नीय‚म् । च‚कारौ पूर्वापेक्ष‚या स‚मुच्च‚यार्थौ ॥
	\pend% ending standard par
      ‚{\tiny $_{lb}$}‚‚{\tiny $_{lb}$}‚\textsuperscript{\textenglish{101/dm}}‚{\tiny $_{lb}$}‚
	  \bigskip
	  \begingroup
	

	  \pstart \leavevmode% starting standard par
	अनुप‚ल‚ब्धिमुदाह‚र्त्तुमाह--
	\pend% ending standard par
       ‚{\tiny $_{lb}$}‚ 
	  \bigskip
	  \begingroup
	

	  \pstart \leavevmode% starting standard par
	त‚त्रानुप‚ल‚ब्धिर्य‚था--न प्र‚देश‚विशेषे क्व‚चिद् घ‚टः, उप‚ल‚ब्धिल‚क्ष‚ण‚प्राप्त‚स्या‚{\tiny $_{lb}$}‚नुप‚ल‚ब्धेरिति ॥ १२ ॥
	\pend% ending standard par
      
	  \endgroup
	‚{\tiny $_{lb}$}‚ 

	  \pstart \leavevmode% starting standard par
	य‚थेत्यादि । य‚थेत्युप‚प्र‚द‚र्श‚नार्थ‚म् । य‚थेय‚म‚नुप‚ल‚ब्धिस्त‚थान्यापि । न त्विय‚मेवेत्य‚र्थः ।‚{\tiny $_{lb}$}‚ प्र‚देश एक‚देशः । विशिष्य‚त इति विशेषः प्र‚तिप‚त्तृप्र‚त्य‚क्षः । तादृश‚श्च न स‚र्वः प्र‚देशः ।‚{\tiny $_{lb}$}‚ त‚दाह--क्व‚चिद् इति । प्र‚तिप‚त्तृप्र‚त्य‚क्षे क्व‚चिदेव प्र‚देश इति ध‚र्मो । न घ‚ट इति साध्य‚म् ।‚{\tiny $_{lb}$}‚ \edtext{\textsuperscript{*}}{\lemma{*}\Bfootnote{उप‚ल‚ब्धिज्ञान‚म् । त‚स्य ल‚क्ष‚णं ज‚निका--\cite{dp-msA}}}उप‚ल‚ब्धिर्ज्ञान‚म् । त‚स्या \edtext{}{\lemma{स्या}\Bfootnote{कार‚ण‚म्--\cite{dp-msD-n}}}ल‚क्ष‚णं ज‚निका साम‚ग्री । त‚या हि\edtext{}{\lemma{हि}\Bfootnote{ह्य‚नुप‚ल० \cite{dp-edH}}} उप‚ल‚ब्धिर्ल‚क्ष्य‚ते । त‚त्प्राप्तोऽर्थो‚{\tiny $_{lb}$}‚\edtext{}{\lemma{त्प्राप्तोऽर्थो}\Bfootnote{आल‚म्ब‚न‚त्वेन--\cite{dp-msD-n}}}ज‚न‚क‚त्वेन साम‚ग्र्य‚न्त‚र्भावात् । उप‚ल‚ब्धिल‚क्ष‚ण‚प्राप्तो दृश्य इत्य‚र्थः । त‚स्यानुप‚ल‚ब्धेः—‚{\tiny $_{lb}$}‚इत्य‚यं हेतुः ।
	\pend% ending standard par
       ‚{\tiny $_{lb}$}‚ 

	  \pstart \leavevmode% starting standard par
	अथ यो य‚त्र नास्ति स क‚थं त‚त्र दृश्यः ? दृश्य‚त्व‚स‚मारोपाद‚स‚न्न‚पि दृश्य उच्य‚ते ।‚{\tiny $_{lb}$}‚ \edtext{\textsuperscript{*}}{\lemma{*}\Bfootnote{य‚श्चार्थ--\cite{dp-msD-n}}}य‚श्चैवं स‚म्भाव्य‚ते--य‚द्य‚साव‚त्र भ‚वेद् दृश्य एव भ‚वेदिति । स त‚त्र अविद्य‚मानोऽपि दृश्यः‚{\tiny $_{lb}$}‚ स‚मारोप्यः । क‚श्चैवं स‚म्भाव्यः ? य‚स्य स‚म‚ग्राणि स्वाल‚म्ब‚न‚द‚र्श‚न‚कार‚णानि भ‚व‚न्ति ।‚{\tiny $_{lb}$}‚ क‚दा च तानि स‚म‚ग्राणि ग‚म्य‚न्ते ? य‚दैक‚ज्ञान‚संस‚र्गिव‚स्त्व‚न्त‚रोप‚ल‚म्भः । एकेन्द्रिय‚ज्ञान‚{\tiny $_{lb}$}‚ग्राह्यं लोच‚नादिप्र‚णिधानाभिमुखं \edtext{}{\lemma{णिधानाभिमुखं}\Bfootnote{प‚ट-भूत‚ल--\cite{dp-msD-n}}}व‚स्तुद्व‚य‚म‚न्योन्यापेक्ष‚मेक‚ज्ञान‚संस‚र्गि \edtext{}{\lemma{र्गि}\Bfootnote{ग‚म्य‚ते \cite{dp-msB}}}क‚थ्य‚ते । त‚योर्हि
	\pend% ending standard par
      
	  \endgroup
	‚{\tiny $_{lb}$}‚

	  \pstart \leavevmode% starting standard par
	\textbf{न घ‚ट} इति घ‚टाभाव‚व्य‚व‚हार‚योग्य‚तेति द्र‚ष्ट‚व्य‚म् । घ‚टाभाव‚स्य प्र‚त्य‚क्ष‚सिद्ध‚त्वात् ।‚{\tiny $_{lb}$}‚ एत‚च्च प‚र‚स्ताद‚भिधास्य‚ते । \textbf{त‚या ल‚क्ष्य‚त} इति ब्रुव‚ता ल‚क्ष्य‚तेऽनेनेति ल‚क्ष‚ण‚मिति व्य‚क्तीकृत‚स् ।‚{\tiny $_{lb}$}‚ \textbf{त‚दि}त्युप‚ल‚ब्धिल‚क्ष‚ण‚म् । क‚थं त‚त्प्रान्त इत्याह--\textbf{ज‚न‚क‚त्वे}न । किं तेनैकेन सा ज‚न्य‚ते ये\leavevmode\ledsidenote{\textenglish{42a/ms}}‚{\tiny $_{lb}$}‚नैव‚मुच्य‚त इत्याह--\textbf{साम‚ग्र्य‚न्त‚र्भावादि}ति ।
	\pend% ending standard par
      ‚{\tiny $_{lb}$}‚

	  \pstart \leavevmode% starting standard par
	स‚मुदायार्थं स्फुट‚य‚ति । \textbf{दृश्यो} द‚र्श‚न‚योग्यः । इतिरेव‚म‚र्थे । एव‚म‚भिधेयो य‚स्यो‚{\tiny $_{lb}$}‚प‚ल‚ब्धिल‚क्ष‚ण‚प्राप्त‚श‚ब्द‚स्य ।
	\pend% ending standard par
      ‚{\tiny $_{lb}$}‚

	  \pstart \leavevmode% starting standard par
	य‚द्य‚विद्य‚मानः स‚मारोपाद् दृश्य उच्य‚ते त‚दा भिन्नेन्द्रिय‚ग्राह्य‚म‚पि त‚त्रानेक‚म‚स्ति ।‚{\tiny $_{lb}$}‚ त‚स्यापि त‚र्हि त‚थात्व‚मायात‚मित्याश‚ङ्क्याह--य‚श्चैव‚मिति । \textbf{चो}ऽव‚धार‚णे । स तादृशो‚{\tiny $_{lb}$}‚ दृश्य‚त‚या स‚मारोप्यो न स‚र्व इत्य‚र्थात् । \textbf{क‚श्चैवं स‚म्भाव्य} इति पृच्छ‚ति । पुनः‚{\tiny $_{lb}$}‚श‚ब्दार्थ‚श्च‚कारः ।
	\pend% ending standard par
      ‚{\tiny $_{lb}$}‚

	  \pstart \leavevmode% starting standard par
	\textbf{य‚स्येति} सिद्धान्ती । य‚स्येति प्र‚तिषेध्य‚स्य । स्व‚मात्मान‚माल‚म्ब‚त इत्याल\textbf{म्ब‚नं । य‚स्य‚{\tiny $_{lb}$}‚ द‚र्श‚न}स्य । त‚स्य \textbf{कार‚णानि} । त‚च्चेन्नास्ति क‚थं त‚द्द‚र्श‚न‚कार‚ण‚साम‚ग्र्याव‚ग‚तिरित्याश‚यः पृच्छ‚ति‚{\tiny $_{lb}$}‚ \textbf{क‚दे}ति । \textbf{च} पूर्व‚व‚त् । एक‚स्मिन् ज्ञाने संस‚र्गः प्र‚तिभासः स य‚स्यास्ति । त‚च्च \textbf{त‚द् व‚स्त्व‚न्त‚रं}‚{\tiny $_{lb}$}‚ चेति त‚था त‚स्यो\textbf{प‚ल‚म्भो} य‚देत्युक्त‚म् ।
	\pend% ending standard par
      ‚{\tiny $_{lb}$}‚‚{\tiny $_{lb}$}‚\textsuperscript{\textenglish{102/dm}}‚{\tiny $_{lb}$}‚
	  \bigskip
	  \begingroup
	

	  \pstart \leavevmode% starting standard par
	स‚तोर्नैक‚निय‚ता भ‚व‚ति प्र‚तिप‚त्तिः । योग्य‚ताया द्व‚योर‚प्य‚विशिष्ट‚त्वात् । त‚स्मादेक‚{\tiny $_{lb}$}‚ज्ञान‚संस‚र्गिणि दृश्य‚माने\edtext{}{\lemma{माने}\Bfootnote{भूत‚ले--\cite{dp-msD-n}}} स‚त्येक‚स्मिन्नित‚र‚त्\edtext{}{\lemma{त्}\Bfootnote{घ‚टः--\cite{dp-msD-n}}} स‚म‚ग्र‚द‚र्श‚न‚साम‚ग्रीकं य‚दि भ‚वेद् दृश्य‚मेव‚{\tiny $_{lb}$}‚ भ‚वेदिति स‚म्भावितं \edtext{}{\lemma{म्भावितं}\Bfootnote{दृश्य‚मा० \cite{dp-msA} \cite{dp-edP} \cite{dp-edH} \cite{dp-edE} \cite{dp-edN}}}दृश्य‚त्व‚मारोप्य‚ते । त‚स्यानुप‚ल‚म्भो दृश्यानुप‚ल‚म्भः । त‚स्मात् स एव‚{\tiny $_{lb}$}‚ \edtext{\textsuperscript{*}}{\lemma{*}\Bfootnote{घ‚टादिविविक्त० \cite{dp-msB}}}घ‚ट‚विविक्त‚प्र‚देश‚स्त‚दाल‚म्ब‚नं च ज्ञानं दृश्यानुप‚ल‚म्भ‚निश्च‚य‚हेतुत्वात् दृश्यानुप‚ल‚म्भ उच्य‚ते ।
	\pend% ending standard par
       ‚{\tiny $_{lb}$}‚ 

	  \pstart \leavevmode% starting standard par
	याव‚द्धि एक‚ज्ञान‚संस‚र्गि व‚स्तु\edtext{}{\lemma{स्तु}\Bfootnote{व‚स्तु त‚ज्ज्ञानं चा[[वा]] न निश्चित‚म् न ताव‚द् \cite{dp-msB} व‚स्तु त‚ज्ज्ञानं वा न निश्चित‚म्‚{\tiny $_{lb}$}‚ न ताव‚द् \cite{dp-msD}}} न निश्चित‚म्, त‚ज्ज्ञानं च, न ताव‚द् दृश्यानुप‚ल‚म्भ‚{\tiny $_{lb}$}‚निश्च‚यः । त‚तो व‚स्तु अनुप‚ल‚म्भ उच्य‚ते, त‚ज्ज्ञानं च । द‚र्श‚न‚निवृत्तिमात्रं तु स्व‚य‚म‚निश्चित‚{\tiny $_{lb}$}‚त्वाद‚ग‚म‚क‚म्\edtext{}{\lemma{म्}\Bfootnote{०ग‚म‚क‚मेव \cite{dp-msC}}} । \edtext{\textsuperscript{*}}{\lemma{*}\Bfootnote{०क‚म् । तादृश‚घ‚ट‚र‚हितः । \cite{dp-msB}}}त‚तो दृश्य‚घ‚ट‚र‚हितः प्र‚देशः, त‚ज्ज्ञानं च व‚च‚न‚साम‚र्थ्यादेव दृश्यानुप‚ल‚म्भ‚{\tiny $_{lb}$}‚रूप‚मुक्तं द्र‚ष्ट‚व्य‚म् ॥
	\pend% ending standard par
      
	  \endgroup
	‚{\tiny $_{lb}$}‚

	  \pstart \leavevmode% starting standard par
	त‚त्र य‚दि निराकारं विज्ञान‚मिति स्थितिस्त‚दैक‚श‚ब्दो द्व‚यादिसंख्यानिरासार्थः । य‚दा‚{\tiny $_{lb}$}‚ तु साकार‚मिति स्थितिस्त‚दा प‚क्ष‚द्व‚य‚म्--एक‚मेव वाऽनेकाकार‚म‚श‚क्य‚विवेच‚नं चित्रं ज्ञान‚म्,‚{\tiny $_{lb}$}‚ प्र‚तिव‚स्त्व‚नेक‚मेव वा त‚त्त‚दाकारानुकारि । पूर्व‚स्मिन् प‚क्षे पूर्व‚व‚देक‚श‚ब्दः । शेष‚प‚क्षे त्वेक‚{\tiny $_{lb}$}‚च‚क्षुराय‚त‚न‚प्र‚भ‚व‚त्वादेक‚रूपाल‚म्ब‚न‚त्वाच्चैक‚श‚ब्दो गौणः ।
	\pend% ending standard par
      ‚{\tiny $_{lb}$}‚

	  \pstart \leavevmode% starting standard par
	न‚नु य‚दि त‚त् प्र‚तिषिध्य‚मान‚म‚नेन भूत‚लादिना स‚हैक‚स्मिन् ज्ञाने प्र‚तिभासेत त‚द‚पेक्ष‚{\tiny $_{lb}$}‚मिद‚मेक‚ज्ञान‚संस‚र्गि क‚थ्येत । याव‚तेद‚मेव नास्तीत्याश‚ङ्क्य त‚थात्व‚मेव त‚यो\textbf{रेकेन्द्रियेत्या}दिना‚{\tiny $_{lb}$}‚ द‚र्श‚य‚ति । \textbf{लोच‚नादीनां प्र‚णिधानं} स्व‚ज्ञानोप‚ज‚न‚ने योग्यीभ‚व‚नं त‚त्राभिमुख‚म‚नुगुण‚म् ।‚{\tiny $_{lb}$}‚ \textbf{अन्योन्यापे}क्ष‚मिति घ‚टाद्य‚पेक्षं भूत‚लादि, त‚द‚पेक्षं च घ‚टादीत्य‚र्थः ।
	\pend% ending standard par
      ‚{\tiny $_{lb}$}‚

	  \pstart \leavevmode% starting standard par
	य‚दि नाम द्व‚यं त‚त्राव‚स्थितं त‚थाप्येक‚मेव त‚त्राव‚भासिष्य‚त इत्याह--\textbf{त‚योरिति ।‚{\tiny $_{lb}$}‚ हि}र्य‚स्मात् । उप‚प‚त्तिमाह--\textbf{योग्य‚ताया} इति । य‚त एव त‚योर्घ‚टादिभूत‚लाद्योरेक‚ज्ञान‚संस‚र्गः‚{\tiny $_{lb}$}‚ स‚म्भ‚वी \textbf{त‚स्मात्} । एक‚स्मिन् भूत‚लादिके दृश्य‚माने स‚तीत‚र‚द् घ‚टादिकं प‚रिपूर्ण‚द‚र्श‚न‚साम‚ग्रीकं‚{\tiny $_{lb}$}‚ ज्ञात‚व्य‚म् । त‚थास‚द‚विद्य‚मान‚म‚पि स‚मारोपाद् दृश्य‚मुच्य‚त इति द‚र्श‚य‚ति ।
	\pend% ending standard par
      ‚{\tiny $_{lb}$}‚

	  \pstart \leavevmode% starting standard par
	\textbf{भ‚वेदिति}ना स‚म्भाव‚नाया आकारः क‚थितः । \textbf{त‚स्यैव} स‚म्भाव‚नार्ह‚स्यानुप‚ल‚म्भ‚स्त‚द्‚{\tiny $_{lb}$}‚विविक्तान्योप‚ल‚म्भ‚रूपः । य‚स्मात्त‚स्यैक‚ज्ञान‚संस‚र्गिणो भूत‚लादेर्द‚र्श‚नात् त‚द‚नुप‚ल‚म्भो निश्चीय‚ते‚{\tiny $_{lb}$}‚ \textbf{त‚स्मात्} ।
	\pend% ending standard par
      ‚{\tiny $_{lb}$}‚

	  \pstart \leavevmode% starting standard par
	न‚नु भूत‚लादिज्ञान‚निश्च‚य एव त‚द‚नुप‚ल‚म्भ‚निश्च‚य‚हेतुत्वाद‚नुप‚ल‚म्भोऽस्तु भूत‚लादिकं तु‚{\tiny $_{lb}$}‚ क‚थ‚मित्याह--\textbf{याव‚दि}ति । \textbf{हि}र्य‚स्मात् । अनेन ज्ञान‚विशिष्टं भूत‚लादि, भूत‚लाद्य‚व‚च्छिन्न‚ञ्च‚{\tiny $_{lb}$}‚ ज्ञानं दृश्य‚स्यानुप‚ल‚म्भ‚मुप‚ल‚म्भाभावं व्य‚व‚ह‚र्त्त‚व्यैक‚स्व‚भावं निश्चाय‚य‚तीति द‚र्शित‚म् । व‚स्त्विति‚{\tiny $_{lb}$}‚ भूत‚लादि । ज्ञान‚मिति त‚द्ग्राहि । च‚स्तुल्योपाय‚त्वं स‚मुच्चिनोति ।
	\pend% ending standard par
      ‚{\tiny $_{lb}$}‚‚{\tiny $_{lb}$}‚\textsuperscript{\textenglish{103/dm}}‚{\tiny $_{lb}$}‚
	  \bigskip
	  \begingroup
	

	  \pstart \leavevmode% starting standard par
	का पुन‚रुप‚ल‚ब्धिल‚क्ष‚ण‚प्राप्तिरित्याह--
	\pend% ending standard par
       ‚{\tiny $_{lb}$}‚ 
	  \bigskip
	  \begingroup
	

	  \pstart \leavevmode% starting standard par
	उप‚ल‚ब्धिल‚क्ष‚ण‚प्राप्तिरुप‚ल‚म्भ‚प्र‚त्य‚यान्त‚र‚साक‚ल्यं स्व‚भाव‚विशेष‚श्च ॥ १३ ॥
	\pend% ending standard par
      
	  \endgroup
	‚{\tiny $_{lb}$}‚ 

	  \pstart \leavevmode% starting standard par
	उप‚ल‚ब्धिल‚क्ष‚ण‚प्राप्तिः--उप‚ल‚ब्धिल‚क्ष‚ण‚प्राप्त‚त्वं घ‚ट‚स्य उप‚ल‚म्भ‚प्र‚त्य‚यान्त‚र‚साक‚ल्य‚मिति ।‚{\tiny $_{lb}$}‚ ज्ञान‚स्य घ‚टोऽपि ज‚न‚कः, अन्ये च च‚क्षुराद‚यः । घ‚टाद् दृश्याद‚न्ये हेत‚वः प्र‚त्य‚यान्त‚राणि ।‚{\tiny $_{lb}$}‚ तेषां साक‚ल्यं स‚न्निधिः । स्व‚भाव एव विशिष्य‚ते त‚द‚न्य‚स्मादिति विशेषो विशिष्ट इत्य‚र्थः ।‚{\tiny $_{lb}$}‚ त‚द‚यं विशिष्टः स्व‚भावः प्र‚त्य‚यान्त‚र‚साक‚ल्यं चैत‚द् द्व‚य‚मुप‚ल‚ब्धिल‚क्ष‚ण‚प्राप्त‚त्वं घ‚टादेर्द्र‚ष्ट‚व्य‚म् ॥
	\pend% ending standard par
      
	  \endgroup
	‚{\tiny $_{lb}$}‚

	  \pstart \leavevmode% starting standard par
	य‚द्येक‚स्मिन् ज्ञाने य‚योः संस‚र्गोऽस्ति त‚योरेक‚त‚रोप‚ल‚म्भ‚स्त‚दित‚रानुप‚ल‚म्भ‚निश्च‚य‚हेतुत्वाद्‚{\tiny $_{lb}$}‚ अनुप‚ल‚म्भ‚स्त‚स्माच्च त‚स्याभाव‚व्य‚हार‚स्त‚दा नील‚ज्ञानानुभ‚वे पीत‚ज्ञानाभाव‚व्य‚व‚हारो न स्यात्‚{\tiny $_{lb}$}‚ त‚योरेक‚ज्ञा\leavevmode\ledsidenote{\textenglish{42b/ms}}न‚संस‚र्गाभावात् । न हि भ‚व‚न्म‚ते ज्ञानं ज्ञानान्त‚रेण वेद्य‚ते, स्व‚संवेद‚न‚त्वा‚{\tiny $_{lb}$}‚भाव‚प्र‚स‚ङ्गादिति चेत् । स‚त्य‚मेत‚त् । केव‚ल‚मेक‚ज्ञान‚संस‚र्गिश‚ब्देनान्योन्याव्य‚भिच‚रितोप‚ल‚म्भ‚त्व‚मिह‚{\tiny $_{lb}$}‚ \textbf{विव‚क्षित‚म्} । त‚च्च ज्ञानेऽप्य‚स्ति । य‚दि हि त‚ज्ज्ञानं विद्य‚मानं स्यात् त‚दा नील‚ज्ञान‚व‚त्संविदित‚{\tiny $_{lb}$}‚मेव भ‚वेत् । न च संवेद्य‚ते । त‚स्मान्नास्तीति व्य‚व‚ह्रिय‚त इति किम‚व‚द्य‚म् ।
	\pend% ending standard par
      ‚{\tiny $_{lb}$}‚

	  \pstart \leavevmode% starting standard par
	स्यादेत‚त् । किं पुन‚र्ज्ञातृज्ञेय‚ध‚र्मोप‚ल‚ब्धिव्युदासेन प‚र्युदास‚वृत्तिना न‚ञा ज्ञातृज्ञेय‚ध‚र्म‚{\tiny $_{lb}$}‚ल‚क्ष‚णा द्विविध प‚ल‚ब्धिः प्र‚तिपाद्य‚ते ? न तूप‚ल‚म्भाभाव‚मात्रं प्र‚स‚ज्य‚प्र‚तिषेधाश्र‚येणोच्य‚ते‚{\tiny $_{lb}$}‚ \textbf{य‚थेश्व‚र‚सेनो} म‚न्य‚त इत्याश‚ङ्क्याह--\textbf{द‚र्श‚न}त्यादि । \textbf{द‚र्श‚न‚मु}प‚ल‚ब्धिस्त‚स्य \textbf{निवृत्ति}र‚भाव‚स्तुच्छ‚रूपः‚{\tiny $_{lb}$}‚ सैव त‚न्मात्रं व‚स्त्व‚न्त‚र‚संस‚र्ग‚विर‚हः । \textbf{तुः} पूर्व‚स्माद‚नुप‚ल‚म्भाद् वैध‚र्म्य‚म‚स्य द्योत‚य‚ति ।‚{\tiny $_{lb}$}‚ \textbf{स्व‚य‚म‚निश्चित‚त्वा}दिति ब्रुव‚ताऽनुप‚ल‚म्भात् त‚त्प्र‚तिप‚त्ताव‚न‚व‚स्थादोष‚प्र‚स‚ङ्गेन त‚स्य साध‚काभावः‚{\tiny $_{lb}$}‚ सूचितः । अनिश्चित‚त्वादेवाग‚म‚कः । एव‚ञ्च व्याच‚क्षाणेन इदं सूचित‚म्--त‚थाविधानुप‚ल‚ब्धिः‚{\tiny $_{lb}$}‚ प्र‚माण‚निवृत्ताव‚प्य‚र्थाभावाभावाद‚भाव‚व्य‚व‚हारे साध्येऽनैकान्तिकीति । \textbf{व‚च‚न‚साम‚र्थ्या}दित्युप‚{\tiny $_{lb}$}‚ल‚ब्धिल‚क्ष‚ण‚प्राप्त‚स्येतिव‚च‚न‚साम‚र्थ्यात्, अन्य‚थैत‚द‚तिरिच्येतेति भावः ॥
	\pend% ending standard par
      ‚{\tiny $_{lb}$}‚

	  \pstart \leavevmode% starting standard par
	न‚नूप‚ल‚ब्धिल‚क्ष‚ण‚प्राप्तः स उच्य‚ते य‚स्योप‚ल‚ब्धिल‚क्ष‚ण‚प्राप्तिर‚स्ति । य‚था य‚स्याप्ति‚{\tiny $_{lb}$}‚र्य‚थार्थ‚द‚र्श‚नादिरूपाऽस्ति स आप्त इत्युच्य‚ते । सा चोप‚ल‚ब्धिल‚क्ष‚ण‚प्राप्तिर्य‚द्यात्म‚म‚नःस‚न्निक‚र्षः,‚{\tiny $_{lb}$}‚ इन्द्रिय‚म‚न संयोगः, इन्द्रियार्थ‚स‚न्निक‚र्षः, विष‚य‚प्र‚काश‚संयोगः, अनेक‚द्र‚व्य‚व‚त्त्व‚म्, रूपं चोद्भूतं‚{\tiny $_{lb}$}‚ स‚माख्यातं त‚दा त्व‚न्म‚तेऽमीषाम‚भावादुप‚ल‚ब्धिल‚क्ष‚ण‚प्राप्तिर‚स‚म्भ‚विनीति म‚न्य‚मानः पृच्छ‚ति‚{\tiny $_{lb}$}‚ \textbf{का पुन}रिति । सामान्य‚विशेषाकाराभ्याम‚यं प्र‚श्नः । आचार्य‚स्यापि नामून्युप‚ल‚ब्धिल‚क्ष‚ण‚{\tiny $_{lb}$}‚प्राप्तिश‚ब्देन विव‚क्षितानि ।
	\pend% ending standard par
      ‚{\tiny $_{lb}$}‚

	  \pstart \leavevmode% starting standard par
	किं त‚र्हीय‚मित्य‚भिप्रायेण य‚दु\textbf{प‚ल‚ब्धिल‚क्ष‚णे}त्यादिप्र‚तिव‚च‚नं त‚दुप‚ल‚ब्धीत्यादिना व्याख्यातु‚{\tiny $_{lb}$}‚मुप‚क्र‚म‚ते । व्याख्येय‚मेवो\textbf{प‚ल‚ब्धिल‚क्ष‚ण‚प्राप्ति}श‚ब्द‚स‚मानार्थेनो\textbf{प‚ल‚ब्धिल‚क्ष‚ण‚प्राप्त‚त्व}श‚ब्देनानु‚{\tiny $_{lb}$}‚व‚द‚ति । अयं चास्याश‚यः--य‚स्योप‚ल‚ब्धिल‚क्ष‚ण‚प्राप्तिस्त‚स्याव‚श्य‚मुप‚ल‚ब्धिल‚क्ष‚ण‚प्राप्त‚त्व‚म‚स्तीति ।‚{\tiny $_{lb}$}‚ अत एवोप‚ल‚ब्धिल‚क्ष‚ण‚प्राप्तिरुप‚ल‚ब्धिल‚क्ष‚ण‚प्राप्त‚त्व‚मित्युक्त‚म् । क‚स्येत्याकाङ्क्षायामाह—‚{\tiny $_{lb}$}‚\textbf{घ‚ट‚स्ये}ति ।
	\pend% ending standard par
      ‚{\tiny $_{lb}$}‚

	  \pstart \leavevmode% starting standard par
	न‚नु साक‚ल्यं नामानेक‚ध‚र्मः । न च ज्ञान‚स्य हेत‚वो ब‚ह‚वः । किञ्च य‚दि प्र‚तिषेध्योऽपि‚{\tiny $_{lb}$}‚ ज्ञान‚स्य हेतुः स्यात्त‚दा त‚स्मात्प्र‚त्य‚याद‚न्ये प्र‚त्य‚याः प्र‚त्य‚यान्त‚राण्युच्य‚न्त इत्याश‚ङ्क्याह--\textbf{ज्ञान‚स्ये}ति । \leavevmode\ledsidenote{\textenglish{104/dm}}‚{\tiny $_{lb}$}‚ 
	  
	कीदृशः स्व‚भाव‚विशेष इत्याह-- ‚{\tiny $_{lb}$}‚ 
	  
	\edtext{\textsuperscript{*}}{\lemma{*}\Bfootnote{यः स्व‚भावः स‚त्स्व‚न्येषूप‚ल‚म्भ‚प्र‚त्य‚येषु य‚त्प्र‚त्य‚क्ष एव भ‚व‚ति स स्व‚भावः \cite{dp-msB} \cite{dp-edP} \cite{dp-edH}‚{\tiny $_{lb}$}‚ स‚त्स्व‚न्येषूप‚ल‚म्भ‚प्र‚त्य‚येषु यः प्र‚त्य‚क्ष एव भ‚व‚ति स स्व‚भावः \cite{dp-edE}}}यः स्व‚भावः स‚त्स्व‚न्येषूप‚ल‚म्भ‚प्र‚त्य‚येषु स‚न् प्र‚त्य‚क्ष एव भ‚व‚ति \edtext{}{\lemma{ति}\Bfootnote{स स्व‚भाव‚विशेषः इति नास्ति \cite{dp-msC} \cite{dp-msD} प्र‚त‚योः}}स स्व‚भाव‚{\tiny $_{lb}$}‚विशेषः ॥ १४ ॥‚{\tiny $_{lb}$}‚ 
	  
	स‚त्स्वित्यादि । उप‚ल‚म्भ‚स्य यानि घ‚टाद् दृश्याद् प्र‚त्य‚यान्त‚राणि तेषु स‚त्सु विद्य‚मानेषु‚{\tiny $_{lb}$}‚ यः स्व‚भावः स‚न् प्र‚त्य‚क्ष एव भ‚व‚ति स स्व‚भाव‚विशेषः । त‚द‚य‚म‚त्रार्थः--एक‚प्र‚तिप‚त्त्र‚पेक्ष‚मिदं‚{\tiny $_{lb}$}‚ प्र‚त्य‚क्ष‚ल‚क्ष‚ण‚म् । त‚था च स‚ति द्र‚ष्टुं प्र‚वृत्त‚स्यैक‚स्य द्र‚ष्टुर्दृश्य‚मान उभ‚य‚वान् भावः । अदृश्य‚{\tiny $_{lb}$}‚मानास्तु देश‚काल‚स्व‚भाव‚विप्र‚कृष्टाः स्व‚भाव‚विशेष‚र‚हिताः प्र‚त्य‚यान्त‚र‚साक‚ल्य‚व‚न्त‚स्तु । यैर्हि‚{\tiny $_{lb}$}‚ प्र‚त्य‚यैः स द्र‚ष्टा प‚श्य‚ति ते स‚न्निहिताः । अत‚श्च स‚न्निहिता\edtext{}{\lemma{न्निहिता}\Bfootnote{०हिताय द्र‚ष्टुं \cite{dp-msA} \cite{dp-edP} \cite{dp-edE} ०हिता यः द्र‚ष्टुं \cite{dp-edH} ०हिता यैः द्र‚ष्टुं \cite{dp-edN}}} य‚द् द्र‚ष्टुं \edtext{}{\lemma{ष्टुं}\Bfootnote{प्र‚वृत्ताः \cite{dp-msB}}}प्र‚वृत्तः सः ।‚{\tiny $_{lb}$}‚ न केव‚लं प्र‚त्य‚यान्त‚र‚साक‚ल्य‚मुप‚ल‚ब्धिल‚क्ष‚ण‚प्राप्तिः किन्त्व‚न्य‚द‚पीत्याह--\textbf{स्व‚भावे}ति । \textbf{च}स्तुल्य‚{\tiny $_{lb}$}‚ब‚ल‚त्वं स‚मुच्चिनोति । \textbf{त‚द‚न्य‚स्मात्} पिशाचादेर्वि\textbf{शिष्य‚ते} । ज्ञान‚ज‚न‚न‚योग्य‚त‚या विशेष‚ण‚त्वेऽ‚{\tiny $_{lb}$}‚प्य‚स्य राज‚द‚न्तादिपाठाद् \textbf{विशेष}श‚ब्द‚स्य पूर्व‚निपाता\leavevmode\ledsidenote{\textenglish{43a/ms}}भावः । क‚र्म‚साध‚न‚स्यैव \textbf{विशेष-}‚{\tiny $_{lb}$}‚ श‚ब्द‚स्यार्थ‚माह--\textbf{विशिष्ट} इति ।
	\pend% ending standard par
      ‚{\tiny $_{lb}$}‚

	  \pstart \leavevmode% starting standard par
	द्व‚य‚मेत‚न्मिलित‚मेवोप‚ल‚ब्धिल‚क्ष‚ण‚प्राप्तिश‚ब्द‚वाच्य‚मुप‚संहार‚व्याजेन \textbf{त‚दित्यादिना} द‚र्श‚य‚ति ।‚{\tiny $_{lb}$}‚ य‚तः प्र‚त्य‚यान्त‚र‚साक‚ल्यं स्व‚भाव‚विशेष‚श्चोप‚ल‚ब्धिल‚क्ष‚ण‚प्राप्तिर्विव‚क्षिता त‚त्त‚स्मादुप‚ल‚ब्धि‚{\tiny $_{lb}$}‚ल‚क्ष‚ण‚प्राप्तिश‚ब्द‚वाच्य‚मुप‚ल‚ब्धिल‚क्ष‚ण‚प्राप्त‚त्वं घ‚टादेः प्र‚तिषेध्य‚स्य ॥
	\pend% ending standard par
      ‚{\tiny $_{lb}$}‚

	  \pstart \leavevmode% starting standard par
	अथ किं स्थ‚वीयान् स्व‚भावः स्व‚भाव‚विशेष उत‚स्वित्प‚र‚रूपामिश्र‚स्व‚ल‚क्ष‚णात्म‚क‚{\tiny $_{lb}$}‚ इत्य‚भिप्रेत्य पृच्छ‚ति \textbf{कीदृश} इति । स‚त्स्वित्याद्युत्त‚र‚मुप‚ल‚म्भ‚स्येत्यादिना व्याच‚ष्टे ।
	\pend% ending standard par
      ‚{\tiny $_{lb}$}‚

	  \pstart \leavevmode% starting standard par
	न‚नु किम‚स्य स‚म्भ‚वोऽस्ति य‚दुत प्र‚त्य‚यान्त‚र‚साक‚ल्ये स‚त्य‚पि स्व‚भावः प्र‚त्य‚क्ष एव‚{\tiny $_{lb}$}‚ भ‚व‚तीति । त‚था हि स‚त्य‚पि घ‚ट‚स्य तादृशे स्व‚भावे विदूर‚व‚र्त्तिनः पुरुष‚स्य लोच‚नादि‚{\tiny $_{lb}$}‚प्र‚णिधानेऽपि नासौ प्र‚त्य‚क्षो भ‚व‚तीत्याश‚ङ्क्याह--\textbf{त‚द‚य‚मि}ति । य‚स्माद् द्व‚य‚मेत‚दुप‚ल‚ब्धिल‚क्ष‚ण‚{\tiny $_{lb}$}‚प्राप्तिम‚वोच‚दाचार्य‚स्त‚त्त‚स्माद‚त्र प्र‚स्तावेऽ\textbf{य‚म‚र्थो} वाच्योऽभिम‚तः ।
	\pend% ending standard par
      ‚{\tiny $_{lb}$}‚

	  \pstart \leavevmode% starting standard par
	कोऽसावित्याह--\textbf{एके}ति । एक‚श्चासौ विव‚क्षितः प्र‚तिप‚त्ता चेति त‚था त‚द‚पेक्ष‚{\tiny $_{lb}$}‚मिदं \textbf{प्र‚त्य‚क्ष‚ल‚क्ष‚ण‚म्} । यः स्व‚भावः स‚त्स्व‚न्येषूप‚ल‚म्भ‚प्र‚त्य‚यान्त‚रेषु स‚न् प्र‚त्य‚क्ष एव भ‚व‚तीत्येवं‚{\tiny $_{lb}$}‚ रूपः । एक‚श्च प्र‚तिप‚त्ता स एव वाच्यो योऽव्य‚व‚धानादिदेशो द्र‚ष्टुं प्र‚वृत्त‚श्च । त‚थाविधे च‚{\tiny $_{lb}$}‚ द्र‚ष्ट‚रि त‚थाविधोऽव‚श्यं प्र‚त्य‚क्ष एव भ‚व‚तीति ।
	\pend% ending standard par
      ‚{\tiny $_{lb}$}‚

	  \pstart \leavevmode% starting standard par
	त‚थापि क‚थं पूर्व‚प‚क्षातिक्र‚म इत्याह--\textbf{त‚था चेति} त‚स्मिश्च प्र‚कारे स‚ति । \textbf{दृश्य‚मान}‚{\tiny $_{lb}$}‚ इति हेतुभावेन विशेष‚ण‚म् । \textbf{उभ‚य‚वान्} स्व‚भाव‚विशेष‚वान् प्र‚त्य‚यान्त‚र‚साक‚ल्य‚वांश्च । य‚द्य‚{\tiny $_{lb}$}‚‚{\tiny $_{lb}$}‚ ‚{\tiny $_{lb}$}‚ \leavevmode\ledsidenote{\textenglish{105/dm}}‚{\tiny $_{lb}$}‚ 
	  
	द्र‚ष्टुम‚प्र‚वृत्त‚स्य तु योग्य‚देश‚स्था अपि द्र‚ष्टुं ते न श‚क्याः प्र‚त्य‚यान्त‚र‚वैक‚ल्य‚व‚न्तः, स्व‚भाव‚विशेष‚{\tiny $_{lb}$}‚युक्तास्तु । दूर‚देश‚कालास्तु उभ‚य‚विक‚लाः । त‚देवं प‚श्य‚तः क‚स्य‚चिन्न प्र‚त्य‚यान्त‚र‚विक‚लो‚{\tiny $_{lb}$}‚ नाम, स्व‚भाव‚विशेष‚विक‚ल‚स्तु भ‚वेत् । अप‚श्य‚त‚स्तु\edtext{}{\lemma{स्तु}\Bfootnote{०स्तु श‚क्यो द्र‚ष्टुं योग्य० \cite{dp-msA} \cite{dp-edP} \cite{dp-edH} \cite{dp-edE} \cite{dp-edN}}} द्र‚ष्टुं श‚क्यो योग्य‚देश‚स्थः प्र‚त्य‚यान्त‚र‚{\tiny $_{lb}$}‚विक‚ल । अन्ये तूभ‚य‚विक‚ला इति ॥‚{\tiny $_{lb}$}‚ दृश्य‚माना अपि स्व‚भाव‚विशेष‚व‚न्त‚स्त‚दा किं स्व‚भाव‚विशेष‚ग्र‚ह‚णेनेत्याह--\textbf{अदृश्य‚माना इति} ।‚{\tiny $_{lb}$}‚ तुर्दृश्य‚मानेभ्योऽदृश्य‚मानान् भिन‚त्ति । देशादिविप्र‚कृष्ट‚त्व‚म‚दृश्य‚मान‚त्वे निब‚न्ध‚न‚म्, हेतु‚{\tiny $_{lb}$}‚भावेन विशेष‚णात् । \textbf{ते प्र‚त्य‚यान्त‚र‚साक‚ल्य‚व‚न्तः । तुः} स्व‚भाव‚विशेष‚विर‚हात्प्र‚त्य‚यान्त‚र‚साक‚ल्य‚{\tiny $_{lb}$}‚व‚त्त्वेन तान् विशिन‚ष्टि । \textbf{द्र‚ष्टुं प्र‚वृत्त‚स्ये}त्य‚स्यानुवृत्ताविदं द्र‚ष्ट‚व्य‚म् ।
	\pend% ending standard par
      ‚{\tiny $_{lb}$}‚

	  \pstart \leavevmode% starting standard par
	अय‚म‚त्र प्र‚क‚र‚णार्थः--एक‚प्र‚तिप‚त्त्र‚पेक्ष‚या य‚स्त‚थाविधः स्व‚भावः सोऽपि य‚द्युप‚ल‚ब्धि‚{\tiny $_{lb}$}‚ल‚क्ष‚ण‚प्राप्तिल‚क्ष‚ण‚त्वेन नोपादीय‚ते, त‚दा तेषाम‚पि देशादिविप्र‚क‚र्षिणां प्र‚त्य‚यान्त‚र‚साक‚ल्य‚{\tiny $_{lb}$}‚मुप‚ल‚ब्धिल‚क्ष‚ण‚प्राप्तिर‚स्तीत्युप‚ल‚ब्धिल‚क्ष‚ण‚प्राप्तानाम‚नुप‚ल‚म्भाद‚भाव‚व्य‚व‚हारः प्र‚व‚र्त्त‚नीयः स्यात् ।‚{\tiny $_{lb}$}‚ न चैत‚द् युज्य‚ते । त‚स्माद् विशिष्ट‚प्र‚तिप‚त्त्र‚पेक्ष‚मिदं स्व‚भाव‚विशेष‚स्य ल‚क्ष‚ण‚मित्युप‚ल‚ब्धि‚{\tiny $_{lb}$}‚ल‚क्ष‚ण‚प्राप्तिल‚क्ष‚णं सूक्त‚मिति ।
	\pend% ending standard par
      ‚{\tiny $_{lb}$}‚

	  \pstart \leavevmode% starting standard par
	प्र‚त्य‚यान्त‚र‚साक‚ल्य‚व‚त्त्व‚मेव तेषां साध‚य‚न्नाह--यैरिति । हिर्य‚स्माद‚र्थे । एत‚देव कुतः‚{\tiny $_{lb}$}‚ सिद्ध्य‚तीत्याह--\textbf{अत} इति ॥ \textbf{अत} इत्य‚यं निपातो व‚क्ष्य‚माण‚हेत्व‚र्थः । \textbf{चो} व‚क्त‚व्य‚मेत‚दित्य‚{\tiny $_{lb}$}‚स्मिन्न‚र्थे । \textbf{स‚न्निहितास्ते} प्र‚त्य‚या य‚द् य‚स्माद् \textbf{द्र‚ष्टुं प्र‚वृत्त‚स्}त‚द् विविक्तं द्र‚ष्टुं प्र‚वृत्तो य‚त‚{\tiny $_{lb}$}‚ इत्य‚र्थः । य‚द् वा त‚देव निरीक्षितुं प्र‚वृत्तो य‚त इति । त‚दा तु\leavevmode\ledsidenote{\textenglish{43b/ms}}प्रेक्षापूर्व‚कारीति द्र‚ष्ट‚व्य‚म् ।‚{\tiny $_{lb}$}‚ द‚र्श‚न‚प्र‚वृत्त‚पुरुषापेक्ष‚या ताव‚द‚र्थ‚स्यैवंप्र‚कार‚व‚त्त्व‚म्, द्र‚ष्टुम‚प्र‚वृत्त‚स्य तु स कीदृश इत्याह—‚{\tiny $_{lb}$}‚\textbf{द्र‚ष्टुम‚प्र‚वृत्त‚स्ये}ति । तुना द्र‚ष्टुं प्र‚वृत्ताद‚प्र‚वृत्त‚स्य भेद‚माह । याव‚त्येव देशे स‚ति त‚स्मिन् प्र‚त्य‚या‚{\tiny $_{lb}$}‚न्त‚रे दृश्य‚न्तेऽर्थाः स एव \textbf{योग्यो देशः,} त‚त्र\textbf{स्थाः} ।
	\pend% ending standard par
      ‚{\tiny $_{lb}$}‚

	  \pstart \leavevmode% starting standard par
	त‚स्माद् दृश्याद‚न्ये ये च‚क्षुराद‚यो हेत‚व‚स्तानि \textbf{प्र‚त्य‚यान्त‚राणि} । तेषां \textbf{वैक‚ल्य‚म‚भाव}‚{\tiny $_{lb}$}‚स्त‚द्व‚न्तः, हेतुभावेन विशेष‚णाद् । अत एव द्र‚ष्टुं ते न श‚क्याः । स्व‚भाव‚विशेष‚स्तु तेषाम‚स्तीति‚{\tiny $_{lb}$}‚ द‚र्श‚य‚न्नाह--\textbf{स्व‚भावे}ति । तुना प्र‚त्य‚यान्त‚र‚वैक‚ल्य‚व‚त्त्वात्स्व‚भाव‚विशेष‚युक्त‚त्वेन‚तान् विशिन‚ष्टि ।‚{\tiny $_{lb}$}‚ त‚थाविध‚पुरुषापेक्ष‚या देश‚काल‚विप्र‚कृष्टानां तु का वार्त्तेत्याह--दूरेति । दूरौ देश‚कालौ येषां‚{\tiny $_{lb}$}‚ ते त‚थोक्ताः । ये हि देशेन विप्र‚कृष्टास्ते दूर‚देशाः, ये च कालेन ते दूर‚काला भ‚व‚न्तीति‚{\tiny $_{lb}$}‚ भावः । । \textbf{तुः} पूर्वेभ्य इमान् भिन‚त्ति । अत्रापि द्र‚ष्टुम‚प्र‚वृत्त‚स्येत्य‚नुव‚र्त्त‚ते । \textbf{त‚देव}मित्यादिनो‚{\tiny $_{lb}$}‚क्त‚मेवोप‚संह‚र‚ति ।
	\pend% ending standard par
      ‚{\tiny $_{lb}$}‚

	  \pstart \leavevmode% starting standard par
	अथ‚वा \textbf{एक‚प्र‚तिप‚त्त्र‚पेक्ष‚मिद}मित्य‚न्य‚था व्याख्याय‚ते--इहोप‚ल‚ब्धिल‚क्ष‚ण‚प्राप्त‚स्येत्य‚नेन‚{\tiny $_{lb}$}‚ देश‚काल‚स्व‚भाव‚विप्र‚कृष्ट‚त‚याऽनुप‚ल‚ब्धिल‚क्ष‚ण‚प्राप्ताः किल व्याव‚र्त्त‚यित‚व्याः । न च तेऽप्य‚नु‚{\tiny $_{lb}$}‚प‚ल‚ब्धिल‚क्ष‚ण‚प्राप्ताः श‚क्या व‚क्तुं य‚तो व्य‚व‚च्छिद्येर‚न्, त‚थापि पिशाचोऽपि स‚जातीयैरुप‚ल‚भ्य‚ते ।‚{\tiny $_{lb}$}‚ एवं देश‚विप्र‚कृष्टोऽपि त‚द्देशीयैः । त‚था काल‚विप्र‚कृष्टोऽपि त‚त्कालिकैरिति व्याव‚र्त्त्याभावादुप‚{\tiny $_{lb}$}‚ \leavevmode\ledsidenote{\textenglish{106/dm}}‚{\tiny $_{lb}$}‚ 
	  
	अनुप‚ल‚ब्धिमुदाहृत्य स्व‚भाव‚मुदाह‚र्त्तुमाह-- ‚{\tiny $_{lb}$}‚ 
	  
	स्व‚भावः \edtext{}{\lemma{भावः}\Bfootnote{स्व‚भावः स‚त्ता० \cite{dp-msC} स्व‚भावः स्व‚स‚त्ताभावि \cite{dp-edE}}}स्व‚स‚त्तामात्र‚भाविनि साध्य‚ध‚र्मे हेतुः ॥ १५ ॥‚{\tiny $_{lb}$}‚ 
	  
	\edtext{\textsuperscript{*}}{\lemma{*}\Bfootnote{स्व‚भाव इत्यादि इति नास्ति \cite{dp-msA} \cite{dp-edP} \cite{dp-edH} \cite{dp-edE} \cite{dp-edN}}}स्व‚भाव इत्यादि । स्व‚भावो हेतुरिति स‚म्ब‚न्धः । कीदृशो हेतुः साध्य‚स्य\edtext{}{\lemma{स्य}\Bfootnote{साध्य‚स्यैव स्व० \cite{dp-msA} \cite{dp-edP} \cite{dp-edH} \cite{dp-edE} \cite{dp-edN} साध्य‚स्य भाव \cite{dp-msB}}} स्व‚भाव‚{\tiny $_{lb}$}‚ इत्याह--स्व‚स्य आत्म‚नः स‚त्ता । सैव केव‚ला स्व‚स‚त्तामात्र‚म् । त‚स्मिन् स‚ति भ‚वितुं शीलं‚{\tiny $_{lb}$}‚ य‚स्येति । यो हेतोरात्म‚नः स‚त्ताम‚पेक्ष्य विद्य‚मानो भ‚व‚ति, न तु हेतुस‚त्ताया व्य‚तिरिक्तं‚{\tiny $_{lb}$}‚ क‚ञ्चिद्धेतुम‚पेक्ष‚ते स\edtext{}{\lemma{स}\Bfootnote{स इति नास्ति \cite{dp-msA}}} स्व‚स‚त्तामात्र‚भावी साध्यः । त‚स्मिन् साध्ये यो हेतुः स स्व‚भावः त‚स्य‚{\tiny $_{lb}$}‚ साध्य‚स्य \edtext{}{\lemma{स्य}\Bfootnote{नान्यः इति नास्ति \cite{dp-msC}}}नान्यः ॥ ‚{\tiny $_{lb}$}‚ 
	  
	उदाह‚र‚ण‚म्-- ‚{\tiny $_{lb}$}‚ 
	  
	य‚था वृक्षोऽयं शिंश‚पात्वादिति\edtext{}{\lemma{पात्वादिति}\Bfootnote{त्वाद् । \cite{dp-msC}}} ॥ १६ ॥‚{\tiny $_{lb}$}‚ 
	  
	\edtext{\textsuperscript{*}}{\lemma{*}\Bfootnote{य‚थेति इति नास्ति \cite{dp-edH} \cite{dp-edE} \cite{dp-edN}}}य‚थेति । अय‚म् इति ध‚र्मो । वृक्ष इति साध्य‚म् । शिंश‚पात्वादिति हेतुः ।‚{\tiny $_{lb}$}‚ त‚द‚य‚म‚र्थः--वृक्ष‚व्य‚व‚हार‚योग्योऽय‚म्, शिंश‚पाव्य‚व‚हार‚योग्य‚त्वादिति । य‚त्र प्र‚चुर‚शिंश‚पे\edtext{}{\lemma{पे}\Bfootnote{शिंश‚प‚देशे--\cite{dp-msC}}}‚{\tiny $_{lb}$}‚ देशेऽविदित‚शिंश‚पाव्य‚व‚हारो ज‚डो \edtext{}{\lemma{डो}\Bfootnote{य‚था \cite{dp-msB}}}य‚दा केन‚चिदुच्चां \edtext{}{\lemma{चिदुच्चां}\Bfootnote{०मुप‚द‚र्श्य \cite{dp-msB}}}शिंश‚पामुपाद‚र्श्योच्य‚ते अयं वृक्षः‚{\tiny $_{lb}$}‚ इति त‚दासौ जाड्याच्छिंश‚पाया उच्च‚त्व‚म‚पि \edtext{}{\lemma{पि}\Bfootnote{वृक्ष‚त्व‚व्य‚व‚हार‚निमि० \cite{dp-msC} ०व्य‚व‚हार‚निमि० \cite{dp-msA} \cite{dp-edP} \cite{dp-edE} \cite{dp-edH} \cite{dp-edN}}}वृक्ष‚व्य‚व‚हार‚स्य निमित्त‚म‚ध‚स्य‚ति त‚दा‚{\tiny $_{lb}$}‚ यामेवानुच्चां \edtext{}{\lemma{यामेवानुच्चां}\Bfootnote{०च्चां शिंश‚पां प‚श्य‚ति तामे० \cite{dp-msA} \cite{dp-edP} \cite{dp-edH} \cite{dp-edE} \cite{dp-edN}}}प‚श्य‚ति शिंश‚पां \edtext{}{\lemma{पां}\Bfootnote{०मेवावृक्ष‚त्व‚म० \cite{dp-msA}}}तामेवावृक्ष‚म‚व‚स्य‚ति । स मूढः शिंश‚पात्व‚मात्र‚निमित्ते‚{\tiny $_{lb}$}‚ ल‚ब्धिल‚क्ष‚ण‚प्राप्त‚स्येति विशेष‚ण‚म‚न‚र्थ‚क‚मित्याश‚ङ्क्याह--\textbf{त‚द‚य‚म‚त्रार्थ} इति । त‚दा तु प्र‚त्य‚क्ष‚{\tiny $_{lb}$}‚श‚ब्देनोप‚ल‚ब्धिल‚क्ष‚ण‚प्राप्त‚त्वं वाच्य‚म्, त‚स्य ल‚क्ष‚ण‚मिदं पूर्वोव‚त‚मिति योज्य‚म् । \textbf{एक‚प्र‚तिप‚त्त्र‚{\tiny $_{lb}$}‚पेक्ष‚मिदं प्र‚त्य‚क्ष‚ल‚क्ष‚ण}मुप‚ल‚ब्धिल‚क्ष‚ण‚मित्य‚र्थः ।
	\pend% ending standard par
      ‚{\tiny $_{lb}$}‚

	  \pstart \leavevmode% starting standard par
	त‚थापि क‚थं चोद्यातिक्र‚म इत्याह--\textbf{त‚था चे}ति । शेषं पूर्व‚मेव कृत‚व्याख्यान‚म् ॥
	\pend% ending standard par
      ‚{\tiny $_{lb}$}‚

	  \pstart \leavevmode% starting standard par
	स‚म्प्र‚ति स्व‚भाव‚हेतुं विव‚रितुम\textbf{नुप‚ल‚ब्धिमि}त्यादिनोप‚क्र‚म‚ते । सामान्य‚वृत्तिर‚प्य‚यं‚{\tiny $_{lb}$}‚ \textbf{स्व‚भाव}श‚ब्दः साध्य‚ध‚र्म‚स्य श्रुत‚त्वात्त‚स्यैव स्व‚भावे व‚र्त्त‚ते इत्य‚भिप्रायेण साध्य‚स्य \textbf{स्व‚भाव}‚{\tiny $_{lb}$}‚ इत्य‚भ्य‚वादीत् । हेतोः स्व‚रूप‚स्य चिन्त‚नात्स्व‚श‚ब्देन त‚स्यैवात्मा विव‚क्षितः । एत‚देव यो‚{\tiny $_{lb}$}‚ हेतोरित्यादिना स्फुट‚य‚ति । त‚स्मिन् साध्ये यो हेतुर्ग‚म‚कः ॥
	\pend% ending standard par
      ‚{\tiny $_{lb}$}‚‚{\tiny $_{lb}$}‚\textsuperscript{\textenglish{107/dm}}‚{\tiny $_{lb}$}‚
	  \bigskip
	  \begingroup
	

	  \pstart \leavevmode% starting standard par
	वृक्ष‚व्य‚व‚हारे प्र‚व‚र्त्त्य‚ते । नोच्च‚त्वादि निमित्तान्त‚र‚मिह वृक्ष‚व्य‚व‚हार‚स्य । अपि तु शिंश‚पा‚{\tiny $_{lb}$}‚त्व‚मात्रं निमित्तं--शिंश‚पाग‚त‚शाखादिम‚त्त्वं निमित्त‚मित्य‚र्थः ॥
	\pend% ending standard par
       ‚{\tiny $_{lb}$}‚ 

	  \pstart \leavevmode% starting standard par
	कार्य‚मुदाह‚र्त्तुमाह--
	\pend% ending standard par
       ‚{\tiny $_{lb}$}‚ 
	  \bigskip
	  \begingroup
	

	  \pstart \leavevmode% starting standard par
	कार्यं य‚था\edtext{}{\lemma{था}\Bfootnote{य‚थाग्निर‚त्र \cite{dp-msB} \cite{dp-msD} \cite{dp-edP} \cite{dp-edH} \cite{dp-edE} \cite{dp-edN}}} व‚ह्निर‚त्र धूमादिति ॥ १७ ॥
	\pend% ending standard par
      
	  \endgroup
	‚{\tiny $_{lb}$}‚ 

	  \pstart \leavevmode% starting standard par
	\edtext{\textsuperscript{*}}{\lemma{*}\Bfootnote{अग्निरिति \cite{dp-msB} \cite{dp-msD} \cite{dp-edP} \cite{dp-edH} \cite{dp-edE} \cite{dp-edN}}}व‚ह्निरिति साध्य‚म् । अत्रेति ध‚र्मी । धूमादिति हेतुः । कार्य‚कार‚ण‚भावो लोके‚{\tiny $_{lb}$}‚ प्र‚त्य‚क्षानुप‚ल‚म्भ‚निब‚न्ध‚नः\edtext{}{\lemma{नः}\Bfootnote{०प‚ल‚म्भः निब‚न्ध‚नं प्र० \cite{dp-msB}}} प्र‚तीत इति न स्व‚भाव‚स्येव कार्य‚स्य ल‚क्ष‚ण‚मुक्त‚म् ॥
	\pend% ending standard par
      
	  \endgroup
	‚{\tiny $_{lb}$}‚

	  \pstart \leavevmode% starting standard par
	\textbf{उदाह‚र‚ण}म‚स्यार्थ‚स्येति प्र‚क‚र‚णात् । उदाह्रिय‚ते प्र‚द‚र्श्य‚ते स्व‚भाव‚हेतुर‚नेनेत्युदाह‚र‚णं‚{\tiny $_{lb}$}‚ स्व‚भाव‚हेतुप्र‚तिपाद‚कं वाक्य‚म् । इदं च स्व‚भाव‚हेतोर‚र्थ‚क‚थ‚नं न तु त‚त्प्र‚योगोप‚द‚र्श‚न‚म् ।‚{\tiny $_{lb}$}‚ प्र‚योग‚स्तु--यः शिंश‚पात्व‚व्य‚व‚हार‚योग्यः, स वृक्ष‚त्व‚व्य‚व‚हार‚योग्यः । य‚था प्र‚व‚र्त्तित‚वृक्ष‚त्व‚व्य‚व‚हारा‚{\tiny $_{lb}$}‚ पूर्वाधिग‚ता शिंश‚पा । शिंश‚पाव्य‚व‚हार‚योग्य‚श्चाय‚मिति ।
	\pend% ending standard par
      ‚{\tiny $_{lb}$}‚

	  \pstart \leavevmode% starting standard par
	न‚नु च यः शिंश‚पां प‚श्य‚ति स वृक्षं जानात्येव । त‚त्क‚थ‚म‚त्र साध्य‚साध‚न‚भाव इत्याह—‚{\tiny $_{lb}$}‚\textbf{त‚द‚य‚मि}ति । य‚स्मात् शिंश‚पा साध‚न‚त्वेनोप‚न्य‚स्ता, वृक्षः साध्य‚त्वेन । न चैत‚द् य‚थाश्रुति‚{\tiny $_{lb}$}‚ स‚ङ्ग‚च्छ‚ते \textbf{त‚त्} त‚स्मात्, वृक्षोऽयं--\textbf{वृक्ष‚व्य‚व‚हार‚यो}\leavevmode\ledsidenote{\textenglish{44a/ms}}\textbf{ग्योऽयं} शिंश‚पात्वात् \textbf{शिंश‚पाव्य‚व‚हार‚{\tiny $_{lb}$}‚योग्य‚त्वादित्य‚य‚म‚र्थो वाच्यः--वृक्षोऽयं शिंश‚पात्वा}दित्य‚स्य वाक्य‚स्येति प्र‚क‚र‚णात् । \textbf{इति}र्वाक्यार्थं‚{\tiny $_{lb}$}‚स्यैव स्व‚रूपं द‚र्श‚य‚ति ।
	\pend% ending standard par
      ‚{\tiny $_{lb}$}‚

	  \pstart \leavevmode% starting standard par
	न‚नु यो विदित‚वृक्ष‚व्य‚व‚हारः स स्व‚यं प्र‚त्य‚क्षेणैव तं व्य‚व‚हारं प्र‚व‚र्त्त‚यिष्य‚ति त‚त्क‚थ‚{\tiny $_{lb}$}‚म‚स्यानुमान‚स्याव‚तार इत्याह--\textbf{त‚त्रे}ति वाक्योप‚क्षेपे । \textbf{निमित्त‚मि}त्य‚न्तं सुबोध‚म् ।
	\pend% ending standard par
      ‚{\tiny $_{lb}$}‚

	  \pstart \leavevmode% starting standard par
	अनेन च प्र‚ब‚न्धेन मूढं प्र‚त्येत‚द् व्य‚व‚हार‚साध‚न‚म‚नुमान‚मिति द‚र्शित‚म् । केव‚ल‚{\tiny $_{lb}$}‚मिद‚म‚त्र निरूप‚णीय‚म् । \add{य‚दा च} वृक्ष‚त्व‚व्य‚हार‚व्युत्प‚त्तिं कार्य‚माण एवारोपितोच्च‚त्वादि‚{\tiny $_{lb}$}‚निमित्तः, त‚दा केन दृष्टान्तेन बोध‚यित‚व्यः ? आदित एव तेन शाखादिम‚त्त्व‚मात्रं निमित्तं‚{\tiny $_{lb}$}‚ न गृहीत‚मिति । स‚त्य‚मेत‚त् । केव‚लं बोधे य‚त्नः क‚र‚णीयः । \textbf{त‚दासौ जाड्यादुच्च‚त्व‚म‚पि‚{\tiny $_{lb}$}‚ निमित्त‚म‚व‚स्य}तीतीद‚मेत‚स्मिन् मूढे प्र‚तिप‚त्त‚रि योज‚नीय‚म् । यः प्र‚थ‚मं ताव‚त् शिंश‚पाग‚तं‚{\tiny $_{lb}$}‚ शाखादिम‚त्त्व‚मात्र‚मेव निमित्त‚म‚व‚साय वृक्ष‚व्य‚व‚हारं प्राव‚र्त्त‚य‚त् प‚श्चाज्जाड्य‚व‚शात्त‚न्मात्रं‚{\tiny $_{lb}$}‚ निमित्तं विस्मृत्यान्य‚देव वृक्ष‚व्य‚व‚हार‚काले उच्च‚त्व‚म‚पि निमित्त‚मासीदिति व्यामुह्य त‚दोच्च‚त्व‚म‚पि‚{\tiny $_{lb}$}‚ वृक्ष‚व्य‚व‚हार‚निमित्त‚म‚व‚क‚ल्प‚य‚तीति । स चैवंभूतो ज‚डः शिंश‚पाव्य‚व‚हार‚योग्य‚त्वेन हेतुना प्र‚थ‚मं‚{\tiny $_{lb}$}‚ प्र‚व‚र्त्तित‚त‚न्मात्र‚निमित्त‚वृक्ष‚व्य‚व‚हार‚या त‚दानीं त‚थास्मारित‚या शिंश‚प‚या दृष्टान्तेन वृक्ष‚व्य‚व‚हार‚{\tiny $_{lb}$}‚योग्य‚तां बोध‚यितुं श‚क्य‚त एव । यः पुन‚रादित एवारोपितोच्च‚त्वादिनिमित्त‚स्तं प्र‚ति हेतूप‚न्यास‚{\tiny $_{lb}$}‚ एव न युज्य‚ते । किन्त‚र्हि ? वृक्ष‚व्य‚व‚हार‚स‚म‚य‚मेवासौ ग्राह‚यित‚व्य इति स‚र्व‚म‚व‚दात‚म् ॥
	\pend% ending standard par
      ‚{\tiny $_{lb}$}‚‚{\tiny $_{lb}$}‚\textsuperscript{\textenglish{108/dm}}‚{\tiny $_{lb}$}‚
	  \bigskip
	  \begingroup
	

	  \pstart \leavevmode% starting standard par
	न‚नु त्रिरूप‚त्वादेक‚मेव लिङ्गं\edtext{}{\lemma{लिङ्गं}\Bfootnote{लिङ्ग‚म‚युक्त‚म् \cite{dp-msB} \cite{dp-msC} \cite{dp-msD} \cite{dp-edP} \cite{dp-edH} \cite{dp-edE} \cite{dp-edN}}} युक्त‚म् । अथ प्र‚कार‚भेदाद्भेदः । एवं स‚ति स्व‚भाव‚हेतो‚{\tiny $_{lb}$}‚रेक‚स्यान‚न्त‚प्र‚कार‚त्वात् त्रित्व‚म‚युक्त‚मित्याह--
	\pend% ending standard par
       ‚{\tiny $_{lb}$}‚ 
	  \bigskip
	  \begingroup
	

	  \pstart \leavevmode% starting standard par
	अत्र द्वौ व‚स्तुसाध‚नौ । एकः प्र‚तिषेध‚हेतुः ॥ १८ ॥
	\pend% ending standard par
      
	  \endgroup
	‚{\tiny $_{lb}$}‚ 

	  \pstart \leavevmode% starting standard par
	अत्र द्वौ इति । अत्रेत्येषु त्रिषु हेतुषु म‚ध्ये द्वौ हेतू व‚स्तुसाध‚नौ--विधेः साध‚नौ ग‚म‚कौ ।‚{\tiny $_{lb}$}‚ एकः प्र‚तिषेध‚स्य हेतुर्ग‚म‚कः । प्र‚तिषेध इति चाभावोऽभाव‚व्य‚व‚हार‚श्चोक्तो द्र‚ष्ट‚व्यः ।
	\pend% ending standard par
      
	  \endgroup
	‚{\tiny $_{lb}$}‚

	  \pstart \leavevmode% starting standard par
	इदानीं कार्य‚हेतुं विव‚रीतुमाह--कार्य‚मिति । हेतोः प्र‚कृत‚त्वात्कार्य‚मिति कार्य‚हेतुमित्य‚{\tiny $_{lb}$}‚\textbf{व‚से}यं सुख‚प्र‚तिप‚त्त्य‚र्थ‚म् ।
	\pend% ending standard par
      ‚{\tiny $_{lb}$}‚

	  \pstart \leavevmode% starting standard par
	साध्यादिस्व‚रूप‚माह \textbf{व‚ह्निरि}ति । एत‚द‚पि हेतोर‚र्थ‚क‚थ‚न‚म् । न तु प्र‚योग‚प्र‚द‚र्श‚न‚म् ।‚{\tiny $_{lb}$}‚ व्याप्तेर्द‚र्श‚यित‚व्याया अप्र‚द‚र्श‚नात् । अनिर्देश्यायाश्च प्र‚तिज्ञाया निर्देशात् । व्याप्तिवेदिन्य‚पि‚{\tiny $_{lb}$}‚ पुंसि हेतुर‚नुवाद्येनैव रूपेण निर्दिश्य‚मान प्र‚थ‚मान्त एव निर्देश्यः--अत्र धूम इति । न तु धूमा‚{\tiny $_{lb}$}‚दिति । न च त‚थाविधं प्र‚त्य‚पि प्र‚तिज्ञा प्र‚योज्या अन्य‚था क एनाम‚साध‚नाङ्गं ब्रूयात् ।‚{\tiny $_{lb}$}‚ साध‚नाङ्ग‚त्वे च श‚त‚मुखी बाधा \textbf{वाद‚न्याय‚स्या}प‚द्येत ।
	\pend% ending standard par
      ‚{\tiny $_{lb}$}‚

	  \pstart \leavevmode% starting standard par
	ईदृश‚स्तु प्र‚योगः क‚र‚णीयः--य‚त्र धूम‚स्त‚त्र स‚र्व‚त्र व‚ह्निर्य‚था म‚हान‚से, धूम‚श्चात्रेति ।‚{\tiny $_{lb}$}‚ स्व‚भावानुप‚ल‚ब्ध्योरिव कार्य‚हेतोर‚पि क‚स्माल्ल‚क्ष‚ण‚माचार्येण न प्र‚णीत‚मित्याश‚ङ्काम‚पाकुर्व‚न्नाह—‚{\tiny $_{lb}$}‚\textbf{कार्ये}ति । \textbf{लो}के व्य‚व‚ह‚र्त्त‚रि ज‚ने । \textbf{प्र‚तीतः} प्र‚सिद्धः । \textbf{इति}र्हेतौ । \textbf{नोक्त‚मा}चार्येणेति शेषः ।
	\pend% ending standard par
      ‚{\tiny $_{lb}$}‚

	  \pstart \leavevmode% starting standard par
	अय‚म‚भिप्रायः--अनुप‚ल‚ब्धौ ख‚लु ब‚ह‚वो विप्र‚तिप‚न्नाः । उप‚ल‚ब्ध्य‚भाव‚मात्र‚म‚नुप‚ल‚ब्धि‚{\tiny $_{lb}$}‚\textbf{मीश्व‚र‚सेनो} म‚न्य‚ते । \textbf{कुमारिल}स्तु व‚स्त्व‚न्त‚र‚स्यैक‚ज्ञान‚संस‚र्गिताम‚न‚पेक्ष्यैवान्य‚मात्र‚स्य ज्ञान‚म‚नुप‚{\tiny $_{lb}$}‚ल‚ब्धिम‚भाव‚प्र‚माण‚त‚या व‚र्ण‚य‚ति । त‚था, विव‚क्षित‚ज्ञानानाधार‚ताल‚क्ष‚ण‚मात्म‚नोऽप‚रिणामं स्वापादि‚{\tiny $_{lb}$}‚साधार‚ण‚म\leavevmode\ledsidenote{\textenglish{44b/ms}}पि त‚थात्वेन व‚र्ण‚य‚ति । य‚दाह--सात्म‚नोऽप‚रिणामो वा विज्ञानं‚{\tiny $_{lb}$}‚ वाऽन्य‚व‚स्तुनि \href{http://sarit.indology.info/?cref=śv.abhāva.11}{श्लोक‚वा० अभाव० ११} इति ।
	\pend% ending standard par
      ‚{\tiny $_{lb}$}‚

	  \pstart \leavevmode% starting standard par
	त‚था स्व‚भावेऽपि हेतौ ब‚ह‚वो विप्र‚तिपेदिरे । केचिद‚र्थान्त‚रापेक्षिण्य‚पि ध‚र्मे स्व‚भावं‚{\tiny $_{lb}$}‚ हेतुम‚ध्य‚व‚सिताः । केचित्तु व‚स्तुनो ध‚र्म‚विशेष‚माश्रितं स्व‚भाव‚मिति ।
	\pend% ending standard par
      ‚{\tiny $_{lb}$}‚

	  \pstart \leavevmode% starting standard par
	त‚द्विप्र‚तिप‚त्तिनिराक‚र‚णार्थं त‚योर्ल‚क्ष‚ण‚माख्यात‚म् । अत्र तु कार्य‚त्व‚रूपे न केचिद्‚{\tiny $_{lb}$}‚ विप्र‚तिप‚द्य‚न्त इति नास्य ल‚क्ष‚ण‚मुक्त‚मिति । कार्य‚कार‚ण‚भावेन ग‚म्य‚ग‚म‚क‚भावे स‚र्व‚था‚{\tiny $_{lb}$}‚ ग‚म्य‚ग‚म‚क‚भाव‚प्र‚स‚ङ्ग इत्यादिकायां विप्र‚तिप‚त्ताव‚पि न कार्य‚स्य ल‚क्ष‚णे विप्र‚तिप‚त्ति ।‚{\tiny $_{lb}$}‚ किन्त‚र्हि ? त‚स्य ग‚म‚क‚त्वे । सा चान्य‚त्र निराकृताऽत्रापि प्राज्ञैः स्व‚य‚म‚भ्यूह्या प्राज्ञ‚ज‚ना‚{\tiny $_{lb}$}‚धिकारेणास्य प्रार‚म्भादिति ॥
	\pend% ending standard par
      ‚{\tiny $_{lb}$}‚

	  \pstart \leavevmode% starting standard par
	स‚म्प्र‚ति त्रिरूपाणि च त्रीण्येवेत्य‚स‚ह‚मानः प्राह--\textbf{न‚न्वि}ति । \textbf{न‚नु} प्र‚श्नः । \textbf{अथ}श‚ब्दो‚{\tiny $_{lb}$}‚ \textbf{य‚दिश‚ब्द}स्यार्थे । \textbf{प्र‚कार‚स्य} स्व‚रूप‚स्य \textbf{भेदा}द् विशेषात् \textbf{भेदो} नानात्व‚म् । एव‚म‚भ्युप‚ग‚मे स‚ति ।‚{\tiny $_{lb}$}‚ \textbf{एक‚स्ये}त्य‚भिन्न‚स्य । अभिन्न‚त्व‚ञ्चास्व‚भाव‚हेतुत्व‚व्यावृत्तेः स‚र्व‚त्र भावात् ।
	\pend% ending standard par
      ‚{\tiny $_{lb}$}‚\textsuperscript{\textenglish{109/dm}}‚{\tiny $_{lb}$}‚
	  \bigskip
	  \begingroup
	

	  \pstart \leavevmode% starting standard par
	त‚द‚य‚म‚र्थः--हेतुः साध्य‚सिद्ध्य‚र्थ‚त्वात् साध्याङ्ग‚म्, साध्यं प्र‚धान‚म् । अत‚श्च साध्योप‚{\tiny $_{lb}$}‚क‚र‚ण‚स्य हेतोः प्र‚धान‚साध्य‚भेदाद्भेदः न \edtext{}{\lemma{न}\Bfootnote{स्व‚रूपात् \cite{dp-msC}}}स्व‚रूप‚भेदात् । साध्य‚श्च क‚श्चिद्विधिः, क‚श्चित्‚{\tiny $_{lb}$}‚ प्र‚तिषेधः । विधिप्र‚तिषेध‚योश्च \edtext{}{\lemma{योश्च}\Bfootnote{प‚र‚स्प‚रं प‚रि० \cite{dp-msB} \cite{dp-msD}}}प‚र‚स्प‚र‚प‚रिहारेणाव‚स्थानात् त‚योर्हेतू भिन्नौ । विधिर‚पि‚{\tiny $_{lb}$}‚ क‚श्चिद्धेतोर्भिन्नः, क‚श्चिद‚भिन्न । भेदाभेद‚योर‚प्य‚न्योन्य‚त्यागेनात्म‚स्थितेर्भिन्नौ हेतू । त‚तः‚{\tiny $_{lb}$}‚ साध्य‚स्य प‚र‚स्प‚र‚विरोधात् हेत‚वो\edtext{}{\lemma{वो}\Bfootnote{हेत‚वोऽपि भिन्ना \cite{dp-msB}}} भिन्नाः, न तु स्व‚त एवेति ॥
	\pend% ending standard par
       ‚{\tiny $_{lb}$}‚ 

	  \pstart \leavevmode% starting standard par
	क‚स्मात् पुन‚स्त्र‚याणां हेतुत्व‚म्, क‚स्माच्चान्येषाम‚हेतुत्व‚मित्याश‚ङ्क्य य‚था त्र‚याणामेव‚{\tiny $_{lb}$}‚ हेतुत्व‚म‚न्येषां चाहेतुत्वं त‚दुभ्यं द‚र्श‚यितुमाह--
	\pend% ending standard par
      
	  \endgroup
	‚{\tiny $_{lb}$}‚
	  \bigskip
	  \begingroup
	

	  \pstart \leavevmode% starting standard par
	स्व‚भाव‚प्र‚तिब‚न्धे हि स‚त्य‚र्थोऽर्थं ग‚म‚येत् ॥ १९ ॥
	\pend% ending standard par
      
	  \endgroup
	‚{\tiny $_{lb}$}‚

	  \pstart \leavevmode% starting standard par
	\textbf{अन‚न्त‚प्र‚कार‚त्वा}दिति ब्रुव‚तोऽयं भावः--स‚विशेष‚ण‚निर्विशेष‚ण‚व्य‚तिरिक्ताव्य‚तिरिक्त‚{\tiny $_{lb}$}‚विशेष‚ण‚त्वादिभेदेनान‚न्त‚स्व‚भाव‚त्वादिति । \textbf{ग‚म‚कावि}ति विवृण्व‚न् साध‚य‚त इति \textbf{साध‚नावि}ति‚{\tiny $_{lb}$}‚ क‚र्त्त‚रि ल्युटं द‚र्श‚य‚ति । अत्र च व‚स्तुनः साध‚नावेवेत्य‚व‚धार‚णीयं न तु व‚स्तुन एवेति । इत‚र‚{\tiny $_{lb}$}‚व्य‚व‚च्छेद‚स्यापि ताभ्यां साध‚नात् । प्र‚तिप‚त्त्र‚ध्य‚व‚सायानुरोधात्तु विधिसाध‚न‚त्व‚म‚न‚योरुच्य‚ते ।‚{\tiny $_{lb}$}‚ प्र‚तिषेध‚स्य हेतुरेवेत्य‚व‚धार‚णीय‚म्, न त्व‚य‚मेवेति, पूर्वाभ्याम‚पि साम‚र्थ्यात्प्र‚तिषेध‚स्य साध‚नात् ।
	\pend% ending standard par
      ‚{\tiny $_{lb}$}‚

	  \pstart \leavevmode% starting standard par
	न‚नु न दृश्यानुप‚ल‚म्भेनाभावः साध्य‚ते, त‚स्य प्र‚त्य‚क्ष‚सिद्ध‚त्वात् । किन्तु व्य‚व‚हारः ।‚{\tiny $_{lb}$}‚ त‚त्क‚थं प्र‚तिषेधोऽनुप‚ल‚म्भ‚साध्यः ? अथाभाव‚व्य‚व‚हारः प्र‚तिषेध उच्य‚ते । त‚र्हि व्याप‚कानुप‚{\tiny $_{lb}$}‚ल‚म्भादिना व्याप्याद्य‚भावे साध्ये केनाभावः साधितः ? त‚तः किम‚त्र प्र‚तिषेध‚श‚ब्देन‚{\tiny $_{lb}$}‚ प्र‚तिप‚त्त‚व्य‚मित्याश‚ङ्क्याह--\textbf{प्र‚तिषेध} इति । \textbf{इतिः} प्र‚तिषेध‚श‚ब्दं प्र‚त्य‚व‚मृश‚ति ।
	\pend% ending standard par
      ‚{\tiny $_{lb}$}‚

	  \pstart \leavevmode% starting standard par
	तेन \textbf{प्र‚तिषेध} इत्य‚नेन श‚ब्दे\textbf{नाभावोऽभाव‚व्य‚व‚हार‚श्चोक्तो द्र‚ष्ट‚व्य} इत्य‚र्थः । एक‚स्य प्र‚तिषेधेन‚{\tiny $_{lb}$}‚ इति मुख्य‚या वृत्त्या स‚ङ्ग्र‚होऽन्य‚स्य प्र‚तिषेधाश्र‚य‚त‚या गौण्या वृत्त्येति भावः । नास्तीति \textbf{ज्ञानं}‚{\tiny $_{lb}$}‚ नास्तीत्य‚भिधानं निःश‚ङ्काऽत्र ग‚म‚नाग‚म‚न‚ल‚क्ष‚णा प्र‚वृत्ति\textbf{र्व्य‚व‚हारः} । स च ह‚ठात्प्र‚व‚र्त्त‚यितुं न‚{\tiny $_{lb}$}‚ श‚क्य‚त इति त‚द्योग्य‚तैव साध्येति द्र‚ष्ट‚व्य‚म् । एवं तु स्व‚भाव‚हेताव‚न्त‚र्भावेऽप्य‚नुप‚ल‚म्भ‚स्य‚{\tiny $_{lb}$}‚ त‚तः पृथ‚क्क‚र‚णं प्र‚तिप‚त्त्र‚ध्य‚व‚साय‚व‚शादित्य‚व‚सेय‚म् ।
	\pend% ending standard par
      ‚{\tiny $_{lb}$}‚

	  \pstart \leavevmode% starting standard par
	न‚नु विधिप्र‚तिषेध‚साध‚न‚त्वेऽप्य‚मीषां त्रिरूप‚त्व‚म‚विशिष्ट‚म् । त‚त्त्वादेव चाभेद‚श्चोदितः ।‚{\tiny $_{lb}$}‚ त‚त्क‚थ‚मिद‚मुत्त‚रं पूर्व‚प‚क्ष‚म‚तिव‚र्त्त‚तामित्याश‚ङ्क्याह--\textbf{त‚द‚य‚मि}ति । य‚स्मादिद‚मुत्त‚रीकृत‚माचार्येण‚{\tiny $_{lb}$}‚ य‚थाश्रुति च पूर्व‚प‚क्षं नातिक्राम‚ति \textbf{त‚त्त‚स्माद‚य‚म‚र्थो} वाक्य‚स्यायं तात्प‚र्यार्थ इत्य‚र्थः । एत‚मेवार्थं‚{\tiny $_{lb}$}‚ हेतुरित्यादिना \textbf{न तु स्व‚त एवेत्य}न्तेन ग्र‚न्थेन प्र‚तिपाद‚य‚ति ॥
	\pend% ending standard par
      ‚{\tiny $_{lb}$}‚

	  \pstart \leavevmode% starting standard par
	अथ क‚थ‚म‚न्योऽर्थोऽन्य‚म‚र्थं न व्य‚भिच‚र‚ति येनैते त्र‚यो हेत‚वः ? य\leavevmode\ledsidenote{\textenglish{45a/ms}}था चामीषां‚{\tiny $_{lb}$}‚ स्व‚साध्य‚साध‚नाद् ग‚म‚क‚त्वं त‚थाऽन्येषाम‚प्य‚कार्य‚स्व‚भावानुप‚ल‚म्भात्म‚नां किं न भ‚व‚तीति म‚न्वानः‚{\tiny $_{lb}$}‚ प्र‚श्नेनोप‚क्र‚म‚ते--\textbf{क‚स्मादि}ति । \textbf{क‚स्मादिति} सामान्य‚तो हेतुं पृच्छ‚ति \textbf{पुन‚रि}ति विशेष‚तः ।‚{\tiny $_{lb}$}‚ \textbf{त्र‚याणाम}नुप‚ल‚ब्ध्यादीनाम् । \textbf{चः} पूर्व‚निमित्तापेक्ष‚या निमित्तान्त‚र‚स‚मुच्च‚यार्थः । \textbf{अन्येषा}‚{\tiny $_{lb}$}‚म‚नीदृशात्म‚नां संयोग्यादीनाम् ।
	\pend% ending standard par
      ‚{\tiny $_{lb}$}‚‚{\tiny $_{lb}$}‚\textsuperscript{\textenglish{110/dm}}‚{\tiny $_{lb}$}‚
	  \bigskip
	  \begingroup
	

	  \pstart \leavevmode% starting standard par
	स्व‚भाव‚प्र‚तिब‚न्ध इति । स्व‚भावेन प्र‚तिब‚न्धः \edtext{}{\lemma{न्धः}\Bfootnote{स्व‚भाव‚प्र‚तिब‚न्धः इति नास्ति \cite{dp-msC}}}स्व‚भाव‚प्र‚तिब‚न्धः । साध‚नं कृता \href{http://sarit.indology.info/?cref=vk-mbh.2.1.33}{व्या० 	म‚हा० २. १. ३३} इति स‚मासः । स्व‚भाव‚प्र‚तिब‚द्ध‚त्वं प्र‚तिब‚द्ध‚स्व‚भाव‚त्व‚मित्य‚र्थः । कार‚णे‚{\tiny $_{lb}$}‚ \edtext{\textsuperscript{*}}{\lemma{*}\Bfootnote{स्व‚भावे स्वोत्प‚त्तौ स‚त्यां प्र‚तिब‚न्धः स्व‚स‚त्तायाः प्र‚तिब‚न्धः ।--\cite{dp-msD-n}}}स्व‚भावे च साध्ये स्व‚भावेन प्र‚तिब‚न्धः कार्य‚स्व‚भाव‚योर‚विशिष्ट इत्येकेन स‚मासेन द्व‚योर‚पि‚{\tiny $_{lb}$}‚ संग्र‚हः । हिर्य‚स्माद‚र्थे । य‚स्मात् स्व‚भाव‚प्र‚तिब‚न्धे स‚ति साध‚नार्थः साध्यार्थ ग‚म‚येत्, त‚स्मात्‚{\tiny $_{lb}$}‚ त्र‚याणां ग‚म‚क‚त्व‚म्, अन्येषाम‚ग‚म‚क‚त्व‚म् ॥
	\pend% ending standard par
       ‚{\tiny $_{lb}$}‚ 

	  \pstart \leavevmode% starting standard par
	क‚स्मात् पुनः स्व‚भाव‚प्र‚तिब‚न्ध एव स‚ति ग‚म्य‚ग‚म‚क‚भावो नान्य‚थेत्याह--
	\pend% ending standard par
       ‚{\tiny $_{lb}$}‚ 
	  \bigskip
	  \begingroup
	

	  \pstart \leavevmode% starting standard par
	त‚द‚प्र‚तिब‚द्ध‚स्य त‚द‚व्य‚भिचार‚निय‚माभावात् ॥ २० ॥
	\pend% ending standard par
      
	  \endgroup
	‚{\tiny $_{lb}$}‚ 

	  \pstart \leavevmode% starting standard par
	\hphantom{.}त‚द‚प्र‚तिब‚द्ध‚स्येति । त‚द् इति स्व‚भाव उक्तः । तेन स्व‚भावेन अप्र‚तिब‚द्धः--त‚द‚प्र‚ति‚{\tiny $_{lb}$}‚ब‚द्धः । यो य‚त्र स्व‚भावेन न प्र‚तिब‚द्धः त‚स्य\edtext{}{\lemma{स्य}\Bfootnote{लिङ्ग‚स्य--\cite{dp-msD-n}}} \edtext{\textsuperscript{*}}{\lemma{*}\Bfootnote{स्व‚भावेन अप्र‚तिब‚द्ध‚स्य--\cite{dp-msD-n}}}त‚द‚प्र‚तिब‚द्ध‚स्य त‚द‚व्य‚भिचार‚निय‚माभावात्\edtext{}{\lemma{माभावात्}\Bfootnote{०निय‚माभावः--\cite{dp-msA} \cite{dp-edP} \cite{dp-edH} \cite{dp-edE}}} ।‚{\tiny $_{lb}$}‚ \edtext{\textsuperscript{*}}{\lemma{*}\Bfootnote{त‚स्याप्र‚तिब‚द्ध‚विष‚य‚स्य \cite{dp-msA} \cite{dp-edP} \cite{dp-edH} त‚स्य प्र‚ति० \cite{dp-edN}}}त‚स्याप्र‚तिब‚न्ध‚विष‚य‚स्याव्य‚भिचारः त‚द‚व्य‚भिचारः, त‚स्य निय‚मः त‚द‚व्य‚भिचार‚निय‚मः,‚{\tiny $_{lb}$}‚ त‚स्याभावात् ।
	\pend% ending standard par
       ‚{\tiny $_{lb}$}‚ 

	  \pstart \leavevmode% starting standard par
	\edtext{\textsuperscript{*}}{\lemma{*}\Bfootnote{अय‚म‚र्थः \cite{dp-msA} \cite{dp-msB} \cite{dp-msC} \cite{dp-msD} \cite{dp-edP} \cite{dp-edH} \cite{dp-edE} \cite{dp-edN}}}त‚द‚य‚म‚र्थः--न हि यो य‚त्र स्व‚भावेन\edtext{}{\lemma{भावेन}\Bfootnote{स्व‚भावेन प्र‚ति \cite{dp-msB}}} न प्र‚तिब‚द्धः, स \edtext{}{\lemma{स}\Bfootnote{त‚म‚प्र‚तिब‚द्ध‚विष० \cite{dp-msA} \cite{dp-edP} \cite{dp-edH}}}त‚म‚प्र‚तिब‚न्ध‚विष‚य‚म‚व‚श्य‚मेव‚{\tiny $_{lb}$}‚ न व्य‚भिच‚र‚तीति नास्ति त‚योर‚व्य‚भिचार‚निय‚मः--अविनाभाव‚निय‚मः । अव्य‚भिचार-
	\pend% ending standard par
      
	  \endgroup
	‚{\tiny $_{lb}$}‚

	  \pstart \leavevmode% starting standard par
	\textbf{स्व‚भावेन} स्व‚रूपेण । साध‚नं कृता \href{http://sarit.indology.info/?cref=vk-mbh.2.1.33}{व्या० म‚हा० २. १. ३३}इति \textbf{पाणिनीय‚भाष्य‚कार‚स्येदं}‚{\tiny $_{lb}$}‚ सूत्र‚म् । तेन क‚र्त्तृक‚र‚णे कृता ब‚हुल‚म् \href{http://sarit.indology.info/?cref=Pā.2.1.32}{पाणिनि २. १. ३२} इति सूत्र‚म‚प‚नीय ग‚लेचोप‚क‚{\tiny $_{lb}$}‚ इत्यादिसिद्ध्य‚र्थ साध‚नं कृता इति सूत्रं कृत‚म् । वार्त्तिक‚सूत्रिकाणां तु तृतीया \href{http://sarit.indology.info/?cref=Pā.2.1.30}{पाणिनि 	२. १. ३०} इति योग‚विभागात्स‚मासोऽव‚सेयः । अनेन च तृतीयास‚मासेनैव कार्य‚स्व‚भाव‚योः‚{\tiny $_{lb}$}‚ संग्र‚हादावृत्त्या ष‚ष्ठीस‚प्त‚मीस‚मासाभ्यां कार्य‚स्व‚भाव‚योः संग्र‚ह इति य‚त्पूर्वैर्व्याख्यातं त‚द‚प‚{\tiny $_{lb}$}‚व्याख्यान‚मिति ख्यापित‚म् ।
	\pend% ending standard par
      ‚{\tiny $_{lb}$}‚

	  \pstart \leavevmode% starting standard par
	स‚म‚स्त‚स्य प‚द‚स्यार्थ‚माह--\textbf{स्व‚भावेति} । अनेन प्र‚तिब‚न्ध‚श‚ब्देन \textbf{प्र‚तिब‚द्ध‚त्व}माय‚त्त‚त्व‚{\tiny $_{lb}$}‚मुच्य‚त इति द‚र्श‚य‚ति । अस्यैवार्थं स्प‚ष्ट‚य‚ति । \textbf{प्र‚तिब‚द्धे}ति । यः स्व‚रूपेण क्व‚चिदाय‚त्त‚स्त‚स्य‚{\tiny $_{lb}$}‚ स्व‚भाव‚स्त‚त्र प्र‚तिब‚द्ध आय‚त्त इत्य‚र्थाभेदेन \textbf{प्र‚तिब‚द्ध‚स्व‚भाव‚त्व‚मित्य‚र्थ} इति स्प‚ष्टीकृत‚म् ।
	\pend% ending standard par
      ‚{\tiny $_{lb}$}‚

	  \pstart \leavevmode% starting standard par
	न‚नु पूर्वेषाम‚भिम‚त‚स‚मास‚व्युदासेन तृतीयास‚मासं द‚र्श‚य‚ता किंस्विद‚ति यो ल‚ब्धः ?‚{\tiny $_{lb}$}‚ केव‚ल‚माहोपुरुषिका प्र‚काशितेत्याश‚ङ्क्य पूर्वं बुद्धिस्थ‚मेव स्फुट‚यितुमाह--\textbf{कार‚ण} इति । क‚स्यासौ
	\pend% ending standard par
      ‚{\tiny $_{lb}$}‚‚{\tiny $_{lb}$}‚\textsuperscript{\textenglish{111/dm}}‚{\tiny $_{lb}$}‚
	  \bigskip
	  \begingroup
	

	  \pstart \leavevmode% starting standard par
	निय‚माच्च ग‚म्य‚ग‚म‚क‚भावः । न हि योग्य‚त‚या प्र‚दीप‚व‚त् प‚रोक्षार्थ‚प्र‚तिप‚त्तिनिमित्त‚मिष्टं‚{\tiny $_{lb}$}‚ लिङ्ग‚म् अपि त्व‚व्य‚भिचारित्वेन निश्चित‚म् । त‚तः स्व‚भाव‚प्र‚तिब‚न्धे \edtext{}{\lemma{न्धे}\Bfootnote{स‚त्य‚विनाभाव‚निश्च‚यः । \cite{dp-msA} \cite{dp-edP} \cite{dp-edH} \cite{dp-edE} \cite{dp-edN}}}स‚त्य‚विनाभावित्व‚{\tiny $_{lb}$}‚निश्च‚यः, त‚तो ग‚म्य‚ग‚म‚क‚भावः । त‚स्मात् स्व‚भाव‚प्र‚तिब‚न्धे स‚त्य‚र्थोऽर्थं ग‚म‚येन्नान्य‚थेति स्थित‚म् ॥
	\pend% ending standard par
       ‚{\tiny $_{lb}$}‚ 

	  \pstart \leavevmode% starting standard par
	न‚नु च प‚राय‚त्त‚स्य प्र‚तिब‚न्धोऽप‚राय‚त्ते । त‚दिह साध्य‚साध‚न‚योः क‚स्य क्व‚{\tiny $_{lb}$}‚ प्र‚तिब‚न्ध इत्याह--
	\pend% ending standard par
      
	  \endgroup
	‚{\tiny $_{lb}$}‚
	  \bigskip
	  \begingroup
	

	  \pstart \leavevmode% starting standard par
	स च प्र‚तिब‚न्धः साध्येऽर्थे लिङ्ग‚स्य ॥ २१ ॥
	\pend% ending standard par
      
	  \endgroup
	‚{\tiny $_{lb}$}‚

	  \pstart \leavevmode% starting standard par
	प्र‚तिब‚न्ध इत्य काङ्क्षायामाह--\textbf{कार्य‚स्व‚भाव‚यो}स्त‚योरेव प्र‚कृत‚त्वात् । \textbf{अविशिष्टः} साधार‚णः ।‚{\tiny $_{lb}$}‚ \textbf{द्व‚योर‚पि} कार्य‚स्व‚भाव‚योर‚तिश‚ये \textbf{स‚ङ्ग्र‚हः} स्वीकारः । अय‚मेवातिश‚य इति भावः । साध‚न‚{\tiny $_{lb}$}‚ल‚क्ष‚णोऽर्थः साध्य‚ल‚क्ष‚ण‚म‚र्थं \textbf{ग‚म‚येत्} बोध‚यितुं श‚व‚नोति । य‚त एवं \textbf{त‚स्माद् अन्येषां} त‚द्व्य‚ति‚{\tiny $_{lb}$}‚रिक्तानाम् । तेषां स्व‚भाव‚प्र‚तिब‚न्धाभावात् । त‚द‚भाव‚श्च तेषां तादात्म्य‚त‚दुत्प‚त्त्य‚भावात् ।‚{\tiny $_{lb}$}‚ त‚द‚न्य‚स्य च स‚म्ब‚न्ध‚स्याभावात् । तादात्म्य‚त‚दुत्प‚त्तिभावे च कार्य‚स्व‚भाव‚योरेवान्त‚र्भाव‚{\tiny $_{lb}$}‚ इति भावः ॥
	\pend% ending standard par
      ‚{\tiny $_{lb}$}‚

	  \pstart \leavevmode% starting standard par
	\textbf{त‚दि}त्यादिना स‚मासं प्र‚द‚र्श्य \textbf{त‚द‚प्र‚तिब‚द्ध‚स्ये}ति योज‚य‚ता \textbf{ध‚र्मोत्त}रेण मूले \textbf{त‚द‚प्र‚तिब‚द्ध‚{\tiny $_{lb}$}‚स्येति} पाठो द‚र्शितः । दृश्य‚ते च ब‚हुश‚स्त‚द‚प्र‚तिब‚द्ध‚स्व‚भाव‚स्येति पाठः । त‚त्रापि पाठे त‚द‚प्र‚तिब‚द्धः‚{\tiny $_{lb}$}‚ साध्याप्र‚तिब‚द्धः स्व‚भावः स्व‚रूपं य‚स्य लिङ्ग‚स्येति विग्र‚हः कार्यः । \textbf{प्र‚तिब‚न्धः} प्र‚तिब‚द्ध‚त्व‚मा‚{\tiny $_{lb}$}‚य‚त्त‚त्वं य‚त्साध‚न‚स्य, त‚स्य \textbf{विष‚योऽव्य‚भिचार}स्तेन विनाऽभ‚वितृत्व‚म् । त‚स्य \textbf{निय‚मो}ऽव‚स्य\edtext{}{\lemma{स्य}\Bfootnote{श्य}}न्ता ।
	\pend% ending standard par
      ‚{\tiny $_{lb}$}‚

	  \pstart \leavevmode% starting standard par
	न‚नु त‚द‚प्र‚तिब‚द्ध‚त्वेऽपि अद्य‚त‚न आदित्योद‚योऽस्त‚म‚य‚म‚प्र‚तिब‚न्ध‚विष‚यं न व्य‚भिच‚र‚तीत्या‚{\tiny $_{lb}$}‚श‚ङ्क्याह--\textbf{त‚द‚य‚म‚र्थ} इति । य‚त एव‚मुक्त‚म् \textbf{त‚त्त‚स्माद‚यं} तात्प‚र्यार्थः । आदित्योद‚योऽपि हि‚{\tiny $_{lb}$}‚ भ‚विष्य‚ति । न च त‚द‚ह‚र‚स्त‚म‚यः, म‚ह‚र्षिणाऽन्येन वा म‚ह‚र्धिना केन‚चित्त‚स्यास्त‚म‚य‚विब‚न्ध‚{\tiny $_{lb}$}‚स‚म्भ‚वात् । त‚स्मादादित्य‚स्य त‚स्माद‚स्त‚म‚य‚योःय‚तैव साध्या--अय‚मादित्योद‚योऽस्त‚म‚य‚योग्य‚{\tiny $_{lb}$}‚ उद‚य‚त्वात् । श्व‚स्त‚नोद‚य‚व‚दिति । स‚ति चैवं स्व‚भाव‚हेतुत्व‚म‚स्यायात‚मिति भावः ।
	\pend% ending standard par
      ‚{\tiny $_{lb}$}‚

	  \pstart \leavevmode% starting standard par
	न‚नु चाव्य‚भिचार‚मात्रेण प्र‚योज‚न‚म्, त‚त्किं निय‚मेनेत्याह--\textbf{अव्य‚भिचारेति । चो}ऽव‚{\tiny $_{lb}$}‚धार‚णे हेतौ वा । एत‚देव कुत इत्याह--\textbf{न ही}ति । \leavevmode\ledsidenote{\textenglish{45b/ms}} \textbf{हि}र्य‚स्मात् । प्र‚दीपो वैध‚र्म्य‚{\tiny $_{lb}$}‚दृष्टान्तः । \textbf{अपि} तु किन्त्व\textbf{व्य‚भिचारित्वेन} साध्य‚स्य प्र‚कृत‚त्वात् साध्याविनाभावित्वेन‚{\tiny $_{lb}$}‚ \textbf{निश्चित‚म्} । निय‚माभावे च कुतोऽव्य‚भिचार‚निश्च‚य इत्य‚स्य तात्प‚र्यार्थः । निश्चीय‚तां‚{\tiny $_{lb}$}‚ त‚द‚व्य‚भिचारोऽन्येषाम‚पि, प्र‚तिब‚न्ध‚स्तु क‚स्मान्मृग्य‚त इत्याह--त‚त इति । य‚तोऽव‚श्य‚म‚व्य‚{\tiny $_{lb}$}‚भिचारो निश्चेत‚व्य‚स्त‚त‚स्त‚स्मात् । अन्य‚थाऽव्य‚भिचार एव न श‚क्य‚ते निश्चेतुमित्य‚र्थाद‚नेन‚{\tiny $_{lb}$}‚ द‚र्शित‚म् । एताव‚ताऽपि क‚थं ग‚म‚क‚त्व‚मित्याह--\textbf{त‚त} इति । त‚तो निश्चिताद‚व्य‚भिचारा‚{\tiny $_{lb}$}‚ल्लिङ्ग‚स्य ग‚म‚क‚त्वे मिद्धे साध्य‚स्यापि ग‚म‚क\edtext{}{\lemma{क}\Bfootnote{ग‚म्य}}त्वं सिद्ध्य‚तीति द्व‚योरुप‚यासः । उक्त‚{\tiny $_{lb}$}‚म‚र्थ‚मुप‚संह‚र‚न्नाह--\textbf{त‚स्मादि}ति । \textbf{अर्थो} लिङ्ग‚ल‚क्ष‚णः, \textbf{अर्थं} लिङ्गिल‚क्ष‚ण‚म् ॥
	\pend% ending standard par
      ‚{\tiny $_{lb}$}‚

	  \pstart \leavevmode% starting standard par
	\textbf{त‚त्त‚स्मादिहानु}मानानुमेय‚चिन्तायां स्व‚भाव‚प्र‚तिब‚न्ध‚चिन्तायां वा ।
	\pend% ending standard par
      ‚{\tiny $_{lb}$}‚\textsuperscript{\textenglish{112/dm}}‚{\tiny $_{lb}$}‚
	  \bigskip
	  \begingroup
	

	  \pstart \leavevmode% starting standard par
	स चेति । स च स्व‚भाव‚प्र‚तिब‚न्धो लिङ्ग‚स्य साध्येऽर्थे । लिङ्गं प‚राय‚त्त‚त्वात्‚{\tiny $_{lb}$}‚ प्र‚तिब‚द्ध‚म् । साध्य‚स्त्व‚र्थोऽप‚राय‚त्त‚त्वात् प्र‚तिब‚न्ध‚वि\unclear{यो} न \edtext{}{\lemma{न}\Bfootnote{न प्र‚तिब० \cite{dp-msA} \cite{dp-msB} \cite{dp-msD} \cite{dp-edP} \cite{dp-edH} \cite{dp-edE} \cite{dp-edN}}}तु प्र‚तिब‚द्ध इत्य‚र्थः । त‚त्राय‚{\tiny $_{lb}$}‚म‚र्थः--तादात्म्याविशेषेऽपि य‚त् प्र‚तिब‚द्धं\edtext{}{\lemma{द्धं}\Bfootnote{लिङ्ग‚म्--\cite{dp-msD-n}}} त‚द् ग‚म‚क‚म् । य‚त् प्र‚तिब‚न्ध‚विष‚यः त‚द् ग‚म्य‚म् ।‚{\tiny $_{lb}$}‚ य‚स्य च ध‚र्म‚स्य \edtext{}{\lemma{स्य}\Bfootnote{य‚न्निय‚तः \cite{dp-msA} \cite{dp-msB} \cite{dp-msC} \cite{dp-msD} \cite{dp-edP} \cite{dp-edH} \cite{dp-edE} \cite{dp-edN}}}यो निय‚तः स्व‚भावः स त‚त्प्र‚तिब‚द्धः । य‚था प्र‚य‚त्नान‚न्त‚रीय‚क‚त्वाख्योऽनित्य‚त्वे ।‚{\tiny $_{lb}$}‚ य‚स्य तु स चान्य‚श्च स्व‚भावः च प्र‚तिब‚न्ध‚विष‚यः, न तु प्र‚तिब‚द्धः । य‚थाऽनित्य‚त्वाख्यः‚{\tiny $_{lb}$}‚ प्र‚य‚त्नान‚न्त‚रीय‚क‚त्वाख्ये । निश्च‚यापेक्षो हि ग‚म्य‚ग‚म‚क‚भावः । प्र‚य‚त्नान्त‚रीय‚क‚त्व‚मेव‚{\tiny $_{lb}$}‚ चानित्य‚त्व‚स्व‚भावं निश्चित‚म् । अत‚स्त‚देव अनित्य‚त्वे प्र‚तिब‚द्ध‚म् । त‚स्मान्निय‚त‚विष‚य एव‚{\tiny $_{lb}$}‚ ग‚म्य‚ग‚म‚क‚भावः नान्य‚थेति ।
	\pend% ending standard par
      
	  \endgroup
	‚{\tiny $_{lb}$}‚

	  \pstart \leavevmode% starting standard par
	न‚नु भिन्न‚योर्ग‚म्य‚ग‚म‚क‚भावे लिङ्गं त‚दुत्प‚त्त्या प‚राय‚त्त‚त्वात्प्र‚तिब‚द्ध‚म् । साध्य‚स्त्व‚प‚राय‚त्त‚{\tiny $_{lb}$}‚त्वात्प्र‚तिब‚न्ध‚विष‚यः । त‚दाश्र‚य‚श्च साध्य‚साध‚न‚भाव‚निय‚मः स्यात् । स्व‚भाव‚योस्तु‚{\tiny $_{lb}$}‚ त‚द्भावे हेतोस्तादात्म्य‚प्र‚तिब‚द्ध‚त्व‚म् । तादात्म्यं चोभ‚योर‚विशिष्ट‚म् । त‚तः प्र‚तिव‚द्ध‚त्वं प्र‚तिब‚न्ध‚{\tiny $_{lb}$}‚विष‚य‚त्वं वा द्व‚योर‚विशिष्ट‚मायाति । त‚दाश्र‚य‚श्च निय‚तः साध्य‚साध‚न‚भावः प्र‚स‚क्त इत्याह—‚{\tiny $_{lb}$}‚\textbf{त‚त्राय‚म‚र्थ} इति । न केव‚ल‚म‚र्थान्त‚र‚त्व इत्य‚पिश‚ब्दः । \textbf{य‚त्प्र‚तिब‚द्ध}मिति य‚त्साध्य‚प्र‚तिब‚द्ध‚त‚या‚{\tiny $_{lb}$}‚ त‚दाय‚त्त‚त‚या निश्चित‚मिति द्र‚ष्ट‚व्य‚म् । \textbf{प्र‚तिब‚न्ध‚विष‚यो}ऽपि त‚त्त्वेन निश्चितो द्र‚ष्ट‚व्यः ।
	\pend% ending standard par
      ‚{\tiny $_{lb}$}‚

	  \pstart \leavevmode% starting standard par
	न‚नु तादात्म्याविशेषादेक‚स्त‚त्र प्र‚तिब‚न्ध‚विष‚य‚त‚या न तु प्र‚तिब‚द्ध‚त‚येत्य‚य‚मेव विभागः कुत‚{\tiny $_{lb}$}‚ इत्याह--\textbf{य‚स्येति} । य‚द्वा तादात्म्यानुभाविनि द्व‚ये किन्त‚त्र प्र‚तिब‚द्धं य‚द् ग‚म‚कं किञ्च प्र‚ति‚{\tiny $_{lb}$}‚ब‚न्ध‚विष‚यो य‚द् ग‚म्य‚मित्य‚जान‚न्तं प्र‚त्याह--\textbf{य‚स्ये}ति । \textbf{चो} हेतौ द्वितीय‚प‚क्षेऽव‚धार‚णे । \textbf{य‚स्य}‚{\tiny $_{lb}$}‚ ध‚र्म‚स्य व्यावृत्तिक‚ल्पित‚स्य \textbf{यो निय‚तः} प्र‚तिनिय‚तः स एव स्व‚भावो न त‚द‚न्योऽपि । स इति‚{\tiny $_{lb}$}‚ \textbf{य‚स्ये}ति ष‚ष्ठ्य‚न्तेनोक्तः प‚रामृष्टः । \textbf{त‚दि}ति \textbf{य} इति प्र‚थ‚मान्तेनोक्तः प्र‚त्य‚व‚मृष्ट‚स्त‚स्मिन् प्र‚तिब‚द्ध‚{\tiny $_{lb}$}‚ इति विग्र‚हः । निश्चीय‚त इति शेषः । प्र‚क‚र‚ण‚ल‚भ्यं वा । कः पुन‚रीदृश इत्याह--\textbf{य‚थे}ति ।‚{\tiny $_{lb}$}‚ \textbf{प्र‚य‚त्नः} पुरुष‚व्यापार‚स्त‚स्या\textbf{न‚न्त‚र‚म}व्य‚व‚धान‚म् । त‚त्र भ‚व इति दिगादित्वाद् य‚त् । त‚तः‚{\tiny $_{lb}$}‚ स्वार्थे क‚न् । त‚स्य भाव‚स्त‚त्त्व‚म् । त‚दाख्या नाम य‚स्य स त‚था । प्र‚य‚त्नान‚न्त‚र्य‚क‚त्व‚स्य‚{\tiny $_{lb}$}‚ ह्य‚नित्य‚त्व‚मेव स्व‚भावो न तु नित्य‚त्व‚म‚पि । त‚तोऽनित्य‚त्वे प्र‚तिब‚द्धं निश्चीय‚ते ।
	\pend% ending standard par
      ‚{\tiny $_{lb}$}‚

	  \pstart \leavevmode% starting standard par
	ईदृश‚स्य ताव‚दियं ग‚तिर‚न्य‚स्य तु का वार्त्तेत्याह--\textbf{य‚स्ये}ति । ध‚र्म‚स्येत्य‚नुवृत्तेर्य‚स्येति‚{\tiny $_{lb}$}‚ ध‚र्म‚स्य । तुर्विशिष्टं ध‚र्मं द‚र्श‚य‚ति । \textbf{स च} सोऽपि प्र‚य‚त्नान‚न्त‚र्याख्योऽन्य‚श्चास‚श्च\edtext{}{\lemma{श्च}\Bfootnote{चा}}‚{\tiny $_{lb}$}‚प्र‚य‚त्नान‚न्त‚रीय‚को व‚न‚कुसुमादिर‚पि । \textbf{स प्र‚तिब‚न्ध}स्य साध‚न‚ग‚त‚प्र‚तिब‚द्ध‚त्व‚स्य \textbf{विष‚यो} गोच‚रः ।‚{\tiny $_{lb}$}‚ एत‚देव व्य‚तिरेक‚मुखेण द्र‚ढ‚य‚ति--\textbf{न त्वि}ति । पुन‚र‚र्थे \textbf{तु}श‚ब्दः । कोऽसावीदृश इत्याह--\textbf{य‚थे}ति ।‚{\tiny $_{lb}$}‚ अनित्य‚त्वाख्यः प्र‚य‚त्नान‚न्त‚र्य‚त्वाख्येन प्र‚तिब‚द्ध इति योज्य‚म् । एव‚ञ्चार्थात्प्र‚तिब‚न्ध‚विष‚ये‚{\tiny $_{lb}$}‚ चेत्य‚व\leavevmode\ledsidenote{\textenglish{46a/ms}}तिष्ठ‚ते । एत‚च्चानित्य‚त्व‚स्य त‚द‚स‚त्स्व‚भाव‚त्वेनानिय‚त‚स्व‚भात्व‚म‚नित्य‚त्व‚सामान्या‚{\tiny $_{lb}$}‚भिप्रायेणोक्त‚म् । अन्य‚था घ‚टादिग‚तानित्य‚त्व‚स्य प्र‚य‚त्नान‚न्त‚रीय‚क‚त्व‚म‚न्त‚रेण कुतोऽव‚स्थान‚म् ।‚{\tiny $_{lb}$}‚ येनान्य‚स्व‚भाव‚त‚याऽस्यानिय‚त‚त्वं स्यादिति ।
	\pend% ending standard par
      ‚{\tiny $_{lb}$}‚‚{\tiny $_{lb}$}‚\textsuperscript{\textenglish{113/dm}}‚{\tiny $_{lb}$}‚
	  \bigskip
	  \begingroup
	

	  \pstart \leavevmode% starting standard par
	क‚स्मात् पुनः स्व‚भाव‚प्र‚तिब‚न्धो \edtext{}{\lemma{न्धो}\Bfootnote{लिङ्ग‚स्य न व‚स्तुन इत्याह--\cite{dp-msB} \cite{dp-msD} \cite{dp-edH} लिङ्ग‚स्य साध्येनेत्याह \cite{dp-edN}}}लिङ्ग‚स्येत्याह--
	\pend% ending standard par
       ‚{\tiny $_{lb}$}‚ 
	  \bigskip
	  \begingroup
	

	  \pstart \leavevmode% starting standard par
	\edtext{\textsuperscript{*}}{\lemma{*}\Bfootnote{व‚स्तुनः \cite{dp-edP}}}व‚स्तुत‚स्तादात्म्यात्\edtext{}{\lemma{स्तादात्म्यात्}\Bfootnote{तादात्म्यात्साध्यार्थादुत्प० \cite{dp-msB} \cite{dp-edP} \cite{dp-edH} \cite{dp-edE} \cite{dp-edN}}} त‚दुत्प‚त्तेश्च ॥ २२ ॥
	\pend% ending standard par
      
	  \endgroup
	‚{\tiny $_{lb}$}‚ 

	  \pstart \leavevmode% starting standard par
	व‚स्तुत इत्यादि । स साध्योऽर्थ\edtext{}{\lemma{साध्योऽर्थ}\Bfootnote{स साध्यः स्व‚भावो--\cite{dp-msB} ऽर्थ‚स्व‚भाव आत्मा य‚स्य \cite{dp-msD}}} आत्मा स्व‚भावो य‚स्य त‚त् त‚दात्मा\edtext{}{\lemma{दात्मा}\Bfootnote{त‚दात्म--\cite{dp-msC} \cite{dp-msD}}} । त‚स्य‚{\tiny $_{lb}$}‚ भाव‚स्तादात्म्य‚म्\edtext{}{\lemma{म्}\Bfootnote{तादात्म्यं त‚त्स्व‚भाव‚त्व‚म् त‚स्मा० \cite{dp-msB}}} । त‚स्माद्धेतोः । य‚तः साध्य‚स्व‚भावं साध‚नं त‚स्मात्\edtext{}{\lemma{स्मात्}\Bfootnote{त‚त् त‚त्र \cite{dp-msB} \cite{dp-edN} \cite{dp-edH}}} त‚त्र स्व‚भाव‚प्र‚तिब‚न्ध\edtext{}{\lemma{न्ध}\Bfootnote{प्र‚तिब‚द्ध‚मित्य‚र्थः--\cite{dp-msA} \cite{dp-msB} \cite{dp-edP} \cite{dp-edH} \cite{dp-edE} \cite{dp-edN}}}‚{\tiny $_{lb}$}‚ इत्य‚र्थः ।
	\pend% ending standard par
       ‚{\tiny $_{lb}$}‚ 

	  \pstart \leavevmode% starting standard par
	य‚दि साध्य‚स्व‚भावं साध‚नं साध्य‚साध‚न‚योर‚भेदात् प्र‚तिज्ञार्थैक‚देशो हेतुः स्यादित्या‚{\tiny $_{lb}$}‚ व‚स्तुत इति । प‚र‚मार्थ‚स‚ता रूपेणाऽभेद‚स्त‚योः । विक‚ल्प‚विष‚य‚स्तु य‚त् स‚मारोपितं रूप‚म् ।‚{\tiny $_{lb}$}‚ त‚द‚पेक्षः साध्य‚साध‚न‚भेदः । \edtext{\textsuperscript{*}}{\lemma{*}\Bfootnote{विक‚ल्प०--\cite{dp-msD-n}}}निश्च‚यापेक्ष\edtext{}{\lemma{यापेक्ष}\Bfootnote{निश्च‚यापेक्ष‚या--\cite{dp-msB} \cite{dp-msC} \cite{dp-msD}}} एव हि ग‚म्य‚ग‚म‚क‚भावः । त‚तो निश्च‚यारूढ‚{\tiny $_{lb}$}‚रूपापेक्ष एव त‚योर्भेदो युक्तः वास्त‚व‚स्त्व‚भेद इति । न केव‚लात् \edtext{}{\lemma{लात्}\Bfootnote{केव‚लं तादा० \cite{dp-msA} \cite{dp-msB} \cite{dp-msD} \cite{dp-edP} \cite{dp-edH} \cite{dp-edE} \cite{dp-edN}}}तादात्म्याद‚पि तु त‚तः‚{\tiny $_{lb}$}‚ साध्याद‚र्थाद् उत्प‚त्तिर्लिङ्ग‚स्य--त‚दुत्प‚त्तेश्च साध्येऽर्थे स्व‚भाव‚प्र‚तिब‚न्धो लिङ्ग‚स्य ॥
	\pend% ending standard par
      
	  \endgroup
	‚{\tiny $_{lb}$}‚

	  \pstart \leavevmode% starting standard par
	स्यादेत‚त्तादात्म्यं ताव‚त्त‚योर‚स्ति । त‚त्किं प्र‚तिब‚द्ध‚त्व‚प्र‚तिब‚न्ध‚विष‚य‚त्व‚निश्च‚येन ग‚म्य‚{\tiny $_{lb}$}‚ग‚म‚क‚त्व‚व्य‚व‚स्थानिब‚न्ध‚नीकृतेनेत्याह--\textbf{निश्च‚ये}ति । हीति य‚स्मात् । त‚द‚पेक्षायाम‚पि किमिति‚{\tiny $_{lb}$}‚ प्र‚य‚त्नान‚न्त‚रीय‚क‚त्व‚मेव ग‚म‚क‚मित्याह--\textbf{प्र‚य‚त्नेति । चो} य‚स्माद‚र्थे । य‚त‚स्त‚त्स्व‚भावं निश्चित‚म्‚{\tiny $_{lb}$}‚ अतोऽस्मात्त‚देवानित्य‚त्वे प्र‚तिब‚द्ध‚मुच्य‚त इति शेषः । य‚त‚स्तादात्म्याविशेषेऽपि य‚त्प्र‚तिब‚द्ध‚त‚या‚{\tiny $_{lb}$}‚ निश्चितं त‚देव ग‚म‚क‚मित‚र‚द् ग‚म्यं \textbf{त‚स्मान्निय‚तः} प्र‚तिनिय‚तः प्र‚य‚त्नान्त‚रीय‚क‚मेव ग‚म‚क‚म्,‚{\tiny $_{lb}$}‚ अनित्य‚त्व च ग‚म्य‚मेवेत्येवंरूपो \textbf{विष‚यो} य‚स्य ग‚म्य‚ग‚म‚क‚भाव‚स्य स त‚था । एत‚देव व्य‚तिरेक‚मुखेण‚{\tiny $_{lb}$}‚ द्र‚ढ‚य‚ति \textbf{नान्य‚थे}ति ॥
	\pend% ending standard par
      ‚{\tiny $_{lb}$}‚

	  \pstart \leavevmode% starting standard par
	न‚नु च य एक‚स्यान्य‚त्र प्र‚तिब‚न्ध‚स्त‚दाय‚त्त‚त्वं स ताव‚न्नाहेतुकः । क‚श्चासौ हेतुरित्य‚भि‚{\tiny $_{lb}$}‚प्रेत्य प्र‚श्न‚य‚ति--\textbf{क‚स्मादि}ति । निमित्त‚प्र‚श्न‚श्चैषः । \textbf{तादात्म्यादिति} मौल‚मुत्त‚र व्याख्यातु‚{\tiny $_{lb}$}‚माह--\textbf{य‚त} इति । \textbf{त‚त्र} साध्ये \textbf{स्व‚भावेन प्र‚तिब‚न्धः} प्र‚तिब‚द्ध‚त्वं लिङ्ग‚स्येति शेषः ।
	\pend% ending standard par
      ‚{\tiny $_{lb}$}‚

	  \pstart \leavevmode% starting standard par
	\textbf{प्र‚तिज्ञा} साध्य‚निर्देशः । त‚स्या अर्थो ध‚र्म‚ध‚र्मिस‚मुदायः । अत्र च साध्य‚साध‚न‚योरैका‚{\tiny $_{lb}$}‚त्म्य‚स्य प्र‚स्तुत‚त्वात्साध्य‚ल‚क्ष‚ण‚स्य प्र‚तिज्ञार्थ‚स्य हेतुत्व‚मास‚क्त‚म् । त‚त‚श्च साध्य‚ध‚र्म‚व‚त्साध‚न‚{\tiny $_{lb}$}‚ध‚र्म‚स्याप्य‚सिद्धिः । सिद्धौ वा हेतुवैय‚र्थ्य‚मिति भावः । य‚दि प‚र‚मार्थ‚तोऽभेदः, क‚थं त‚र्हि‚{\tiny $_{lb}$}‚ भेद‚निब‚न्ध‚नो ग‚म्य‚ग‚म‚क‚भाव इत्याह--\textbf{विक‚ल्पे}ति । तुः पार‚मार्थिकाद‚भेदाद् वैध‚र्म्य‚माह ।‚{\tiny $_{lb}$}‚ कोऽसौ विक‚ल्प‚विष‚यः ? य‚दि बाह्य‚स्त‚दा त‚द‚व‚स्थो दोष इत्याह--\textbf{य‚दि}ति । त‚म‚पेक्ष‚त‚{\tiny $_{lb}$}‚ ‚{\tiny $_{lb}$}‚ \leavevmode\ledsidenote{\textenglish{114/dm}}‚{\tiny $_{lb}$}‚ 
	  
	क‚स्मान्निमित्त‚द्व‚यात् स्व‚भाव‚प्र‚तिब‚न्धो लिङ्ग‚स्य नान्य‚स्मादित्याह-- ‚{\tiny $_{lb}$}‚ 
	  
	अत‚त्स्व‚भाव‚स्यात‚दुत्प‚त्तेश्च त‚त्राप्र‚तिब‚द्ध‚स्व‚भाव‚त्वात् ॥ २३ ॥‚{\tiny $_{lb}$}‚ 
	  
	अत‚त्स्व‚भाव‚स्येति । स स्व‚भावोऽस्य\edtext{}{\lemma{भावोऽस्य}\Bfootnote{०भावो य‚स्य सो \cite{dp-msD} ०भावो य‚स्येति सो \cite{dp-msC}}} सोऽयं त‚त्स्व‚भावः । न त‚त्स्व‚भावोऽत‚त्स्व‚भावः ।‚{\tiny $_{lb}$}‚ त‚स्मादुत्प‚त्तिर‚स्य सोऽयं त‚दुत्प‚त्तिः । न त‚थाऽत‚दुत्प‚त्तिः । यो य‚त्स्व‚भावो य‚दुत्प‚त्तिश्च‚{\tiny $_{lb}$}‚ न भ‚व‚ति त‚स्य अत‚त्स्व‚भाव‚स्य, अत‚दुत्प‚त्तेश्च । त‚त्र \edtext{}{\lemma{त्र}\Bfootnote{असाध्य[[ध्येऽ]]कार‚णे च--\cite{dp-msD-n}}}अत‚त्स्व‚भावे अनुत्पाद‚के चाप्र‚तिब‚द्धः‚{\tiny $_{lb}$}‚ स्व‚भावोऽस्येति \edtext{}{\lemma{भावोऽस्येति}\Bfootnote{०ति अप्र‚ति० \cite{dp-msD}}}सोऽय‚म‚प्र‚तिब‚द्ध‚स्व‚भावः । त‚स्य \edtext{}{\lemma{स्य}\Bfootnote{त‚स्य भाव‚स्त‚स्मात्--\cite{dp-msD} \cite{dp-msC}}}भावोऽप्र‚तिब‚द्ध‚स्व‚भाव‚त्व‚म् । त‚स्माद‚प्र‚तिब‚द्ध‚{\tiny $_{lb}$}‚स्व‚भाव‚त्वात् । य‚द्य‚त‚त्स्व‚भावेऽनुत्पाद‚के च क‚श्चित् प्र‚तिब‚द्ध‚स्व‚भावो भ‚वेद्, भ‚वेद‚न्य‚तोऽपि\edtext{}{\lemma{तोऽपि}\Bfootnote{संयोग‚स‚म‚वायादेः--\cite{dp-msD-n}}}‚{\tiny $_{lb}$}‚ निमित्तात्\edtext{}{\lemma{निमित्तात्}\Bfootnote{निमित्त‚त्वात् स्व० \cite{dp-msC}}} स्व‚भाव‚प्र‚तिब‚न्धः । प्र‚तिब‚द्ध‚स्व‚भाव‚त्वं हि स्व‚भाव‚प्र‚तिब‚न्धः । न चान्यः\edtext{}{\lemma{चान्यः}\Bfootnote{संयोग्यादिः--\cite{dp-msD-n}}}‚{\tiny $_{lb}$}‚ क‚श्चिदाय‚त्त‚स्व‚भावः । त‚स्मात् तादात्म्य‚त‚दुत्प‚त्तिभ्यामेव स्व‚भाव‚प्र‚तिब‚न्धः ॥‚{\tiny $_{lb}$}‚ इति \textbf{त‚द‚पेक्षः} । इदं साध‚न‚मिदं साध्य‚मिति \textbf{साध्य‚साध‚न}रूपो \textbf{भेदो} नानात्व‚मित्य‚र्थः । य‚दि‚{\tiny $_{lb}$}‚ नाम क‚ल्प‚नानिर्मितो भेद‚स्त‚थापि क‚थं ग‚म्य‚ग‚म‚क‚भाव इत्याह--\textbf{निश्च‚ये}ति । \textbf{हि}र्य‚स्मात् ।‚{\tiny $_{lb}$}‚ \textbf{निश्च‚यापेक्ष} इति निश्च‚य‚विष‚यीकृत‚रूपापेक्ष इत्य‚र्थः । य‚त एवं त‚त‚स्त‚स्मात्त‚योः साध्य‚साध‚न‚यो‚{\tiny $_{lb}$}‚र्भेदो नानात्वं युक्त्या स‚ङ्ग‚तो \textbf{युक्तः} । व‚स्तुनोऽकृत्रिमाद् रूपादाग‚तो \textbf{वास्त‚वः} ।
	\pend% ending standard par
      ‚{\tiny $_{lb}$}‚

	  \pstart \leavevmode% starting standard par
	स्यान्म‚त‚म्--भेदेन क‚ल्पित‚योर्न तादात्म्यं ग‚म्य‚ग‚म‚क‚भाव‚निब‚न्ध‚न‚म‚स्ति । वास्त‚वेन‚{\tiny $_{lb}$}‚ च रूपेणैक‚त्वान्न ग‚म्य‚ग‚म‚क‚भाव इति क‚थं स्व‚भाव‚हेतोर्ग‚म‚क‚त्व‚म् ? नाहेतुत्व‚म् । य‚द्द‚र्श‚न‚द्वारा‚{\tiny $_{lb}$}‚यातावेतौ ध‚र्मौ त‚थाप्र‚तीय‚मानौ त‚त् ताव‚त्प‚र‚मार्थ‚त‚स्त‚दात्म‚क‚मित्येक‚स्य ध‚र्मान्त‚रा‚{\tiny $_{lb}$}‚व्य‚भिचारः । वास्त‚वं तादात्म्य‚ग‚तं च य‚स्य ग‚म‚क‚त्वं स स्व‚भाव‚हेतुरुच्य‚त इति को विरोधः ?
	\pend% ending standard par
      ‚{\tiny $_{lb}$}‚

	  \pstart \leavevmode% starting standard par
	अयं प्र‚क‚र‚णार्थः--न निश्च‚य‚स्थे स‚मारोपिते रूपे स‚मारोपित‚त्वेनाध्य‚व‚सीय‚माने ग‚म्य‚{\tiny $_{lb}$}‚ग‚म‚के किन्तु स्व‚ल‚क्ष‚ण‚त्वेनाध्य‚व‚सीय‚माने । त‚त्र तादात्म्य‚म‚स्ति । एत‚दुक्तं भ‚व‚ति--आरोप्य‚माणं‚{\tiny $_{lb}$}‚ रूप‚मारोपित‚भेद\leavevmode\ledsidenote{\textenglish{46b/ms}}म् । आरोपित‚स‚दृशं च स्व‚ल‚क्ष‚ण‚म् । तेनारोपितेन रूपेणानुग‚म्य‚मानं‚{\tiny $_{lb}$}‚ भिन्न‚म‚घ्य‚व‚सीय‚ते । त‚द‚व‚ध्य‚व‚सित‚भेद‚निब‚न्ध‚नो ग‚म्य‚ग‚म‚क‚भाव‚स्त‚स्य स्व‚त‚श्च तादात्म्य‚मिति ।
	\pend% ending standard par
      ‚{\tiny $_{lb}$}‚

	  \pstart \leavevmode% starting standard par
	द्वितीयं प्र‚तिब‚न्ध‚कार‚णं व्याख्यातुमाह--\textbf{न केव‚ला}दिति । \textbf{चः} साधार‚णं निमित्तं‚{\tiny $_{lb}$}‚ स‚मुच्चिनोति ॥
	\pend% ending standard par
      ‚{\tiny $_{lb}$}‚

	  \pstart \leavevmode% starting standard par
	न‚नु च स‚म‚वायादितोऽपि निमित्तात्प्र‚तिब‚न्धो नास‚म्भ‚वी । त‚त्क‚थं तादात्म्यात्त‚दुत्प‚त्तेरेव‚{\tiny $_{lb}$}‚ च स उच्य‚ते इत्य‚भिप्राय‚वान् पृच्छ‚ति--\textbf{क‚स्मादि}ति । अन्य‚निमित्त‚स्यान‚भिधाना\textbf{न्निमित्त‚{\tiny $_{lb}$}‚द्व‚यादि}त्याह ।
	\pend% ending standard par
      ‚{\tiny $_{lb}$}‚

	  \pstart \leavevmode% starting standard par
	अत‚त्स्व‚भाव‚स्येत्यादि ब्रुव‚त‚श्चाचार्य‚स्याय‚माश‚यः--भ‚वेदेवान्य‚तः स‚म्ब‚न्धात्प्र‚तिब‚न्धो‚{\tiny $_{lb}$}‚ ‚{\tiny $_{lb}$}‚ \leavevmode\ledsidenote{\textenglish{115/dm}}‚{\tiny $_{lb}$}‚ य‚दि स‚म‚वायादिर‚न्यः स‚म्ब‚न्धः प्र‚माण‚बाधितो न भ‚वेत् । न चासौ न बाध्य‚ते । त‚त्कुतो‚{\tiny $_{lb}$}‚ऽसाव‚स‚न्न‚स्य निमित्तं भ‚वेदिति ।
	\pend% ending standard par
      ‚{\tiny $_{lb}$}‚

	  \pstart \leavevmode% starting standard par
	स‚मुदायार्थं व्याच‚ष्टे--यो य‚त्स्व‚भाव इति । अप्र‚तिब‚द्ध‚स्व‚भाव‚त्वादिति मूल‚स्य भाव‚{\tiny $_{lb}$}‚प्र‚त्य‚यं त्य‚क्त्वा विग्र‚ह‚माह--\textbf{अप्र‚तिब‚द्ध} इति । \textbf{त‚स्य भाव‚स्}त‚त्त्व‚म् । \textbf{त‚स्मादि}ति तु योज्य‚म् ।‚{\tiny $_{lb}$}‚ अमुमेवार्थं \textbf{य‚दी}त्यादिना स्फुट‚य‚ति । क‚स्मात्पुन‚र‚न्य‚तो निमित्तान्न भ‚व‚तीत्याह--\textbf{प्र‚तिब}द्धेति ।‚{\tiny $_{lb}$}‚ हिर्य‚स्मात् । प्र‚तिब‚द्ध‚स्व‚भाव‚त्व‚मेवान्य‚स्यान्य‚त्र भ‚विष्य‚तीत्याह--\textbf{न चे}ति । \textbf{चो}ऽव‚धार‚णे हेतौ‚{\tiny $_{lb}$}‚ वा । अन्य‚स्य स‚म्ब‚न्ध‚स्याभावात्तादात्म्य‚त‚दुत्प‚त्त्य‚भावे चेति प्र‚क‚र‚ण‚ल‚भ्यं कृत्वा \textbf{न चान्यः‚{\tiny $_{lb}$}‚ क‚श्चिदाय‚त्त‚स्व‚भाव} इत्युक्त‚म् ।
	\pend% ending standard par
      ‚{\tiny $_{lb}$}‚

	  \pstart \leavevmode% starting standard par
	न‚नु चास‚त्य‚पि तादात्म्ये त‚दुत्प‚त्तौ चान्य‚त्रास्व‚भावेऽनुत्पाद‚के चान्य‚त्प्र‚तिब‚द्धं य‚था—‚{\tiny $_{lb}$}‚आत‚पो वृक्ष‚च्छायायाम् । तुलाया अर्वाग्भाग‚न‚म‚नाव‚न‚म‚ने प‚र‚भागोन्न‚म‚नाव‚न‚म‚न‚योः । अर्वाग्भागः‚{\tiny $_{lb}$}‚ प‚र‚भागे । र‚सो रूपे । पाणिः पाद‚योः । अप‚त‚ज्ज‚ल‚माधारे । ब‚लाका स‚लिले । न‚दीपूर‚{\tiny $_{lb}$}‚ उप‚रिवृत्तायां वृष्टौ । च‚न्द्रोद‚यः स‚मुद्र‚वृद्धौ कुमुद‚विकासे च । कृत्तिकोद‚यो रोहिण्युद‚ये ।‚{\tiny $_{lb}$}‚ पिपिलिकोत्स‚र‚णं म‚त्स्य‚विकार‚श्च वृष्टौ । श‚र‚दि ज‚ल‚प्र‚सादोऽग‚स्त्योद‚ये । विशिष्टो मेघो‚{\tiny $_{lb}$}‚\add{द}यो व‚र्ष‚क‚र्म‚णि । अद्यादित्योद‚योऽस्त‚म‚ये श्व‚स्त‚नोद‚ये च । कुष्माण्ड‚गुड‚कोऽन्तःस्थित‚बीजे ।‚{\tiny $_{lb}$}‚ प‚रिव्राज‚को द‚ण्डे । स‚न्त्र‚स्तो न‚कुलः स‚र्पे । किय‚द् वा श‚क्य‚ते निद‚र्श‚यितुम् ? एताव‚त्तूच्य‚ते—‚{\tiny $_{lb}$}‚य‚द् येनाविनाभूतं दृश्य‚ते त‚त्त‚त्र प्र‚तिब‚द्ध‚म् । त‚स्य च लिङ्ग‚म् । अत एव त्रीण्येव लिङ्गानीति‚{\tiny $_{lb}$}‚ संख्यानिय‚मोऽप्य‚युक्तः । केव‚लं लिङ्ग‚स्य रूपाण्येव व‚क्त‚व्यानि । य‚द्द‚र्श‚नात् हेतुत्व‚म‚व‚सीय‚त इति ।
	\pend% ending standard par
      ‚{\tiny $_{lb}$}‚

	  \pstart \leavevmode% starting standard par
	नैष दोषः । अमीषां म‚ध्ये येषां प्र‚तिब‚न्धोऽस्ति तेषां तादात्म्य‚त‚दुत्प‚त्त्योर‚न्य‚त‚र‚{\tiny $_{lb}$}‚स‚म्भ‚वाद्, येषु च त‚द‚भाव‚स्तेषाम‚प्र‚तिब‚न्धाद‚ग‚म‚क‚त्वात् । त‚थाहि वृक्ष‚स्य छायायामेक‚साम‚ग्र्य‚{\tiny $_{lb}$}‚धीन‚त‚यैव प्र‚तिब‚न्धः । त‚त‚स्त‚त्प्र‚तिप‚त्तिः कार्य‚लिङ्ग‚जैव । छाया हि प्र‚तिभास‚मान‚रूप‚संस्थान‚{\tiny $_{lb}$}‚व‚ती शैत्याद्य‚र्थ‚क्रियाकारिणी व‚स्त्वेव, न त्वेवालोकाभावः । सा च पूर्व‚स्मादालोकोपादानात्‚{\tiny $_{lb}$}‚ पूर्व‚वृक्ष‚क्ष‚णाद् वृक्ष‚क्ष‚णेन सार्ध‚मुत्प‚द्य‚ते । त‚थाऽर्वाग्भाग‚न‚म‚नाव‚न‚म‚ने अपि तुलायाः प‚र‚भागो‚{\tiny $_{lb}$}‚न्न‚म‚नाव‚न‚म‚नाभ्यामेव स‚मं पुरुष‚प्र‚य‚त्नादेव त‚थाविधात्त‚दुपादान‚स‚ह\leavevmode\ledsidenote{\textenglish{47a/ms}}कारिण उत्प‚द्येते ।‚{\tiny $_{lb}$}‚ त‚थाऽर्वाग्भाग‚प‚र‚भाग‚योर‚स‚रूप‚योर‚प्येक‚साम‚ग्र्य‚धीन‚तैव । पाणिस्त्व‚प्र‚तिब‚द्ध एव, व्य‚ङ्ग‚स्यापि‚{\tiny $_{lb}$}‚ स‚म्भ‚वात् । अव्य‚भिचारे चैक‚संस‚र्गाधीन‚तैव निब‚न्ध‚न‚म् । तादृशं च ज‚ल‚माधार‚स्य कार्य‚मेव,‚{\tiny $_{lb}$}‚ तादृशी च ब‚लाका स‚लिल‚स्य । न‚दीपूरोऽपि त‚थाविध उप‚रिवृष्टेः । दृश्यादृश्य‚स‚मुदाय‚श्च‚{\tiny $_{lb}$}‚ य‚थायोगं स‚र्व‚त्र ध‚र्मी क‚र्त्त‚व्यः । न‚दीपूरे चान्योऽपि प्र‚कारो व‚क्तुं श‚क्यः । न‚दी‚{\tiny $_{lb}$}‚ ध‚र्मिणी । उप‚रिवृष्टिम‚द्देश‚स‚ब‚न्धित्व‚म‚स्याः साध्य‚म् । त‚थाविध‚पूर‚त्व‚मात्रं हेतुः । एवं च‚न्द्रो‚{\tiny $_{lb}$}‚द‚य‚स‚मुद्र‚वृद्धिकुमुद‚विकाशा\edtext{}{\lemma{विकाशा}\Bfootnote{सा}}नाम‚प्येक‚साम‚ग्र्य‚धीन‚तैव । एक‚स्मादेव म‚हाभूत‚विशेषात्‚{\tiny $_{lb}$}‚ काल‚व्य‚व‚हार‚विष‚यादेत‚दुत्पादापेक्षिण‚स्तेषामुत्पादात् । त‚था य एव कृत्तिकोद‚य‚हेतु‚{\tiny $_{lb}$}‚र्म‚हाभूत‚विशेषः काल‚संज्ञितः स एव क‚तिप‚य‚काल‚व्य‚व‚धानेन रोहिण्युद‚य‚हेतुरिति त‚द्द‚र्श‚नाद्‚{\tiny $_{lb}$}‚ हेतोस्त‚ज्ज‚न‚न‚योग्य‚ताध‚र्मोऽनुमीय‚ते एव । त‚था पिपीलिकोत्स‚र‚ण‚स्य म‚त्स्य‚विकार‚स्य च‚{\tiny $_{lb}$}‚ यो हेतुः स एव क‚तिप‚य‚काल‚व्य‚व‚धानेन व‚र्ष‚क‚र‚ण‚योग्य‚स्त‚तः पूर्व‚व‚द् हेतुध‚र्मानुमान‚म्, रूप‚र‚स‚यो‚{\tiny $_{lb}$}‚रिवैक‚साम‚ग्र्य‚धीन‚त‚यैव वा त‚त्स‚म‚कालिक‚व‚र्ष‚णानुमान‚म् । त‚दा त्व‚श्र‚व‚णीय‚ब‚हिःस्थित‚श‚ब्द‚ग‚र्भ‚{\tiny $_{lb}$}‚गृहादिव्य‚व‚स्थितोऽनुमाता प्र‚त्येत‚व्यः । त‚था श‚र‚दादिज‚ल‚प्र‚सादोऽग‚स्त्योद‚य‚स्य कार्य‚मेव । अथ‚{\tiny $_{lb}$}‚ ज‚ल‚प्र‚सादं दृष्ट्वोदेष्य‚तीत्य‚नुमीय‚ते, त‚दा त‚स्मादेव म‚हाभूतात्काल‚संज्ञिताज्ज‚ल‚प्र‚सादः । स \leavevmode\ledsidenote{\textenglish{116/dm}}‚{\tiny $_{lb}$}‚ 
	  
	भ‚व‚तु नाम तादात्म्य‚त‚दुत्प‚त्तिभ्यामेव स्व‚भाव‚प्र‚तिब‚न्धः । कार्य‚स्व‚भाव‚योरेव तु‚{\tiny $_{lb}$}‚ग‚म‚क‚त्वं क‚थ‚मित्याह-- ‚{\tiny $_{lb}$}‚ 
	  
	ते च तादात्म्य‚त‚दुत्प‚त्ती स्व‚भाव‚कार्य‚योरेवेति ताभ्यामेव व‚स्तुसिद्धिः ॥ २४ ॥‚{\tiny $_{lb}$}‚ 
	  
	ते चेति । इतिः त‚स्माद‚र्थे । य‚स्मात् स्व‚भावे कार्ये एव च तादात्म्य‚त‚दुत्प‚त्ती स्थिते,‚{\tiny $_{lb}$}‚ त‚न्निब‚न्ध‚न‚श्च ग‚म्य‚ग‚म‚क‚भाव‚स्त‚स्मात् ताभ्यामेव कार्य‚स्व‚भावाभ्यां व‚स्तुनो विधेः\edtext{}{\lemma{विधेः}\Bfootnote{साध्य‚स्य--\cite{dp-msD-n}}} सिद्धिः ॥ ‚{\tiny $_{lb}$}‚ 
	  
	अथ प्र‚तिषेध‚सिद्धिर‚दृश्यानुप‚ल‚म्भाद‚पि क‚स्मान्नेष्टेत्याह-- ‚{\tiny $_{lb}$}‚ 
	  
	प्र‚तिषेध‚सिद्धिर‚पि\edtext{}{\lemma{पि}\Bfootnote{०सिद्धिर्य‚थोक्ता० \cite{dp-edE}}} य‚थोक्ताया एवानुप‚ल‚ब्धेः ॥ २५ ॥‚{\tiny $_{lb}$}‚ 
	  
	प्र‚तिषेध‚व्य‚व‚हार‚स्य सिद्धिर्य‚थोक्ता या दृश्यानुप‚ल‚ब्धिस्त‚त एव भ‚व‚ति य‚त‚स्त‚स्मा‚{\tiny $_{lb}$}‚द‚न्य‚तो\edtext{}{\lemma{तो}\Bfootnote{अदृश्यानुप‚ल‚ब्धेः--\cite{dp-msD-n}}} नोक्ता ॥ ‚{\tiny $_{lb}$}‚ 
	  
	त‚त‚स्ताव‚त् क‚स्माद्भ‚व‚तीत्याह-- ‚{\tiny $_{lb}$}‚ 
	  
	स‚ति व‚स्तुनि त‚स्या\edtext{}{\lemma{स्या}\Bfootnote{त‚स्यास‚म्भ‚वात् \cite{dp-msC}}} अस‚म्भ‚वात् ॥ २६ ॥‚{\tiny $_{lb}$}‚ 
	  
	स‚ति त‚स्मिन् प्र‚तिषेध्ये व‚स्तुनि, य‚स्माद् दृश्यानुप‚ल‚ब्धिर्न स‚म्भ‚व‚ति त‚स्माद्--अस‚म्भ‚वात्‚{\tiny $_{lb}$}‚ त‚तः प्र‚तिषेध‚सिद्धिः ॥‚{\tiny $_{lb}$}‚ च क‚तिप‚य‚काल‚व्य‚व‚धानेन त‚दुद‚य‚निमित्त‚मिति पूर्व‚व‚द् हेतुध‚र्मानुमान‚म् । मेघ‚स्यापि त‚थाविध‚{\tiny $_{lb}$}‚स्यात्य‚न्तायोग्य‚ताव्यावृत्त्या वृष्टिक‚र‚ण‚योग्य‚ताऽनुमेया । न तु भाविव‚र्षं व्य‚भिचार‚स‚म्भ‚वात् ।‚{\tiny $_{lb}$}‚ सा च स्व‚भाव‚भूतैवानुमीय‚त इति तादात्म्य‚मेव निब‚न्ध‚न‚म् । आदित्योद‚य‚स्य तु प्र‚भावातिश‚य‚{\tiny $_{lb}$}‚व‚ता योग्यादिना विब‚न्ध‚स‚म्भ‚वात् नास्त्येवाविनाभावः । अन्य‚थाद्य ग‚र्द‚भ‚द‚र्श‚न‚स्याप्य‚स्त‚म‚य‚श्व‚{\tiny $_{lb}$}‚स्त‚नोद‚य‚योस्त‚थात्वं स्यात् । एवं तु युक्त‚म्--अय‚मुदेता अस्त‚म‚य‚श्व‚स्त‚नोद‚य‚योग्य इति । त‚था‚{\tiny $_{lb}$}‚ चोद‚य‚त‚थाविध‚योग्य‚त‚योस्तादात्म्य‚मेव निब‚न्ध‚न‚म् । कुष्माण्ड‚स्यापि बीजेनैक‚साम‚ग्र्य‚धीन‚तैव ।‚{\tiny $_{lb}$}‚ प‚रिव्राज‚क‚न‚कुलौ द‚ण्ड‚स‚र्प‚योर‚प्र‚तिब‚द्धावेव । अन्य‚थापि स‚म्भ‚वात् । किय‚द् वा श‚क्य‚ते प‚रि‚{\tiny $_{lb}$}‚ह‚र्त्तुम् ? एताव‚दुच्य‚ते--अस‚ति तादात्म्ये त‚दुत्प‚त्तौ वा क‚स्य‚चित्क्व‚चित्प्र‚तिब‚न्धे ताद्रूप्येण च‚{\tiny $_{lb}$}‚ ग‚म‚क‚त्वे स‚र्वं स‚र्व‚त्र प्र‚तिब‚द्धं त‚द्ग‚म‚कं प्र‚स‚ज्येतेति ॥
	\pend% ending standard par
      ‚{\tiny $_{lb}$}‚

	  \pstart \leavevmode% starting standard par
	\textbf{कार्य‚स्व‚भाव‚योरेव तु ग‚म‚क‚त्वं क‚थ‚मि}ति ब्रुव‚तः पूर्व‚प‚क्ष‚वादिनोऽय‚माश‚यः--तादात्म्य‚{\tiny $_{lb}$}‚त‚दुप‚त्ती एवान्य‚स्य भ‚विष्य‚तः, त‚त‚श्च ग‚म‚क‚त्व‚मिति । कार्य‚स्व‚भाव‚योरिति द्व‚यो\add{रु}च्चार‚णे‚{\tiny $_{lb}$}‚ चायं त‚स्य भावः--भ‚व‚द्भिरेवानुप‚ल‚म्भोऽन‚योर‚न्त‚र्भावित इति ॥
	\pend% ending standard par
      ‚{\tiny $_{lb}$}‚

	  \pstart \leavevmode% starting standard par
	\edtext{\textsuperscript{*}}{\lemma{*}\Bfootnote{अस्प‚ष्ट‚म्--सं०}}\add{... ... ...}नुप‚ल‚ब्धिरित्युक्ते कुतोस्य पूर्व‚प‚क्ष‚स्योत्थान‚म् ?‚{\tiny $_{lb}$}‚ स‚त्य‚मेत‚त्, केव‚लं त‚देवा\edtext{}{\lemma{देवा}\Bfootnote{अस्प‚ष्ट‚म्--सं०}}\add{... ... ...}मित्य‚स्या अप्य‚नुप‚ल\leavevmode\ledsidenote{\textenglish{47b/ms}}ब्धेः‚{\tiny $_{lb}$}‚ ‚{\tiny $_{lb}$}‚ \leavevmode\ledsidenote{\textenglish{117/dm}}‚{\tiny $_{lb}$}‚ 
	  
	अथ त‚त एव क‚स्मादित्याह-- ‚{\tiny $_{lb}$}‚ 
	  
	अन्य‚था चानुप‚ल‚ब्धिल‚क्ष‚ण‚प्राप्तेषु देश‚काल‚स्व‚भाव‚वि\edtext{}{\lemma{वि}\Bfootnote{०प्र‚कृष्टेष्वात्म‚प्र०--\cite{dp-msB} \cite{dp-edP} \cite{dp-edH} \cite{dp-edE} \cite{dp-edN}}}प्र‚कृष्टेष्व‚र्थेष्वात्म‚प्र‚त्य‚क्ष‚{\tiny $_{lb}$}‚निवृत्तेर‚भाव‚निश्च‚याभावात् ॥ २७ ॥‚{\tiny $_{lb}$}‚ 
	  
	अन्य‚था चेति । स‚ति व‚स्तुनि त‚स्या अदृश्यानुप‚ल‚ब्धेः स‚म्भ‚वादित्य‚न्य‚थाश‚ब्दार्थः ।‚{\tiny $_{lb}$}‚ एत‚स्मात् कार‚णात् नान्य‚स्या\edtext{}{\lemma{स्या}\Bfootnote{अदृश्यानुप‚ल‚ब्धेः--\cite{dp-msD-n}}} अनुप‚ल‚ब्धेः प्र‚तिषेध‚सिद्धिः । ‚{\tiny $_{lb}$}‚ 
	  
	कुत एत‚त्--स‚त्य‚पि व‚स्तुनि त‚स्याः स‚म्भ‚व इत्याह--अनुप‚ल‚ब्धिल‚क्ष‚ण‚प्राप्तेष्वित्यादि ।‚{\tiny $_{lb}$}‚ इह प्र‚त्य‚यान्त‚र‚साक‚ल्यात् स्व‚भाव‚विशेषाच्चोप‚ल‚ब्धिल‚क्ष‚ण‚प्राप्तोऽर्थ\edtext{}{\lemma{प्राप्तोऽर्थ}\Bfootnote{प्राप्तार्थ उक्तः \cite{dp-msA} \cite{dp-msB} \cite{dp-edP} \cite{dp-edH} प्राप्तेऽर्थे \cite{dp-msC}}} उक्तः । \edtext{\textsuperscript{*}}{\lemma{*}\Bfootnote{द्व‚योरेक‚स्याप्य० \cite{dp-msB} द्व‚योरेकैक‚स्याभावे--\cite{dp-msC}}}द्व‚योरेकैक‚{\tiny $_{lb}$}‚स्याप्य‚भावेऽनुप‚ल‚ब्धिल‚क्ष‚ण‚प्राप्तोऽर्थ\edtext{}{\lemma{प्राप्तोऽर्थ}\Bfootnote{प्राप्तेऽर्थे उच्य‚ते--\cite{dp-msC}}} उच्य‚ते । ‚{\tiny $_{lb}$}‚ 
	  
	त‚दिहानुप‚ल‚ब्धिल‚क्ष‚ण‚प्राप्तेष्विति प्र‚त्य‚यान्त‚र‚वैक‚ल्य‚व‚न्त उक्ताः । देश‚काल‚स्व‚भाव‚{\tiny $_{lb}$}‚विप्र‚कृष्टेष्विति \edtext{}{\lemma{कृष्टेष्विति}\Bfootnote{स्व‚भाव‚विशेष‚विप्र‚कृष्टाः \cite{dp-msA} \cite{dp-edP} \cite{dp-edH} \cite{dp-edE} \cite{dp-edN} स्व‚भाव‚विप्र‚कृष्टा--\cite{dp-msC}}}स्व‚भाव‚विशेष‚र‚हिता उक्ताः । देश‚श्च काल‚श्च स्व‚भाव‚श्च तैर्विप्र‚कृष्टा‚{\tiny $_{lb}$}‚ इति विग्र‚हः । तेष्व‚भाव‚निश्च‚य‚स्याभावात् । स‚त्य‚पि व‚स्तुनि \edtext{}{\lemma{स्तुनि}\Bfootnote{अदृश्यानुप‚ल‚ब्धेः--\cite{dp-msD-n} । त‚स्याभावः \cite{dp-edP} \cite{dp-edH} \cite{dp-edE} \cite{dp-edN}}}त‚स्या भाव इष्टः । ‚{\tiny $_{lb}$}‚ 
	  
	क‚स्मान्निश्च‚याभाव इत्याह--तेषु प्र‚तिप‚त्तुरात्म‚नो य‚त् प्र‚त्य‚क्षं त‚स्य निवृत्तेः कार‚णात्‚{\tiny $_{lb}$}‚ निश्च‚याभावः । य‚स्माद‚नुप‚ल‚ब्धिल‚क्ष‚ण‚प्राप्तेष्वात्म‚प्र‚त्य‚क्ष‚निवृत्तेर‚भाव‚निश्च‚याभावः, त‚स्मात्‚{\tiny $_{lb}$}‚ स‚त्य‚पि व‚स्तुनि आत्म‚प्र‚त्य‚क्ष‚निवृत्तिल‚क्ष‚णाया अदृश्यानुप‚ल‚ब्धेः स‚म्भ‚वः । त‚तो य‚थोक्ताया‚{\tiny $_{lb}$}‚ एव प्र‚तिषेध‚सिद्धिः ॥‚{\tiny $_{lb}$}‚ प्र‚तिषेध‚स‚म्भ‚वादित्य‚भिप्रायेण पूर्व‚प‚क्ष‚प्र‚वृत्तेर‚दोष एषः । \textbf{प्र‚तिषे}ध‚श‚ब्देन व्य‚व‚हारोऽभिप्रेत इति‚{\tiny $_{lb}$}‚ \textbf{प्र‚तिषेध‚व्य‚व‚हार} इति विवृत‚म् । मूले त्व‚पिश‚ब्दः साध्यान्त‚र‚स‚मुच्च‚ये । काऽसौ य‚थोक्ते‚{\tiny $_{lb}$}‚त्याह--येति ।
	\pend% ending standard par
      ‚{\tiny $_{lb}$}‚

	  \pstart \leavevmode% starting standard par
	मूल‚साम‚र्थ्य‚स्थित‚म‚भिव्य‚न‚वित--\textbf{त‚स्माद‚न्य‚तो}ऽदृश्यानुप‚ल‚ब्धे\textbf{र्नोव‚ता प्र‚तिषेध‚सिद्धिरि}ति‚{\tiny $_{lb}$}‚ प्र‚कृतेन स‚म्ब‚न्धः ॥
	\pend% ending standard par
      ‚{\tiny $_{lb}$}‚

	  \pstart \leavevmode% starting standard par
	य‚द्य‚दृश्यानुप‚ल‚ब्धेर्न भ‚व‚ति त‚दाऽनुप‚ल‚ब्धित्वाविशेषे विव‚क्षिताया अपि मा भूदित्य‚भि‚{\tiny $_{lb}$}‚प्रेत्याह--\textbf{त‚त‚स्ताव}दिति । त‚स्माद‚स‚म्भ‚वात् कार‚णा\textbf{त‚त} इति त‚त एव दृश्यानुप‚ल‚ब्धेरिति‚{\tiny $_{lb}$}‚ विव‚क्षित‚मित‚र‚थाऽन्य‚स्या अपि प्र‚तिषेध‚सिद्धिक‚थ‚न‚प्र‚स‚ङ्गात् ॥
	\pend% ending standard par
      ‚{\tiny $_{lb}$}‚

	  \pstart \leavevmode% starting standard par
	न‚नु त‚तोऽप्य‚स्त्य‚न्य‚तोऽपि । क‚थं पुन‚स्त‚त एवेति निय‚मो ल‚भ्य‚ते इत्य‚भिप्राय‚वानाह—‚{\tiny $_{lb}$}‚अथेति । \textbf{अथ}श‚ब्दः प्र‚श्ने । \textbf{अन्य‚था चे}त्युत्त‚रं व्याच‚क्षाण इहैव‚च्छेदं द‚र्श‚य‚ति च‚श‚ब्द‚ञ्च‚{\tiny $_{lb}$}‚ य‚स्माद‚र्थे । दृश्यानुप‚ल‚ब्धेरुक्तात्प्र‚काराद‚दृश्यानुप‚ल‚ब्धेर‚न्य‚प्र‚कार‚त्व‚म‚न्य‚थात्वं विव‚क्षित‚माचार्य‚{\tiny $_{lb}$}‚‚{\tiny $_{lb}$}‚ ‚{\tiny $_{lb}$}‚ \leavevmode\ledsidenote{\textenglish{118/dm}}‚{\tiny $_{lb}$}‚ 
	  
	अथेयं दृश्यानुप‚ल‚ब्धिः क‚स्मिन् काले प्र‚माण‚म्, किंस्व‚भावा, किंव्यापारा चेत्याह-- ‚{\tiny $_{lb}$}‚ 
	  
	अमूढ‚स्मृतिसंस्कार‚स्यातीत‚स्य व‚र्त्त‚मान‚स्य च प्र‚तिप‚त्तृप्र‚त्य‚क्ष‚स्य \edtext{}{\lemma{स्य}\Bfootnote{निवृत्तिर‚नुप‚ल‚ब्धिर‚भाव० \cite{dp-msC} \cite{dp-msD}}}निवृत्ति‚{\tiny $_{lb}$}‚र‚भाव‚व्य‚व‚हार‚प्र‚व‚र्त्त‚नी\edtext{}{\lemma{नी}\Bfootnote{०हार‚साध‚नी \cite{dp-msB} \cite{dp-msC} \cite{dp-msD} \cite{dp-edP} \cite{dp-edH} \cite{dp-edE} \cite{dp-edN}}} ॥ २८ ॥‚{\tiny $_{lb}$}‚ 
	  
	\edtext{\textsuperscript{*}}{\lemma{*}\Bfootnote{अमूढेति \cite{dp-msA} \cite{dp-edP} \cite{dp-edE} नास्ति \cite{dp-edH} \cite{dp-edN}}}अमूढेत्यादि । प्र‚तिप‚त्तुः प्र‚त्य‚क्षो घ‚टादिर‚र्थः, त‚स्य निवृत्तिर‚नुप‚ल‚ब्धिः त‚द्भाव‚स्य‚{\tiny $_{lb}$}‚भावेति याव‚त् । अत एवाभावो न साध्यः स्व‚भावानुप‚ल‚ब्धेः, सिद्ध‚त्वात् । \edtext{\textsuperscript{*}}{\lemma{*}\Bfootnote{य‚था च प्र‚तिप‚त्तृप्र‚त्य‚क्ष‚निवृत्तिर‚नुप‚ल‚ब्धिः प्र‚देश‚स्त‚ज्ज्ञानं चोच्य‚ते त‚था अविद्य‚मानो‚{\tiny $_{lb}$}‚ऽपीत्यादिना द‚र्श‚य‚ति--\cite{dp-msD-n}}}अविद्य‚मानोऽपि‚{\tiny $_{lb}$}‚ \edtext{\textsuperscript{*}}{\lemma{*}\Bfootnote{०मानोपि घ‚टा० \cite{dp-msC}}}च घ‚टादिरेक‚ज्ञान‚संस‚र्गिणि \edtext{}{\lemma{र्गिणि}\Bfootnote{०र्गिर्णि भास० \cite{dp-msA} \cite{dp-edE} \cite{dp-edP}}}भूत‚ले भास‚माने स‚म‚ग्र‚साम‚ग्रीको ज्ञाय‚मानो \edtext{}{\lemma{मानो}\Bfootnote{दृश्य‚मान‚त‚या \cite{dp-msB} \cite{dp-msD} \cite{dp-edH} \cite{dp-edN}}}दृश्त‚या स‚म्भावित‚{\tiny $_{lb}$}‚त्वात् प्र‚त्य‚क्ष उक्तः । अत एक‚ज्ञान‚संस‚र्गो\edtext{}{\lemma{र्गो}\Bfootnote{संस‚र्गात् \cite{dp-msB}}} दृश्य‚मानोऽर्थ‚स्त‚ज्ज्ञानं च प्र‚त्य‚क्ष‚निवृत्तिरुच्य‚ते ।‚{\tiny $_{lb}$}‚ त‚तो हि दृश्य‚मानाद‚र्थात् त‚द्बुद्धेश्च स‚म‚ग्र‚द‚र्श‚न‚साम‚ग्रीक‚त्वेन प्र‚त्य‚क्ष‚त‚या स‚म्भावित‚स्य निवृत्ति‚{\tiny $_{lb}$}‚र‚व‚सीय‚ते । त‚स्माद‚र्थ‚ज्ञाने एव प्र‚त्य‚क्ष‚स्य घ‚ट‚स्याभाव उच्य‚ते । न तु निवृत्तिमात्र‚{\tiny $_{lb}$}‚मिहाभावः, निवृत्तिमात्राद् दृश्य‚निवृत्त्य‚निश्च‚यात् ।‚{\tiny $_{lb}$}‚ स्येति द‚र्श‚य‚न्नाह--\textbf{स‚तीति । एत‚स्मा}त्स‚ति व‚स्तुनि त‚त्स‚म्भ‚वात् । मूले त्व\textbf{न्य‚था चानुप‚ल‚ब्धि‚{\tiny $_{lb}$}‚ल‚क्ष‚ण‚प्राप्ते}ष्वित्येक‚वाक्य‚त‚यैवार्थः स‚ङ्ग‚च्छ‚ते । ना य‚श‚ब्दार्थ‚व्याख्या, नाप्युत्त‚र‚प‚द‚व्याख्याने पूर्व‚{\tiny $_{lb}$}‚प‚क्ष‚व‚च‚ना\edtext{}{\lemma{ना}\Bfootnote{न}}प्र‚यासः क‚श्चित् । त‚था तु न प्र‚क्रान्तं \textbf{ध‚र्मोत्त‚रेणेति} किम‚त्र कुर्मः ? \textbf{कुत एत‚दिति}‚{\tiny $_{lb}$}‚ सामान्येनोक्त्वा विशेष‚निष्ठं क‚रोति स‚त्य\textbf{पीति} । एत‚च्च \textbf{विग्र‚ह} इत्य‚न्तं सुग‚म‚म् । \textbf{तेष्व}नुप‚{\tiny $_{lb}$}‚ल‚ब्धिल‚क्ष‚ण‚प्राप्तेष्व‚भाव‚निश्च‚याभावात् । स‚त्य‚पि व‚स्तुनि त‚स्या अनुप‚ल‚ब्धेर्भाव इष्टः ।‚{\tiny $_{lb}$}‚ अनेनैत‚दाह--नास्माक‚म‚त्र प्र‚माण‚म‚स्ति य‚त्स‚त्येव व‚स्तुनि सा भ‚व‚तीति । किन्तु त‚स्यां स‚त्याम‚पि‚{\tiny $_{lb}$}‚ य‚स्मात्प्र‚त्य‚यो दोलाय‚ते त‚स्मादेव‚मुच्य‚त इति ।
	\pend% ending standard par
      ‚{\tiny $_{lb}$}‚

	  \pstart \leavevmode% starting standard par
	स‚म्प्र‚ति निश्च‚याभाव‚स्याव‚धिं प‚र्येष‚माण आह--\textbf{क‚स्मा}त्स‚काशादिति । प‚र‚प्र‚त्य‚क्ष‚{\tiny $_{lb}$}‚निवृत्तेर‚श‚क्य‚निश्च‚य‚त्वे त‚न्निवृत्त्य‚र्थ‚मात्म‚प्र‚त्य‚क्ष‚ग्र‚ह‚ण‚म् । न‚नु य‚द्य‚य‚म‚पादान‚प्र‚श्नो न तु हेतु‚{\tiny $_{lb}$}‚प्र‚श्न‚स्त‚दा क‚थ‚मिद‚माह--\textbf{त‚स्य निवृत्तेः कार‚णान्निश्च‚याभाव} इति चेत् । न । अन्यार्थ‚त्वात्‚{\tiny $_{lb}$}‚ कार‚ण‚श‚ब्द‚स्य । निश्च‚याभाव‚स्य शाब्देन न्यायेन जाय‚मान‚स्यैषा प्र‚कृतिः कार‚ण‚म् । ज‚निक‚र्त्तुः‚{\tiny $_{lb}$}‚ प्र‚कृतिः \href{http://sarit.indology.info/?cref=Pā.1.4.30}{पाणिनि १. ४. ३०}इत्य‚नेन ल‚ब्धापादान‚संज्ञ‚काद‚स्मादित्य‚र्थ‚स्य विव‚क्षित‚त्वात् । अन्य‚था‚{\tiny $_{lb}$}‚ त्व‚स‚म‚ञ्ज‚सं स्यात् । \textbf{य‚स्मादि}त्यादिनोव‚त‚म‚र्थ‚मुप‚संह‚र‚ति । एत‚च्च \textbf{प्र‚तिषेध‚सिद्धिरित्येत}द‚न्तं‚{\tiny $_{lb}$}‚ सुग‚म‚म् ॥
	\pend% ending standard par
      ‚{\tiny $_{lb}$}‚

	  \pstart \leavevmode% starting standard par
	स‚म्प्र‚त्य‚नुप‚ल‚ब्धेर‚नुमान‚ज्ञान‚हेतुत्वात् प्रामाण्यं स्व‚भाव‚विशेषो व्यापार‚श्चोक्तोऽपि‚{\tiny $_{lb}$}‚ काल‚पुरुष‚विशेष‚प‚रिग्र‚हेण व‚क्तुम्--अथेत्यादिना प्र‚श्न‚पूर्व‚मुप‚क्र‚म‚ते । अथ‚श‚ब्द आर‚म्भे पूर्व‚व‚त् ।
	\pend% ending standard par
      ‚{\tiny $_{lb}$}‚‚{\tiny $_{lb}$}‚\textsuperscript{\textenglish{119/dm}}‚{\tiny $_{lb}$}‚
	  \bigskip
	  \begingroup
	

	  \pstart \leavevmode% starting standard par
	न‚नु च दृश्य‚निवृत्तिर‚व‚सीय‚ते दृश्यानुप‚ल‚म्भात् । स‚त्य‚मेवैत‚त् । केव‚ल‚मेक‚ज्ञान‚{\tiny $_{lb}$}‚संस‚र्गिणि दृश्य‚माने घ‚टो य‚दि भ‚वेद् दृश्य एव भ‚वेदिति दृश्यः स‚म्भावितः । \edtext{\textsuperscript{*}}{\lemma{*}\Bfootnote{विविक्त‚प्र‚देश‚ज्ञानात्०--\cite{dp-msD-n}}}त‚तो दृश्यानु‚{\tiny $_{lb}$}‚प‚ल‚ब्धिर्निश्चिता । \edtext{\textsuperscript{*}}{\lemma{*}\Bfootnote{दृश्यानुप‚ल‚म्भ‚निश्च \cite{dp-msC} \cite{dp-msD}}}दृश्यानुप‚ल‚ब्धिनिश्च‚य‚साम‚र्थ्यादेव च\edtext{}{\lemma{च}\Bfootnote{०देव दृश्या० \cite{dp-edE}}} दृश्याभावो निश्चितः । \edtext{\textsuperscript{*}}{\lemma{*}\Bfootnote{अमुमेवार्थं व्य‚तिरेक‚मुखेन भाव‚य‚ति--\cite{dp-msD-n}}}य‚दि‚{\tiny $_{lb}$}‚ हि दृश्य‚स्त‚त्र भ‚वेद् दृश्यानुप‚ल‚म्भो न भ‚वेत् । अतो दृश्यानुप‚ल‚म्भ‚निश्च‚याद् द‚श्याभावः‚{\tiny $_{lb}$}‚ साम‚र्थ्याद‚व‚सितः, न\edtext{}{\lemma{न}\Bfootnote{न तु व्य‚व० \cite{dp-msA} \cite{dp-edP} \cite{dp-edH} \cite{dp-edE} \cite{dp-edN}}} व्य‚व‚हृत इति दृश्यानुप‚ल‚म्भेन व्य‚व‚ह‚र्त्त‚व्यः ।
	\pend% ending standard par
       ‚{\tiny $_{lb}$}‚ 

	  \pstart \leavevmode% starting standard par
	त‚स्माद‚र्थान्त‚र‚म्--एक‚ज्ञान‚संस‚र्गि दृश्य‚मान‚म्, त‚ज्ज्ञानं च प्र‚त्य‚क्ष‚निवृत्तिनिश्च‚य‚हेतुत्वात्‚{\tiny $_{lb}$}‚ प्र‚त्य‚क्ष‚निवृत्तिरुक्तं द्र‚ष्ट‚व्य‚म् ।
	\pend% ending standard par
      
	  \endgroup
	‚{\tiny $_{lb}$}‚

	  \pstart \leavevmode% starting standard par
	प्र‚त्य‚क्ष‚प‚रिच्छेद्य‚त्वात् \textbf{प्र‚त्य‚क्षो घ‚टादिः । निवृत्ति}श‚ब्देनाचार्य‚स्यानुप‚ल‚ब्धिर्विव‚क्षितेति‚{\tiny $_{lb}$}‚ द‚र्श‚य‚ति \textbf{त‚स्य निवृत्तिर‚नुप‚ल‚ब्धिरिति} । अनुप‚ल‚ब्धिश‚ब्देनापि विव‚क्षित‚क‚र्त्तृक‚र्म‚ध‚र्मोप‚ल‚ब्धि‚{\tiny $_{lb}$}‚प‚र्युदासेनान्य‚देक‚ज्ञान‚संस‚र्गि व‚स्तु त‚ज्ज्ञानं च विव‚क्षित‚म् । \leavevmode\ledsidenote{\textenglish{48a/ms}}एत‚देव स्प‚ष्ट‚य‚ति \textbf{त‚द‚भाव‚{\tiny $_{lb}$}‚स्व‚भावेति याव‚दि}ति । त‚स्य प्र‚तिषेध्य‚स्य घ‚टादेर‚भावो विशिष्टो भाव‚स्त‚त्स्व‚भावा । त‚द‚भाव‚{\tiny $_{lb}$}‚स्व‚भाव‚श‚ब्देन यावान‚र्थ उक्त त‚द‚नुप‚ल‚ब्धिश‚ब्देनापीति इति याव‚दित्य‚स्यार्थः । य‚तोऽन्यो‚{\tiny $_{lb}$}‚प‚ल‚ब्धिरेव त‚द‚नुप‚ल‚ब्धिः । सैव च त‚द‚भावो नान्योऽत एवास्मादेव कार‚णाद‚भावो घ‚टादेर्न‚{\tiny $_{lb}$}‚ \textbf{साध्यः} । कुतो न साध्य इत्याह--\textbf{स्व‚भावानुप‚ल‚ब्धे}र्लिङ्गात् । कुतो न साध्य इत्याह—‚{\tiny $_{lb}$}‚\textbf{सिद्ध‚त्वा}त् निश्चित‚त्वात् घ‚टाभाव‚स्येति प्र‚क‚र‚णात् ।
	\pend% ending standard par
      ‚{\tiny $_{lb}$}‚

	  \pstart \leavevmode% starting standard par
	एवं म‚न्य‚ते--त‚देक‚ज्ञान‚संस‚र्गि व‚स्तु त‚ज्ज्ञानं च घ‚टाद्य‚नुप‚ल‚ब्धिस्त‚द‚भाव‚श्च । त‚च्चेन्द्रिय‚जेन‚{\tiny $_{lb}$}‚ प्र‚त्य‚क्षेण स्व‚संवेद‚नेन च सिद्ध‚मिति न लिङ्गाद‚भावः साध्य‚त इति । न‚न्व‚विद्य‚मानो घ‚टादिः क‚थं‚{\tiny $_{lb}$}‚ प्र‚त्य‚क्षः ? अथ प्र‚त्य‚क्षः, क‚थं त‚द‚नुप‚ल‚ब्धिरुच्य‚त इत्याह--\textbf{अविद्य‚मानोऽपि चे}ति । न केव‚लं‚{\tiny $_{lb}$}‚ विद्य‚मानः प्र‚त्य‚क्ष उच्य‚ते इत्य‚पिश‚ब्दः । \add{\textbf{स‚म‚ग्रा} स‚म‚स्ता \textbf{साम‚ग्री}}कार‚ण‚क‚लापो य‚स्येति विग्र‚हः ।‚{\tiny $_{lb}$}‚ शेषाद्विभाषा \href{http://sarit.indology.info/?cref=Pā.5.4.154}{पाणिनि ५. ४. १५४} इति क‚प् । न क‚पि \href{http://sarit.indology.info/?cref=Pā.7.4.14}{पाणिनि ७. ४. १४} इति ह्र‚स्व‚त्व‚{\tiny $_{lb}$}‚प्र‚तिषेधः । \textbf{ज्ञाय‚मानो} निश्चीय‚मानो \add{घ‚टादिरेक‚ज्ञान}संस‚र्गिणि प्र\edtext{}{\lemma{प्र}\Bfootnote{अस्प‚ष्ट‚म्--सं०}}...त‚दुप‚ल‚ब्धेरेवेत्य‚व‚{\tiny $_{lb}$}‚धार‚णीय‚म् । \textbf{प्र‚त्य‚क्ष‚निवृ}त्तिर्घ‚टाद्य‚नुप‚ल‚ब्धिः । क‚स्मात्तु द्व‚यं त‚थोच्य‚त इत्याह--त‚तो हीति । हीति‚{\tiny $_{lb}$}‚ य‚स्मात् । य‚स्माद‚मू एव दृश्य‚घ‚टादितुच्छ‚रूप‚निवृत्त्य‚व‚सेय‚हेतू \textbf{त‚स्मात्} कार‚णात् । \textbf{अर्थ}‚{\tiny $_{lb}$}‚ एक‚ज्ञान‚संस‚र्गिव‚स्त्व‚न्त‚र‚म्, ज्ञानं च त‚स्यैव । एत‚देव व्य‚तिरेक‚मुखेण द्र‚ढ‚य‚ति--न त्विति ।‚{\tiny $_{lb}$}‚ \textbf{निवृत्तिमात्र‚मु}प‚ल‚ब्ध्य‚भाव‚मात्र‚म् । त‚स्य त‚थात्वे को दोष इत्याह--\textbf{निवृत्तिमात्रादि}ति ।‚{\tiny $_{lb}$}‚ \textbf{दृश्य‚निवृत्त्य‚निश्च‚याद्} दृश्याभावानिश्च‚यात् । इह निवृत्तिमात्र‚स्य प्र‚स‚ज्य‚प्र‚तिषेधात्म‚नो‚{\tiny $_{lb}$}‚ निश्चेतुम‚श‚क्य‚त्वान्न हेतुत्वं युज्य‚त इति \textbf{ध‚र्मोत्त‚र‚स्या}श‚यो \textbf{निवृत्तिमात्राद् दृश्य‚निवृत्त्य‚निश्च‚या}‚{\tiny $_{lb}$}‚दिति ब्रुव‚तः । पूर्व‚प‚क्ष‚वादिना त्वेवं ज्ञात‚म्--निवृत्तिमात्रान्निर्विशेष‚णाद‚य‚मेवं प्र‚तिषेध‚ति ।‚{\tiny $_{lb}$}‚ त‚द‚हं स‚विशेष‚णं निवृत्तिमात्र‚मेव द‚र्श‚यामीति प्र‚मोद‚मान आह--\textbf{न‚नु चे}ति ।‚{\tiny $_{lb}$}‚ \textbf{दृश्य‚निवृत्ति}र्दृश्याभावः । \textbf{दृश्यानुप‚ल‚म्भादि}त्य‚त्रानुप‚ल‚म्भ‚श‚ब्देनोप‚ल‚म्भाभाव‚मात्रं विव‚क्षित‚{\tiny $_{lb}$}‚मित‚र‚था पूर्व‚प‚क्ष‚वादिनः प्र‚कृतं हीयेत । अनेन स‚विशेष‚ण‚मेवोप‚ल‚म्भाभाव‚मात्रं प्र‚स‚ज्य‚प्र‚ति‚{\tiny $_{lb}$}‚‚{\tiny $_{lb}$}‚ ‚{\tiny $_{lb}$}‚ \leavevmode\ledsidenote{\textenglish{120/dm}}‚{\tiny $_{lb}$}‚ 
	  
	य‚था\edtext{}{\lemma{था}\Bfootnote{प्र‚तिप‚त्तृप्र‚त्य‚क्ष‚स्यैत‚द् व्याख्याय अतीत‚स्य व‚र्त्त‚मान‚स्यैत‚द् विशेष‚ण‚द्व‚यं व्याच‚ष्टे--\cite{dp-msD-n}}} चैक‚ज्ञान‚संस‚र्गिणि प्र‚त्य‚क्षे घ‚ट‚स्य प्र‚त्य‚क्ष‚त्व‚मारोपित‚म् अस‚तोऽपि, त‚था त‚स्मिन्नेक‚{\tiny $_{lb}$}‚ज्ञान‚संस‚र्गिण्य‚तीते\edtext{}{\lemma{तीते}\Bfootnote{०तीते व‚र्त्त‚माने चामूढ‚स्मृतिसंस्कारे च घ‚ट० \cite{dp-msA} \cite{dp-msB} \cite{dp-msC} \cite{dp-msD} \cite{dp-edP} \cite{dp-edH} \cite{dp-edN}}} चामूढ‚स्मृतिसंस्कारे, व‚र्त्त‚माने च घ‚ट‚स्य त‚त्त\edtext{}{\lemma{त्त}\Bfootnote{त‚द्रूप \cite{dp-msA} \cite{dp-msB} \cite{dp-edP} \cite{dp-edH} \cite{dp-edE} \cite{dp-edN}}}द्रू\edtext{}{\lemma{द्रू}\Bfootnote{अतीतादि--\cite{dp-msD-n}}} प‚मारोपित‚म‚स‚त इति‚{\tiny $_{lb}$}‚ द्र‚ष्ट‚व्य‚म् । अनेन च\edtext{}{\lemma{च}\Bfootnote{अनेन दृश्या \cite{dp-msA} \cite{dp-msB} \cite{dp-edP} \cite{dp-edH} \cite{dp-edE} \cite{dp-edN}}} दृश्यानुप‚ल‚ब्धिः प्र‚त्य‚क्ष‚घ‚ट‚निवृत्तिस्व‚भावोक्ता । सा च सिद्धा ।‚{\tiny $_{lb}$}‚ तेन न\edtext{}{\lemma{न}\Bfootnote{तेन घ‚टा० \cite{dp-msB}}} घ‚टाभावः साध्यः, अपि तु अभाव‚व्य‚व‚हार इत्युक्त‚म् ।‚{\tiny $_{lb}$}‚ षेध‚रूपं लिङ्ग‚म‚स्तु, न तु न‚ञः प‚र्युदास‚वृत्त्या त‚देक‚ज्ञान‚संस‚र्गि व‚स्तु, त‚ज्ज्ञानं चेति पूर्व‚प‚क्ष‚वादी‚{\tiny $_{lb}$}‚ द‚र्श‚य‚ति । \textbf{स‚त्य}मित्यादिना प्र‚तिविध‚त्ते । किन्त्वेक\textbf{ज्ञान‚संस‚र्गिणि} भूत‚लादौ \textbf{दृश्य‚माने} स‚ति‚{\tiny $_{lb}$}‚ \textbf{दृश्यः स‚म्भावित} आरोपितः स प्र‚तिषेध्य इति प्र‚क‚र‚णात् । \textbf{त‚तो} दृश्य‚त्व‚स‚मारोपात् \textbf{दृश्यानु‚{\tiny $_{lb}$}‚प‚ल‚ब्धि}र्दृश्य‚ज्ञानाभाव‚स्तुच्छ‚रूपो व्य‚व‚ह‚र्त्त‚व्य‚मात्रं \textbf{निश्चिता} भ‚व‚ति । निश्चीय‚तामुप‚ल‚ब्ध्य‚{\tiny $_{lb}$}‚भावो ज्ञेयाभाव‚स्ताव‚न्न निश्चित इति त‚न्निश्च‚यार्थं निवृत्तिमात्रं व्याप‚रिष्य‚त इत्याह--\textbf{दृश्येति} ।‚{\tiny $_{lb}$}‚ दृश्य‚ज्ञानाभाव‚निश्च‚य‚साम‚र्थ्याद‚न्य‚थाऽनु\leavevmode\ledsidenote{\textenglish{48b/ms}}प‚प‚त्तेः । \textbf{चो} हेतौ । \textbf{दृश्य‚स्य} ज्ञेय‚स्या\textbf{भावो}‚{\tiny $_{lb}$}‚ व्य‚व‚ह‚र्त्त‚व्यैक‚रूपः । साम‚र्थ्य‚मेव \textbf{य‚दी}त्यादिना द‚र्श‚य‚ति । हिर्य‚स्माद‚र्थे । \textbf{दृश्यानुप‚ल‚म्भ} इति‚{\tiny $_{lb}$}‚ दृश्योप‚ल‚म्भाव इत्य‚र्थः । \textbf{अतो}ऽस्माद् \textbf{दृश्यानुप‚ल‚म्भ‚निश्च‚यात्} । दृश्य‚स्य \textbf{ज्ञेय‚स्याभाव} उक्ता‚{\tiny $_{lb}$}‚\textbf{त्साम‚र्थ्याद}व‚सितः । दृश्योप‚ल‚म्भाभाव‚निश्च‚य‚स्त्वेक‚ज्ञान‚संस‚र्गिव‚स्त्व‚न्त‚रोप‚ल‚ग्भेनेति द्र‚ष्ट‚व्य‚म् ।
	\pend% ending standard par
      ‚{\tiny $_{lb}$}‚

	  \pstart \leavevmode% starting standard par
	न‚नु य‚दि ज्ञेयाभावोऽप्य‚व‚सित‚स्त‚र्हि क‚थं लिङ्गेन साध्य‚त इत्याह--\textbf{न व्य‚व‚हृत} इति ।‚{\tiny $_{lb}$}‚ अदृष्टानाम‚पि स‚त्त्व‚श‚ङ्क‚या न प्र‚त्य‚क्षं व्य‚व‚हार‚यितुं श‚क्नोतीति भावः । केन त‚र्हि व्य‚व‚ह्रिय‚त‚{\tiny $_{lb}$}‚ इत्याह--\textbf{दृश्यानुप‚ल‚म्भेन} लिङ्ग‚भूतेन ।
	\pend% ending standard par
      ‚{\tiny $_{lb}$}‚

	  \pstart \leavevmode% starting standard par
	\edtext{\textsuperscript{*}}{\lemma{*}\Bfootnote{अस्प‚ष्ट‚म्--सं०}}...ऽभाव‚व्य‚व‚हार एव \edtext{}{\lemma{एव}\Bfootnote{अस्प‚ष्ट‚म्--सं०}}\add{... ... ...}\textbf{ज्ञानं चे}ति । च‚कार‚स्तुल्य‚क‚क्ष‚तां द‚र्श‚य‚ति ।‚{\tiny $_{lb}$}‚ द्व‚योश्च निवृत्तिनिश्च‚य‚हेतुत्व‚म्; ज्ञान‚म‚न्त‚रेण--य‚स्माद‚यं केव‚लः प्र‚देश‚स्त‚स्मात् घ‚टादिर्नास्ति‚{\tiny $_{lb}$}‚इत्य‚ध्य‚व‚सातुम‚श‚क्य‚त्वात्; त‚था विष‚य‚म‚न्त‚रेण--य‚स्मात्केव‚ल‚प्र‚देशाप‚रोक्षीक‚र‚णं त‚स्मात् त‚ज्ज्ञानं‚{\tiny $_{lb}$}‚ नास्ति इति निश्चेतुम‚श‚क्य‚त्वादिति द्र‚ष्ट‚व्य‚म् । प्र‚त्य‚क्ष‚स्य घ‚टादे\textbf{निवृत्तिनिश्च‚य‚हेतुत्वात्प्र‚त्य‚क्ष‚{\tiny $_{lb}$}‚निवृत्ति\edtext{}{\lemma{निवृत्ति}\Bfootnote{अस्प‚ष्ट‚म्--सं०}}\add{रुक्तं द्र‚ष्ट‚व्य‚म्}} ।
	\pend% ending standard par
      ‚{\tiny $_{lb}$}‚

	  \pstart \leavevmode% starting standard par
	न‚नु च प्र‚तिषेध्य‚स्य घ‚टादेर‚स‚तोऽपि त‚थाऽस्तु प्र‚त्य‚क्ष‚त्व‚म् । य‚त्पुन‚स्त‚त्र नासीदेव त‚स्य‚{\tiny $_{lb}$}‚ क‚थ‚म‚तीत‚त्व‚म्; य‚च्च त‚त्र नास्त्येव त‚स्य क‚थं व‚र्त्त‚मान‚त्व‚म्, क‚थं च त‚त्रानुप‚ल‚ब्धेर्व्यापार‚{\tiny $_{lb}$}‚ इत्याश‚ङ्काम‚पाकुर्व‚न्नाह--य‚थेति । येन प्र‚कारेण घ‚टो य‚दि भ‚वेद् दृश्य एव भ‚वेदित्येवं‚{\tiny $_{lb}$}‚ रूपेण । \textbf{चो} हेतौ । \textbf{त‚स्मिन्नेक‚ज्ञान‚संस‚र्गिण्य‚तीतेऽमूढ‚स्मृतिसंस्कारे । चो} व‚क्त‚व्यान्त‚र‚स‚मुच्च‚ये ।‚{\tiny $_{lb}$}‚ \textbf{व‚र्त्त‚माने च} । स‚मुच्च‚ये च‚कारः । \textbf{त‚था} तेन प्र‚कारेण--य‚दि त‚त्र पूर्वं घ‚टः स्थितो य‚दि स्यात्,‚{\tiny $_{lb}$}‚ उप‚ल‚ब्धः स्यात्, न चोप‚ल‚भ्य‚ते--इत्येव‚मात्म‚ना \textbf{घ‚ट‚स्या}रोपात् प्र‚त्य‚क्ष‚स्य त‚दानी\textbf{म‚स‚त‚{\tiny $_{lb}$}‚स्त‚द्रूप‚म}तीत‚त्वं व‚र्त्त‚मान‚त्व‚ञ्\textbf{चारोपित‚मा}रोप‚सिद्ध‚म् \textbf{इतिरे}वं द्र‚ष्ट‚व्याकारं द‚र्श‚य‚ति । \textbf{द्र‚ष्ट‚व्यं}‚{\tiny $_{lb}$}‚ ज्ञात‚व्य‚मिति योज‚नीय‚मिद‚म् ।
	\pend% ending standard par
      ‚{\tiny $_{lb}$}‚‚{\tiny $_{lb}$}‚\textsuperscript{\textenglish{121/dm}}‚{\tiny $_{lb}$}‚
	  \bigskip
	  \begingroup
	

	  \pstart \leavevmode% starting standard par
	\edtext{\textsuperscript{*}}{\lemma{*}\Bfootnote{प‚दानां प्र‚योज‚नं प्र‚तिपाद्येदानीं स‚म्ब‚न्ध‚म‚र्थं च द‚र्श‚य‚ति--\cite{dp-msD-n}}}अमूढोऽभ्र‚ष्टो द‚र्श‚नाहितः स्मृतिज‚न‚न‚रूपः संस्कारो य‚स्मिन् घ‚टादौ स त‚थोक्तः ।‚{\tiny $_{lb}$}‚ त‚स्य अतीत‚स्य प्र‚तिप‚त्तृप्र‚त्य‚क्ष‚स्येति स‚म्ब‚न्धः । व‚र्त्त‚मान‚स्य च प्र‚तिप‚त्तृप्र‚त्य‚क्ष‚स्येति स‚म्ब‚न्धः ।‚{\tiny $_{lb}$}‚ अमूढ‚स्मृतिसंस्कार‚ग्र‚ह‚णं तु न व‚र्त्त‚मान‚विशेष‚ण‚म् । य‚स्माद‚तीते घ‚ट‚विविक्त‚प्र‚देश‚द‚र्श‚ने स्मृति‚{\tiny $_{lb}$}‚संस्कारो मूढो दृश्य‚घ‚टानुप‚ल‚म्भे दृश्ये च घ‚टे मूढो भ‚व‚ति । व‚र्त्त‚माने तु\edtext{}{\lemma{तु}\Bfootnote{व‚र्त्त‚माने च घ‚ट० \cite{dp-msA} \cite{dp-msB} \cite{dp-edP} \cite{dp-edH} \cite{dp-edE} \cite{dp-edN}}} घ‚ट‚र‚हित‚प्र‚देश‚द‚र्श‚ने न‚{\tiny $_{lb}$}‚ स्मृतिसंस्कार‚मोहः । अत एव \edtext{}{\lemma{एव}\Bfootnote{अतएव न घ‚टानुप‚ल‚म्भे नापि घ‚टे मोहः \cite{dp-msA} \cite{dp-msC} \cite{dp-msD} \cite{dp-edP} \cite{dp-edH} \cite{dp-edE} \cite{dp-edN} अत एव घ‚टा‚{\tiny $_{lb}$}‚नुप‚ल‚म्भे नापि घ‚टे मोहः--\cite{dp-msB}}}न घ‚टाभावे, नापि घ‚टानुप‚ल‚म्भे मोहः । त‚स्मान्न व‚र्त्त‚मान‚निषेध्य‚{\tiny $_{lb}$}‚विशेष‚ण‚म‚मूढ‚स्मृतिसंस्कार‚ग्र‚ह‚ण‚म्, \edtext{\textsuperscript{*}}{\lemma{*}\Bfootnote{०ह‚ण‚म् व्य‚भि० \cite{dp-msC}}}स्मृतिसंस्कार‚व्य‚भिचाराभावाद् व‚र्त्त‚मान‚स्यार्थ‚स्य । अत एव‚{\tiny $_{lb}$}‚ व‚र्त्त‚मान‚स्य चेति च‚श‚ब्दः कृतः, विशेष‚ण‚र‚हित‚स्य व‚र्त्त‚मान‚स्य विशेष‚ण‚व‚तातीतेन स‚मुच्च‚यो‚{\tiny $_{lb}$}‚ य‚था विज्ञायेतेति\edtext{}{\lemma{विज्ञायेतेति}\Bfootnote{विज्ञायेत । त‚द‚य० \cite{dp-msB}}} ।
	\pend% ending standard par
       ‚{\tiny $_{lb}$}‚ 

	  \pstart \leavevmode% starting standard par
	त‚द‚य‚म‚र्थः--अतीतोऽनुप‚ल‚म्भः स्फुटः\edtext{}{\lemma{स्फुटः}\Bfootnote{स्फुटं \cite{dp-msB} \cite{dp-edP} \cite{dp-edH} \cite{dp-edE} \cite{dp-edN}}} स्म‚र्य‚माणः प्र‚माण‚म्, व‚र्त्त‚मान‚श्च । त‚तो‚{\tiny $_{lb}$}‚ नासीदिह घ‚टः, अनुप‚ल‚ब्ध‚त्वात्, नास्ति अनुप‚ल‚भ्य‚मान‚त्वात् इति श‚क्यं ज्ञातुम् । न तु‚{\tiny $_{lb}$}‚ न भ‚विष्य‚त्य‚त्र घ‚टः, \edtext{\textsuperscript{*}}{\lemma{*}\Bfootnote{अनुप‚ल‚भ्य‚मान‚त्वात्--\cite{dp-msA}}}अनुप‚ल‚प्स्य‚मान‚त्वात् इति\edtext{}{\lemma{इति}\Bfootnote{इति ज्ञातुं श‚क्य‚म्--\cite{dp-msC}}} श‚क्यं ज्ञातुम् । अनाग‚ताया अनुप‚ल‚ब्धेः‚{\tiny $_{lb}$}‚ स‚त्त्व‚स‚न्देहादिति काल‚विशेषोऽनुप‚ल‚ब्धेर्व्याख्यातः ।
	\pend% ending standard par
      
	  \endgroup
	‚{\tiny $_{lb}$}‚

	  \pstart \leavevmode% starting standard par
	न‚नु दृश्यानुप‚ल‚ब्धिः क‚स्मिन् काले प्र‚माणं किंस्व‚भावा किंव्यापारा चेति त्रित‚यं पृष्ट‚{\tiny $_{lb}$}‚ आचार्य‚स्त‚त्रानेन किमाख्यात‚माचार्येणेत्याश‚ङ्काम‚पाकुर्व‚न्नाह--\textbf{अनेने}ति । \textbf{अनेन} प्र‚तिप‚त्तृप्र‚त्य‚क्ष‚स्य‚{\tiny $_{lb}$}‚ निवृत्तिरिति व‚च‚नेन । \textbf{चो}ऽव‚धार‚णे । स्व\textbf{भाव‚स्ये}\edtext{}{\lemma{स्व}\Bfootnote{भावे}}त्य‚स्यान‚न्त‚रं द्र‚ष्ट‚व्यः ।‚{\tiny $_{lb}$}‚ प्र‚त्य‚क्ष‚स्य घ‚ट‚स्य । प्र‚स‚ज्य‚प्र‚तिषेध‚ल‚क्ष‚ण‚निवृत्तिनिश्च‚य‚हेतुत्वात्प्र‚त्य‚क्ष‚घ‚ट‚निवृत्तिः प्र‚त्य‚क्ष‚घ‚टानुप‚{\tiny $_{lb}$}‚ल‚ब्धिः । न‚ञः प‚र्युदास‚वृत्त्या । त‚देक‚ज्ञान‚संस‚र्गि व‚स्तु त‚ज्ज्ञानं च द्व‚यं \textbf{स्व‚भावो} रूपं य‚स्या‚{\tiny $_{lb}$}‚ दृश्यानुप‚ल‚ब्धेः सा त‚थोव‚ता व्याख्यातेति शेषः ।
	\pend% ending standard par
      ‚{\tiny $_{lb}$}‚

	  \pstart \leavevmode% starting standard par
	न‚नु दृश्यानुप‚ल‚ब्धिरूपं लिङ्ग‚म‚स्तु प‚र्युदास‚रूप‚म्, साध्य‚स्तु प्र‚स‚ज्य‚प्र‚तिषेध‚रूपः किं न‚{\tiny $_{lb}$}‚ भ‚व‚तीत्याह--\textbf{सेति । चो} य‚स्मात् । \textbf{सा} त‚थाविधाऽनुप‚ल‚ब्धि\add{... ... ...}उक्त‚माचार्येण‚{\tiny $_{lb}$}‚ साऽ\textbf{भाव‚व्य‚व‚हार‚प्र‚व‚र्त्त‚नी}त्य‚नेन श‚ब्देनेति भावः । \textbf{ध‚र्मोत्त‚रेण} चात एव \textbf{अभावो न साध्य} इत्य‚नेन ।
	\pend% ending standard par
      ‚{\tiny $_{lb}$}‚

	  \pstart \leavevmode% starting standard par
	घ‚ट‚विविक्त‚प्र\leavevmode\ledsidenote{\textenglish{49a/ms}}देश‚द‚र्श‚नेनाहित आरोपितः संस्कारो विशिष्ट‚श‚क्तियुक्त‚विज्ञानात्मिका‚{\tiny $_{lb}$}‚ वास‚ना, न तु प‚राभिम‚तो भाव‚नाख्यः । अनुरूप‚स्म‚र‚णं ज‚न‚यितुम‚नीशानो \textbf{मूढ} इति व्य‚प‚दिश्य‚ते ।‚{\tiny $_{lb}$}‚ \textbf{दृश्य‚घ‚टानुप‚ल‚म्भे} दृश्य‚स्य घ‚ट‚स्योप‚ल‚ब्ध्य‚भावे तुच्छ‚रूपे । अत एव च दृश्ये \textbf{घ‚टे} । अमूढ‚स्मृति‚{\tiny $_{lb}$}‚संस्कार‚ग्र‚ह‚णं क‚स्मान्न व‚र्त्त‚मान‚विशेष‚ण‚मित्याह--\textbf{व‚र्त्त‚माने त्वि}ति । तुर‚तीताद्व‚र्त्त‚मान‚स्य वैध‚र्म्य‚{\tiny $_{lb}$}‚माह । \textbf{अतो} मोहाभावान्न \textbf{घ‚टाभावे} व्य‚व‚ह‚र्त्त‚व्यैक‚रूपे । \textbf{नापि घ‚टानुप‚ल‚म्भे} घ‚टोप‚ल‚ब्ध्य‚भावे तुच्छ‚{\tiny $_{lb}$}‚रूपे । प्र‚त्य‚क्ष‚व‚दारोपाद् \textbf{व‚र्त्त‚मान‚स्यार्थ‚स्य} घ‚टादेः । \textbf{अत एव} व‚र्त्त‚मान‚ग्र‚ह‚ण‚स्य‚{\tiny $_{lb}$}‚ निर्विशेष‚ण‚त्वादेव ।
	\pend% ending standard par
      ‚{\tiny $_{lb}$}‚\footnote{अस्प‚ष्ट‚म्--सं०}\textsuperscript{\textenglish{122/dm}}‚{\tiny $_{lb}$}‚
	  \bigskip
	  \begingroup
	

	  \pstart \leavevmode% starting standard par
	\hphantom{.}व्यापारं द‚र्श‚य‚ति । अभाव‚स्य व्य‚व‚हारः नास्ति इत्येव‚माकारं ज्ञान‚म्, श‚ब्द‚श्चैव‚{\tiny $_{lb}$}‚माकारः, निःश‚ङ्कं\edtext{}{\lemma{ङ्कं}\Bfootnote{निःश‚ङ्क‚ग‚माग‚म‚ल‚क्ष० \cite{dp-msD} \cite{dp-msB} निःश‚ङ्का ग‚म‚नाग‚म‚न‚यो [[?]] ल‚क्ष० \cite{dp-msC}}} ग‚म‚नाग‚म‚न‚ल‚क्ष‚णा च प्र‚वृत्तिः कायिकोऽभाव‚व्य‚व‚हारः । घ‚टाभावे हि‚{\tiny $_{lb}$}‚ ज्ञाते निःश‚ङ्कं ग‚न्तुमाग‚न्तुं च प्र‚व‚र्त्त‚ते ।
	\pend% ending standard par
       ‚{\tiny $_{lb}$}‚ 

	  \pstart \leavevmode% starting standard par
	\edtext{\textsuperscript{*}}{\lemma{*}\Bfootnote{त‚देव‚म‚स्य \cite{dp-msC} \cite{dp-msD} त‚देव‚मेत‚स्य \cite{dp-msA} \cite{dp-edP} \cite{dp-edH} \cite{dp-edE} \cite{dp-edN} त‚देव त‚स्य--\cite{dp-msB}}}त‚देत‚स्य त्रिविध‚स्याप्य\edtext{}{\lemma{स्याप्य}\Bfootnote{प्य‚भाव‚स्य व्य‚व० \cite{dp-msC}}}भाव‚व्य‚व‚हार‚स्य दृश्यानुप‚ल‚ब्धिः \edtext{}{\lemma{ब्धिः}\Bfootnote{प्र‚व‚र्त्त‚नी इति नास्ति \cite{dp-msA} \cite{dp-msB} \cite{dp-msC} \cite{dp-msD} \cite{dp-edP} \cite{dp-edH} \cite{dp-edE} \cite{dp-edN}}}प्र‚व‚र्त्त‚नी साध‚नी प्र‚व‚र्त्तिका ।
	\pend% ending standard par
       ‚{\tiny $_{lb}$}‚ 

	  \pstart \leavevmode% starting standard par
	\hphantom{.}य‚द्य‚पि च नास्ति घ‚टः इति ज्ञान‚म‚नुप‚ल‚ब्धेरेव भ‚व‚ति, अय‚मेव चाभाव‚निश्च‚यः, त‚थापि‚{\tiny $_{lb}$}‚ य‚स्मात् प्र‚त्य‚क्षेण केव‚लः प्र‚देश उप‚ल‚ब्ध‚स्त‚स्मात् इह घ‚टो नास्ति इत्येवं \edtext{}{\lemma{इत्येवं}\Bfootnote{त्येवं प्र‚त्य० \cite{dp-msC}}}च प्र‚त्य‚क्ष‚व्यापार‚म‚नु‚{\tiny $_{lb}$}‚स‚र‚त्य‚भाव‚निश्च‚यः; त‚स्मात् प्र‚त्य‚क्ष‚स्य केव‚ल‚प्र‚देश‚ग्र‚ह‚ण‚व्यापारानुसार्य‚भाव‚निश्च‚यः प्र‚त्य‚क्ष‚कृतः ।
	\pend% ending standard par
      
	  \endgroup
	‚{\tiny $_{lb}$}‚

	  \pstart \leavevmode% starting standard par
	न‚नु य‚था विक‚ल्पेन विष‚यीक्रिय‚माणोऽतीतोऽनुप‚ल‚म्भः प्र‚माण‚मुच्य‚ते त‚था विक‚ल्प‚स्या‚{\tiny $_{lb}$}‚व्याह‚त‚प्र‚स‚र‚त्वाद‚नाग‚तोऽप्य‚नुप‚ल‚म्भो विक‚ल्प्य‚मानः किं न त‚थाप्र‚माण‚मित्याह--त‚द‚य‚मिति ।‚{\tiny $_{lb}$}‚ य‚स्मात्केव‚ल‚प्र‚देश‚द‚र्श‚नाहितः संस्कारोऽतीते घ‚टादाव‚मूढो गृह्य‚ते, व‚र्त्त‚माने तु त‚स्मिन् स्मृति‚{\tiny $_{lb}$}‚संस्कारे मोहो न स‚म्भ‚व‚त्येव त‚त्त‚स्माद‚यं तात्प‚र्यार्थः । अनुप‚ल‚म्भ‚निश्च‚य‚हेतु\textbf{त्वाद‚नुप‚ल‚म्भः}‚{\tiny $_{lb}$}‚ केव‚ल‚प्र‚देशादिः \textbf{स्फुटो} य‚थाऽसौ केव‚लोऽनुभूत‚स्त‚था स्मृत्वा विष‚यीक्रिय‚माणः \textbf{प्र‚माण‚म् । व‚र्त्त‚{\tiny $_{lb}$}‚मान‚श्च} तादृग‚नुप‚ल‚म्भः \textbf{स्फुटो}ऽभ्रान्तेन ज्ञानेन गृह्य‚माण इति द्र‚ष्ट‚व्य‚म् । य‚त ईदृशोऽनुप‚ल‚म्भः‚{\tiny $_{lb}$}‚ प्र‚माणं \textbf{त‚त}स्त‚स्मात् । कुतो न श‚क्य‚ते ज्ञातुमित्याह--\textbf{अनाग‚ताया} इति । त‚थात्वेन निश्चितो हि‚{\tiny $_{lb}$}‚ हेतुर्ग‚म‚कोऽन्य‚था स‚न्दिग्धासिद्ध‚ता हेतुदोषः स्यादित्य‚भिप्रायः । प्र‚त्य‚क्ष‚निवृत्तिश‚ब्देन ताव‚द्‚{\tiny $_{lb}$}‚ दृश्यानुप‚ल‚ब्धेः स्व‚भावो द‚र्शितोऽ\textbf{मूढेत्या}दिना तु किं द‚र्शित‚मित्याह--कालेति । \add{\textbf{इतिरेव‚म‚र्थे ते}}‚{\tiny $_{lb}$}‚नैवं \textbf{काल‚विशेषोऽनुप‚ल‚ब्धेर्व्याख्यात} इति । \textbf{काल‚विशेषो}ऽतीतो व‚र्त्त‚मान‚श्च \textbf{व्याख्यातः} क‚थितो‚{\tiny $_{lb}$}‚ऽनेनेति शेषः । एत‚च्चातीते व‚र्त्त‚माने च कालेऽनुप‚ल‚ब्धेः प्रामाण्याख्यान‚म‚स्योप‚ल‚क्ष‚णं द्र‚ष्ट‚व्य‚म्,‚{\tiny $_{lb}$}‚ कार्य‚स्व‚भाव‚हेत्वोर‚पि त‚योः काल‚योः प्रामाण्यात् । त‚था ह्यासीद‚त्र व‚ह्निर्धूम‚स्योप‚ल‚ब्ध‚त्वात् ।‚{\tiny $_{lb}$}‚ अस्ति व‚ह्निरिह धूम‚स्योप‚ल‚भ्य‚मान‚त्वात् । त‚थाऽसीदिह पाद‚पः शिंश‚पाया उप‚ल‚ब्ध‚त्वात् ।‚{\tiny $_{lb}$}‚ अस्तीह वृक्षः शिंश‚पाया उप‚ल‚भ्य‚मान‚त्वाद् इत्य‚पि भ‚व‚त्येव ।
	\pend% ending standard par
      ‚{\tiny $_{lb}$}‚

	  \pstart \leavevmode% starting standard par
	न‚नु दृश्यानुप‚ल‚म्भे भ‚व‚तु ज्ञानाभिधान‚ल‚क्ष‚णो व्य‚व‚हारः । कायिक‚स्तु क‚थ‚मित्याह--\textbf{घ‚टे}ति ।‚{\tiny $_{lb}$}‚ हिर्य‚स्मात् । \textbf{प्र‚व‚र्त्त‚त} इति योग्य‚त‚योच्य‚ते । प्र‚वृत्तियोग्य‚स्ताव‚द् भ‚व‚तीति । अत एव चाभाव‚{\tiny $_{lb}$}‚योग्य‚ता साध्योच्य‚ते ।
	\pend% ending standard par
      ‚{\tiny $_{lb}$}‚

	  \pstart \leavevmode% starting standard par
	\textbf{त‚देत‚स्ये}ति लोकोक्तिरेषा । त‚च्चैत‚च्चेति विग्र‚हः कार्यः । य‚द्वा य‚तोऽभाव‚{\tiny $_{lb}$}‚निश्च‚योऽनुप‚ल‚ब्धिनिमित्त‚क‚स्त‚त्त‚स्मात् । \textbf{अपि}र‚तिश‚ये । आस्तामेक‚स्य द्व‚योर्वा प्र‚व‚र्त्त‚नी,‚{\tiny $_{lb}$}‚ त्रिविध‚स्य याव‚त्प्र‚व‚र्त्त‚नीत्य‚र्थः । \textbf{प्र‚व‚र्त‚नी}त्य‚स्य द्व‚य‚मेत‚द् विव‚र‚णं स्प‚ष्टार्थं \textbf{साध‚नी प्र‚व‚र्त्तिके}‚{\tiny $_{lb}$}‚\leavevmode\ledsidenote{\textenglish{49b/ms}}\textbf{ति} प्र‚व‚र्त्त‚य‚तीति प्र‚व‚र्त्त‚नीति योग्य‚त‚योक्त‚म् । साभाव‚त्रित‚य‚म‚भाव‚व्य‚व‚हारं प्र‚व‚र्त्त‚यितुं‚{\tiny $_{lb}$}‚ योग्या ताव‚न्मात्र‚निमित्त‚क‚त्वात्त‚स्य । स‚त्य‚र्थित्वे पुरुष‚स्तान् व्य‚व‚हारानाच‚र‚तु मा वा ।‚{\tiny $_{lb}$}‚ अत एवाभाव‚व्य‚व‚हार‚योग्य‚ता प्र‚देशादेः साध्य‚ते दृश्यानुप‚ल‚म्भेनेति प‚र‚मार्थः ।
	\pend% ending standard par
      ‚{\tiny $_{lb}$}‚‚{\tiny $_{lb}$}‚\textsuperscript{\textenglish{123/dm}}‚{\tiny $_{lb}$}‚
	  \bigskip
	  \begingroup
	

	  \pstart \leavevmode% starting standard par
	किञ्च । दृश्यानुप‚ल‚म्भ‚निश्च‚य‚क‚र‚ण‚साम‚र्थ्यादेव पूर्वोक्त‚या नीत्या प्र‚त्य‚क्षेणैवाभावो‚{\tiny $_{lb}$}‚ निश्चितः । केव‚ल‚म‚दृष्टानाम‚पि स‚त्त्व‚स‚म्भ‚वात्, स‚त्त्व‚श‚ङ्क‚या न श‚क्नोत्य‚स‚त्त्वं\edtext{}{\lemma{त्त्वं}\Bfootnote{श‚क्नोत्य‚भावं व्य० \cite{dp-msC}}} व्य‚व‚ह‚र्त्तुम् ।‚{\tiny $_{lb}$}‚ अतोऽनुप‚ल‚म्भोऽभावं\edtext{}{\lemma{म्भोऽभावं}\Bfootnote{०ल‚म्भो व्य‚व० \cite{dp-msA}}} व्य‚व‚हार‚य‚ति--दृश्यो य‚तोऽनुप‚ल‚ब्धः, त‚स्मान्नास्ति इति । अतो‚{\tiny $_{lb}$}‚ दृश्यानुप‚ल‚म्भोऽभाव‚ज्ञानं कृतं प्र‚व‚र्त्त‚य‚ति, न तु अकृतं क‚रोती\edtext{}{\lemma{रोती}\Bfootnote{क‚रोत्य‚भाव० \cite{dp-msC}}}त्य‚भाव‚निश्च‚योऽनुप‚ल‚म्भात्‚{\tiny $_{lb}$}‚ प्र‚वृत्तोऽपि प्र‚त्य‚क्षेण कृतोऽनुप‚ल‚म्भेन प्र‚व‚र्त्तित उक्त इत्य‚भाव‚व्य‚व‚हार‚प्र\edtext{}{\lemma{प्र}\Bfootnote{०हारे प्र‚व‚र्त्त्य‚नुप‚ल‚ब्धिः \cite{dp-msC} प्र‚व‚र्त‚न्युप‚ल \cite{dp-msA} प्र‚व‚र्तिन्युप‚ल \cite{dp-edP} \cite{dp-edH} प्र‚व‚र्तिन्य‚नु० \cite{dp-msB}}} व‚र्त्त‚न्य‚नुल‚ब्धिः ॥
	\pend% ending standard par
       ‚{\tiny $_{lb}$}‚ 

	  \pstart \leavevmode% starting standard par
	क‚स्मात् पुन‚र‚तीते व‚र्त्त‚माने चानुप‚ल‚ब्धिर्ग‚मिकेत्याह--
	\pend% ending standard par
       ‚{\tiny $_{lb}$}‚ 
	  \bigskip
	  \begingroup
	

	  \pstart \leavevmode% starting standard par
	त‚स्या एवाभाव‚निश्च‚यात् ॥ २९ ॥
	\pend% ending standard par
      
	  \endgroup
	‚{\tiny $_{lb}$}‚ 

	  \pstart \leavevmode% starting standard par
	त‚स्या एव य‚थोक्त‚कालाया अनुप‚ल‚ब्धेर‚भाव‚निश्च‚यात् । अनाग‚ता ह्य‚नुप‚ल‚ब्धिः स्व‚य‚मेव‚{\tiny $_{lb}$}‚ स‚न्दिग्ध‚स्व‚भावा । त‚स्या असिद्धाया नाऽभाव‚निश्च‚योऽपि त्व‚तीत‚व‚र्त्त‚मानाया इति ॥
	\pend% ending standard par
      
	  \endgroup
	‚{\tiny $_{lb}$}‚

	  \pstart \leavevmode% starting standard par
	न‚नु य‚दि नास्तीत्येव‚माकारं ज्ञान‚म‚नुप‚ल‚ब्धेर्लिङ्गाद् भ‚व‚ति, क‚थं त‚र्हि प्र‚त्य‚क्षाव‚सितोऽभाव‚{\tiny $_{lb}$}‚ उक्त इत्याह--\textbf{य‚द्य‚पि चे}ति निपात‚स‚मुदायो य‚द्य‚पिश‚ब्द‚व‚द् विशेषाभिधान‚निमित्ताभ्युप‚ग‚मे‚{\tiny $_{lb}$}‚ व‚र्त्त‚ते । अभाव‚निश्च‚य‚श‚ब्द‚सामानाधिक‚र‚ण्याद‚य‚मेवेति निर्देशः । च‚श‚ब्दो व‚क्त‚व्य‚मेत‚दित्य‚स्यार्थे‚{\tiny $_{lb}$}‚ऽत्र व‚र्त्त‚ते । अभाव‚निश्च‚य‚स्य त‚दा स्थैर्य‚लाभाद‚नुप‚ल‚म्भादेवेत्युक्तं द्र‚ष्ट‚व्य‚म् । \textbf{त‚थापी}ति‚{\tiny $_{lb}$}‚ लौकिकोक्तिरियं निग‚दाभिधानार‚म्भे । \textbf{त‚स्मा}च्छ‚ब्देन य‚स्माच्छ‚ब्द आक्षिप्तः । तेनाय‚म‚र्थः ।‚{\tiny $_{lb}$}‚ य‚स्मात्केव‚ल‚प्र‚देश‚ग्र‚ह‚ण‚ल‚क्ष‚ण‚व्यापारानुसारी घ‚टो नास्तीत्य‚भाव‚निश्च‚यः \textbf{त‚स्मात्प्र‚त्य‚क्ष‚कृत} उच्य‚त‚{\tiny $_{lb}$}‚ इति शेषः । क‚स्य व्यापारानुसारीत्याकाङ्क्षायामुक्त‚म्--\textbf{प्र‚त्य‚क्ष‚स्य} प्र‚माण‚विशेष‚स्येति ।
	\pend% ending standard par
      ‚{\tiny $_{lb}$}‚

	  \pstart \leavevmode% starting standard par
	अथ स्याद्--य‚द्य‚यं प्र‚त्य‚क्ष‚कृत‚स्त‚दा प्र‚त्य‚क्ष‚प्र‚व‚र्त्तितोऽपि । त‚त्किं दृश्यानुप‚ल‚म्भेन क्रिय‚त‚{\tiny $_{lb}$}‚ इत्याश‚ङ्क्याह--\textbf{किञ्चे}ति व‚क्त‚व्यान्त‚र‚स‚मुच्च‚ये । त‚द्विविक्त‚प्र‚देशादिग्राहिणा \textbf{प्र‚त्य‚क्षेणैवाभावो}‚{\tiny $_{lb}$}‚ निषेध्याभावः प्र‚स‚ज्य‚प्र‚तिषेधात्मा \textbf{निश्चि}तः । क‚थं ? \textbf{दृश्य‚स्यानुप‚ल‚म्भ} उप‚ल‚म्भाभाव‚{\tiny $_{lb}$}‚स्तुच्छ‚रूप‚स्त\textbf{न्निश्च‚य‚क‚र‚ण‚साम‚र्थ्या}द‚न्य‚थाऽनुप‚प‚त्तेः । \textbf{पूर्वोक्त‚या नीत्या} युक्त्या य‚दि हि दृश्य‚स्त‚त्र‚{\tiny $_{lb}$}‚ भ‚वेत्, दृश्यानुप‚ल‚म्भो न भ‚वेदित्येव‚मात्मिक‚या । य‚दि प्र‚त्य‚क्ष‚मित्थं प्र‚तिषेध्याभावं निश्चाय‚य‚ति,‚{\tiny $_{lb}$}‚ व्य‚व‚हार‚यितुम‚पि श‚क्नोत्येवेत्याह--\textbf{केव‚लं} किन्तु \textbf{न श‚क्नो}ति \textbf{व्य‚व‚ह‚र्त्तुमित्य}न्त‚र्भूत‚णिज‚र्थ‚त्वान्नि‚{\tiny $_{lb}$}‚र्देश‚स्य व्य‚व‚हार‚यितुमित्य‚र्थः । कुतो \textbf{न} श‚क्नोतीत्याह--\textbf{स‚त्त्व‚श‚ङ्क‚ये}ति प्र‚तिषेध्य\textbf{स‚त्त्व‚स्य}‚{\tiny $_{lb}$}‚ स्व‚रूप‚स्य \textbf{स‚त्त्व‚श‚ङ्क‚या} स‚न्देहेन हेतुना ।
	\pend% ending standard par
      ‚{\tiny $_{lb}$}‚

	  \pstart \leavevmode% starting standard par
	न‚न्व‚द‚र्श‚नेऽपि क‚थं स‚न्देह इत्या\textbf{ह--अदृष्टे}ति । तेन प्र‚त्य‚क्षेणादृष्टानाम‚पि पिशाचादीनां‚{\tiny $_{lb}$}‚ \textbf{स‚त्व‚स्य} स‚द्भाव‚स्य \textbf{स‚म्भ‚वात्} स‚म्भाव्य‚मान‚त्वात् नित्यं श‚ङ्क्य‚मानानुप‚ल‚म्भ‚व्य‚भि‚{\tiny $_{lb}$}‚चारो ह्य‚भाव इति भावः । अतः प्र‚त्य‚क्ष‚स्य त‚त्राश‚क्त‚त्वात् साऽपि क‚थं व्य‚व‚हार‚य‚तीत्याह—‚{\tiny $_{lb}$}‚दृश्य इति । अतोऽस्मात्कार‚णात्कृतं प्र‚त्य‚क्षेणेति प्र‚क‚र‚णात् । एत‚देव व्य‚तिरेक‚मुखेण‚{\tiny $_{lb}$}‚ द्र‚ड‚य‚ति--\textbf{न त्वि}ति । य‚स्मात् प्र‚त्य‚क्ष‚व्यापारानुसार्य‚भाव‚निश्च‚य \textbf{इति}स्त‚स्मात् । \textbf{अभाव}‚{\tiny $_{lb}$}‚‚{\tiny $_{lb}$}‚ ‚{\tiny $_{lb}$}‚ \leavevmode\ledsidenote{\textenglish{124/dm}}‚{\tiny $_{lb}$}‚ 
	  
	स‚म्प्र‚त्य‚नुप‚ल‚ब्धेः प्र‚कार‚भेदं द‚र्श‚यितुमाह-- ‚{\tiny $_{lb}$}‚ 
	  
	सा च प्र‚योग‚भेदादेकाद‚श‚प्र‚कारा ॥ ३० ॥‚{\tiny $_{lb}$}‚ 
	  
	सा च एषानुप‚ल‚ब्धिः \edtext{}{\lemma{ब्धिः}\Bfootnote{ब्धिरेकाद‚श प्र‚कारा अस्या \cite{dp-msA} \cite{dp-edP} \cite{dp-edE}}}एकाद‚श‚प्र‚कारा--एकाद‚श प्र‚कारा अस्या इत्येकाद‚श‚प्र‚कारा । ‚{\tiny $_{lb}$}‚ 
	  
	कुतः प्र‚कार‚भेदः ? प्र‚योग‚भेदात् । प्र‚योगः प्र‚युक्तिः श‚ब्द‚स्याभिधाव्यापार\edtext{}{\lemma{स्याभिधाव्यापार}\Bfootnote{भिधान‚व्या० \cite{dp-msA} \cite{dp-msB} \cite{dp-msC} \cite{dp-msD} \cite{dp-edP} \cite{dp-edH} \cite{dp-edE} \cite{dp-edN}}}‚{\tiny $_{lb}$}‚ उच्य‚ते । श‚ब्दो हि साक्षात् क्व‚चिद‚र्थान्त‚राभिधायी\edtext{}{\lemma{राभिधायी}\Bfootnote{शीतादिविरुद्ध‚व‚ह्न्य‚भिधायी--\cite{dp-msD-n}}}, क्व‚चित् \edtext{}{\lemma{चित्}\Bfootnote{वृक्षाप्र‚द्य [[?]] भाव‚प्र‚तिषेधाभिधायी--\cite{dp-msD-n}}}प्र‚तिषेधान्त‚राभिधायी ।‚{\tiny $_{lb}$}‚ स‚र्व‚त्रैव तु दृश्यानुप‚ल‚ब्धिर‚श‚ब्दोपात्तापि ग‚म्य‚त इति वाच‚क‚व्यापार‚भेदाद‚नुप‚ल‚म्भ‚प्र‚कार‚भेदो न‚{\tiny $_{lb}$}‚ तु स्व‚रूप‚भेदादिति याव‚त् ॥ ‚{\tiny $_{lb}$}‚ 
	  
	प्र‚कार‚भेदान् आह--‚{\tiny $_{lb}$}‚ 
	  
	स्व‚भावानुप‚ल‚ब्धिर्य‚था--नात्र धूम उप‚ल‚ब्धिल‚क्ष‚ण‚प्राप्त‚स्यानुप‚ल‚ब्धेरिति ॥ ३१ ॥‚{\tiny $_{lb}$}‚ निश्च‚योऽनुप‚ल‚म्भात् प्र‚वृत्तोऽपि दृढीभूतोऽपि \textbf{प्र‚त्य‚क्षेण} केव‚ल‚प्र‚देशादिवेदिना साम‚र्थ्यात् कृत‚{\tiny $_{lb}$}‚ उत्पादितोऽ\textbf{नुप‚ल‚म्भेन} दृश्यानुप‚ल‚म्भेन \textbf{प्र‚व‚र्त्तितः} साधित \textbf{उक्त} आचार्येणा\textbf{भाव‚व्य‚व‚हार‚प्र‚व‚र्त्त‚नी‚{\tiny $_{lb}$}‚त्य‚नेन} श‚ब्देनेति बुद्धिस्थ‚म् । \textbf{इतीत्या}दिनोप‚संहारः । \textbf{इति}रेव‚मुक्तेन क्र‚मेणा\textbf{नुप‚ल‚ब्धि}र्दृश्या‚{\tiny $_{lb}$}‚नुप‚ल‚ब्धिस्त‚स्या एव प्र‚कृत‚त्वात् ।
	\pend% ending standard par
      ‚{\tiny $_{lb}$}‚

	  \pstart \leavevmode% starting standard par
	\textbf{क‚स्मादित्यादि व‚र्त्त‚मानाया} इत्य‚न्तं सुग‚म‚म् ।
	\pend% ending standard par
      ‚{\tiny $_{lb}$}‚

	  \pstart \leavevmode% starting standard par
	अनाग‚ताया अप्य‚नुप‚ल‚ब्धेर‚भाव‚निश्च‚यः क‚स्मान्न भ‚व‚तीत्याह--\textbf{अनाग‚तेति । हि}र्य‚स्माद‚र्थे ।‚{\tiny $_{lb}$}‚ \textbf{त‚स्या} इति प‚ञ्च‚म्य‚न्त‚मिद‚म् । अय‚मेव च साम‚र्थ्याव‚सितो मौलो\add{र्थोऽ\textbf{नाग‚ता हि}} इत्यादिना‚{\tiny $_{lb}$}‚ \textbf{ध‚र्मोत्त‚रेण\edtext{}{\lemma{रेण}\Bfootnote{अस्प‚ष्ट‚म्--सं०}}}...काङ्क्षोप‚श‚मार्थं पुर‚स्तादुक्तः ।
	\pend% ending standard par
      ‚{\tiny $_{lb}$}‚

	  \pstart \leavevmode% starting standard par
	स्यादेत‚त्--\textbf{प्र‚तिषेध‚सिद्धिरि}त्यादिना \textbf{निश्च‚याभावादि}त्य‚न्तेन \leavevmode\ledsidenote{\textenglish{50a/ms}} ग्र‚न्थेनास्यार्थ‚स्य‚{\tiny $_{lb}$}‚ \textbf{ग‚त‚त्वात्त‚स्या एवाभाव‚निश्च‚यादि}त्य‚य‚माचार्यीयो ग्र‚न्थः पुन‚रुक्त इति । न पुन‚रुक्तः । य‚तो‚{\tiny $_{lb}$}‚ य‚थाऽतीतेऽपि काले घ‚टादेस्त‚त्काल‚व‚र्त्तिदृश्यानुप‚ल‚ब्धेर‚तीतायाः स्मृत्यारूढाया स्व\edtext{}{\lemma{स्व}\Bfootnote{अ}}भाव‚निश्च‚य‚{\tiny $_{lb}$}‚स्त‚वाऽनाग‚तेऽपि काले घ‚टादेर‚भाव‚निश्च‚यः किं न भ‚व‚तीति केनापाकृतं येनायं पुन‚रुक्तः स्यादिति ॥
	\pend% ending standard par
      ‚{\tiny $_{lb}$}‚

	  \pstart \leavevmode% starting standard par
	स‚म्प्र‚ति \add{अनुप‚ल‚ब्धेः} \textbf{प्र‚कार}स्य स्व‚रूप‚स्य भेदं नानात्वं\edtext{}{\lemma{नानात्वं}\Bfootnote{अस्प‚ष्ट‚म्--सं०}}...प्र‚कार‚भेद\edtext{}{\lemma{भेद}\Bfootnote{अस्प‚ष्ट‚म्--सं०}}...‚{\tiny $_{lb}$}‚ द्र‚ष्ट‚व्य‚म् । \textbf{चो} व‚क्त‚व्यान्त‚र‚स‚मुच्च‚ये । \textbf{सेति} मूलानुवादः । \textbf{एषे}ति त‚स्य व्याख्यान‚म् ।‚{\tiny $_{lb}$}‚ एकाद‚श‚ग्र‚ह‚णं चाचार्य‚स्योप‚ल‚क्ष‚णार्थं य‚था \textbf{प्र‚माण‚वार्त्ति}के\edtext{}{\lemma{के}\Bfootnote{अस्प‚ष्ट‚म्--सं०}}...ग्र‚ह‚णं\edtext{}{\lemma{णं}\Bfootnote{अस्प‚ष्ट‚म्--सं०}}...षोड‚श‚प्र‚कारेति‚{\tiny $_{lb}$}‚ तु द्र‚ष्ट‚व्य‚म् । एत‚च्च कार‚ण‚विरुद्ध‚कार्योप‚ल‚ब्धिव्याख्यानान‚न्त‚रं द‚र्श‚यिष्यामः । विव‚क्षिता‚{\tiny $_{lb}$}‚न्योप‚ल‚म्भैक‚रूप‚त्वाद् दृश्यानुप‚ल‚ब्धेः, क‚थ‚म‚यं भेद उप‚प‚द्येतेत्य‚भिप्रेत्य पृच्छ‚ति \textbf{कुत} इति ।‚{\tiny $_{lb}$}‚ ‚{\tiny $_{lb}$}‚ \leavevmode\ledsidenote{\textenglish{125/dm}}‚{\tiny $_{lb}$}‚ 
	  
	स्व‚भावेत्यादि । प्र‚तिषेध्य‚स्य यः स्व‚भाव‚स्त‚स्यानुप‚ल‚ब्धिर्य‚थेति । अत्रेति ध‚र्मी । न‚{\tiny $_{lb}$}‚ धूम इति साध्य‚म्, उप‚ल‚ब्धिल‚क्ष‚ण‚प्राप्त‚स्यानुप‚ल‚ब्धेरिति हेतुः, अयं च हेतुः पूर्व‚व‚द्व्याख्येयः ॥ ‚{\tiny $_{lb}$}‚ 
	  
	प्र‚तिषेध्य‚स्य य‚त् कार्यं त‚स्यानुप‚ल‚ब्धिरुदाह्निय‚ते-- ‚{\tiny $_{lb}$}‚ 
	  
	कार्यानुप‚ल‚ब्धिर्य‚था--नेहाप्र‚तिब‚द्ध‚साम‚र्थ्यानि धूम‚कार‚णानि स‚न्ति,‚{\tiny $_{lb}$}‚ धूमाभावादिति\edtext{}{\lemma{धूमाभावादिति}\Bfootnote{इति नास्ति । \cite{dp-msB} \cite{dp-edP} \cite{dp-edH} \cite{dp-edE} \cite{dp-edN}}} ॥ ३२ ॥‚{\tiny $_{lb}$}‚ 
	  
	य‚थेति । इहेति ध‚र्मी । अप्र‚तिब‚द्ध‚म् अनुप‚ह‚तं धूम‚ज‚न‚नं प्र‚ति साम‚र्थ्यं येषां तान्य‚{\tiny $_{lb}$}‚प्र‚तिब‚द्ध‚साम‚र्थ्यानि न स‚न्तीति साध्य‚म् । धूमाभाव‚दिति हेतुः ।‚{\tiny $_{lb}$}‚ \textbf{प्र‚योग‚भेदा}दित्याचार्यीय‚मुत्त‚र‚म‚नूद्य प्र‚योग‚श‚ब्दं व्याच‚ष्टे \textbf{प्र‚यो}ग इति । \textbf{प्र‚युक्त्य}र्थ‚माह--\textbf{श‚ब्दे}ति ।‚{\tiny $_{lb}$}‚ श‚ब्द‚श‚ब्देन प्र‚क‚र‚णाद् वाच‚कः श‚ब्दो गृह्य‚ते । \textbf{अभिधा} अर्थ‚प्र‚काश‚न‚म् । त‚त्र \textbf{व्यापारो} व्यापृतिः‚{\tiny $_{lb}$}‚ प्र‚वृत्तिः, य‚द्वाऽभिधा अर्थ‚प्र‚काश‚नं त‚ल्ल‚क्ष‚णो व्यापार‚स्त‚स्य प्र‚योग‚स्य भेदाद् भिद्य‚मान‚त्वादिति‚{\tiny $_{lb}$}‚ तु सुबोध‚त्वात् \textbf{ध‚र्मोत्त‚रेण} न व्याख्यात‚म् ।
	\pend% ending standard par
      ‚{\tiny $_{lb}$}‚

	  \pstart \leavevmode% starting standard par
	न‚नु श‚ब्द‚स्यैवानुप‚ल‚म्भ‚वाच‚क‚स्यानुप‚ल‚म्भ एव वाच्य‚स्त‚त्क‚थं प्र‚योग‚भेदो येनानुप‚ल‚म्भ‚स्य‚{\tiny $_{lb}$}‚ प्र‚कार‚भेद उच्य‚त इत्याह--श‚ब्दो \textbf{हीति । हि}र्य‚स्मात् । \textbf{साक्षा}द‚व्य‚व‚धानेन । \textbf{क्व‚चिद्} विरुद्धोप‚{\tiny $_{lb}$}‚ल‚म्भादौ प्र‚तिषेध्याच्छीत‚स्प‚र्शादेर‚र्थान्त‚र‚म‚ग्न्याद्य‚भिध‚त्ते । क्व‚चिद् व्याप‚कानुप‚ल‚ब्ध्यादौ‚{\tiny $_{lb}$}‚ विव‚क्षितात् शिंश‚पादिप्र‚तिषेधात्प्र‚तिषेधान्त‚रं वृक्षादिप्र‚तिषेध‚म‚भिध‚त्ते । य‚द्य‚र्थान्त‚र‚विधिर‚र्था‚{\tiny $_{lb}$}‚न्त‚र‚प्र‚तिषेध‚श्च क्रिय‚ते त‚र्हि\edtext{}{\lemma{र्हि}\Bfootnote{अस्प‚ष्ट‚म्--सं०}}\add{... ... ...}द्य‚त इत्याह--\textbf{स‚र्व‚त्रे}ति । \textbf{तुर्वि}शेषार्थः ।‚{\tiny $_{lb}$}‚ अश‚ब्दोपात्ता स्व‚वाच‚क‚प‚दानुपात्ता । य‚था चाश‚ब्दोपात्ताऽपि सा प्र‚तीय‚ते त‚था पुर‚स्ताद‚भिधास्य‚ते ।‚{\tiny $_{lb}$}‚ \textbf{अपिर‚व}धार‚णे अतिश‚ये वा । \textbf{इति}स्त‚स्माद‚र्थे एव‚म‚र्थे\edtext{}{\lemma{र्थे}\Bfootnote{अस्प‚ष्ट‚म्--सं०}}\add{... ... ...}अत एवार्थीं‚{\tiny $_{lb}$}‚ ग‚तिमाश्रित्योक्तं न तु शाब्दीमिति द्र‚ष्ट‚व्य‚म् ॥
	\pend% ending standard par
      ‚{\tiny $_{lb}$}‚

	  \pstart \leavevmode% starting standard par
	अनुप‚ल‚म्भ‚स्य प्र‚कृत‚त्वात् \textbf{प्र‚कार‚भेदान्}--इति म‚न्त‚व्य‚म् । \textbf{त‚स्यानुप‚ल‚ब्धिः} पूर्वोक्त‚या‚{\tiny $_{lb}$}‚ नीत्या त‚द्विविक्तः प्र‚देश‚स्त‚ज्ज्ञानं चाव‚सेय‚म् । एवं कार्यानुप‚ल‚ब्ध्यादिषु द्विरूपैवाऽनुप‚{\tiny $_{lb}$}‚ल‚ब्धि\edtext{}{\lemma{ब्धि}\Bfootnote{अस्प‚ष्ट‚म्--सं०}}...ति निद‚र्श‚नेन ।
	\pend% ending standard par
      ‚{\tiny $_{lb}$}‚

	  \pstart \leavevmode% starting standard par
	न‚नु कीदृश्युप‚ल‚ब्धिल‚क्ष‚ण‚प्राप्तिः ? क‚थ‚ञ्चाविद्य‚मान\edtext{}{\lemma{मान}\Bfootnote{अस्प‚ष्ट‚म्--सं०}}\add{... ... ...}\textbf{अय‚ञ्चेति ।‚{\tiny $_{lb}$}‚ पूर्व}स्मिन्न‚नुप‚ल‚ब्धिल‚क्ष‚णाख्यान इव । \textbf{चो}\leavevmode\ledsidenote{\textenglish{50b/ms}}य‚स्माद‚र्थे, अव‚धार‚णार्थे वा \textbf{पूर्व‚व‚दि}त्य‚स्यान‚न्त‚रं‚{\tiny $_{lb}$}‚ द्र‚ष्ट‚व्यः । एत‚च्च स्व‚भावानुप‚ल‚म्भ‚स्यार्थ‚क‚थ‚नं कृत‚माचार्येण न प्र‚योगो द‚र्शितः, असाध‚नाङ्ग‚स्य‚{\tiny $_{lb}$}‚ प्र‚तिज्ञाया उपादानात् साध‚नाङ्ग‚स्य च व्याप्तेर‚प्र‚द‚र्श‚नात् हेतोश्चानुवाद्य‚रूप‚स्य प्र‚थ‚मान्त‚स्या‚{\tiny $_{lb}$}‚निर्देशात् । एव‚मित‚रासु स‚र्वास्वेवानुप‚ल‚ब्धिषु बोद्ध‚व्य‚म् ।
	\pend% ending standard par
      ‚{\tiny $_{lb}$}‚

	  \pstart \leavevmode% starting standard par
	प्र‚योगः पुन‚रीदृशः--य‚द् य‚त्रोप‚ल‚ब्धिल‚क्ष‚ण‚प्राप्तं स‚न्नोप‚ल‚भ्य‚ते त‚त्स‚र्वं त‚त्रास‚द्व्य‚व‚हार‚{\tiny $_{lb}$}‚योग्य‚म्--य‚था--तुर‚ङ्ग‚मोत्त‚माङ्गे शृङ्ग‚म् । नोप‚ल‚भ्य‚ते चात्रोप‚ल‚ब्धिल‚क्ष‚ण‚प्राप्तो धूम इति ।‚{\tiny $_{lb}$}‚ अनेन सामान्यादिनिराक‚र‚णे दृश्यानुप‚ल‚म्भः प्र‚योक्त‚व्य इति द‚र्शित‚माचार्येण तुल्य‚न्याय‚त्वा‚{\tiny $_{lb}$}‚दित्य व‚सेय‚म् ।
	\pend% ending standard par
      ‚{\tiny $_{lb}$}‚‚{\tiny $_{lb}$}‚\textsuperscript{\textenglish{126/dm}}‚{\tiny $_{lb}$}‚
	  \bigskip
	  \begingroup
	

	  \pstart \leavevmode% starting standard par
	कार‚णानि च नाव‚श्यं कार्य‚व‚न्ति भ‚व‚न्तीति कार्याद‚र्श‚नाद‚प्र‚तिब‚द्ध‚साम‚र्थ्यानामेवाभावः‚{\tiny $_{lb}$}‚ साध्यः,\edtext{\textsuperscript{*}}{\lemma{*}\Bfootnote{साध्य‚ते न त्व० \cite{dp-msC}}} न त्व‚न्येषाम् । अप्र‚तिब‚द्ध‚श‚क्तीनि चान्त्य‚क्ष‚ण‚भावीन्येव, अन्येषां प्र‚तिब‚न्ध‚स‚म्भ‚वात् ।
	\pend% ending standard par
       ‚{\tiny $_{lb}$}‚ 

	  \pstart \leavevmode% starting standard par
	कार्यानुप‚ल‚ब्धिश्च य‚त्र कार‚ण‚म‚दृश्यं त‚त्र प्र‚युज्य‚ते । दृश्ये तु कार‚णे दृश्यानुप‚ल‚ब्धिरेव‚{\tiny $_{lb}$}‚ ग‚मिका ।
	\pend% ending standard par
       ‚{\tiny $_{lb}$}‚ 

	  \pstart \leavevmode% starting standard par
	त‚त्र \edtext{}{\lemma{त्र}\Bfootnote{०गृह‚स्योप‚रि० \cite{dp-msB} \cite{dp-msC} \cite{dp-msD}}}ध‚व‚ल‚गृहोप‚रिस्थितो गृहाङ्ग‚ण‚म‚प‚श्य‚न्न‚पि च‚र्तुषु पार्श्वेष्व‚ङ्ग‚ण‚भित्तिप‚र्य‚न्तं प‚श्य‚ति ।‚{\tiny $_{lb}$}‚ भित्तिप‚र्य‚न्त‚स‚मं \edtext{}{\lemma{मं}\Bfootnote{आलोक‚त‚म‚सी \add{आकाश} इति बौद्ध‚म‚त‚म्--\cite{dp-msD-n}}}चालोक‚संज्ञ‚क‚माकाश‚देशं धूम‚विविक्तं प‚श्य‚ति । त‚त्र धूमाभाव‚निश्च‚याद् ।‚{\tiny $_{lb}$}‚ य‚द्देश‚स्थेन व‚ह्निना ज‚न्य‚मानो धूम‚स्त‚द्देशः स्यात् । त‚स्य च\edtext{}{\lemma{च}\Bfootnote{एवार्थः--\cite{dp-msD-n} । त‚स्य व‚ह्ने० \cite{dp-msB}}} व‚ह्नेर‚प्र‚तिब‚द्ध‚साम‚र्थ्य‚स्याभावः‚{\tiny $_{lb}$}‚ प्र‚तिप‚त्त‚व्यः । त‚द्गृहाङ्ग‚ण‚देशेन\edtext{}{\lemma{देशेन}\Bfootnote{त‚द्गृहाङ्ग‚ण‚देश‚स्थेन व‚ह्निना--\cite{dp-msC} \cite{dp-msD} त‚द्गृहाङ्ग‚ण‚देशेन भ[[च]] \cite{dp-msA} त‚द्गृहाङ्ग‚ण‚{\tiny $_{lb}$}‚स्थेन च व‚ह्निना--\cite{dp-msB} त‚द्गृहाङ्ग‚ण‚देशेन व‚ह्नि \cite{dp-edP} \cite{dp-edH} \cite{dp-edE}}} च व‚ह्निना ज‚न्य‚मानो धूम‚स्त‚द्देशः स्यात् । त‚स्मात् त‚द्देश‚स्य‚{\tiny $_{lb}$}‚ \edtext{\textsuperscript{*}}{\lemma{*}\Bfootnote{अप्र‚तिब‚द्ध‚साम‚र्थ्य‚स्येत्य‚र्थः--\cite{dp-msD-n}}}व‚ह्नेर‚भावः प्र‚तिप‚त्त‚व्यः ।
	\pend% ending standard par
       ‚{\tiny $_{lb}$}‚ 

	  \pstart \leavevmode% starting standard par
	त‚द्गृहाङ्ग‚ण‚देशं भित्तिप‚रिक्षिप्तं भित्तिप‚र्य‚न्त‚प‚रिक्षिप्तेन चालोकात्म‚ना धूम‚विविक्तेना‚{\tiny $_{lb}$}‚काश‚देशेन स‚ह ध‚र्मिणं क‚रोति ।
	\pend% ending standard par
      
	  \endgroup
	‚{\tiny $_{lb}$}‚

	  \pstart \leavevmode% starting standard par
	इह--अनुप‚ल‚म्भः केव‚ल‚प्र‚देशादिः, अभाव‚व्य‚व‚हार‚योग्य‚ता च साध्या, दृश्यानुप‚ल‚म्भ‚स्य‚{\tiny $_{lb}$}‚ तादात्म्य‚ल‚क्ष‚ण एव प्र‚तिब‚न्धो ग‚म्य‚ग‚म‚क‚भाव‚निब‚न्ध‚न‚म्--इति केषाञ्चिन्म‚त‚म् । केचित्तु—‚{\tiny $_{lb}$}‚अनुप‚ल‚म्भाऽभाव‚व्य‚व‚हार‚योग्य‚त‚योर्ग‚म्य‚ग‚म‚क‚भावे विप‚र्य‚य‚तादात्म्य‚ल‚क्ष‚णः प्र‚तिब‚न्धो निमित्त‚मिति‚{\tiny $_{lb}$}‚ प्र‚तिपेदिरे । अत एवानुप‚ल‚म्भः कार्य‚स्व‚भावाभ्यां भेदेन निर्दिष्टः, स्व‚साध्ये प्र‚तिब‚न्धा‚{\tiny $_{lb}$}‚न‚पेक्ष‚णाद्, इत‚र‚योश्च त‚द्वैप‚रीत्यादिति च । एवं चैत‚त् स‚माद‚ध‚तीति । त‚था हि--उप‚{\tiny $_{lb}$}‚ल‚ब्ध्य‚व्य‚भिचारात् तादृशी अनुप‚ल‚ब्धेरेव स‚त्ता । त‚त उप‚ल‚ब्धिल‚क्ष‚ण‚प्राप्त‚स‚त्त्वे त‚दुप‚ल‚म्भ‚यो‚{\tiny $_{lb}$}‚स्तादात्म्यादेवानुप‚ल‚म्भास‚द्व्य‚व‚हार‚योर्ग‚म्य‚ग‚म‚क‚भावः । य‚दि तु त‚द्विविक्त‚प्र‚देशाद्युप‚ल‚म्भ‚{\tiny $_{lb}$}‚रूप‚स्यानुप‚ल‚म्भ‚स्य त‚द‚स‚द्व्य‚व‚हार‚योग्तायाश्च य‚त्तादात्म्यं त‚न्निमित्त‚मुच्येत, त‚दा त‚स्य‚{\tiny $_{lb}$}‚ भूत‚लादेर‚नुप‚ल‚ब्धिल‚क्ष‚ण‚प्राप्ताऽस‚द्व्य‚व‚हार‚योग्य‚ताऽप्यात्म‚भूतैवेति पिशाचाद्य‚भाव‚व्य‚व‚हार‚म‚प्य‚{\tiny $_{lb}$}‚नुप‚ल‚म्भः साध‚येत् त‚त्प्र‚तिब‚न्धादिति ॥
	\pend% ending standard par
      ‚{\tiny $_{lb}$}‚

	  \pstart \leavevmode% starting standard par
	न‚नु कार्याभावात् कार‚णाभावः साध्य‚ताम् । कि\textbf{म‚प्र‚तिब‚द्ध‚साम‚र्थ्य‚स्ये}ति साध्य‚ध‚र्म‚{\tiny $_{lb}$}‚विशेष‚णेनेत्याह--\textbf{कार‚णानीति । चो} य‚स्माद् । \textbf{इति}स्त‚स्मात् । \textbf{साध्यः} साध‚यितुं युज्य‚त इत्य‚र्थः ।‚{\tiny $_{lb}$}‚ एत‚देव व्य‚तिरेक‚मुखेण द्र‚ढ‚य‚ति \textbf{न त्विति । तु}र्वैध‚र्म्यार्थ एव‚कारार्थो वा । \textbf{नान्येषां}‚{\tiny $_{lb}$}‚ कार‚ण‚कार‚ण‚त‚योप‚च‚रित‚कुर्व‚द्रूपाणाम् ।
	\pend% ending standard par
      ‚{\tiny $_{lb}$}‚

	  \pstart \leavevmode% starting standard par
	अथ प‚रिपूर्ण‚या अपि साम‚ग्र्या विधार‚क‚स‚म्भ‚वात् प्र‚तिब‚न्धः स‚म्भाव्य‚ते । त‚द‚स‚म्भ‚वीदं‚{\tiny $_{lb}$}‚ ‚{\tiny $_{lb}$}‚ \leavevmode\ledsidenote{\textenglish{127/dm}}‚{\tiny $_{lb}$}‚ 
	  
	त‚स्माद् \edtext{}{\lemma{स्माद्}\Bfootnote{दृश्य‚मानाका० \cite{dp-msC} \cite{dp-msA}}}दृश्य‚मानादृश्य‚मानाकाश‚देशाव‚य‚वः प्र‚त्य‚क्षाप्र‚त्य‚क्ष‚स‚मुदायो व‚ह्न‚य‚भाव‚प्र‚तीति‚{\tiny $_{lb}$}‚साम‚र्थ्यायातो ध‚र्मी, न दृश्य‚मान\edtext{}{\lemma{मान}\Bfootnote{दृश्य‚मान इव \cite{dp-msC}}} एव । इह--इति तु प्र‚त्य‚क्ष‚निर्देशो दृश्य‚मान‚भागापेक्षः । ‚{\tiny $_{lb}$}‚ 
	  
	न केव‚ल‚मिहैव दृश्यादृश्य‚स‚मुदायो ध‚र्मी, अपि त्व‚न्य‚त्रापि । श‚ब्द‚स्य क्ष‚णिक‚त्वे‚{\tiny $_{lb}$}‚ साध्ये क‚श्चिदेव श‚ब्दः प्र‚त्य‚क्षोऽन्य‚स्तु प‚रोक्ष‚स्त‚द्व‚दिहापि । य‚था चात्र ध‚र्मी साध्य‚प्र‚तिप‚त्त्य‚{\tiny $_{lb}$}‚धिक‚र‚ण‚भूतो दृश्यादृश्याव‚य‚वो द‚र्शित‚स्त‚द्व‚दुत्त‚रेष्व‚पि प्र‚योगेषु स्व‚यं प्र‚तिप‚त्त‚व्यः ॥‚{\tiny $_{lb}$}‚ विशेष‚ण‚मित्याह--\textbf{अप्र‚तिब‚द्धे}ति च‚श‚ब्द‚स्तुश‚ब्द‚स्यार्थे हेतौ वा । कुत एत‚दित्याह—‚{\tiny $_{lb}$}‚\textbf{अन्येषामि}ति । अन्त्य‚क्ष‚ण‚प्राप्तानि च कार‚णानि योगिनाऽपि प्र‚तिब‚द्धु\edtext{}{\lemma{द्धु}\Bfootnote{न्धु}}म‚श‚क्यानि,‚{\tiny $_{lb}$}‚ इत‚र‚था ताद्रूप्य‚मेव हीयेत । अत‚स्त‚थाविधानि कार‚णानि कार्याव्य‚भिचारीणि कार्येण व्याप्य‚न्ते ।‚{\tiny $_{lb}$}‚ त‚तः कार्यं निव‚र्त‚मानं कार‚णानि तादृशानि निव‚र्त्त‚य‚तीति द्र‚ष्ट‚व्य‚म् ।
	\pend% ending standard par
      ‚{\tiny $_{lb}$}‚

	  \pstart \leavevmode% starting standard par
	इह य‚द्य‚प्य‚नेनान्त्य‚क्ष‚णेनेदं कार्यं कृत‚मिति नाव‚ग‚तं त‚थापि व‚स्तुनः काचिद‚व‚स्था कार्यं‚{\tiny $_{lb}$}‚ कुर्व‚ती प्र‚तीतैव य‚द‚न‚न्त‚रं कार्य‚मुप‚ल‚ब्ध‚म् । सैव चाव‚स्थाऽत्रापि निषिध्य‚ते । अत एवायं‚{\tiny $_{lb}$}‚ क्ष‚णिकाक्ष‚णिक‚साधार‚णाव‚स्थातिश‚य‚निषेधः ।
	\pend% ending standard par
      ‚{\tiny $_{lb}$}‚

	  \pstart \leavevmode% starting standard par
	न‚नु कार्य‚स्य ता\leavevmode\ledsidenote{\textenglish{51a/ms}}व‚द् दृश्य‚स्यात्रानुप‚ल‚म्भो हेतुरुपादात‚व्यः । त‚त्किम‚स्य‚{\tiny $_{lb}$}‚ स‚म्भ‚वोऽस्ति य‚त्कार्य‚मेव दृश्यं न तु कार‚णं येन कार्यानुप‚ल‚ब्धिः प्र‚युज्य‚त इत्याश‚ङ्क्य‚{\tiny $_{lb}$}‚ विष‚य‚म‚स्या द‚र्श‚यितुमाह--\textbf{कार्ये}ति ।
	\pend% ending standard par
      ‚{\tiny $_{lb}$}‚

	  \pstart \leavevmode% starting standard par
	कोऽसावेवंविधो विष‚य इत्याह--\textbf{त‚त्रे}ति वाक्योप‚क्षेपे ।
	\pend% ending standard par
      ‚{\tiny $_{lb}$}‚

	  \pstart \leavevmode% starting standard par
	य‚दि पुन‚र्गृहाङ्ग‚णं प‚श्येत् दृश्यानुप‚ल‚ब्धिरेव प्र‚युक्ता स्यादित्य‚भिप्राथेणाह--\textbf{गृहाङ्ग‚ण‚{\tiny $_{lb}$}‚म‚प‚श्य‚न्न‚पीति । अपि}र‚व‚धार‚णे । \textbf{अङ्ग‚ण‚भित्तेः प‚र्य‚न्त‚म}व‚सानं । \textbf{भित्तिप‚र्य‚न्तेन स‚मं} तुल्य‚म् ।
	\pend% ending standard par
      ‚{\tiny $_{lb}$}‚

	  \pstart \leavevmode% starting standard par
	\textbf{सौत्रान्तिका}नामालोक‚त‚मः स्व‚भाव एवाक श इति । त‚न्म‚य‚त्या\edtext{}{\lemma{त्या}\Bfootnote{त‚न्म‚त्या}} आलोक‚{\tiny $_{lb}$}‚संज्ञाक‚मित्युक्त‚म् ।
	\pend% ending standard par
      ‚{\tiny $_{lb}$}‚

	  \pstart \leavevmode% starting standard par
	अत्राकाशे किं प्र‚त्येत‚व्य‚मित्याह--\textbf{य‚द्देशे}ति । य‚दा \textbf{य‚द्देशे चे}ति पाठ‚स्त‚दा यो‚{\tiny $_{lb}$}‚ देशोऽस्येति विग्र‚हः । य‚दा तु \textbf{य‚द्देश‚स्थे}नेति त‚दा य‚श्चासौ देश‚श्च त‚त्र तिष्ठ‚तीति \textbf{त‚द्देश‚स्थ}‚{\tiny $_{lb}$}‚ इति । स देशोऽस्येति विग्र‚हः ।
	\pend% ending standard par
      ‚{\tiny $_{lb}$}‚

	  \pstart \leavevmode% starting standard par
	एताव‚ताऽपि न ज्ञाय‚ते क‚स्तेन प्र‚तिप‚त्त्रा ध‚र्मी कृत इत्याह--\textbf{त‚द्गृहे}ति । स चासौ‚{\tiny $_{lb}$}‚ \textbf{गृहाङ्ग‚ण‚देश}श्चेति स‚मासः । अङ्ग‚ण‚स्य प्र‚कृत‚त्वाद् \textbf{भित्तिः} त‚स्यैव, त‚या \textbf{प‚रिक्षिप्तं} प‚रिच्छिन्नं‚{\tiny $_{lb}$}‚ \textbf{ध‚र्मिणं क‚रोतीति} स‚म्ब‚न्धः । किमेताव‚देवेत्याह \textbf{भित्ती}ति । \textbf{भित्ति}र‚ङ्ग‚ण‚स्यैव । \textbf{भि}त्तेश्च‚{\tiny $_{lb}$}‚ प्रान्तो वाच्य‚स्त‚स्याः \textbf{प‚र्य‚न्तोऽव}सानं निष्ठा तेन \textbf{प‚रिक्षिप्तं} प‚रिच्छिन्नं तेन । \textbf{च}स्तुल्य‚ब‚ल‚त्व‚{\tiny $_{lb}$}‚स‚मुच्च‚यार्थः ।
	\pend% ending standard par
      ‚{\tiny $_{lb}$}‚

	  \pstart \leavevmode% starting standard par
	\textbf{त‚स्मादि}त्यादिनोप‚संहार‚व्याजेन लोकाध्य‚व‚साय‚सिद्धं ध‚र्मिणं द‚र्श‚य‚ति । \textbf{दृश्यादृश्य‚{\tiny $_{lb}$}‚स‚मुदायो} लोकेनैक‚त्वेनाक‚लितः । तावान् देशो दृश्य‚मानः किमात्म‚कः ? \textbf{देशाव‚य‚वो} भागो य‚स्य‚{\tiny $_{lb}$}‚ ‚{\tiny $_{lb}$}‚ \leavevmode\ledsidenote{\textenglish{128/dm}}‚{\tiny $_{lb}$}‚ 
	  
	प्र‚तिषेध्य‚स्य व्याप्य‚स्य यो व्याप‚को ध‚र्म‚स्त‚स्यानुप‚ल‚ब्धिरुदाह्निय‚ते--‚{\tiny $_{lb}$}‚ स त‚था । क‚थ‚मेवंविधो ध‚र्मीत्याह--\textbf{व‚ह‚न्य‚भावेति} । व‚ह्निरिति विशिष्टो धूम‚ज‚न‚नेऽव्य‚{\tiny $_{lb}$}‚व‚धेय‚श‚क्तिः । त‚स्याप्युप‚ल‚क्ष‚ण‚त्वाद‚न्य‚स्य धूम‚ज‚न‚नेऽप्र‚तिब‚द्ध‚स्याभाव‚प्र‚तीतिर्ग्र‚हीत‚व्या ।
	\pend% ending standard par
      ‚{\tiny $_{lb}$}‚

	  \pstart \leavevmode% starting standard par
	अय‚म‚र्थः--लोक‚स्ताव‚त्त‚थाविध‚देशे धूम‚म‚नुप‚ल‚भ‚मान‚स्ताव‚ति देशे त‚थाभूत‚व‚ह्न्य‚भावं‚{\tiny $_{lb}$}‚ प्र‚त्येति । न चैत‚देवंविधं ध‚र्मिण‚म‚न्त‚रेण घ‚ट‚त इति \textbf{साम‚र्थ्य‚म}न्य‚थाऽनुप‚प‚त्तिस्त‚स्माद् \textbf{आयात}‚{\tiny $_{lb}$}‚ उप‚स्थितः । अत एवाचार्येणापि 
	    \pend% close preceding par
	  
	    \begin{quote}
	  
	    
	    \stanza[\smallbreak]
	इष्टं विरुद्ध‚कार्येऽपि देश‚कालाद्य‚पेक्ष‚ण‚म् ।\&[\smallbreak]


	
	    \end{quote}
	  
	    \pstart  \leavevmode% new par for following
	    \hphantom{.}
	   \href{http://sarit.indology.info/?cref=pv.3.5}{प्र‚माण‚वा० ३. ५}‚{\tiny $_{lb}$}‚ इति ब्रुव‚तैवं ध‚र्मीष्ट एवेति भावः ।
	\pend% ending standard par
      ‚{\tiny $_{lb}$}‚

	  \pstart \leavevmode% starting standard par
	न‚न्व‚प्र‚त्य‚क्ष‚स्यापि ध‚र्मित्वे क‚थ‚मिहेतीद‚मो ह‚प्र‚त्य‚यान्त‚स्य निर्देश इत्याश‚ङ्क्याह—‚{\tiny $_{lb}$}‚इहेति । तुश‚ब्दो य‚स्माद‚र्थे । \textbf{इति}रिह‚श‚ब्द‚स्य निर्दिष्ट‚स्याकारं प्र‚त्य‚व‚मृश‚ति । प्र‚त्य‚क्ष‚{\tiny $_{lb}$}‚व‚स्तुप्र‚तिपाद‚को निर्देश‚स्त‚थोक्तः ।
	\pend% ending standard par
      ‚{\tiny $_{lb}$}‚

	  \pstart \leavevmode% starting standard par
	न‚नु य‚दि दृश्यादृश्य‚स‚मुदायो ध‚र्मीं घ‚ट‚ते त‚दा कार्य‚स्व‚भाव‚हेत्वोर‚पि किम‚यं न स‚म्भ‚व‚ती‚{\tiny $_{lb}$}‚त्याह--\textbf{न केव‚ल‚मि}ति । \textbf{इहैव} कार्यानुप‚ल‚म्भ एव । \textbf{अपि तु} किन्त्व\textbf{न्य‚त्रापि} कार्य‚विशेषे‚{\tiny $_{lb}$}‚ स्व‚भाव‚विशेषे च त‚त्र त‚त्र कार्य‚हेतोर्धूम-वाक्याद‚ग्नि-पौरुषेय‚त्वादिसिद्धौ त‚थाविधो ध‚र्मी‚{\tiny $_{lb}$}‚ सुव्य‚क्त इति न त‚त्र द‚र्शित‚स्तुल्य‚त्वा\edtext{}{\lemma{त्वा}\Bfootnote{न्या}}य‚त‚या वा \textbf{द्र‚ष्ट‚व्यः} ।
	\pend% ending standard par
      ‚{\tiny $_{lb}$}‚

	  \pstart \leavevmode% starting standard par
	क‚थं हेताव‚प्येवंविध‚स्य ध‚र्मिणः स‚म्भ‚व इत्थाह--\textbf{श‚ब्द‚स्ये}ति । \textbf{क‚श्चिदेव} श्रूय‚माणः‚{\tiny $_{lb}$}‚ \textbf{प्र‚त्य‚क्षोऽन्य‚स्त्व}श्रूय‚माणः \textbf{प‚रोक्षः} । अनेनाव‚श्यं दृश्यादृश्य‚श‚ब्द‚स‚मुदायोऽत्र ध‚र्मीति द‚र्शित‚म् ।‚{\tiny $_{lb}$}‚ दृश्यादृश्यात्म‚ना च श‚ब्देन ध‚र्मिणा भाव्य‚मिति प्र‚तिपाद्य‚ज‚न‚संश‚यारोपाभ्यामायात‚म् । न‚हि‚{\tiny $_{lb}$}‚ प्र\leavevmode\ledsidenote{\textenglish{51b/ms}}तिपाद्यः श‚ब्द‚स्यानित्य‚त्वे संश‚यानो विप‚र्य‚स्य‚ति, अर्थात् घ‚ट‚श‚ब्द एव, श्रूय‚माण एव च‚{\tiny $_{lb}$}‚ स‚न्देग्धि, विप‚र्य‚स्य‚ति वा । किन्त्विहे\edtext{}{\lemma{किन्त्विहे}\Bfootnote{ह}}घ‚ट‚प‚टादिश‚ब्दे श्रूय‚माणे श्रुते श्रोष्य‚माणे च । त‚त‚स्त‚द‚{\tiny $_{lb}$}‚नुरोधात् प्र‚त्य‚क्षाप्र‚त्य‚क्ष‚श‚ब्द‚स‚मुदायोऽनित्य‚त्वे साध्ये ध‚र्मीं प्र‚स‚ह्य प‚तितः । \textbf{त‚द्व‚दिहापि‚{\tiny $_{lb}$}‚ कार्यानुप‚ल‚म्भे} ।
	\pend% ending standard par
      ‚{\tiny $_{lb}$}‚

	  \pstart \leavevmode% starting standard par
	अमुमेव न्याय‚म‚न्य‚त्राप्य‚तिदिश‚न्नाह--\textbf{य‚था चेति । चो}ऽव‚धार‚णे \textbf{साध्य‚प्र‚तिप‚त्त्य‚धि‚{\tiny $_{lb}$}‚क‚र‚ण‚भूत} इति विशेष‚ण‚व्याजेन ध‚र्मिणो ल‚क्ष‚ण‚मुक्त‚म् ।
	\pend% ending standard par
      ‚{\tiny $_{lb}$}‚

	  \pstart \leavevmode% starting standard par
	कार्यानुप‚ल‚ब्धिप्र‚योग‚स्त्वेवं क‚र्त्त‚व्यः । य‚त्र य‚स्य कार्य‚मुप‚ल‚ब्धिल‚क्ष‚ण‚प्राप्तं नोप‚ल‚भ्य‚ते‚{\tiny $_{lb}$}‚ त‚त्तु त‚ज्ज‚न‚नाप्र‚तिब‚द्ध‚साम‚र्थ्यं नास्ति । य‚था क्व‚चिद् दृश्य‚मानेऽङ्कुरे त‚थाविधं बीज‚म् ।‚{\tiny $_{lb}$}‚ नोप‚ल‚भ्य‚ते चात्रोप‚ल‚ब्धिल‚क्ष‚ण‚प्राप्तो धूम इति ।
	\pend% ending standard par
      ‚{\tiny $_{lb}$}‚

	  \pstart \leavevmode% starting standard par
	अनेन च कार्यानुप‚ल‚म्भ‚प्र‚द‚र्श‚नेन य‚ज्ज‚ल्पितं \textbf{ज‚ल्प‚म‚होद‚धिना} निःश‚ब्दे देशे श‚ब्द‚मात्रा‚{\tiny $_{lb}$}‚भावे साध्ये क‚स्त‚देक‚ज्ञान‚संस‚र्गिव‚स्त्व‚न्त‚रोप‚ल‚म्भो येन श‚ब्दाभाव‚व्य‚व‚हारो बौद्धानां भ‚वेद् ‚{\tiny $_{lb}$}‚ इति त‚त्प्र‚त्युक्तं द्र‚ष्ट‚व्य‚म् । त‚थाहि--नेहाप्र‚तिब‚द्ध‚साम‚र्थ्यानि श्रोत्र‚ज्ञान‚कार‚णानि स‚न्ति ।‚{\tiny $_{lb}$}‚ श्रोत्र‚ज्ञानाभावादिति कार्यानुप‚ल‚ब्धिः स्फुटैव । श्रोत्र‚ज्ञानानुप‚ल‚म्भ‚श्च त‚द‚न्य‚ज्ञानोप‚ल‚म्भ‚रूपः‚{\tiny $_{lb}$}‚ संविदितोऽस्त्येव । एक‚ज्ञान‚संस‚र्गित्वं चान्योन्याव्य‚भिच‚रितोप‚ल‚म्भ‚त्व‚मित्युक्तं पुर‚स्तात् ।‚{\tiny $_{lb}$}‚ प्र‚योग‚स्त्व‚न‚न्त‚र‚व‚द्विज्ञात‚व्य इति ॥
	\pend% ending standard par
      \textsuperscript{\textenglish{129/dm}}‚{\tiny $_{lb}$}‚
	  \bigskip
	  \begingroup
	
	  \bigskip
	  \begingroup
	

	  \pstart \leavevmode% starting standard par
	व्याप‚कानुप‚ल‚ब्धिर्य‚था\edtext{}{\lemma{था}\Bfootnote{नुप‚ल‚ब्धेर्य० \cite{dp-msC}}}--नात्र शिंश‚पा, वृक्षाभावादिति\edtext{}{\lemma{वृक्षाभावादिति}\Bfootnote{इति नास्ति \cite{dp-edE}}} ॥ ३३ ॥
	\pend% ending standard par
      
	  \endgroup
	‚{\tiny $_{lb}$}‚ 

	  \pstart \leavevmode% starting standard par
	य‚थेति । अत्रेति\edtext{}{\lemma{अत्रेति}\Bfootnote{अत्र ध‚र्मी--\cite{dp-msA} \cite{dp-msB} \cite{dp-edP} \cite{dp-edH} \cite{dp-edE}}} ध‚र्मी । न शिंश‚पेति शिंश‚पाऽभावः साध्यः । वृक्ष‚स्य व्याप‚क‚स्या‚{\tiny $_{lb}$}‚भावादिति हेतुः ।
	\pend% ending standard par
       ‚{\tiny $_{lb}$}‚ 

	  \pstart \leavevmode% starting standard par
	इय‚म‚प्य‚नुप‚ल‚ब्धिर्व्याप्य‚स्य शिंश‚पात्व‚स्या\edtext{}{\lemma{स्या}\Bfootnote{शिंश‚पात्व‚दृश्य‚स्याभाव \cite{dp-msB} शिंश‚पात्व‚स्य दृश्याभावे \cite{dp-msA} \cite{dp-edP} \cite{dp-edH} \cite{dp-edE} \cite{dp-edN} शिंश‚पात्व‚स्य‚{\tiny $_{lb}$}‚ दृश्य‚स्याभावे--\cite{dp-msC} \cite{dp-msD}}}ऽदृश्य‚स्याभावे\edtext{}{\lemma{स्याभावे}\Bfootnote{साध्ये \cite{dp-msD-n} ।}} प्र‚युज्य‚ते । उप‚ल‚ब्धिल‚क्ष‚ण‚{\tiny $_{lb}$}‚प्राप्ते तु व्याप्ये दृश्यानुप‚ल‚ब्धिर्ग‚मिका । त‚त्र य‚दा पूर्वाप‚रावुप‚श्लिप्टौ स‚मुन्न‚तौ देशौ भ‚व‚तः,‚{\tiny $_{lb}$}‚ त‚योरेक‚स्त‚रुग‚ह‚नोपेतो\edtext{}{\lemma{नोपेतो}\Bfootnote{०पेताऽप‚र \cite{dp-msC}}}ऽप‚र‚श्चैक‚शिलाघ‚टितो निर्वृक्ष‚क‚क्ष‚कः\edtext{}{\lemma{कः}\Bfootnote{तृणोत्क‚रः--\cite{dp-msD-n} । ०क‚क्षः । द्र० \cite{dp-msA} \cite{dp-msB} \cite{dp-msC} \cite{dp-msD} \cite{dp-edP} \cite{dp-edH} \cite{dp-edE} \cite{dp-edN}}} । द्र‚ष्टापि त‚त्स्थान् वृक्षान्‚{\tiny $_{lb}$}‚ प‚श्य‚न्न‚पि शिंश‚पादिभेदं\edtext{}{\lemma{पादिभेदं}\Bfootnote{०भेदं न यो विवे--\cite{dp-msA} \cite{dp-edP} \cite{dp-edH} \cite{dp-edE}}} यो न विवेच‚य‚ति, त‚स्य वृक्ष‚त्वं प्र‚त्य‚क्षं अप्र‚त्य‚क्षं\edtext{}{\lemma{क्षं}\Bfootnote{०त्य‚क्षं शिंश० \cite{dp-msA} \cite{dp-msB} \cite{dp-msC} \cite{dp-edH} \cite{dp-edE} \cite{dp-edN}}} तु शिंश‚पात्व‚म् । स‚{\tiny $_{lb}$}‚ हि निर्वृक्ष एक‚शिलाघ‚टिते वृक्षाभावं दृश्य‚त्वाद् दृश्यानुप‚ल‚म्भाद‚व‚स्य‚ति । शिंश‚पात्वाभावं तु‚{\tiny $_{lb}$}‚ व्याप‚क‚स्य वृक्ष‚त्व‚स्याभावादिति । तादृशे\edtext{}{\lemma{तादृशे}\Bfootnote{तादृश‚विष० \cite{dp-msA}}} विष‚येऽस्या\edtext{}{\lemma{येऽस्या}\Bfootnote{स्याः प्र‚योगोऽभाव‚साध‚नाय \cite{dp-msC}}} अभाव‚साध‚नाय प्र‚योगः ॥
	\pend% ending standard par
       ‚{\tiny $_{lb}$}‚ 
	  \bigskip
	  \begingroup
	

	  \pstart \leavevmode% starting standard par
	स्व‚भाव‚विरुद्धोप‚ल‚ब्धिर्य‚था--नात्र शीत‚स्प‚र्शो \edtext{}{\lemma{र्शो}\Bfootnote{अग्नेरिति--\cite{dp-msB} \cite{dp-edP} \cite{dp-edH} \cite{dp-edE} \cite{dp-edN}}}व‚ह्नेरिति ॥ ३४ ॥
	\pend% ending standard par
      
	  \endgroup
	‚{\tiny $_{lb}$}‚ 

	  \pstart \leavevmode% starting standard par
	प्र‚तिषेध्य‚स्य स्व‚भावेन विरुद्ध‚स्योप‚ल‚ब्धिरुद्राह्रिय‚ते \edtext{}{\lemma{ते}\Bfootnote{०ह्रिय‚ते । अत्रेति--\cite{dp-msB} \cite{dp-msC} \cite{dp-msD}}}य‚थेति । अत्रेति ध‚र्मी । न‚{\tiny $_{lb}$}‚ शीत‚स्प‚र्श \edtext{}{\lemma{र्श}\Bfootnote{इति नास्ति \cite{dp-msA}}}इति शीत‚स्प‚र्श‚प्र‚तिषेधः साध्यः । व‚ह्नेरिति हेतुः । इयं चानुप‚ल‚ब्धिस्त‚त्र‚{\tiny $_{lb}$}‚ प्र‚योक्त‚व्या य‚त्र शीत‚स्प‚र्शोदृश्ऽयः, दृश्ये\edtext{}{\lemma{दृश्ये}\Bfootnote{दृश्ये तु दृश्या० \cite{dp-msB} \cite{dp-msC} \cite{dp-msD}}} दृश्यानुप‚ल‚ब्धिप्र‚योगात् ।
	\pend% ending standard par
      
	  \endgroup
	‚{\tiny $_{lb}$}‚

	  \pstart \leavevmode% starting standard par
	व्याप‚कानुप‚ल‚ब्धिं व्याख्यातुमाह--\textbf{प्र‚तिषेध्य‚स्ये}ति । तादात्म्याविशेषेऽपि य‚था क‚श्चिदेव‚{\tiny $_{lb}$}‚ ध‚र्मो व्याप्य इत‚रो व्याप‚क‚श्च त‚था प्रागेव \textbf{ध‚र्मोत्त‚रेण} निर्णीत‚म् । \textbf{ध‚र्म} इति च ध‚र्म्य‚पेक्ष‚या‚{\tiny $_{lb}$}‚ वृक्ष‚त्वादि । \textbf{न शिंश‚पे}ति न शिंश‚पात्व‚मित्य‚र्थः । \textbf{वृक्ष‚स्ये}ति च ध‚र्मिणा ध‚र्म‚स्य वृक्ष‚त्व‚स्य निर्देशः ।
	\pend% ending standard par
      ‚{\tiny $_{lb}$}‚

	  \pstart \leavevmode% starting standard par
	न‚नूप‚ल‚ब्धिल‚क्ष‚ण‚प्राप्त‚स्य ताव‚द् वृक्ष‚त्व‚स्यानुप‚ल‚ब्धिः प्र‚योक्त‚व्या । त‚था च शिंश‚पात्व‚{\tiny $_{lb}$}‚म‚पि दृश्य‚मेव निषेध्य‚मिति दृश्यानुप‚ल‚ब्धिरेव प्र‚योगार्हेत्याह--\textbf{इय‚म‚पी}ति । न केव‚लं‚{\tiny $_{lb}$}‚ पूर्विका विशिष्टे विष‚ये किन्त्वि\textbf{य‚म‚पी}त्य‚पिश‚ब्दः । \textbf{शिंश‚पात्व‚स्यादृश्य‚स्येति}--य‚दि स्याद्‚{\tiny $_{lb}$}‚ दृश्य‚मेव स्यादिति स‚म्भाव‚नाम‚तिवृत्त‚स्येत्य‚र्थः । एव‚मुत्त‚र‚त्राप्य‚दृश्य‚त्व‚मीदृश‚मेव द्र‚ष्ट‚व्य‚म् ।‚{\tiny $_{lb}$}‚ त‚स्याभावे साध्य इत्य‚ध्याहारः ।
	\pend% ending standard par
      ‚{\tiny $_{lb}$}‚‚{\tiny $_{lb}$}‚\textsuperscript{\textenglish{130/dm}}‚{\tiny $_{lb}$}‚
	  \bigskip
	  \begingroup
	

	  \pstart \leavevmode% starting standard par
	त‚स्माद् य‚त्र व‚र्ण‚विशेषाद् व‚ह्निर्दृश्यः, शीत‚स्प‚र्शो दूर‚स्थ‚त्वात्\edtext{}{\lemma{त्वात्}\Bfootnote{दूर‚त्वात् \cite{dp-msB} \cite{dp-msD}}} स‚न्न‚प्य‚दृश्यः, त‚त्रास्याः‚{\tiny $_{lb}$}‚ प्र‚योगः ॥
	\pend% ending standard par
       ‚{\tiny $_{lb}$}‚ 
	  \bigskip
	  \begingroup
	

	  \pstart \leavevmode% starting standard par
	विरुद्ध‚कार्योप‚ल‚ब्धिर्य‚था--नात्र शीत‚स्प‚र्शो धूमादिति ॥ ३५ ॥
	\pend% ending standard par
      
	  \endgroup
	‚{\tiny $_{lb}$}‚ 

	  \pstart \leavevmode% starting standard par
	प्र‚तिषेध्येन य‚द् विरुद्धं त‚त्कार्य‚स्योप‚ल‚ब्धिर्ग‚मिका--य‚थेति । अत्रेति ध‚र्मी । न‚{\tiny $_{lb}$}‚ शीत‚स्प‚र्श इति शीत‚स्प‚र्शाभावः साध्यः । धूमादिति हेतुः । य‚त्र शीत‚स्प‚र्शः स‚न् दृश्यः
	\pend% ending standard par
      
	  \endgroup
	‚{\tiny $_{lb}$}‚

	  \pstart \leavevmode% starting standard par
	कोऽसावेवंविधो विष‚य इत्याह--त‚त्रेति वाक्योप‚न्यासे । \textbf{उप‚श्लिष्टौ} प्र‚त्यास‚न्नौ‚{\tiny $_{lb}$}‚ \textbf{स‚मुन्न}तावुच्चौ । त‚रूणां \textbf{ग‚ह‚नं} ग‚ह्व‚रं \textbf{तेनोपेतो} युक्तः । द्वितीय \textbf{एक‚या शिल‚या घ‚टितो} निर्मित‚{\tiny $_{lb}$}‚ एक‚शिलारूप‚स्तूप इति याव‚त् । त‚त्त्वेनैव च \textbf{निर्वृक्ष‚क‚क्ष‚कः । क‚क्ष}स्तृण‚म् । निर्ग‚तौ वृक्ष‚क‚क्षौ‚{\tiny $_{lb}$}‚ य‚त इति विग्र‚हः । भ‚व‚त्वेवं त‚थापि क‚थ‚म‚स्याः प्र‚योग इत्याह--\textbf{द्र‚ष्टापीति । अपिर}व‚धार‚णे—‚{\tiny $_{lb}$}‚\textbf{न विवेच‚य‚ती}त्य‚स्यान‚न्त‚रं द्र‚ष्ट‚व्य‚म् \edtext{}{\lemma{म्}\Bfootnote{व्यः}} । \textbf{त‚स्य} तादृश‚स्य द्र‚ष्टुः ।
	\pend% ending standard par
      ‚{\tiny $_{lb}$}‚

	  \pstart \leavevmode% starting standard par
	भ‚व‚तु द्र‚ष्टु\add{स्}ताव‚द् दूर‚देश‚स्थायित‚या शिंश‚पाया अविवेक‚स्त‚थापि येनैव वृक्षाभावं‚{\tiny $_{lb}$}‚ प्र‚तिप‚द्य‚ते तेनैव शिंश‚पाऽभाव‚म‚पि किं न प्र‚तिप‚द्य‚त इत्याह--\textbf{स ही}ति । \textbf{हि}र्य‚स्मात् । क‚थं‚{\tiny $_{lb}$}‚ दृश्यानुप‚ल‚म्भाद‚व‚स्य‚तीत्याह--\textbf{दृश्य‚त्वाद्} वृक्ष‚त्व‚स्येति प्र‚क‚र‚णात् । कुत‚स्त‚र्हि शिंश‚पा\leavevmode\ledsidenote{\textenglish{52a/ms}}‚{\tiny $_{lb}$}‚त्वाभाव‚म‚वैतीत्याह--\textbf{शिंश}पेति । \textbf{तु}श‚ब्दो वैध‚र्म्ये । \textbf{इति}स्त‚स्माद‚र्थे । \textbf{अभाव}श‚ब्देनाभावोऽभाव‚{\tiny $_{lb}$}‚व्य‚व‚हार‚श्चोक्तो द्र‚ष्ट‚व्यः । एव‚मुत्त‚र‚त्रापि प्र‚त्येय‚म् ।
	\pend% ending standard par
      ‚{\tiny $_{lb}$}‚

	  \pstart \leavevmode% starting standard par
	प्र‚योगः पुन‚रीदृशः कार्यः--य‚त्र य‚स्य व्याप‚कं नास्ति न त‚त् त‚त्रास्ति । य‚था--अस‚ति‚{\tiny $_{lb}$}‚ प्र‚मेय‚त्वे प्रामाण्य‚म् । नास्ति च वृक्ष‚त्वं शिंश‚पात्व‚स्य व्याप‚क‚मिति । अनेन व्याप‚कानुप‚{\tiny $_{lb}$}‚ल‚म्भ‚स्य व्याप्याभावे ग‚म‚क‚त्व‚प्र‚तिपाद‚नेन नित्यानाम‚र्थ‚क्रियाकारित्वाभावः क्र‚म‚यौग‚प‚द्य‚यो‚{\tiny $_{lb}$}‚व्य‚पिक‚योर‚भावादित्यादि द‚र्शितं द्र‚ष्ट‚व्य‚म् ॥
	\pend% ending standard par
      ‚{\tiny $_{lb}$}‚

	  \pstart \leavevmode% starting standard par
	\textbf{प्र‚तिषेध्येत्या}दिना स्व‚भाव‚विरुद्धोप‚ल‚ब्धिं व्याच‚ष्टे--मूले तूदाह्रिय‚त इत्य‚ध्याहार \textbf{इति}‚{\tiny $_{lb}$}‚ द‚र्श‚य‚न्नाह--\textbf{उदाह्रिय‚त} इति ।
	\pend% ending standard par
      ‚{\tiny $_{lb}$}‚

	  \pstart \leavevmode% starting standard par
	सुख‚प्र‚तिप‚त्त्य‚र्थं मौल ध‚र्म्यादिप्र‚विभागं द‚र्श‚य‚न्नाह--अत्रेति ।
	\pend% ending standard par
      ‚{\tiny $_{lb}$}‚

	  \pstart \leavevmode% starting standard par
	एतेनेश्व‚रेऽप्येकोपादानादिविक‚ल्पे स‚म्प्र‚दानादिविक‚ल्पाभावो विरुद्धोप‚ल‚ब्धिप्र‚स‚ङ्गेन‚{\tiny $_{lb}$}‚ द‚र्शितः । विक‚ल्प‚स्य विक‚ल्पान्त‚रेण स‚हास्थितिल‚क्ष‚ण‚स्य विरोध‚स्य स्व‚स‚न्ताने सिद्ध‚त्वात् । त‚त‚श्च‚{\tiny $_{lb}$}‚ त‚स्यानिरूप्य‚क‚र्त्तृत्व‚मायात‚म् । अन्य‚था युग‚प‚द्दृष्टोत्पादानाम‚नुत्प‚त्तिः प्र‚स‚ज्येत । अनिरूप्य‚क‚र्त्तृत्वे‚{\tiny $_{lb}$}‚ चाऽऽधिप‚त्य‚मात्रेण क‚र्त्तृत्वं स्यात् । त‚था च क‚र्म‚णा सिद्ध‚साध‚न‚त्व‚मीश्व‚र‚साध‚नानां कार्य‚त्वादीना‚{\tiny $_{lb}$}‚मित्यादि द‚र्शित‚म् ॥
	\pend% ending standard par
      ‚{\tiny $_{lb}$}‚

	  \pstart \leavevmode% starting standard par
	विरुद्ध‚कार्योप‚ल‚ब्धिं व्याच‚क्षाण आह--\textbf{प्र‚तिषेध्येने}ति । मूल‚ग‚मिका विव‚क्षिताभाव‚{\tiny $_{lb}$}‚प्र‚तिपाद‚केति विव‚क्षित‚मिति द‚र्श‚यितुमाह--\textbf{ग‚मिके}ति । पूर्व‚व‚द् ध‚र्म्यादिक‚थ‚न‚म् ।
	\pend% ending standard par
      ‚{\tiny $_{lb}$}‚\textsuperscript{\textenglish{131/dm}}‚{\tiny $_{lb}$}‚
	  \bigskip
	  \begingroup
	

	  \pstart \leavevmode% starting standard par
	स्यात् त‚त्र‚दृश्यानुप‚ल‚ब्धिर्ग‚मिका । य‚त्र विरुद्धो व‚ह्निः प्र‚त्य‚क्षः, त‚त्र विरुद्धोप‚ल‚ब्धिर्ग‚मिका\edtext{}{\lemma{मिका}\Bfootnote{ब्धिः । द्व‚यो० \cite{dp-msA} \cite{dp-msB} \cite{dp-edP} \cite{dp-edH} \cite{dp-edE} \cite{dp-edN}}} ।‚{\tiny $_{lb}$}‚ द्व‚योर‚पि तु प‚रोक्ष‚त्वे \edtext{}{\lemma{त्वे}\Bfootnote{विरोध‚का० \cite{dp-msA}}}विरुद्व‚कार्योप‚ल‚ब्धिः प्र‚युज्य‚ते ।
	\pend% ending standard par
       ‚{\tiny $_{lb}$}‚ 

	  \pstart \leavevmode% starting standard par
	\edtext{\textsuperscript{*}}{\lemma{*}\Bfootnote{य‚त्र \cite{dp-msA}}}त‚त्र स‚म‚स्ताप‚व‚र‚क‚स्थं शीतं निव‚र्त्त‚यितुं स‚म‚र्थ‚स्याग्नेर‚नुमाप‚कं य‚दा विशिष्टं धूम‚क‚लापं‚{\tiny $_{lb}$}‚ निर्यान्त‚म‚प‚व‚र‚कात् प‚श्य‚ति, त‚दा विशिष्टाद्व‚ह्नेर‚नुमितात् शीत‚स्प‚र्श‚निवृत्तिम‚नुमिमीते\edtext{}{\lemma{नुमिमीते}\Bfootnote{निवृत्तिर‚नुमीय‚ते \cite{dp-msA} \cite{dp-msC}}} । इह‚{\tiny $_{lb}$}‚ दृश्य‚मान‚द्वार‚प्र‚देश‚स‚हितः \edtext{}{\lemma{हितः}\Bfootnote{स‚र्वाप‚व‚र‚का० \cite{dp-msA} \cite{dp-edP} \cite{dp-edH} \cite{dp-edE} \cite{dp-edN}}}स‚र्वोऽप‚व‚र‚काभ्य‚न्त‚र‚देशो ध‚र्मी साध्य‚प्र‚तिप‚त्त्य‚नुस‚र‚णात् पूर्व‚व‚द्‚{\tiny $_{lb}$}‚ द्र‚ष्ट‚व्य इति\edtext{}{\lemma{इति}\Bfootnote{इति नास्ति \cite{dp-msA} \cite{dp-msB} \cite{dp-msC} \cite{dp-edP} \cite{dp-edH} \cite{dp-edE} \cite{dp-edN}}} ॥
	\pend% ending standard par
       ‚{\tiny $_{lb}$}‚ 
	  \bigskip
	  \begingroup
	

	  \pstart \leavevmode% starting standard par
	विरुद्ध‚व्याप्तोप‚ल‚ब्धिर्य‚था--न ध्रुव‚भावी भूत‚स्यापि भाव‚स्य विनाशः,‚{\tiny $_{lb}$}‚ हेत्व‚न्त‚रापेक्ष‚णादिति\edtext{}{\lemma{णादिति}\Bfootnote{इति नास्ति \cite{dp-edE}}} ॥ ३६ ॥
	\pend% ending standard par
      
	  \endgroup
	‚{\tiny $_{lb}$}‚ 

	  \pstart \leavevmode% starting standard par
	प्र‚तिषेध्य‚स्य य‚द् विरुद्धं तेन व्याप्त‚स्य ध‚र्मान्त‚र‚स्य उप‚ल‚ब्धिरुदाह‚र्त्त‚व्या । य‚थेति ।‚{\tiny $_{lb}$}‚ ध्रुव‚म् अव‚श्यं भ‚व‚तीति\edtext{}{\lemma{तीति}\Bfootnote{भ‚व‚ति ध्रु० \cite{dp-msB} \cite{dp-msC} \cite{dp-msD}}} ध्रुव‚भावी नेति ध्रुव‚भावित्व‚निषेधः\edtext{}{\lemma{निषेधः}\Bfootnote{०त्व‚प्र‚तिषेधः--\cite{dp-msC}}} साध्यः । विनाशो ध‚र्मी ।
	\pend% ending standard par
      
	  \endgroup
	‚{\tiny $_{lb}$}‚

	  \pstart \leavevmode% starting standard par
	क‚स्मिन्नियं प्र‚योक्त‚व्येत्याह--य‚त्रेति । द्व‚योर्विरुद्ध‚शीत‚स्प‚र्श‚योः । \textbf{अपि}र‚व‚धार‚णे ।‚{\tiny $_{lb}$}‚ \textbf{तुः} पूर्व‚स्माद् वैध‚र्म्ये ।
	\pend% ending standard par
      ‚{\tiny $_{lb}$}‚

	  \pstart \leavevmode% starting standard par
	कः पुन‚रीदृशो विष‚य इत्याह--\textbf{त‚त्रेति} ।
	\pend% ending standard par
      ‚{\tiny $_{lb}$}‚

	  \pstart \leavevmode% starting standard par
	न‚नु च प्र‚दीप‚शिखाप्र‚भावे\edtext{}{\lemma{भावे}\Bfootnote{भ‚वे}} धूमेऽपि न शीत‚स्प‚र्शाभावः, त‚त्क‚थ‚मियं ग‚मिकेत्याह—‚{\tiny $_{lb}$}‚\textbf{स‚म‚स्ते}ति । \edtext{\textsuperscript{*}}{\lemma{*}\Bfootnote{प्र‚तौ स‚र्व‚त्र अव‚व‚र‚केत्यादि दृश्य‚ते टीकायां तु अप‚व‚र‚केति ।}}अप‚व‚र‚क‚ग्र‚ह‚णं शीत‚स्थानोप‚ल‚क्ष‚णार्थ‚म् । \textbf{अप‚व‚र‚कात् निर्यान्तं} निर्ग‚च्छ‚न्त‚म् ।‚{\tiny $_{lb}$}‚ अन्य‚त्र च ग‚म्य‚मानो धूमः क‚थ‚म‚न्य‚त्र शीताभावं साध‚य‚तीत्याह--इहेति । \textbf{इह} विरुद्ध‚कार्यो‚{\tiny $_{lb}$}‚प‚ल‚ब्धौ । \textbf{दृश्य‚मान}श्चासौ \textbf{द्वार‚देश}श्च तेन \textbf{स‚हितः ।}
	\pend% ending standard par
      ‚{\tiny $_{lb}$}‚

	  \pstart \leavevmode% starting standard par
	उप‚प‚त्तिमाह--\textbf{साध्येति । साध्य‚स्य} शीत‚स्प‚र्शाभाव‚स्य \textbf{प्र‚तिप‚त्ति}र‚व‚बोध‚स्त‚स्य प्र‚तिप\textbf{त्ते‚{\tiny $_{lb}$}‚र‚नुस‚र‚णं} निरूप‚णं त‚स्मात् । \textbf{पूर्व‚व‚दि}ति य‚थापूर्वं कार्यानुप‚ल‚म्भे व‚ह्न्याद्य‚भाव‚प्र‚तीतिसाम‚र्थ्या‚{\tiny $_{lb}$}‚यात‚स्तादृशो ध‚र्मी त‚द्व‚त् ।
	\pend% ending standard par
      ‚{\tiny $_{lb}$}‚

	  \pstart \leavevmode% starting standard par
	अय‚म‚स्य भावः--शीत‚स्प‚र्शाभाव‚प्र‚तीतिरेवेयं विमृश्य‚माणाऽव‚श्य‚मेवंविध‚ध‚र्मिण‚माक‚र्ष‚तीति ।
	\pend% ending standard par
      ‚{\tiny $_{lb}$}‚

	  \pstart \leavevmode% starting standard par
	प्र‚योगः पुन‚र‚स्या एवं क‚र्त्त‚व्यः--य‚त्र धूम‚विशेष‚स्त‚त्र शीत‚स्प‚र्शाभावः । य‚था म‚हान‚सादौ ।‚{\tiny $_{lb}$}‚ त‚थाविध‚श्चात्र धूम इति । एत‚च्चात्य‚न्ताभ्यासाज्झ‚टिति धूम‚द‚र्श‚नाच्छीत‚स्प‚र्शा‚{\tiny $_{lb}$}‚भाव‚प्र‚तीत्युद‚ये विरुद्ध‚कार्योप‚ल‚म्भ‚ज‚मेक‚म‚नुमान‚माचार्येणोक्त‚मिति द्र‚ष्ट‚व्य‚म् । अन‚भ्यास‚द‚श‚या‚{\tiny $_{lb}$}‚ ‚{\tiny $_{lb}$}‚ \leavevmode\ledsidenote{\textenglish{132/dm}}‚{\tiny $_{lb}$}‚ 
	  
	भूत‚स्यीपि भाव‚स्येति ध‚र्मिविशेष‚ण‚म् । भूत‚स्य जात‚स्यापि विन‚श्व‚रः स्व‚भावो नाव‚श्य‚म्भावी,‚{\tiny $_{lb}$}‚ किमुताजात‚स्येति अपिश‚ब्दार्थः । \edtext{\textsuperscript{*}}{\lemma{*}\Bfootnote{ज‚न‚नाद् \cite{dp-msA} \cite{dp-edP} \cite{dp-edH} \cite{dp-edE}}}ज‚न‚काद्धेतोर‚न्यो हेतुः हेत्व‚न्त‚रं मुद्ग‚रादि\edtext{}{\lemma{रादि}\Bfootnote{मुद्ग‚रादिः \cite{dp-msC}}} । त‚द‚पेक्ष‚ते‚{\tiny $_{lb}$}‚ विन‚श्व‚रः\edtext{}{\lemma{रः}\Bfootnote{विन‚श्व‚र‚स्यापेक्ष० \cite{dp-msA}}} । त‚स्यापेक्षुणादिति हेतुः । हेत्व‚न्त‚रापेक्ष‚णं \edtext{}{\lemma{णं}\Bfootnote{पेक्ष‚णं नाध्रुव० \cite{dp-msB}}}नामाध्रुव‚भावित्वेन व्याप्तं य‚था‚{\tiny $_{lb}$}‚ वास‚सि राग‚स्य \edtext{}{\lemma{स्य}\Bfootnote{राज‚कादि० \cite{dp-msB} \cite{dp-msC} \cite{dp-msD}}}र‚ञ्ज‚मादिहेत्व‚न्त‚रापेक्ष‚ण‚म‚ध्रुव‚भावित्वेन व्याप्त‚म्\edtext{}{\lemma{म्}\Bfootnote{व्याप्तं त‚द्व‚द् । ध्रुव० \cite{dp-msC}}} । ध्रुव‚भावित्व‚विरुद्धं‚{\tiny $_{lb}$}‚ चाध्रुव‚भावित्व‚म् । विनाश‚श्च विन‚श्व‚र‚स्व‚भावात्मा हेत्व‚न्त‚रापेक्ष इष्टः । त‚तो विरुद्ध‚{\tiny $_{lb}$}‚व्याप्त‚हेत्व‚न्त‚रापेक्ष‚ण‚द‚र्श‚नाद् ध्रुव‚भावित्व‚निषेधः । ‚{\tiny $_{lb}$}‚ 
	  
	इह ध्रुव‚भावित्वं नित्य‚त्व‚म्, अध्रुव‚भावित्वं \edtext{}{\lemma{भावित्वं}\Bfootnote{च नास्ति \cite{dp-msB} \cite{dp-msD}}}चानित्य‚त्व‚म् । नित्य‚त्वानित्य‚त्व‚योश्च‚{\tiny $_{lb}$}‚ प‚र‚स्प‚र‚प‚रिहारेणाव‚स्थानादेक‚त्र विरोधः । त‚था\edtext{}{\lemma{था}\Bfootnote{स्व‚भावानुप‚ल‚ब्धिरूप‚ता स्यात् । स्व‚भावानुप‚ल‚ब्धिरूपा चाभ्युपेता पूर्वाचार्यै‚{\tiny $_{lb}$}‚रित्याह--\cite{dp-msD-n}}} च स‚ति प‚र‚स्प‚र‚प‚रिहार‚व‚तोर्द्व‚योर्य‚दैकं‚{\tiny $_{lb}$}‚ दृश्य‚ते त‚त्र द्वितीय‚स्य तादात्म्य‚निषेधः कार्यः । तादात्म्य‚निषेध‚श्च \edtext{}{\lemma{श्च}\Bfootnote{ध‚श्च त‚याभ्यु० \cite{dp-msA}}}दृश्य‚त‚याऽभ्युप‚ग‚त‚स्य‚{\tiny $_{lb}$}‚ स‚म्भ‚व‚ति । \edtext{\textsuperscript{*}}{\lemma{*}\Bfootnote{य एवं--\cite{dp-msA}}}य‚त एवं तादात्म्य‚निषेधः क्रिय‚ते--य‚द्य‚यं दृश्य‚मानो नित्यो भ‚वेन्नित्य‚रूपो दृश्येत ।‚{\tiny $_{lb}$}‚ न च नित्य‚रूपो दृश्य‚ते । त‚स्मान्न नित्यः । एवं च प्र‚तिषेध्य‚स्य नित्य‚त्व‚स्य दृश्य‚माना\edtext{}{\lemma{माना}\Bfootnote{दृश्य‚मानात्म‚त्व‚म‚भ्यु \cite{dp-msA} \cite{dp-edP} \cite{dp-edH} \cite{dp-edE} \cite{dp-edN}}}‚{\tiny $_{lb}$}‚त्म‚क‚त्व‚म‚भ्युप‚ग‚म्य प्र‚तिषेधःकृतो भ‚व‚ति । \edtext{\textsuperscript{*}}{\lemma{*}\Bfootnote{अथ न व‚स्त्वेक‚त्व‚विरोधोऽन‚योः प‚रं यो निषेधो ध्रुव‚भावित्व‚स्य विधीय‚ते स ।‚{\tiny $_{lb}$}‚ य‚द्य‚या [[?]] दृश्य‚त्वे स‚ति पूर्वानुप‚ल‚ब्धिष्विव भ‚वेत् त‚दा नास्यानुप‚ल‚ब्धेः । अथ भ‚व‚तु‚{\tiny $_{lb}$}‚ नित्य‚त्व‚स्याव‚स्तुन एवं निषेधः, पिशाचादीनां तु स‚तां क‚थं निषेध इत्याह--\cite{dp-msD-n}}}व‚स्तुनोऽप्य‚दृश्य‚स्य पिशाचादेर्य‚दि\edtext{}{\lemma{दि}\Bfootnote{य‚दैव \cite{dp-msB}}}दृश्य‚घ‚टात्म‚क\edtext{}{\lemma{क}\Bfootnote{०त्म‚त्वं० \cite{dp-msA} \cite{dp-edP} \cite{dp-edH} \cite{dp-edE} \cite{dp-edN}}}-‚{\tiny $_{lb}$}‚ पुन‚र‚ज्ञातेऽनुमाने कार्य‚लिङ्ग‚ज‚विरुद्धोप‚ल‚म्भ‚जे भ‚व‚तः । त‚थाहि--य‚त्र धूम‚स्त‚त्र स‚र्व‚त्र व‚ह्निर्य‚थाऽ‚{\tiny $_{lb}$}‚य‚स्कार‚कुट्याम्, धूम‚श्चात्रेति कार्य‚लिङ्ग‚ज‚मेक‚म‚त्र निय‚त‚प्राग्भावि, त‚द‚नु य‚त्र व‚ह्निर्न त‚त्र‚{\tiny $_{lb}$}‚ शीत‚स्प‚र्शो य‚था र‚स‚व‚तीप्र‚देशे, व‚ह्निश्चात्रेति विरुद्धोप‚ल‚म्भ‚जं द्वितीय‚मिति ॥
	\pend% ending standard par
      ‚{\tiny $_{lb}$}‚

	  \pstart \leavevmode% starting standard par
	\textbf{प्र‚तिषेध्य‚स्ये}त्यादिना विरुद्ध‚व्याप्तोप‚ल‚ब्धिं\leavevmode\ledsidenote{\textenglish{52b/ms}} व्याच‚ष्टे । पूर्व‚व‚त्साध्यादिप्र‚द‚र्श‚न‚म् ।
	\pend% ending standard par
      ‚{\tiny $_{lb}$}‚

	  \pstart \leavevmode% starting standard par
	\textbf{किमुते}ति निपात‚स‚मुदायः किम्पुन‚रित्य‚स्यार्थे व‚र्त्त‚ते ।
	\pend% ending standard par
      ‚{\tiny $_{lb}$}‚

	  \pstart \leavevmode% starting standard par
	न‚नु किम‚जात‚स्यापि व‚स्तुनो नाश‚म‚व‚श्यं भाविनं केचिदिच्छ‚न्ति येनापिश‚ब्दः स‚मुच्च‚ये‚{\tiny $_{lb}$}‚ व्याख्याय‚त इति ? नैष दोषः । अजात‚स्य ताव‚द‚निष्ट‚त्वादेव नाव‚श्य‚म्भावी विनाशः,‚{\tiny $_{lb}$}‚ जात‚स्यापि नाव‚श्य‚म्भावीतीत्थं मूलेऽपिश‚ब्दः । केव‚लं \textbf{किमुताजात‚स्ये}ति व्याच‚क्षाणेन \textbf{ध‚र्मोत्त‚रेणा}य‚{\tiny $_{lb}$}‚म‚र्थो न व्य‚क्तीकृतः । त‚त्रापि किम्पुन‚र‚जात‚स्य य‚स्य विनाश एव नेष्ट इत्य‚भिप्रायेण‚{\tiny $_{lb}$}‚ ‚{\tiny $_{lb}$}‚ \leavevmode\ledsidenote{\textenglish{133/dm}}‚{\tiny $_{lb}$}‚ 
	  
	त्व‚निषेधः क्रिय‚ते दृश्यात्म‚क\edtext{}{\lemma{क}\Bfootnote{०त्म‚त्वं० \cite{dp-msA} \cite{dp-edP} \cite{dp-edH} \cite{dp-edE} \cite{dp-edN}}} त्व‚म‚भ्युप‚ग‚म्य क‚र्त्त‚व्यः । \edtext{\textsuperscript{*}}{\lemma{*}\Bfootnote{य‚द्य‚यं दृश्य० \cite{dp-msA} \cite{dp-edP} \cite{dp-edH} \cite{dp-edE} \cite{dp-edN}}}य‚द्य‚यं घ‚टो दृश्य‚मानः पिशाचात्मा‚{\tiny $_{lb}$}‚ भ‚वेत् पिशाचो दृष्टो भ‚वेत् । न च दृष्टः । त‚स्मात् न पिशाच इति । दृश्यात्म‚त्वाभ्युप‚ग‚म‚{\tiny $_{lb}$}‚पूर्व‚को दृश्य‚माने घ‚टादौ\edtext{}{\lemma{टादौ}\Bfootnote{०माने व‚स्तुनि घ‚टादौ \cite{dp-msC}}} व‚स्तुनि व‚स्तुनोऽव‚स्तुनो वा दृश्य‚स्यादृश्य‚स्य च तादात्म्य\edtext{}{\lemma{तादात्म्य}\Bfootnote{निषेधः \cite{dp-msA} \cite{dp-msB} \cite{dp-edP} \cite{dp-edH} \cite{dp-edE} \cite{dp-edN}}}प्र‚तिषेधः ।‚{\tiny $_{lb}$}‚ त‚था च स‚ति य‚था घ‚ट‚स्य दृश्य‚त्व‚म‚भ्युप‚ग‚म्य \edtext{}{\lemma{म्य}\Bfootnote{निपेधः \cite{dp-msB}}}प्र‚तिषेधो दृश्यानुप‚ल‚म्भादेव त‚द्व‚त् स‚र्व‚स्य‚{\tiny $_{lb}$}‚ प‚र‚स्प‚र‚प‚रिहार‚व‚तोऽन्य‚त्र दृश्य‚माने निषेधो दृश्यानुप‚ल‚म्भादेव । त‚था \edtext{}{\lemma{था}\Bfootnote{तादात्म्य‚निषेध‚संसूच‚क‚स्य व्याप‚कानुप‚ल‚ब्धिप्र‚योग‚स्य । स च एवं--नित्य‚स्य‚{\tiny $_{lb}$}‚ स‚त्ता स्थिरोप‚ल‚म्भ‚त्वेन व्याप्ता । त‚स्य स्थिरोप‚ल‚म्भ‚विष‚य‚त्व‚स्य त‚त्र घ‚टादौ अनुप‚ल‚ब्ध्या‚{\tiny $_{lb}$}‚ नित्य‚स‚त्ताया व्याप्ता\add{याः}निषेधः--\cite{dp-msD-n}}}चास्यैवंजातीय‚क‚स्य‚{\tiny $_{lb}$}‚ प्र‚योग‚स्य स्व‚भावानुप‚ल‚ब्धाव‚न्त‚र्भावः ॥‚{\tiny $_{lb}$}‚ योज‚नीयः । य‚द्वा प्राग‚भाव‚स्यानादेर‚जात‚स्यापि नाश‚म‚व‚श्य‚म्भाविनं केचिदिच्छ‚न्तीति त‚द‚पेक्ष‚या‚{\tiny $_{lb}$}‚ \textbf{अपिश‚ब्दः} स‚मुच्च‚ये । \textbf{त‚न्मु}द्ग‚रा\textbf{द्य‚पेक्ष‚ते विन‚श्व‚रो} विनंष्टुमिति शेषः । \textbf{त‚स्य} हेत्व‚न्त‚र‚स्य ।
	\pend% ending standard par
      ‚{\tiny $_{lb}$}‚

	  \pstart \leavevmode% starting standard par
	प्र‚योगः पुन‚रीदृशः क‚र्त्त‚व्यः--य‚द्य‚द‚व‚स्थाप्राप्तौ हेत्व‚न्त‚र‚म‚पेक्ष‚ते न त‚द‚व‚श्यं त‚द्रूपं भ‚व‚ति ।‚{\tiny $_{lb}$}‚ य‚था व‚स्त्रं र‚क्त‚रूप‚ताप‚त्तौ राग‚द्र‚व्य‚संयोगापेक्षं नाव‚श्यं र‚क्तं भ‚व‚ति । अपेक्ष‚ते च भावो‚{\tiny $_{lb}$}‚ विनंष्टुं हेत्व‚न्त‚र‚मिति विरुद्ध‚व्याप्तोप‚ल‚ब्धिप्र‚स‚ङ्ग एषः । अत एव मूलेऽपिश‚ब्दः प्र‚स‚ङ्ग‚साध‚न‚त्व‚{\tiny $_{lb}$}‚प्र‚स‚ङ्गार्थो ल‚क्ष्य‚ते । स्व‚त‚न्त्र‚साध‚नं तु विरुद्ध‚व्याप्तोप‚ल‚म्भाख्य‚मेवं द्र‚ष्ट‚व्य‚म्--यो‚{\tiny $_{lb}$}‚ विरुद्ध‚ध‚र्म‚संस‚र्ग‚वान्नासावेको य‚था द्र‚व‚क‚ठिने । विरुद्ध‚ध‚र्म‚संस‚र्ग‚वांश्च सामान्यादिरिति ।
	\pend% ending standard par
      ‚{\tiny $_{lb}$}‚

	  \pstart \leavevmode% starting standard par
	न‚नु च कोऽर्थ‚योर्विरोधः, किञ्चास्य विरोध‚स्य साध‚कं प्र‚माण‚मित्याश‚ङ्काम‚पाक‚र्त्तु‚{\tiny $_{lb}$}‚माह--इहेति । \textbf{नित्य‚त्व}श‚ब्देनाव‚श्य‚म्भावित्व‚म्, \textbf{अनित्य‚त्व}श‚ब्देनाऽन‚व‚श्य‚भ्भावित्व‚मुक्तं द्र‚ष्ट‚व्य‚म् ।‚{\tiny $_{lb}$}‚ अन्य‚था केन नित्यो विनाशोऽभ्युपेतो येनास्याऽनित्य‚ताऽपो\edtext{}{\lemma{ताऽपो}\Bfootnote{ऽऽपा}}द्येत । य‚च्च पूर्वं \textbf{ध्रुव‚भावित्व}‚{\tiny $_{lb}$}‚श‚ब्दं विवृण्व‚ताऽनेन \textbf{ध्रुव‚म‚व‚श्यं भ‚व‚ती}ति विवृतं त‚च्च व्याह‚न्येत ।
	\pend% ending standard par
      ‚{\tiny $_{lb}$}‚

	  \pstart \leavevmode% starting standard par
	स‚म्प्र‚ति विरोध‚मुप‚पाद‚य‚ति--\textbf{नित्यानित्य‚योरिति । चो} हेतौ ।
	\pend% ending standard par
      ‚{\tiny $_{lb}$}‚

	  \pstart \leavevmode% starting standard par
	इदानीं प‚र‚स्प‚र‚प‚रिहार‚स्थित‚ल‚क्ष‚ण‚विरोध‚व्य‚व‚स्थाप‚कं दृश्यानुप‚ल‚म्भं द‚र्श‚यितुमाह—‚{\tiny $_{lb}$}‚\textbf{त‚था चे}ति । क‚थं दृश्य‚त‚याऽभ्युप\add{ग}त‚स्य निषेध इत्याह--\textbf{य‚त} इति । \textbf{एवं} व‚क्ष्य‚माणेन‚{\tiny $_{lb}$}‚ प्र‚कारेण । त‚मेवाह--\textbf{य‚द्य‚य‚मि}ति । \textbf{न च} नैवं \textbf{नित्य‚रूपो}ऽव‚श्य‚म्भाविस्व‚रूपो \textbf{दृश्य‚ते} प्र‚तीय‚ते ।‚{\tiny $_{lb}$}‚ य‚द्य‚प्येवं त‚थापि दृश्यात्म‚काभ्युप‚ग‚म इत्याह--\textbf{एव‚मि}ति \textbf{चो} य‚स्मात् । \textbf{एव}म‚न‚न्त‚रोक्तेन‚{\tiny $_{lb}$}‚ क्र‚मेण दृश्य‚मान‚स्यादृश्येनाव‚स्तुनाऽन्योन्य‚प‚रिहार‚स्थित‚ल‚क्ष‚ण‚विरोध‚व्य‚व‚स्थायां ताव‚देवं दृश्यानु‚{\tiny $_{lb}$}‚प‚ल‚म्भ उपायः । व‚स्तुनाऽप्य‚दृश्येन त‚थात्व‚व्य‚व‚स्थायाम‚य‚मेवोपाय इति द‚र्श‚यितुं \textbf{व‚स्तुनोऽपी}त्या‚{\tiny $_{lb}$}‚दिनोप‚क्र‚म‚ते । न केव‚लं क‚ल्पित‚स्याव‚स्तुन \textbf{इत्य‚पि}श‚ब्दः । \textbf{इति}र्हेतावेव‚म‚र्थे वा । \textbf{दृश्य‚स्ये}ति‚{\tiny $_{lb}$}‚ व‚स्त्व‚पेक्ष‚या ।
	\pend% ending standard par
      ‚{\tiny $_{lb}$}‚‚{\tiny $_{lb}$}‚\textsuperscript{\textenglish{134/dm}}‚{\tiny $_{lb}$}‚
	  \bigskip
	  \begingroup
	
	  \bigskip
	  \begingroup
	

	  \pstart \leavevmode% starting standard par
	कार्य‚विरुद्धोप‚ल‚ब्धिर्य‚था--नेहाप्र‚तिब‚द्ध‚साम‚र्थ्यानि शीत‚कार‚णानि स‚न्ति,‚{\tiny $_{lb}$}‚ \edtext{\textsuperscript{*}}{\lemma{*}\Bfootnote{अग्नेरिति--\cite{dp-msB} \cite{dp-msC} \cite{dp-edP} \cite{dp-edH} \cite{dp-edE} \cite{dp-edN}}}व‚ह्नेरिति ॥ ३७ ॥
	\pend% ending standard par
      
	  \endgroup
	‚{\tiny $_{lb}$}‚ 

	  \pstart \leavevmode% starting standard par
	प्र‚तिषेध्य‚स्य य‚त् कार्यं त‚स्य य‚द्विरुद्धं त‚स्योप‚ल‚ब्धेरुदाह‚र‚ण‚म्--य‚थेति । इहेति ध‚र्मी ।‚{\tiny $_{lb}$}‚ अप्र‚तिब‚द्धं साम‚र्थ्य येषां शीत‚कार‚णानां शीत‚ज‚न‚नं प्र‚ति\edtext{}{\lemma{ति}\Bfootnote{प्र‚ति न तानि स‚न्ति इति--\cite{dp-msA} \cite{dp-edP} \cite{dp-edH} \cite{dp-edE} \cite{dp-edN}}}, तानि न स‚न्ति इति साध्य‚म् ।‚{\tiny $_{lb}$}‚ व‚ह्नेरिति हेतुः ।
	\pend% ending standard par
       ‚{\tiny $_{lb}$}‚ 

	  \pstart \leavevmode% starting standard par
	य‚त्र शीत‚कार‚णानि अदृश्यानि, शीत‚स्प‚र्शोऽप्य‚दृश्यः, त‚त्रायं हेतुः प्र‚योक्त‚व्यः । दृश्य‚त्वे‚{\tiny $_{lb}$}‚ तु शीत‚स्प‚र्श‚स्य त‚त्कार‚णानां वा कार्यानुप‚ल‚ब्धिर्दृश्यानुप‚ल‚ब्धिर्वा ग‚मिका । त‚स्मादेषाप्य‚{\tiny $_{lb}$}‚भाव‚साध‚नी । त‚तो य‚स्मिन् \edtext{}{\lemma{स्मिन्}\Bfootnote{देशे \cite{dp-msA} \cite{dp-msB} \cite{dp-msC} \cite{dp-msD} \cite{dp-edP} \cite{dp-edH} \cite{dp-edE} \cite{dp-edN}}}उद्देशे स‚द‚पि शीत‚कार‚ण‚म‚दृश्यं शीत‚स्प‚र्श‚श्च\edtext{}{\lemma{श्च}\Bfootnote{अदृश्यः--\cite{dp-msD-n}}} दूर‚स्थ‚त्वात्‚{\tiny $_{lb}$}‚ प्र‚तिप‚त्तुर्व‚ह्निर्भास्व‚र‚व‚र्ण‚त्वाद् दूराद‚पि दृश्य‚स्त‚त्रायं प्र‚योग इति\edtext{}{\lemma{इति}\Bfootnote{इति नास्ति \cite{dp-msA} \cite{dp-msB} \cite{dp-edP} \cite{dp-edH} \cite{dp-edE} \cite{dp-edN}}} ॥
	\pend% ending standard par
      
	  \endgroup
	‚{\tiny $_{lb}$}‚

	  \pstart \leavevmode% starting standard par
	वाश‚ब्दार्थ‚श्च‚कार इति केचित् ।
	\pend% ending standard par
      ‚{\tiny $_{lb}$}‚

	  \pstart \leavevmode% starting standard par
	\hphantom{.}अन्ये तु अव‚स्तुनोऽदृश्य‚त्व‚स्य सिद्ध‚त्वात् किं त‚द‚नुवादेन कार्य‚म् ? त‚तो द्व‚य‚म‚प्येत‚द्‚{\tiny $_{lb}$}‚ व‚स्त्व‚पेक्ष‚या योज्य‚म् । \textbf{व‚स्तुनो दृश्य‚स्य} घ‚टादेः, \textbf{अदृश्य‚स्य} पिशाचादेः । अन्य‚था पिशाचादिव‚स्तुन‚{\tiny $_{lb}$}‚स्त‚थात्वं नोक्तं स्यात् । प्र‚कृतं च त‚देव इति प्र‚तिप‚न्नाः ।
	\pend% ending standard par
      ‚{\tiny $_{lb}$}‚

	  \pstart \leavevmode% starting standard par
	भ‚व‚त्वेवं त‚तः किं सिद्ध‚मित्याह--त‚था \textbf{च स‚ती}ति । \textbf{त‚द्व‚द्} घ‚ट‚व‚त् । \textbf{अन्य}\leavevmode\ledsidenote{\textenglish{53a/ms}}\textbf{त्र}‚{\tiny $_{lb}$}‚ अन्य‚स्मिन् दृश्य‚माने व‚स्तुनि ।
	\pend% ending standard par
      ‚{\tiny $_{lb}$}‚

	  \pstart \leavevmode% starting standard par
	न‚नु भ‚व‚तु दृश्यानुप‚ल‚म्भाद् दृश्य‚मानेङ्गुल्यादौ स‚र्व‚स्य सुमेर्वादेस्तादात्म्य‚निषेध‚स्त‚थाप्यु‚{\tiny $_{lb}$}‚क्तास्व‚नुप‚ल‚ब्धिषु कुत्राय‚म‚न्त‚र्थ‚व‚तीत्याश‚ङ्काम‚पाक‚र्त्तुमुप‚संहार‚व्याजेनातिदेश‚म‚प्याह--\textbf{एव‚मिति ।‚{\tiny $_{lb}$}‚ एवंजातीय‚क‚स्यै}व‚म्प्र‚कार‚व‚तः । एव‚म्प्र‚कार‚स्येत्युक्ते व‚च‚न‚भं \edtext{}{\lemma{भं}\Bfootnote{व‚च‚न‚ल‚भ्यं ?}} स्यात् । त‚त्रान्त‚र्भावो‚{\tiny $_{lb}$}‚ द‚र्शित एवेति भावः ।
	\pend% ending standard par
      ‚{\tiny $_{lb}$}‚

	  \pstart \leavevmode% starting standard par
	अथ य‚दि दृश्यानुप‚ल‚म्भाद‚न्य‚त्रान्य‚स्य दृश्य‚स्यादृस्य‚श्य वा तादात्म्य‚निषेधः क‚थं विरुद्धे\edtext{}{\lemma{विरुद्धे}\Bfootnote{द्ध}}‚{\tiny $_{lb}$}‚व्याप्तोप‚ल‚ब्धेर‚व‚तार इति चेत् । न दोषः । विरोध‚प्र‚तिप‚त्तिकाले दृश्यानुप‚ल‚म्भ‚स्य व्यापारात् ।‚{\tiny $_{lb}$}‚ त‚द‚व‚ग‚त‚विरोधेन तु व्याप्तं य‚त्र दृश्य‚ते विरुद्ध‚व्याप्तोप‚ल‚म्भादेव विव‚क्षिताभाव‚प्र‚तीतिरिति‚{\tiny $_{lb}$}‚ किम‚व‚द्य‚म् ॥
	\pend% ending standard par
      ‚{\tiny $_{lb}$}‚

	  \pstart \leavevmode% starting standard par
	प्र‚तिषेध्य‚स्येत्यादिना कार्य‚विरुद्धोप‚ल‚ब्धिं विवृणोति । पूर्व‚व‚द् ध‚र्म्यादिप्र‚द‚र्श‚न‚म् ।‚{\tiny $_{lb}$}‚ \textbf{व‚ह्नेरि}ति शीत‚निव‚र्त्त‚न‚क्ष‚माद् विशिष्टादिति द्र‚ष्ट‚व्य‚म् । अन्य‚था प्र‚ति\edtext{}{\lemma{ति}\Bfootnote{दी}}पाद्यात्म‚नः शीतऽ‚{\tiny $_{lb}$}‚निव‚र्त्त‚क‚त्वेनानैकान्तिक‚ताप‚त्तेः ।
	\pend% ending standard par
      ‚{\tiny $_{lb}$}‚‚{\tiny $_{lb}$}‚\textsuperscript{\textenglish{135/dm}}‚{\tiny $_{lb}$}‚
	  \bigskip
	  \begingroup
	
	  \bigskip
	  \begingroup
	

	  \pstart \leavevmode% starting standard par
	व्याप‚क‚विरुद्धोप‚ल‚ब्धिर्य‚था--नात्र तुषार‚स्प‚र्शो \edtext{}{\lemma{र्शो}\Bfootnote{अग्नेरिति \cite{dp-msB} \cite{dp-msC} \cite{dp-edP} \cite{dp-edH} \cite{dp-edN} \cite{dp-edE}}}व‚ह्नेरिति ॥ ३८ ॥
	\pend% ending standard par
      
	  \endgroup
	‚{\tiny $_{lb}$}‚ 

	  \pstart \leavevmode% starting standard par
	प्र‚तिषेध्य‚स्य य‚द् व्याप‚कं तेन य‚द् विरुद्धं त‚स्योप‚ल‚ब्धिरुदाह‚र्त्त‚व्या य‚थेति । अत्रेति‚{\tiny $_{lb}$}‚ ध‚र्मी । तुषार‚स्प‚र्शो नेति साध्य‚म् । \edtext{\textsuperscript{*}}{\lemma{*}\Bfootnote{अग्नेरिति \cite{dp-msD} \cite{dp-msB}}}व‚ह्नेरिति हेतुः ।
	\pend% ending standard par
       ‚{\tiny $_{lb}$}‚ 

	  \pstart \leavevmode% starting standard par
	य‚त्र \edtext{}{\lemma{त्र}\Bfootnote{य‚त्र प्र‚तिषेध्य‚तुषार० \cite{dp-msC}}}व्याप्य‚स्तुषार‚स्प‚र्शो व्याप‚क‚श्च\edtext{}{\lemma{श्च}\Bfootnote{व्याप‚कं च शी० \cite{dp-msD}}} शीत‚स्प‚र्शो न दृश्य‚स्त‚त्रायं हेतुः । त‚योर्दृश्य‚त्वे\edtext{}{\lemma{त्वे}\Bfootnote{य‚थासंख्य‚म्--\cite{dp-msD-n}}}‚{\tiny $_{lb}$}‚ स्व‚भाव‚स्य व्याप‚क‚स्य चानुप‚ल‚ब्धिर्य‚तः प्र‚योक्त‚व्या\edtext{}{\lemma{व्या}\Bfootnote{प्र‚योग‚श्चैव‚म्--नात्र तुषार‚स्प‚र्शः, उप‚ल‚ब्धिल‚क्ष‚ण‚प्राप्त‚स्यानुप‚ल‚ब्धेः । नात्र तुषार‚{\tiny $_{lb}$}‚स्प‚र्शः, शीत‚स्प‚र्शाभावात् ।}} । त‚था च\edtext{}{\lemma{च}\Bfootnote{च नास्ति \cite{dp-msB} \cite{dp-edP} \cite{dp-edE}}} स‚त्य‚भाव‚साध‚नीय‚म् ।‚{\tiny $_{lb}$}‚ दूर‚व‚र्त्तिन‚श्च प्र‚तिप‚त्तुस्तुषार‚स्प‚र्शः शीत‚स्प‚र्श‚विशेषः, शीत‚मात्रं \edtext{}{\lemma{मात्रं}\Bfootnote{च नास्ति \cite{dp-msB}}}च प‚रोक्ष‚म् । व‚ह्निस्तु‚{\tiny $_{lb}$}‚ रूप‚विशेषाद् दूर‚स्थोऽपि प्र‚त्य‚क्षः । त‚तो व‚ह्नेः शीत‚मात्राभावः । त‚तः शीत‚विशेषेतुषार‚स्प‚र्शा‚{\tiny $_{lb}$}‚भाव‚निश्च‚यः । शीत‚विशेष‚स्य शीत‚सामान्येन व्याप्त‚त्वादिति \edtext{}{\lemma{त्वादिति}\Bfootnote{विशिष्टे विष० \cite{dp-msB}}}विशिष्ट‚विष‚येऽस्याः प्र‚योगः ॥
	\pend% ending standard par
       ‚{\tiny $_{lb}$}‚ 
	  \bigskip
	  \begingroup
	

	  \pstart \leavevmode% starting standard par
	कार‚णानुप‚ल‚ब्धिर्य‚था--नात्र धूमो \edtext{}{\lemma{धूमो}\Bfootnote{अग्न्य‚भावादिति । \cite{dp-msB} \cite{dp-msD} \cite{dp-edP} \cite{dp-edH} \cite{dp-edN} अग्न्य‚भावात् \cite{dp-edE}}}व‚हाय‚भावादिति ॥ ३९ ॥
	\pend% ending standard par
      
	  \endgroup
	‚{\tiny $_{lb}$}‚ 

	  \pstart \leavevmode% starting standard par
	प्र‚तिषेध्य‚स्य य‚त् कार‚णं त‚स्यानुप‚ल‚ब्धेरुदाह‚र‚णं य‚थेति । अत्रेति \textbf{ध‚र्मी} । न धूम
	\pend% ending standard par
      
	  \endgroup
	‚{\tiny $_{lb}$}‚

	  \pstart \leavevmode% starting standard par
	कीदृशि विष‚येऽस्याः प्र‚योग इत्याह--य‚त्रेति । न केव‚लं पूर्व इत्य‚पिश‚ब्दः । अभावो‚{\tiny $_{lb}$}‚ऽभाव‚व्य‚व‚हार‚श्चाभाव‚श‚ब्देनोक्तः ।
	\pend% ending standard par
      ‚{\tiny $_{lb}$}‚

	  \pstart \leavevmode% starting standard par
	स्यान्म‚त‚म् । क‚थं पुनः शीत‚स्प‚र्श‚शीत‚कार‚णेऽदृश्ये व‚ह्निस्तु दृश्यः स‚म्भ‚व‚ति येनास्याः‚{\tiny $_{lb}$}‚ प्र‚योगो घ‚ट‚त इत्याह--\textbf{य‚स्मिन्नि}ति । उद्देशे प्र‚देशे \textbf{प्र‚तिप‚त्तुर्दूर‚स्थ‚त्वादिति} शीत‚स्प‚र्श‚शीत‚कार‚ण‚यो‚{\tiny $_{lb}$}‚र‚दृश्य‚त्वे कार‚ण‚म् । \textbf{भास्व‚र‚व‚र्ण‚त्वादि}ति व‚ह्नेर्दृश्य‚त्वे निब‚न्ध‚न‚म् । \textbf{भास्व‚रो} भास‚न‚शीलो‚{\tiny $_{lb}$}‚ \textbf{व‚र्णो} य‚स्य त‚द्भाव‚स्त‚स्मात् । न केव‚लं निक‚ट इत्य‚पिश‚ब्दः । त‚त्र त‚स्मिन् देशे ।
	\pend% ending standard par
      ‚{\tiny $_{lb}$}‚

	  \pstart \leavevmode% starting standard par
	प्र‚योगः पुन‚रेवं कार्यः--य‚त्र विशिष्टो व‚ह्निर्न त‚त्र शीतोप‚ज‚न‚नाप्र‚तिब‚द्ध‚श‚क्तीनि‚{\tiny $_{lb}$}‚ शीत‚कार‚णानि । य‚था क्व‚चिद‚नुभूते प्र‚देशे । त‚थाभूत‚श्चात्र व‚ह्निरिति ।
	\pend% ending standard par
      ‚{\tiny $_{lb}$}‚

	  \pstart \leavevmode% starting standard par
	एत‚च्चाभ्यासाज्झ‚टिति व‚ह्निद‚र्श‚नेन त‚थाभूत‚शीत‚कार‚णाभाव‚प्र‚तीतिज‚न्म‚न्येकं कार्य‚{\tiny $_{lb}$}‚विरुद्धोप‚ल‚म्भ‚ज‚म‚नुमान‚मुक्त‚माचार्येणेति द्र‚ष्ट‚व्य‚म् । अन्य‚था तु विरुद्धोप‚ल‚म्भ‚कार्यानुप‚ल‚म्भ‚जे‚{\tiny $_{lb}$}‚ द्वे एते अनुमाने । त‚था हि य‚त्र व‚ह्निर्न त‚त्र शीत‚स्प‚र्श इति स्व‚भाव‚विरुद्धोप‚ल‚म्भ‚ज‚मेक‚{\tiny $_{lb}$}‚म‚नुमान‚म् । य‚त्र च य‚त्कार्यं नास्ति, न त‚त्र त‚त्कार‚णं त‚ज्ज‚न‚नाप्र‚तिब‚द्ध‚साम‚र्थ्य‚म‚स्तीति‚{\tiny $_{lb}$}‚ कार्यानुप‚ल‚म्भ‚जं द्वितीय‚मिति ॥
	\pend% ending standard par
      ‚{\tiny $_{lb}$}‚‚{\tiny $_{lb}$}‚\textsuperscript{\textenglish{136/dm}}‚{\tiny $_{lb}$}‚
	  \bigskip
	  \begingroup
	

	  \pstart \leavevmode% starting standard par
	इति साध्य‚म् । व‚ह्न्य‚भावादिति हेतुः । य‚त्र कार्यं स‚द‚पि \edtext{}{\lemma{पि}\Bfootnote{स‚द‚पि न दृश्यं भ‚व‚ति \cite{dp-msC} स‚द‚पि दृश्यं न भ‚व‚ति \cite{dp-msA} \cite{dp-edP} \cite{dp-edH} \cite{dp-edE} \cite{dp-edN}}}अदृश्यं भ‚व‚ति त‚त्रायं प्र‚योगः ।‚{\tiny $_{lb}$}‚ दृश्ये तु कार्ये दृश्यानुप‚ल‚ब्धिर्ग‚मिका । त‚तोऽय‚म‚प्य‚भाव‚साध‚नः\edtext{}{\lemma{नः}\Bfootnote{०साध‚कः \cite{dp-msC}}} । निष्क‚म्पाय‚त‚स‚लिल‚पूरिते‚{\tiny $_{lb}$}‚ ह्र‚दे हेम‚न्तोचित‚बाष्प‚योद्ग‚मे विर‚ले स‚न्ध्यात‚म‚सि स‚ति स‚न्न‚पि त‚त्र धूमो न दृश्य‚त\edtext{}{\lemma{त}\Bfootnote{दृश्य इति \cite{dp-msA} \cite{dp-edP} \cite{dp-edH} \cite{dp-edE} \cite{dp-edN}}} इति‚{\tiny $_{lb}$}‚ कार‚णानुप‚ल‚ब्ध्या \edtext{}{\lemma{ब्ध्या}\Bfootnote{प्र‚तिषिध्य‚ते \cite{dp-msB}}}प्र‚तिषेध्य‚ते । व‚ह्निस्तु य‚दि त‚स्याम्भ‚स उप‚रि प्ल‚व‚मानो भ‚वेत्\edtext{}{\lemma{वेत्}\Bfootnote{भ‚वेज्ज्व‚लितो रूप० \cite{dp-msA} \cite{dp-edP} \cite{dp-edH} \cite{dp-edN} भ‚वेज्ज्व‚लित‚रूप० \cite{dp-edE}}} प्र‚ज्व‚लितो,‚{\tiny $_{lb}$}‚ रूप‚विशेषादेवोप‚ल‚ब्धो भ‚वेत् । अज्व‚लित‚स्तु \edtext{}{\lemma{स्तु}\Bfootnote{तु व‚न‚म‚ध्य० \cite{dp-msB}}}इन्ध‚न‚म‚ध्य‚निविष्टो भ‚वेत् । त‚त्रापि द‚ह‚नाधि‚{\tiny $_{lb}$}‚क‚र‚ण‚मिन्ध‚नं प्र‚त्य‚क्ष‚मिति स्व‚रूपेण, आधार‚रूपेण वा दृश्य\edtext{}{\lemma{दृश्य}\Bfootnote{दृश्य‚मान‚रूप एव \cite{dp-msC}}} एव व‚ह्निरिति त‚त्रास्य\edtext{}{\lemma{त्रास्य}\Bfootnote{त‚त्रास्याः प्र‚योगः \cite{dp-msA} \cite{dp-msB} \cite{dp-msC} \cite{dp-msD} \cite{dp-edP} \cite{dp-edH} \cite{dp-edE} \cite{dp-edN}}} प्र‚योग‚{\tiny $_{lb}$}‚ \edtext{\textsuperscript{*}}{\lemma{*}\Bfootnote{इति नास्ति \cite{dp-msA} \cite{dp-msB} \cite{dp-edP} \cite{dp-edH} \cite{dp-edE} \cite{dp-edN}}}इति ॥
	\pend% ending standard par
      
	  \endgroup
	‚{\tiny $_{lb}$}‚

	  \pstart \leavevmode% starting standard par
	व्याप‚क‚विरुद्धोप‚ल‚ब्धिं व्याख्यातुमाह--\textbf{प्र‚तिषे}ध्येत्यादि । पूर्व‚व‚द् ध‚र्म्यादिप्र‚द‚र्श‚न‚म् ।‚{\tiny $_{lb}$}‚ अत्रापि विशिष्टाद् \textbf{व‚ह्नेरि}ति द्र‚ष्ट‚व्य‚म् ।
	\pend% ending standard par
      ‚{\tiny $_{lb}$}‚

	  \pstart \leavevmode% starting standard par
	अस्यापि प्र‚योग‚विष‚य‚माह--\textbf{य‚त्रेति} । क‚थं त‚योर‚दृश्य‚त्व‚म्, व‚ह्नेश्च दृश्य‚त्व‚म्, क‚थं‚{\tiny $_{lb}$}‚ च न शीत‚स्प‚र्श एव तुषार‚स्प‚र्श इत्याश‚ङ्कात्रित‚य‚म‚पाकुर्व‚न्नाह--\textbf{दूरेति । चो} हेतौ निय‚मे वा ।‚{\tiny $_{lb}$}‚ त‚योर्भेद‚मुप‚पाद‚य‚ति \textbf{तुषारे}ति । \textbf{शीत‚मात्र‚म}शीत‚व्यावृत्तिमात्र‚म् । \textbf{त‚तः} शीत‚मात्राऽभावात् ।‚{\tiny $_{lb}$}‚ \textbf{शीत‚विशेष‚श्चासौ तुषार‚स्प‚र्श}श्चेति विग्र‚हः ।
	\pend% ending standard par
      ‚{\tiny $_{lb}$}‚

	  \pstart \leavevmode% starting standard par
	क‚थं त‚द‚भाव‚निश्च‚य इत्याह--\textbf{शीत‚विशेष‚स्ये}ति । एष च वास्त‚वो निवृत्तिक्र‚मः‚{\tiny $_{lb}$}‚ प‚राम‚र्श‚द‚शायां द‚र्शितो न तु त‚त्प्र‚योग\leavevmode\ledsidenote{\textenglish{53b/ms}}कालिकः । त‚थात्वे हि नैक‚म‚नुमान‚मिदं स्यात् ।‚{\tiny $_{lb}$}‚ \textbf{इती}त्यादिनोप‚संह‚र‚ति । \textbf{इति}रेव‚म‚र्थे । त‚स्माद‚र्थे वा । एतेन य‚द् य‚त्र निय‚त‚स‚होप‚ल‚म्भं‚{\tiny $_{lb}$}‚ त‚त्त‚तो न भिद्य‚ते । य‚थैक‚स्माच्च‚न्द्र‚म‚सो द्वितीय‚श्च‚न्द्र‚मा । निय‚त‚स‚होप‚ल‚म्भ‚स्तु नीलादिर्ज्ञानेने‚{\tiny $_{lb}$}‚त्यादि द‚र्शितं द्र‚ष्ट‚व्य‚म् ॥
	\pend% ending standard par
      ‚{\tiny $_{lb}$}‚

	  \pstart \leavevmode% starting standard par
	कार‚णानुप‚ल‚ब्धिं विव‚रिषुराह--\textbf{प्र‚तिषेध्ये}ति ।
	\pend% ending standard par
      ‚{\tiny $_{lb}$}‚

	  \pstart \leavevmode% starting standard par
	न‚नु द्व‚योर‚पि तुल्य‚स्व‚ज्ञान‚ज‚न‚न‚योग्य‚तारूप‚त्वात् तुल्य‚दृश्य‚त्व‚मिति क‚थ‚म‚स्याः प्र‚योग‚{\tiny $_{lb}$}‚ इत्याश‚ङ्क्य विष‚य‚म‚स्या द‚र्श‚यितुं \textbf{य‚त्रे}त्यादिनोप‚क्र‚म‚ते ।
	\pend% ending standard par
      ‚{\tiny $_{lb}$}‚

	  \pstart \leavevmode% starting standard par
	न‚नु म‚नोमोद‚कोप‚योग‚मात्र‚मेत‚त् न पुन‚रीदृशो विष‚योऽस्ति य‚त्राग्निरेव दृश्यो न धूम‚{\tiny $_{lb}$}‚ इति क‚थं पूर्वोक्तातिक्र‚म इत्याह--\textbf{निष्क‚म्पे}ति । ह्र‚दो ज‚लाधार‚विशेषः । \textbf{निर्ग‚तः क‚म्प}श्च‚ल‚नं‚{\tiny $_{lb}$}‚ य‚स्मात्स त‚था स चासा\textbf{वाय‚तो} म‚हानिति त‚था । स चासौ स‚लिल‚पूरित‚श्चेत्येवं विग्र‚हः कार्यः ।‚{\tiny $_{lb}$}‚ \textbf{आय‚त}ग्र‚ह‚णेन ह्र‚द‚स्य म‚ह‚त्त्वाद् बाष्पे भूय‚स्त्व‚म‚त एव धूम‚स्य त‚तो भेदेनानुप‚ल‚क्ष‚ण‚मिति द‚र्श‚य‚ति ।
	\pend% ending standard par
      ‚{\tiny $_{lb}$}‚

	  \pstart \leavevmode% starting standard par
	पुनः किं विशिष्टे ? हेम‚न्ते हेम‚न्त‚संज्ञ‚के काले । उचितोऽधिकृत‚श्चासौ \textbf{बाष्प}श्चेति‚{\tiny $_{lb}$}‚ ‚{\tiny $_{lb}$}‚ \leavevmode\ledsidenote{\textenglish{137/dm}}‚{\tiny $_{lb}$}‚ 
	  
	कार‚ण‚विरुद्धोप‚ल‚धिर्य‚था--नास्य रोम‚ह‚र्षादिविशेषाः,‚{\tiny $_{lb}$}‚ स‚न्निहित‚द‚ह‚न‚विशेष‚त्वादिति ॥ ४० ॥‚{\tiny $_{lb}$}‚ 
	  
	प्र‚तिषेध्य‚स्य य‚त् कार‚णं त‚स्य य‚द्विरुद्धं त‚स्योप‚ल‚ब्धेरुदाह‚र‚णं य‚थेति । अस्येति‚{\tiny $_{lb}$}‚ ध‚र्मी । रोम्णां ह‚र्ष उद्भेदः । स आदिर्येषां द‚न्त‚वीणादीनां शीत‚कृतानाम्, ते विशि \edtext{}{\lemma{विशि}\Bfootnote{विशेष्य‚न्ते \cite{dp-msB}}}ष्य‚न्ते‚{\tiny $_{lb}$}‚ त‚द‚न्येभ्यो भ‚य‚श्र‚द्धादिकृतेभ्य इति रोम‚ह‚र्षादिविशेषाः । ते न स‚न्तीति साध्य‚म् । द‚ह‚न एव‚{\tiny $_{lb}$}‚ विशिष्य‚ते\edtext{}{\lemma{ते}\Bfootnote{विशेष्य‚न्तेऽन्य‚स्मा० \cite{dp-msB} विशिष्य‚तेऽन्य‚स्मा० \cite{dp-msC} \cite{dp-msD}}} त‚द‚न्य‚स्माद्द‚ह‚नाच्छीत‚निव‚र्त्त‚न‚साम‚र्थ्येनेति द‚ह‚न‚विशेषः । क‚श्चिद् द‚ह‚नः स‚न्न‚पि‚{\tiny $_{lb}$}‚ न शीत‚निव‚र्त्त‚न‚क्ष‚मो य‚था प्र‚दीपः । तादृश‚निवृत्त‚ये विशेष‚ग्र‚ह‚ण‚म् । स‚न्निहितो द‚ह‚न‚विशेषो‚{\tiny $_{lb}$}‚ य‚स्य स त‚थोक्तः । त‚स्य भाव‚स्त‚स्मादिति हेतुः । य‚त्र शीत‚स्प‚र्शः स‚न्न‚प्य‚दृश्यो रोम‚ह‚र्षादि‚{\tiny $_{lb}$}‚विशेषाश्चादृश्याः, त‚त्रायं प्र‚योगः । रोम‚ह‚र्षादिविशेष‚स्य दृश्य‚त्वे दृश्यानुप‚ल‚ब्धिः प्र‚योक्त‚व्या ।‚{\tiny $_{lb}$}‚ शीत‚स्प‚र्श‚स्य दृश्य‚त्वे कार‚णानुप‚ल‚ब्धिः । त‚स्माद‚भाव‚साध‚नोऽय‚म् । रूप‚विशेषाद्धि दूराद्‚{\tiny $_{lb}$}‚ त‚था त‚स्योद्ग‚म ऊर्ध्वं ग‚म‚नं य‚स्मिन् य‚स्माद् \textbf{वा} स त‚था । काल‚विशेषेऽप्य‚स्याः प्र‚योग इति‚{\tiny $_{lb}$}‚ द‚र्श‚य‚ति विर‚ले । स‚न्ध्याकालोचितं त‚मः \textbf{स‚न्ध्यात‚मः} त‚स्मिन् विर‚ले म‚न्द‚प्र‚चारे । कुत्र ? ह्र‚दे‚{\tiny $_{lb}$}‚ त‚थाविधे च त‚म‚सि स‚ति । \textbf{इति}स्त‚स्माद् । व‚ह्नेर‚पि त‚त्रेयं ग‚तिर्भ‚विष्य‚तीत्याह—‚{\tiny $_{lb}$}‚\textbf{व‚ह्निस्त्विति} । तुः पूर्व‚व‚त् । \textbf{अम्भ‚स} इति ष‚ष्ठी पुनः ष‚ष्ठ्य‚त‚स‚र्थे \href{http://sarit.indology.info/?cref=Pā.2.3.30}{पाणिनि २. ३. ३०}‚{\tiny $_{lb}$}‚त्यादिना उप‚रिश‚ब्द‚स्यात‚स‚र्थ‚प्र‚त्य‚यान्त‚त्वात् । त‚थाहि ऊर्ध्वं\add{र्ध्व}श‚ब्दाद् उप‚र्युप‚रिष्टाद् ‚{\tiny $_{lb}$}‚ \href{http://sarit.indology.info/?cref=Pā.5.3.31}{पाणिनि ५. ३. ३१}इतिरित्प्र‚त्य‚यो निपातितः । तेनैव सूत्रेणोर्ध्व‚श‚ब्द‚स्योपादेशोऽपि ।
	\pend% ending standard par
      ‚{\tiny $_{lb}$}‚

	  \pstart \leavevmode% starting standard par
	\textbf{प्ल‚व‚मानो}ऽव‚तिष्ठ‚मानः । अनेकार्थ‚त्वाद् धातोर्ग‚च्छ‚न्निति वा । \textbf{स्व‚रूपेण} ज्वाला‚{\tiny $_{lb}$}‚रूपेणाधार‚रूपेणेन्ध‚न‚निविष्टेन । \textbf{इति}र्हेतौ । \textbf{त‚त्र} त‚स्मिन् स्थान‚विशेषे । \textbf{अस्य} कार‚णानुप‚ल‚म्भ‚स्य ।
	\pend% ending standard par
      ‚{\tiny $_{lb}$}‚

	  \pstart \leavevmode% starting standard par
	त‚मिस्रायामेव तु रात्रौ निराधार‚के प्र‚देशे कार‚णानुप‚ल‚ब्धेः प्र‚योगः सुक‚रः, त‚त्र व‚ह्नेर्दृश्य‚{\tiny $_{lb}$}‚त्वाद् धूम‚स्य स‚तोऽप्य‚दृश्य‚त्वात् । अनेन पुन‚रेवंविधं विष‚यं प‚रित्य‚ज्यान्यं विष‚य‚मुप‚पाद‚य‚ता‚{\tiny $_{lb}$}‚ किमित्यात्माऽऽयासित इति न प्र‚तीमः ।
	\pend% ending standard par
      ‚{\tiny $_{lb}$}‚

	  \pstart \leavevmode% starting standard par
	एवं तु प्र‚योगः कार्यः--य‚त्र य‚स्य कार‚णं नास्ति न त‚त्त‚त्रास्ति । य‚था बीजाभावेऽ‚{\tiny $_{lb}$}‚ ङ्कुरः । नास्ति चात्र धूम‚स्य कार‚णं व‚ह्निरिति ।
	\pend% ending standard par
      ‚{\tiny $_{lb}$}‚

	  \pstart \leavevmode% starting standard par
	एतेन य‚त्र य‚त्र विज्ञान‚स्य कार‚णं विज्ञानं नास्ति न त‚त्र विज्ञान‚मुप‚प‚द्य‚ते । य‚थोप‚ल‚श‚क‚ले ।‚{\tiny $_{lb}$}‚ नास्ति च प्राग्भ‚वीयं विज्ञानं क‚ल‚लाव‚स्थायामिति कार‚णानुप‚ल‚ब्धिप्र‚स‚ङ्गः सूचित‚स्तुल्य‚न्यायार्थः ॥
	\pend% ending standard par
      ‚{\tiny $_{lb}$}‚

	  \pstart \leavevmode% starting standard par
	कार‚ण‚विरुद्धोप‚ल‚ब्धिं व्याख्यातुमाह \textbf{प्र‚तिषेध्य‚स्ये}ति । पूर्व‚व‚द् ध‚र्म्यादिप्र‚द‚र्श‚न‚म् । \textbf{उद्भेदः}‚{\tiny $_{lb}$}‚ पुल‚क इत्य‚र्थः । \textbf{द‚न्त‚वीणा}ऽध‚रोप‚रिस्थित‚द‚न्त‚प‚ङ्क्तिभ्यां स‚त्व‚र‚म‚भिह‚न्य‚मानाभ्यां\add{... ... ...}‚{\tiny $_{lb}$}‚ क‚ट‚क‚ट‚क‚र‚ण‚म् । \textbf{आदि}श‚ब्देन श‚रीर‚क‚म्प‚स्य ग्र‚ह‚ण‚म् । \edtext{\textsuperscript{*}}{\lemma{*}\Bfootnote{टीकायां नास्ति सुखादीति पाठः--सं०}}\textbf{सुखादि}त्यादिग्र‚हेण ह‚र्ष‚वीर‚र‚स‚योर्ग्र‚ह‚ण‚म् ।
	\pend% ending standard par
      ‚{\tiny $_{lb}$}‚‚{\tiny $_{lb}$}‚\textsuperscript{\textenglish{138/dm}}‚{\tiny $_{lb}$}‚
	  \bigskip
	  \begingroup
	

	  \pstart \leavevmode% starting standard par
	व्ध‚नं प‚श्य‚ति । शीत‚स्प‚र्श‚स्त्व‚दृश्यो रोम‚ह‚र्षादिविशेषाश्च । तेषां कार‚ण‚विरुद्धोप‚ल‚ब्ध्याऽभावं\edtext{}{\lemma{ब्ध्याऽभावं}\Bfootnote{०भावः प्र‚ति \cite{dp-edP} \cite{dp-edH} \cite{dp-edE} \cite{dp-edN} भाव प्र‚ति \cite{dp-msA}}}‚{\tiny $_{lb}$}‚ प्र‚तिप‚द्य‚त इति त‚त्रास्य प्र‚योग\edtext{}{\lemma{योग}\Bfootnote{त‚त्रास्याः प्र‚योग०--\cite{dp-msA} \cite{dp-msB} \cite{dp-msC} \cite{dp-msD} \cite{dp-edP} \cite{dp-edH} \cite{dp-edE} \cite{dp-edN}}} इति ॥
	\pend% ending standard par
       ‚{\tiny $_{lb}$}‚ 
	  \bigskip
	  \begingroup
	

	  \pstart \leavevmode% starting standard par
	कार‚ण‚विरुद्ध‚कार्योप‚ल‚ब्धिर्य‚था--न रोम‚ह‚र्षादिविशेष‚युक्त‚पुरुष‚वान‚यं‚{\tiny $_{lb}$}‚ प्र‚देशः, धूमादिति ॥ ४१ ॥
	\pend% ending standard par
      
	  \endgroup
	‚{\tiny $_{lb}$}‚ 

	  \pstart \leavevmode% starting standard par
	प्र‚तिषेध्य‚स्य य‚त् कार‚णं त‚स्य य‚द् विरुद्धं त‚स्य य‚त् कार्यं त‚स्योप‚ल‚ब्धिरुदाह‚र्त्त‚व्या—‚{\tiny $_{lb}$}‚य‚थेति अयं\edtext{}{\lemma{अयं}\Bfootnote{अयं देश इति \cite{dp-msA} \cite{dp-edP} \cite{dp-edH} \cite{dp-edE} \cite{dp-edN}}} प्र‚देश इति ध‚र्मी । योगो युक्त‚म् । रोम‚ह‚र्षादिविशेषैर्युव‚तं \edtext{}{\lemma{तं}\Bfootnote{०ह‚र्ष‚विशेष० \cite{dp-msC} रोम‚ह‚र्षादिविशेष‚युक्त‚म् नास्ति--\cite{dp-msB}}}रोम‚ह‚र्षादि‚{\tiny $_{lb}$}‚विशेष‚युक्त‚म् । त‚स्य स‚म्ब‚न्धी \edtext{}{\lemma{न्धी}\Bfootnote{पुरुषो नास्ति--\cite{dp-msC}}}पुरुषो \edtext{}{\lemma{पुरुषो}\Bfootnote{विशेष‚गुण‚युक्तः \cite{dp-msB}}}रोम‚ह‚र्षादिविशेष‚युक्त‚पुरुषः । त‚द्वान् न भ‚व‚तीति‚{\tiny $_{lb}$}‚ साध्य‚म् । धूमादिति हेतुः ।
	\pend% ending standard par
      
	  \endgroup
	‚{\tiny $_{lb}$}‚

	  \pstart \leavevmode% starting standard par
	किं \leavevmode\ledsidenote{\textenglish{54a/ms}} शीत‚निव‚र्त्त‚नाऽक्ष‚मोऽप्य‚स्ति द‚ह‚नो येन त‚तो विशिष्य‚त इत्याह--\textbf{क‚श्चिदिति ।‚{\tiny $_{lb}$}‚ स‚न्निहितो द‚ह‚न‚विशेषो} य‚स्येति विगृह्ण‚न् स‚न्निहित‚श्चासौ द‚ह‚न‚विशेष‚श्चेति य‚द‚न्येन व्याख्य तं‚{\tiny $_{lb}$}‚ त‚द‚प‚ह‚स्त‚य‚ति । त‚दा हि व्य‚धिक‚र‚णासिद्धो हेतुः स्यादिति ।
	\pend% ending standard par
      ‚{\tiny $_{lb}$}‚

	  \pstart \leavevmode% starting standard par
	क्व पुन‚र‚स्याः प्र‚योग इत्याह--य‚त्रेति ।
	\pend% ending standard par
      ‚{\tiny $_{lb}$}‚

	  \pstart \leavevmode% starting standard par
	न‚नु शीत‚स्प‚र्श‚रोम‚ह‚र्ष‚विशेषाणाम‚दृश्य‚त्वे क‚थं व‚ह्नेर्दृश्य‚त्व‚मित्याह--\textbf{रूपेति । हि}र्य‚स्मात् ।‚{\tiny $_{lb}$}‚ \textbf{इतिरे}व‚म‚न‚न्त‚रोक्तेन न्यायेन । त‚त्र विशेषेऽस्य कार‚ण‚विरुद्धोप‚ल‚म्भ‚स्य । एष तु प्र‚योगोऽभिधानीयः‚{\tiny $_{lb}$}‚य‚त्र य‚त्कार‚ण‚विरुद्ध‚म‚स्ति न त‚त्त‚त्रास्ति । य‚था श्लेष्म‚विरुद्धे पित्ते न श्लैष्मिको व्याधिः ।‚{\tiny $_{lb}$}‚ अस्ति च रोम‚ह‚र्षादिकार‚ण‚विरुद्धो व‚ह्निर‚त्रेति ।
	\pend% ending standard par
      ‚{\tiny $_{lb}$}‚

	  \pstart \leavevmode% starting standard par
	एत‚द‚प्य‚त्य‚न्ताभ्यासाज्झ‚टिति स‚न्निहित‚द‚ह‚न‚विशेष‚त्वाव‚ग‚म‚मात्रे रोम‚ह‚र्षादिविशेषाभाव‚{\tiny $_{lb}$}‚प्र‚तीत्युद‚ये स‚त्येक‚माचार्येणोक्त‚म् । \textbf{ध‚र्मोत्त‚रेणापि} त‚था व्याख्याय‚त इति द्र‚ष्ट‚व्य‚म् । अन्य‚था‚{\tiny $_{lb}$}‚ तु विरुद्धोप‚ल‚म्भ‚कार‚णानुप‚ल‚म्भ‚स‚म्भ‚वे द्वे इमे अनुमाने । त‚था हि--य‚त्र व‚ह्निर्न त‚त्र शीत‚स्प‚र्श‚{\tiny $_{lb}$}‚ इति विरुद्धोप‚ल‚म्भ‚ज‚मेक‚म‚नुमान‚म् । य‚त्र शीत‚स्प‚र्शाभावो न त‚त्र त‚त्कार्य‚रोम‚ह‚र्षादीति‚{\tiny $_{lb}$}‚ कार‚णानुप‚ल‚म्भ‚जं द्वितीय‚मिति ॥
	\pend% ending standard par
      ‚{\tiny $_{lb}$}‚

	  \pstart \leavevmode% starting standard par
	\textbf{प्र‚तिषेध्य‚स्ये}त्यादिना कार‚ण‚विरुद्ध‚कार्योप‚ल‚ब्धिं व्याच‚ष्टे । पूर्व‚व‚द् ध‚र्म्यादिप्र‚द‚र्श‚न‚म् ।
	\pend% ending standard par
      ‚{\tiny $_{lb}$}‚

	  \pstart \leavevmode% starting standard par
	रोम‚ह‚र्षादिविशेषैर्युक्त‚श्चासौ पुरुष‚श्चेति क‚र्म‚धार‚यं कृत्वा स विद्य‚ते य‚त्र स त‚द्वानिति‚{\tiny $_{lb}$}‚ \textbf{शान्त‚भ‚द्रेण} व्याख्यात‚म् । त‚च्चाव‚द्य‚म् । य‚तः क‚र्म‚धार‚य‚म‚त्त्व‚र्थीयाद् ब‚हुव्रीहिरेव लाघ‚वेन ‚{\tiny $_{lb}$}‚ इति व‚च‚नाद् रोम‚ह‚र्षादिविशेष‚युक्तः पुरुषो य‚त्रेत्येवं विशिष्टे प्र‚देशेऽव‚ग‚ते \textbf{किं} म‚त्व‚र्थीयेनेति‚{\tiny $_{lb}$}‚ म‚न्य‚मानो म‚हावैयाक‚र‚णोऽयं \textbf{ध‚र्मोत्त‚रः} प्राहः--\textbf{योगो युक्त‚मि}ति भावे निष्ठा । य‚द्येवं कृष्णः स‚र्पो‚{\tiny $_{lb}$}‚ ‚{\tiny $_{lb}$}‚ \leavevmode\ledsidenote{\textenglish{139/dm}}‚{\tiny $_{lb}$}‚ 
	  
	\edtext{\textsuperscript{*}}{\lemma{*}\Bfootnote{रोम‚ह‚र्ष‚विशेष \cite{dp-msD} \cite{dp-msB}}}रोम‚ह‚र्षादिविशेष‚स्य प्र‚त्य‚क्ष‚त्वे दृश्यानुप‚ल‚ब्धिः । कार‚ण‚स्य शीत‚स्प‚र्श‚स्य प्र‚त्य‚क्ष‚त्वे‚{\tiny $_{lb}$}‚ कार‚णानुप‚ल‚ब्धिः । व‚ह्नेस्तु\edtext{}{\lemma{ह्नेस्तु}\Bfootnote{व‚ह्नेः प्र‚त्य० \cite{dp-msC} \cite{dp-msD}}} प्र‚त्य‚क्ष‚त्वे कार‚ण‚विरुद्धोप‚ल‚ब्धिः प्र‚योक्त‚व्या । त्र‚याणाम‚प्य‚दृश्य‚त्वेऽयं‚{\tiny $_{lb}$}‚ प्र‚योगः । त‚स्माद‚भाव‚साध‚नोऽय‚म् । \edtext{\textsuperscript{*}}{\lemma{*}\Bfootnote{य‚त्र \cite{dp-msA}}}त‚त्र दूर‚स्थ‚स्य प्र‚तिप‚त्तुर्द‚ह‚न‚शीत‚स्प‚र्श‚रोम‚ह‚र्षादिविशेषा‚{\tiny $_{lb}$}‚ अप्र‚त्य‚क्षाः स‚न्तोऽपि, धूम‚स्तु प्र‚त्य‚क्षो य‚त्र, त‚त्रैत‚त् प्र‚माण‚म् । धूम‚स्तु \edtext{}{\lemma{स्तु}\Bfootnote{यादृश‚स्त‚स्मिन्देशे \cite{dp-msA} \cite{dp-msC} \cite{dp-edP} \cite{dp-edH} \cite{dp-edE} \cite{dp-edN}}}यादृश‚स्त‚द्देशे स्थितं‚{\tiny $_{lb}$}‚ शीतं निव‚र्त्त‚यितुं स‚म‚र्थ‚स्य व‚ह्नेर‚नुमाप‚कः स इह ग्राह्यः । धूम‚मात्रेण \edtext{}{\lemma{मात्रेण}\Bfootnote{तु नास्ति--\cite{dp-msA}}}तु व‚ह्निमात्रेऽनुमितेऽपि‚{\tiny $_{lb}$}‚ न शीत‚स्प‚र्श‚निवृत्तिः, नापि \edtext{}{\lemma{नापि}\Bfootnote{ह‚र्षादिनिवृत्ति \cite{dp-msD} \cite{dp-msB}}}रोम‚ह‚र्षादिविशेष‚निवृत्तिर‚व‚सातुं \edtext{}{\lemma{सातुं}\Bfootnote{श‚क्य‚त इति \cite{dp-msC} श‚क्येति धूम० \cite{dp-msA} श‚क्येते न धू० \cite{dp-edH}}}श‚क्येति न धूम‚मात्रं हेतुरिति‚{\tiny $_{lb}$}‚ द्र‚ष्ट‚व्य‚मिति ॥‚{\tiny $_{lb}$}‚ य‚स्मिन् व‚ल्मीके, लोहितः शालिर्य‚स्मिन् ग्रामे, गौरः ख‚रो य‚स्मिन्न‚र‚ण्य इति ब‚हुव्रीहिणा भ‚वित‚व्य‚म् ।‚{\tiny $_{lb}$}‚ त‚त‚श्च कृष्ण‚म‚र्प‚वान् व‚ल्मीको, लोहित‚शालिमान् ग्रामः, गौर‚ख‚र‚व‚द‚र‚ण्य‚मिति न स्यात् ।‚{\tiny $_{lb}$}‚ साध‚व‚श्चामी प्र‚योगास्त‚त्क‚थ‚म‚नेनैवं व्याख्यात‚मिति चेत् । नैष दोषः । य‚स्मात् क‚र्म‚धार‚य‚म‚त्व‚र्थीयाद्‚{\tiny $_{lb}$}‚ ब‚हुव्रीहिर्लाघ‚वेन इतीदं व‚च‚नं संज्ञाश‚ब्दं व‚र्ज‚यित्वा वेदित‚व्य‚म् । संज्ञाश‚ब्दाश्चैते कृष्ण‚स‚र्प‚{\tiny $_{lb}$}‚लोहित‚शालिगौर‚ख‚र‚श‚ब्दा इति साधूक्तं \textbf{योगो युक्त‚मि}ति । \textbf{रोम‚ह‚र्षादिविशेष‚युक्त‚मि}ति रोम‚{\tiny $_{lb}$}‚ह‚र्षादिविशेष‚योग इत्य‚र्थः । \textbf{त‚स्य} त‚द्युक्त‚स्य त‚द्योग‚स्य \textbf{स‚म्ब‚न्धी । स‚म्ब‚न्धीत्य‚ने}न स‚म्ब‚न्ध‚{\tiny $_{lb}$}‚ष‚ष्ठीयं स‚म‚स्य‚त इति द‚र्श‚य‚ति । य‚तोऽयं रोम‚ह‚र्षादिविशेष‚योगः स्वात्म‚ना पुरुषं व्य‚व‚च्छिन‚त्ति ।‚{\tiny $_{lb}$}‚ त‚तो व्य‚व‚च्छेद‚कः स‚न्पुरुषं स्व‚स‚म्ब‚न्धिन‚मुप‚पाद‚य‚ति ।
	\pend% ending standard par
      ‚{\tiny $_{lb}$}‚

	  \pstart \leavevmode% starting standard par
	क‚दाऽयं प्र‚योगो द्र‚ष्ट‚व्य इत्याह--\textbf{त्र‚याणामि}ति व‚ह्निशीत‚स्प‚र्श‚रोम‚ह‚र्षादिविशेषाणाम् ।‚{\tiny $_{lb}$}‚ \textbf{अपि}र‚व‚धार‚णे । क‚स्मादेव‚मित्याह--\textbf{रोम‚ह‚र्षादिविशेष‚स्ये}त्यादि । हिश‚ब्दार्थ‚श्चात्रार्थाद् द्र‚ष्ट‚व्यः ।‚{\tiny $_{lb}$}‚ य‚त एवं त‚स्मात् । \leavevmode\ledsidenote{\textenglish{54b/ms}} \textbf{अय‚मि}त्य‚य‚म‚पीति द्र‚ष्ट‚व्य‚म् ।
	\pend% ending standard par
      ‚{\tiny $_{lb}$}‚

	  \pstart \leavevmode% starting standard par
	क्व पुन‚स्त्र‚याणाम‚प्र‚त्य‚क्ष‚त्वं धूम‚स्य तु प्र‚त्य‚क्ष‚त्व‚मित्याह--त‚त्रेति वाक्योप‚क्षेपे । विद्य‚माना‚{\tiny $_{lb}$}‚ अप्य‚प्र‚त्य‚क्षा य‚द्य‚भ‚विष्य‚न्, निय‚त‚मुपाल‚प्स्य‚न्तेति स‚म्भाव‚नाम‚तिवृत्ताः । \textbf{दूर‚स्थ‚स्ये}ति हेतु\textbf{भावेन}‚{\tiny $_{lb}$}‚ विशेष‚ण‚म् । त‚तोऽय‚म‚र्थः--दूर‚स्थ‚त्वात् प्र‚तिप‚त्तुस्ते स‚न्तोऽप्य‚प्र‚त्य‚क्षा इति ।
	\pend% ending standard par
      ‚{\tiny $_{lb}$}‚

	  \pstart \leavevmode% starting standard par
	त‚त्र स्थाने । \textbf{एत‚त्} कार‚ण‚विरुद्ध‚कार्य‚रूपं साध‚नं \textbf{प्र‚माण‚मि}त्य‚नुमानाख्य‚प्र‚माण‚{\tiny $_{lb}$}‚ज‚न‚क‚वात् । अदूर‚स्थ‚त्वे तु प्र‚तिप‚त्तुः प्राकारादिव्य‚व‚हितोद्देशेऽयं प्र‚योगो द्र‚ष्ट‚व्यः । धूम‚श‚ब्देन‚{\tiny $_{lb}$}‚ विशिष्टो धूमो विव‚क्षित इति द‚र्श‚य‚ति--\textbf{धूम‚स्त्वि}ति । तुश‚ब्दो विशेषार्थः ।
	\pend% ending standard par
      ‚{\tiny $_{lb}$}‚

	  \pstart \leavevmode% starting standard par
	एत‚देव व्य‚तिरेक‚मुखेण द्र‚ढ‚य‚न्नाह--\textbf{धूम‚मात्रेणे}ति । \textbf{तुः} पूर्व‚स्माद् वैध‚र्म्य‚माह । \textbf{इति}‚{\tiny $_{lb}$}‚र्हेतौ । द्वितीय इतिरेव‚म‚र्थः ।
	\pend% ending standard par
      ‚{\tiny $_{lb}$}‚‚{\tiny $_{lb}$}‚\textsuperscript{\textenglish{140/dm}}‚{\tiny $_{lb}$}‚
	  \bigskip
	  \begingroup
	

	  \pstart \leavevmode% starting standard par
	य‚द्येकः प्र‚तिषेध‚हेतुरुक्तः क‚थ‚मेकाद‚शाऽभाव‚हेत‚व इत्याह--
	\pend% ending standard par
       ‚{\tiny $_{lb}$}‚ 
	  \bigskip
	  \begingroup
	

	  \pstart \leavevmode% starting standard par
	इसे स‚र्वे कार्यानुप‚ल‚ध्याद‚यो द‚शानुप‚ल‚ब्धिप्र‚योगाः स्व‚भावानुप‚ल‚ब्धौ‚{\tiny $_{lb}$}‚ स‚ङ्ग्र‚ह‚मुप‚यान्ति ॥ ४२ ॥
	\pend% ending standard par
      
	  \endgroup
	‚{\tiny $_{lb}$}‚ 

	  \pstart \leavevmode% starting standard par
	\edtext{\textsuperscript{*}}{\lemma{*}\Bfootnote{इमे स‚र्वे इत्यादि नास्ति--\cite{dp-edH} \cite{dp-edN} इम इत्यादि \cite{dp-msA} \cite{dp-msB} \cite{dp-msD} \cite{dp-edE} \cite{dp-edP}}}इमे स‚र्वे इत्यादि । इमेऽनुप‚ल‚ब्धिप्र‚योगाः । इद‚मान‚न्त‚र‚प्र‚क्रान्ता\edtext{}{\lemma{क्रान्ता}\Bfootnote{०न्त‚र‚प्र‚योगान्ता \cite{dp-edH} \cite{dp-edE} ०न्त‚र‚प्र‚योक्तानां नि० \cite{dp-msA}}} निर्दिष्टाः ।‚{\tiny $_{lb}$}‚ त‚त्र किय‚ताम‚पि ग्र‚ह‚णे प्र‚स‚क्त आह--कार्यानुप‚ल‚ब्ध्याद‚य इति । कार्यानुप‚ल‚ब्ध्यादीनाम‚पि‚{\tiny $_{lb}$}‚ त्र‚याणां च‚तुर्णां वा ग्र‚ह‚णे प्र‚स‚क्त\edtext{}{\lemma{क्त}\Bfootnote{प्र‚स‚क्ते स‚त्या \cite{dp-edP} \cite{dp-edH} \cite{dp-edE} \cite{dp-edN} प्र‚स‚क्तेत्याह \cite{dp-msA}}} आह--द‚शेति । \edtext{\textsuperscript{*}}{\lemma{*}\Bfootnote{त‚त्र नास्ति--\cite{dp-msA} \cite{dp-msB} \cite{dp-edP} \cite{dp-edH} \cite{dp-edE} \cite{dp-edN}}}त‚त्र द‚शानाम‚प्युदाहृत‚मात्राणां ग्र‚ह‚ण‚{\tiny $_{lb}$}‚प्र‚स‚ङ्गे \edtext{}{\lemma{ङ्गे}\Bfootnote{ङ्गे आह--\cite{dp-msB}}}स‚त्याह--स‚र्व इति ।
	\pend% ending standard par
       ‚{\tiny $_{lb}$}‚ 

	  \pstart \leavevmode% starting standard par
	एत‚दुक्तं भ‚व‚ति--अप्र‚युक्ता\edtext{}{\lemma{युक्ता}\Bfootnote{भ‚व‚ति । प्र‚युक्ता \cite{dp-msC}}} अपि प्र‚युक्तोदाह‚र‚ण‚स‚दृशाश्च स‚र्व एवेति । द‚श‚ग्र‚ह‚ण‚{\tiny $_{lb}$}‚म‚न्त‚रेण स‚र्व‚ग्र‚ह‚णे क्रिय‚माणे प्र‚युक्तोदाह‚र‚ण‚कार्त्स्न्यं \edtext{}{\lemma{कार्त्स्न्यं}\Bfootnote{ग‚म्य‚ते \cite{dp-msA} \cite{dp-msB} \cite{dp-msC} \cite{dp-msD} \cite{dp-edP} \cite{dp-edH} \cite{dp-edE} \cite{dp-edN}}}ग‚म्येत । द‚श‚ग्र‚ह‚णात्\edtext{}{\lemma{णात्}\Bfootnote{द‚श‚ग्र‚ह‚णोदाह‚र‚ण० \cite{dp-msB}}} \edtext{\textsuperscript{*}}{\lemma{*}\Bfootnote{०दाहृत‚का० \cite{dp-msC}}}तूदाह‚र‚ण‚{\tiny $_{lb}$}‚कार्त्स्न्येऽव‚ग‚ते स‚र्व‚ग्र‚ह‚ण‚म‚तिरिच्य‚मान‚मुदाहृत‚स‚दृश‚कार्त्स्न्याव‚ग‚त‚ये\edtext{}{\lemma{ये}\Bfootnote{०याधिग‚त‚ये \cite{dp-msC}}} \edtext{\textsuperscript{*}}{\lemma{*}\Bfootnote{जात‚म् \cite{dp-edE}}}जाय‚ते ।
	\pend% ending standard par
      
	  \endgroup
	‚{\tiny $_{lb}$}‚

	  \pstart \leavevmode% starting standard par
	प्र‚योगः पुन‚रीदृशो वाच्यः--य‚त्र य‚त्कार‚ण‚विरुद्ध‚कार्य‚म‚स्ति त‚त्र त‚न्नास्ति । य‚था‚{\tiny $_{lb}$}‚ रुदित‚विशेषे स‚ति न स्मित‚विशेषः । रोम‚ह‚र्षादिविशेष‚युक्त‚पुरुष‚व‚त्त्व‚कार‚ण‚शीत‚स्प‚र्श‚विरुद्ध‚{\tiny $_{lb}$}‚व‚ह्निकार्य‚ञ्चात्र धूम इति ।
	\pend% ending standard par
      ‚{\tiny $_{lb}$}‚

	  \pstart \leavevmode% starting standard par
	एत‚द‚प्य‚त्य‚न्ताभ्यासाज्झ‚टिति धूम‚द‚र्श‚नेन रोम‚ह‚र्षादियुक्त‚पुरुष‚व‚त्त्वाभाव‚प्र‚तीत्युद‚ये स‚ति‚{\tiny $_{lb}$}‚ कार‚ण‚विरुद्व‚कार्योप‚ल‚ब्धिज‚मेक‚म‚नुमान‚मुक्त‚माचार्येणेति द्र‚ष्ट‚व्य‚म् । अन्य‚था कार्य‚हेतुविरुद्धो‚{\tiny $_{lb}$}‚प‚ल‚म्भ‚कार‚णानुप‚ल‚म्भ‚स‚म्भ‚वानि त्रीण्य‚मून्य‚नुमानानि । त‚था हि--त‚देयं प‚रिपाटिः--य‚त्र‚{\tiny $_{lb}$}‚ धूम‚स्त‚त्राग्निरिति कार्य‚हेतुज‚मेक‚म‚नुमान‚म् । य‚त्र व‚ह्निर्न त‚त्र शीत‚स्प‚र्श इति विरुद्धोप‚ल‚म्भ‚जं‚{\tiny $_{lb}$}‚ द्वितीय‚म् । य‚त्र शीत‚स्प‚र्शाभावो न त‚त्र त‚त्कार्य‚रोम‚ह‚र्षादिविशेष‚युक्त‚पुरुष‚भाव इति कार‚णानु‚{\tiny $_{lb}$}‚प‚ल‚म्भ‚जं तृतीय‚मिति ।
	\pend% ending standard par
      ‚{\tiny $_{lb}$}‚

	  \pstart \leavevmode% starting standard par
	एते च प्र‚कारा अनुप‚ल‚ब्धेरुप‚ल‚क्ष‚णं वेदित‚व्याः । अन्यासाम‚पि विधान‚स‚म्भ‚वात् ।‚{\tiny $_{lb}$}‚ त‚थाहि व्याप‚क‚विरुद्ध‚कार्योप‚ल‚ब्धिर‚प्य‚स्ति--य‚था \unclear{ा}त्र तुषार‚स्प‚र्शो धूमादिति । कार्य‚विरुद्ध‚{\tiny $_{lb}$}‚कार्योप‚ल‚ब्धिर‚प्य‚स्ति । य‚था--नेहाप्र‚तिब‚द्ध‚साम‚र्थ्यानि शीत‚कार‚णानि स‚न्ति धूमादिति ।‚{\tiny $_{lb}$}‚ व्याप‚क‚विरुद्ध‚व्याप्तोप‚ल‚ब्धिर‚प्य‚स्ति--य‚था नायं नित्यः क‚दाचित्कार्य‚कारित्वादिति । प्र‚तिषेध्य‚स्य‚{\tiny $_{lb}$}‚ नित्य‚त्व‚स्य व्याप‚कं निर‚तिश‚य‚त्व‚म् । त‚स्य विरुद्धं सातिश‚य‚त्व‚म् । तेन व्याप्तं क‚दाचित्कार्य‚{\tiny $_{lb}$}‚कारित्व‚मिति । आसाञ्च य‚थास्वं य‚थायोगं प्र‚योगाः स्व‚य‚मूह्याः ।
	\pend% ending standard par
      ‚{\tiny $_{lb}$}‚‚{\tiny $_{lb}$}‚\textsuperscript{\textenglish{141/dm}}‚{\tiny $_{lb}$}‚
	  \bigskip
	  \begingroup
	

	  \pstart \leavevmode% starting standard par
	ते स्व‚भावानुप‚ल‚ब्धौ स‚ङ्ग्र‚हं \edtext{}{\lemma{हं}\Bfootnote{त‚त्स्वाभाव्येन--\cite{dp-msD-n}}}तादात्म्येन ग‚च्छ‚न्ति । स्व‚भावानुप‚ल‚ब्धिस्व‚भावा\edtext{}{\lemma{भावा}\Bfootnote{स्व‚भावा आत्मोत्पाद‚काः । भाव‚ध्व‚निरुत्पाद‚क‚प‚र्यायः--\cite{dp-msD-n}}}‚{\tiny $_{lb}$}‚ इत्य‚र्थः ॥
	\pend% ending standard par
       ‚{\tiny $_{lb}$}‚ 

	  \pstart \leavevmode% starting standard par
	न‚नु च स्व‚भावानुप‚ल‚ब्धिप्र‚योगाद् भिद्य‚न्ते कार्यानुप‚ल‚ब्ध्याद‚यः । त‚त् \edtext{}{\lemma{त्}\Bfootnote{०म‚न्त‚र्भाव इत्याह--\cite{dp-msC}}}क‚थ‚म‚न्त‚{\tiny $_{lb}$}‚र्भ‚व‚न्ति ? इत्याह--
	\pend% ending standard par
       ‚{\tiny $_{lb}$}‚ 
	  \bigskip
	  \begingroup
	

	  \pstart \leavevmode% starting standard par
	पार‚म्प‚र्येणार्थान्त‚र‚विधिप्र‚तिपेधाभ्यां प्र‚योग‚भेदेऽपि ॥ ४३ ॥
	\pend% ending standard par
      
	  \endgroup
	‚{\tiny $_{lb}$}‚ 

	  \pstart \leavevmode% starting standard par
	प्र‚योग‚भेदेऽपि--प्र‚योग‚स्य श‚ब्द‚व्यापार‚स्य भेदेऽपि अन्त‚र्भ‚व‚न्ति । क‚थं प्र‚योग‚भेद‚{\tiny $_{lb}$}‚ इत्याह--अर्थान्त‚र‚विधीति\edtext{}{\lemma{विधीति}\Bfootnote{विधीत्यादि \cite{dp-edP} \cite{dp-edH} \cite{dp-edE} \cite{dp-edN}}} । \edtext{\textsuperscript{*}}{\lemma{*}\Bfootnote{प्र‚तिषेध्याद‚र्थान्त‚र‚स्य \cite{dp-msB}}}प्र‚तिषेध्याद‚र्थाद‚र्थान्त‚र‚स्य विधिरुप‚ल‚ब्धिः स्व‚भाव‚विरुद्धाद्यु‚{\tiny $_{lb}$}‚प‚ल‚ब्धिप्र‚योगेषु । प्र‚तिषेधः कार्यानुप‚ल‚ब्ध्यादिषु प्र‚योगेषु । अर्थान्त‚र‚विधिना, अर्थान्त‚र-
	\pend% ending standard par
      
	  \endgroup
	‚{\tiny $_{lb}$}‚

	  \pstart \leavevmode% starting standard par
	केचित्तु नेहाप्र‚तिब‚द्ध‚साम‚र्थ्यानि व‚ह्निकार‚णानि स‚न्ति, तुषार‚स्प‚र्शादिति कार्य‚विरुद्ध‚{\tiny $_{lb}$}‚व्याप्तोप‚ल‚ब्धिमिच्छ‚न्ति । नात्र धूम‚स्तुषार‚स्प‚र्शादिति कार‚ण‚विरुद्ध‚व्याप्तोप‚ल‚ब्धिम‚पीति ॥
	\pend% ending standard par
      ‚{\tiny $_{lb}$}‚

	  \pstart \leavevmode% starting standard par
	\textbf{सा च प्र‚योग‚भेदादेकाद‚श} प्र‚कारेति य‚दुक्तं त‚द‚स‚ह‚मान‚श्चोद‚य‚ति \textbf{य‚दी}ति । \edtext{\textsuperscript{*}}{\lemma{*}\Bfootnote{न्याय‚बिन्दुः २. १७.}}अत्र द्वौ‚{\tiny $_{lb}$}‚ \textbf{व‚स्तुसाध‚नावेकः प्र‚तिषेध‚हेतु}रित्य‚नेनैकः \textbf{प्र‚तिषेध‚हेतुरुक्त} इति चोद‚यितुराश‚यः । \textbf{कार्ये}त्यादिना‚{\tiny $_{lb}$}‚ द‚श‚ग्र‚ह‚ण‚स्य तात्प‚र्यार्थं व्याच‚ष्टे । अपिश‚ब्दः श‚ङ्कायाम् । \textbf{स‚र्व}ग्र‚ह‚ण‚स्यापि तात्प‚र्यार्थ‚माह—‚{\tiny $_{lb}$}‚त‚त्रेति वाक्योप‚न्यासे । \textbf{अपिः} पूर्व‚व‚त् । \textbf{उदाहृत} एवो\textbf{दाहृत‚मात्राणि} तेषाम् । अनेन‚{\tiny $_{lb}$}‚ द्र‚व्य‚कार्त्स्न्य‚वृत्तः स‚र्व‚श‚ब्दो गृहीत इति द‚र्शित‚म् ।
	\pend% ending standard par
      ‚{\tiny $_{lb}$}‚

	  \pstart \leavevmode% starting standard par
	त‚र्हि \leavevmode\ledsidenote{\textenglish{55a/ms}} \textbf{स‚र्व}ग्र‚ह‚ण‚मेवास्तु किं द‚श‚ग्र‚ह‚णेनेत्याह--द‚शेति । \textbf{प्र‚युक्तोदाह‚र‚ण‚कार्त्स्न्यं‚{\tiny $_{lb}$}‚ ग‚म्येतेति} लोके स‚र्वे प‚दातु\edtext{}{\lemma{दातु}\Bfootnote{त}}योऽत्र योद्धारः इत्यादौ स‚र्व‚श‚ब्द‚स्योप‚द‚र्शित‚कार्त्स्न्य‚वृत्त‚स्य‚{\tiny $_{lb}$}‚ द‚र्श‚नादुक्त‚म् । य‚द्येवं द‚श‚ग्र‚ह‚णेऽप्येवं किं न स्यादित्याह--\textbf{द‚श‚ग्र‚ह‚णादि}ति । तुर्द‚श‚ग्र‚ह‚ण‚र‚हित‚{\tiny $_{lb}$}‚प‚क्षाद् वैध‚र्म्य‚माह । \textbf{अतिरिच्य‚मान‚म}धिकीभ‚व‚द् ग‚तार्थं स‚दिति याव‚त् उप‚द‚र्शित‚तुल्याव‚बोधाय‚{\tiny $_{lb}$}‚ स‚म्प‚द्य‚ते ।
	\pend% ending standard par
      ‚{\tiny $_{lb}$}‚

	  \pstart \leavevmode% starting standard par
	स्यादेत‚त्--इमे स‚र्वे द‚शानुप‚ल‚ब्धिप्र‚योगा इत्येताव‚तैव किय‚तां ग्र‚ह‚ण‚प्र‚स‚ङ्गो निराकृत‚{\tiny $_{lb}$}‚ एवेति क‚थं कार्यानुप‚ल‚ब्ध्यादिग्र‚ह‚ण\textbf{माचार्य‚स्य} नातिरिच्य‚ते, क‚थं च \textbf{ध‚र्मोत्त‚र‚स्यैषा} तात्प‚र्यार्थ‚{\tiny $_{lb}$}‚व्याख्या--\textbf{त‚त्किय‚ताम‚पि प्र‚स‚क्त आ}हेत्य‚प‚र्यालोचित‚व्याख्यानं \add{न} भ‚व‚तीति चेत् । नैष‚{\tiny $_{lb}$}‚ दोषः । त‚थाहि--प्राक्त‚नैकाद‚श‚ग्र‚ह‚ण‚स्योप‚ल‚क्ष‚णार्थ‚त्वेनान्येषाम‚पि व्याप‚क‚विरुद्ध‚व्याप्तोप‚{\tiny $_{lb}$}‚ल‚ब्ध्यादिप्र‚योगाणाम‚भिम‚त‚त्वादाद्यान् कार्याऽनुप‚ल‚ब्ध्यादिप्र‚योगान्विहाय द‚श‚संख्यापूर‚णं कृतं‚{\tiny $_{lb}$}‚ भ‚व‚त्येवेति त‚दाश‚ङ्कानिवृत्त्य‚र्थं कार्यानुप‚ल‚ब्ध्यादिग्र‚ह‚णं कृत‚मा\textbf{चार्येण । ध‚र्मोत्त‚रेणा}ऽपि त‚था‚{\tiny $_{lb}$}‚ व्याख्यात‚मिति । न त‚र्हि द‚श‚ग्र‚ह‚णं क‚र्त्त‚व्य‚मिति चेत् । न । अस्योप‚ल‚क्ष‚णार्थ‚त्वाद‚दोष एषः ।‚{\tiny $_{lb}$}‚ ‚{\tiny $_{lb}$}‚ \leavevmode\ledsidenote{\textenglish{142/dm}}‚{\tiny $_{lb}$}‚ 
	  
	प्र‚तिषेधेन च प्र‚योगा भिद्य‚न्ते । य‚दि प्र‚योगान्त‚रेष्व‚र्थान्त‚र‚विधिप्र‚तिषेधौ क‚थं \edtext{}{\lemma{थं}\Bfootnote{क‚थ‚म‚न्त‚र्भ० \cite{dp-msC}}}त‚र्हि‚{\tiny $_{lb}$}‚ अन्त‚र्भ‚व‚न्ति ? इत्याह--पार‚म्प‚र्येणेति प्र‚णालिक‚येत्य‚र्थः । ‚{\tiny $_{lb}$}‚ 
	  
	एत‚दुक्तं भ‚व‚ति--न साक्षादेते प्र‚योगा दृश्यानुप‚ल‚ब्धिम‚भिद‚ध‚ति, दृश्यानुप‚ल‚ब्ध्य‚{\tiny $_{lb}$}‚व्य‚भिचारिणं त्व‚र्थान्त‚र‚स्य विधिं निषेधं वाऽभिद‚ध‚ति । त‚तः प्र‚णालिक‚यामीषां स्व‚भावानुप‚ल‚ब्धौ‚{\tiny $_{lb}$}‚ स‚ङ्ग्र‚हो न साक्षादिति ॥ ‚{\tiny $_{lb}$}‚ 
	  
	य‚दि प्र‚योग‚भेदादेष\edtext{}{\lemma{भेदादेष}\Bfootnote{भेदेन भेदः \cite{dp-msA} \cite{dp-edP} \cite{dp-edH} \cite{dp-edE} \cite{dp-edN} प्र‚योग‚भेदादेव भेदः \cite{dp-msB} \cite{dp-msC}}} भेदः; प‚रार्थानुमाने व‚क्त‚व्य एषः । श‚ब्द‚भेदो हि प्र‚योग‚भेदः ।‚{\tiny $_{lb}$}‚ श‚ब्द‚श्च\edtext{}{\lemma{श्च}\Bfootnote{श‚ब्द‚स्तु प‚रा० \cite{dp-msB} \cite{dp-msC}}} प‚रार्थानुमान‚मित्याश‚ङ्क्याह-- ‚{\tiny $_{lb}$}‚ 
	  
	प्र‚योग‚द‚र्श‚नाभ्यासात् स्व‚य‚म‚प्येवं व्य‚व‚च्छेद‚प्र‚तीतिर्भ‚व‚तीति\edtext{}{\lemma{तीति}\Bfootnote{प्र‚तीतिरिति स्वार्थानुमानेऽप्य‚स्याः प्र‚भेद‚निर्देशः--\cite{dp-msC}}} स्वार्थेऽप्य‚नुमाने‚{\tiny $_{lb}$}‚ऽस्याः प्र‚योग‚निर्देशः ॥ ४४ ॥‚{\tiny $_{lb}$}‚ 
	  
	प्र‚योग‚द‚र्श‚नेत्यादि । प्र‚योगाणां \edtext{}{\lemma{योगाणां}\Bfootnote{शास्त्र‚घ‚टितानाम् \cite{dp-msA} \cite{dp-edP} \cite{dp-edH} \cite{dp-edE} \cite{dp-edN} शास्त्र‚प‚रिघ‚टितानाम् \cite{dp-msB} \textbf{शास्त्र‚ग‚दितानां}—‚{\tiny $_{lb}$}‚पाठान्त‚र‚म्--\cite{dp-msD-n}}}शास्त्र‚प‚रिप‚ठितानां द‚र्श‚न‚मुप‚ल‚म्भः । त‚स्याभ्यासः‚{\tiny $_{lb}$}‚ पुनः पुन‚राव‚र्त्त‚न‚म् । त‚स्मान्निमित्तात् । स्व‚य‚म‚पीति प्र‚तिप‚त्तुरात्म‚नोऽपि । एव‚म् इत्य‚{\tiny $_{lb}$}‚न‚न्त‚रोक्तेन \edtext{}{\lemma{रोक्तेन}\Bfootnote{प्र‚योग‚द‚र्श‚नाभ्यास‚क्र‚मेण--\cite{dp-msD-n}}}क्र‚मेण । व्य‚व‚च्छेद‚स्य प्र‚तिषेध‚स्य प्र‚तीतिर्भ‚व‚तीति\edtext{}{\lemma{तीति}\Bfootnote{भ‚व‚ति इतिश० \cite{dp-msC} \cite{dp-msB}}} इति\textbf{श‚ब्द‚स्त‚स्माद‚र्थे} ।‚{\tiny $_{lb}$}‚ क‚थं \textbf{स‚ङ्ग्र‚ह‚म}न्त‚र्भावं \textbf{ग‚च्छ‚न्ती}त्याह--\textbf{तादात्म्येने}ति । त‚स्याः स्व‚भावानुप‚ल‚ब्धेरात्मा‚{\tiny $_{lb}$}‚ त‚दात्मा त‚स्य भाव‚स्तेन स्व‚भावानुप‚ल‚ब्धित्वेन । \textbf{अनुप‚ल‚ब्धिस्व‚भावा} इत्य‚र्थः--इतीदं स्प‚ष्टी‚{\tiny $_{lb}$}‚क‚र‚ण‚म‚पि स्व‚भावानुप‚ल‚ब्धिस्मार‚क‚त्वेन त‚त्स्व‚भावा इति द्र‚ष्ट‚व्य‚म् ॥
	\pend% ending standard par
      ‚{\tiny $_{lb}$}‚

	  \pstart \leavevmode% starting standard par
	\edtext{\textsuperscript{*}}{\lemma{*}\Bfootnote{अस्प‚ष्ट‚म्--सं०}}...रेव किं न श‚ब्दाख्या य‚त इति चेत् न--\textbf{पार‚म्प‚र्य}ग्र‚ह‚ण‚व्याघात‚प्र‚स‚ङ्गात्, \textbf{न साक्षादेत}‚{\tiny $_{lb}$}‚ इत्यादिव‚क्ष्य‚माण‚ध‚र्मोत्त‚रीय‚व्याख्यान‚व्याघात‚प्र‚स‚ङ्गाच्चेति । \textbf{श‚ब्द}स्य \textbf{व्यापारो}ऽभिधाल‚क्ष‚णः‚{\tiny $_{lb}$}‚ त‚स्य \textbf{भेदे} भिद्य‚मान‚त्वेऽ\textbf{पि । स्व‚भाव‚विरुद्धादी}त्यादिग्र‚ह‚णेन कार‚ण‚विरुद्धादीनां ग्र‚ह‚ण‚म् ।‚{\tiny $_{lb}$}‚ \textbf{कार्यानुप‚ल‚ब्ध्यादी}त्यादिग्र‚ह‚णेन व्याप‚कानुप‚ल‚ब्ध्यादीनां ग्र‚ह‚ण‚म् । \textbf{अर्थान्त‚र‚विधिप्र‚तिषेधाभ्या}‚{\tiny $_{lb}$}‚मिति मूले क‚र‚ण‚तृतीयाद्विव‚च‚नान्त‚मेत‚दिति द‚र्श‚य‚न्नाह--\textbf{अर्थान्त‚रे}ति । \textbf{च}स्तुल्य‚ब‚ल‚त्वं‚{\tiny $_{lb}$}‚ स‚मुच्चिनोति । \textbf{भिद्य‚न्ते} नानारूपा भ‚व‚न्ति । \textbf{प्र‚योगान्त‚रेष्वि}त्य‚न्त‚र‚श‚ब्दोऽन्य‚व‚च‚नः‚{\tiny $_{lb}$}‚ स्व‚भावानुप‚ल‚ब्ध्य‚पेक्ष‚या । \textbf{प‚र‚म्प‚रा} प‚रिपाटिः । सैव पार‚म्प‚र्य‚मिति स्वार्थिकः प्र‚त्य‚यः ।‚{\tiny $_{lb}$}‚ एत‚देव स्प‚ष्ट‚य‚ति--\textbf{प्र‚णालिक‚ये}ति ।
	\pend% ending standard par
      ‚{\tiny $_{lb}$}‚

	  \pstart \leavevmode% starting standard par
	न‚नु य‚द्य‚मीषां दृश्यानुप‚ल‚ब्धाव‚न्त‚र्भाव‚स्त‚दा साक्षात्त‚द‚भिधानं त‚थात्वे च क‚थं \textbf{पार‚म्प‚र्येणे}‚{\tiny $_{lb}$}‚त्याश‚ङ्क्याह--\textbf{एत‚दुक्तं भ‚व‚ति} । य‚दि नाभिद‚ध‚ति त‚दा--\textbf{न साक्षादि}ति न क‚र्त्त‚व्य‚म् ।‚{\tiny $_{lb}$}‚ ‚{\tiny $_{lb}$}‚ \leavevmode\ledsidenote{\textenglish{143/dm}}‚{\tiny $_{lb}$}‚ 
	  
	त‚द‚य‚म‚र्थः--य‚स्मात् स्व‚य‚म‚प्येव‚म‚नेनोपायेन\edtext{}{\lemma{नेनोपायेन}\Bfootnote{म‚नेन मेयेन--पाठः--\cite{dp-msD-n}}} प्र‚तिप‚द्य‚ते प्र‚योगाभ्यासात्, त‚स्मात्‚{\tiny $_{lb}$}‚ स्व‚प्र‚तिप‚त्ताव‚प्युप‚युज्य‚मान‚स्यास्य प्र‚योग‚भेद‚स्य स्वार्थानुमाने निर्देशः । \edtext{\textsuperscript{*}}{\lemma{*}\Bfootnote{य‚त् पुन‚स्त्रिरूपं लिङ्गाख्यान‚म्--\cite{dp-msD-n}}}य‚त् पुनः प‚र‚प्र‚ति‚{\tiny $_{lb}$}‚प‚त्तावेवोप‚युज्य‚ते त‚त् प‚रार्थानुमान एव व‚क्त‚व्य‚मिति ॥‚{\tiny $_{lb}$}‚ स‚ङ्ग्र‚ह‚श्च क‚थ‚मित्याह--\textbf{दृश्ये}ति । \textbf{तु}र्विशेषार्थे य‚स्माद‚र्थे वा । \textbf{विधि}म‚ग्न्यादेः । \textbf{निषेधं} व्याप‚कादेः ।‚{\tiny $_{lb}$}‚ च‚कारो वाश‚ब्दार्थे । त‚त‚स्त‚द‚व्य‚भिचारिविधिनिषेधाभिधानात् । \textbf{न साक्षात्}‚{\tiny $_{lb}$}‚ नाव्य‚व‚धानेन । अर्थान्त‚र‚विधिप्र‚तिषेध‚योश्च दृश्यानुप‚ल‚म्भाव्य‚भिचारित्वं कार्य‚कार‚ण‚भावादि‚{\tiny $_{lb}$}‚ग्र‚ह‚ण‚काल‚प्र‚वृत्त‚दृश्यानुप‚ल‚म्भ‚स्मार‚कादि द्र‚ष्ट‚व्य‚म् ।
	\pend% ending standard par
      ‚{\tiny $_{lb}$}‚

	  \pstart \leavevmode% starting standard par
	\textbf{एष भे}\leavevmode\ledsidenote{\textenglish{55b/ms}}द इति स्व‚भावानुप‚ल‚ब्ध्यादिरूपः । क‚स्मात्त‚त्र वाच्य इत्याह \textbf{श‚ब्दे}ति ।‚{\tiny $_{lb}$}‚ \textbf{हि}र्य‚स्मात् । \textbf{श‚ब्द‚भेद}स्त्रिरूप‚लिङ्ग‚वाक्य‚नानात्व‚म् । य‚द्य‚प्येवं त‚थापि क‚थं त‚त्र व‚क्त‚व्य‚{\tiny $_{lb}$}‚ इत्याह--श‚ब्द‚श्चेति । \textbf{चो} हेतौ ।
	\pend% ending standard par
      ‚{\tiny $_{lb}$}‚

	  \pstart \leavevmode% starting standard par
	\textbf{शास्त्र‚प‚रिप‚ठितानामि}ति शास्त्र‚प‚रिप‚ठित‚द्वारेण प‚रिज्ञातानां स्व‚भावाद्य‚नुप‚ल‚ब्ध्यादि‚{\tiny $_{lb}$}‚वाच‚कानां वाक्यानामिति द्र‚ष्ट‚व्य‚म् । उप‚ल‚म्भो द्विविधो वाच्य‚रूपो वाच‚क‚रूप‚श्च । अत‚{\tiny $_{lb}$}‚ एवाव‚र्त्त‚न‚म‚पि द्वेधा श‚ब्द‚रूपाव‚र्त्त‚न‚म्, अर्थाव‚र्त्त‚नं च । त‚त्रार्थाव‚र्त्त‚नं पुनः पुन‚श्चेत‚सि निवेश‚न‚म् ।‚{\tiny $_{lb}$}‚ श‚ब्दाव‚र्त्त‚नं पुनः पुन‚रुच्चार‚ण‚म् ।
	\pend% ending standard par
      ‚{\tiny $_{lb}$}‚

	  \pstart \leavevmode% starting standard par
	मूले \textbf{स्व‚यं}श‚ब्द आत्म‚न इति ष‚ष्ठ्य‚र्थे व‚र्त्त‚मानो गृहीत इत्याश‚येनाह--\textbf{प्र‚तिप‚त्तुरात्म‚न}‚{\tiny $_{lb}$}‚ इति । स्वार्थानुमान‚प्र‚स्तावात्प्र‚तिप‚त्तृश‚ब्देन य‚स्त्रिरूपेण लिङ्गेन प‚रोक्ष‚म‚र्थं प्र‚तिप‚द्य‚ते स गृह्य‚ते ।‚{\tiny $_{lb}$}‚ \textbf{अपि}श‚ब्दात्प‚रोऽपि त‚स्मात्प्र‚तिप‚द्य‚त इति स‚म्ब‚न्ध‚नीय‚म् । मूले तु न केव‚लं प‚र‚स्येति‚{\tiny $_{lb}$}‚ योज‚नीय‚म् । \textbf{अन‚न्त‚रोक्तेन} प‚रिप‚ठित‚स्व‚भावानुप‚ल‚ब्ध्यादिसूचितेन स्व‚भावानुप‚ल‚ब्ध्यादिप्र‚योग‚{\tiny $_{lb}$}‚क्र‚मेण ।
	\pend% ending standard par
      ‚{\tiny $_{lb}$}‚

	  \pstart \leavevmode% starting standard par
	य‚त \textbf{इति}श‚ब्द‚स्त‚स्माद‚र्थे त‚त्त‚स्माद् \textbf{अयं} व‚क्ष्य‚माणोऽर्थः । \textbf{त‚स्माच्छ}ब्देन य‚स्मा‚{\tiny $_{lb}$}‚ च्छ‚ब्द‚स्यान्व‚याद् \textbf{य‚स्मादि}त्युक्त‚म् । \textbf{अनेन} स्व‚भावानुप‚ल‚ब्ध्यादिप्र‚योग‚ल‚क्ष‚णेनो\textbf{पायेन प्र‚तिप‚द्य‚त}‚{\tiny $_{lb}$}‚ इत्याश‚ङ्क्य पूर्व‚मेव स्म‚र‚य‚ति \textbf{प्र‚योगाऽभ्यासादि}ति ।
	\pend% ending standard par
      ‚{\tiny $_{lb}$}‚

	  \pstart \leavevmode% starting standard par
	स्यान्म‚त‚म्--न स्व‚य‚मुच्च‚रितः श‚ब्द‚स्त‚त्प्र‚तिप‚त्तेर्निमित्त‚म् । प्र‚तिप‚न्ने श‚ब्द‚प्र‚योगात् ॥‚{\tiny $_{lb}$}‚ अन्य‚था प्र‚तिनिय‚त‚प्र‚योगायोगात् । त‚त्क‚थं पिष्ट‚पेष‚ण‚कारी श‚ब्द उपाय‚त्वेनोच्य‚त इति‚{\tiny $_{lb}$}‚ नैत‚द‚स्ति । य‚तो लिङ्ग‚द‚र्श‚नेनान्य‚तो वा निमित्तात्प्र‚बुद्ध‚वास‚नो म‚न्द‚प्र‚चारार्थ‚स्म‚र‚णोऽत्य‚न्ताभ्य‚स्त‚{\tiny $_{lb}$}‚प्र‚योग‚स्त‚थाविध‚प्र‚योग‚मुच्चार्यैवं त‚त्त्व‚म‚व‚गाह‚मान‚स्त‚द‚र्थं प्र‚तिप‚द्य‚ते--य‚था क‚श्चिद‚भ्यासात्स‚ति‚{\tiny $_{lb}$}‚ ध‚र्मिणि ध‚र्माणां लोके चिन्ता प्र‚व‚र्त्त‚त इत्युच्चार्यैवास्यार्थं प्र‚तिप‚द्य‚ते, त‚द्व‚त् त्रिरूपाख्यानं वाक्य‚{\tiny $_{lb}$}‚मुच्चार्यैव क‚श्चिद‚भ्य‚स्त‚प्र‚योगः प‚रोक्ष‚म‚र्थं प्र‚तिप‚द्य‚ते । त‚तोऽय‚मुपायो भ‚व‚त्येव ।
	\pend% ending standard par
      ‚{\tiny $_{lb}$}‚

	  \pstart \leavevmode% starting standard par
	त‚तो प‚र‚स्यापि प्र‚तीत्युद‚यात् प‚रार्थानुमान‚म‚पि स्यादिति चेत् । भ‚व‚तु । का क्ष‚तिः ?‚{\tiny $_{lb}$}‚ स्व‚प्र‚तिप‚त्तिप्र‚योज‚नं स‚त्स्वार्थानुमानं त‚दैव च तेनान्यः प्र‚तिप‚द्य‚त इति प‚र‚प्र‚तिप‚त्तिप्र‚योज‚नं स‚त्‚{\tiny $_{lb}$}‚ प‚रार्थानुमान‚म् । अत एव‚म‚श‚ब्दोऽपि क‚श्चिन्निय‚तोऽस्तीत्य‚वाचामेति ।
	\pend% ending standard par
      ‚{\tiny $_{lb}$}‚‚{\tiny $_{lb}$}‚\textsuperscript{\textenglish{144/dm}}‚{\tiny $_{lb}$}‚
	  \bigskip
	  \begingroup
	

	  \pstart \leavevmode% starting standard par
	न‚नु च कार्यानुप‚ल‚ब्ध्यादिषु कार‚णादीनाम‚दृश्यानामेव \edtext{}{\lemma{दृश्यानामेव}\Bfootnote{०मेव प्र‚तिषेधः \cite{dp-msA} \cite{dp-edP} \cite{dp-edH} \cite{dp-edE} \cite{dp-edN}}}निषेधः, दृश्य‚निषेधे स्व‚भावा‚{\tiny $_{lb}$}‚\edtext{}{\lemma{भावा}\Bfootnote{०नुप‚ल‚म्भ‚प्र० \cite{dp-msA} \cite{dp-msB} \cite{dp-edP} \cite{dp-edH} \cite{dp-edE} \cite{dp-edN}}}नुप‚ल‚ब्धिप्र‚योग‚प्र‚स‚ङ्गात् । त‚था च स‚ति\edtext{}{\lemma{ति}\Bfootnote{अदृश्यानां निषेधे स‚ति--\cite{dp-msD-n}}} न तेषां\edtext{}{\lemma{तेषां}\Bfootnote{कार‚णादीनाम्--\cite{dp-msD-n}}} दृश्यानुप‚ल‚ब्धेर्निषेधः । त‚त् क‚थ‚{\tiny $_{lb}$}‚मेषां प्र‚योगाणां दृश्यानुप‚ल‚ब्धाव‚न्त‚र्भाव इत्याह--
	\pend% ending standard par
       ‚{\tiny $_{lb}$}‚ 
	  \bigskip
	  \begingroup
	

	  \pstart \leavevmode% starting standard par
	स‚र्व‚त्र \edtext{}{\lemma{त्र}\Bfootnote{चास्याम‚भावाभाव‚व्य‚व \cite{dp-edE}}}चास्याम‚भाव‚व्य‚व‚हार‚साध‚न्याम‚नुप‚ल‚ब्धौ येषां स्व‚भाव‚विरुद्धादीनामुप‚{\tiny $_{lb}$}‚ल‚ब्ध्या\edtext{}{\lemma{ब्ध्या}\Bfootnote{विरुद्धानामुप० \cite{dp-msC}}} कार‚णादीनाम‚नुप‚ल‚ब्ध्या च प्र‚तिषेध उक्त‚स्तेषामुप‚ल‚धिल‚क्ष‚ण‚प्राप्ताना‚{\tiny $_{lb}$}‚मेवोप‚ल‚ब्धिर‚नुप‚ल‚ब्धिश्च वेदित‚व्या ॥ ४५ ॥
	\pend% ending standard par
      
	  \endgroup
	‚{\tiny $_{lb}$}‚ 

	  \pstart \leavevmode% starting standard par
	\edtext{\textsuperscript{*}}{\lemma{*}\Bfootnote{स‚र्व‚त्र चेत्यादि नास्ति \cite{dp-edH} \cite{dp-edN}}}स‚र्व‚त्र चेत्यादि । अभाव‚श्च \edtext{}{\lemma{श्च}\Bfootnote{अभाव‚श्च त‚स्य च व्य‚व‚हारोऽभाव‚व्य० \cite{dp-msA} \cite{dp-edP} \cite{dp-edH} \cite{dp-edE}}}त‚द्व्य‚व‚हार‚श्च अभाव‚व्य‚व‚हारौ । र‚व‚भावानुप‚{\tiny $_{lb}$}‚ल‚ब्धाव‚भाव‚व्य‚व‚हारः साध्य । शिष्टेष्व‚भावः । त‚योः साध‚न्याम‚नुप‚ल‚ब्धौ । स‚र्व‚त्र चेति‚{\tiny $_{lb}$}‚ च‚श‚ब्दो हिश‚ब्द‚स्यार्थे । य‚स्मात् स‚र्व‚त्रानुप‚ल‚ब्धौ\edtext{}{\lemma{ब्धौ}\Bfootnote{ल‚ब्धौ स‚त्यां--\cite{dp-msB} \cite{dp-edH} L.}} येषां प्र‚तिषेध उक्त‚स्तेषामुप‚ल‚ब्धिल‚क्ष‚ण‚{\tiny $_{lb}$}‚प्राप्तानां \edtext{}{\lemma{प्राप्तानां}\Bfootnote{दृश्य‚मानानामेव \cite{dp-msB}}}दृश्यानामेवं \edtext{}{\lemma{दृश्यानामेवं}\Bfootnote{मेव स प्र‚ति० \cite{dp-msA} \cite{dp-edP} \cite{dp-edH} \cite{dp-edE} \cite{dp-edN}}}प्र‚तिषेध‚स्त‚स्माद् दृश्यानुप‚ल‚ब्धाव‚न्त‚र्भावः ।
	\pend% ending standard par
       ‚{\tiny $_{lb}$}‚ 

	  \pstart \leavevmode% starting standard par
	कुत एत‚द् दृश्यानामेवेत्याह--स्व‚भावेत्यादि । अत्रापि च‚कारो हेत्व‚र्थः । य‚स्मात्‚{\tiny $_{lb}$}‚ स्व‚भाव‚विरुद्ध आदिर्येषां तेषामुप‚ल‚ब्ध्या, कार‚ण‚मादिर्येषां तेषाम‚नुप‚ल‚ब्ध्या प्र‚तिषेध उक्त‚स्त‚स्माद्‚{\tiny $_{lb}$}‚ दृश्यानामेव प्र‚तिषेध इत्य‚र्थः ।
	\pend% ending standard par
       ‚{\tiny $_{lb}$}‚ 

	  \pstart \leavevmode% starting standard par
	य‚दि नाम स्व‚भाव‚विरुद्धाद्युप‚ल‚ब्ध्या कार‚णाद्य‚नुप‚ल‚ब्ध्या\edtext{}{\lemma{ब्ध्या}\Bfootnote{कार‚णानुप‚ल‚ब्ध्या--\cite{dp-msC}}} च प्र‚तिषेध उक्त‚स्त‚थापि‚{\tiny $_{lb}$}‚ क‚थं दृश्यानामेव प्र‚तिषेध इत्याह--उप‚ल‚ब्धिरित्यादि । अत्रापि च‚कारो हेत्व‚र्थः । य‚स्माद्‚{\tiny $_{lb}$}‚ ये विरोधिनः, व्याप्य‚व्याप‚क‚भूताः, कार्य‚कार‚ण‚भूताश्च ज्ञातास्तेषाम‚व‚श्य‚मेवोप‚ल‚ब्धिः, उप‚ल‚ब्धि‚{\tiny $_{lb}$}‚पूर्वा चानुप‚ल‚ब्धिर्वेदित‚व्या \edtext{}{\lemma{व्या}\Bfootnote{ज्ञात‚व्या नास्ति \cite{dp-msA} \cite{dp-msB} \cite{dp-edP} \cite{dp-edH} \cite{dp-edE} \cite{dp-edN}}}ज्ञात‚व्या । उप‚ल‚ब्ध्य‚नुप‚ल‚ब्धी च द्वे येषां स्त‚स्ते दृश्या एव ।‚{\tiny $_{lb}$}‚ त‚स्मात् स्व‚भाव‚विरुद्धाद्युप‚ल‚ब्ध्या कार‚णाद्य‚नुप‚ल‚ब्ध्या चोप‚ल‚ब्ध्य‚नुप‚ल‚ब्धिम‚तां विरुद्धादीनां‚{\tiny $_{lb}$}‚ प्र‚तिषेधः क्रिय‚माणो दृश्यानामेव कृतो द्र‚ष्ट‚व्यः ।
	\pend% ending standard par
      
	  \endgroup
	‚{\tiny $_{lb}$}‚

	  \pstart \leavevmode% starting standard par
	केचित्पुन‚रेवं व्याच‚क्ष‚ते--स्व‚य‚मित्यादिना ग्र‚न्थेन वार्त्तिक‚कृतेद‚मुक्त‚म्--स्व‚भावादीनाम‚नुप‚{\tiny $_{lb}$}‚ल‚ब्ध्या विरुद्धादीनाञ्चोप‚ल‚ब्ध्या य‚थायोग‚म‚भावं त‚द्व्य‚व‚हारं च प्र‚योग‚निर‚पेक्ष एव प्र‚तिप‚त्ता‚{\tiny $_{lb}$}‚ प्र‚त्येति । न केव‚लं प्र‚योगाभ्यासात् प्र‚तिप‚त्तिस‚म‚य एव प्र‚योग‚मुच्चार‚य‚ति । न तु त‚तो‚{\tiny $_{lb}$}‚ऽपूर्व‚म‚व‚ग‚च्छ‚तीत‚र‚था प्र‚तिनिय‚त‚श‚ब्दोच्चार‚णं न भ‚वेदिति । \textbf{अनेनोपायेने}ति \textbf{चोपाय} इहो\textbf{पाय}‚{\tiny $_{lb}$}‚स्त‚थाश‚ब्दोच्चार‚ण‚क्र‚म‚स्तेनेति व‚र्ण‚य‚न्ति । एतेन चानुप‚ल‚ब्धिप्र‚योग‚स‚म‚र्थ‚न‚न्यायेन ।
	\pend% ending standard par
      ‚{\tiny $_{lb}$}‚‚{\tiny $_{lb}$}‚\textsuperscript{\textenglish{145/dm}}‚{\tiny $_{lb}$}‚
	  \bigskip
	  \begingroup
	

	  \pstart \leavevmode% starting standard par
	ब‚हुषु चोद्येषु प्र‚क्रान्तेषु प‚रिहार‚स‚मुच्च‚यार्थ‚श्च‚कारो हेत्व‚र्थो भ‚व‚ति । य‚स्मादिदं चेदं‚{\tiny $_{lb}$}‚ च स‚माधान‚म‚स्ति त‚स्मात् त‚त्त‚च्चोद्य‚म‚युक्त‚मिति च‚कारार्थः ॥
	\pend% ending standard par
       ‚{\tiny $_{lb}$}‚ 

	  \pstart \leavevmode% starting standard par
	क‚स्मात् पुनः प्र‚तिषेध्यानां विरुद्धादीनामुप‚ल‚ब्ध्य‚नुप‚ल‚ब्धी वेदित‚व्ये इत्याह--
	\pend% ending standard par
       ‚{\tiny $_{lb}$}‚ 
	  \bigskip
	  \begingroup
	

	  \pstart \leavevmode% starting standard par
	अन्येषां विरोध‚कार्य‚कार‚ण‚भावाभावासिद्धेः\edtext{}{\lemma{भावाभावासिद्धेः}\Bfootnote{०कार‚ण‚भावासिद्धेः \cite{dp-edE} ०कार‚ण‚भावासिद्धिः \cite{dp-msB} \cite{dp-edP} \cite{dp-edH}}} ॥ ४६ ॥
	\pend% ending standard par
      
	  \endgroup
	‚{\tiny $_{lb}$}‚ 

	  \pstart \leavevmode% starting standard par
	अन्येषामिति । उप‚ल‚ब्ध्य‚नुप‚ल‚ब्धिम‚द्भ्योऽन्येऽनुप‚ल‚ब्धा एव ये तेषां विरोध‚श्च‚{\tiny $_{lb}$}‚ कार्य‚कार‚ण‚भाव‚श्च केन‚चित्स‚हाभाव‚श्च व्याप्य‚स्य\edtext{}{\lemma{स्य}\Bfootnote{व्याप्य‚स्येति व्याप्य‚रूपाणामिति व्याख्येय‚म् । ब‚हुस्थानेषु पाठोऽपि न--\cite{dp-msD-n}}} व्याप‚क‚स्याभावे\edtext{}{\lemma{स्याभावे}\Bfootnote{व्याप‚काभावे \cite{dp-msC}}} न सिध्य‚ति य‚स्मात् त‚तो‚{\tiny $_{lb}$}‚ विरोध\edtext{}{\lemma{विरोध}\Bfootnote{विरोधिकार्य० \cite{dp-msA} \cite{dp-msB} \cite{dp-edP} \cite{dp-edE} \cite{dp-edH} \cite{dp-edN}}} कार्य‚कार‚ण‚भावाभावासिद्धेः कार‚णाद् उप‚ल‚ब्ध्य‚नुप‚ल‚ब्धिम‚न्त एवं विरुद्धाद‚यो‚{\tiny $_{lb}$}‚ निषेध्याः । उभ‚य‚व‚न्त‚श्च दृश्या एव । त‚स्माद् दृश्यानामेव प्र‚तिषेधः ।
	\pend% ending standard par
       ‚{\tiny $_{lb}$}‚ 

	  \pstart \leavevmode% starting standard par
	त‚द‚य‚म‚र्थः । विरोध‚श्च\edtext{}{\lemma{श्च}\Bfootnote{विरोधः कार्य० \cite{dp-msA} \cite{dp-edP} \cite{dp-edH} \cite{dp-edE}}} कार्य‚कार‚ण‚भाव‚श्च व्याप‚काभावे व्याप्याभाव‚श्च दृश्यानु‚{\tiny $_{lb}$}‚प‚ल‚ब्धेरेवेति । \edtext{\textsuperscript{*}}{\lemma{*}\Bfootnote{०संनिधाने प‚रा० \cite{dp-msC}}}एक‚संनिधाव‚प‚राभाव‚प्र‚तीतौ ज्ञातो विरोधः । कार‚णाभिम‚ताभावे च‚{\tiny $_{lb}$}‚ कार्याभिम‚ताभाव‚प्र‚त्य‚येऽव‚सितः\edtext{}{\lemma{सितः}\Bfootnote{०व‚सित‚कार्य० \cite{dp-edP} \cite{dp-edH} \cite{dp-msA} \cite{dp-msB}}} कार्य‚कार‚ण‚भावः । व्याप‚काभिम‚ताभावे च \edtext{}{\lemma{च}\Bfootnote{व्याप्याभावे \cite{dp-msA} \cite{dp-msB} \cite{dp-msD} \cite{dp-edP} \cite{dp-edH} \cite{dp-edE} \cite{dp-edN}}}व्याप्याभि-
	\pend% ending standard par
      
	  \endgroup
	‚{\tiny $_{lb}$}‚

	  \pstart \leavevmode% starting standard par
	अन्ये पुन‚र‚न्य‚था व्य‚व‚स्थिताः--\textbf{स्व‚य‚म‚पी}त्यादिकं नाविर्भूत‚प्र‚योग‚म‚धिकृत्योक्त‚म्, किन्त्व‚न्त‚{\tiny $_{lb}$}‚र्ज‚ल्पाकार‚प्र‚वृत्तं स्व‚प्र‚तिप‚त्तिकाल‚भा\leavevmode\ledsidenote{\textenglish{56a/ms}}विन‚मिति ।
	\pend% ending standard par
      ‚{\tiny $_{lb}$}‚

	  \pstart \leavevmode% starting standard par
	अत्र च साध्व‚साधु वा व्याख्यानं साधुभिरेव ज्ञात‚व्य‚मिति ।
	\pend% ending standard par
      ‚{\tiny $_{lb}$}‚

	  \pstart \leavevmode% starting standard par
	स्यादेत‚त्--य‚था प्र‚योग‚भेदः स्वार्थानुमाने क‚थ्य‚ते त‚था च न किञ्चिद् वाच्यं प‚रार्थानु‚{\tiny $_{lb}$}‚माने स्यादित्याश‚ङ्क्याह--\textbf{य‚त्पुन‚रिति । प‚र‚प्र‚तिप‚त्तावेव,} न तु स्व‚प्र‚तिप‚त्ताव‚पीत्य‚व‚धार‚णार्थः ॥
	\pend% ending standard par
      ‚{\tiny $_{lb}$}‚

	  \pstart \leavevmode% starting standard par
	स‚म्प्र‚ति दृश्यानुप‚ल‚ब्धाव‚न्त‚र्भावं स‚र्वानुप‚ल‚ब्धीनाम‚स‚ह‚मान आह--\textbf{न‚नु चेति । त‚था‚{\tiny $_{lb}$}‚ च स‚ति} कार‚णादीनाम‚दृश्यानां निषेध‚प्र‚कारे स‚ति ।
	\pend% ending standard par
      ‚{\tiny $_{lb}$}‚

	  \pstart \leavevmode% starting standard par
	\textbf{शिष्टेषु} प‚रिशिष्टेषु \textbf{अभाव} इत्य‚भावोपीति द्र‚ष्ट‚व्य‚म् । न त्व‚भाव एव व्य‚व‚हार‚स्यापि‚{\tiny $_{lb}$}‚ साध‚नात् ।
	\pend% ending standard par
      ‚{\tiny $_{lb}$}‚

	  \pstart \leavevmode% starting standard par
	\textbf{कार‚णा\add{द्य}नुप‚ल‚ब्ध्या} च क‚र‚ण‚भूत‚या । कार्य‚कार‚ण‚भावादिग्र‚ह‚ण‚काले योप‚ल‚ब्धिर‚नु‚{\tiny $_{lb}$}‚प‚ल‚ब्धिश्च पूर्व‚मासीत् त‚द्व‚तां \textbf{प्र‚तिषेधः क्रिय‚माणो दृश्यानामेव कृतो द्र‚ष्ट‚व्यो} ज्ञात‚व्यः । य‚था‚{\tiny $_{lb}$}‚ अर्थ‚विरोधादिग्र‚ह‚ण‚कालेऽव‚श्यंभाविनी दृश्यानुप‚ल‚ब्धिस्त‚थाऽन‚न्त‚र‚मेव \textbf{ध‚र्मोत्त‚रेण} प्र‚साध‚यिष्य‚ते ।
	\pend% ending standard par
      ‚{\tiny $_{lb}$}‚

	  \pstart \leavevmode% starting standard par
	न‚नु च \textbf{कार‚णादीनां चे}त्य‚नेन प्र\edtext{}{\lemma{प्र}\Bfootnote{च}}कारेणाव‚श्यं स‚मुच्च‚यार्थेन भाव्य‚म्, त‚त्क‚थं हेत्व‚र्थे‚{\tiny $_{lb}$}‚ व्याख्याय‚त इत्याश‚ङ्क्य पूर्वं बुद्धिस्थं स्प‚ष्ट‚य‚न्नाह--\textbf{ब‚हुष्विति} ।
	\pend% ending standard par
      ‚{\tiny $_{lb}$}‚

	  \pstart \leavevmode% starting standard par
	एव‚म्म‚न्य‚ते--स‚मुच्च‚यार्थे व‚र्त्त‚मान एवायं हेत्व‚र्थे व‚र्त्त‚ते । न त्वेवं स‚मुच्च‚यार्थो‚{\tiny $_{lb}$}‚ निराक्रिय‚ते, हेतूनां प‚र‚स्प‚र‚स‚मुच्च‚य‚स्य प्र‚तीय‚मान‚त्वात् । त‚था स न हेत्व‚र्थो भ‚व‚ति । \textbf{इतिरेवं‚{\tiny $_{lb}$}‚ च‚कार}स्यार्थः प्र‚योज‚न‚म् ॥
	\pend% ending standard par
      ‚{\tiny $_{lb}$}‚‚{\tiny $_{lb}$}‚\textsuperscript{\textenglish{146/dm}}‚{\tiny $_{lb}$}‚
	  \bigskip
	  \begingroup
	

	  \pstart \leavevmode% starting standard par
	म‚ताभावे निश्चिते निश्चितो व्याप्य‚व्याप‚क‚भावः । त‚त्र\edtext{}{\lemma{त्र}\Bfootnote{त‚त्--\cite{dp-msC}}} व्याप्य‚व्याप‚क‚भाव‚प्र‚तीतेर्निमित्त‚म‚भावः‚{\tiny $_{lb}$}‚ प्र‚तिप‚त्त‚व्यः । इह गृहीते वृक्षाभावे हि शिंश‚पात्वाभाव‚प्र‚तीतौ \edtext{}{\lemma{तीतौ}\Bfootnote{०भाव‚प्र‚तीतौ व्या० \cite{dp-msB}}}प्र‚तीतो व्याप्य‚व्याप‚क‚भावः ।‚{\tiny $_{lb}$}‚ अभाव‚प्र‚तिप‚त्तिश्च स‚र्व‚त्र दृश्यानुप‚ल‚ब्धेरेव । त‚स्माद्विरोध‚म् कार्य‚कार‚ण‚भाव‚म्, व्याप्य‚व्याप‚क‚{\tiny $_{lb}$}‚भावं च स्म‚र‚ता विरोध-कार्य‚कार‚ण‚भाव-व्याप्य‚व्याप‚क‚भाविविष‚याभाव‚प्र‚तिप‚त्ति\edtext{}{\lemma{त्ति}\Bfootnote{देश‚काल‚स्व‚भाव‚विप्र‚कृष्टाः पिशाचाद‚य‚य‚स्तेषां पिशाचादीनां विरोध‚श्च केन‚चिद‚ग्निना‚{\tiny $_{lb}$}‚ स‚ह न सिध्य‚तीति स‚म्ब‚न्धः । त‚था कार्य‚कार‚ण‚भाव‚श्च पिशाचादीनां केन‚चिद्धूमेन सार्धं‚{\tiny $_{lb}$}‚ न सिध्य‚ति । \cite{dp-msD-n} ।}} निब‚न्ध‚नं‚{\tiny $_{lb}$}‚ दृश्यानुप‚ल‚ब्धिः स्म‚र्त‚व्या । दृश्यानुप‚ल‚ब्ध्य‚स्म‚र‚णे विरोधादीनाम‚स्म‚र‚ण‚म् । त‚था च स‚ति‚{\tiny $_{lb}$}‚ न विरुद्धादिविधिप्र‚तिषेधाभ्यामित‚राभाव‚प्र‚तीतिः स्यात् । विरोधादिग्र‚ह‚ण‚काल‚भाविन्यां च‚{\tiny $_{lb}$}‚ दृश्यानुप‚ल‚ब्धाव‚व‚श्य‚स्म‚र्त‚व्यायां त‚त एवाभाव‚प्र‚तीतिः ।
	\pend% ending standard par
       ‚{\tiny $_{lb}$}‚ 

	  \pstart \leavevmode% starting standard par
	त‚त्र य‚द्य‚पि संप्र‚तित‚नी \edtext{}{\lemma{नी}\Bfootnote{संप्र‚ति नास्ति--\cite{dp-msA} \cite{dp-edP} \cite{dp-edH} \cite{dp-edE} \cite{dp-edN}}}नास्ति दृश्यानुप‚ल‚ब्धिर्विरोधादिग्र‚ह‚ण‚काले त्वासीत् । या‚{\tiny $_{lb}$}‚ दृश्यानुप‚ल‚ब्धिः संप्र‚ति स्म‚र्य‚माणा सैवाभाव‚प्र‚तिप‚त्तिनिब‚न्ध‚न‚म् । त‚तः संप्र‚ति नास्ति \edtext{}{\lemma{नास्ति}\Bfootnote{दृश्योप‚ल‚ब्धि० \cite{dp-msA} \cite{dp-msB} \cite{dp-edP} \cite{dp-edH} \cite{dp-edE} \cite{dp-edN}}}दृश्यानु‚{\tiny $_{lb}$}‚प‚ल‚ब्धिरित्य‚भाव‚साध‚न‚त्वेन दृश्यानुप‚ल‚ब्धिप्र‚योगाद् भिद्य‚न्ते कार्यानुप‚ल‚ब्ध्यादिप्र‚योगाः ।
	\pend% ending standard par
      
	  \endgroup
	‚{\tiny $_{lb}$}‚

	  \pstart \leavevmode% starting standard par
	\textbf{विरुद्ध}श‚ब्देन प्र‚तिषेध्य‚स्य विरुद्धं ग्राह्य‚म् । \textbf{आदि}श‚ब्देन विरुद्ध‚कार्यादीनां ग्र‚ह‚ण‚म् ।‚{\tiny $_{lb}$}‚ येषामेक‚दोप‚ल‚ब्धिस्तेभ्योऽन्येऽ\textbf{नुप‚ल‚ब्धा एव} । क‚दाचित्क्व‚चिद‚ज्ञाता एव । \textbf{व्याप‚क‚स्याभावेऽ‚{\tiny $_{lb}$}‚भाव‚श्च व्याप्य‚स्य न सिद्ध्यिति य‚स्मात्} ।
	\pend% ending standard par
      ‚{\tiny $_{lb}$}‚

	  \pstart \leavevmode% starting standard par
	अय‚माश‚यः--य‚दि पूर्वं व्याप‚काभिम‚त‚स्याभावे व्याप्याभिम‚ताभावो निश्चितो भ‚वेत् त‚दा‚{\tiny $_{lb}$}‚ व्याप्य‚व्याप‚क‚भावः सिद्ध्येत्, त‚दा च व्याप‚कानुप‚ल‚ब्धिर्ग‚मिका स्यात्, नान्य‚देति ।
	\pend% ending standard par
      ‚{\tiny $_{lb}$}‚

	  \pstart \leavevmode% starting standard par
	अय‚म‚त्र प्र‚क‚र‚णार्थः--प्र‚ब‚न्धेन भ‚व‚तो य‚द्भावे य‚स्याभाव‚स्त‚स्य विरोध‚ग‚तिर्य‚त्स्व‚{\tiny $_{lb}$}‚भाव‚श्च येनोप‚ल‚भ्य‚ते तेन स‚ह कार्य‚कार‚ण‚भावोऽपि प‚ञ्च‚प्र‚त्य‚क्षानुप‚ल‚म्भ‚स‚म‚धिग‚म्यः, व्याप्य‚{\tiny $_{lb}$}‚व्याप‚क‚भावोऽपि प्र‚त्य‚क्षानुप‚ल‚म्भाव‚सेय इति क‚थ‚म‚दृश्य‚स्य सिद्ध्य‚न्तीति ।
	\pend% ending standard par
      ‚{\tiny $_{lb}$}‚

	  \pstart \leavevmode% starting standard par
	न‚नु भ‚व‚न्तु तेऽन्य‚त्र दृश्याः, त‚त्र ताव‚द‚दृश्या एव व‚क्त‚व्याः । दृश्य‚त्वे दृश्यानुप‚ल‚म्भ‚{\tiny $_{lb}$}‚प्र‚योगात् । त‚त्क‚थं दृश्यानुप‚ल‚ब्धावित‚रासाम‚नुप‚ल‚ब्धीनाम‚न्त‚र्भाव इत्याश‚ङ्क्य त‚था त‚त्रान्त‚र्भा‚{\tiny $_{lb}$}‚व‚स्त‚था द‚र्श‚यितुमाह--\textbf{त‚दिति} । य‚स्माद‚न्य‚त्र दृश्य‚त्वेऽपि विरुद्धादीनां दृश्यानुप‚ल‚म्भेऽन्त‚र्भावो‚{\tiny $_{lb}$}‚ न घ‚ट‚ते, स चाचार्येणोक्त‚स्त‚त्त‚स्मा\textbf{द‚य‚म‚र्थः--स‚र्व‚त्र चे}त्यादेर्वाक्य‚स्य तात्प‚र्यार्थः । क‚थं‚{\tiny $_{lb}$}‚ \textbf{दृश्यानुप‚ल‚ब्धेरि}त्य‚मुम‚र्थं ताव‚त्प्र‚साध‚य‚ति--\textbf{एके}ति ।
	\pend% ending standard par
      ‚{\tiny $_{lb}$}‚

	  \pstart \leavevmode% starting standard par
	कार्य‚कार‚ण‚भावे का वार्तेत्याह--\textbf{कार‚णे}ति ।
	\pend% ending standard par
      ‚{\tiny $_{lb}$}‚

	  \pstart \leavevmode% starting standard par
	य‚द्येवं व्याप्य‚व्याप‚क‚भाव‚स्य का ग‚तिरित्याह--\textbf{व्याप‚के}ति ।
	\pend% ending standard par
      ‚{\tiny $_{lb}$}‚

	  \pstart \leavevmode% starting standard par
	न‚नु \textbf{च} व्याप्य‚व्याप‚क‚भाव‚निश्च‚ये त‚योरेक‚त्वात् किम‚भाव‚निश्च‚येनेत्याह--त‚त्र व्याप्य‚{\tiny $_{lb}$}‚व्याप‚क‚भाव‚निश्च‚ये क‚र्त्त\leavevmode\ledsidenote{\textenglish{56b/ms}}व्ये । क‚थ‚म‚भाव‚प्र‚तीतिर्व्याप्य‚व्याप‚क‚भाव‚प्र‚तीतेर्निमित्त‚{\tiny $_{lb}$}‚मित्याह--इहेति ।
	\pend% ending standard par
      ‚{\tiny $_{lb}$}‚‚{\tiny $_{lb}$}‚\textsuperscript{\textenglish{147/dm}}‚{\tiny $_{lb}$}‚
	  \bigskip
	  \begingroup
	

	  \pstart \leavevmode% starting standard par
	विरुद्ध‚विधिना, कार‚णादिनिषेधेन च य‚तो दृश्यानुप‚ल‚ब्धिराक्षिप्ता त‚तो दृश्यानुप‚{\tiny $_{lb}$}‚ल‚ब्धेरेव\edtext{}{\lemma{ब्धेरेव}\Bfootnote{०ल‚ब्धिरेव \cite{dp-msC}}} कालान्त‚र‚वृत्तायाः स्मृतिविष‚य‚भूताया अभाव‚प्र‚तिप‚त्तिः । अमीषां च प्र‚योगाणां‚{\tiny $_{lb}$}‚ दृश्यानुप‚ल‚ब्धाव‚न्त‚र्भावः । त‚द‚नेन स‚र्वेण दृश्यानुप‚ल‚ब्धाव‚न्त‚र्भावो द‚शानाम‚नुप‚ल‚ब्धिप्र‚योगाणां‚{\tiny $_{lb}$}‚ पार‚म्प‚र्येण द‚र्शित इति वेदित‚व्य‚म् ॥
	\pend% ending standard par
      
	  \endgroup
	‚{\tiny $_{lb}$}‚

	  \pstart \leavevmode% starting standard par
	आस्तां स‚र्व‚त्र विरोधादाव‚भाव‚प्र‚तीतिः, दृश्यानुप‚ल‚ब्धिस्तु क्वोप‚युज्य‚त इत्याह—‚{\tiny $_{lb}$}‚\textbf{अभावे}ति । \textbf{चो} हेतौ ।
	\pend% ending standard par
      ‚{\tiny $_{lb}$}‚

	  \pstart \leavevmode% starting standard par
	न‚नु य‚दि नाम विरोधादिग्र‚ह‚ण‚काले दृश्यानुप‚ल‚ब्धिरासीत्, त‚थापि न सा विरुद्धोप‚{\tiny $_{lb}$}‚\textbf{ल‚ब्धि-व्याप‚कानु}प‚ल‚ब्ध्यादिप्र‚योग‚विष‚ये स‚म्प्र‚त्य‚नुव‚र्त्त‚ते । त‚त्क‚थं विरुद्धोप‚ल‚ब्ध्यादीनां त‚त्रान्त‚र्भाव‚{\tiny $_{lb}$}‚ इत्याह--\textbf{त‚स्मादि}ति । य‚तोऽभाव‚प्र‚तीतिम‚न्त‚रेण न विरोधादिसिद्धिः, अभाव‚सिद्धिश्च दृश्यानु‚{\tiny $_{lb}$}‚प‚ल‚ब्धे\textbf{स्त‚स्माद्} हेतो\textbf{र्विरोधा}दिकं \textbf{स्म‚र‚ते}ति विरुद्धोप‚ल‚म्भ‚व्याप‚कानुप‚ल‚म्भादिप्र‚योग‚काल इति‚{\tiny $_{lb}$}‚ द्र‚ष्ट‚व्य‚म् । त‚द‚स्म‚र‚णे हि य‚तो विरुद्धादिरिहास्ति, य‚तोऽयं व्याप‚कादिर्नास्ति, त‚स्मात्त‚त्त‚न्ना‚{\tiny $_{lb}$}‚स्तीत्य‚स्याः प्र‚तीतेर‚योगात्--इत्य‚पि द्र‚ष्ट‚व्य‚म् । \textbf{स्म‚र‚ते}ति च त‚दानीं विरोधादेर्ग्र‚ह‚णाद् गृहीत‚स्यैव‚{\tiny $_{lb}$}‚ विक‚ल्प‚नादुक्त‚म् । कुतः पुन‚र‚व‚श्य‚स्म‚र्त्त‚व्या सेत्याह--\textbf{दृश्ये}ति ।
	\pend% ending standard par
      ‚{\tiny $_{lb}$}‚

	  \pstart \leavevmode% starting standard par
	अथ स्यात्--प्राक्त‚नी दृश्यानुप‚ल‚ब्धिः स‚दा स्म‚र्य‚ताम्, त‚थापि क‚थ‚म‚सौ विव‚क्षिताभाव‚{\tiny $_{lb}$}‚सिद्धावुप‚योगं भ‚ज‚ते, येनात्म‚नीत‚रा अनुप‚ल‚ब्धौ\edtext{}{\lemma{ब्धौ}\Bfootnote{ब्धी}}र‚न्त‚र्भाव‚य‚तीत्याह--\textbf{विरोधे}ति । \textbf{चो}‚{\tiny $_{lb}$}‚ य‚स्मात्त‚स्यां निय‚त‚स्म‚र‚णायां स‚त्याम् ।
	\pend% ending standard par
      ‚{\tiny $_{lb}$}‚

	  \pstart \leavevmode% starting standard par
	न‚नु विरुद्धोप‚ल‚ब्धिकार‚णाद्य‚नुप‚ल‚ब्धिप्र‚योग‚विष‚ये सा नास्ति त‚त्क‚थ‚म‚विद्य‚माना सैवा‚{\tiny $_{lb}$}‚भाव‚प्र‚तीतेर्निब‚न्ध‚न‚मित्याह--त‚त्रेति वाक्योप‚न्यासे । \textbf{स‚म्प्र‚ती}दानीं \textbf{स्म‚र्य‚माणा सैवाभाव‚प्र‚ती‚{\tiny $_{lb}$}‚तेर्निब‚न्ध‚नं} विरुद्धाद्य‚भाव‚ज्ञान‚स्य क‚र‚ण‚म् ।
	\pend% ending standard par
      ‚{\tiny $_{lb}$}‚

	  \pstart \leavevmode% starting standard par
	क‚थं त‚र्हि दृश्यानुप‚ल‚ब्धेर्विरुद्धोप‚ल‚ब्ध्यादीनां भेद इत्याह--\textbf{त‚त} इति । \textbf{त‚तो} विरोधादि‚{\tiny $_{lb}$}‚ग्र‚ह‚ण‚काल‚प्र‚वृत्ताया दृश्यानुप‚ल‚ब्धेः स्म‚र्य‚माण‚त्वात्, \textbf{स‚म्प्र‚ति} सा \textbf{नास्ति । इति}स्त‚स्माद‚{\tiny $_{lb}$}‚त्राभावः साध्य‚ते तेना\textbf{भाव‚साध‚न‚त्वेन} त‚तो \textbf{भिद्य‚न्ते} विरुद्धोप‚ल‚ब्ध्यादिप्र‚योगाः ।
	\pend% ending standard par
      ‚{\tiny $_{lb}$}‚

	  \pstart \leavevmode% starting standard par
	\textbf{विरुद्ध‚विधिने}त्यादिनोक्त‚म‚र्थ‚मुप‚संह‚र‚ति । \textbf{विरुद्ध‚विधिने}त्युप‚ल‚क्ष‚ण‚म । विरुद्ध‚कार्या‚{\tiny $_{lb}$}‚दिविधिनाऽपि दृश्यानुप‚ल‚म्भाक्षेपात् । अन्य‚था तासां त‚त्रान्त‚र्भावो न स्यात् ।
	\pend% ending standard par
      ‚{\tiny $_{lb}$}‚

	  \pstart \leavevmode% starting standard par
	न‚नु स्वात‚न्त्र्येण त्व‚या त‚स्या एव प्राक्प्र‚वृत्ताया अभाव‚निश्च‚यो द‚र्शित‚स्त‚त्र चानुप‚ल‚ब्धी‚{\tiny $_{lb}$}‚नाम‚न्त‚र्भावः, न त्वाचार्य‚स्याय‚म‚भिप्रेत इत्याश‚ङ्क्याचार्य‚स्यैवाय‚म‚भिप्रेतोऽर्थ इति द‚र्श‚य‚न्नाह—‚{\tiny $_{lb}$}‚\textbf{त‚द‚नेनेति । अनेन इमे स‚र्व} \href{http://sarit.indology.info/?cref=2.42}{२. ४२} इत्यादिना\edtext{}{\lemma{इत्यादिना}\Bfootnote{अत्र मूले प्र‚दीपानुसारी पाठो नोप‚ल‚भ्य‚ते । त‚त्र तु विरोध‚{\tiny $_{lb}$}‚कार्य‚कार‚ण‚भावाभावासिद्धेः २. ४६ इति ल‚भ्य‚ते ।}} \textbf{व्याप‚क‚भावासिद्धे}रित्य‚न्तेन ।
	\pend% ending standard par
      ‚{\tiny $_{lb}$}‚

	  \pstart \leavevmode% starting standard par
	य‚दि साक्षात्त‚स्याम‚न्त‚र्भाव‚स्त‚दा द‚श्यानुप‚ल‚ब्धिः स्यात्, न बिरुद्धोप‚ल‚ब्ध्यादिभेद इत्याह—‚{\tiny $_{lb}$}‚\textbf{पार‚म्प‚र्येणे}ति । विरुद्धादिप्र‚योग‚काले न साऽस्ति केव‚लं प्राक्प्र‚वृत्ता सा स्म‚र्य‚त इति ।
	\pend% ending standard par
      ‚{\tiny $_{lb}$}‚

	  \pstart \leavevmode% starting standard par
	अय‚म‚त्र प्र‚क‚र‚णार्थः--दृश्यानुप‚ल‚म्भ‚स्याव‚क्त‚व्य‚त्वेन द‚शानाम‚प्य‚नुप‚ल‚ब्धोनां त‚त्रान्त‚र्भावः,‚{\tiny $_{lb}$}‚ विरुद्धाद्य‚भाव‚प्र‚तीताव‚नुप‚योग‚श्चेति ।
	\pend% ending standard par
      ‚{\tiny $_{lb}$}‚‚{\tiny $_{lb}$}‚\textsuperscript{\textenglish{148/dm}}‚{\tiny $_{lb}$}‚
	  \bigskip
	  \begingroup
	

	  \pstart \leavevmode% starting standard par
	उक्ता दृश्यानुप‚ल‚ब्धिर‚भावे, अभाव‚व्य‚व‚हारे च साध्ये प्र‚माण‚म् । अदृश्यानुप‚ल‚ब्धिस्तु\edtext{}{\lemma{ब्धिस्तु}\Bfootnote{ल‚ब्धिः किं \cite{dp-msA} \cite{dp-edP} \cite{dp-edH} \cite{dp-edE}}}‚{\tiny $_{lb}$}‚ किंस्व‚भावा, किंव्यापारा\edtext{}{\lemma{किंव्यापारा}\Bfootnote{पारेत्याह--\cite{dp-msB} \cite{dp-msD}}} चेत्याह--
	\pend% ending standard par
       ‚{\tiny $_{lb}$}‚ 
	  \bigskip
	  \begingroup
	

	  \pstart \leavevmode% starting standard par
	विप्र‚कृष्ट‚विष‚या\edtext{}{\lemma{या}\Bfootnote{विष‚यानुप०--\cite{dp-msB} \cite{dp-msC} \cite{dp-edH} \cite{dp-edE} \cite{dp-edN}}} पुन‚र‚नुप‚ल‚ब्धिः प्र‚त्य‚क्षानुमान‚निवृत्तिल‚क्ष‚णा संश‚य‚हेतुः ॥ ४७ ॥
	\pend% ending standard par
      
	  \endgroup
	‚{\tiny $_{lb}$}‚ 

	  \pstart \leavevmode% starting standard par
	विप्र‚कृष्टेत्यादि । विप्र‚कृष्ट‚स्त्रिभिर्देश‚काल‚स्व‚भाव‚विप्र‚क‚र्षैर्य‚स्या विष‚यः सा विप्र‚कृष्ट‚{\tiny $_{lb}$}‚विष‚येति संश‚य‚हेतुः । किंस्व‚भावा सेत्याह--प्र‚त्य‚क्षानुमान‚निवृत्तिर्ल‚क्ष‚णं स्व‚भावो य‚स्याः‚{\tiny $_{lb}$}‚ सा प्र‚त्य‚क्षानुमान‚निवृत्तिल‚क्ष‚णा । न ज्ञान‚ज्ञेय‚स्व‚भावेति याव‚त् ॥
	\pend% ending standard par
      
	  \endgroup
	‚{\tiny $_{lb}$}‚

	  \pstart \leavevmode% starting standard par
	केचित्पुन‚र‚त्रैवं ब्रुव‚ते--इहैक‚ज्ञान‚संस‚र्गिव‚स्त्व‚न्त‚रोप‚ल‚म्भोऽनुप‚ल‚म्भः । न च शीत‚स्य‚{\tiny $_{lb}$}‚ निषेधे साध्ये दूर‚त्वाद् व‚ह्नेर्भास्व‚र‚रूपोप‚ल‚ब्धिः शीत‚स्प‚र्शानुप‚ल‚ब्धिर्युज्य‚ते । येनानुप‚ल‚ब्धिः‚{\tiny $_{lb}$}‚ सिद्ध्येत्, रूप‚स्प‚र्श‚योरेक‚ज्ञान‚संस‚र्गित्वाभावात् । \leavevmode\ledsidenote{\textenglish{57a/ms}} न च विरोध‚ग्र‚ह‚ण‚काल‚प्र‚वृत्त‚{\tiny $_{lb}$}‚दृश्यानुप‚ल‚म्भ‚स्मार‚क‚त्वेन दृश्यानुप‚ल‚म्भ‚त्वं वाच्य‚म्, प्राक्प्र‚वृत्त‚प्र‚त्य‚क्ष‚स्मार‚क‚त्वेनापि प्र‚त्य‚क्ष‚त्व‚{\tiny $_{lb}$}‚प्र‚संङ्गात् । अत एव विरोधादिग्र‚ह‚ण‚काल‚प्र‚वृत्त‚दृश्यानुप‚ल‚म्भ‚स्मार‚क‚त्वेनानुप‚ल‚ब्धीनां त‚त्रान्त‚र्भावो‚{\tiny $_{lb}$}‚ न युज्य‚ते । नापि स्मृतायास्त‚स्या एवाभाव‚निश्च‚यः, व्याप्तिग्राह‚क‚प्र‚माण‚स्मार‚क‚त्वेन प‚रार्थानु‚{\tiny $_{lb}$}‚मान‚स्य त‚त्प्र‚माणान्त‚र्भाव‚प्र‚स‚ङ्गात्, त‚त एव स्म‚र्य‚माणात् प्र‚माणाद् विव‚क्षित‚प्र‚तीतिप्र‚स‚ङ्गाच्च ।‚{\tiny $_{lb}$}‚ त‚स्मात्स‚र्व‚त्रैव स‚म्प्र‚तित‚नो दृश्यानुप‚ल‚म्भो द‚र्श‚नीय‚स्त‚द्ब‚लेनैवाभाव‚निश्च‚यो वाच्यः, न तु‚{\tiny $_{lb}$}‚ प्राक्प्र‚वृत्ताद् दृश्यानुप‚ल‚म्भात् स्मृत्या विष‚यीकृताद‚भाव‚निश्च‚यः । नापि त‚त्स्मार‚क‚त्वेनानुप‚{\tiny $_{lb}$}‚ल‚ब्धीनां त‚थात्व‚मिति ।
	\pend% ending standard par
      ‚{\tiny $_{lb}$}‚

	  \pstart \leavevmode% starting standard par
	य‚द्येवं क‚थ‚ङ्कारं स प्र‚द‚र्श्य‚तामिति चेत् । उच्य‚ते । दूराद् व‚ह्ने रूप‚विशेषं दृष्ट्वा‚{\tiny $_{lb}$}‚ य‚त्रैवंविध‚रूप‚विशेष‚स्त‚त्र ताव‚द्देश‚व्याप‚क‚स्तुषार‚स्प‚र्श‚विशेषोऽस्ति । य‚था म‚हान‚सादौ त‚था‚{\tiny $_{lb}$}‚विध‚मेव‚रूप‚मित्यानुमानिकी विशिष्टोष्ण‚स्प‚र्श‚प्र‚तीतिः । सैव च शीत‚स्प‚र्शानुप‚ल‚ब्धिरुष्ण‚शीत‚{\tiny $_{lb}$}‚स्प‚र्श‚योरेक‚ज्ञान‚संस‚र्गित्वात् । विव‚क्षितोप‚ल‚म्भाद‚न्य उप‚ल‚म्भोऽनुप‚ल‚म्भः । स क्व‚चित्प्र‚त्य‚क्षात्मा‚{\tiny $_{lb}$}‚ क्व‚चिद‚नुमानात्मेति न शास्त्र‚विरोधो न युक्तिविरोधः । त‚त एव त‚द्दृश्यानुप‚ल‚म्भाच्छीत‚स्प‚{\tiny $_{lb}$}‚र्शाभाव‚प्र‚तीतिः ।
	\pend% ending standard par
      ‚{\tiny $_{lb}$}‚

	  \pstart \leavevmode% starting standard par
	आह‚त्य‚न्तं \edtext{}{\lemma{न्तं}\Bfootnote{आह‚त्य}} दृश्यानुप‚ल‚ब्धेर‚नुद‚याद् दृश्यानुप‚ल‚ब्धेर्भेदेन निर्देशः । अत एव चानु‚{\tiny $_{lb}$}‚मिताऽनुमान‚मेत‚त् । केव‚ल‚म‚त्य‚न्ताभ्यासाञ्झ‚टिति त‚थाप्र‚तीत्युद‚ये स‚त्येक‚म‚नुमान‚मुक्त‚म् ।‚{\tiny $_{lb}$}‚ व‚स्तुत‚स्त्व‚नेक‚म‚नुमान‚मेत‚त् । एवं व्याप‚क‚विरुद्ध‚कार्योप‚ल‚ब्ध्यादाव‚पि स‚र्वं द्र‚ष्ट‚व्य‚म् । त‚था‚{\tiny $_{lb}$}‚ च व्याप‚क‚विरुद्धोप‚ल‚ब्ध्यादिष्व‚प‚र‚म‚नुमान‚मेकं प्ल‚व‚मान‚म‚व‚सेय‚म् । य‚योश्च प‚र‚स्प‚र‚प‚रिहा‚{\tiny $_{lb}$}‚र‚स्थित‚ल‚क्ष‚णो विरोध‚स्त‚त्र दृश्यानुप‚ल‚ब्धिः स्फुटैव । तेन विरुद्ध‚व्याप्तोप‚ल‚ब्ध्यादिषु स‚म्प्र‚त्येव‚{\tiny $_{lb}$}‚ दृश्यानुप‚ल‚ब्धिर‚स्ति । भेद‚स्तु पार‚म्प‚र्येण त‚दुद‚यात् । एवं व्याप‚क‚कार‚णानुप‚ल‚ब्ध्यादिष्व‚{\tiny $_{lb}$}‚कार‚ण‚व्याप‚कादेर्ध‚र्म‚स्यैवानुप‚ल‚म्भाद‚भावे \edtext{}{\lemma{भावे}\Bfootnote{सिद्धे त‚त एव साम‚र्थ्यात्कार्यादेर‚भावाव‚साय इतीदा‚{\tiny $_{lb}$}‚नीन्त‚न‚स्यैव दृश्यानुप‚ल‚म्भ‚स्योप‚योगो}} न प्राक्त‚न‚स्य स्मृत्यादिविष‚यीकृत‚स्य । दृश्यानुप‚{\tiny $_{lb}$}‚ल‚म्भ‚स्य च साक्षात्कार‚णे व्यापारात् पार‚म्प‚र्येण च विरुद्धाद्य‚भावे भावे व्यापारात् दृश्यानुप‚{\tiny $_{lb}$}‚ल‚ब्धेर्भेदेन कार‚णानुप‚ल‚ब्ध्यादीनां प्र‚योग इति ॥
	\pend% ending standard par
      ‚{\tiny $_{lb}$}‚\textsuperscript{\textenglish{149/dm}}‚{\tiny $_{lb}$}‚
	  \bigskip
	  \begingroup
	

	  \pstart \leavevmode% starting standard par
	न‚नु च प्र‚माणात् प्र‚मेय‚स‚त्ताव्य‚व‚स्था । त‚तः प्र‚माणाभाबात् प्र‚मेयाभाव‚प्र‚तित्तिर्यु‚{\tiny $_{lb}$}‚क्तेत्याह--
	\pend% ending standard par
       ‚{\tiny $_{lb}$}‚ 
	  \bigskip
	  \begingroup
	

	  \pstart \leavevmode% starting standard par
	प्र‚माण‚निवृत्ताव‚प्य‚र्थाभावासिद्धेरिति ॥ ४८ ॥‚{\tiny $_{lb}$}‚ ॥ \edtext{\textsuperscript{*}}{\lemma{*}\Bfootnote{॥ द्वितीय‚प‚रिच्छेदः ॥--\cite{dp-msD} \cite{dp-msB} ॥ न्याय‚बिन्दुप्र‚क‚र‚णे द्वितीयः प‚रिच्छेदः स‚माप्तः ॥--\cite{dp-edE}}}स्वार्थानुमान‚प‚रिच्छेदो द्वितीयः स‚माप्तः ॥
	\pend% ending standard par
      
	  \endgroup
	‚{\tiny $_{lb}$}‚ 

	  \pstart \leavevmode% starting standard par
	प्र‚माण‚निवृत्ताव‚पीत्याह । कार‚णं व्याप‚कं च निव‚र्त‚मानं कार्यं व्याप्यं च निव‚र्त‚येत् ।‚{\tiny $_{lb}$}‚ न च प्र‚माणं प्र‚मेय‚स्य कार‚णं नापि व्याप‚क‚म् । अतः प्र‚माण‚योर्निवृत्ताव‚पि अर्थ‚स्य प्र‚मेय‚स्य‚{\tiny $_{lb}$}‚ निवृत्तिर्न सिध्य‚ति । त‚तोऽसिद्धेः संश‚य‚हेतुर‚दृश्यानुप‚ल‚ब्धिः, न निश्च‚य‚हेतुः । य‚त् पुनः‚{\tiny $_{lb}$}‚ प्र‚माण‚स‚त्त‚या प्र‚मेय‚स‚त्ता सिध्य‚ति त‚द् युक्त‚म् । प्र‚मेय‚कार्य हि प्र‚माण‚म् । न च कार‚ण‚{\tiny $_{lb}$}‚म‚न्त‚रेण कार्य‚म‚स्ति । न \edtext{}{\lemma{न}\Bfootnote{न च--\cite{dp-msB} \cite{dp-msD}}}तु कार‚णान्य‚व‚श्यं कार्य‚व‚न्ति भ‚व‚न्ति । त‚स्मात् प्र‚माणात्‚{\tiny $_{lb}$}‚ प्र‚मेय‚स‚त्ता व्य‚व‚स्थाप्या, न प्र‚माणामावात् प्र‚मेयाभाव‚व्य‚व‚स्थेति ॥
	\pend% ending standard par
       ‚{\tiny $_{lb}$}‚ 

	  \pstart \leavevmode% starting standard par
	॥ \edtext{\textsuperscript{*}}{\lemma{*}\Bfootnote{॥ न्याय‚बिन्दुटीकायां द्वितीयः प‚रिच्छेदः स‚माप्तः ॥ \cite{dp-msA} \cite{dp-msB} \cite{dp-edP} \cite{dp-edH} \cite{dp-edE} ॥ न्याय‚{\tiny $_{lb}$}‚बिन्दुटीकायां द्वितीयः प‚रिच्छेदः ॥ \cite{dp-msD}}}आचार्य‚ध‚र्मोत्त‚र‚कृतायां न्याय‚बिन्दुटीकायां स्वार्थानुमानो द्वितीयः प‚रिच्छेदः ।
	\pend% ending standard par
      
	  \endgroup
	‚{\tiny $_{lb}$}‚

	  \pstart \leavevmode% starting standard par
	स‚म्प्र‚त्य‚दृश्यानुप‚ल‚ब्धिम‚धिकृत्योक्तं व्याच‚क्षाण आह--\textbf{उक्तेति ।}
	\pend% ending standard par
      ‚{\tiny $_{lb}$}‚

	  \pstart \leavevmode% starting standard par
	न‚नु य‚दि प्र‚त्य‚क्षाऽनुमान‚निवृत्तिमात्र‚रूपाऽदृश्यानुप‚ल‚ब्धिर‚भावे साध्ये संश‚य‚हेतुर‚नैकान्तिकी‚{\tiny $_{lb}$}‚ त‚र्हि सा त‚त‚श्च हेत्वाभासाव‚स‚र एव व‚क्त‚व्या । त‚त् किमिहोच्च‚य‚त इति चेत् । नैत‚म\edtext{}{\lemma{म}\Bfootnote{द}}स्ति । य‚तो‚{\tiny $_{lb}$}‚ न सा व‚च‚न‚व्य‚क्त्याऽनुप‚ल‚ब्धिक्ष‚ण‚प्राप्त‚स्यानुप‚ल‚ब्धिर‚स‚द्व्य‚व‚हारे साध्ये न प्र‚माण‚म्, ऐकान्तिक‚{\tiny $_{lb}$}‚स‚द्व्य‚व‚हार‚निषेधे तु प्र‚माण‚मिति प्र‚द‚र्श‚नात् ।
	\pend% ending standard par
      ‚{\tiny $_{lb}$}‚

	  \pstart \leavevmode% starting standard par
	\textbf{न ज्ञान‚ज्ञेय‚स्व‚भावेति याव‚दि}त्यार्थं न्याय‚माश्रित्योक्तं न शाब्द‚मिति द्र‚ष्ट‚व्य‚म् ॥
	\pend% ending standard par
      ‚{\tiny $_{lb}$}‚

	  \pstart \leavevmode% starting standard par
	य‚दि प्र‚माण‚निवृत्त्या\leavevmode\ledsidenote{\textenglish{57b/ms}}प्र‚मेयानिवृत्तिस्त‚र्हि त‚त्स‚त्त‚याऽपि न प्र‚मेय‚स‚त्ता सिद्ध्ये‚{\tiny $_{lb}$}‚दित्याश‚ङ्क्य त‚त्रोप‚प‚त्तिं द‚र्श‚य‚न्नाह--\textbf{य‚त्पुन‚रि}ति । न‚नु कार‚ण‚म‚प्य‚व‚श्यं कार्य‚व‚द् भ‚व‚ति‚{\tiny $_{lb}$}‚ त‚त‚श्च स‚ति ज्ञेये ज्ञानेनाप्य‚व‚श्य‚भाव्य‚म् । त‚च्चेन्नास्ति ज्ञेय‚म‚पि नास्त्येव । त‚तः सिद्ध्य‚त्येवाऽ‚{\tiny $_{lb}$}‚दृश्य‚स्याप्य‚भाव इत्याश‚ङ्क्याह--\textbf{न त्वि}ति । \textbf{तु}ना कार्य‚ध‚र्मात् कार‚ण‚ध‚र्म‚स्य वैध‚र्म्य‚माह । एत‚च्च‚{\tiny $_{lb}$}‚ कार‚ण‚मात्राभिप्रायेणोक्त‚म्, त‚था चैत्त‚था प्रागेव निर्णीत‚म् । \textbf{त‚स्मादि}त्यादिनाऽस्यैव प्र‚कृत‚{\tiny $_{lb}$}‚स्योप‚संहार इति ॥
	\pend% ending standard par
      ‚{\tiny $_{lb}$}‚

	  \pstart \leavevmode% starting standard par
	॥ प‚ण्डित\textbf{जितारि}शिष्य‚दु\textbf{र्वेक‚मिश्र}विर‚चित\textbf{ध‚र्मोत्त‚र‚निब‚न्ध}स्य द्वितीयः प‚रिच्छेदः ॥
	\pend% ending standard par
      
	    
	    \endnumbering% ending numbering from div
	    \endgroup
	    
	  
	  
	% new div opening: depth here is 0
	
	    
	    \begingroup
	    \beginnumbering% beginning numbering from div depth=0
	    
	  
\chapter*[{तृतीयः प‚रार्थानुमान‚प‚रिच्छेदः ।}]{तृतीयः प‚रार्थानुमान‚प‚रिच्छेदः ।}\textsuperscript{\textenglish{150/dm}}‚{\tiny $_{lb}$}‚
	  \bigskip
	  \begingroup
	

	  \pstart \leavevmode% starting standard par
	स्वार्थ-प‚रार्थानुमान‚योः स्वार्थं व्याख्याय प‚रार्थ व्याख्यातुकाम आह--
	\pend% ending standard par
       ‚{\tiny $_{lb}$}‚ 
	  \bigskip
	  \begingroup
	

	  \pstart \leavevmode% starting standard par
	त्रिरूप‚लिङ्गाख्यानं \edtext{}{\lemma{लिङ्गाख्यानं}\Bfootnote{प‚रार्थानुमा० \cite{dp-msB} \cite{dp-edP} \cite{dp-edH} \cite{dp-edE} \cite{dp-edN}}}प‚रार्थ‚म‚नुमान‚म् ॥ १ ॥
	\pend% ending standard par
      
	  \endgroup
	‚{\tiny $_{lb}$}‚ 

	  \pstart \leavevmode% starting standard par
	त्रिरूप‚लिङ्गाख्यान‚मिति । त्रीणि रूपाणि--अन्व‚य-व्य‚तिरेक-प‚क्ष‚ध‚र्म‚त्व‚संज्ञ‚कानि य‚स्य‚{\tiny $_{lb}$}‚ त‚त् त्रिरूप‚म् । त्रिरूपं च \edtext{}{\lemma{च}\Bfootnote{च लिङ्ग चं \cite{dp-msA}}}त‚ल्लिङ्गं च त‚स्याख्यान‚म् । आख्याय‚ते प्र‚काश्य‚तेऽनेनेति—‚{\tiny $_{lb}$}‚त्रिरूपं \edtext{}{\lemma{त्रिरूपं}\Bfootnote{त्रिरूप‚लि० \cite{dp-edE}}}लिङ्ग‚मिति आख्यान‚म् । किं पुन‚स्त‚त् ? व‚च‚न‚म् । व‚च‚नेन हि त्रिरूपं\edtext{}{\lemma{त्रिरूपं}\Bfootnote{त्रिरूप‚लि० \cite{dp-msC}}} लिङ्ग‚मा‚{\tiny $_{lb}$}‚ख्याय‚ते । प‚र‚स्मायिदं \edtext{}{\lemma{स्मायिदं}\Bfootnote{प‚र‚स्मायिति प‚रा० \cite{dp-edE}}}प‚रार्थ‚म् ॥
	\pend% ending standard par
       ‚{\tiny $_{lb}$}‚ 

	  \pstart \leavevmode% starting standard par
	न‚नु \edtext{}{\lemma{नु}\Bfootnote{न‚नु स‚म्य० \cite{dp-msA}}}च स‚म्य‚ग्ज्ञानात्म‚क‚म‚नुमान‚मुक्त‚म् । त‚त् किम‚र्थं संप्र‚ति व‚च‚नात्म‚क‚म‚नुमान‚{\tiny $_{lb}$}‚मुच्य‚त इत्याह--
	\pend% ending standard par
       ‚{\tiny $_{lb}$}‚ 
	  \bigskip
	  \begingroup
	

	  \pstart \leavevmode% starting standard par
	कार‚णे कार्योप‚चारात् ॥ २ ॥
	\pend% ending standard par
      
	  \endgroup
	‚{\tiny $_{lb}$}‚ 

	  \pstart \leavevmode% starting standard par
	कार‚णे कार्योप‚चारादिति । \edtext{\textsuperscript{*}}{\lemma{*}\Bfootnote{त्रिरूप‚लिङ्गाल‚म्ब‚ना स्मृतिः \cite{dp-msC} \cite{dp-msD} \cite{dp-msB}}}त्रिरूप‚लिङ्गाभिधानात् त्रिरूप‚लिङ्ग‚स्मृतिरुत्प‚द्य‚ते\edtext{}{\lemma{ते}\Bfootnote{श्रोतुः--\cite{dp-msD-n}}} ।‚{\tiny $_{lb}$}‚ स्मृतेश्चानुमान‚म् । \edtext{\textsuperscript{*}}{\lemma{*}\Bfootnote{त‚स्यानुमान‚स्य--\cite{dp-msA} \cite{dp-msB} \cite{dp-edP} \cite{dp-edH} \cite{dp-edE} \cite{dp-edN}}}त‚स्माद् अनुमान‚स्य प‚र‚म्प‚र‚या त्रिरूप‚लिङ्गाभिधानं कार‚ण‚म् । त‚स्मिन्‚{\tiny $_{lb}$}‚ कार‚णे व‚च‚ने कार्य‚स्य अनुमान‚स्योप‚चारः स‚मारोपः क्रिय‚ते । त‚तः स‚मारोपात् कार‚णं व‚च‚न-
	\pend% ending standard par
      
	  \endgroup
	‚{\tiny $_{lb}$}‚

	  \pstart \leavevmode% starting standard par
	त्रिरूपं लिङ्गं ज्ञात‚म‚पि व‚क्तुम‚विदुषो बाल‚स्य व्युत्पाद‚नार्थं त्रिरूप‚लिङ्गाख्यान‚ल‚क्ष‚णं‚{\tiny $_{lb}$}‚ य‚त्प‚रार्थ‚म‚नुमान‚मुक्तं त‚द् व्याख्यातुं \textbf{स्वार्थेत्यादि}ना प्र‚स्तौति । द्व‚यो रूप‚योर‚भिधानादेक‚स्य‚{\tiny $_{lb}$}‚ ग‚म्य‚मान‚त्वा\textbf{दाख्याय‚ते प्र‚काश्य‚तेऽनेनेति त्रिरूपं लिङ्ग‚मि}ति विवृतं न त्व‚भिधीय‚तेऽनेनेति ।‚{\tiny $_{lb}$}‚ अभिधेय‚स्य ग‚म्य‚मान‚स्य च प्र‚काश्य‚त्वं तुल्य‚मिति प्र‚काश्य‚ते इत्य‚नेन द्व‚योः स‚ङ्ग्र‚हः ।‚{\tiny $_{lb}$}‚ येनार्थ‚क्र‚मेणात्म‚नः प‚रोक्षार्थ‚ज्ञान‚मुत्प‚न्नं तेनैव क्र‚मेण प‚र‚स‚न्ताने लिङ्गिज्ञानोत्पिपाद‚यिष‚या‚{\tiny $_{lb}$}‚ त्रिरूप‚स्य लिङ्ग‚स्य ख्याप‚कं य‚द् व‚च‚नं त‚त्प‚रार्थ‚म‚नुमान‚मिति द्र‚ष्ट‚व्य‚म् ॥
	\pend% ending standard par
      ‚{\tiny $_{lb}$}‚

	  \pstart \leavevmode% starting standard par
	कार‚णे व‚च‚ने कार्य‚स्य ज्ञान‚स्योप‚चारात् स‚मारोपात् ।
	\pend% ending standard par
      ‚{\tiny $_{lb}$}‚

	  \pstart \leavevmode% starting standard par
	क‚थं पुन‚र्व‚च‚न‚स्यानुमान‚हेतुत्व‚मित्याह--\textbf{त्रिरूपेति} ।
	\pend% ending standard par
      ‚{\tiny $_{lb}$}‚‚{\tiny $_{lb}$}‚\textsuperscript{\textenglish{151/dm}}‚{\tiny $_{lb}$}‚
	  \bigskip
	  \begingroup
	

	  \pstart \leavevmode% starting standard par
	म‚नुमान‚श‚ब्देनोच्य‚ते । औप‚चारिकं\edtext{}{\lemma{चारिकं}\Bfootnote{औप‚चार‚क‚म्--\cite{dp-msA}}} व‚च‚न‚म‚नुमान‚म्, न मुख्य‚मित्य‚र्थः । न याव‚त्\edtext{}{\lemma{त्}\Bfootnote{न च याव‚त् \cite{dp-msA} \cite{dp-msB} \cite{dp-msD} \cite{dp-edP} \cite{dp-edH} \cite{dp-edE} \cite{dp-edN}}} किंचिदु‚{\tiny $_{lb}$}‚प‚चाराद‚नुमान‚श‚ब्देन व‚क्तुं श‚क्यं ताव‚त् स‚र्वं व्याख्येय‚म् । किन्त्व‚नुमानं व्याख्यातुकामेनानु‚{\tiny $_{lb}$}‚मान‚स्व‚रूप‚स्य\edtext{}{\lemma{स्य}\Bfootnote{स्व‚रूप‚स्यैव व्या० \cite{dp-msC}}} व्याख्येय‚त्वान्निमित्तं व्याख्येय‚म् । निमित्तं च त्रिरूपं लिङ्ग‚म् । त‚च्च स्व‚यं‚{\tiny $_{lb}$}‚ वा प्र‚तीत‚म‚नुमान‚स्य निमित्तं भ‚व‚ति, प‚रेण वा प्र‚तिपादितं भ‚व‚ति\edtext{}{\lemma{ति}\Bfootnote{भ‚व‚ति नास्ति \cite{dp-msA} \cite{dp-msC} \cite{dp-edP} \cite{dp-edE} \cite{dp-edN}}} । त‚स्माल्लिङ्ग‚स्य स्व‚रूपं‚{\tiny $_{lb}$}‚ \edtext{\textsuperscript{*}}{\lemma{*}\Bfootnote{च नास्ति \cite{dp-msA} \cite{dp-edP} \cite{dp-edH} \cite{dp-edE} \cite{dp-edN}}}च व्याख्येय‚म्, त‚त्प्र‚तिपाद‚क‚श्च श‚ब्दः । त‚त्र स्व‚रूपं स्वार्थानुमाने व्याख्यात‚म् । प्र‚तिपाद‚{\tiny $_{lb}$}‚क‚श्च\edtext{}{\lemma{श्च}\Bfootnote{०द‚कः श‚ब्द \cite{dp-msA} \cite{dp-msB} \cite{dp-edP} \cite{dp-edH} \cite{dp-edE} \cite{dp-edN}}} श‚ब्द इह व्याख्येयः । त‚तः प्र‚तिपाद‚कं श‚ब्द‚म‚व‚श्यं व‚क्त‚व्यं द‚र्श‚य‚न् अनुमान‚श‚ब्दे‚{\tiny $_{lb}$}‚नोक्त‚वानाचार्य इति प‚र‚मार्थः ॥
	\pend% ending standard par
       ‚{\tiny $_{lb}$}‚ 

	  \pstart \leavevmode% starting standard par
	प‚रार्थानुमान‚स्य प्र‚कार‚भेदं द‚र्श‚यितुमाह--
	\pend% ending standard par
      
	  \endgroup
	‚{\tiny $_{lb}$}‚
	  \bigskip
	  \begingroup
	

	  \pstart \leavevmode% starting standard par
	त‚द् द्विविध‚म् ॥ ३ ॥
	\pend% ending standard par
      
	  \endgroup
	‚{\tiny $_{lb}$}‚

	  \pstart \leavevmode% starting standard par
	न‚नु च त्रिरूप‚लिङ्गाभिधानाद‚व‚ग‚ते स‚ति ध‚र्मिणि लिङ्गं ज्ञाय‚ते । त‚स्य तु व्याप्तिः‚{\tiny $_{lb}$}‚ स्म‚र्य‚ते । त‚त्क‚थं त्रिरूप‚लिङ्ग‚व‚च‚नात् त‚त्स्मृतिरुत्प‚द्य‚ते इत्युच्य‚त इति चेत् । उच्य‚ते ।‚{\tiny $_{lb}$}‚ गृह्य‚माण‚म‚पि धूमादिव‚स्तु न ताव‚ल्लिङ्गं याव‚द् व‚ह्न्यादिसाध्याविनाभूत‚त‚या न ज्ञाय‚ते ।‚{\tiny $_{lb}$}‚ त‚थात्वं च त‚स्य न त‚दा ग्राह्य‚म‚पि तु पूर्व‚गृहीत‚मेव स्म‚र्त्त‚व्य‚मिति सूक्तं \textbf{त्रिरूप‚लिङ्ग‚स्मृति‚{\tiny $_{lb}$}‚रुत्प‚द्य‚त} इति । \textbf{स्मृते}रिति प‚क्ष‚ध‚र्म‚ग्र‚ह‚ण‚स‚हिताया इति द्र‚ष्ट‚व्य‚म् ।
	\pend% ending standard par
      ‚{\tiny $_{lb}$}‚

	  \pstart \leavevmode% starting standard par
	अय‚म‚र्थः--व‚च‚न‚म‚पि त्रिरूपं लिङ्गं स्म‚र‚य‚त् प‚रोक्षार्थ‚ज्ञान‚स्य प‚र‚म्प‚र‚या कार‚णं भ‚व‚दु‚{\tiny $_{lb}$}‚प‚चाराद‚नुमान‚मुच्य‚त इति ।
	\pend% ending standard par
      ‚{\tiny $_{lb}$}‚

	  \pstart \leavevmode% starting standard par
	अथाबाधित‚त्वाद्य‚पि लिङ्ग‚स्य ल‚क्ष‚ण‚मित्याच‚क्ष‚ते केचिदिति विप्र‚तिप‚त्तिद‚र्श‚नात् त‚द्व्यु‚{\tiny $_{lb}$}‚त्पाद‚नं युक्त‚म्, न तु त‚द्व‚च‚न‚म्, त‚स्य विप्र‚तिप‚त्त्य‚भावादिति चेत् । न अत्राप्य‚व्याप्तिव्य‚ति‚{\tiny $_{lb}$}‚रेकाभ्यां निग‚द‚न्तो विप्र‚तिप‚न्ना इत्य‚स्यापि व्युत्पाद‚नं न्याय्य‚म् ।
	\pend% ending standard par
      ‚{\tiny $_{lb}$}‚

	  \pstart \leavevmode% starting standard par
	अथाऽपि स्यात्, य‚दि प‚र‚म्प‚र‚याऽनुमान‚हेतुत्वेन व‚च‚न‚मुप‚चाराद‚नुमान‚मिति व्युत्पाद्य‚ते‚{\tiny $_{lb}$}‚ त‚र्हि जिज्ञासास्वास्थ्यादिक‚म‚पि प‚र‚म्प‚र‚याऽनुमान‚हेतुत्वाद‚नुमान‚श‚ब्देन व‚क्तुं श‚क्य‚मिति त‚द‚पि‚{\tiny $_{lb}$}‚ किं नोच्य‚त इत्याह--\textbf{न याव‚दि}ति ।
	\pend% ending standard par
      ‚{\tiny $_{lb}$}‚

	  \pstart \leavevmode% starting standard par
	न‚नु स्वास्थ्यादिक‚म‚पि निमित्त‚मिति त‚द‚व‚स्थो दोषः । न । \textbf{निमित्तं व्याख्पेय}मित्य‚{\tiny $_{lb}$}‚व्य‚हित‚म‚साधार‚णं निमित्त‚माख्येय‚मित्य‚र्थः ।
	\pend% ending standard par
      ‚{\tiny $_{lb}$}‚

	  \pstart \leavevmode% starting standard par
	न‚नु स्व‚यं प्र‚तीतं लिङ्ग‚म‚नुमान‚स्य निमित्त‚म् । त‚त्किं त‚द्व‚च‚नेन व्याख्यातेनेत्याह—‚{\tiny $_{lb}$}‚\textbf{त‚च्चे}ति । \textbf{चो} य‚स्माद‚र्थे । वाश‚ब्दो विक‚ल्पार्थः । य‚तः प‚रेण प्र‚तिपादित‚म‚पि त‚ल्लिङ्ग‚{\tiny $_{lb}$}‚म‚नुमान‚स्य निमित्तं \textbf{त‚त}स्त‚स्माद\textbf{व‚श्यं व‚क्त‚व्यं प्र}\leavevmode\ledsidenote{\textenglish{58a/ms}}\textbf{तिपाद‚कं} लिङ्ग‚प्र‚तिपाद‚कं व‚च‚नं \textbf{द‚र्श‚य‚न्न‚{\tiny $_{lb}$}‚नुमान‚श‚ब्देनोक्त‚वानाचार्यः} ।
	\pend% ending standard par
      ‚{\tiny $_{lb}$}‚‚{\tiny $_{lb}$}‚‚{\tiny $_{lb}$}‚\textsuperscript{\textenglish{152/dm}}‚{\tiny $_{lb}$}‚
	  \bigskip
	  \begingroup
	

	  \pstart \leavevmode% starting standard par
	त‚द् द्विविध‚मिति । त‚दिति प‚रार्थानुमान‚म् । द्वौ विधौ प्र‚कारौ य‚स्य त‚द् द्विविध‚म् ॥‚{\tiny $_{lb}$}‚ कुतो द्विविध‚मित्याह--
	\pend% ending standard par
       ‚{\tiny $_{lb}$}‚ 
	  \bigskip
	  \begingroup
	

	  \pstart \leavevmode% starting standard par
	प्र‚योग‚भेदात् ॥ ४ ॥
	\pend% ending standard par
      
	  \endgroup
	‚{\tiny $_{lb}$}‚ 

	  \pstart \leavevmode% starting standard par
	प्र‚योग‚स्य श‚ब्द‚व्यापार‚स्य भेदात् । प्र‚युक्तिः प्र‚योगोर्थाभिधान‚म् । श‚ब्द‚स्यार्थाभि‚{\tiny $_{lb}$}‚धान‚व्यापार‚भेदाद् द्विविध‚म‚नुमान‚म् ॥
	\pend% ending standard par
       ‚{\tiny $_{lb}$}‚ 

	  \pstart \leavevmode% starting standard par
	त‚देवाभिधान‚व्यापार‚निब‚न्ध‚नं\edtext{}{\lemma{नं}\Bfootnote{अभिधान‚स्य व्यापारो निब‚न्ध‚नं य‚स्य--\cite{dp-msD-n}}} द्वैविध्यं द‚र्श‚यितुमाह--
	\pend% ending standard par
       ‚{\tiny $_{lb}$}‚ 
	  \bigskip
	  \begingroup
	

	  \pstart \leavevmode% starting standard par
	माध‚र्म्य‚व‚द्वैध‚र्म्य‚व‚च्चेति\edtext{}{\lemma{च्चेति}\Bfootnote{व‚च्च ॥ \cite{dp-msC}}} ॥ ५ ॥
	\pend% ending standard par
      
	  \endgroup
	‚{\tiny $_{lb}$}‚ 

	  \pstart \leavevmode% starting standard par
	साध‚र्म्य‚व‚द्वैध‚र्म्म‚व‚च्चेति । स‚मानो ध‚र्मोऽस्य\edtext{}{\lemma{र्मोऽस्य}\Bfootnote{ध‚र्मो य‚स्य \cite{dp-msC} \cite{dp-msA} \cite{dp-edP} \cite{dp-edH} \cite{dp-edE} \cite{dp-edN}}} सोऽयं स‚ध‚र्मा । त‚स्य भावः साध‚र्म्य‚म् ।‚{\tiny $_{lb}$}‚ विस‚दृशो ध‚र्मोऽस्य विध‚र्मा । विध‚र्म‚णो भावो वैध‚र्म्य‚म् । दृष्टान्त‚ध‚र्मिणा स‚ह साध्य‚ध‚र्मिणः‚{\tiny $_{lb}$}‚ सादृश्यं हेतुकृतं साध‚र्म्य‚मुच्य‚ते । असादृश्यं च हेतुकृतं वैध‚र्म्यंमुच्य‚ते । \edtext{\textsuperscript{*}}{\lemma{*}\Bfootnote{त‚योः साध‚र्म्य‚वैध‚र्म्य‚योः--\cite{dp-msD-n}}}त‚त्र य‚स्य साध‚न‚{\tiny $_{lb}$}‚वाक्य‚स्य साध‚र्म्य‚म‚भिधेयं त‚त् साध‚र्म्य‚व‚त् । य‚था--य‚त कृत‚कं त‚द‚नित्यं य‚था घ‚टः, त‚था च‚{\tiny $_{lb}$}‚ कृत‚कः श‚ब्द इत्य‚त्र कृत‚क‚त्व‚कृतं दृष्टान्त‚साध्य‚ध‚र्मिणोः सादृश्य‚म‚भिधेय‚म् । य‚स्य तु वैध‚र्म्य‚म‚{\tiny $_{lb}$}‚भिघेयं त‚द् वैध‚र्म्य‚व‚त् । य‚था--य‚न्नित्यं त‚द‚कृत‚कं दृष्टं य‚थाकाश‚म् । श‚ब्द‚स्तु कृत‚क इति‚{\tiny $_{lb}$}‚ \edtext{\textsuperscript{*}}{\lemma{*}\Bfootnote{इति अकृत‚क‚त्व‚कृत‚म्--\cite{dp-msC}}}कृत‚क‚त्वाऽकृत‚क‚त्व‚कृतं श‚ब्दाकाश‚योः साध्य‚दृष्टान्त‚ध‚र्मिणोर‚सादृश्य‚मिहाभिधेय‚म् ॥
	\pend% ending standard par
       ‚{\tiny $_{lb}$}‚ 

	  \pstart \leavevmode% starting standard par
	य‚द्य‚न‚योः प्र‚योग‚योर‚भिधेयं भिन्नं क‚थं त‚र्हि त्रिरूपं लिङ्ग‚म‚भिन्नं प्र‚काश्य‚मित्याह--
	\pend% ending standard par
       ‚{\tiny $_{lb}$}‚ 
	  \bigskip
	  \begingroup
	

	  \pstart \leavevmode% starting standard par
	नान‚योर‚र्थ‚तः क‚श्चिद्भेदः ॥ ६ ॥
	\pend% ending standard par
      
	  \endgroup
	‚{\tiny $_{lb}$}‚ 

	  \pstart \leavevmode% starting standard par
	नान‚योर‚र्थ‚त इति अर्थः प्र‚योज‚न‚म्\edtext{}{\lemma{म्}\Bfootnote{०ज‚न‚म् प्र‚काश‚यित‚व्यं व‚स्तु य‚दुद्दिश्यानु० \cite{dp-msA} \cite{dp-edP} \cite{dp-edH} \cite{dp-edE} \cite{dp-edN}}} । य‚त् प्र‚योज‚नं प्र‚काश‚यित‚व्यं व‚स्तु उद्दिश्या‚{\tiny $_{lb}$}‚नुमाने प्र‚युज्येते, त‚तः \edtext{}{\lemma{तः}\Bfootnote{प्र‚योज‚नान्नान‚योर्भेदः--\cite{dp-msB} \cite{dp-msD}}}प्र‚योज‚नाद‚न‚योर्न भेदः क‚श्चित् । त्रिरूपं हि लिङ्गं प्र‚काश‚यित‚व्य‚म् ।‚{\tiny $_{lb}$}‚ त‚दुद्दिश्य द्वे अप्येते प्र‚युज्येते । द्वाभ्याम‚पि त्रिरूपं लिङ्गं प्र‚काश्य‚त एव । त‚तः प्र‚काश‚{\tiny $_{lb}$}‚यित‚व्यं प्र‚योज‚न‚म‚न‚योर‚भिन्न‚म् । त‚था च न त‚तो भेदः क‚श्चित् ॥
	\pend% ending standard par
      
	  \endgroup
	‚{\tiny $_{lb}$}‚

	  \pstart \leavevmode% starting standard par
	अनेनैत‚दाह-न स्वास्थ्यादि प्र‚तिप‚न्न‚म‚पि प‚रोक्षार्थ‚प्र‚तिपाद‚कं येन त‚दुच्येत । त‚द्व‚च‚न‚{\tiny $_{lb}$}‚म‚व‚श्यं द‚र्श‚यित‚व्य\textbf{म‚नुमान‚श‚ब्देना}भिल‚प्येत । स्व‚य‚म‚श‚क्त‚म‚पि तु हेतुव‚च‚नं प‚रोक्ष‚प्र‚काश‚न‚व‚स्तुसू‚{\tiny $_{lb}$}‚च‚क‚त्वाद‚नुमान‚श‚ब्देनोक्त‚मिति । \textbf{इति प‚र‚मार्थ} एव‚म‚स्योप‚चार‚स्य प‚र‚मः प्र‚कृष्टोऽर्थः प्र‚योज‚न‚म् ॥
	\pend% ending standard par
      ‚{\tiny $_{lb}$}‚

	  \pstart \leavevmode% starting standard par
	विध‚श‚ब्देन च विगृह्ण‚तोऽभिप्रायः प्रागेव प्र‚द‚र्शितः ॥
	\pend% ending standard par
      ‚{\tiny $_{lb}$}‚

	  \pstart \leavevmode% starting standard par
	अभिधान‚म‚र्थ‚प्र‚काश‚न‚म् ॥
	\pend% ending standard par
      ‚{\tiny $_{lb}$}‚‚{\tiny $_{lb}$}‚\textsuperscript{\textenglish{153/dm}}‚{\tiny $_{lb}$}‚
	  \bigskip
	  \begingroup
	

	  \pstart \leavevmode% starting standard par
	अभिधेय‚भेदोऽपि त‚र्हि न स्यादित्याह--
	\pend% ending standard par
       ‚{\tiny $_{lb}$}‚ 
	  \bigskip
	  \begingroup
	

	  \pstart \leavevmode% starting standard par
	अन्य‚त्र प्र‚योग‚भेदात् ॥ ७ ॥
	\pend% ending standard par
      
	  \endgroup
	‚{\tiny $_{lb}$}‚ 

	  \pstart \leavevmode% starting standard par
	अन्य‚त्र प्र‚योग‚भेदादिति । प्र‚योगोऽभिधानं वाच‚क‚त्व‚म् । वाच‚क‚त्व‚भेदाद‚न्यो भेदः‚{\tiny $_{lb}$}‚ प्र‚योज‚न‚कृतो नास्तीत्य‚र्थः ।
	\pend% ending standard par
       ‚{\tiny $_{lb}$}‚ 

	  \pstart \leavevmode% starting standard par
	एत‚दुक्तं भ‚व‚ति । अन्य‚द‚भिधेय‚म‚न्य‚त् प्र‚काश्यं प्र‚योज‚न‚म् । त‚त्राभिघेयापेक्ष‚या‚{\tiny $_{lb}$}‚ वाच‚क‚त्वं भिद्य‚ते । प्र‚काश्यं त्व‚भिन्न‚म् । अन्व‚ये हि क‚थिते व‚क्ष्य‚माणेन न्यायेन व्य‚तिरेक‚{\tiny $_{lb}$}‚ग‚तिर्भ‚व‚ति । व्य‚तिरेके चान्व‚य‚ग‚तिः । त‚त‚स्त्रिरूपं लिङ्गं प्र‚काश्य‚म‚भिन्न‚म् । न च य‚त्राभि‚{\tiny $_{lb}$}‚धेय‚भेद‚स्त‚त्र साम‚र्थ्य‚ग‚म्योऽप्य‚र्थो\edtext{}{\lemma{र्थो}\Bfootnote{प्र‚योज‚न‚म्--\cite{dp-msD-n}}} भिद्य‚ते । य‚स्मात् पीनो देव‚द‚त्तो दिवा न भुङ्क्ते पीनो‚{\tiny $_{lb}$}‚ देव‚द‚त्तो रात्रौ भुङ्क्ते इत्य‚न‚योर्वाक्य‚योर‚भिधेय‚भेदेपि ग‚म्य‚मान‚मेक‚मेव\edtext{}{\lemma{मेव}\Bfootnote{ग‚म्यं दिवा भोज‚नाभाव‚विशेषं पीन‚त्वं रात्रिभोज‚न‚कार्य‚मेक‚मेव--\cite{dp-msD-n}}} त‚द्व‚दिहाभिधेय‚भेदेपि‚{\tiny $_{lb}$}‚ ग‚म्य‚सानं व‚स्त्वेक‚मेव ॥
	\pend% ending standard par
      
	  \endgroup
	‚{\tiny $_{lb}$}‚

	  \pstart \leavevmode% starting standard par
	केन क‚स्य किं कृतं च साध‚र्म्यं वैध‚र्म्यं चेत्याह--\textbf{दृष्टान्ते}ति । \textbf{सादृश्यं हेतुकृत‚मि}ति‚{\tiny $_{lb}$}‚ हेतुस‚द्भाव‚द्वार‚क‚म् । \textbf{असादृश्यं च हेतुकृत‚मि}ति हेतुस‚द्भावास‚द्भाव‚द्वार‚कं द्र‚ष्ट‚व्य‚म् । व‚तुब‚र्थं‚{\tiny $_{lb}$}‚ प्र‚योज‚यितुमाह--\textbf{त‚त्रेति} वाक्योप‚न्यासे । \textbf{य‚स्य वाक्य‚स्य साध‚र्म्य} सादृश्य‚म\textbf{भिधेय‚म}स्ति ।‚{\tiny $_{lb}$}‚ एत‚देवोदाह‚र‚णेन द‚र्श‚य‚न्नाह--\textbf{य‚थे}ति । \textbf{य‚त्कृत‚क‚मि}ति । य‚द् य‚द् कृत‚क‚मिति वीप्सार्थों‚{\tiny $_{lb}$}‚ विव‚क्षितः, \textbf{त‚दि}त्य‚त्रापि । \textbf{त‚था च कृत‚कः श‚ब्द} इति प‚क्ष‚ध‚र्म‚ताक‚थ‚न‚मिद‚म् । न त्वेवं‚{\tiny $_{lb}$}‚ प‚क्ष‚ध‚र्मो द‚र्श‚नीयः । कृत‚क‚श्च श‚ब्दः इत्येताव‚तैव ग‚तार्थ‚त्वेन त‚थाश‚ब्द‚स्य वैय‚र्थ्यात् । त‚तः‚{\tiny $_{lb}$}‚ कृत‚क‚श्च श‚ब्दः इत्येव द‚र्श‚नीयः । इत‚र‚था प‚रेषामिवोप‚न‚य‚प्र‚योगः स्यात् । स चायुक्त‚{\tiny $_{lb}$}‚ इति । \textbf{य‚न्नित्य‚मि}ति साध्याभावानुवाद\textbf{स्त‚द‚कृत‚क‚मि}ति साध‚नाभाव‚विधिः ॥
	\pend% ending standard par
      ‚{\tiny $_{lb}$}‚

	  \pstart \leavevmode% starting standard par
	\textbf{अभिधेयं भिन्न‚मि}ति ब्रुव‚तोऽय‚माश‚यः--साध‚र्म्य‚व‚त्प्र‚योग‚स्यान्व‚यः प‚क्ष‚ध‚र्म‚ता चाभिधेया ।‚{\tiny $_{lb}$}‚ \textbf{अभिन्नं} साधार‚णं \textbf{प्र‚काश्यं} द्व‚योः प्र‚योग‚योः ।
	\pend% ending standard par
      ‚{\tiny $_{lb}$}‚

	  \pstart \leavevmode% starting standard par
	\textbf{अर्थः} प्र‚योज‚न‚वाच्याचार्येणोक्तो नाभिधेय‚वाचीति द‚र्श‚य‚ति \textbf{य‚दुद्दिश्ये}ति स्प‚ष्ट‚य‚न्‚{\tiny $_{lb}$}‚ प्र‚युज्य‚तेऽनेनेति \textbf{प्र‚योज‚नं} साध‚न‚वाक्य‚स्य प्र‚व‚र्त्त‚कं लिङ्ग‚व‚स्तूक्तं न फ‚लं प्र‚योज‚न‚मिति द‚र्श‚य‚ति ।‚{\tiny $_{lb}$}‚ \textbf{प्र‚काश‚यित‚व्यं} रूप‚त्र‚य‚योगिलिङ्गं त\textbf{च्चाभिन्नं} साधार‚णं द्व‚योर‚पि प्र‚योग‚योः, द्वाभ्याम‚पि त‚स्यैव‚{\tiny $_{lb}$}‚ प्र‚तिपाद‚नात् । अनुमान‚हेतुत्वाद‚नुमाने साध‚र्म्य‚वैध‚र्म्य‚व‚ती वाक्ये क‚थिते । \textbf{त‚त} इति \textbf{प्र‚योज‚न-}‚{\tiny $_{lb}$}‚ स‚मानाधिक‚र‚णं न हेतुप‚द‚मेत‚त् ॥
	\pend% ending standard par
      ‚{\tiny $_{lb}$}‚

	  \pstart \leavevmode% starting standard par
	न‚न्व‚भिधेय‚मेव प्र‚काश्यं त‚त्क‚थं प्र‚काश्याभेद इत्याश‚ङ्क्याह--\textbf{एत‚दुक्तं भ‚व‚ती}ति ।‚{\tiny $_{lb}$}‚ \textbf{अन्य‚त्प्र‚काश्य‚मि}ति साम‚र्थ्य‚ग‚म्यं प्र‚काश्य‚म् । न तु त‚त्राभिधाव्यापार इत्याकूत‚म् । \textbf{त‚त्रे}ति‚{\tiny $_{lb}$}‚ वाक्योप‚क्षेपे ।
	\pend% ending standard par
      ‚{\tiny $_{lb}$}‚

	  \pstart \leavevmode% starting standard par
	य‚दि साध‚र्म्य‚व‚द्वाक्येऽन्व‚योऽभिधेय‚स्त‚र्हि क‚थं व्य‚तिरेकः प्र‚काश्य‚तां ग‚तः ? बैध‚र्म्य‚वाक्ये‚{\tiny $_{lb}$}‚ ‚{\tiny $_{lb}$}‚ \leavevmode\ledsidenote{\textenglish{154/dm}}‚{\tiny $_{lb}$}‚ 
	  
	त‚त्र साध‚र्म्य‚व‚त्प्रोगः\edtext{}{\lemma{त्प्रोगः}\Bfootnote{०र्म्य‚व‚त् य‚दुप०--\cite{dp-msD} \cite{dp-msB} \cite{dp-edP} \cite{dp-edH} \cite{dp-edE} \cite{dp-edN}}}--य‚दुप‚ल‚ब्धिल‚क्ष‚ण‚प्राप्तं स‚न्नोप‚ल‚भ्य‚ते सोऽस‚{\tiny $_{lb}$}‚द्व्य‚व‚हार‚विष‚यः सिद्धः । य‚थाऽन्यः क‚श्रिद् दृष्टः श‚श‚विषाणादिः । नोप‚ल‚भ्य‚ते‚{\tiny $_{lb}$}‚ च क्व‚चित् प्र‚देश‚विशेष\edtext{}{\lemma{विशेष}\Bfootnote{विष‚ये उप० \cite{dp-msC}}} उप‚ल‚ब्धिल‚क्ष‚ण‚प्राप्तो घ‚ट\edtext{}{\lemma{ट}\Bfootnote{घ‚ट इति ॥ \cite{dp-msD} \cite{dp-msB} \cite{dp-edP} \cite{dp-edH} \cite{dp-edE} \cite{dp-edN}}} इत्य‚नुप‚ल‚ब्धिप्र‚योगः ॥ ८ ॥‚{\tiny $_{lb}$}‚ 
	  
	त‚त्रेति त‚योः साध‚र्म्य‚वैध‚र्म्य‚व‚तोर‚नुमान‚योः साध‚र्म्य‚व‚त् \edtext{}{\lemma{त्}\Bfootnote{ताव‚दुहार‚ण‚मुदाह‚र्त्तुम‚नुप० \cite{dp-msC} \cite{dp-msB}}}ताव‚दुहार‚न्न‚नुप‚ल‚ब्धिमाह—‚{\tiny $_{lb}$}‚य‚दित्यादिना । य‚दुप‚ल‚ब्धिल‚क्ष‚ण‚प्राप्तं--य‚द् दृश्यं स‚न्नोप‚ल‚भ्य‚ते इत्य‚नेन दृश्यानुप‚ल‚म्भोऽ‚{\tiny $_{lb}$}‚नूद्य‚ते । सोऽस‚द्व्य‚व‚हार‚विष‚यः\edtext{}{\lemma{यः}\Bfootnote{विष‚य‚स्त‚द० \cite{dp-msB} \cite{dp-msC} \cite{dp-msD}}} सिद्धः--त‚द‚स‚दिति व्य‚व‚ह‚र्त‚व्य‚मित्य‚र्थः । अनेनास‚द्व्य‚व‚हार‚{\tiny $_{lb}$}‚योग्य‚त्व‚स्य\edtext{}{\lemma{स्य}\Bfootnote{योग्य‚त्वे विधिः \cite{dp-msB}}} विधिः कृतः । त‚त‚श्चास‚द्व्य‚व‚हार‚योग्य‚त्वे\edtext{}{\lemma{त्वे}\Bfootnote{०हार‚स्य योग्य० \cite{dp-msA} \cite{dp-msB} \cite{dp-edP} \cite{dp-edH} \cite{dp-edE} \cite{dp-edN}}} दृश्यानुप‚ल‚म्भो निय‚तः क‚थितः ।‚{\tiny $_{lb}$}‚ दृश्य‚म‚नुप‚ल‚ब्ध‚म‚स‚द्व्य‚व‚हार‚योग्य‚मेवेत्य‚र्थः । साध‚न‚स्य च साध्येऽर्थे निय‚त‚त्व‚क‚थ‚नं व्याप्ति-‚{\tiny $_{lb}$}‚ च य‚दि व्य‚तिरेकोऽभिधेयः, क‚थ‚म‚न्व‚यः प्र‚काश्य‚तामाप‚न्न इत्याह--\textbf{अन्व‚य} इति । हिर्य‚स्मात् ।‚{\tiny $_{lb}$}‚ \textbf{व‚क्ष्य‚माणेन साध‚र्म्येणापि ही} त्यादिना प्र‚तिपाद‚यिष्य‚माणेन न्यायेन युक्ता ।
	\pend% ending standard par
      ‚{\tiny $_{lb}$}‚

	  \pstart \leavevmode% starting standard par
	स्यान्म‚त‚म्--य‚योर‚भिधेयं भिन्नं त‚योः साम‚र्थ्य‚ग‚म्य‚म‚प्य‚व‚श्यं भिद्य‚ते--य‚था ग‚ङ्गायां‚{\tiny $_{lb}$}‚ घोषः, कूपे ग‚र्ग‚कुल‚मित्याश‚ङ्क्याह--\textbf{न चे}ति । \textbf{चो} य‚स्माद‚र्थे । \textbf{अर्थः} प्र‚योज‚नं \textbf{य‚दुद्दिश्य‚{\tiny $_{lb}$}‚ प्र‚वृत्तं वाक्य‚म्} ।
	\pend% ending standard par
      ‚{\tiny $_{lb}$}‚

	  \pstart \leavevmode% starting standard par
	उप‚प‚त्तिमाह--\textbf{य‚स्मादि}ति । एक‚स्य वाक्य‚स्य दिवा भोज‚नाभावोऽभिधेयोऽन्य‚स्य‚{\tiny $_{lb}$}‚ रात्रिभोज‚न‚मित्य‚न‚योर्वाक्य\leavevmode\ledsidenote{\textenglish{58b/ms}}योर‚भिधेय‚भेदोऽस्ति । त‚स्मिन् स‚त्य‚पि य‚था भोज‚न‚विशिष्टं‚{\tiny $_{lb}$}‚ देव‚द‚त्त‚स्य पीन‚त्वं प्र‚तिपाद्यं न भिद्य‚ते \textbf{त‚द्व‚द‚त्राभिधेय‚स्या}न्व‚य‚प‚क्ष‚ध‚र्म‚ताल‚क्ष‚ण‚स्य व्य‚तिरेक‚{\tiny $_{lb}$}‚प‚क्ष‚ध‚र्म‚ताल‚क्ष‚ण‚स्य \textbf{भेदेऽपि ग‚म्य‚मान‚मेक‚म}भिन्न‚म् ।
	\pend% ending standard par
      ‚{\tiny $_{lb}$}‚

	  \pstart \leavevmode% starting standard par
	अथ ग‚म्य‚यानं साम‚र्थ्यात् प्र‚तीय‚मान‚मुच्य‚ते । त‚च्च दृष्टान्त‚दार्ष्टान्तिक‚योर्भिद्य‚त एव ।‚{\tiny $_{lb}$}‚ त‚थाहि दिवा भोज‚न‚निषेध‚वाक्य‚स्य ग‚म्य‚मानं रात्रिभोज‚नं रात्रिभोज‚न‚विधान‚वाक्य‚स्य तु‚{\tiny $_{lb}$}‚ ग‚म्य‚मानं दिवा भोज‚न‚निषेध‚न‚म्, त‚था दार्ष्टान्तिकेपि \textbf{साध‚न} \edtext{\textsuperscript{*}}{\lemma{*}\Bfootnote{साध‚र्म्य}} व‚द् वाक्ये व्य‚तिरेको‚{\tiny $_{lb}$}‚ ग‚म्य‚मानो वैध‚र्म्य‚व‚द्वाक्ये चान्व‚यो ग‚म्य‚मानः । य‚देक‚स्याभिधेयं त‚देक‚स्य ग‚म्य‚मान‚म्, य‚द‚न्य‚स्य‚{\tiny $_{lb}$}‚ ग‚म्य‚मानं त‚दित‚र‚स्याभिधेय‚मिति संक्षेपः । त‚तः क‚थ‚मुच्य‚ते वाक्य‚योर्ग‚म्य‚मान‚मेक‚मिति ।‚{\tiny $_{lb}$}‚ किन्नोच्य‚ते ? ग‚म्य‚मान‚श‚ब्द‚स्येहान्यार्थ‚स्य विव‚क्षित‚त्वात् । त‚थाहि ग‚म्य‚मान‚श‚ब्देनात्राभिधेयं‚{\tiny $_{lb}$}‚ साम‚र्थ्य‚प्र‚काश्य‚ञ्च । य‚त्तु द्व‚यं प्र‚तीय‚मान‚तामात्रेणोपाधिना विव‚क्षितं त‚च्च दृष्टान्त‚वाक्य‚यो‚{\tiny $_{lb}$}‚र्भोज‚न‚निमित्त‚पीन‚त्व‚ल‚क्ष‚णं \textbf{दा}र्ष्टान्तिक‚वाक्य‚योश्चान्व‚य‚तिरेक‚प‚क्ष‚ध‚र्म‚तात्म‚क‚रूप‚त्र‚य‚योग‚लिङ्ग‚{\tiny $_{lb}$}‚ल‚क्ष‚ण‚मेक‚म‚भिन्न‚मित्य‚न‚व‚द्य‚मेत‚त् ॥
	\pend% ending standard par
      ‚{\tiny $_{lb}$}‚

	  \pstart \leavevmode% starting standard par
	साध‚र्म्य‚म‚भिधेयं य‚स्य विद्य‚ते त‚दुदाह‚र‚णेन द‚र्श‚य\textbf{न्न‚नुप‚ल‚ब्धिमाह । ताव‚च्छ‚ब्दः क्र‚मे} ।‚{\tiny $_{lb}$}‚ ‚{\tiny $_{lb}$}‚ \leavevmode\ledsidenote{\textenglish{155/dm}}‚{\tiny $_{lb}$}‚ 
	  
	क‚थ‚न‚म् । य‚थोक्त‚म् व्याप्ति\edtext{}{\lemma{व्याप्ति}\Bfootnote{व्याप्तिव्या० \cite{dp-msA}}} र्व्याप‚क‚स्य त‚त्र भाव एव, व्याप्य‚स्य \edtext{}{\lemma{स्य}\Bfootnote{स्य च \cite{dp-msB} \cite{dp-msC} \cite{dp-msD} \cite{dp-edP} \cite{dp-edH} \cite{dp-edE} \cite{dp-edN}}}वा त‚त्रैव भावः इति ।‚{\tiny $_{lb}$}‚ \href{http://sarit.indology.info/?cref=hb.1.4}{हेतु० पृ० ५३} व्याप्तिसाध‚न‚स्य प्र‚माण‚स्य विष‚यो दृष्टान्तः । त‚मेव द‚र्श‚यितुमाह--य‚थान्य‚{\tiny $_{lb}$}‚ इति । साध्य‚ध‚र्मिणोऽन्यो दृष्टान्त इत्य‚र्थः । दृष्ट इति प्र‚माणेन\edtext{}{\lemma{माणेन}\Bfootnote{प्र‚त्य‚क्षेण निश्चित इति न व्याक‚र्त‚व्य‚म्--\cite{dp-msD-n}}} निश्चितः । श‚श‚विषाणं हि‚{\tiny $_{lb}$}‚ न च‚क्षुषा विष‚यीकृत‚म् अपि तु प्र‚माणेन दृश्यानुप‚ल‚म्भेनास‚द्व्य‚व‚हार‚योग्यं विज्ञात‚म् ।‚{\tiny $_{lb}$}‚ श‚श‚विषाण‚मादिर्य‚स्यास‚द्व्य‚व‚हार‚विष‚य‚स्य स त‚थोक्तः । श‚श‚विषाणादौ हि दृश्यानुप‚ल‚म्भ‚{\tiny $_{lb}$}‚मात्र‚निमित्तोऽस‚द्व‚य‚व‚हारः\edtext{}{\lemma{हारः}\Bfootnote{निमित्तान्त‚राभावोप‚द‚र्श‚नेन दृश्यानुप‚ल‚ब्धिव्य‚तिरिक्तास‚म्भ‚विनिमित्त‚श्चास‚द्व्य‚{\tiny $_{lb}$}‚व‚हारः--\cite{dp-msD-n}}} प्र‚माणेन सिद्धः । त‚त एव\edtext{}{\lemma{एव}\Bfootnote{त‚तः प्र‚मा० \cite{dp-msC} दृश्यानुप‚ल‚म्भादेव--\cite{dp-msD-n}}} प्र‚माणाद‚नेन\edtext{}{\lemma{नेन}\Bfootnote{व्याप्तिसाध‚केन--\cite{dp-msD-n}}} वाक्येनाभिधीय‚माना‚{\tiny $_{lb}$}‚ व्याप्तिर्ज्ञात‚व्या । ‚{\tiny $_{lb}$}‚ 
	  
	संप्र‚ति व्याप्तिं क‚थ‚यित्वा दृश्यानुप‚ल‚म्भ‚स्य प‚क्ष‚ध‚र्म‚त्वं द‚र्श‚यितुमाह--नोप‚ल‚भ्य‚ते चेति ।‚{\tiny $_{lb}$}‚ स च द्विरावृत्त्याऽनुप‚ल‚ब्धिमित्य‚त्रापि द्र‚ष्ट‚व्यः । तेनाय‚म‚र्थः--साध‚र्म्य‚व‚त्ताव‚दुदाह‚र‚न्न‚नुप‚{\tiny $_{lb}$}‚ल‚ब्धिमाह । प‚श्चाद् वैध‚र्म्य‚व‚दुदाह‚र‚न् व‚क्ष्य‚ति । त‚थाऽनुप‚ल‚ब्धिं ताव‚दाह प‚श्चात्स्व‚भाव‚कार्ये‚{\tiny $_{lb}$}‚ व‚क्ष्य‚तीति ।
	\pend% ending standard par
      ‚{\tiny $_{lb}$}‚

	  \pstart \leavevmode% starting standard par
	न‚न्वाचार्येणैव साध‚र्म्य‚व‚द् वैध‚र्म्य‚व‚च्चोदाहृत‚म‚त्रेति किमिति \textbf{ध‚र्मोत्त‚रेण}--य‚था य‚त्कृत‚क‚{\tiny $_{lb}$}‚मित्यादिना साध‚र्म्य‚व‚द् वैध‚र्म्य‚व‚च्चोदाहृतं प्रागिति चेत् । नैष दोषः । दृष्टान्त‚साध्य‚ध‚र्मिणोः‚{\tiny $_{lb}$}‚ सादृश्याख्यं साध‚र्म्य‚म्, वैस‚दृश्याख्यं वैध‚र्म्यं च हेतुद्वार‚क‚मेव न तु सामान्येन, अन्य‚था प्र‚तियोग्य‚{\tiny $_{lb}$}‚पेक्ष‚याऽपि साध‚र्म्यं वैध‚र्म्यं च केन‚चिदाकारेणास्तीति न त‚न्निराकृतं स्यादिति द‚र्श‚यितुं त‚दुदा‚{\tiny $_{lb}$}‚हृत‚म्, न त्वाचार्येण नोदाहृत‚मित्युदाहृत‚मिति । तेनात्राप्याचार्यीये निद‚र्श‚ने हेतुकृत‚मेव‚{\tiny $_{lb}$}‚ त‚त्प्र‚त्येत‚व्यं व्य‚व‚स्थित‚म् ।
	\pend% ending standard par
      ‚{\tiny $_{lb}$}‚

	  \pstart \leavevmode% starting standard par
	\textbf{अस‚द्व्य‚व‚हारः}--अस‚दिति ज्ञान‚म‚स‚दित्य‚भिधानं निःश‚ङ्का च ग‚माग‚मादिका प्र‚वृत्तिः ।‚{\tiny $_{lb}$}‚ य‚तो दृश्यानुप‚ल‚म्भोऽभूद‚तोऽस‚द्व्य‚व‚हार‚योग्य‚त्वं विहित‚म् । त‚त‚स्त‚स्मात् । \textbf{चो}ऽव‚धार‚णे ।
	\pend% ending standard par
      ‚{\tiny $_{lb}$}‚

	  \pstart \leavevmode% starting standard par
	अय‚माश‚यः--य‚द‚नूद्य‚ते त‚द् व्याप्य‚म् । य‚द् विधीय‚ते त‚द् व्याप‚क‚म् । व्याप्यं च‚{\tiny $_{lb}$}‚ व्याप‚के निय‚तं भ‚व‚तीति । एव‚मुत्त‚र‚त्राप्य‚नुवाद‚विधिक्र‚मो द्र‚ष्ट‚व्यः ।
	\pend% ending standard par
      ‚{\tiny $_{lb}$}‚

	  \pstart \leavevmode% starting standard par
	अथ द्व्य‚व‚य‚वे साध‚न‚वाक्ये द‚र्श‚यित‚व्ये व्याप्तिः प‚क्ष‚ध‚र्म‚ता च द‚र्श‚नीया । न चात्र‚{\tiny $_{lb}$}‚ व्याप्तिरुप‚द‚र्शिता, केव‚ल‚म‚नुवाद‚विधिक्र‚मो द‚र्शितः, प‚क्ष‚ध‚र्म‚ता च द‚र्श‚यिष्य‚ते । त‚त्क‚थं प‚रिपूर्णं‚{\tiny $_{lb}$}‚ साध‚न‚वाक्य‚मिदं भ‚विष्य‚तीत्याश‚ङ्क्याह--\textbf{साध‚न‚स्ये}ति । \textbf{चो} हेतौ ।
	\pend% ending standard par
      ‚{\tiny $_{lb}$}‚

	  \pstart \leavevmode% starting standard par
	न‚नु प‚रोक्षार्थ‚प्र‚तिप‚त्तौ स‚र्व‚थाऽनुप‚योगी दृष्टान्त‚स्त‚त्किं तेनाख्यातेनेत्याह \textbf{व्याप्ती}ति ।‚{\tiny $_{lb}$}‚ व्याप्य‚व्याप‚क‚ध‚र्म‚ल‚क्ष‚णा \leavevmode\ledsidenote{\textenglish{59a/ms}} व्याप्तिः साध्य‚ते निश्चीय‚ते येन प्र‚माणेन त‚स्य \textbf{विष‚यो} य‚त्र‚{\tiny $_{lb}$}‚ ‚{\tiny $_{lb}$}‚ \leavevmode\ledsidenote{\textenglish{156/dm}}‚{\tiny $_{lb}$}‚ 
	  
	प्र‚देश एक‚देशः पृथिव्याः । स एव विशिष्य‚तेऽन्य‚स्मादिति विशेषः एकः । प्र‚देश‚विशेष‚{\tiny $_{lb}$}‚ इत्येक‚स्मिन् प्र‚देशे । क्व‚चिदिति । प्र‚तिप‚त्तुः प्र‚त्य‚क्ष\edtext{}{\lemma{क्ष}\Bfootnote{प्र‚त्य‚क्षे । एको० \cite{dp-msC} \cite{dp-msD} \cite{dp-edE} \cite{dp-edN}}} एकोऽपि प्र‚देशः । स एवाभाव‚{\tiny $_{lb}$}‚व्य‚व‚हाराधिक‚र‚णं यः प्र‚तिप‚त्तुः प्र‚त्य‚क्षो नान्यः । उप‚ल‚ब्धिल‚क्ष‚ण‚प्राप्त इति दृश्यः । य‚था‚{\tiny $_{lb}$}‚ चास‚तोऽपि घ‚ट‚स्य स‚मारोपित‚मुप‚ल‚ब्धिल‚क्ष‚ण‚प्राप्त‚त्वं त‚था व्याख्यात‚म् ॥ ‚{\tiny $_{lb}$}‚ 
	  
	स्व‚भाव‚हेतोः साध‚र्म्य‚व‚न्तं प्र‚योगं द‚र्श‚यितुमाह-- ‚{\tiny $_{lb}$}‚ 
	  
	त‚था स्व‚भाव‚हेतोः प्र‚योगः--य‚त् स‚त् त‚त् स‚र्व‚म‚नित्य‚म्, य‚था घ‚टादि‚{\tiny $_{lb}$}‚ रिति शुद्ध‚स्य\edtext{}{\lemma{स्य}\Bfootnote{शुद्ध‚स्व‚भाव‚स्य प्र‚योगः \cite{dp-msC} ।}} स्व‚भाव‚हेतोः प्र‚योगः ॥ ९ ॥‚{\tiny $_{lb}$}‚ 
	  
	त‚थेति । य‚थाऽनुप‚ल‚ब्धेस्त‚था स्व‚भाव‚हेतोः साध‚र्म्य‚वान् प्र‚योग इत्य‚र्थः । य‚त् स‚दिति‚{\tiny $_{lb}$}‚ प्र‚वृत्तं प्र‚माणं साध्य‚साध‚न‚योर्व्याप्तिम‚व‚स्य‚ति । स च विष‚यो \textbf{दृष्टो} निश्चितः साध्य‚साध‚न‚यो‚{\tiny $_{lb}$}‚र‚न्तोऽव‚सानं य‚थायोगं निय‚त‚त्व‚निय‚म‚विष‚य‚त्व‚निपुणो य‚स्मिन्निति व्युत्प‚त्त्या दृष्टान्त‚श‚ब्दोऽ‚{\tiny $_{lb}$}‚भिल‚प्यः । त‚मेव ख्याप‚यितुमाहाचार्यः । अनेनैत‚दाकूत‚म्--व्याप्तिसाध‚क‚प्र‚माण‚स्याधिक‚र‚ण‚तां‚{\tiny $_{lb}$}‚ ग‚च्छ‚न् दृष्टान्तः साध‚नाव‚य‚व‚स्य व्याप्तेः प्र‚तिप‚त्त्य‚ङ्ग‚म् । न तु साक्षात्साध‚न‚स्य । नापि‚{\tiny $_{lb}$}‚ साध्य‚सिद्धेः । त‚द्व‚च‚न‚म‚पि त‚त्स्मार‚क‚त्वेन साध‚न‚वाक्य उप‚युज्य‚ते । अत एव व‚च‚न‚साध‚न‚{\tiny $_{lb}$}‚वाक्य‚स्याव‚य‚वोऽथ च प्र‚योक्त‚व्य इति । कुतोऽन्य इत्याह--\textbf{साध्य‚ध‚र्मिण} इति । \textbf{श‚श‚वि}‚{\tiny $_{lb}$}‚षाणादेश्च व्याप्तिसाध‚क‚प्र‚माणाधिक‚र‚ण‚त्वेन दृष्टान्त‚रूप‚त्वाद् \textbf{दृष्टान्त इत्य‚र्थ} इति स्प‚ष्ट‚य‚ति ।
	\pend% ending standard par
      ‚{\tiny $_{lb}$}‚

	  \pstart \leavevmode% starting standard par
	न‚नु दृष्ट‚श्च‚क्षुषा ज्ञात इति किं न व्याख्याय‚ते ? किं पुन‚रेवं व्याख्याय‚त इत्याह—‚{\tiny $_{lb}$}‚\textbf{श‚शेति । ही}ति य‚स्मात् । \textbf{विष‚यीकृतं} विज्ञात‚मिति चातीते निष्ठां प्र‚युञ्जानः प्राग्भावि‚{\tiny $_{lb}$}‚ व्याप्तिग्र‚ह‚णं द‚र्श‚य‚ति । क‚थं पुनः श‚श‚विषाणादि दृष्टान्तो येन सा ख्याप्य‚त इत्याह--\textbf{श‚शेति} ।‚{\tiny $_{lb}$}‚ \textbf{हि}र्य‚स्मात् । \textbf{दृश्यानुप‚ल‚म्भ} एव \textbf{त‚न्मात्रं} त‚न्नि\textbf{मित्तं} य‚स्य स त‚था । अनेन व्याप्तिसाध‚क‚{\tiny $_{lb}$}‚प्र‚माणाधिक‚र‚ण‚त्वात्त‚स्य दृष्टान्त‚रूप‚तामाह । किं त‚त्प्र‚माणं येन त‚त्र प्र‚वृत्तेन दृश्यानुप‚ल‚म्भा‚{\tiny $_{lb}$}‚भाव‚व्य‚व‚हार‚योग्य‚त्व‚योर्व्याप्तिः साध्य‚त इति चेत् । उच्य‚ते । \textbf{वाद‚न्यायोक्तेन} न्यायेन बुद्धिव्य‚{\tiny $_{lb}$}‚प‚देशाभावादेर‚स‚द्व्य‚व‚हारानिमित्त‚त्वेन निमित्तान्त‚राभावे दृश्यानुप‚ल‚म्भ एवान्य‚निर‚पेक्षो‚{\tiny $_{lb}$}‚ निमित्त‚म् । य‚च्च य‚न्मात्र‚निमित्तं त‚त्त‚स्मिन् स‚ति भ‚व‚ति । य‚था बीजादिसाम‚ग्रीमात्र‚नि‚{\tiny $_{lb}$}‚मित्तोऽङ्कुरः स‚ति त‚स्मिन् भ‚व‚ति । दृश्यानुप‚ल‚म्भ‚मात्र‚निमित्त‚श्चास‚द्व्य‚व‚हार इत्य‚नुमानं त‚त्र‚{\tiny $_{lb}$}‚ प्र‚बृत्तं साध्य‚साध‚न‚योर्व्याप्तिम‚व‚स्य‚तीति ।
	\pend% ending standard par
      ‚{\tiny $_{lb}$}‚

	  \pstart \leavevmode% starting standard par
	\textbf{अनेन वाक्येन} य‚दुप‚ल‚ब्धिल‚क्ष‚ण‚प्राप्त‚मित्यादिनाऽ\textbf{भिधीय‚माना} प्र‚काश्य‚माना । त‚त \textbf{एव}‚{\tiny $_{lb}$}‚ प्राक्प्र‚वृत्ताद‚न‚न्त‚रोक्ताद‚नुमानात्म‚नः प्र‚माणाद् \textbf{व्याप्तिर्ज्ञात‚व्या} ।
	\pend% ending standard par
      ‚{\tiny $_{lb}$}‚

	  \pstart \leavevmode% starting standard par
	एत‚दुक्तं भ‚व‚ति त‚त्प्र‚माण‚सिद्धैव व्याप्तिर‚नेन वाक्येन स्म‚र्य‚त इति ॥
	\pend% ending standard par
      ‚{\tiny $_{lb}$}‚

	  \pstart \leavevmode% starting standard par
	\textbf{स्व‚भावेत्यादि}ना स्व‚भाव‚हेतोः साध‚र्म्य‚व‚त्प्र‚योगं विव‚रितुमुप‚क्र‚म‚ते । \textbf{स‚र्व}श‚ब्द‚स्याऽ‚{\tiny $_{lb}$}‚शेष‚ताऽर्थः । त‚यैव प्र‚तिपादित‚या साध‚न‚स्य साध्याय‚त्त‚ताख्यो यो निय‚मः स प्र‚तिपादितो‚{\tiny $_{lb}$}‚ ‚{\tiny $_{lb}$}‚ \leavevmode\ledsidenote{\textenglish{157/dm}}‚{\tiny $_{lb}$}‚ स‚त्त्व‚म‚नूद्य त‚त् स‚र्व‚म‚नित्य‚मित्य‚नित्य‚त्वं विधीय‚ते । स‚र्व‚ग्र‚ह‚णं च निय‚मार्थ‚म् । स‚र्व‚म‚नित्य‚म् ।‚{\tiny $_{lb}$}‚ न किञ्चिन्नानित्य‚म् । य‚त् स‚त् त‚द‚नित्य‚मेव । अनित्य‚त्वाद‚न्य‚त्र नित्य‚त्वे स‚त्त्वं नास्तीत्येवं‚{\tiny $_{lb}$}‚ स‚त्त्व‚म‚नित्य‚त्वे साध्ये निय‚तं ख्यापितं भ‚व‚ति । त‚था च स‚ति व्याप्तिप्र‚द‚र्श‚न‚वाक्य‚मिद‚म् ।‚{\tiny $_{lb}$}‚ य‚था घ‚टादिरिति \edtext{}{\lemma{टादिरिति}\Bfootnote{व्याप्तिसाध‚न‚स्य \cite{dp-msA} \cite{dp-msB} \cite{dp-msC} \cite{dp-msD} \cite{dp-edP} \cite{dp-edH} \cite{dp-edE} \cite{dp-edN} नित्य‚क्र‚म‚यौग‚प‚द्याभ्याम् अर्थ‚{\tiny $_{lb}$}‚क्रियाविरोधादिति विप‚र्य‚य‚बाध‚कं प्र‚माण‚म्--\cite{dp-msD-n}}}व्याप्तिसाध‚क‚स्य प्र‚माण‚स्य विष‚य‚क‚थ‚न‚मेत‚त् । शुद्ध‚स्येति निर्विशेष‚ण‚स्य‚{\tiny $_{lb}$}‚ स्व‚भाव‚स्य \edtext{}{\lemma{स्य}\Bfootnote{प्र‚योग‚स्य विशेष‚णं द‚र्श० \cite{dp-msB}}}प्र‚योगः ।
	\pend% ending standard par
      ‚{\tiny $_{lb}$}‚

	  \pstart \leavevmode% starting standard par
	स‚विशेष‚णं द‚र्श‚यितुमाह--
	\pend% ending standard par
      ‚{\tiny $_{lb}$}‚
	  \bigskip
	  \begingroup
	

	  \pstart \leavevmode% starting standard par
	य‚दुत्प‚त्तिम‚त् त‚द‚नित्य‚मिति स्व‚भाव‚भूत‚ध‚र्म‚भेदेन स्व‚भाव‚स्य प्र‚योगः ॥ १० ॥
	\pend% ending standard par
      
	  \endgroup
	‚{\tiny $_{lb}$}‚
	  \bigskip
	  \begingroup
	

	  \pstart \leavevmode% starting standard par
	य‚दुत्प‚त्तिम‚दिति ।\edtext{\textsuperscript{*}}{\lemma{*}\Bfootnote{य‚दुत्प‚त्तिः \cite{dp-msC} य‚दुत्प‚त्तिम‚दिति उत्प‚त्तिम‚त्त्व‚म‚नू० \cite{dp-msA}}} उत्प‚तिः स्व‚रूप‚लाभो\edtext{}{\lemma{लाभो}\Bfootnote{लाभःस य‚स्यास्ति--\cite{dp-msB}}} य‚स्यास्ति त‚द् उत्प‚त्तिम‚त् । उत्प‚त्ति‚{\tiny $_{lb}$}‚ मित्त्व‚म‚नूद्य त‚द‚नित्य‚मित्य‚नित्य‚त्व‚विधिः\edtext{}{\lemma{विधिः}\Bfootnote{विधेः \cite{dp-msA} \cite{dp-edP} \cite{dp-edH}}} । त‚था च स‚त्युत्प‚त्तिम‚त्त्व‚म‚नित्य‚त्वे निय‚त‚माख्यात‚म् ।
	\pend% ending standard par
       ‚{\tiny $_{lb}$}‚ 

	  \pstart \leavevmode% starting standard par
	स्व‚भावं\edtext{}{\lemma{भावं}\Bfootnote{अग्रे स्व‚यं दुर्वेकेण स्व‚भाव‚भूतः स्व‚भावात्म‚कः इत्यादिना स्व‚भाव‚भूतः इत्येव‚{\tiny $_{lb}$}‚ ध‚र्मोत्त‚र‚संम‚तः पाठ इति गृहीत‚स्त‚थापि अत्र तेनैव स्व‚भावं भूतः इत्येवंरूपेण गृहीतोऽस्ति‚{\tiny $_{lb}$}‚ इति व्याख्यानुरोधाद् भाति-सं० । स्व‚भाव‚भूतः-\cite{dp-msA} \cite{dp-msB} \cite{dp-msC} \cite{dp-msD} \cite{dp-edP} \cite{dp-edH} \cite{dp-edE} \cite{dp-edN}}} भूतः \edtext{}{\lemma{भूतः}\Bfootnote{स्व‚भावात्म‚को \cite{dp-msA} \cite{dp-msB} \cite{dp-msC} \cite{dp-msD} \cite{dp-edP} \cite{dp-edH} \cite{dp-edE} \cite{dp-edN}}}त‚दात्म‚को ध‚र्मः । त‚स्य भेदेन । भेदं हेतूकृत्य प्र‚योगः । \edtext{\textsuperscript{*}}{\lemma{*}\Bfootnote{न‚नु स्व‚भाव‚भूत‚स्य क‚थं भ‚द इत्याह--\cite{dp-msD-n}}}अनुत्प‚{\tiny $_{lb}$}‚न्नेभ्यो हि व्यावृत्तिमाश्रित्योत्प‚न्नो भाव\edtext{}{\lemma{भाव}\Bfootnote{भाव उच्य‚ते \cite{dp-msA} \cite{dp-msB} \cite{dp-msD} \cite{dp-edP} \cite{dp-edH} \cite{dp-edE} \cite{dp-edN}}} इत्युच्य‚ते । सैव व्यावृत्तिर्य‚दा व्यावृत्त्य‚न्त‚र‚निर‚पेक्षा‚{\tiny $_{lb}$}‚ व‚क्तुमिष्य‚ते त‚दा व्य‚तिरेकिणीव निर्दिश्य‚ते--भाव‚स्य उत्प‚त्तिरिति । त‚या च व्य‚ति‚{\tiny $_{lb}$}‚रिक्त‚येवोत्प‚त्त्या विशिष्टं\edtext{}{\lemma{विशिष्टं}\Bfootnote{विशिष्टं च व‚स्तु--\cite{dp-msC}}} व‚स्तु उत्प‚त्तिम‚दुक्त‚म् । तेन स्व‚भाव‚भूतेन ध‚र्मेण क‚ल्पित‚भेदेन
	\pend% ending standard par
      
	  \endgroup
	‚{\tiny $_{lb}$}‚

	  \pstart \leavevmode% starting standard par
	भ‚व‚तीति \textbf{निय‚म‚ख्याप‚नार्थं स‚र्व‚ग्र‚ह‚णं} भ‚व‚ति । अन्य‚था निःशेष‚त्वानुप‚प‚त्तेरिति । \textbf{स‚र्व‚मि}त्या‚{\tiny $_{lb}$}‚द्य‚स्यैव स्प‚स्टीक‚र‚ण‚म् । \textbf{त‚था च स‚ति} निय‚त‚त्व‚निय‚म‚विष‚य‚त्व‚ख्याप‚न‚प्र‚कारे स‚ति । \textbf{इदं‚{\tiny $_{lb}$}‚ वाक्यं} य‚त्स‚त्त‚द‚नित्य‚मित्यात्म\textbf{क‚म् । व्याप्तिसाध‚क‚स्य प्र‚माण‚स्येति} य‚स्य क्र‚माक्र‚माऽयोगो न त‚स्य‚{\tiny $_{lb}$}‚ क्व‚चित्साम‚र्थ्यं य‚थाऽऽकाश‚कुशेश‚य‚स्य । अस्ति चाक्ष‚णिके स इति व्याप‚कानुप‚ल‚म्भ‚स‚म्भ‚व‚स्या‚{\tiny $_{lb}$}‚नुमान‚स्येति द्र‚ष्ट‚व्य‚म् । एत‚च्च ब‚हुवाच्य‚म‚न्य‚त्र विप‚ञ्चितं नेहाप्र‚कृत‚त्वात्प्र‚त‚न्य‚ते ।
	\pend% ending standard par
      ‚{\tiny $_{lb}$}‚

	  \pstart \leavevmode% starting standard par
	क‚थं पुन‚रुत्प‚त्तिर्भाव‚स्य विशेष‚ण‚मित्याह--\textbf{स्व‚भाव‚मि}ति । \textbf{स्व‚भावं भूतः} प्राप्त इति‚{\tiny $_{lb}$}‚ क‚र्त्त‚रि निष्ठा द्वितीया\href{http://sarit.indology.info/?cref=Pā.2.1.24}{पाणिनि २. १. २४.}इति योग‚विभागा\leavevmode\ledsidenote{\textenglish{59b/ms}}त्स‚मासः । अस्यैव‚{\tiny $_{lb}$}‚ स्प‚ष्टीक‚र‚णं \textbf{त‚दात्म‚क} इति । य‚दि स्व‚भावः क‚थं विशेष‚ण‚म्, भेदेन त‚स्य द‚र्श‚नाद् इत्याह—‚{\tiny $_{lb}$}‚‚{\tiny $_{lb}$}‚ ‚{\tiny $_{lb}$}‚ \leavevmode\ledsidenote{\textenglish{158/dm}}‚{\tiny $_{lb}$}‚ 
	  
	विशिष्टः स्व‚भावः प्र‚युक्तो द्र‚ष्ट‚व्यः ॥ ‚{\tiny $_{lb}$}‚ 
	  
	य‚त् कृत‚कं त‚द‚नित्य‚मित्युपाधिभेदेन ॥ ११ ॥‚{\tiny $_{lb}$}‚ 
	  
	य‚त् कृत‚क‚मिति कृत‚क‚त्व‚म‚नूद्य अनित्य‚त्वं विधीय‚त इति \edtext{}{\lemma{इति}\Bfootnote{अनिय‚त‚त्वे--\cite{dp-msA}}}अनित्य‚त्वे निय‚तं कृत‚क‚त्व‚{\tiny $_{lb}$}‚मुक्त‚म् । अतो व्याप्तिर‚नित्य‚त्वेन कृत‚क‚त्व‚स्य द‚र्शिता । उपाधिभेदेन स्व‚भाव‚स्य प्र‚योग इति‚{\tiny $_{lb}$}‚ संब‚न्धः । उपाधिर्विशेष‚ण‚म् । त‚स्य भेदेन भिन्नेनोपाधिना विशिष्टः स्व‚भावः प्र‚युक्त इत्य‚र्थः । ‚{\tiny $_{lb}$}‚ 
	  
	\edtext{\textsuperscript{*}}{\lemma{*}\Bfootnote{प‚रार्थानुमाने--\cite{dp-msD-n}}}इह क‚दाचिच्छुद्ध एवार्थ उच्य‚ते, क‚दाचिद‚व्य‚तिरिक्तेन विशेष‚णेन विशिष्टः क‚दाचि‚{\tiny $_{lb}$}‚द्व्य‚तिरिक्तेन । देव‚द‚त्त इति शुद्धः, ल‚म्ब‚क‚र्ण इत्य‚भिन्न‚क‚र्ण‚द्व‚य‚विशिष्टः, चित्र‚गुरिति व्य‚ति‚{\tiny $_{lb}$}‚रिक्त‚चित्र‚ग‚वीविशिष्टः । त‚द्व‚त् स‚त्त्वं शुद्ध‚म्, उत्प‚तिम‚त्त्व‚म‚व्य‚तिरिक्त‚विशेष‚ण‚म्, कृत‚क‚त्वं‚{\tiny $_{lb}$}‚ व्य‚तिरिक्त‚विशेष‚ण‚म् ॥‚{\tiny $_{lb}$}‚ त‚स्येति । \textbf{भेदेन} विशेष्याव्य‚तिरिक्त‚त‚या विशेष‚क‚त्व‚ल‚क्ष‚णेन विक‚ल्प‚स‚न्द‚र्शितेन । स्व‚भाव‚भूतः‚{\tiny $_{lb}$}‚ स्व‚भावात्म‚को ध‚र्म इति च प‚र‚मार्थाभिप्रायेणोक्त‚म् । \textbf{भेदेने}तीयं तृतीया हेताविति द‚र्श‚य‚न्नाह—‚{\tiny $_{lb}$}‚\textbf{भेद‚मि}ति । व्य‚व‚हार‚सिद्धं \textbf{भेद}मुत्प‚त्तुः स‚काशाद‚न्य‚त्वं \textbf{हेतूकृत्य} निब‚न्ध‚नीकृत्य \textbf{प्र‚योगः} स‚वि‚{\tiny $_{lb}$}‚\textbf{शेष‚ण‚स्य} स्व‚भाव‚हेतोरिति प्र‚क‚र‚णात् ।
	\pend% ending standard par
      ‚{\tiny $_{lb}$}‚

	  \pstart \leavevmode% starting standard par
	क‚ल्प‚न‚याऽपि क‚थं भेदो येनोत्प‚त्त्या विशिष्ट‚मुत्प‚त्तिम‚दुच्य‚त इत्याह--\textbf{अनुत्प‚न्नेभ्य}‚{\tiny $_{lb}$}‚ इति । हिर्य‚स्मात् । \textbf{अनुत्प‚न्नेभ्य} आकाशादिभ्यो \textbf{व्यावृत्तिं} व्य‚व‚च्छेद\textbf{माश्रित्य} प‚रिक‚ल्प्य ।‚{\tiny $_{lb}$}‚ य‚दि व्यावृत्त्याश्र‚येणोत्प‚न्नो भाव उच्य‚ते त‚र्हि क‚थ‚मुत्प‚त्तिर‚स्येति प्र‚योग इत्याह--\textbf{सैवेति ।‚{\tiny $_{lb}$}‚ व्यावृत्त्य‚न्त‚रं} म‚ह‚त्त्वादि \textbf{त‚न्निर‚पेक्षा व‚क्तुमिष्य‚ते} य‚दा \textbf{त‚दा । तेन} प‚र‚मार्थः स्व‚भाव‚भूते‚{\tiny $_{lb}$}‚नोत्प‚त्त्याख्येन \textbf{ध‚र्मेण क‚ल्पितः} स‚मारोपितो \textbf{भेदो}ऽर्थान्त‚र‚त्वं य‚स्य ये\edtext{}{\lemma{ये}\Bfootnote{ते}}न अव्य‚तिरिक्तेन‚{\tiny $_{lb}$}‚ विशेष‚णेन विशिष्ट‚स्य स्व‚भाव‚हेतोः प्र‚योग इत्य‚र्थः ।
	\pend% ending standard par
      ‚{\tiny $_{lb}$}‚

	  \pstart \leavevmode% starting standard par
	पूर्व‚म‚व्य‚तिरिक्त‚विशेष‚ण‚विशिष्ट‚स्य स्व‚भाव‚स्य प्र‚योगः । अधुना तु भिन्न‚विशेष‚ण‚{\tiny $_{lb}$}‚विशिष्ट‚स्येति भेद‚स्त‚दाह--\textbf{भिन्नेने}ति । य‚द्वा \textbf{भिन्नेन} पूर्व‚स्माद‚न्यादृशेन स‚ङ्केत‚व‚शाद‚न्त‚र्भा‚{\tiny $_{lb}$}‚वितेन, न तु विद्य‚मान‚स्व‚वाच‚केन । अत एवाय‚म‚न्य‚तो भिद्य‚ते प्र‚योगः ।
	\pend% ending standard par
      ‚{\tiny $_{lb}$}‚

	  \pstart \leavevmode% starting standard par
	\textbf{इहेति} प‚र‚प्र‚तिपाद‚नार्थे श‚ब्द‚प्र‚योगे । \textbf{शुद्धो} निर्विशेष‚णः । अथ किमेकोऽर्थः शुद्धः‚{\tiny $_{lb}$}‚ क‚दाचिद‚व्य‚तिरिक्तोपाधिना विशिष्टः; क‚दाचिद् व्य‚तिरिक्त‚विशेष‚ण‚विशिष्टो दृष्टः शिष्टैः‚{\tiny $_{lb}$}‚ प्र‚युज्य‚मानो येनैव‚मुच्य‚मानं प‚र‚भागं पुष्णातीत्याह--\textbf{देव‚द‚त्त} इति । व‚क्ष्य‚माण‚त‚द्व‚त्श‚ब्दात्‚{\tiny $_{lb}$}‚ य‚द्व‚त्श‚ब्दोऽत्र द्र‚ष्ट‚व्यः । \textbf{चि}त्रा चासौ गौश्चेति गोर‚त‚द्धित‚लुकि \href{http://sarit.indology.info/?cref=Pā.5.4.92}{पाणिनि ५. ४. ६२} इति‚{\tiny $_{lb}$}‚ ट‚च्, टित्वाङ्ङीप् त‚या विशिष्टः । य‚थाक्र‚म‚मेव दृष्टान्त‚दार्ष्टान्तिक‚योज‚ना कार्या ।
	\pend% ending standard par
      ‚{\tiny $_{lb}$}‚

	  \pstart \leavevmode% starting standard par
	\textbf{चित्र‚गु}श‚ब्देन कृत‚क‚श‚ब्द‚स्य साम्यं नास्तीति म‚न्वानः प‚र आह--\textbf{न‚नु} चेति । \textbf{कार‚णा नां‚{\tiny $_{lb}$}‚ व्यापारो} निय‚त‚प्राग्भाव \unclear{स्त}द‚तिरेकिणो व्यापार‚स्याभावात् । न‚नु च कृत‚क‚श‚ब्दे न विशेष‚ण‚{\tiny $_{lb}$}‚वाचिश‚ब्दोऽस्तीति य‚दुक्तं त‚त्त‚द‚व‚द‚स्थ‚मेवेत्याह--\textbf{य‚द्य‚पीति । अन्त‚र्भावितं} प्र‚काशित‚म् । क‚थं‚{\tiny $_{lb}$}‚ ‚{\tiny $_{lb}$}‚ \leavevmode\ledsidenote{\textenglish{159/dm}}‚{\tiny $_{lb}$}‚ 
	  
	न‚नु च चित्र‚गुश‚ब्दे व्य‚तिरिक्त‚स्य विशेष‚ण‚स्य वाच‚क‚श्चित्र‚श‚ब्दो गोश‚ब्द‚श्चास्ति ।‚{\tiny $_{lb}$}‚ कृत‚क‚श‚ब्दे तु निर्विशेष‚ण‚वाचिनः श‚ब्द‚स्य प्र‚योगोस्तीत्याश‚ङ्क्याह-- ‚{\tiny $_{lb}$}‚ 
	  
	अपेक्षित‚प‚र‚व्यापारो हि भावः स्व‚भाव‚निष्प‚त्तौ कृत‚क इति ॥ १२ ॥‚{\tiny $_{lb}$}‚ 
	  
	अपेक्षितेति । प‚रेषां कार‚णानां व्यापारः स्व‚भाव‚स्य निष्प‚त्तौ--निष्प‚त्त्य‚र्थ‚म‚पेक्षितः‚{\tiny $_{lb}$}‚ प‚र‚व्यापारो य‚न स त‚थोक्तः । हीति य‚स्माद‚र्थे । य‚स्माद‚पेक्षित‚प‚र‚व्यापारः कृत‚क उच्य‚ते‚{\tiny $_{lb}$}‚ त‚स्माद् व्य‚तिरिक्तेन विशेष‚णेन विशिष्टः स्व‚भाव उच्य‚ते । य‚द्य‚पि व्य‚तिरिक्तं विशेष‚ण‚प‚दं‚{\tiny $_{lb}$}‚ न\edtext{}{\lemma{न}\Bfootnote{न च प्र‚यु० \cite{dp-msB}}} प्र‚युक्तं त‚थापि कृत‚क‚श‚ब्देनैव व्य‚तिरिक्तं \edtext{}{\lemma{तिरिक्तं}\Bfootnote{विशेष‚ण‚म‚न्त० \cite{dp-msA} \cite{dp-edP} \cite{dp-edH} \cite{dp-edN}}}विशेष‚ण‚प‚द‚म‚न्त‚र्भावित‚म् । अत एव‚{\tiny $_{lb}$}‚ संज्ञाप्र‚कारोऽयं कृत‚क‚श‚ब्दो य‚स्मात् संज्ञायाम‚यं क‚न्प्र‚त्य‚यो विहितः । य‚त्र च विशेष‚ण‚म‚न्त‚र्भा‚{\tiny $_{lb}$}‚व्य‚ते त‚त्र विशेष‚ण‚प‚दं न प्र‚युज्य‚ते । ‚{\tiny $_{lb}$}‚ 
	  
	क्व‚चित्तु\edtext{}{\lemma{चित्तु}\Bfootnote{क्व‚चित् प्र० \cite{dp-msA} \cite{dp-msB} \cite{dp-edP} \cite{dp-edH} \cite{dp-edE} \cite{dp-edN}}} प्र‚तीय‚मानं विशेष‚णं य‚था कृत इत्युक्ते ह‚तुभिरित्य‚त‚त् प्र‚ताय‚ते । त‚त्र\edtext{}{\lemma{त्र}\Bfootnote{त‚त्र हेतु \cite{dp-msB}}}‚{\tiny $_{lb}$}‚ च हेतुश‚ब्दः प्र‚युज्य‚ते, क‚दाचिन्न वा प्र‚युज्य‚ते ॥ ‚{\tiny $_{lb}$}‚ 
	  
	एवं प्र‚त्य‚य‚भेद‚भेदित्वाद‚योऽपि\edtext{}{\lemma{योऽपि}\Bfootnote{०द‚यो द्र० \cite{dp-msB} \cite{dp-edP} \cite{dp-edH} \cite{dp-edE} \cite{dp-edN}}} द्र‚ष्ट‚व्याः ॥ १३ ॥‚{\tiny $_{lb}$}‚ 
	  
	प्र‚युज्य‚मान‚स्व\edtext{}{\lemma{स्व}\Bfootnote{स्व‚श‚ब्दो विशेष‚ण‚श‚ब्दः--\cite{dp-msD-n}}} श‚ब्द‚श्च य‚था प्र‚त्य‚य‚भेद‚भेदिश‚ब्दे\edtext{}{\lemma{ब्दे}\Bfootnote{प्र‚त्य‚य‚भेद‚श‚ब्दे \cite{dp-msB}}} प्र‚त्य‚य‚भेद‚श‚ब्दः\edtext{}{\lemma{ब्दः}\Bfootnote{प्र‚त्य‚य‚भेदः \cite{dp-msA} \cite{dp-msB} \cite{dp-edP} \cite{dp-edH} \cite{dp-edE} \cite{dp-edN}}} । य‚था च\edtext{}{\lemma{च}\Bfootnote{य‚था कृत० \cite{dp-msC}}}‚{\tiny $_{lb}$}‚ कृत‚क‚श‚ब्दो भिन्न‚विशेष‚ण‚स्व‚भावाभिधायी एवं प्र‚त्य‚य‚भेद‚भेदित्व‚मादिर्येषां प्र‚य‚त्नान्त‚रीय‚{\tiny $_{lb}$}‚क‚त्वादीनां तेऽपि स्व‚भाव‚हेतोः प्र‚योगा भिन्न‚विशेष‚ण‚स्व‚भावाभिधायिनो द्र‚ष्ट‚व्याः । ‚{\tiny $_{lb}$}‚ 
	  
	प्र‚त्य‚यानां कार‚णानां भेदो विशेष‚स्तेन प्र‚त्य‚य‚कालाभेदेनं भेत्तुं शीलं य‚स्य‚स प्र‚त्य‚य‚भेद‚भेदी‚{\tiny $_{lb}$}‚ श‚ब्द‚स्त‚स्य भावः प्र‚त्य‚य‚भेद‚भेदित्व‚म् । त‚तः प्र‚त्य‚य‚भेद‚भेदित्वाच्छ‚ब्द‚स्य कृत‚क‚त्वं साध्य‚ते ।‚{\tiny $_{lb}$}‚ प्र‚य‚त्नान‚न्त‚रीय‚क‚त्वाद‚नित्य‚त्व‚म्\edtext{}{\lemma{म्}\Bfootnote{त्वं साध्य‚ते । \cite{dp-msA} \cite{dp-edP} \cite{dp-edH} \cite{dp-edE} \cite{dp-edN}}} । त‚त्र प्र‚त्य‚य‚भेद‚श‚ब्दो व्य‚तिरिक्त‚विशेष‚णाभिधायी‚{\tiny $_{lb}$}‚ प्र‚त्य‚य‚भेद‚भेदिश‚ब्दे प्र‚युक्तः । प्र‚य‚त्नान‚न्त‚रीय‚क‚श‚ब्दे च प्र‚य‚त्न‚श‚ब्दः ।‚{\tiny $_{lb}$}‚ \textbf{पुनः} कृत‚क‚श‚ब्देनान्त‚र्भावित‚मित्याह--\textbf{य‚स्मादिति । संज्ञायां} नाम्नि क‚न्प्र‚त्य‚यो विहित‚स्त‚{\tiny $_{lb}$}‚स्माद‚न्त‚र्भावित‚मिति । \textbf{अत एव} संज्ञाया क‚नो विधानादेवायं \textbf{कृत‚क‚श‚ब्दः संज्ञाप्र‚कारः} संज्ञा‚{\tiny $_{lb}$}‚विशेषः संज्ञाश‚ब्द इति यावात् ।
	\pend% ending standard par
      ‚{\tiny $_{lb}$}‚

	  \pstart \leavevmode% starting standard par
	अन्त‚र्भावेऽपि क‚थं विशेष‚ण‚प‚दाप्र‚योग इत्याह--\textbf{य‚त्रेति । चो} य‚स्माद‚र्थे । अथाव‚सि‚{\tiny $_{lb}$}‚त‚स्याप्य‚स्ति प्र‚योगो य‚था कृत‚क इत्युक्ते हेतुनेति प्र‚तीताव‚पि हेतुश‚ब्द‚प्र‚योग इत्याह--\textbf{क्व‚चि‚{\tiny $_{lb}$}‚दि}ति । \textbf{तुः} पूर्व‚स्माद् वैध‚र्म्यं \edtext{}{\lemma{र्म्यं}\Bfootnote{र्म्ये}} । \textbf{प्र‚तीय‚मानं} स्व‚त उत्पादायोगात् साम‚र्थ्याद‚व‚सीय‚मान‚म् ।‚{\tiny $_{lb}$}‚ ‚{\tiny $_{lb}$}‚ \leavevmode\ledsidenote{\textenglish{160/dm}}‚{\tiny $_{lb}$}‚ 
	  
	त‚देवं त्रिविधः स्व‚भाव‚हेतुप्र‚योगो\edtext{}{\lemma{योगो}\Bfootnote{०हेतुयोगो \cite{dp-msB}}} द‚र्शितः शुद्धोऽव्य‚तिरिक्त‚विशेष‚णो व्य‚तिरिक्त‚{\tiny $_{lb}$}‚विशेष‚ण‚श्च । ‚{\tiny $_{lb}$}‚ 
	  
	\edtext{\textsuperscript{*}}{\lemma{*}\Bfootnote{एत‚द‚र्थ‚म्--\cite{dp-msB} \cite{dp-msD}}}एव‚म‚र्थं चैत‚दाख्यात‚म्--वाच‚क‚भेदान्मा भूत् क‚स्य‚चित्स्व‚भाव‚हेताव‚पि प्र‚युक्ते‚{\tiny $_{lb}$}‚ व्यामोह इति ॥ ‚{\tiny $_{lb}$}‚ 
	  
	\edtext{\textsuperscript{*}}{\lemma{*}\Bfootnote{स‚न्नोत्प० \cite{dp-edE}}}स‚न्नुत्प‚त्तिमान् कृत‚को वा श‚ब्द इति प‚क्ष‚ध‚र्मोप‚द‚र्श‚न‚म्\edtext{}{\lemma{म्}\Bfootnote{अस्मिन् सूत्रे न किंचित्केन‚चिद्व्याख्यात‚म्--सं०}} ॥ १४ ॥‚{\tiny $_{lb}$}‚ 
	  
	अथ किमेते स्व‚भाव‚हेत‚वः सिद्ध‚स‚म्ब‚न्धे स्व‚भावे साध्ये प्र‚योक्त‚व्या आहोस्विद‚सिद्ध‚{\tiny $_{lb}$}‚स‚म्ब‚न्ध इत्याश‚ङ्क्य सिद्ध‚स‚म्ब‚न्धे प्र‚योक्त‚व्या इति द‚र्श‚यितुमाह-- ‚{\tiny $_{lb}$}‚ 
	  
	स‚र्व एते साध‚न‚ध‚र्मा य‚थास्वं प्र‚माणौः सिद्ध‚साध‚न‚ध‚र्म‚मात्रानुव‚न्ध एव‚{\tiny $_{lb}$}‚ साध्य‚ध‚र्मेऽव‚ग‚न्त‚व्याः ॥ १५ ॥‚{\tiny $_{lb}$}‚ 
	  
	स‚र्व एत इति । ग‚म‚क‚त्वात् साध‚नानि, प‚राश्रित‚त्वाच्च ध‚र्माः, साध‚न‚ध‚र्मा एव‚{\tiny $_{lb}$}‚ अय‚म‚स्याश‚यः--न कृत‚क‚श‚ब्देन प्र‚तिपादित‚स्यार्थ‚स्य अन्य‚थानुप‚प‚त्त्या हेतुव्यापारोऽत्र‚{\tiny $_{lb}$}‚ प्र‚तीय‚ते येनात्रापि विशेष‚ण \leavevmode\ledsidenote{\textenglish{60a/ms}} प‚द‚स्य प्र‚योगो व‚क्तुरिच्छातः स्याद् वा न वा किन्तु स्वोत्प‚त्ता‚{\tiny $_{lb}$}‚व‚पेक्षित‚प‚र‚व्यापार‚स्यैवार्थ‚स्येदं नामेति कुतोऽन‚योः साम्य‚मिति ।
	\pend% ending standard par
      ‚{\tiny $_{lb}$}‚

	  \pstart \leavevmode% starting standard par
	एवं ताव‚त्क‚श्चिद् विशेष‚ण‚भूतोऽर्थोऽर्थात्प्र‚तीय‚मानः स्व‚श‚ब्देनोच्य‚ते न वा व‚क्तुरिच्छा‚{\tiny $_{lb}$}‚व‚शादिति प्र‚तिपाद्य क‚श्चित्पुन‚र्विशेष‚ण‚भूतोऽर्थः प्र‚तीय‚मानोऽप्य‚व‚श्यं स्व‚श‚ब्देनोपादेय इति‚{\tiny $_{lb}$}‚ द‚र्श‚यितुमाह--\textbf{प्र‚युज्य‚माने}ति । \textbf{प्र‚युज्य‚मान} उपादीय‚मानः । \textbf{स्व‚श‚ब्दः} स्व‚वाच‚को य‚स्य‚{\tiny $_{lb}$}‚ विशेष‚ण‚रूप‚स्यार्थ‚स्य स त‚थोक्तः । कोऽसावीदृश इत्याह--\textbf{य‚थे}ति । \textbf{प्र‚त्य‚य‚भेदः} प्र‚त्य‚य‚भेद‚{\tiny $_{lb}$}‚ल‚क्ष‚णोऽर्थो विशेष‚णात्माऽव‚श्यं स्व‚वाच‚केन प्र‚त्य‚य‚भेद‚श‚ब्देनाभिधीय‚त इति प्र‚क‚र‚णात् ।
	\pend% ending standard par
      ‚{\tiny $_{lb}$}‚

	  \pstart \leavevmode% starting standard par
	\textbf{भेत्तुं} भिदां ग‚न्तुं । क‚स्मिन् साध्ये साध‚न‚मिद‚मित्याह--\textbf{श‚ब्द‚स्ये}ति । एत‚च्च‚{\tiny $_{lb}$}‚ \textbf{मीमांस‚का}दीनां प्र‚ति द्र‚ष्ट‚व्य‚म् ।
	\pend% ending standard par
      ‚{\tiny $_{lb}$}‚

	  \pstart \leavevmode% starting standard par
	\textbf{त‚देव‚मि}त्यादिनोप‚संहारः । स्व‚भाव‚हेतोः साध‚र्म्य‚व‚त्प्र‚योग‚मात्रे द‚र्श‚यित‚व्ये किम‚नेक‚स्य‚{\tiny $_{lb}$}‚ स्व‚भाव‚हेतोः प्र‚योगो द‚र्शित इत्याश‚ङ्क्य फ‚ल‚म‚स्योप‚द‚र्श‚य‚न्नाह--\textbf{एव‚म‚र्थं चैत‚दि}ति । \textbf{एवं}‚{\tiny $_{lb}$}‚ व‚क्ष्य‚माण‚कोऽर्थः प्र‚योज‚नं य‚स्येति विग्र‚हः कार्यः । \textbf{चो}ऽव‚धार‚णे हेतौ वा । \textbf{एत‚दि}ति त्रैविध्य‚म् ।‚{\tiny $_{lb}$}‚ \textbf{एव‚म‚र्थं चैत}त् \textbf{हेतुजात‚मि}ति क्व‚चित्पाठ‚स्त‚त्र च \textbf{जातं} वृन्दं द्र‚ष्ट‚व्य‚म् । त‚मेवार्थं \textbf{वाच‚के}त्यादिना‚{\tiny $_{lb}$}‚ \textbf{द‚र्श‚य‚ति । वाच‚के व्यामोहो} भ्र‚म‚स्त‚स्मात् । क‚स्य‚चित्प्र‚तिप‚त्तुः स्व‚भाव‚हेताव‚पि प्र‚युक्ते‚{\tiny $_{lb}$}‚ व्यामोहो नायं स्व‚भावः इति विप‚र्य‚य‚ज्ञान‚म् ।
	\pend% ending standard par
      ‚{\tiny $_{lb}$}‚

	  \pstart \leavevmode% starting standard par
	एत‚दुक्तं भ‚व‚ति--य‚दि त्र‚याणाम‚न्य‚त‚म उपादीयेत त‚दा क‚दाचिद‚न्येनान्य‚था प्र‚युक्ते‚{\tiny $_{lb}$}‚ स्व‚भाव‚हेतौ शास्त्रोक्त‚स्व‚भाव‚हेतुवाच‚क\add{भिन्न}त्वात् नायं स्व‚भाव‚वाच‚कः इति वाच‚के‚{\tiny $_{lb}$}‚ ‚{\tiny $_{lb}$}‚ \leavevmode\ledsidenote{\textenglish{161/dm}}‚{\tiny $_{lb}$}‚ 
	  
	साध‚न‚ध‚र्म‚मात्र‚म् । मात्र‚श‚ब्देनाधिक‚स्यापेक्ष‚णीय‚स्य निरासः । त‚स्यानुब‚न्धोऽनुग‚म‚न‚म‚न्व‚यः ।‚{\tiny $_{lb}$}‚ सिद्धः साध‚न‚ध‚र्म‚मात्रानुब‚न्धो य‚स्य स त‚थोक्तः । केन सिद्ध इत्याह--य‚थास्वं प्र‚माणैरिति\edtext{}{\lemma{माणैरिति}\Bfootnote{प्र‚माणैर्य‚स्य साध‚न‚ध‚र्म‚स्य य‚दात्मीयं \cite{dp-msC} \cite{dp-msD} प्र‚माणैर्य‚स्य य‚दात्मीयं--\cite{dp-msB}}} ।‚{\tiny $_{lb}$}‚ य‚स्य साध्य‚ध‚र्म‚स्य य‚दात्मीयं प्र‚माणं तेनैव प्र‚माणेन सिद्ध इत्य‚र्थः । स्व‚भाव‚हेतूनां च ब‚हुभे‚{\tiny $_{lb}$}‚द‚त्वात् संब‚न्ध‚साध‚नान्य‚पि प्र‚माणानि ब‚हूनीति प्र‚माणैरिति ब‚हुव‚च‚न‚निर्देशः । ग‚म‚यित‚{\tiny $_{lb}$}‚व्य‚त्वात् साध्यः, प‚राश्रित‚त्वाच्च ध‚र्मः साध्य‚ध‚र्मः । ‚{\tiny $_{lb}$}‚ 
	  
	त‚द‚यं प‚र‚मार्थः--न हेतुः प्र‚दीप‚व‚द् योग्य‚त‚या ग‚म‚कोऽपि तु नान्त‚रीय‚क‚त‚या विनि‚{\tiny $_{lb}$}‚श्चितः\edtext{}{\lemma{श्चितः}\Bfootnote{अथ नान्त‚रीय‚त्वानिश्च‚येऽपि लिङ्ग‚स्य प‚रोक्षार्थ‚प्र‚तिपाद‚क‚त्वं स्यादित्याह--\cite{dp-msD-n}}} । साध्याविनाभावित्व‚निश्च‚य‚न‚मेव\edtext{}{\lemma{मेव}\Bfootnote{०मेव हेतोः \cite{dp-msB}}} हि हेतोः साध्य‚प्र‚तिपाद‚न‚व्यापारो नान्यः‚{\tiny $_{lb}$}‚ क‚श्चित् ।‚{\tiny $_{lb}$}‚ व्यामुह्य नायं स्व‚भावः इति साध‚नेऽपि व्यामुह्येत । स व्यामोहो मा भूदित्येत‚द‚र्थं त्रिविधः‚{\tiny $_{lb}$}‚ स्व‚भाव‚हेतुरुक्तः ।
	\pend% ending standard par
      ‚{\tiny $_{lb}$}‚

	  \pstart \leavevmode% starting standard par
	अथेति स‚म्बोध‚ने । \textbf{सिद्धो} निश्चितः \textbf{स‚म्ब‚न्ध}स्तादात्म्य‚ल‚क्ष‚णो य‚स्य त‚स्मिन् । \textbf{य‚स्य‚{\tiny $_{lb}$}‚ साध्य‚ध‚र्म‚स्य य‚दात्मीय‚मिति} येन प्र‚माणेन य‚स्य साध्य‚स्य साध‚न‚व्याप\add{क}त्वं निश्चीय‚ते ।‚{\tiny $_{lb}$}‚ त‚देव त‚स्येत्य‚भिप्रायेणोक्त‚म् । व्य‚क्तिभेद‚विव‚क्ष‚या \textbf{ब‚हूनी}त्युक्त‚म् ।
	\pend% ending standard par
      ‚{\tiny $_{lb}$}‚

	  \pstart \leavevmode% starting standard par
	साध्य‚साध‚न‚योः स‚म्ब‚न्धो वास्त‚वोऽस्तु । किं त‚न्निश्च‚येनेत्याह--\textbf{त‚द‚य‚मिति} । य‚त‚{\tiny $_{lb}$}‚ एवं त‚त्त‚स्मा\textbf{द‚य‚म}भिधास्य‚मान‚स्तात्प‚र्यार्थः \textbf{स‚र्व} इत्यादेर्वाक्य‚स्य । \textbf{प्र‚दीपो} वैध‚र्म्य‚दृष्टान्तः ।‚{\tiny $_{lb}$}‚ \textbf{योग्य‚त‚या} त‚थाश‚क्य‚त‚या ।
	\pend% ending standard par
      ‚{\tiny $_{lb}$}‚

	  \pstart \leavevmode% starting standard par
	न‚नु प‚रोक्षार्थ‚प्र‚तिपाद‚न‚व्यापार‚स‚मावेशाद् हेतुर्ग‚म‚कः त‚त्किं साध्य‚नान्त‚रीय‚क‚त्व‚निश्च‚येने‚{\tiny $_{lb}$}‚त्याह--साध्येति । हिर्य‚स्माद‚र्थे । \textbf{साध्याविनाभावित्व‚निश्च‚य‚न‚म}त्रापीद‚मेत‚त्स्व‚भाव‚मिति‚{\tiny $_{lb}$}‚ निश्च‚यः । त‚देव \textbf{हेतोः प्र‚तिपाद‚न‚व्यापारः} प‚रोक्षार्थ‚प्र‚तिपाद‚न‚ल‚क्ष‚णो व्यापारः प‚रोक्षार्थ‚प्र‚तिपा‚{\tiny $_{lb}$}‚द‚क‚त्व‚मित्य‚र्थः । य‚द्येव‚म‚निश्चित‚स‚म्ब‚न्धेऽपि साध्य‚ध‚र्मे साध‚न‚ध‚र्म‚स्त‚न्नान्त‚रीय‚क‚त‚या निश्चेतुं‚{\tiny $_{lb}$}‚ श‚क्य इति किं सिद्ध‚स‚म्ब‚न्धेन साध्येनानुसृतेनेत्याह--\textbf{प्र‚थ‚म‚मि}ति । \textbf{प्र‚थ‚मं} हेतुप्र‚योगात्प्राक्,‚{\tiny $_{lb}$}‚ \textbf{बाध‚केन} साध्य‚विप‚र्य‚यो हेतो\leavevmode\ledsidenote{\textenglish{60b/ms}}\edtext{}{\lemma{हेतो}\Bfootnote{नैत‚त्प‚त्रं प्र‚तिबिम्बित‚म्--सं०}} \add{... ... ...‚{\tiny $_{lb}$}‚ ... ... ...‚{\tiny $_{lb}$}‚ ... ... ...‚{\tiny $_{lb}$}‚ ... ... ...}‚{\tiny $_{lb}$}‚ ...\leavevmode\ledsidenote{\textenglish{61a/ms}} साध्य‚साध‚न‚भाव इति चेत् । य‚द्द‚र्श‚न‚द्वारायातावेतौ कृत‚क‚त्वानित्य‚त्व‚विक‚ल्पौ‚{\tiny $_{lb}$}‚ व्यावृत्तिनिष्ठौ प‚र‚मार्थ‚त‚स्त‚स्य तादात्म्यादित्युक्त‚प्राय‚मित्य‚दोषः । तादात्म्याव‚सायः कुत इति चेद्‚{\tiny $_{lb}$}‚ विप‚र्य‚ये वाध‚क‚प्र‚माण‚व‚शात् । त‚त एव त‚र्हि साध्यं सिद्ध‚मिति किं साध्य‚त इति चेत् । न ।‚{\tiny $_{lb}$}‚ त‚तो ध‚र्म्य‚न‚व‚च्छेदेन प्ल‚व‚मानाकारायाः प्र‚तीतेर‚प्र‚वृत्त्य‚ङ्ग‚स्योद‚यात् । य‚त्पुन‚रियं घ‚र्म‚काल‚विशे‚{\tiny $_{lb}$}‚‚{\tiny $_{lb}$}‚ ‚{\tiny $_{lb}$}‚ \leavevmode\ledsidenote{\textenglish{162/dm}}‚{\tiny $_{lb}$}‚ 
	  
	\edtext{\textsuperscript{*}}{\lemma{*}\Bfootnote{प्राग‚नुमानात्--\cite{dp-msD-n}}}प्र‚थ‚मं बाध‚केन प्र‚माणेन साध्य‚प्र‚तिब‚न्धो निश्चेत‚व्यो हेतोः । पुन‚र‚नुमान‚काले\edtext{}{\lemma{काले}\Bfootnote{०कालेन साध‚नं--\cite{dp-msA} \cite{dp-msB} \cite{dp-edP} \cite{dp-edH} \cite{dp-edE}}}‚{\tiny $_{lb}$}‚ साध‚नं साध्य\edtext{}{\lemma{साध्य}\Bfootnote{साध्यान‚न्त‚रीय‚कं \cite{dp-msA} \cite{dp-msB} \cite{dp-edP} \cite{dp-edH} \cite{dp-edE}}} नान्त‚रीय‚कं सामान्येन स्म‚र्त‚व्य‚म् । कृत‚क‚त्वं नामानित्य\edtext{}{\lemma{नामानित्य}\Bfootnote{०त्य‚त्व‚स्व० \cite{dp-msA} \cite{dp-msB} \cite{dp-edP} \cite{dp-edH} \cite{dp-edE} \cite{dp-edN}}} स्व‚भाव‚मिति सामा‚{\tiny $_{lb}$}‚न्येन स्मृत‚म‚र्थ\edtext{}{\lemma{र्थ}\Bfootnote{स्मृत‚म‚र्थाय--\cite{dp-msB}}} पुन‚र्विशेषे योज‚य‚ति--इद‚म‚पि कृत‚क‚त्वं श‚ब्दे व‚र्त‚मान‚म‚नित्य\edtext{}{\lemma{नित्य}\Bfootnote{०त्य‚त्व‚स्व० \cite{dp-msC}}} स्व‚भाव‚मेवेति । ‚{\tiny $_{lb}$}‚ 
	  
	त‚त्र सामान्य‚स्म‚र‚णं लिङ्ग‚ज्ञान‚म्\edtext{}{\lemma{म्}\Bfootnote{लिङ्ग‚निश्चाय‚कं ज्ञान‚म्--\cite{dp-msD-n}}} । विशिष्ट‚स्य तु श‚ब्द‚ग‚त‚कृत‚क‚त्व‚स्या\edtext{}{\lemma{स्या}\Bfootnote{कृत‚क‚स्या० \cite{dp-edH}}} ऽनित्य‚{\tiny $_{lb}$}‚त्व‚स्व‚भाव‚स्य स्म‚र‚ण‚म‚नुमान‚ज्ञान‚म्\edtext{}{\lemma{म्}\Bfootnote{मानं ज्ञानं--\cite{dp-msC}}} । त‚था च स‚त्य‚विनाभावित्व‚ज्ञान‚मेव प‚रोक्षार्थ‚{\tiny $_{lb}$}‚प्र‚तिपाद‚क‚त्वं नाम । तेन निश्चित‚त‚न्मात्रानुब‚न्धे साध्य‚ध‚र्मे स्व‚भाव‚हेत‚वः प्र‚योक्त‚व्या‚{\tiny $_{lb}$}‚ नान्य‚त्रेत्युक्त‚म् ॥ ‚{\tiny $_{lb}$}‚ 
	  
	य‚द्येवं स‚म्ब‚न्धो निश्चेत‚व्यः साध्य‚स्य साध‚नेन स‚ह । साध‚न‚ध‚र्म‚मात्रानुब‚न्ध‚स्तु‚{\tiny $_{lb}$}‚ साध्य‚स्य क‚स्मान्निश्चितो मृग्य‚त इत्याह-- ‚{\tiny $_{lb}$}‚ 
	  
	\edtext{\textsuperscript{*}}{\lemma{*}\Bfootnote{त‚स्यैव--इति नास्ति--\cite{dp-edP} \cite{dp-edH}}}त‚स्यैव त‚त्स्व‚भाव‚त्वात् ॥ १६ ॥‚{\tiny $_{lb}$}‚ 
	  
	त‚स्यैवेति सिद्ध‚साध‚न‚ध‚र्म‚मात्रानुब‚न्ध‚स्य । त‚त्स्व‚भाव‚त्वादिति साध‚न‚ध‚र्म‚र‚व‚भाव‚त्वात् ।‚{\tiny $_{lb}$}‚ यो हि साध्य‚ध‚र्मः साध‚न‚ध‚र्म‚मात्रानुब‚न्ध‚वात् स एव त‚स्य साध‚न‚ध‚र्म‚स्य स्व‚भावो नान्यः ॥ ‚{\tiny $_{lb}$}‚ 
	  
	भ‚व‚तु ईदृश एव स्व‚भावः । स्व‚भाव एव तु साध्ये क‚स्माद्धेतुप्र‚योगः ? ‚{\tiny $_{lb}$}‚ 
	  
	स्व‚भाव‚स्य च\edtext{}{\lemma{च}\Bfootnote{०स्य हेतु० \cite{dp-msC}}} हेतुत्वात् ॥ १७ ॥‚{\tiny $_{lb}$}‚ 
	  
	स्व‚भाव‚स्य च\edtext{}{\lemma{च}\Bfootnote{०स्य हेतु० \cite{dp-msC}}} हेतुत्वात् । स्व‚भाव एव\edtext{}{\lemma{एव}\Bfootnote{स्व‚भाव एव हेतु--\cite{dp-msB} स्व‚भाव इह \cite{dp-msA} \cite{dp-edP} \cite{dp-edH} \cite{dp-edN}}} इह हेतुः प्र‚क्रान्तः । त‚स्मात् स एव‚{\tiny $_{lb}$}‚ साध्यः क‚र्त‚व्यः यः साध‚न‚स्य स्व‚भावः स्यात् । साध‚न‚ध‚र्म‚मात्रानुब‚न्ध‚वांश्च\edtext{}{\lemma{वांश्च}\Bfootnote{०न्ध‚श्च स्व० \cite{dp-msA} \cite{dp-edP} \cite{dp-edE} \cite{dp-edH} \cite{dp-edN}}} स्व‚भावो नान्यः ॥ ‚{\tiny $_{lb}$}‚ 
	  
	य‚दि साध्य‚ध‚र्मः साध‚न‚स्य स्व‚भावः\edtext{}{\lemma{भावः}\Bfootnote{स्व‚भावः प्र‚ति० \cite{dp-msA} \cite{dp-msB} \cite{dp-msD} \cite{dp-edP} \cite{dp-edE} \cite{dp-edH} \cite{dp-edN}}} स्यात् प्र‚तिज्ञार्थैक‚देश‚स्त‚र्हि हेतुः स्यादित्याह-- ‚{\tiny $_{lb}$}‚ 
	  
	व‚स्तुत‚स्त‚योस्तादात्म्य‚म्\edtext{}{\lemma{म्}\Bfootnote{०त्म्यात् \cite{dp-msB} \cite{dp-msD} \cite{dp-edP} \cite{dp-edH} \cite{dp-edE} \cite{dp-edN}}} ॥ १८ ॥‚{\tiny $_{lb}$}‚ 
	  
	व‚स्तुत इति । व‚स्तुतः प‚र‚मार्थ‚तः साध्य‚साध‚न‚योस्तादात्म्य‚म् । स‚मारोपित‚स्तु‚{\tiny $_{lb}$}‚ साध्य‚साध‚न‚भेदः\edtext{}{\lemma{भेदः}\Bfootnote{०साध‚न‚योर्भेदः \cite{dp-msA} \cite{dp-edP} \cite{dp-edH} \cite{dp-edE} \cite{dp-edN}}} । साध्य‚साध‚न‚भावो हि निश्च‚यारूढे रूपे । निश्च‚यारूढं च रूपं स‚मारोपितेन‚{\tiny $_{lb}$}‚ षान‚व‚च्छेदेन त‚दात्म‚ताप्र‚तीतिः प्र‚वृत्त्य‚ङ्ग‚मिय‚म‚स्मादेव लिङ्गादिति किम‚व‚द्य‚म् ? एवं स‚त्त्व‚हेता‚{\tiny $_{lb}$}‚व‚पि द्र‚ष्ट‚व्य‚म् । एत‚च्चोक्त‚म‚पि स्वार्थानुमानेऽधिकाभिधानार्थं पुन‚रुक्त‚मिहेति द्र‚ष्ट‚व्य‚म् ॥
	\pend% ending standard par
      ‚{\tiny $_{lb}$}‚‚{\tiny $_{lb}$}‚\textsuperscript{\textenglish{163/dm}}‚{\tiny $_{lb}$}‚
	  \bigskip
	  \begingroup
	

	  \pstart \leavevmode% starting standard par
	भेदेनेत‚रेत‚र‚व्यावृत्तिकृतेन भिन्न‚मिति अन्य‚त् साध‚न‚म्, अन्य‚त् साध्य‚म् । दूराद्धि शाखादिमान‚र्थो‚{\tiny $_{lb}$}‚ बृक्ष इति निश्चीय‚ते न शिंश‚पेति । अथ च स एव वृक्षः सैव शिंश‚पा । त‚स्माद‚भिन्न‚म‚पि व‚स्तु‚{\tiny $_{lb}$}‚ निश्च‚यो भिन्न‚माद‚र्श‚य‚ति व्यावृत्तिभेदेन । त‚स्मान्निश्च‚यारूढ‚रूपापेक्ष‚या अन्य‚त् साध‚न‚म अन्य‚त्‚{\tiny $_{lb}$}‚ साध्य‚म् । अतो न प्र‚तिज्ञार्थैक‚देशो हेतुः । वास्त‚वं च तादात्म्य‚मिति ॥
	\pend% ending standard par
       ‚{\tiny $_{lb}$}‚ 

	  \pstart \leavevmode% starting standard par
	क‚स्मात् पुनः साध‚न‚ध‚र्म‚मात्रानुब‚न्ध्येव\edtext{}{\lemma{न्ध्येव}\Bfootnote{०ब‚न्धे च साध्यः--\cite{dp-msB}}} साध्यः स्व‚भावो नान्य इत्याह--
	\pend% ending standard par
       ‚{\tiny $_{lb}$}‚ 
	  \bigskip
	  \begingroup
	

	  \pstart \leavevmode% starting standard par
	त‚न्निष्प‚त्ताव‚निष्प‚न्न‚स्य त‚त्स्व‚भाव‚त्वाभावात्\edtext{}{\lemma{त्वाभावात्}\Bfootnote{त‚त्स्व‚भावात् \cite{dp-msC}}} ॥ १९ ॥
	\pend% ending standard par
      
	  \endgroup
	‚{\tiny $_{lb}$}‚ 

	  \pstart \leavevmode% starting standard par
	त‚न्निष्प‚त्ताविति । यो हि य‚न्नानुब‚ध्नाति स\edtext{}{\lemma{स}\Bfootnote{स त‚न्निष्प‚त्ताव‚निष्प‚न्न‚स्य साध‚न \cite{dp-msA}}} त‚न्निष्प‚त्ताव‚निष्प‚न्नः । त‚स्य त‚न्निष्प‚त्ता‚{\tiny $_{lb}$}‚व‚निष्प‚न्न‚स्य साध‚न‚स्व‚भाव‚त्व‚म‚युक्त‚म् । य‚तो निष्प‚त्त्य‚निष्प‚त्ती भावाभाव‚रूपे । भावाभावौ‚{\tiny $_{lb}$}‚ च प‚र‚स्प‚र‚प‚रिहारेण स्थितौ । य‚दि च पूर्व‚निष्प‚न्न‚स्य, अनिष्प‚न्न‚स्य चैक्यं भ‚वेदेक‚स्यैवार्थ‚स्य‚{\tiny $_{lb}$}‚ भावाभावौ स्यातां युग‚प‚त् । न च विरुद्ध‚योर्भावाभाव‚योरैक्यं युज्य‚ते, विरुद्ध‚ध‚र्म‚संस‚र्गात्म‚क‚{\tiny $_{lb}$}‚त्वादेक‚त्वाभाव‚स्य ।\edtext{\textsuperscript{*}}{\lemma{*}\Bfootnote{अय‚मेव भेदोभाव‚नां विरुद्ध‚ध‚र्माध्यासः, कार‚ण‚भेद‚श्च--\cite{dp-msD-n}}}
	\pend% ending standard par
       ‚{\tiny $_{lb}$}‚ 

	  \pstart \leavevmode% starting standard par
	किञ्च प‚श्चादुत्प‚द्य‚मानं पूर्व‚निष्प‚न्नाद्भिन्न‚हेतुक‚म् । हेतुभेद‚पूर्व‚क‚श्च कार्य‚भेदः । त‚तो‚{\tiny $_{lb}$}‚ निष्प‚न्नानिष्प‚न्न‚योर्विरुद्ध‚ध‚र्म‚संस‚र्गात्म‚को भेदो भेद‚हेतुश्च कार‚ण‚भेद इति कुत एक‚त्व‚म् ?‚{\tiny $_{lb}$}‚ त‚स्मात् साध‚न‚ध‚र्म‚मात्रानुब‚न्ध्येव साध्यः स्व‚भावो नान्यः ॥
	\pend% ending standard par
       ‚{\tiny $_{lb}$}‚ 

	  \pstart \leavevmode% starting standard par
	मा भूत् प‚श्चान्निष्प‚न्नः पूर्व‚ज‚स्य स्व‚भावः । साध्य‚स्तु क‚स्मान्न भ‚व‚तीत्याह--
	\pend% ending standard par
       ‚{\tiny $_{lb}$}‚ 
	  \bigskip
	  \begingroup
	

	  \pstart \leavevmode% starting standard par
	व्य‚भिचार‚संभ‚वाच्च ॥ २० ॥
	\pend% ending standard par
      
	  \endgroup
	‚{\tiny $_{lb}$}‚ 

	  \pstart \leavevmode% starting standard par
	\edtext{\textsuperscript{*}}{\lemma{*}\Bfootnote{व्य‚भिचारेत्यादि--नास्ति--\cite{dp-msA} \cite{dp-msB} \cite{dp-msD} \cite{dp-edP} \cite{dp-edH} \cite{dp-edE} \cite{dp-edN}}}व्य‚भिचारेत्यादि । पूर्व‚जेन प‚श्चान्निष्प‚न्न‚स्य व्य‚भिचारः प‚रित्यागो य‚स्त‚स्य संभ‚वाच्च
	\pend% ending standard par
      
	  \endgroup
	‚{\tiny $_{lb}$}‚

	  \pstart \leavevmode% starting standard par
	न‚नु किम‚स्य स‚म्भ‚वोऽस्ति य‚दुतैकात्म‚न एवैका व्यावृत्तिर्निश्चीय‚ते नेत‚रेत्याह--दूरादिति ।‚{\tiny $_{lb}$}‚ हिर्य‚स्मात् । \textbf{अथ चे}ति निपात‚स‚मुदायः प्र‚तिपिपाद‚यिषित‚प‚र‚मार्थ‚द्योत‚कः । \textbf{वृक्षः शिंश‚पे}ति‚{\tiny $_{lb}$}‚ चोप‚ल‚क्ष‚ण‚मेत‚त् ।
	\pend% ending standard par
      ‚{\tiny $_{lb}$}‚

	  \pstart \leavevmode% starting standard par
	\textbf{त‚स्मादि}त्यादिनोप‚संहारः । \textbf{वास्त‚वं} व‚स्तुस्व‚रूपादाग‚त‚म‚स‚मारोपित‚मित्य‚र्थः । तुश‚ब्दार्थ‚{\tiny $_{lb}$}‚श्च‚कारः ॥
	\pend% ending standard par
      ‚{\tiny $_{lb}$}‚

	  \pstart \leavevmode% starting standard par
	\textbf{यः} साध्य‚ध‚र्मो \textbf{यं} साध‚न‚ध‚र्मं \textbf{नानुब‚ध्नाति,} नानुग‚च्छ‚ति, त‚स्मिन् स‚ति निय‚मेन नोप‚ति ठ‚त‚{\tiny $_{lb}$}‚ इति याव‚त् । क‚स्मात् त‚त्स्व‚भाव‚त्व‚म‚युक्त‚मित्याह--\textbf{य‚त} इति । भ‚व‚तां भावाभाव‚रूपे त‚थाऽपि‚{\tiny $_{lb}$}‚ किं त‚योस्त‚त्स्व‚भाव‚त्व‚मित्याह--\textbf{भावे}ति । चो य‚स्माद‚र्थे । \textbf{य‚दिच्छ}\edtext{\textsuperscript{*}}{\lemma{*}\Bfootnote{य‚दि चे}}ति च‚श‚ब्दो‚{\tiny $_{lb}$}‚ व‚क्त‚व्यान्त‚र‚स‚मुच्च‚ये ।
	\pend% ending standard par
      ‚{\tiny $_{lb}$}‚‚{\tiny $_{lb}$}‚\textsuperscript{\textenglish{164/dm}}‚{\tiny $_{lb}$}‚
	  \bigskip
	  \begingroup
	

	  \pstart \leavevmode% starting standard par
	न पूर्व‚निष्प‚न्न‚स्य प‚श्चान्निष्प‚न्नः साध्यः । त‚स्मात् साध‚न‚ध‚र्म‚मात्रानुब‚न्ध्येव\edtext{}{\lemma{न्ध्येव}\Bfootnote{०ब‚न्ध्येव यः स्व‚भावः \cite{dp-msB} \cite{dp-msC} \cite{dp-msD} \cite{dp-edE}}} स्व‚भावः । स‚{\tiny $_{lb}$}‚ एव च साध्यः । त‚था च सिद्ध‚साध‚न‚ध‚र्म‚मात्रानुब‚न्ध एव स्व‚भावे स्व‚भाव‚हेत‚वः प्र‚योक्त‚व्या‚{\tiny $_{lb}$}‚ इति स्थित‚म् ॥
	\pend% ending standard par
       ‚{\tiny $_{lb}$}‚ 
	  \bigskip
	  \begingroup
	

	  \pstart \leavevmode% starting standard par
	कार्य‚हेतोः\edtext{}{\lemma{हेतोः}\Bfootnote{कार्य‚हेतुप्र‚योगः \cite{dp-msC} कार्य‚हेतोर‚पि प्र‚योगः \cite{dp-msB} \cite{dp-msD} \cite{dp-edP} \cite{dp-edH} \cite{dp-edE} \cite{dp-edN}}} प्र‚योगः--य‚त्र धूम‚स्त‚त्राग्निः । य‚था म‚हान‚सादौ । अस्ति‚{\tiny $_{lb}$}‚ चेह धूम इति ॥ २१ ॥
	\pend% ending standard par
      
	  \endgroup
	‚{\tiny $_{lb}$}‚ 

	  \pstart \leavevmode% starting standard par
	कार्य‚हेतोः प्र‚योगः । साध‚र्म्य‚वानिति प्र‚क‚र‚णाद‚पेक्ष‚णीय‚म् । य‚त्र धूम इति धूम‚म‚नूद्य‚{\tiny $_{lb}$}‚ त‚त्राग्निरित्य‚ग्ने\edtext{}{\lemma{ग्ने}\Bfootnote{त्य‚ग्निविधः--\cite{dp-msB}}} र्विधिः । त‚था च \edtext{}{\lemma{च}\Bfootnote{व्याप्तिर्व्याप‚क‚स्य त‚त्र भाव एवेत्यादिकः--\cite{dp-msD-n}}}निय‚मार्थः पूर्व‚व‚द‚व‚ग‚न्त‚व्यः । त‚द‚नेन कार्य‚कार‚ण‚भाव‚निमित्ता‚{\tiny $_{lb}$}‚ व्याप्तिर्द‚र्शिता ।
	\pend% ending standard par
       ‚{\tiny $_{lb}$}‚ 

	  \pstart \leavevmode% starting standard par
	व्याप्तिसाध‚न‚प्र‚माण‚विष‚यं द‚र्श‚यितुमाह--य‚था म‚हान‚सादाविति । म‚हान‚सादौ हि‚{\tiny $_{lb}$}‚ प्र‚त्य‚क्षानुप‚ल‚म्भाभ्यां कार्य‚कार‚ण‚भावात्माविनाभावो निश्चितः ।
	\pend% ending standard par
       ‚{\tiny $_{lb}$}‚ 

	  \pstart \leavevmode% starting standard par
	अस्ति चेहेति साध्य‚ध‚र्मिणि प‚क्ष‚ध‚र्मोप‚संहारः ॥
	\pend% ending standard par
       ‚{\tiny $_{lb}$}‚ 
	  \bigskip
	  \begingroup
	

	  \pstart \leavevmode% starting standard par
	इहापि सिद्ध एव कार्य‚कार‚ण‚भावे कार‚णे साध्ये \edtext{}{\lemma{साध्ये}\Bfootnote{कार्यं हेतु० \cite{dp-msC} \cite{dp-msD}}}कार्य‚हेतुर्व‚क्त‚व्यः ॥ २२ ॥
	\pend% ending standard par
      
	  \endgroup
	‚{\tiny $_{lb}$}‚ 

	  \pstart \leavevmode% starting standard par
	इहापीति । न केव‚लं स्व‚भाव‚हेताविहापि कार्य‚हेतौ\edtext{}{\lemma{हेतौ}\Bfootnote{कार्य‚हेतोः \cite{dp-msB}}} । सिद्ध एवेति निश्चिते कार्य‚{\tiny $_{lb}$}‚कार‚ण‚त्वे । कार्य‚कार‚ण‚भाव\edtext{}{\lemma{भाव}\Bfootnote{कार्य‚कार‚ण‚त्व‚निश्च‚यो \cite{dp-msA} \cite{dp-edP} \cite{dp-edH} \cite{dp-edN} कार्य‚कार‚ण‚निश्च‚यो \cite{dp-msC}}} निश्च‚यो ह्य‚व‚श्यं क‚र्त‚व्यः । य‚तो न योग्य‚त‚या हेतुर्ग‚म‚कोऽपि तु‚{\tiny $_{lb}$}‚ नान्त‚रीय‚क‚त्वादित्युक्त‚म् ॥
	\pend% ending standard par
      
	  \endgroup
	‚{\tiny $_{lb}$}‚

	  \pstart \leavevmode% starting standard par
	एवं विरुद्ध‚ध‚र्म‚संस‚र्गात्म‚क‚भेदं प्र‚तिपाद्य त‚स्य हेतुं कार‚ण‚भेदं प्र‚तिपाद‚यितुमाह—‚{\tiny $_{lb}$}‚\textbf{किञ्चे}ति निपात‚स‚मुदायो व‚क्त‚व्यान्त‚र‚स‚मुच्च‚ये । पूर्व‚निष्प‚न्न‚व‚स्तुहेतुक‚त्वे त‚दैवोत्प‚त्तिप्र‚स‚ङ्गेन‚{\tiny $_{lb}$}‚ प‚श्चादुत्पादायोगादिति भावो \textbf{भिन्न‚हेतुक‚मिति} ब्रुव‚तः । भिन्न‚हेतुक‚त्वेऽपि क‚थं भेद इत्याह—‚{\tiny $_{lb}$}‚\textbf{हेतुभेदे}ति । \textbf{चो} हेतौ । \textbf{त‚त} इत्यादिनोऽप‚संहारः । \textbf{त‚स्मादि}त्यादिनाऽऽद्य‚स्योप‚संहारः ॥
	\pend% ending standard par
      ‚{\tiny $_{lb}$}‚

	  \pstart \leavevmode% starting standard par
	पूर्व‚स्मिन् काले जातः \textbf{पूर्व‚जः । व्य‚भिचार}स्त‚द‚न्त‚रेणाऽपि केव‚ल‚स्य स्थितिः । \textbf{चो}ऽस्व‚{\tiny $_{lb}$}‚भाव‚तापेक्ष‚याऽसाध्य‚तां स‚मुच्चिनोति । \textbf{त‚स्मादि}त्यादिना साध‚न‚ध‚र्म‚मात्रानुब‚न्धिन‚स्त‚त्स्व‚भाव‚त्वं‚{\tiny $_{lb}$}‚ साध‚न‚स्व‚भाव‚त्व‚ञ्चोप‚संह‚र‚ति । च‚श‚ब्दः \textbf{साध्य} इत्य‚स्यान‚न्त‚रं त‚त्स्व‚भाव‚त्वापेक्ष‚या साध्य‚त्वं‚{\tiny $_{lb}$}‚ स‚मुच्चिनोति । \textbf{त‚था च} त‚न्निष्प‚त्तावेव निष्प‚न्न‚स्य तादात्म्ये त‚न्मात्रानुब‚न्धिन एव च साध्य‚त्वे स‚ति ॥
	\pend% ending standard par
      ‚{\tiny $_{lb}$}‚‚{\tiny $_{lb}$}‚\textsuperscript{\textenglish{165/dm}}‚{\tiny $_{lb}$}‚
	  \bigskip
	  \begingroup
	

	  \pstart \leavevmode% starting standard par
	साध‚र्म्य‚वान्स्व‚भाव‚कार्यानुप‚ल‚म्भानां प्र‚योगो द‚र्शितः । वैध‚र्म्य‚व‚न्तं द‚र्श‚यितुमाह--
	\pend% ending standard par
       ‚{\tiny $_{lb}$}‚ 
	  \bigskip
	  \begingroup
	

	  \pstart \leavevmode% starting standard par
	वैध‚र्म्य‚व‚तः\edtext{}{\lemma{तः}\Bfootnote{वैध‚र्म‚व‚तः \cite{dp-edE}}} प्र‚योगः--य‚त् स‚दुप‚ल‚ब्धिल‚क्ष‚ण‚प्राप्तं त‚दुप‚ल‚भ्य‚त‚{\tiny $_{lb}$}‚ एव । य‚था नीलादिविशेषः । न चैव‚मिहोप‚ल‚ब्धिल‚क्ष‚ण‚प्राप्त‚स्य स‚त‚{\tiny $_{lb}$}‚ उप‚ल‚ब्धिर्घ‚ट‚स्येत्य‚नुप‚ल‚ब्धिप्र‚योगः ॥ २३ ॥
	\pend% ending standard par
      
	  \endgroup
	‚{\tiny $_{lb}$}‚ 

	  \pstart \leavevmode% starting standard par
	वैध‚र्म्य‚व‚त \edtext{}{\lemma{त}\Bfootnote{वैध‚र्म्य‚व‚तः य‚त् स‚दिति \cite{dp-msC}}}इति य‚त् स‚दुप‚ल‚ब्धिल‚क्ष‚ण‚प्राप्त‚मिति य‚त् स‚त् दृश्य‚मित्य‚स्तित्वानुवादः ।‚{\tiny $_{lb}$}‚ त‚दुप‚ल‚भ्य‚त इत्युप‚ल‚म्भ‚विधिः । \edtext{\textsuperscript{*}}{\lemma{*}\Bfootnote{त‚स्मात्--\cite{dp-msD-n}}}त‚द‚नेन दृश्य‚स्य स‚त्त्वं द‚र्श‚न‚विष‚य‚त्वेन व्याप्तं क‚थित‚म्,‚{\tiny $_{lb}$}‚ अस‚त्त्व‚निवृत्तिश्च स‚त्त्व‚म । अनुप‚ल‚म्भ‚निवृत्तिश्च उप‚ल‚म्भः । तेन साध्य‚निवृत्त्य‚नुवादेन‚{\tiny $_{lb}$}‚ साध‚न‚निवृत्तिर्विहिता । त‚था च\edtext{}{\lemma{च}\Bfootnote{स‚त्त्व‚म्--\cite{dp-msD-n}}} साध्य‚निवृत्तिः साध‚न‚निवृत्तौ निय‚त‚त्वात् साध‚न‚निवृत्त्या‚{\tiny $_{lb}$}‚ व्याप्ता क‚थिता । य‚दि च ध‚र्मिणि साध्य‚ध‚र्मो न भ‚वेद् \edtext{}{\lemma{वेद्}\Bfootnote{हेतुर‚पि । हेत्व० \cite{dp-msB} \cite{dp-edP} \cite{dp-edH} हेतुर‚पि न स्यात्--\cite{dp-msD}}}हेतुर‚पि न भ‚वेत् । \edtext{\textsuperscript{*}}{\lemma{*}\Bfootnote{उप‚ल‚म्भः--\cite{dp-msD-n}}}हेत्व‚भावेन‚{\tiny $_{lb}$}‚ \edtext{\textsuperscript{*}}{\lemma{*}\Bfootnote{स‚त्त्व‚स्य--\cite{dp-msD-n}}}साध्याभाव‚स्य व्याप्त‚त्वात् । अस्ति च हेतुः\edtext{}{\lemma{हेतुः}\Bfootnote{अनुप‚ल‚भ्य‚मानः--\cite{dp-msD-n}}} । अतो व्याप‚क‚स्य साध‚नाभा\edtext{}{\lemma{नाभा}\Bfootnote{उप‚ल‚म्भः--\cite{dp-msD-n}}} व‚स्याभावाद्‚{\tiny $_{lb}$}‚ व्याप्य‚स्य\edtext{}{\lemma{स्य}\Bfootnote{स‚त्त्व‚स्य--\cite{dp-msD-n}}} साध्याभाव‚स्याभाव इति साध्य‚ग‚ति\edtext{}{\lemma{ति}\Bfootnote{साध्य‚निश्च‚यो भ‚व‚ति \cite{dp-msA} \cite{dp-msB} \cite{dp-msC} \cite{dp-msD} \cite{dp-edP} \cite{dp-edH} \cite{dp-edE} \cite{dp-edN}}} र्भ‚व‚ति । त‚तो वैध‚र्म्य‚प्र‚योगे साध‚नाभावे‚{\tiny $_{lb}$}‚ साध्याभावो\edtext{}{\lemma{साध्याभावो}\Bfootnote{निय‚मः \cite{dp-msB}}} निय‚तो द‚र्श‚नीयः स‚र्व‚त्रिति न्यायः ॥
	\pend% ending standard par
      
	  \endgroup
	‚{\tiny $_{lb}$}‚

	  \pstart \leavevmode% starting standard par
	\textbf{प्र‚क‚र‚णा}त्साध‚र्म्य‚व‚त्प्र‚योग‚द‚र्श‚न‚प्र‚स्तावात् । \textbf{त‚था चे}ति धूमानुवादेनाग्निविधाने स‚तीत्य‚र्थः ।‚{\tiny $_{lb}$}‚ \textbf{निय‚मो}ऽव्य‚भिचारः, त‚ल्ल‚क्ष‚णोऽर्थः प्र‚तिपाद्य‚त‚याऽभिधेयः प्र‚योज‚नं वाऽस्य--य‚त्र धूम इत्यादेः‚{\tiny $_{lb}$}‚ प्र‚योग‚स्येति प्र‚स्तावात्, सोऽ\textbf{नुग‚न्त‚व्यः} प्र‚त्येत‚व्यः । \textbf{पूर्व‚व‚द‚नु}प‚ल‚ब्ध्यादिव‚त् ।
	\pend% ending standard par
      ‚{\tiny $_{lb}$}‚

	  \pstart \leavevmode% starting standard par
	एत‚देव व्य‚न‚क्ति \textbf{त‚दि}ति य‚त एवं \textbf{त‚त्त}स्मात् । \textbf{व्याप्ति}र‚विनाभावः, साध्य‚निय‚त‚त्वं‚{\tiny $_{lb}$}‚ साध‚न‚स्येति याव‚त् । \textbf{द‚र्शिता} प्र‚द‚र्शिता । किंनिमित्ता सेत्याह--\textbf{कार्ये}ति ।
	\pend% ending standard par
      ‚{\tiny $_{lb}$}‚

	  \pstart \leavevmode% starting standard par
	अय‚माश‚यः--व्याप्तिः ख‚लुः प्र‚तिब‚न्धः साध्याय‚त्त‚त्व‚म् । त‚च्चार्थान्त‚र‚स्यार्थान्त‚रे प्र‚ति‚{\tiny $_{lb}$}‚ब‚द्ध‚त्वं कार्य‚कार‚ण‚भाव‚व‚शादिति स एव \textbf{निमित्तं} त‚स्य, अन‚र्थान्त‚र‚स्य तु तादा\leavevmode\ledsidenote{\textenglish{61b/ms}}त्म्य‚म् ।
	\pend% ending standard par
      ‚{\tiny $_{lb}$}‚

	  \pstart \leavevmode% starting standard par
	\textbf{म‚हान‚सः} सूप‚कार‚शाला । \textbf{आदि}श‚ब्देनाय‚स्कार‚कुट्यादेर्ग्र‚ह‚ण‚म् । किं त‚त्र व्याप्तिसाध‚कं‚{\tiny $_{lb}$}‚ प्र‚माणं य‚द‚पेक्ष‚या त‚स्य विष‚य‚त्व‚मित्याह--\textbf{प्र‚त्य‚क्षे}ति । हिर्य‚स्माद‚र्थे । प्र‚त्य‚क्षानुप‚ल‚म्भानां‚{\tiny $_{lb}$}‚ प्र‚त्येकं जात्येक‚त्व‚विव‚क्ष‚या \textbf{प्र‚त्य‚क्षानुप‚ल‚म्भाभ्यामि}त्युक्त‚म् । प‚र‚मार्थ‚त‚स्तु त्रिभिर‚नुप‚ल‚म्भैर्द्वाभ्यां‚{\tiny $_{lb}$}‚ प्र‚त्य‚क्षाभ्यामित्य‚व‚सेय‚म् । \textbf{कार्य‚कार‚ण‚भावादात्मा} निश्च‚यारूढः स्व‚भावो य‚स्येति विग्र‚हः कार्योऽ‚{\tiny $_{lb}$}‚न्य‚था युक्तिविरोधः स्व‚व‚च‚न‚विरोध‚श्चाऽस्य स्यात् । \textbf{अविनाभावो}ऽव्य‚भिचारः साध्याय‚त्त‚ता‚{\tiny $_{lb}$}‚ साध‚न‚स्येति याव‚त् ।
	\pend% ending standard par
      ‚{\tiny $_{lb}$}‚

	  \pstart \leavevmode% starting standard par
	\textbf{प‚क्ष‚ध‚र्म}स्यो\textbf{प‚संहारो} ढौक‚नं त‚त्र स‚त्त्व‚प्र‚द‚र्श‚न‚मित्य‚र्थः ॥
	\pend% ending standard par
      ‚{\tiny $_{lb}$}‚‚{\tiny $_{lb}$}‚\textsuperscript{\textenglish{166/dm}}‚{\tiny $_{lb}$}‚
	  \bigskip
	  \begingroup
	

	  \pstart \leavevmode% starting standard par
	स्व‚भाव‚हेतोर्वैध‚र्म्य‚प्र‚योग‚माह--
	\pend% ending standard par
       ‚{\tiny $_{lb}$}‚ 
	  \bigskip
	  \begingroup
	

	  \pstart \leavevmode% starting standard par
	अस‚त्य‚नित्य‚त्वे नास्त्येव\edtext{}{\lemma{नास्त्येव}\Bfootnote{नास्ति स‚त्त्व‚म्--\cite{dp-msB} \cite{dp-msD} \cite{dp-edP} \cite{dp-edH} \cite{dp-edE} \cite{dp-edN}}} स‚त्त्व‚मुत्प‚त्तिम‚त्त्वं कृत‚क‚त्वं वा ।\edtext{\textsuperscript{*}}{\lemma{*}\Bfootnote{असंश्च \cite{dp-msB} \cite{dp-edP} \cite{dp-edH} \cite{dp-edN}}}‚{\tiny $_{lb}$}‚ संश्च श‚ब्द उत्प‚त्तिमान् कृत‚को वेति स्व‚भाव‚हेतोः प्र‚योगः ॥ २४ ॥
	\pend% ending standard par
      
	  \endgroup
	‚{\tiny $_{lb}$}‚ 

	  \pstart \leavevmode% starting standard par
	अस‚त्य‚नित्य‚त्व इति । इहानित्य‚त्व‚स्य साध्य‚स्याभावो हेतोर‚भावे निय‚त\edtext{}{\lemma{त}\Bfootnote{निय‚मः \cite{dp-msB}}} उच्य‚ते ।‚{\tiny $_{lb}$}‚ तेन हेत्व‚भावेन\edtext{}{\lemma{भावेन}\Bfootnote{स‚त्त्वेन--\cite{dp-msD-n}}} साध्याभावो व्याप्त\edtext{}{\lemma{व्याप्त}\Bfootnote{नित्य‚त्व‚म्--\cite{dp-msD-n}}} उक्त‚स्त्रिष्व‚पि स्व‚भाव‚हेतुषु । स‚न्नुत्प‚त्तिमान् कृत‚को‚{\tiny $_{lb}$}‚ वा श‚ब्द इति त्र‚याणाम‚पि प‚क्ष‚ध‚र्म‚त्व‚प्र‚द‚र्श‚न‚म् । इह\edtext{}{\lemma{इह}\Bfootnote{श‚ब्दे ध‚र्मिणि--\cite{dp-msD-n}}} च साध‚नाभाव‚स्य व्याप‚क‚स्याभाव‚{\tiny $_{lb}$}‚ उक्तः । त‚तो व्याप्योऽपि साध्याभावो निव‚र्त‚त\edtext{}{\lemma{त}\Bfootnote{निवृत्त इति \cite{dp-msA} \cite{dp-edP} \cite{dp-edH} \cite{dp-edE} \cite{dp-edN}}} इति साध्य‚ग‚तिः ॥
	\pend% ending standard par
       ‚{\tiny $_{lb}$}‚ 

	  \pstart \leavevmode% starting standard par
	कार्य‚हेतोर्वैध\edtext{}{\lemma{हेतोर्वैध}\Bfootnote{वैध‚र्म्य‚प्र‚यो० \cite{dp-msA} \cite{dp-edP} \cite{dp-edH} \cite{dp-edE} \cite{dp-edN}}} र्म्य‚व‚त्प्र‚योग‚माह--
	\pend% ending standard par
       ‚{\tiny $_{lb}$}‚ 
	  \bigskip
	  \begingroup
	

	  \pstart \leavevmode% starting standard par
	अस‚त्य‚ग्नौ न भ‚व‚त्येव धूमः ।‚{\tiny $_{lb}$}‚ अत्र चास्ति धूम\edtext{}{\lemma{धूम}\Bfootnote{अत्र चास्तीति कार्य--\cite{dp-msC} \cite{dp-msD}}} इति कार्य‚हेतोः प्र‚योगः ॥ २५ ॥
	\pend% ending standard par
      
	  \endgroup
	‚{\tiny $_{lb}$}‚ 

	  \pstart \leavevmode% starting standard par
	अस‚त्य‚ग्नाविति । इहापि\edtext{}{\lemma{इहापि}\Bfootnote{इहापि च \cite{dp-msC} \cite{dp-msD}}} व‚ह्न्य‚भावो धूमाभावेन व्याप्त उक्तः । \edtext{\textsuperscript{*}}{\lemma{*}\Bfootnote{अत्र चास्ति धूम इति \cite{dp-msC} \cite{dp-msD}}}अस्ति‚{\tiny $_{lb}$}‚ चात्र धूम इति व्याप‚क‚स्य धूमाभाव‚स्याभाव उक्तः । त‚तो व्याप्य‚स्य व‚ह्न्य‚भाव‚स्याभावे‚{\tiny $_{lb}$}‚ साध्य‚ग‚तिः ॥
	\pend% ending standard par
       ‚{\tiny $_{lb}$}‚ 

	  \pstart \leavevmode% starting standard par
	न‚नु च साध‚र्म्य‚व‚ति\edtext{}{\lemma{ति}\Bfootnote{साध‚र्म्य‚व्य‚तिरेको \cite{dp-msB}}} व्य‚तिरेको नोक्तः । वैध‚र्म्य‚व‚ति चान्व‚यः । त‚त् क‚थ‚मेत‚त्‚{\tiny $_{lb}$}‚ त्रिरूप‚लिङ्गाख्यान‚मित्याह--
	\pend% ending standard par
      
	  \endgroup
	‚{\tiny $_{lb}$}‚

	  \pstart \leavevmode% starting standard par
	दृष्टान्त‚दार्ष्टान्तिक‚योर्हेतुस‚द्भावास‚द्भाव‚द्वार‚कं \textbf{वैध‚र्म्यं} विद्य‚ते प्र‚तिपाद्य‚त‚या य‚स्य‚{\tiny $_{lb}$}‚ तं \textbf{द‚र्श‚यितुमाह वार्तिक‚कारः ।}
	\pend% ending standard par
      ‚{\tiny $_{lb}$}‚

	  \pstart \leavevmode% starting standard par
	\textbf{त‚था चेति} साध्य‚निवृत्त्य‚नुवादेन साध‚न‚निवृत्तिप्र‚कारे स‚तीत्य‚र्थः । \textbf{साध्य‚निवृत्तिर‚भावः‚{\tiny $_{lb}$}‚ साध‚न}स्य \textbf{निवृत्त्या}ऽभावेन \textbf{व्याप्ता} आत्म‚निय‚तीकृता \textbf{क‚थिता} प्र‚काशिता । कुत इत्याह—‚{\tiny $_{lb}$}‚\textbf{साध‚न}स्य \textbf{निवृत्ता}व‚भावे \textbf{निय‚त‚त्वाद}व्य‚भिचारित्वात्साध्य‚निवृत्तेरिति प्र‚क्र‚मात् । य‚तो य‚त्र‚{\tiny $_{lb}$}‚ साध्याभाव‚स्त‚न्निय‚त‚त्व‚म‚स्य । इमामेव व्याप्तिं व्य‚न‚क्ति \textbf{य‚दी}ति । \textbf{चो} हेतौ । \textbf{ध‚र्मिणीत्य‚ने}न‚{\tiny $_{lb}$}‚ ध‚र्मिमात्र‚मुप‚द‚र्श‚य‚न् स‚र्वोप‚संहार‚व‚तीं व्याप्तिमाह । कुतः पुनः साध्याभावे साध‚नाभाव इत्याह—‚{\tiny $_{lb}$}‚\textbf{हेत्व‚भावेनेति । अस्ति हेतु}र्श्दृयानुप‚ल‚म्भः । अनेन व्याप‚क‚स्य साध‚नाभाव‚ल‚क्ष‚ण‚स्याभावो द‚र्शितः ।‚{\tiny $_{lb}$}‚ \textbf{अतो} व्याप‚काभावात्साध्याभावः स‚द्व्य‚व‚हार‚योग्य‚त्व‚ल‚क्ष‚णो व्याप्यो निव‚र्त्त‚ते । य‚त एव‚म्
	\pend% ending standard par
      ‚{\tiny $_{lb}$}‚‚{\tiny $_{lb}$}‚\textsuperscript{\textenglish{167/dm}}‚{\tiny $_{lb}$}‚
	  \bigskip
	  \begingroup
	
	  \bigskip
	  \begingroup
	

	  \pstart \leavevmode% starting standard par
	साध‚र्म्येणापि हि प्र‚योगेऽर्थाद्व‚ध‚र्म्य‚ग‚तिरिति\edtext{}{\lemma{तिरिति}\Bfootnote{ग‚तिः ।--\cite{dp-msC}}} ॥ २६ ॥
	\pend% ending standard par
      
	  \endgroup
	‚{\tiny $_{lb}$}‚ 

	  \pstart \leavevmode% starting standard par
	साध‚र्म्येणेति । साध‚र्म्येणापि अभिधेयेन युक्ते प्र‚योगे क्रिय‚माणे अर्थात्\edtext{}{\lemma{अर्थात्}\Bfootnote{अर्थादिति--\cite{dp-msC}}} साम‚र्थ्यात्‚{\tiny $_{lb}$}‚ वैध‚र्म्य‚स्य व्य‚तिरेक‚स्य ग‚तिर्भ‚व‚तीति\edtext{}{\lemma{तीति}\Bfootnote{भ‚व‚ति । ही \cite{dp-msB} \cite{dp-msC} \cite{dp-msD}}} । हीति य‚स्मात् । त‚स्मात् त्रिरूप‚लिङ्गाख्यान‚मेत‚त्\edtext{}{\lemma{त्}\Bfootnote{मेव त‚त् \cite{dp-msC}}} ।
	\pend% ending standard par
       ‚{\tiny $_{lb}$}‚ 

	  \pstart \leavevmode% starting standard par
	य‚दि नाम व्य‚तिरेकोऽन्व‚य‚व‚ता\edtext{}{\lemma{ता}\Bfootnote{व‚ति नोक्तोऽन्व‚य० \cite{dp-msA} \cite{dp-edP} \cite{dp-edE} \cite{dp-msC}}} नोक्त‚स्त‚थापि अन्व‚य‚व‚च‚न‚साम‚र्थ्यादेवाव‚सीय‚ते ॥‚{\tiny $_{lb}$}‚ क‚थ‚म् ?
	\pend% ending standard par
       ‚{\tiny $_{lb}$}‚ 
	  \bigskip
	  \begingroup
	

	  \pstart \leavevmode% starting standard par
	अस‚ति त‚स्मिन् साध्येन हेतोर‚न्व‚याभावात् ॥ २७ ॥
	\pend% ending standard par
      
	  \endgroup
	‚{\tiny $_{lb}$}‚ 

	  \pstart \leavevmode% starting standard par
	अस‚ति त‚स्मिन् व्य‚तिरेके\edtext{}{\lemma{तिरेके}\Bfootnote{व्य‚तिरेक‚बु० \cite{dp-msA}}} \edtext{\textsuperscript{*}}{\lemma{*}\Bfootnote{बुद्ध्य‚ध्य‚व‚सिते \cite{dp-msA} \cite{dp-msB} \cite{dp-edP} \cite{dp-edH} \cite{dp-edE} \cite{dp-edN}}}बुद्ध्याध्य‚व‚सिते साध्येन हेतोर‚न्व‚य‚स्य \edtext{}{\lemma{स्य}\Bfootnote{बुद्ध्याव‚सित‚स्य \cite{dp-msA} \cite{dp-edP} \cite{dp-edH} \cite{dp-edE} \cite{dp-edN} बुद्ध्य‚व‚सित‚स्य \cite{dp-msB} \cite{dp-msD}}}बुद्ध्याध्य‚व‚सित-\edtext{\textsuperscript{*}}{\lemma{*}\Bfootnote{०सित‚त्वाभावात् \cite{dp-msC}}}‚{\tiny $_{lb}$}‚ स्याभावात् । \edtext{\textsuperscript{*}}{\lemma{*}\Bfootnote{साध्ये निय‚त‚मित्यादिनाऽन्व‚य‚बा[[बो]]ध‚साम‚र्थ्यात् व्य‚तिरेकं द‚र्श‚य‚ति--\cite{dp-msD-n}}}साध्ये निय‚तं साध‚न‚म‚न्व‚य‚वाक्याद‚व‚स्य‚ता साध्याभावे साध‚नं नाश‚ङ्क‚नीय‚म् ।
	\pend% ending standard par
      
	  \endgroup
	‚{\tiny $_{lb}$}‚

	  \pstart \leavevmode% starting standard par
	इतिस्त‚स्मात् । \textbf{साध्य}स्यास‚द्व्य‚व‚हार‚योग्य‚त्व‚स्य \textbf{ग‚ति}र‚व‚सायो भ‚व‚ति, \textbf{स‚र्व‚त्र} हेतुत्र‚य‚वैध‚र्म्य‚{\tiny $_{lb}$}‚प्र‚योगे ॥
	\pend% ending standard par
      ‚{\tiny $_{lb}$}‚

	  \pstart \leavevmode% starting standard par
	स्व‚भाव‚हेतुम‚धिकृत्याह--\textbf{स्व‚भावे}ति । वैध‚र्म्य‚प्र‚तिपाद‚कः प्र‚योग‚स्त‚थोक्तः । क‚थं पुन‚र‚त्र‚{\tiny $_{lb}$}‚ साध्य‚निश्च‚यो जाय‚त इत्याह--इहेति स्व‚भाव‚हेतुप्र‚योग‚त्र‚ये । \textbf{चो} य‚स्मात् \textbf{साध‚नाभाव‚स्य}‚{\tiny $_{lb}$}‚ स‚त्त्वादिनिवृत्तेर\textbf{भावः} स‚त्त्वादिविधि\textbf{रुक्तः । त‚त}स्त‚स्मात् । \textbf{व्याप्योऽपि} क्ष‚णिक‚त्वाभावोऽपि ।‚{\tiny $_{lb}$}‚ \textbf{अपिः} साध‚नाभाव‚निवृत्त्य‚पेक्ष‚या साध्याभाव‚निवृत्तिं स‚मुच्चिनोति । य‚त एव‚मितिस्त‚स्मात् ।‚{\tiny $_{lb}$}‚ \textbf{साध्य‚स्य} क्ष‚णिक‚त्व‚स्य \textbf{ग‚ति}र्निश्च‚य इति ॥
	\pend% ending standard par
      ‚{\tiny $_{lb}$}‚

	  \pstart \leavevmode% starting standard par
	कार्य‚हेतुमुद्दिश्याह--\textbf{कार्येति} । न केव‚लं पूर्व‚योरित्य‚पिश‚ब्दः । \textbf{धूमाभाव‚स्याभावो} धूम‚स‚त्तैव‚{\tiny $_{lb}$}‚ प्र‚तिषेध‚प्र‚तिषेध‚स्य विधिरूप‚त्वादेवं पूर्व‚त्रापि विज्ञेय‚म् । त‚तो व्याप‚क‚स्य घूमाभाव‚स्याभावात्‚{\tiny $_{lb}$}‚ \textbf{व्याप्य‚स्य व‚ह्र्य‚भाव‚स्याभाव}\edtext{\textsuperscript{*}}{\lemma{*}\Bfootnote{वे}} न्याय‚सिद्धे स‚ति ॥
	\pend% ending standard par
      ‚{\tiny $_{lb}$}‚

	  \pstart \leavevmode% starting standard par
	न केव‚लं व्य‚तिरेकेणाभिधेयेन युक्त इ\textbf{त्य‚पिश}ब्दः । \textbf{अभिधेयेन} साक्षाद‚भिधाविश्राम‚{\tiny $_{lb}$}‚विष‚येण । \textbf{साम‚र्थ्याद}न्य‚थाऽनुप‚प‚त्तेः ।
	\pend% ending standard par
      ‚{\tiny $_{lb}$}‚

	  \pstart \leavevmode% starting standard par
	एत‚दिह ज्ञात‚व्य‚म्--अन्व‚य‚व्य‚तिरेक‚योर्भेद‚स्य व्यावृत्तिनिब‚न्ध‚न‚त्वाद् व‚स्तुत‚स्तादात्म्यात्‚{\tiny $_{lb}$}‚ स्व‚भाव‚हेतुजानुमान‚ब‚लादित‚र‚प्र‚तीतिर्न त्व‚न्य‚थाऽनुप‚प‚त्तिल‚क्ष‚णार्थाप‚त्तिर‚नेनोच्य‚त इति ।
	\pend% ending standard par
      ‚{\tiny $_{lb}$}‚‚{\tiny $_{lb}$}‚\textsuperscript{\textenglish{168/dm}}‚{\tiny $_{lb}$}‚
	  \bigskip
	  \begingroup
	

	  \pstart \leavevmode% starting standard par
	इत‚र‚था \edtext{}{\lemma{था}\Bfootnote{साध्ये निय‚त० \cite{dp-msC}}}साध्य‚निय‚त‚मेव न प्र‚तीतं स्यात् । साध्याभावे च साध‚नाभाव‚ग‚तिर्व्य‚तिरेक‚ग‚तिः ।‚{\tiny $_{lb}$}‚ अतः साध्य‚निय‚त‚स्य साध‚न‚स्याभिधान‚साम‚र्थ्याद‚न्व‚य‚वाक्येऽव‚सितो व्य‚तिरेकः ॥
	\pend% ending standard par
       ‚{\tiny $_{lb}$}‚ 
	  \bigskip
	  \begingroup
	

	  \pstart \leavevmode% starting standard par
	त‚था वैध‚र्म्येणाप्य‚न्व‚य‚ग‚तिः ॥ २८ ॥
	\pend% ending standard par
      
	  \endgroup
	‚{\tiny $_{lb}$}‚ 

	  \pstart \leavevmode% starting standard par
	त‚थेति । य‚थाऽन्व‚य‚वाक्ये त‚थाऽर्थादेव वैध‚र्म्येण प्र‚योगेऽन्व‚य‚स्यान‚भिधीय‚मान‚स्यापि ग‚तिः ॥‚{\tiny $_{lb}$}‚ क‚थ‚म् ?
	\pend% ending standard par
       ‚{\tiny $_{lb}$}‚ 
	  \bigskip
	  \begingroup
	

	  \pstart \leavevmode% starting standard par
	अस‚ति त‚स्मिन् साध्याभावे हेत्व‚भाव‚स्यासिद्धेः ॥ २९ ॥
	\pend% ending standard par
      
	  \endgroup
	‚{\tiny $_{lb}$}‚ 

	  \pstart \leavevmode% starting standard par
	अस‚ति त‚स्मिन् अन्व‚ये बुद्धिगृहीते \edtext{}{\lemma{बुद्धिगृहीते}\Bfootnote{गृहीते ते साध्या--\cite{dp-msA} \cite{dp-edP} \cite{dp-edH}}}साध्याभावे हेत्व‚भाव‚स्यासिद्धेर‚न‚व‚सायात् ।‚{\tiny $_{lb}$}‚ हेत्व‚भावे साध्याभावं निय‚तं व्य‚तिरेक‚वाक्याद‚व‚स्य‚ता हेतुसंभ‚वे साध्याभावो नाश‚ङ्क‚नीयः ।‚{\tiny $_{lb}$}‚ इत‚र‚था हेत्व‚भावे\edtext{}{\lemma{भावे}\Bfootnote{साध्याभारो[[वो]]ऽन‚ग्निः--\cite{dp-msD-n}}} निय‚तो\edtext{}{\lemma{तो}\Bfootnote{निय‚तः साध्याभावो न स्यात्--\cite{dp-msA} \cite{dp-msB} \cite{dp-msD} \cite{dp-edP} \cite{dp-edH} \cite{dp-edE} \cite{dp-edN}}} न स्यात् प्र‚तीतः । हेतुस‚त्त्वे च साध्य‚स‚त्त्व\edtext{}{\lemma{त्त्व}\Bfootnote{साध्य‚स‚त्त्वं ग‚तिः--\cite{dp-msA}}} ग‚तिर‚न्व‚य‚ग‚तिः ।‚{\tiny $_{lb}$}‚ अतः साध‚नाभाव\edtext{}{\lemma{नाभाव}\Bfootnote{साध‚नाभावे निय‚त‚स्य--\cite{dp-msC}}} निय‚त‚स्य साध्याभाव‚स्याभिधान‚साम‚र्थ्याद् व्य‚तिरेक‚वाक्येऽन्व‚य‚ग‚तिः ॥
	\pend% ending standard par
      
	  \endgroup
	‚{\tiny $_{lb}$}‚

	  \pstart \leavevmode% starting standard par
	एत‚देव द‚र्श‚यितुमाह--\textbf{य‚दि नामे}ति । विशेषाभिधान‚निमित्ताभ्युप‚ग‚मे चायं निपात‚{\tiny $_{lb}$}‚स‚मुदायः । \textbf{अन्व‚य‚व‚ता}ऽन्व‚येनाभिधेयेन युक्ते प्र‚योगेणेति प्र‚स्तावात् । \textbf{अन्व‚य‚व‚च‚न‚साम‚र्थ्याद‚न्व}‚{\tiny $_{lb}$}‚\leavevmode\ledsidenote{\textenglish{62a/ms}}याभिधान‚ब‚लात् ॥
	\pend% ending standard par
      ‚{\tiny $_{lb}$}‚

	  \pstart \leavevmode% starting standard par
	अभिप्राय‚म‚जानानः प‚र आह--\textbf{क‚थ‚मि}ति ।
	\pend% ending standard par
      ‚{\tiny $_{lb}$}‚

	  \pstart \leavevmode% starting standard par
	\textbf{अस‚ती}त्यादि सिद्धान्त‚वादी । अभिधेय‚त‚या स्थित इत्याश‚ङ्काम‚पाक‚र्त्तुमाह--\textbf{बुद्ध्येति} ।‚{\tiny $_{lb}$}‚ तादात्म्य‚त‚दुत्प‚त्तिनिब‚न्ध‚ने प्र‚तिब‚न्धेन प्र‚तिब‚द्ध‚त्वात्साध‚न‚मिदं साध्याभावे न भ‚व‚त्येवेति बुद्ध‚याऽ‚{\tiny $_{lb}$}‚\textbf{ध्य‚व‚सिते} विष‚यीकृतेऽस‚ति । \textbf{साध्येन हेतोर‚न्व‚य‚स्य}--अन्वीय‚मान‚त्व‚स्य--य‚त्र साध‚नं त‚त्र‚{\tiny $_{lb}$}‚ स‚र्व‚त्राव‚श्यं साध्य‚मित्येवंरूप‚स्य \textbf{बुद्ध्याऽध्य‚व‚सित‚स्ये}त्येत‚द‚न्व‚य‚वाक्योप‚स्थापित‚या बुद्ध्या गृहीत‚{\tiny $_{lb}$}‚स्याभावाद‚भाव‚प्र‚स‚ङ्गादित्य‚र्थः ।
	\pend% ending standard par
      ‚{\tiny $_{lb}$}‚

	  \pstart \leavevmode% starting standard par
	एत‚दुक्तं भ‚व‚ति । य‚दि साध्याभावेऽपि साध‚नं स्यात् त‚दा य‚त्रैवादः साध्याभावेऽपि‚{\tiny $_{lb}$}‚ वृत्त‚मिष्य‚ते त‚त्रैव त‚त्साध‚न‚म‚प्य‚स्ति न च साध्य‚मिति क‚थं य‚त्र य‚त्र साध‚न‚ध‚र्म‚स्त‚त्र त‚त्र साध्य‚ध‚र्म‚{\tiny $_{lb}$}‚ इति स‚र्वोप‚संहारेणान्व‚य उक्तः स्यादिति । त‚स्माद् हेतोर‚न्व‚याभावाद् हेतोः \textbf{साध्ये निय‚तं}‚{\tiny $_{lb}$}‚ साध्याविनाभाविसाध‚न\textbf{म‚न्व‚य‚वाक्वादिवा}\edtext{}{\lemma{न}\Bfootnote{क्याद‚व}}\textbf{स्य‚ता} प्र‚तिय‚ता \textbf{साध्याभावे साध‚नं नाश‚ङ्क‚नीयं}‚{\tiny $_{lb}$}‚ न स‚न्देह‚नीय‚म् । आश‚ङ्कानिषेधेन च विप‚र्य‚योऽत्य‚न्तं निषिद्धः ।
	\pend% ending standard par
      ‚{\tiny $_{lb}$}‚

	  \pstart \leavevmode% starting standard par
	क‚थं पुन‚स्तेनैवं नाश‚ङ्क‚नीय‚मित्याह--\textbf{इत‚र‚थे}ति--अतोऽन्येन प्र‚कारेण । अस्तु साध्याभावे‚{\tiny $_{lb}$}‚ साध‚नाभावाव‚सायो व्य‚तिरेक‚स्तु क‚थं प्र‚तीय‚त इत्याह--\textbf{साध्येति । चो} य‚स्माद‚र्थे । य‚त‚{\tiny $_{lb}$}‚ ‚{\tiny $_{lb}$}‚ \leavevmode\ledsidenote{\textenglish{169/dm}}‚{\tiny $_{lb}$}‚ 
	  
	य‚दि नामाकाशादौ साध्याभावे साध‚नाभाव‚स्त‚थापि किमिति हेतुसंभ‚वे साध्य‚संभ‚व‚{\tiny $_{lb}$}‚ इत्याह-- ‚{\tiny $_{lb}$}‚ 
	  
	न हि स्व‚भाव‚प्र‚तिब‚न्धेऽस‚त्येक‚स्य निवृत्ताव‚प‚र‚स्य निय‚मेन‚{\tiny $_{lb}$}‚ निवृत्तिः ॥ ३० ॥‚{\tiny $_{lb}$}‚ 
	  
	न‚हीति । \edtext{\textsuperscript{*}}{\lemma{*}\Bfootnote{भावः--उत्पादः, स‚त्ता वा--\cite{dp-msD-n}}}स्व‚भावेन प्र‚तिब‚न्धो य‚स्त‚स्मिन्न‚स‚त्येक‚स्य साध्य‚स्य निवृत्त्या नाप‚र‚स्य‚{\tiny $_{lb}$}‚ साध‚न‚स्य निय‚मेन युक्ता निय‚म‚व‚ती निवृत्तिः ॥ ‚{\tiny $_{lb}$}‚ 
	  
	स च द्विप्र‚कारः स‚र्व‚स्य । तादात्म्य‚ल‚क्ष‚ण‚स्त‚दुत्प‚त्तिल‚क्ष‚ण‚श्चे‚{\tiny $_{lb}$}‚त्युक्त‚म् ॥ ३१ ॥‚{\tiny $_{lb}$}‚ एव\textbf{म‚तो} हेतोरित्यादिनोप‚संहारः । अस‚ति व्य‚तिरेके प्र‚तिब‚न्धानाक्षेपाद‚न्व‚य‚स्यैवास‚त्त्वाद‚स‚त‚श्च‚{\tiny $_{lb}$}‚ स‚त्त्वेन त‚त्प्र‚तिपाद‚नायोगात्प्रेक्षाव‚ताम‚न्व‚य‚व‚च‚न‚मेव न प्र‚युक्तं स्यादिति स‚मुदायार्थः ॥
	\pend% ending standard par
      ‚{\tiny $_{lb}$}‚

	  \pstart \leavevmode% starting standard par
	\textbf{वैध‚र्म्येणा}भिधेयेन युक्त इत्य‚ध्याहारः । \textbf{प्र‚योगे} साध‚न‚वाच‚क‚श‚ब्द‚स‚मूहे ॥
	\pend% ending standard par
      ‚{\tiny $_{lb}$}‚

	  \pstart \leavevmode% starting standard par
	\textbf{क‚थ‚मिति} प‚रः ।
	\pend% ending standard par
      ‚{\tiny $_{lb}$}‚

	  \pstart \leavevmode% starting standard par
	\textbf{अस‚ती}त्यादि सिद्धान्त‚वादी । \textbf{बुद्धिगृहीत} इति बुद्ध्य‚न्त‚र‚गृहीत इति ग्राह्य‚म् । एत‚देव‚{\tiny $_{lb}$}‚ प्र‚तिपाद‚य‚न्नाह--\textbf{हेत्व‚भाव} इति । \textbf{इत‚र‚था} हेतुस‚द्भावे साध्याभाव‚स‚म्भ‚व‚प्र‚कारे स‚ति \textbf{निय‚तो न‚{\tiny $_{lb}$}‚ स्यात्प्र‚तीतः} साध्याभाव इति शेषः ।
	\pend% ending standard par
      ‚{\tiny $_{lb}$}‚

	  \pstart \leavevmode% starting standard par
	मा भून्निय‚तोऽनुग‚तः किं न‚श्छिन्न‚मित्याह--\textbf{हेतुस‚त्त्वे} इति । \textbf{चः} स‚मुच्च‚ये ।
	\pend% ending standard par
      ‚{\tiny $_{lb}$}‚

	  \pstart \leavevmode% starting standard par
	अय‚माश‚यः--स‚ति साध‚नेऽव‚श्यं साध्य‚मित्येवंल‚क्ष‚णोऽन्व‚योऽस्त्येव । केव‚लं व्य‚तिरेक‚{\tiny $_{lb}$}‚वाक्यान्न प्र‚तीय‚त इत्युच्य‚ते पूर्व‚प‚क्ष‚वादिना । य‚दा च हेत्व‚भावे न निय‚तः साध्याभावः स‚म्भाव्य‚त‚{\tiny $_{lb}$}‚ इति कुतो य‚त्र य‚त्र साध‚नं त‚त्र त‚त्र साध्य‚मित्येवंरूपोऽन्व‚यः सिद्ध्येत् । त‚त्रैवं स‚म्भाव‚ना‚{\tiny $_{lb}$}‚विष‚ये हेतुभावेऽपि साध्याभावादिति ।
	\pend% ending standard par
      ‚{\tiny $_{lb}$}‚

	  \pstart \leavevmode% starting standard par
	य‚त एव‚म‚तोऽस्माद् हेतोरित्यादिनोप‚संहारः । अत्राप्य‚य‚माश‚यः--य‚दि य‚त्र साध‚नं‚{\tiny $_{lb}$}‚ त‚त्राव‚श्यं साध्य‚मिति न स्यात्त‚दा त‚त्रैव ताव‚द‚स‚त्य‚पि साध्ये साध‚नं वृत्त‚मिति कुतः साध्याभावे‚{\tiny $_{lb}$}‚ साध‚नं न व‚र्त्त‚त इत्येवंरूपो व्य‚तिरेकः सिद्ध्येदिति ॥
	\pend% ending standard par
      ‚{\tiny $_{lb}$}‚

	  \pstart \leavevmode% starting standard par
	अत्राभिप्राय‚म‚प‚रिज्ञाय‚मानः \edtext{}{\lemma{मानः}\Bfootnote{रिजानानः}} प्राह--\textbf{य‚दि नामेति} ।
	\pend% ending standard par
      ‚{\tiny $_{lb}$}‚

	  \pstart \leavevmode% starting standard par
	\textbf{न‚ही}त्यादि प्र‚तिविधान‚माचार्यीयं न‚हीत्यादिना व्याच‚ष्टे । अयं च मौलो हिश‚ब्दः‚{\tiny $_{lb}$}‚ प‚श्चाद् व्याख्यास्य‚ते । \textbf{निय‚मेना}व‚श्यंत‚या । या चाव‚श्यं \textbf{भाविनी निवृत्तिः} सा निय‚मेन युक्ता‚{\tiny $_{lb}$}‚ भ‚व‚तीत्य‚र्थ‚क‚थ‚न‚मेत‚त् ॥
	\pend% ending standard par
      ‚{\tiny $_{lb}$}‚\textsuperscript{\textenglish{170/dm}}‚{\tiny $_{lb}$}‚
	  \bigskip
	  \begingroup
	

	  \pstart \leavevmode% starting standard par
	स च स्व‚भाव‚प्र‚तिब‚न्धो द्विप्र‚कारः स‚र्व‚स्य \edtext{}{\lemma{स्य}\Bfootnote{हेतोः--\cite{dp-msD-n} । प्र‚तिब‚द्ध‚स्य इति नास्ति \cite{dp-msA} \cite{dp-edP} \cite{dp-edH}}}प्र‚तिब‚द्ध‚स्य । तादात्म्यं ल‚क्ष‚णं निमित्तं य‚स्य‚{\tiny $_{lb}$}‚ स त‚थोक्तः । त‚दुत्प‚त्तिर्ल‚क्ष‚णं निमित्तं य‚स्य स त‚थोक्तः । यो य‚त्र प्र‚तिब‚द्ध‚स्त‚स्य स‚{\tiny $_{lb}$}‚ प्र‚तिब‚न्ध‚विष‚योऽर्थः स्व‚भावः कार‚णं वा स्यात् । अन्य‚स्मिन् प्र‚तिब‚द्ध‚त्वानुप‚प‚त्तेः । त‚स्माद्‚{\tiny $_{lb}$}‚ \unclear{द्वि}प्र‚कारः स इत्युक्त‚म् । स च सार्ध्येऽर्थे लिङ्ग‚स्य इत्य‚त्रान्त‚रेऽभिहितः ॥
	\pend% ending standard par
       ‚{\tiny $_{lb}$}‚ 
	  \bigskip
	  \begingroup
	

	  \pstart \leavevmode% starting standard par
	तेनं हि निवृत्तिं क‚थ‚य‚ता प्र‚तिब‚न्धो द‚र्श‚नीयः । त‚स्मात् निवृत्ति‚{\tiny $_{lb}$}‚व‚च‚न‚माक्षिप्त‚प्र‚तिब‚न्धोप‚द‚र्श‚न‚भेव भ‚व‚ति । य‚च्च प्र‚तिब‚न्धोप‚द‚र्श‚नं \edtext{}{\lemma{नं}\Bfootnote{त‚द‚न्व‚य--\cite{dp-msC}}}त‚देवान्व‚{\tiny $_{lb}$}‚य‚व‚च‚न‚मित्येकेनापि वाक्येनान्व‚य‚मुखेन व्य‚तिरेक‚मुखेन वा प्र‚युक्तेन स‚प‚क्षास‚प‚{\tiny $_{lb}$}‚क्ष‚योर्लिङ्ग‚स्य स‚द‚स‚त्त्व‚ख्याप‚नं \edtext{}{\lemma{नं}\Bfootnote{स‚द‚स‚त्त्वाख्याप‚नं--\cite{dp-msC}}}कृतं भ‚व‚तीति नाव‚श्यं वाक्य‚द्व‚य‚प्र‚योगः ॥ ३२ ॥
	\pend% ending standard par
      
	  \endgroup
	‚{\tiny $_{lb}$}‚ 

	  \pstart \leavevmode% starting standard par
	\edtext{\textsuperscript{*}}{\lemma{*}\Bfootnote{पूर्व‚सूत्रोक्तः--\cite{dp-msD-n}}}हिर्य‚स्माद‚र्थे । य‚स्मात् स्व‚भाव‚प्र‚तिब‚न्धे निव‚र्त्य‚निव‚र्त‚क‚भाव‚स्तेन\edtext{}{\lemma{स्तेन}\Bfootnote{तेनेत्युप‚संह‚र‚ति--\cite{dp-msD-n}}} साध्य‚स्य निवृत्तौ‚{\tiny $_{lb}$}‚ साध‚न‚स्य\edtext{}{\lemma{स्य}\Bfootnote{साध‚न‚निवृत्तिं \cite{dp-msC} \cite{dp-msD}}} निवृत्तिं क‚थ‚य‚ता प्र‚तिब‚न्धो निव‚र्त्य‚निव‚र्त‚क‚योर्द‚र्श‚नीयः । य‚दि हि साध‚नं साध्ये‚{\tiny $_{lb}$}‚ प्र‚तिब‚द्धं भ‚वेद् एवं साध्य‚निवृत्तौ \edtext{}{\lemma{निवृत्तौ}\Bfootnote{निवृत्तौ निय‚मेन--\cite{dp-msB} \cite{dp-msC} \cite{dp-msD}}}त‚न्निय‚मेन निव‚र्तेत । य‚त‚श्च त‚स्य प्र‚तिब‚न्धो द‚र्श‚नीयः‚{\tiny $_{lb}$}‚ त‚स्मात् साध्य‚निवृत्तौ य‚त् साध‚न‚निवृत्तिव‚च‚नं\edtext{}{\lemma{नं}\Bfootnote{अन्त‚र्भावित‚म्--\cite{dp-msD-n} । तेनाक्षिप्त प्र‚तिब‚न्घोप‚द‚र्श‚नं त‚देवान्व‚य--\cite{dp-msB}}} तेनाक्षिप्तं प्र‚तिब‚न्धोप‚द‚र्श‚न‚म् । य‚च्च‚{\tiny $_{lb}$}‚ त‚दाक्षिप्तं\edtext{}{\lemma{दाक्षिप्तं}\Bfootnote{क्षिप्त‚प्र‚ति \cite{dp-msA} \cite{dp-edP} \cite{dp-edH} \cite{dp-edE} \cite{dp-edN}}} प्र‚तिब‚न्धोप‚द‚र्श‚नं त‚देवान्व‚य‚व‚च‚न‚म् । प्र‚तिब‚न्ध‚श्चेद‚व‚श्यं द‚र्श‚यित‚व्यो न व‚क्त‚{\tiny $_{lb}$}‚व्य‚स्त‚र्ह्य‚न्व‚यः । य‚स्माद् दृष्टान्ते प्र‚माणेन प्र‚तिब‚न्धो\edtext{}{\lemma{न्धो}\Bfootnote{व‚च‚न‚रूपो य‚त्र य‚त्र धूम‚स्त‚त्राग्निरित्येवं न व‚क्त‚व्य‚स्त‚र्ह्य‚न्व‚यः प्र‚तिब‚न्ध‚शून्यः ।‚{\tiny $_{lb}$}‚ साध्य‚हेत्वोस्तादात्म्य‚त‚दुत्प‚त्तिरूप‚प्र‚तिब‚न्धे स्थिते सिद्ध एवान्व‚य इति भावः--\cite{dp-msD-n} ।}} द‚र्श्य‚मान एवान्व‚यो नाप‚रः क‚श्चित्,
	\pend% ending standard par
      
	  \endgroup
	‚{\tiny $_{lb}$}‚

	  \pstart \leavevmode% starting standard par
	अथ स्व‚भाव‚प्र‚तिब‚न्ध‚श्चेदेक‚निवृत्ताव‚प‚र‚निवृत्तिनिब‚न्ध‚नं त‚दा कार्य‚हेतोरेव व्य‚तिरेको न‚{\tiny $_{lb}$}‚ स्व‚भाव‚हेतो\leavevmode\ledsidenote{\textenglish{62b/ms}}रिति\edtext{}{\lemma{रिति}\Bfootnote{पाठोऽत्र घृष्टः ।}}...\textbf{त‚स्या}सौ न त‚र्हि कार्यंहेतावित्याह--\textbf{स चे}ति । \textbf{चो} य‚स्माद‚र्थे ।‚{\tiny $_{lb}$}‚ \textbf{स्व‚भावेन प्र‚तिब‚न्धः} प्र‚तिब‚द्ध‚त्वं साध्याय‚त्त‚त्व‚म् । क‚स्यासावित्याश‚ङ्कायामाह--\textbf{प्र‚तिब‚द्ध‚स्य}‚{\tiny $_{lb}$}‚ साध‚न‚स्य । स‚र्व‚स्येत्य‚नेन व्याप्तिं द‚र्श‚य‚ति । त‚त्र संयोगादिनिमित्त‚श‚ङ्काव्युदासायाभिम‚तं‚{\tiny $_{lb}$}‚ द्वितं\edtext{}{\lemma{द्वितं}\Bfootnote{द्वैतं}}द‚र्श‚य‚न्नाह--तादात्म्य‚मित्यादि । ल‚क्ष्य‚तेऽनेनेति \textbf{ल‚क्ष‚ण‚म्} । अत एवाह \textbf{निमित्त‚मि}ति‚{\tiny $_{lb}$}‚ किम्पुन‚स्तेन स्व‚भावेन\edtext{}{\lemma{भावेन}\Bfootnote{पाठोऽत्र घृष्टः ।}}...मित्याह--\textbf{अन्य‚स्मिन्न}स्व‚भावेऽकार‚णे च । संयोग‚स‚म‚वाय‚योः‚{\tiny $_{lb}$}‚ प्र‚माण‚बाधित‚त्वेन निमित्त‚त्वानुप‚प‚त्तेरिति भावः । य‚त एवं \textbf{त‚स्माद्} हेतोः । \textbf{स} इति‚{\tiny $_{lb}$}‚ प्र‚तिब‚न्धः ॥
	\pend% ending standard par
      ‚{\tiny $_{lb}$}‚

	  \pstart \leavevmode% starting standard par
	पूर्व‚कं हिश‚ब्द‚मिदानीं य‚थायोगं व्याच‚ष्टे--हिरिति य‚स्माद‚र्थ‚वृत्तिं हिश‚ब्द‚म् । अस्य‚{\tiny $_{lb}$}‚ ‚{\tiny $_{lb}$}‚ \leavevmode\ledsidenote{\textenglish{171/dm}}‚{\tiny $_{lb}$}‚ 
	  
	त‚स्मान्निव‚र्त्य‚निव‚र्त‚क‚योः\edtext{}{\lemma{योः}\Bfootnote{०र्त‚क‚प्र‚तिब--\cite{dp-msA} \cite{dp-msD} \cite{dp-edP} \cite{dp-edH} \cite{dp-edE}}} प्र‚तिब‚न्धो ज्ञात‚व्यः । त‚था चान्व‚य एव ज्ञातो भ‚व‚ति । इति‚{\tiny $_{lb}$}‚श‚ब्दो हेतौ । य‚स्माद‚न्व‚येऽपि\edtext{}{\lemma{येऽपि}\Bfootnote{अन्व‚ये व्य० \cite{dp-msA} \cite{dp-edP} \cite{dp-edH} \cite{dp-edE} \cite{dp-edN}}} व्य‚तिरेक‚ग‚तिः व्य‚तिरेके चान्व‚य‚ग‚तिः, त‚स्माद् एकेनापि स‚प‚क्षे‚{\tiny $_{lb}$}‚ चास‚प‚क्षे च स‚त्त्वास‚त्त्व‚योः ख्याप‚नं कृत‚म् । ‚{\tiny $_{lb}$}‚ 
	  
	अन्व‚यो मुख‚मुपायोऽभिधेय‚त्वाद् य‚स्य त‚द् अन्व‚य‚मुखं वाक्य‚म् । एवं व्य‚तिरेको मुखं‚{\tiny $_{lb}$}‚ \edtext{\textsuperscript{*}}{\lemma{*}\Bfootnote{मुख‚म‚स्य--\cite{dp-msC}}}य‚स्येति । इति\edtext{}{\lemma{इति}\Bfootnote{इतिक‚र‚णो हेतौ \cite{dp-msA}}} हेतौ । य‚स्मादेकेनापि वाक्येन द्व‚य‚ग‚तिस्त‚स्मादेक‚स्मिन् साध‚न‚वाक्ये‚{\tiny $_{lb}$}‚ द्व‚योर‚न्व‚य‚व्य‚तिरेक‚वाक्य‚योर‚व‚श्य‚मेव प्र‚योगो न क‚र्त्त‚व्यः । ‚{\tiny $_{lb}$}‚ 
	  
	अर्थ‚ग‚त्य‚र्थो हि श‚ब्द‚प्र‚योगः । अर्थ‚श्चेद‚व‚ग‚तः, किं श‚ब्द‚प्र‚योगेण ? \edtext{\textsuperscript{*}}{\lemma{*}\Bfootnote{एक‚मेव त्व‚न्व‚य \cite{dp-msA} \cite{dp-edP} \cite{dp-edH} \cite{dp-edE} \cite{dp-edN}}}एक‚मेवान्व‚य‚वाक्यं‚{\tiny $_{lb}$}‚ व्य‚तिरेक‚वाक्यं वा प्र‚योक्त‚व्य‚म् ॥‚{\tiny $_{lb}$}‚ 
	  
	अनुप‚ल‚ब्धाव‚पि--य‚त् स‚द् उप‚र्लाब्ध‚ल‚क्ष‚ण‚प्राप्तं त‚द् उप‚ल‚भ्य‚त‚{\tiny $_{lb}$}‚ एवेत्युक्ते--अनुप‚ल‚भ्य‚मानं तादृश‚म‚स‚दिति प्र‚तीतेर‚न्व‚य‚सिद्धिः ॥ ३३ ॥‚{\tiny $_{lb}$}‚ हिश‚ब्द‚स्य अर्थं कृत्वा श‚ब्द‚प‚दार्थ‚क‚स्या\edtext{}{\lemma{स्या}\Bfootnote{पाठोऽत्र घृष्टः ।}}...एव‚मुक्ते व‚क्ष्य‚माणे च स‚र्व‚त्र द्र‚ष्ट‚व्य‚म् ।‚{\tiny $_{lb}$}‚ अमुमेव य‚स्माद‚र्थ‚म‚पेक्ष्य तेनेति योज‚यितुमिति द‚र्श‚यितुमाह--\textbf{य‚स्मादिति} । एत‚च्च \textbf{न‚हीत्यादि}‚{\tiny $_{lb}$}‚वाक्य‚स्य प्र‚काश्य‚म‚र्थं गृहीत्वोक्तं न त्व‚भिधेय‚म्, निवृत्तिनिषेध‚स्यैव त‚त्राभिधेय‚त्वात् ।
	\pend% ending standard par
      ‚{\tiny $_{lb}$}‚

	  \pstart \leavevmode% starting standard par
	न‚नु प्र‚तिब‚न्धः प्र‚तिब‚द्ध‚त्व‚म् । स च निव‚र्त्त‚मान‚स्यैव न निव‚र्त्त‚क‚स्य । \textbf{य‚दि ही}त्यादिना‚{\tiny $_{lb}$}‚ च\edtext{}{\lemma{च}\Bfootnote{पाठोऽत्र घृष्टः ।}}...द‚र्श‚यिष्य‚ति । त‚त्क‚थ‚मिह निव‚र्त्य‚निव‚र्त्त‚क‚योरित्युक्त‚म् । स‚त्य‚म् । केव‚ल‚म‚त्र‚{\tiny $_{lb}$}‚ प्र‚तिब‚न्ध‚श‚ब्देन प्र‚तिब‚द्ध‚त्वं प्र‚तिब‚न्ध‚विष‚य‚त्वं च विव‚क्षित‚म् ।
	\pend% ending standard par
      ‚{\tiny $_{lb}$}‚

	  \pstart \leavevmode% starting standard par
	तेनाय‚म‚र्थः । प्र‚तिब‚न्ध‚विष‚ये प्र‚तिब‚द्ध‚त्वं द‚र्श‚नीय‚म् । त‚था च न क‚श्चिद् दोषः ।
	\pend% ending standard par
      ‚{\tiny $_{lb}$}‚

	  \pstart \leavevmode% starting standard par
	क‚स्मात्पुनः प्र‚तिब‚न्धो द‚र्श‚नीय इत्याश‚ङ्क्याह--\textbf{य‚दी}ति । हीति य‚स्मात् । \textbf{तेन}‚{\tiny $_{lb}$}‚ व्य‚तिरेक‚व\textbf{च‚नेनाक्षिप्तं} प्र‚काशितं । प्र‚तिब‚न्ध‚स्तादात्म्य‚त‚दुत्प‚त्तिनिब‚न्ध‚नं द‚र्श‚य‚ष्यिते प्र‚काश्य‚तेऽ‚{\tiny $_{lb}$}‚नेनेति त‚था । प्र‚तिब‚न्धोऽव‚श्य‚द‚र्श‚यित‚व्योऽन्य‚था व्य‚तिरेक‚स्यैवासिद्धेरिति भावः । भ‚व‚तु‚{\tiny $_{lb}$}‚ त‚त्त‚था--\textbf{य‚दि}ति । \textbf{चोऽ}व‚धार‚णे । त‚द‚य‚म‚र्थः--य‚देवाक्षिप्त‚प्र‚तिब‚न्धोप‚द‚र्श‚नं \textbf{त‚देवान्व‚य‚व‚च‚न‚{\tiny $_{lb}$}‚म}न्व‚य‚प्र‚काश‚न‚म् ।
	\pend% ending standard par
      ‚{\tiny $_{lb}$}‚

	  \pstart \leavevmode% starting standard par
	न‚नूप‚द‚र्श्य‚तां प्र‚तिब‚न्धो\edtext{}{\lemma{न्धो}\Bfootnote{पाठोऽत्र घृष्टः ।}}...स क‚थं तेनोक्तो भ‚व‚तीत्याह--\textbf{प्र‚तिब‚न्ध} इति । \textbf{य‚द्य‚व‚श्यं‚{\tiny $_{lb}$}‚ द‚र्श‚यित‚व्यो} निय‚मेन ख्याप‚नीय‚स्त\textbf{र्हि न व‚क्त‚व्यः} प्र‚तिपाद‚यित‚व्योऽन्व‚यः ।
	\pend% ending standard par
      ‚{\tiny $_{lb}$}‚

	  \pstart \leavevmode% starting standard par
	न‚नु च दृष्टान्तेन प्र‚तिब‚न्ध‚साध‚न‚केन प्र‚माणेन केव‚लं प्र‚तिब‚न्धः प्र‚द‚र्श्य‚ते, न त्व‚न्व‚यः ।‚{\tiny $_{lb}$}‚ त‚त्क‚थं व‚क्त‚व्य‚स्तेन वाक्येनेत्याह--\textbf{य‚स्मादि}ति । त‚स्य तादात्म्य‚निब‚न्ध‚न‚स्य त‚दुत्प‚त्तिनिब‚न्ध‚न‚स्य‚{\tiny $_{lb}$}‚ वाऽन्व‚यात्म‚क‚त्वादिति भावः ॥
	\pend% ending standard par
      ‚{\tiny $_{lb}$}‚‚{\tiny $_{lb}$}‚\textsuperscript{\textenglish{172/dm}}‚{\tiny $_{lb}$}‚
	  \bigskip
	  \begingroup
	

	  \pstart \leavevmode% starting standard par
	अनुप‚ल‚ब्धाव‚पि व्य‚तिरेकेणो\edtext{}{\lemma{तिरेकेणो}\Bfootnote{केण युक्तेन \cite{dp-msB}}} क्तेनान्व‚य‚ग‚तिः । य‚त् स‚द् उप‚ल‚ब्धिल‚क्ष‚ण‚प्राप्त‚मिति‚{\tiny $_{lb}$}‚ साध्य‚स्य--अस‚द्व्य‚व‚हार‚योग्य‚त्व‚स्य निवृत्तिं दृश्य‚स‚त्त्व\edtext{}{\lemma{त्त्व}\Bfootnote{स‚त्त्व‚स्व‚रूपामाह--\cite{dp-msC}}} रूपामाह । त‚दुप‚ल‚भ्य‚त एवेत्य‚नुप‚ल‚म्भ‚स्य‚{\tiny $_{lb}$}‚ निवृत्तिमुप‚ल‚म्भ‚रूपामाह । त‚द‚नेन साध्य‚निवृत्तिः साध‚न‚निवृत्त्या व्याप्ता द‚र्शिता । य‚दि च‚{\tiny $_{lb}$}‚ साध‚न‚संभ‚वेपि \edtext{}{\lemma{वेपि}\Bfootnote{स‚त्त्व‚म्--\cite{dp-msD-n}}}साध्य‚निवृत्तिर्भ‚वेत् न साध‚नाभावेन\edtext{}{\lemma{नाभावेन}\Bfootnote{उप‚ल‚म्भेन--\cite{dp-msD-n}}} व्याप्ता भ‚वेत् । अतो व्याप्तिं\edtext{}{\lemma{व्याप्तिं}\Bfootnote{व्याप्तिप्र‚ति \cite{dp-msA}}} प्र‚ति‚{\tiny $_{lb}$}‚प‚द्य‚मानेन साध‚न‚संभ‚वः साध्य‚संभ‚वेन व्याप्तः प्र‚तिप‚त्त‚व्यः । अत एवाह--अनुप‚ल‚भ्य‚मानं‚{\tiny $_{lb}$}‚ तादृश‚मिति दृश्य‚म‚स‚दिति प्र‚तीतेः\edtext{}{\lemma{तीतेः}\Bfootnote{प्र‚तीतेः इति नास्ति \cite{dp-msA}}} संप्र‚त्य‚याद् अन्व‚य‚सिद्धिरिति ॥
	\pend% ending standard par
       ‚{\tiny $_{lb}$}‚ 
	  \bigskip
	  \begingroup
	

	  \pstart \leavevmode% starting standard par
	द्व‚योर‚प्य‚न‚योः प्र‚योग‚यो\edtext{}{\lemma{यो}\Bfootnote{प्र‚योगेऽव‚श्यं \cite{dp-msB} \cite{dp-edP} \cite{dp-edH} प्र‚योगे नाव‚श्यं--\cite{dp-edE} \cite{dp-edN}}}र्नाव‚श्यं प‚क्ष‚निर्देशः ॥ ३४ ॥
	\pend% ending standard par
      
	  \endgroup
	‚{\tiny $_{lb}$}‚ 

	  \pstart \leavevmode% starting standard par
	य‚त‚श्च साध‚नं साध्य‚ध‚र्म‚प्र‚तिब‚द्धं तादात्म्य-द‚दुत्प‚त्तिभ्यां प्र‚तिप‚त्त‚व्यं द्व‚योर‚पि प्र‚योग‚योः,‚{\tiny $_{lb}$}‚ त‚स्मात् प‚क्षोऽव‚श्य‚मेव न निर्देश्यः । य‚त् साध‚नं साध्य‚निय‚तं प्र‚तीतं त‚त एव साध्य‚ध‚र्मिणि‚{\tiny $_{lb}$}‚ दृष्टात्\edtext{}{\lemma{दृष्टात्}\Bfootnote{दृष्ट्वा \cite{dp-msA} \cite{dp-edP} \cite{dp-edH}}} साध्य‚प्र‚तीतिः । अतो न किंचित् साध्य‚निर्देशेनेति ॥
	\pend% ending standard par
      
	  \endgroup
	‚{\tiny $_{lb}$}‚

	  \pstart \leavevmode% starting standard par
	\textbf{निवृत्तिव‚च‚नं} चैत‚दुप‚ल‚क्ष‚णं द्र‚ष्ट‚व्य‚म् । तेनान्व‚य‚व‚च‚नेपि स‚र्वं य‚थायोगं द्र‚ष्ट‚व्य‚म् ।
	\pend% ending standard par
      ‚{\tiny $_{lb}$}‚

	  \pstart \leavevmode% starting standard par
	य‚स्मादेव‚म‚नुवाद‚विधिक्र‚म‚स्त‚त् त‚स्माद‚नेन वाक्येन \textbf{द‚र्शिता} प्र‚काशिता । भ‚व‚त्व‚भाक्यो‚{\tiny $_{lb}$}‚र्व्याप्य‚व्याप‚क‚भाव‚स्त‚थाप्य‚न्व‚यः क‚थं सिद्ध्य‚तीत्याह--\textbf{य‚दी}ति । \textbf{चो} हेतौ । \textbf{न व्याप्ता भ‚वेत्‚{\tiny $_{lb}$}‚ साध्य‚निवृत्तिरि}ति प्र‚कृत‚त्वात् ॥
	\pend% ending standard par
      ‚{\tiny $_{lb}$}‚

	  \pstart \leavevmode% starting standard par
	\textbf{त‚स्मात्प‚क्षोव‚श्य‚मेव न} निर्देश्य इत्य‚नेन \textbf{नाव‚श्यं प‚क्ष‚निर्देश} इत्य‚स्यार्थः क‚थितः ।
	\pend% ending standard par
      ‚{\tiny $_{lb}$}‚

	  \pstart \leavevmode% starting standard par
	\hphantom{.}एव‚ञ्च व्याच‚क्षाणेन य‚त्कैश्चित्स्व‚यूथ्यैर्विद्व‚स्य‚मानैः अव‚श्यं प‚क्ष‚निर्देशो न; किन्त‚र्हि ?‚{\tiny $_{lb}$}‚ क‚दाचिन्निर्देशः, क‚दाचिन्न इति व्याख्यातं त‚द‚प‚ह‚स्तितं द्र‚ष्ट‚व्य‚म् । क‚दाचिद‚पि त‚स्य प्र‚योगार्ह‚त्वे‚{\tiny $_{lb}$}‚ प्र‚तिज्ञायाः साध‚नाङ्ग‚त्व‚प्र‚स‚ङ्गात् । त‚थात्वे च \textbf{वाद‚न्याय‚स्य}\leavevmode\ledsidenote{\textenglish{63a/ms}}\edtext{}{\lemma{च}\Bfootnote{अस्मिन् प‚त्रे अधिकं घृष्टं व‚र्त‚ते । अत एव स‚म्य‚क् न प‚ठ‚य‚ते--सं०}} \add{विरोधः स्यात्} ।
	\pend% ending standard par
      ‚{\tiny $_{lb}$}‚

	  \pstart \leavevmode% starting standard par
	न‚न्व‚स‚ति साध्य‚निर्देशे कुत‚स्त‚द‚व‚ग‚तिर्येन त‚द‚निर्देश इत्याश‚ङ्क्याह--\textbf{य‚दि}ति । \textbf{साध्य‚{\tiny $_{lb}$}‚निय‚तं} साध्य‚नान्त‚रीय‚क‚म् । प्र‚तिब‚न्ध‚साध‚केन प्र‚माणेनेति बुद्धिस्थ‚म् । \textbf{त‚त एव} साध‚नात्‚{\tiny $_{lb}$}‚ \textbf{ध‚र्मिणि} विवादास्प‚दीभूते \textbf{दृष्टात्} प्र‚माणेनाव‚ग‚तात् \textbf{साध्य}स्य ध‚र्म‚ध‚र्मिस‚मुदाय‚स्य \textbf{प्र‚तीति}र्भ‚व‚ति ।
	\pend% ending standard par
      ‚{\tiny $_{lb}$}‚

	  \pstart \leavevmode% starting standard par
	एत‚दुक्तं भ‚व‚ति...प्र‚तीतार्थ‚प्र‚तिपाद‚केन क‚र्त्त‚व्य‚मिति । \textbf{श‚ङ्क‚राचार्य}...ईश्व‚र‚{\tiny $_{lb}$}‚कार‚णे...विशेषः प्र‚तीयेतेत्यादिना वाक्य‚प्र‚ब‚न्धेन विरुद्ध‚श‚ङ्काव्य‚व‚च्छेदार्थं साध्य‚व‚च‚न‚मिति‚{\tiny $_{lb}$}‚ स‚माधानात् । त‚थाहि त‚स्य प्र‚ब‚न्ध‚स्याय‚म‚र्थः--अस‚ति साध्य‚व‚च‚ने य‚त् कृत‚कं त‚द् स‚र्व‚म‚नित्यं‚{\tiny $_{lb}$}‚ ‚{\tiny $_{lb}$}‚ \leavevmode\ledsidenote{\textenglish{173/dm}}‚{\tiny $_{lb}$}‚ य‚थाः...कुत‚श्चिद् भ्रान्तिनिमित्तादीदृशं व्याप्तिव‚च‚नं संभाव्य‚ते । अथ त‚थाविधा‚{\tiny $_{lb}$}‚भिप्रायो व‚क्ता कृत‚क‚त्वं प्र‚युञ्जानः त‚स्य नित्य‚त्वेन व्याप्तिं ब्रूयात् । त‚द‚युक्त‚म् । य‚त‚स्त‚{\tiny $_{lb}$}‚योर्व्याप्तिं ब्रुवाणो...क‚थ‚येत्...चेत् साध‚न\add{... ... ...}‚{\tiny $_{lb}$}‚ साध‚न‚विक‚लः साध्य‚विक‚लो वा मा भूद् दृष्टान्त इति य‚त् कृत‚कं त‚द‚नित्य‚मिति प्र‚योगोपि विरुद्ध‚{\tiny $_{lb}$}‚वाद्येव द्र‚ष्ट‚व्यः । ... त‚देत‚द्भौ \add{ताख्यानं} किम‚त्र ब्रूमः त‚था हि कः ख‚लु प्रेक्षावान्‚{\tiny $_{lb}$}‚ साध‚न‚विक‚लं विहाय‚सं साध्य‚विक‚लं च कुम्भ‚मालोच‚यितुमीशानोऽभिप्रेत‚नित्य‚त्व‚विरुद्धेनानित्य‚{\tiny $_{lb}$}‚त्वेन साध्ये \edtext{}{\lemma{साध्ये}\Bfootnote{ध्य}} विक‚ल‚त‚या कुम्भ‚स‚न्निभ एव घ‚टे व्याप्तिं द‚र्श‚येत् । भ्रान्त्या चेत् ।‚{\tiny $_{lb}$}‚ साध‚न‚वैक‚ल्य‚माकाश‚स्य\add{... ...}वाऽभिप्रेतेन नित्य‚त्वेन कृत‚क‚त्व‚स्य व्याप्निं न‚{\tiny $_{lb}$}‚ प्र‚द‚र्श‚येत् । त‚त्र साध‚न‚वैक‚ल्य‚म्, कुम्भे च साध्य‚वैक‚ल्य...कुम्भ‚तुल्येपि घ‚टे व्यामुह्य‚ति,‚{\tiny $_{lb}$}‚ भ्रान्तेर्निय‚त‚निमित्त‚त्वादिति चेत् । एवं त‚र्हि प‚क्ष‚व‚च‚नेपि क‚थं...संभाव्य‚त्वात् ।‚{\tiny $_{lb}$}‚ ...स‚मानोपि विसंवाद‚नाभिप्रायो न...। न चाप्युत्प‚त्तौ न चान्य‚दास्यान्यादृशं‚{\tiny $_{lb}$}‚ व‚च‚न‚मिदानीं तु द्रुतादिभेद‚भिन्न‚मित्यादिना प्र‚कारेण व‚च‚न‚विशेषेण ज्ञानेन कार्य‚भूतेनाविप‚रीतोऽ‚{\tiny $_{lb}$}‚भिप्रायोऽव‚धार्य‚ते । तेन\add{... ...}त‚द‚नित्य‚मित्य‚भिधात‚रि...विशेषः । व‚च‚न‚सांक‚र्यान्नैवं‚{\tiny $_{lb}$}‚ त‚त्र निश्च‚य इति चेत् । एत‚त् प‚क्ष‚व‚च‚नेपि स‚मान‚म् । एवं निश्चेतुं श‚क्येति त‚स्य त‚त्र‚{\tiny $_{lb}$}‚ त‚थात्वाव‚ग‚मो भ‚विष्य‚तीति चेत् स‚र्वं स‚मान‚म‚न्य‚त्राऽविशेषात् । किञ्च स एवं वादी त‚प‚स्वी‚{\tiny $_{lb}$}‚ ...\leavevmode\ledsidenote{\textenglish{63b/ms}}स्व‚य‚मेव ताव‚द् दुष्य‚ति । न हि क‚श्चित्साध‚न‚वादी प्र‚तिज्ञाहेतूदाह‚र‚णान्ये‚{\tiny $_{lb}$}‚वाभिधायोप‚र‚म‚ते । किन्त‚र्हि ? निग‚म‚न‚म‚प्युपाद‚त्ते--त‚स्माद‚नित्यः श‚ब्द इति । एव‚मुक्ते‚{\tiny $_{lb}$}‚ च कुतो नित्य‚त्व‚श‚ङ्का, य‚तः कृत‚क‚त्व‚स्य विरुद्ध‚ता भ‚वेत् ? त‚त‚श्च निग‚म‚नेनैव विरुद्ध‚श‚ङ्का‚{\tiny $_{lb}$}‚व्य‚व‚च्छेद‚स्य कृत‚त्वात्प‚क्ष‚व‚च‚न‚म‚पार्थ‚क‚म् । य‚दा\textbf{हाक्ष‚पादः}\edtext{}{\lemma{दा}\Bfootnote{अत्र अक्ष‚पाद‚व‚च‚न‚त्वेन य‚दुद्धृतं त‚न्नास्ति न्याय‚सूत्रे किन्तु वार्तिके--साध्य‚विप‚रीत‚{\tiny $_{lb}$}‚प्र‚स‚ङ्ग‚प्र‚तिषेधार्थं य‚त् पुन‚र‚भिधानं त‚त् निग‚म‚न‚म् इति व‚र्त‚ते--\href{http://sarit.indology.info/?cref=}{१. १. ३९.}}}--साध्ये विप‚रीत‚श‚ङ्काव्य‚व‚च्छेदार्थ‚{\tiny $_{lb}$}‚ निग‚म‚न‚मिति । त‚था हेतुव‚च‚न‚स्याप्य‚नुत्थान‚मायात‚म् । निग‚म‚नेऽनाश्वास इति चेद् । ह‚न्त‚{\tiny $_{lb}$}‚ प्र‚तिज्ञाव‚च‚नेऽप्य‚नाश्वास‚स्तुल्यः । य‚त्त‚त्र स‚माधानं त‚न्निग‚म‚नेऽपि भ‚विष्य‚ति । त‚स्मात् त‚न्म‚तेऽपि‚{\tiny $_{lb}$}‚ निग‚म‚नादेवाभिप्रेत‚साध्य‚प्र‚तीतेर्न विरुद्धाश‚ङ्कानिरासार्थं प‚क्ष‚व‚च‚नं क‚र‚णीय‚म् । त‚द‚यं य‚था नाम‚{\tiny $_{lb}$}‚ क‚श्चित् स्वाङ्गुलिज्वाल‚या प‚रं दिध‚क्षुः स प‚रं द‚हेद्वा न वा, स्वाङ्गुलिदाह‚मेव ताव‚द‚नुभ‚व‚तीति‚{\tiny $_{lb}$}‚ वृत्तान्तो जातः ।
	\pend% ending standard par
      ‚{\tiny $_{lb}$}‚

	  \pstart \leavevmode% starting standard par
	य‚द्येवं निग‚म‚न‚म‚प्य‚पार्थ‚क‚माप‚द्य‚त इति चेत् । अय‚म‚प‚रोऽस्तु दोषः । क‚थं नाम \textbf{ताथाग‚ता‚{\tiny $_{lb}$}‚ ज}य‚न्ति ? केव‚लं स‚ति निग‚म‚ने विरुद्ध‚श‚ङ्काव्य‚व‚च्छेदार्थं प‚क्ष‚व‚च‚नं न कार्य‚म् । निग‚म‚नेनैव‚{\tiny $_{lb}$}‚ त‚दाश‚ङ्काव्य‚व‚च्छेद‚स्य कृत‚त्वादित्युच्य‚त इति ।
	\pend% ending standard par
      ‚{\tiny $_{lb}$}‚

	  \pstart \leavevmode% starting standard par
	\textbf{त्रिलोच‚नः} पुन\textbf{र्न्याय‚भाष्य‚टीका}यामिद‚म‚वादीत्--साध्य‚व‚च‚न‚म‚साध‚नाङ्ग‚व‚च‚नं न भ‚व‚ति,‚{\tiny $_{lb}$}‚ य‚तो विवादेषु प‚र‚प्र‚तिप‚त्तिम‚धिकृत्य न प्र‚योग‚निय‚मः श‚क्यः । प‚टुम‚न्दादिभावेन प‚र‚प्र‚तिप‚त्तीना‚{\tiny $_{lb}$}‚म‚न‚व‚स्थानात् । त‚था हि हेतुव‚च‚नादेव क‚श्चित्प्र‚त्येति । क‚श्चित्पुन‚र‚न्त‚रेणापि हेतुव‚च‚नं‚{\tiny $_{lb}$}‚ व‚क्तृस्व‚रूप‚प‚रिशील‚नात्प्रागेव श‚ब्द‚निष्प‚त्तेरोष्ठादिस्थान‚व्यापारोप‚ल‚ब्धेर्व‚क्तुर‚भिप्रेत‚म‚न्वेति ।‚{\tiny $_{lb}$}‚ त‚स्माद‚न‚पेक्षित‚प‚र‚प्र‚तिप‚त्तिरेवायं ज्ञाता ज्ञान‚स्थ‚म‚र्थं प्र‚तिपाद‚य‚न्तं त‚स्य स्व‚प्र‚तिप‚त्त्याऽऽरूढ‚स्यार्थ‚स्य‚{\tiny $_{lb}$}‚ \leavevmode\ledsidenote{\textenglish{174/dm}}‚{\tiny $_{lb}$}‚ 
	  
	\edtext{\textsuperscript{*}}{\lemma{*}\Bfootnote{एव‚मे० \cite{dp-msA} \cite{dp-msB} \cite{dp-edP} \cite{dp-edH} \cite{dp-edE} एत‚मे० \cite{dp-edN}}}एन‚मेवार्थ‚म‚नुप‚ल‚ब्धिप्र‚योगे द‚र्श‚य‚ति-- ‚{\tiny $_{lb}$}‚ 
	  
	य‚स्मात् साध‚र्म्य‚व‚त्प्र‚योगेऽपि--य‚दुप‚ल‚ब्धिल‚क्ष‚ण‚प्राप्तं स‚न्नोप‚ल‚भ्य‚ते‚{\tiny $_{lb}$}‚ सोऽस‚द्व्य‚व‚हार‚विष‚यः\edtext{}{\lemma{यः}\Bfootnote{विष‚यः सिद्धः \cite{dp-msC}}} । नोप‚ल‚भ्य‚ते चात्रोप‚ल‚ब्धिल‚क्ष‚ण‚प्राप्तो घ‚ट इत्युक्ते‚{\tiny $_{lb}$}‚ साम‚र्थ्यादेव नेह घ‚ट इति भ‚व‚ति ॥ ३५ ॥‚{\tiny $_{lb}$}‚ 
	  
	साध‚र्म्य‚व‚ति प्र‚योगेपि साम‚र्थ्यादेव नेह प्र‚देशे\edtext{}{\lemma{देशे}\Bfootnote{नेह घ‚ट० \cite{dp-msA} \cite{dp-msB} \cite{dp-edP} \cite{dp-edH} \cite{dp-edN}}} घ‚ट इति भ‚व‚ति । ‚{\tiny $_{lb}$}‚ 
	  
	किं पुन‚स्त‚त् साम‚र्थ्य‚मित्याह--य‚दुप‚ल‚ब्धिल‚क्ष‚ण‚प्राप्तं\edtext{}{\lemma{प्राप्तं}\Bfootnote{प्राप्त मिति । अनु० \cite{dp-msA} \cite{dp-edP} \cite{dp-edH}}} स‚न्नोप‚ल‚भ्य‚ते--इत्य‚नुप‚ल‚म्भा‚{\tiny $_{lb}$}‚नुवादः । सोऽस‚द्व्य‚व‚हार‚विष‚यः--इत्य‚स‚द्व्य‚व‚हार‚योग्य‚त्व‚विधिः । त‚था च स‚ति दृश्यानु-‚{\tiny $_{lb}$}‚ वाच‚कं श‚ब्दं प्र‚योक्तुम‚र्ह‚ति । स्व‚प्र‚तिप‚त्तिश्च लिङ्ग‚जा ज्ञाप‚नीय‚ध‚र्म‚विशिष्टं ध‚र्णिम‚भिनिविश‚ते ।‚{\tiny $_{lb}$}‚ त‚स्मात्प‚र‚स्य विवाद‚यित्रा ज्ञान‚स्थ‚म‚र्थं प‚रो बोद्ध‚व्य इति स एव प‚रं प्र‚त्युपाय इति ।
	\pend% ending standard par
      ‚{\tiny $_{lb}$}‚

	  \pstart \leavevmode% starting standard par
	त‚देत\textbf{त्कार्प‚टिक‚क‚र्णाट}र‚टित‚म‚श्र‚द्धेयं धीम‚ताम् । त‚था हि--स‚त्य‚म्, स्व‚प्र‚तिप‚त्त्याऽऽरूढ‚{\tiny $_{lb}$}‚ एवार्थः प‚र‚स्मै प्र‚तिपाद्य‚ते । केव‚ल‚मिद‚मालोच्य‚ताम्--किं प‚क्ष‚ध‚र्म‚व‚च‚नाद् व्याप्तिव‚च‚न‚स‚हिता‚{\tiny $_{lb}$}‚त्सोऽर्थः प्र‚तिपादितो भ‚व‚ति न‚वेति । प्र‚तिपाद‚ने किं प्र‚तीत‚प्र‚त्याय‚केन त‚द्व‚च‚नेन कार्य‚म् ?‚{\tiny $_{lb}$}‚ ताव‚तो व‚च‚नात्त‚त्प्र‚तिप‚त्तिम‚प‚ह्न‚वानेव \edtext{}{\lemma{वानेव}\Bfootnote{नेन}} तु नाप‚ह‚नुतं नाम किञ्चिद् । अव‚श्यं चैत‚द‚न्य‚था‚{\tiny $_{lb}$}‚ स्वार्थानुमान‚काले प्र‚तिज्ञाव‚च‚न‚म‚न्त‚रेण क‚थं प्र‚तिप‚त्तिः स्यात् ? स्व‚प्र‚तिप‚त्तिकाले च याव‚तोऽ‚{\tiny $_{lb}$}‚ थात्साध्य‚प्र‚तीतिरासीत् प‚रार्थानुमान‚कालेऽपि ताव‚त एव व‚च‚न‚मुपादेय‚म् । त‚त्र च न प्र‚ज्ञाप‚नीय‚{\tiny $_{lb}$}‚ध‚र्म‚विशिष्ट‚ध‚र्मिद‚र्श‚न‚पूर्व‚कादिसाध‚नादिद‚र्श‚नात् साध्य‚प्र‚तीतिरासीत् । किन्त‚र्हि ? प‚क्ष‚{\tiny $_{lb}$}‚ध‚र्म‚द‚र्श‚नात् त‚द‚विनाभाव‚स्म‚र‚ण‚स‚हितादिति ताव‚त एव व‚च‚नं न्याय्य‚म् ।
	\pend% ending standard par
      ‚{\tiny $_{lb}$}‚

	  \pstart \leavevmode% starting standard par
	\textbf{अथो}क्तं प‚टुम‚न्दादि भावेन प‚र‚प्र‚तीतीनाम‚न‚व‚स्थानान्न श‚क्य‚ते प्र‚योग‚निय‚मः क‚र्त्तुमिति ।‚{\tiny $_{lb}$}‚ स‚त्य‚मुक्त‚म्\leavevmode\ledsidenote{\textenglish{64a/ms}} केव‚लं स्व‚व‚धाय कृत्योत्थाप‚न‚प्रायं त‚त् । य‚तः प‚टुम‚न्दादिभेदेन प्र‚ति‚{\tiny $_{lb}$}‚प‚तृणाम‚नेक‚प्र‚कार‚त्वात्स्याद‚पि क‚श्चिद् यः प‚ञ्चाव‚य‚वेऽपि वाक्ये प्र‚युक्ते पूर्वं संश‚य‚जिज्ञासा‚{\tiny $_{lb}$}‚दिव‚च‚न‚म‚न्त‚रेण न बुद्ध्य‚ते बोध‚यित‚व्य‚मिति त‚द्व‚च‚न‚स्याप्य‚व‚श्य‚प्र‚योज्य‚त्वाद‚व‚य‚व‚त्वाद‚व‚ग‚त‚{\tiny $_{lb}$}‚\edtext{}{\lemma{त}\Bfootnote{द‚प‚ग‚तं}} प‚ञ्चाव‚य‚व‚त्वं साध‚न‚वाक्य‚स्य । अभ्युप‚ग‚मे च \textbf{गौड्.अकाश्मीर}पुरुष‚विधायो \edtext{}{\lemma{विधायो}\Bfootnote{विष‚यो}}‚{\tiny $_{lb}$}‚पाख्यानं कुतूह‚लास्प‚द‚म‚व‚त‚र‚ते । प्र‚तिज्ञाहे\textbf{तू}दाह‚र‚णोप‚न‚य‚निग‚म‚नान्य‚वाव‚य‚वा इति शास्त्र‚स्थिते‚{\tiny $_{lb}$}‚र‚प‚सिद्धान्तोऽपि दीप्ताज्ञः पार्थिव इव निगृह‚णाति ।
	\pend% ending standard par
      ‚{\tiny $_{lb}$}‚

	  \pstart \leavevmode% starting standard par
	अथ किम‚स्य स‚म्भ‚वोऽस्ति यो निर्दिष्टे हि साध्ये साध‚ने वाऽभिहिते निद‚र्शिते चोदाह‚र‚णे‚{\tiny $_{lb}$}‚ कृतेऽप्युप‚न‚ये निग‚मिते च स‚र्वाव‚य‚व‚व्यापारे साध्यं न बुध्य‚त इति ? न‚नु अस्यापि प्र‚तिप‚त्तुः‚{\tiny $_{lb}$}‚ किम‚स्ति स‚म्भ‚वो य‚त्र ध‚र्मिणि साध‚नं बोधितः, त‚स्य साध्याविनाभावितां स्म‚रितोऽपि य‚स्त‚त्र‚{\tiny $_{lb}$}‚ साध्यं नाव‚बुध्य‚त इति ? स‚म्भ‚व‚ति बुद्धिमान्द्यादिति चेत् । स‚र्वं स‚मान‚मिद‚म‚न्य‚त्राभिनिवेशा‚{\tiny $_{lb}$}‚दित्य‚लं विस्त‚रेण ।
	\pend% ending standard par
      ‚{\tiny $_{lb}$}‚‚{\tiny $_{lb}$}‚\textsuperscript{\textenglish{175/dm}}‚{\tiny $_{lb}$}‚
	  \bigskip
	  \begingroup
	

	  \pstart \leavevmode% starting standard par
	प‚ल‚म्भोऽस‚द्व्य‚व‚हार‚योग्य‚त्वेन व्याप्तो द‚र्शितः । नोप‚ल‚भ्य‚ते च इत्यादिना\edtext{}{\lemma{इत्यादिना}\Bfootnote{०भ्य‚ते इत्या० \cite{dp-msA} \cite{dp-msB} \cite{dp-edP} \cite{dp-edH} \cite{dp-edN}}} साध्य‚ध‚र्मिणि‚{\tiny $_{lb}$}‚ स‚त्त्वं लिङ्ग‚स्य द‚र्शित‚म् । य‚दि च साध्य‚ध‚र्म‚स्त‚त्र साध्य‚ध‚र्मिणि न भ‚वेत् साध‚न‚र्मोऽपि न‚{\tiny $_{lb}$}‚ भ‚वेत् । साध्य‚निय‚त‚त्वात् त‚स्य साध‚न‚ध‚र्म‚स्येति साम‚र्थ्य‚म ॥
	\pend% ending standard par
       ‚{\tiny $_{lb}$}‚ 
	  \bigskip
	  \begingroup
	

	  \pstart \leavevmode% starting standard par
	त‚था वैध‚र्म्य‚व‚त्प्र‚योगेऽपि--यः स‚द्व्य‚व‚हार‚विष‚य उप‚ल‚ब्धिल‚क्ष‚ण‚प्राप्तः,‚{\tiny $_{lb}$}‚ स उप‚ल‚भ्य‚त एव । न\edtext{}{\lemma{न}\Bfootnote{न च \cite{dp-msC}}} त‚थाऽत्र तादृशो घ‚ट उप‚ल‚भ्य‚त इत्युक्ते साम‚र्थ्या‚{\tiny $_{lb}$}‚देव नेह स‚द्व्य‚व‚हार‚विष‚य इति भ‚व‚ति ॥ ३६ ॥
	\pend% ending standard par
      
	  \endgroup
	‚{\tiny $_{lb}$}‚ 

	  \pstart \leavevmode% starting standard par
	य‚था साध‚र्म्य‚व‚त्प्र‚योगे त‚था वैध‚र्म्य‚व‚त्प्र‚योगेऽपि साम‚र्थ्यादेव नेह स‚द्व्य‚व‚हार\edtext{}{\lemma{हार}\Bfootnote{०व्य‚व‚हार‚स्य विष० \cite{dp-msC} \cite{dp-msD}}}विष‚योऽस्ति‚{\tiny $_{lb}$}‚ घ‚ट इति भ‚व‚ति ।
	\pend% ending standard par
       ‚{\tiny $_{lb}$}‚ 

	  \pstart \leavevmode% starting standard par
	साम‚र्थ्य द‚र्श‚यितुमाह--यः स‚द्व्य‚व‚हार‚विष‚य इति विद्य‚मानः । उप‚ल‚ब्धिल‚क्ष‚ण‚प्राप्त इति‚{\tiny $_{lb}$}‚ दृश्यः इत्येषा साध्य‚निवृत्तिः । \edtext{\textsuperscript{*}}{\lemma{*}\Bfootnote{स नास्ति \cite{dp-msA} \cite{dp-msB} \cite{dp-msC} \cite{dp-msD} \cite{dp-edP} \cite{dp-edH}}}स उप‚ल‚भ्य‚त एवेति साध‚न‚निवृत्तिरिति । अनेन च\edtext{}{\lemma{च}\Bfootnote{च नास्ति \cite{dp-msC} \cite{dp-edE} अनेन न \cite{dp-edH}}} साध्य‚{\tiny $_{lb}$}‚निवृत्तिः साध‚न‚निवृत्त्या व्याप्ता द‚र्शिता । न त‚थेति--य‚थाऽन्यो दृश्य उप‚ल‚भ्य‚ते न त‚थात्र‚{\tiny $_{lb}$}‚ प्र‚देशे तादृश इति दृश्यो घ‚ट उप‚ल‚भ्य‚त इति । अनेन साध्य‚निवृत्तेर्व्यापिका निवृत्तिर‚स‚ती‚{\tiny $_{lb}$}‚ साध्य‚ध‚र्मिणि द‚र्शिता । य‚दि च\edtext{}{\lemma{च}\Bfootnote{च न साध्य० \cite{dp-msA} \cite{dp-msB} \cite{dp-edP} \cite{dp-edH} \cite{dp-edN}}} साध्य‚ध‚र्मः साध्य‚ध‚र्मिणि न स्यात् साध‚न‚ध‚र्मोऽपि\edtext{}{\lemma{र्मोऽपि}\Bfootnote{०र्मिणि भ‚वेत् साध‚न० \cite{dp-msA} \cite{dp-msB} \cite{dp-edP} \cite{dp-edH} \cite{dp-edE} \cite{dp-edN}}} न भ‚वेत् ।
	\pend% ending standard par
      
	  \endgroup
	‚{\tiny $_{lb}$}‚

	  \pstart \leavevmode% starting standard par
	त‚द‚पि\edtext{}{\lemma{पि}\Bfootnote{य‚द‚पि}}\textbf{न्याय‚भाष्य‚टीका-वातिंक‚योर्विश्व‚रूपोद्योत‚क‚रावा}ह‚तुः पुरा विष‚य‚निरूप‚ण‚{\tiny $_{lb}$}‚पूर्व‚क‚मेव हि क‚र‚ण‚व्याप‚र‚णं दृष्ट‚म् । क‚र‚णं च साध‚न\edtext{}{\lemma{न}\Bfootnote{नं}}व्यापार‚यित‚व्य‚म् । अतो विष‚य‚{\tiny $_{lb}$}‚निरूप‚णं साध्य‚व‚च‚नेन क्रिय‚ते, अन्य‚था क‚र‚ण‚प्र‚व‚र्त्त‚न‚स्याश‚क्य‚त्वादिति ।\edtext{}{\lemma{तुः}\Bfootnote{नोप‚ल‚भ्य‚ते न्याय‚वार्तिके--सं०}} त‚द‚पि न च‚तुर‚स्र‚म् ।‚{\tiny $_{lb}$}‚ य‚तो य‚दि हेतुं प्र‚युञ्जानेन विष‚यः सिसाध‚यिषितोऽर्थों निरूप‚यित‚व्यो बुद्धौ निवेश‚नीय इत्य‚भि‚{\tiny $_{lb}$}‚म‚त‚म्, त‚दाऽभ्युप‚ग‚म एवोत्त‚र‚म् । न‚हि क‚श्चित् साध्य‚म‚निश्चित्यैव प‚र‚प्र‚तिप‚त्त‚ये साध‚न‚वाक्य‚{\tiny $_{lb}$}‚म‚भिध‚त्ते । अथ व‚च‚नेन क‚र‚ण‚स्य हेतोः स विष‚यो द‚र्श‚यित‚व्य इति म‚तिः, त‚दा तेनैव ताव‚द्‚{\tiny $_{lb}$}‚ द‚र्शितेन कोऽर्थः ? य‚दि प‚र‚स्य प्र‚तीतिर‚न्य‚था न स्यात्स‚र्वं शोभेतेत्युत्त‚र‚मिति किं क्षुण्ण‚क्षो‚{\tiny $_{lb}$}‚दीक‚र‚णेन ?
	\pend% ending standard par
      ‚{\tiny $_{lb}$}‚

	  \pstart \leavevmode% starting standard par
	\textbf{अध्य‚य‚नः} पुना \textbf{रुचिटीका}यामिद‚म‚वोच‚त् ध‚र्म‚विशिष्ट‚स्य ध‚र्मिणो निर्देशः क्रिय‚ते श्रोतु‚{\tiny $_{lb}$}‚राश्वास‚नार्थ‚म् । न त्वादौ ध‚र्म‚विशिष्ट‚स्य ध‚र्मिणो निर्देशो युक्तः । अयुक्त‚तां \edtext{}{\lemma{तां}\Bfootnote{ता}} त‚स्य‚{\tiny $_{lb}$}‚ प्र‚तिप‚त्ताव‚दृष्ट‚त्वात् । त‚त्र प्र‚देश‚मात्र‚मुप‚ल‚भ‚ते, त‚त्स्थं च ध‚र्म‚म् । त‚तोऽविनाभावं स्म‚र‚ति ।‚{\tiny $_{lb}$}‚ त‚द‚न‚न्त‚रं त‚देवेद‚मिति प‚रामृश‚ति । त‚तो विशिष्ट‚तां प्र‚देश‚स्य प्र‚तिप‚द्य‚ते, न त्वादेव \edtext{}{\lemma{त्वादेव}\Bfootnote{त्वादावेव}} ।‚{\tiny $_{lb}$}‚ प‚राम‚र्श‚स्य च स्वार्थ‚पूर्व‚क‚त्व‚म् । न च स्वार्थे ध‚र्म‚विशिष्ट‚स्य ध‚र्मिणो द‚र्श‚न‚म‚स्ति ।‚{\tiny $_{lb}$}‚ तेन प‚र‚प्र‚तिप‚त्ताव‚पि न कार्य‚म् । आदौ तु क्रिय‚ते, प्र‚तिपाद्य‚स्यास्थोत्पाद‚नार्थ‚मिति । ‚{\tiny $_{lb}$}‚ \leavevmode\ledsidenote{\textenglish{176/dm}}‚{\tiny $_{lb}$}‚ 
	  
	अस्ति च साध‚न‚ध‚र्म इति साम‚र्थ्य‚म् ।\edtext{\textsuperscript{*}}{\lemma{*}\Bfootnote{साम‚र्थ्यात् त‚तः \cite{dp-msA} \cite{dp-msB} \cite{dp-edP} \cite{dp-edH} \cite{dp-edN}}} अतः साम‚र्थ्यात् नात्स्य‚त्र घ‚ट इति प्र‚तीतेर्न प‚क्ष‚निर्देशः ।‚{\tiny $_{lb}$}‚ एवं कार्य‚स्व‚भाव‚हेत्वोर‚पि साम‚र्थ्यात् संप्र‚त्य‚य इति न \edtext{}{\lemma{न}\Bfootnote{प‚क्षो निर्देश्यः \cite{dp-edE}}}प‚क्ष‚निर्देशः ॥ ‚{\tiny $_{lb}$}‚ 
	  
	कीदृशः पुनः \edtext{}{\lemma{पुनः}\Bfootnote{प‚क्षः निर्देशः \cite{dp-msC}}}प‚क्ष इति निर्देश्यः ? ॥ ३७ ॥‚{\tiny $_{lb}$}‚ 
	  
	कीदृशः पुर‚र‚र्थः प‚क्ष इति--अनेन‚श‚ब्देन निर्देश्यो व‚क्त‚व्यः ? इत्याह-- ‚{\tiny $_{lb}$}‚ 
	  
	स्व‚रूपेणैव स्व‚य‚मिष्टो\edtext{}{\lemma{मिष्टो}\Bfootnote{०मिष्टो निरा० \cite{dp-msB} \cite{dp-edP}}}ऽनिराकृतः प‚क्ष इति\edtext{}{\lemma{इति}\Bfootnote{इति निर्देश्यः--\cite{dp-msC}}} ॥ ३८ ॥‚{\tiny $_{lb}$}‚ 
	  
	स्व‚रूपेणैवेति साध्य‚त्वेनैव । स्व‚य‚मिति वादिना । इष्ट इति--नोक्त एवापि त्विष्टो‚{\tiny $_{lb}$}‚ऽपीत्य‚र्थः । एवंभूतः स‚न् प्र‚त्य‚क्षादिभिः अनिराकृतो \edtext{}{\lemma{अनिराकृतो}\Bfootnote{०कृतोऽर्थो यः स \cite{dp-msB} \cite{dp-msC} \cite{dp-msD}}}योऽर्थः स प‚क्ष इत्युच्य‚ते । ‚{\tiny $_{lb}$}‚ 
	  
	अथ य‚दि \edtext{}{\lemma{दि}\Bfootnote{अथ य‚दि न प‚क्षो \cite{dp-msA} \cite{dp-msB} \cite{dp-edP} \cite{dp-edH} \cite{dp-edE} \cite{dp-edN}}}प‚क्षो न निर्देश्यः, क‚थ‚म‚निर्देश्य‚स्य ल‚क्ष‚ण‚मुक्त‚म् ? न साध‚न‚वाक्याव‚य‚{\tiny $_{lb}$}‚व‚त्वाद‚स्य ल‚क्ष‚ण‚मुक्त‚म‚पि त्व‚साध्यं \edtext{}{\lemma{साध्यं}\Bfootnote{किञ्च‚त्--\cite{dp-msB}}}केचित् साध्य‚म्, साध्यं चासाध्यं \edtext{}{\lemma{चासाध्यं}\Bfootnote{केचित् नास्ति--\cite{dp-msC} \cite{dp-msA} \cite{dp-edP} \cite{dp-edH} \cite{dp-edE} \cite{dp-edN}}}केचित् प्र‚तिप‚न्नाः ।‚{\tiny $_{lb}$}‚ त‚त् साध्यासाध्य‚विप्र‚तिप‚त्तिनिकार‚णार्थं प‚क्ष‚ल‚क्ष‚ण‚मुक्त‚म् ॥ ‚{\tiny $_{lb}$}‚ 
	  
	स्व‚रूपेणेष्ट इत्य‚स्य विव‚र‚ण‚म-- ‚{\tiny $_{lb}$}‚ 
	  
	स्व‚रूपेणेति साध्य‚त्वेनेष्टः ॥ ३९ ॥‚{\tiny $_{lb}$}‚ 
	  
	साध्य‚त्वेनेष्ट इति । प‚क्ष‚स्य साध्य‚त्वान्नाप‚र‚म‚स्ति रूप‚म् । अतः स्व‚रूपं साध्य‚त्व‚मिति ॥ ‚{\tiny $_{lb}$}‚ 
	  
	एव‚श‚ब्दं विव‚रितुमाह-- ‚{\tiny $_{lb}$}‚ 
	  
	स्व‚रूपेणैवेति साध्य‚त्बेनैवेष्टो\edtext{}{\lemma{त्बेनैवेष्टो}\Bfootnote{०त्वेनेष्टो \cite{dp-msC} \cite{dp-msB} \cite{dp-edH} \cite{dp-edP} \cite{dp-edE} \cite{dp-edN}}} न साध‚न‚त्वेनापि ॥ ४० ॥‚{\tiny $_{lb}$}‚ 
	  
	स्व‚रूपेणैवेति । न‚नु चैव‚श‚ब्दः केव‚ल एव प्र‚त्य‚व‚म‚र्ष्ट‚व्य‚स्त‚त्\edtext{}{\lemma{त्}\Bfootnote{त‚त्क‚थ‚म्--\cite{dp-msB}}} किम‚र्थं स्व‚रूप‚श‚ब्देन‚{\tiny $_{lb}$}‚ तेन तु त‚प‚स्विना ब‚हूक्तं स‚म‚ञ्ज‚सं । केव‚लं प्र‚तिप‚त्तुराश्वासेनैवोत्पादितेन किं प्र‚योज‚न‚म् ?‚{\tiny $_{lb}$}‚ क‚थं चासौ स‚न्दिग्धार्थाभिधायिनः प्र‚तिज्ञाव‚च‚नादास्थामुत्पाद‚य‚तीति स‚मीचीनं निरूपित‚म्!‚{\tiny $_{lb}$}‚ आस्था ख‚लु इद‚मेव म‚न्येऽथेत्य‚भिस‚म्प्र‚त्य‚यः । सा क‚थं व‚च‚न‚मात्राज्जायेत ? जातौ वा‚{\tiny $_{lb}$}‚ साध‚नाद्य‚भिधानं न क‚थं वैय‚र्थ्य‚म‚श्नुवीतेत्य‚लं ब‚हुना ॥
	\pend% ending standard par
      ‚{\tiny $_{lb}$}‚

	  \pstart \leavevmode% starting standard par
	अत्र साम‚र्थ्यात्स्व‚यं श‚ब्द‚स्य वादिनेति विवृतिः कृता न तु स्व‚यंश‚ब्द‚स्य वादिनेत्य‚र्थः ।‚{\tiny $_{lb}$}‚ एत‚च्चान‚न्त‚र‚मेव द‚र्श‚यिष्य‚ते ॥
	\pend% ending standard par
      ‚{\tiny $_{lb}$}‚‚{\tiny $_{lb}$}‚\textsuperscript{\textenglish{177/dm}}‚{\tiny $_{lb}$}‚
	  \bigskip
	  \begingroup
	

	  \pstart \leavevmode% starting standard par
	स‚ह प्र‚त्य‚व‚मृष्टः ? उच्य‚ते । एव‚श‚ब्दो निपातो द्योत‚कः । प‚दान्त‚राभिहित‚स्यार्थ‚स्य‚{\tiny $_{lb}$}‚ विशेषं द्योत‚य‚ति इति प‚दान्त‚रेण विशेष्य‚वाचिना स‚ह निर्दिष्टः । न साध‚न‚त्वेनापीति । य‚त्‚{\tiny $_{lb}$}‚ साध‚न‚त्वेन निर्दिष्टं त‚त् साध‚न‚त्वेनेष्ट‚म् । असिद्ध‚त्वाच्च\edtext{}{\lemma{त्वाच्च}\Bfootnote{०त्वात् साध्य० \cite{dp-msB}}} साध्य‚त्वेनापीष्ट‚म् । त‚स्य‚{\tiny $_{lb}$}‚ निवृत्त्य‚र्थ\edtext{}{\lemma{र्थ}\Bfootnote{०त्त्य‚र्थ‚म्--\cite{dp-msD}}} एव‚श‚ब्दः ॥
	\pend% ending standard par
       ‚{\tiny $_{lb}$}‚ 

	  \pstart \leavevmode% starting standard par
	त‚दुदाह‚र‚ति--
	\pend% ending standard par
       ‚{\tiny $_{lb}$}‚ 
	  \bigskip
	  \begingroup
	

	  \pstart \leavevmode% starting standard par
	य‚था श‚ब्द‚स्यानित्य‚त्वे साध्ये चाक्षुष‚त्वं हेतुः, श‚ब्देऽसिद्ध‚त्वात्‚{\tiny $_{lb}$}‚ साध्य‚म् । न पुन‚स्त‚दिह साध्य‚त्वेनैवेष्ट‚म्\edtext{}{\lemma{म्}\Bfootnote{साध्य‚त्वेनेष्ट‚म् \cite{dp-msC} \cite{dp-msD} \cite{dp-edE}}}, साध‚न‚त्वेनाभिधानात्\edtext{}{\lemma{त्वेनाभिधानात्}\Bfootnote{०त्वेनाप्य‚भिधानात् \cite{dp-msB} \cite{dp-edP} \cite{dp-edH} \cite{dp-edE} \cite{dp-edN}}} ॥ ४१ ॥
	\pend% ending standard par
      
	  \endgroup
	‚{\tiny $_{lb}$}‚ 

	  \pstart \leavevmode% starting standard par
	\edtext{\textsuperscript{*}}{\lemma{*}\Bfootnote{य‚थेति नास्ति \cite{dp-msA}}}य‚थेति । श‚ब्द‚स्यानित्य‚त्वे साध्ये चाक्षुष‚त्वं हेतुः श‚ब्देऽसिद्ध‚त्वात् साध्य‚म्--इत्य‚नेन‚{\tiny $_{lb}$}‚ साध्य‚त्वेनेष्टिमाह ।
	\pend% ending standard par
       ‚{\tiny $_{lb}$}‚ 

	  \pstart \leavevmode% starting standard par
	त‚द् इति चाक्षुष‚त्व‚म् । इहेति श‚ब्दे । साध्य‚त्वेनैवेष्ट‚म्--इति साध्य‚त्वेनेष्टिनिय‚मा‚{\tiny $_{lb}$}‚भाव‚माह । साध‚न‚त्वेनाभिधानाद् इति--य‚तः साध‚न‚त्वेनाभिहित‚म्, अतः साध‚न‚त्वेनापीष्ट‚म् ।‚{\tiny $_{lb}$}‚ न साध्य‚त्वेनैवेति ॥
	\pend% ending standard par
       ‚{\tiny $_{lb}$}‚ 

	  \pstart \leavevmode% starting standard par
	स्व‚य‚मित्य‚नेन स्व‚यंश‚ब्दं व्याख्येय‚मुप‚क्षिप्य त‚स्यार्थ‚माह--
	\pend% ending standard par
       ‚{\tiny $_{lb}$}‚ 
	  \bigskip
	  \begingroup
	

	  \pstart \leavevmode% starting standard par
	स्व‚य‚मिति वादिना ॥ ४२ ॥
	\pend% ending standard par
      
	  \endgroup
	‚{\tiny $_{lb}$}‚ 

	  \pstart \leavevmode% starting standard par
	वादिनेति । स्व‚यंश‚ब्दो निपात आत्म‚न इति \edtext{}{\lemma{इति}\Bfootnote{नाशं स्व‚य‚मिच्छ‚तीत्यादौ आत्म‚नो नाश‚मिच्छ‚तीत्य‚र्थः--\cite{dp-msD-n}}}ष‚ष्ठ्य‚न्त‚स्यात्म‚नेति च तृतीयान्त‚{\tiny $_{lb}$}‚स्याथ\edtext{}{\lemma{स्याथ}\Bfootnote{०स्यार्थेन युक्तः--\cite{dp-msB}}} व‚र्त्त‚ते । त‚दिह तृतीयान्त‚स्यात्म‚श‚ब्द‚स्यार्थे वृत्तः स्व‚यंश‚ब्दः । आत्म‚श‚ब्द‚श्च स‚म्ब‚न्धि‚{\tiny $_{lb}$}‚श‚ब्दः । वादी च प्र‚त्यास‚न्नः\edtext{}{\lemma{न्नः}\Bfootnote{प्र‚त्यास‚न्न‚भूतः य‚स्य \cite{dp-msA} \cite{dp-msB} \cite{dp-edP} \cite{dp-edH}}} । त‚तो य‚स्य वादिन आत्मा तृतीयार्थ‚युक्तः\edtext{}{\lemma{युक्तः}\Bfootnote{तृतीयार्थेन युक्तः \cite{dp-msC} \cite{dp-msD}}} स एव\edtext{}{\lemma{एव}\Bfootnote{एव नास्ति--\cite{dp-msB}}}‚{\tiny $_{lb}$}‚ तृतीयार्थ‚युक्तो निर्दिष्टो वादिनेति । न\edtext{}{\lemma{न}\Bfootnote{न‚नु \cite{dp-msA} \cite{dp-msB} \cite{dp-edP} \cite{dp-edH}}} तु स्व‚यंश‚ब्द‚स्य वादिनेत्येष प‚र्यायः ॥
	\pend% ending standard par
      
	  \endgroup
	‚{\tiny $_{lb}$}‚

	  \pstart \leavevmode% starting standard par
	प‚क्ष‚स्यानुमेय‚स्य ॥
	\pend% ending standard par
      ‚{\tiny $_{lb}$}‚

	  \pstart \leavevmode% starting standard par
	स्व‚रूप‚श‚ब्देनेति स‚हार्थे तृतीया ।
	\pend% ending standard par
      ‚{\tiny $_{lb}$}‚

	  \pstart \leavevmode% starting standard par
	\leavevmode\ledsidenote{\textenglish{64b/ms}} \textbf{उच्य‚त} इति सिद्धान्त‚वादी ।
	\pend% ending standard par
      ‚{\tiny $_{lb}$}‚

	  \pstart \leavevmode% starting standard par
	य‚त्साध‚न‚त्वेनेष्टं त‚त्क‚थं साध्य‚त्वेनापीष्टं भ‚व‚तीत्याह--\textbf{असिद्ध\add{त्वा}दिति । चो} य‚स्मात् ।‚{\tiny $_{lb}$}‚ साध्य‚त्वेनेष्टोऽपि य‚दा साध‚न‚त्वेनोक्त‚स्त दाऽप‚क्ष इत्येव‚म‚र्थ \textbf{एव}श‚ब्द इति स‚मुदायार्थः ॥
	\pend% ending standard par
      ‚{\tiny $_{lb}$}‚‚{\tiny $_{lb}$}‚\textsuperscript{\textenglish{178/dm}}‚{\tiny $_{lb}$}‚
	  \bigskip
	  \begingroup
	

	  \pstart \leavevmode% starting standard par
	कः पुन‚र‚सौ वादीत्याह--
	\pend% ending standard par
       ‚{\tiny $_{lb}$}‚ 
	  \bigskip
	  \begingroup
	

	  \pstart \leavevmode% starting standard par
	य‚स्त‚दा साध‚न‚माह ॥ ४३ ॥
	\pend% ending standard par
      
	  \endgroup
	‚{\tiny $_{lb}$}‚ 

	  \pstart \leavevmode% starting standard par
	य‚स्त‚दा--इति वाद‚काले साध‚न‚माह । अनेक‚वादिस‚म्भ‚वेऽपि\edtext{}{\lemma{वेऽपि}\Bfootnote{०स‚म्भ‚वे स्व‚यं० \cite{dp-msB} \cite{dp-msC} \cite{dp-msD}}} स्व‚यंश‚ब्द‚वाच्य‚स्य‚{\tiny $_{lb}$}‚ वादिनो विशेष‚ण‚मेत‚त् ।
	\pend% ending standard par
       ‚{\tiny $_{lb}$}‚ 

	  \pstart \leavevmode% starting standard par
	य‚द्येवं\edtext{}{\lemma{द्येवं}\Bfootnote{य‚द्येव--\cite{dp-msA} \cite{dp-msB} \cite{dp-edP} \cite{dp-edH}}} वादिन इष्टः साध्यः--इत्युक्त‚म् । एतेन च किमुक्तेन ? अनेन\edtext{}{\lemma{अनेन}\Bfootnote{अनेन च \cite{dp-msC}}} त‚दा‚{\tiny $_{lb}$}‚ वाद‚काले तेन वादिना स्व‚यं यो ध‚र्मः साध‚यितुमिष्टः स एव साध्यो\edtext{}{\lemma{साध्यो}\Bfootnote{साध्यो ध‚र्मो नेत‚र इत्यु० \cite{dp-msC}}} नेत‚रो \edtext{}{\lemma{रो}\Bfootnote{ध‚र्म नास्ति \cite{dp-msB}}}ध‚र्म इत्युक्तं‚{\tiny $_{lb}$}‚ भ‚व‚ति । वादिनोऽनिष्ट‚ध‚र्म‚साध्य‚त्व‚निव‚र्त्त‚न‚म‚स्य व‚च‚न‚स्य फ‚ल‚मिति याव‚त् ॥
	\pend% ending standard par
       ‚{\tiny $_{lb}$}‚ 

	  \pstart \leavevmode% starting standard par
	अथ क‚स्मिन् स‚त्य‚न्य‚ध‚र्म‚साध्य‚त्व‚स्य\edtext{}{\lemma{स्य}\Bfootnote{साध्य‚त्व‚स‚म्भ० \cite{dp-msA} \cite{dp-msC} \cite{dp-edP} \cite{dp-edH} \cite{dp-edE}}} स‚म्भ‚वो य‚न्निवृत्त्य‚र्थं \edtext{}{\lemma{र्थं}\Bfootnote{०त्त्य‚र्थं चेदं \cite{dp-msA} \cite{dp-msD} \cite{dp-edP} \cite{dp-edH} \cite{dp-edE} \cite{dp-edN} ०त्त्य‚र्थं चैत‚त्--\cite{dp-msB}}}त‚द् व‚क्त‚व्य‚मित्याह--
	\pend% ending standard par
       ‚{\tiny $_{lb}$}‚ 
	  \bigskip
	  \begingroup
	

	  \pstart \leavevmode% starting standard par
	एतेन य‚द्य‚पि क्व‚चिच्छास्त्रे स्थितः साध‚न‚माह, त‚च्छास्त्र‚कारेण त‚स्मिन्‚{\tiny $_{lb}$}‚ ध‚र्मिएय‚नेक‚ध‚र्माभ्युप‚ग‚मेऽपि य‚स्त‚दा तेन वादिना\edtext{}{\lemma{वादिना}\Bfootnote{तेन स्व‚यं वादिना ध‚र्मः साध० \cite{dp-msC}}} ध‚र्मः स्व‚यं साध‚यितु-
	\pend% ending standard par
      
	  \endgroup
	
	  \endgroup
	‚{\tiny $_{lb}$}‚

	  \pstart \leavevmode% starting standard par
	\textbf{त‚त्त}स्माद‚र्थ‚द्व‚य‚वृत्तित्वात् । \textbf{इह} प‚क्ष‚ल‚क्ष‚णे \textbf{स्व‚यंश‚ब्दो} गृहीत इति शेषः । तृतीया‚{\tiny $_{lb}$}‚प्र‚तिपाद्योऽर्थोऽत्रैष‚ण‚क‚र्त्तृत्व‚म् । \textbf{आत्म‚ना} इष्ट इत्य‚त्र तृतीयायाः क‚र्त्त‚रि विधानात् ॥
	\pend% ending standard par
      ‚{\tiny $_{lb}$}‚

	  \pstart \leavevmode% starting standard par
	\textbf{अनेक‚वादिस‚म्भ‚वेऽपि} श‚ब्द‚ग‚ताकाश‚गुण‚त्वादिवादिभूय‚स्त्वेऽपि । वादित्वं च योग्य‚त‚या ।‚{\tiny $_{lb}$}‚ न तु त‚दा स्व‚प‚र‚प‚क्ष‚सिद्ध्य‚सिद्ध्य‚र्थ‚व‚च‚न‚ल‚क्ष‚ण‚वाद‚प्र‚णेतारः । \textbf{विशेष‚णं} व्य‚व‚च्छेद‚क\textbf{मेत‚द् य‚स्त‚दा‚{\tiny $_{lb}$}‚साध‚न‚माहे}ति व‚च‚न‚म् ।
	\pend% ending standard par
      ‚{\tiny $_{lb}$}‚

	  \pstart \leavevmode% starting standard par
	\textbf{य‚द्येव‚मि}ति प‚रः । अयं च निपात‚स‚मुदायोऽनिष्टापाद‚न‚प्रार‚म्भे व‚र्त्त‚ते । \textbf{इत्युक्त‚म}नेन‚{\tiny $_{lb}$}‚ वाक्येनेति शेषः ।
	\pend% ending standard par
      ‚{\tiny $_{lb}$}‚

	  \pstart \leavevmode% starting standard par
	उच्य‚तामेवं को दोष इत्याह--\textbf{एतेनेति । च}श‚ब्दोऽपिश‚ब्द‚स्यार्थे । शास्त्र‚कारेष्ट‚म‚पि‚{\tiny $_{lb}$}‚ वादीष्टं भ‚व‚ति । त‚त्कोऽतिश‚योऽनेन प्र‚तिपादित इति चोद‚यितुराश‚यः । \textbf{अनेने}ति सिद्धान्त‚वादी ।‚{\tiny $_{lb}$}‚ \textbf{अनेन} य‚स्त‚दा साध‚न‚माहेति विशेष‚णाव‚च्छिन्नेन स्व‚यंश‚ब्देन ।
	\pend% ending standard par
      ‚{\tiny $_{lb}$}‚

	  \pstart \leavevmode% starting standard par
	एत‚दुक्तं भ‚व‚ति । य‚च्छास्त्राभ्युप‚ग‚मेनापि वादी क्व‚चित्साध‚न‚म‚भिध‚त्ते, त‚च्छास्त्र‚कारेण‚{\tiny $_{lb}$}‚ त‚त्र याव‚दिष्टं ताव‚च्चेत्त‚स्य साध्य‚त्वेनेष्टं त‚देष्ट‚मित्येव कृतं स्यात्, न तु स्व‚य‚मिति । \textbf{नेत‚र‚{\tiny $_{lb}$}‚ इति} त‚च्छास्त्र‚कारेष्टोऽम्ब‚र‚गुण‚त्वादिरिति बुद्धिस्थ‚म् । \textbf{वादिन} इति आद्य‚स्यैव व्य‚क्तीक‚र‚ण‚म् ॥
	\pend% ending standard par
      ‚{\tiny $_{lb}$}‚

	  \pstart \leavevmode% starting standard par
	अथेत्याम‚न्त्र‚णे ।
	\pend% ending standard par
      ‚{\tiny $_{lb}$}‚‚{\tiny $_{lb}$}‚\textsuperscript{\textenglish{179/dm}}‚{\tiny $_{lb}$}‚
	  \bigskip
	  \begingroup
	
	  \bigskip
	  \begingroup
	

	  \pstart \leavevmode% starting standard par
	मिष्टः, स एव साध्यो नेत‚र इत्युक्तं भ‚व‚ति ॥ ४४ ॥
	\pend% ending standard par
      
	  \endgroup
	‚{\tiny $_{lb}$}‚ 

	  \pstart \leavevmode% starting standard par
	त‚च्छास्त्र‚कारेणेति । य‚च्छास्त्रं तेन वादिनाऽभ्युप‚ग‚तं त‚च्छास्त्र‚कारेण त‚स्मिन् साध्य‚{\tiny $_{lb}$}‚ध‚र्मिणि अनेक‚स्य \edtext{}{\lemma{स्य}\Bfootnote{अनेक‚ध‚र्म० \cite{dp-msC}}}ध‚र्म‚स्याभ्युप‚ग‚मे स‚ति अन्य‚ध‚र्म‚साध्य‚त्व‚स‚म्भ‚वः । त‚थाहि--शास्त्रं येनाभ्यु‚{\tiny $_{lb}$}‚प‚ग‚तं \edtext{}{\lemma{तं}\Bfootnote{त‚स्मिन् सिद्धो \cite{dp-edE}}}त‚त्सिद्धो ध‚र्मः स‚र्व एव तेन साध्य इत्य‚स्ति विप्र‚तिप‚त्तिः । अनेनापास्य‚ते । अनेक‚{\tiny $_{lb}$}‚ध‚र्माभ्युप‚ग‚मेऽपि स‚ति स एव साध्यो यो वादिन इष्टो नान्य इति ।
	\pend% ending standard par
       ‚{\tiny $_{lb}$}‚ 

	  \pstart \leavevmode% starting standard par
	न‚नु च शास्त्रान‚पेक्षं \edtext{}{\lemma{पेक्षं}\Bfootnote{तादात्म्य‚त‚दुत्प‚त्तेः--\cite{dp-msD-n}}}व‚स्तुब‚ल‚प्र‚वृत्तं लिङ्ग‚म् । अतोऽन‚पेक्ष‚णीय‚त्वान्न शास्त्रे‚{\tiny $_{lb}$}‚ स्थित्वा वादः क‚र्त्त‚व्यः । स‚त्य‚म् । आहोपुरुषिक‚या तु य‚द्य‚पि क्व‚चिच्छास्त्रे स्थित इति‚{\tiny $_{lb}$}‚ किञ्चिच्छास्त्र‚म‚भ्युप‚ग‚तः साध‚न‚माह, त‚थापि य एव त‚स्येष्टः स एव\edtext{}{\lemma{एव}\Bfootnote{स एव त‚स्य \cite{dp-msA} \cite{dp-edE}}} साध्य\edtext{}{\lemma{साध्य}\Bfootnote{साध्य‚त इति--\cite{dp-msC}}} इति ज्ञाप‚ना‚{\tiny $_{lb}$}‚येद‚मुक्त‚म् ॥
	\pend% ending standard par
       ‚{\tiny $_{lb}$}‚ 

	  \pstart \leavevmode% starting standard par
	इष्ट इतीष्ट‚श‚ब्द‚मुप‚क्षिप्य व्याच‚ष्टे--
	\pend% ending standard par
       ‚{\tiny $_{lb}$}‚ 
	  \bigskip
	  \begingroup
	

	  \pstart \leavevmode% starting standard par
	इष्ट इति य‚त्रार्थे विवादेन साध‚न‚मुप‚न्य‚स्तं त‚स्य सिद्धिमिच्छ‚ता‚{\tiny $_{lb}$}‚ सो \edtext{}{\lemma{सो}\Bfootnote{सोऽर्थोऽनु० \cite{dp-msC}}}ऽनुक्तोऽपि व‚च‚नेन साध्यः ॥ ४५ ॥
	\pend% ending standard par
      
	  \endgroup
	‚{\tiny $_{lb}$}‚ 

	  \pstart \leavevmode% starting standard par
	\hphantom{.}य‚त्रार्थ आत्म‚नि विरुद्धो वादः प्र‚क्रान्तः--नास्ति आत्मा--इत्यात्म‚प्र‚तिषेध‚वाद आत्म‚{\tiny $_{lb}$}‚स‚त्तावाद‚विरुद्धः, विधिप्र‚तिषेध‚योर्विरोधात् । तेन विवादेन हेतुना साध‚न‚मुप‚न्य‚स्तं त‚स्या‚{\tiny $_{lb}$}‚त्मार्थ‚स्य सिद्धिं निश्च‚य‚म् इच्छ‚ता वादिना सोऽर्थः साध्य इत्युक्तं भ‚व‚ति इष्ट‚श‚ब्देन । य‚त्‚{\tiny $_{lb}$}‚ त‚द् इत्युक्तं भ‚व‚ति इति ग्र‚ह‚ण‚म‚न्ते त‚दिहापेक्ष्य वाक्यं \edtext{}{\lemma{वाक्यं}\Bfootnote{वाक्यं प‚रिस‚मा० \cite{dp-msA} \cite{dp-msB} \cite{dp-msC} \cite{dp-msD} \cite{dp-edP} \cite{dp-edH} \cite{dp-edE} \cite{dp-edN}}}स‚माप‚यित‚व्य‚म् ।
	\pend% ending standard par
       ‚{\tiny $_{lb}$}‚ 

	  \pstart \leavevmode% starting standard par
	य‚द्य‚पि प‚रार्थानुमान उक्त एव साध्यो युक्तः, अनुक्तोपि \edtext{}{\lemma{अनुक्तोपि}\Bfootnote{तुश‚ब्द‚स्त‚थापीत्य‚र्थे--\cite{dp-msD-n}}}तु व‚च‚नेन साध्यः, साम‚र्थ्यो‚{\tiny $_{lb}$}‚क्त‚त्वात् त‚स्य ॥
	\pend% ending standard par
      
	  \endgroup
	‚{\tiny $_{lb}$}‚

	  \pstart \leavevmode% starting standard par
	इत्य‚स्ति विप्र‚तिप‚त्तिर्न्याय‚विरुद्धा प्र‚तिप‚त्तिः केषाञ्चित् । \textbf{अनेना}त्म‚विशेष‚णेनापास्य‚ते ।‚{\tiny $_{lb}$}‚ \textbf{अनेके}त्यादिनोप‚संह‚र‚ति । \textbf{अनेक‚ध‚र्माभ्युप‚ग‚मेऽपि} शास्त्र‚कार‚स्य त‚त्र ध‚र्मिण्य‚नेक‚ध‚र्मोप‚ग‚मे‚{\tiny $_{lb}$}‚ स‚त्य‚पि । \textbf{वादिन} इत्यात्म‚न इति ष‚ष्ठ्य‚न्त‚स्यार्थे वृत्तं स्व‚यंश‚ब्द‚मु\add{पा}दाय ।
	\pend% ending standard par
      ‚{\tiny $_{lb}$}‚

	  \pstart \leavevmode% starting standard par
	\textbf{आहोपुरुषिक‚याऽभ्युप‚ग‚त} इति क‚र्त्त‚रीयं निष्ठा ॥
	\pend% ending standard par
      ‚{\tiny $_{lb}$}‚

	  \pstart \leavevmode% starting standard par
	\textbf{विरुद्धः} प‚र‚स्य चाभिप्रेत‚विप‚रीतार्थोप‚स्थाप‚को वादः स्व‚प‚र‚प‚क्ष‚योः सिद्ध्य‚सिद्ध्य‚र्थं‚{\tiny $_{lb}$}‚ व‚च‚न‚म् । \textbf{प्र‚क्रान्तः} प्र‚वृत्तः । \textbf{इत्युक्तं भ‚व‚ती}ति नात्र श्रूय‚ते त‚त्क‚थ‚मेवं व्याख्याय‚त इत्याह—‚{\tiny $_{lb}$}‚य‚त्त‚दिति । लोकोक्तिश्चैषा । इहेष्ट‚प‚द‚विव‚र‚णे । \textbf{स‚माप‚यित‚व्यं} स‚ङ्ग‚तार्थं क‚र्त्त‚व्य‚म् ।
	\pend% ending standard par
      ‚{\tiny $_{lb}$}‚‚{\tiny $_{lb}$}‚\textsuperscript{\textenglish{180/dm}}‚{\tiny $_{lb}$}‚
	  \bigskip
	  \begingroup
	

	  \pstart \leavevmode% starting standard par
	कुत एत‚दित्याह--
	\pend% ending standard par
       ‚{\tiny $_{lb}$}‚ 
	  \bigskip
	  \begingroup
	

	  \pstart \leavevmode% starting standard par
	त‚द‚धिक‚र‚ण‚त्वाद्विवाद‚स्य ॥ ४६ ॥
	\pend% ending standard par
      
	  \endgroup
	‚{\tiny $_{lb}$}‚ 

	  \pstart \leavevmode% starting standard par
	\edtext{\textsuperscript{*}}{\lemma{*}\Bfootnote{त‚दित्यादि त‚दिति \cite{dp-msA} \cite{dp-msB} \cite{dp-msC} \cite{dp-msD} \cite{dp-edP} \cite{dp-edH} \cite{dp-edE} \cite{dp-edN}}}त‚दिति सोऽर्थोऽधिक‚र‚ण‚म् आश्र‚यो य‚स्य स त‚द‚धिक‚र‚णो विवादः । त‚स्य‚{\tiny $_{lb}$}‚ भाव‚स्त‚त्त्व‚म् । त‚स्मादिति ।
	\pend% ending standard par
       ‚{\tiny $_{lb}$}‚ 

	  \pstart \leavevmode% starting standard par
	एत‚दुक्तं भ‚व‚ति--य‚स्माद्विवादं निराक‚र्त्तुमिच्छ‚ता वादिना साध‚न‚मुप‚न्य‚स्तं त‚स्माद्‚{\tiny $_{lb}$}‚ य‚द् अधिक‚र‚णं विवाद‚स्य त‚देव साध्य‚म् । य‚तो विरुद्धं वाद‚म‚प‚नेतुं साध‚न‚मुप‚न्य‚स्तं त‚च्चेत् न‚{\tiny $_{lb}$}‚ साध्यं किमिदानीं \edtext{}{\lemma{किमिदानीं}\Bfootnote{ज‚ग‚ति निय‚त‚म्--\cite{dp-msB} \cite{dp-msC} \cite{dp-msD} \cite{dp-edP} \cite{dp-edH} \cite{dp-edE} \cite{dp-edN}}}जातिनिय‚तं किंचित् साध्यं स्यादिति ॥
	\pend% ending standard par
       ‚{\tiny $_{lb}$}‚ 

	  \pstart \leavevmode% starting standard par
	अनुक्त‚म‚पि प‚रार्थानुमाने साध्य‚मिष्ट‚म् । \edtext{\textsuperscript{*}}{\lemma{*}\Bfootnote{०मिष्ट‚मुदाह‚र‚ति--\cite{dp-msB} साध्यं दृष्ट‚मुदाह० \cite{dp-msC} \cite{dp-msD}}}त‚दुदाह‚र‚ति--
	\pend% ending standard par
       ‚{\tiny $_{lb}$}‚ 
	  \bigskip
	  \begingroup
	

	  \pstart \leavevmode% starting standard par
	य‚था प‚रार्थाश्च‚क्षुराद‚यः संघात‚त्वाच्छ‚य‚नास‚नाद्य‚ङ्ग‚व‚दिति । \edtext{\textsuperscript{*}}{\lemma{*}\Bfootnote{०दिति । आत्मार्था इत्य‚नुत्था [[क्ता]] प्यात्मार्थ‚ता साध्य‚ते तेन--\cite{dp-msC}}}अत्रात्मार्था‚{\tiny $_{lb}$}‚ इत्य‚नुक्ताव‚प्यात्मार्थ‚ता साध्या । \edtext{\textsuperscript{*}}{\lemma{*}\Bfootnote{साध्या । अनेन \cite{dp-msB} \cite{dp-edP} \cite{dp-edH} \cite{dp-edE}}}तेन नोक्त‚मात्र‚मेव साध्य‚म्--इत्युक्तं‚{\tiny $_{lb}$}‚ भ‚व‚ति ॥ ४७ ॥
	\pend% ending standard par
      
	  \endgroup
	‚{\tiny $_{lb}$}‚ 

	  \pstart \leavevmode% starting standard par
	प‚रार्था इति । च‚क्षुरादिर्येषां श्रोत्रादीनां ते\edtext{}{\lemma{ते}\Bfootnote{ते नास्ति \cite{dp-msB}}} च‚क्षुराद‚य इति ध‚र्मी । प‚र‚स्मायिमे‚{\tiny $_{lb}$}‚ प‚रार्था इति साध्यं पारार्थ्य‚म् । स‚ङ्घात‚त्वादिति हेतुः । व्याप्तिविष‚य‚प्र‚द‚र्श‚नं \edtext{}{\lemma{नं}\Bfootnote{च नास्ति--\cite{dp-msA} \cite{dp-msB} \cite{dp-msD} \cite{dp-edP} \cite{dp-edH} \cite{dp-edE} \cite{dp-edN}}}च श‚य‚नास‚ना‚{\tiny $_{lb}$}‚द्य‚ङ्ग‚व‚दिति । श‚य‚न‚मास‚नं च ते आदी\edtext{}{\lemma{आदी}\Bfootnote{आदिर्य‚स्य \cite{dp-msD}}} य‚स्य त‚च्छ‚य‚नास‚नादि पुरुषोप‚भोगाङ्गं संघात‚रूप‚म् ।‚{\tiny $_{lb}$}‚ त‚द्व‚द‚त्र \edtext{}{\lemma{त्र}\Bfootnote{त‚द्व‚द‚त्र य‚त्प्र‚माणे--\cite{dp-msA} \cite{dp-msB} \cite{dp-edP} \cite{dp-edH}}}प्र‚माणे य‚द‚प्यात्मार्थाश्च‚क्षुराद‚य इत्यात्मार्थ‚ता नोक्ता \edtext{}{\lemma{नोक्ता}\Bfootnote{नोक्ताप्या० \cite{dp-msB} अनुक्ताप्या--\cite{dp-msA} \cite{dp-msC} \cite{dp-msD} \cite{dp-edP} \cite{dp-edH} \cite{dp-edE} \cite{dp-edN}}}अनुक्ताव‚प्यात्मार्थ‚ता साध्या ।
	\pend% ending standard par
      
	  \endgroup
	‚{\tiny $_{lb}$}‚

	  \pstart \leavevmode% starting standard par
	\textbf{अनुक्तोऽपि तु व‚च‚नेन । व‚च‚नेन} साक्षाद‚भिधाव्यापार‚विष‚य‚त्व‚म‚नापादितोऽपि ।‚{\tiny $_{lb}$}‚ \textbf{तुर्विशेष‚दीप‚ने । साध्यः} साध्य एव । कुत इत्याह--\textbf{साम‚र्थ्योक्त‚त्वात्त‚स्य} बुद्धिस्थ‚स्यात्मादेः ।
	\pend% ending standard par
      ‚{\tiny $_{lb}$}‚

	  \pstart \leavevmode% starting standard par
	एत‚दुक्तं भ‚व‚ति--प‚रार्थानुमाने उक्तोऽर्थः साध्यः । उक्त‚श्च प्र‚काशित उच्य‚ते । प्र‚का‚{\tiny $_{lb}$}‚श्य‚मान‚ता च साक्षाद‚भिधाव्यापार‚विष‚य‚त‚या च साम‚र्थ्य‚ग‚म्य‚त‚या च । उक्त‚म \edtext{}{\lemma{म}\Bfootnote{उक्त‚ता}} तु‚{\tiny $_{lb}$}‚ प्र‚काशित‚ताख्या द्व‚योर‚प्य‚विशिष्टेति ॥
	\pend% ending standard par
      ‚{\tiny $_{lb}$}‚

	  \pstart \leavevmode% starting standard par
	\textbf{कुत एत‚दि}ति साम‚र्थ्योक्त‚त्व\textbf{मिति} हेतो\textbf{राह वार्त्तिक‚कारः ।}
	\pend% ending standard par
      ‚{\tiny $_{lb}$}‚‚{\tiny $_{lb}$}‚‚{\tiny $_{lb}$}‚‚{\tiny $_{lb}$}‚‚{\tiny $_{lb}$}‚\textsuperscript{\textenglish{181/dm}}‚{\tiny $_{lb}$}‚
	  \bigskip
	  \begingroup
	

	  \pstart \leavevmode% starting standard par
	त‚था हि--सांख्येनोक्त‚म्--अस्ति आत्मा । त‚द्विरुद्धं बौद्धेनोक्तं--नास्त्यात्मेति । त‚तः‚{\tiny $_{lb}$}‚ सांख्येन स्व‚वाद‚विरुद्धं बौद्ध‚वाद हेतूकृत्य विरुद्ध‚वाद‚निराक‚र‚णाय स्व‚वाद‚प्र‚तिष्ठाप‚नाय च‚{\tiny $_{lb}$}‚ साध‚न‚मुप‚न्य‚स्त‚म् । अतोऽनुक्ताव‚प्यात्मार्थ‚ता\edtext{}{\lemma{ता}\Bfootnote{०नुक्ताप्या० \cite{dp-msA} \cite{dp-msB} \cite{dp-msC} \cite{dp-msD} \cite{dp-edP} \cite{dp-edH} \cite{dp-edE} \cite{dp-edN}}} साध्या, \edtext{\textsuperscript{*}}{\lemma{*}\Bfootnote{य आत्म‚प्र‚तिषेध‚वादोऽधिक‚र‚णं य‚स्य--\cite{dp-msD-n}}}त‚द‚धिक‚र‚ण‚त्वाद् विवाद‚स्य ।\edtext{\textsuperscript{*}}{\lemma{*}\Bfootnote{श‚य‚नादिषु \cite{dp-msA}}} श‚य‚ना‚{\tiny $_{lb}$}‚स‚नादिषु हि पुरुषोप‚भोगाङ्गेष्वात्मार्थ‚त्वेनान्व‚यो\edtext{}{\lemma{यो}\Bfootnote{०र्थ‚त्बेन प्र‚सिद्धः \cite{dp-msA}}} न प्र‚सिद्धः । स‚ङ्घात‚त्व‚स्य \edtext{}{\lemma{स्य}\Bfootnote{प‚रार्थ‚मा० \cite{dp-msA} \cite{dp-msB} \cite{dp-edP} \cite{dp-edH} प‚रार्थ‚त्व‚मा० \cite{dp-edN}}}पारार्थ्य‚मात्रेण‚{\tiny $_{lb}$}‚ तु सिद्धः । त‚तः प‚रार्था इत्युक्त‚म् ।
	\pend% ending standard par
       ‚{\tiny $_{lb}$}‚ 

	  \pstart \leavevmode% starting standard par
	च‚क्षुराद‚य इत्य‚त्रादिग्र‚ह‚णाद्विज्ञान‚म‚पि\edtext{}{\lemma{पि}\Bfootnote{बौद्धानां म‚ते प‚र‚माणुरूपं ज्ञान‚म‚तः स‚ङ्घात‚रूप‚त्व‚म्--\cite{dp-msD-n}}} प‚रार्थं साध‚यितुमिष्ट‚म् । विज्ञानाच्च प‚र‚{\tiny $_{lb}$}‚ आत्मैव स्यात् ।
	\pend% ending standard par
       ‚{\tiny $_{lb}$}‚ 

	  \pstart \leavevmode% starting standard par
	\edtext{\textsuperscript{*}}{\lemma{*}\Bfootnote{प‚र‚स्य नास्ति--\cite{dp-msB}}}प‚र‚स्यार्थ‚कारि विज्ञानं सेत्स्य‚तीति साम‚र्थ्यादात्मार्थ‚त्वं सिध्य‚ति च‚क्षुरादीनामिति म‚त्वा‚{\tiny $_{lb}$}‚ प‚रार्थ‚ग्र‚ह‚णं कृत‚म् । तेनेष्ट\edtext{}{\lemma{तेनेष्ट}\Bfootnote{साध्य‚व‚च० \cite{dp-msA} \cite{dp-msB} \cite{dp-edP} \cite{dp-edH} \cite{dp-edE}, \cite{dp-edN}}} साध्य‚त्व‚व‚च‚नेन नोक्त‚मात्र‚म्, अपि तु प्र‚तिवादिनो विवादा‚{\tiny $_{lb}$}‚स्प‚द‚त्वाद् वादिनः\edtext{}{\lemma{वादिनः}\Bfootnote{वादिना \cite{dp-edE}}} साध‚यितुमिष्ट‚म्-उक्त‚म्, अनुक्तं वा प्र‚क‚र‚ण‚ग‚म्यं साध्य‚मित्युक्तं भ‚व‚ति ॥
	\pend% ending standard par
      
	  \endgroup
	‚{\tiny $_{lb}$}‚

	  \pstart \leavevmode% starting standard par
	य‚द्य‚पि त‚न्मूलो विवाद‚स्त‚थापि अभिधाव्यापार‚विश्राम‚भूमिरेवार्थः साध्यः । न चात्मा‚{\tiny $_{lb}$}‚दिर्विवादाधिक‚र‚ण‚म् । त‚थाभूत‚स्त‚त्क‚थं साध्य इत्याश‚ङ् क्याह--\textbf{एत‚दुक्तं} भ‚व‚तीति । \textbf{य‚दा}‚{\tiny $_{lb}$}‚त्मादि \textbf{अधिक‚र‚ण‚मा}स्प‚दं \textbf{विवाद‚स्य}--अस्तीदं नास्तीद‚मित्यात्म‚क‚स्य प‚र‚स्प‚र‚विरुद्ध‚स्य \textbf{वाद‚स्य} ।‚{\tiny $_{lb}$}‚ अत्रोप‚प\leavevmode\ledsidenote{\textenglish{65a/ms}} त्तिमाह--\textbf{य‚त} इति । विरुद्धं स्व‚प‚क्ष‚प्र‚त्य‚नीकं नास्तीद‚मिति वाद‚म् । \textbf{त‚द्}‚{\tiny $_{lb}$}‚ विरुद्ध‚वादाप‚नेतृहेतूप‚न्यास‚विष‚यं \textbf{चेद्} य‚दि \textbf{न साध्यं} न जिज्ञाप‚यिषित\textbf{मिदानी}मेत‚स्मिन्न‚भ्युप‚ग‚मे‚{\tiny $_{lb}$}‚ किं \textbf{जातिनिय‚तं} साध्य‚त्व‚जातिनिय‚तं जातिव‚शं \textbf{किञ्चिद्} व‚स्तु \textbf{साध्यं स्याद्} भ‚वितुम‚र्ह‚ति ।‚{\tiny $_{lb}$}‚ \textbf{क्षेपे किमः} प्र‚योगान्न किञ्चिदित्य‚र्थोऽव‚तिष्ठ‚ते ॥
	\pend% ending standard par
      ‚{\tiny $_{lb}$}‚

	  \pstart \leavevmode% starting standard par
	\textbf{अनुक्त‚म‚पि} साक्षाद‚भिधाव्यापाराविष‚योऽपि । \textbf{संहा \edtext{}{\lemma{संहा}\Bfootnote{धा}}} त‚त्वाद‚नेक‚रूप‚त्वात् । काल‚वि‚{\tiny $_{lb}$}‚शेषान‚पेक्षं चैत‚द् द्र‚ष्ट‚व्य‚म् । तेन क्र‚मेण युग‚प‚द् वा संह‚तं त‚दिति संह‚त‚रूपं विज्ञान‚म‚पि क्र‚मेणा‚{\tiny $_{lb}$}‚नेक‚रूप‚म‚त‚स्त‚त्रापि संहा \edtext{}{\lemma{संहा}\Bfootnote{धा}} त‚त्वं सिद्धिमिति त‚द‚प्यादिश‚ब्देन स‚ङ्गृहीतं ध‚र्मि क‚र्त्त‚व्य‚म् । अत‚{\tiny $_{lb}$}‚ एवान‚न्त‚र‚म् \textbf{अत्रादिश‚ब्दाद् विज्ञान‚म‚पि} इति व‚क्ष्य‚ते । अन्य‚था तु च‚क्षुरादीनां विज्ञान‚ल‚क्ष‚ण‚{\tiny $_{lb}$}‚प‚रार्थ‚तासिद्धावा \edtext{}{\lemma{तासिद्धावा}\Bfootnote{व}} पि नाभिप्रेतं \textbf{सांख्य‚स्य} सिद्ध्येत् । \textbf{अनुक्ताव‚प्य}न‚भिधानेऽपि त‚स्येत्य‚र्थात् ।‚{\tiny $_{lb}$}‚ \textbf{क्व‚चित्पुन‚र‚नुक्ताप्यात्मार्थ‚ते}ति पाठः । त‚त्रार्ज‚वेनै \add{व} स‚म्ब‚न्धः । साध‚नोप‚न्यासाश्र‚य‚त्वेन‚{\tiny $_{lb}$}‚ प्र‚कृत‚त्वात्त‚स्या इति भावः ।
	\pend% ending standard par
      ‚{\tiny $_{lb}$}‚

	  \pstart \leavevmode% starting standard par
	\textbf{त‚था ही}त्या\textbf{दिनै}त‚देव प्र‚तिपाद‚य‚ति । \textbf{हेतूकृत्य} निमित्तीकृत्य । \textbf{चः} फ‚ल‚स‚मुच्च‚ये ।‚{\tiny $_{lb}$}‚ \textbf{अनुक्ताव‚पी}ति पूर्व‚व‚द्वाच्य‚म् । य‚द्य‚यं त‚स्याभिप्राय‚स्त‚दाऽऽत्मार्था इत्येव किं न ब्र‚वीतीत्या‚{\tiny $_{lb}$}‚श‚ङ्क्य येनाभिप्रायेणैव‚म‚वादीत्तं द‚र्श‚यितुमाह--\textbf{श‚य‚नेति । हि}र्य‚स्माद‚र्थे ।
	\pend% ending standard par
      ‚{\tiny $_{lb}$}‚‚{\tiny $_{lb}$}‚\textsuperscript{\textenglish{182/dm}}‚{\tiny $_{lb}$}‚
	  \bigskip
	  \begingroup
	

	  \pstart \leavevmode% starting standard par
	अनिराकृत इति--एत‚ल्ल‚क्ष‚ण‚योगेऽपि यः साध‚यितुमिष्टोऽप्य‚र्थः‚{\tiny $_{lb}$}‚ प्र‚त्य‚क्षानुमान‚प्र‚तीतिस्व‚व‚च‚नैर्निराक्रिय‚ते, न स प‚क्ष\edtext{}{\lemma{क्ष}\Bfootnote{प‚क्षः द‚र्श० \cite{dp-msC}}} इति प्र‚द‚र्श‚नार्थ‚म् ॥ ४८ ॥
	\pend% ending standard par
      
	  \endgroup
	‚{\tiny $_{lb}$}‚
	  \bigskip
	  \begingroup
	

	  \pstart \leavevmode% starting standard par
	अनिराकृत इति व्याख्येय‚म् । एत‚दिति--अन‚न्त‚र‚प्र‚क्रान्तं य‚त् प‚क्ष‚ल‚क्ष‚ण‚मुक्तं‚{\tiny $_{lb}$}‚ साध्य‚त्वेनेष्टेत्यादि\edtext{}{\lemma{त्वेनेष्टेत्यादि}\Bfootnote{साध्य‚त्वेनेष्ट‚त्वादि \cite{dp-msC} \cite{dp-msD}}}--एत‚ल्ल‚क्ष‚णेन योगेऽप्य‚र्थो न प‚क्ष इति प्र‚द‚र्श‚नार्थ‚म् \edtext{}{\lemma{म्}\Bfootnote{प्र‚द‚र्श‚नाय \cite{dp-msA} \cite{dp-msC} \cite{dp-edP} \cite{dp-edH}}}प्र‚तिपाद‚नाय अनि‚{\tiny $_{lb}$}‚राकृत‚ग्र‚ह्णं कृत‚म् ।
	\pend% ending standard par
       ‚{\tiny $_{lb}$}‚ 

	  \pstart \leavevmode% starting standard par
	कीदृशोऽर्थो न प‚क्षः साध‚यितुमिष्टोऽपीत्याह--यः साध‚यितुमिष्टोऽर्थः--प्र‚त्य‚क्षं चानुमानं‚{\tiny $_{lb}$}‚ च प्र‚तीतिश्च स्व‚व‚च‚नं च--\edtext{\textsuperscript{*}}{\lemma{*}\Bfootnote{तैर्नि० \cite{dp-edE} \cite{dp-edN}}}एतैर्निराक्रिय‚ते--विप‚रीतः साध्य‚ते \edtext{}{\lemma{ते}\Bfootnote{साध्य‚ते स न प‚क्ष \cite{dp-msB} \cite{dp-msD}}}न स प‚क्ष इति ॥
	\pend% ending standard par
       ‚{\tiny $_{lb}$}‚ 
	  \bigskip
	  \begingroup
	

	  \pstart \leavevmode% starting standard par
	त‚त्र प्र‚त्य‚क्ष‚निराकृतो य‚था--अश्राव‚णः श‚ब्द इति ॥ ४९ ॥
	\pend% ending standard par
      
	  \endgroup
	‚{\tiny $_{lb}$}‚ 

	  \pstart \leavevmode% starting standard par
	त‚त्रेति । तेषु च‚तुर्षु प्र‚त्य‚क्षादिनिराकृतेषु\edtext{}{\lemma{क्षादिनिराकृतेषु}\Bfootnote{०कृतेषु निरा० \cite{dp-msB}}} प्र‚त्य‚क्ष‚निराकृतः कीदृशः ? य‚थेति ।‚{\tiny $_{lb}$}‚ य‚थाऽयं प्र‚त्य‚क्ष‚निराकृत‚स्त‚थाऽन्येऽपि द्र‚ट‚व्या इति य‚थाश‚ब्दार्थः ।
	\pend% ending standard par
       ‚{\tiny $_{lb}$}‚ 

	  \pstart \leavevmode% starting standard par
	श्र‚व‚णेन ग्राह्यः श्राव‚णः । न श्राव‚णोऽश्राव‚णः । श्रोत्रेण न ग्राह्य इति प्र‚तिज्ञार्थः ।‚{\tiny $_{lb}$}‚ श्रोत्राग्राह्य‚त्वं श‚ब्द‚स्य प्र‚त्य‚क्ष‚सिद्धेन श्रोत्र\edtext{}{\lemma{श्रोत्र}\Bfootnote{श्रोत्र‚ज्ञान‚ग्राह्य०--\cite{dp-msD-n}}} ग्राह्य‚त्वेन बाध्य‚ते ॥
	\pend% ending standard par
      
	  \endgroup
	‚{\tiny $_{lb}$}‚

	  \pstart \leavevmode% starting standard par
	न‚नु च‚क्षुरादीनां विज्ञान‚ल‚क्ष‚ण‚प‚रार्थ‚ता सेत्स्य‚ति । त‚त्क‚थ‚मात्मार्थ‚तासिद्धिरित्याह \textbf{च‚क्षुराद‚य‚{\tiny $_{lb}$}‚ इत्य‚त्रे}ति । त‚थापि कुत‚स्त‚त्सिद्धिरित्याह \textbf{विज्ञानाच्चे}ति । \textbf{चो} य‚स्मात् । अथ विज्ञान‚स्यापि‚{\tiny $_{lb}$}‚ ध‚र्मित्वे क‚थ‚मात्मार्थ‚तासिद्धिरित्याश‚ङ्क्य स्प‚ष्ट‚यितुमाह--\textbf{प‚र‚स्ये}ति । \textbf{साम‚र्थ्यादात्मार्थ‚त्वं‚{\tiny $_{lb}$}‚ सिद्ध्य‚ति च‚क्षुरादीनामित्येवं म‚त्वा प‚रार्थ‚ग्र‚ह‚णं कृतं सांख्येने}ति प्र‚स्तावात्, अध्याहारे वा,‚{\tiny $_{lb}$}‚ तेषां विज्ञानार्थ‚ताया अपि स‚म्भाव्य‚त्वात् । त‚त् सिद्धैकं साम‚र्थ्य‚मित्याश‚ङ्क्य \textbf{प‚र‚स्ये}ति योज्य‚म् ।
	\pend% ending standard par
      ‚{\tiny $_{lb}$}‚

	  \pstart \leavevmode% starting standard par
	\textbf{अर्थ‚कारि} प्र‚योज‚न‚कारि \textbf{विज्ञान‚म}पीति द्र‚ष्ट‚व्य‚म् । \textbf{सेत्स्य‚ती}ति ब्रुव‚तोऽयं भावः ।‚{\tiny $_{lb}$}‚ \textbf{आदि}श‚ब्दाद् विज्ञान‚स्यापि त‚थात्वे साध्ये विज्ञानं प‚रार्थ‚कारि सेत्स्य‚तीति । \textbf{इति}र्हेतौ । अनेन‚{\tiny $_{lb}$}‚ साम‚र्ध्यं च‚क्षुरादीनामात्मार्थ‚तासिद्धौ द‚र्शित‚म् ।
	\pend% ending standard par
      ‚{\tiny $_{lb}$}‚

	  \pstart \leavevmode% starting standard par
	\textbf{तेने}त्य‚स्य मूल‚स्य व्याख्यान\textbf{मिष्ट‚साध‚न \edtext{}{\lemma{न}\Bfootnote{साध्य‚त्व}} व‚च‚नेनेति वादिनः साध‚यितु}मिच्छ‚या‚{\tiny $_{lb}$}‚ विष‚यीकृत‚म् ।
	\pend% ending standard par
      ‚{\tiny $_{lb}$}‚

	  \pstart \leavevmode% starting standard par
	त‚च्च द्विविध‚मिति द‚र्श‚य‚न्नाह--\textbf{उक्त}मित्यादि । \textbf{उक्तं} साक्षाद‚भिधाविष‚यीकृत‚म् ।‚{\tiny $_{lb}$}‚ त‚द्विप‚रीत\textbf{म‚नुक्त‚म्} । क‚थं त‚र्हि त‚त्साध्य‚मित्याह--\textbf{प्र‚क‚र‚णे}ति । \textbf{प्र‚क‚र‚णे}न साध्योप‚न्यासा‚{\tiny $_{lb}$}‚श्र‚य‚त‚या प्र‚कृत‚त्वेन \textbf{ग‚म्यं} प्र‚काश्यं साध्यं साध्य‚मेवे\textbf{त्युक्तं भ‚व‚ति} ॥
	\pend% ending standard par
      ‚{\tiny $_{lb}$}‚\textsuperscript{\textenglish{183/dm}}‚{\tiny $_{lb}$}‚
	  \bigskip
	  \begingroup
	
	  \bigskip
	  \begingroup
	

	  \pstart \leavevmode% starting standard par
	अनुमान‚निराकृतो य‚था--नित्यः श‚ब्द इति ॥ ५० ॥
	\pend% ending standard par
      
	  \endgroup
	‚{\tiny $_{lb}$}‚ 

	  \pstart \leavevmode% starting standard par
	\edtext{\textsuperscript{*}}{\lemma{*}\Bfootnote{श्र‚व‚णेन्द्रिय प्र‚भ‚व‚ज्ञानेनेत्य‚र्थः--\cite{dp-msD-n}}}अनुमान‚निराकृतो य‚था\edtext{}{\lemma{था}\Bfootnote{य‚था नास्ति \cite{dp-msC} \cite{dp-msD} \cite{dp-edH} \cite{dp-edE} \cite{dp-edN}}} नित्यः श‚ब्द इति । श‚ब्द‚स्य प्र‚तिज्ञातं नित्य‚त्व‚म् अनित्य‚{\tiny $_{lb}$}‚त्वेनानुमान‚सिद्धेन निराक्रिय‚ते ॥
	\pend% ending standard par
       ‚{\tiny $_{lb}$}‚ 
	  \bigskip
	  \begingroup
	

	  \pstart \leavevmode% starting standard par
	प्र‚तीतिनिराकृतो य‚था--अच‚न्द्रः श‚शीति ॥ ५१ ॥
	\pend% ending standard par
      
	  \endgroup
	‚{\tiny $_{lb}$}‚ 

	  \pstart \leavevmode% starting standard par
	प्र‚तीत्या निराकृतः अच‚न्द्र इति च‚न्द्र‚श‚ब्द‚वाच्यो न भ‚व‚ति श‚शीति प्र‚तिज्ञातार्थः ।\edtext{\textsuperscript{*}}{\lemma{*}\Bfootnote{प्र‚तिज्ञार्थः--\cite{dp-msC} \cite{dp-msD}}}‚{\tiny $_{lb}$}‚ अयं च प्र‚तीत्या निराकृतः । प्र‚तीतोऽर्थ उच्य‚ते विक‚ल्प‚विज्ञान‚विष‚यः । प्र‚तीतिः प्र‚तीत‚त्वं
	\pend% ending standard par
      
	  \endgroup
	‚{\tiny $_{lb}$}‚

	  \pstart \leavevmode% starting standard par
	\textbf{एत‚ल्ल‚क्ष‚ण‚योगेऽपी}त्य‚स्यार्थ‚क‚थ‚न‚मिद\textbf{मेत‚ल्ल‚क्ष‚णेन योगेऽप्य‚र्थ} इति ।
	\pend% ending standard par
      ‚{\tiny $_{lb}$}‚

	  \pstart \leavevmode% starting standard par
	न‚नु च प्र‚त्य‚क्षादिभिर्निराक्रिय‚तेऽप‚सार्य‚त इति किल म‚त‚म् । न चाक्षा\edtext{}{\lemma{चाक्षा}\Bfootnote{र्था}}प‚सार‚णं‚{\tiny $_{lb}$}‚ प्र‚त्य‚क्षादिध‚र्मोऽपि तु व‚स्तुव्य‚व‚स्थाप‚न‚मित्याह--\textbf{विप‚रीत} इति ॥
	\pend% ending standard par
      ‚{\tiny $_{lb}$}‚

	  \pstart \leavevmode% starting standard par
	श्रूय‚तेऽनेनेति श्र\textbf{व‚णं} श्रोत्रेन्द्रियं तेन \textbf{ग्राह्यः} उप‚ल‚ब्धः । त‚ज्ज्ञान‚ग्राह्य‚त्वाच्च त‚द्ग्राह्य‚त्व‚{\tiny $_{lb}$}‚मुक्त‚म् । न‚ञा स‚मास‚माह--\textbf{नेति} । स‚म‚स्त‚स्यार्थ‚माह--श्रोत्रेणेति । अय‚म‚स्याश‚यः—‚{\tiny $_{lb}$}‚अश्राव‚णः श‚ब्दः--श्रोत्र‚जे\leavevmode\ledsidenote{\textenglish{65b/ms}}न ज्ञानेन नानुभूय‚त इति यः प्र‚तिजानीते त‚स्य सा प्र‚तिज्ञा‚{\tiny $_{lb}$}‚ श्र‚व‚णेन्द्रिय‚जेन प्र‚त्य‚क्षेण श‚ब्दाल‚म्ब‚नेन बाध्य‚ते श‚ब्द‚ग‚ताऽप्र‚तिभास‚न‚विप‚रीत‚स्य त‚त्प्र‚तिभास‚न‚स्य‚{\tiny $_{lb}$}‚ तेनोप‚स्थाप‚नादिति ।
	\pend% ending standard par
      ‚{\tiny $_{lb}$}‚

	  \pstart \leavevmode% starting standard par
	एतेन त‚न्निराकृत‚म्, \textbf{य‚दुद्द्योत‚क‚रेणो}क्त‚म्\edtext{}{\lemma{म्}\Bfootnote{न्याय‚वार्त्तिकं द्र‚ष्ट‚व्य‚म्--१. १. ३३. पृ० ११३.}}--अश्राव‚णः श‚ब्द इति प्र‚त्य‚क्ष‚विरुद्धोदाह‚र‚णं‚{\tiny $_{lb}$}‚ व‚र्ण‚य‚ति । न प्र‚त्य‚क्ष‚स्य विष‚यो ज्ञातो नानुमान‚स्य । किं कार‚ण‚म् ? इन्द्रिय‚वृत्तीनाम‚{\tiny $_{lb}$}‚तीन्द्रिय‚त्वात् । श्राव‚ण‚त्व‚ञ्चेन्द्रिय‚वृत्तिः । सा क‚थं प्र‚त्य‚क्षा भ‚विष्य‚ति ? त‚स्माद‚नुमान‚{\tiny $_{lb}$}‚विरुद्ध‚स्योदाह‚र‚ण‚मिद‚म् । अनुष्णो व‚ह्निः कृत‚क‚त्वादिति प्र‚त्य‚क्ष‚विरुद्ध‚स्येति ।
	\pend% ending standard par
      ‚{\tiny $_{lb}$}‚

	  \pstart \leavevmode% starting standard par
	न ह्म‚श्राव‚ण‚श‚ब्देन श‚ब्दाख्ये विष‚ये ज्ञानोत्प‚त्तौ श्रोत्रेन्द्रिय‚स्य वृत्तेर‚भावोऽभिम‚तो य‚स्य‚{\tiny $_{lb}$}‚ वादिन‚स्तं प्र‚ति श‚ब्द‚विष‚य‚श्र‚व‚णेन्द्रिय‚वृत्तेः प्र‚त्य‚क्ष‚विरुद्ध‚त्वात्प्र‚त्य‚क्ष‚निराकृत‚मिद‚माचार्येणोक्त‚म्,‚{\tiny $_{lb}$}‚ येनोच्य‚तेऽतीन्द्रियेन्द्रिय‚वृत्तिः क‚थं प्र‚त्य‚क्षा येनेद‚मुदाह‚र‚णं संग‚च्छेतेति । किन्त‚र्हि ? यः‚{\tiny $_{lb}$}‚ क‚श्चिद् व्यामोह‚माहात्म्याद् य‚देत‚च्छ्रोत्र‚ग्राह्यं रूप‚म‚द्व‚यं त‚न्नास्तीति प्र‚तीजानीते तं वादिविशेषं‚{\tiny $_{lb}$}‚ प्र‚तीति क‚थं न प्र‚त्य‚क्ष‚विरुद्धोदाह‚र‚ण‚मिद‚मिति ।
	\pend% ending standard par
      ‚{\tiny $_{lb}$}‚

	  \pstart \leavevmode% starting standard par
	एत‚दुक्तं भ‚व‚ति । श‚ब्दो नास्तीत्येव‚म‚पि ब्रुवाण‚स्यास्ति प्र‚त्य‚क्ष‚बाधा । केव‚लं विष‚यो‚{\tiny $_{lb}$}‚ निषेधोऽनेक‚मार्गः । श‚ब्दो नास्ति व्यापित‚या नित्य‚त‚येत्यादि । त‚त्रास‚ति श्राव‚ण‚श‚ब्दे स‚र्व‚स्यैव‚{\tiny $_{lb}$}‚ निषेधे प्र‚त्य‚क्ष‚बाधा श‚ङ्क्येत । श्राव‚ण‚श‚ब्देन तु श्रुतिमात्र‚ग्राह्य‚मेव य‚द्रूपं त‚न्निषेधे प्र‚त्य‚क्ष‚{\tiny $_{lb}$}‚बाधा न तु सामान्य‚ध‚र्म‚निषेध इति ख्याप्य‚त इति ॥
	\pend% ending standard par
      ‚{\tiny $_{lb}$}‚‚{\tiny $_{lb}$}‚\textsuperscript{\textenglish{184/dm}}‚{\tiny $_{lb}$}‚
	  \bigskip
	  \begingroup
	

	  \pstart \leavevmode% starting standard par
	विक‚ल्प‚विज्ञान‚विष‚य‚त्व‚मुच्य‚ते । तेन विक‚ल्प‚ज्ञानेन\edtext{}{\lemma{ज्ञानेन}\Bfootnote{विक‚ल्प‚विज्ञान‚विष‚य‚त्वेन प्र० \cite{dp-msB} \cite{dp-edP} \cite{dp-edH} \cite{dp-edE} विक‚ल्प‚ज्ञान‚विष‚य‚त्वेन \cite{dp-msC} \cite{dp-msD}‚{\tiny $_{lb}$}‚ विक‚ल्प‚विज्ञानेन \cite{dp-msA}}} प्र‚तीतिरूपेण श‚शिन‚श्च‚न्द्र‚श‚ब्द‚वाच्य‚त्वं‚{\tiny $_{lb}$}‚ सिद्ध‚मेव । त‚था हि--य‚द्विक‚ल्प\edtext{}{\lemma{ल्प}\Bfootnote{य‚द्विक‚ल्प‚ज्ञान० \cite{dp-msB} \cite{dp-edP} \cite{dp-edH} \cite{dp-edN} य‚ज्ज्ञान‚ग्राह्यं \cite{dp-msA}}} विज्ञान‚ग्राह्यं त‚च्छ‚ब्दाकार‚संस‚र्ग‚योग्य‚म् । य‚च्छ‚ब्दाकार‚संस‚र्ग‚{\tiny $_{lb}$}‚योग्यं त‚त् साङ्केतिकेन श‚ब्देन व‚क्तुं श‚क्य‚म् । अतः प्र‚तीतिरूपेण विक‚ल्प‚विज्ञान‚विष‚य‚त्वेन सिद्धं‚{\tiny $_{lb}$}‚ च‚न्द्र‚श‚ब्द‚वाच्य‚त्व‚म‚च‚न्द्र‚त्व‚स्य बाध‚क‚म् । स्व‚भाव‚हेतुश्च प्र‚तीतिः । य‚स्माद्विक‚ल्प‚विष‚य‚त्व‚{\tiny $_{lb}$}‚मात्रानुब‚न्धिनी साङ्केतिक‚श‚ब्द‚वाच्य‚ता, त‚तः स्व‚भाव‚हेतुसिद्धं च‚न्द्र‚श‚ब्द‚वाच्य‚त्व‚म‚वाच्य‚त्व‚स्य‚{\tiny $_{lb}$}‚ बाध‚कं द्र‚ष्ट‚व्य‚म् ॥
	\pend% ending standard par
      
	  \endgroup
	‚{\tiny $_{lb}$}‚

	  \pstart \leavevmode% starting standard par
	\textbf{प्र‚तीत्या} विक‚ल्प‚विज्ञान‚रूपेण विष‚यिणा विष‚य‚स्य निर्देशात् । एत‚देव द‚र्श‚य‚ति \textbf{प्र‚तीत}‚{\tiny $_{lb}$}‚ इति । \textbf{प्र‚तीत उच्य‚ते} व्य‚व‚ह्रिय‚ते, साक्षात्कृत‚स्याप्य‚जाताध्य‚व‚साय‚स्य त‚थाव्य‚व‚हारा‚{\tiny $_{lb}$}‚भावादिति भावः । विष‚यिणा विष‚य‚ग‚तो ध‚र्म उक्त इति स्फुट‚य‚ति \textbf{प्र‚तीतिः} प्र‚तीत‚त्व‚रूपेण‚{\tiny $_{lb}$}‚ \textbf{प्र‚तीत‚त्व‚मि}ति । तेन विक‚ल्प‚ज्ञानेनेति विक‚ल्प‚विज्ञान‚विष‚य‚त्वेनेति ज्ञेयं \textbf{प्र‚तीतिरूपेण ।‚{\tiny $_{lb}$}‚ त‚था} हीत्य‚नेनैत‚देवोप‚पाद‚य‚ति ।
	\pend% ending standard par
      ‚{\tiny $_{lb}$}‚

	  \pstart \leavevmode% starting standard par
	भ‚व‚तु श‚ब्दाकार‚संस‚र्ग‚योग्य‚त्व‚म् । त‚च्छ‚ब्द‚वाच्य‚ता तु क‚थ‚मित्याह--\textbf{य‚दिति । अत}‚{\tiny $_{lb}$}‚ इत्यादिना प्र‚तीतेर्बाध‚क‚त्वं द‚र्श‚य‚ति । \textbf{अच‚न्द्र‚त्व‚स्या}च‚न्द्र‚श‚ब्द‚वाच्य‚त्व‚स्य । किंसाध‚न‚सिद्ध‚मिदं‚{\tiny $_{lb}$}‚ च‚न्द्र‚श‚ब्द‚वाच्य‚त्व‚मित्याह--\textbf{स्व‚भावे}ति । स्व‚भाव‚हेतुल‚क्ष‚णं योज‚य‚न्नाह--\textbf{य‚स्मादि}ति । \textbf{त‚त}‚{\tiny $_{lb}$}‚ इत्यादिनोप‚संहारः ।
	\pend% ending standard par
      ‚{\tiny $_{lb}$}‚

	  \pstart \leavevmode% starting standard par
	एवं तु प्र‚योगो द्र‚ष्ट‚व्यः--योऽर्थो विक‚ल्प‚विज्ञान‚विष‚यः स साङ्केतिकेन श‚ब्देन व‚क्तुं‚{\tiny $_{lb}$}‚ श‚क्यः । य‚था शाखादिमान‚र्थो वृक्ष‚श‚ब्देन । विक‚ल्प‚विज्ञान‚विष‚य‚श्च श‚शीति ।
	\pend% ending standard par
      ‚{\tiny $_{lb}$}‚

	  \pstart \leavevmode% starting standard par
	इह केन‚चिच्छ‚ब्देन क‚स्य‚चिद‚भिधातुम‚श‚क्य‚त्वं वास्त‚वे प्र‚तिनिय‚तार्थ‚श‚ब्द‚स‚म्ब‚न्धे स‚ति‚{\tiny $_{lb}$}‚ स्यात् । स चान्य‚त्र प्र‚तिषिद्धः । पारिशेष्याज्ज्ञानात्म‚न्यारूढ‚स्यार्थ‚स्य श‚ब्द‚स‚म्ब‚न्धः क‚र्त्तुं‚{\tiny $_{lb}$}‚ क‚स्य श‚क्यो य‚स्तेन श‚ब्दाकारेण स‚ह नैक‚स्मिन् ज्ञानेन संसृज्य‚ते । अनिय‚तार्थं च विज्ञान‚मिति‚{\tiny $_{lb}$}‚ त‚दारूढोऽर्थ‚स्त‚द‚भिधानाकार‚संस‚र्ग‚योग्य एव । त‚स्मात्तेन श‚ब्देनाभिधातुम‚श‚क्य‚त्व‚म‚त‚दाकार‚{\tiny $_{lb}$}‚संस‚र्ग‚योग्य‚त्वेन व्याप्त‚म् । व्या\leavevmode\ledsidenote{\textenglish{66a/ms}}प‚क‚विरुद्धं च त‚द‚भिधानाकार‚संस‚र्ग‚योग्य‚त्व‚म् । तेन‚{\tiny $_{lb}$}‚ च विक‚ल्प‚विज्ञान‚विष‚य‚त्वं व्याप्त‚म् । त‚देवं विक‚ल्प‚विज्ञान‚विष‚य‚त्वं त‚द्व्याप‚क‚विरुद्ध‚व्याप्त‚{\tiny $_{lb}$}‚त्वात्तेनापि विरुद्ध्य‚ते । त‚त‚श्च त‚द्विरुद्धेन श‚क्य‚त्वेन व्याप्त इति स्व‚भाव‚हेतुः । त‚स्माद्‚{\tiny $_{lb}$}‚ विक‚ल्प‚विज्ञान‚विष‚य‚त्व‚मेव प्र‚तीतिः प्र‚सिद्धिर्व्य‚व‚हार‚श्चोच्य‚ते । अन‚या प्र‚तीत्या य‚त्साधितं‚{\tiny $_{lb}$}‚ श‚शिन‚श्च‚न्द्र‚श‚ब्द‚वाच्य‚त्वं त‚त्स्व‚विरुद्ध‚स्य त‚द‚न‚भिधेय‚त्व‚स्य बाध‚कं भ‚व‚ति । तेन प्र‚तीतेर्वि‚{\tiny $_{lb}$}‚क‚ल्प‚विज्ञान‚ल‚क्ष‚णाया जातो ध‚र्म इष्ट‚श‚ब्दाभिधेय‚त्व‚ल‚क्ष‚ण‚स्तेनान‚भिधेय‚त्व‚स्य बाध‚नात्प्र‚तीति‚{\tiny $_{lb}$}‚बाधोच्य‚त इति प‚र‚मार्थः ।
	\pend% ending standard par
      ‚{\tiny $_{lb}$}‚

	  \pstart \leavevmode% starting standard par
	य‚द्वा प्र‚तीतेस्त‚थारूपाया जात एवेष्ट‚श‚ब्दाभिधेय‚त्व‚ल‚क्ष‚णो ध‚र्मः प्र‚तीतिश‚ब्देनोक्तः,‚{\tiny $_{lb}$}‚ ‚{\tiny $_{lb}$}‚ \leavevmode\ledsidenote{\textenglish{185/dm}}‚{\tiny $_{lb}$}‚ 
	  
	स्व‚व‚च‚न‚निराकृतो य‚था-नानुमानं प्र‚माण‚म् ॥ ५२ ॥‚{\tiny $_{lb}$}‚ 
	  
	स्व‚व‚च‚नं प्र‚तिज्ञार्थ‚स्यात्मीयो वाच‚कः श‚ब्दः । तेन निराकृतः प्र‚तिज्ञार्थो न साध्यः ।‚{\tiny $_{lb}$}‚ य‚था नानुमानं प्र‚माण‚म्--इत्य‚त्र\edtext{}{\lemma{त्र}\Bfootnote{प्र‚माण‚म् । अत्र \cite{dp-msA} \cite{dp-msB} \cite{dp-msD} \cite{dp-edP} \cite{dp-edH} \cite{dp-edE} \cite{dp-edN}}} अनुमान‚स्य प्रामाण्य‚निषेधः प्र‚तिज्ञार्थः । स\edtext{}{\lemma{स}\Bfootnote{स चानुमा० \cite{dp-msC}}} नानुमानं‚{\tiny $_{lb}$}‚ प्र‚माण‚म्--इत्य‚नेन स्व‚वाच‚केन वाक्येन बाध्य‚ते । वाक्यं हि एत‚त् प्र‚युज्य‚मानं व‚क्तुः‚{\tiny $_{lb}$}‚ \edtext{\textsuperscript{*}}{\lemma{*}\Bfootnote{शाब्द‚स्य प्र‚त्य० \cite{dp-msA} \cite{dp-msB} \cite{dp-msC} \cite{dp-msD} \cite{dp-edP} \cite{dp-edH} \cite{dp-edE} \cite{dp-edN}}}शाब्द‚प्र‚त्य‚य‚स्य स‚द‚र्थ‚त्व‚मिष्टं सूच‚य‚ति । त‚थाहि-म‚द्वाक्याद् योऽर्थ‚स‚म्प्र‚त्य‚य‚स्त‚वोत्प‚द्य‚ते सोऽस‚त्यार्थ‚{\tiny $_{lb}$}‚ इति द‚र्श‚य‚न् वाक्य‚मेव नोच्चार‚येद्व‚क्ता, व‚च‚नार्थ‚श्चेद‚स‚त्यः प‚रेण ज्ञात‚व्यो व‚च‚न‚म‚पार्थ‚क‚म् ।‚{\tiny $_{lb}$}‚ योऽपि हि स‚र्वं मिथ्या ब्र‚वीमीति\edtext{}{\lemma{वीमीति}\Bfootnote{ब्र‚वीति व‚क्ति \cite{dp-msB}}} व‚क्ति सोऽप्य‚स्य वाक्य‚स्य स‚त्यार्थ‚त्व‚माद‚र्श‚य‚न्नेव वाक्य‚मु‚{\tiny $_{lb}$}‚च्चार‚य‚ति । य‚द्येत‚द्वाक्यं स‚त्यार्थ‚माद‚र्शित‚म्, एवं वाक्यान्त‚राण्यात्मीयान्य‚स‚त्यार्थानि\edtext{}{\lemma{त्यार्थानि}\Bfootnote{०न्य‚स‚त्यानि \cite{dp-msA}}} द‚र्शितानि‚{\tiny $_{lb}$}‚ भ‚व‚न्ति । \edtext{\textsuperscript{*}}{\lemma{*}\Bfootnote{एत‚देव इत्यादि भ‚व‚न्ति इत्य‚न्तं सूत्र‚त्वेन \cite{dp-edH} प्र‚तौ मुद्रित‚म् । किन्तु नास्त्येत‚त्‚{\tiny $_{lb}$}‚ सूत्र‚म्--सं०}}एत‚देव तु य‚द्य‚स‚त्यार्थ‚म्, अन्यान्य‚स‚त्यार्भानि न द‚र्शितानि भ‚व‚न्ति । त‚त‚श्च न‚{\tiny $_{lb}$}‚ किञ्चिदुच्चार‚ण‚स्य फ‚ल‚मिति नोच्चार‚येत् । त‚स्माद्वाक्य‚प्र‚भ‚वं वाक्यार्थाल‚म्ब‚नं विज्ञानं‚{\tiny $_{lb}$}‚ स‚त्यार्थं द‚र्श‚य‚न्नेव व‚क्ता \edtext{}{\lemma{क्ता}\Bfootnote{अनुमान‚प्रामाण्य‚निषेध‚ल‚क्ष‚ण‚म्--\cite{dp-msD-n}}}वाक्य‚मुच्चार‚य‚ति । त‚था\edtext{}{\lemma{था}\Bfootnote{य‚था \cite{dp-msA}}} च स‚ति बाह्य‚व‚स्तुनान्त‚रीय‚कं श‚ब्दं‚{\tiny $_{lb}$}‚ द‚र्श‚य‚ता श‚ब्द‚जं विज्ञानं स‚त्यार्थं द‚र्श‚यित‚व्य‚म् । त‚तो बाह्यार्थ‚कार्याच्छ‚ब्दादुत्प‚न्नं विज्ञानं‚{\tiny $_{lb}$}‚ स‚त्यार्थ‚माद‚र्श‚य‚ता\edtext{}{\lemma{ता}\Bfootnote{आद‚र्श‚यिता--\cite{dp-msA}}} कार्य‚लिङ्ग‚ज‚म‚नुमानं प्र‚माणं शाब्दं द‚र्शितं भ‚व‚ति ।‚{\tiny $_{lb}$}‚ प्र‚तीतिमात्रादेव सिद्धो योऽर्थः स इह बाध‚क इति द‚र्श‚यितुम् । एत‚च्च स्व‚भाव‚हेतुत्वं क‚ल्पित‚{\tiny $_{lb}$}‚मिष्ट‚म्, न वास्त‚व‚म्, श‚शिनो विक‚ल्प‚विज्ञान‚विष‚य‚त्व‚स्याऽऽध्य‚व‚सानिक‚त्वात् । अन्य‚थाऽनु‚{\tiny $_{lb}$}‚मान‚निराकृतान्नास्य पृथ‚ङ्निर्देशः स्यादिति ॥
	\pend% ending standard par
      ‚{\tiny $_{lb}$}‚

	  \pstart \leavevmode% starting standard par
	स्व‚श‚ब्देनात्म‚व‚च‚नेन प्र‚कृत‚त्वात्साध्य‚स्यात्मा गृह्य‚त इति अभिप्रायेणाह--प्र‚तिज्ञार्थ‚{\tiny $_{lb}$}‚स्यात्मीय इति । तेन \textbf{निराकृत} इति त‚दुप‚स्थापितेनानुमान‚प्रामाण्येन निराकृत इति द्र‚ष्ट‚व्य‚म् ।‚{\tiny $_{lb}$}‚ क‚थं निराक्रिय‚त इत्याह--\textbf{वाक्य‚मि}ति । य‚स्माद् \textbf{एत‚द् वाक्यं प्र‚युज्य‚मानं} स‚द् \textbf{व‚क्तुः शाब्द‚{\tiny $_{lb}$}‚प्र‚त्य‚य‚स्य} श‚ब्द‚प्र‚भ‚व‚स्य ज्ञान‚स्य \textbf{स‚द‚र्थ‚त्वं} स‚त्यार्थ‚त्व\textbf{मिष्टं सूच‚य‚ति} प्र‚काश‚य‚ति ।
	\pend% ending standard par
      ‚{\tiny $_{lb}$}‚

	  \pstart \leavevmode% starting standard par
	प्र‚युक्त‚म‚पि क‚थं त‚थाक‚रोतीत्याह--\textbf{त‚था ही}ति । \textbf{नोच्चार‚ये}दुच्चार‚यितुं नार्ह‚ति, अपार्थ‚{\tiny $_{lb}$}‚क‚त्वादिति बुद्धिस्थ‚म् । \textbf{व‚च‚ने}त्यादिना त्वेत‚देव व्य‚न‚क्ति-न स‚र्वं व‚च‚नं प्र‚युज्य‚मानं त‚थाकारि य‚था‚{\tiny $_{lb}$}‚ स‚र्वं मिथ्या ब्र‚वीमीति व‚च‚न‚मित्याह--\textbf{योऽपी}ति । हिर्य‚स्मात् । क‚थं त‚थाद‚र्श‚य‚न्नुच्चार‚य‚{\tiny $_{lb}$}‚तीत्याह--य‚द्येत‚दित्यादि । भ‚व‚तु वाक्य‚प्र‚भ‚वं वाक्यार्थाल‚म्ब‚नं ज्ञानं स‚त्यार्थ‚म्, त‚थापि क‚थ‚{\tiny $_{lb}$}‚म‚नुमान‚प्रामाण्यं व‚च‚नेनोप‚स्थाप्य‚ते येनानुमान‚प्रामाण्य‚प्र‚तिषेध‚स्त‚द्व‚च‚नोप‚स्थापितानुमान‚प्रामाण्येन‚{\tiny $_{lb}$}‚ ‚{\tiny $_{lb}$}‚ \leavevmode\ledsidenote{\textenglish{186/dm}}‚{\tiny $_{lb}$}‚ 
	  
	त‚स्मात् नानुमानं प्र‚माण‚म्--इति ब्रुव‚ता शाब्द‚स्य प्र‚त्य‚य‚स्यास‚न्न‚र्थो\edtext{}{\lemma{र्थो}\Bfootnote{०स्यास‚न् ग्राह्य उक्तो \cite{dp-msA} \cite{dp-edP} \cite{dp-edH} \cite{dp-edN}}} ग्राह्य उक्तः ।‚{\tiny $_{lb}$}‚ अस‚द‚र्थ‚त्व‚मेव ह्य‚प्रामाण्य‚मुच्य‚ते, नान्य‚त् । श‚ब्दोच्चार‚ण‚साम‚र्थ्याच्चार्थाविनाभावी स्व‚श‚ब्दो‚{\tiny $_{lb}$}‚ द‚र्शितः । त‚था च \edtext{}{\lemma{च}\Bfootnote{स‚न्नार्थो--\cite{dp-edE}}}स‚न्न‚र्थो द‚र्शितः । \edtext{\textsuperscript{*}}{\lemma{*}\Bfootnote{नानुमानं प्र‚माणं इत्य‚स्माच्छ‚ब्दाद्योऽर्थो बुध्य‚ते तेन ज‚नितो नानुमान इत्यादिकः‚{\tiny $_{lb}$}‚ श‚ब्दः न प्र‚त्येष्य‚ति नास्तिको व्य‚भिचारात्--\cite{dp-msD-n}}}त‚तः \edtext{}{\lemma{तः}\Bfootnote{अध्यारोपित०--\cite{dp-msD-n}}}क‚ल्पिताद‚र्थ‚कार्याच्छ‚ब्दाच्छाब्द\edtext{}{\lemma{ब्दाच्छाब्द}\Bfootnote{०च्छ‚ब्द‚प्र० \cite{dp-msA}}} प्र‚त्य‚यार्थ‚स्या‚{\tiny $_{lb}$}‚नुमितं स‚त्त्वं प्र‚तिज्ञाय‚मान‚म‚स‚त्त्वं प्र‚तिब‚ध्नाति । ‚{\tiny $_{lb}$}‚ 
	  
	त‚देवं स्व‚व‚च‚नानुमितेन स‚त्त्वेनास‚त्त्वं बाध्य‚मानं\edtext{}{\lemma{मानं}\Bfootnote{वाच्य‚मानं \cite{dp-msA} \cite{dp-msB} \cite{dp-edP} \cite{dp-edH} \cite{dp-edE} \cite{dp-edN}}} स्व‚व‚च‚नेन बाधित‚मुक्त‚मित्य‚य‚म‚त्रार्थः । ‚{\tiny $_{lb}$}‚ 
	  
	अन्ये त्वाहुः--अभिप्राय‚कार्याच्छ‚ब्दाज्जातं ज्ञान‚म‚भिप्रायाल‚म्ब‚न‚म् । स‚द‚र्थ‚मिच्छ‚तः‚{\tiny $_{lb}$}‚ श‚ब्द‚प्र‚योगः । तेनाप्रामाण्यं प्र‚तिज्ञातं बाध्य‚त इति ।‚{\tiny $_{lb}$}‚ निराक्रिय‚माणः स्व‚व‚च‚न‚निराकृतो भ‚व‚तीत्याश‚ङ्क्याह--त‚था च स‚तीति--श‚ब्द‚ज्ञान‚स्य‚{\tiny $_{lb}$}‚ स‚त्यार्थ‚त्व‚प्र‚तिपाद‚नाभिप्रायेण वाक्योच्चार‚ण‚प्र‚कारे स‚ति । \textbf{बाह्य‚व‚स्तुनान्त‚रीय‚कं} त‚द‚विना‚{\tiny $_{lb}$}‚भाविनं द‚र्श‚य‚ता स‚त्यार्थं द‚र्श‚यित‚व्यं द‚र्श‚यितुं युज्य‚ते श‚क्य‚त इत्य‚र्थः । य‚त एवं त‚त‚स्त‚स्मात् ।‚{\tiny $_{lb}$}‚ \textbf{बाह्यार्थ‚कार्यादि}ति स‚ति भेदे त‚दुत्प‚त्त्यैव नान्त‚रीय‚क‚त्व‚स‚म्भ‚वादिति भावः । \textbf{शाब्द‚मि}ति‚{\tiny $_{lb}$}‚ प्र‚कृत‚त्वाज्ज्ञानं \textbf{कार्य‚लिङ्ग‚जं} श‚ब्द‚रूप‚कार्य‚लिङ्ग‚ज‚मिति हेतुभावेन विशेष‚ण\textbf{म‚नुमानं प्र‚माणं द‚र्शितं‚{\tiny $_{lb}$}‚ प्र‚काशितं भ‚व‚ति} ।
	\pend% ending standard par
      ‚{\tiny $_{lb}$}‚

	  \pstart \leavevmode% starting standard par
	\textbf{त‚स्मादित्या}दिनोप‚संह‚र‚ति ।
	\pend% ending standard par
      ‚{\tiny $_{lb}$}‚

	  \pstart \leavevmode% starting standard par
	न‚न्व‚नुमान‚स्याऽस‚न् ग्राह्य इति युज्य‚ते व‚क्तुम् । त‚त्किं शाब्द‚स्य प्र‚त्य‚य‚स्येत्युव‚त‚मिति‚{\tiny $_{lb}$}‚ चेत् । नैष दोषः । शाब्द‚स्यापि प्र‚त्य‚य‚स्योक्तेन न्यायेनानुमान‚त्वात् । स‚र्वानुमान‚प्रामाण्य‚{\tiny $_{lb}$}‚प्र‚तिषेधे चास्यापि प्र‚तिषिद्ध‚त्वात् । अस्यैव चानुमान‚स्यानुमान‚प्रामाण्य‚प्र‚तिषेध‚ल‚क्ष‚ण‚प्र‚तिज्ञार्थ‚{\tiny $_{lb}$}‚बाध‚क‚त्वादुप‚न्यासो युक्त‚रूपः ।
	\pend% ending standard par
      ‚{\tiny $_{lb}$}‚

	  \pstart \leavevmode% starting standard par
	य‚द्येव‚म‚प्रामाण्य‚मुक्त‚म् त‚च्च बाध्य\leavevmode\ledsidenote{\textenglish{66b/ms}}त इति व‚क्तुं युज्य‚ते । त‚त्किमेव‚मुक्त‚मित्याह‚{\tiny $_{lb}$}‚\textbf{अस‚द‚र्थ‚त्व‚मि}ति । \textbf{ही}ति य‚स्मात् । आस्ताम‚स‚न् ग्राह्योऽभिहितः किम‚त इत्याह--\textbf{श‚ब्देति} ।‚{\tiny $_{lb}$}‚ \textbf{चो} हेताव‚व‚धार‚णे वा । श‚ब्दोऽप्य‚त‚स्त‚था द‚र्शित‚स्त‚थापि किमायात‚मित्याह--\textbf{त‚था चे}ति श‚ब्द‚स्य‚{\tiny $_{lb}$}‚ बाह्यार्थाविनाभाविप्र‚द‚र्श‚न‚प्र‚कारे स‚ति \textbf{स‚न्न‚र्थो}ऽध्य‚व‚सेयो \textbf{द‚र्शितः} । स‚द‚र्थ‚त्वं प्रामाण्य‚ल‚क्ष‚णं‚{\tiny $_{lb}$}‚ द‚र्शित‚मिति याव‚त् । त‚स्य स‚द‚र्थ‚त्वं प्र‚द‚र्श्य‚तां बाधा तु क‚थ‚मित्याह \textbf{त‚त} इति । \textbf{शाब्द‚प्र‚त्य‚य‚स्य}‚{\tiny $_{lb}$}‚ योऽर्थो ज्ञाप्योऽनुमान‚प्रामाण्य‚ल‚क्ष‚ण‚स्त‚स्यानेनैव श‚ब्द‚लिङ्गेनानुमाने\textbf{नानुमितं स‚त्त्वं} क‚र्त्तृ \textbf{प्र‚तिज्ञाय‚{\tiny $_{lb}$}‚मान}म‚स‚द‚र्थ‚म‚प्रामाण्य‚ल‚क्ष‚णं क‚र्म‚भूतं \textbf{प्र‚तिब‚ध्नाति} । निराक‚रोतीति व‚क्त‚व्ये \textbf{प्र‚तिब‚ध्नाती}ति‚{\tiny $_{lb}$}‚ \textbf{ब्रुवाणः} प‚र‚मार्थ‚तोऽस्यावास्त‚व‚त्वान्न बाधा किन्त्वेत‚दुप‚स्थापितेत‚र‚योः प‚र‚स्प‚र‚प्र‚तिब‚न्ध इति‚{\tiny $_{lb}$}‚ सूच‚य‚ति ।
	\pend% ending standard par
      ‚{\tiny $_{lb}$}‚‚{\tiny $_{lb}$}‚\textsuperscript{\textenglish{187/dm}}‚{\tiny $_{lb}$}‚
	  \bigskip
	  \begingroup
	

	  \pstart \leavevmode% starting standard par
	त‚द‚युक्त‚म् । य‚त इह प्र‚तीतेः स्व‚भाव‚हेतुत्व‚म्, स्व‚व‚च‚न‚स्य च कार्य‚हेतुत्वं क‚ल्पित‚मिष्ट‚म् ।‚{\tiny $_{lb}$}‚ न वास्त‚व‚म्\edtext{}{\lemma{म्}\Bfootnote{वास्त‚व‚मिति \cite{dp-msC} \cite{dp-msD}}} अभिप्राय‚कार्य‚त्वं च वास्त‚व‚मेव श‚ब्द‚स्य । त‚त‚स्त‚दिह न गृह्य‚ते ।
	\pend% ending standard par
       ‚{\tiny $_{lb}$}‚ 

	  \pstart \leavevmode% starting standard par
	किञ्च । य‚था--अनुमान‚म‚निच्छ‚न्\edtext{}{\lemma{न्}\Bfootnote{अग्न्य‚व्य० \cite{dp-msD}}} व‚ह्न्य‚व्य‚भिचारित्वं धूम‚स्य न प्र‚त्येति, त‚था‚{\tiny $_{lb}$}‚ श‚ब्द‚स्याप्य‚भिप्रायाव्य‚भिचारित्वं न प्र‚त्येष्य‚ति । बाह्य‚व‚स्तुप्र‚त्याय‚नाय च श‚ब्दः प्र‚युज्य‚ते ।‚{\tiny $_{lb}$}‚ त‚न्न श‚ब्द‚स्याभिप्रायाविनाभावित्वाभ्युप‚ग‚म‚पूर्व‚कः श‚ब्द‚प्र‚योगः । \edtext{\textsuperscript{*}}{\lemma{*}\Bfootnote{अपि च इति य‚स्माद‚र्थेऽव्य‚य‚म्--\cite{dp-msD-n} । अपि च इत्यार‚भ्य श‚ब्द‚प्र‚योगः इत्य‚न्तः‚{\tiny $_{lb}$}‚ पाठो नास्ति--\cite{dp-msB}}}अपि च, न स्वाभिप्राय‚{\tiny $_{lb}$}‚ निवेद‚नाय श‚ब्द उच्चार्य‚ते, अपि तु बाह्य \edtext{}{\lemma{बाह्य}\Bfootnote{बाह्य‚व‚स्तुस‚त्त्व \cite{dp-msA} \cite{dp-msB} \cite{dp-msC} \cite{dp-msD} \cite{dp-edP} \cite{dp-edH} \cite{dp-edE} \cite{dp-edN}}}स‚त्त्व‚प्र‚तिपाद‚नाय, त‚स्माद् बाह्य‚व‚स्त्व‚विनाभावि‚{\tiny $_{lb}$}‚त्वाभ्युप‚ग‚म‚पूर्व‚कः श‚ब्द‚प्र‚योगः । त‚तः पूर्व‚क‚मेव\edtext{}{\lemma{मेव}\Bfootnote{पूर्व‚मेव \cite{dp-edE}}} व्याख्यात‚म‚न‚व‚द्य‚म्\edtext{}{\lemma{म्}\Bfootnote{व्याख्यान‚म० \cite{dp-msA} \cite{dp-msB} \cite{dp-msC} \cite{dp-msD} \cite{dp-edP} \cite{dp-edH} \cite{dp-edE} \cite{dp-edN}}} ॥
	\pend% ending standard par
       ‚{\tiny $_{lb}$}‚ 
	  \bigskip
	  \begingroup
	

	  \pstart \leavevmode% starting standard par
	इति च‚त्वारः प‚क्षाभासा निराकृता भ‚व‚न्ति ॥ ५३ ॥
	\pend% ending standard par
      
	  \endgroup
	‚{\tiny $_{lb}$}‚ 

	  \pstart \leavevmode% starting standard par
	एवं\edtext{}{\lemma{एवं}\Bfootnote{एवं स‚ति--\cite{dp-msA} \cite{dp-edP} \cite{dp-edN}}}च स‚ति--अनिराकृत‚ग्र‚ह‚णेनान‚न्त‚रोक्ताश्च‚त्वारः प‚क्ष‚व‚दाभास‚न्त इति प‚क्षाभासा‚{\tiny $_{lb}$}‚ निर‚स्ता भ‚व‚न्ति ॥
	\pend% ending standard par
       ‚{\tiny $_{lb}$}‚ 

	  \pstart \leavevmode% starting standard par
	स‚म्प्र‚ति प‚क्ष‚ल‚क्ष‚ण‚प‚दानि येषां व्य‚व‚च्छेद‚कानि तेषां व्य‚व‚छेदेन यादृशः \edtext{}{\lemma{यादृशः}\Bfootnote{प‚क्षो--\cite{dp-msC}}}प‚क्षार्थो ल‚भ्य‚ते‚{\tiny $_{lb}$}‚ तं द‚र्श‚यितुं व्य‚व‚च्छेद्यान् संक्षिप्य द‚र्श‚य‚ति--
	\pend% ending standard par
      
	  \endgroup
	‚{\tiny $_{lb}$}‚

	  \pstart \leavevmode% starting standard par
	न‚न्व‚नुमान‚प्रामाण्य‚म‚स‚त्त्वं बाध‚ते न तु स्व‚व‚च‚नं त‚त्क‚थं स्व‚व‚च‚न‚निराकृतोदाह‚र‚ण‚मिद‚{\tiny $_{lb}$}‚मुक्त‚मित्याश‚ङ्क्य स्व‚व‚च‚न‚निराकृत इत्य‚त्र यादृशोऽर्थो विव‚क्षित‚स्तं द‚र्श‚य‚ति \textbf{त‚देव‚मि}ति ।‚{\tiny $_{lb}$}‚ अत्रेति स्व‚व‚च‚न‚निराकृते । य‚थानुमानं न प्र‚माण‚मित्य‚त्र ।
	\pend% ending standard par
      ‚{\tiny $_{lb}$}‚

	  \pstart \leavevmode% starting standard par
	स्वाभिप्रायेणैव व्याख्याय \textbf{अन्ये त्वि}त्यादिना पूर्व‚व्याख्यानं दूष‚यितुमाह । तुना स्व‚व्या‚{\tiny $_{lb}$}‚ख्यानाद् वैस‚दृश्य‚माह ।
	\pend% ending standard par
      ‚{\tiny $_{lb}$}‚

	  \pstart \leavevmode% starting standard par
	\textbf{न वास्त‚व‚मि}ति ब्रुव‚तो य‚दीदं वास्त‚व‚म‚नुमानं स्यात्त‚दानुमान‚निराकृतोदाह‚र‚णान्न‚{\tiny $_{lb}$}‚ पृथ‚गुच्येतेति । \textbf{किञ्चे}ति व‚क्त‚व्यान्त‚राभ्युच्च‚ये । \textbf{बाह्य‚व‚स्तुप्र‚त्याय‚नाय चे}ति \textbf{च}कारोऽपि‚{\tiny $_{lb}$}‚ व‚क्त‚व्यान्त‚र‚स‚मुच्च‚ये । य‚त एवं \textbf{त‚त‚स्मा}त् । स्व‚स्योच्चार‚यितुम‚भिप्रायो व‚क्तुकाम‚ता,‚{\tiny $_{lb}$}‚ त‚द‚व्य‚भिचारित्वा\textbf{भ्युप‚ग‚मः पूर्वो} य‚स्य त‚था न भ‚व‚ति । पूर्वं \textbf{बाह्य‚व‚स्तुप्र‚त्याय‚नाये}त्य‚नेन‚{\tiny $_{lb}$}‚ साम‚र्थ्यान्न स्वाभिप्राय‚प्र‚त्याय‚नायेत्युक्त‚म‚पि स्फुटं द‚र्श‚यितुं \textbf{त‚न्न श‚ब्द‚स्याभिप्रायाविनाभा‚{\tiny $_{lb}$}‚वित्वाभ्युप‚ग‚म‚पूर्व‚कः श‚ब्द‚प्र‚योग} इत्य‚नेन साम‚र्थ्याद् बाह्म‚व‚स्त्व‚विनाभावित्वाभ्युप‚ग‚म‚पूर्व‚क इति‚{\tiny $_{lb}$}‚ द‚र्शित‚म‚पि साक्षाद् द‚र्श‚यितु\textbf{म‚पि चेत्या}दिनोप‚क्र‚म‚ते । \textbf{अपि चे}ति पूर्वोक्त‚स्य व‚क्त‚व्यान्त‚र‚{\tiny $_{lb}$}‚ द्योत‚क‚स्य स्प‚ष्टीक‚र‚ण‚म् । य‚त्राप्य‚स‚त्त्व‚प्र‚तिपाद‚नाय श‚ब्दः प्र‚युज्य‚ते त‚त्रापि विव‚क्षित‚स‚त्त्व‚{\tiny $_{lb}$}‚विविक्त‚स्यान्य‚स‚त्त्व‚स्य प्र‚तिपाद‚नाद् \textbf{बाह्य‚स‚त्त्व‚प्र‚तिपाद‚नाये}त्युक्त‚म् । य‚द्वा \textbf{स‚त्त्व}ग्र‚ह‚ण‚स्योप‚{\tiny $_{lb}$}‚‚{\tiny $_{lb}$}‚ ‚{\tiny $_{lb}$}‚ \leavevmode\ledsidenote{\textenglish{188/dm}}‚{\tiny $_{lb}$}‚ 
	  
	\edtext{\textsuperscript{*}}{\lemma{*}\Bfootnote{एवं नास्ति \cite{dp-msB} \cite{dp-edP} \cite{dp-edH}}}एवं सिद्ध‚स्य, असिद्ध‚स्यापि साध‚न‚त्वेनाभिम‚त‚स्य, स्व‚यं वादिना त‚दा‚{\tiny $_{lb}$}‚ साध‚यितुम‚निष्ट‚स्य, उक्त‚मात्र‚स्य निराकृत‚स्य च विप‚र्य‚येण साध्यः । तेनैव‚{\tiny $_{lb}$}‚ स्व‚रूपेणाभिम‚तो वादिन इष्टोऽनिराकृतः प‚क्ष इति प‚क्ष‚ल‚क्ष‚ण‚म‚न‚व‚द्यं\edtext{}{\lemma{द्यं}\Bfootnote{ल‚क्ष‚ण‚म‚व‚द्यं \cite{dp-msB} \cite{dp-edP}}} द‚र्शितं‚{\tiny $_{lb}$}‚ भ‚व‚ति ॥ ५४ ॥‚{\tiny $_{lb}$}‚ 
	  
	एव‚म्--इत्य‚न‚न्त‚रोक्त‚क्र‚मेण\edtext{}{\lemma{मेण}\Bfootnote{०रोक्तेन क्र‚मेण \cite{dp-msB} \cite{dp-msC} \cite{dp-msD}}} । सिद्ध‚स्य विप‚र्य‚येण विप‚रीत‚त्वेन हेतुना साध्यो‚{\tiny $_{lb}$}‚ द्र‚ष्ट‚व्यः । य‚स्माद‚र्थात् सिद्धोऽर्थो विप‚रीतः, स साध्य इत्य‚र्थः । सिद्ध‚श्च विप‚रीतोऽसिद्ध‚स्य ।‚{\tiny $_{lb}$}‚ त‚स्माद् असिद्धः साध्यः । असिद्धोऽपि न स‚र्वोऽपि तु साध‚न‚त्वेनोक्त‚स्यासिद्ध‚स्यापि विप‚र्य‚येण ।‚{\tiny $_{lb}$}‚ स्व‚र्य वादिना साध‚यितुम‚निष्ट‚स्य असिद्ध‚स्य विप‚र्य‚येण । त‚था उक्त‚मात्र‚स्य असिद्ध‚स्यापि‚{\tiny $_{lb}$}‚ विप‚र्य‚येण । त‚था निराकृत‚स्यासिद्ध‚स्यापि विप‚र्य‚येण साध्यः । ‚{\tiny $_{lb}$}‚ 
	  
	य‚श्चायं प‚ञ्च‚भिर्व्य‚व‚च्छेद्यै र‚हितोऽर्थोऽसिद्धो\edtext{}{\lemma{हितोऽर्थोऽसिद्धो}\Bfootnote{असिद्धो इत्यादिप‚दान‚न्त‚रं \cite{dp-msA} प्र‚तौ \cite{dp-msD} प्र‚तौ च संख्याङ्काः द‚त्ता व‚र्त्त‚न्ते--सं०}}ऽसाध‚नं वादिनः स्व‚यं साध‚यितुमिष्ट‚{\tiny $_{lb}$}‚ उक्तोऽनुक्तो वा प्र‚माणैर‚निराकृतः साध्यः, स एवासौ स्व‚रूपेणैव स्व‚य‚मिष्टोऽनिराकृत एतैः‚{\tiny $_{lb}$}‚ प‚दैरुक्त इत्य‚र्थः । य‚श्चायं साध्यः स प‚क्ष \edtext{}{\lemma{क्ष}\Bfootnote{इति नास्ति \cite{dp-msA} \cite{dp-msB} \cite{dp-msD} \cite{dp-edP} \cite{dp-edH} \cite{dp-edE} \cite{dp-edN}}}इति उच्य‚ते । इतिश‚ब्द एव‚म‚र्थे । एवं प‚क्ष‚{\tiny $_{lb}$}‚ल‚क्ष‚ण‚म‚न‚व‚द्य‚मिति । अविद्य‚मान‚म‚व‚द्यं दोषो य‚स्य त‚द‚न‚व‚द्य‚म् । द‚र्शितं क‚थित‚म् । ‚{\tiny $_{lb}$}‚ 
	  
	त्रिरूप‚लिङ्गाख्यानं प‚रिस‚माप‚य्य\edtext{}{\lemma{य्य}\Bfootnote{स‚माप्य \cite{dp-edE}}} प्र‚स‚ङ्गाग‚तं च प‚क्ष‚ल‚क्ष‚ण‚म‚भिधाय हेत्वाभासान्‚{\tiny $_{lb}$}‚ व‚क्तुकाम‚स्तेषां प्र‚स्तावं र‚च‚य‚ति त्रिरूपेत्यादिना-- ‚{\tiny $_{lb}$}‚ 
	  
	त्रिरूप‚लिङ्गाख्यानं प‚रार्थानुमान‚मित्युक्त‚म् । त‚त्र त्र‚याणां रूपाणामेक‚{\tiny $_{lb}$}‚स्यापि रूप‚स्यानुक्तौ साध‚नाभासः ॥ ५५ ॥‚{\tiny $_{lb}$}‚ 
	  
	एत‚दुक्तं भ‚व‚ति--त्रिरूप‚लिङ्गं\edtext{}{\lemma{लिङ्गं}\Bfootnote{लिङ्गाख्यानं व‚क्तु० \cite{dp-msA} \cite{dp-msB} \cite{dp-edP} \cite{dp-edH} \cite{dp-edE} \cite{dp-edN}}} व‚क्तुकामेन स्फुंट त‚द्व‚क्त‚व्य‚म् । एवं च त‚त् स्फुट‚मुक्तं‚{\tiny $_{lb}$}‚ ल‚क्ष‚ण‚त्वाद् बाह्य‚स‚त्त्व‚प्र‚तिपाद‚नायेत्य‚पि द्र‚ष्ट‚व्य‚म् । स‚र्व‚था न स्वाभिप्राय‚स्य स‚त्त्व‚म‚स‚त्त्वं वा‚{\tiny $_{lb}$}‚ प्र‚तिपाद‚यितुं श‚ब्द‚प्र‚योग इत्य‚ने \add{न} हेतुम् । य‚त एवं \textbf{त‚त}स्त‚स्मात्\textbf{पूर्व‚क‚मेव} य‚न्म‚या व्याख्यात‚{\tiny $_{lb}$}‚\textbf{म‚न‚व‚द्य‚म}प‚ग‚त‚दोष‚म् ॥
	\pend% ending standard par
      ‚{\tiny $_{lb}$}‚

	  \pstart \leavevmode% starting standard par
	\textbf{इति च‚त्वारः प‚क्षाभासा निराकृता भ‚व‚न्ती}ति मूलं व्याच‚क्षाण आह--\textbf{एव‚मि}ति ।‚{\tiny $_{lb}$}‚ \textbf{निर‚स्ता भ‚व‚न्ति} प‚क्ष‚त्वेनेति प्र‚स्तावात् ॥
	\pend% ending standard par
      ‚{\tiny $_{lb}$}‚

	  \pstart \leavevmode% starting standard par
	\textbf{प‚क्ष इत्युच्य‚ते} व्य‚क्तीक्रिय‚त इति व्युत्प‚त्त्येति भावः ॥
	\pend% ending standard par
      ‚{\tiny $_{lb}$}‚

	  \pstart \leavevmode% starting standard par
	न‚नु त्रिरूप‚लिङ्गाख्यानं प्र‚कृत‚मुक्त‚मेव । त‚त् किं हेत्वाभासाख्यान‚म‚प्र‚कृतं क्रिय‚त‚{\tiny $_{lb}$}‚ ‚{\tiny $_{lb}$}‚ \leavevmode\ledsidenote{\textenglish{189/dm}}‚{\tiny $_{lb}$}‚ 
	  
	भ‚व‚ति य‚दि त‚च्च, त‚त्प्र‚तिरूप‚कं\edtext{}{\lemma{कं}\Bfootnote{प्र‚तिरूपं बोध्य‚ते \cite{dp-msA} \cite{dp-msB} \cite{dp-edP} \cite{dp-edH} लिङ्गाभास‚म्--\cite{dp-msD-n}}} चोच्य‚ते । हेय‚ज्ञाने\edtext{}{\lemma{ज्ञाने}\Bfootnote{हेय‚ज्ञाते \cite{dp-edE}}} हि त‚द्विविक्त‚मुपादेयं सुज्ञातं भ‚व‚तीति ।‚{\tiny $_{lb}$}‚ त्रिरूप‚लिङ्गाख्यानं \edtext{}{\lemma{लिङ्गाख्यानं}\Bfootnote{प‚रार्थानु--\cite{dp-msA} \cite{dp-msB} \cite{dp-edP} \cite{dp-edH} \cite{dp-edE} \cite{dp-edN}}}प‚रार्थ‚म‚नुमान‚म् \cref{nb.3.1} इति प्राग् उक्त‚म् । ‚{\tiny $_{lb}$}‚ 
	  
	त‚त्रेति त‚स्मिन् स‚ति । त्रिरूप‚लिङ्गाख्याने प‚रार्थानुमाने\edtext{}{\lemma{रार्थानुमाने}\Bfootnote{प‚रार्थेऽनुमाने--\cite{dp-msC}}} स‚तीत्य‚र्थः । त्र‚याणां रूपाणां‚{\tiny $_{lb}$}‚ म‚ध्य एक‚स्याप्य‚नुक्तौ । अपिश‚ब्दाद् द्व‚योर‚पि । साध‚न‚स्य आभासः स‚दृशं साध‚न‚स्य, न‚{\tiny $_{lb}$}‚ साध‚न‚मित्य‚र्थः । त्र‚याणां रूपाणां न्यून‚ता नाम साध‚न‚दोषः ॥ ‚{\tiny $_{lb}$}‚ 
	  
	उक्ताव‚प्य‚सिद्धौ स‚न्देहे वा प्र‚तिपाद्य‚प्र‚तिपाद‚क‚योः ॥ ५६ ॥‚{\tiny $_{lb}$}‚ 
	  
	न केव‚ल‚म‚नुक्तावुक्ताव‚प्य‚सिद्धौ स‚न्देहे वा । क‚स्येत्याह--प्र‚तिपाद्य‚स्य प्र‚तिवादिनः,‚{\tiny $_{lb}$}‚ प्र‚तिपाद‚क‚स्य च वादिनो हेत्वाभासः ॥ ‚{\tiny $_{lb}$}‚ 
	  
	अथ \edtext{}{\lemma{अथ}\Bfootnote{क‚स्य रूप० \cite{dp-msA} \cite{dp-edP} \cite{dp-edH} \cite{dp-edE} \cite{dp-edN} हेत्वाभासोऽप्येक‚स्य रूप० \cite{dp-msC}}}क‚स्यैक‚स्यारूप‚स्यासिद्धौ स‚न्देहे\edtext{}{\lemma{न्देहे}\Bfootnote{स‚न्देहे वाक्यं सं० \cite{dp-msA}}} वा किंसंज्ञ‚को हेत्वाभास इत्याह--‚{\tiny $_{lb}$}‚ 
	  
	एक‚स्य रूप‚स्य ध‚र्मिस‚म्ब‚न्ध‚स्यासिद्धौ स‚न्देहे \edtext{}{\lemma{न्देहे}\Bfootnote{स‚न्देहे चासि० \cite{dp-msB} \cite{dp-msC} \cite{dp-msD} \cite{dp-edP} \cite{dp-edH} \cite{dp-edE}}}वाऽसिद्धो हेत्वाभासः ॥ ५७ ॥‚{\tiny $_{lb}$}‚ इत्याश‚ङ्क्याह--\textbf{एत‚दुक्तं भ‚व‚ती}ति । त्रिरूप‚लिङ्गाख्यान‚स्य स्फुटाभिधानार्थ‚त्वं चोप‚ल‚क्ष‚णं‚{\tiny $_{lb}$}‚ तेन विप्र‚तिप‚त्तिनिराक‚र‚णार्थं चेति द्र‚ष्ट‚व्य‚म्, स‚न्दिग्ध‚विप‚क्ष‚व्यावृत्तिक‚त्वादोष \edtext{}{\lemma{त्वादोष}\Bfootnote{दिषु}}‚{\tiny $_{lb}$}‚ विप्र‚तिप‚त्तिद‚र्श‚नात् ।
	\pend% ending standard par
      ‚{\tiny $_{lb}$}‚

	  \pstart \leavevmode% starting standard par
	स‚दृ\leavevmode\ledsidenote{\textenglish{67a/ms}} शं साध‚न‚स्येत्य‚र्थ‚क‚थ‚न‚मेत‚द् । व्युत्प‚त्तिस्त्वाभास‚न\textbf{माभासः} प्र‚तिभासः, साध‚न‚{\tiny $_{lb}$}‚स्येवाभासः प्र‚तिभासोऽस्येति त‚था । \textbf{न्यून‚ता} ऊन‚त्व‚म‚प‚रिपूर्ण‚ता । क‚स्येत्याकाङ्क्षायामाह—‚{\tiny $_{lb}$}‚\textbf{त्र‚याणामि}ति । य‚स्मादेक‚स्य द्व‚योर्वाऽनुक्तौ त्रीणि प‚रिपूर्णं प्र‚तिपादितानि न भ‚व‚न्ति त‚स्मा‚{\tiny $_{lb}$}‚त्त्र‚याणां न्यून‚तेत्य‚र्थः ।
	\pend% ending standard par
      ‚{\tiny $_{lb}$}‚

	  \pstart \leavevmode% starting standard par
	न‚नु हीनाङ्ग‚त्वं न साध‚न‚दोषः । विद्य‚मानेऽपि हि रूप‚त्र‚ये द्व‚योरेक‚स्य वा व‚क्त्राऽन‚{\tiny $_{lb}$}‚भिधाने स‚ति न्य‚न‚तायाः स‚म्भ‚वात् । त‚त्क‚थं व‚क्तृदोषः साध‚न‚दोष इत्युच्य‚त इति चेत् ।‚{\tiny $_{lb}$}‚ स‚त्य‚म्; केव‚लं नात्र साध‚न‚श‚ब्देन लिङ्ग‚म‚भिप्रेत‚म् । किं त‚र्हि ? त‚त्प्र‚तिपाद‚कं वाक्यं त‚स्य‚{\tiny $_{lb}$}‚ चाप‚रिपूर्ण‚ता दोषो भ‚व‚त्येव । व‚क्तृदोष‚स्तु निमित्त\add{म}प‚रिपूर्ण‚ताया इति किम‚व‚द्य‚म् ?
	\pend% ending standard par
      ‚{\tiny $_{lb}$}‚

	  \pstart \leavevmode% starting standard par
	एव‚मुप‚ल‚क्ष‚णार्थ‚त्वाद‚स्य य‚थाऽन्य‚त‚मेनापि रूपेण हीन‚त्वान्न्यून‚ता साध‚न‚दोष‚स्त‚था हेतू‚{\tiny $_{lb}$}‚दाह‚र‚णाद‚प्याधिक्यं साध‚न‚दोषो \textbf{वार्त्तिक}कार‚स्याभिप्रेतो द्र‚ष्ट‚व्यः, उभ‚य‚त्रापि त्व‚साध‚नाङ्ग‚{\tiny $_{lb}$}‚व‚च‚नाद् वादिना निग्र‚हो विव‚क्षित इत्य‚पीति ॥
	\pend% ending standard par
      ‚{\tiny $_{lb}$}‚

	  \pstart \leavevmode% starting standard par
	\textbf{प्र‚तिपाद्य‚स्ये}त्यादिना प्र‚तिपाद्य‚प्र‚तिपाद‚क‚श‚ब्द‚योरेवार्थं व्याच‚ष्टे । आचार्य‚स्य तु‚{\tiny $_{lb}$}‚ प्र‚तिपाद्य‚स्य प्र‚तिपाद‚क‚स्य च प्र‚तिपाद्य‚प्र‚तिपाद‚क‚योश्चेति व्य‚स्त‚स‚म‚स्त‚निर्देशोऽभिप्रेतः ।
	\pend% ending standard par
      ‚{\tiny $_{lb}$}‚‚{\tiny $_{lb}$}‚\textsuperscript{\textenglish{190/dm}}‚{\tiny $_{lb}$}‚
	  \bigskip
	  \begingroup
	

	  \pstart \leavevmode% starting standard par
	एक‚स्य रूप‚स्येति । ध‚र्मिणा स‚ह स‚म्ब‚न्धः ध‚र्मिस‚म्ब‚न्धः । ध‚र्मिणि स‚त्त्वं हेतोः ।‚{\tiny $_{lb}$}‚ त‚स्य असिद्धौ स‚न्देहे वा असिद्ध‚संज्ञाको हेत्वाभासः । असिद्ध‚त्वादेव च ध‚र्मिण्य‚प्र‚तिप‚त्तिहेतुः ।‚{\tiny $_{lb}$}‚ न साध्य‚स्य, न विरुद्ध‚स्य, न संश‚य‚स्य हेतुर‚पि त्व‚प्र‚तिप‚त्तिहेतुः । न क‚स्य‚चिद‚तः प्र‚तिप‚त्तिरिति‚{\tiny $_{lb}$}‚ कृत्वा । अयं चार्थोऽसिद्ध‚संज्ञाक‚र‚णादेव प्र‚तिप‚त्त‚व्यः ॥
	\pend% ending standard par
       ‚{\tiny $_{lb}$}‚ 

	  \pstart \leavevmode% starting standard par
	उदाह‚र‚ण‚माह--
	\pend% ending standard par
       ‚{\tiny $_{lb}$}‚ 
	  \bigskip
	  \begingroup
	

	  \pstart \leavevmode% starting standard par
	य‚था--अनित्यः श‚ब्द इति साध्ये चाक्षुष‚त्व‚मुभ‚यासिद्ध‚म् ॥ ५८ ॥
	\pend% ending standard par
      
	  \endgroup
	‚{\tiny $_{lb}$}‚ 

	  \pstart \leavevmode% starting standard par
	य‚थेत्यादि । \edtext{\textsuperscript{*}}{\lemma{*}\Bfootnote{नित्यः \cite{dp-msB}}}अनित्यः श‚ब्द इत्य‚नित्य‚त्व‚विशिष्टे श‚ब्दे साध्ये चाक्षुष‚त्वं च‚क्षुर्ग्राह्य‚त्वं‚{\tiny $_{lb}$}‚ श‚ब्दे \edtext{}{\lemma{ब्दे}\Bfootnote{द्व‚योर्द्व‚योर‚पि \cite{dp-msB}}}द्व‚योर‚पि वादिप्र‚तिवादिनोर‚सिद्ध‚म् ॥
	\pend% ending standard par
       ‚{\tiny $_{lb}$}‚ 
	  \bigskip
	  \begingroup
	

	  \pstart \leavevmode% starting standard par
	चेत‚नास्त‚र‚व इति साध्ये स‚र्व‚त्व‚ग‚प‚ह‚र‚णे म‚र‚णं\edtext{}{\lemma{णं}\Bfootnote{म‚र‚णादिति प्र‚ति० \cite{dp-msC}}} प्र‚तिवाद्य‚सिद्ध‚म्, विज्ञानेन्द्रिया‚{\tiny $_{lb}$}‚युर्निरोध‚ल‚क्ष‚ण‚स्य\edtext{}{\lemma{स्य}\Bfootnote{०स्यानेन म‚र‚ण‚स्याभ्यु० \cite{dp-msC}}} म‚र‚ण‚स्यानेनाभ्युप‚ग‚मात्, त‚स्य च त‚रुष्व‚स‚म्भ‚वात् ॥ ५९ ॥
	\pend% ending standard par
      
	  \endgroup
	‚{\tiny $_{lb}$}‚ 

	  \pstart \leavevmode% starting standard par
	चेत‚नास्त‚र‚व इति त‚रूणां चैत‚न्ये साध्ये । स‚र्वा त्व‚क् स‚र्व‚त्व‚क् । त‚स्या अप‚ह‚र‚णे‚{\tiny $_{lb}$}‚ स‚ति म‚र‚णं दिग‚म्ब‚रैरुप‚न्य‚स्तं प्र‚तिवादिनो बौद्ध‚स्यासिद्ध‚म् ।
	\pend% ending standard par
       ‚{\tiny $_{lb}$}‚ 

	  \pstart \leavevmode% starting standard par
	क‚स्माद‚सिद्ध‚मित्याह--विज्ञानं चेन्द्रियं चायुश्चेति द्व‚न्द्वः\edtext{}{\lemma{न्द्वः}\Bfootnote{चायुश्च रूपा० \cite{dp-msA} \cite{dp-edP} \cite{dp-edH} चायुश्च त‚त्र \cite{dp-msB} \cite{dp-edN}}} । त‚त्र विज्ञानं\edtext{}{\lemma{विज्ञानं}\Bfootnote{विज्ञान‚च‚क्षु--\cite{dp-msB}}} च‚क्षुरादि‚{\tiny $_{lb}$}‚ज‚नित‚म्\edtext{}{\lemma{म्}\Bfootnote{ज‚यित--\cite{dp-msB} च‚क्षुरादिविज्ञानं रूपा० \cite{dp-edE} च‚क्षुरादिज्ञानं रूपा० \cite{dp-msC}}} । रूपादिविज्ञानोत्प‚त्त्या य‚द‚नुमितं \edtext{}{\lemma{नुमितं}\Bfootnote{कार्यान्त० \cite{dp-msB} \cite{dp-edP}}}कायान्त‚र्भूतं च‚क्षुर्गोल‚कादिस्थितं\edtext{}{\lemma{कादिस्थितं}\Bfootnote{स्थित‚रूप‚म्--\cite{dp-msA} \cite{dp-msB} \cite{dp-edP} \cite{dp-edH} \cite{dp-edN}}} रूपं
	\pend% ending standard par
      
	  \endgroup
	‚{\tiny $_{lb}$}‚

	  \pstart \leavevmode% starting standard par
	अयं हेत्वाभासः कीदृशीं प्र‚तिप‚त्तिं प्र‚सूत इत्याह--\textbf{असिद्ध‚त्वादेवेति} । तुश‚ब्दार्थ‚श्च‚कारः ।‚{\tiny $_{lb}$}‚ क‚भ‚मेत‚त्प्राप्य‚त इत्याह--\textbf{अय‚ञ्चे}ति । \textbf{चो} य‚स्माद‚र्थे । क‚स्याश्चिद‚पि प्र‚तिप‚त्तेहेतुत‚या न‚{\tiny $_{lb}$}‚ सिद्ध इति \textbf{असिद्ध} उक्त इत्य‚भिप्रायः ॥
	\pend% ending standard par
      ‚{\tiny $_{lb}$}‚

	  \pstart \leavevmode% starting standard par
	\textbf{द्व‚योर‚पि वादिप्र‚तिवादिनोर‚सिद्ध‚म}निश्चित‚म् । अनेन व्य‚धिक‚र‚णासिद्धोऽप्युक्तो‚{\tiny $_{lb}$}‚ द्र‚ष्ट‚व्यः । य‚था राज्ञोऽयं प्रासादः, काक‚स्य कार्ष्ण्यादिति । त‚स्यापि कार्ष्ण्य‚स्य प्रासादे‚{\tiny $_{lb}$}‚ ध‚र्मिणि उभ‚योर‚सिद्ध‚त्वात्, केव‚लं त‚त्रासिद्धोऽप्य‚र्थ‚ध‚र्मिग‚त‚त्वेनोपादानात् त‚था व्य‚प‚दिश्य‚ते ।
	\pend% ending standard par
      ‚{\tiny $_{lb}$}‚

	  \pstart \leavevmode% starting standard par
	न‚नु व्य‚धिक‚र‚ण‚म‚पि लिङ्गं ग‚म‚कं दृष्ट‚म् । य‚था प्र‚त्य‚ग्र‚श‚राव‚द‚र्श‚नं भ्रान्तिम \add{त्}‚{\tiny $_{lb}$}‚प्र‚त्य‚ग्र‚श‚राव‚त्व‚स्य भ्रान्तिम‚च्च‚क्र‚व‚त्त्वे । अस्य च हेतुत्वाद् । देश एव हि ध‚र्मी अविदूर‚{\tiny $_{lb}$}‚कुलाल‚स‚म्ब‚न्धित्वं साध्य‚म् त‚स्य च ध‚र्मिणः प्र‚त्य‚ग्र‚श‚राव‚स‚म्ब‚न्धित्वं भ्रान्तिम‚च्च‚क्र‚स‚म्ब‚न्धित्व‚ञ्च‚{\tiny $_{lb}$}‚ ध‚र्मो भ‚व‚त्येव । न त्व‚त्र कुलाल‚स्य ध‚र्मिणः स‚द्भावः साध्य‚ते, येनैव त‚दुच्येतेति ।
	\pend% ending standard par
      ‚{\tiny $_{lb}$}‚

	  \pstart \leavevmode% starting standard par
	त‚था, विशेष‚णासिद्ध‚विशेष्यासिद्धाव‚प्य‚नेनैवोक्तौ द्र‚ष्ट‚व्यौ । य‚थाऽनित्यः श‚ब्दोऽन‚भि‚{\tiny $_{lb}$}‚ \leavevmode\ledsidenote{\textenglish{191/dm}}‚{\tiny $_{lb}$}‚ 
	  
	त‚दिन्द्रिय‚म् । आयुरिति लोके प्राणा \edtext{}{\lemma{प्राणा}\Bfootnote{उच्य‚ते--\cite{dp-msB}}}उच्य‚न्ते । न चाग‚म‚सिद्ध‚मिह युज्य‚ते व‚क्तुम् । अतः‚{\tiny $_{lb}$}‚ \edtext{\textsuperscript{*}}{\lemma{*}\Bfootnote{प्र‚माण‚स्व० \cite{dp-edH}}}प्राण‚स्व‚भाव‚मायुरिह । तेषां निरोधो निवृत्तिः । स ल‚क्ष‚णं त‚त्त्वं य‚स्य त‚त् त‚थोक्त‚म् ।‚{\tiny $_{lb}$}‚ त‚थाभूत‚स्य म‚र‚ण‚स्य अनेन बौद्धेन प्र‚तिज्ञात‚त्वात् । ‚{\tiny $_{lb}$}‚ 
	  
	य‚दि नामैवं त‚थापि क‚थ‚म‚सिद्ध‚मित्याह--त‚स्य च विज्ञानादिनिरोधात्म‚क‚स्य \edtext{}{\lemma{स्य}\Bfootnote{त‚रुष्व‚भावात्--\cite{dp-msB}}}त‚रुष्व‚{\tiny $_{lb}$}‚स‚म्भ‚वात् । स‚त्तापूर्व‚को निरोधः । त‚त‚श्च यो विज्ञान‚निरोधं त‚रुष्विच्छेत् स क‚थं विज्ञानं‚{\tiny $_{lb}$}‚ नेच्छेत् । त‚स्माद् विज्ञानानिष्टेर्निरोधोऽपि नेष्ट‚स्त‚रुषु । ‚{\tiny $_{lb}$}‚ 
	  
	न‚नु च शोषोऽपि म‚र‚ण‚मुच्य‚ते । स च त‚रुषु सिद्धः । स‚त्य‚म् । केव‚लं विज्ञान‚{\tiny $_{lb}$}‚स‚त्त‚या\edtext{}{\lemma{या}\Bfootnote{स‚त्ताया \cite{dp-msA}}} व्याप्तं य‚त् म‚र‚णं त‚दिह हेतुः । विज्ञान‚निरोध‚श्च त‚त्स‚त्त‚या व्याप्तः, न शोष‚मात्र‚म् ।‚{\tiny $_{lb}$}‚ त‚तो\edtext{}{\lemma{तो}\Bfootnote{त‚त्र \cite{dp-msA}}} य‚न्म‚र‚णं\edtext{}{\lemma{णं}\Bfootnote{म‚र‚ण‚हेतुः--\cite{dp-msA} \cite{dp-msB} \cite{dp-edP} \cite{dp-edH} \cite{dp-edN}}} हेतुस्त‚त् त‚रुष्व‚सिद्ध‚म् । य‚त्तु\edtext{}{\lemma{त्तु}\Bfootnote{य‚च्च \cite{dp-msD}}} सिद्धं शोषात्म‚कं त‚द‚हेतुः । ‚{\tiny $_{lb}$}‚ 
	  
	दिग‚म्ब‚र‚स्तु साध्येन \edtext{}{\lemma{साध्येन}\Bfootnote{विज्ञानेन्द्रियायुर्निरोध‚ल‚क्ष‚ण‚म्--\cite{dp-msD-n}}}व्याप्त‚म‚व्याप्तं\edtext{}{\lemma{व्याप्तं}\Bfootnote{शोष‚ल‚क्ष‚ण‚म् ।}} वा म‚र‚ण‚म‚विविच्य म‚र‚ण‚मात्रं हेतुमाह ।‚{\tiny $_{lb}$}‚ त‚द‚स्य वादिनो हेतुभूतं\edtext{}{\lemma{हेतुभूतं}\Bfootnote{हेतुज्ञात्त‚तं \cite{dp-msA} विज्ञानेन्द्रियायुर्निरोध‚ल‚क्ष‚ण‚म्--\cite{dp-msD-n}}} म‚र‚णं न ज्ञात‚म् । अज्ञानात् सिद्धं शोष‚रूप‚म्, शोष‚रूप‚स्य म‚र‚ण‚स्य‚{\tiny $_{lb}$}‚ त‚रुषु द‚र्श‚नात् । प्र‚तिवादिन‚स्तु ज्ञात‚म‚तोऽसिद्ध‚म् । य‚दा तु वादिनोऽपि ज्ञातं त‚दा वादिनो‚{\tiny $_{lb}$}‚प्य‚सिद्धं स्यादिति न्यायः ॥‚{\tiny $_{lb}$}‚ धेय‚त्वे स‚ति प्र‚मेय‚त्वात् । अनित्यः श‚ब्दः प्र‚मेय‚त्वे स‚ति अन‚भिधेय‚त्वादिति विशेष‚ण‚विशिष्ट‚स्य‚{\tiny $_{lb}$}‚ रूप‚स्य त‚त्र ध‚र्मिणि द्व‚योर‚पि वादिप्र‚तिवादिनोर‚सिद्ध‚त्वात् । केव‚लं त‚त्रासिद्धो विशेष‚ण‚{\tiny $_{lb}$}‚विशिष्ट‚त‚योपात्त‚स्य रूप‚स्य त‚थाऽसिद्धेस्त‚था त‚था व्य‚प‚दिश्य‚त इति ॥
	\pend% ending standard par
      ‚{\tiny $_{lb}$}‚

	  \pstart \leavevmode% starting standard par
	\textbf{चेत‚ना} इति । चेत‚य‚न्त इति \textbf{चेत‚नाः} । दिश एवाम्ब‚रं येषामिति व्युत्प‚त्त्या \textbf{दिग‚म्ब‚राः}‚{\tiny $_{lb}$}‚ क्ष‚प‚ण‚का उच्य‚न्ते । तैः किम्प्र‚माण‚स‚म‚धिग‚त‚मिन्द्रिय‚मित्याह--\textbf{रूपादी}ति । स‚त्स्व‚न्येषु‚{\tiny $_{lb}$}‚ कार‚णेष्व‚व्यापृते च‚क्षुरादौ रूपादिज्ञान‚म‚नुत्प‚द्य‚मानं स्वोत्प‚त्तौ कार‚णा\leavevmode\ledsidenote{\textenglish{67b/ms}}न्त‚र‚म‚पेक्ष‚णीयं‚{\tiny $_{lb}$}‚ सूच‚य‚ति । प्र‚णिहिते तु च‚क्षुरादौ जाय‚मानं त‚त्र‚स्थं त‚त् किम‚पि कार‚ण‚म‚स्तीति ख्याप‚य‚ति ।‚{\tiny $_{lb}$}‚ अत एवाह--\textbf{कायान्त‚र्भूतं} कायाश्रित‚म् । सामान्येनोक्तं काया\textbf{श्रि}त‚त्व‚म् । विशिष्ट‚ज्ञानाभि‚{\tiny $_{lb}$}‚प्रायेणोक्तं विशिष्टाश्र‚याश्रित‚त्वं द‚र्श‚य‚ति \textbf{च‚क्षुरि}ति । \textbf{आदि}श‚ब्देन र‚स‚नादिप‚रिग्र‚हः ।‚{\tiny $_{lb}$}‚ प्र‚स‚न्नार्थादेर‚पि स‚द्भावान्न गोल‚कादिरेवेन्द्रिय‚मिति भावः । किमायुरित्याह--\textbf{आयुरिति} आयुः‚{\tiny $_{lb}$}‚श‚ब्देनेत्य‚र्थः । \textbf{लोके} व्य‚व‚ह‚र्त्त‚रि ज‚ने प्राणोऽन्तःश‚रीरे र‚स‚म‚ल‚धातूनां प्रेर‚णादिहेतुरेकः स‚न्‚{\tiny $_{lb}$}‚ क्रियाभेदोद \edtext{}{\lemma{क्रियाभेदोद}\Bfootnote{भेदाद}} पानादिसंज्ञां ल‚भ‚त इति त‚त्त‚द‚व‚स्थाविव‚क्ष‚या \textbf{प्राणा} इति ब‚हुव‚च‚न‚म् ।
	\pend% ending standard par
      ‚{\tiny $_{lb}$}‚

	  \pstart \leavevmode% starting standard par
	न‚न्वाग‚मे जीवितेन्द्रिय‚मायुरित्युक्त‚म् । त‚त्किमेवं व्याख्याय‚त इत्याह--न चेति ।‚{\tiny $_{lb}$}‚ \textbf{चो}ऽव‚धार‚णे हेतौ वा । य‚त एव\textbf{म‚तः} कार‚णात् । \textbf{इह} प्र‚माण‚सिद्ध‚व‚स्तूप‚द‚र्श‚न‚प्र‚स्तावे त‚था‚{\tiny $_{lb}$}‚‚{\tiny $_{lb}$}‚ ‚{\tiny $_{lb}$}‚ \leavevmode\ledsidenote{\textenglish{192/dm}}‚{\tiny $_{lb}$}‚ 
	  
	अचेत‚नाः सुखाद‚य इति साध्य उत्प‚त्तिम‚त्व‚म् अनित्य‚त्वं\edtext{}{\lemma{त्वं}\Bfootnote{अनित्यं \cite{dp-msB} \cite{dp-edP} \cite{dp-edH} \cite{dp-edE} \cite{dp-edN}}} वा सांख्य‚स्य‚{\tiny $_{lb}$}‚ स्व‚यं वादिनोऽसिद्ध‚म् ॥ ६० ॥‚{\tiny $_{lb}$}‚ 
	  
	अचेत‚नाः सुखाद‚य इति--सुख‚मादिर्येषां दुःखादीनां ते सुखाद‚यः । तेषाम‚चैत‚न्ये साध्ये‚{\tiny $_{lb}$}‚ उत्प‚त्तिम‚त्त्व‚म्, अनित्य‚त्वं वा लिङ्ग‚मुप‚न्य‚स्त‚म् । य उत्प‚त्तिम‚न्तोऽनित्या वा ते न चेत‚नाः ।‚{\tiny $_{lb}$}‚ य‚था रूपाद‚यः । त‚था चोत्प‚त्तिम‚न्तोऽनित्या वा सुखाद‚य‚स्त‚स्माद‚चेत‚नाः । चैत‚न्यं तु पुरुष‚स्य‚{\tiny $_{lb}$}‚ \edtext{\textsuperscript{*}}{\lemma{*}\Bfootnote{स्वं रूप‚म्--\cite{dp-msA} \cite{dp-msB} \cite{dp-edP} \cite{dp-edH} \cite{dp-edN}}}स्व‚रूप‚म् । अत्र चोत्प‚त्तिम‚त्त्व‚म‚नित्य‚त्वं वा प‚र्यायेण हेतुर्न युग‚प‚त् । त‚च्च द्व‚य‚म‚पि‚{\tiny $_{lb}$}‚ सांख्य‚स्य वादिनो न सिद्ध‚म् । प‚रार्थो\edtext{}{\lemma{रार्थो}\Bfootnote{प‚रार्थादि हे० \cite{dp-msB}}} हि हेतूप‚न्यासः । तेन यः प‚र‚स्य सिद्धः स हेतु‚{\tiny $_{lb}$}‚र्व‚क्त‚व्यः । प‚र‚स्य चास‚त उत्पाद उत्प‚त्तिम‚त्त्व‚म्, स‚त‚श्च निर‚न्व‚यो विनाशोऽनित्य‚त्वं सिद्ध‚म् ।‚{\tiny $_{lb}$}‚ तादृशं च द्व‚य‚म‚पि सांख्य‚स्यासिद्ध‚म् । इहाप्य‚नित्य‚त्वोत्प‚त्तिम‚त्त्व‚साध‚नाद् वादिनोऽसिद्ध‚म् ।‚{\tiny $_{lb}$}‚ य‚दि त्व‚नित्य‚त्वोत्प‚त्तिम‚त्त्व‚योः \edtext{}{\lemma{योः}\Bfootnote{प्र‚माणं \cite{dp-msA} \cite{dp-msC} \cite{dp-edP} \cite{dp-edH} \cite{dp-edE} \cite{dp-edN}}}प्रामाण्यं वादिनो \edtext{}{\lemma{वादिनो}\Bfootnote{विज्ञानं \cite{dp-msC}}}ज्ञातं स्याद् \edtext{}{\lemma{स्याद्}\Bfootnote{त‚दा नास्ति \cite{dp-msA} \cite{dp-edP} \cite{dp-edH} \cite{dp-edE} \cite{dp-edN}}}त‚दा वादिनोऽपि सिद्धं स्यात् ।‚{\tiny $_{lb}$}‚ त‚तः प्र‚माणाप‚रिज्ञानादिदं वादिनोऽसिद्ध‚म् ॥ ‚{\tiny $_{lb}$}‚ 
	  
	संदिग्धासिद्धं द‚र्श‚यितुमाह-- ‚{\tiny $_{lb}$}‚ 
	  
	त‚था स्व‚यं त‚दाश्र‚य‚ण‚स्य वा संदेहेऽसिद्धः ॥ ६१ ॥‚{\tiny $_{lb}$}‚ 
	  
	स्व‚य‚मिति हेतोरात्म‚नः स‚न्देहेऽसिद्धः । त‚दाश्र‚य‚ण‚स्य \edtext{}{\lemma{स्य}\Bfootnote{०स्य चेति \cite{dp-msA} \cite{dp-msB} \cite{dp-msC} \cite{dp-msD} \cite{dp-edP} \cite{dp-edH} \cite{dp-edE}}}वेति--त‚स्य हेतोराश्र‚य‚ण‚म्—‚{\tiny $_{lb}$}‚आश्रीय‚तेऽस्मिन् हेतुरित्याश्र‚य‚णं हेतोर्व्य‚तिरिक्त आश्र‚य‚भूतः\edtext{}{\lemma{भूतः}\Bfootnote{आश्र‚य‚भूत‚सा० \cite{dp-edE}}} साध्य‚ध‚र्मी क‚थ्य‚ते । त‚त्र हि‚{\tiny $_{lb}$}‚ त‚स्यैव रूप‚स्य । \textbf{विज्ञान‚स‚त्त‚या व्याप्तं य‚दि}ति य‚स्मिन् म‚र‚णेऽव‚श्यं प्रागासीद् विज्ञान‚म्, त‚द्‚{\tiny $_{lb}$}‚ विज्ञान‚स‚त्त‚या व्याप्त‚मुक्त‚म् । त‚च्च श्वासोष्म‚प‚रिस्प‚न्दादिविग‚म‚ल‚क्ष‚ण‚म् । \textbf{दिग‚म्ब‚र‚स्यापि‚{\tiny $_{lb}$}‚ क‚थं} सिद्ध‚मित्याह--\textbf{अज्ञानादिति} । चैत‚न्याव्य‚भिचारिणो म‚र‚ण‚स्याज्ञानात् । अनेनैत‚दाह—‚{\tiny $_{lb}$}‚य‚दि तेन साध्य‚व्याप्तं म‚र‚णं म‚र‚ण‚श‚ब्द‚मात्र‚स‚म‚तां बिभ्र‚तः शोष‚मात्राद् भेदेन विवेचितं स्यात्‚{\tiny $_{lb}$}‚ केव‚ल‚म‚ज्ञानात्त‚त्सिद्ध‚मुच्य‚त इति ।
	\pend% ending standard par
      ‚{\tiny $_{lb}$}‚

	  \pstart \leavevmode% starting standard par
	एत‚देवाह--\textbf{य‚दा त्वि}ति । एव‚मुत्त‚र‚त्रापि द्र‚ष्ट‚व्य‚म् ।
	\pend% ending standard par
      ‚{\tiny $_{lb}$}‚

	  \pstart \leavevmode% starting standard par
	\textbf{सुख‚म}नुकूल‚वेद‚नीय‚म् । आदिश‚ब्दादिच्छाद्वेषादिप‚रिग्र‚हः । पुरुष‚स्य सांख्य‚प‚रिक‚ल्पित‚{\tiny $_{lb}$}‚स्यात्म‚नः । पुरुष‚श्चेत‚य‚ते बुद्धिर‚ध्य‚व‚स्य‚तीति सिद्धान्तात् । \textbf{सांख्य‚स्य} संख्य‚या प‚ञ्च‚{\tiny $_{lb}$}‚विंश‚तित‚त्त्वानीत्य‚न‚या व्य‚व‚ह‚र‚तीति \textbf{सांख्यो} योग‚रूढिश्चैषा, \textbf{क‚पिल} एव त‚थोच्य‚ते ।
	\pend% ending standard par
      ‚{\tiny $_{lb}$}‚

	  \pstart \leavevmode% starting standard par
	न‚नूत्प‚त्तिम‚त्त्वं कृत‚क‚त्व‚म् वा स्व‚सिद्ध‚मेव तेनोप‚न्य‚स‚नीय‚मुप‚न्य‚स्तं च । त‚त्क‚थं वाद्य‚सिद्ध‚ते‚{\tiny $_{lb}$}‚त्याह--\textbf{प‚रार्थो} हीति । हिर्य‚स्मात् । \textbf{प‚रार्थः} प‚र‚प्र‚तिप‚त्तिप्र‚योज‚नः । \textbf{निर‚न्व‚यः} स‚र्व‚थोच्छेदः ।‚{\tiny $_{lb}$}‚ ‚{\tiny $_{lb}$}‚ \leavevmode\ledsidenote{\textenglish{193/dm}}‚{\tiny $_{lb}$}‚ 
	  
	हेतुर्व‚र्त्त‚मानो ग‚म‚क‚त्वेनाश्रीय‚ते । त‚स्याश्र‚य‚ण‚स्य स‚न्देहे स‚न्दिग्धः ॥ ‚{\tiny $_{lb}$}‚ 
	  
	\edtext{\textsuperscript{*}}{\lemma{*}\Bfootnote{स्वात्म‚ना \cite{dp-msA} \cite{dp-msB} \cite{dp-msC} \cite{dp-msD} \cite{dp-edP} \cite{dp-edE} \cite{dp-edH} \cite{dp-edN}}}आत्म‚ना स‚न्दिह्य‚मान‚मुदाह‚र्त्तुमाह-- ‚{\tiny $_{lb}$}‚ 
	  
	य‚था बाष्पादिभावेन संदिह्य‚मानो भूत‚स‚ङ्घातोऽग्निसिद्धावुप‚दिश्य‚मानः‚{\tiny $_{lb}$}‚ \edtext{\textsuperscript{*}}{\lemma{*}\Bfootnote{संदिग्धासिद्धः नास्ति \cite{dp-msC}}}संदिग्धासिद्धः ॥ ६२ ॥‚{\tiny $_{lb}$}‚ 
	  
	य‚थेति । बाष्प आदिर्य‚स्य स बाष्पादिः । त‚द्भावेन बाष्पादित्वेन संदिह्य‚मानो‚{\tiny $_{lb}$}‚ भूत‚संङ्घात इति भूतानां पृथिव्यादीनां संङ्घातः स‚मूहः । अग्निसिद्धौ--अग्निसिद्ध्य‚र्थ‚म् उपादी‚{\tiny $_{lb}$}‚य‚मानोऽसिद्धः । ‚{\tiny $_{lb}$}‚ 
	  
	एत‚दुक्तं भ‚व‚ति--य‚दा धूमोऽपि बाष्पादित्वेन संदिग्धो भ‚व‚ति त‚दाऽसिद्धः, ग‚म‚क‚रूपा‚{\tiny $_{lb}$}‚निश्च‚यात् । धूम‚त‚या निश्चितो \edtext{}{\lemma{निश्चितो}\Bfootnote{व‚ह्निकार्य‚त्वाद्ग० \cite{dp-msC}}}व‚ह्निज‚न्य‚त्वाद् ग‚म‚कः । य‚दा तु संदिग्ध‚स्त‚दा न ग‚म‚क‚{\tiny $_{lb}$}‚ इत्य‚सिद्ध‚ताख्यो दोषः ॥ ‚{\tiny $_{lb}$}‚ 
	  
	आश्र‚य‚णासिद्ध‚मुदाह‚र‚ति-- ‚{\tiny $_{lb}$}‚ 
	  
	य‚थेह निकुञ्जे\edtext{}{\lemma{निकुञ्जे}\Bfootnote{निगुञ्जे--\cite{dp-msC}}} म‚यूरः केकायितादिति ॥ ६३ ॥‚{\tiny $_{lb}$}‚ 
	  
	य‚थेति । इह निकुञ्ज इति ध‚र्मी । प‚र्व‚तोप‚रिभागेन तिर्य‚ङ्निर्ग‚तेन प्र‚च्छादितो‚{\tiny $_{lb}$}‚ भूमागो निकुञ्जः । म‚यूर इति साध्य‚म् । केकायितादिति हेतुः । केकायितं--म‚यूर‚ध्व‚निः ॥‚{\tiny $_{lb}$}‚ न तु विन‚ष्ट‚स्यापि स‚त्त्व‚र‚ज‚स्त‚मोरूपेणानुग‚म इष्टः । न केव‚लं पूर्व‚मेवेत्य‚पिश‚ब्दः । \textbf{अनित्य‚{\tiny $_{lb}$}‚त्वोत्प‚त्तिम‚त्त्व}योर्य‚त्साध‚नं प्र‚माणं त‚स्य \edtext{}{\lemma{स्य}\Bfootnote{स्याऽ}} ज्ञानाद‚निश्च‚यात् य‚दीत्यादिनैत‚देव द्र‚ढ‚य‚ति ॥
	\pend% ending standard par
      ‚{\tiny $_{lb}$}‚

	  \pstart \leavevmode% starting standard par
	\textbf{स्व‚य}मि\textbf{त्यात्म‚न} इति ष‚ष्ठ्य‚न्त‚स्यानुव‚र्त्त‚ते । हेतोश्च प्र‚कृत‚त्वाद् \textbf{हेतो}रिति विवृणोति ।‚{\tiny $_{lb}$}‚ आश्रीय‚ते साध‚न‚त्वेनोपादीय‚ते \textbf{अस्मिन्निति} ॥
	\pend% ending standard par
      ‚{\tiny $_{lb}$}‚

	  \pstart \leavevmode% starting standard par
	य‚स्यात्म‚नः स‚न्देहः स आत्म‚ना स‚न्दिह्य‚मानो भ‚व‚तीत्य‚भिप्रायेणाह--\textbf{आत्म‚ना स‚न्दिह्य‚{\tiny $_{lb}$}‚मान‚मि}ति । \textbf{आदि}श‚ब्देन नीहारादिप‚रिग्र‚हः ।
	\pend% ending standard par
      ‚{\tiny $_{lb}$}‚

	  \pstart \leavevmode% starting standard par
	न‚नु य‚द्य‚सौ प‚र‚मार्थ‚तो धूम‚स्त‚दा स‚न्देहेऽपि किं न ग‚म‚क इत्याह--\textbf{एत‚दुक्तं भ‚व‚तीति} ।‚{\tiny $_{lb}$}‚ किं त‚द् ग‚म‚कं रूपं येनानिश्चित इत्याह--\textbf{व‚ह्नी}ति । \textbf{य‚दा त्वि}त्यादिनोक्त‚मेव स्प‚ष्ट‚य‚ति ।‚{\tiny $_{lb}$}‚ \textbf{त‚दा न ग‚म‚क} इति ब्रुव‚त‚श्चाय‚माश‚यो व‚ह्निज‚न्य‚त्व‚स्यैव ग‚म‚क‚त्व‚निब‚न्ध‚न‚स्य त‚दाऽनिश्चित‚त्वात् ।
	\pend% ending standard par
      ‚{\tiny $_{lb}$}‚

	  \pstart \leavevmode% starting standard par
	एतेन स‚न्दिग्ध‚विशेष‚णासिद्धः स‚न्दिग्ध‚विशेष्यासिद्ध‚श्चोक्तो द्र‚ष्ट‚व्यः \leavevmode\ledsidenote{\textenglish{68a/ms}} य‚था‚{\tiny $_{lb}$}‚ ष‚ड्जादिस‚त्त्व‚स‚न्देहे म‚यूर‚श‚ब्दोऽयं ष‚ड्जादिम‚त्त्वे स‚ति अव‚र्णात्म‚क‚त्वात् । अव‚र्णात्म‚क‚त्वे स‚ति‚{\tiny $_{lb}$}‚ ‚{\tiny $_{lb}$}‚ \leavevmode\ledsidenote{\textenglish{194/dm}}‚{\tiny $_{lb}$}‚ 
	  
	\edtext{\textsuperscript{*}}{\lemma{*}\Bfootnote{क‚थ‚माश्र० \cite{dp-msA} \cite{dp-msB} \cite{dp-edP} \cite{dp-edH} \cite{dp-edE}}}क‚थ‚म‚य‚माश्र‚य‚णासिद्ध इत्याह-- ‚{\tiny $_{lb}$}‚ 
	  
	त‚दापात‚देश‚विभ्र‚मे ॥ ६४ ॥‚{\tiny $_{lb}$}‚ 
	  
	त‚दापात\edtext{}{\lemma{दापात}\Bfootnote{त‚द‚घात--\cite{dp-msA}}} इति । त‚स्य केकायित‚स्यापात \edtext{}{\lemma{स्यापात}\Bfootnote{आग‚मः \cite{dp-msC}}}आग‚म‚नं त‚स्य देशः स उच्य‚ते य‚स्माद‚{\tiny $_{lb}$}‚ वेशादाग‚च्छ‚ति केकायित‚म् । त‚स्य विभ्र‚मे व्यामोहे स‚त्य‚य‚माश्र‚य‚णासिद्धः । निर‚न्त‚रेषु‚{\tiny $_{lb}$}‚ व‚हुषु निकुञ्जेषु स‚त्सु य‚दा केकायितापात‚निकुञ्जे\edtext{}{\lemma{निकुञ्जे}\Bfootnote{केकायितापात‚विभ्र‚मः \cite{dp-msA} \cite{dp-edP} \cite{dp-edH}}} विभ्र‚मः--किम‚स्मान्निकुञ्जात् केकायित‚{\tiny $_{lb}$}‚माग‚त‚म् । आहोस्विद‚न्य‚स्मादिति\edtext{}{\lemma{स्मादिति}\Bfootnote{आहोस्विद‚स्मादिति--\cite{dp-msA} \cite{dp-edP} \cite{dp-edH} \cite{dp-edN}}}, \edtext{\textsuperscript{*}}{\lemma{*}\Bfootnote{तादाश्र‚य० \cite{dp-msA} \cite{dp-msB} \cite{dp-edP} \cite{dp-edH} \cite{dp-edE} \cite{dp-edN}}}त‚दाय‚माश्र‚य‚णासिद्ध इति ॥ ‚{\tiny $_{lb}$}‚ 
	  
	ध‚र्मिणोऽसिद्धाव‚प्य‚सिद्ध‚त्व‚मुदाह‚र‚ति-- ‚{\tiny $_{lb}$}‚ 
	  
	ध‚र्म्य‚सिद्धाव‚प्य‚सिद्धः--य‚था स‚र्व‚ग‚त आत्मेति साध्ये स‚र्व‚त्रोप‚ल‚भ्य‚{\tiny $_{lb}$}‚मान‚गुण‚त्व‚म् ॥ ६५ ॥‚{\tiny $_{lb}$}‚ 
	  
	य‚थेति । स‚र्व‚स्मिन् ग‚तः स्थितः स‚र्व‚ग‚तो व्यापीति याव‚त् । व्यापित्व आत्म‚नः साध्ये‚{\tiny $_{lb}$}‚ स‚र्व‚त्रोप‚ल‚भ्य‚मान‚गुण‚त्वं लिङ्ग‚म् । स‚र्व‚त्र देश उप‚ल‚भ्य‚मानाः सुख‚दुःखेच्छाद्वेषाद‚यो गुणा‚{\tiny $_{lb}$}‚ य‚स्यात्म‚न‚स्त‚स्य भाव‚स्त‚त्त्व‚म् । न गुणा गुणिन‚म‚न्त‚रेण व‚र्त्त‚न्ते । गुणानां गुणिनि स‚म‚वायात् ।‚{\tiny $_{lb}$}‚ निष्क्रिय‚श्चात्मा । त‚त‚श्च य‚दि व्यापी न भ‚वेत् क‚थं दाक्षिणाप‚थ उप‚ल‚ब्धाः सुखाद‚यो म‚ध्य‚देश‚{\tiny $_{lb}$}‚ उप‚ल‚भ्येर‚न् । त‚स्मात् स‚र्व‚ग‚त आत्मा । ‚{\tiny $_{lb}$}‚ 
	  
	त‚दिह बौद्ध‚स्यात्मैव न सिद्धः, किमुत स‚र्व‚त्रोप‚ल‚भ्य‚मान‚गुण‚त्वं सिध्येत् त‚स्येत्य‚सिद्धौ\edtext{}{\lemma{सिद्धौ}\Bfootnote{घ‚र्मिति [[णि]] हेतोः स‚म्ब‚न्ध‚स्य स‚त्त्व‚स्येत्य‚सिद्धौ--\cite{dp-msD-n}}}‚{\tiny $_{lb}$}‚ हेत्वाभासः । पूर्व‚माश्र‚य‚ण‚संदेहेन ध‚र्मिणि संदेह उक्तः । संप्र‚ति त्व‚सिद्धो ध‚र्म्युक्त इत्य‚न‚{\tiny $_{lb}$}‚न‚योर्विशेषः ।‚{\tiny $_{lb}$}‚ ष‚ड्जादिम‚त्त्वादिति । उभ‚य‚त्रापि विशेष‚ण‚विशिष्ट‚स्य रूप‚स्य वादिप्र‚तिवादिनोर्द्व‚योर‚प्य‚निश्चित‚{\tiny $_{lb}$}‚त्वात्केव‚लं विशेष‚ण‚स‚न्देहेन च विशेष‚ण‚विशिष्टेन रूपेणासिद्ध इति त‚था व्य‚प‚दिश्य‚त इति ॥
	\pend% ending standard par
      ‚{\tiny $_{lb}$}‚

	  \pstart \leavevmode% starting standard par
	\textbf{प‚र्व‚तोप‚रिभागेन तिर्य‚ग्निर्ग}तेनेति च भूभाग इति चोप‚ल‚क्ष‚णं द्र‚ष्ट‚व्य‚म् । न तु त‚थाविध‚{\tiny $_{lb}$}‚ एव निकुञ्जः, प‚र्व‚त‚ग‚ह्व‚र‚देश‚स्यैव निकुञ्ज‚श‚ब्दाभिल‚प्य वात् ॥
	\pend% ending standard par
      ‚{\tiny $_{lb}$}‚

	  \pstart \leavevmode% starting standard par
	\textbf{य‚स्माद् देशादाग‚च्छ‚ती}ति व‚च‚न‚व्य‚क्त्या चोत्प‚न्नः श‚ब्द‚श्च‚तुर्दिव‚क श‚ब्द‚स‚न्तानं ज‚न‚य‚ति,‚{\tiny $_{lb}$}‚ स च ज‚ल‚त‚र‚ङ्ग‚न्यायेन श्रोत्र‚देश‚माग‚तो गृहीत इति द‚र्श‚य‚ति ॥
	\pend% ending standard par
      ‚{\tiny $_{lb}$}‚

	  \pstart \leavevmode% starting standard par
	\textbf{द्वेषादी}त्य\textbf{त्रादि}ग्र‚ह‚णेन बुद्धिप्र‚य‚त्नादीनां ग्र‚ह‚ण‚म् । सामान्य‚वान् गुणः संयोग‚विभाग‚यो‚{\tiny $_{lb}$}‚र‚न‚पेक्षो न कार‚ण‚म् इति गुण‚ल‚क्ष‚ण‚योगाद् \textbf{गुणाः । स‚म‚वाया}त्स‚म‚वेत‚त्वात् । प्र‚तिषिद्धानां‚{\tiny $_{lb}$}‚ ‚{\tiny $_{lb}$}‚ \leavevmode\ledsidenote{\textenglish{195/dm}}‚{\tiny $_{lb}$}‚ 
	  
	त‚देव‚मेक‚स्य\edtext{}{\lemma{स्य}\Bfootnote{स‚र्वेष्व‚पि असिद्धेषु एकं रूपं प‚क्ष‚ध‚र्म‚त्व‚म‚सिद्ध‚म्--\cite{dp-msD-n}}} रूप‚स्य \edtext{}{\lemma{स्य}\Bfootnote{ध‚र्मिब‚द्ध० \cite{dp-msA} \cite{dp-edP} \cite{dp-msB} \cite{dp-edH}}}ध‚र्मिस‚म्ब‚द्ध‚स्यासिद्धाव‚सिद्धो हेत्वाभासः ॥ ‚{\tiny $_{lb}$}‚ 
	  
	त‚थैक‚स्य रूप‚स्यास‚प‚क्षेऽस‚त्त्व‚स्यासिद्धाव‚नैकान्तिको हेत्वाभासः ॥ ६६ ॥‚{\tiny $_{lb}$}‚ 
	  
	\edtext{\textsuperscript{*}}{\lemma{*}\Bfootnote{त‚था प‚र० \cite{dp-msA} \cite{dp-msB} \cite{dp-edP} \cite{dp-edH}}}त‚थाऽप‚र‚स्यैक‚स्य रूप‚स्य--\edtext{\textsuperscript{*}}{\lemma{*}\Bfootnote{विप‚क्षे--\cite{dp-msD-n}}}अस‚प‚क्षेऽस‚त्त्वा\textbf{ख्य}स्यासिद्धाव‚नैकान्तिको हेत्वाभासः । ‚{\tiny $_{lb}$}‚ 
	  
	एकोऽन्त एकान्तो निश्च‚यः । स प्र‚योज‚न‚म‚स्येत्यैकान्तिकः\edtext{}{\lemma{स्येत्यैकान्तिकः}\Bfootnote{०म‚स्यैकान्तिकः \cite{dp-msC} \cite{dp-msD}}} । नैकान्तिकोऽनैकान्तिकः ।‚{\tiny $_{lb}$}‚ य‚स्मान्न साध्य‚स्य न विप‚र्य‚य‚स्य निश्च‚योऽपि तु त‚द्विप‚रीतः संश‚यः । साध्येत‚र‚योः संश‚य‚हेतु‚{\tiny $_{lb}$}‚र‚नैकान्तिक उक्तः ॥‚{\tiny $_{lb}$}‚ च न स‚म‚वाय इति । \textbf{निष्क्रिय‚त्वं} च क्रियाया मूर्त्त‚द्र‚व्य‚वृत्तित्वात्, आत्म‚न‚श्चाम‚र्त्त‚त्वादिति‚{\tiny $_{lb}$}‚ सिद्धान्त‚स्थितेः ।
	\pend% ending standard par
      ‚{\tiny $_{lb}$}‚

	  \pstart \leavevmode% starting standard par
	\textbf{किमुते}ति निपातः किम्पुन‚रित्य‚स्यार्थे व‚र्त्त‚ते ।
	\pend% ending standard par
      ‚{\tiny $_{lb}$}‚

	  \pstart \leavevmode% starting standard par
	आश्र‚य‚णासिद्ध‚ध‚र्म्य‚सिद्ध‚योः कियान् भेद इत्याश‚ङ्क्य भेद‚मुप‚पाद‚य‚न्नाह--पूर्व‚मिति ।‚{\tiny $_{lb}$}‚ अय‚म‚र्थः--पूर्वं प‚र‚मार्थ‚तो विद्य‚मानोऽपि हेत्वाधार‚रूप‚त‚या स‚न्देहाद‚निश्चित इति त‚द् ध‚र्म्य‚सिद्ध‚{\tiny $_{lb}$}‚ उक्तः । \textbf{स‚म्प्र‚ति तु} स‚र्व‚थैवासौ ध‚र्मित्वेनासिद्ध उच्य‚त इति । ध‚र्म्य‚सिद्ध एवाश्र‚यासिद्ध‚{\tiny $_{lb}$}‚ उच्य‚त इति । तेन नाश्र‚यासिद्धो नाम अन्यः प्र‚भेदः ।
	\pend% ending standard par
      ‚{\tiny $_{lb}$}‚

	  \pstart \leavevmode% starting standard par
	अन्य‚थासिद्ध‚स्त्व‚सिद्ध एव न भ‚व‚तीति न त‚स्यान्त‚र्भाव‚श्चिन्त्य‚ते । त‚था ह्य‚न्य‚थासिद्ध‚{\tiny $_{lb}$}‚ इति कोऽर्थः ? किम‚न्य‚थैव सिद्ध आहोस्वित् अन्य‚थाऽपि सिद्धः ? न‚नु य‚द्य‚न्य‚थैव, त‚दा‚{\tiny $_{lb}$}‚ जिज्ञाप‚यिषित‚विप‚र्य‚येणैव सिद्ध उप‚प‚न्नो नानेन प्र‚कारेणेति विरुद्ध एव । अथान्य‚थाऽपि‚{\tiny $_{lb}$}‚ सिद्धः, त‚दैत‚स्माद‚न्येनापि प्र‚कारेण सिद्धोऽय‚मित्य‚य‚म‚पि भ‚विष्य‚ति । न च साध्य‚मिति स‚न्दिग्ध‚{\tiny $_{lb}$}‚विप‚क्ष‚व्यावृत्तिर‚नैकान्तिक एवेति ॥
	\pend% ending standard par
      ‚{\tiny $_{lb}$}‚

	  \pstart \leavevmode% starting standard par
	अनैकान्तिक‚श‚ब्द‚स्य व्युत्प‚त्तिमाह--\textbf{एक} इति । \textbf{एक} इति ब्रुव‚न्नेक‚स्ता\edtext{}{\lemma{स्ता}\Bfootnote{क‚श्चा}}सौ साध्य‚{\tiny $_{lb}$}‚ल‚क्ष‚णैकार्थ‚विष‚य‚त्वाद‚न्त‚श्च क‚थाव‚सान‚हेतुत्वादाकाङ्क्षोप‚श‚म‚हेतुत्वाद् वेति द‚र्श‚य‚ति । स‚म‚स्तं‚{\tiny $_{lb}$}‚ प‚द‚म‚र्थ‚ञ्चाह--\textbf{एकान्तो निश्च‚य} इति । साध्येत‚र‚योरेक‚त‚र‚निश्च‚य‚फ‚ल इत्य‚र्थः । त‚द्विरुद्धे‚{\tiny $_{lb}$}‚ चायं न‚ञ् द्र‚ष्ट‚व्यः । \textbf{य‚स्मादि}त्यादिनैत‚देव स्फुट‚य‚ति । \textbf{य‚स्माद्}धेतोरित्य‚पादाने चेयं प‚ञ्च‚मी ।‚{\tiny $_{lb}$}‚ किन्तु \textbf{त‚द्विप‚रीतो} निश्च‚य‚विप‚रीतः य‚त्त‚दोर्नित्य‚म‚भिस‚म्ब‚न्धात्स इति द्र‚ष्ट‚व्य‚म् । क्व संश‚य‚{\tiny $_{lb}$}‚ इत्याह--\textbf{साध्येति} ।
	\pend% ending standard par
      ‚{\tiny $_{lb}$}‚

	  \pstart \leavevmode% starting standard par
	य‚द्वा क‚थ‚म‚नैकान्तिको भ‚व‚तीत्याह--\textbf{य‚स्मादि}ति हेतुप‚ञ्च‚मी । ऐकान्तिक‚प्र‚तिषेधेनान्यः‚{\tiny $_{lb}$}‚ प्र‚तिप‚त्तिहेतुरुक्तो न‚ञेंत्य‚भिप्रायेणाह--\textbf{संश‚य‚हेतु}रिति । तेन नासिद्ध‚स्य त‚थात्व‚प्र‚स‚ङ्गः ।‚{\tiny $_{lb}$}‚ अथ‚वैकान्त‚निय‚त‚त्वादेक इत्य‚न्त इति च निश्च‚यः प्रोक्तः । त‚थाहि स‚र्वोऽयं प‚दार्थ‚भेद‚{\tiny $_{lb}$}‚ ‚{\tiny $_{lb}$}‚ \leavevmode\ledsidenote{\textenglish{196/dm}}‚{\tiny $_{lb}$}‚ 
	  
	त‚मुदाह‚र‚ति-- ‚{\tiny $_{lb}$}‚ 
	  
	य‚था श‚ब्द‚स्यानित्य‚त्वादिके ध‚र्मे साध्ये प्र‚मेय‚त्वादिको ध‚र्मः स‚प‚क्ष‚{\tiny $_{lb}$}‚विप‚क्ष‚योः स‚र्व‚त्रैक‚देशे वा व‚र्त्त‚मानः ॥ ६७ ॥‚{\tiny $_{lb}$}‚ 
	  
	य‚थेत्यादिना । अनित्य‚त्व‚मादिर्य‚स्याऽसौ\edtext{}{\lemma{स्याऽसौ}\Bfootnote{०र्य‚स्य सोऽनि० \cite{dp-msA} \cite{dp-msB} \cite{dp-msD} \cite{dp-edP} \cite{dp-edH} \cite{dp-edE} \cite{dp-edN}}} अनित्य‚त्वादिको ध‚र्मः । आदिश‚ब्दाद‚प्र‚य‚त्ना‚{\tiny $_{lb}$}‚न‚न्त‚रीय‚क‚त्वं प्र‚य‚त्नान‚न्त‚रीय‚क‚त्वं \edtext{}{\lemma{त्वं}\Bfootnote{नित्य‚त्वं नास्ति \cite{dp-msB}}}नित्य‚त्वं च प‚रिगृह्य‚ते । प्र‚मेय‚त्व‚म् आदिर्य‚स्य स‚{\tiny $_{lb}$}‚ प्र‚मेय‚त्वादिकः । आदिश‚ब्दाद‚नित्य‚त्व‚म्, पुन‚र‚नित्य‚त्व‚म्, अमूर्त्त‚त्वं च \edtext{}{\lemma{च}\Bfootnote{गृह्य‚न्ते \cite{dp-msB}}}गृह्य‚ते । श‚ब्द‚स्य‚{\tiny $_{lb}$}‚ ध‚र्मिणोऽनित्य‚त्वादिके ध‚र्मे साध्ये प्र‚मेय‚त्वादिको ध‚र्मोऽनैकान्तिकः । च‚तुर्णाम‚पि \edtext{}{\lemma{पि}\Bfootnote{नास्ति हि--\cite{dp-msA} \cite{dp-msB} \cite{dp-edP} \cite{dp-edH} \cite{dp-edN}}}हि विप‚क्षेऽ‚{\tiny $_{lb}$}‚स‚त्त्व‚म‚सिद्ध‚म् । ‚{\tiny $_{lb}$}‚ 
	  
	त‚थाहि-अनित्यः श‚ब्दः प्र‚मेय‚त्वात्\edtext{}{\lemma{त्वात्}\Bfootnote{प्र‚मेय‚त्वात्, आकाश‚व‚त् घ‚ट‚व‚दिति--\cite{dp-msA} \cite{dp-msB} \cite{dp-msD} \cite{dp-edP} \cite{dp-edH} \cite{dp-edE} \cite{dp-edN}}} घ‚ट‚व‚द्-आकाश‚व‚दिति प्र‚मेय‚त्वं स‚प‚क्ष‚विप‚क्ष‚व्यापि । ‚{\tiny $_{lb}$}‚ 
	  
	अप्र‚य‚त्नान‚न्त‚रीय‚कः श‚ब्दोऽनित्य‚त्वात्, विद्युदाकाश‚व‚द् \edtext{}{\lemma{द्}\Bfootnote{काश‚घ‚ट‚व० \cite{dp-msA}}}घ‚ट‚व‚च्च--इत्य‚नित्य‚त्वं स‚प‚क्षै‚{\tiny $_{lb}$}‚क‚देश‚वृत्ति--विद्युदादाव‚स्ति, नाकाशादौ; \edtext{\textsuperscript{*}}{\lemma{*}\Bfootnote{इत आर‚म्भ नाकाशादौ प‚र्य‚न्तः पाठो नास्ति \cite{dp-msB}}}विप‚क्ष‚व्यापि--\edtext{\textsuperscript{*}}{\lemma{*}\Bfootnote{०व्यापि स‚र्व‚त्र प्र‚य‚त्नान‚न्त‚रीय‚के भावात् \cite{dp-msC}}}प्र‚य‚त्नान‚न्त‚रीय‚के स‚र्व‚त्र भावात्\edtext{}{\lemma{भावात्}\Bfootnote{अनित्य‚त्व‚स्य--\cite{dp-msD-n}}} । ‚{\tiny $_{lb}$}‚ 
	  
	अनित्य‚त्वात् प्र‚य‚त्नान‚न्त‚रीय‚कः श‚ब्दो घ‚ट‚व‚द् विद्युदाकाश‚व‚च्च--इत्य‚नित्य‚त्वं विप‚क्षैक‚{\tiny $_{lb}$}‚देश‚वृत्ति--विद्युदादाव‚स्ति नाकाशादौ । स‚प‚क्ष‚व्यापि\edtext{}{\lemma{व्यापि}\Bfootnote{व्यापि स‚प‚क्षे स‚र्व‚त्र \cite{dp-msC} \cite{dp-msD}}} स‚र्व‚त्र प्र‚य‚त्नान‚न्त‚रीय‚के भावात्\edtext{}{\lemma{भावात्}\Bfootnote{साध्यैः स‚ह क्र‚मात् योज्य‚ते--\cite{dp-msD-n}}} । ‚{\tiny $_{lb}$}‚ 
	  
	नित्यः श‚ब्दोऽमूर्त्त‚त्वाद् आकाश‚प‚र‚माणुव‚त्,\edtext{\textsuperscript{*}}{\lemma{*}\Bfootnote{नित्य‚त्व‚स्य--\cite{dp-msD-n}}} क‚र्म‚घ‚ट‚व‚च्च । इत्य‚मूर्त‚त्व‚मुभ‚यैक‚{\tiny $_{lb}$}‚देश‚वृत्ति--उभ‚योरेक‚देश आकाशे क‚र्म‚णि च व‚र्त‚ते । प‚र‚माणौ तु स‚प‚क्षैक‚देशे घ‚टादौ च‚{\tiny $_{lb}$}‚ विप‚क्षैक‚देशे न व‚र्त्त‚ते । मूर्त्त‚त्वात् घ‚ट‚प‚र‚माणुप्र‚भृतीनाम् । ‚{\tiny $_{lb}$}‚ 
	  
	नित्यास्तु प‚र‚माण‚वो \edtext{}{\lemma{वो}\Bfootnote{वैशेषिकैर‚प्य‚भ्युप० \cite{dp-msC}}}वैशेषिकैर‚भ्युप‚ग‚म्य‚न्ते । त‚तः स‚प‚क्षान्त‚र्ग‚ताः ।‚{\tiny $_{lb}$}‚ एक‚स्मिन्ना\edtext{}{\lemma{स्मिन्ना}\Bfootnote{न्न}}न्तेऽव‚तिष्ठेते--नित्यो वाऽनित्यो वेत्यादिरूपेण । \textbf{स प्र‚योज‚न‚म‚स्य} स त‚था । न‚{\tiny $_{lb}$}‚ त‚थाऽ\textbf{नैकान्तिकः} ॥
	\pend% ending standard par
      ‚{\tiny $_{lb}$}‚

	  \pstart \leavevmode% starting standard par
	\textbf{विप‚क्षेऽस‚त्त्व‚म‚सिद्धं} स‚र्व‚त्रेति द्र‚ष्ट‚व्य‚म् । \leavevmode\ledsidenote{\textenglish{68b/ms}} स‚द‚कार‚ण‚व‚न्नित्य‚म् \href{http://sarit.indology.info/?cref=vsū.4.1.1}{वै० सू० ४. 	१. १.} इति नित्य‚ल‚क्ष‚ण‚योगान्नित्या इष्टाः \textbf{प‚र‚माण‚वः} ॥
	\pend% ending standard par
      ‚{\tiny $_{lb}$}‚‚{\tiny $_{lb}$}‚\textsuperscript{\textenglish{197/dm}}‚{\tiny $_{lb}$}‚
	  \bigskip
	  \begingroup
	

	  \pstart \leavevmode% starting standard par
	अस्य च‚तुर्विध‚स्य प‚क्ष‚ध‚र्म‚स्यास‚त्त्व‚म‚सिद्धं विप‚क्षे । त‚तोऽनैकान्तिक‚ता ॥
	\pend% ending standard par
       ‚{\tiny $_{lb}$}‚ 
	  \bigskip
	  \begingroup
	

	  \pstart \leavevmode% starting standard par
	त‚था--अस्यैव रूप‚स्य संदेहेऽप्य‚नैकान्तिक एव ॥ ६८ ॥
	\pend% ending standard par
      
	  \endgroup
	‚{\tiny $_{lb}$}‚ 

	  \pstart \leavevmode% starting standard par
	य‚था चास्य रूप‚स्यासिद्धाव‚नैकान्तिक‚स्त‚था अस्यैव विप‚क्षेऽस‚त्त्वाख्य‚स्य\edtext{}{\lemma{स्य}\Bfootnote{०ख्य‚रूप‚स्य \cite{dp-msC}}} रूप‚स्य संदेहे‚{\tiny $_{lb}$}‚ऽनैकान्तिकः ॥
	\pend% ending standard par
       ‚{\tiny $_{lb}$}‚ 

	  \pstart \leavevmode% starting standard par
	त‚मुदाह‚र‚ति--
	\pend% ending standard par
       ‚{\tiny $_{lb}$}‚ 
	  \bigskip
	  \begingroup
	

	  \pstart \leavevmode% starting standard par
	य‚थाऽस‚र्व‚ज्ञः क‚श्चिद्विव‚क्षितः पुरुषो रागादिमान् वेति साध्ये व‚क्तृत्वादिको‚{\tiny $_{lb}$}‚ ध‚र्मः स‚न्दिग्ध‚विप‚क्ष‚व्यावृत्तिकः ॥ ६९ ॥
	\pend% ending standard par
      
	  \endgroup
	‚{\tiny $_{lb}$}‚ 

	  \pstart \leavevmode% starting standard par
	य‚थेति । अस‚र्व‚ज्ञ इत्य‚स‚र्व‚ज्ञ‚त्वं साध्य‚म् । क‚श्चिद्विव‚क्षित इति व‚क्तुर‚भिप्रेतः पुरुषो‚{\tiny $_{lb}$}‚ ध‚र्मीं । रागा आदिर्य‚स्य द्वेषादेः स रागादिः । स य‚स्यास्ति स रागादिमान् इति द्वितीयं‚{\tiny $_{lb}$}‚ साध्य‚म् । \edtext{\textsuperscript{*}}{\lemma{*}\Bfootnote{वेति ग्र‚ह० \cite{dp-msC} \cite{dp-msD}}}वाग्र‚ह‚णं रागादिम‚त्त्व‚स्य पृथ‚क्साध्य‚त्व‚ख्याप‚नार्थ‚म् । त‚तोऽस‚र्व‚ज्ञ‚त्वे रागादिम‚त्त्वे‚{\tiny $_{lb}$}‚ वा\edtext{}{\lemma{वा}\Bfootnote{म‚त्त्वे च साध्ये \cite{dp-msC}}} साध्ये प्र‚कृते व‚क्तृत्वं--व‚च‚न‚श‚क्तिस्त‚दादिर्य‚स्योन्मेष‚निमेषादेः स व‚क्तृत्वादिको ध‚र्मोऽ‚{\tiny $_{lb}$}‚नैकान्तिकः ।
	\pend% ending standard par
       ‚{\tiny $_{lb}$}‚ 

	  \pstart \leavevmode% starting standard par
	स‚न्दिग्धा\edtext{}{\lemma{न्दिग्धा}\Bfootnote{संदिग्ध‚वि० \cite{dp-msB}}} विप‚क्षा व्यावृत्तिर्य‚स्य स त‚थोक्तः । अस‚र्व‚ज्ञ‚त्वे साध्ये स‚र्व‚ज्ञ‚त्वं विप‚क्षः ।‚{\tiny $_{lb}$}‚ त‚त्र व‚च‚नादेः स‚त्त्व‚म‚स‚त्त्वं वा स‚न्दिग्ध‚म् । अतो न ज्ञाय‚ते किं\edtext{}{\lemma{किं}\Bfootnote{किं नास्ति \cite{dp-msA} \cite{dp-msB} \cite{dp-msD} \cite{dp-edP} \cite{dp-edH} \cite{dp-edE} \cite{dp-edN}}} व‚क्ता स‚र्व‚ज्ञ उतास‚र्व‚ज्ञ‚{\tiny $_{lb}$}‚ इत्य‚नैकान्तिकं व‚क्तृत्व‚म् ।
	\pend% ending standard par
       ‚{\tiny $_{lb}$}‚ 

	  \pstart \leavevmode% starting standard par
	न‚नु च स‚र्व‚ज्ञो व‚क्ता नोप‚ल‚भ्य‚ते त‚त्क‚थं व‚च‚नं स‚र्व‚ज्ञे स‚न्दिग्ध‚म् ? अत एव
	\pend% ending standard par
       ‚{\tiny $_{lb}$}‚ 
	  \bigskip
	  \begingroup
	

	  \pstart \leavevmode% starting standard par
	\edtext{\textsuperscript{*}}{\lemma{*}\Bfootnote{स‚र्व‚त्रैक‚देशे वा स‚र्व‚ज्ञो \cite{dp-msB} \cite{dp-edP} \cite{dp-edH}}}स‚र्व‚ज्ञो व‚क्ता नोप‚ल‚भ्य‚ते इत्येवंप्र‚कार‚स्यानुप‚ल‚भ्भ‚स्यादृश्यात्म‚विष‚य‚त्वोन‚{\tiny $_{lb}$}‚ \edtext{\textsuperscript{*}}{\lemma{*}\Bfootnote{संदेहे हेतु \cite{dp-msB} \cite{dp-edP} \cite{dp-edH} \cite{dp-edE}}}संदेह‚हेतुत्वाद् ।\edtext{\textsuperscript{*}}{\lemma{*}\Bfootnote{त्वाद‚स‚र्व० \cite{dp-msB} \cite{dp-msD} \cite{dp-edP} \cite{dp-edH} \cite{dp-edE}}} त‚तोऽस‚र्व‚ज्ञ‚विप‚र्य‚याद्व‚क्तृत्वादेर्व्यावृत्तिः स‚न्दिग्धा ॥ ७० ॥
	\pend% ending standard par
      
	  \endgroup
	‚{\tiny $_{lb}$}‚ 

	  \pstart \leavevmode% starting standard par
	\hphantom{.}स‚र्व‚ज्ञो व‚क्ता नोप‚ल‚भ्य‚ते इत्येव‚म्प्र‚कार‚स्य--\edtext{\textsuperscript{*}}{\lemma{*}\Bfootnote{०जातीय‚क‚स्य--\cite{dp-msB}}}एवंजातीय‚स्यानुप‚ल‚म्भ‚स्य संदेह‚हेतुत्वात् ।‚{\tiny $_{lb}$}‚ कुत इत्याह--अदृश्य\edtext{}{\lemma{अदृश्य}\Bfootnote{अदृश्यात्मा \cite{dp-msA} \cite{dp-msB} \cite{dp-edP} \cite{dp-edH} \cite{dp-edE} \cite{dp-edN}}} आत्मा विष‚यो य‚स्य त‚स्य भावोऽदृश्यात्म‚विष‚य‚त्वं तेन स‚न्देह‚हेतुत्व‚म् ।
	\pend% ending standard par
      
	  \endgroup
	‚{\tiny $_{lb}$}‚

	  \pstart \leavevmode% starting standard par
	\textbf{द्वेषादे}रित्यादिग्र‚ह‚णेन मोहादेर्ग्र‚ह‚ण‚म् ॥
	\pend% ending standard par
      ‚{\tiny $_{lb}$}‚

	  \pstart \leavevmode% starting standard par
	\textbf{अत एवा}नुप‚ल‚म्भ‚मात्रादेवेति सिद्धान्ती । अमुमेवार्थं मूलेन संस्य‚न्द‚य‚न्नाह--\textbf{स‚र्व‚ज्ञ} इति ।‚{\tiny $_{lb}$}‚ \textbf{तेना}दृश्य‚विष‚य‚त्वेन हेतुना \textbf{स‚न्देह‚हेतुत्वं} स‚न्देह‚हेतुत्वादिति । हेतुप‚ञ्च‚मीमिदानीं व्याच‚ष्टे—‚{\tiny $_{lb}$}‚य‚त इति ॥
	\pend% ending standard par
      ‚{\tiny $_{lb}$}‚‚{\tiny $_{lb}$}‚\textsuperscript{\textenglish{198/dm}}‚{\tiny $_{lb}$}‚
	  \bigskip
	  \begingroup
	

	  \pstart \leavevmode% starting standard par
	य‚तोऽदृश्य‚विष‚योऽनुप‚ल‚म्भः\edtext{}{\lemma{म्भः}\Bfootnote{संश‚य‚हे० \cite{dp-msA} \cite{dp-edP} \cite{dp-edH} \cite{dp-edE} \cite{dp-edN}}} स‚न्देह‚हेतुर्न निश्च‚य‚हेतुस्त‚तोऽस‚र्व‚ज्ञ‚विप‚क्षात् स‚र्व‚ज्ञाद् व‚क्तृत्वा‚{\tiny $_{lb}$}‚देर्व्यावृत्तिः स‚न्दिग्धा ॥
	\pend% ending standard par
       ‚{\tiny $_{lb}$}‚ 

	  \pstart \leavevmode% starting standard par
	नानुप‚ल‚म्भात् \edtext{}{\lemma{म्भात्}\Bfootnote{०भात् संदिग्धे व‚क्तृ० \cite{dp-msB}}}स‚र्व‚ज्ञे व‚क्तृत्व‚म‚स‚द्ब्रूमः । अपि तु स‚र्वंज्ञ‚त्वेन स‚ह व‚क्तृत्व‚स्य विरोधात् ।‚{\tiny $_{lb}$}‚ एत‚न्न\edtext{}{\lemma{न्न}\Bfootnote{एत‚न्न नास्ति \cite{dp-edE}}} ।
	\pend% ending standard par
       ‚{\tiny $_{lb}$}‚ 
	  \bigskip
	  \begingroup
	

	  \pstart \leavevmode% starting standard par
	व‚क्तृत्व‚स‚र्व‚ज्ञ‚त्व‚योर्विरोधाभावाच्च यः स‚र्व‚ज्ञः स व‚क्ता न भ‚व‚तीत्य‚{\tiny $_{lb}$}‚द‚र्श‚नेपि व्य‚तिरेको न सिध्य‚ति, संदेहात् ॥ ७१ ॥
	\pend% ending standard par
      
	  \endgroup
	‚{\tiny $_{lb}$}‚ 

	  \pstart \leavevmode% starting standard par
	स‚र्व‚ज्ञ‚त्व‚व‚क्तृत्व‚योर्विरोधो नास्ति । विरोधाभावाच्च कार‚णाद् व्य‚तिरेको न सिध्य‚ति—‚{\tiny $_{lb}$}‚इति संब‚न्धः ।
	\pend% ending standard par
       ‚{\tiny $_{lb}$}‚ 

	  \pstart \leavevmode% starting standard par
	व्याप्तिम‚न्तं व्य‚तिरेकं द‚र्श‚य‚ति--यः स‚र्व‚ज्ञ इति । साध्याभाव‚रूपं स‚र्व‚ज्ञ‚त्व‚म‚नूद्य‚{\tiny $_{lb}$}‚ \edtext{\textsuperscript{*}}{\lemma{*}\Bfootnote{न स व‚क्ता भ‚व‚ति \cite{dp-msA} \cite{dp-edP} \cite{dp-edH} \cite{dp-edE} \cite{dp-edN}}} स व‚क्ता न भ‚व‚ति इति साध‚न‚स्य व‚क्तृत्व‚स्याभावो विधीय‚ते । तेन साध्याभावः साध‚नाभावे‚{\tiny $_{lb}$}‚ निय‚त‚त्वात् \edtext{}{\lemma{त्वात्}\Bfootnote{साध‚र्म्य‚भावेन \cite{dp-msB}}}साध‚नाभावेन व्याप्त उक्त इति । व्याप्तिमानीदृशो व्य‚तिरेको विरोधे स‚ति‚{\tiny $_{lb}$}‚ व‚क्तृत्व‚स‚र्व‚ज्ञ‚त्व‚योः सिध्येत् । न चास्ति विरोधः । त‚स्मान्न सिध्य‚तीति\edtext{}{\lemma{तीति}\Bfootnote{सिध्य‚ति । कुतः \cite{dp-msA} \cite{dp-edP} \cite{dp-edH} \cite{dp-edE} \cite{dp-edN}}} । कुत इत्याह—‚{\tiny $_{lb}$}‚संदेहात् । य‚तो विरोधाभावः, त‚स्मात् संदेहः । स‚न्देहाद् व्य‚तिरेकासिद्धिः ॥
	\pend% ending standard par
       ‚{\tiny $_{lb}$}‚ 

	  \pstart \leavevmode% starting standard par
	क‚थं विरोधाभावः ?
	\pend% ending standard par
       ‚{\tiny $_{lb}$}‚ 
	  \bigskip
	  \begingroup
	

	  \pstart \leavevmode% starting standard par
	द्विविधो हि प‚दार्थानां विरोधः ॥ ७२ ॥
	\pend% ending standard par
      
	  \endgroup
	‚{\tiny $_{lb}$}‚ 

	  \pstart \leavevmode% starting standard par
	\edtext{\textsuperscript{*}}{\lemma{*}\Bfootnote{हिर्य‚स्माद‚र्थेद्विवि० \cite{dp-msD} हिर्य‚स्मात्--\cite{dp-msB}}}हीति य‚स्माद् द्विविध एव विरोधो नान्यः, त‚स्मान्न व‚क्तृत्व‚स‚र्वंज्ञ‚त्व‚योर्विंरोधः ॥
	\pend% ending standard par
       ‚{\tiny $_{lb}$}‚ 

	  \pstart \leavevmode% starting standard par
	कः पुन‚र‚सौ द्विविधो विरोध इत्याह--
	\pend% ending standard par
       ‚{\tiny $_{lb}$}‚ 
	  \bigskip
	  \begingroup
	

	  \pstart \leavevmode% starting standard par
	अविक‚ल‚कार‚ण‚स्य भ‚व‚तोऽन्य‚भावेऽभावाद्\edtext{}{\lemma{भावेऽभावाद्}\Bfootnote{०भावः । अभा० \cite{dp-msB} \cite{dp-edP} \cite{dp-edH}}} विरोध‚ग‚तिः \edtext{}{\lemma{तिः}\Bfootnote{ग‚तिरिति--\cite{dp-msC}}}॥ ७३ ॥
	\pend% ending standard par
      
	  \endgroup
	‚{\tiny $_{lb}$}‚ 

	  \pstart \leavevmode% starting standard par
	अविक‚ल‚कार‚ण‚स्येति । अविक‚लानि स‚म‚ग्राणि कार‚णानि य‚स्य स त‚थोक्तः । य‚स्य‚{\tiny $_{lb}$}‚ कार‚ण‚वैक‚ल्याद‚भावो न त‚स्य केन‚चिद‚पि विरोध‚ग‚तिः । त‚द‚र्थ‚म् अविक‚ल‚कार‚ण‚ग्र‚ह‚ण‚म् ।
	\pend% ending standard par
       ‚{\tiny $_{lb}$}‚ 

	  \pstart \leavevmode% starting standard par
	न‚नु च य‚स्यापि कार‚ण‚साक‚ल्यं त‚स्यापि निवृत्तिर‚श‚क्या केन‚चिद‚पि क‚र्त्तुम् । त‚त्‚{\tiny $_{lb}$}‚ कुतो विरोध‚ग‚तिः ? एवं त‚र्हि अविक‚ल‚कार‚ण‚स्यापि य‚त्कृतात् कार‚ण‚वैक‚ल्याद् अभाव‚स्तेन‚{\tiny $_{lb}$}‚ विरोध‚ग‚तिः ।
	\pend% ending standard par
      
	  \endgroup
	‚{\tiny $_{lb}$}‚

	  \pstart \leavevmode% starting standard par
	\textbf{य‚स्ये}त्यादिनाऽविक‚ल‚कार‚ण‚स्य फ‚लं व‚र्ण‚य‚ति । एवं त‚र्हीत्युत्त‚र‚म् । त‚र्हि त‚स्मिन्‚{\tiny $_{lb}$}‚ काले । \textbf{एवं} बोद्ध‚व्य‚मित्य‚र्थः । \textbf{अपि} स‚म्भाव‚नायाम् । न्याय‚ब‚लादेवं स‚म्भाव‚याम इत्य‚र्थः ।‚{\tiny $_{lb}$}‚ \textbf{विरोध}स्य ग‚तिः प्र‚तिप‚त्तिः ।
	\pend% ending standard par
      ‚{\tiny $_{lb}$}‚\textsuperscript{\textenglish{199/dm}}‚{\tiny $_{lb}$}‚
	  \bigskip
	  \begingroup
	

	  \pstart \leavevmode% starting standard par
	त‚था च स‚ति यो य‚स्य विरुद्धः स त‚स्य किञ्चित्क‚र एव । त‚था हि--शीत‚स्प‚र्श‚स्य‚{\tiny $_{lb}$}‚ ज‚न‚को भूत्वा शीत‚स्प‚र्शान्त‚र‚ज‚न‚न‚श‚क्तिं प्र‚तिब‚ध्न‚न् शीत‚स्प‚र्श‚स्य निव‚र्त्त‚को विरुद्धः । त‚स्माद्धेतु‚{\tiny $_{lb}$}‚वैक‚ल्य‚कारी विरुद्धो ज‚न‚क एक निव‚र्त्त्य‚स्य । स‚हान‚व‚स्थान‚विरोध‚श्चाय‚म् । त‚तो विरुद्ध‚यो‚{\tiny $_{lb}$}‚रेक‚स्मिन्न‚पि क्ष‚णे स‚हाव‚स्थानं प‚रिह‚र्त्त‚व्य‚म् । दूर‚स्थ‚योर्विरोधाभावाच्च निक‚ट‚स्थ‚योरेव‚{\tiny $_{lb}$}‚ निव‚र्त्त्य‚निव‚र्त्त‚क‚भावः ।
	\pend% ending standard par
       ‚{\tiny $_{lb}$}‚ 

	  \pstart \leavevmode% starting standard par
	त‚स्माद्यो य‚स्य निव‚र्त्त‚कः स तं य‚दि प‚रं तृतीये क्ष‚णे निव‚र्त्त‚य‚ति । प्र‚थ‚मे क्ष‚णे स‚न्निप‚त‚{\tiny $_{lb}$}‚न्न‚स‚म‚र्थाव‚स्थाधान‚योग्यो\edtext{}{\lemma{योग्यो}\Bfootnote{स्थाधाने योग्यो \cite{dp-msD} ०स्थान‚योग्यो \cite{dp-msA} \cite{dp-msB} \cite{dp-edP} \cite{dp-edH} \cite{dp-edE}}} भ‚व‚ति । द्वितीये विरुद्ध‚म‚स‚म‚र्थं क‚रोति । तृतीये त्व‚स‚म‚र्थे‚{\tiny $_{lb}$}‚ निवृत्ते त‚द्देश‚माक्राम‚ति ।
	\pend% ending standard par
       ‚{\tiny $_{lb}$}‚ 

	  \pstart \leavevmode% starting standard par
	त‚त्रालोको ग‚तिध‚र्मा क्र‚मेण ज‚ल‚त‚र‚ङ्ग‚न्यायेन \edtext{}{\lemma{न्यायेन}\Bfootnote{त‚द्देश‚मा० \cite{dp-msC} \cite{dp-msD}}}देश‚माक्राम‚न्\edtext{}{\lemma{न्}\Bfootnote{०माक्राम‚य‚न् \cite{dp-msB} \cite{dp-edN}}} य‚दाऽन्ध‚कार‚निर‚न्त‚र-\edtext{\textsuperscript{*}}{\lemma{*}\Bfootnote{कारे निर० \cite{dp-msA} \cite{dp-msB} \cite{dp-edP} \cite{dp-edH} \cite{dp-edE} \cite{dp-edN}}}‚{\tiny $_{lb}$}‚ मालोक‚क्ष‚णं ज‚न‚य‚ति त‚दाऽऽलोक‚स‚मीप‚व‚र्त्तिन‚म‚न्ध‚कार‚म‚स‚म‚र्थं ज‚न‚य‚ति । त‚तोऽसाम‚र्थ्यं त‚स्प य‚स्य‚{\tiny $_{lb}$}‚ स‚मीप‚व‚र्त्त्यालोकः । \edtext{\textsuperscript{*}}{\lemma{*}\Bfootnote{अस‚म‚र्थ्ये \cite{dp-msA} असाम‚र्थ्ये \cite{dp-edP} \cite{dp-edH} \cite{dp-edE} \cite{dp-edN}}}अस‚म‚र्थे निवृत्ते \edtext{}{\lemma{निवृत्ते}\Bfootnote{तादृशो \cite{dp-msA} \cite{dp-msB} \cite{dp-edP} \cite{dp-edH} \cite{dp-edN}}}त‚द्देशो जाय‚त आलोक इत्येवं क्र‚मेणाऽऽलोकेनान्ध‚{\tiny $_{lb}$}‚कारोऽप‚नेयः । त‚थोष्ण‚स्प‚र्शेन शीत‚स्प‚र्शो निव‚र्त्त‚नीयः ।
	\pend% ending standard par
      
	  \endgroup
	‚{\tiny $_{lb}$}‚

	  \pstart \leavevmode% starting standard par
	किम‚तः सिद्ध‚मित्याह--त‚था चेति कार‚ण‚वैक‚ल्य‚कारिणो विरोधाव‚ग‚म‚प्र‚कारे स‚ति ।‚{\tiny $_{lb}$}‚ किञ्चित्क‚र‚त्व‚मेव त‚था हीत्यादिना द‚र्श‚य‚ति । य‚था चास्य ज‚न‚क‚त्वं त‚थाऽन‚न्त‚र‚मेव व्य‚क्ती‚{\tiny $_{lb}$}‚क‚रिष्य‚ते ।
	\pend% ending standard par
      ‚{\tiny $_{lb}$}‚

	  \pstart \leavevmode% starting standard par
	न‚नु किं क‚तिप‚य‚क्ष‚ण‚स‚हित‚योः प‚श्चान्निव‚र्त्त्य‚निव‚र्त्त‚क‚भावेन विरोधोऽथ‚वाऽन्य‚थेत्या‚{\tiny $_{lb}$}‚श‚ङ्क्याह--स‚हेति । \textbf{चो} य‚स्मात् त‚त‚स्त‚स्मात् । न केव‚लं ब‚हुषु क्ष‚णेष्वित्य‚पि श‚ब्दः ।‚{\tiny $_{lb}$}‚ \textbf{स‚हाव‚स्थान‚मे}क‚त्र स्थितिः । निक‚टाव‚स्थानं तु न प‚रिह‚र्त्त‚व्य‚मिति बुद्धिस्थ‚म् । \textbf{प‚रिह‚र्त्त‚व्यं} नाङ्गी‚{\tiny $_{lb}$}‚क‚र्त्त‚व्य‚म् । त‚योरेक‚स्मिन्न‚पि क्ष‚णे स‚ह‚स्थित्य‚भावात् क‚थ‚मेव‚म‚ङ्गीक्रिय‚ते ? अत एव । न‚{\tiny $_{lb}$}‚ स‚ह‚स्थित‚योः प‚श्चाद् विरोध इति वा कृत‚म‚नेन ।
	\pend% ending standard par
      ‚{\tiny $_{lb}$}‚

	  \pstart \leavevmode% starting standard par
	य‚द्येवं क्व‚चित्प्र‚देशे व‚र्त्त‚मान आलोक‚स्त्रिलोकीव्य‚व‚स्थितानि त‚मांस्य‚नेनैव क्र‚मेणाप‚न‚ये‚{\tiny $_{lb}$}‚दिति न क्व‚चित् त‚मांस्य‚व‚तिष्ठेर‚न्नित्याह--\textbf{निक‚ट‚स्थ‚योरि}ति य‚योर्निव‚र्त्त्य‚निव‚र्त्त‚क‚भावो दृष्ट‚{\tiny $_{lb}$}‚स्त‚योर्निक‚ट‚स्थ‚योरेव न तु निक‚ट‚स्थ‚योर‚व‚श्यं निव‚र्त्त्य‚निर्व‚र्त्त‚क‚भाव इत्य‚स्यार्थो द्र‚ष्ट‚व्यः । त‚योरेव‚{\tiny $_{lb}$}‚ क‚थं त‚थाभाव इत्याश‚ङ्कायां \textbf{दूर‚स्थायोरि}ति योज्य‚म् । \textbf{चो}ऽव‚धार‚णे । य‚तः किञ्चित्क‚र‚स्यैव‚{\tiny $_{lb}$}‚ निव‚र्त्त‚क‚त्वं \textbf{त‚स्माद्} हेतोः \textbf{प‚रं} प्र‚कृष्टं य‚था भ‚व‚ति । एत‚देवोप‚पाद‚य‚न्नाह--\textbf{प्र‚थ‚म} इति \textbf{स‚न्निय‚त‚{\tiny $_{lb}$}‚न्नि}क‚टीभ‚व‚न्निव‚र्त्त‚क इति प्र‚क‚र‚णात् । \textbf{अस‚म‚र्था} चोपादेय‚क्ष‚ण‚निर्माणे अश‚क्ता\textbf{व‚स्था} य‚स्यान्ध‚{\tiny $_{lb}$}‚कार‚क्ष‚ण‚स्य त‚स्याऽऽ\textbf{धान‚मु}त्पाद‚न‚म्, त‚त्र \textbf{योग्यः} स‚म‚र्थो भ‚व‚ति । \textbf{द्वितीये} क्ष‚णे इत्य‚नुव‚र्त्त‚ते ।‚{\tiny $_{lb}$}‚ \textbf{विरुद्ध‚म}न्ध‚कार\textbf{म‚स‚म‚र्थं} स‚जातीय‚क्ष‚णान्त‚र‚ज‚न‚नाक्ष‚मं क‚रोति । तृतीये क्ष‚णेऽऽ\textbf{स‚म‚र्थे} त‚स्मिन्‚{\tiny $_{lb}$}‚ ‚{\tiny $_{lb}$}‚ \leavevmode\ledsidenote{\textenglish{200/dm}}‚{\tiny $_{lb}$}‚ 
	  
	य‚दा त्वालोक‚स्त‚त्रैवान्ध‚कार‚देशे ज‚न्य‚ते त‚दा य‚तः क्ष‚णाद‚न्ध‚कार‚देश‚स्यालोक‚स्य ज‚न‚कः\edtext{}{\lemma{कः}\Bfootnote{ज‚न‚क‚क्ष‚णः \cite{dp-msA} \cite{dp-msB} \cite{dp-edP} \cite{dp-edH} \cite{dp-edE} \cite{dp-edN}}}‚{\tiny $_{lb}$}‚ क्ष‚णः उत्प‚द्य‚ते त‚त एवान्ध‚कारोऽन्ध‚कारान्त‚र‚ज‚न‚नास‚म‚र्थ\edtext{}{\lemma{र्थ}\Bfootnote{अन्ध‚कारान्त‚रास‚म‚र्थः \cite{dp-msA} अन्ध‚कारान्त‚राज‚न‚नास‚म‚र्थः \cite{dp-msB} \cite{dp-msC}}} उत्प‚न्नः । \edtext{\textsuperscript{*}}{\lemma{*}\Bfootnote{त‚तोऽसाम‚र्थ्याव० \cite{dp-msB}}}त‚तोऽस‚म‚र्थाव‚स्था‚{\tiny $_{lb}$}‚ज‚न‚क‚त्व‚मेव निव‚र्त्त‚क‚त्व‚म् । ‚{\tiny $_{lb}$}‚ 
	  
	अत‚श्च य‚स्मिन् क्ष‚णे ज‚न‚क‚स्त‚त‚स्तृतीये क्ष‚णे निवृत्तो विरुद्धो य‚दि शीघ्रं निव‚र्त्त‚ते । ‚{\tiny $_{lb}$}‚ 
	  
	ज‚न्य‚ज‚न‚क‚भाव‚च्च \edtext{}{\lemma{च्च}\Bfootnote{स‚न्त‚न‚योः \cite{dp-msA}}}स‚न्तान‚योर्विरोधो न क्ष‚ण‚योः । य‚द्य‚पि च न स‚न्तानो नाम व‚स्तु‚{\tiny $_{lb}$}‚ त‚थापि स‚न्तानिनो व‚स्तुभूताः । त‚तोऽयं प‚र‚मार्थः--न क्ष‚ण‚योर्विरोधः । अपि तु ब‚हूनां‚{\tiny $_{lb}$}‚ व‚न्ध्य‚क्ष‚णे \textbf{निवृत्ते} स्व‚र‚स‚तो निरुद्धे त‚द्देशं त‚स्यास‚म‚र्थ‚क्ष‚ण‚स्य देशं स्थान‚मा\textbf{क्राम‚ति,} त‚द्देशो‚{\tiny $_{lb}$}‚ भ‚व‚ति निव‚र्त्त‚क इत्य‚र्थात् ।
	\pend% ending standard par
      ‚{\tiny $_{lb}$}‚

	  \pstart \leavevmode% starting standard par
	इह क‚श्चिन्निव‚र्त्त‚क आलोको यामेव दिश‚माक्राम‚ति त‚द्दिग्व‚र्त्तिन‚मेव स्व‚विरुद्धं ग‚ति‚{\tiny $_{lb}$}‚क्र‚मेणैव निव‚र्त्त‚य‚ति । क‚श्चित्पुन‚र्विरुद्धाव‚ष्ट‚ब्ध एव देशे स‚मुत्प‚न्न‚मात्र एवानेक‚दिग्व‚र्त्तिनं विरुद्धं‚{\tiny $_{lb}$}‚ झ‚टिति निव‚र्त्त‚य‚ति । त‚त्र न ज्ञाय‚ते क‚स्य क‚थं किञ्चित्क‚र‚त‚या निव‚र्त्त‚क‚त्व‚मित्याह--\textbf{त‚त्रेति}‚{\tiny $_{lb}$}‚ वाक्योप‚क्षेपे । देश‚म‚भिम‚तं स्थान‚मा\textbf{क्रामं}स्त‚द्देशो भ‚व‚न्नालोक इत्य‚र्थः । \textbf{अन्ध‚कार‚निर‚न्त‚र‚{\tiny $_{lb}$}‚म}न्ध‚काराव्य‚व‚हित‚म् । \textbf{आलोक‚स‚मीप‚व‚र्त्तिन‚मि}ति त‚ज्ज‚न्य‚मानालोक‚स‚मीप‚व‚र्त्तिन‚म् । \textbf{अस‚म‚र्थ}‚{\tiny $_{lb}$}‚म‚न्ध‚कारान्त‚र‚ज‚न‚नाश‚क्तं ज‚न‚य‚ति । य‚त एवं त‚त‚स्त‚स्य ज‚न‚क‚त्व‚म् । त‚त‚स्त‚स्मात्स‚मीप‚{\tiny $_{lb}$}‚\textbf{व‚र्त्त्यालोक} इति ग‚तिध‚र्मेति द्र‚ष्ट‚व्य‚म् । \textbf{असाम‚र्थ्यं} चो\leavevmode\ledsidenote{\textenglish{69a/ms}}पादेय‚क्ष‚णोप‚ज‚न‚नं प्र‚तीति‚{\tiny $_{lb}$}‚ प्र‚स्तावाद‚व‚सेय‚म् । \textbf{अस‚म‚र्थे} त‚स्मिन्न‚न्ध‚कारे \textbf{निवृत्ते} स्व‚र‚स‚तोवि\edtext{}{\lemma{तोवि}\Bfootnote{नि}}रुद्धे स‚ति । सोऽस‚म‚र्थान्ध‚{\tiny $_{lb}$}‚कार‚क्ष‚ण‚देशो देशो य‚स्य स त‚था जाय‚ते \textbf{आलोकः} । इतिस्त‚स्मात् । \textbf{एव‚म}न‚न्त‚रोक्तेन \textbf{क्र‚मेण}‚{\tiny $_{lb}$}‚ प‚रिपाट्या \textbf{ग‚तिध‚र्मालोक‚स्}त‚द्देशाक्र‚म‚णाय स‚न्निप‚त‚न्न‚स‚म‚र्थाव‚स्थाऽऽधान‚योग्यो भ‚व‚ति । द्वितीये‚{\tiny $_{lb}$}‚ क्ष‚णेऽस‚म‚र्थं ज‚न‚य‚ति । तृतीये त‚द्देशो जाय‚त इत्य‚न‚न्त‚रोक्तः क्र‚मो विभ‚ज्य योज‚नीयः ।
	\pend% ending standard par
      ‚{\tiny $_{lb}$}‚

	  \pstart \leavevmode% starting standard par
	अमुमेव क्र‚म‚म‚न्य‚त्रादिश‚न्नाह--\textbf{त‚थे}ति । य‚था--प्रालोकान्ध‚कार‚योर्निव‚र्त्त्य‚निव‚र्त्त‚क‚{\tiny $_{lb}$}‚भाव‚स्तेन प्र‚कारेण \textbf{उष्ण‚स्प‚र्शेन} ग‚तिध‚र्मेण दृष्टान्त‚व‚शाद् द्र‚ष्ट‚व्य‚म् ।
	\pend% ending standard par
      ‚{\tiny $_{lb}$}‚

	  \pstart \leavevmode% starting standard par
	ग‚तिध‚र्म‚ण‚स्ताव‚दालोक‚स्यायं क्र‚मः । विरुद्धाक्रान्त‚देश‚म‚ध्योत्प‚न्न‚स्य कीदृश इत्याह—‚{\tiny $_{lb}$}‚य‚देति । तुर्विशेष‚णार्थः । \textbf{य‚तः क्ष‚णादालोक‚स्य ज‚न‚कः क्ष‚ण उत्प‚द्य‚ते} । कीदृश‚स्यालोक‚{\tiny $_{lb}$}‚क्ष‚ण‚स्येत्याह--\textbf{अन्ध‚कारे}ति । \textbf{अन्ध‚कार‚देश‚स्य} निव‚र्त्त्याऽन्ध‚कार‚स‚म्ब‚न्धी देशो य‚स्य स त‚था‚{\tiny $_{lb}$}‚ \textbf{त‚त एव} त‚मोदेशालोकोत्पाद‚क्ष‚ण‚योरेक‚साम‚ग्र्य‚धीन‚तामाह । य‚तोऽन्ध‚कार‚देशालोक‚हेतूत्पाद‚क‚स्य‚{\tiny $_{lb}$}‚ क्ष‚ण‚स्य व‚न्ध्यान्ध‚काराधाय‚क‚त्व‚तो हेतोर‚विद्य‚मानं स‚जातीय‚ज‚न्म‚नि साम‚र्थ्यं य‚स्या अव‚स्थाया‚{\tiny $_{lb}$}‚ अन्ध‚कार‚स‚म्ब‚न्धिन्याः सा त‚था । त‚ज्ज‚न‚क‚त्व‚मेवालोक‚स्येति प्र‚क‚र‚णात् ।
	\pend% ending standard par
      ‚{\tiny $_{lb}$}‚

	  \pstart \leavevmode% starting standard par
	अत्रापि तृतीये क्ष‚णे प‚रं निव‚र्त्त‚क‚त्व‚मिति द‚र्श‚य‚न्नाह--\textbf{अत‚श्चे}ति । \textbf{चो}ऽव‚धार‚णे । अत्रापि‚{\tiny $_{lb}$}‚ \textbf{प्र‚थ‚मे क्ष‚णे}ऽन्ध‚कार‚देशालोक‚हेतूत्पाद‚कः क्ष‚णः स‚मुद्भ‚व‚न्नेवान्ध‚कारास‚म‚र्थाव‚स्थात‚द्देशालोक‚हेतुज‚न‚न‚{\tiny $_{lb}$}‚‚{\tiny $_{lb}$}‚ ‚{\tiny $_{lb}$}‚ \leavevmode\ledsidenote{\textenglish{201/dm}}‚{\tiny $_{lb}$}‚ 
	  
	क्ष‚णानाम् । य‚तः स‚त्सु द‚ह‚न‚क्ष‚णेषु प्र‚वृत्ता अपि शीत‚क्ष‚णा निवृत्तिध‚र्माणो भ‚व‚न्तीति स‚न्तान‚यो‚{\tiny $_{lb}$}‚र्निव‚र्त्त्य‚निव‚र्त्त‚क‚त्व‚निमित्ते च विरोधे स्थिते स‚र्वेषां प‚र‚माणूनां स‚त्य‚प्येक‚देशाव‚स्थानाभावे न‚{\tiny $_{lb}$}‚ विरोधः, इत‚रेत‚र‚स‚न्तानानिव‚र्त्त‚नात् तेषाम् । ग‚तिध‚र्मा चालोको यां \edtext{}{\lemma{यां}\Bfootnote{दिशं क्राम० \cite{dp-msC}}}दिश‚माक्राम‚ति‚{\tiny $_{lb}$}‚ \edtext{\textsuperscript{*}}{\lemma{*}\Bfootnote{त‚द्विव‚र्त्तिनः \cite{dp-msA}}}त‚द्दिग्व‚र्त्तिनो विरोधिस‚न्तानान् निव‚र्त्त‚य‚ति । त‚तोऽप‚व‚र‚कैक‚देश‚स्था प्र‚दीप‚प्र‚भाऽन्ध‚कार‚निक‚ट‚{\tiny $_{lb}$}‚व‚र्त्तिव्य‚पि नान्ध‚कारं निव‚र्त्त‚य‚ति, \edtext{\textsuperscript{*}}{\lemma{*}\Bfootnote{अन्ध‚कारायाक्रान्तायां \cite{dp-msA}}}अन्ध‚काराक्रान्तायां दिश्यालोक‚क्ष‚णान्त‚र‚ज‚न‚नासाम‚र्थ्यात् ।‚{\tiny $_{lb}$}‚ कार‚णासाम‚र्थ्य‚हेतुत्व‚कृतं\edtext{}{\lemma{कृतं}\Bfootnote{०हेतुकृतं--\cite{dp-msA} \cite{dp-edP} \cite{dp-edH} \cite{dp-edE} \cite{dp-edN}}} स‚न्तान‚निष्ठ‚मेव विरोधं द‚र्श‚य‚ता \edtext{}{\lemma{ता}\Bfootnote{भ‚व‚तेति \cite{dp-edP} \cite{dp-edH}}}भ‚व‚त इति कृत‚म् । भ‚व‚तः‚{\tiny $_{lb}$}‚ प्र‚ब‚न्धेन\edtext{}{\lemma{न्धेन}\Bfootnote{०न्धेन व‚र्त्त० \cite{dp-msA} \cite{dp-msB} \cite{dp-msD} \cite{dp-edP} \cite{dp-edH} \cite{dp-edE} \cite{dp-edN}}} प्र‚व‚र्त्त‚मान‚स्य\edtext{}{\lemma{स्य}\Bfootnote{०स्य स‚न्तान० \cite{dp-msC}}} शीत‚स्प‚र्श‚स‚न्तान‚स्याभावोऽन्य‚स्योष्ण‚स‚न्तान‚स्य भावे स‚तीति ।‚{\tiny $_{lb}$}‚ योग्यो भ‚व‚ति । द्वितीयेऽन्ध‚कार‚देशालोकोत्पाद‚क‚क्ष‚ण‚विरुद्धान‚न्ध‚कारान‚स‚म‚र्थान् ज‚न‚य‚ति ।‚{\tiny $_{lb}$}‚ तृतीये त्व‚स‚म‚र्थेषु निवृत्तेषु त‚द्देश आलोको जाय‚त इति प्र‚त्येत‚व्य‚म् । त‚था शीताक्रान्त‚देश‚{\tiny $_{lb}$}‚म‚ध्योत्प‚न्नेनोष्ण‚स्प‚र्शेन स्थित‚ध‚र्म‚णा त‚थैव शीत‚स्प‚र्शो निव‚र्त्त‚नीय इत्य‚पि द्र‚ष्ट‚व्य‚म् ।
	\pend% ending standard par
      ‚{\tiny $_{lb}$}‚

	  \pstart \leavevmode% starting standard par
	न‚नु च येनालोक‚क्ष‚णेन स‚न्निप‚ति \edtext{}{\lemma{ति}\Bfootnote{प‚त}} तान्ध‚कार‚क्ष‚णोऽस‚म‚र्थो ज‚न्य‚ते न तेन त‚द्देश‚{\tiny $_{lb}$}‚ आक्र‚म्य‚ते । येन चाक्र‚म्य‚ते न तेनास‚म‚र्थो ज‚न्य‚ते । त‚था योऽन्ध‚कार‚स्त‚त्स‚न्निप‚त‚न‚काल‚भावी‚{\tiny $_{lb}$}‚ नासौ त‚द्विरुद्धः । य‚श्चास‚म‚र्थ‚स्त‚ज्ज‚न्मा सोऽपि त‚ज्ज‚न्य‚त्वाद‚विरोधी । ये चानुत्प‚त्तिध‚र्माण‚{\tiny $_{lb}$}‚स्तेऽप्य‚स‚त्त्वात्क‚थं तैर्विरुद्धा इत्याश‚ङ्क्याह--\textbf{ज‚न्य‚ज‚न‚क‚भावादि}ति । \textbf{चो}ऽव‚धार‚णे \textbf{स‚न्तान‚यो}‚{\tiny $_{lb}$}‚रित्य‚स्यान‚न्त‚रं द्र‚ष्ट‚व्यः ।
	\pend% ending standard par
      ‚{\tiny $_{lb}$}‚

	  \pstart \leavevmode% starting standard par
	अय‚माश‚यः--ज‚न्य‚ज‚न‚क‚भाव‚विशेष एवायं निव‚र्त्त्य‚निव‚र्त्त‚क‚भावः अर्वाग्द‚र्शी च न‚{\tiny $_{lb}$}‚ क्ष‚ण‚योः कार्य‚कार‚ण‚भावं विभाव‚यितुं विभ‚व‚ति । अपि तु स‚न्तान‚योस्त‚तोऽन्ध‚कार‚क्ष‚ण‚प्र‚ब‚न्ध‚{\tiny $_{lb}$}‚मेक‚त्वेनाव‚साय निव‚र्त्त्यं विरुद्ध‚म‚ध्य‚व‚स्यालोक‚क्ष‚ण‚प्र‚ब‚न्धं चैक‚त्वेनाधिम‚च्य त‚द्विरोधिन‚{\tiny $_{lb}$}‚म‚धिमुञ्च‚तीति ।
	\pend% ending standard par
      ‚{\tiny $_{lb}$}‚

	  \pstart \leavevmode% starting standard par
	प‚र‚मार्थ‚दृष्ट्या चेदं क्ष‚णोल्लेखेनाख्याय‚ते । न तु लोक‚स्थित्याश्र‚येण । न त‚र्हि प‚र‚{\tiny $_{lb}$}‚मार्थ‚तो विरोध इति चेत् । किं वै कार्य‚कार‚ण‚भाव‚विशेष एवैवंविधो न विद्य‚ते, येनैवं व‚क्तु‚{\tiny $_{lb}$}‚म‚ध्य‚व‚सितो भ‚वानिति ? एत‚च्चान‚न्त‚र‚मेव निरूप‚यिष्य‚ते ।
	\pend% ending standard par
      ‚{\tiny $_{lb}$}‚

	  \pstart \leavevmode% starting standard par
	न‚नु न स‚न्तानिव्य‚तिरेकेण स‚न्तानो नामान्यः स‚म्भ‚वी । त‚त्क‚थं\leavevmode\ledsidenote{\textenglish{69b/ms}}द्व‚योः स‚न्तान‚योर्विरोध‚{\tiny $_{lb}$}‚ उच्य‚त इत्याह \textbf{य‚द्य‚पी}त्य‚नुम‚तौ । य‚तः स‚न्तानिनो व‚स्तुभूताः स‚न्ति \textbf{त‚तो} हेतोर‚यं व‚क्ष्य‚माण‚कः ।‚{\tiny $_{lb}$}‚ उप‚प‚त्तिमाह--\textbf{य‚त} इति । य‚स्माद् ग\edtext{}{\lemma{ग}\Bfootnote{शी}}त‚क्ष‚ण‚प्र‚ब‚न्ध‚स्याभावान्निवृत्तिध‚र्म‚क‚त्व‚म् । त‚था‚{\tiny $_{lb}$}‚ य‚तः स‚त्स्वालोके\edtext{}{\lemma{त्स्वालोके}\Bfootnote{क}}क्ष‚णेषु प्र‚वृत्ता अप्य‚न्ध‚कार‚क्ष‚णान्नि\edtext{}{\lemma{णान्नि}\Bfootnote{णा नि}}वृत्तिध‚र्माणो भ‚व‚न्तीति द्र‚ष्ट‚व्य‚म् ।‚{\tiny $_{lb}$}‚ अन्ध‚कारादिक्ष‚ण‚प्र‚ब‚न्ध‚स्य\edtext{}{\lemma{स्य}\Bfootnote{कोष्ठ‚कान्त‚र्ग‚तः पाठः व्य‚र्थः--सं० ।}} \edtext{\textsuperscript{*}}{\lemma{*}\Bfootnote{आलोकादिक्ष‚ण‚प्र‚ब‚न्ध‚स्य}} आलोकादिक्ष‚ण‚प्र‚ब‚न्धेन स‚ह विरोध इति‚{\tiny $_{lb}$}‚ प्र‚क‚र‚णार्थः ।
	\pend% ending standard par
      ‚{\tiny $_{lb}$}‚‚{\tiny $_{lb}$}‚\textsuperscript{\textenglish{202/dm}}‚{\tiny $_{lb}$}‚
	  \bigskip
	  \begingroup
	

	  \pstart \leavevmode% starting standard par
	ये त्वाहुर्न विरोधो वास्त‚व इति त इदं व‚क्त‚व्याः--य‚था न निष्प‚न्ने कार्ये क‚श्चिज्ज‚न्य-
	\pend% ending standard par
      
	  \endgroup
	‚{\tiny $_{lb}$}‚

	  \pstart \leavevmode% starting standard par
	य‚दि येन स‚ह य‚स्यैक‚देशास्थितिर्न भ‚व‚ति तेन त‚स्य स‚हान‚व‚स्थान‚ल‚क्ष‚णो विरोध‚स्त‚र्हि‚{\tiny $_{lb}$}‚ स‚र्व एव प‚र‚माण‚वः स‚प्र‚तिघ‚त्वाद‚न्योन्य‚देश‚प‚रिहारेण व‚र्त्त‚न्त इति स‚र्वेषामेव प‚र‚माणूनाम‚यं‚{\tiny $_{lb}$}‚ विरोधः किन्न व्य‚व‚स्थाप्य‚त इत्याश‚ङ्क्याह--\textbf{स‚न्तान‚योरि}ति । \textbf{चो}ऽव‚धार‚णे । हेतुमाह—‚{\tiny $_{lb}$}‚\textbf{इत‚रेत‚रे}ति । य‚तः स‚त्स्व‚पि तेषु स‚र्व एव स‚न्तानेन प्र‚व‚ह‚न्ति त‚तः स‚न्तानाऽनिव‚र्त्त‚नं तेषाम् ।
	\pend% ending standard par
      ‚{\tiny $_{lb}$}‚

	  \pstart \leavevmode% starting standard par
	न‚नु य‚द्यालोकान्ध‚कार‚योर्निव‚र्त्त्य‚निव‚र्त्त‚क‚भावेन विरोध‚स्त‚र्हि प्र‚दीप‚म‚ल्लिकात‚ल‚व‚र्त्त्येव‚{\tiny $_{lb}$}‚ व‚र‚कात्माण निव‚र्त्ती\edtext{}{\lemma{र्त्ती}\Bfootnote{?}} अन्ध‚कार‚स्त‚त्स‚मीप‚व‚र्त्तिनाऽऽलोकेन किं न निव‚र्त्त्य‚त इत्याश‚ङ्क्याह—‚{\tiny $_{lb}$}‚\textbf{ग‚तिध‚र्मे}ति । \textbf{चो} य‚स्माद‚र्थे । \textbf{त‚द्दिग्व‚र्त्तिन} एवेत्य‚र्थाद् द्र‚ष्ट‚व्यः । य‚तो य‚द्दिग‚भिमुख‚ग‚ति‚{\tiny $_{lb}$}‚रालोक‚स्त‚दाक्र‚म्य‚माण‚दिग्व‚र्त्तिन एव विरोधिस‚न्तानान्निव‚र्त्त‚य‚ति । त‚त‚स्त‚स्मात्कार‚णात् ।
	\pend% ending standard par
      ‚{\tiny $_{lb}$}‚

	  \pstart \leavevmode% starting standard par
	कुतो न निव‚र्त्त‚य‚तीत्याह--\textbf{अन्ध‚कारेति । अन्ध‚काराक्रान्ताया}मित्य‚नेन दिशोऽन्त\edtext{}{\lemma{दिशोऽन्त}\Bfootnote{ऽन्ध}}‚{\tiny $_{lb}$}‚काराक्रान्त‚त्व‚मालोक‚क्ष‚णान्त‚राज‚न‚नासाम‚र्थ्य‚कार‚णं नोक्त‚म् । किन्त‚र्हि ? वास्त‚वानुवादः‚{\tiny $_{lb}$}‚ कृतः । या सा दिग‚न्ध‚काराक्रान्ता दृश्य‚ते त‚त्र त‚स्य त‚ज्ज‚न‚नासाम‚र्थ्यादित्य‚र्थः । अन्य‚थाऽ‚{\tiny $_{lb}$}‚न्ध‚काराक्रान्त‚त्व‚मेव त‚स्य न स्यात् । आलोकेन स‚मीप‚व‚र्त्तिनाऽन्ध‚काराप‚न‚यास‚म्भ‚वादिति‚{\tiny $_{lb}$}‚ क‚थ‚मेनं संग‚च्छेत ।
	\pend% ending standard par
      ‚{\tiny $_{lb}$}‚

	  \pstart \leavevmode% starting standard par
	अय‚म‚त्र प‚र‚मार्थः--दृश्य‚ते ताव‚त्काचिद‚न्ध‚कार‚मात्रा निक‚ट‚स्थितेनाप्यालोकेनाऽनिव‚र्त्तिता ।‚{\tiny $_{lb}$}‚ दृष्ट‚श्चान्य‚स्याव‚व‚र‚क‚व‚र्त्तिनोऽन्ध‚कार‚प्र‚च‚य‚स्योच्छेदः । त‚स्मादालोक‚स्यालोकान्त‚र‚ज‚न‚नासाम‚र्थ्य‚{\tiny $_{lb}$}‚म‚न्य‚त्र तु साम‚र्थ्यं त‚त्त्व‚चिन्त‚कैर‚चिन्त्य‚त्वात्प्र‚तीत्य‚स‚मुत्पाद‚स्य क‚ल्प्य‚त इति । अत एव य‚यो‚{\tiny $_{lb}$}‚र्ज‚न्य‚ज‚न‚क‚भावेन निव‚र्त्त्य‚निव‚र्त्त‚क‚भावो नास्ति त‚योः प्र‚दीप‚म‚ल्लिकादित‚ल‚व‚र्त्त्य‚न्ध‚कार‚त‚दास‚न्ना‚{\tiny $_{lb}$}‚लोक‚योर्न‚विरोधः । प्रायोवृत्त्या तु तौ विरोधेनाव‚बुद्ध्येते । अत एव पूर्वं \textbf{दूर‚स्थ‚योर्विंरोधा‚{\tiny $_{lb}$}‚भावाच्च निक‚ट‚स्थ‚योरेव निव‚र्त्त्यंनिव‚र्त्त‚क‚भावः\textbf{}} इत्युक्त‚म्, न तु निक‚ट‚स्थ‚योर्निव‚र्त्त्य‚निव‚र्त्त‚क‚{\tiny $_{lb}$}‚भाव एव इति । स‚ति निव‚र्त्त्य‚निव‚र्त्त‚क‚त्वे निक‚ट‚स्थ‚योरेव, न तु निक‚ट‚स्थ‚योर‚व‚श्यं निव‚र्त्त्य‚{\tiny $_{lb}$}‚निव‚र्त्त‚क‚भावः इति च व्याख्यात‚मेव ।
	\pend% ending standard par
      ‚{\tiny $_{lb}$}‚

	  \pstart \leavevmode% starting standard par
	स‚म्प्र‚ति ज‚न्य‚ज‚न‚क‚भाव‚निब‚न्ध‚नं स‚न्तान‚ग‚त‚मेव च विरोधं स्व‚यं प्र‚तिपादित‚माचार्य‚{\tiny $_{lb}$}‚स्याप्य‚भिप्रेत‚मेत‚दिति द‚र्श‚य‚न्नाह--\textbf{कार‚णैरिति} \edtext{\textsuperscript{*}}{\lemma{*}\Bfootnote{णेति}} । \textbf{कार‚ण‚स्य} निव‚र्त्त‚यित‚व्य‚स्य शीत‚{\tiny $_{lb}$}‚स्प‚र्शादेर्य‚द\textbf{साम‚र्थ्यं} स‚जातीय‚क्ष‚ण‚निर्माणेऽश‚क्त‚त्वं त‚त्र य‚द्धेतुत्वं निव‚र्त्त‚क‚स्य \textbf{त‚त्कृतं} त‚त्प्र‚युक्त‚म् ।‚{\tiny $_{lb}$}‚ अत एव \textbf{स‚न्तान‚निष्ठं स‚न्ताने} क्ष‚ण‚प्र‚ब‚न्धे \textbf{निष्ठा} व्य‚व‚स्थाप्य‚त‚या प‚र्य‚व‚सानं य‚स्य तं \textbf{द‚र्श‚य‚ता}‚{\tiny $_{lb}$}‚ प्र‚काश\leavevmode\ledsidenote{\textenglish{70a/ms}}य‚ताऽऽचार्येणेत्य‚र्थात् ।
	\pend% ending standard par
      ‚{\tiny $_{lb}$}‚

	  \pstart \leavevmode% starting standard par
	ये पुनः \textbf{शान्त‚भ‚द्राद‚यः}--\textbf{न} ताव‚दालोकादेरुत्प‚न्नेनान्ध‚कारादिना विरोधः, त‚स्यातीत‚{\tiny $_{lb}$}‚त्वेनास‚त्त्वात् । न चोत्पित्सुना स‚ह, त‚स्याप्य‚नाग‚त‚त‚याऽस‚त्त्वात् । नाऽपि व‚र्त्त‚मानेन, त‚स्यापि‚{\tiny $_{lb}$}‚ त‚ज्ज‚न्म‚त‚याऽविरोधित्वात् । त‚स्मान्न विरोधो नाम द्विष्ठः स‚म्ब‚न्धोऽस्ति । किन्तु काल्प‚निक‚{\tiny $_{lb}$}‚ एव । अत एवाचार्येण \textbf{विरोध‚ग‚तिरि}त्य‚भिधायि । न तु विरोध इति ।--इति व्याख्यात‚{\tiny $_{lb}$}‚व‚न्त‚स्तान् व‚च‚न‚भ‚ङ्ग्या निराचिकीर्षुराह--\textbf{ये त्वि}ति ।
	\pend% ending standard par
      \textsuperscript{\textenglish{203/dm}}‚{\tiny $_{lb}$}‚
	  \bigskip
	  \begingroup
	

	  \pstart \leavevmode% starting standard par
	ज‚न‚क‚भावो नाम \edtext{}{\lemma{नाम}\Bfootnote{दृष्टोऽस्ति--\cite{dp-msA} \cite{dp-msB} \cite{dp-edP} \cite{dp-edH} \cite{dp-edE} \cite{dp-edN}}}द्विष्ठोऽस्ति । कार‚ण‚पूर्विका तु कार्यंवृत्तिः\edtext{}{\lemma{कार्यंवृत्तिः}\Bfootnote{कार्य‚प्र‚वृत्तिः--\cite{dp-msA} \cite{dp-msB} \cite{dp-msC} \cite{dp-msD} \cite{dp-edP} \cite{dp-edH} \cite{dp-edE} \cite{dp-edN}}} । अतो वास्त‚व एव ।‚{\tiny $_{lb}$}‚ त‚द्व‚त् न निवृत्ते व‚स्तुनि क‚श्चित् \edtext{}{\lemma{श्चित्}\Bfootnote{क‚श्चिदिष्टो \cite{dp-msA} \cite{dp-msB} \cite{dp-edP} \cite{dp-edH} \cite{dp-edN} क‚श्चिद् दृष्टो--\cite{dp-edE}}}द्विष्ठो नाम विरोधोऽस्ति । द‚ह‚न‚निमित्तं तु शीत‚स्प‚र्श‚स्य‚{\tiny $_{lb}$}‚ \edtext{\textsuperscript{*}}{\lemma{*}\Bfootnote{क्ष‚णान्त‚रासाम० \cite{dp-msA} \cite{dp-msB} \cite{dp-edP} \cite{dp-edH} \cite{dp-edE} \cite{dp-edN}}}क्ष‚णान्त‚र‚ज‚न‚नासाम‚र्थ्य‚म् । अतो\edtext{}{\lemma{अतो}\Bfootnote{त‚तो--\cite{dp-msC}}} विरोधोऽपि वास्त‚व एव ॥
	\pend% ending standard par
       ‚{\tiny $_{lb}$}‚ 

	  \pstart \leavevmode% starting standard par
	उदाह‚र‚ण‚माह--
	\pend% ending standard par
       ‚{\tiny $_{lb}$}‚ 
	  \bigskip
	  \begingroup
	

	  \pstart \leavevmode% starting standard par
	शीतोष्ण‚स्प‚र्श‚व‚त् ॥ ७४ ॥
	\pend% ending standard par
      
	  \endgroup
	‚{\tiny $_{lb}$}‚ 

	  \pstart \leavevmode% starting standard par
	\edtext{\textsuperscript{*}}{\lemma{*}\Bfootnote{शीत‚ञ्चो० \cite{dp-msD}}}शीत‚श्चोष्ण‚श्च \textbf{तावेव} स्प‚र्शौ त‚योरिव । शीतोष्ण‚स्प‚र्श‚योर्हिं पूर्व‚व‚द्विरोधो योज‚नीयः ॥
	\pend% ending standard par
       ‚{\tiny $_{lb}$}‚ 

	  \pstart \leavevmode% starting standard par
	द्वितीय‚म‚पि विरोधं द‚र्श‚यितुमाह--
	\pend% ending standard par
       ‚{\tiny $_{lb}$}‚ 
	  \bigskip
	  \begingroup
	

	  \pstart \leavevmode% starting standard par
	प‚र‚स्प‚र‚प‚रिहार‚स्थित‚ल‚क्ष‚ण‚त‚या \edtext{}{\lemma{या}\Bfootnote{वा नास्ति--\cite{dp-msC}}}वा \edtext{}{\lemma{वा}\Bfootnote{वा भाव‚व‚त् \cite{dp-msB} \cite{dp-edP} \cite{dp-edH}}}भावामाव‚व‚त् ॥ ७५ ॥
	\pend% ending standard par
      
	  \endgroup
	‚{\tiny $_{lb}$}‚ 

	  \pstart \leavevmode% starting standard par
	प‚र‚स्प‚र‚स्य\edtext{}{\lemma{स्य}\Bfootnote{प‚र‚स्प‚रं \cite{dp-msA} \cite{dp-msB} \cite{dp-edP} प‚र‚स्प‚र‚प‚रि० \cite{dp-edH} \cite{dp-edE}}} प‚रिहारः प‚रित्याग‚स्तेन स्थितं ल‚क्ष‚णं रूपं य‚योस्त‚द्भावः प‚र‚स्प‚र‚प‚रिहार‚{\tiny $_{lb}$}‚स्थित‚ल‚क्ष‚ण‚ता\edtext{}{\lemma{ता}\Bfootnote{त‚या नास्ति--\cite{dp-msC}}} त‚या ।
	\pend% ending standard par
       ‚{\tiny $_{lb}$}‚ 

	  \pstart \leavevmode% starting standard par
	इह य‚स्मिन् प‚रिच्छिद्य‚माने य‚द् व्य‚व‚च्छिद्य‚ते त‚त् प‚रिच्छिद्य‚मान‚म‚व‚च्छिद्य‚मान‚{\tiny $_{lb}$}‚प‚रिहारेण स्थित‚रूपं द्र‚ष्ट‚व्य‚म् । नीले च प‚रिच्छिद्य‚माने ताद्रूप्य‚प्र‚च्युतिर‚व‚च्छिद्य‚ते, त‚द‚व्य‚{\tiny $_{lb}$}‚व‚च्छेदे नीलाप‚रिच्छेद‚प्र‚स‚ङ्गात् । त‚स्माद्व‚स्तुनो भावाभावौ प‚र‚स्प‚र‚प‚रिहारेण स्थित‚रूपौ ।‚{\tiny $_{lb}$}‚ नीलात्तु य‚द‚न्य‚द्रूपं त‚न्नीलाभावाव्य‚भिचारि । नील‚स्य दृश्य‚स्य पीतादावुप‚ल‚भ्य‚मानेऽनुप‚ल‚म्भाद-
	\pend% ending standard par
      
	  \endgroup
	‚{\tiny $_{lb}$}‚

	  \pstart \leavevmode% starting standard par
	\textbf{कार‚ण‚पूर्विका} कार‚ण‚त्वेनाभिम‚त‚प‚दार्थ‚स‚त्तापूर्विका कार्यंस्य कार्य‚त्वेनाभिम‚त‚स्य \textbf{वृत्तिः}‚{\tiny $_{lb}$}‚ प्र‚वृत्तिर्भाव इति याव‚त् । तुर्विशेष‚णार्थः । य‚त एव\textbf{म‚तो} हेतो\textbf{र्वास्त‚वः} पार‚मार्थिकः । \textbf{अन्य‚था}‚{\tiny $_{lb}$}‚ कार्य‚कार‚ण‚भावोऽप्य‚वास्त‚वोऽस्त्विति भावः ।
	\pend% ending standard par
      ‚{\tiny $_{lb}$}‚

	  \pstart \leavevmode% starting standard par
	न‚नु किं कार्य‚कार‚ण‚भावोऽपि द्विष्ठः स‚म्ब‚न्धः क‚श्चिदिष्टो येनैव‚मुच्य‚त इति चेत् ।‚{\tiny $_{lb}$}‚ न । कार‚ण‚पूर्विकायाः कार्य‚वृत्तेर्वास्त‚व‚त्वात् । इहापि त‚र्हि द‚ह‚नादिनिमित्तं शीत‚स्प‚र्शादे‚{\tiny $_{lb}$}‚र्ज‚न‚नासाम‚र्थ्यं वास्त‚व‚म‚स्तु । न तु विरोधः स‚म्ब‚न्ध इति चेत् । न । एताव‚तोऽन्य‚स्मात्कार्य‚{\tiny $_{lb}$}‚कार‚ण‚भावाद‚स्य कार्य‚कार‚ण‚भाव‚स्य विशेष‚रूप‚त्वाभ्युप‚ग‚मात् । अस्माभिर‚पीदृश एव कार्य‚{\tiny $_{lb}$}‚कार‚ण‚भाव‚विशेषो विरोध इत्युच्य‚त इति क‚थ‚म‚य‚म‚वास्त‚वः स्यादिति ॥
	\pend% ending standard par
      ‚{\tiny $_{lb}$}‚

	  \pstart \leavevmode% starting standard par
	पूर्वंव‚त्पूर्वोप‚द‚र्शित‚व‚त् ॥
	\pend% ending standard par
      ‚{\tiny $_{lb}$}‚‚{\tiny $_{lb}$}‚\textsuperscript{\textenglish{204/dm}}‚{\tiny $_{lb}$}‚
	  \bigskip
	  \begingroup
	

	  \pstart \leavevmode% starting standard par
	भाव‚निश्च‚यात् । य‚था च नीलं\edtext{}{\lemma{नीलं}\Bfootnote{नील‚म‚भावं \cite{dp-msA} \cite{dp-msB} \cite{dp-msC} \cite{dp-edP} \cite{dp-edH}}} स्वाभावं प‚रिह‚र‚ति, \edtext{\textsuperscript{*}}{\lemma{*}\Bfootnote{त‚मिव अभाव‚व‚त्--\cite{dp-msD-n}}}त‚द्व‚द् अभावाव्य‚भिचारि पीतादिक‚{\tiny $_{lb}$}‚म‚पीति\edtext{}{\lemma{पीति}\Bfootnote{म‚पि । त‚था \cite{dp-msA} \cite{dp-msB} \cite{dp-edP} \cite{dp-edH} \cite{dp-edE} \cite{dp-edN}}} । त‚था च भावाभाव‚योः साक्षाद्विरोधः,\edtext{\textsuperscript{*}}{\lemma{*}\Bfootnote{विरोधौ \cite{dp-msA} क्षाद्विरोधः कः क‚स्य \cite{dp-msB}}} व‚स्तुनोस्त्व‚न्योन्याभावाव्य‚भिचारित्वा‚{\tiny $_{lb}$}‚द्विरोधः ।
	\pend% ending standard par
       ‚{\tiny $_{lb}$}‚ 

	  \pstart \leavevmode% starting standard par
	क‚स्य चान्य‚त्राभावाव‚सायः ? यो निय‚ताकारोऽर्थः,\edtext{\textsuperscript{*}}{\lemma{*}\Bfootnote{र्थः, न तु \cite{dp-edE}}} त‚स्य । न त्व‚निय‚ताकारः,\edtext{\textsuperscript{*}}{\lemma{*}\Bfootnote{०कारोऽर्थः क्ष० \cite{dp-msA} \cite{dp-msB} \cite{dp-edP} \cite{dp-edH} \cite{dp-edE} \cite{dp-edN}}}‚{\tiny $_{lb}$}‚ क्ष‚णिक‚त्वादिव‚त् । क्ष‚णिक‚त्वं हि स‚र्वेषां नीलादीनां स्व‚रूपात्म‚क‚म् । अतो न निय‚ताकार‚म् ।‚{\tiny $_{lb}$}‚ \edtext{\textsuperscript{*}}{\lemma{*}\Bfootnote{अतः \cite{dp-msD} \cite{dp-edE} प्र‚त्य‚न्त‚रे य‚तः इति इति \cite{dp-msD} प्र‚तौ टिप्प‚णं व‚र्त्त‚ते ।}}य‚तः क्ष‚णिक‚त्व‚प‚रिहारेण न किञ्चिद् दृश्य‚ते ।
	\pend% ending standard par
      
	  \endgroup
	‚{\tiny $_{lb}$}‚

	  \pstart \leavevmode% starting standard par
	न‚नु स‚र्व‚मेव व‚स्तु स‚त्त्व‚र‚ज‚स्त‚मोरूपेणैक‚मिति क‚थ‚म‚न्योन्य‚रूप‚प‚रित्याग इत्याह--\textbf{इहे}ति ।‚{\tiny $_{lb}$}‚ \textbf{य‚द् व्य‚व‚च्छिद्य‚ते} य‚न्न प‚रिच्छिद्य‚ते । अप‚रिच्छेद‚स्यैव व्य‚व‚च्छेद‚रूप‚त्वात् । \textbf{अव‚च्छिद्य‚मान‚प‚रि‚{\tiny $_{lb}$}‚हारेण} व्य‚व‚च्छिद्य‚मान‚प‚रिहारेण \textbf{स्थितं} व्य‚व‚स्थितं रूपं स्व‚रूपं य‚स्य त‚त्त‚था । किम्पुन‚रिदं‚{\tiny $_{lb}$}‚ प्र‚सिद्ध‚मित्याह--\textbf{नील‚मि}ति । \textbf{चो} य‚स्मात् । त‚देव रूपं त‚द्रूप‚म्, त‚द्रूप‚मेव ताद्रूप्य‚म् त‚स्य प्र‚च्युति‚{\tiny $_{lb}$}‚र‚भावो \textbf{व्य}व‚ह‚र्त्त‚व्यैक‚रूपः प्र‚स‚ज्य‚प्र‚तिषेधात्मा तुच्छ‚रूपः । उप‚प‚त्तिमाह--\textbf{त‚द‚व्य‚व‚च्छेद} इति ।‚{\tiny $_{lb}$}‚ य‚त एवं \textbf{त‚स्मात्} कार‚णात् । य‚दि भावाभाव‚योर्विरोधः, न त‚र्हि नील‚पीत‚योः स स्यादित्याह—‚{\tiny $_{lb}$}‚\textbf{नीलादि}ति । तुर्विशेष‚द्योत‚कः । अभावाव्य‚भिचारित्व‚मेव साध‚य‚न्नाह--नील‚स्येति । भ‚व‚त्येवं‚{\tiny $_{lb}$}‚ नील‚स्य पीतादाव‚भावः, न तु त‚त्प‚रिहारेण त‚द् व्य‚व‚स्थित‚मित्याह--य‚थेति । \textbf{चो} य‚स्माद‚र्थे ।‚{\tiny $_{lb}$}‚ नीलं क‚र्त्तृ स्वाभावं स्व‚भावं \edtext{}{\lemma{भावं}\Bfootnote{?}} स च मान‚भाव‚श्च \edtext{}{\lemma{श्च}\Bfootnote{?}} तं न व्य‚भिच‚र‚तीति त‚था । एवं‚{\tiny $_{lb}$}‚ स‚ति किं व्य‚व‚स्थित‚मित्याह--\textbf{त‚था} चेति नील‚स्य साक्षात्स्वाभाव‚प‚रिहार‚प्र‚कारे त‚द‚व्य‚भि‚{\tiny $_{lb}$}‚चारित्वाद‚र्थान्त‚र‚प‚रिहार‚प्र‚कारे च स‚ति ।
	\pend% ending standard par
      ‚{\tiny $_{lb}$}‚

	  \pstart \leavevmode% starting standard par
	न‚तु \edtext{}{\lemma{तु}\Bfootnote{नु}} य‚द् य‚द‚भावाव्य‚भिचारि य \edtext{}{\lemma{य}\Bfootnote{त}} त्त‚त्तेन विरुद्ध्य‚ते । त‚स्य न‚{\tiny $_{lb}$}‚ त‚दात्म‚क‚त्वेनाभावाव‚साय‚स्तादात्म्याभाव\add{आ}व‚साय‚फ‚ल‚त्वाद‚न्य‚स्य विरोध‚स्य । त‚र्हिक्ष‚णिक‚त्व‚म‚पि‚{\tiny $_{lb}$}‚ नीलाभावाव्य‚भिचारित्वान्नीलेन विरुद्ध्य‚मानं न नीलात्म‚कं स्यात् । त‚द‚पि नीलाभाव‚व‚देव‚{\tiny $_{lb}$}‚ अन्य‚था क्ष‚णिक‚त्वं नीलात्म‚तैव स्यात् । त‚था च याव‚त्क्ष‚णिकं ताव‚न्नील‚मिति कृत्स्ना त्रिलोकी‚{\tiny $_{lb}$}‚ नीलैव स्यादिति म‚न‚सि निधायाह--\textbf{क‚स्य चेति} । तुश‚ब्दार्थ‚श्च‚कारः ।
	\pend% ending standard par
      ‚{\tiny $_{lb}$}‚

	  \pstart \leavevmode% starting standard par
	प‚र‚मु\leavevmode\ledsidenote{\textenglish{70b/ms}}खेन प्र‚श्नं कृत्वा प्र‚श्न‚विस‚र्ज‚न‚माह--य इति । \textbf{निय‚त}स्य प्र‚तिनिय‚त‚स्य‚{\tiny $_{lb}$}‚ \textbf{व‚स्तुन आकारः} स्व‚रूप‚मिति विग्र‚हीत‚व्य‚म् । एत‚देब व्य‚तिरेक‚मुखेणाह--\textbf{न त्वि}ति । न‚{\tiny $_{lb}$}‚ पुन‚र‚निय‚त‚स्य स‚र्व‚व‚स्तुस्व‚रूपात्म‚क‚स्य । त‚देव द‚र्शंय‚ति--\textbf{क्ष‚णिक‚त्वादिव‚दिति । आदि}श‚ब्दात्प‚र‚{\tiny $_{lb}$}‚माणुम‚य‚त्वादिप‚रिग्र‚हः । अनेनैत‚दाह--अभावाव्य‚भिचारित्वेऽपि निय‚ताकारेण तेन स‚म\add{म}स्य‚{\tiny $_{lb}$}‚ विरोधो नानिय‚ताकारेणेति ।
	\pend% ending standard par
      ‚{\tiny $_{lb}$}‚

	  \pstart \leavevmode% starting standard par
	अनिय‚ताकार‚त्व‚म‚स्योप‚पाद‚य‚न्नाह--\textbf{क्ष‚णिक‚त्वं} हीति । हिर्य‚स्माद‚र्थे । स‚न्मात्रानु‚{\tiny $_{lb}$}‚ब‚न्धित्वात्क्ष‚णिक‚स्येत्य‚भिप्रायः । य‚त एव्र\textbf{म‚तः} कार‚णात् । न त्व‚य‚म‚र्थः--\textbf{निय‚तः} प्र‚तिनिय‚त‚{\tiny $_{lb}$}‚ ‚{\tiny $_{lb}$}‚ \leavevmode\ledsidenote{\textenglish{205/dm}}‚{\tiny $_{lb}$}‚ 
	  
	य‚द्येव‚म‚भावोऽपि न निय‚ताकारः । \edtext{\textsuperscript{*}}{\lemma{*}\Bfootnote{क‚थं न निय० \cite{dp-msA} \cite{dp-msB} \cite{dp-edP} \cite{dp-edH} \cite{dp-edN}}}क‚थ‚म‚निय‚ताकारो नाम ? याव‚ता व‚स्तुरूप‚{\tiny $_{lb}$}‚विविक्ताकारः क‚ल्पितोऽभावः । त‚तो दृष्टं क‚ल्पितं वा निय‚तं रूप‚म‚न्य‚त्रास‚दित्य‚व‚सीय‚ते\edtext{}{\lemma{ते}\Bfootnote{०त्राऽस‚द‚व‚सीय‚ते \cite{dp-msA} \cite{dp-msB} \cite{dp-msD} \cite{dp-edP} \cite{dp-edH} \cite{dp-edE} \cite{dp-edN}}} ।‚{\tiny $_{lb}$}‚ नानिय‚त‚म् । एवं \edtext{}{\lemma{एवं}\Bfootnote{नित्य‚त्वे पिशा० \cite{dp-msA} \cite{dp-edP} \cite{dp-edH}}}नित्य‚त्व‚पिशाचादिर‚पि निय‚ताकारः क‚ल्पितो द्र‚ष्ट‚व्यः । एकात्म‚त्व-\edtext{\textsuperscript{*}}{\lemma{*}\Bfootnote{एकात्म‚क‚त्व \cite{dp-msA} \cite{dp-msB} \cite{dp-msC} \cite{dp-msD} \cite{dp-edP} \cite{dp-edH} \cite{dp-edE} \cite{dp-edN}}}‚{\tiny $_{lb}$}‚ विरोध‚श्चाय‚म् । य‚योर्हि \edtext{}{\lemma{योर्हि}\Bfootnote{प‚र‚स्प‚रेणाव० \cite{dp-msB}}}प‚र‚स्प‚र‚प‚रिहारेणाव‚स्थानं त‚योरेक‚त्वाभावः ।‚{\tiny $_{lb}$}‚ \textbf{आकारो} य‚स्येति । एवं हि क्ष‚णिक‚त्वादेर‚पि निय‚ताकार‚त्वं स्यात् । त‚थाहि प‚र‚म‚स‚ङ्कुचित‚{\tiny $_{lb}$}‚काल‚व‚र्त्तिरूप‚त्वेन निय‚ताकार‚त्वात् । \textbf{ध‚र्मोत्त‚रो}ऽपि \textbf{क्ष‚णिक‚त्वं} हि \textbf{स‚र्वेषां नीलादीनां स्व‚रूपात्म‚क‚{\tiny $_{lb}$}‚मि}ति ब्रुवाणो \textbf{निय‚ताकार} इत्य‚त्र ष‚ष्ठीत‚त्पुरुष‚म‚भिव्य‚न‚क्ति इत‚र‚था क्ष‚णिक‚त्व‚स्य हि स‚र्वो‚{\tiny $_{lb}$}‚ नीलादिः स्व‚रूप‚म् इत्य‚भिद‚ध्यादिति । एव‚ञ्चा \edtext{}{\lemma{ञ्चा}\Bfootnote{च}} क्ष‚णिक‚स्यापि न नीलेऽभावाव‚साय‚स्त‚स्य‚{\tiny $_{lb}$}‚ स‚र्व‚नीलादिव‚स्त्वात्म‚क‚त्वेनानिय‚ताकार‚त्वात् । येन च क्ष‚णिकं क‚ल्पितं न तेन प्र‚तिनिय‚त‚{\tiny $_{lb}$}‚व‚स्त्वात्म‚क \edtext{}{\lemma{क}\Bfootnote{कं}} क‚ल्पित‚म‚त एव नील‚ग्राहि प्र‚त्य‚क्षं क्ष‚णिक‚त्वाक्ष‚णिक‚त्व‚योरुदासीनं नील‚मात्रे‚{\tiny $_{lb}$}‚ प्र‚माण‚म् । त‚था च नील‚स्याक्ष‚णिक‚त्व‚प‚रिहारेणाव‚स्थानं क्ष‚णिक‚त्व‚सिद्धेः प्राङ् निश्चेतु‚{\tiny $_{lb}$}‚म‚श‚क्य‚मिति न्याय‚ब‚लात्प्राप्त‚म् ।
	\pend% ending standard par
      ‚{\tiny $_{lb}$}‚

	  \pstart \leavevmode% starting standard par
	एव‚ञ्चाविरोध‚रूप‚विवेच‚के \textbf{ध‚र्मोत्त‚रे} स‚त्त्य‚पि ये केचिद् द्विष्य‚मान‚ज‚ल्प‚म‚होद‚धिप्र‚भृत‚यो‚{\tiny $_{lb}$}‚ विरोध‚चोद्य‚प‚रिजिहीर्ष‚या प‚र‚स्प‚र‚प‚रिहार‚स्थित‚ल‚क्ष‚णं विरोधं प‚रिहारीकुर्व‚न्ति तैर‚यं क‚स्य \textbf{चा‚{\tiny $_{lb}$}‚न्य‚त्राभावाव‚सा\add{यः ।}यो निय‚ताकारो न त्व‚निय‚ताकारः क्ष‚णिक‚त्वादिरि}ति ध‚र्मोत्त‚र‚स्य ग्र‚न्थो न‚{\tiny $_{lb}$}‚ दृष्टो न चार्थ‚स्य स‚मीचीनिधिस \edtext{}{\lemma{मीचीनिधिस}\Bfootnote{?}} ज्ञात इति ल‚क्ष्य‚ते ।
	\pend% ending standard par
      ‚{\tiny $_{lb}$}‚

	  \pstart \leavevmode% starting standard par
	य‚द्य‚य‚म‚प‚रिहारः, क‚स्त‚त्र प‚रिहार इति चेत् । य‚थैत‚त्प‚रिह्रिय‚ते त‚था \textbf{विशेषाख्यान}‚{\tiny $_{lb}$}‚ एवास्माभिर‚भ्य‚धायीति त‚त एवापेक्षित‚व्य‚म् । इह पुन‚र‚प्र‚कृत‚त्वान्नोच्य‚त इति ।
	\pend% ending standard par
      ‚{\tiny $_{lb}$}‚

	  \pstart \leavevmode% starting standard par
	य‚दि निय‚ताकारं व‚स्तु प‚रिह‚र‚ति नानिय‚ताकार‚म्, त‚र्हि भावो नाभावं प‚रिहृत्य तिष्ठेत्‚{\tiny $_{lb}$}‚ त‚स्यानिय‚ताकार‚त्वाद् इत्याश‚ङ्क‚मान आह--\textbf{य‚द्येव‚मिति} । एव‚ञ्चेद‚भ्युप‚ग‚म्य‚ते त‚देत्य‚र्थात् ।‚{\tiny $_{lb}$}‚ न केव‚लं क्ष‚णिक‚त्वादीत्य‚पि श‚ब्दः । क‚थ‚मिति सिद्धान्ती ।
	\pend% ending standard par
      ‚{\tiny $_{lb}$}‚

	  \pstart \leavevmode% starting standard par
	\textbf{याव‚ते}ति तृतीयान्त‚प्र‚तिनिरूप‚को निपातोऽत्र य‚स्मादित्य‚स्यार्थे व‚र्त्त‚ते । \textbf{व‚स्तुरूप‚विविक्तो}‚{\tiny $_{lb}$}‚ दृश्य‚नीलादिस्व‚भाव‚र‚हित \textbf{आकारो} य‚स्येति विग्र‚हः ।
	\pend% ending standard par
      ‚{\tiny $_{lb}$}‚

	  \pstart \leavevmode% starting standard par
	क‚ल्पित‚ग्र‚ह‚णेनैत‚दाह--नाभावो नाम क‚श्चित् प्र‚माण‚सिद्धोऽस्ति । केव‚लं क‚ल्पिक‚या‚{\tiny $_{lb}$}‚ बुद्ध्या त‚था स‚मारोपित इति । य‚त एवं त‚स्माद् \textbf{दृष्टं} प्र‚माणाव‚ग‚तं \textbf{क‚ल्पित‚म्} आग‚माश्र‚येणा‚{\tiny $_{lb}$}‚न्य‚था वा स‚मारोपित‚म् । \textbf{अन्य‚त्र} त‚तोऽन्य‚स्मिन्, नीलादौ वाऽनिय‚त‚म्, न नीलाद्यात्म‚कं स‚त्‚{\tiny $_{lb}$}‚ त‚त्रैवाऽस\add{दि} त्य‚व‚सीय‚ते । अमुमेव न्याय‚म‚न्य‚त्रादिश‚न्ना\leavevmode\ledsidenote{\textenglish{71a/ms}}ह--एव‚मिति । य‚थाऽभावो निय‚ता‚{\tiny $_{lb}$}‚कार एव क‚ल्पितो नानिय‚ताकार एवं \textbf{नित्य‚त्व‚म‚पि} स‚र्व‚कालाव‚स्थायित्व‚ल‚क्ष‚णं \textbf{निय‚ताकार‚मेव}‚{\tiny $_{lb}$}‚ ‚{\tiny $_{lb}$}‚ \leavevmode\ledsidenote{\textenglish{206/dm}}‚{\tiny $_{lb}$}‚ 
	  
	अत एव लाक्ष‚णिकोऽयं विरोध उच्य‚ते । ल‚क्ष‚णं रूपं व‚स्तूनां प्र‚योज‚न‚म‚स्येति कृत्वा ।‚{\tiny $_{lb}$}‚ विरोधेन ह्य‚नेन व‚स्तुत‚त्त्वं विभ‚क्तं व्य‚व‚स्थाप्य‚ते । अत एव दृश्य‚माने रूपे य‚न्निषिध्य‚ते‚{\tiny $_{lb}$}‚ त‚द् दृश्य‚मेवाभ्युप‚ग‚म्य निषिध्य‚ते । त‚था हि--अभावोऽपि पिशाचोऽपि य‚दा पीते निषेद्धुमिष्य‚ते‚{\tiny $_{lb}$}‚ त‚दा दृश्यात्म‚त‚या निषेध्य इति दृश्य‚त्व‚म‚भ्युप‚ग‚म्य दृश्यानुप‚ल‚ब्धेरेव निषेधः । त‚था च स‚ति‚{\tiny $_{lb}$}‚ रूपे प‚रिच्छिद्य‚मान एक‚स्मिंस्त‚द‚भावो दृश्यो व्य‚व‚च्छिद्य‚ते । \edtext{\textsuperscript{*}}{\lemma{*}\Bfootnote{अथ य‚त्त‚द‚भाव‚व‚त् पा [[पी]] तादि त‚त् क‚थं व्य‚व‚च्छिद्य‚ते इत्याह--\cite{dp-msD-n}}}य‚च्च त‚द‚भाव‚व‚न्निय‚ताकारं‚{\tiny $_{lb}$}‚ \textbf{क‚ल्पित‚म्,} न तु स‚र्व‚स्य नीलादेः । अक्ष‚णिक‚त्वं तु स‚र्व‚स्य नीलादेः स्व‚रूपात्म‚कं स‚र्व‚स्यैवानेक‚{\tiny $_{lb}$}‚क्ष‚ण‚स्थायित्वात् । न चाक्ष‚णिक एव नित्यः, स‚तोऽकार‚ण‚स्याकाशादेः किय‚त एव त‚थात्वात् ।‚{\tiny $_{lb}$}‚ त‚था \textbf{पिशाच‚त्व‚म‚प्य}स्थिस्नायुम‚य‚सूचीव‚क्त्रादिरूप‚स्यैव स्व‚रूपं \textbf{क‚ल्पित‚मि}ति । त‚स्यापि नीला‚{\tiny $_{lb}$}‚कार‚त्वान्नीलादिना \edtext{}{\lemma{त्वान्नीलादिना}\Bfootnote{ता}} । \textbf{अयं} विरोधः । य‚द्वा निय‚तः प्र‚तिनिय‚त आकारः स्व‚भावो य‚स्य‚{\tiny $_{lb}$}‚ स त‚था निषेधेना\textbf{निय‚ताकारः} । त‚दा तु स‚र्व‚नीलाद्य‚नात्म‚क‚त्व‚स‚र्व‚नीलाद्यात्म‚क‚त्वे निय‚ताकार‚{\tiny $_{lb}$}‚त्वामिय‚ताकार‚त्वे वाच्ये । तेन न क्ष‚णिक‚त्वादौ निय‚ताकार‚त्व‚स्य प्र‚स‚ङ्गः । \textbf{स‚र्देषां स्व‚रूपा‚{\tiny $_{lb}$}‚त्म‚क‚मिति} च विव‚र‚ण‚म‚र्थाभेदेन नेय‚मिति ।
	\pend% ending standard par
      ‚{\tiny $_{lb}$}‚

	  \pstart \leavevmode% starting standard par
	न‚नु चानेनापि विरोधेन विरोधिनोः स‚हाव‚स्थानं निषिध्य‚ते । पूर्वेणापि प‚र‚स्प‚र‚प‚रि‚{\tiny $_{lb}$}‚हाराव‚स्थानं प्र‚तिपाद्य‚त इति क‚थ‚म‚न्योन्यान‚न्त‚र्भाव इत्याश‚ङ्क्याह--\textbf{एकात्म‚त्वेति । चो}‚{\tiny $_{lb}$}‚ य‚स्माद‚र्थे । विरुद्ध‚योरेकात्म‚निषेध‚को विरोध \textbf{एकात्म‚विरोध} उक्तः । क‚थ‚म‚स्य त‚था‚{\tiny $_{lb}$}‚त्व‚मित्याह--\textbf{य‚योरि}ति । हिर्य‚स्माद‚र्थे ।
	\pend% ending standard par
      ‚{\tiny $_{lb}$}‚

	  \pstart \leavevmode% starting standard par
	य‚त एतेन विरोधेन विरुद्ध‚योरेकात्म‚त्वं निषिध्य‚ते \textbf{अत एवा}स्मादेव कार‚णात् ।‚{\tiny $_{lb}$}‚ क‚थ‚मीदृशो विरोधो भ‚व‚ता \textbf{लाक्ष‚णि}क‚श‚ब्देनाभिधीय‚ते इत्याह--\textbf{ल‚क्ष‚ण}मिति । विभ‚क्त‚स्व‚रूपं‚{\tiny $_{lb}$}‚ \textbf{प्र‚योज‚नं} व्य‚व‚स्थाप्य‚त‚या साध्य‚म्, प्र‚युज्य‚ते अनेन इति वा \textbf{प्र‚योज‚नं} प्र‚योज‚क‚स्य । \textbf{इति‚{\tiny $_{lb}$}‚ कृत्वा} एवं व्युत्पाद्य ईदृश्या व्युत्प‚त्त्येति याव‚त् ।
	\pend% ending standard par
      ‚{\tiny $_{lb}$}‚

	  \pstart \leavevmode% starting standard par
	क‚थ‚मेत‚त्प्र‚योज‚न‚मित्याह--\textbf{विरोधेने}ति । \textbf{ही}ति य‚स्मात् । \textbf{विभ‚क्त}म‚न्येन विभ‚क्तं‚{\tiny $_{lb}$}‚ य‚तोऽ\textbf{नेनेति विरोधेन} नीलादेर्विभ‚क्त‚रूप‚व्य‚व‚स्थाप‚नाद‚न्येन स‚हैकात्म्यं निषिध्य‚ते । \textbf{अत एवा}‚{\tiny $_{lb}$}‚स्मादेव कार‚णात् । \textbf{दृश्य‚माने रूपे} प्र‚तीय‚माने व‚स्तुस्व‚रूपे \textbf{य‚न्निषिध्य‚ते} दृश्य‚मानात्म‚क‚त्वेन‚{\tiny $_{lb}$}‚ प्र‚तिषिध्य‚ते ।
	\pend% ending standard par
      ‚{\tiny $_{lb}$}‚

	  \pstart \leavevmode% starting standard par
	न‚नु दृश्ये व‚स्तुनि दृश्यान्त‚र‚स्य दृश्य‚त्वाभ्युप‚ग‚म‚पूर्व‚को निषेधो युक्तो न त्व‚दृश्य‚स्येत्या‚{\tiny $_{lb}$}‚श‚ङ्क्याह--\textbf{त‚था ही}ति । न केव‚लं भाव इत्य‚पिश‚ब्दः । न केव‚ल‚म‚भाव इत्य‚पिश‚ब्दः ।‚{\tiny $_{lb}$}‚ \textbf{दृश्यात्म‚त‚या} दृश्य‚पीतात्म‚त‚या \textbf{निषेध्यो} निषेधार्हः, नायं दृश्य‚मानः पीतः, अभावः, पिशाचो वा‚{\tiny $_{lb}$}‚ ताद्रूप्येणाप्र‚तिभास‚नादित्येवं निषेधादित्य‚भिप्रायः । इतिर्हेतौ । \textbf{दृश्यानुप‚ल‚ब्धेरे}वान्य‚स्य‚{\tiny $_{lb}$}‚ तादात्म्येनान्य‚स्मि\textbf{न्निषेधः} ।
	\pend% ending standard par
      ‚{\tiny $_{lb}$}‚

	  \pstart \leavevmode% starting standard par
	अथ स्यात्प्र‚त्य‚क्ष‚मेवात्र नील‚स्य पीतात्म‚ताऽभाव‚व्य‚व‚हारं क‚रोति । त‚त् किमेव‚मुच्य‚ते ?‚{\tiny $_{lb}$}‚ अथोक्त‚मेत‚द‚दृष्टानाम‚पि स‚त्त्व‚संज्ञ‚या न श‚क्तो व्य‚व‚हार‚यितुमिति चेत् । न । इह तादात्म्य‚{\tiny $_{lb}$}‚ \leavevmode\ledsidenote{\textenglish{207/dm}}‚{\tiny $_{lb}$}‚ 
	  
	रूपं त‚द‚पि दृश्यं व्य‚व‚च्छिद्य‚ते । त‚तः स्व‚प्र‚च्युतिव‚त् प्र‚च्युतिम‚न्तोऽपि व्य‚व‚च्छिन्ना इति ये‚{\tiny $_{lb}$}‚ प‚र‚स्प‚र‚प‚रिहार‚स्थित‚रूपाः स‚र्वे तेऽनेन निषिद्धैक‚त्वा इति । स‚त्य‚पि चास्मिन् विरोधे‚{\tiny $_{lb}$}‚ स‚हाव‚स्थानं स्याद‚पि । ‚{\tiny $_{lb}$}‚ 
	  
	त‚तो भिन्न‚व्यापारौ विरोधौ । एकेन विरोधेन शीतोष्ण‚स्प‚र्श‚योरेक‚त्वं वार्य‚ते । अन्येन‚{\tiny $_{lb}$}‚ स‚हाव‚स्थान‚म् । भिन्न‚विष‚यौ\edtext{}{\lemma{यौ}\Bfootnote{भिन्न‚प्र‚वृत्तिविष‚यौ \cite{dp-msA} \cite{dp-edP} \cite{dp-edH} \cite{dp-edN}}} च । स‚क‚ले व‚स्तुन्य‚व‚स्तुनि च प‚र‚स्प‚र‚प‚रिहार‚विरोधः ।‚{\tiny $_{lb}$}‚ व‚स्तुन्येव क‚तिप‚ये \edtext{}{\lemma{ये}\Bfootnote{स‚हाव‚स्थान० \cite{dp-msB} \cite{dp-msC}}}स‚हान‚व‚स्थान‚विरोधः । त‚स्माद्भिन्न‚व्यापारौ भिन्न‚विष‚यौ च । त‚तो‚{\tiny $_{lb}$}‚ नान‚योर‚न्योन्यान्त‚र्भाव इति ॥‚{\tiny $_{lb}$}‚ 
	  
	स च द्विविधोऽपि विरोधो व‚क्तृत्व‚स‚र्व‚ज्ञ‚त्व‚योर्न स‚म्भ‚व‚ति ॥ ७६ ॥‚{\tiny $_{lb}$}‚ निषेधात् । आधेय‚निषेधे ह्य‚यं न्यायो न तु दृश्य‚मानात्म‚तानिषेध इति । स‚त्य‚मेत‚त् । केव‚ल‚{\tiny $_{lb}$}‚म‚त्य‚न्त‚मूढं प्र‚त्येत‚दुक्त‚मित्य‚दोषः । किमेवं स‚ति सिद्धिमित्याह--\textbf{त‚था च त‚ति} ऐकात्म्य‚निषेधे‚{\tiny $_{lb}$}‚ स‚र्व‚स्य दृश्यात्म‚त‚या निषेध‚प्र‚कारे स‚ति । \textbf{त‚द‚भाव}स्त‚स्य प‚रिच्छिद्य‚मान‚स्य स्व‚रूप‚स्य नीलादे‚{\tiny $_{lb}$}‚र‚भाव‚स्त\textbf{द‚भावो दृश्यो} दृश्यात्म\leavevmode\ledsidenote{\textenglish{71b/ms}}कः स‚न् \textbf{व्य‚व‚च्छिद्य‚ते} तादात्म्येन निषिध्य‚ते । अयं‚{\tiny $_{lb}$}‚ दृश्य‚मानो नीलो नाभावः तुच्छ‚रूपेण अभाव‚रूपेणाप्र‚तिभास‚नादिति कृत्वा दृश्य‚मान‚रूपात्म‚त‚या‚{\tiny $_{lb}$}‚ निषेधादिति भावः ।
	\pend% ending standard par
      ‚{\tiny $_{lb}$}‚

	  \pstart \leavevmode% starting standard par
	भ‚व‚तु प‚रिच्छिद्य‚मानाऽभाव‚स्य दृश्य‚स्य व्य‚व‚च्छेद‚स्त‚द‚व्य‚भिचारिण‚स्तु निषेधे का‚{\tiny $_{lb}$}‚ वार्त्तेत्याह--\textbf{य‚च्चेति} । अपिश‚ब्दार्थ‚श्च‚कारः । \textbf{त‚द‚भावो} विद्य‚तेऽस्येति त‚था ।
	\pend% ending standard par
      ‚{\tiny $_{lb}$}‚

	  \pstart \leavevmode% starting standard par
	य‚दि त‚द‚भाव‚वांस्तादात्म्येन प्र‚तिषिध्य‚ते त‚र्हि क्ष‚णिक‚त्व‚म‚पि पूर्वोक्तेन न्यायेन नीला‚{\tiny $_{lb}$}‚भाव‚व‚दिति त‚द‚पि तादात्म्य‚त‚या व्य‚व‚च्छेद्यं स्यादित्याह--\textbf{निय‚ताकार‚मिति} । एत‚च्च पूर्व‚मेव‚{\tiny $_{lb}$}‚ कृत‚व्याख्यान‚म् । य‚तो द्व‚योर‚प्य‚भाव‚त‚द्व‚तोर्दृश्य‚मानात्म‚त‚या निषेधाद् द्श्य‚योरेव निषेध‚स्त‚त‚{\tiny $_{lb}$}‚स्त‚स्मात्\textbf{स्व‚प्र‚च्युतिरिव} स्वाभाव इव व्य‚व‚च्छिन्ना निषिद्ध‚तादात्म्याः । इतिस्त‚स्मात् । \textbf{स‚र्व}‚{\tiny $_{lb}$}‚ग्र‚ह‚णं कार्त्स्न्य‚प्र‚तिपाद‚नार्थ‚म् । अनेनेति विरोधेन \textbf{निषिद्ध‚मेक‚त्वं} येषामिति विग्र‚हः । \textbf{विरोधे}‚{\tiny $_{lb}$}‚ प‚र‚स्प‚र‚प‚रिहार‚स्थितात्म‚ल‚क्ष‚णे स‚हैक‚त्र लोक‚प्र‚तीतिसिद्धे देशेऽव‚स्थानं स्थितिः स्यात् । \textbf{अपिः}‚{\tiny $_{lb}$}‚ स‚म्भाव‚नायाम् । य‚स्माद‚नेन विरोधेन नैक‚त्राव‚स्थानं निषिद्ध्य‚ते किन्त्वेकात्म‚क‚त्व‚म् । पूर्वेण‚{\tiny $_{lb}$}‚ चैक‚त्राव‚स्थानं न त्वेकात्म‚क‚त्व‚म् ।
	\pend% ending standard par
      ‚{\tiny $_{lb}$}‚

	  \pstart \leavevmode% starting standard par
	\textbf{त‚तः} कार‚णाद् \textbf{भिन्नौ} नानाभूतौ \textbf{व्यापारौ} य‚योस्तौ त‚थोक्तौ । भिन्न‚व्यापार‚त्व‚{\tiny $_{lb}$}‚मेवान‚योरेकेत्यादिना स्फुट‚य‚ति । न केव‚लं व्यापार‚भेदाद‚न‚योर्भेदेनोप‚न्यासः । किन्तु‚{\tiny $_{lb}$}‚ विष‚य‚भेदाद‚पीत्याह--\textbf{भिन्न‚विष‚यौ} चेति । न केव‚लं भिन्न‚व्यापारौ भिन्न‚विष‚याव‚पीत्य‚पि‚{\tiny $_{lb}$}‚श‚ब्दार्थ‚श्च‚कारः । भिन्न‚विष‚य‚त्व‚मेव द‚र्श‚य‚न्नाह--\textbf{स‚क‚ल} इति ॥
	\pend% ending standard par
      ‚{\tiny $_{lb}$}‚

	  \pstart \leavevmode% starting standard par
	भ‚व‚तूक्त‚ल‚क्ष‚णो द्विविध एव विरोधः । त‚थाप्य‚न‚योर‚न्य‚त‚र एव विरोधो व‚क्तृत्व‚स‚र्व‚ज्ञ‚{\tiny $_{lb}$}‚त्व‚योर्भ‚विष्य‚तीत्याह--\textbf{स चे}ति । \textbf{चो} य‚स्मात्सोऽय‚म‚न‚न्त‚रोक्तो \textbf{द्विविधो} नास्ति । अपि‚{\tiny $_{lb}$}‚र‚तिश‚ये ।
	\pend% ending standard par
      ‚{\tiny $_{lb}$}‚‚{\tiny $_{lb}$}‚\textsuperscript{\textenglish{208/dm}}‚{\tiny $_{lb}$}‚
	  \bigskip
	  \begingroup
	

	  \pstart \leavevmode% starting standard par
	स चायं द्विविधोऽपि विरोधो व‚क्तृत्वं च स‚र्व‚ज्ञ‚त्वं च त‚योर्न स‚म्भ‚व‚ति । न ह्य‚विक‚ल‚{\tiny $_{lb}$}‚कार‚ण‚स्य स‚र्व‚ज्ञ‚त्व‚स्य व‚क्तृत्व‚भावाद‚भाव‚ग‚तिः\edtext{}{\lemma{तिः}\Bfootnote{ग‚तिरिति--\cite{dp-msD}}} । स‚र्व‚ज्ञ‚त्वं ह्य‚दृश्य‚म् । अदृष्ट‚स्य चाभावो‚{\tiny $_{lb}$}‚ नाव‚सीय‚ते । त‚तो नानेन विरोध‚ग‚तिर्भ‚व‚ति ।
	\pend% ending standard par
       ‚{\tiny $_{lb}$}‚ 

	  \pstart \leavevmode% starting standard par
	न च व‚क्तृत्व‚प‚रिहारेण स‚र्व‚ज्ञ‚त्व‚म‚व‚स्थित‚म् । काष्ठाद‚यो हि\edtext{}{\lemma{हि}\Bfootnote{०द‚योऽपि व‚क्तृ० \cite{dp-msA} \cite{dp-msB} \cite{dp-edP} \cite{dp-edH} \cite{dp-edE} \cite{dp-edN} ०द‚योऽपि हि व‚क्तृ \cite{dp-msC}}} व‚क्तृत्व‚प‚रिहृताः ।‚{\tiny $_{lb}$}‚ तेषाम‚पि स‚र्व‚ज्ञ‚त्व‚प्र‚स‚ङ्गात् । नापि स‚र्व‚ज्ञ‚त्व‚प‚रिहारेण व‚क्तृत्व‚म् । काष्ठादीनाम‚पि व‚क्तृत्व‚{\tiny $_{lb}$}‚प्र‚स‚ङ्गात् । त‚त एवाविरोधाद् व‚क्तृत्व\edtext{}{\lemma{क्तृत्व}\Bfootnote{विधाने न स‚र्व० \cite{dp-msA} \cite{dp-msB} \cite{dp-edP} \cite{dp-edH} \cite{dp-edE} \cite{dp-edN}}} विधेर्न स‚र्व‚ज्ञ‚त्व‚निषेधः ॥
	\pend% ending standard par
       ‚{\tiny $_{lb}$}‚ 

	  \pstart \leavevmode% starting standard par
	स्यादेत‚त्--य‚दि नास्त्येव विरोधो घ‚ट‚प‚ट‚योरिव स्याद‚पि त‚योः स‚हाव‚स्थितिद‚र्श‚न‚म्\edtext{}{\lemma{म्}\Bfootnote{०र्श‚न‚म् । अद‚र्श‚नात् तु--\cite{dp-msA} \cite{dp-msB} \cite{dp-msD} \cite{dp-edP} \cite{dp-edH} \cite{dp-edE} \cite{dp-edN}}} ।‚{\tiny $_{lb}$}‚ स‚हाव\unclear{व‚स्थित्य‚द‚र्श‚नात्तु विरोध‚ग‚तिः । विरोधा\edtext{}{\lemma{विरोधा}\Bfootnote{विरोधाद‚भा \cite{dp-msB} \cite{dp-msD}}}च्चाभाव‚ग‚तिरित्याश‚ङ्क्याह--}
	\pend% ending standard par
      
	  \endgroup
	‚{\tiny $_{lb}$}‚
	  \bigskip
	  \begingroup
	

	  \pstart \leavevmode% starting standard par
	न\edtext{}{\lemma{न}\Bfootnote{न च विरु० \cite{dp-msC}}} चाविरुद्ध‚विधेर‚नुप‚ल‚ब्धाव‚प्य‚भाव‚ग‚तिः ॥ ७७ ॥
	\pend% ending standard par
      
	  \endgroup
	‚{\tiny $_{lb}$}‚

	  \pstart \leavevmode% starting standard par
	त‚त्राद्य‚स्य ताव‚द‚भावं \textbf{न ही}त्यादिना द‚र्श‚य‚ति । हीति य‚स्मात् । कुतो नाभाव‚ग‚ति‚{\tiny $_{lb}$}‚रित्याह--\textbf{स‚र्व‚ज्ञ‚त्वं} हीति । हिर्य‚स्मात् । अदृश्य‚स्यापि किं माभाव‚ग‚तिरित्याह--\textbf{अदृष्ट‚स्}येति ।‚{\tiny $_{lb}$}‚ \textbf{चो} य‚स्माद‚र्थे । य‚त एवं त‚त‚स्त‚स्मात् । अनेन व‚क्तृत्वेन विरोध‚ग‚तिर्नास्ति त‚स्य स‚र्व‚ज्ञ‚{\tiny $_{lb}$}‚त्व‚स्येत्य‚र्थात् ।
	\pend% ending standard par
      ‚{\tiny $_{lb}$}‚

	  \pstart \leavevmode% starting standard par
	द्वितीय‚स्य विरोध‚स्य \add{अभावं} प्र‚तिपाद‚य‚न्नाह--\textbf{न चे}ति । \textbf{चः} प्र‚तिषेध‚स‚मुच्च‚ये ।‚{\tiny $_{lb}$}‚ उप‚प‚त्तिमाह--\textbf{काष्ठे}ति । हिर्य‚स्मात् । कुतो व‚क्तृत्व‚प‚रिहारेण स‚र्व‚ज्ञ‚त्वं नाब‚स्थित‚मित्याह—‚{\tiny $_{lb}$}‚\textbf{तेषाम‚पि} काष्ठादीनां स‚र्व‚ज्ञ‚त्व‚स्य प्र‚स‚ङ्गात्प्र‚स‚क्तेः । कुत‚स्तेषां त‚थात्व‚प्र‚स‚ङ्ग इत्याश‚ङ्क्य‚{\tiny $_{lb}$}‚ योज‚नीयं \textbf{काष्ठाद‚य} इति । हिर्य‚स्मात् । \textbf{काष्ठाद‚यो व‚क्तृत्वेन} व‚च‚न‚श‚क्त्या \textbf{प‚रिहृता}‚{\tiny $_{lb}$}‚स्त्य‚क्ताः । व‚क्तृत्व‚मेव स‚र्व‚ज्ञ‚त्व‚प‚रिहारेण व्य‚व‚स्थितं भ‚विष्य‚तीत्याह--\textbf{नापीति । अपिः}‚{\tiny $_{lb}$}‚ प्र‚तिषेध‚स‚मुच्च‚ये । अत्रापि तामेवोप‚प‚त्तिमाह--काष्ठेति । इहापि काष्ठाद‚यो हि स‚र्व‚ज्ञ‚{\tiny $_{lb}$}‚त्वेन प‚रिहृता इति द्र‚ष्ट‚व्य‚म् ।
	\pend% ending standard par
      ‚{\tiny $_{lb}$}‚

	  \pstart \leavevmode% starting standard par
	स्यादेत‚त्--व‚क्तृत्व‚स‚र्व‚ज्ञ‚त्व‚योः प‚र‚स्प‚र‚प‚रिहार‚स्थित‚ल‚क्ष‚ण‚ताविरोधेऽपि का क्ष‚तिर्येन‚{\tiny $_{lb}$}‚ त‚न्निषेधः कृतः । त‚था ह्य‚न्योन्य‚प‚रिहारेणाव‚स्थानेऽपि य‚द् व‚क्तृत्वं त‚त्स‚र्व‚ज्ञ‚त्वं मा भूत्,‚{\tiny $_{lb}$}‚ स‚र्व‚ज्ञ‚त्वं वा व‚क्तृत्व‚म् । त‚योस्त्वेक‚त्र‚स्थितिर‚विरुद्धैव । त‚द‚न‚न्त‚र‚मेवोक्तं \textbf{ध‚र्मोत्त‚रेण} \leavevmode\ledsidenote{\textenglish{72a/ms}}‚{\tiny $_{lb}$}‚ \textbf{स‚त्य‚पि चास्मिन् विरोधे स‚हाव‚स्थानं स्याद‚पी}ति ।
	\pend% ending standard par
      ‚{\tiny $_{lb}$}‚

	  \pstart \leavevmode% starting standard par
	त‚देत‚द‚स‚त् । येषां हि व‚क्तृत्व‚स‚र्व‚ज्ञ‚त्व‚ल‚क्ष‚णौ ध‚र्मिणो भिन्नावेव ध‚र्मौ तेषामिदं शोभ‚ते‚{\tiny $_{lb}$}‚ \textbf{न तु ताथाग‚तानां} ध‚र्म‚ध‚र्मिणोर्वास्त‚व‚म‚भेद‚मिच्छ‚ताम् । त‚था हि य‚दि व‚क्तृत्वं स‚र्व‚ज्ञ‚त्व‚प‚रि‚{\tiny $_{lb}$}‚‚{\tiny $_{lb}$}‚ ‚{\tiny $_{lb}$}‚ \leavevmode\ledsidenote{\textenglish{209/dm}}‚{\tiny $_{lb}$}‚ 
	  
	न चाविरुद्ध‚विधेरिति अनुप‚ल‚ब्धाव‚पि नायं विरुद्ध‚विधिः य‚द्य‚पि \edtext{}{\lemma{पि}\Bfootnote{य‚द्य‚पि च \cite{dp-msA} \cite{dp-msB} \cite{dp-edP} \cite{dp-edH} \cite{dp-edE} \cite{dp-edN}}}स‚हाव‚स्थाना‚{\tiny $_{lb}$}‚नुप‚ल‚म्भ‚स्त‚थापि न त‚योर्विरोधः । य‚स्मान्न स‚हानुप‚ल‚म्भ‚मात्राद् विरोधोऽपि तु द्व‚योरुप‚ल‚भ्य‚{\tiny $_{lb}$}‚मान‚योर्निव\edtext{}{\lemma{योर्निव}\Bfootnote{निर्व‚त्य \cite{dp-msA}}} र्त्य‚निव‚र्त‚क‚भावाव‚सायात् । त‚स्माद‚नुप‚ल‚ब्धाव‚पि न व‚क्तृत्व‚विधे र्विरुद्ध‚विधिः ।\edtext{\textsuperscript{*}}{\lemma{*}\Bfootnote{व‚क्तृत्व‚विरोधिविरुद्ध‚विधिः \cite{dp-msA} \cite{dp-msB} \cite{dp-edP} \cite{dp-edH} \cite{dp-edN}}}‚{\tiny $_{lb}$}‚ अतोऽस्मान्नान्य‚स्याभाव‚ग‚तिः ॥ ‚{\tiny $_{lb}$}‚ 
	  
	त‚था न व‚क्तृत्वाद् रागादिम‚त्त्व‚ग‚तिः । य‚तो य‚दि व‚च‚नादि रागादीनां कार्यं स्याद्व‚च‚{\tiny $_{lb}$}‚नादे रागादिग‚तिः स्यात् । रागादिनिवृत्तौ व‚च‚नादिनिवृत्तिः\edtext{}{\lemma{नादिनिवृत्तिः}\Bfootnote{व‚च‚न‚निवृ० \cite{dp-msB} \cite{dp-msC} \cite{dp-msD}}} स्यात् । न च कार्य‚म् ।‚{\tiny $_{lb}$}‚ कुतः-- ‚{\tiny $_{lb}$}‚ 
	  
	रागादीनां व‚च‚नादेश्च कार्य‚कार‚ण‚भावासिद्धेः ॥ ७८ ॥‚{\tiny $_{lb}$}‚ 
	  
	रागादीनां व‚च‚नादेश्च कार्य‚कार‚ण‚भाव‚स्याऽसिद्धेः कार‚णान्न कार्य‚म् । अतोऽस्मान्न‚{\tiny $_{lb}$}‚ ग‚तिः ॥ ‚{\tiny $_{lb}$}‚ 
	  
	मा भूद्रागादिकार्यं व‚च‚नं स‚ह‚चारि तु भ‚व‚ति । त‚तो रागादौ स‚ह‚चारिणि निवृत्ते‚{\tiny $_{lb}$}‚ निव‚र्त्तंते\edtext{}{\lemma{र्त्तंते}\Bfootnote{निव‚र्त्तिष्य‚ते \cite{dp-msC}}} व‚च‚न‚मित्याश‚ङ्क्याह-- ‚{\tiny $_{lb}$}‚ 
	  
	अर्थान्त‚र‚स्य \edtext{}{\lemma{स्य}\Bfootnote{वा कार‚ण० \cite{dp-msA} \cite{dp-msB} \cite{dp-edP} \cite{dp-edH} \cite{dp-edE} वाऽकार‚ण० \cite{dp-edN}}}चाकार‚ण‚स्य निवृत्तौ न व‚च‚नादेर्निवृत्तिः ॥ ७९ ॥‚{\tiny $_{lb}$}‚ 
	  
	अर्थान्त‚र‚स्य\edtext{}{\lemma{स्य}\Bfootnote{वा कार‚ण० \cite{dp-msA} \cite{dp-msB} \cite{dp-edP} \cite{dp-edH} \cite{dp-edE} वाऽकार‚ण० \cite{dp-edN}}} चाकार‚ण‚स्य निवृत्तौ स‚ह‚चारित्व‚द‚र्श‚न‚मात्रेण नान्य‚स्य व‚च‚नादेर्निवृत्तिः ।‚{\tiny $_{lb}$}‚ अतो व‚क्तृत्वं भ‚वेद्रागादिविर‚ह‚श्च ॥ ‚{\tiny $_{lb}$}‚ 
	  
	इति स‚न्दिग्ध\edtext{}{\lemma{न्दिग्ध}\Bfootnote{स‚न्दिग्ध‚व्यावृत्ते [[त्ति]] काऽनै० \cite{dp-msC}}} व्य‚तिरेकोऽनैकान्तिको व‚च‚नादिः ॥ ८० ॥‚{\tiny $_{lb}$}‚ हृतं स्यात् त‚दा व‚क्तृत्व‚स्य व‚क्तुर‚भेदात्स‚र्व‚ज्ञ‚त्व‚स्य च स‚र्व‚ज्ञाद्, व‚क्तैव स‚र्व‚ज्ञो न स्यात्, स‚र्व‚ज्ञ‚{\tiny $_{lb}$}‚ एव वा व‚क्तेति युक्त‚म‚न‚योः प‚र‚स्प‚र‚प‚रिहार‚स्थित‚ल‚क्ष‚ण‚ताऽभाव‚प्र‚तिपाद‚न‚मिति ।
	\pend% ending standard par
      ‚{\tiny $_{lb}$}‚

	  \pstart \leavevmode% starting standard par
	\textbf{य‚दि नास्त्येव विरोधः} स‚हान‚व‚स्थान‚ल‚क्ष‚ण इति द्र‚ष्ट‚व्य‚म् । अन्य‚था \textbf{घ‚ट‚प‚ट‚यो}र्दृष्टान्त‚ता‚{\tiny $_{lb}$}‚ न स्यात् । उप‚संहारे चाय‚म‚र्थो व्य‚क्तीक‚रिष्य‚ते । उक्तं विरुद्ध‚त्वे स‚हाव‚स्थानाद‚र्श‚नं‚{\tiny $_{lb}$}‚ कार‚ण‚म्, त‚त्किमेव‚मुच्य‚त इत्याश‚ङ्क्याह--\textbf{य‚द्य‚पीति । त‚थापि} तेनापि प्र‚कारेण । \textbf{अतो}‚{\tiny $_{lb}$}‚ हेतो\textbf{र‚स्मा}द् व‚क्तृत्वाद् \textbf{अन्य‚स्य} स‚र्व‚ज्ञ‚त्व‚स्य \textbf{नाभाव}प्र‚तिप‚त्तिः ॥
	\pend% ending standard par
      ‚{\tiny $_{lb}$}‚

	  \pstart \leavevmode% starting standard par
	य‚था व‚क्तृत्वाद‚स‚र्व‚ज्ञ‚त्व‚ग‚तिर्नास्ति त‚था व‚क्तृत्वाद् रागादिम‚त्त्व‚स्यापि ग‚तिर्नास्तीति‚{\tiny $_{lb}$}‚ जिज्ञाप‚यिषुराह--\textbf{त‚थे}ति । आदिग्र‚ह‚णाद् द्वेषादिप‚रिग्र‚हः ।
	\pend% ending standard par
      ‚{\tiny $_{lb}$}‚

	  \pstart \leavevmode% starting standard par
	\textbf{रागा}दिश‚ब्द‚स‚न्निधानाद्रागादि \textbf{स‚ह‚चारी}ति बोद्ध‚व्य‚म् । य‚था स‚न्दिग्ध‚विप‚क्ष‚व्यावृत्तिक‚{\tiny $_{lb}$}‚साध‚न‚दोष‚स्त‚था प्रागेवाभिहित‚मिति न पुन‚रुच्य‚ते ॥
	\pend% ending standard par
      ‚{\tiny $_{lb}$}‚\textsuperscript{\textenglish{210/dm}}‚{\tiny $_{lb}$}‚
	  \bigskip
	  \begingroup
	

	  \pstart \leavevmode% starting standard par
	इति श‚ब्द‚स्त‚स्माद‚र्थे । \edtext{\textsuperscript{*}}{\lemma{*}\Bfootnote{त‚स्माद‚स‚र्व‚ज्ञ‚त्वावीत‚राग‚त्व‚विप‚र्य‚यात् विप‚क्षात् स‚र्व‚ज्ञ‚त्वा [[त्व]] वीत‚रागादिम‚त्त्वात्‚{\tiny $_{lb}$}‚ स‚न्दिग्धो--\cite{dp-msB} त‚स्माद‚स‚र्व‚ज्ञ‚रागादिम‚त्त्व‚विप‚र्य‚यात् विप‚क्षात् स‚र्व‚ज्ञ‚त्वाद‚रागादिम‚त्त्वाच्च स‚न्दिग्धो \cite{dp-msC}}}त‚स्माद‚स‚र्व‚ज्ञ‚त्व‚विप‚र्य‚याद् विप‚क्षात्स‚र्व‚ज्ञ‚त्वाद्, रागादिम‚त्त्व‚{\tiny $_{lb}$}‚विप‚र्य‚याद‚रागादिम‚त्त्वात् स‚न्दिग्धो व्य‚तिरेको व‚च‚नादेः ।\edtext{\textsuperscript{*}}{\lemma{*}\Bfootnote{व‚च‚नादिः \cite{dp-msC}}} अतोऽनैकान्तिको व‚च‚नादिः ॥
	\pend% ending standard par
       ‚{\tiny $_{lb}$}‚ 

	  \pstart \leavevmode% starting standard par
	एव‚मेकैक‚रूपादिसिद्धिस‚न्देहे हेतुदोषान् आख्याय द्व‚योर्द्व‚यो रूप‚योर‚सिद्धिस‚न्देहे‚{\tiny $_{lb}$}‚ हेतुदोषान् व‚क्तुकाम\edtext{}{\lemma{क्तुकाम}\Bfootnote{कामा \cite{dp-msC}}} आह--
	\pend% ending standard par
       ‚{\tiny $_{lb}$}‚ 
	  \bigskip
	  \begingroup
	

	  \pstart \leavevmode% starting standard par
	द्व‚यो रूप‚योर्विप‚र्य‚य‚सिद्धौ विरुद्धः ॥ ८१ ॥
	\pend% ending standard par
      
	  \endgroup
	‚{\tiny $_{lb}$}‚ 

	  \pstart \leavevmode% starting standard par
	द्व‚योरिति\edtext{}{\lemma{योरिति}\Bfootnote{द्व‚योः । \cite{dp-msD}}} । द्व‚यो रूप‚योर्विप‚र्य‚य‚सिद्धौ स‚त्त्यां विरुद्धः ॥
	\pend% ending standard par
       ‚{\tiny $_{lb}$}‚ 

	  \pstart \leavevmode% starting standard par
	त्रीणि च रूपाणि स‚न्ति । त‚तो विशेष‚ज्ञाप\edtext{}{\lemma{ज्ञाप}\Bfootnote{ज्ञानार्थ० \cite{dp-msC}}}नार्थ‚माह--
	\pend% ending standard par
       ‚{\tiny $_{lb}$}‚ 
	  \bigskip
	  \begingroup
	

	  \pstart \leavevmode% starting standard par
	क‚योर्द्व‚योः ? ॥ ८२ ॥
	\pend% ending standard par
      
	  \endgroup
	‚{\tiny $_{lb}$}‚ 

	  \pstart \leavevmode% starting standard par
	क‚योर्द्व‚योरिति ॥
	\pend% ending standard par
       ‚{\tiny $_{lb}$}‚ 

	  \pstart \leavevmode% starting standard par
	विशिष्टे रूपे द‚र्श‚य‚ति--
	\pend% ending standard par
       ‚{\tiny $_{lb}$}‚ 
	  \bigskip
	  \begingroup
	

	  \pstart \leavevmode% starting standard par
	स‚प‚क्षे स‚त्व‚स्य, अस‚प‚क्षे चास‚त्व‚स्य । य‚था कृत‚क‚त्वं\edtext{}{\lemma{त्वं}\Bfootnote{स्य च कृत० \cite{dp-msC}}} प्र‚य‚त्नान‚न्त‚रीय‚{\tiny $_{lb}$}‚ क‚त्वं च नित्य‚त्वे साध्ये विरुद्धो हेत्वाभासः ॥ ८३ ॥
	\pend% ending standard par
      
	  \endgroup
	‚{\tiny $_{lb}$}‚ 

	  \pstart \leavevmode% starting standard par
	स‚प‚क्षे स‚त्त्व‚स्य, अस‚प‚क्षे चास‚त्त्व‚स्य विप‚र्य‚य‚सिद्धाविति\edtext{}{\lemma{सिद्धाविति}\Bfootnote{सिद्धिरिति \cite{dp-msC}}} स‚म्ब‚न्धः । कृत‚क‚त्व‚मिति‚{\tiny $_{lb}$}‚ स्व‚भाव‚हेतुः । प्र‚य‚त्नान‚न्त‚रीय‚क‚त्व मिति कार्य‚हेतुः\edtext{}{\lemma{हेतुः}\Bfootnote{कार्य‚हेतोः \cite{dp-edH}}} । प्र‚य‚त्नान‚न्त‚रीय‚क\edtext{}{\lemma{क}\Bfootnote{०क‚त्व‚श० \cite{dp-msC}}} श‚ब्देन हि प्र‚य‚त्ना‚{\tiny $_{lb}$}‚न‚न्त‚रं ज‚न्म ज्ञानं च प्र‚य‚त्नान‚न्त‚रीय‚क‚मुच्य‚ते । ज‚न्म जाय‚मान‚स्य स्व‚भावः । ज्ञानं ज्ञेय‚स्य‚{\tiny $_{lb}$}‚ कार्य‚म् । त‚दिह प्र‚य‚त्नान‚न्त‚रं \edtext{}{\lemma{रं}\Bfootnote{प्र‚य‚त्नान‚न्त‚रीय‚क‚त्व‚ग्राह‚क‚म्--\cite{dp-msD-n}}}ज्ञानं गृह्य‚ते । \edtext{\textsuperscript{*}}{\lemma{*}\Bfootnote{त‚तः का० \cite{dp-msC} \cite{dp-msD}}}तेन कार्य‚हेतुः ।
	\pend% ending standard par
      
	  \endgroup
	‚{\tiny $_{lb}$}‚

	  \pstart \leavevmode% starting standard par
	स‚र्वानैकान्तिक‚प्र‚कारानुक्त्वा विरुद्ध‚त्वाख्यं हेतुदोष‚म‚भिद‚धानो वार्तिक‚कार‚स्याभिप्राय‚म्‚{\tiny $_{lb}$}‚ \textbf{एव‚मि}त्यादिना द‚र्श‚य‚ति ।
	\pend% ending standard par
      ‚{\tiny $_{lb}$}‚

	  \pstart \leavevmode% starting standard par
	व्य‚क्तिभेद‚विव‚क्ष‚या हेतुदोषानिति ब‚हुव‚च‚नेनाह ॥
	\pend% ending standard par
      ‚{\tiny $_{lb}$}‚

	  \pstart \leavevmode% starting standard par
	\textbf{विशिष्टे} रूपे द्वे \textbf{द‚र्श‚य‚ति} । य‚दि द्वाव‚पि स्व‚भाव‚हेतू त‚दा किं द्व‚याभिधानेन ?‚{\tiny $_{lb}$}‚ एकेनापि स्व‚भाव‚हेतुना विरुद्ध‚त्व‚स्य द‚र्शित‚त्वादित्याश‚ङ्क्य स‚म‚र्थ‚न‚माह--\textbf{कृत‚क‚त्व‚मि}ति । न‚नु‚{\tiny $_{lb}$}‚ स्व‚भाव‚हेतुत्वेन प्र‚य‚त्नान‚न्त‚रीय‚क‚त्व‚म‚न्य‚त्र द‚र्शित‚म् । त‚त्क‚थ‚म‚नेनैव‚मुच्य‚त इत्याह--\textbf{प्र‚य‚त्नेति} ।‚{\tiny $_{lb}$}‚ \textbf{हिर्य}स्मात् । प्र‚य‚त्न‚स्य पुरुष‚व्यापार‚स्या\textbf{न‚न्त‚र‚म}व्य‚व‚हितं \textbf{ज‚न्म ज्ञानं च} त‚द्विष‚य‚मुच्य‚ते ।
	\pend% ending standard par
      ‚{\tiny $_{lb}$}‚\footnote{चेति \cite{dp-msC}}‚{\tiny $_{lb}$}‚\textsuperscript{\textenglish{211/dm}}‚{\tiny $_{lb}$}‚
	  \bigskip
	  \begingroup
	

	  \pstart \leavevmode% starting standard par
	एतौ हेतू नित्य‚त्वे साध्ये विरुद्धौ हेत्वाभासौ ॥
	\pend% ending standard par
       ‚{\tiny $_{lb}$}‚ 

	  \pstart \leavevmode% starting standard par
	क‚स्मात् पुन‚रेतौ विरुद्धावित्याह--
	\pend% ending standard par
       ‚{\tiny $_{lb}$}‚ 
	  \bigskip
	  \begingroup
	

	  \pstart \leavevmode% starting standard par
	अन‚योः स‚प‚क्षेऽस‚त्व‚म्, अस‚प‚क्षे च स‚त्व‚मिति विप‚र्य‚य‚सिद्धिः\edtext{}{\lemma{सिद्धिः}\Bfootnote{०द्धिरिति--\cite{dp-msC}}} ॥ ८४ ॥
	\pend% ending standard par
      
	  \endgroup
	‚{\tiny $_{lb}$}‚ 

	  \pstart \leavevmode% starting standard par
	अन‚योरिति । स‚प‚क्षे हि\edtext{}{\lemma{हि}\Bfootnote{हि नास्ति \cite{dp-msA} \cite{dp-msB} \cite{dp-edP} \cite{dp-edH} \cite{dp-edE} \cite{dp-edN}}} नित्ये क‚त‚क‚त्व‚प्र‚य‚त्नान‚न्त‚रीय‚क‚त्व‚योर‚स‚त्त्व‚मेव निश्चित‚म् ।‚{\tiny $_{lb}$}‚ अनित्ये विप‚क्षे एव स‚त्त्वं निश्चित‚मिति विप‚र्य‚य‚सिद्धिः ॥
	\pend% ending standard par
       ‚{\tiny $_{lb}$}‚ 

	  \pstart \leavevmode% starting standard par
	क‚स्मात् पुन‚र्विप‚र्य‚य‚सिद्धाव‚प्येतौ विरुद्धावित्याह--
	\pend% ending standard par
       ‚{\tiny $_{lb}$}‚ 
	  \bigskip
	  \begingroup
	

	  \pstart \leavevmode% starting standard par
	एतौ च साध्य‚विप‚र्य‚य‚साध‚नाद्विरुद्धौ ॥ ८५ ॥
	\pend% ending standard par
      
	  \endgroup
	‚{\tiny $_{lb}$}‚ 

	  \pstart \leavevmode% starting standard par
	एतौ च साध्य‚स्य नित्य‚त्व‚स्य विप‚र्य‚य‚म्--अनित्य‚त्वं साध‚य‚तः । त‚तः\edtext{}{\lemma{तः}\Bfootnote{त‚तः नास्ति--\cite{dp-msA} \cite{dp-msB} \cite{dp-edP} \cite{dp-edH} \cite{dp-edN}}} साध्य‚विप‚र्य‚य‚{\tiny $_{lb}$}‚साध‚नाद्विरुद्धौ ॥
	\pend% ending standard par
       ‚{\tiny $_{lb}$}‚ 

	  \pstart \leavevmode% starting standard par
	य‚दि साध्य‚विप‚र्य‚य‚साध‚नाद्विरुद्धावेतौ, उक्तं च प‚रार्थानुमाने साध्य‚म्, न त्व‚नुक्त‚म्;‚{\tiny $_{lb}$}‚ इष्टं च अनुक्त‚म् । अतोऽन्य इष्ट‚विघात‚कृदाभ्यामिति द‚र्श‚य‚न्नाह--
	\pend% ending standard par
       ‚{\tiny $_{lb}$}‚ 
	  \bigskip
	  \begingroup
	

	  \pstart \leavevmode% starting standard par
	\edtext{\textsuperscript{*}}{\lemma{*}\Bfootnote{त‚त्र च--\cite{dp-msB} \cite{dp-edP} \cite{dp-edH}}}न‚नु च तृतीयोऽपीष्ट‚विघात‚कृद् विरुद्धः ॥ ८६ ॥
	\pend% ending standard par
      
	  \endgroup
	‚{\tiny $_{lb}$}‚ 

	  \pstart \leavevmode% starting standard par
	न‚नु च तृतीयोऽपि विरुद्ध उक्तः\edtext{}{\lemma{उक्तः}\Bfootnote{प्र‚माण‚विनिश्च‚यादौ--\cite{dp-msD-n}}} । उक्त‚विप‚र्य‚य‚साध‚नौ द्वौ । तृतीयोऽय‚मिष्ट‚स्य‚{\tiny $_{lb}$}‚ श‚ब्देनानुपात्त‚स्य \edtext{}{\lemma{स्य}\Bfootnote{विधानं--\cite{dp-msA} \cite{dp-msB} \cite{dp-edP} \cite{dp-edH}}}विघातं क‚रोति विप‚र्य‚य‚साध‚नादिति इष्ट‚विघात‚कृत् ॥
	\pend% ending standard par
       ‚{\tiny $_{lb}$}‚ 

	  \pstart \leavevmode% starting standard par
	त‚मुदाह‚र‚ति--
	\pend% ending standard par
      
	  \endgroup
	‚{\tiny $_{lb}$}‚
	  \bigskip
	  \begingroup
	

	  \pstart \leavevmode% starting standard par
	य‚था प‚रार्थाश्च‚क्षुराद‚यः स‚ङ्घात‚त्वाच्छ‚य‚नास‚नाद्य‚ङ्ग‚व‚दिति ॥ ८७ ॥
	\pend% ending standard par
      
	  \endgroup
	‚{\tiny $_{lb}$}‚

	  \pstart \leavevmode% starting standard par
	न‚नु \textbf{प्र‚य‚त्नान‚न्त‚रीय}क\add{त्व}श‚ब्द उपात्त‚स्त‚त्किंप्र‚य‚त्नान‚न्त‚रीय‚क‚श‚ब्द‚स्यार्थ उच्य‚त‚{\tiny $_{lb}$}‚ इत्याह--\textbf{प्र‚य‚त्नान‚न्त‚रीय‚क‚त्व}मिति प्र‚य‚त्नान‚न्त‚रीय‚क‚त्व‚श‚ब्देनापि त‚देवोक्तं व‚स्तुत इत्य‚र्थः । तेन‚{\tiny $_{lb}$}‚ श‚ब्देन द्व‚य‚स्याभिधाने क‚दा स्व‚भाव‚स्याभिधानं क‚दा कार्य‚स्येत्याश‚ङ्क्य \textbf{ज‚न्मे}त्यादिना विभ‚ज‚ते ।‚{\tiny $_{lb}$}‚ य‚तो ज्ञान‚स्याप्य‚भिधानं \textbf{त‚त्त}स्मा\textbf{दिह} विरुद्धोदाह‚र‚ण‚प्र‚क्र‚मे ।
	\pend% ending standard par
      ‚{\tiny $_{lb}$}‚

	  \pstart \leavevmode% starting standard par
	अनुप‚ल‚म्भ‚स्त्व‚न‚योरेवान्त‚र्भूत‚त्वान्न पृथ‚गुप‚द‚र्शितः । एत‚योरुदाहृत‚योरेव सोऽपि सुज्ञान‚{\tiny $_{lb}$}‚ इति भाव उन्नेयः ।
	\pend% ending standard par
      ‚{\tiny $_{lb}$}‚

	  \pstart \leavevmode% starting standard par
	ल‚क्ष्य‚ते चाय\textbf{माचार्यं}स्याश‚यः--स‚प‚क्षावृत्तौ स‚त्यां व्याप्त्याऽव्याप्त्या वा विप‚क्ष‚वृत्तो‚{\tiny $_{lb}$}‚ विरुद्ध एवेति द‚र्श‚यितुमुभ‚योपादान‚म् । अन्य‚था कृत‚क‚त्व‚मात्रे प्र‚द‚र्शिते स‚प‚क्षावृत्तौ विप‚क्ष‚व्याप‚क‚{\tiny $_{lb}$}‚ एव यः स विरुद्धो न तु प्र‚य‚त्नान‚न्त‚रीय‚क‚व‚द् यो विप‚क्ष‚व्यापीति श‚ङ्का स्यात् । त‚था‚{\tiny $_{lb}$}‚ ‚{\tiny $_{lb}$}‚ \leavevmode\ledsidenote{\textenglish{212/dm}}‚{\tiny $_{lb}$}‚ 
	  
	य‚थेति । च‚क्षुराद‚य इति ध‚र्मी । प‚रोऽर्थः प्र‚योज‚नं \edtext{}{\lemma{नं}\Bfootnote{प‚रोऽर्थः प्र‚योज‚नं सं०--\cite{dp-msA} \cite{dp-msB} \cite{dp-msC} \cite{dp-msD} \cite{dp-edP} \cite{dp-edH} \cite{dp-edE} \cite{dp-edN}}}प‚रार्थः प्र‚योज‚कः संस्कार्य‚{\tiny $_{lb}$}‚ उप‚क‚र्त्त‚व्यो येषां ते प‚रार्थाः-इति साध्य‚म् । संघात‚त्वात् स‚ञ्चित‚रूप‚त्वादिति हेतुः । च‚क्षुराद‚यो‚{\tiny $_{lb}$}‚ हि प‚र‚माणुस‚ञ्चिति\edtext{}{\lemma{ञ्चिति}\Bfootnote{०संच‚य‚रू \cite{dp-msC} \cite{dp-msD}}}रूपाः । त‚तः संघात‚रूपा उच्य‚न्ते । श‚य‚न‚मास‚नं चादिर्य‚स्य त‚च्छ‚य‚ना‚{\tiny $_{lb}$}‚स‚नादि । त‚देवाङ्गं पुरुषोप‚भोगाङ्ग‚त्वात् । अयं व्याप्तिप्र‚द‚र्श‚न‚विष‚यो दृष्टान्तः । अत्र हि‚{\tiny $_{lb}$}‚ पारार्थ्येन संह‚त‚त्वं\edtext{}{\lemma{त्वं}\Bfootnote{संघात‚त्वं \cite{dp-msD}}} व्याप्त‚म् । य‚तः \edtext{}{\lemma{तः}\Bfootnote{श‚य‚नाद‚यः \cite{dp-msD}}}श‚य‚नास‚नाद‚यः संघात‚रूपाः पुरुष‚स्य भोगिनो भ‚व‚न्त्यु‚{\tiny $_{lb}$}‚प‚कार‚का इति प‚रार्था उच्य‚न्ते ॥ ‚{\tiny $_{lb}$}‚ 
	  
	क‚थ‚म‚य‚मिष्ट‚विघात‚कृदित्याह-- ‚{\tiny $_{lb}$}‚ 
	  
	त‚दिष्टासंह‚त‚पारार्थ्य‚विप‚र्य‚य‚साध‚नाद्विरुद्धः ॥ ८८ ॥‚{\tiny $_{lb}$}‚ 
	  
	त‚दिष्टासंह‚त‚पारार्थ्य‚विप‚र्य‚य‚साध‚नादिति । असंह‚ते विष‚ये पारार्थ्य‚म‚संह‚त‚पारार्थ्य‚म् ।\edtext{\textsuperscript{*}}{\lemma{*}\Bfootnote{०र्थ्य‚म् । इष्टासंह‚त‚पारार्थ्यं त‚स्य सांख्य‚स्य वादिन इष्टासंह‚त‚पारार्थ्य‚म् \cite{dp-msC} \cite{dp-msD}}}‚{\tiny $_{lb}$}‚ त‚स्य सांख्य‚स्य वादिन इष्ट‚म‚संह‚त‚पारार्थ्यं त‚दिष्टासंह‚त‚पाराथ्यंम् । त‚स्य विप‚र्य‚यः संह‚त‚{\tiny $_{lb}$}‚पारार्थ्यं नाम । त‚स्य साध‚नाद् विरुद्धः । ‚{\tiny $_{lb}$}‚ 
	  
	आत्मा अस्ति इति ब्रुवाणः सांख्यः कुत एत‚द् इति प‚र्य‚नुयुक्तो बौद्धेनेद‚मात्म‚नः सिद्ध‚ये‚{\tiny $_{lb}$}‚ प्र‚माण‚माह । त‚स्माद‚संह‚त‚स्यात्म‚न उप‚कार‚क‚त्वं साध्यं च‚क्षुरादीनाम् । अयं तु हेतुर्विप‚र्य‚य‚{\tiny $_{lb}$}‚व्याप्तः । य‚स्माद्यो य‚स्योप‚कार‚कः स त‚स्य ज‚न‚कः । ज‚न्य‚मान‚श्च युग‚प‚त्\edtext{}{\lemma{त्}\Bfootnote{०मान‚स्तु युग० \cite{dp-msD}}} क्र‚मेण वा‚{\tiny $_{lb}$}‚ भ‚व‚ति\edtext{}{\lemma{ति}\Bfootnote{भ‚व‚तीति \cite{dp-msD}}} संह‚तः । त‚स्मात् प‚रार्थाश्च‚क्षुराद‚यः\edtext{}{\lemma{यः}\Bfootnote{च‚क्षुराद‚य इति संहं० \cite{dp-msA} \cite{dp-msB} \cite{dp-edP} \cite{dp-edH} \cite{dp-edN}}} संह‚त‚प‚रार्था इति सिद्ध‚म् ॥‚{\tiny $_{lb}$}‚ प्र‚य‚त्नान‚न्त‚रीय‚क‚त्व‚मात्रे प्र‚द‚र्शिते स‚प‚क्षावृत्ताव‚व्याप्त्यैव यो विप‚क्षे व‚र्त्त‚ते स एव विरुद्धो नान्य‚{\tiny $_{lb}$}‚ इति श‚ङ्का स्यात्, एताव‚तापि \leavevmode\ledsidenote{\textenglish{72b/ms}} विप‚र्य‚य‚सिद्धेरुप‚प‚त्तेः । य‚दि तु कार्य‚हेतुरुदाहृत‚{\tiny $_{lb}$}‚ इत्युच्य‚ते, त‚दानुप‚ल‚ब्धिर‚प्युदाह‚र्त्त‚व्या स्यात् । न च त‚स्यास्त‚त्रान्त‚र्भावान्नोदाह‚र‚ण‚मिति युक्त‚म्,‚{\tiny $_{lb}$}‚ अन्व‚यादिप्र‚द‚र्श‚नेऽपि अनुदाह‚र‚ण‚प्र‚स‚ङ्गात् । एवं \textbf{ध‚र्मोत्त‚रेण} क‚थं न व्याख्यात‚मिति न प्र‚तीमः ॥
	\pend% ending standard par
      ‚{\tiny $_{lb}$}‚

	  \pstart \leavevmode% starting standard par
	अर्थ‚श‚ब्दः प्र‚योज‚ने वृत्त‚स्त‚स्य च प्र‚योज‚य‚तीति व्युत्प‚त्त्या प्र‚योज‚क‚श‚ब्दाभिल‚प्य‚त्व‚मित्य‚भि‚{\tiny $_{lb}$}‚प्रेत्य \textbf{प‚रार्थः प्र‚योज‚क} इत्युक्तः । प‚र‚स्मिन्न‚र्थः प्र‚योज‚नं येषामिति ग‚म‚क‚त्वाद् व्य‚धिक‚र‚ण‚{\tiny $_{lb}$}‚ब‚हुव्रीहौ तु स‚र्वं स‚म‚ञ्ज‚सं केव‚ल‚म‚नेन त‚था न व्याख्यात‚मिति न विद्मः । \textbf{अङ्गं} निमित्त‚म् ।‚{\tiny $_{lb}$}‚ क‚स्य त‚द‚ङ्ग‚मित्युच्य‚त इत्याह--\textbf{पुरुषे}ति ॥
	\pend% ending standard par
      ‚{\tiny $_{lb}$}‚

	  \pstart \leavevmode% starting standard par
	क्र‚मेण युग‚प‚द् वाऽपि द्वेधाप्य‚संह‚तं द्र‚ष्ट‚व्य‚म् ।
	\pend% ending standard par
      ‚{\tiny $_{lb}$}‚

	  \pstart \leavevmode% starting standard par
	कुतः पुन‚र‚संह‚त‚विष‚यं पारार्थ्य‚मिष्टं \textbf{सांख्य‚स्या}ख्याय‚त इत्याश‚ङ्क्याह \textbf{आत्मेति} । असंह‚तो‚{\tiny $_{lb}$}‚प‚कार‚क‚त्वं ताव‚च्च‚क्षुरादीनां तेनैषित‚व्य‚म‚न्य‚थाऽऽत्माऽसिद्धेः । य‚च्च त‚द‚संह‚त‚रूपं स एवा‚{\tiny $_{lb}$}‚त्मेत्य‚भिप्रायेणोक्त‚म\textbf{संह‚त‚स्यात्म‚न} इति ॥
	\pend% ending standard par
      ‚{\tiny $_{lb}$}‚\textsuperscript{\textenglish{213/dm}}‚{\tiny $_{lb}$}‚
	  \bigskip
	  \begingroup
	

	  \pstart \leavevmode% starting standard par
	अयं च विरुद्ध आचार्य‚दिङ् नागेनोक्तः--
	\pend% ending standard par
       ‚{\tiny $_{lb}$}‚ 
	  \bigskip
	  \begingroup
	

	  \pstart \leavevmode% starting standard par
	स \edtext{}{\lemma{स}\Bfootnote{स क‚स्मा० \cite{dp-msC}}}इह क‚स्मान्नोक्तः ॥ ८९ ॥
	\pend% ending standard par
      
	  \endgroup
	‚{\tiny $_{lb}$}‚ 

	  \pstart \leavevmode% starting standard par
	स क‚स्माद् वार्त्तिक‚कारेण स‚ता त्व‚या नोक्तः ? इत‚र\edtext{}{\lemma{र}\Bfootnote{वार्त्तिंक‚कारः--\cite{dp-msD-n}}} आह--
	\pend% ending standard par
       ‚{\tiny $_{lb}$}‚ 
	  \bigskip
	  \begingroup
	

	  \pstart \leavevmode% starting standard par
	अन‚योरेवान्त‚र्भावात् ॥ ९० ॥
	\pend% ending standard par
      
	  \endgroup
	‚{\tiny $_{lb}$}‚ 

	  \pstart \leavevmode% starting standard par
	अन‚योरेव साध्य‚विप‚र्य‚य‚साध‚न‚योर‚न्त‚र्भावात् ।
	\pend% ending standard par
       ‚{\tiny $_{lb}$}‚ 

	  \pstart \leavevmode% starting standard par
	न‚नु चोक्त‚विप‚र्य‚पं न साध‚य‚ति । त‚त् क‚थ‚मुक्त‚विप‚र्य‚य‚साध‚न‚योरेवान्त‚र्भाव इत्याह--
	\pend% ending standard par
       ‚{\tiny $_{lb}$}‚ 
	  \bigskip
	  \begingroup
	

	  \pstart \leavevmode% starting standard par
	न ह्य‚य‚माभ्यां साध्य‚विप‚र्य‚य‚साध‚न‚त्वेन भिद्य‚ते ॥ ९१ ॥
	\pend% ending standard par
      
	  \endgroup
	‚{\tiny $_{lb}$}‚ 

	  \pstart \leavevmode% starting standard par
	न ह्य‚य‚मिति । हीति य‚स्माद‚र्थे । य‚स्माद् अय‚मिष्ट‚विधात‚कृदाभ्यां हेतुभ्यां साध्य‚{\tiny $_{lb}$}‚विप‚र्य‚य‚स्य\edtext{}{\lemma{स्य}\Bfootnote{विप‚र्य‚य‚साध०--\cite{dp-msA} \cite{dp-msB} \cite{dp-edP} \cite{dp-edH} \cite{dp-edE} \cite{dp-edN}}} साध‚न‚त्वेन न भिद्य‚ते । य‚था तौ साध्य‚विप‚र्य‚य‚साध‚नौ त‚थाऽय‚म‚पीति । उक्त‚{\tiny $_{lb}$}‚विप‚र्य‚यं तु साध‚य‚तु\edtext{}{\lemma{तु}\Bfootnote{ध्य‚तु मा वा \cite{dp-edE} \cite{dp-msD}}} वा मा वा किमुक्त‚विप‚र्य‚य‚साध‚नेन । त‚स्माद‚न‚योरेवान्त‚र्भावः ॥
	\pend% ending standard par
       ‚{\tiny $_{lb}$}‚ 

	  \pstart \leavevmode% starting standard par
	न‚नु चोक्त‚मेव साध्यं\edtext{}{\lemma{साध्यं}\Bfootnote{साध्ये \cite{dp-msA} \cite{dp-msB} \cite{dp-edP} \cite{dp-edH} \cite{dp-edN}}} त‚त् क‚थं साध्य‚विप‚र्य‚य‚साध‚न‚त्वेनाभेद इत्याह--
	\pend% ending standard par
       ‚{\tiny $_{lb}$}‚ 
	  \bigskip
	  \begingroup
	

	  \pstart \leavevmode% starting standard par
	न‚हीष्टोक्त‚योः साध्य‚त्वेन क‚श्चिद्विशेष इति ॥ ९२ ॥
	\pend% ending standard par
      
	  \endgroup
	‚{\tiny $_{lb}$}‚ 

	  \pstart \leavevmode% starting standard par
	न हीति य‚स्मादिष्टोक्त‚योः \edtext{}{\lemma{योः}\Bfootnote{प‚र‚स्प‚र‚स्य \cite{dp-msD} \cite{dp-msB} \cite{dp-edP} \cite{dp-edH} \cite{dp-edE}}}प‚र‚स्प‚र‚स्मात् साध्य‚त्वेन न क‚श्चिद्विशेषो भेद इति ।‚{\tiny $_{lb}$}‚ त‚स्माद‚न‚योरेवान्त‚र्भाव इत्युप‚संहारः ।
	\pend% ending standard par
       ‚{\tiny $_{lb}$}‚ 

	  \pstart \leavevmode% starting standard par
	प्र‚तिवादिनो हि य‚ज्जिज्ञासितं त‚त् प्र‚क‚र‚णाप‚न्न‚म् । य‚च्च प्र‚क‚र‚णाप‚न्नं त‚त् साध‚नेच्छ‚या‚{\tiny $_{lb}$}‚ विष‚यीकृतं साध्य‚मिष्ट‚मुक्त‚म‚नुक्तं वा, न तूक्त‚मात्र‚मेव साध्य‚म् । तेनाविशेष इति ॥
	\pend% ending standard par
       ‚{\tiny $_{lb}$}‚ 
	  \bigskip
	  \begingroup
	

	  \pstart \leavevmode% starting standard par
	द्व‚यो रूप‚योरेक‚स्यासिद्धाव‚प‚र‚स्य च स‚न्देहेऽनैकान्तिकः ॥ ९३ ॥
	\pend% ending standard par
      
	  \endgroup
	‚{\tiny $_{lb}$}‚ 

	  \pstart \leavevmode% starting standard par
	द्व‚यो रूप‚यो\edtext{}{\lemma{यो}\Bfootnote{रूप‚योर‚सिद्धौ विरुद्धः--\cite{dp-msA} \cite{dp-msB} \cite{dp-msC} \cite{dp-msD} \cite{dp-edP} \cite{dp-edH} \cite{dp-edN}}} र्विप‚र्य‚य‚सिद्धौ विरुद्ध उक्तः । \edtext{\textsuperscript{*}}{\lemma{*}\Bfootnote{अन‚योस्तु द्व‚यो० \cite{dp-msD} अन‚योर्द्व‚योर्म० \cite{dp-msA} \cite{dp-msB} \cite{dp-edP} \cite{dp-edH} \cite{dp-edE} \cite{dp-edN}}}त‚योस्तु द्व‚योर्म‚ध्य एक‚स्यासिद्धौ, अप‚र‚स्य‚{\tiny $_{lb}$}‚ च स‚न्देहेऽनैकान्तिकः ॥
	\pend% ending standard par
      
	  \endgroup
	‚{\tiny $_{lb}$}‚

	  \pstart \leavevmode% starting standard par
	\textbf{इत‚र} इति चोद‚काद‚न्यो \textbf{वार्तिक‚कार} इत्य‚र्थात् ।
	\pend% ending standard par
      ‚{\tiny $_{lb}$}‚

	  \pstart \leavevmode% starting standard par
	न‚ह्य‚क्त‚विप‚र्य‚य‚साध‚क‚त्वेन विरुद्ध उच्य‚ते । किन्त‚र्हि ? साध्य‚विप‚र्य‚य‚साध‚क‚त्वेन ।‚{\tiny $_{lb}$}‚ त‚स्मात् \textbf{किमुक्त‚विप‚र्य‚य‚साध‚नेने}त्युक्त‚म् ॥
	\pend% ending standard par
      ‚{\tiny $_{lb}$}‚

	  \pstart \leavevmode% starting standard par
	त‚मेव साध्य‚त्वेनाभेदं साध‚य‚न्नाह--\textbf{प्र‚तिवादिनो ही}ति । हिर्य‚स्माद‚र्थे । \textbf{तेन} साध्येच्छ‚या‚{\tiny $_{lb}$}‚ विष‚यीकृत‚मात्र‚स्य साध्य‚त्वेना\textbf{विशेषो}ऽभेदः ॥
	\pend% ending standard par
      ‚{\tiny $_{lb}$}‚‚{\tiny $_{lb}$}‚\textsuperscript{\textenglish{214/dm}}‚{\tiny $_{lb}$}‚
	  \bigskip
	  \begingroup
	

	  \pstart \leavevmode% starting standard par
	कीदृशोऽसावित्याह--
	\pend% ending standard par
       ‚{\tiny $_{lb}$}‚ 
	  \bigskip
	  \begingroup
	

	  \pstart \leavevmode% starting standard par
	य‚था वीत‚रागः \edtext{}{\lemma{रागः}\Bfootnote{क‚श्चित् नास्ति--\cite{dp-msC}}}क‚श्चित् स‚र्व‚ज्ञो वा, व‚क्तृत्वादिति\edtext{}{\lemma{क्तृत्वादिति}\Bfootnote{व‚क्तृत्वात् । व्य० \cite{dp-msC}}} । व्य‚तिरेको‚{\tiny $_{lb}$}‚ऽत्रासिद्धः । स‚न्दिग्धोऽन्व‚यः ॥ ९४ ॥
	\pend% ending standard par
      
	  \endgroup
	‚{\tiny $_{lb}$}‚ 

	  \pstart \leavevmode% starting standard par
	य‚थेति । विग‚तो रागो य‚स्य स वीत‚राग इत्येकं साध्य‚म् । स‚र्व‚ज्ञो वेति द्वितीय‚म् ।‚{\tiny $_{lb}$}‚ व‚क्तृत्वादिति हेतुः । व्य‚तिरेकोऽत्रासिद्ध इति । स्वात्म‚न्येव स‚रागे चास‚र्व‚ज्ञे च विप‚क्षे‚{\tiny $_{lb}$}‚ व‚क्तृत्वं दृष्ट‚म् । अतोऽसिद्धो व्य‚तिरेकः । स‚न्दिग्धोऽन्व‚यः ॥
	\pend% ending standard par
       ‚{\tiny $_{lb}$}‚ 

	  \pstart \leavevmode% starting standard par
	कुत इत्याह--
	\pend% ending standard par
       ‚{\tiny $_{lb}$}‚ 
	  \bigskip
	  \begingroup
	

	  \pstart \leavevmode% starting standard par
	स‚र्व‚ज्ञ‚वीत‚राग‚योर्विप्र‚क‚र्षाद्व‚च‚नादेस्त‚त्र स‚त्त्व‚म‚स‚त्त्वं वा स‚न्दिग्ध‚म् ॥ ९५ ॥
	\pend% ending standard par
      
	  \endgroup
	‚{\tiny $_{lb}$}‚ 

	  \pstart \leavevmode% starting standard par
	स‚प‚क्ष‚भूत‚योः स‚र्व‚ज्ञ‚वीत‚राग‚योर्विप्र‚क‚र्षादित्य‚तीन्द्रिय‚त्वाद् व‚च‚नादेरिन्द्रिय‚ग‚म्य‚स्यापि त‚त्र‚{\tiny $_{lb}$}‚ अतीन्द्रिय‚योः स‚र्व‚ज्ञ‚त्व‚वी \edtext{}{\lemma{वी}\Bfootnote{ज्ञ‚वी}} त‚राग‚योः स‚त्त्व‚म‚स‚त्त्वं वा स‚न्दिग्ध‚म् । त‚त‚श्च न ज्ञाय‚ते किं‚{\tiny $_{lb}$}‚ ब‚क्तृत्वात् स‚र्वंज्ञ उत नेत्य‚नैकान्तिकं इति ॥
	\pend% ending standard par
       ‚{\tiny $_{lb}$}‚ 

	  \pstart \leavevmode% starting standard par
	स‚म्प्र‚ति द्व‚योरेव स‚न्देहेऽनैकान्तिकं व‚क्तुमाह--
	\pend% ending standard par
       ‚{\tiny $_{lb}$}‚ 
	  \bigskip
	  \begingroup
	

	  \pstart \leavevmode% starting standard par
	अन‚योरेव द्व‚यो रूप‚योः स‚न्देहेऽनैकान्तिकः ॥ ९६ ॥
	\pend% ending standard par
      
	  \endgroup
	‚{\tiny $_{lb}$}‚ 

	  \pstart \leavevmode% starting standard par
	अन‚योरेव--अन्व‚य-व्य‚तिरेक‚रूप‚योः स‚न्देहात् संश‚य‚हेतुः ।
	\pend% ending standard par
       ‚{\tiny $_{lb}$}‚ 

	  \pstart \leavevmode% starting standard par
	उदाह‚र‚ण‚म्\edtext{}{\lemma{म्}\Bfootnote{उदाह‚र‚णं य‚था \cite{dp-msD} उदाह‚र‚णं च \cite{dp-msC}}}--
	\pend% ending standard par
       ‚{\tiny $_{lb}$}‚ 
	  \bigskip
	  \begingroup
	

	  \pstart \leavevmode% starting standard par
	य‚था\edtext{}{\lemma{था}\Bfootnote{य‚था नास्ति \cite{dp-msB} \cite{dp-msC} \cite{dp-msD} \cite{dp-edP} \cite{dp-edH}}} सात्म‚कं जीव‚च्छ‚रीरं प्राणादिम‚त्वादिति\edtext{}{\lemma{त्वादिति}\Bfootnote{०म‚त्वात् ।--\cite{dp-msC}}} ॥ ९७ ॥
	\pend% ending standard par
      
	  \endgroup
	‚{\tiny $_{lb}$}‚ 

	  \pstart \leavevmode% starting standard par
	य‚थेति\edtext{}{\lemma{थेति}\Bfootnote{य‚थेति नास्ति--\cite{dp-msA} \cite{dp-msB} \cite{dp-msC} \cite{dp-msD} \cite{dp-edP} \cite{dp-edH} \cite{dp-edN}}} । स‚हात्म‚ना व‚र्त्तंते सात्म‚क‚मिति साध्य‚म् । श‚रीर‚मिति ध‚र्मी । जीव‚द्ग्र‚ह‚णं‚{\tiny $_{lb}$}‚ ध‚र्मिविशेष‚ण‚म् । मृते ह्यात्मानं नेच्छ‚ति ।
	\pend% ending standard par
       ‚{\tiny $_{lb}$}‚ 

	  \pstart \leavevmode% starting standard par
	प्राणाः \edtext{}{\lemma{प्राणाः}\Bfootnote{प्राणा आश्वा० \cite{dp-msA} \cite{dp-msB} \cite{dp-edP} \cite{dp-edH} \cite{dp-edE} \cite{dp-edN}}}श्वासाद‚य आदिर्य‚स्योन्मेष‚निमेषादेः प्राणिध‚र्म‚स्य स प्राणादिः । स य‚स्यास्ति‚{\tiny $_{lb}$}‚ त‚त् प्राणादिम‚त् जीव‚च्छ‚रीर‚म् । त‚स्य भाव‚स्त‚त्त्व‚म् । त‚स्मादित्येष हेतुः । अय‚म‚साधार‚णः‚{\tiny $_{lb}$}‚ संश‚य‚हेतुरुप‚पाद‚यित‚व्यः ॥
	\pend% ending standard par
      
	  \endgroup
	‚{\tiny $_{lb}$}‚

	  \pstart \leavevmode% starting standard par
	\textbf{जीव}त्प्राणान् धार‚य‚च्च त‚च्छ‚रीरं चेति विग्र‚हः । \textbf{उन्मेष}श्च‚क्षुर्विकाश आदिर्य‚स्य \textbf{निमेषा}‚{\tiny $_{lb}$}‚देस्त‚स्य \textbf{प्राणिध‚र्म‚स्य} जीव‚ध‚र्म‚स्य । \textbf{असाधार‚णः} स‚प‚क्ष‚विप‚क्षावृत्तेः । विवादाध्यासित‚स्यैव‚{\tiny $_{lb}$}‚ ध‚र्मिणो ध‚र्म इत्य‚र्थः । असाधार‚ण‚त्वादेव \textbf{संश‚य‚हेतुरि}ति हेतुभावेन विशेष‚ण‚म् ॥
	\pend% ending standard par
      ‚{\tiny $_{lb}$}‚‚{\tiny $_{lb}$}‚\textsuperscript{\textenglish{215/dm}}‚{\tiny $_{lb}$}‚
	  \bigskip
	  \begingroup
	

	  \pstart \leavevmode% starting standard par
	प‚क्ष‚ध‚र्म‚स्य च द्वाभ्यां कार‚णाभ्यां संश‚य‚हेतुत्व‚म् । संश‚य‚विष‚यौ यावाकारौ ताभ्यां‚{\tiny $_{lb}$}‚ स‚र्व‚स्य व‚स्तुनः संग्र‚हात् । त‚योश्च व्याप‚क‚योराकार‚योरेक‚त्रापि वृत्त्य‚निश्च‚यात् । \edtext{\textsuperscript{*}}{\lemma{*}\Bfootnote{य‚काभ्यां \cite{dp-msA} \cite{dp-msB} \cite{dp-msC} \cite{dp-msD} \cite{dp-edP} \cite{dp-edH} \cite{dp-edN}}}याभ्यां‚{\tiny $_{lb}$}‚ ह्याकाराभ्यां स‚र्वं व‚स्तु न संगृह्य‚ते त‚योराकार‚योर्न संश‚यः । प्र‚कारान्त‚र‚स‚म्भ‚वे हि प‚क्ष‚ध‚र्मो‚{\tiny $_{lb}$}‚ ध‚र्मिण‚म‚वियुक्तं द्व‚योरेकेन ध‚र्मेण द‚र्श‚यितुं न श‚क्नुयात् । अतो न संश‚य‚हेतुः स्यात् ।‚{\tiny $_{lb}$}‚ द्व‚योर्ध‚र्म‚योर‚निय‚तं भावं द‚र्श‚य‚न् संश‚य‚हेतुः । द्व‚योस्त्व‚निय‚त‚म‚पि भावं द‚र्श‚यितुम‚श‚क्तो\edtext{}{\lemma{क्तो}\Bfootnote{श‚श‚विषाणादिः--\cite{dp-msD-n}}}ऽप्र‚तिप‚त्ति‚{\tiny $_{lb}$}‚हेतुः । निय‚तं तु\edtext{}{\lemma{तु}\Bfootnote{तु नास्ति \cite{dp-msA} \cite{dp-msB} \cite{dp-edP} \cite{dp-edH} \cite{dp-edE} \cite{dp-edN}}} भावं द‚र्श‚य‚न् \edtext{}{\lemma{न्}\Bfootnote{स‚म्य‚ग् नास्ति \cite{dp-msA} \cite{dp-msB} \cite{dp-edP} \cite{dp-edH} \cite{dp-edE} \cite{dp-edN}}}स‚म्य‚ग् हेतुर्विरुद्धो वा स्यात् । त‚स्माद् \edtext{}{\lemma{स्माद्}\Bfootnote{य‚काभ्यां \cite{dp-msB} \cite{dp-msC} \cite{dp-msD} \cite{dp-edP} \cite{dp-edH} \cite{dp-edN}}}याभ्यां स‚र्वं व‚स्तु‚{\tiny $_{lb}$}‚ संगृह्य‚ते त‚योः\edtext{}{\lemma{योः}\Bfootnote{त‚योराकार‚योः सं० \cite{dp-msC}}} संश‚य‚हेतुर्य‚दि त‚योरेक‚त्रापि स‚द्भाव‚निश्च‚यो न स्यात् । स‚द्भाव‚निश्च‚ये तु‚{\tiny $_{lb}$}‚ य‚द्येक‚त्र निय‚त‚स‚त्तानिश्च‚यो\edtext{}{\lemma{यो}\Bfootnote{०यो विरुद्धो हेतुर्वा स्यात् \cite{dp-msA} \cite{dp-msB} \cite{dp-msD} \cite{dp-edP} \cite{dp-edH} \cite{dp-edE} \cite{dp-edN}}} हेतुर्विरुद्धो वा स्यात् । अनिय‚त‚स‚त्तानिश्च‚ये तु साधार‚णानैकान्तिकः,‚{\tiny $_{lb}$}‚ स‚न्दिग्ध‚विप‚क्ष‚व्यावृत्तिकः, स‚न्दिग्धान्व‚योऽसिद्ध‚व्य‚तिरेको वा स्यात् । एक‚त्रापि तु
	\pend% ending standard par
      
	  \endgroup
	‚{\tiny $_{lb}$}‚

	  \pstart \leavevmode% starting standard par
	क‚थं पुन‚र‚यं संश‚य‚हेतुरुत्पाद‚यितुं श‚क्य‚ते याव‚ता नास्मात्सात्म‚क‚त्व‚स्यानात्म‚क‚त्व‚स्य‚{\tiny $_{lb}$}‚ वा प्र‚तिप‚त्तिर्जाय‚ते । त‚तोऽप्र‚तिप‚त्तिरेवासाधार‚ण इ\textbf{त्युद्द्योत‚क‚र‚म}त‚माश‚ङ्क्य य‚त्र \edtext{}{\lemma{त्र}\Bfootnote{य‚त् न}}‚{\tiny $_{lb}$}‚ हीत्युक्तं वार्तिक‚कृता त‚द‚व‚तार‚यितुं भूमिकां र‚च‚य‚न्नाह--\textbf{प‚क्ष‚ध‚र्म‚स्ये}ति । \textbf{चो} य‚स्माद‚र्थे ।‚{\tiny $_{lb}$}‚ प‚क्ष‚स्य ध‚र्म‚स्य स‚तो \textbf{द्वाभ्यां कार‚णाभ्यां} निमित्ताभ्याम् । \textbf{स‚र्व‚स्य} निःशेष‚स्य संग्र‚हाज्ज्ञाप‚नात् ।‚{\tiny $_{lb}$}‚ \textbf{त‚योर्व्याप‚क‚योराकार‚योः} सात्म‚क‚त्वानात्म‚क‚त्वाख्य‚योर्विष‚य‚भूत‚योरेक‚त्रापि विष‚येऽ\textbf{निश्च‚यात्त}स्य‚{\tiny $_{lb}$}‚ प‚क्ष‚ध‚र्म‚स्येत्य‚र्थात् ।
	\pend% ending standard par
      ‚{\tiny $_{lb}$}‚

	  \pstart \leavevmode% starting standard par
	न‚नु संश‚य्य‚मान‚योराकार‚योः स‚र्व‚व‚स्तुव्याप‚नेन किं ? येन त‚थात्वं त‚योरुप‚व‚र्ण्य‚त‚{\tiny $_{lb}$}‚ इत्याह--\textbf{याभ्यामि}ति । हीति य‚स्मात् । क‚थं पुन‚स्त‚योराकार‚योर्न संश‚य इत्याह--\textbf{प्र‚कारान्त‚रे}ति ।‚{\tiny $_{lb}$}‚ हिर्य‚स्माद‚र्थे । \textbf{प‚क्ष‚ध‚र्मः} स‚न् \textbf{ध‚र्मिणं} तं प‚क्षं \textbf{द्व‚योरेकेन ध‚र्मेण} सात्म‚क‚त्वाख्येन‚{\tiny $_{lb}$}‚ अनात्म‚त्वाख्येन वा । \textbf{अवियुक्तं} युक्तं स‚म्ब‚द्ध‚मिति याव‚त् । \textbf{न श‚क्नोति द‚र्श‚यितुं} प्र‚कारान्त‚रेण‚{\tiny $_{lb}$}‚ स‚म्ब‚न्ध‚स्य त‚स्य स‚म्भ‚वात् । य‚त एव\textbf{म‚तः} स प‚क्ष‚ध‚र्मो \textbf{न संश‚य‚हेतुः स्या}त् । \textbf{त‚यो}राकार‚योरिति‚{\tiny $_{lb}$}‚ साम‚र्थ्यात् ।
	\pend% ending standard par
      ‚{\tiny $_{lb}$}‚

	  \pstart \leavevmode% starting standard par
	अथ क‚थं सात्म‚क‚त्वानात्म‚क‚त्वे द्वौ ध‚र्मौ द‚र्श‚य‚न्न‚पि संश‚य‚हेतुरुच्य‚त इत्याह--\textbf{द्व‚योरि}ति ।‚{\tiny $_{lb}$}‚ द‚र्श\leavevmode\ledsidenote{\textenglish{73a/ms}}\textbf{य‚न्नि}ति हेतौ श‚तुर्विधानाद‚निय‚त‚भाव‚प्र‚द‚र्श‚नादित्य‚र्थो बोद्ध‚व्यः ।
	\pend% ending standard par
      ‚{\tiny $_{lb}$}‚

	  \pstart \leavevmode% starting standard par
	न‚नु निय‚ताभाव‚म‚द‚र्श‚य‚न्न‚प्र‚तिप‚त्तिहेतुरेवायं युज्य‚त इत्याह--\textbf{द्व‚योस्त्वि}ति । तुश‚ब्दो‚{\tiny $_{lb}$}‚ य‚स्माद‚र्थे । निय‚त‚प्र‚द‚र्श‚क‚स्यापि किं न त‚थात्व‚मित्याह--\textbf{निय‚त‚त्वं\edtext{}{\lemma{त्वं}\Bfootnote{य‚तं}} त्वि}ति । \textbf{तुः} पूर्व‚व‚द्‚{\tiny $_{lb}$}‚ विशेष‚णार्थो वा । य‚स्मादेवं \textbf{त‚स्माद्} हेतोः ।
	\pend% ending standard par
      ‚{\tiny $_{lb}$}‚

	  \pstart \leavevmode% starting standard par
	\textbf{निय‚तो}ऽन्य‚त्रान‚नुगामी \textbf{स‚त्तानिश्च‚यो} य‚स्य स त‚था । विरुद्धोऽपि विप‚र्य‚ये स‚म्य‚घेतु‚{\tiny $_{lb}$}‚रेवेत्येक‚त्र स‚त्तानिश्च‚ये \textbf{विरुद्धो वा स्यादि}त्युक्त‚म् ।
	\pend% ending standard par
      ‚{\tiny $_{lb}$}‚‚{\tiny $_{lb}$}‚\textsuperscript{\textenglish{216/dm}}‚{\tiny $_{lb}$}‚
	  \bigskip
	  \begingroup
	

	  \pstart \leavevmode% starting standard par
	वृत्त्य‚निश्च‚याद‚साधार‚णानैकान्तिको भ‚व‚ति । त‚तोऽसाधार‚णानैकान्तिक‚स्यानैकान्तिक‚त्वे हेतुद्व‚यं‚{\tiny $_{lb}$}‚ द‚र्श‚यितुमाह--
	\pend% ending standard par
       ‚{\tiny $_{lb}$}‚ 
	  \bigskip
	  \begingroup
	

	  \pstart \leavevmode% starting standard par
	न हि \edtext{}{\lemma{हि}\Bfootnote{सात्म‚कानात्म‚का० \cite{dp-msC}}}सात्म‚क‚निरात्म‚काभ्याम‚न्यो राशिर‚स्ति य‚त्रायं\edtext{}{\lemma{त्रायं}\Bfootnote{य‚त्र प्राणा० \cite{dp-msD} \cite{dp-msB} \cite{dp-edP} \cite{dp-edH} \cite{dp-edE} \cite{dp-edN} प्राणादिर्व‚र्त्तेत \cite{dp-edN}}} प्राणादिर्व‚र्त्त‚ते‚{\tiny $_{lb}$}‚ \edtext{\textsuperscript{*}}{\lemma{*}\Bfootnote{र्व‚र्तेत}} ॥ ९८ ॥
	\pend% ending standard par
      
	  \endgroup
	‚{\tiny $_{lb}$}‚ 

	  \pstart \leavevmode% starting standard par
	न‚हीति । स‚हात्म‚ना व‚र्त्त‚ते सात्म‚कः । निष्क्रान्त आत्मा य‚स्मात् स निरात्म‚कः ।‚{\tiny $_{lb}$}‚ \edtext{\textsuperscript{*}}{\lemma{*}\Bfootnote{आभ्यां \cite{dp-msD}}}ताभ्यां य‚स्मान्नान्यो राशिर‚स्ति । किंभूतः ? य‚त्रायं व‚स्तुध‚र्मः प्राणादिर्व‚र्त्तेत ? त‚स्माद‚यं‚{\tiny $_{lb}$}‚ \edtext{\textsuperscript{*}}{\lemma{*}\Bfootnote{०द‚यं द्व‚योर्भ० \cite{dp-msC}}}त‚योर्भ‚व‚ति संश‚य‚हेतुः ॥
	\pend% ending standard par
       ‚{\tiny $_{lb}$}‚ 

	  \pstart \leavevmode% starting standard par
	क‚स्माद‚न्य‚राश्य‚भाव इत्याह--
	\pend% ending standard par
       ‚{\tiny $_{lb}$}‚ 
	  \bigskip
	  \begingroup
	

	  \pstart \leavevmode% starting standard par
	आत्म‚नो\edtext{}{\lemma{नो}\Bfootnote{आत्म‚वृत्ति० \cite{dp-msC}}} वृति-व्य‚व‚च्छेदाभ्यां \edtext{}{\lemma{च्छेदाभ्यां}\Bfootnote{स‚र्व‚स्य सं० \cite{dp-msC}}}स‚र्व‚संग्र‚हात् ॥ ९९ ॥
	\pend% ending standard par
      
	  \endgroup
	‚{\tiny $_{lb}$}‚ 

	  \pstart \leavevmode% starting standard par
	आत्म‚नो वृत्तिः स‚द्भावो व्य‚व‚च्छेदोऽभावः । ताभ्यां स‚र्व‚स्य व‚स्तुनः संग्र‚हात् क्रोडी‚{\tiny $_{lb}$}‚क‚र‚णात् । य‚त्र ह्यात्मा अस्ति त‚त् सात्म‚क‚म् । \edtext{\textsuperscript{*}}{\lemma{*}\Bfootnote{त‚द‚न्य‚न्नि० \cite{dp-msA} \cite{dp-msB} \cite{dp-edP} \cite{dp-edH} \cite{dp-edN}}}अन्य‚न्निरात्म‚क‚म् । त‚तो नान्यो राशिर‚स्ति—‚{\tiny $_{lb}$}‚\edtext{\textsuperscript{*}}{\lemma{*}\Bfootnote{इति नास्ति \cite{dp-msA} \cite{dp-msB} \cite{dp-msC} \cite{dp-msD} \cite{dp-edP} \cite{dp-edH} \cite{dp-edN}}}इति संश‚य‚हेतुत्व‚कार‚ण‚म् ॥
	\pend% ending standard par
      
	  \endgroup
	‚{\tiny $_{lb}$}‚

	  \pstart \leavevmode% starting standard par
	य‚दैक‚त्रैव स‚त्तानिश्च‚यो विरुद्धो वा स्यादित्युक्त‚म्, य‚दैक‚त्रैव स‚त्तानिश्च‚यो नास्ति त‚दा‚{\tiny $_{lb}$}‚ का वार्त्तेत्याह--\textbf{अनिय‚तेति} । \textbf{तु}र्विशेषार्थः । \textbf{अनिय‚तो}ऽत्रैवायं व‚र्त्त‚त इत्येवंरूप‚निय‚म‚शून्यो यः‚{\tiny $_{lb}$}‚ \textbf{स‚त्तानिश्च‚य}स्त‚स्मिन् स‚ति । य‚दोभ‚य‚त्र स‚त्तानिश्च‚य‚स्त‚दा स‚प‚क्ष‚विप‚क्ष‚साधार‚ण‚त्वात्साधार‚णः,‚{\tiny $_{lb}$}‚ य‚दा तु विप‚क्ष‚वृत्तिस‚म्भाव‚नायाम‚निश्च‚य\edtext{}{\lemma{य}\Bfootnote{निय‚त}}स‚त्तानिश्च‚य‚स्त‚दा स‚न्दिग्ध‚विप‚क्ष‚व्यावृत्तिकः ।‚{\tiny $_{lb}$}‚ य‚दा पुनः स‚प‚क्षे वृत्तिस‚न्देह‚नानिय‚त‚स‚त्तानिश्च‚य‚स्त‚दा स‚न्धिग्धान्व‚यासिद्ध‚व्य‚तिरेकः । य‚दा तु‚{\tiny $_{lb}$}‚ स‚प‚क्षास‚प‚क्ष‚योरेक‚त्रापि स‚त्तानिश्च‚यो नास्ति त‚दा स‚प‚क्षास‚प‚क्षावृत्तेर‚साधार‚णः । एत‚देवाह—‚{\tiny $_{lb}$}‚\textbf{एक‚त्रापीति} । य‚तः प‚क्ष‚ध‚र्म‚स्योक्ताभ्यां कार‚णाभ्यां संश‚य‚हेतुत्वं \textbf{त‚त}स्त‚स्मात् \textbf{द‚र्श‚यितुं}‚{\tiny $_{lb}$}‚ द‚र्श‚यिष्यामीति म‚त्त्वा ।
	\pend% ending standard par
      ‚{\tiny $_{lb}$}‚

	  \pstart \leavevmode% starting standard par
	राशिः प्र‚कारः । व‚स्तुध‚र्म‚त्वं च प्राणादेर‚व‚स्तुनि श‚श‚विषाणादाव‚वृत्तेः । य‚तो राश्य‚{\tiny $_{lb}$}‚न्त‚राभाव\textbf{स्त‚स्मा}त्कार‚णाद‚यं प्राणादिम‚त्त्वाख्यो हेतुः । \textbf{त‚योः} सात्म‚क‚निरात्म‚क‚योः ।
	\pend% ending standard par
      ‚{\tiny $_{lb}$}‚

	  \pstart \leavevmode% starting standard par
	साधार‚ण‚स्य ध‚र्म‚स्य संश‚य‚हेतुत्वे द्वे कार‚णे । त‚त्रामुना \textbf{न‚हीत्या}दिना मूलेन संश‚य‚विष‚याभ्या‚{\tiny $_{lb}$}‚माकाराभ्यां स‚र्व‚व‚स्तुसंग्र‚ह एकं कार‚ण‚मुक्त‚म् । \textbf{नाप्य‚न‚यो}रित्यादिना तु त‚योरेक‚त्रापि‚{\tiny $_{lb}$}‚ वृत्त्य‚निश्च‚यो द्वितीयं कार‚ण‚मुक्त‚मिति द‚र्श‚यितुमाह--\textbf{संश‚येति} ॥
	\pend% ending standard par
      ‚{\tiny $_{lb}$}‚‚{\tiny $_{lb}$}‚\textsuperscript{\textenglish{217/dm}}‚{\tiny $_{lb}$}‚
	  \bigskip
	  \begingroup
	

	  \pstart \leavevmode% starting standard par
	प्र‚काराभ्यां \edtext{}{\lemma{काराभ्यां}\Bfootnote{स‚र्व‚संग्र‚हं \cite{dp-msA} \cite{dp-msB} \cite{dp-msC} \cite{dp-edP} \cite{dp-edH} \cite{dp-edE} \cite{dp-edN}}}स‚र्व‚व‚स्तुस‚ङ्ग्र‚हं प्र‚तिपाद्य द्वितीय‚माह--
	\pend% ending standard par
       ‚{\tiny $_{lb}$}‚ 
	  \bigskip
	  \begingroup
	

	  \pstart \leavevmode% starting standard par
	नाप्य‚न‚योरेक‚त्र वृत्तिनिश्च‚यः ॥ १०० ॥
	\pend% ending standard par
      
	  \endgroup
	‚{\tiny $_{lb}$}‚ 

	  \pstart \leavevmode% starting standard par
	नाप्य‚न‚योः सात्म‚कानात्म‚क‚योर्म‚ध्य एक‚त्र सात्म‚केऽनात्म‚के\edtext{}{\lemma{के}\Bfootnote{०के निरात्म‚के \cite{dp-msC}}} वा वृत्तेः स‚द्भाव‚स्य‚{\tiny $_{lb}$}‚ निश्च‚योऽस्ति । द्वाव‚पि राशी त्य‚क्त्वा न व‚र्त्त‚ते प्राणादिः, व‚स्तुध‚र्म‚त्वात् । त‚त‚श्चान‚योरेव‚{\tiny $_{lb}$}‚ व‚र्त्त‚ते\edtext{}{\lemma{ते}\Bfootnote{व‚र्त्त‚ते एताव० \cite{dp-msC}}} इत्येताव‚देव ज्ञात‚म् । विशेषे तु वृत्तिनिश्च‚यो नास्तीत्य‚य‚म‚र्थः\edtext{}{\lemma{र्थः}\Bfootnote{नास्तीत्य‚र्थः \cite{dp-msC} \cite{dp-msD}}} ॥
	\pend% ending standard par
       ‚{\tiny $_{lb}$}‚ 

	  \pstart \leavevmode% starting standard par
	त‚दाह--
	\pend% ending standard par
       ‚{\tiny $_{lb}$}‚ 
	  \bigskip
	  \begingroup
	

	  \pstart \leavevmode% starting standard par
	\edtext{\textsuperscript{*}}{\lemma{*}\Bfootnote{सात्म‚क‚त्वेन निरात्म‚क० \cite{dp-msB} \cite{dp-edP} \cite{dp-edH} \cite{dp-edE} \cite{dp-edN}}}सात्म‚क‚त्वेनाऽनात्म‚क‚त्वेन वा प्र‚सिद्धे प्राणादेर‚सिद्धेः\edtext{}{\lemma{सिद्धेः}\Bfootnote{र‚सिद्धिस्ताभ्यां न व्य‚तिरिच्य‚ते \cite{dp-edE}}} ॥ १०१ ॥
	\pend% ending standard par
      
	  \endgroup
	‚{\tiny $_{lb}$}‚ 

	  \pstart \leavevmode% starting standard par
	\edtext{\textsuperscript{*}}{\lemma{*}\Bfootnote{सात्म‚क‚त्वेन निरात्म‚क० \cite{dp-msD}}}सात्म‚क‚त्वेनाऽनात्म‚क‚त्वेन वा विशेषेण युक्ते प्र‚सिद्धे निश्चिते व‚स्तुनि प्राणादेर्ध‚र्म‚स्य‚{\tiny $_{lb}$}‚ \edtext{\textsuperscript{*}}{\lemma{*}\Bfootnote{०र्ध‚र्म‚स्यासिद्धेर‚नैकान्तिकोऽनिश्चि० \cite{dp-msA} \cite{dp-msB} \cite{dp-edP} \cite{dp-edH} \cite{dp-edE} \cite{dp-edN} ०र्ध‚र्म‚स्यासिद्धेर‚निश्चि० \cite{dp-msC}}}स‚र्व‚व‚स्तुव्यापिनोः प्र‚कार‚योरेक‚त्र निय‚त‚स‚द्भाव‚स्यासिद्धेर‚नैकान्तिकः, अनिश्चित‚त्वात् ।
	\pend% ending standard par
       ‚{\tiny $_{lb}$}‚ 

	  \pstart \leavevmode% starting standard par
	त‚देव‚म‚साधार‚ण‚स्य ध‚र्म‚स्यानैकान्तिक‚त्वे कार‚ण‚द्व‚य‚म‚भिहित‚म् ॥
	\pend% ending standard par
      
	  \endgroup
	‚{\tiny $_{lb}$}‚

	  \pstart \leavevmode% starting standard par
	\textbf{प्र‚काराभ्या}मात्म‚व्य‚व‚च्छेद‚रूपाभ्यामाकाराभ्याम् । \textbf{वृत्तिः} प्र‚वृत्तिर‚र्थात् भाव एवाव‚{\tiny $_{lb}$}‚तिष्ठ‚त इत्य‚भिप्रायेणा\textbf{ह--वृत्तेः स‚द्भाव‚स्ये}ति ।
	\pend% ending standard par
      ‚{\tiny $_{lb}$}‚

	  \pstart \leavevmode% starting standard par
	य‚द्येवं त‚योर्न व‚र्त्त‚त इत्येव किं न स्यात् ? त‚था च क‚थं संश‚य‚हेतुरित्याश‚ङ्क्य‚{\tiny $_{lb}$}‚ यादृशोऽस्यार्थोऽभिप्रेत‚स्तं स्फुट‚यितुमाह--\textbf{द्वाव‚पीति ।} कुतो न व‚र्त्त‚त इत्याह--\textbf{व‚स्तुध‚र्म‚त्वा‚{\tiny $_{lb}$}‚दि}ति । प्रायादे \edtext{}{\lemma{प्रायादे}\Bfootnote{प्राणादे}} रिति विभ‚क्तिविप‚रिणामेन स‚म्ब‚न्ध‚नीय‚म् । व‚स्तुना वाऽव‚श्यं‚{\tiny $_{lb}$}‚ सात्म‚केनाऽनात्म‚केन वा भाव्य‚मिति भावः ।
	\pend% ending standard par
      ‚{\tiny $_{lb}$}‚

	  \pstart \leavevmode% starting standard par
	त‚तो व‚स्तुस‚त्त्वेन सिद्ध‚प‚रित्यागेनान्य‚त्रावृत्तेः कार‚णात् । \textbf{अन‚योः} सात्म‚कानात्म‚क‚योः ।‚{\tiny $_{lb}$}‚ एव‚कारेणान्य‚त्रा \edtext{}{\lemma{त्रा}\Bfootnote{त्र}} वृत्तिनिषेधं स्प‚ष्ट‚य‚ति । इतिरेताव‚तः स्व‚रूपं द‚र्श‚य‚ति । य‚दिद‚म‚न‚न्त‚रोक्त‚{\tiny $_{lb}$}‚मेत‚त् प‚रिमाणं य‚स्य प्र‚मेय‚स्य त‚द् \textbf{एताव‚द्} व‚स्तुत‚त्त्वं निश्चित‚म् । कुत‚स्त‚र्हि नास्य वृत्ति‚{\tiny $_{lb}$}‚निश्च‚य इत्याह--\textbf{विशेषे त्वि}ति । \textbf{विशेषे} विशिष्टे प्र‚कारे । \textbf{तु}रिमाम‚व‚स्थां भेद‚व‚तीमाह ।‚{\tiny $_{lb}$}‚ तेन सात्म‚क‚त्व \edtext{}{\lemma{त्व}\Bfootnote{सात्म‚क}} एवानात्म‚क एवेत्य‚र्थः । \textbf{वृत्तेः} स्व‚भाव‚स्य प्राणादेरिति प्र‚क‚र‚णात् ।‚{\tiny $_{lb}$}‚ \textbf{इति}रेव‚म‚र्थे । \textbf{अर्थो}ऽभिधेयो य‚स्य \textbf{नाप्य‚न‚योरेक‚त्र वृत्तिनिश्च‚यः} इत्य‚स्य मौल‚स्य वाक्य‚स्ये‚{\tiny $_{lb}$}‚त्य‚र्थात् ॥
	\pend% ending standard par
      ‚{\tiny $_{lb}$}‚

	  \pstart \leavevmode% starting standard par
	य‚स्मादेव‚मेत‚द् व‚क्तुं युज्य‚ते, नान्य‚था त‚त्त‚स्मा \leavevmode\ledsidenote{\textenglish{73b/ms}} \textbf{दाह वार्त्तिक‚कारः ।} किमाहेत्याह‚{\tiny $_{lb}$}‚ \textbf{सात्मेति । त‚देव‚मि}त्यादिनोप‚संह‚र‚ति । एवं च व्याच‚क्षाणेन \add{न} म‚या स्वात‚न्त्र्येण \textbf{प‚क्ष‚ध‚र्म‚स्ये}‚{\tiny $_{lb}$}‚‚{\tiny $_{lb}$}‚ ‚{\tiny $_{lb}$}‚ \leavevmode\ledsidenote{\textenglish{218/dm}}‚{\tiny $_{lb}$}‚ 
	  
	प‚क्ष‚ध‚र्म‚श्च भ‚व‚न्\edtext{}{\lemma{न्}\Bfootnote{भ‚व‚त् स‚र्वः \cite{dp-msA}}} स‚र्वः साधार‚णोऽसाधार‚णो वा भ‚व‚त्य‚नैकान्तिकः । त‚स्मादुप‚संहार‚{\tiny $_{lb}$}‚व्याजेन प‚क्ष‚ध‚र्म‚त्वं द‚र्श‚य‚ति-- ‚{\tiny $_{lb}$}‚ 
	  
	त‚स्माज्जीव‚च्छ‚रीर‚स‚म्ब‚न्धी प्राणादिः सात्म‚काद‚नात्म‚काच्च स‚र्व‚स्माद्‚{\tiny $_{lb}$}‚ व्यावृत्त‚त्वोनासिद्धेस्ताभ्यां\edtext{}{\lemma{त्वोनासिद्धेस्ताभ्यां}\Bfootnote{०नासिद्धिः ॥ \cite{dp-edE} ताभ्यामि त्यादि नास्ति \cite{dp-edE}}} न व्य‚तिरिच्य‚ते ॥१०२॥‚{\tiny $_{lb}$}‚ 
	  
	त‚स्मादित्यादिना । जीव‚च्छ‚रीर‚स्य स‚म्ब‚न्धी प‚क्ष‚ध‚र्म इत्य‚र्थः । य‚स्मात् त‚योरेक‚त्रापि‚{\tiny $_{lb}$}‚ न निवृत्तिनिश्च‚य‚स्त‚स्मात् ताभ्यां न व्य‚तिरिच्य‚ते । ‚{\tiny $_{lb}$}‚ 
	  
	व‚स्तुध‚मे हि स‚र्व‚व‚स्तुव्यापिनोः प्र‚कार‚योरेक‚त्र निय‚त‚स‚द्भावो निश्चितः प्र‚कारान्त‚रान्नि‚{\tiny $_{lb}$}‚व‚र्त्तेत ।\edtext{\textsuperscript{*}}{\lemma{*}\Bfootnote{०रान्निव‚र्त्त‚ते । त‚त एवाह--\cite{dp-msD} ०रान्निव‚र्त्तेत । त‚त एवाह \cite{dp-msC} \cite{dp-msA} \cite{dp-msB} \cite{dp-edP} \cite{dp-edH} \cite{dp-edE} \cite{dp-edN}}} अत एवाह--सात्म‚काद‚नात्म‚काच्च स‚र्व‚स्माद् व‚स्तुनो व्यावृत्त‚त्वेनासिद्धेरित ।‚{\tiny $_{lb}$}‚ प्राणादिस्ताव‚त् कुत‚श्चिद् घ‚टादेर्निवृत्त एव । त‚त एताव‚द‚व‚सातुं श‚क्य‚म्--सात्म‚काद‚नात्म‚काद्वा‚{\tiny $_{lb}$}‚ किय‚तो निवृत्तः । स‚र्व‚स्मात् \edtext{}{\lemma{स्मात्}\Bfootnote{तु नास्ति--\cite{dp-msC}}}तु निवृत्तो नाव‚सीय‚ते । त‚तो न कुत‚श्चिद् व्य‚तिरेकः ॥‚{\tiny $_{lb}$}‚ दिनाऽसाधार‚ण‚स्य संश‚य‚हेतुत्व‚निमित्त‚द्व‚य‚मादितो द‚र्शित‚म् । किं त‚र्हि ? \textbf{वार्त्तिक‚कारेणैवैत‚द}‚{\tiny $_{lb}$}‚भिहित‚मिति द‚र्शित‚म् ॥
	\pend% ending standard par
      ‚{\tiny $_{lb}$}‚

	  \pstart \leavevmode% starting standard par
	न‚नु चासाधार‚ण‚स्य प्राणादेर‚नैकान्तिक‚त्व‚कार‚ण‚द्व‚य‚म‚न‚न्त‚रोक्त‚म‚भिधीय‚ताम् । \textbf{त‚स्मा‚{\tiny $_{lb}$}‚दि}त्यादिना तु श‚रीर‚स‚म्ब‚न्धित्व‚म‚स्य क‚स्मादाचार्यो द‚र्श‚य‚तीत्याश‚ङ्कां निराचिकीर्षुः \textbf{प‚क्ष‚{\tiny $_{lb}$}‚ध‚र्म‚श्चे}त्यादिनोप‚क्र‚म‚ते । \textbf{चो} य‚स्माद‚र्थे । अप‚क्ष‚ध‚र्म‚स्त्व‚सिद्ध‚त्वाख्याम‚न्यामेव दोष‚जातिम‚नुश्नुत‚{\tiny $_{lb}$}‚ इति भावः प‚क्ष‚ध‚र्म‚त्वं प्र‚द‚र्श‚य‚तो \textbf{वार्तिक‚कार}स्योन्नेयः । \textbf{वा}श‚ब्देनानिय‚त‚प्र‚भेदेऽनास्\textbf{थां} द‚र्श‚य‚ति ।‚{\tiny $_{lb}$}‚ न त्व‚साधार‚ण‚त्वाख्यं प‚क्षान्त‚र‚म्, स‚न्दिग्ध‚विप‚क्ष‚व्यावृत्तिकादेर‚स‚ङ्ग्र‚ह‚प्र‚स‚ङ्गात् । य‚स्माद‚प‚क्ष‚{\tiny $_{lb}$}‚ध‚र्मो नानैकान्तिक उप‚व‚र्णितेनाभिप्रायेण \textbf{त‚स्मा}त्कार‚णात् । \textbf{द‚र्श‚य‚ति} प्र‚काश‚य‚ति ।
	\pend% ending standard par
      ‚{\tiny $_{lb}$}‚

	  \pstart \leavevmode% starting standard par
	केन द‚र्श‚य‚तीत्याकाङ्क्षायामाह--\textbf{त‚स्मादित्यादिने}ति । प‚क्ष‚ध‚र्म‚त्व‚प्र‚द‚र्श‚नं तु \textbf{जीव‚च्छ‚रीर‚{\tiny $_{lb}$}‚स‚म्ब‚न्धी}ति व‚च‚नं द्र‚ष्ट‚व्य‚म् । \textbf{त‚स्मादि}त्य‚नेन य‚स्मादित्याक्षिप्तं द‚र्श‚य‚न्नाह--\textbf{य‚स्मादि}ति ।‚{\tiny $_{lb}$}‚ \textbf{एक‚त्रापि न वृत्तिनिश्च‚य}स्त‚स्य प्राणादेरित्य‚र्थात् । \textbf{त‚स्मा}त्कार‚णात् \textbf{ताभ्यां} सात्म‚क‚त्वानात्म‚{\tiny $_{lb}$}‚क‚त्वाभ्यां \textbf{न व्य‚तिरिच्य‚ते} न निव‚र्त्त‚ते, त‚द‚संस्प‚र्शी न भ‚व‚तीति याव‚त् ।
	\pend% ending standard par
      ‚{\tiny $_{lb}$}‚

	  \pstart \leavevmode% starting standard par
	त‚योरेक‚त्र वृत्त्य‚निश्च‚येऽपि क‚थं ताभ्यां न व्य‚तिरिच्य‚त इत्याह--\textbf{व‚स्त्विति । ही}ति‚{\tiny $_{lb}$}‚ य‚स्माद‚र्थे । \textbf{प्र‚कार‚यो}स्त‚द्वृत्तिव्य‚व‚च्छेद‚रूप‚योः स्व‚रूप‚योराकार‚योरिति याव‚त् । त‚योर्ध‚र्म‚{\tiny $_{lb}$}‚ \textbf{एक‚त्र निय‚तः} अत्रैवायं व‚र्त्त‚त इति निय‚म‚वान् \textbf{स‚द्भावः} स‚त्त्वं य‚स्य स त‚था । \textbf{प्र‚कारान्त‚रा‚{\tiny $_{lb}$}‚न्नि}य‚त‚स‚द्भाव‚विष‚याद‚न्य‚स्मादाकारात् । \textbf{निव‚र्त्तेत} निव‚र्तितुम‚र्ह‚ति, त‚न्न संस्पृशेदिति याव‚त् ।‚{\tiny $_{lb}$}‚ एक‚त्र वृत्त्य‚निश्च‚याच्च नायं त‚थेत्य‚भिप्रायः ।
	\pend% ending standard par
      ‚{\tiny $_{lb}$}‚

	  \pstart \leavevmode% starting standard par
	य‚त एव‚मेत‚द् भ‚व‚ति, न चायं प्राणादिस्त‚था, अत \textbf{एवा}हाचार्यः । किमाहेत्याह—‚{\tiny $_{lb}$}‚\textbf{सात्म‚कादि}त्यादि । \textbf{स‚र्व‚स्मादि}ति प्र‚त्येकं स‚म्ब‚द्ध \edtext{}{\lemma{द्ध}\Bfootnote{न्द्ध}} व्य‚म् ।
	\pend% ending standard par
      ‚{\tiny $_{lb}$}‚‚{\tiny $_{lb}$}‚\textsuperscript{\textenglish{219/dm}}‚{\tiny $_{lb}$}‚
	  \bigskip
	  \begingroup
	

	  \pstart \leavevmode% starting standard par
	य‚द्येव‚म‚न्व‚योऽस्तु त‚योर्निश्चित इत्याह--
	\pend% ending standard par
       ‚{\tiny $_{lb}$}‚ 
	  \bigskip
	  \begingroup
	

	  \pstart \leavevmode% starting standard par
	न\edtext{}{\lemma{न}\Bfootnote{न च त‚त्रा० \cite{dp-msC} \cite{dp-msD}}} त‚त्रान्वेति ॥ १०३ ॥
	\pend% ending standard par
      
	  \endgroup
	‚{\tiny $_{lb}$}‚ 

	  \pstart \leavevmode% starting standard par
	न\edtext{}{\lemma{न}\Bfootnote{न च त‚त्र \cite{dp-msD}}} त‚त्र सात्म‚केऽनात्म‚के वाऽर्थेन्वेति--अन्व‚य‚वान् प्राणादिः ॥
	\pend% ending standard par
       ‚{\tiny $_{lb}$}‚ 

	  \pstart \leavevmode% starting standard par
	कुत इत्याह--
	\pend% ending standard par
       ‚{\tiny $_{lb}$}‚ 
	  \bigskip
	  \begingroup
	

	  \pstart \leavevmode% starting standard par
	एकात्म‚न्य‚प्य‚सिद्धेः ॥ १०४ ॥
	\pend% ending standard par
      
	  \endgroup
	‚{\tiny $_{lb}$}‚ 

	  \pstart \leavevmode% starting standard par
	\edtext{\textsuperscript{*}}{\lemma{*}\Bfootnote{एकात्म‚न्य‚पीति--नास्ति \cite{dp-msA} \cite{dp-msB} \cite{dp-msC} \cite{dp-msD} \cite{dp-edP} \cite{dp-edH} \cite{dp-edN}}}एकात्म‚न्य‚पीति । एकात्म‚नि सात्म‚केऽनात्म‚के वाऽसिद्धेः कार‚णात् । व‚स्तुध‚र्म‚त‚या‚{\tiny $_{lb}$}‚ त‚योर्द्व‚योरेक‚त्र\edtext{}{\lemma{त्र}\Bfootnote{०र्द्व‚योर‚प्येक‚त्र \cite{dp-msC} \cite{dp-msD} ०क‚त्र ताव‚त् व‚र्त्त‚त इ० \cite{dp-msC}}} वा व‚र्त्त‚त इत्य‚व‚सितः प्राणादिः । न तु सात्म‚क एव निरात्म‚क एव वा‚{\tiny $_{lb}$}‚ व‚र्त्त‚त इति कुतोऽन्व‚य‚निश्च‚यः ॥
	\pend% ending standard par
       ‚{\tiny $_{lb}$}‚ 

	  \pstart \leavevmode% starting standard par
	न‚नु च प्र‚तिवादिनो न किञ्च‚त् सात्म‚क‚म‚स्ति । त‚तोऽस्य हेतोर्न सात्म‚केऽन्व‚यो\edtext{}{\lemma{यो}\Bfootnote{अनुग‚म‚नं स‚द्भाव इत्य‚र्थः \cite{dp-msD-n}}} न‚{\tiny $_{lb}$}‚ व्य‚तिरेक\edtext{}{\lemma{तिरेक}\Bfootnote{व्यावृत्तिः--अभाव इत्य‚र्थः--\cite{dp-msD-n}}} इत्य‚न्व‚य‚व्य‚तिरेक‚योर‚भाव‚निश्च‚यः सात्म‚के, न तु स‚द्भाव‚संश‚य इत्याश‚ङ्क्याह--
	\pend% ending standard par
       ‚{\tiny $_{lb}$}‚ 
	  \bigskip
	  \begingroup
	

	  \pstart \leavevmode% starting standard par
	नापि \edtext{}{\lemma{नापि}\Bfootnote{सात्म‚कान्निरात्म‚का० \cite{dp-msB} \cite{dp-msD} \cite{dp-edP} \cite{dp-edH} \cite{dp-edE} \cite{dp-edN}}}सात्म‚काद‚नात्म‚काच्च त‚स्यान्व‚य‚व्य‚तिरेक‚योर‚भाव‚निश्च‚यः ॥ १०५ ॥
	\pend% ending standard par
      
	  \endgroup
	
	  \endgroup
	‚{\tiny $_{lb}$}‚

	  \pstart \leavevmode% starting standard par
	न‚नु घ‚ट‚प‚टादेर‚नेक‚स्मात्प्राणादिर्निव‚र्त्त‚मानो दृष्ट‚स्त‚त्क‚थं त‚स्य व्यावृत्त‚त्वेनासिद्धि‚{\tiny $_{lb}$}‚रित्याह--\textbf{प्राणादिरिति । त‚तः} स‚र्व‚स्मात् सात्म‚काद‚नात्म‚काच्च निवृत्त्य‚न‚व‚सायात् । कुतः ?‚{\tiny $_{lb}$}‚ सात्म‚काद‚नात्म‚काच्च प्र‚तिब‚न्धासिद्धेरिति चात्र स‚र्व‚त्राभिप्रायः ।
	\pend% ending standard par
      ‚{\tiny $_{lb}$}‚

	  \pstart \leavevmode% starting standard par
	न‚नु किमुच्य‚ते \textbf{न कुत‚श्चिदि}ति ? याव‚ता निरात्म‚कादेव व्य‚तिरेकोऽस्याव‚सातुं श‚क्यः,‚{\tiny $_{lb}$}‚ बौद्धेन घ‚टादेर्निरात्म‚क‚त्वेनेष्ट‚त्वादिति चेत् । य‚द्येवं जीव‚च्छ‚रीर‚म‚पि \textbf{बौद्धेन} त‚थात्वेनेष्ट‚मिति‚{\tiny $_{lb}$}‚ त‚स्यापि त‚थात्वं किन्न भ‚वेत् । अभ्युप‚ग‚मेन च सात्म‚कानात्म‚के विभ‚ज्य हेतुं क‚थ‚य‚ता ग‚मिक‚त्व‚{\tiny $_{lb}$}‚मिति य‚त्किञ्चिदेत‚त् ॥
	\pend% ending standard par
      ‚{\tiny $_{lb}$}‚

	  \pstart \leavevmode% starting standard par
	त‚योरिति विष‚य‚स‚प्त‚मी । त‚स्य प्राणादेरिति च शेषः ॥
	\pend% ending standard par
      ‚{\tiny $_{lb}$}‚

	  \pstart \leavevmode% starting standard par
	न‚नु व‚स्तुध‚र्मेण तेनाव‚श्यं क्वापि निय‚तेन भाव्य‚म् । त‚त्क‚थ‚म‚न्व‚याभाव इत्याह—‚{\tiny $_{lb}$}‚\textbf{व‚स्तुध‚र्म‚त‚ये}ति । न चानिय‚त‚वृत्तिनिश्च‚योऽन्व‚यो नामेति भावः ।
	\pend% ending standard par
      ‚{\tiny $_{lb}$}‚

	  \pstart \leavevmode% starting standard par
	न‚नु चासाधार‚ण‚त्वान्निरात्म‚केऽन्व‚य‚निश्च‚यो मा भूद् । व्य‚तिरेक‚निश्च‚य‚स्त्व‚स्तु, निरात्म‚के‚{\tiny $_{lb}$}‚ घ‚टादौ प्राणादेर‚द‚र्श‚नादिति चेत् । न । त‚स्यैव \leavevmode\ledsidenote{\textenglish{74a/ms}} श‚रीर‚स्य निरात्म‚क‚त्व‚स‚म्भाव‚नायां‚{\tiny $_{lb}$}‚ स‚र्व‚स्मान्निरात्म‚कान्निवृत्तिनिश्च‚याभावात् । न च त‚थानिश्च‚य‚निमित्तं प्र‚तिब‚न्ध‚निश्च‚योऽ‚{\tiny $_{lb}$}‚ ऽस्तीति ॥
	\pend% ending standard par
      ‚{\tiny $_{lb}$}‚‚{\tiny $_{lb}$}‚\textsuperscript{\textenglish{220/dm}}‚{\tiny $_{lb}$}‚
	  \bigskip
	  \begingroup
	

	  \pstart \leavevmode% starting standard par
	नापि सात्म‚काद् व‚स्तुनः त‚स्य प्राणादेर‚न्व‚य‚व्य‚तिरेक‚योर‚भाव‚निश्च‚यः । नापि च‚{\tiny $_{lb}$}‚ निरात्म‚कात् । सात्म‚काद‚नात्म‚कादिति च प‚ञ्च‚मी व्य‚तिरेक‚श‚ब्दापेक्ष‚या द्र‚ष्ट‚व्या ॥
	\pend% ending standard par
       ‚{\tiny $_{lb}$}‚ 

	  \pstart \leavevmode% starting standard par
	क‚थ‚म‚न्व‚य‚व्य‚तिरेक‚योर्नाभाव‚निश्च‚य इत्याह--
	\pend% ending standard par
       ‚{\tiny $_{lb}$}‚ 
	  \bigskip
	  \begingroup
	

	  \pstart \leavevmode% starting standard par
	\edtext{\textsuperscript{*}}{\lemma{*}\Bfootnote{एक‚स्याभा० \cite{dp-msC}}}एकाभाव‚निश्च‚य‚स्याप‚र\edtext{}{\lemma{र}\Bfootnote{०स्याप‚राभाव‚नान्त० \cite{dp-msB} \cite{dp-edP} \cite{dp-edH} ०स्याप‚र‚भाव‚नान्त‚री० \cite{dp-msD} \cite{dp-edE}}} भाव‚निश्च‚य‚नान्त‚रीय‚क‚त्वात् ॥ १०६ ॥
	\pend% ending standard par
      
	  \endgroup
	‚{\tiny $_{lb}$}‚ 

	  \pstart \leavevmode% starting standard par
	एक‚स्यान्व‚य‚स्य व्य‚तिरेक‚स्य वा योऽभाव‚निश्च‚यः \edtext{}{\lemma{यः}\Bfootnote{स एवाप‚र‚स्य \cite{dp-edE} \cite{dp-edN}}}सोऽप‚र‚स्य द्वितीय‚स्य \edtext{}{\lemma{स्य}\Bfootnote{भावे निश्च० \cite{dp-msA} \cite{dp-msB} \cite{dp-edP} \cite{dp-edH}}}भाव‚निश्च‚य‚{\tiny $_{lb}$}‚नान्त‚रीय‚कः \edtext{}{\lemma{कः}\Bfootnote{०रीय‚कः भ‚व‚ति निश्च० \cite{dp-msA} \cite{dp-msB} \cite{dp-edP} \cite{dp-edH}}}भाव‚निश्च‚य‚स्याव्य‚भिचारी । त‚स्य भाव‚स्त‚त्त्वं त‚स्मात् । य‚त एकाभाव‚निश्च‚यो‚{\tiny $_{lb}$}‚ऽप‚र‚भाव‚निश्च‚य\edtext{}{\lemma{य}\Bfootnote{निश्च‚य नास्ति \cite{dp-msA} \cite{dp-msC}}} नान्त‚रीय‚कः, त‚स्मान्न द्व‚योरेक‚त्राभाव‚निश्च‚यः ॥
	\pend% ending standard par
       ‚{\tiny $_{lb}$}‚ 

	  \pstart \leavevmode% starting standard par
	क‚स्मात् पुन‚रेक‚स्याभाव‚निश्च‚योऽप‚र‚स‚द्भाव‚निश्च‚याऽव्य‚भिचारीत्याह--
	\pend% ending standard par
       ‚{\tiny $_{lb}$}‚ 
	  \bigskip
	  \begingroup
	

	  \pstart \leavevmode% starting standard par
	\edtext{\textsuperscript{*}}{\lemma{*}\Bfootnote{अत एवान्व‚य० \cite{dp-msC}}}अन्व‚य-व्य‚तिरेक‚योर‚न्योन्य‚व्य‚व‚च्छेद‚रूप‚त्वात् । \edtext{\textsuperscript{*}}{\lemma{*}\Bfootnote{अत एव \cite{dp-msD} \cite{dp-msB} \cite{dp-edP} \cite{dp-edH} \cite{dp-edE} \cite{dp-edN}}}त‚त एवान्व‚य‚{\tiny $_{lb}$}‚व्य‚तिरेक‚योः स‚न्देहाद‚नैकान्तिकः ॥ १०७ ॥
	\pend% ending standard par
      
	  \endgroup
	‚{\tiny $_{lb}$}‚ 

	  \pstart \leavevmode% starting standard par
	अन्व‚य‚व्य‚तिरेक‚योर‚न्योन्य‚व्य‚व‚छेद‚रूप‚त्वादिति । अन्योन्य‚स्य व्य‚व‚च्छेदोऽभावः, स‚{\tiny $_{lb}$}‚ एव रूपं य‚योस्त‚योर्भाव‚स्त‚त्त्व‚म् त‚स्मात् कार‚णात् ।
	\pend% ending standard par
       ‚{\tiny $_{lb}$}‚ 

	  \pstart \leavevmode% starting standard par
	अन्व‚य‚व्य‚तिरेकौ भावाभावौ । भावाभावौ च प‚र‚स्प‚र‚व्य‚व‚च्छेद‚रूपौ । य‚स्य व्य‚व‚च्छेदेन‚{\tiny $_{lb}$}‚ य‚त् प‚रिच्छिद्य‚ते त‚त् त‚त्प‚रिहारेण व्य‚व‚स्थित‚म् । स्वाभाव‚व्य‚व‚च्छेदेन च भावः प‚रिच्छिद्य‚ते ।
	\pend% ending standard par
       ‚{\tiny $_{lb}$}‚ 

	  \pstart \leavevmode% starting standard par
	त‚स्मात् स्वाभाव‚व्य‚व‚च्छेदेन भावो व्य‚व‚स्थितः । अभावो हि नीरूपो यादृशो विक‚ल्पेन‚{\tiny $_{lb}$}‚ द‚र्शितः । नीरूप‚तां च व्य‚व‚च्छिद्य रूप‚माकार‚व‚त् प‚रिच्छिद्य‚ते । त‚था च स‚त्य‚न्व‚याभावो‚{\tiny $_{lb}$}‚ व्य‚तिरेकः, व्य‚तिरेकाभाव‚श्चान्व‚यः ।
	\pend% ending standard par
       ‚{\tiny $_{lb}$}‚ 

	  \pstart \leavevmode% starting standard par
	त‚तोऽन्व‚याभावे निश्चिते व्य‚तिरेको निश्चितो भ‚व‚ति । व्य‚तिरेकाभावे च निश्चिते‚{\tiny $_{lb}$}‚ऽन्व‚यो निश्चितो भ‚व‚ति ।
	\pend% ending standard par
      
	  \endgroup
	‚{\tiny $_{lb}$}‚

	  \pstart \leavevmode% starting standard par
	\textbf{न‚नु चे}त्यादि \textbf{त‚स्मात्कार‚णा}दित्येत‚द‚न्तं स्प‚ष्टार्थ तेन न व्याख्याय‚ते ।
	\pend% ending standard par
      ‚{\tiny $_{lb}$}‚

	  \pstart \leavevmode% starting standard par
	अन्योन्य‚व्य‚व‚च्छेद‚रूप‚त्व‚मेवान्व‚य‚व्य‚तिरेक‚योः क‚थ‚मित्याश‚ङ‚क्याह--\textbf{अन्व‚य‚व्य‚तिरेका‚{\tiny $_{lb}$}‚वि}ति । अभाव‚रूप‚त्व‚ञ्च व्य‚तिरेक‚स्य प्र‚तीतिसिद्ध‚स्य बोद्ध‚व्य‚म् । भ‚व‚तां तौ त‚थारूपौ‚{\tiny $_{lb}$}‚ किम‚त इत्याह--\textbf{भावाभावावि}ति । \textbf{चो} यास्माद‚र्थे । भ‚व‚त्वेवं त‚थापि क‚थं त‚योर‚न्योन्य‚प‚रि‚{\tiny $_{lb}$}‚हारेणाव‚स्थान‚मित्याह--\textbf{य‚स्ये}ति ।
	\pend% ending standard par
      ‚{\tiny $_{lb}$}‚

	  \pstart \leavevmode% starting standard par
	न‚न्व‚त्र क‚स्य व्य‚व‚च्छेदेन किं प‚रिच्छिद्य‚ते येन त‚त्प‚रिहारेह‚ण त‚द् व्य‚व‚तिष्ठ‚त इत्याह—‚{\tiny $_{lb}$}‚\textbf{स्वाभावे}ति । हेत्व‚र्थ‚श्च‚कारः । ताव‚त्कासा \edtext{}{\lemma{त्कासा}\Bfootnote{कोऽसा}} व‚भावो नाम य‚द्व्य‚व‚च्छेदेन भावः‚{\tiny $_{lb}$}‚ ‚{\tiny $_{lb}$}‚ \leavevmode\ledsidenote{\textenglish{221/dm}}‚{\tiny $_{lb}$}‚ 
	  
	त‚स्माद् य‚दि नाम सात्म‚क‚म‚व‚स्तु निरात्म‚कं च व‚स्तु, त‚थापि \edtext{}{\lemma{थापि}\Bfootnote{त‚थापि न त‚योः \cite{dp-msA} \cite{dp-msB} \cite{dp-edP} \cite{dp-edH} \cite{dp-edE} \cite{dp-edN}}}त‚योर्न प्राणादेर‚न्व‚य‚{\tiny $_{lb}$}‚व्य‚तिरेक‚योर‚भाव‚निश्च‚यः । एक‚त्र\edtext{}{\lemma{त्र}\Bfootnote{एक‚व‚स्तु० \cite{dp-msA} \cite{dp-msB} \cite{dp-edP} \cite{dp-edH} \cite{dp-edE} \cite{dp-edN}}} व‚स्तुन्येक‚स्य\edtext{}{\lemma{स्य}\Bfootnote{एक‚व‚स्तु० \cite{dp-msA} \cite{dp-msB} \cite{dp-edP} \cite{dp-edH} \cite{dp-edE} \cite{dp-edN}}} व‚स्तुनो युग‚प‚द्भावाभाव‚विरोधात्‚{\tiny $_{lb}$}‚ त‚योर‚भाव‚निश्च‚यायोगात्\edtext{}{\lemma{यायोगात्}\Bfootnote{निश्च‚य‚योगात् \cite{dp-msB}}} । ‚{\tiny $_{lb}$}‚ 
	  
	न च प्र‚तिवाद्य‚नुरोधात् सात्म‚कानात्म‚के व‚स्तुनी स‚द‚स‚ती । किन्तु प्र‚माणानुरोधाद् ।‚{\tiny $_{lb}$}‚ इत्युभे स‚न्दिग्धे । त‚त‚स्त‚योः प्राणादिम‚त्त्व‚स्य स‚द‚स‚त्त्व‚संश‚यः ।\edtext{\textsuperscript{*}}{\lemma{*}\Bfootnote{स‚द‚स‚त्त्वानिश्च‚यः \cite{dp-msC}}} ‚{\tiny $_{lb}$}‚ 
	  
	य‚त एव क्व‚चिद‚न्व‚य-व्य‚तिरेक‚योर्न भाव‚निश्च‚यो \edtext{}{\lemma{यो}\Bfootnote{नाभाव‚नि० \cite{dp-msC}}}नाप्य‚भाव‚निश्च‚यः, त‚त एवान्व‚य‚{\tiny $_{lb}$}‚व्य‚तिरेक‚योः स‚न्देहः । ‚{\tiny $_{lb}$}‚ 
	  
	य‚दि तु क्व‚चिद\edtext{}{\lemma{चिद}\Bfootnote{क्व‚चिद‚न्व‚य० \cite{dp-msC} \cite{dp-msD}}} प्य‚न्व‚य-व्य‚तिरेक‚योरेक‚स्याप्य‚भाव‚निश्च‚यः स्यात्, स एव द्वितीय‚स्य‚{\tiny $_{lb}$}‚ भाव‚निश्च‚य इत्य‚न्व‚य‚व्य‚तिरेक‚स‚न्देह एव न स्यात् । य‚त‚श्च\edtext{}{\lemma{श्च}\Bfootnote{य‚त‚स्तु \cite{dp-msC} \cite{dp-msD}}} न क्व‚चिद्भावाभाव‚निश्च‚य‚स्त‚त‚{\tiny $_{lb}$}‚ एवान्व‚य‚व्य‚तिरेक‚योः स‚न्देहः । स‚न्देहाच्चानैकान्तिकः\edtext{}{\lemma{न्देहाच्चानैकान्तिकः}\Bfootnote{०कान्तिक इत्याह । \cite{dp-edE}}} ॥‚{\tiny $_{lb}$}‚ प‚रिच्छिद्य‚त इत्याह--\textbf{त‚था चे}ति नीरूप‚ताव्य‚व‚च्छेदेन रूप‚स्य प्र‚तिष्ठिताकार‚व‚तः प‚रिच्छेद‚प्र‚कारे‚{\tiny $_{lb}$}‚ स‚ति । \textbf{त‚तो}ऽन्योन्याभाव‚रूप‚त्वाद‚न‚योः ।
	\pend% ending standard par
      ‚{\tiny $_{lb}$}‚

	  \pstart \leavevmode% starting standard par
	न‚नु \textbf{बौद्धानां} सात्म‚कं नाम नास्त्येवेत्य‚व‚स्तु । स‚न्मात्रं तु निरात्म‚क‚म‚तो व‚स्तु । त‚त्र‚{\tiny $_{lb}$}‚ व‚स्तु\add{नि}निरात्म‚को\edtext{}{\lemma{को}\Bfootnote{के}} हेतोर‚न्व‚य‚व्य‚तिरेक‚योर‚भाव‚निश्च‚यो मा भूत्सात्म‚के त्व‚व‚स्तुनि स क‚थं‚{\tiny $_{lb}$}‚ न स्यादित्याश‚ङ्क्योप‚संहार‚व्याजेनाह \textbf{त‚स्मादिति} । य‚स्माद् विधिप्र‚तिषेध‚योरेक‚प्र‚तिषेधोऽप‚र‚{\tiny $_{lb}$}‚विधिनान्त‚रीय‚क‚स्त‚स्मात् । क‚थं न भाव‚निश्च‚य‚स्त‚योरित्याश‚ङ‚क्योप‚प‚त्तिमाह--एक‚त्रेति ।‚{\tiny $_{lb}$}‚ त‚योर‚न्व‚य‚व्य‚तिरेक‚योर्भावाभावात्म‚नो\textbf{र‚भाव‚निश्च‚य}स्या\textbf{योगा}द‚नुप‚प‚त्तिः । क‚थ‚म‚योग इत्याह—‚{\tiny $_{lb}$}‚\textbf{एक‚स्ये}ति । काल‚भेदे किं न युज्य‚ते इत्याह--\textbf{युग‚प‚दि}ति ।
	\pend% ending standard par
      ‚{\tiny $_{lb}$}‚

	  \pstart \leavevmode% starting standard par
	इदं च प्र‚तिवाद्य‚भ्युप‚ग‚म‚ब‚लात्सात्म‚कानात्म‚क‚योः । स‚द‚स‚त्त्व‚म‚भ्युप‚ग‚म्योक्त‚म् । त‚देव तु‚{\tiny $_{lb}$}‚ न युज्य‚त इति द‚र्श‚य‚न्नाह--\textbf{न चे}ति । \textbf{चो} व‚क्त‚व्यान्त‚र‚स‚मुच्च‚ये । प्र‚क‚र‚णादिह \textbf{प्र‚तिवादी‚{\tiny $_{lb}$}‚ बौ}द्ध‚स्त‚द‚नुरोध‚व‚शादिष्ट्य‚निष्टिव‚शा\textbf{त्सात्म‚क‚म‚स‚न्निरात्म‚कं स‚दि}ति य‚थायोगं योज‚नीय‚म् ।‚{\tiny $_{lb}$}‚ एवं ह्य‚वास्त‚व‚म‚नुमानं स्यात् न व‚स्तुब‚ल‚प्र‚वृत्त‚मिति भावः । य‚द्येवं ते स‚द‚स‚ती न भ‚व‚तः क‚थं‚{\tiny $_{lb}$}‚ नामेत्याह--\textbf{किन्त्विति} । प्र‚माणं चेदं निय‚तं व‚र्त्त‚ते । आत्म‚न्येव च विवाद‚वृत्तेर‚न्याऽपि‚{\tiny $_{lb}$}‚ \edtext{\textsuperscript{*}}{\lemma{*}\Bfootnote{न्य‚त्रापि ?}} सात्म‚क‚त्व‚म‚नात्म‚क‚त्वेन \edtext{}{\lemma{त्वेन}\Bfootnote{क‚त्वं वा न ?}} व्य‚व‚तिष्ठ‚त इति भावः । \textbf{इति}स्त‚स्मात् ।‚{\tiny $_{lb}$}‚ उभे सात्म‚क‚त्वानात्म‚क‚त्वे । य‚त एवं \textbf{त‚तः} कार‚णात्सात्म‚कानात्म‚क‚योः \textbf{स‚द‚स‚त्त्व‚योः संश‚यः} ।‚{\tiny $_{lb}$}‚ क‚स्येत्याकाङ्क्षायामुक्त‚म्--\textbf{प्राणादिम‚त्त्व‚स्येति ।}
	\pend% ending standard par
      ‚{\tiny $_{lb}$}‚‚{\tiny $_{lb}$}‚\textsuperscript{\textenglish{222/dm}}‚{\tiny $_{lb}$}‚
	  \bigskip
	  \begingroup
	

	  \pstart \leavevmode% starting standard par
	क‚स्माद‚नैकान्तिक इत्याह\edtext{}{\lemma{इत्याह}\Bfootnote{इत्याह नास्ति \cite{dp-msA} \cite{dp-msB} \cite{dp-msC} \cite{dp-msD} \cite{dp-edP} \cite{dp-edH} \cite{dp-edN}}}--
	\pend% ending standard par
       ‚{\tiny $_{lb}$}‚ 
	  \bigskip
	  \begingroup
	

	  \pstart \leavevmode% starting standard par
	साध्येत‚र‚योर‚तो निश्च‚याभावात् ॥१०८॥
	\pend% ending standard par
      
	  \endgroup
	‚{\tiny $_{lb}$}‚ 

	  \pstart \leavevmode% starting standard par
	साध्य‚स्य, इत‚र‚स्य च विरुद्ध‚स्य अतः--स‚न्दिग्धान्व‚य‚य‚तिरेकान्निश्च‚याभावात् ।‚{\tiny $_{lb}$}‚ स‚प‚क्ष‚विप‚क्ष‚योर्हि स‚द‚स‚त्त्व‚स‚न्देहे न साध्य‚स्य न विरुद्ध‚स्य सिद्धिः\edtext{}{\lemma{सिद्धिः}\Bfootnote{असिद्धिः \cite{dp-msB}}} । न च सात्म‚कानात्म‚काभ्यां\edtext{}{\lemma{काभ्यां}\Bfootnote{०भ्यां च प‚रः \cite{dp-msA} \cite{dp-msB} \cite{dp-edP} \cite{dp-edH} \cite{dp-edN}}}‚{\tiny $_{lb}$}‚ प‚रः प्र‚कारः संभ‚व‚ति । त‚तः प्राणादिम‚त्त्वाद् ध‚र्मिणि जीव‚च्छ‚रीरे संश‚य आत्म‚भावा‚{\tiny $_{lb}$}‚भाव‚योरित्य‚नैकान्तिकः प्राणादिरिति ॥
	\pend% ending standard par
      
	  \endgroup
	‚{\tiny $_{lb}$}‚

	  \pstart \leavevmode% starting standard par
	अस्तु स‚द‚स‚त्त्व‚संश‚योऽन्व‚य‚व्य‚तिरेक‚निश्च‚य‚स्तु किन्न भ‚व‚तीत्याह--\textbf{य‚त} इति । न \textbf{भाव‚निश्च‚यो‚{\tiny $_{lb}$}‚ नाभाव‚निश्च‚य} इत्येक‚त्र सात्म‚केऽनात्म‚के वेति द्र‚ष्ट‚व्य‚म् । सात्म‚केऽनात्म‚के वा प्राणादेः‚{\tiny $_{lb}$}‚ स‚द‚स‚त्त्व‚निश्च‚याभावादेवान्व‚य‚व्य‚तिरेक‚योः स‚न्देहो नान्य‚थेति प्र‚तिपाद‚यितुमाह--\textbf{य‚दि} त्विति ।‚{\tiny $_{lb}$}‚ तुरिमाम‚व‚स्थां भेद‚व‚तीमाह ।
	\pend% ending standard par
      ‚{\tiny $_{lb}$}‚

	  \pstart \leavevmode% starting standard par
	उक्त‚मेवोप‚संह‚र‚न्नाह--य‚त‚श्चेति । \textbf{चो}ऽव‚धार‚णे । त‚स्मात् \textbf{स‚न्देहात् अनैकान्तिकः}‚{\tiny $_{lb}$}‚ प्राणादिम‚त्त्वाख्यो हेतुरिति प्र‚क‚र‚णात् ॥
	\pend% ending standard par
      ‚{\tiny $_{lb}$}‚

	  \pstart \leavevmode% starting standard par
	\textbf{स‚न्दिग्धाव‚न्व‚य‚व्य‚तिरेकौ} य‚स्य त‚त्त‚था, त‚स्मात्साध्य‚स्य विरुद्ध‚स्य वा \textbf{निश्च‚याभावात् ।‚{\tiny $_{lb}$}‚ स‚प‚क्षे}त्यादि\leavevmode\ledsidenote{\textenglish{74b/ms}}नैत‚देव स‚म‚र्थ‚य‚ते । हिर्य‚स्मात् । \textbf{स‚प‚क्ष‚विप‚क्ष‚यो}र्विष‚य‚भूत‚योर्हेतोः \textbf{स‚द‚स‚त्त्व‚{\tiny $_{lb}$}‚स‚न्देहे न साध्य‚स्या}नुमित्सित‚स्य \textbf{विरुद्ध‚स्य} विप‚र्य‚य‚स्य \textbf{सिद्धि}र्निश्च‚यः । विरुद्धोऽपि विप‚र्य‚ये‚{\tiny $_{lb}$}‚ स‚म्य‚ग्धेतुरित्य‚भिप्रायेणेद‚मुक्त‚म् ।
	\pend% ending standard par
      ‚{\tiny $_{lb}$}‚

	  \pstart \leavevmode% starting standard par
	\textbf{न चे}त्यादि \textbf{प्राणादिरि}त्य‚न्तं सुग‚म‚म् ।
	\pend% ending standard par
      ‚{\tiny $_{lb}$}‚

	  \pstart \leavevmode% starting standard par
	ईदृश एव चासाधार‚णो हेतुः कैश्चि\textbf{न्नैयायिकै}र‚नुप‚संहार्य इत्युक्त‚म् । त‚तोऽनुप‚संहार्योऽयं‚{\tiny $_{lb}$}‚ हेत्वाभास इति श‚ब्द‚श्र‚व‚णार्थ...साहः क‚र‚णीयः ।
	\pend% ending standard par
      ‚{\tiny $_{lb}$}‚

	  \pstart \leavevmode% starting standard par
	\textbf{उद्द्योय‚क‚र}स्तु श्राव‚ण‚त्वाख्येऽसाधार‚ण‚हेतावाचार्य\textbf{दिग्नागेन} द‚र्शित इद‚म‚वादीत्—‚{\tiny $_{lb}$}‚य‚द्येत‚च्छ्राव‚ण‚त्वं नित्यानित्य‚योर्दृष्टं स्याज्ज‚न‚येत्त‚योः संश‚य‚मूर्ध्व‚त्व‚मिव स्थाणुपुरुष‚योः । न‚{\tiny $_{lb}$}‚ च दृष्ट‚म् । त‚स्मान्नायं संश‚य‚हेतुर‚पि त्व‚प्र‚तिप‚त्तिहेतुरेव । अथ श्राव‚ण‚त्वं व‚स्तुध‚र्मः । व‚स्तुना‚{\tiny $_{lb}$}‚ नित्येन भाव्य‚म‚नित्येन वा प्र‚कारान्त‚राभावात् । न च त‚योरेक‚त्रापि दृष्ट‚म् । अत‚स्त‚योः‚{\tiny $_{lb}$}‚ संश‚यं क‚रोति । त‚र्हि व‚स्तुध‚र्म‚त्वात् संश‚यो न श्राव‚ण‚त्वादिति । तुल्य‚न्याय‚त‚याऽत्राप्य‚{\tiny $_{lb}$}‚साधार‚णे त‚दीय‚मिद‚मीदृशं च‚र्चित‚मास‚ज्य‚त एवेति क‚थ‚म‚यं संश‚य‚हेतुरुप‚पाद‚यित‚व्यः इति‚{\tiny $_{lb}$}‚ साधूक्तं तेन । केव‚लं ग‚म‚क‚रूप‚विवेच‚ने स‚मीचीन‚म‚नो न प्र‚हित‚म् । य‚तो व‚स्तुध‚र्म‚त्वं श्राव‚ण‚{\tiny $_{lb}$}‚त्व‚स्य नित्याकार‚संस्प‚र्शिज्ञान‚ज‚न‚ने निब‚न्ध‚न‚म् । न तु त‚स्मादेव व‚स्तुध‚र्मादुभ‚याकार‚संस्प‚र्शी‚{\tiny $_{lb}$}‚ प्र‚त्य‚यो दोलाय‚ते न च य‚द्य‚स्य प्र‚तिप‚त्तिकार‚णे कार‚ण‚म् त‚त एव सा प्र‚तिप‚त्तिः, न तु त‚स्मादिति‚{\tiny $_{lb}$}‚ श‚क्य‚ते व‚क्तुम् । त‚दुत्प‚त्तेर‚ग्निप्र‚तिप‚त्तिर्न तु धूमादित्य‚स्याभिधान‚प्र‚स‚ङ्गात् ।
	\pend% ending standard par
      ‚{\tiny $_{lb}$}‚‚{\tiny $_{lb}$}‚\textsuperscript{\textenglish{223/dm}}‚{\tiny $_{lb}$}‚

	  \pstart \leavevmode% starting standard par
	किञ्चैवं प्र‚मेय‚त्वादेर‚पि न संश‚यः स्यात् । श‚क्य‚ते हि त‚त्रापि व‚क्तुमुभ‚य‚त्र द‚र्श‚नात्‚{\tiny $_{lb}$}‚ संश‚यो न प्र‚मेय‚त्वादिति । अथ त‚स्य ताव‚दुभ‚य‚त्र द‚र्श‚नं तेन त‚स्मादुच्य‚ते । य‚द्येवं व‚स्तुध‚र्म‚{\tiny $_{lb}$}‚त्व‚म‚पि श्राव‚ण‚त्व‚स्यैवेति क‚थं न त‚स्माद‚सौ । अपि चोर्ध्व‚त्व‚म‚पि य‚द्य‚पि स्थाणुपुरुष‚योर‚पि‚{\tiny $_{lb}$}‚ दृष्टं त‚थापि ताव‚त्त‚त्रान्त‚रेणात्रैव भ‚विष्य‚तीति प‚र्य‚नुयोगे स‚तीद‚मेव वाच्य‚म्--य‚दुतोर्ध्व‚त्वं नाम‚{\tiny $_{lb}$}‚ व‚स्तुध‚र्मः । व‚स्तुना चैवंविधिः--स्थाणुना पुरुषेण वाऽव‚श्यं भाव्य‚मिति । त‚था च व‚स्तुध‚र्मादेव‚{\tiny $_{lb}$}‚ संश‚यो नोर्ध्व‚त्वादित्य‚निष्टापाद‚नं--केन निराक्रियेतेत्य‚लं विस्त‚रेण ।
	\pend% ending standard par
      ‚{\tiny $_{lb}$}‚

	  \pstart \leavevmode% starting standard par
	साध‚न‚स्य सिद्धेर्य‚न्नाङ्ग‚म‚सिद्धो विरुद्धोऽनैकान्तिको हेत्वाभासः । त‚स्यापि व‚च‚नं वादिनो‚{\tiny $_{lb}$}‚ निग्र‚ह‚स्थान‚म‚स‚म‚र्थोपादानात् । त‚स्मादेवंविधो हेत्वाभासः स्व‚य‚म‚प्र‚योज्यः प‚र‚प्र‚युक्त‚श्चाव‚श्य‚{\tiny $_{lb}$}‚मुद्भाव‚यित‚व्य इति हेत्वाभास‚व्युत्पाद‚ने \textbf{वार्त्तिक‚कार}स्याभिप्रायो बोद्ध‚व्यः ।
	\pend% ending standard par
      ‚{\tiny $_{lb}$}‚

	  \pstart \leavevmode% starting standard par
	स्यादेत‚त्--अस‚म‚र्थंविशेष‚णोऽस‚म‚र्थ‚विशेष्य‚श्चास्ति प्र‚भेदः । य‚थाऽनित्यः प्र‚मेय‚त्वे‚{\tiny $_{lb}$}‚ स‚ति कृत‚क‚त्वात् । अत्र कृत‚क‚त्वं विशेष्य‚मेव साध्य‚सिद्धौ स‚म‚र्थ‚म्, न तु प्र‚मेय‚त्वं विशेष‚ण‚{\tiny $_{lb}$}‚मित्य‚स‚म‚र्थं विशेष‚ण‚म्, य‚त्र विशेष्य‚मेव स‚म‚र्थ‚मिति कृत्वा भ‚व‚त्य‚स‚म‚र्थ‚विशेष‚णो हेतुः ।‚{\tiny $_{lb}$}‚ य‚ञ्च \edtext{}{\lemma{ञ्च}\Bfootnote{च्चा}} नित्यः श‚ब्दः कृत‚क‚त्वे स‚ति प्र‚मेय‚त्वादिति । अत्र हि कृत‚क‚त्वं विशेष‚ण‚मेव‚{\tiny $_{lb}$}‚ साध्य‚सिद्धौ स‚म‚र्थ‚म्, न तु प्र‚मेय‚त्वं विशेष्य‚मित्य‚स‚म‚र्थं विशेष्य‚म् । य‚त्र हि विशेष‚ण‚मे \leavevmode\ledsidenote{\textenglish{75a/ms}} व‚{\tiny $_{lb}$}‚ स‚म‚र्थ‚मिति कृत्वा भ‚व‚त्य‚य‚म‚स‚म‚र्थ‚विशेष्यो हेतुः । शेष‚मुभ‚यीविधास्व‚न्त‚र्भाव्य‚ताम् । न ताव‚द‚{\tiny $_{lb}$}‚सिद्धे, द्व‚योर‚पि ध‚र्मिणि सिद्धेः । न च विरुद्धे, विप‚र्य‚य‚व्याप्त्य‚भावात् । नाप्य‚नैकान्तिके कृत‚क‚{\tiny $_{lb}$}‚त्व‚विशिष्ट‚प्र‚मेय‚त्व‚स्य प्र‚मेय‚त्व‚विशिष्ट‚कृत‚क‚त्व‚स्य च साध्याऽव्य‚भिचारात् । त‚स्माद‚सिद्ध‚त्वा‚{\tiny $_{lb}$}‚देर‚न्य एवायं हेतुदोष‚प्र‚कारः प्राप्त इति ।
	\pend% ending standard par
      ‚{\tiny $_{lb}$}‚

	  \pstart \leavevmode% starting standard par
	त‚देत‚द‚व‚द्य‚म्, हेत्व‚दोषात् । य‚दि ह्येव‚म‚यं प्र‚युक्तो हेतुर्द्विष्येत् \edtext{}{\lemma{हेतुर्द्विष्येत्}\Bfootnote{त}} त‚दाऽस्यामीषु‚{\tiny $_{lb}$}‚ हेतुराशिष्व‚न्त‚र्भाव‚श्चिन्त्येत, अन्यो वा हेतुदोषोऽभ्युप‚ग‚म्येत । याव‚ता नैव‚म‚यं प्र‚युक्तोऽन्य‚थेति‚{\tiny $_{lb}$}‚ साध्य‚साध‚नादिति । न त‚र्ह्येवं वादी निगृह्य‚त इति चेत् । किं न निगृह्य‚ते, असाध‚नाङ्ग‚{\tiny $_{lb}$}‚व‚च‚नात् ? उभ‚य‚त्रापि साध्य‚सिद्ध्य‚न‚ङ्ग‚स्य प्र‚मेय‚त्व‚स्यास‚म‚र्थ‚स्याभिधानात् । य‚था च‚{\tiny $_{lb}$}‚ साध्य‚सिद्ध्य‚न‚ङ्ग‚स्य व‚च‚ने निग्र‚होऽव‚श्य‚म्भावी, अनिग्र‚हे वा दोषः, त‚था \textbf{वाद‚न्यायेऽवादीन्न्या‚{\tiny $_{lb}$}‚वादी}ति त‚त‚स्त‚द‚पेक्षित‚व्यः । त‚तोऽय‚म‚र्थो व‚क्तृदोष एव न हेतुदोषः । तेनान‚न्त‚र्भावेऽपि न‚{\tiny $_{lb}$}‚ हेत्वाभासानुष‚ङ्ग इति ।
	\pend% ending standard par
      ‚{\tiny $_{lb}$}‚

	  \pstart \leavevmode% starting standard par
	भ‚व‚तु ताव‚द‚त्रेयं ग‚तिः । सिद्ध‚साध‚ने तु साध‚ने किं भ‚विष्य‚ति ? न ताव‚त् सिद्ध‚साध‚नं‚{\tiny $_{lb}$}‚ साध‚न\add{म}सिद्ध‚त्वाद्य‚न्य‚त‚म‚दोष‚दूषितं साध्य‚साध‚न‚साम‚र्थ्याप्र‚च्युतेरिति ।
	\pend% ending standard par
      ‚{\tiny $_{lb}$}‚

	  \pstart \leavevmode% starting standard par
	अत्रोच्य‚ते--इह हेतुर्द्वेधा दुष्य‚ति । क‚श्चिद‚साम‚र्थ्यात्, अप‚रो वैय‚र्थ्यात् । त‚त्रासाम‚र्थ्य‚{\tiny $_{lb}$}‚ एव दोषो \textbf{वार्त्तिक‚कारेणाऽ}न‚न्त‚रोक्तेन क्र‚मेण त्रिधा द‚र्शितः । न तु वैय‚र्थ्य‚ल‚क्ष‚णः । सिद्ध‚{\tiny $_{lb}$}‚साध‚नं तु वैय‚र्थ्य‚ल‚क्ष‚णोऽन्य एवायं हेतोः स्व‚ग‚तो दोष इति क‚स्माद‚स्यान्त‚र्भाव‚श्चिन्त‚नीयः ?‚{\tiny $_{lb}$}‚ य‚दाहाचार्यः--अन्य‚थानिष्ठं\edtext{}{\lemma{थानिष्ठं}\Bfootnote{ष्टं}}भ‚वेद् विफ‚ल‚मेव वा । त‚था न साध्य‚त्वे वैक‚ल्याद् ‚{\tiny $_{lb}$}‚ इत्यादीति ।
	\pend% ending standard par
      ‚{\tiny $_{lb}$}‚

	  \pstart \leavevmode% starting standard par
	व‚क्तृदोष एवैष इत्य‚पि वार्त्ता, य‚थायोगं प‚रिपूर्ण‚साध‚न‚रूपाभिधानाद‚नुप‚युक्तान‚भि‚{\tiny $_{lb}$}‚धानाच्च व‚क्तुर‚दुष्ट‚त्वात् । व‚क्ताऽयं हेतुनिश्चितो \edtext{}{\lemma{हेतुनिश्चितो}\Bfootnote{ता}}ऽर्य‚प्र‚युक्तो वैय‚र्थ्य‚म‚नुभ‚व‚ति । न तु \leavevmode\ledsidenote{\textenglish{224/dm}}‚{\tiny $_{lb}$}‚ 
	  
	त्र‚याणां रूपाणाम‚सिद्धौ\edtext{}{\lemma{सिद्धौ}\Bfootnote{०म‚सिद्धिसंदेहे हेतु० \cite{dp-msC}}} स‚न्देहे च\edtext{}{\lemma{च}\Bfootnote{वा \cite{dp-msD}}} हेतुदोषानुप‚पाद्योप‚संह‚र‚न्नाह-- ‚{\tiny $_{lb}$}‚ 
	  
	\edtext{\textsuperscript{*}}{\lemma{*}\Bfootnote{एवं त्र० \cite{dp-msB} \cite{dp-msD} \cite{dp-edP} \cite{dp-edH} \cite{dp-edE} \cite{dp-edN}}}एव‚मेषां त्र‚याणां रूपाणामेकैक‚स्य र्द्व‚योर्द्व‚योर्वा रूप‚योर‚सिद्धौ संदेहे‚{\tiny $_{lb}$}‚ वा\edtext{}{\lemma{वा}\Bfootnote{च \cite{dp-msB} \cite{dp-edP} \cite{dp-edH} \cite{dp-edE} \cite{dp-edN}}} य‚थायोग‚म‚सिद्ध‚विरुद्धानैकान्तिकास्र‚यो हेत्वाभासाः ॥ १०९ ॥‚{\tiny $_{lb}$}‚ 
	  
	एव‚मित्य‚न‚न्त‚रोक्तेन क्र‚मेण । एषां म‚ध्य एकैकं रूपं \edtext{}{\lemma{रूपं}\Bfootnote{य‚द‚सिद्धं \cite{dp-msA} \cite{dp-msB} \cite{dp-edP} \cite{dp-edH} \cite{dp-edN}}}य‚दाऽसिद्धं स‚न्दिग्धं वा\edtext{}{\lemma{वा}\Bfootnote{वा नास्ति \cite{dp-msB}}} भ‚व‚ति,‚{\tiny $_{lb}$}‚ द्वे द्वे वाऽसिद्धे संदिग्धे वा\edtext{}{\lemma{वा}\Bfootnote{वा नास्ति \cite{dp-msB}}} भ‚व‚तः, त‚दासिद्ध‚श्च विरुद्ध‚श्चानैकान्तिक‚श्च ते हेत्वाभासाः ।‚{\tiny $_{lb}$}‚ य‚थायोग‚मिति । य‚स्यासिद्धौ संदेहे वा यो हेत्वाभासो युज्य‚ते स त‚स्याऽसिद्धेः संदेहाच्च‚{\tiny $_{lb}$}‚ व्य‚व‚स्थाप्य‚त इति य‚स्य य‚स्य \edtext{}{\lemma{स्य}\Bfootnote{येन नास्ति \cite{dp-msB}}}येन येन योगो य‚थायोग‚मिति ॥ ‚{\tiny $_{lb}$}‚ 
	  
	विरुद्धाव्य‚भिचार्य‚पि संश‚य‚हेतुरुक्तः । स इह क‚स्मान्नोक्तः ? ॥ ११० ॥‚{\tiny $_{lb}$}‚ 
	  
	न‚नु चाऽऽचार्येण विरुद्धाव्य‚भिचार्य‚पि संश‚य‚हेतुरुक्तः । हेत्व‚न्त‚र‚साधित‚स्य \edtext{}{\lemma{स्य}\Bfootnote{०स्य य‚द्विरुद्धं त‚न्न \cite{dp-msB} \cite{dp-msD}}}विरुद्धं‚{\tiny $_{lb}$}‚ य‚त् त‚न्न व्य‚भिच‚र‚तीति\edtext{}{\lemma{तीति}\Bfootnote{०र‚ति स विरु० \cite{dp-msA} \cite{dp-msB} \cite{dp-edP} \cite{dp-edH} \cite{dp-edE} \cite{dp-edN}}} विरुद्धाव्य‚भिचारी । य‚दि वा विरुद्ध‚श्चासौ साध‚नान्त‚र‚सिद्ध‚स्य‚{\tiny $_{lb}$}‚ ध‚र्म‚स्य विरुद्ध‚साध‚नात्, अव्य‚भिचारी च स्व‚साध्याव्य‚भिचाराद्विरुद्धाव्य‚भिचारी ॥‚{\tiny $_{lb}$}‚ स्व‚तोऽ\edtext{}{\lemma{तोऽ}\Bfootnote{तो}}दुष्ट‚स्त‚तो व‚क्तृदोषो युज्य‚त एवेति चेत् । त‚र्हि विरुद्ध‚त्व‚म‚पि व‚क्तृदोषोऽस्तु‚{\tiny $_{lb}$}‚ न्याय‚स्य स‚मान‚त्वात् । श‚क्य‚ते हि त‚त्राप्येव‚म‚भिधातुम्--व‚क्ताऽय‚म‚न‚नुरूपे साध्ये प्र‚युक्ते विप‚र्य‚य‚{\tiny $_{lb}$}‚साधानाद् विरुद्ध‚ताम‚नुभ‚व‚ति न त्व‚यं स्व‚तो दुष्टो नामेति । विव‚क्षितार्थ‚साध‚नासाम‚र्थ्यं‚{\tiny $_{lb}$}‚ ताव‚द‚स्य स्व‚तोऽस्ति तेना\edtext{}{\lemma{तेना}\Bfootnote{न}}हेतुदोष एवाय‚मिति चेत् । इहापि निश्चितार्थ‚निश्च‚य‚नं ताव‚द‚स्य‚{\tiny $_{lb}$}‚ स्व‚तोऽस्तीति क‚थं न वैय‚र्थ्यं त‚स्य दोष इति चिन्त्य‚तामिति ।
	\pend% ending standard par
      ‚{\tiny $_{lb}$}‚

	  \pstart \leavevmode% starting standard par
	निग्र‚ह‚स्त्वेवंवादिनोऽसाध‚नाङ्ग‚व‚च‚नाद् बोद्ध‚व्यः । सिद्धि\add{ः}साध‚नं त‚द‚ङ्गंम् ध‚र्मो य‚स्या‚{\tiny $_{lb}$}‚र्थ‚स्य विवादाश्र‚य‚स्य वाद‚प्र‚स्तावाद् हेतोः । स साध‚नाङ्गः । त‚था यो न भ‚व‚ति त‚स्या‚{\tiny $_{lb}$}‚प्र‚स्तुत‚स्याभिधानादिति कृत्वेति स‚र्व‚मेवाव‚दात‚म् ।
	\pend% ending standard par
      ‚{\tiny $_{lb}$}‚

	  \pstart \leavevmode% starting standard par
	केचित्पुन‚रेव‚म‚सिद्धेऽन्त‚र्भाव‚यितुं प्र‚य‚त‚न्ते । अन्ये तु विरुद्धे । य‚था च तेषां प्र‚य‚तिर्य‚था‚{\tiny $_{lb}$}‚ त‚द‚भ‚ज‚मान‚म‚भिधानं त‚था \textbf{स्व‚यूथ्य‚विचार} एवाभिहित इति त‚त एवापेक्षित‚व्य इति ॥
	\pend% ending standard par
      ‚{\tiny $_{lb}$}‚

	  \pstart \leavevmode% starting standard par
	\textbf{त्र‚याणामि}त्यादि \textbf{व्य‚व‚स्थाप्य‚त} इत्येत‚द‚न्तं सुबोध‚म् ।
	\pend% ending standard par
      ‚{\tiny $_{lb}$}‚

	  \pstart \leavevmode% starting standard par
	\textbf{य‚स्ये}ति हेत्वाभास‚स्य । स‚र्व‚हेत्वाभास‚स‚ङ्ग्र‚ह‚णार्थं \textbf{य‚स्ये}ति द्विरुक्तं \textbf{येन} \leavevmode\ledsidenote{\textenglish{75b/ms}} \textbf{येन}‚{\tiny $_{lb}$}‚ दोषेण \textbf{योगः} स‚म्ब‚न्धः । एत‚च्चार्थ‚क‚थ‚न‚म् । योगान‚तिक्र‚मेणेति विग्र‚हः कार्यः ॥
	\pend% ending standard par
      ‚{\tiny $_{lb}$}‚

	  \pstart \leavevmode% starting standard par
	\textbf{न‚नु चे}त्यादीत्याहेत्येत‚द‚न्तं सुग‚म‚म् ।
	\pend% ending standard par
      ‚{\tiny $_{lb}$}‚‚{\tiny $_{lb}$}‚\textsuperscript{\textenglish{225/dm}}‚{\tiny $_{lb}$}‚
	  \bigskip
	  \begingroup
	

	  \pstart \leavevmode% starting standard par
	स‚त्य‚म् । उक्त आचार्येण । म‚या त्विह नोक्तः । क‚स्मादित्याह--
	\pend% ending standard par
       ‚{\tiny $_{lb}$}‚ 
	  \bigskip
	  \begingroup
	

	  \pstart \leavevmode% starting standard par
	अनुमान‚विष‚येऽ\edtext{}{\lemma{येऽ}\Bfootnote{०विष‚ये त‚स्यासं० \cite{dp-msC}}}स‚म्भ‚वात् ॥ १११ ॥
	\pend% ending standard par
      
	  \endgroup
	‚{\tiny $_{lb}$}‚ 

	  \pstart \leavevmode% starting standard par
	अनुमान‚स्य विष‚यः प्र‚माण‚सिद्धं त्रैरूप्य‚म् । य‚तो हि अनुमान‚स‚म्भ‚वः\edtext{}{\lemma{वः}\Bfootnote{अनुमान‚स‚द्भावः \cite{dp-msA} \cite{dp-msB} \cite{dp-msC} \cite{dp-edP} \cite{dp-edH} \cite{dp-edN} अनुमान‚स्य स‚म्भ‚वः \cite{dp-edE}}} सोऽनुमान‚स्य‚{\tiny $_{lb}$}‚ विष‚यः । प्र‚माण‚सिद्धाच्च त्रैरूप्याद‚नुमान‚स‚म्भ‚वः\edtext{}{\lemma{वः}\Bfootnote{अनुमान‚स‚द्भावः \cite{dp-msA} \cite{dp-msB} \cite{dp-edP} \cite{dp-edH} \cite{dp-edN}}} । त‚स्मात् त‚देवानुमान‚विष‚यः । त‚स्मिन्‚{\tiny $_{lb}$}‚ प्र‚क्रान्ते न विरुद्धाव्य‚भिचारिस‚म्भ‚वः । प्रामाण‚सिद्धे हि त्रैरूप्ये प्र‚स्तुते स एव हेत्वाभासः‚{\tiny $_{lb}$}‚ स‚म्भ‚ब‚ति य‚स्य प्र‚माण‚सिद्धं रूप‚म् । न च विरुद्धाव्य‚भिचारिणः प्र‚माण‚सिद्ध‚म‚स्ति रूप‚म् ।‚{\tiny $_{lb}$}‚ अतो न स‚म्भ‚वः । त‚तोऽस‚म्भ‚वात्\edtext{}{\lemma{वात्}\Bfootnote{त‚तोऽस‚म्भ‚वो नोक्तः \cite{dp-msA} \cite{dp-msB} \cite{dp-edP} \cite{dp-edH}}} नोक्तः ॥
	\pend% ending standard par
       ‚{\tiny $_{lb}$}‚ 

	  \pstart \leavevmode% starting standard par
	क‚स्माद‚संभ‚व इत्याह--
	\pend% ending standard par
       ‚{\tiny $_{lb}$}‚ 
	  \bigskip
	  \begingroup
	

	  \pstart \leavevmode% starting standard par
	न हि स‚म्भ‚वोऽस्ति कार्य‚स्व‚भाव‚योरुक्त‚ल‚क्ष‚ण‚योर‚नुप‚ल‚म्भ‚स्य च\edtext{}{\lemma{च}\Bfootnote{वा \cite{dp-msD}}}‚{\tiny $_{lb}$}‚ विरुद्ध‚तायाः ॥११२॥
	\pend% ending standard par
      
	  \endgroup
	‚{\tiny $_{lb}$}‚ 

	  \pstart \leavevmode% starting standard par
	न हीति । य‚स्मान्न स‚म्भ‚वोऽस्ति विरुद्ध‚तायाः । कार्य च स्व‚भाव‚श्च त‚यो‚{\tiny $_{lb}$}‚रुक्त‚ल‚क्ष‚ण‚योरिति ।
	\pend% ending standard par
      
	  \endgroup
	‚{\tiny $_{lb}$}‚

	  \pstart \leavevmode% starting standard par
	\textbf{अनुमान‚स्य} त्रैरूप्याल्लिङ्ग‚स‚म्ब‚न्धिनोऽन्य‚स्मात् \textbf{त्रैरूप्यं विष‚यः} । य‚था म‚त्स्यानां विष‚यो‚{\tiny $_{lb}$}‚ ज‚ल‚मिति । एत‚देवाह--\textbf{य‚त} इति । हिर्य‚स्माद‚र्थे । \textbf{प्र‚क्रान्ते} प्र‚स्तुते अनुमान‚विष‚यो\edtext{}{\lemma{यो}\Bfootnote{ये}}‚{\tiny $_{lb}$}‚ वै\edtext{}{\lemma{वै}\Bfootnote{त्रै}}रूप्ये स‚ति ।
	\pend% ending standard par
      ‚{\tiny $_{lb}$}‚

	  \pstart \leavevmode% starting standard par
	क‚स्मात्त‚त्र विरुद्धाव्य‚भिचारिणोऽस‚म्भ‚व इत्याह--\textbf{प्र‚माणे}ति । हीति य‚स्मात् । स‚{\tiny $_{lb}$}‚ एव \textbf{हेत्वाभासो} विरुद्धाव्य‚भिचार्याख्यः \textbf{स‚म्भ‚व‚ति य‚स्य प्र‚माणेन सिद्धं रूपं} प‚क्ष‚ध‚र्मान्व‚य‚व्य‚ति‚{\tiny $_{lb}$}‚रेकात्म‚क‚मिति विव‚क्षित‚म् ।
	\pend% ending standard par
      ‚{\tiny $_{lb}$}‚

	  \pstart \leavevmode% starting standard par
	अय‚माश‚यः--य‚दि त‚देकेन द्वाभ्यां वा रूपाभ्यां हीनं स्यात्त‚देष्वेव हेत्वाभासेष्व‚न्त‚र्भ‚वेत् ।‚{\tiny $_{lb}$}‚ न त्वेत‚द‚तिरिक्तो विरुद्धाव्य‚भिचारी नाम हेत्वाभासो भ‚वेत् । भ‚व‚ता त्व‚नेन प्र‚माण‚सिद्ध‚{\tiny $_{lb}$}‚त्रैरूप्येणैव भाव्य‚मिति ।
	\pend% ending standard par
      ‚{\tiny $_{lb}$}‚

	  \pstart \leavevmode% starting standard par
	अस्तु त‚स्य त‚थात्व‚मित्याह--\textbf{न चे}ति । \textbf{चो}ऽव‚धार‚णे, व्य‚क्त‚मेत‚दित्य‚स्मिन्न‚र्थे वा ।
	\pend% ending standard par
      ‚{\tiny $_{lb}$}‚

	  \pstart \leavevmode% starting standard par
	एवं ब्रुव‚त‚श्चास्याय‚म‚भिप्रायः--व‚स्तुनः प‚र‚स्प‚र‚विरुद्ध‚रूप‚द्व‚यास‚म्भ‚वाद् अव‚श्य‚म‚न‚योरेक‚{\tiny $_{lb}$}‚म‚स‚म्पूर्णाङ्ग‚मिति ॥
	\pend% ending standard par
      ‚{\tiny $_{lb}$}‚

	  \pstart \leavevmode% starting standard par
	कुतः पुन‚र‚नुमान‚विष‚येऽस्यास‚म्भ‚वोऽव‚सीय‚त इत्य‚भिप्रेत्य पृच्छ‚ति प‚रः--\textbf{क‚स्मादि}ति‚{\tiny $_{lb}$}‚ ‚{\tiny $_{lb}$}‚ \leavevmode\ledsidenote{\textenglish{226/dm}}‚{\tiny $_{lb}$}‚ 
	  
	कार्य‚स्य कार‚णाज्ज‚न्म‚ल‚क्ष‚णं त‚त्त्व‚म् । स्व‚माव‚स्य च साध्य‚व्याप्त‚त्वं त‚त्त्व‚म् । य‚त्‚{\tiny $_{lb}$}‚ कार्य‚म्, य‚श्च स्व‚भावः, स क‚थ‚मात्म‚वार‚णं व्याप‚कं च स्व‚भावं प‚रित्य‚ज्य भ‚वेद् येन विरुद्धः‚{\tiny $_{lb}$}‚ स्यात् । अनुप‚ल‚म्भ‚स्य च उक्त‚ल‚क्ष‚ण‚स्येति । दृश्यानुप‚ल‚म्भ‚त्वं \edtext{}{\lemma{त्वं}\Bfootnote{०त्व‚म‚नुप० \cite{dp-msA} \cite{dp-msB} \cite{dp-msC} \cite{dp-edP} \cite{dp-edH} \cite{dp-edE} \cite{dp-edN}}}चानुप‚ल‚म्भ‚ल‚क्ष‚ण‚म् ।‚{\tiny $_{lb}$}‚ त‚स्यापि\edtext{}{\lemma{स्यापि}\Bfootnote{त‚स्यापि च स्व‚भावा० \cite{dp-msA} \cite{dp-msB} \cite{dp-edP} \cite{dp-edH}}} व‚स्त्व‚भावाव्य‚भिचारित्वान्न विरुद्ध‚त्व‚स‚म्भ‚वः\edtext{}{\lemma{वः}\Bfootnote{०स‚म्भ‚वः स्यात् । एते० \cite{dp-msA} \cite{dp-edP} \cite{dp-edH} ०स‚म्भ‚वः स्यादेति [[?]] त‚त् एते० \cite{dp-msB}}} ॥ ‚{\tiny $_{lb}$}‚ 
	  
	स्यादेत‚त्--एतेभ्योऽन्यो भ‚विष्य‚तीत्याह-- ‚{\tiny $_{lb}$}‚ 
	  
	न चान्योऽव्य‚भिचारी ॥ ११३ ॥‚{\tiny $_{lb}$}‚ 
	  
	न चान्य एतेभ्योऽव्य‚भिचारी त्रिभ्यः । अत \edtext{}{\lemma{अत}\Bfootnote{अत एव तेष्वेव \cite{dp-msA} \cite{dp-msB} \cite{dp-edP} \cite{dp-edH} \cite{dp-edE} \cite{dp-edN} अत एवैतेष्वेव \cite{dp-msC}}}एवैष्वेव हेतुत्व‚म् ॥ ‚{\tiny $_{lb}$}‚ 
	  
	क्व त‚र्ह्याचार्य‚दिग्नागेनायं हेतुदोप उक्त इत्याह-- ‚{\tiny $_{lb}$}‚ 
	  
	त‚स्माद‚व‚स्तुद‚र्श‚न‚ब‚ल‚प्र‚वृत्त‚माग‚माश्र‚य‚म‚नुमान‚माश्रित्य त‚द‚र्थ‚विचारेषु‚{\tiny $_{lb}$}‚ विरुद्वाव्य‚भिचारी साध‚न‚दोष उक्तः ॥ ११४ ॥‚{\tiny $_{lb}$}‚ 
	  
	य‚स्माद् व‚स्तुब‚ल‚प्र‚वृत्तेऽनुमाने न स‚म्भ‚व‚ति त‚स्माद् आग‚माश्र‚य‚म‚नुमान‚माश्रित्य‚{\tiny $_{lb}$}‚व्य‚भिचार्युक्तः । आग‚म‚सिद्धं हि य‚स्यानुमान‚स्य लिङ्ग‚त्रैरूप्यं त‚स्याग‚म आश्र‚यः । ‚{\tiny $_{lb}$}‚ 
	  
	न‚नु चाग‚म‚सिद्ध‚म‚पि त्रैरूप्यं प्र‚माण‚सिद्ध‚मित्याह--अव‚स्तुद‚र्श‚न‚ब‚ल‚प्र‚वृत्त‚मिति ।‚{\tiny $_{lb}$}‚ अव‚स्तुनो द‚र्श‚नं विक‚ल्प‚मात्र‚म् त‚स्य ब‚लं साम‚र्थ्य‚म् । त‚तः प्र‚वृत्त‚म्--अप्र‚माणाद्विक‚ल्प‚मात्राद्‚{\tiny $_{lb}$}‚ व्य‚व‚स्थितं त्रैरूप्य‚माग‚म‚सिद्ध‚म‚नुमान‚स्य । न तु प्र‚माणात् ।‚{\tiny $_{lb}$}‚ \textbf{न‚ही}त्य‚त्र‚स्थ‚स्य हिश‚ब्द‚स्यार्थो \textbf{य‚स्मादि}त्य‚नेनोक्तः । \textbf{विरुद्ध‚तायाः} एक‚साध‚न‚साधित‚स्यार्थ‚स्य‚{\tiny $_{lb}$}‚ प्र‚त्य‚नीक‚प‚क्ष‚साध‚न‚रूप‚तायाः ।
	\pend% ending standard par
      ‚{\tiny $_{lb}$}‚

	  \pstart \leavevmode% starting standard par
	एत‚दुक्तं म‚व‚ति--स्व‚साध्याव्य‚भिचारिणा हि भ‚व‚ता कार्येण स्व‚भावेन वा भाव्य‚म् ।‚{\tiny $_{lb}$}‚ \textbf{न च} व‚स्तुन‚स्त‚द‚त‚त्स्व‚भावौ स्तो येन त‚द‚त‚त्स्व‚भावाव्य‚भिचारिणौ द्वौ हेतू स‚न्निप‚त‚न्तौ विरुद्धा‚{\tiny $_{lb}$}‚\textbf{व्य‚भिचारिणौ} स्यातामिति । येन कार्य‚स्व‚भाव‚योः कार‚ण‚व्याप‚क‚विधिना कृस‚त‚द्भावेन विरुद्धं‚{\tiny $_{lb}$}‚ त‚त्रैव ध‚र्मिणि हेत्व‚न्त‚र‚साधितार्थं विरुद्ध‚साध‚नं भ‚वेत् ।
	\pend% ending standard par
      ‚{\tiny $_{lb}$}‚

	  \pstart \leavevmode% starting standard par
	य‚द्येव‚म‚नुप‚ल‚म्भे त‚त्स‚म्भ‚विष्य‚तीत्याह--\textbf{अनुप‚ल‚म्भ‚स्ये}ति । चः पूर्वापेक्षः स‚मुच्च‚ये ॥
	\pend% ending standard par
      ‚{\tiny $_{lb}$}‚

	  \pstart \leavevmode% starting standard par
	\textbf{य‚त} एत‚द‚तिरिक्तोऽव्य‚भिचारी \add{न} स‚म्म‚त \add{\textbf{अत}} एवास्मादेव कार‚णात् । \textbf{एष्वेव}‚{\tiny $_{lb}$}‚ कार्य‚स्व‚भावानुप‚ल‚म्भेष्वेव ॥
	\pend% ending standard par
      ‚{\tiny $_{lb}$}‚

	  \pstart \leavevmode% starting standard par
	\textbf{क‚थ‚मा}ग‚माश्र‚य‚त्व‚म‚नुमान‚स्येत्याह--\textbf{आग‚म‚सिद्ध‚मि}ति । हिर्य‚स्माद‚र्थे । स‚ति त‚स्मिन्ना‚{\tiny $_{lb}$}‚ग‚मेऽनुमान‚स्य प्र‚वृत्तेर‚सावाश्र‚य‚स्त‚स्य ।
	\pend% ending standard par
      ‚{\tiny $_{lb}$}‚‚{\tiny $_{lb}$}‚\textsuperscript{\textenglish{227/dm}}‚{\tiny $_{lb}$}‚
	  \bigskip
	  \begingroup
	

	  \pstart \leavevmode% starting standard par
	त‚त् त‚र्ह्य‚नुमान\edtext{}{\lemma{नुमान}\Bfootnote{नुमानेनाग‚म० \cite{dp-msA} \cite{dp-msB} \cite{dp-edP} \cite{dp-edH}}} माग‚म‚सिद्ध\edtext{}{\lemma{सिद्ध}\Bfootnote{सिद्धं त्रैरू० \cite{dp-msC}}} त्रैरूप्यं क्वाधिकृत‚मित्याह--\edtext{\textsuperscript{*}}{\lemma{*}\Bfootnote{०त्याह त‚स्याग‚म० \cite{dp-msB}}}त‚द‚र्थेति । त‚स्याग‚म‚स्य‚{\tiny $_{lb}$}‚ योऽर्थोऽतीन्द्रियः प्र‚त्य‚क्षानुमानाभ्याम‚विष‚यीकृतः सामान्यादिस्त‚स्य विचारेषु प्र‚क्रान्तेषु आग‚माश्र‚य‚{\tiny $_{lb}$}‚म‚नुमानं स‚म्भ‚व‚ति । त‚दाश्र‚यो विरुद्धाव्य‚भिचार्युक्त आचार्येणेति ॥
	\pend% ending standard par
       ‚{\tiny $_{lb}$}‚ 

	  \pstart \leavevmode% starting standard par
	क‚स्मात् पुन‚राग‚माश्र‚ये\edtext{}{\lemma{ये}\Bfootnote{०श्र‚योऽप्य \cite{dp-msA}}} प्य‚नुमाने स‚म्भ‚व इत्याह--
	\pend% ending standard par
       ‚{\tiny $_{lb}$}‚ 
	  \bigskip
	  \begingroup
	

	  \pstart \leavevmode% starting standard par
	शास्त्र‚काराणाम‚र्थेषु भ्रान्त्या \edtext{}{\lemma{भ्रान्त्या}\Bfootnote{विप‚रीत‚स्य स्व‚भा० \cite{dp-msB} \cite{dp-edP} \cite{dp-edH} \cite{dp-edE}}}विप‚रीत‚स्व‚भावो\edtext{}{\lemma{भावो}\Bfootnote{भाव‚स्योप‚सं० \cite{dp-edN}}} प‚संहार‚स‚म्भ‚वात् ॥ ११५ ॥
	\pend% ending standard par
      
	  \endgroup
	‚{\tiny $_{lb}$}‚ 

	  \pstart \leavevmode% starting standard par
	शास्त्र‚कृतां विप‚रीत‚स्य व‚स्तुविरुद्ध‚स्य स्व‚भाव‚स्य उप‚संहारो ढौक‚न‚म‚र्थेषु । त‚स्य‚{\tiny $_{lb}$}‚ संभ‚वाद् विरुद्धाव्य‚भिचारिस‚म्भ‚वः । भ्रान्त्येति विप‚र्यासेन । विप‚र्य‚स्ता हि शास्त्र‚काराः\edtext{}{\lemma{काराः}\Bfootnote{०कारास्तं त‚म‚स‚न्तं स्व‚भा० \cite{dp-edE}}}‚{\tiny $_{lb}$}‚ स‚न्त‚म‚स‚न्तं स्व‚भाव‚मारोप‚य‚न्तीति ॥
	\pend% ending standard par
       ‚{\tiny $_{lb}$}‚ 

	  \pstart \leavevmode% starting standard par
	य‚दि शास्त्र‚कृतोऽपि भ्रान्ताः, अन्येष्व‚पि पुरुषेषु क आश्वास इत्याह--
	\pend% ending standard par
       ‚{\tiny $_{lb}$}‚ 
	  \bigskip
	  \begingroup
	

	  \pstart \leavevmode% starting standard par
	न ह्य‚स्य स‚म्भ‚वो\edtext{}{\lemma{वो}\Bfootnote{स‚म्भ‚वोऽस्ति य‚था० \cite{dp-msC}}} य‚थाव‚स्थित‚व‚स्तुस्थितिष्वात्म‚कार्यानुप‚ल‚म्भेषु\edtext{}{\lemma{म्भेषु}\Bfootnote{०ष्वात्म‚कार्येषूप‚ल‚म्भेषु--\cite{dp-msB} \cite{dp-edP} \cite{dp-edH}}} ॥ ११६ ॥
	\pend% ending standard par
      
	  \endgroup
	‚{\tiny $_{lb}$}‚ 

	  \pstart \leavevmode% starting standard par
	न‚हीति । न हेतुषु क‚ल्प‚न‚या हेतुत्व‚व्य‚व‚स्था । अपि तु व‚स्तुस्थित्या । त‚तो‚{\tiny $_{lb}$}‚ य‚थाव‚स्थित‚व‚स्तुस्थितिष्वात्म‚कार्यानुप‚ल‚म्भेष्व‚स्य स‚म्भ‚वो नास्ति ।
	\pend% ending standard par
       ‚{\tiny $_{lb}$}‚ 

	  \pstart \leavevmode% starting standard par
	अव‚स्थितं प‚र‚मार्थ‚स‚द्व‚स्तु त‚द‚न‚तिक्रान्ता य‚थाव‚स्थिता \edtext{}{\lemma{स्थिता}\Bfootnote{०स्थित‚व‚स्तु० \cite{dp-msC}}}व‚स्तुस्थितिर्व्य‚व‚स्था\edtext{}{\lemma{स्था}\Bfootnote{०स्थितिव्य‚व० \cite{dp-msA} \cite{dp-msB} \cite{dp-edP} \cite{dp-edH} \cite{dp-edE}}} येषां ते‚{\tiny $_{lb}$}‚ य‚थाव‚स्थित‚व‚स्तुस्थित‚यः । ते हि य‚था व‚स्तु स्थितं त‚था स्थिताः\edtext{}{\lemma{स्थिताः}\Bfootnote{त‚था स्थापिता न \cite{dp-msC} \cite{dp-msD}}} । न क‚ल्प‚न‚या । \edtext{\textsuperscript{*}}{\lemma{*}\Bfootnote{अतः \cite{dp-msA} \cite{dp-msB} \cite{dp-edP} \cite{dp-edH} \cite{dp-edE} \cite{dp-edN}}}त‚त‚स्तेषु‚{\tiny $_{lb}$}‚ न भ्रान्तेर‚व‚काशोऽस्ति येन विरुद्धाव्य‚भिचारिस‚म्भ‚वः स्यात् ॥
	\pend% ending standard par
       ‚{\tiny $_{lb}$}‚ 

	  \pstart \leavevmode% starting standard par
	त‚त्र विरुद्धाव्य‚भिचारिण्युदाह‚र‚ण‚म्--
	\pend% ending standard par
      
	  \endgroup
	‚{\tiny $_{lb}$}‚

	  \pstart \leavevmode% starting standard par
	\textbf{न‚नु चेत्यादि विप‚र्य‚स्ता ही}त्येत‚द‚न्तं सुग‚म‚म् । \textbf{शास्त्र‚कारा} इति तीर्थिक‚शास्त्र‚प्र‚णेतार इति‚{\tiny $_{lb}$}‚ द्र‚ष्ट‚व्य‚म्, त‚द्व‚च‚न‚स्यैव प्र‚माण‚बाधित‚त्वेन तेषामेव विप‚र्य‚स्त‚त्वात् ॥
	\pend% ending standard par
      ‚{\tiny $_{lb}$}‚

	  \pstart \leavevmode% starting standard par
	\textbf{अन्येष्व‚पी}ति कार्यादिहेतुप्र‚योक्तृषु ।
	\pend% ending standard par
      ‚{\tiny $_{lb}$}‚

	  \pstart \leavevmode% starting standard par
	\textbf{य‚थाव‚स्थित‚व‚स्तुस्थितिष्विति}--अस्य तात्प‚र्यार्थ‚माह--\textbf{न हेतुष्वि}ति । त‚तः‚{\tiny $_{lb}$}‚ क‚ल्प‚न‚या हेतुत्वाद्य\edtext{}{\lemma{हेतुत्वाद्य}\Bfootnote{त्व‚व्य}}व‚स्थायाः । अर्थ‚क्रियास‚म‚र्थ‚त्वं\edtext{}{\lemma{त्वं}\Bfootnote{न्तु}} \textbf{प‚र‚मार्थ‚स‚त्} । क‚थं ते त‚थारूपा‚{\tiny $_{lb}$}‚ इत्याह--\textbf{ते ही}ति । \textbf{ते} कार्याद‚यो हिर्य‚स्माद‚र्थे । हेतुभावे चैत‚द्विशेष‚ण‚म् । य‚त‚स्ते य‚था‚{\tiny $_{lb}$}‚व‚स्थित‚य‚स्त‚त‚स्तेष्व\edtext{}{\lemma{स्तेष्व}\Bfootnote{षु}}स‚म्भ‚वो नास्तीत्य‚र्थः । स‚त्यां स्थितौ किं न स‚म्भ‚व इत्याह--\textbf{त‚त} इति ।‚{\tiny $_{lb}$}‚ ‚{\tiny $_{lb}$}‚ ‚{\tiny $_{lb}$}‚ ‚{\tiny $_{lb}$}‚ \leavevmode\ledsidenote{\textenglish{228/dm}}‚{\tiny $_{lb}$}‚ 
	  
	\edtext{\textsuperscript{*}}{\lemma{*}\Bfootnote{अत्रोदा० \cite{dp-edE}}}त‚त्रोदाह‚र‚ण‚म्--य‚त् स‚र्व‚देशाव‚स्थितैः \edtext{}{\lemma{स्थितैः}\Bfootnote{०तैः स‚म्ब‚न्धिभिः स‚म्ब० \cite{dp-msC} तैः स्व‚स‚म्व‚न्धिभिः स‚म्ब० \cite{dp-msB} \cite{dp-edP} \cite{dp-edH} \cite{dp-edN}}}स्व‚स‚म्ब‚न्धिभिर्युग‚प‚द‚भिस‚म्ब‚ध्य‚ते‚{\tiny $_{lb}$}‚ त‚त् स‚र्व‚ग‚त‚म् । य‚थाऽऽकाश‚म् ।\edtext{\textsuperscript{*}}{\lemma{*}\Bfootnote{०काश‚मिति \cite{dp-msC}}} अभिस‚म्ब‚ध्य‚ते च\edtext{}{\lemma{च}\Bfootnote{च नास्त्रि \cite{dp-msB} \cite{dp-edP} \cite{dp-edH} \cite{dp-edE}}} स‚र्वादेशाव‚स्थितैः‚{\tiny $_{lb}$}‚ स्व‚स‚म्ब‚न्धिभिर्युग‚प‚त् सामान्य‚मिति ॥ ११७ ॥‚{\tiny $_{lb}$}‚ 
	  
	य‚त् स‚र्व‚स्मिन् देशेऽव‚स्थितैः स्व‚स‚म्ब‚न्धिभिर्युग‚प‚द‚भिस‚म्ब‚ध्य‚ते \edtext{}{\lemma{ते}\Bfootnote{ध्य‚ते त‚त्स‚र्वः \cite{dp-msA} \cite{dp-msB} \cite{dp-edP} \cite{dp-edH} \cite{dp-edE} \cite{dp-edN}}}इति स‚र्व‚देशाव‚स्थि‚{\tiny $_{lb}$}‚त‚र‚भिस‚म्ब‚ध्य‚मान‚त्वं सामान्य‚स्य अनूद्य स‚र्व‚ग‚त‚त्वं विधीय‚ते । तेन युग‚प‚द‚भिस‚म्ब‚ध्य‚मान‚त्वं‚{\tiny $_{lb}$}‚ स‚र्व‚ग‚त‚त्वे निय‚तं तेन व्याप्तं क‚थ्य‚ते । ‚{\tiny $_{lb}$}‚ 
	  
	इह सामान्यं क‚णाद‚म‚ह‚र्षिणा निष्क्रियं दृश्य‚मेकं\edtext{}{\lemma{मेकं}\Bfootnote{दृश्य‚मेवोक्त‚म् \cite{dp-msB}}} चोक्त‚म् । युग‚प‚च्च स‚र्वैः स्वैः‚{\tiny $_{lb}$}‚ \edtext{\textsuperscript{*}}{\lemma{*}\Bfootnote{स‚र्वैः स्वैः स्वैः स‚म्ब० \cite{dp-msA} \cite{dp-edP} \cite{dp-edH} \cite{dp-edE} \cite{dp-edN} स‚र्वैः स्वैः स्वैः स्व‚स‚म्ब० \cite{dp-msB}}}स‚म्ब‚न्धिभिः स‚म‚वायेन संब‚द्ध‚म् । त‚त्र पैलुकेन क‚णाद‚शिष्येण व्य‚क्तिषु व्य‚क्तिर‚हितेषु च‚{\tiny $_{lb}$}‚ य‚त‚स्ते क‚ल्प‚न‚या नः स्थापितास्त‚तः कार‚णात् । \textbf{भ्रान्ते}र्विप‚र्यास‚स्याव‚स‚रोऽ\textbf{व‚काशः । येन}‚{\tiny $_{lb}$}‚ भ्रान्त्य‚व‚काशेन ॥
	\pend% ending standard par
      ‚{\tiny $_{lb}$}‚

	  \pstart \leavevmode% starting standard par
	क‚स्यानुवादेनात्र क‚स्य विधिरित्या\leavevmode\ledsidenote{\textenglish{76a/ms}}ह--\textbf{स‚र्वे}ति । \textbf{स‚र्व‚देशाव‚स्थितैः}--स्व‚{\tiny $_{lb}$}‚स‚म्ब‚न्धिभिरित्य‚र्थात् । य‚त एव‚म‚नुवाद‚विधिक्र‚म‚स्तेन हेतुना ।
	\pend% ending standard par
      ‚{\tiny $_{lb}$}‚

	  \pstart \leavevmode% starting standard par
	न‚नु स‚र्वैः स्व‚स‚म्ब‚न्धिभिर्युग‚प‚द‚भिस‚म्ब‚न्धो नाम सामान्य‚स्य युग‚प‚त्स‚र्व‚स‚म्ब‚न्धिस‚म‚वाय‚{\tiny $_{lb}$}‚ एव । स‚र्व‚ग‚त‚त्व‚म‚पीद‚मेवास्येति ।
	\pend% ending standard par
      ‚{\tiny $_{lb}$}‚

	  \pstart \leavevmode% starting standard par
	क‚थ‚म‚न‚योर्व्यावृत्तितोऽपि भेद‚स‚म्ब‚न्ध‚भाव‚तो ग‚म्य‚ग‚म‚क‚भाव इति चेत् । नैष दोषः ।‚{\tiny $_{lb}$}‚ नानादेश‚स्थैः स्व‚स‚म्ब‚न्धिभिः शाव‚लेयादिभिर्युग‚प‚द‚भिस‚म्ब‚न्धो हेतुः । स‚म्ब‚न्धिदेश‚त‚द‚न्त‚राल‚{\tiny $_{lb}$}‚व्यापित्वं तु साध्य‚मिति ग‚म्य‚ग‚म‚क‚भावो न विरुध्य‚ते । स‚र्व‚स‚म्ब‚न्धिभिर्युग‚प‚द‚भिस‚म्ब‚न्ध‚श्चा‚{\tiny $_{lb}$}‚ग‚त्वाऽनाग‚च्छ‚द्भिरिति द्र‚ष्ट‚व्य‚म् ।
	\pend% ending standard par
      ‚{\tiny $_{lb}$}‚

	  \pstart \leavevmode% starting standard par
	अथ केन विरुद्धो\edtext{}{\lemma{विरुद्धो}\Bfootnote{द्धा}}व्य‚भिचारिप्र‚स‚व‚बीजं ध‚र्म‚द्व‚य‚योः किम‚भ्युप‚ग‚तं येन त‚योः स‚न्नि‚{\tiny $_{lb}$}‚पाताद् विरुद्ध\edtext{}{\lemma{विरुद्ध}\Bfootnote{द्धा}}व्य‚भिचारिस‚म्भ‚व इत्याह--\textbf{इहे}ति । \textbf{इह} सामान्य‚प‚दार्थ‚विचार‚प्र‚क्र‚मे ।‚{\tiny $_{lb}$}‚ क‚ण‚म‚त्तीति \textbf{क‚णादः} । रूढिव‚शाच्चायं श‚ब्दः \textbf{काश्य‚पे} मुनौ व‚र्त्त‚ते । स चासौ म‚ह‚र्षिश्चेति ।‚{\tiny $_{lb}$}‚ हेतुभावेनास्य विशेष‚ण‚त्वात् क‚णाद‚त्वादेव म‚ह‚र्षिः । एवं त‚स्य हि काष्ठाग‚ता निःस्पृह‚ता य‚तोऽ‚{\tiny $_{lb}$}‚ न्य‚र‚वो\edtext{}{\lemma{वो}\Bfootnote{?}}स्व‚भोज्यादिक‚म‚पि प‚रित्य‚ज्य क‚ण‚मात्रं भुक्त्वा ध्यानादिक‚माच‚र‚ति । अथोऽसाव‚{\tiny $_{lb}$}‚न्येभ्यः सातिश‚य‚वान् भ‚व‚तीति । \textbf{निष्क्रियं} क्रियाशून्य‚म‚मूर्त्त‚त्वात् । \textbf{एक}म‚नानारूप‚म्, प्र‚त्येकं‚{\tiny $_{lb}$}‚ स्वाश्र‚येषु ल‚क्ष‚णाविशेषाद्, विशेष‚ल‚क्ष‚णाभावाच्च । न तु स‚म‚वायादेकं त्रिलोक्यां सामान्य‚म्,‚{\tiny $_{lb}$}‚ प्र‚त्य‚य‚भेदात् प‚र‚स्प‚र‚तोऽन्य‚त्वात् । गोत्वादीनाञ्च निष्क्रिय‚त्वेन स‚हाऽस्यैक‚त्वं स‚मुच्चिनोति ।‚{\tiny $_{lb}$}‚ ‚{\tiny $_{lb}$}‚ \leavevmode\ledsidenote{\textenglish{229/dm}}‚{\tiny $_{lb}$}‚ 
	  
	देशेषु सामान्यं स्थितं साध‚यितुं \edtext{}{\lemma{यितुं}\Bfootnote{प्र‚माण‚मुप० \cite{dp-msC}}}प्र‚माण‚मिद‚मुप‚न्य‚स्त‚म् य‚थाकाश‚मिति--व्याप्तिप्र‚द‚र्श‚न‚{\tiny $_{lb}$}‚विष‚यो दृष्टान्तः । आकाश‚म‚पि हि स‚र्व‚देशाव‚स्थितैर्वृक्षादिभिः स्व‚संयोगिभिर्युग‚प‚द‚भिस‚म्ब‚ध्य‚मानं‚{\tiny $_{lb}$}‚ स‚र्व‚ग‚तं च । अभिस‚म्ब‚ध्य‚ते च\edtext{}{\lemma{च}\Bfootnote{०ते वा स‚र्व० \cite{dp-msB}}} स‚र्व‚देशाव‚स्थितैः स्व‚स‚म्ब‚न्धिभिरिति हेतोः प‚क्ष‚ध‚र्म‚त्व‚प्र‚द‚र्श‚न‚म् ॥ ‚{\tiny $_{lb}$}‚ 
	  
	अस्य स्व‚भाव‚हेतुत्वं \edtext{}{\lemma{हेतुत्वं}\Bfootnote{प्र‚योज‚यितुमाह--\cite{dp-msA} \cite{dp-msB} \cite{dp-edP} \cite{dp-edH} \cite{dp-edN} योज‚य‚न्नाह \cite{dp-edE}}}योज‚यितुमाह-- ‚{\tiny $_{lb}$}‚ 
	  
	त‚त्स‚म्ब‚न्धिस्व‚भाव‚मात्रानुब‚न्धिनी त‚द्देश‚स‚न्निहित‚स्व‚भाव‚ता ॥ ११८ ॥‚{\tiny $_{lb}$}‚ 
	  
	त‚त्स‚म्ब‚न्धीति । तेषां स‚र्व‚देशाव‚स्थितानां द्र‚व्याणां स‚म्ब‚न्धी सामान्य‚स्य स्व‚भावः स एव‚{\tiny $_{lb}$}‚ त‚त्स‚म्ब‚न्धिस्व‚भाव‚मात्र‚म् । त‚द‚नुब‚ध्नातीति त‚द‚नुब‚न्धिनी । ‚{\tiny $_{lb}$}‚ 
	  
	कासावित्याह--त‚द्देश‚संन्निहित‚स्व‚भाव‚ता । तेषां स‚म्ब‚न्धिनां देश‚स्त‚द्देशः । त‚द्देशे‚{\tiny $_{lb}$}‚ स‚न्निहितः स्व‚भावो य‚स्य त‚त् त‚द्देश‚संनिहित‚स्व‚भाव‚म्\edtext{}{\lemma{म्}\Bfootnote{स्व‚भावः \cite{dp-msC} \cite{dp-msD} \cite{dp-edE}}} । त‚स्य भाव‚स्त‚त्ता । य‚स्य हि येषां‚{\tiny $_{lb}$}‚ स‚म्ब‚न्धी स्व‚भावः त‚न्निय‚मेन तेषां देशे स‚न्निहितं भ‚व‚ति । त‚त‚स्त‚त्स‚म्ब‚न्धित्वानुब‚न्धिनी त‚द्देश‚{\tiny $_{lb}$}‚संनिहित‚ता सामान्य‚स्य ॥‚{\tiny $_{lb}$}‚ \textbf{युग‚प‚दे}क‚काल‚म् । \textbf{चः} पूर्वापेक्ष‚या स‚मुच्य‚ये \textbf{स‚म‚वायेन} स‚म्ब‚न्धेन स‚म्ब‚द्ध‚त्व‚म् । एव‚म‚भिहिते‚{\tiny $_{lb}$}‚ \textbf{क‚णादेन} त‚च्छिष्येण \textbf{पैलुकेन} । पील‚वः प‚र‚माण‚वः । पीलुपाके चायं पीलुश‚ब्द उप‚चारास‚ता‚{\tiny $_{lb}$}‚स्त्त्येन \edtext{}{\lemma{स्त्त्येन}\Bfootnote{उप‚च‚रितोऽस्ति । तेन}} निमित्तेन व्य‚व‚ह‚र‚तीति \textbf{पैलुकः} । तेनाविव‚क्षिताऽऽन्त‚र‚भेद‚स्य‚{\tiny $_{lb}$}‚युग‚प‚त्स‚र्व‚स‚म्ब‚न्ध‚मात्र‚हेतुत्वादाकाश‚स्य दृष्टान्त‚रूप‚ता द्र‚ष्ट‚व्या । न तु सामान्य‚स्येवास्य‚{\tiny $_{lb}$}‚ स‚म्ब‚न्धिभिः स‚म‚वायेन स‚म्ब‚न्धः । संयोग‚ल‚क्ष‚णेनास्य स‚म्ब‚न्धेन स‚म्ब‚न्धात् ।
	\pend% ending standard par
      ‚{\tiny $_{lb}$}‚

	  \pstart \leavevmode% starting standard par
	एत‚देवाभिप्रेत्याह--\textbf{आकाश‚म‚पी}ति । हीति य‚स्मात् । \textbf{स्व‚संयोगिभि}रिति वास्त‚वा‚{\tiny $_{lb}$}‚नुवादः । \textbf{न} त्येत‚त् प्र‚कृताङ्ग‚म् । दार्ष्टान्तिकेऽस्यानुप‚प‚त्तेः । \textbf{चो}ऽभिस‚म्ब‚न्ध\edtext{}{\lemma{न्ध}\Bfootnote{न्ध्य}}मान‚त्वेन‚{\tiny $_{lb}$}‚ स‚ह स‚र्व‚ग‚त‚त्व‚स्यैक‚विष‚य‚तां स‚मुच्चिनोति ॥
	\pend% ending standard par
      ‚{\tiny $_{lb}$}‚

	  \pstart \leavevmode% starting standard par
	\textbf{द्र‚व्याणां} ग‚वादीनाम् । एत‚च्च गोत्वादिसामान्य‚विव‚क्ष‚योक्त‚म् । उप‚ल‚क्ष‚णं द्र‚व्य‚ग्र‚ह‚णं‚{\tiny $_{lb}$}‚ क‚र्त्त‚व्य‚म्, इत‚र‚थोत्क्षेप‚ण‚त्वादिसामान्य‚स्यास‚ङ्ग्र‚हः स्यात् ।
	\pend% ending standard par
      ‚{\tiny $_{lb}$}‚

	  \pstart \leavevmode% starting standard par
	एत‚च्च \textbf{त‚स्य भाव} इत्येद‚न्तं सुग‚म‚म् ।
	\pend% ending standard par
      ‚{\tiny $_{lb}$}‚

	  \pstart \leavevmode% starting standard par
	\edtext{\textsuperscript{*}}{\lemma{*}\Bfootnote{०ते वा स‚र्व० \cite{dp-msB}}}\textbf{त‚स्मादि}त्य‚नेनार्थाग‚तं स्व‚भाव‚हेतुत्व‚निमित्तं द‚र्श‚य‚ति । \textbf{न} तु त‚द्धित‚प्र‚त्य‚यान्ते‚{\tiny $_{lb}$}‚ प‚ञ्च‚म्य‚स्ति यां व्याच‚क्षीत । अयं त्व‚स्यार्थः--य‚स्माद् युग‚प‚त् स‚र्व‚देशाव‚स्थित‚स‚म्ब‚न्धिस‚म्ब‚न्धः‚{\tiny $_{lb}$}‚ स्व‚स‚त्तामात्रानुब‚न्धिनि साध्ये हेतुः, त‚स्मात्स्व‚भाव हेतुत्व‚म‚स्येति ।
	\pend% ending standard par
      ‚{\tiny $_{lb}$}‚

	  \pstart \leavevmode% starting standard par
	न‚नु त‚त्स‚म्ब‚न्धिनोऽपि त‚द्देश‚स‚न्निहित‚स्व‚भाव‚तैव कुतो येनैवं भ‚व‚तीत्याह--\textbf{य‚स्ये}ति ।‚{\tiny $_{lb}$}‚ हिर्य‚स्मात् । य‚तोऽयं सामान्य‚न्यायः \textbf{त‚त}स्त\leavevmode\ledsidenote{\textenglish{76b/ms}}स्मात् । य‚द्वा स‚र्व‚स‚म्ब‚न्धित्वेऽपि क‚स्मा‚{\tiny $_{lb}$}‚‚{\tiny $_{lb}$}‚ ‚{\tiny $_{lb}$}‚ \edtext{\textsuperscript{*}}{\lemma{*}\Bfootnote{अत्र मूले \textbf{त‚स्य भावः । त‚स्मात्} इति पाठः क‚ल्प्यः--सं०}} \leavevmode\ledsidenote{\textenglish{230/dm}}‚{\tiny $_{lb}$}‚ 
	  
	न‚नु च ग‚वां स‚म्ब‚न्धी स्वामी । न च \edtext{}{\lemma{च}\Bfootnote{न च त‚द्देशे स‚न्नि० \cite{dp-msA} \cite{dp-msB} \cite{dp-edP} \cite{dp-edH} \cite{dp-edE} न च त‚द्देश‚संनि० \cite{dp-edN}}}गोदेशे स‚न्निहित‚स्व‚भावः । \edtext{\textsuperscript{*}}{\lemma{*}\Bfootnote{०भावः स्वामी \cite{dp-msA} \cite{dp-msB} \cite{dp-edP} \cite{dp-edH} \cite{dp-edE} \cite{dp-edN}}}त‚त् क‚र्थं \edtext{}{\lemma{र्थं}\Bfootnote{क‚थं संबं० \cite{dp-msA} \cite{dp-msB} \cite{dp-msC} \cite{dp-msD} \cite{dp-edP} \cite{dp-edH} \cite{dp-edN}}}त‚त्स‚म्ब‚{\tiny $_{lb}$}‚न्धित्वात् त‚द्देश‚त्व‚मित्याह-- ‚{\tiny $_{lb}$}‚ 
	  
	न हि यो य‚त्र नास्ति त‚द्देश‚मात्म‚ना व्याप्नोतीति स्व‚भाव‚हेतुप्र‚योगः ॥ ११९ ॥‚{\tiny $_{lb}$}‚ 
	  
	न हीति । यो य‚त्र देशे नास्ति स देशो य‚स्य स त‚द्देशः तं न व्याप्नोत्यात्म‚ना स्व‚रूपेण । ‚{\tiny $_{lb}$}‚ 
	  
	इह सामान्य‚स्य त‚द्व‚तां च स‚म‚वाय‚ल‚क्ष‚णः \edtext{}{\lemma{णः}\Bfootnote{ल‚क्ष‚ण‚स‚म्ब० \cite{dp-msA}}}स‚म्ब‚न्धः । स चाभिन्न‚देश‚योरेव । \edtext{\textsuperscript{*}}{\lemma{*}\Bfootnote{अनेन \cite{dp-msB}}}तेन‚{\tiny $_{lb}$}‚ य‚त्र य‚त् स‚म‚वेतं \edtext{}{\lemma{वेतं}\Bfootnote{क‚र्त्तृ--\cite{dp-msD-n}}}त‚त् \edtext{}{\lemma{त्}\Bfootnote{क‚र्म--\cite{dp-msD-n}}}त‚दात्मी न रूपेण क्रोडीकुर्व‚त् \edtext{}{\lemma{त्}\Bfootnote{स‚म‚वाय‚रूप० \cite{dp-msC}}}स‚म‚वायिरूप‚देशे स्वात्मानं निवेश‚य‚ति ।‚{\tiny $_{lb}$}‚ त्त‚द्देश‚स‚न्निहित‚स्व‚भाव‚तेत्याह--\textbf{त‚स्मादि}ति । \textbf{त‚स्मा}त्त‚त्स‚म्ब‚न्धिमात्रानुब‚न्धिनीति । त‚द्देश‚{\tiny $_{lb}$}‚स‚न्निहित‚स्व‚भाव‚ताऽऽकाश‚स्य दृष्टा \textbf{त‚स्मात्} कार‚णात् । \textbf{य‚स्य} व‚स्तुन‚स्तेषां \textbf{\edtext{\textsuperscript{*}}{\lemma{*}\Bfootnote{नो येषां}} स‚म्ब‚न्धी‚{\tiny $_{lb}$}‚ स्व‚भावः । हि}र‚व‚धार‚णे \textbf{स‚म्ब}न्धीत्य‚स्मात्प‚रो द्र‚ष्ट‚व्यः । त‚द् व‚स्तु \textbf{निय‚मेनाव}श्यंत‚या‚{\tiny $_{lb}$}‚ \textbf{तेषां} स‚म्ब‚न्धीनां \textbf{देशे स‚न्निहि}तं \textbf{भ‚व‚ति} । य‚त एवं सामान्य‚न्याय\textbf{स्त‚तः । त‚स्मादि}ति पाठे‚{\tiny $_{lb}$}‚ भाव‚ग‚तिः । \edtext{\textsuperscript{*}}{\lemma{*}\Bfootnote{अत्र मूले त‚स्य भावः । क‚स्मात् ? इति पाठः क‚ल्प्यः--सं०}}\textbf{क‚स्मादि}ति \textbf{तु} क्व‚चित्पुस्त‚के पाठः । स तु युक्त‚रूपः । त‚त्स‚म्ब‚न्धिस्व‚भाव‚{\tiny $_{lb}$}‚माव‚मात्रानुब‚न्धिनी त‚द्देश‚स‚न्निहित‚स्व‚भाव‚तेति व्याख्या । \textbf{क‚स्मादेत‚दि}ति कार‚णाकाड्क्षास‚म्ब‚न्धात् ।‚{\tiny $_{lb}$}‚ त‚द‚न‚न्त‚रं च \textbf{य‚स्य हीत्या}देः सामान्योत्त‚र‚स्य, \textbf{त‚त} इत्यादेश्चोप‚संहार‚व्य‚प‚देशेन विशेषोत्त‚र‚स्य‚{\tiny $_{lb}$}‚ \textbf{सुमो} \edtext{\textsuperscript{*}}{\lemma{*}\Bfootnote{यो}} ज्य‚त्वादिति ।
	\pend% ending standard par
      ‚{\tiny $_{lb}$}‚

	  \pstart \leavevmode% starting standard par
	य‚त्र तु \textbf{त‚स्य भाव‚स्त‚त्ते}ति पाठः त‚त्र स‚र्व‚म‚व‚दात‚म् ॥
	\pend% ending standard par
      ‚{\tiny $_{lb}$}‚

	  \pstart \leavevmode% starting standard par
	न‚नु चैक‚देश‚स्थ‚मेव सामान्यं युग‚प‚त्स‚र्वैः स‚म्ब‚न्धिभिर‚भिस‚म्भ‚न्त्स्य‚ते । त‚त्किं त‚स्य‚{\tiny $_{lb}$}‚ व्यापारास‚म्भ‚वेनावेदितेनेत्याह--इहेति ।
	\pend% ending standard par
      ‚{\tiny $_{lb}$}‚

	  \pstart \leavevmode% starting standard par
	\textbf{स} इति स‚म‚वायः । च‚कारः पुनः श‚ब्द‚स्यार्थे । \textbf{अभिन्न‚देश‚यो}रिति लोक‚प्र‚सिद्ध‚{\tiny $_{lb}$}‚देशापेक्ष‚योक्त‚म् न तु शास्त्र‚प्र‚सिद्ध‚देशापेक्ष‚येति द्र‚ष्ट‚व्य‚म् । अन्य‚था य‚दा \edtext{}{\lemma{दा}\Bfootnote{था प‚ट}} त‚न्तूनां‚{\tiny $_{lb}$}‚ स‚म‚वायो न स्यात् । प‚ट‚स्य त‚न्न‚था \edtext{}{\lemma{था}\Bfootnote{त‚न्त‚वो}} देशः । त‚न्तूनां पुन‚रंश‚वः । सामान्य‚त‚द्व‚तोश्च‚{\tiny $_{lb}$}‚ न स्यात् । गोत्व‚सामान्य‚स्य गौर्देशः । गोश्च सास्नाद‚योऽव‚य‚वा इति ।
	\pend% ending standard par
      ‚{\tiny $_{lb}$}‚

	  \pstart \leavevmode% starting standard par
	अथ‚वा सामान्य‚ल‚क्ष‚ण‚युगा \edtext{}{\lemma{युगा}\Bfootnote{योगा}} पेक्ष‚या अभिन्न‚देश‚त्वं विव‚क्षित‚म् । न तु प्र‚त्येका‚{\tiny $_{lb}$}‚पेक्ष‚म् । तेनाय‚म‚र्थः--कुण्ड‚ब‚द‚र‚व‚द् य‚त्र द्वाव‚पि स‚म्ब‚न्धिनौ भिन्न‚देशौ न त‚योः स‚म‚वायः । य‚योस्त्वे‚{\tiny $_{lb}$}‚क‚त‚र‚स्यान्य‚त‚रो देश‚स्त‚यो स‚म‚वाय इति । एव‚ञ्च प‚ट‚त‚न्तूनां सामान्य‚त‚द्व‚तोश्च नास‚ङ्ग्र‚ह इति ।‚{\tiny $_{lb}$}‚ येन कार‚णेनाभिन्न‚देश‚योरेव स‚म‚वाय\textbf{स्तेन} प्र‚थ‚म‚व्याख्याने \textbf{स‚म‚वायिरूप‚स्य} स‚म‚वायिस्व‚भाव‚स्य‚{\tiny $_{lb}$}‚ देश इति । द्वितीय‚व्याख्याने \textbf{स‚म‚वायिरूप}मेव स‚म‚वायिस्व‚भाव एव \textbf{देश} इति विगृह्य त‚स्मिन्निति‚{\tiny $_{lb}$}‚ ‚{\tiny $_{lb}$}‚ \leavevmode\ledsidenote{\textenglish{231/dm}}‚{\tiny $_{lb}$}‚ 
	  
	\edtext{\textsuperscript{*}}{\lemma{*}\Bfootnote{त‚द्देशे रूप० \cite{dp-msD}}}त‚द्देश‚रूप‚निवेश‚न‚मेव त‚त्क्रोडीक‚र‚ण‚म् । त‚त‚स्त‚त्स‚म‚वायः । ‚{\tiny $_{lb}$}‚ 
	  
	त‚स्माद् य‚द् य‚त्र स‚म‚वेतं त‚त् \edtext{}{\lemma{त्}\Bfootnote{त‚त् त‚त्र द्र० \cite{dp-edE}}}त‚द्द्र‚व्यं व्याप्नुव‚दात्म‚ना त‚द्देशे संन्निहितं भ‚व‚ति । ‚{\tiny $_{lb}$}‚ 
	  
	त‚द‚य‚म‚र्थः--त‚द्देश‚स्थ‚व‚स्तुव्याप‚नं त‚द्देश‚स‚त्त‚या व्याप्त‚म् । त‚द्देश‚स‚त्ताऽभावे\edtext{}{\lemma{त्ताऽभावे}\Bfootnote{त‚द्देश‚स‚त्ताया अभावे \cite{dp-msB} \cite{dp-msC}}} त‚द्व्याप‚ना‚{\tiny $_{lb}$}‚भावाद् व्याप‚न‚ल‚क्ष‚ण स‚म‚वाय‚संम्ब‚न्धो न स्यात् । अस्ति च व्याप‚न‚म् । अत‚स्त‚द्देशे स‚न्निहित‚त्व‚{\tiny $_{lb}$}‚मिति । त‚द‚यं स्व‚भाव‚हेतुः ॥ ‚{\tiny $_{lb}$}‚ 
	  
	पैठ‚र‚प्र‚योगं द‚र्श‚य‚न्नाह-- ‚{\tiny $_{lb}$}‚ 
	  
	द्वितीयोऽपि प्र‚योगः--य‚दुप‚ल‚ब्धिल‚क्ष‚ण‚प्राप्तं स‚न्नोप‚ल‚भ्य‚ते न त‚त्‚{\tiny $_{lb}$}‚ त‚त्रास्ति । त‚द्य‚था--क्क‚चिद‚विद्य‚मानो घ‚टः ।\edtext{\textsuperscript{*}}{\lemma{*}\Bfootnote{घ‚ट इति \cite{dp-msC}}} नोप‚ल‚भ्य‚ते चोप‚ल‚ब्धिल‚क्ष‚ण‚{\tiny $_{lb}$}‚प्राप्तं सामान्यं व्य‚क्त‚य‚न्त‚रालेष्विति । अय‚म‚नुप‚ल‚म्भः\edtext{}{\lemma{म्भः}\Bfootnote{ल‚म्भ‚प्र‚योगः स्व० \cite{dp-msD} \cite{dp-msB} \cite{dp-edP} \cite{dp-edH} \cite{dp-edE} \cite{dp-edN}}} स्व‚भाव‚श्च प‚र‚स्प‚र‚{\tiny $_{lb}$}‚विरुद्धार्थ‚साध‚नादेक‚त्र संश‚यं ज‚न‚य‚तः ॥ १२० ॥‚{\tiny $_{lb}$}‚ 
	  
	द्वितीयोऽपि--इति । य‚दुप‚ल‚ब्धेर्ल‚क्ष‚ण‚तां विष‚य‚तां प्राप्तं दृश्य‚मित्य‚र्थः । एतेन दृश्या‚{\tiny $_{lb}$}‚नुप‚ल‚ब्धिम‚नूद्य न \edtext{}{\lemma{न}\Bfootnote{०नूद्य त‚त्त‚त्र \cite{dp-msB} नूद्य त [[न]] त‚त्त‚त्र \cite{dp-msA} नूद्य त‚त्त‚त्त‚त्र \cite{dp-edP} \cite{dp-edH}}}त‚त् त‚त्रास्ति इत्य‚स‚द्व्य‚व‚हार्य‚त्वं\edtext{}{\lemma{त्वं}\Bfootnote{व्य‚व‚हार‚विष‚य‚त्वं विहित‚म् \cite{dp-msD}}} विहित‚म् । त‚तो \edtext{}{\lemma{तो}\Bfootnote{व्याप्य‚स्य दृ० \cite{dp-edE}}}व्याप्य‚दृश्यानुप‚{\tiny $_{lb}$}‚ल‚ब्धेर्व्याप‚क‚म‚स‚द्व्य‚व‚हार्य‚त्वं द‚र्शित‚म् । त‚द्य‚थेति क्व‚चिद‚स‚न् घ‚टो दृष्टान्तः ।‚{\tiny $_{lb}$}‚ योज्य‚म् । उभ‚य‚त्रापि तु \textbf{स‚म‚वाय} \edtext{\textsuperscript{*}}{\lemma{*}\Bfootnote{यि}} श‚ब्देनाधारोऽभिप्रेतः । \textbf{स्वात्मानं निवेश‚य‚त्यु}प‚न‚य‚ति ।
	\pend% ending standard par
      ‚{\tiny $_{lb}$}‚

	  \pstart \leavevmode% starting standard par
	न‚नु त‚द्व्याप‚नं त‚त्क्रोडीक‚र‚ण‚म‚भिप्रेत‚म् । त‚त्क‚थं स‚म्ब‚न्धिनि स्वात्म‚नि निवेश‚नं‚{\tiny $_{lb}$}‚ व्याख्याय‚त इत्याह--त‚द्देश इति । स चासौ देश‚श्च त‚त्र रूप‚स्य स्व‚रूप‚स्य \textbf{निवेश‚न}मुप‚न‚य‚न‚म् ।‚{\tiny $_{lb}$}‚ \textbf{त‚त}स्त‚स्मात्त‚द्देश‚रूप‚निवेश‚नात्त‚स्यः । स‚म्ब‚न्धिनः स‚म‚वायः । \textbf{त‚स्मादित्या}दिनोप‚संहारः ।
	\pend% ending standard par
      ‚{\tiny $_{lb}$}‚

	  \pstart \leavevmode% starting standard par
	न‚नु त‚द्व्याप‚न‚म‚पि भ‚विष्य‚ति, न च त‚द्देश‚स‚न्निहित‚स्व‚भाव‚तेत्याश‚ङ्क्याह--\textbf{त‚द‚य‚मिति} ।‚{\tiny $_{lb}$}‚ य‚त आत्म‚ना त‚द्व्याप‚न‚ल‚क्ष‚णेन स‚म्ब‚न्धेन त‚द्व्याप्य‚मान‚देश‚स‚न्निधान‚मुक्तं \textbf{त‚त्त}स्माद‚यं तात्प‚{\tiny $_{lb}$}‚\textbf{र्यार्थः} । स चासौ \textbf{देश}श्च त‚त्र‚स्थ\textbf{व‚स्तुव्याप‚नं} लोक‚प्र‚सिद्ध‚देशापेक्ष‚या त‚स्य देश\textbf{स्त‚द्देश}स्त‚त्र या‚{\tiny $_{lb}$}‚ \textbf{स‚त्ता} विद्य‚मान‚ता त‚या \textbf{व्याप्त‚म्} । अन्य‚था तु स्व‚रूपेण व्याप‚नास‚म्भ‚वादित्य‚भिप्रायः ।
	\pend% ending standard par
      ‚{\tiny $_{lb}$}‚

	  \pstart \leavevmode% starting standard par
	त‚देव व्य‚तिरेक‚मुखेणोप‚पाद‚य‚न्ना\leavevmode\ledsidenote{\textenglish{77a/ms}}ह--\textbf{त‚द्देशेति} । नास्त्येवायं स‚म्ब‚न्ध इति \add{चेदाह--}‚{\tiny $_{lb}$}‚ \textbf{अस्ति चे}ति । \textbf{चो}ऽव‚धार‚णे । य‚त उक्तेन क्र‚मेण स्व‚भाव‚ल‚क्ष‚ण‚योगोऽस्यास्ति । \textbf{त‚त्त‚स्माद‚यं}‚{\tiny $_{lb}$}‚ युग‚प‚त् स‚र्व‚स‚म्ब‚न्ध्य‚भिस‚म्ब‚न्ध‚ल‚क्ष‚णो हेतुः \textbf{स्व‚भावः ॥}
	\pend% ending standard par
      ‚{\tiny $_{lb}$}‚‚{\tiny $_{lb}$}‚\textsuperscript{\textenglish{232/dm}}‚{\tiny $_{lb}$}‚
	  \bigskip
	  \begingroup
	

	  \pstart \leavevmode% starting standard par
	प‚क्ष‚ध‚र्म‚त्वं द‚र्श‚यितुमाह--नोप‚ल‚भ्य‚ते चेति । \edtext{\textsuperscript{*}}{\lemma{*}\Bfootnote{व्य‚क्तेर‚न्त‚रालं \cite{dp-msA} \cite{dp-msB} \cite{dp-edP} \cite{dp-edH}}}व्य‚क्त्य‚न्त‚रालं--व्य‚क्त्य‚न्त‚रं च व्य‚क्ति‚{\tiny $_{lb}$}‚शून्यं चाकाश‚म् । दृश्य‚म‚पि क‚स्यांचिद्व्य‚क्तौ गोसामान्य‚म‚श्वादिषु व्य‚क्त्य‚न्त‚रेषु व्य‚क्तिशून्ये\edtext{}{\lemma{क्तिशून्ये}\Bfootnote{रेषु शून्ये चाका० \cite{dp-msA}}}‚{\tiny $_{lb}$}‚ चाकाशे \edtext{}{\lemma{चाकाशे}\Bfootnote{चोप‚ल‚भ्य० \cite{dp-msA} \cite{dp-msB} \cite{dp-edP} \cite{dp-edH}}}नोप‚ल‚भ्य‚ते । त‚स्मान्न तेष्व‚स्तीति ग‚म्य‚ते ।
	\pend% ending standard par
       ‚{\tiny $_{lb}$}‚ 

	  \pstart \leavevmode% starting standard par
	अय‚म‚नुप‚ल‚म्भः पूर्वोक्त‚श्च स्व‚भावः \edtext{}{\lemma{भावः}\Bfootnote{प‚र‚स्प‚र‚विरु० \cite{dp-msA} \cite{dp-msB} \cite{dp-msC} \cite{dp-edP} \cite{dp-edH} \cite{dp-edE} \cite{dp-edN}}}प‚र‚स्प‚र‚स्य विरुद्धौ याव‚र्थौ त‚योः साध‚नात्‚{\tiny $_{lb}$}‚ तावेक‚स्मिन् ध‚र्मिणि संश‚यं ज‚न‚य‚तः । न ह्येकोऽर्थः प‚र‚स्प‚र‚विरुद्ध‚स्व‚भावो भ‚वितुम‚र्ह‚ति । \edtext{\textsuperscript{*}}{\lemma{*}\Bfootnote{स्व‚भावेन--\cite{dp-msD-n}}}एकेन‚{\tiny $_{lb}$}‚ चात्र \edtext{}{\lemma{चात्र}\Bfootnote{व्य‚क्त्य‚न्त‚रेषु--नास्ति \cite{dp-msB}}}व्य‚क्त्य‚न्त‚रेषु व्य‚क्तिशून्ये चाकाशे स‚त्त्व‚म्, अप‚रेण चानुप‚ल‚म्भेनास‚त्त्वं साध्य‚ते । न
	\pend% ending standard par
      
	  \endgroup
	‚{\tiny $_{lb}$}‚

	  \pstart \leavevmode% starting standard par
	\textbf{क‚णा}द‚स्यैवाप‚रः शिष्यः \textbf{पैठ‚र}स्त‚स्य \textbf{प्र‚योग‚म्} । पिठ‚रोऽव‚य‚विद्र‚व्य‚म् । पूव‚व‚दुप‚चारात्पिठ‚र‚{\tiny $_{lb}$}‚श‚ब्द‚स्तेन व्य‚व‚ह‚र‚तीति त‚थोक्तः ।
	\pend% ending standard par
      ‚{\tiny $_{lb}$}‚

	  \pstart \leavevmode% starting standard par
	\textbf{अस‚द‚व्य‚व‚हार्य‚त्व}म‚स‚दिति व्य‚व‚हार‚णीय‚त्वं \textbf{विहित‚म् । क‚स्याञ्चि}त्सास्नादिम‚त्यां व्याप्तौ‚{\tiny $_{lb}$}‚ \edtext{\textsuperscript{*}}{\lemma{*}\Bfootnote{\textbf{व्य‚क्तौ}}} व्य‚ज्य‚ते सामान्य‚म‚न‚येति \textbf{व्य‚क्तिः} ।
	\pend% ending standard par
      ‚{\tiny $_{lb}$}‚

	  \pstart \leavevmode% starting standard par
	प्रागुक्त‚स्ताव‚त्स्व‚भावः, अयं तु किंसंज्ञ‚को हेतुरित्याह--\textbf{अय‚मिति} । प्रागुक्तं स्मार‚य‚ति‚{\tiny $_{lb}$}‚ \textbf{पूर्वे}ति । तुश‚ब्द‚श्च‚श‚ब्द‚स्यार्थे । \textbf{तावे}तौ हेतू \textbf{ध‚र्मिण्येक‚स्मिन्} सामान्याख्ये । \textbf{संश‚यं} प्र‚कृत‚योः‚{\tiny $_{lb}$}‚ साध्य‚योरित्य‚र्थात् क‚थं संश‚यं ज‚न‚य‚त इत्याह । एक‚स्यैव तौ विव‚क्षित‚स‚र्व‚ग‚त‚त्वास‚र्वंग‚त‚त्व‚{\tiny $_{lb}$}‚ल‚क्ष‚णौ स्व‚भावौ भ‚विष्य‚त इत्याह--\textbf{न हीति} । हिर्य‚स्मात् । \textbf{प‚र‚स्प‚र‚विरुद्धौ स्व‚भावौ}‚{\tiny $_{lb}$}‚ य‚स्येति त‚था ।
	\pend% ending standard par
      ‚{\tiny $_{lb}$}‚

	  \pstart \leavevmode% starting standard par
	न‚नु चात्र विरुद्धावेव ध‚र्मावेक‚स्य सामान्य‚स्य द्वाभ्यामेताभ्यां सिद्ध्येते इति । त‚त्किमेत‚{\tiny $_{lb}$}‚दुच्य‚त इत्याह--\textbf{एकेनेति । चो} य‚स्माद‚र्थे । \textbf{एकेन}ति प्रागुक्तेन स्व‚भावेन । \textbf{अप‚रेणे}ति‚{\tiny $_{lb}$}‚ प‚श्चादुक्तेन । त‚मेवाह \textbf{अनुप‚ल‚म्भेन} प्र‚क‚र‚णाद् दृश्यानुप‚ल‚म्भेनेति नेय‚म् । साध‚य‚तां त‚र्हि‚{\tiny $_{lb}$}‚ व्य‚क्त्य‚न्त‚राले सामान्य‚स्य स‚त्त्व‚म‚स‚त्त्वं च एतौ हेतू का क्ष‚तिरिह--इत्याह \textbf{न चेति । चो}ऽव‚{\tiny $_{lb}$}‚धार‚णे । स‚त्त्व‚म‚स‚त्त्वं च द्व‚योर्द्वाभ्यां साध‚ने किम‚नुप‚प‚न्न‚मित्याश‚ङ्क्याह--\textbf{एक‚स्यापि} । काल‚भेदे‚{\tiny $_{lb}$}‚ किन्नैवं स‚म्भ‚व‚तीत्याश‚ङ्क्याह--\textbf{एक‚दैवेति} । क‚स्याप्येक‚दैवाधिक‚र‚ण‚भेदेऽप्येत‚दित्याश‚ङ्क्याह—‚{\tiny $_{lb}$}‚\textbf{एक‚त्रेति ।} क‚थ‚म‚युक्त‚मित्याह--\textbf{त‚योरि}ति । \textbf{त‚योः} स‚त्त्वास‚त्त्व‚योः । \textbf{विरोधा}त्प‚र‚स्प‚र‚प‚रि‚{\tiny $_{lb}$}‚हार‚व्य‚व‚स्थित‚रूप‚त्वात् ।
	\pend% ending standard par
      ‚{\tiny $_{lb}$}‚

	  \pstart \leavevmode% starting standard par
	क‚थं पुन‚राग‚माश्र‚यानुमानाश्र‚य‚त्वं विरुद्धाव्य‚भिचारि\add{णी} त्याश‚ङ‚क्योप‚संहार‚व्याजेन‚{\tiny $_{lb}$}‚ य‚थाऽन‚योस्त‚थात्वं त‚था द‚र्श‚य‚न्नाह--\textbf{त‚दि}ति । य‚त एवं त‚त्त‚द्व‚देकं सामान्यं \textbf{क‚णादेन} उक्त‚म्,‚{\tiny $_{lb}$}‚ त‚द्रूप‚विचारे त‚च्छिष्याभ्यामेवं प्र‚क्रान्तं \textbf{त‚त्} त‚स्मादा\textbf{ग‚म‚सि}द्ध‚स्याग‚म‚प्र‚तिपादित‚स्यानुप‚ल‚म्भेनापि‚{\tiny $_{lb}$}‚ व्य‚क्त‚त्य‚न्त‚रालास‚त्त्व‚प्र‚तिपाद‚न‚द्वार‚म्\edtext{}{\lemma{म्}\Bfootnote{द्वाराऽ}}स‚र्व‚ग‚त‚त्व‚स्य साध‚नात् । स‚र्व‚ग‚त‚त्वास‚र्व‚ग‚त‚त्व‚योः‚{\tiny $_{lb}$}‚ \textbf{साध्य‚योरेता}वित्याह । एवाव‚न्तं \edtext{}{\lemma{न्तं}\Bfootnote{?}} त‚थोक्तौ हेतू ।
	\pend% ending standard par
      ‚{\tiny $_{lb}$}‚

	  \pstart \leavevmode% starting standard par
	विरुद्धाव्य‚भिचारित्व‚मेवान‚योरुप‚द‚र्श‚य‚न्नाह--\textbf{य‚त} इति । \textbf{च}कारः पूर्वापेक्ष‚या स‚मुच्च‚ये ।‚{\tiny $_{lb}$}‚ ‚{\tiny $_{lb}$}‚ \leavevmode\ledsidenote{\textenglish{233/dm}}‚{\tiny $_{lb}$}‚ 
	  
	चैक‚स्यैक‚दैक‚त्र स‚त्त्व‚म‚स‚त्त्वं च\edtext{}{\lemma{च}\Bfootnote{वा \cite{dp-msD}}} युक्त‚म्, त‚योर्विरोधात् । \edtext{\textsuperscript{*}}{\lemma{*}\Bfootnote{त‚स्मादाग‚म० \cite{dp-msB}}}त‚दाग‚म‚सिद्ध‚स्य सामान्य‚स्य‚{\tiny $_{lb}$}‚ स‚र्व‚ग‚त‚त्वास‚र्व‚ग‚त‚त्व‚योः साध्य‚योरेतौ विरुद्धाव्य‚भिचारिणौ जातौ । \edtext{\textsuperscript{*}}{\lemma{*}\Bfootnote{अथ प‚र‚स्प‚र‚विरुद्धाव्य‚भिचारिहेतुदोषः क‚णाद‚शिष्य‚योरेवाय‚म्, न क‚णाद‚स्येत्याह--\cite{dp-msD-n}}}य‚तः सामान्य‚स्यैक‚स्य‚{\tiny $_{lb}$}‚ युग‚प‚त् स‚र्व‚देशाव‚स्थितैर‚भिस‚म्ब‚न्धित्वं\edtext{}{\lemma{न्धित्वं}\Bfootnote{०भिस‚म्ब‚द्ध‚त्व‚म् \cite{dp-msB} भिस‚म्ब‚न्ध‚त्व‚म् \cite{dp-msD}}} चाभ्युप‚ग‚त‚म्, दृश्य‚त्वं च । त‚तः स‚र्व‚स‚म्ब‚न्धित्वात्‚{\tiny $_{lb}$}‚ स‚र्व‚ग‚त‚त्व‚म्, \edtext{\textsuperscript{*}}{\lemma{*}\Bfootnote{दृश्य‚त्वाद‚न्त‚रालानुप‚ल० \cite{dp-msA} \cite{dp-msB} \cite{dp-msC} \cite{dp-msD} \cite{dp-edP} \cite{dp-edH} \cite{dp-edE} \cite{dp-edN}}}दृश्य‚त्वाद‚न्त‚रालाद‚नुप‚ल‚म्भाद‚स‚र्व‚ग‚त‚त्व‚म् । त‚तः \edtext{}{\lemma{तः}\Bfootnote{क‚णाद‚द्वारेण--\cite{dp-msD-n}}}शास्त्र‚कारेणैव विरुद्ध‚व्याप्त‚त्व‚{\tiny $_{lb}$}‚म‚प‚श्य‚ता विरुद्ध‚व्याप्तौ ध‚र्मावुक्त्वा\edtext{}{\lemma{र्मावुक्त्वा}\Bfootnote{ध‚र्मावुक्तौ । इह विरुद्धा० \cite{dp-msB} ध‚र्मावुक्तौ विरु० \cite{dp-msC} \cite{dp-msD}}} विरुद्धाव्य‚भिचार्य‚व‚काशो द‚त्त इति । न च\edtext{}{\lemma{च}\Bfootnote{च नास्ति \cite{dp-msB} \cite{dp-msD}}} व‚स्तुन्य‚स्य‚{\tiny $_{lb}$}‚ \edtext{\textsuperscript{*}}{\lemma{*}\Bfootnote{०स्य हेतोः स‚म्भ‚वः \cite{dp-msB} \cite{dp-msD} विरुद्धाव्य‚भिचारिणः--\cite{dp-msD-n}}}स‚म्भ‚वः । इत्युक्ता हेत्वाभासाः ॥ ‚{\tiny $_{lb}$}‚ 
	  
	न‚नु च साध‚नाव‚य‚व‚त्वाद् य‚था हेत‚व उक्तास्त‚त्प्र‚स‚ङ्गेन \edtext{}{\lemma{ङ्गेन}\Bfootnote{च नास्ति--\cite{dp-msD}}}च हेत्वाभासाः, त‚था साध‚ना‚{\tiny $_{lb}$}‚व‚य‚व‚त्वाद् दृष्टान्ता व‚क्त‚व्यास्त‚त्प्र‚स‚ङ्गेन च दृष्टान्ताभासाः । त‚त् क‚थं नोक्ता इत्याह--‚{\tiny $_{lb}$}‚ त‚तोऽभ्युप‚ग‚तात्स‚र्व‚स‚म्ब‚न्धिनः स‚र्व‚स‚म्ब‚न्धित्वात् । स‚र्व‚ग‚त‚त्व‚म‚न्त‚रेण त‚द‚स‚म्भ‚वादिति भावः ।‚{\tiny $_{lb}$}‚ अनुप‚ल‚म्भेऽपि क‚थ‚म‚स‚र्व‚ग‚त‚त्वं निश्चेतुं श‚क्य‚त इत्याह--दृश्य‚त्वात्त‚त इत्य‚नुव‚र्त्त‚ते ।
	\pend% ending standard par
      ‚{\tiny $_{lb}$}‚

	  \pstart \leavevmode% starting standard par
	न‚नु चान्त‚रालेऽनुप‚ल‚म्भादिति युज्य‚ते व‚क्तुम् । त‚त्किमुक्त\textbf{म‚न्त‚रालादि}ति । स‚त्य‚मेत‚त् ।‚{\tiny $_{lb}$}‚ केव‚ल\textbf{म‚नुप‚ल‚म्भाद‚नु}प‚ल‚म्भ‚व्य‚व‚हारादित्य‚र्थ‚विव‚क्षित‚त्वाद‚दोषः । क्व‚चित्पु\textbf{न‚र‚न्त‚रालानुप‚ल‚म्भादि}ति‚{\tiny $_{lb}$}‚ पाठः । त‚त्र च य‚थायोगं स‚मासः ।
	\pend% ending standard par
      ‚{\tiny $_{lb}$}‚

	  \pstart \leavevmode% starting standard par
	न‚नु य‚दि शास्त्रात्म‚काऽऽग‚म‚कारेणास्य प‚र‚स्प‚र‚विरुद्ध‚स्व‚भावाव‚हं किञ्चिद् द्व‚य‚मुक्तं भ‚वेत्,‚{\tiny $_{lb}$}‚ विरुद्धाव्य‚भिचार्य‚व‚काशः । याव‚तेद‚मेव नास्तीत्याश‚ङ्काप‚नोद‚व्याजेनोप‚संह‚र‚न्नाह--\textbf{त‚त इति} ।‚{\tiny $_{lb}$}‚ \leavevmode\ledsidenote{\textenglish{77b/ms}} य‚त एवं शास्त्र‚कारेणैवेद‚ञ्चेद‚ञ्चाभ्युप‚ग‚तं \textbf{त‚त}स्त‚स्मात् \textbf{शास्त्र‚कारेणैव} प्र‚क‚र‚णा\textbf{त्क‚णादेन‚{\tiny $_{lb}$}‚ विरुद्धाव्य‚भिचारिणोऽव‚काशो द‚त्तः} । किं कृत्वा तेन त‚द‚व‚काशो द‚त्त इत्याह--\textbf{विरुद्धेति} ।‚{\tiny $_{lb}$}‚ प‚र‚स्प‚र‚साध्य‚विरुद्ध‚त्व‚व्याप्तिध‚र्मौ प्र‚क‚र‚णाद् युग‚प‚त्स‚र्व‚स‚म्ब‚न्धिस‚म्ब‚न्ध‚दृश्य‚त्वाख्या \edtext{}{\lemma{त्वाख्या}\Bfootnote{ख्यौ}}‚{\tiny $_{lb}$}‚ व‚क्तुर\edtext{}{\lemma{क्तुर}\Bfootnote{?}}दृश्य‚त्व‚विष‚य‚स्यानुप‚ल‚म्भ‚स्य विरुद्ध‚व्याप्त‚त्वाद् दृश्य‚त्वं विरुद्ध‚व्याप्त‚मिति तु द्र‚ष्ट‚व्य‚म् ।
	\pend% ending standard par
      ‚{\tiny $_{lb}$}‚

	  \pstart \leavevmode% starting standard par
	क‚थं पुन‚स्त‚योर्ध‚र्म‚योः प‚र‚स्प‚र‚विरुद्धार्थ‚व्याप्त‚त्वं विदुषा तेनैव‚मुच्येत येन त‚द‚व‚काश‚दानं‚{\tiny $_{lb}$}‚ त‚स्य क‚ल्प्य‚त इत्याह--\textbf{विरुद्ध‚व्याप्त‚त्व‚मि}ति । प‚र‚स्प‚र‚विरुद्ध‚स‚र्व‚ग‚त‚त्वास‚र्व‚ग‚त‚त्व‚ल‚क्ष‚णार्थ‚व्याप्त‚त्वं‚{\tiny $_{lb}$}‚ त‚योर्ध‚र्म‚योरित्य‚र्थात् । \textbf{अप‚श्य‚ता} अनालोच‚य‚ता भ्रान्त्या ताव‚द‚भिधायेति याव‚त् । इतिस्त‚स्मात्‚{\tiny $_{lb}$}‚ \textbf{न व‚स्तुनि} न व‚स्तुव‚ल‚प्र‚वृत्तेऽनुमानेऽस्य विरुद्धाव्य‚भिचारिणो जातिविव‚क्ष‚यैक‚व‚च‚न‚म् । \textbf{इति}रेव‚म‚र्थे ।‚{\tiny $_{lb}$}‚ तेनाय‚म‚र्थः--एव‚मुक्तेन प्र‚कारेणो\textbf{क्ताः} क‚थिता \textbf{हेत्वाभासाः,} कुत‚श्चित्साम्याद् हेतुव‚दाभासाः ॥
	\pend% ending standard par
      ‚{\tiny $_{lb}$}‚

	  \pstart \leavevmode% starting standard par
	साम्प्र‚तं हेतौ त‚दाभासे च क‚थिते तुल्य‚न्याय‚त‚याऽप‚र‚म‚पि स्वाभास‚स‚हितं किं नोक्त‚मिति‚{\tiny $_{lb}$}‚ ‚{\tiny $_{lb}$}‚ \leavevmode\ledsidenote{\textenglish{234/dm}}‚{\tiny $_{lb}$}‚ 
	  
	त्रिरूपो\edtext{}{\lemma{त्रिरूपो}\Bfootnote{त्रिल‚क्ष‚णो हेतु० \cite{dp-msC}}} हेतुरुक्तः । ताव‚ता \edtext{}{\lemma{ता}\Bfootnote{ताव‚तैवार्थ० \cite{dp-msB} \cite{dp-msD} \cite{dp-edP} \cite{dp-edH} \cite{dp-edE} \cite{dp-edN} ताव‚ता वा [[चा]]र्थ‚प्र‚तीतिसिद्धेरिति \cite{dp-msC}}}चार्थ‚प्र‚तीतिरिति न पृथ‚ग्दृष्टान्तो नाम‚{\tiny $_{lb}$}‚ साध‚नाव‚य‚वः क‚श्चित् । तेन नास्य ल‚क्ष‚णं पृथ‚गुच्य‚ते ग‚तार्थ‚त्वात् ॥ १२१ ॥‚{\tiny $_{lb}$}‚ 
	  
	त्रिरूपो हेतुरुक्तः, त‚त् किं दृष्टान्तैः ? ‚{\tiny $_{lb}$}‚ 
	  
	स्यादेत‚त्--ताव‚ता नार्थ‚प्र‚तीतिरित्याह--ताव‚ता \edtext{}{\lemma{ता}\Bfootnote{ताव‚ता वे[[चे]]ति \cite{dp-msC} ताव‚तैवेति \cite{dp-msB} \cite{dp-msD} \cite{dp-edP} \cite{dp-edH} \cite{dp-edE} \cite{dp-edN}}}चेति । उक्त‚ल‚क्ष‚णेनैव हेतुना‚{\tiny $_{lb}$}‚ भ‚व‚ति साध्य‚प्र‚तीतिः । \edtext{\textsuperscript{*}}{\lemma{*}\Bfootnote{त‚तः \cite{dp-msB}}}अतः स एव ग‚म‚कः । \edtext{\textsuperscript{*}}{\lemma{*}\Bfootnote{त‚तः नास्ति \cite{dp-msA} \cite{dp-edP} \cite{dp-edH}}}त‚त‚स्त‚द्व‚च‚न‚मेव साध‚न‚म् । न दृष्टान्तो‚{\tiny $_{lb}$}‚ नाम नाध‚न‚स्याव‚य‚वः । य‚त‚श्चायं नाव‚य‚वः, तेन नास्य दृष्टान्त‚स्य ल‚क्ष‚णं हेतुल‚क्ष‚णात् पृथ‚गुच्य‚ते । ‚{\tiny $_{lb}$}‚ 
	  
	क‚थं त‚र्हि हेतोर्व्याप्तिनिश्च‚यो य‚द्य‚दृष्टान्त‚को हेतुरिति चेत् । नोच्य‚ते हेतुर‚दृष‚टान्त‚क‚{\tiny $_{lb}$}‚ एव । अपि तु न हेतोः पृथ‚ग्दृष्टान्तो नाम हेत्व‚न्त‚र्भूत एव दृष्टान्तः । अत एवोक्तं नास्य‚{\tiny $_{lb}$}‚ ल‚क्ष‚णं पृथ‚गुच्य‚त \edtext{}{\lemma{त}\Bfootnote{इति नास्ति \cite{dp-msB} \cite{dp-msD}}}इति । न त्वेव‚मुक्त‚म्--नास्य ल‚क्ष‚ण‚मुच्य‚त इति ।\edtext{\textsuperscript{*}}{\lemma{*}\Bfootnote{इति नास्ति \cite{dp-msB}}}‚{\tiny $_{lb}$}‚ चोद‚य‚न्नाह--\textbf{न‚नु चे}ति । अथ क‚थं साध‚नाव‚य‚व‚त्वं हेतूनां येन \textbf{साध‚नाव‚य‚त्वाद् य‚था हेत‚व उक्ता}‚{\tiny $_{lb}$}‚ इत्युच्य‚ते ? त‚था हि साध्य‚ते निश्चीय‚ते साध्य‚म‚नेनेति \textbf{साध‚न‚म्} । प‚क्ष‚ध‚र्मान्व‚य‚{\tiny $_{lb}$}‚व्य‚तिरेक‚व‚ल्लिङ्ग‚मुच्य‚ते । त‚दाख्यानादेव च वाक्य‚म‚पि \textbf{साध‚न‚मु}च्य‚ते । न तु वाक्यात्साध्य‚{\tiny $_{lb}$}‚सिद्ध्युप‚योगित‚या हेतुर‚पि त‚देव त‚द्व‚च‚न‚म‚पीति । द्वेधाऽपि तं स‚मुदाय‚म‚पेक्ष्यास्याव‚य‚व‚त्वं येनैव‚{\tiny $_{lb}$}‚मुच्य‚त इति । न । अभिप्रायाप‚रिज्ञानात् । इहैवं पूर्व‚प‚क्ष‚वादी म‚न्य‚ते । प‚रोक्षोऽर्थो निश्चीय‚मानो‚{\tiny $_{lb}$}‚ हेतुदृष्टान्ताभ्यां निश्चीय‚ते । न तु हेतुनैव । अदृष्टान्त‚क‚स्य हेतोः साध्य‚साध‚नाश‚क्तेः ।‚{\tiny $_{lb}$}‚ त‚त‚श्च हेतुदृष्टान्त‚स‚मुदायः साध्य‚स्य साध‚न‚म् । त‚त्र य‚था स‚मुदायापेक्ष‚या हेतुरूपोऽव‚य‚व‚{\tiny $_{lb}$}‚ उक्त‚स्त‚दाभास‚श्च त‚था साध‚नाव‚य‚व‚त्वाविशेष‚त्वाद् दृष्टान्त‚रूपोऽप‚रोऽय‚म‚व‚य‚वः स‚प्र‚तिप‚क्षः‚{\tiny $_{lb}$}‚ किन्नोक्त इति । एव‚ञ्चोद‚य‚न्न‚य‚मेव साध‚नाव‚य‚व‚त्वाद्य‚भिधान‚मिति । \textbf{त‚त्प्र‚स‚ङ्गेन} त‚त्प्र‚स्तावेन ।
	\pend% ending standard par
      ‚{\tiny $_{lb}$}‚

	  \pstart \leavevmode% starting standard par
	\textbf{उक्तं} प‚क्ष‚ध‚र्मान्व‚य‚व्य‚तिरेकात्म‚कं \textbf{ल‚क्ष‚णं} य‚स्य तेन । \textbf{एव}कारेण दृष्टान्त‚स्य साक्षात्‚{\tiny $_{lb}$}‚ सिद्ध्युप‚योगितां निर‚स्य‚ति । य‚त एव\textbf{म‚तो} हेतोः, \textbf{स} इति हेतुः । निय‚मेन दृष्टान्त‚स्य‚{\tiny $_{lb}$}‚ ग‚म‚क‚रूप‚विर‚हं द्र‚ढ‚य‚ति । य‚तः साक्षात् सिद्धिरुप‚जाय‚ते, \textbf{स एव} ग‚म‚य‚ति प्र‚त्याय‚य‚ति साध्य‚{\tiny $_{lb}$}‚मिति कृत्वा उच्य‚ते, नान्य इति भावः । य‚स्माद् हेतोरेव ग‚म‚क‚त्वं \textbf{त‚त}स्त‚स्मात् \textbf{त‚स्य} हेतो‚{\tiny $_{lb}$}‚ स्त्रिरूप‚स्य \textbf{व‚च‚नं साध‚नं} साध‚नाभिधानं न दृष्टान्त‚स्येत्य‚र्थात् । \textbf{ताव‚ता चार्थ‚प्र‚तीति}रिति‚{\tiny $_{lb}$}‚ \textbf{मौल‚मितिश‚ब्द‚म‚पेक्ष्य न पृथ‚ग्दृष्टान्तो नाम साध‚नाव‚य‚व} इति मूलं व्याच‚ष्टे \textbf{न दृष्टान्त} इति ।‚{\tiny $_{lb}$}‚ अर्थ‚रूपो \textbf{दृष्टान्त}स्त‚द्व‚च‚नं वा, \textbf{न साध‚न‚स्या}र्थ‚रूप‚स्य व‚च‚न‚स्य वा । \textbf{नाव‚य‚वो} नैक‚देशः ।‚{\tiny $_{lb}$}‚ हेत्व‚न्त‚र‚र्भूत‚त्वं च व्याप्तिग्राह‚क‚प्र‚माणाधिक‚र‚ण‚त‚या हेतावुप‚योगात् । न त‚स्य स्व‚रू\leavevmode\ledsidenote{\textenglish{78a/ms}}पेण‚{\tiny $_{lb}$}‚ ‚{\tiny $_{lb}$}‚ \leavevmode\ledsidenote{\textenglish{235/dm}}‚{\tiny $_{lb}$}‚ 
	  
	य‚द्येवं हेतूप‚योगिनोऽपि ल‚क्ष‚णं व‚क्त‚व्य‚मेवेत्याह--ग‚तार्थ‚त्वात्--ग‚तोऽर्थः प्र‚योज‚न‚म‚भिधेयं‚{\tiny $_{lb}$}‚ वा य‚स्य दृष्टान्त‚ल‚क्ष‚ण‚स्य । \edtext{\textsuperscript{*}}{\lemma{*}\Bfootnote{त‚त् नास्ति \cite{dp-edE} \cite{dp-edN}}}त‚त् त‚था । त‚स्य भाव‚स्त‚त्त्व‚म् । त‚स्मात् । दृष्टान्त‚ल‚क्ष‚णं‚{\tiny $_{lb}$}‚ ह्य‚च्य‚ते दृष्टान्त‚प्र‚तीतिर्य‚था स्यात् । दृष्टान्त‚श्च हेतुल‚क्ष‚णादेवाव‚सितः । त‚तो दृष्टान्त‚{\tiny $_{lb}$}‚ल‚क्ष‚ण‚स्य य‚त् प्र‚योज‚न‚म्--दृष्टान्त‚प्र‚तीतिस्त‚द् ग‚तं निष्प‚न्न‚म् । अभिधेयं वा ग‚तं ज्ञातं\edtext{}{\lemma{ज्ञातं}\Bfootnote{ज्ञानं \cite{dp-msB} \cite{dp-edP} \cite{dp-edH} \cite{dp-edE} ग‚तं दृष्टा० \cite{dp-msA}}}‚{\tiny $_{lb}$}‚ दृष्टान्ताख्य‚म् ॥ ‚{\tiny $_{lb}$}‚ 
	  
	क‚थं \edtext{}{\lemma{थं}\Bfootnote{ग‚तार्थ‚मि० \cite{dp-edE}}}ग‚तार्थ‚त्व‚मित्याह-- ‚{\tiny $_{lb}$}‚ 
	  
	हेतोः स‚प‚क्ष एव स‚त्त्व‚म‚स‚प‚क्षाच्च स‚र्व‚तो व्याव‚त्ती \edtext{}{\lemma{त्ती}\Bfootnote{व्यावृत्त‚रूप० \cite{dp-msB} \cite{dp-edP} व्यावृत्तो रूप० \cite{dp-edH}}}रूप‚मुक्त‚म‚भेदेन ।‚{\tiny $_{lb}$}‚ पुन‚र्विशेषेण \edtext{}{\lemma{र्विशेषेण}\Bfootnote{०स्व‚भाव‚योर्ज‚न्म० \cite{dp-msB} \cite{dp-msD} \cite{dp-edP} \cite{dp-edH} \cite{dp-edE} \cite{dp-edN}}}कार्य‚स्व‚भाव‚योरुक्त‚ल‚क्ष‚ण‚योर्ज‚न्म‚त‚न्मात्रानुब‚न्धौ द‚र्श‚नीयाव‚क्तौ ।‚{\tiny $_{lb}$}‚ त‚च्च द‚र्श‚य‚ता-य‚त्र धूम‚स्त‚त्राग्निः, अस‚त्य‚ग्नौ न क्व‚चिद् धूमो य‚था म‚हान‚से‚{\tiny $_{lb}$}‚त‚र‚योः; य‚त्र कृत‚क‚त्वं त‚त्रानित्य‚त्व‚म्, अनित्य‚त्वाभावो कृत‚क‚त्वाऽस‚म्भ‚वो\edtext{}{\lemma{वो}\Bfootnote{स‚म्भ‚वोऽस्ति \cite{dp-msC}}}‚{\tiny $_{lb}$}‚ य‚था घ‚टाकाश‚योः--इति द‚र्श‚नीय‚म् । न ह‚य‚न्य‚था स‚प‚क्ष‚विप‚क्ष‚योः स‚द‚स‚त्वे‚{\tiny $_{lb}$}‚ य‚थोक्त‚प्र‚कारे श‚क्ये द‚र्श‚यितुम् । त‚त्कार्य‚तानिय‚मः\edtext{}{\lemma{मः}\Bfootnote{निय‚मं \cite{dp-msC}}} कार्य‚लिङ्ग‚स्य, स्व‚भाव‚{\tiny $_{lb}$}‚लिङ्ग‚स्य \edtext{}{\lemma{स्य}\Bfootnote{वा \cite{dp-msC}}}च स्व‚भावेन\edtext{}{\lemma{भावेन}\Bfootnote{स्व‚भाव‚व्याप्तिरिति \cite{dp-msC}}} व्याप्तिः । \edtext{\textsuperscript{*}}{\lemma{*}\Bfootnote{अस्मिन्न‚र्थे \cite{dp-msC}}}अस्मिंश्चार्थे द‚र्शिते द‚र्शित एव दृष्टान्तो‚{\tiny $_{lb}$}‚ भ‚व‚ति । एताव‚न्मात्र‚रूप‚त्वात् त‚स्येति\edtext{}{\lemma{स्येति}\Bfootnote{इति नास्ति \cite{dp-msC}}} ॥ १२२ ॥‚{\tiny $_{lb}$}‚ 
	  
	हेतो \edtext{}{\lemma{हेतो}\Bfootnote{रूप‚म‚भेदेनोक्तं सा० \cite{dp-msA} \cite{dp-msD} \cite{dp-edP} \cite{dp-edH} \cite{dp-edE}}}रूप‚मुक्त‚म‚भेदेन सामान्येन । साधार‚णं कार्य‚स्व‚भावानुप‚ल‚म्भानामेत‚ल्ल‚क्ष‚ण‚{\tiny $_{lb}$}‚मित्य‚र्थः । किं पुन‚स्त‚त् ? स‚प‚क्ष एव \edtext{}{\lemma{एव}\Bfootnote{य‚त् नास्ति \cite{dp-msB} \cite{dp-msC} \cite{dp-msD}}}य‚त् स‚त्त्व‚म्, विप‚क्षाच्च स‚र्व‚स्मात् व्यावृत्तिर्या ।‚{\tiny $_{lb}$}‚ रूप‚द्व‚य‚मेत‚द‚भेदेनोक्त‚म् । ‚{\tiny $_{lb}$}‚ 
	  
	न च सामान्य‚मुक्त‚म‚पि श‚क्यं ज्ञातुम् । अत‚स्त‚देव विशेष‚निष्ठं व‚क्त‚व्य‚म् । अतः‚{\tiny $_{lb}$}‚ पुन‚र‚पि विशेषेण विशेष‚व‚न्तौ ज‚न्म‚त‚न्मात्रानुब‚न्धौ द‚र्श‚नीयावुक्तौ । कार्य‚स्य ज‚न्म ज्ञात‚व्य‚{\tiny $_{lb}$}‚मुक्त‚म् । ज‚न्म‚नि हि \edtext{}{\lemma{हि}\Bfootnote{विज्ञाते \cite{dp-msA} \cite{dp-edP} \cite{dp-edH} \cite{dp-edE}}}ज्ञाते कार्य‚स्य स‚प‚क्ष एव स‚त्त्वं विप‚क्षाच्च स‚र्व‚स्माद् व्यावृत्तिर्ज्ञाता‚{\tiny $_{lb}$}‚ हेताव‚न्त‚र्भावः, साक्षात्साध्य‚सिद्ध्युप‚योगिताप्र‚स‚ङ्गात् । न च साऽस्य स‚म्भ‚विनीति । अनेका‚{\tiny $_{lb}$}‚र्थ‚त्वाद् वा हेतो \edtext{}{\lemma{हेतो}\Bfootnote{धातो}} \textbf{र्ग‚तं निष्प‚न्न‚मि}ति विवृत‚म् । ग‚त्य‚र्थानां ज्ञानार्थ‚त्वाद‚पि \textbf{ग‚तं ज्ञात}मित्य‚पि‚{\tiny $_{lb}$}‚ चोक्त‚म् ॥
	\pend% ending standard par
      ‚{\tiny $_{lb}$}‚

	  \pstart \leavevmode% starting standard par
	न‚नु च हेतुः स‚प‚क्ष‚वृत्तिना विप‚क्ष\edtext{}{\lemma{क्ष}\Bfootnote{क्षा}}वृत्तेनैव च रूपेण ग‚म‚कः । त‚च्चेत् क‚थितं किं‚{\tiny $_{lb}$}‚ विशेष‚ल‚क्ष‚ण‚क‚थ‚नेनेत्याह--\textbf{न चे}ति । \textbf{चोऽ}व‚धार‚णे हेतौ वा । \textbf{सामान्यं} साधार‚णं प्र‚क‚र‚णा‚{\tiny $_{lb}$}‚‚{\tiny $_{lb}$}‚ ‚{\tiny $_{lb}$}‚ \leavevmode\ledsidenote{\textenglish{236/dm}}‚{\tiny $_{lb}$}‚ 
	  
	भ‚व‚ति । स्व‚भाव‚स्य त‚न्मात्रानुब‚न्धो द‚र्श‚नीय उक्तः । त‚दिति साध‚न‚म् । त‚देव त‚न्मात्र‚म्‚{\tiny $_{lb}$}‚ साध‚न‚मात्र‚म्\edtext{}{\lemma{म्}\Bfootnote{मात्र‚स्यानु० \cite{dp-msC} \cite{dp-msD} साध‚न‚मात्र‚म् । साध‚न‚मात्र‚स्यानुब० \cite{dp-msB} \cite{dp-edN}}} । त‚स्यानुब‚न्धोऽनुग‚म‚न‚म्--साध‚न‚मात्र‚स्य\edtext{}{\lemma{स्य}\Bfootnote{मात्र‚भावे \cite{dp-msA} \cite{dp-msB} \cite{dp-msD} \cite{dp-edP} \cite{dp-edH} \cite{dp-edE} \cite{dp-edN}}} भावे भावः साध्य‚स्य । \edtext{\textsuperscript{*}}{\lemma{*}\Bfootnote{अथ साध‚न‚ध‚र्म‚मात्रानुब‚न्धः साध्य‚स्य त‚थापि तादात्म्यं न स्यादित्याह--\cite{dp-msD-n}}}त‚न्मात्र‚{\tiny $_{lb}$}‚भावित्व‚मेव हि साध्य‚स्य तादात्म्य‚म् । साध‚न‚स्य\edtext{}{\lemma{स्य}\Bfootnote{य‚दा साध‚न‚स्य च स्व० \cite{dp-msC}}} य‚दा स्व‚भावो ज्ञातो भ‚व‚ति, त‚दा स्व‚भाव‚{\tiny $_{lb}$}‚हेतोः--स‚प‚क्ष एव स‚त्त्व‚म्, विप‚क्षाच्च स‚र्व‚स्माद् व्यावृत्तिर्ज्ञाता भ‚व‚ति । ‚{\tiny $_{lb}$}‚ 
	  
	त‚देवं\edtext{}{\lemma{देवं}\Bfootnote{त‚देव \cite{dp-edE}}} सामान्य‚ल‚क्ष‚णं विशेषात्म‚कं ज्ञात‚व्यं नान्य‚था । त‚तो विशेष‚ल‚क्ष‚ण‚मुक्त‚म् । ‚{\tiny $_{lb}$}‚ 
	  
	किम‚तो\edtext{}{\lemma{तो}\Bfootnote{विशेष‚ल‚क्ष‚णात्--\cite{dp-msD-n}}} य‚दि नामैव\edtext{}{\lemma{नामैव}\Bfootnote{नामैवं त‚दित्याह \cite{dp-edE} प्रागुक्त‚न्यायेन--\cite{dp-msD-n}}} मित्याह--त‚च्च\edtext{}{\lemma{च्च}\Bfootnote{त‚त्र \cite{dp-msA} \cite{dp-msB} \cite{dp-edP} \cite{dp-edH} \cite{dp-edE}}} सामान्य‚ल‚क्ष‚णं द‚र्श‚यितुकामेन विशेष‚ल‚क्ष‚णं‚{\tiny $_{lb}$}‚ द‚र्श‚य‚तैवं\edtext{}{\lemma{तैवं}\Bfootnote{०तैवं च द‚र्श० \cite{dp-msB}}} द‚र्श‚नीय‚म्--इति स‚म्ब‚न्धः । य‚त्र धूम‚स्त‚त्राग्निरिति कार्य‚हेतोव्याप्तिर्द‚शिता ।‚{\tiny $_{lb}$}‚ व्याप्तिश्च कार्य‚कार‚ण‚भाव‚साध‚नात्\edtext{}{\lemma{नात्}\Bfootnote{प्र‚त्य‚क्षानुप‚ल‚म्भाभ्याम्--\cite{dp-msD-n}}} प्र‚माणान्निश्चीय‚ते । त‚तो य‚था म‚हान‚स इति द‚र्श‚नीय‚म् ।‚{\tiny $_{lb}$}‚ ल्ल‚क्ष‚णं \textbf{न श‚क्यं ज्ञातुमि}ति प्र‚वृत्त्युप‚योगित‚या न श‚क्य‚मिति म‚न्त‚व्य‚म् । न तु सामान्य‚ल‚क्ष‚ण‚स्य‚{\tiny $_{lb}$}‚ वाक्यात्प्र‚तीतिर्न भ‚व‚त्येव । \textbf{विशेष‚व‚न्तौ} च विशिष्टावित्य‚र्थः । एत‚देव विभ‚ज्य‚मान आचार्ये‚{\tiny $_{lb}$}‚\edtext{}{\lemma{आचार्ये}\Bfootnote{आह--\textbf{कार्य‚स्ये}}} त्यादि ।
	\pend% ending standard par
      ‚{\tiny $_{lb}$}‚

	  \pstart \leavevmode% starting standard par
	न‚नूक्तेऽप्य‚स्मिन् य‚दि सामान्य‚ल‚क्ष‚ण‚प्र‚तीतिर्नास्ति किम‚नेनोक्तेनापीति । आह—‚{\tiny $_{lb}$}‚\textbf{ज‚न्मे}ति । \textbf{हि}र्य‚स्माद‚र्थे । न‚नु स्व‚भाव‚हेतौ साध‚न‚स्व‚भाव‚ता साध्य‚स्य द‚र्श‚यितुं युज्य‚ते, त‚त्किं‚{\tiny $_{lb}$}‚ त‚न्मात्रानुब‚न्धो द‚र्श‚नीय उक्त इत्याह--\textbf{त‚न्मात्रे}ति । \textbf{हि}र्य‚स्मात् । अथ त‚न्मात्रानुव‚न्धे‚{\tiny $_{lb}$}‚ द‚र्शितेऽपि क‚थं सामान्य‚ल‚क्ष‚ण‚प्र‚तिप‚त्तिरित्याह--\textbf{साध‚न‚स्ये}ति । \textbf{त‚देव‚मि}त्यादिनोप‚संहारः ।
	\pend% ending standard par
      ‚{\tiny $_{lb}$}‚

	  \pstart \leavevmode% starting standard par
	अनुप‚ल‚ब्धेश्चान‚योरेवान्त‚र्भावान्न पृथ‚ग् विशिष्ट‚ल‚क्ष‚णाभिधान‚मित्य‚व‚सेय‚म् ।
	\pend% ending standard par
      ‚{\tiny $_{lb}$}‚

	  \pstart \leavevmode% starting standard par
	अथ किं हेतोर‚न्व‚य‚व्य‚तिरेकावेव ल‚क्ष‚णं येनैत‚द् द्व‚य‚मेव ल‚क्ष‚ण‚त‚या स्म‚र्य‚ते । न चैत‚त्,‚{\tiny $_{lb}$}‚ प‚क्ष‚ध‚र्म‚ताया अपि ल‚क्ष‚ण‚त्वाद‚भिहित‚त्वाच्चेति चेत् । स‚त्य‚म् । केव‚लं न हेतोः स‚र्व‚रूप‚मिह‚{\tiny $_{lb}$}‚ प्र‚क्रान्त‚म् । किन्तु य‚द्रूप‚प्र‚द‚र्श‚नेन दृष्टान्त‚प्र‚द‚र्श‚नं कृतं भ‚व‚ति त‚त्प्र‚कृत‚मिति ।
	\pend% ending standard par
      ‚{\tiny $_{lb}$}‚

	  \pstart \leavevmode% starting standard par
	य‚द्य‚प्ये\textbf{वं य‚दि नाम} सामान्य‚विशेष‚ल‚क्ष‚णाऽभिधान‚म‚तोऽभिधानात् \textbf{किं} भ‚व‚ति ? किमः‚{\tiny $_{lb}$}‚ क्षेपे प्र‚योगात्, न किञ्चिद् भ‚व‚तीत्य‚र्थः । \textbf{य \edtext{}{\lemma{य}\Bfootnote{त}} दित्युत्त‚र}म् । \textbf{चो} य‚स्मात् । व्याप्तिप्र‚द‚र्श‚ने‚{\tiny $_{lb}$}‚ऽपि क‚थं दृष्टान्ताख्यान‚म‚व‚त‚र‚तीत्याह--\textbf{व्याप्तिश्चेति} । हेत्व‚र्थ‚श्च‚कारः । कार्य‚कार‚ण‚भाव‚{\tiny $_{lb}$}‚साध‚नात्प्र‚माणान्निश्चीय‚तां व्याप्तिस्त‚थापि दृष्टान्तः क‚थं प्र‚कृतो भ‚व‚तींत्याह--\textbf{त‚त} इति ।‚{\tiny $_{lb}$}‚ दृष्टान्त‚म‚न्त‚रेण त‚देव प्र‚माणं न प्र‚व‚र्त्तेत । त‚द‚प्र‚वृत्तौ च सामान्य‚ल‚क्ष‚ण‚मेव न ज्ञायेतेति भावः ।‚{\tiny $_{lb}$}‚ ‚{\tiny $_{lb}$}‚ \leavevmode\ledsidenote{\textenglish{237/dm}}‚{\tiny $_{lb}$}‚ 
	  
	अस‚त्य‚ग्नौ न भ‚व‚त्येव धूम इति व्य‚तिरेको द‚र्शितः । स च य‚थेत‚र‚स्मिन्निति द‚र्श‚नीयः ।‚{\tiny $_{lb}$}‚ व‚ह्निनिवृत्तिर्हि धूम‚निवृत्तौ निय‚ता द‚र्श‚नीया । सा च म‚हान‚सादित‚र‚त्रेति द‚र्श‚नीया । ‚{\tiny $_{lb}$}‚ 
	  
	य‚त्र कृत‚क‚त्वं त‚त्रानित्य‚त्व‚मिति स्व‚भाव‚हेतोर्व्याप्तिर्द‚र्शिता । अनित्य‚त्वाभावे न भ‚व‚त्येव‚{\tiny $_{lb}$}‚ कृत‚क‚त्व‚मिति व्य‚तिरेको द‚र्शितः । व्याप्तेश्च साध‚कं प्र‚माणं \edtext{}{\lemma{माणं}\Bfootnote{साध‚र्म्य‚म् दृ० \cite{dp-msB}}}साध‚र्म्य‚दृष्टान्ते द‚र्श‚नीय‚म् ।‚{\tiny $_{lb}$}‚ प्र‚सिद्ध‚व्याप्तिक‚स्य च हेतोः साध्य‚निवृत्तौ निवृत्तिर्निय‚ता\edtext{}{\lemma{ता}\Bfootnote{निवृत्तिर्द‚र्श० \cite{dp-msA} \cite{dp-msB} \cite{dp-msD} \cite{dp-edP} \cite{dp-edH} \cite{dp-edN} \cite{dp-edE}}} द‚र्श‚नीया । त‚द‚व‚श्यं य‚था घ‚टे, य‚था‚{\tiny $_{lb}$}‚ आकाशे चेति द‚र्श‚नीय‚म् । ‚{\tiny $_{lb}$}‚ 
	  
	क‚स्मादेव‚मित्याह--न हीति । य‚स्माद‚न्य‚था सामान्य‚ल‚क्ष‚ण‚रूपे स‚प‚क्ष‚विप‚क्ष‚योः स‚द‚स‚त्त्वे‚{\tiny $_{lb}$}‚ य‚थोक्त‚प्र‚कारे इति निय‚ते--स‚प‚क्ष एव स‚त्त्व‚म्, विप‚क्षेऽस‚त्त्व‚मेवेति निय‚मो य‚थोक्त‚प्र‚कारः--ते‚{\tiny $_{lb}$}‚ न श‚क्ये द‚र्श‚यितुम् । विशेष‚ल‚क्ष‚णे हि द‚र्शिते य‚थोक्त‚प्र‚कारे\edtext{}{\lemma{कारे}\Bfootnote{य‚थोक्त‚ल‚क्ष‚णे स‚द० \cite{dp-msC}}} स‚द‚स‚त्त्वे द‚र्शिते भ‚व‚तः । न‚{\tiny $_{lb}$}‚ च विशेष‚ल‚क्ष‚ण‚म‚न्य‚था श‚क्यं द‚र्श‚यितुम् । ‚{\tiny $_{lb}$}‚ 
	  
	त‚स्य साध्य‚स्य कार्य‚म्--त‚त्कार्यं धूमः । त‚स्य भाव‚स्त‚त्कार्य‚ता । सैव निय‚मो‚{\tiny $_{lb}$}‚ अन्व‚य‚प्र‚द‚र्श‚नेन साध‚र्म्य‚दृष्टान्ताख्यान‚माख्याय व्य‚तिरेकोक्त्याऽपि वैध‚र्म्य‚दृष्टान्तौ \textbf{किं}‚{\tiny $_{lb}$}‚ \edtext{\textsuperscript{*}}{\lemma{*}\Bfootnote{दृष्टान्तोक्तिं}} द‚र्श‚यितुमाह--\textbf{अस‚त्य‚ग्नाविति} । व्य‚तिरेक‚प्र‚द‚र्श‚नेऽपि क‚थं दृष्टान्ताऽऽ‚{\tiny $_{lb}$}‚प‚त‚न‚मिति आश‚ङ्क्याह--\textbf{स चेति । चो} य‚स्माद‚र्थे । \textbf{इत‚र‚स्मिन्न}ग्निम‚त्प्र‚देशाद‚न्य‚स्मिन्‚{\tiny $_{lb}$}‚ म‚हाह्र‚दादौ । य‚त्र क‚योश्चिद् व्याप्य‚व्याप‚क‚भावो द‚र्श‚यित‚व्य‚स्त‚त्रास्तु दृष्टान्तोप‚निपातः,‚{\tiny $_{lb}$}‚ व्य‚तिरेकोप‚द‚र्श‚ने किं तेनेत्याह--\textbf{व‚ह्नीति । ही}ति य‚स्मात् । अत्रापि विध्योर्विप‚र्य‚येण व्याप्य‚{\tiny $_{lb}$}‚व्याप‚क‚भावो द‚र्श‚यित‚व्य इत्य‚र्थः । एत‚च्च व्य‚तिरेकाख्यानं दृष्टान्तोप‚द‚र्श‚न‚मुद्दिष्ट‚विष‚यं‚{\tiny $_{lb}$}‚ प्र‚योग‚म‚धिकृत्योक्त‚मित्य‚धिग‚न्त‚व्य‚म् ।
	\pend% ending standard par
      ‚{\tiny $_{lb}$}‚

	  \pstart \leavevmode% starting standard par
	\textbf{य‚त्रे}त्यादिना स्व‚भाव‚हेतुम‚धिकृत्य दृष्टान्त‚स्य ग‚तार्थ‚त्वं द‚र्श‚यितुमुप‚क्र‚म‚ते । एवं प्र‚द‚र्श‚य‚तां‚{\tiny $_{lb}$}‚ व्याप्तिः, दृष्टान्ताव‚तार‚क‚स्तु\edtext{}{\lemma{स्तु}\Bfootnote{र‚स्तु}}क‚थ‚मित्याह--\textbf{व्याप्तेश्चे}ति । \textbf{चो} य‚स्मात् । प्र‚माणाऽ‚{\tiny $_{lb}$}‚प्र‚द‚र्श‚ने व्याप्तेर‚सिद्धेः । त‚च्च प्र‚माण‚म्, अन्त‚रेणाधिक‚र‚णं न स‚म्भ‚व‚तीति भावः । य‚द्येवं‚{\tiny $_{lb}$}‚ व्य‚तिरेक \leavevmode\ledsidenote{\textenglish{78b/ms}} प्र‚द‚र्श‚ने कृतं दृष्टान्तेनेत्याह--\textbf{प्र‚सिद्धेति} । द‚र्श्य‚तामेव, त‚थापि दृष्टान्तोप‚{\tiny $_{lb}$}‚निपातः क‚थ‚मित्याह--\textbf{त‚दि}ति । य‚स्मादेव द‚र्श‚नीयं त‚त्त‚स्मात् । स‚च्चास‚च्च स‚द‚स‚ती त‚यो‚{\tiny $_{lb}$}‚र्भाव‚स्ते प्र‚द‚र्श‚न‚क्रियापेक्ष‚या च द्वितीयाद्विव‚च‚नान्त‚मेत‚त् । यादृश उक्तो \textbf{य‚थोक्तः प्र‚कारः}‚{\tiny $_{lb}$}‚ स्व‚रूपं य‚योस्ते त‚थोक्ते । अस्यैवार्थ‚माह--\textbf{निय‚ते} इति । स‚प‚क्षास‚प‚क्ष‚स‚द‚स‚त्त्व‚योरुक्त‚मेव प्र‚कारं‚{\tiny $_{lb}$}‚ स्प‚ष्ट‚य‚न्नाह--\textbf{स‚प‚क्ष} इति । \textbf{इति}र्निय‚म‚स्याकारं द‚र्श‚य‚ति । \textbf{ते} य‚थोक्त‚प्र‚कारे स‚द‚स‚त्त्वे ।‚{\tiny $_{lb}$}‚ क‚थं ते न श‚क्ये द‚र्शंयितुमित्याह--\textbf{विशेषे}ति । \textbf{हि}र्य‚स्माद‚र्थे । द‚र्श्य‚तां त‚र्हि विशेष‚ल‚क्ष‚ण‚म् ।‚{\tiny $_{lb}$}‚ दृष्टान्त‚स्य तु किमायात‚मित्याह--\textbf{न चे}ति । \textbf{चो} य‚स्माद‚र्थेऽव‚धार‚णे वा । य‚द्विशेष‚ल‚क्ष‚णं‚{\tiny $_{lb}$}‚ दृष्टान्त‚म‚न्त‚रेणाश‚क्य‚प्र‚द‚र्श‚नं त‚त्स्व‚रूपाख्यान‚म् । मूलं व्याख्यातुमाह--\textbf{त‚स्ये}त्यादि । विशेष‚{\tiny $_{lb}$}‚‚{\tiny $_{lb}$}‚ ‚{\tiny $_{lb}$}‚ \leavevmode\ledsidenote{\textenglish{238/dm}}‚{\tiny $_{lb}$}‚ 
	  
	य‚त‚स्त‚त्कार्य‚त‚या धूमो द‚ह‚ने निय‚तः । सोऽयं त‚त्कार्य‚तानिय‚मो विशेष‚ल‚क्ष‚ण‚रूपोऽन्य‚थ द‚र्श‚यितु‚{\tiny $_{lb}$}‚म‚श‚क्यः । स्व‚भाव‚लिङ्ग‚स्य च स्व‚भावेन साध्येन व्याप्तिर्विशेष‚ल‚क्ष‚ण‚रूपा न श‚क्या द‚र्श‚यितुम् ।‚{\tiny $_{lb}$}‚ य‚स्मात् कार्य‚कार‚ण‚भाव‚स्तादात्म्यं च म‚हान‚से घ‚टे च ज्ञात‚व्य‚म्, त‚स्माद् व्याप्तिसाध‚नं प्र‚माणं‚{\tiny $_{lb}$}‚ द‚र्श‚य‚ता \edtext{}{\lemma{ता}\Bfootnote{साध्य‚दृष्टा० \cite{dp-msB}}}साध‚र्म्य‚दृष्टान्तो द‚र्श‚नीयः । वैध‚र्म्य‚दृष्टान्त‚स्तु प्र‚सिद्धे त‚त्कार्य‚त्वे कार‚णाभावे‚{\tiny $_{lb}$}‚ कार्याभाव‚प्र‚तिप‚त्त्य‚र्थ‚म्\edtext{}{\lemma{म्}\Bfootnote{०त्त्य‚र्थः \cite{dp-edE}}} । त‚त एव नाव‚श्यं व‚स्तु भ‚व‚ति । कार‚णाभावे कार्याभावो व‚स्तु‚{\tiny $_{lb}$}‚न्य‚व‚स्तुनि वा भ‚व‚ति । त‚तो व‚स्त्व‚व‚स्तु वा वैध‚र्म्य‚दृष्टान्त इष्य‚ते । ‚{\tiny $_{lb}$}‚ 
	  
	त‚स्माद् दृष्टान्त‚म‚न्त‚रेण\edtext{}{\lemma{रेण}\Bfootnote{दृष्टान्त‚व्य‚तिरेकेण हे० \cite{dp-msA} \cite{dp-edP} \cite{dp-edH} \cite{dp-edE} \cite{dp-edN}}} न हेतोर‚न्व‚यो व्य‚तिरेको वा\edtext{}{\lemma{वा}\Bfootnote{वा न श‚क्यो \cite{dp-msA} \cite{dp-edP} \cite{dp-edH} \cite{dp-edE} \cite{dp-edN}}} श‚क्यो द‚र्श‚यितुम् अतो‚{\tiny $_{lb}$}‚ हेतुरूपाख्यानादेव हेतोर्व्याप्तिसाध‚न‚स्य\edtext{}{\lemma{स्य}\Bfootnote{०साध‚क‚स्य \cite{dp-edE}}} प्र‚माण‚स्य द‚र्श‚कः साध‚र्म्य‚दृष्टान्तः । ‚{\tiny $_{lb}$}‚ 
	  
	प्र‚सिद्ध‚व्याप्तिक‚स्य साध्याभावे हेत्व‚भाव\edtext{}{\lemma{भाव}\Bfootnote{भाव‚द‚र्श० \cite{dp-msC} \cite{dp-msD}}} प्र‚द‚र्श‚नाद्वैध‚र्म्य‚दृष्टान्त उपादेय इति \edtext{}{\lemma{इति}\Bfootnote{च एवार्थे--\cite{dp-msD-n}}}च‚{\tiny $_{lb}$}‚ द‚र्शितं भ‚व‚ति ।‚{\tiny $_{lb}$}‚ ल‚क्ष‚ण‚स्य स्व‚रूप‚माख्यायाख्यान्य‚थाप्र‚द‚र्श‚नाश‚क्य‚त्वं द‚र्श‚य‚न्नाह--\textbf{सोऽय‚मि}ति ।
	\pend% ending standard par
      ‚{\tiny $_{lb}$}‚

	  \pstart \leavevmode% starting standard par
	वैध‚र्म्य‚दृष्टान्त‚स्त‚र्हि न प्र‚द‚र्श‚नीय इत्याह--\textbf{वैध‚र्म्येति । तुः} साध‚र्म्य‚दृष्टान्ताद् वैध‚र्म्य‚{\tiny $_{lb}$}‚दृष्टान्तं भेद‚व‚न्तं द‚र्श‚य‚ति । एत‚च्च य‚दा व्य‚तिरेक‚मुखेण प्र‚योगः क्रिय‚ते, त‚त्कालाभिप्रायेणोच्य‚त‚{\tiny $_{lb}$}‚ इति द्र‚ष्ट‚व्य‚म् । न त्वेक‚स्मिन् प्र‚योगे द्व‚योप‚न्यासः स‚म्भ‚वी । एत‚च्च प्रागेव निर्लोठित‚म् ।‚{\tiny $_{lb}$}‚ य‚तः \textbf{प्र‚सिद्धे त‚त्कार्य‚त्वे कार‚णाभावे कार्याभाव‚प्र‚तिप‚त्त्य‚थं} द‚र्श‚नीयो वैध‚र्म्य‚दृष्टान्त‚स्त‚त एव‚{\tiny $_{lb}$}‚ त‚स्मादेव कार‚णात् । \textbf{नाव‚श्यं निय‚मे \add{न} व‚स्तु भ‚व‚ति} । व‚स्त्व‚प्य‚व‚स्त्व‚पि वैध‚र्म्य‚दृष्टान्तो‚{\tiny $_{lb}$}‚ भ‚व‚तीत्य‚र्थः । य‚दि हि क‚स्य‚चिद् विधिना क‚स्य‚चिद् विधिर्द‚र्श‚यित‚व्यः स्यात्, त‚दा ग‚म \edtext{}{\lemma{म}\Bfootnote{ग‚ग‚ना}}‚{\tiny $_{lb}$}‚ दिर‚स‚न् क‚थं क‚स्य‚चिदाधारः स्यादिति, न सिद्ध्येत्त‚दुप‚द‚र्श‚न‚म् । य‚दा त्वेक‚स्य व्याप्य‚स्याभावे‚{\tiny $_{lb}$}‚ अप‚र‚स्य व्याप‚क‚स्याभावो द‚र्श‚यित‚व्य‚स्त‚दा व्याप‚क‚स्याऽभावेऽव‚स्तुनि सुष्ठु स‚म्भ‚व‚ति । इत‚र‚थाऽ‚{\tiny $_{lb}$}‚भावेऽपि अस्य भावो विहितो भ‚वेत् । सोऽपि नेति चेत्, अय‚मेवाभाव इत्य‚भिप्रायः । वैध‚र्म्य‚{\tiny $_{lb}$}‚दृष्टान्त‚ग्र‚ह‚णं चैत‚देक‚ध‚र्माभावेनाप‚र‚ध‚र्माभावोप‚द‚र्श‚न‚विष‚योप‚ल‚क्ष‚णं द्र‚ष्ट‚व्य‚म‚न्य‚था साध‚र्म्य‚दृष्टा‚{\tiny $_{lb}$}‚न्ते त्व‚व‚श्यं व‚स्तु स आश्र‚य इष्ट इति स्यात् । त‚था च स‚ति ब‚ह्व‚स‚म‚ञ्ज‚सं स्यादिति । य‚तः‚{\tiny $_{lb}$}‚ कार‚णाभावात्कार्याभावो व‚स्तुन्य‚व‚स्तुनि च भ‚व‚ति \textbf{त‚तः} कार‚णाद् \textbf{व‚स्तु} म‚हाह्र‚दादि । \textbf{अव‚स्तु}‚{\tiny $_{lb}$}‚ आकाशादि । वाश‚ब्द‚स्तुल्य‚ब‚ल‚त्वं स‚मुच्चिनोति ।
	\pend% ending standard par
      ‚{\tiny $_{lb}$}‚

	  \pstart \leavevmode% starting standard par
	\textbf{त‚स्मादि}त्यादिना प्र‚कृत‚मुप‚संह‚र‚ति । व्याप्तिसाध‚क‚प्र‚माणाधिक‚र‚ण‚त्वाद् द‚र्श‚य‚तीति‚{\tiny $_{lb}$}‚ \textbf{द‚र्श‚कः} । य‚द्येवं साध्य‚साध‚क‚व्याप्तिप्र‚साध‚क‚प्र‚माण‚प्र‚द‚र्श‚क‚त्वात्साध‚र्म्य‚दृष्टान्त एवोपादेयो न‚{\tiny $_{lb}$}‚ वैध‚र्म्य‚दृष्टान्त इत्याह--\textbf{प्र‚सिद्धे}तिं । \textbf{वैध‚र्म्य‚दृष्टान्त उपादेय} इति योज‚यित्वा कुत उपादेय‚{\tiny $_{lb}$}‚ इत्याश‚ङ्कायां \textbf{प्र‚सिद्धेत्या}दिहेतुप‚दं योज्य‚म् । \textbf{साध्याभावे हेत्व‚भाव‚प्र‚द‚र्श‚नात्} । त‚त्रेति‚{\tiny $_{lb}$}‚ बुद्धिस्थ‚म् । य‚द्वा प्र‚द‚र्श्य‚तेऽस्मिन्निति प्र‚द‚र्श‚य‚तीति वा \textbf{प्र‚द‚र्श‚नो हेत्व‚भाव‚स्य प्र‚द‚र्श‚न} इति त‚था ।‚{\tiny $_{lb}$}‚ भाव‚प्र‚धान‚त्वान्निर्देश‚स्य ह‚त्व‚भाव‚प्र‚द‚र्श‚न‚त्वादित्य‚र्थः ।
	\pend% ending standard par
      ‚{\tiny $_{lb}$}‚‚{\tiny $_{lb}$}‚\textsuperscript{\textenglish{239/dm}}‚{\tiny $_{lb}$}‚
	  \bigskip
	  \begingroup
	

	  \pstart \leavevmode% starting standard par
	अस्मिंश्चार्थे द‚र्शिते द‚र्शित एव दृष्टान्तो भ‚व‚ति । योऽय‚म‚र्थो व्याप्तिसाध‚न‚प्र‚माण‚{\tiny $_{lb}$}‚प्र‚द‚र्श‚नः\edtext{}{\lemma{नः}\Bfootnote{प्र‚माण‚द‚र्शिनः \cite{dp-msA} \cite{dp-msB} \cite{dp-edP} \cite{dp-edH} \cite{dp-edN}}} क‚श्चिदुपादेयो निवृत्तिप्र‚द‚र्श‚न\edtext{}{\lemma{न}\Bfootnote{निवृत्तिप्र‚द‚र्श‚क‚श्च \cite{dp-msB}}} श्च--इत्य‚स्मिन्न‚र्थे \edtext{}{\lemma{र्थे}\Bfootnote{प्र‚द‚र्शिते \cite{dp-msA} \cite{dp-edP} \cite{dp-edH} \cite{dp-edE} \cite{dp-edN}}}द‚र्शिते द‚र्शितो दृष्टान्त‚{\tiny $_{lb}$}‚ \edtext{\textsuperscript{*}}{\lemma{*}\Bfootnote{दृष्टान्तः । क‚स्मादित्याह \cite{dp-edE}}}इत्याह--एताव‚न्मात्रं रूपं य‚स्य त‚स्य भाव‚स्त‚त्त्व‚म्, त‚स्मादिति । एताव‚देव हि रूपं‚{\tiny $_{lb}$}‚ दृष्टान्त‚स्य, य‚दुत व्याप्तिसाध‚न‚प्र‚माण\edtext{}{\lemma{माण}\Bfootnote{०प्र‚माण‚द‚र्श‚न‚त्वं \cite{dp-msA} \cite{dp-msB} \cite{dp-msD} \cite{dp-edP} \cite{dp-edH} \cite{dp-edE}}} प्र‚द‚र्श‚क‚त्वं नाम साध‚र्म्य‚दृष्टान्त‚स्य, प्र‚सिद्ध‚व्याप्तिक‚स्य‚{\tiny $_{lb}$}‚ च\edtext{}{\lemma{च}\Bfootnote{वा \cite{dp-msA} \cite{dp-msB} \cite{dp-msD} \cite{dp-edP} \cite{dp-edH} \cite{dp-edE} \cite{dp-edN}}} साध्य‚निवृत्तौ साध‚न‚निवृत्तिप्र‚द‚र्श‚क‚त्व‚मित्येत\edtext{}{\lemma{मित्येत}\Bfootnote{वैध‚र्म्य‚दृष्टान्त‚स्य एत‚त् नास्ति \cite{dp-msB} वैध‚र्म्य‚दृष्टान्त‚स्य नास्ति. \cite{dp-msD}}} द्वैध‚र्म्य‚दृष्टान्त‚स्य । \edtext{\textsuperscript{*}}{\lemma{*}\Bfootnote{त‚त् \cite{dp-msA} \cite{dp-msD} \cite{dp-edP} \cite{dp-edH} \cite{dp-edE} \cite{dp-edN} प्रागुक्त‚न्यायेन--\cite{dp-msD-n}}}एत‚च्च हेतुरूपाख्याना‚{\tiny $_{lb}$}‚देवाख्यात‚मिति किं दृष्टान्त‚ल‚क्ष‚णेन ? ॥
	\pend% ending standard par
       ‚{\tiny $_{lb}$}‚ 
	  \bigskip
	  \begingroup
	

	  \pstart \leavevmode% starting standard par
	एतेनैव दृष्टान्त‚दोषा अपि निर‚स्ता भ‚व‚न्ति ॥ १२३ ॥
	\pend% ending standard par
      
	  \endgroup
	‚{\tiny $_{lb}$}‚ 

	  \pstart \leavevmode% starting standard par
	एतेनैव च हेतुरूपाख्यानाद् दृष्टान्त‚त्व‚प्र‚द‚र्श‚नेन दृष्टान्त‚दोषा\edtext{}{\lemma{दोषा}\Bfootnote{दृष्टान्त‚स्य दोषाः \cite{dp-msA} \cite{dp-edP} \cite{dp-edH} \cite{dp-edE} \cite{dp-edN}}} दृष्टान्ताभासाः\edtext{}{\lemma{दृष्टान्ताभासाः}\Bfootnote{निर‚स‚न‚द्वारा क‚थिताः--\cite{dp-msD-n}}} क‚थिता‚{\tiny $_{lb}$}‚ भ‚व‚न्ति । त‚थाहि--पूर्वोक्त‚सिद्ध‚ये य उपादीय‚मानोऽपि\edtext{}{\lemma{मानोऽपि}\Bfootnote{०पि न स‚म‚र्थः \cite{dp-msB}}} दृष्टान्तो न स‚म‚र्थः स्व‚कार्यं‚{\tiny $_{lb}$}‚ साध‚यितुं स दृष्टान्त‚दोष इति साम‚र्थ्यादुक्तं\edtext{}{\lemma{र्थ्यादुक्तं}\Bfootnote{०र्थ्यादित्येत‚दुक्तं \cite{dp-msB}}} भ‚व‚ति ॥
	\pend% ending standard par
      
	  \endgroup
	‚{\tiny $_{lb}$}‚

	  \pstart \leavevmode% starting standard par
	\textbf{अस्मिंश्चेत्या}दि मूल‚म‚नूद्य व्याच‚ष्टे \textbf{योऽय‚मि}ति । पूर्व‚व‚त्प्र‚द‚र्श‚न‚श‚ब्द‚स्य व्युत्प‚त्तिः,‚{\tiny $_{lb}$}‚ स‚मास‚श्च क‚र्त्त‚व्यः । प्र‚सिद्धायां व्याप्तौ साध्या\leavevmode\ledsidenote{\textenglish{79a/ms}}भावे हेतोर्निवृत्तिप्र‚द‚र्श‚न इत्य‚र्थो द्र‚ष्ट‚व्यः ।‚{\tiny $_{lb}$}‚ \textbf{एताव‚देवै}ताव‚न्मात्र‚म् । अन्य‚द‚प्य‚स्य रूप‚म‚स्तीत्याह--\textbf{एताव‚देवे}ति । हिर्य‚स्मादेत‚त् प‚रिमाण‚{\tiny $_{lb}$}‚म‚स्येति त‚था । \textbf{एव}कारेणान्य‚स्य ताद्रूप्य‚निरासो दृढीकृतः । किं त‚द्रूप‚मित्याह--\textbf{य‚दुते}ति ।‚{\tiny $_{lb}$}‚ निपात‚स‚मुदाय‚श्चायं य‚देत‚दित्य‚स्यार्थे ॥
	\pend% ending standard par
      ‚{\tiny $_{lb}$}‚

	  \pstart \leavevmode% starting standard par
	येषु दृष्टान्त‚त्वेनोपात्तेषु स्व‚कार्य‚का\edtext{}{\lemma{का}\Bfootnote{क}}र‚णासाम‚र्थ्ये दोषः स‚म्भ‚व‚ति ते कुत‚श्चित्सामा‚{\tiny $_{lb}$}‚न्याद् दृष्टान्त‚व‚दाभास‚माना दृष्टान्ताभासा भ‚व‚न्तीति अर्थं न्याय‚माश्रित्याह--\textbf{दृष्टान्त‚दोषा‚{\tiny $_{lb}$}‚ दृष्टान्ताभासाः क‚थिता भ‚व‚न्तीति} । य‚स्मात्ते दृष्टान्ताभास‚त्वे\add{न} क‚थिता भ‚व‚न्त्य‚त एव दृष्टा‚{\tiny $_{lb}$}‚न्त‚त्वेन \textbf{निर‚स्ता भ‚व‚न्ती}त्य‚त एव मूल‚म‚र्थ‚तो व्याख्यात‚मित्य‚व‚ग‚न्त‚व्य‚म् । दृष्टान्त‚त‚त्त्व‚प्र‚द‚र्श‚नेन‚{\tiny $_{lb}$}‚ य‚था दृष्टान्ताभासा\edtext{}{\lemma{दृष्टान्ताभासा}\Bfootnote{स}}क‚थ‚नं कृतं भ‚व‚ति त‚था द‚र्श‚यितुं \textbf{त‚था ही}त्यादिनोप‚क्र‚म‚ते । \textbf{पूर्वोक्त}स्य‚{\tiny $_{lb}$}‚ ज‚न्म‚त‚न्मात्रानुब‚न्ध‚स्य \textbf{सिद्ध‚ये} निश्च‚याय । \textbf{स्व‚कार्यं} साध‚न‚स‚म्भ‚वे साध्य‚प्र‚द‚र्श‚न‚स‚म्भ‚व‚ल‚क्ष‚ण‚म्,‚{\tiny $_{lb}$}‚ सिद्ध‚व्याप्तिक‚स्य च हेतोः साध्याभावे साध‚न‚निवृत्तिप्र‚द‚र्श‚न‚ल‚क्ष‚णं च \textbf{साध‚यितुं य उपादीय‚{\tiny $_{lb}$}‚मानोऽपि न स‚म‚र्थः, स दृष्टान्त‚दोषो} दुष्टो दृष्टान्तः, दृष्टान्ताभास इति याव‚त् । \textbf{साम‚र्थ्यात्}‚{\tiny $_{lb}$}‚ स्व‚कार्याकार‚ण‚ल‚क्ष‚णात् ॥
	\pend% ending standard par
      ‚{\tiny $_{lb}$}‚‚{\tiny $_{lb}$}‚\textsuperscript{\textenglish{240/dm}}‚{\tiny $_{lb}$}‚
	  \bigskip
	  \begingroup
	

	  \pstart \leavevmode% starting standard par
	दृष्टान्ताभासामुदाह‚र‚ति--
	\pend% ending standard par
       ‚{\tiny $_{lb}$}‚ 
	  \bigskip
	  \begingroup
	

	  \pstart \leavevmode% starting standard par
	य‚था नित्यः श‚ब्दोऽमूर्त्त‚त्वात् । क‚र्म‚व‚त् प‚र‚माणुव‚द् घ‚ट‚व‚दिति । एते‚{\tiny $_{lb}$}‚ दृष्टान्ताभासाः साध्य‚साध‚न‚ध‚र्मोभ‚य‚विक‚लाः ॥ १२४ ॥
	\pend% ending standard par
      
	  \endgroup
	‚{\tiny $_{lb}$}‚ 

	  \pstart \leavevmode% starting standard par
	य‚था \edtext{}{\lemma{था}\Bfootnote{य‚थेति नित्यः \cite{dp-msA} \cite{dp-msB} \cite{dp-edP} \cite{dp-edH} \cite{dp-edE} \cite{dp-edN}}}नित्यः श‚ब्द इति \edtext{}{\lemma{इति}\Bfootnote{इति नित्य‚त्वे साध्ये श‚ब्द‚स्यामू० \cite{dp-msD} \cite{dp-msB}}}श‚ब्द‚स्य नित्य‚त्वे साध्येऽमूर्त‚त्वादिति हेतुः । साध‚र्म्येण‚{\tiny $_{lb}$}‚ क‚र्म‚व‚त् प‚र‚माणुव‚द् घ‚ट‚व‚दित्येते दृष्टान्ता उप‚न्य‚स्ताः । एते च दृष्टान्त‚दोषाः । साध्यं च‚{\tiny $_{lb}$}‚ साध‚नं चोभ‚यं चेति\edtext{}{\lemma{चेति}\Bfootnote{च । तैर्वि० \cite{dp-msC}}} । तैर्विक‚लाः । साध्य‚विक‚लं क‚र्म, त‚स्याऽनित्य‚त्वात् । साध‚न‚विक‚लः‚{\tiny $_{lb}$}‚ प‚र‚माणुः, मूर्त्त‚त्वात् प‚र‚माणूनाम । \edtext{\textsuperscript{*}}{\lemma{*}\Bfootnote{अस‚र्व‚ग‚तं द्र० \cite{dp-msA} \cite{dp-msB} \cite{dp-edP} \cite{dp-edH} \cite{dp-edN}}}अस‚र्व‚ग‚त‚द्र‚व्य‚प‚रिमाणं मूर्त्तिः । अस‚र्व‚ग‚ताश्च द्र‚व्य‚रूपाश्च‚{\tiny $_{lb}$}‚ प‚र‚माण‚वः । नित्यास्तु वैशेषिकैरिष्य‚न्ते । त‚तो न साध्य‚विक‚लः\edtext{}{\lemma{लः}\Bfootnote{साध्य‚विक‚लाः \cite{dp-msC} \cite{dp-msD}}} । घ‚ट‚स्तूभ‚य‚विक‚लः,‚{\tiny $_{lb}$}‚ अनित्य‚त्वान्मूर्त्त‚त्वाच्च घ‚ट‚स्येति\edtext{}{\lemma{स्येति}\Bfootnote{घ‚ट‚स्य । \cite{dp-edE}}} ॥
	\pend% ending standard par
       ‚{\tiny $_{lb}$}‚ 
	  \bigskip
	  \begingroup
	

	  \pstart \leavevmode% starting standard par
	त‚था स‚न्दिग्ध‚साध्य‚ध‚र्माद‚य‚श्च । य‚था रागादिमान‚यं व‚च‚नाद्र‚थ्या‚{\tiny $_{lb}$}‚पुरुष‚व‚त् । म‚र‚ण‚ध‚र्माऽयं पुरुषो रागादिम‚त्वाद्र‚थ्यापुरुष‚व‚त् । अस‚र्व‚ज्ञोऽयं‚{\tiny $_{lb}$}‚ रागादिम‚त्वाद्र‚थ्यापुरुष‚व‚दिति ॥ १२५ ॥
	\pend% ending standard par
      
	  \endgroup
	‚{\tiny $_{lb}$}‚ 

	  \pstart \leavevmode% starting standard par
	त‚था\edtext{}{\lemma{था}\Bfootnote{त‚थेति \cite{dp-edE}}} संदिग्धः साध्य‚ध‚र्मो य‚स्मिन् स संदिग्ध‚साध्य‚ध‚र्मः । स आदिर्येषां ते त‚थोक्ताः ।‚{\tiny $_{lb}$}‚ संदिग्ध‚साध्य‚ध‚र्मः । संदिग्ध‚साध‚न‚ध‚र्मः संदिग्धोभ‚यः\edtext{}{\lemma{यः}\Bfootnote{०भ‚य‚ध‚र्मः । \cite{dp-msC}}} ।
	\pend% ending standard par
      
	  \endgroup
	‚{\tiny $_{lb}$}‚

	  \pstart \leavevmode% starting standard par
	\textbf{दृष्टान्ताभासानि}त्यादि \textbf{प‚र‚माणूनामि}त्येत‚द‚न्तं सुग‚म‚म् ।
	\pend% ending standard par
      ‚{\tiny $_{lb}$}‚

	  \pstart \leavevmode% starting standard par
	क‚थं प‚र‚माण‚वः साध‚न‚विक‚ला न साध्य‚विक‚ला इत्याह--अ\textbf{स‚र्वे}ति । \textbf{प‚रिमाणं} मान‚{\tiny $_{lb}$}‚व्य‚व‚हार‚कार‚ण‚म् । त‚च्च गुण‚त्वाद् द्र‚व्याश्र‚यीति । \textbf{द्र‚व्य}ग्र‚ह‚णेन वास्त‚वं रूप‚म‚नूदित‚म् । त‚त्र‚{\tiny $_{lb}$}‚ य‚दि द्र‚व्य‚प‚रिमाणं मूर्त्तिरित्येव ताव‚दुच्य‚ते त‚दाऽऽकाशादेर‚पि द्र‚व्य‚स्य प‚र‚म‚म‚ह‚त्त्व‚नाम‚धेयं‚{\tiny $_{lb}$}‚ प‚रिमाण‚म‚स्तीति मूर्त्त‚त्वं प्र‚स‚ज्येत । अत‚स्त‚न्निवृत्त्य‚र्थं द्र‚व्ये विशेष‚ण‚म\textbf{स‚र्व‚ग‚त}ग्र‚ह‚ण‚म् । त‚त्र प‚र‚माणोः‚{\tiny $_{lb}$}‚ प‚रिमाणं भ‚व‚त्य‚स‚र्व‚ग‚त‚स्य द्र‚व्य‚स्येति द‚र्श‚य‚न्नाह--\textbf{अस‚र्व‚ग‚ताश्चे}ति । स‚म‚वायिकार‚णं द्र‚व्यं‚{\tiny $_{lb}$}‚ गुण‚व‚द्वेति द्र‚व्य‚ल‚क्ष‚ण‚योगाद् द्र‚व्य‚रूपः । \textbf{च}कारौ पूर्वापेक्ष‚या एक‚विष‚य‚त्व‚म‚न‚योः स‚मुच्चिनुतः ।‚{\tiny $_{lb}$}‚ त‚त् पुनः प‚र‚माणोः प‚रिमाणं पारिमाण्ड‚ल‚संज्ञ‚कं ज्ञात‚व्य‚म् । इय‚ता साध‚न‚वैक‚ल्यं द‚र्शित‚म् ।
	\pend% ending standard par
      ‚{\tiny $_{lb}$}‚

	  \pstart \leavevmode% starting standard par
	साध्यावैक‚ल्यं द‚र्श‚य‚न्नाह--\textbf{नित्यास्त्वि}ति । तुनेनार्थ‚मान्त‚रेण \edtext{}{\lemma{रेण}\Bfootnote{?}} विशिन‚ष्टि । द्र‚व्य‚{\tiny $_{lb}$}‚गुण‚क‚र्म‚सामान्य‚विशेष‚स‚म‚वायात्म‚कैः प‚दार्थ‚विशेषैर्व्य‚व‚ह‚र‚न्तीति \textbf{वैशेषिकाः,} रूढ\edtext{}{\lemma{रूढ}\Bfootnote{ढे}} श्चाभ्यु‚{\tiny $_{lb}$}‚प‚ग‚त\textbf{क‚णाद‚शास्त्रा} एवोच्य‚न्ते । अथ\add{वा} ष‚ट्प‚दार्थीप्र‚तिपाद‚क‚त‚या विशिष्य‚ते त‚द‚न्य‚स्मा‚{\tiny $_{lb}$}‚च्छास्त्रादिति विशेषः \textbf{काणादं शास्त्रं} विव‚क्षित‚म् । \edtext{\textsuperscript{*}}{\lemma{*}\Bfootnote{तुल‚ना--त‚द‚धीते त‚द्वेद \href{http://sarit.indology.info/?cref=Pā.4.3.59}{पाणिनि ४. २. ५६}.}}त‚द् विद‚न्त्य‚धीय‚ते वा इति \textbf{वैशेषिका‚{\tiny $_{lb}$}‚स्तैरिष्य‚न्त} इति व‚च‚न‚व्य‚क्त्या चेष्टिमात्र‚मेत‚न्न पुन‚र‚त्र प्र‚माण‚म‚स्तीति सूच‚य‚ति ।
	\pend% ending standard par
      ‚{\tiny $_{lb}$}‚\footnote{दिति साध्य० \cite{dp-msB} \cite{dp-edP} \cite{dp-edH} \cite{dp-edE} \cite{dp-edN}}‚{\tiny $_{lb}$}‚\textsuperscript{\textenglish{241/dm}}‚{\tiny $_{lb}$}‚
	  \bigskip
	  \begingroup
	

	  \pstart \leavevmode% starting standard par
	उदाह‚र‚ण‚म्--रागादिमानिति रागादिम‚त्त्वं साध्य‚म् । व‚च‚नाद् इति हेतुः । र‚थ्या‚{\tiny $_{lb}$}‚पुरुष‚व‚दिति दृष्टान्ते\edtext{}{\lemma{दृष्टान्ते}\Bfootnote{दृष्टान्तः रा० \cite{dp-msA} \cite{dp-msB} \cite{dp-msD} \cite{dp-edP} \cite{dp-edH} \cite{dp-edE} \cite{dp-edN}}} रागादिम‚त्त्वं स‚न्दिग्ध‚म् । म‚र‚णं ध‚र्मोऽस्येति म‚र‚ण‚ध‚र्मा । त‚स्य भावो‚{\tiny $_{lb}$}‚ म‚र‚ण‚ध‚र्म‚त्वं साध्य‚म् । अयं पुरुष इति ध‚र्मी । रागादिम‚त्त्वादिति हेतुः । र‚थ्यापुरुषे दृष्टान्ते‚{\tiny $_{lb}$}‚ स‚न्दिग्धं साध‚न‚म् । साध्यं तु निश्चितं म‚र‚ण‚ध‚र्म‚त्व‚मिति । अस‚र्व‚ज्ञ इति । अस‚र्व‚ज्ञ‚त्वं‚{\tiny $_{lb}$}‚ साध्य‚म् । रागादिम‚त्त्वादिति हेतुः । त‚दुभ‚य‚म‚पि र‚थ्यापुरुषे दृष्टान्ते स‚न्दिग्ध‚म् । अस‚र्व‚ज्ञ‚त्वं‚{\tiny $_{lb}$}‚ रागादिम‚त्त्वं चेति ॥
	\pend% ending standard par
       ‚{\tiny $_{lb}$}‚ 
	  \bigskip
	  \begingroup
	

	  \pstart \leavevmode% starting standard par
	\edtext{\textsuperscript{*}}{\lemma{*}\Bfootnote{त‚था नास्ति \cite{dp-msB} \cite{dp-edP} \cite{dp-edH} \cite{dp-edE} \cite{dp-edN}}}त‚थाऽन‚न्व‚योऽप्र‚द‚र्शितान्व‚य‚श्च । य‚था - यो व‚क्ता स रागादिमान्,‚{\tiny $_{lb}$}‚ इष्ट‚पुरुष‚व‚त् । अनित्यः श‚ब्दः कृत‚क‚त्वाद् घ‚ट‚व‚दिति ॥ १२६ ॥
	\pend% ending standard par
      
	  \endgroup
	‚{\tiny $_{lb}$}‚ 

	  \pstart \leavevmode% starting standard par
	त‚थाऽन‚न्व‚य इति । य‚स्मिन् दृष्टान्ते साध्य‚साध‚न‚योः स‚म्भ‚व‚मात्रं दृश्य‚ते, न तु साध्येन‚{\tiny $_{lb}$}‚ व्याप्तो हेतुः, सोऽन‚न्व‚यः । अप्र‚द‚र्शितान्व‚य‚श्च--य‚स्मिन् दृष्टान्ते विद्य‚मानोऽप्य‚न्व‚यो न‚{\tiny $_{lb}$}‚ प्र‚द‚र्शितो व‚क्त्रा सोऽप्र‚द‚र्शितान्व‚यः ।
	\pend% ending standard par
       ‚{\tiny $_{lb}$}‚ 

	  \pstart \leavevmode% starting standard par
	अन‚न्व‚य‚मुदाह‚र‚ति--य‚थेति । यो व‚क्तेति व‚क्तृत्व‚म‚नूद्य स रागादिमानिति रागादि‚{\tiny $_{lb}$}‚म‚त्त्वं\edtext{}{\lemma{त्त्वं}\Bfootnote{०म‚त्त्वे वि० \cite{dp-msA}}} विहित‚म् । त‚तो व‚क्तृत्व‚स्य रागादिम‚त्त्वं \edtext{}{\lemma{त्त्वं}\Bfootnote{वे}} प्र‚तिनिय‚मः । तेन व्याप्तिरुक्ता । इष्ट‚{\tiny $_{lb}$}‚पुरुष‚व‚दिति । इष्ट‚ग्र‚ह‚णेन प्र‚तिवाद्य‚पि \edtext{}{\lemma{पि}\Bfootnote{गृह्य‚ते \cite{dp-msA} \cite{dp-edP} \cite{dp-edH} \cite{dp-edE} \cite{dp-edN}}}संगृह्य‚ते वाद्य‚पि । तेन व‚क्तृत्व‚रागादिम‚त्त्व‚योः‚{\tiny $_{lb}$}‚ स‚त्त्व‚मात्र‚मिष्टे पुरुषे सिद्ध‚म् । व्याप्तिस्तु न सिद्धा । तेनाऽन‚न्व‚यो दृष्टान्त इति ।
	\pend% ending standard par
       ‚{\tiny $_{lb}$}‚ 

	  \pstart \leavevmode% starting standard par
	अनित्यः श‚ब्द इत्य‚नित्य‚त्वं साध्य‚म् । कृत‚क‚त्वादिति हेतुः । घ‚ट‚व‚दिति\edtext{}{\lemma{दिति}\Bfootnote{०दित्य‚त्र दृ० \cite{dp-msA} \cite{dp-edP} \cite{dp-edH} \cite{dp-edE}}} दृष्टान्ते‚{\tiny $_{lb}$}‚ न प्र‚द‚र्शितोऽन्व‚यः ।
	\pend% ending standard par
      
	  \endgroup
	‚{\tiny $_{lb}$}‚

	  \pstart \leavevmode% starting standard par
	\textbf{त‚त} इत्यादि \textbf{म‚र‚ण‚ध‚र्म‚त्व‚मि}त्येत‚द‚न्तं सुज्ञान‚म् ।
	\pend% ending standard par
      ‚{\tiny $_{lb}$}‚

	  \pstart \leavevmode% starting standard par
	न‚नु म‚र‚ण‚ध‚र्म‚त्व‚म‚प्य‚स्य क‚थं निश्चितं येन हेतुरेव स‚न्दिग्ध उच्य‚त इति चेत् । स‚त्य‚म् ।‚{\tiny $_{lb}$}‚ केव‚लं प्र‚सिद्धिस‚माश्र‚येणैव‚मुक्त‚मित्य‚व‚सेय‚म् । अन्वीय‚मान‚त्वं साध‚न‚स्य साध्येनान्व‚यः । स‚{\tiny $_{lb}$}‚ च प्र‚तिब‚न्ध‚साध‚क‚प्र‚माणाक्षेपात् प्र‚सिद्ध्य‚ति । य‚त्र तु त‚न्नास्ति केव‚लं स‚म्भ‚व‚मात्रं साह‚च‚र्य‚मात्रं‚{\tiny $_{lb}$}‚ दृश्य‚त इत्य‚भिप्रायः ॥
	\pend% ending standard par
      ‚{\tiny $_{lb}$}‚

	  \pstart \leavevmode% starting standard par
	\textbf{व‚क्तृत्व‚स्य} हेतो \textbf{रागादिम‚त्त्वे} \leavevmode\ledsidenote{\textenglish{79b/ms}} साध्ये \textbf{प्र‚तिनिय‚मः} प्र‚तिनिय‚त‚त्व‚मुक्त‚मिति शेषः ।‚{\tiny $_{lb}$}‚ \textbf{तेन} साध्य‚निय‚त‚त्वेन \textbf{व्याप्तिर}न‚यो\textbf{रुक्ता} प्र‚द‚र्शिता । एवंविधानुवाद‚विध्युप‚द‚र्श‚ने त‚था प्र‚तीते‚{\tiny $_{lb}$}‚राह‚त्योद‚याद् \textbf{व्याप्तिरुक्ते}त्युक्त‚म् । न त्व‚साव‚न‚योर्वाक्य‚तोऽस्ति । य‚तो द्व‚योर‚पि वादिप्र‚ति‚{\tiny $_{lb}$}‚वादिनोदृष्टान्त‚त्वेन स‚ङ्ग्र‚हः \textbf{तेन} हेतुना \textbf{स‚त्त्व‚मात्रं} स‚म्भ‚व‚मात्रं साह‚च‚र्य‚मात्र‚मिति याव‚त् ।‚{\tiny $_{lb}$}‚ अस्ति चेष्टिः स‚ह‚मात्रोप‚द‚र्श‚न‚म‚दोषाव‚ह‚मित्याह--\textbf{व्याप्तिस्त्वि}ति । \textbf{तुः} स‚त्त्व‚मात्राद् व्याप्तिं‚{\tiny $_{lb}$}‚ भिन‚त्ति । \textbf{न सिद्धा न} प्र‚माण‚निश्चिता । अन‚योर्व्याप्तिसाध‚क‚प्र‚माणाभावादिति भावः ।
	\pend% ending standard par
      ‚{\tiny $_{lb}$}‚

	  \pstart \leavevmode% starting standard par
	अप्र‚द‚र्शितान्व‚य‚मुदाह‚र‚न्नाह--\textbf{अनित्य} इति ।
	\pend% ending standard par
      ‚{\tiny $_{lb}$}‚‚{\tiny $_{lb}$}‚\textsuperscript{\textenglish{242/dm}}‚{\tiny $_{lb}$}‚
	  \bigskip
	  \begingroup
	

	  \pstart \leavevmode% starting standard par
	इह य‚द्य‚पि कृत‚क‚त्वेन घ‚ट‚स‚दृशः श‚ब्द‚स्त‚थापि नानित्य‚त्वेनापि स‚दृशः प्र‚त्येतुं श‚क्य‚ते\edtext{}{\lemma{ते}\Bfootnote{श‚क्योऽति० \cite{dp-msA} \cite{dp-edP} \cite{dp-edH} \cite{dp-edE} \cite{dp-edN}}}‚{\tiny $_{lb}$}‚ऽतिप्र‚स‚ङ्गात्\edtext{}{\lemma{ङ्गात्}\Bfootnote{पाक्योऽपि प्राप्तः--\cite{dp-msD-n}}} । य‚दि तु कृत‚क‚त्व‚म् अनित्य\edtext{}{\lemma{अनित्य}\Bfootnote{अनित्य‚त्व \cite{dp-msA} \cite{dp-msC} \cite{dp-edP} \cite{dp-edH} \cite{dp-edE} \cite{dp-edN}}} स्व‚भावं\edtext{}{\lemma{भावं}\Bfootnote{ज्ञातं \cite{dp-msC}}} विज्ञातं भ‚व‚त्येवं\edtext{}{\lemma{त्येवं}\Bfootnote{०त्येव कृ० \cite{dp-msC}}} कृत‚क‚त्वाद‚नित्य‚त्व‚{\tiny $_{lb}$}‚प्र‚तीतिः स्यात् । त‚स्माद् य‚त् कृत‚कं त‚द‚नित्य‚मिति कृत‚क‚त्व‚म‚नित्य‚त्वे\edtext{}{\lemma{त्वे}\Bfootnote{०त्य‚त्व‚निय० \cite{dp-msA} \cite{dp-edP} \cite{dp-edH} \cite{dp-edE} \cite{dp-edN}}} निय‚त‚म‚भिधाय निय‚म‚{\tiny $_{lb}$}‚साध‚नायान्व‚य‚वाक्यार्थ‚प्र‚तिप‚त्तिविष‚यो दृष्टान्त उपादेयः । स च प्र‚द‚र्शितान्व‚य एव । अनेन‚{\tiny $_{lb}$}‚ त्व‚न्व‚य‚वाक्य‚म‚नुक्त्वैव दृष्टान्त उपात्तः । ईदृश‚श्च साध‚र्म्य‚मात्रेणैवोप‚योगी । न च साध‚र्म्यात्‚{\tiny $_{lb}$}‚ साध्य‚सिद्धिः । अतोऽन्व‚यार्थो दृष्टान्त‚स्त‚द‚र्थ‚श्चानेन नोपात्तः । साध‚र्म्यार्थ‚श्चोपात्तो निरुप‚योग
	\pend% ending standard par
      
	  \endgroup
	‚{\tiny $_{lb}$}‚

	  \pstart \leavevmode% starting standard par
	न‚नु घ‚टोप‚द‚र्श‚नेन कृत‚क‚त्वेन ताव‚द् घ‚ट‚स‚दृशः श‚ब्दो द‚र्शितः । त‚था चानित्य‚त्वेनापि‚{\tiny $_{lb}$}‚ स‚दृशो द‚र्शित‚स्त‚त‚श्च व्याप्तिर्द‚र्शितैवेत्याह--इहेति । इहानित्य‚त्व‚सिद्धिप्र‚स्तावे । कुत‚स्त‚था‚{\tiny $_{lb}$}‚\textbf{प्र‚त्येतुम‚श‚क्य} इत्याह--\textbf{अतिप्र‚स‚ङ्गादिति} । मूर्त्त‚त्वादि\edtext{}{\lemma{त्वादि}\Bfootnote{दे}}र‚पि सादृश्याग‚म \edtext{}{\lemma{म}\Bfootnote{सादृश्याव‚ग‚मात्}}‚{\tiny $_{lb}$}‚ \textbf{प्र‚स‚क्ति \edtext{}{\lemma{क्ति}\Bfootnote{क्ते}} रिष्टं ध‚र्म‚म‚ति}क्रान्तः \textbf{प्र‚स‚ङ्गोऽतिप्र‚स‚ङ्गः} त‚स्मात् । न त‚र्हि कृत‚क‚त्वाद‚नित्य‚त्वं क‚दा‚{\tiny $_{lb}$}‚चिद‚पि प्र‚त्येत‚व्य‚मित्याह--\textbf{य‚दि त्वि}ति । तुरिमाम‚व‚स्थां विशेष‚व‚तीं द‚र्श‚य‚ति । य‚दि कृत‚क‚{\tiny $_{lb}$}‚त्वानुवाद‚पूर्विका \edtext{}{\lemma{पूर्विका}\Bfootnote{पूर्व‚क}} विधानेनार्थान्त‚र‚त्वे स‚ति न नित्य‚त्व‚स्व‚भावं कृत‚क‚त्वं प्र‚तीतं \textbf{भ‚व‚त्येवं}‚{\tiny $_{lb}$}‚ त‚त्स्वाभाव्येन प्र‚तिप‚त्तौ \textbf{कृत‚क‚त्वाद‚नित्य‚त्व‚प्र‚तीतिः स्या}त् । य‚तः कृत‚क‚त्व‚स्यानित्य‚त्व‚स्व‚भावा‚{\tiny $_{lb}$}‚व‚ग‚मात्त‚त‚स्त‚त्प्र‚तीतिर्नान्य‚था \textbf{त‚स्माद्} हेतोः । \textbf{अन्व‚य‚वाक्य‚स्य} साध्य‚निय‚त‚त्वं साध‚न‚स्या\textbf{र्थो}ऽ‚{\tiny $_{lb}$}‚भिधेयः, \textbf{त‚त्प्र‚तिप‚त्तिविष}य‚स्त‚त्प्र‚द‚र्श‚न इत्य‚र्थः । अन्व‚य‚वाक्यार्थ‚प्र‚तिप‚त्तिविष‚योऽपि दृष्टान्तो‚{\tiny $_{lb}$}‚ य‚द्य‚प्र‚द‚र्शितान्व‚य‚स्त‚दा सोऽपि निरुप‚योगः किमित्युपादेय इत्याह--\textbf{स चे}ति । \textbf{चो}ऽव‚धार‚णे‚{\tiny $_{lb}$}‚ य‚स्माद‚र्थे वा । अनेन क‚थं नामाय‚मुपात्तो येनैव‚म‚भिधीय‚त इत्याह--\textbf{अनेनेति । अनेन}‚{\tiny $_{lb}$}‚ वादिना । तुर्विशेषार्थः । अन्व‚य‚प्र‚तिपाद‚कं \textbf{वाक्य‚म‚न्व‚य‚वाक्य‚म्} । त‚द‚नुक्त्वैव । अथैव‚{\tiny $_{lb}$}‚मुपात्तोऽपि य‚द्युप‚युज्य‚ते त‚दा का क्ष‚तिरित्याह--\textbf{ईदृश‚श्चे}ति । \textbf{च}कारोऽस्येमाम‚व‚स्थां भेद‚व‚तीमाह ।‚{\tiny $_{lb}$}‚ स‚मानः प्र‚क‚र‚णाद् घ‚टेन स‚दृशो ध‚र्मो य‚स्यासौ \textbf{स‚ध‚र्मा} । त‚स्य भावः \textbf{साध‚र्म्य‚म्} । त‚देव त‚न्\textbf{मात्र‚म् ।‚{\tiny $_{lb}$}‚ मात्र‚ग्र}ह‚णेन विशिष्टं साध‚र्म्य‚म‚पाक‚रोति । तेनैव‚कारेण निराकृत‚निरास‚मेव द्र‚ढ‚य‚ति । साध‚र्म्य‚{\tiny $_{lb}$}‚प्र‚द‚र्श‚न‚मात्रेणैवाय‚मुप‚योग‚वानित्य‚र्थः । अथैत‚त्प‚द‚द‚र्शितात्साध‚र्म्याद‚पि य‚दि साध्यं सिद्ध्य‚ति‚{\tiny $_{lb}$}‚ त‚दा क‚थ‚म‚य‚म‚नुप‚युक्त इत्याह--न \textbf{चे}ति । \textbf{चो}ऽव‚धार‚णे य‚स्माद‚र्थे वा ।
	\pend% ending standard par
      ‚{\tiny $_{lb}$}‚

	  \pstart \leavevmode% starting standard par
	एवं ब्रुव‚तोऽयं भावः--अनेन ख‚लु साध‚र्म्यं प्र‚द‚र्श‚नीय‚म् । कृत‚क‚त्वेनैव च साध‚र्म्यं‚{\tiny $_{lb}$}‚ प्र‚द‚र्श्याय‚ञ्च‚रितार्थो भ‚विष्य‚ति । न च कृत‚क‚त्वेन घ‚ट‚साध‚र्म्ये श‚ब्द‚स्याव‚ग‚तेऽप्य‚नित्य‚त्वेनापि‚{\tiny $_{lb}$}‚ त‚त्साध‚र्म्याव‚ग‚मोऽव‚श्य‚म्भावीति श‚क्य‚म‚भिधातुम् । मूर्त्त‚त्वादिनापि सादृश्याव‚ग‚मेऽस्यानिवार्य‚त्व‚{\tiny $_{lb}$}‚प्र‚स‚ङ्गादिति । य‚तः साध‚र्म्य‚मात्रान्न साध्य‚सिद्धिर‚तः कार‚णात् । \textbf{अन्व‚योऽन्वीय‚मान‚त्वं‚{\tiny $_{lb}$}‚ साध‚न‚स्य साध्येन सोऽर्थः} प्र‚योज \leavevmode\ledsidenote{\textenglish{80a/ms}} नं य‚स्य स त‚था न साध‚र्म्य‚मात्र‚प्र‚द‚र्श‚नार्थ इत्य‚र्थात् ।‚{\tiny $_{lb}$}‚ अनेनाप्य‚न्व‚यार्थ एवाय‚मुपात्त इत्याह--\textbf{त‚द‚र्थ‚श्चे}ति । \textbf{चो} य‚स्माद‚र्थे । सोऽन्व‚योऽर्थो य‚स्य स‚{\tiny $_{lb}$}‚ त‚था । \textbf{अनेन} वादिना । किम‚र्थ‚स्त‚र्ह्य‚नेनाय‚मुपात्त इत्याह--\textbf{साध‚र्म्ये}ति । \textbf{चो} य‚स्माद‚र्थे ।‚{\tiny $_{lb}$}‚ ‚{\tiny $_{lb}$}‚ \leavevmode\ledsidenote{\textenglish{243/dm}}‚{\tiny $_{lb}$}‚ 
	  
	इति व‚क्तृदोषाद‚यं दृष्टान्त‚दोषः । व‚क्त्रा ह्य‚त्र प‚रः प्र‚तिपाद‚यित‚व्यः । त‚तो य‚दि नाम न‚{\tiny $_{lb}$}‚ दुष्टं व‚स्तु त‚थापि व‚क्त्रा दुष्टं द‚र्शित‚मिति दुष्ट‚मेव ॥ ‚{\tiny $_{lb}$}‚ 
	  
	त‚था विप‚रीतान्व‚यः--य‚द‚नित्यं त‚त् कृत‚क‚मिति ॥ १२७ ॥‚{\tiny $_{lb}$}‚ 
	  
	त‚था विप‚रीतोऽन्व‚यो य‚स्मिन् दृष्टान्ते स त‚थोक्तः । त‚मेवोदाह‚र‚ति--य‚द‚नित्यं त‚त्‚{\tiny $_{lb}$}‚ कृत‚क‚मिति । कृत‚क‚त्व‚म‚नित्य‚त्व‚निय‚तं दृष्टान्ते\edtext{}{\lemma{दृष्टान्ते}\Bfootnote{दृष्टान्तेन \cite{dp-edE}}} द‚र्श‚नीय‚म् । एवं कृत‚क‚त्वाद‚नित्य‚त्व‚ग‚तिः‚{\tiny $_{lb}$}‚ स्यात् । अत्र त्व‚नित्य‚त्वं कृत‚क‚त्वे\edtext{}{\lemma{त्वे}\Bfootnote{०क‚त्व‚निय‚तं \cite{dp-msD}}} निय‚तं द‚र्शित‚म् । कृत‚क‚त्वं \edtext{}{\lemma{त्वं}\Bfootnote{त्व‚निय‚त‚मेवोक्त‚म‚नित्य‚त्वे \cite{dp-msC} \cite{dp-msD}}}त्व‚निय‚त‚मेवानित्य‚त्वे ।\edtext{\textsuperscript{*}}{\lemma{*}\Bfootnote{अत्र त‚तः कृत‚क‚त्व‚म‚नित्य‚त्वे निय‚त‚मेव इत्य‚धिकः पाठोऽस्ति \cite{dp-edE} प्र‚तौ ।}}‚{\tiny $_{lb}$}‚ त‚तो यादृश‚मिह कृत‚क‚त्व‚म‚निय‚त‚म‚नित्य‚त्वे \edtext{}{\lemma{त्वे}\Bfootnote{प्र‚द‚र्शित‚म् \cite{dp-msA} \cite{dp-msB} \cite{dp-edP} \cite{dp-edH} \cite{dp-edE} \cite{dp-edN}}}द‚र्शितं तादृशान्नास्त्य‚नित्य‚त्व‚प्र‚तीतिः । त‚थाहि—‚{\tiny $_{lb}$}‚य‚द‚नित्य‚मित्य‚नित्य‚त्व‚म‚नूद्य त‚त् कृत‚क‚मिति कृत‚क‚त्वं विहित‚म् । अतोऽनित्य‚त्वं निय‚त‚मुक्तं‚{\tiny $_{lb}$}‚ कृत‚क‚त्वे, न तु कृत‚क‚त्व‚म‚नित्य‚त्वे । त‚तो य‚थाऽनित्य‚त्वाद‚निय‚तात् प्र‚य‚त्नान‚न्त‚रीय‚क‚त्वे न‚{\tiny $_{lb}$}‚ प्र‚य‚त्नान‚न्त‚रीय‚क‚त्व‚प्र‚तीतिः, त‚द्व‚त् कृत‚क‚त्वाद‚नित्य‚त्व‚प्र‚तिप‚त्तिर्न स्याद्, अनित्य‚त्वेऽनिय‚त‚त्वात्‚{\tiny $_{lb}$}‚ कृत‚क‚त्व‚स्य ।‚{\tiny $_{lb}$}‚ \textbf{साध‚र्म्यार्थः} साध‚र्म्य‚प्र‚तिपाद‚न‚प्र‚योज‚न \textbf{उपात्त}स्त‚स्मा\textbf{न्निरुप‚योगः} स । त‚स्माद‚र्थें वा । अयं‚{\tiny $_{lb}$}‚ च‚श‚ब्दात् प‚रो द्र‚ष्ट‚व्यः । न‚नु कृत‚क‚त्वानित्य‚त्व‚योस्ताव‚द् व‚स्तुतोऽन्व‚योऽस्त्येव । त‚त्क‚थं‚{\tiny $_{lb}$}‚ विद्य‚मानेऽपि त‚स्मिन् त‚थाऽप्र‚द‚र्श‚न‚मात्रेणासौ दृष्टान्तो दुष्य‚तीत्याह--\textbf{व‚क्तृदोषा}दिति । \textbf{व‚क्ता‚{\tiny $_{lb}$}‚ ही}त्यादिनैत‚देव स‚म‚र्थ‚य‚ते । न‚नु व‚क्तैवासौ त‚था प्र‚द‚र्श‚य‚न्न‚प‚राध्य‚तु, अन्य‚स्य दृष्टान्त‚त‚{\tiny $_{lb}$}‚प‚स्विनः कोऽप‚राध इति चेत् । स्व‚कार्याऽक‚र‚ण‚मेव दोषः । त‚द‚क‚र‚णं त‚स्य स्व‚त एव \edtext{}{\lemma{एव}\Bfootnote{एवाऽ}}‚{\tiny $_{lb}$}‚साम‚र्थ्याद्, अन्येनान्य‚थाप्र‚द‚र्श‚नाद् वाऽस्तु । किमेताव‚ता त‚द‚क‚र‚णं त‚स्य नास्त्येवेति सूक्तं‚{\tiny $_{lb}$}‚ \textbf{व‚क्तुदोषाद‚यं दृष्टान्त‚दोष} इति ॥
	\pend% ending standard par
      ‚{\tiny $_{lb}$}‚

	  \pstart \leavevmode% starting standard par
	\textbf{विप‚रीतोऽन्व‚य} इति वैप‚रीत्येन प्र‚द‚र्श‚नाद् विप‚रीत उक्तो न तु विप‚रीतोन्व‚योऽस्त्येव ।‚{\tiny $_{lb}$}‚ कीदृशोऽविप‚रीतोऽन्व‚यो य‚स्माद‚यं विप‚रीत इत्याह--\textbf{कृत‚क‚त्व‚मि}ति । एवं प्र‚द‚र्श‚ने को गुण‚{\tiny $_{lb}$}‚ इत्याह--\textbf{एव‚मि}ति । एवं कृत‚क‚त्व‚स्यानित्य‚त्वे निय‚त‚त्व‚प्र‚द‚र्श‚ने स‚ति । अत्र पुनः कुत्र किं‚{\tiny $_{lb}$}‚ निय‚त‚मित्याह--\textbf{अत्रेति} । तुरिमाम‚व‚स्थां विशेष‚व‚तीं द‚र्श‚य‚ति । अनित्य‚त्वानुवादेन‚{\tiny $_{lb}$}‚ कृत‚क‚त्व‚स्य विधानाद् विधीय‚मान‚स्य व्याप‚क‚त‚या निय‚म‚विष‚य‚त्वादित्य‚भिस‚न्धिः कृत‚क‚त्वं‚{\tiny $_{lb}$}‚ पुनः कीदृशं द‚र्शित‚मित्याह--\add{कृ\textbf{त‚क‚त्व‚मि}ति} । तुर‚नित्य‚त्वात् कृत‚क‚त्वं भिन‚त्ति । \textbf{अनि}त्य‚त्वे‚{\tiny $_{lb}$}‚\textbf{ऽनिय‚त‚मि}ति कोऽर्थोऽनिप्य‚त्वेऽपि कृत‚क‚त्त्वं भ‚व‚ति, अन्त‚रेण कृत‚क‚त्व‚मिति । अनित्य‚त्वा‚{\tiny $_{lb}$}‚निय‚ताद‚पि कृत‚क‚त्वादि \edtext{}{\lemma{त्वादि}\Bfootnote{द}} नित्य‚त्वं प्र‚तिप‚श्येत इत्याह--\textbf{त‚त} इति । य‚त एव‚म‚नुवाद‚विधिक्र‚मे‚{\tiny $_{lb}$}‚ कृत‚क‚त्व‚म‚नित्य‚त्वानिय‚तं द‚र्शितं भ‚व‚ति, त‚त‚स्त‚स्मादिह प्र‚योगे । \textbf{यादृश}मिति विधीय‚मान‚म् ।‚{\tiny $_{lb}$}‚ पूर्व‚मेवंवादिनाऽ\textbf{नित्य‚त्वं कृत‚क‚त्वे} निय‚तं \textbf{द‚र्शित‚म्,} न कृत‚क‚त्व‚म‚नित्य‚त्वे चेति प्र‚तिज्ञामात्रेणोक्त‚म‚धुना‚{\tiny $_{lb}$}‚ तु येन प्र‚कारेण त‚स्यैव प्र‚द‚र्श‚न‚मायातं \textbf{त‚थाही}त्यादिना त‚द् द‚र्श‚य‚ति । अनित्य‚त्वं प्र‚य‚त्नान‚न्त‚रीय‚क‚{\tiny $_{lb}$}‚‚{\tiny $_{lb}$}‚ ‚{\tiny $_{lb}$}‚ \leavevmode\ledsidenote{\textenglish{244/dm}}‚{\tiny $_{lb}$}‚ 
	  
	य‚द्य‚पि च कृत‚क‚त्वं व‚स्तुस्थित्याऽनित्य‚त्वे निय‚तं \edtext{}{\lemma{तं}\Bfootnote{त‚थाप्य‚निय‚तं नास्ति \cite{dp-msB}}}त‚थाप्य‚निय‚तं व‚क्त्रा\edtext{}{\lemma{क्त्रा}\Bfootnote{प्र‚द‚र्शित‚म् \cite{dp-msC}}} द‚र्शित‚म् ।‚{\tiny $_{lb}$}‚ अतः \edtext{}{\lemma{अतः}\Bfootnote{अत‚स्त‚त् स्व‚यं न दुष्ट० \cite{dp-msA} \cite{dp-msB} \cite{dp-edP} \cite{dp-edH} \cite{dp-edE} \cite{dp-edN}}}स्व‚य‚म‚दुष्ट‚म‚पि \edtext{}{\lemma{पि}\Bfootnote{व‚क्तुर्दोषात्--\cite{dp-msA} \cite{dp-edP} \cite{dp-edH} \cite{dp-edE}}}व‚क्तृदोषाद् दुष्ट‚म् । ‚{\tiny $_{lb}$}‚ 
	  
	त‚स्माद् विप‚रीतान्व‚योऽपि व‚क्तुर‚प‚राधात्, न व‚स्तुतः \edtext{}{\lemma{स्तुतः}\Bfootnote{नः}} । प‚रार्थानुमाने च\edtext{}{\lemma{च}\Bfootnote{च नास्ति \cite{dp-msC} \cite{dp-msD}}} व‚क्तु‚{\tiny $_{lb}$}‚र‚पि \edtext{}{\lemma{पि}\Bfootnote{न केव‚लं हेतोः--\cite{dp-msD-n}}}दोष‚श्चिन्त्य‚त इति ॥ ‚{\tiny $_{lb}$}‚ 
	  
	साध‚र्म्येण दृष्टान्त‚दोषाः \edtext{}{\lemma{दोषाः}\Bfootnote{दृष्टान्त‚दोषाः इति नास्ति--\cite{dp-msB} \cite{dp-edP} \cite{dp-edH} \cite{dp-edE} \cite{dp-edN}}}॥ १२८ ॥‚{\tiny $_{lb}$}‚ 
	  
	साध‚र्म्येण\edtext{}{\lemma{र्म्येण}\Bfootnote{०र्म्येण त‚द्दृ० \cite{dp-msB}}} न‚व दृष्टान्त‚दोषा उक्ताः ॥ ‚{\tiny $_{lb}$}‚ 
	  
	वैध‚र्म्येणापि\edtext{}{\lemma{र्म्येणापि}\Bfootnote{०पि न‚व दृ० \cite{dp-msA} \cite{dp-edP} \cite{dp-edH} \cite{dp-edE} \cite{dp-edN}}} दृष्टान्त‚दोषान्\edtext{}{\lemma{दोषान्}\Bfootnote{व‚क्तुकाम आह \cite{dp-msB} \cite{dp-msD}}} व‚क्तुमाह-- ‚{\tiny $_{lb}$}‚ 
	  
	वैध‚र्म्येणापि--प‚र‚माणुव‚त् क‚र्म‚व‚द् आकाश‚व‚दिति साध्याद्य‚व्य‚तिरेकिणः ॥ १२९ ॥‚{\tiny $_{lb}$}‚ 
	  
	नित्य‚त्वे श‚ब्द‚स्य साध्ये हेताव‚मूर्त्त‚त्वे \edtext{}{\lemma{त्वे}\Bfootnote{प‚र‚माणुव‚द्व‚ध० \cite{dp-msA} \cite{dp-edP} \cite{dp-edH} \cite{dp-edE} प‚र‚माणुर्वैध‚र्म्येण \cite{dp-msC}}}प‚र‚माणुर्वैध‚र्म्य‚दृष्टान्तः साध्याव्य‚तिरेकी ।‚{\tiny $_{lb}$}‚ नित्य‚त्वात् प‚र‚माणूनाम् । क‚र्म साध‚नाव्य‚तिरेकि, अमूर्त्त‚त्वात् क‚र्म‚णः आकाश‚मुभ‚या‚{\tiny $_{lb}$}‚व्य‚तिरेकि, नित्य‚त्वाद‚मूर्त्त‚त्वाच्च ।‚{\tiny $_{lb}$}‚ त्व‚म‚न्त‚रेणाचिर‚प्र‚भादौ दृश्य‚मान‚म‚निय‚तं त‚त्र । त‚त्र य‚थाऽनित्य‚त्वात्प्र‚य‚त्नान‚न्त‚रीय‚क‚त्वा‚{\tiny $_{lb}$}‚प्र‚तीतिस्त‚द्व‚त् कृत‚क‚त्वाद‚प्य‚नित्य \add{त्व} निय‚तान्नानित्य‚त्व‚प्र‚तीतिर्भ‚वितुम‚र्ह‚ति ।
	\pend% ending standard par
      ‚{\tiny $_{lb}$}‚

	  \pstart \leavevmode% starting standard par
	न‚नु भ‚व‚तु अनित्य‚त्वात्प्र‚य‚त्नान‚न्त‚रीय‚क‚त्वाप्र‚तीतिर्व‚स्तुत‚स्त‚स्य त‚त्रानिय‚त‚त्वात्कृत‚क‚त्वं तु‚{\tiny $_{lb}$}‚ प‚र‚मार्थ‚तो निय‚त‚म‚नित्य‚त्वे । त‚त्कुत‚स्त‚स्मात्त‚स्याप्र‚तीतिरित्याह--\textbf{य‚द्य‚पित्या}दि । \textbf{एवं} ब्रुव‚तोऽयं‚{\tiny $_{lb}$}‚ भावः--व‚स्तुत‚श्छेद‚न‚स्व‚भावोऽपि प‚र‚शुर्य‚दाच्छेद‚केन भ्रान्त्याऽन्य‚था वा उद्व‚र्त्त्य बाहुमुद्य‚म्य‚{\tiny $_{lb}$}‚ द्वैधीभावार्थं ध‚वादौ निपात्य‚ते, त‚दा तेन स्व‚य‚म‚दुष्टेनापि य‚था छिदा न स‚म्पाद्य‚ते, त‚द्व‚द‚नेनापि‚{\tiny $_{lb}$}‚ प‚रोक्षार्थ‚प्र‚तीतिर्न स‚म्पाद्य‚ते, स्व‚य‚म‚दुष्टेनापीति ।
	\pend% ending standard par
      ‚{\tiny $_{lb}$}‚

	  \pstart \leavevmode% starting standard par
	य‚स्मादेवं त \leavevmode\ledsidenote{\textenglish{80b/ms}} स्माद् हेतोर्व‚क्तुर‚नुमाप‚यितु\textbf{र‚प‚राधाद्} दोषाद् \textbf{विप‚रीतान्व‚योऽपि}‚{\tiny $_{lb}$}‚ दृष्टान्त‚दोष इति दोषः । न केव‚ल‚म‚प्र‚द‚र्शितान्व‚य इत्याह--\textbf{न व‚स्तुनः} कृत‚क‚त्व‚स्याप‚राधात् ।‚{\tiny $_{lb}$}‚ त‚त्ताव‚त्स्व‚तोऽदुष्ट‚म् । त‚त्किं व‚क्तृदोषेण चिन्तितेनेत्याह--\textbf{प‚रार्थे}ति । न केव‚लं साध‚न‚स्य‚{\tiny $_{lb}$}‚ दोष‚श्चिन्त्य‚त इत्य‚पि श‚ब्दात् ॥
	\pend% ending standard par
      ‚{\tiny $_{lb}$}‚

	  \pstart \leavevmode% starting standard par
	य‚तोऽल्पीयो नास्ति स \textbf{प‚र‚माणुः} वैध‚र्म्य‚प्र‚तिपाद \add{न} विष‚योपात्त‚त्वान् \textbf{वैध‚र्म्य‚दृष्टान्त‚{\tiny $_{lb}$}‚ उक्तः} । \textbf{साध्य}स्या\textbf{व्य‚तिरेको} निवृत्त्य‚भावः सोऽस्यास्तीति त‚थोक्तः ॥
	\pend% ending standard par
      ‚{\tiny $_{lb}$}‚\textsuperscript{\textenglish{245/dm}}‚{\tiny $_{lb}$}‚
	  \bigskip
	  \begingroup
	

	  \pstart \leavevmode% starting standard par
	साध्य‚मादिर्येषां तानि साध्यादीनि साध्य‚साध‚नोभ‚यानि । तेषाम‚व्य‚तिरेको \edtext{}{\lemma{तिरेको}\Bfootnote{०रेको वृत्त्य० \cite{dp-msA} \cite{dp-edP} \cite{dp-edH} \cite{dp-edN} ०रेको निवृत्ता[[त्त्य]] भावः \cite{dp-msB}}}निवृत्त्य‚{\tiny $_{lb}$}‚भावः । स येषाम‚स्ति ते साध्याद्य‚व्य‚तिरेकिणः । ते चोदाहृताः ॥
	\pend% ending standard par
       ‚{\tiny $_{lb}$}‚ 

	  \pstart \leavevmode% starting standard par
	अप‚रानुदाह‚र्त्तुमाह--
	\pend% ending standard par
       ‚{\tiny $_{lb}$}‚ 
	  \bigskip
	  \begingroup
	

	  \pstart \leavevmode% starting standard par
	त‚था संदिग्ध‚साध्य‚व्य‚तिरेकाद‚यः । य‚थाऽस‚र्व‚ज्ञाः क‚पिलाद‚योऽनाप्ता वा‚{\tiny $_{lb}$}‚ अविद्य‚मान‚स‚र्व‚ज्ञ‚ताप्त‚तालिङ्ग‚भूत‚प्र‚माणातिश‚य‚शास‚न‚त्वादिति । अत्र \edtext{}{\lemma{अत्र}\Bfootnote{वैध‚र्म्येणोदा० \cite{dp-msC}}}वैध‚{\tiny $_{lb}$}‚र्म्योदाह‚र‚ण‚म्--यः स‚र्व‚ज्ञ आप्तो वा स ज्योतिर्ज्ञानादिक‚मुप‚दिष्ट‚वान् ।‚{\tiny $_{lb}$}‚ \edtext{\textsuperscript{*}}{\lemma{*}\Bfootnote{त‚द्य‚था \cite{dp-msB} \cite{dp-msD} \cite{dp-edP} \cite{dp-edH} \cite{dp-edE} \cite{dp-edN}}}य‚था--\edtext{\textsuperscript{*}}{\lemma{*}\Bfootnote{वृष‚भ \cite{dp-msC}}}ऋष‚भ‚व‚र्ध‚मानादिरिति । \edtext{\textsuperscript{*}}{\lemma{*}\Bfootnote{रिति वैध‚र्म्योदाह‚र‚णाद‚स‚र्व० \cite{dp-msC}}}त‚त्रास‚र्व‚ज्ञ‚तानाप्त‚त‚योः साध्य‚ध‚र्म‚योः‚{\tiny $_{lb}$}‚ संदिग्धो व्य‚तिरेकः ॥१३०॥
	\pend% ending standard par
      
	  \endgroup
	‚{\tiny $_{lb}$}‚ 

	  \pstart \leavevmode% starting standard par
	त‚थेति । साध्य‚स्य व्य‚तिरेकः साध्य‚व्य‚तिरेकः । संदिग्धः साध्य‚व्य‚तिरेको य‚स्मिन् स‚{\tiny $_{lb}$}‚ संदिग्ध‚साध्य‚व्य‚तिरेकः । स आदिर्येषां ते त‚थोक्ताः ।
	\pend% ending standard par
       ‚{\tiny $_{lb}$}‚ 

	  \pstart \leavevmode% starting standard par
	संदिग्ध‚साध्य‚व्य‚तिरेक‚मुदाह‚र्त्तुमाह--य‚थेति । अस‚र्व‚ज्ञा इत्येकं साध्य‚म् । अनाप्ता‚{\tiny $_{lb}$}‚ अक्षीण‚दोषा इति द्वितीय‚म् । क‚पिलाद‚य इति ध‚र्मीं । अविद्य‚मान‚स‚र्व‚ज्ञ‚तेत्यादि हेतुः । स‚र्व‚ज्ञ‚ता‚{\tiny $_{lb}$}‚ च आप्त‚ता च त‚योर्लिङ्ग‚भूतः प्र‚माणातिश‚यो लिङ्गात्म‚कः प्र‚माण‚विशेषः । अविद्य‚मानः‚{\tiny $_{lb}$}‚ स‚र्व‚ज्ञ‚ताप्त‚तालिङ्ग‚भूतः प्र‚माणातिश‚यो य‚स्मिन् त‚त् त‚थोक्तं शास‚न‚म् । तादृशं शास‚नं येषां‚{\tiny $_{lb}$}‚ ते\edtext{}{\lemma{ते}\Bfootnote{क‚पिलाद‚यः--\cite{dp-msD-n}}} त‚थोक्ताः । तेषां भाव‚स्त‚त्त्व‚म् । त‚स्मात् । प्र‚माणातिश‚यो ज्योतिर्ज्ञानोप‚देश इहाभिप्रेतः ।‚{\tiny $_{lb}$}‚ य‚दि हि क‚पिलाद‚यः स‚र्व‚ज्ञा आप्ता वा स्युस्त‚दा ज्योतिर्ज्ञानादिकं क‚स्मान्नोप‚दिष्ट‚व‚न्तः ? न‚{\tiny $_{lb}$}‚ चोप‚दिष्ट‚व‚न्तः । त‚स्मान्न स‚र्व‚ज्ञा आप्ता वा ।
	\pend% ending standard par
       ‚{\tiny $_{lb}$}‚ 

	  \pstart \leavevmode% starting standard par
	अत्र प्र‚माणे वैध‚र्म्योदाह‚र‚ण‚म् । यः स‚र्व‚ज्ञ आप्तो वा स ज्योतिर्ज्ञानादिकं स‚र्व‚ज्ञ‚ताप्त‚ता‚{\tiny $_{lb}$}‚लिङ्ग‚भूत‚मुप‚दिष्ट‚वान् । य‚था ऋष‚भो व‚र्ध‚मान‚श्च तावादी य‚स्य स ऋष‚भ-व‚र्ध‚मानादिर्दि-
	\pend% ending standard par
      
	  \endgroup
	‚{\tiny $_{lb}$}‚

	  \pstart \leavevmode% starting standard par
	प्र‚क‚र‚णात्क्षीण‚दोष‚त्व‚माप्त‚त्व‚ल‚क्ष‚ण‚म‚स्य विव‚क्षित‚मिति त‚त्वा \textbf{अनाप्ता अक्षीण‚दोषा} इति‚{\tiny $_{lb}$}‚ व्याच‚ष्टे । \textbf{क‚पिलः सांख्य}द‚र्श‚न‚कारो मुनिः । आदिश‚ब्देन \textbf{गौत‚मा}देः स‚ङ्ग्र‚हः । \textbf{शास‚नं} मार्ग‚{\tiny $_{lb}$}‚स्त‚त्प्र‚णीत‚शास्त्र‚मिति याव‚त् ।
	\pend% ending standard par
      ‚{\tiny $_{lb}$}‚

	  \pstart \leavevmode% starting standard par
	न‚नु तेषाम‚पि शास‚ने लिङ्गात्म‚कः प्र‚माणातिश‚योऽनेकोऽस्त्येव । त‚त्क‚थं हेतुर‚य‚म‚सिद्धो न‚{\tiny $_{lb}$}‚ भ‚व‚तीत्याह--\textbf{प्र‚माणातिश‚य} इति । \textbf{ज्योतिषां} ग्र‚ह‚न‚क्ष‚त्राणां \textbf{ज्ञानं} त‚स्यो\textbf{प‚देशः} । ज्योतिर्ज्ञान‚स्यो‚{\tiny $_{lb}$}‚प‚ल‚क्ष‚ण‚त्वाज्ज्योतिर्ज्ञानाद्युप‚देश इत्य‚व‚ग‚न्त‚व्य‚म् । त‚द‚यं स‚मुदायार्थः--य‚स्मात्तैर्ज्योतिर्ज्ञान‚म‚न्या‚{\tiny $_{lb}$}‚तीन्द्रिय‚ज्ञानं च स्व‚शास‚ने नोप‚दिष्टं त‚स्मात्ते न त‚थारूपा इति । \textbf{य‚दि ही}त्यादिनैत‚देव द‚र्श‚य‚ति ।‚{\tiny $_{lb}$}‚ ‚{\tiny $_{lb}$}‚ \leavevmode\ledsidenote{\textenglish{246/dm}}‚{\tiny $_{lb}$}‚ 
	  
	ग‚म्ब‚राणां शास्ता\edtext{}{\lemma{शास्ता}\Bfootnote{शास्ता साध्य‚व्य‚तिरेकः स‚र्व० \cite{dp-msC}}} स‚र्व‚ज्ञ‚श्च\edtext{}{\lemma{श्च}\Bfootnote{च नास्ति \cite{dp-msA} \cite{dp-msB} \cite{dp-edP} \cite{dp-edH} \cite{dp-edE} \cite{dp-edN}}} आप्त‚श्चेति । त‚दिह वैध‚र्म्योदाह‚र‚णाद् ऋष‚भादेर‚स‚र्व‚ज्ञ‚त्व‚स्या‚{\tiny $_{lb}$}‚नाप्त‚तायाश्च व्य‚तिरेको व्यावृत्तिः संदिग्धा । य‚तो ज्योतिर्ज्ञानं चोप‚दिशेद् अस‚र्व‚ज्ञाश्च भ‚वेद्‚{\tiny $_{lb}$}‚ अनाप्ता वा । कोऽत्र विरोधः ? नैमित्तिक‚मेत‚ज्ज्ञानं व्य‚भिचारि न स‚र्व‚ज्ञ‚त्व‚म‚नुमाप‚येत् ॥‚{\tiny $_{lb}$}‚ आदिश‚ब्देनान्यातीन्द्रिय‚ज्ञानं संगृह्य‚ते । ऋष‚भ‚व‚र्ध‚मान‚नाम‚धेयाव‚ह्नीकाणामिष्ट‚देवौ ।‚{\tiny $_{lb}$}‚ आदिश‚ब्देन \textbf{पार्श्व‚नाथारिष्ट‚नेमिप्र‚भृ}तेः स‚ङ्ग्र‚हः । \textbf{शास्ता} स्व‚दृष्ट‚म‚न्योप‚देष्टा । \textbf{स‚र्व‚ज्ञ‚श्चा‚{\tiny $_{lb}$}‚प्त‚श्चे}त्य‚नेन द्व‚य‚स्यापि साध्य‚स्याभावं द‚र्श‚य‚ति । य‚दाह--\textbf{अक‚ल‚ङ्कः}--य‚दि सूक्ष्मे व्य‚व‚हिते वा‚{\tiny $_{lb}$}‚ व‚स्तुनि बुद्धिर‚त्य‚न्त‚प‚रोक्षे न स्यात्क‚थं त‚र्हि ज्योतिर्ज्ञानाविसंवादः ? ज्योतिर्ज्ञान‚म‚पि हि स‚र्व‚ज्ञ‚{\tiny $_{lb}$}‚\textbf{प्र‚व‚र्त्तित‚मेव,} एत‚स्माद‚विसंवादिनो ज्योतिर्ज्ञानात्स‚र्व‚ज्ञ‚सिद्धिः । त‚दुक्त‚म्--
	\pend% ending standard par
      ‚{\tiny $_{lb}$}‚
	  \bigskip
	  \begingroup
	
	    \begin{quote}
	  
	    
	    \stanza[\smallbreak]
	धीर‚त्य‚न्त‚प‚रोक्षेऽर्थे न चेत्पुंसां कुतः पुनः ।&ज्योतिर्ज्ञानाविसंवादः श्रुत‚त्वाच्चेत्साध‚नान्त‚र‚म् ॥\&[\smallbreak]


	
	    \end{quote}
	  \href{http://sarit.indology.info/?cref=svi.p413}{सिद्धिवि०, पृ० ४१३} इति ।
	  \endgroup
	‚{\tiny $_{lb}$}‚

	  \pstart \leavevmode% starting standard par
	क‚थं पुनः स‚त्य‚पि ज्योतिर्ज्ञानाद्युप‚देशे विप‚क्षादृष‚भादेर‚स‚र्व‚ज्ञ‚त्वादेर्व्यावृत्तिः स‚न्दिग्धेत्याह—‚{\tiny $_{lb}$}‚\textbf{य‚त} इति । अत्र स‚र्व‚ज्ञ‚तायाम‚निष्टायां स‚त्याम‚पि ज्योतिर्ज्ञानाद्युप‚देशे \textbf{को विरोधो}ऽनुप‚प‚त्तिः ?‚{\tiny $_{lb}$}‚ क्षेपे \textbf{किमः} प्र‚योगान्न क‚श्चिदित्य‚र्थः ।
	\pend% ending standard par
      ‚{\tiny $_{lb}$}‚

	  \pstart \leavevmode% starting standard par
	अथास‚र्व‚ज्ञ‚त्वे त‚स्यैत‚स्मादिदं ग्र‚होप‚रागादि भावि त‚त‚श्चैवं भावीति ज्ञानं क‚थं वृत्तं येन‚{\tiny $_{lb}$}‚ स‚म्वादि त‚थोप‚दिशेत् । त‚स्मात् स‚र्व‚ज्ञ एवासाविति निश्च‚य इत्याश‚ङ्क्याह--\textbf{नैभित्तिक‚मेत}‚{\tiny $_{lb}$}‚दिति । एत‚ज्ज्योतिर्ज्ञानादिकं निमित्तात्प‚र‚म्प‚र‚या कार‚णाद् । विष‚येण च विष‚यिणो‚{\tiny $_{lb}$}‚ निर्देशान्निमित्त‚द‚र्श‚नादाग‚तं \textbf{नैमित्तिक‚म्,} अत एव त‚दुप‚देष्टुः स‚र्व‚ज्ञ‚ता व्य‚भिच‚र‚तीति । त‚था‚{\tiny $_{lb}$}‚ स‚र्व‚ज्ञ‚ताम‚न्त‚रेणापि भ‚व‚तीदानीन्त‚न‚ज्योतिषिकाणामिवाऽतीन्द्रियोप‚रागादिज्ञान‚मित्य‚भिप्रायः ।‚{\tiny $_{lb}$}‚ त‚थाभूतं \textbf{ज्ञानं न स‚र्व‚ज्ञ‚ताम‚नुमाप‚येद}नुमाप‚यितुं श‚क्नोति ।
	\pend% ending standard par
      ‚{\tiny $_{lb}$}‚

	  \pstart \leavevmode% starting standard par
	न‚नु च त‚थाभूतेन भाविव‚स्तुना स‚ह क‚स्य‚चित्कार्य‚कार‚ण‚भाव एव तेन क‚श्चिद् ज्ञातः ।‚{\tiny $_{lb}$}‚ य‚द्य\leavevmode\ledsidenote{\textenglish{81a/ms}}साव‚स‚र्व‚ज्ञो भ‚वेद् अस‚र्व‚ज्ञ‚श्च क‚थ‚मुप‚दिशेदिति चेत् । न । एत‚द‚न्य‚तोऽपि ज्ञात्वा त‚दुप‚{\tiny $_{lb}$}‚देश‚स‚म्भ‚वात् । त‚स्यान्य‚त‚स्त‚थाविधात् । न चादिमान् संसारः । येन स एवाद्य‚स्त‚था ज्ञानी‚{\tiny $_{lb}$}‚ स‚र्व‚ज्ञः, स चास्माक‚मृष‚भादिरित्य‚प्युच्येत । अथ‚वा--य‚द्य‚साव‚स‚र्व‚ज्ञ‚स्त‚दा त‚स्य त‚थाभूत‚स्य‚{\tiny $_{lb}$}‚ साध्य‚साध‚न‚भाव‚स्याविदुष उप‚देशाद‚स्म‚दादीनाम‚तीन्द्रियोप‚रागादिज्ञानं संवादि च क‚थं भ‚वेदि‚{\tiny $_{lb}$}‚त्याश‚ङ्क्याह--\textbf{नैमित्तिक}मिति ।
	\pend% ending standard par
      ‚{\tiny $_{lb}$}‚

	  \pstart \leavevmode% starting standard par
	अय‚म‚र्थः--कार‚ण‚द‚र्श‚न‚स्व‚भाव‚कार्य‚ज्ञान‚मेत‚त्; एत‚च्च भाविव‚स्तुव्य‚तिरेकेण भ‚वेद‚पि,‚{\tiny $_{lb}$}‚ नाव‚श्यं कार‚णानि कार्य‚व‚न्ति भ‚व‚न्तीति न्यायात् । एत‚च्च व्य‚भिचारि ज्ञान‚मुप‚दिश्य‚मानं‚{\tiny $_{lb}$}‚ नोप‚देष्टुः स‚र्व‚ज्ञ‚ताम‚नुमाप‚यितुं क‚ल्प्य‚ते ।
	\pend% ending standard par
      ‚{\tiny $_{lb}$}‚

	  \pstart \leavevmode% starting standard par
	अथ त‚देव त‚स्य त‚थाभूत‚स्य कार‚ण‚व‚स्तुन‚स्त‚त्त‚द्भाविव‚स्तु प्र‚ति कार‚ण‚त्वं क‚थं जानीया‚{\tiny $_{lb}$}‚त्क‚थं चोप‚दिशेद् य‚द्य‚साव‚स‚र्व‚ज्ञ इति चेत् । अन्य‚त‚स्त‚द्धि ज्ञात्वा ज्ञातं \edtext{}{\lemma{ज्ञातं}\Bfootnote{नं}} चोप‚देश‚श्च त‚स्योप‚{\tiny $_{lb}$}‚ ‚{\tiny $_{lb}$}‚ \leavevmode\ledsidenote{\textenglish{247/dm}}‚{\tiny $_{lb}$}‚ 
	  
	स‚न्दिग्ध‚साध‚न‚व्य‚तिरेको य‚था--न त्र‚यीविदा ब्राह्म‚णेन ग्राह्य‚व‚च‚नः‚{\tiny $_{lb}$}‚ क‚श्चिद् \edtext{}{\lemma{श्चिद्}\Bfootnote{क‚श्चित् पुरुषो \cite{dp-msB} \cite{dp-edP} \cite{dp-edH} \cite{dp-edE}}}विव‚क्षितः पुरुषो रागादिम‚त्त्वादिति । अत्र वैध‚र्म्योदाह‚र‚ण‚म्--ये‚{\tiny $_{lb}$}‚ ग्राह्य‚व‚च‚ना न ते रागादिम‚न्तः । त‚द्य‚था \edtext{}{\lemma{था}\Bfootnote{गोत‚मा० \cite{dp-msD}}}गौत‚माद‚यो ध‚र्म‚शास्त्राणां\edtext{}{\lemma{शास्त्राणां}\Bfootnote{ध‚र्म‚शास्त्र‚प्र‚णे० \cite{dp-msC}}}‚{\tiny $_{lb}$}‚ प्र‚णेतार इति । गौत‚मादिभ्यो रागादिम‚त्त‚व‚स्य साध‚न‚ध‚र्म‚स्य व्यावृत्तिः‚{\tiny $_{lb}$}‚ स‚न्दिग्धा ॥ १३१ ॥  ‚{\tiny $_{lb}$}‚ 
	  
	स‚न्दिग्धः साध‚न‚व्य‚तिरेको य‚स्मिन् स त‚थोक्तः । त‚मुदाह‚र‚ति--य‚थेति । ऋक्साम‚य‚जंषि‚{\tiny $_{lb}$}‚ त्रीणि त्र‚यी तां वेत्तीति\edtext{}{\lemma{वेत्तीति}\Bfootnote{वेत्ति त्र‚यी० \cite{dp-msA} \cite{dp-msB} \cite{dp-msD} \cite{dp-edP} \cite{dp-edH} \cite{dp-edE} \cite{dp-edN}}} त्र‚यीवित् । तेन न ग्राह्यं व‚च‚नं य‚स्येति साध्य‚म् । विव‚क्षित इति‚{\tiny $_{lb}$}‚ क‚पिलादिर्ध‚र्मी । रागादिम‚त्त्वादिति हेतुः । ‚{\tiny $_{lb}$}‚ 
	  
	अत्र प्र‚माणे वैध‚र्म्योदाह‚र‚ण‚म्--साध्याभावः साध‚नाभावेन \edtext{}{\lemma{नाभावेन}\Bfootnote{०भावेन व्याप्तो य‚त्र द‚र्श्य‚ते--\cite{dp-msA} \cite{dp-edP} \cite{dp-edH} \cite{dp-edE}}}य‚त्र व्याप्तो द‚र्श्य‚ते त‚द्‚{\tiny $_{lb}$}‚ वैध‚र्म्योदाह‚र‚ण‚म् । ग्राह्यं व‚च‚नं येषां ते ग्राह्य‚व‚च‚ना इति साध्य‚निवृत्तिम‚नूद्य न ते रागादिम‚न्त‚{\tiny $_{lb}$}‚ इति साध‚नाभावो विहितः । गौत‚म आदिर्येषां ते त‚थोक्ता म‚न्वाद‚यो ध‚र्म‚शास्त्राणि स्मृत‚य‚{\tiny $_{lb}$}‚स्तेषां क‚र्त्तारः त्र‚यीविदा हि ब्राह्म‚णेन ग्राह्य‚व‚च‚ना ध‚र्म‚शास्त्र‚कृतो वीत‚रागाश्च त इतीह‚{\tiny $_{lb}$}‚ \edtext{\textsuperscript{*}}{\lemma{*}\Bfootnote{इतीह ध‚र्मिव्य० \cite{dp-msC} \cite{dp-msD} इति ध‚र्मी \cite{dp-msA} \cite{dp-edP} \cite{dp-edH} \cite{dp-edN} इति ध‚र्मिव्य० \cite{dp-edE}}}ध‚र्मी व्य‚तिरेक‚विष‚यो गौत‚माद‚य इति । गौत‚मादिभ्यो रागादिम‚त्त्व‚स्य साध‚न‚स्य निवृत्तिः‚{\tiny $_{lb}$}‚ स‚न्दिग्धा । य‚द्य‚पि ते ग्राह्य‚व‚च‚नास्त्र‚यीविद‚स्त‚थापि\edtext{}{\lemma{थापि}\Bfootnote{०विदा त‚थापि \cite{dp-msA} \cite{dp-edP} \cite{dp-edH} \cite{dp-edE} \cite{dp-edN}}} किं स‚रागा उत वीत‚रागा इति स‚न्देहः ॥ ‚{\tiny $_{lb}$}‚ 
	  
	स‚न्दिग्धोभ‚य‚व्य‚तिरेको य‚था--अवीत‚रागाः क‚पिलाद‚यः, प‚रिग्र‚हाग्र‚ह योगादिति ।‚{\tiny $_{lb}$}‚ अत्र वैध‚र्म्येणोदाह‚र‚ण‚म्--यो वीत‚रागो न त‚स्य प‚रिग्र‚हाग्र‚हः ।   ‚{\tiny $_{lb}$}‚ प‚द्य‚ते । त‚स्याप्य‚न्य‚स्माद् । अनादिश्च संसार इति क‚थ‚मेताव‚न्मात्रेण व‚र्ध‚मानाहेः स‚र्व‚ज्ञ‚त्व‚{\tiny $_{lb}$}‚सिद्धिरिति ॥
	\pend% ending standard par
      ‚{\tiny $_{lb}$}‚

	  \pstart \leavevmode% starting standard par
	त्र‚योऽव‚य‚वा य‚स्याः संह‚तेरिति \textbf{त्र‚यी}त्य‚द‚न्तात् ङीप् । किम‚त्र साध्या \add{भावा} नुवादेन साध‚ना‚{\tiny $_{lb}$}‚भावो विहितो येनैत‚द् वैध‚र्म्योदाह‚र‚णं भ‚व‚तीत्याश‚ङ्क्याह--\textbf{साध्ये}ति । \textbf{गौत‚मोऽक्ष‚पादा}प‚र‚नामा‚{\tiny $_{lb}$}‚ \textbf{न्याय‚सूत्र‚स्या}पि प्र‚णेता मुनिः । \textbf{म‚नु}रिति स्मृतिकारो मुनिः । आदिश‚ब्दाद् \textbf{विश्व‚रूप‚याज्ञ‚{\tiny $_{lb}$}‚व‚ल्क्य‚संवृत्ता}देः स‚ङ्ग्र‚हः । इतिस्त‚स्मात् । ध‚र्मित्व‚मात्र‚जिज्ञापिष‚या \textbf{ध‚र्मी व्य‚तिरेक‚विष‚य} इति‚{\tiny $_{lb}$}‚ अभिधाय त‚स्यैव विशेष‚निष्ठ‚प्र‚तिपाद‚नेच्छ‚या \textbf{गौत‚माद‚य} इत्युक्त‚म् । \edtext{\textsuperscript{*}}{\lemma{*}\Bfootnote{स‚न्दिग्ध‚म्--स०}}इन्द्रिय‚म‚न‚स्कार\add{ज}मेत‚{\tiny $_{lb}$}‚द्राजा\edtext{}{\lemma{द्राजा}\Bfootnote{मेत‚ज्ज्ञा}}न‚मित्यादिव‚त् त‚स्माद् वैध‚र्मी दृष्टान्तो गौत‚मादिरिति वाक्यार्थः । त‚थात्वं च तेषां‚{\tiny $_{lb}$}‚ त‚थात्वेनोपादानान्न तु प‚र‚मार्थ‚त इत्य‚व‚सेय‚म् । ग्राह्य‚व‚च‚न‚त्वेऽपि तेषां क‚थं ताद्रूप्य‚स‚न्देह‚{\tiny $_{lb}$}‚ इत्याह--\textbf{य‚द्य‚पी}ति । \textbf{उते}ति प‚क्षान्त‚र‚मुद्द्योत‚य‚ति ॥
	\pend% ending standard par
      ‚{\tiny $_{lb}$}‚‚{\tiny $_{lb}$}‚‚{\tiny $_{lb}$}‚‚{\tiny $_{lb}$}‚‚{\tiny $_{lb}$}‚‚{\tiny $_{lb}$}‚‚{\tiny $_{lb}$}‚‚{\tiny $_{lb}$}‚\textsuperscript{\textenglish{248/dm}}‚{\tiny $_{lb}$}‚
	  \bigskip
	  \begingroup
	
	  \bigskip
	  \begingroup
	

	  \pstart \leavevmode% starting standard par
	य‚थ‚र्ष‚भादेरिति\edtext{}{\lemma{भादेरिति}\Bfootnote{भादेः । ऋष० \cite{dp-msC}}} । ऋष‚भादेर‚वीत‚राग‚त्व‚प‚रिग्र‚ह‚ग्र‚ह‚योग‚योः साध्य‚साध‚न‚ध‚र्म‚योः‚{\tiny $_{lb}$}‚ स‚न्दिग्धो व्य‚तिरेकः ॥१३२॥
	\pend% ending standard par
      
	  \endgroup
	‚{\tiny $_{lb}$}‚ 

	  \pstart \leavevmode% starting standard par
	स‚न्दिग्ध उभ‚योर्व्य‚तिरेको य‚स्मिन् स त‚थोक्तः । त‚मुदाह‚र‚ति य‚थेति । अवीत‚रागा‚{\tiny $_{lb}$}‚ इति रागादिम‚त्त्वं साध्य‚म् । क‚पिलाद‚य इति ध‚र्मी । प‚रिग्र‚हो ल‚भ्य‚मान‚स्य स्वीकारः प्र‚थ‚मः ।‚{\tiny $_{lb}$}‚ स्वीकारादूर्ध्वं य‚द् गार्ध्यं मात्स‚र्यं स आग्र‚हः । प‚रिग्र‚ह‚श्च आग्र‚ह‚श्च, ताभ्यां योगात् । क‚पिला‚{\tiny $_{lb}$}‚द‚यो ल‚भ्य‚मानं स्वीकुर्व‚न्ति स्वीकृतं न मुञ्च‚न्ति--इति ते रागादिम‚न्तो ग‚म्य‚न्ते । अत्र प्र‚माणे‚{\tiny $_{lb}$}‚ वैध‚र्म्योदाह‚र‚ण‚म्--य‚त्र साध्याभावे साध‚नाभावो द‚र्श‚यित‚व्यः । यो वीत‚राग इति साध्याभाव‚{\tiny $_{lb}$}‚म‚नूद्य, न त‚स्य प‚रिग्र‚हाग्र‚हाविति साध‚नाभावो विहितः । य‚थ‚र्ष‚भादेरिति दृष्टान्तः । एत‚स्मादृ‚{\tiny $_{lb}$}‚ ष‚भादेर्दृष्टान्ताद् अवीत‚राग‚त्व‚स्य साध्य‚स्य प‚रिग्र‚हाग्र‚ह‚योग‚स्य\edtext{}{\lemma{स्य}\Bfootnote{योग‚त्व‚स्य \cite{dp-msA}}} च साध‚न‚स्य \edtext{}{\lemma{स्य}\Bfootnote{व्यावृत्तिः \cite{dp-msC} \cite{dp-edE}}}निवृत्तिः‚{\tiny $_{lb}$}‚ स‚न्दिग्धा । ऋष‚भादीनां हि प‚रिग्र‚हाग्र‚ह‚योगोऽपि स‚न्दिग्धो वीत‚राग‚त्वं च । य‚दि नाम त‚त्सिद्धान्ते‚{\tiny $_{lb}$}‚ वीत‚रागाश्च निष्प‚रिग्र‚हाश्च \edtext{}{\lemma{हाश्च}\Bfootnote{प‚रिप‚ठ्य‚न्ते \cite{dp-msD}}}प‚ठ्य‚न्ते त‚थापि स‚न्देह एव ॥
	\pend% ending standard par
       ‚{\tiny $_{lb}$}‚ 

	  \pstart \leavevmode% starting standard par
	अप‚रान‚पि \edtext{}{\lemma{पि}\Bfootnote{अप‚राण्य‚पि त्रीण्युदा० \cite{dp-msD}}}त्रीनुदाह‚र्त्तुमाह--
	\pend% ending standard par
       ‚{\tiny $_{lb}$}‚ 
	  \bigskip
	  \begingroup
	

	  \pstart \leavevmode% starting standard par
	अव्य‚तिरेको य‚था--अवीत‚रागोऽयं\edtext{}{\lemma{रागोऽयं}\Bfootnote{०रागो व‚क्तृ० \cite{dp-msD} \cite{dp-msB} \cite{dp-edP} \cite{dp-edH} \cite{dp-edE} \cite{dp-edN}}} व‚क्तृत्वात् । वैध‚र्म्येणोदा‚{\tiny $_{lb}$}‚ह‚र‚ण‚म्\edtext{}{\lemma{म्}\Bfootnote{वैध‚र्म्योदाह‚र० \cite{dp-msB} \cite{dp-edP} \cite{dp-edH} \cite{dp-edE} \cite{dp-edN} त‚वात् । य‚त्रावी० \cite{dp-msC}}}--\edtext{\textsuperscript{*}}{\lemma{*}\Bfootnote{य‚त्र वीत० \cite{dp-msB} \cite{dp-edP} \cite{dp-edH} \cite{dp-edE}}}य‚त्राऽवीत‚राग‚त्वं नास्ति, \edtext{\textsuperscript{*}}{\lemma{*}\Bfootnote{नास्ति स व‚क्ता \cite{dp-msB} \cite{dp-edP} \cite{dp-edH} \cite{dp-edE} नास्ति स न व‚क्ता \cite{dp-msC}}}न स व‚क्ता । य‚था--उप‚ल‚ख‚ण्ड‚{\tiny $_{lb}$}‚ इति । य‚द्य‚प्युप‚ल‚ख‚ण्डादुभ‚यं व्यावृत्तं \edtext{}{\lemma{व्यावृत्तं}\Bfootnote{व्यावृत्त‚या स‚र्वो--\cite{dp-msB} \cite{dp-edP} \cite{dp-edH} व्यावृत्तं यो स‚र्वो० \cite{dp-edE} ०वृत्तं त‚था स‚र्वो० \cite{dp-msC}}}त‚थापि स‚र्वो वीत‚रागो न व‚क्तेति‚{\tiny $_{lb}$}‚ व्याप्त्या व्य‚तिरेकासिद्धेर‚व्य‚तिरेकः ॥ १३३ ॥
	\pend% ending standard par
      
	  \endgroup
	
	  \endgroup
	‚{\tiny $_{lb}$}‚

	  \pstart \leavevmode% starting standard par
	उभ‚य‚श\edtext{}{\lemma{श}\Bfootnote{उभ‚श}}ब्द‚स्य द्विव‚च‚नान्त‚स्य प्र‚योग‚द‚र्श‚नादुभ‚योरित्युभ‚श‚ब्देनार्थ‚माह । ल‚ब्ध‚मिदं‚{\tiny $_{lb}$}‚ व‚स्तु म‚त्तोऽन्य‚त्र न‚राम‚दि \edtext{}{\lemma{दि}\Bfootnote{मागादि}}ति तु विशेषोऽत्र मात्स‚र्य‚म‚भिप्रेत \textbf{आग्र‚हः} । न मुञ्च‚ति‚{\tiny $_{lb}$}‚\edtext{}{\lemma{ति}\Bfootnote{न्ति}} नान्य‚स्मै द‚द‚ति । अनेनैव रूपेण वैध‚र्म्योदाह‚र‚णं भ‚व‚ति । नान्य‚थेति द्र‚ढ‚यितुमुक्त‚म‚पि‚{\tiny $_{lb}$}‚ स्मार‚य‚न्नाह--\textbf{य‚त्रे}ति । \textbf{ऋष‚भादीनामि}त्य‚नेन वैध‚र्म्योदाह‚र‚णाद् \textbf{ऋष‚भादेः} साध्य‚साध‚न‚यो‚{\tiny $_{lb}$}‚र्व्यावृत्तिस‚न्देहं द‚र्श‚य‚ति ।
	\pend% ending standard par
      ‚{\tiny $_{lb}$}‚

	  \pstart \leavevmode% starting standard par
	न‚न्व‚स्म‚दाग‚मे त‚द्गुण‚द्व‚य‚योगिन‚स्ते क‚थितास्त‚त्क‚थ‚म‚न‚योस्त‚तो व्यावृत्तिः स‚न्दिह्य‚त‚{\tiny $_{lb}$}‚ इत्याह--\textbf{य‚दि नामे}ति । \textbf{प‚ठ्य‚न्त} इति च व‚च‚न‚व्य‚क्त्या च पाठ‚मात्रेण तेषां त‚द्गुण‚योगः सिद्धः‚{\tiny $_{lb}$}‚ न तु प्र‚माणेनेति द‚र्श‚य‚ति । अत एवाह--त\textbf{थापी}ति ॥
	\pend% ending standard par
      ‚{\tiny $_{lb}$}‚

	  \pstart \leavevmode% starting standard par
	\textbf{त्रीनि}ति दृष्टान्त‚दोषान् ।
	\pend% ending standard par
      ‚{\tiny $_{lb}$}‚‚{\tiny $_{lb}$}‚‚{\tiny $_{lb}$}‚‚{\tiny $_{lb}$}‚\textsuperscript{\textenglish{249/dm}}‚{\tiny $_{lb}$}‚
	  \bigskip
	  \begingroup
	

	  \pstart \leavevmode% starting standard par
	अविद्य‚मानो व्य‚तिरेको य‚स्मिन् सोऽव्य‚तिरेकः । अवीत‚राग इति रागादिम‚त्त्वं साध्य‚म् ।‚{\tiny $_{lb}$}‚ व‚क्तृत्वादिति हेतुः । इह व्य‚तिरेक‚माह--य‚त्रावीत‚राग‚त्वं नास्तीति साध्याभावानुवादः । त‚त्र‚{\tiny $_{lb}$}‚ व‚क्तृत्व‚म‚पि नास्ति--इति साध‚नाभाव‚विधिः । तेन साध‚नाभावेन साध्याभावो व्याप्त उक्तः ।‚{\tiny $_{lb}$}‚ \edtext{\textsuperscript{*}}{\lemma{*}\Bfootnote{अत्र दृष्टा० \cite{dp-msD}}}दृष्टान्तो य‚थोप‚ल‚ख‚ण्ड \edtext{}{\lemma{ण्ड}\Bfootnote{०ख‚ण्डेति० \cite{dp-msA} \cite{dp-msB} \cite{dp-edP} \cite{dp-edH}}}इति ।
	\pend% ending standard par
       ‚{\tiny $_{lb}$}‚ 

	  \pstart \leavevmode% starting standard par
	क‚थ‚म‚व्य‚म‚व्य‚तिरेको याव‚तोप‚ल‚ख‚ण्डादुभ‚यं\edtext{}{\lemma{यं}\Bfootnote{०भ‚य‚म‚पि नि० \cite{dp-msD}}} निवृत्त‚म् ? किम‚तः ?\edtext{\textsuperscript{*}}{\lemma{*}\Bfootnote{य‚द्युप‚ल० \cite{dp-msA} \cite{dp-msB} \cite{dp-edP} \cite{dp-edH} \cite{dp-edE}}} य‚द्य‚पि उप‚ल‚{\tiny $_{lb}$}‚ख‚ण्डादुभ‚यं व्यावृत्तं स‚राग‚त्वं च व‚क्तृत्वं च\edtext{}{\lemma{च}\Bfootnote{य‚त्र वीत‚राग‚त्वं त‚त्र व‚क्तृत्वं नास्ति--\cite{dp-msD-n}}}, त‚थापि व्याप्त्या\edtext{}{\lemma{व्याप्त्या}\Bfootnote{व्याप्तो व्य० \cite{dp-msC}}} व्य‚तिरेको य‚स्त‚स्याऽसिद्धेः‚{\tiny $_{lb}$}‚ कार‚णाद् अव्य‚तिरेकोऽय‚म् ।
	\pend% ending standard par
       ‚{\tiny $_{lb}$}‚ 

	  \pstart \leavevmode% starting standard par
	कीदृशी पुन‚र्व्याप्तिरित्याह--स‚र्वो वीत‚राग इति साध्याभावानुवादः । न व‚क्तेति‚{\tiny $_{lb}$}‚ साध‚नाभाव‚विधिः । तेन साध्याभावः साध‚नाभाव\edtext{}{\lemma{नाभाव}\Bfootnote{साध‚नाभावे नि० \cite{dp-msD} साध‚नाभावो नि० \cite{dp-msC}}}निय‚तः \edtext{}{\lemma{तः}\Bfootnote{स्थापितो--\cite{dp-msA}}}ख्यापितो भ‚व‚तीति\edtext{}{\lemma{तीति}\Bfootnote{भ‚व‚ति । ई० \cite{dp-msD} \cite{dp-msB}}} । इदृशी‚{\tiny $_{lb}$}‚ व्याप्तिः । त‚या व्य‚तिरेको न सिद्धः । अस्य चार्थ‚स्य प्र‚सिद्ध‚ये दृष्टान्तः । त‚त् स्व‚कार्या‚{\tiny $_{lb}$}‚ ऽक‚र‚णाद् दुष्टः ॥
	\pend% ending standard par
       ‚{\tiny $_{lb}$}‚ 
	  \bigskip
	  \begingroup
	

	  \pstart \leavevmode% starting standard par
	अप्र‚द‚र्शित‚व्य‚तिरेको य‚था--अनित्यः श‚ब्दः, कृत‚क‚त्वादाकाश‚व‚दिति‚{\tiny $_{lb}$}‚ व‚ध‚र्म्येण\edtext{}{\lemma{र्म्येण}\Bfootnote{नास्ति वैध‚र्म्येण \cite{dp-edE} वैध‚र्म्येणापि \cite{dp-msB} \cite{dp-msD} \cite{dp-edP} \cite{dp-edH} न मूल‚त्वेनापि तु टीकास्थं‚{\tiny $_{lb}$}‚ गृहीतं \cite{dp-edN} प्र‚तौ ।}} ॥१३४॥
	\pend% ending standard par
      
	  \endgroup
	
	  \endgroup
	‚{\tiny $_{lb}$}‚

	  \pstart \leavevmode% starting standard par
	येनाय‚म‚नुवाद‚विधिक्र‚म‚स्तेन हेतुना । \textbf{याव‚ते}ति तृतीयान्त‚प्र‚तिरूप‚को य‚स्मादित्य‚स्यार्थे व‚र्त्त‚मानोऽत्र‚{\tiny $_{lb}$}‚ गृहीतः । उ\leavevmode\ledsidenote{\textenglish{81b/ms}}\textbf{प‚ल‚ख‚ण्डा}च्छिलाश‚क‚लात् । \textbf{उभ}यं साध्य‚साध‚न‚म् । \textbf{किम‚त} इति सिद्धान्ती ।‚{\tiny $_{lb}$}‚ अत उभ‚य‚निवृत्ते किं भ‚व‚ति ? न किञ्चिदित्य‚र्थः । न‚नूप‚ल‚ख‚ण्डात्ताव‚द् वैध‚र्मीदृष्टान्तादुभ‚यं‚{\tiny $_{lb}$}‚ निवृत्तं द‚र्शितं येन त‚त्किमेव‚मुच्य‚त इत्याह--\textbf{य‚द्य‚पी}ति । \textbf{व्याप्त्या} स‚र्व‚रागित्व‚ज‚न्य‚तास्वीकारेण ।‚{\tiny $_{lb}$}‚ त‚स्य व्याप्तिम‚तो व्य‚तिरेक‚स्याऽ\textbf{सिद्धे}र‚निश्च‚यात् । \textbf{कीदृशी}ति सामान्य‚तः पृच्छ‚ति । \textbf{पुन}रिति‚{\tiny $_{lb}$}‚ विशेष‚तः । \textbf{इति}र‚न‚न्त‚रोक्तं श‚ब्दं प‚रामृश‚ति । \textbf{तेनाने}न श‚ब्देनेत्य‚र्थः । न व‚क्त‚व्येत्य‚त्रापीति पूर्व‚व‚त् ।‚{\tiny $_{lb}$}‚ येनैव‚म‚नुवाद‚विधिक्र‚म\textbf{स्तेन । इति}रीदृश्या व्याप्तेः स्व‚रूपं प्र‚काश‚य‚ति य‚स्माद‚र्थे वा । त‚त्प्र‚तीत्थं‚{\tiny $_{lb}$}‚भूत‚ल‚क्ष‚ण‚त‚येयं तृतीया, साध‚नाभावेन साध्याभाव‚न्यायेन ल‚क्ष‚णा व्याप्तिरीदृशीत्य‚र्थः । ईदृशं‚{\tiny $_{lb}$}‚ व्याप्तिम‚न्तं व्य‚तिरेकं प्र‚साध‚यितुम‚स‚म‚र्थोऽयं क‚थ‚म‚यं दृष्टान्तोऽत्रेष्ट इत्याह--\textbf{अस्य चे}ति । \textbf{अस्य}‚{\tiny $_{lb}$}‚ साध‚नाभावे साध्याऽभाव‚निय‚त‚ल‚क्ष‚ण‚स्या\textbf{र्थ‚स्य । चो} य‚स्माद‚र्थेऽव‚धार‚णे वा । \textbf{प्र‚सिद्ध‚ये} निश्च‚यार्थं‚{\tiny $_{lb}$}‚ दृ\textbf{ष्टान्त} उपादीय‚त इति शेषः, प्र‚क‚र‚ण‚ल‚भ्यं वा, न चायं त‚थाप्र‚द‚र्श‚न इत्य‚मुम‚र्थं प्र‚क‚र‚ण‚{\tiny $_{lb}$}‚ग‚म्यं कृत्वा । \textbf{त‚त्त}स्मात् \textbf{स्व‚कार्याक‚र‚णाद् दुष्ट} इत्युक्त‚म् । य‚द् वा भ‚व‚त्व‚स्यार्थ‚स्य प्र‚सिद्ध‚ये‚{\tiny $_{lb}$}‚ ‚{\tiny $_{lb}$}‚ ‚{\tiny $_{lb}$}‚ ‚{\tiny $_{lb}$}‚ ‚{\tiny $_{lb}$}‚ ‚{\tiny $_{lb}$}‚ ‚{\tiny $_{lb}$}‚ ‚{\tiny $_{lb}$}‚ ‚{\tiny $_{lb}$}‚ ‚{\tiny $_{lb}$}‚ \leavevmode\ledsidenote{\textenglish{250/dm}}‚{\tiny $_{lb}$}‚ 
	  
	अप्र‚द‚र्शितो व्य‚तिरेको य‚स्मिन् स त‚थोक्तः । अनित्यः श‚ब्द इत्य‚नित्य‚त्वं साध्य‚म् ।‚{\tiny $_{lb}$}‚ कृत‚क‚त्वादिति हेतुः । आकाश‚व‚दिति वैध‚र्म्येण दृष्टान्तः । ‚{\tiny $_{lb}$}‚ 
	  
	इह प‚रार्थानुमाने प‚र‚स्माद‚र्थः प्र‚तिप‚त्त‚व्यः । स शुद्धोऽपि स्व‚तो य‚दि प‚रेणाशुद्धः‚{\tiny $_{lb}$}‚ ख्याप्य‚ते स ताव‚द्य‚था प्र‚काशित‚स्त‚था न युक्तः । य‚था युक्त‚स्त‚था न प्र‚काशितः । प्र‚काशित‚श्च‚{\tiny $_{lb}$}‚ हेतुः । अतो व‚क्तुर‚प‚राधाद‚पि प‚रार्थानुमाने हेतुर्दृष्टान्तो वा दुष्टः स्याद‚पि । न च सादृश्या‚{\tiny $_{lb}$}‚व‚सादृश्याद्वा साध्य‚प्र‚तिप‚त्तिः, अपि तु साध्य‚निय‚ताद्धेतोः । अतः साध्य‚निय‚तो हेतुर‚न्व‚य‚वाक्येन‚{\tiny $_{lb}$}‚ व्य‚तिरेक‚वाक्येन \edtext{}{\lemma{वाक्येन}\Bfootnote{च व‚क्त० \cite{dp-msA} \cite{dp-msB} \cite{dp-edP} \cite{dp-edH} \cite{dp-edE} \cite{dp-edN}}}वा व‚क्त‚व्यः । अन्य‚था ग‚म‚को नोक्तः स्यात् । स त‚थोक्तो दृष्टान्तेन\edtext{}{\lemma{दृष्टान्तेन}\Bfootnote{दृष्टान्तेनासिद्धो \cite{dp-msB}}}‚{\tiny $_{lb}$}‚ दृष्टान्त‚स्त‚थाप्य‚यं क‚थं दुष्ट इत्याह--\textbf{त‚त्स्व‚कार्ये}ति । त‚च्च \textbf{त‚त्स्व‚कार्यं} च । \textbf{साध‚नाभावे}‚{\tiny $_{lb}$}‚ साध्याभाव‚निय‚त‚ख्याप‚न‚ल‚क्ष‚णं चेति । त‚था त‚स्या\textbf{क‚र‚णाद}स‚म्पाद‚नाद् \textbf{दुष्ट} इति ॥
	\pend% ending standard par
      ‚{\tiny $_{lb}$}‚

	  \pstart \leavevmode% starting standard par
	अथ प‚र‚मार्थ‚त‚स्ताव‚द् दृष्टान्ते न‚भ‚सि साध्याभावोऽप्य‚स्ति, साध‚नाभाव‚श्च । त‚त्क‚थ‚म‚{\tiny $_{lb}$}‚प्र‚द‚र्शित‚व्य‚तिरेको दृष्टान्तो दुष्ट इत्याह--\textbf{इहे}ति । \textbf{प‚र‚स्मा}त्साध‚न‚वादिनः । \textbf{अर्थो} हेतुल‚क्ष‚णः ।‚{\tiny $_{lb}$}‚ प्र‚क‚र‚णात्साध्याभावे साध‚नाभाव‚ल‚क्ष‚ण‚श्च । \textbf{स स्व‚तः शुद्धो} व‚स्तुवृत्त्या प‚रिशुद्धः । त‚थात्वेन‚{\tiny $_{lb}$}‚ विद्य‚मान इति याव‚त् । न केव‚ल‚म‚शुद्धः--इत्य‚पिश‚ब्दात् । \textbf{प‚रेण} साध‚न‚प्र‚योक्त्रा । \textbf{य‚था‚{\tiny $_{lb}$}‚ प्र‚काशितो} व्य‚तिरेक‚मात्र‚वान् प्र‚काशितो व्याप्तिशून्य‚श्च प्र‚काशितः । \textbf{त‚था न युक्तो} नोप‚युक्तः‚{\tiny $_{lb}$}‚ साध्य‚सिद्धौ । य‚दि हि साध्याभावानुवादेन साध्या\edtext{}{\lemma{साध्या}\Bfootnote{ध‚ना}}भावो विधीय‚ते दृष्टान्ते प्र‚द‚र्श्येतैव‚{\tiny $_{lb}$}‚म‚सौ हेतुः साध्य‚सिद्ध्य‚ङ्ग‚व्य‚तिरेक‚वान् सिद्ध्येत् । एव‚मेव चाऽसौ व्याप्तिम‚द्व्य‚तिरेकः प्र‚सिद्ध्येत् ।‚{\tiny $_{lb}$}‚ त‚त्प्र‚द‚र्श‚न‚श्च दृष्टान्तोऽदुष्टो भ‚वेदित्य‚भिप्रायः ।
	\pend% ending standard par
      ‚{\tiny $_{lb}$}‚

	  \pstart \leavevmode% starting standard par
	य‚दि नाम त‚था न प्र‚काशित‚स्त‚थापि त‚दुप‚योगी हेतुर्दृष्टान्तो वा त‚था किं न प्र‚तिप‚द्य‚त‚{\tiny $_{lb}$}‚ इत्याह--\textbf{प्र‚काशित‚श्चे}ति । \textbf{चो} य‚स्माद‚र्थे । \textbf{हेतुरि}त्युप‚ल‚क्ष‚ण‚म् । तेन दृष्टान्तोऽपि द्र‚ष्ट‚व्यः ।‚{\tiny $_{lb}$}‚ अ\edtext{}{\lemma{अ}\Bfootnote{य}}त एव‚म् \textbf{अतो} अस्माद् हेतोः । न केव‚लं व‚चोव्य‚व‚स्थिताद् दोषादित्य‚पिश‚ब्देनाह ।‚{\tiny $_{lb}$}‚ य‚द्य‚पि दृष्टान्त एव प्र‚कृत‚स्त‚थापि हेतुर‚प्येवंविधः स्व‚तोऽदुष्टोऽपि व‚क्तृदोषादेव दुष्य‚तीति‚{\tiny $_{lb}$}‚ तुल्य‚न्याय‚त‚या प्र‚स‚ङ्ग‚न द‚र्शित‚म् । य‚द्वा य‚थैवंविधो हेतुर्व‚क्तृदोषाद् दुष्टो भ‚व‚ति त‚द्व‚द् दृष्टान्तोऽ‚{\tiny $_{lb}$}‚पीति दृष्टान्तार्थं हेतोः व‚क्त्र‚प‚राधेना\edtext{}{\lemma{राधेना}\Bfootnote{न}} दुष्ट‚त्व‚ख्याप‚नं कृत‚मिति स‚र्व‚म‚व‚दात‚म् ।
	\pend% ending standard par
      ‚{\tiny $_{lb}$}‚

	  \pstart \leavevmode% starting standard par
	\leavevmode\ledsidenote{\textenglish{82a/ms}} न‚नु च य‚था कृत‚क‚त्वेनाकाश‚विध‚र्मा श‚ब्दः प्र‚तीय‚ते त‚थाऽनित्य‚त्वेनापि त‚द्विध‚र्मा‚{\tiny $_{lb}$}‚ भ‚विष्य‚ति त‚त्क‚थ‚म‚नुप‚युक्त इत्याह--\textbf{न चे}ति । \textbf{चो}ऽव‚धार‚णे हेतौ वा । एवंव‚द‚तोऽय‚माश‚यः—‚{\tiny $_{lb}$}‚य‚द्येकेन ध‚र्मेण वैध‚र्म्य\edtext{}{\lemma{र्म्य}\Bfootnote{र्म्ये}} प्र‚तीतेऽप‚रेणापि त‚द्वैध‚म्य‚प्र‚तीतिर‚व‚श्य‚म्भाविनी, त‚दा मूर्त्त‚त्वेनापि‚{\tiny $_{lb}$}‚ श‚ब्द‚स्य त‚द्वैध‚र्म्य‚प्र‚तीतिः प्र‚स‚ज्येतेति । तुल्य‚न्याय‚त‚याऽन्व‚य‚वाक्य‚म‚धिकृत्य \textbf{सादृश्यादि}त्युक्त‚म् ।‚{\tiny $_{lb}$}‚ \textbf{सादृश्याद‚सादृश्याद्} वेति साध‚र्म्य‚वैध‚र्म्य‚दृष्टान्त‚प्र‚तिपादितादिति प्र‚क‚र‚णात् ।
	\pend% ending standard par
      ‚{\tiny $_{lb}$}‚

	  \pstart \leavevmode% starting standard par
	व्य‚तिरेक‚वाक्येनापि साध‚नाभावेनापि निय‚म‚ख्याप‚न‚द्वारा साध्य एव हेतोर्निय‚त‚त्व‚{\tiny $_{lb}$}‚ख्याप‚नाद् \textbf{व्य‚तिरेक‚वाक्येन चे}त्युक्त‚म् ।
	\pend% ending standard par
      ‚{\tiny $_{lb}$}‚‚{\tiny $_{lb}$}‚\textsuperscript{\textenglish{251/dm}}‚{\tiny $_{lb}$}‚
	  \bigskip
	  \begingroup
	

	  \pstart \leavevmode% starting standard par
	सिद्धो द‚र्श‚यित‚व्यः । त‚स्माद् दृष्टान्तो नामान्व‚य‚व्य‚तिरेक‚वाक्यार्थ‚प्र‚द‚र्श‚नः\edtext{}{\lemma{नः}\Bfootnote{प्र‚द‚र्श‚नार्थे । \cite{dp-msC} \cite{dp-msD}}} । न चेह व्य‚तिरेक‚{\tiny $_{lb}$}‚वाक्यं प्र‚युक्त‚म् । अतो वैध‚र्म्य‚दृष्टान्त इहासादृश्य‚मात्रेण\edtext{}{\lemma{मात्रेण}\Bfootnote{सादृश्य‚भावेन साध० \cite{dp-msA} \cite{dp-msB} \cite{dp-edP} \cite{dp-edH} \cite{dp-edN}}} साध‚क उप‚न्य‚स्तः । न च त‚था‚{\tiny $_{lb}$}‚ साध‚कः । व्य‚तिरेक‚विष‚य‚त्वेन स साध‚कः । च न त‚थोप‚न्य‚स्त इति\edtext{}{\lemma{इति}\Bfootnote{इति । अतोऽप्र० \cite{dp-msA} \cite{dp-edP} \cite{dp-edH} \cite{dp-edE} \cite{dp-edN} इति । अप्र० \cite{dp-msC}}} अय‚म‚प्र‚द‚र्शित‚{\tiny $_{lb}$}‚व्य‚तिरेको व‚क्तुर‚प‚राधाद् दुष्टः ॥
	\pend% ending standard par
       ‚{\tiny $_{lb}$}‚ 
	  \bigskip
	  \begingroup
	

	  \pstart \leavevmode% starting standard par
	विप‚रीत‚व्य‚तिरेको \edtext{}{\lemma{तिरेको}\Bfootnote{य‚था नास्ति \cite{dp-msD}}}य‚था--य‚द‚कृत‚कं त‚न्नित्यं भ‚व‚तीति ॥१३५॥
	\pend% ending standard par
      
	  \endgroup
	‚{\tiny $_{lb}$}‚ 

	  \pstart \leavevmode% starting standard par
	\edtext{\textsuperscript{*}}{\lemma{*}\Bfootnote{विप‚रीतो इत्यार‚भ्य त‚मुदाह‚र‚ति प‚र्य‚न्तः पाठो दुर्वेक‚स‚मीप‚स्थैक‚स्मिन्नाद‚र्शे नासी‚{\tiny $_{lb}$}‚दिति व्याख्यानुरोधात् ज्ञाय‚ते--सं०}}विप‚रीतो व्य‚तिरेको य‚स्मिन् वैध‚र्म्य‚दृष्टान्ते स त‚थोक्तः । त‚मुदाह‚र‚ति--\edtext{\textsuperscript{*}}{\lemma{*}\Bfootnote{य‚था नास्ति \cite{dp-msA} \cite{dp-msB} \cite{dp-edP} \cite{dp-edH} \cite{dp-edE} \cite{dp-edN}}}य‚था य‚द-
	\pend% ending standard par
      
	  \endgroup
	‚{\tiny $_{lb}$}‚

	  \pstart \leavevmode% starting standard par
	अन्व‚य‚वाक्ये साध‚न‚म‚नूद्य साध्यं विधात‚व्य‚म् । व्य‚तिरेक‚वाक्ये च साध्याभाव‚म‚नूद्य‚{\tiny $_{lb}$}‚ साध‚नाभावो विधात‚व्यः । त‚थैव हेतोः साध्य‚निय‚त‚त्वाभिधानादित्य‚स्यार्थः ।
	\pend% ending standard par
      ‚{\tiny $_{lb}$}‚

	  \pstart \leavevmode% starting standard par
	क‚स्माद‚सौ वाक्य‚द्व‚येनाप्येव व‚क्त‚व्य इत्याह--\textbf{अन्य‚थेति} । अस्माद‚न्येन प्र‚कारेण‚{\tiny $_{lb}$}‚ \textbf{ग‚म‚कः} प‚रोक्षार्थ‚प्र‚काश‚को \textbf{नोक्तः स्यात्} ।
	\pend% ending standard par
      ‚{\tiny $_{lb}$}‚

	  \pstart \leavevmode% starting standard par
	काम‚म‚सावेव‚मुच्य‚ताम् । दृष्टान्त‚स्तु क‚थ‚म‚त्राधिक्रिय‚त इत्याह--\textbf{स} इति । \textbf{स} हेतु‚{\tiny $_{lb}$}‚\textbf{स्त‚थोक्तः} साध्य‚निय‚त उक्तः । \textbf{दृष्टान्तेन} साध‚र्म्य‚व‚ता वैध‚र्म्य‚व‚ता च क‚र‚णेन \textbf{सिद्धो निश्चितो‚{\tiny $_{lb}$}‚ द‚र्श‚यित‚व्यः} ।
	\pend% ending standard par
      ‚{\tiny $_{lb}$}‚

	  \pstart \leavevmode% starting standard par
	न‚न्वेव‚म‚पि न ज्ञाय‚ते किंव्यापारो दृष्टान्त इहोप‚युज्य‚ते इत्याश‚ङ्क्योप‚संहार‚व्याजेनाह—‚{\tiny $_{lb}$}‚\textbf{त‚स्मादि}ति । \textbf{नाम}श‚ब्दः प्र‚सिद्धाविह । \textbf{अन्व‚य‚व्य‚तिरेक‚वाक्य}योर‚था \edtext{}{\lemma{था}\Bfootnote{र्थोऽ}} भिधेयः--\textbf{उक्तेन}‚{\tiny $_{lb}$}‚ प्र‚कारेण हेतोः साध्य‚निय‚त‚त्व‚म्--प्र‚द‚र्श्यः । तं प्र‚द‚र्श‚य‚तीति त‚था । य‚द्वा प्र‚द‚र्श्य‚तेऽनेनेति‚{\tiny $_{lb}$}‚ प्र‚द‚र्श्य‚तेऽस्मिन्निति \textbf{प्र‚द‚र्श‚नः} । त‚स्य \textbf{प्र‚द‚र्श‚न} इति विग्र‚हः ।
	\pend% ending standard par
      ‚{\tiny $_{lb}$}‚

	  \pstart \leavevmode% starting standard par
	य‚द्येव‚म‚य‚म‚पि वैध‚र्म्य‚दृष्टान्त‚स्त‚थाकार्येवात्रोप‚योज्य‚त इत्याह--\textbf{न चेति । चो}‚{\tiny $_{lb}$}‚ य‚स्माद‚र्थे । \textbf{इह} प्र‚योगे । \textbf{व्य‚तिरेक}ख्याप‚कं साध्याभावानुवादेन साध‚नाभाव‚विधाय‚कं‚{\tiny $_{lb}$}‚ \textbf{वाक्}य‚मित्य‚र्थः । अत‚स्त‚थाभूत‚वाक्य‚प्र‚योगाद वैध‚र्म्य‚दृष्टान्त आकाशः । \textbf{इहा}नित्य‚त्व‚साध‚न‚{\tiny $_{lb}$}‚प्र‚योगे साध्य‚ध‚र्मिणोऽ\textbf{सादृश्यं} केव‚लं य‚त्त‚न्मात्रेण त‚न्मात्र‚प्र‚द‚र्श‚नेन, विष‚येण विष‚यिणो निर्देशात् ।‚{\tiny $_{lb}$}‚ \textbf{साध‚को} निश्चाय‚को हेतोः साध्य‚निय‚त‚त्व‚स्येति प्र‚क‚र‚णात् ।
	\pend% ending standard par
      ‚{\tiny $_{lb}$}‚

	  \pstart \leavevmode% starting standard par
	य‚दि त‚न्मात्रेणापि साध‚क‚स्त‚दा का क्ष‚तिरित्याह--\textbf{न चे}ति । \textbf{त‚थे}त्य‚सादृश्य‚मात्रेण ।‚{\tiny $_{lb}$}‚ वाद्युक्तेन ध‚र्मेण साध्य‚ध‚र्मिणोऽसादृश्याव‚ग‚मे ध‚र्मान्त‚रेणाप्य‚सादृश्याव‚ग‚मोऽव‚श्य‚म्भावीति युज्य‚ते‚{\tiny $_{lb}$}‚ऽतिप्र‚स‚ङ्गादित्य‚भिप्रायः ।
	\pend% ending standard par
      ‚{\tiny $_{lb}$}‚‚{\tiny $_{lb}$}‚‚{\tiny $_{lb}$}‚‚{\tiny $_{lb}$}‚‚{\tiny $_{lb}$}‚‚{\tiny $_{lb}$}‚\textsuperscript{\textenglish{252/dm}}‚{\tiny $_{lb}$}‚
	  \bigskip
	  \begingroup
	

	  \pstart \leavevmode% starting standard par
	कृत‚क‚मित्यादि । इहान्व‚य\edtext{}{\lemma{य}\Bfootnote{व्य‚तिरेक‚वाक्याभ्यां \cite{dp-msA} \cite{dp-msB} \cite{dp-msD} \cite{dp-edP} \cite{dp-edH} \cite{dp-edE} \cite{dp-edN}}} व्य‚तिरेकाभ्यां साध्य‚निय‚तो हेतुर्द‚र्श‚यित‚व्यः । य‚दा च साध्य‚निय‚तो‚{\tiny $_{lb}$}‚ हेतुर्द‚र्श‚यित‚व्य‚स्त‚दा व्य‚तिरेक‚वाक्ये साध्याभावः साध‚नाभावे निय‚तो द‚र्श‚यित‚व्यः । एवं हि‚{\tiny $_{lb}$}‚ हेतुः साध्य‚निय‚तो द‚र्शितः स्यात् । य‚दि तु साध्याभावः साध‚नाभावे निय‚तो नाख्याय‚ते साध‚न‚{\tiny $_{lb}$}‚स‚त्तायाम‚पि साध्याभावः स‚म्भाव्येत\edtext{}{\lemma{म्भाव्येत}\Bfootnote{स‚म्भाव्य‚ते \cite{dp-msB}}} । त‚था च साध‚नं साध्य‚निय‚तं न प्र‚तीयेत\edtext{}{\lemma{तीयेत}\Bfootnote{प्र‚तीय‚ते \cite{dp-msB} \cite{dp-msC}}} । त‚स्मात्‚{\tiny $_{lb}$}‚ साध्याभावः साध‚नाभावे निय‚तो व‚क्त‚व्यः । विप‚रीत‚व्य‚तिरेके च साध‚नाभावः साध्याभावे‚{\tiny $_{lb}$}‚ निय‚त उच्य‚ते । न साध्याभावः साध‚नाभावे । त‚था हि--य‚द‚कृत‚क‚मिति साध‚नाभाव‚म‚नूद्य‚{\tiny $_{lb}$}‚ त‚न्नित्य‚मिति साध्याभाव‚विधिः ।
	\pend% ending standard par
       ‚{\tiny $_{lb}$}‚ 

	  \pstart \leavevmode% starting standard par
	त‚तोऽय‚म‚र्थः--अकृत‚को नित्य एव । त‚था च स‚ति अकृत‚क‚त्वं नित्य‚त्वे साध्याभावे‚{\tiny $_{lb}$}‚ निय‚त‚मुक्त‚म्, न नित्य‚त्वं साध‚नाभावे । त‚तो न साध्य‚निय‚तं हेतुं व्य‚तिरेक‚वाक्य‚माह । त‚था‚{\tiny $_{lb}$}‚ च विप‚रीत‚व्य‚तिरेकोऽपि व‚क्तुर‚प‚राधाद् दुष्टः ॥
	\pend% ending standard par
      
	  \endgroup
	‚{\tiny $_{lb}$}‚

	  \pstart \leavevmode% starting standard par
	य‚द्येव‚म‚साध‚कः क‚थं नाम साध‚क इत्याह--\textbf{व्य‚तिरेके}ति । \textbf{व्य‚तिरेक‚वि}ष‚य‚त्वेनेति‚{\tiny $_{lb}$}‚ व्य‚तिरेक‚प्र‚तिप‚त्तिविष‚य‚त्वेन । \textbf{चो} य‚स्माद‚र्थे, व्य‚क्त‚मेत‚दित्य‚स्मिन्न‚थे वा । अनेनापि त‚थैवो‚{\tiny $_{lb}$}‚प‚न्य‚स्त इत्याह--\textbf{न चे}ति । \textbf{चो}ऽव‚धार‚णे । व्य‚तिरेक‚वाक्य‚म‚नुक्त्वैव त‚स्योपादानादित्य‚भिप्रायः ।‚{\tiny $_{lb}$}‚ \textbf{इति}स्त‚स्माद‚थे\unclear{} एव‚म‚र्थे\unclear{} वा ॥
	\pend% ending standard par
      ‚{\tiny $_{lb}$}‚

	  \pstart \leavevmode% starting standard par
	विप‚रीत‚व्य‚तिरेकं व्याच‚क्षाण आह--\textbf{य‚दा चे}ति । विप‚रीतान्व‚य‚श \leavevmode\ledsidenote{\textenglish{82b/ms}} ब्द‚स्य‚{\tiny $_{lb}$}‚ व्युत्प‚त्तौ द‚र्शितायां विप‚रीत‚व्य‚तिरेक‚श‚ब्द‚स्यापि--\textbf{विप‚रीतो} वैप‚रीत्येन प्र‚द‚र्श‚नाद् \textbf{व्य‚तिरेको‚{\tiny $_{lb}$}‚ य‚स्मिन् दृष्टान्ते स त‚थोक्त} इति--व्युत्प‚त्तिर्द‚र्शिता भ‚व‚त्येवेति चाभिप्रायेण नोक्ता । \textbf{त‚मुदा‚{\tiny $_{lb}$}‚ह‚र‚ती}ति सुज्ञान‚त्वान्नोक्त‚मिति प्र‚तिप‚त्त‚व्य‚म् । य‚त्र तु पुस्त‚के \textbf{विप‚रीत‚व्य‚तिरेको य‚थे}त्य‚स्य‚{\tiny $_{lb}$}‚ मूल‚स्य व्याख्यान‚ग्र‚न्थोऽस्ति त‚त्र स‚र्व‚म‚व‚दात‚म् । अस‚तीयं ग‚तिर‚स्माभिर्द‚र्शिता ।
	\pend% ending standard par
      ‚{\tiny $_{lb}$}‚

	  \pstart \leavevmode% starting standard par
	न‚नु किं नाम व्य‚तिरेक‚वाक्येन द‚र्श‚नीय‚म् ? य‚द्वैप‚रीत्येन द‚र्श‚नाद‚यं विप‚रीत‚व्य‚तिरेक‚{\tiny $_{lb}$}‚ उच्य‚त इत्याह--\textbf{य‚दा} चेति । \textbf{चो}ऽव‚धार‚णे । \textbf{साध्य‚निय‚त} इत्य‚स्मात्प‚रः प्र‚तिप‚त्त‚व्यः । \textbf{य‚दा}‚{\tiny $_{lb}$}‚ य‚स्मिन् काले प्र‚तिपाद‚न‚काल इत्य‚र्थात् । कालान्त‚रे त‚थाप्र‚द‚र्श‚नानुप‚प‚त्तेः । हेतोः साध्ये‚{\tiny $_{lb}$}‚ निय‚त‚त्व‚प्र‚द‚र्श‚न‚ञ्चाव‚श्य‚कार्य‚म‚न्य‚था ग‚म‚को नोक्तः स्यादित्य‚भिप्रायः । \textbf{त‚दा} त‚स्मिन् काले ।‚{\tiny $_{lb}$}‚ साध्याभावानुवादेन साध‚नाभावोऽभिधात‚व्य इत्य‚स्यार्थः । क‚स्मात्पुन‚रेवं द‚र्श‚यित‚व्य इत्याह—‚{\tiny $_{lb}$}‚\textbf{एवं ही}ति । \textbf{ही}ति य‚स्मात् । \textbf{एवं} साध्याभाव‚स्य साध‚नाभावे निय‚त‚त्व‚प्र‚द‚र्श‚न‚प्र‚कारे स‚ति ।
	\pend% ending standard par
      ‚{\tiny $_{lb}$}‚

	  \pstart \leavevmode% starting standard par
	अथान्य‚थाप्र‚द‚र्श‚नेऽपि य‚दि साध‚नं साध्य‚निय‚तं प्र‚तीय‚ते त‚दा त‚थाप्र‚द‚र्श‚नेनैव किं प्र‚योज‚न‚{\tiny $_{lb}$}‚मित्याह--\textbf{य‚दि} त्विति । तुस्त‚थाऽनाख्यानाव‚स्थां भेद‚व‚तीं द‚र्श‚य‚ति । \textbf{त‚था च} साध‚न‚स‚त्ताया‚{\tiny $_{lb}$}‚म‚पि साध्याभाव‚स‚म्भाव‚नाप्र‚कारे स‚ति । य‚स्मादेवं \textbf{त‚स्मादि}त्युप‚संहारः । अस्मिन् प्र‚योगे‚{\tiny $_{lb}$}‚ किं नामोच्य‚त इत्याह--\textbf{विप‚रीते}ति । तुश‚ब्दार्थ‚श्च‚कारः । \textbf{त‚था ही}त्यादिनैत‚देव प्र‚ति‚{\tiny $_{lb}$}‚पाद‚य‚ति । य‚त एव‚म‚नुवाद‚विधिस्त‚तः । \textbf{त‚था चा}कृत‚क‚स्य नित्य‚त्वोक्तिप्र‚कारे स‚ति ।‚{\tiny $_{lb}$}‚ ‚{\tiny $_{lb}$}‚ ‚{\tiny $_{lb}$}‚ \leavevmode\ledsidenote{\textenglish{253/dm}}‚{\tiny $_{lb}$}‚ 
	  
	दृष्टान्त‚दोषानुदाहृत्य दुष्ट‚त्व‚निब‚न्ध‚न‚त्वं द‚र्श‚यितुमाह-- ‚{\tiny $_{lb}$}‚ 
	  
	न ह्येभिर्दृष्टान्ताभासैर्हेतोः सामान्य‚ल‚क्ष‚णं स‚प‚क्ष एव स‚त्वं विप‚क्षे च‚{\tiny $_{lb}$}‚ स‚र्व‚त्रास‚त्व‚मेव निश्च‚येन श‚क्यं द‚र्श‚यितुं विशेष‚ल‚क्ष‚णं वा । त‚द‚र्थाप‚त्यैषां‚{\tiny $_{lb}$}‚ निरासो \edtext{}{\lemma{निरासो}\Bfootnote{निरासो वेदित‚व्यः \cite{dp-msB} \cite{dp-msD} \cite{dp-edP} \cite{dp-edH} \cite{dp-edE} \cite{dp-edN}}}द्र‚ष्ट‚व्यः ॥१३६॥‚{\tiny $_{lb}$}‚ 
	  
	न‚ह्येभिरिति । साध्य‚निय‚त‚हेतुप्र‚द‚र्श‚नाय हि दृष्टान्ता व‚क्त‚व्याः । एभिश्च हेतोः‚{\tiny $_{lb}$}‚ स‚प‚क्ष एव स‚त्त्वं विप‚क्षे च स‚र्व‚त्रास‚त्त्व‚मेव य‚त् सामान्य‚ल‚क्ष‚णं त‚त् निश्च‚येन न श‚क्यं द‚र्श‚यितुम् । ‚{\tiny $_{lb}$}‚ 
	  
	न‚नु च सामान्य‚ल‚क्ष‚णं विशेष‚निष्ठ‚मेव प्र‚तिप‚त्त‚व्यं न स्व‚त एवेत्याह--विशेष‚ल‚क्ष‚णं‚{\tiny $_{lb}$}‚ वा । य‚दि विशेष‚ल‚क्ष‚णं प्र‚तिपाद‚यितुं श‚क्येत स्यादेव सामान्य‚ल‚क्ष‚ण‚प्र‚तिप‚त्तिः । विशेष‚{\tiny $_{lb}$}‚ल‚क्ष‚ण‚मेव तु न श‚क्य‚मेभिः प्र‚तिपाद‚यितुम् । त‚स्माद‚र्थाप‚त्त्या साम‚र्थ्येन\edtext{}{\lemma{र्थ्येन}\Bfootnote{साम‚र्थ्येनेति न तेषां \cite{dp-msA} \cite{dp-msB} \cite{dp-edP} साम‚र्थ्येनेति तेषां \cite{dp-edE} \cite{dp-edH} साम‚र्थ्येन तेषां \cite{dp-msD}}} एषां निराक‚र‚णं‚{\tiny $_{lb}$}‚ द्र‚ष्ट‚व्य‚म् । साध्य‚निय‚त‚साध‚न‚प्र‚तीत‚ये\edtext{}{\lemma{ये}\Bfootnote{प्र‚तिप‚त्त‚ये उपा० \cite{dp-msB}}} उपात्ताः । त‚द‚स‚म‚र्था दुष्टाः, \edtext{\textsuperscript{*}}{\lemma{*}\Bfootnote{स्व‚कार्य‚क‚र‚णात् \cite{dp-msA} \cite{dp-msB} \cite{dp-edP} \cite{dp-edH}}}स्व‚कार्याक‚र‚णादिति‚{\tiny $_{lb}$}‚ \edtext{\textsuperscript{*}}{\lemma{*}\Bfootnote{असाम‚र्थ्य‚म् इति संशोधितं \cite{dp-msC} \cite{dp-msD} प्र‚त‚योः । इति साम‚र्थ्य‚म्--\cite{dp-msA} \cite{dp-msB} \cite{dp-edP} \cite{dp-edH} \cite{dp-edE} \cite{dp-edN}}}असाम‚र्थ्य‚म् । इय‚ता साध‚न‚मुक्त‚म् ॥‚{\tiny $_{lb}$}‚ \textbf{अकृत‚क‚त्वं} कृत‚क‚त्व‚स्य साध‚न‚स्याभावः । \textbf{नित्य‚त्वे}ऽनित्य‚त्य‚त्व‚ल‚क्ष‚ण‚साध्याभावे । \textbf{न नित्य‚त्वं}‚{\tiny $_{lb}$}‚ साध्याभाव‚ल‚क्ष‚णं \textbf{साध‚नाभावे} कृत‚क‚त्व‚ल‚क्ष‚ण‚साध‚नाभावेऽकृत‚क‚त्व इत्य‚र्थात् । य‚त एवं \textbf{त‚तो}‚{\tiny $_{lb}$}‚ हेतो\textbf{र्व्य‚तिरेक‚वाक्यं} क‚र्त्तृ \textbf{हेतुं} क‚र्म‚भूतं \textbf{न साध्य‚निय‚त‚माह} । उक्त‚या नीत्या साध‚नाभावः‚{\tiny $_{lb}$}‚ साध्याभावे निय‚त‚त्वात्त‚म‚न्त‚रेण न भ‚वेत् । न तु य‚त्र साध्याभाव‚स्त‚त्राव‚श्यं साध‚नाभाव इति‚{\tiny $_{lb}$}‚ साध्य‚म‚न्त‚रेणापि साध‚नं भ‚वेत् । त‚त‚श्च साध्यानिय‚तं साध‚न‚मित्य‚भिप्रायः । \textbf{त‚था च} साध्य‚{\tiny $_{lb}$}‚निय‚त‚हेत्व‚प्र‚द‚र्श‚न‚प्र‚कारे स‚ति \textbf{व‚क्तुरे}वंवाक्य‚प्र‚योक्तुः ॥
	\pend% ending standard par
      ‚{\tiny $_{lb}$}‚

	  \pstart \leavevmode% starting standard par
	\textbf{निश्च‚येना}व‚श्य‚न्त‚या । \textbf{विशेष‚निष्ठ‚मेव प्र‚त्येत‚व्य‚मि}ति ब्रुव‚तोऽयं भावः--य‚दि नामा‚{\tiny $_{lb}$}‚मीभिः स‚प‚क्ष एव स‚त्त्वं विप‚क्षे स‚र्व‚त्रास‚त्त्वं निश्च‚येन श‚क्य‚ते द‚र्श‚यितुम्, त‚थाप्येते विशेष‚ल‚क्ष‚णं‚{\tiny $_{lb}$}‚ सामान्य‚ल‚क्ष‚ण‚प्र‚तिप‚त्त्य‚ङ्गं प्र‚तिपाद‚य‚न्त उप‚योक्ष्य‚न्त इति । अत्र \textbf{विशेष‚ल‚क्ष‚ण‚श्चे}त्युत्त‚रं \textbf{य‚दीत्या}‚{\tiny $_{lb}$}‚दिना व्याच‚ष्टे । य‚तः सामान्य‚ल‚क्ष‚णं विशेष‚ल‚क्ष‚णं वा न श‚क्य‚मेभिर्द‚र्श‚यितुम् । \textbf{त‚स्मात्}‚{\tiny $_{lb}$}‚ कार‚णात् । \textbf{अर्थाप‚त्त्ये}त्य‚स्य व्याख्यान‚म् \textbf{साम‚र्थ्येन} सामान्य‚विशेष‚ल‚क्ष‚णाप्र‚तिपाद‚न‚ल‚क्ष‚ण‚न ।‚{\tiny $_{lb}$}‚ \textbf{एषां} दृष्टान्ताभासानां \textbf{निराक‚र‚णं} दृष्टान्त‚रूप‚त्वेनेत्य‚र्थात् ।
	\pend% ending standard par
      ‚{\tiny $_{lb}$}‚

	  \pstart \leavevmode% starting standard par
	क‚थ‚म‚मी दुष्टा येन त‚थात्वेन निराक‚र‚ण‚मेषामित्याश‚ङ्क्योप‚संह‚र‚न्नाह--\textbf{साध्ये}ति ।‚{\tiny $_{lb}$}‚ \textbf{त‚द‚स‚म‚र्था}स्त‚द‚प्र \edtext{}{\lemma{प्र}\Bfootnote{त‚त्प्र}} तीतिकार‚णाश‚क्ताः । असाम‚र्थ्य‚मेव क‚थं येनासाम \leavevmode\ledsidenote{\textenglish{83a/ms}}र्थ्याद् दुष्टा‚{\tiny $_{lb}$}‚ उच्य‚न्त इत्याह--\textbf{स्व‚कार्य}स्य हेतोः साध्य‚निय‚त‚त्व‚प्र‚द‚र्श‚न‚ल‚क्ष‚ण‚स्या\textbf{क‚र‚णात्} । न‚नु त‚द‚क‚र‚ण‚{\tiny $_{lb}$}‚मेवासाम‚र्थ्य‚मुक्त‚मिति चेत् । स‚त्य‚म् । केव‚ल‚म‚साम‚र्थ्य‚व्य‚व‚हारापेक्ष‚यैव‚मुक्त‚मित्य‚व‚सेय‚म् । इति‚{\tiny $_{lb}$}‚ स्त‚स्माद‚र्थे, एव‚म‚र्थे वा । \textbf{असाम‚र्थ्य‚मेषा}मित्य‚र्थात् । साध्यादिविक‚ल‚स्यान‚न्व‚याप्र‚द‚र्शितान्व‚या‚{\tiny $_{lb}$}‚‚{\tiny $_{lb}$}‚ ‚{\tiny $_{lb}$}‚ ‚{\tiny $_{lb}$}‚ ‚{\tiny $_{lb}$}‚ \leavevmode\ledsidenote{\textenglish{254/dm}}‚{\tiny $_{lb}$}‚ 
	  
	दूष‚णं व‚क्तुमाह-- ‚{\tiny $_{lb}$}‚ 
	  
	दूष‚णा \edtext{}{\lemma{णा}\Bfootnote{न्यून‚तायुक्तिः--\cite{dp-msB} \cite{dp-edP} \cite{dp-edH} दूष‚णानि न्यू० \cite{dp-edE}}}न्यून‚ताद्युक्तिः ॥ १३७ ॥‚{\tiny $_{lb}$}‚ 
	  
	\edtext{\textsuperscript{*}}{\lemma{*}\Bfootnote{दूष‚णानि कानि द्र‚ष्ट‚व्यानि--\cite{dp-edE}}}दूष‚णा का द्र‚ष्ट‚व्या ? न्यून‚तादीनामुवितः । उच्य‚तेऽन‚येत्युवित‚र्व‚च‚न‚म् न्यून‚{\tiny $_{lb}$}‚तादे\edtext{}{\lemma{तादे}\Bfootnote{न्यून‚तादिर्व \cite{dp-msA} \cite{dp-msB} \cite{dp-edP} \cite{dp-edH} न्यून‚तादिव‚च० \cite{dp-edE} \cite{dp-edN}}}र्व‚च‚न‚म् ॥ ‚{\tiny $_{lb}$}‚ 
	  
	दूष‚णं विव‚रीतुमाह-- ‚{\tiny $_{lb}$}‚ 
	  
	ये पूर्वं न्यून‚तादायः साध‚न‚दोषा उक्तास्तेषामुद्भाव‚नं दूष‚ण‚म् । तेन‚{\tiny $_{lb}$}‚ प‚रेष्टार्थ‚सिद्धिप्र‚तिब‚न्धात् ॥१३८॥‚{\tiny $_{lb}$}‚ 
	  
	ये पूर्वं न्यून‚ताद‚योऽसिद्ध‚विरुद्धानैकान्तिका उक्तास्तेषामुद्भाव‚कं य‚द्व‚च‚नं त‚द् दूष‚ण‚म् । ‚{\tiny $_{lb}$}‚ 
	  
	न‚नु च न्यून‚ताद‚यो न विप‚र्य‚य‚साध‚नाः । त‚त् क‚थं दूष‚ण‚मित्याह--तेन न्यून‚तादि‚{\tiny $_{lb}$}‚व‚च‚नेन प‚रेषामिष्ट‚श्चासाव‚र्थ‚श्च त‚स्य सिद्धिः निश्च‚य‚स्त‚स्याः प्र‚तिब‚न्धात् । नाव‚श्यं विप‚र्य‚य‚{\tiny $_{lb}$}‚साध‚नादेव दूष‚णं विरुद्ध‚व‚त् । अपि तु प‚र‚स्याभिप्रेत‚निश्च‚य\edtext{}{\lemma{य}\Bfootnote{०निश्च‚य‚निब‚न्ध० \cite{dp-msA} \cite{dp-msB} \cite{dp-msC} \cite{dp-msD} \cite{dp-edP} \cite{dp-edH} \cite{dp-edN} निश्च‚य‚प्र‚तिब‚न्ध० \cite{dp-edE}}}विब‚न्धान्निश्च‚याभावो भ‚व‚ति‚{\tiny $_{lb}$}‚ निश्च‚य‚विप‚र्य‚य इत्य‚स्त्येव\edtext{}{\lemma{स्त्येव}\Bfootnote{इत्य‚स्ति विप० \cite{dp-msA} \cite{dp-edP} \cite{dp-edH} \cite{dp-edE}}} विप‚र्य‚य‚सिद्धिरिति । \edtext{\textsuperscript{*}}{\lemma{*}\Bfootnote{उक्तंदूष‚ण‚म् \cite{dp-edE}}}उक्ता \edtext{}{\lemma{उक्ता}\Bfootnote{नास्ति दूष‚णा \cite{dp-msA} \cite{dp-msB} \cite{dp-edP} \cite{dp-edH}}}दूष‚णा ॥‚{\tiny $_{lb}$}‚ देर‚पि दृष्टान्ताभास‚स्यासाध‚नाङ्ग‚व‚च‚नाद् वादिनो निग्र‚होऽसाम‚र्थ्योपादानान्न्याय‚प्राप्तः । त‚स्मा‚{\tiny $_{lb}$}‚देवंविधाः स्व‚वाक्ये व‚र्ज‚नीयाः । प‚रोपात्ताश्च चोद‚नीया इति दृष्टान्ताभास‚व्युत्पाद‚ने \textbf{वार्त्तिक‚{\tiny $_{lb}$}‚कृतो}ऽभिप्रायः प्र‚त्येत‚व्य इति ।
	\pend% ending standard par
      ‚{\tiny $_{lb}$}‚

	  \pstart \leavevmode% starting standard par
	स‚म्प्र‚ति सुख‚ग्र‚ह‚णार्थ‚मुक्त‚प्र‚ब‚न्ध‚स्याचार्यीय‚स्य प्र‚तिपादित‚म‚व‚च्छिन्द‚न्नाह--इय‚तेति‚{\tiny $_{lb}$}‚ \textbf{त्रिरूप‚लिङ्गाख्}यान‚मित्यादिनैत‚द‚न्तेन, इदं प‚रिमाण‚म‚स्ये\textbf{तीय‚त् तेनेय‚ता} म‚हावाक्येन \textbf{साध‚न‚मुक्त}‚{\tiny $_{lb}$}‚माचार्येणेति शेषः । प्र‚स‚ङ्गाग‚त‚स्यानेक‚स्यापि त‚द‚भिधान‚म‚तिवृत्त‚म्, त‚द‚पि साध‚न‚प्र‚तिपाद‚न एव‚{\tiny $_{lb}$}‚ साक्षात्प‚र‚म्प‚र‚या वा स‚मुप‚युक्त‚म् । साध‚नाभासाभिधान‚म‚पि त‚स्यैव स्फुटाव‚ग‚मार्थं त‚त्रैवोप‚{\tiny $_{lb}$}‚युक्त‚मिति म‚न्य‚मानेनोक्त\textbf{मिय‚ता साध‚न‚मुक्त‚मि}ति ।
	\pend% ending standard par
      ‚{\tiny $_{lb}$}‚

	  \pstart \leavevmode% starting standard par
	इति भूत‚दूष‚णोद्भाव‚ना \textbf{दूष‚णा} । दुषेर्णिज‚न्ताद् युचंकृत्वा टाप्क‚र्त्त‚व्यः । एत‚च्च‚{\tiny $_{lb}$}‚ \textbf{त‚द‚दूष‚ण‚मि}त्य‚न्तं सुबोध‚म् ।
	\pend% ending standard par
      ‚{\tiny $_{lb}$}‚

	  \pstart \leavevmode% starting standard par
	विप‚रीत‚साध‚न‚स्यैव दूष‚ण‚त्वात्क‚थं न्यून‚ताद्युक्ति \edtext{}{\lemma{ताद्युक्ति}\Bfootnote{क्ते}} र्दूष‚ण‚त्व‚मित्य‚भिप्रेत्याह--\textbf{न‚नु चे}ति ।‚{\tiny $_{lb}$}‚ अत्र \textbf{तेने}त्याद्युत्त‚रं व्याच‚क्षाण आह--\textbf{तेने}त्यादि ।
	\pend% ending standard par
      ‚{\tiny $_{lb}$}‚‚{\tiny $_{lb}$}‚‚{\tiny $_{lb}$}‚‚{\tiny $_{lb}$}‚‚{\tiny $_{lb}$}‚‚{\tiny $_{lb}$}‚‚{\tiny $_{lb}$}‚\textsuperscript{\textenglish{255/dm}}‚{\tiny $_{lb}$}‚
	  \bigskip
	  \begingroup
	
	  \bigskip
	  \begingroup
	

	  \pstart \leavevmode% starting standard par
	दूष‚णाभासास्तु जात‚यः ॥ १३९ ॥
	\pend% ending standard par
      
	  \endgroup
	‚{\tiny $_{lb}$}‚ 

	  \pstart \leavevmode% starting standard par
	दूष‚णाभासा इति । \edtext{\textsuperscript{*}}{\lemma{*}\Bfootnote{दूष‚णाव‚त् \cite{dp-msB}}}दूष‚ण‚व‚दाभास‚न्त इति दूष‚णाभासाः । के ते ? जात‚यः \edtext{}{\lemma{यः}\Bfootnote{इति नास्ति \cite{dp-msA} \cite{dp-msB} \cite{dp-edP} \cite{dp-edH} \cite{dp-edE} \cite{dp-edN}}}इति ।‚{\tiny $_{lb}$}‚ जातिश‚ब्दः सादृश्य‚व‚च‚नः । उत्त‚र‚स‚दृशानि जात्युत्त‚राणि\edtext{}{\lemma{राणि}\Bfootnote{०राणीति \cite{dp-msA} \cite{dp-edP} \cite{dp-edH} \cite{dp-edE}}} । उत्त‚र‚स्थान‚प्र‚युक्त‚त्वाद् उत्त‚र‚{\tiny $_{lb}$}‚स‚दृशानि जात्युत्त‚राणि ॥
	\pend% ending standard par
       ‚{\tiny $_{lb}$}‚ 

	  \pstart \leavevmode% starting standard par
	त‚देवोत्त‚र‚सादृश्य‚मुत्त‚र‚स्थान‚प्र‚युक्त्वेन द‚र्श‚यितुमाह--
	\pend% ending standard par
       ‚{\tiny $_{lb}$}‚ 
	  \bigskip
	  \begingroup
	

	  \pstart \leavevmode% starting standard par
	\edtext{\textsuperscript{*}}{\lemma{*}\Bfootnote{अनुभूत \cite{dp-msB} \cite{dp-edP} \cite{dp-edH}}}अभूत‚दोषोद्भाव‚नानि जात्युत्त‚राणीति ॥ १४० ॥
	\pend% ending standard par
      
	  \endgroup
	‚{\tiny $_{lb}$}‚ 

	  \begin{center}%% label @type='head'
	\textbf{॥ \footnote{तृतीय‚प‚रिच्छेदः स‚माप्तः इति नास्ति \cite{dp-msC} \cite{dp-msD}}‚तृतीय‚प‚रिच्छेदः स‚माप्तः ॥}
	\end{center}
	‚{\tiny $_{lb}$}‚ 

	  \begin{center}%% label @type='head'
	\textbf{॥ न्याय‚बिन्दुः स‚माप्तः ॥ ल‚घुध‚र्मोत्त‚र‚सूत्रं स‚माप्त‚मिति ॥}
	\end{center}
	
	  \endgroup
	‚{\tiny $_{lb}$}‚

	  \pstart \leavevmode% starting standard par
	न‚नूक्तं विप‚रीत‚साध‚नं दूष‚ण‚म् । त‚त्क‚थं प‚रेष्टार्थ‚सिद्धिप्र‚तिब‚न्ध‚क‚स्यापि न्यून‚तादिव‚च‚न‚स्य‚{\tiny $_{lb}$}‚ त‚थात्व‚मुच्य‚त इत्याश‚ङ्काम‚पाकुर्व‚न्नाह--\textbf{नाव‚श्य‚मिति} । किन्त्व‚प‚र‚स्य सिसाध‚यिषितार्थ\textbf{निश्च‚य‚{\tiny $_{lb}$}‚विब‚न्धा}द् अप्य‚त्रार्थाद् द्र‚ष्ट‚व्य‚म् । अन्य‚थाऽ\textbf{व‚श्यं} ग्र‚ह‚ण‚म‚व‚धार‚णं \textbf{विरुद्ध‚व‚दि}त्य‚पि दुर्योजं स्यात् ।‚{\tiny $_{lb}$}‚ मृत्वा शीर्त्वा च त‚द्योज‚ने व‚क्तुर‚कौश‚लं स्यादिति ।
	\pend% ending standard par
      ‚{\tiny $_{lb}$}‚

	  \pstart \leavevmode% starting standard par
	य‚दि त्व‚व‚श्यं विप‚र्य‚य‚साध‚न‚त्व‚स्यैव दूष‚ण‚त्व‚मिति निर्ब‚न्ध‚स्त‚दा त‚द‚प्य‚स्य न्यून‚तादिव‚च‚न‚{\tiny $_{lb}$}‚स्यास्तीति द‚र्श‚य‚न्नाह--\textbf{निश्च‚ये}ति । वाश‚ब्दः प‚क्षान्त‚र‚म‚व‚द्योत‚य‚ति । इतिस्त‚स्मा\textbf{द‚स्त्येव} ॥
	\pend% ending standard par
      ‚{\tiny $_{lb}$}‚

	  \pstart \leavevmode% starting standard par
	सादृश्यार्थ‚वृत्तेर‚पि जातिश‚ब्द‚स्य द‚र्श‚ना\textbf{ज्जातिश‚ब्दः सादृश्य‚व‚च‚न} इत्याह ।
	\pend% ending standard par
      ‚{\tiny $_{lb}$}‚

	  \pstart \leavevmode% starting standard par
	न‚नु जातिश‚ब्दः सादृश्य‚व‚च‚न‚त्वाद‚स्त्व‚य‚म‚र्थः--जात‚यः स‚दृशा इति । कानि पुन‚स्तानि‚{\tiny $_{lb}$}‚ केन च स‚दृशानीति न ज्ञाय‚त इत्याश‚ङ्काम‚पाकुर्व‚न्नाह--\textbf{उत्त‚रे}ति । एत‚च्चाचार्येणैव विव‚र‚णे‚{\tiny $_{lb}$}‚ स्प‚ष्टीकृत‚मिति म‚न्य‚ते । त‚द‚य‚म‚र्थः-जातिश‚ब्देन जात्युत्त‚र‚मेवात्र \textbf{वार्त्तिक‚का}र‚स्य विव‚क्षित‚मिति ।
	\pend% ending standard par
      ‚{\tiny $_{lb}$}‚

	  \pstart \leavevmode% starting standard par
	क‚थं पुन‚र्दूष‚णाभासानां जात्युत्त‚र‚श‚ब्द‚वाच्य‚त्व‚मित्याश‚ङ्क्याह--\textbf{उत्त‚रे}ति । ल‚क्ष्य‚ते‚{\tiny $_{lb}$}‚ चाय‚माचार्य‚स्याश‚यो य‚दुतोत्त‚र‚स्थाने जाय‚त इति । विव‚र‚णेऽप्युत्त‚र‚स्थाने जाय‚मान‚त्वाज्जाय‚त‚{\tiny $_{lb}$}‚ उत्त‚र‚त्वेनाभास‚ना\textbf{दुत्त‚राणी}ति । एवं तु क‚थ‚म‚नेन \add{न} व्याख्यात‚मिति न प्र‚तीमः ॥
	\pend% ending standard par
      ‚{\tiny $_{lb}$}‚

	  \pstart \leavevmode% starting standard par
	\textbf{जात्युत्त‚र}श‚ब्द‚स्य विग्र‚हं द‚र्श‚य‚न्नाह--\textbf{जात्येति । अभूत‚दोषोद्भाव‚नानि जात्युत्त‚राणी}ति‚{\tiny $_{lb}$}‚ ब्रुव‚ता \textbf{वार्त्तिक‚कृता} भूत‚दोषोद्भाव‚नं तु य‚द् दूष‚णाख्यं त‚देवोत्तीर्य‚ते अनिष्ट‚प‚क्षाद‚नेनोत्तार‚य‚ति,‚{\tiny $_{lb}$}‚ निर्वाह‚य‚ति वा स्व‚प‚क्ष‚मिति । \leavevmode\ledsidenote{\textenglish{83b/ms}}...मिति सूच‚य‚ति । एत‚च्च जात्युत्त‚र‚व्युत्पाद‚न‚{\tiny $_{lb}$}‚माचार्य‚स्य हेत्वाभास‚व‚न्न प्र‚योगार्थ‚म् । य‚था \textbf{नैयायिका} म‚न्य‚न्ते--अत्य‚न्त‚प‚राजीय‚मानाव‚स्थायां‚{\tiny $_{lb}$}‚ ‚{\tiny $_{lb}$}‚ ‚{\tiny $_{lb}$}‚ ‚{\tiny $_{lb}$}‚ ‚{\tiny $_{lb}$}‚ \leavevmode\ledsidenote{\textenglish{256/dm}}‚{\tiny $_{lb}$}‚ 
	  
	अभूत‚स्यास‚त्य‚स्य दोष‚स्य उद्भाव‚नानि । उद्भाव्य‚त \edtext{}{\lemma{त}\Bfootnote{एतैरुद्भा \cite{dp-msC}}}एतैरित्युद्धाव‚नानि व‚च‚नानि ।‚{\tiny $_{lb}$}‚ तानि जात्युत्त‚राणि । जात्या सादृश्येनोत्त‚राणि जात्युत्त‚राणीति ॥ ‚{\tiny $_{lb}$}‚ 
	  
	क‚तिप‚य‚प‚द‚व‚स्तुव्याख्य‚या य‚न्म‚याप्तं कुश‚ल‚म‚म‚ल‚मिन्दोरंशुव‚न्न्याय‚बिन्दोः । ‚{\tiny $_{lb}$}‚ 
	  
	प‚द‚म‚ज‚र‚म‚वाप्य ज्ञान‚ध‚र्मोत्त‚रं य‚ज् ज‚ग‚दुप‚कृतिमात्र‚व्यापृतिः\edtext{}{\lemma{व्यापृतिः}\Bfootnote{व्यापृतः \cite{dp-msC} \cite{dp-msD}}} स्याम‚तोऽह‚म् ॥ ‚{\tiny $_{lb}$}‚ 
	  
	आचार्य‚ध‚र्मोत्त‚र\edtext{}{\lemma{र}\Bfootnote{स‚माप्तेयं न्याय‚बिन्दुटीका कृतिराचार्य‚ध‚र्मोत्त‚र‚स्य ॥ आ. ०र‚पाद‚विर‚चितायां \cite{dp-msB} \cite{dp-msD}}} विर‚चितायां न्याय‚बिन्दुटीकायां तृतीयः प‚रिच्छेदः स‚माप्तः ॥\edtext{\textsuperscript{*}}{\lemma{*}\Bfootnote{\cite{dp-msA} \cite{dp-msB} प्र‚तिषु--स‚ह‚स्र‚मेकं श्लोकानां त‚था श‚त‚च‚तुष्ट‚य‚म् । स‚प्त‚स‚प्त‚तिसंयुक्तं निपुणं प‚रिपिण्डित‚म् ॥  ‚{\tiny $_{lb}$}‚ \cite{dp-msD} प्र‚तौ--० पिण्डित‚म् ॥ १४७ ॥ मंग‚लं म‚हा श्री ॥‚{\tiny $_{lb}$}‚ \cite{dp-msC} प्र‚तौ--स‚माप्त‚मिति ॥ संव‚त् १४९० व‚र्षे मार्ग‚शिर सुदि ३ र‚वौ श्री ख‚र‚त‚र‚ग‚च्छे‚{\tiny $_{lb}$}‚ श्री जिन‚राज‚सूरिप‚ट्टे, श्री श्री जिन‚भ‚द्र‚सूरिराज्ये प‚रीक्ष‚गूर्ज‚र‚सुत‚ध‚र‚णाकेन‚{\tiny $_{lb}$}‚ लिखापितं ॥ शुभं भ‚व‚तु ॥ क‚ल्याण‚म‚स्तु ॥ न्याय‚बिन्दुसूत्र‚वृत्ति...पुरोहित‚{\tiny $_{lb}$}‚ह‚रीयाकेन लिखित‚म् ॥}}‚{\tiny $_{lb}$}‚ जातिर्हेत्वाभास‚श्च प्र‚योक्त‚व्य इति । किञ्च दूष‚ण‚स्व‚रूप‚स्य स्फुटार्थ‚बोध‚नार्थ‚म् । हेय‚ज्ञाने‚{\tiny $_{lb}$}‚ हि त‚द्विविक्त‚मुपादेयं सुज्ञातं भ‚व‚तीति जाति \edtext{}{\lemma{जाति}\Bfootnote{तेः}} हेत्वाभासानाञ्च स्फुट‚स्व‚रूप‚प‚रिज्ञान‚स्य‚{\tiny $_{lb}$}‚ प्र‚योज‚न‚म् स्व‚वाक्ये प‚रिव‚र्ज‚नं प‚र‚प्र‚युक्तानाम‚पि दोषोद्भाव‚न‚मित‚र‚था व्यामोहः स्यादि‚{\tiny $_{lb}$}‚त्युक्त‚प्राय‚म् ।
	\pend% ending standard par
      ‚{\tiny $_{lb}$}‚

	  \pstart \leavevmode% starting standard par
	त‚त्र जात्युत्त‚र‚स्योदाह‚र‚णं य‚था--अनित्यः श‚ब्दः कृत‚क‚त्वाद् घ‚ट‚व‚दित्युक्ते किमिदं‚{\tiny $_{lb}$}‚ कृत‚क‚त्वं श‚ब्द‚ग‚तं हेतुत्वेनोप‚नीत‚माहोस्विद् घ‚ट‚ग‚त‚म् । य‚दि श‚ब्द‚ग‚तं त‚स्य घ‚टो दृष्टान्तो‚{\tiny $_{lb}$}‚ऽस‚म्भ‚वाद‚व्याप्तेर‚नैकान्तिको\add{... ... ...}प्र‚त्य‚व‚स्थानात्मिका‚{\tiny $_{lb}$}‚ जाति\add{... ... ...}साध‚र्म्यादिति नैयायिक\add{... ...}‚{\tiny $_{lb}$}‚ \add{... ... ...}\textbf{आचार्य}स्य त्व‚य‚माश‚यो\add{... ... ...}‚{\tiny $_{lb}$}‚ \add{... ... ...}त‚थाहि \textbf{नैयायिका}\add{... ...}प्र‚तिज्ञाप‚द‚यो‚{\tiny $_{lb}$}‚र्विरोध‚माहुस्त‚था--प्र‚य‚त्नान‚न्त‚रीय‚कः श‚ब्दः\add{... ... ...}प्र‚य‚त्नान‚न्त‚रीय‚क‚त्वादि\add{...‚{\tiny $_{lb}$}‚ ... ...}हेतुमाच‚क्ष‚ते । त‚यैव दूष‚णा\add{... ... ...}‚{\tiny $_{lb}$}‚ श‚ब्द‚स्य ध‚र्मित्वात् । न चेयं जाति\add{... ... ...}भ‚व‚ति । त‚त‚श्च य‚स्यैव‚{\tiny $_{lb}$}‚ प्र‚त्य‚व‚स्थान‚स्य\add{... ... ...}त‚त्प्र‚यो‚{\tiny $_{lb}$}‚ \add{... ... ...}‚{\tiny $_{lb}$}‚ \add{... ... ...}जात्युत्त‚राणीत्य‚त्रेतिश‚ब्दो‚{\tiny $_{lb}$}‚ माहावाक्य‚प‚रिस‚माप्तौ ॥
	\pend% ending standard par
      ‚{\tiny $_{lb}$}‚‚{\tiny $_{lb}$}‚‚{\tiny $_{lb}$}‚‚{\tiny $_{lb}$}‚\textsuperscript{\textenglish{257/dm}}‚{\tiny $_{lb}$}‚

	  \pstart \leavevmode% starting standard par
	\textbf{आचार्य‚श्रीध‚र्म‚कीर्त्ति}विर‚चित‚स्यास्य \textbf{न्याय‚बिन्दु}संज्ञ‚क‚स्य प्र‚क‚र‚ण‚स्य य‚थाव‚द‚र्थ‚प्र‚काशिकां‚{\tiny $_{lb}$}‚ म‚हाप‚टीय‚सीमीदृशीं \add{व्याख्यां} \add{विर‚च}य‚ता म‚या\add{... ...}किम‚पि पुण्य‚मुपार्जितं‚{\tiny $_{lb}$}‚ त‚द‚नेन तादृशीम‚व‚स्थां प्राप्य स‚क‚ल‚स‚त्त्वोप‚कारं\add{... ...}मित्य‚ध्याश‚यो मे\add{... ... ...}क्रिया‚{\tiny $_{lb}$}‚योगात्सात्मीकृत‚प‚रार्थ‚क‚र‚णोऽयं \textbf{ध‚र्मो}त्त‚रः \textbf{क‚तिप‚ये}त्यादिना\add{... ... ...}श्लोक‚माह ।
	\pend% ending standard par
      ‚{\tiny $_{lb}$}‚

	  \pstart \leavevmode% starting standard par
	अस्यायं स‚मुदायार्थः । \textbf{न्याय‚बिन्दोः} किय‚त्\add{... ... ...}कुश‚ल‚माप्त‚म‚तः‚{\tiny $_{lb}$}‚ कुश‚ला\textbf{द‚ज‚रं ज्ञान‚ध‚र्मोत्त‚रं} च प‚दं त‚द‚वाप्य \textbf{ज‚ग‚दुप‚कृतिमात्र}\add{व्यापृतिः} स्यामिति ।\add{... ...‚{\tiny $_{lb}$}‚ ... ...}बुद्ध्य‚ते । प‚द्य‚न्ते ग‚म्य‚न्तेऽर्था एभिरिति प‚दानि वाक्यानि तेषां व‚स्तुप्र‚तिपाद्य‚त‚यास्ति‚{\tiny $_{lb}$}‚ त‚म‚भिधेय‚रूपं\add{... ...}शेषो ज्ञेयः । \textbf{आप्तं} प्राप्तं कुश‚लं सुकृत‚म् । किं \textbf{कुर्व‚ता} ? \leavevmode\ledsidenote{\textenglish{84a/ms}}...‚{\tiny $_{lb}$}‚ ...भ‚व‚ति पुण्य‚म‚पि कुश‚ल‚ञ्च । य‚था प‚र\add{... ...}प‚रोद्भ‚वापि । त‚तो व्य‚भिचार‚संभ‚वा‚{\tiny $_{lb}$}‚द्विशेष‚ण‚म् । किंव‚न्निर्म‚ल‚म् ? \textbf{इन्दोरंशुव‚दि}ति । \textbf{इन्दो}श्च‚न्द्र‚म‚सोंऽ\textbf{श‚वः} किर‚णास्त इव । एवं‚{\tiny $_{lb}$}‚विध‚विधानेन य‚त् पुण्यं ज‚न्य‚ते त‚द‚व‚श्य‚म‚म‚ल‚त‚यैत‚त्तुल्यं भ‚व‚तीति भावः । \textbf{प‚दं} प्र‚तिष्ठाम‚व‚स्था‚{\tiny $_{lb}$}‚मिति याव‚त् । किं\add{... ...}द्य‚ते...य‚त्र त‚त्त‚था । ज‚राग्र‚ह‚ण‚स्योप‚ल‚क्ष‚ण‚त्वात् मृत्योर‚पि‚{\tiny $_{lb}$}‚ स‚ङ्ग्र‚हो ज्ञात‚व्यः । तेनाय‚म‚र्थः--अज‚र‚म‚म‚र्त्यं चेति । अथ‚वा ज‚रानिर्देशेनैव द‚ण्डापूप‚न्यायेन‚{\tiny $_{lb}$}‚ मृत्योः प्र‚तिषेधः कृत एवेत्य‚व‚सेय‚म् । पुन‚र‚पि त‚द्विशिन‚ष्टि--\textbf{ज्ञाने}ति । ज्ञानं हेयोपादेय‚त‚त्त्व‚स्य‚{\tiny $_{lb}$}‚ साभ्युपाय‚स्याव‚बोधो विव‚क्षितः । \textbf{ध‚र्मं}श्च स‚र्वोप‚क‚र‚ण‚निव‚र्त्त‚कोऽदृष्टः । तावेवोत्त‚राव‚धिकौ‚{\tiny $_{lb}$}‚ य‚त्र प‚दे त‚त्त‚था । ताभ्यां \textbf{चोत्त}रं श्रेष्ठ‚म् । य‚त्त‚दोश्च नित्य‚म‚भिस‚म्ब‚धेन त‚च्छ‚ब्द‚स्य ल‚ब्ध‚त्वात्‚{\tiny $_{lb}$}‚ त‚द\textbf{वाप्ये}त्य‚र्थोऽव‚तिष्ठ‚ते । \textbf{ज‚ग‚तो} जीव‚लोक‚स्य \textbf{उप‚कृति}रुप‚कारः । सैव त\textbf{न्मात्र‚म्} । त‚द्\textbf{व्यापृति}‚{\tiny $_{lb}$}‚र्व्यापारो व्याप्रिय‚माण‚ता य‚स्य म‚म सोऽहं त‚था ।
	\pend% ending standard par
      ‚{\tiny $_{lb}$}‚

	  \pstart \leavevmode% starting standard par
	नावाशंसाविष‚येऽस्मिन्नाशीर्लिङ्गा भूयास‚मिति श‚ब्द‚सिद्धे\add{र्नैव‚म}नेन व‚क्त‚व्य‚म् त‚त्‚{\tiny $_{lb}$}‚ किमेव‚म‚वादीदिति चेत् । न । आशंसाविष‚य‚त्वाभावात् । य‚त ऽएवंविधानुष्ठान‚ज‚न्म‚ना‚{\tiny $_{lb}$}‚ पुण्यातिश‚येन एव‚म्भूत‚प‚द‚प्राप्तेस्त‚तोऽपि म‚मैवंविध‚क्रिय‚स्य स‚म्भाव्य‚मान‚त्वादेव...‚{\tiny $_{lb}$}‚ ...य‚मेत‚दित्य‚भिप्रायात् स‚र्व‚म‚व‚दात‚मिति ॥
	\pend% ending standard par
      ‚{\tiny $_{lb}$}‚
	    
	    \stanza[\smallbreak]
	गुरो\textbf{र्जिता}रेर‚धिग‚म्य धीध‚नं म‚या हि टीका विवृता प‚टीय‚सी ।&कुतूह‚लेनापि त‚द‚त्र युज्य‚ते निरीक्ष‚णं साधु विवेच‚कानाम् ॥&अज्ञो ज‚न‚स्त्य‚ज‚ति ल‚ब्ध‚म‚पीह र‚त्नं काचेन तुल्य‚मिति च‚लाय‚तेति\edtext{}{\lemma{तेति}\Bfootnote{मिति चंच‚ल‚मान‚सोऽपि}} ।&एताव‚तैव त‚द‚लंङ्क‚र‚णं न किं स्यात् किं वाऽऽद‚रेण त‚दुपाद‚द‚ते न ध‚न्याः ॥&इमं निब‚न्धं विधिव‚द्विधाय \add{म‚या ह्य}वाप्तं सुकृतंथ‚क‚ञ्चित् ।&इहैव ज‚न्म‚न्य‚थ तेन स‚त्त्वा अन‚न्त‚संबोधिम‚वाप्नुव‚न्तु ॥&॥ प‚ण्डित‚दुर्वैक‚मिश्र‚विर‚चित‚ध‚र्मोत्त‚र‚प्र‚दीपो नाम निब‚न्धः स‚माप्तः ॥\&[\smallbreak]


	
	    
	    \endnumbering% ending numbering from div
	    \endgroup
	    
	  % running endDocumentHook
     \backmatter 
	 \chapter{The TEI Header}
	 \begin{minted}[fontfamily=rmfamily,fontsize=\footnotesize,breaklines=true]{xml}
       <teiHeader xmlns="http://www.tei-c.org/ns/1.0" xml:lang="en">
   <fileDesc>
      <titleStmt>
         <title type="main" subtype="basetext">Nyāyabindu</title>
         <title type="sub" subtype="commentary" n="1">Nyāyabinduṭīkā</title>
         <title type="sub" subtype="commentary" n="2">Dharmottarapradīpa</title>
         <author role="baseauthor">Dharmakīrti</author>
         <author role="commentator" n="1">Dharmottara</author>
         <author role="commentator" n="2">Durveka Miśra</author>
         <funder>Deutsche Forschungsgemeinschaft</funder>
         <funder>The National Endowment for the Humanities</funder>
         <principal>
	           <persName>Birgit Kellner</persName>
	        </principal>
         <respStmt>
            <resp>data entry by</resp>
            <name key="name aurorachana">Aurorachana, Auroville</name>
         </respStmt>
         <respStmt xml:id="sarit-encoder-dp">
            <resp>prepared for SARIT by</resp>
            <persName key="name person lo">Liudmila Olalde</persName>
         </respStmt>
      </titleStmt>
      <editionStmt>
         <p> </p>
      </editionStmt>
      <publicationStmt>
         <publisher>SARIT: Search and Retrieval of Indic Texts. DFG/NEH Project (NEH-No.
	HG5004113), 2013-2016 </publisher>
         <idno>2015-08-07</idno>
         <availability status="restricted">
            <p>Copyright Notice:</p>
            <p>Copyright 2015-2016 SARIT</p>
            <licence> 
	              <p>Distributed under a <ref target="https://creativecommons.org/licenses/by-sa/4.0/">Creative Commons Attribution-ShareAlike 4.0 International licence.</ref> Under this licence, you are free to:</p>
	              <list>
                  <item>Share — copy and redistribute the material in any medium or format.</item>
                  <item>Adapt — remix, transform, and build upon the material for any purpose, even commercially.</item>
               </list>
	              <p>The licensor cannot revoke these freedoms as long as you follow the license terms.</p>
	              <p>Under the following terms:</p>
	              <list>
                  <item>Attribution — You must give appropriate credit, provide a link to the license, and indicate if changes were made. You may do so in any reasonable manner, but not in any way that suggests the licensor endorses you or your use.</item>
                  <item>ShareAlike — If you remix, transform, or build upon the material, you must distribute your contributions under the same license as the original.</item>
               </list>
	              <p>More information and fuller details of this license are given on the Creative Commons website.</p>
	           </licence>
            <p>SARIT assumes no responsibility for unauthorised use that infringes the rights of any copyright owners, known or unknown.</p>
         </availability>
         <date>2015</date>
      </publicationStmt>
      <sourceDesc>
         <bibl xml:id="dp-malvania-book">
	           <title type="main">Dharmottarapradīpa</title>
	           <title type="sub">Being a sub-commentary on Dharmottara's Nyāyabinduṭikā, a commentary on Dharmakīrti's Nyāyabindu</title>
	           <author>Dharmakīrti</author>
	           <author>Dharmottara</author>
	           <author>Durveka Miśra</author>
	           <editor xml:id="ed-dm">
               <forename>Dalsukhbhai</forename> 
               <surname>Malvania</surname>
            </editor>
	           <publisher>Kashiprasad Jayaswal Research Institute</publisher>
	           <pubPlace>Patna</pubPlace>
	           <date>1971</date>
	           <note>Revised second edition</note>
	        </bibl>
         <listWit>
            <head>List of manuscripts and published works utilised by Malvania. These descriptions are extracted from  Malvanias  introduction to his <ref target="#dp-malvania-book">edition</ref> pp. iii-viii).</head>
            <witness xml:id="dp-ms-dp">
	              <msDesc>
                  <msIdentifier>
                     <idno resp="#sarit-encoder-dp">Dharmottarapradīpa-MS</idno>
                     <altIdentifier>
                        <idno/>
                        <!-- is there a standard identifier? --></altIdentifier>
                  </msIdentifier>
                  <msContents>
                     <msItem>
                        <author>Durveka Miśra</author>
                        <title>Dharmottarapradīpa</title>
                     </msItem>
                  </msContents>
                  <physDesc>
                     <objectDesc>
                        <p>84 palm-leaves, written in Newari script. The first leaf is in a mutilated condition.</p>
                     </objectDesc>
                  </physDesc>
                  <history>
                     <p>Malvania had no access to the manuscript itself, but merely  to the photos procured by Rāhula Sāṅkṛtyāyana of the palm-leaf manuscripts of Sanskrit works in Tibet that were preserved in the Bihar Research Society, Patna, in thirteen albums. According to Malvania's description: "the 13th album contains the Dharmottarapradīpa in 28 plates. [...] Each of the plates bears the caption हे० बि० अ० suggesting that the work is Hetu-bindu-anuṭīkā. The work however is Nyāya-bindu-anuṭīkā, which has been called Dharmottara-pradīpa by the commentator himself. The original copy covers 84 leaves. It is written in Newari script. When the photocopy was made, the 60th leaf was not reversed. Consequently, 60A has been photographed twice, whereas there is no photo of the reverse, i. e., 60B. The MS is correct, but here and there it is indistinct. The first leaf is in a mutilated condition." (Malvania, "Introduction", <ref target="#dp-malvania-book">Dharmottarapradīpa</ref>, p. vii).</p>
                  </history>
               </msDesc>

	           </witness>
            <witness xml:id="dp-msA">
	              <msDesc>
                  <msIdentifier>
                     <idno>A</idno>
                     <altIdentifier>
                        <idno/>
                        <!-- is there a standard identifier? --></altIdentifier>
                  </msIdentifier>
                  <msContents>
                     <msItem>
                        <author>Dharmottara</author>
                        <title>Nyāyabinduṭīkā</title>
                     </msItem>
                  </msContents>
                  <physDesc>
                     <objectDesc>
                        <p>Palm-leaf manuscript.</p>
                     </objectDesc>
                  </physDesc>
                  <history>
                     <p>Palm-leaf manuscript belonging to the Śāntinātha Jaina Bhaṇḍāra Khambhāt. It was written in Vikrama Saṃvat 1229 (A.D. 1172). For a detailed description see <bibl>Peterson, A Third Report of Operaions in Search of Sanskrit MSS. in the Bombay Circle. 1887, No. 215.</bibl>(Malvania, "Introduction", <ref target="#dp-malvania-book">Dharmottarapradīpa</ref>, p. v.)</p>
                  </history>
               </msDesc>
	           </witness>
            <witness xml:id="dp-msB">
	              <msDesc resp="#ed-dm">
                  <msIdentifier>
                     <idno>B</idno>
                     <altIdentifier>
                        <idno/>
                        <!-- is there a standard identifier? --></altIdentifier>
                  </msIdentifier>
                  <msContents>
                     <msItem>
                        <author>Dharmakīrti</author>
                        <title>Nyāyabindu</title>
                        <author>Dharmottara</author>
                        <title>Nyāyabinduṭīkā</title>
                     </msItem>
                  </msContents>
                  <physDesc>
                     <objectDesc>
                        <p/>
                     </objectDesc>
                  </physDesc>
                  <history>
                     <p>This manuscript is included in the Bhau Daji collection of MSS. in the Bombay Branch of the Royal Asiatic Society. It is recorded as Laghu-dharmottara-sūtra. (Malvania, "Introduction", <ref target="#dp-malvania-book">Dharmottarapradīpa</ref>, p. v)</p>
                  </history>
               </msDesc>
	           </witness>
            <witness xml:id="dp-msC">
	              <msDesc>
                  <msIdentifier>
                     <idno>C</idno>
                     <altIdentifier>
                        <idno/>
                        <!-- is there a standard identifier? --></altIdentifier>
                  </msIdentifier>
                  <msContents>
                     <msItem>
                        <author>Dharmakīrti</author>
                        <title>Nyāyabindu</title>
                        <author>Dharmottara</author>
                        <title>Nyāyabinduṭīkā</title>
                     </msItem>
                  </msContents>
                  <physDesc>
                     <objectDesc>
                        <p>Palm-leaf manuscript</p>
                     </objectDesc>
                  </physDesc>
                  <history>
                     <p>Manuscript of the Jaina Jñāna Bhaṇḍāra, Jaisalmer. Palm-leaf manuscript no. 364. It was copied by Purohita Hariyāka at the instance of Śrāvaka Dharaṇāka at the time of Ācārya Śrī Jinabhadra Sūri, the pupil of Śrī Jinarāja Sūri, in V.S. 1490 (1433 A.D.). Muni Śrī Puṇyavijayaji noted down the variants of this MS.(Malvania, "Introduction", <ref target="#dp-malvania-book">Dharmottarapradīpa</ref>, pp. v-vi)</p>
                  </history>
               </msDesc>
	           </witness>
            <witness xml:id="dp-msD">
	              <msDesc>
                  <msIdentifier>
                     <idno>D</idno>
                     <altIdentifier>
                        <idno/>
                        <!-- is there a standard identifier? --></altIdentifier>
                  </msIdentifier>
                  <msContents>
                     <msItem>
                        <author>Dharmakīrti</author>
                        <title>Nyāyabindu</title>
                        <author>Dharmottara</author>
                        <title>Nyāyabinduṭīkā</title>
                     </msItem>
                  </msContents>
                  <physDesc>
                     <objectDesc>
                        <p>Palm-leaf manuscript.</p>
                     </objectDesc>
                  </physDesc>
                  <history>
                     <p>The manuscript belongs to the Bhaṇḍāra at Jaisalmer. It is the Palm-leaf manuscript no. 376. It appears to be older than <ref target="#dp-msC">C</ref>. It has been corrected by some reader. Muni Śrī Puṇyavijayaji conjectures that it belongs to the second half of the 13th century. Muni Śrī Puṇyavijayaji noted down the variants of this MS. (Malvania, "Introduction", <ref target="#dp-malvania-book">Dharmottarapradīpa</ref>, p. vi.)</p>
                  </history>
               </msDesc>
	           </witness>
            <witness xml:id="dp-msD-n">Marginal notes in manuscript <ref target="#dp-msD">D</ref>. Some of these notes  are included in the text of the manuscript <ref target="dp-msB">B</ref>. These notes were printed in Malvania's edition with the mark "टि०" in the footnotes.</witness>
            <witness xml:id="dp-edP">P = <ref target="http://east.uni-hd.de/bib/5186/">Peter Peterson's edition of the Nyāyabinduṭīka</ref>, published by the Asiatic Society of Bengal in the Bibliotheca Indica in 1889.The Nyāyabinduṭīkā was prepared on the basis of MSS. <ref target="#dp-msA">A</ref> and <ref target="#dp-msB">B</ref>, whereas the text of the Nyayabindu was finalised on that of the MS. <ref target="#dp-msB">B</ref> only. The second edition (1929) has no changes.
	  </witness>
            <witness xml:id="dp-edE">E = <ref target="http://east.uni-hd.de/bib/5509/">Stscherbatsky's edition of the Nyāyabindu and the Nyāyabinduṭīkā</ref>, published in the Bibliotheca Buddhica from Petrograd in 1918. Prepared on the basis of <ref target="#dp-edP">Peterson's edition</ref> and the MS of the Nyāyabinduṭīkā belonging to Denison Ross.
	  </witness>
            <witness xml:id="dp-edH">H = <ref target="http://east.uni-hd.de/bib/5196/">Chandra Shekhar Shāstrī's edition</ref> published in Banaras in the Haridas Sanskirt Series Volume No. 22, under the title of Nyāya-bindu in 1924. Republished in 1954 without any alteration.
	  </witness>
            <witness xml:id="dp-edN">N = <ref target="http://east.uni-hd.de/bib/5226/">Śrī P. I. Tarkas' edition of the Nyāyabindu and the Nyāyabinduṭīkā</ref>, published as volume 1 of the Nūtana Sanskrit Granthamāla of Akola in 1952. Chiefly based on <ref target="dp-edE">Stscherbatsky's edition</ref>.
	  </witness>
         </listWit>
      </sourceDesc>
   </fileDesc>
   <encodingDesc>
      <p>The texts is structured according to the pages of the printed edition. The different text layers (base-text, commentary 1 and commentary 2) were encoded as follows:
      <list>
            <item>The Dharmottarapradīpa is the uppermost level and was enclosed in the body-element. It is subdivided in 3 parts encoded as &lt;div type="chapter" ana="dp"&gt;.</item>
            <item>The Nyāyabindu was enclosed in &lt;quote type="basetext" ana="nb"&gt;.</item>
            <item>The Nyāyabinduṭīkā was enclosed in &lt;quote type="commentary1" ana="nbṭ"&gt;.</item>
         </list>
      </p>
      <p>Line breaks: In the source file, there were two types of line breaks: returns (and possible surrounding space) and hyphens+returns. These were replaced with lb-elements. The ed-attribute "dm" refers to Malvania's ed<ref sameAs="#dp-malvania-book"/>.</p>
      <p>The folio numbers of the Dharmottarapradīpa-manuscript were encoded as pb-elements with the attribute ed="ms", wich refers to the <ref target="#dp-ms-dp">manuscript</ref> used by Malvania, described above in the source description.</p>
      <p>The page breaks of <ref sameAs="#dp-malvania-book">Malvania's edition</ref> were encoded as pb-elements with the ed-attribute "dm".</p>
      <p>Round and square brackets were replaced by SARIT with the following TEI-elements:
      <list>
            <item>References to other works were enclosed in &lt;ref cRef=""&gt;. The attribute cert="unknown" indicates that cRef was not checked by the encoder; whereas  cert="high" indicates that the value of the cRef was checked by the encoder.</item>
            <item>Text in square brackets was enclosed in &lt;add&gt;, following Malvania's own explanation of the use of square brackets in his introduction (p. viii).</item>
            <item>Text in round brackets  was enclosed in &lt;note type="correction"&gt;, following Malvania's own explanation of the use of round brackets in his introduction (p. viii).</item>
            <item>All suspension points ("...") were enclosed in &lt;add type="gap"&gt;. Malvania used this mark to indicate "the portion that could not be read" ("Introduction", p. viii).</item>
         </list>
      </p>
      <p>Bold characters were enclosed in &lt;hi rend="bold"&gt;</p>
      <p>Double quotes were replaced by quote-elements.</p>
      <p>Single quotes were replaced by &lt;q&gt;.</p>
      <p>Characters that were not readable in the printed edition available to SARIT were enclosed &lt;unclear reason="illegible"&gt;.</p>
      <p>Abbreviations used in the cRef- and ana-attributes in this file: <!-- this is a provisory list and has to be replaced by a refsDecl -->
      <list ana="abbreviations">
            <item>ak = Vasubandhu's Abhidharmakośa</item>
            <item>dp = Durveka Miśra's, Dharmottarapradīpa</item>
            <item>hb = Dharmakīrti's Hetubindu. Chapter numbers in the cRef-attributes correspond to: <bibl>Hetubindu of Dharmakīti: a point on probans. A Sanskrit version translated with an introduction and notes by Pradeep P. Gokhale. Delhi: Sri Satguru Publications, 1997.<ref target="http://www.dsbcproject.org/node/7643"/>
               </bibl>
            </item>
            <item>hbṭ = Bhaṭṭa Arcaṭa's Hetubinduṭikā. Chapter numbers in the cRef-attributes correspond to: <bibl>Sanghavi, Sukhlalji, Muni Śrī Jambuvijayaji, eds. 1949. Hetubinduṭīkā of Bhaṭṭa Arcaṭa with the Sub-Commentary Entitled Āloka of Durveka Miśra. Baroda: Oriental Institute, pp. 1-229.<ref target="http://east.uni-hd.de/buddh/ind/15/41/494/ http://www.dsbcproject.org/node/7724"/>
               </bibl>
            </item>
            <item>htu = Jitari's <ref target="http://east.uni-hd.de/buddh/ind/26/74/559/ http://www.dsbcproject.org/node/7707">Hetutattvopadeśa</ref>.</item>
            <item>kā = Bhāmaha's Kāvyālaṃkāra </item>
            <item>kāv = Kāvyālaṃkāravṛtti </item>
            <item>MBh = Mahābhārata </item>
            <item>nā = Nyāyāvatāra </item>
            <item>nb = Dharmakīrti's Nyāyabindu</item>
            <item>nbṭ = Dharmottara's Nyāyabinduṭīkā</item>
            <item>nsū = Nyāyasūtra; sūtra-numbers in the cRef-attributes correspond to the <ref target="https://www.worldcat.org/title/srigautamamunipranitanyayasutrani-va&#130;tsya&#130;na&#130;mu&#130;ni&#130;krta&#130;bha&#130;sya&#130;vi&#130;sva&#130;na&#130;tha&#130;bha&#130;tta&#130;ca&#130;rya&#130;krta&#130;vrtti&#130;sa&#130;meta&#130;ni/oclc/644135949">1922 edition.</ref>
            </item>
            <item>nbh = Vātsyāyana's Nyāyasūtrabhāṣya; page numbers in the cRef-attributes correspond to the <ref target="https://www.worldcat.org/title/srigautamamunipranitanyayasutrani-va&#130;tsya&#130;na&#130;mu&#130;ni&#130;krta&#130;bha&#130;sya&#130;vi&#130;sva&#130;na&#130;tha&#130;bha&#130;tta&#130;ca&#130;rya&#130;krta&#130;vrtti&#130;sa&#130;meta&#130;ni/oclc/644135949">1922 edition.</ref>
            </item>
            <item>nv = Uddyotakara's Nyāyavārttika. The page- and line-numbers in the cRef-attributes were take from <persName>Yasuhiro Okazaki</persName>'s <ref target="http://indology.info/etexts/archive/texts/uddyotakara-nyayavarttika.zip">index</ref> and refer to: <bibl>V. P. Dvivedin ed.: Nyāya-Vārttika, Bibliotheca Indica, Calcutta, 1887; rep. Delhi 1986.</bibl>
            </item>
            <item>Pā = Pāṇini's Aṣṭādhyāyī</item>
            <item>pv = Dharmakīrti's Pramāṇavārttika; the verse numbers correspond to <ref target="http://east.uni-hd.de/buddh/ind/36/120/193/">Sāṅkṛtyāyana's edition</ref>.</item>
            <item>pv-pandey = Dharmakīrti's Pramāṇavārttika; the verse numbers correspond to <ref target="http://east.uni-hd.de/buddh/ind/7/16/658/">Pandey's edition</ref>.</item>
            <item>vk-mbh = Vyākaraṇa-mahābhāṣya</item>
            <item>vsū = Kaṇāda's Vaiśeṣikasūtra; the verse numbers correspond to: <bibl>Jambūvijaya, ed. Vaiśeṣikasūtra of Kaṇāda: with the commentary of Candrānanda. Baroda: Oriental Institiute, 1982.</bibl>
            </item>
            <item>svi = Akalaṅka's Siddhiviniścaya. Page numbers might correspond to: <bibl>Mahendra Kumar Jain, ed. Siddhiviniścaya of Akalaṅka edited with the commentary Siddhivniścayaṭīkā of Anantavīrya. 2 vols. Delhi: Bhāratīya Jñānapīṭha Prakāśana (Jñānapīṭha Mūrtidevī Jaina Granthamālā 10), 1953-1957.</bibl>
            </item>
            <item>śv = Kumārila's Ślokavārttika:
	<list>
                  <item>śv-abhāva</item>
                  <item>śv-pratyakṣa</item>
               </list>
            </item>
         </list>
      </p>
      <p>The footnotes were encoded as note-elements with their corresponding n-attribute, which indicates the page and the note number. The note-elements are located were the footnote references appear in the printed edition. When several references point to the the same footnote, the note-element was duplicated and the number was extended with "a", "b", etc. This concerns the following notes: 
      <list>
            <item>2-1</item>
            <item>4-1</item>
            <item>51-1</item>
            <item>116-5</item>
            <item>120-7</item>
            <item>124-5</item>
            <item>125-2</item>
            <item>162-11</item>
            <item>166-1</item>
            <item>170-11</item>
            <item>171-6</item>
            <item>209-6</item>
            <item>229-2</item>
         </list>
      </p>
      <p>For some footnotes there were no references in the corresponding pages. Since the footnotes appear in the book, they were encoded as note-elements with the type-attribute "noreference". The notes listed below were relocated; the cert-attribute indicates thee degree of certainty of the location assigned by the encoder.
      <list>
            <item>95-3</item>
            <item>148-3</item>
            <item>151-2</item>
            <item>170-3</item>
            <item>172-7</item>
            <item>175-1</item>
            <item>175-7</item>
            <item>180-4</item>
            <item>180-5</item>
            <item>190-4</item>
            <item>198-9</item>
            <item>209-2</item>
            <item>209-3</item>
            <item>209-4</item>
            <item>210-6</item>
            <item>212-6</item>
            <item>227-9</item>
            <item>244-5</item>
            <item>244-7</item>
            <item>248-7</item>
         </list>
      The following notes could not be assigned a position within the text flow and are found at the end of the page, where they appeared in the printed edition.
      <list>
            <item>14-6</item>
            <item>20-3</item>
            <item>121-9</item>
            <item>210-7</item>
            <item>229-5</item>
            <item>240-1</item>
         </list>
      </p>
   </encodingDesc>
   <revisionDesc>
      <change when="2015-09-05" who="pma">Resolved references to notes.</change>
      <change when="2015-09-13" who="#sarit-encoder-dp">Corrected footnote reference:
      <list>
            <item> 95-8 to 95-7</item>
            <item>Second occurence of 121-4 to 121-7</item>
            <item>First occurence of 182-5 to 182-3</item>
         </list>
      </change>
      <change when="2015-09-14" who="#sarit-encoder-dp">Fixed problems with notes for which there where no references.</change>
      <change when="2015-12-30" who="#sarit-encoder-dp">Added @xml:lang to the front-element.</change>
      <change when="2016-03-08" who="#sarit-encoder-dp">Deleted the Index to the Nyāyabindu, which I had added manually.</change>
   </revisionDesc>
</teiHeader>
	 \end{minted}
       
      \clearpage
      \begin{english}
      \printshorthands
      \printbibliography
      \end{english}
    
\end{document}
