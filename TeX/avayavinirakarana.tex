\documentclass[article,12pt,a4paper]{memoir}%
    
      %% useful for debugging
      %% \usepackage{syntonly}%
      %%\syntaxonly%
    
	  \usepackage[normalem]{ulem}
	  \usepackage{eulervm}
	  \usepackage{xltxtra}
  \usepackage{polyglossia}
  \PolyglossiaSetup{sanskrit}{
  hyphenmins={2,3},% default is {1,3}
  }
  \setdefaultlanguage{sanskrit}
  % english etc. should also be available, notes and bib
  \setotherlanguages{english,german,italian,french}
  \setotherlanguage[numerals=arabic]{tibetan}
  \usepackage{fontspec}
  %% redefine some chars (either changed by parsing, or not commonly in font)
  \catcode`⃥=\active \def⃥{\textbackslash}
  \catcode`‿=\active \def‿{\textunderscore}
  \catcode`❴=\active \def❴{\{}
  \catcode`❵=\active \def❵{\}}
  \catcode`〔=\active \def〔{{[}}% translate 〔OPENING TORTOISE SHELL BRACKET
  \catcode`〕=\active \def〕{{]}}% translate 〕CLOSING TORTOISE SHELL BRACKET
  \catcode` =\active \def {\,}
  \catcode`·=\active \def·{\textbullet}
  %% BREAK PERMITTED HERE: \discretionary{-}{}{}\nobreak\hspace{0pt}
  \catcode`‚=\active \def‚{\-}
  \catcode`ꣵ=\active \defꣵ{%
  म्\textsuperscript{cb}%for candrabindu
  }
  %% show a lot of tolerance
  \tolerance=9000
  \def\textJapanese{\fontspec{Kochi Mincho}}
  \def\textChinese{\fontspec{HAN NOM A}}
  \def\textKorean{\fontspec{Baekmuk Gulim} }
  % make sure English font is there
  \newfontfamily\englishfont[Mapping=tex-text]{TeX Gyre Schola}
    % set up a devanagari font
  \newfontfamily\devanagarifont[Script=Devanagari,Mapping=devanagarinumerals,AutoFakeBold=1.5,AutoFakeSlant=0.3]{Chandas}
	\newfontfamily\rmlatinfont[Mapping=tex-text]{TeX Gyre Pagella}
	\newfontfamily\tibetanfont[Script=Tibetan,Scale=1.2]{Tibetan Machine Uni}
  \newcommand\bo\tibetanfont
  
    \defaultfontfeatures{Scale=MatchLowercase,Mapping=tex-text}
	\setmainfont{Chandas}
    \setsansfont{TeX Gyre Bonum}
  
  \setmonofont{DejaVu Sans Mono}
	  %% page layout start: fit to a4 and US letterpaper (example in memoir.pdf)
	  %% page layout start
	  % stocksize (actual size of paper in the printer) is a4 as per class
	  % options;
	  
	  % trimming, i.e., which part should be cut out of the stock (this also
	  % sets \paperheight and \paperwidth):
	  % \settrimmedsize{0.9\stockheight}{0.9\stockwidth}{*}
	  % \settrimmedsize{225mm}{150mm}{*}
	  % % say where you want to trim
	  \setlength{\trimtop}{\stockheight}    % \trimtop = \stockheight
	  \addtolength{\trimtop}{-\paperheight} %           - \paperheight
	  \setlength{\trimedge}{\stockwidth}    % \trimedge = \stockwidth
	  \addtolength{\trimedge}{-\paperwidth} %           - \paperwidth
	  % % this makes trims equal on top and bottom (which means you must cut
	  % % twice). if in doubt, cut on top, so that dust won't settle when book
	  % % is in shelf
	  \settrims{0.5\trimtop}{0.5\trimedge}

	  % figure out which font you're using
	  \setxlvchars
	  \setlxvchars
	  % \typeout{LENGTH: lxvchars: \the\lxvchars}
	  % \typeout{LENGTH: xlvchars: \the\xlvchars}

	  % set the size of the text block next:
	  % this sets \textheight and \textwidth (not the whole page including
	  % headers and footers)
	  \settypeblocksize{230mm}{130mm}{*}

	  % left and right margins:
	  % this way spine and edge margins are the same
	  % \setlrmargins{*}{*}{*}
	  \setlrmargins{*}{*}{1.5}

	  % upper and lower, same logic as before
	  % \setulmargins{*}{*}{*}% upper = lower margin
	  % \uppermargin = \topmargin + \headheight + \headsep
	  %\setulmargins{*}{*}{1.5}% 1.5*upper = lower margin
	  \setulmargins{*}{*}{1.5}% 

	  % header and footer spacings
	  \setheadfoot{2\baselineskip}{2\baselineskip}

	  % \setheaderspaces{ headdrop }{ headsep }{ ratio }
	  \setheaderspaces{*}{*}{1.5}

	  % see memman p. 51 for this solution to widows/orphans 
	  \setlength{\topskip}{1.6\topskip}
	  % fix up layout
	  \checkandfixthelayout
	  %% page layout end
	
	  \sloppybottom
	
	    % numbering depth
	    \maxtocdepth{section}
	    % set up layout of toc
	    \setpnumwidth{4em}
	    \setrmarg{5em}
	    \setsecnumdepth{all}
	    \newenvironment{docImprint}{\vskip 6pt}{\ifvmode\par\fi }
	    \newenvironment{docDate}{}{\ifvmode\par\fi }
	    \newenvironment{docAuthor}{\ifvmode\vskip4pt\fontsize{16pt}{18pt}\selectfont\fi\itshape}{\ifvmode\par\fi }
	    % \newenvironment{docTitle}{\vskip6pt\bfseries\fontsize{18pt}{22pt}\selectfont}{\par }
	    \newcommand{\docTitle}[1]{#1}
	    \newenvironment{titlePart}{ }{ }
	    \newenvironment{byline}{\vskip6pt\itshape\fontsize{16pt}{18pt}\selectfont}{\par }
	    % setup title page; see CTAN /info/latex-samples/TitlePages/, and memoir
	  \newcommand*{\plogo}{\fbox{$\mathcal{SARIT}$}}
	  \newcommand*{\makeCustomTitle}{\begin{english}\begingroup% from example titleTH, T&H Typography
	  \thispagestyle{empty}
	  \raggedleft
	  \vspace*{\baselineskip}
	  
	      % author(s)
	    {\Large Aśoka}\\[0.167\textheight]
	    % maintitle
	    {\Huge Avayavinirākaraṇa}\\[\baselineskip]
	    {\Large SARIT}\\\vspace*{\baselineskip}\plogo\par
	  \vspace*{3\baselineskip}
	  \endgroup
	  \end{english}}
	  \newcommand{\gap}[1]{}
	  \newcommand{\corr}[1]{($^{x}$#1)}
	  \newcommand{\sic}[1]{($^{!}$#1)}
	  \newcommand{\reg}[1]{#1}
	  \newcommand{\orig}[1]{#1}
	  \newcommand{\abbr}[1]{#1}
	  \newcommand{\expan}[1]{#1}
	  \newcommand{\unclear}[1]{($^{?}$#1)}
	  \newcommand{\add}[1]{($^{+}$#1)}
	  \newcommand{\deletion}[1]{($^{-}$#1)}
	  \newcommand{\quotelemma}[1]{\textcolor{cyan}{#1}}
	  \newcommand{\name}[1]{#1}
	  \newcommand{\persName}[1]{#1}
	  \newcommand{\placeName}[1]{#1}
	  % running latexPackages template
     \usepackage[x11names]{xcolor}
     \definecolor{shadecolor}{gray}{0.95}
     \usepackage{longtable}
     \usepackage{ctable}
     \usepackage{rotating}
     \usepackage{lscape}
     \usepackage{ragged2e}
     
	 \usepackage{titling}
	 \usepackage{marginnote}
	 \renewcommand*{\marginfont}{\color{black}\rmlatinfont\scriptsize}
	 \setlength\marginparwidth{.75in}
	 \usepackage{graphicx}
	 \graphicspath{{images/}}
	 \usepackage{csquotes}
       
	 \def\Gin@extensions{.pdf,.png,.jpg,.mps,.tif}
       
      \usepackage[noend,series={A,B}]{reledmac}
       % simplify what ledmac does with fonts, because it breaks. From the documentation of ledmac:
       % The notes are actually given seven parameters: the page, line, and sub-line num-
       % ber for the start of the lemma; the same three numbers for the end of the lemma;
       % and the font specifier for the lemma. 
       \makeatletter
       \def\select@lemmafont#1|#2|#3|#4|#5|#6|#7|%
       {}
       \makeatother
       \setlength{\stanzaindentbase}{20pt}
     \setstanzaindents{3,2,2,2,2,2,2,2,2,}
     % \setstanzapenalties{1,5000,10500}
     \lineation{page}
     % \linenummargin{inner}
     \linenumberstyle{arabic}
     \firstlinenum{5}
    \linenumincrement{5}
    \renewcommand*{\numlabfont}{\normalfont\scriptsize\color{black}}
    \addtolength{\skip\Afootins}{1.5mm}
    \Xnotenumfont{\bfseries\footnotesize}
    \sidenotemargin{outer}
    \linenummargin{inner}
    \Xarrangement{twocol}
    \arrangementX{twocol}
    %% biblatex stuff start
	 \usepackage[backend=biber,%
	 citestyle=authoryear,%
	 bibstyle=authoryear,%
	 language=english,%
	 sortlocale=en_US,%
	 ]{biblatex}
	 
		 \addbibresource[location=remote]{https://raw.githubusercontent.com/paddymcall/Stylesheets/HEAD/profiles/sarit/latex/bib/sarit.bib}
	 \renewcommand*{\citesetup}{%
	 \rmlatinfont
	 \biburlsetup
	 \frenchspacing}
	 \renewcommand{\bibfont}{\rmlatinfont}
	 \DeclareFieldFormat{postnote}{:#1}
	 \renewcommand{\postnotedelim}{}
	 %% biblatex stuff end
	 
	 \setcounter{errorcontextlines}{400}
       
	 \usepackage{lscape}
	 \usepackage{minted}
       
	   % pagestyles
	   \pagestyle{ruled}
	   
	   \makeoddfoot{ruled}{{\tiny\rmlatinfont \textit{Compiled: \today}}}{}{\rmlatinfont\thepage}
	   \makeevenfoot{ruled}{\rmlatinfont\thepage}{}{{\tiny\rmlatinfont \textit{Compiled: \today}}}
	   
	 
	   \usepackage{perpage}
           \MakePerPage{footnote}
	 
       \usepackage[destlabel=true,% use labels as destination names; ; see dvipdfmx.cfg, option 0x0010, if using xelatex
       pdftitle={Avayavinirākaraṇa // Aśoka},
       pdfauthor={SARIT: Search and Retrieval of Indic Texts. DFG/NEH Project (NEH-No.
	HG5004113), 2013-2017 },
       unicode=true]{hyperref}
       
       \renewcommand\UrlFont{\rmlatinfont}
       \newcounter{parCount}
       \setcounter{parCount}{0}
       % cleveref should come last; note: also consider zref, this could become more useful than cleveref?
       \usepackage[english]{cleveref}% clashes with eledmac < 1.10.1 standard
       \crefname{parCount}{§}{§§}
     
\begin{document}
    
     \makeCustomTitle
     \let\tabcellsep&
	\frontmatter
	\tableofcontents
	% \listoffigures
	% \listoftables
	\cleardoublepage
        \mainmatter 
	  
	% new div opening: depth here is 0
	
	    
	    \begingroup
	    \beginnumbering% beginning numbering from div depth=0
	    
	  
\chapter*[{॥ अव‚य‚विनिराक‚र‚ण‚म् ॥}]{॥ अव‚य‚विनिराक‚र‚ण‚म् ॥}\textsuperscript{\textenglish{1/tha}}‚{\tiny $_{lb}$}‚\textsuperscript{\textenglish{78b/msR}}

	  \refstepcounter{parCount}
	  \pstart \leavevmode% starting standard par
	॥ न‚मो लोक‚नाथाय\edtext{}{\lemma{नाथाय}\Bfootnote{न‚मः स‚म‚न्त‚भ‚द्राय \cite{avaya-S}}} ॥
	{\color{gray}{\textsuperscript{§~\theparCount}}}
	\pend% ending standard par
      ‚{\tiny $_{lb}$}‚
	    
	    \stanza[\smallbreak]
	  अभ्यासातिश‚य‚स्त‚था न विहित‚स्तादृक् श्रुतं नार्जितं&‚{\tiny $_{lb}$}‚त‚न्मे चाप‚ल‚म‚र्थ‚लेश‚म‚धुना निर्मातुमीहे य‚दि ।&‚{\tiny $_{lb}$}‚एत‚त् स‚र्व‚म‚वेक्ष्य य‚न्निग‚दितं त‚त् प‚ण्डितैर्दृश्य‚तां&‚{\tiny $_{lb}$}‚प्रीण‚न्त्येव विचेष्टितानि\edtext{}{\lemma{विचेष्टितानि}\Bfootnote{चेष्टितानि \cite{avaya-S}}} हि शिशोः प्रायः पितॄणां म‚नः ॥\&[\smallbreak]
	  
	  
	  ‚{\tiny $_{lb}$}‚

	  \refstepcounter{parCount}
	  \pstart \leavevmode% starting standard par
	यो विरुद्ध‚ध‚र्माध्यास‚वान् नासावेकः । य‚था घ‚टादिर‚र्थः । विरुद्ध‚ध‚र्मा‚{\tiny $_{lb}$}‚ध्यास‚वांश्च स्थूलो नीलादिर‚र्थ इति व्याप‚क‚विरुद्धोप‚ल‚ब्धिः । इह\edtext{}{\lemma{इह}\Bfootnote{°मान‚स्थूलो \cite{avaya-S}}} दृश्य‚{\tiny $_{lb}$}‚मानः स्थूलो नीलादिर‚र्थो ध‚र्मी । स चानुभ‚वाव‚सितोऽस‚ति बाध‚के प्र‚त्य‚क्ष‚{\tiny $_{lb}$}‚सिद्धो हेतोराश्र‚यासिद्धिं निह‚न्ति ।
	{\color{gray}{\textsuperscript{§~\theparCount}}}
	\pend% ending standard par
      ‚{\tiny $_{lb}$}‚

	  \refstepcounter{parCount}
	  \pstart \leavevmode% starting standard par
	न‚नु भ‚व‚द्भिर‚स‚न्नेवाव‚य‚वी प्र‚तिज्न्ँआय‚ते, स क‚थं प्र‚त्य‚क्ष‚सिद्धः ? नाव‚य‚{\tiny $_{lb}$}‚विन‚मिह ध‚र्मिणं प्र‚तिप‚न्नाः स्मः, किं तु प्र‚तिभास‚मानं स्थूल‚नीलादिक‚{\tiny $_{lb}$}‚म‚र्थ‚म् । न त‚र्हि विरुद्ध‚ध‚र्म‚संस‚र्गाद‚व‚य‚विन एक‚त्वं प्र‚तिषिद्धं\edtext{}{\lemma{तिषिद्धं}\Bfootnote{निषिद्धं \cite{avaya-S}}} स्यात् ।
	{\color{gray}{\textsuperscript{§~\theparCount}}}
	\pend% ending standard par
      ‚{\tiny $_{lb}$}‚

	  \refstepcounter{parCount}
	  \pstart \leavevmode% starting standard par
	य‚द्येवं क‚स्त‚र्हि भ‚व‚तोऽव‚य‚वी ? एकोऽनेकाव‚य‚व‚स‚म‚वेतोऽर्थ इति‚{\tiny $_{lb}$}‚ चेत् ? स किं प्र‚तिभास‚मानात् स्थूल‚नीलादेर‚न्योऽन‚न्यो वा ? न ताव‚द‚न्यः,‚{\tiny $_{lb}$}‚ दृश्य‚त्वेनाभ्युप‚ग‚मात् । न च प्र‚तिभास‚मान‚स्थूल‚नीलार्थ‚व्य‚तिरेकेणाप‚रः‚{\tiny $_{lb}$}‚ प्र‚तिभाति । अन‚न्य‚श्चेत् ? त‚स्यैक‚त्व‚प्र‚तिक्षेपे क‚थ‚म‚प्र‚तिषिद्ध‚मेक‚त्व‚म‚व‚{\tiny $_{lb}$}‚य‚विनः ?
	{\color{gray}{\textsuperscript{§~\theparCount}}}
	\pend% ending standard par
      ‚{\tiny $_{lb}$}‚

	  \refstepcounter{parCount}
	  \pstart \leavevmode% starting standard par
	न‚नु च प्र‚तिभास‚मान‚श्चेद‚र्थो नाव‚य‚विरूपो ध‚र्मी, किं त‚र्हि रूप‚म‚स्या‚{\tiny $_{lb}$}‚विशिष्टं प्र‚सिद्धं\edtext{}{\lemma{सिद्धं}\Bfootnote{°स्याव‚शिष्ट‚प्र‚सिद्धं \cite{avaya-S}}} भ‚विष्य‚तीति ? उच्य‚ते । प्र‚तिभास‚मानः स्थूलो नीलाकारः‚{\tiny $_{lb}$}‚ प्र‚त्य‚क्ष‚सिद्धं रूप‚म् अव‚य‚वित्वान‚भ्युप‚ग‚मेऽप्य‚व‚शिष्य‚ते । न चैवं म‚न्त‚व्य‚म् ।‚{\tiny $_{lb}$}‚ न प‚र‚माणुसंच‚य‚वादिनां स्थूलाकारः क‚श्चिद‚र्थोऽस्ति, यः प्र‚सिद्धो ध‚र्मी‚{\tiny $_{lb}$}‚ स्यादिति । अव‚य‚व्य‚न‚भ्युप‚ग‚मेऽपि हि निर‚व‚य‚वानेकात्म‚कः स्थूलोऽर्थोऽभ्युप‚ग‚म्य‚{\tiny $_{lb}$}‚मानः केन निषिद्धः ?
	{\color{gray}{\textsuperscript{§~\theparCount}}}
	\pend% ending standard par
      ‚{\tiny $_{lb}$}‚\textsuperscript{\textenglish{2/tha}}

	  \refstepcounter{parCount}
	  \pstart \leavevmode% starting standard par
	स्यान्म‚त‚म्, निर‚व‚य‚वाः प‚र‚माण‚वः क‚थं प्र‚त्य‚क्षे स्थूलेनाकारेणाव‚{\tiny $_{lb}$}‚भासेर‚न्, स्थूल‚सूक्ष्म‚योर्विरोधादिति ? त‚द‚पि न युक्त‚म्, प‚र‚माण‚व एव हि‚{\tiny $_{lb}$}‚ प‚र‚रूप‚देश‚प‚रिहारेणोत्प‚न्नाः प‚र‚स्प‚र‚स‚हिता अव‚भास‚माना देश‚वितान‚व‚न्तो‚{\tiny $_{lb}$}‚ भास‚न्ते । वित‚त‚देश‚त्वं च स्थूल‚त्व‚म् ।
	{\color{gray}{\textsuperscript{§~\theparCount}}}
	\pend% ending standard par
      ‚{\tiny $_{lb}$}‚

	  \refstepcounter{parCount}
	  \pstart \leavevmode% starting standard par
	य‚त् तूक्त‚म्, स्थूल‚सूक्ष्म‚योर्विरोध इति, त‚त् किं प‚र‚रूप‚देश‚प‚रिहार‚व‚ती‚{\tiny $_{lb}$}‚ निर‚व‚य‚वानामुत्प‚त्तिः, प‚र‚स्प‚र‚स‚हितानामेक‚विज्न्ँआनाव‚भासित्वं \edtext{}{\lemma{भासित्वं}\Bfootnote{वा विरुद्ध‚म् \cite{avaya-S}}}वाविरुद्ध‚मिति‚{\tiny $_{lb}$}‚ व‚क्तुम‚ध्य‚व‚सित‚म् ।
	{\color{gray}{\textsuperscript{§~\theparCount}}}
	\pend% ending standard par
      ‚{\tiny $_{lb}$}‚

	  \refstepcounter{parCount}
	  \pstart \leavevmode% starting standard par
	इदं चेद‚विरुद्ध‚म्\edtext{}{\lemma{म्}\Bfootnote{चेद् विरुद्ध‚म् \cite{avaya-S}}}, स्थूलाकारः किं विरुद्धः ? अतएव निर‚व‚य‚वेषु‚{\tiny $_{lb}$}‚ ब‚हुष्वेक‚स्मिन् विज्न्ँआने प्र‚तिभास‚मानेषु भ‚व‚न् प्र‚तिभास‚काल‚भावी प्र‚तिभास‚{\tiny $_{lb}$}‚ध‚र्मः स्थूलाकारो न तु वास्त‚वः, प्र‚त्येक‚म‚भावात् । प्र‚तिभासात् प्राग् उर्ध्वं‚{\tiny $_{lb}$}‚ वा न त‚र्हि स्थूल इति चेत्--न, त‚दापि प्र‚तिभास‚योग्य‚तासं\leavevmode\ledsidenote{\textenglish{79a/msR}}‚{\tiny $_{lb}$}‚भ‚वात् । य‚दैव हि संचिता भ‚वेयुः, त‚दैव प्र‚तिभास‚योग्याः प‚र‚माण‚वः । य‚दा‚{\tiny $_{lb}$}‚ च प्र‚तिभास‚योग्यास्त‚दा स्थूलाः ।
	{\color{gray}{\textsuperscript{§~\theparCount}}}
	\pend% ending standard par
      ‚{\tiny $_{lb}$}‚

	  \refstepcounter{parCount}
	  \pstart \leavevmode% starting standard par
	य‚स्त्वाह निर‚व‚य‚वेषु ब‚हुषु प्र‚तिभास‚मानेष्व‚व‚श्य‚म‚न्त‚रेणापि प्र‚ति‚{\tiny $_{lb}$}‚भासित‚व्य‚म् । अन्त‚रान‚व‚भासे प‚र‚स्प‚र‚विविक्ता एव नाव‚भासिताः स्युः ।‚{\tiny $_{lb}$}‚ विविक्तान‚व‚भासे चाणुमात्र‚कं पिण्डो भ‚वेत् । न चान्त‚र‚म‚व‚भास‚मान‚मुत्प‚{\tiny $_{lb}$}‚श्यामः । त‚द‚यं निर‚न्त‚र एक एव स्थूलो न\edtext{}{\lemma{न}\Bfootnote{न \textit{om} \cite{avaya-S}}} निर‚व‚य‚वानेकात्म‚को भ‚वितु‚{\tiny $_{lb}$}‚म‚र्ह‚ति, अपि त्वेकः स्थूलात्म‚क एवेति । सोऽप्येवं वाच्यः । किं विजातीय‚{\tiny $_{lb}$}‚प‚र‚माण्व‚न्त‚र‚म‚न्त‚र‚म्, आहोस्वित् शून्याकाश‚योगः ? त‚त्र शून्याकाश‚{\tiny $_{lb}$}‚योग‚स्ताव‚द‚व‚स्तुत‚या नेन्द्रिय‚प्र‚त्य‚क्ष‚गोच‚रः, अर्थ‚साम‚र्थ्य‚भाविनि त‚त्रार्थ‚स्यैव‚{\tiny $_{lb}$}‚ प्र‚तिभासोप‚प‚त्तेः । विजातीयं त्विन्द्रियान्त‚र‚ग्राह्यं क‚थ‚मिन्द्रियान्त‚र‚ज्न्ँआने‚{\tiny $_{lb}$}‚ अव‚भासेत, अविष‚य‚त्वात् ? त‚त्र केनान्त‚रेणाव‚भासित‚व्य‚मिति न विद्मः ।
	{\color{gray}{\textsuperscript{§~\theparCount}}}
	\pend% ending standard par
      ‚{\tiny $_{lb}$}‚

	  \refstepcounter{parCount}
	  \pstart \leavevmode% starting standard par
	य‚त् पुन‚रुच्य‚ते, अन्त‚रालान‚व‚भासे विविक्ता नाव‚भासेर‚न्निति, त‚त्रापि‚{\tiny $_{lb}$}‚ किमेषाम‚न्त‚रं विवेकः प‚र‚रूप‚शून्य‚ता वेति चिन्त्य‚म् । निरूप‚य‚न्त‚स्तु प‚र‚रूप‚{\tiny $_{lb}$}‚शून्य‚तामेव विवेकं व‚स्तूनां प‚श्यामः । प‚र‚रूप‚शून्याश्चेमे भास‚मानाः क‚थं‚{\tiny $_{lb}$}‚ विविक्ता नाव‚भासेर‚न्निति ?
	{\color{gray}{\textsuperscript{§~\theparCount}}}
	\pend% ending standard par
      ‚{\tiny $_{lb}$}‚‚{\tiny $_{lb}$}‚\textsuperscript{\textenglish{3/tha}}

	  \refstepcounter{parCount}
	  \pstart \leavevmode% starting standard par
	न‚नु च न प्र‚त्येकं प‚र‚माणूनां स्थूलाकारः, त‚तः स‚मुदितानामेष्ट‚व्यः ।‚{\tiny $_{lb}$}‚ य‚था च नील‚प‚र‚माणूनां प्र‚त्येक‚म‚संभ‚व‚न् पीताकारो ब‚हुष्व‚पि न दृश्य‚ते,‚{\tiny $_{lb}$}‚ त‚था प्र‚त्येक‚म‚संभ‚व‚न् स्थूलाकारः क‚थं ब‚हुषु स्यादिति ?
	{\color{gray}{\textsuperscript{§~\theparCount}}}
	\pend% ending standard par
      ‚{\tiny $_{lb}$}‚

	  \refstepcounter{parCount}
	  \pstart \leavevmode% starting standard par
	उक्त‚म‚त्र, अविरुद्धो निर‚व‚य‚वाणां स्थूलाकार इति । नील‚विरुद्ध‚स्तु‚{\tiny $_{lb}$}‚ पीताकारः । त‚तः पीत‚विरुद्धं नीलाकारं विभ्राणेषु ब‚हुष्व‚पि क‚थं पीताकारो‚{\tiny $_{lb}$}‚ भ‚वेत् ? नैवं निर‚व‚य‚व‚त्व‚विरुद्धः स्थूलाकारः, य‚थोक्त‚न्यायात् । त‚देवं‚{\tiny $_{lb}$}‚ प्र‚त्य‚क्ष‚सिद्धः स्थूलोऽर्थ इह ध‚र्मीति व्य‚व‚स्थित‚म् ।
	{\color{gray}{\textsuperscript{§~\theparCount}}}
	\pend% ending standard par
      ‚{\tiny $_{lb}$}‚

	  \refstepcounter{parCount}
	  \pstart \leavevmode% starting standard par
	त‚देवं व्य‚व‚स्थिते\edtext{}{\lemma{स्थिते}\Bfootnote{त‚देव‚म‚व‚स्थिते \cite{avaya-S}}} ध‚र्मिणि हेतोः स‚त्त्व‚म‚स‚त्त्वं वा निरूप्य\edtext{}{\lemma{निरूप्य}\Bfootnote{निरूप्य‚म् \cite{avaya-S}}} त‚त्र पाण्या‚{\tiny $_{lb}$}‚दावेक‚स्मिन् क‚म्प‚माने स्थूलोऽर्थः स‚क‚म्प‚निष्क‚म्पे रूपे युग‚प‚त् प्र‚तिप‚द्य‚मानः‚{\tiny $_{lb}$}‚ क‚थं विरुद्ध‚ध‚र्म‚संस‚र्ग‚वान् न स्यात् ? स‚क‚म्प‚निष्क‚म्प‚योर्हि रूप‚योः प‚र‚स्प‚रा‚{\tiny $_{lb}$}‚भावाव्य‚भिचार‚निमित्त‚कोऽस्ति विरोधः । भावाभाव‚योरेव हि प‚र‚स्प‚र‚{\tiny $_{lb}$}‚प‚रिहारात्म‚को विरोधः । व‚स्तुनोस्तु त‚द‚भावाव्य‚भिचारेणैव । इह च स‚क‚म्प‚{\tiny $_{lb}$}‚निष्क‚म्पे रूपे गृह्ण‚त् प्र‚त्य‚क्ष‚मेव साम‚र्थ्यात् प‚र‚स्प‚राभावं साध‚य‚ति ।‚{\tiny $_{lb}$}‚ व्य‚व‚हार‚य‚ति तु निर्विशेष‚णैवानुप‚ल‚ब्धिः ।
	{\color{gray}{\textsuperscript{§~\theparCount}}}
	\pend% ending standard par
      ‚{\tiny $_{lb}$}‚

	  \refstepcounter{parCount}
	  \pstart \leavevmode% starting standard par
	स्यादेत‚त् । पाण्यादावेक‚स्मिन्न‚व‚य‚वे क‚म्प‚माने नाव‚य‚विनः क‚म्प‚{\tiny $_{lb}$}‚\leavevmode\ledsidenote{\textenglish{79b/msR}}रूप‚म‚भ्युपेय‚म् । अव‚य‚व एव हि त‚दा क्रियावान् दृश्य‚ते । न चेदं‚{\tiny $_{lb}$}‚ म‚न्त‚व्य‚म्, अव‚य‚वे क्रियाव‚ति त‚दाधेयेनाव‚य‚विनापि क्रियाव‚ता भ‚वित‚व्य‚म्,‚{\tiny $_{lb}$}‚ य‚था र‚थे च‚ल‚ति त‚दारुढोऽपि च‚ल‚तीति । अव‚य‚वाव‚य‚विक‚र्म‚णोर्भिन्न‚निमित्त‚{\tiny $_{lb}$}‚त्वात्, निमित्तायौग‚प‚द्याच्च ।
	{\color{gray}{\textsuperscript{§~\theparCount}}}
	\pend% ending standard par
      ‚{\tiny $_{lb}$}‚

	  \refstepcounter{parCount}
	  \pstart \leavevmode% starting standard par
	य‚दा हि आत्म‚नः पाणिक‚म्प‚नेच्छा भ‚व‚ति, त‚तः प्र‚य‚त्नः, त‚दा‚{\tiny $_{lb}$}‚ प्र‚य‚त्न‚व‚दात्म‚पाणिसंयोगात् पाणौ क्रिया । य‚दा तु श‚रीर‚क‚म्प‚नार्थः प्र‚य‚त्न‚विशेष‚{\tiny $_{lb}$}‚ आत्म‚नो जाय‚ते, त‚दा त‚त्प्र‚य‚त्न‚व‚दात्म‚श‚रीर‚संयोगात् श‚रीरे क‚र्मेति निमित्त‚{\tiny $_{lb}$}‚भेदः । अतो \edtext{}{\lemma{अतो}\Bfootnote{भिन्न‚मिति त‚योः \cite{avaya-S}}}भिन्न‚निमित्त‚योरेक‚स्मिन् जाते क‚थं निमित्तान्त‚र‚प्र‚तिब‚द्ध‚ज‚न्मा‚{\tiny $_{lb}$}‚ त‚द‚भावे त‚द‚प‚रोऽपि त‚थैव जायेत ? र‚थ‚त‚दारुढ‚क‚र्म‚णोस्तु स‚त्य‚पि निमित्त‚{\tiny $_{lb}$}‚भेदे निमित्त‚यौग‚प‚द्यात् स‚ह‚भावः ।
	{\color{gray}{\textsuperscript{§~\theparCount}}}
	\pend% ending standard par
      ‚{\tiny $_{lb}$}‚

	  \refstepcounter{parCount}
	  \pstart \leavevmode% starting standard par
	त‚था हि नोद‚नं र‚थ‚क‚र्म‚णीऽस‚म‚वायिकार‚ण‚म्, त‚दारुढ‚क‚र्म‚ण‚स्तु नोद्य‚{\tiny $_{lb}$}‚संयोगः । त‚योश्च स‚ह‚भावात् क‚र्म‚णी अपि यौग‚प‚द्येन जायेते । त‚देत‚त्‚{\tiny $_{lb}$}‚ \leavevmode\ledsidenote{\textenglish{4/tha}} स‚क‚ल‚मालोच्य भ‚द‚न्त‚ध‚र्मोत्त‚रेण इद‚मुक्त‚म्--न चात्राव‚य‚वः क्रियावान् ।‚{\tiny $_{lb}$}‚ अव‚य‚वेषु हि क्रियाव‚स्तु विभागो जाय‚ते, क्रियाया विभागार‚म्भं प्र‚ति निर‚पेक्ष‚{\tiny $_{lb}$}‚कार‚ण‚त्वात् । तेन च संयोगेऽस‚म‚वायिकार‚णे \edtext{}{\lemma{णे}\Bfootnote{निव‚र्त‚ते \cite{avaya-S}}}निव‚र्तिते निव‚र्तेताव‚य‚वि‚{\tiny $_{lb}$}‚द्र‚व्य‚मिति ।
	{\color{gray}{\textsuperscript{§~\theparCount}}}
	\pend% ending standard par
      ‚{\tiny $_{lb}$}‚

	  \refstepcounter{parCount}
	  \pstart \leavevmode% starting standard par
	इद‚माहुर‚त्र क‚णाद‚शिष्याः--नोद‚नाद‚भिघाताद्\edtext{}{\lemma{भिघाताद्}\Bfootnote{°द‚विघाताद्वा \cite{avaya-msR}}} वा जाय‚मानः क्रिया‚{\tiny $_{lb}$}‚विशेषो द्र‚व्यार‚म्भ‚क‚संयोग‚प्र‚तिद्व‚न्द्विविभागार‚म्भ‚को दृष्टः । अन्य‚त् तु विभाग‚{\tiny $_{lb}$}‚मात्र‚मार‚भ‚ते, न तु य‚थोक्त‚विशेष‚म् । अन्य‚स्मिन् प‚द्म‚संकोच‚विकाशादिके‚{\tiny $_{lb}$}‚ क‚र्म‚णि अव‚य‚व‚संयोग‚निवृत्तेर‚द‚र्श‚नात् । त‚द्व‚त् प्र‚य‚त्न‚ज‚न्य‚स्यापि क‚र्म‚णो‚{\tiny $_{lb}$}‚ नास्ति संयोग‚निव‚र्त‚न‚साम‚र्थ्य‚म् । त‚तो नास्ति द्र‚व्य‚निवृत्तिरिति ।
	{\color{gray}{\textsuperscript{§~\theparCount}}}
	\pend% ending standard par
      ‚{\tiny $_{lb}$}‚

	  \refstepcounter{parCount}
	  \pstart \leavevmode% starting standard par
	अय‚म‚त्र स‚माधिः । इहाव‚य‚विनि क्रियाव‚ति निय‚मेनाव‚य‚वैर‚पि क्रिया‚{\tiny $_{lb}$}‚व‚द्भिर्भ‚वित‚व्य‚म् । अन्य‚थाव‚य‚व‚संसृष्टेभ्य आकाश‚देशेभ्यो विभागोऽव‚य‚वासं‚{\tiny $_{lb}$}‚सृष्टैश्च संयोगोऽव‚य‚विनः क्रियाव‚तः स्यात् । अव‚य‚वास्तु क्रियाविर‚हिणः‚{\tiny $_{lb}$}‚ पूर्वाक्रान्तेभ्यो न‚भोदेशेभ्यो नाप‚स‚र्पेयुः । अव‚य‚विस‚माक्रान्तैश्च देशैर्नाभि‚{\tiny $_{lb}$}‚संब‚ध्येर‚न् ।
	{\color{gray}{\textsuperscript{§~\theparCount}}}
	\pend% ending standard par
      ‚{\tiny $_{lb}$}‚

	  \refstepcounter{parCount}
	  \pstart \leavevmode% starting standard par
	न चाव‚य‚विसंयोग‚विभागाभ्याम‚व‚य‚व‚संयोग‚विभागौ वाच्यौ, कार्य‚{\tiny $_{lb}$}‚संयोग‚विभाग‚योः कार‚ण‚संयोग‚विभागौ प्र‚ति निमित्त‚भावान‚भ्युप‚ग‚मात् ।‚{\tiny $_{lb}$}‚ कार‚ण‚संयोग‚विभागावेव हि कार्य‚संयोग‚विभागावार‚भेते ।
	{\color{gray}{\textsuperscript{§~\theparCount}}}
	\pend% ending standard par
      ‚{\tiny $_{lb}$}‚

	  \refstepcounter{parCount}
	  \pstart \leavevmode% starting standard par
	न च विप‚र्य‚यः, सिद्धान्त‚हानिप्र‚स‚ङ्गात् । तेनाव‚य‚वाव‚य‚विनां पृथ‚ग्देश‚{\tiny $_{lb}$}‚त्वात् \leavevmode\ledsidenote{\textenglish{80a/msR}} आधार्याधार‚भाव‚हानौ स‚म‚वायोऽपि न व्य‚व‚तिष्ठेत । त‚स्माद्‚{\tiny $_{lb}$}‚ अव‚य‚विनि क्रियाव‚ति अव‚य‚वाः क्रियाव‚न्तः इत्य‚काम‚कैर‚पि वैशेषिकैः इद‚म‚भ्यु‚{\tiny $_{lb}$}‚पेय‚म् । य‚था चाव‚य‚विक्रियायाम‚व‚य‚वाः क्रियाव‚न्तः, त‚था अव‚य‚वेष्व‚पि‚{\tiny $_{lb}$}‚ क्रियाव‚त्सु त‚द‚व‚य‚वैः क्रियाव‚द्भिर्भाव्य‚मिति । अन‚या दिशा द्व्य‚णुके क्रिया‚{\tiny $_{lb}$}‚व‚ति त‚दार‚म्भ‚कौ प‚र‚माणू क्रियाव‚न्तौ म‚न्त‚व्यौ ।
	{\color{gray}{\textsuperscript{§~\theparCount}}}
	\pend% ending standard par
      ‚{\tiny $_{lb}$}‚

	  \refstepcounter{parCount}
	  \pstart \leavevmode% starting standard par
	अत‚श्च‚लाव‚य‚व‚संब‚न्धिन‚श्च‚लाः प‚र‚माण‚वः । निष्क्रियाव‚य‚व‚संब‚न्धि‚{\tiny $_{lb}$}‚न‚स्तु निष्क्रियाः । प‚र‚माणुक्रियायां त‚दार‚ब्ध‚म‚पि द्व्य‚णुक‚द्र‚व्यं क्रियाव‚त्‚{\tiny $_{lb}$}‚ प्र‚स‚ज्येत, स‚र्वाव‚य‚व‚क्रियायाः कार्य‚क्रियाविनाभूत‚त्वात् ।
	{\color{gray}{\textsuperscript{§~\theparCount}}}
	\pend% ending standard par
      ‚{\tiny $_{lb}$}‚\textsuperscript{\textenglish{5/tha}}

	  \refstepcounter{parCount}
	  \pstart \leavevmode% starting standard par
	एक‚स्य प‚र‚माणोः क्रिया न प‚र‚स्येति चेत् ? स‚क्रिय‚निष्क्रियौ त‚र्हि‚{\tiny $_{lb}$}‚ प‚र‚माणू प‚र‚स्प‚रं विभ‚ज्य‚मानावुज्झित‚द्र‚व्यार‚म्भ‚क‚संयोगौ स्याताम् । त‚त‚श्च‚{\tiny $_{lb}$}‚ द्र‚व्य‚नाशः । एवं च द्व्य‚णुक‚द्र‚व्य‚स्य क्रियाव‚त्त्वे त्र्य‚णुकेऽपि क्रिया स्यादित्य‚नेन‚{\tiny $_{lb}$}‚ क्र‚मेण प‚रिदृश्य‚मानोऽच‚ल‚श्च‚लः स्याद‚व‚य‚वः । त‚स्मात् निष्क्रियाव‚य‚व‚संब‚{\tiny $_{lb}$}‚न्धिनो निष्क्रियाः प‚र‚माण‚वः च‚लाव‚य‚व‚संब‚न्धिभ्य‚श्च‚लेभ्यो विभ‚ज्य‚माना‚{\tiny $_{lb}$}‚ द्र‚व्यार‚म्भ‚क‚संयोग‚विनाश‚व‚न्तः स्युः । निर‚व‚य‚वो हि प‚र‚माणुर्य‚म‚णुमात्र‚न‚भो‚{\tiny $_{lb}$}‚देश‚माक्र‚म्य प‚र‚माण्व‚न्त‚रेण संयुज्य‚ते, त‚तो विभ‚ज्य‚मानः क‚थ‚म‚नुज्झित‚{\tiny $_{lb}$}‚प‚र‚माण्व‚न्त‚र‚प्र‚त्यास‚त्तिकः स्यात् ? साव‚य‚वो हि भाव एकेनाव‚य‚वेन व‚स्त्व‚{\tiny $_{lb}$}‚न्त‚र‚संयुक्त एवाव‚य‚वान्त‚रैश्च‚लैः पूर्वाक्रान्तान् न‚भोदेशान् विर‚ह‚थ्य देशा‚{\tiny $_{lb}$}‚न्त‚र‚माक्रामेत् । निर‚व‚य‚वेषु त्व‚त्य‚न्त‚म‚स्त‚मितेयं क‚थेति युक्तं द्र‚व्य‚नाश‚प्र‚स‚ङ्ग‚म्‚{\tiny $_{lb}$}‚ उत्प‚श्यामः ।
	{\color{gray}{\textsuperscript{§~\theparCount}}}
	\pend% ending standard par
      ‚{\tiny $_{lb}$}‚

	  \refstepcounter{parCount}
	  \pstart \leavevmode% starting standard par
	अथ वान्य‚थायं विरुद्ध‚ध‚र्म‚संस‚र्गः । त‚था ह्यावृते एक‚स्मिन् पाण्यादौ‚{\tiny $_{lb}$}‚ स्थूल‚स्यावृतानावृते रूपे युग‚प‚दुद्भ‚व‚न्ती विरुद्ध‚ध‚र्म‚द्व‚य‚संयोग‚म‚स्यावेद‚य‚तः ।‚{\tiny $_{lb}$}‚ न चानावृतैक‚रूप एवाय‚मिति श‚क्यं व‚क्तुम्, अर्द्धाव‚र‚णेऽपि अनावृत‚स्य‚{\tiny $_{lb}$}‚ पूर्व‚व‚द् द‚र्श‚न‚प्र‚स‚ङ्गात् ।
	{\color{gray}{\textsuperscript{§~\theparCount}}}
	\pend% ending standard par
      ‚{\tiny $_{lb}$}‚

	  \refstepcounter{parCount}
	  \pstart \leavevmode% starting standard par
	\edtext{\textsuperscript{*}}{\lemma{*}\Bfootnote{त‚त् \textit{om} \cite{avaya-S}}}त‚द‚व‚य‚व‚द‚र्श‚नाय‚त्तोप‚ल‚ब्धेः त‚द‚दृष्टाव‚स्यादृष्टिरिति चेत् ? अदृष्टिरेव‚{\tiny $_{lb}$}‚ त‚र्हि अस्यास्तु, न तु दृष्ट्य‚दृष्टी ।
	{\color{gray}{\textsuperscript{§~\theparCount}}}
	\pend% ending standard par
      ‚{\tiny $_{lb}$}‚

	  \refstepcounter{parCount}
	  \pstart \leavevmode% starting standard par
	न‚न्व‚स्त्येवाव‚य‚विद‚र्श‚न‚म‚व‚य‚वाव‚र‚णेऽपीति कोऽयं प्र‚स‚ङ्गः ? एवं त‚र्हि‚{\tiny $_{lb}$}‚ स्थूलः प्राग्व‚दुप‚ल‚भ्येत ।
	{\color{gray}{\textsuperscript{§~\theparCount}}}
	\pend% ending standard par
      ‚{\tiny $_{lb}$}‚

	  \refstepcounter{parCount}
	  \pstart \leavevmode% starting standard par
	इद‚म‚त्राह क‚श्चित् । भूयोऽव‚य‚वेन्द्रिय‚स‚न्निक‚र्ष‚स‚हायोऽव‚य‚वीन्द्रिय‚स‚न्नि‚{\tiny $_{lb}$}‚क‚र्षः स्थूलोप‚ल‚ब्धेर्निमित्त‚म् । न चार्द्धाव‚र‚णे भूय‚साम‚व‚य‚वानाम‚स्तीन्द्रिय‚{\tiny $_{lb}$}‚स‚न्निक‚र्षः । त‚तो न स्थूल‚ग्र‚ह‚ण‚मिति ।
	{\color{gray}{\textsuperscript{§~\theparCount}}}
	\pend% ending standard par
      ‚{\tiny $_{lb}$}‚

	  \refstepcounter{parCount}
	  \pstart \leavevmode% starting standard par
	सोऽप्येवं प्र‚ष्ट‚व्यः । किं स्थूलो नामाव‚य‚विनोऽन्यः ? स ए\add{व} वानेक‚{\tiny $_{lb}$}‚व्यापी स्थूलः ? त‚त्र य‚द्य‚व‚य‚व्येव स्थूलः, त‚दा \leavevmode\ledsidenote{\textenglish{80b/msR}} त‚द्ग्र‚ह‚णे क‚थं न‚{\tiny $_{lb}$}‚ स्थूल‚ग्र‚ह‚ण‚मिति चिन्त्य‚म् ।
	{\color{gray}{\textsuperscript{§~\theparCount}}}
	\pend% ending standard par
      ‚{\tiny $_{lb}$}‚

	  \refstepcounter{parCount}
	  \pstart \leavevmode% starting standard par
	अथ स्थूल‚त्वाख्यः प‚रिमाण‚विशेषो गुणः, \edtext{\textsuperscript{*}}{\lemma{*}\Bfootnote{स द्र‚व्या° \cite{avaya-S}}}स च द्र‚व्याद‚न्य एवेति‚{\tiny $_{lb}$}‚ चेत् ? एवं त‚र्हि प‚रिमाण‚र‚हित‚मेत‚द्\edtext{}{\lemma{द्}\Bfootnote{°मेव त‚त् \cite{avaya-S}}} द्र‚व्य‚मुप‚ल‚भ्य‚ते । न त्व‚स्यान्येनान्येन‚{\tiny $_{lb}$}‚ \leavevmode\ledsidenote{\textenglish{6/tha}} प‚रिमाणेन योगः । न चास्यानेक‚प‚रिमाण‚क‚ल्प‚नापि साध्वी, युग‚प‚त् स‚र्वेषा‚{\tiny $_{lb}$}‚म‚नुप‚ल‚ब्धेः । न च \edtext{}{\lemma{च}\Bfootnote{द‚र्श‚न‚स्य \cite{avaya-msR}}}दृश्य‚स्य द‚र्श‚न‚निवृत्तिर्युक्ता ।
	{\color{gray}{\textsuperscript{§~\theparCount}}}
	\pend% ending standard par
      ‚{\tiny $_{lb}$}‚

	  \refstepcounter{parCount}
	  \pstart \leavevmode% starting standard par
	अव‚य‚व‚स्यैव त‚त् प‚रिमाण‚मुप‚ल‚भ्य‚त इति चेत् ? य‚द्येवं स एव त‚र्हि‚{\tiny $_{lb}$}‚ अव‚य‚वः स्वेन प‚रिमाणेन संब‚द्धः प्र‚त्य‚क्षोऽस्त्वाव‚र‚ण‚काले नाव‚य‚वी ।
	{\color{gray}{\textsuperscript{§~\theparCount}}}
	\pend% ending standard par
      ‚{\tiny $_{lb}$}‚

	  \refstepcounter{parCount}
	  \pstart \leavevmode% starting standard par
	स्यादेत‚त् । एकार्थ‚स‚म‚वायाद् भ्रान्तिनिमित्ताद‚व‚य‚व्येवाव‚य‚व‚प‚रिमानेन‚{\tiny $_{lb}$}‚ संब‚द्धः प्र‚तिभातीति, त‚द‚पि न शोभ‚न‚म् । द्व‚योः स‚म‚प‚रिमाण‚योर‚व‚य‚वाव‚य‚{\tiny $_{lb}$}‚विनोः प्र‚तिभास‚प्र‚स‚ङ्गः । य‚स्य हि म‚ह‚तोऽव‚य‚व‚स्य त‚त् प‚रिमाण‚म्, स‚{\tiny $_{lb}$}‚ ताव‚त् स्वेन प‚रिमाणेन संब‚द्धः प्र‚तिभाति । अव‚य‚विन‚श्च त‚त्प‚रिमाण‚{\tiny $_{lb}$}‚संब‚न्ध‚प्र‚तिभासाभ्युप‚ग‚मे द्व‚योः स‚म‚प‚रिमाण‚योर‚व‚य‚वाव‚य‚विनोः प्र‚तिभास‚{\tiny $_{lb}$}‚ आस‚ज्य‚ते ।
	{\color{gray}{\textsuperscript{§~\theparCount}}}
	\pend% ending standard par
      ‚{\tiny $_{lb}$}‚

	  \refstepcounter{parCount}
	  \pstart \leavevmode% starting standard par
	न चास्ति य‚थोक्तः प्र‚तिभासः । प‚रिमाण‚र‚हितः सोऽव‚य‚वः प्र‚तिभास‚त‚{\tiny $_{lb}$}‚ इति चेत् ? स्यादेत‚त् । य‚स्याव‚य‚व‚स्य प‚रिमाणेनाव‚य‚वी संब‚द्धः प्र‚तिभाति‚{\tiny $_{lb}$}‚ स स्वं प‚रिमाणं प‚रित्य‚ज्याभातीति । इद‚म‚पि प‚रिमाण‚विर‚हिणोऽव‚य‚व‚स्या‚{\tiny $_{lb}$}‚दृष्टेर‚श‚क्यं क‚ल्प‚यितुम् । एकार्थ‚स‚म‚वायाच्च भ्रान्तिनिमित्ताद‚ल्प‚त‚राव‚य‚व‚{\tiny $_{lb}$}‚प‚रिमाण‚वान‚प्य‚व‚य‚वी प्र‚तिभासेत । न च बाध‚क‚म‚न्त‚रेण भ्रान्तिर‚पि श‚क्या‚{\tiny $_{lb}$}‚ व्य‚व‚स्थाप‚यितुम् ।
	{\color{gray}{\textsuperscript{§~\theparCount}}}
	\pend% ending standard par
      ‚{\tiny $_{lb}$}‚

	  \refstepcounter{parCount}
	  \pstart \leavevmode% starting standard par
	अस्तु त‚र्हि स्थूलोऽव‚य‚व एव प्र‚त्य‚क्षः । \edtext{\textsuperscript{*}}{\lemma{*}\Bfootnote{एव‚म‚नाव° \cite{avaya-msR}}}एव‚माव‚र‚ण‚काले इवाना‚{\tiny $_{lb}$}‚व‚र‚णाव‚स्थायाम‚पि स्थूल‚त‚रोऽव‚य‚वः प्र‚त्य‚क्षोऽस्तु, प‚र‚म‚ध्य‚व‚र्तिनाम‚व‚य‚वाना‚{\tiny $_{lb}$}‚मिन्द्रिय‚स‚न्निक‚र्षाभावे स्थूल‚त‚माव‚य‚विद‚र्श‚नानुप‚प‚त्तेः । स्यान्म‚तं निखिलाव‚य‚व‚{\tiny $_{lb}$}‚द‚र्श‚न न स्थूल‚प्र‚त्य‚क्ष‚कार‚ण‚म्, किं तु भूयोऽव‚य‚व‚द‚र्श‚न‚मिति ।
	{\color{gray}{\textsuperscript{§~\theparCount}}}
	\pend% ending standard par
      ‚{\tiny $_{lb}$}‚

	  \refstepcounter{parCount}
	  \pstart \leavevmode% starting standard par
	इद‚म‚पि न स‚म्य‚क् । अभिमुखाव‚स्थित‚स्य हि प‚र्व‚तादेर‚र्वाचीनाव‚य‚व‚{\tiny $_{lb}$}‚द‚र्श‚ने न त‚था स्थूल‚प्र‚तिप‚त्तिः, य‚था अर्वाक्प‚र‚म‚ध्य‚व‚र्तिनां द‚र्श‚ने । त‚तो न‚{\tiny $_{lb}$}‚ याव‚त् निर‚व‚शेषाव‚य‚व‚द‚र्श‚न‚म्, ताव‚त् क‚थं स्थूल‚त‚मार्थ‚प्र‚तिप‚त्तिः स्यात् ?‚{\tiny $_{lb}$}‚ न चाव‚य‚वा अर्वाक्प‚र‚म‚ध्य‚व‚र्तिनो युग‚प‚द् दृश्य‚न्ते । त‚त् क‚थ‚म‚व‚य‚वी स्थूलो‚{\tiny $_{lb}$}‚ दृश्येत ?
	{\color{gray}{\textsuperscript{§~\theparCount}}}
	\pend% ending standard par
      ‚{\tiny $_{lb}$}‚

	  \refstepcounter{parCount}
	  \pstart \leavevmode% starting standard par
	क्र‚मेण दिशाम‚व‚य‚वानां प्र‚त्य‚क्षीक‚र‚णे प्र‚त्य‚क्षः स्थूलोऽव‚य‚वीति चेत् ?‚{\tiny $_{lb}$}‚ अन्य‚दा तु कः प्र‚त्य‚क्ष इति विमृष्य‚म् । अव‚य‚व इति चेत् ? अव‚य‚वा‚{\tiny $_{lb}$}‚ \edtext{\textsuperscript{*}}{\lemma{*}\Bfootnote{प्र‚तिभास इति \cite{avaya-S}}} \leavevmode\ledsidenote{\textenglish{7/tha}} अपि प‚र‚म‚ध्य‚व‚र्तिनो न युग‚प‚द् दृश्य‚न्त इति क‚थं सोऽपि प्र‚त्य‚क्षः‚{\tiny $_{lb}$}‚ स्यात् ? त‚देवं नाव‚य‚वी नाव‚य‚वाः प्र‚त्य‚क्षा इति न किन्ँचिद् दृश्येतेति ।‚{\tiny $_{lb}$}‚ त‚त् \leavevmode\ledsidenote{\textenglish{81a/msR}} सिद्ध‚मावृतानावृत‚रूपः स्थूलोऽर्थ इति ।
	{\color{gray}{\textsuperscript{§~\theparCount}}}
	\pend% ending standard par
      ‚{\tiny $_{lb}$}‚

	  \refstepcounter{parCount}
	  \pstart \leavevmode% starting standard par
	त‚था रागारागाभ्यां विरोधः संभाव‚नीयः । त‚था ह्येक‚स्मिन् \edtext{}{\lemma{स्मिन्}\Bfootnote{उक्ते \cite{avaya-S}}}र‚क्ते‚{\tiny $_{lb}$}‚ स्थूलोऽर्थो र‚क्तार‚क्ते रूपे युग‚प‚त् प्र‚तिप‚द्य‚मानो विरुद्ध‚रूप‚द्व‚य‚योग‚मात्म‚नः‚{\tiny $_{lb}$}‚ प्र‚काश‚य‚ति । न‚नु अव‚य‚व एव र‚क्तोऽव‚य‚वी त्व‚र‚क्तैक‚रूप एवेति । य‚द्येवं‚{\tiny $_{lb}$}‚ र‚क्तेऽव‚य‚वेऽर‚क्त‚रूपोऽव‚य‚वी दृश्येत । न चैव‚म् । अथ वा राग‚द्र‚व्य‚संयोगो‚{\tiny $_{lb}$}‚ हि र‚क्त‚त्व‚म् । अव‚य‚व‚स्य च राग‚द्र‚व्येण संयोगेऽव‚य‚विनोऽपि तेन भाव्य‚म्,‚{\tiny $_{lb}$}‚ अव‚श्यं हि कार‚ण‚संयोगिना कार्य‚म‚पि संयुज्य‚त इति स‚म‚यात् ।
	{\color{gray}{\textsuperscript{§~\theparCount}}}
	\pend% ending standard par
      ‚{\tiny $_{lb}$}‚

	  \refstepcounter{parCount}
	  \pstart \leavevmode% starting standard par
	य‚स्त्वाह--र‚क्त एवाव‚य‚वी । अस्ति हि कुङ्कुमार‚क्ते प‚टाव‚य‚वे‚{\tiny $_{lb}$}‚ कुङ्कुमार‚क्तः प‚ट इति प्र‚त्य‚यः । त‚तो राग‚द्र‚व्य‚संयुक्त एवाव‚य‚वीति‚{\tiny $_{lb}$}‚ त‚स्यापि व‚र्णान्त‚रान‚व‚भासः स्याद‚व‚य‚विनः । राग‚द्र‚व्यं हि प्र‚त्यासीद‚त् निज‚{\tiny $_{lb}$}‚रूपं व‚स्तुन‚स्तिरोध‚त्ते । स्वेन च रूपेण द्र‚व्यं संब‚ध्नाति । य‚था र‚क्तोऽव‚य‚वः‚{\tiny $_{lb}$}‚ तिरोहित‚स‚ह‚ज‚रूपो राग‚द्र‚व्य‚स‚म‚वायिना रूपेण संब‚न्धी प्र‚तिभाति । अव‚य‚{\tiny $_{lb}$}‚विन्य‚पि राग‚द्र‚व्य‚संयोगिनि व‚र्णान्त‚रान‚व‚भास‚प्र‚स‚ङ्गो दुर्वारः । त‚न्न र‚क्त‚{\tiny $_{lb}$}‚ एवाव‚य‚वीति श‚क्यं व‚क्तुम् ।
	{\color{gray}{\textsuperscript{§~\theparCount}}}
	\pend% ending standard par
      ‚{\tiny $_{lb}$}‚

	  \refstepcounter{parCount}
	  \pstart \leavevmode% starting standard par
	अन्य‚स्तु संयोग‚स्याव्याप्य‚वृत्त्या स‚माध‚त्ते । स ह्याह, श‚ब्द इवाव्याप्य‚{\tiny $_{lb}$}‚वृत्तिः संयोगः । तेनैकार्थो र‚क्त‚श्चार‚क्त‚श्च । सोऽप्येवं प‚र्य‚नुयोज्यः,‚{\tiny $_{lb}$}‚ क‚थ‚म‚व्याप्य‚वृत्तिः ? य‚दि हि स्वाश्र‚ये स‚म‚वेतो रूपादिव‚द् व्याप्य‚वृत्तिरेवाय‚म्,‚{\tiny $_{lb}$}‚ अस‚म‚वेत‚श्चावृत्तिरेवाप्य‚द्र‚व्येष्विव ग‚न्धः ।
	{\color{gray}{\textsuperscript{§~\theparCount}}}
	\pend% ending standard par
      ‚{\tiny $_{lb}$}‚

	  \refstepcounter{parCount}
	  \pstart \leavevmode% starting standard par
	एक‚त्र संयोग‚स्य भावाभावाव्यापिनी\edtext{}{\lemma{भावाभावाव्यापिनी}\Bfootnote{°व्यापिनो \cite{avaya-S}}} वृत्तिरिति चेत् ? स्यादेत‚त् ।‚{\tiny $_{lb}$}‚ य‚था विरुद्धाव‚पि रूप‚र‚सावेक‚माश्र‚येते, त‚था संयोग‚स्याप्येक‚त्र भावाभावौ‚{\tiny $_{lb}$}‚ युग‚प‚त् स्याताम् ।
	{\color{gray}{\textsuperscript{§~\theparCount}}}
	\pend% ending standard par
      ‚{\tiny $_{lb}$}‚

	  \refstepcounter{parCount}
	  \pstart \leavevmode% starting standard par
	अहो मोह‚विजृम्भित‚म् । अभावो हि भाव‚निवृत्तिरूपः, नास्य भाव‚{\tiny $_{lb}$}‚निवृत्तिं हित्वा रूपान्त‚र‚मीक्ष्य‚ते । य‚श्च य‚न्निवृत्तिरूपः स क‚थं त‚स्मिन् स‚त्येव‚{\tiny $_{lb}$}‚ भ‚व‚ति, भावे वा त‚न्निवृत्तिरूप‚तां ज‚ह्यात् ? त‚था हि अन‚लं प‚श्य‚न्न‚पि‚{\tiny $_{lb}$}‚ स‚लिलार्थी त‚त्र प्र‚व‚र्तेत । ज‚ल‚विविक्त‚स्यान‚ल‚स्य द‚र्श‚नात्, ज‚लाभाव‚सिद्धेः‚{\tiny $_{lb}$}‚ \leavevmode\ledsidenote{\textenglish{8/tha}} अप्र‚वृत्तिरिति चेत् ? भ‚व‚तु अनुप‚ल‚म्भात् ज‚लाभाव‚सिद्धिः, त‚थापि ज‚ल‚स‚त्तां‚{\tiny $_{lb}$}‚ संभाव‚य‚न् ज‚लार्थी प्र‚व‚र्तेत ।
	{\color{gray}{\textsuperscript{§~\theparCount}}}
	\pend% ending standard par
      ‚{\tiny $_{lb}$}‚

	  \refstepcounter{parCount}
	  \pstart \leavevmode% starting standard par
	न‚नु त‚त्र य‚दि ज‚लं स्यात्, उप‚ल‚भ्येत । किम‚तः ? अतोऽनुप‚ल‚म्भाद्‚{\tiny $_{lb}$}‚ अभावो ज‚ल‚स्येति । य‚द्य‚प्य‚भ्युप‚ग‚तैव ज‚लाभाव‚सिद्धिः, त‚थापि त‚द‚र्थिन‚स्त‚{\tiny $_{lb}$}‚द्भाव‚श‚ङ्क‚या प्र‚वृत्तिः स्यात् । युग‚प‚देक‚त्र संयोग‚स्य भावाभावौ दृश्येते । तेनैवं‚{\tiny $_{lb}$}‚ क‚ल्प‚यामो न स्वेच्छ‚या ।
	{\color{gray}{\textsuperscript{§~\theparCount}}}
	\pend% ending standard par
      ‚{\tiny $_{lb}$}‚

	  \refstepcounter{parCount}
	  \pstart \leavevmode% starting standard par
	न‚नु किम‚भावो भाव‚प्र‚तिषेधात्म‚कः प्र‚तीय‚ते, अन्य‚था वा ? त‚त्र भाव‚नि‚{\tiny $_{lb}$}‚वृत्तिरूपे\leavevmode\ledsidenote{\textenglish{81b/msR}}ऽभावे सिद्धे क‚थं भावोप‚ल‚ब्धिर्न भ्रान्ता स्यात् ? भावाप्र‚तिषेधा‚{\tiny $_{lb}$}‚त्म‚क‚श्च नाभावः । नाम‚मात्रं तु स्यात् । न च नाम‚मात्राद‚र्थ‚स्य त‚थाभावः ।‚{\tiny $_{lb}$}‚ रूप‚र‚स‚योस्तु न प‚र‚स्प‚र‚निवृत्ती रूप‚मिति क‚थं त‚दुदाह‚र‚ण‚मिह शोभेत ?‚{\tiny $_{lb}$}‚ न च रूप‚र‚स‚योरेक‚त्र स‚म‚वायोऽस्माभिर‚नुम‚न्य‚ते । श‚ब्दोऽप्येव‚मेवाव्याप्य‚{\tiny $_{lb}$}‚वृत्तिर‚सिद्धः । स क‚थं प्र‚कृत‚संश‚य‚निवृत्त‚ये क‚ल्प्येतेति ? अलं ब‚हुलापित‚या\edtext{}{\lemma{या}\Bfootnote{°भाषित‚या \cite{avaya-S}}} ।
	{\color{gray}{\textsuperscript{§~\theparCount}}}
	\pend% ending standard par
      ‚{\tiny $_{lb}$}‚

	  \refstepcounter{parCount}
	  \pstart \leavevmode% starting standard par
	अथ वा स्थूलोऽर्थ‚स्त‚द‚त‚द्देशः\edtext{}{\lemma{द्देशः}\Bfootnote{त‚द्देशः \cite{avaya-msR}}} प्र‚तीय‚ते । \edtext{\textsuperscript{*}}{\lemma{*}\Bfootnote{°त‚द्देश‚यो° \cite{avaya-S}}}त‚द‚त‚द्देश‚योश्च प‚र‚स्प‚रा‚{\tiny $_{lb}$}‚भावाव्य‚भिचार‚निमित्तोऽस्ति विरोधः । अतो विरुद्ध‚ध‚र्म‚संस‚र्गः स्थूल‚स्य ।
	{\color{gray}{\textsuperscript{§~\theparCount}}}
	\pend% ending standard par
      ‚{\tiny $_{lb}$}‚

	  \refstepcounter{parCount}
	  \pstart \leavevmode% starting standard par
	स्यादेत‚त् । क‚थं त‚द‚त‚द्देश‚योः प‚र‚स्प‚राभावाव्य‚भिचारः ? उच्य‚ते,‚{\tiny $_{lb}$}‚ इह ताव‚देक‚स्मिन् देशे प‚रिच्छिद्य‚माने ताद्रूप्य‚प्र‚च्युतिर‚स्य व्य‚व‚च्छिद्य‚ते,‚{\tiny $_{lb}$}‚ त‚द‚व्य‚व‚च्छेदे त‚त्प‚रिच्छेदाभाव‚प्र‚स‚क्तेः । प्र‚च्युतिव‚च्च प्र‚च्युतिम‚द‚पि देशान्त‚रं‚{\tiny $_{lb}$}‚ व्य‚व‚च्छिद्य‚ते । य‚दि हि प‚रिच्छिद्य‚मानो देशान्त‚र‚स्व‚भावो भ‚वेत्, त‚दा‚{\tiny $_{lb}$}‚ त‚द्रूप एवोप‚ल‚ब्धो भ‚वेत् । देशान्त‚र‚रूपं तु विर‚ह‚य्य स्वेन रूपेण प्र‚काश‚मानो‚{\tiny $_{lb}$}‚ देशो देशान्त‚रासंसृष्ट इत्य‚व‚सीय‚ते । य‚था च देश‚स्य देशान्त‚रासंस‚र्गः,‚{\tiny $_{lb}$}‚ त‚था तेनाधार‚भूतेन देशेन य‚द् व्याप्तं रूपं त‚द‚पि देशान्त‚र‚संस‚र्गाद‚व‚च्छिन्नं\edtext{}{\lemma{च्छिन्नं}\Bfootnote{°द‚विच्छिन्नं \cite{avaya-S}°द् विच्छिन्नं ?}}‚{\tiny $_{lb}$}‚ भ‚व‚ति । क‚थं हि तेन देशेन व्याप्तं रूपं त‚द‚भाव‚व‚ति देशान्त‚रे व‚र्तेत, य‚था‚{\tiny $_{lb}$}‚ एकेन देशेन व्याप्तो घ‚टो न देशान्त‚रे व‚र्त‚ते ?
	{\color{gray}{\textsuperscript{§~\theparCount}}}
	\pend% ending standard par
      ‚{\tiny $_{lb}$}‚

	  \refstepcounter{parCount}
	  \pstart \leavevmode% starting standard par
	देशान्त‚रेऽनुप‚ल‚म्भाद‚प्र‚वृत्तिरिति चेत् ? विप्र‚कृष्टे देशान्त‚रे क‚थ‚म‚स्या‚{\tiny $_{lb}$}‚भावः प्र‚तिप‚त्त‚व्यः ? त‚स्मादिद‚म‚काम‚केनापि वाच्य‚म्, य‚दुत एक‚देश‚व्याप्तं‚{\tiny $_{lb}$}‚ रूपं न देशान्त‚रे व‚र्त‚त इति, त‚स्य तेन व्याप‚न‚प्र‚स‚ङ्गात् ।
	{\color{gray}{\textsuperscript{§~\theparCount}}}
	\pend% ending standard par
      ‚{\tiny $_{lb}$}‚\textsuperscript{\textenglish{9/tha}}

	  \refstepcounter{parCount}
	  \pstart \leavevmode% starting standard par
	त‚द‚भाव‚ति देशान्त‚रे व‚र्त‚मानोऽपि तेन देशेन व्याप्येत । न च व्याप्ति‚{\tiny $_{lb}$}‚र‚स्य श‚क्याव‚सातुम्, भागान्त‚रासंभ‚वात् ।
	{\color{gray}{\textsuperscript{§~\theparCount}}}
	\pend% ending standard par
      ‚{\tiny $_{lb}$}‚

	  \refstepcounter{parCount}
	  \pstart \leavevmode% starting standard par
	एतेन त‚न्निर‚स्तं य‚दाह क‚श्चित्--य‚थैको भाव‚स्त‚द‚त‚द्देशं ज‚न‚य‚न्न‚{\tiny $_{lb}$}‚ विरुध्य‚त इति, त‚द‚पि न प्र‚कृतानुरूप‚म् । त‚था हि भावाभावावेव\edtext{}{\lemma{भावाभावावेव}\Bfootnote{एव \textit{om} \cite{avaya-msR}}} प‚र‚स्प‚र‚{\tiny $_{lb}$}‚प‚रिहारेण विरुद्धौ न व‚स्तुनी । व‚स्तुनोस्तु प‚र‚स्प‚राभावाव्य‚भिचारेण‚{\tiny $_{lb}$}‚ विरोधः । तेन य‚देव व‚स्तु व‚स्त्व‚न्त‚र‚प्र‚च्युतिम‚त्, त‚देव तेन विरुद्ध‚म् ।
	{\color{gray}{\textsuperscript{§~\theparCount}}}
	\pend% ending standard par
      ‚{\tiny $_{lb}$}‚

	  \refstepcounter{parCount}
	  \pstart \leavevmode% starting standard par
	न चैक‚कार्य‚निर्व‚र्त‚न‚श‚क्तिः कार्यान्त‚र‚श‚क्त्य‚भावाव्य‚भिचारिणी । अनुप‚{\tiny $_{lb}$}‚ल‚म्भो हि व‚स्तुनो व‚स्त्व‚न्त‚राभावाव्य‚भिचारं साध‚य‚ति । एक‚कार्य‚निर्व‚र्त‚न‚{\tiny $_{lb}$}‚श‚क्तिम‚ति च रूपे गृह्य‚माणे कार्यान्त‚र‚निर्व‚र्त्त‚न‚श‚क्तिर‚पि प‚रिच्छिद्य‚त इति‚{\tiny $_{lb}$}‚ क‚थं त‚द‚भावः ? एक‚देश‚संब‚द्धं तु रूपं देशान्त‚र‚संस‚र्गिरूप‚प‚रिहारेणोप‚ल‚भ्य‚{\tiny $_{lb}$}‚मानं \edtext{}{\lemma{मानं}\Bfootnote{भाव्य‚चारितुं स‚च्च विरुद्ध‚म् \cite{avaya-msR}}}त‚द‚भावाव्य‚भिचारि तेन विरुद्ध‚म् । य‚थात्य‚न्त‚स‚दृशोर्व‚स्तुनोर्युग‚प‚द्‚{\tiny $_{lb}$}‚ \leavevmode\ledsidenote{\textenglish{82a/msR}} उप‚ल‚भ्य‚मान‚योः स‚त्य‚पि \edtext{}{\lemma{पि}\Bfootnote{चाकाराभेदे \cite{avaya-msR}}}चाकार‚भेदे दूराद‚नुप‚ल‚क्ष्य‚माण‚भेद‚योर्देश‚{\tiny $_{lb}$}‚भेद‚मात्र‚निमित्त‚कं प्र‚त्य‚क्षाव‚सितं\edtext{}{\lemma{सितं}\Bfootnote{प्र‚त्य‚क्षाश्रितं \cite{avaya-msR}}} विरोध‚माश्रित्य भेदोऽव‚स्थाप्य‚ते ।
	{\color{gray}{\textsuperscript{§~\theparCount}}}
	\pend% ending standard par
      ‚{\tiny $_{lb}$}‚

	  \refstepcounter{parCount}
	  \pstart \leavevmode% starting standard par
	य‚स्त्वाह--य‚थैकं च‚क्षुर्विज्न्ँआनं भिन्नेषु च‚क्षुरादिषु व‚र्त‚ते त‚द‚धीनोत्पाद‚{\tiny $_{lb}$}‚त‚या, त‚थान्योऽपि \edtext{}{\lemma{थान्योऽपि}\Bfootnote{देश‚वृत्ति° ?}}भिन्न‚देश‚निवृत्तिर्न भेत्स्य‚त इति, सोऽपि देश‚भेद‚निमित्ते‚{\tiny $_{lb}$}‚ विरोधे अव‚स्थाप्य‚माने व‚स्तुनः कार‚ण‚भेद‚निमित्तं विरोध‚मास‚न्ँज‚य‚न्‚{\tiny $_{lb}$}‚ न नैपुण्य‚मात्म‚नो निवेद‚य‚ति । न हि विज्न्ँआन‚स्य देशोऽस्ति क‚श्चित्,‚{\tiny $_{lb}$}‚ अमूर्त‚त्वात् ।
	{\color{gray}{\textsuperscript{§~\theparCount}}}
	\pend% ending standard par
      ‚{\tiny $_{lb}$}‚

	  \refstepcounter{parCount}
	  \pstart \leavevmode% starting standard par
	स्यादेत‚त् । य‚था देश‚भेद‚निमित्तो विरोधः, त‚था कार‚ण‚भेद‚निमित्तोऽपि‚{\tiny $_{lb}$}‚ स्यात् । को हि व‚स्तुतो विशेषः कार‚ण‚भेदाद् देश‚भेद‚स्येति ?
	{\color{gray}{\textsuperscript{§~\theparCount}}}
	\pend% ending standard par
      ‚{\tiny $_{lb}$}‚

	  \refstepcounter{parCount}
	  \pstart \leavevmode% starting standard par
	उक्त‚मिह । प‚र‚स्प‚राभावाव्य‚भिचार‚निमित्तो व‚स्तूनाम‚स्ति विरोधः ।‚{\tiny $_{lb}$}‚ स देश‚भेदे\edtext{}{\lemma{भेदे}\Bfootnote{देश‚भेदेऽपि \cite{avaya-S}}} संनिधीय‚ते, न कार‚ण‚भेदे । देश‚भेद‚व‚ती हि रूपेऽन्योन्य‚{\tiny $_{lb}$}‚प‚रिहारेणोप‚ल‚भ्य‚माने प‚र‚स्प‚राभावाव्य‚भिचारिणी भ‚व‚तो न कार‚ण‚{\tiny $_{lb}$}‚भेद‚व‚ती ।
	{\color{gray}{\textsuperscript{§~\theparCount}}}
	\pend% ending standard par
      ‚{\tiny $_{lb}$}‚\textsuperscript{\textenglish{10/tha}}

	  \refstepcounter{parCount}
	  \pstart \leavevmode% starting standard par
	त‚देवं क‚म्प‚रागाव‚र‚ण‚भावाभाव‚कृते देश‚भेद‚निब‚न्ध‚ने च च‚तुर्थे विरुद्ध‚{\tiny $_{lb}$}‚ध‚र्म‚संस‚र्गेऽव‚य‚विविष‚ये व्य‚व‚स्थापिते प‚क्ष‚ध‚र्म‚त्वं सिद्धं हेतोः । अधुना व्याप्ति‚{\tiny $_{lb}$}‚रेवास्य स्व‚साध्येन स‚म‚र्थ‚नीया ।
	{\color{gray}{\textsuperscript{§~\theparCount}}}
	\pend% ending standard par
      ‚{\tiny $_{lb}$}‚

	  \refstepcounter{parCount}
	  \pstart \leavevmode% starting standard par
	इह विरुद्ध‚ध‚र्म‚संस‚र्ग‚विर‚ह‚मात्र‚निब‚न्ध‚नोऽभेद‚व्य‚व‚हारो व‚स्तूनां दृष्टः‚{\tiny $_{lb}$}‚ अहेतोर‚योगात्, निमित्तान्त‚र‚स्य चाद‚र्श‚नात् । त‚तो विरुद्ध‚ध‚र्म‚संस‚र्गेऽपि‚{\tiny $_{lb}$}‚ भ‚व‚न्न‚भेद‚व्य‚व‚हारो व्याप‚कं \edtext{}{\lemma{कं}\Bfootnote{निमित्त‚व‚त्त्वं \cite{avaya-S}}}निमित्त‚त्वं ज‚ह्यात् । त‚तो व्याप‚कानुप‚ल‚ब्ध्या‚{\tiny $_{lb}$}‚ त‚स्माद् व्यावृत्तो\edtext{}{\lemma{व्यावृत्तो}\Bfootnote{त‚स्माद् व्या° \textit{om} \cite{avaya-msR}}} विरुद्ध‚ध‚र्म‚विर‚हेण व्याप्य‚ते । त‚द्विरुद्ध‚श्च विरुद्ध‚ध‚र्म‚{\tiny $_{lb}$}‚संस‚र्गः । तेनेयं व्याप‚क‚विरुद्धोप‚ल‚ब्धिभूमिका ।
	{\color{gray}{\textsuperscript{§~\theparCount}}}
	\pend% ending standard par
      ‚{\tiny $_{lb}$}‚

	  \refstepcounter{parCount}
	  \pstart \leavevmode% starting standard par
	एवं प्र‚साधितेऽस्यास्त्रैरूप्येऽसिद्ध‚विरुद्धानैकान्तिका दोषा नाव‚काशं‚{\tiny $_{lb}$}‚ ल‚भ‚न्त इति ।
	{\color{gray}{\textsuperscript{§~\theparCount}}}
	\pend% ending standard par
      ‚{\tiny $_{lb}$}‚
	    
	    \stanza[\smallbreak]
	  एवं म‚या ब‚हुषु दुर्म‚तिनिर्मितेषु&‚{\tiny $_{lb}$}‚प्र‚त्युद्धृतेषु ख‚लु दूष‚ण‚क‚ण्ट‚केषु ।&‚{\tiny $_{lb}$}‚आचार्य‚नीतिप‚थ एव विशोधितोऽय-&‚{\tiny $_{lb}$}‚मुत्सार्य म‚त्स‚र‚म‚नेन ज‚नः प्र‚यातु ॥\&[\smallbreak]
	  
	  
	  ‚{\tiny $_{lb}$}‚
		
		\pstart
		\begin{center}
	      ॥ स‚माप्तं चेद‚म‚व‚य‚विनिराक‚र‚ण‚मिति ॥
		\end{center}
		\pend
		
	      
	    
	    \endnumbering% ending numbering from div
	    \endgroup
	    
	  % running endDocumentHook
     \backmatter 
	 \chapter{The TEI Header}
	 \begin{minted}[fontfamily=rmfamily,fontsize=\footnotesize,breaklines=true]{xml}
       <teiHeader xmlns="http://www.tei-c.org/ns/1.0" xml:lang="en">
   <fileDesc>
      <titleStmt>
         <title>Avayavinirākaraṇa</title>
         <author>Aśoka</author>
         <funder>Deutsche Forschungsgemeinschaft</funder>
         <funder>The National Endowment for the Humanities</funder>
         <principal>
	           <persName>Birgit Kellner</persName>
	        </principal>
         <respStmt>
            <resp>data entry by</resp>
            <name xml:id="aurorachana">Aurorachana, Auroville</name>
         </respStmt>
         <respStmt>
            <resp>prepared for SARIT by</resp>
            <persName xml:id="sarit-encoder-avaya">Liudmila Olalde</persName>
         </respStmt>
      </titleStmt>
      <editionStmt>
         <p> </p>
      </editionStmt>
      <publicationStmt>
         <publisher>SARIT: Search and Retrieval of Indic Texts. DFG/NEH Project (NEH-No.
	HG5004113), 2013-2017 </publisher>
         <availability status="restricted">
            <p>Copyright Notice:</p>
            <p>Copyright 2016 SARIT</p>
            <licence> 
	              <p>Distributed under a <ref target="https://creativecommons.org/licenses/by-sa/4.0/">Creative Commons Attribution-ShareAlike 4.0 International licence.</ref> Under this licence, you are free to:</p>
	              <list>
                  <item>Share — copy and redistribute the material in any medium or format.</item>
                  <item>Adapt — remix, transform, and build upon the material for any purpose, even commercially.</item>
               </list>
	              <p>The licensor cannot revoke these freedoms as long as you follow the license terms.</p>
	              <p>Under the following terms:</p>
	              <list>
                  <item>Attribution — You must give appropriate credit, provide a link to the license, and indicate if changes were made. You may do so in any reasonable manner, but not in any way that suggests the licensor endorses you or your use.</item>
                  <item>ShareAlike — If you remix, transform, or build upon the material, you must distribute your contributions under the same license as the original.</item>
               </list>
	              <p>More information and fuller details of this license are given on the Creative Commons website.</p>
	           </licence>
            <p>SARIT assumes no responsibility for unauthorised use that infringes the rights of any copyright owners, known or unknown.</p>
         </availability>
         <date>2016</date>
      </publicationStmt>
      <sourceDesc>
         <bibl xml:id="avaya-thakur-book">
	           <title type="main">Aśokanibandhau Avayavinirākaraṇaṃ Samānyadūṣaṇaṃ ca</title>
	           <author>Aśoka</author>
	           <editor xml:id="avaya-ed">Anantalal Thakur</editor>
	           <publisher>Kashi Prasad Jayaswal Research Institute</publisher>
	           <pubPlace>Patna</pubPlace>
	           <date>1974</date>
	           <note>The manuscript consulted by Thakur is described below.</note>
	        </bibl>
         <listWit>
            <witness xml:id="avaya-msR">
	              <msDesc>
                  <msIdentifier>
                     <idno>Ratnakīrtinibandha</idno>
                     <altIdentifier>
                        <idno>R</idno>
                        <note>This description is based on Thakur's introduction to his <ref target="#avaya-thakur-book">1974 edition</ref> (pp. vii-viii).</note>
                     </altIdentifier>
                  </msIdentifier>
                  <msContents>
                     <msItem>
                        <author>Ratnakīrti</author>
                        <title>Ratnakīrtinibandha</title>
                     </msItem>
                     <msItem>
                        <author>Aśoka</author>
                        <title>Avayavinirākaraṇa</title>
                     </msItem>
                     <msItem>
                        <author>Aśoka</author>
                        <title>Sāmānyaduṣaṇa</title>
                     </msItem>
                  </msContents>
                  <physDesc>
                     <objectDesc>
                        <p>Thakur had access only to photographs of the palm-leave manuscript.</p>
                        <p>The <title>Avayavinirākaraṇa</title> covers folios 78b1-82a5. The <title>Sāmānyaduṣana</title> covers folios 82a5-84b1. Both texts are placed between the <title>Citrādvaitaprakāśavāda</title> and the <title>Santānāntaradūṣaṇa</title>. The scribe is the same as in the <title>Ratnakīrtinibandha</title>. Each side of a folio consists of six lines of ca. 138 syllables. The inner four lines leave a pag of 12 syllables for two perforations in the palm-leaves.  The script id proto-Maithili.</p>
                     </objectDesc>
                  </physDesc>
                  <history>
                     <p>Photographs of manuscripts brought by Rahula Sankrityayana from Tibet discovered by Anantalal Thakur while editing the <title>Ratnakīrtinibandhāvalī</title>.</p>
                  </history>
               </msDesc>

	           </witness>
            <witness xml:id="avaya-S">
	              <bibl>
	                 <title>Six Buddhist Nyāya Tracts in Sanskrit</title>
	                 <editor>Mahāmahopadhyāya Haraprasād Shāstri</editor>
	                 <publisher>The Asiatic Society</publisher>
	                 <pubPlace>Calcutta</pubPlace>
	                 <date>1910</date>
	              </bibl>
	           </witness>
         </listWit>
      </sourceDesc>
   </fileDesc>
   <encodingDesc>
      <p>
         <list>
            <item>In the source file, there were two types of line breaks: returns (and possible surrounding space) and hyphens+returns. These were replaced with lb-elements. The ed-attribute "tha" refers to <ref target="#avaya-thakur-book">Thakur's 1974 edition</ref>.</item>
            <item>The folio numbers were encoded as pb-elements with the ed-attribute "msR", which refers to the <ref target="#avaya-msR">manuscript used by Thakur</ref>.</item>
            <item>Additions in square brackets were enclosed in an add-element.</item>
            <item>Underlined items were enclosed in a <tag>hi rend="underlined"</tag>.</item>
            <item>In the note-elements the <tag>lb</tag>s were removed.</item>
            <item>The footnotes were encoded as note-elements with their corresponding n-attribute. The footnote references were replaced by the note-elements.</item>
         </list>
      </p>
      <p xml:id="avaya-fn">There was no footnote reference for note 2 on page 6. Hence it was given the attribute place="bottom".</p>
   </encodingDesc>
   <profileDesc><!-- ... --></profileDesc>
   <revisionDesc>
      <change/>
   </revisionDesc>
</teiHeader>
	 \end{minted}
       
      \clearpage
      \begin{english}
      \printshorthands
      \printbibliography
      \end{english}
    
\end{document}
