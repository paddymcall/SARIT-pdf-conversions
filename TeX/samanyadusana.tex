\documentclass[article,12pt,a4paper]{memoir}%
    
      %% useful for debugging
      %% \usepackage{syntonly}%
      %%\syntaxonly%
    
	  \usepackage[normalem]{ulem}
	  \usepackage{eulervm}
	  \usepackage{xltxtra}
  \usepackage{polyglossia}
  \PolyglossiaSetup{sanskrit}{
  hyphenmins={2,3},% default is {1,3}
  }
  \setdefaultlanguage{sanskrit}
  % english etc. should also be available, notes and bib
  \setotherlanguages{english,german,italian,french}
  \setotherlanguage[numerals=arabic]{tibetan}
  \usepackage{fontspec}
  %% redefine some chars (either changed by parsing, or not commonly in font)
  \catcode`⃥=\active \def⃥{\textbackslash}
  \catcode`‿=\active \def‿{\textunderscore}
  \catcode`❴=\active \def❴{\{}
  \catcode`❵=\active \def❵{\}}
  \catcode`〔=\active \def〔{{[}}% translate 〔OPENING TORTOISE SHELL BRACKET
  \catcode`〕=\active \def〕{{]}}% translate 〕CLOSING TORTOISE SHELL BRACKET
  \catcode` =\active \def {\,}
  \catcode`·=\active \def·{\textbullet}
  %% BREAK PERMITTED HERE: \discretionary{-}{}{}\nobreak\hspace{0pt}
  \catcode`‚=\active \def‚{\-}
  \catcode`ꣵ=\active \defꣵ{%
  म्\textsuperscript{cb}%for candrabindu
  }
  %% show a lot of tolerance
  \tolerance=9000
  \def\textJapanese{\fontspec{Kochi Mincho}}
  \def\textChinese{\fontspec{HAN NOM A}}
  \def\textKorean{\fontspec{Baekmuk Gulim} }
  % make sure English font is there
  \newfontfamily\englishfont[Mapping=tex-text]{TeX Gyre Schola}
    % set up a devanagari font
  \newfontfamily\devanagarifont[Script=Devanagari,Mapping=devanagarinumerals,AutoFakeBold=1.5,AutoFakeSlant=0.3]{Chandas}
	\newfontfamily\rmlatinfont[Mapping=tex-text]{TeX Gyre Pagella}
	\newfontfamily\tibetanfont[Script=Tibetan,Scale=1.2]{Tibetan Machine Uni}
  \newcommand\bo\tibetanfont
  
    \defaultfontfeatures{Scale=MatchLowercase,Mapping=tex-text}
	\setmainfont{Chandas}
    \setsansfont{TeX Gyre Bonum}
  
  \setmonofont{DejaVu Sans Mono}
	  %% page layout start: fit to a4 and US letterpaper (example in memoir.pdf)
	  %% page layout start
	  % stocksize (actual size of paper in the printer) is a4 as per class
	  % options;
	  
	  % trimming, i.e., which part should be cut out of the stock (this also
	  % sets \paperheight and \paperwidth):
	  % \settrimmedsize{0.9\stockheight}{0.9\stockwidth}{*}
	  % \settrimmedsize{225mm}{150mm}{*}
	  % % say where you want to trim
	  \setlength{\trimtop}{\stockheight}    % \trimtop = \stockheight
	  \addtolength{\trimtop}{-\paperheight} %           - \paperheight
	  \setlength{\trimedge}{\stockwidth}    % \trimedge = \stockwidth
	  \addtolength{\trimedge}{-\paperwidth} %           - \paperwidth
	  % % this makes trims equal on top and bottom (which means you must cut
	  % % twice). if in doubt, cut on top, so that dust won't settle when book
	  % % is in shelf
	  \settrims{0.5\trimtop}{0.5\trimedge}

	  % figure out which font you're using
	  \setxlvchars
	  \setlxvchars
	  % \typeout{LENGTH: lxvchars: \the\lxvchars}
	  % \typeout{LENGTH: xlvchars: \the\xlvchars}

	  % set the size of the text block next:
	  % this sets \textheight and \textwidth (not the whole page including
	  % headers and footers)
	  \settypeblocksize{230mm}{130mm}{*}

	  % left and right margins:
	  % this way spine and edge margins are the same
	  % \setlrmargins{*}{*}{*}
	  \setlrmargins{*}{*}{1.5}

	  % upper and lower, same logic as before
	  % \setulmargins{*}{*}{*}% upper = lower margin
	  % \uppermargin = \topmargin + \headheight + \headsep
	  %\setulmargins{*}{*}{1.5}% 1.5*upper = lower margin
	  \setulmargins{*}{*}{1.5}% 

	  % header and footer spacings
	  \setheadfoot{2\baselineskip}{2\baselineskip}

	  % \setheaderspaces{ headdrop }{ headsep }{ ratio }
	  \setheaderspaces{*}{*}{1.5}

	  % see memman p. 51 for this solution to widows/orphans 
	  \setlength{\topskip}{1.6\topskip}
	  % fix up layout
	  \checkandfixthelayout
	  %% page layout end
	
	  \sloppybottom
	
	    % numbering depth
	    \maxtocdepth{section}
	    % set up layout of toc
	    \setpnumwidth{4em}
	    \setrmarg{5em}
	    \setsecnumdepth{all}
	    \newenvironment{docImprint}{\vskip 6pt}{\ifvmode\par\fi }
	    \newenvironment{docDate}{}{\ifvmode\par\fi }
	    \newenvironment{docAuthor}{\ifvmode\vskip4pt\fontsize{16pt}{18pt}\selectfont\fi\itshape}{\ifvmode\par\fi }
	    % \newenvironment{docTitle}{\vskip6pt\bfseries\fontsize{18pt}{22pt}\selectfont}{\par }
	    \newcommand{\docTitle}[1]{#1}
	    \newenvironment{titlePart}{ }{ }
	    \newenvironment{byline}{\vskip6pt\itshape\fontsize{16pt}{18pt}\selectfont}{\par }
	    % setup title page; see CTAN /info/latex-samples/TitlePages/, and memoir
	  \newcommand*{\plogo}{\fbox{$\mathcal{SARIT}$}}
	  \newcommand*{\makeCustomTitle}{\begin{english}\begingroup% from example titleTH, T&H Typography
	  \thispagestyle{empty}
	  \raggedleft
	  \vspace*{\baselineskip}
	  
	      % author(s)
	    {\Large Aśoka}\\[0.167\textheight]
	    % maintitle
	    {\Huge Sāmānyadūṣaṇa}\\[\baselineskip]
	    {\Large SARIT}\\\vspace*{\baselineskip}\plogo\par
	  \vspace*{3\baselineskip}
	  \endgroup
	  \end{english}}
	  \newcommand{\gap}[1]{}
	  \newcommand{\corr}[1]{($^{x}$#1)}
	  \newcommand{\sic}[1]{($^{!}$#1)}
	  \newcommand{\reg}[1]{#1}
	  \newcommand{\orig}[1]{#1}
	  \newcommand{\abbr}[1]{#1}
	  \newcommand{\expan}[1]{#1}
	  \newcommand{\unclear}[1]{($^{?}$#1)}
	  \newcommand{\add}[1]{($^{+}$#1)}
	  \newcommand{\deletion}[1]{($^{-}$#1)}
	  \newcommand{\quotelemma}[1]{\textcolor{cyan}{#1}}
	  \newcommand{\name}[1]{#1}
	  \newcommand{\persName}[1]{#1}
	  \newcommand{\placeName}[1]{#1}
	  % running latexPackages template
     \usepackage[x11names]{xcolor}
     \definecolor{shadecolor}{gray}{0.95}
     \usepackage{longtable}
     \usepackage{ctable}
     \usepackage{rotating}
     \usepackage{lscape}
     \usepackage{ragged2e}
     
	 \usepackage{titling}
	 \usepackage{marginnote}
	 \renewcommand*{\marginfont}{\color{black}\rmlatinfont\scriptsize}
	 \setlength\marginparwidth{.75in}
	 \usepackage{graphicx}
	 \graphicspath{{images/}}
	 \usepackage{csquotes}
       
	 \def\Gin@extensions{.pdf,.png,.jpg,.mps,.tif}
       
      \usepackage[noend,series={A,B}]{reledmac}
       % simplify what ledmac does with fonts, because it breaks. From the documentation of ledmac:
       % The notes are actually given seven parameters: the page, line, and sub-line num-
       % ber for the start of the lemma; the same three numbers for the end of the lemma;
       % and the font specifier for the lemma. 
       \makeatletter
       \def\select@lemmafont#1|#2|#3|#4|#5|#6|#7|%
       {}
       \makeatother
       \setlength{\stanzaindentbase}{20pt}
     \setstanzaindents{3,2,2,2,2,2,2,2,2,2,2,2,2,}
     % \setstanzapenalties{1,5000,10500}
     \lineation{page}
     % \linenummargin{inner}
     \linenumberstyle{arabic}
     \firstlinenum{5}
    \linenumincrement{5}
    \renewcommand*{\numlabfont}{\normalfont\scriptsize\color{black}}
    \addtolength{\skip\Afootins}{1.5mm}
    \Xnotenumfont{\bfseries\footnotesize}
    \sidenotemargin{outer}
    \linenummargin{inner}
    \Xarrangement{twocol}
    \arrangementX{twocol}
    %% biblatex stuff start
	 \usepackage[backend=biber,%
	 citestyle=authoryear,%
	 bibstyle=authoryear,%
	 language=english,%
	 sortlocale=en_US,%
	 ]{biblatex}
	 
		 \addbibresource[location=remote]{https://raw.githubusercontent.com/paddymcall/Stylesheets/HEAD/profiles/sarit/latex/bib/sarit.bib}
	 \renewcommand*{\citesetup}{%
	 \rmlatinfont
	 \biburlsetup
	 \frenchspacing}
	 \renewcommand{\bibfont}{\rmlatinfont}
	 \DeclareFieldFormat{postnote}{:#1}
	 \renewcommand{\postnotedelim}{}
	 %% biblatex stuff end
	 
	 \setcounter{errorcontextlines}{400}
       
	 \usepackage{lscape}
	 \usepackage{minted}
       
	   % pagestyles
	   \pagestyle{ruled}
	   
	   \makeoddfoot{ruled}{{\tiny\rmlatinfont \textit{Compiled: \today}}}{}{\rmlatinfont\thepage}
	   \makeevenfoot{ruled}{\rmlatinfont\thepage}{}{{\tiny\rmlatinfont \textit{Compiled: \today}}}
	   
	 
	   \usepackage{perpage}
           \MakePerPage{footnote}
	 
       \usepackage[destlabel=true,% use labels as destination names; ; see dvipdfmx.cfg, option 0x0010, if using xelatex
       pdftitle={Sāmānyadūṣaṇa // Aśoka},
       pdfauthor={SARIT: Search and Retrieval of Indic Texts. DFG/NEH Project (NEH-No.
	HG5004113), 2013-2017 },
       unicode=true]{hyperref}
       
       \renewcommand\UrlFont{\rmlatinfont}
       \newcounter{parCount}
       \setcounter{parCount}{0}
       % cleveref should come last; note: also consider zref, this could become more useful than cleveref?
       \usepackage[english]{cleveref}% clashes with eledmac < 1.10.1 standard
       \crefname{parCount}{§}{§§}
     
\begin{document}
    
     \makeCustomTitle
     \let\tabcellsep&
	\frontmatter
	\tableofcontents
	% \listoffigures
	% \listoftables
	\cleardoublepage
        \mainmatter 
	  
	% new div opening: depth here is 0
	
	    
	    \begingroup
	    \beginnumbering% beginning numbering from div depth=0
	    
	  
\chapter*[{॥ सामान्य‚दूष‚ण‚म् ॥}]{॥ सामान्य‚दूष‚ण‚म् ॥}\textsuperscript{\textenglish{11/tha}}‚{\tiny $_{lb}$}‚\textsuperscript{\textenglish{82a/msR}}‚{\tiny $_{lb}$}‚
	    
	    \stanza[\smallbreak]
	  व्याप‚कं नित्य‚मेकं च सामान्यं यैः प्र‚क‚ल्पित‚म् ।&‚{\tiny $_{lb}$}‚मोह‚ग्र‚न्थिच्छिदे तेषां त‚द‚भावः प्र‚साध्य‚ते ॥\&[\smallbreak]
	  
	  
	  ‚{\tiny $_{lb}$}‚

	  \refstepcounter{parCount}
	  \pstart \leavevmode% starting standard par
	क‚थ‚मिद‚म‚व‚ग‚म्य‚ते प‚र‚स्प‚र‚विल‚क्ष‚ण‚क्ष‚णेषु\edtext{}{\lemma{णेषु}\Bfootnote{°क्ष‚ण‚ल‚क्ष‚णेषु \cite{sādū-msR}}} प्र‚त्य‚क्ष‚स‚मीक्ष्य‚माणेष्व‚भिन्न‚{\tiny $_{lb}$}‚धीध्व‚निप्र‚स‚व‚निब‚न्ध‚न‚म‚नुयायिरूपं सामान्यं न मान्यं म‚नीषिणामिति ?‚{\tiny $_{lb}$}‚ साध‚क‚प्र‚माण‚विर‚हाद् बाध‚क‚प्र‚माण‚संभ‚वाच्चेति ब्रूमः ।
	{\color{gray}{\textsuperscript{§~\theparCount}}}
	\pend% ending standard par
      ‚{\tiny $_{lb}$}‚
	    
	    \stanza[\smallbreak]
	  त‚था हि य‚दिदं सामान्य‚साध‚न‚म‚नुमान‚म‚भिधीय‚ते प‚रैः,&‚{\tiny $_{lb}$}‚य‚द‚नुग‚ताकारं ज्न्ँआनं त‚द‚नुग‚त‚व‚स्तुनिब‚न्ध‚न‚म् ।&‚{\tiny $_{lb}$}‚य‚था ब‚हुषु पुष्पेषु स्र‚क् स्र‚गिति ज्न्ँआन‚म् ।\&[\smallbreak]
	  
	  
	  ‚{\tiny $_{lb}$}‚

	  \refstepcounter{parCount}
	  \pstart \leavevmode% starting standard par
	\leavevmode\ledsidenote{\textenglish{82b/msR}} अस्ति च प‚र‚स्प‚र‚संप‚र्क‚विक‚ल‚क‚लासु कार्यादिव्य‚क्तिष्व‚नुग‚ता‚{\tiny $_{lb}$}‚कारं विज्न्ँआनं त‚द‚नैकान्तिक‚तादोषाक्रान्त‚श‚रीर‚त्वान्न त‚द्भाव‚साध‚नायाल‚म् ।‚{\tiny $_{lb}$}‚ य‚तो भ‚व‚ति ब‚हुषु पाच‚केषु पाच‚कः पाच‚क इति एकाकार‚प‚राम‚र्श‚प्र‚त्य‚यः ।‚{\tiny $_{lb}$}‚ न च तेष्व‚नुग‚त‚मेकं व‚स्तु स‚म‚स्ति । त‚द्भावे हि प्रागेव त‚थाविध‚प्र‚त्य‚योत्पाद‚{\tiny $_{lb}$}‚प्र‚स‚ङ्गो दुर्वार‚प्र‚चारः ।
	{\color{gray}{\textsuperscript{§~\theparCount}}}
	\pend% ending standard par
      ‚{\tiny $_{lb}$}‚

	  \refstepcounter{parCount}
	  \pstart \leavevmode% starting standard par
	क्रियोप‚कारापेक्षाणां स्व‚ल‚क्ष‚णानां सामान्य‚व्य‚न्ँज‚क‚त्वाद‚य‚म‚दोष इति‚{\tiny $_{lb}$}‚ चेत् ? नैत‚द‚स्ति । नित्यानाम‚नाधेयातिश‚य‚त‚यानुप‚कारिणि स‚ह‚कारिण्य‚पेक्षा‚{\tiny $_{lb}$}‚योगात् । सातिश‚य‚त्वे वा प्र‚तिक्ष‚णं विश‚रारुश‚रीर‚त्वात् क्रिया कुत इति‚{\tiny $_{lb}$}‚ दोषो दुष्प‚रिह‚रः । क्रियानिब‚न्ध‚न‚त्वात् पाच‚केष्व‚नुग‚ताकार‚प्र‚त्य‚य‚स्य नानै‚{\tiny $_{lb}$}‚कान्तिक‚तादोष इत्य‚पि वार्त‚म् । प्र‚तिभेदं भिद्य‚मानानां क‚र्म‚णां त‚न्निब‚न्ध‚न‚त्वा‚{\tiny $_{lb}$}‚योगात् । भिन्नानाम‚प्य‚भिन्नाकार‚ज्न्ँआन‚निब‚न्ध‚न‚त्वे व्य‚क्तीनाम‚पि त‚थाभावो‚{\tiny $_{lb}$}‚ न राज‚द‚ण्ड‚निवारितः । त‚त‚श्च सामान्य‚मेव नोपेयं स्यादिति मूल‚ह‚रं‚{\tiny $_{lb}$}‚ प‚क्ष‚माश्र‚य‚ता देवानांप्रियेण सुष्ठु अनुकूल‚माच‚रित‚म् । अनेनैव न्यायेन क्रिया‚{\tiny $_{lb}$}‚कार‚क‚संब‚न्ध‚म‚भिन्न‚ज्न्ँआन‚निब‚न्ध‚न‚मुप‚क‚ल्प‚य‚न् प्र‚तिक्षिप्तः ।
	{\color{gray}{\textsuperscript{§~\theparCount}}}
	\pend% ending standard par
      ‚{\tiny $_{lb}$}‚‚{\tiny $_{lb}$}‚\textsuperscript{\textenglish{12/tha}}

	  \refstepcounter{parCount}
	  \pstart \leavevmode% starting standard par
	पाक‚क्रियात्व‚निब‚न्ध‚नः पाच‚केष्व‚नुग‚ताकारः प्र‚त्य‚यः । त‚ता नानै‚{\tiny $_{lb}$}‚कान्तिक‚तादोष इत्य‚पि न म‚न्त‚व्य‚म् । न ह्य‚र्थान्त‚र‚संब‚न्धिनी जातिर‚र्था‚{\tiny $_{lb}$}‚न्त‚र‚प्र‚त्य‚योत्प‚त्तिहेतुः, अतिप्र‚स‚ङ्गात् ।
	{\color{gray}{\textsuperscript{§~\theparCount}}}
	\pend% ending standard par
      ‚{\tiny $_{lb}$}‚

	  \refstepcounter{parCount}
	  \pstart \leavevmode% starting standard par
	स्यादेत‚त् । स‚म‚वेत‚स‚म‚वाय‚संब‚न्ध‚ब‚लात् पाक‚क्रियासामान्यं पाच‚केष्व‚{\tiny $_{lb}$}‚भिन्नाकारं प‚राम‚र्श‚प्र‚त्य‚य‚मुप‚ज‚न‚य‚ति । त‚तो न य‚थोक्तो दोषः ।
	{\color{gray}{\textsuperscript{§~\theparCount}}}
	\pend% ending standard par
      ‚{\tiny $_{lb}$}‚

	  \refstepcounter{parCount}
	  \pstart \leavevmode% starting standard par
	त‚दिद‚म‚प्य‚सार‚म् । य‚त उद‚यान‚न्त‚राप‚व‚र्गित‚या\edtext{}{\lemma{या}\Bfootnote{य‚त् °व‚र्जित‚या \cite{sādū-S}}} क‚र्म‚णामेवासंभ‚वात्,‚{\tiny $_{lb}$}‚ विन‚ष्टे क‚र्म‚णि त‚त्सामान्यं न क‚र्म‚णि त‚द‚भावादेव, नापि क‚र्त‚रीति संब‚द्ध‚{\tiny $_{lb}$}‚संब‚न्धोऽप्य‚स्य नास्तीति नाभिन्न‚प्र‚त्य‚य‚हेतुः । त‚स्मात् स्थित‚मेत‚त्‚{\tiny $_{lb}$}‚ त‚द‚नैकान्तिक‚तादोष‚दुष्ट‚त्वात् नेद‚म‚नुमानं सामान्य‚स‚त्तासाध‚नाय प‚र्याप्त‚{\tiny $_{lb}$}‚मिति ।
	{\color{gray}{\textsuperscript{§~\theparCount}}}
	\pend% ending standard par
      ‚{\tiny $_{lb}$}‚

	  \refstepcounter{parCount}
	  \pstart \leavevmode% starting standard par
	इत‚श्चापि न सामान्य‚स‚त्तासाध‚न‚मिद‚म‚नुमान‚म् । य‚थैव हि प‚र‚स्प‚रा‚{\tiny $_{lb}$}‚स‚ङ्कीर्ण‚स्व‚भावा अपि शाब‚लेयाद‚यो भावाः क‚याचिदेव त‚देक‚कार्य‚प्र‚ति‚{\tiny $_{lb}$}‚निय‚म‚ल‚क्ष‚ण‚या स्व‚हेतुब‚लायात‚या प्र‚कृत्या त‚देक‚म‚भिम‚त‚म‚नुग‚त‚रूप‚मुप‚{\tiny $_{lb}$}‚कुर्व‚ते, त‚द‚प‚र‚सामान्यान्त‚र‚म‚न्त‚रेणान्य‚थान‚व‚स्थाप्र‚स‚ङ्गात्, त‚था \edtext{}{\lemma{था}\Bfootnote{त‚मेकं \cite{sādū-S}}}त‚वैकं‚{\tiny $_{lb}$}‚ प‚राम‚र्श‚प्र‚त्य‚य‚मुप‚ज‚न‚य‚न्तु किम‚न्त‚राल‚ग‚डुना व्य‚तिरेक‚व‚ता सामान्येनो‚{\tiny $_{lb}$}‚प‚ग‚तेन ?
	{\color{gray}{\textsuperscript{§~\theparCount}}}
	\pend% ending standard par
      ‚{\tiny $_{lb}$}‚

	  \refstepcounter{parCount}
	  \pstart \leavevmode% starting standard par
	अथोच्य‚ते । प्र‚तिनिय‚त‚श‚क्त‚यः स‚र्व\leavevmode\ledsidenote{\textenglish{83a/msR}}भावाः । एत‚च्च सामान्याप‚{\tiny $_{lb}$}‚लापिभिर‚पि निय‚त‚म‚भ्युप‚ग‚म‚नीय‚म् । अन्य‚था कुतः शालिबीजं शाल्य‚ङ्कुर‚{\tiny $_{lb}$}‚मेव ज‚न‚य‚ति न कोद्र‚वाङ्कुर‚मिति प‚र‚प‚र्य‚नुयोगे भाव‚प्र‚कृतिं मुक्त्वा किम‚प‚र‚{\tiny $_{lb}$}‚मिह व‚च‚नीय‚म‚स्ति ?
	{\color{gray}{\textsuperscript{§~\theparCount}}}
	\pend% ending standard par
      ‚{\tiny $_{lb}$}‚

	  \refstepcounter{parCount}
	  \pstart \leavevmode% starting standard par
	एत‚च्चोत्त‚र‚म‚स्माक‚म‚पि न व‚नौकःकुल‚क‚व‚लित‚म् । त‚था हि व‚य‚{\tiny $_{lb}$}‚म‚प्येवं श‚क्ता एव व‚क्तुम् । सामान्य‚मेवोप‚क‚र्तुं श‚क्तिर्व्य‚क्तीनां भेदाविशेषेऽपि‚{\tiny $_{lb}$}‚ न त‚देकं विज्न्ँआन‚मुप‚ज‚न‚यितुमिति ।
	{\color{gray}{\textsuperscript{§~\theparCount}}}
	\pend% ending standard par
      ‚{\tiny $_{lb}$}‚

	  \refstepcounter{parCount}
	  \pstart \leavevmode% starting standard par
	अनुत्त‚रं ब‚त दोष‚स‚ङ्क‚ट‚म‚त्र‚भ‚वान् दृष्टिदोषेण \edtext{}{\lemma{दृष्टिदोषेण}\Bfootnote{प्र‚विश्य \cite{sādū-S}}}प्र‚वेश्य‚मानोऽपि नात्मान‚{\tiny $_{lb}$}‚मात्म‚ना संवेद‚य‚ते । त‚था हि शालिबीज‚त‚द‚ङ्कुर‚योर‚ध्य‚क्षानुप‚ल‚म्भ‚निब‚न्ध‚ने‚{\tiny $_{lb}$}‚ कार्य‚कार‚ण‚भावेऽव‚ग‚ते शालिबीजं शाल्य‚ङ्कुरं ज‚न‚यितुं श‚क्तं न कोद्र‚वाङ्कुर‚{\tiny $_{lb}$}‚मिति श‚क्य‚म‚भिधातुम् । नैवं सामान्य‚त‚द्व‚तोरुप‚कार्योप‚कार‚क‚भावः कुत‚श्च‚न‚{\tiny $_{lb}$}‚ \leavevmode\ledsidenote{\textenglish{13/tha}} प्र‚माणात् निश्चितः । त‚त् क‚थ‚मिद‚मुत्त‚र‚म‚भिधीय‚मान‚माद‚धीत साधिमान‚{\tiny $_{lb}$}‚मित्य‚ल‚म‚लीक‚निर्ब‚न्ध‚नेन ? न साध‚क‚प्र‚माण‚विर‚ह‚मात्रेण प्रेक्षाव‚ताम्‚{\tiny $_{lb}$}‚ अस‚द्व्य‚व‚हारः । त‚त‚स्त‚द‚भाव‚साध‚क‚म‚नुमान‚म‚भिधीय‚मान‚म‚स्माभिराक‚ल्प्य‚{\tiny $_{lb}$}‚ताम्\edtext{}{\lemma{ताम्}\Bfootnote{°राक‚ल्य° \cite{sādū-S}}} ।
	{\color{gray}{\textsuperscript{§~\theparCount}}}
	\pend% ending standard par
      ‚{\tiny $_{lb}$}‚

	  \refstepcounter{parCount}
	  \pstart \leavevmode% starting standard par
	य‚द् य‚दुप‚ल‚ब्धिल‚क्ष‚ण‚प्राप्तं स‚न्नोप‚ल‚भ्य‚ते, त‚त्त‚द‚स‚दिति प्रेक्षाव‚द्भिः‚{\tiny $_{lb}$}‚ व्य‚व‚ह‚र्त‚व्य‚म् ।
	{\color{gray}{\textsuperscript{§~\theparCount}}}
	\pend% ending standard par
      ‚{\tiny $_{lb}$}‚

	  \refstepcounter{parCount}
	  \pstart \leavevmode% starting standard par
	य‚थाम्ब‚राम्बुरुह‚म् ।
	{\color{gray}{\textsuperscript{§~\theparCount}}}
	\pend% ending standard par
      ‚{\tiny $_{lb}$}‚

	  \refstepcounter{parCount}
	  \pstart \leavevmode% starting standard par
	नोप‚ल‚भ्य‚ते चोप‚ल‚ब्धिल‚क्ष‚ण‚प्राप्तं सामान्यं क्व‚चिद‚पीति स्व‚भावानु‚{\tiny $_{lb}$}‚प‚ल‚ब्धिः ।
	{\color{gray}{\textsuperscript{§~\theparCount}}}
	\pend% ending standard par
      ‚{\tiny $_{lb}$}‚

	  \refstepcounter{parCount}
	  \pstart \leavevmode% starting standard par
	न चात्रासिद्धिदोषोद्भाव‚न‚या प्र‚त्य‚व‚स्थात‚व्य‚म् । त‚था हि अत्रासिद्धिः‚{\tiny $_{lb}$}‚ भ‚व‚न्ती स्व‚रूप‚तो विशेष‚ण‚तो वा भ‚वेत् ? त‚त्र न ताव‚दाद्या\edtext{}{\lemma{दाद्या}\Bfootnote{°दाद्यं \cite{sādū-S}}} संभ‚व‚ति,‚{\tiny $_{lb}$}‚ अन्योप‚ल‚म्भ‚रूप‚स्यानुप‚ल‚म्भ‚स्याभ्युप‚ग‚मात् । त‚स्य च स्व‚संवेद‚न‚प्र‚त्य‚क्ष‚{\tiny $_{lb}$}‚साक्षात्कृत‚स्व‚रूप‚त्वात् कुतः स्व‚रूपासिद्धिदोषाव‚काशः ?
	{\color{gray}{\textsuperscript{§~\theparCount}}}
	\pend% ending standard par
      ‚{\tiny $_{lb}$}‚

	  \refstepcounter{parCount}
	  \pstart \leavevmode% starting standard par
	अथोच्य‚ते स्व‚संवेद‚न‚मेव न संभ‚व‚ति, स्वात्म‚नि क्रियाविरोधात् । न हि‚{\tiny $_{lb}$}‚ त‚यैवासिधार‚या सैवासिधारा च्छिद्य‚ते; त‚देवाङ्गुल्य‚ग्रं तेनैवाङ्गुल्य‚ग्रेण स्पृश्य‚त‚{\tiny $_{lb}$}‚ इति । अतोऽसिद्ध एवायं हेतुः । \edtext{\textsuperscript{*}}{\lemma{*}\Bfootnote{त‚देदं \cite{sādū-msR}}}त‚दिदं स्व‚संवेद‚न‚श‚ब्दार्थाप‚रिज्न्ँआन‚{\tiny $_{lb}$}‚विजृम्भित‚मेव प्र‚क‚ट‚य‚ति वाचः ।
	{\color{gray}{\textsuperscript{§~\theparCount}}}
	\pend% ending standard par
      ‚{\tiny $_{lb}$}‚

	  \refstepcounter{parCount}
	  \pstart \leavevmode% starting standard par
	त‚था हि क‚ल‚स‚क‚ल‚धौत‚कुव‚ल‚यादिभ्यो व्यावृत्तं विज्न्ँआन‚मुप‚जाय‚ते । तेन‚{\tiny $_{lb}$}‚ बोध‚रूप‚त‚योत्प‚त्तिरेवास्य स्व‚संवित्तिरुच्य‚ते, प्र‚काश‚व‚त् । न क‚र्म‚क‚र्तृक्रिया‚{\tiny $_{lb}$}‚भावात् । एक‚स्यानंश‚रूप‚स्य त्रैरूप्यानुप‚प‚त्तितः । य‚थैव हि प्र‚काश‚कान्त‚र‚{\tiny $_{lb}$}‚निर‚पेक्षः प्र‚काशः प्र‚काश‚मान आत्म‚नः प्र‚काश‚क उच्य‚ते, त‚था ज्न्ँआन‚म‚पि‚{\tiny $_{lb}$}‚ ज्न्ँआनान्त‚र‚निर‚पेक्षं प्र‚काश‚मान‚मात्म‚नः प्र‚काश‚क‚मुच्य‚ते ।
	{\color{gray}{\textsuperscript{§~\theparCount}}}
	\pend% ending standard par
      ‚{\tiny $_{lb}$}‚

	  \refstepcounter{parCount}
	  \pstart \leavevmode% starting standard par
	त‚तोऽयं प‚र\leavevmode\ledsidenote{\textenglish{83b/msR}}मार्थः । न ज्न्ँआनं ज्न्ँआनान्त‚र‚संवेद्य‚मुप‚प‚द्य‚ते, नाप्य‚{\tiny $_{lb}$}‚संविदित‚मुच्य‚ते । य‚था प्र‚कारे च स्व‚संवेद‚न‚श‚ब्दार्थे विव‚क्षिते न किन्ँचिद्‚{\tiny $_{lb}$}‚ \edtext{\textsuperscript{*}}{\lemma{*}\Bfootnote{व‚च‚नीय‚क‚म‚स्ति \cite{sādū-S}}}व‚च‚नीय‚म‚स्ति कुतो य‚थोक्त‚दोषाव‚स‚रः ?
	{\color{gray}{\textsuperscript{§~\theparCount}}}
	\pend% ending standard par
      ‚{\tiny $_{lb}$}‚\textsuperscript{\textenglish{14/tha}}

	  \refstepcounter{parCount}
	  \pstart \leavevmode% starting standard par
	नापि विशेष‚णासिद्ध्यासिद्धिरुद्भाव‚नीया । उप‚ल‚ब्धिल‚क्ष‚ण‚प्राप्त‚त‚या‚{\tiny $_{lb}$}‚ सामान्य‚स्य स्व‚य‚मेव प‚रैरुप‚ग‚मात् । त‚थान‚भ्युप‚ग‚मे वा न सामान्य‚ब‚लेन‚{\tiny $_{lb}$}‚ \edtext{\textsuperscript{*}}{\lemma{*}\Bfootnote{वा कुलेया° \cite{sādū-S}}}बाहुलेयादिष्व‚नुग‚ताकारौ धीध्व‚नी स्याताम् । न हि य‚तो य‚त्र ज्न्ँआनाभिधान‚{\tiny $_{lb}$}‚प्र‚वृत्तिः, \edtext{\textsuperscript{*}}{\lemma{*}\Bfootnote{°ल‚म्भ‚ने ?}}त‚द‚नुप‚ल‚क्ष‚णे त‚स्य प्र‚तीतिर्भ‚व‚ति, द‚ण्डिव‚त् ।
	{\color{gray}{\textsuperscript{§~\theparCount}}}
	\pend% ending standard par
      ‚{\tiny $_{lb}$}‚

	  \refstepcounter{parCount}
	  \pstart \leavevmode% starting standard par
	य‚त् पुन‚रिद‚मुद्द्योतित‚म् उद‚द्योत‚क‚रेण, किं सामान्यं प्र‚तिप‚द्य‚से न वा ?‚{\tiny $_{lb}$}‚ य‚दि प्र‚तिप‚द्य‚से, क‚थ‚म‚प‚ह्नुषे ? अथ न प्र‚तिप‚द्य‚से, त‚दा त‚स्यासिद्ध‚त्वादा‚{\tiny $_{lb}$}‚श्र‚यासिद्धो हेतुः ।
	{\color{gray}{\textsuperscript{§~\theparCount}}}
	\pend% ending standard par
      ‚{\tiny $_{lb}$}‚

	  \refstepcounter{parCount}
	  \pstart \leavevmode% starting standard par
	त‚दिदं त‚स्य \edtext{}{\lemma{स्य}\Bfootnote{°पिता° \cite{sādū-S}}}ध‚र्मिस्व‚रूप‚तान‚भिज्न्ँअताविजृम्भित‚माभाति, य‚तो न व‚यं‚{\tiny $_{lb}$}‚ ब‚हीरूप‚त‚या सामान्यं ध‚र्मित‚याङ्गीकुर्म‚हे, अन्त‚र्मात्रानिवेशिनो भावाभावो‚{\tiny $_{lb}$}‚भ‚यानुभ‚याहित‚वास‚नाप‚रिपाक‚प्र‚भ‚व‚स्याध्य‚स्त‚ब‚हिर्व‚स्तुनो ज्न्ँआनाकार‚स्य ध‚र्मि‚{\tiny $_{lb}$}‚त‚योप‚योगात्\edtext{}{\lemma{योगात्}\Bfootnote{°नुभ‚वाहित° \cite{sādū-S} °प‚ग‚मात् \cite{sādū-msR}}} । स च स्व‚संवेद‚न‚प्र‚त्य‚क्ष‚सिद्ध‚त‚या न श‚क्यः प्र‚तिक्षेप्तुम् ।
	{\color{gray}{\textsuperscript{§~\theparCount}}}
	\pend% ending standard par
      ‚{\tiny $_{lb}$}‚

	  \refstepcounter{parCount}
	  \pstart \leavevmode% starting standard par
	त‚द‚त्र ध‚र्मिणि व्य‚व‚स्थिताः स‚द‚स‚त्त्वे चिन्त‚य‚न्ति । किम‚यं सामान्य‚श‚ब्द‚{\tiny $_{lb}$}‚विक‚ल्प‚प्र‚तिभासार्थो ध‚र्मी प‚र‚प‚रिक‚ल्पित‚ब‚हिःसामान्य‚निब‚न्ध‚नो वेति ? त‚स्य‚{\tiny $_{lb}$}‚ बाह्यानुपादान‚त्वे \edtext{}{\lemma{त्वे}\Bfootnote{साध्ये त‚थानु° ?}}साध्य‚त‚यानुप‚ल‚म्भो हेतुः । न पुन‚स्त‚स्यैवाभावः प्र‚साध्य‚ते,‚{\tiny $_{lb}$}‚ त‚द्विष‚य‚श‚ब्दाप्र‚योग‚प्र‚स‚ङ्गात् । एवंविधे च ध‚र्मिणि विव‚क्षिते कुत आश्र‚या‚{\tiny $_{lb}$}‚सिद्धिदोषः ?
	{\color{gray}{\textsuperscript{§~\theparCount}}}
	\pend% ending standard par
      ‚{\tiny $_{lb}$}‚

	  \refstepcounter{parCount}
	  \pstart \leavevmode% starting standard par
	य‚त् तूच्य‚ते, प्र‚त्य‚क्ष‚प्र‚माण‚सिद्ध‚स्व‚भाव‚त‚या सामान्य‚स्यासिद्ध एवायं‚{\tiny $_{lb}$}‚ हेतुरिति त‚द‚युक्त‚म्, त‚स्य स्व‚रूपेणाप्र‚तिभास‚नात् ।
	{\color{gray}{\textsuperscript{§~\theparCount}}}
	\pend% ending standard par
      ‚{\tiny $_{lb}$}‚

	  \refstepcounter{parCount}
	  \pstart \leavevmode% starting standard par
	इद‚मेव हि प्र‚त्य‚क्ष‚स्य प्र‚त्य‚क्ष‚त्व‚म्, य‚त् स्व‚रूप‚स्य स्व‚बुद्धौ स‚म‚र्प‚ण‚म् ।‚{\tiny $_{lb}$}‚ इदं पुन‚र्मूल्यादान‚क्र‚यि सामान्यं स्व‚रूपं च नाद‚र्श‚य‚ति, प्र‚त्य‚क्ष‚तां च स्वी‚{\tiny $_{lb}$}‚क‚र्तुमिच्छ‚ति । त‚था हि न व‚यं प‚र‚स्प‚रास‚ङ्कीर्ण‚शाब‚लेयादिव्य‚क्तिभेद‚प्र‚ति‚{\tiny $_{lb}$}‚भास‚न‚वेलायां त‚द्विल‚क्ष‚ण‚म‚प‚र‚म‚नुग‚त‚म‚ध्य‚क्षेणेक्षाम‚हे, \edtext{\textsuperscript{*}}{\lemma{*}\Bfootnote{क‚ण्ठेश‚न° \cite{sādū-S}}}क‚ण्ठेगुण‚मिव भूतेषु,‚{\tiny $_{lb}$}‚ शाब‚लेय‚सामान्य‚बुद्धेर‚सिद्धेः । त‚त् क‚थ‚म‚दृष्ट‚क‚ल्प‚न‚यात्मान‚मात्म‚ना विप्र‚ल‚{\tiny $_{lb}$}‚भेम‚हि ? इति नासिद्धो हेतुः ।
	{\color{gray}{\textsuperscript{§~\theparCount}}}
	\pend% ending standard par
      ‚{\tiny $_{lb}$}‚

	  \refstepcounter{parCount}
	  \pstart \leavevmode% starting standard par
	नाप्य‚नैकान्तिक‚ता श‚ङ्काविष‚य‚म‚तिप‚त‚ति, विप‚क्ष‚वृत्त्य‚द‚र्श‚नात् । अस‚प‚क्षे‚{\tiny $_{lb}$}‚ संभ‚वानुप‚ल‚म्भात् साधार‚णानैकान्तिक‚ता मा भूत्, स‚न्दिग्ध‚विप‚क्ष‚व्यावृत्ति‚{\tiny $_{lb}$}‚क‚ता तु प्र‚तिब‚न्धाद‚र्श‚नाद‚निवारित‚प्र‚स‚रैव ।
	{\color{gray}{\textsuperscript{§~\theparCount}}}
	\pend% ending standard par
      ‚{\tiny $_{lb}$}‚\textsuperscript{\textenglish{15/tha}}

	  \refstepcounter{parCount}
	  \pstart \leavevmode% starting standard par
	\edtext{\textsuperscript{*}}{\lemma{*}\Bfootnote{त‚देत‚न्न स‚मा° \cite{sādū-S}}}त‚देत‚द‚नालोचित‚त‚र्क‚क‚र्क‚श‚धियाम‚भिधान‚म्, विप‚र्य‚ये बाध‚क‚प्र‚माण‚{\tiny $_{lb}$}‚साम‚र्थ्याद‚प\leavevmode\ledsidenote{\textenglish{84a/msR}}सारित‚स‚द्भाव‚त्वात् त‚दाश‚ङ्कायाः । त‚था हि अस‚त्त्वे साध्ये‚{\tiny $_{lb}$}‚ स‚त्त्वं विप‚क्षः । त‚त्र प्र‚त्य‚क्ष‚वृत्त्या भ‚वित‚व्य‚म्, य‚तो य‚द् य‚दाविक‚लाप्र‚तिह‚त‚{\tiny $_{lb}$}‚साम‚र्थ्य‚म्, त‚त् त‚दा भ‚व‚त्येव । त‚द् य‚था अविक‚ल‚ब‚ल‚स‚क‚ल‚कार‚ण‚क‚लापो‚{\tiny $_{lb}$}‚ऽङ्कुरः । स‚ति च च‚क्षुरादिसाक‚ल्ये दृश्ये व‚स्तुन्य‚विक‚लाप्र‚तिब‚द्ध‚श‚क्तिकार‚णं‚{\tiny $_{lb}$}‚ प्र‚त्य‚क्षं ज्न्ँआन‚मिति स्व‚भाव‚हेतुः । त‚तो विरुद्धोप‚ल‚म्भाद् विप‚क्षाद् व्याव‚र्त‚{\tiny $_{lb}$}‚मानो हेतुर‚स‚द्व्य‚व‚हार‚योग्य‚त्वेन व्याप्य‚त इति व्याप्तिसिद्धेर्नानैकान्तिकः ।
	{\color{gray}{\textsuperscript{§~\theparCount}}}
	\pend% ending standard par
      ‚{\tiny $_{lb}$}‚

	  \refstepcounter{parCount}
	  \pstart \leavevmode% starting standard par
	अभिम‚त‚साध्य‚प्र‚तिब‚न्ध‚सिद्धेस्तु विरुद्ध‚ता दूर‚त‚र‚स‚मुत्सारित‚र‚भ‚स‚{\tiny $_{lb}$}‚प्र‚स‚रैव ।
	{\color{gray}{\textsuperscript{§~\theparCount}}}
	\pend% ending standard par
      ‚{\tiny $_{lb}$}‚

	  \refstepcounter{parCount}
	  \pstart \leavevmode% starting standard par
	त‚तोऽसिद्ध‚तादिदूष‚ण‚श‚ङ्काक‚ल‚ङ्कान‚ङ्किताद्धेतोः\edtext{}{\lemma{ङ्किताद्धेतोः}\Bfootnote{अतः...\add{क ?} ल‚ङ्कि \add{कृ}ता° \cite{sādū-S}}} प्र‚स्तुत‚व‚स्तुसिद्धौ सिद्ध‚म्‚{\tiny $_{lb}$}‚ अस‚त्त्वं सामान्य‚स्येत्य‚ल‚म‚तिब‚हुविस्त‚र‚विसारिण्या क‚थ‚येति विर‚म्य‚ते ।
	{\color{gray}{\textsuperscript{§~\theparCount}}}
	\pend% ending standard par
      ‚{\tiny $_{lb}$}‚

	  \refstepcounter{parCount}
	  \pstart \leavevmode% starting standard par
	न च व‚स्तुसंस्थान‚व‚त् सामान्यं\edtext{}{\lemma{सामान्यं}\Bfootnote{सामान्यं न ? \cite{sādū-msR}}} व्य‚क्तेर्ल‚क्ष‚ण‚म् । न चानुवृत्त‚व्यावृत्त‚{\tiny $_{lb}$}‚व‚र्णाद्यात्म‚के जातिव्य‚क्ती व‚र्णादिनिय‚त‚प्र‚तिभास‚प्र‚तीतिप्र‚स‚ङ्गात्, व्य‚क्तेरे‚{\tiny $_{lb}$}‚वासौ प्र‚तिभास इति चेत् ? कोऽप‚र‚स्त‚र्हि सामान्य‚स्यानुग‚ताकार इति चेत् ?‚{\tiny $_{lb}$}‚ न‚नु व‚र्ण‚संस्थाने विर‚ह‚य्य किम‚प‚र‚म‚नुगामि विद्य‚ते ? जातिव्य‚क्त्योः स‚म‚वाय‚{\tiny $_{lb}$}‚ब‚लाद‚विभावित‚विभाग‚योः क्षीरोद‚क‚योरिव प‚र‚स्प‚र‚मिश्र‚णेन प्र‚तिप‚त्तिरिति‚{\tiny $_{lb}$}‚ चेत् ? न त‚र्हि सामान्य‚विशेष‚योरेक‚त‚र‚स्यापि रूपं गृहीत‚म् । स्व‚रूपाग्र‚ह‚णे‚{\tiny $_{lb}$}‚ अन‚योर‚प्य‚ग्र‚ह‚ण‚मिति निराल‚म्ब‚नैव सा तादृशी प्र‚तिप‚त्तिरिति प‚र‚मार्थ आवे‚{\tiny $_{lb}$}‚दित‚स्ताव‚त् । निराल‚म्ब‚न‚या च प्र‚तीत्या व्य‚व‚स्थाप्य‚मानं सामान्यं सुव्य‚व‚{\tiny $_{lb}$}‚स्थापित‚म् ।
	{\color{gray}{\textsuperscript{§~\theparCount}}}
	\pend% ending standard par
      ‚{\tiny $_{lb}$}‚
	    
	    \stanza[\smallbreak]
	  त‚स्माद् विशेष्यासिद्ध्यापि नाय‚म‚सिद्धो हेतुः ।&‚{\tiny $_{lb}$}‚स‚प‚क्षे व‚र्त‚मानो विरुद्ध इत्य‚पि न म‚न्त‚व्यः ।&‚{\tiny $_{lb}$}‚अनैकान्तिक‚ताप्य‚स्य न संभाव‚नाम‚र्ह‚ति ।\&[\smallbreak]
	  
	  
	  ‚{\tiny $_{lb}$}‚

	  \refstepcounter{parCount}
	  \pstart \leavevmode% starting standard par
	अस‚द्व्य‚व‚हारान‚पेक्ष‚त्वेन हि दृश्यानुप‚ल‚म्भो व्याप्तः । य‚दि हि स‚न्न‚पि‚{\tiny $_{lb}$}‚ त‚त्र न प्र‚व‚र्त‚येत्, इह सापेक्षः स्यात् । त‚तो विप‚क्षाद् व्याप‚क‚विरुद्धाव‚रुद्धाद्‚{\tiny $_{lb}$}‚ \leavevmode\ledsidenote{\textenglish{16/tha}} व्याव‚र्त‚मानोऽस‚द्व्य‚व‚हारे विश्राम्य‚तीति । अत‚स्तेनास‚द्व्य‚व‚हारेणानुप‚ल‚म्भो‚{\tiny $_{lb}$}‚ \edtext{\textsuperscript{*}}{\lemma{*}\Bfootnote{व्याप्य‚त इति \cite{sādū-S}}}व्याप्त इति कुतोऽनेकान्तः ? त‚त‚श्च स एवार्थः स‚मायातः ।‚{\tiny $_{lb}$}‚ 
	    \pend% close preceding par
	  
	    
	    \stanza[\smallbreak]
	  एतासु प‚न्ँच‚स्व‚व‚भास‚नीषु&‚{\tiny $_{lb}$}‚प्र‚त्य‚क्ष‚बोधे स्फुट‚म‚ङ्गुलीषु ।&‚{\tiny $_{lb}$}‚साधार‚णं ष‚ष्ठ‚मिहेक्ष‚ते यः&‚{\tiny $_{lb}$}‚शृङ्गं शिर‚स्यात्म‚न ईक्ष‚ते\edtext{}{\lemma{ते}\Bfootnote{ईष्य‚ते \cite{sādū-msR}}} सः ॥\&[\smallbreak]
	  
	  
	  
	    \pstart  \leavevmode% new par for following
	    \hphantom{.}
	  ‚{\tiny $_{lb}$}‚ इति ।
	{\color{gray}{\textsuperscript{§~\theparCount}}}
	\pend% ending standard par
      ‚{\tiny $_{lb}$}‚
	    
	    \stanza[\smallbreak]
	  स‚र्व‚स्य च पूर्वोक्त‚स्यायं प‚र‚मार्थः ।&‚{\tiny $_{lb}$}‚प्र‚त्य‚क्ष‚प्र‚तिभासिव‚र्ष्म‚सु न प‚न्ँच‚स्व‚ङ्गुलीषु स्थितं\edtext{}{\lemma{स्थितं}\Bfootnote{°व‚र्त्म न प‚न्ँच…लीषु स्थित‚म् \cite{sādū-S}, °भासि ध‚र्म‚सु \cite{sādū-S}. \textenglish{See →} Intro. p. V, °ष्व‚व‚स्थित‚म् \cite{sādū-msR}.}}&‚{\tiny $_{lb}$}‚सामान्यं प्र‚तिभास‚ते न च विक‚ल्पाकार‚बुद्धौ त‚था ।&‚{\tiny $_{lb}$}‚ता एवास्फुट‚मूर्त‚योऽत्र हि विभास‚न्ते \leavevmode\ledsidenote{\textenglish{84b/msR}} न जातिस्त‚तः&‚{\tiny $_{lb}$}‚सादृश्य‚भ्र‚म‚कार‚णौ पुन‚रिमावेकोप‚ल‚ब्धिध्व‚नी ॥&‚{\tiny $_{lb}$}‚इति सामान्य‚सिद्धिदूष‚णा दिक् प्र‚साधिता\edtext{}{\lemma{साधिता}\Bfootnote{प्र‚सारिता \cite{sādū-S}.}} ॥\&[\smallbreak]
	  
	  
	  ‚{\tiny $_{lb}$}‚
		
		\pstart
		\begin{center}
	      ॥ कृतिरियं प‚ण्डिताशोक‚स्य ॥
		\end{center}
		\pend
		
	      
	    
	    \endnumbering% ending numbering from div
	    \endgroup
	    
	  % running endDocumentHook
     \backmatter 
	 \chapter{The TEI Header}
	 \begin{minted}[fontfamily=rmfamily,fontsize=\footnotesize,breaklines=true]{xml}
       <teiHeader xmlns="http://www.tei-c.org/ns/1.0" xml:lang="en">
   <fileDesc>
      <titleStmt>
         <title>Sāmānyadūṣaṇa</title>
         <author>Aśoka</author>
         <funder>Deutsche Forschungsgemeinschaft</funder>
         <funder>The National Endowment for the Humanities</funder>
         <principal>
	           <persName>Birgit Kellner</persName>
	        </principal>
         <respStmt>
            <resp>data entry by</resp>
            <name xml:id="aurorachana">Aurorachana, Auroville</name>
         </respStmt>
         <respStmt>
            <resp>prepared for SARIT by</resp>
            <persName xml:id="sarit-encoder-sādū">Liudmila Olalde</persName>
         </respStmt>
      </titleStmt>
      <editionStmt>
         <p> </p>
      </editionStmt>
      <publicationStmt>
         <publisher>SARIT: Search and Retrieval of Indic Texts. DFG/NEH Project (NEH-No.
	HG5004113), 2013-2017 </publisher>
         <availability status="restricted">
            <p>Copyright Notice:</p>
            <p>Copyright 2016 SARIT</p>
            <licence> 
	              <p>Distributed under a <ref target="https://creativecommons.org/licenses/by-sa/4.0/">Creative Commons Attribution-ShareAlike 4.0 International licence.</ref> Under this licence, you are free to:</p>
	              <list>
                  <item>Share — copy and redistribute the material in any medium or format.</item>
                  <item>Adapt — remix, transform, and build upon the material for any purpose, even commercially.</item>
               </list>
	              <p>The licensor cannot revoke these freedoms as long as you follow the license terms.</p>
	              <p>Under the following terms:</p>
	              <list>
                  <item>Attribution — You must give appropriate credit, provide a link to the license, and indicate if changes were made. You may do so in any reasonable manner, but not in any way that suggests the licensor endorses you or your use.</item>
                  <item>ShareAlike — If you remix, transform, or build upon the material, you must distribute your contributions under the same license as the original.</item>
               </list>
	              <p>More information and fuller details of this license are given on the Creative Commons website.</p>
	           </licence>
            <p>SARIT assumes no responsibility for unauthorised use that infringes the rights of any copyright owners, known or unknown.</p>
         </availability>
         <date>2016</date>
      </publicationStmt>
      <sourceDesc>
         <bibl xml:id="sādū-thakur-book">
	           <title type="main">Aśokanibandhau Avayavinirākaraṇaṃ Samānyadūṣaṇaṃ ca</title>
	           <author>Aśoka</author>
	           <editor xml:id="sādū-ed">Anantalal Thakur</editor>
	           <publisher>Kashi Prasad Jayaswal Research Institute</publisher>
	           <pubPlace>Patna</pubPlace>
	           <date>1974</date>
	           <note>The manuscript consulted by Thakur is described below.</note>
	        </bibl>
         <listWit>
            <witness xml:id="sādū-msR">
	              <msDesc>
                  <msIdentifier>
                     <idno>Ratnakīrtinibandha</idno>
                     <altIdentifier>
                        <idno>R</idno>
                        <note>This description is based on Thakur's introduction to his <ref target="#sādū-thakur-book">1974 edition</ref> (pp. vii-viii).</note>
                     </altIdentifier>
                  </msIdentifier>
                  <msContents>
                     <msItem>
                        <author>Ratnakīrti</author>
                        <title>Ratnakīrtinibandha</title>
                     </msItem>
                     <msItem>
                        <author>Aśoka</author>
                        <title>Avayavinirākaraṇa</title>
                     </msItem>
                     <msItem>
                        <author>Aśoka</author>
                        <title>Sāmānyaduṣaṇa</title>
                     </msItem>
                  </msContents>
                  <physDesc>
                     <objectDesc>
                        <p>Thakur had access only to photographs of the palm-leave manuscript.</p>
                        <p>The <title>Avayavinirākaraṇa</title> covers folios 78b1-82a5. The <title>Sāmānyaduṣana</title> covers folios 82a5-84b1. Both texts are placed between the <title>Citrādvaitaprakāśavāda</title> and the <title>Santānāntaradūṣaṇa</title>. The scribe is the same as in the <title>Ratnakīrtinibandha</title>. Each side of a folio consists of six lines of ca. 138 syllables. The inner four lines leave a pag of 12 syllables for two perforations in the palm-leaves.  The script id proto-Maithili.</p>
                     </objectDesc>
                  </physDesc>
                  <history>
                     <p>Photographs of manuscripts brought by Rahula Sankrityayana from Tibet discovered by Anantalal Thakur while editing the <title>Ratnakīrtinibandhāvalī</title>.</p>
                  </history>
               </msDesc>

	           </witness>
            <witness xml:id="sādū-S">
	              <bibl>
	                 <title>Six Buddhist Nyāya Tracts in Sanskrit</title>
	                 <editor>Mahāmahopadhyāya Haraprasād Shāstri</editor>
	                 <publisher>The Asiatic Society</publisher>
	                 <pubPlace>Calcutta</pubPlace>
	                 <date>1910</date>
	              </bibl>
	           </witness>
         </listWit>
      </sourceDesc>
   </fileDesc>
   <encodingDesc>
      <p>
         <list>
            <item>In the source file, there were two types of line breaks: returns (and possible surrounding space) and hyphens+returns. These were replaced with lb-elements. The ed-attribute "tha" refers to <ref target="#sādū-thakur-book">Thakur's 1974 edition</ref>.</item>
            <item>The folio numbers were encoded as pb-elements with the ed-attribute "msR", which refers to the <ref target="#sādū-msR">manuscript used by Thakur</ref>.</item>
            <item>Round brackets (1 occurrence) were encoded as <tag>hi rend="brackets"</tag>.</item>
            <item>Additions in square brackets were enclosed in an add-element.</item>
            <item>Underlined items were enclosed in a <tag>hi rend="underlined"</tag>.</item>
            <item>In the note-elements the <tag>lb</tag>s were removed.</item>
            <item>The footnotes were encoded as note-elements with their corresponding n-attribute. The footnote references were replaced by the note-elements.</item>
         </list>
      </p>
   </encodingDesc>
   <profileDesc><!-- ... --></profileDesc>
   <revisionDesc>
      <change who="#sarit-encoder-sādū" when="2016-05-26">Corrected the second reference to footnote 13-3 to 13-4.</change>
   </revisionDesc>
</teiHeader>
	 \end{minted}
       
      \clearpage
      \begin{english}
      \printshorthands
      \printbibliography
      \end{english}
    
\end{document}
