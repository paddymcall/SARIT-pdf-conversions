\documentclass[article,12pt,a4paper]{memoir}%
    \usepackage{syntonly}%
    %%\syntaxonly%
    
	  \usepackage[normalem]{ulem}
	  \usepackage{eulervm}
	  \usepackage{xltxtra}
  \usepackage{polyglossia}
  \PolyglossiaSetup{sanskrit}{
  hyphenmins={2,3},% default is {1,3}
  }
  \setdefaultlanguage{sanskrit}
  % english should be available, notes and stuff
  \setotherlanguage{english}
  \setotherlanguage{german}
  \setotherlanguage[numerals=arabic]{tibetan}
  \usepackage{fontspec}
  %% redefine some chars (either changed by parsing, or not commonly in font)
  \catcode`⃥=\active \def⃥{\textbackslash}
  \catcode`‿=\active \def‿{\textunderscore}
  \catcode`❴=\active \def❴{\{}
  \catcode`❵=\active \def❵{\}}
  \catcode`〔=\active \def〔{{[}}% translate 〔OPENING TORTOISE SHELL BRACKET
  \catcode`〕=\active \def〕{{]}}% translate 〕CLOSING TORTOISE SHELL BRACKET
  \catcode` =\active \def {\,}
  \catcode`·=\active \def·{\textbullet}
  %% BREAK PERMITTED HERE: \discretionary{-}{}{}\nobreak\hspace{0pt}
  \catcode`‚=\active \def‚{\-}
  \catcode`ꣵ=\active \defꣵ{%
  म्\textsuperscript{cb}%for candrabindu
  }
  %% show a lot of tolerance
  \tolerance=9000
  \def\textJapanese{\fontspec{Kochi Mincho}}
  \def\textChinese{\fontspec{HAN NOM A}}
  \def\textKorean{\fontspec{Baekmuk Gulim} }
  % make sure English font is there
  \newfontfamily\englishfont[Mapping=tex-text]{TeX Gyre Schola}
    % set up a devanagari font
  \newfontfamily\devanagarifont[Script=Devanagari,Mapping=devanagarinumerals,AutoFakeBold=1.5,AutoFakeSlant=0.3]{Chandas}
	\newfontfamily\rmlatinfont[Mapping=tex-text]{TeX Gyre Pagella}
	\newfontfamily\tibetanfont[Script=Tibetan,Scale=1.2]{Tibetan Machine Uni}
  \newcommand\bo\tibetanfont
  
    \defaultfontfeatures{Scale=MatchLowercase,Mapping=tex-text}
	\setmainfont{Chandas}
    \setsansfont{TeX Gyre Bonum}
  
  \setmonofont{DejaVu Sans Mono}
	  %% page layout start: fit to a4 and US letterpaper (example in memoir.pdf)
	  %% page layout start
	  % stocksize (actual size of paper in the printer) is a4 as per class
	  % options;
	  
	  % trimming, i.e., which part should be cut out of the stock (this also
	  % sets \paperheight and \paperwidth):
	  % \settrimmedsize{0.9\stockheight}{0.9\stockwidth}{*}
	  % \settrimmedsize{225mm}{150mm}{*}
	  % % say where you want to trim
	  \setlength{\trimtop}{\stockheight}    % \trimtop = \stockheight
	  \addtolength{\trimtop}{-\paperheight} %           - \paperheight
	  \setlength{\trimedge}{\stockwidth}    % \trimedge = \stockwidth
	  \addtolength{\trimedge}{-\paperwidth} %           - \paperwidth
	  % % this makes trims equal on top and bottom (which means you must cut
	  % % twice). if in doubt, cut on top, so that dust won't settle when book
	  % % is in shelf
	  \settrims{0.5\trimtop}{0.5\trimedge}

	  % figure out which font you're using
	  \setxlvchars
	  \setlxvchars
	  % \typeout{LENGTH: lxvchars: \the\lxvchars}
	  % \typeout{LENGTH: xlvchars: \the\xlvchars}

	  % set the size of the text block next:
	  % this sets \textheight and \textwidth (not the whole page including
	  % headers and footers)
	  \settypeblocksize{230mm}{130mm}{*}

	  % left and right margins:
	  % this way spine and edge margins are the same
	  % \setlrmargins{*}{*}{*}
	  \setlrmargins{*}{*}{1.5}

	  % upper and lower, same logic as before
	  % \setulmargins{*}{*}{*}% upper = lower margin
	  % \uppermargin = \topmargin + \headheight + \headsep
	  %\setulmargins{*}{*}{1.5}% 1.5*upper = lower margin
	  \setulmargins{*}{*}{1.5}% 

	  % header and footer spacings
	  \setheadfoot{2\baselineskip}{2\baselineskip}

	  % \setheaderspaces{ headdrop }{ headsep }{ ratio }
	  \setheaderspaces{*}{*}{1.5}

	  % see memman p. 51 for this solution to widows/orphans 
	  \setlength{\topskip}{1.6\topskip}
	  % fix up layout
	  \checkandfixthelayout
	  %% page layout end
	
	  \sloppybottom
	
	    % numbering depth
	    \maxtocdepth{section}
	    % set up layout of toc
	    \setpnumwidth{4em}
	    \setrmarg{5em}
	    \setsecnumdepth{all}
	    \newenvironment{docImprint}{\vskip 6pt}{\ifvmode\par\fi }
	    \newenvironment{docDate}{}{\ifvmode\par\fi }
	    \newenvironment{docAuthor}{\ifvmode\vskip4pt\fontsize{16pt}{18pt}\selectfont\fi\itshape}{\ifvmode\par\fi }
	    % \newenvironment{docTitle}{\vskip6pt\bfseries\fontsize{18pt}{22pt}\selectfont}{\par }
	    \newcommand{\docTitle}[1]{#1}
	    \newenvironment{titlePart}{ }{ }
	    \newenvironment{byline}{\vskip6pt\itshape\fontsize{16pt}{18pt}\selectfont}{\par }
	    % setup title page; see CTAN /info/latex-samples/TitlePages/, and memoir
	  \newcommand*{\plogo}{\fbox{$\mathcal{SARIT}$}}
	  \newcommand*{\makeCustomTitle}{\begin{english}\begingroup% from example titleTH, T&H Typography
	  \thispagestyle{empty}
	  \raggedleft
	  \vspace*{\baselineskip}
	  
	      % author(s)
	    {\Large Dharmakīrti and Manorathanandin}\\[0.167\textheight]
	    % maintitle
	    {\Huge Pramāṇavārttika}\\[\baselineskip]
	    % titlesubtitle
	    {\small  — Pramāṇavārttikavṛtti}\\[\baselineskip]
	    {\Large SARIT}\\\vspace*{\baselineskip}\plogo\par
	  \vspace*{3\baselineskip}
	  \endgroup
	  \end{english}}
	  \newcommand{\gap}[1]{}
	  \newcommand{\corr}[1]{($^{x}$#1)}
	  \newcommand{\sic}[1]{($^{!}$#1)}
	  \newcommand{\reg}[1]{#1}
	  \newcommand{\orig}[1]{#1}
	  \newcommand{\abbr}[1]{#1}
	  \newcommand{\expan}[1]{#1}
	  \newcommand{\unclear}[1]{($^{?}$#1)}
	  \newcommand{\add}[1]{($^{+}$#1)}
	  \newcommand{\deletion}[1]{($^{-}$#1)}
	  \newcommand{\quotelemma}[1]{\textcolor{cyan}{#1}}
	  \newcommand{\name}[1]{#1}
	  \newcommand{\persName}[1]{#1}
	  \newcommand{\placeName}[1]{#1}
	  % running latexPackages template
     \usepackage[x11names]{xcolor}
     \definecolor{shadecolor}{gray}{0.95}
     \usepackage{longtable}
     \usepackage{ctable}
     \usepackage{rotating}
     \usepackage{lscape}
     \usepackage{ragged2e}
     
	 \usepackage{titling}
	 \usepackage{marginnote}
	 \renewcommand*{\marginfont}{\color{black}\rmlatinfont\scriptsize}
	 \setlength\marginparwidth{.75in}
	 \usepackage{graphicx}
	 \graphicspath{{images/}}
	 \usepackage{csquotes}
       
	 \def\Gin@extensions{.pdf,.png,.jpg,.mps,.tif}
       
      \usepackage[noend,series={A,B}]{reledmac}
       % simplify what ledmac does with fonts, because it breaks. From the documentation of ledmac:
       % The notes are actually given seven parameters: the page, line, and sub-line num-
       % ber for the start of the lemma; the same three numbers for the end of the lemma;
       % and the font specifier for the lemma. 
       \makeatletter
       \def\select@lemmafont#1|#2|#3|#4|#5|#6|#7|%
       {}
       \makeatother
       \setlength{\stanzaindentbase}{20pt}
     \setstanzaindents{3,2,2,2,2,2,2,2,2,2,2,}
     % \setstanzapenalties{1,5000,10500}
     \lineation{page}
     % \linenummargin{inner}
     \linenumberstyle{arabic}
     \firstlinenum{5}
    \linenumincrement{5}
    \renewcommand*{\numlabfont}{\normalfont\scriptsize\color{black}}
    \addtolength{\skip\Afootins}{1.5mm}
    \Xnotenumfont{\bfseries\footnotesize}
    \sidenotemargin{outer}
    \linenummargin{inner}
    \Xarrangement{twocol}
    \arrangementX{twocol}
    %% biblatex stuff start
	 \usepackage[backend=biber,%
	 citestyle=authoryear,%
	 bibstyle=authoryear,%
	 language=english,%
	 sortlocale=en_US,%
	 ]{biblatex}
	 
		 \addbibresource[location=remote]{https://raw.githubusercontent.com/paddymcall/Stylesheets/HEAD/profiles/sarit/latex/bib/sarit.bib}
	 \renewcommand*{\citesetup}{%
	 \rmlatinfont
	 \biburlsetup
	 \frenchspacing}
	 \renewcommand{\bibfont}{\rmlatinfont}
	 \DeclareFieldFormat{postnote}{:#1}
	 \renewcommand{\postnotedelim}{}
	 %% biblatex stuff end
	 
	 \setcounter{errorcontextlines}{400}
       
	 \usepackage{lscape}
	 \usepackage{minted}
       
	   % pagestyles
	   \pagestyle{ruled}
	   
	   \makeoddfoot{ruled}{{\tiny\rmlatinfont \textit{Compiled: \today}}}{}{\rmlatinfont\thepage}
	   \makeevenfoot{ruled}{\rmlatinfont\thepage}{}{{\tiny\rmlatinfont \textit{Compiled: \today}}}
	   
	 
	   \usepackage{perpage}
           \MakePerPage{footnote}
	 
       \usepackage[destlabel=true,% use labels as destination names; ; see dvipdfmx.cfg, option 0x0010, if using xelatex
       pdftitle={Dharmakīrti's Pramāṇavārttika with a commentary by Manorathanandin},
       pdfauthor={SARIT: Search and Retrieval of Indic Texts. DFG/NEH Project (NEH-No. HG5004113), 2013-2016 }]{hyperref}
       \hyperbaseurl{}
       \renewcommand\UrlFont{\rmlatinfont}
       \usepackage[english]{cleveref}% clashes with eledmac < 1.10.1!
       % \newcommand{\cref}{\href}
     
\begin{document}
    
     \makeCustomTitle
     \let\tabcellsep&
	\frontmatter
	\tableofcontents
	% \listoffigures
	% \listoftables
	\cleardoublepage
         \begin{english}
      \chapter[Title Page]{Title Page}
    \begin{docTitle}  \begin{titlePart} \persName{Dharmakīrti}'s Pramāṇavārttika with a commentary by \persName{Manorathanandin}\end{titlePart} \end{docTitle} \begin{docAuthor} Dharmakīrti\end{docAuthor} \begin{docAuthor} Manorathanandin\end{docAuthor}
      \cleardoublepage
    \begin{sanskrit}\par
न‚मो म‚ञ्जुश्रिये ॥\end{sanskrit}\end{english}\mainmatter 
	  
	% new div opening: depth here is 0
	
	    
	    \begingroup
	    \beginnumbering% beginning numbering from div depth=0
	    
	  
\chapter*[{प्र‚थ‚मः प‚रिच्छेदः ॥: प्र‚माण‚सिद्धिः}]{प्र‚थ‚मः प‚रिच्छेदः ॥: प्र‚माण‚सिद्धिः}\textsuperscript{\textenglish{001/s}}\textsuperscript{\textenglish{1b/MA}}
	  
	% new div opening: depth here is 1
	
	  
	% new div opening: depth here is 2
	
	    
	    \stanza[\smallbreak]
	विमुक्ताव‚र‚ण‚क्लेशं दीप्ताखिल‚गुण‚श्रियं ।&स्वैक‚वेद्यात्म‚स‚म्प‚त्तिं न‚म‚स्यामि म‚हामुनिम् ॥\&[\smallbreak]


	
	    
	    \stanza[\smallbreak]
	स्व‚य‚म‚पि कृतिनां म‚ह‚द्भिर‚न्यैर‚पि ग‚मितो ब‚हुविस्त‚रैर्न्न योय‚म् ।&त‚द‚पि च सुग‚मो न म‚द्विधानामिति विवृतिच्छ‚ल‚तः क‚रोमि चिन्ताम् ॥\&[\smallbreak]


	
	    
	    \stanza[\smallbreak]
	अह‚म‚पि न निजैक‚लाभ‚लुब्धो न च प‚र‚कृत्य‚र‚साभिलाष‚मुक्तः ।&फ‚ल‚ति पुन‚रियं प‚रार्थ‚वाञ्छाव्र‚त‚तिर‚भीष्ट‚फ‚लानि पुण्य‚भाजाम् ॥\&[\smallbreak]


	\label{div_pvv.1.1}
	  
	% new div opening: depth here is 2
	

	  \begin{center}%% label @type='head'
	\textbf{क. न‚म‚स्कार‚श्लोकः}
	\end{center}
	

	  \pstart \leavevmode% starting standard par
	शास्त्रा\edtext{}{\edlabel{pvv.1-1}\label{pvv.1-1}\lemma{शास्त्रा}\Bfootnote{स्तुत्या पुण्योप‚च‚यात् ।}}दाव‚विध्नेन त‚त्स‚माप्त्य‚र्थं‚{\tiny $_{1}$}‚ भ‚ग‚व‚ति प्र‚साद‚ज‚न‚ने श्रोतृ\edtext{}{\edlabel{pvv.1-2}\label{pvv.1-2}\lemma{श्रोतृ}\Bfootnote{व्याख्यातुश्चेति प‚रार्थ‚द‚र्श‚नात् ।}}ज‚नानुग्र‚हार्थ‚ञ्च ‚{\tiny $_{lb}$}‚स्तुतिपूर्व‚क‚माचार्यो न‚म‚स्कार‚श्लोक‚माह ।
	\pend% ending standard par
      
	  \bigskip
	  \begingroup
	
	    \large
	  
	    \begin{quote}
	  
	    
	    \stanza[\smallbreak]
	विधूत‚क‚ल्प‚नाजाल‚ग‚म्भीरोदार‚मूर्त‚ये ।&न‚मः स‚म‚न्त‚भ‚द्राय स‚म‚न्त‚स्फ‚र‚ण‚त्विषे ॥ १ ॥\&[\smallbreak]


	
	    \end{quote}
	  
	  \endgroup
	

	  \pstart \leavevmode% starting standard par
	विधूतं\edtext{}{\edlabel{pvv.1-3}\label{pvv.1-3}\lemma{विधूतं}\Bfootnote{स‚र्व्वाव‚र‚ण‚विग‚मात् ।}} विध्व‚स्तं अनुत्प‚त्तिक‚ध‚र्म‚तामापादितं ‚{\color{DodgerBlue3}‚क‚ल्प‚ना} ग्राह्य‚ग्राह‚काध्यारोपः सैव ‚{\tiny $_{lb}$}‚‚{\color{DodgerBlue3}‚जालं} ब‚न्ध‚न‚हेतुत्वात् यासां ता विधूत‚क‚ल्प‚नाजालाः । एतेन ध‚र्म‚काय उक्तः\edtext{}{\edlabel{pvv.1-4}\label{pvv.1-4}\lemma{उक्तः}\Bfootnote{पूजेयं न‚मः श‚ब्दात्प्र‚णाम‚तः शिष्टैश्च तेश्च । त‚च्च स्व‚प‚रार्थ‚त‚दुभ‚य‚स‚म्प‚त्तिस्त‚तः अवृत्तिहानिगाम्भीर्य्यौदार्य‚विशेषैस्त्रिभिः स्वार्थ उक्तः ।}} । द्व‚य‚{\tiny $_{lb}$}‚शून्य‚ताया ध‚र्म‚धातुत्वात् (।) त‚द‚धिग‚म‚स्य ध‚र्म‚काय‚त्वात् । ग‚म्भीराश्च ख‚ड्ग‚{\tiny $_{lb}$}‚\leavevmode\ledsidenote{\textenglish{002/s}} श्राव‚काद्य‚विष‚य‚त्वात् । उदाराश्च स‚क‚ल‚ज्ञेय‚स‚त्वार्थ‚{\tiny $_{2}$}‚व्याप‚नादिति ‚{\color{DodgerBlue3}‚ग‚म्भीरो‚{\tiny $_{lb}$}‚दाराः} । आभ्यां साम्भोगिक‚नैर्म्माणिक‚कायावुक्तौ त‚योरेव स्व‚रूप‚त्वात् । ‚{\tiny $_{lb}$}‚विधूत‚क‚ल्प‚नाजाला ग‚म्भीरोदारा ‚{\color{DodgerBlue3}‚मूर्त्त‚यो} य‚स्य स ‚{\color{DodgerBlue3}‚विधूत‚क‚ल्प‚नाजाल‚ग‚म्भोरोदार‚{\tiny $_{lb}$}‚म‚र्त्तिः} (।) एतेन स्वार्थ‚स‚म्प‚दुक्ता त्रिकाय‚ल‚क्ष‚ण‚त्वात्त‚स्याः ।
	\pend% ending standard par
      

	  \pstart \leavevmode% starting standard par
	\hphantom{.}‚{\color{DodgerBlue3}‚स‚म‚न्तं} निर‚व‚शेषं भ‚द्रं क‚ल्याणं प‚रार्थ‚स‚म्प‚त्स‚म्भार‚ल‚क्ष‚णं य‚स्माद‚सौ ‚{\color{DodgerBlue3}‚स‚म‚न्त‚भ‚द्रः} (।) अन‚या भ‚ग‚व‚न्नाम‚व्युत्प‚त्त्या प‚रार्थ‚स‚म्प‚द‚भिहिता\edtext{}{\edlabel{pvv.2-1}\label{pvv.2-1}\lemma{भिहिता}\Bfootnote{त‚द‚र्थिनां य‚था ।}} । स‚म‚न्त‚तः स्फ‚{\tiny $_{3}$}‚र‚न्तीति ‚{\tiny $_{lb}$}‚स‚म‚न्त‚स्फ‚र‚ण्यः त्विषः ताव (त्) त्विषो देश‚ना य‚स्य स ‚{\color{DodgerBlue3}‚स‚म‚न्त‚स्फ‚र‚ण‚त्विट्} व‚स्तुत‚त्त्वाव‚भास‚नोपाय‚ता च त्विड्देश‚न‚योः साध‚र्म्यं (।) अनेन प‚रार्थ‚स‚म्प‚दुपायो ‚{\tiny $_{lb}$}‚द‚र्शितः । देश‚नाद्वारेण भ‚ग‚व‚ता ज‚ग‚द‚र्थ‚क‚र‚णात् । एतेन स्तुतिरुक्ता असाधार‚णानां ‚{\tiny $_{lb}$}‚स्व‚प‚रार्थ‚स‚म्प‚त्तित‚दुपायानामुप‚द‚र्श‚नात् । स‚र्व्व‚त्र न‚मःश\edtext{}{\edlabel{pvv.2-2}\label{pvv.2-2}\lemma{मःश}\Bfootnote{अकारान्तः स‚कारान्त‚स्तृतीयार्थ इत्य‚न्ये । ज‚न‚कायः ।}}ब्द‚योगाच्च‚तुर्थी । अनेन ‚{\tiny $_{lb}$}‚न‚म‚स्कारोभिहितः । य‚दा तु स‚म‚न्त‚भ‚द्र‚श‚ब्दो‚{\tiny $_{4}$}‚ रूढ्या बोधिस‚त्त्व‚विशेषे व‚र्त‚ते ‚{\tiny $_{lb}$}‚त‚दापि प‚द‚व्याख्यानं पूर्व‚व‚देव । अय‚न्तु विशेषः । विधूत‚क‚ल्प‚नाजाल‚त्वं बोधि‚{\tiny $_{lb}$}‚स‚त्त्व‚भूम्याव‚र‚ण‚प्र‚हाण‚तो वेदित‚व्यं । गाम्भीर्य‚न्तु ख‚ड्ग‚श्राव‚काविष‚य‚त्वात् । औदार्य‚न्तु ‚{\tiny $_{lb}$}‚द‚र्श भूमीश्व‚र‚बोधिस‚त्त्व‚माहात्म्यातिश‚य‚तः । काय‚त्र‚य‚न्तु बोधिस‚त्त्वानाम‚प्य‚स्ति ‚{\tiny $_{lb}$}‚प्र‚क‚र्ष‚निष्ठाग‚म‚नात्तु भ‚ग‚व‚तां व्य‚व‚स्थाप्य‚ते । देश‚ना च प्र‚सिद्धैव तेषां ॥ (१) ॥
	\pend% ending standard par
      \label{div_pvv.1.2}
	  
	% new div opening: depth here is 2
	

	  \begin{center}%% label @type='head'
	\textbf{ख. शास्त्रार‚म्भ‚प्र‚योज‚न‚म्}
	\end{center}
	

	  \pstart \leavevmode% starting standard par
	श्रोतृदोष‚बाहुल्याच्छा‚{\tiny $_{5}$}‚स्त्रेण प‚रोप‚कार‚म‚प‚श्य‚न् सूक्ताभ्यास‚भावित‚चित्त‚ता‚{\tiny $_{lb}$}‚मेवात्म‚नः शास्त्रार‚म्भ‚कार‚ण‚न्द‚र्श‚य‚न् व‚क्रोक्त्या दोष‚ताप‚न‚य‚नेन शास्त्रे श्रोतृन् ‚{\tiny $_{lb}$}‚प्र‚व‚र्त‚यितुम(ा) ह ।
	\pend% ending standard par
      
	  \bigskip
	  \begingroup
	
	    \large
	  
	    \begin{quote}
	  
	    
	    \stanza[\smallbreak]
	\edtext{\textsuperscript{*}}{\edlabel{pvv.2-asterisk}\label{pvv.2-asterisk}\lemma{*}\Bfootnote{द्र‚ष्ट‚व्यं प‚रिशिष्टं ।१-४}} प्रायः प्राकृत‚स‚क्तिर‚प्र‚तिब‚ल‚प्र‚ज्ञो ज‚नः केव‚लं,‚{\tiny $_{lb}$}‚ नान‚र्थ्येव सुभाषितैः प‚रिग‚ता विद्वेष्ट्य‚पीर्ष्याम‚लैः ।&तेनायं न प‚रोप‚कार इति न‚श्चिन्तापि चेत‚स्त‚तः,‚{\tiny $_{lb}$}‚ सूक्ताभ्यास‚विब‚र्द्धित‚व्य‚स‚न‚मित्य‚त्रानुब‚द्ध‚स्पृह‚म् ॥ २ ॥\&[\smallbreak]


	
	    \end{quote}
	  
	  \endgroup
	

	  \pstart \leavevmode% starting standard par
	\hphantom{.}‚{\color{DodgerBlue3}‚प्रायो} भूयान् बाहुल्येन वा ‚{\color{DodgerBlue3}‚ज‚नः प्राकृतेषु} ब‚हिःशास्त्रेषु ‚{\color{DodgerBlue3}‚स‚क्ति}‚र‚भिष्व‚ङ्गो य‚स्य ‚{\tiny $_{lb}$}‚स प्राकृत‚स‚क्तिर‚नेन कुप्र‚ज्ञ‚त्वं श्रोतृदोष उक्तः । ‚{\color{DodgerBlue3}‚अप्र‚तिब‚ला} शास्त्रार्थ‚ग्र‚ह‚णं प्र‚त्य‚{\tiny $_{lb}$}‚‚{\color{DodgerBlue3}‚श‚क्ता प्र‚ज्ञा} य‚स्यासाव‚प्र‚तिब‚ल‚प्र‚ज्ञः अनेनाज्ञ‚त्व‚मुक्तं ।‚{\tiny $_{6}$}‚ ‚{\color{DodgerBlue3}‚केव‚लं, नान‚र्थ्येव सूभाषितैः} । ‚{\tiny $_{lb}$}‚किन्तु सुभाषिताभिधायिनं ईर्ष्या प‚र‚संप‚त्तौ चेत‚सो व्यारोषः सैव म‚ल‚श्चित्त\edtext{}{\edlabel{pvv.2-3}\label{pvv.2-3}\lemma{श्चित्त}\Bfootnote{आत्मात्मीयास्त्रैधातुकाश्च चैत्ताः स‚वास‚नाः ।}}म‚लिनी‚{\tiny $_{lb}$}‚\leavevmode\ledsidenote{\textenglish{003/s}} क‚र‚णात् । तैः ‚{\color{DodgerBlue3}‚प‚रिग‚तो} युक्तः स‚न् ‚{\color{DodgerBlue3}‚विद्वेष्ट्य‚पि । ईर्ष्याम‚लैरिति} व्य‚क्त्य‚पेक्ष‚या ‚{\tiny $_{lb}$}‚ब‚हुव‚च‚नं । अनेन य‚थाक्र‚म‚म‚न‚र्थित्व‚म‚माध्य‚स्थ्य‚ञ्चोक्तं (।) ‚{\color{DodgerBlue3}‚तेन श्रोतृदोष‚क‚लार्पेन ‚{\tiny $_{lb}$}‚अय‚मा}‚रिप्सितो वार्त्तिकाख्यो ग्र‚न्थः (।) प‚र‚मुप‚क‚रोतीति ‚{\color{DodgerBlue3}‚प‚रोप‚कार} इति ‚{\color{DodgerBlue3}‚नोऽस्माक‚{\tiny $_{lb}$}‚ञ्चि‚{\tiny $_{7}$}‚न्तापि} नास्ति । क‚थ‚न्त‚र्हि शास्त्र‚क‚र‚णे प्र‚वृत्तिरित्याह चेत‚श्चिरं ‚{\color{DodgerBlue3}‚दीर्घ‚कालं \leavevmode\ledsidenote{\textenglish{2a/MA}} ‚{\tiny $_{lb}$}‚सूक्त}‚स्या‚{\color{DodgerBlue3}‚भ्यासेन विव‚र्द्धित‚व्य‚स‚नं} विस्तारिताभिष्व‚ङ्ग‚{\color{DodgerBlue3}‚मिति}‚हेतो‚{\color{DodgerBlue3}‚र‚त्र वार्तिक‚क‚र‚णे‚{\tiny $_{lb}$}‚ऽनुब‚द्ध‚स्पृहं} जाताभिलाषं । एतेन कुप्र‚ज्ञ‚तादिदोष‚जात‚मात्म‚नो बोधिताः श्रोतार‚{\tiny $_{lb}$}‚स्त‚त्प‚रिहारेण शास्त्रे प्र‚व‚र्तिता एव भ‚व‚न्ति ॥ (२)
	\pend% ending standard par
      
	  
	% new div opening: depth here is 2
	

	  \begin{center}%% label @type='head'
	\textbf{ग. प्र‚माण‚सिद्धिः}
	\end{center}
	

	  \pstart \leavevmode% starting standard par
	अय‚माचा र्यो बृह‚दाचार्यीय प्र मा ण स मु च्च य शा स्त्रे वार्त्तिकं चिकीर्षुः ‚{\tiny $_{lb}$}‚स्व‚तः कृत‚भ‚ग‚व‚न्न‚म‚स्कार(:) त‚च्छास्त्रा‚{\tiny $_{1}$}‚र‚म्भ‚स‚म‚ये त‚दाचार्य‚कृत‚भ‚ग‚व‚न्न‚म‚स्कार‚{\tiny $_{lb}$}‚श्लोकं व्याख्यातुकामः प्र‚थ‚मं\edtext{}{\edlabel{pvv.3-1}\label{pvv.3-1}\lemma{मं}\Bfootnote{द्वितीयां स‚म्वित्तिसिद्धिम् । प्र‚माणं भूतो जातो भ‚ग‚वान् मान‚मिव किन्त‚{\tiny $_{lb}$}‚दित्याह ।}} प्र‚माण‚सामान्य‚ल‚क्ष‚ण‚माह (।)
	\pend% ending standard par
      
	  
	% new div opening: depth here is 1
	
\chapter*[{१. प्र‚माण‚ल‚क्ष‚ण‚म्}]{१. प्र‚माण‚ल‚क्ष‚ण‚म्}

	  \begin{center}%% label @type='head'
	\textbf{(१) अविसंवादि ज्ञान‚म्}
	\end{center}
	\label{div_pvv.1.3}
	  
	% new div opening: depth here is 2
	
	  \bigskip
	  \begingroup
	
	    \large
	  
	    \begin{quote}
	  
	    
	    \stanza[\smallbreak]
	\label{pv.1.3a}\flagstanza{\tiny\textenglish{...v.1.3a}}प्र‚माण‚म‚विसंवादि ज्ञानं;\&[\smallbreak]


	
	    \end{quote}
	  
	  \endgroup
	

	  \pstart \leavevmode% starting standard par
	ज्ञानं प्र‚माणं\edtext{}{\edlabel{pvv.3-2}\label{pvv.3-2}\lemma{माणं}\Bfootnote{प्र‚माणं स‚म्य‚ग्ज्ञान‚म‚पूर्व्व‚गोच‚र‚मिति ल‚क्ष‚णं ।}}नाज्ञान‚मिन्द्रियार्थ‚स‚न्निक‚र्षादि । कीदृश‚म‚विसंवादि । विसंवा‚{\tiny $_{lb}$}‚द‚नं विसंवादो व‚ञ्ज‚नं त‚द्योगाद्विसंवादि । न त‚थाऽसाव‚विसंवादि । अविस‚म्वाद‚{\tiny $_{lb}$}‚न‚मुक्त‚मित्य‚र्थः । किं पुन‚रित्याह (।)
	\pend% ending standard par
      
	  \bigskip
	  \begingroup
	
	    \large
	  
	    \begin{quote}
	  
	    
	    \stanza[\smallbreak]
	\label{pv.1.3b}\flagstanza{\tiny\textenglish{...v.1.3b}}अर्थ‚क्रियास्थितिः ।&अविसंवाद‚नं;\&[\smallbreak]


	
	    \end{quote}
	  
	  \endgroup
	

	  \pstart \leavevmode% starting standard par
	य‚थोप‚द‚र्शितार्थ‚स्य क्रियायाः स्थि‚{\tiny $_{2}$}‚तिः प्र‚माण‚योग्य‚ताऽविसंवाद‚नं(।) अत‚श्च य‚तो ‚{\tiny $_{lb}$}‚ज्ञानाद‚र्थं प‚रिच्छिद्यापि\edtext{}{\edlabel{pvv.3-3}\label{pvv.3-3}\lemma{रिच्छिद्यापि}\Bfootnote{म‚रुम‚रीच्यादौ ।}} न प्र‚व‚र्त‚ते प्र‚वृत्तो वा कुत‚श्चित्प्र‚तिब‚न्धादेर‚र्थ‚क्रियान्नाधि‚{\tiny $_{lb}$}‚ग‚च्छ‚ति । त‚द‚पि प्र‚माण‚मेव प्र‚माण‚योग्य‚ताल‚क्ष‚ण‚स्याविसंवाद‚स्य स‚त्त्वात् । सैव ‚{\tiny $_{lb}$}‚प्र‚माण‚योग्य‚ता क‚थ‚म‚स‚त्याम‚र्थ‚क्रियाप्राप्तौ निश्चीय‚त इति चेत् (।) य‚त्ताव‚द‚स‚कृद्‚{\tiny $_{lb}$}‚व्य‚व‚हाराभ्यासाद्द‚र्श‚न‚मात्रेणोप‚ल‚क्षित‚भ्र‚म‚विविक्त‚स्व‚रूप‚विशेषं साध‚ना‚{\tiny $_{3}$}‚ध्य‚क्षं त‚स्य\edtext{}{\edlabel{pvv.3-4}\label{pvv.3-4}\lemma{स्य}\Bfootnote{प्र‚मेय‚स्य ।}} ‚{\tiny $_{lb}$}‚\leavevmode\ledsidenote{\textenglish{004/s}} स्व‚त एव प्र‚माण‚योग्य‚तानिश्च‚यः कृत्रिमाकृत्रिम‚म‚णिरुप्यादित‚त्व‚निश्च‚य‚व‚त् । ‚{\tiny $_{lb}$}‚अंनुमान‚स्य च साध्य‚प्र‚तिब‚द्ध‚ज‚न्म‚नो व्य‚भिचाराश‚ङ्काविर‚हात् । अर्थ‚क्रियानिर्भा‚{\tiny $_{lb}$}‚स‚न्तु प्र‚त्य‚क्षं स्व‚त एवार्थ‚क्रियानुभ‚वात्म‚कं न त‚त्र प‚रार्थ‚क्रियाऽपेक्ष्य‚त इति त‚द‚पि ‚{\tiny $_{lb}$}‚स्व‚तो निश्चित‚प्रामाण्यं । अत एवार्थ‚क्रियाप‚रंप‚रानुस‚र‚णाद‚न‚व‚स्थादोषोपि दुस्थ ‚{\tiny $_{lb}$}‚एव । य‚त्त्व‚न‚भ्य‚स्त‚द‚शायां‚{\tiny $_{4}$}‚ संदिग्ध‚प्रामाण्य‚मुत्प‚त्तौ\edtext{}{\edlabel{pvv.4-1}\label{pvv.4-1}\lemma{त्तौ}\Bfootnote{स‚त्यां ।}} त‚स्यार्थ‚क्रियाज्ञानाद‚नुमानाद्वा ‚{\tiny $_{lb}$}‚प्रामाण्यं निश्चीय‚ते । एत‚च्चाविसंवाद‚नं बाह्यार्थेत‚र‚वाद‚योः स‚मानं प्र‚माण‚{\tiny $_{lb}$}‚ल‚क्ष‚णं (।) वि ज्ञा न न‚येपि\edtext{}{\edlabel{pvv.4-2}\label{pvv.4-2}\lemma{येपि}\Bfootnote{अव्याप‚क‚त्वं निर‚स्य‚न्नाह ।}} साध‚न‚निर्भास‚ज्ञानान‚न्त‚र\edtext{}{\edlabel{pvv.4-3}\label{pvv.4-3}\lemma{र}\Bfootnote{व‚ह्निज्ञानान्त‚रं दाहादिज्ञानं ।}} म‚र्थंक्रिया\edtext{}{\edlabel{pvv.4-4}\label{pvv.4-4}\lemma{र्थंक्रिया}\Bfootnote{र‚विच‚न्द्राम्बुद‚चित्रादीनां द‚र्श‚न‚मेवार्थ‚क्रियास्थितिः ।}}निर्भास‚ज्ञान‚मेव\edtext{}{\edlabel{pvv.4-5}\label{pvv.4-5}\lemma{मेव}\Bfootnote{य‚द‚र्थाकारं ज्ञानं त‚द् बाह्यार्थाविनाभावि य‚था अर्थ‚क्रियानिर्भासं ।}} ‚{\tiny $_{lb}$}‚संवादः । अतो विज्ञ‚प्तिमात्र‚त्वे प्र‚माणेत‚र‚विभाग‚व्य‚व‚हारोऽसंकीर्ण्णः।
	\pend% ending standard par
      

	  \pstart \leavevmode% starting standard par
	न‚नु श‚ब्द‚ग‚न्ध‚र‚स‚स्प‚र्शान् चित्र‚रूप‚ञ्च प‚श्य‚तो ज्ञान‚स्य प‚र‚म‚र्थ‚क्रियाज्ञानं ‚{\tiny $_{lb}$}‚नास्तीति‚{\tiny $_{5}$}‚ त‚त्प्र‚माण‚न्न\edtext{}{\edlabel{pvv.4-6}\label{pvv.4-6}\lemma{न्न}\Bfootnote{विना भ्रान्तिं प्र‚युक्ते ।}} स्यादित्याह ।
	\pend% ending standard par
      
	  \bigskip
	  \begingroup
	
	    \large
	  
	    \begin{quote}
	  
	    
	    \stanza[\smallbreak]
	\label{pv.1.3c}\flagstanza{\tiny\textenglish{...v.1.3c}}शाब्देप्य‚भिप्राय‚निवेद‚नात् ॥ ३ ॥\&[\smallbreak]


	
	    \end{quote}
	  
	  \endgroup
	

	  \pstart \leavevmode% starting standard par
	\hphantom{.}‚{\color{DodgerBlue3}‚शाब्दे} श‚ब्द‚ज‚निते ज्ञानेऽपि श‚ब्दाद् ग‚न्धादिविष‚येऽपि ‚{\color{DodgerBlue3}‚अभिप्राय}‚स्याभिप्रेतार्थ‚{\tiny $_{lb}$}‚क्रिया (या) ‚{\color{DodgerBlue3}‚निवेद‚नात्} प्र‚तिपाद‚नात्प्रामाण्यं (।) अर्थ‚क्रिया हि क्व‚चित्स्व‚रूप‚प्र‚ति‚{\tiny $_{lb}$}‚प‚त्तिरेव । क्व‚चित्त‚तोऽन्या य‚थास‚म्भ‚वं व्य‚व‚हार‚विष‚यः । त‚त्प्राप‚ण‚ञ्च प्रामाण्य‚{\tiny $_{lb}$}‚मिति नाव्याप‚कं प्र‚माण‚ल‚क्ष‚ण‚म् । (३)
	\pend% ending standard par
      \label{div_pvv.1.4}
	  
	% new div opening: depth here is 2
	

	  \pstart \leavevmode% starting standard par
	न‚नु श‚ब्द‚स्यार्थ‚प्र‚तिब‚न्धाभावान्न प्रामाण्यं स्यादिष्य‚ते चानुमान‚त्वादि‚{\tiny $_{6}$}‚त्याह(।)
	\pend% ending standard par
      
	  \bigskip
	  \begingroup
	
	    \large
	  
	    \begin{quote}
	  
	    
	    \stanza[\smallbreak]
	व‚क्तृव्यापार‚विष‚यो योर्थो बुद्धौ प्र‚काश‚ते ।&प्रामाण्यं त‚त्र श‚ब्द‚स्य नार्थ‚त‚त्त्व‚निब‚न्ध‚न‚म् ॥ ४ ॥\&[\smallbreak]


	
	    \end{quote}
	  
	  \endgroup
	

	  \pstart \leavevmode% starting standard par
	\hphantom{.}‚{\color{DodgerBlue3}‚व‚क्तृर्व्यापारो} विव‚क्षा ‚{\color{DodgerBlue3}‚त‚स्य विष‚यो योऽर्थः} स‚मारोपित‚ब‚ही रूपो ज्ञानाकारः ‚{\tiny $_{lb}$}‚प्र‚काश‚ते ‚{\color{DodgerBlue3}‚बुद्धौ} विव‚क्षा\edtext{}{\edlabel{pvv.4-7}\label{pvv.4-7}\lemma{क्षा}\Bfootnote{संकेत‚ब‚लात् ।}}त्मिकायां (।) ‚{\color{DodgerBlue3}‚त‚त्र श‚ब्द‚स्य प्रामाण्यं} लिङ्ग‚त्वं । श‚ब्दादुच्च‚{\tiny $_{lb}$}‚रिताद्विव‚क्षितार्थ‚प्र‚तिभासी विक‚ल्पो\edtext{}{\edlabel{pvv.4-8}\label{pvv.4-8}\lemma{ल्पो}\Bfootnote{विक‚ल्प‚श‚ब्द . . . न न‚दी . . . . . . । . . .}}नुमीय‚त इत्य‚र्थः । त‚त्कार्य‚त्वात्त‚च्छ‚ब्द‚स्य । ‚{\color{DodgerBlue3}‚न ‚{\tiny $_{lb}$}‚पुन‚र‚र्थ‚त‚त्त्व‚निब‚न्ध‚नं} त‚त्प्र‚तिब‚न्धाभावात् ॥ (४)
	\pend% ending standard par
      \label{div_pvv.1.5}
	  
	% new div opening: depth here is 2
	

	  \pstart \leavevmode% starting standard par
	न‚नु घ\edtext{}{\edlabel{pvv.4-9}\label{pvv.4-9}\lemma{घ}\Bfootnote{येन ज्ञात्वा प्र‚वृत्त‚स्यार्थ‚संवाद‚स्त‚च्चेत्प्र‚माणं घ‚ट‚विक‚ल्पोपि स्यात्प्र‚मा ॥}}टोय‚मित्यादिज्ञानात्प्र‚व‚र्त‚मान‚स्य स‚म्ब‚न्धोस्त्येवेति त‚त् प्र‚माणं ‚{\tiny $_{lb}$}‚स्यात् (।) इत्याह ।
	\pend% ending standard par
      \textsuperscript{\textenglish{005/s}}
	  \bigskip
	  \begingroup
	
	    \large
	  
	    \begin{quote}
	  
	    
	    \stanza[\smallbreak]
	गृहीत‚ग्र‚ह‚णान्नेष्टं सांवृतं, धोप्र‚माण‚ता ।&प्र‚वृत्तेस्त‚त्प्र‚धान‚त्वात् हेयोपादेय‚व‚स्तुनि ॥ ५ ॥\&[\smallbreak]


	
	    \end{quote}
	  
	  \endgroup
	

	  \pstart \leavevmode% starting standard par
	\hphantom{.}‚{\color{DodgerBlue3}‚गृहीत‚ग्र‚{\tiny $_{7}$}‚ह‚णान्नेष्टं सांवृतं} द‚र्श‚नोत्त‚र‚कालं सांवृतं विक‚ल्प‚ज्ञानं प्र‚माणं नेष्टं \leavevmode\ledsidenote{\textenglish{2b/MA}} ‚{\tiny $_{lb}$}‚द‚र्श‚न‚गृहीत‚स्यैव ग्र‚ह‚णात् तेनैव च प्राप‚यितुं श‚क्य‚त्वात् सांवृत‚म\edtext{}{\edlabel{pvv.5-1}\label{pvv.5-1}\lemma{म}\Bfootnote{. . . . . . . घ‚टः । त‚द्ग‚त‚स‚त्ता म‚हासामान्यं । त‚त्संख्यान्त‚र्ग्ग‚तः । उत्क्षेप‚णं क‚र्म त‚स्यैवैते व्य‚प‚देशा इति सांवृताः ।}}किञ्चित्क‚र‚मेव । ‚{\tiny $_{lb}$}‚क‚स्मात्पुन‚र्द्धियः ‚{\color{DodgerBlue3}‚प्र‚माण}‚तेष्य‚ते नेन्द्रियादेः(।) ‚{\color{DodgerBlue3}‚हेयोपादेय‚व‚स्तु}‚विष‚यायाः ‚{\color{DodgerBlue3}‚प्र‚वृत्ते\edtext{}{\edlabel{pvv.5-2}\label{pvv.5-2}\lemma{वृत्ते}\Bfootnote{ज्ञात्वैव पुंसः प्र‚वृत्तेः ।}}स्त‚त्‚{\tiny $_{lb}$}‚प्र‚धान‚त्वात्} ज्ञान‚प्र‚धान‚त्वात् धिय एव प्रामाण्यं (।) न हीन्द्रिय‚म‚स्तीत्येव प्र‚वृत्तिः ‚{\tiny $_{lb}$}‚किन्त‚र्हि ज्ञान‚स‚द्भ‚वात् साध‚क‚त‚म‚ञ्च प्र‚माणं त‚स्याव्य‚व‚हित‚व्यापार‚{\tiny $_{1}$}‚त्वात् । (५)
	\pend% ending standard par
      \label{div_pvv.1.6}
	  
	% new div opening: depth here is 2
	

	  \pstart \leavevmode% starting standard par
	एवं फ‚लार्थिनां प्र‚वृत्तिव्य‚व‚हार‚कारित्वेन धियः प्रामाण्यं प्र‚तिपादितं । साम्प्र‚त‚{\tiny $_{lb}$}‚म‚धिग‚म‚फ‚ल‚विभाग‚कारित्व‚माह ।
	\pend% ending standard par
      
	  \bigskip
	  \begingroup
	
	    \large
	  
	    \begin{quote}
	  
	    
	    \stanza[\smallbreak]
	\label{pv.1.6a}\flagstanza{\tiny\textenglish{...v.1.6a}}विष‚याकार‚भेदाच्च धियोधिग‚म‚भेद‚तः ।\&[\smallbreak]


	
	    \end{quote}
	  
	  \endgroup
	

	  \pstart \leavevmode% starting standard par
	\hphantom{.}‚{\color{DodgerBlue3}‚धियो} विष‚य‚स्येवाकारो ‚{\color{DodgerBlue3}‚विष‚याकारः\edtext{}{\edlabel{pvv.5-3}\label{pvv.5-3}\lemma{याकारः}\Bfootnote{रूपं}}} नीलादिस्त‚स्य ‚{\color{DodgerBlue3}‚भेदात् विशेषाद‚धि‚{\tiny $_{lb}$}‚ग‚म\edtext{}{\edlabel{pvv.5-4}\label{pvv.5-4}\lemma{म}\Bfootnote{धियः ।}}} स्यार्थ‚प्र‚तीते‚{\color{DodgerBlue3}‚र्भेदा}‚द्विशेषाद्धिय एव प्रामाण्यं नील‚स्व‚रूपं हि ज्ञानं नील‚प्र‚तीति‚{\tiny $_{lb}$}‚र‚न्यादृश‚म‚न्य‚थेति धीरेव प्र‚माणं ।
	\pend% ending standard par
      

	  \pstart \leavevmode% starting standard par
	न‚नु य‚थाधिग‚म‚साध‚न\edtext{}{\edlabel{pvv.5-5}\label{pvv.5-5}\lemma{न}\Bfootnote{अधिग‚म‚स्य साध‚न‚मिन्द्रियादुत्प‚त्तेः ॥}}माकार‚स्त‚थेन्द्रिय‚म‚पि त‚दुत्प‚त्तेर‚त आह‚{\tiny $_{2}$}‚ (।)
	\pend% ending standard par
      
	  \bigskip
	  \begingroup
	
	    \large
	  
	    \begin{quote}
	  
	    
	    \stanza[\smallbreak]
	\label{pv.1.6b}\flagstanza{\tiny\textenglish{...v.1.6b}}भावादेवास्य त‚द्भावे;\&[\smallbreak]


	
	    \end{quote}
	  
	  \endgroup
	

	  \pstart \leavevmode% starting standard par
	\hphantom{.}त‚स्याकार‚स्य ‚{\color{DodgerBlue3}‚भावेऽस्या}‚धिग‚म‚स्य ‚{\color{DodgerBlue3}‚भावादेव} साध‚न‚म‚व्य‚व‚हित‚त्वात्‌न त्विन्द्रियादि ‚{\tiny $_{lb}$}‚त‚द्भावेपि ज्ञानानुत्प‚त्ताव‚धिग‚माभावात् । क‚श्चि\edtext{}{\edlabel{pvv.5-6}\label{pvv.5-6}\lemma{श्चि}\Bfootnote{मीमांस‚कः ।}}दाह (।) स‚र्व‚ज्ञानानाम‚बाधित‚त्व‚{\tiny $_{lb}$}‚ल‚क्ष‚णं प्रामाण्यं \edtext{}{\edlabel{pvv.5-7}\label{pvv.5-7}\lemma{प्रामाण्यं}\Bfootnote{उत्त‚र‚काल‚भाविनान‚भ्यास‚जेन चेत् ।}}स्व‚त एव सिध्य‚ते\edtext{}{\edlabel{pvv.5-8}\label{pvv.5-8}\lemma{ते}\Bfootnote{विष‚याकार‚स्य स्व‚स‚म्वेद‚नात् ज्ञान‚स‚त्तासिद्धिः ।}} बाध‚कार‚ण‚दोष‚ज्ञानाभ्यां क्व‚चित्त‚द‚पोह्य‚ते ‚{\tiny $_{lb}$}‚य‚था शुक्तिकायां र‚ज‚त‚ज्ञाने\edtext{}{\edlabel{pvv.5-9}\label{pvv.5-9}\lemma{ज्ञाने}\Bfootnote{स्व‚तः प्रामाण्य‚स्य ज्ञाते ज्ञाने त‚दात्म‚भूत‚स्य प्रामाण्य‚स्यापि ज्ञात‚त्वात् ।}} च‚न्द्र‚द्व‚य‚द‚र्श‚ने वा । त‚च्चेद‚म‚युक्तं य‚तः (।)
	\pend% ending standard par
      
	  \bigskip
	  \begingroup
	
	    \large
	  
	    \begin{quote}
	  
	    
	    \stanza[\smallbreak]
	\label{pv.1.6c}\flagstanza{\tiny\textenglish{...v.1.6c}}स्व‚रूप‚स्य स्व‚तो ग‚तिः ॥ ६ ॥\&[\smallbreak]


	
	    \end{quote}
	  
	  \endgroup
	\textsuperscript{\textenglish{006/s}}

	  \pstart \leavevmode% starting standard par
	\hphantom{.}‚{\color{DodgerBlue3}‚स्व‚रूप‚स्य स्व‚तो ग‚ति\edtext{}{\edlabel{pvv.6-1}\label{pvv.6-1}\lemma{ति}\Bfootnote{ज्ञानं । साध‚नं ।}}} र्न प्रामाण्य‚स्य । स्व‚तो हि प्रामाण्य‚स्याभिव्य‚क्ति\edtext{}{\edlabel{pvv.6-1-bis}\label{pvv.6-1-bis}\lemma{क्ति}\Bfootnote{ज्ञानं । साध‚नं ।}}‚{\tiny $_{lb}$}‚र्व्व‚क्त‚व्या ।‚{\tiny $_{3}$}‚ न तूत्प‚त्तिः । ज्ञानात्म‚भूत‚स्य स्व‚स्मादुत्प‚त्तिविरोधात् । य‚दि च स्व‚तो‚{\tiny $_{lb}$}‚ऽबाधित‚त्वं प्रामाण्य‚म‚भिव्य‚क्त्या व्य‚व‚स्थापितं न त‚स्य बाध‚कानां स‚ह‚स्रेणापि बाधो‚{\tiny $_{lb}$}‚युक्तः । अथ स‚म्भ‚व‚ति बाध‚के बाध‚काद‚र्श‚नं य‚त्र त‚त्राबाधित‚त्व‚म‚पोद्य‚ते बाध‚क‚{\tiny $_{lb}$}‚द‚र्श‚नेन । एव‚न्त‚र्हि बाध‚काद‚र्श‚नान्नाबाधित‚त्वं किन्त‚र्हि बाध‚काभावात् । त‚स्य ‚{\tiny $_{lb}$}‚चाद‚र्श‚नाद‚न्य‚न्न साध‚नं । त‚च्चेद‚साध‚नं नाबाधित‚त्वं नाम प्रामाण्यं नाप्य‚स्य स्व‚{\tiny $_{4}$}‚तः ‚{\tiny $_{lb}$}‚सिद्धिः । अतः प्र‚थ‚मं बाध‚काद‚र्श‚नात्प्र‚स‚क्त‚म‚बाधित‚त्वं बाध‚द‚र्श‚नाद‚पोद्य‚त इति ‚{\tiny $_{lb}$}‚किम‚त्रायुक्तं । ईदृश एव बाध्य\edtext{}{\edlabel{pvv.6-2}\label{pvv.6-2}\lemma{बाध्य}\Bfootnote{स‚तो न वाचा स‚म्वादे वाऽस‚तः स्व‚य‚मेवास‚त्त्वात् । ---सिद्धान्ती}}बाध‚क‚भावः स‚र्व्व‚तः ॥
	\pend% ending standard par
      

	  \pstart \leavevmode% starting standard par
	कीद‚शोऽत्र प्र‚स\edtext{}{\edlabel{pvv.6-3}\label{pvv.6-3}\lemma{स}\Bfootnote{अबाधित‚त्वं प्र‚स‚क्त‚मिति ।}}ङ्गार्थः (।) किं स‚त्त्व‚मुत स‚त्त्व‚निश्च‚यौ । स‚म्भाव‚नामात्र‚म्वा । ‚{\tiny $_{lb}$}‚त‚त्र न ताव‚द‚न‚योः प‚क्ष‚योर‚प‚वादो युक्तः । स‚तः केन‚चिद‚पि बाधितुम‚श‚क्य‚त्वात् ‚{\tiny $_{lb}$}‚अन्त्येपि न स्व\edtext{}{\edlabel{pvv.6-4}\label{pvv.6-4}\lemma{स्व}\Bfootnote{प्र‚माण‚मिद‚मिति निश्च‚य‚रूपा न च व्य‚क्तिः । प्र‚माण‚त‚दाभास‚योरुत्प‚त्तिकाले संश‚याद‚भ्यास‚म्विना ॥}}तोऽबाधित‚त्व‚निश्च‚यः त‚द्विरुद्ध‚त्वात् स‚म्भाव‚नायाः । अनिश्चित‚मेव ‚{\tiny $_{lb}$}‚त‚त्रा‚{\tiny $_{5}$}‚बाधित‚त्वं क‚थ\edtext{}{\edlabel{pvv.6-5}\label{pvv.6-5}\lemma{थ}\Bfootnote{निश्चित‚त्वादित्वेन ।}}म‚न्य‚थोत्प‚द्य‚त इति चेत् । न त‚र्हि स्व‚तःप्रामाण्य‚निश्च‚य इति ‚{\tiny $_{lb}$}‚य‚त्रापि बाध‚द‚र्श‚नं नास्ति त‚त्राप्य‚नाश्वास एव (।) एव‚ञ्च बाध्य‚बाध‚क‚भावः ‚{\tiny $_{lb}$}‚‚{\color{DodgerBlue3}‚स‚द‚स‚त्ताप‚क्ष‚योर‚स‚ङ्ग\edtext{}{\edlabel{pvv.6-6}\label{pvv.6-6}\lemma{ङ्ग}\Bfootnote{स्व‚तःप्र‚माण्य‚सिद्धिरित्युत्प‚त्तिर्व्य‚क्तिर्व्वा स्यात् सिद्धः श‚ब्दः सिद्ध ओद‚न‚व‚त् । नोत्प‚त्तिः । जाताजात‚ज्ञान‚योर‚कार‚क‚त्वात् । अस‚त्त्वे नाजात‚स्य जात‚स्य प्रामाण्यात्स‚त्वात् ।}}तो} वेदित‚व्यः ॥(६)
	\pend% ending standard par
      \label{div_pvv.1.7}
	  
	% new div opening: depth here is 2
	
	  \bigskip
	  \begingroup
	
	    \large
	  
	    \begin{quote}
	  
	    
	    \stanza[\smallbreak]
	\label{pv.1.7a}\flagstanza{\tiny\textenglish{...v.1.7a}}\edtext{\textsuperscript{*}}{\edlabel{pvv.6-asterisk}\label{pvv.6-asterisk}\lemma{*}\Bfootnote{द्र‚ष्ट‚व्यं प‚रिशिष्टं ।६}} प्रामाण्यं व्य‚व‚हारेण;\&[\smallbreak]


	
	    \end{quote}
	  
	  \endgroup
	

	  \pstart \leavevmode% starting standard par
	य‚दि स्व‚रूप‚मात्रं स्व‚तो ग‚म्य‚ते‚{\tiny $_{1}$}‚ न प्रामाण्यं क‚थ‚न्त‚र्हि त‚द‚व‚ग‚म्य‚मित्याह । ‚{\tiny $_{lb}$}‚‚{\color{DodgerBlue3}‚प्रामाण्य‚म्व्य‚व‚हारेणार्थ‚क्रियाज्ञानेन} (।) य‚स्य साध‚न‚ज्ञान‚स्य तादात्म्याद‚नुभूतेपि ‚{\tiny $_{lb}$}‚प्रामाण्ये साश‚ङ्का व्य‚व‚ह‚र्त्तारोऽन‚भ्या‚{\tiny $_{6}$}‚स‚व‚शाद‚नुत्प‚न्नानुरूप‚निश्च‚याः त‚त्रार्थ‚क्रियाज्ञा‚{\tiny $_{lb}$}‚नेन प्रामाण्य‚निश्च‚यः । अन्य‚त्र तु विभ्र\edtext{}{\edlabel{pvv.6-7}\label{pvv.6-7}\lemma{विभ्र}\Bfootnote{इति ब‚हिर‚र्थे प्रामाण्याभावोस्य ।}}म‚श‚ङ्कास‚ङ्कोचादुत्प‚त्तावेव स्व‚रूप\edtext{}{\edlabel{pvv.6-8}\label{pvv.6-8}\lemma{रूप}\Bfootnote{स्व‚स‚म्वेद‚नेन ।}}स्य ‚{\tiny $_{lb}$}‚प्रामाण्य‚स्य स्व‚तो ग‚तिरित्युक्तं ॥ अथ‚वा\edtext{}{\edlabel{pvv.6-9}\label{pvv.6-9}\lemma{वा}\Bfootnote{स्व‚रूप‚स्य स्व‚तो ग‚तिः । प्रामाण्यं व्य‚व‚हारेणेत्य‚स्य व्याख्यान्त‚र‚माह ।}} च‚क्षुर्विज्ञानेन रूप‚क्ष‚ण एको दृश्य‚ते न भावी ‚{\tiny $_{lb}$}‚प्राप्यो नापि स्प‚र्शः त‚त्क‚य‚म‚न्य‚द‚र्श‚न‚म‚न्य‚प्राप्त्या प्र‚माणं । एवं ह्य‚तिप्र‚स‚ङ्गः स्यात् । ‚{\tiny $_{lb}$}‚\leavevmode\ledsidenote{\textenglish{007/s}} अनुमान‚ञ्च व्याप्तिग्र‚ह‚ण‚सापेक्षं व्याप्तिश्च प्र‚त्य‚क्षेण पुरोव‚र्त्तिरूप‚मात्र‚ग्राहिणा क‚थं‚{\tiny $_{7}$}‚ ‚{\tiny $_{lb}$}‚श‚क्य‚ग्र‚हा । देश‚काल‚व्य‚क्तिव्याप्त्या च व्याप्तिरुच्य‚ते य‚त्र य‚त्र धूम‚स्त‚त्र त‚त्राग्नि\leavevmode\ledsidenote{\textenglish{3a/MA}}‚{\tiny $_{lb}$}‚रिति प्र‚त्य‚क्ष‚पृष्ठ‚ज‚श्च विक‚ल्पो न प्र‚माणं प्र‚माण‚व्यापारानुकारी त्व‚साविष्य‚ते । ‚{\tiny $_{lb}$}‚य‚त्राय‚म‚ध्य‚क्ष‚व्यापार‚म‚तिक्र‚म्याधिक‚मारोप‚य‚ति ‚{\color{DodgerBlue3}‚त‚त्र न प्र‚माणं अमूल‚क‚त्वात्त‚स्य} प्र‚माण‚प्र‚मेय‚स्य । विजातीय‚व्यावृत्तेर‚ध्य‚क्षेण दृष्ट‚त्वाद‚स्त्येव मूल‚मिति चेत् न ‚{\tiny $_{lb}$}‚स‚जातीय‚व्यावृत्त्या विशेषित‚त्वात्त‚स्याः । अन्य‚था शाब‚लेय‚नाश‚म्प्र‚तिय‚ता प्र‚त्य‚क्षेण ‚{\tiny $_{lb}$}‚गोमा‚{\tiny $_{1}$}‚त्र‚नाशो व्य‚व‚स्थाप्येत । अनुमानाच्च व्याप्तिग्र‚ह‚णेऽन‚व‚स्थाप‚त्तिः । अत ‚{\tiny $_{lb}$}‚आह । ‚{\color{DodgerBlue3}‚स्व‚रूप‚स्य स्व‚तो ग‚तिः} ॥
	\pend% ending standard par
      

	  \pstart \leavevmode% starting standard par
	स्व‚रूप‚मात्रं स्व‚तो ग‚म्य‚ते न प्राप्य‚रूप‚सापेक्षं प्रामाण्य‚न्नाम किञ्चिद‚स्ति । ‚{\tiny $_{lb}$}‚क‚थ‚न्त‚र्हि त‚द्व्य‚व‚स्थेत्याह । ‚{\color{DodgerBlue3}‚प्रामाण्य‚म्व्य‚व‚हारेण ।}
	\pend% ending standard par
      

	  \pstart \leavevmode% starting standard par
	सांव्य‚व‚हारिक‚स्येदं प्र‚माण‚स्य ल‚क्ष‚णं संव्य‚व‚हार‚श्च भाविभूत‚रूपादिक्ष‚णानामे‚{\tiny $_{lb}$}‚क‚त्वेन संवाद‚विष‚योन‚व‚गीतः स‚र्व्व‚स्य । साध्य‚साध‚न‚योरेक‚व्य‚क्ति\edtext{}{\edlabel{pvv.7-1}\label{pvv.7-1}\lemma{क्ति}\Bfootnote{य‚दि धूमो व‚ह्नेर‚न्य‚तोपि जायेतेह व‚ह्नेर्न जायेत द्विकार‚ण‚म‚कार‚णं य‚तः ।}}द‚र्श‚ने‚{\tiny $_{2}$}‚ ‚{\color{DodgerBlue3}‚स‚म‚स्त}‚{\tiny $_{lb}$}‚त‚ज्जातीय‚त‚थात्व‚व्य‚व‚स्थानं स‚म्वा\edtext{}{\edlabel{pvv.7-2}\label{pvv.7-2}\lemma{म्वा}\Bfootnote{साध्याधिग‚तिः साध‚नं सारूप्ये त‚योः ।}} द‚म‚व‚धार‚य‚न्ति व्य‚व‚ह\edtext{}{\edlabel{pvv.7-3}\label{pvv.7-3}\lemma{ह}\Bfootnote{प्राग‚दृष्टो धूमाधीः ।}}र्त्तारः । त‚द‚नुरोधात् ‚{\tiny $_{lb}$}‚प्रामाण्य‚म्व्य‚व‚स्थाप्य‚ते । त‚त्त्व‚त‚स्तु स्व‚स‚म्वेद‚न‚मात्र‚म‚प्र‚वृत्तिनिवृत्तिकं ॥
	\pend% ending standard par
      

	  \pstart \leavevmode% starting standard par
	न‚नु य‚त्र ताव‚द‚भ्य‚स्त‚साध‚न‚ज्ञानादिषु निर‚स्त‚भ्र‚मा व्य‚व‚हारिण‚स्तेषां स्व‚त ‚{\tiny $_{lb}$}‚एव प्रामाण्य‚निश्च‚यः य‚त्त्व‚न‚भ्य‚स्त‚साध‚नं ज्ञानं त‚स्यापि व्य‚व‚हारेणेति निष्फ‚लं शास्त्र‚{\tiny $_{lb}$}‚प्र‚ण‚य‚न‚मित्याह ।
	\pend% ending standard par
      
	  \bigskip
	  \begingroup
	
	    \large
	  
	    \begin{quote}
	  
	    
	    \stanza[\smallbreak]
	\label{pv.1.7b}\flagstanza{\tiny\textenglish{...v.1.7b}}शास्त्रं मोह‚निर्त‚नं ।\&[\smallbreak]


	
	    \end{quote}
	  
	  \endgroup
	

	  \pstart \leavevmode% starting standard par
	य‚दि व्य‚व‚हार‚तः प्र‚माण‚स्व‚रूप‚{\tiny $_{3}$}‚सिद्धिः प‚र‚स्प‚र‚विरो\edtext{}{\edlabel{pvv.7-4}\label{pvv.7-4}\lemma{विरो}\Bfootnote{स‚म्य‚ग्ज्ञानाद् ध‚र्मास्तित्व‚प‚र‚लोक‚निश्च‚यः त‚तो मोक्षाधिग‚मात् प्र‚त्य‚क्ष‚पृष्ठ‚{\tiny $_{lb}$}‚विक‚ल्पाख्यं (।)}}धीनि ल‚क्ष‚ण‚शास्त्राणि न स्युः । त‚स्माच्छास्त्रेण ल‚क्ष‚णोप‚द‚र्श‚नात् त‚द्विष‚यः संमोहो निव‚र्त‚नीयः । येन\edtext{}{\edlabel{pvv.7-5}\label{pvv.7-5}\lemma{येन}\Bfootnote{अस्याज्ञात‚स्य ग्राह्य‚त्वेपि नाय‚म‚र्थः ।}} प‚र‚लो‚{\tiny $_{lb}$}‚क‚निःश्रेय‚सादेर्व्य‚व‚हाराप्र‚सिद्ध‚स्य सिद्धिर्भ‚व‚ति ।
	\pend% ending standard par
      

	  \begin{center}%% label @type='head'
	\textbf{(२) अज्ञातार्थ‚प्र‚काश‚क‚म्}
	\end{center}
	

	  \pstart \leavevmode% starting standard par
	त‚देव‚म‚विस‚म्वाद‚नं प्र‚माण‚ल‚क्ष‚ण‚मुक्त‚भिदानीम‚न्य‚दाह ।
	\pend% ending standard par
      \textsuperscript{\textenglish{008/s}}
	  \bigskip
	  \begingroup
	
	    \large
	  
	    \begin{quote}
	  
	    
	    \stanza[\smallbreak]
	\label{pv.1.7c}\flagstanza{\tiny\textenglish{...v.1.7c}}अज्ञातार्थ‚प्र‚काशो वा;\&[\smallbreak]


	
	    \end{quote}
	  
	  \endgroup
	

	  \pstart \leavevmode% starting standard par
	प्र‚काश‚नं प्र‚काशोऽज्ञा\edtext{}{\edlabel{pvv.8-1}\label{pvv.8-1}\lemma{काशोऽज्ञा}\Bfootnote{बुद्धोप्य‚न्याज्ज्ञात‚ज्ञानात् अत्राप्य‚विस‚म्वादादेव ।}} त‚स्यार्थ‚स्य प्र‚काशो ज्ञानं (।) त‚त्प्र‚माणं । अर्थ‚ग्र‚ह‚णेन ‚{\tiny $_{lb}$}‚द्विच‚न्द्रादिज्ञान‚स्य निरासः । अज्ञात‚ग्र‚ह‚णेन साम्वृत‚{\tiny $_{4}$}‚स्याव‚य‚व्यादिविष‚य‚स्य । पृथ‚ग् ‚{\tiny $_{lb}$}‚गृहीतानामेव रूपादीनामेक‚त्वेन विक‚ल्प‚नात् (।) स्म‚र‚ण‚ञ्च पूर्व‚गृहीतार्थ‚विक‚ल्प‚{\tiny $_{lb}$}‚रूप‚त्वान्नाधिक‚ग्राहि (।) गृहीते च प्राक्त‚न‚मेव प्र‚माणं । इदानीन्तु स्म‚र‚ण‚म‚प्र\edtext{}{\edlabel{pvv.8-2}\label{pvv.8-2}\lemma{प्र}\Bfootnote{पूर्व्व‚दृष्ट‚स्य य‚द‚स्तित्वं ।}} व‚र्त‚कं ‚{\tiny $_{lb}$}‚त‚स्यैव स‚न्देहात् ॥
	\pend% ending standard par
      

	  \pstart \leavevmode% starting standard par
	न‚न्व‚विस‚म्वादादेवाज्ञातार्थ‚प्र‚काशो ज्ञात‚व्यः । अन्य‚था पीत‚शंख‚ज्ञान‚म‚पि प्र‚माणं ‚{\tiny $_{lb}$}‚स्यात् । त‚था चाविस‚म्वादित्व‚मेव प्र‚माण‚म‚स्तु किम‚नेनाभि\edtext{}{\edlabel{pvv.8-3}\label{pvv.8-3}\lemma{नेनाभि}\Bfootnote{ज्ञान‚ञ्चास‚च्च स्यान्न बाह्ये प्र‚काश‚कं य‚था विक‚ल्प‚कं ।}} हितेन‚{\tiny $_{5}$}‚ (।) स्यादे‚{\tiny $_{lb}$}‚त‚द्य‚दि स‚म्भ‚वित्व‚मात्रे ल‚क्ष‚णं स्यात् । किं नूद्दिष्ट\edtext{}{\edlabel{pvv.8-4}\label{pvv.8-4}\lemma{नूद्दिष्ट}\Bfootnote{य‚द्य‚न‚धि (ग‚म) विष‚यं प्र‚माणं प्र‚त्य‚क्षाग्र(?गृ) हीतं सामान्यं स्व‚ल‚क्ष ‚{\tiny $_{lb}$}‚ण‚विष‚य‚त्वात् प‚र‚न्तु प्र‚त्य‚क्ष‚ग्राह्यं रूपिस‚म‚वायात्त‚च्चाक्षुष‚माह मीमांस‚कादेः ।}} त्वेनान्य‚था ज्ञान‚त्व‚स‚त्त्वादिक‚म‚पि ‚{\tiny $_{lb}$}‚ल‚क्ष‚णं स्यात् ।
	\pend% ending standard par
      

	  \pstart \leavevmode% starting standard par
	न‚न्व‚विस‚म्वादिभ्योऽज्ञातार्थ‚प्र‚काश‚कं ज्ञाय‚ते\edtext{}{\edlabel{pvv.8-5}\label{pvv.8-5}\lemma{ते}\Bfootnote{विस‚म्वादिनः प्र‚काश‚क‚त्वायोगात् ।}} न तु ज्ञान‚त्वादिभ्य इति पूर्व्व‚स्या‚{\tiny $_{lb}$}‚पेक्ष‚णीय‚ता ल‚क्ष‚णेन । न तु प‚रेषामिति विशेषः । य‚द्येवं त‚दाऽविस‚म्वादित्वेप्य‚{\tiny $_{lb}$}‚ज्ञातार्थ‚प्र‚काश‚न‚म‚पेक्ष‚त एव (।) नान्य‚था सांवृत‚स्य निरासः श‚क्यः क‚र्त्तुं (।) ‚{\tiny $_{lb}$}‚त‚स्मादुभ‚य‚म‚पि प‚र‚स्प‚र‚सापेक्ष‚मे‚{\tiny $_{6}$}‚व ल‚क्ष‚ण‚म्बोद्ध‚व्यं ॥ (७)
	\pend% ending standard par
      \label{div_pvv.1.8}
	  
	% new div opening: depth here is 2
	

	  \pstart \leavevmode% starting standard par
	न‚नु स्व‚ल‚क्ष‚ण‚प्र‚तीतेरूर्द्ध्वं सामा\edtext{}{\edlabel{pvv.8-6}\label{pvv.8-6}\lemma{सामा}\Bfootnote{प्र‚माण‚स‚मुच्च‚ये आग‚मे च ।}} न्य‚विष‚यं ज्ञान‚म‚ज्ञातार्थ‚प्र‚काश‚क‚त्वात्प्र‚माणं ‚{\tiny $_{lb}$}‚प्राप्तं त‚देवाह (।)
	\pend% ending standard par
      
	  \bigskip
	  \begingroup
	
	    \large
	  
	    \begin{quote}
	  
	    
	    \stanza[\smallbreak]
	\label{pv.1.7d}\flagstanza{\tiny\textenglish{...v.1.7d}}स्व‚रूपाधिग‚तेः प‚र‚म् ॥ ७ ॥\&[\smallbreak]


	
	    \end{quote}
	  
	  \endgroup
	
	  \bigskip
	  \begingroup
	
	    \large
	  
	    \begin{quote}
	  
	    
	    \stanza[\smallbreak]
	\label{pv.1.8a}\flagstanza{\tiny\textenglish{...v.1.8a}}प्राप्तं सामान्य‚विज्ञानं ;\&[\smallbreak]


	
	    \end{quote}
	  
	  \endgroup
	

	  \pstart \leavevmode% starting standard par
	प्र‚माण‚मिति शेषः\edtext{}{\edlabel{pvv.8-7}\label{pvv.8-7}\lemma{शेषः}\Bfootnote{अस्तु वाऽन‚धिग‚म‚स्त‚थापि ।}} ।
	\pend% ending standard par
      

	  \pstart \leavevmode% starting standard par
	अत्राह ।
	\pend% ending standard par
      
	  \bigskip
	  \begingroup
	
	    \large
	  
	    \begin{quote}
	  
	    
	    \stanza[\smallbreak]
	\label{pv.1.8b}\flagstanza{\tiny\textenglish{...v.1.8b}}अविज्ञाते स्व‚ल‚क्ष‚णे (।)&य‚ज्ज्ञान‚मित्य‚भिप्रायात्;\&[\smallbreak]


	
	    \end{quote}
	  
	  \endgroup
	\textsuperscript{\textenglish{009/s}}

	  \pstart \leavevmode% starting standard par
	अज्ञात‚स्व‚ल‚क्ष‚ण‚विष\edtext{}{\edlabel{pvv.9-1}\label{pvv.9-1}\lemma{विष}\Bfootnote{ज्ञान‚त्वादीनां ।}}यं ‚{\color{DodgerBlue3}‚य‚ज्ज्ञानं} त‚त्प्र‚माणं न ज्ञात‚विष‚य‚{\color{DodgerBlue3}‚मित्य‚भिप्रायान्नातिप्र}‚{\tiny $_{lb}$}‚स\edtext{}{\edlabel{pvv.9-2}\label{pvv.9-2}\lemma{स}\Bfootnote{अन‚धिग‚ते स्व‚ल‚क्ष‚णे य‚द‚न‚धि (ग‚म) विष‚य‚मिति स‚विशेष‚णं ल‚क्ष‚णं वाच्यं ।}} ङ्गः ॥ एव‚न्त‚र्हि अनुमान‚म‚पि सामा\edtext{}{\edlabel{pvv.9-3}\label{pvv.9-3}\lemma{सामा}\Bfootnote{श‚ब्दः प्र‚त्य‚क्षोऽन्य‚थाऽश्र‚यासिद्धिः स्यात् । दृष्ट‚स्य श‚ब्दानित्य‚त्व‚स्य प्र‚त्या‚{\tiny $_{lb}$}‚य‚नात् ज्ञात‚स्य ग्राह्य‚त्वे . . . . . .।}}न्य‚विष‚य‚त्वात् प्र‚माणं न स्यात् । ‚{\tiny $_{lb}$}‚नैत‚द‚स्ति त‚द‚पि‚{\tiny $_{7}$}‚ च ल‚क्ष\edtext{}{\edlabel{pvv.9-4}\label{pvv.9-4}\lemma{क्ष}\Bfootnote{अध्य‚क्षेण तु स एवागृहीतो य‚त्र निश्च‚यो ज‚नितः । व्याव‚हारिकाधिकारात् ‚{\tiny $_{lb}$}‚अन्यापोह‚विष‚याच्च ॥}} ण‚मेवानित्यादिरूप‚त‚या विष‚यीक‚रोति ॥
	\pend% ending standard par
      \textsuperscript{\textenglish{3b/MA}}‚{\tiny $_{lb}$}‚

	  \pstart \leavevmode% starting standard par
	किं पुन‚र‚धिग‚त‚स्व‚ल‚क्ष‚ण‚विष‚य‚मेव प्र‚माण‚मिष्टं ।
	\pend% ending standard par
      
	  \bigskip
	  \begingroup
	
	    \large
	  
	    \begin{quote}
	  
	    
	    \stanza[\smallbreak]
	\label{pv.1.8d}\flagstanza{\tiny\textenglish{...v.1.8d}}स्व‚ल‚क्ष‚ण‚विचार‚तः ॥ ८ ॥\&[\smallbreak]


	
	    \end{quote}
	  
	  \endgroup
	

	  \pstart \leavevmode% starting standard par
	\hphantom{.}‚{\color{DodgerBlue3}‚स्व‚ल‚क्ष‚ण‚विचार‚तो}‚ऽर्थ‚क्रियार्थिभिः स्व‚ल‚क्ष‚ण\edtext{}{\edlabel{pvv.9-5}\label{pvv.9-5}\lemma{ण}\Bfootnote{अनुप‚ल‚ब्धिश्च प्र‚देशः ज्ञान‚म्वेति व‚स्तुतो व‚स्त्व‚धिष्ठानैव ।}} मेव प्र‚माणेनान्विष्य‚ते त‚स्यैवार्थ‚{\tiny $_{lb}$}‚क्रियासाध‚न‚त्वा\edtext{}{\edlabel{pvv.9-6}\label{pvv.9-6}\lemma{त्वा}\Bfootnote{आत्माकाशादौ त‚द्व‚स्तुभूत‚शून्या त‚द्बुद्धिरेवाश्र‚यः ।}}त् । य‚देव च तैर‚न्विष्य‚ते त‚देव शास्त्रे विचार्य‚ते सांव्य‚व‚हारिक‚{\tiny $_{lb}$}‚प्र‚माणाधिकारात्\edtext{}{\edlabel{pvv.9-7}\label{pvv.9-7}\lemma{माणाधिकारात्}\Bfootnote{आकाशादिविभुत्व‚व‚दीश्व‚र‚प्रामाण्यं क‚टाक्ष‚य‚ति । द्विविधेन य‚थोक्तेन ल‚क्ष‚णेन ‚{\tiny $_{lb}$}‚निर्दिष्टं य‚देत‚त् प्र‚माणं । प्र‚माण‚साध‚र्म्य‚न्तु साध‚यिष्य‚माणं सिद्धं कृत्वोदाहृतं ।}}॥ (८)
	\pend% ending standard par
      \label{div_pvv.1.9}
	  
	% new div opening: depth here is 2
	

	  \begin{center}%% label @type='head'
	\textbf{(३) भ‚ग‚व‚तः प्रामाण्य‚म्}
	\end{center}
	

	  \pstart \leavevmode% starting standard par
	य‚थोक्त‚द्विविध‚ल‚क्ष‚ण‚मुक्तं य‚त्प्र‚माणं (।)
	\pend% ending standard par
      
	  \bigskip
	  \begingroup
	
	    \large
	  
	    \begin{quote}
	  
	    
	    \stanza[\smallbreak]
	\label{pv.1.9a}\flagstanza{\tiny\textenglish{...v.1.9a}}त‚द्व‚त् प्र‚माणं भ‚ग‚वान्;\&[\smallbreak]


	
	    \end{quote}
	  
	  \endgroup
	

	  \pstart \leavevmode% starting standard par
	\hphantom{.}‚{\color{DodgerBlue3}‚त‚द्व‚द् भ‚ग‚वान् प्र‚माणं} । य‚थाभिहित‚स्य स‚त्य‚च‚तुष्ट‚य‚स्याविस‚म्वाद‚नात्त‚स्यैव ‚{\tiny $_{lb}$}‚प‚रैर‚ज्ञात‚स्य प्र\edtext{}{\edlabel{pvv.9-8}\label{pvv.9-8}\lemma{प्र}\Bfootnote{ल‚क्ष‚णं ।}}काश‚नाच्च‚{\tiny $_{1}$}‚ ॥
	\pend% ending standard par
      

	  \pstart \leavevmode% starting standard par
	\hphantom{.}य‚द्येवं न‚म‚स्कार‚श्लोके प्र‚माणायेत्येवास्तु प्र‚गाण‚भूतायेति किम‚र्थ‚मित्याह ।
	\pend% ending standard par
      
	  \bigskip
	  \begingroup
	
	    \large
	  
	    \begin{quote}
	  
	    
	    \stanza[\smallbreak]
	\label{pv.1.9b}\flagstanza{\tiny\textenglish{...v.1.9b}}अभू\edtext{}{\edlabel{pvv.9-9}\label{pvv.9-9}\lemma{अभू}\Bfootnote{अजात‚त्व‚निवृत्त्य‚र्थ जात‚त्वोक्तं ।}}त‚विनिवृत्त‚ये (।)&भूतोन्क्तिः;\&[\smallbreak]


	
	    \end{quote}
	  
	  \endgroup
	\textsuperscript{\textenglish{010/s}}

	  \pstart \leavevmode% starting standard par
	\hphantom{.}‚{\color{DodgerBlue3}‚भूत‚श‚ब्द}‚निर्देशोऽभूत‚स्य नित्य‚स्य निवृत्त्य‚र्थ ‚{\color{DodgerBlue3}‚नित्यं प्र‚माणं ना}‚स्तीत्य‚र्थः ।
	\pend% ending standard par
      
	  \bigskip
	  \begingroup
	
	    \large
	  
	    \begin{quote}
	  
	    
	    \stanza[\smallbreak]
	\label{pv.1.9c}\flagstanza{\tiny\textenglish{...v.1.9c}}साध‚नापेक्षा त‚तो युक्तां प्र‚माण‚ता ॥ ९ ॥\&[\smallbreak]


	
	    \end{quote}
	  
	  \endgroup
	

	  \pstart \leavevmode% starting standard par
	त‚तः साध\edtext{}{\edlabel{pvv.10-1}\label{pvv.10-1}\lemma{साध}\Bfootnote{य‚दा भ‚ग‚व‚ज्ज्ञान‚मुत्प‚न्नं त‚दा नाक‚स्मिक‚मिति स्व‚कार‚णं सूच‚य‚तीति अनुष्ठित‚{\tiny $_{lb}$}‚प्रामाण्याविप‚रीत‚साध‚न‚श्च भ‚ग‚वानिति स्व‚भाव‚हेतुः ।}}नापेक्षा प्र‚माण‚ता युक्ता भ‚ग‚व‚त‚श्च प्रामाण्य‚साध‚नं व‚क्ष्य‚माणं (९) ।
	\pend% ending standard par
      \label{div_pvv.1.10_1.11}
	  
	% new div opening: depth here is 2
	

	  \begin{center}%% label @type='head'
	\textbf{(४) ईश्व‚रादेर‚प्रामाण्य‚म्}
	\end{center}
	

	  \begin{center}%% label @type='head'
	\textbf{क. नित्यानिंत्य‚योर‚प्र‚माण‚ता}
	\end{center}
	

	  \begin{center}%% label @type='head'
	\textbf{(क) नित्य‚स्याप्र‚माण‚ता}
	\end{center}
	
	  \bigskip
	  \begingroup
	
	    \large
	  
	    \begin{quote}
	  
	    
	    \stanza[\smallbreak]
	\label{pv.1.10a}\flagstanza{\tiny\textenglish{....1.10a}}नित्यं प्र‚माणं नैवास्ति प्रामाण्यात्;\&[\smallbreak]


	
	    \end{quote}
	  
	  \endgroup
	

	  \pstart \leavevmode% starting standard par
	क‚स्मात् पुन‚र्नित्यं प्र‚माणं नैवास्ति(।)आ\edtext{}{\edlabel{pvv.10-2}\label{pvv.10-2}\lemma{आ}\Bfootnote{नित्य‚मीश्व‚रं नैयायिकाः प्र‚माण‚माहुः । आसंसार‚मेकं प्र‚तिस‚त्त्वं बुद्धिं ‚{\tiny $_{lb}$}‚प्र‚माण‚माहुः सांख्याः ।}}ह व‚स्तुनोर्थ‚क्रियाकारिणः स‚तो ‚{\tiny $_{lb}$}‚ग‚तेर्ज्ञान‚स्य ‚{\color{DodgerBlue3}‚प्रामाण्यान्नास्ति नित्यं प्र‚माणं} । अत्रैव कार‚ण‚माह ।
	\pend% ending standard par
      
	  \bigskip
	  \begingroup
	
	    \large
	  
	    \begin{quote}
	  
	    
	    \stanza[\smallbreak]
	\label{pv.1.10b}\flagstanza{\tiny\textenglish{....1.10b}}व‚स्तुसंग‚तेः ।&ज्ञेयानित्य‚त‚या त‚स्या अध्रौव्यात्;\&[\smallbreak]


	
	    \end{quote}
	  
	  \endgroup
	

	  \pstart \leavevmode% starting standard par
	\hphantom{.}‚{\color{DodgerBlue3}‚ज्ञेय}‚स्य व‚स्तुनोऽर्थ‚क्रियाका‚{\tiny $_{1}$}‚रित्वेना‚{\color{DodgerBlue3}‚नित्य}‚त्वात् ‚{\color{DodgerBlue3}‚त‚स्या व‚स्तुस‚ङ्ग‚ते}‚र‚पि ‚{\tiny $_{lb}$}‚त‚ज्ज‚न्यायाऽ(?अ) ‚{\color{DodgerBlue3}‚ध्रौव्याद}‚नित्य‚त्वात् ।
	\pend% ending standard par
      

	  \pstart \leavevmode% starting standard par
	स्यादेत‚द् (।) अनित्य‚विष‚य‚म‚नित्य‚मेव ज्ञानं केव‚लं य‚स्य त‚त् ज्ञानं स ज्ञाता ‚{\tiny $_{lb}$}‚‚{\color{DodgerBlue3}‚नित्यो} भ‚विष्य‚तीत्याह (।)
	\pend% ending standard par
      
	  \bigskip
	  \begingroup
	
	    \large
	  
	    \begin{quote}
	  
	    
	    \stanza[\smallbreak]
	\label{pv.1.10c}\flagstanza{\tiny\textenglish{....1.10c}}क्र‚म‚ज‚न्म‚नां ॥ १० ॥\&[\smallbreak]


	
	    \end{quote}
	  
	  \endgroup
	
	  \bigskip
	  \begingroup
	
	    \large
	  
	    \begin{quote}
	  
	    
	    \stanza[\smallbreak]
	\label{pv.1.11a}\flagstanza{\tiny\textenglish{....1.11a}}नित्यादुत्प‚त्तिविश्लेषाद‚पेक्षाया अयोग‚तः ।\&[\smallbreak]


	
	    \end{quote}
	  
	  \endgroup
	

	  \pstart \leavevmode% starting standard par
	\hphantom{.}ज्ञान‚स्य नित्यात् ज्ञातुरु‚{\color{DodgerBlue3}‚त्प‚त्तेर्व्वि\edtext{}{\edlabel{pvv.10-3}\label{pvv.10-3}\lemma{त्तेर्व्वि}\Bfootnote{अयोगात्}} श्लेषात्} । नित्यं हि स‚दैक‚रूपं य‚दि ‚{\tiny $_{lb}$}‚‚{\color{DodgerBlue3}‚क्र‚म‚ज‚न्म‚नां} ज्ञानानाम‚र्ज‚न‚स‚म‚र्थं । स‚कृदेव तानि कुर्य्यात् । अथ स‚म‚र्थ‚म‚पि नित्यं ‚{\tiny $_{lb}$}‚क्र‚मिस‚ह‚कार्य्य‚पेक्ष‚या क्र‚मेण क‚रोति त‚द‚युक्त‚{\color{DodgerBlue3}‚म‚पेक्षाया अयोगात्} ।
	\pend% ending standard par
      

	  \pstart \leavevmode% starting standard par
	क‚स्मात् स‚ह‚कार्य‚पेक्षा‚{\tiny $_{3}$}‚ न युक्तेत्याह (।),
	\pend% ending standard par
      
	  \bigskip
	  \begingroup
	
	    \large
	  
	    \begin{quote}
	  
	    
	    \stanza[\smallbreak]
	\label{pv.1.11b}\flagstanza{\tiny\textenglish{....1.11b}}क‚थ‚ञ्चिन्नोप‚कार्य‚त्वात्;\&[\smallbreak]


	
	    \end{quote}
	  
	  \endgroup
	\textsuperscript{\textenglish{011/s}}

	  \pstart \leavevmode% starting standard par
	\hphantom{.}नित्य‚स्य स‚र्व्व‚दाऽविशिष्ट‚स्व‚भाव‚स्य प‚रैः स‚ह‚कारिभिः ‚{\color{DodgerBlue3}‚क‚थ‚ञ्चिन्नोप‚कार्य‚त्वात्} क्व त‚द‚पेक्षा । त‚तः स‚र्व्व‚ज्ञानानि स‚कृदेव कुर्य्यादित्य‚वार्य्यः प्र‚स‚ङ्गः ॥
	\pend% ending standard par
      

	  \begin{center}%% label @type='head'
	\textbf{(ख) अनित्य‚स्याप्य‚प्र‚माण‚ता}
	\end{center}
	

	  \pstart \leavevmode% starting standard par
	स्या\edtext{}{\edlabel{pvv.11-1}\label{pvv.11-1}\lemma{स्या}\Bfootnote{यो य‚त्साध‚न‚म‚विप‚रीत‚म‚नुतिष्ठ‚ति त‚स्य त‚त्प्राप्तिर्भ‚व‚ति । य‚थातुर‚स्यारोग्य‚{\tiny $_{lb}$}‚साध‚न‚म‚विप‚रीत‚म‚नुतिष्ठ‚तः ।}} देत‚त् (।) स‚न्तान‚विशेषान्निस‚र्ग‚सिद्धांप‚राप‚र‚क्ष‚णात्म‚कं साध‚नापेक्षा\edtext{}{\edlabel{pvv.11-2}\label{pvv.11-2}\lemma{नापेक्षा}\Bfootnote{ईश्व‚र‚ज्ञानं ।}} शून्य‚{\tiny $_{lb}$}‚म‚नित्य‚मेव प्र‚माणं भ‚विष्य‚तीत्याह ।
	\pend% ending standard par
      
	  \bigskip
	  \begingroup
	
	    \large
	  
	    \begin{quote}
	  
	    
	    \stanza[\smallbreak]
	\label{pv.1.11c}\flagstanza{\tiny\textenglish{....1.11c}}अनित्येप्य‚प्र‚माण‚ता ॥११ ॥\&[\smallbreak]


	
	    \end{quote}
	  
	  \endgroup
	

	  \pstart \leavevmode% starting standard par
	\hphantom{.}‚{\color{DodgerBlue3}‚अनित्येपि अपि}‚श‚ब्दान्नित्येप्य‚{\color{DodgerBlue3}‚प्र‚माण‚ता} साध‚नाभाव इत्य‚र्थः ॥ (११)
	\pend% ending standard par
      
	  
	% new div opening: depth here is 2
	

	  \pstart \leavevmode% starting standard par
	न‚नु स‚न्त्येव साध‚नानि य‚था स्थित्वा प्र‚वृत्तेः संस्थान‚विशेषाद‚र्थ‚क्रियासाध‚न‚त्वात्‚{\tiny $_{4}$}‚ ‚{\tiny $_{lb}$}‚कार्य‚त्वादेश्च विम‚त्य‚धिक‚र‚णानि त‚नुभुव‚न‚क‚र\edtext{}{\edlabel{pvv.11-3}\label{pvv.11-3}\lemma{र}\Bfootnote{इन्द्रिय‚श‚रीरादीनां कालान्त‚रं स्थित्वा प्र‚वृत्तेः कार्य‚हेतुषु ।}}णादीन्युपादानाद्य‚भिज्ञ‚बुद्धिम‚त्पूर्व्व‚{\tiny $_{lb}$}‚काणि तुर्य्यादिव‚त् प्रासादादिव‚त् वास्यादिव‚त् घ‚टादिव‚च्चे\edtext{}{\edlabel{pvv.11-4}\label{pvv.11-4}\lemma{च्चे}\Bfootnote{वैध‚र्म्येणात्य‚न्ताभावः ।}} त्येव‚मादीनि । स्थि‚{\tiny $_{lb}$}‚त्वा प्र‚वृत्त्याद‚यः । तुरीत‚न्त्वादिषूपादानाद्य‚भिज्ञ‚बुद्धिम‚त्पूर्व्व‚क‚त्व‚मात्रेणोप‚ल‚ब्ध‚व्याप्त‚{\tiny $_{lb}$}‚योधिक‚र‚ण‚सिद्धा\edtext{}{\edlabel{pvv.11-5}\label{pvv.11-5}\lemma{सिद्धा}\Bfootnote{य‚त्प्र‚सिद्धाव‚न्य‚प्र‚क‚र‚ण‚सिद्धिः सोऽधिक‚र‚ण‚सिद्धान्तः । (न्या॰ सू॰ १।१।३०) ‚{\tiny $_{lb}$}‚य‚था सांख्य‚स्यान्त‚राभ‚व‚निषेधे आत्मैव स‚ञ्च‚र‚त्य‚श‚रीर इति अप‚रीक्षिताभ्युप‚ग‚मात् ॥}}न्त‚न्यायेन नित्य‚व्यापिस‚र्व्व‚ज्ञ‚नित्य‚बुद्ध्याश्र‚यात्म‚विशेष‚विशिष्ट‚मेव ‚{\tiny $_{lb}$}‚प‚क्षे साध्य‚मुप‚न‚य‚न्ति धूम इव प‚र्व‚त‚व‚र्त्तिनं द‚ह‚नं । न ह्य‚नित्येनाव्यापिना वा‚{\tiny $_{5}$}‚ नाना‚{\tiny $_{lb}$}‚देश‚काल‚कार्य‚जातं श‚क्य‚क्रियं । नापि स‚र्व्व‚स्य कार्य‚स्योपादान‚कार‚णान्य‚स‚र्व्व‚विद् वेदितुं ‚{\tiny $_{lb}$}‚स‚म‚र्थः । नाप्य‚नित्य‚या बुद्ध्याऽतीतानाग‚त‚काल‚व‚र्त्ति व‚स्तुजातं ज्ञातुं श‚क्यं(।) त‚द्वेद‚{\tiny $_{lb}$}‚नाच्च स‚र्व्व‚वित्(।) न चैते हेत‚वोऽसिद्धा ध‚र्मिणि स‚त्त्व‚निश्च‚यात् । न च विरुद्धाः ‚{\tiny $_{lb}$}‚स‚प‚क्षे स‚त्त्वात्(।) नानैकान्तिकाः साध्य‚साध‚न‚योर्व्याप्तिनिश्च‚यात् । न च कालात्य‚{\tiny $_{lb}$}‚याप‚दिष्टाः साध्य‚साध‚न‚बाध‚नाभावात्(।) नापि प्र‚क‚र‚ण‚स‚मा विप‚र्य‚य‚साध‚क‚हेत्व‚{\tiny $_{lb}$}‚भावा‚{\tiny $_{6}$}‚दिति । अत्राह ।
	\pend% ending standard par
      \label{div_pvv.1.12}
	  
	% new div opening: depth here is 2
	

	  \begin{center}%% label @type='head'
	\textbf{(ख. ईश्व‚र‚दूष‚ण‚म्)}
	\end{center}
	

	  \begin{center}%% label @type='head'
	\textbf{(क) स‚न्निवेश‚मात्रान्नेश्व‚रानुमान‚म्}
	\end{center}
	
	  \bigskip
	  \begingroup
	
	    \large
	  
	    \begin{quote}
	  
	    
	    \stanza[\smallbreak]
	\label{pv.1.12a}\flagstanza{\tiny\textenglish{....1.12a}}स्थित्वा प्र‚वृत्तिः संस्थान‚विशेषार्थ‚क्रियादिषु ।&इष्ट‚सिद्धिः;\&[\smallbreak]


	
	    \end{quote}
	  
	  \endgroup
	\textsuperscript{\textenglish{012/s}}

	  \pstart \leavevmode% starting standard par
	\hphantom{.}‚{\color{DodgerBlue3}‚स्थित्वा प्र‚वृ\edtext{}{\edlabel{pvv.12-1}\label{pvv.12-1}\lemma{वृ}\Bfootnote{उप‚द‚र्शित‚म‚र्थ प्राप्तुकामा काय‚वाग्‏व्यापार‚स‚हाया बुद्धिः प्र‚वृत्तिः ।}}त्तिः-संस्थान‚विशेषार्थ‚क्रियार्थ‚क्रियादिषु} । साध‚नेषूपादानाद्य‚भिज्ञ‚{\tiny $_{lb}$}‚बुद्धिम‚त्पूर्व्व‚क‚त्वे साध्ये ‚{\color{DodgerBlue3}‚इष्ट‚सिद्धि}‚र‚स्माकं । व‚य‚म‚पि साधार‚णासाधार‚ण‚चेत‚ना‚{\tiny $_{lb}$}‚ल‚क्ष‚ण‚क‚र्म‚निर्मितं ज‚ग‚द्विचित्र‚मिच्छामः(।) त‚त्साध‚य‚ता च प‚रेण साहाय्य‚क‚म‚नुष्ठितं । ‚{\tiny $_{lb}$}‚येन साध्य‚ग‚तेन विशेषेण विना ध‚र्मिणि लिङ्ग‚म‚नुप‚प‚न्नं त‚स्यैवाधिक‚र‚ण‚सिद्धान्तेन ‚{\tiny $_{lb}$}‚\leavevmode\ledsidenote{\textenglish{4a/MA}} प्र‚तीतिर्य‚था प‚र्व्व‚त‚व‚र्त्तिनो धूमाद्व‚ह्नेः प‚र्व्व‚{\tiny $_{7}$}‚त‚व‚र्त्तित्व‚स्य(।) न ह्य‚न्य‚देश‚स्थेनाग्निना ‚{\tiny $_{lb}$}‚ज‚न्य‚मान‚स्य धूम‚स्य प‚र्व‚त‚व‚र्त्तित्व‚मुप‚प‚द्य‚ते(।) न‚त्वेवं नित्य‚त्व‚स‚र्व्व‚ज्ञ‚त्व‚व्यापित्वादि ‚{\tiny $_{lb}$}‚विना चेत‚न‚स्य त‚नुभुव‚नादिग‚तं स्थित्वा प्र‚वृत्त्यादिक‚म‚नुप‚प‚न्नं(।) नित्य‚त्वादिविप‚र्य‚य‚{\tiny $_{lb}$}‚योगिनापि चेत‚नेन क्रिय‚माणं त‚द् घ‚ट‚त एव ॥
	\pend% ending standard par
      

	  \pstart \leavevmode% starting standard par
	न‚नूक्त‚मेवानित्याव्यापिनः स‚र्व्व‚देश‚काल‚व‚र्त्ति कार्य‚म‚कार्यं अस‚र्व्व‚विद‚श्च ‚{\tiny $_{lb}$}‚स‚र्व्वोपादानाद्य‚भिज्ञ‚ता नास्तीत्यादिना । स‚त्य‚मुक्त‚म‚युक्त‚न्तूक्तं क‚र्त्तुरेक‚त्वासि‚{\tiny $_{1}$}‚द्धेः । ‚{\tiny $_{lb}$}‚एक‚स्य क‚र्त्तुर‚नेक‚देश‚कालं कार्यं कुर्व्वाण‚स्य त‚दुपादानादिकं जानान‚स्य नित्य‚त्वा‚{\tiny $_{lb}$}‚दिक‚म‚न्त‚रेण स्थित्वा प्र‚वृत्त्यादि नोप‚प‚द्य‚त इति स्याद‚पि तादृश‚स्याधिक‚र‚ण‚{\tiny $_{lb}$}‚सिद्धान्तेन प्र‚तिप‚त्तिः । अनेकेनापि तु नानादेश‚काल‚व‚र्त्तिना स्व‚स्व‚कार्य‚स्योपादा‚{\tiny $_{lb}$}‚नादिजान‚ता क्रिय‚माणं स्थित्वा प्र‚वृत्त्यादिस‚ङ्ग‚त‚मेवात एवैक‚त्व‚स्यापि त‚त एव ‚{\tiny $_{lb}$}‚सिद्धिर‚युक्ताऽनेक‚स्यापि हेतुत्व‚योगात् । प‚क्षायो‚{\tiny $_{2}$}‚ग‚व्य‚व‚च्छेद एव तु साध्य‚स्याधि‚{\tiny $_{lb}$}‚क‚र‚ण‚सिद्धान्तेन सिध्य‚तु न तु त‚द‚धिको व‚ह्नेरिव चान्द‚न‚त्वादिः ॥
	\pend% ending standard par
      

	  \pstart \leavevmode% starting standard par
	न‚न्व‚नेक एव ते चेत‚नाव‚न्तोभिम‚ताः\edtext{}{\edlabel{pvv.12-2}\label{pvv.12-2}\lemma{ताः}\Bfootnote{बौद्धेन ।}}(।) न चोपादानादिकं त‚न्वादीनां ते ‚{\tiny $_{lb}$}‚जान‚ते । नापि त‚त्क‚र्तृ त्व‚मात्म‚नो म‚न्य‚न्ते त‚त्क‚थ‚म‚मी क‚र्त्तारः । \edtext{\textsuperscript{*}}{\edlabel{pvv.12-3}\label{pvv.12-3}\lemma{*}\Bfootnote{ईश्व‚रादिकं विशेषं त्य‚क्त्वा सामान्येन चेत‚नामात्र‚पूर्व्व‚त्वं बौद्धेनेष्ट‚सिद्धिरुक्ता न युक्ता ।}}अथाधिप‚त्य‚मात्रे‚{\tiny $_{lb}$}‚णैषां क‚र्त्तृत्वं न तूप‚क‚र‚णाद्यायोज‚नेन व्यापारेण य‚था च‚न्द्र‚स्य च‚न्द्र‚कान्त‚द्रुतौ । ‚{\tiny $_{lb}$}‚तेनोपादानाद्य‚भिज्ञ‚ता क‚र्तृत्वाभि‚{\tiny $_{3}$}‚मान‚श्च नैषां । एव‚न्त‚र्ह्युपादानाद्य‚भिज्ञ‚चेत‚न‚{\tiny $_{lb}$}‚पूर्व्व‚क‚त्वे साध्ये नेष्ट\edtext{}{\edlabel{pvv.12-4}\label{pvv.12-4}\lemma{नेष्ट}\Bfootnote{बौ[द्धः] प्राह ।}} सिद्धिर्व्व‚क्त‚व्या । य‚द्येवं भ‚व‚त एव सूक्ष्मेक्षिका, त‚दा ‚{\tiny $_{lb}$}‚चेत‚नापूर्व्व‚क‚त्व‚मात्रं साध‚य, दृष्टान्ते च कुम्भ‚कार‚त्व‚व‚त् स‚द‚प्युपादानाद्य‚भिज्ञ‚त्व‚म‚{\tiny $_{lb}$}‚प्र‚योज‚कं म‚न्य‚स्व ॥
	\pend% ending standard par
      

	  \pstart \leavevmode% starting standard par
	स‚त्क‚थं प‚रिह‚र्त्त‚व्य‚मिति चेत्(।) त‚त्किं\edtext{}{\edlabel{pvv.12-5}\label{pvv.12-5}\lemma{त्किं}\Bfootnote{साध्ये ।}} कुम्भ‚कार‚त्व‚म‚प‚रिहार्य । \edtext{\textsuperscript{*}}{\edlabel{pvv.12-6}\label{pvv.12-6}\lemma{*}\Bfootnote{ईश्व‚रं साध‚य‚तः}}अप्र‚योज‚क‚{\tiny $_{lb}$}‚त्वाद् त्य‚ज्य‚त इति चेत् उपादानाद्य‚भि\edtext{}{\edlabel{pvv.12-7}\label{pvv.12-7}\lemma{भि}\Bfootnote{कुम्भ‚कृच्चेत् युज्य‚ते । ईश्व‚रोप्य‚त्र न प्र‚योज‚को दृष्टः इति त्य‚ज्य‚तां ।}}ज्ञ‚त्वेपि स‚मान‚मेत‚त्(।) उपादानादिक‚म‚{\tiny $_{lb}$}‚जान‚तोपि बुद्धिम‚तोऽनेक‚स्याधि‚{\tiny $_{4}$}‚ष्ठा\edtext{}{\edlabel{pvv.12-8}\label{pvv.12-8}\lemma{ष्ठा}\Bfootnote{म‚यूर‚च‚न्द्रिकोपादानं म‚यूरो न वेत्य‚र्थे च त‚द‚धिष्ठानाच्च‚न्द्रिकोत्प‚द्य‚ते ।}}न‚मात्रेणापि कार्य्याणामुत्प‚त्तियोगात् ।
	\pend% ending standard par
      \textsuperscript{\textenglish{013/s}}
	  \bigskip
	  \begingroup
	
	    \large
	  
	    \begin{quote}
	  
	    
	    \stanza[\smallbreak]
	\label{pv.1.12b}\flagstanza{\tiny\textenglish{....1.12b}}असिद्धिर्वा दृष्टान्ते संश‚योऽथ‚वा ॥ १२ ॥\&[\smallbreak]


	
	    \end{quote}
	  
	  \endgroup
	

	  \pstart \leavevmode% starting standard par
	\hphantom{.}अथ नित्य‚त्वादिविशिष्ट‚पुरुष‚पूर्व्व‚क‚त्व‚मेव साध्यं त‚दाऽ‚{\color{DodgerBlue3}‚सिद्धिर्दृष्टान्ते} । साध्य‚{\tiny $_{lb}$}‚शून्यो दृष्टान्त इत्य‚र्थः । न हि क्व‚चिद् दृष्टान्ते तादृशं साध्य‚मुप‚ल‚ब्धं येन व्याप्तिः ‚{\tiny $_{lb}$}‚प्र‚तीयेत(।) असिद्ध‚व्याप्तिक‚श्च हेतुर‚नैकान्तिक एव । स्थित्वा प्र‚वृत्तेर‚र्थ‚क्रियास‚म‚र्थ‚{\tiny $_{lb}$}‚त्वादिति हेत्वोः ‚{\color{DodgerBlue3}‚संश\edtext{}{\edlabel{pvv.13-1}\label{pvv.13-1}\lemma{संश}\Bfootnote{स‚न्दिग्धासिद्धोयं य‚थाग्निज‚न्य‚त्वेनानिश्चितो धूमः ।}}योऽथ‚वाऽ}‚नैकान्तिक‚त्वं तेनैव त्व‚भि\edtext{}{\edlabel{pvv.13-2}\label{pvv.13-2}\lemma{भि}\Bfootnote{स य‚थाऽन्यान‚पेक्षः क‚रोत्येवं दृष्ट‚कार‚णेपि स्यात् स्थित्वा स्थित्वा कार्यं ‚{\tiny $_{lb}$}‚प्र‚व‚र्त‚य‚ति ।}}म‚त‚पुरुषेण । न ह्य‚सौ ‚{\tiny $_{lb}$}‚स्थि\edtext{}{\edlabel{pvv.13-3}\label{pvv.13-3}\lemma{स्थि}\Bfootnote{संश‚य‚श्चात्र दृष्ट‚कार‚ण‚स्य त‚द‚पेक्ष‚त्वे सोपि त‚था स्यात् । न च त‚था । ‚{\tiny $_{lb}$}‚व...नामेकः प्र‚धानं म‚न्दाश्च‚र(?)भात्य‚त्र । एकाभावेप्य‚भावा कार्य‚स्य गुण‚प्र‚धा‚{\tiny $_{lb}$}‚न‚भाव इति कृत्वा स्थित्वेत्यादि उप‚योगिनां ज्ञान‚स्य प्र‚धान‚त्वानिश्च‚यात् संश‚यः ।}}त्वा प्र‚व‚र्त‚मानः कार्य्येष्व‚र्थ‚{\tiny $_{5}$}‚ क्रिया\edtext{}{\edlabel{pvv.13-4}\label{pvv.13-4}\lemma{क्रिया}\Bfootnote{क्र‚म‚क‚र‚ण‚म‚पि न नित्य‚त्वादेव ।}}कारी वा त‚था पुरुषाधिष्ठितः । अन‚व‚स्था‚{\tiny $_{lb}$}‚प्र‚स‚ङ्गात् ।(१२)
	\pend% ending standard par
      \label{div_pvv.1.13}
	  
	% new div opening: depth here is 2
	
	  \bigskip
	  \begingroup
	
	    \large
	  
	    \begin{quote}
	  
	    
	    \stanza[\smallbreak]
	\label{pv.1.13a}\flagstanza{\tiny\textenglish{....1.13a}}सिद्धं यादृग‚धिष्ठातृभावाभावानुवृत्तिम‚त् ।&स‚न्निवेशादि;\&[\smallbreak]


	
	    \end{quote}
	  
	  \endgroup
	

	  \pstart \leavevmode% starting standard par
	\hphantom{.}अथ‚वा ‚{\color{DodgerBlue3}‚या\edtext{}{\edlabel{pvv.13-5}\label{pvv.13-5}\lemma{या}\Bfootnote{अर्थ‚क्रियाऽक‚र‚णेपि न व‚स्तुत्व‚म‚न‚व ‚{\tiny $_{lb}$}‚स्थातः । स्थित्वा प्र‚वृत्तिर‚पि न नित्य‚त्वादिति नित्य‚प‚क्षे ।}}दृ}‚श‚बुद्धिम‚त्पूर्व्व‚कं ‚{\color{DodgerBlue3}‚स‚न्निवेशादि} दृष्टान्ते दृष्टं त‚स्य ध‚र्मिण्य‚सिद्धि‚{\tiny $_{lb}$}‚रित्याह । घ‚टे दृष्टान्त‚ध‚र्मिणि ‚{\color{DodgerBlue3}‚स‚न्निवेशादि यादृशं} पृथुबुध्नोद‚रादि दृष्टैक‚व्याक्ति‚{\tiny $_{lb}$}‚जात्या विशेषित‚म‚धि‚{\color{DodgerBlue3}‚ष्ठातुः} पुंसोऽन्व‚य‚व्य‚तिरेकानुविधान‚व‚त् व्य‚व‚हार‚प्र‚ग‚ल्भ‚पुरुषाणां ‚{\tiny $_{lb}$}‚त‚त्सिद्धान्तानुरोध‚र‚हितानां प्र‚त्य‚क्ष‚ब‚लेनानुरूप‚निश्च‚योत्पा‚{\color{DodgerBlue3}‚दात्सिद्धं} निश्चितं ॥
	\pend% ending standard par
      
	  \bigskip
	  \begingroup
	
	    \large
	  
	    \begin{quote}
	  
	    
	    \stanza[\smallbreak]
	\label{pv.1.13c}\flagstanza{\tiny\textenglish{....1.13c}}त‚द्युक्तं त‚स्माद् य‚द‚नुमीय‚ते ॥१३ ॥\&[\smallbreak]


	
	    \end{quote}
	  
	  \endgroup
	

	  \pstart \leavevmode% starting standard par
	त‚स्मात्स‚न्निवेशाद‚प‚र‚{\tiny $_{6}$}‚त्रानुप‚ल‚ब्ध‚पुरुष‚ज‚न्म‚नि घ‚टे य‚द् बुद्धिम‚द‚धिष्ठा‚{\tiny $_{lb}$}‚न‚म‚{\color{DodgerBlue3}‚नुमीय‚ते त‚द्युक्तं} त‚स्यैव पुरुष‚कार्य‚त्वेन निश्च‚यात् ॥ \edtext{\textsuperscript{*}}{\edlabel{pvv.13-6}\label{pvv.13-6}\lemma{*}\Bfootnote{न पुन‚र्व‚स्तुभेदे ।}}अस‚न्निवेश‚व्यावृत्तं स‚न्नि‚{\tiny $_{lb}$}‚वेश‚मात्रं तु स‚द‚पि न त‚त्कार्य‚त‚या प्र‚त्य‚क्ष‚मुप‚स्थाप‚य‚ति । प्र‚त्य‚क्ष‚व्यापार‚विवादे च ‚{\tiny $_{lb}$}‚प‚टुप्र‚चारा व्य‚व‚हारिणः श‚र‚णं (।) न हि क‚श्चिद् व्य‚व‚हारी घ‚टं पुरुष‚कृतं प‚श्य‚न् ‚{\tiny $_{lb}$}‚श‚रावादि प‚र्व्व‚तादिक‚म्वा त‚त्कृत‚म‚व‚धार‚य‚ति (।) य‚दा तु श‚रावादीन‚पि त‚त ‚{\tiny $_{lb}$}‚उद‚य‚मासाद‚य‚तः‚{\tiny $_{7}$}‚ प‚श्य‚ति त‚दा तान‚पि त‚त्कृतान‚वैति (।) अतः स‚न्निवेश‚विशेषं\leavevmode\ledsidenote{\textenglish{4b/MA}} ‚{\tiny $_{lb}$}‚पुरुष‚कार्यं दृष्ट‚व‚तः स‚न्निवेश‚मात्रात्त‚द‚नुमान‚म‚युक्तं (। १३)
	\pend% ending standard par
      \label{div_pvv.1.14}
	  
	% new div opening: depth here is 2
	\textsuperscript{\textenglish{014/s}}

	  \pstart \leavevmode% starting standard par
	एत‚देवाह (।)
	\pend% ending standard par
      
	  \bigskip
	  \begingroup
	
	    \large
	  
	    \begin{quote}
	  
	    
	    \stanza[\smallbreak]
	\label{pv.1.14}\flagstanza{\tiny\textenglish{...v.1.14}}व‚स्तुभेदे प्र‚सिद्ध‚स्य श‚ब्द‚साम्याद‚भेदिनः ।&न युक्तानुमितिः पाण्डुद्र‚व्यादिव‚द् हुताश‚ने ॥ १४ ॥\&[\smallbreak]


	
	    \end{quote}
	  
	  \endgroup
	

	  \pstart \leavevmode% starting standard par
	\hphantom{.}‚{\color{DodgerBlue3}‚व‚स्तुभेदे} घ‚टे ‚{\color{DodgerBlue3}‚प्र‚सिद्ध‚स्य} पुरुष‚पूर्व्व‚क‚त्व‚स्य स‚न्निवेश इति ‚{\color{DodgerBlue3}‚श‚ब्द‚साम्याद‚भेदिनः} स‚न्निवेश‚मात्रात् प‚र्व्व‚तादौ ‚{\color{DodgerBlue3}‚न युक्तानुमितिः पाण्डुद्र‚व्यादिव‚द् हुताश‚ने} । य‚था ‚{\tiny $_{lb}$}‚पाण्डुविशेष‚स्य धूम‚स्य कार‚ण‚त्वेन दृष्टे व‚ह्नौ पाण्डुश‚ब्द‚साम्याद‚भेदिनो य‚तः कुत‚{\tiny $_{lb}$}‚श्चिद् पाण्डुद्र‚व्याद् धूमादेर‚नुमान‚{\tiny $_{1}$}‚म‚नुचित‚म‚तो य‚त्त‚द् बुद्धिम‚द्‏व्याप्तं स‚न्निवेशा\edtext{}{\edlabel{pvv.14-1}\label{pvv.14-1}\lemma{न्निवेशा}\Bfootnote{ध‚टादेः ।}} दि ‚{\tiny $_{lb}$}‚त\edtext{}{\edlabel{pvv.14-2}\label{pvv.14-2}\lemma{त}\Bfootnote{प‚र्व्व‚तादौ ।}}द्ध‚र्भिणि नास्तीत्य‚सिद्धिर्हेतूनां ॥ (१४)
	\pend% ending standard par
      \label{div_pvv.1.15}
	  
	% new div opening: depth here is 2
	

	  \pstart \leavevmode% starting standard par
	न‚नु स‚न्निवेशादिसामान्याद‚स‚न्निवेशादिव्यावृत्त्या हेत‚वो भ‚विष्य‚न्तीत्य‚त आह (।)
	\pend% ending standard par
      
	  \bigskip
	  \begingroup
	
	    \large
	  
	    \begin{quote}
	  
	    
	    \stanza[\smallbreak]
	\label{pv.1.15}\flagstanza{\tiny\textenglish{...v.1.15}}अन्य‚था कुम्भ‚कारेण मृद्विकार‚स्य क‚स्य‚चित् ।&घ‚टादेः क‚र‚णात् सिध्येद् व‚ल्मीक‚स्यापि त‚त्कृतिः ॥ १५ ॥\&[\smallbreak]


	
	    \end{quote}
	  
	  \endgroup
	

	  \pstart \leavevmode% starting standard par
	\hphantom{.}‚{\color{DodgerBlue3}‚अन्य‚था} य‚दि साध्य‚व्याप्तं विशेषं त्य‚क्त्वा सामान्यं लिङ्गं क्रिय‚ते त‚दा ‚{\color{DodgerBlue3}‚कुम्भ‚{\tiny $_{lb}$}‚कारेण क‚स्य‚चिद् घ‚टादेर्मृद्विकार‚स्य क‚र‚णाद्व‚ल्मीक‚स्यापि} मृद्वि\edtext{}{\edlabel{pvv.14-3}\label{pvv.14-3}\lemma{मृद्वि}\Bfootnote{घ‚ट‚व‚त् ।}}कार‚त्वा‚{\color{DodgerBlue3}‚त्तेन} कुम्भ\edtext{}{\edlabel{pvv.14-4}\label{pvv.14-4}\lemma{कुम्भ}\Bfootnote{घ‚ट‚स्य कृत‚कृत्वं श‚ब्दे नास्तीति य‚थाकार्यें जात्युत्त‚रं त‚थेद‚मिति चेत् ।}}‚{\tiny $_{lb}$}‚कारेण ‚{\color{DodgerBlue3}‚कृतिः} क‚र‚णं ‚{\color{DodgerBlue3}‚सिध्येत।} । (१५)
	\pend% ending standard par
      \label{div_pvv.1.16}
	  
	% new div opening: depth here is 2
	

	  \pstart \leavevmode% starting standard par
	स्यादेत‚त् (।) न मृद्विकार‚त्वं कुम्भ‚कार‚व्याप्तं त‚म‚न्त‚रेणापि व‚ल्मी\edtext{}{\edlabel{pvv.14-2-bis}\label{pvv.14-2-bis}\lemma{ल्मी}\Bfootnote{प‚र्व्व‚तादौ ।}}क‚स्योत्प‚त्तेः । ‚{\tiny $_{lb}$}‚य‚द्ये\edtext{}{\edlabel{pvv.14-5}\label{pvv.14-5}\lemma{द्ये}\Bfootnote{व‚ह्नेः ।}}वं बुद्धिम‚न्त‚म‚न्त‚रेणैत‚त्त‚न्वादि जाय‚त इति त‚द‚पि प‚क्षो न स्यात् । अथ प‚क्षेण न ‚{\tiny $_{lb}$}‚व्य‚भिचारः व‚ल्मीकेप्य‚यं न्यायः स‚मानः (।) त‚स्मात्स‚न्निवेशादिसामान्यं व्याप्त्य‚{\tiny $_{lb}$}‚सिद्धेर‚नैकान्तिक‚मेव ॥\edtext{\textsuperscript{*}}{\edlabel{pvv.14-6}\label{pvv.14-6}\lemma{*}\Bfootnote{न‚न्वेत‚त् कार्य‚स‚मं सामान्ये हेतौ विशेष‚प‚रिक‚ल्प‚नादिति ॥}} न चैत‚त्कार्य‚स‚मं दूष‚णं य‚तः (।)
	\pend% ending standard par
      
	  \bigskip
	  \begingroup
	
	    \large
	  
	    \begin{quote}
	  
	    
	    \stanza[\smallbreak]
	\label{pv.1.16}\flagstanza{\tiny\textenglish{...v.1.16}}साध्येनानुग‚मात् कार्ये सामान्येनापि साध‚ने ।&स‚म्ब‚न्धिभेदाद् भेदोक्तिदोषः कार्य‚स‚मो म‚तः ॥ १६ ॥\&[\smallbreak]


	
	    \end{quote}
	  
	  \endgroup
	

	  \pstart \leavevmode% starting standard par
	\hphantom{.}‚{\color{DodgerBlue3}‚साध्येनानि}‚त्य‚त्वेना‚{\color{DodgerBlue3}‚नुग‚माद्} व्याप‚नात् ‚{\color{DodgerBlue3}‚कार्ये} कृत‚क‚त्वे ‚{\color{DodgerBlue3}‚सामान्येनापि साध‚ने} कृते ‚{\tiny $_{lb}$}‚‚{\color{DodgerBlue3}‚स‚म्ब‚धिनोः} साध्य‚ध‚र्मिदृष्टान्त‚ध‚र्मिणो‚{\color{DodgerBlue3}‚र्भेदात्} स‚म्ब‚द्ध‚स्य साध‚न‚स्य‚{\tiny $_{3}$}‚ ‚{\color{DodgerBlue3}‚भेदोक्त्या} य‚दि ‚{\tiny $_{lb}$}‚साध्य‚ध‚र्मिग‚तं कार्य‚त्वं हेतुस्त‚दा नान्व‚य‚सिद्धिर‚थ दृष्टान्त‚ग‚तं त‚दाऽसिद्धो हेतुरिति ‚{\tiny $_{lb}$}‚यो दोषः ‚{\color{DodgerBlue3}‚स कार्य‚स‚मो म‚तः} ॥ (१६)
	\pend% ending standard par
      \label{div_pvv.1.17}
	  
	% new div opening: depth here is 2
	

	  \pstart \leavevmode% starting standard par
	अत्र तु स‚न्निवेशादिसामान्यं न साध्य‚व्या\edtext{}{\edlabel{pvv.14-7}\label{pvv.14-7}\lemma{व्या}\Bfootnote{अविशिष्ट‚संस्थान‚सामान्य‚स्यैवाभावात् ।}}प्तं सिद्ध‚मिति विशेषेऽसिद्ध‚त्वं सामान्ये ‚{\tiny $_{lb}$}‚\leavevmode\ledsidenote{\textenglish{015/s}} चानैकान्तिक‚दू\edtext{}{\edlabel{pvv.15-1}\label{pvv.15-1}\lemma{दू}\Bfootnote{कुम्भ‚कारादिनैव कृत‚त्व‚स‚म्भाव‚न‚या ।}}ष‚णं न जात्युत्त‚रं ॥
	\pend% ending standard par
      

	  \pstart \leavevmode% starting standard par
	\hphantom{.}न‚नु स‚न्निवेशादिश‚ब्द‚वाच्योऽर्थः ‚{\color{DodgerBlue3}‚कुम्भे बुद्धिम‚द्‏व्याप्तः प्र‚तीतः स च त‚न्वा-} दिष्व‚पि दृश्य‚ते त‚तो विशेषे विक‚ल्पो न युक्त इत्याह (।)
	\pend% ending standard par
      
	  \bigskip
	  \begingroup
	
	    \large
	  
	    \begin{quote}
	  
	    
	    \stanza[\smallbreak]
	\label{pv.1.17}\flagstanza{\tiny\textenglish{...v.1.17}}जात्य‚न्त‚रे प्र‚सिद्ध‚स्य श‚ब्द‚सामान्य‚द‚र्श‚नात् ।&न युक्तं साध‚नं गोत्वाद् वागादीनां विषाणिव‚त् ॥ १७ ॥\&[\smallbreak]


	
	    \end{quote}
	  
	  \endgroup
	

	  \pstart \leavevmode% starting standard par
	\hphantom{.}‚{\color{DodgerBlue3}‚जात्य‚न्त‚रे‚{\tiny $_{4}$}‚} जातिविशेषे घ‚ट‚स‚न्निवेशादौ ‚{\color{DodgerBlue3}‚प्र‚सिद्ध‚स्य} साध्य‚स्य बुद्धिम‚त्पूर्व्व‚{\tiny $_{lb}$}‚क‚त्व‚स्य तं विशेषं प‚रित्य‚ज्य ‚{\color{DodgerBlue3}‚श‚ब्द‚सामान्य‚द‚र्श‚नाद्} स‚न्निवेशादिमात्रेण ‚{\color{DodgerBlue3}‚न युक्तं साध‚नं} वागादीनां ‚{\color{DodgerBlue3}‚गोत्वात्} गोश‚ब्द‚वाच्य‚त्वाद्विषाणिव‚त् विषाणित्व‚स्येव न युक्तं साध‚नं ‚{\tiny $_{lb}$}‚विशिष्ट‚जातेरेव विषाण‚व्याप्त्युप‚ल‚ब्धेः । (१७)
	\pend% ending standard par
      \label{div_pvv.1.18}
	  
	% new div opening: depth here is 2
	

	  \pstart \leavevmode% starting standard par
	किञ्च (।)
	\pend% ending standard par
      
	  \bigskip
	  \begingroup
	
	    \large
	  
	    \begin{quote}
	  
	    
	    \stanza[\smallbreak]
	\label{pv.1.18}\flagstanza{\tiny\textenglish{...v.1.18}}विव‚क्षाप‚र‚त‚न्त्र‚त्वान्न श‚ब्दाः स‚न्ति कुत्र वा ।&त‚द्‏भावाद्; अर्थ‚सिद्धौ तु स‚र्वं स‚र्व‚स्य सिध्य‚ति ॥ १८ ॥\&[\smallbreak]


	
	    \end{quote}
	  
	  \endgroup
	

	  \pstart \leavevmode% starting standard par
	\hphantom{.}‚{\color{DodgerBlue3}‚विव‚क्षाप‚र‚त‚न्त्र‚त्वान्न श\edtext{}{\edlabel{pvv.15-2}\label{pvv.15-2}\lemma{श}\Bfootnote{य‚द्य‚पि न प्र‚व‚र्त्तितः श‚ब्दः । त‚थापि य‚दैव प्र‚व‚र्त्त्य‚ते त‚दैव भ‚व‚तीति ।}}ब्दाः स‚न्ति कुत्र वा} स‚र्व्व‚त्रैव स‚न्ति । त‚स्य श‚ब्द‚स्य ‚{\tiny $_{lb}$}‚‚{\color{DodgerBlue3}‚भावात् । अर्थ‚सिद्धौ} स‚त्यां ‚{\color{DodgerBlue3}‚स‚{\tiny $_{5}$}‚र्व्व} य‚थास‚मीहितं साध्यं ‚{\color{DodgerBlue3}‚स‚र्व्व‚स्य} पुंसः ‚{\color{DodgerBlue3}‚सिध्य‚ति} उक्त‚मीश्व‚र‚साध‚न‚स्य दूष‚णं ॥ (१८)
	\pend% ending standard par
      \label{div_pvv.1.19}
	  
	% new div opening: depth here is 2
	

	  \pstart \leavevmode% starting standard par
	अमुमेव न्याय‚म‚न्य‚त्राप्य‚तिदिश‚न्नाह ।
	\pend% ending standard par
      
	  \bigskip
	  \begingroup
	
	    \large
	  
	    \begin{quote}
	  
	    
	    \stanza[\smallbreak]
	\label{pv.1.19}\flagstanza{\tiny\textenglish{...v.1.19}}एतेन का पि ला दीनाम‚चैत‚न्यादि चिन्तित‚म् ।&अनित्यादेश्च चैत‚न्य म‚र‚णात् त्व‚ग‚पोह‚तः ॥ १९ ॥\&[\smallbreak]


	
	    \end{quote}
	  
	  \endgroup
	

	  \pstart \leavevmode% starting standard par
	\hphantom{.}‚{\color{DodgerBlue3}‚एतेन} श‚ब्द‚सामान्य‚मात्र‚स्यार्थ‚शून्य‚स्याहेतुत्व‚क‚थ‚नेन य‚त्का‚{\color{DodgerBlue3}‚पिलादे}‚र्ब्बुद्धिसुखा‚{\color{DodgerBlue3}‚दी‚{\tiny $_{lb}$}‚ना\edtext{}{\edlabel{pvv.15-3}\label{pvv.15-3}\lemma{ना}\Bfootnote{रूपादिव‚त् ।}}म}‚नित्य‚त्वोत्प‚त्तिम‚त्त्वादिहेतुतोऽचेत‚न‚त्व‚मिष्टं त‚था दि ग म्ब रा णां चैत‚न्यं ‚{\tiny $_{lb}$}‚त‚रूणां स‚र्व्व‚स्याः ‚{\color{DodgerBlue3}‚त्व‚चोऽपोह}‚तोऽप‚ग‚ते‚{\color{DodgerBlue3}‚र्म‚र}‚णाद‚भिम‚तं ‚{\color{DodgerBlue3}‚त‚च्चिन्तितं} वेदित‚व्यं (।) य‚था ‚{\tiny $_{lb}$}‚ह्य‚प्र‚च्युत‚प्राच्य‚रूप‚स्य तिरोधा‚{\tiny $_{6}$}‚न‚म‚नित्य‚त्वं सां\edtext{}{\edlabel{pvv.15-4}\label{pvv.15-4}\lemma{सां}\Bfootnote{आस‚र्ग‚प्र‚ल‚यान्नित्यैका बुद्धिर्न वेद‚ना, प्र‚कृतिर्भोग्या भोक्ता पुरुषः सांख्य‚स्य (।) ‚{\tiny $_{lb}$}‚सांख्यः स्व‚स्व‚भावाच्युत‚स्य तिरोधान‚म‚नित्य‚तामाहातिरोधानं बौद्ध‚स्यासिद्धं । निर‚{\tiny $_{lb}$}‚न्व‚य‚नाशः सांख्य‚स्य । आत्म‚नः स‚ञ्च‚र‚न्तो वृक्षाद्य‚व‚स्था भ‚व‚न्तीति क्ष‚प‚णः । ‚{\tiny $_{lb}$}‚अनित्य‚ता सामान्या सिद्धिर्व्विनिश्च‚येस्ति ।}}ख्य स्येष्टं बौ द्ध स्य तु निर‚न्व‚य‚विना‚{\tiny $_{lb}$}‚शित्वं । त‚स्य य‚थाक्र‚म‚मुपादाने प्र‚तिवाद्य‚सिद्ध‚ता वाद्य‚सिद्ध‚ता च ।
	\pend% ending standard par
      \textsuperscript{\textenglish{016/s}}

	  \pstart \leavevmode% starting standard par
	न‚नूभ‚यासिद्ध‚म‚नित्य‚त्व‚म‚स्ति किञ्चिदृते श‚ब्द‚साम्याद् त‚था विज्ञानेन्द्रिया‚{\tiny $_{lb}$}‚युर्निरोध‚ल‚क्ष‚णं म‚र‚ण‚मिष्टं बौद्ध‚सिद्धान्ते त‚स्य च त‚रुष्व\edtext{}{\edlabel{pvv.16-1}\label{pvv.16-1}\lemma{रुष्व}\Bfootnote{त‚त्सिद्धौ चैत‚न्यं सिद्ध‚मिति साध्यं ।}} सिद्धिः । चेत‚न‚त्व‚स्यै\edtext{}{\edlabel{pvv.16-2}\label{pvv.16-2}\lemma{स्यै}\Bfootnote{न शोष‚मात्र‚स्य ।}} व ‚{\tiny $_{lb}$}‚साध्य‚त्वात् । न ह्य‚सिद्धेषु त\edtext{}{\edlabel{pvv.16-3}\label{pvv.16-3}\lemma{त}\Bfootnote{विज्ञानादिषु त‚त्र ।}}न्निरोधो युक्तः शोष\edtext{}{\edlabel{pvv.16-4}\label{pvv.16-4}\lemma{शोष}\Bfootnote{शोषोम‚र‚ण‚मित्य‚पि न व‚र्ण्ण‚वादिनानेकान्तात् । अपि तु य‚च्छोष‚व‚त्त‚च्चेत‚नाव‚न्न सिद्धं विप‚र्य‚येऽबाधात् ॥}}मात्र‚न्तु त‚रुषु म‚र‚ण‚मुप‚चारा‚{\tiny $_{lb}$}‚\leavevmode\ledsidenote{\textenglish{5a/MA}} दुच्य‚ते । य‚दि च म‚र‚ण‚वाच्य‚त्व‚मात्रं हेतुः‚{\tiny $_{7}$}‚ त‚दा तैला\edtext{}{\edlabel{pvv.16-5}\label{pvv.16-5}\lemma{तैला}\Bfootnote{घृत‚हिङ्ग्वादिषु ।}}दिष्व‚पि त‚त्स‚त्त्वात् साधा‚{\tiny $_{lb}$}‚र‚णानैकान्तिक‚ता ॥(१९)
	\pend% ending standard par
      \label{div_pvv.1.20}
	  
	% new div opening: depth here is 2
	

	  \pstart \leavevmode% starting standard par
	न‚न्वेवं कृत‚क‚त्वादिक‚म‚पि हेतु\edtext{}{\edlabel{pvv.16-6}\label{pvv.16-6}\lemma{हेतु}\Bfootnote{वैशेषिकेणापौरुषेय‚श‚ब्द‚निषेधाय मीमांस‚क‚कृतः ।}} र्न्न स्यादाकाश‚गुण‚श्श‚ब्द‚स्य ध‚र्मो बौद्ध‚स्यासिद्धः । ‚{\tiny $_{lb}$}‚अन्य‚था चान्य‚स्येत्य‚त आह ।
	\pend% ending standard par
      
	  \bigskip
	  \begingroup
	
	    \large
	  
	    \begin{quote}
	  
	    
	    \stanza[\smallbreak]
	\label{pv.1.20}\flagstanza{\tiny\textenglish{...v.1.20}}व‚स्तुस्व‚रूपेऽसिद्धेऽयं न्यायः सिद्धे विशेष‚ण‚म् ।&अबाध‚क‚म‚सिद्धाव‚प्याकाशाश्र‚य‚व‚द् ध्व‚नेः ॥ २० ॥\&[\smallbreak]


	
	    \end{quote}
	  
	  \endgroup
	

	  \pstart \leavevmode% starting standard par
	\hphantom{.}‚{\color{DodgerBlue3}‚व‚स्तुस्व‚रूपेऽसिद्धेऽय‚न्न्यायः} य‚थोक्तासिद्धिचोद‚नाल‚क्ष‚णः व‚स्तुस्व‚रूपे तु ‚{\tiny $_{lb}$}‚ध‚र्मिणि हेतौ ‚{\color{DodgerBlue3}‚सिद्धे विशे\edtext{}{\edlabel{pvv.16-7}\label{pvv.16-7}\lemma{विशे}\Bfootnote{व्याप्यो हेतुर्न चायं विशेष‚ण‚व्याप्तः । हेतुवादीष्ट‚ञ्च साध्यं विशेष‚ण‚ञ्च त‚द‚निष्टं ।}} ष‚ण‚म‚सिद्धाव‚प्य‚बाध‚कं} किमिवाका‚{\color{DodgerBlue3}‚शाश्र‚य‚व‚द् ध्व‚नेः} । य‚था ‚{\tiny $_{lb}$}‚श‚ब्द‚स्याकाश‚गुण‚त्वं विशेष‚ण‚म‚पि न बाध‚कं ध‚र्मितायाः कृत‚क‚त्वादिहेतोर्व्वा विशे‚{\tiny $_{lb}$}‚ष‚णासिद्धा‚{\tiny $_{1}$}‚व‚पि हि श‚ब्दो ध‚र्मी प्र‚त्य‚क्ष‚सिद्धः । कृत‚क‚त्वादि चानुमान‚सि\edtext{}{\edlabel{pvv.16-8}\label{pvv.16-8}\lemma{सि}\Bfootnote{यः प्र‚त्य‚य‚भेद‚भेदी स कृत‚कः ।}}द्धं ‚{\tiny $_{lb}$}‚ताव‚तैव च साध्य‚साध‚न‚भावो निर्व्विरोधः । (२०)
	\pend% ending standard par
      \label{div_pvv.1.21_1.22}
	  
	% new div opening: depth here is 2
	

	  \pstart \leavevmode% starting standard par
	य‚त्र श‚ब्दोप्य‚सिद्धो व‚स्तु तु सिद्धं त‚त्र क‚थ‚मित्याह ।
	\pend% ending standard par
      
	  \bigskip
	  \begingroup
	
	    \large
	  
	    \begin{quote}
	  
	    
	    \stanza[\smallbreak]
	\label{pv.1.21}\flagstanza{\tiny\textenglish{...v.1.21}}असिद्धाव‚पि श‚ब्द‚स्य सिद्धे व‚स्तुनि सिध्य‚ति ।&औ लू क्य स्य य‚था बौद्धेनोक्तं मूर्त्यादिसाध‚न‚म् ॥ २१ ॥\&[\smallbreak]


	
	    \end{quote}
	  
	  \endgroup
	

	  \pstart \leavevmode% starting standard par
	\hphantom{.}‚{\color{DodgerBlue3}‚असिद्धाव‚पि श‚ब्द‚स्य सिद्धे व‚स्तुनि} साध‚नाभिम‚ते ‚{\color{DodgerBlue3}‚सिध्य‚ति}\edtext{}{\edlabel{pvv.16-9}\label{pvv.16-9}\lemma{ते}\Bfootnote{व‚स्त्वेव व‚स्तुनः प्र‚तिब‚न्धाद् ग‚म‚कं न श‚ब्दः ।}} साध्यो\edtext{}{\edlabel{pvv.16-10}\label{pvv.16-10}\lemma{साध्यो}\Bfootnote{स‚प्र‚तिभासादिवादिनां ।}}र्थो ‚{\color{DodgerBlue3}‚य‚था ‚{\tiny $_{lb}$}‚औलूक्य‚स्य} वै शे षि क स्य प‚र‚माणूनाम‚नित्य‚त्व‚साध‚नार्थं ‚{\color{DodgerBlue3}‚बौद्धेन मूर्त्त्यादिसाध‚न}\edtext{}{\edlabel{pvv.16-11}\label{pvv.16-11}\lemma{नार्थं}\Bfootnote{अनित्याः प‚र‚माण‚वो मूर्त्त‚त्वाद् घ‚टादिव‚त् ।}} ‚{\tiny $_{lb}$}‚\leavevmode\ledsidenote{\textenglish{017/s}} ‚{\color{DodgerBlue3}‚मुक्तं} श‚ब्दासिद्धाव‚पि सिध्य‚ति (।) त‚था हि वैशेषिक‚स्यास‚र्व\edtext{}{\edlabel{pvv.17-1}\label{pvv.17-1}\lemma{र्व}\Bfootnote{द्विविधं द्र‚व्यं स‚र्व्व‚ग‚तं पृथिव्यादि अस‚र्व्व‚ग‚तं घ‚टादि ।}} ग‚तं द्र‚व्य‚प‚रिमाणं ‚{\tiny $_{lb}$}‚मूर्त्तिरिष्टा बौद्ध‚स्य स्प‚र्श‚व‚ति\edtext{}{\edlabel{pvv.17-2}\label{pvv.17-2}\lemma{ति}\Bfootnote{न हि स प‚रिमाणं नाम किञ्चिदिच्छ‚ति ।}} सा प्र‚सिद्धा । त‚तो नोभ‚य‚स‚म्प्र‚तिप‚न्ना‚{\tiny $_{2}$}‚ मूर्त्तिश‚ब्द‚{\tiny $_{lb}$}‚वृत्तिर्व्वाच्य‚भेदात् ।---
	\pend% ending standard par
      

	  \pstart \leavevmode% starting standard par
	---श‚ब्द‚स्यासिद्धाव‚पि च स्प‚र्श‚व‚त्त्व‚ल‚क्ष‚णोर्थो द्व‚योर‚पि सिद्धः स एव हेतुत्वेनाभि‚{\tiny $_{lb}$}‚प्रेत इति भ‚व‚ति साध‚न‚म‚त‚श्च (।)
	\pend% ending standard par
      
	  \bigskip
	  \begingroup
	
	    \large
	  
	    \begin{quote}
	  
	    
	    \stanza[\smallbreak]
	\label{pv.1.22}\flagstanza{\tiny\textenglish{...v.1.22}}त‚स्यैव व्य‚भिचारादौ श‚ब्देप्य‚व्य‚भिचारिणि ।&दोष‚व‚त् साध‚नं ज्ञेयं व‚स्तुनो व‚स्तुसिद्धितः ॥ २२ ॥\&[\smallbreak]


	
	    \end{quote}
	  
	  \endgroup
	

	  \pstart \leavevmode% starting standard par
	\hphantom{.}‚{\color{DodgerBlue3}‚त‚स्यैवा}‚र्थ‚स्य ‚{\color{DodgerBlue3}‚व्य‚भिचारा}‚दावादिश‚ब्दाद‚सिद्ध‚त्वे विप‚र्य्य‚य‚व्याप्तौ च ‚{\color{DodgerBlue3}‚श‚ब्देप्य\edtext{}{\edlabel{pvv.17-3}\label{pvv.17-3}\lemma{ब्देप्य}\Bfootnote{य‚था विषाणी शाब‚लेयः क‚ल‚भो वा गोत्वात् ह‚स्तित्वाद्वा ग‚म‚नात् ह‚स्त‚योगाद्वेत्य‚र्थः ।}}व्य‚{\tiny $_{lb}$}‚भिचारिणि दोष‚व‚त्साध‚नं ज्ञेयं} क‚स्मा‚{\color{DodgerBlue3}‚द्व‚स्तुनो} हेतो‚{\color{DodgerBlue3}‚र्व्व‚स्तुनः} साध्य‚स्य ‚{\color{DodgerBlue3}‚सिद्धितः} । एवं ‚{\tiny $_{lb}$}‚साध्य‚व्याप्तार्थ‚शून्य‚श‚ब्द‚मात्र‚काणीश्व‚र‚साध‚नानि दोष‚व‚न्ति बोद्ध‚व्यानि । (२२)
	\pend% ending standard par
      \label{div_pvv.1.23}
	  
	% new div opening: depth here is 2
	

	  \begin{center}%% label @type='head'
	\textbf{(ख) ईश्व‚र‚बाध‚कं प्र‚माण‚म्}
	\end{center}
	

	  \pstart \leavevmode% starting standard par
	न‚नु किं पुन‚रीश्व‚र‚स्य बाध‚कं\edtext{}{\edlabel{pvv.17-4}\label{pvv.17-4}\lemma{कं}\Bfootnote{विश्व‚कार‚ण‚मीश्व‚र इत्य‚त्र ।}} प्र‚माण‚मि‚{\tiny $_{3}$}‚त्याह (।)
	\pend% ending standard par
      
	  \bigskip
	  \begingroup
	
	    \large
	  
	    \begin{quote}
	  
	    
	    \stanza[\smallbreak]
	\label{pv.1.23}\flagstanza{\tiny\textenglish{...v.1.23}}य‚था त‚त् कार‚णं व‚स्तु त‚थैव त‚द‚कार‚ण‚म् ।&य‚दा त‚त् कार‚णं केन म‚तं नेष्ट‚म‚कार‚ण‚म् ॥ २३ ॥\&[\smallbreak]


	
	    \end{quote}
	  
	  \endgroup
	

	  \pstart \leavevmode% starting standard par
	\hphantom{.}‚{\color{DodgerBlue3}‚य‚था} स‚दृशेन स्व‚भावेन ‚{\color{DodgerBlue3}‚त‚दी}\edtext{}{\edlabel{pvv.17-5}\label{pvv.17-5}\lemma{भावेन}\Bfootnote{योऽकार‚काव‚स्थाऽविशिष्टौ न स क‚रोति स इव अविशिष्ट‚श्चाय‚मिति व्याप‚कानुप‚ल‚ब्धिः ।}} श्व‚राख्यं ‚{\color{DodgerBlue3}‚व‚स्तु कार‚ण‚मिष्टं} स‚र्ग्गाव‚स्थायां ‚{\color{DodgerBlue3}‚त‚थैव} तेनैव स्व‚भावेन स‚र्ग्गात् प्राक् त‚न्न ‚{\color{DodgerBlue3}‚कार‚णं} केन विशेषेण म‚तं । न त्व\edtext{}{\edlabel{pvv.17-6}\label{pvv.17-6}\lemma{त्व}\Bfootnote{अकार‚ण‚मेवेष्य‚तां कार‚ण‚व‚स्थायाम‚प्य‚कार‚काव‚स्थाऽविशेषात् ।}}कार‚ण‚मिष्टं । ‚{\tiny $_{lb}$}‚कार‚ण‚त्वं ह्य‚कार‚काव‚स्थाविशिष्ट‚त्वेन व्याप्तं त‚द‚भावात्कार‚ण‚त्वाभावः । (२३)
	\pend% ending standard par
      \label{div_pvv.1.24}
	  
	% new div opening: depth here is 2
	

	  \begin{center}%% label @type='head'
	\textbf{a. अकार‚कं न कार‚ण‚म्}
	\end{center}
	

	  \pstart \leavevmode% starting standard par
	य‚दि पुन‚र‚कार‚काव‚स्थाऽविशिष्टोपीश्व‚रः कार‚ण‚मुच्य‚ते त‚दा (।)
	\pend% ending standard par
      
	  \bigskip
	  \begingroup
	
	    \large
	  
	    \begin{quote}
	  
	    
	    \stanza[\smallbreak]
	\label{pv.1.24}\flagstanza{\tiny\textenglish{...v.1.24}}श‚स्त्रौष‚धाभिस‚म्ब‚न्धाच्चैत्र‚स्य व्र‚ण‚रोह‚णे ।&अस‚म्ब‚द्ध‚स्य किं स्थाणोः कार‚ण‚त्वं न क‚ल्प्य‚ते ॥ २४ ॥\&[\smallbreak]


	
	    \end{quote}
	  
	  \endgroup
	\textsuperscript{\textenglish{018/s}}

	  \pstart \leavevmode% starting standard par
	\hphantom{.}‚{\color{DodgerBlue3}‚चैत्र‚स्य श‚स्त्रौष‚ध‚योः स‚म्ब‚न्धाद्} व्र‚णे ‚{\color{DodgerBlue3}‚व्र‚ण‚रोह‚णे} च वृत्तेऽ‚{\color{DodgerBlue3}‚स‚म्ब‚द्ध‚स्य} व्यापार‚{\tiny $_{lb}$}‚द्वारेणाप्र‚त्यास‚न्न‚{\tiny $_{4}$}‚स्य ‚{\color{DodgerBlue3}‚स्थाणोः किं कार‚ण‚त्वं न क‚ल्प्य‚ते} निमित्त‚स्य स‚मान‚त्वात् । (२४)
	\pend% ending standard par
      \label{div_pvv.1.25}
	  
	% new div opening: depth here is 2
	

	  \pstart \leavevmode% starting standard par
	न‚न्व‚कार‚णाव‚स्थातो व्यापार‚स‚मावेशाद‚स्ति विशेषः कार‚ण‚व‚स्थायामित्याह\edtext{}{\edlabel{pvv.18-1}\label{pvv.18-1}\lemma{स्थायामित्याह}\Bfootnote{पूर्व्व‚स्यासिद्धिं प‚रिह‚र‚ति श‚श‚विषाण‚व‚त् ।}} ।
	\pend% ending standard par
      
	  \bigskip
	  \begingroup
	
	    \large
	  
	    \begin{quote}
	  
	    
	    \stanza[\smallbreak]
	\label{pv.1.25a}\flagstanza{\tiny\textenglish{....1.25a}}स्व‚भाव‚भेदेन विना व्यापारोपि न युज्य‚ते ।\&[\smallbreak]


	
	    \end{quote}
	  
	  \endgroup
	

	  \pstart \leavevmode% starting standard par
	\hphantom{.}‚{\color{DodgerBlue3}‚स्व‚भाव‚भेदेन विना} न केव‚लं कार‚क‚त्वं (।) ‚{\color{DodgerBlue3}‚व्यापारोपि} निर्व्यापार‚स्य नित्य‚स्य ‚{\tiny $_{lb}$}‚‚{\color{DodgerBlue3}‚न युज्य‚ते} (।)
	\pend% ending standard par
      

	  \pstart \leavevmode% starting standard par
	किञ्च (।)
	\pend% ending standard par
      
	  \bigskip
	  \begingroup
	
	    \large
	  
	    \begin{quote}
	  
	    
	    \stanza[\smallbreak]
	\label{pv.1.25b}\flagstanza{\tiny\textenglish{....1.25b}}नित्य‚स्याव्य‚तिरेकित्वात् साम‚र्थ्य‚ञ्च दुर‚न्व‚य‚म् ॥ २५ ॥\&[\smallbreak]


	
	    \end{quote}
	  
	  \endgroup
	

	  \pstart \leavevmode% starting standard par
	\hphantom{.}‚{\color{DodgerBlue3}‚नित्य‚स्याव्य‚तिरेकित्वात् साम‚र्थ्य‚ञ्च दुर‚न्व‚यं} दुर‚व‚ग‚मं न ह्य‚स्तीति कार‚ण‚म् ‚{\tiny $_{lb}$}‚(।)अपि तु य‚द‚भावात्कार्याभावः स त‚त्कार‚ण‚म‚न्य‚थाऽकाशादीनाम‚पि हेतुत्व‚प्र‚स‚ङ्गः ।‚{\tiny $_{5}$}‚ ‚{\tiny $_{lb}$}‚(२५)
	\pend% ending standard par
      \label{div_pvv.1.26}
	  
	% new div opening: depth here is 2
	

	  \pstart \leavevmode% starting standard par
	अपि च (।)
	\pend% ending standard par
      
	  \bigskip
	  \begingroup
	
	    \large
	  
	    \begin{quote}
	  
	    
	    \stanza[\smallbreak]
	\label{pv.1.26}\flagstanza{\tiny\textenglish{...v.1.26}}येषु स‚त्सु भ‚व‚त्येव य‚त् तेभ्योऽन्य‚स्य क‚ल्प‚ने ।&त‚द्धेतुत्वेन स‚र्व‚त्र हेतूनाम‚न‚व‚स्थितिः ॥ २६ ॥\&[\smallbreak]


	
	    \end{quote}
	  
	  \endgroup
	

	  \pstart \leavevmode% starting standard par
	\hphantom{.}‚{\color{DodgerBlue3}‚येषु} कार‚णेषु ‚{\color{DodgerBlue3}‚स‚त्सु\edtext{}{\edlabel{pvv.18-2}\label{pvv.18-2}\lemma{त्सु}\Bfootnote{निमित्त‚कार‚ण‚मीश‚स्त‚न्तुवाय‚व‚दित्याह ।}}} य‚त्कार्य‚{\color{DodgerBlue3}‚म्भ‚व‚त्येव तेभ्यः} कार‚णेभ्यो‚{\color{DodgerBlue3}‚न्य‚स्य} प‚दार्थ‚स्य त‚त्कार्य‚{\tiny $_{lb}$}‚‚{\color{DodgerBlue3}‚हेतुत्वेन} क\edtext{}{\edlabel{pvv.18-3}\label{pvv.18-3}\lemma{क}\Bfootnote{य‚दा त‚दा त‚त्कार‚णं ।}}ल्प‚ने ‚{\color{DodgerBlue3}‚स‚र्व्व‚त्र} कार्य‚{\color{DodgerBlue3}‚हेतूनाम‚न‚व‚स्थितिः} प्राप्नोत्य‚प‚राप‚र‚क‚ल्प‚न‚या (।) ‚{\tiny $_{lb}$}‚त‚स्माद् दृष्ट‚साम‚र्थ्या एव क्षितिबीजाद‚यः कार‚ण‚म‚ङ्कुर‚स्य नेश्व‚रादिर‚दृष्ट‚सा‚{\tiny $_{lb}$}‚म‚र्थ्यः । (२६)
	\pend% ending standard par
      \label{div_pvv.1.27}
	  
	% new div opening: depth here is 2
	

	  \pstart \leavevmode% starting standard par
	न‚नु क्षित्यादिर‚प्य‚कार‚काव‚स्थातो न विशिष्ट‚स्व\edtext{}{\edlabel{pvv.18-4}\label{pvv.18-4}\lemma{स्व}\Bfootnote{एतेन पूर्व्व‚स्यानेकान्त‚माह । असंस्कृत‚क्षेत्राद‚यः (।)}} भावः कार‚णाव‚स्थायाम‚ङ्‚{\tiny $_{lb}$}‚कुर‚स्येत्याह (।)
	\pend% ending standard par
      
	  \bigskip
	  \begingroup
	
	    \large
	  
	    \begin{quote}
	  
	    
	    \stanza[\smallbreak]
	\label{pv.1.27}\flagstanza{\tiny\textenglish{...v.1.27}}स्व‚भाव‚प‚रिणा\edtext{}{\lemma{*}\Afootnote{{\rmlatinfont Correction: }मे न {\rmlatinfont (sic!)}; मेन {\rmlatinfont (corr by \url{frw})}}} हेतुर‚ङ् कुर‚ज‚न्म‚नि ।&भूम्यादिस्त‚स्य संस्कारे त‚द्विशेष‚स्य द‚र्श‚नात् ॥ २७ ॥\&[\smallbreak]


	
	    \end{quote}
	  
	  \endgroup
	

	  \pstart \leavevmode% starting standard par
	उप‚स‚र्प‚ण‚प्र‚त्य‚या\edtext{}{\edlabel{pvv.18-5}\label{pvv.18-5}\lemma{या}\Bfootnote{आद्युत्प‚त्तौ ज‚ग‚तो निमित्त‚मिति चेन्न तेनैवानेकान्त ईश्व‚रान्त‚र‚प्र‚स‚ङ्गात् ‚{\tiny $_{lb}$}‚त‚न्मात्र‚हेतुत्वे क‚र्म‚नैफ‚ल्यं ॥}}द‚दृष्ट‚स‚ह\edtext{}{\edlabel{pvv.18-6}\label{pvv.18-6}\lemma{ह}\Bfootnote{वृष्ट्यादि ।}} कारिणः प्राप्त‚कार्योत्पादानुगु‚{\tiny $_{6}$}‚णातिश‚या ‚{\color{DodgerBlue3}‚भूम्यादिः ‚{\tiny $_{lb}$}‚स्व‚भाव‚प‚रिणा\edtext{}{\lemma{*}\Afootnote{{\rmlatinfont Correction: }मे न {\rmlatinfont (sic!)}; मेन {\rmlatinfont (corr by \url{frw})}}}} कार्यानुगुणातिश‚य‚तार‚त‚म्य‚मुक्ताप‚राप‚र‚क्ष‚ण‚ल‚क्ष‚णेनान्त्याव‚स्था‚{\tiny $_{lb}$}‚\leavevmode\ledsidenote{\textenglish{019/s}} प्राप्ताऽ‚{\color{DodgerBlue3}‚ङ्कुर‚ज‚न्म‚नि हेतु}‚र्भ‚व‚ति (।) स तु पूर्व्व‚प‚रैक‚रूपः क्र‚माक्र‚म‚योर‚र्थ‚क्रिया‚{\tiny $_{lb}$}‚विरोधात् । कुत एत‚दिति चेत् त‚स्य भूम्यादेः क‚र्ष‚ण‚पांसुप्र‚क्षेपादिना ‚{\color{DodgerBlue3}‚संस्कारे} त‚स्याङ‚कुर‚स्य पुष्ट‚त‚रादि‚{\color{DodgerBlue3}‚विशेष‚स्य द‚र्श‚नात्} । (२७)
	\pend% ending standard par
      \label{div_pvv.1.28}
	  
	% new div opening: depth here is 2
	
	  \bigskip
	  \begingroup
	
	    \large
	  
	    \begin{quote}
	  
	    
	    \stanza[\smallbreak]
	\label{pv.1.28}\flagstanza{\tiny\textenglish{...v.1.28}}य‚था विशेषेण विना विष‚येन्द्रिय‚संह‚तिः ।&बुद्धेर्हेतुस्तंथेदं चेन्न त‚त्रापि विशेष‚तः ॥ २८ ॥\&[\smallbreak]


	
	    \end{quote}
	  
	  \endgroup
	

	  \pstart \leavevmode% starting standard par
	\hphantom{.}न‚नु ‚{\color{DodgerBlue3}‚य‚था विशेषेण विना विष‚येन्द्रिय‚संह‚ति}‚र‚क्षेप‚क्रियाध‚र्मिणी ‚{\color{DodgerBlue3}‚बुद्धेर्हेतु}‚{\tiny $_{7}$}‚र्भ‚व‚ति1\leavevmode\ledsidenote{\textenglish{5b/MA}} ‚{\tiny $_{lb}$}‚‚{\color{DodgerBlue3}‚त‚थेद‚मी}‚श्व‚रादि व‚स्तुविशेष\edtext{}{\edlabel{pvv.19-1}\label{pvv.19-1}\lemma{स्तुविशेष}\Bfootnote{पुन‚र‚नेकान्त‚त्व‚मिह ।}} म्विना स‚ह‚कारिस‚न्निधानेन कार्यं क‚रोतीति ‚{\color{DodgerBlue3}‚चेत्} । ‚{\tiny $_{lb}$}‚न (।) ‚{\color{DodgerBlue3}‚त‚त्रापि} विष‚येन्द्रियादिसंह‚तौ प्राग‚व‚स्थाव‚दुप‚स‚र्प्प‚ण‚प्र‚त्य‚य‚ज‚निताद्विज्ञान‚ज‚न‚न‚{\tiny $_{lb}$}‚श‚क्त‚क्ष‚ण‚प्र‚ज्ञाल‚क्ष‚णा‚{\color{DodgerBlue3}‚द्विशेषात्} । (२८)
	\pend% ending standard par
      \label{div_pvv.1.29}
	  
	% new div opening: depth here is 2
	

	  \pstart \leavevmode% starting standard par
	अन्य\edtext{}{\edlabel{pvv.19-2}\label{pvv.19-2}\lemma{अन्य}\Bfootnote{नातिश‚योत्प‚त्त्याऽपि तु संयोगं ज‚न‚य‚न्ति स‚हिताः संयोगात्कार्य‚मित्याह ‚{\tiny $_{lb}$}‚य‚थाकार्य‚ज‚न‚नाय नालं त‚था संयोगेपि ।}}था (।)
	\pend% ending standard par
      
	  \bigskip
	  \begingroup
	
	    \large
	  
	    \begin{quote}
	  
	    
	    \stanza[\smallbreak]
	\label{pv.1.29}\flagstanza{\tiny\textenglish{...v.1.29}}पृथ‚क् पृथ‚ग‚श‚क्तानां स्व‚भावातिश‚येऽस‚ति ।&संह‚ताव‚प्य‚साम‚र्थ्यं स्यात् सिद्धोऽतिश‚य‚स्त‚तः ॥ २९ ॥\&[\smallbreak]


	
	    \end{quote}
	  
	  \endgroup
	

	  \pstart \leavevmode% starting standard par
	\hphantom{.}‚{\color{DodgerBlue3}‚पृथ‚क् पृथ‚ग‚श‚क्तानां} विष‚येन्द्रियाणां ‚{\color{DodgerBlue3}‚स्व‚भावातिश‚येऽस‚ति संह‚ताव‚प्य\edtext{}{\edlabel{pvv.19-3}\label{pvv.19-3}\lemma{प्य}\Bfootnote{कार्यासाम‚र्थ्य‚व‚त् ।}} साम‚र्थ्यं ‚{\tiny $_{lb}$}‚स्यात्} । ज्ञानार्ज‚न‚म्प्र‚ति स्व‚रूपाभेदात् । उत्प‚द्य‚ते च ज्ञानं ‚{\color{DodgerBlue3}‚सिद्धोतिश‚य‚स्त‚तो} ज्ञानो‚{\tiny $_{lb}$}‚त्पादात् । (२९)
	\pend% ending standard par
      \label{div_pvv.1.30}
	  
	% new div opening: depth here is 2
	

	  \begin{center}%% label @type='head'
	\textbf{b. संह‚तौ हेतुता नेश्व‚रादौ}
	\end{center}
	
	  \bigskip
	  \begingroup
	
	    \large
	  
	    \begin{quote}
	  
	    
	    \stanza[\smallbreak]
	\label{pv.1.30}\flagstanza{\tiny\textenglish{...v.1.30}}त‚स्मात् पृथ‚ग‚श‚क्तेषु येषु संभाव्य‚ते गुणः ।&संह‚तौ हेतुता तेषां नेश्व‚रादेर‚भेद‚तः ॥ ३० ॥\&[\smallbreak]


	
	    \end{quote}
	  
	  \endgroup
	

	  \pstart \leavevmode% starting standard par
	\hphantom{.}‚{\color{DodgerBlue3}‚त‚स्मात् पृथ‚ग‚श‚क्तेषु येषु‚{\tiny $_{1}$}‚ स‚म्भाव्य‚ते गुणः} स्व‚रूपान्त‚रोत्पाद‚ल‚क्ष‚णं ‚{\color{DodgerBlue3}‚संह‚तौ हेतुता ‚{\tiny $_{lb}$}‚तेषां} क्ष‚णिकानां ‚{\color{DodgerBlue3}‚नेश्व‚रादेर\edtext{}{\edlabel{pvv.19-4}\label{pvv.19-4}\lemma{रादेर}\Bfootnote{आदिना स्थिरात्म‚नां ग्र‚हः ।}} भेद‚तः} । ईश्व‚र‚प्र‚धान‚पुरुषादेर‚कार‚काभिन्न‚स्व‚रूप‚त्वान्न ‚{\tiny $_{lb}$}‚हेतुत्व‚मित्यु\edtext{}{\edlabel{pvv.19-5}\label{pvv.19-5}\lemma{मित्यु}\Bfootnote{त‚स्मात् स्थित‚मेव त‚त्र नित्यं प्र‚माण‚मिति ।}}प‚संहारः । उक्त‚मीश्व‚रादिदूष‚णं ॥ (३०) ॥
	\pend% ending standard par
      
	  
	% new div opening: depth here is 1
	
\chapter*[{२. भ‚ग‚वान् प्र‚माण‚म्}]{२. भ‚ग‚वान् प्र‚माण‚म्}

	  \begin{center}%% label @type='head'
	\textbf{(१) ज्ञान‚व‚त्वात्}
	\end{center}
	\label{div_pvv.1.31}
	  
	% new div opening: depth here is 2
	

	  \pstart \leavevmode% starting standard par
	भ‚ग‚व‚तोपि साध‚नाभावाद‚प्रामाण्यं प‚र‚म‚तेनाश‚ङ्क‚ते ।
	\pend% ending standard par
      \textsuperscript{\textenglish{020/s}}
	  \bigskip
	  \begingroup
	
	    \large
	  
	    \begin{quote}
	  
	    
	    \stanza[\smallbreak]
	\label{pv.1.31}\flagstanza{\tiny\textenglish{...v.1.31}}प्रामाण्य‚ञ्च प‚रोक्षार्थ‚ज्ञानं त‚त्साध‚न‚स्य च ।&अभावात्; नास्त्य‚नुष्ठान‚मिति केचित् प्र‚च‚क्ष‚ते ॥ ३१ ॥\&[\smallbreak]


	
	    \end{quote}
	  
	  \endgroup
	

	  \pstart \leavevmode% starting standard par
	\hphantom{.}‚{\color{DodgerBlue3}‚प्रामाण्य‚ञ्च प‚रोक्षार्थ‚ज्ञानं}‚, न स‚र्व्व‚स्येष्य‚ते । ‚{\color{DodgerBlue3}‚त‚त्साध‚न‚स्याभावात् । अनु‚{\tiny $_{lb}$}‚ष्ठा\edtext{}{\edlabel{pvv.20-1}\label{pvv.20-1}\lemma{ष्ठा}\Bfootnote{य‚त्साध‚नानुष्ठानात्प्रामाण्यं ।}}नं} क‚स्य‚चिन्ना‚{\color{DodgerBlue3}‚स्तीति} क‚थ‚न्त‚थाविध‚प्र‚माणोप‚प‚त्तिरि‚{\color{DodgerBlue3}‚ति केचित्} जै‚{\tiny $_{2}$}‚ मि नी याः ‚{\tiny $_{lb}$}‚‚{\color{DodgerBlue3}‚प्र‚च‚क्ष‚ते} । (३१)
	\pend% ending standard par
      \label{div_pvv.1.32}
	  
	% new div opening: depth here is 2
	

	  \pstart \leavevmode% starting standard par
	अत्राह ।
	\pend% ending standard par
      
	  \bigskip
	  \begingroup
	
	    \large
	  
	    \begin{quote}
	  
	    
	    \stanza[\smallbreak]
	\label{pv.1.32}\flagstanza{\tiny\textenglish{...v.1.32}}ज्ञान‚वान् मृग्य‚ते क‚श्चित् त‚दुक्त‚प्र‚तिप‚त्त‚ये ।&अज्ञोप‚देश‚क‚र‚णे विप्र‚ल‚म्भ‚न‚श‚ङ्‏किभिः ॥ ३२ ॥\&[\smallbreak]


	
	    \end{quote}
	  
	  \endgroup
	

	  \pstart \leavevmode% starting standard par
	न ख‚लु व्य‚स‚नित‚या प्र‚माण‚म‚न्विष्य‚ते प्रेक्षाव‚द्‏भिर‚पि तु स्व‚र्ग्गाप‚व‚र्ग‚प्र‚धान‚{\tiny $_{lb}$}‚पुरुषार्थं प्रेप्सुभिः त‚द्विष‚य‚{\color{DodgerBlue3}‚ज्ञान‚वान् क‚श्चिद‚न्विष्य‚ते त‚दुक्त}‚स्योपाय‚स्य ‚{\color{DodgerBlue3}‚प्र‚तिप‚त्त‚ये}‚{\tiny $_{lb}$}‚ऽनुष्ठानार्थं ‚{\color{DodgerBlue3}‚अज्ञोप‚देश‚क‚र‚णे विप्र‚ल‚म्भ‚न‚श‚ङ्किभिः} विस‚म्वादं स‚म्भाव‚य‚द्‏भिः ॥ (३२)
	\pend% ending standard par
      \label{div_pvv.1.33}
	  
	% new div opening: depth here is 2
	
	  \bigskip
	  \begingroup
	
	    \large
	  
	    \begin{quote}
	  
	    
	    \stanza[\smallbreak]
	\label{pv.1.33}\flagstanza{\tiny\textenglish{...v.1.33}}त‚स्माद‚नुष्ठेय‚ग‚तं ज्ञान‚म‚स्य विचार्य‚ताम् ।&कीट‚संख्याप‚रिज्ञानं त‚स्य नः क्वोप‚युज्य‚ते ॥ ३३ ॥\&[\smallbreak]


	
	    \end{quote}
	  
	  \endgroup
	

	  \pstart \leavevmode% starting standard par
	\hphantom{.}‚{\color{DodgerBlue3}‚दुःखोप‚श‚मोपायोप‚देष्टुर्ज्ञानं मृग्य‚ते य‚त‚स्त‚स्माद‚नुष्ठेय‚ग‚तं संसार‚दुःख‚प्र‚श‚मोपायं ‚{\tiny $_{lb}$}‚ज्ञान‚म‚स्य प्र‚माण‚पुरुष‚स्य विचार्य‚तां । अनुप‚योगि कीट‚संख्याप‚रिज्ञान‚न्त‚स्योप‚देष्टु‚{\tiny $_{lb}$}‚र्नोऽस्माकं न क्व‚चित्पुरुषार्थे उप‚युज्य‚ते इति न त‚द्विचार्य‚मिति प्र‚तिज्ञा प्र‚त्युप‚का‚{\tiny $_{lb}$}‚राद्य‚पेक्षास‚मं स‚र्व‚स‚त्वेषु। त‚स्माद्य}‚देवं प्रेक्षाव‚ता‚{\color{DodgerBlue3}‚म‚नुष्ठेय‚न्त}‚द्विष‚य‚मुप‚देष्टुं ‚{\color{DodgerBlue3}‚ज्ञान‚मु}‚प‚युक्तं ‚{\tiny $_{lb}$}‚नान्य‚विष‚यं । (३३)
	\pend% ending standard par
      \label{div_pvv.1.34}
	  
	% new div opening: depth here is 2
	

	  \begin{center}%% label @type='head'
	\textbf{(२) हेयोपादेय‚वेद‚क‚त्वात् न तु स‚र्व‚वेद‚क‚त्वात्}
	\end{center}
	
	  \bigskip
	  \begingroup
	
	    \large
	  
	    \begin{quote}
	  
	    
	    \stanza[\smallbreak]
	\label{pv.1.34}\flagstanza{\tiny\textenglish{...v.1.34}}हेयोपादेय‚त‚त्त्व‚स्य साभ्युपाय‚स्य वेद‚कः ।&यः प्र‚माण‚म‚साविष्टो न तु स‚र्व‚स्य वेद‚कः ॥ ३४ ॥\&[\smallbreak]


	
	    \end{quote}
	  
	  \endgroup
	

	  \pstart \leavevmode% starting standard par
	\hphantom{.}त‚स्माद्धेय‚त‚त्त्व‚स्य दुःख‚स‚त्य‚स्य ‚{\color{DodgerBlue3}‚साभ्युपाय‚स्य} स‚मुद‚य‚{\tiny $_{3}$}‚स‚त्यान्वित‚{\color{DodgerBlue3}‚स्योपादेय‚{\tiny $_{lb}$}‚त‚त्त्व‚स्य} निरोध‚स‚त्य‚स्य ‚{\color{DodgerBlue3}‚साभ्युपाय‚स्य} मार्ग्ग‚स‚त्य‚स‚हित‚स्य प्र‚माण‚प‚रिशुद्ध‚स्य यो ‚{\tiny $_{lb}$}‚वेद‚कः ‚{\color{DodgerBlue3}‚स प्र‚माण‚मिष्टो न तु स‚र्व्व‚स्य} य‚स्य क‚स्य‚चिद्वि‚{\color{DodgerBlue3}‚वेद‚कः} । न ख‚लु स‚क‚ल‚ज्ञानादार्य‚{\tiny $_{lb}$}‚स‚त्य‚च‚तुष्ट‚य‚देश‚नाऽपि तु त‚ज्ज्ञान‚त्वात् त‚दुप‚देष्टृत‚यैव च प्रामाण्य‚मिष्य‚ते (। ३४)
	\pend% ending standard par
      \label{div_pvv.1.35}
	  
	% new div opening: depth here is 2
	

	  \pstart \leavevmode% starting standard par
	त‚देवाह ॥
	\pend% ending standard par
      \textsuperscript{\textenglish{021/s}}
	  \bigskip
	  \begingroup
	
	    \large
	  
	    \begin{quote}
	  
	    
	    \stanza[\smallbreak]
	\label{pv.1.35}\flagstanza{\tiny\textenglish{...v.1.35}}दूरं प‚श्य‚तु वा मा वा त‚त्त्व‚मिष्ट‚न्तु प‚श्य‚तु ।&प्र‚माणं दूर‚द‚र्शी चेदेत गृध्रानुपास्म‚हे ॥ ३५ ॥\&[\smallbreak]


	
	    \end{quote}
	  
	  \endgroup
	

	  \pstart \leavevmode% starting standard par
	\hphantom{.}‚{\color{DodgerBlue3}‚दू\edtext{}{\edlabel{pvv.21-1}\label{pvv.21-1}\lemma{दू}\Bfootnote{अतीन्द्रियं स‚र्व्वातीन्द्रियं ।}} रं प‚श्य‚तु वा मा वा द्राक्षीत्त‚त्व‚मिष्ट}‚न्त्वार्य‚स‚त्य‚च‚तुष्ट‚यं ‚{\color{DodgerBlue3}‚प‚श्य‚ति} ताव‚तैव भ‚ग‚{\tiny $_{lb}$}‚वान् ‚{\color{DodgerBlue3}‚प्र‚माण}‚म‚न्य‚थाऽत‚त्त्व‚द‚र्श्य‚पि प्र‚माणं ‚{\color{DodgerBlue3}‚दूर‚द‚{\tiny $_{4}$}‚र्शी चेदिष्य‚ते । एताग‚च्छ‚त मुमुक्ष‚वो ‚{\tiny $_{lb}$}‚गृध्रानुपास्म‚हे} दूर‚द‚र्शिनो दीर्घ‚श्रुती\edtext{}{\edlabel{pvv.21-2}\label{pvv.21-2}\lemma{श्रुती}\Bfootnote{आकाश‚गान् तिर‚श्चः प‚र‚चित्तानुसारिणः क्ष‚णिकादीन् ।}}श्च (? न् च) व‚राहानित्युप‚ह‚स‚ति । उक्त‚म‚{\tiny $_{lb}$}‚भिम‚तं प्रामा\edtext{}{\edlabel{pvv.21-3}\label{pvv.21-3}\lemma{प्रामा}\Bfootnote{पुरुषार्थ‚ज्ञ‚त्वेन । य‚तः स‚त्याव‚बोधाद्ध‚र्म‚च‚क्रादौ भ‚ग‚वान् सार्व‚ज्ञं ‚{\tiny $_{lb}$}‚प्र‚तिज्ञात‚वात् ।}}ण्यं (३५) ॥
	\pend% ending standard par
      \label{div_pvv.1.36}
	  
	% new div opening: depth here is 2
	

	  \begin{center}%% label @type='head'
	\textbf{(३) कारुणिक‚त्वात् प्र‚माण‚म्}
	\end{center}
	

	  \begin{center}%% label @type='head'
	\textbf{क. ज‚न्मान्त‚र‚सिद्धिः}
	\end{center}
	

	  \begin{center}%% label @type='head'
	\textbf{(क‚रुणा ज‚न्मान्त‚राभ्यासात्)}
	\end{center}
	

	  \pstart \leavevmode% starting standard par
	न‚न्वीदृश‚स्य प्र‚माण‚स्य किं साध‚न‚मित्याह (।)
	\pend% ending standard par
      
	  \bigskip
	  \begingroup
	
	    \large
	  
	    \begin{quote}
	  
	    
	    \stanza[\smallbreak]
	\label{pv.1.36}\flagstanza{\tiny\textenglish{...v.1.36}}साध‚नं क‚रुणाभ्यासात् सा बुद्धेर्देह‚संश्र‚यात् ।&असिद्धाऽभ्यास इति चेन्नाश्र‚य‚प्र‚तिषेध‚तः ॥ ३६ ॥\&[\smallbreak]


	
	    \end{quote}
	  
	  \endgroup
	

	  \pstart \leavevmode% starting standard par
	\hphantom{.}‚{\color{DodgerBlue3}‚साध‚नं क‚रुणा} दुःखादुःख‚हेतोश्च स‚मुद्ध‚र‚ण‚काम‚ता क‚रुणा सा भ‚ग‚व‚तः प्रामा‚{\tiny $_{lb}$}‚ण्य‚स्य साध‚नं । सैव क‚रुणा इत्य‚त आह । ‚{\color{DodgerBlue3}‚अभ्यासात्सा} गौत्र‚विशेषात्क‚ल्याण‚मित्र‚{\tiny $_{lb}$}‚संस‚र्ग्गाद‚नुश‚य‚द‚र्श‚नाच्च क‚श्चिन्म‚हास‚त्त्वः कृ‚{\tiny $_{5}$}‚पायामुप‚जात‚स्पृहः सा\edtext{}{\edlabel{pvv.21-4}\label{pvv.21-4}\lemma{सा}\Bfootnote{वृत्तिस्तु निरुप‚द्र‚व‚भूतार्थ‚स्व‚भाव‚स्ये\href{http://sarit.indology.info/?cref=pv.1.212}{(प्र॰१।२१२)}त्य‚त्रोक्तो ।}} द‚र‚निर‚न्त‚रा‚{\tiny $_{lb}$}‚नेक‚ज‚न्म‚प‚र‚म्प‚राप्र‚भ‚वाभ्यासेन सात्मीभूत‚कृप‚या प्रेर्य‚माणः स‚र्व्व‚स‚त्त्वानां स‚मुद‚य‚{\tiny $_{lb}$}‚हान्या दुःख‚हानाय मार्ग‚भाव‚न‚या निरोध‚प्राप‚णाय च देश‚नां क‚र्तुकामः स्व‚य‚म‚साक्षा‚{\tiny $_{lb}$}‚त्कृत‚स्य देश‚नायां विप्र‚ल‚म्भ‚स‚म्भाव‚नाच्च‚तुरार्य‚स‚त्यानि साक्षात् क‚रोतीति भ‚ग‚व‚ति ‚{\tiny $_{lb}$}‚साध‚नं कृपा प्रामाण्य‚स्य । ‚{\color{DodgerBlue3}‚बुद्धेर्देह‚संश्र‚याद‚सिद्धोऽभ्यास इति चेत्} बुद्धिर्देह‚माश्रिता ‚{\tiny $_{lb}$}‚कार्य‚{\tiny $_{6}$}‚त्वात् प्र‚दीप‚मिव प्र‚भा । श‚क्तिरूप‚त्वाद्वा म‚द्य‚मिव म‚द‚श‚क्तिः । गुण‚त्वाद्वा ‚{\tiny $_{lb}$}‚प‚ट‚मिव शुक्ल‚ता । त्रेधाप्याश्र‚य‚विनाशे त‚स्य नाशात् । कुतो ज‚न्मान्त‚राणि क‚थ‚म्वा ‚{\tiny $_{lb}$}‚तेष्व‚भ्यासः कृपादेरिति चा र्व्वा काः ।
	\pend% ending standard par
      \textsuperscript{\textenglish{022/s}}

	  \pstart \leavevmode% starting standard par
	\hphantom{.}त‚देत‚न्न युक्त‚मा‚{\color{DodgerBlue3}‚श्र‚य‚प्र‚तिषेध‚तो} बुद्धेः । बुद्धिर‚हितः कायो ना\edtext{}{\edlabel{pvv.22-1}\label{pvv.22-1}\lemma{ना}\Bfootnote{कार‚ण‚त्वेनाश्र‚यो न ।}}श्र‚यः ‚{\tiny $_{lb}$}‚कार‚ण‚त्वात् गुणित्वात् श‚क्तिम‚त्त्वाद्वा । (३६)
	\pend% ending standard par
      \label{div_pvv.1.37}
	  
	% new div opening: depth here is 2
	

	  \begin{center}%% label @type='head'
	\textbf{(क) भूत‚चैत‚न्य‚म‚त‚निरासः\footnote{\label{pvv.22-asterisk}  द्र‚ष्ट‚व्यं प‚रिशिष्टं ।९,३७}}
	\end{center}
	

	  \begin{center}%% label @type='head'
	\textbf{बीज‚प‚क्ष‚निरासः}
	\end{center}
	

	  \begin{center}%% label @type='head'
	\textbf{a. त‚त्र कार‚ण‚त्वं प्र‚तिषेद्धुमाह ।}
	\end{center}
	
	  \bigskip
	  \begingroup
	
	    \large
	  
	    \begin{quote}
	  
	    
	    \stanza[\smallbreak]
	\label{pv.1.37}\flagstanza{\tiny\textenglish{...v.1.37}}प्राणापानेन्द्रिय‚धियां देहादेव न केव‚लात् ।&स‚जातिनिर‚पेक्षाणां ज‚न्म ज‚न्म‚प‚रिग्र‚हे ॥ ३७ ॥\&[\smallbreak]


	
	    \end{quote}
	  
	  \endgroup
	
	  \bigskip
	  \begingroup
	
	    \large
	  
	    \begin{quote}
	  
	    
	    \stanza[\smallbreak]
	\label{pv.1.38a}\flagstanza{\tiny\textenglish{....1.38a}}अतिप्र‚स‚ङ्गात्;\&[\smallbreak]


	
	    \end{quote}
	  
	  \endgroup
	

	  \pstart \leavevmode% starting standard par
	\hphantom{.}‚{\color{DodgerBlue3}‚प्राणो} वायुरूद्‏र्ध्व‚ग त‚द्विप‚रीतो‚{\color{DodgerBlue3}‚ऽपानः इन्द्रि}‚याणि च‚क्षुरादीनि । ‚{\color{DodgerBlue3}‚धी}‚र्बुद्धिः‚{\tiny $_{7}$}‚ ‚{\tiny $_{lb}$}‚तासां स‚जातिनिर‚पेक्षाणां कार‚ण‚भूत‚पूर्व्व‚{\color{DodgerBlue3}‚स‚जाति}‚प्राणादिपुञ्ज‚{\color{DodgerBlue3}‚निर‚पेक्षाणां देहादेव ‚{\tiny $_{lb}$}‚\leavevmode\ledsidenote{\textenglish{6a/MA}} केव‚लान्न ज‚न्म} भ‚व‚ति (।) कुत इत्याह (।) ‚{\color{DodgerBlue3}‚ज‚न्म‚प‚रिग्र‚हे}‚ऽति (प्र) स‚ङ्गात् (।) ‚{\tiny $_{lb}$}‚य‚दि म‚हाभूतेभ्य एव केव‚लेभ्यः प्राणादीनां ज‚न्म‚ग्र‚ह‚स्त‚दा स‚र्व्व‚स्माद् भ‚वेयु‚{\tiny $_{lb}$}‚रिति स‚र्वं प्राणिम‚यं ज‚ग‚त् स्यात् (।) न चास्त्येत‚त्त‚स्मात्पूर्व्व‚स‚जातिसापेक्षाणा‚{\tiny $_{lb}$}‚मेवाक्षादीनां देहाज्ज‚न्मेति पूर्व्व‚ज‚न्म‚प्र‚तिब‚न्ध‚सिद्धिः । (३७)
	\pend% ending standard par
      \label{div_pvv.1.38}
	  
	% new div opening: depth here is 2
	
	  \bigskip
	  \begingroup
	
	    \large
	  
	    \begin{quote}
	  
	    
	    \stanza[\smallbreak]
	\label{pv.1.38b}\flagstanza{\tiny\textenglish{....1.38b}}भाविज‚न्म‚प‚र‚म्प‚रासिद्ध्य‚र्थ‚म‚प्याह (।)\&[\smallbreak]


	
	    \end{quote}
	  
	  \endgroup
	
	  \bigskip
	  \begingroup
	
	    \large
	  
	    \begin{quote}
	  
	    
	    \stanza[\smallbreak]
	\label{pv.1.38c}\flagstanza{\tiny\textenglish{....1.38c}}य‚द् दृष्टं प्र‚तिस‚न्धान‚श‚क्तिम‚त् ।&किमासीत् त‚स्य य‚न्नास्ति प‚श्चाद् येन न स‚न्धिम‚त् ॥ ३८ ॥\&[\smallbreak]


	
	    \end{quote}
	  
	  \endgroup
	

	  \pstart \leavevmode% starting standard par
	य‚त्प्राणापानादिम‚ध्याव‚{\tiny $_{1}$}‚स्थायां\edtext{}{\edlabel{pvv.22-2}\label{pvv.22-2}\lemma{स्थायां}\Bfootnote{व‚र्त‚मानायां । त‚द‚पि प्राणादीनां स‚त्त्वात् पूर्व्वाव‚स्थाव‚त् नाविक‚ल‚कार‚णो ‚{\tiny $_{lb}$}‚न स‚न्ध‚त्ते । प्राणापानेन्द्रिय‚धियां ।}} प्राणादीनामुत्पाद‚न‚श‚क्तियुक्तं दृष्टं ‚{\color{DodgerBlue3}‚त‚स्य ‚{\tiny $_{lb}$}‚किमासीत् प्र‚तिस‚न्धान}‚कालेऽधिकं ‚{\color{DodgerBlue3}‚य‚त् प‚श्चा}‚न्म‚र‚ण‚का\edtext{}{\edlabel{pvv.22-3}\label{pvv.22-3}\lemma{का}\Bfootnote{कुत‚स्त‚द्देह‚नाशे बुद्धेर‚भावः ।}}ले ‚{\color{DodgerBlue3}‚नास्ति\edtext{}{\edlabel{pvv.22-4}\label{pvv.22-4}\lemma{नास्ति}\Bfootnote{त‚दापि प्राणादीनां स‚त्त्वात् पूर्व्वाव‚स्थाव‚त् नाविक‚ल‚कार‚णो न संध‚त्ते ।}} येन} त‚द्वैक‚ल्यात्त‚दा ‚{\tiny $_{lb}$}‚‚{\color{DodgerBlue3}‚न स‚न्धि}‚म‚त्स‚म‚ग्राप्र‚तिब‚द्ध‚कार‚ण‚त्वात् प्र‚तिस‚न्धानं प्राप्त‚मित्य‚र्थः । (३८)
	\pend% ending standard par
      \label{div_pvv.1.39}
	  
	% new div opening: depth here is 2
	

	  \pstart \leavevmode% starting standard par
	b. स्यादेत‚त् (।) पूर्व्व‚स‚जातिहेतुक‚मिन्द्रियादि न भ‚व‚ति किन्तु देह‚हेतुक‚मेवेदं न ‚{\tiny $_{lb}$}‚चातिप्र‚स‚ङ्गः केषाञ्चिदेव भूत‚प‚रिणामानां देहात्म‚कानां त‚द्धेतुत्वात् । अन्येषाञ्च ‚{\tiny $_{lb}$}‚त‚द्विरुद्ध‚स्व‚भावानाम‚हेतुत्वात् सुव‚र्ण्ण‚बीजाबीज‚पाषाण‚व‚त्‚{\tiny $_{2}$}‚ (।) अत्राह ।
	\pend% ending standard par
      \textsuperscript{\textenglish{023/s}}
	  \bigskip
	  \begingroup
	
	    \large
	  
	    \begin{quote}
	  
	    
	    \stanza[\smallbreak]
	\label{pv.1.39}\flagstanza{\tiny\textenglish{...v.1.39}}न स क‚श्चित् पृथिव्यादेरंशो य‚त्र न ज‚न्त‚वः ।&संस्वेद‚जाद्या जाय‚न्ते स‚र्वं बीजात्म‚कं त‚तः ॥ ३९ ॥\&[\smallbreak]


	
	    \end{quote}
	  
	  \endgroup
	

	  \pstart \leavevmode% starting standard par
	\hphantom{.}‚{\color{DodgerBlue3}‚न स क‚श्चित्पृथिव्यादेरंशः} प्र‚देशो ‚{\color{DodgerBlue3}‚य‚त्र ज‚न्त‚वः संस्वेद‚जाद्या} आद्य‚श‚ब्दाज्ज‚रा‚{\tiny $_{lb}$}‚युजाण्ड‚ज‚प्र‚भृत‚यो ‚{\color{DodgerBlue3}‚न जाय‚न्ते (।) त‚तः स‚र्वं} भूत‚प‚रिण‚तिजातं प्राणादिज‚न‚ने ‚{\color{DodgerBlue3}‚बीजात्म‚क}‚{\tiny $_{lb}$}‚मिति नास्ति बीज‚विरुद्ध‚स्व‚भाव‚ता क‚स्य‚चित् । सुव‚र्ण्णासुव‚र्ण्ण‚बीज‚त्व‚न्तु पाषाणा‚{\tiny $_{lb}$}‚दीनां सुव‚र्ण्ण‚प‚र‚माणूनामेव स‚द‚स‚त्त्वाभ्यामिति विष‚मो दृष्टान्तः । (३९)
	\pend% ending standard par
      \label{div_pvv.1.40}
	  
	% new div opening: depth here is 2
	

	  \pstart \leavevmode% starting standard par
	न‚नु भूत‚मात्र‚हेतुक‚त्वाविशेषेपि प‚रिणाम‚स्य य‚था विशेषः सुव‚र्ण्ण‚प‚र‚माणुम‚य‚त्वे‚{\tiny $_{lb}$}‚त‚राभ्यान्त‚{\tiny $_{3}$}‚था प्राणादिहेतुत्वाहेतुत्वाभ्याम‚पि स्यात् ।
	\pend% ending standard par
      

	  \pstart \leavevmode% starting standard par
	उक्त‚मेवात्र स‚र्व्व‚त्र प्राणिनां दृष्टेः । स‚र्व्व त‚द्बीजात्म‚क‚मिति नास्त्येवाबीजा‚{\tiny $_{lb}$}‚त्म‚क‚ता क‚स्य‚चित्प‚रिणाम‚स्य । अत्रापि वा तुल्यः प्र‚स‚ङ्गः । य‚दि भूत‚मात्र‚हेतुकोऽयं ‚{\tiny $_{lb}$}‚बीज‚प‚रिणामः स‚र्व्व‚स्त‚था स्यात् हेत्व‚विशेषे कार्य‚विशेषायोगात् ।
	\pend% ending standard par
      

	  \begin{center}%% label @type='head'
	\textbf{ii. श‚क्तिम‚त्त्व‚निरासः}
	\end{center}
	

	  \pstart \leavevmode% starting standard par
	a. न‚नु भूतान्य‚प्य‚वान्त‚रानेक‚विविध‚विशेष‚भाञ्जि विचित्राः प‚रिण‚तीर्ज‚न‚य‚न्तीति ‚{\tiny $_{lb}$}‚न स‚मान‚ताप्र‚स‚ङ्गः । न ताव‚त्स‚विशेषो भूत‚मात्रात्‚{\tiny $_{4}$}‚ स‚र्व्व‚त्र प्र‚स‚ङ्गात् । न चान्य‚तो‚{\tiny $_{lb}$}‚ऽन्य‚स्याभावात् । अस्माक‚न्तु क‚र्मापि स‚ह‚कारि स‚म्म‚तं त‚द्वैचित्र्यात् विचित्रं कार्य‚{\tiny $_{lb}$}‚जात‚मुचितं । श‚क्तिप‚क्षं निषेद्धुमाह ।
	\pend% ending standard par
      
	  \bigskip
	  \begingroup
	
	    \large
	  
	    \begin{quote}
	  
	    
	    \stanza[\smallbreak]
	\label{pv.1.40}\flagstanza{\tiny\textenglish{...v.1.40}}त‚त् स्व‚जात्य‚न‚पेक्षाणाम‚क्षादीनां स‚मुद्भ‚वे ।&प‚रिणामो य‚थैक‚स्य स्यात् स‚र्व‚स्याविशेष‚तः ॥ ४० ॥\&[\smallbreak]


	
	    \end{quote}
	  
	  \endgroup
	

	  \pstart \leavevmode% starting standard par
	\hphantom{.}त‚त्त‚स्माद् भूत‚मात्र‚हेतुत्वात् ‚{\color{DodgerBlue3}‚स्व‚जात्य‚न‚पेक्षाणाम‚क्षादीनां स‚मुद्भ‚वे} स्वीक्रिय‚{\tiny $_{lb}$}‚माणे ‚{\color{DodgerBlue3}‚प‚रिणामो य‚थैक‚स्य} देह‚स्यात‚द्भूत‚प्राणापानेन्द्रिय‚चैत‚न्य‚श‚क्तित‚या ‚{\color{DodgerBlue3}‚त‚था ‚{\tiny $_{lb}$}‚स‚र्व्व‚स्य} लोष्टादेर‚पि स्याद्धेतोर‚विशेष‚तः । अत एव हि प्र‚स‚ङ्गाद् गुण‚प‚क्षोपि प्र‚ति‚{\tiny $_{5}$}‚‚{\tiny $_{lb}$}‚क्षिप्तो बोद्ध‚व्यः । त‚स्मादाश्र‚य‚त्व‚म‚पि नास्त्य‚निन्द्रिय‚स्य काय‚स्य कार‚ण‚त्व‚श‚क्ति‚{\tiny $_{lb}$}‚म‚त्वे गुण\edtext{}{\edlabel{pvv.23-1}\label{pvv.23-1}\lemma{गुण}\Bfootnote{चार्वाको नित्यानि भूतान्येव प‚रिण‚त्याद्भुत‚चैत‚न्यानि स‚त्व‚व्य‚प‚देश‚{\tiny $_{lb}$}‚भाञ्जि । तान्येव विरोधिप्र‚त्य‚याद‚न‚भिव्य‚क्त‚चैत‚न्यानीत्युक्तेः ।}}त्वानाम‚योग‚तः । (४०)
	\pend% ending standard par
      \label{div_pvv.1.41}
	  
	% new div opening: depth here is 2
	
	  \bigskip
	  \begingroup
	
	    \large
	  
	    \begin{quote}
	  
	    
	    \stanza[\smallbreak]
	\label{pv.1.41a}\flagstanza{\tiny\textenglish{....1.41a}}b. सेन्द्रियोपि कायो बुद्धेराश्र‚यो न युक्त इति व‚क्तुमाह ।&प्र‚त्येक‚मुप‚घातेऽपि नेन्द्रियाणां म‚नोम‚तेः ।&उप‚घातोस्ति;\&[\smallbreak]


	
	    \end{quote}
	  
	  \endgroup
	\textsuperscript{\textenglish{024/s}}

	  \pstart \leavevmode% starting standard par
	\hphantom{.}‚{\color{DodgerBlue3}‚इन्द्रियाणां प्र‚त्येकं} य‚थास्वं प्र‚त्य‚यैरु\edtext{}{\edlabel{pvv.24-1}\label{pvv.24-1}\lemma{यैरु}\Bfootnote{प्र‚सुप्तिर्नाम व्याधिनाऽन्त‚र्ब्ब‚हिः कायेन्द्रियेपि नाशिते बुद्धेर्भावात् ?}} ‚{\color{DodgerBlue3}‚प‚घातेपि}\edtext{\textsuperscript{*}}{\edlabel{pvv.24-2}\label{pvv.24-2}\lemma{*}\Bfootnote{नानिन्द्रियो भूत‚देहे केश‚न‚खाग्रादौ चैत‚न्य‚प्र‚स‚ङ्गात् ।}} स‚ति ‚{\color{DodgerBlue3}‚म‚नो}‚म‚तेर्व्विक‚ल्प‚बुद्धेरुप‚घात ‚{\tiny $_{lb}$}‚(ो) प‚टुम‚न्द‚तादिल‚क्ष‚णो नास्ति (।) य‚द्धि य‚दाश्रितं त‚त्त‚द्विकारे विक्रिय‚ते य‚था ‚{\tiny $_{lb}$}‚घ‚ट‚स्य दाहादौ त‚च्छुक्ल‚त्वादि(।) त‚स्मान्न त‚दाश्रिता बुद्धिः‚{\tiny $_{6}$}‚ ।
	\pend% ending standard par
      

	  \pstart \leavevmode% starting standard par
	विप‚र्य‚य‚मेवाख्यातुमाह ।
	\pend% ending standard par
      
	  \bigskip
	  \begingroup
	
	    \large
	  
	    \begin{quote}
	  
	    
	    \stanza[\smallbreak]
	\label{pv.1.41b}\flagstanza{\tiny\textenglish{....1.41b}}भ‚ङ्गेऽस्यास्तेषां भ‚ङ्ग‚श्च दृश्य‚ते ॥ ४१ ॥\&[\smallbreak]


	
	    \end{quote}
	  
	  \endgroup
	

	  \pstart \leavevmode% starting standard par
	\hphantom{.}भ‚य‚शोकादिभिर‚स्या म‚नोबुद्धे‚{\color{DodgerBlue3}‚र्भ‚ङ्गे}\edtext{}{\edlabel{pvv.24-3}\label{pvv.24-3}\lemma{नोबुद्धे}\Bfootnote{य‚स्मिँस्थिते य‚न्निव‚र्त‚ते त‚त्त‚तो भिन्नं य‚थोद‚केऽग्निः (।) स्थिते मृत‚श‚रीरे ‚{\tiny $_{lb}$}‚निव‚र्त(?र्त्य) ते प्राणादिभिरिति स्व‚भाव‚हेतुः । य‚त्र पूर्व्व‚स्थिते प्राग‚विद्य‚मानं त‚त्र ‚{\tiny $_{lb}$}‚प‚श्चात् भ‚व‚ति त‚त्त‚तो भिन्नं त‚द्य‚था पूर्व्व‚स्थिते घ‚टे प्राग‚स‚न् प‚श्चात्क्रिय‚माणो ‚{\tiny $_{lb}$}‚दीपः (।) पूर्व्व‚स्थितेषु भूतेषु च त‚त्राविद्य‚मानाः स्युः प्राणाद‚य इति स्व‚भावः ‚{\tiny $_{lb}$}‚पूर्व्वाप‚र‚ज‚न्म‚साध‚न‚द्व‚यं ।}} (नाशे) विकारे स‚ति ‚{\color{DodgerBlue3}‚तेषा}‚मिन्द्रियाणां ‚{\tiny $_{lb}$}‚‚{\color{DodgerBlue3}‚भ‚ङ्गो} विकार‚{\color{DodgerBlue3}‚श्च दृश्य‚ते} । (४१)
	\pend% ending standard par
      \label{div_pvv.1.42}
	  
	% new div opening: depth here is 2
	
	  \bigskip
	  \begingroup
	
	    \large
	  
	    \begin{quote}
	  
	    
	    \stanza[\smallbreak]
	\label{pv.1.42}\flagstanza{\tiny\textenglish{...v.1.42}}त‚स्मात् स्थित्याश्र‚यो बुद्धेर्बुद्धिमेव स‚माश्रितः ।&क‚श्चिन्निमित्त‚म‚क्षाणां त‚स्माद‚क्षाणि बुद्धितः ॥ ४२ ॥\&[\smallbreak]


	
	    \end{quote}
	  
	  \endgroup
	

	  \pstart \leavevmode% starting standard par
	ह‚र्षादिना च प‚रिपुष्टिरिति विक‚ल्प‚बुद्धिविकार‚विकारित्वादिन्द्रियाण्येव ‚{\tiny $_{lb}$}‚त‚दाश्रितानि (।) त‚स्माद् ‚{\color{DodgerBlue3}‚बुद्धेः स्थि}‚तेराश्र‚यः स‚मान‚जातीयः ‚{\color{DodgerBlue3}‚क‚श्चिन्न} सेन्द्रियः\edtext{}{\edlabel{pvv.24-4}\label{pvv.24-4}\lemma{सेन्द्रियः}\Bfootnote{पूर्व्व‚क्ष‚णो बुद्धेः ।}} ‚{\tiny $_{lb}$}‚कायः । स त‚र्हि त‚दाश्रितो भ‚विष्य‚ति न स च ‚{\color{DodgerBlue3}‚बुद्धिमेव स‚माश्रितः । निमि\edtext{}{\edlabel{pvv.24-5}\label{pvv.24-5}\lemma{निमि}\Bfootnote{ज‚न‚क‚श्च ।}}त्त‚ञ्चा‚{\tiny $_{lb}$}‚क्षाणां त‚स्माद‚क्षाणि बुद्धितो} न बुद्धिस्तेभ्यः । (४२)
	\pend% ending standard par
      \label{div_pvv.1.43}
	  
	% new div opening: depth here is 2
	

	  \begin{center}%% label @type='head'
	\textbf{(ख) विज्ञान‚सिद्धिः}
	\end{center}
	

	  \begin{center}%% label @type='head'
	\textbf{I. अन्व‚य‚तः}
	\end{center}
	

	  \begin{center}%% label @type='head'
	\textbf{A. प्र‚तिस‚न्धिः}
	\end{center}
	

	  \begin{center}%% label @type='head'
	\textbf{a. कायाश्र‚यो बुद्धिः}
	\end{center}
	

	  \begin{center}%% label @type='head'
	\textbf{b. प्र‚तिस‚न्ध्याक्षेपिका बुद्धिः}
	\end{center}
	
	  \bigskip
	  \begingroup
	
	    \large
	  
	    \begin{quote}
	  
	    
	    \stanza[\smallbreak]
	\label{pv.1.43a}\flagstanza{\tiny\textenglish{....1.43a}}यादृश्याक्षेपिका साऽसीत् प‚श्चाद‚प्य‚स्तु तादृशी ।\&[\smallbreak]


	
	    \end{quote}
	  
	  \endgroup
	

	  \pstart \leavevmode% starting standard par
	\hphantom{.}त‚तो ज‚न्मादौ ‚{\color{DodgerBlue3}‚यादृशी} दृष्टाऽत्म‚ग्र‚ह‚योगिनी बुद्धिरा‚{\color{DodgerBlue3}‚क्षेपिका} बुद्धीन्द्रियादी‚{\tiny $_{lb}$}‚नामासीत् ‚{\color{DodgerBlue3}‚प‚श्चाद‚पि} म‚र‚णाव‚स्थायाम‚पि ‚{\color{DodgerBlue3}‚तादृ\edtext{}{\edlabel{pvv.24-6}\label{pvv.24-6}\lemma{तादृ}\Bfootnote{अविक‚ल‚कार‚ण‚त्वात् ।}} श्या}‚क्षेपिका भ‚व‚तु । ज‚न्मान्त‚र‚स्य ‚{\tiny $_{lb}$}‚\leavevmode\ledsidenote{\textenglish{025/s}} श‚रीरा\edtext{}{\edlabel{pvv.25-1}\label{pvv.25-1}\lemma{रीरा}\Bfootnote{सिद्धोऽभ्यासो ज‚न्मान्त‚र‚साध‚नात् । नापि बुद्धेर्देह आश्र‚यो येन देह‚नाशे बुद्धि‚{\tiny $_{lb}$}‚नाशाद‚भ्यासो न स्यात् ।}}न्त‚र‚स‚म्ब‚द्ध‚बुद्धीन्द्रियाद्युत्पाद‚न‚ल‚क्ष‚ण‚स्येति पूर्व्वोक्त‚निग‚म‚नं ॥
	\pend% ending standard par
      

	  \begin{center}%% label @type='head'
	\textbf{C. कायाश्रितं म‚नोविज्ञान‚म्}
	\end{center}
	

	  \pstart \leavevmode% starting standard par
	न\edtext{}{\edlabel{pvv.25-2}\label{pvv.25-2}\lemma{न}\Bfootnote{आग‚म‚विरोध‚माह चार्व्वाकः ।}}नु कायाश्रित‚त्वं म‚न‚सोप्युक्तं भ‚ग‚व‚ताऽन्योन्यानु\edtext{}{\edlabel{pvv.25-3}\label{pvv.25-3}\lemma{ताऽन्योन्यानु}\Bfootnote{अन्योन्य‚बीज‚कं ह्येत‚द्द्व‚यं य‚दुत सेन्द्रिय‚त्वं क‚त्यो विज्ञान‚ञ्चेत्य‚भिध‚र्मो ।}}विधायित्वं काय‚म‚न‚सो-\leavevmode\ledsidenote{\textenglish{6b/MA}} ‚{\tiny $_{lb}$}‚रिति व‚द‚ता । त‚त्क‚थ‚मित्याह (।)
	\pend% ending standard par
      
	  \bigskip
	  \begingroup
	
	    \large
	  
	    \begin{quote}
	  
	    
	    \stanza[\smallbreak]
	\label{pv.1.43b}\flagstanza{\tiny\textenglish{....1.43b}}त‚ज्ज्ञानैरुप‚कार्य‚त्वादुक्तं कायाश्रितं म‚नः ॥ ४३ ॥\&[\smallbreak]


	
	    \end{quote}
	  
	  \endgroup
	

	  \pstart \leavevmode% starting standard par
	\hphantom{.}‚{\color{DodgerBlue3}‚त‚ज्ज्ञानैः} काय‚विष‚यैर्ज्ञानैरूपादिग्राहि‚{\color{DodgerBlue3}‚भिरुप‚कार्य‚त्वात्} म‚न‚सः सुखोत्साहादि‚{\tiny $_{lb}$}‚रूपे‚{\color{DodgerBlue3}‚णोक्तं} भ ग व ता ‚{\color{DodgerBlue3}‚कायाश्रितं‚{\tiny $_{1}$}‚ म‚नो}‚विज्ञानं न साक्षात्त‚दुत्प‚त्तेः । (४३)
	\pend% ending standard par
      \label{div_pvv.1.44}
	  
	% new div opening: depth here is 2
	

	  \begin{center}%% label @type='head'
	\textbf{d. अन्योन्य‚हेतुके काय‚म‚न‚सी}
	\end{center}
	

	  \pstart \leavevmode% starting standard par
	न\edtext{}{\edlabel{pvv.25-4}\label{pvv.25-4}\lemma{न}\Bfootnote{य‚तः स‚म‚स्त‚व्य‚स्तेन्द्रिय‚घातेपि न म‚नोघातः ।}} न्विन्द्रियाणि विना न बुद्धि\edtext{}{\edlabel{pvv.25-5}\label{pvv.25-5}\lemma{बुद्धि}\Bfootnote{च‚क्षुरादिज्ञान‚प‚ञ्च‚कं बुद्धिः । य‚थेन्द्रिय‚धीर‚क‚ल्पं रूपादि गृह्णाति त‚था ‚{\tiny $_{lb}$}‚म‚नोपि गृह्णीयात् कायेन ज‚नित‚त्वात्सेन्द्रियेण ।}} रिति युक्त‚माश्र‚य‚त्व‚मेषामित्याह ।
	\pend% ending standard par
      
	  \bigskip
	  \begingroup
	
	    \large
	  
	    \begin{quote}
	  
	    
	    \stanza[\smallbreak]
	\label{pv.1.44}\flagstanza{\tiny\textenglish{...v.1.44}}य‚द्य‚प्य‚क्षैर्विना बुद्धिर्न्न तान्य‚पि त‚या विना ।&त‚थाप्य‚न्योन्य‚हेतुत्वं त‚तोऽप्य‚न्योन्य‚हेतुके ॥ ४४ ॥\&[\smallbreak]


	
	    \end{quote}
	  
	  \endgroup
	

	  \pstart \leavevmode% starting standard par
	\hphantom{.}‚{\color{DodgerBlue3}‚य‚द्य‚प्य‚क्षै}‚र्विना बुद्धिर्न भ‚व‚ति । ‚{\color{DodgerBlue3}‚तान्य‚प्य}‚क्षाणि ‚{\color{DodgerBlue3}‚त‚या विना न} भ‚व‚न्ति । भूत‚मात्रा‚{\tiny $_{lb}$}‚दुत्प‚त्तेः स‚र्व्व‚स्मादुत्प‚त्तिप्र‚स‚ङ्गादित्युक्तेः । ‚{\color{DodgerBlue3}‚त‚थाप्य‚न्योन्य‚हेतुत्व‚मुक्त‚म्भ‚व‚ति । त‚तो‚{\tiny $_{lb}$}‚प्य‚न्योन्य‚हेतुके} काय‚म‚न‚सी म‚ध्याव‚स्थाव‚त् अनादितादृक्‏प्र‚वाह‚व‚ती इति सिद्धः ‚{\tiny $_{lb}$}‚प‚र‚लोकः ।(४४)
	\pend% ending standard par
      \label{div_pvv.1.45}
	  
	% new div opening: depth here is 2
	

	  \pstart \leavevmode% starting standard par
	कि\edtext{}{\edlabel{pvv.25-6}\label{pvv.25-6}\lemma{कि}\Bfootnote{य‚दा कात‚र‚स्यान्यं शोणितादिविकृतं दृष्ट्वैन्द्रिय‚ज्ञानं विकृतं म‚नो विकार‚य‚ति । ‚{\tiny $_{lb}$}‚य‚दि नित्यात्कायादेव प्राणाद‚य‚स्त‚दा ।}}ञ्च ।
	\pend% ending standard par
      
	  \bigskip
	  \begingroup
	
	    \large
	  
	    \begin{quote}
	  
	    
	    \stanza[\smallbreak]
	\label{pv.1.45}\flagstanza{\tiny\textenglish{...v.1.45}}नाक्र‚मात् क्र‚मिणो भावो नाप्य‚पेक्षाऽविशेषिणः ।&क्र‚माद् भ‚व‚न्ती धीः कायात् क्र‚मं त‚स्यापि शंस‚ति ॥ ४५ ॥\&[\smallbreak]


	
	    \end{quote}
	  
	  \endgroup
	\textsuperscript{\textenglish{026/s}}

	  \pstart \leavevmode% starting standard par
	\hphantom{.}‚{\color{DodgerBlue3}‚नाक्र‚मात् क्र‚मिणः}\edtext{\textsuperscript{*}}{\edlabel{pvv.26-1}\label{pvv.26-1}\lemma{*}\Bfootnote{क्ष‚णिक‚स्य ।}} कार्य‚स्य ‚{\color{DodgerBlue3}‚भावः} । क्र‚म‚र‚हित‚त्वात् कार‚ण‚स्य । त\edtext{}{\edlabel{pvv.26-2}\label{pvv.26-2}\lemma{त}\Bfootnote{स‚र्द‚था विशेषेण हेतुरिति चेत् इत्य‚स्य न निय‚मः ।}} न्निष्पा\edtext{}{\edlabel{pvv.26-3}\label{pvv.26-3}\lemma{न्निष्पा}\Bfootnote{इन्द्रिय‚विज्ञानादीनि ।}} ‚{\tiny $_{lb}$}‚द्यानि कार्या‚{\tiny $_{2}$}‚णि स‚कृज्जायेर‚न् । क्र‚म‚व‚तः स‚ह‚कारिणोऽ‚{\color{DodgerBlue3}‚पेक्ष्य} क्र‚माज्ज‚न‚यिष्य‚तीति चेत् ‚{\tiny $_{lb}$}‚‚{\color{DodgerBlue3}‚नाप्य‚विशेषिणः} स्थिरैक‚रूप‚स्य प‚रैर‚नाधेय‚विशेष‚स्य प‚रेषां स‚ह‚कारिणाम‚पेक्षाऽस्ति । ‚{\tiny $_{lb}$}‚त‚स्मा‚{\color{DodgerBlue3}‚त्क्र‚माद् भ‚व‚न्ती धीः कायात्क्र‚म‚न्त‚स्यापि} काय‚स्य शंस‚ति\edtext{}{\edlabel{pvv.26-4}\label{pvv.26-4}\lemma{ति}\Bfootnote{अनुमाप‚य‚ति ।}}। (४५)
	\pend% ending standard par
      \label{div_pvv.1.46}
	  
	% new div opening: depth here is 2
	

	  \pstart \leavevmode% starting standard par
	त‚त‚श्च (।)
	\pend% ending standard par
      
	  \bigskip
	  \begingroup
	
	    \large
	  
	    \begin{quote}
	  
	    
	    \stanza[\smallbreak]
	\label{pv.1.46}\flagstanza{\tiny\textenglish{...v.1.46}}प्र‚तिक्ष‚ण‚म‚पूर्व‚स्य पूर्वः पूर्वः क्ष‚णो भ‚वेत् ।&त‚स्य हेतुर‚तो हेतुर्दृष्ट एवास्तु स‚र्व‚दा ॥ ४६ ॥\&[\smallbreak]


	
	    \end{quote}
	  
	  \endgroup
	

	  \pstart \leavevmode% starting standard par
	\hphantom{.}‚{\color{DodgerBlue3}‚प्र‚तिक्ष‚ण‚म‚पूर्व्व‚स्य} बुद्धीन्द्रिय‚काय‚स‚मुदाय‚स्य कार्य‚स्य पूर्व्वः ‚{\color{DodgerBlue3}‚पूर्व्वः क्ष‚ण}‚स्तादृशो ‚{\tiny $_{lb}$}‚‚{\color{DodgerBlue3}‚हेतुर्भ‚वेद‚तः} कार‚णाद‚न‚न्त‚र‚स्य बुद्धीन्द्रियादेर्हेतुर्म‚ध्याव‚स्थाव‚द् ‚{\color{DodgerBlue3}‚दृष्ट एव} बुद्धीन्द्रियादि‚{\tiny $_{lb}$}‚क‚लापः‚{\tiny $_{3}$}‚ ‚{\color{DodgerBlue3}‚स‚र्व्व‚दा} ऐहिक‚ज‚न्मादौ वामुत्रिक‚ज‚न्मादौ चा‚{\color{DodgerBlue3}‚स्तु} । (४६)
	\pend% ending standard par
      \label{div_pvv.1.47}
	  
	% new div opening: depth here is 2
	

	  \begin{center}%% label @type='head'
	\textbf{e. चित्तान्त‚रेण स‚न्धान‚म्}
	\end{center}
	

	  \pstart \leavevmode% starting standard par
	न‚नु य‚दि स‚विज्ञान‚काय‚त्वात् त‚थाभूत‚ज‚न‚नानुमानेन प‚र‚लोक‚सिद्धिः त‚दा म‚र‚ण‚{\tiny $_{lb}$}‚चित्त‚त्वाच्चित्तान्त‚राप्र‚तिस‚न्धान‚म‚र्ह‚च्च‚र‚म‚चित्त‚व‚दिति क‚स्मान्नानुमीय‚त इत्याह (।)
	\pend% ending standard par
      
	  \bigskip
	  \begingroup
	
	    \large
	  
	    \begin{quote}
	  
	    
	    \stanza[\smallbreak]
	\label{pv.1.47}\flagstanza{\tiny\textenglish{...v.1.47}}चित्तान्त‚र‚स्य स‚न्धाने को विरोधोन्त्य‚चेत‚सः ।&त‚द्व‚द‚प्य‚र्ह‚त‚श्चित्तं अस‚न्धानं कुतो म‚त‚म् ॥ ४७ ॥\&[\smallbreak]


	
	    \end{quote}
	  
	  \endgroup
	

	  \pstart \leavevmode% starting standard par
	\hphantom{.}‚{\color{DodgerBlue3}‚चित्तान्त‚र‚स्य स‚न्धाने को विरोधोऽन्त्य‚चेत‚सो} म‚र‚ण‚चित्त‚स्य न क‚श्चित् । ‚{\tiny $_{lb}$}‚त‚था हि(।)न ताव‚न्म‚र‚ण‚चित्तेन चित्तान्त‚र‚स‚न्धान‚स्य स‚हा\edtext{}{\edlabel{pvv.26-5}\label{pvv.26-5}\lemma{हा}\Bfootnote{प्रीतिदौर्म‚न‚स्यादित्वेन भिन्ना प्र‚तिक्ष‚णं वेद्य‚माना बुद्धिर्य‚दि नित्येष्य‚ते त‚दा‚{\tiny $_{lb}$}‚त‚त्त्वान्त‚रं प‚ञ्च‚मं स्यात् आलोकान्ध‚कार‚व‚त् ।}}न‚व‚स्थान‚ल‚क्ष‚णो विरोधः ‚{\tiny $_{lb}$}‚निव‚र्त्त्य‚निव‚र्त‚क‚त्वाभावात् । नापि प‚र‚स्प‚र‚प‚रिहार‚स्थितिल‚क्ष‚णः\edtext{}{\edlabel{pvv.26-6}\label{pvv.26-6}\lemma{णः}\Bfootnote{भावाभाव‚व‚त् ! म‚नोज्ञानं ।}} ‚{\tiny $_{4}$}‚ अम\edtext{}{\edlabel{pvv.26-7}\label{pvv.26-7}\lemma{अम}\Bfootnote{न तु संतान‚चित्त (?) व्य‚व‚च्छेदेन ।}} र‚ण‚चित्त‚{\tiny $_{lb}$}‚व्य‚व‚च्छेदेन म‚र‚ण‚चित्त‚स्याव‚स्थानात् । ‚{\color{DodgerBlue3}‚त‚द्व‚द‚प्य‚र्ह‚त‚श्चित्त‚म‚स‚न्धानं कुतः} प्र‚माणात् ‚{\tiny $_{lb}$}‚‚{\color{DodgerBlue3}‚म‚तं} येन दृष्टान्तः स्यात् । न ह्य‚र्ह‚न् भ‚व‚तां\edtext{}{\edlabel{pvv.26-8}\label{pvv.26-8}\lemma{तां}\Bfootnote{चार्व्वाकाणाम् ।}}सिद्धः । त‚द्‏बाध‚नाय य‚त्नात् । (४७)
	\pend% ending standard par
      \label{div_pvv.1.48}
	  
	% new div opening: depth here is 2
	

	  \pstart \leavevmode% starting standard par
	अथाभ्युप‚ग‚म्य‚ते त‚दा त‚स्य क्लेश‚विसंयोग‚कृत‚म‚स‚न्धानं नान्य‚था स च न ‚{\tiny $_{lb}$}‚पृथ‚ग्‏ज‚नानामिति क‚थ‚न्तेषां म‚र‚ण‚चित्त‚म‚स‚न्धानं । क्लेश‚विसंयोगो हि प्र‚तिस‚न्धान‚{\tiny $_{lb}$}‚विरोधी ल‚क्ष्य‚ते न म‚र‚ण‚चित्तं ॥\edtext{\textsuperscript{*}}{\edlabel{pvv.26-9}\label{pvv.26-9}\lemma{*}\Bfootnote{अन‚न्य‚स‚त्त्व‚नेय‚स्येत्यादिना प्र‚तिस‚न्धानं स्नेहात् व‚क्ष्य‚ते न सोऽर्ह‚तः । भावि‚{\tiny $_{lb}$}‚कार‚णानुप‚ल‚ब्धिः । एवं म‚न्य‚ते न प‚रेण ।}}
	\pend% ending standard par
      \textsuperscript{\textenglish{027/s}}

	  \pstart \leavevmode% starting standard par
	स्यादेत‚द्(।) भ‚व‚त्सिद्धान्त‚सिद्ध‚म‚र्ह‚च्चित्त‚म‚स‚{\tiny $_{5}$}‚न्धान‚न्त‚त एव दृष्टान्त‚सिद्धिः । ‚{\tiny $_{lb}$}‚अत्राह (।)
	\pend% ending standard par
      
	  \bigskip
	  \begingroup
	
	    \large
	  
	    \begin{quote}
	  
	    
	    \stanza[\smallbreak]
	\label{pv.1.48}\flagstanza{\tiny\textenglish{...v.1.48}}असिद्धार्थः प्र‚माणेन किं सिद्धान्तोऽनुग‚म्य‚ते ॥&हेतोर्वैक‚ल्य‚त‚स्त‚च्चेत् किन्त‚देवाऽत्र नोदित‚म् ॥ ४८ ॥\&[\smallbreak]


	
	    \end{quote}
	  
	  \endgroup
	

	  \pstart \leavevmode% starting standard par
	\hphantom{.}‚{\color{DodgerBlue3}‚असिद्धार्थो}‚ऽनिश्चितार्थः ‚{\color{DodgerBlue3}‚प्र‚माणेन किं सिद्धान्तोऽनुग‚म्य‚ते} त‚द‚नुरोधेन प‚र‚लोक‚स्या‚{\tiny $_{lb}$}‚भ्युप‚ग‚म‚प्र‚स‚ङ्गात् । अथ ‚{\color{DodgerBlue3}‚हेतो}‚राश्वास‚प्र‚श्वासेन्द्रिय‚पाट\edtext{}{\edlabel{pvv.27-1}\label{pvv.27-1}\lemma{पाट}\Bfootnote{अर्ह‚त‚श्चित्त‚स्य हेतोः ।}} वादे‚{\color{DodgerBlue3}‚र्वैक‚ल्य‚तो भ‚र‚ण‚चित्त‚स्य} त‚द‚प्र‚तिस‚न्धान‚मिति चेत् । ‚{\color{DodgerBlue3}‚किन्त‚देव} हेतुवैक\edtext{}{\edlabel{pvv.27-2}\label{pvv.27-2}\lemma{हेतुवैक}\Bfootnote{म‚र‚ण‚चित्त‚त्वादिति हित्वा ।}} ल्य\edtext{}{\edlabel{pvv.27-3}\label{pvv.27-3}\lemma{ल्य}\Bfootnote{अत्र र‚थ्यापुरुषे ।}} म‚त्राप्र‚तिस‚न्धाने साध‚न‚त्वे ‚{\color{DodgerBlue3}‚नोदितं} । ‚{\tiny $_{lb}$}‚दृष्टान्त‚विक‚ल‚न्तु म‚र‚ण‚चित्त‚मुक्तं । आश्वासादिहेतुत्व‚निषेध‚ञ्च व‚क्ष्य‚ति । (४८)
	\pend% ending standard par
      \label{div_pvv.1.49}
	  
	% new div opening: depth here is 2
	

	  \begin{center}%% label @type='head'
	\textbf{f. काय‚स्याहेतुत्व‚म्}
	\end{center}
	
	  \bigskip
	  \begingroup
	
	    \large
	  
	    \begin{quote}
	  
	    
	    \stanza[\smallbreak]
	\label{pv.1.49}\flagstanza{\tiny\textenglish{...v.1.49}}त‚द्धीव‚द् ग्र‚ह‚ण‚प्राप्तेर्म‚नोज्ञानं न सेन्द्रियात् ।&ज्ञानोत्पाद‚न‚साम‚र्थ्य‚भेदान्न स‚क‚लाद‚पि ॥ ४९ ॥\&[\smallbreak]


	
	    \end{quote}
	  
	  \endgroup
	

	  \pstart \leavevmode% starting standard par
	काय‚श्च हेतुर्भ‚व‚न् सेन्द्रियो वा स्याद‚निन्द्रियो वा । सेन्द्रियोपि प्र‚त्येक‚{\tiny $_{lb}$}‚मिन्द्रियैः स‚हितः स‚म‚स्तैर्व्वा (।) त‚त्र प्र‚थ‚म‚प‚क्षे त‚स्या इन्द्रिय‚धिय\edtext{}{\edlabel{pvv.27-4}\label{pvv.27-4}\lemma{धिय}\Bfootnote{अव्य‚व‚धानेन च । त‚स्मादुत्त‚र‚बुद्धिज‚न‚न‚स‚म‚र्था वास‚नारूपा क‚र्म्म‚संज्ञिता ‚{\tiny $_{lb}$}‚पूर्व्व‚बुद्धिरेवाश्र‚य इति स्थितं ।}} इव विक‚ल्प्य‚मानेषु ‚{\tiny $_{lb}$}‚रूपादिषु स्प‚ष्ट‚त‚र‚स्य ‚{\color{DodgerBlue3}‚ग्र‚ह‚ण‚स्य प्राप्तेः । म‚नोज्ञानं न सेन्द्रियात कायात्} । इन्द्रिय‚{\tiny $_{lb}$}‚ज‚न्य‚स्य स्प‚ष्ट‚त‚याऽभ्रान्त‚ज्ञानादिषु व्याप्तिदृष्टेः । द्वितीय‚प‚क्षेप्याह (।) प्र‚त्येक‚{\tiny $_{lb}$}‚मिन्द्रियाणां रूपादिग्र‚ह‚ण‚प्र‚तिनिय‚त‚{\color{DodgerBlue3}‚ज्ञानोत्पाद‚न‚साम‚र्थ्य‚भेदात्} दृष्टात् । ‚{\color{DodgerBlue3}‚न स‚क‚ला-} द‚पीन्द्रिय‚क‚लापात्‚{\tiny $_{7}$}‚ प्र‚तिनिय‚त‚विष‚याग्राहिणो म‚नोविज्ञान‚स्य स‚म्भ‚वः\edtext{}{\edlabel{pvv.27-5}\label{pvv.27-5}\lemma{वः}\Bfootnote{सामान्य‚विष‚य‚त्वाद‚स्य ।}} । एकेन्द्रि-\leavevmode\ledsidenote{\textenglish{7a/MA}} ‚{\tiny $_{lb}$}‚य‚वैक‚ल्येप्य‚नुत्पाद‚प्र‚स\edtext{}{\edlabel{pvv.27-6}\label{pvv.27-6}\lemma{स}\Bfootnote{तृतीयः त‚दा स‚म्ब‚न्धाद् ग‚वाश्व‚व‚त् पृथ‚ग्‏भावः स्यात् ।}}ङ्गाच्च । (४९)
	\pend% ending standard par
      \label{div_pvv.1.50}
	  
	% new div opening: depth here is 2
	

	  \pstart \leavevmode% starting standard par
	नाप्य‚निन्द्रियो हेतुरित्याह
	\pend% ending standard par
      
	  \bigskip
	  \begingroup
	
	    \large
	  
	    \begin{quote}
	  
	    
	    \stanza[\smallbreak]
	\label{pv.1.50a}\flagstanza{\tiny\textenglish{....1.50a}}अचेत‚न‚त्वान्नान्य‚स्माद्;\&[\smallbreak]


	
	    \end{quote}
	  
	  \endgroup
	

	  \pstart \leavevmode% starting standard par
	\hphantom{.}‚{\color{DodgerBlue3}‚अचेत‚न‚त्वान्नान्य‚स्माद}‚निन्द्रियात् केश‚न‚खादेरिव म‚नोज्ञानं ।
	\pend% ending standard par
      

	  \pstart \leavevmode% starting standard par
	न‚न्व‚चेत‚न‚त्वं किमिन्द्रिय‚ज्ञान‚र‚हित‚त्व‚मुत म‚नोज्ञान‚विमुक्त‚त्वं(।) त‚त्राद्य‚मिष्ट‚मेव ‚{\tiny $_{lb}$}‚\leavevmode\ledsidenote{\textenglish{028/s}} इन्द्रिय‚स्याभावे त‚ज्ज्ञानाभावात् । अन्त्ये साध्याविशिष्ट‚त्वं हेतोः । म‚नोज्ञान‚स्यैव ‚{\tiny $_{lb}$}‚साध्य‚त्वात् ।
	\pend% ending standard par
      

	  \pstart \leavevmode% starting standard par
	उच्य‚ते (।) य‚था स्प‚र्शा\edtext{}{\edlabel{pvv.28-1}\label{pvv.28-1}\lemma{र्शा}\Bfootnote{स्प‚र्श‚मात्र‚स्य केव‚लास‚म्भ‚वादादिश‚ब्दः । त‚ज्ज्ञानैरुप‚कार्य‚त्वादिनोक्त‚म‚पि ।}}द‚यः स्प‚र्श‚ज्ञानेन‚{\tiny $_{1}$}‚ चेत‚य‚न्ते न त‚था न‚ख‚केशाद‚य ‚{\tiny $_{lb}$}‚इत्य‚चेत‚नाः । चेत‚नाप्र‚तिब‚द्ध‚ञ्च म‚नोविज्ञानं त‚द‚भावे न स्यात् ।
	\pend% ending standard par
      

	  \pstart \leavevmode% starting standard par
	(B.क‚र्म)
	\pend% ending standard par
      ‚{\tiny $_{lb}$}‚

	  \pstart \leavevmode% starting standard par
	(a. क‚र्म‚भेदात् काय‚म‚न‚सोः स‚ह‚स्थितिः)
	\pend% ending standard par
      ‚{\tiny $_{lb}$}‚

	  \pstart \leavevmode% starting standard par
	\hphantom{.}न‚नु य‚दि न काय आश्र‚यः क‚थं ‚{\color{DodgerBlue3}‚स‚ह‚स्थिति}‚रित्य‚त आह ।
	\pend% ending standard par
      
	  \bigskip
	  \begingroup
	
	    \large
	  
	    \begin{quote}
	  
	    
	    \stanza[\smallbreak]
	\label{pv.1.50b}\flagstanza{\tiny\textenglish{....1.50b}}हेत्व‚भेंदात् स‚ह‚स्थितिः ।&अक्ष‚व‚द् रूप‚र‚स‚व‚द‚र्थ‚द्वारेण विक्रिया ॥ ५० ॥\&[\smallbreak]


	
	    \end{quote}
	  
	  \endgroup
	

	  \pstart \leavevmode% starting standard par
	हेतोः क‚र्म्म‚संज्ञित‚स्याभेदात् । एक‚साम‚ग्रीप्र‚तिब‚द्ध‚त्वात् स‚ह‚स्थितिः । न ‚{\tiny $_{lb}$}‚त्वाश्र‚याश्र‚यिभावात् ।\edtext{\textsuperscript{*}}{\edlabel{pvv.28-2}\label{pvv.28-2}\lemma{*}\Bfootnote{किमिव ।}} ‚{\color{DodgerBlue3}‚अक्ष‚व‚द्रूप‚र‚स‚व‚त्} । य‚थाऽक्षाणि रूप‚र‚साद‚य‚श्च प‚र‚स्प‚र‚{\tiny $_{lb}$}‚म‚नाश्र‚याश्र‚यिभूता अप्येक‚साम‚ग्र्‏य‚धीन‚त्वात् स‚ह‚स्थितिम‚न्तः ।\edtext{\textsuperscript{*}}{\edlabel{pvv.28-3}\label{pvv.28-3}\lemma{*}\Bfootnote{निय‚त‚स‚ह‚भावो नेति च तुल्यं देह‚म‚न‚सोर्वा क‚ल्प्य‚म‚नोमात्र‚त्वात् ।}} न‚नु यो य‚द्विकारेण ‚{\tiny $_{lb}$}‚विक्रिय‚ते स त‚दाश्रितो‚{\tiny $_{2}$}‚ य‚था च‚क्षुरादिविकारेण विक्रिय‚माण‚न्त‚ज्ज्ञानं च‚क्षुराद्या‚{\tiny $_{lb}$}‚श्रितं । विष‚श्लेष्मादिना काय‚विकारे च विक्रिय‚ते म‚नोविज्ञानं अत‚स्त‚दाश्रित‚{\tiny $_{lb}$}‚मित्याह (।) त‚ज्ज्ञानैरुप‚कार्य‚त्वादिनोक्त‚म‚पि । ‚{\color{DodgerBlue3}‚अर्थ‚द्वारेण विक्रिया} आल‚म्व्य‚माना ‚{\tiny $_{lb}$}‚हि श‚स्त्र‚प्र‚हाराद‚यो विकार‚य‚न्ति मान‚सं न त्वाश्र‚य\edtext{}{\edlabel{pvv.28-4}\label{pvv.28-4}\lemma{य}\Bfootnote{एतेन कार‚णात्प्राङ् निष्प‚न्नं त‚त्स‚ह‚भावि वा न कार्य‚मिति भिन्नाभिन्न‚स‚न्ता‚{\tiny $_{lb}$}‚न‚योः सामान्यः कार्य‚कार‚ण‚भावः ।}}त्वेन ।(५०)
	\pend% ending standard par
      \label{div_pvv.1.51}
	  
	% new div opening: depth here is 2
	

	  \begin{center}%% label @type='head'
	\textbf{i. उप‚कार‚को निव‚र्त्त‚क‚श्च हेतुः}
	\end{center}
	
	  \bigskip
	  \begingroup
	
	    \large
	  
	    \begin{quote}
	  
	    
	    \stanza[\smallbreak]
	\label{pv.1.51}\flagstanza{\tiny\textenglish{...v.1.51}}स‚त्तोप‚कारिणो य‚स्य नित्यं त‚द‚नुब‚न्ध‚तः ।&स हेतुः स‚प्त‚मी त‚स्मादुत्पादादिति चोच्य‚ते ॥ ५१ ॥\&[\smallbreak]


	
	    \end{quote}
	  
	  \endgroup
	

	  \pstart \leavevmode% starting standard par
	\hphantom{.}न चोप‚कार‚क इत्येवाश्र‚यः किन्तु निर्व्व‚र्त‚कः । त‚दाह । ‚{\color{DodgerBlue3}‚य‚स्य} स‚त्ता निर्व‚र्त्य‚स्य ‚{\tiny $_{lb}$}‚‚{\color{DodgerBlue3}‚उप‚कारिणी} स हेतुः स एवाश्र‚यः । क‚थ‚मुप‚कारिणी(।) ‚{\color{DodgerBlue3}‚नित्यं त‚स्य} निर्व‚र्त्त्य‚स्या‚{\tiny $_{lb}$}‚‚{\color{DodgerBlue3}‚नुब‚न्ध‚तो}‚{\tiny $_{3}$}‚ऽनुव‚र्त‚नात् । य‚स्य तु क‚दा\edtext{}{\edlabel{pvv.28-5}\label{pvv.28-5}\lemma{दा}\Bfootnote{स्व‚स‚न्निधिभावेन. . . . त्या कार्य‚कारीत्य‚र्थः मृत्पिण्डं घ‚ट‚कारीव । य‚थोपा‚{\tiny $_{lb}$}‚घ्यायः त‚द‚स‚त्त्वेऽप्य‚स्य स‚त्त्वं ।}} चिदुप‚कार‚क‚त्व‚म‚सौ विशेष‚स्यैव हेतुर्न ध‚र्मिणः ‚{\tiny $_{lb}$}‚\leavevmode\ledsidenote{\textenglish{029/s}} त‚द‚भावेऽपि त‚द्भावात् । चित्त‚मात्र‚प्र‚तिब‚द्ध‚ञ्च चित्तं त‚द्विशेष‚स्तु कायादिहेतुकः । ‚{\tiny $_{lb}$}‚अतो नायं हेतुः न च त‚न्निवृ\edtext{}{\edlabel{pvv.29-1}\label{pvv.29-1}\lemma{न्निवृ}\Bfootnote{आरूप्ये भावात् ।}} त्त्या निवृत्तिश्चित्त‚स्य । नित्यानुब‚न्धित‚या हेतुत्व‚म‚भि‚{\tiny $_{lb}$}‚प्रेत्य चास्मिन् स‚तीदं भ‚व‚तीति स‚प्त‚म्युच्य‚ते ।\edtext{\textsuperscript{*}}{\edlabel{pvv.29-2}\label{pvv.29-2}\lemma{*}\Bfootnote{कार्योत्पाद‚न‚स‚म‚र्थ‚स्य पूर्व भावः कार‚ण‚त्वं स‚प्त‚म्या प‚ञ्च‚म्या च निर्दिष्टं ‚{\tiny $_{lb}$}‚त‚दुत्त‚र‚त्वं कार्य‚त्वं प्र‚थ‚म‚माह ।}} त‚स्मादिति च प‚ञ्च‚मी त‚द्धेत्व‚नुव‚द्धां ‚{\tiny $_{lb}$}‚कार्य‚ताञ्चाभिप्रेत्यो‚{\color{DodgerBlue3}‚त्पादादिति चो\edtext{}{\edlabel{pvv.29-3}\label{pvv.29-3}\lemma{चो}\Bfootnote{न चेदृशोन्व‚यः काय‚चित्त‚योर‚स्ति दृश्य‚ते वा ।}}च्य‚ते} । (५१)
	\pend% ending standard par
      \label{div_pvv.1.52}
	  
	% new div opening: depth here is 2
	

	  \pstart \leavevmode% starting standard par
	न‚नु हेतुरुप‚कार‚कः क‚थ\edtext{}{\edlabel{pvv.29-4}\label{pvv.29-4}\lemma{थ}\Bfootnote{प‚रं पृच्छ‚ति । य‚दि व‚ह्न्यादिव‚त्त‚था प्र‚दीप‚हेतुः त‚था देह‚श्चैत‚न्यादेश्च ।}}ञ्चिदुप‚कुर्व्व‚न् दृश्य‚ते च‚{\tiny $_{4}$}‚ क‚दाचिद्देह‚श्चित्त‚म‚त ‚{\tiny $_{lb}$}‚आह (।)
	\pend% ending standard par
      
	  \bigskip
	  \begingroup
	
	    \large
	  
	    \begin{quote}
	  
	    
	    \stanza[\smallbreak]
	\label{pv.1.52}\flagstanza{\tiny\textenglish{...v.1.52}}अस्तूप‚कार‚को वापि क‚दाचिच्चित्त‚स‚न्त‚तेः ।&व‚ह्न्यादिव‚द् घ‚टादीनां विनिवृत्तिर्न ताव‚ता ॥ ५२ ॥\&[\smallbreak]


	
	    \end{quote}
	  
	  \endgroup
	

	  \pstart \leavevmode% starting standard par
	\hphantom{.}‚{\color{DodgerBlue3}‚अस्तूप‚कार\edtext{}{\edlabel{pvv.29-5}\label{pvv.29-5}\lemma{कार}\Bfootnote{कायाश्रित‚ज्ञान‚द्वारेण ।}}को वापि क‚दाचिच्चित्त‚स‚न्त‚ते}‚र्देहो ‚{\color{DodgerBlue3}‚व‚ह्न्यादिव‚द् घ‚टादीनां न च ‚{\tiny $_{lb}$}‚ताव‚तो}‚प‚कार‚क‚निवृत्त्या उप‚कार्य‚स्य ‚{\color{DodgerBlue3}‚निवृत्तिः} । न हि व‚ह्निर्घ‚ट‚स्यापाक‚मुप‚कारं ‚{\tiny $_{lb}$}‚कुर्व्व‚न्न‚पि स्व‚निवृत्त्या निव‚र्त‚कः । (५२)
	\pend% ending standard par
      \label{div_pvv.1.53}
	  
	% new div opening: depth here is 2
	

	  \pstart \leavevmode% starting standard par
	किञ्च\edtext{}{\edlabel{pvv.29-6}\label{pvv.29-6}\lemma{किञ्च}\Bfootnote{य‚दि देहः कार‚णं बुद्धेः ।}} (।)
	\pend% ending standard par
      
	  \bigskip
	  \begingroup
	
	    \large
	  
	    \begin{quote}
	  
	    
	    \stanza[\smallbreak]
	\label{pv.1.53a}\flagstanza{\tiny\textenglish{....1.53a}}अनिवृत्तिप्र‚स‚ङ्ग‚श्च देहे तिष्ठ‚ति चेत‚सः ।\&[\smallbreak]


	
	    \end{quote}
	  
	  \endgroup
	

	  \pstart \leavevmode% starting standard par
	\hphantom{.}‚{\color{DodgerBlue3}‚अनेवृत्तिप्र‚स‚ङ्ग‚श्च देहे तिष्ठ‚ति चेत‚सः} । स्व‚जातिनिर‚पेक्ष‚स्य देह‚मात्र‚हेतु‚{\tiny $_{lb}$}‚क‚स्याविक‚ल‚हेतोर‚नुत्प‚त्त्य‚योगात् त‚तो याव‚द्धेतुस्ताव‚न्न म‚र‚णं भ‚वेत् ।
	\pend% ending standard par
      

	  \begin{center}%% label @type='head'
	\textbf{(ii. न प्राणापान‚हेतुकं चित्त‚म्)}
	\end{center}
	

	  \pstart \leavevmode% starting standard par
	प्राणापानाव‚पि चित्त‚कार‚णं‚{\tiny $_{5}$}‚ त‚योर्व्वैक‚ल्यान्म‚र‚णाव‚स्थायान्न चित्तोत्पाद इति ‚{\tiny $_{lb}$}‚चेत् । आह (।)
	\pend% ending standard par
      
	  \bigskip
	  \begingroup
	
	    \large
	  
	    \begin{quote}
	  
	    
	    \stanza[\smallbreak]
	\label{pv.1.53b}\flagstanza{\tiny\textenglish{....1.53b}}त‚द्भाव‚भावाद् व‚श्य‚त्वात् प्राणाप‚नौ त‚तो न त‚त् ॥ ५३ ॥\&[\smallbreak]


	
	    \end{quote}
	  
	  \endgroup
	

	  \pstart \leavevmode% starting standard par
	\hphantom{.}त‚स्य चित्त‚स्य ‚{\color{DodgerBlue3}‚भावे भावात् प्राणापा\edtext{}{\edlabel{pvv.29-7}\label{pvv.29-7}\lemma{प्राणापा}\Bfootnote{आकुञ्च‚न‚प्र‚सार‚ण‚व‚त् ।}} नौ त‚तो} भ‚व‚तः । ‚{\color{DodgerBlue3}‚चित्त‚व‚श्य‚त्वाच्च} त‚त ‚{\tiny $_{lb}$}‚एव तौ न तु ताभ्यां त‚च्चित्तं विप‚र्य‚यात् ।(५३)
	\pend% ending standard par
      \textsuperscript{\textenglish{030/s}}\label{div_pvv.1.54}
	  
	% new div opening: depth here is 2
	

	  \pstart \leavevmode% starting standard par
	एत‚द्धेतुद्व‚य‚सिद्ध्य‚र्थ‚माह (।)
	\pend% ending standard par
      
	  \bigskip
	  \begingroup
	
	    \large
	  
	    \begin{quote}
	  
	    
	    \stanza[\smallbreak]
	\label{pv.1.54a}\flagstanza{\tiny\textenglish{....1.54a}}प्रेर‚णाक‚र्ष‚णो वायोः प्र‚य‚त्नेन विना कुतः ।\&[\smallbreak]


	
	    \end{quote}
	  
	  \endgroup
	

	  \pstart \leavevmode% starting standard par
	\hphantom{.}ब‚हिर‚न्त‚श्च ‚{\color{DodgerBlue3}‚प्रेर‚णाक‚र्ष‚णे वायो}‚र्य‚थाक्र‚मं प्राणा\edtext{}{\edlabel{pvv.30-1}\label{pvv.30-1}\lemma{प्राणा}\Bfootnote{वृत्तिकृता (?) एत‚त्प‚द‚म‚न‚ङ्गीकृत्य व्याख्यात‚म् । पुनः प‚र‚म‚त‚माश‚ङ्क‚ते ‚{\tiny $_{lb}$}‚न कार्यं चैत‚न्य‚स्य कार‚णं किन्तु ।}}पानौ पुरुष‚स्य बुद्धिल‚क्ष‚णं ‚{\tiny $_{lb}$}‚प्र‚य‚त्न‚{\color{DodgerBlue3}‚म्वि\edtext{}{\edlabel{pvv.30-2}\label{pvv.30-2}\lemma{म्वि}\Bfootnote{त‚द्‏भावात् प्रेर‚णाक‚र्ष‚ण‚व‚त् ।}}ना} कुतः स‚म्भ‚व‚त‚स्त‚स्मात् त‚द‚धीनौ ।
	\pend% ending standard par
      

	  \pstart \leavevmode% starting standard par
	किञ्च (।)
	\pend% ending standard par
      
	  \bigskip
	  \begingroup
	
	    \large
	  
	    \begin{quote}
	  
	    
	    \stanza[\smallbreak]
	\label{pv.1.54b}\flagstanza{\tiny\textenglish{....1.54b}}निर्ह्रासातिश‚याप‚त्तिनिर्ह्रासातिश‚यात् त‚योः ॥ ५४ ॥\&[\smallbreak]


	
	    \end{quote}
	  
	  \endgroup
	

	  \pstart \leavevmode% starting standard par
	\hphantom{.}य‚दि प्राणापान‚हेतुक‚ञ्चितं त‚दा चित्त‚स्य ‚{\color{DodgerBlue3}‚निर्ह्रासातिश‚या‚{\tiny $_{6}$}‚प‚त्ति}‚र‚प‚क‚र्षोत्क‚र्षा‚{\tiny $_{lb}$}‚प‚त्तिः । ‚{\color{DodgerBlue3}‚निर्ह्रासातिश‚यात्त‚योः} प्राणापान‚योः कार‚ण‚विशेषानुकारित्वात् कार्य\edtext{}{\edlabel{pvv.30-3}\label{pvv.30-3}\lemma{कार्य}\Bfootnote{य‚था मृताव‚स्थायां श‚रीरे तिष्ठ‚ति किन्त‚च्चैत‚न्यं न भ‚व‚तीति चोद्यं बौद्धेन ‚{\tiny $_{lb}$}‚त‚था प्राणापानाव‚पि क‚स्मान्निव‚र्त्त्येते येन त‚द‚भावान्न चैत‚न्य‚म् ।}}विशे‚{\tiny $_{lb}$}‚ष‚स्य । (५४)
	\pend% ending standard par
      \label{div_pvv.1.55}
	  
	% new div opening: depth here is 2
	
	  \bigskip
	  \begingroup
	
	    \large
	  
	    \begin{quote}
	  
	    
	    \stanza[\smallbreak]
	\label{pv.1.55a}\flagstanza{\tiny\textenglish{....1.55a}}तुल्यः प्र‚स‚ङ्गोपि त‚योः;\&[\smallbreak]


	
	    \end{quote}
	  
	  \endgroup
	

	  \pstart \leavevmode% starting standard par
	\edtext{\textsuperscript{*}}{\edlabel{pvv.30-4}\label{pvv.30-4}\lemma{*}\Bfootnote{प‚रः मृत‚देहे तिष्ठिति चित्तं किन्न स्यादिति बौद्ध‚स्य प्र‚स‚ङ्गेन स‚ह ।}}अपि च ‚{\color{DodgerBlue3}‚त‚योर‚पि} प्राणापान‚योर्देहे तिष्ठ‚त्य‚निवृ\edtext{}{\edlabel{pvv.30-5}\label{pvv.30-5}\lemma{निवृ}\Bfootnote{अनेकान्त‚त्व‚मुक्तं प‚र‚स्य ।}}त्ति‚{\color{DodgerBlue3}‚प्र‚स‚ङ्ग‚स्तुल्य}‚स्त‚द्धेतुत्वा‚{\tiny $_{lb}$}‚‚{\color{DodgerBlue3}‚त्त‚योः} । त‚तो म‚र‚ण‚काले प्राणापान‚वैक‚ल्याभावाच्चित्तानिवृत्तिप्र‚स‚ङ्ग‚स्त‚द‚व‚स्थः ।
	\pend% ending standard par
      

	  \pstart \leavevmode% starting standard par
	(b. स्यित्यावेध‚क‚त्वात् कार‚णं क‚र्म)
	\pend% ending standard par
      ‚{\tiny $_{lb}$}‚

	  \pstart \leavevmode% starting standard par
	\hphantom{.}स्यादेत‚त् (।) ‚{\color{DodgerBlue3}‚चित्त‚कार‚णे}‚ऽपि चित्ते स्वीक्रिय‚माणे म‚र‚ण‚स‚म‚ये चित्तान‚पायात् ‚{\tiny $_{lb}$}‚चित्तोत्प‚त्तौ याव‚द्देहेऽनिवृत्तिप्र‚{\tiny $_{7}$}‚स‚ङ्ग इत्याह (सिद्धान्ती)।
	\pend% ending standard par
      
	  \bigskip
	  \begingroup
	
	    \large
	  
	    \begin{quote}
	  
	    
	    \stanza[\smallbreak]
	\label{pv.1.55b}\flagstanza{\tiny\textenglish{....1.55b}}न तुल्यं चित्त‚कार‚णे ।&स्थित्यावेध‚क‚म‚न्य‚च्च य‚तः कार‚ण‚मिष्य‚ते ॥ ५५ ॥\&[\smallbreak]


	
	    \end{quote}
	  
	  \endgroup
	

	  \pstart \leavevmode% starting standard par
	\hphantom{.}‚{\color{DodgerBlue3}‚न तुल्यं चित्त‚कार‚णे} चैत‚न्ये प्र‚स‚ञ्ज‚नं । न ख‚ल्व‚नुव‚र्त‚क‚मेव कार‚ण‚मिष्टं ‚{\tiny $_{lb}$}‚येन नित्यं त‚स्मिन्नाश्र‚ये स्यात् । किन्त‚र्हि (।) ‚{\color{DodgerBlue3}‚स्थित्यावेध‚क‚म‚न्य‚च्च} क‚र्म्माख्यं ‚{\tiny $_{lb}$}‚‚{\color{DodgerBlue3}‚कार‚णं य‚त इष्य‚ते} त‚स्मात्तेन क‚र्म‚णा याव‚त्कालं त‚द्देहे स्थितिराक्षिप्ता त‚त ऊर्ध्व ‚{\tiny $_{lb}$}‚हेत्व‚पायान्न प्र‚व‚र्त‚ते देहान्त‚रे तु व‚र्त‚त इत्य‚स‚मान‚त्वं ।(५५)
	\pend% ending standard par
      \label{div_pvv.1.56}
	  
	% new div opening: depth here is 2
	\textsuperscript{\textenglish{031/s}}

	  \begin{center}%% label @type='head'
	\textbf{(i. अन्य‚था दोषाभावे पुन‚रुज्जीव‚न‚म्)}
	\end{center}
	
	  \bigskip
	  \begingroup
	
	    \large
	  
	    \begin{quote}
	  
	    
	    \stanza[\smallbreak]
	\label{pv.1.56}\flagstanza{\tiny\textenglish{...v.1.56}}न दोषैर्विगुणो दोहो हेतुर्व‚र्त्यादिव‚द् य‚दि ।&मृते श‚मीकृते दोषे पुन‚रुज्जीव‚नं भ‚वेत् ॥ ५६ ॥\&[\smallbreak]


	
	    \end{quote}
	  
	  \endgroup
	

	  \pstart \leavevmode% starting standard par
	\hphantom{.}दौषैर्व्वातादिभिर्विगुणीकृतो ‚{\color{DodgerBlue3}‚देहो} न हेतुश्चैत‚न्य‚स्य प्राणापानादेश्च ‚{\color{DodgerBlue3}‚व‚र्त्त्यादि‚{\tiny $_{lb}$}‚व‚द् य‚दि} । य‚था विगुणी‚{\tiny $_{8}$}‚कृतो व‚र्त्त्यादिर्न दीप‚स्य हेतुः । त‚दा ‚{\color{DodgerBlue3}‚मृते}\edtext{}{\edlabel{pvv.31-1}\label{pvv.31-1}\lemma{दा}\Bfootnote{क‚स्य‚चित्स्व‚भाव‚स्याकाराभिम‚तेनाकार्य‚त्वाद‚किञ्चित्क‚रो नाश्र‚यः । अस‚द्धि ‚{\tiny $_{lb}$}‚श‚श‚विषाण‚क‚ल्पं किंमाश्र‚येत ।}} श‚रीरे चैत‚न्यादि-\leavevmode\ledsidenote{\textenglish{7b/MA}} ‚{\tiny $_{lb}$}‚नाशान्मृत‚स्य श‚मीभ‚व‚न्ति दोषा इति ‚{\color{DodgerBlue3}‚श‚मीकृते दोषे} विकार‚विकारिण्यारोग्य‚{\tiny $_{lb}$}‚लाभाद्धेतोर्देह‚स्य ‚{\color{DodgerBlue3}‚पुन‚रुज्जीव‚नं} प्राणा\edtext{}{\edlabel{pvv.31-2}\label{pvv.31-2}\lemma{प्राणा}\Bfootnote{त‚स्माद्दोषाभावेऽप्य‚भावान्न दोष‚स‚न्निधिमात्रं चैत‚न्य‚वृत्तिं प्र‚तिब‚ध्नातीति ‚{\tiny $_{lb}$}‚स्थितं ।}} पानं चैत‚न्योत्प‚त्तिल‚क्ष‚{\color{DodgerBlue3}‚ण‚म्भ\edtext{}{\edlabel{pvv.31-3}\label{pvv.31-3}\lemma{म्भ}\Bfootnote{प‚रः प्राह ।}} वेत्} । (५६)
	\pend% ending standard par
      \label{div_pvv.1.57}
	  
	% new div opening: depth here is 2
	

	  \begin{center}%% label @type='head'
	\textbf{ii. देह‚श्चित्त‚स्य नोपादान‚म्}
	\end{center}
	
	  \bigskip
	  \begingroup
	
	    \large
	  
	    \begin{quote}
	  
	    
	    \stanza[\smallbreak]
	\label{pv.1.57}\flagstanza{\tiny\textenglish{...v.1.57}}निवृत्तेप्य‚न‚ले काष्ठ‚विकाराविनिवृत्तिव‚त् ।&त‚स्याऽनिवृत्तिरिति चेन्न चिकित्साप्र‚योग‚तः ॥ ५७ ॥\&[\smallbreak]


	
	    \end{quote}
	  
	  \endgroup
	

	  \pstart \leavevmode% starting standard par
	\hphantom{.}‚{\color{DodgerBlue3}‚निवृत्तेप्य‚न‚ले} हेतौ ‚{\color{DodgerBlue3}‚काष्ठ‚विकार}‚स्याङ्गारादेर‚{\color{DodgerBlue3}‚निवृत्तिव‚त् । त‚स्य} चैत‚न्यादि‚{\tiny $_{lb}$}‚निरोध‚स्या‚{\color{DodgerBlue3}‚निवृत्तिर्निवृत्तिर्न} भ‚व‚ति त‚द्दोषे मृत‚श‚रीरान्निवृत्ते‚{\color{DodgerBlue3}‚पीति चेत्\edtext{}{\edlabel{pvv.31-4}\label{pvv.31-4}\lemma{चेत्}\Bfootnote{प‚रिह‚र‚ति}} । न ‚{\tiny $_{lb}$}‚चिकित्साप्र‚योग‚तः} । (५७)
	\pend% ending standard par
      \label{div_pvv.1.58}
	  
	% new div opening: depth here is 2
	

	  \pstart \leavevmode% starting standard par
	स्या‚{\tiny $_{1}$}‚देत‚द् (।) य‚द्य‚निव‚र्त्त्या दोषाणां विकार(ा)ः स्यात्(?स्युः) दृश्य‚न्ते च ‚{\tiny $_{lb}$}‚चिकित्स्य‚माना दोष‚विकारा ज्व‚राद‚यः ।
	\pend% ending standard par
      

	  \pstart \leavevmode% starting standard par
	एत\edtext{}{\edlabel{pvv.31-5}\label{pvv.31-5}\lemma{एत}\Bfootnote{पूर्व्वोक्तं स्प‚ष्ट‚य‚ति ।}}दे‚{\tiny $_{1}$}‚वाह ।
	\pend% ending standard par
      
	  \bigskip
	  \begingroup
	
	    \large
	  
	    \begin{quote}
	  
	    
	    \stanza[\smallbreak]
	\label{pv.1.58}\flagstanza{\tiny\textenglish{...v.1.58}}अपुन‚र्भाव‚तः किञ्चिद् विकार‚ज‚न‚नं क्व‚चित् ।&किञ्चिद् विप‚र्य‚याद‚ग्निर्य‚था काष्ठ‚सुव‚र्ण‚योः ॥ ५८ ॥\&[\smallbreak]


	
	    \end{quote}
	  
	  \endgroup
	

	  \pstart \leavevmode% starting standard par
	\hphantom{.}‚{\color{DodgerBlue3}‚किञ्चि}‚द्व‚स्तु ‚{\color{DodgerBlue3}‚क्व‚चि}‚द्विकार्ये ‚{\color{DodgerBlue3}‚विका}‚र‚स्य ‚{\color{DodgerBlue3}‚ज‚न‚नं} ज‚न‚क‚{\color{DodgerBlue3}‚म‚पुन‚र्भाव‚तो} य‚था विकार्य‚{\tiny $_{lb}$}‚स्यापुन‚र्भावो निर्व्विकार‚त्वं पुन‚र्न भ‚व‚ति ‚{\color{DodgerBlue3}‚य‚थाग्नि काष्ठे (।) किञ्चिदेव त‚द्विप‚र्य्य-} याद्य‚था‚{\color{DodgerBlue3}‚ग्नि सुव‚र्ण्णे} । (५८)
	\pend% ending standard par
      \label{div_pvv.1.59}
	  
	% new div opening: depth here is 2
	
	  \bigskip
	  \begingroup
	
	    \large
	  
	    \begin{quote}
	  
	    
	    \stanza[\smallbreak]
	\label{pv.1.59}\flagstanza{\tiny\textenglish{...v.1.59}}आद्य‚स्याल्पोप्य‚संहार्यः प्र‚त्यानेय‚स्तु य‚त्कृतः ।&विकारः स्यात् पुन‚र्भावः त‚स्य हेम्नि ख‚र‚त्व‚व‚त् ॥ ५९ ॥\&[\smallbreak]


	
	    \end{quote}
	  
	  \endgroup
	\textsuperscript{\textenglish{032/s}}

	  \pstart \leavevmode% starting standard par
	\hphantom{.}‚{\color{DodgerBlue3}‚त‚त्राद्य‚स्य} विकार‚स्या‚{\color{DodgerBlue3}‚ल्पोपि} विकारः श्याम‚तादिकोऽसंहार्योऽनिव‚र्त‚नीयः । ‚{\tiny $_{lb}$}‚‚{\color{DodgerBlue3}‚प्र‚त्यानेयो} निव‚र्त‚नीय‚स्तु ‚{\color{DodgerBlue3}‚य‚त्कृ‚{\tiny $_{2}$}‚तो विकार‚स्त‚स्य} विकार्य‚स्य ‚{\color{DodgerBlue3}‚पुन‚र्भावः स्या}‚त् ‚{\tiny $_{lb}$}‚विकार‚निवृत्त्या पूर्व्वाव‚स्थाप‚त्तिः । व‚ह्निकृते द्र‚व‚त्वे निवृत्ते हेम्नि ख‚र‚त्व‚व‚त् । दृढ‚त्व‚{\tiny $_{lb}$}‚मिव ॥‚{\tiny $_{2}$}‚ (५९)
	\pend% ending standard par
      \label{div_pvv.1.60}
	  
	% new div opening: depth here is 2
	

	  \pstart \leavevmode% starting standard par
	न‚नु\edtext{}{\edlabel{pvv.32-1}\label{pvv.32-1}\lemma{नु}\Bfootnote{चिकित्स्य‚ते च ज्ञानादिकृतो विकारः त‚स्मात्पूर्व्व‚प्र‚स‚ङ्ग‚स्त‚द‚व‚स्थः ।}} य‚दि निव‚र्त‚नीयो दोष‚विकार‚स्त‚दा न क‚श्चिद‚साध्यो वि\edtext{}{\edlabel{pvv.32-2}\label{pvv.32-2}\lemma{वि}\Bfootnote{काष्ठ‚विकार‚व‚देन‚माह प‚रः ।}}धिः स्यादित्याह ॥
	\pend% ending standard par
      
	  \bigskip
	  \begingroup
	
	    \large
	  
	    \begin{quote}
	  
	    
	    \stanza[\smallbreak]
	\label{pv.1.60}\flagstanza{\tiny\textenglish{...v.1.60}}दुर्ल‚भ‚त्वात् स‚माधातुर‚साध्यं किञ्चिदीरित‚म् ।&आयुःक्ष‚याद् वा; दोषे तु केव‚ले नास्त्य‚साध्य‚ता ॥ ६० ॥\&[\smallbreak]


	
	    \end{quote}
	  
	  \endgroup
	

	  \pstart \leavevmode% starting standard par
	\hphantom{.}‚{\color{DodgerBlue3}‚य‚त्किञ्चिद‚साध्यं} व्या\edtext{}{\edlabel{pvv.32-3}\label{pvv.32-3}\lemma{व्या}\Bfootnote{सिद्धान्ती ।}}धिजात‚मुक्तं त‚{\color{DodgerBlue3}‚त्स‚माधातुः} स‚द्यो दोष‚श‚म‚नौष‚ध‚र‚सा‚{\tiny $_{lb}$}‚य‚नादेर्दुर्ल\edtext{}{\edlabel{pvv.32-4}\label{pvv.32-4}\lemma{नादेर्दुर्ल}\Bfootnote{असाध्य‚व्याधिश्च त‚त्त‚त्‏क‚र्म‚व‚शात् । अकाल‚म‚र‚णे ॥}} ‚{\color{DodgerBlue3}‚भ‚त्वात्} क‚र्म्माक्षिप्त‚स्यायुष‚स्त‚च्छ‚रीर‚स‚ह‚च‚र‚चैत‚न्यादिस्थितिहेतोर्व्वा ‚{\tiny $_{lb}$}‚क्ष\edtext{}{\edlabel{pvv.32-5}\label{pvv.32-5}\lemma{क्ष}\Bfootnote{काल‚म‚र‚णे ।}}यात् । भूत‚मात्र‚वादि‚{\tiny $_{3}$}‚नः ‚{\color{DodgerBlue3}‚केव‚ले दोषे तु} स्वीक्रिय‚माणे ‚{\color{DodgerBlue3}‚नास्त्य‚साध्य‚ता} क‚स्य‚चिद् ‚{\tiny $_{lb}$}‚व्याधेः । भूत‚मात्रार‚ब्ध‚स्य दोष‚स्य चिकित्सादृष्टेः । विशे\edtext{}{\edlabel{pvv.32-6}\label{pvv.32-6}\lemma{विशे}\Bfootnote{तुल्य‚व्याध्योस्तुल्यौष‚ध‚योश्च चिर‚क्षिप्रोप‚श‚म‚श्च दृश्य‚त एव क‚र्म‚कृतः ।}}ष‚कारिण‚श्च हेतोर‚{\tiny $_{lb}$}‚भावात् । (६०)
	\pend% ending standard par
      \label{div_pvv.1.61}
	  
	% new div opening: depth here is 2
	

	  \pstart \leavevmode% starting standard par
	त‚था (।)
	\pend% ending standard par
      
	  \bigskip
	  \begingroup
	
	    \large
	  
	    \begin{quote}
	  
	    
	    \stanza[\smallbreak]
	\label{pv.1.61}\flagstanza{\tiny\textenglish{...v.1.61}}मृते विषादिसंहारात् त‚द्दंश‚च्छेद‚तोऽपि वा ।&विकार‚हेतोर्विग‚मे स नोच्छ्‏व‚सिति किं पुनः ॥ ६१ ॥\&[\smallbreak]


	
	    \end{quote}
	  
	  \endgroup
	

	  \pstart \leavevmode% starting standard par
	\hphantom{.}विषादिना ‚{\color{DodgerBlue3}‚मृते} प्राणिनि स‚ति दंश\edtext{}{\edlabel{pvv.32-7}\label{pvv.32-7}\lemma{दंश}\Bfootnote{याव‚ज्जीव‚ति ताव‚द् व्याप्नोति विषं मूते तु दंश‚स्थानं यातीति निय‚मः ।}} स्थाने ‚{\color{DodgerBlue3}‚विषादेः संहारात्} संग‚ण‚नात्त‚स्य ‚{\color{DodgerBlue3}‚दंश}‚{\tiny $_{lb}$}‚स्थान‚स्य ‚{\color{DodgerBlue3}‚च्छेद‚तोपि वा विकार‚हेतो}‚र्म‚र‚ण‚हेतोर्व्विषादे‚{\color{DodgerBlue3}‚र्व्विग‚मात् स} मृतः ‚{\color{DodgerBlue3}‚पुन‚र्नोच्छ्‏व‚{\tiny $_{lb}$}‚सिति} । क‚स्मात् (।) चैत‚न्य‚हेतोर्देह‚स्य विकार‚त्वात् युक्त‚{\tiny $_{4}$}‚मुच्छ्‏व‚सितुं । (६१)
	\pend% ending standard par
      \label{div_pvv.1.62}
	  
	% new div opening: depth here is 2
	

	  \pstart \leavevmode% starting standard par
	किञ्च (।)
	\pend% ending standard par
      
	  \bigskip
	  \begingroup
	
	    \large
	  
	    \begin{quote}
	  
	    
	    \stanza[\smallbreak]
	\label{pv.1.62}\flagstanza{\tiny\textenglish{...v.1.62}}उपादानाविकारेण नोपादेय‚स्य विक्रिया ।&क‚र्त्तुं श‚क्याऽविकारेण मृदः कुण्डादिनो य‚था ॥ ६२ ॥\&[\smallbreak]


	
	    \end{quote}
	  
	  \endgroup
	

	  \pstart \leavevmode% starting standard par
	\hphantom{.}य‚द्य‚पादानं देह‚श्चित्त‚स्य त‚दो‚{\color{DodgerBlue3}‚पादान}‚स्य देह‚स्या‚{\color{DodgerBlue3}‚विकारे}‚ण विकारं विनो‚{\color{DodgerBlue3}‚पादेय‚स्य} चित्त‚स्य ‚{\color{DodgerBlue3}‚विक्रिया न क‚र्त्तुं श‚क्या} स्यात् । ‚{\color{DodgerBlue3}‚य‚था} कुण्डाद्युपादान‚भूताया ‚{\color{DodgerBlue3}‚मृदो} विकारेण ‚{\tiny $_{lb}$}‚\leavevmode\ledsidenote{\textenglish{033/s}} विना कु‚{\color{DodgerBlue3}‚ण्डादिनो} विकारः क‚र्त्तुभ‚श‚क्यः । (६२)
	\pend% ending standard par
      \label{div_pvv.1.63}
	  
	% new div opening: depth here is 2
	

	  \pstart \leavevmode% starting standard par
	क‚स्मादेव‚मित्याह (।)
	\pend% ending standard par
      
	  \bigskip
	  \begingroup
	
	    \large
	  
	    \begin{quote}
	  
	    
	    \stanza[\smallbreak]
	\label{pv.1.63}\flagstanza{\tiny\textenglish{...v.1.63}}अविकृत्य हि य‚द् व‚स्तु यः प‚दार्थों विकार्य‚ते ।&उपादानं न त‚त् त‚स्य युक्तं गोग‚व‚यादिव‚त् ॥ ६३ ॥\&[\smallbreak]


	
	    \end{quote}
	  
	  \endgroup
	

	  \pstart \leavevmode% starting standard par
	\hphantom{.}हिर्य‚स्मात् ‚{\color{DodgerBlue3}‚अविकृत्य य‚द्व‚स्तु} किञ्चिद्यः ‚{\color{DodgerBlue3}‚प‚दार्थो विकार्य‚ते उपादान‚न्न त‚त्त‚स्य} प‚दार्थ‚स्य यु‚{\color{DodgerBlue3}‚क्तं गोग‚व‚यादिव‚त्} गोग‚व‚य‚यो‚{\tiny $_{5}$}‚रिव नोपादानोपादेय‚भावः । एक‚स्या‚{\tiny $_{lb}$}‚विकारेणाप‚र‚स्य विकारात् । (६३)
	\pend% ending standard par
      \label{div_pvv.1.64}
	  
	% new div opening: depth here is 2
	

	  \begin{center}%% label @type='head'
	\textbf{(II. व्य‚तिरेक‚तः)}
	\end{center}
	

	  \begin{center}%% label @type='head'
	\textbf{A. आश्र‚याश्र‚यिभाव‚निरासः}
	\end{center}
	

	  \begin{center}%% label @type='head'
	\textbf{a. सामान्येन आश्र‚याश्र‚यिभाव‚निरासः}
	\end{center}
	

	  \begin{center}%% label @type='head'
	\textbf{i. न काय‚चेत‚सोः स‚ह‚स्थितिराश्र‚याश्र‚यित‚या}
	\end{center}
	
	  \bigskip
	  \begingroup
	
	    \large
	  
	    \begin{quote}
	  
	    
	    \stanza[\smallbreak]
	\label{pv.1.64a}\flagstanza{\tiny\textenglish{....1.64a}}चेतःश‚रीर‚योरेवं त‚द्धेतोः कार्य‚ज‚न्म‚नः ।\&[\smallbreak]


	
	    \end{quote}
	  
	  \endgroup
	

	  \pstart \leavevmode% starting standard par
	\hphantom{.}‚{\color{DodgerBlue3}‚चेतःश‚रीर‚योरेवं} श‚रीर‚म‚विकृत्यैव भ‚य‚शोकादिना स‚म‚न‚न्त‚र‚प्र‚त्य‚य‚विकार‚{\tiny $_{lb}$}‚मात्रेण चेत‚सो विकारोत्प‚त्तेः नोपादानोपादेय‚भावः । य‚दि नाम देहोऽपि क‚थ‚ञ्चि‚{\tiny $_{lb}$}‚दुप‚कार‚क‚श्चित्त‚स्याव‚स्थाविशेष‚हेतुत्वात् त‚थापि स‚न्तान‚हेतुत्वाभावात् नोपादान‚म‚{\tiny $_{lb}$}‚त‚स्त‚न्निवृत्त्या न चैत‚न्य‚निवृत्तिः‚{\tiny $_{6}$}‚ । अव‚स्थाविशेष एव निव‚र्तेत ।
	\pend% ending standard par
      

	  \pstart \leavevmode% starting standard par
	\hphantom{.}क‚थ‚न्त‚र्हि स‚हाव (स्था) नं ‚{\color{DodgerBlue3}‚चेतःश‚रीर‚यो}‚रित्याह ॥
	\pend% ending standard par
      
	  \bigskip
	  \begingroup
	
	    \large
	  
	    \begin{quote}
	  
	    
	    \stanza[\smallbreak]
	\label{pv.1.64b}\flagstanza{\tiny\textenglish{....1.64b}}स‚ह‚कारात् स‚ह‚स्थान‚म‚ग्निताम्र‚द्र‚व‚त्व‚व‚त् ॥ ६४ ॥\&[\smallbreak]


	
	    \end{quote}
	  
	  \endgroup
	

	  \pstart \leavevmode% starting standard par
	\hphantom{.}त‚स्य चेत‚सः श‚रीर‚स्य च ‚{\color{DodgerBlue3}‚हेतोः} पूर्व्व‚चित्त‚क्ष‚ण‚स्य क‚ल‚लादेश्च । ‚{\color{DodgerBlue3}‚स‚ह‚कारात्स‚ह-} कार‚णात् ‚{\color{DodgerBlue3}‚कार्य}‚योश्चित्त‚देह‚यो‚{\color{DodgerBlue3}‚र्ज‚न्म‚न} उत्पादात् त‚योः स‚ह‚स्थानं भ‚व‚ति । य‚था‚{\color{DodgerBlue3}‚ग्नि‚{\tiny $_{lb}$}‚ताम्र}‚ज‚न्म‚नोर्व्व‚ह्निताम्र‚द्र‚व‚त्व‚योः स‚हाव‚स्थितिः ॥ (६४)
	\pend% ending standard par
      \label{div_pvv.1.65}
	  
	% new div opening: depth here is 2
	

	  \begin{center}%% label @type='head'
	\textbf{(ii र्स्थितिस्थात्रोर‚व्य‚तिरेकात्)}
	\end{center}
	

	  \pstart \leavevmode% starting standard par
	न‚नु देह‚श्चित्त‚स्याश्र‚य‚स्त‚तः स‚ह‚स्थितिः स्यादित्याह (।)
	\pend% ending standard par
      
	  \bigskip
	  \begingroup
	
	    \large
	  
	    \begin{quote}
	  
	    
	    \stanza[\smallbreak]
	\label{pv.1.65}\flagstanza{\tiny\textenglish{...v.1.65}}अनाश्र‚यात्स‚द‚स‚तोर्नाश्र‚यः स्थितिकार‚ण‚म् ।&स‚त‚श्चेदाश्र‚यो नास्याः स्थातुर‚व्य‚तिरेक‚तः ॥ ६५ ॥\&[\smallbreak]


	
	    \end{quote}
	  
	  \endgroup
	

	  \pstart \leavevmode% starting standard par
	\hphantom{.}‚{\color{DodgerBlue3}‚अना\edtext{}{\edlabel{pvv.33-1}\label{pvv.33-1}\lemma{अना}\Bfootnote{आश्र‚य‚त्वायोगात् ।}} श्र‚यात् स‚द‚स‚तोर्नाश्र‚यः न} ह्य‚कार‚ण‚{\tiny $_{7}$}‚माश्र‚यः । अतिप्र‚स‚ङ्गात् । ‚{\color{DodgerBlue3}‚स‚त‚श्च} \leavevmode\ledsidenote{\textenglish{034/s}} निष्प‚न्न‚त्वात् ना\edtext{}{\edlabel{pvv.34-1}\label{pvv.34-1}\lemma{ना}\Bfootnote{य‚स्य पुन‚र‚दृष्टं किञ्चिद‚प‚रं स‚ह‚कारि नास्ति य‚च्छ‚क्तिक्ष‚याद‚स्थिभि‚{\tiny $_{lb}$}‚श्चैत‚न्ये स्यात् ।}}श्र‚यः क‚श्चित् । अस‚तोऽपि वा कार‚णं किञ्चित् स्याच्\edtext{}{\edlabel{pvv.34-2}\label{pvv.34-2}\lemma{स्याच्}\Bfootnote{एकाहिद‚ष्ट‚योर‚पि काल‚द‚ष्टो म्रिय‚तेऽन्यो जीव‚तीत्य‚स्ति ।}} न त्वाश्र‚यः ‚{\tiny $_{lb}$}‚स्व‚रूप‚स्यैवाभावात् ॥ स‚तः ‚{\color{DodgerBlue3}‚स्थितिकार‚ण}‚माश्र‚य‚श्चेदिष्य‚ते (।) नैत‚द‚पि युक्त‚{\color{DodgerBlue3}‚म‚स्याः ‚{\tiny $_{lb}$}‚स्थातुर‚व्य‚तिरेक‚तः} । न हि स्थितिर्नाम स्थातुः प‚दार्थात् भिन्ना यां कुर्व‚त आश्र‚य‚त्वं । ‚{\tiny $_{lb}$}‚(६५)
	\pend% ending standard par
      \label{div_pvv.1.66_1.67}
	  
	% new div opening: depth here is 2
	

	  \begin{center}%% label @type='head'
	\textbf{(iii न्य‚तिरेकेऽपि नाश्र‚यः)}
	\end{center}
	
	  \bigskip
	  \begingroup
	
	    \large
	  
	    \begin{quote}
	  
	    
	    \stanza[\smallbreak]
	\label{pv.1.66a}\flagstanza{\tiny\textenglish{....1.66a}}व्य‚तिरेकेऽपि त‚द्धेतुस्तेन भाव‚स्य किं कृत‚म् ।\&[\smallbreak]


	
	    \end{quote}
	  
	  \endgroup
	

	  \pstart \leavevmode% starting standard par
	\hphantom{.}‚{\color{DodgerBlue3}‚व्य‚तिरेके}‚ऽपि वा स्वीक्रिय‚माणे ‚{\color{DodgerBlue3}‚त\edtext{}{\edlabel{pvv.34-3}\label{pvv.34-3}\lemma{त}\Bfootnote{स्थितेः ।}}द्धे}‚तुराश्र‚यः स्यात् । ‚{\color{DodgerBlue3}‚तेन} चाश्र‚याभिम‚तेन ‚{\tiny $_{lb}$}‚\leavevmode\ledsidenote{\textenglish{8a/MA}} ‚{\color{DodgerBlue3}‚भाव‚स्य} स्थितिम‚तः ‚{\color{DodgerBlue3}‚किं कृतं‚{\tiny $_{8}$}‚} येनासावाश्र‚यः ।
	\pend% ending standard par
      

	  \begin{center}%% label @type='head'
	\textbf{(iv. नोत्य‚न्न‚या स्थित्या भाव‚स्थाप‚ना)}
	\end{center}
	

	  \pstart \leavevmode% starting standard par
	स्यादेत‚द् (।) भाव‚स‚म्ब‚न्धिनी स्थितिर्भावं स्थाप‚य‚ति तेन त‚त्क‚र्त्तुराश्र‚य‚त्व‚{\tiny $_{lb}$}‚मिर्त्याह (।)
	\pend% ending standard par
      
	  \bigskip
	  \begingroup
	
	    \large
	  
	    \begin{quote}
	  
	    
	    \stanza[\smallbreak]
	\label{pv.1.66b}\flagstanza{\tiny\textenglish{....1.66b}}अविनाश‚प्र‚स‚ङ्गः स नाश‚हेतोर्म‚तो य‚दि ॥ ६६ ॥\&[\smallbreak]


	
	    \end{quote}
	  
	  \endgroup
	
	  \bigskip
	  \begingroup
	
	    \large
	  
	    \begin{quote}
	  
	    
	    \stanza[\smallbreak]
	\label{pv.1.67a}\flagstanza{\tiny\textenglish{....1.67a}}तुल्यः प्र‚स‚ङ्ग‚स्त‚त्रापि;\&[\smallbreak]


	
	    \end{quote}
	  
	  \endgroup
	

	  \pstart \leavevmode% starting standard par
	य‚दि स्थित्योत्प‚न्न‚या भावः स्थाप्य‚ते त‚दान क‚दाचिद‚स्य भा\edtext{}{\edlabel{pvv.34-4}\label{pvv.34-4}\lemma{भा}\Bfootnote{न स्थितिकार‚णात्कुण्डादिवाश्र‚यः किन्तु ।}}व‚स्य विनाशः स्यात् ‚{\tiny $_{lb}$}‚‚{\color{DodgerBlue3}‚स नाश‚हेतो}‚र्मुद्‏ग‚रा‚{\color{DodgerBlue3}‚देर्म‚तो य‚दि} (। ६६) ‚{\color{DodgerBlue3}‚तुल्यः प्र‚स‚ङ्ग‚स्त‚त्रापि} । नाशोऽपि न ‚{\tiny $_{lb}$}‚ताव‚द् भावाद‚व्य‚तिरिक्तः क्रिय‚ते त‚स्योत्प‚न्न‚त्वात् । व्य‚तिरिक्तेऽपि नाशे कृते भाव‚{\tiny $_{lb}$}‚स्त‚द‚व‚स्थ इति प्राग्‏व‚दुप‚ल‚म्भादिप्र‚स‚ङ्गः ।
	\pend% ending standard par
      

	  \pstart \leavevmode% starting standard par
	किञ्च (।)
	\pend% ending standard par
      

	  \begin{center}%% label @type='head'
	\textbf{(v. नाश‚स्य स‚हेतुत्वे स्थितिहेतुर्निष्फ‚लः)}
	\end{center}
	
	  \bigskip
	  \begingroup
	
	    \large
	  
	    \begin{quote}
	  
	    
	    \stanza[\smallbreak]
	\label{pv.1.67b}\flagstanza{\tiny\textenglish{....1.67b}}किं पुनः स्थितिहेतुना ॥&आ नाश‚काग‚मात् स्थानं त‚त‚श्चेद् व‚स्तुध‚र्म‚ता ॥ ६७ ॥\&[\smallbreak]


	
	    \end{quote}
	  
	  \endgroup
	
	  \bigskip
	  \begingroup
	
	    \large
	  
	    \begin{quote}
	  
	    
	    \stanza[\smallbreak]
	\label{pv.1.68a}\flagstanza{\tiny\textenglish{....1.68a}}नाशास्य;\&[\smallbreak]


	
	    \end{quote}
	  
	  \endgroup
	

	  \pstart \leavevmode% starting standard par
	\hphantom{.}य‚दि नाश‚हेतुना नाशः क्रिय‚ते त‚दा ‚{\color{DodgerBlue3}‚किं पुनः स्थितिहेतुना}‚श्र‚येण याव‚न्ना‚{\tiny $_{1}$}‚श‚हे‚{\tiny $_{lb}$}‚तुर्नाप‚त‚ति ताव‚त् स्व‚य‚मेव स्थास्य‚ति । आप‚तिताच्च त‚स्मान्नैष र‚क्ष‚ण‚क्ष‚म इति ‚{\tiny $_{lb}$}‚\leavevmode\ledsidenote{\textenglish{035/s}} किम‚नेन स्वीकृतेन । ‚{\color{DodgerBlue3}‚आ नाश‚काग‚माद्धि नाश‚काग‚म‚नं याव‚त्त‚त आश्र‚यात् स्था‚{\tiny $_{lb}$}‚न}‚ञ्चेदिष्य‚ते । एव‚न्त‚र्हि व‚स्तु‚{\color{DodgerBlue3}‚ध‚र्म‚ता ना\edtext{}{\edlabel{pvv.35-1}\label{pvv.35-1}\lemma{ना}\Bfootnote{य‚दि स्थाप‚कं विना न तिष्ठ‚ति ।}} श‚स्य} । (६७)
	\pend% ending standard par
      \label{div_pvv.1.68}
	  
	% new div opening: depth here is 2
	

	  \pstart \leavevmode% starting standard par
	य‚दि न‚श्व‚रो भावः\edtext{}{\edlabel{pvv.35-2}\label{pvv.35-2}\lemma{भावः}\Bfootnote{नाश‚शीलः स्व‚य‚ञ्चेद् स्यात ।}} त‚दाश्र‚येण नाश‚कोप‚निपातं ‚{\color{DodgerBlue3}‚याव‚त्स्थाप्येतान्य\edtext{}{\edlabel{pvv.35-3}\label{pvv.35-3}\lemma{त्स्थाप्येतान्य}\Bfootnote{अन‚श्व‚र‚त्वे ।}}था} तु स्व‚य‚मेव स्थितिमान् किमाश्र‚येण (।)
	\pend% ending standard par
      
	  \bigskip
	  \begingroup
	
	    \large
	  
	    \begin{quote}
	  
	    
	    \stanza[\smallbreak]
	\label{pv.1.68b}\flagstanza{\tiny\textenglish{....1.68b}}स‚त्य‚बाधोसाविति किं स्थितिहेतुना ॥&य‚था ज‚लादेराधार इति चेत् तुल्य‚म‚त्र च ॥ ६८ ॥\&[\smallbreak]


	
	    \end{quote}
	  
	  \endgroup
	

	  \pstart \leavevmode% starting standard par
	\hphantom{.}एवं ‚{\color{DodgerBlue3}‚स‚त्य\edtext{}{\edlabel{pvv.35-4}\label{pvv.35-4}\lemma{त्य}\Bfootnote{विनाशित्वेऽस्य नाश‚हेतुनापि न किञ्चिन्न‚ख‚व‚त्त्वादेव ।}}बाधोऽसाविति किं स्थितिहेतुना\edtext{}{\edlabel{pvv.35-5}\label{pvv.35-5}\lemma{स्थितिहेतुना}\Bfootnote{भाव‚स्य नाशित्वात् ।}}} विद्य‚माने व‚स्तुनि स्व‚भा\edtext{}{\edlabel{pvv.35-6}\label{pvv.35-6}\lemma{भा}\Bfootnote{कुण्डाद‚यः किं कुर्व्वाणा आश्र‚य इति ।}}व‚त्वात् ‚{\tiny $_{lb}$}‚अबाधो बाध‚र‚हितोसौ नाश इति किं स्थिति‚{\tiny $_{2}$}‚हेतुना स्वीकृतेनापि । ‚{\color{DodgerBlue3}‚य‚था ज‚लादेः} स‚त एवाधारो घ‚टादिस्त‚था चित्त‚स्य देह ‚{\color{DodgerBlue3}‚इति चेत् । तुल्य‚म‚त्र च} प्रागुक्तं ‚{\tiny $_{lb}$}‚स‚क‚लं ॥ (६८)
	\pend% ending standard par
      \label{div_pvv.1.69}
	  
	% new div opening: depth here is 2
	

	  \begin{center}%% label @type='head'
	\textbf{(vi. भाव‚स‚न्त‚तेर्हेतुत्वादाश्र‚य‚त्व‚म्)}
	\end{center}
	

	  \pstart \leavevmode% starting standard par
	क‚थ‚न्त‚र्ह्याधार‚व्य‚व‚हारः स‚म‚र्थ‚नीय इत्या\edtext{}{\edlabel{pvv.35-3-bis}\label{pvv.35-3-bis}\lemma{इत्या}\Bfootnote{अन‚श्व‚र‚त्वे ।}} ह ।
	\pend% ending standard par
      
	  \bigskip
	  \begingroup
	
	    \large
	  
	    \begin{quote}
	  
	    
	    \stanza[\smallbreak]
	\label{pv.1.69}\flagstanza{\tiny\textenglish{...v.1.69}}प्र‚तिक्ष‚ण‚विनाशे हि भावानां भाव‚स‚न्त‚तेः ।&त‚थोत्प‚त्तेः स‚हेतुत्वादाश्र‚योऽयुक्त‚म‚न्य‚था ॥ ६९ ॥\&[\smallbreak]


	
	    \end{quote}
	  
	  \endgroup
	

	  \pstart \leavevmode% starting standard par
	\hphantom{.}भावानां हि विन‚श्व‚र‚स्व‚भाव‚त‚या ‚{\color{DodgerBlue3}‚प्र‚तिक्ष‚ण‚विनाशे} यो भावः स‚ह‚कारी ‚{\tiny $_{lb}$}‚भ‚वास‚न्त‚तेः त‚था तादृश्याः स्वोपादान‚दे\edtext{}{\edlabel{pvv.35-7}\label{pvv.35-7}\lemma{दे}\Bfootnote{आधार‚स्य ।}}शाया ‚{\color{DodgerBlue3}‚उत्प‚त्ते}‚र्न्निमित्तं ‚{\color{DodgerBlue3}‚स‚हेतुत्वादाश्र‚यो} न ‚{\color{DodgerBlue3}‚त्व‚न्य‚था}‚ऽयुक्त‚त्वात् । (६९)
	\pend% ending standard par
      \label{div_pvv.1.70}
	  
	% new div opening: depth here is 2
	

	  \begin{center}%% label @type='head'
	\textbf{(b. i विशेष‚णाश्र‚याश्र‚यिभाव‚दूष‚ण‚म्)}
	\end{center}
	

	  \begin{center}%% label @type='head'
	\textbf{ii गुण‚सामान्य‚क‚र्म‚णां दूष‚ण‚म्}
	\end{center}
	

	  \pstart \leavevmode% starting standard par
	एवं सामान्येनाश्र‚याश्र‚यिभाव‚दूष‚ण‚म‚भिधाय द्र‚व्य‚दूष‚णादौ विशेषे दूष‚णं ‚{\tiny $_{lb}$}‚व‚क्तुमाह ।
	\pend% ending standard par
      
	  \bigskip
	  \begingroup
	
	    \large
	  
	    \begin{quote}
	  
	    
	    \stanza[\smallbreak]
	\label{pv.1.70}\flagstanza{\tiny\textenglish{...v.1.70}}स्यादाधारो ज‚लादीनां ग‚म‚न‚प्र‚तिब‚न्ध‚तः ।&अग‚तीनां किमाधारैर्गुण‚सामान्य‚क‚र्म‚णाम् ॥ ७० ॥\&[\smallbreak]


	
	    \end{quote}
	  
	  \endgroup
	

	  \pstart \leavevmode% starting standard par
	\hphantom{.}स्यादाधारो ‚{\color{DodgerBlue3}‚ज‚लादीनां} प्र‚स‚र्प‚ण‚ध‚र्म‚णां ‚{\color{DodgerBlue3}‚ग‚म‚न‚प्र‚तिब‚न्ध‚तः} कुण्डादिः ‚{\color{DodgerBlue3}‚अग‚ती}‚नां ‚{\tiny $_{lb}$}‚\leavevmode\ledsidenote{\textenglish{036/s}} ‚{\color{DodgerBlue3}‚निष्क्रिय‚त्वात्किमाधारैः} गुणिव्य‚क्त्यादि‚{\color{DodgerBlue3}‚भिर्गुण‚सामान्य‚क‚र्म‚णां} प‚दार्थानां । (७०)
	\pend% ending standard par
      \label{div_pvv.1.71}
	  
	% new div opening: depth here is 2
	

	  \begin{center}%% label @type='head'
	\textbf{(B. स‚म‚वाय‚स‚म‚वायिभाव‚निरासः)}
	\end{center}
	
	  \bigskip
	  \begingroup
	
	    \large
	  
	    \begin{quote}
	  
	    
	    \stanza[\smallbreak]
	\label{pv.1.71}\flagstanza{\tiny\textenglish{...v.1.71}}एतेन स‚म‚वाय‚श्च स‚म‚वायि च कार‚ण‚म् ।&व्य‚व‚स्थित‚त्वं जात्यादेर्निर‚स्त‚म‚न‚पाश्र‚यात् ॥ ७१ ॥\&[\smallbreak]


	
	    \end{quote}
	  
	  \endgroup
	

	  \pstart \leavevmode% starting standard par
	\hphantom{.}‚{\color{DodgerBlue3}‚एतेनाश्र}‚याश्र‚यिभाव‚प्र‚तिषेधेन ‚{\color{DodgerBlue3}‚स‚म‚वा}‚यो\edtext{}{\edlabel{pvv.36-1}\label{pvv.36-1}\lemma{यो}\Bfootnote{अपृथ‚क् सिद्धानां द्र‚व्य‚गुणादीनां ।}} ऽयुत‚सिद्धानामाधार्याधार‚भूतानामि‚{\tiny $_{lb}$}‚हेति प्र‚त्य‚य‚हेतुर्य‚था व्य‚क्तिसामान्य‚योः ‚{\color{DodgerBlue3}‚स‚म‚वायिकार‚ण‚ञ्च} स्व‚स‚म‚वेत‚कार्य‚ज‚न‚कं ‚{\tiny $_{lb}$}‚य‚थात्मादि बुद्ध्यादीनां ‚{\color{DodgerBlue3}‚व्य‚व‚स्थित‚त्वं जात्यादेः} कासुचिदेव व्य‚क्ति‚{\tiny $_{4}$}‚षु\edtext{}{\edlabel{pvv.36-2}\label{pvv.36-2}\lemma{षु}\Bfootnote{ग‚व्येव ।}} गोत्वं ‚{\tiny $_{lb}$}‚व‚र्त‚ते केषु\edtext{}{\edlabel{pvv.36-3}\label{pvv.36-3}\lemma{केषु}\Bfootnote{आदिना ।}} चिदेव च देहाकार‚प‚रिण‚तेषु चैत‚न्य‚मित्या\edtext{}{\edlabel{pvv.36-4}\label{pvv.36-4}\lemma{मित्या}\Bfootnote{आश्र‚य‚निषेधे स‚म‚वायादेव सिद्धेः । अंकुरादीनां वा ।}} दि ‚{\color{DodgerBlue3}‚निर‚स्त‚म‚न‚पाश्र‚यादा-} श्र‚य‚प्र‚तिषेधात् । त‚न्मूल‚त्वाच्चासां व्य‚व‚स्थानां । (७१)
	\pend% ending standard par
      \label{div_pvv.1.72}
	  
	% new div opening: depth here is 2
	

	  \begin{center}%% label @type='head'
	\textbf{(स्थितिस्थाप‚क‚तानिरासे संग्र‚ह‚श्लोकः)}
	\end{center}
	

	  \pstart \leavevmode% starting standard par
	उक्त‚म‚र्थं श्लोक‚त्र‚येण संगृह्ण‚न्नाह ।
	\pend% ending standard par
      
	  \bigskip
	  \begingroup
	
	    \large
	  
	    \begin{quote}
	  
	    
	    \stanza[\smallbreak]
	\label{pv.1.72}\flagstanza{\tiny\textenglish{...v.1.72}}प‚र‚तो भाव‚नाश‚श्चेत् त‚स्य किं स्थितिहेतुना ।&स विन‚श्येद् विनाऽप्य‚न्यैर‚श‚क्ताः स्थितिहेत‚वः ॥ ७२ ॥\&[\smallbreak]


	
	    \end{quote}
	  
	  \endgroup
	

	  \pstart \leavevmode% starting standard par
	\hphantom{.}‚{\color{DodgerBlue3}‚प‚र‚तो} मुद्ग‚रादे‚{\color{DodgerBlue3}‚र्भाव}‚स्यान‚श्व‚र‚स्य ‚{\color{DodgerBlue3}‚नाश‚श्चेत् त‚स्य किं स्थितिहेतुना}‚श्र‚येण ‚{\tiny $_{lb}$}‚स्व‚य‚म‚न‚श्व‚र‚त्वादेव न न‚क्ष्य‚ति । अथ न‚श्व‚र‚स्व‚भावोसौ त‚दा ‚{\color{DodgerBlue3}‚स विन‚श्येत विना‚{\tiny $_{lb}$}‚प्य‚न्यै}‚र्नाश‚हेतुभिरं‚{\color{DodgerBlue3}‚श‚क्ताः स्थितिहेत‚वो} भावं न‚श्व‚रं स्थाप‚यितुं न‚श्व\edtext{}{\edlabel{pvv.36-5}\label{pvv.36-5}\lemma{श्व}\Bfootnote{स‚र्व्वः । नाश‚य‚ति}}व‚र‚{\tiny $_{5}$}‚स्व‚भाव‚{\tiny $_{lb}$}‚स्याव‚श्यं नाशात् । (७२)
	\pend% ending standard par
      \label{div_pvv.1.73}
	  
	% new div opening: depth here is 2
	

	  \pstart \leavevmode% starting standard par
	किञ्च (।)
	\pend% ending standard par
      
	  \bigskip
	  \begingroup
	
	    \large
	  
	    \begin{quote}
	  
	    
	    \stanza[\smallbreak]
	\label{pv.1.73a}\flagstanza{\tiny\textenglish{....1.73a}}स्थितिमान् साश्र‚यः स‚र्वः स‚र्वोत्प‚त्तौ च साश्र‚यः ।\&[\smallbreak]


	
	    \end{quote}
	  
	  \endgroup
	

	  \pstart \leavevmode% starting standard par
	\hphantom{.}‚{\color{DodgerBlue3}‚स्थितिमान् साश्र‚यः स‚र्व्वो} भावः । त‚त्र यो नाम क‚श्चि\edtext{}{\edlabel{pvv.36-6}\label{pvv.36-6}\lemma{श्चि}\Bfootnote{यो नामानित्याश्र‚य‚स्त‚स्याप्य‚न्यो याव‚न्नित्याः प‚र‚माण‚व इत्याह ।}}न्नित्याश्र‚यो य‚था ‚{\tiny $_{lb}$}‚सुखादिरात्माश्रितः । स‚र्व्वः स नित्यं स्थितिमान् स्यात् स्थाप‚क‚स्य स‚दा स्थितेः । ‚{\tiny $_{lb}$}‚क‚श्चिद‚नित्याश्र‚यो य‚था शुक्ल‚त्वादिः कार्य‚द्र‚व्याश्रितः । ‚{\color{DodgerBlue3}‚स‚र्व्वोत्प‚त्ताव}‚स‚र्व्व‚श्चो‚{\tiny $_{lb}$}‚त्प‚द्य‚मानः । ‚{\color{DodgerBlue3}‚साश्र‚य} इति द्र‚व्यादिर‚पि साश्र‚यः । त‚दाश्र‚योपि क‚पालादिः ख‚ण्डो‚{\tiny $_{lb}$}‚व‚य‚विषु स‚म‚वेतः ।
	\pend% ending standard par
      \textsuperscript{\textenglish{037/s}}

	  \pstart \leavevmode% starting standard par
	ते चान्येष्विति याव‚त् प‚र‚माण‚व आश्र‚याव‚ध\edtext{}{\edlabel{pvv.37-1}\label{pvv.37-1}\lemma{ध}\Bfootnote{इति नित्य‚मेव स्थितिः ।}}यः‚{\tiny $_{6}$}‚ तेषां नित्य‚त्वात्त‚दाश्रितं ‚{\tiny $_{lb}$}‚द्व्य‚णुकं नित्य‚मित्य‚न‚या प‚र‚म्प‚र‚या गुणोपि नित्यः सादित्याह (।)
	\pend% ending standard par
      
	  \bigskip
	  \begingroup
	
	    \large
	  
	    \begin{quote}
	  
	    
	    \stanza[\smallbreak]
	\label{pv.1.73b}\flagstanza{\tiny\textenglish{....1.73b}}त‚स्मात् स‚र्व‚स्य भाव‚स्य न विनाशः क‚दाच‚न ॥ ७३ ॥\&[\smallbreak]


	
	    \end{quote}
	  
	  \endgroup
	

	  \pstart \leavevmode% starting standard par
	\hphantom{.}‚{\color{DodgerBlue3}‚त‚स्मात्स‚र्व्व‚स्य भाव‚स्य} बुद्ध्या\edtext{}{\edlabel{pvv.37-2}\label{pvv.37-2}\lemma{बुद्ध्या}\Bfootnote{नित्य‚भूताश्रित‚त्वात् चैत‚न्य‚स्यापि ।}}देः शुक्लादेश्च ‚{\color{DodgerBlue3}‚न विना\edtext{}{\edlabel{pvv.37-3}\label{pvv.37-3}\lemma{विना}\Bfootnote{अस्ति च नाश इति किमाश्र‚य‚स्वीकारेण ।}}शः क‚दाच‚न} प्राप्नोति । (७३)
	\pend% ending standard par
      \label{div_pvv.1.74}
	  
	% new div opening: depth here is 2
	

	  \pstart \leavevmode% starting standard par
	किञ्च (।)
	\pend% ending standard par
      
	  \bigskip
	  \begingroup
	
	    \large
	  
	    \begin{quote}
	  
	    
	    \stanza[\smallbreak]
	\label{pv.1.74}\flagstanza{\tiny\textenglish{...v.1.74}}स्व‚यं विन‚श्व‚रात्मा चेत् त‚स्य कः स्थाप‚कः प‚रः ।&स्व‚यं न न‚श्व‚रात्मा चेत् त‚स्य कः स्थाप‚कः प‚रः ॥ ७४ ॥\&[\smallbreak]


	
	    \end{quote}
	  
	  \endgroup
	

	  \pstart \leavevmode% starting standard par
	\hphantom{.}‚{\color{DodgerBlue3}‚स्व‚य‚म्विन‚श्व}‚रात्मा चेत् भाव‚स्त‚स्य ‚{\color{DodgerBlue3}‚कः स्थाप‚कः प‚र} आश्र‚या‚{\color{DodgerBlue3}‚भिम‚तो न क‚श्चि}‚{\tiny $_{lb}$}‚द‚सामार्थ्यात् । ‚{\color{DodgerBlue3}‚स्व‚यं न न‚श्व‚रात्मा चेत् त‚स्य कः स्थाप‚कः प‚रः} स्व‚य‚म‚विनाशित‚यैव ‚{\tiny $_{lb}$}‚स्थितेर्व्वैय‚र्थ्यात्स्थाप‚क‚स्य । (७४)
	\pend% ending standard par
      \label{div_pvv.1.75}
	  
	% new div opening: depth here is 2
	

	  \begin{center}%% label @type='head'
	\textbf{(c. उपादानोपादेय‚भाव‚निरासः)}
	\end{center}
	

	  \pstart \leavevmode% starting standard par
	पुन‚श्चित्त‚श‚रीर‚योरुपादानोपादेय‚तां निषेद्ध्ु‚{\tiny $_{7}$}‚माह(।)
	\pend% ending standard par
      
	  \bigskip
	  \begingroup
	
	    \large
	  
	    \begin{quote}
	  
	    
	    \stanza[\smallbreak]
	\label{pv.1.75}\flagstanza{\tiny\textenglish{...v.1.75}}बुद्धिव्यापार‚भेदेन निर्ह्रासातिश‚याव‚पि ।&प्र‚ज्ञादेर्भ‚व‚तो देह‚निर्ह्रासातिश‚यौ विना ॥ ७५ ॥\&[\smallbreak]


	
	    \end{quote}
	  
	  \endgroup
	

	  \pstart \leavevmode% starting standard par
	\hphantom{.}‚{\color{DodgerBlue3}‚बुद्धिव्यापार‚भेदेन} न म‚नोज्ञान‚स्याभ्यास‚विशेषेण ‚{\color{DodgerBlue3}‚निर्ह्रासाति}‚श‚यावु\edtext{}{\edlabel{pvv.37-4}\label{pvv.37-4}\lemma{यावु}\Bfootnote{स‚न्मित्रास‚न्मित्र‚संयोगादिना ।}}प‚च‚याप‚{\tiny $_{lb}$}‚च‚या‚{\color{DodgerBlue3}‚व‚पि प्र‚ज्ञादे}‚रादिश‚ब्दान्मैत्रीक‚रुणावैराग्या\edtext{}{\edlabel{pvv.37-5}\label{pvv.37-5}\lemma{रुणावैराग्या}\Bfootnote{नाशः स‚हेतुको निर्हेतुको वा उभ‚य‚था स्थितिहेतुवैफ‚ल्यार्थ ।}}दीनां ‚{\color{DodgerBlue3}‚भ‚व‚तो देह‚स्य निर्ह्रासातिश‚यौ ‚{\tiny $_{lb}$}‚विना} । त‚स्माद् बुद्धिरेवोपादान‚कार‚णं त‚द्विकार‚विकारित्वात् । न देहो विप‚र्य‚{\tiny $_{lb}$}‚यात् । (७५)
	\pend% ending standard par
      \label{div_pvv.1.76}
	  
	% new div opening: depth here is 2
	
	  \bigskip
	  \begingroup
	
	    \large
	  
	    \begin{quote}
	  
	    
	    \stanza[\smallbreak]
	\label{pv.1.76a}\flagstanza{\tiny\textenglish{....1.76a}}इदं दीप‚प्र‚भादीनामाश्रितानां न विद्य‚ते ॥\&[\smallbreak]


	
	    \end{quote}
	  
	  \endgroup
	

	  \pstart \leavevmode% starting standard par
	\hphantom{.}इद‚माश्र‚य‚विकारं विनापि विकारित्वं ‚{\color{DodgerBlue3}‚दीप‚प्र‚भादी\edtext{}{\edlabel{pvv.37-6}\label{pvv.37-6}\lemma{भादी}\Bfootnote{य‚त् प्र‚ज्ञादीनां देहोत्क‚र्षाप‚क‚र्ष‚निर‚पेक्ष‚त्वं बुद्ध्य‚धीन‚त्व‚ञ्च ।}}नामाश्रितानां न विद्य‚ते} त\edtext{}{\edlabel{pvv.37-7}\label{pvv.37-7}\lemma{त}\Bfootnote{त‚न्न दीप‚प्र‚भादीनां त‚न्न देह‚माश्र‚य इत्य‚भिप्रायः ॥}}द्विकार‚विकारित्वात् ।
	\pend% ending standard par
      \textsuperscript{\textenglish{038/s}}\textsuperscript{\textenglish{8b/MA}}

	  \pstart \leavevmode% starting standard par
	अथ देहाद‚पि स्व‚स्थात्प्र‚ज्ञादेरुत्क‚र्षो दृश्य‚त इति‚{\tiny $_{3}$}‚ त‚द्विकार‚विकारित्व‚म‚स्त्ये‚{\tiny $_{lb}$}‚वेत्याह ।
	\pend% ending standard par
      
	  \bigskip
	  \begingroup
	
	    \large
	  
	    \begin{quote}
	  
	    
	    \stanza[\smallbreak]
	\label{pv.1.76b}\flagstanza{\tiny\textenglish{....1.76b}}स्यात् त‚तोऽपि विशेषोऽस्य न चित्तेऽनुप‚कारिणि ॥ ७६ ॥\&[\smallbreak]


	
	    \end{quote}
	  
	  \endgroup
	

	  \pstart \leavevmode% starting standard par
	\hphantom{.}‚{\color{DodgerBlue3}‚स्यात् त‚तो} देहा‚{\color{DodgerBlue3}‚द‚पि विशेषो}‚स्य प्र‚ज्ञादे\edtext{}{\edlabel{pvv.38-1}\label{pvv.38-1}\lemma{ज्ञादे}\Bfootnote{न मुख्यः किन्तु प‚र‚म्प‚र‚या ।}} ‚{\color{DodgerBlue3}‚र्न चित्तेऽनुप‚कारिणि} । चित्तं हि स्व‚स्थ‚{\tiny $_{lb}$}‚देहोप‚कृत‚सौम‚न‚स्य‚म‚भ्यास‚विशेष‚व‚त् प्र‚ज्ञादिक‚मुत्क‚र्ष‚य‚ति । न गुणेपि चित्ते देह ‚{\tiny $_{lb}$}‚ए\edtext{}{\edlabel{pvv.38-2}\label{pvv.38-2}\lemma{ए}\Bfootnote{उत्क‚र्ष‚ति ।}}व । (७६)
	\pend% ending standard par
      \label{div_pvv.1.77}
	  
	% new div opening: depth here is 2
	

	  \pstart \leavevmode% starting standard par
	अमुमेव न्यायं रागादावाह ।
	\pend% ending standard par
      
	  \bigskip
	  \begingroup
	
	    \large
	  
	    \begin{quote}
	  
	    
	    \stanza[\smallbreak]
	\label{pv.1.77}\flagstanza{\tiny\textenglish{...v.1.77}}रागादिवृद्धिः पुष्ट्यादेः क‚दाचित् सुख‚दुःख‚जा ।&त‚योश्च धातुसाम्यादेर‚न्त‚र‚र्थ‚स्य स‚न्निधेः ॥ ७७ ॥\&[\smallbreak]


	
	    \end{quote}
	  
	  \endgroup
	

	  \pstart \leavevmode% starting standard par
	या\edtext{}{\edlabel{pvv.38-3}\label{pvv.38-3}\lemma{या}\Bfootnote{बुद्ध्युत्क‚र्षादिनिर‚पेक्षः केव‚लो देहो रागादिनिमित्त‚मित्याह ।}}पि ‚{\color{DodgerBlue3}‚रागादिवृद्धिः पुष्ट\edtext{}{\edlabel{pvv.38-4}\label{pvv.38-4}\lemma{पुष्ट}\Bfootnote{धातुसाम्याद् वैष‚म्याच्चान्तःस्प्र‚ष्ट‚व्य‚विशेषेण काय‚विज्ञान‚म‚नुगृह्य‚ते विक्रिय‚ते च त‚द्विक‚ल्पं ज‚न‚य‚ति त‚तो राग‚द्वेषौ प‚र‚म्प‚र‚या कायात् ।}}यादेः} सापि न स‚र्व्व‚दापि तु ‚{\color{DodgerBlue3}‚क‚दाचित्} प्र‚कृत्या ‚{\tiny $_{lb}$}‚म‚न्द‚राग‚स्य प्र‚तिसंख्यान‚ब‚लिन‚श्च पुष्ट‚स्यापि रागावृद्धेः । य‚दापि भ‚व‚ति त‚दापि न ‚{\tiny $_{lb}$}‚केव‚लात्पुष्ट्या‚{\tiny $_{1}$}‚देः किन्तु ‚{\color{DodgerBlue3}‚सुख‚दुःख‚जा} । सुखाद्रागः दुःखाद् द्वेषः इति न चित्त‚निर‚पेक्षो ‚{\tiny $_{lb}$}‚रागादिहेतुर्देहः । ‚{\color{DodgerBlue3}‚त‚योश्च} सुख‚दुःख‚यो‚{\color{DodgerBlue3}‚र्द्धातुसाम्यादेर‚न्त‚र‚र्थ‚स्या}‚नुग्राह‚क‚स्यान्तः ‚{\tiny $_{lb}$}‚स्प्र‚ष्ट‚व्य‚विशेष‚स्यान्त‚र‚स्य स्प‚र्श‚ज्ञान‚विष‚यीकृत‚स्य ‚{\color{DodgerBlue3}‚स‚न्निधेर्ज‚न्म} । सुख‚दुःख‚ज्ञाने अपि ‚{\tiny $_{lb}$}‚विशिष्ट‚विष‚य‚पूर्व्व‚ज्ञान‚सापेक्षे एव न देह‚मात्र‚ज‚न्ये इत्य‚र्थः । (७७)
	\pend% ending standard par
      \label{div_pvv.1.78}
	  
	% new div opening: depth here is 2
	
	  \bigskip
	  \begingroup
	
	    \large
	  
	    \begin{quote}
	  
	    
	    \stanza[\smallbreak]
	\label{pv.1.78}\flagstanza{\tiny\textenglish{...v.1.78}}एतेन स‚न्निपातादेः स्मृतिभ्रंशाद‚यो ग‚ताः ।&विकार‚य‚ति धीरेव ह्य‚न्त‚र‚र्थ‚विशेष‚जा ॥ ७८ ॥\&[\smallbreak]


	
	    \end{quote}
	  
	  \endgroup
	

	  \pstart \leavevmode% starting standard par
	\hphantom{.}‚{\color{DodgerBlue3}‚एते}‚नान्त‚रोक्त‚न्यायेन ‚{\color{DodgerBlue3}‚स‚न्निपातादे}‚रादिग्र‚ह‚णाज्ज्व‚रादेः ‚{\color{DodgerBlue3}‚स्मृतिभ्रंशाद‚यो ग‚{\tiny $_{2}$}‚ता} व्याख्याताः । धीरेव हि पूर्व्विकाऽ‚{\color{DodgerBlue3}‚न्त‚र‚र्थ‚विशेषा}‚द्धातुवैष‚म्याज्जाता त‚द्‏ग्राहि\edtext{}{\edlabel{pvv.38-5}\label{pvv.38-5}\lemma{ग्राहि}\Bfootnote{वैष‚म्योप‚प्लुता}}णीं ‚{\tiny $_{lb}$}‚चित्त‚स‚न्त‚तिं ‚{\color{DodgerBlue3}‚विकार‚य‚ति} स्मृतिप्र‚मोषाद्युप‚ह‚तां क‚रोति । (७८)
	\pend% ending standard par
      \label{div_pvv.1.79}
	  
	% new div opening: depth here is 2
	
	  \bigskip
	  \begingroup
	
	    \large
	  
	    \begin{quote}
	  
	    
	    \stanza[\smallbreak]
	\label{pv.1.79}\flagstanza{\tiny\textenglish{...v.1.79}}शार्दूल‚शोणितादीनां स‚न्तानातिश‚ये क्व‚चित् ।&मोहाद‚यः स‚म्भ‚व‚न्ति श्र‚व‚णेक्ष‚ण‚तो य‚था ॥ ७९ ॥\&[\smallbreak]


	
	    \end{quote}
	  
	  \endgroup
	\textsuperscript{\textenglish{039/s}}

	  \pstart \leavevmode% starting standard par
	\hphantom{.}य‚था ‚{\color{DodgerBlue3}‚शार्दू\edtext{}{\edlabel{pvv.39-1}\label{pvv.39-1}\lemma{शार्दू}\Bfootnote{देह‚श्चित्त‚मुप‚क‚रोति चित्तं प्र‚ज्ञादीन् । चित्तेऽनुप‚कारिणि स‚ति तु न विशेषः । व्याघ्र इहेति श्रुत्वा बिभेति ।}} ल‚शोणितादीनां} य‚थाक्र‚मं ‚{\color{DodgerBlue3}‚श्र‚व‚णेक्ष‚ण‚तः स‚न्तानातिश‚ये क्व‚चिद्} भीरुत‚मे ‚{\color{DodgerBlue3}‚मोहाद‚य} आदिश‚ब्दाद् भ‚य‚रोम‚ह‚र्षाद‚यो विष‚य‚विकृत‚बुद्धिद्वारेणैव ‚{\tiny $_{lb}$}‚‚{\color{DodgerBlue3}‚स‚म्भ‚व‚न्ति} । न हि मोहादीनां शार्दूल‚शोणिताद‚य उपादान‚कार‚णानि किन्तु विष‚याः ‚{\tiny $_{lb}$}‚सं‚{\tiny $_{3}$}‚न्तः प‚र‚म्प‚र‚योप‚कार‚काः त‚था राग‚स्मृतिभ्रंशाद‚योपि बोद्ध‚व्याः । (७९)
	\pend% ending standard par
      \label{div_pvv.1.80}
	  
	% new div opening: depth here is 2
	
	  \bigskip
	  \begingroup
	
	    \large
	  
	    \begin{quote}
	  
	    
	    \stanza[\smallbreak]
	\label{pv.1.80}\flagstanza{\tiny\textenglish{...v.1.80}}त‚स्मात् स्व‚स्यैव संस्कारं निय‚मेनानुव‚र्त‚ते ।&त‚न्नान्त‚रीय‚कं चित्त‚म‚त‚श्चित्त‚स‚माश्रित‚म् ॥ ८० ॥\&[\smallbreak]


	
	    \end{quote}
	  
	  \endgroup
	

	  \pstart \leavevmode% starting standard par
	\hphantom{.}‚{\color{DodgerBlue3}‚त‚स्मा}‚त् स्व‚{\color{DodgerBlue3}‚स्यैव संस्कारं निय‚मेनानुव\edtext{}{\edlabel{pvv.39-2}\label{pvv.39-2}\lemma{मेनानुव}\Bfootnote{स‚मान‚जातीय‚विक‚ल्प‚विज्ञान‚स्य पूर्व‚क‚स्य ।}}र्त‚ते । त‚न्नान्त‚रीय‚कं चित्त}‚मेषित‚व्य‚{\tiny $_{lb}$}‚‚{\color{DodgerBlue3}‚म‚त‚श्चित्त‚माश्रितं} चित्तं चित्त‚सँस्कार‚स्यैवानुव‚र्त‚नात् । देह‚संस्कार‚न्तु व्य‚भि‚{\tiny $_{lb}$}‚च‚र‚ति प्र‚तिसंख्यान‚ब‚लिनामित्युक्तं । (८०)
	\pend% ending standard par
      \label{div_pvv.1.81}
	  
	% new div opening: depth here is 2
	

	  \pstart \leavevmode% starting standard par
	चेतःश‚रीर‚योर्भेद‚प‚क्षेणाश्र‚याश्र‚यिभावो ऽयुक्त इति प्र‚तिपाद्य श‚क्तिप‚क्षेप्य‚{\tiny $_{lb}$}‚भेदात्म‚के दोष‚माह (।)
	\pend% ending standard par
      
	  \bigskip
	  \begingroup
	
	    \large
	  
	    \begin{quote}
	  
	    
	    \stanza[\smallbreak]
	\label{pv.1.81}\flagstanza{\tiny\textenglish{...v.1.81}}य‚था श्रुतादिसंस्कारः कृत‚श्चेत‚सि चेत‚सि ।&कालेन व्य‚ज्य‚तेऽभेदात् स्याद् देहेपि त‚तो गुणः ॥ ८१ ॥\&[\smallbreak]


	
	    \end{quote}
	  
	  \endgroup
	

	  \pstart \leavevmode% starting standard par
	\hphantom{.}‚{\color{DodgerBlue3}‚य‚था श्रुतादिसं‚{\tiny $_{4}$}‚स्कारः कृत‚श्चेत‚सि} पुन‚र्य‚था-प्र‚बोध‚प्र‚त्य‚यं चेत‚सि ‚{\color{DodgerBlue3}‚कालेन क्र‚म-} भाविना ‚{\color{DodgerBlue3}‚व्य‚ज्य‚ते} त‚था ‚{\color{DodgerBlue3}‚अभेदाच्चित्त}‚श‚रीर‚योः ‚{\color{DodgerBlue3}‚स्याद् देहेपि त‚तः} संस्कार‚प्र‚बोध‚कात् ‚{\tiny $_{lb}$}‚प्र‚त्य‚यात् गुणोऽभिव्य‚क्तः (।) त‚त‚श्च देहं प‚श्य‚ता ‚{\color{DodgerBlue3}‚श्रुतादिसंस्कारोपि} त‚दात्म‚भूतो ‚{\tiny $_{lb}$}‚दृश्यो दृश्येत ।\edtext{\textsuperscript{*}}{\edlabel{pvv.39-3}\label{pvv.39-3}\lemma{*}\Bfootnote{न च दृश्य‚ते । श्रुतादिसंस्कारेण संस्क्रिय‚माणेपि म‚नोविज्ञाने म‚नो न देह‚संस्कारः ।}}त‚स्माद् देह‚स्याश्र‚य‚त्व‚प्र‚तिषेधात् त‚द्विनाशे चित्त‚विनाशो नेति ‚{\tiny $_{lb}$}‚ज‚न्म‚प‚र‚म्प‚रासु युक्तः कृपाभ्यासः । (८१)
	\pend% ending standard par
      \label{div_pvv.1.82}
	  
	% new div opening: depth here is 2
	

	  \begin{center}%% label @type='head'
	\textbf{(ग) पुन‚र्ज‚न्म‚प‚रिग्र‚हः}
	\end{center}
	

	  \pstart \leavevmode% starting standard par
	क‚थं पुन‚र्ज‚न्म‚प‚रिग्र‚ह इत्याह (।)
	\pend% ending standard par
      
	  \bigskip
	  \begingroup
	
	    \large
	  
	    \begin{quote}
	  
	    
	    \stanza[\smallbreak]
	\label{pv.1.82}\flagstanza{\tiny\textenglish{...v.1.82}}अन‚न्य‚स‚त्व‚नेय‚स्य हीन‚स्थान‚प‚रिग्र‚हः ।&आत्म‚स्नेह‚व‚तो दुःख‚सुख‚त्यागाप्तिवाञ्छ‚या ॥ ८२ ॥\&[\smallbreak]


	
	    \end{quote}
	  
	  \endgroup
	\textsuperscript{\textenglish{040/s}}

	  \pstart \leavevmode% starting standard par
	\hphantom{.}‚{\color{DodgerBlue3}‚अन‚न्य‚{\tiny $_{5}$}‚स‚त्व‚नेय‚स्य} ईश्व‚र‚प्र‚तिषेधात् । ‚{\color{DodgerBlue3}‚हीन‚स्थान‚प\edtext{}{\edlabel{pvv.40-1}\label{pvv.40-1}\lemma{प}\Bfootnote{अप‚राधीन‚स्य म‚क्षिकाणाम‚शुचिस्थान‚ग्र‚ह‚कामिनां स्त्रीकुण‚प‚श‚रीरादिप‚रिग्र‚ह‚व‚त् ।}}रिग्र‚हो} ग‚र्भ‚स्याश्र‚य‚त्वेन ‚{\tiny $_{lb}$}‚स्वीकारः ‚{\color{DodgerBlue3}‚आत्म‚स्नेह‚व‚तः} स‚तृष्ण‚स्य दुःखे सुख‚मिति विप‚र्यासः । त‚स्य ‚{\color{DodgerBlue3}‚दुःख‚सुख‚यो‚{\tiny $_{lb}$}‚र्य‚थाक्र‚मं त्यागाप्तिवाञ्छ‚या} । स‚तृष्णो हि दुःखे सुख‚मिति विप‚र्य‚स्तः आत्म‚नि ‚{\tiny $_{lb}$}‚स्निग्धो ज‚न्माक्षेप‚क‚क‚र्म‚व‚शात् सुख‚हेतुं ग‚र्भ‚स्थानं म‚न्य‚मानः प‚रिगृह्णाति । (८२)
	\pend% ending standard par
      \label{div_pvv.1.83}
	  
	% new div opening: depth here is 2
	

	  \begin{center}%% label @type='head'
	\textbf{I. अविद्या-तृष्णे ब‚न्ध‚कार‚ण‚म्}
	\end{center}
	

	  \pstart \leavevmode% starting standard par
	त‚त‚श्च (।)
	\pend% ending standard par
      
	  \bigskip
	  \begingroup
	
	    \large
	  
	    \begin{quote}
	  
	    
	    \stanza[\smallbreak]
	\label{pv.1.83}\flagstanza{\tiny\textenglish{...v.1.83}}दुःखे विप‚र्यास‚म‚तिः तृष्णा चाब‚न्ध‚कार‚ण‚म् ।&ज‚न्मिनो य‚स्य ते न स्तो न स ज‚न्माधिग‚च्छ‚ति ॥ ८३ ॥\&[\smallbreak]


	
	    \end{quote}
	  
	  \endgroup
	

	  \pstart \leavevmode% starting standard par
	\hphantom{.}‚{\color{DodgerBlue3}‚दुः\edtext{}{\edlabel{pvv.40-2}\label{pvv.40-2}\lemma{दुः}\Bfootnote{दुःखे ग‚र्भादिस्थानेभिर‚तिः सुख‚मेत‚दिति ।}}खे विप‚र्यास‚म‚तिस्तृष्णा चाब‚न्ध‚कार‚णं} आश्लेष‚हेतु‚{\color{DodgerBlue3}‚र्ज‚न्मिनः}‚{\tiny $_{6}$}‚ । तृष्ण‚या ‚{\tiny $_{lb}$}‚आत्म‚स्नेहोप्याक्षिप्तो हेतुव‚द्वेदित‚व्यः । य‚स्य तून्मू\edtext{}{\edlabel{pvv.40-3}\label{pvv.40-3}\lemma{तून्मू}\Bfootnote{प‚रिण‚तोऽभिर‚तिपुरःस‚रः प्र‚पात‚पातादिविल‚क्ष‚णः ।}}लितात्म‚ग्र‚ह‚स्य ‚{\color{DodgerBlue3}‚ते} विप‚र्यास‚स्तृष्णा ‚{\tiny $_{lb}$}‚‚{\color{DodgerBlue3}‚च न स्तो} न विद्येते ‚{\color{DodgerBlue3}‚न स ज‚न्माधिग‚च्छ‚ति} ॥ (८३)
	\pend% ending standard par
      \label{div_pvv.1.84}
	  
	% new div opening: depth here is 2
	

	  \begin{center}%% label @type='head'
	\textbf{II. ग‚त्याग‚त्योर‚द‚र्श‚नं इन्द्रियापाट‚वात्}
	\end{center}
	
	  \bigskip
	  \begingroup
	
	    \large
	  
	    \begin{quote}
	  
	    
	    \stanza[\smallbreak]
	\label{pv.1.84}\flagstanza{\tiny\textenglish{...v.1.84}}ग‚त्याग‚ती न दृष्टे चेदिन्द्रियाणाम‚पाट‚वात् ।&अदृष्टिर्म‚न्द‚नेत्र‚स्य त‚नुधूमाग‚तिर्य‚था ॥ ८४ ॥\&[\smallbreak]


	
	    \end{quote}
	  
	  \endgroup
	

	  \pstart \leavevmode% starting standard par
	\hphantom{.}भाविज‚न्म‚न्य‚तीताच्च ज‚न्म‚नो य‚थाक्र‚मं ‚{\color{DodgerBlue3}‚ग‚त्याग‚ती न दृष्टे चेत् । इन्द्रिया‚{\tiny $_{lb}$}‚णाम‚पाट‚वात्सा अदृष्टि}‚र‚द‚र्श‚नं ते सूक्ष्म‚स्यान्त‚राभ‚व‚श‚रीर‚स्य । ‚{\color{DodgerBlue3}‚म‚न्द‚नेत्र\edtext{}{\edlabel{pvv.40-4}\label{pvv.40-4}\lemma{नेत्र}\Bfootnote{किमिवेत्याह उप‚ह‚त‚च‚क्षुषः ।}}स्य} पुंस‚स्त\edtext{}{\edlabel{pvv.40-5}\label{pvv.40-5}\lemma{स्त}\Bfootnote{विर‚लः ।}} ‚{\color{DodgerBlue3}‚न‚धू}‚म‚स्याग‚तिर‚द‚र्श‚नं ‚{\color{DodgerBlue3}‚य‚था} । न ह्य‚दृश्य‚स्याद‚र्श‚नाद‚भावः । (८४)
	\pend% ending standard par
      \label{div_pvv.1.85}
	  
	% new div opening: depth here is 2
	

	  \begin{center}%% label @type='head'
	\textbf{III. मूर्त्त‚स्याऽपि मूर्त्ते प्र‚वेशः}
	\end{center}
	\textsuperscript{\textenglish{9a/MA}}

	  \pstart \leavevmode% starting standard par
	न‚नु मूर्त्तं न मूर्त्तान्त‚र‚{\tiny $_{7}$}‚म‚नुप्र‚विश‚ति प्र‚तिघातात्(।) मूर्त्त‚ञ्चान्त‚राभ‚व‚श‚रीर‚{\tiny $_{lb}$}‚मित्य\edtext{}{\edlabel{pvv.40-6}\label{pvv.40-6}\lemma{मित्य}\Bfootnote{क‚थं ग‚र्भ प्र‚विश‚ति ।}}त आह (।)
	\pend% ending standard par
      
	  \bigskip
	  \begingroup
	
	    \large
	  
	    \begin{quote}
	  
	    
	    \stanza[\smallbreak]
	\label{pv.1.85}\flagstanza{\tiny\textenglish{...v.1.85}}त‚नुत्वान्मूर्त‚म‚पि तु किञ्चित् क्व‚चिद‚श‚क्तिम‚त् ।&ज‚ल‚व‚त् सूत‚व‚द्धेम्नि नादृष्टेनास‚देव वा ॥ ८५ ॥\&[\smallbreak]


	
	    \end{quote}
	  
	  \endgroup
	\textsuperscript{\textenglish{041/s}}

	  \pstart \leavevmode% starting standard par
	\hphantom{.}‚{\color{DodgerBlue3}‚त‚नुत्वान्मूर्त्त‚म‚पि तु किञ्चित्क्व‚चि}\edtext{\textsuperscript{*}}{\edlabel{pvv.41-1}\label{pvv.41-1}\lemma{*}\Bfootnote{प्र‚विश‚ति ।}}न्मूर्त्ते‚{\color{DodgerBlue3}‚ऽश‚क्तिम‚द}‚प्र‚तिघात‚व‚त् ‚{\color{DodgerBlue3}‚ज‚ल‚व‚त्} घ‚टे ‚{\tiny $_{lb}$}‚‚{\color{DodgerBlue3}‚सूत‚व‚द्} हेम्नि । ज‚ल‚सूतौ हि मूर्त्ताव‚पि घ‚ट‚हेम्नी भिन्द‚न्तौ दृश्ये\edtext{}{\edlabel{pvv.41-2}\label{pvv.41-2}\lemma{दृश्ये}\Bfootnote{सूर्य‚र‚श्म‚य‚श्च स्फ‚टिकं भित्वेन्ध‚नं विश‚न्तीत्य‚नेकान्तः ।}}तेऽ(?अ)न्त‚रा‚{\tiny $_{lb}$}‚भ‚व‚श‚रीर‚म‚श‚क्तिम‚त् न दृष्ट‚मिति चेत् ‚{\color{DodgerBlue3}‚नाद}‚ष्टेना‚{\color{DodgerBlue3}‚स‚देव} वा भ‚व‚ति तादृश‚म‚न्त‚राभ‚व‚{\tiny $_{lb}$}‚श‚रीरं । (८५)
	\pend% ending standard par
      \label{div_pvv.1.86}
	  
	% new div opening: depth here is 2
	

	  \begin{center}%% label @type='head'
	\textbf{IV. नोपादान‚भूतं श‚रीरं बुद्धेः}
	\end{center}
	

	  \begin{center}%% label @type='head'
	\textbf{A. अव‚य‚विनिरासः}
	\end{center}
	

	  \pstart \leavevmode% starting standard par
	किञ्च (।) श‚रीर‚मुपादानं बुद्धेर्भ‚व‚दे\edtext{}{\edlabel{pvv.41-3}\label{pvv.41-3}\lemma{दे}\Bfootnote{निर‚व‚य‚वं ।}}क‚म‚व‚य‚विरूपं वा स्यात् । अनेकं प‚र‚{\tiny $_{lb}$}‚माणुस‚ञ्च‚य‚स्व‚भावं‚{\tiny $_{1}$}‚ वा । त‚त्राव‚य‚विरूपं किम‚व‚य‚वा एव ह‚स्ताद‚य उत तेभ्योऽन्य‚त् ‚{\tiny $_{lb}$}‚(।) द्व‚य‚म‚पि प्र‚तिषेद्ध्ुमाह\edtext{}{\edlabel{pvv.41-4}\label{pvv.41-4}\lemma{तिषेद्ध्ुमाह}\Bfootnote{एक‚ञ्चेत्}} ।
	\pend% ending standard par
      
	  \bigskip
	  \begingroup
	
	    \large
	  
	    \begin{quote}
	  
	    
	    \stanza[\smallbreak]
	\label{pv.1.86}\flagstanza{\tiny\textenglish{...v.1.86}}पाण्यादिक‚म्पे स‚र्व‚स्य क‚म्प‚प्राप्तेर्विरोधिनः ।&एक‚स्मिन् क‚र्म‚णोऽयोगात् स्यात् पृथ‚क् सिद्धिर‚न्य‚था ॥ ८६ ॥\&[\smallbreak]


	
	    \end{quote}
	  
	  \endgroup
	

	  \pstart \leavevmode% starting standard par
	\hphantom{.}‚{\color{DodgerBlue3}‚पाण्यादिक‚म्पे} स‚र्व्व‚स्य क‚म्प‚प्राप्तेः । य‚दि पाण्याद‚योऽव‚य‚वा एव अव‚य‚व्येक‚{\tiny $_{lb}$}‚रूपः त‚दा पाण्यादेः क‚म्पे स‚ति स‚र्व्व‚स्य पादादेर‚पि क‚म्पः प्राप्नोति । ‚{\color{DodgerBlue3}‚एक}‚स्मिँस्त‚{\tiny $_{lb}$}‚स्मिन् ‚{\color{DodgerBlue3}‚क‚र्म‚णः} क‚म्प‚स्य विरोधिनोऽक‚म्प‚स्या‚{\color{DodgerBlue3}‚योगात्} । एक‚न्तु द्र‚व्यं त‚त्स‚म‚वेत‚श्च ‚{\tiny $_{lb}$}‚क‚म्प इति स‚र्व्वं क‚म्पेत । अव‚य‚वानामेकाव‚य‚विरूप‚त्वं हेतुः प‚राभ्युप‚ग‚म‚सिद्धः । ‚{\tiny $_{lb}$}‚स‚र्व्व‚स्य क‚म्प‚प्र‚स‚ङ्गः । न च क‚म्पोस्ति इति साध्याभावेनैकाव‚य‚विरूप‚त्वाभाव‚{\tiny $_{lb}$}‚प्र‚स‚ङ्ग‚विप‚र्य‚यः । एव‚म्व‚क्ष्य‚माणाव‚पि प्र‚स‚ङ्ग‚त‚द्विप‚र्य‚यौ वेदित‚व्यौ ॥ ‚{\color{DodgerBlue3}‚अथाव‚य‚वेभ्यो} भिन्नोऽव‚य‚वी । अत एवैक‚स्मिन्न‚व‚य‚वे क‚म्प‚माने नाव‚य‚वान्त‚र‚स्य क‚म्पः त‚दापि ‚{\color{DodgerBlue3}‚स्यात् ‚{\tiny $_{lb}$}‚पृथ}‚क्‚{\color{DodgerBlue3}‚‏सिद्धिर‚न्य}‚थाऽव‚य‚वाव‚य‚विनोर्भेदे पृथ‚क्‏क‚म्प‚मानाद‚व‚य‚वाद‚क‚म्प‚मान‚स्याव‚य‚विनः ‚{\tiny $_{lb}$}‚स‚म‚वेत‚स्य‚{\tiny $_{3}$}‚ भेदेन त‚त्रै\edtext{}{\edlabel{pvv.41-5}\label{pvv.41-5}\lemma{त्रै}\Bfootnote{य‚था एक‚देश‚ल‚ग्न‚मुद‚कं व‚स्त्रैक‚देश एव दृश्य‚ते त‚द्व‚त् ।}}वाव‚य‚वे सिद्धिः स्यात् व‚स्त्रो\edtext{}{\edlabel{pvv.41-6}\label{pvv.41-6}\lemma{स्त्रो}\Bfootnote{पृथ‚ग्‏भावः स्याद‚व‚य‚वाव‚य‚विनोः अव‚य‚वी भेदेन दृश्येत ।}}द‚क‚व‚त् । अत्राप्य‚व‚य‚वाव‚य‚{\tiny $_{lb}$}‚विनोर्भेदः प‚राभ्युप‚ग‚म‚सिद्धो हेतुः । पृथ‚क्‏सिद्धिः प्र‚स‚ज्य‚ते । साध्याभावे साध\edtext{}{\edlabel{pvv.41-7}\label{pvv.41-7}\lemma{साध}\Bfootnote{स‚र्वाक‚म्पार्थं पृथ‚क् स्वीकृतिः पृभ‚ग‚व‚य‚व्य‚सिद्धौ अव‚य‚वाव‚य‚विभेद‚स्य साध‚न‚स्याभावः । त‚तः स‚र्व्व‚क‚म्प‚प्र‚स‚ङ्ग‚स्याप‚रिहारः ।}}ना‚{\tiny $_{lb}$}‚भावो विप‚र्य‚यः । एव‚म्व‚क्ष्य‚माणौ च प्र‚स‚ङ्ग‚विप‚र्य‚यौ । (८६)
	\pend% ending standard par
      \label{div_pvv.1.87}
	  
	% new div opening: depth here is 2
	
	  \bigskip
	  \begingroup
	
	    \large
	  
	    \begin{quote}
	  
	    
	    \stanza[\smallbreak]
	\label{pv.1.87}\flagstanza{\tiny\textenglish{...v.1.87}}एक‚स्य चावृतौ स‚र्व‚स्यावृतिः स्याद‚नावृतौ ।&दृश्येत र‚क्ते चैक‚स्मिन् रागोऽर‚क्त‚स्य वाऽग‚तिः ॥ ८७ ॥\&[\smallbreak]


	
	    \end{quote}
	  
	  \endgroup
	\textsuperscript{\textenglish{042/s}}

	  \pstart \leavevmode% starting standard par
	\hphantom{.}अथाभेद‚प‚क्षे ‚{\color{DodgerBlue3}‚एक‚स्या}‚व‚य‚व‚स्या‚{\color{DodgerBlue3}‚वृतौ स‚र्व्व‚स्यावृतिश्च स्यादि}‚ति प्र‚स‚ङ्गः । ‚{\tiny $_{lb}$}‚भेद‚प‚क्ष‚माश्रित्या‚{\color{DodgerBlue3}‚नावृ}‚तौ चाव‚य‚विनः स्वीक्रिय‚माणायामावृत एवाव‚य‚वेऽनावृतोसौ‚{\tiny $_{4}$}‚ ‚{\tiny $_{lb}$}‚दृश्येतेति प्र‚स‚ङ्गः । अथाभेद‚प‚क्षे ‚{\color{DodgerBlue3}‚र‚क्ते चैक‚स्मिन्न}‚व‚य‚वे स‚र्व्व‚त्राव‚य‚वे ‚{\color{DodgerBlue3}‚रागो} दृश्येतेति ‚{\tiny $_{lb}$}‚प्र‚स‚ङ्गः । भेद‚प‚क्षे तु र‚क्त एवाव‚य‚वेऽ‚{\color{DodgerBlue3}‚र‚क्त‚स्य} चाव‚य‚विनो ‚{\color{DodgerBlue3}‚वाऽग‚तिः} स्यादिति ‚{\tiny $_{lb}$}‚‚{\color{DodgerBlue3}‚प्र‚स‚ङ्गः} । (८७)
	\pend% ending standard par
      \label{div_pvv.1.88_1.89}
	  
	% new div opening: depth here is 2
	

	  \pstart \leavevmode% starting standard par
	स‚र्व्व‚त्र साध्याभावेन साध‚नाभावः प्र‚स‚ङ्ग‚विप‚र्य‚यः । त‚माह (।)
	\pend% ending standard par
      
	  \bigskip
	  \begingroup
	
	    \large
	  
	    \begin{quote}
	  
	    
	    \stanza[\smallbreak]
	\label{pv.1.88a}\flagstanza{\tiny\textenglish{....1.88a}}नास्त्येक‚स‚मुदायोऽस्माद‚नेक‚त्वेऽपि पूर्व‚व‚त् ।\&[\smallbreak]


	
	    \end{quote}
	  
	  \endgroup
	

	  \pstart \leavevmode% starting standard par
	\hphantom{.}‚{\color{DodgerBlue3}‚नास्त्येक}‚स्मिन् नास्त्येकोऽव‚य‚वी ‚{\color{DodgerBlue3}‚स‚मुदा}‚योऽव‚य‚वाना‚{\color{DodgerBlue3}‚म‚स्मा}‚त्क‚म्पादिसाध्या‚{\tiny $_{lb}$}‚भावात् । अ\edtext{}{\edlabel{pvv.42-1}\label{pvv.42-1}\lemma{अ}\Bfootnote{आवृताव‚य‚वाद‚व‚य‚विभेद‚प‚क्षेपि । अव‚य‚विद‚र्श‚न‚प्र‚स‚ङ्ग‚स्य साध्य‚स्याभावेनावृताद् भेद‚स्य साध‚न‚स्याभावो विप‚र्य‚यः । त‚त‚श्चाभेद‚प‚क्ष इव स‚र्व्वाव‚र‚णादि स्यात् ।}}व‚य‚वाव‚य‚विनोर‚{\color{DodgerBlue3}‚नेक‚त्वेपि पूर्व्व‚व‚त्} । अभेद‚प‚क्ष इव साध्याभा‚{\tiny $_{5}$}‚वेन ‚{\tiny $_{lb}$}‚साध‚नाभावो विप‚र्य‚यः ।
	\pend% ending standard par
      
	  \bigskip
	  \begingroup
	
	    \large
	  
	    \begin{quote}
	  
	    
	    \stanza[\smallbreak]
	\label{pv.1.88b}\flagstanza{\tiny\textenglish{....1.88b}}अविशेषाद‚णुत्वाच्च न ग‚तिश्चेन्न सिध्य‚ति ॥ ८८ ॥\&[\smallbreak]


	
	    \end{quote}
	  
	  \endgroup
	
	  \bigskip
	  \begingroup
	
	    \large
	  
	    \begin{quote}
	  
	    
	    \stanza[\smallbreak]
	\label{pv.1.89a}\flagstanza{\tiny\textenglish{....1.89a}}अविशेषः;\&[\smallbreak]


	
	    \end{quote}
	  
	  \endgroup
	

	  \pstart \leavevmode% starting standard par
	अथ श‚रीरादौ प्र‚त्य‚क्ष‚दृष्टे ध‚र्मिणि क‚म्पाक‚म्पादिविरुद्ध‚ध‚र्म्माध्या\edtext{}{\edlabel{pvv.42-2}\label{pvv.42-2}\lemma{र्म्माध्या}\Bfootnote{अत्रोद्योत‚क‚राद‚य आहुर्य‚दि}}सात् । ‚{\tiny $_{lb}$}‚स्व‚त‚न्त्र\edtext{}{\edlabel{pvv.42-3}\label{pvv.42-3}\lemma{न्त्र}\Bfootnote{सिद्धान्त ।}}हेतोरेक‚त्व‚प्र‚तिषेधः साध‚नीय इति । अव‚य‚विनोऽभावात् प‚र‚माणुपुंज‚रूपं ‚{\tiny $_{lb}$}‚श‚रीरादि त‚द‚पि प‚र‚स्प‚र‚स‚ङ्ग‚माव‚स्थातः\edtext{}{\edlabel{pvv.42-4}\label{pvv.42-4}\lemma{स्थातः}\Bfootnote{स‚काशात् ।}} पुञ्जाव‚स्थाया‚{\color{DodgerBlue3}‚म‚विशेषाद्} विशेषाभावात् ‚{\tiny $_{lb}$}‚‚{\color{DodgerBlue3}‚अणुत्वाच्च} । द‚र्श‚नान‚र्ह‚सूक्ष्म‚त‚यापि श‚रीरादेर्न ‚{\color{DodgerBlue3}‚ग‚तिश्चेत्\edtext{}{\edlabel{pvv.42-5}\label{pvv.42-5}\lemma{तिश्चेत्}\Bfootnote{अस्माकं म‚ते}}। न सिध्य‚ति (८८) ‚{\tiny $_{lb}$}‚अविशेषः} ।
	\pend% ending standard par
      

	  \begin{center}%% label @type='head'
	\textbf{(a. आव‚र‚णाद्य‚भाव‚निरासः)}
	\end{center}
	
	  \bigskip
	  \begingroup
	
	    \large
	  
	    \begin{quote}
	  
	    
	    \stanza[\smallbreak]
	\label{pv.1.89b}\flagstanza{\tiny\textenglish{....1.89b}}विशिष्टानामैन्द्रिय‚त्व‚म‚तोऽन‚णुः ।\&[\smallbreak]


	
	    \end{quote}
	  
	  \endgroup
	

	  \pstart \leavevmode% starting standard par
	प‚र‚स्प‚रास‚ङ्ग‚तेभ्यः प\edtext{}{\edlabel{pvv.42-6}\label{pvv.42-6}\lemma{प}\Bfootnote{क्ष‚णिक‚त्वात् ।}}र‚मा‚{\tiny $_{6}$}‚णुभ्योऽदृष्ट‚स‚ह‚कारिभ्यो दृश्यानामेवान्योन्य‚संह‚ता‚{\tiny $_{lb}$}‚नामुत्प‚त्तेः । तेषां ‚{\color{DodgerBlue3}‚विशिष्टानामैन्द्रिय}‚त्व‚मिन्द्रिय‚ग्राह्य‚त्वं अत\edtext{}{\edlabel{pvv.42-7}\label{pvv.42-7}\lemma{अत}\Bfootnote{येऽतीन्द्रिया अण‚वो न त एव प‚श्चात् किन्तु तानाश्रित्यान्य एव विशिष्टा उत्प‚द्य‚न्ते ।}} ऐन्द्रिय‚त्वा‚{\color{DodgerBlue3}‚द‚न‚णु}‚रिष्टः ‚{\tiny $_{lb}$}‚(।) इन्द्रियागोच‚रेष्व‚णुत्वं प्र‚सिद्धं त‚देव तु हेतुकृतं । पुञ्जीभूतास्तु दृश्य‚माना नाण‚व ‚{\tiny $_{lb}$}‚\leavevmode\ledsidenote{\textenglish{043/s}} उच्य‚न्ते किन्तु श‚रीरादिव्य‚प‚देश्याः । य‚था त‚न्त‚वः प‚टाव‚स्था न त‚न्त‚वः उच्य‚न्तेऽपि ‚{\tiny $_{lb}$}‚तु प‚ट इत्य‚सिद्धौ हेतू\edtext{}{\edlabel{pvv.43-1}\label{pvv.43-1}\lemma{हेतू}\Bfootnote{अविशेषाद‚णुत्वाच्चेति प‚रोक्तौ ।}} ।
	\pend% ending standard par
      
	  \bigskip
	  \begingroup
	
	    \large
	  
	    \begin{quote}
	  
	    
	    \stanza[\smallbreak]
	\label{pv.1.89c}\flagstanza{\tiny\textenglish{....1.89c}}एतेनाव‚र‚णादीनाम‚भाव‚श्च निराकृतः ॥ ८९ ॥\&[\smallbreak]


	
	    \end{quote}
	  
	  \endgroup
	

	  \pstart \leavevmode% starting standard par
	\hphantom{.}‚{\color{DodgerBlue3}‚एतेन} प‚र‚माणूनां पूर्व्वाव‚स्थातो विशिष्ट‚त्व‚क‚थ‚नेनाव‚{\tiny $_{7}$}‚र\edtext{}{\edlabel{pvv.43-2}\label{pvv.43-2}\lemma{र}\Bfootnote{अव‚य‚वित्वायोगात् ।}} ‚{\color{DodgerBlue3}‚णाधारादीनाम‚भावः}\leavevmode\ledsidenote{\textenglish{9b/MA}} ‚{\tiny $_{lb}$}‚प‚रैर‚संह‚ताव‚स्थायामिव यः प्र‚तिपादित‚तः ‚{\color{DodgerBlue3}‚स च निराकृतो} बोद्ध‚व्यः । ‚{\color{DodgerBlue3}‚केचित् प‚र‚माण‚वः} संह‚ता जाता आव‚र‚ण‚धार‚णादिक्ष‚मा भ‚व‚न्ति नान्ये । (८९)
	\pend% ending standard par
      \label{div_pvv.1.90}
	  
	% new div opening: depth here is 2
	
	  \bigskip
	  \begingroup
	
	    \large
	  
	    \begin{quote}
	  
	    
	    \stanza[\smallbreak]
	\label{pv.1.90}\flagstanza{\tiny\textenglish{...v.1.90}}क‚थं वा सूत‚हेमादिमिश्रं त‚प्तोप‚लादि वा ।&दृश्यं पृथ‚ग‚श‚क्तानाम‚क्षादीनां ग‚तिः क‚थ‚म् ॥ ९० ॥\&[\smallbreak]


	
	    \end{quote}
	  
	  \endgroup
	

	  \pstart \leavevmode% starting standard par
	\hphantom{.}‚{\color{DodgerBlue3}‚क‚थ‚म्वा सूत‚हेमादिमिश्रं} पिष्टिकाव‚स्थायां तेजःप‚र‚मा\edtext{}{\edlabel{pvv.43-3}\label{pvv.43-3}\lemma{मा}\Bfootnote{य‚था य‚द्येकं द्र‚व्यं नेष्य‚ते त‚दा य‚था तेषां पूर्व्वाव‚स्थायां प‚र‚माणुरूपं नावृनो/?/ति न द‚धाति च त‚था प‚श्चाद‚प्य‚विशेषान्नावृणुयान्न द‚ध्याच्चेति निराकृतं । तैज‚साः प‚र‚माण‚वः काष्ठोप‚लादिप‚र‚माणुभिः प‚र‚स्प‚र‚संभेदेनाविश‚न्तीति प‚रः ।}}णुस‚ञ्च‚य‚रूपे ‚{\color{DodgerBlue3}‚त‚प्तो}‚{\tiny $_{lb}$}‚प‚लादि वा मिश्रं क‚थं दृश्यं विजातीयानां द्र‚व्यानार‚म्भात् न त‚द‚व‚य‚वि द्र‚व्यं । प‚र‚{\tiny $_{lb}$}‚माण‚व‚श्च त्व‚न्म‚ते न दृश्या इति न तेषां द‚र्श‚नं स्यात् । य‚दि च प‚र‚माण‚वः पृथ‚ग‚{\tiny $_{lb}$}‚व‚स्थायां‚{\tiny $_{1}$}‚ ज्ञान‚ज‚न‚नेऽश‚क्ता इति पुञ्जाव‚स्था अपि त‚था । त‚दा ‚{\color{DodgerBlue3}‚पृथ‚ग‚श‚क्तानाम‚{\tiny $_{lb}$}‚क्षादीनां}\edtext{}{\edlabel{pvv.43-4}\label{pvv.43-4}\lemma{दा}\Bfootnote{च‚क्षूरूपालोक‚म‚न‚सिकाराणां ।}} ज्ञान‚ज‚न‚ने संह‚तौ ‚{\color{DodgerBlue3}‚ग‚तिर्ज्ञानं क‚थं} । (९०)
	\pend% ending standard par
      \label{div_pvv.1.91}
	  
	% new div opening: depth here is 2
	
	  \bigskip
	  \begingroup
	
	    \large
	  
	    \begin{quote}
	  
	    
	    \stanza[\smallbreak]
	\label{pv.1.91}\flagstanza{\tiny\textenglish{...v.1.91}}संयोगाच्चेत् स‚मानोऽत्र प्र‚स‚ङ्गो हेम‚सूत‚योः ।&दृश्यः संयोग इति चेत् कुतोऽदृश्याश्र‚ये ग‚तिः ॥ ९१ ॥\&[\smallbreak]


	
	    \end{quote}
	  
	  \endgroup
	

	  \pstart \leavevmode% starting standard par
	त‚स्माद‚नैकान्तिक‚म‚पि हेतुद्व‚यं नेन्द्रियादेर्ज्ञान‚ज‚न्म किन्तु संयो\edtext{}{\edlabel{pvv.43-5}\label{pvv.43-5}\lemma{संयो}\Bfootnote{आत्मा म‚न‚सा युज्य‚ते म‚नः इन्द्रियेण इन्द्रिय‚म‚र्थेन त‚तो ज्ञानोत्प‚त्तिः ।}}गात्त‚दीयादिति ‚{\tiny $_{lb}$}‚चेत् । ‚{\color{DodgerBlue3}‚स‚मानोऽत्र प्र‚स‚ङ्गः} । य‚था पृथ‚गिन्द्रियाद‚यः ‚{\color{DodgerBlue3}‚संयोग‚न्न ज‚न‚य‚न्ति । त‚था}‚{\tiny $_{lb}$}‚मिलिता अपि न ज‚न‚येयुः । ‚{\color{DodgerBlue3}‚हेम‚सूत‚योर्दृश्यः संयोगो} दृश्य‚त ‚{\color{DodgerBlue3}‚इति चेत् । कुतोऽदृश्या‚{\tiny $_{lb}$}‚श्र‚ये ग‚तिः} । हेम‚सूत‚प‚र‚माण‚वो हि संयोग‚स्याश्र‚या‚{\tiny $_{2}$}‚स्तेषाम‚दृश्य‚त्वे क‚थ‚न्त‚दाश्रि‚{\tiny $_{lb}$}‚त‚स्य संयोग‚स्य द‚र्श‚नं (।) न हि क‚श्चित् पिशाच‚योः संयोग‚मुप‚ल‚भ‚ते (।) किञ्च (।) ‚{\tiny $_{lb}$}‚नानाद्र‚व्यार‚ब्ध‚पान‚कादिर‚पि संयोगो गुण एषित‚व्यः । त‚स्मिन् ‚{\color{DodgerBlue3}‚मृष्टं पान‚कं स्व‚रूपं} पान‚क‚मिति निर्गुण‚त्वाद् गुणानां विरुद्धः ॥ (९१)
	\pend% ending standard par
      \label{div_pvv.1.92}
	  
	% new div opening: depth here is 2
	
	  \bigskip
	  \begingroup
	
	    \large
	  
	    \begin{quote}
	  
	    
	    \stanza[\smallbreak]
	\label{pv.1.92}\flagstanza{\tiny\textenglish{...v.1.92}}र‚स‚रूपादियोग‚श्च संयोग उप‚चार‚तः ।&इष्ट‚श्चेद् बुद्धिभेदोऽस्तु पंक्तिर्दीर्घेति वा क‚थ‚म् ॥ ९२ ॥\&[\smallbreak]


	
	    \end{quote}
	  
	  \endgroup
	\textsuperscript{\textenglish{044/s}}

	  \pstart \leavevmode% starting standard par
	नानागुणिद्र‚व्येषु संयोगो ॥ र‚स‚रूपाद‚य‚श्च स‚म‚वेता इत्येकार्थ‚स‚म‚वायात्त‚द्ध‚र्म‚स्य ‚{\tiny $_{lb}$}‚पान‚कादिषूप\edtext{}{\edlabel{pvv.44-1}\label{pvv.44-1}\lemma{कादिषूप}\Bfootnote{द्र‚व्य‚ध‚र्मे गुणे ।}} ‚{\color{DodgerBlue3}‚चार‚तो र‚स‚रूपादियोग इष्ट‚श्चेत्} । त‚र्हि क्षीरा\edtext{}{\edlabel{pvv.44-2}\label{pvv.44-2}\lemma{क्षीरा}\Bfootnote{आदिना क्षीरोद‚कादि ।}}दौ‚{\tiny $_{3}$}‚ पान‚कादौ च ‚{\tiny $_{lb}$}‚मुख्यामुख्य‚त्वेन स्प‚ष्टास्प‚ष्ट‚त‚या मृष्टादि‚{\color{DodgerBlue3}‚बुद्धेर्भेदः} स्यात्\edtext{}{\edlabel{pvv.44-3}\label{pvv.44-3}\lemma{स्यात्}\Bfootnote{य‚दि य‚त्र प‚र‚माणुषु र‚साद‚यः स‚म‚वेतास्त‚त्र संयोगोपीति संयोगो र‚सोप‚चार‚स्त‚र्हि क्षारादिव‚त्पान‚के र‚स‚बुद्धिर्माभूद‚स्ति च ।}} । न हि माण‚व‚केऽग्नि‚{\tiny $_{lb}$}‚बुद्धिरुप‚चाराद‚ग्नाविव भ‚व‚ति । ‚{\color{DodgerBlue3}‚पंक्तिर्दीर्घेति वा क‚थं}\edtext{\textsuperscript{*}}{\edlabel{pvv.44-4}\label{pvv.44-4}\lemma{*}\Bfootnote{य‚द्येकार्थ‚स‚म‚वायात्त‚द्ध‚र्मोप‚चार‚तो उप‚देशः पंक्तिः संयोग‚त्वाद् गुणः । न त‚त्र दैर्ध्य गुणोस्ति नापि प‚क्षिषु क‚थ‚मुप‚चार‚तोपि निर्द्देशः । न हि प्र‚त्येकं प‚ङिक्त‚षु दैर्ध्य‚म‚स्ति त‚थाभूतं । येनैकार्थ‚स‚म‚वाय उप‚चार‚वीजं स्यात् ।}} । पंक्तिषु स‚म‚वेता प‚ङ‏क्तिः ‚{\tiny $_{lb}$}‚सं\edtext{}{\edlabel{pvv.44-5}\label{pvv.44-5}\lemma{सं}\Bfootnote{य‚द्य‚व‚य‚विनं विनाऽद‚र्श‚नं त‚दा क‚थं सूत‚हेम‚मिश्रं दृश्य‚ते । न हि त‚द्‏द्र‚व्यार‚म्भ‚कं विजातीयानार‚म्भात् अन्य‚था स‚र्व स‚र्व्वैरार‚भ्येतेति सिद्धान्तः ।}}ख्या गुणो वा स्यात्(।) न च त‚त्र दैर्ध्य‚म‚स्ति निर्गुण‚त्वाद् गुणानां (।) नापि ‚{\tiny $_{lb}$}‚प‚ङ‏क्तिषु दैर्ध्य‚म‚स्ति येनैकार्थ‚स‚म‚वायादुप‚चारः स्यात् (। ९२)
	\pend% ending standard par
      \label{div_pvv.1.93}
	  
	% new div opening: depth here is 2
	

	  \begin{center}%% label @type='head'
	\textbf{(b. संख्यादिनिरासः)}
	\end{center}
	

	  \pstart \leavevmode% starting standard par
	एत‚च्चाभ्युप‚ग‚म्योक्तं न तु संख्याद‚यः स‚न्ति त‚दाह (।)
	\pend% ending standard par
      
	  \bigskip
	  \begingroup
	
	    \large
	  
	    \begin{quote}
	  
	    
	    \stanza[\smallbreak]
	\label{pv.1.93}\flagstanza{\tiny\textenglish{...v.1.93}}संख्यासंयोग‚क‚र्मादेर‚पि त‚द्व‚त्-स्व‚रूप‚तः ।&अभिलापाच्च भेदेन रूपं बुद्धौ न भास‚ते ॥ ९३ ॥\&[\smallbreak]


	
	    \end{quote}
	  
	  \endgroup
	

	  \pstart \leavevmode% starting standard par
	\hphantom{.}‚{\color{DodgerBlue3}‚संख्यासंयोग‚क‚र्म्मादे}‚(ः।) आदिश‚ब्दाद्विभाग‚प‚र‚त्वाप‚र‚त्व‚सामान्यादेर‚पि ‚{\tiny $_{lb}$}‚‚{\color{DodgerBlue3}‚त‚द्व‚त्स्व‚रूप‚तो} द्र‚व्य‚स्व‚भावाद् ‚{\color{DodgerBlue3}‚भेदेनाभिलापाच्च} । संख्यासंयोग इत्यादिकात् । ‚{\color{DodgerBlue3}‚बुद्धौ} द्र‚व्य‚ग्राहिण्यां ‚{\color{DodgerBlue3}‚रूपं न भास‚ते}‚ऽभास‚मान‚ञ्च दृश्याभिम‚तं दृश्यानुप‚ल‚ब्ध‚या बाधितं ॥ ‚{\tiny $_{lb}$}‚(९३)
	\pend% ending standard par
      \label{div_pvv.1.94}
	  
	% new div opening: depth here is 2
	

	  \pstart \leavevmode% starting standard par
	य‚दि संख्याद‚यो न स‚न्ति क‚थ‚मेको घ‚टः संयुक्तो म‚हान् प‚त‚तीत्यादि व्य‚प‚दिश्य‚ते ‚{\tiny $_{lb}$}‚ऽर्थ‚भेदाभावे प‚र्य्याय‚ता प्राप्नोतीत्य‚त आह ।
	\pend% ending standard par
      
	  \bigskip
	  \begingroup
	
	    \large
	  
	    \begin{quote}
	  
	    
	    \stanza[\smallbreak]
	\label{pv.1.94}\flagstanza{\tiny\textenglish{...v.1.94}}श‚ब्द‚ज्ञाने विक‚ल्पेन व‚स्तुभेदानुसारिणा ।&गुणादिष्विव क‚ल्प्यार्थे न‚ष्टाजातेषु वा य‚था ॥ ९४ ॥\&[\smallbreak]


	
	    \end{quote}
	  
	  \endgroup
	

	  \pstart \leavevmode% starting standard par
	\hphantom{.}‚{\color{DodgerBlue3}‚श‚ब्द‚ज्ञाने} एको घ‚ट इत्यादिके क‚ल्प्यार्थे क‚ल्पितार्थे ‚{\color{DodgerBlue3}‚विक‚{\tiny $_{5}$}‚ल्पेन} (।) कीदृशेन ‚{\color{DodgerBlue3}‚व‚स्तु‚{\tiny $_{lb}$}‚भेदानुसारिणा} व‚स्तुनो भेदो विजातीयाद् व्यावृत्तिस्ताम्विष‚य‚त्वेनानुस‚र‚ता ‚{\color{DodgerBlue3}‚गुणादि}‚{\tiny $_{lb}$}‚\leavevmode\ledsidenote{\textenglish{045/s}} ‚{\color{DodgerBlue3}‚ष्विव} य‚था प‚ङ्क्त्यादौ ‚{\color{DodgerBlue3}‚एका म‚ह‚ती ग‚च्छ‚तीत्यादिश‚ब्द‚ज्ञाने । न हि त‚त्र संख्याद‚यः} स‚न्ति निर्गुण‚त्वाद् गुणानां । ‚{\color{DodgerBlue3}‚न‚ष्टाजातेषु वा य‚था । एको द्वौ ब‚ह‚वो न‚ष्टा भ‚वि-} ष्य‚न्ति वेति श‚ब्द‚ज्ञान‚व‚त् । ‚{\color{DodgerBlue3}‚न‚ष्ट‚म‚जात‚ञ्च स्व‚य‚मेव नास्ति किं पुन‚स्त‚त्र संख्याद‚यो} भ‚विष्य‚न्ति । (९४)
	\pend% ending standard par
      \label{div_pvv.1.95}
	  
	% new div opening: depth here is 2
	
	  \bigskip
	  \begingroup
	
	    \large
	  
	    \begin{quote}
	  
	    
	    \stanza[\smallbreak]
	\label{pv.1.95}\flagstanza{\tiny\textenglish{...v.1.95}}म‚तो य‚द्युप‚चारोऽत्र स इष्टो य‚न्निब‚न्ध‚नः ।&स एव स‚र्व‚भावेषु हेतुः किं नेष्य‚ते त‚योः ॥ ९५ ॥\&[\smallbreak]


	
	    \end{quote}
	  
	  \endgroup
	

	  \pstart \leavevmode% starting standard par
	\hphantom{.}‚{\color{DodgerBlue3}‚म‚तो य‚द्युप‚चारोत्र} गु‚{\tiny $_{6}$}‚णादिषु संख्यादिव्य‚प‚देश‚स्य ‚{\color{DodgerBlue3}‚स} उप‚चार ‚{\color{DodgerBlue3}‚इष्टो य‚न्नि‚{\tiny $_{lb}$}‚ब‚न्ध‚नो}\edtext{}{\edlabel{pvv.45-1}\label{pvv.45-1}\lemma{चार}\Bfootnote{लोक‚व्य‚व‚हार‚निब‚न्ध‚नः ।}} गुणादिषु संख्याद्युप‚चार‚स्य हेतुर्यः ‚{\color{DodgerBlue3}‚स एव स‚र्व्व‚भावेषु हेतुः किन्नेष्य‚ते त‚योः} श‚ब्द‚ज्ञान‚योः येन संख्याद‚यः क‚ल्प्य‚न्ते प्र‚माण‚बाधिताः ॥ (९५)
	\pend% ending standard par
      \label{div_pvv.1.96}
	  
	% new div opening: depth here is 2
	
	  \bigskip
	  \begingroup
	
	    \large
	  
	    \begin{quote}
	  
	    
	    \stanza[\smallbreak]
	\label{pv.1.96}\flagstanza{\tiny\textenglish{...v.1.96}}उप‚चारो न स‚र्व‚त्र य‚दि भिन्न‚विशेष‚ण‚म् ।&मुख्य‚मित्येव च कुतोऽभिन्नेऽभिन्नार्थ‚तेति चेत् ॥ ९६ ॥\&[\smallbreak]


	
	    \end{quote}
	  
	  \endgroup
	

	  \pstart \leavevmode% starting standard par
	\hphantom{.}‚{\color{DodgerBlue3}‚उप‚चारो न स‚र्व्व‚त्र य‚दि} मुख्ये स‚त्युप‚चारो भ‚व‚ति\edtext{}{\edlabel{pvv.45-2}\label{pvv.45-2}\lemma{ति}\Bfootnote{पृष्टः सिद्धान्तिनैकं मुख्य‚मिति आह ।}}न तु स‚र्व्व‚त्रैवा\edtext{}{\edlabel{pvv.45-3}\label{pvv.45-3}\lemma{त्रैवा}\Bfootnote{तादृशं ते मुख्य‚ल‚क्ष‚णं । अविशिष्टं स्व‚तो द्र‚व्यं विशिष्य‚ते संख्यादिना त‚त् विशेष‚णं त‚स्य भिन्नं ।}}सौ ॥ ‚{\color{DodgerBlue3}‚भिन्न‚{\tiny $_{lb}$}‚विशेष‚णं} मुख्यं य‚त्र भिन्नं विशेष‚ण‚म‚स्ति ‚{\color{DodgerBlue3}‚त‚न्मुख्यं अन्य‚त्रोप‚चारः । भिन्न‚विशे‚{\tiny $_{7}$}‚ष‚णं}\leavevmode\ledsidenote{\textenglish{10a/MA}} ‚{\tiny $_{lb}$}‚‚{\color{DodgerBlue3}‚मुख्य‚मित्ये}‚त‚देव ‚{\color{DodgerBlue3}‚कुतो} निश्चितं ‚{\color{DodgerBlue3}‚अभिन्नेऽभिन्नार्थ‚तेति चेत्} । (९६)
	\pend% ending standard par
      \label{div_pvv.1.97}
	  
	% new div opening: depth here is 2
	

	  \pstart \leavevmode% starting standard par
	\hphantom{.}य‚दि संख्यादि भिन्नं नास्ति त‚दा घ‚ट एकः ‚{\color{DodgerBlue3}‚संयुक्तो म‚हान् ग‚च्छ‚तीति श‚ब्दानां} प‚र्याय‚ता स्यात् । न चास्ति ॥
	\pend% ending standard par
      
	  \bigskip
	  \begingroup
	
	    \large
	  
	    \begin{quote}
	  
	    
	    \stanza[\smallbreak]
	\label{pv.1.97}\flagstanza{\tiny\textenglish{...v.1.97}}अन‚र्थान्त‚र‚हेतुत्वेप्य‚प‚र्यायः सितादिषु ।&संख्यादियोगिनः श‚ब्दास्त‚त्राप्य‚र्थान्त‚रं य‚दि ॥ ९७ ॥\&[\smallbreak]


	
	    \end{quote}
	  
	  \endgroup
	

	  \pstart \leavevmode% starting standard par
	\hphantom{.}‚{\color{DodgerBlue3}‚न‚न्व\edtext{}{\edlabel{pvv.45-4}\label{pvv.45-4}\lemma{न्व}\Bfootnote{त‚स्मान्नार्थ‚विष‚याः सिद्धा । अत्र बौद्ध आहानेकान्त‚तां ।}}न‚र्थान्त‚र‚हेतुत्वेप्य‚प‚र्यायः सितादि}‚गुणे\edtext{}{\edlabel{pvv.45-5}\label{pvv.45-5}\lemma{गुणे}\Bfootnote{गुणेन गुणान्त‚र‚मित्य‚र्थान्त‚राभावात् य‚थैकं शुक्लं संयुक्तं विभ‚क्त‚मित्यादि ।}}षु ‚{\color{DodgerBlue3}‚संख्या\edtext{}{\edlabel{pvv.45-6}\label{pvv.45-6}\lemma{संख्या}\Bfootnote{अर्थान्त‚र‚निमित्तं ।}}दियोगिनः श‚ब्दा} दृश्य‚न्ते । ‚{\color{DodgerBlue3}‚अत्राप्य‚र्थान्त‚रं य‚दि} । सितादौ संख्यादि तिष्ठ‚ति । ‚{\color{DodgerBlue3}‚त‚त‚स्त‚च्छ‚ब्दानाम}‚{\tiny $_{lb}$}‚प‚र्याय‚ताप‚त्तिः । (९७)
	\pend% ending standard par
      \label{div_pvv.1.98}
	  
	% new div opening: depth here is 2
	\textsuperscript{\textenglish{046/s}}
	  \bigskip
	  \begingroup
	
	    \large
	  
	    \begin{quote}
	  
	    
	    \stanza[\smallbreak]
	\label{pv.1.98}\flagstanza{\tiny\textenglish{...v.1.98}}गुण‚द्र‚व्याविशेषः स्याद् भिन्नो व्यावृत्तिभेद‚तः ।&स्याद‚न‚र्थान्त‚रार्थ‚त्वेप्य‚क‚र्माद्र‚व्य‚श‚ब्द‚व‚त् ॥ ९८ ॥\&[\smallbreak]


	
	    \end{quote}
	  
	  \endgroup
	

	  \pstart \leavevmode% starting standard par
	\hphantom{.}एव‚ञ्च स‚ति ‚{\color{DodgerBlue3}‚गुणानां द्र‚व्याणाञ्चाविशेषः स्यात्} संकीर्ण्ण‚ल‚क्ष‚ण‚त्वात्‚{\tiny $_{1}$}‚ । ‚{\tiny $_{lb}$}‚क्रियाव‚त् गुण‚व‚त्स‚म‚वायिकार‚णं द्र‚व्य‚मिति ल‚क्ष‚ण\edtext{}{\edlabel{pvv.46-1}\label{pvv.46-1}\lemma{ण}\Bfootnote{गुण‚द्र‚व्य‚योः गुण‚क‚र्म‚णोश्च स‚म‚वायात् ।}}स्य द्व‚योर‚पि भावात् । क‚थं पुन‚र‚{\tiny $_{lb}$}‚भिन्नार्थ‚त्वेप्य‚प‚र्याय‚त्व\edtext{}{\edlabel{pvv.46-2}\label{pvv.46-2}\lemma{त्व}\Bfootnote{बौद्ध‚स्यापि ।}}मित्याह । विजातीयेभ्योऽनेकासंयुक्तादिभ्यो ‚{\color{DodgerBlue3}‚व्यावृत्तेर्भेदात् ‚{\tiny $_{lb}$}‚भिन्नः} प्र‚त्य‚य‚श्चैको घ‚टः संयुक्तो घ‚ट इत्यादि । अक‚र्म द्र‚व्य‚म‚द्र‚व्यं क‚र्मेति\edtext{}{\edlabel{pvv.46-3}\label{pvv.46-3}\lemma{र्मेति}\Bfootnote{अक‚र्म‚क‚त्वादौ प‚रेण व्य‚तिरेको नेष्टोऽभिम‚ताद‚तो व्यावृत्तिमुखेनैव ते स्थाप्याः अक‚र्म‚त्व‚म‚द्र‚व्य‚त्व‚ञ्चास्येति ।}} श‚ब्दाविव ‚{\tiny $_{lb}$}‚भिन्नो व्य‚तिरिक्तार्थाभावेपि । न हि त्व‚न्म‚तेप्य‚क‚र्म‚द्र‚व्य‚श‚ब्द‚योर‚र्थो द्र‚व्य\edtext{}{\edlabel{pvv.46-4}\label{pvv.46-4}\lemma{व्य}\Bfootnote{अत्र व्यावृत्तिरेव श‚र‚णं ।}}क‚र्म‚भ्यां ‚{\tiny $_{lb}$}‚भिन्नो व‚स्तुभूतोस्ति ॥ (९८)
	\pend% ending standard par
      \label{div_pvv.1.99}
	  
	% new div opening: depth here is 2
	

	  \begin{center}%% label @type='head'
	\textbf{(B. संख्याद्य‚भावेऽप्येक‚त्व‚संयोग‚योर्व्य‚प‚देशः)}
	\end{center}
	

	  \pstart \leavevmode% starting standard par
	य‚दि न स‚न्ति संख्याद‚यः क‚थं घ‚ट‚स्यै\edtext{}{\edlabel{pvv.46-5}\label{pvv.46-5}\lemma{स्यै}\Bfootnote{व्य‚तिरेक‚ष‚ष्ठी ।}}क‚त्वं सं‚{\tiny $_{2}$}‚योगो वा इत्यादि व्य‚प‚देश इत्याह ।
	\pend% ending standard par
      
	  \bigskip
	  \begingroup
	
	    \large
	  
	    \begin{quote}
	  
	    
	    \stanza[\smallbreak]
	\label{pv.1.99}\flagstanza{\tiny\textenglish{...v.1.99}}व्य‚तिरेकीव य‚च्चापि सूच्य‚ते भाव‚वाचिभिः ।&संख्यादित‚द्व‚तः श‚ब्दैस्त‚द्ध‚र्मान्त‚र‚भेद‚क‚म् ॥ ९९ ॥\&[\smallbreak]


	
	    \end{quote}
	  
	  \endgroup
	

	  \pstart \leavevmode% starting standard par
	\hphantom{.}‚{\color{DodgerBlue3}‚व्य‚तिरे}‚कीव भेद‚व‚दिव ‚{\color{DodgerBlue3}‚य‚च्चापि संख्यादि त‚द्व‚तो} द्र‚व्याद् ‚{\color{DodgerBlue3}‚भाव‚वाचिभिः} द्र‚व्या‚{\tiny $_{lb}$}‚भिधायिभिः ‚{\color{DodgerBlue3}‚श‚ब्दैः सूच्य‚ते} घ‚ट‚स्यैक‚त्व‚मित्यादि । त‚त्सूच्य‚मान‚मेक‚त्वादि‚{\color{DodgerBlue3}‚ध‚र्म्मान्त‚र}‚स्य ‚{\tiny $_{lb}$}‚शुक्ल‚त्वादेः ‚{\color{DodgerBlue3}‚भेद‚कं} । य‚द्य‚पि घ‚टानैक‚त्वाद‚यो भिन्ना व‚स्तुत‚स्त‚थाप्येक‚त्व‚शुक्ल‚त्वाद‚यो ‚{\tiny $_{lb}$}‚घ‚टात्प‚र‚स्प‚रं चानुवृत्त्य‚न‚नुवृत्तिभ्यां क‚ल्पित‚भेदाः । (९९)
	\pend% ending standard par
      \label{div_pvv.1.100}
	  
	% new div opening: depth here is 2
	
	  \bigskip
	  \begingroup
	
	    \large
	  
	    \begin{quote}
	  
	    
	    \stanza[\smallbreak]
	\label{pv.1.100}\flagstanza{\tiny\textenglish{....1.100}}श्रुतिस्त‚न्मात्र‚जिज्ञासोर‚नाक्षिप्ताखिलाप‚रा ।&भिन्नं ध‚र्म‚मिवाच‚ष्टे योगोऽङ्गुल्या इति क्व‚चित् ॥ १०० ॥\&[\smallbreak]


	
	    \end{quote}
	  
	  \endgroup
	

	  \pstart \leavevmode% starting standard par
	तेषु य‚दैको ध‚र्मः प्र‚तिपि\edtext{}{\edlabel{pvv.46-6}\label{pvv.46-6}\lemma{तिपि}\Bfootnote{प्र‚तिप‚त्तुमिच्छुः ।}}त्सितः त‚दा भेदेन निर्दि‚{\tiny $_{3}$}‚ष्टः स ध‚र्मान्त‚र‚प्र‚तिक्षेप‚को ‚{\tiny $_{lb}$}‚भ‚व‚ति घ‚ट‚स्यैक‚त्व‚मित्यादि । त‚दा ‚{\color{DodgerBlue3}‚त‚न्मात्र}‚स्यैक‚ध‚र्म‚मात्र‚प्र‚तिपित्सोः प्र‚तिपाद्य‚स्यान्‚{\tiny $_{lb}$}‚‚{\color{DodgerBlue3}‚रोधेन च त‚था} संकेत‚व‚शात् प्र‚युक्ता ‚{\color{DodgerBlue3}‚श्रुतिर‚नाक्षिप्तो}‚ऽविष‚यीकृतोऽ‚{\color{DodgerBlue3}‚प‚रोखिलो} ध‚र्मो (व्य‚व‚च्छेद‚कं ।) य‚थास‚म्भ‚वी य‚या सा तादृशी ध‚र्मिणो ध‚र्म्मान्त‚रेभ्य‚श्च ‚{\tiny $_{lb}$}‚‚{\color{DodgerBlue3}‚भिन्नं} निष्कृष्ट‚भिव ‚{\color{DodgerBlue3}‚ध‚र्म‚माच‚ष्टे} य‚थाऽ‚{\color{DodgerBlue3}‚ङ्गुल्या योग} इति । (१००)
	\pend% ending standard par
      \label{div_pvv.1.101}
	  
	% new div opening: depth here is 2
	
	  \bigskip
	  \begingroup
	
	    \large
	  
	    \begin{quote}
	  
	    
	    \stanza[\smallbreak]
	\label{pv.1.101}\flagstanza{\tiny\textenglish{....1.101}}युक्ताङ्गुलीति स‚र्वेषां आक्षेपाद् ध‚र्मिवाचिनी ।&ख्यातैकार्थाभिधानेऽपि त‚था विहित‚संस्थितिः ॥ १०१ ॥\&[\smallbreak]


	
	    \end{quote}
	  
	  \endgroup
	\textsuperscript{\textenglish{047/s}}

	  \begin{center}%% label @type='head'
	\textbf{(a. ध‚र्म‚वाचिन्येव श्रुतिर्ध‚र्मिवाचिनी)}
	\end{center}
	

	  \pstart \leavevmode% starting standard par
	य‚दा तु स एव ध‚र्मो ध‚र्मान्त‚र‚स‚म्ब‚न्ध‚योगो जिज्ञासित‚स्त‚दा त‚था‚{\tiny $_{4}$}‚ संकेताद् ‚{\tiny $_{lb}$}‚‚{\color{DodgerBlue3}‚युक्ताङ्गुली}‚ति श्रुतिः ‚{\color{DodgerBlue3}‚स‚र्व्वेषां} ध‚र्म्मान्त‚राणा\edtext{}{\edlabel{pvv.47-1}\label{pvv.47-1}\lemma{राणा}\Bfootnote{अङ्गुलीग‚तानां ।}} ‚{\color{DodgerBlue3}‚माक्षेपाद् ध‚र्मिवाचिनी ख्याता} । एक‚स्य ‚{\tiny $_{lb}$}‚विजातीय‚व्यावृत्तिल‚क्ष‚ण‚स्यार्थ‚स्या‚{\color{DodgerBlue3}‚भिधानेप्य}‚य‚म्विभागो युक्तः । ‚{\color{DodgerBlue3}‚य‚स्मात्तेन प्र‚कारेण ‚{\tiny $_{lb}$}‚विहित‚संस्थितिः} कृत‚व्य‚व‚स्था सा श्रुतिः संकेतेन । (१०१)
	\pend% ending standard par
      \label{div_pvv.1.102}
	  
	% new div opening: depth here is 2
	

	  \begin{center}%% label @type='head'
	\textbf{(b. स‚मुदाय‚वाचिनी श्रुति)}
	\end{center}
	

	  \pstart \leavevmode% starting standard par
	\hphantom{.}त‚देवं ध‚र्म‚वाचिन्य‚पि श्रुतिर्द्ध‚र्मान्त‚र‚प्र‚तिक्षेपाप्र‚तिक्षेपाभ्यां ‚{\color{DodgerBlue3}‚ध‚र्म‚वाचिनी ध‚र्मि}‚{\tiny $_{lb}$}‚वाचिनी वेति निर्दिष्टं ।
	\pend% ending standard par
      

	  \pstart \leavevmode% starting standard par
	इदानीं स‚मु\edtext{}{\edlabel{pvv.47-2}\label{pvv.47-2}\lemma{मु}\Bfootnote{रूप‚र‚सादेः ।}} दाय‚वाचिनीं द‚र्श‚य‚ति ।
	\pend% ending standard par
      
	  \bigskip
	  \begingroup
	
	    \large
	  
	    \begin{quote}
	  
	    
	    \stanza[\smallbreak]
	\label{pv.1.102}\flagstanza{\tiny\textenglish{....1.102}}रूपादिश‚क्तिभेदानाम‚नाक्षेपेण व‚र्त्त‚ते ।&त‚त्स‚मान‚फ‚लाऽहेतुव्य‚व‚च्छेदे घ‚ट‚श्रुतिः ॥ १०२ ॥\&[\smallbreak]


	
	    \end{quote}
	  
	  \endgroup
	

	  \pstart \leavevmode% starting standard par
	घ‚ट‚व्य‚{\tiny $_{5}$}‚प‚देश‚भाजां ‚{\color{DodgerBlue3}‚रूपादी}‚नाम‚वान्त‚र‚रंज‚ना\edtext{}{\edlabel{pvv.47-3}\label{pvv.47-3}\lemma{ना}\Bfootnote{र‚क्तं चित्तं स्याद्विशिष्ट‚रूपादौ च‚क्षुर्व्विज्ञानादि ।}} ‚{\color{DodgerBlue3}‚दिश‚क्तिभेदानाम‚नाक्षेपेण तेषां} रूपादीनां ‚{\color{DodgerBlue3}‚य‚त्स‚मानं} फ‚ल‚मुद‚काह‚र‚णादि त‚स्याहेतोर‚श्वादे‚{\color{DodgerBlue3}‚र्व्य‚व‚च्छेदे} प्र‚तिपाद्ये ‚{\color{DodgerBlue3}‚घ‚ट}\edtext{}{\edlabel{pvv.47-4}\label{pvv.47-4}\lemma{तिपाद्ये}\Bfootnote{रूप‚र‚साद्ये नैक‚त्र द्र‚व्ये एक‚कार्य‚कार‚ण‚श‚क्तिख्याप‚नाय ।}}‚{\tiny $_{lb}$}‚श्रुतिर्व्व‚र्त‚ते उद‚काह‚र‚णाद्य‚हेतुव्यावृत्तिं स‚मुदाय‚वाची घ‚ट‚श‚ब्द आहेत्य‚र्थः । (१०२)
	\pend% ending standard par
      \label{div_pvv.1.103}
	  
	% new div opening: depth here is 2
	
	  \bigskip
	  \begingroup
	
	    \large
	  
	    \begin{quote}
	  
	    
	    \stanza[\smallbreak]
	\label{pv.1.103}\flagstanza{\tiny\textenglish{....1.103}}अतो न रूपं घ‚ट इत्येकाधिक‚र‚ण श्रुतिः ।&भेदोय‚मीदृशो जातिस‚मुदायाभिधायिनोः ॥ १०३ ॥\&[\smallbreak]


	
	    \end{quote}
	  
	  \endgroup
	

	  \pstart \leavevmode% starting standard par
	\hphantom{.}अतः स‚मुदायाऽभिधायित्वात् न रूपं घ‚ट इत्येकाधिक‚र‚णा श्रुतिः । ‚{\color{DodgerBlue3}‚रूप‚श‚ब्दो} हि ध‚र्म‚वाची घ‚ट‚श‚ब्द‚स्तु स‚मुदायाभिधायी । क‚थ‚म‚न‚{\tiny $_{6}$}‚योरेकार्थ‚ता । ‚{\color{DodgerBlue3}‚भेदोऽय‚मी}‚{\tiny $_{lb}$}‚दृ\edtext{}{\edlabel{pvv.47-5}\label{pvv.47-5}\lemma{दृ}\Bfootnote{स्व‚विक‚ल्पोप‚र‚चितं न मुख्यं गौर्व्वाहीक‚व‚दुप‚चार‚श्च, न गुणेषु संख्यादिवृत्तिः । न ह्येकं रूपं एको घ‚ट इति प्र‚त्य‚ये लोक‚स्याव‚साय‚भेदः ।}}शो ‚{\color{DodgerBlue3}‚जातिस‚मुदायाभिधायिनो}‚र्घ‚ट‚त्व‚घ‚ट‚श‚ब्द‚योश्च ज्ञेयः । (१०३)
	\pend% ending standard par
      \label{div_pvv.1.104}
	  
	% new div opening: depth here is 2
	

	  \pstart \leavevmode% starting standard par
	स‚मुदायाभिधायी चाप्र‚तिक्षिप्त\edtext{}{\edlabel{pvv.47-6}\label{pvv.47-6}\lemma{तिक्षिप्त}\Bfootnote{जातिश‚ब्द‚त्वात् ।}}ध‚र्मान्त‚र एव भ‚व‚तीति सौव\edtext{}{\edlabel{pvv.47-7}\label{pvv.47-7}\lemma{सौव}\Bfootnote{स‚मुदाय‚भेदापेक्ष‚याऽन्यानाक्षेपाय ।}}र्ण्णो घ‚ट ‚{\tiny $_{lb}$}‚इत्यादि सामानाधिक‚र‚ण्यं । स च\edtext{}{\edlabel{pvv.47-8}\label{pvv.47-8}\lemma{च}\Bfootnote{स‚मुदायाभिधायी ।}} द्विविधोऽनेक‚वृत्तिर‚न्य‚था\edtext{}{\edlabel{pvv.47-9}\label{pvv.47-9}\lemma{था}\Bfootnote{अनेक‚घ‚ट‚वृक्षादौ ।}} च । ‚{\color{DodgerBlue3}‚त‚था घ‚टादि}‚{\tiny $_{lb}$}‚श‚ब्दो\edtext{}{\edlabel{pvv.47-10}\label{pvv.47-10}\lemma{ब्दो}\Bfootnote{एक‚गिरौ स्व‚स‚मुदायापेक्षः ।}}वि न्ध्या द्रि श‚ब्द‚श्च । य‚दि रूपाद‚यः केव‚ला नास्त्य‚व‚य‚वी त‚दा क‚थं घ‚ट‚स्य ‚{\tiny $_{lb}$}‚\leavevmode\ledsidenote{\textenglish{048/s}} रूपाद‚य इति स‚म्ब‚न्ध (।) इत्या\edtext{}{\edlabel{pvv.48-1}\label{pvv.48-1}\lemma{इत्या}\Bfootnote{द्विधा रूपादीनां श‚क्तिः सामान्या य‚था घ‚टादेरुद‚काह‚र‚णादि । प्र‚तिनिय‚ता च च‚क्षुर्व्विज्ञानादिज‚निका ।}}ह (।)
	\pend% ending standard par
      
	  \bigskip
	  \begingroup
	
	    \large
	  
	    \begin{quote}
	  
	    
	    \stanza[\smallbreak]
	\label{pv.1.104}\flagstanza{\tiny\textenglish{....1.104}}रूपाद‚यो घ‚ट‚स्येति त‚त्सामान्योप‚स‚र्ज‚नाः ।&त‚च्छ‚क्तिभेदाः ख्याप्य‚न्ते वाच्योऽन्योपि दिशाऽन‚या ॥ १०४ ॥\&[\smallbreak]


	
	    \end{quote}
	  
	  \endgroup
	

	  \pstart \leavevmode% starting standard par
	\hphantom{.}‚{\color{DodgerBlue3}‚रूपाद‚यो घ‚ट‚स्य} इति स‚म्ब‚न्ध‚वाचिन्या श्रुत्या ‚{\color{DodgerBlue3}‚त‚त्सामा‚{\tiny $_{7}$}‚न्योप‚स‚र्ज‚ना} घ‚ट‚त्व‚{\tiny $_{lb}$}‚\leavevmode\ledsidenote{\textenglish{10b/MA}} सामान्य‚विशेषितास्तेषां रूपादीनां श‚क्तिभेदा र‚ञ्ज‚नाद‚यः ख्याप्य‚न्ते घ‚ट‚व्य‚प‚देश‚{\tiny $_{lb}$}‚विष‚य‚स‚मुदायान्त‚र्ग्ग‚तं र‚ञ्ज‚न‚क्ष‚म‚रूपं निष्कृष्योच्य‚त इत्य‚र्थः । अन्योपि च‚न्द‚न‚स्य ‚{\tiny $_{lb}$}‚ग‚न्ध इत्यादिव्य‚प‚देशोऽन‚या दिशा वाच्यः । विस्त‚र‚स्तृतीय‚प‚रि\edtext{}{\edlabel{pvv.48-2}\label{pvv.48-2}\lemma{रि}\Bfootnote{सामान्यादिचिन्तायां (३।५५) ।}}च्छेद एवास्य । ‚{\tiny $_{lb}$}‚त‚देव‚म‚व‚य‚व्यादीनां प्र‚तिषेधात् पान‚कादिरिव प‚र‚माणुपुञ्ज‚रूप एव देहः प्र‚त्य‚क्षेणे‚{\tiny $_{lb}$}‚क्ष्य‚त इति स्थितं ॥ (१०४)
	\pend% ending standard par
      \label{div_pvv.1.105}
	  
	% new div opening: depth here is 2
	

	  \begin{center}%% label @type='head'
	\textbf{(घ) विज्ञानं कार‚ण‚म्}
	\end{center}
	

	  \begin{center}%% label @type='head'
	\textbf{I. प‚र‚माणूनां स‚मुदितानां न बुद्धिहेतुत्व‚म्}
	\end{center}
	

	  \pstart \leavevmode% starting standard par
	न चास्य बुद्धिहेतुत्वं युक्त‚{\tiny $_{1}$}‚मिति प्र‚कृत‚म‚नुब‚न्धं प‚र‚माणूनां स‚मुदितानां प्र‚त्येकं ‚{\tiny $_{lb}$}‚बुद्धिहेतुत्वाभावं व‚क्तुमाह ।
	\pend% ending standard par
      
	  \bigskip
	  \begingroup
	
	    \large
	  
	    \begin{quote}
	  
	    
	    \stanza[\smallbreak]
	\label{pv.1.105}\flagstanza{\tiny\textenglish{....1.105}}हेतुत्वे च स‚म‚स्तानामेकाङ्ग‚विक‚लेपि न ।&प्र‚त्येक‚म‚पि साम‚र्थ्ये युग‚प‚द् ब‚हुस‚म्भ‚वः ॥ १०५ ॥\&[\smallbreak]


	
	    \end{quote}
	  
	  \endgroup
	

	  \pstart \leavevmode% starting standard par
	\hphantom{.}य‚दि ‚{\color{DodgerBlue3}‚स‚म‚स्तानां} देह‚प‚र‚माणूनां ‚{\color{DodgerBlue3}‚हेतुत्वं} त‚दाऽव\edtext{}{\edlabel{pvv.48-3}\label{pvv.48-3}\lemma{दाऽव}\Bfootnote{क‚र्ण्णादि ।}}य‚व‚च्छेदादिना ‚{\color{DodgerBlue3}‚एकाङ्ग‚विक‚लेपि} देहे न स्यात् बुद्धिः (।) न चैत‚द‚स्ति । अथ प्र‚त्येकं ते स‚म‚र्थास्त‚दा ‚{\color{DodgerBlue3}‚प्र‚त्येक‚म‚पि ‚{\tiny $_{lb}$}‚साम‚र्थ्ये} स्वीक्रिय‚माणे ‚{\color{DodgerBlue3}‚युग‚प‚द् ब‚हुस‚म्भ‚वः} । प‚र‚माणुसंख्यानि ज्ञानानि युग‚प‚ज्जाय‚रेन् ‚{\tiny $_{lb}$}‚स‚र्व्वेषां प्र‚त्येकं साम‚र्थ्यात् । (१०५)
	\pend% ending standard par
      \label{div_pvv.1.106}
	  
	% new div opening: depth here is 2
	

	  \begin{center}%% label @type='head'
	\textbf{(II. प्राणापान‚योर‚नियाम‚क‚ता)}
	\end{center}
	

	  \pstart \leavevmode% starting standard par
	अथ प्राणापानाभ्यां नियाम‚काभ्यामेक‚मेव ज्ञान‚म‚भिव्य‚ज्य‚ते‚{\tiny $_{2}$}‚ त‚तो न युग‚{\tiny $_{lb}$}‚प‚द‚नेका व्य‚क्तिरित्य‚त आह (।)
	\pend% ending standard par
      
	  \bigskip
	  \begingroup
	
	    \large
	  
	    \begin{quote}
	  
	    
	    \stanza[\smallbreak]
	\label{pv.1.106}\flagstanza{\tiny\textenglish{....1.106}}नानेक‚त्व‚स्य तुल्य‚त्वात् प्राणापानौ नियाम‚कौ ॥&एक‚त्वेऽपि ब‚हुव्य‚क्तिस्त‚द्धेतोर्नित्य‚स‚न्निधेः ॥ १०६ ॥\&[\smallbreak]


	
	    \end{quote}
	  
	  \endgroup
	

	  \pstart \leavevmode% starting standard par
	\hphantom{.}त‚न्न । प‚र‚माणुस‚ञ्च‚यात्म‚क‚त्वेन देह‚व‚त् प्राणापान‚योर‚{\color{DodgerBlue3}‚प्य‚नेक‚त्व‚स्य तुल्य‚त्वात्} । ‚{\tiny $_{lb}$}‚‚{\color{DodgerBlue3}‚तेषां प्र‚त्येक‚म‚भिव्य‚ञ्ज‚क‚त्वे युग‚प‚द‚नेका}‚भिव्य‚क्तिप्र‚स‚ङ्गादेक‚ज्ञानाभिव्य‚क्तौ ‚{\tiny $_{lb}$}‚\leavevmode\ledsidenote{\textenglish{049/s}} प्राणापानौ नियाम‚कौ न युक्तौ । ‚{\color{DodgerBlue3}‚एक‚त्वेपि} प्राण‚स्या\edtext{}{\edlabel{pvv.49-1}\label{pvv.49-1}\lemma{स्या}\Bfootnote{अथ माभूदेव दोष इति प्राणादिरेकं द्र‚व्य‚मिष्य‚ते ।}}पान‚स्य च ‚{\color{DodgerBlue3}‚ब‚हूनां नाना}‚काल‚{\tiny $_{lb}$}‚भाविनां ज्ञानानां युग‚प‚द‚{\color{DodgerBlue3}‚भिव्य‚क्तिः} स्यात् । त‚स्याभिव्य\edtext{}{\edlabel{pvv.49-2}\label{pvv.49-2}\lemma{स्याभिव्य}\Bfootnote{नित्य‚त्वाद‚स्य । य‚त् स‚न्निहिताविक‚ल‚कार‚णं त‚द् भ‚व‚त्येवेति न्यायादाह ।}}क्ते‚{\color{DodgerBlue3}‚र्हेतोः} प्राणा\edtext{}{\edlabel{pvv.49-3}\label{pvv.49-3}\lemma{प्राणा}\Bfootnote{देह‚स्य च ।}}देर्नित्यं ‚{\tiny $_{lb}$}‚स‚न्निधेः ॥ (१०६)
	\pend% ending standard par
      \label{div_pvv.1.107}
	  
	% new div opening: depth here is 2
	
	  \bigskip
	  \begingroup
	
	    \large
	  
	    \begin{quote}
	  
	    
	    \stanza[\smallbreak]
	\label{pv.1.107}\flagstanza{\tiny\textenglish{....1.107}}नानेक‚हेतुरिति चेन्नाविशेषात् क्र‚माद‚पि ॥&नैक‚प्राणेप्य‚नेकार्थ‚ग्र‚ह‚णान्निय‚म‚स्त‚तः ॥ १०७ ॥\&[\smallbreak]


	
	    \end{quote}
	  
	  \endgroup
	

	  \pstart \leavevmode% starting standard par
	\hphantom{.}युग‚प‚{\color{DodgerBlue3}‚न्नानेक}‚स्या‚{\tiny $_{3}$}‚‚{\color{DodgerBlue3}‚भिव्य‚क्तिहेतुः} प्राणा‚{\color{DodgerBlue3}‚दिरिति चेत्}\edtext{}{\edlabel{pvv.49-4}\label{pvv.49-4}\lemma{प्राणा}\Bfootnote{प्राणापान‚देह‚स्य य आत्मा विज्ञान‚हेतुः प‚श्चात्स पूर्व्व‚म‚पीति ।}} (।) ‚{\color{DodgerBlue3}‚न} स्याद्धेतुर‚{\color{DodgerBlue3}‚विशेषात्} ।\edtext{\textsuperscript{*}}{\edlabel{pvv.49-5}\label{pvv.49-5}\lemma{*}\Bfootnote{अनेक‚त्वेपि प्राणापान‚योर्देह‚स्य च एक‚त्रैव ज्ञाने साम‚र्थ्यं न ब‚हुषु चेत् क्र‚मेणानेक‚ज्ञान‚हेतुत्व‚म‚नेनाभ्युपेय‚मेव ।}} ‚{\tiny $_{lb}$}‚‚{\color{DodgerBlue3}‚क्र‚माद‚पि} ॥
	\pend% ending standard par
      

	  \pstart \leavevmode% starting standard par
	किञ्चै\edtext{}{\edlabel{pvv.49-6}\label{pvv.49-6}\lemma{किञ्चै}\Bfootnote{य‚द्य‚प्युक्त‚मेकः प्राणादिरेकां धियं व्य‚न‚क्तीति त‚न्न ।}}क‚स्मिन्न‚पि ‚{\color{DodgerBlue3}‚प्राणेऽनेकेषा}‚म‚{\color{DodgerBlue3}‚र्थानां ग्र‚ह‚णाद}‚नेका‚{\color{DodgerBlue3}‚भिर्ब्बु}‚द्धि\edtext{}{\edlabel{pvv.49-7}\label{pvv.49-7}\lemma{द्धि}\Bfootnote{अनेक‚क्ष‚ण‚भावित्वादेक‚प्राण‚स्य एक‚प्राण‚कालेऽनेका बुद्धिः प्र‚व‚र्त‚ते निव‚र्त‚ते च ।}}भिरेकः प्राणः ‚{\tiny $_{lb}$}‚एकां बुद्धिम‚भिव्य‚न‚क्तीति नाय‚न्त‚त एक‚प्राणा‚{\color{DodgerBlue3}‚न्निय‚मः} । (१०७)
	\pend% ending standard par
      \label{div_pvv.1.108}
	  
	% new div opening: depth here is 2
	
	  \bigskip
	  \begingroup
	
	    \large
	  
	    \begin{quote}
	  
	    
	    \stanza[\smallbreak]
	\label{pv.1.108}\flagstanza{\tiny\textenglish{....1.108}}एक‚याऽनेक‚विज्ञाने बुद्ध्याऽस्तु स‚कृदेव त‚त् ।&अविरोधात्; क्र‚मेणापि मा भूत् त‚द‚विशेष‚तः ॥ १०८ ॥\&[\smallbreak]


	
	    \end{quote}
	  
	  \endgroup
	

	  \pstart \leavevmode% starting standard par
	\hphantom{.}‚{\color{DodgerBlue3}‚एक\edtext{}{\edlabel{pvv.49-8}\label{pvv.49-8}\lemma{एक}\Bfootnote{आस‚र्ग‚प्र‚ल‚यादेरेकैव बुद्धिरिति सिद्धान्तात् ।}}या बुद्ध्याऽनेक‚वि\edtext{}{\edlabel{pvv.49-9}\label{pvv.49-9}\lemma{वि}\Bfootnote{अथ क्र‚म‚व‚त्येकैव त‚दा बुद्धिरिष्य‚ते त‚दा ।}}ज्ञाने} च स्वीक्रिय‚माणेऽ‚{\color{DodgerBlue3}‚स्तु स‚कृदेव त‚त्} याव‚द् ‚{\tiny $_{lb}$}‚ग्र‚हीत‚व्य‚ग्र‚ह‚ण‚{\color{DodgerBlue3}‚म‚विरोधात्} य‚द्येक‚स्या अनेक‚ग्र‚ह‚णं विरुध्य‚ते क‚तिचित्प‚दार्थ‚ग्र‚ह‚ण‚{\tiny $_{lb}$}‚म‚प्येक‚दा न स्यात् । न विरुध्य‚ते चेत् याव‚द्ग्र‚{\tiny $_{4}$}‚हीत‚व्यं गृह्णीया\edtext{}{\edlabel{pvv.49-10}\label{pvv.49-10}\lemma{गृह्णीया}\Bfootnote{दीर्घ‚श्च सितादौ याव‚न्त‚स्ते त‚त्रैव ।}}त् । ‚{\color{DodgerBlue3}‚अन्य‚था ‚{\tiny $_{lb}$}‚क्र‚मेणापि मा भूद}‚नेक‚ग्र‚ह‚णं । ‚{\color{DodgerBlue3}‚त‚द‚विशेष‚तो} बुद्धेर्व्विशेषाभावात् । (१०८)
	\pend% ending standard par
      \label{div_pvv.1.109}
	  
	% new div opening: depth here is 2
	
	  \bigskip
	  \begingroup
	
	    \large
	  
	    \begin{quote}
	  
	    
	    \stanza[\smallbreak]
	\label{pv.1.109}\flagstanza{\tiny\textenglish{....1.109}}ब‚ह‚वः क्ष‚णिकाः प्राणा अस्व‚जातीय‚कालिकाः ।&तादृशामेव चित्तानां क‚ल्प्य‚न्ते य‚दि कार‚ण‚म् ॥ १०९ ॥\&[\smallbreak]


	
	    \end{quote}
	  
	  \endgroup
	

	  \pstart \leavevmode% starting standard par
	\hphantom{.}‚{\color{DodgerBlue3}‚ब‚ह\edtext{}{\edlabel{pvv.49-11}\label{pvv.49-11}\lemma{ह}\Bfootnote{नैकः प्राणोऽनेक‚विज्ञान‚हेतुः किन्तु ।}}वःक्ष‚णिकाः प्राणा अस्व‚जातीय‚कालिका} अस‚ह\edtext{}{\edlabel{pvv.49-12}\label{pvv.49-12}\lemma{ह}\Bfootnote{य‚देकः प्राणो न त‚दाऽप‚रः ।}}भाविनः । ‚{\color{DodgerBlue3}‚तादृ}\edtext{\textsuperscript{*}}{\edlabel{pvv.49-13}\label{pvv.49-13}\lemma{*}\Bfootnote{क्र‚मिणां ।}} ‚{\tiny $_{lb}$}‚शामेक‚क्ष‚णिकानां ब‚हूनाम‚स‚ह‚भाविनां ‚{\color{DodgerBlue3}‚चित्तानां क‚ल्प्य‚न्ते य‚दि कार‚णं} । (। १०९)
	\pend% ending standard par
      \label{div_pvv.1.110}
	  
	% new div opening: depth here is 2
	\textsuperscript{\textenglish{050/s}}
	  \bigskip
	  \begingroup
	
	    \large
	  
	    \begin{quote}
	  
	    
	    \stanza[\smallbreak]
	\label{pv.1.110}\flagstanza{\tiny\textenglish{....1.110}}क्र‚म‚व‚न्तः क‚थं ते स्युः क्र‚म‚व‚द्धेतुना विना ।&पूर्व‚स्व‚जातिहेतुत्वे न स्यादाद्य‚स्य स‚म्भ‚वः ॥ ११० ॥\&[\smallbreak]


	
	    \end{quote}
	  
	  \endgroup
	

	  \pstart \leavevmode% starting standard par
	\hphantom{.}त‚दां ‚{\color{DodgerBlue3}‚क्र‚म‚व‚न्तः क‚थ‚न्ते} प्राणाः ‚{\color{DodgerBlue3}‚स्युः क्र‚म‚व‚द्धेतुना विना} । श‚री\edtext{}{\edlabel{pvv.50-1}\label{pvv.50-1}\lemma{री}\Bfootnote{अथैत‚द्दोष‚त‚या देहं त्य‚क्त्वा पूर्व्व‚पूर्व्व‚प्राण‚हेतुकाः प्राणा इष्टास्त‚दा ।}}रं तेषां हेतुः ‚{\tiny $_{lb}$}‚त‚च्चाक्र‚मं । न चाक्र‚मात्क्र‚मिकार्यं युक्तं ‚{\color{DodgerBlue3}‚पूर्व्व‚स्व‚जा}‚तिहेतुत्वे ‚{\color{DodgerBlue3}‚न स्यादाद्य‚स्य} प्राण‚{\tiny $_{5}$}‚स्य ‚{\tiny $_{lb}$}‚‚{\color{DodgerBlue3}‚स‚म्भ‚वः} । (११०)
	\pend% ending standard par
      \label{div_pvv.1.111}
	  
	% new div opening: depth here is 2
	

	  \begin{center}%% label @type='head'
	\textbf{(III. प‚र‚लोकानाग‚त‚स्य प्राणे निर्हेतुक‚ता)}
	\end{center}
	

	  \pstart \leavevmode% starting standard par
	न हि प‚र‚लोकाग‚तः प्राणोस्ति य आद्य‚स्य श‚रीर‚स‚म्ब‚न्धिनो हेतुः स्यात् । ‚{\tiny $_{lb}$}‚एत‚देवाह ।
	\pend% ending standard par
      
	  \bigskip
	  \begingroup
	
	    \large
	  
	    \begin{quote}
	  
	    
	    \stanza[\smallbreak]
	\label{pv.1.111}\flagstanza{\tiny\textenglish{....1.111}}त‚द्धेतुस्तादृशो नास्ति स‚ति वाऽक‚नेता ध्रुव‚म् ।&प्राणानां भिन्न‚देश‚त्वात् स‚कृज्ज‚न्म धियाम‚तः ॥ १११ ॥\&[\smallbreak]


	
	    \end{quote}
	  
	  \endgroup
	

	  \pstart \leavevmode% starting standard par
	\edtext{\textsuperscript{*}}{\edlabel{pvv.50-2}\label{pvv.50-2}\lemma{*}\Bfootnote{त‚दा ज्ञानेपि प‚टुत्वादि ।}}त‚द्धेतुस्तादृशो नास्ति । ‚{\color{DodgerBlue3}‚स‚ति} च त‚द्धे\edtext{}{\edlabel{pvv.50-3}\label{pvv.50-3}\lemma{द्धे}\Bfootnote{अभ्युप‚ग‚म्याह ।}}ताव‚{\color{DodgerBlue3}‚नेक‚ता ध्रुवं प्राणानां भिन्न‚देश‚त्वात्} । ‚{\tiny $_{lb}$}‚दे\edtext{}{\edlabel{pvv.50-4}\label{pvv.50-4}\lemma{दे}\Bfootnote{ब‚हुत्वं यूथ‚व‚न्म‚न्य‚ते ।}}श‚भेदात् प्राणाः प्र‚तिदेशं भिन्ना ‚{\color{DodgerBlue3}‚अतो}‚ऽनेक‚त्वात् प्राणानां प्र‚त्येकं स‚म‚र्थानां तेभ्यो ‚{\tiny $_{lb}$}‚व्य‚ञ्च‚कोऽनेक‚त्वाद‚तो ‚{\color{DodgerBlue3}‚धियां ज‚न्म स‚कृत्} स्यादिति प्र‚स‚ङ्गः । (१११)
	\pend% ending standard par
      \label{div_pvv.1.112}
	  
	% new div opening: depth here is 2
	
	  \bigskip
	  \begingroup
	
	    \large
	  
	    \begin{quote}
	  
	    
	    \stanza[\smallbreak]
	\label{pv.1.112}\flagstanza{\tiny\textenglish{....1.112}}य‚द्येक‚कालिकोऽनेकोऽप्येक‚चैत‚न्य‚कार‚ण‚म् ।&एक‚स्यापि न वैक‚ल्ये स्यान्म‚न्द‚श्व‚सितादिषु ॥ ११२ ॥\&[\smallbreak]


	
	    \end{quote}
	  
	  \endgroup
	

	  \pstart \leavevmode% starting standard par
	\hphantom{.}‚{\color{DodgerBlue3}‚य‚द्येक‚कालिकोऽनेकः} प्राण\edtext{}{\edlabel{pvv.50-5}\label{pvv.50-5}\lemma{प्राण}\Bfootnote{युग‚प‚द्‏ब‚हुस‚म्भ‚व‚निरासार्थं ।}} ‚{\color{DodgerBlue3}‚एक‚चैत‚न्य‚कार‚ण}‚मिष्य‚ते त‚दै‚{\color{DodgerBlue3}‚क‚स्यापि} प्राण‚स्य ‚{\color{DodgerBlue3}‚म‚न्द‚{\tiny $_{lb}$}‚श्व‚सितादिषु वैक‚{\tiny $_{6}$}‚ ल्ये} स‚ति ‚{\color{DodgerBlue3}‚न स्याच्चै}‚त‚न्यं कार‚णानाम‚स‚म‚ग्र‚त्वात् । (११२)
	\pend% ending standard par
      \label{div_pvv.1.113}
	  
	% new div opening: depth here is 2
	
	  \bigskip
	  \begingroup
	
	    \large
	  
	    \begin{quote}
	  
	    
	    \stanza[\smallbreak]
	\label{pv.1.113}\flagstanza{\tiny\textenglish{....1.113}}अथ हेतुर्य‚थाभावं ज्ञानेऽपि स्याद् विशिष्ट‚ता ।&न हि त‚त् त‚स्य कार्यं य‚द् य‚स्य भेदान्न भिद्य‚ते ॥ ११३ ॥\&[\smallbreak]


	
	    \end{quote}
	  
	  \endgroup
	

	  \pstart \leavevmode% starting standard par
	\hphantom{.}‚{\color{DodgerBlue3}‚अथ हेतुर्य‚थाभावं} य‚थास‚म्भ‚वं प्राणानां हेतुत्वं स्यात् । त‚दा ‚{\color{DodgerBlue3}‚ज्ञानेपि स्याद् ‚{\tiny $_{lb}$}‚विशिष्ट‚ता} (।) प्राणानामुप‚च‚याप‚च‚याभ्यां ज्ञान‚म‚पि तादृशं स्यात् । ‚{\color{DodgerBlue3}‚न हि त‚त्त‚स्य ‚{\tiny $_{lb}$}‚कार्यं}\edtext{\textsuperscript{*}}{\edlabel{pvv.50-6}\label{pvv.50-6}\lemma{*}\Bfootnote{न च ज्ञानं त‚द‚र्थ‚कारि ।}}युक्तं ‚{\color{DodgerBlue3}‚य‚द्य‚स्य भेदान्न भिद्य‚ते} । (११३)
	\pend% ending standard par
      \label{div_pvv.1.114}
	  
	% new div opening: depth here is 2
	

	  \begin{center}%% label @type='head'
	\textbf{(IV. श‚क्तिनिय‚माद् न धियां स‚कृज्ज‚न्त)}
	\end{center}
	

	  \pstart \leavevmode% starting standard par
	त्व‚न्म‚तेपि धियां स‚कृज्ज‚न्म क‚स्मान्न भ‚व‚तीत्याह ।
	\pend% ending standard par
      \textsuperscript{\textenglish{051/s}}
	  \bigskip
	  \begingroup
	
	    \large
	  
	    \begin{quote}
	  
	    
	    \stanza[\smallbreak]
	\label{pv.1.114a}\flagstanza{\tiny\textenglish{...1.114a}}विज्ञानं श‚क्तिनिय‚मादेक‚मेक‚स्य कार‚ण‚म् ।\&[\smallbreak]


	
	    \end{quote}
	  
	  \endgroup
	

	  \pstart \leavevmode% starting standard par
	\hphantom{.}‚{\color{DodgerBlue3}‚विज्ञानं श‚क्तिनिय‚मादेकंविज्ञान‚मे\edtext{}{\edlabel{pvv.51-1}\label{pvv.51-1}\lemma{मे}\Bfootnote{स‚जातीयं ।}} कं} । ‚{\color{DodgerBlue3}‚श‚क्तेः} स्व‚का\edtext{}{\edlabel{pvv.51-2}\label{pvv.51-2}\lemma{का}\Bfootnote{न युग‚प‚त्स‚मान‚जातीयानि विज्ञानानि विजातीयानि तु स्युः ।}}र‚ण‚कृताया ‚{\color{DodgerBlue3}‚निय‚मादे‚{\tiny $_{lb}$}‚क‚स्य} विज्ञान‚स्य ‚{\color{DodgerBlue3}‚कार‚ण}‚मिति न स‚कृद् धियां ज‚न्म ।
	\pend% ending standard par
      

	  \pstart \leavevmode% starting standard par
	कुत एव‚{\tiny $_{7}$}‚दिति चेदाह ।
	\pend% ending standard par
      
	  \bigskip
	  \begingroup
	
	    \large
	  
	    \begin{quote}
	  
	    
	    \stanza[\smallbreak]
	\label{pv.1.114b}\flagstanza{\tiny\textenglish{...1.114b}}अन्यार्थास‚क्तिविगुणे ज्ञाने नार्थान्त‚राग्र‚हात् ॥ ११४ ॥\&[\smallbreak]


	
	    \end{quote}
	  
	  \endgroup
	

	  \pstart \leavevmode% starting standard par
	\hphantom{.}अन्य‚स्मिन्न‚र्थे आस‚क्त्या पुनः पुनः प्र‚वृत्यात्मिक‚या ‚{\color{DodgerBlue3}‚विगुणे} विष‚यान्त‚र‚स‚ञ्चारि-\leavevmode\ledsidenote{\textenglish{11a/MA}} ‚{\tiny $_{lb}$}‚ज्ञानोत्पाद‚विरोधानि ‚{\color{DodgerBlue3}‚ज्ञाने} पूर्व्व‚के स‚ति विष‚यान्त‚र‚स्याग्र‚ह‚णात् । अविगुणे तु ग्र‚ह‚{\tiny $_{lb}$}‚णात्\edtext{}{\edlabel{pvv.51-3}\label{pvv.51-3}\lemma{णात्}\Bfootnote{कालुष्य‚प्र‚सादादिम‚त्स्व‚पूर्व्व‚पूर्व्व‚विज्ञानेषूत्त‚राण्य‚पि त‚था स्युः ।}}। त‚स्मात् ज्ञान‚कार्यं ज्ञानं । त‚द‚न्व‚य‚व्य‚तिरेकानुविधानात् । (११४)
	\pend% ending standard par
      \label{div_pvv.1.115}
	  
	% new div opening: depth here is 2
	

	  \begin{center}%% label @type='head'
	\textbf{(ङ) क‚र्म‚सिद्धिः}
	\end{center}
	

	  \begin{center}%% label @type='head'
	\textbf{I. स‚ह‚स्थितिकार‚णं क‚र्म,}
	\end{center}
	
	  \bigskip
	  \begingroup
	
	    \large
	  
	    \begin{quote}
	  
	    
	    \stanza[\smallbreak]
	\label{pv.1.115}\flagstanza{\tiny\textenglish{....1.115}}श‚रीरात् स‚कृदुत्प‚न्ना धीः स्व‚जात्या निय‚म्य‚ते ।&प‚र‚त‚श्चेत् स‚म‚र्थ‚स्य देह‚स्य विर‚तिः कुतः ॥ ११५ ॥\&[\smallbreak]


	
	    \end{quote}
	  
	  \endgroup
	

	  \pstart \leavevmode% starting standard par
	\hphantom{.}अथ ‚{\color{DodgerBlue3}‚श‚रीरा\edtext{}{\edlabel{pvv.51-4}\label{pvv.51-4}\lemma{रीरा}\Bfootnote{ग‚र्भादौ काय‚प‚र‚माण‚वो ज‚न‚य‚न्त्येकं चैत‚न्यं ।}} त्स‚कृ}‚त्प्र‚थ‚म‚मुत्प‚न्ना धीः ‚{\color{DodgerBlue3}‚स्व‚जात्या निय‚म्य‚ते प‚र‚त} एक‚स्या बुद्धेरेका ‚{\tiny $_{lb}$}‚बुद्धिर्भ‚व‚तीति न स‚कृज्ज‚न्म‚प्र‚स‚ङ्गः । न‚नु बुद्धिज‚न‚ने प्र‚थ‚मं ‚{\color{DodgerBlue3}‚स‚म‚र्थ‚स्य देह‚स्य} प‚श्चा‚{\color{DodgerBlue3}‚द्विर‚ति}‚{\tiny $_{1}$}‚स्त‚ज्ज‚न‚नात् ‚{\color{DodgerBlue3}‚कुतः} । येन बुद्धिर्न्नियामिका स्यात् प‚र‚तः । (११५)
	\pend% ending standard par
      \label{div_pvv.1.116}
	  
	% new div opening: depth here is 2
	
	  \bigskip
	  \begingroup
	
	    \large
	  
	    \begin{quote}
	  
	    
	    \stanza[\smallbreak]
	\label{pv.1.116}\flagstanza{\tiny\textenglish{....1.116}}अनाश्र‚यान्निवृत्ते स्याच्छ‚रीरे चेत‚सः स्थितिः ।&केव‚ल‚स्येति चेच्चित्त‚स‚न्तान‚स्थितिकार‚ण‚म् ॥ ११६ ॥\&[\smallbreak]


	
	    \end{quote}
	  
	  \endgroup
	

	  \pstart \leavevmode% starting standard par
	\hphantom{.}न‚नु य‚दि बुद्धेर्न देह आश्र‚य‚स्त‚दाऽ‚{\color{DodgerBlue3}‚नाश्र‚यादाश्र}‚याभावात् ‚{\color{DodgerBlue3}‚निवृत्ते श‚रीरे केव‚ल‚स्य ‚{\tiny $_{lb}$}‚चेत‚सः स्थितिः स्यादिति चेत्} । स्यादेत‚त्\edtext{}{\edlabel{pvv.51-5}\label{pvv.51-5}\lemma{त्}\Bfootnote{आरूप्य‚भाव ।}} श‚रीरेण स‚ह ‚{\color{DodgerBlue3}‚चित्त‚स‚न्तान‚स्य स्थितिकार‚णं} दृष्टं\edtext{}{\edlabel{pvv.51-6}\label{pvv.51-6}\lemma{दृष्टं}\Bfootnote{प‚ञ्चाय‚त‚नं}}स‚हायं य‚त्क‚र्म (११६ ।)
	\pend% ending standard par
      \label{div_pvv.1.117}
	  
	% new div opening: depth here is 2
	

	  \begin{center}%% label @type='head'
	\textbf{(II. आमुत्रिक‚देह‚हेतुः प‚ञ्चाय‚त‚न‚मैहिक‚म्)}
	\end{center}
	
	  \bigskip
	  \begingroup
	
	    \large
	  
	    \begin{quote}
	  
	    
	    \stanza[\smallbreak]
	\label{pv.1.117a}\flagstanza{\tiny\textenglish{...1.117a}}त‚द्धेतुवृत्तिलाभाय नाङ्ग‚तां य‚दि ग‚च्छ‚ति ।\&[\smallbreak]


	
	    \end{quote}
	  
	  \endgroup
	

	  \pstart \leavevmode% starting standard par
	\hphantom{.}a. त‚द्य‚दि त‚स्य पार‚लौकिक‚स्य ‚{\color{DodgerBlue3}‚देह‚स्य हेतु}‚रैहिक‚म‚न्त्यं प‚ञ्चा\edtext{}{\edlabel{pvv.51-7}\label{pvv.51-7}\lemma{ञ्चा}\Bfootnote{इन्द्रियं ।}}य‚त‚नं त‚स्य ‚{\color{DodgerBlue3}‚वृत्तिः} \leavevmode\ledsidenote{\textenglish{052/s}} पार‚लौकिक‚देह‚ज‚न‚नाद्याभिमुख्यं । त‚स्य ‚{\color{DodgerBlue3}‚लाभाय} प्राप्त‚येऽ‚{\color{DodgerBlue3}‚ङ्ग‚तां} स‚ह‚कारितां ‚{\color{DodgerBlue3}‚न ‚{\tiny $_{lb}$}‚ग‚च्छ‚ति} त‚दा केव‚लं चित्तं तिष्ठ‚ति । य‚था विरूपे धातौ ।
	\pend% ending standard par
      

	  \pstart \leavevmode% starting standard par
	कः पुन‚रामुत्रिक‚देह‚हेतुः य‚स्य चित्तं स‚ह‚कारीत्याह ।
	\pend% ending standard par
      
	  \bigskip
	  \begingroup
	
	    \large
	  
	    \begin{quote}
	  
	    
	    \stanza[\smallbreak]
	\label{pv.1.117b}\flagstanza{\tiny\textenglish{...1.117b}}हेतुर्देहान्त‚रोत्प‚त्तौ प‚ञ्चाय‚त‚न‚मैहिक‚म् ॥ ११७ ॥\&[\smallbreak]


	
	    \end{quote}
	  
	  \endgroup
	

	  \pstart \leavevmode% starting standard par
	\hphantom{.}‚{\color{DodgerBlue3}‚हेतुर्देहान्त‚रोत्प‚त्तौ प‚ञ्चाय‚त‚नं} प‚ञ्चेन्द्रियाणि । ऐहिक‚मिदं ज‚न्म‚भ‚वं (। ११७)
	\pend% ending standard par
      \label{div_pvv.1.118}
	  
	% new div opening: depth here is 2
	
	  \bigskip
	  \begingroup
	
	    \large
	  
	    \begin{quote}
	  
	    
	    \stanza[\smallbreak]
	\label{pv.1.118}\flagstanza{\tiny\textenglish{....1.118}}त‚द‚ङ्ग‚भाव‚हेतुत्व‚निषेधेऽनुप‚ल‚म्भ‚न‚म् ।&अनिश्च‚य‚क‚रं प्रोक्तं । इन्द्रियाद्य‚पि शेष‚व‚त् ॥ ११८ ॥\&[\smallbreak]


	
	    \end{quote}
	  
	  \endgroup
	

	  \pstart \leavevmode% starting standard par
	\hphantom{.}य‚च्च त‚योः क‚र्म-ऐहिक‚प‚ञ्चाय‚त‚न‚योर्य‚थाक्र‚म‚{\color{DodgerBlue3}‚म‚ङ्ग‚भाव‚हेतुत्व‚योः} स‚ह‚कारितो‚{\tiny $_{lb}$}‚पादान‚त्व‚यो‚{\color{DodgerBlue3}‚र्निषेधे} क‚र्त्त‚व्येऽ\edtext{}{\edlabel{pvv.52-1}\label{pvv.52-1}\lemma{व्येऽ}\Bfootnote{न दृष्ट‚मेत‚त्स‚ह‚कारित्वादीति ।}} ‚{\color{DodgerBlue3}‚नुप‚ल‚म्भ‚नं} प‚रैरुच्य‚ते । त‚द‚{\color{DodgerBlue3}‚निश्च‚य‚क}‚र‚म‚नैकान्तिक‚म‚{\tiny $_{lb}$}‚दृश्य‚विष‚य‚त्वात् ‚{\color{DodgerBlue3}‚प्रोक्तं इन्द्रियाद्य‚पि शेष‚व‚त्} ।‚{\tiny $_{3}$}‚ य‚ज‚पीन्द्रियादि श‚रीरान्त‚र‚स‚म्ब‚न्धी‚{\tiny $_{lb}$}‚न्द्रियादि प्र‚तिस‚न्धातृ न भ\edtext{}{\edlabel{pvv.52-2}\label{pvv.52-2}\lemma{भ}\Bfootnote{म‚ध्याव‚स्थेन्द्रिय‚व‚त् । अव्याप्तेः ।}} \edtext{\textsuperscript{*}}{\edlabel{pvv.52-3a}\label{pvv.52-3a}\lemma{*}\Bfootnote{a अन्त्यं श‚रीरं ।\begin{english} --- Placement of note uncertain; marked with a question mark in the edition (see encoding description for details).\end{english}}} व‚तीन्द्रिय‚त्वादेरिति त‚द‚पि शे\edtext{}{\edlabel{pvv.52-3}\label{pvv.52-3}\lemma{शे}\Bfootnote{य‚था चैत्रेन्द्रियं मैत्रेन्द्रिय‚स्यातिप‚र‚स्य विजातीयासंन्धानं दृष्टान्तः । स्व‚स्य तु ‚{\tiny $_{lb}$}‚स‚जातीयं स‚न्धाय‚कं पूर्व्वेन्द्रियेण प‚र‚स‚न्धान‚दृष्टेः ।}} \edtext{\textsuperscript{*}}{\edlabel{pvv.52-4a}\label{pvv.52-4a}\lemma{*}\Bfootnote{a अन्त्य‚चित्तं न देह‚स‚ह‚कारि देहान्त‚रोत्पाद‚ने चित्त‚त्वात् पूर्व्व‚चित्त‚व‚ता\begin{english} --- Placement of note uncertain; marked with a question mark in the edition (see encoding description for details)\end{english}}} ष‚व‚द‚नैकान्तिकं बोद्ध‚व्यं । ‚{\tiny $_{lb}$}‚आदिश‚ब्दात्प्राणापान‚त्वादि ज्ञान‚त्वादि\edtext{}{\edlabel{pvv.52-4}\label{pvv.52-4}\lemma{त्वादि}\Bfootnote{ज्ञानास‚ह‚कारित्वे पूतीभावः स्यात् ।}}च । (११८)
	\pend% ending standard par
      \label{div_pvv.1.119}
	  
	% new div opening: depth here is 2
	

	  \pstart \leavevmode% starting standard par
	b विरुद्ध‚त्व‚म‚पि द‚र्श‚य‚ति ।
	\pend% ending standard par
      
	  \bigskip
	  \begingroup
	
	    \large
	  
	    \begin{quote}
	  
	    
	    \stanza[\smallbreak]
	\label{pv.1.119}\flagstanza{\tiny\textenglish{....1.119}}दृष्टा च श‚क्तिः पूर्वेषामिन्द्रियाणां स्व‚जातिषु ।&विकार‚द‚र्श‚नात् सिद्धं अप‚राप‚र‚ज‚न्म च ॥ ११९ ॥\&[\smallbreak]


	
	    \end{quote}
	  
	  \endgroup
	

	  \pstart \leavevmode% starting standard par
	\hphantom{.}‚{\color{DodgerBlue3}‚दृष्टा च श‚क्तिः पूर्व्वेषामिन्द्रियाणां स्व‚जातिषु} क‚र्त्त‚व्येषु म‚ध्याव‚स्थायां ‚{\tiny $_{lb}$}‚त‚त इन्द्रियादित्वात् । स्व‚जातिप्र‚तिस‚न्धातृत्व‚मेवैषां युक्तं । म‚ध्याव‚स्थायाञ्च ‚{\tiny $_{lb}$}‚पूर्व्वाव‚स्थातः पाट\edtext{}{\edlabel{pvv.52-5}\label{pvv.52-5}\lemma{पाट}\Bfootnote{प्र‚स‚न्नाविल‚त्वादिना ।}}वादि‚{\color{DodgerBlue3}‚विकार‚द‚र्श‚नात् सिद्धं} प्र‚तिक्ष‚ण‚{\color{DodgerBlue3}‚म‚प‚राप‚र‚ज‚न्म चे}‚न्द्रिया‚{\tiny $_{4}$}‚दी‚{\tiny $_{lb}$}‚नामिति नादृष्टान्तो हेतुः । (११९)
	\pend% ending standard par
      \label{div_pvv.1.120}
	  
	% new div opening: depth here is 2
	

	  \pstart \leavevmode% starting standard par
	c न‚नु श‚रीरादेवेन्द्रियादीनां ज‚न्मेत्याह ।
	\pend% ending standard par
      
	  \bigskip
	  \begingroup
	
	    \large
	  
	    \begin{quote}
	  
	    
	    \stanza[\smallbreak]
	\label{pv.1.120}\flagstanza{\tiny\textenglish{....1.120}}श‚रीराद् य‚दि त‚ज्ज‚न्म प्र‚स‚ङ्गः पूर्व‚व‚द् भ‚वेत् ।&चित्ताच्चेत् त‚त एवास्तु ज‚न्म देहान्त‚र‚स्य च ॥ १२० ॥\&[\smallbreak]


	
	    \end{quote}
	  
	  \endgroup
	\textsuperscript{\textenglish{053/s}}

	  \pstart \leavevmode% starting standard par
	\hphantom{.}‚{\color{DodgerBlue3}‚श‚रीराद्य‚दि} तेषामिन्द्रियादीनां ‚{\color{DodgerBlue3}‚ज‚न्म}‚{\tiny $_{1}$}‚त‚दा ‚{\color{DodgerBlue3}‚पूर्व्व‚व}‚द्धे\edtext{}{\edlabel{pvv.53-1}\label{pvv.53-1}\lemma{द्धे}\Bfootnote{युग‚प‚द् ब‚हुत्वं मृतेपि ।}} तुत्वे च स‚म‚स्तानामित्यादि‚{\tiny $_{lb}$}‚नोक्तः ‚{\color{DodgerBlue3}‚प्र‚स‚ङ्गो भ‚वेत्} । ‚{\color{DodgerBlue3}‚चित्ताच्चे}‚दिन्द्रिय‚चित्तादीनां ज‚न्म श‚रीराज्ज‚न्म‚नि ‚{\tiny $_{lb}$}‚दोष‚द‚र्श‚नादिष्य‚ते त‚दान्त्याव‚स्थायाम‚प्य‚वि‚{\color{DodgerBlue3}‚क‚ल‚त्वा}‚च्चित्त‚स्य ‚{\color{DodgerBlue3}‚त‚त} एवास्तु ‚{\color{DodgerBlue3}‚ज‚न्म देहा‚{\tiny $_{lb}$}‚न्त‚र‚स्य} प‚ञ्चाय‚त‚न‚रूप‚स्यान‚ग‚त‚स्य । (१२०)
	\pend% ending standard par
      \label{div_pvv.1.121}
	  
	% new div opening: depth here is 2
	
	  \bigskip
	  \begingroup
	
	    \large
	  
	    \begin{quote}
	  
	    
	    \stanza[\smallbreak]
	\label{pv.1.121}\flagstanza{\tiny\textenglish{....1.121}}त‚स्मान्न हेतुवैक‚ल्यात् स‚र्वेषाम‚न्त्य‚चेत‚साम् ।&अस‚न्धिरीदृशं तेन शेष‚व‚त् साध‚नं म‚त‚म् ॥ १२१ ॥\&[\smallbreak]


	
	    \end{quote}
	  
	  \endgroup
	

	  \pstart \leavevmode% starting standard par
	d य‚त‚श्चित्त‚मेव चित्त‚स्य हेतुः । तृष्णाक‚र्म‚स‚हाय‚ञ्च प‚ञ्चाय‚{\tiny $_{5}$}‚त‚न‚स्य । ‚{\color{DodgerBlue3}‚त‚स्मान्न ‚{\tiny $_{lb}$}‚हेतुवैक‚ल्यात् । स‚र्व्वेषाम‚न्त्य‚चेत‚साम‚स‚न्धिः} चित्त‚स्य प‚ञ्चाय‚त‚न‚स्य ‚{\color{DodgerBlue3}‚च} हेत्व‚वै‚{\tiny $_{lb}$}‚क‚ल्यात् कार्योत्पाद‚स्याव‚श्य‚म्भावित्वात् (।) तेनेदृश‚म‚न्त्य‚चित्त‚त्वादि ‚{\color{DodgerBlue3}‚शेष}‚व‚द‚{\tiny $_{lb}$}‚नैकान्तिकं ‚{\color{DodgerBlue3}‚साध‚नं म‚तं} ॥ (१२१)
	\pend% ending standard par
      \label{div_pvv.1.122_1.123}
	  
	% new div opening: depth here is 2
	

	  \begin{center}%% label @type='head'
	\textbf{(ख. युक्तः क‚रुणाभ्यासः)}
	\end{center}
	

	  \begin{center}%% label @type='head'
	\textbf{(क) चित्त‚मात्र‚प्र‚तिब‚द्ध‚त्वात्}
	\end{center}
	

	  \pstart \leavevmode% starting standard par
	त‚देवं चित्त‚मात्र‚प्र‚तिब‚द्ध‚त्वाच्चित्त‚ज‚न्म‚नो देह‚निवृत्ताव‚पि ज‚न्म‚प‚र‚म्प‚रा‚{\tiny $_{lb}$}‚स‚म्भ‚वे युक्तः कृपाभ्यास इत्य‚भ्यासात्सेति स‚म‚र्थितं ।
	\pend% ending standard par
      
	  \bigskip
	  \begingroup
	
	    \large
	  
	    \begin{quote}
	  
	    
	    \stanza[\smallbreak]
	\label{pv.1.122a}\flagstanza{\tiny\textenglish{...1.122a}}अभ्यासेन विशेषेऽपि ल‚ङ्घ‚नोद‚क‚ताप‚व‚त् ।&स्व‚भावातिक्र‚मो मा भूदिति चेद्;\&[\smallbreak]


	
	    \end{quote}
	  
	  \endgroup
	

	  \pstart \leavevmode% starting standard par
	\hphantom{.}‚{\color{DodgerBlue3}‚न‚न्व‚भ्यासेन विशेषेपि} स‚त्य‚ल्पीय‚सि ‚{\color{DodgerBlue3}‚स्व‚भाव‚स्य} कृपादेस्त‚द्विप‚क्ष‚{\tiny $_{6}$}‚संकीर्ण‚त्व‚{\tiny $_{lb}$}‚स्यातिक्र‚मो विप‚क्षाव्य‚व‚कीर्ण्ण‚स्व‚र‚स‚प्र‚वृत्त‚कृपादिम‚य‚ता सात्मीभावो ‚{\color{DodgerBlue3}‚मा भूत् ‚{\tiny $_{lb}$}‚लंघ‚नोद‚क‚ताप‚व‚त्} ।
	\pend% ending standard par
      

	  \begin{center}%% label @type='head'
	\textbf{(ख) पुन‚र्य‚त्नापेक्षा}
	\end{center}
	

	  \pstart \leavevmode% starting standard par
	न हि पुरुषोत्य‚र्थं ल‚ङ्घ‚ने कृताभ्यासो योज‚न‚म‚र्द्ध‚योज‚नं वा ल‚ङ्घ‚य‚ति । नाप्युद‚{\tiny $_{lb}$}‚क‚मेकान्तं ताप्य‚मानं द‚ह‚नीभ‚व‚ति । किन्तु प्र‚कृतिसिद्धात् ल‚ङ्घ‚नात् स्प‚र्शाच्च ‚{\tiny $_{lb}$}‚विशेष‚मात्रं भ‚व‚ति य‚था (।) त‚थोत्क‚र्ष‚मात्रं स्यात् कृप‚या न तु सात्मीभाव ‚{\color{DodgerBlue3}‚इति ‚{\tiny $_{lb}$}‚चेत्} । अत्राह (।)
	\pend% ending standard par
      
	  \bigskip
	  \begingroup
	
	    \large
	  
	    \begin{quote}
	  
	    
	    \stanza[\smallbreak]
	\label{pv.1.122b}\flagstanza{\tiny\textenglish{...1.122b}}आहितः स चेत् ॥ १२२ ॥\&[\smallbreak]


	
	    \end{quote}
	  
	  \endgroup
	
	  \bigskip
	  \begingroup
	
	    \large
	  
	    \begin{quote}
	  
	    
	    \stanza[\smallbreak]
	\label{pv.1.123}\flagstanza{\tiny\textenglish{....1.123}}पुन‚र्य‚त्न‚म‚पेक्षेत य‚दि स्याच्चास्थिराश्र‚यः ।&विशेषो नैव व‚र्द्धेत स्व‚भाव‚श्च न तादृशः ॥ १२३ ॥\&[\smallbreak]


	
	    \end{quote}
	  
	  \endgroup
	\textsuperscript{\textenglish{054/s}}

	  \pstart \leavevmode% starting standard par
	\hphantom{.}‚{\color{DodgerBlue3}‚आहितः स चेत्} विशेष (:। १२२) आधाय‚क‚निवृत्तावात्म‚लाभाय ‚{\color{DodgerBlue3}‚पुन‚र्य‚त्न‚म‚{\tiny $_{lb}$}‚पेक्षेत} न स्व‚र‚स‚वाही स्यात् । ल‚ङ्घ‚नं य‚थाभ्य‚स्त‚म‚पि पुन‚र्य‚त्नापेक्ष‚यैव प्र‚व‚र्त‚ते न ‚{\tiny $_{lb}$}‚‚{\color{DodgerBlue3}‚स्व‚र‚स‚वाहि} । ‚{\color{DodgerBlue3}‚य‚दि स्याच्चास्थिराश्र‚य} उद‚क‚ताप‚व‚त् । क्वाथ्य‚मानं ह्य‚द‚कं क्षीय\edtext{}{\edlabel{pvv.54-1}\label{pvv.54-1}\lemma{क्षीय}\Bfootnote{न ज्व‚ल‚ति ।}} त ‚{\tiny $_{lb}$}‚एव इत्य‚स्थिराश्र‚य उद‚क‚तापः पुन‚र्य‚त्नापेक्षी च स्व‚र‚स‚वाहित्वाभावात् । त‚दा ‚{\tiny $_{lb}$}‚\leavevmode\ledsidenote{\textenglish{11b/MA}} ‚{\color{DodgerBlue3}‚विशेषो नैव व‚र्द्धेत} प्र‚क‚र्ष‚निष्ठां न ग‚च्छेत् (।) ‚{\color{DodgerBlue3}‚तादृश‚श्च} विशेषो ‚{\color{DodgerBlue3}‚नैव स्व‚भावः} ।\edtext{\textsuperscript{*}}{\edlabel{pvv.54-2}\label{pvv.54-2}\lemma{*}\Bfootnote{लंघ‚नादिविशेष‚व‚न्न प्र‚कृतिः । अन्यान‚पेक्ष‚त्वात्स्व‚भाव‚स्याकाश‚व‚त् ।}} ‚{\tiny $_{lb}$}‚प्र‚कृतिर्व्विशेष\edtext{}{\edlabel{pvv.54-3}\label{pvv.54-3}\lemma{कृतिर्व्विशेष}\Bfootnote{नैवं प्र‚ज्ञाद‚योऽभ्यासात् काष्ठानिष्टां प्र‚तिष्ठिताः ।}}व‚त् । हे\edtext{}{\edlabel{pvv.54-4}\label{pvv.54-4}\lemma{हे}\Bfootnote{विशेष‚स्य ।}}तुस‚न्निधा‚{\tiny $_{8}$}‚न‚व्य‚व‚धान‚सापेक्ष‚त्वात् प्र‚वृत्तिनिवृत्त्योः । ‚{\tiny $_{lb}$}‚(१२३)
	\pend% ending standard par
      \label{div_pvv.1.124}
	  
	% new div opening: depth here is 2
	

	  \pstart \leavevmode% starting standard par
	त‚था च (।)
	\pend% ending standard par
      
	  \bigskip
	  \begingroup
	
	    \large
	  
	    \begin{quote}
	  
	    
	    \stanza[\smallbreak]
	\label{pv.1.124}\flagstanza{\tiny\textenglish{....1.124}}त‚त्रोप‚युक्त‚श‚क्तीनां विशेषानुत्त‚रान् प्र‚ति ।&साध‚नानाम‚साम‚र्थ्यान्नित्य‚ञ्चानाश्र‚य‚स्थितेः ॥ १२४ ॥\&[\smallbreak]


	
	    \end{quote}
	  
	  \endgroup
	

	  \pstart \leavevmode% starting standard par
	\hphantom{.}‚{\color{DodgerBlue3}‚त‚त्र} पूर्व्व‚दृष्ट एव विशेषे ‚{\color{DodgerBlue3}‚उप‚युक्त‚श‚क्तीनां साध‚नानां} य‚त्नादीनां पुन‚र‚पि ‚{\tiny $_{lb}$}‚‚{\color{DodgerBlue3}‚विशेषानुत्त‚रान् प्र‚त्य‚साम‚र्थ्यात् नित्य‚ञ्चानाश्र‚य‚स्थितेः} । (१२४)
	\pend% ending standard par
      \label{div_pvv.1.125}
	  
	% new div opening: depth here is 2
	
	  \bigskip
	  \begingroup
	
	    \large
	  
	    \begin{quote}
	  
	    
	    \stanza[\smallbreak]
	\label{pv.1.125}\flagstanza{\tiny\textenglish{....1.125}}विशेष‚स्यास्व‚भाव‚त्वाद् वृद्धाव‚प्याहितो य‚दा ।&नापेक्षेत पून‚र्य‚त्नं य‚त्नोन्यः स्याद् विशेष‚कृत् ॥ १२५ ॥\&[\smallbreak]


	
	    \end{quote}
	  
	  \endgroup
	

	  \pstart \leavevmode% starting standard par
	\hphantom{.}आश्र‚य‚स्थित्य‚भावात् तादृश‚स्य ‚{\color{DodgerBlue3}‚विशेष‚स्यास्व‚भाव‚त्वात् वृद्धाव‚पि} व्य‚व‚स्थितो‚{\tiny $_{lb}$}‚त्क‚र्ष‚तैव । पुन‚र्य‚त्नापेक्षित्वेनास्थिराश्र‚य‚त्वेन व्य‚व‚स्थितोत्क‚र्ष‚ता व्याप्तेत्य‚र्थः । ‚{\color{DodgerBlue3}‚य‚दा} तु विशेष ‚{\color{DodgerBlue3}‚आहितो नापेक्षेत पुन‚र्य‚त्नं} प्रागुत्प‚न्न‚स्यात्म‚नो लाभायापि तु स्व‚र‚{\tiny $_{1}$}‚स‚वाही ‚{\tiny $_{lb}$}‚‚{\color{DodgerBlue3}‚भ‚व‚ति त‚दा य‚त्नोऽन्यः} क्रिय‚माणो ‚{\color{DodgerBlue3}‚विशेष‚कृत्} य‚थाभ्यास‚मुत्त‚रोत्त‚र‚विशेषाधायी ‚{\tiny $_{lb}$}‚भ‚व‚ति । (१२५)
	\pend% ending standard par
      \label{div_pvv.1.126}
	  
	% new div opening: depth here is 2
	

	  \begin{center}%% label @type='head'
	\textbf{(ग) स्व‚र‚सेनाभ्यास‚जः क‚रुणादिप्र‚वाहः}
	\end{center}
	
	  \bigskip
	  \begingroup
	
	    \large
	  
	    \begin{quote}
	  
	    
	    \stanza[\smallbreak]
	\label{pv.1.126}\flagstanza{\tiny\textenglish{....1.126}}काष्ठ‚पार‚द‚हेमादेर‚ग्न्यादेरिव चेत‚सि ।&अभ्यास‚जाः प्र‚व‚र्त्त‚न्ते स्व‚र‚सेन कृपाद‚यः ॥ १२६ ॥\&[\smallbreak]


	
	    \end{quote}
	  
	  \endgroup
	

	  \pstart \leavevmode% starting standard par
	\hphantom{.}‚{\color{DodgerBlue3}‚काष्ठ‚पार‚देह‚मादेर‚ग्न्यादेरिव} । य‚थाग्निना हेम‚चार‚ण‚जार\edtext{}{\edlabel{pvv.54-5}\label{pvv.54-5}\lemma{जार}\Bfootnote{तुष‚दाह्ये प्र‚क्षिप्तानां त‚त्रैव क्ष‚य‚न‚य‚नं ।}}णादिना पुट‚पाका‚{\tiny $_{lb}$}‚दिना य‚थाक्र‚मं का\edtext{}{\edlabel{pvv.54-6}\label{pvv.54-6}\lemma{का}\Bfootnote{अग्निना ।}}ष्ठे पा\edtext{}{\edlabel{pvv.54-7}\label{pvv.54-7}\lemma{पा}\Bfootnote{हेमादिना ।}}र‚दे हेम्नी\edtext{}{\edlabel{pvv.54-8}\label{pvv.54-8}\lemma{हेम्नी}\Bfootnote{पुटादिना क‚ल्याण‚सुव‚र्ण्ण‚ता ।}} व द‚र‚दाहः । रुप्य‚र‚ञ्ज‚न‚साम‚र्थ्य‚व‚र्ण्णिका\edtext{}{\edlabel{pvv.54-9}\label{pvv.54-9}\lemma{र्ण्णिका}\Bfootnote{वानी (?) ।}}‚{\tiny $_{lb}$}‚\leavevmode\ledsidenote{\textenglish{055/s}} वृद्ध‚य आहिताः स्व‚र‚स‚वाहिन्यो न पुन‚र्य‚त्न‚सापेक्षाः । तेषु य‚दा पुन‚र्व‚ह्न्याद‚यो ‚{\tiny $_{lb}$}‚व्याप्रिय‚न्ते त‚दा स‚म‚धिक‚म‚ङ्गारा\edtext{}{\edlabel{pvv.55-1}\label{pvv.55-1}\lemma{ङ्गारा}\Bfootnote{याव‚द् भ‚स्म‚शात् स्यात् ।}} दिविशेष‚माद‚ध‚ति (।) ‚{\color{DodgerBlue3}‚त‚थाभ्या‚{\tiny $_{2}$}‚स‚जाः कृपाद‚यः} पुन‚र्य‚त्नान‚पेक्षित्वात् स्थिराश्र‚य‚त्वाच्च ‚{\color{DodgerBlue3}‚स्व‚र‚सेन प्र‚व‚र्त‚न्ते} । (१२६)
	\pend% ending standard par
      \label{div_pvv.1.127}
	  
	% new div opening: depth here is 2
	
	  \bigskip
	  \begingroup
	
	    \large
	  
	    \begin{quote}
	  
	    
	    \stanza[\smallbreak]
	\label{pv.1.127}\flagstanza{\tiny\textenglish{....1.127}}त‚स्मात् स तेषामुत्प‚न्नः स्व‚भावो जाय‚ते गुणः ।&त‚दुत्त‚रोत्त‚रो य‚त्नो विशेष‚स्य विधाय‚कः ॥ १२७ ॥\&[\smallbreak]


	
	    \end{quote}
	  
	  \endgroup
	

	  \pstart \leavevmode% starting standard par
	\hphantom{.}‚{\color{DodgerBlue3}‚त‚स्मात्} स्व‚र‚स‚वाहित्वात्स ‚{\color{DodgerBlue3}‚तेषा}‚म‚भ्यास‚व‚तां पुंसा‚{\color{DodgerBlue3}‚मुत्प‚न्नो} गुणः ‚{\color{DodgerBlue3}‚कृपादिः स्व‚भावो} जाय‚ते म‚न‚सः प्र‚कृतिर्भ‚व‚ति । ‚{\color{DodgerBlue3}‚त‚दुत्त‚रोत्त‚रो य‚त्नः} पूर्व्व‚पूर्व्वाभ्यासाद‚प‚राप‚रः प्र‚य‚त्नो ‚{\tiny $_{lb}$}‚विशेष‚कृद् भ‚व‚ति पूर्व्व‚प्र‚य‚त्न‚कृत‚स्य ‚{\color{DodgerBlue3}‚विशेष‚स्य} सुस्थित‚त्वात् । आधिक्याधान‚मेवा‚{\tiny $_{lb}$}‚प‚र‚य‚त्नात् । (१२७)
	\pend% ending standard par
      \label{div_pvv.1.128}
	  
	% new div opening: depth here is 2
	
	  \bigskip
	  \begingroup
	
	    \large
	  
	    \begin{quote}
	  
	    
	    \stanza[\smallbreak]
	\label{pv.1.128}\flagstanza{\tiny\textenglish{....1.128}}य‚स्माच्च तुल्य‚जातीय‚पूर्व‚बीज‚प्र‚वृद्ध‚यः ।&कृपादिबुद्ध‚य‚स्तासां स‚त्य‚भ्यासे कुतः स्थितिः ॥ १२८ ॥\&[\smallbreak]


	
	    \end{quote}
	  
	  \endgroup
	

	  \pstart \leavevmode% starting standard par
	\hphantom{.}‚{\color{DodgerBlue3}‚य‚स्माच्च} कार‚णा‚{\color{DodgerBlue3}‚त्तुल्य‚जातीया}‚त् ‚{\color{DodgerBlue3}‚पूर्व्व}‚स्मात् बीजाद्वास‚नाग‚{\tiny $_{3}$}‚र्भ‚स‚म‚न‚न्त‚र‚{\tiny $_{lb}$}‚प्र‚त्य‚यात् ‚{\color{DodgerBlue3}‚प्र‚वृद्धि}‚रुत्क‚र्षो यासान्तास्त‚था ‚{\color{DodgerBlue3}‚कृपादिबुद्ध‚यः} (।) ‚{\color{DodgerBlue3}‚तासां स‚त्य‚भ्यासे कुतः} स्थितिर्व्य‚व‚स्थितोत्क‚र्ष‚ता । (१२८)
	\pend% ending standard par
      \label{div_pvv.1.129}
	  
	% new div opening: depth here is 2
	
	  \bigskip
	  \begingroup
	
	    \large
	  
	    \begin{quote}
	  
	    
	    \stanza[\smallbreak]
	\label{pv.1.129}\flagstanza{\tiny\textenglish{....1.129}}न चैवं लंघ‚नादेव लंघ‚नं ब‚ल‚य‚त्न‚योः ।&त‚द्धेत्वोः स्थित‚श‚क्तित्वाल्लंघ‚न‚स्य स्थितात्म‚ता ॥ १२९ ॥\&[\smallbreak]


	
	    \end{quote}
	  
	  \endgroup
	

	  \pstart \leavevmode% starting standard par
	\hphantom{.}‚{\color{DodgerBlue3}‚न चैवं लंघ‚नादेव लंघ‚नं} य‚था कृपादिभ्य एव कृपाद‚यः । त‚था न लंघ‚नादेव ‚{\tiny $_{lb}$}‚लंघ‚न‚म‚पि तु ब‚ल‚य‚त्नाभ्यां ‚{\color{DodgerBlue3}‚ब‚ल‚य‚त्न‚योस्त‚द्धेत्वोः स्थित‚श‚क्तित्वात्} साम‚र्थ्य‚निय‚मात् ‚{\tiny $_{lb}$}‚लंघ‚न‚स्य ‚{\color{DodgerBlue3}‚स्थितात्म‚ता} व्य‚व‚स्थितोत्क‚र्ष‚ता भ‚व‚ति । (१२९)
	\pend% ending standard par
      \label{div_pvv.1.130}
	  
	% new div opening: depth here is 2
	

	  \pstart \leavevmode% starting standard par
	य‚दि ब‚ल‚य‚त्नाभ्यामेव लंघ‚नं न स्व‚भाव‚जातीयात्त‚दाभ्यासात्प्राग‚पि ताव‚त्प‚रि‚{\tiny $_{lb}$}‚मा‚{\tiny $_{4}$}‚णं स्यादित्याह (।)
	\pend% ending standard par
      
	  \bigskip
	  \begingroup
	
	    \large
	  
	    \begin{quote}
	  
	    
	    \stanza[\smallbreak]
	\label{pv.1.130}\flagstanza{\tiny\textenglish{....1.130}}त‚स्यादौ देह‚वैगुण्यात् प‚श्चाद्व‚द‚विलंघ‚न‚म् ।&श‚नैर्य‚त्नेन वैगुण्ये निर‚स्ते स्व‚ब‚ले स्थितिः ॥ १३० ॥\&[\smallbreak]


	
	    \end{quote}
	  
	  \endgroup
	

	  \pstart \leavevmode% starting standard par
	\hphantom{.}‚{\color{DodgerBlue3}‚त‚स्य} लंघ‚यितुरादाव‚भ्यासात् पूर्व्वं ‚{\color{DodgerBlue3}‚देह‚वैगुण्यात्} श्लेष्मादिकृत‚गौर‚वात् ‚{\color{DodgerBlue3}‚प‚श्चाद्व}‚{\tiny $_{lb}$}‚द‚भ्यासान‚न्त‚र‚मिवा‚{\color{DodgerBlue3}‚विलंघ‚नं श‚नैर्य‚त्नेन} व्यायामादिना ‚{\color{DodgerBlue3}‚वैगुण्ये निर‚स्ते स्व‚ब‚ले स्थितिः} श‚रीर‚स्य भ‚व‚ति । तेन पूर्व्व‚स्माल्लंघ‚नं विशिष्य‚ते ब‚लानुरूप‚स्थितिक‚ञ्च । (१३०)
	\pend% ending standard par
      \label{div_pvv.1.131}
	  
	% new div opening: depth here is 2
	

	  \begin{center}%% label @type='head'
	\textbf{(घ) क‚रुणा स्व‚बीज‚प्र‚भ‚वा}
	\end{center}
	

	  \pstart \leavevmode% starting standard par
	म‚नोगुणास्तु स‚त्य‚भ्यासे विप‚क्षान‚भ्यासे च प्र‚क‚र्षंनिष्टां ग‚च्छ‚न्ति । त‚दाहे (।)
	\pend% ending standard par
      \textsuperscript{\textenglish{056/s}}
	  \bigskip
	  \begingroup
	
	    \large
	  
	    \begin{quote}
	  
	    
	    \stanza[\smallbreak]
	\label{pv.1.131}\flagstanza{\tiny\textenglish{....1.131}}कृपा स्व‚बीज‚प्र‚भ‚वा स्व‚बीज‚प्र‚भ‚वैर्न चेत् ।&विप‚क्षैर्बाध्य‚ते चित्ते प्र‚यात्य‚त्य‚न्त‚सात्म‚ताम् ॥ १३१ ॥\&[\smallbreak]


	
	    \end{quote}
	  
	  \endgroup
	

	  \pstart \leavevmode% starting standard par
	\hphantom{.}‚{\color{DodgerBlue3}‚कृपा स्व‚बीज‚प्र‚भ‚वा} भूयोऽभ्य‚स्त‚स्व‚जा\edtext{}{\edlabel{pvv.56-1}\label{pvv.56-1}\lemma{जा}\Bfootnote{न स‚र्व्वः किन्तु य‚स्य बीजं कृपायाम‚स्ति ।}} तीय‚सँस्कार‚व‚त् स‚म‚न‚न्त‚र‚प्र‚त्य‚य‚{\tiny $_{lb}$}‚प्र‚सूतार्थ‚{\color{DodgerBlue3}‚बीज‚प्र‚भ‚वैर्न चेत् विप‚क्षै}‚र्द्वेषादिभि‚{\color{DodgerBlue3}‚र्ब्बाध्य‚ते} स्वोत्प‚त्त्या व्याह‚न्य‚ते ‚{\color{DodgerBlue3}‚चित्ते} चित्त‚{\tiny $_{lb}$}‚स‚न्ताने ‚{\color{DodgerBlue3}‚प्र‚यात्य‚त्य‚न्त‚सात्म‚तां} विप‚क्षासंकीर्ण्ण‚सात्म‚तां प्र‚कृतितां । (१३१)
	\pend% ending standard par
      \label{div_pvv.1.132}
	  
	% new div opening: depth here is 2
	
	  \bigskip
	  \begingroup
	
	    \large
	  
	    \begin{quote}
	  
	    
	    \stanza[\smallbreak]
	\label{pv.1.132}\flagstanza{\tiny\textenglish{....1.132}}त‚थापि मूल‚म‚भ्यासः पूर्वः पूर्वः प‚र‚स्य तु ।&कृपावैराग्य‚बोधादेश्चित्त‚ध‚र्म‚स्य पाट‚वे ॥ १३२ ॥\&[\smallbreak]


	
	    \end{quote}
	  
	  \endgroup
	

	  \pstart \leavevmode% starting standard par
	\hphantom{.}‚{\color{DodgerBlue3}‚त‚था हि मूलं} कार‚ण‚म‚भ्यासः ‚{\color{DodgerBlue3}‚पूर्व्वः पूर्व्वः प‚र‚स्यो}‚त्त‚र‚स्य ‚{\color{DodgerBlue3}‚कृपावैराग्य‚बो\edtext{}{\edlabel{pvv.56-2}\label{pvv.56-2}\lemma{बो}\Bfootnote{बुद्ध‚यादेः ।}} धादे‚{\tiny $_{lb}$}‚श्चित्त‚ध‚र्म‚स्य} म‚नोगुण‚स्य ‚{\color{DodgerBlue3}‚पाट‚वे} प्र\edtext{}{\edlabel{pvv.56-3}\label{pvv.56-3}\lemma{प्र}\Bfootnote{कार‚ण‚मिति स‚म्ब‚न्धः ।}}क‚र्षे न तूत्प‚त्तौ । त‚स्याः स‚म्भ‚वात् । (। १३२)
	\pend% ending standard par
      \label{div_pvv.1.133}
	  
	% new div opening: depth here is 2
	

	  \begin{center}%% label @type='head'
	\textbf{(ङ) अभ्य‚सात् क‚रुणात्म‚क‚त्व‚म्}
	\end{center}
	

	  \pstart \leavevmode% starting standard par
	त‚तः (।)
	\pend% ending standard par
      
	  \bigskip
	  \begingroup
	
	    \large
	  
	    \begin{quote}
	  
	    
	    \stanza[\smallbreak]
	\label{pv.1.133}\flagstanza{\tiny\textenglish{....1.133}}कृपात्म‚क‚त्व‚म‚भ्यासाद् घृणावैराग्य‚राग‚व‚त् ।&निष्प‚न्न‚क‚रुणोत्क‚र्ष‚प‚र‚दुःख‚क्ष‚मेस्तिः ॥ १३३ ॥\&[\smallbreak]


	
	    \end{quote}
	  
	  \endgroup
	

	  \pstart \leavevmode% starting standard par
	\hphantom{.}‚{\color{DodgerBlue3}‚कृपात्म‚क‚त्व‚म‚भ्यासाद्} भ‚व‚ति ‚{\color{DodgerBlue3}‚घृणाबैराग्य‚राग‚व‚त्} । य‚थाभ्यासात् घृणा ‚{\tiny $_{lb}$}‚क्व‚चिद्विष‚ये वैराग्यं राग‚श्च सात्मीभ‚व‚ति (।) त‚दे‚{\tiny $_{6}$}‚व‚म‚भ्यासात् ‚{\color{DodgerBlue3}‚कृपा} प्र‚{\color{DodgerBlue3}‚क‚र्ष}‚विशे\edtext{}{\edlabel{pvv.56-4}\label{pvv.56-4}\lemma{विशे}\Bfootnote{नैरात्म्य‚द‚र्श‚नं ।}}‚{\tiny $_{lb}$}‚ष‚व‚ती ‚{\color{DodgerBlue3}‚निष्प‚द्य‚त} इति स‚म‚र्थित‚र्मिय\edtext{}{\edlabel{pvv.56-5}\label{pvv.56-5}\lemma{र्मिय}\Bfootnote{साध‚नं क‚र‚णेभिप्र‚स्तुत्य कृपावैराग्य‚राग‚व‚दित्य‚न्तेन ॥}}ता च ज‚ग‚द्धितैषित्व‚ञ्च (।) व्याख्यातं ‚{\tiny $_{lb}$}‚कृपायाः प्रामाण्य‚साध‚न‚त्वं (। १३३)
	\pend% ending standard par
      \label{div_pvv.1.134}
	  
	% new div opening: depth here is 2
	

	  \begin{center}%% label @type='head'
	\textbf{(४) शास्तृत्वात् भ‚ग‚वान् प्र‚माण‚म्}
	\end{center}
	

	  \begin{center}%% label @type='head'
	\textbf{क. शास्तृत्व‚व्याख्यान‚म्}
	\end{center}
	

	  \pstart \leavevmode% starting standard par
	शास्तृत्व‚व्याख्यानाय (द‚यां) द‚र्श‚यितुमाह (।)
	\pend% ending standard par
      
	  \bigskip
	  \begingroup
	
	    \large
	  
	    \begin{quote}
	  
	    
	    \stanza[\smallbreak]
	\label{pv.1.134}\flagstanza{\tiny\textenglish{....1.134}}द‚यावान् दुःख‚हानार्थ‚मुपायेष्व‚भियुज्य‚ते ।&प‚रोक्षोपेय‚त‚द्धेतोस्त‚दाख्यानं हि दुष्क‚र‚म् ॥ १३४ ॥\&[\smallbreak]


	
	    \end{quote}
	  
	  \endgroup
	

	  \pstart \leavevmode% starting standard par
	\hphantom{.}‚{\color{DodgerBlue3}‚द‚यावान्} बोधिस‚त्त्वः प‚र‚दुःखं श‚म‚यितुकामः दुःख‚हानार्थ‚मात्म‚नः ‚{\color{DodgerBlue3}‚उपायेषु} दुःख‚श‚म‚नोपायेष्व‚{\color{DodgerBlue3}‚भियुज्य‚ते} । क‚स्मात्पुनः प‚र‚दुःख‚श‚म‚नोपायेषु युज्य‚त ‚{\tiny $_{lb}$}‚\leavevmode\ledsidenote{\textenglish{057/s}} इत्याह (।) ‚{\color{DodgerBlue3}‚प‚रोक्ष उपेयो} दुःख‚प्र‚श‚मः ‚{\color{DodgerBlue3}‚त‚द्धेतुश्च} मा\edtext{}{\edlabel{pvv.57-1}\label{pvv.57-1}\lemma{मा}\Bfootnote{अशुचिविज्ञ‚गुप्त्या (?) सुख‚दुःख‚योरुद्वेगानुद्वेगौ च घृणाद्य‚र्थः ।}}र्गो य‚स्य‚{\tiny $_{7}$}‚ त‚स्य ‚{\color{DodgerBlue3}‚त‚दाख्यां\edtext{}{\edlabel{pvv.57-2}\label{pvv.57-2}\lemma{दाख्यां}\Bfootnote{न ह्य‚मार्ग‚ज्ञोऽविप‚रीत‚मार्ग्गोप‚देशोऽधिक्रिय‚ते ।}} नं} य‚स्माद् ‚{\color{DodgerBlue3}‚दुष्क‚रं} (। १३४)
	\pend% ending standard par
      \label{div_pvv.1.135}
	  
	% new div opening: depth here is 2
	

	  \begin{center}%% label @type='head'
	\textbf{(ख. दुःख‚हेतुप‚रीक्ष‚ण‚म्)}
	\end{center}
	

	  \pstart \leavevmode% starting standard par
	त‚त्र साक्षात्क‚र‚णे प‚रीक्ष‚ण‚म‚ङ्गं त‚दाह (।)
	\pend% ending standard par
      
	  \bigskip
	  \begingroup
	
	    \large
	  
	    \begin{quote}
	  
	    
	    \stanza[\smallbreak]
	\label{pv.1.135}\flagstanza{\tiny\textenglish{....1.135}}युक्त्याग‚माभ्यां विमृश‚न् दुःख‚हेतुं प‚रीक्ष‚ते ।&त‚स्यानित्यादि रूपं च दुःख‚स्यैव विशेष‚णैः ॥ १३५ ॥\&[\smallbreak]


	
	    \end{quote}
	  
	  \endgroup
	

	  \pstart \leavevmode% starting standard par
	\hphantom{.}‚{\color{DodgerBlue3}‚यु\edtext{}{\edlabel{pvv.57-3}\label{pvv.57-3}\lemma{यु}\Bfootnote{एतेन युक्त्यायुक्तिशून्याग‚माग्र‚हः । त‚र्क‚मात्र‚त्याग आग‚मेन त‚त्र निग्र‚ह‚स्था‚{\tiny $_{lb}$}‚नानान्त‚त्व‚ज्ञानान्मोक्ष इति नैयायिकः । प्र‚कृतिपुरुषान्त‚ज्ञानादिति सांख्याः क‚र्म‚क्ष‚या‚{\tiny $_{lb}$}‚दिति दिग‚म्ब‚राः ।}} क्त्याग‚माभ्या}‚म‚नुमान‚प्र‚व‚च‚नाभ्यां प‚र‚स्प‚र‚म‚विरुद्धाभ्यां ‚{\color{DodgerBlue3}‚विमृश‚न्} विचार‚य‚न् दुःख‚स्य ज‚न्म‚नो ‚{\color{DodgerBlue3}‚हेतुं प‚रीक्ष‚ते} मुमुक्षुः ‚{\color{DodgerBlue3}‚त‚स्य} दुःख‚हेतोर‚{\color{DodgerBlue3}‚नित्यादिरूपं} आदिश‚ब्दान्निव‚र्त‚न‚योग्य‚तादिक‚ञ्च प‚रीक्ष‚ते । ‚{\color{DodgerBlue3}‚क‚थ‚मित्याह । दुःख‚स्यैव विशेष‚णैः । ‚{\tiny $_{lb}$}‚कादाचित्क‚त्वादिभिः} । (१३५)
	\pend% ending standard par
      \label{div_pvv.1.136_1.137}
	  
	% new div opening: depth here is 2
	

	  \pstart \leavevmode% starting standard par
	क‚स्मात् पुन‚र्दुःख‚स्य हेतोर‚नित्य‚त्वादि प‚रीक्ष‚णीय‚मित्याह ।
	\pend% ending standard par
      
	  \bigskip
	  \begingroup
	
	    \large
	  
	    \begin{quote}
	  
	    
	    \stanza[\smallbreak]
	\label{pv.1.136}\flagstanza{\tiny\textenglish{....1.136}}य‚त‚स्त‚था स्थिते हेतौ निवृत्तिर्नेति प‚श्य‚ति ।&फ‚ल‚स्य हेतोर्हानार्थं त‚द्विप‚क्षं प‚रीक्ष‚ते ॥ १३६ ॥\&[\smallbreak]


	
	    \end{quote}
	  
	  \endgroup
	

	  \pstart \leavevmode% starting standard par
	\hphantom{.}‚{\color{DodgerBlue3}‚य‚त‚स्त‚था स्थिते हेतौ नित्य‚त्वात्स‚दास्थिते हेतौ फ‚ल‚स्य दुःख‚स्य निवृत्तिर्नेति ‚{\tiny $_{lb}$}‚प‚श्य‚ति जानाति(।) त‚स्मात् दुःख‚स्य हेतोर्हानार्थ‚ञ्च त‚स्या विप‚क्षं प‚रीक्ष‚ते दुःख‚हेतु‚{\tiny $_{lb}$}‚बिरुद्धं, य‚स्याभ्यासाद् दुःख‚हेतुर‚पैति} ।
	\pend% ending standard par
      
	  \bigskip
	  \begingroup
	
	    \large
	  
	    \begin{quote}
	  
	    
	    \stanza[\smallbreak]
	\label{pv.1.137}\flagstanza{\tiny\textenglish{....1.137}}साध्य‚ते त‚द्विप‚क्षोपि हेतो रूपाव‚बोध‚तः ॥&आत्मात्मीय‚ग्र‚ह‚कृतः स्नेहः संस्कार‚गोच‚रः ॥ १३७ ॥\&[\smallbreak]


	
	    \end{quote}
	  
	  \endgroup
	

	  \pstart \leavevmode% starting standard par
	\hphantom{.}‚{\color{DodgerBlue3}‚साध्य‚ते} निश्चीय‚ते ‚{\color{DodgerBlue3}‚त‚द्विप‚क्षोपि हेतो रूपाव‚बोध‚तः} । ज्ञाते हि हे\edtext{}{\edlabel{pvv.57-4}\label{pvv.57-4}\lemma{हे}\Bfootnote{दुःख‚हेतौ नित्य‚शुच्याद्याकारे ।}} तौ त‚द्विरोधी ‚{\tiny $_{lb}$}‚बोद्धुं श‚क्यः । ‚{\color{DodgerBlue3}‚आत्मा\edtext{}{\edlabel{pvv.57-5}\label{pvv.57-5}\lemma{आत्मा}\Bfootnote{त‚च्छून्ये त‚द‚भिनिवेशः । अस्य चाविद्या स‚ह‚जा ।}}त्मीय‚ग्र‚हा}‚भ्यां ‚{\color{DodgerBlue3}‚कृतः स्नेहः सँस्कार‚गोच‚रः‚{\tiny $_{8}$}‚} (। १३७)
	\pend% ending standard par
      \label{div_pvv.1.138}
	  
	% new div opening: depth here is 2
	\textsuperscript{\textenglish{058/s}}

	  \begin{center}%% label @type='head'
	\textbf{(ग. नैरात्म्य‚द‚र्श‚न‚तो वास‚नाहानिः)}
	\end{center}
	
	  \bigskip
	  \begingroup
	
	    \large
	  
	    \begin{quote}
	  
	    
	    \stanza[\smallbreak]
	\label{pv.1.138}\flagstanza{\tiny\textenglish{....1.138}}हेतुविरोधि नैरात्म्य‚द‚र्श‚नं त‚स्य बाध‚क‚म् ॥&ब‚हुशो ब‚हुधोपायं कालेन ब‚हुनास्य च ॥ १३८ ॥\&[\smallbreak]


	
	    \end{quote}
	  
	  \endgroup
	\textsuperscript{\textenglish{12a/MA}}

	  \pstart \leavevmode% starting standard par
	अध्यात्म‚स्क\edtext{}{\edlabel{pvv.58-1}\label{pvv.58-1}\lemma{स्क}\Bfootnote{संस्काराख्येषु ।}}न्धेषु त‚दुप‚कार‚केषु च बाह्येषु स्नेहोस्य ‚{\color{DodgerBlue3}‚हेतु}‚र्दुःख‚स्या\edtext{}{\edlabel{pvv.58-2}\label{pvv.58-2}\lemma{स्या}\Bfootnote{कः पुन‚र‚सौ ज‚न्म‚ल‚क्ष‚ण‚स्य दुःख‚स्य हेतुस्त‚द्विरुद्धो वा ध‚र्म्म इत्याह (।) दुःखे ‚{\tiny $_{lb}$}‚विप‚र्य्य‚सेत्यादिनोक्तः स्नेहः कीदृश इत्य‚त्रात्मेत्यादिदुःख‚भूता आत्मीय‚र‚हिताः ‚{\tiny $_{lb}$}‚स्क‚न्धाद‚य एवास्य गोच‚रो विष‚यः ।}}त्मात्मीय‚{\tiny $_{lb}$}‚ग्र‚ह‚व‚तः स्निग्ध‚स्य तृष्ण‚या ज‚न्म‚प‚रिग्र‚हात् । त‚स्य ‚{\color{DodgerBlue3}‚नैरात्म्य‚द‚र्श‚नं विरोधि} विप‚रीता‚{\tiny $_{lb}$}‚ल‚म्ब‚नाकार‚त्वात् ‚{\color{DodgerBlue3}‚बाध‚कं} विप‚क्षः । एतं दुःख‚हेतुं त‚द्विप‚क्ष‚ञ्चाग‚मादुप‚श्रुत्यानुमाना‚{\tiny $_{lb}$}‚न्निश्चित्य ‚{\color{DodgerBlue3}‚ब‚हुशो} अनेक‚शो ‚{\color{DodgerBlue3}‚ब‚हुधोपाय‚म}‚नेक‚प्र‚कारं ‚{\color{DodgerBlue3}‚कालेन च ब‚हुनास्य} बोधिस‚त्त्व‚स्य ‚{\tiny $_{lb}$}‚(। १३८)
	\pend% ending standard par
      \label{div_pvv.1.139}
	  
	% new div opening: depth here is 2
	
	  \bigskip
	  \begingroup
	
	    \large
	  
	    \begin{quote}
	  
	    
	    \stanza[\smallbreak]
	\label{pv.1.139}\flagstanza{\tiny\textenglish{....1.139}}ग‚च्छ‚न्त्य‚भ्य‚स्य‚त‚स्त‚त्र गुण‚दोषाः प्र‚काश‚ताम् ।&बुद्धेश्च पाट‚वाद्धेतोर्वास‚नातः प्र‚हीय‚ते ॥ १३९ ॥\&[\smallbreak]


	
	    \end{quote}
	  
	  \endgroup
	

	  \pstart \leavevmode% starting standard par
	\hphantom{.}‚{\color{DodgerBlue3}‚अभ्य‚स्य‚तो} भाव‚य‚तः ‚{\color{DodgerBlue3}‚त‚त्र} दुःख‚हेतौ त‚द्विप‚क्षे च ‚{\color{DodgerBlue3}‚गुण‚दोषा} य‚थायोगं‚{\tiny $_{1}$}‚ ‚{\color{DodgerBlue3}‚प्र‚काश‚तां ‚{\tiny $_{lb}$}‚ग‚च्छ‚न्ति} । अभ्यासाधीनो हि भाव्य‚मान‚बुद्ध्याकार‚विश‚दीभावः ।
	\pend% ending standard par
      

	  \pstart \leavevmode% starting standard par
	\hphantom{.}‚{\color{DodgerBlue3}‚अतो}‚ऽभ्यासाद् ‚{\color{DodgerBlue3}‚बुद्धेश्च पाट‚वाद्धेतो}‚रात्म‚ग्र‚ह‚स्य तृष्णायाश्च ‚{\color{DodgerBlue3}‚वास‚ना} । ‚{\tiny $_{lb}$}‚काय‚वाग्बुद्धिवैगुण्य‚हेतुतः श‚क्तिलेशः ‚{\color{DodgerBlue3}‚प्र‚ही\edtext{}{\edlabel{pvv.58-3}\label{pvv.58-3}\lemma{ही}\Bfootnote{एक‚मात्मान‚न्द‚म‚यिष्यामीति नैषामात्म‚ग्र‚होस्ति ।}} य‚ते} निःशेष‚म‚पैति । (१३९)
	\pend% ending standard par
      \label{div_pvv.1.140}
	  
	% new div opening: depth here is 2
	

	  \begin{center}%% label @type='head'
	\textbf{(घ. प्र‚त्येक‚बुद्धादिभ्यो बुद्धे विशेषः}
	\end{center}
	
	  \bigskip
	  \begingroup
	
	    \large
	  
	    \begin{quote}
	  
	    
	    \stanza[\smallbreak]
	\label{pv.1.140a}\flagstanza{\tiny\textenglish{...1.140a}}प‚रार्थ‚वृत्तेः ख‚ङ्गादेर्विशेषोऽयं म‚हामुनेः ।\&[\smallbreak]


	
	    \end{quote}
	  
	  \endgroup
	

	  \pstart \leavevmode% starting standard par
	\hphantom{.}‚{\color{DodgerBlue3}‚अय‚मेव} वास‚नाहानिल‚क्ष‚णः ख‚ङ्गः प्र‚त्येक‚बुद्ध ‚{\color{DodgerBlue3}‚आदि}‚र्य‚स्य श्रा\edtext{}{\edlabel{pvv.58-4}\label{pvv.58-4}\lemma{श्रा}\Bfootnote{वास‚नास्ति ।}}व‚क‚स्य त‚स्मात् ‚{\tiny $_{lb}$}‚स‚काशात् ‚{\color{DodgerBlue3}‚म‚हामु\edtext{}{\edlabel{pvv.58-5}\label{pvv.58-5}\lemma{हामु}\Bfootnote{क्लेश‚विसंयोगे}}नेः} स‚म्य‚क्स‚म्बुद्ध‚स्य ‚{\color{DodgerBlue3}‚विशेषः}\edtext{}{\edlabel{pvv.58-6}\label{pvv.58-6}\lemma{स्य}\Bfootnote{स च प‚रार्थ‚वृत्तित्वात् ।}} स्वार्थ‚स‚म्प‚त्तेः ।
	\pend% ending standard par
      

	  \pstart \leavevmode% starting standard par
	न‚न्वेव‚मुपायाभ्यासो द‚र्शितो न शास्तृत्वं त‚{\tiny $_{2}$}‚स्योप‚देष्ट्ठ‚त्वादित्याह ।
	\pend% ending standard par
      
	  \bigskip
	  \begingroup
	
	    \large
	  
	    \begin{quote}
	  
	    
	    \stanza[\smallbreak]
	\label{pv.1.140b}\flagstanza{\tiny\textenglish{...1.140b}}उपायाभ्यास एवायं ताद‚र्थ्याच्छास‚नं म‚त‚म् ॥ १४० ॥\&[\smallbreak]


	
	    \end{quote}
	  
	  \endgroup
	

	  \pstart \leavevmode% starting standard par
	\hphantom{.}‚{\color{DodgerBlue3}‚उपायाभ्यास एवायं शास‚नं म‚तं ताद‚र्थ्यात्} शास‚नार्थ‚त्वात् । कार‚णे कार्योप‚{\tiny $_{lb}$}‚चारात् । (१४०)
	\pend% ending standard par
      \label{div_pvv.1.141}
	  
	% new div opening: depth here is 2
	\textsuperscript{\textenglish{059/s}}
	  \bigskip
	  \begingroup
	
	    \large
	  
	    \begin{quote}
	  
	    
	    \stanza[\smallbreak]
	\label{pv.1.141a}\flagstanza{\tiny\textenglish{...1.141a}}निष्प‚त्तेः प्र‚थ‚मं भावाद्धेतुरुक्त‚मिदं द्व‚य‚म् ॥\&[\smallbreak]


	
	    \end{quote}
	  
	  \endgroup
	

	  \pstart \leavevmode% starting standard par
	\hphantom{.}सुग‚त‚त्त्व‚स्य फ‚ल‚स्य ‚{\color{DodgerBlue3}‚निष्प‚त्तेः प्र‚थ‚मं भावा}‚त्ताव‚{\color{DodgerBlue3}‚देत‚त् द्व‚यं} हि हितैषित्वं शास्तृत्वं ‚{\tiny $_{lb}$}‚‚{\color{DodgerBlue3}‚हेतुरुक्तं} हेत्व‚व‚स्थाया अभिधानात् ॥
	\pend% ending standard par
      

	  \begin{center}%% label @type='head'
	\textbf{(५) सुग‚त‚त्वात् भ‚ग‚वान् प्र‚माण‚म्}
	\end{center}
	

	  \pstart \leavevmode% starting standard par
	सुग‚त‚त्वं व्याचिख्यासुराह (।)
	\pend% ending standard par
      
	  \bigskip
	  \begingroup
	
	    \large
	  
	    \begin{quote}
	  
	    
	    \stanza[\smallbreak]
	\label{pv.1.141b}\flagstanza{\tiny\textenglish{...1.141b}}हेतोः प्र‚हाणं त्रिगुणं सुग‚त‚त्व‚म‚निश्र‚यात् ॥ १४१ ॥\&[\smallbreak]


	
	    \end{quote}
	  
	  \endgroup
	

	  \pstart \leavevmode% starting standard par
	\hphantom{.}‚{\color{DodgerBlue3}‚हेतोः} स‚मुदाय‚स्य ‚{\color{DodgerBlue3}‚प्र‚हाणं} निरोधः ‚{\color{DodgerBlue3}‚सुग‚त‚त्वं} त‚च्च ‚{\color{DodgerBlue3}‚त्रिगुणं} गुण‚त्र‚य‚युक्तं (।) ‚{\tiny $_{lb}$}‚सुश‚ब्द‚स्य त्रिविधोऽर्थः प्र‚श‚स्त‚ता सुरूप‚व‚त् । अपुन‚रावृत्तिः अन‚ष्ट‚{\tiny $_{8}$}‚ज्व‚र‚व‚त् । ‚{\tiny $_{lb}$}‚निःशेष‚ता च अपूर्ण्ण‚घ‚ट‚व‚त् ।
	\pend% ending standard par
      

	  \pstart \leavevmode% starting standard par
	\hphantom{.}त‚त्र प्र‚श‚स्तं भ‚ग‚वान् ज्ञात‚वान् सुग‚त इति प्र‚श‚स्तार्थ‚माह । ‚{\color{DodgerBlue3}‚अनिःश्र‚याद}‚{\tiny $_{lb}$}‚नाश्र‚य\edtext{}{\edlabel{pvv.59-1}\label{pvv.59-1}\lemma{य}\Bfootnote{लोके हि सुखं त‚द‚नुब‚न्धि च प्र‚श‚स्तं त‚द्विप‚रीतं सास्र‚वं ।}}णाद् । (१४१)
	\pend% ending standard par
      \label{div_pvv.1.142}
	  
	% new div opening: depth here is 2
	
	  \bigskip
	  \begingroup
	
	    \large
	  
	    \begin{quote}
	  
	    
	    \stanza[\smallbreak]
	\label{pv.1.142a}\flagstanza{\tiny\textenglish{...1.142a}}दुःख‚स्य श‚स्तं नैरात्म्य‚दृष्टेश्च युक्तितोऽपि वा ।\&[\smallbreak]


	
	    \end{quote}
	  
	  \endgroup
	

	  \pstart \leavevmode% starting standard par
	\hphantom{.}‚{\color{DodgerBlue3}‚दुःख‚स्य} श‚स्तं सुग‚त‚त्वं । त‚त् पुन‚र्दुःखानाश्र‚य‚णं ‚{\color{DodgerBlue3}‚नैरात्म्य‚दृष्टेः} । आत्म‚द‚र्शी ‚{\tiny $_{lb}$}‚ह्यात्म‚नि स्निह्य‚न् त‚द्दुःख‚सुख‚प‚रिहार‚प्राप्तीच्छ‚या ज‚न्म दुःख‚रूप‚माद‚त्ते । प्र‚ही‚{\tiny $_{lb}$}‚णात्म‚द‚र्श‚न‚स्तु नैतादृश इति प्र‚हीण‚दुःखोपायः ‚{\color{DodgerBlue3}‚युक्तितोपि वा} युक्तिप‚रिदृष्टेनोपा\edtext{}{\edlabel{pvv.59-2}\label{pvv.59-2}\lemma{रिदृष्टेनोपा}\Bfootnote{अन‚न्त‚रोक्तेन ।}}‚{\tiny $_{lb}$}‚येन वा ग‚म‚{\tiny $_{4}$}‚नात्त‚त्सुग‚त‚त्त्वं प्र‚श‚स्तं ।
	\pend% ending standard par
      

	  \pstart \leavevmode% starting standard par
	अथ‚वाऽपुन‚रावृत्त्या ग‚म‚नं सुग‚त‚त्वं (।) त‚दाख्यातुमाह ।
	\pend% ending standard par
      
	  \bigskip
	  \begingroup
	
	    \large
	  
	    \begin{quote}
	  
	    
	    \stanza[\smallbreak]
	\label{pv.1.142b}\flagstanza{\tiny\textenglish{...1.142b}}पुन‚रावृत्तिरित्युक्तौ ज‚न्म‚दोष‚स‚मुद्भ‚वौ ॥ १४२ ॥\&[\smallbreak]


	
	    \end{quote}
	  
	  \endgroup
	

	  \pstart \leavevmode% starting standard par
	\hphantom{.}‚{\color{DodgerBlue3}‚ज‚न्म‚नो} रागादेश्च ‚{\color{DodgerBlue3}‚दोष‚स्य स‚मुद्भ‚वौ} पुनः पुन‚राव‚र्त‚नात् ‚{\color{DodgerBlue3}‚पुन‚रावृत्तिरित्युक्तौ} (१४२)
	\pend% ending standard par
      \label{div_pvv.1.143_1.144_1.145}
	  
	% new div opening: depth here is 2
	

	  \begin{center}%% label @type='head'
	\textbf{(क. आत्म‚द‚र्श‚न‚बीज‚हानात् मुक्तिः)}
	\end{center}
	
	  \bigskip
	  \begingroup
	
	    \large
	  
	    \begin{quote}
	  
	    
	    \stanza[\smallbreak]
	\label{pv.1.143a}\flagstanza{\tiny\textenglish{...1.143a}}आत्म‚द‚र्श‚न‚बीज‚स्य हानाद‚पुन‚राग‚मः ।&त‚द्भूत‚भिन्नात्म‚त‚या;\&[\smallbreak]


	
	    \end{quote}
	  
	  \endgroup
	

	  \pstart \leavevmode% starting standard par
	\hphantom{.}नैरात्म्य‚भाव‚नासात्म्ये तु । ‚{\color{DodgerBlue3}‚आत्म‚द‚र्श‚न‚स्य} ज‚न्म‚प्र‚ब‚न्ध‚{\color{DodgerBlue3}‚बीज‚स्य हानाद‚पुन‚रा‚{\tiny $_{lb}$}‚ग‚मो}‚ऽपुन‚रावृत्तिः । त‚त्पुन‚रात्म‚द‚र्श‚न‚बीज‚स्य हानं ‚{\color{DodgerBlue3}‚भूता}\edtext{}{\edlabel{pvv.59-3}\label{pvv.59-3}\lemma{हानं}\Bfootnote{भूत‚त्वेनात्म‚द‚र्श‚न‚स‚मारोपान्याकार‚ताभिज्ञ‚त्वेन त‚द्विरुद्ध‚ता । आत्म‚त्वेन ‚{\tiny $_{lb}$}‚स्व‚भाव‚ता क‚थिता (।) भूतो भिन्न आत्मा य‚स्य त‚द्भाव‚स्त‚या ।}} त्स‚त्यात् नैरात्म्याद् ‚{\color{DodgerBlue3}‚भिन्ना} \leavevmode\ledsidenote{\textenglish{060/s}} ‚{\color{DodgerBlue3}‚त्म‚त‚या}‚ऽन्य‚त्वात् । न हि सात्मीभूत‚प्र‚तिप‚क्ष‚स्य विप‚क्ष‚बी‚{\tiny $_{5}$}‚ज‚स‚म्भ‚वः । ‚{\color{DodgerBlue3}‚निःशेष}‚म्वाग‚म‚{\tiny $_{lb}$}‚नात् सुग‚त‚त्वं (।)
	\pend% ending standard par
      

	  \pstart \leavevmode% starting standard par
	त‚दाह (।)
	\pend% ending standard par
      
	  \bigskip
	  \begingroup
	
	    \large
	  
	    \begin{quote}
	  
	    
	    \stanza[\smallbreak]
	\label{pv.1.143b}\flagstanza{\tiny\textenglish{...1.143b}}शेष‚म‚क्लेश‚निर्ज्व‚र‚म् ॥ १४३ ॥\&[\smallbreak]


	
	    \end{quote}
	  
	  \endgroup
	
	  \bigskip
	  \begingroup
	
	    \large
	  
	    \begin{quote}
	  
	    
	    \stanza[\smallbreak]
	\label{pv.1.144a}\flagstanza{\tiny\textenglish{...1.144a}}काय‚वाग्बुद्धिवैगुण्यं मार्गोंक्त्य‚प‚टुताऽपि वा ।&अशेष‚हान‚म‚भ्यासाद्;\&[\smallbreak]


	
	    \end{quote}
	  
	  \endgroup
	

	  \pstart \leavevmode% starting standard par
	\hphantom{.}‚{\color{DodgerBlue3}‚काय}‚वैगुण्य‚म‚चाप‚लेप्युत्प्लुत्य ग‚म‚नादि । वाग्वैगुण्यं मानाभावेपि वृष‚ली‚{\tiny $_{lb}$}‚वादादि । ‚{\color{DodgerBlue3}‚बुद्धिवैगुण्यं} नित्या‚{\color{DodgerBlue3}‚स‚माधाना}‚द‚व्याकृत‚चित्ताव‚स्थानं एत‚त् त्र‚यं शेषं स‚क‚ल‚{\tiny $_{lb}$}‚क्लेशोप‚क्लेश‚प्र‚श‚माद‚क्लेशं । निर्ज्व‚र‚ञ्चादोष‚मूल‚त्वात् । ‚{\color{DodgerBlue3}‚मार्ग्ग}‚स्य क्ष‚णिक‚नैरात्म्य‚{\tiny $_{lb}$}‚भाव‚नादे‚{\color{DodgerBlue3}‚रुक्ताव‚प‚टुतापि वा} शेषं । त‚त्प‚रित्यागाद‚{\color{DodgerBlue3}‚शेष‚हान‚म‚भ्यासा}‚दिति निःशेष‚ग‚म‚{\tiny $_{6}$}‚‚{\tiny $_{lb}$}‚नात्सुग‚त‚त्वं ।
	\pend% ending standard par
      

	  \pstart \leavevmode% starting standard par
	द‚र्शितं त्रिगुणं सुग‚त‚त्वं ॥ ॰ ॥
	\pend% ending standard par
      

	  \begin{center}%% label @type='head'
	\textbf{(ख. बुद्ध‚त्व‚बाध‚क‚युक्तिनिरासः}
	\end{center}
	

	  \pstart \leavevmode% starting standard par
	त‚देवं स‚र्व्व‚ज्ञ‚स्य स‚म्भ‚वानुमानं प्र‚तिपाद्य त‚द्बाध‚कं दूष‚यितुमाह (।)
	\pend% ending standard par
      
	  \bigskip
	  \begingroup
	
	    \large
	  
	    \begin{quote}
	  
	    
	    \stanza[\smallbreak]
	\label{pv.1.144b}\flagstanza{\tiny\textenglish{...1.144b}}उक्त्यादेर्दोष‚संक्ष‚यः ॥ १४४ ॥\&[\smallbreak]


	
	    \end{quote}
	  
	  \endgroup
	
	  \bigskip
	  \begingroup
	
	    \large
	  
	    \begin{quote}
	  
	    
	    \stanza[\smallbreak]
	\label{pv.1.145}\flagstanza{\tiny\textenglish{....1.145}}नेत्येके व्य‚तिरेकोस्य संदिग्धो व्य‚भित‚चार्य‚तः ।&अक्ष‚यित्वं च दोषाणां नित्य‚त्वाद‚नुपाय‚तः ॥ १४५ ॥\&[\smallbreak]


	
	    \end{quote}
	  
	  \endgroup
	

	  \pstart \leavevmode% starting standard par
	\hphantom{.}‚{\color{DodgerBlue3}‚एके} जै मि नी या उक्त्यादेर्हेतो र‚थ्यापुरुष‚व‚त् रागादिदो\edtext{}{\edlabel{pvv.60-1}\label{pvv.60-1}\lemma{रागादिदो}\Bfootnote{रागादिमान् विव‚क्षितः पुरुषो व‚क्तृत्वात् ।}}ष‚संक्ष‚यः क‚स्य‚चिन्ना‚{\tiny $_{lb}$}‚स्तीत्याहुः । ‚{\color{DodgerBlue3}‚व्य‚ति\edtext{}{\edlabel{pvv.60-2}\label{pvv.60-2}\lemma{ति}\Bfootnote{अस‚ति रागादिम‚त्वे न भ‚व‚ति व‚क्तृत्व‚मिति ।}}रेको} विप‚क्षाद् व्यावृत्तिः अस्य व‚क्तृत्वादिहेतोः ‚{\color{DodgerBlue3}‚संदिग्धोऽतो ‚{\tiny $_{lb}$}‚व्य‚भिचार्य}‚नैकान्तिकोय‚मिति विस्त‚र‚तो विप‚ञ्च‚यिष्य‚ते ।
	\pend% ending standard par
      

	  \pstart \leavevmode% starting standard par
	\hphantom{.}किञ्च (।) ‚{\color{DodgerBlue3}‚अक्ष‚यित्वं दोषाणां} यो म‚न्य‚ते स ‚{\color{DodgerBlue3}‚नित्य‚त्वाद्वा}‚ऽकाश‚व‚त्‚{\tiny $_{7}$}‚ ‚{\color{DodgerBlue3}‚अनुपाय‚त} उपायाभावात् वा । न‚ष्ट‚स्य पुन‚रुन्म‚ज्ज‚न‚व‚त् । (१४४) (१४५)
	\pend% ending standard par
      \label{div_pvv.1.146}
	  
	% new div opening: depth here is 2
	

	  \pstart \leavevmode% starting standard par
	एषां त्र‚याणाम‚पि हेतूनाम‚सिद्ध‚त्वं द‚र्श‚य‚ति (।)
	\pend% ending standard par
      
	  \bigskip
	  \begingroup
	
	    \large
	  
	    \begin{quote}
	  
	    
	    \stanza[\smallbreak]
	\label{pv.1.146}\flagstanza{\tiny\textenglish{....1.146}}उपाय‚स्याप‚रिज्ञानाद‚पि वा प‚रिक‚ल्प‚येत् ।&हेतुम‚त्त्वात् विरुद्ध‚स्य हेतोर‚भ्यास‚तः क्ष‚यात्त् ॥ १४६ ॥\&[\smallbreak]


	
	    \end{quote}
	  
	  \endgroup
	\textsuperscript{\textenglish{061/s}}

	  \pstart \leavevmode% starting standard par
	\hphantom{.}‚{\color{DodgerBlue3}‚उपाय‚स्याप‚रिज्ञानाद् वा} मूर्ख‚स्य श‚ब्द‚ज्ञान‚व‚त् । ‚{\color{DodgerBlue3}‚प‚रि\edtext{}{\edlabel{pvv.61-1}\label{pvv.61-1}\lemma{रि}\Bfootnote{क‚श्चित्स‚वेता (?) ।}} क‚ल्प‚येत्} ।
	\pend% ending standard par
      

	  \pstart \leavevmode% starting standard par
	\hphantom{.}‚{\color{DodgerBlue3}‚हेतुम‚त्त्वाद् विरुद्ध‚स्य} हेतुम‚त्त्वाद् दोषाणाम‚नित्य‚त्वं त‚तोऽसिद्धं नित्य‚त्वं दोषाणां । ‚{\tiny $_{lb}$}‚हेतोरात्म‚द‚र्श‚न‚स्य विरुद्ध‚स्य नैरात्म्य‚स्या‚{\color{DodgerBlue3}‚भ्यास‚तः क्ष‚यादुपायाभावोप्य‚सिद्धः} । (१४६)
	\pend% ending standard par
      \label{div_pvv.1.147}
	  
	% new div opening: depth here is 2
	
	  \bigskip
	  \begingroup
	
	    \large
	  
	    \begin{quote}
	  
	    
	    \stanza[\smallbreak]
	\label{pv.1.147a}\flagstanza{\tiny\textenglish{...1.147a}}हेतुस्व‚भाव‚ज्ञानेन त‚ज्ज्ञान‚म‚पि साध्य‚ते ॥ ॰ ॥\&[\smallbreak]


	
	    \end{quote}
	  
	  \endgroup
	

	  \pstart \leavevmode% starting standard par
	\hphantom{.}‚{\color{DodgerBlue3}‚हेतो}‚रात्म‚द‚र्श‚न‚स्य ‚{\color{DodgerBlue3}‚स्व‚भाव‚ज्ञानेन त‚त्-ज्ञान‚म‚पि} त‚न्निवृत्त्युपाय‚स्य त‚{\tiny $_{8}$}‚द्वि-\leavevmode\ledsidenote{\textenglish{12b/MA}} ‚{\tiny $_{lb}$}‚प‚र्य‚य‚रूप‚स्य ज्ञान‚ञ्च ‚{\color{DodgerBlue3}‚साध्य‚ते} । य‚था दाने ज्ञाते त‚द्विप‚र्य‚य‚रूप‚त्वान्मात्स‚र्य‚स्य त‚द्वि‚{\tiny $_{lb}$}‚प‚क्ष‚ताऽव‚सीय‚त इति तृतीयोप्य‚सिद्धः । प्र‚त्युक्तं स‚र्व्व‚ज्ञ‚स्य बाध‚नं ॥
	\pend% ending standard par
      

	  \begin{center}%% label @type='head'
	\textbf{(६) तायित्वाद् भ‚ग‚वान् प्र‚माण‚म्}
	\end{center}
	

	  \pstart \leavevmode% starting standard par
	तायित्वं व्याख्यातुमाह ।
	\pend% ending standard par
      
	  \bigskip
	  \begingroup
	
	    \large
	  
	    \begin{quote}
	  
	    
	    \stanza[\smallbreak]
	\label{pv.1.147b}\flagstanza{\tiny\textenglish{...1.147b}}तायः स्व‚दृष्ट‚मार्गोक्तिः, वैफ‚ल्याद् व‚क्ति नानृत‚म् ॥ १४७ ॥\&[\smallbreak]


	
	    \end{quote}
	  
	  \endgroup
	

	  \pstart \leavevmode% starting standard par
	\hphantom{.}दुःख‚हेतुनिव‚र्त‚क‚त्वेन ‚{\color{DodgerBlue3}‚स्व‚यं दृष्ट‚स्य \edtext{}{\edlabel{pvv.61-2}\label{pvv.61-2}\lemma{स्य}\Bfootnote{अनुलोम‚तः पूर्व्व‚पूर्व्वाज्ज‚ग‚द्धितैषित्वादेरुत्त‚रोत्त‚र‚स्य स‚म्भाव‚नानुमानेनात्य‚{\tiny $_{lb}$}‚न्ताभाव‚निरासः । नाव‚श्यं कार‚णं कार्य‚व‚दिति न निय‚म‚हेतुः ।}}मार्ग्ग}‚स्यो‚{\color{DodgerBlue3}‚क्ते}‚र्देश‚ना ‚{\color{DodgerBlue3}‚तायः} । कार‚णे\edtext{}{\edlabel{pvv.61-3}\label{pvv.61-3}\lemma{णे}\Bfootnote{आस्र‚व‚क्ष‚याऽनुशास‚नीप्रातिहार्य‚नाम‚के कार‚णे ।}} कार्योप‚{\tiny $_{lb}$}‚चारात् । त‚या हि स‚त्त्वान् ताय‚ते त‚द्योगात् तायित्वं (।) स च ‚{\color{DodgerBlue3}‚वैफ‚ल्याद्व‚क्ति ‚{\tiny $_{lb}$}‚नानृतं} । आत्म‚सुखाद्य‚भिलाषादिना क‚श्चिद‚स‚त्यं व‚द‚ति अ‚{\tiny $_{1}$}‚ज्ञानाद्वा । प्र‚हीणात्म‚{\tiny $_{lb}$}‚द‚र्श‚न‚स्य साक्षात्कृत‚त‚त्त्व‚स्य त‚दुभ‚यं नास्ति । (१४७)
	\pend% ending standard par
      \label{div_pvv.1.148}
	  
	% new div opening: depth here is 2
	

	  \begin{center}%% label @type='head'
	\textbf{(क‚रुणाहेतुकं स‚त्याभिधान‚म्)}
	\end{center}
	

	  \pstart \leavevmode% starting standard par
	विशेष‚तः स‚त्याभिधान‚हेतुरेव कृपाऽस्तीत्याह (।)
	\pend% ending standard par
      
	  \bigskip
	  \begingroup
	
	    \large
	  
	    \begin{quote}
	  
	    
	    \stanza[\smallbreak]
	\label{pv.1.148a}\flagstanza{\tiny\textenglish{...1.148a}}द‚यालुत्वात् प‚रार्थ‚ञ्च स‚र्वार‚म्भाभियोग‚तः ।&त‚स्मात् प्र‚माणं;\&[\smallbreak]


	
	    \end{quote}
	  
	  \endgroup
	

	  \pstart \leavevmode% starting standard par
	\hphantom{.}‚{\color{DodgerBlue3}‚द‚यालुत्वा}‚च्च ‚{\color{DodgerBlue3}‚प‚रार्थ‚ञ्च स‚र्व्व‚स्य} मार्ग्गाभ्यासादेरा‚{\color{DodgerBlue3}‚र‚म्भेऽभियोग‚तः} प‚रार्थ‚मेवो‚{\tiny $_{lb}$}‚द्दिश्य भ‚ग‚वान‚भिसंबुद्धः क‚थ‚न्त‚स्य मिथ्याभिधानेन स‚त्त्व‚व‚ञ्च‚नाश‚ङ्कापि । ‚{\color{DodgerBlue3}‚त‚स्मा}‚{\tiny $_{lb}$}‚त्तायित्वात् ‚{\color{DodgerBlue3}‚प्र‚माणं} भ‚ग‚वान् । य‚थादृष्टार्थ‚प्र‚व‚क्तृत्वं हि स‚म्वादित्व‚मेवेति प्र‚थ‚म‚{\tiny $_{lb}$}‚प्र‚माण‚ल‚क्ष‚ण‚योगात् प्रामा‚{\tiny $_{2}$}‚ण्य‚म‚नेनोक्तं ।
	\pend% ending standard par
      \textsuperscript{\textenglish{062/s}}

	  \begin{center}%% label @type='head'
	\textbf{क. तायः च‚तुःस‚त्त्य‚प्र‚काश‚न‚म्}
	\end{center}
	

	  \pstart \leavevmode% starting standard par
	द्वितीय‚ल‚क्ष‚ण‚योग‚म‚प्याह\edtext{}{\edlabel{pvv.62-1}\label{pvv.62-1}\lemma{प्याह}\Bfootnote{अज्ञातार्थ‚प्र‚काशो वा द्वितीय‚ल‚क्ष‚णं ।}} (।)
	\pend% ending standard par
      
	  \bigskip
	  \begingroup
	
	    \large
	  
	    \begin{quote}
	  
	    
	    \stanza[\smallbreak]
	\label{pv.1.148b}\flagstanza{\tiny\textenglish{...1.148b}}तायो वा च‚तुःस‚त्य‚प्र‚काश‚न‚म् ॥ १४८ ॥\&[\smallbreak]


	
	    \end{quote}
	  
	  \endgroup
	

	  \pstart \leavevmode% starting standard par
	\hphantom{.}‚{\color{DodgerBlue3}‚तायो वा च‚तुःस‚त्य‚प्र‚काश‚नं} । प‚रै\edtext{}{\edlabel{pvv.62-2}\label{pvv.62-2}\lemma{रै}\Bfootnote{अनुलोम‚तो व्याख्यायेदानीं प्र‚तिलोम‚तः कार्य‚प्र‚तिप‚त्त्या कार‚ण‚सिद्धिद‚र्श‚नेन ‚{\tiny $_{lb}$}‚प्रामाण्य‚माह ।}}र‚ज्ञात‚स्य स‚त्य‚च‚तुष्ट‚य‚स्य प्र‚काश‚न‚म्वा ‚{\tiny $_{lb}$}‚तायः । त‚द्योगात् तायी प्र‚माणं भ‚ग‚वानुक्तः । (१४८)
	\pend% ending standard par
      \label{div_pvv.1.149}
	  
	% new div opening: depth here is 2
	

	  \begin{center}%% label @type='head'
	\textbf{ख. च‚त्वारि आर्य-स‚त्यानि}
	\end{center}
	

	  \begin{center}%% label @type='head'
	\textbf{(क) दुःख‚स‚त्त्य‚म्}
	\end{center}
	

	  \begin{center}%% label @type='head'
	\textbf{I. संसारिणः स्क‚न्धा दुःख‚म्}
	\end{center}
	

	  \pstart \leavevmode% starting standard par
	आर्य‚स‚त्येषु दुःख‚माह ।
	\pend% ending standard par
      
	  \bigskip
	  \begingroup
	
	    \large
	  
	    \begin{quote}
	  
	    
	    \stanza[\smallbreak]
	\label{pv.1.149a}\flagstanza{\tiny\textenglish{...1.149a}}दुःखं संसारिणः स्क‚न्धाः;\&[\smallbreak]


	
	    \end{quote}
	  
	  \endgroup
	

	  \pstart \leavevmode% starting standard par
	\hphantom{.}रूप‚वेद‚नासंज्ञासंस्कार‚विज्ञानाख्याः प‚ञ्च ‚{\color{DodgerBlue3}‚स्क‚न्धाः} । ज‚न्म‚म‚र‚ण‚प्र‚ब‚न्धः ‚{\tiny $_{lb}$}‚‚{\color{DodgerBlue3}‚संसारः} । त‚द्व\edtext{}{\edlabel{pvv.62-3}\label{pvv.62-3}\lemma{द्व}\Bfootnote{संसारिणः स्क‚न्धाः ।}}न्तो ‚{\color{DodgerBlue3}‚दुःखं} तिसृभिर्दुः ख‚ताभिः ।
	\pend% ending standard par
      

	  \pstart \leavevmode% starting standard par
	न‚नु य‚दि स्क‚न्धा एव प्र‚तीत्य‚स‚मुत्प‚न्ना न तु क‚श्चित्स‚त्त्वो यः संस‚र‚ति त‚दा ‚{\tiny $_{lb}$}‚‚{\color{DodgerBlue3}‚रागा‚{\tiny $_{3}$}‚द‚यो} यादृच्छिका अहेत‚वः स्युरित्याह (।)
	\pend% ending standard par
      
	  \bigskip
	  \begingroup
	
	    \large
	  
	    \begin{quote}
	  
	    
	    \stanza[\smallbreak]
	\label{pv.1.149b}\flagstanza{\tiny\textenglish{...1.149b}}रागादेः पाट‚वेक्ष‚णात् ।&अभ्यासान्न य‚दृच्छातोऽहेतोर्ज‚न्म‚विरोध‚तः ॥ १४९ ॥\&[\smallbreak]


	
	    \end{quote}
	  
	  \endgroup
	

	  \pstart \leavevmode% starting standard par
	\hphantom{.}अभ्यासा‚{\color{DodgerBlue3}‚द्रागादेः पाट‚व‚स्येक्ष‚णात् । अभ्यासादेव} ते भ‚व‚न्ति ‚{\color{DodgerBlue3}‚न} तु ‚{\color{DodgerBlue3}‚य‚दृच्छातः} । ‚{\tiny $_{lb}$}‚न ह्य‚कार‚णाद्विशेष‚स‚म्भ‚वः । नाप्य‚हेतुका रागाद‚योऽ‚{\color{DodgerBlue3}‚हेतो}‚र्हेतुर‚हित‚स्य ‚{\color{DodgerBlue3}‚ज‚न्म‚विरो‚{\tiny $_{lb}$}‚ध‚तः} । न ह्याकाशं क‚दाचिज्जाय‚ते । (१४९)
	\pend% ending standard par
      \label{div_pvv.1.150}
	  
	% new div opening: depth here is 2
	

	  \begin{center}%% label @type='head'
	\textbf{II. रागादीनां वातादिदोष‚ज‚त्व‚निरासः}
	\end{center}
	

	  \pstart \leavevmode% starting standard par
	स्यादेत‚त् (।) वात‚प्र‚कृतिर्मोह‚वान् । पित्त‚प्र‚कृतिर्द्वेष‚वान् । श्लेष्म‚प्र‚कृती राग‚{\tiny $_{lb}$}‚वानिति वातादिध‚र्मो दोष‚ग‚ण इत्याह ।
	\pend% ending standard par
      \textsuperscript{\textenglish{063/s}}
	  \bigskip
	  \begingroup
	
	    \large
	  
	    \begin{quote}
	  
	    
	    \stanza[\smallbreak]
	\label{pv.1.150}\flagstanza{\tiny\textenglish{....1.150}}व्य‚भिचारान्न वातादिध‚र्मः प्र‚कृतिसंक‚रात् ।&अदोष‚श्च त‚द‚न्योऽपि ध‚र्मः किं त‚स्य नेक्ष्य‚ते ॥ १५० ॥\&[\smallbreak]


	
	    \end{quote}
	  
	  \endgroup
	

	  \pstart \leavevmode% starting standard par
	\hphantom{.}a. ‚{\color{DodgerBlue3}‚न व्य‚भिचाराद् वातादिध‚र्मो} रागादिः‚{\tiny $_{4}$}‚ । वात‚प्र‚कृतिर‚पि ‚{\color{DodgerBlue3}‚न मोह‚ब‚हुलः} । ‚{\tiny $_{lb}$}‚पित्त‚प्र‚कृतिर‚पि न प‚टुद्वेषः । श्लेष्म‚प्र‚कृतिश्च नोद्भूत‚राग‚विशेषः क‚श्चिद् दृश्य‚त ‚{\tiny $_{lb}$}‚इति वातादिव्य‚भिचारिणो मोहाद‚यो न त‚द्धेत‚वः ॥ ‚{\color{DodgerBlue3}‚प्र‚कृतिसंकाराद‚दोष‚श्चेत्} संकीर्ण्ण‚प्र‚कृत‚यो हि पुरुषाः प्र‚त्येकं वात‚पित्त‚श्लेष्म‚णां स‚त्त्वात् । अतो ‚{\color{DodgerBlue3}‚दोषाणां न} कार‚ण‚व्य‚भिचारः । य‚द्येवं त‚स्या‚{\tiny $_{2}$}‚द्वेषादेर‚न्यो\edtext{}{\edlabel{pvv.63-1}\label{pvv.63-1}\lemma{न्यो}\Bfootnote{एकः कोप‚नः प्राज्ञः प्र‚स्वेदादिमांश्च न स्याद‚स्ति च (।) पित्त‚गुण‚स‚मुदायः ‚{\tiny $_{lb}$}‚कोप‚प‚रिणाम‚काले प्र‚ज्ञाप‚रिणामोपि मा भूत् (।) न हि लाभे प्र‚वेश‚च्छेदेपि ‚{\tiny $_{lb}$}‚पित्त‚प्र‚कृतेः निःश‚र‚णं युक्तं ।}} ध‚र्मः ख‚र‚त्वादिः\edtext{}{\edlabel{pvv.63-2}\label{pvv.63-2}\lemma{त्वादिः}\Bfootnote{ग‚न्धादिर्न दृश्य‚ते कुतः ।\begin{english} --- Placement of note uncertain; marked with a question mark in the edition (see encoding description for details).\end{english}}} ‚{\color{DodgerBlue3}‚त‚स्य वातादेः ‚{\tiny $_{lb}$}‚किन्नेक्ष्य‚ते\edtext{}{\edlabel{pvv.63-3}\label{pvv.63-3}\lemma{ते}\Bfootnote{न दृश्य‚ते च त‚द‚स‚देत‚त् ।\begin{english} --- Placement of note uncertain; marked with a question mark in the edition (see encoding description for details).\end{english}}}} । (१५०)
	\pend% ending standard par
      \label{div_pvv.1.151}
	  
	% new div opening: depth here is 2
	

	  \pstart \leavevmode% starting standard par
	b. अथ प्र‚{\tiny $_{5}$}‚त्येकं स\edtext{}{\edlabel{pvv.63-4}\label{pvv.63-4}\lemma{स}\Bfootnote{वातादीनां ।}}र्व्वेषां रागादिर्ध‚र्म‚स्त‚तो न व्य‚भिचार इत्याह (।)
	\pend% ending standard par
      
	  \bigskip
	  \begingroup
	
	    \large
	  
	    \begin{quote}
	  
	    
	    \stanza[\smallbreak]
	\label{pv.1.151}\flagstanza{\tiny\textenglish{....1.151}}न स‚र्व‚ध‚र्मः स‚र्वेंषां स‚म‚राग‚प्र‚स‚ङ्ग‚तः ।&रूपादिव‚द‚दोष‚श्चेत तुल्यं त‚त्रापि चोद‚न‚म् ॥ १५१ ॥\&[\smallbreak]


	
	    \end{quote}
	  
	  \endgroup
	

	  \pstart \leavevmode% starting standard par
	\hphantom{.}‚{\color{DodgerBlue3}‚न स‚र्व्व‚ध‚र्मः स‚र्व्वेषां स\edtext{}{\edlabel{pvv.63-5}\label{pvv.63-5}\lemma{स}\Bfootnote{एवं तुल्य‚द्वेषाद‚यः ।}}म‚स्य राग‚स्य प्र‚स‚ङ्गात्} । नानाप्र‚कृतिक‚त्वेपि रागादि‚{\tiny $_{lb}$}‚हेतोः स‚मान‚त्वात् । ‚{\color{DodgerBlue3}‚रूपादिव‚द‚दोष‚श्चेत्} । य‚था भूत‚मात्र‚हेतुक‚त्वेपि रू\edtext{}{\edlabel{pvv.63-6}\label{pvv.63-6}\lemma{रू}\Bfootnote{क्व‚चिद्र‚सः क्व‚चित् स्प‚र्शः ।}}पाद‚य उत्कृ‚{\tiny $_{lb}$}‚ष्य‚न्तेऽय‚कृष्य‚न्ते च क्व‚चित् त‚था रागाद‚योपीति चेत् । ‚{\color{DodgerBlue3}‚तुल्यं त‚त्र} रूपादा‚{\color{DodgerBlue3}‚व‚पि चोद‚नं} स‚म‚त्व‚स्य । (१५१)
	\pend% ending standard par
      \label{div_pvv.1.152}
	  
	% new div opening: depth here is 2
	
	  \bigskip
	  \begingroup
	
	    \large
	  
	    \begin{quote}
	  
	    
	    \stanza[\smallbreak]
	\label{pv.1.152a}\flagstanza{\tiny\textenglish{...1.152a}}आधिप‚त्यं विशिष्टानां य‚दि त‚त्र न क‚र्म‚णाम् ।\&[\smallbreak]


	
	    \end{quote}
	  
	  \endgroup
	

	  \pstart \leavevmode% starting standard par
	\hphantom{.}‚{\color{DodgerBlue3}‚आधिप‚त्यं विशिष्टानां क‚र्म‚णां} श्रुताश्रुत‚ल‚क्ष‚णानां । ‚{\color{DodgerBlue3}‚य‚दि त‚त्र} रूपादौ कार्ये ‚{\tiny $_{lb}$}‚नेष्य‚ते‚{\tiny $_{6}$}‚ भूत‚स‚ह‚कारिणां क‚र्म‚णां वैशिष्ट्यात् रूपादिविशेष इत्य‚र्थः ।
	\pend% ending standard par
      

	  \pstart \leavevmode% starting standard par
	स्यादेत‚न्न दोष‚मात्राद्रागाद‚योऽपि तु तेषां प‚रिणाम\edtext{}{\edlabel{pvv.63-7}\label{pvv.63-7}\lemma{रिणाम}\Bfootnote{रागादेर्यः स्व‚स्व‚कीयः प‚रिणाम‚विशेषः त‚द‚भावात् ।}} विशेषात् य‚था व्याध‚यः ।\edtext{\textsuperscript{*}}{\edlabel{pvv.63-8}\label{pvv.63-8}\lemma{*}\Bfootnote{द्वेष‚प‚रिणाम‚भावेपि ।}} ‚{\tiny $_{lb}$}‚त‚तो न स‚म‚राग‚तादिप्र‚स‚ङ्ग इत्याह (।)
	\pend% ending standard par
      
	  \bigskip
	  \begingroup
	
	    \large
	  
	    \begin{quote}
	  
	    
	    \stanza[\smallbreak]
	\label{pv.1.152b}\flagstanza{\tiny\textenglish{...1.152b}}विशेषेपि च दोषाणाम‚विशेषाद्;\&[\smallbreak]


	
	    \end{quote}
	  
	  \endgroup
	\textsuperscript{\textenglish{064/s}}

	  \pstart \leavevmode% starting standard par
	\hphantom{.}‚{\color{DodgerBlue3}‚विशेषेपि चे दोषाणां} प्र‚कोपादिनाऽ‚{\color{DodgerBlue3}‚विशेषात्} रागादीनां न दोष‚प‚रिणाम‚{\tiny $_{lb}$}‚हेतुता । (१५२)
	\pend% ending standard par
      \label{div_pvv.1.153_1.154_1.155}
	  
	% new div opening: depth here is 2
	

	  \pstart \leavevmode% starting standard par
	न‚न्व‚विशेषादित्य‚सिद्धो हेतुः । क‚फाद्युत्क‚र्षे रागाद्युत्क‚र्ष‚दृष्टेरित्या\edtext{}{\edlabel{pvv.64-1}\label{pvv.64-1}\lemma{दृष्टेरित्या}\Bfootnote{निजो योस्ति क‚र्क‚श‚त्वादौ ।}} ह ।
	\pend% ending standard par
      
	  \bigskip
	  \begingroup
	
	    \large
	  
	    \begin{quote}
	  
	    
	    \stanza[\smallbreak]
	\label{pv.1.152c}\flagstanza{\tiny\textenglish{...1.152c}}असिद्ध‚ता ॥ १५२ ॥\&[\smallbreak]


	
	    \end{quote}
	  
	  \endgroup
	
	  \bigskip
	  \begingroup
	
	    \large
	  
	    \begin{quote}
	  
	    
	    \stanza[\smallbreak]
	\label{pv.1.153a}\flagstanza{\tiny\textenglish{...1.153a}}न विकाराद् विकारेण स‚र्वेषां न च स‚र्व‚जाः ।\&[\smallbreak]


	
	    \end{quote}
	  
	  \endgroup
	

	  \pstart \leavevmode% starting standard par
	\hphantom{.}C. ‚{\color{DodgerBlue3}‚नासिद्ध‚ता}‚ऽविशेष‚स्य । त‚था हि ‚{\color{DodgerBlue3}‚स‚र्व्वेषां} क‚फादीनां विकारे‚{\tiny $_{7}$}‚णोत्क‚र्षेण पीड‚या ‚{\tiny $_{lb}$}‚विकारात् द्वेषो भ‚व‚ति न रागाद‚यः । स‚र्व्व‚ज‚त्वाद‚दोष इति चेत् । न च ‚{\color{DodgerBlue3}‚स‚र्व्व‚जाः} । ‚{\tiny $_{lb}$}‚स‚म‚राग‚तादिप्र‚स‚ङ्गादित्युक्तेः ॥
	\pend% ending standard par
      

	  \pstart \leavevmode% starting standard par
	किञ्च (।)
	\pend% ending standard par
      
	  \bigskip
	  \begingroup
	
	    \large
	  
	    \begin{quote}
	  
	    
	    \stanza[\smallbreak]
	\label{pv.1.153b}\flagstanza{\tiny\textenglish{...1.153b}}कार‚णे व‚र्द्ध‚माने च कार्य‚हानिर्न युज्य‚ते ॥ १५३ ॥\&[\smallbreak]


	
	    \end{quote}
	  
	  \endgroup
	
	  \bigskip
	  \begingroup
	
	    \large
	  
	    \begin{quote}
	  
	    
	    \stanza[\smallbreak]
	\label{pv.1.154a}\flagstanza{\tiny\textenglish{...1.154a}}तापादिष्विव;\&[\smallbreak]


	
	    \end{quote}
	  
	  \endgroup
	

	  \pstart \leavevmode% starting standard par
	\hphantom{.}स‚न्निपाताव‚स्थायां कार‚णे क‚फादिके ‚{\color{DodgerBlue3}‚व‚र्द्ध‚माने कार्य‚स्य} रागादे‚{\color{DodgerBlue3}‚र्हानिर्न ‚{\tiny $_{lb}$}‚युज्य‚ते} (१५३) ‚{\color{DodgerBlue3}‚तापादिष्विव} तापादीनामिव पित्तादिवृद्धौ ।
	\pend% ending standard par
      

	  \pstart \leavevmode% starting standard par
	d. स्यादेत‚द् (।) दोषाणां साम्ये रागाद‚यो भ‚व‚न्ति वैष‚म्ये तु द्वेषाद‚यः । त‚तो ‚{\tiny $_{lb}$}‚\leavevmode\ledsidenote{\textenglish{13a/MA}} विशेषेपि दोषाणां न विशिष्य‚न्ते रागाद‚य इत्य‚सिद्धो‚{\tiny $_{8}$}‚ हेतुरित्याह (।)
	\pend% ending standard par
      
	  \bigskip
	  \begingroup
	
	    \large
	  
	    \begin{quote}
	  
	    
	    \stanza[\smallbreak]
	\label{pv.1.154b}\flagstanza{\tiny\textenglish{...1.154b}}रागादेर्विकारोऽपि सुखादिजः ।\&[\smallbreak]


	
	    \end{quote}
	  
	  \endgroup
	

	  \pstart \leavevmode% starting standard par
	\hphantom{.}‚{\color{DodgerBlue3}‚रागादेर्दो}‚ष‚साम्याव‚स्थायां ‚{\color{DodgerBlue3}‚विकारोपि} वृद्धिल‚क्ष‚णो यः स ‚{\color{DodgerBlue3}‚सुखादिजः} आन्त‚र‚{\tiny $_{lb}$}‚धातुसाम्य‚स्प‚र्श‚प्र‚भ‚वेन सुखादिना व‚र्द्ध‚न्ते रागाद\edtext{}{\edlabel{pvv.64-2}\label{pvv.64-2}\lemma{रागाद}\Bfootnote{प‚रा श्लेष्मादिवृद्धाव‚पि य‚न्न रागादिवृद्धिः सा ।}}यः ।
	\pend% ending standard par
      
	  \bigskip
	  \begingroup
	
	    \large
	  
	    \begin{quote}
	  
	    
	    \stanza[\smallbreak]
	\label{pv.1.154c}\flagstanza{\tiny\textenglish{...1.154c}}वैष‚म्य‚जेन दुःखेन रागास्यानुद्भ‚वो य‚दि ॥ १५४ ॥\&[\smallbreak]


	
	    \end{quote}
	  
	  \endgroup
	
	  \bigskip
	  \begingroup
	
	    \large
	  
	    \begin{quote}
	  
	    
	    \stanza[\smallbreak]
	\label{pv.1.155a}\flagstanza{\tiny\textenglish{...1.155a}}वाच्यं केनोद्भ‚वः साम्यान्म‚द‚वृद्धिः स्म‚र‚स्त‚तः ।\&[\smallbreak]


	
	    \end{quote}
	  
	  \endgroup
	

	  \pstart \leavevmode% starting standard par
	\hphantom{.}‚{\color{DodgerBlue3}‚वैष‚म्य‚जेन} दुःखेन द्वेष‚स्योत्पाद‚नात् त‚द्विरुद्ध‚स्य ‚{\color{DodgerBlue3}‚रागास्यानुद्भ‚वो य‚दि} म‚तः ‚{\tiny $_{lb}$}‚(१५४) त‚दा ‚{\color{DodgerBlue3}‚वाच्यं केन} हेतूना राग‚{\color{DodgerBlue3}‚स्योद्भ‚वः} । ‚{\color{DodgerBlue3}‚साम्याद्} दोषाणां ‚{\color{DodgerBlue3}‚म‚द}‚स्य शुक्र‚स्य ‚{\tiny $_{lb}$}‚‚{\color{DodgerBlue3}‚वृद्धिः} (।) ‚{\color{DodgerBlue3}‚स्म‚रो} रागः । शुक्र‚वृद्धेर्य‚दीष्ट‚मेत‚द‚प्य‚युक्तं य‚तः (।)
	\pend% ending standard par
      
	  \bigskip
	  \begingroup
	
	    \large
	  
	    \begin{quote}
	  
	    
	    \stanza[\smallbreak]
	\label{pv.1.155b}\flagstanza{\tiny\textenglish{...1.155b}}रागी विष‚म‚दोषोऽपि दृष्टः साम्येऽपि नाप‚रः ॥ १५५ ॥\&[\smallbreak]


	
	    \end{quote}
	  
	  \endgroup
	

	  \pstart \leavevmode% starting standard par
	\hphantom{.}‚{\color{DodgerBlue3}‚रागी विष‚म‚दो\edtext{}{\edlabel{pvv.64-3}\label{pvv.64-3}\lemma{दो}\Bfootnote{धातुवैष‚म्येपि ।}}षोऽपि} राग‚च‚रितः क‚श्चिद् ‚{\color{DodgerBlue3}‚दृष्टः साम्येपि नाप‚रः} प्र‚तिसंख्या‚{\tiny $_{1}$}‚न‚{\tiny $_{lb}$}‚ब‚ली म‚न्द‚राग‚प्र‚कृतिर्व्वा दृष्टः । (१५५)
	\pend% ending standard par
      \label{div_pvv.1.156}
	  
	% new div opening: depth here is 2
	\textsuperscript{\textenglish{065/s}}
	  \bigskip
	  \begingroup
	
	    \large
	  
	    \begin{quote}
	  
	    
	    \stanza[\smallbreak]
	\label{pv.1.156a}\flagstanza{\tiny\textenglish{...1.156a}}क्ष‚याद‚सृक‚स्रुतोऽप्य‚न्ये;\&[\smallbreak]


	
	    \end{quote}
	  
	  \endgroup
	

	  \pstart \leavevmode% starting standard par
	\hphantom{.}‚{\color{DodgerBlue3}‚क्ष‚या}‚च्छुक्र‚स्यासुक‚श्रु (? स्रु) तो र‚क्तं ‚{\color{DodgerBlue3}‚क्ष‚र}\edtext{}{\edlabel{pvv.65-1}\label{pvv.65-1}\lemma{क्तं}\Bfootnote{इन्द्रियेण ।}} ‚{\color{DodgerBlue3}‚न्तोप्य‚न्ये राग‚ब‚हुला दृष्टा इति} शुक्र‚म‚पि न राग‚हेतुः ।
	\pend% ending standard par
      

	  \pstart \leavevmode% starting standard par
	किञ्च (।)
	\pend% ending standard par
      
	  \bigskip
	  \begingroup
	
	    \large
	  
	    \begin{quote}
	  
	    
	    \stanza[\smallbreak]
	\label{pv.1.156b}\flagstanza{\tiny\textenglish{...1.156b}}नैक‚स्त्रीनिय‚तो म‚दः ।\&[\smallbreak]


	
	    \end{quote}
	  
	  \endgroup
	

	  \pstart \leavevmode% starting standard par
	नैक‚स्त्रीनिय‚तो म‚दः । न ह्येकां स्त्रिय‚म‚पेक्ष्य शुक्रं शुक्रीभ‚व‚ति । अपि तु ‚{\tiny $_{lb}$}‚स‚र्व्वाः ।
	\pend% ending standard par
      
	  \bigskip
	  \begingroup
	
	    \large
	  
	    \begin{quote}
	  
	    
	    \stanza[\smallbreak]
	\label{pv.1.156c}\flagstanza{\tiny\textenglish{...1.156c}}तेनैक‚स्यां न तीव्रः स्याद्, अङ्ग‚रूपाद्य‚पीति चेत् ॥ १५६ ॥\&[\smallbreak]


	
	    \end{quote}
	  
	  \endgroup
	

	  \pstart \leavevmode% starting standard par
	\hphantom{.}‚{\color{DodgerBlue3}‚तेन} साधार‚ण‚त्वेन शुक्र‚स्य त‚ज्ज‚न्यो राग ‚{\color{DodgerBlue3}‚एक‚स्यां} स्त्रियां ‚{\color{DodgerBlue3}‚न तीव्रः स्यात्\edtext{}{\edlabel{pvv.65-2}\label{pvv.65-2}\lemma{स्यात्}\Bfootnote{क‚स्य‚चिद् भ‚व‚ति च त‚तो न म‚दः कार‚ण राग‚स्य ।}}} किन्तु साधार‚णः । ‚{\color{DodgerBlue3}‚अङ्ग‚रूपाद्य\edtext{}{\edlabel{pvv.65-3}\label{pvv.65-3}\lemma{रूपाद्य}\Bfootnote{रूप उप‚चार‚गुण‚रागी चेति त्रिविध आदिना रागी ।}}पीति चेत्} । (१५६)
	\pend% ending standard par
      \label{div_pvv.1.157}
	  
	% new div opening: depth here is 2
	

	  \pstart \leavevmode% starting standard par
	e. रूप‚यौव‚नोप‚चारादि च स‚ह‚कारि राग‚वृद्धेरिति चेत् ।
	\pend% ending standard par
      
	  \bigskip
	  \begingroup
	
	    \large
	  
	    \begin{quote}
	  
	    
	    \stanza[\smallbreak]
	\label{pv.1.157}\flagstanza{\tiny\textenglish{....1.157}}न स‚र्वेषाम‚नेकान्तान्न चाप्य‚निय‚तो भ‚वेत् ।&अगुण‚ग्राहिणोऽपि स्यात् अङ्गं सो ऽपि गुण‚ग्र‚हः ॥ १५७ ॥\&[\smallbreak]


	
	    \end{quote}
	  
	  \endgroup
	

	  \pstart \leavevmode% starting standard par
	\hphantom{.}न युक्त‚मेत‚त् । ‚{\color{DodgerBlue3}‚स‚र्व्वेषां} रूपादीना‚{\color{DodgerBlue3}‚म‚नेकान्त}‚त्वात्‚{\tiny $_{2}$}‚ । रूपादिर‚हितेष्व‚पि रागो‚{\tiny $_{lb}$}‚त्क‚र्ष‚दृष्टेः । किञ्च (।) य‚दि शुक्रं रूपादि च स्त्रियाः कार‚णं ‚{\color{DodgerBlue3}‚राग‚स्य} त‚दा ‚{\color{DodgerBlue3}‚न चाप्य‚{\tiny $_{lb}$}‚निय‚तोऽ}‚विष‚यीकृत‚स्त्रीविशेषः साधार‚णो रागो न ‚{\color{DodgerBlue3}‚भ‚वेत्} । रूपादिस‚ह‚कारिणो‚{\tiny $_{lb}$}‚ऽव्यापारात् । त‚था रूप‚व‚त्या गुण‚म‚शुभाभाव‚नाभावितं ‚{\color{DodgerBlue3}‚गृह्ण‚तोपि} रागः स्यात् । ‚{\tiny $_{lb}$}‚शुक्र‚रूप‚योस्त‚द्धेत्वोः स‚द्भावात् । अङ्ग‚न्निमित्तं ‚{\color{DodgerBlue3}‚सोपि गुण‚ग्र‚ह}‚स्त‚तोऽशुभां भाव‚य‚तो ‚{\tiny $_{lb}$}‚न स्याद्रागः (। १५७)
	\pend% ending standard par
      \label{div_pvv.1.158}
	  
	% new div opening: depth here is 2
	
	  \bigskip
	  \begingroup
	
	    \large
	  
	    \begin{quote}
	  
	    
	    \stanza[\smallbreak]
	\label{pv.1.158a}\flagstanza{\tiny\textenglish{...1.158a}}य‚दि स‚र्वो गुण‚ग्राही स्याद्, हेतोर‚विशेष‚तः ।\&[\smallbreak]


	
	    \end{quote}
	  
	  \endgroup
	

	  \pstart \leavevmode% starting standard par
	\hphantom{.}‚{\color{DodgerBlue3}‚य‚दि स‚र्व्वः} शुभाशुभ‚भाव‚को ‚{\color{DodgerBlue3}‚गुण‚ग्रा‚{\tiny $_{3}$}‚ही} स्यात् । गुण‚ग्र‚ह‚स्य ‚{\color{DodgerBlue3}‚हेतो रूपादेः} स‚र्व्वान् ‚{\color{DodgerBlue3}‚प्र‚त्य‚विशेष‚तः} ।
	\pend% ending standard par
      

	  \pstart \leavevmode% starting standard par
	किञ्च (।)
	\pend% ending standard par
      
	  \bigskip
	  \begingroup
	
	    \large
	  
	    \begin{quote}
	  
	    
	    \stanza[\smallbreak]
	\label{pv.1.158b}\flagstanza{\tiny\textenglish{...1.158b}}य‚द‚व‚स्थो म‚तो रागी न द्वेषी स्याच्च तादृशः ॥ १५८ ॥\&[\smallbreak]


	
	    \end{quote}
	  
	  \endgroup
	

	  \pstart \leavevmode% starting standard par
	\hphantom{.}‚{\color{DodgerBlue3}‚य‚द‚व‚स्थः} श्लेष्म‚प्र‚कृतिस्थः पुरुषो ‚{\color{DodgerBlue3}‚रागी म‚त‚स्त}‚द‚प‚स्थो ‚{\color{DodgerBlue3}‚न द्वेषी स्यात्} । (१५८)
	\pend% ending standard par
      \label{div_pvv.1.159_1.160_1.161_1.162_1.163_1.164_1.165_1.166}
	  
	% new div opening: depth here is 2
	

	  \pstart \leavevmode% starting standard par
	कुत इत्याह (।)
	\pend% ending standard par
      \textsuperscript{\textenglish{066/s}}
	  \bigskip
	  \begingroup
	
	    \large
	  
	    \begin{quote}
	  
	    
	    \stanza[\smallbreak]
	\label{pv.1.159a}\flagstanza{\tiny\textenglish{...1.159a}}त‚योर‚स‚म‚रूप‚त्वान्निय‚म‚श्चात्र नेक्ष्य‚ते ।\&[\smallbreak]


	
	    \end{quote}
	  
	  \endgroup
	

	  \pstart \leavevmode% starting standard par
	\hphantom{.}‚{\color{DodgerBlue3}‚त‚यो} राग‚द्वेष‚योर‚स‚म‚रूप‚त्वाद्विरुद्ध‚त्वात् त‚दुत्पादिक‚योर‚व‚स्थ‚योर‚पि विरोधः ॥ ‚{\tiny $_{lb}$}‚स्यादेत‚द्रागोत्पादिकायाम‚व‚स्थायां द्वेषो न भ‚व‚त्येवेत्याह (।) ‚{\color{DodgerBlue3}‚निय‚म‚श्चात्र नेक्ष्य‚ते} । ‚{\tiny $_{lb}$}‚श्लेष्माव‚स्थो रागी न द्वेषीति नात्र निय‚मः ।
	\pend% ending standard par
      

	  \pstart \leavevmode% starting standard par
	त‚वापि क‚स्माद‚मी स‚म‚राग‚{\tiny $_{4}$}‚प्र‚स‚ङ्गाद‚यो दोषा न भ‚व‚न्तीत्याह (।)
	\pend% ending standard par
      
	  \bigskip
	  \begingroup
	
	    \large
	  
	    \begin{quote}
	  
	    
	    \stanza[\smallbreak]
	\label{pv.1.159b}\flagstanza{\tiny\textenglish{...1.159b}}स‚जातिवास‚नाभेद‚प्र‚तिब‚द्ध‚प्र‚वृत्त‚यः ॥ १५९ ॥\&[\smallbreak]


	
	    \end{quote}
	  
	  \endgroup
	
	  \bigskip
	  \begingroup
	
	    \large
	  
	    \begin{quote}
	  
	    
	    \stanza[\smallbreak]
	\label{pv.1.160a}\flagstanza{\tiny\textenglish{...1.160a}}य‚स्य रागाद‚य‚स्त‚स्य नैते दोषाः प्र‚स‚ङ्गिनः ।\&[\smallbreak]


	
	    \end{quote}
	  
	  \endgroup
	

	  \pstart \leavevmode% starting standard par
	\hphantom{.}‚{\color{DodgerBlue3}‚स‚जातिवास‚ना} आत्मात्मीय‚ग्र‚ह‚मूल‚स्य स\edtext{}{\edlabel{pvv.66-1}\label{pvv.66-1}\lemma{स}\Bfootnote{स‚त्काय‚द‚र्श‚न‚स्य ।}} जातेः पूर्व्व‚पूर्व्वाभ्य‚स्त‚स्य रागादे‚{\tiny $_{lb}$}‚र्व्वास‚नाऽप‚राप‚र‚रागादिज‚निकाः श‚क्त‚यः तासां ‚{\color{DodgerBlue3}‚भेदः} प‚र‚स्प‚र‚तः त‚त्र ‚{\color{DodgerBlue3}‚प्र‚तिब‚द्धा ‚{\tiny $_{lb}$}‚प्र‚वृत्ति}‚र्ज‚न्म येषां ते त‚था (१५९) ‚{\color{DodgerBlue3}‚रागाद‚यो य‚स्य} बौद्ध‚स्य म‚ते ‚{\color{DodgerBlue3}‚त‚स्य नैते}‚ऽन‚न्त‚र‚{\tiny $_{lb}$}‚मुक्ता ‚{\color{DodgerBlue3}‚दोषाः प्र‚स‚ङ्गिनः} ।
	\pend% ending standard par
      

	  \begin{center}%% label @type='head'
	\textbf{III. रागादीनां भूत‚ध‚र्म‚त्व‚निरासः}
	\end{center}
	
	  \bigskip
	  \begingroup
	
	    \large
	  
	    \begin{quote}
	  
	    
	    \stanza[\smallbreak]
	\label{pv.1.160b}\flagstanza{\tiny\textenglish{...1.160b}}एतेन भूत‚ध‚र्म‚त्वं निषिद्धं;\&[\smallbreak]


	
	    \end{quote}
	  
	  \endgroup
	

	  \pstart \leavevmode% starting standard par
	\hphantom{.}‚{\color{DodgerBlue3}‚एतेन} वातादिध‚र्म‚त्व‚निषेधेन ‚{\color{DodgerBlue3}‚भूत‚ध‚र्म‚त्वं निषिद्धं} रागादेर्ब्बोंद्ध‚व्यं ‚{\color{DodgerBlue3}‚दोषा\edtext{}{\edlabel{pvv.66-2}\label{pvv.66-2}\lemma{दोषा}\Bfootnote{वात‚पित्त‚श्लेष्म‚णां क्र‚मात् म‚ह‚दादित्वेन रागादेर्भूत‚{\tiny $_{lb}$}‚ध‚र्म‚त्वं न भूत/?/त्वे हेतुर‚यं ।}}णां} म‚रुत्तेजोऽम्भः स्व‚भाव‚त्वात् ॥
	\pend% ending standard par
      

	  \pstart \leavevmode% starting standard par
	भ‚व‚न्तु स‚भा\edtext{}{\edlabel{pvv.66-3}\label{pvv.66-3}\lemma{भा}\Bfootnote{स्व‚स‚दृश‚हेतुका रागाद‚यः सुराश‚क्त्योरिव ।}}ग‚हेतुकाः पृथिव्याद्याश्रितास्तु स्युर्द्ध‚व‚लादिव‚दित्याह ।
	\pend% ending standard par
      
	  \bigskip
	  \begingroup
	
	    \large
	  
	    \begin{quote}
	  
	    
	    \stanza[\smallbreak]
	\label{pv.1.160c}\flagstanza{\tiny\textenglish{...1.160c}}निश्च‚य‚स्य च ॥ १६० ॥\&[\smallbreak]


	
	    \end{quote}
	  
	  \endgroup
	
	  \bigskip
	  \begingroup
	
	    \large
	  
	    \begin{quote}
	  
	    
	    \stanza[\smallbreak]
	\label{pv.1.161a}\flagstanza{\tiny\textenglish{...1.161a}}निषेधान्न पृथिव्यादिनिःश्रिता ध‚व‚लाद‚यः ।\&[\smallbreak]


	
	    \end{quote}
	  
	  \endgroup
	

	  \pstart \leavevmode% starting standard par
	\hphantom{.}‚{\color{DodgerBlue3}‚निश्र‚य‚स्या}‚श्र‚य‚स्य ‚{\color{DodgerBlue3}‚निषेधाद}‚नाश्र‚यात्स‚द‚स‚तोरित्यादिना ‚{\color{DodgerBlue3}‚न पृथिव्यादिनिःश्रिता ‚{\tiny $_{lb}$}‚ध‚व‚लाद‚योपि} । कुत एव रागाद‚यः ।
	\pend% ending standard par
      

	  \pstart \leavevmode% starting standard par
	क‚थ‚न्त‚र्हि भूतान्याश्रित्योपादाय रूप‚मुत्प‚द्य‚त इतीष्ट‚मित्याह (।)
	\pend% ending standard par
      
	  \bigskip
	  \begingroup
	
	    \large
	  
	    \begin{quote}
	  
	    
	    \stanza[\smallbreak]
	\label{pv.1.161b}\flagstanza{\tiny\textenglish{...1.161b}}त‚दुपादाय-श‚ब्द‚श्च हेत्व‚र्थः स्वाश्र‚येण च ॥ १६१ ॥\&[\smallbreak]


	
	    \end{quote}
	  
	  \endgroup
	
	  \bigskip
	  \begingroup
	
	    \large
	  
	    \begin{quote}
	  
	    
	    \stanza[\smallbreak]
	\label{pv.1.162a}\flagstanza{\tiny\textenglish{...1.162a}}अविनिर्भाग‚व‚र्तित्वाद् रूपादेराश्र‚योऽपि वा\&[\smallbreak]


	
	    \end{quote}
	  
	  \endgroup
	\textsuperscript{\textenglish{067/s}}

	  \pstart \leavevmode% starting standard par
	\hphantom{.}‚{\color{DodgerBlue3}‚त‚दुपादाय श‚ब्द‚श्च} तानि भूतानि उपादाय श‚ब्द‚श्च ‚{\color{DodgerBlue3}‚हेत्व‚र्थः} । भूतानि हेतू‚{\tiny $_{lb}$}‚कृत्योपादाय रूप‚मुत्प‚द्य‚त इत्य‚र्थः । ‚{\color{DodgerBlue3}‚स्वा\edtext{}{\edlabel{pvv.67-1}\label{pvv.67-1}\lemma{स्वा}\Bfootnote{अभ्युप‚ग‚म्याह ।}}श्र‚येण} भूत‚च‚तुष्केण ‚{\color{DodgerBlue3}‚रूपादे}‚रेक‚साम‚ग्र्‏य‚धी‚{\tiny $_{lb}$}‚न‚त्वेन (१६१) ‚{\color{DodgerBlue3}‚अविनि‚{\tiny $_{6}$}‚र्भाग‚व‚र्त्तित्वा}‚द्विभागेनान‚व‚स्थिते‚{\color{DodgerBlue3}‚राश्र‚योपि वा} भूत‚च‚तुष्कं ॥
	\pend% ending standard par
      
	  \bigskip
	  \begingroup
	
	    \large
	  
	    \begin{quote}
	  
	    
	    \stanza[\smallbreak]
	\label{pv.1.162b}\flagstanza{\tiny\textenglish{...1.162b}}म‚दादिश‚क्तेरिव चेद् विनिर्भागो;\&[\smallbreak]


	
	    \end{quote}
	  
	  \endgroup
	

	  \pstart \leavevmode% starting standard par
	\hphantom{.}स्यादेत‚त् (।) सुराया ‚{\color{DodgerBlue3}‚म‚द}‚श‚क्तेरा‚{\color{DodgerBlue3}‚दि}‚श‚ब्दात् क\edtext{}{\edlabel{pvv.67-2}\label{pvv.67-2}\lemma{क}\Bfootnote{ध‚त्तूर}}न‚कादेरुन्माद‚{\color{DodgerBlue3}‚श‚क्ते\edtext{}{\edlabel{pvv.67-3}\label{pvv.67-3}\lemma{क्ते}\Bfootnote{शुक्तिभावे, म‚द्य‚निवृत्तेर्व्व‚स्त्व‚निवृत्तौ ।}}रिवाश्रित}‚{\tiny $_{lb}$}‚त्वेपि ‚{\color{DodgerBlue3}‚विनिर्भागो} भूतैः स‚ह चैत‚न्य‚स्य स्यात् ।---
	\pend% ending standard par
      

	  \pstart \leavevmode% starting standard par
	a. भूत‚चैन्य‚योर्भेदात्
	\pend% ending standard par
      

	  \pstart \leavevmode% starting standard par
	य‚था म‚द‚श‚क्तिर‚हिता व्याप‚न्ना सुरा दृश्य‚ते । एवं चैत‚न्य‚र‚हितानि भूतान्य‚पि ‚{\tiny $_{lb}$}‚स्युरिति चेत् (।)
	\pend% ending standard par
      
	  \bigskip
	  \begingroup
	
	    \large
	  
	    \begin{quote}
	  
	    
	    \stanza[\smallbreak]
	\label{pv.1.162c}\flagstanza{\tiny\textenglish{...1.162c}}न व‚स्तुनः ॥ १६२ ॥\&[\smallbreak]


	
	    \end{quote}
	  
	  \endgroup
	
	  \bigskip
	  \begingroup
	
	    \large
	  
	    \begin{quote}
	  
	    
	    \stanza[\smallbreak]
	\label{pv.1.163}\flagstanza{\tiny\textenglish{....1.163}}श‚क्तिर‚र्थान्त‚रं व‚स्तु न‚श्येन्नाश्रित‚माश्र‚ये ।&तिष्ठ‚त्य‚विक‚ले याति, त‚त्तुल्यं चेन्न भेद‚तः ॥ १६३ ॥\&[\smallbreak]


	
	    \end{quote}
	  
	  \endgroup
	
	  \bigskip
	  \begingroup
	
	    \large
	  
	    \begin{quote}
	  
	    
	    \stanza[\smallbreak]
	\label{pv.1.164a}\flagstanza{\tiny\textenglish{...1.164a}}भूत‚चैत‚न‚योः;\&[\smallbreak]


	
	    \end{quote}
	  
	  \endgroup
	

	  \pstart \leavevmode% starting standard par
	\hphantom{.}‚{\color{DodgerBlue3}‚न व‚स्तुनः} सुरादेः ‚{\color{DodgerBlue3}‚श‚क्तिर‚र्थान्त‚रं} किन्तु ‚{\color{DodgerBlue3}‚व‚स्त्वे}‚वार्थ‚क्रियाश‚क्तं शुक्त्याद्य‚व‚स्थायां ‚{\tiny $_{lb}$}‚‚{\color{DodgerBlue3}‚न\edtext{}{\edlabel{pvv.67-4}\label{pvv.67-4}\lemma{न}\Bfootnote{क्ष‚णिक‚त्वात् ।}}श्येत्} । अस‚म‚र्थ‚स्य चोत्प‚त्ति \edtext{}{\edlabel{pvv.67-5}\label{pvv.67-5}\lemma{त्ति}\Bfootnote{अथ सुरास्थितैव त‚दाह ।}} ‚{\color{DodgerBlue3}‚र्नाश्रितं} श‚क्त्याद्य‚{\color{DodgerBlue3}‚विक‚ल आश्र‚ये तिष्ठ‚ति‚{\tiny $_{7}$}‚ यात्य-} प‚ति । ‚{\color{DodgerBlue3}‚त‚त्तुल्य‚ञ्चेत्} भूत‚चेत‚न‚योर\edtext{}{\edlabel{pvv.67-6}\label{pvv.67-6}\lemma{योर}\Bfootnote{शुक्ल‚त्वादिव‚द् भूताश्रित‚त्वं ।}}प्यैकात्म्यं ।‚{\tiny $_{5}$}‚ त‚द्भूत‚चैत‚न्ये भूते न‚ष्टे निश्चेत‚नं ‚{\tiny $_{lb}$}‚भूतान्त‚र‚मुत्प‚द्य‚त इति चेत् । ‚{\color{DodgerBlue3}‚न} युक्त‚मेत‚त् । ‚{\color{DodgerBlue3}‚भेद‚तो} (१६३) भूत‚चैत‚न्य‚योः ।---
	\pend% ending standard par
      

	  \pstart \leavevmode% starting standard par
	क‚थ‚मित्याह (।)
	\pend% ending standard par
      
	  \bigskip
	  \begingroup
	
	    \large
	  
	    \begin{quote}
	  
	    
	    \stanza[\smallbreak]
	\label{pv.1.164b}\flagstanza{\tiny\textenglish{...1.164b}}भिन्न‚प्र‚तिभासाव‚बोध‚तः ।&अविकार‚ञ्च काय‚स्य तुल्य‚रूपं भ‚वेन्म‚नः ॥ १६४ ॥\&[\smallbreak]


	
	    \end{quote}
	  
	  \endgroup
	
	  \bigskip
	  \begingroup
	
	    \large
	  
	    \begin{quote}
	  
	    
	    \stanza[\smallbreak]
	\label{pv.1.165a}\flagstanza{\tiny\textenglish{...1.165a}}रूपादिव‚त्;\&[\smallbreak]


	
	    \end{quote}
	  
	  \endgroup
	

	  \pstart \leavevmode% starting standard par
	\hphantom{.}‚{\color{DodgerBlue3}‚भिन्न‚प्र‚तिभासाव‚बोध‚तः} । भिन्नाकार‚ज्ञान‚विष‚य‚त‚या भिन्ने\edtext{}{\edlabel{pvv.67-7}\label{pvv.67-7}\lemma{भिन्ने}\Bfootnote{भिन्नाकार‚ज्ञानाभ्याम‚न‚योर्ग्र‚हाद‚न्य‚थातिप्र‚स‚ङ्गः विश्व‚मेकं स्यात् काय‚म‚न‚{\tiny $_{lb}$}‚इन्द्रियाभ्यां च ग्राह्य‚त्वात् । य‚दा श‚रीर‚म‚नाल‚म्व्य विष‚यान्त‚र‚माल‚म्ब‚ते त‚दा‚{\tiny $_{lb}$}‚ऽश‚रीराकारोद‚यात् ।}} भूत‚चैत‚न्ये ‚{\color{DodgerBlue3}‚नान्य‚था} \leavevmode\ledsidenote{\textenglish{068/s}} ‚{\color{DodgerBlue3}‚क्व‚चिद‚पि भेद‚सिद्धिः} । य‚दि च देह‚चित‚योरैक्यं त‚दा ‚{\color{DodgerBlue3}‚अविकार‚ञ्च काय‚स्य} याव‚न्न ‚{\tiny $_{lb}$}‚विक्रिय‚ते देह‚न्ताव‚त्तुल्य‚मेकाकार‚{\color{DodgerBlue3}‚म्भ‚वेन्म‚नो}\edtext{}{\edlabel{pvv.68-1}\label{pvv.68-1}\lemma{मेकाकार}\Bfootnote{भिन्न‚भिन्नार्थाकारं म‚नोज्ञानं न स्यात् ।}}(१६४) ‚{\color{DodgerBlue3}‚रूपादिव‚त्} ।
	\pend% ending standard par
      

	  \begin{center}%% label @type='head'
	\textbf{(b. वास‚नाभेद‚तो भेदात्)}
	\end{center}
	

	  \pstart \leavevmode% starting standard par
	स्यादेत‚त् (।) एक‚{\tiny $_{8}$}‚रूप‚त्वेपि देह‚स्यार्थानां नानारूप‚त्वात् ज्ञान‚म‚पि त‚थे‚{\tiny $_{lb}$}‚\leavevmode\ledsidenote{\textenglish{13b/MA}} त्याह (।)
	\pend% ending standard par
      
	  \bigskip
	  \begingroup
	
	    \large
	  
	    \begin{quote}
	  
	    
	    \stanza[\smallbreak]
	\label{pv.1.165b}\flagstanza{\tiny\textenglish{...1.165b}}विक‚ल्प‚स्य कैवार्थ‚प‚र‚त‚न्त्र‚ता ।\&[\smallbreak]


	
	    \end{quote}
	  
	  \endgroup
	

	  \pstart \leavevmode% starting standard par
	\hphantom{.}‚{\color{DodgerBlue3}‚विक‚ल्प‚स्य कैवार्थ‚प‚र‚त‚न्त्र‚ता} । येनार्थ‚नानात्वात् क‚ल्प‚नापि त‚था स्यात् । ‚{\tiny $_{lb}$}‚अन‚पेक्षित‚स‚न्निध‚यः उपादान‚व‚शेन विक‚ल्पाः प्र‚व‚र्त‚न्ते इति वास‚नाभेदादेषां भेदः ।
	\pend% ending standard par
      
	  \bigskip
	  \begingroup
	
	    \large
	  
	    \begin{quote}
	  
	    
	    \stanza[\smallbreak]
	\label{pv.1.165c}\flagstanza{\tiny\textenglish{...1.165c}}अन‚पेक्ष्य य‚दा कायं वास‚नाबोध‚कार‚ण‚म् ॥ १६५ ॥\&[\smallbreak]


	
	    \end{quote}
	  
	  \endgroup
	
	  \bigskip
	  \begingroup
	
	    \large
	  
	    \begin{quote}
	  
	    
	    \stanza[\smallbreak]
	\label{pv.1.166}\flagstanza{\tiny\textenglish{....1.166}}ज्ञानं स्यात् क‚स्य‚चित् किञ्चित् कुत‚श्चित् तेन किञ्च‚न ।&अविज्ञान‚स्य विज्ञानानुपादानाच्च सिध्य‚ति ॥ १६६ ॥\&[\smallbreak]


	
	    \end{quote}
	  
	  \endgroup
	

	  \pstart \leavevmode% starting standard par
	एव‚ञ्चा \edtext{}{\edlabel{pvv.68-2}\label{pvv.68-2}\lemma{ञ्चा}\Bfootnote{स्व‚म‚तं स‚म‚र्थ‚य‚ते ।}} ‚{\color{DodgerBlue3}‚न‚पेक्ष्य य‚दा कायं} किञ्चिज्ज्ञानं ‚{\color{DodgerBlue3}‚वास‚नाबोध‚स्य} प्र‚बोध‚स्य ‚{\color{DodgerBlue3}‚कार‚णं} स्यात् । (१६५) ‚{\color{DodgerBlue3}‚क‚स्य\edtext{}{\edlabel{pvv.68-3}\label{pvv.68-3}\lemma{स्य}\Bfootnote{कार्य‚स्य ।}}चि}‚दुत्प‚द्य‚मान‚स्य ज्ञान‚स्य त‚दा ‚{\color{DodgerBlue3}‚तेन} प्र‚बोध‚कानुरोधेन ‚{\tiny $_{lb}$}‚‚{\color{DodgerBlue3}‚कुत‚श्चि}‚ज्ज्ञानाद‚न‚न्त‚रं ‚{\color{DodgerBlue3}‚किञ्च‚न ज्ञानं स्यादिति} सुस्थ‚म‚स्य‚{\tiny $_{1}$}‚ प‚क्षे । ‚{\color{DodgerBlue3}‚अविज्ञा}\edtext{\textsuperscript{*}}{\edlabel{pvv.68-4}\label{pvv.68-4}\lemma{*}\Bfootnote{एत‚च्च त‚योर्व्विरोधात् ।}} ‚{\color{DodgerBlue3}‚न‚स्य} विज्ञान‚शून्य‚स्य लोष्टादे‚{\color{DodgerBlue3}‚र्विज्ञानानुपादानाच्च\edtext{}{\edlabel{pvv.68-5}\label{pvv.68-5}\lemma{र्विज्ञानानुपादानाच्च}\Bfootnote{न भूत‚ध‚र्म‚त्व‚प्र‚तिक्षेपादेरेव ।}} सिध्य‚ति} विज्ञानादेव विज्ञानं ।\edtext{\textsuperscript{*}}{\edlabel{pvv.68-6}\label{pvv.68-6}\lemma{*}\Bfootnote{अविज्ञानान्न विज्ञान‚मिति य‚दुक्तं त‚दिष्ट‚मेव सांख्य‚स्येति सिद्ध‚साध‚नं ।}} ‚{\tiny $_{lb}$}‚(१६६)
	\pend% ending standard par
      \label{div_pvv.1.167_1.168_1.169}
	  
	% new div opening: depth here is 2
	

	  \begin{center}%% label @type='head'
	\textbf{(c. अविज्ञान‚तो विज्ञानानुत्पादात् ।)}
	\end{center}
	
	  \bigskip
	  \begingroup
	
	    \large
	  
	    \begin{quote}
	  
	    
	    \stanza[\smallbreak]
	\label{pv.1.167}\flagstanza{\tiny\textenglish{....1.167}}विज्ञान‚श‚क्तिस‚म्ब‚न्धादिष्ट‚ञ्चेत् स‚र्व‚व‚स्तुनः ।&एत‚च्छांख्य‚प‚शोः कोऽन्यः स‚ल‚ज्जो व‚क्तुमीह‚ते ॥ १६७ ॥\&[\smallbreak]


	
	    \end{quote}
	  
	  \endgroup
	
	  \bigskip
	  \begingroup
	
	    \large
	  
	    \begin{quote}
	  
	    
	    \stanza[\smallbreak]
	\label{pv.1.168a}\flagstanza{\tiny\textenglish{...1.168a}}अदृष्ट‚पूर्व‚म‚स्तीति तृणाग्रे क‚रिणां श‚त‚म् ।\&[\smallbreak]


	
	    \end{quote}
	  
	  \endgroup
	

	  \pstart \leavevmode% starting standard par
	\hphantom{.}‚{\color{DodgerBlue3}‚स‚र्व्व‚स्य व‚स्तुनो विज्ञान‚श‚क्तिस‚म्ब‚न्धाद}‚विज्ञानाद्विज्ञानानुत्पाद‚न‚{\color{DodgerBlue3}‚मिष्टं\edtext{}{\edlabel{pvv.68-7}\label{pvv.68-7}\lemma{मिष्टं}\Bfootnote{स‚र्व्व‚त्र श‚क्तिव्य‚क्ती त‚न्म‚तेन ।}}} चेत् । ‚{\tiny $_{lb}$}‚ए\edtext{}{\edlabel{pvv.68-8}\label{pvv.68-8}\lemma{ए}\Bfootnote{सिद्ध‚साध‚न‚त्वं प‚रिह‚र‚ति ।}} त‚द्विज्ञान‚स्य श‚क्तिरूप‚त‚याऽव‚स्थानं सां ख्य‚{\color{DodgerBlue3}‚प‚शोः} ‚{\color{DodgerBlue3}‚कोऽन्यः स‚ल‚ज्जो व‚क्तुमीह‚ते} । ‚{\tiny $_{lb}$}‚(१६७) यो ब्रूयाद‚{\color{DodgerBlue3}‚दृष्ट}‚म‚पि ‚{\color{DodgerBlue3}‚तृणाग्रे क‚रिणां श‚त‚म‚स्तीति}‚।
	\pend% ending standard par
      \textsuperscript{\textenglish{069/s}}

	  \pstart \leavevmode% starting standard par
	\hphantom{.}त‚थाहि य‚दि श‚क्तिर‚भिव्य‚क्तिरूपाद् ‚{\color{DodgerBlue3}‚विज्ञानाद‚न्या त‚दाऽविज्ञान‚त्वं सिद्धं} । ‚{\tiny $_{lb}$}‚अथान्य‚था त‚दा (।)
	\pend% ending standard par
      
	  \bigskip
	  \begingroup
	
	    \large
	  
	    \begin{quote}
	  
	    
	    \stanza[\smallbreak]
	\label{pv.1.168b}\flagstanza{\tiny\textenglish{...1.168b}}य‚द् रूपं दृश्य‚तां यातं त‚द् रूपं प्राङ् न दृश्य‚ते ॥ १६८ ॥\&[\smallbreak]


	
	    \end{quote}
	  
	  \endgroup
	
	  \bigskip
	  \begingroup
	
	    \large
	  
	    \begin{quote}
	  
	    
	    \stanza[\smallbreak]
	\label{pv.1.169a}\flagstanza{\tiny\textenglish{...1.169a}}श‚त‚धा विप्र‚कीर्णेऽपि हेतौ त‚द् विद्य‚ते क‚थ‚म् ।\&[\smallbreak]


	
	    \end{quote}
	  
	  \endgroup
	

	  \pstart \leavevmode% starting standard par
	य‚द्रूपं ख्या‚{\tiny $_{2}$}‚नाख्यं ‚{\color{DodgerBlue3}‚दृश्य‚तां यात}‚म‚भिव्य‚क्ताव‚स्थायां ‚{\color{DodgerBlue3}‚त‚द्रूपं प्राक्} श‚क्त्य‚व‚स्थायां ‚{\tiny $_{lb}$}‚‚{\color{DodgerBlue3}‚श‚त‚धापि विप्र‚कीर्ण्णे हेतौ न दृश्य‚ते} (१६८) । ‚{\color{DodgerBlue3}‚क‚थ‚न्त‚द्विद्य‚त} इति नाविज्ञान‚त्व‚म‚{\tiny $_{lb}$}‚सिद्धं (।)
	\pend% ending standard par
      

	  \pstart \leavevmode% starting standard par
	किञ्च (।)
	\pend% ending standard par
      
	  \bigskip
	  \begingroup
	
	    \large
	  
	    \begin{quote}
	  
	    
	    \stanza[\smallbreak]
	\label{pv.1.169b}\flagstanza{\tiny\textenglish{...1.169b}}रागाद्य‚निय‚मोऽपूर्व‚प्रादुर्भावे प्र‚स‚ज्य‚ते ॥ १६९ ॥\&[\smallbreak]


	
	    \end{quote}
	  
	  \endgroup
	

	  \pstart \leavevmode% starting standard par
	\hphantom{.}न चेत् प‚र‚लोकादाग‚च्छ‚ति चित्त‚स‚न्तानः त‚दाऽ‚{\color{DodgerBlue3}‚पूर्व्व}‚स‚त्त्व‚{\color{DodgerBlue3}‚प्रादुर्भावे ‚{\tiny $_{lb}$}‚रागाद्य}‚निय‚म‚श्च‚क्षुराद्य‚निय‚म‚व‚त् ‚{\color{DodgerBlue3}‚प्र‚स‚ज्य‚ते} । य‚था च‚क्षुःक‚र‚च‚र‚ण‚श्याम‚तानिय‚मः ‚{\tiny $_{lb}$}‚पुरुषे नेक्ष्य‚ते ‚{\color{DodgerBlue3}‚क‚दाचित् क‚स्य‚चिद‚भाव‚द‚र्श‚नात्} । त‚था रागी रागित‚रो नीराग‚श्च ‚{\tiny $_{lb}$}‚क‚श्चित् स्यात् । (१६९)
	\pend% ending standard par
      \label{div_pvv.1.170_1.171}
	  
	% new div opening: depth here is 2
	

	  \pstart \leavevmode% starting standard par
	(d. भूतात्म‚त‚या स‚मान‚राग‚ताप्र‚स‚ङ्गात् ।
	\pend% ending standard par
      
	  \bigskip
	  \begingroup
	
	    \large
	  
	    \begin{quote}
	  
	    
	    \stanza[\smallbreak]
	\label{pv.1.170}\flagstanza{\tiny\textenglish{....1.170}}भूतात्म‚ताऽन‚तिक्रान्तः स‚र्वो रागादिमान् य‚दि ।&स‚र्वः स‚मान‚रागः स्याद् भूतातिश‚य‚तो न चेत् ॥ १७० ॥\&[\smallbreak]


	
	    \end{quote}
	  
	  \endgroup
	
	  \bigskip
	  \begingroup
	
	    \large
	  
	    \begin{quote}
	  
	    
	    \stanza[\smallbreak]
	\label{pv.1.171}\flagstanza{\tiny\textenglish{....1.171}}भूतानां प्राणिताऽभेदेऽप्य‚यं भेदो य‚दाश्र‚यः ।&त‚न्निर्ह्रासातिश‚य‚व‚त् त‚द्भावात् तानि हाप‚येत् ॥ १७१ ॥\&[\smallbreak]


	
	    \end{quote}
	  
	  \endgroup
	

	  \pstart \leavevmode% starting standard par
	\hphantom{.}‚{\color{DodgerBlue3}‚भूतात्म‚ता}‚या ‚{\color{DodgerBlue3}‚अन‚तिक्रान्तेः स‚र्व्वः} पुमान् ‚{\color{DodgerBlue3}‚रागादि\edtext{}{\edlabel{pvv.69-1}\label{pvv.69-1}\lemma{रागादि}\Bfootnote{आदिना सुखाद‚यः ।}}मान् य‚दि} त‚दा ‚{\color{DodgerBlue3}‚स‚र्व्वंः स‚मान‚{\tiny $_{lb}$}‚रागः स्यात्} । भूतात्म‚ताया अविशेषात् । ‚{\color{DodgerBlue3}‚भूता}‚नाम‚वान्त‚रा‚{\color{DodgerBlue3}‚द‚तिश‚य‚तः स‚मान‚राग‚ता}‚{\tiny $_{lb}$}‚प्र‚स‚ङ्गो ‚{\color{DodgerBlue3}‚न चेत्} (१७०) ‚{\color{DodgerBlue3}‚भूतानां प्राणि}‚ताया अ‚{\color{DodgerBlue3}‚भेदेप्य\edtext{}{\edlabel{pvv.69-2}\label{pvv.69-2}\lemma{भेदेप्य}\Bfootnote{अयं प्राणी प्राणित‚रः प्राणित‚म इत्य‚भेदे ।}}यं भेद} उत्क‚ट‚म‚न्द‚राग‚त्वादिको ‚{\tiny $_{lb}$}‚‚{\color{DodgerBlue3}‚य‚दाश्र‚यो} य‚द‚वान्त‚र‚विशेष‚व‚त्\edtext{}{\edlabel{pvv.69-3}\label{pvv.69-3}\lemma{त्}\Bfootnote{भूतातिश‚यं ।}} कार‚ण‚माश्रित्य । त‚त्कार‚णं ‚{\color{DodgerBlue3}‚निर्ह्रासातिश‚य‚व‚त्} अप‚च\edtext{}{\edlabel{pvv.69-4}\label{pvv.69-4}\lemma{च}\Bfootnote{यो य‚त्रोत्क‚र्ष‚वान् स त‚त्र स‚म्भ‚व‚दुच्छेद‚ध‚र्म्माद्य‚निब‚न्ध‚नं ‚{\tiny $_{lb}$}‚रागादि निव‚र्त‚येत् त‚त्र शुक्ल‚त्वादिहेतुव‚त् ।}}य‚तार‚त‚म्य‚व‚त् ‚{\color{DodgerBlue3}‚त‚द्भावाद्रागादिम‚त्त्वात् तानि} भूतानि ‚{\color{DodgerBlue3}‚हाप‚येत्} भ्रंश‚येदिति ‚{\tiny $_{lb}$}‚नीरा‚{\tiny $_{4}$}‚गोपि क‚श्चित् स‚त्त्वः स्यात् । (१७१)
	\pend% ending standard par
      \label{div_pvv.1.172}
	  
	% new div opening: depth here is 2
	\textsuperscript{\textenglish{070/s}}
	  \bigskip
	  \begingroup
	
	    \large
	  
	    \begin{quote}
	  
	    
	    \stanza[\smallbreak]
	\label{pv.1.172}\flagstanza{\tiny\textenglish{....1.172}}न चेद् भेदेऽपि रागादिहेतुतुल्यात्म‚ताक्ष‚यः ।&स‚र्व‚त्र रागः स‚दृशः स्याद्धेतोस्स‚दृशात्म‚नः ॥ १७२ ॥\&[\smallbreak]


	
	    \end{quote}
	  
	  \endgroup
	

	  \pstart \leavevmode% starting standard par
	\hphantom{.}‚{\color{DodgerBlue3}‚भेदे म‚न्दो}‚त्क‚ट‚रागादिज‚न‚केऽवान्त‚र‚भूत‚विशेषेपि ‚{\color{DodgerBlue3}‚रागादि}‚हेतुत्वेन या ‚{\tiny $_{lb}$}‚‚{\color{DodgerBlue3}‚तुल्यात्म\edtext{}{\edlabel{pvv.70-1}\label{pvv.70-1}\lemma{तुल्यात्म}\Bfootnote{रागादिहेतुः सामान्यं, त‚स्य य‚दि क्ष‚यः स्यात्त‚दा विरागः स्यात्, न ‚{\tiny $_{lb}$}‚चास्ति क्ष‚यः (।) य‚तोऽक्ष‚य‚स्तेन वीत‚रागो न चेत् त‚दा ।}}ता} स‚दृश‚ता त‚स्याः ‚{\color{DodgerBlue3}‚क्ष‚यो न चेत् स‚र्व्व‚त्र} पुंसि ‚{\color{DodgerBlue3}‚स‚दृशो रागादिः स्यात्} । ‚{\tiny $_{lb}$}‚‚{\color{DodgerBlue3}‚स‚दृशात्म‚नो हेतो}‚र्भावात् । (१७२)
	\pend% ending standard par
      \label{div_pvv.1.173}
	  
	% new div opening: depth here is 2
	
	  \bigskip
	  \begingroup
	
	    \large
	  
	    \begin{quote}
	  
	    
	    \stanza[\smallbreak]
	\label{pv.1.173}\flagstanza{\tiny\textenglish{....1.173}}न हि गोप्र‚त्य‚य‚स्यास्ति स‚मानात्म‚भुवः क्व‚चित् ।&तार‚त‚म्यं पृथिव्यादौ प्राणितादेरिहापि वा ॥ १७३ ॥\&[\smallbreak]


	
	    \end{quote}
	  
	  \endgroup
	

	  \pstart \leavevmode% starting standard par
	\hphantom{.}‚{\color{DodgerBlue3}‚न\edtext{}{\edlabel{pvv.70-2}\label{pvv.70-2}\lemma{न}\Bfootnote{तुल्य‚हेतुकं विशिष्य‚त इति दृष्टान्त‚माह}} हि गोप्र‚त्य‚य‚स्यास्ति स‚मानात्म‚भुवः} स‚दृश‚कार‚णोत्प‚त्तेः । ‚{\color{DodgerBlue3}‚क्व‚चित्} शाब‚ले‚{\tiny $_{lb}$}‚यादौ ‚{\color{DodgerBlue3}‚तार\edtext{}{\edlabel{pvv.70-3}\label{pvv.70-3}\lemma{तार}\Bfootnote{गोत‚रो गोत‚म इति ।}} त‚म्य}‚म‚पि तु स‚र्व्व‚त्र स‚मान‚तैव । ‚{\color{DodgerBlue3}‚पृथिव्यादौ} स‚दृशे हेतौ ‚{\color{DodgerBlue3}‚प्राणितादेरिह} चा र्व्वा क म‚ते‚{\tiny $_{5}$}‚‚{\color{DodgerBlue3}‚पि वा} विशेषो न विद्य‚ते\edtext{}{\edlabel{pvv.70-4}\label{pvv.70-4}\lemma{ते}\Bfootnote{य‚था न प्राणी प्राणित‚र इति त‚था रागी रागित‚र इत्य‚पि न युक्तं तुल्य‚हेतुस‚म्भ‚वात् ।}}। (१७३)
	\pend% ending standard par
      \label{div_pvv.1.174}
	  
	% new div opening: depth here is 2
	
	  \bigskip
	  \begingroup
	
	    \large
	  
	    \begin{quote}
	  
	    
	    \stanza[\smallbreak]
	\label{pv.1.174}\flagstanza{\tiny\textenglish{....1.174}}औष्ण‚य‚स्य तार‚त‚म्येऽपि नानुष्णोऽग्निः क‚दाच‚न ।&त‚थेहापीति चेन्नाग्नेरौष्ण्याद् भेद‚निषेध‚तः ॥ १७४ ॥\&[\smallbreak]


	
	    \end{quote}
	  
	  \endgroup
	

	  \pstart \leavevmode% starting standard par
	\hphantom{.}‚{\color{DodgerBlue3}‚औष्ण्य‚स्य तार‚त‚म्येपि} खादिराग्न्यादौ\edtext{}{\edlabel{pvv.70-5}\label{pvv.70-5}\lemma{खादिराग्न्यादौ}\Bfootnote{सांख्यादिराह ।}} ‚{\color{DodgerBlue3}‚नानुष्णोऽग्निः क‚दाच‚न} । ‚{\color{DodgerBlue3}‚त‚थेह} रागादि‚{\tiny $_{lb}$}‚‚{\color{DodgerBlue3}‚तार‚त‚म्येपि} न वीत‚रागः क‚श्चि‚{\color{DodgerBlue3}‚दिति} ‚{\color{DodgerBlue3}‚चेत्} क‚दाच‚न नैत‚द्युक्तं । ‚{\color{DodgerBlue3}‚अग्नेरौष्ण्याद् भेद‚स्य ‚{\tiny $_{lb}$}‚निषेध‚तः} । नैत‚द्युक्तं । भास्व‚र‚रूपोष्ण‚स्प‚र्शादिर‚ग्निरुच्य‚ते । तेनौष्ण्याभावेऽग्निरेव ‚{\tiny $_{lb}$}‚न स्यात् । रागादिस्तु भूतेभ्योऽन्य‚स्त‚द‚भावेपि तेषां भावात् । (१७४)
	\pend% ending standard par
      \label{div_pvv.1.175}
	  
	% new div opening: depth here is 2
	

	  \pstart \leavevmode% starting standard par
	अतः स‚विशे\edtext{}{\edlabel{pvv.70-6}\label{pvv.70-6}\lemma{विशे}\Bfootnote{ध‚र्मिणो गुणा इति विशेष‚णं न चौष्ण्य‚स्याग्निर्ध‚र्मी ।}}ष‚णं हेतुन्द‚र्श‚य‚ति ।
	\pend% ending standard par
      
	  \bigskip
	  \begingroup
	
	    \large
	  
	    \begin{quote}
	  
	    
	    \stanza[\smallbreak]
	\label{pv.1.175}\flagstanza{\tiny\textenglish{....1.175}}तार‚त‚म्यानुभ‚विनो य‚स्यान्य‚स्य स‚तो गुणाः ।&ते क्व‚चित् प्र‚तिह‚न्य‚न्ते त‚द्भेदे ध‚व‚लादिव‚त् ॥ १७५ ॥\&[\smallbreak]


	
	    \end{quote}
	  
	  \endgroup
	

	  \pstart \leavevmode% starting standard par
	\hphantom{.}‚{\color{DodgerBlue3}‚गुणेभ्योऽन्य‚स्य} य‚स्य ‚{\color{DodgerBlue3}‚स‚तो} ध‚र्मिणो ये ‚{\color{DodgerBlue3}‚तार‚त‚{\tiny $_{6}$}‚म्यानुभाविनो गुणास्ते क्व‚चिद्} ध‚र्मिणि ‚{\color{DodgerBlue3}‚प्र‚तिह\edtext{}{\edlabel{pvv.70-7}\label{pvv.70-7}\lemma{तिह}\Bfootnote{उच्छिद्य‚न्ते ।}}न्य‚न्ते} । ‚{\color{DodgerBlue3}‚त‚द्भेदे} भूत‚भेदे ‚{\color{DodgerBlue3}‚ध‚व‚लादिव‚त्} ।\edtext{\textsuperscript{*}}{\edlabel{pvv.70-8}\label{pvv.70-8}\lemma{*}\Bfootnote{कृष्ट‚स्य ज‚न्तोर्दृ ष्टेः ।}}न हि स‚र्व्वो भूत‚प‚रिणामः ‚{\tiny $_{lb}$}‚शुक्लः । (१७५)
	\pend% ending standard par
      \label{div_pvv.1.176}
	  
	% new div opening: depth here is 2
	\textsuperscript{\textenglish{071/s}}

	  \begin{center}%% label @type='head'
	\textbf{(f. न रूप‚व‚द् रागोऽपि भूत‚ध‚र्मः)}
	\end{center}
	

	  \pstart \leavevmode% starting standard par
	स्यादेत‚द् (।) भूत‚ध‚र्मो रूपादिर्य‚थाव‚श्यं अभूत्वा भ‚व‚ति त‚था रागोपि देहिनः ‚{\tiny $_{lb}$}‚स्यादित्याह (।)
	\pend% ending standard par
      
	  \bigskip
	  \begingroup
	
	    \large
	  
	    \begin{quote}
	  
	    
	    \stanza[\smallbreak]
	\label{pv.1.176}\flagstanza{\tiny\textenglish{....1.176}}रूपादिव‚न्न निय‚म‚स्तेषां भूताविभाग‚तः ॥&त‚त् तुल्य‚ञ्चेन्न रागादेः स‚होत्प‚त्तिप्र‚स‚ङ्ग‚तः ॥ १७६ ॥\&[\smallbreak]


	
	    \end{quote}
	  
	  \endgroup
	

	  \pstart \leavevmode% starting standard par
	\hphantom{.}‚{\color{DodgerBlue3}‚रूपा\edtext{}{\edlabel{pvv.71-1}\label{pvv.71-1}\lemma{रूपा}\Bfootnote{सामान्य‚म‚त्र विशेषे व्य‚भिचारात् ।}}\edtext{}{\edlabel{pvv.71-1a}\label{pvv.71-1a}\lemma{रूपा}\Bfootnote{1a स‚ह‚वृत्तिनिय‚मेनेत्य‚र्थः ।\begin{english} --- Placement of note uncertain; marked with a question mark in the edition (see encoding description for details).\end{english}}} दिव‚न्न निय‚मो} रागादे‚{\color{DodgerBlue3}‚स्तेषां} रूपादीनां ‚{\color{DodgerBlue3}‚भूताविभाग‚तः} । न हि रूपादि‚{\tiny $_{lb}$}‚सामान्यं विना भूता\edtext{}{\edlabel{pvv.71-2}\label{pvv.71-2}\lemma{भूता}\Bfootnote{सामान्यं क‚र्तृ ।}} नि व‚र्त‚ते । रागादेर‚पि त‚द्भूताविनिर्भाग‚व‚र्त्तित्वं ‚{\color{DodgerBlue3}‚तुल्य‚ञ्चेत्} । ‚{\tiny $_{lb}$}‚नैत‚द‚स्ति । ‚{\color{DodgerBlue3}‚रागा\edtext{}{\edlabel{pvv.71-3}\label{pvv.71-3}\lemma{रागा}\Bfootnote{द्वेषादेर्युग‚प‚त् रागादिहेतुः ।}}} देर्भूतैः ‚{\color{DodgerBlue3}‚स‚होत्प‚त्तिप्र‚स‚ङ्ग‚{\tiny $_{7}$}‚तो} ॥ विष‚यैः कादाचित्क‚स‚न्निधानै‚{\tiny $_{lb}$}‚र्निय‚मित‚त्वात् । (१७६)
	\pend% ending standard par
      \label{div_pvv.1.177}
	  
	% new div opening: depth here is 2
	

	  \pstart \leavevmode% starting standard par
	न स‚होत्पाद‚प्र‚स‚ङ्ग इति चेदाह (।)
	\pend% ending standard par
      
	  \bigskip
	  \begingroup
	
	    \large
	  
	    \begin{quote}
	  
	    
	    \stanza[\smallbreak]
	\label{pv.1.177}\flagstanza{\tiny\textenglish{....1.177}}विक‚ल्प्य‚विष‚य‚त्वाच्च विष‚या न नियाम‚काः ।&स‚भाग‚हेतुविर‚हाद् रागादेर्निय‚मो न वा ॥ १७७ ॥\&[\smallbreak]


	
	    \end{quote}
	  
	  \endgroup
	

	  \pstart \leavevmode% starting standard par
	\hphantom{.}विक‚ल्प्यः क‚ल्पित‚स्त‚{\color{DodgerBlue3}‚द्विष‚य‚त्वाच्च} रागादे‚{\color{DodgerBlue3}‚र्व्विष‚या} रूपाद‚योऽग्राह्यात्वान्न ‚{\color{DodgerBlue3}‚निया‚{\tiny $_{lb}$}‚म‚काः} । किञ्चाविज्ञान‚स्य विज्ञान‚हेतुत्वात् नेत्यु\edtext{}{\edlabel{pvv.71-4}\label{pvv.71-4}\lemma{नेत्यु}\Bfootnote{अविज्ञानं न विज्ञान‚हेतुरित्युक्तं प‚रं प्र‚ति । इदानीम‚पि प‚रं प्र‚त्याह ।}}क्तं । त्व‚न्म‚ते ‚{\color{DodgerBlue3}‚स‚भाग‚स्य} च\edtext{}{\edlabel{pvv.71-5}\label{pvv.71-5}\lemma{च}\Bfootnote{वा स‚मुच्च‚यार्थः ।}} ‚{\color{DodgerBlue3}‚हेतो‚{\tiny $_{lb}$}‚र्विर‚हा\edtext{}{\edlabel{pvv.71-6}\label{pvv.71-6}\lemma{हा}\Bfootnote{य‚द्य‚पि तृणाद‚यो ह‚स्त्यात्म‚ना प‚रिण‚तास्त‚त्रापि तेषां न त‚च्छ‚क्तिर‚स्ति ।}}त् रागादेर्निय‚मो} देश‚काल‚स्व‚भाव‚विष‚यो ‚{\color{DodgerBlue3}‚न वा} स्याद‚हेतुत्वात् । (१७७)
	\pend% ending standard par
      \label{div_pvv.1.178_1.179}
	  
	% new div opening: depth here is 2
	

	  \begin{center}%% label @type='head'
	\textbf{(f. न भूतान्येद हेतुः)}
	\end{center}
	‚{\tiny $_{lb}$}‚

	  \pstart \leavevmode% starting standard par
	भूतान्येव हि हेतुरिति चेत् ।
	\pend% ending standard par
      
	  \bigskip
	  \begingroup
	
	    \large
	  
	    \begin{quote}
	  
	    
	    \stanza[\smallbreak]
	\label{pv.1.178a}\flagstanza{\tiny\textenglish{...1.178a}}स‚र्व‚दा स‚र्व‚बुद्धीनां ज‚न्म वा हेतुस‚न्निधेः ।\leavevmode\ledsidenote{\textenglish{14a/MA}}\&[\smallbreak]


	
	    \end{quote}
	  
	  \endgroup
	

	  \pstart \leavevmode% starting standard par
	\hphantom{.}त‚र्हि ‚{\color{DodgerBlue3}‚स‚र्व्व‚दा स‚र्व्व‚बुद्धीनां} सुख‚दुःखेच्छाद्वेष‚राग‚{\tiny $_{8}$}‚कृपादीनां ‚{\color{DodgerBlue3}‚ज‚न्म वा स्यात् । ‚{\tiny $_{lb}$}‚हेतो}‚र्भूत‚संघात‚स्यं ‚{\color{DodgerBlue3}‚स‚न्निधेः} । त‚देवं रा\edtext{}{\edlabel{pvv.71-7}\label{pvv.71-7}\lemma{रा}\Bfootnote{दुःख‚स‚त्य‚व्याख्यार‚म्भे ।}}गादेः पाट‚वेक्ष‚णादित्यादिना च‚तुर्ण्णाम‚रूपिणां ‚{\tiny $_{lb}$}‚स्क‚न्धानां स‚भाग‚हेतुक‚त्वे साधिते साधितं संसारित्व‚मुपादान‚स्क‚न्धानां । त एव दुःख‚{\tiny $_{lb}$}‚मित्युक्ताः ।
	\pend% ending standard par
      \textsuperscript{\textenglish{072/s}}

	  \begin{center}%% label @type='head'
	\textbf{IV. च‚तुराक‚रं दुःख‚स‚त्य‚म्}
	\end{center}
	

	  \pstart \leavevmode% starting standard par
	दुःख‚स‚त्य‚ञ्चानित्य‚तो दुःख‚तः शून्य‚तोऽनात्म‚त‚श्चेति च‚तुराकार‚माख्यातुमाह (।)
	\pend% ending standard par
      
	  \bigskip
	  \begingroup
	
	    \large
	  
	    \begin{quote}
	  
	    
	    \stanza[\smallbreak]
	\label{pv.1.178b}\flagstanza{\tiny\textenglish{...1.178b}}क‚दाचिदुप‚ल‚म्भात् त‚द‚ध्रुवं दोष‚निश्र‚यात् ॥ १७८ ॥\&[\smallbreak]


	
	    \end{quote}
	  
	  \endgroup
	
	  \bigskip
	  \begingroup
	
	    \large
	  
	    \begin{quote}
	  
	    
	    \stanza[\smallbreak]
	\label{pv.1.179a}\flagstanza{\tiny\textenglish{...1.179a}}दुःखं हेतुव‚श‚त्वाच्च न चात्मा नाप्य‚धिष्ठित‚म् ।\&[\smallbreak]


	
	    \end{quote}
	  
	  \endgroup
	

	  \pstart \leavevmode% starting standard par
	\hphantom{.}‚{\color{DodgerBlue3}‚क‚दाचिदुप‚ल‚म्भाद्} दुःख‚म‚{\color{DodgerBlue3}‚ध्रुव‚म}‚नित्यं । ‚{\color{DodgerBlue3}‚दोष‚निश्र‚या\edtext{}{\edlabel{pvv.72-1}\label{pvv.72-1}\lemma{या}\Bfootnote{सास्र‚व‚त्वात् उत्प‚त्त्या क्लिश्य‚न्ति विपाकेन च ।}}त्} रागादिदोषाश्र‚येणोत्प‚{\tiny $_{lb}$}‚त्तेः । (१७८) ‚{\color{DodgerBlue3}‚हेतुव‚श‚त्वाच्च} । स‚र्व्वं प‚र‚व‚शं‚{\tiny $_{1}$}‚ ‚{\color{DodgerBlue3}‚दुःख}‚मिति न्यायात् दुःखं त‚त् । ‚{\color{DodgerBlue3}‚न ‚{\tiny $_{lb}$}‚चात्मा}‚श्र‚यं । अना\edtext{}{\edlabel{pvv.72-2}\label{pvv.72-2}\lemma{अना}\Bfootnote{हेतुव‚श‚त्वाच्च ।}}त्म‚नः आत्म‚विल‚क्ष‚ण‚त्वात् । ‚{\color{DodgerBlue3}‚ना\edtext{}{\edlabel{pvv.72-3}\label{pvv.72-3}\lemma{ना}\Bfootnote{नात्मीयं}}प्य‚धिष्ठितं} । अधिष्ठातुरात्म\edtext{}{\edlabel{pvv.72-4}\label{pvv.72-4}\lemma{अधिष्ठातुरात्म}\Bfootnote{स‚भाग‚तो जात‚त्वात् ।}}नो‚{\tiny $_{lb}$}‚ऽभावात् । अनेन शून्य‚त इत्याख्यातं ।
	\pend% ending standard par
      

	  \pstart \leavevmode% starting standard par
	क‚स्मात्पुन‚रात्मा नाधिष्ठातेत्याह---
	\pend% ending standard par
      
	  \bigskip
	  \begingroup
	
	    \large
	  
	    \begin{quote}
	  
	    
	    \stanza[\smallbreak]
	\label{pv.1.179b}\flagstanza{\tiny\textenglish{...1.179b}}नाकार‚ण‚म‚धिष्ठाता नित्यं वा कार‚णं क‚थ‚म् ॥ १७९॥\&[\smallbreak]


	
	    \end{quote}
	  
	  \endgroup
	

	  \pstart \leavevmode% starting standard par
	\hphantom{.}‚{\color{DodgerBlue3}‚नाकार‚ण‚म‚धिष्ठाता}‚ऽतिप्र‚स‚ङ्गात् । ‚{\color{DodgerBlue3}‚नित्यं वा} द्र‚व्यं ‚{\color{DodgerBlue3}‚कार‚णं क‚थं} (।) त‚स्य क्र‚म‚{\tiny $_{lb}$}‚यौग‚प‚द्याभ्याम‚र्थ‚क्रियाविर‚हात् । (१७९)
	\pend% ending standard par
      \label{div_pvv.1.180_1.181_1.182_1.183_1.184}
	  
	% new div opening: depth here is 2
	
	  \bigskip
	  \begingroup
	
	    \large
	  
	    \begin{quote}
	  
	    
	    \stanza[\smallbreak]
	\label{pv.1.180a}\flagstanza{\tiny\textenglish{...1.180a}}त‚स्माद‚नेक‚मेक‚स्माद् भिन्न‚कालं न जाय‚ते ।\&[\smallbreak]


	
	    \end{quote}
	  
	  \endgroup
	

	  \pstart \leavevmode% starting standard par
	\hphantom{.}‚{\color{DodgerBlue3}‚त‚स्माद‚नेकं भिन्न‚काल}‚दृश्य‚मानं सुख‚दुःखादिकार्यं ‚{\color{DodgerBlue3}‚नैक‚स्माज्जाय‚ते} एक‚स्या‚{\tiny $_{lb}$}‚नेक‚क‚र‚णे स‚म‚र्थ‚स्य स‚कृदेव त‚त्क्रियाप्र‚स‚ङ्गात् ।
	\pend% ending standard par
      
	  \bigskip
	  \begingroup
	
	    \large
	  
	    \begin{quote}
	  
	    
	    \stanza[\smallbreak]
	\label{pv.1.180b}\flagstanza{\tiny\textenglish{...1.180b}}कार्यानुत्पाद‚तोऽन्येषु स‚ङ्ग‚तेष्व‚पि हेतुषु ॥ १८० ॥\&[\smallbreak]


	
	    \end{quote}
	  
	  \endgroup
	
	  \bigskip
	  \begingroup
	
	    \large
	  
	    \begin{quote}
	  
	    
	    \stanza[\smallbreak]
	\label{pv.1.181a}\flagstanza{\tiny\textenglish{...1.181a}}हेत्व‚न्त‚रानुमानं स्यान्नैत‚न् नित्येषु विद्य‚ते ।\&[\smallbreak]


	
	    \end{quote}
	  
	  \endgroup
	

	  \pstart \leavevmode% starting standard par
	किञ्चा‚{\tiny $_{2}$}‚नेकेषु ‚{\color{DodgerBlue3}‚हेतुषु} स‚ङ्ग‚{\color{DodgerBlue3}‚तेषु} मिथो मिलितेष्व‚पि ‚{\color{DodgerBlue3}‚कार्यानुत्पाद}‚तो (१८०) ‚{\tiny $_{lb}$}‚‚{\color{DodgerBlue3}‚हेत्व‚न्त‚रानुमानं स्यात्} (।) य‚था रूपालोक‚म‚न‚स्कारेषु स‚त्स्व‚पि च‚क्षुर्व्विज्ञान‚म‚नुत्प‚द्य‚{\tiny $_{lb}$}‚मानं च‚क्षुर‚नुमाप‚य‚ति । ‚{\color{DodgerBlue3}‚नैत‚त्का}‚र्यानुत्प‚त्त्याऽनुमानं ‚{\color{DodgerBlue3}‚नित्येषु विद्य‚ते} तेषाम‚व्य‚ति\edtext{}{\edlabel{pvv.72-5}\label{pvv.72-5}\lemma{ति}\Bfootnote{त‚द्व्य‚तिरेकेपि कार्योत्प‚त्तेः ।}}रेकि‚{\tiny $_{lb}$}‚त्वात् (।)
	\pend% ending standard par
      

	  \begin{center}%% label @type='head'
	\textbf{(ख) स‚मुद‚य‚स‚त्य‚म्}
	\end{center}
	

	  \begin{center}%% label @type='head'
	\textbf{I. च‚तुराकारः स‚मुद‚यः}
	\end{center}
	

	  \pstart \leavevmode% starting standard par
	च‚तुराकारं दुःखं व्याख्याय स‚मुद‚य‚तो हेतुतः प्र‚त्य‚य‚तः प्र‚भ‚व‚त‚श्चेति च‚तुराकारं ‚{\tiny $_{lb}$}‚‚{\color{DodgerBlue3}‚स‚मुदायं व्याख्यातुमाह} (।)
	\pend% ending standard par
      \textsuperscript{\textenglish{073/s}}
	  \bigskip
	  \begingroup
	
	    \large
	  
	    \begin{quote}
	  
	    
	    \stanza[\smallbreak]
	\label{pv.1.181b}\flagstanza{\tiny\textenglish{...1.181b}}कादाचित्क‚त‚या सिद्धा दुःख‚स्यास्य स‚हेतुता ॥ १८१ ॥\&[\smallbreak]


	
	    \end{quote}
	  
	  \endgroup
	
	  \bigskip
	  \begingroup
	
	    \large
	  
	    \begin{quote}
	  
	    
	    \stanza[\smallbreak]
	\label{pv.1.182a}\flagstanza{\tiny\textenglish{...1.182a}}नित्यं स‚त्व‚म‚स‚त्वं वाऽहेतोर्बाह्यान‚पेक्ष‚णात् ॥\&[\smallbreak]


	
	    \end{quote}
	  
	  \endgroup
	

	  \pstart \leavevmode% starting standard par
	\hphantom{.}a. ‚{\color{DodgerBlue3}‚कादाचित्क‚त‚या दुःख‚स्य स‚हेतुता सिद्धा} (१८१)। ‚{\color{DodgerBlue3}‚नित्यं स‚त्त्व‚म‚स‚त्त्वं‚{\tiny $_{3}$}‚म्वा ‚{\tiny $_{lb}$}‚हेतो}‚र्भ‚व‚ति य‚थाऽऽकाश‚स्य श‚श‚विषाण‚स्य बाह्या‚{\color{DodgerBlue3}‚न‚पेक्ष‚णात्} । य एव च दुःख‚हेतुः स ‚{\tiny $_{lb}$}‚एव स‚मुदायः ।
	\pend% ending standard par
      
	  \bigskip
	  \begingroup
	
	    \large
	  
	    \begin{quote}
	  
	    
	    \stanza[\smallbreak]
	\label{pv.1.182b}\flagstanza{\tiny\textenglish{...1.182b}}तैक्ष्ण्यादीनां य‚था नास्ति कार‚णं क‚ण्ट‚कादिषु ॥ १८२ ॥\&[\smallbreak]


	
	    \end{quote}
	  
	  \endgroup
	
	  \bigskip
	  \begingroup
	
	    \large
	  
	    \begin{quote}
	  
	    
	    \stanza[\smallbreak]
	\label{pv.1.183a}\flagstanza{\tiny\textenglish{...1.183a}}त‚थाऽकार‚ण‚मेत‚त् स्यादिति केचित् प्र‚च‚क्ष‚ते ।\&[\smallbreak]


	
	    \end{quote}
	  
	  \endgroup
	

	  \pstart \leavevmode% starting standard par
	न\edtext{}{\edlabel{pvv.73-1}\label{pvv.73-1}\lemma{न}\Bfootnote{कः प‚द्म‚नाल‚द‚ल‚केश‚र‚क‚र्ण्णिकानां संस्थान‚व‚र्ण्ण‚र‚च‚नामृदुतादिहेतुः (।) ‚{\tiny $_{lb}$}‚प‚त्राणि चित्र‚य‚ति कोत्र प‚त‚त्रिणाम्वा स्वाभाविकं ज‚ग‚दिदं निय‚तं त‚थैव ॥ ‚{\tiny $_{lb}$}‚दुःख‚म‚हेतुकं चार्व्वाकः ।}}नु ‚{\color{DodgerBlue3}‚य‚था क‚ण्ट‚कादिषु तैक्ष्ण्यादीनां कार‚णं नास्ति} (१८२) ‚{\color{DodgerBlue3}‚त‚थाऽकार‚ण‚मेत‚त्} दुःखं स्यात् । त‚त् कुतः स‚मुद‚य ‚{\color{DodgerBlue3}‚इति केचित्} स्व‚भाव‚वादिनः ‚{\color{DodgerBlue3}‚प्र‚च‚क्ष‚ते} (।)
	\pend% ending standard par
      

	  \pstart \leavevmode% starting standard par
	b. ते एवं व‚क्त‚व्या (:।)
	\pend% ending standard par
      
	  \bigskip
	  \begingroup
	
	    \large
	  
	    \begin{quote}
	  
	    
	    \stanza[\smallbreak]
	\label{pv.1.183b}\flagstanza{\tiny\textenglish{...1.183b}}स‚त्येव य‚स्मिन् य‚ज्ज‚न्म विकारे वाऽपि विक्रिया ॥ १८३ ॥\&[\smallbreak]


	
	    \end{quote}
	  
	  \endgroup
	
	  \bigskip
	  \begingroup
	
	    \large
	  
	    \begin{quote}
	  
	    
	    \stanza[\smallbreak]
	\label{pv.1.184a}\flagstanza{\tiny\textenglish{...1.184a}}त‚त् त‚स्य कार‚णं प्राहुस्त‚त् तेषाम‚पि विद्य‚ते ।\&[\smallbreak]


	
	    \end{quote}
	  
	  \endgroup
	

	  \pstart \leavevmode% starting standard par
	\hphantom{.}‚{\color{DodgerBlue3}‚स‚त्येव य‚स्मिन्} व‚स्तुनि ‚{\color{DodgerBlue3}‚य‚स्य ज‚न्म} । य‚स्य ‚{\color{DodgerBlue3}‚विकारे} स‚त्येव वा य‚स्य ‚{\color{DodgerBlue3}‚विक्रिया} (१८३) ‚{\color{DodgerBlue3}‚त‚त्त‚स्य} ज‚न्मिनो विकारिण‚श्च ‚{\color{DodgerBlue3}‚कार‚णं प्राहु}‚र्व्विद्वांसः । त‚ज्ज‚न्म स‚त्येव ‚{\tiny $_{lb}$}‚बीजोद‚क‚पृथिव्यादिषु‚{\tiny $_{4}$}‚ त‚दुत्क‚र्षाप‚क‚र्षादिविकारे च विकृत‚त्वं ‚{\color{DodgerBlue3}‚तेषां} क‚ण्ट‚कादीनाम‚{\tiny $_{lb}$}‚प्य‚स्तीति तेपि स‚हेतुका एव । एवं स्क‚न्धा अपि ॥
	\pend% ending standard par
      

	  \pstart \leavevmode% starting standard par
	न१नु स्प‚र्शे स‚ति भ‚व‚ति च‚क्षुर्व्विज्ञानं । अस‚ति च न भ‚व‚ति\edtext{}{\edlabel{pvv.73-2}\label{pvv.73-2}\lemma{ति}\Bfootnote{स्प‚र्श‚व‚त्येवं संह‚ते च‚क्षुर्व्विज्ञानं म‚न्य‚ते ।}} (।) न च त‚त् ‚{\tiny $_{lb}$}‚कार‚ण‚म‚तोऽतिव्याप्तिरित्याह (।)
	\pend% ending standard par
      
	  \bigskip
	  \begingroup
	
	    \large
	  
	    \begin{quote}
	  
	    
	    \stanza[\smallbreak]
	\label{pv.1.184b}\flagstanza{\tiny\textenglish{...1.184b}}स्प‚र्श‚स्य रूप‚हेतुत्वाद् द‚र्श‚नेऽस्ति निमित्त‚ता ॥ १८४ ॥\&[\smallbreak]


	
	    \end{quote}
	  
	  \endgroup
	

	  \pstart \leavevmode% starting standard par
	\hphantom{.}‚{\color{DodgerBlue3}‚स्प‚र्श‚स्य} रूपाद्य‚विनिर्भा\edtext{}{\edlabel{pvv.73-3}\label{pvv.73-3}\lemma{विनिर्भा}\Bfootnote{स्प‚र्शीभूत‚च‚तुष्कात्मा उपादाय रूप‚स्य हेतुः ।}}गिनः स‚ह‚कारिभावेन ‚{\color{DodgerBlue3}‚रूप‚हेतुत्वात् । द‚र्श‚ने} च‚क्षु‚{\tiny $_{lb}$}‚र्व्विज्ञानेऽ‚{\color{DodgerBlue3}‚स्ति निमित्त‚ता} पार‚म्प‚र्येणेति नातिव्याप्तिः । (१८४)
	\pend% ending standard par
      \label{div_pvv.1.185_1.186}
	  
	% new div opening: depth here is 2
	

	  \pstart \leavevmode% starting standard par
	एत‚च्च व्य‚तिरे\edtext{}{\edlabel{pvv.73-4}\label{pvv.73-4}\lemma{तिरे}\Bfootnote{अतिव्याप्ति ।}}क‚म‚भ्युप‚ग‚म्योक्तं न तु रूप‚मुप‚द‚र्श्य स्प‚र्शाभावे नेत्र‚बुद्धेर‚भावः ‚{\tiny $_{lb}$}‚श‚क्य‚{\tiny $_{5}$}‚द‚र्श‚नः । रूप‚स्प‚र्श‚योर‚विनिर्भाग‚व‚र्त्तित्वात् । त‚स्य च दुःख‚स्य (।)
	\pend% ending standard par
      \textsuperscript{\textenglish{074/s}}
	  \bigskip
	  \begingroup
	
	    \large
	  
	    \begin{quote}
	  
	    
	    \stanza[\smallbreak]
	\label{pv.1.185a}\flagstanza{\tiny\textenglish{...1.185a}}नित्यानां प्र‚तिषेधेन नेश्व‚रादेश्च स‚म्भ‚वः ।&असाम‚र्थ्याद‚तो हेतुर्भ‚व‚वाञ्छा;\&[\smallbreak]


	
	    \end{quote}
	  
	  \endgroup
	

	  \pstart \leavevmode% starting standard par
	\hphantom{.}‚{\color{DodgerBlue3}‚नित्यानां} क्र‚माक्र‚माभ्याम‚र्थ‚क्रियायाम‚साम‚र्थ्यात् । ‚{\color{DodgerBlue3}‚प्र‚तिषेधेन}\edtext{\textsuperscript{*}}{\edlabel{pvv.74-1}\label{pvv.74-1}\lemma{*}\Bfootnote{कार‚ण‚म्विकृतिङ्ग‚च्छ‚ज्जाय‚तेन्य‚स्य कार‚ण‚मित्यादिना ।}} च ‚{\color{DodgerBlue3}‚नेश्व‚रादे}‚रा‚{\tiny $_{lb}$}‚दिग्र‚ह‚णात् प्र‚धान‚पुरुषादेः कार‚णात्स‚म्भ‚व उत्पादः । अतो नित्याद‚नुत्प‚त्तेर्दुःख‚स्य ‚{\tiny $_{lb}$}‚हेतु‚{\color{DodgerBlue3}‚र्भ‚व‚वाञ्छा} ज‚न्म‚तृष्णा, ज‚न्म‚स्थानाव‚स्था\edtext{}{\edlabel{pvv.74-2}\label{pvv.74-2}\lemma{स्था}\Bfootnote{अव‚स्था म‚नुष्य‚देवादि ।}} स‚त्त्वाद्य‚भिला\edtext{}{\edlabel{pvv.74-3}\label{pvv.74-3}\lemma{भिला}\Bfootnote{स‚हायाः स‚त्त्वाः उप‚क‚र‚ण‚ञ्च‚न्द‚नादि । ग‚र्भादि ।}}षात्मिका । (१८५)
	\pend% ending standard par
      
	  \bigskip
	  \begingroup
	
	    \large
	  
	    \begin{quote}
	  
	    
	    \stanza[\smallbreak]
	\label{pv.1.185b}\flagstanza{\tiny\textenglish{...1.185b}}प‚रिग्र‚हः ॥ १८५ ॥\&[\smallbreak]


	
	    \end{quote}
	  
	  \endgroup
	
	  \bigskip
	  \begingroup
	
	    \large
	  
	    \begin{quote}
	  
	    
	    \stanza[\smallbreak]
	\label{pv.1.186a}\flagstanza{\tiny\textenglish{...1.186a}}य‚स्माद् देश‚विशेष‚स्य त‚त्प्राप्त्याशाकृतो तृणाम् ।\&[\smallbreak]


	
	    \end{quote}
	  
	  \endgroup
	

	  \pstart \leavevmode% starting standard par
	\hphantom{.}‚{\color{DodgerBlue3}‚य‚स्माद् देश‚विशेष‚स्य} प‚रिग्र‚हः ‚{\color{DodgerBlue3}‚त‚त्प्राप्ति}‚तृष्णा‚{\color{DodgerBlue3}‚कृतो नृणां} । नृश‚ब्दः प्राण्युप‚{\tiny $_{lb}$}‚ल‚क्ष‚णः । त‚तो ग‚र्भ‚स्था‚{\tiny $_{6}$}‚नादान‚म‚पि त‚त्तृष्णाकृत‚मेव । (१८६)
	\pend% ending standard par
      \label{div_pvv.1.187_1.188_1.189}
	  
	% new div opening: depth here is 2
	

	  \pstart \leavevmode% starting standard par
	\hphantom{.}c. न‚नूक्तं भ ग व ता त‚त्र क‚त‚म‚स्स‚मुद‚य आर्य‚स‚त्यं पौन‚र्भाविकी न\edtext{}{\edlabel{pvv.74-4}\label{pvv.74-4}\lemma{न}\Bfootnote{द्व‚यो राग‚योः स‚म‚व‚धानाभावादाह । व‚र्त‚मानार्थाल‚म्ब‚नाकृष्टा प्रीतिर्न‚न्दी ।}}न्दी ‚{\tiny $_{lb}$}‚राग‚स‚ह‚ग‚ता त‚त्र त‚त्राभिनान्दिनी य‚दुत काम‚तृष्णा भ‚व‚तृष्णा विभ‚व‚तृष्णा चे\edtext{}{\edlabel{pvv.74-asterisk}\label{pvv.74-asterisk}\lemma{ता}\Bfootnote{दीघ‚निकाय ।२२}} ति ‚{\tiny $_{lb}$}‚त‚त्क‚थ‚मेका भ‚व‚तृष्णोच्य‚ते स‚मुद‚य‚स‚त्य‚मिति । अत्राह (।)
	\pend% ending standard par
      
	  \bigskip
	  \begingroup
	
	    \large
	  
	    \begin{quote}
	  
	    
	    \stanza[\smallbreak]
	\label{pv.1.186b}\flagstanza{\tiny\textenglish{...1.186b}}सा भ‚वेच्छाऽप्त्य‚नाप्तीच्छोः प्र‚वृत्तिः सुख‚दुःख‚योः ॥ १८६ ॥\&[\smallbreak]


	
	    \end{quote}
	  
	  \endgroup
	
	  \bigskip
	  \begingroup
	
	    \large
	  
	    \begin{quote}
	  
	    
	    \stanza[\smallbreak]
	\label{pv.1.187a}\flagstanza{\tiny\textenglish{...1.187a}}य‚तोऽपि प्राणिनः काम‚विभ‚वेच्छे च ते म‚ते ।\&[\smallbreak]


	
	    \end{quote}
	  
	  \endgroup
	

	  \pstart \leavevmode% starting standard par
	\hphantom{.}‚{\color{DodgerBlue3}‚य‚तः} कार‚णात् प्राणिनः ‚{\color{DodgerBlue3}‚सुख‚दुःख‚योः} क्र‚मेणा‚{\color{DodgerBlue3}‚प्त्य‚नाप्तीच्छोः} प्र‚वृत्तिर्ग‚र्भ‚स्थान‚{\tiny $_{lb}$}‚प‚रिग्र‚हायातः ‚{\color{DodgerBlue3}‚सा भ‚वेच्छापि} (१८६) ‚{\color{DodgerBlue3}‚काम‚विभ‚वे}‚च्छे ‚{\color{DodgerBlue3}‚च ते म‚ते} । सुख‚प्राप्तीच्छा ‚{\tiny $_{lb}$}‚काम‚तृ‚{\tiny $_{7}$}‚ष्णा । दुःख‚वियोगेच्छा विभ‚व‚तृष्णा । भ‚व‚तृष्णायां सुख‚दुःख‚प्राप्तिप‚रि‚{\tiny $_{lb}$}‚हारेच्छापूर्व्विकायां ग‚र्भ‚स्थानोपादानेच्छात्मिकायां द्व‚योर‚पि संग्र‚हाद‚विरोधः । ‚{\tiny $_{lb}$}‚(१८७)
	\pend% ending standard par
      
	  \bigskip
	  \begingroup
	
	    \large
	  
	    \begin{quote}
	  
	    
	    \stanza[\smallbreak]
	\label{pv.1.187b}\flagstanza{\tiny\textenglish{...1.187b}}स‚र्व‚त्र चात्म‚स्नेह‚स्य हेतुत्वात् संप्र‚व‚र्त‚ते ॥ १८७ ॥\&[\smallbreak]


	
	    \end{quote}
	  
	  \endgroup
	
	  \bigskip
	  \begingroup
	
	    \large
	  
	    \begin{quote}
	  
	    
	    \stanza[\smallbreak]
	\label{pv.1.188a}\flagstanza{\tiny\textenglish{...1.188a}}असुखे सुख‚संज्ञ‚स्य;\&[\smallbreak]


	
	    \end{quote}
	  
	  \endgroup
	

	  \pstart \leavevmode% starting standard par
	\hphantom{.}स‚मुद‚य‚स्य च स‚मुद‚यात्म‚क‚त्वात् ‚{\color{DodgerBlue3}‚स‚र्व्व‚त्र} विष‚येऽ‚{\color{DodgerBlue3}‚सुखे} सुख‚र‚हिते ‚{\color{DodgerBlue3}‚सुख‚संज्ञ}‚स्य ‚{\tiny $_{lb}$}‚दुःख‚विप‚र्य‚य‚स्त‚स्यानेनाशुचौ शुचिविप‚र्यासोपि क‚थितः । न ह्य‚शुचौ त‚था म‚न्य‚मान ‚{\tiny $_{lb}$}‚क‚स्य‚चित् सुख‚संज्ञाऽऽत्म‚स्नेहेनेति ‚{\color{DodgerBlue3}‚आत्म‚स्नेह‚स्य हेतुत्वादिति} । अह‚ङ्कार‚म‚म‚कारो‚{\tiny $_{lb}$}‚\leavevmode\ledsidenote{\textenglish{075/s}} त्थापित‚स्य ।\edtext{\textsuperscript{*}}{\edlabel{pvv.75-1}\label{pvv.75-1}\lemma{*}\Bfootnote{अहंकार‚म‚म‚कारो य‚स्य ।}} अनेनात्म‚नि आत्म‚स्नेह‚विप‚र्यास उ‚{\tiny $_{3}$}‚क्तः । संप्र‚व‚र्त‚त इत्य‚नेनानित्य‚{\tiny $_{lb}$}‚विप‚र्यासः सूचितः । न हि नित्य‚विप‚र्यास‚म्विना फ‚लार्थी प्र‚व‚र्त‚ते ।
	\pend% ending standard par
      \textsuperscript{\textenglish{14b/MA}}‚{\tiny $_{lb}$}‚
	  \bigskip
	  \begingroup
	
	    \large
	  
	    \begin{quote}
	  
	    
	    \stanza[\smallbreak]
	\label{pv.1.188b}\flagstanza{\tiny\textenglish{...1.188b}}त‚स्मात् तृष्णा भ‚वाश्र‚यः ॥ ॥\&[\smallbreak]


	
	    \end{quote}
	  
	  \endgroup
	

	  \pstart \leavevmode% starting standard par
	\hphantom{.}त‚देवं च‚तुर्व्विप‚र्यास‚वासित‚मान‚स एवा‚{\color{DodgerBlue3}‚त्म‚स्नेहात्} सुख‚दुःख‚प्राप्तिप‚रिजिहीर्ष‚या ‚{\tiny $_{lb}$}‚‚{\color{DodgerBlue3}‚प्र‚व‚र्त‚ते} । त‚था ग‚र्भ‚स्थानेपि सुख‚दुःख‚प्राप्तिप‚रिहारेच्छैव तृष्णा । ‚{\color{DodgerBlue3}‚त‚स्मात्तृष्णा ‚{\tiny $_{lb}$}‚भ‚व‚स्याश्र‚यो} हेतुः । अनेन हेतुत आख्यातं ।
	\pend% ending standard par
      
	  \bigskip
	  \begingroup
	
	    \large
	  
	    \begin{quote}
	  
	    
	    \stanza[\smallbreak]
	\label{pv.1.188c}\flagstanza{\tiny\textenglish{...1.188c}}विर‚क्त‚ज‚न्मादृष्टेरित्याचार्याः संप्र‚च‚क्ष‚ते ॥ १८८ ॥\&[\smallbreak]


	
	    \end{quote}
	  
	  \endgroup
	
	  \bigskip
	  \begingroup
	
	    \large
	  
	    \begin{quote}
	  
	    
	    \stanza[\smallbreak]
	\label{pv.1.189a}\flagstanza{\tiny\textenglish{...1.189a}}अदेह‚रागादृष्टेश्च देहाद् राग‚स‚मुद्भ‚वः ।\&[\smallbreak]


	
	    \end{quote}
	  
	  \endgroup
	

	  \pstart \leavevmode% starting standard par
	\hphantom{.}न‚नु विर‚क्त‚स्य ज‚न्मादृष्टेरि \href{http://sarit.indology.info/?cref=nsū.3.1.25}{(न्याय‚सू॰ ३।१।२५)} ‚{\color{DodgerBlue3}‚त्याचार्या} गौ त मा ‚{\tiny $_{lb}$}‚द\edtext{}{\edlabel{pvv.75-2}\label{pvv.75-2}\lemma{द}\Bfootnote{व‚सुव‚न्धुनैव‚मुक्ते चार्व्वाक आह ।}}योपि ‚{\color{DodgerBlue3}‚संप्र‚च‚क्ष‚ते} (। १८८) ‚{\color{DodgerBlue3}‚ऽदेह‚स्य रागादृष्टेश्च देहाद्राग‚स‚मुद्भ‚वः} स्थित‚स्तंतो ‚{\tiny $_{lb}$}‚न देही वीत‚रागः ।
	\pend% ending standard par
      

	  \pstart \leavevmode% starting standard par
	अत्राह (।)
	\pend% ending standard par
      
	  \bigskip
	  \begingroup
	
	    \large
	  
	    \begin{quote}
	  
	    
	    \stanza[\smallbreak]
	\label{pv.1.189b}\flagstanza{\tiny\textenglish{...1.189b}}निमित्तोप‚ग‚मादिष्ट‚मुपादानं तु वार्य‚ते ॥ १८९ ॥\&[\smallbreak]


	
	    \end{quote}
	  
	  \endgroup
	

	  \pstart \leavevmode% starting standard par
	\hphantom{.}निमित्त‚स्य स‚ह‚कारिकार‚ण‚स्यो‚{\color{DodgerBlue3}‚प‚ग‚मात्} देहोस्य राग‚स्य स‚हाकारिकार‚ण‚{\color{DodgerBlue3}‚मिष्टं} त‚तो नानिष्ट‚माप‚द्य‚ते । ‚{\color{DodgerBlue3}‚उपादान‚न्तु} देहो राग‚स्य ‚{\color{DodgerBlue3}‚वार्य‚ते} । राग एव तू\edtext{}{\edlabel{pvv.75-3}\label{pvv.75-3}\lemma{तू}\Bfootnote{देह‚स्य}}पादान‚कार‚णं । ‚{\tiny $_{lb}$}‚न च रागो ज‚न्म‚हेतुः । विर‚क्त‚स्य क‚रुण‚या ज‚न्म‚स‚म्भ‚वात् । र\edtext{}{\edlabel{pvv.75-4}\label{pvv.75-4}\lemma{र}\Bfootnote{न राग‚मात्रात्}}क्त‚स्यापि तृष्ण‚यैव ‚{\tiny $_{lb}$}‚ज‚न्म‚ग्र‚हः । अनेन प्र‚त्य‚य‚त इति व्याख्या\edtext{}{\edlabel{pvv.75-5}\label{pvv.75-5}\lemma{व्याख्या}\Bfootnote{प्र‚भ‚वाकार‚माह ।}}तं । (१८९)
	\pend% ending standard par
      \label{div_pvv.1.190}
	  
	% new div opening: depth here is 2
	
	  \bigskip
	  \begingroup
	
	    \large
	  
	    \begin{quote}
	  
	    
	    \stanza[\smallbreak]
	\label{pv.1.190}\flagstanza{\tiny\textenglish{....1.190}}इमां तु युक्तिम‚न्विच्छ‚न् बाध‚ते स्व‚म‚तं स्व‚य‚म् ।&ज‚न्म‚ना स‚ह‚भाव‚श्चेत् जातानां राग‚द‚र्श‚नात् ॥ १९० ॥\&[\smallbreak]


	
	    \end{quote}
	  
	  \endgroup
	

	  \pstart \leavevmode% starting standard par
	\hphantom{.}d. विर‚क्त‚ज‚न्मादृष्टेरिती‚{\color{DodgerBlue3}‚मां युक्तिम‚न्विच्छ‚न्} चा र्व्वा कः ‚{\color{DodgerBlue3}‚स्व‚म‚तं स्व‚यं‚{\tiny $_{2}$}‚ ‚{\tiny $_{lb}$}‚बाध‚ते} । राग‚हेतुको देहः त‚द्धेतुक‚श्च राग इति । अन्योन्य‚हेतुत्वात् ज‚न्म‚प्र‚ब‚न्ध‚सिद्धेः । ‚{\tiny $_{lb}$}‚वीत‚रागाभ्युप‚ग‚माच्च स्व‚म‚त‚बाधास्य । ‚{\color{DodgerBlue3}‚ज‚न्म‚ना स‚ह‚भावो} रागादीनां । न पूर्व्वं ‚{\tiny $_{lb}$}‚रागोस्ति जातानां ‚{\color{DodgerBlue3}‚राग‚द‚र्श‚नात्} । अतो न रागो देह‚हेतुरिति चेत् । (१९०)
	\pend% ending standard par
      \label{div_pvv.1.191_1.192_1.193_1.194}
	  
	% new div opening: depth here is 2
	
	  \bigskip
	  \begingroup
	
	    \large
	  
	    \begin{quote}
	  
	    
	    \stanza[\smallbreak]
	\label{pv.1.191a}\flagstanza{\tiny\textenglish{...1.191a}}स‚भाग‚जातेः प्राक् सिद्धिः;\&[\smallbreak]


	
	    \end{quote}
	  
	  \endgroup
	\textsuperscript{\textenglish{076/s}}

	  \begin{center}%% label @type='head'
	\textbf{(II. तृष्णा ज‚न्म‚स‚मुद‚यः)}
	\end{center}
	‚{\tiny $_{lb}$}‚

	  \pstart \leavevmode% starting standard par
	न‚न्वेवं देहोपि न स्यात् राग‚हेतुः स‚ह‚भावात् । न चाहेतुक‚ता । त‚तः स‚भा‚{\tiny $_{lb}$}‚गात् स‚जातीयाद्राग‚{\color{DodgerBlue3}‚ज्जाते}‚रुत्पादात् । ‚{\color{DodgerBlue3}‚प्राग्रा}‚ग‚स्य ‚{\color{DodgerBlue3}‚सिद्धि}‚रित्यायातं ।
	\pend% ending standard par
      

	  \begin{center}%% label @type='head'
	\textbf{(III. क‚र्माऽपि)}
	\end{center}
	

	  \pstart \leavevmode% starting standard par
	न‚न्व‚विद्या तृष्णा क‚र्म च ज‚न्म‚कार‚णं‚{\tiny $_{3}$}‚ त‚त्क‚थं तृष्णैव केव‚ला स‚मुद‚य उक्त ‚{\tiny $_{lb}$}‚इत्याह (।)
	\pend% ending standard par
      
	  \bigskip
	  \begingroup
	
	    \large
	  
	    \begin{quote}
	  
	    
	    \stanza[\smallbreak]
	\label{pv.1.191b}\flagstanza{\tiny\textenglish{...1.191b}}कार‚ण‚त्वेऽपि नोदित‚म् ।&अज्ञानं; उक्ता तृष्णैव स‚न्तान‚प्रेर‚णाद् भ‚वे ॥ १९१ ॥\&[\smallbreak]


	
	    \end{quote}
	  
	  \endgroup
	
	  \bigskip
	  \begingroup
	
	    \large
	  
	    \begin{quote}
	  
	    
	    \stanza[\smallbreak]
	\label{pv.1.192a}\flagstanza{\tiny\textenglish{...1.192a}}आन‚न्त‚र्याच्च क‚र्मापि स‚ति त‚स्मिन्न‚संभ‚वात् ।\&[\smallbreak]


	
	    \end{quote}
	  
	  \endgroup
	

	  \pstart \leavevmode% starting standard par
	\hphantom{.}‚{\color{DodgerBlue3}‚कार‚ण‚त्वेपि नोदित‚म‚ज्ञान}‚म‚विद्या मोहाप‚र‚संज्ञ‚कं । ‚{\color{DodgerBlue3}‚उक्ता तृष्णैव} त‚यैव ‚{\color{DodgerBlue3}‚स‚न्ता‚{\tiny $_{lb}$}‚न}‚स्य प‚ञ्च‚स्क‚न्ध‚स‚न्त‚तेः ‚{\color{DodgerBlue3}‚प्रेर‚णात्\edtext{}{\edlabel{pvv.76-1}\label{pvv.76-1}\lemma{णात्}\Bfootnote{इय‚मेव प्र‚भ‚विष्णुत्वात् प्र‚भ‚वः ।}}} किम‚र्थं ‚{\color{DodgerBlue3}‚भ‚वे}‚(१९१) ज‚न्म‚निमित्तं ‚{\color{DodgerBlue3}‚क‚र्मापि} हेतुत्वेऽपि ‚{\tiny $_{lb}$}‚नोक्तं स‚मुद‚य‚त्वेन ‚{\color{DodgerBlue3}‚स‚ति त‚स्मिन्न}‚ज्ञाने क‚र्म‚णि च । तृष्णाऽसंमुखीभावेऽ‚{\color{DodgerBlue3}‚भावा}‚ज्ज‚{\tiny $_{lb}$}‚न्म‚नः । आन‚न्त‚र्याच्च तृष्णायाः स‚तोर‚पि मोह‚क‚र्म‚णोस्तृष्णाया असंमुखीभावे ‚{\tiny $_{lb}$}‚ज‚न्म‚न आक्षेप‚क‚त्वासंभ‚वात्\edtext{}{\edlabel{pvv.76-2}\label{pvv.76-2}\lemma{वात्}\Bfootnote{आन‚न्त‚र्य्याच्च तृष्णायाः ।}} य‚था विमुक्ति (:) चित्त‚स्येत्युक्तः स‚मुद‚यः ।
	\pend% ending standard par
      
	  \bigskip
	  \begingroup
	
	    \large
	  
	    \begin{quote}
	  
	    
	    \stanza[\smallbreak]
	\label{pv.1.192b}\flagstanza{\tiny\textenglish{...1.192b}}त‚द‚नान्य‚न्तिकं हेतोः प्र‚तिब‚न्धादिस‚म्भ‚वात् ॥ १९२ ॥\&[\smallbreak]


	
	    \end{quote}
	  
	  \endgroup
	

	  \pstart \leavevmode% starting standard par
	त‚देत‚द्य‚थोक्त‚कार‚ण (स‚मुद‚य) स्व‚भावं दुःख‚म‚नात्य‚न्तिकं संभ‚व‚दुच्छेदं दुःख‚{\tiny $_{lb}$}‚हेतोस्तृष्णायाः प्र‚तिब‚न्ध‚स्य संभाव्य‚{\tiny $_{4}$}‚मान‚त्वात् । आदिश‚ब्दाद‚विद्यादेः स‚ह‚कारिणो ‚{\tiny $_{lb}$}‚वैक‚ल्य‚{\color{DodgerBlue3}‚स‚म्भ‚वात्} ।
	\pend% ending standard par
      

	  \pstart \leavevmode% starting standard par
	अनेन निरोध एव नास्तीति वादि\edtext{}{\edlabel{pvv.76-3}\label{pvv.76-3}\lemma{वादि}\Bfootnote{मीमांस‚काद‚यः ।}}नः प्र‚तिनिरोध‚त इति क‚थितं ।
	\pend% ending standard par
      

	  \begin{center}%% label @type='head'
	\textbf{(ग) निरोध‚स‚त्य‚म्}
	\end{center}
	
	  \bigskip
	  \begingroup
	
	    \large
	  
	    \begin{quote}
	  
	    
	    \stanza[\smallbreak]
	\label{pv.1.193a}\flagstanza{\tiny\textenglish{...1.193a}}संसारित्वाद‚निर्मोक्षो नेष्ट‚त्वाद‚प्र‚सिद्धितः ।\&[\smallbreak]


	
	    \end{quote}
	  
	  \endgroup
	

	  \pstart \leavevmode% starting standard par
	\hphantom{.}न‚नु ‚{\color{DodgerBlue3}‚संसारित्वाद‚निर्म्मोक्षो} मुक्तिर्नास्ति क‚स्य‚चित् । त‚त्क‚थं निरोध‚स‚म्भाव‚ना । ‚{\tiny $_{lb}$}‚‚{\color{DodgerBlue3}‚नैष दोष इष्ट‚त्वात्} । को नाम मुक्तिं संसारिण इच्छ‚ति । ‚{\color{DodgerBlue3}‚अप्र‚सि}‚द्धित‚स्त‚स्य । ‚{\tiny $_{lb}$}‚न हि संसारीं क‚श्चिद‚स्ति । किन्तु दुःखं केव‚लं हेतुब‚लात्प्र‚व‚र्त‚ते त‚द‚भावाच्च न ‚{\tiny $_{lb}$}‚भ‚व‚तीति ब्रूमः ।
	\pend% ending standard par
      \textsuperscript{\textenglish{077/s}}

	  \begin{center}%% label @type='head'
	\textbf{(I. संसार्य‚भावे मुक्तिव्य‚व‚स्था)}
	\end{center}
	

	  \pstart \leavevmode% starting standard par
	य‚दि संसारी क‚श्चिन्नास्ति को‚{\tiny $_{5}$}‚ मुक्त्य‚र्थी किम‚र्थं प्र‚व‚र्त‚ते इत्याह (।)
	\pend% ending standard par
      
	  \bigskip
	  \begingroup
	
	    \large
	  
	    \begin{quote}
	  
	    
	    \stanza[\smallbreak]
	\label{pv.1.193b}\flagstanza{\tiny\textenglish{...1.193b}}याव‚च्चात्म‚नि न प्रेम्णो हानिः स प‚रित‚स्य‚ति ॥ १९३ ॥\&[\smallbreak]


	
	    \end{quote}
	  
	  \endgroup
	
	  \bigskip
	  \begingroup
	
	    \large
	  
	    \begin{quote}
	  
	    
	    \stanza[\smallbreak]
	\label{pv.1.194a}\flagstanza{\tiny\textenglish{...1.194a}}ताव‚द् दुःखित‚मारोप्य न च स्व‚स्थोऽव‚तिष्ठ‚ते ॥\&[\smallbreak]


	
	    \end{quote}
	  
	  \endgroup
	

	  \pstart \leavevmode% starting standard par
	\hphantom{.}‚{\color{DodgerBlue3}‚याव‚च्चात्म‚नि} एक‚त्वाहंकार‚विष‚येषु स्क‚न्धेषु ‚{\color{DodgerBlue3}‚प्रेम्णः} स्नेह‚स्य ‚{\color{DodgerBlue3}‚न हानि}‚स्ताव‚द् ‚{\tiny $_{lb}$}‚दुःखित‚मात्मान‚मारोप्य ‚{\color{DodgerBlue3}‚स} प्राण्य‚भिम‚तो दुःख‚स‚न्तानः ‚{\color{DodgerBlue3}‚प‚रित‚स्य‚ति} (१९३) दुःख‚{\tiny $_{lb}$}‚मास्ते । ‚{\color{DodgerBlue3}‚न च} दुःख‚हेत्व‚प‚ग‚मोपायाभ्यां स क्लेश‚म्विना ‚{\color{DodgerBlue3}‚स्व‚स्थोऽव‚तिष्ठ‚ते} ।
	\pend% ending standard par
      \textsuperscript{\textenglish{15a/MA}}‚{\tiny $_{lb}$}‚

	  \pstart \leavevmode% starting standard par
	किन्तु (।)
	\pend% ending standard par
      
	  \bigskip
	  \begingroup
	
	    \large
	  
	    \begin{quote}
	  
	    
	    \stanza[\smallbreak]
	\label{pv.1.194b}\flagstanza{\tiny\textenglish{...1.194b}}मिथ्याध्यारोप‚हानार्थं य‚त्नोऽस‚त्य‚पि मोक्त‚रि ॥ १९४ ॥\&[\smallbreak]


	
	    \end{quote}
	  
	  \endgroup
	

	  \pstart \leavevmode% starting standard par
	\hphantom{.}‚{\color{DodgerBlue3}‚मिथ्याध्यारो}‚प‚स्य संसारित्वाध्य‚व‚साय‚स्य ‚{\color{DodgerBlue3}‚हानार्थं य‚त्नोऽस‚त्य‚पि} क‚स्मिँश्चि‚{\tiny $_{lb}$}‚दात्मादौ ‚{\color{DodgerBlue3}‚मोक्त‚रि} । न हि य‚थाव‚स्त्त्वेव व्य‚व‚{\tiny $_{6}$}‚हारः । किन्तु य‚थाव‚साय‚ञ्च । ‚{\tiny $_{lb}$}‚त‚थाहि र‚ज्जुर‚पि स‚र्पाध्य‚व‚साय‚विष‚य‚त्वात् प‚रिहार‚विष‚यः । एव‚म‚ह‚मेव ब‚द्धोऽह‚{\tiny $_{lb}$}‚मेव मोक्ष्यामीत्य‚ध्यारोपान्मुक्त्य‚र्थं व्यायामः । (१९४)
	\pend% ending standard par
      \label{div_pvv.1.195_1.196_1.197}
	  
	% new div opening: depth here is 2
	

	  \begin{center}%% label @type='head'
	\textbf{(II. मुक्तानां संसारे स्थितिः)}
	\end{center}
	‚{\tiny $_{lb}$}‚

	  \pstart \leavevmode% starting standard par
	भ‚व‚त्वात्म‚ग्र‚ह‚विप‚र्य‚स्तानां सुखाद्य‚भिलाषात्प्र‚वृत्तिल‚क्ष‚णा संसारे स्थितिरुन्मी‚{\tiny $_{lb}$}‚लितात्म‚ग्र‚ह‚योनिस‚क‚ल‚दोष‚राश‚य‚स्तु क‚स्मादास‚त इत्याह (।)
	\pend% ending standard par
      
	  \bigskip
	  \begingroup
	
	    \large
	  
	    \begin{quote}
	  
	    
	    \stanza[\smallbreak]
	\label{pv.1.195a}\flagstanza{\tiny\textenglish{...1.195a}}अव‚स्था वीत‚रागाणां द‚य‚या क‚र्म‚णाऽपि वा ।\&[\smallbreak]


	
	    \end{quote}
	  
	  \endgroup
	

	  \pstart \leavevmode% starting standard par
	\hphantom{.}a. ‚{\color{DodgerBlue3}‚अव‚स्था वीत‚रागाणां द‚य‚यापि क‚र्म‚णापि वा} वीत‚मोहानाम‚पि दुःखाद् दुःख‚{\tiny $_{7}$}‚‚{\tiny $_{lb}$}‚हेतोश्च लोक‚मुद्ध‚र्तुकाम‚त‚या स्थितिः । ताव‚त्कालानुब‚न्धिश‚रीराक्षेप‚केन क‚र्म‚णा ‚{\tiny $_{lb}$}‚वा स्थितिः ।
	\pend% ending standard par
      

	  \pstart \leavevmode% starting standard par
	त‚देवाह (।)
	\pend% ending standard par
      
	  \bigskip
	  \begingroup
	
	    \large
	  
	    \begin{quote}
	  
	    
	    \stanza[\smallbreak]
	\label{pv.1.195b}\flagstanza{\tiny\textenglish{...1.195b}}आक्षिप्तेऽविनिवृत्तीष्टेः ।\&[\smallbreak]


	
	    \end{quote}
	  
	  \endgroup
	

	  \pstart \leavevmode% starting standard par
	\hphantom{.}‚{\color{DodgerBlue3}‚आक्षिप्ते} क‚र्म‚णा कायेऽविनिवृत्ते‚{\color{DodgerBlue3}‚र्निवृत्य}‚भाव‚{\color{DodgerBlue3}‚स्येष्टेः} । य‚द्येवं ज‚न्मान्त‚रा‚{\color{DodgerBlue3}‚क्षेप‚क‚स्य} क‚र्म‚णः स‚द्भावात् भ‚वा\edtext{}{\edlabel{pvv.77-1}\label{pvv.77-1}\lemma{वा}\Bfootnote{त‚त्र च ते प्रेक्ष‚काः ।}} न्त‚र‚ञ्च स्यादित्याह---
	\pend% ending standard par
      
	  \bigskip
	  \begingroup
	
	    \large
	  
	    \begin{quote}
	  
	    
	    \stanza[\smallbreak]
	\label{pv.1.195c}\flagstanza{\tiny\textenglish{...1.195c}}स‚ह‚कारिक्ष‚याद‚ल‚म् ॥ १९५ ॥\&[\smallbreak]


	
	    \end{quote}
	  
	  \endgroup
	
	  \bigskip
	  \begingroup
	
	    \large
	  
	    \begin{quote}
	  
	    
	    \stanza[\smallbreak]
	\label{pv.1.196a}\flagstanza{\tiny\textenglish{...1.196a}}नाक्षेप्तुम‚प‚रं क‚र्म भ‚व‚तृष्णाविलंघिनाम् ।\&[\smallbreak]


	
	    \end{quote}
	  
	  \endgroup
	\textsuperscript{\textenglish{078/s}}

	  \pstart \leavevmode% starting standard par
	\hphantom{.}‚{\color{DodgerBlue3}‚स‚ह‚कारिण} आत्मात्मीयादिविप‚र्यासिज्ञान‚स्य तृष्णायाश्च ‚{\color{DodgerBlue3}‚क्ष‚यान्नालं} (१९५) ‚{\tiny $_{lb}$}‚न श‚क्तं ‚{\color{DodgerBlue3}‚क‚र्माक्षेप्तुम}‚प‚रं भ‚वं ‚{\color{DodgerBlue3}‚भ‚व‚तृष्णाविलंधिनां} । तृष्णालंघ‚ना नैरात्म्य‚दृष्टिरि‚{\tiny $_{lb}$}‚त्यात्मात्मीयादिमोह‚निवृत्तिश्चोक्ता । अनेन स‚हाकारिवैक‚ल्य‚मुक्तं ।
	\pend% ending standard par
      

	  \pstart \leavevmode% starting standard par
	न‚नु द‚या‚{\tiny $_{1}$}‚ स‚त्त्व‚द‚र्श‚नात् त‚च्च क्षीणं मुक्तानां त‚त्क‚थ‚न्द‚य‚या स्थितिरित्याह (।)
	\pend% ending standard par
      
	  \bigskip
	  \begingroup
	
	    \large
	  
	    \begin{quote}
	  
	    
	    \stanza[\smallbreak]
	\label{pv.1.196b}\flagstanza{\tiny\textenglish{...1.196b}}दुःख‚ज्ञानेऽविरुद्ध‚स्य पूर्व‚संस्कार‚वाहिनी ॥ १९६ ॥\&[\smallbreak]


	
	    \end{quote}
	  
	  \endgroup
	
	  \bigskip
	  \begingroup
	
	    \large
	  
	    \begin{quote}
	  
	    
	    \stanza[\smallbreak]
	\label{pv.1.197a}\flagstanza{\tiny\textenglish{...1.197a}}व‚स्तुध‚र्मो द‚योत्प‚त्तिर्न सा स‚त्वानुरोधिनी ।\&[\smallbreak]


	
	    \end{quote}
	  
	  \endgroup
	

	  \pstart \leavevmode% starting standard par
	\hphantom{.}‚{\color{DodgerBlue3}‚दुःख}‚स्यानित्य‚दुःख‚शून्यानात्म‚काकार‚स्य ‚{\color{DodgerBlue3}‚ज्ञाने} स‚त्य‚{\color{DodgerBlue3}‚विरुद्ध‚स्य} द्वेषाभावात् ‚{\tiny $_{lb}$}‚स‚र्व्व‚त्राप्र‚तिह\edtext{}{\edlabel{pvv.78-1}\label{pvv.78-1}\lemma{तिह}\Bfootnote{द्वेषो हि कृपाविरोधी स प्र‚हीण‚द्वेषः ।}} त‚स्य ‚{\color{DodgerBlue3}‚पूर्व्व‚संस्कार‚वाहिनी} (१९६) पूर्व्वाभ्या\edtext{}{\edlabel{pvv.78-2}\label{pvv.78-2}\lemma{पूर्व्वाभ्या}\Bfootnote{क‚रुणाया । शास्तुः ।}}स‚प्र‚वृत्ता या ‚{\color{DodgerBlue3}‚द‚योत्प‚त्तिः ‚{\tiny $_{lb}$}‚सा न स‚त्त्वानुरोधिनी} स‚त्त्व‚दृष्टिव‚शा किन्तु व‚स्तुनो दुःख‚स्य कृपाविष‚य‚त‚याऽभ्य‚स्त‚स्य ‚{\tiny $_{lb}$}‚ध\edtext{}{\edlabel{pvv.78-3}\label{pvv.78-3}\lemma{ध}\Bfootnote{एतेन ध‚र्माल‚म्ब‚नी कृपोक्ता}}र्मः । उन्मूलितात्म‚दृष्टीनाम‚पि दुःख‚स्य कृपाविष‚य‚त‚याऽभ्य‚स्त‚स्य संमुखीभाव‚{\tiny $_{lb}$}‚मात्रेण द‚योत्प‚द्य‚त इत्य‚र्थः ।
	\pend% ending standard par
      

	  \pstart \leavevmode% starting standard par
	b. एव‚न्त‚र्हि रागोपि मुक्तानां‚{\tiny $_{2}$}‚ स्यादित्याह (।)
	\pend% ending standard par
      
	  \bigskip
	  \begingroup
	
	    \large
	  
	    \begin{quote}
	  
	    
	    \stanza[\smallbreak]
	\label{pv.1.197b}\flagstanza{\tiny\textenglish{...1.197b}}आत्मान्त‚र‚स‚मारोपाद् रागो ध‚र्मेऽत‚दात्म‚के ॥ १९७ ॥\&[\smallbreak]


	
	    \end{quote}
	  
	  \endgroup
	

	  \pstart \leavevmode% starting standard par
	\hphantom{.}‚{\color{DodgerBlue3}‚आ\edtext{}{\edlabel{pvv.78-4}\label{pvv.78-4}\lemma{आ}\Bfootnote{उक्त‚स्क‚न्धेभ्योऽन्य‚त्वाद‚न्त‚श‚ब्दः ।}}त्मान्त‚र‚स्य} स्थिर‚सुखात्मात्मीय‚रूप‚स्या‚{\color{DodgerBlue3}‚रोपात् ध‚र्मे} स्क‚न्ध‚मात्र‚रूपेऽ‚{\color{DodgerBlue3}‚त‚दात्म‚के} व‚स्तुनोऽस्थिरादिस्व‚भावे ‚{\color{DodgerBlue3}‚रागो}‚ऽभिष्व‚ङ्ग‚रूपो भ‚व‚ति ॥ (१९७) ।
	\pend% ending standard par
      \label{div_pvv.1.198_1.199_1.200}
	  
	% new div opening: depth here is 2
	
	  \bigskip
	  \begingroup
	
	    \large
	  
	    \begin{quote}
	  
	    
	    \stanza[\smallbreak]
	\label{pv.1.198a}\flagstanza{\tiny\textenglish{...1.198a}}दुःख‚स‚न्तान‚संस्प‚र्श‚मात्रेणैव द‚योद‚यः ।\&[\smallbreak]


	
	    \end{quote}
	  
	  \endgroup
	

	  \pstart \leavevmode% starting standard par
	\hphantom{.}‚{\color{DodgerBlue3}‚द‚योद‚य‚स्तु दुःख‚स‚न्तान‚स्य संस्प‚र्शो} द‚र्श‚नं त‚{\color{DodgerBlue3}‚न्मात्रेणैवं} भ\edtext{}{\edlabel{pvv.78-5}\label{pvv.78-5}\lemma{भ}\Bfootnote{इति रागाद‚न्या कृपा ।}}व‚ति । न त‚त्र स‚त्त्व‚{\tiny $_{lb}$}‚द‚र्श‚नापेक्षा ।
	\pend% ending standard par
      

	  \pstart \leavevmode% starting standard par
	c. य‚था दुःख‚द‚र्श‚नात् द‚योत्प‚त्तिस्त‚थाप‚कारिणि द्वेषोपि स्यादित्याह (।)
	\pend% ending standard par
      
	  \bigskip
	  \begingroup
	
	    \large
	  
	    \begin{quote}
	  
	    
	    \stanza[\smallbreak]
	\label{pv.1.198b}\flagstanza{\tiny\textenglish{...1.198b}}मोह‚श्च मूलं दोषाणां स च स‚त्व‚ग्र‚हः;\&[\smallbreak]


	
	    \end{quote}
	  
	  \endgroup
	

	  \pstart \leavevmode% starting standard par
	\hphantom{.}‚{\color{DodgerBlue3}‚मोह‚श्च\edtext{}{\edlabel{pvv.78-6}\label{pvv.78-6}\lemma{श्च}\Bfootnote{अमूढ‚स्य पापावृत्तेः ।}}} मूल‚मादिकार‚णं ‚{\color{DodgerBlue3}‚दोषाणां स च} मोहः ‚{\color{DodgerBlue3}‚स‚त्त्व‚ग्र‚हः} । उन्मूलित‚स‚त्त्व‚{\tiny $_{lb}$}‚दृष्टेश्च । (१९८)
	\pend% ending standard par
      
	  \bigskip
	  \begingroup
	
	    \large
	  
	    \begin{quote}
	  
	    
	    \stanza[\smallbreak]
	\label{pv.1.198c}\flagstanza{\tiny\textenglish{...1.198c}}विना ॥ १९८ ॥\&[\smallbreak]


	
	    \end{quote}
	  
	  \endgroup
	
	  \bigskip
	  \begingroup
	
	    \large
	  
	    \begin{quote}
	  
	    
	    \stanza[\smallbreak]
	\label{pv.1.199a}\flagstanza{\tiny\textenglish{...1.199a}}तेनोद्य‚हेतौ न द्वेषो, न दोषोऽतः कृपा म‚ता ।\&[\smallbreak]


	
	    \end{quote}
	  
	  \endgroup
	\textsuperscript{\textenglish{079/s}}

	  \pstart \leavevmode% starting standard par
	\hphantom{.}‚{\color{DodgerBlue3}‚तेन}‚{\tiny $_{3}$}‚ स‚त्त्व‚ग्र‚हेण विनाऽ‚{\color{DodgerBlue3}‚द्य‚हेता}‚व‚प‚कारिणि ‚{\color{DodgerBlue3}‚न द्वेषो}‚स्ति आत्म‚नोऽद‚र्श‚ना‚{\tiny $_{lb}$}‚त्त‚द‚प‚कार‚भ्रान्त्य‚भावात् । ‚{\color{DodgerBlue3}‚अतो} दोष‚मूल‚स्यात्म‚ग्र‚ह‚स्याभावादुत्प‚द्य‚माना ‚{\color{DodgerBlue3}‚कृपा ‚{\tiny $_{lb}$}‚न दोषो म‚ता} (।)
	\pend% ending standard par
      

	  \pstart \leavevmode% starting standard par
	b. य‚दि पूर्व्व‚क‚र्म्मावेध‚त‚स्त‚दैव न निर्व्वाणं\edtext{}{\edlabel{pvv.79-1}\label{pvv.79-1}\lemma{निर्व्वाणं}\Bfootnote{किन्तु स्थितिस्त‚र्हि कृपाप‚र‚व‚श‚स्य ।}} त‚दा सांसारिक‚तैव स्यादित्याह (।)
	\pend% ending standard par
      
	  \bigskip
	  \begingroup
	
	    \large
	  
	    \begin{quote}
	  
	    
	    \stanza[\smallbreak]
	\label{pv.1.199b}\flagstanza{\tiny\textenglish{...1.199b}}नामुक्तिः पूर्व‚संस्कार‚क्ष‚येऽन्याप्र‚तिस‚न्धितः ॥ १९९ ॥\&[\smallbreak]


	
	    \end{quote}
	  
	  \endgroup
	
	  \bigskip
	  \begingroup
	
	    \large
	  
	    \begin{quote}
	  
	    
	    \stanza[\smallbreak]
	\label{pv.1.200}\flagstanza{\tiny\textenglish{....1.200}}अक्षीण‚श‚क्तिः संस्कारो येषां तिष्ठ‚न्ति तेऽन‚धाः ॥&म‚न्द‚त्वात् क‚रुणायाश्च न य‚त्नः स्थाप‚ने म‚हान् ॥ २०० ॥\&[\smallbreak]


	
	    \end{quote}
	  
	  \endgroup
	

	  \pstart \leavevmode% starting standard par
	\hphantom{.}‚{\color{DodgerBlue3}‚नामुक्तिः} किन्तु मुक्तिरेव ‚{\color{DodgerBlue3}‚पूर्व्व‚संस्कार‚क्ष‚ये} पूर्व्व‚क‚र्मावेध‚क्ष‚ये ‚{\color{DodgerBlue3}‚स‚त्य‚न्य‚स्य} दुःख‚स्य ‚{\tiny $_{lb}$}‚हेतुवैक‚ल्याद‚{\color{DodgerBlue3}‚प्र‚तिस‚न्धितः} । (१९९) येषां पुन‚र्म‚हाकृपाणां प्राणि‚{\tiny $_{4}$}‚धान‚प‚रिपुष्ट‚स्य ‚{\tiny $_{lb}$}‚ज‚न्माक्षेप‚क‚क‚र्म‚णः ‚{\color{DodgerBlue3}‚सँस्कारोऽक्षीण‚श‚क्तिस्तेऽन‚धाः} स‚म्य‚क्‏स‚म्बुद्धाः याव‚दाकाश‚{\tiny $_{lb}$}‚‚{\color{DodgerBlue3}‚न्तिष्ठ‚न्त्येव} । श्राव‚काणान्तु क‚र्म‚णो निय‚त‚काल‚स्थितिक‚देहाक्षेप‚क‚त्वा‚{\color{DodgerBlue3}‚न्म‚न्द‚त्वात्} क‚रुणाया ‚{\color{DodgerBlue3}‚य‚त्न‚श्च म‚हान् स्था}‚प‚ने नास्तीति ‚{\color{DodgerBlue3}‚न} स‚दा स्थितिः (। २००)
	\pend% ending standard par
      \label{div_pvv.1.201_1.202_1.203_1.204_1.205}
	  
	% new div opening: depth here is 2
	
	  \bigskip
	  \begingroup
	
	    \large
	  
	    \begin{quote}
	  
	    
	    \stanza[\smallbreak]
	\label{pv.1.201a}\flagstanza{\tiny\textenglish{...1.201a}}तिष्ठ‚न्त्येव प‚राधीना येषां तु म‚ह‚ती कृपा ।\&[\smallbreak]


	
	    \end{quote}
	  
	  \endgroup
	

	  \pstart \leavevmode% starting standard par
	\hphantom{.}तिष्ठ‚न्त्येव स‚र्व्व‚दा ते ‚{\color{DodgerBlue3}‚प‚राधीनाः} प‚रेषामुप‚क‚र‚णीकृतात्मानो म‚हामुन‚यः । ‚{\tiny $_{lb}$}‚येषाम‚कार‚ण‚व‚त्स‚लानां ‚{\color{DodgerBlue3}‚म‚ह}‚ती ‚{\color{DodgerBlue3}‚कृपा} ।
	\pend% ending standard par
      

	  \begin{center}%% label @type='head'
	\textbf{III. स‚त्कार्य‚दृष्टेर्विग‚मः}
	\end{center}
	
	  \bigskip
	  \begingroup
	
	    \large
	  
	    \begin{quote}
	  
	    
	    \stanza[\smallbreak]
	\label{pv.1.201b}\flagstanza{\tiny\textenglish{...1.201b}}स‚त्काय‚दृष्टेर्विग‚मादाद्य एवाभ‚वो भ‚वेत् ॥ २०१ ॥\&[\smallbreak]


	
	    \end{quote}
	  
	  \endgroup
	
	  \bigskip
	  \begingroup
	
	    \large
	  
	    \begin{quote}
	  
	    
	    \stanza[\smallbreak]
	\label{pv.1.202a}\flagstanza{\tiny\textenglish{...1.202a}}मार्गे चेत् स‚ह‚जाहानेर्न हानौ वा भ‚वः कुतः ।\&[\smallbreak]


	
	    \end{quote}
	  
	  \endgroup
	

	  \pstart \leavevmode% starting standard par
	\hphantom{.}a. न‚न्वाद्य एव मार्ग्गे द‚र्श‚न‚मार्ग्गे ‚{\color{DodgerBlue3}‚स‚त्का‚{\tiny $_{5}$}‚य‚दृष्टेर्व्वि}‚ग‚मात् स्रोत आप‚न्न‚स्य ‚{\color{DodgerBlue3}‚भ‚वो} ज‚न्मान्त‚र‚ब‚न्धो ‚{\color{DodgerBlue3}‚न भ‚वे}‚दिति ‚{\color{DodgerBlue3}‚चेत्} (२०१)। ‚{\color{DodgerBlue3}‚स‚ह‚जाहानेर्न}‚। द्विधा हि स‚त्काय‚{\tiny $_{lb}$}‚दृष्टिराभिसाँस्कारिकी या स्क‚न्ध‚व्य‚तिरिक्तात्माध्य‚व‚सायिनी, स‚ह‚जा च । त‚त्र ‚{\tiny $_{lb}$}‚प्र‚थ‚मा द‚र्श‚न‚मार्ग्गे हीय‚ते ‚{\color{DodgerBlue3}‚न} द्वितीया भाव‚नामार्ग‚हेया । सा च मोहः तृष्णायाश्च ‚{\tiny $_{lb}$}‚हेतुरिति भ‚व‚ति ज‚न्म‚प्र‚ब‚न्धः । य‚दि तु प‚टुत‚र‚प्र‚ज्ञ‚स्याद्य एव मार्ग्गो मार्ग्गान्त‚र‚{\tiny $_{lb}$}‚स्व‚भावः त‚दा ‚{\color{DodgerBlue3}‚हानौ वा} स‚ह‚जाया आत्म‚दृष्टेः पुन‚{\color{DodgerBlue3}‚र्भ‚वः कुत}‚{\tiny $_{6}$}‚ः ।
	\pend% ending standard par
      

	  \pstart \leavevmode% starting standard par
	b. कीदृशं पुन‚स्त‚त्स‚ह‚ज‚स‚त्त्व‚द‚र्श‚न‚मित्याह (।)
	\pend% ending standard par
      
	  \bigskip
	  \begingroup
	
	    \large
	  
	    \begin{quote}
	  
	    
	    \stanza[\smallbreak]
	\label{pv.1.202b}\flagstanza{\tiny\textenglish{...1.202b}}सुखी भ‚वेयं दुःखी वा मा भूव‚मिति तृष्य‚तः ॥ २०२ ॥\&[\smallbreak]


	
	    \end{quote}
	  
	  \endgroup
	
	  \bigskip
	  \begingroup
	
	    \large
	  
	    \begin{quote}
	  
	    
	    \stanza[\smallbreak]
	\label{pv.1.203a}\flagstanza{\tiny\textenglish{...1.203a}}यैवाऽह‚मिति धीः सैव स‚ह‚जं स‚त्त्व‚द‚र्श‚न‚म् ।\&[\smallbreak]


	
	    \end{quote}
	  
	  \endgroup
	\textsuperscript{\textenglish{080/s}}

	  \pstart \leavevmode% starting standard par
	\hphantom{.}‚{\color{DodgerBlue3}‚सुखी भ‚वेयं दुःखी वा माभूव‚मिति तृष्य‚तः} । कांक्ष‚माण‚स्य ‚{\color{DodgerBlue3}‚यैवाह‚मिति धीः ‚{\tiny $_{lb}$}‚सैव स‚ह‚जं स‚त्त्व‚द‚र्श‚न}‚मुच्य‚ते ।
	\pend% ending standard par
      

	  \pstart \leavevmode% starting standard par
	c. त‚द‚प्र‚हीणं स्रोत-आप‚न्न‚स्येति क‚थं ज्ञाय‚त इत्याह (।)
	\pend% ending standard par
      
	  \bigskip
	  \begingroup
	
	    \large
	  
	    \begin{quote}
	  
	    
	    \stanza[\smallbreak]
	\label{pv.1.203b}\flagstanza{\tiny\textenglish{...1.203b}}न ह्य‚प‚श्य‚न्न‚ह‚मिति क‚श्चिदात्म‚नि स्निह्य‚ति ॥ २०३ ॥\&[\smallbreak]


	
	    \end{quote}
	  
	  \endgroup
	
	  \bigskip
	  \begingroup
	
	    \large
	  
	    \begin{quote}
	  
	    
	    \stanza[\smallbreak]
	\label{pv.1.204a}\flagstanza{\tiny\textenglish{...1.204a}}न चात्म‚नि विना प्रेम्णा सुख‚कामोऽभिधाव‚ति ।\&[\smallbreak]


	
	    \end{quote}
	  
	  \endgroup
	

	  \pstart \leavevmode% starting standard par
	\hphantom{.}‚{\color{DodgerBlue3}‚न‚ह्य‚प‚श्य‚न्न‚ह‚मिति क‚श्चिदात्म‚नि स्निह्य‚ति} किन्तु प‚श्य‚न्नेव (२०३) ।
	\pend% ending standard par
      

	  \pstart \leavevmode% starting standard par
	\hphantom{.}‚{\color{DodgerBlue3}‚न चात्म‚नि विना प्रेम्णा सुख‚कामः} क‚श्चिद् ग‚र्भ‚स्थानादि‚{\color{DodgerBlue3}‚म‚भिधाव‚ति} । अभिधाव‚ति ‚{\tiny $_{lb}$}‚\leavevmode\ledsidenote{\textenglish{15b/MA}} च ग‚र्भ‚स्थानं (।) प्र‚हीणाभिसंस्कारिक स‚त्त्व‚दृष्टिर‚पि स्रोत आप‚न्नं‚{\tiny $_{7}$}‚ त‚स्याप्र‚हीणं ‚{\tiny $_{lb}$}‚स‚ह‚जं स‚त्त्व‚द‚र्श‚नं ।
	\pend% ending standard par
      

	  \begin{center}%% label @type='head'
	\textbf{(IV. बंध‚मोक्ष‚व्य‚व‚स्था)}
	\end{center}
	

	  \pstart \leavevmode% starting standard par
	न‚नु स‚त्यात्म‚नि ब‚न्ध‚मोक्षावेकाधिक‚र‚णौ युक्तौ । नेत्याह ।\edtext{\textsuperscript{*}}{\edlabel{pvv.80-1}\label{pvv.80-1}\lemma{*}\Bfootnote{स‚त्य‚म‚प्यात्म‚नि ब‚न्धाद्य‚भाव‚माह ।}}
	\pend% ending standard par
      
	  \bigskip
	  \begingroup
	
	    \large
	  
	    \begin{quote}
	  
	    
	    \stanza[\smallbreak]
	\label{pv.1.204b}\flagstanza{\tiny\textenglish{...1.204b}}दुःख‚स्योत्पाद‚हेतुत्वं ब‚न्धः, नित्य‚स्य त‚त् कुतः ॥ २०४ ॥\&[\smallbreak]


	
	    \end{quote}
	  
	  \endgroup
	
	  \bigskip
	  \begingroup
	
	    \large
	  
	    \begin{quote}
	  
	    
	    \stanza[\smallbreak]
	\label{pv.1.205a}\flagstanza{\tiny\textenglish{...1.205a}}अदुःखोत्पाद‚हेतुत्वं मोक्षः, नित्य‚स्य त‚त् कुतः ।\&[\smallbreak]


	
	    \end{quote}
	  
	  \endgroup
	

	  \pstart \leavevmode% starting standard par
	\hphantom{.}a. ‚{\color{DodgerBlue3}‚दुःख‚स्यो}‚पादान‚स्क‚न्धाना‚{\color{DodgerBlue3}‚मुत्पाद‚हेतुत्वं} ब‚न्धः । ‚{\color{DodgerBlue3}‚त‚त्कुतो नित्य‚स्य} क्र‚म‚यौग‚प‚द्या‚{\tiny $_{lb}$}‚भ्याम‚र्थ‚क्रिया विर‚हान् (२०४) । ‚{\color{DodgerBlue3}‚अदुःखोत्पाद‚हेतुत्वं} दुःखं प्र‚त्य‚य‚हेतुता ‚{\color{DodgerBlue3}‚मोक्षः । ‚{\tiny $_{lb}$}‚त‚च्च नित्य\edtext{}{\edlabel{pvv.80-2}\label{pvv.80-2}\lemma{नित्य}\Bfootnote{ब‚न्धासिद्धो ।}} स्य कुतः} । न हि पूर्व्वाप‚रैक‚स्य भाव‚स्य दुःख‚हेतोः प‚श्चाद‚हेतुत्वं युक्तं ।
	\pend% ending standard par
      

	  \pstart \leavevmode% starting standard par
	b. स्यादेत‚त् (।) न नित्य‚स्य हेतुत्वं ब‚न्ध‚मोक्षौ च युक्ताविति । नित्य‚त्वानित्य‚{\tiny $_{1}$}‚ ‚{\tiny $_{lb}$}‚‚{\color{DodgerBlue3}‚त्वाभ्याम‚वाच्य‚स्य पुद्ग‚ल‚स्य} तौ भ‚विष्य‚त इति म‚न्वानं वै भा षि क म्प्र‚{\tiny $_{lb}$}‚त्याह (।)
	\pend% ending standard par
      
	  \bigskip
	  \begingroup
	
	    \large
	  
	    \begin{quote}
	  
	    
	    \stanza[\smallbreak]
	\label{pv.1.205b}\flagstanza{\tiny\textenglish{...1.205b}}अनित्य‚त्वेन योऽवाच्यः स हेतुर्न हि क‚स्य‚चित् ॥ २०५ ॥\&[\smallbreak]


	
	    \end{quote}
	  
	  \endgroup
	

	  \pstart \leavevmode% starting standard par
	\hphantom{.}‚{\color{DodgerBlue3}‚अनित्य‚त्वेन योऽवाच्यः} । अनित्य‚त्व‚मुप‚ल‚क्ष‚णं । नित्य‚त्वेनाप्य‚वाच्यः ‚{\color{DodgerBlue3}‚स‚हेतुर्न हि ‚{\tiny $_{lb}$}‚क‚स्य‚चित्} । त‚स्मैवाभावात् । त‚थाहि य‚द्य‚साव‚नित्यो न भ‚व‚ति स्यान्नित्यः । (२०५)
	\pend% ending standard par
      \label{div_pvv.1.206}
	  
	% new div opening: depth here is 2
	

	  \pstart \leavevmode% starting standard par
	अथ नित्यो न भ‚व‚ति स्याद‚नित्यः । अन्योन्याभाव‚ल‚क्ष‚ण‚त्वाद‚न‚योरेक‚विधि ‚{\tiny $_{lb}$}‚प्र‚तिषेध‚स्याप‚र‚प्र‚तिष‚ध‚विधिनान्त‚रीय‚क‚त्वात् न क्व‚चिद्व‚स्तुनि द्व‚य‚प्र‚तिषेध‚स‚म्भ‚व ‚{\tiny $_{lb}$}‚इति पुद्ग‚ल‚स्याभावाद‚हेतु\edtext{}{\edlabel{pvv.80-2-bis}\label{pvv.80-2-bis}\lemma{हेतु}\Bfootnote{ब‚न्धासिद्धो ।}}त्वं अतो ‚{\color{DodgerBlue3}‚ब‚न्ध‚मोक्षाव‚प्य‚वाच्ये} पुद्ग‚ले ‚{\color{DodgerBlue3}‚न युज्येते ‚{\tiny $_{lb}$}‚क‚थ‚ञ्च‚न} ।
	\pend% ending standard par
      \textsuperscript{\textenglish{081/s}}

	  \pstart \leavevmode% starting standard par
	C. अथ नित्य‚त्वेनावाच्य‚त्वान्न दोष इति चेत् । न‚न्वेव‚म‚नित्य एवोक्तः स्यात् । ‚{\tiny $_{lb}$}‚त‚थाहि\edtext{}{\edlabel{pvv.81-1}\label{pvv.81-1}\lemma{थाहि}\Bfootnote{न नित्य‚म‚न्य‚देव किञ्चिद्याव‚ता ।}} (।)
	\pend% ending standard par
      
	  \bigskip
	  \begingroup
	
	    \large
	  
	    \begin{quote}
	  
	    
	    \stanza[\smallbreak]
	\label{pv.1.206}\flagstanza{\tiny\textenglish{....1.206}}नित्यं त‚माहुर्विद्वांसो यः स्व‚भावो न न‚श्य‚ति ॥ २०६ ॥\&[\smallbreak]


	
	    \end{quote}
	  
	  \endgroup
	

	  \pstart \leavevmode% starting standard par
	\hphantom{.}‚{\color{DodgerBlue3}‚नित्यं त‚माहुर्व्विद्वांसो यः स्व‚भावो:} स‚र्व‚दा ‚{\color{DodgerBlue3}‚न न‚श्य‚ति} स चेदीदृशो न भ‚व‚त्य‚{\tiny $_{lb}$}‚नित्य एव स्यात् । (२०६)
	\pend% ending standard par
      \label{div_pvv.1.207}
	  
	% new div opening: depth here is 2
	
	  \bigskip
	  \begingroup
	
	    \large
	  
	    \begin{quote}
	  
	    
	    \stanza[\smallbreak]
	\label{pv.1.207a}\flagstanza{\tiny\textenglish{...1.207a}}त्य‚क्त्वेमां ह्रेप‚णीं दृष्टिम‚तोऽनित्यः स उच्य‚ताम् ॥\&[\smallbreak]


	
	    \end{quote}
	  
	  \endgroup
	

	  \pstart \leavevmode% starting standard par
	\hphantom{.}अक्षे‚{\color{DodgerBlue3}‚मां} पुद्ग‚ल‚{\color{DodgerBlue3}‚दृष्टिं} पुद्‏ग‚ल‚नैरात्म्य‚वादिनः संस्कारानित्य‚तावा‚{\color{DodgerBlue3}‚दिन‚श्च शास्तुः} शिष्याणां ‚{\color{DodgerBlue3}‚ह्रेप‚णीं} ल‚ज्जाव‚र्द्ध‚नीं ‚{\color{DodgerBlue3}‚त्य‚क्त्वाऽनित्यः स} पुद्ग‚ल ‚{\color{DodgerBlue3}‚उच्य‚तां} येन ‚{\color{DodgerBlue3}‚ब‚न्ध‚{\tiny $_{3}$}‚मोक्षौ} युज्येते । एकाकारो निरोधो व्याख्यातः । X ॥ X ।
	\pend% ending standard par
      

	  \begin{center}%% label @type='head'
	\textbf{(घ) मार्ग‚स‚त्य‚म्}
	\end{center}
	

	  \begin{center}%% label @type='head'
	\textbf{(च‚तुराकारं मार्ग‚स‚त्य‚म्)}
	\end{center}
	

	  \pstart \leavevmode% starting standard par
	त्र‚य आकारा व‚क्ष्य‚न्ते ।
	\pend% ending standard par
      
	  \bigskip
	  \begingroup
	
	    \large
	  
	    \begin{quote}
	  
	    
	    \stanza[\smallbreak]
	\label{pv.1.207b}\flagstanza{\tiny\textenglish{...1.207b}}उक्तो मार्गः, त‚द‚भ्यासादाश्र‚यः प‚रिव‚र्त‚ते ॥ २०७ ॥\&[\smallbreak]


	
	    \end{quote}
	  
	  \endgroup
	

	  \pstart \leavevmode% starting standard par
	मार्ग‚स‚त्यं च‚तुराकारं व‚क्तुमाह (।)
	\pend% ending standard par
      

	  \pstart \leavevmode% starting standard par
	\hphantom{.}a. ‚{\color{DodgerBlue3}‚उक्तो मार्गः} शास्तृप‚द‚व्याख्याव‚स‚रे\edtext{}{\edlabel{pvv.81-2}\label{pvv.81-2}\lemma{रे}\Bfootnote{उपायाभ्यास एवाय‚न्तादात्म्याच्छास‚नं म‚त‚मि \href{http://sarit.indology.info/?cref=pv.1.140}{(१।१४०)} त्य‚नेन भ्रान्तिनिव‚र्त‚काग‚म‚न‚काल‚मात्र‚म‚स्याव‚स्थानात् ।}} नैरात्म्य‚द‚र्श‚न‚ल‚क्ष‚णः । ‚{\color{DodgerBlue3}‚त‚स्याभ्यासादा‚{\tiny $_{lb}$}‚श्र‚यः} क्लेश‚वास‚नाभूत‚या ल य वि ज्ञा नं ‚{\color{DodgerBlue3}‚प‚रिव‚र्त‚ते} क्लिष्ट‚द‚शानिरोधात् क्लेश‚विसं‚{\tiny $_{lb}$}‚युक्त‚चित्त‚प्र‚ब‚न्धात्म‚ना प‚रिण‚म‚ति । (२०७)
	\pend% ending standard par
      \label{div_pvv.1.208_1.209_1.210}
	  
	% new div opening: depth here is 2
	

	  \pstart \leavevmode% starting standard par
	अनेन मार्ग‚त इति मार्गाकारो द‚र्शितः क्लेश‚विसंयोग‚हेतुत्वा‚{\tiny $_{4}$}‚त् ।
	\pend% ending standard par
      
	  \bigskip
	  \begingroup
	
	    \large
	  
	    \begin{quote}
	  
	    
	    \stanza[\smallbreak]
	\label{pv.1.208a}\flagstanza{\tiny\textenglish{...1.208a}}सात्म्येऽपि दोष‚भाव‚श्चेन्मार्ग‚व‚त् ।\&[\smallbreak]


	
	    \end{quote}
	  
	  \endgroup
	

	  \pstart \leavevmode% starting standard par
	\hphantom{.}मार्ग‚स्याभ्यास‚प्र‚क‚र्षात् ‚{\color{DodgerBlue3}‚सात्म्येपि} प्र‚कृतित्वे च प्राप्ते पुन‚{\color{DodgerBlue3}‚र्दोषा}‚णां मोहादीनां ‚{\tiny $_{lb}$}‚‚{\color{DodgerBlue3}‚भावः} प्राप्नोति ‚{\color{DodgerBlue3}‚चेत् । मार्ग‚व‚त्} य‚था ब‚न्धाव‚स्थायां दोष‚सात्म्येपि मार्गोऽभ्यास‚{\tiny $_{lb}$}‚व‚शादाविर्भ‚व‚ति ।
	\pend% ending standard par
      

	  \pstart \leavevmode% starting standard par
	अत्राह (।)
	\pend% ending standard par
      
	  \bigskip
	  \begingroup
	
	    \large
	  
	    \begin{quote}
	  
	    
	    \stanza[\smallbreak]
	\label{pv.1.208b}\flagstanza{\tiny\textenglish{...1.208b}}नाविभुत्व‚तः ।\&[\smallbreak]


	
	    \end{quote}
	  
	  \endgroup
	\textsuperscript{\textenglish{082/s}}

	  \pstart \leavevmode% starting standard par
	\hphantom{.}‚{\color{DodgerBlue3}‚नाविभुत्व‚तः} असाम‚र्थ्यात् । मार्ग‚सात्म्येऽपि स्थित‚स्य चेत‚सि न दोषाणा‚{\tiny $_{lb}$}‚मुत्प‚त्तुं साम‚र्थ्य\edtext{}{\edlabel{pvv.82-1}\label{pvv.82-1}\lemma{र्थ्य}\Bfootnote{प्र‚कृतिशुद्ध‚हेम‚व‚त् ।}}म‚स्ति । त‚न्निदान‚भूत‚स्य स‚त्व‚द‚र्श‚न‚स्योन्मूलित‚त्वात् ।
	\pend% ending standard par
      

	  \pstart \leavevmode% starting standard par
	\hphantom{.}‚{\color{DodgerBlue3}‚अनेन} शान्त‚त इति निरोधाकार उक्तो दोषाणां स‚र्व्व‚था शान्त‚त्वात् ।
	\pend% ending standard par
      

	  \begin{center}%% label @type='head'
	\textbf{(ग. स‚त्काय‚दृष्टिः)}
	\end{center}
	

	  \begin{center}%% label @type='head'
	\textbf{(क) स‚त्व‚द‚र्श‚नाभावे कार‚ण‚म्}
	\end{center}
	

	  \pstart \leavevmode% starting standard par
	स‚त्व‚द‚र्श‚न‚मेव पुनः क‚स्मान्न भ‚व‚तीत्याह (।)
	\pend% ending standard par
      
	  \bigskip
	  \begingroup
	
	    \large
	  
	    \begin{quote}
	  
	    
	    \stanza[\smallbreak]
	\label{pv.1.208c}\flagstanza{\tiny\textenglish{...1.208c}}विष‚य‚ग्र‚ह‚णं ध‚र्मों विज्ञान‚स्य य‚थास्ति सः ॥ २०८ ॥\&[\smallbreak]


	
	    \end{quote}
	  
	  \endgroup
	
	  \bigskip
	  \begingroup
	
	    \large
	  
	    \begin{quote}
	  
	    
	    \stanza[\smallbreak]
	\label{pv.1.209a}\flagstanza{\tiny\textenglish{...1.209a}}गृह्य‚ते सोऽस्य ज‚न‚को विद्य‚मानात्म‚नेति च ।\&[\smallbreak]


	
	    \end{quote}
	  
	  \endgroup
	

	  \pstart \leavevmode% starting standard par
	\hphantom{.}‚{\color{DodgerBlue3}‚विष‚य‚ग्र‚ह‚ण‚म्विज्ञान‚स्य} ताव‚त्स‚ति ग्राह्य‚ग्राह‚क‚भावो ‚{\color{DodgerBlue3}‚ध‚र्म\edtext{}{\edlabel{pvv.82-2}\label{pvv.82-2}\lemma{र्म}\Bfootnote{अर्थ‚ग्राहिज्ञान‚मिच्छ‚तो व‚क्त‚व्यः ।}}: य‚था} चास्ति ‚{\color{DodgerBlue3}‚स} स विष‚य‚स्त‚था ‚{\color{DodgerBlue3}‚गृह्य‚ते} । विज्ञानेन विष‚यिणा (२०८) । ‚{\color{DodgerBlue3}‚स} च विष‚योस्य विज्ञान‚स्य ‚{\tiny $_{lb}$}‚‚{\color{DodgerBlue3}‚ज‚न‚को विद्य‚मानेनात्म‚ना} य‚थाव‚स्थितेन रूपेण ।
	\pend% ending standard par
      
	  \bigskip
	  \begingroup
	
	    \large
	  
	    \begin{quote}
	  
	    
	    \stanza[\smallbreak]
	\label{pv.1.209b}\flagstanza{\tiny\textenglish{...1.209b}}एषा प्र‚कृतिर‚स्यास्त‚न्निमित्तान्त‚र‚तः स्ख‚ल‚त् ॥ २०९ ॥\&[\smallbreak]


	
	    \end{quote}
	  
	  \endgroup
	
	  \bigskip
	  \begingroup
	
	    \large
	  
	    \begin{quote}
	  
	    
	    \stanza[\smallbreak]
	\label{pv.1.210a}\flagstanza{\tiny\textenglish{...1.210a}}व्यावृत्तौ प्र‚त्य‚यापेक्ष‚म‚दृढं स‚र्प‚बुद्धिव‚त् ।\&[\smallbreak]


	
	    \end{quote}
	  
	  \endgroup
	

	  \pstart \leavevmode% starting standard par
	\hphantom{.}य‚थाव‚स्थित‚व‚स्तुग्र‚ह‚ण‚ञ्च ‚{\color{DodgerBlue3}‚नात्मैषा प्र‚कृतिः} स्व‚भावो ज्ञान‚स्य विष‚यिणः । ‚{\tiny $_{lb}$}‚य‚थास्व‚भावं स्व‚ग्राहिज्ञान‚ज‚न‚न‚ञ्च विष‚य‚स्य प्र‚कृतिः । अस्याः प्र‚कृतेस्त‚ज्ज्ञानं ‚{\tiny $_{lb}$}‚‚{\color{DodgerBlue3}‚निमित्तान्त‚र‚तः}‚{\tiny $_{6}$}‚ आन्त‚राद‚विद्यारूपादाग‚न्तुकाच्च विष‚य‚दोषादेः ‚{\color{DodgerBlue3}‚स्ख‚ल‚द्वि}‚ष‚य‚ग्र‚ह‚ण‚{\tiny $_{lb}$}‚‚{\color{DodgerBlue3}‚विप‚रीताकारं\edtext{}{\edlabel{pvv.82-3}\label{pvv.82-3}\lemma{रीताकारं}\Bfootnote{स‚त्वाध्य‚व‚सायि ।}}} भ‚व‚ति (। २०९) ‚{\color{DodgerBlue3}‚त‚च्च व्यावृत्तौ} विष‚य‚विप‚रीत‚ग्र‚ह‚णाकार‚ताया ‚{\tiny $_{lb}$}‚निवृत्त्य‚र्थ ‚{\color{DodgerBlue3}‚प्र‚त्य‚यापेक्षं} भ्रान्तिनिव‚र्त‚क‚कार‚ण‚म‚पेक्ष्य‚माण‚{\color{DodgerBlue3}‚म‚दृढ‚म}‚स्था\edtext{}{\edlabel{pvv.82-4}\label{pvv.82-4}\lemma{स्था}\Bfootnote{म‚न्द‚म‚न्द‚प्र‚काशे स‚र्प्पोचिते प्र‚देशे ।}}ष्णु ‚{\color{DodgerBlue3}‚स‚र्प‚बुद्धिव‚त्} । ‚{\tiny $_{lb}$}‚य‚था स‚र्प‚बुद्धी र‚ज्वा भ्रान्तिनिमित्ता\edtext{}{\edlabel{pvv.82-5}\label{pvv.82-5}\lemma{भ्रान्तिनिमित्ता}\Bfootnote{अस‚ह‚जाग‚न्तुर्दोषः । नैरात्म्यं प्र‚कृतिः प्र‚मासिद्धिः ।}} जाता र‚ज्जुस्व‚रूप‚ग्राहिणः प्र‚त्य‚यान्निवृत्ता न ‚{\tiny $_{lb}$}‚पुन‚रुद्भ‚व‚तीति । त‚था भ्रान्तिनिमित्त‚निरासात् दृष्टे नैरात्म्ये व‚स्तुनि स‚ति नास्ति ‚{\tiny $_{lb}$}‚\leavevmode\ledsidenote{\textenglish{16a/MA}} स‚त्त्वे दृष्टिस‚म्भ‚वः‚{\tiny $_{7}$}‚ ज्ञान‚स्य विष‚य‚स्व‚रूप‚ग्र‚ह‚ण‚प्र‚व‚ण‚त्वात् । विष‚य‚स्य च स्वाकारा‚{\tiny $_{lb}$}‚र्प‚ण‚प्र‚वृत्त‚त्वात् ।
	\pend% ending standard par
      

	  \pstart \leavevmode% starting standard par
	a. किञ्च (।)
	\pend% ending standard par
      
	  \bigskip
	  \begingroup
	
	    \large
	  
	    \begin{quote}
	  
	    
	    \stanza[\smallbreak]
	\label{pv.1.210b}\flagstanza{\tiny\textenglish{...1.210b}}प्र‚भास्व‚र‚मिद‚ञ्चित्तं प्र‚कृत्याग‚न्त‚वो म‚लाः ॥ २१० ॥\&[\smallbreak]


	
	    \end{quote}
	  
	  \endgroup
	

	  \pstart \leavevmode% starting standard par
	\hphantom{.}‚{\color{DodgerBlue3}‚प्र‚भास्व‚र‚म‚ना}‚त्म‚भूत‚दोष‚स‚ञ्च‚य‚{\color{DodgerBlue3}‚मिदं चित्तं प्र‚कृत्या} स्व‚भावेन । ये तु म‚नोदोषा ‚{\tiny $_{lb}$}‚\leavevmode\ledsidenote{\textenglish{083/s}} दृश्य‚न्ते ते भ्रान्तिनिमित्तोप‚नीत‚त्वादा‚{\color{DodgerBlue3}‚ग‚न्त‚वो}‚ऽस्व‚भाव‚भूताश्चेत‚सः । त‚मः ‚{\color{DodgerBlue3}‚तुहिनाद‚य} इव न‚भ‚सः (२१०) ।
	\pend% ending standard par
      \label{div_pvv.1.211_1.212_1.213_1.214_1.215_1.216_1.217_1.218_1.219_1.220_1.221_1.222}
	  
	% new div opening: depth here is 2
	
	  \bigskip
	  \begingroup
	
	    \large
	  
	    \begin{quote}
	  
	    
	    \stanza[\smallbreak]
	\label{pv.1.211a}\flagstanza{\tiny\textenglish{...1.211a}}त‚त्प्राग‚प्य‚स‚म‚र्थानां प‚श्चाच्छ‚क्तिः क त‚न्म‚ये ॥\&[\smallbreak]


	
	    \end{quote}
	  
	  \endgroup
	

	  \pstart \leavevmode% starting standard par
	\hphantom{.}‚{\color{DodgerBlue3}‚त‚त्प्राग‚पि} त‚स्मान्नैरात्म्य‚द‚र्श‚नात्पूर्व्व‚म‚पि त‚तः श्रुत‚चिन्ताध्य‚व‚सानेप्यापात‚{\tiny $_{lb}$}‚विष्क‚म्भ‚नादुत्प‚त्तुम‚{\color{DodgerBlue3}‚स‚म‚र्थानां} म‚लानां ‚{\color{DodgerBlue3}‚प‚श्चा}‚न्मार्ग‚निष्प‚त्तौ ‚{\color{DodgerBlue3}‚श‚क्ति}‚रुत्प‚त्तुं ‚{\color{DodgerBlue3}‚क्व त‚न्म‚ये} मार्ग‚सात्म्ये‚{\tiny $_{1}$}‚ स्व‚रूपे (।)
	\pend% ending standard par
      

	  \pstart \leavevmode% starting standard par
	एत‚देवाह (।)
	\pend% ending standard par
      
	  \bigskip
	  \begingroup
	
	    \large
	  
	    \begin{quote}
	  
	    
	    \stanza[\smallbreak]
	\label{pv.1.211b}\flagstanza{\tiny\textenglish{...1.211b}}नालं प्र‚रोढुम‚त्य‚न्तं स्य‚न्दिन्याम‚ग्निव‚द् भुवि ॥ २११ ॥\&[\smallbreak]


	
	    \end{quote}
	  
	  \endgroup
	
	  \bigskip
	  \begingroup
	
	    \large
	  
	    \begin{quote}
	  
	    
	    \stanza[\smallbreak]
	\label{pv.1.212a}\flagstanza{\tiny\textenglish{...1.212a}}बाध‚कोत्प‚त्तिसाम‚र्थ्य‚ग‚र्भे श‚क्तोऽपि व‚स्तुनि ।\&[\smallbreak]


	
	    \end{quote}
	  
	  \endgroup
	

	  \pstart \leavevmode% starting standard par
	\hphantom{.}श‚क्तोऽप्य‚द्ध‚तोपि म‚नोम‚लो ‚{\color{DodgerBlue3}‚नाल‚म‚त्य‚न्तं प्र‚रोढुं व‚स्तुनि} चित्त‚स‚न्ताने कीदृशे ‚{\tiny $_{lb}$}‚बाध‚कोत्प‚त्तिसाम‚र्थ्य‚ग‚र्भे स‚र्व‚दृष्टि‚{\color{DodgerBlue3}‚बाध‚क}‚स्य नैरात्म्य‚द‚र्श‚न‚स्य मार्ग‚स‚त्यो‚{\color{DodgerBlue3}‚त्प‚त्तिस्त‚स्यां ‚{\tiny $_{lb}$}‚साम‚र्थ्यं} प्र‚भ‚विष्णुत्वं ‚{\color{DodgerBlue3}‚त‚द्ग‚र्भे} आन‚न्त‚र्य‚स्य ग‚र्भे त‚स्मिन् ‚{\color{DodgerBlue3}‚स्य‚न्दिन्यां\edtext{}{\edlabel{pvv.83-1}\label{pvv.83-1}\lemma{न्दिन्यां}\Bfootnote{नित्य‚मुद‚क‚स्राविण्यां ।}} भुवि अग्निव‚त्} (२११) । य‚था हि व‚ह्नेर्हेतुब‚लादुत्प‚न्नोपि स्य‚न्दिन्यां भुवि बाध‚क‚व‚त्यां नात्य‚न्तं ‚{\tiny $_{lb}$}‚प्र‚रोह‚ति त‚था मार्गोत्प‚त्तिसाम‚र्थ्य‚ग‚र्भे चेत‚स्युत्प‚न्ना अपि म‚ला नात्य‚न्तं विरोह‚न्ति । ‚{\tiny $_{lb}$}‚सात्मीभूत‚मा‚{\tiny $_{2}$}‚र्ग्गें तु हेतुवैक‚ल्यान्नोत्प‚द्य‚न्त एव ।
	\pend% ending standard par
      

	  \begin{center}%% label @type='head'
	\textbf{(ख) नैरात्म्य‚द‚र्श‚ने दोषोत्प‚त्तिव्य‚व‚स्था}
	\end{center}
	

	  \pstart \leavevmode% starting standard par
	क‚थ‚ञ्च नैरात्म्य‚द‚र्शिनो म‚लोत्प‚त्तिराशंक्य‚ते । न ताव‚त् हेतुसाक‚ल्यात्स‚त्व‚{\tiny $_{lb}$}‚दृष्टेर्हेतुभूताया अभावात् नैरात्म्य‚द‚र्श‚न‚व‚त् ‚{\color{DodgerBlue3}‚स‚त्त्व‚द‚र्श‚न‚ञ्च भाव‚न‚योत्प‚न्नं हेतुरिति} चेत् ॥
	\pend% ending standard par
      

	  \pstart \leavevmode% starting standard par
	\hphantom{.}कुत‚स्त‚स्य ‚{\color{DodgerBlue3}‚भाव‚ना} । किन्नैरात्म्य‚स्य सो\edtext{}{\edlabel{pvv.83-2}\label{pvv.83-2}\lemma{सो}\Bfootnote{नैरात्म्ये य‚दि दोषः स्यात्त‚दा स‚त्त्व‚द‚र्श‚नं भाव‚येत् । त‚च्च नास्ति दोष‚त्र‚य‚स्याप्य‚भावात् ।}}प‚द्र‚व‚त्वात् । किञ्चाभूत‚त्वेन भ्र‚म‚{\tiny $_{lb}$}‚हेतुत्वात् । अस्व‚भावेनोप‚ह‚न्तुं श‚क्य‚त्वाद्वा । एत‚त्त्र‚य‚म‚स‚ङ्ग‚त‚मित्याह (।)
	\pend% ending standard par
      
	  \bigskip
	  \begingroup
	
	    \large
	  
	    \begin{quote}
	  
	    
	    \stanza[\smallbreak]
	\label{pv.1.212b}\flagstanza{\tiny\textenglish{...1.212b}}निरुप‚द्र‚व‚भूतार्थ‚स्व‚भाव‚स्य विप‚र्य‚यैः ॥ २१२ ॥\&[\smallbreak]


	
	    \end{quote}
	  
	  \endgroup
	
	  \bigskip
	  \begingroup
	
	    \large
	  
	    \begin{quote}
	  
	    
	    \stanza[\smallbreak]
	\label{pv.1.213a}\flagstanza{\tiny\textenglish{...1.213a}}न बाधा य‚त्न‚व‚त्वेऽपि बुद्धेस्त‚त्प‚क्ष‚पात‚तः ॥\&[\smallbreak]


	
	    \end{quote}
	  
	  \endgroup
	

	  \pstart \leavevmode% starting standard par
	\hphantom{.}दोष‚राशेरुद्वेज‚क‚स्य प्र‚हाणेन ‚{\color{DodgerBlue3}‚निरुप‚द्र‚व‚स्य} प्र‚माण‚स‚म्वादि‚{\tiny $_{3}$}‚त्वेन ‚{\color{DodgerBlue3}‚भूतार्थ‚स्य} स‚त्यार्थ‚स्यानारोपित‚त्वेन ‚{\color{DodgerBlue3}‚स्व‚भाव‚स्य} प्र‚कृतेर्नैरात्म्य‚स्याभिरुचित‚विष‚य‚स्य विप‚र्य‚ये‚{\tiny $_{lb}$}‚ष्वात्माद्याकारेष्व‚भ्यासे सोप‚द्र‚व‚त्वादिना ‚{\color{DodgerBlue3}‚प्र‚य‚त्न एव} ताव‚न्न ‚{\color{DodgerBlue3}‚स‚म्भ‚व‚ति प्रेक्ष‚स्य} । ‚{\tiny $_{lb}$}‚स‚म्भ‚वेपि वा ‚{\color{DodgerBlue3}‚विप‚र्य‚यैः} (। २१२)
	\pend% ending standard par
      \textsuperscript{\textenglish{084/s}}

	  \pstart \leavevmode% starting standard par
	\hphantom{.}a. ‚{\color{DodgerBlue3}‚न बाधा} नैरात्म्य‚स्य सात्मीभूत‚स्य स्व‚भाव‚स्यास्ति । \edtext{\textsuperscript{*}}{\edlabel{pvv.84-1}\label{pvv.84-1}\lemma{*}\Bfootnote{चित्त‚संप्र‚युक्त‚त्वात् चैत्तानां संप्र‚युक्ताऽविद्येत्याह ।}} ‚{\color{DodgerBlue3}‚बुद्धेस्त‚त्र दोष‚प्र‚ति‚{\tiny $_{lb}$}‚प‚क्षे गुण‚व‚ति मार्ग्गे प‚क्ष‚पातात्} ।
	\pend% ending standard par
      

	  \pstart \leavevmode% starting standard par
	न हि स्व‚भावः साक्षात्कृतोऽन्य‚था क‚र्तु श‚क्यः । अनेन प्र‚णीत इति निरोधा‚{\tiny $_{lb}$}‚कारो द‚र्शितः ॥
	\pend% ending standard par
      

	  \pstart \leavevmode% starting standard par
	b. न‚नु य‚दि नैरात्म्य‚सात्म‚त्व‚द‚र्श‚न‚योः प‚र‚स्प‚र‚भेदाद् बा\edtext{}{\edlabel{pvv.84-2}\label{pvv.84-2}\lemma{बा}\Bfootnote{विरोधात् ।}}धा त‚दा राग‚{\tiny $_{4}$}‚प्र‚तिघ‚{\tiny $_{lb}$}‚योर‚पि स्या\edtext{}{\edlabel{pvv.84-3}\label{pvv.84-3}\lemma{स्या}\Bfootnote{न चास्त्येक‚भाव‚न‚याऽप‚र‚स्यात्य‚न्त‚निरोधो लोक इत्य‚नेकान्त‚माह ।}}दित्याह (।)
	\pend% ending standard par
      
	  \bigskip
	  \begingroup
	
	    \large
	  
	    \begin{quote}
	  
	    
	    \stanza[\smallbreak]
	\label{pv.1.213b}\flagstanza{\tiny\textenglish{...1.213b}}आत्म‚ग्र‚हैक‚योनित्वात्;&कार्य‚कार‚ण‚भाव‚तः ॥ २१३ ॥\&[\smallbreak]


	
	    \end{quote}
	  
	  \endgroup
	
	  \bigskip
	  \begingroup
	
	    \large
	  
	    \begin{quote}
	  
	    
	    \stanza[\smallbreak]
	\label{pv.1.214a}\flagstanza{\tiny\textenglish{...1.214a}}राग‚प्र‚तिध‚योर्बाधा भेदेऽपि न प‚र‚स्प‚र‚म् ।\&[\smallbreak]


	
	    \end{quote}
	  
	  \endgroup
	

	  \pstart \leavevmode% starting standard par
	\hphantom{.}‚{\color{DodgerBlue3}‚आत्म‚ग्र‚ह एको योनि}‚ष्कार‚णं य‚योस्तौ त‚योर्भाव‚स्त‚स्मादेक‚कार‚ण‚{\color{DodgerBlue3}‚त्वात्} राग‚{\tiny $_{lb}$}‚प्र‚तिध‚योर्भेद\edtext{}{\edlabel{pvv.84-4}\label{pvv.84-4}\lemma{योर्भेद}\Bfootnote{ल‚क्ष‚ण‚भेदेपि नात्य‚न्त‚विरोध‚स्त‚योः ।}}पि न प‚र‚स्प‚रं बाधा रूप‚र‚स‚योरिव । त‚था ‚{\color{DodgerBlue3}‚कार्य‚कार‚ण‚भाव‚तोपि} (२१३) नान्योन्यं ‚{\color{DodgerBlue3}‚बाधा} । त‚थाहि य‚दैक‚स्मिन् राग‚स्त‚दा त‚द‚प‚कारिणि द्वेषः । ‚{\tiny $_{lb}$}‚य‚दा च क्व‚चिद् द्वेष‚स्त‚दा त‚द‚प‚कारिणि रागोपीति प‚र‚स्प‚र‚कार्य‚कार‚ण‚भावान्नास्ति ‚{\tiny $_{lb}$}‚विरोधः । च‚क्षु\edtext{}{\edlabel{pvv.84-5}\label{pvv.84-5}\lemma{क्षु}\Bfootnote{भेदेप्येक‚म‚न्य‚हेतुः ।}}रादिबुद्धीनामिव ।
	\pend% ending standard par
      

	  \pstart \leavevmode% starting standard par
	c. न‚नु द्वेषादिप्र‚तिप‚{\tiny $_{5}$}‚क्षा मैत्र्याद‚यो न च तानुच्छिन्द‚न्तीत्याह (।)
	\pend% ending standard par
      
	  \bigskip
	  \begingroup
	
	    \large
	  
	    \begin{quote}
	  
	    
	    \stanza[\smallbreak]
	\label{pv.1.214b}\flagstanza{\tiny\textenglish{...1.214b}}मोहाविरोधान्मैत्र्यादेर्न्नात्य‚न्तं दोष‚निग्र‚हः ॥ २१४ ॥\&[\smallbreak]


	
	    \end{quote}
	  
	  \endgroup
	
	  \bigskip
	  \begingroup
	
	    \large
	  
	    \begin{quote}
	  
	    
	    \stanza[\smallbreak]
	\label{pv.1.215a}\flagstanza{\tiny\textenglish{...1.215a}}त‚न्मूलाश्च म‚लाः स‚र्वे;\&[\smallbreak]


	
	    \end{quote}
	  
	  \endgroup
	

	  \pstart \leavevmode% starting standard par
	\hphantom{.}‚{\color{DodgerBlue3}‚मोहाविरोधान्मैत्र्यादेः} । मैत्रीक‚रुणाद‚यो मोहाविरोधिनोऽत‚श्च दोष‚कार‚णा‚{\tiny $_{lb}$}‚वैक‚ल्या‚{\color{DodgerBlue3}‚न्नात्य‚न्तं} द्वेषादि‚{\color{DodgerBlue3}‚दोषा}‚णां ‚{\color{DodgerBlue3}‚निग्र‚हो} मैत्र्यादेः । आपात‚विष्क‚म्भ‚न‚मात्र‚न्तु ‚{\tiny $_{lb}$}‚भ‚व‚ति मोह‚स्यान‚पायात् (२१४) । ‚{\color{DodgerBlue3}‚त‚न्मूलाश्च म‚लाः स‚र्व्वे} प्र‚सूय‚न्ते ।
	\pend% ending standard par
      

	  \begin{center}%% label @type='head'
	\textbf{(ग) मोहः स‚त्काय‚दृष्टिः}
	\end{center}
	

	  \pstart \leavevmode% starting standard par
	न‚नु स‚त्त्व‚द‚र्श‚नं दोष‚मूलं न मोह इत्याह (।)
	\pend% ending standard par
      
	  \bigskip
	  \begingroup
	
	    \large
	  
	    \begin{quote}
	  
	    
	    \stanza[\smallbreak]
	\label{pv.1.215b}\flagstanza{\tiny\textenglish{...1.215b}}स च स‚त्काय‚द‚र्श‚न‚म् ।\&[\smallbreak]


	
	    \end{quote}
	  
	  \endgroup
	

	  \pstart \leavevmode% starting standard par
	\hphantom{.}‚{\color{DodgerBlue3}‚स च} मोहः ‚{\color{DodgerBlue3}‚स‚त्काय‚द‚र्श‚नं} ।
	\pend% ending standard par
      \textsuperscript{\textenglish{085/s}}

	  \pstart \leavevmode% starting standard par
	\hphantom{.}न‚नु मोहोऽसंप्र‚ख्यान‚रूपः । ‚{\color{DodgerBlue3}‚स‚त्त्व‚दृष्टिः तु विप‚रीतार्थ‚प्र‚तिप‚त्तिरूपा । त‚त्क‚थं} मोह एव स‚त्त्व‚द‚र्श‚न‚मित्याह (।)
	\pend% ending standard par
      
	  \bigskip
	  \begingroup
	
	    \large
	  
	    \begin{quote}
	  
	    
	    \stanza[\smallbreak]
	\label{pv.1.215c}\flagstanza{\tiny\textenglish{...1.215c}}विद्यायाः प्र‚तिप‚क्ष‚त्वाच्चैत्त‚त्त्वेनोप‚ल‚ब्धितः ॥ २१५ ॥\&[\smallbreak]


	
	    \end{quote}
	  
	  \endgroup
	
	  \bigskip
	  \begingroup
	
	    \large
	  
	    \begin{quote}
	  
	    
	    \stanza[\smallbreak]
	\label{pv.1.216a}\flagstanza{\tiny\textenglish{...1.216a}}मिथ्योप‚ल‚ब्धिर‚ज्ञानं युक्तेश्चान्य‚द्युक्तिम‚त् ॥\&[\smallbreak]


	
	    \end{quote}
	  
	  \endgroup
	

	  \pstart \leavevmode% starting standard par
	\hphantom{.}‚{\color{DodgerBlue3}‚विद्यायाः प्र‚तिप‚क्ष‚त्वात्} । विद्याया नैरात्म्य‚दृष्टेर्व्विप‚क्षोऽविद्या । ‚{\color{DodgerBlue3}‚स चा}‚{\tiny $_{lb}$}‚प्र‚ख्यान\edtext{}{\edlabel{pvv.85-1}\label{pvv.85-1}\lemma{ख्यान}\Bfootnote{अभाव‚स्त‚स्य ।}}मात्र‚म्वा\edtext{}{\edlabel{pvv.85-2}\label{pvv.85-2}\lemma{म्वा}\Bfootnote{त‚द‚र्थः ।}}रूपादि वा न भ‚व‚ति निर्व्वाणेपि ‚{\color{DodgerBlue3}‚त‚योर्भावात् ।\edtext{\textsuperscript{*}}{\edlabel{pvv.85-3}\label{pvv.85-3}\lemma{*}\Bfootnote{विद्याविरुद्धाऽविद्या ।}} किन्तु मिथ्योप}‚{\tiny $_{lb}$}‚ल‚ब्धिर‚ज्ञान‚म‚विद्याऽध‚र्मानृत‚व‚त् । विद्यायाः स‚द‚र्थ‚त्वात्\edtext{}{\edlabel{pvv.85-4}\label{pvv.85-4}\lemma{त्वात्}\Bfootnote{युग‚प‚द‚नुत्प‚त्तिस्तु रागानिमित्ताद् ग्र‚हाघात‚व‚स्तुब‚हुलीक‚र‚णाद्य‚साधार‚णं कार‚ण‚व‚त्वात् ।}} । ‚{\color{DodgerBlue3}‚चैत्त‚त्वेनोप‚ल‚ब्धित‚श्च} । ‚{\tiny $_{lb}$}‚चैत्त‚त्वेन क‚र‚णेनोप‚ल‚ब्धिरूप\edtext{}{\edlabel{pvv.85-5}\label{pvv.85-5}\lemma{ब्धिरूप}\Bfootnote{अविद्यायाः ।}} ‚{\color{DodgerBlue3}‚त्वाच्च} । (२१५) ‚{\color{DodgerBlue3}‚आश्र‚याल‚म्ब‚नाकार‚काल‚द्र‚व्य}‚{\tiny $_{lb}$}‚स‚म‚तादिभिः स‚मं प्र‚युक्ताः संप्र‚युक्ता इति संप्र‚युक्त‚ल‚क्ष‚णं । ‚{\color{DodgerBlue3}‚न चासंप्र‚ख्यान‚स्य} नीरूप‚स्याल‚म्ब‚नाकार‚योग इति ‚{\color{DodgerBlue3}‚मिथ्याज्ञान}‚म‚विद्या । ‚{\color{DodgerBlue3}‚उक्ते}\edtext{\textsuperscript{*}}{\edlabel{pvv.85-6}\label{pvv.85-6}\lemma{*}\Bfootnote{युक्तिं प्र‚तिपाद्याग‚म‚माह ।}}ऽर्थे विरोधात् । ‚{\color{DodgerBlue3}‚भ ग}‚{\tiny $_{lb}$}‚व‚{\tiny $_{7}$}‚ता प्युक्तं याः काश्च‚न लोक‚व्य‚व‚हारोप‚प‚त्त‚यः स‚र्व्वास्ता आत्माभिनिवेश‚तो\leavevmode\ledsidenote{\textenglish{16b/MA}} ‚{\tiny $_{lb}$}‚भ‚व‚न्ति आत्माभिनिवेश‚विग‚म‚तो न भ‚व‚न्ति इत्य‚नेन स‚त्त्व‚दृष्टिरेव ज‚न्म‚हेतुरुक्ता । ‚{\tiny $_{lb}$}‚आत्माभिनिवेश‚ल‚क्ष‚ण‚त्वात्त\edtext{}{\edlabel{pvv.85-7}\label{pvv.85-7}\lemma{त्वात्त}\Bfootnote{अविद्यायाः ।}}स्याः । ‚{\color{DodgerBlue3}‚अतोऽन्य‚द‚संप्र‚ख्यान‚ल‚क्ष‚ण‚म‚ज्ञान‚म‚युक्तं निर्व्वा}‚{\tiny $_{lb}$}‚णेपि त‚त्स\edtext{}{\edlabel{pvv.85-8}\label{pvv.85-8}\lemma{त्स}\Bfootnote{असंप्र‚ख्यान‚स्य स‚त्त्वात् ।}}त्त्वात् ।
	\pend% ending standard par
      

	  \pstart \leavevmode% starting standard par
	a. न‚नु य‚द्य‚विद्या दृष्टिरेव त‚था च दृष्टिसंप्र‚युक्ताऽविद्येति संप्र‚युक्तार्थो ‚{\tiny $_{lb}$}‚न स्यात् । न सा तेनैव स‚म्प्र‚युक्ता किन्तु स‚क‚ल‚क्लेशानुग‚ताऽविद्या\edtext{}{\edlabel{pvv.85-9}\label{pvv.85-9}\lemma{ताऽविद्या}\Bfootnote{त‚स्मात्पूर्व्वोक्तैवाविद्या ।}} (।) स‚त्काय‚दृष्टिस्तु ‚{\tiny $_{lb}$}‚त‚देक‚दे\edtext{}{\edlabel{pvv.85-10}\label{pvv.85-10}\lemma{दे}\Bfootnote{य‚था प‚लाशादियुक्तं व‚नं पाण्याद‚दियुक्तं श‚रीरं ॥}}शः‚{\tiny $_{1}$}‚ त‚त‚श्च‚ग‚माविरोध इत्या\edtext{}{\edlabel{pvv.85-11}\label{pvv.85-11}\lemma{इत्या}\Bfootnote{इत्य‚भिप्राय आह ।}}ह । (।)
	\pend% ending standard par
      
	  \bigskip
	  \begingroup
	
	    \large
	  
	    \begin{quote}
	  
	    
	    \stanza[\smallbreak]
	\label{pv.1.216b}\flagstanza{\tiny\textenglish{...1.216b}}व्याख्येयोऽत्र विरोधो यः;\&[\smallbreak]


	
	    \end{quote}
	  
	  \endgroup
	

	  \pstart \leavevmode% starting standard par
	\hphantom{.}‚{\color{DodgerBlue3}‚अत्र} विद्यानिर्देशे आग‚म‚{\color{DodgerBlue3}‚विरोध यः} प्र‚स‚ज‚ति ‚{\color{DodgerBlue3}‚स सा\edtext{}{\edlabel{pvv.85-12}\label{pvv.85-12}\lemma{सा}\Bfootnote{अविद्या ।}}मान्य‚विशे\edtext{}{\edlabel{pvv.85-13}\label{pvv.85-13}\lemma{विशे}\Bfootnote{दृष्टि ।}}ष‚भावेन} भेद‚क‚ल्प\edtext{}{\edlabel{pvv.85-14}\label{pvv.85-14}\lemma{ल्प}\Bfootnote{पाणेरिव श‚रीरात् ।}}न‚या ‚{\color{DodgerBlue3}‚व्याख्येयः} स‚म‚र्थ‚नीयः । य‚था प‚लाश‚युक्त‚म्व‚न‚मिति । दृष्टिस्व‚भावा‚{\tiny $_{lb}$}‚ऽविद्याप्राधान्येन क्लेश‚हेतुरित्युप‚द‚र्श‚न‚ञ्च प्र‚योज‚नं ।
	\pend% ending standard par
      

	  \pstart \leavevmode% starting standard par
	b. प्र‚कृत‚माह (।) य‚त‚श्चैव‚मात्म‚द‚र्श‚न‚प्र‚भूताः स‚र्व्व‚क्लेशाः ।
	\pend% ending standard par
      \textsuperscript{\textenglish{086/s}}
	  \bigskip
	  \begingroup
	
	    \large
	  
	    \begin{quote}
	  
	    
	    \stanza[\smallbreak]
	\label{pv.1.216c}\flagstanza{\tiny\textenglish{...1.216c}}त‚द्विरोधाच्च त‚न्म‚यैः ॥ २१६ ॥\&[\smallbreak]


	
	    \end{quote}
	  
	  \endgroup
	
	  \bigskip
	  \begingroup
	
	    \large
	  
	    \begin{quote}
	  
	    
	    \stanza[\smallbreak]
	\label{pv.1.217a}\flagstanza{\tiny\textenglish{...1.217a}}विरोधः शून्य‚तादृष्टेः स‚र्व‚दोषैः प्र‚सिध्य‚ति ।\&[\smallbreak]


	
	    \end{quote}
	  
	  \endgroup
	

	  \pstart \leavevmode% starting standard par
	\hphantom{.}त‚या ‚{\color{DodgerBlue3}‚स‚त्त्व}‚दृष्ट्या विरोधाच्च ‚{\color{DodgerBlue3}‚शून्य‚तादृष्टे}‚र्नैरात्म्य‚दृष्टेः । ‚{\color{DodgerBlue3}‚त‚न्म‚यैः} (२१६) ‚{\tiny $_{lb}$}‚स‚त्त्व‚दृष्टिहेतुकैः ‚{\color{DodgerBlue3}‚स‚र्व्व‚दोषै}‚र्व्विरोधः ‚{\color{DodgerBlue3}‚सिध्य‚ति} । शीत‚विरुद्ध‚स्याग्नेरिव त‚त्कार्ये ‚{\tiny $_{lb}$}‚रोम‚ह‚र्षादिभिर‚तः सा‚{\tiny $_{2}$}‚त्मीभूत‚नैरात्म्यानां न पुन‚र्दोष‚लेशोत्प‚त्तिर‚नेन निःश‚र‚ण‚त ‚{\tiny $_{lb}$}‚इति निरोधाकारो निर्द्दिष्टः दोषेभ्यः स‚र्व्व‚था निःश‚र‚णात् ।
	\pend% ending standard par
      

	  \begin{center}%% label @type='head'
	\textbf{(घ) दोषाः प्र‚तीत्य‚स‚मुत्प‚न्नाः}
	\end{center}
	
	  \bigskip
	  \begingroup
	
	    \large
	  
	    \begin{quote}
	  
	    
	    \stanza[\smallbreak]
	\label{pv.1.217b}\flagstanza{\tiny\textenglish{...1.217b}}नाक्ष‚यः प्राणिध‚र्म‚त्वाद् रूपादिव‚द‚सिद्धितः ॥ २१७ ॥\&[\smallbreak]


	
	    \end{quote}
	  
	  \endgroup
	
	  \bigskip
	  \begingroup
	
	    \large
	  
	    \begin{quote}
	  
	    
	    \stanza[\smallbreak]
	\label{pv.1.218a}\flagstanza{\tiny\textenglish{...1.218a}}स‚म्ब‚न्धे प्र‚तिप‚क्ष‚स्य त्याग‚स्याद‚र्श‚नाद‚पि ।\&[\smallbreak]


	
	    \end{quote}
	  
	  \endgroup
	

	  \pstart \leavevmode% starting standard par
	\hphantom{.}a. स्यादेत‚{\color{DodgerBlue3}‚द‚क्ष‚यो} रागादिः ‚{\color{DodgerBlue3}‚प्राणिध‚र्म‚त्वाद्रूपादिव‚त् न} युक्त‚मेत‚द‚{\color{DodgerBlue3}‚सिद्धितः} । न हि ‚{\tiny $_{lb}$}‚‚{\color{DodgerBlue3}‚प्राणी} क‚श्चिद‚स्ति य‚द्ध‚र्मा रागाद‚यः सिद्ध्य‚न्ति । स‚त्त्वे दृष्टौ तु स‚त्यां केव‚ल\edtext{}{\edlabel{pvv.86-1}\label{pvv.86-1}\lemma{ल}\Bfootnote{र‚ज्ज्वां स‚र्प‚बुद्धिव‚द‚पैति च ।}}मुप‚{\tiny $_{lb}$}‚ल‚भ्य‚न्त इति प्र‚तीत्य‚स‚मुत्पाद‚मात्र‚मेत‚त् (२१७) । तेषां ‚{\color{DodgerBlue3}‚प्र‚तिप‚क्ष‚स्य} नैरात्म्य‚द‚र्श‚{\tiny $_{lb}$}‚न‚स्य स‚म्ब‚न्धे संमुखीभावे त्याग‚स्यापात‚विष्क‚म्भ‚ण‚{\tiny $_{3}$}‚स्या‚{\color{DodgerBlue3}‚द‚र्श‚नाद‚पि} स‚म्भ‚व‚त्प्र‚ति‚{\tiny $_{lb}$}‚प‚क्ष‚त्वेनोच्छेद‚स‚म्भ‚वात् नाक्ष‚यित्वं ।
	\pend% ending standard par
      

	  \pstart \leavevmode% starting standard par
	b. स्यादे\edtext{}{\edlabel{pvv.86-2}\label{pvv.86-2}\lemma{स्यादे}\Bfootnote{वैध‚र्म्य‚दृष्टान्त‚माह ।}}त‚त् (।)ताम्रादीनाम‚ग्नियोगाद् द्र‚वाव‚स्थायां न‚ष्ट‚म‚पि काठिन्यं ‚{\tiny $_{lb}$}‚पुनः शीत‚स‚म्प‚र्कादुत्प‚द्य‚ते । त‚द्व‚न्न‚ष्टानाम‚पि दोषाणां मार्ग‚सात्म्ये पुनः कुत‚श्चि‚{\tiny $_{lb}$}‚द्धेतोरुत्प‚त्तिः स्यात् । अत्राह (।)
	\pend% ending standard par
      
	  \bigskip
	  \begingroup
	
	    \large
	  
	    \begin{quote}
	  
	    
	    \stanza[\smallbreak]
	\label{pv.1.218b}\flagstanza{\tiny\textenglish{...1.218b}}न काठिन्य‚व‚दुत्प‚त्तिः पुन‚र्दोष‚विरोधिनः ॥ २१८ ॥\&[\smallbreak]


	
	    \end{quote}
	  
	  \endgroup
	
	  \bigskip
	  \begingroup
	
	    \large
	  
	    \begin{quote}
	  
	    
	    \stanza[\smallbreak]
	\label{pv.1.219a}\flagstanza{\tiny\textenglish{...1.219a}}सात्म‚त्वेनान‚पाय‚त्वात् अनेकान्ताच्च भ‚स्म‚व‚त् ।\&[\smallbreak]


	
	    \end{quote}
	  
	  \endgroup
	

	  \pstart \leavevmode% starting standard par
	\hphantom{.}‚{\color{DodgerBlue3}‚न काठिन्य‚व‚दुत्प‚त्तिः पुन‚र्दो}‚षाणां । दोष‚{\color{DodgerBlue3}‚विरोधिनो} (२१८ ) नैरात्म्य‚स्य ‚{\tiny $_{lb}$}‚‚{\color{DodgerBlue3}‚सात्म‚त्वेन} प्र‚कृतित्वेना‚{\color{DodgerBlue3}‚न‚पायात्} । न ह्य‚व्याह‚ते विरोधिनि त‚द्विरुद्ध‚स्योत्प‚त्तिर‚ग्ना‚{\tiny $_{lb}$}‚विव शीत‚काठि‚{\tiny $_{4}$}‚न्यं कादाचित्क‚त्वाद‚ग्निनिवृत्तौ त‚द्विरोधिन्या द्र‚व‚तायाः स्व‚र‚स‚{\tiny $_{lb}$}‚निरोधादुत्प‚द्य‚तेऽ‚{\color{DodgerBlue3}‚नैकान्ताच्च भ‚स्म‚व‚त्} । य‚था भ‚स्म‚नि भूते पुन‚र्न काष्ठोत्प‚त्तिस्त‚था ‚{\tiny $_{lb}$}‚नैरात्म्य‚सात्म‚तायां न पुन‚र्न‚ष्टानां दोषाणामुत्प‚त्तिरित्य‚नैकान्तिक‚ता न‚ष्टोत्प‚त्तेः ।
	\pend% ending standard par
      

	  \pstart \leavevmode% starting standard par
	\hphantom{.}C. न‚न्वात्म‚भाव‚न‚या‚{\color{DodgerBlue3}‚पि मोक्षोस्ति} त‚त्किं नैरात्म्य‚भाव‚न‚या य‚दाहुरात्मा म‚न्त‚व्यो ‚{\tiny $_{lb}$}‚निदिध्यासित‚व्य इत्यादि । अत्राह (।)
	\pend% ending standard par
      
	  \bigskip
	  \begingroup
	
	    \large
	  
	    \begin{quote}
	  
	    
	    \stanza[\smallbreak]
	\label{pv.1.219b}\flagstanza{\tiny\textenglish{...1.219b}}यः प‚श्य‚त्यात्मानं त‚त्रास्याह‚मिति शाश्व‚तः स्नेहः ॥ २१९ ॥\&[\smallbreak]


	
	    \end{quote}
	  
	  \endgroup
	
	  \bigskip
	  \begingroup
	
	    \large
	  
	    \begin{quote}
	  
	    
	    \stanza[\smallbreak]
	\label{pv.1.220a}\flagstanza{\tiny\textenglish{...1.220a}}स्नेहात् सुखेषु तृष्य‚ति तृष्णा दोषांस्तिर‚स्कुरुते ।\&[\smallbreak]


	
	    \end{quote}
	  
	  \endgroup
	\textsuperscript{\textenglish{087/s}}

	  \pstart \leavevmode% starting standard par
	\hphantom{.}‚{\color{DodgerBlue3}‚यः प‚श्य‚त्यात्मानं त‚त्रा}‚त्म‚न्य‚{\color{DodgerBlue3}‚स्य} द्र‚ष्टु‚{\color{DodgerBlue3}‚र‚ह‚मिति शाश्व‚तोऽनुपा}‚यि‚{\color{DodgerBlue3}‚स्नेहो} भ‚व‚ति (२१९) ।
	\pend% ending standard par
      

	  \pstart \leavevmode% starting standard par
	\hphantom{.}‚{\color{DodgerBlue3}‚स्नेहा}‚दात्म‚स्ने‚{\tiny $_{5}$}‚हात्सुखेषु ‚{\color{DodgerBlue3}‚तृष्य‚ति । तृष्णा}‚वान् भ‚व‚तीति । ‚{\color{DodgerBlue3}‚तृष्णा च सुख‚सा-} ध‚न‚त्वेनाध्य‚व‚सितानां व‚स्तूनां ‚{\color{DodgerBlue3}‚दोषान}‚शुचित्वादीन् ‚{\color{DodgerBlue3}‚तिर‚स्कुरुते} प्र‚च्छाद‚य‚ति दोष‚ति‚{\tiny $_{lb}$}‚र‚स्क‚र‚णात् ।
	\pend% ending standard par
      
	  \bigskip
	  \begingroup
	
	    \large
	  
	    \begin{quote}
	  
	    
	    \stanza[\smallbreak]
	\label{pv.1.220b}\flagstanza{\tiny\textenglish{...1.220b}}गुण‚द‚र्शी प‚रितृष्य‚न् म‚मेति त‚त्साध‚नान्युपाद‚त्ते ॥ २२० ॥\&[\smallbreak]


	
	    \end{quote}
	  
	  \endgroup
	
	  \bigskip
	  \begingroup
	
	    \large
	  
	    \begin{quote}
	  
	    
	    \stanza[\smallbreak]
	\label{pv.1.221a}\flagstanza{\tiny\textenglish{...1.221a}}तेनात्माभिनिवेशो याव‚त् ताव‚त् स संसारे ॥\&[\smallbreak]


	
	    \end{quote}
	  
	  \endgroup
	

	  \pstart \leavevmode% starting standard par
	\hphantom{.}‚{\color{DodgerBlue3}‚गुण‚द‚र्शी} शुचित्वेष्ट‚त्व‚गुणान् प‚श्य‚न् ‚{\color{DodgerBlue3}‚प‚रितृष्य‚न्} म‚मेति ‚{\color{DodgerBlue3}‚म‚मेदं सुख‚मिति} ग‚र्द्ध‚{\tiny $_{lb}$}‚मान‚स्त‚स्य सुख‚स्य ‚{\color{DodgerBlue3}‚साध‚नानि} ग‚र्भ‚ग‚म‚नादी‚{\color{DodgerBlue3}‚न्युपाद‚त्ते} । (२२० )
	\pend% ending standard par
      

	  \pstart \leavevmode% starting standard par
	तेनात्म‚द‚र्श‚न‚मूल\edtext{}{\edlabel{pvv.87-1}\label{pvv.87-1}\lemma{मूल}\Bfootnote{आत्म‚द‚र्श‚नं मूलं य‚स्य ज‚न्मादेः ।}}त्वेन ज‚न्मादेरा‚{\color{DodgerBlue3}‚त्माभिनिवेशो याव‚त्ताव‚त् स} आत्म‚द‚र्शी ‚{\color{DodgerBlue3}‚संसार} एव (।)
	\pend% ending standard par
      

	  \pstart \leavevmode% starting standard par
	d. न केव‚लं ज‚न्म‚प्र‚ब‚न्ध‚स्त‚स्य‚{\tiny $_{6}$}‚ दोषा अपि स‚म‚स्ताः स‚न्तीत्याह (।)
	\pend% ending standard par
      
	  \bigskip
	  \begingroup
	
	    \large
	  
	    \begin{quote}
	  
	    
	    \stanza[\smallbreak]
	\label{pv.1.221b}\flagstanza{\tiny\textenglish{...1.221b}}अत्म‚नि स‚ति प‚र‚संज्ञा स्व‚प‚र‚विभागात् प‚रिग्र‚ह‚द्वेषौ ॥ २२१ ॥\&[\smallbreak]


	
	    \end{quote}
	  
	  \endgroup
	
	  \bigskip
	  \begingroup
	
	    \large
	  
	    \begin{quote}
	  
	    
	    \stanza[\smallbreak]
	\label{pv.1.222a}\flagstanza{\tiny\textenglish{...1.222a}}अन‚योः संप्र‚तिब‚द्धाः स‚र्वे दोषाः प्र‚जाय‚न्ते ॥\&[\smallbreak]


	
	    \end{quote}
	  
	  \endgroup
	

	  \pstart \leavevmode% starting standard par
	\hphantom{.}‚{\color{DodgerBlue3}‚आत्म‚नि स‚ति} त‚तोऽन्य‚स्मिन् ‚{\color{DodgerBlue3}‚प‚र‚संज्ञा} प‚र‚बुद्धिर्भ‚व‚ति । ‚{\color{DodgerBlue3}‚स्व‚प‚र‚विभागाच्च} कार‚णात् स्व‚प‚र‚योर्य‚थाक्र‚मं ‚{\color{DodgerBlue3}‚प‚रिग्र}‚होऽभिष्व‚{\color{DodgerBlue3}‚ङ्गो द्वेषः} प‚रित्याग‚स्तौ भ‚व‚तः (२२१) । ‚{\tiny $_{lb}$}‚‚{\color{DodgerBlue3}‚अन‚यो}‚र‚नुन‚य‚प्र‚तिषेध‚योः ‚{\color{DodgerBlue3}‚संप्र‚तिब‚द्धाः स‚र्व्वे दोषा} राग‚मात्स‚र्येर्ष्याद‚यः ‚{\color{DodgerBlue3}‚प्र‚जाय‚न्ते} ।
	\pend% ending standard par
      

	  \pstart \leavevmode% starting standard par
	\hphantom{.}e. य‚द्य‚प्यात्म‚नि स्नेह‚वान् त‚थाप्य‚त्मीये ‚{\color{DodgerBlue3}‚सुख‚साध‚ने वैराग्यान्न संस‚र‚तीति ‚{\tiny $_{lb}$}‚चेत्} । नैत‚द्युक्तं य‚त (ः।)
	\pend% ending standard par
      
	  \bigskip
	  \begingroup
	
	    \large
	  
	    \begin{quote}
	  
	    
	    \stanza[\smallbreak]
	\label{pv.1.222b}\flagstanza{\tiny\textenglish{...1.222b}}निय‚मेनात्म‚नि स्निह्यंस्त‚दीये न विर‚ज्य‚ते ॥ २२२ ॥\&[\smallbreak]


	
	    \end{quote}
	  
	  \endgroup
	

	  \pstart \leavevmode% starting standard par
	\hphantom{.}‚{\color{DodgerBlue3}‚आत्म‚नि स्निह्य‚न्} प्रीय‚माण‚{\color{DodgerBlue3}‚स्त‚दीय आत्मी}‚{\tiny $_{7}$}‚ये ‚{\color{DodgerBlue3}‚सुख‚सा}‚ध‚ने ‚{\color{DodgerBlue3}‚निय‚मेन} (‚{\color{DodgerBlue3}‚न}‚) ‚{\color{DodgerBlue3}‚विर‚{\tiny $_{lb}$}‚ज्य‚ते}‚ऽभिष्व‚ज‚त्येव त‚त्क‚थ‚मात्मीय‚विरागान्मुक्तिः ॥ आत्म‚स्नेह‚स्यात्मीय‚वैराग्य‚विरो‚{\tiny $_{lb}$}‚धित्वात् । (२२२)
	\pend% ending standard par
      \label{div_pvv.1.223}
	  
	% new div opening: depth here is 2
	

	  \pstart \leavevmode% starting standard par
	f. त‚मे\edtext{}{\edlabel{pvv.87-2}\label{pvv.87-2}\lemma{मे}\Bfootnote{आत्म‚स्नेहं ।}}व त्य‚ज‚तीति चेत् आह (।)
	\pend% ending standard par
      
	  \bigskip
	  \begingroup
	
	    \large
	  
	    \begin{quote}
	  
	    
	    \stanza[\smallbreak]
	\label{pv.1.223}\flagstanza{\tiny\textenglish{....1.223}}ने चास्त्यात्म‚नि निर्दोषे स्नेहाप‚ग‚म‚कार‚ण‚म्\edtext{}{\edlabel{pvv.87-3}\label{pvv.87-3}\lemma{म्}\Bfootnote{नैत‚च्छ्लोकार्द्ध विवृतं वृत्तिकृता ।}} ॥&स्नेहः स‚दोष इति चेत् त‚तः किं त‚स्य व‚र्ज‚न‚म् ॥ २२३ ॥\&[\smallbreak]


	
	    \end{quote}
	  
	  \endgroup
	\textsuperscript{\textenglish{088/s}}

	  \pstart \leavevmode% starting standard par
	य\edtext{}{\edlabel{pvv.88-1}\label{pvv.88-1}\lemma{य}\Bfootnote{न‚चात्म‚नि निर्दोषे स्नेंहाप‚ग‚म‚कार‚ण‚म‚स्ति इतिव‚क्तुमुचितं व्याख्यान‚म् ।}}द्य‚प्यात्मा निर्दोष‚स्त‚थापि ‚{\color{DodgerBlue3}‚स्नेहः स‚दोष इति चेत् । त‚तः} स‚दोष‚त्वात् ‚{\color{DodgerBlue3}‚किं ‚{\tiny $_{lb}$}‚क‚र्त‚व्यं} त‚स्य स्नेह‚स्य ‚{\color{DodgerBlue3}‚व‚र्ज‚न‚म्} (। २२३)
	\pend% ending standard par
      \label{div_pvv.1.224_1.225_1.226}
	  
	% new div opening: depth here is 2
	
	  \bigskip
	  \begingroup
	
	    \large
	  
	    \begin{quote}
	  
	    
	    \stanza[\smallbreak]
	\label{pv.1.224a}\flagstanza{\tiny\textenglish{...1.224a}}अदुषितेऽस्य विष‚ये न श‚क्यं त‚स्य व‚र्ज‚न‚म् ।\&[\smallbreak]


	
	    \end{quote}
	  
	  \endgroup
	

	  \pstart \leavevmode% starting standard par
	\hphantom{.}‚{\color{DodgerBlue3}‚अदूषितेस्य विष‚य} आत्म‚नि ‚{\color{DodgerBlue3}‚न श‚क्यं त‚स्य व‚र्ज‚नं} । न हि स्नेहः स्व‚गुण‚दोषा‚{\tiny $_{lb}$}‚\leavevmode\ledsidenote{\textenglish{17a/MA}} भ्यामुपादीय‚ते त्य‚ज्य‚ते वा‚{\tiny $_{8}$}‚ किन्तु विष‚य‚स्य विष‚य‚श्च निर्दोष इति क‚थ‚म‚स्य ‚{\tiny $_{lb}$}‚व\edtext{}{\edlabel{pvv.88-2}\label{pvv.88-2}\lemma{व}\Bfootnote{इद‚मेव स‚म‚र्थ‚य‚ते ।}}र्ज‚नं ।
	\pend% ending standard par
      

	  \pstart \leavevmode% starting standard par
	\hphantom{.}‚{\color{DodgerBlue3}‚किञ्च (।)}
	\pend% ending standard par
      
	  \bigskip
	  \begingroup
	
	    \large
	  
	    \begin{quote}
	  
	    
	    \stanza[\smallbreak]
	\label{pv.1.224b}\flagstanza{\tiny\textenglish{...1.224b}}प्र‚हाणिरिच्छाद्वेषादेर्गुण‚दोषानुब‚न्धिनः ॥ २२४ ॥\&[\smallbreak]


	
	    \end{quote}
	  
	  \endgroup
	
	  \bigskip
	  \begingroup
	
	    \large
	  
	    \begin{quote}
	  
	    
	    \stanza[\smallbreak]
	\label{pv.1.225a}\flagstanza{\tiny\textenglish{...1.225a}}त‚योर‚दृष्टिर्विष‚ये;\&[\smallbreak]


	
	    \end{quote}
	  
	  \endgroup
	

	  \pstart \leavevmode% starting standard par
	\hphantom{.}‚{\color{DodgerBlue3}‚इच्छाद्वेषादेर्गुण‚दोषानुब‚न्धिनो} य‚थाक्र‚मं विष‚य‚स्य गुण‚दोषानुव‚र्तिनः प्र‚हाणिः ‚{\tiny $_{lb}$}‚प्र‚हाण्युपायः (२२४) । त‚योर्गुण‚दोष‚योर‚{\color{DodgerBlue3}‚दृष्टिर्विष\edtext{}{\edlabel{pvv.88-3}\label{pvv.88-3}\lemma{दृष्टिर्विष}\Bfootnote{इच्छाविष‚ये गुण‚दृष्टेः प्र‚व‚र्त‚तो द्वेषो दोष‚दृष्टेः तेनात्म‚दृष्ट्याऽग‚तो भिष्व‚ङ्ग‚स्त(द्)दृष्टेर‚पैति ।}}ये}‚ऽन‚न्योपाय‚ताद‚र्श‚नार्थ‚मुप‚चारः । ‚{\tiny $_{lb}$}‚विष‚य‚गुण‚दोषाद‚र्श‚ने एवेच्छा द्वेषादि‚{\color{DodgerBlue3}‚प्र‚हाण्यु}‚पाय इत्य‚र्थः ॥
	\pend% ending standard par
      

	  \pstart \leavevmode% starting standard par
	न‚न्व‚दृष्टोपि इत्यादिर‚निच्छामात्रात् त्य‚ज्य‚मानो दृश्य‚त\edtext{}{\edlabel{pvv.88-4}\label{pvv.88-4}\lemma{त}\Bfootnote{त‚त्किमात्म‚नि त‚द्विष‚येऽव‚श्य‚न्त‚याऽद‚र्श‚न‚म‚पेक्ष‚ते ।}} इत्याह (।)
	\pend% ending standard par
      
	  \bigskip
	  \begingroup
	
	    \large
	  
	    \begin{quote}
	  
	    
	    \stanza[\smallbreak]
	\label{pv.1.225b}\flagstanza{\tiny\textenglish{...1.225b}}न तु बाह्येषु यः क्र‚मः ।\&[\smallbreak]


	
	    \end{quote}
	  
	  \endgroup
	

	  \pstart \leavevmode% starting standard par
	न तु बाह्येषु व‚स्तुषु यः क्र‚मोऽनिच्छामात्र‚कृत‚त्याग‚रूपः स आन्त‚रेष्व‚पि ‚{\tiny $_{lb}$}‚स्नेहादिषु युक्तः । बाह्याधीनं बाह्य‚म‚निच्छ‚या त्य‚क्तुं श‚क्यं आत्म‚द‚र्श‚नाधीन‚न्तु ‚{\tiny $_{lb}$}‚न श‚क्य‚प‚रिहारं । अविक‚ल‚हेतुत्वेनोत्प‚त्तेः ।
	\pend% ending standard par
      

	  \pstart \leavevmode% starting standard par
	किञ्च (।)
	\pend% ending standard par
      
	  \bigskip
	  \begingroup
	
	    \large
	  
	    \begin{quote}
	  
	    
	    \stanza[\smallbreak]
	\label{pv.1.225c}\flagstanza{\tiny\textenglish{...1.225c}}न हि स्नेह‚गुणात् स्नेहः किन्त्व‚र्थ‚गुण‚द‚र्श‚नात् ॥ २२५ ॥\&[\smallbreak]


	
	    \end{quote}
	  
	  \endgroup
	
	  \bigskip
	  \begingroup
	
	    \large
	  
	    \begin{quote}
	  
	    
	    \stanza[\smallbreak]
	\label{pv.1.226a}\flagstanza{\tiny\textenglish{...1.226a}}कार‚णेऽविक‚ले त‚स्मिन् कार्यं केन निवार्य‚ते ।\&[\smallbreak]


	
	    \end{quote}
	  
	  \endgroup
	

	  \pstart \leavevmode% starting standard par
	\hphantom{.}न हि ‚{\color{DodgerBlue3}‚स्नेह‚गुणात्स्नेहः क्रिय‚ते किन्त्व‚र्थ}‚स्य विष‚य‚स्य ‚{\color{DodgerBlue3}‚गुण‚दोष‚द‚र्श‚नात्} जाय‚ते ‚{\tiny $_{lb}$}‚(२२५) ‚{\color{DodgerBlue3}‚कार‚णेऽविक‚ले त‚स्मिन्} विष‚य‚गुणे ‚{\color{DodgerBlue3}‚कार्यं} स्नेहः ‚{\color{DodgerBlue3}‚केन निवार्य‚ते} । न ‚{\tiny $_{lb}$}‚केन‚चित् ।
	\pend% ending standard par
      
	  \bigskip
	  \begingroup
	
	    \large
	  
	    \begin{quote}
	  
	    
	    \stanza[\smallbreak]
	\label{pv.1.226b}\flagstanza{\tiny\textenglish{...1.226b}}का वा स‚दोष‚ता दृष्टा स्नेहे दुःख‚स‚माश्र‚यः ॥ २२६ ॥\&[\smallbreak]


	
	    \end{quote}
	  
	  \endgroup
	\textsuperscript{\textenglish{089/s}}

	  \pstart \leavevmode% starting standard par
	\hphantom{.}‚{\color{DodgerBlue3}‚का वा स‚दोष‚ता दृष्टा स्नेहे} येनायं व‚र्ज‚यित‚व्यः ॥ ‚{\color{DodgerBlue3}‚दुःख‚स्य स‚माश्र‚य‚श्चेद्दोषः} । ‚{\tiny $_{lb}$}‚त‚था ह्यात्म‚नि स्निह्य‚न् त‚त्सुख‚सा‚{\tiny $_{2}$}‚ध‚नेषु तृष्णावान् दुःख‚भूतं संसार‚मुपाद‚त्ते ॥ ‚{\tiny $_{lb}$}‚(२२६)
	\pend% ending standard par
      \label{div_pvv.1.227}
	  
	% new div opening: depth here is 2
	
	  \bigskip
	  \begingroup
	
	    \large
	  
	    \begin{quote}
	  
	    
	    \stanza[\smallbreak]
	\label{pv.1.227a}\flagstanza{\tiny\textenglish{...1.227a}}त‚थापि न विरागोऽत्र स्व‚त्व‚दृष्टेर्य‚थात्म‚नि ।\&[\smallbreak]


	
	    \end{quote}
	  
	  \endgroup
	

	  \pstart \leavevmode% starting standard par
	\hphantom{.}‚{\color{DodgerBlue3}‚त‚थापि} दुःख‚हेतुत्वेपि ‚{\color{DodgerBlue3}‚न विरोगोऽत्र} स्नेहे ‚{\color{DodgerBlue3}‚य‚था}‚त्म‚नि स्व‚त्व‚{\color{DodgerBlue3}‚दृष्टेः} दुःख‚निदान‚{\tiny $_{lb}$}‚भूताया हेता‚{\color{DodgerBlue3}‚वात्म‚नि} न विरागः ।
	\pend% ending standard par
      

	  \pstart \leavevmode% starting standard par
	य‚दि दुःख‚हेतौ विराग‚स्त‚था स्व‚त्वे दृष्टिद्वारेण स‚र्व्वं दुःख‚मिति त‚स्य हेतावात्म‚न्येव ‚{\tiny $_{lb}$}‚स युक्तः । न चास्त्येत‚त् ।
	\pend% ending standard par
      
	  \bigskip
	  \begingroup
	
	    \large
	  
	    \begin{quote}
	  
	    
	    \stanza[\smallbreak]
	\label{pv.1.227b}\flagstanza{\tiny\textenglish{...1.227b}}न तैर्विना दुःख‚हेतुरात्मा चेत् तेऽपि तादृशाः ॥ २२७ ॥\&[\smallbreak]


	
	    \end{quote}
	  
	  \endgroup
	

	  \pstart \leavevmode% starting standard par
	\hphantom{.}‚{\color{DodgerBlue3}‚न तैः स्नेह}‚बुद्धीन्द्रियादिभिरात्मीयै‚{\color{DodgerBlue3}‚र्व्विना दुःख‚हेतु}‚रात्मा चेत् । ‚{\color{DodgerBlue3}‚तेपि} स्नेहा‚{\tiny $_{lb}$}‚द‚यः ‚{\color{DodgerBlue3}‚तादृशा} आत्मान‚म‚न्त‚रेण न दुःख‚हेत‚वः । (२२७)
	\pend% ending standard par
      \label{div_pvv.1.228}
	  
	% new div opening: depth here is 2
	
	  \bigskip
	  \begingroup
	
	    \large
	  
	    \begin{quote}
	  
	    
	    \stanza[\smallbreak]
	\label{pv.1.228}\flagstanza{\tiny\textenglish{....1.228}}निर्दोषं द्व‚य‚म‚प्येवं वैराग्यान्न द्व‚योस्त‚तः ॥&दुःख‚भाव‚न‚या स्याच्चेद‚हिद‚ष्टाङ्ग‚हानिव‚त् ॥ २२८ ॥\&[\smallbreak]


	
	    \end{quote}
	  
	  \endgroup
	

	  \pstart \leavevmode% starting standard par
	\hphantom{.}‚{\color{DodgerBlue3}‚एवं} प‚र‚स्प‚र‚सापेक्ष‚त्वे ‚{\color{DodgerBlue3}‚निर्दोषं‚{\tiny $_{3}$}‚ द्व‚य‚म‚पि} स्नेहादिरात्मा च । ‚{\color{DodgerBlue3}‚वैराग्य‚न्न द्व‚यो}‚र‚पि ‚{\tiny $_{lb}$}‚‚{\color{DodgerBlue3}‚त‚तः} का\edtext{}{\edlabel{pvv.89-1}\label{pvv.89-1}\lemma{का}\Bfootnote{क‚र‚णीयं ।}}र्यं । त‚था च संसारो दोषाश्च दुर्व्वारा हेतुसाक‚ल्यात् ॥ प्र‚वृत्तिः ‚{\tiny $_{lb}$}‚य‚दि स्नेहादिषु ‚{\color{DodgerBlue3}‚दुःख‚भाव‚न‚या} हानिः ‚{\color{DodgerBlue3}‚स्यात् । अहिद‚ष्ट‚स्याङ्ग‚स्य हानिव‚त्} । ‚{\tiny $_{lb}$}‚य‚थात्मीय‚म‚प्य‚हिद‚ष्ट‚म‚ङ्गं दुःख‚व‚शाद् विर‚ज्य त्य‚ज्य‚ते । अनुप‚भोगाश्र‚य‚त्वात् । ‚{\tiny $_{lb}$}‚(२२८)
	\pend% ending standard par
      \label{div_pvv.1.229_1.230_1.231}
	  
	% new div opening: depth here is 2
	
	  \bigskip
	  \begingroup
	
	    \large
	  
	    \begin{quote}
	  
	    
	    \stanza[\smallbreak]
	\label{pv.1.229}\flagstanza{\tiny\textenglish{....1.229}}आत्मीय‚बुद्धिहान्याऽत्र त्यागो न तु विय‚र्य‚ये ॥&उप‚भोगाश्र‚य‚त्वेन गृहीतेष्विन्द्रियादिषु ॥ २२९ ॥\&[\smallbreak]


	
	    \end{quote}
	  
	  \endgroup
	
	  \bigskip
	  \begingroup
	
	    \large
	  
	    \begin{quote}
	  
	    
	    \stanza[\smallbreak]
	\label{pv.1.230a}\flagstanza{\tiny\textenglish{...1.230a}}स्व‚त्व‚धीः केन वार्येत वैराग्यं त‚त्र त‚त् कुतः ॥\&[\smallbreak]


	
	    \end{quote}
	  
	  \endgroup
	

	  \pstart \leavevmode% starting standard par
	\hphantom{.}‚{\color{DodgerBlue3}‚आत्मीय‚बुद्धिहान्या} । त‚त्राहिद‚ष्टाङ्गे ‚{\color{DodgerBlue3}‚त्यागो न तु विप‚र्य‚ये} आत्मीय‚बुद्धि‚{\tiny $_{lb}$}‚स‚त्तायां । य‚स्मादु‚{\color{DodgerBlue3}‚प‚भोग}‚स्या‚{\color{DodgerBlue3}‚श्र‚य‚त्वेन} कार‚ण‚त्वेन ‚{\color{DodgerBlue3}‚गृहीतेष्विन्द्रियादिषु} (२२९) ‚{\tiny $_{lb}$}‚‚{\color{DodgerBlue3}‚स्व‚त्वे धीरा}‚{\tiny $_{4}$}‚त्मीय‚त्व‚बुद्धिः ‚{\color{DodgerBlue3}‚के\edtext{}{\edlabel{pvv.89-2}\label{pvv.89-2}\lemma{के}\Bfootnote{त‚स्मात्सा ।}}न} हेतुना ‚{\color{DodgerBlue3}‚वार्येत न} केन‚चित् । त‚त्कुत‚{\color{DodgerBlue3}‚स्त‚त्रो}‚प‚भोग‚{\tiny $_{lb}$}‚साध‚ने स्वीयाव‚य‚वे ‚{\color{DodgerBlue3}‚वैराग्यं} येन त्य‚ज्य‚ते । ‚{\color{DodgerBlue3}‚त‚तो} य‚त्त्य‚ज्य‚ते ‚{\color{DodgerBlue3}‚आत्मीय‚बुद्धिहान्या} एव । न चैवं स्नेहादिष्वात्मीय‚बुद्धिहानिर‚स्ति येनैषां त्यागः स्यात् ।
	\pend% ending standard par
      \textsuperscript{\textenglish{090/s}}
	  \bigskip
	  \begingroup
	
	    \large
	  
	    \begin{quote}
	  
	    
	    \stanza[\smallbreak]
	\label{pv.1.230b}\flagstanza{\tiny\textenglish{...1.230b}}प्र‚त्य‚क्ष‚मेव स‚र्व‚स्य केशादिषु क‚लेव‚रात् ॥ २३० ॥\&[\smallbreak]


	
	    \end{quote}
	  
	  \endgroup
	
	  \bigskip
	  \begingroup
	
	    \large
	  
	    \begin{quote}
	  
	    
	    \stanza[\smallbreak]
	\label{pv.1.231a}\flagstanza{\tiny\textenglish{...1.231a}}च्युतेषु स‚धृणा बुद्धिर्जाय‚तेऽन्येषु स‚स्पृहा ।*\&[\smallbreak]


	
	    \end{quote}
	  
	  \endgroup
	

	  \pstart \leavevmode% starting standard par
	\hphantom{.}एत‚च्च ‚{\color{DodgerBlue3}‚प्र‚त्य‚क्ष‚मेव स‚र्व्व‚स्य केशादिषु क‚लेव‚रात्} । (२३० ) ‚{\color{DodgerBlue3}‚च्युतेष्वा}‚त्मीय‚{\tiny $_{lb}$}‚बुद्धिविष‚येषु ‚{\color{DodgerBlue3}‚स‚धृणा} बुद्धिर्जाय‚ते । ज‚न‚स्यान्य‚त्राच्युतेष्वात्मीय‚बुद्धिविष‚येषु स‚स्पृहा ।
	\pend% ending standard par
      

	  \begin{center}%% label @type='head'
	\textbf{(ङ) आत्मात्मीय‚बुद्ध्योर्हानिः}
	\end{center}
	

	  \pstart \leavevmode% starting standard par
	दुःख‚भाव‚न‚या‚{\tiny $_{5}$}‚ आत्मीय‚बुद्धिरेव हीय‚त इति चेत् । न युक्त‚मिदं (।)
	\pend% ending standard par
      
	  \bigskip
	  \begingroup
	
	    \large
	  
	    \begin{quote}
	  
	    
	    \stanza[\smallbreak]
	\label{pv.1.231b}\flagstanza{\tiny\textenglish{...1.231b}}स‚म‚वायादिस‚म्ब‚न्ध‚ज‚निता त‚त्र हि स्व‚धीः ॥ २३१ ॥\&[\smallbreak]


	
	    \end{quote}
	  
	  \endgroup
	

	  \pstart \leavevmode% starting standard par
	\hphantom{.}‚{\color{DodgerBlue3}‚स‚म‚वायादिस‚म्ब‚न्ध‚ज‚निता} हि य‚स्मात्त‚त्र बुद्ध्यादौ ‚{\color{DodgerBlue3}‚स्व‚धीः} । त‚था चात्म‚नः ‚{\tiny $_{lb}$}‚सुखादीना स‚म‚वायः स‚म्ब‚न्धः । श‚रीरेण संयोगः । श‚रीराश्रितैः रूपादिभिः संयुक्त‚{\tiny $_{lb}$}‚स‚म‚वायः । श्रोत्रेन्द्रियेण संयोग‚श्च‚क्षुरादिभिः संयोगिसंयोग आत्म‚स‚म्ब‚न्धः । (२३१)
	\pend% ending standard par
      \label{div_pvv.1.232}
	  
	% new div opening: depth here is 2
	
	  \bigskip
	  \begingroup
	
	    \large
	  
	    \begin{quote}
	  
	    
	    \stanza[\smallbreak]
	\label{pv.1.232}\flagstanza{\tiny\textenglish{....1.232}}स त‚थैवेति सा दोष‚दृष्टाव‚पि न हीय‚ते ।&स‚म‚वायाद्य‚भावेऽपि स‚र्व‚त्रास्त्युप‚कारिता ॥ २३२ ॥\&[\smallbreak]


	
	    \end{quote}
	  
	  \endgroup
	

	  \pstart \leavevmode% starting standard par
	\hphantom{.}‚{\color{DodgerBlue3}‚स} दुःख‚भाव‚नायाम‚पि ‚{\color{DodgerBlue3}‚त‚थैवेति सा} स्व‚धी‚{\color{DodgerBlue3}‚र्दोष‚दृष्टाव‚पि न हीय‚ते} निमित्त‚स्या‚{\tiny $_{lb}$}‚वैक‚ल्यात् । अथ स\edtext{}{\edlabel{pvv.90-1}\label{pvv.90-1}\lemma{स}\Bfootnote{सांख्यं प्र‚त्याह (।) प्र‚कृतिपुरुषान्त‚र‚ज्ञानान्मुक्तेः ।}}म‚वायादिर्नास्त्येव त‚दा ।‚{\tiny $_{6}$}‚ ‚{\color{DodgerBlue3}‚स‚म‚वायाद्य‚भावेऽपि स‚र्व्व‚त्र} बुद्ध्या‚{\tiny $_{lb}$}‚दाव‚{\color{DodgerBlue3}‚स्त्युप‚कारिता} त‚त्कृता । (२३२)
	\pend% ending standard par
      \label{div_pvv.1.233}
	  
	% new div opening: depth here is 2
	
	  \bigskip
	  \begingroup
	
	    \large
	  
	    \begin{quote}
	  
	    
	    \stanza[\smallbreak]
	\label{pv.1.233}\flagstanza{\tiny\textenglish{....1.233}}दुःखोप‚कारान्न भ‚वेद‚ङ्‏गुल्यामिव चेत् स्व‚धीः ।&न ह्येकान्तेन त‚द् दुखं भूय‚सा स‚विषान्न‚व‚त् ॥ २३३ ॥\&[\smallbreak]


	
	    \end{quote}
	  
	  \endgroup
	

	  \pstart \leavevmode% starting standard par
	त‚त्र स्व‚धीर‚श\edtext{}{\edlabel{pvv.90-2}\label{pvv.90-2}\lemma{श}\Bfootnote{उप‚भोगाङ्ग‚त्वेन वृत्तेः ।}}क्य‚धार‚णा ‚{\color{DodgerBlue3}‚दुःखोप‚कारात्} दुःखोप‚निधाना\edtext{}{\edlabel{pvv.90-3}\label{pvv.90-3}\lemma{निधाना}\Bfootnote{उप‚स‚मीपे निधान‚म‚र्प‚ण‚न्त‚तः}}त् ‚{\color{DodgerBlue3}‚न भ‚वे}‚द‚हिद‚ष्टा‚{\tiny $_{lb}$}‚या‚{\color{DodgerBlue3}‚म‚ङ्गुल्यामिव} स्नेह‚बुद्ध्यादौ ‚{\color{DodgerBlue3}‚स्व‚धी}‚रिति ‚{\color{DodgerBlue3}‚चेत् । न ह्येकान्तेन} त‚त्स्नेहादि‚{\color{DodgerBlue3}‚दुःखं} दुःख‚हेतुः प‚र्यायेण सुख‚हेतुत्वाद‚पि । किन्तु ‚{\color{DodgerBlue3}‚भूय‚सा} त‚द्दुःखं ‚{\color{DodgerBlue3}‚स‚वि\edtext{}{\edlabel{pvv.90-4}\label{pvv.90-4}\lemma{वि}\Bfootnote{य‚थात्मा न त्य‚ज्य‚ते त‚थाय‚म‚पि ।}}षान्न‚व‚त्} । प‚रिण‚ति‚{\tiny $_{lb}$}‚दुःख‚हेतुर‚पि विषान्न‚मापात‚सुखं च । (२३३)
	\pend% ending standard par
      \label{div_pvv.1.234}
	  
	% new div opening: depth here is 2
	

	  \pstart \leavevmode% starting standard par
	a. न‚नु स‚विष‚म‚न्नं सुख‚मिश्र‚ञ्च वैराग्य‚विष‚यः स्व‚हित‚कामानामेवं‚{\tiny $_{7}$}‚ स्नेहादि‚{\tiny $_{lb}$}‚र‚पि स्यादित्याह (।)
	\pend% ending standard par
      
	  \bigskip
	  \begingroup
	
	    \large
	  
	    \begin{quote}
	  
	    
	    \stanza[\smallbreak]
	\label{pv.1.234a}\flagstanza{\tiny\textenglish{...1.234a}}विशिष्ट‚सुख‚स‚ङ्गात् स्यात् त‚द्विरुद्धे विरागिता ।\&[\smallbreak]


	
	    \end{quote}
	  
	  \endgroup
	

	  \pstart \leavevmode% starting standard par
	\hphantom{.}‚{\color{DodgerBlue3}‚विशिष्टे} सुखे सुख‚साध‚ने त‚दात्व‚प‚रिणाभ‚योर‚नुग्र‚हीत‚रि विषादिदोष‚र‚हित‚{\tiny $_{lb}$}‚भोज‚नादौ ‚{\color{DodgerBlue3}‚स‚ङ्गाद}‚भिष्व‚ङ्गात् ‚{\color{DodgerBlue3}‚स्यात्त‚द्विरुद्धे} स‚विषान्नादौ ‚{\color{DodgerBlue3}‚विरागिता} ।
	\pend% ending standard par
      \textsuperscript{\textenglish{091/s}}

	  \pstart \leavevmode% starting standard par
	वैराग्याच्च (।)
	\pend% ending standard par
      
	  \bigskip
	  \begingroup
	
	    \large
	  
	    \begin{quote}
	  
	    
	    \stanza[\smallbreak]
	\label{pv.1.234b}\flagstanza{\tiny\textenglish{...1.234b}}किञ्चित् प‚रित्य‚जेत् सौख्यं विशिष्ट‚सुख‚तृष्ण‚या ॥ २३४ ॥\&[\smallbreak]


	
	    \end{quote}
	  
	  \endgroup
	

	  \pstart \leavevmode% starting standard par
	\hphantom{.}‚{\color{DodgerBlue3}‚किञ्चित्सौख्यं} प‚रिण‚तिदुःख‚ब‚हुलं ‚{\color{DodgerBlue3}‚प‚रित्य‚जे}‚दात्म‚कामो ‚{\color{DodgerBlue3}‚विशिष्ट‚स्य सुख‚स्य} प‚रिणामाविरुद्ध‚स्य ‚{\color{DodgerBlue3}‚तृष्ण‚या} अभिलाषेण । य‚दा त्वात्म‚नि स‚ति न किञ्चित्सुखैक‚रूपं ‚{\tiny $_{lb}$}‚स‚र्व्वं सुखं दुःख‚संभिश्रं त‚दा क्व प‚रिहारः क‚स्मिन्न‚नुरागः‚{\tiny $_{8}$}‚ स्वीकारो न चेच्छ‚या\leavevmode\ledsidenote{\textenglish{17b/MA}} ‚{\tiny $_{lb}$}‚श‚क्य‚प‚रिहाराः स्नेहाद‚यः त‚त्कार‚ण‚स्यात्म‚नोऽवैक‚ल्यादित्युक्तं ॥ (२३४)
	\pend% ending standard par
      \label{div_pvv.1.235}
	  
	% new div opening: depth here is 2
	

	  \pstart \leavevmode% starting standard par
	b. न‚न्व\edtext{}{\edlabel{pvv.91-1}\label{pvv.91-1}\lemma{न्व}\Bfootnote{य‚दि नैरात्म्य‚मेव स‚र्व्व‚ध‚र्म्माणां व‚स्तुत‚स्त‚दाऽप्र‚वृत्तिस्त‚त्कार्य स्यान्न चास्तीत्याह । नैराश्ये नैरात्म्ये त‚त्त्वे मूल‚व्याख्या, स्यादेत‚द्य‚दि नैरात्म्य‚ज्ञानं स्यात् किन्त्व‚ज्ञानादारोपात् प्र‚व‚र्त‚ते ।}}नैरात्म्य‚प‚क्षेपि दुःखं स‚क‚लं । ‚{\color{DodgerBlue3}‚त‚न्निरोधः प‚र‚सुखं । त‚त्क्व‚चित्तृष्ण‚या} प्र‚वृत्तिर्न स्यादित्याह (।)
	\pend% ending standard par
      
	  \bigskip
	  \begingroup
	
	    \large
	  
	    \begin{quote}
	  
	    
	    \stanza[\smallbreak]
	\label{pv.1.235}\flagstanza{\tiny\textenglish{....1.235}}नैरात्म्ये तु य‚थालाभ‚मात्म‚स्नेहात् प्र‚व‚र्त‚ते ।&अलाभे म‚त्त‚कासिन्या दृष्टा तिर्य‚क्षु कामिता ॥ २३५ ॥\&[\smallbreak]


	
	    \end{quote}
	  
	  \endgroup
	

	  \pstart \leavevmode% starting standard par
	\hphantom{.}‚{\color{DodgerBlue3}‚नैरात्म्य‚त}‚त्त्वेऽधायोपादा\edtext{}{\edlabel{pvv.91-2}\label{pvv.91-2}\lemma{त्त्वेऽधायोपादा}\Bfootnote{आत्मारोपात् ।}} ‚{\color{DodgerBlue3}‚त्म‚स्नेहा}‚त्सुख‚विप‚र्यासात् दुःखेष्व‚पि सुख‚त‚याऽध्य‚व‚{\tiny $_{lb}$}‚सितेषु विष‚येषु य‚थालाभं प्राप्त्य‚नुक्र‚मेण प्र‚वृत्तिर्भ‚व‚ति निर्व्वाण‚सुख‚व्युत्प‚त्त्य‚भावात् । ‚{\tiny $_{lb}$}‚व्युत्प‚त्ताव‚पि त‚न्मार्ग‚सात्म‚त्वाभावात् । सात्मीकृत‚मार्ग्गास्तु क्व‚चिन्न प्र‚व‚र्त‚न्त‚{\tiny $_{1}$}‚ ‚{\tiny $_{lb}$}‚एव । त‚था ‚{\color{DodgerBlue3}‚चालाभे म‚त्त‚कासिन्या} म‚त्त‚ग‚ज‚गामिन्या कामुक‚स्य ‚{\color{DodgerBlue3}‚तिर्य‚क्षु कामिता ‚{\tiny $_{lb}$}‚दृष्टा} । ब‚ल‚वानात्म‚स्नेहो विशिष्ट‚सुख‚साध‚न‚स्यालाभे सुखाभास‚हेतौ च प्र‚व‚र्त‚{\tiny $_{lb}$}‚य‚ति । (२३५)
	\pend% ending standard par
      \label{div_pvv.1.236_1.237}
	  
	% new div opening: depth here is 2
	

	  \pstart \leavevmode% starting standard par
	c. किञ्च (।) य‚दि बुद्धीन्द्रिय‚श‚रीरादीनां दुःख‚हेतुत्वात् तेषु वैराग्यात्त‚{\tiny $_{lb}$}‚त्त्यागात् कैव‚ल्य‚मा‚{\color{DodgerBlue3}‚त्म‚न इष्टं} । मुक्तिद‚शायान्त‚दात्म‚भोगादिस‚क‚ल‚प‚रिच्छेदाभावात् ‚{\tiny $_{lb}$}‚नाशाविशेषान्नाश एवेष्टः स्यात् (।) त‚च्चेद‚म‚युक्त‚मित्याह (।)
	\pend% ending standard par
      
	  \bigskip
	  \begingroup
	
	    \large
	  
	    \begin{quote}
	  
	    
	    \stanza[\smallbreak]
	\label{pv.1.236a}\flagstanza{\tiny\textenglish{...1.236a}}य‚स्यात्मा व‚ल्ल‚भ‚स्त‚स्य स नाशं क‚थ‚मिच्छ‚ति ।\&[\smallbreak]


	
	    \end{quote}
	  
	  \endgroup
	

	  \pstart \leavevmode% starting standard par
	\hphantom{.}‚{\color{DodgerBlue3}‚य\edtext{}{\edlabel{pvv.91-3}\label{pvv.91-3}\lemma{य}\Bfootnote{सांख्य‚स्य ।}}स्यात्मा व‚ल्ल‚भ\edtext{}{\edlabel{pvv.91-4}\label{pvv.91-4}\lemma{भ}\Bfootnote{देही एव मुक्तः ।}}स्त‚स्य स क‚थं नाश‚मिच्छ‚ति} न नाशं ‚{\color{DodgerBlue3}‚कैव‚ल्य‚{\tiny $_{2}$}‚म‚प्र‚तीति}‚{\tiny $_{lb}$}‚विष‚य‚त्वादिच्छ‚ति ।
	\pend% ending standard par
      

	  \pstart \leavevmode% starting standard par
	क‚थं पुनः केव‚ल‚मात्मान‚मिच्छ‚न् नाश‚मिच्छ‚तीत्याह (।)
	\pend% ending standard par
      
	  \bigskip
	  \begingroup
	
	    \large
	  
	    \begin{quote}
	  
	    
	    \stanza[\smallbreak]
	\label{pv.1.236b}\flagstanza{\tiny\textenglish{...1.236b}}निवृत्त‚स‚र्वानुभ‚व‚व्य‚व‚हार‚गुण‚श्र‚य‚म् ॥ २३६ ॥\&[\smallbreak]


	
	    \end{quote}
	  
	  \endgroup
	
	  \bigskip
	  \begingroup
	
	    \large
	  
	    \begin{quote}
	  
	    
	    \stanza[\smallbreak]
	\label{pv.1.237a}\flagstanza{\tiny\textenglish{...1.237a}}इच्छेत् प्रेम क‚थं;\&[\smallbreak]


	
	    \end{quote}
	  
	  \endgroup
	\textsuperscript{\textenglish{092/s}}

	  \pstart \leavevmode% starting standard par
	\hphantom{.}‚{\color{DodgerBlue3}‚निवृत्तः स‚र्व्व}‚स्याव‚{\color{DodgerBlue3}‚नुभ‚व‚व्य‚व\edtext{}{\edlabel{pvv.92-1}\label{pvv.92-1}\lemma{व}\Bfootnote{सुखादिभोक्तेत्यादि ।}}हार}‚स्य स‚माश्र‚य आश्र‚य‚णं य‚स्मात्तं नाश‚ल‚क्ष‚णा‚{\tiny $_{lb}$}‚विशिष्ट‚मित्य‚र्थः । नाशा\edtext{}{\edlabel{pvv.92-2}\label{pvv.92-2}\lemma{नाशा}\Bfootnote{नैरात्म्ये विशेषात् ।}}विशिष्ट‚ञ्चात्मानं ‚{\color{DodgerBlue3}‚क‚थं प्रेम} स्नेहातिश‚य ‚{\color{DodgerBlue3}‚इच्छेत्} । ‚{\tiny $_{lb}$}‚‚{\color{DodgerBlue3}‚य‚स्मात्} (।)
	\pend% ending standard par
      
	  \bigskip
	  \begingroup
	
	    \large
	  
	    \begin{quote}
	  
	    
	    \stanza[\smallbreak]
	\label{pv.1.237b}\flagstanza{\tiny\textenglish{...1.237b}}प्रेम्णः प्र‚कृतिर्न हि तादृशी ।\&[\smallbreak]


	
	    \end{quote}
	  
	  \endgroup
	

	  \pstart \leavevmode% starting standard par
	\hphantom{.}‚{\color{DodgerBlue3}‚प्रेम्णः प्र‚कृतिस्तादृशी} स्व‚विष‚य‚नाशैष‚ण‚स्व‚भावा न भ‚व‚ति (।) त‚स्मात् (।)
	\pend% ending standard par
      
	  \bigskip
	  \begingroup
	
	    \large
	  
	    \begin{quote}
	  
	    
	    \stanza[\smallbreak]
	\label{pv.1.237c}\flagstanza{\tiny\textenglish{...1.237c}}स‚र्व‚थात्म‚ग्र‚हः स्नेह‚मात्म‚नि द्र‚ढ‚य‚त्य‚ल‚म् ॥ २३७ ॥\&[\smallbreak]


	
	    \end{quote}
	  
	  \endgroup
	

	  \pstart \leavevmode% starting standard par
	\hphantom{.}‚{\color{DodgerBlue3}‚स‚र्व्व‚था आत्म‚ग्र‚ह आत्म‚नि स्नेह‚म‚ल‚म}‚त्य‚र्थ ‚{\color{DodgerBlue3}‚द्र‚ढ‚य‚ति} (। २३७)
	\pend% ending standard par
      \label{div_pvv.1.238_1.239}
	  
	% new div opening: depth here is 2
	

	  \pstart \leavevmode% starting standard par
	स आत्म‚ग्र‚ह‚श्चात्मोप‚कारिषु ।
	\pend% ending standard par
      
	  \bigskip
	  \begingroup
	
	    \large
	  
	    \begin{quote}
	  
	    
	    \stanza[\smallbreak]
	\label{pv.1.238a}\flagstanza{\tiny\textenglish{...1.238a}}आत्मीय‚स्नेह‚बीज‚न्तु त‚द‚व‚स्थं व्य‚व‚स्थित‚म् ।\&[\smallbreak]


	
	    \end{quote}
	  
	  \endgroup
	

	  \pstart \leavevmode% starting standard par
	\hphantom{.}‚{\color{DodgerBlue3}‚आत्मीय‚स्नेह‚बीजं त‚द‚व‚स्थं व्य‚व‚स्थित}‚{\tiny $_{3}$}‚मिति त‚त्प्र‚तिब‚द्धानां दोषाणाञ्चानि‚{\tiny $_{lb}$}‚वृत्तिः ॥
	\pend% ending standard par
      

	  \pstart \leavevmode% starting standard par
	d. स्यादेत‚द् (।) आत्मीये दोष‚द‚र्श‚नाद्वैराग्य‚मुत्प‚द्य‚ते इत्याह (।)
	\pend% ending standard par
      
	  \bigskip
	  \begingroup
	
	    \large
	  
	    \begin{quote}
	  
	    
	    \stanza[\smallbreak]
	\label{pv.1.238b}\flagstanza{\tiny\textenglish{...1.238b}}य‚त्नेऽप्यात्मीय‚वैराग्यं गुण‚लेश‚स‚माश्र‚यात् ॥ २३८ ॥\&[\smallbreak]


	
	    \end{quote}
	  
	  \endgroup
	
	  \bigskip
	  \begingroup
	
	    \large
	  
	    \begin{quote}
	  
	    
	    \stanza[\smallbreak]
	\label{pv.1.239a}\flagstanza{\tiny\textenglish{...1.239a}}वृत्तिमान् प्र‚तिब‚ध्नाति, त‚द्दोषान् संवृणोति च ।\&[\smallbreak]


	
	    \end{quote}
	  
	  \endgroup
	

	  \pstart \leavevmode% starting standard par
	\hphantom{.}दोष‚द‚र्श‚नात् ‚{\color{DodgerBlue3}‚य‚त्नेपि} स‚ति ताव‚त्काल‚मा‚{\color{DodgerBlue3}‚त्मीये}‚षु ‚{\color{DodgerBlue3}‚वैराग्यं} य‚दुत्प‚न्नं त‚दात्म‚स्नेहो ‚{\tiny $_{lb}$}‚‚{\color{DodgerBlue3}‚वृत्तिमाना}‚त्मीयेषु ‚{\color{DodgerBlue3}‚गुण‚ले}‚श‚स्य सुख‚साध‚न‚त्व‚स्य ‚{\color{DodgerBlue3}‚स‚माश्र‚य}‚णात् (२३८) ‚{\color{DodgerBlue3}‚प्र‚तिब‚ध्नाति । ‚{\tiny $_{lb}$}‚त‚द्दोषाँश्च} दुःख‚साध‚नादीन् ‚{\color{DodgerBlue3}‚संवृणोति} । त‚त् कुत आत्म‚स्नेह‚व‚त आत्मीये वैराग्य‚{\tiny $_{lb}$}‚योगः । आत्म‚स्नेह‚स्य स‚र्व्व‚दोष‚मूल‚त्वात्‚{\tiny $_{4}$}‚ (।)
	\pend% ending standard par
      
	  \bigskip
	  \begingroup
	
	    \large
	  
	    \begin{quote}
	  
	    
	    \stanza[\smallbreak]
	\label{pv.1.239b}\flagstanza{\tiny\textenglish{...1.239b}}आत्म‚न्य‚पि विराग‚श्चेदिदानीं यो विर‚ज्य‚ते ॥ २३९ ॥\&[\smallbreak]


	
	    \end{quote}
	  
	  \endgroup
	

	  \pstart \leavevmode% starting standard par
	\hphantom{.}‚{\color{DodgerBlue3}‚आत्म‚न्य‚पि विराग‚श्चे}‚द्वाध्य‚ते । न‚नूक्त‚म‚त्र । न चात्म‚नि निर्दोषे स्नेहाप‚ग‚म‚{\tiny $_{lb}$}‚कार‚ण‚म‚स्ति । भ‚व‚तु ताव‚त्त‚थापीदानीम‚त्रापि प‚क्षे यो त‚त्र ‚{\color{DodgerBlue3}‚विर‚ज्य‚ते} (। २३९)
	\pend% ending standard par
      \label{div_pvv.1.240}
	  
	% new div opening: depth here is 2
	
	  \bigskip
	  \begingroup
	
	    \large
	  
	    \begin{quote}
	  
	    
	    \stanza[\smallbreak]
	\label{pv.1.240}\flagstanza{\tiny\textenglish{....1.240}}त्य‚ज‚त्य‚सौ य‚थात्मानं व्य‚र्थाऽतो दुःख‚भाव‚ना ।&दुःख‚भाव‚न‚याऽप्येष दुःख‚मेव विभाव‚येत् ॥ २४० ॥\&[\smallbreak]


	
	    \end{quote}
	  
	  \endgroup
	

	  \pstart \leavevmode% starting standard par
	\hphantom{.}तेन स तं ‚{\color{DodgerBlue3}‚त्य‚ज‚ति य‚थात्मानं} । न ह्यात्म‚नि विर‚क्तोपि तं त्य‚ज‚ति, त‚थात्मी‚{\tiny $_{lb}$}‚येपि विर‚क्त‚स्तं न त्य‚क्ष्य‚तीति ‚{\color{DodgerBlue3}‚व्य‚र्थ[ा]तो दुःख‚भाव}‚नाऽत्म‚नोऽत्यागात् । ‚{\color{DodgerBlue3}‚दुःख-} \leavevmode\ledsidenote{\textenglish{093/s}} ‚{\color{DodgerBlue3}‚भाव‚न‚यापि} एष भाव‚को ‚{\color{DodgerBlue3}‚दुःख‚मेव} भाव्य‚मानं ‚{\color{DodgerBlue3}‚विभाव‚येत्} प्र‚का\edtext{}{\edlabel{pvv.93-1}\label{pvv.93-1}\lemma{का}\Bfootnote{प्र‚त्य‚क्षाव‚सान‚त्वाद् भाव‚नायाः ।}}श‚येत् । (२४०)
	\pend% ending standard par
      \label{div_pvv.1.241}
	  
	% new div opening: depth here is 2
	
	  \bigskip
	  \begingroup
	
	    \large
	  
	    \begin{quote}
	  
	    
	    \stanza[\smallbreak]
	\label{pv.1.241}\flagstanza{\tiny\textenglish{....1.241}}प्र‚त्य‚क्षं पूर्व‚म‚पि त‚त् त‚थापि न विराग‚वान् ॥&य‚द्य‚प्येक‚त्र दोषेण त‚त्क्ष‚णं च‚लिता म‚तिः ॥ २४१ ॥\&[\smallbreak]


	
	    \end{quote}
	  
	  \endgroup
	

	  \pstart \leavevmode% starting standard par
	\hphantom{.}त‚च्च भाव‚नातः ‚{\color{DodgerBlue3}‚पूर्व्व‚म‚पि प्र‚त्य‚क्ष}‚मेव दुःख‚मात्म\edtext{}{\edlabel{pvv.93-2}\label{pvv.93-2}\lemma{मात्म}\Bfootnote{इति व्य‚र्था भाव‚ना ।}}स्नेहादिश‚स्त्र‚प्र‚हाराद्य‚नुभ‚व‚{\tiny $_{lb}$}‚काले ‚{\color{DodgerBlue3}‚त‚थापि} प्र‚त्य‚क्षीकृतात्म‚स्नेहादिदुःख‚त्वे‚{\tiny $_{5}$}‚‚{\color{DodgerBlue3}‚पि} न ‚{\color{DodgerBlue3}‚विराग‚वान्} तेषु क‚श्र्चित्त‚दा । ‚{\tiny $_{lb}$}‚त‚तो भाव‚नाप्र‚क‚र्षेप्येवं स्यात् साक्षात्क‚र‚ण‚त्वात्त‚स्य । ‚{\color{DodgerBlue3}‚त‚च्च न} विराग‚हेतुः प्रागिव । ‚{\tiny $_{lb}$}‚य‚द्य‚प्येक‚त्राप‚राध‚कारिणि दोष‚द‚र्श‚नात् ‚{\color{DodgerBlue3}‚त‚त्क्ष‚णं} निय‚त‚काल‚म‚नुरागा‚{\color{DodgerBlue3}‚च्च‚लिता म‚ति-} र्व्विराग‚भ‚ज‚नात् (। २४१)
	\pend% ending standard par
      \label{div_pvv.1.242}
	  
	% new div opening: depth here is 2
	
	  \bigskip
	  \begingroup
	
	    \large
	  
	    \begin{quote}
	  
	    
	    \stanza[\smallbreak]
	\label{pv.1.242}\flagstanza{\tiny\textenglish{....1.242}}विर‚क्तो नैव त‚त्रापि कामीव व‚नितान्त‚रे ॥&त्याज्योपादेय‚भेदे हि स‚क्तिर्यैवैक‚भाविनी ॥ २४२ ॥\&[\smallbreak]


	
	    \end{quote}
	  
	  \endgroup
	

	  \pstart \leavevmode% starting standard par
	\hphantom{.}त‚थाप्य‚सौ ‚{\color{DodgerBlue3}‚त‚त्र विर‚क्तः} स‚र्व्व‚था प‚र्यायेण रागोत्प‚त्तेः । किं पुन‚र‚न्य‚त्र ‚{\color{DodgerBlue3}‚कामीव} क्व‚चित् कामिन्यां राग‚कारिण्यां विर‚क्तोपि ‚{\color{DodgerBlue3}‚न व‚नितान्त‚रे} विर‚क्तः । त‚स्याम‚पि ‚{\tiny $_{lb}$}‚वा क्र‚मे‚{\tiny $_{6}$}‚ण ।
	\pend% ending standard par
      

	  \pstart \leavevmode% starting standard par
	\hphantom{.}किञ्च (।) प्र‚तिधानुन‚य‚विष‚य‚त्वात् ‚{\color{DodgerBlue3}‚त्याज्योपादेय‚भे\edtext{}{\edlabel{pvv.93-3}\label{pvv.93-3}\lemma{भे}\Bfootnote{आत्म‚नीत्यादिव‚द‚य‚मुप‚देशः ।}}दे हि स‚ति स‚क्तिरा‚{\tiny $_{lb}$}‚स‚क्तिर्यैवे}‚क‚स्मिन् ‚{\color{DodgerBlue3}‚भाविनी} द्वेष‚विष‚य‚त‚याऽनुराग‚विष‚य‚त‚या वा (। २४२)
	\pend% ending standard par
      \label{div_pvv.1.243}
	  
	% new div opening: depth here is 2
	
	  \bigskip
	  \begingroup
	
	    \large
	  
	    \begin{quote}
	  
	    
	    \stanza[\smallbreak]
	\label{pv.1.243}\flagstanza{\tiny\textenglish{....1.243}}सा बीजं स‚र्व‚स‚क्तीनां प‚र्यायेण स‚मुद्भ‚वे ॥&निर्दोष‚विष‚यः स्नेहो निर्दोषः साध‚नानि च ॥ २४३ ॥\&[\smallbreak]


	
	    \end{quote}
	  
	  \endgroup
	

	  \pstart \leavevmode% starting standard par
	\hphantom{.}e. ‚{\color{DodgerBlue3}‚सा} स‚क्ति‚{\color{DodgerBlue3}‚र्बीजं} कार‚णं ‚{\color{DodgerBlue3}‚स‚र्व्व‚स‚क्तीनां प‚र्यायेण} प‚रिपाट्‏या ‚{\color{DodgerBlue3}‚स‚मुद्भ‚व}‚निमित्तं । ‚{\tiny $_{lb}$}‚त‚था हि क्व‚चिद् द्वेषास‚क्त्या त‚द‚नुकूल‚प्र‚तिकूल‚योः प्र‚तिधानुन‚यौ भ‚व‚तः । ‚{\tiny $_{lb}$}‚त‚थानुरागास‚क्त्यापि क्व‚चित्त‚योरेवानु‚{\color{DodgerBlue3}‚न‚य‚द्वेषौ भ‚व‚तः} । त‚देव‚म‚मात्नो निर्दोष‚त्वा‚{\tiny $_{lb}$}‚‚{\color{DodgerBlue3}‚न्निर्दोष‚विष‚यः} स्नेहो ‚{\color{DodgerBlue3}‚निर्दो‚{\tiny $_{7}$}‚षः} स्व‚यं ‚{\color{DodgerBlue3}‚उप‚भोग‚साध‚नानि} चेन्द्रिय‚श‚रीरादिनि ‚{\tiny $_{lb}$}‚श‚ब्द‚र‚स‚रूपादीनि निर्दोषाणि । आत्मादीनां स‚र्व्वेषां प्र‚त्येकं दुःख‚हेतुत्वाभावात् ।\edtext{\textsuperscript{*}}{\edlabel{pvv.93-4}\label{pvv.93-4}\lemma{*}\Bfootnote{निर्द्दोष‚स्यात्म‚नः स‚तः ।}} ‚{\tiny $_{lb}$}‚(२४३)
	\pend% ending standard par
      \label{div_pvv.1.244}
	  
	% new div opening: depth here is 2
	
	  \bigskip
	  \begingroup
	
	    \large
	  
	    \begin{quote}
	  
	    
	    \stanza[\smallbreak]
	\label{pv.1.244}\flagstanza{\tiny\textenglish{....1.244}}एताव‚देव च ज‚ग‚त् क्वेदानीं स विर‚ज्य‚ते ॥&स‚दोष‚ताऽपि चेत् त‚स्य त‚त्रात्म‚न्य‚पि सा स‚मा ॥ २४४ ॥\&[\smallbreak]


	
	    \end{quote}
	  
	  \endgroup
	\textsuperscript{\textenglish{094/s}}

	  \pstart \leavevmode% starting standard par
	\hphantom{.}‚{\color{DodgerBlue3}‚एता\edtext{}{\edlabel{pvv.94-1}\label{pvv.94-1}\lemma{एता}\Bfootnote{आत्म‚स्नेहः साध‚न‚ञ्च ।}}व‚देव च ज‚ग‚त्} । त्रिभिर्ज‚ग‚तः संग्र‚हात् । ‚{\color{DodgerBlue3}‚क्वेदानीं स मो}‚क्तुकामो ‚{\color{DodgerBlue3}‚विर‚{\tiny $_{lb}$}‚ज्य‚ते} ॥ स‚मुदायाद्दोष‚द‚र्श‚नात् त‚त एव विर‚ज्य‚ते चेत् । न‚न्वेव‚मात्म‚न्य‚पि वैराग्यं ‚{\tiny $_{lb}$}‚प्राप्तं ॥ न केव‚लं गुण‚व‚त्ता ‚{\color{DodgerBlue3}‚स‚दोष‚तापि चे\edtext{}{\edlabel{pvv.94-2}\label{pvv.94-2}\lemma{चे}\Bfootnote{दुःखाश्र‚य‚त्वात् ।}}त्त‚स्य} स्नेहेन्द्रियादेः । ‚{\color{DodgerBlue3}‚त‚त्रात्म‚न्य‚पि} कैव‚ल्येनेष्टे ‚{\color{DodgerBlue3}‚सा} स‚दोष‚ता‚{\tiny $_{3}$}‚ ‚{\color{DodgerBlue3}‚स‚मा} त‚स्या अपि स्नेहादिदोष‚व‚त्त्वात् । (२४४)
	\pend% ending standard par
      \label{div_pvv.1.245}
	  
	% new div opening: depth here is 2
	\textsuperscript{\textenglish{18a/MA}}

	  \pstart \leavevmode% starting standard par
	f. एव‚न्त‚र्ह्यात्म‚दोष‚मेव वैराग्य‚भाव‚नाया ज‚ह्यादिति चेत् ।
	\pend% ending standard par
      
	  \bigskip
	  \begingroup
	
	    \large
	  
	    \begin{quote}
	  
	    
	    \stanza[\smallbreak]
	\label{pv.1.245a}\flagstanza{\tiny\textenglish{...1.245a}}त‚त्राविर‚क्त‚स्त‚द्दोषे क्वेदानीं स विर‚ज्य‚ते ।\&[\smallbreak]


	
	    \end{quote}
	  
	  \endgroup
	

	  \pstart \leavevmode% starting standard par
	\hphantom{.}‚{\color{DodgerBlue3}‚त‚त्रात्म}‚न्य‚{\color{DodgerBlue3}‚विर‚क्त}‚स्त‚{\color{DodgerBlue3}‚द्दोषे क्वेदा}‚नीमात्म‚द‚र्श‚न‚काले ‚{\color{DodgerBlue3}‚स} मुमुक्षु‚{\color{DodgerBlue3}‚र्विर}‚ज्य‚ते । य‚था ‚{\tiny $_{lb}$}‚स‚दोषेप्यात्म‚न्यात्म‚द‚र्श‚नाद‚विर‚क्त‚स्त‚था त‚द्दोषेपि स्नेहादात्मीय‚त्व‚द‚र्श‚नान्न विर‚{\tiny $_{lb}$}‚ज्येत (।) अपि च (।)
	\pend% ending standard par
      
	  \bigskip
	  \begingroup
	
	    \large
	  
	    \begin{quote}
	  
	    
	    \stanza[\smallbreak]
	\label{pv.1.245b}\flagstanza{\tiny\textenglish{...1.245b}}गुण‚द‚र्श‚न‚स‚म्भूतं स्नेहं बाधित‚दोष‚दृक् ॥ २४५ ॥\&[\smallbreak]


	
	    \end{quote}
	  
	  \endgroup
	

	  \pstart \leavevmode% starting standard par
	\hphantom{.}‚{\color{DodgerBlue3}‚गुण‚द‚र्श‚न‚स‚म्भूतं स्नेहं बार्धित‚दोष}‚स्य ‚{\color{DodgerBlue3}‚दृक्} दृष्टिः (२४५)
	\pend% ending standard par
      \label{div_pvv.1.246_1.247}
	  
	% new div opening: depth here is 2
	
	  \bigskip
	  \begingroup
	
	    \large
	  
	    \begin{quote}
	  
	    
	    \stanza[\smallbreak]
	\label{pv.1.246}\flagstanza{\tiny\textenglish{....1.246}}स चेन्द्रियादौ न त्वेवं बालादेर‚पि स‚म्भ‚वात् ।&दोष‚व‚त्य‚पि स‚द्भावात् स्व‚भावाद् गुण‚व‚त्य‚पि ॥ २४६ ॥\&[\smallbreak]


	
	    \end{quote}
	  
	  \endgroup
	
	  \bigskip
	  \begingroup
	
	    \large
	  
	    \begin{quote}
	  
	    
	    \stanza[\smallbreak]
	\label{pv.1.247a}\flagstanza{\tiny\textenglish{...1.247a}}अन्य‚त्र;\&[\smallbreak]


	
	    \end{quote}
	  
	  \endgroup
	

	  \pstart \leavevmode% starting standard par
	\hphantom{.}‚{\color{DodgerBlue3}‚स च} स्नेह‚जे‚{\color{DodgerBlue3}‚न्द्रियादावे}‚वं गुण\edtext{}{\edlabel{pvv.94-3}\label{pvv.94-3}\lemma{गुण}\Bfootnote{आत्मीय‚बुद्धेर‚बाध‚कं दुःख‚द‚र्श‚नं क‚थ‚मात्मीय‚स्नेह‚म‚प‚नुदेत् ।}}द‚र्श‚नान्न दृष्टः । ‚{\color{DodgerBlue3}‚बालादेर‚पि} गुण‚प‚रीक्षाऽभ्यु‚{\tiny $_{lb}$}‚(प‚ग)म्य च‚क्षुरादावात्मी‚{\tiny $_{1}$}‚य‚त्व‚मात्रेण स्नेह‚स्य ‚{\color{DodgerBlue3}‚स‚म्भ‚वात्} । गुण‚दोष‚द‚र्श‚नात् न स्नेह‚{\tiny $_{lb}$}‚भावाभावी । किन्त्वात्मीय‚त्व‚द‚र्श‚नाद‚र्श‚नात्त‚द‚न्व‚य‚व्य‚तिरेकानुविधानादि\edtext{}{\edlabel{pvv.94-4}\label{pvv.94-4}\lemma{तिरेकानुविधानादि}\Bfootnote{त‚देवाह ।}}ति । द‚र्श‚{\tiny $_{lb}$}‚य‚ति च\edtext{}{\edlabel{pvv.94-5}\label{pvv.94-5}\lemma{च}\Bfootnote{शिष्यान् प‚र‚तः पूर्व्वार्थ‚स‚म‚र्थ‚न‚माचार्यः ।}} स्व‚कीये च‚क्षुरादौ गुण‚विक‚ले काण‚त्वादि‚{\color{DodgerBlue3}‚दोष‚व‚त्य‚प्या}‚त्मीय‚त्व‚प‚राम‚र्शात् ‚{\tiny $_{lb}$}‚स्नेह‚स्य ‚{\color{DodgerBlue3}‚स‚द्भावाद}‚न्य‚त्र प‚र‚कीये नेत्रादौ दोष‚र‚हिते ‚{\color{DodgerBlue3}‚गुण‚व‚त्य}\edtext{}{\edlabel{pvv.94-6}\label{pvv.94-6}\lemma{हिते}\Bfootnote{स्नेहाभावात् ।}}पि (। २४६)
	\pend% ending standard par
      
	  \bigskip
	  \begingroup
	
	    \large
	  
	    \begin{quote}
	  
	    
	    \stanza[\smallbreak]
	\label{pv.1.247b}\flagstanza{\tiny\textenglish{...1.247b}}आत्मीय‚तायां वा व्य‚तीतादौ विहानितः ।&त‚त एव च नात्मीय‚बुद्धेर‚पि गुणेक्ष‚ण‚म् ॥ २४७ ॥\&[\smallbreak]


	
	    \end{quote}
	  
	  \endgroup
	
	  \bigskip
	  \begingroup
	
	    \large
	  
	    \begin{quote}
	  
	    
	    \stanza[\smallbreak]
	\label{pv.1.248a}\flagstanza{\tiny\textenglish{...1.248a}}कार‚ण‚म्;\&[\smallbreak]


	
	    \end{quote}
	  
	  \endgroup
	

	  \pstart \leavevmode% starting standard par
	\hphantom{.}‚{\color{DodgerBlue3}‚आत्मीय‚तायाम‚पि वाऽतीतादौ} केश‚न‚खादौ आदिश‚ब्दाल्लूनाङ्गुल्यादौ व‚र्त‚{\tiny $_{lb}$}‚मानेन स्नेह आत्मीय‚त्वेनादृष्टे‚{\color{DodgerBlue3}‚र्व्विहानितः} प‚रि‚{\tiny $_{2}$}‚त्यागात् स्व‚त्व‚स्य ‚{\color{DodgerBlue3}‚त‚त एव च} बाला‚{\tiny $_{lb}$}‚देर‚पि भावात् । ‚{\color{DodgerBlue3}‚आत्मीय‚बुद्धेर‚पि न गुणेक्ष‚णं कार‚णं} किन्त्वात्म‚द‚र्श‚न‚मेव (। २४७)
	\pend% ending standard par
      \label{div_pvv.1.248_1.249}
	  
	% new div opening: depth here is 2
	\textsuperscript{\textenglish{095/s}}
	  \bigskip
	  \begingroup
	
	    \large
	  
	    \begin{quote}
	  
	    
	    \stanza[\smallbreak]
	\label{pv.1.248b}\flagstanza{\tiny\textenglish{...1.248b}}हीय‚ते साऽपि त‚स्मान्नागुण‚द‚र्श‚नात् ।\&[\smallbreak]


	
	    \end{quote}
	  
	  \endgroup
	

	  \pstart \leavevmode% starting standard par
	\hphantom{.}‚{\color{DodgerBlue3}‚त‚स्माद् गुण}‚द‚र्श‚न‚हेतुक‚त्वाभावात् सा आत्मीय‚बुद्धिर‚पि अगुण‚स्य ‚{\color{DodgerBlue3}‚दोष‚स्य ‚{\tiny $_{lb}$}‚द‚र्श‚ना}‚न्न हीय‚ते । कार‚ण‚विरुद्धो हि ध‚र्मी निव‚र्त‚कः ‚{\color{DodgerBlue3}‚क‚स्य‚चिद्य‚थाग्नी रोमाञ्च}‚{\tiny $_{lb}$}‚विशेष‚स्य । आत्म‚द‚र्श‚नं हि स्नेहात्मीय‚दृशादेः कार‚णं न च त‚द्विरोधिनी ‚{\color{DodgerBlue3}‚दोष‚दृक्} ।
	\pend% ending standard par
      
	  \bigskip
	  \begingroup
	
	    \large
	  
	    \begin{quote}
	  
	    
	    \stanza[\smallbreak]
	\label{pv.1.248c}\flagstanza{\tiny\textenglish{...1.248c}}अपि चास‚द्गुणारोपः स्नेहात् त‚त्र हि दृश्य‚ते ॥ २४८ ॥\&[\smallbreak]


	
	    \end{quote}
	  
	  \endgroup
	
	  \bigskip
	  \begingroup
	
	    \large
	  
	    \begin{quote}
	  
	    
	    \stanza[\smallbreak]
	\label{pv.1.249a}\flagstanza{\tiny\textenglish{...1.249a}}त‚स्मात् त‚त्कार‚ण‚बाधी विधिस्तं बाध‚ते क‚थ‚म् ।\&[\smallbreak]


	
	    \end{quote}
	  
	  \endgroup
	

	  \pstart \leavevmode% starting standard par
	\hphantom{.}‚{\color{DodgerBlue3}‚अपि चास‚तां गुणानामारोप‚स्त‚त्रा}‚त्मीये ‚{\color{DodgerBlue3}‚स्नेहा‚{\tiny $_{3}$}‚द्धि} य‚स्माद् ‚{\color{DodgerBlue3}‚दृश्य‚ते} (२४८) ‚{\tiny $_{lb}$}‚‚{\color{DodgerBlue3}‚त‚स्मात्त}‚स्य स्नेहादेः ‚{\color{DodgerBlue3}‚कार‚ण}‚स्यात्म‚द‚र्श‚न‚स्या‚{\color{DodgerBlue3}‚बाधी} अबाध‚को ‚{\color{DodgerBlue3}‚विधि}‚र्दीक्षा दुःख‚भाव‚{\tiny $_{lb}$}‚नादिरूपः ‚{\color{DodgerBlue3}‚तं} स्नेहादिं ‚{\color{DodgerBlue3}‚बाध‚ते क‚थं} । कार‚णानिवृत्त्या कार्य‚निषेध‚स्य क‚र्तुम‚श‚क्य‚त्वात् ।
	\pend% ending standard par
      

	  \begin{center}%% label @type='head'
	\textbf{(च) प्र‚कृतिपुरुष‚योर्भेंद‚प्र‚तीताव‚पि न मोक्षः}
	\end{center}
	

	  \pstart \leavevmode% starting standard par
	सां ख्या स्तु म‚न्यंते । चेत‚नाचेत‚न‚योः पुरुष‚प्र‚धा\edtext{}{\edlabel{pvv.95-1}\label{pvv.95-1}\lemma{धा}\Bfootnote{प्र‚कृतिः}}न‚योर्याव‚दैक्यं म‚न्य‚ते पुरुषः । ‚{\tiny $_{lb}$}‚ताव‚त्स स्नेह‚वान् अयुक्त‚श्च भेद‚प्र‚तीतौ न स्नेहो वियुक्त‚श्चेति । अत्राह\edtext{}{\edlabel{pvv.95-2}\label{pvv.95-2}\lemma{अत्राह}\Bfootnote{एक‚बुद्धिरेवेन्द्रियादिष्वात्म‚नो नास्त्य‚तः पृथ‚गात्म‚नो वेद‚नादित्याह ।}} (।)
	\pend% ending standard par
      
	  \bigskip
	  \begingroup
	
	    \large
	  
	    \begin{quote}
	  
	    
	    \stanza[\smallbreak]
	\label{pv.1.249b}\flagstanza{\tiny\textenglish{...1.249b}}प‚राप‚र‚प्रार्थ‚नातो विनाशोत्पाद‚बुद्धितः ॥ २४९ ॥\&[\smallbreak]


	
	    \end{quote}
	  
	  \endgroup
	

	  \pstart \leavevmode% starting standard par
	\hphantom{.}काण‚त्वादिदोष‚युक्ता ‚{\color{DodgerBlue3}‚प‚राप‚र}‚स्य विशिष्ट‚विशिष्ट‚स्य च‚क्षुःश‚{\tiny $_{4}$}‚रीरादिक‚स्य ‚{\tiny $_{lb}$}‚‚{\color{DodgerBlue3}‚प्रार्थ‚नातः} । आत्म‚न‚श्चान्य‚स्यान‚भिलाष‚तः । त‚स्मात्पृथ‚ग्भूत‚मात्मान‚म‚य‚म‚मुक्तोपि ‚{\tiny $_{lb}$}‚ज‚नो वेत्ति । त‚था ‚{\color{DodgerBlue3}‚विनाशोत्पाद‚बुद्धितः} (। २४९)
	\pend% ending standard par
      \label{div_pvv.1.250}
	  
	% new div opening: depth here is 2
	
	  \bigskip
	  \begingroup
	
	    \large
	  
	    \begin{quote}
	  
	    
	    \stanza[\smallbreak]
	\label{pv.1.250}\flagstanza{\tiny\textenglish{....1.250}}इन्द्रियादौ पृथ‚ग्भूत‚मात्मानं वेत्त्य‚यं ज‚नः ।&त‚स्मान्नैक‚त्व‚दृष्ट्यापि स्नेहः । स्निह्य‚न् स आत्म‚नि ॥ २५० ॥\&[\smallbreak]


	
	    \end{quote}
	  
	  \endgroup
	

	  \pstart \leavevmode% starting standard par
	\hphantom{.}श‚रीरे‚{\color{DodgerBlue3}‚न्द्रियादौ} विप‚र्य‚याच्चात्म‚नि ‚{\color{DodgerBlue3}‚भिन्न‚मात्मानं} तेभ्यो ‚{\color{DodgerBlue3}‚वेत्ति । त‚स्मान्नैक‚त्व‚{\tiny $_{lb}$}‚दृष्ट्यापि स्नेहः} किन्त्वात्म‚द‚र्श‚नात् ॥ ‚{\color{DodgerBlue3}‚स} आत्म‚द‚र्शी ‚{\color{DodgerBlue3}‚स्निह्य‚न्नात्म‚नि} (। २५०)
	\pend% ending standard par
      \label{div_pvv.1.251}
	  
	% new div opening: depth here is 2
	
	  \bigskip
	  \begingroup
	
	    \large
	  
	    \begin{quote}
	  
	    
	    \stanza[\smallbreak]
	\label{pv.1.251}\flagstanza{\tiny\textenglish{....1.251}}उप‚ल‚म्भान्त‚र‚ङ्गेषु प्र‚कृत्यैवानुर‚ज्य‚ते ॥&प्र‚त्युत्प‚न्नात् तु यो दुःखान्निर्वेदो द्वेष ईदृशः ॥ २५१ ॥\&[\smallbreak]


	
	    \end{quote}
	  
	  \endgroup
	
	  \bigskip
	  \begingroup
	
	    \large
	  
	    \begin{quote}
	  
	    
	    \stanza[\smallbreak]
	\label{pv.1.252a}\flagstanza{\tiny\textenglish{...1.252a}}न वैराग्यं;\&[\smallbreak]


	
	    \end{quote}
	  
	  \endgroup
	

	  \pstart \leavevmode% starting standard par
	\hphantom{.}‚{\color{DodgerBlue3}‚उप‚ल‚म्भान्त‚र‚ङ्गेषू}‚प‚भोग‚साध‚नेष्विन्द्रियादिषु ‚{\color{DodgerBlue3}‚प्र‚कृत्या} स्व‚भावेनै‚{\color{DodgerBlue3}‚वानुर‚ज्य‚ते} । ‚{\tiny $_{lb}$}‚‚{\color{DodgerBlue3}‚प्र‚त्युत्प‚न्नात्तु} व‚र्त‚मानात्पुन‚{\tiny $_{5}$}‚‚{\color{DodgerBlue3}‚र्दुःखान्निर्व्वेदो यः} स न ‚{\color{DodgerBlue3}‚वैराग्यं} किन्तु ‚{\color{DodgerBlue3}‚द्वेष ईदृशः\edtext{}{\edlabel{pvv.95-3}\label{pvv.95-3}\lemma{ईदृशः}\Bfootnote{स्यादेत‚दात्मीय‚स्नेह‚स्यात्मीय‚बुद्धिरेव हेतुः सा तु गुण‚द‚र्श‚नादित्याह ।}}} । (२५१)
	\pend% ending standard par
      \label{div_pvv.1.252}
	  
	% new div opening: depth here is 2
	\textsuperscript{\textenglish{096/s}}
	  \bigskip
	  \begingroup
	
	    \large
	  
	    \begin{quote}
	  
	    
	    \stanza[\smallbreak]
	\label{pv.1.252b}\flagstanza{\tiny\textenglish{...1.252b}}त‚द‚प्य‚स्य स्नेहोऽव‚स्थान्त‚रेष‚णात् ।&द्वेष‚स्य दुःख‚योनित्वात् स ताव‚न्मात्र‚संस्थितिः ॥ २५२ ॥\&[\smallbreak]


	
	    \end{quote}
	  
	  \endgroup
	

	  \pstart \leavevmode% starting standard par
	\hphantom{.}य‚स्मा‚{\color{DodgerBlue3}‚त्त‚दापि} निर्व्वेदाव‚स्थायां ‚{\color{DodgerBlue3}‚स्नेहो}‚स्यास्ति न च विर‚क्त‚स्य स्नेह‚स‚म्भ‚वः । ‚{\tiny $_{lb}$}‚त‚द‚स्तित्व‚मेव कुत इति चेत् । ‚{\color{DodgerBlue3}‚अव‚स्थान्त}‚र‚स्य दुःख‚हेतोर्निर्व्वेद‚कारिण्यां अव‚स्थायां ‚{\tiny $_{lb}$}‚विल‚क्ष‚ण‚स्यै‚{\color{DodgerBlue3}‚ष‚णात्} । न हि स्नेह‚म‚न्त‚रेणैक‚त्यागाद‚प‚र‚वाञ्छा । ‚{\color{DodgerBlue3}‚द्वेष‚स्य दुःख‚स्य योनि‚{\tiny $_{lb}$}‚त्वात् । स} निर्व्वेदाख्यो द्वेषो याव‚द् दुःख‚म‚नुव‚र्त‚ते ‚{\color{DodgerBlue3}‚ताव‚न्मात्रं} ताव‚त्काल‚प‚रिमाणं ‚{\tiny $_{lb}$}‚‚{\color{DodgerBlue3}‚संस्थिति}‚र‚स्ये‚{\tiny $_{6}$}‚ति (। २५२)
	\pend% ending standard par
      \label{div_pvv.1.253_1.254}
	  
	% new div opening: depth here is 2
	

	  \begin{center}%% label @type='head'
	\textbf{(छ) हानोपादान‚हानित औदासीन्य‚म्}
	\end{center}
	
	  \bigskip
	  \begingroup
	
	    \large
	  
	    \begin{quote}
	  
	    
	    \stanza[\smallbreak]
	\label{pv.1.253a}\flagstanza{\tiny\textenglish{...1.253a}}त‚स्मिन् निवृत्तेप्र‚कृतिं स्वामेव भ‚ज‚ते पुनः ।\&[\smallbreak]


	
	    \end{quote}
	  
	  \endgroup
	

	  \pstart \leavevmode% starting standard par
	\hphantom{.}त‚था ‚{\color{DodgerBlue3}‚त‚स्मिन्} दुःखे कार‚ण‚निरोधा‚{\color{DodgerBlue3}‚न्निवृत्ते पुनः स्वामेव प्र‚कृतिं} विष‚येष्व‚विराग‚{\tiny $_{lb}$}‚ल‚क्ष‚णां ‚{\color{DodgerBlue3}‚भ‚ज‚ते} स‚त्त्व‚द‚र्शी ।
	\pend% ending standard par
      

	  \pstart \leavevmode% starting standard par
	कीदृशं त‚र्हि वैराग्यं युक्तं\edtext{}{\edlabel{pvv.96-1}\label{pvv.96-1}\lemma{युक्तं}\Bfootnote{आह ।}} ।
	\pend% ending standard par
      
	  \bigskip
	  \begingroup
	
	    \large
	  
	    \begin{quote}
	  
	    
	    \stanza[\smallbreak]
	\label{pv.1.253b}\flagstanza{\tiny\textenglish{...1.253b}}औदासीन्यं तु स‚र्व‚त्र त्यागोपादान‚हानितः ॥ २५३ ॥\&[\smallbreak]


	
	    \end{quote}
	  
	  \endgroup
	
	  \bigskip
	  \begingroup
	
	    \large
	  
	    \begin{quote}
	  
	    
	    \stanza[\smallbreak]
	\label{pv.1.254a}\flagstanza{\tiny\textenglish{...1.254a}}वासीच‚न्द‚न‚क‚ल्पानां वैराग्यं नाम क‚थ्य‚ते ।\&[\smallbreak]


	
	    \end{quote}
	  
	  \endgroup
	

	  \pstart \leavevmode% starting standard par
	स‚त्त्व‚दृष्ट्य‚भावात् स‚र्व्व‚त्र विष\edtext{}{\edlabel{pvv.96-2}\label{pvv.96-2}\lemma{विष}\Bfootnote{आत्म‚भावे त‚दुप‚क‚र‚णे च ।}}ये प्र‚तिकूल‚त्वानुकूल‚त्वाभ्यां ‚{\color{DodgerBlue3}‚अन‚ध्य‚व‚सिते ‚{\tiny $_{lb}$}‚त्यागोप\edtext{}{\edlabel{pvv.96-3}\label{pvv.96-3}\lemma{त्यागोप}\Bfootnote{उद्वेगोहं म‚मेति ग्र‚ह‚श्च ।}}दान‚योर्हानितो (२५३) वासीच‚न्द्र‚न‚क‚ल्प‚नां} वासीच‚न्द‚न‚योः क‚ल्पाः ‚{\tiny $_{lb}$}‚स‚दृशा ये वासीच‚न्द‚न‚क‚ल्पा वा ये तेषां साक्षात्कृत‚नैरात्म्य‚त‚त्त्वानामौदासीन्य‚म‚नु‚{\tiny $_{lb}$}‚न‚य‚प्र‚तिघ‚{\tiny $_{7}$}‚र‚हित‚त्वं ‚{\color{DodgerBlue3}‚पुन‚र्वैराग्यं नाम} आग‚म‚प्र‚सिद्धं ‚{\color{DodgerBlue3}‚क‚थ्य‚ते} ।
	\pend% ending standard par
      

	  \begin{center}%% label @type='head'
	\textbf{I. संस्कार‚दुःख‚भावात् दुःख‚भाव‚ना}
	\end{center}
	

	  \pstart \leavevmode% starting standard par
	न‚नु य‚दि दुःख‚भाव‚न‚या स्नेहादिहान्या न मुक्तिः त‚त्क‚थं भ‚ग‚व‚तोक्ता दुःख‚{\tiny $_{lb}$}‚भाव‚नेत्याह (।)
	\pend% ending standard par
      
	  \bigskip
	  \begingroup
	
	    \large
	  
	    \begin{quote}
	  
	    
	    \stanza[\smallbreak]
	\label{pv.1.254b}\flagstanza{\tiny\textenglish{...1.254b}}संस्कार‚दुःख‚तां म‚त्वा क‚थिता दुःख‚भाव‚ना ॥ २५४ ॥\&[\smallbreak]


	
	    \end{quote}
	  
	  \endgroup
	

	  \pstart \leavevmode% starting standard par
	\hphantom{.}‚{\color{DodgerBlue3}‚संस्कार‚दुःख‚तां म‚त्त्वा क‚थिता दुःख‚भाव‚ना} । न हि दुःख‚दुःख‚ताम‚भिस‚न्धाय ‚{\tiny $_{lb}$}‚त‚द्भाव‚नोक्ता किन्त‚र्हि संस्कार‚दुःख‚तां । (२५४)
	\pend% ending standard par
      \label{div_pvv.1.155}
	  
	% new div opening: depth here is 2
	

	  \pstart \leavevmode% starting standard par
	a. सैव किमुच्य‚त इत्याह (।)
	\pend% ending standard par
      
	  \bigskip
	  \begingroup
	
	    \large
	  
	    \begin{quote}
	  
	    
	    \stanza[\smallbreak]
	\label{pv.1.255a}\flagstanza{\tiny\textenglish{...1.255a}}सा च नः प्र‚त्य‚योत्प‚त्तिः सा नैरात्म्य‚दृगाश्र‚यः ।\&[\smallbreak]


	
	    \end{quote}
	  
	  \endgroup
	\textsuperscript{\textenglish{097/s}}

	  \pstart \leavevmode% starting standard par
	\hphantom{.}‚{\color{DodgerBlue3}‚सा च संस्कार‚दुःख‚ता नो}‚ऽस्माकं सौ ग ता नां ‚{\color{DodgerBlue3}‚प्र‚त्य‚योत्प‚त्ति}‚र्हेतुपार‚त‚न्त्र्यं । ‚{\color{DodgerBlue3}‚सा} प्र‚त्य‚योत्प‚त्ति‚{\color{DodgerBlue3}‚र्नैरात्म्य‚स्य\edtext{}{\edlabel{pvv.97-1}\label{pvv.97-1}\lemma{स्य}\Bfootnote{सापि न साक्षाद्भाविनामुक्त‚मिति ।}} दृशो} द‚{\tiny $_{8}$}‚र्श‚न‚स्या‚{\color{DodgerBlue3}‚श्र‚यः} कार‚णं ।
	\pend% ending standard par
      \textsuperscript{\textenglish{18b/MA}}‚{\tiny $_{lb}$}‚

	  \pstart \leavevmode% starting standard par
	\hphantom{.}b. त‚था हि हेतुफ‚ल‚भूताः ‚{\color{DodgerBlue3}‚क्ष‚ण‚क्ष‚यिणो भावाः प्र‚वृत्त‚यो नात्म‚रूपा नाप्यात्माधि}‚{\tiny $_{lb}$}‚ष्ठिता इति संस्कार‚दुःख‚ताभाव‚ना नैरात्म्य‚द‚र्श‚नानुकूला सैव च मुक्तिहेतुरित्याह(।)
	\pend% ending standard par
      
	  \bigskip
	  \begingroup
	
	    \large
	  
	    \begin{quote}
	  
	    
	    \stanza[\smallbreak]
	\label{pv.1.255b}\flagstanza{\tiny\textenglish{...1.255b}}मुक्तिस्तु शून्य‚तादृष्टेस्त‚द‚र्थाः शेष‚भाव‚नाः ॥ २५५ ॥\&[\smallbreak]


	
	    \end{quote}
	  
	  \endgroup
	

	  \pstart \leavevmode% starting standard par
	\hphantom{.}‚{\color{DodgerBlue3}‚मुक्तिस्तु शून्य‚ताया} निरात्म‚ताया ‚{\color{DodgerBlue3}‚दृष्टेः । शेष‚स्य} नित्य‚दुःखादे‚{\color{DodgerBlue3}‚र्भाव‚नास्त‚द‚र्था} निरात्म‚द‚र्श‚नार्थाः । (२५५)
	\pend% ending standard par
      \label{div_pvv.1.256}
	  
	% new div opening: depth here is 2
	

	  \begin{center}%% label @type='head'
	\textbf{II. अनित्य‚दुःखानात्म‚ता}
	\end{center}
	
	  \bigskip
	  \begingroup
	
	    \large
	  
	    \begin{quote}
	  
	    
	    \stanza[\smallbreak]
	\label{pv.1.256a}\flagstanza{\tiny\textenglish{...1.256a}}अनित्यात् प्राह तेनैव दुःखं; दुःखान्निरात्म‚ताम् ॥\&[\smallbreak]


	
	    \end{quote}
	  
	  \endgroup
	

	  \pstart \leavevmode% starting standard par
	\hphantom{.}‚{\color{DodgerBlue3}‚तेनैवा}‚नित्य‚दुःख‚भाव‚नायाः शून्य‚ताभाव‚नानुकूल‚त्वेन भ‚ग‚वान‚नित्याद‚नित्य‚त्वाद् ‚{\tiny $_{lb}$}‚‚{\color{DodgerBlue3}‚दुःखं} सं‚{\tiny $_{1}$}‚सारिस्क‚न्धानां हा (नाद्) ‚{\color{DodgerBlue3}‚दुःखाद् दुःख‚त्वान्निरात्म‚तामाह} (।) त‚द्य‚था रूपं भिक्ष‚वो नित्य‚म‚नित्यं वा । अनित्यं भ‚द‚न्त । य‚द‚नित्यं त‚द् दुःखं सुख‚म्वा । ‚{\tiny $_{lb}$}‚दुःख‚म्भ‚द‚न्त । य‚द‚नित्यं दुःखं विप‚रिणाम ध‚र्म‚कं क‚ल्प्य‚न्नु त‚देवं द्र‚ष्टुः एत‚न्म‚म एषोह‚{\tiny $_{lb}$}‚म‚स्मि एष मे आत्मेति । नो हीदं भ‚द‚न्त ।\edtext{}{\edlabel{pvv.97-2}\label{pvv.97-2}\lemma{था}\Bfootnote{म‚ज्झिम‚निकाये म‚हापुण्ण‚म‚सुत्त‚न्ते (१०९) ।}} इत्येवं हेतुफ‚ल‚भाव\edtext{}{\edlabel{pvv.97-3}\label{pvv.97-3}\lemma{भाव}\Bfootnote{हेतुफ‚ल‚भाव‚क‚थ‚नेनात्म‚द‚र्श‚न‚मेव ।}}नेनात्म‚द‚र्श‚न‚मेव मुक्ते‚{\tiny $_{lb}$}‚रुपाय इति क‚थितं । त‚देवाचार्येणोक्तं (।) उक्तो मार्गः त‚द‚भ्यासादाश्र‚यः प‚रिव‚र्त‚त ‚{\tiny $_{lb}$}‚इति नास्ति विरो‚{\tiny $_{2}$}‚धः ।
	\pend% ending standard par
      

	  \pstart \leavevmode% starting standard par
	यः पुन‚रात्म‚द‚र्शी सोऽविर‚क्त एव ।
	\pend% ending standard par
      
	  \bigskip
	  \begingroup
	
	    \large
	  
	    \begin{quote}
	  
	    
	    \stanza[\smallbreak]
	\label{pv.1.256b}\flagstanza{\tiny\textenglish{...1.256b}}अविर‚क्त‚श्च तृष्णावान् स‚र्वार‚म्भ‚स‚माश्रितः ॥ २५६ ॥\&[\smallbreak]


	
	    \end{quote}
	  
	  \endgroup
	

	  \pstart \leavevmode% starting standard par
	\hphantom{.}‚{\color{DodgerBlue3}‚अविर‚क्त‚श्च तृष्णावान्} । हान्युपादान‚ल‚क्ष‚णान् क‚र्म‚प्र‚स‚व‚हेतून् ‚{\color{DodgerBlue3}‚स‚मा‚{\tiny $_{lb}$}‚श्रितः} । (२५६)
	\pend% ending standard par
      \label{div_pvv.1.257}
	  
	% new div opening: depth here is 2
	
	  \bigskip
	  \begingroup
	
	    \large
	  
	    \begin{quote}
	  
	    
	    \stanza[\smallbreak]
	\label{pv.1.257}\flagstanza{\tiny\textenglish{....1.257}}सोऽमुक्तः क्लेश‚क‚र्म‚भ्यां संसारी नाम तादृशः ॥&आत्मीय‚मेव यो नेच्छेद् भोक्ताप्य‚स्य न विद्य‚ते ॥ २५७ ॥\&[\smallbreak]


	
	    \end{quote}
	  
	  \endgroup
	

	  \pstart \leavevmode% starting standard par
	\hphantom{.}‚{\color{DodgerBlue3}‚सोऽमुक्तः क्लेश‚क‚र्म‚भ्यां} आत्म‚द‚र्श‚न‚प्र‚वृत्तिकार‚ण‚काभ्यां ‚{\color{DodgerBlue3}‚संसारी नाम प्र‚सिद्धः ‚{\tiny $_{lb}$}‚तादृशः} । त‚देव‚मात्म‚नि स‚ति नात्मीय‚त्यागः । त‚थाऽमुक्तिरित्युक्तं । भ‚व‚तु वा‚{\tiny $_{lb}$}‚ऽत्मीय‚त्याग‚स्त‚थाप्या‚{\color{DodgerBlue3}‚त्मीय‚मेव यो नेच्छे}‚त्त‚न्म‚तेऽस्यात्मीय‚स्य ‚{\color{DodgerBlue3}‚भोक्ता न विद्य‚ते} । ‚{\tiny $_{lb}$}‚भोग्यापेक्ष‚त्वात् भोक्तृत्व‚स्य‚{\tiny $_{3}$}‚ (२५७) ।
	\pend% ending standard par
      \label{div_pvv.1.258_1.159_1.160_1.161_1.162_1.163}
	  
	% new div opening: depth here is 2
	\textsuperscript{\textenglish{098/s}}

	  \begin{center}%% label @type='head'
	\textbf{III. संसारी क्लेश‚क‚र्म‚भ्याम‚मुक्तः}
	\end{center}
	

	  \begin{center}%% label @type='head'
	\textbf{(ज) स‚त्काय‚दृष्टिर्मूल‚म्}
	\end{center}
	
	  \bigskip
	  \begingroup
	
	    \large
	  
	    \begin{quote}
	  
	    
	    \stanza[\smallbreak]
	\label{pv.1.258}\flagstanza{\tiny\textenglish{....1.258}}आत्माऽपि न त‚दा त‚स्य क्रियाभोगौ हि ल‚क्ष‚ण‚म् ।&त‚स्माद‚नादिस‚न्तान‚तुल्य‚जातीय‚बीजिकाम् ॥ २५८ ॥\&[\smallbreak]


	
	    \end{quote}
	  
	  \endgroup
	
	  \bigskip
	  \begingroup
	
	    \large
	  
	    \begin{quote}
	  
	    
	    \stanza[\smallbreak]
	\label{pv.1.259a}\flagstanza{\tiny\textenglish{...1.259a}}उत्खात‚मूलां कुरुत स‚त्व‚दृष्टिं मुमुक्ष‚वः ।\&[\smallbreak]


	
	    \end{quote}
	  
	  \endgroup
	

	  \pstart \leavevmode% starting standard par
	\hphantom{.}भोक्त्र‚भावे ‚{\color{DodgerBlue3}‚आत्मापि ना}‚स्तीति प्र‚स‚ङ्गात् । कुत इति चेत् । हिर्य‚स्माद‚स्या‚{\tiny $_{lb}$}‚त्म‚नः ‚{\color{DodgerBlue3}‚क्रियाभोगौ ल‚क्ष‚णं} क‚र्त्ता भोक्ता चात्मोच्य‚ते (।) य‚दा चात्मीय‚मेव ‚{\tiny $_{lb}$}‚नास्ति । किम‚र्थं क‚र्म क‚र्त‚व्यं किम्वा भोक्त‚व्यं । क‚र्तृत्व‚भोक्तृत्वाभावादात्मा‚{\tiny $_{lb}$}‚भाव एव स्वीकृतः स्यात् । त‚स्मात्स‚त्यात्म‚नि आत्मीयं त‚त्स्नेहादिस‚त्त्वेऽनुच्छेद ‚{\tiny $_{lb}$}‚एवं संसार‚स्य । ‚{\color{DodgerBlue3}‚त‚स्मात्} संसारादुद्विज‚माना मुमुक्ष‚व उ\edtext{}{\edlabel{pvv.98-1}\label{pvv.98-1}\lemma{उ}\Bfootnote{य‚थोक्त‚मिह याव‚च्च‚तुर्थः श्र‚म‚णः शून्याः प‚र‚प्र‚वादाः श्र‚म‚णैर्ब्राह्म‚णैर्व्वा ।}}त्खात‚मूलामुद्ध्ृत‚कार‚णां ‚{\tiny $_{lb}$}‚स‚त्त्व‚दृष्टिं कु‚{\tiny $_{4}$}‚रुत ।
	\pend% ending standard par
      

	  \pstart \leavevmode% starting standard par
	न‚नु किम‚स्या मूल‚मित्याह (।)
	\pend% ending standard par
      

	  \pstart \leavevmode% starting standard par
	\hphantom{.}‚{\color{DodgerBlue3}‚अनादिस‚न्तान‚स्तुल्य‚जातीयः} पूर्व्व‚पूर्व्व‚स‚त्त्व‚द‚र्श‚न‚स्व‚भावोऽविद्यारूपो ‚{\color{DodgerBlue3}‚बीजं} कार‚णं य‚स्यास्तां (२५८) ‚{\color{DodgerBlue3}‚स‚त्त्व‚द}‚र्श‚न‚म‚विद्यास्व‚भावं पूर्व्व‚पूर्व्व‚मुत्त‚रोत्त‚र‚स्य ‚{\color{DodgerBlue3}‚स‚त्त्व}‚{\tiny $_{lb}$}‚द‚र्श‚न‚स्य हेतुरित्य‚र्थः ।
	\pend% ending standard par
      

	  \begin{center}%% label @type='head'
	\textbf{(आग‚म‚मात्रेण न मुक्तिः)}
	\end{center}
	

	  \pstart \leavevmode% starting standard par
	न‚नूक्त‚मीश्व‚रेणाग‚मेऽस्त्यात्मा मोक्ष‚श्चास्य दीक्षाविधिनेति । त‚त्किम‚त्र ‚{\tiny $_{lb}$}‚चिन्त्य‚ते त‚त्कार‚णा वा धीविधिस्तं बाध‚ते क‚थ‚मित्यादिनोक्त‚त्वात् (।) अत्राह (।)
	\pend% ending standard par
      
	  \bigskip
	  \begingroup
	
	    \large
	  
	    \begin{quote}
	  
	    
	    \stanza[\smallbreak]
	\label{pv.1.259b}\flagstanza{\tiny\textenglish{...1.259b}}आग‚म‚स्य त‚थाभाव‚निब‚न्ध‚न‚म‚प‚श्य‚ताम् ॥ २५९ ॥\&[\smallbreak]


	
	    \end{quote}
	  
	  \endgroup
	
	  \bigskip
	  \begingroup
	
	    \large
	  
	    \begin{quote}
	  
	    
	    \stanza[\smallbreak]
	\label{pv.1.260a}\flagstanza{\tiny\textenglish{...1.260a}}मुक्तिमाग‚म‚मात्रेण व‚द‚न्न प‚रितोष‚कृद् ।\&[\smallbreak]


	
	    \end{quote}
	  
	  \endgroup
	

	  \pstart \leavevmode% starting standard par
	\hphantom{.}‚{\color{DodgerBlue3}‚आग‚म‚स्य त‚थाभाव‚स्य} प्र‚तिपादितार्थ‚स‚म्वादित्व‚स्य ‚{\color{DodgerBlue3}‚निब‚न्ध‚नं} हेतुम‚{\color{DodgerBlue3}‚प‚श्य‚तां}‚{\tiny $_{5}$}‚ । ‚{\tiny $_{lb}$}‚(२५९) मुमुक्षूणा‚{\color{DodgerBlue3}‚माग‚म‚मात्रेण मुक्तिं व‚द‚न्} न ‚{\color{DodgerBlue3}‚प‚रितोष}‚कृद् भ‚व‚ति ।
	\pend% ending standard par
      

	  \pstart \leavevmode% starting standard par
	\hphantom{.}‚{\color{DodgerBlue3}‚दीक्षाऽकिञ्चित्क‚री---}
	\pend% ending standard par
      

	  \pstart \leavevmode% starting standard par
	न‚न्व‚स्ति प्रामाण्य‚निब‚न्ध‚न‚माग‚म‚स्य दीक्षाविधिस्पृष्ट‚स्यानारोह‚ध‚र्म‚क‚त्व‚द‚र्श‚नं । ‚{\tiny $_{lb}$}‚य‚था हि बीजं दीक्षाविधिस्पृष्टं न प्र‚रोह‚ति त‚था पुमान‚पि दीक्षितो न ‚{\tiny $_{lb}$}‚पुन‚र्भ‚व‚तीत्याह (।)
	\pend% ending standard par
      \textsuperscript{\textenglish{099/s}}
	  \bigskip
	  \begingroup
	
	    \large
	  
	    \begin{quote}
	  
	    
	    \stanza[\smallbreak]
	\label{pv.1.260b}\flagstanza{\tiny\textenglish{...1.260b}}नालं बीजादिसंसिद्धो विधिः पुंसाम‚ज‚न्म‚ने ॥ २६० ॥\&[\smallbreak]


	
	    \end{quote}
	  
	  \endgroup
	
	  \bigskip
	  \begingroup
	
	    \large
	  
	    \begin{quote}
	  
	    
	    \stanza[\smallbreak]
	\label{pv.1.261a}\flagstanza{\tiny\textenglish{...1.261a}}तैलाभ्य‚ङ्गाग्निहादेर‚पि मुक्ति प्र‚स‚ङ्ग‚तः ॥\&[\smallbreak]


	
	    \end{quote}
	  
	  \endgroup
	

	  \pstart \leavevmode% starting standard par
	\hphantom{.}नालं श‚क्तो ‚{\color{DodgerBlue3}‚बीजादिषु संसिद्धो विधिर्दी}‚क्षायाः ‚{\color{DodgerBlue3}‚पुंसाम‚ज‚न्म‚ने} । (२६०) ‚{\tiny $_{lb}$}‚‚{\color{DodgerBlue3}‚तैलाभ्य‚ङ्गाग्निदाह‚देर‚पि} संसार‚{\color{DodgerBlue3}‚न्मुक्तिप्र‚स‚ङ्ग‚तः} । तैलेनाभ्य‚क्तं बीज‚म‚ग्निना ‚{\tiny $_{lb}$}‚च स्पृष्टं न प्र‚रोह‚ति\edtext{}{\edlabel{pvv.99-1}\label{pvv.99-1}\lemma{ति}\Bfootnote{क्षोद‚श्वेद‚विद‚ल‚न‚मादिना ।}} त‚था पुरुषो‚{\tiny $_{6}$}‚पि तैलाभ्य‚ङ्गाग्निदाहाभ्यां न पुन‚र्भ‚वेत् ।
	\pend% ending standard par
      

	  \begin{center}%% label @type='head'
	\textbf{I. आत्म‚नोऽमूर्त्त‚त्वे न पाप‚गौर‚व‚लाघ‚व‚म्}
	\end{center}
	

	  \pstart \leavevmode% starting standard par
	\hphantom{.}a. स्यादेत‚त् । दीक्षायाः ‚{\color{DodgerBlue3}‚प्राक्} पाप‚गुरोरुत्त‚रं तुल‚या लाघ‚वात् । पापाभावोप‚{\tiny $_{lb}$}‚ल‚म्भ आग‚म‚प्रामाण्य‚निब‚न्ध‚न‚मित्याह (।)
	\pend% ending standard par
      
	  \bigskip
	  \begingroup
	
	    \large
	  
	    \begin{quote}
	  
	    
	    \stanza[\smallbreak]
	\label{pv.1.261b}\flagstanza{\tiny\textenglish{...1.261b}}प्राग् गुरोर्लाघ‚वात् प‚श्चान्न पाप‚ह‚र‚णं कृत‚म् ॥ २६१ ॥\&[\smallbreak]


	
	    \end{quote}
	  
	  \endgroup
	
	  \bigskip
	  \begingroup
	
	    \large
	  
	    \begin{quote}
	  
	    
	    \stanza[\smallbreak]
	\label{pv.1.262a}\flagstanza{\tiny\textenglish{...1.262a}}मा भूद् गौर‚व‚मेवास्य न पापं गुर्व‚मूर्त्तितः ॥\&[\smallbreak]


	
	    \end{quote}
	  
	  \endgroup
	

	  \pstart \leavevmode% starting standard par
	\hphantom{.}प्राग् गुरोर्लाघ‚वात् प‚श्चान्न पाप‚ह‚र‚णं कृतं । दीक्षायाः ‚{\color{DodgerBlue3}‚प्राग् गुरोः प‚श्चाल्लाघ‚{\tiny $_{lb}$}‚वात्} दीक्ष‚या ‚{\color{DodgerBlue3}‚न पाप‚ह‚र‚ण}‚स्य दीक्षित‚स्य किन्तु ‚{\color{DodgerBlue3}‚गौर‚व‚मेवास्य कृतं} स‚त् (२६१) ‚{\tiny $_{lb}$}‚‚{\color{DodgerBlue3}‚मा भूदि}‚ति क‚स्मान्न क‚ल्प्य‚ते । लाघ‚वं हि गौर‚व‚विरोधि दृश्य‚मानं त‚द‚भाव‚मेव ग‚म‚{\tiny $_{lb}$}‚येत् । न पाप‚भावं । पाप‚मेव गुर्विति‚{\tiny $_{7}$}‚चेत् । ‚{\color{DodgerBlue3}‚न पापं गुरु । अमूर्त्तितो} मूर्त्त‚त्वा‚{\tiny $_{lb}$}‚भावात् । भूर्त्त‚ध‚र्मो हि गौर‚वं क‚थं पाप‚स्यामूर्त्त‚स्य स्यात् ।
	\pend% ending standard par
      

	  \pstart \leavevmode% starting standard par
	न‚नु त्व‚त्प‚क्षेऽपि नैरात्म्य‚द‚र्श‚ने भूतेपि क‚स्मान्न ज‚न्मेत्याह (।)
	\pend% ending standard par
      
	  \bigskip
	  \begingroup
	
	    \large
	  
	    \begin{quote}
	  
	    
	    \stanza[\smallbreak]
	\label{pv.1.262b}\flagstanza{\tiny\textenglish{...1.262b}}मिथ्याज्ञान‚त‚दुद्भूत‚त‚र्ष‚स‚ञ्चेत‚नाव‚शात् ॥ २६२ ॥\&[\smallbreak]


	
	    \end{quote}
	  
	  \endgroup
	
	  \bigskip
	  \begingroup
	
	    \large
	  
	    \begin{quote}
	  
	    
	    \stanza[\smallbreak]
	\label{pv.1.263a}\flagstanza{\tiny\textenglish{...1.263a}}हीन‚स्थान‚ग‚तिर्ज‚न्म त‚त‚स्त‚च्छिन्न जाय‚ते ।\&[\smallbreak]


	
	    \end{quote}
	  
	  \endgroup
	

	  \pstart \leavevmode% starting standard par
	\hphantom{.}‚{\color{DodgerBlue3}‚मिथ्याज्ञानं} दुःखे विप‚र्यास‚म‚तिः । ‚{\color{DodgerBlue3}‚त‚दुद्भूत‚स्त‚र्षो} मिथ्याज्ञान‚प्र‚भ‚वा तृष्णा । ‚{\tiny $_{lb}$}‚ताभ्यां संप्र‚युक्ते ‚{\color{DodgerBlue3}‚चेत‚ने} । त‚द्व‚शाद्या ‚{\color{DodgerBlue3}‚हीन‚स्थान‚ग‚ति}‚स्त‚{\color{DodgerBlue3}‚ज्ज‚न्मे}‚त्युक्तं । ‚{\color{DodgerBlue3}‚अत‚स्त‚च्छित्} । ‚{\tiny $_{lb}$}‚अज्ञान‚तृष्णाच्छेद‚को नैरात्म्य‚द‚र्शी ‚{\color{DodgerBlue3}‚न जात‚ये} कार‚णाभावात् ।
	\pend% ending standard par
      

	  \pstart \leavevmode% starting standard par
	त‚देवाह (।)
	\pend% ending standard par
      
	  \bigskip
	  \begingroup
	
	    \large
	  
	    \begin{quote}
	  
	    
	    \stanza[\smallbreak]
	\label{pv.1.263b}\flagstanza{\tiny\textenglish{...1.263b}}त‚योरेव हि साम‚र्थ्यं जातौ त‚न्मात्र‚भाव‚तः ॥ २६३ ॥\&[\smallbreak]


	
	    \end{quote}
	  
	  \endgroup
	

	  \pstart \leavevmode% starting standard par
	\hphantom{.}‚{\color{DodgerBlue3}‚त‚यो‚{\tiny $_{8}$}‚रेवा}‚ज्ञान‚तृष्ण‚यो‚{\color{DodgerBlue3}‚र्हि साम‚र्थ्यं जातौ} ज‚न्म‚निमित्तं ‚{\color{DodgerBlue3}‚त‚न्मात्रेण\edtext{}{\edlabel{pvv.99-2}\label{pvv.99-2}\lemma{न्मात्रेण}\Bfootnote{इह च तृष्ण‚या देशान्त‚रं कुर्य्यात् । तृष्णाज्ञान‚भावाभावानुकारात् ।}} भाव‚तः} ॥
	\pend% ending standard par
      \textsuperscript{\textenglish{19a/MA}}‚{\tiny $_{lb}$}‚

	  \pstart \leavevmode% starting standard par
	ते च दीक्षित‚स्यापि स्त इति स जाय‚ते । (२६३)
	\pend% ending standard par
      \textsuperscript{\textenglish{100/s}}\label{div_pvv.1.264_1.265_1.266_1.267_1.268_1.269_1.270}
	  
	% new div opening: depth here is 2
	

	  \pstart \leavevmode% starting standard par
	c. न‚नु क‚र्मापि ज‚न्म‚कार‚ण‚मिष्टं त‚त्क‚थ‚म‚ज्ञान‚तृष्णे एवोक्ते इत्याह (।)
	\pend% ending standard par
      
	  \bigskip
	  \begingroup
	
	    \large
	  
	    \begin{quote}
	  
	    
	    \stanza[\smallbreak]
	\label{pv.1.264}\flagstanza{\tiny\textenglish{....1.264}}ते चेत‚ने स्व‚यं क‚र्मेत्य‚ख‚ण्डं ज‚न्म‚कार‚ण‚म् ॥&ग‚तिप्र‚तीत्योःक‚र‚णान्याश्र‚य‚स्तान्य‚दृष्ट‚तः ॥ २६४ ॥\&[\smallbreak]


	
	    \end{quote}
	  
	  \endgroup
	
	  \bigskip
	  \begingroup
	
	    \large
	  
	    \begin{quote}
	  
	    
	    \stanza[\smallbreak]
	\label{pv.1.265a}\flagstanza{\tiny\textenglish{...1.265a}}अदृष्ट‚नाशाद‚ग‚तिः त‚त्संस्कारो न चेत‚ना ॥\&[\smallbreak]


	
	    \end{quote}
	  
	  \endgroup
	

	  \pstart \leavevmode% starting standard par
	\hphantom{.}‚{\color{DodgerBlue3}‚ते चेत‚ने} मिथ्याज्ञान‚त‚दुद्भूत‚र्ष‚संचेत‚ने ‚{\color{DodgerBlue3}‚स्व‚य}‚मात्म‚ना पूर्व्व‚शुभाशुभ‚क‚र्म‚सँस्कार‚{\tiny $_{lb}$}‚स‚हाये ‚{\color{DodgerBlue3}‚क‚र्म} क‚र्म‚स्व‚भावे ‚{\color{DodgerBlue3}‚इत्य‚ख‚ण्ड‚म}‚न्यूनं ‚{\color{DodgerBlue3}‚ज‚न्म‚कार‚ण}‚मुक्त‚मिति न विरोधः । स्यादे‚{\tiny $_{lb}$}‚त‚द्(।) ‚{\color{DodgerBlue3}‚ग‚तिप्र‚तीत्यो}‚र‚भिम‚त‚देश‚ग‚म‚न‚स्य ज्ञान‚स्य च ‚{\color{DodgerBlue3}‚क‚र‚णानी}‚न्द्रियाणि । ‚{\color{DodgerBlue3}‚आश्र‚यः} कार‚णं‚{\tiny $_{1}$}‚ इन्द्रियेभ्य उत्प‚न्नेन ज्ञानेन विष‚यं प‚रिच्छिद्य प्र‚वृत्तेः । तानीन्द्रियाणि ‚{\tiny $_{lb}$}‚‚{\color{DodgerBlue3}‚चादृष्ट‚तः} शुभादिल‚क्ष‚णाद् (२६४) । दीक्ष‚या चा‚{\color{DodgerBlue3}‚दृष्ट‚नाशात्त}‚त्कार्याणां कार‚णा‚{\tiny $_{lb}$}‚नाम‚नुत्प‚त्तेर्व्विष‚य‚स्य प‚रिच्छेद‚तृष्ण‚योर‚भावात् ‚{\color{DodgerBlue3}‚अग‚ति}‚र्ज‚न्म‚स्थान इति नास्ति ‚{\tiny $_{lb}$}‚पुन‚र्भ‚वो दीक्षित‚स्य (।) ‚{\color{DodgerBlue3}‚त‚द‚दृ}\edtext{\textsuperscript{*}}{\edlabel{pvv.100-1}\label{pvv.100-1}\lemma{*}\Bfootnote{विंश‚ति कुश‚लाकुश‚लाः ।}}ष्ट‚ञ्चात्म‚{\color{DodgerBlue3}‚संस्कारो न चे\edtext{}{\edlabel{pvv.100-2}\label{pvv.100-2}\lemma{चे}\Bfootnote{बौद्धोक्तः ।}}त‚ना} स्यात् दीक्षित‚स्यापि ‚{\tiny $_{lb}$}‚सास्तीति ज‚न्मापि\edtext{}{\edlabel{pvv.100-3}\label{pvv.100-3}\lemma{न्मापि}\Bfootnote{इति प‚राभिप्रायः ।}} स्यात ।
	\pend% ending standard par
      

	  \pstart \leavevmode% starting standard par
	b. अत्रा\edtext{}{\edlabel{pvv.100-4}\label{pvv.100-4}\lemma{अत्रा}\Bfootnote{स‚माधानं ।}}ह (।)
	\pend% ending standard par
      
	  \bigskip
	  \begingroup
	
	    \large
	  
	    \begin{quote}
	  
	    
	    \stanza[\smallbreak]
	\label{pv.1.265b}\flagstanza{\tiny\textenglish{...1.265b}}साम‚र्थ्यं क‚र‚णोत्प‚त्तेर्भावाभावानुवृत्तितः ॥ २६५ ॥\&[\smallbreak]


	
	    \end{quote}
	  
	  \endgroup
	
	  \bigskip
	  \begingroup
	
	    \large
	  
	    \begin{quote}
	  
	    
	    \stanza[\smallbreak]
	\label{pv.1.266a}\flagstanza{\tiny\textenglish{...1.266a}}दृष्टं बुद्धेर्न चान्य‚स्या स‚न्ति तानि न‚य‚न्ति किम् ।\&[\smallbreak]


	
	    \end{quote}
	  
	  \endgroup
	

	  \pstart \leavevmode% starting standard par
	बुद्धे\edtext{}{\edlabel{pvv.100-5}\label{pvv.100-5}\lemma{बुद्धे}\Bfootnote{मिथ्याज्ञान‚त‚र्ष‚युतायाः ।}}‚{\color{DodgerBlue3}‚र्भावाभावानुवृत्तितो}‚ऽन्व‚य‚व्य‚तिरेकानुवृत्त्या ‚{\color{DodgerBlue3}‚क‚र‚णा}‚नामेक‚{\tiny $_{2}$}‚स्माद्देशाद‚{\tiny $_{lb}$}‚प\edtext{}{\edlabel{pvv.100-6}\label{pvv.100-6}\lemma{प}\Bfootnote{य‚त्रोत्प‚त्स्य‚ते ।}}र‚देश‚स‚म्ब‚द्धाना‚{\color{DodgerBlue3}‚मुत्प‚त्तेः} कार‚णा‚{\color{DodgerBlue3}‚त्सामार्थ्यं बुद्धे}‚रिन्द्रिय‚ज‚न‚नं प्र‚ति(२६५)‚{\color{DodgerBlue3}‚दृष्टं} नान्य‚स्य संस्कार‚रूप‚स्यादृष्ट‚स्य त‚द‚न्व‚य‚व्य‚तिरेकानुविधानानुप‚ल‚म्भात् । सा बुद्धि‚{\tiny $_{lb}$}‚श्चास्ति देशान्त‚र‚स‚म्ब‚द्ध‚क‚र‚ण‚ज‚निका दीक्षित‚स्या ‚{\color{DodgerBlue3}‚त‚त्कि}‚न्तानि क‚र‚णानि ग‚र्भ‚स्थान‚न्न ‚{\tiny $_{lb}$}‚‚{\color{DodgerBlue3}‚य‚न्ति} ग‚च्छ‚न्ति हेत्व‚वैक‚ल्याद् भ‚वित‚व्यं (२६६) ।
	\pend% ending standard par
      

	  \pstart \leavevmode% starting standard par
	c. ग‚म‚नेनाव‚श्यं दीक्ष‚योप‚ह‚ता बुद्धिर्देशान्त‚रं नेतुमिन्द्रियाण्य‚श‚क्ता चेत् ।
	\pend% ending standard par
      
	  \bigskip
	  \begingroup
	
	    \large
	  
	    \begin{quote}
	  
	    
	    \stanza[\smallbreak]
	\label{pv.1.266b}\flagstanza{\tiny\textenglish{...1.266b}}धार‚ण‚प्रेर‚ण‚क्षोभ‚निरोधाश्चेत‚नाव‚शाः ॥ २६६ ॥\&[\smallbreak]


	
	    \end{quote}
	  
	  \endgroup
	
	  \bigskip
	  \begingroup
	
	    \large
	  
	    \begin{quote}
	  
	    
	    \stanza[\smallbreak]
	\label{pv.1.267}\flagstanza{\tiny\textenglish{....1.267}}न स्युस्तेषाम‚साम‚र्थ्ये त‚स्य दीक्षाद्य‚न‚न्त‚र‚म् ।&अथ बुद्धेस्त‚दाभावान्न स्युः स‚न्धीय‚ते म‚लैः ॥ २६७ ॥\&[\smallbreak]


	
	    \end{quote}
	  
	  \endgroup
	

	  \pstart \leavevmode% starting standard par
	न‚न्वेव‚न्त‚स्य मुमुक्षोर्दी‚{\tiny $_{3}$}‚‚{\color{DodgerBlue3}‚क्षान‚न्त‚रं} बुद्धेर‚{\color{DodgerBlue3}‚साम‚र्थ्ये तेषा}‚मिन्द्रियाणां स्व‚विष‚ये ‚{\tiny $_{lb}$}‚व्य‚व‚स्थाप‚नं धार‚णं । त‚त्रा\edtext{}{\edlabel{pvv.100-7}\label{pvv.100-7}\lemma{त्रा}\Bfootnote{आकृष्य न‚य‚नं ।}}योज‚नं ‚{\color{DodgerBlue3}‚प्रेर‚णं । क्षोभो}\edtext{}{\edlabel{pvv.100-8}\label{pvv.100-8}\lemma{नं}\Bfootnote{भ्रूभ‚ङ्गादि ।}} विकारः । स्व‚विष‚यान्निव‚र्त‚नं ‚{\tiny $_{lb}$}‚\leavevmode\ledsidenote{\textenglish{101/s}} ‚{\color{DodgerBlue3}‚निरोधः} । ते बुद्धिनिब‚न्ध‚न‚वृत्त‚यो ‚{\color{DodgerBlue3}‚न स्युः} ॥ ‚{\color{DodgerBlue3}‚अथ बुद्धेस्त‚दा} म‚र‚ण‚काले‚{\color{DodgerBlue3}‚ऽभावान्न स्युः} धार‚ण‚प्रेर‚णाद‚यः ।
	\pend% ending standard par
      

	  \pstart \leavevmode% starting standard par
	\hphantom{.}अत्राह । स‚न्धीय‚ते ज‚न्य‚ते ‚{\color{DodgerBlue3}‚म‚लै}‚र्मिथ्याज्ञानैर्मिथ्याज्ञानात्म‚स्नेहादिभिर्बुद्धिर्म‚र‚ण‚{\tiny $_{lb}$}‚स‚म‚येपीति कुतो बुद्ध्य‚भावः । (२६७)
	\pend% ending standard par
      
	  \bigskip
	  \begingroup
	
	    \large
	  
	    \begin{quote}
	  
	    
	    \stanza[\smallbreak]
	\label{pv.1.268}\flagstanza{\tiny\textenglish{....1.268}}बुद्धेस्तेषाम‚साम‚र्थ्ये जीव‚तोऽपि स्युर‚क्ष‚माः ॥&निर्ह्रासातिश‚यात् पुष्टौ प्र‚तिप‚क्ष‚स्व‚प‚क्ष‚योः ॥ २६८ ॥\&[\smallbreak]


	
	    \end{quote}
	  
	  \endgroup
	
	  \bigskip
	  \begingroup
	
	    \large
	  
	    \begin{quote}
	  
	    
	    \stanza[\smallbreak]
	\label{pv.1.269a}\flagstanza{\tiny\textenglish{...1.269a}}दोषाः स्व‚बीज‚स‚न्ताना दीक्षितेऽप्य‚निवारिताः ॥\&[\smallbreak]


	
	    \end{quote}
	  
	  \endgroup
	

	  \pstart \leavevmode% starting standard par
	\hphantom{.}‚{\color{DodgerBlue3}‚अथ} म‚ला अपि दीक्ष‚योप‚ह‚त‚साम‚र्थ्या न बुद्धि स‚न्द‚{\tiny $_{4}$}‚ध‚ति । त‚दा ‚{\color{DodgerBlue3}‚तेषां} म‚ला‚{\tiny $_{lb}$}‚नाम‚{\color{DodgerBlue3}‚साम‚र्थ्ये} स्वीक्रिय‚माणे ‚{\color{DodgerBlue3}‚जीव‚तोपि} दीक्षित‚स्य म‚ला बुद्धिस‚न्धानं ‚{\color{DodgerBlue3}‚प्र‚त्य‚क्ष‚माः स्युः} । ‚{\tiny $_{lb}$}‚दीक्षा म‚लानां बाधिकेत्य‚पि मिथ्या । त‚था प्र‚तिप‚क्ष‚स्य नैरात्म्य‚भाव‚नायाः स्व‚प‚क्ष‚स्या‚{\tiny $_{lb}$}‚योनिशोम‚न‚स्कार‚स्याभ्यासात् ‚{\color{DodgerBlue3}‚पुष्टौ} स‚त्यां दोषाणां य‚थाक्र‚मं ‚{\color{DodgerBlue3}‚न‚र्ह्रा}‚साद‚प‚च‚यात् । ‚{\tiny $_{lb}$}‚‚{\color{DodgerBlue3}‚अतिश‚या}‚द‚भिवृद्धे (२६८) ‚{\color{DodgerBlue3}‚र्दोषाः} स्व‚कीयाद् ‚{\color{DodgerBlue3}‚बीजा}‚त्तुल्य‚जातीय‚कार‚णात् ‚{\color{DodgerBlue3}‚स‚न्तानो} येषां ते ‚{\color{DodgerBlue3}‚दीक्षितेपि} दोष‚बीजे‚{\tiny $_{5}$}‚ प‚रिपोष‚व‚त्य‚{\color{DodgerBlue3}‚निवारिताः} ।
	\pend% ending standard par
      

	  \begin{center}%% label @type='head'
	\textbf{II. आत्म‚नो नित्य‚त्वे न पुन‚र्ज‚न्म}
	\end{center}
	

	  \pstart \leavevmode% starting standard par
	स्यादेत‚द् (।) आत्म‚नोपि ग‚र्भ‚ग‚त‚क‚र‚णादिज‚न‚ने व्यापारः स एव दीक्ष‚या निरुद्ध ‚{\tiny $_{lb}$}‚इति न पुन‚र्ज‚न्मेत्याह (।)
	\pend% ending standard par
      
	  \bigskip
	  \begingroup
	
	    \large
	  
	    \begin{quote}
	  
	    
	    \stanza[\smallbreak]
	\label{pv.1.269b}\flagstanza{\tiny\textenglish{...1.269b}}नित्य‚स्य निर‚पेक्ष‚त्वात् क्र‚मोत्प‚त्तिर्विरुध्य‚ते ॥ २६९ ॥\&[\smallbreak]


	
	    \end{quote}
	  
	  \endgroup
	
	  \bigskip
	  \begingroup
	
	    \large
	  
	    \begin{quote}
	  
	    
	    \stanza[\smallbreak]
	\label{pv.1.270}\flagstanza{\tiny\textenglish{....1.270}}क्रियायाम‚क्रियायाञ्च क्रिय‚योः स‚दृशात्म‚नः ॥&ऐक्य‚ञ्च हेतुफ‚ल‚योर्व्य‚तिरेक‚स्त‚त‚स्त‚योः ॥ २७० ॥\&[\smallbreak]


	
	    \end{quote}
	  
	  \endgroup
	

	  \pstart \leavevmode% starting standard par
	\hphantom{.}नित्य‚स्यानुप‚कार्य‚त‚या ‚{\color{DodgerBlue3}‚निर‚पेक्ष‚त्वात्} क‚र‚णादीनां ‚{\color{DodgerBlue3}‚क्र‚मे\edtext{}{\edlabel{pvv.101-1}\label{pvv.101-1}\lemma{मे}\Bfootnote{पूर्व्व कृत‚वान् प‚श्चाद‚पि ग‚र्भे व्याप्रिय‚त इति क्र‚मः ।}}णोत्प‚त्तिर्व्वि}‚रुध्य‚ते ‚{\tiny $_{lb}$}‚स‚म‚त्व‚हेतुस‚द्भावात् स‚कृदुत्पाद‚प्र‚स‚क्तेः । (२६९) इन्द्रियादेः क्रियायाम‚क्रियायाञ्च ‚{\tiny $_{lb}$}‚स‚दृशात्म‚न‚स्तुल्य‚रूप‚स्यात्म‚न‚स्त‚योः काल‚योस्ते ते ‚{\color{DodgerBlue3}‚विरुध्येते} ।
	\pend% ending standard par
      

	  \pstart \leavevmode% starting standard par
	य‚द्य‚सौ कार्य‚क‚र‚ण‚स्व‚भावः‚{\tiny $_{6}$}‚ त‚दा कार्यादेव ‚{\color{DodgerBlue3}‚क्रियावि}‚रामोस्य विरुध्य‚ते । ‚{\tiny $_{lb}$}‚एव‚{\color{DodgerBlue3}‚म‚क्रियायाम‚पि} वाच्यं । किञ्च (।) आत्म‚नः क‚र्म‚क‚र्तृता हेतुत्वं भोक्तृत्व‚ञ्च ‚{\color{DodgerBlue3}‚फ‚लं} ते ‚{\tiny $_{lb}$}‚चात्म‚न एक‚रूप‚स्य रूपे इति हेतुफ‚लाभेद‚प्र‚स‚ङ्गः । ‚{\color{DodgerBlue3}‚व्य‚तिरेको भेद‚स्त‚त आत्म‚न‚स्त‚योः} क‚र्तृ त्व‚भोक्तृत्व‚योर्ध‚र्म‚योस्त‚तो नैक्य‚प्र‚स‚ङ्ग इति चेत् । (२७० ।)
	\pend% ending standard par
      \textsuperscript{\textenglish{102/s}}\label{div_pvv.1.271_1.272_1.273_1.274_1.275_1.276}
	  
	% new div opening: depth here is 2
	

	  \begin{center}%% label @type='head'
	\textbf{III. नैरात्म्ये स्मृतिसंग‚तिः}
	\end{center}
	
	  \bigskip
	  \begingroup
	
	    \large
	  
	    \begin{quote}
	  
	    
	    \stanza[\smallbreak]
	\label{pv.1.271a}\flagstanza{\tiny\textenglish{...1.271a}}क‚र्तृभोक्तृत्व‚हानिः स्यात् साम‚र्थ्य‚ञ्च न सिध्य‚ति ।\&[\smallbreak]


	
	    \end{quote}
	  
	  \endgroup
	

	  \pstart \leavevmode% starting standard par
	\hphantom{.}एवं ‚{\color{DodgerBlue3}‚क‚र्तृ त्व‚भोक्तृत्वाहानिः स्यात्} । आत्म‚नोऽत‚त्स्व‚भाव‚त्वात् । त‚त्स‚म्ब‚न्धात् ‚{\tiny $_{lb}$}‚क‚र्ता भोक्ता च । अनुप‚कृत‚स्य स‚म्ब‚न्धित्वेऽति‚{\tiny $_{7}$}‚प्र‚स‚ङ्गात् उप‚कृत‚त्वं वेदित‚व्यं ‚{\tiny $_{lb}$}‚त‚च्चाश‚क्य‚साध‚नं य‚स्मान्नित्य‚स्याव्य‚तिरेकित्वात् ‚{\color{DodgerBlue3}‚साम‚र्थ्य‚ञ्च न सिध्य‚ति} ।
	\pend% ending standard par
      

	  \pstart \leavevmode% starting standard par
	न‚नु य‚द्यात्मा नास्ति । त‚दाऽन्येनानुभूतं क‚र्म च कृत‚म‚न्यः स्म‚र‚ति भुड‚त्क्ते ‚{\tiny $_{lb}$}‚फ‚ल‚मिति स्यात्त‚था चातिप्र‚स‚ङ्ग इत्याह (।)
	\pend% ending standard par
      
	  \bigskip
	  \begingroup
	
	    \large
	  
	    \begin{quote}
	  
	    
	    \stanza[\smallbreak]
	\label{pv.1.271b}\flagstanza{\tiny\textenglish{...1.271b}}अन्य‚स्म‚र‚ण‚भोगादिप्र‚स‚ङ्ग‚श्च न बाध‚काः ॥ २७१ ॥\&[\smallbreak]


	
	    \end{quote}
	  
	  \endgroup
	
	  \bigskip
	  \begingroup
	
	    \large
	  
	    \begin{quote}
	  
	    
	    \stanza[\smallbreak]
	\label{pv.1.272a}\flagstanza{\tiny\textenglish{...1.272a}}अस्मृतेः; क‚स्य चित् । तेन ह्य‚नुभूतेः स्मृतोद्भ‚वः ।\&[\smallbreak]


	
	    \end{quote}
	  
	  \endgroup
	

	  \pstart \leavevmode% starting standard par
	\hphantom{.}‚{\color{DodgerBlue3}‚अन्य‚स्य स्म‚र‚ण‚भोगादिप्र‚स‚ङ्गाश्च न बाध‚का} भ‚व‚न्ति (२७१) अस्मृतेः । ‚{\tiny $_{lb}$}‚‚{\color{DodgerBlue3}‚क‚स्य‚चित्} स्म‚र्त्तुर‚भावात् । एवं भोगोपि नास्ति भोक्त्र‚भावात् । ‚{\color{DodgerBlue3}‚तेन} हि त‚स्मात् ‚{\tiny $_{lb}$}‚\leavevmode\ledsidenote{\textenglish{19b/MA}} स्म‚र्त्र‚भावात्‚{\tiny $_{8}$}‚ कार‚णाद्विष‚याणाम‚{\color{DodgerBlue3}‚नुभूतेः} स‚काशात् ‚{\color{DodgerBlue3}‚स्मृति}‚रेव स्मृतं त‚स्यो‚{\color{DodgerBlue3}‚द्भ‚वः} । ‚{\tiny $_{lb}$}‚व‚स्तुध‚र्मो ह्येष य‚द‚नुभ‚वः प‚टीयान् स्म‚र‚ण‚बीजाधान‚द्वारेण स्म‚र‚णं ज‚न‚य‚ति । ‚{\tiny $_{lb}$}‚शुभाशुभ‚चेत‚नाश्च संस्कारा भोगाकारं स‚म्विदं प्र‚व‚र्त‚य‚न्ति (।) त‚त्किं स्म‚र्तृ‚{\tiny $_{lb}$}‚भोक्तृदुर्ग्र‚हेण । त‚त्त‚दाकारः प्र‚तीत्य‚स‚मुत्प‚न्नो बुद्धिप्र‚ब‚न्ध एव केव‚लो न तु संसारी ‚{\tiny $_{lb}$}‚नाम क‚श्चित् ॥
	\pend% ending standard par
      

	  \pstart \leavevmode% starting standard par
	h. य‚दि नास्त्यात्मा क‚थं स कोप‚दृष्टिः संसा‚{\tiny $_{1}$}‚र‚प्र‚वृत्तिर्व्वेत्याह (।)
	\pend% ending standard par
      
	  \bigskip
	  \begingroup
	
	    \large
	  
	    \begin{quote}
	  
	    
	    \stanza[\smallbreak]
	\label{pv.1.272b}\flagstanza{\tiny\textenglish{...1.272b}}स्थिरं सुखं म‚माह‚ञ्चेत्यादि स‚त्य‚च‚तुष्ट‚ये ॥ २७२ ॥\&[\smallbreak]


	
	    \end{quote}
	  
	  \endgroup
	
	  \bigskip
	  \begingroup
	
	    \large
	  
	    \begin{quote}
	  
	    
	    \stanza[\smallbreak]
	\label{pv.1.273a}\flagstanza{\tiny\textenglish{...1.273a}}अभूतान् षोड‚शाकारान् आरोप्य प‚रितृष्य‚ति ॥\&[\smallbreak]


	
	    \end{quote}
	  
	  \endgroup
	

	  \pstart \leavevmode% starting standard par
	\hphantom{.}‚{\color{DodgerBlue3}‚स्थिर\edtext{}{\edlabel{pvv.102-1}\label{pvv.102-1}\lemma{स्थिर}\Bfootnote{नित्य‚मिति वाच्ये क्ष‚णात् प‚रं स्थायी स‚र्व्वो नित्य इत्य‚र्थः ।}}म}‚क्ष‚णिकं सुखं त्रिदुःख‚ताविप‚रीतं म‚मेत्यात्मीय‚म‚ह‚मित्यात्माहं‚{\tiny $_{lb}$}‚कार‚श्चेति दुःख‚स‚त्य‚स्य विप‚रीता आकारा इत्याद्यान् प्र‚ति स‚त्यं\edtext{}{\edlabel{pvv.102-2}\label{pvv.102-2}\lemma{त्यं}\Bfootnote{अस‚मुद‚या हेत्व‚प्र‚भ‚वाद्याः ।}} च‚तुरः कृत्वा ‚{\tiny $_{lb}$}‚‚{\color{DodgerBlue3}‚षोड‚शाकारान्-भूतान्} प्राक् य‚थोक्त‚भूताकार‚विप‚रीतान् ‚{\color{DodgerBlue3}‚स‚त्य‚च‚तुष्ट‚ये} (२७२)‚{\tiny $_{lb}$}‚ऽरोप‚य‚तीति भ्रान्तिरेव स‚त्काय‚दृष्टिः स्व‚बीज‚प्र‚भ‚वा । ‚{\color{DodgerBlue3}‚आरो}\edtext{\textsuperscript{*}}{\edlabel{pvv.102-3}\label{pvv.102-3}\lemma{*}\Bfootnote{स स‚त्काय‚दृष्टिमान् तामारोप्य च ।}}प्य च किञ्चित् ‚{\tiny $_{lb}$}‚स्व‚सूख‚साध‚न‚ञ्च म‚न्य‚मान‚स्त‚द‚र्थं ‚{\color{DodgerBlue3}‚प‚रितृष्य‚ति} । तृष्ण‚या च ज‚न्म‚{\tiny $_{2}$}‚स्थानोपांदान‚मिति ‚{\tiny $_{lb}$}‚संसार‚प्र‚वृत्तिः ।
	\pend% ending standard par
      \textsuperscript{\textenglish{103/s}}

	  \begin{center}%% label @type='head'
	\textbf{घ. स‚म्य‚ग्दृष्टिनैरात्म्य‚दृष्टिः}
	\end{center}
	

	  \pstart \leavevmode% starting standard par
	मार्ग‚विप‚र्य‚यं संसार‚हेतुमुक्त्वा मार्ग‚माह (।)
	\pend% ending standard par
      
	  \bigskip
	  \begingroup
	
	    \large
	  
	    \begin{quote}
	  
	    
	    \stanza[\smallbreak]
	\label{pv.1.273b}\flagstanza{\tiny\textenglish{...1.273b}}त‚त्रैव त‚द्विरुद्धार्थ‚त‚त्त्वाकारानुरोधिनी ॥ २७३ ॥\&[\smallbreak]


	
	    \end{quote}
	  
	  \endgroup
	
	  \bigskip
	  \begingroup
	
	    \large
	  
	    \begin{quote}
	  
	    
	    \stanza[\smallbreak]
	\label{pv.1.274a}\flagstanza{\tiny\textenglish{...1.274a}}ह‚न्ति सानुच‚रां तृष्णां स‚म्य‚ग्दृष्टिः सुभाविता ॥\&[\smallbreak]


	
	    \end{quote}
	  
	  \endgroup
	

	  \pstart \leavevmode% starting standard par
	\hphantom{.}‚{\color{DodgerBlue3}‚त‚त्र} स‚त्य‚च‚तुष्ट‚य ‚{\color{DodgerBlue3}‚एव म्य‚ग्दृष्टिर्ने}‚रात्म्य‚दृष्टिः ‚{\color{DodgerBlue3}‚त‚द्विरुद्धार्थ‚त‚त्त्वाकारानु‚{\tiny $_{lb}$}‚रोधिनी} (२७३) तेषां स्थिर‚सुखाद्याकाराणाम‚विद्यारोपितानां विरुद्धोऽर्थ‚स्त‚स्य ‚{\tiny $_{lb}$}‚त‚त्त्वानि भूता आकारा अनित्या सुखाद‚यः षोड‚शाकारास्तान‚नुरोद्धुं ‚{\color{DodgerBlue3}‚शीलं य‚स्याः} सा त‚था सुभाविता साद‚र‚निर‚न्त‚र‚दीर्घ‚कालाभ्यास‚प्रा‚{\tiny $_{3}$}‚प्त‚वैश‚द्या ‚{\color{DodgerBlue3}‚ह‚न्ति तृष्णां ज‚न्म-} हेतुं ‚{\color{DodgerBlue3}‚सानुच‚रां} मात्स‚र्यादिप‚रिवारां ॥
	\pend% ending standard par
      

	  \begin{center}%% label @type='head'
	\textbf{(क) तृष्णाक्ष‚यात् मोक्षः}
	\end{center}
	

	  \pstart \leavevmode% starting standard par
	न‚नु तृष्णाक्ष‚येपि क‚र्म‚देह‚योर्ज‚न्म‚हेत्वोर्भावात् ज‚न्म किं न भ‚व‚तीत्याह (।)
	\pend% ending standard par
      
	  \bigskip
	  \begingroup
	
	    \large
	  
	    \begin{quote}
	  
	    
	    \stanza[\smallbreak]
	\label{pv.1.274b}\flagstanza{\tiny\textenglish{...1.274b}}त्रिहेतोर्नोद्भ‚वः क‚र्म‚देह‚योः स्थित‚योर‚पि ॥ २७४ ॥\&[\smallbreak]


	
	    \end{quote}
	  
	  \endgroup
	
	  \bigskip
	  \begingroup
	
	    \large
	  
	    \begin{quote}
	  
	    
	    \stanza[\smallbreak]
	\label{pv.1.275a}\flagstanza{\tiny\textenglish{...1.275a}}एकाभावाद् विना बीजं नाङ्कुर‚स्येव स‚म्भ‚वः ॥\&[\smallbreak]


	
	    \end{quote}
	  
	  \endgroup
	

	  \pstart \leavevmode% starting standard par
	\hphantom{.}त्रिहेतोस्तृष्णाक‚र्म‚देह‚हेतोर्ज‚न्म‚न‚स्तृष्णायाः क्ष‚ये क‚र्म‚{\color{DodgerBlue3}‚देह‚योः स्थित‚योर‚पि ‚{\tiny $_{lb}$}‚नोद्भ‚वः} उत्प‚त्तिर्न भ‚व‚ति (२७४) ‚{\color{DodgerBlue3}‚एकाभावात्} हेतुसाम‚ग्र्‏य‚वैक‚ल्यात् । ‚{\color{DodgerBlue3}‚विना} ‚{\color{DodgerBlue3}‚बीजं} क्षित्युद‚कादिभावेपि ‚{\color{DodgerBlue3}‚नाङ्कुर‚स्योद्भ‚वः} ।
	\pend% ending standard par
      

	  \pstart \leavevmode% starting standard par
	a. न‚नु त्रिहेतोर्ज‚न्म‚न एकाभावे‚{\tiny $_{4}$}‚पि चेद‚नुत्प‚त्तिः त‚त्किं क\edtext{}{\edlabel{pvv.103-1}\label{pvv.103-1}\lemma{क}\Bfootnote{तृष्णामुपेक्ष्याप्र‚हीणामेव ।}}र्म्म‚णो देह‚स्य वा ‚{\tiny $_{lb}$}‚क्ष‚यो नाभ्य‚स्य‚ते । प‚रे चाहुः क‚र्म‚क्ष‚यान्मुक्तिरिति । अत्राह (।)
	\pend% ending standard par
      
	  \bigskip
	  \begingroup
	
	    \large
	  
	    \begin{quote}
	  
	    
	    \stanza[\smallbreak]
	\label{pv.1.275b}\flagstanza{\tiny\textenglish{...1.275b}}अस‚म्भ‚वाद् विप‚क्ष‚स्य न हानिः क‚र्म‚देह‚योः ॥ २७५ ॥\&[\smallbreak]


	
	    \end{quote}
	  
	  \endgroup
	
	  \bigskip
	  \begingroup
	
	    \large
	  
	    \begin{quote}
	  
	    
	    \stanza[\smallbreak]
	\label{pv.1.276a}\flagstanza{\tiny\textenglish{...1.276a}}अश‚क्य‚त्वाच्च तृष्णायां स्थितायां पुन‚रुद्भ‚वात् ॥\&[\smallbreak]


	
	    \end{quote}
	  
	  \endgroup
	

	  \pstart \leavevmode% starting standard par
	\hphantom{.}‚{\color{DodgerBlue3}‚अस‚म्भ‚वाद्विप‚क्ष‚स्य न हानिः क‚र्म‚देह‚योर‚स्ति} । (२७५) ‚{\color{DodgerBlue3}‚अश‚क्य‚त्वाच्च}‚। ‚{\tiny $_{lb}$}‚स‚त्य‚पि वा विप‚क्षे\edtext{}{\edlabel{pvv.103-2}\label{pvv.103-2}\lemma{क्षे}\Bfootnote{बोद्ध‚म‚श‚क्ये ।}} त‚द‚भ्यासाद् देह‚क‚र्म‚निवृत्तिर‚श‚क्य‚क्रिया । ‚{\color{DodgerBlue3}‚तृष्णायां स्थितायां} त‚त्प्र‚चित‚स्यात्म‚ग्र‚ह‚व‚तो ग‚र्भ‚स्थान‚प‚रिग्र‚हे स‚ति ‚{\color{DodgerBlue3}‚पुन‚रुद्भ‚वात्} देह‚स्य श‚रीरि‚{\color{DodgerBlue3}‚ण‚श्च} तृष्ण‚यैव प्र‚वृत्त्या स‚र्व्व‚त्र शुभाशुभ‚प्र‚वृत्तेश्च (।)
	\pend% ending standard par
      

	  \pstart \leavevmode% starting standard par
	b. एव‚न्त‚र्हि तृष्णा क‚र्म च क्ष‚प‚यित‚व्यं मुमुक्षुणेत्याह (।)
	\pend% ending standard par
      
	  \bigskip
	  \begingroup
	
	    \large
	  
	    \begin{quote}
	  
	    
	    \stanza[\smallbreak]
	\label{pv.1.276b}\flagstanza{\tiny\textenglish{...1.276b}}द्व‚य‚क्ष‚यार्थं य‚त्ने च व्य‚र्थः क‚र्म‚क्ष‚र्थ श्र‚मः ॥ २७६ ॥\&[\smallbreak]


	
	    \end{quote}
	  
	  \endgroup
	\textsuperscript{\textenglish{104/s}}

	  \pstart \leavevmode% starting standard par
	\hphantom{.}‚{\color{DodgerBlue3}‚द्व‚य‚क्ष‚यार्थं य‚त्ने} क्रिय‚माणे ‚{\color{DodgerBlue3}‚क‚र्म‚क्ष‚ये व्य‚र्थः श्र‚मः} । तृष्णाक्ष‚य‚मात्राज्ज‚न्माभाव‚{\tiny $_{lb}$}‚सिद्धेः स‚द‚पि क‚र्मानुप‚युक्त‚मित्य‚लं त‚त्क्ष‚य‚प्र‚यासेन । न च क‚र्म‚क्ष‚यः प्र‚ति\edtext{}{\edlabel{pvv.104-1}\label{pvv.104-1}\lemma{ति}\Bfootnote{स‚त्यां तृष्णायां ।}}प‚क्षाभावात् ‚{\tiny $_{lb}$}‚श‚क्य इत्युक्तं ।
	\pend% ending standard par
      

	  \begin{center}%% label @type='head'
	\textbf{(ख) अक्षीण‚क‚र्म‚णो न मोक्षः}
	\end{center}
	

	  \pstart \leavevmode% starting standard par
	स‚न्ताप‚क्लेशोप‚भोगात् पूर्व्वार्जित‚क‚र्म‚क्ष‚योऽप‚र‚स्य चाकार‚णं त‚तो मुक्तिरित्य‚पि ‚{\tiny $_{lb}$}‚मो\edtext{}{\edlabel{pvv.104-2}\label{pvv.104-2}\lemma{मो}\Bfootnote{दिग‚म्ब‚र‚स्य ।}}हः । (२७६)
	\pend% ending standard par
      \label{div_pvv.1.277}
	  
	% new div opening: depth here is 2
	

	  \pstart \leavevmode% starting standard par
	त‚थाहि (।)
	\pend% ending standard par
      
	  \bigskip
	  \begingroup
	
	    \large
	  
	    \begin{quote}
	  
	    
	    \stanza[\smallbreak]
	\label{pv.1.277}\flagstanza{\tiny\textenglish{....1.277}}फ‚ल‚वैचित्र्य‚दृष्टेश्च श‚क्तिभेदोऽनुमीय‚ते ।&क‚र्म‚णां ताप‚संक्लेशात् नैक‚रूपात् त‚तः क्ष‚यः ॥ २७७ ॥\&[\smallbreak]


	
	    \end{quote}
	  
	  \endgroup
	

	  \pstart \leavevmode% starting standard par
	\hphantom{.}क‚र्म‚णां ‚{\color{DodgerBlue3}‚फ‚ल‚वैचित्र्य‚स्य\edtext{}{\edlabel{pvv.104-3}\label{pvv.104-3}\lemma{स्य}\Bfootnote{भोगाभोग‚रोगारोग्यादिः ।}}} नानाग‚त्युप‚भोग्याने‚{\tiny $_{6}$}‚ क‚विधोप‚क‚र‚ण‚साध्य‚विविध‚सुख‚{\tiny $_{lb}$}‚दुःखोप‚भोग‚प्र‚कार‚स्य ‚{\color{DodgerBlue3}‚दृष्टेश्च श‚क्तिभेदः} साम‚र्थ्य‚नानात्व‚म‚{\color{DodgerBlue3}‚नुमीय‚तेऽतो} नानाप्र‚कार‚{\tiny $_{lb}$}‚फ‚ल‚ज‚न‚न‚साम‚र्थ्यात् का\edtext{}{\edlabel{pvv.104-4}\label{pvv.104-4}\lemma{का}\Bfootnote{उत्पित्सुश्लोकाव‚ताराय प‚र‚म‚तं लिख्य‚ते । स्यादेत‚त् (।) तृष्णा हि क‚र्म‚निदानं ‚{\tiny $_{lb}$}‚त‚स्याश्च मार्ग्गात् क्ष‚यो निदानिनोपि क‚र्म‚णः क्ष‚यात् त्व‚यापि क‚र्म‚णां क्ष‚य एवेष्टः । ‚{\tiny $_{lb}$}‚येपि दोष‚विरोधिन उपाया नैरात्म्य‚द‚र्श‚न‚ल‚क्ष‚णा इष्य‚न्ते । दोष‚निदानं तृष्णान्ते ‚{\tiny $_{lb}$}‚उप‚घ्न‚न्ति । अस‚त्यां तृष्णायां अनाग‚त‚दोषानुत्प‚त्तिः । उत्पित्सुदोष‚निर्घातात्ते ‚{\tiny $_{lb}$}‚श‚क्ता । अनेन चातीत‚क‚र्म‚निदान‚तृष्णादिबाध‚को न भ‚व‚त्येव मार्ग्ग‚स्त‚दुत्प‚त्तिकाले ‚{\tiny $_{lb}$}‚त‚स्याभावादेवेत्य‚सिद्धिरुक्ता प‚रे (रैः?) । य‚व‚गोधूमाद्य‚नेक‚बीजेष्व‚नेकाङ‚कुर‚व‚त् ।}}र‚णा‚{\color{DodgerBlue3}‚देक‚रूपा}‚त् फ‚ला‚{\color{DodgerBlue3}‚त्ताप‚संक्लेशान्न क‚र्म‚णां क्ष‚यः} । (२७७)
	\pend% ending standard par
      \label{div_pvv.1.278}
	  
	% new div opening: depth here is 2
	

	  \pstart \leavevmode% starting standard par
	स‚र्व्वेषां क‚र्म‚णां ताप एव फ‚ल‚मिति चेत् ।
	\pend% ending standard par
      
	  \bigskip
	  \begingroup
	
	    \large
	  
	    \begin{quote}
	  
	    
	    \stanza[\smallbreak]
	\label{pv.1.278}\flagstanza{\tiny\textenglish{....1.278}}फ‚लं क‚थ‚ञ्चित् त‚ज्ज‚न्य‚म‚ल्पं स्यान्न विजातिम‚त् ॥&अथाऽपि त‚प‚सः श‚क्त्या श‚क्तिस‚ङ्क‚र‚संक्ष‚यैः ॥ २७८ ॥\&[\smallbreak]


	
	    \end{quote}
	  
	  \endgroup
	

	  \pstart \leavevmode% starting standard par
	\hphantom{.}‚{\color{DodgerBlue3}‚त‚स्य} क‚र्म‚णो ‚{\color{DodgerBlue3}‚ज‚न्यं फ‚लं\edtext{}{\edlabel{pvv.104-5}\label{pvv.104-5}\lemma{लं}\Bfootnote{क्षेत्रासंस्कारेल्प‚फ‚ल‚व‚त् ।}} क‚थ‚ञ्चित्} जुगुप्सादिभिर‚ल्पं ‚{\color{DodgerBlue3}‚स्यात् न} तु ‚{\color{DodgerBlue3}‚विजाति‚{\tiny $_{lb}$}‚\leavevmode\ledsidenote{\textenglish{20a/MA}} म‚त्} । दानं द‚त्त्वा हिंसित्वा वा जुगुप्सु‚{\tiny $_{7}$}‚स्त‚योः फ‚ल‚म‚ल्पीयोऽनुभ‚व‚ति । कार‚ण‚ता ‚{\tiny $_{lb}$}‚न वापादानात् । न तु शुभ‚स्य‚दुःख‚म‚शुभ‚स्य वा सुखं फ‚लं भ‚वितुम‚र्ह‚ति ॥ ‚{\color{DodgerBlue3}‚अथापि त‚प‚सः ‚{\tiny $_{lb}$}‚श‚क्त्या श‚क्तिसंक‚रेण} ताप‚क्लेश‚मात्र‚फ‚लेन तानि हीय‚न्ते । त‚पःश‚क्त्या क‚र्म‚णां संक्ष‚{\tiny $_{lb}$}‚येण वा ज‚न्माभावः । (२७८)
	\pend% ending standard par
      \label{div_pvv.1.279_1.280_1.281}
	  
	% new div opening: depth here is 2
	\textsuperscript{\textenglish{105/s}}
	  \bigskip
	  \begingroup
	
	    \large
	  
	    \begin{quote}
	  
	    
	    \stanza[\smallbreak]
	\label{pv.1.279a}\flagstanza{\tiny\textenglish{...1.279a}}क्लेशात् कुत‚श्चिद्धीयेताशेष‚म‚क्लेश‚लेश‚तः ॥\&[\smallbreak]


	
	    \end{quote}
	  
	  \endgroup
	

	  \pstart \leavevmode% starting standard par
	\hphantom{.}b. य‚च्च किञ्चिद‚व‚शिष्टं त‚त् ‚{\color{DodgerBlue3}‚क्लेशात् कुत‚श्चित्} केशोल्लु‚{\color{DodgerBlue3}‚ञ्च‚नादेः क्षीय‚ते} । ‚{\tiny $_{lb}$}‚क‚र्म‚क्ष‚याच्च मुक्तिः । अत्राह\edtext{}{\edlabel{pvv.105-1}\label{pvv.105-1}\lemma{अत्राह}\Bfootnote{अत्र श‚क्तेः क्ष‚यः साङ्क‚र्य‚म्वेति प‚क्षौ द्वौ ।}} (।)
	\pend% ending standard par
      

	  \pstart \leavevmode% starting standard par
	\hphantom{.}‚{\color{DodgerBlue3}‚हीयेताशेष‚म‚क्लेश‚लेश‚तः} । य‚दि त‚प‚सा क‚र्म‚क्ष‚योऽशेषं क‚र्म हीयेताक्लेश‚तो ‚{\tiny $_{lb}$}‚विनैव केशोल्लुञ्च‚नादि‚{\tiny $_{1}$}‚दुःखात्क‚र्म‚णः क्षीण‚त्वाद्य‚था नार‚कादि दुःखं न भ‚व‚ति ‚{\tiny $_{lb}$}‚त‚थाऽल्पीयोपि न स्यात् । श‚क्तिसांक‚र्येपि लेश‚तः स‚न्ताप‚क्लेशात् केव‚लात् क‚र्म ‚{\tiny $_{lb}$}‚हीयेत । न दुःखान्त‚रानुब‚न्धी संसार‚प्र‚ब‚न्धः त‚प‚स्विनः स्यात् ।
	\pend% ending standard par
      
	  \bigskip
	  \begingroup
	
	    \large
	  
	    \begin{quote}
	  
	    
	    \stanza[\smallbreak]
	\label{pv.1.279b}\flagstanza{\tiny\textenglish{...1.279b}}य‚दीष्ट‚म‚प‚रं क्लेशात् त‚त् त‚पः क्लेश एव चेत् ॥ २७९ ॥\&[\smallbreak]


	
	    \end{quote}
	  
	  \endgroup
	
	  \bigskip
	  \begingroup
	
	    \large
	  
	    \begin{quote}
	  
	    
	    \stanza[\smallbreak]
	\label{pv.1.280a}\flagstanza{\tiny\textenglish{...1.280a}}त‚त् क‚र्म‚फ‚ल‚मित्य‚स्मान्न श‚क्तेः स‚ङ्क‚रादिक‚म् ॥\&[\smallbreak]


	
	    \end{quote}
	  
	  \endgroup
	

	  \pstart \leavevmode% starting standard par
	\hphantom{.}त‚प‚सः श‚क्त्या श‚क्तिस‚ङ्क‚र‚संक्ष‚य‚श्च त‚दा व‚क्तुं श‚क्यो ‚{\color{DodgerBlue3}‚य‚दि स्यादिष्टं ‚{\tiny $_{lb}$}‚क्लेशाद‚प‚र\edtext{}{\edlabel{pvv.105-2}\label{pvv.105-2}\lemma{र}\Bfootnote{केशोल्लुञ्च‚नादेर‚न्य‚त् कुश‚ल‚रूपं त‚पोपीष्टं स्यात् ।}}}‚म‚न्य‚त्त‚पो नान्य‚था । ‚{\color{DodgerBlue3}‚क्लेश एव चेत्त‚त्त‚पः} । (२७९) ‚{\color{DodgerBlue3}‚त‚त्क्लेश‚रूपं} त‚पः ‚{\color{DodgerBlue3}‚क‚र्म‚फ‚लं} (त‚पोऽव‚शेषित‚स्य क‚र्म‚णः फ‚ल‚मिष्ट) ‚{\color{DodgerBlue3}‚मित्य‚स्मात्क‚र्म‚फ‚ल‚भूतात्त‚प\edtext{}{\edlabel{pvv.105-3}\label{pvv.105-3}\lemma{प}\Bfootnote{फ‚ल‚स्य क‚र्तृत्वायोगात् संक‚र‚क्ष‚ये ।}}सः ‚{\tiny $_{lb}$}‚श‚क्तिसंक‚रादिकं न‚{\tiny $_{2}$}‚} युक्तं । आदिश‚ब्दात्संक्ष‚य‚श्च ।
	\pend% ending standard par
      

	  \pstart \leavevmode% starting standard par
	C. त‚स्य\edtext{}{\edlabel{pvv.105-4}\label{pvv.105-4}\lemma{स्य}\Bfootnote{य‚न्म‚ते मार्ग्गात्तृष्णाक्ष‚ये क‚र्म‚णोपि क्ष‚य इष्टः ।}} म‚ते तु (।)
	\pend% ending standard par
      
	  \bigskip
	  \begingroup
	
	    \large
	  
	    \begin{quote}
	  
	    
	    \stanza[\smallbreak]
	\label{pv.1.280b}\flagstanza{\tiny\textenglish{...1.280b}}उत्पित्सुदोष‚निर्घाताद् येऽपि दोष‚विरोधिनः ॥ २८० ॥\&[\smallbreak]


	
	    \end{quote}
	  
	  \endgroup
	
	  \bigskip
	  \begingroup
	
	    \large
	  
	    \begin{quote}
	  
	    
	    \stanza[\smallbreak]
	\label{pv.1.281a}\flagstanza{\tiny\textenglish{...1.281a}}त‚ज्जे क‚र्माणि श‚क्ताः स्युः कृत‚हानिः क‚थं भ‚वेत् ॥\&[\smallbreak]


	
	    \end{quote}
	  
	  \endgroup
	

	  \pstart \leavevmode% starting standard par
	\hphantom{.}उत्पित्सोस्तृष्णादे‚{\color{DodgerBlue3}‚र्दोष}‚स्य ‚{\color{DodgerBlue3}‚निर्घातात् येपि दोष‚विरोधिनो} नैरात्म्याभ्यासाद‚य ‚{\tiny $_{lb}$}‚उपायाः (२८०) ‚{\color{DodgerBlue3}‚त‚ज्जे} तृष्णादिदोष‚प्र‚भ‚वे ‚{\color{DodgerBlue3}‚क‚र्म‚णि} कार‚ण‚वार‚ण‚द्वारेण व्याह‚न्तुं ‚{\tiny $_{lb}$}‚श‚क्ता न प्राग्ज‚निते त‚स्योत्प\edtext{}{\edlabel{pvv.105-5}\label{pvv.105-5}\lemma{स्योत्प}\Bfootnote{पूर्व्व‚कृत‚क‚मेक्ष‚यो नेति न कृत‚स्य हानिः ।}}न्न‚त्वात् । अतः ‚{\color{DodgerBlue3}‚कृत‚स्य} क‚र्म‚णो ‚{\color{DodgerBlue3}‚हानिः ‚{\tiny $_{lb}$}‚क‚थ‚म्भ‚वेत्} ।
	\pend% ending standard par
      

	  \pstart \leavevmode% starting standard par
	d. न‚नु य‚था दोषेभ्यः क‚र्म त‚था क‚र्म‚णो दोषाश्च भ‚व‚न्तीत्य‚क्षीण‚क‚र्म‚णो न ‚{\tiny $_{lb}$}‚स्यात् मुक्तिरित्याह (।)
	\pend% ending standard par
      
	  \bigskip
	  \begingroup
	
	    \large
	  
	    \begin{quote}
	  
	    
	    \stanza[\smallbreak]
	\label{pv.1.281b}\flagstanza{\tiny\textenglish{...1.281b}}दोषा न क‚र्म‚णो दुष्टः क‚रोति न विप‚र्य‚यात् ॥ २८१ ॥\&[\smallbreak]


	
	    \end{quote}
	  
	  \endgroup
	\textsuperscript{\textenglish{106/s}}

	  \pstart \leavevmode% starting standard par
	\hphantom{.}‚{\color{DodgerBlue3}‚दोषा न क‚र्म‚णो} भ‚व‚न्ति किन्तु दोषै‚{\color{DodgerBlue3}‚र्दुष्टः} प्राणी क‚र्म‚क‚रो भ‚व‚ति ‚{\color{DodgerBlue3}‚न विप‚{\tiny $_{3}$}‚‚{\tiny $_{lb}$}‚र्य‚यात्} । नादुष्टः क‚र्म क‚रोति नैरात्म्य‚द‚र्शिन‚स्तृष्णाभावात् । क्व‚चित् प्र‚वृत्तिनिवृत्त्य‚{\tiny $_{lb}$}‚स‚म्भ‚वात् । (२८१)
	\pend% ending standard par
      \label{div_pvv.1.282}
	  
	% new div opening: depth here is 2
	

	  \pstart \leavevmode% starting standard par
	न‚नु क‚र्म‚णः शुभात् सुखं सुखाद‚भिलाषोऽभिलाषाच्च राग इति क‚र्म‚णो दोष‚{\tiny $_{lb}$}‚ज‚न्मेत्याह (।)
	\pend% ending standard par
      
	  \bigskip
	  \begingroup
	
	    \large
	  
	    \begin{quote}
	  
	    
	    \stanza[\smallbreak]
	\label{pv.1.282a}\flagstanza{\tiny\textenglish{...1.282a}}मिथ्याविक‚ल्पेन विना नाभिलाषः सुखाद‚पि ।\&[\smallbreak]


	
	    \end{quote}
	  
	  \endgroup
	

	  \pstart \leavevmode% starting standard par
	\hphantom{.}य‚श्च ‚{\color{DodgerBlue3}‚सुखाद‚प्य‚भिलाषो} राग‚हेतुर्दृश्य‚ते स च ‚{\color{DodgerBlue3}‚मिथ्याविक‚ल्पेन विना} स्थिर‚{\tiny $_{lb}$}‚सुख‚म‚दीयाहंकार‚विक‚ल्प‚न‚म‚न्त‚रेण न भ‚व‚तीत्यायोनिशोम‚न‚स्कार एव दोष‚हेतुः ‚{\tiny $_{lb}$}‚न क‚र्म । त‚तः स‚त्य‚पि क‚र्म‚ण्युन्मूलितात्म‚दृष्ट‚यो निर्दोषा निर्व्वान्ति । निर्व्वाण‚ञ्च‚{\tiny $_{4}$}‚ ‚{\tiny $_{lb}$}‚दुःख‚निरोध‚ल‚क्ष‚णं । दुःखं प‚रिज्ञाय त‚त्स‚मुद‚यं मार्ग‚भाव‚न‚या प्र‚हाय प्राप्य‚ते नान्य‚{\tiny $_{lb}$}‚थेति ।
	\pend% ending standard par
      

	  \pstart \leavevmode% starting standard par
	च‚तुःस‚त्य‚प्र‚काश‚क एव मुमुक्षूणामुपास्य इति तायित्व‚मुक्तं ।
	\pend% ending standard par
      

	  \begin{center}%% label @type='head'
	\textbf{ङ. तायात् सुग‚त‚त्व‚सिद्धिः}
	\end{center}
	

	  \pstart \leavevmode% starting standard par
	\hphantom{.}त‚देव‚म‚नुलोम‚तः >प्र‚माण‚भूताये त्यादि प‚ञ्च‚प‚दानि व्याख्याय प्र‚तिलोम‚तो ‚{\tiny $_{lb}$}‚लिङ्ग‚लैङ्गिक‚त्वं द‚र्श‚य‚न्नाह (।)
	\pend% ending standard par
      
	  \bigskip
	  \begingroup
	
	    \large
	  
	    \begin{quote}
	  
	    
	    \stanza[\smallbreak]
	\label{pv.1.282b}\flagstanza{\tiny\textenglish{...1.282b}}तायात् त‚त्त्व‚स्थिराशेष‚विशेष‚ज्ञान‚साध‚न‚म् ॥ २८२ ॥\&[\smallbreak]


	
	    \end{quote}
	  
	  \endgroup
	

	  \pstart \leavevmode% starting standard par
	\hphantom{.}‚{\color{DodgerBlue3}‚तायाच्च‚तुः}‚स‚त्य‚प्र‚काश‚ल‚क्ष‚णात् लिङ्गात् ‚{\color{DodgerBlue3}‚त‚त्त्व‚स्थिराशेष}‚विशेष‚{\color{DodgerBlue3}‚ज्ञान}‚स्य\edtext{}{\edlabel{pvv.106-1}\label{pvv.106-1}\lemma{स्य}\Bfootnote{त‚त्त्व‚ञ्च स्थिर‚ञ्चाशेष‚ञ्च तैर्व्विशिष्य‚त इति विशेष‚ज्ञानं ।}} ‚{\tiny $_{lb}$}‚त्रिगुण‚स्य सुग‚त‚त्व‚स्य ‚{\color{DodgerBlue3}‚साध\edtext{}{\edlabel{pvv.106-2}\label{pvv.106-2}\lemma{साध}\Bfootnote{च‚तुःस‚त्य‚विष‚य‚म‚शेष‚त्वं ।}}नं} सिद्धिः । त‚त्त्व‚स्य क्ष‚णिक‚नैरात्म्य‚स्य ज्ञानात् प्र‚श‚स्तं । ‚{\tiny $_{lb}$}‚(२८२)
	\pend% ending standard par
      \label{div_pvv.1.283_1.284}
	  
	% new div opening: depth here is 2
	
	  \bigskip
	  \begingroup
	
	    \large
	  
	    \begin{quote}
	  
	    
	    \stanza[\smallbreak]
	\label{pv.1.283a}\flagstanza{\tiny\textenglish{...1.283a}}बोधार्थ‚त्वाद् ग‚मेः;\&[\smallbreak]


	
	    \end{quote}
	  
	  \endgroup
	

	  \pstart \leavevmode% starting standard par
	अपुन‚रावृ‚{\tiny $_{5}$}‚त्त्या च स्थिरं निःशेषं विशेष‚ज्ञानं त्रिगुणं सुग‚त‚त्वं ‚{\color{DodgerBlue3}‚बोधार्थ‚त्वाद्} ग‚मेर्ग‚त‚श‚ब्दः प्र‚कृतिः । न हि स‚म्वादिनोऽसाक्षात्कृत‚स्यार्थ‚स्य उप‚देशः श‚क्य‚क्रियः (।) ‚{\tiny $_{lb}$}‚न चानुमानेन ज्ञात‚स्योप‚देश इति युक्त‚मुक्तं । क्ष‚णिक‚त्व‚नैरात्म्यादिविष‚य‚स्यानु‚{\tiny $_{lb}$}‚मान‚स्य भ‚ग‚व‚दुप‚देश‚म‚न्त‚रेणोत्प‚त्तिबीजाभावात् अगृहीतोप‚देशानाम‚भावात् । ‚{\tiny $_{lb}$}‚न च नित्य‚प‚रोक्ष‚स्यार्थ‚स्य स्थैर्यादिविप‚र्य‚याध्य‚व‚सायिनः क‚श्चिन्निश्च‚योऽस्ति । ‚{\tiny $_{lb}$}‚\leavevmode\ledsidenote{\textenglish{107/s}} त‚स्मात्प्र‚माण‚स‚म्वादि‚{\tiny $_{6}$}‚नः प‚रोक्षार्थ‚स्योप‚देश‚स्त‚त्साक्षात्कार‚पूर्व्व‚क एवेति युक्तं ‚{\tiny $_{lb}$}‚तायित्वात् सुग‚त‚त्वानुमानं भ‚ग‚व‚तः । स च भ‚ग‚वान् तायः सुग‚त‚त्वात् त्रिगुणात् ‚{\tiny $_{lb}$}‚गुणानुक्र‚मेण ।
	\pend% ending standard par
      
	  \bigskip
	  \begingroup
	
	    \large
	  
	    \begin{quote}
	  
	    
	    \stanza[\smallbreak]
	\label{pv.1.283b}\flagstanza{\tiny\textenglish{...1.283b}}बाह्य‚शैक्षाशैक्षाधिक‚स्त‚तः ॥\&[\smallbreak]


	
	    \end{quote}
	  
	  \endgroup
	

	  \pstart \leavevmode% starting standard par
	\hphantom{.}‚{\color{DodgerBlue3}‚बाह्य‚शैक्षाशैक्षे}‚भ्योऽ‚{\color{DodgerBlue3}‚धिकः} ये लौकिक‚भाव‚नामार्गेण वीत‚रागा बाह्या अत‚त्व‚{\tiny $_{lb}$}‚द‚र्शिन‚स्तेभ्यः त‚त्त्व‚द‚र्शित्वाद‚धिकः । ये शैक्षा अबाह्याः प‚रिहाणिध‚र्माण‚स्तेभ्यो‚{\tiny $_{lb}$}‚ऽपुन‚रावृत्त्या । ये चाशैक्षाः श्राव‚का अप्र‚हीण‚क्लेश‚वास‚ना असाक्षात्कृत‚स‚र्व्वाकार‚{\tiny $_{lb}$}‚व‚स्त‚व‚स्तेभ्यो निःशेष‚प्र‚{\tiny $_{7}$}‚तीत्या । त‚स्मात् सुग‚त‚त्वात् शास‚नं शास्तृत्व‚म‚नुमीय‚ते ।
	\pend% ending standard par
      

	  \pstart \leavevmode% starting standard par
	कि पुनः शास‚न‚मित्याह (।)
	\pend% ending standard par
      
	  \bigskip
	  \begingroup
	
	    \large
	  
	    \begin{quote}
	  
	    
	    \stanza[\smallbreak]
	\label{pv.1.283c}\flagstanza{\tiny\textenglish{...1.283c}}प‚रार्थ‚ज्ञान‚घ‚ट‚नं त‚स्मात् त‚च्छास‚नं त‚तः ॥ २८३ ॥\&[\smallbreak]


	
	    \end{quote}
	  
	  \endgroup
	
	  \bigskip
	  \begingroup
	
	    \large
	  
	    \begin{quote}
	  
	    
	    \stanza[\smallbreak]
	\label{pv.1.284a}\flagstanza{\tiny\textenglish{...1.284a}}द‚याप‚रार्थ‚तंत्र‚त्वं ;\&[\smallbreak]


	
	    \end{quote}
	  
	  \endgroup
	

	  \pstart \leavevmode% starting standard par
	\hphantom{.}‚{\color{DodgerBlue3}‚त‚च्छास‚नं} कार‚णे कार्योप‚चारात् ‚{\color{DodgerBlue3}‚प‚रार्थं य‚ज्ज्ञानं} सुग‚त‚त्वं त‚द‚र्थं ‚{\color{DodgerBlue3}‚घ‚ट‚नं} व्यायामः । ‚{\tiny $_{lb}$}‚बुद्ध‚त्व‚साध‚न‚मार्गाभ्यास इत्य‚र्थः । न ह्युपाय‚म‚न्त‚रेणोपेय‚स‚म्भ‚वः । ‚{\color{DodgerBlue3}‚त‚तः} शास‚नाद् ‚{\tiny $_{lb}$}‚(२८३) ‚{\color{DodgerBlue3}‚द‚याप‚रार्थ‚त‚न्त्र‚त्वं} प‚रार्थ‚प्र‚धान‚त्वं ज‚ग‚द्धितैषित्व‚म‚नुमीय‚ते इत्य‚र्थः ।
	\pend% ending standard par
      

	  \pstart \leavevmode% starting standard par
	\hphantom{.}न‚नु नाव‚श्यं कारुणिक‚स्य मोक्ष‚मार्गाभ्यासः । ‚{\color{DodgerBlue3}‚स्वार्थ‚बुद्ध्यापि बाह्यानामिव} स‚म्भ‚वात् त‚त्क‚थ‚मुपायाभ्यासाद्द‚यानुमान‚मित्याह (।)
	\pend% ending standard par
      
	  \bigskip
	  \begingroup
	
	    \large
	  
	    \begin{quote}
	  
	    
	    \stanza[\smallbreak]
	\label{pv.1.284b}\flagstanza{\tiny\textenglish{...1.284b}}सिद्धार्थ‚स्याऽविराम‚तः ॥\&[\smallbreak]


	
	    \end{quote}
	  
	  \endgroup
	

	  \pstart \leavevmode% starting standard par
	\hphantom{.}‚{\color{DodgerBlue3}‚सिद्धार्थ‚स्य} निष्प‚न्न‚मोक्ष‚ल‚क्ष‚णात्म‚स‚म्वाद‚स्यापि सुग‚त‚स्य ख‚ड्गादिव‚त् प‚रा‚{\tiny $_{lb}$}‚र्थ‚क्रियातो‚{\color{DodgerBlue3}‚ऽविराम‚तो}‚ऽनिवृत्तेः फ‚लाव‚स्थायान्द‚यास‚द्भावाद्धेत्व‚व‚स्थायाम‚पि त‚स्या‚{\tiny $_{lb}$}‚स्तित्व‚म‚नुमान‚ञ्च ।
	\pend% ending standard par
      

	  \pstart \leavevmode% starting standard par
	ज‚ग‚द्धितैषित्व‚स्य सुग‚त‚त्व‚शास्तृत्व‚तायित्व‚स‚हित‚स्य प्रामाण्य‚साध‚न‚त्व‚माह (।)
	\pend% ending standard par
      
	  \bigskip
	  \begingroup
	
	    \large
	  
	    \begin{quote}
	  
	    
	    \stanza[\smallbreak]
	\label{pv.1.284c}\flagstanza{\tiny\textenglish{...1.284c}}द‚य‚या श्रेय आच‚ष्टे;\&[\smallbreak]


	
	    \end{quote}
	  
	  \endgroup
	

	  \pstart \leavevmode% starting standard par
	\hphantom{.}य‚तो ‚{\color{DodgerBlue3}‚द‚य‚या} ज‚ग‚द्धितैषित्वेन ‚{\color{DodgerBlue3}‚श्रेय आच‚ष्टे} । निर्द‚य‚स्तु विस‚म्वाद‚नाभिप्रायोपि ‚{\tiny $_{lb}$}‚ब्रूयात् ।
	\pend% ending standard par
      

	  \pstart \leavevmode% starting standard par
	स‚द‚योप्य‚भूत‚म‚ज्ञो व‚क्तीत्याह (।)
	\pend% ending standard par
      
	  \bigskip
	  \begingroup
	
	    \large
	  
	    \begin{quote}
	  
	    
	    \stanza[\smallbreak]
	\label{pv.1.284d}\flagstanza{\tiny\textenglish{...1.284d}}ज्ञानात् स‚त्यं स‚साध‚न‚म् ॥ २८४ ॥\&[\smallbreak]


	
	    \end{quote}
	  
	  \endgroup
	

	  \pstart \leavevmode% starting standard par
	\hphantom{.}‚{\color{DodgerBlue3}‚ज्ञाना}‚त् सुग‚त‚त्वात् भूत‚माच‚ष्टे । त‚च्च ज्ञानं ‚{\color{DodgerBlue3}‚स‚साध‚नं} विद्य‚मानोपायाभ्यासं विद्य‚{\tiny $_{lb}$}‚मान‚शास्तृत्व‚मित्य‚र्थः । (२८४)
	\pend% ending standard par
      \label{div_pvv.1.285_1.286_1.287}
	  
	% new div opening: depth here is 2
	
	  \bigskip
	  \begingroup
	
	    \large
	  
	    \begin{quote}
	  
	    
	    \stanza[\smallbreak]
	\label{pv.1.285a}\flagstanza{\tiny\textenglish{...1.285a}}त‚च्चाभियोग‚वान् व‚क्तुं य‚त‚स्त‚स्मात् प्र‚माण‚ता ॥\&[\smallbreak]


	
	    \end{quote}
	  
	  \endgroup
	\textsuperscript{\textenglish{108/s}}

	  \pstart \leavevmode% starting standard par
	\hphantom{.}‚{\color{DodgerBlue3}‚त‚च्च} स‚त्य‚च‚तुष्ट‚यं विनेयानां ‚{\color{DodgerBlue3}‚व‚क्तुम‚भियोगः} साद‚र‚स‚त‚त‚प्र‚वृत्तिस्त‚द्वान् तायी ‚{\tiny $_{lb}$}‚चेत्य‚र्थः । ‚{\color{DodgerBlue3}‚त‚स्मात्} कारुणिक‚त्वात् सुग‚त‚त्वात् शास्तृत्वात् तायित्वाच्च भ‚ग‚व‚तः ‚{\tiny $_{lb}$}‚‚{\color{DodgerBlue3}‚प्र‚माण‚ता} य‚थोप‚द‚र्शितार्थ‚स‚म्वाद‚क‚ताऽन्यैर‚ज्ञात‚च‚तुःस‚त्यार्थ‚प्र‚काश‚क‚ता वाऽनुमीय‚त ‚{\tiny $_{lb}$}‚इत्य‚र्थः ।
	\pend% ending standard par
      

	  \pstart \leavevmode% starting standard par
	एव‚ञ्चानुमानानुमेय‚व्य‚व‚हारे स्थिते प्रामाण्यात्तायित्वं हितैषित्वादु‚{\tiny $_{8}$}‚पाया‚{\tiny $_{lb}$}‚\leavevmode\ledsidenote{\textenglish{20b/MA}} भ्यासाच्च सुग‚त‚त्व‚म्भ‚व‚तीत्युक्तं ।
	\pend% ending standard par
      

	  \begin{center}%% label @type='head'
	\textbf{(७) संवाद‚क‚त्वात् भ‚ग‚वान् प्र‚माण‚म्}
	\end{center}
	

	  \pstart \leavevmode% starting standard par
	क‚स्मा\edtext{}{\edlabel{pvv.108-1}\label{pvv.108-1}\lemma{स्मा}\Bfootnote{प्र‚माण‚भूतायेत्येव स्तुतिप‚द‚माह \cref{ps.1.1} । [\edlabel{ps.1.1}प्र‚माण‚भूताय ज‚ग‚द्धितैषिणे प्र‚ण‚म्य शास्त्रे सुग‚ताय तायिने । प्र‚माण‚सिद्ध‚यै स्व‚म‚तात् स‚मुच्च‚यः क‚रिष्य‚ते विप्र‚सृतादिहैक‚तः ॥ (१।१) ॥ ]}}त्पुन‚र‚नेक‚गुण‚स‚म्भार‚स‚म्भ‚वेपि प्रामाण्येनैव भ‚ग‚व‚तः स्तुतिरित्याह (।)
	\pend% ending standard par
      
	  \bigskip
	  \begingroup
	
	    \large
	  
	    \begin{quote}
	  
	    
	    \stanza[\smallbreak]
	\label{pv.1.285b}\flagstanza{\tiny\textenglish{...1.285b}}उप‚देश‚त‚थाभाव‚स्तुतिस्त‚दुप‚देश‚तः ॥ २८५ ॥\&[\smallbreak]


	
	    \end{quote}
	  
	  \endgroup
	
	  \bigskip
	  \begingroup
	
	    \large
	  
	    \begin{quote}
	  
	    
	    \stanza[\smallbreak]
	\label{pv.1.286a}\flagstanza{\tiny\textenglish{...1.286a}}प्र‚माण‚त‚त्व‚सिध्य‚र्थं;\&[\smallbreak]


	
	    \end{quote}
	  
	  \endgroup
	

	  \pstart \leavevmode% starting standard par
	\hphantom{.}‚{\color{DodgerBlue3}‚उप‚देश‚स्य त‚थाभावः} स‚म्वाद‚क‚त्वं प्रामाण्यं । तेन ‚{\color{DodgerBlue3}‚स्तुति}‚राचार्येण कृता । त‚स्य ‚{\tiny $_{lb}$}‚भ‚ग‚व‚त ‚{\color{DodgerBlue3}‚उप‚देश‚तः} (२८५) ‚{\color{DodgerBlue3}‚प्र‚माण}‚स्य ‚{\color{DodgerBlue3}‚त‚त्त्वं} ल‚क्ष‚णं त‚{\color{DodgerBlue3}‚त्सिध्य}‚र्थं भ ग व द्दे श‚नायाः\edtext{}{\edlabel{pvv.108-2}\label{pvv.108-2}\lemma{नायाः}\Bfootnote{स‚काशात्प्र‚माण‚योः ।}} ‚{\tiny $_{lb}$}‚‚{\color{DodgerBlue3}‚प्र‚माण‚विनिश्च‚यो} नोत्प्रेक्षामात्रेणेत्याख्यातुमित्य‚र्थः ।
	\pend% ending standard par
      

	  \pstart \leavevmode% starting standard par
	\hphantom{.}a. न‚नु नील‚स‚म‚ङ्गी पुरुषो नीलं जानाति नो तु नील‚मिति ब्रुव‚ता भ‚ग‚{\tiny $_{lb}$}‚व‚ता प्र‚त्य‚क्षं द‚र्शितं । अनुमानं नोक्तं । क‚थ‚माग‚मात् प्र‚त्येत‚व्य‚मित्याह (।)
	\pend% ending standard par
      
	  \bigskip
	  \begingroup
	
	    \large
	  
	    \begin{quote}
	  
	    
	    \stanza[\smallbreak]
	\label{pv.1.286b}\flagstanza{\tiny\textenglish{...1.286b}}अनुमानेऽप्य‚वार‚णात् ।\&[\smallbreak]


	
	    \end{quote}
	  
	  \endgroup
	
	  \bigskip
	  \begingroup
	
	    \large
	  
	    \begin{quote}
	  
	    
	    \stanza[\smallbreak]
	\label{pv.1.286c}\flagstanza{\tiny\textenglish{...1.286c}}प्र‚योग‚द‚र्श‚नाद् वाऽस्य;\&[\smallbreak]


	
	    \end{quote}
	  
	  \endgroup
	

	  \pstart \leavevmode% starting standard par
	\hphantom{.}‚{\color{DodgerBlue3}‚अनुमानेप्य‚वार‚णादि}‚ष्टिर्द‚र्शिता । शून्याः प‚र‚प्र‚वादा इत्यादिना शाब्दादेरेव ‚{\tiny $_{lb}$}‚‚{\color{DodgerBlue3}‚निषेधात् प्र‚योग}\edtext{}{\edlabel{pvv.108-3}\label{pvv.108-3}\lemma{शाब्दादेरेव}\Bfootnote{अन्यान्य‚प्याक्षिप्तानीति ग्राह्याणि स्युरित्याह ।}}स्य प‚रार्थानुमान‚रूप‚स्य ‚{\color{DodgerBlue3}‚द‚र्श‚नाद्वाग‚मेऽस्या}‚नुमान‚स्य निर्देशः कृत एव ‚{\tiny $_{lb}$}‚‚{\color{DodgerBlue3}‚भ‚ग‚व‚ता} (।)
	\pend% ending standard par
      

	  \pstart \leavevmode% starting standard par
	त‚मेव प्र‚योग‚माह (।)
	\pend% ending standard par
      
	  \bigskip
	  \begingroup
	
	    \large
	  
	    \begin{quote}
	  
	    
	    \stanza[\smallbreak]
	\label{pv.1.286d}\flagstanza{\tiny\textenglish{...1.286d}}य‚त् किञ्चिदुद‚यात्म‚क‚म् ॥ २८६ ॥\&[\smallbreak]


	
	    \end{quote}
	  
	  \endgroup
	
	  \bigskip
	  \begingroup
	
	    \large
	  
	    \begin{quote}
	  
	    
	    \stanza[\smallbreak]
	\label{pv.1.287a}\flagstanza{\tiny\textenglish{...1.287a}}निरोध‚ध‚र्म‚कं स‚र्वं त‚दित्यादाव‚नेक‚धा ॥\&[\smallbreak]


	
	    \end{quote}
	  
	  \endgroup
	\textsuperscript{\textenglish{109/s}}

	  \pstart \leavevmode% starting standard par
	\hphantom{.}‚{\color{DodgerBlue3}‚य‚त्किञ्चिदुद‚यात्म‚कं} (२८६) ‚{\color{DodgerBlue3}‚त‚त्स‚र्वं निरोध‚ध‚र्म‚क‚मित्यादावाग‚म‚वाक्येऽनेक‚धा} स्व‚भावादिलिङ्ग‚ज‚म‚नेक‚प्र‚कार‚म‚नुमानं दुश्य‚ते ।
	\pend% ending standard par
      

	  \pstart \leavevmode% starting standard par
	b. क‚थं पुन‚र‚नेनानुमान‚मुक्त‚मित्याह (।)
	\pend% ending standard par
      
	  \bigskip
	  \begingroup
	
	    \large
	  
	    \begin{quote}
	  
	    
	    \stanza[\smallbreak]
	\label{pv.1.287b}\flagstanza{\tiny\textenglish{...1.287b}}अनुमानाश्र‚यो लिङ्ग‚म‚विनाभाव‚ल‚क्ष‚ण‚म् ॥ २८७ ॥\&[\smallbreak]


	
	    \end{quote}
	  
	  \endgroup
	
	  \bigskip
	  \begingroup
	
	    \large
	  
	    \begin{quote}
	  
	    
	    \stanza[\smallbreak]
	\label{pv.1.288}\flagstanza{\tiny\textenglish{....1.288}}व्याप्तिप्र‚द‚र्श‚नाद्धेतोः साध्येनोक्त‚ञ्च त‚त् स्फुट‚म् ॥\&[\smallbreak]


	
	    \end{quote}
	  
	  \endgroup
	

	  \pstart \leavevmode% starting standard par
	\hphantom{.}‚{\color{DodgerBlue3}‚अनुमान‚स्याश्र‚यः} कार‚णं ‚{\color{DodgerBlue3}‚लिङ्ग}‚म्व‚क्त‚व्य‚म‚{\tiny $_{2}$}‚नुमान‚निर्देशार्थ‚म‚न्य‚थाऽश‚क्य‚त्वात् ।
	\pend% ending standard par
      

	  \pstart \leavevmode% starting standard par
	\hphantom{.}किं ल‚क्ष‚ण‚मित्याह (।) ‚{\color{DodgerBlue3}‚अविनाभावः} साध्याव्य‚भिचारित्वं ‚{\color{DodgerBlue3}‚त‚ल्ल‚क्ष‚णं} य‚स्य ‚{\tiny $_{lb}$}‚त‚त्त‚था । (२८७) स चाविनाभावो लिङ्गं ल‚क्ष‚णं ‚{\color{DodgerBlue3}‚हेतो}‚रुद‚य‚ध‚र्म‚क‚त्व‚स्य ‚{\color{DodgerBlue3}‚साध्येन} निरोध‚ध‚र्म‚क‚त्वेन ‚{\color{DodgerBlue3}‚व्याप्तेः प्र‚द‚र्श‚नात्} य‚त्किञ्चिदुद‚य‚ध‚र्म‚कं ‚{\color{DodgerBlue3}‚त‚त्स‚र्व्वं निरोध‚ध‚र्म-} क‚मि \edtext{\textsuperscript{*}}{\edlabel{pvv.109-1}\label{pvv.109-1}\lemma{*}\Bfootnote{म‚हाव‚ग्गे १।१।८}}त्यादिना ‚{\color{DodgerBlue3}‚स्फुटः} प्र‚व्य‚क्तो द‚र्शित (:।)
	\pend% ending standard par
      

	  \pstart \leavevmode% starting standard par
	इत्य‚नुमान‚प्रामाण्य‚निर्देशोपि भ‚ग‚व‚दुप‚ज्ञ‚मेव (।) त‚देवं भ‚ग‚वानेव प्र‚माण‚भूत‚{\tiny $_{lb}$}‚स्तायी मुमुक्षुभिरुपास्यो नेत‚र इति‚{\tiny $_{3}$}‚ द‚र्श‚नार्थ‚मा चा र्ये ण ‚{\color{DodgerBlue3}‚त‚स्य स्तुतिरुक्तेति युक्तं} ॥
	\pend% ending standard par
      
	    
	    \pstart
	    \begin{center}
	  आचार्य म नो र थ न न्दि कृतायां प्र‚माणाव‚र्त्तिक‚वृत्तौ प्र‚थ‚मः प‚रिच्छेदः ॥
	    \end{center}
	    \pend
	  
	  
	    
	    \endnumbering% ending numbering from div
	    \endgroup
	    
	  
	  
	% new div opening: depth here is 0
	
	    
	    \begingroup
	    \beginnumbering% beginning numbering from div depth=0
	    
	  
\chapter*[{द्वितीयः प‚रिच्छेदःद्र‚ष्ट‚व्यं प‚रिशिष्टं ।५: प्र‚त्य‚क्ष‚म्}]{द्वितीयः प‚रिच्छेदःद्र‚ष्ट‚व्यं प‚रिशिष्टं ।५
	: प्र‚त्य‚क्ष‚म्}
	  
	% new div opening: depth here is 1
	

	  \pstart \leavevmode% starting standard par
	\leavevmode\ledsidenote{\textenglish{110/s}}प्र‚थ‚म‚प‚रिच्छेदेन प्र‚माण‚सामान्य‚ल‚क्ष‚णं व्य‚व‚स्थाप्य विशेष‚ल‚क्ष‚ण‚माख्यातुं ‚{\tiny $_{lb}$}‚द्वितीय‚प‚रिच्छेदार‚म्भः ।
	\pend% ending standard par
      
	  
	% new div opening: depth here is 1
	
\chapter*[{(१. प्र‚माण‚संख्याविप्र‚तिप‚त्तिनिरासः)}]{(१. प्र‚माण‚संख्याविप्र‚तिप‚त्तिनिरासः)}\label{div_pvv.2.1_2.2}
	  
	% new div opening: depth here is 2
	

	  \pstart \leavevmode% starting standard par
	विप्र‚तिप‚त्त‚य‚श्चात्र संख्याल‚क्ष‚ण‚गोच‚र‚फ‚ल‚विष‚याः स‚न्ति । त‚त्र संख्या‚{\tiny $_{lb}$}‚विप्र‚तिप‚त्तिनिराक‚र‚णार्थ‚माह (।)
	\pend% ending standard par
      
	  \bigskip
	  \begingroup
	
	    \large
	  
	    \begin{quote}
	  
	    
	    \stanza[\smallbreak]
	\label{pv.2.1a}\flagstanza{\tiny\textenglish{...v.2.1a}}मानं द्विविधं विष‚य‚द्वैविध्यात्;\&[\smallbreak]


	
	    \end{quote}
	  
	  \endgroup
	

	  \pstart \leavevmode% starting standard par
	\hphantom{.}‚{\color{DodgerBlue3}‚मानं द्विविधं} । य‚त्त‚त्प्र‚माण‚म‚विस‚म्वादित्वाद‚ज्ञातार्थ‚प्र‚काश‚क‚त्वात्सामान्य-‚{\tiny $_{4}$}‚ ‚{\tiny $_{lb}$}‚ल‚क्ष‚ण‚मुक्तं त‚द् द्विविधं । प्र‚त्य‚क्षानुमान‚भेदेन । क‚स्माद् (।) ‚{\color{DodgerBlue3}‚विष‚य}‚स्य स्व‚ल‚क्ष‚ण‚सा‚{\tiny $_{lb}$}‚मान्य‚ल‚क्ष‚ण‚रूप‚त‚या ‚{\color{DodgerBlue3}‚द्वैविध्यात्} । शाब्दादिक‚म‚पि हि प्र‚माण‚म्भ‚व‚त्स‚विष‚यं व‚क्त‚व्यं (।) ‚{\tiny $_{lb}$}‚विष‚य‚श्च स्व‚सामान्य‚ल‚क्ष‚णाद‚तिरिक्तो नास्ति । त‚त‚स्त‚द्विष‚य‚त्वे प्र‚त्य‚क्षानुमान‚{\tiny $_{lb}$}‚तैव । नापि सामान्य‚विशेषात्म‚क एकोस्ति विष‚यः\edtext{}{\edlabel{pvv.110-1}\label{pvv.110-1}\lemma{यः}\Bfootnote{नीलोत्प‚लादि प‚र‚स्य अर्थ‚साम‚र्थ्ये स्व‚ल‚क्ष‚ण‚त्व‚मेव ।}} । प‚र‚स्प‚र‚विरुद्ध‚योरैकात्म्या‚{\tiny $_{lb}$}‚योगात् ।
	\pend% ending standard par
      

	  \pstart \leavevmode% starting standard par
	न‚नु क‚थं विष‚य‚द्वैविध्य‚सिद्धिः (।) न प्र‚त्य‚क्षान्नाप्य‚नुमान‚तो य‚थाक्र‚मं स्व‚ल‚क्ष‚ण‚{\tiny $_{lb}$}‚सामा‚{\tiny $_{5}$}‚न्य‚ल‚क्ष‚ण‚त्वाद‚न‚योः । द्वाभ्यां द्व‚य‚सिद्धिरिति चेत् ।
	\pend% ending standard par
      

	  \pstart \leavevmode% starting standard par
	न‚नु प्र‚त्य‚क्ष‚स्य सामान्याविष‚य‚त्वे साध्य‚साध‚न‚स‚म्ब‚न्धाग्र‚ह‚णाद‚नुमान‚मेव न ‚{\tiny $_{lb}$}‚स्यात् । सामान्य‚विष‚य‚त्वे च प्र‚त्य‚क्ष‚त्वादेव त‚त्सिद्धेर्व्विफ‚ल‚म‚नुमान‚स्य प्रामाण्य‚{\tiny $_{lb}$}‚क‚ल्प‚नं ।
	\pend% ending standard par
      

	  \pstart \leavevmode% starting standard par
	अत्रोच्य‚ते । प्र‚त्य‚क्ष‚म‚पि स्व‚ल‚क्ष‚णं विष‚यीकुर्व्व‚त् त‚त्स‚म्भ‚वि विजातीय‚{\tiny $_{lb}$}‚व्यावृत्त्युप‚क‚ल्पितं सामान्यं पृष्ठ‚विक‚ल्पेन निश्चिन्व‚त्त‚द्विष‚य‚म‚पि निश्च‚य‚विष‚येण ‚{\tiny $_{lb}$}‚\leavevmode\ledsidenote{\textenglish{111/s}} च प्र‚त्य‚क्ष‚विष‚य‚व्य‚व‚स्था । एव‚न्त‚र्हि स्व‚ल‚क्ष‚ण‚विष‚य‚ता‚{\tiny $_{6}$}‚ न स्यादिति चेत्\edtext{}{\edlabel{pvv.111-1}\label{pvv.111-1}\lemma{चेत्}\Bfootnote{न‚नु य‚द‚तीन्द्रियं केशादिव्य‚व‚हितं स्व‚ल‚क्ष‚णं, य‚च्च सामान्यं न गोच‚रोनुमान‚स्य, त‚स्य क‚थं व्य‚व‚स्था ।  --- य‚ज्जातीयैः प्र‚माणैश्च य‚ज्जातीयार्थ‚द‚र्श‚नं । भ‚वेदिदानीं लोक‚स्य त‚था कालान्त‚रेष्व‚पी ति ‚{\tiny $_{lb}$}‚\href{http://sarit.indology.info/?cref=śv}{[कुमारिल]व‚च‚नात्} ।}} । न(।) ‚{\tiny $_{lb}$}‚स‚जातीय‚व्यावृत्त‚त्वेनापि त‚तो निश्च‚यात्। द्वे च व्यावृत्ती स्व‚ल‚क्ष‚णे स्तो निश्चिते च ‚{\tiny $_{lb}$}‚प्र‚त्य‚क्ष‚ब‚लात्। न चैव‚म‚प्य‚नुमान‚स्य वैय‚र्थ्य (।) न हि सामान्य‚मित्येव प्र‚त्य‚क्ष‚विष‚यः । ‚{\tiny $_{lb}$}‚प‚रोक्षे त‚स्याप्र‚वृत्तेः । न च य‚देक‚दाऽप‚रोक्षं त‚त्स‚र्व‚दा त‚था । स्व‚ल‚क्ष‚णं क‚दाचिद‚{\tiny $_{lb}$}‚प‚रोक्ष‚म‚प्य‚न्य‚दा प‚रोक्षं एवं सामान्य‚म‚पि । त‚तोऽप‚रोक्षे सामान्ये गृहीतायां व्याप्तौ ‚{\tiny $_{lb}$}‚प‚रोक्षे त‚स्मिन्न‚नुमान‚वृत्तिरिति न क‚श्चिद्विरोधः । त‚स्मात्‚{\tiny $_{7}$}‚ प्र‚त्य‚क्ष‚त्वाद्वा विष‚य‚{\tiny $_{lb}$}‚द्वैविध्य‚सिद्धिः प्र‚त्य‚क्षानुमानाभ्यां ‚{\color{DodgerBlue3}‚वेत्युभ‚य}‚थाप्युप‚प‚न्नं ।
	\pend% ending standard par
      

	  \pstart \leavevmode% starting standard par
	विष‚य‚द्वैविध्य‚मेव क‚स्मादित्याह (।)
	\pend% ending standard par
      
	  \bigskip
	  \begingroup
	
	    \large
	  
	    \begin{quote}
	  
	    
	    \stanza[\smallbreak]
	\label{pv.2.1b}\flagstanza{\tiny\textenglish{...v.2.1b}}श‚क्त्य‚श‚क्तितः ।&अर्थ‚क्रियायां;\&[\smallbreak]


	
	    \end{quote}
	  
	  \endgroup
	

	  \pstart \leavevmode% starting standard par
	\hphantom{.}‚{\color{DodgerBlue3}‚श‚क्त्य‚श‚क्तितोऽर्थ‚क्रियायां} । स्व‚ल‚क्ष‚ण‚स्यार्थ‚क्रियाश‚क्त‚त्वात् । ‚{\color{DodgerBlue3}‚विजातीय‚व्या}‚{\tiny $_{lb}$}‚वृत्त्युप‚क‚ल्पित‚स्य च सामान्य‚स्याश‚क्त‚त्वात् विष‚य‚द्वैविध्यं । न ह्येक‚स्य ‚{\color{DodgerBlue3}‚विरुद्धाविमौ} ध‚र्मौ युज्येते । य‚द्य‚न‚र्थ‚क्रियाकारि सामान्यं ‚{\color{DodgerBlue3}‚केशोण्डुक‚ज्ञान‚प्र‚तिभासि केशाद्य‚पि} सामान्यं स्यात् ।
	\pend% ending standard par
      
	  \bigskip
	  \begingroup
	
	    \large
	  
	    \begin{quote}
	  
	    
	    \stanza[\smallbreak]
	\label{pv.2.1c}\flagstanza{\tiny\textenglish{...v.2.1c}}केशादिर्न्नार्थोन‚र्थाधिमोक्ष‚तः ॥ १ ॥\&[\smallbreak]


	
	    \end{quote}
	  
	  \endgroup
	

	  \pstart \leavevmode% starting standard par
	\hphantom{.}न ‚{\color{DodgerBlue3}‚के\edtext{}{\edlabel{pvv.111-2}\label{pvv.111-2}\lemma{के}\Bfootnote{अन‚र्थ‚क्रियातो न स्व‚ल‚क्ष‚णं, स्प‚ष्ट‚प्र‚तिभास्य‚न‚न्व‚यित्वाभ्यां न च सामान्य‚मिति विष‚यान्त‚र‚त्व‚म‚स्य ॥}}शादिर}‚र्थः सामान्य‚रूपोऽ‚{\color{DodgerBlue3}‚न‚र्थाधिमोक्ष‚तः} । (१) य‚त्र हि व्य‚व‚ह‚{\tiny $_{8}$}‚र्तॄ\edtext{}{\edlabel{pvv.111-3}\label{pvv.111-3}\lemma{र्तॄ}\Bfootnote{अथारोप्यैत‚त् अन्य‚थोत्प‚त्तिसारूप्याभ्यां विष‚य‚त्वे आकारो ब‚हिर्भासिकेशादेर्नोत्पाद‚को न स‚रूप‚को नापि स‚म्वित्त‚स्येति चोद्यान‚व‚काश एव सांव्य‚व‚हारि ‚{\tiny $_{lb}$}‚प्र‚माण‚मेत‚त् । देशादिविप्र‚कृष्ट‚न्तु (।) य‚ज्जातीयैः प्र‚माणैश्च य‚ज्जातीयार्थ‚द‚र्श‚नं । भ‚वेदिदानीं लोक‚स्य त‚था कालान्त‚रेष्व‚पि  \href{http://sarit.indology.info/?cref=śv}{[कुमारिल‚स्य]} ॥}}णा-\leavevmode\ledsidenote{\textenglish{21a/MA}} ‚{\tiny $_{lb}$}‚म‚र्थाध्य‚व‚सायः सोऽर्थः स्व‚ल‚क्ष‚णं सामान्य‚म्वा स्यात् । य‚त्र पुन‚र‚र्थ‚बुद्धिरेव नास्ति ‚{\tiny $_{lb}$}‚स क‚थं सामान्य‚मुच्य‚तां ।
	\pend% ending standard par
      \textsuperscript{\textenglish{112/s}}
	  \bigskip
	  \begingroup
	
	    \large
	  
	    \begin{quote}
	  
	    
	    \stanza[\smallbreak]
	\label{pv.2.2}\flagstanza{\tiny\textenglish{...pv.2.2}}स‚दृशास‚दृश‚त्वाच्च विष‚याविष‚य‚त्व‚तः ।&श‚ब्द‚स्यान्य‚निमित्तानां भावे धीः स‚द‚स‚त्व‚तः ॥ २ ॥\&[\smallbreak]


	
	    \end{quote}
	  
	  \endgroup
	

	  \pstart \leavevmode% starting standard par
	\hphantom{.}‚{\color{DodgerBlue3}‚त‚था\edtext{}{\edlabel{pvv.112-1}\label{pvv.112-1}\lemma{था}\Bfootnote{सांख्य‚म‚तेनाह ।}} त‚दृशास‚दृश‚त्वाच्च} विष‚य‚द्वैविध्यं । स‚दृशं सामान्यं स‚र्व‚व्य‚क्तिसाधार‚ण‚{\tiny $_{lb}$}‚त्वात् । अस‚दृशं स्व‚ल‚क्ष‚णं स‚र्व्व‚तो व्यावृत्त‚त्वात् । अन‚योश्चान्योन्य‚व्य‚व‚च्छेद‚रूप‚{\tiny $_{lb}$}‚त्वात् न राश्य‚न्त‚रं । त‚तो य‚दि क‚ल्प्य‚मानं स‚दृशं त‚दा सामान्य‚मेव त‚त् । ‚{\tiny $_{lb}$}‚अथास‚दृशं स्व‚ल‚क्ष‚ण‚मेवेति द्वैविध्य‚मेव विष‚य‚स्य । त‚था\edtext{}{\edlabel{pvv.112-2}\label{pvv.112-2}\lemma{था}\Bfootnote{ज्ञान‚द्वारेण निराकृत्य श‚ब्द‚मुखेनाह ।}} श‚ब्द‚स्य विष‚याविष‚य‚{\tiny $_{lb}$}‚त्व‚त‚श्च द्वैविध्यं । ‚{\color{DodgerBlue3}‚श‚ब्द‚स्य} विष‚यः ‚{\color{DodgerBlue3}‚सामान्यं} । अविष‚यः स्व‚ल‚क्ष‚णं । न च ‚{\tiny $_{lb}$}‚श‚ब्द‚विष‚याविष‚याभ्या‚{\color{DodgerBlue3}‚म‚न्यो}‚स्ति स‚र्व्व‚स्य संग्र‚हात् द्वैविध्य‚मेव । त‚था विष‚याद‚{\tiny $_{lb}$}‚‚{\color{DodgerBlue3}‚न्य‚षान्निमित्तानां} म‚न‚स्कार‚व‚त् साद्गुण्य‚संकेत‚ग्र‚ह‚णानां ‚{\color{DodgerBlue3}‚भावे} ग्राहिकाया धियः ‚{\tiny $_{lb}$}‚सा\edtext{}{\edlabel{pvv.112-3}\label{pvv.112-3}\lemma{सा}\Bfootnote{व्य‚व‚ह‚र्तृ व्य‚व‚सायान्न व‚स्तुतः ।}}मान्ये ‚{\color{DodgerBlue3}‚स‚त्त्वात्} स्व‚ल\edtext{}{\edlabel{pvv.112-4}\label{pvv.112-4}\lemma{ल}\Bfootnote{म‚न‚स्कारादीनां भावेपि य‚द‚भावे धियोऽभाव‚स्त‚त्स्व‚ल‚क्ष‚णं ।}}क्ष‚णे चाभावात् विष‚य‚द्वैविध्यं । य‚त्र विष‚य‚व्य‚तिरिक्त‚{\tiny $_{lb}$}‚निमित्त‚स‚द्भावे भ‚व‚ति बुद्धिस्त‚त्सामान्यं (।) य‚त्र\edtext{}{\edlabel{pvv.112-5}\label{pvv.112-5}\lemma{त्र}\Bfootnote{विष‚ये स‚त्येव बुद्धिर्भ‚व‚ति ।}}तु न भ‚व‚ति त‚त् स्व‚ल‚क्ष‚णं । ‚{\tiny $_{lb}$}‚प्र‚कारान्त‚र‚ञ्च न स‚म्भ‚व‚तीति बुद्धिविष‚याविष‚य‚त्वे सामान्य‚स्व‚ल‚क्ष‚ण‚तैवेति द्वैविध्य‚{\tiny $_{lb}$}‚मेव विष‚य‚स्य । (२)
	\pend% ending standard par
      
	  
	% new div opening: depth here is 1
	
\chapter*[{(२. स‚त्य‚द्व‚य‚चिन्ता)}]{(२. स‚त्य‚द्व‚य‚चिन्ता)}\label{div_pvv.2.3}
	  
	% new div opening: depth here is 2
	

	  \pstart \leavevmode% starting standard par
	त‚देवार्थ‚क्रियासाम‚र्थ्यादिकं स्व‚ल‚क्ष‚णा‚{\tiny $_{2}$}‚दौ योज‚य‚न्नाह ।
	\pend% ending standard par
      
	  \bigskip
	  \begingroup
	
	    \large
	  
	    \begin{quote}
	  
	    
	    \stanza[\smallbreak]
	\label{pv.2.3}\flagstanza{\tiny\textenglish{...pv.2.3}}अर्थ‚क्रियास‚म‚र्थ य‚त् त‚द‚त्र प‚र‚मार्थ‚स‚त् ।&अन्य‚त् संवृतिस‚त् प्रोक्तं; ते स्व‚सामान्य‚ल‚क्ष‚णो ॥ ३ ॥\&[\smallbreak]


	
	    \end{quote}
	  
	  \endgroup
	

	  \pstart \leavevmode% starting standard par
	\hphantom{.}‚{\color{DodgerBlue3}‚अर्थ‚क्रियायां} ज्ञानादिकायां स्व‚रूपोप‚धानेन ‚{\color{DodgerBlue3}‚स‚म‚र्थं य‚त्त‚द‚त्र} व‚स्तुविचारो ‚{\color{DodgerBlue3}‚प‚र‚मार्थ‚{\tiny $_{lb}$}‚स‚त्} । एवं य‚द‚स‚दृशं श‚ब्दाविष‚योऽन्य‚निमित्त‚भावे ज्ञानाभाव‚श्च त‚त्प‚र‚मार्थ‚स‚त् । ‚{\tiny $_{lb}$}‚‚{\color{DodgerBlue3}‚अतोऽन्य‚द}‚श‚क्तं स‚दृशं श‚ब्द‚विष‚यः । अन्य‚निमित्त‚भावे बुद्धेर्व्विष‚य‚श्च त‚त् ‚{\color{DodgerBlue3}‚संवृतिस‚त् ‚{\tiny $_{lb}$}‚प्रोक्तं} क‚ल्प‚नामात्र‚व्य‚व‚हार्य‚त्वात् । ते प‚र‚मार्थ‚संवृती ‚{\color{DodgerBlue3}‚स्व‚सामान्य‚ल‚क्ष‚णे} । (३)
	\pend% ending standard par
      
	  
	% new div opening: depth here is 1
	
\chapter*[{(३. सामान्य‚त‚त्क‚ल्य‚नानिरासः)}]{(३. सामान्य‚त‚त्क‚ल्य‚नानिरासः)}\label{div_pvv.2.4}
	  
	% new div opening: depth here is 2
	
	  \bigskip
	  \begingroup
	
	    \large
	  
	    \begin{quote}
	  
	    
	    \stanza[\smallbreak]
	\label{pv.2.4a}\flagstanza{\tiny\textenglish{...v.2.4a}}अश‚क्तं स‚र्व‚मिति चेद् बीजादेर‚ङ्कुरादिषु ।&दृष्टा श‚क्तिः;\&[\smallbreak]


	
	    \end{quote}
	  
	  \endgroup
	

	  \pstart \leavevmode% starting standard par
	\hphantom{.}स्व‚ल‚क्ष‚ण‚सामान्य‚ल‚क्ष‚णे इष्टे ॥ ‚{\color{DodgerBlue3}‚स‚र्व्व}‚म‚र्थ‚कारित्वेने\edtext{}{\edlabel{pvv.112-6}\label{pvv.112-6}\lemma{कारित्वेने}\Bfootnote{माध्य‚मिको सिद्ध‚तामाह ।}}ष्ट‚{\color{DodgerBlue3}‚म‚श‚क्तं} । न किञ्चित् ‚{\tiny $_{lb}$}‚क‚र्त्तुं स‚म‚{\tiny $_{3}$}‚‚{\color{DodgerBlue3}‚र्थ‚मिति चेत् । बीजादेः} कार‚णाभिम‚त‚स्या‚{\color{DodgerBlue3}‚ङ्कुरादौ} कार्य‚संम‚ते ‚{\color{DodgerBlue3}‚दृष्टा श‚क्ति-} \leavevmode\ledsidenote{\textenglish{113/s}} र्ज‚न‚न‚ल‚क्ष‚णा । बीजान्व‚य‚व्य‚तिरेकानुविधाय्य‚ङ्कुरो दृश्य‚ते (।) इद‚मेव कार‚ण‚स्य ‚{\tiny $_{lb}$}‚श‚क्त‚त्वं य‚त्प्राग‚दृष्ट‚स्य त‚द्भाव एव भावः ॥
	\pend% ending standard par
      
	  \bigskip
	  \begingroup
	
	    \large
	  
	    \begin{quote}
	  
	    
	    \stanza[\smallbreak]
	\label{pv.2.4b}\flagstanza{\tiny\textenglish{...v.2.4b}}म‚ता सा चेत् संवृत्या;\&[\smallbreak]


	
	    \end{quote}
	  
	  \endgroup
	

	  \pstart \leavevmode% starting standard par
	\hphantom{.}सा श‚क्तिः ‚{\color{DodgerBlue3}‚संवृत्या म‚ता चेत्} कार्य‚कार‚ण‚भावो हि व्य‚व‚हार‚मात्र‚तः सिद्धः न ‚{\tiny $_{lb}$}‚प‚र‚मार्थ‚तः । न ताव‚त्प्र‚त्य‚क्षं त‚द्ग्र‚ह‚ण‚स‚म‚र्थं बीजाङ्कुर‚ग्राहिणोः प्र‚त्य‚क्ष‚योः ‚{\tiny $_{lb}$}‚स्व‚विष‚य‚मात्र‚व्य‚व‚स्थाप‚नात् केनान्व‚य‚व्य‚तिरेक‚ग्र‚ह‚णं । क्र‚मेण द्व‚योर्गृहीत‚योस्त‚द्ब‚ल‚{\tiny $_{4}$}‚‚{\tiny $_{lb}$}‚भाविना स्म‚र‚णेन ग्र‚ह‚ण‚मिति चेत् ।
	\pend% ending standard par
      

	  \pstart \leavevmode% starting standard par
	न‚नु केनान्व‚य‚व्य‚तिरेकौ गृहीतौ । न प्र‚त्येकं बीजाङ्कुर‚ज्ञानाभ्यां स्व‚स्व‚विष‚य‚{\tiny $_{lb}$}‚ग्र‚ह‚णात् । नापि द्वाभ्यां ज्ञान‚योर्ज्ञेय‚योश्च साहित्याभावात् । असाहित्ये बीजाङ्कुर‚{\tiny $_{lb}$}‚मात्र‚स्य ग्र‚ह‚णं नान्व‚य‚व्य‚तिरेक‚योः । क्र‚म‚ग्र‚ह‚ण‚मेव कार्य‚कार‚ण‚भाव‚ग्र‚ह‚ण‚म् । त‚त्तु घ‚ट‚{\tiny $_{lb}$}‚कुलाल‚योर‚प्य‚स्ति इति चेत् (।) न च क्र‚मोपि केन‚चिच्छ‚क्य‚ग्र‚ह‚णः\edtext{}{\edlabel{pvv.113-1}\label{pvv.113-1}\lemma{णः}\Bfootnote{स्व‚रूप‚मात्र‚वेद‚नात् ।}}प्र‚तियोग्य‚वेद‚ना\edtext{}{\edlabel{pvv.113-2}\label{pvv.113-2}\lemma{ना}\Bfootnote{अक्र‚म‚वेद(ना)पेक्ष‚त्वात् ।}}त् । ‚{\tiny $_{lb}$}‚पूर्वाप‚र‚ग्र‚ह‚ण‚म‚त एव नास्ति स्व‚ज्ञानेन व‚र्त‚मान‚ता‚{\tiny $_{5}$}‚ग्र‚ह‚णाच्च । कार्य‚काले च कार‚णं ‚{\tiny $_{lb}$}‚पूर्व्व‚मुच्य‚ते त‚दा च त‚देव नास्ति । त‚देत‚न्मृत‚स्यारोग्यं । अथ य‚दैव बीजं त‚दैवाङ्कु‚{\tiny $_{lb}$}‚रात् पूर्व्वं न तु प‚श्चात् अस्य पूर्व्व‚त्वं स‚म्भ‚व‚ति ॥
	\pend% ending standard par
      

	  \pstart \leavevmode% starting standard par
	न‚न्वेवं पूर्व्व‚त‚या प्र‚तिभासोस्य प्राप्तः । न चैत‚द‚स्ति । अङ्कुर‚साहित्यं पूर्व्व‚त्वं ‚{\tiny $_{lb}$}‚त‚च्च गृह्य‚त एवेति चेत् । तादृशं पूर्व्व‚त्व‚म‚न्येषाम‚प्य‚स्तीति तेपि कार‚णानि ‚{\tiny $_{lb}$}‚स्युः । किञ्च (।) पूर्व्व‚त्वं व‚र्त‚मान‚कालात्प्राग्भावित्व‚मुच्य‚ते (।) त‚द्य‚दि व‚स्तुनो रूपं ‚{\tiny $_{lb}$}‚त‚दा व‚र्त‚मानं क‚दापि न‚{\tiny $_{6}$}‚ स्यात् । व‚र्त‚मान‚तात‚त्प्राग्भावित्व‚योर्व्विरोधात् । स्यादे‚{\tiny $_{lb}$}‚त‚द‚ङ्कुर‚व‚र्त‚मान‚तायाः प्राग्भावित्वं पूर्व्व‚त्वं । त‚च्च बीज‚स्य व‚र्त‚मान‚त्वेनाविरुद्ध‚{\tiny $_{lb}$}‚मिति न त‚स्याभावः । एव‚न्त‚र्हि बीज‚ग्र‚ह‚णे पूर्व्व‚ताग्र‚ह‚णं प्राप्तं । न च बीज‚स्व‚रूप‚{\tiny $_{lb}$}‚मिव पूर्व्व‚ताम‚पि त‚द्ग्राहिणि ज्ञाने क‚श्चिदुप‚ल‚भ‚ते ।
	\pend% ending standard par
      

	  \pstart \leavevmode% starting standard par
	न\edtext{}{\edlabel{pvv.113-3}\label{pvv.113-3}\lemma{न}\Bfootnote{प‚रः ।}}न्वापेक्षिक‚मिदं पूर्व्व‚त्वं प्र‚तियोगिनोऽप्र‚तीतौ क‚थं प्र‚तीय‚तां । य\edtext{}{\edlabel{pvv.113-4}\label{pvv.113-4}\lemma{य}\Bfootnote{माध्य‚मिकः ।}}द्येवं व‚स्तुनो ‚{\tiny $_{lb}$}‚नेदं केव‚लं भावान्त‚रापेक्ष‚या व्य‚व‚ह्रिय‚त इत्या(या)तं‚{\tiny $_{7}$}‚(।) न हि व‚स्तुरूपं स‚ति व‚स्तुनि\leavevmode\ledsidenote{\textenglish{21b/MA}} ‚{\tiny $_{lb}$}‚नास्ति (।) य‚च्चास्ति तेन व‚स्तुरूपाव‚भासिनि ज्ञाने प्र‚तिभासित‚व्य‚मेव । अन्य‚था ज्ञान‚{\tiny $_{lb}$}‚ज्ञेय‚योर्व्विष‚य‚विष‚यितैव न स्यात् (।) त‚स्मान्नाध्य‚क्षात् कार्य‚कार‚ण‚ताग्र‚हः । अत‚श्च ‚{\tiny $_{lb}$}‚नानुमानाद‚पि । न हि स‚र्व्व‚दा प‚रोक्षेऽन्येऽनुमान‚वृत्तिः व्याप्तिग्र‚ह‚ण‚पूर्व्व‚क‚त्वात्त‚स्य । ‚{\tiny $_{lb}$}‚नापि प्र‚त्य‚क्ष‚ब‚ल‚भाविस्म‚र‚णं त‚द्ग्र‚ह‚ण‚प्र‚व‚णं(।)न हि त‚त्स्व‚त‚न्त्रं प्र‚माणं किन्तु प्र‚माण‚{\tiny $_{lb}$}‚\leavevmode\ledsidenote{\textenglish{114/s}} व्यापार‚व्य‚व‚स्थाप‚कं । य‚दि य‚थानुभ‚वं प्र‚व‚र्त‚ते नान्य‚था । न च कार्य‚{\tiny $_{1}$}‚कार‚ण‚भावा‚{\tiny $_{lb}$}‚नुभ‚वो भूत इत्युक्तं । अतः स्म‚र‚ण‚त्व‚म‚प्य‚स्य नास्ति । अनूभूत‚विक‚ल्प‚न‚स्य स्म‚र‚ण‚{\tiny $_{lb}$}‚त्वात् (।) त‚तो विक‚ल्प‚मात्र‚मेत‚त् न त‚तो व‚स्तुव्य‚व‚स्थेति संवृत्त्यैवाविचारित‚र‚म‚णी‚{\tiny $_{lb}$}‚य‚या कार्य‚कार‚ण‚भाव‚व्य‚व‚हारो न प‚र‚मार्थ‚तः ।
	\pend% ending standard par
      
	  \bigskip
	  \begingroup
	
	    \large
	  
	    \begin{quote}
	  
	    
	    \stanza[\smallbreak]
	\label{pv.2.4c}\flagstanza{\tiny\textenglish{...v.2.4c}}अस्तु य‚था त‚था ॥ ४ ॥\&[\smallbreak]


	
	    \end{quote}
	  
	  \endgroup
	

	  \pstart \leavevmode% starting standard par
	\hphantom{.}‚{\color{DodgerBlue3}‚अस्तु\edtext{}{\edlabel{pvv.114-1}\label{pvv.114-1}\lemma{अस्तु}\Bfootnote{सिद्धान्ती(---)अर्थ‚क्रिया ताव‚द‚स्ति । अन्य‚था ज‚ग‚द्व्यालोपः ।}} य‚था त‚था} । सांवृत‚म‚पि कार्य‚कार‚ण‚भाव‚माश्रित्य साध्य‚साध‚नादि‚{\tiny $_{lb}$}‚व्य‚व‚हार‚स‚म्वाद‚स‚म्प्र‚त्य‚यात् स‚माप्तो लोक‚व्य‚व‚हारः । सांव्य‚व‚हारिक‚ञ्च प्र‚माणं ‚{\tiny $_{lb}$}‚ताव‚तैव सुस्थं । (४)
	\pend% ending standard par
      \label{div_pvv.2.5}
	  
	% new div opening: depth here is 2
	

	  \begin{center}%% label @type='head'
	\textbf{(१) सामान्य‚चिन्ता}
	\end{center}
	

	  \begin{center}%% label @type='head'
	\textbf{(क) सामान्ये नार्थ‚क्रियासाम‚र्थ्य‚म्}
	\end{center}
	
	  \bigskip
	  \begingroup
	
	    \large
	  
	    \begin{quote}
	  
	    
	    \stanza[\smallbreak]
	\label{pv.2.5}\flagstanza{\tiny\textenglish{...pv.2.5}}सास्ति स‚र्व‚त्र चेद् बुद्धेर्नान्व‚य‚व्य‚तिरेक‚योः ।&सामान्य‚ल‚क्ष‚णेऽदृष्टेः च‚क्षूरूपादिबुद्धिव‚त् ॥ ५ ॥\&[\smallbreak]


	
	    \end{quote}
	  
	  \endgroup
	

	  \pstart \leavevmode% starting standard par
	\hphantom{.}‚{\color{DodgerBlue3}‚सार्थ}‚क्रिया ‚{\color{DodgerBlue3}‚स‚{\tiny $_{2}$}‚र्व्व‚त्र} स्व‚ल‚क्ष‚णे सामा\edtext{}{\edlabel{pvv.114-2}\label{pvv.114-2}\lemma{सामा}\Bfootnote{च‚क्षुःश्रोत्रादिरूप‚श‚ब्दादि ।}}न्येप्य‚{\color{DodgerBlue3}‚स्ति चेत् न} सामान्य‚ल‚क्ष‚णे । ‚{\tiny $_{lb}$}‚अस्तु ताव‚द‚न्य‚स्य कार्य‚स्य ‚{\color{DodgerBlue3}‚बुद्धेर‚प्य‚न्व‚य‚व्य‚तिरेक}‚योर‚दृष्टेः । अन्त्यं हि भावानां ‚{\tiny $_{lb}$}‚कार्यं बुद्धिः । सापि य‚द‚न्व‚य‚व्य‚तिरेकौ नानुविध‚त्ते (।) त‚स्य कुतोऽर्थ‚क्रिया । प्र‚त्य‚क्षं ‚{\tiny $_{lb}$}‚ताव‚न्न ‚{\color{DodgerBlue3}‚सामान्य}‚विष‚यं ‚{\color{DodgerBlue3}‚स्व‚ल‚क्ष‚ण}‚मात्र‚स्य प्र‚तिभास‚नात् । घ‚ट इत्यादिविक‚ल्प‚बुद्धिस्तु ‚{\tiny $_{lb}$}‚सामान्याव‚सायात् त\edtext{}{\edlabel{pvv.114-3}\label{pvv.114-3}\lemma{त}\Bfootnote{सामान्य‚विष‚या ।}}द्विष‚या । सापि स‚म‚याभो\edtext{}{\edlabel{pvv.114-4}\label{pvv.114-4}\lemma{याभो}\Bfootnote{चित्ताभोगो म‚न‚स्कारः ।}}गादिमात्रादुत्प‚त्तेर्न सामान्यान्व‚य‚{\tiny $_{lb}$}‚व्य‚तिरेकानुविधायिनी‚{\tiny $_{3}$}‚ । ‚{\color{DodgerBlue3}‚च‚क्षूरूपादिबुद्धिव‚त्} च‚क्षूरूपादिज‚बुद्धेः । च‚क्षुराद्य‚न्व‚य‚{\tiny $_{lb}$}‚व्य‚तिरेकानुविधान‚मिव साम‚र्थ्ये स‚ति भावादेकापायेप्य‚भावाच्च । नैवं सामान्य‚{\tiny $_{lb}$}‚बुद्धिराभोग‚मात्रादुत्प‚त्तेः । (५)
	\pend% ending standard par
      \label{div_pvv.2.6}
	  
	% new div opening: depth here is 2
	
	  \bigskip
	  \begingroup
	
	    \large
	  
	    \begin{quote}
	  
	    
	    \stanza[\smallbreak]
	\label{pv.2.6}\flagstanza{\tiny\textenglish{...pv.2.6}}एतेन स‚म‚याभोगाद्य‚न्त‚र‚ङ्गानुरोध‚तः ।&घ‚टोत्क्षेप‚ण‚सामान्य‚संख्यादिषु धियो ग‚ताः ॥ ६ ॥\&[\smallbreak]


	
	    \end{quote}
	  
	  \endgroup
	

	  \pstart \leavevmode% starting standard par
	\hphantom{.}‚{\color{DodgerBlue3}‚एतेन} सामान्य‚स्य ज्ञान‚मात्र‚कार्येप्य‚साम‚र्थ्य‚क‚थ‚नेन ‚{\color{DodgerBlue3}‚स‚म‚याभोगादे}‚र‚न्त‚र‚ङ्ग‚स्या‚{\tiny $_{lb}$}‚‚{\color{DodgerBlue3}‚नुरोध‚तः । घ‚टो}‚ऽव‚य‚वि द्र‚व्यं । ‚{\color{DodgerBlue3}‚उत्क्षेप‚णं} क‚र्म । ‚{\color{DodgerBlue3}‚सामान्यं संख्या} गुणः स‚म‚वायः ‚{\color{DodgerBlue3}‚आदि} श‚ब्दात्संयोग‚विभागाद‚यः । तेषु या ‚{\color{DodgerBlue3}‚धियो} विक‚ल्पिका भ‚व‚न्ति ता ‚{\color{DodgerBlue3}‚ग‚ता} व्याख्याता‚{\tiny $_{4}$}‚ः । ‚{\tiny $_{lb}$}‚\leavevmode\ledsidenote{\textenglish{115/s}} न हि रूपा\edtext{}{\edlabel{pvv.115-1}\label{pvv.115-1}\lemma{रूपा}\Bfootnote{न‚नु प्र‚त्य‚क्ष‚ग‚म्य‚मेवोत्क्षेप‚णादीत्याह (।)}}दिव्य‚तिरिक्तं द्र‚व्यं । अप‚राप‚र‚देशिज‚न्य‚ह‚स्तादिक्ष‚णाद् भिन्नं क‚र्म, ‚{\tiny $_{lb}$}‚व्य‚क्तिव्य‚तिरेकि सामान्यं प्र‚त्य‚क्ष‚बुद्धाव‚भास‚ते । विक‚ल्प‚बुद्धिर‚स्तु क‚ल्पिका । सा च\edtext{}{\edlabel{pvv.115-2}\label{pvv.115-2}\lemma{च}\Bfootnote{स‚म‚याभोगं व्याच‚ष्टे ।}} ‚{\tiny $_{lb}$}‚संकेत‚सँस्कार‚प्र‚बोध‚मात्र‚भाविनी नार्थाधीनेति न ते\edtext{}{\edlabel{pvv.115-3}\label{pvv.115-3}\lemma{ते}\Bfootnote{अर्थानाम‚नुमाज‚न‚ने ।}}षां साम‚र्थ्यं स‚म‚र्थ‚य‚ति ॥ (६)
	\pend% ending standard par
      \label{div_pvv.2.7}
	  
	% new div opening: depth here is 2
	

	  \pstart \leavevmode% starting standard par
	न‚नु य‚द्य‚र्थ‚निर‚पेक्षाणाम‚न्य‚निमित्तानां भावे य‚त्र बुद्धिस्त‚त्सामान्यं त‚दा ‚{\tiny $_{lb}$}‚तैमिरिक‚बुद्धिप्र‚तिभासिनः केशाद‚योपि सामान्यं स्तुः । तेष्व‚पि धियोऽर्थ‚निर‚पेक्ष‚{\tiny $_{lb}$}‚च‚क्षुरादिमात्र‚निमि‚{\tiny $_{5}$}‚त्त‚त्वादित्याह ।
	\pend% ending standard par
      
	  \bigskip
	  \begingroup
	
	    \large
	  
	    \begin{quote}
	  
	    
	    \stanza[\smallbreak]
	\label{pv.2.7a}\flagstanza{\tiny\textenglish{...v.2.7a}}केशाद‚यो न सामान्य‚म‚न‚र्थाभिनिवेश‚तः ॥\&[\smallbreak]


	
	    \end{quote}
	  
	  \endgroup
	

	  \pstart \leavevmode% starting standard par
	\hphantom{.}‚{\color{DodgerBlue3}‚केशाद‚य}‚स्तैमिरिक‚प्र‚तिभासिनो ‚{\color{DodgerBlue3}‚न सामान्य‚म‚न‚र्थाभिनिवेश\edtext{}{\edlabel{pvv.115-4}\label{pvv.115-4}\lemma{र्थाभिनिवेश}\Bfootnote{प्र‚त्य‚क्ष‚विष‚येप्य‚र्थाभिनिवेश‚मात्रं नार्थ इति सूच‚न‚या म‚ध्याप्र‚वेश‚माह ।}}तः} । न हि तेषु ‚{\tiny $_{lb}$}‚विष‚य‚बुद्धिर्व्य‚व‚हारिणां विष‚ये च सामान्य‚चिन्तेयं । य‚दि ज्ञेय‚त्वेनाध्य‚व‚सायाभावान्न ‚{\tiny $_{lb}$}‚सामान्यं त‚दाऽभावोप्य‚नुप‚ल‚ब्धिविष‚यः ‚{\color{DodgerBlue3}‚सामान्यं न स्यात्} ।---
	\pend% ending standard par
      

	  \pstart \leavevmode% starting standard par
	त‚त‚श्चानुप‚ल‚ब्धिर‚नुमानं न भ‚वेदित्याह (।)
	\pend% ending standard par
      
	  \bigskip
	  \begingroup
	
	    \large
	  
	    \begin{quote}
	  
	    
	    \stanza[\smallbreak]
	\label{pv.2.7b}\flagstanza{\tiny\textenglish{...v.2.7b}}ज्ञेय‚त्वेन ग्र‚हाद् दोषो नाभावेषु प्र‚स‚ज्य‚ते ॥ ७ ॥\&[\smallbreak]


	
	    \end{quote}
	  
	  \endgroup
	

	  \pstart \leavevmode% starting standard par
	\hphantom{.}‚{\color{DodgerBlue3}‚नाभावेषु} सामान्य‚रूप‚त्वाभाव‚{\color{DodgerBlue3}‚दोषः प्र‚स‚ज्य‚ते ज्ञेय‚त्वेन ग्र‚हात्} । य‚द्य‚प्य‚भावेऽर्थ‚{\tiny $_{lb}$}‚रूप‚ताया अध्य‚व‚सायो नास्ति ज्ञे\edtext{}{\edlabel{pvv.115-5}\label{pvv.115-5}\lemma{ज्ञे}\Bfootnote{भाव‚व्यावृत्तिः ।}}य‚त‚या त्व‚स्ति लोक‚स्य (।) ‚{\color{DodgerBlue3}‚त‚च्च ज्ञे‚{\tiny $_{6}$}‚यं} य‚द‚र्थ‚कारि ‚{\tiny $_{lb}$}‚त‚त्सामान्य‚मेव । (७)
	\pend% ending standard par
      \label{div_pvv.2.8_2.9}
	  
	% new div opening: depth here is 2
	

	  \pstart \leavevmode% starting standard par
	य‚दि तु (।)
	\pend% ending standard par
      
	  \bigskip
	  \begingroup
	
	    \large
	  
	    \begin{quote}
	  
	    
	    \stanza[\smallbreak]
	\label{pv.2.8a}\flagstanza{\tiny\textenglish{...v.2.8a}}तेषाम‚पि त‚थाभावेऽप्र‚तिषेधात्;\&[\smallbreak]


	
	    \end{quote}
	  
	  \endgroup
	

	  \pstart \leavevmode% starting standard par
	\hphantom{.}‚{\color{DodgerBlue3}‚तेषां} तैमिरिक‚ग‚म्यानां केशादीना‚{\color{DodgerBlue3}‚म‚पि त‚थाभावे} ज्ञानान्त‚रेण ज्ञेय‚त्वे ‚{\tiny $_{lb}$}‚स‚मान‚रूप‚तेष्य‚ते त‚दाऽ‚{\color{DodgerBlue3}‚प्र‚तिषेधादि}‚ष्ट‚मेवास्माकं । तैमिरिक‚ज्ञान‚ग‚म्याः केशाद‚य ‚{\tiny $_{lb}$}‚इति य‚दा बुद्ध्य‚न्त‚रेण प‚रामृश्य‚न्ते त‚दा केशाद‚यो ज्ञेय‚त्वेन सामान्य‚मिष्टा एव ॥
	\pend% ending standard par
      

	  \pstart \leavevmode% starting standard par
	\hphantom{.}य‚दि तैमिरिक‚दृष्टाः केशाद‚यो न व‚स्तु त‚दा ‚{\color{DodgerBlue3}‚स्फुटाभ‚ता} न स्यादित्याह ।
	\pend% ending standard par
      
	  \bigskip
	  \begingroup
	
	    \large
	  
	    \begin{quote}
	  
	    
	    \stanza[\smallbreak]
	\label{pv.2.8b}\flagstanza{\tiny\textenglish{...v.2.8b}}स्फुटाभ‚ता ।&ज्ञान‚रूप‚त‚यार्थ‚त्वात्;\&[\smallbreak]


	
	    \end{quote}
	  
	  \endgroup
	

	  \pstart \leavevmode% starting standard par
	\hphantom{.}‚{\color{DodgerBlue3}‚ज्ञान‚रूप‚त‚या} ज्ञानाकार‚त‚या केशादीनाम‚र्थ‚त्वात्।‚{\tiny $_{7}$}‚ स्व‚ल‚क्ष‚ण‚त्वात्स्फुटाभ‚ता ॥
	\pend% ending standard par
      \textsuperscript{\textenglish{22a/MA}}‚{\tiny $_{lb}$}‚

	  \pstart \leavevmode% starting standard par
	य‚दि ज्ञानाकार‚त‚या केशाद‚यः स्व‚ल‚क्ष‚ण‚न्त‚दा सामान्यं क‚थ‚मित्याह (।)
	\pend% ending standard par
      \textsuperscript{\textenglish{116/s}}
	  \bigskip
	  \begingroup
	
	    \large
	  
	    \begin{quote}
	  
	    
	    \stanza[\smallbreak]
	\label{pv.2.8c}\flagstanza{\tiny\textenglish{...v.2.8c}}केशादीति म‚तिः पुनः ॥ ८ ॥\&[\smallbreak]


	
	    \end{quote}
	  
	  \endgroup
	
	  \bigskip
	  \begingroup
	
	    \large
	  
	    \begin{quote}
	  
	    
	    \stanza[\smallbreak]
	\label{pv.2.9a}\flagstanza{\tiny\textenglish{...v.2.9a}}सामान्य‚विष‚या;\&[\smallbreak]


	
	    \end{quote}
	  
	  \endgroup
	

	  \pstart \leavevmode% starting standard par
	\hphantom{.}‚{\color{DodgerBlue3}‚केशादीति पुन}‚र्व्विक‚ल्पिका ‚{\color{DodgerBlue3}‚म‚तिः} तैमिरिकोप‚ल‚ब्ध‚केशाद्य‚ध्य‚व‚सायिनी ‚{\tiny $_{lb}$}‚‚{\color{DodgerBlue3}‚सामान्य‚विष‚या} । अय‚म‚र्थः (।) तैमिरिक‚दृष्टं केशादि सामान्य‚रूपेण व्य‚व‚स्य‚न्ती ‚{\tiny $_{lb}$}‚विक‚ल्पिका बुद्धिर‚ध्य‚व‚सायेन विष‚येण सामान्य‚विष‚या । तैमिरिक‚धीस्तु स्व‚ल‚क्ष‚ण‚{\tiny $_{lb}$}‚विष‚या बुद्ध्याकार‚स्य स्व‚ल‚क्ष‚ण‚त्वात् ॥ (८)
	\pend% ending standard par
      

	  \pstart \leavevmode% starting standard par
	किन्त‚र्हि निर्व्विष‚य‚मित्याह (।)
	\pend% ending standard par
      
	  \bigskip
	  \begingroup
	
	    \large
	  
	    \begin{quote}
	  
	    
	    \stanza[\smallbreak]
	\label{pv.2.9b}\flagstanza{\tiny\textenglish{...v.2.9b}}केश‚प्र‚तिभास‚म‚न‚र्थ‚क‚म् ।\&[\smallbreak]


	
	    \end{quote}
	  
	  \endgroup
	

	  \pstart \leavevmode% starting standard par
	\hphantom{.}तैमिरिक‚{\color{DodgerBlue3}‚केश‚प्र‚तिभासं} ज्ञान‚{\color{DodgerBlue3}‚म‚न‚र्थ}‚क‚म्बाह्य‚केशाभावात् ।
	\pend% ending standard par
      
	  \bigskip
	  \begingroup
	
	    \large
	  
	    \begin{quote}
	  
	    
	    \stanza[\smallbreak]
	\label{pv.2.9c}\flagstanza{\tiny\textenglish{...v.2.9c}}ज्ञान‚रूप‚त‚यार्थ‚त्वे सामान्ये चेत् प्र‚स‚ज्य‚ते ॥ ९ ॥\&[\smallbreak]


	
	    \end{quote}
	  
	  \endgroup
	

	  \pstart \leavevmode% starting standard par
	\hphantom{.}न‚नु ‚{\color{DodgerBlue3}‚ज्ञान‚रूप‚त‚यार्थ‚त्वे} केशादीति विक‚ल्प‚बुद्धिप्र‚तिभासिनि ‚{\color{DodgerBlue3}‚सामान्ये} स्व‚ल‚क्ष‚ण‚ता ‚{\tiny $_{lb}$}‚‚{\color{DodgerBlue3}‚प्र‚स‚ज्य‚ते चेत्} (। ९)
	\pend% ending standard par
      \label{div_pvv.2.10}
	  
	% new div opening: depth here is 2
	
	  \bigskip
	  \begingroup
	
	    \large
	  
	    \begin{quote}
	  
	    
	    \stanza[\smallbreak]
	\label{pv.2.10a}\flagstanza{\tiny\textenglish{....2.10a}}त‚थेष्ट‚त्वाद‚दोषः;\&[\smallbreak]


	
	    \end{quote}
	  
	  \endgroup
	

	  \pstart \leavevmode% starting standard par
	\hphantom{.}त‚था ज्ञानाकार‚त‚या सामान्य‚स्यापि स्व‚ल‚क्ष‚ण‚ताया ‚{\color{DodgerBlue3}‚इष्ट‚त्वाद‚दोषः} ॥
	\pend% ending standard par
      

	  \begin{center}%% label @type='head'
	\textbf{(ख) सामान्य‚स्य ज्ञानाकार‚ता}
	\end{center}
	

	  \pstart \leavevmode% starting standard par
	क‚थ‚न्त‚र्हि सामान्य‚रूप‚तेत्याह (।)
	\pend% ending standard par
      
	  \bigskip
	  \begingroup
	
	    \large
	  
	    \begin{quote}
	  
	    
	    \stanza[\smallbreak]
	\label{pv.2.10b}\flagstanza{\tiny\textenglish{....2.10b}}अर्थ‚रूप‚त्वेन स‚मान‚ता ।\&[\smallbreak]


	
	    \end{quote}
	  
	  \endgroup
	

	  \pstart \leavevmode% starting standard par
	अर्थ‚रूप‚त्वेनाध्य‚व‚सीय‚मान‚ज्ञेय‚रूप‚त्वेन स‚मान‚ता न तु बुद्ध्याकार‚त्वेन ।
	\pend% ending standard par
      

	  \begin{center}%% label @type='head'
	\textbf{(ग) विजातीय‚व्याव‚र्त‚कं सामान्य‚म्}
	\end{center}
	

	  \pstart \leavevmode% starting standard par
	\hphantom{.}‚{\color{DodgerBlue3}‚क‚थं पुन‚र‚र्थ‚रूप‚त्वे}‚नापि ‚{\color{DodgerBlue3}‚स‚मान‚ते}‚त्याह (।)
	\pend% ending standard par
      
	  \bigskip
	  \begingroup
	
	    \large
	  
	    \begin{quote}
	  
	    
	    \stanza[\smallbreak]
	\label{pv.2.10c}\flagstanza{\tiny\textenglish{....2.10c}}स‚र्व‚त्र स‚म‚रूप‚त्वात् त‚द्व्यावृत्तिस‚माश्र‚यात् ॥ १० ॥\&[\smallbreak]


	
	    \end{quote}
	  
	  \endgroup
	

	  \pstart \leavevmode% starting standard par
	\hphantom{.}‚{\color{DodgerBlue3}‚स‚र्व्व‚त्र} व्य‚क्तिषु ‚{\color{DodgerBlue3}‚त‚द्व्यावृत्तिस‚माश्र‚यात्} विजातीय‚व्यावृत्त्याश्र‚येण दृश्य‚{\tiny $_{lb}$}‚विक‚ल्प्यैक‚त्वाध्य‚व‚सायादुप‚क‚ल्पित‚स्य सा‚{\tiny $_{2}$}‚मान्य‚स्य ‚{\color{DodgerBlue3}‚स‚म‚रूप‚त्वात्} साधार‚ण‚रूप‚त्वा‚{\tiny $_{lb}$}‚द‚र्थ‚त्वेन सामान्य‚मुच्य‚ते ज्ञानाकार‚स्य सामान्य‚स्य संप्र‚ति व्य‚तिरिक्त‚स्य\edtext{}{\edlabel{pvv.116-1}\label{pvv.116-1}\lemma{स्य}\Bfootnote{ज्ञानाकारात् ।}}च पूर्व्वं ‚{\tiny $_{lb}$}‚प्र‚तिषेधात् ॥ (१०)
	\pend% ending standard par
      \textsuperscript{\textenglish{117/s}}\label{div_pvv.2.11}
	  
	% new div opening: depth here is 2
	

	  \pstart \leavevmode% starting standard par
	रूपादिस्व‚भाव‚मेवाविशेषेण सामान्यं किन्नेष्य‚त इत्याह (।)
	\pend% ending standard par
      
	  \bigskip
	  \begingroup
	
	    \large
	  
	    \begin{quote}
	  
	    
	    \stanza[\smallbreak]
	\label{pv.2.11a}\flagstanza{\tiny\textenglish{....2.11a}}न त‚द्व‚स्त्व‚भिधेय‚त्वात् साफ‚ल्याद‚क्ष‚संह‚तेः ॥\&[\smallbreak]


	
	    \end{quote}
	  
	  \endgroup
	

	  \pstart \leavevmode% starting standard par
	\hphantom{.}त‚त्सामान्यं न ‚{\color{DodgerBlue3}‚व‚स्तुरू}‚पादिस्व‚भाव‚{\color{DodgerBlue3}‚म‚भिधेय‚त्वात्} । श‚ब्द\edtext{}{\edlabel{pvv.117-1}\label{pvv.117-1}\lemma{ब्द}\Bfootnote{श‚ब्द‚स्याय‚न्ध‚र्मो य‚दुत श‚ब्द‚ज्ञान‚गोच‚र‚त्वं ।}}ध‚र्म‚व‚त् ‚{\color{DodgerBlue3}‚सामान्यं} श‚ब्द‚ज्ञान‚गोच‚रः । न च श‚ब्द‚विष‚यो व‚स्तु (।) क‚स्मादित्याह (।) ‚{\color{DodgerBlue3}‚साफ‚ल्याद‚क्ष‚संह‚तेः} । ‚{\tiny $_{lb}$}‚य‚दि श‚ब्द‚विष‚यो व‚स्तु भ‚वेत् त‚दा रूपादिश‚ब्दादेवार्थादेर‚{\tiny $_{3}$}‚पि रूपादिप्र‚तीतौ न ‚{\tiny $_{lb}$}‚किञ्चिद‚क्षैः । न चैवं । त‚तो व‚स्तुविष‚येणेन्द्रिय‚ज्ञानेन श‚ब्द‚स्य न तुल्य‚विष‚य‚ता । ‚{\tiny $_{lb}$}‚त‚था चाव‚स्तुतैव शाब्द‚ज्ञानाव‚भासिनः सामान्य‚स्य न रूपादिता ॥
	\pend% ending standard par
      

	  \pstart \leavevmode% starting standard par
	न\edtext{}{\edlabel{pvv.117-2}\label{pvv.117-2}\lemma{न}\Bfootnote{वैभाषिक‚म‚त‚माश‚ङ्क‚ते ।}}नु रूपाद‚यो न श‚ब्द‚स्य विष‚यः किन्त‚र्हि ना\edtext{}{\edlabel{pvv.117-3}\label{pvv.117-3}\lemma{ना}\Bfootnote{विज्ञान‚वाद्य‚तिरिक्ते ।}}म‚निमित्ते विप्र‚युक्त‚संस्कार‚संज्ञे ‚{\tiny $_{lb}$}‚ते विज्ञानाद् व्य‚तिरिक्ते वै भा षि क स्येष्टे । त‚दापि (।)
	\pend% ending standard par
      
	  \bigskip
	  \begingroup
	
	    \large
	  
	    \begin{quote}
	  
	    
	    \stanza[\smallbreak]
	\label{pv.2.11b}\flagstanza{\tiny\textenglish{....2.11b}}नामादिव‚च‚ने व‚क्तृश्रोतृवाच्यानुब‚न्धिनि ॥ ११ ॥\&[\smallbreak]


	
	    \end{quote}
	  
	  \endgroup
	

	  \pstart \leavevmode% starting standard par
	\hphantom{.}‚{\color{DodgerBlue3}‚नामादिव‚च‚न} इष्य‚माणे किन्त‚न्नामादिकं व‚क्त‚रि श्रोत‚र्य‚र्थे वा ‚{\color{DodgerBlue3}‚स‚म्ब‚द्ध‚म-} स‚म्ब‚द्ध‚मेव वा क्व‚चित्\edtext{}{\edlabel{pvv.117-4}\label{pvv.117-4}\lemma{चित्}\Bfootnote{इति च‚त्वारो विक‚ल्पाः ।}} । स‚र्व्व‚था ‚{\color{DodgerBlue3}‚व‚क्तृ‚{\tiny $_{4}$}‚श्रोतृवाच्यानुब‚न्धिनि}\edtext{}{\edlabel{pvv.117-5}\label{pvv.117-5}\lemma{था}\Bfootnote{व‚क्तृस‚म्ब‚न्धिनि । श्रोतृस‚म्ब‚न्धिनि । अर्थ‚स‚म्ब‚न्धिनि ।}}। (११)
	\pend% ending standard par
      \label{div_pvv.2.12}
	  
	% new div opening: depth here is 2
	
	  \bigskip
	  \begingroup
	
	    \large
	  
	    \begin{quote}
	  
	    
	    \stanza[\smallbreak]
	\label{pv.2.12a}\flagstanza{\tiny\textenglish{....2.12a}}अस‚म्ब‚न्धिनि नामादाव‚र्थे स्याद‚प्र‚व‚र्त्त‚न‚म् ॥\&[\smallbreak]


	
	    \end{quote}
	  
	  \endgroup
	

	  \pstart \leavevmode% starting standard par
	\hphantom{.}‚{\color{DodgerBlue3}‚अस‚म्ब‚न्धिनि वा नामादौ} श‚ब्देन चोदिते‚{\color{DodgerBlue3}‚ऽर्थेऽप्र‚व‚र्त‚नं स्याद}‚चोदित‚त्वात् ॥\edtext{\textsuperscript{*}}{\edlabel{pvv.117-6}\label{pvv.117-6}\lemma{*}\Bfootnote{वै॰ ।}}
	\pend% ending standard par
      

	  \pstart \leavevmode% starting standard par
	भ‚व‚तु ताव‚द‚न्य‚स‚म्ब‚न्धिनि अस‚म्ब‚न्धिनि वा नामादाव‚र्थेऽप्र‚व‚र्त‚नं । अर्थ‚{\tiny $_{lb}$}‚स‚म्ब‚न्धिनि तु क‚थ‚म‚प्र‚वृत्तिः ।\edtext{\textsuperscript{*}}{\edlabel{pvv.117-7}\label{pvv.117-7}\lemma{*}\Bfootnote{सिद्धान्ती, अर्थ‚स्य श‚ब्देनाचोदित‚त्वात् ।}} अचोदित‚त्वात् ।
	\pend% ending standard par
      
	  \bigskip
	  \begingroup
	
	    \large
	  
	    \begin{quote}
	  
	    
	    \stanza[\smallbreak]
	\label{pv.2.12b}\flagstanza{\tiny\textenglish{....2.12b}}सारूप्याद् भ्रान्तितो वृत्तिर‚र्थे चेत् स्यान्न स‚र्व‚दा ॥ १२ ॥\&[\smallbreak]


	
	    \end{quote}
	  
	  \endgroup
	
	  \bigskip
	  \begingroup
	
	    \large
	  
	    \begin{quote}
	  
	    
	    \stanza[\smallbreak]
	\label{pv.2.13a}\flagstanza{\tiny\textenglish{....2.13a}}देश‚भ्रान्तिश्च,\&[\smallbreak]


	
	    \end{quote}
	  
	  \endgroup
	

	  \pstart \leavevmode% starting standard par
	\hphantom{.}न हि देव‚द‚त्ते प्र‚तिपादिते त‚त्पित‚रि प्र‚वृत्तिः ॥ निमित्त‚स्यार्थ‚{\color{DodgerBlue3}‚सारूप्यात् । ‚{\tiny $_{lb}$}‚त‚द्भ्रान्तितोऽर्थे} प्र‚वृत्ति‚{\color{DodgerBlue3}‚श्चेत्} स‚म्भाव्य‚त एत‚त् किन्तु ‚{\color{DodgerBlue3}‚न स्यात् स‚र्व्व‚दा} । न हि ‚{\tiny $_{lb}$}‚य‚म‚ल‚क‚योर्निय‚मेन भ्रान्त्याऽ\edtext{}{\edlabel{pvv.117-8}\label{pvv.117-8}\lemma{भ्रान्त्याऽ}\Bfootnote{याव‚ताऽयं पुरुषः स‚र्व्व‚दार्थ एव दृष्टः श‚ब्दात्प्र‚व‚र्त‚मानः । न क‚दाचिन्नास्ति व‚क्त्रादिस‚म्ब‚द्धे । त‚न्न भ्रान्त्या वृत्तिः ।}} न्य‚त्र प्र‚वृत्तिः । क‚दाचित्त‚त्रापि द‚र्श‚{\tiny $_{5}$}‚नात् । त‚था\edtext{}{\edlabel{pvv.117-9}\label{pvv.117-9}\lemma{था}\Bfootnote{देश‚भ्रान्त्यार्थे वृत्तिरिति चेत् ।}} ‚{\tiny $_{lb}$}‚\leavevmode\ledsidenote{\textenglish{118/s}} ‚{\color{DodgerBlue3}‚देश‚भ्रान्तिश्च न} स्यात् । व‚क्तृश्रोत्रादिस‚म्ब‚न्धिनि निय‚त‚देशे नामादौ प्र‚तिपादिते ‚{\tiny $_{lb}$}‚त‚द‚न्य‚देशे घ‚टादौ सारूप्याद‚पि न यु\edtext{}{\edlabel{pvv.118-1}\label{pvv.118-1}\lemma{यु}\Bfootnote{त‚थाहि त्व‚येदं क‚र‚णीय‚मिति नियुक्तः पुमान् सारूप्याद‚न्य‚त्र न प्र‚व‚र्त‚त एवेति दृष्टेः ।}} क्ता प्र‚वृत्तिः ॥ (१२)
	\pend% ending standard par
      \label{div_pvv.2.13}
	  
	% new div opening: depth here is 2
	

	  \pstart \leavevmode% starting standard par
	न‚नु त्व‚न्म‚तेऽपि ज्ञानाकार‚स्याभिधेय‚त्वात्क‚थं बाह्ये प्र‚वृत्तिरित्याह (।)
	\pend% ending standard par
      
	  \bigskip
	  \begingroup
	
	    \large
	  
	    \begin{quote}
	  
	    
	    \stanza[\smallbreak]
	\label{pv.2.13b}\flagstanza{\tiny\textenglish{....2.13b}}न ज्ञाने तुल्य‚मुत्प‚त्तितो धियः ।&त‚थाविधायाः;\&[\smallbreak]


	
	    \end{quote}
	  
	  \endgroup
	

	  \pstart \leavevmode% starting standard par
	\hphantom{.}‚{\color{DodgerBlue3}‚न ज्ञाने} ज्ञानाकारे वाच्येऽर्थेऽप्र‚व‚र्त‚नं ‚{\color{DodgerBlue3}‚तुल्यं} । धिय‚स्त‚{\color{DodgerBlue3}‚थाविधाया} ब‚हिस्त्वेनाध्य‚व‚सिताकाराया ‚{\color{DodgerBlue3}‚उत्प‚त्तितः} । श‚ब्द‚ज‚निता हि बुद्धिर्व्व‚स्तुतः ‚{\tiny $_{lb}$}‚स्वांशाल‚म्ब‚नाप्य‚नाद्य‚विद्याव‚शाद् ब‚हिर्व्विष‚या व्य\edtext{}{\edlabel{pvv.118-2}\label{pvv.118-2}\lemma{व्य}\Bfootnote{दृश्य‚विक‚ल्प‚योरैक्येन वृत्तेः ।}}व‚सीय‚त इति‚{\tiny $_{6}$}‚युक्त‚म‚र्थे ‚{\tiny $_{lb}$}‚प्र‚व‚र्त‚नं ॥
	\pend% ending standard par
      
	  \bigskip
	  \begingroup
	
	    \large
	  
	    \begin{quote}
	  
	    
	    \stanza[\smallbreak]
	\label{pv.2.13c}\flagstanza{\tiny\textenglish{....2.13c}}अन्य‚त्र त‚त्रानुप‚ग‚माद् धियः ॥ १३ ॥\&[\smallbreak]


	
	    \end{quote}
	  
	  \endgroup
	

	  \pstart \leavevmode% starting standard par
	\hphantom{.}‚{\color{DodgerBlue3}‚अन्य‚त्र} नामादिविष\edtext{}{\edlabel{pvv.118-3}\label{pvv.118-3}\lemma{नामादिविष}\Bfootnote{य‚स्यापि न नामादिर्व्विप्र‚युक्तोऽभिधेयः ।}}यिणि ज्ञाने त‚द‚र्थाध्य‚व‚सायात् । अर्थे प्र‚व‚र्त‚नं न युक्तं । ‚{\tiny $_{lb}$}‚‚{\color{DodgerBlue3}‚निराकार}‚बुद्धिवादि ‚{\color{DodgerBlue3}‚वै भा षि क} म‚ते बाह्यार्थ‚प्र‚तिभासाया ज्ञानाकाराया धियोऽनुप‚{\tiny $_{lb}$}‚ग‚म‚तः । (१३)
	\pend% ending standard par
      \label{div_pvv.2.14}
	  
	% new div opening: depth here is 2
	
	  \bigskip
	  \begingroup
	
	    \large
	  
	    \begin{quote}
	  
	    
	    \stanza[\smallbreak]
	\label{pv.2.14a}\flagstanza{\tiny\textenglish{....2.14a}}बाह्यार्थ‚प्र‚तिभासाया उपाये वाऽप्र‚माणाता ।&विज्ञान‚व्य‚तिरिक्त‚स्य;\&[\smallbreak]


	
	    \end{quote}
	  
	  \endgroup
	

	  \pstart \leavevmode% starting standard par
	य‚दि ज्ञेयाकारा बुद्धिः स्यात् स्यात्त‚त्प्र‚तीत्याऽभिमानात्प्र‚वृत्तिर‚पि । अप्र‚वृत्ति‚{\tiny $_{lb}$}‚दोष‚द‚र्श‚नाद् ‚{\color{DodgerBlue3}‚बाह्यार्थ‚प्र‚तिभासाया} बुद्धेरु\edtext{}{\edlabel{pvv.118-4}\label{pvv.118-4}\lemma{बुद्धेरु}\Bfootnote{अभ्युप‚ग‚मे ।}}पाये स्वीकारे ‚{\color{DodgerBlue3}‚वाऽप्र‚माण‚ता} नामादेः ‚{\tiny $_{lb}$}‚विप्र‚युक्त‚संस्कार‚स्य ‚{\color{DodgerBlue3}‚विज्ञान‚व्य‚तिरिक्त‚स्य} (।)
	\pend% ending standard par
      

	  \pstart \leavevmode% starting standard par
	क‚थ‚मित्याह (।)
	\pend% ending standard par
      
	  \bigskip
	  \begingroup
	
	    \large
	  
	    \begin{quote}
	  
	    
	    \stanza[\smallbreak]
	\label{pv.2.14b}\flagstanza{\tiny\textenglish{....2.14b}}व्य‚तिरेकाप्र‚सिद्धितः ॥ १४ ॥\&[\smallbreak]


	
	    \end{quote}
	  
	  \endgroup
	\textsuperscript{\textenglish{22b/MA}}

	  \pstart \leavevmode% starting standard par
	\hphantom{.}‚{\color{DodgerBlue3}‚व्य‚तिरेकाप्र‚सिद्धितः} ।‚{\tiny $_{7}$}‚ प्र‚तिभास‚मान‚स्याकार‚स्य ज्ञान‚त्वात् । न चेन्द्रियादिषु ‚{\tiny $_{lb}$}‚स‚त्स्व‚पि विज्ञान‚कार्यानुत्प‚त्तितोऽर्थ इव नामादिर‚पि श‚क्य‚व्य‚व‚स्थानः । न हि ‚{\tiny $_{lb}$}‚गृहीत‚संकेत‚स्य श‚ब्द‚श्रुतौ क्व‚चिद‚र्थाप्र‚ति\edtext{}{\edlabel{pvv.118-5}\label{pvv.118-5}\lemma{ति}\Bfootnote{प‚रार्थानुमानेन ।}}प‚त्तिः ॥ (१४)
	\pend% ending standard par
      \label{div_pvv.2.15}
	  
	% new div opening: depth here is 2
	
	  \bigskip
	  \begingroup
	
	    \large
	  
	    \begin{quote}
	  
	    
	    \stanza[\smallbreak]
	\label{pv.2.15a}\flagstanza{\tiny\textenglish{....2.15a}}स‚र्व‚ज्ञानार्थ‚व‚त्वाच्चेत् स्व‚प्नादाव‚न्य‚थेक्ष‚णात् ।&अयुक्तं;\&[\smallbreak]


	
	    \end{quote}
	  
	  \endgroup
	\textsuperscript{\textenglish{119/s}}

	  \pstart \leavevmode% starting standard par
	\hphantom{.}न‚न्व‚स्ति प्र‚माणं ‚{\color{DodgerBlue3}‚स‚र्व्व}‚स्य ‚{\color{DodgerBlue3}‚ज्ञान}‚स्या‚{\color{DodgerBlue3}‚र्थ‚व‚त्त्वात्} । शाब्द‚म‚पि ज्ञान‚म‚र्थ‚व‚देव । न ‚{\tiny $_{lb}$}‚च रूपाद‚यो विष‚य इति पारिशेष्यान्नामादिक‚मेवेति चेत् । ‚{\color{DodgerBlue3}‚स्व‚प्नादावा}‚दिश‚ब्दात् ‚{\tiny $_{lb}$}‚तैमिरिक‚ज्ञानादिष्व‚{\color{DodgerBlue3}‚न्य‚था} अर्थ‚शून्य‚{\color{DodgerBlue3}‚स्येक्ष‚णात्} स\edtext{}{\edlabel{pvv.119-1}\label{pvv.119-1}\lemma{स}\Bfootnote{अनैकान्तिक ।}}र्व्व‚ज्ञानार्थ‚व‚च्छाब्द‚स्य‚{\tiny $_{1}$}‚ नामादि‚{\tiny $_{lb}$}‚विष‚य‚त्वानुमान‚म‚युक्तं । स्व‚प्न‚विज्ञान‚म‚पि निमित्त‚विष‚य‚मेवेति चेदाह (।)
	\pend% ending standard par
      
	  \bigskip
	  \begingroup
	
	    \large
	  
	    \begin{quote}
	  
	    
	    \stanza[\smallbreak]
	\label{pv.2.15b}\flagstanza{\tiny\textenglish{....2.15b}}न च संस्कारान्नीलादिप्र‚तिभास‚तः ॥ १५ ॥\&[\smallbreak]


	
	    \end{quote}
	  
	  \endgroup
	

	  \pstart \leavevmode% starting standard par
	\hphantom{.}न च ‚{\color{DodgerBlue3}‚संस्कारान्नि}‚मित्ताख्यात् स्व‚प्न‚ज्ञानं ‚{\color{DodgerBlue3}‚नीलादे}‚र्व्व‚र्ण‚संस्थान‚विशेष‚तः ‚{\color{DodgerBlue3}‚प्र‚ति‚{\tiny $_{lb}$}‚भास‚तः} । न च विप्र‚युक्त‚संस\edtext{}{\edlabel{pvv.119-2}\label{pvv.119-2}\lemma{संस}\Bfootnote{सुखादिव‚त् ।}}का/?/रो व‚र्ण्ण‚संस्थान‚विशेष‚वान् ॥ (१५)
	\pend% ending standard par
      \label{div_pvv.2.16}
	  
	% new div opening: depth here is 2
	

	  \pstart \leavevmode% starting standard par
	नीलादिरेवासाव‚र्थ इति चेत् (।)
	\pend% ending standard par
      
	  \bigskip
	  \begingroup
	
	    \large
	  
	    \begin{quote}
	  
	    
	    \stanza[\smallbreak]
	\label{pv.2.16a}\flagstanza{\tiny\textenglish{....2.16a}}नीलाद्य‚प्र‚तिघातान्न;\&[\smallbreak]


	
	    \end{quote}
	  
	  \endgroup
	

	  \pstart \leavevmode% starting standard par
	\hphantom{.}स्व‚प्न‚प्र‚तिभासि न ‚{\color{DodgerBlue3}‚नीलादि} व‚स्तु ‚{\color{DodgerBlue3}‚अप्र‚तिघातात्} । नीलाद‚यो ‚{\color{DodgerBlue3}‚ह्य‚र्थाः} स्व‚देशे ‚{\tiny $_{lb}$}‚प‚दार्थान्त‚र‚स्य व्याघात‚काः स्व‚प्नोप‚ल‚ब्धास्तु नैवं । पिहित‚द्वाराव‚व‚र‚कोद‚र‚सुप्त‚{\tiny $_{2}$}‚‚{\tiny $_{lb}$}‚स्थानान्त‚र‚ग‚म‚ना(त्) ह‚स्तियूथादिद‚र्श‚नात् ।
	\pend% ending standard par
      

	  \pstart \leavevmode% starting standard par
	किन्त‚र्हि त‚दित्याह (।)
	\pend% ending standard par
      
	  \bigskip
	  \begingroup
	
	    \large
	  
	    \begin{quote}
	  
	    
	    \stanza[\smallbreak]
	\label{pv.2.16b}\flagstanza{\tiny\textenglish{....2.16b}}ज्ञानं त‚द्योग्य‚देश‚कैः ।&अज्ञात‚स्य स्व‚यं ज्ञानात्;\&[\smallbreak]


	
	    \end{quote}
	  
	  \endgroup
	

	  \pstart \leavevmode% starting standard par
	\hphantom{.}‚{\color{DodgerBlue3}‚ज्ञानं त‚त्} नीलादि ‚{\color{DodgerBlue3}‚योग्य‚देश‚कैः} स्व‚प्नाय‚मा\edtext{}{\edlabel{pvv.119-3}\label{pvv.119-3}\lemma{मा}\Bfootnote{य‚दुत्पाद्य‚रूपं योगी प‚श्य‚ति न तेन व्य‚भिचार‚स्त‚त्रान्य‚स्यायोग्य‚त्वात् ।}}नेन स‚मान‚देशैः पुरुषान्त‚रै‚{\color{DodgerBlue3}‚र‚ज्ञा}‚{\tiny $_{lb}$}‚त‚स्य स्व‚यं ‚{\color{DodgerBlue3}‚ज्ञानात्} । स्व‚य‚म्वेद्य‚म‚सामान्य‚मिति ज्ञान‚ल‚क्ष‚णं । त‚च्च स्व‚प्न‚दृष्ट‚{\tiny $_{lb}$}‚नीलादिषु व्य‚क्तं ।
	\pend% ending standard par
      
	  \bigskip
	  \begingroup
	
	    \large
	  
	    \begin{quote}
	  
	    
	    \stanza[\smallbreak]
	\label{pv.2.16c}\flagstanza{\tiny\textenglish{....2.16c}}नामाद्येतेन व‚र्णित‚म् ॥ १६ ॥\&[\smallbreak]


	
	    \end{quote}
	  
	  \endgroup
	

	  \pstart \leavevmode% starting standard par
	\hphantom{.}‚{\color{DodgerBlue3}‚एतेन} स्व‚प्न‚दृष्ट‚स्य ज्ञान‚त्व‚साध‚नेन ‚{\color{DodgerBlue3}‚नामादि व‚र्ण्णितं} । नामादिक‚म‚पि ‚{\tiny $_{lb}$}‚त‚न्न भ‚व‚ति योग्य‚देश‚कैर‚ज्ञात‚स्य स्व‚यं ज्ञानात् ज्ञानं त‚त्\edtext{}{\edlabel{pvv.119-4}\label{pvv.119-4}\lemma{त्}\Bfootnote{नामादि ।}}। त‚स्मान्नास्त्य‚र्थः ‚{\tiny $_{lb}$}‚सामा‚{\tiny $_{3}$}‚न्य‚बुद्धिरिति । (१६)
	\pend% ending standard par
      \label{div_pvv.2.17}
	  
	% new div opening: depth here is 2
	

	  \pstart \leavevmode% starting standard par
	अपि च (।)
	\pend% ending standard par
      
	  \bigskip
	  \begingroup
	
	    \large
	  
	    \begin{quote}
	  
	    
	    \stanza[\smallbreak]
	\label{pv.2.17}\flagstanza{\tiny\textenglish{...v.2.17}}सैवेष्टार्थ‚व‚ती केन च‚क्षुरादिम‚तिः स्मृता ।&अर्थ‚साम‚र्थ्य‚दृष्टेश्चेद‚न्य‚त् प्राप्त‚म‚न‚र्थ‚क‚म् ॥ १७ ॥\&[\smallbreak]


	
	    \end{quote}
	  
	  \endgroup
	

	  \pstart \leavevmode% starting standard par
	\hphantom{.}यैव रूपादिविष‚य‚त्वे‚{\color{DodgerBlue3}‚नेष्टा सैव} च‚क्षुरादिम‚ति‚{\color{DodgerBlue3}‚र‚र्थ‚व‚ती केन} हेतुना ‚{\color{DodgerBlue3}‚म‚ता । ‚{\tiny $_{lb}$}‚अर्थ}‚स्य रूपा‚{\color{DodgerBlue3}‚देश्च‚क्षुरादिम‚ति}‚ज‚न‚ने ‚{\color{DodgerBlue3}‚साम‚र्थ्य‚दृष्टेश्चेत् । अन्य}‚सामान्य‚विष‚यं ‚{\tiny $_{lb}$}‚\leavevmode\ledsidenote{\textenglish{120/s}} विक‚ल्प‚ज्ञान‚{\color{DodgerBlue3}‚म‚न‚र्थ‚कं प्राप्तं}‚(।) न हि य‚था च‚क्षुरादिबुद्धेर‚र्थ‚व्य‚तिरेकाद् व्य‚तिरेकः ‚{\tiny $_{lb}$}‚त‚था सामान्य‚बुद्धेर्व्य‚तिरेकः । आभोग‚मात्रेण भावात् ॥ (१७)
	\pend% ending standard par
      \label{div_pvv.2.18}
	  
	% new div opening: depth here is 2
	
	  \bigskip
	  \begingroup
	
	    \large
	  
	    \begin{quote}
	  
	    
	    \stanza[\smallbreak]
	\label{pv.2.18a}\flagstanza{\tiny\textenglish{....2.18a}}अप्र‚वृत्तिर‚स‚म्ब‚न्धेप्य‚र्थ‚स‚म्ब‚न्ध‚व‚द् य‚दि ।\&[\smallbreak]


	
	    \end{quote}
	  
	  \endgroup
	

	  \pstart \leavevmode% starting standard par
	न\edtext{}{\edlabel{pvv.120-1}\label{pvv.120-1}\lemma{न}\Bfootnote{त‚देवं नामादाव‚प्र‚वृत्तिः ।}} केव‚लं व‚क्तृश्रोतृस‚म्ब‚न्धिनि\edtext{}{\edlabel{pvv.120-2}\label{pvv.120-2}\lemma{न्धिनि}\Bfootnote{स‚म्ब‚न्धेपि ।}} ‚{\color{DodgerBlue3}‚अस‚म्ब‚न्धेपि} नामा‚{\tiny $_{4}$}‚दाव‚र्थेऽ‚{\color{DodgerBlue3}‚प्र‚वृत्तिः} स्यात् । ‚{\tiny $_{lb}$}‚‚{\color{DodgerBlue3}‚अर्थ‚स‚म्ब\edtext{}{\edlabel{pvv.120-3}\label{pvv.120-3}\lemma{म्ब}\Bfootnote{अर्थेन स‚म्ब‚द्धं ।}}न्ध‚व‚त्} नामाद्य‚र्थे प्र‚वृत्त्य‚र्थं ‚{\color{DodgerBlue3}‚य‚दी}‚ष्य‚ते त‚दा अतीतानाग‚तं नामादि त‚द‚भिधा‚{\tiny $_{lb}$}‚यिनां श‚ब्दानां वाच्यं न स्यात् । अभूत् मा न्धा ता भ‚विष्य‚ति शं ख श्च‚क्र‚व‚र्त्तीति (।)
	\pend% ending standard par
      

	  \pstart \leavevmode% starting standard par
	क‚स्मादित्याह (।)
	\pend% ending standard par
      
	  \bigskip
	  \begingroup
	
	    \large
	  
	    \begin{quote}
	  
	    
	    \stanza[\smallbreak]
	\label{pv.2.18b}\flagstanza{\tiny\textenglish{....2.18b}}अतीतानाग‚तं वाच्यं न स्याद‚र्थेन त‚त्क्ष‚यात् ॥ १८ ॥\&[\smallbreak]


	
	    \end{quote}
	  
	  \endgroup
	

	  \pstart \leavevmode% starting standard par
	\hphantom{.}‚{\color{DodgerBlue3}‚अर्थेनातीतानाग‚तेन} स‚ह त‚स्य नामादेः स‚म्ब‚न्धिनः ‚{\color{DodgerBlue3}‚स्यात्} स्व‚रूपा\edtext{}{\edlabel{pvv.120-4}\label{pvv.120-4}\lemma{रूपा}\Bfootnote{अनाग‚त‚स्य चानुत्प‚त्तेः ।}} ‚{\tiny $_{lb}$}‚भावात् ॥ (१८)
	\pend% ending standard par
      \label{div_pvv.2.19}
	  
	% new div opening: depth here is 2
	

	  \begin{center}%% label @type='head'
	\textbf{(२) प‚र‚म‚ते दोषः}
	\end{center}
	

	  \begin{center}%% label @type='head'
	\textbf{क. व्य‚क्तिसामान्य‚योर्भेदे दोषः}
	\end{center}
	

	  \pstart \leavevmode% starting standard par
	अथ (।)
	\pend% ending standard par
      
	  \bigskip
	  \begingroup
	
	    \large
	  
	    \begin{quote}
	  
	    
	    \stanza[\smallbreak]
	\label{pv.2.19}\flagstanza{\tiny\textenglish{...v.2.19}}सामान्य‚ग्र‚ह‚णाच्छ‚ब्दाद‚प्र‚संगो म‚तो य‚दि ।&त‚न्न केव‚ल‚सामान्याग्र‚ह‚णाद् ग्र‚ह‚णेपि वा ॥ १९ ॥\&[\smallbreak]


	
	    \end{quote}
	  
	  \endgroup
	

	  \pstart \leavevmode% starting standard par
	\hphantom{.}‚{\color{DodgerBlue3}‚श‚ब्दात् सामान्य‚ग्र\edtext{}{\edlabel{pvv.120-5}\label{pvv.120-5}\lemma{ग्र}\Bfootnote{आनुष‚ङ्गिक‚म‚प‚नीय प्र‚कृतं सामान्य‚माह (।) श‚ब्देन सामान्य‚स्यैव ग्र‚हाद्रूपादौ स‚फ‚ल‚म‚क्षं ॥}}ह‚णात्} त‚त्प्र‚तीत्या फ‚ल‚व‚न्त्य‚क्षाणीति तेषां वैफ‚ल्य‚स्या‚{\color{DodgerBlue3}‚प्र‚स‚ङ्गो ‚{\tiny $_{lb}$}‚म‚तो य‚दि}‚(।) त‚देत‚न्न युक्तं (।) ‚{\color{DodgerBlue3}‚केव‚ल}‚स्य व्य‚क्तिशून्य‚स्य ‚{\color{DodgerBlue3}‚सामान्य‚{\tiny $_{5}$}‚स्याग्र‚ह‚णात्} । ‚{\tiny $_{lb}$}‚व्य‚क्तिव्यंग्यं हि सामान्यं व्य‚ञ्ज‚काग्र‚ह‚णे क‚थं गृह्य‚ते । केव‚ल‚स्य ‚{\color{DodgerBlue3}‚ग्र‚ह‚णेपि वा} । (१९)
	\pend% ending standard par
      \label{div_pvv.2.20}
	  
	% new div opening: depth here is 2
	
	  \bigskip
	  \begingroup
	
	    \large
	  
	    \begin{quote}
	  
	    
	    \stanza[\smallbreak]
	\label{pv.2.20}\flagstanza{\tiny\textenglish{...v.2.20}}अत‚त्स‚मान‚ता-व्य‚क्ती तेन नित्योप‚ल‚म्भ‚न‚म् ।&नित्य‚त्वाच्च य‚दि व्य‚क्तिर्व्य‚क्तेः प्र‚त्य‚क्ष‚तां प्र‚ति ॥ २० ॥\&[\smallbreak]


	
	    \end{quote}
	  
	  \endgroup
	

	  \pstart \leavevmode% starting standard par
	\hphantom{.}‚{\color{DodgerBlue3}‚अत‚त्स\edtext{}{\edlabel{pvv.120-6}\label{pvv.120-6}\lemma{त्स}\Bfootnote{स‚मानानां भावः स‚मान‚ता व्य‚क्तेः सामान्यं न स्यादित्य‚र्थः । व्य‚क्तिर‚भिव्य‚क्तिर‚पि न स्यात् ।}}मान‚ता}‚-व्य क्ती । य‚त्प्र‚तीत्या य‚त्प्र‚तीय‚ते त‚त्त‚स्य सामान्यं व्यंग्य‚ञ्च । ‚{\tiny $_{lb}$}‚स्व‚त‚न्त्र‚प्र‚तीतिस्तु न सामान्यं व्य‚ङ्ग्यं वा । ‚{\color{DodgerBlue3}‚तेना}‚प‚राधीन‚प्र‚तीतित्वेन ‚{\color{DodgerBlue3}‚नित्योप‚ल‚म्भ‚नं} सामान्यं प्राप्तं । व्य‚क्तिस‚द‚स‚त्त्व‚योर‚प्युप‚ल‚ब्धिः स्यात् । स्व‚प्र‚तीतौ व्य‚क्तिनिर‚{\tiny $_{lb}$}‚पेक्ष‚त्वात् । त‚थोप‚ल‚भ्य‚स्व‚भाव\edtext{}{\edlabel{pvv.120-7}\label{pvv.120-7}\lemma{भाव}\Bfootnote{व्य‚क्त्य‚नुत्प‚त्तिविनाशेपि ।}}स्य सामान्य‚स्य ‚{\color{DodgerBlue3}‚नित्य‚त्वाच्च नित्य‚मुप‚ल‚म्भ‚नं} भ‚वेत् । ‚{\tiny $_{lb}$}‚\leavevmode\ledsidenote{\textenglish{121/s}} य‚दि व्य‚क्तेर्व्विशेषात्‚{\tiny $_{6}$}‚ व्य‚क्तिर‚भि‚{\color{DodgerBlue3}‚व्य‚क्तिः} सामान्य‚स्य ‚{\color{DodgerBlue3}‚प्र‚त्य‚क्ष‚तां प्र‚ति} प्र‚त्य‚क्ष‚भाव‚{\tiny $_{lb}$}‚निमित्त‚मिष्टा । प्र‚त्य‚क्ष‚भूत‚सामान्य‚विष‚या प्र‚तीतिर्व्य‚ञ्जिकाया व्य‚क्तेर्भ‚व‚ति । ‚{\tiny $_{lb}$}‚अन्या तु सामान्य‚बुद्धिर्व्य‚ञ्ज‚क‚म‚न्त‚रेणापि ॥ (२०)
	\pend% ending standard par
      \label{div_pvv.2.21}
	  
	% new div opening: depth here is 2
	

	  \pstart \leavevmode% starting standard par
	न‚नु (।)
	\pend% ending standard par
      
	  \bigskip
	  \begingroup
	
	    \large
	  
	    \begin{quote}
	  
	    
	    \stanza[\smallbreak]
	\label{pv.2.21}\flagstanza{\tiny\textenglish{...v.2.21}}आत्म‚नि ज्ञान‚ज‚न‚ने य‚च्छ‚क्तं श‚क्त‚मेव त‚त् ।&अथाश‚क्तं क‚दाचिच्चेद‚श‚क्तं स‚र्व‚दैव त‚त् ॥ २१ ॥\&[\smallbreak]


	
	    \end{quote}
	  
	  \endgroup
	

	  \pstart \leavevmode% starting standard par
	\hphantom{.}‚{\color{DodgerBlue3}‚य\edtext{}{\edlabel{pvv.121-1}\label{pvv.121-1}\lemma{य}\Bfootnote{सिद्धान्त्याह (।) सामान्यं त‚त्स्व‚भाव‚तः प्र‚त्य‚क्ष‚ज‚न‚कं न वेति ।}}त्सा}‚मान्य‚{\color{DodgerBlue3}‚मात्म‚नि} प्र‚त्य‚क्षात्म‚क‚{\color{DodgerBlue3}‚ज्ञान‚ज‚न‚ने} व्य‚ञ्ज‚क‚व्यापार‚काले ‚{\color{DodgerBlue3}‚श‚क्त-} म‚न्य‚दापि ‚{\color{DodgerBlue3}‚श‚क्त‚मे}‚व त‚न्नित्यैक‚स्व‚भाव‚त‚याऽनुप‚कार्य‚त्वात् । त‚त‚श्च केव‚ल‚स्य ‚{\tiny $_{lb}$}‚सामान्य‚स्य प्र‚त्य‚क्ष‚ज‚न‚नाद‚क्ष‚वैक‚ल्य‚प्र‚स‚ङ्गः । ‚{\color{DodgerBlue3}‚अथाश‚क्तं त‚त्क‚दाचित्}‚{\tiny $_{7}$}‚ केव‚लाद्य‚व‚स्था-\leavevmode\ledsidenote{\textenglish{23a/MA}} ‚{\tiny $_{lb}$}‚याश्चेत् त‚दा ‚{\color{DodgerBlue3}‚स‚र्व्व‚दै}‚वाश‚क्तं ‚{\color{DodgerBlue3}‚त‚दु}‚पेयं एक‚स्य स्व‚भाव‚द्व‚यायोगात् स्व‚भाव‚भेद‚{\tiny $_{lb}$}‚ल‚क्ष‚ण‚त्वाद्व‚स्तुभेद‚स्य । (२१)
	\pend% ending standard par
      \label{div_pvv.2.22}
	  
	% new div opening: depth here is 2
	

	  \pstart \leavevmode% starting standard par
	एत‚देव स्फुट‚य‚ति (।)
	\pend% ending standard par
      
	  \bigskip
	  \begingroup
	
	    \large
	  
	    \begin{quote}
	  
	    
	    \stanza[\smallbreak]
	\label{pv.2.22}\flagstanza{\tiny\textenglish{...v.2.22}}त‚स्य श‚क्तिर‚श‚क्तिर्वा या स्व‚भावेन संस्थिता ।&नित्य‚त्वाद‚पि किं त‚स्य क‚स्तां क्ष‚प‚यितुं क्ष‚मः ॥ २२ ॥\&[\smallbreak]


	
	    \end{quote}
	  
	  \endgroup
	

	  \pstart \leavevmode% starting standard par
	\hphantom{.}त‚स्य सामान्य‚स्य ‚{\color{DodgerBlue3}‚श‚क्तिर‚श‚क्तिर्व्वा} स्व‚विष‚य‚ज्ञान‚ज‚न‚नादौ ‚{\color{DodgerBlue3}‚या स्व‚भावेन} संस्थिता । तां श‚क्तिम‚श‚क्तिम्वा ‚{\color{DodgerBlue3}‚नित्य‚त्वाद‚चिकि}‚त्स्य‚स्यान‚प‚नेय‚प्राचीन‚स्व‚भाव‚स्य ‚{\tiny $_{lb}$}‚कोऽन्यः ‚{\color{DodgerBlue3}‚क्ष‚प‚यितुं क्ष‚मः} । (२२)
	\pend% ending standard par
      \label{div_pvv.2.23}
	  
	% new div opening: depth here is 2
	
	  \bigskip
	  \begingroup
	
	    \large
	  
	    \begin{quote}
	  
	    
	    \stanza[\smallbreak]
	\label{pv.2.23}\flagstanza{\tiny\textenglish{...v.2.23}}त‚च्च सामान्य‚विज्ञान‚म‚नुरुन्ध‚न् विभाव्य‚ते ।&नीलाद्याकार‚लेशो यः स त‚स्मिन् केन निर्मितः ॥ २३ ॥\&[\smallbreak]


	
	    \end{quote}
	  
	  \endgroup
	

	  \pstart \leavevmode% starting standard par
	किन्तु य‚त्त‚द्व्य‚क्तिनिर‚पेक्ष‚मि\edtext{}{\edlabel{pvv.121-2}\label{pvv.121-2}\lemma{मि}\Bfootnote{सामान्य‚विष‚यं ।}}ष्टं‚{\color{DodgerBlue3}‚त‚च्च सामान्य‚विज्ञान‚म‚नुरुन्ध‚न्} विष‚य‚भावेनानु‚{\tiny $_{lb}$}‚व‚र्त‚मानो ‚{\color{DodgerBlue3}‚यो विभाव्य‚ते नीलाद्याकार‚लेशोऽसं}‚{\tiny $_{1}$}‚पूर्ण्ण‚स्फुटीभावः ‚{\color{DodgerBlue3}‚स \edtext{}{\edlabel{pvv.121-3}\label{pvv.121-3}\lemma{स}\Bfootnote{विक‚ल्पाविक‚ल्पा त‚न्म‚तेन ।}}त‚स्मिन्} सामान्य‚{\tiny $_{lb}$}‚ज्ञाने ‚{\color{DodgerBlue3}‚केना}‚र्थेन ‚{\color{DodgerBlue3}‚निर्मितः} । न ताव‚ज्जात्या त‚स्यास्त\edtext{}{\edlabel{pvv.121-4}\label{pvv.121-4}\lemma{स्यास्त}\Bfootnote{व्य‚क्त्याकारास‚म्भ‚वात् ।}}द‚स‚म्भ‚वात् । नापि व्य‚क्त्यात‚त्र ‚{\tiny $_{lb}$}‚त‚स्या\edtext{}{\edlabel{pvv.121-5}\label{pvv.121-5}\lemma{स्या}\Bfootnote{? व्य‚क्तिरिति ।}} अप्र‚तिभास‚नात् । त‚त‚श्च केव‚ल‚सामान्य‚ग्राहि त‚त् ज्ञान‚मिति न युक्तं । ‚{\tiny $_{lb}$}‚विशेष‚ग्र‚ह‚णे चाक्ष‚वैफ‚ल्यं त‚द‚व‚स्थं\edtext{}{\edlabel{pvv.121-6}\label{pvv.121-6}\lemma{स्थं}\Bfootnote{? अभ्युप‚ग‚न्त‚व्यं ।}}॥ (२३)
	\pend% ending standard par
      \label{div_pvv.2.24}
	  
	% new div opening: depth here is 2
	

	  \pstart \leavevmode% starting standard par
	स्यादेत‚त् (।) द्विविधो भावानां प्र‚त्य‚यः । प्र‚त्य‚क्षोऽप्र‚त्य‚क्ष‚श्च । त‚त्र (।)
	\pend% ending standard par
      \textsuperscript{\textenglish{122/s}}
	  \bigskip
	  \begingroup
	
	    \large
	  
	    \begin{quote}
	  
	    
	    \stanza[\smallbreak]
	\label{pv.2.24a}\flagstanza{\tiny\textenglish{....2.24a}}प्र‚त्य‚क्ष‚प्र‚त्य‚यार्थ‚त्वान्नाक्षाणां व्य‚र्थ‚तेति चेत् ।\&[\smallbreak]


	
	    \end{quote}
	  
	  \endgroup
	

	  \pstart \leavevmode% starting standard par
	श‚ब्दादिभावेषु प्र‚त्य‚योऽप्र‚त्य‚क्षः । प्र‚त्य‚क्ष‚स्तु अक्षेभ्य इति न त‚द्वैफ‚ल्यं । ‚{\tiny $_{lb}$}‚अत्राह (।)
	\pend% ending standard par
      
	  \bigskip
	  \begingroup
	
	    \large
	  
	    \begin{quote}
	  
	    
	    \stanza[\smallbreak]
	\label{pv.2.24b}\flagstanza{\tiny\textenglish{....2.24b}}सैवैक‚रूपाच्छ‚ब्दादेर्भिन्नाभासा म‚तिः कुतः ॥ २४ ॥\&[\smallbreak]


	
	    \end{quote}
	  
	  \endgroup
	

	  \pstart \leavevmode% starting standard par
	\hphantom{.}‚{\color{DodgerBlue3}‚सैव}\edtext{\textsuperscript{*}}{\edlabel{pvv.122-1}\label{pvv.122-1}\lemma{*}\Bfootnote{स्प‚ष्टास्प‚ष्टा ।}} प्र‚त्य‚क्षाप्र‚त्य‚{\tiny $_{2}$}‚क्षाभासा ‚{\color{DodgerBlue3}‚भिन्नाभासा} म\edtext{}{\edlabel{pvv.122-2}\label{pvv.122-2}\lemma{म}\Bfootnote{त‚त्त्वान्य‚त्वावाच्य‚स्य सामान्य‚त्वे श‚श‚शृङ्गादेः प्र‚स‚ङ्ग‚श्चेत् ।}}तिरेक‚रूपात् श‚ब्दादेरादिश‚ब्दाद् ‚{\tiny $_{lb}$}‚ग‚न्ध‚र‚सादेः कुतः । एक‚रूप‚विष‚या च भिन्न‚प्र‚तिभासा चेति विरुद्धं । (२४)
	\pend% ending standard par
      \label{div_pvv.2.25}
	  
	% new div opening: depth here is 2
	

	  \pstart \leavevmode% starting standard par
	किञ्च (।) जातिर्जातिम‚तो रूपाद् भिन्नाऽभिन्ना वा ।
	\pend% ending standard par
      
	  \bigskip
	  \begingroup
	
	    \large
	  
	    \begin{quote}
	  
	    
	    \stanza[\smallbreak]
	\label{pv.2.25}\flagstanza{\tiny\textenglish{...v.2.25}}न जातिर्जातिम‚द् व्य‚क्तिरूपं येनाप‚राश्र‚य‚म् ।&सिद्धं; पृथ‚क् चेत् कार्य‚त्वं ह्य‚पेक्षेत्य‚भिधीय‚ते ॥ २५ ॥\&[\smallbreak]


	
	    \end{quote}
	  
	  \endgroup
	

	  \pstart \leavevmode% starting standard par
	\hphantom{.}त‚त्र न ताव‚{\color{DodgerBlue3}‚ज्‏जातिर्जातिम\edtext{}{\edlabel{pvv.122-3}\label{pvv.122-3}\lemma{जातिर्जातिम}\Bfootnote{अभेदः ।}}देव व्य‚क्तिरूपं येन} कार‚णेना‚{\color{DodgerBlue3}‚प‚राश्र‚य‚म‚न}‚न्यानुयायि ‚{\tiny $_{lb}$}‚‚{\color{DodgerBlue3}‚सिद्धं} । न ह्येक‚स्या व्य‚क्ते रूप‚म‚न्य‚त्रास्ति । स‚र्व्वानुयायि च सामान्य‚मिष्टं व्य‚क्तेः ‚{\tiny $_{lb}$}‚स‚काशात् ‚{\color{DodgerBlue3}‚पृथ‚क् चेत्सा}‚मान्यं । अस्येदं सामान्य‚मिति भाविक‚स‚म्ब‚न्धा‚{\tiny $_{3}$}‚नुप‚प‚त्तिः । ‚{\tiny $_{lb}$}‚अपेक्षाल‚क्ष‚णः स‚म्ब‚न्ध‚श्चेत् । ‚{\color{DodgerBlue3}‚न‚न्व‚पेक्षेति कार्य‚त्व}‚मुच्य‚ते त‚त्सामान्य‚स्य नित्य‚स्या‚{\tiny $_{lb}$}‚स‚म्भ‚वि ॥ (२५)
	\pend% ending standard par
      \label{div_pvv.2.26}
	  
	% new div opening: depth here is 2
	
	  \bigskip
	  \begingroup
	
	    \large
	  
	    \begin{quote}
	  
	    
	    \stanza[\smallbreak]
	\label{pv.2.26}\flagstanza{\tiny\textenglish{...v.2.26}}निष्प‚त्तेर‚प‚राधीन‚म‚पि कार्यं स्व‚हेतुतः ।&स‚म्ब‚ध्य‚ते क‚ल्प‚न‚या किम‚कार्यं क‚थ‚ञ्च‚न ॥ २६ ॥\&[\smallbreak]


	
	    \end{quote}
	  
	  \endgroup
	

	  \pstart \leavevmode% starting standard par
	\hphantom{.}‚{\color{DodgerBlue3}‚कार्य‚म‚पि निष्य‚त्ते}‚र्नित्य‚त्वाद‚{\color{DodgerBlue3}‚प‚र\edtext{}{\edlabel{pvv.122-4}\label{pvv.122-4}\lemma{र}\Bfootnote{य‚द‚प्य‚ङ्कुरादि बीजादेः त‚द‚पि निष्प‚न्न‚म‚प‚राधीनं ।}}ाधीनं} स‚र्व्व‚त्र निराशंसं व‚स्तुतो न क्व‚चि‚{\color{DodgerBlue3}‚त्स‚म्ब‚{\tiny $_{lb}$}‚ध्य‚ते} । केव‚लं ‚{\color{DodgerBlue3}‚क‚ल्प\edtext{}{\edlabel{pvv.122-5}\label{pvv.122-5}\lemma{ल्प}\Bfootnote{बीजादेर‚ङ्कुरादीति ।}}न‚या} कार‚णात्म‚नि स‚म्ब‚ध्य‚ते । य‚त्तु ‚{\color{DodgerBlue3}‚क‚थ‚ञ्च‚न} सामान्य‚{\color{DodgerBlue3}‚म‚कार्य ‚{\tiny $_{lb}$}‚त‚त्किं} क्व‚चित् स‚म्भ\edtext{}{\edlabel{pvv.122-6}\label{pvv.122-6}\lemma{म्भ}\Bfootnote{स‚म्ब‚न्ध‚म‚नुभ‚विष्य‚ति ।}}त्स्य‚ते । (२६)
	\pend% ending standard par
      \label{div_pvv.2.27}
	  
	% new div opening: depth here is 2
	
	  \bigskip
	  \begingroup
	
	    \large
	  
	    \begin{quote}
	  
	    
	    \stanza[\smallbreak]
	\label{pv.2.27a}\flagstanza{\tiny\textenglish{....2.27a}}अन्य‚त्वे त‚द‚स‚म्ब‚द्धं । सिद्धाऽतो निःस्व‚भाव‚ता ।&जातिप्र‚संगोऽभाव‚स्य न;\&[\smallbreak]


	
	    \end{quote}
	  
	  \endgroup
	

	  \pstart \leavevmode% starting standard par
	त‚स्माद् (।)
	\pend% ending standard par
      

	  \pstart \leavevmode% starting standard par
	\hphantom{.}व्य‚क्तेः स‚काशा‚{\color{DodgerBlue3}‚द‚न्य‚त्वे त‚त्} सामान्य‚{\color{DodgerBlue3}‚म‚स‚म्ब‚द्ध‚म‚तो}‚ऽस्य ‚{\color{DodgerBlue3}‚निःस्व‚भाव‚ता ‚{\tiny $_{lb}$}‚सिद्धा} व्य‚क्तिभ्य‚स्त‚त्त्वान्य‚त्वाभ्यां व्य‚व‚स्था‚{\tiny $_{4}$}‚प‚यितुम‚श‚क्य‚त्वात् ॥ य‚द्य‚पि निः स्व‚भावा ‚{\tiny $_{lb}$}‚जातिस्त‚थापि ना‚{\color{DodgerBlue3}‚भाव}‚स्य श‚श‚विषाणादेर‚पि जातेर्जातिरूप‚तायाः प्र‚स‚ङ्गः । न हि ‚{\tiny $_{lb}$}‚यो योऽभावः स जातिरुच्य‚ते । किन्तु सामान्यं य‚त्त‚न्निःस्व‚भावं ।
	\pend% ending standard par
      \textsuperscript{\textenglish{123/s}}

	  \pstart \leavevmode% starting standard par
	न‚नु निःस्व‚भाव‚त्वे क‚थं सामान्य‚मित्याह (।)
	\pend% ending standard par
      
	  \bigskip
	  \begingroup
	
	    \large
	  
	    \begin{quote}
	  
	    
	    \stanza[\smallbreak]
	\label{pv.2.27b}\flagstanza{\tiny\textenglish{....2.27b}}अपेक्षाभाव‚त‚स्त‚योः ॥ २७ ॥\&[\smallbreak]


	
	    \end{quote}
	  
	  \endgroup
	

	  \pstart \leavevmode% starting standard par
	\hphantom{.}‚{\color{DodgerBlue3}‚अपेक्षाभाव‚त‚स्त‚योः} । शाब‚लेयागोव्यावृत्त्योः प‚र‚स्प‚र‚म‚पेक्षाभाव‚त‚स्त‚द्व्यावृत्ति‚{\tiny $_{lb}$}‚रेव सामान्यं न श‚श‚विषाणादिः । न हि त‚द‚पेक्षा क्व‚चिद‚स्ति । अत‚द्‏व्यावृत्ति‚{\tiny $_{5}$}‚‚{\tiny $_{lb}$}‚स्तु निःस्व‚भावाप्य‚नुगामिप्र‚त्य‚य‚हेतुः सामान्यं । अभाव‚त्व‚मिव प्राग‚भावादिषु ‚{\tiny $_{lb}$}‚प‚दार्थ‚त्व‚मिव द्र‚व्यादिषु ॥ (२७)
	\pend% ending standard par
      \label{div_pvv.2.28}
	  
	% new div opening: depth here is 2
	
	  \bigskip
	  \begingroup
	
	    \large
	  
	    \begin{quote}
	  
	    
	    \stanza[\smallbreak]
	\label{pv.2.28}\flagstanza{\tiny\textenglish{...v.2.28}}त‚स्माद‚रूपा रूपाणां माश्र‚येणोप‚क‚ल्पिता ।&त‚द्विशेषाव‚गाहार्थैर्जातिः श‚ब्दैः प्र‚काश्य‚ते ॥ २८ ॥\&[\smallbreak]


	
	    \end{quote}
	  
	  \endgroup
	

	  \pstart \leavevmode% starting standard par
	\hphantom{.}‚{\color{DodgerBlue3}‚त‚स्माद्व}‚स्तुतो जा‚{\color{DodgerBlue3}‚तिर‚रूपा} निःस्व‚भावा ‚{\color{DodgerBlue3}‚रूपाणां} शाब‚लेयादीनां ‚{\color{DodgerBlue3}‚नाश्र‚येणोप‚{\tiny $_{lb}$}‚क‚ल्पिता} जातिः । श‚ब्दैस्त‚द्विशेषाव‚गाहार्थैः । ते च ते विशेषाश्च त‚द्वि‚{\color{DodgerBlue3}‚शेषा}‚स्तेषाम‚{\tiny $_{lb}$}‚‚{\color{DodgerBlue3}‚व‚गाहः} प्र‚वृत्तिविष‚य‚त्वेन व्याप‚नं ‚{\color{DodgerBlue3}‚सोऽर्थः} प्र‚योज‚नं येषां तैः स‚कृदेक‚कार्यानेक‚प्र‚तिप‚त्त्य‚र्थं ‚{\tiny $_{lb}$}‚त‚दाश्र‚येणोप‚क‚ल्पिता ‚{\color{DodgerBlue3}‚जाति}‚र‚{\tiny $_{6}$}‚त‚द्व्यावृत्तिल‚क्ष‚णा ‚{\color{DodgerBlue3}‚श‚ब्दै}‚र‚भिधेया । (२८)
	\pend% ending standard par
      \label{div_pvv.2.29_2.30}
	  
	% new div opening: depth here is 2
	

	  \pstart \leavevmode% starting standard par
	य‚दि जातिर्निःस्व‚भावा क‚थं स्व‚भाव‚विशिष्टा व्य‚व‚सीय‚त इत्याह (।)
	\pend% ending standard par
      
	  \bigskip
	  \begingroup
	
	    \large
	  
	    \begin{quote}
	  
	    
	    \stanza[\smallbreak]
	\label{pv.2.29a}\flagstanza{\tiny\textenglish{....2.29a}}त‚स्यां रूपाव‚भासोयं त‚त्वेनार्थ‚स्य वा ग्र‚हः ।&भ्रान्तिः सा;\&[\smallbreak]


	
	    \end{quote}
	  
	  \endgroup
	

	  \pstart \leavevmode% starting standard par
	\hphantom{.}‚{\color{DodgerBlue3}‚त‚स्यां} जातौ रूप‚स्य स्व‚भाव‚स्या‚{\color{DodgerBlue3}‚व‚भासो य‚स्त‚त्वेन} जातिस्व‚भावेन ‚{\color{DodgerBlue3}‚वार्थ‚स्य ‚{\tiny $_{lb}$}‚यो ग्र‚ग्रो भ्रान्तिः सा} जातेनिःस्व‚भाव‚त्वात् विशेषात्म‚क‚त्वाभावात् ।
	\pend% ending standard par
      

	  \pstart \leavevmode% starting standard par
	किन्त‚र्हि भ्रान्तेर्ब्बीज‚मित्याह (।)
	\pend% ending standard par
      
	  \bigskip
	  \begingroup
	
	    \large
	  
	    \begin{quote}
	  
	    
	    \stanza[\smallbreak]
	\label{pv.2.29b}\flagstanza{\tiny\textenglish{....2.29b}}अनादिकालीन‚द‚र्श‚नाभ्यास‚निर्मिता ॥ २९ ॥\&[\smallbreak]


	
	    \end{quote}
	  
	  \endgroup
	

	  \pstart \leavevmode% starting standard par
	\hphantom{.}‚{\color{DodgerBlue3}‚अनादिकालीना}‚नाम‚नादिकालिकानां त‚थाभूताध्य‚व‚साय‚ज्ञानानाम‚{\color{DodgerBlue3}‚भ्यासेन ‚{\tiny $_{lb}$}‚निर्मिता} । (२९)
	\pend% ending standard par
      \textsuperscript{\textenglish{23b/MA}}‚{\tiny $_{lb}$}‚

	  \pstart \leavevmode% starting standard par
	य‚द्य‚न्य‚व्यावृत्तिः सामान्यं किन्त‚स्य रूप‚मित्या‚{\tiny $_{7}$}‚ह (।)
	\pend% ending standard par
      
	  \bigskip
	  \begingroup
	
	    \large
	  
	    \begin{quote}
	  
	    
	    \stanza[\smallbreak]
	\label{pv.2.30}\flagstanza{\tiny\textenglish{...v.2.30}}अर्थानां य‚च्च सामान्य‚म‚न्य‚व्यावृत्तिल‚क्ष‚ण‚म् ।&य‚न्निष्ठास्त इमे श‚ब्दा न रूपं त‚स्य किञ्च‚न ॥ ३० ॥\&[\smallbreak]


	
	    \end{quote}
	  
	  \endgroup
	

	  \pstart \leavevmode% starting standard par
	\hphantom{.}‚{\color{DodgerBlue3}‚अर्थानां} विशेषाणां ‚{\color{DodgerBlue3}‚य‚च्च सामान्य‚म‚न्य‚व्यावृत्तिल‚क्ष‚णं । य‚न्निष्ठा} य‚द्विष‚या‚{\color{DodgerBlue3}‚स्ते ‚{\tiny $_{lb}$}‚इमे} सांकेतिकाः ‚{\color{DodgerBlue3}‚श‚ब्दास्त‚स्य रूपं} स्व‚भावो ‚{\color{DodgerBlue3}‚न किञ्च‚न} । व‚स्तुतः क‚ल्पित‚त्वात् । ‚{\tiny $_{lb}$}‚(३०)
	\pend% ending standard par
      \label{div_pvv.2.31_2.32}
	  
	% new div opening: depth here is 2
	

	  \pstart \leavevmode% starting standard par
	न‚नु बुद्ध्याकारः स स्व‚भावः स एव त‚र्हि सामान्य‚म्भ‚विष्य‚तीत्याह(।)
	\pend% ending standard par
      \textsuperscript{\textenglish{124/s}}
	  \bigskip
	  \begingroup
	
	    \large
	  
	    \begin{quote}
	  
	    
	    \stanza[\smallbreak]
	\label{pv.2.31a}\flagstanza{\tiny\textenglish{....2.31a}}सामान्य‚बुद्धौ सामान्येनारूपायाम‚वीक्ष‚णात् ।&अर्थ‚भ्रान्तिर‚पीष्येत सामान्यं साऽपिः\&[\smallbreak]


	
	    \end{quote}
	  
	  \endgroup
	

	  \pstart \leavevmode% starting standard par
	\hphantom{.}‚{\color{DodgerBlue3}‚सामान्य‚बुद्धाव‚रूपायां} ग्रा\edtext{}{\edlabel{pvv.124-1}\label{pvv.124-1}\lemma{ग्रा}\Bfootnote{बुद्धिप्र‚तिभासः सामान्य‚मित्य‚त्राह ।}}ह्य‚रूप‚र‚हितायाम‚पि ‚{\color{DodgerBlue3}‚सामान्येन} विजातीय‚व्यावृत्त्युप‚{\tiny $_{lb}$}‚क‚ल्पिता-भेदेनाकारेण भेदेष्व‚र्थेष्वी‚{\color{DodgerBlue3}‚क्ष‚णात्} सामान्य‚म‚पि य‚द्व्य‚व‚स्थाप्य‚ते सा‚{\color{DodgerBlue3}‚प्य‚र्थ‚{\tiny $_{lb}$}‚भ्रान्तिरिष्येत‚{\tiny $_{1}$}‚} विवेच‚कैः । न हि बुद्ध्याकारः ‚{\color{DodgerBlue3}‚सामान्य}‚मुक्तं स्व‚ल‚क्ष‚ण‚त्वात् ।
	\pend% ending standard par
      
	  \bigskip
	  \begingroup
	
	    \large
	  
	    \begin{quote}
	  
	    
	    \stanza[\smallbreak]
	\label{pv.2.31b}\flagstanza{\tiny\textenglish{....2.31b}}अभिप्ल‚वात् ॥ ३१ ॥\&[\smallbreak]


	
	    \end{quote}
	  
	  \endgroup
	
	  \bigskip
	  \begingroup
	
	    \large
	  
	    \begin{quote}
	  
	    
	    \stanza[\smallbreak]
	\label{pv.2.32}\flagstanza{\tiny\textenglish{...v.2.32}}अर्थ‚रूप‚त‚या त‚त्वेनाभावाच्च न रूपिणी ।&निःस्व‚भाव‚त‚याऽवाच्यं कुत‚श्चिद् व‚च‚नान्म‚त‚म् ॥ ३२ ॥\&[\smallbreak]


	
	    \end{quote}
	  
	  \endgroup
	

	  \pstart \leavevmode% starting standard par
	\hphantom{.}अर्थ‚निष्ठ‚{\color{DodgerBlue3}‚त‚या} तु सामान्ये गृह्य‚माणेऽर्थ‚भ्र‚मः\edtext{}{\edlabel{pvv.124-2}\label{pvv.124-2}\lemma{मः}\Bfootnote{अत‚स्मिँस्त‚द्ग्र‚हात् ।}} । त\edtext{}{\edlabel{pvv.124-3}\label{pvv.124-3}\lemma{त}\Bfootnote{त‚त्रैवाह हेतुं ।}}था बुद्ध्याकार‚स्यार्थ‚रूप‚त्वेन ‚{\tiny $_{lb}$}‚व्य‚क्तिष्व\edtext{}{\edlabel{pvv.124-4}\label{pvv.124-4}\lemma{क्तिष्व}\Bfootnote{भाति स्वाकार‚म‚र्थेर्प्प‚य‚न्ती ।}}भिप्ल‚वात् अभिस‚म्ब‚न्धात् । ‚{\color{DodgerBlue3}‚त\edtext{}{\edlabel{pvv.124-5}\label{pvv.124-5}\lemma{त}\Bfootnote{व‚स्तुतोर्थ‚रूप‚त‚या ।}}त्त्वे}‚नार्थ‚रूप‚त्वे‚{\color{DodgerBlue3}‚नाभावा\edtext{}{\edlabel{pvv.124-6}\label{pvv.124-6}\lemma{नाभावा}\Bfootnote{नाप्य‚र्थ‚ध‚र्मोन्यापोहोत्र युक्तः ।}}च्च} । ‚{\color{DodgerBlue3}‚न ‚{\tiny $_{lb}$}‚रूपि\edtext{}{\edlabel{pvv.124-7}\label{pvv.124-7}\lemma{रूपि}\Bfootnote{निःस्व‚भावापूर्व्व‚हेतोश्च}}णी} त‚था-भूत‚बाह्य‚विष‚य‚व‚ती सामान्य‚बुद्धिः अत‚श्च । त‚त‚श्च ‚{\color{DodgerBlue3}‚निःस्व‚भाव‚त‚या} सामान्यं भेदाभेदाभ्याम‚{\color{DodgerBlue3}‚वाच्यं} व्य‚क्तिभ्यः । स्व‚भावं हि भिन्न‚म‚भिन्नं वा स्यात् । ‚{\tiny $_{lb}$}‚‚{\color{DodgerBlue3}‚कुत‚श्चिद्}‚सामान्यात् भेदेन ‚{\color{DodgerBlue3}‚व‚च‚नात्} य‚दि व‚स्तुसामा‚{\tiny $_{2}$}‚न्यं ‚{\color{DodgerBlue3}‚म‚तं} । (३१, ३२)
	\pend% ending standard par
      \label{div_pvv.2.33}
	  
	% new div opening: depth here is 2
	
	  \bigskip
	  \begingroup
	
	    \large
	  
	    \begin{quote}
	  
	    
	    \stanza[\smallbreak]
	\label{pv.2.33}\flagstanza{\tiny\textenglish{...v.2.33}}य‚दि व‚स्तुनि व‚स्तूनाम‚वाच्य‚त्वं क‚थ‚ञ्च‚न ।&नैव वाच्य‚मुपादान‚भेदाद् भेदोप‚चार‚तः ॥ ३३ ॥\&[\smallbreak]


	
	    \end{quote}
	  
	  \endgroup
	

	  \pstart \leavevmode% starting standard par
	य‚द्येवं व्य‚क्तेर\edtext{}{\edlabel{pvv.124-8}\label{pvv.124-8}\lemma{क्तेर}\Bfootnote{व‚स्तुत्वात् । ? अश्व ।}}पि भेदाभेदाभ्यां वाच्यं स्यात् । य‚स्मान्न ‚{\color{DodgerBlue3}‚व‚स्तूनाम‚वाच्य‚त्वं ‚{\tiny $_{lb}$}‚क‚थ‚ञ्च‚न} । य‚था हि सामान्यं स्व‚रूप‚व‚त्त्वात् सामान्यान्त‚राद् भेदेनोच्य‚ते त‚था ‚{\tiny $_{lb}$}‚व्य‚क्तेर‚पि त‚त्त्वान्य‚त्वाभ्यां य‚थास‚म्भ‚व‚मुच्येत\edtext{}{\edlabel{pvv.124-9}\label{pvv.124-9}\lemma{मुच्येत}\Bfootnote{व‚स्तुभ‚व‚त्स‚त्त‚त्वान्य‚त्व‚म्वा नातिक्राम‚ति ।}} । त‚स्माद्य‚त् कुत‚श्चिद‚पि व‚स्तुन‚स्त‚{\tiny $_{lb}$}‚त्वान्य‚त्वाभ्याम‚वाच्यं त‚द‚व‚स्त्विति स‚विशेष‚णो हेतुः । अथ‚वा निर्व्विशेष‚णे हेतौ ना‚{\tiny $_{lb}$}‚सिद्धिः । य‚त\edtext{}{\edlabel{pvv.124-10}\label{pvv.124-10}\lemma{त}\Bfootnote{प‚टाद‚पि घ‚ट‚त्व‚म‚न्य‚त्वेन नैव वाच्यं । न चोप‚च‚रित‚भेदेन व‚स्तुत्वं । न हि क‚लाय‚विद‚लं स्व‚र्ण्ण‚मुप‚च‚र्य हेम‚वान् (?व‚त्) स्यात् ।}}सामान्यान्त‚राद‚पि सामान्यं भेदेन ‚{\color{DodgerBlue3}‚नैव वाच्य‚मुपा}‚दान‚स्य‚{\tiny $_{3}$}‚ क‚र्क‚{\tiny $_{lb}$}‚शाब‚लेयादे‚{\color{DodgerBlue3}‚र्भेदा}‚त्त‚दाश्र‚योत्प‚न्न‚बुद्ध्याल‚म्ब‚ना अश्व‚त्व‚गोत्वादीनां ‚{\color{DodgerBlue3}‚भेद\edtext{}{\edlabel{pvv.124-11}\label{pvv.124-11}\lemma{भेद}\Bfootnote{न तु भिन्न‚सामान्ये त‚द्रूपाद‚य एव केव‚ला भिन्नाः । उप‚च‚रित‚भेदान्न व‚स्तुत्व‚सिद्धिः ।}}स्योप‚चारात्} । ‚{\tiny $_{lb}$}‚(३३)
	\pend% ending standard par
      \label{div_pvv.2.34}
	  
	% new div opening: depth here is 2
	\textsuperscript{\textenglish{125/s}}

	  \pstart \leavevmode% starting standard par
	किञ्च (।)
	\pend% ending standard par
      
	  \bigskip
	  \begingroup
	
	    \large
	  
	    \begin{quote}
	  
	    
	    \stanza[\smallbreak]
	\label{pv.2.34}\flagstanza{\tiny\textenglish{...v.2.34}}अतीतानाग‚तेप्य‚र्थे सामान्य‚विनिब‚न्ध‚नाः ।&श्रुत‚यो निविश‚न्ते स‚द‚स‚द्ध‚र्मः क‚थ‚म्भ‚वेत् ॥ ३४ ॥\&[\smallbreak]


	
	    \end{quote}
	  
	  \endgroup
	

	  \pstart \leavevmode% starting standard par
	\hphantom{.}‚{\color{DodgerBlue3}‚अतीतानाग‚तेप्य‚र्थे सामान्य‚निब‚न्ध‚ना} सामान्याश्र‚याः ‚{\color{DodgerBlue3}‚श्रुत‚यो निविश‚न्ते} व्य‚व‚{\tiny $_{lb}$}‚तिष्ठ‚न्ते । आसीत् घ‚टो भ‚विष्य‚तीत्याद‚यः । त‚था चास‚तो घ‚ट‚स्य सामान्यं ध‚र्म ‚{\tiny $_{lb}$}‚इत्युक्तं स्यात् । त‚च्च सामान्यं ‚{\color{DodgerBlue3}‚स‚द‚स‚तो}‚ऽतीतादे‚{\color{DodgerBlue3}‚र्ध‚र्मः क‚थ‚म्भ‚वेत्} । न हि तैक्ष्ण्यं ‚{\tiny $_{lb}$}‚श‚श‚विषाण‚स्य भ‚व‚ति\edtext{}{\edlabel{pvv.125-1}\label{pvv.125-1}\lemma{ति}\Bfootnote{अतीतादौ न सामान्य‚निब‚न्ध‚ना श‚ब्द‚वृत्तिः किन्तूप‚चारादित्याह ।}}। (३४)
	\pend% ending standard par
      \label{div_pvv.2.35}
	  
	% new div opening: depth here is 2
	
	  \bigskip
	  \begingroup
	
	    \large
	  
	    \begin{quote}
	  
	    
	    \stanza[\smallbreak]
	\label{pv.2.35}\flagstanza{\tiny\textenglish{...v.2.35}}उप‚चारात् त‚दिष्टं चेद् व‚र्त्त‚मान‚घ‚ट‚स्य का ।&प्र‚त्यास‚त्तिर‚भावेन या प‚टादौ न विद्य‚ते ॥ ३५ ॥\&[\smallbreak]


	
	    \end{quote}
	  
	  \endgroup
	

	  \pstart \leavevmode% starting standard par
	\hphantom{.}‚{\color{DodgerBlue3}‚अथोप‚चारात्त}‚द‚स‚द्ध‚र्म‚त्व‚{\color{DodgerBlue3}‚मिष्टं} सामान्य‚स्य न व‚स्तुत इति ‚{\color{DodgerBlue3}‚चेत्} । स\edtext{}{\edlabel{pvv.125-2}\label{pvv.125-2}\lemma{स}\Bfootnote{घ‚ट‚स्य रूपादिभ्योऽवाच्य‚त्वेपि प‚टाद् घ‚टे भिन्न‚ता......वाच्य एव । व‚र्त‚माने घ‚ट‚त्व‚द‚र्श‚नात् ।}}ति घ‚टे ‚{\tiny $_{lb}$}‚त‚द्ध‚र्म‚तादृष्टेर‚स‚त्य‚पि‚{\tiny $_{4}$}‚ त‚स्मिन् सा क‚ल्प्य‚त इत्य‚र्थः ।
	\pend% ending standard par
      

	  \pstart \leavevmode% starting standard par
	न‚नु सामान्य‚स‚म्ब‚न्धिनो\edtext{}{\edlabel{pvv.125-3}\label{pvv.125-3}\lemma{न्धिनो}\Bfootnote{व‚र्त्त‚मान‚स्य ।}}ऽपि घ‚ट‚स्या‚{\color{DodgerBlue3}‚भावे}‚नातीतादिघ‚ट‚ल‚क्ष‚णेन का ‚{\color{DodgerBlue3}‚प्र‚त्यास}‚त्ति‚{\tiny $_{lb}$}‚रुप‚चार‚निब‚न्ध‚न‚म‚स्ति ‚{\color{DodgerBlue3}‚या प‚टादौ न विद्य‚ते} य‚द‚भावा\edtext{}{\edlabel{pvv.125-4}\label{pvv.125-4}\lemma{भावा}\Bfootnote{य‚दि व‚स्तुभूत‚सामान्याश्र‚येण श‚ब्द‚वृत्तिस्त‚दातीतादौ न स्यात् ।}}त् घ‚ट‚स‚म्ब‚न्धिता सामान्य‚स्य ‚{\tiny $_{lb}$}‚प\edtext{}{\edlabel{pvv.125-5}\label{pvv.125-5}\lemma{प}\Bfootnote{प‚टे य‚स्याः प्र‚त्यास‚त्तेर‚भावान्नोप‚चार इति ।}}टे नोप‚च‚र्य‚ते (।) सादृश्यं प्र‚त्यास‚त्तिरित चेत् । त‚न्न(।) स‚द‚स‚तो सादृश्याभावात् ॥ ‚{\tiny $_{lb}$}‚प्राक् तादृगासीदिति चेत् । य‚दासीन्न (त‚दा) त‚दुप‚चारः स‚म्ब‚न्ध‚स्य स‚त्त्वात् । य‚दा ‚{\tiny $_{lb}$}‚य‚नास्ति त‚दापि न सादृश्यं । स‚ति च किञ्चिदुप‚च‚र्येत नास‚ति । (३५)
	\pend% ending standard par
      \label{div_pvv.2.36}
	  
	% new div opening: depth here is 2
	

	  \pstart \leavevmode% starting standard par
	किञ्च (।)
	\pend% ending standard par
      
	  \bigskip
	  \begingroup
	
	    \large
	  
	    \begin{quote}
	  
	    
	    \stanza[\smallbreak]
	\label{pv.2.36}\flagstanza{\tiny\textenglish{...v.2.36}}बुद्धेर‚स्ख‚लिता वृत्तिर्मुख्यारोपित‚योः स‚दा ।&सिंहे माण‚व‚के त‚द्व‚द् घोष‚णाप्य‚स्ति लौकिकी ॥ ३६ ॥\&[\smallbreak]


	
	    \end{quote}
	  
	  \endgroup
	

	  \pstart \leavevmode% starting standard par
	\hphantom{.}‚{\color{DodgerBlue3}‚मुख्यारोपित‚योर‚र्थ‚योर्बु‚{\tiny $_{5}$}‚द्धे}‚र्ग्रांहिकाया ‚{\color{DodgerBlue3}‚अस्ख‚लि\edtext{}{\edlabel{pvv.125-6}\label{pvv.125-6}\lemma{लि}\Bfootnote{अस्ख‚लिता वृत्तिरित्याव‚र्त्त्य य‚त्नात् ।}}ता} दृढा ‚{\color{DodgerBlue3}‚वृत्तिः} । ‚{\color{DodgerBlue3}‚त‚न्त्राद}‚{\tiny $_{lb}$}‚वृत्तिश्च ‚{\color{DodgerBlue3}‚स‚दा} । य‚था ‚{\color{DodgerBlue3}‚सिंहे माण‚व‚के} च सिंह‚बुद्धेंर‚स्ख‚लिता । स्ख‚लिता च वृत्तिरिति ‚{\tiny $_{lb}$}‚‚{\color{DodgerBlue3}‚लौकिक्य‚पि घोष‚णास्ति} न केव‚लं प्रामाणिकी । न चातीतानाग‚त‚घ‚टादिबुद्धिः ‚{\tiny $_{lb}$}‚स्ख‚ल‚न्ती जाय‚ते येनातीतानाग‚त‚व्य‚क्तिध‚र्म‚ता सामान्य‚स्यारोपिता स्यात् । (३६)
	\pend% ending standard par
      \label{div_pvv.2.37}
	  
	% new div opening: depth here is 2
	\textsuperscript{\textenglish{126/s}}
	  \bigskip
	  \begingroup
	
	    \large
	  
	    \begin{quote}
	  
	    
	    \stanza[\smallbreak]
	\label{pv.2.37}\flagstanza{\tiny\textenglish{...v.2.37}}य‚त्र रूढ्याऽस‚द‚र्थोपि ज‚नैः श‚ब्दो निवेशितः ।&स मुख्य‚स्त‚त्र त‚त्साम्याद् गौणोन्य‚त्र स्ख‚ल‚द्ग‚तिः ॥ ३७ ॥\&[\smallbreak]


	
	    \end{quote}
	  
	  \endgroup
	

	  \pstart \leavevmode% starting standard par
	\hphantom{.}त‚स्मा‚{\color{DodgerBlue3}‚द्य‚त्र}\edtext{}{\edlabel{pvv.126-1}\label{pvv.126-1}\lemma{स्मा}\Bfootnote{सिंहादौ ।}} विष‚ये‚{\color{DodgerBlue3}‚ऽस‚द‚र्थो} वाच्य‚र‚हितोपि ‚{\color{DodgerBlue3}‚श‚ब्दो} रूढ्या वाच‚क‚त्वेन ‚{\color{DodgerBlue3}‚ज‚नैर्निवे‚{\tiny $_{lb}$}‚शितः} संकेतितः ‚{\color{DodgerBlue3}‚स मृख्यः त‚त्रार्थे । त‚त्साम्यात्} । त‚द्विष‚य‚सादृश्या‚{\color{DodgerBlue3}‚द‚न्य‚त्र} स श‚ब्दः ‚{\tiny $_{lb}$}‚‚{\color{DodgerBlue3}‚स्ख‚{\tiny $_{6}$}‚ल‚द्व‚त्तिर}‚दृढ‚त‚या प्र‚त्य‚य‚हैतुः । ‚{\color{DodgerBlue3}‚गौणः} । त‚त‚श्च स‚द‚र्थ‚विष‚य‚त्वं मुख्य‚त्वं अस‚द‚र्थ‚{\tiny $_{lb}$}‚विष‚य‚त्व‚ञ्चामुख्य‚त्व‚मिति मुख्यागौण‚ल‚क्ष‚ण‚म‚पास्तं । संकेत‚व‚शेन निय‚माभावात् । ‚{\tiny $_{lb}$}‚(३७)
	\pend% ending standard par
      \label{div_pvv.2.38}
	  
	% new div opening: depth here is 2
	
	  \bigskip
	  \begingroup
	
	    \large
	  
	    \begin{quote}
	  
	    
	    \stanza[\smallbreak]
	\label{pv.2.38}\flagstanza{\tiny\textenglish{...v.2.38}}य‚था भावेप्य‚भावाख्यां य‚थाक‚ल्प‚न‚मेव वा ।&कुर्याद‚श‚क्ते श‚क्ते वा प्र‚धानादिश्रुतिं ज‚नः ॥ ३८ ॥\&[\smallbreak]


	
	    \end{quote}
	  
	  \endgroup
	

	  \pstart \leavevmode% starting standard par
	\hphantom{.}‚{\color{DodgerBlue3}‚य‚था भावेपि} पुत्रादौ त‚त्कार्यास‚म‚र्थ‚त्वाद‚स‚त्क‚ल्पे‚{\color{DodgerBlue3}‚ऽभावाख्यां} श‚श‚विषाणं ब‚न्ध्या‚{\tiny $_{lb}$}‚सुत इत्यादिकां ज‚नः कुर्य्यात् । त‚त्र भावेप्य‚मुख्योऽभाव‚श‚ब्दः । अभावे तु मुख्यः ‚{\tiny $_{lb}$}‚त\edtext{}{\edlabel{pvv.126-2}\label{pvv.126-2}\lemma{त}\Bfootnote{पुन‚र्व्य‚भिचार‚माह ।}}था ‚{\color{DodgerBlue3}‚य‚था क‚ल्प‚न‚मेव} सां ख्या द्य‚भिम‚ते व‚स्तुतो‚{\color{DodgerBlue3}‚ऽश‚क्ते} प्र‚धानादौ ‚{\color{DodgerBlue3}‚श‚क्ते} वा पुरुषादा\leavevmode\ledsidenote{\textenglish{24a/MA}}‚{\tiny $_{lb}$}‚व‚नेक‚कार्य‚स‚म\edtext{}{\edlabel{pvv.126-3}\label{pvv.126-3}\lemma{म}\Bfootnote{अव‚स्तुनः कुतो भेदः ।}} ‚{\tiny $_{7}$}‚र्थे त‚त्साम्यात् । ‚{\color{DodgerBlue3}‚प्र‚धानादिश्रुतिं ज‚नः कुर्य्यात्} । स प्र‚धान‚श‚ब्दो‚{\tiny $_{lb}$}‚ऽभां\edtext{}{\edlabel{pvv.126-4}\label{pvv.126-4}\lemma{ऽभां}\Bfootnote{क‚र्तृत्वादिमारोप्य त्रिगुणीम‚ये प्र‚धान‚क‚ल्प‚ना त‚स्य च लोकेनानिष्टेरुप‚र‚त‚व्यापारेऽनेक‚कार्य‚स‚म‚र्थेऽयं सांख्य‚पुरुष इति ।}}व एव मुख्यो भावे चामुख्यः । त‚स्मात्स‚द‚स‚द‚र्थ‚विष‚य‚ता मुख्य‚गौण‚विष‚य‚तेति ‚{\tiny $_{lb}$}‚व्य‚भिचारिल‚क्ष‚णं\edtext{}{\edlabel{pvv.126-5}\label{pvv.126-5}\lemma{णं}\Bfootnote{सांख्य‚म‚ते प्राह ।}}। (३८)
	\pend% ending standard par
      \label{div_pvv.2.39}
	  
	% new div opening: depth here is 2
	

	  \pstart \leavevmode% starting standard par
	य‚थासंकेत‚मेव तु श‚ब्द‚वृत्तिरिति युक्त्तं न चैत‚द्व‚स्तुविष(य)त्वे न्याय्यं । ‚{\tiny $_{lb}$}‚त‚था\edtext{}{\edlabel{pvv.126-6}\label{pvv.126-6}\lemma{था}\Bfootnote{अभिध‚या वाच्य‚त्वाभ्याम‚व‚स्तुत्वात्त‚त्रैवोप‚प‚त्त्य‚न्त‚र‚माह ।}}हि (।)
	\pend% ending standard par
      
	  \bigskip
	  \begingroup
	
	    \large
	  
	    \begin{quote}
	  
	    
	    \stanza[\smallbreak]
	\label{pv.2.39a}\flagstanza{\tiny\textenglish{....2.39a}}श‚ब्देभ्यो यादृशी बुद्धिर्न‚ष्टेऽन‚ष्टेपि दृश्य‚ते ।&तादृश्येव;\&[\smallbreak]


	
	    \end{quote}
	  
	  \endgroup
	

	  \pstart \leavevmode% starting standard par
	\hphantom{.}‚{\color{DodgerBlue3}‚श‚ब्देभ्यो यादृशी} यादृशाकारा ‚{\color{DodgerBlue3}‚बुद्धिः न‚ष्टे} विष‚ये । तादृश्येवान‚ष्टेपि ‚{\tiny $_{lb}$}‚विष‚येऽव्याप्त‚तेन्द्रिय‚स्य ‚{\color{DodgerBlue3}‚दृश्य‚ते} । न हि श‚ब्द‚ज‚निता घ‚ट‚बुद्धिर्निमीलित‚न‚य‚न‚स्य ‚{\tiny $_{lb}$}‚न‚ष्टेऽन‚ष्टे वा घ‚टे विशिष्य‚ते । तुल्याकार‚त्वात् । व्याप्तेन्द्रिय‚स्य तु श‚ब्दं श्रृण्व‚{\tiny $_{1}$}‚‚{\tiny $_{lb}$}‚तो या स्प‚ष्टा बुद्धिः सा प्र‚त्य‚क्षैव न श‚ब्द‚कृता । त‚स्माद‚व‚स्तुविष‚यैव शाब्दी बुद्धिः ।
	\pend% ending standard par
      
	  \bigskip
	  \begingroup
	
	    \large
	  
	    \begin{quote}
	  
	    
	    \stanza[\smallbreak]
	\label{pv.2.39b}\flagstanza{\tiny\textenglish{....2.39b}}स‚द‚र्थानां नैत‚च्छोत्रादिचेत‚साम् ॥ ३९ ॥\&[\smallbreak]


	
	    \end{quote}
	  
	  \endgroup
	\textsuperscript{\textenglish{127/s}}

	  \pstart \leavevmode% starting standard par
	\hphantom{.}‚{\color{DodgerBlue3}‚स‚द‚र्थानां} व‚स्तुविष‚याणां तु ‚{\color{DodgerBlue3}‚श्रोत्रा}‚दीन्द्रिय‚जातानां ‚{\color{DodgerBlue3}‚चेत‚सां नैत}‚द्विष‚य‚स‚द‚स‚त्ता‚{\tiny $_{lb}$}‚काल‚योः साम्यं श‚ब्द‚बुद्धेरिवार्थाभावे इन्द्रिय‚बुद्धेर‚नुत्प‚त्तेः । (३९)
	\pend% ending standard par
      \label{div_pvv.2.40}
	  
	% new div opening: depth here is 2
	

	  \pstart \leavevmode% starting standard par
	न‚नु न‚ष्टेऽन‚ष्टेपि व‚स्तुनि ये शाब्दो चेत‚सी जायेते ।
	\pend% ending standard par
      
	  \bigskip
	  \begingroup
	
	    \large
	  
	    \begin{quote}
	  
	    
	    \stanza[\smallbreak]
	\label{pv.2.40a}\flagstanza{\tiny\textenglish{....2.40a}}सामान्य‚मात्र‚ग्र‚ह‚णात् सामान्यं चेत‚सोर्द्व‚योः ॥\&[\smallbreak]


	
	    \end{quote}
	  
	  \endgroup
	

	  \pstart \leavevmode% starting standard par
	\hphantom{.}ताभ्यां ‚{\color{DodgerBlue3}‚स‚मान्य‚मात्र‚स्य ग्र‚ह‚णात्} । त‚यो‚{\color{DodgerBlue3}‚र्द्व‚योश्चेत‚सोः सामान्यं} साम्यं ‚{\color{DodgerBlue3}‚ग्राह्य}‚{\tiny $_{lb}$}‚प्र‚तिभास‚कृत‚म‚स्ति । एत‚च्चायुक्तं य‚स्मात् (।)
	\pend% ending standard par
      
	  \bigskip
	  \begingroup
	
	    \large
	  
	    \begin{quote}
	  
	    
	    \stanza[\smallbreak]
	\label{pv.2.40b}\flagstanza{\tiny\textenglish{....2.40b}}त‚स्यापि केव‚ल‚स्य प्राग् ग्र‚ह‚णं विनिवारित‚म् ॥ ४० ॥\&[\smallbreak]


	
	    \end{quote}
	  
	  \endgroup
	

	  \pstart \leavevmode% starting standard par
	\hphantom{.}‚{\color{DodgerBlue3}‚त‚स्यापि} सामान्य‚स्य ‚{\color{DodgerBlue3}‚केव‚ल‚स्य} व्य‚क्ति‚{\tiny $_{2}$}‚शून्य‚स्य ‚{\color{DodgerBlue3}‚ग्र‚ह‚णं} (२।१४) ‚{\color{DodgerBlue3}‚प्राग्निवा‚{\tiny $_{lb}$}‚रित‚म्} अत‚त्स‚मान‚ताऽव्य‚क्ती (२।२०) इत्यादिना (४०)
	\pend% ending standard par
      \label{div_pvv.2.41}
	  
	% new div opening: depth here is 2
	

	  \begin{center}%% label @type='head'
	\textbf{ख. व्य‚क्तिसामान्य‚योर‚भेदे दोषः}
	\end{center}
	

	  \pstart \leavevmode% starting standard par
	भिन्नं सामान्यं निषिध्याभिन्न‚म‚पि निषेद्ध्ुमाह (।)
	\pend% ending standard par
      
	  \bigskip
	  \begingroup
	
	    \large
	  
	    \begin{quote}
	  
	    
	    \stanza[\smallbreak]
	\label{pv.2.41}\flagstanza{\tiny\textenglish{...v.2.41}}प‚र‚स्प‚र‚विशिष्टानाम‚विशिष्टं क‚थं भ‚वेत् ।&त‚था द्विरूप‚तायां वा त‚द् व‚स्त्वेकं क‚थं भ‚वेत् ॥ ४१ ॥\&[\smallbreak]


	
	    \end{quote}
	  
	  \endgroup
	

	  \pstart \leavevmode% starting standard par
	\hphantom{.}‚{\color{DodgerBlue3}‚प\edtext{}{\edlabel{pvv.127-1}\label{pvv.127-1}\lemma{प}\Bfootnote{पूर्व्वं न जातिर्जातिम‚देव केव‚ल‚मित्युक्त‚म‚धुना त‚त्र युक्तिरुच्य‚ते ।}}र‚स्प‚र‚विशिष्टानां} विशेषाणा‚{\color{DodgerBlue3}‚म‚विशिष्ट‚म‚भिन्नं रूपं क‚थ‚म्भ‚वेत्} । ‚{\tiny $_{lb}$}‚अव‚श्यं हि व्य‚क्तीनां भेदःक‚थ‚ञ्चिद‚ङ्गीक‚र्त्त‚व्यः । स‚त्त्व‚र\edtext{}{\edlabel{pvv.127-2}\label{pvv.127-2}\lemma{र}\Bfootnote{योपि ज‚ग‚त एक‚त्व‚मिच्छ‚ति तेनापि ।}}ज‚स्त‚म‚सामिव प्र‚कृतिचैत‚{\tiny $_{lb}$}‚न्य‚योरिव वाऽन्य‚था सामान्य‚मेव न स्यात् । भिन्नानाम‚भिन्न‚स्य रूप‚स्य त‚त्त्वा\edtext{}{\edlabel{pvv.127-3}\label{pvv.127-3}\lemma{त्त्वा}\Bfootnote{सामान्य‚त्वात् ।}}त् । ‚{\tiny $_{lb}$}‚ये च भिन्न‚स्व‚भावास्ते नाभिन्ना भ‚वितुम‚र्ह‚न्ति विरुद्ध‚त्वात् । अथैक‚स्यापि स‚{\tiny $_{3}$}‚मान‚{\tiny $_{lb}$}‚म‚स‚मान‚ञ्च द्वे रूपे । त‚था ‚{\color{DodgerBlue3}‚द्विरूप‚तायां त‚द्व‚स्त्वेकं क‚थ‚म्भ‚वेत्} । अन्यो हि स‚मानाद‚{\tiny $_{lb}$}‚स‚मानः स्व‚भावः । अत‚श्च द्वे व‚स्तुनी स्यातां न‚त्वेकं द्विरूपं । (४१)
	\pend% ending standard par
      \label{div_pvv.2.42}
	  
	% new div opening: depth here is 2
	

	  \pstart \leavevmode% starting standard par
	अथ द्व‚योरेकं त‚द्रूपं सामान्यं ।
	\pend% ending standard par
      
	  \bigskip
	  \begingroup
	
	    \large
	  
	    \begin{quote}
	  
	    
	    \stanza[\smallbreak]
	\label{pv.2.42a}\flagstanza{\tiny\textenglish{....2.42a}}ताभ्यां त‚द‚न्य‚देव स्याद् य‚दि रूपं स‚मं त‚योः ।\&[\smallbreak]


	
	    \end{quote}
	  
	  \endgroup
	

	  \pstart \leavevmode% starting standard par
	\hphantom{.}‚{\color{DodgerBlue3}‚य‚दि त‚यो}‚र्द्व‚योः ‚{\color{DodgerBlue3}‚स‚मं} स‚मानं ‚{\color{DodgerBlue3}‚रूपं} त‚दा (।) ‚{\color{DodgerBlue3}‚ताभ्या}‚मेवा‚{\color{DodgerBlue3}‚न्य‚देव त‚त्स्यात्} ।
	\pend% ending standard par
      

	  \pstart \leavevmode% starting standard par
	पृथ‚ग्भूत‚मेव सामान्यं भ‚विष्य‚ति को दोष इति चेदाह (।)
	\pend% ending standard par
      
	  \bigskip
	  \begingroup
	
	    \large
	  
	    \begin{quote}
	  
	    
	    \stanza[\smallbreak]
	\label{pv.2.42b}\flagstanza{\tiny\textenglish{....2.42b}}त‚योरिति न स‚म्ब‚न्धो व्यावृत्तिस्तु न दुष्य‚ति ॥ ४२ ॥\&[\smallbreak]


	
	    \end{quote}
	  
	  \endgroup
	\textsuperscript{\textenglish{128/s}}

	  \pstart \leavevmode% starting standard par
	\hphantom{.}‚{\color{DodgerBlue3}‚त‚यो}‚स्त‚त्सामान्य‚{\color{DodgerBlue3}‚मिति न स‚म्ब‚न्धः} । उप‚कार्योप‚कार‚क‚त्वाभावात् । त‚था‚{\tiny $_{lb}$}‚स‚म्ब‚न्धेऽतिप्र‚स‚ङ्गात् ।‚{\tiny $_{4}$}‚ अस्म‚न्म‚ते ‚{\color{DodgerBlue3}‚तु व्यावृत्तिः} सामान्यं ‚{\color{DodgerBlue3}‚न दुष्य‚ति} । अत‚त्कार्य‚{\tiny $_{lb}$}‚व्यावृ\edtext{}{\edlabel{pvv.128-1}\label{pvv.128-1}\lemma{व्यावृ}\Bfootnote{त‚र्ज्ज‚न्या य‚थाङ‚गुष्ठाद् भेद‚स्त‚था प‚रिशिष्टानामेव । व‚स्तुत्वे ।}}त्तेर‚व‚स्तुत्वात् । अन्य‚थाऽन्यान‚न्य‚त्व‚प‚क्षोक्तो दोषः । (४२)
	\pend% ending standard par
      \label{div_pvv.2.43}
	  
	% new div opening: depth here is 2
	
	  \bigskip
	  \begingroup
	
	    \large
	  
	    \begin{quote}
	  
	    
	    \stanza[\smallbreak]
	\label{pv.2.43a}\flagstanza{\tiny\textenglish{....2.43a}}त‚स्मात् स‚मान‚तैवास्मिन् सामान्येऽव‚स्तुल‚क्ष‚ण‚म् ।\&[\smallbreak]


	
	    \end{quote}
	  
	  \endgroup
	

	  \pstart \leavevmode% starting standard par
	\hphantom{.}य‚तः स‚मान‚त्वेनाव‚स्तुता ‚{\color{DodgerBlue3}‚त‚स्मात्स‚मान‚तैवास्मिन् सामा}‚न्येऽ‚{\color{DodgerBlue3}‚व‚स्तुल‚क्ष‚णं} य‚त्सा‚{\tiny $_{lb}$}‚मान्यं त‚द‚व‚स्थितिव्याप्तिसिद्धेः । भेदेऽभेदे च व‚स्तुत्वायोगात् प्र‚कारान्त‚र‚स्य ‚{\tiny $_{lb}$}‚चाभावात् ।
	\pend% ending standard par
      

	  \pstart \leavevmode% starting standard par
	किञ्च (।)
	\pend% ending standard par
      
	  \bigskip
	  \begingroup
	
	    \large
	  
	    \begin{quote}
	  
	    
	    \stanza[\smallbreak]
	\label{pv.2.43b}\flagstanza{\tiny\textenglish{....2.43b}}कार्य‚ञ्चेत् त‚द‚नेकं स्यान्न‚श्व‚र‚ञ्च न त‚न्म‚त‚म् ॥ ४३ ॥\&[\smallbreak]


	
	    \end{quote}
	  
	  \endgroup
	

	  \pstart \leavevmode% starting standard par
	\hphantom{.}सामान्यं कार्य‚म‚कार्य‚म्वा स्यात् । स‚म्ब‚न्धिनीनां व्य‚क्तीनां ‚{\color{DodgerBlue3}‚कार्य‚ञ्चेत्} प्र‚तिव्य‚क्ति सामान्योक्ता‚{\color{DodgerBlue3}‚व‚नेकं स्यात्} । एक‚ञ्च सामा‚{\tiny $_{5}$}‚ न्य‚मिष्टं । अनेक‚त्वे सामान्य‚{\tiny $_{lb}$}‚रूप‚तानाशः । ‚{\color{DodgerBlue3}‚न‚श्व‚र‚ञ्च न त‚त्सा}‚मान्यं ‚{\color{DodgerBlue3}‚म‚तं} । कार्य‚त्वान्न‚श्व‚र‚ञ्च त‚त्प्राप्नोति ‚{\tiny $_{lb}$}‚कार्य‚त्व‚स्य नाशित्वेन स्व‚भाव‚स्य व्याप्तेः । (४३)
	\pend% ending standard par
      \label{div_pvv.2.44}
	  
	% new div opening: depth here is 2
	

	  \pstart \leavevmode% starting standard par
	किञ्च (।)
	\pend% ending standard par
      
	  \bigskip
	  \begingroup
	
	    \large
	  
	    \begin{quote}
	  
	    
	    \stanza[\smallbreak]
	\label{pv.2.44a}\flagstanza{\tiny\textenglish{....2.44a}}व‚स्तुमात्रानुब‚न्धित्वाद् विनाश‚स्य न नित्य‚ता ।\&[\smallbreak]


	
	    \end{quote}
	  
	  \endgroup
	

	  \pstart \leavevmode% starting standard par
	\hphantom{.}‚{\color{DodgerBlue3}‚व‚स्तु}‚मात्रा‚{\color{DodgerBlue3}‚नुब‚न्धित्वात् नाश‚स्य} व‚स्तुनो\edtext{}{\edlabel{pvv.128-2}\label{pvv.128-2}\lemma{स्तुनो}\Bfootnote{किन्तु क्ष‚णिक‚ता स्यात् ।}} ‚{\color{DodgerBlue3}‚न नित्य‚ता} स्यात् । अ\edtext{}{\edlabel{pvv.128-3}\label{pvv.128-3}\lemma{अ}\Bfootnote{व‚स्तुत्वे च सामान्य‚स्यानित्य‚तैव । अनित्य‚त्वेऽनेक‚त्वाद‚सामान्य‚त्वं ।}}नित्य‚ता‚{\tiny $_{lb}$}‚विर‚हे तु व‚स्तुविर‚होपि व्याप्याभाव‚स्य व्याप‚काभाव‚निय‚त‚त्वात् ।
	\pend% ending standard par
      

	  \pstart \leavevmode% starting standard par
	अथ द्वितीयः प‚क्षः । त‚दा (।)
	\pend% ending standard par
      
	  \bigskip
	  \begingroup
	
	    \large
	  
	    \begin{quote}
	  
	    
	    \stanza[\smallbreak]
	\label{pv.2.44b}\flagstanza{\tiny\textenglish{....2.44b}}अस‚म्ब‚न्ध‚श्च जातीनाम‚कार्य‚त्वाद‚रूप‚ता ॥ ४४ ॥\&[\smallbreak]


	
	    \end{quote}
	  
	  \endgroup
	

	  \pstart \leavevmode% starting standard par
	\hphantom{.}‚{\color{DodgerBlue3}‚अकार्य‚त्वात्} जातीनां स्व‚व्य‚क्तिभिः स‚ह स‚म्ब‚न्ध‚श्च न स्यात् । कार्य‚कार‚ण‚{\tiny $_{lb}$}‚भावाभावे त‚स्येद‚मिति स‚म्ब‚न्ध‚स्यानुत्प‚त्तिरित्युक्तं‚{\tiny $_{6}$}‚ कार्य‚त्वं ह्य‚पेक्षेत्य‚भिधीय‚ते ‚{\tiny $_{lb}$}‚(२।२५) इत्य‚त्रान्त‚रेऽ‚{\color{DodgerBlue3}‚कार्य‚त्वाज्जातीनाम‚रूप}‚ता निःस्व‚भाव‚ता । उत्प‚द्य‚मानं हि ‚{\tiny $_{lb}$}‚स‚स्व‚भावं भ‚वेन्नेत‚र‚त् । नि\edtext{}{\edlabel{pvv.128-4}\label{pvv.128-4}\lemma{नि}\Bfootnote{लोकेनास‚म्म‚त‚त्वाद‚स‚न्मुख्यः प्र‚धानं कुतः । देव‚द‚त्तादिः स‚न्न‚मुख्यः}}याम‚काभावेन स्व‚भावाभाव‚प्र‚स‚ङ्गात् । (४४)
	\pend% ending standard par
      \label{div_pvv.2.45}
	  
	% new div opening: depth here is 2
	

	  \pstart \leavevmode% starting standard par
	शाब्द‚प्र‚त्य‚य‚स्याव‚स्तुविष‚य‚तायामुप‚प‚त्त्य‚न्त‚र‚माह (।)
	\pend% ending standard par
      \textsuperscript{\textenglish{129/s}}
	  \bigskip
	  \begingroup
	
	    \large
	  
	    \begin{quote}
	  
	    
	    \stanza[\smallbreak]
	\label{pv.2.45}\flagstanza{\tiny\textenglish{...v.2.45}}य‚च्च व‚स्तुब‚लाज्ज्ञानं जाय‚ते त‚द‚पेक्ष‚ते ।&न स‚ङ्केतं न सामान्य‚बुद्धिष्वेत‚द् विभाव्य‚ते ॥ ४५ ॥\&[\smallbreak]


	
	    \end{quote}
	  
	  \endgroup
	

	  \pstart \leavevmode% starting standard par
	\hphantom{.}‚{\color{DodgerBlue3}‚य‚च्च} ज्ञान‚मिन्द्रिय‚जं ‚{\color{DodgerBlue3}‚व‚स्तुनो रूपादेर्ब्ब‚लाज्जाय‚ते त‚न्न संकेत‚म‚पेक्ष‚ते,} बाल‚ब‚धिरादेर‚पि भावात् । ‚{\color{DodgerBlue3}‚सामान्य‚बुद्धिषु त्वेत‚त् संकेतान‚पेक्ष‚त्वं न विभाव्य‚ते} संकेत‚ग्र‚ह‚ण‚स्म‚र‚णापेक्ष‚त्वात् । त‚स्मा‚{\tiny $_{7}$}‚न्न व‚स्तुब‚ल‚भाविन्य‚स्ता\edtext{}{\edlabel{pvv.129-1}\label{pvv.129-1}\lemma{स्ता}\Bfootnote{सामान्य‚बुद्ध‚यः ।}}ः । त‚थात्वे स‚ती‚{\tiny $_{lb}$}‚न्द्रियार्थ‚स‚न्निपाते क्षेपायोगात्\edtext{}{\edlabel{pvv.129-2}\label{pvv.129-2}\lemma{क्षेपायोगात्}\Bfootnote{? संकेतापेक्षा न स्यात्}}। (४५)
	\pend% ending standard par
      \label{div_pvv.2.46}
	  
	% new div opening: depth here is 2
	
	  \bigskip
	  \begingroup
	
	    \large
	  
	    \begin{quote}
	  
	    
	    \stanza[\smallbreak]
	\label{pv.2.46}\flagstanza{\tiny\textenglish{...v.2.46}}याप्य‚भेदानुगा बुद्धिः काचिद् व‚स्तुद्व‚येक्ष‚णे ।&स‚ङ्केतेन विना सार्थ‚प्र‚त्यास‚त्तिनिब‚न्ध‚ना ॥ ४६ ॥\&[\smallbreak]


	
	    \end{quote}
	  
	  \endgroup
	

	  \pstart \leavevmode% starting standard par
	या\edtext{}{\edlabel{pvv.129-3}\label{pvv.129-3}\lemma{या}\Bfootnote{प्र‚त्य‚क्षं सामान्य‚मिति वादिनं श‚ङ्क‚ते ।}}पि ‚{\color{DodgerBlue3}‚संकेतेन विना व‚स्तुद्व‚येक्ष‚णे} शाव‚लेयं दृष्ट्वा बाहुलेयं प‚श्य‚तः ‚{\tiny $_{lb}$}‚अभेदानुग‚मात् स एवाय‚मित्य‚भेद‚म‚ध्य‚व‚स्य‚न्ती बुद्धिरुत्प‚द्य‚ते शाव‚लेय‚क‚र्क‚द‚र्श‚ने ‚{\tiny $_{lb}$}‚तु नोत्प‚द्य‚ते । ‚{\color{DodgerBlue3}‚साऽर्था\edtext{}{\edlabel{pvv.129-4}\label{pvv.129-4}\lemma{साऽर्था}\Bfootnote{सिद्धान्त‚य‚ति}}नां} शाव‚लेयादीनाम‚स‚त्य‚पि सामान्ये या ‚{\color{DodgerBlue3}‚प्र‚त्यास‚त्तिरे-} क‚बुद्ध्यादिकार्य‚त्वं त‚{\color{DodgerBlue3}‚न्निब‚न्ध‚ना} । (४६)
	\pend% ending standard par
      \label{div_pvv.2.47}
	  
	% new div opening: depth here is 2
	

	  \pstart \leavevmode% starting standard par
	सामान्यं विनैक‚कार्य‚तैव न स्यादित्याह (।)
	\pend% ending standard par
      
	  \bigskip
	  \begingroup
	
	    \large
	  
	    \begin{quote}
	  
	    
	    \stanza[\smallbreak]
	\label{pv.2.47}\flagstanza{\tiny\textenglish{...v.2.47}}प्र‚त्यास‚त्तिर्विना जात्या य‚थेष्टा च‚क्षुरादिषु ।&ज्ञान‚कार्येषु जातिर्वा य‚यान्वेति विभाग‚तः ॥ ४७ ॥\&[\smallbreak]


	
	    \end{quote}
	  
	  \endgroup
	

	  \pstart \leavevmode% starting standard par
	\hphantom{.}‚{\color{DodgerBlue3}‚प्र‚त्यास‚त्तिर्विना जान्या । च‚क्षुरादिर्ये}‚{\tiny $_{8}$}‚षां विष‚यालोक‚म‚न‚स्काराणां तेषु\leavevmode\ledsidenote{\textenglish{24b/MA}} ‚{\tiny $_{lb}$}‚ज्ञान‚कार्येषु ज्ञान‚हेतुषु जात्यादि विना रूप‚ज्ञानैक‚कार्य‚ज‚न‚क‚त्वं प्र‚त्यास‚त्ति‚{\color{DodgerBlue3}‚र्य‚थेष्टा} । ‚{\tiny $_{lb}$}‚य‚था वा शाव‚लेयाद्बाहुलेयादीनां क‚र्कादीनाञ्च तुल्ये भेदे ‚{\color{DodgerBlue3}‚य‚था} प्र‚त्यास‚त्या ‚{\tiny $_{lb}$}‚जात्य‚न्त‚रं विनैव ‚{\color{DodgerBlue3}‚विभाग‚तो जातिर‚न्वेति} । शाव‚लेय‚बाहुलेयादिष्वेव गोत्वं स‚म‚वेतं ‚{\tiny $_{lb}$}‚न क‚र्कादिषु । सैवानुगामिप्र‚त्य‚य‚निब‚न्ध‚न‚मास्ताम‚लं जात्या । (४७)
	\pend% ending standard par
      \label{div_pvv.2.48}
	  
	% new div opening: depth here is 2
	

	  \begin{center}%% label @type='head'
	\textbf{>ग. न च‚क्षुरादिभिः प्र‚त्येयं सामान्य‚म्}
	\end{center}
	

	  \pstart \leavevmode% starting standard par
	इत‚श्च न व‚स्तु सामान्यं (।)
	\pend% ending standard par
      
	  \bigskip
	  \begingroup
	
	    \large
	  
	    \begin{quote}
	  
	    
	    \stanza[\smallbreak]
	\label{pv.2.48a}\flagstanza{\tiny\textenglish{....2.48a}}क‚थ‚ञ्चिद‚पि विज्ञाने त‚द्रूपान‚व‚भास‚तः ।\&[\smallbreak]


	
	    \end{quote}
	  
	  \endgroup
	

	  \pstart \leavevmode% starting standard par
	\hphantom{.}दृश्य‚त्वेनाभिम‚त‚स्य ‚{\color{DodgerBlue3}‚त‚द्रूप‚स्य} स्व‚ग्राहिणि‚{\tiny $_{1}$}‚ ‚{\color{DodgerBlue3}‚विज्ञाने क‚थ‚ञ्चिद‚प्य‚न‚व‚भास‚तः} ।
	\pend% ending standard par
      \textsuperscript{\textenglish{130/s}}

	  \pstart \leavevmode% starting standard par
	न‚नु स‚न्त्य‚पीन्द्रियाणि नोप‚ल‚भ्य‚न्ते । त‚तोऽत्रानुप‚ल‚म्भाद‚स‚त्वं न । न हीन्द्रियाणि ‚{\tiny $_{lb}$}‚स्व‚ग्राहिणि ज्ञाने प्र‚त्य‚व‚भास‚मान‚त्वात् स‚न्तीष्य‚न्ते (।) किन्त‚र्हि (।) स‚त्स्व‚पि ‚{\tiny $_{lb}$}‚विष‚य‚म‚न‚स्कारादिषु क‚दाचित्प्र‚व‚र्त‚ते ज्ञानं क‚दाचिन्नेति व्य‚भिचार‚ब‚लाद‚तीन्द्रियाणि ‚{\tiny $_{lb}$}‚कानिचिदिन्द्रिय‚व्य‚प‚देश्यानि व्य‚व‚स्थाप्य‚न्ते ।
	\pend% ending standard par
      
	  \bigskip
	  \begingroup
	
	    \large
	  
	    \begin{quote}
	  
	    
	    \stanza[\smallbreak]
	\label{pv.2.48b}\flagstanza{\tiny\textenglish{....2.48b}}य‚दि नामेन्द्रियाणां स्याद् द्र‚ष्टा भासेत त‚द्व‚पुः ॥ ४८ ॥\&[\smallbreak]


	
	    \end{quote}
	  
	  \endgroup
	
	  \bigskip
	  \begingroup
	
	    \large
	  
	    \begin{quote}
	  
	    
	    \stanza[\smallbreak]
	\label{pv.2.49a}\flagstanza{\tiny\textenglish{....2.49a}}रूप‚व‚त्वात्;\&[\smallbreak]


	
	    \end{quote}
	  
	  \endgroup
	

	  \pstart \leavevmode% starting standard par
	\hphantom{.}य‚दि त्व‚तीन्द्रियाणाम‚तीन्द्रिय‚द‚र्शी द्र‚ष्टा नाम स्यात् । ‚{\color{DodgerBlue3}‚भासेत} रूप‚व‚त्वात्तेषा‚{\tiny $_{lb}$}‚मिन्द्रियाणां ‚{\color{DodgerBlue3}‚व‚पुः} । (४८)
	\pend% ending standard par
      \label{div_pvv.2.49}
	  
	% new div opening: depth here is 2
	
	  \bigskip
	  \begingroup
	
	    \large
	  
	    \begin{quote}
	  
	    
	    \stanza[\smallbreak]
	\label{pv.2.49b}\flagstanza{\tiny\textenglish{....2.49b}}न जातीनां केव‚लानाम‚द‚र्श‚नात् ।&व्य‚क्तिग्र‚हे च त‚च्छ्र‚ब्द‚रूपाद‚न्य‚न्न दृश्य‚ते ॥ ४९ ॥\&[\smallbreak]


	
	    \end{quote}
	  
	  \endgroup
	

	  \pstart \leavevmode% starting standard par
	\hphantom{.}‚{\color{DodgerBlue3}‚जाती}‚नान्तु दु‚{\tiny $_{2}$}‚श्याभिम‚तानां ‚{\color{DodgerBlue3}‚केव‚लानां} व्य‚क्तिस्व‚रूप‚व्य‚तिरिक्तानाम‚{\color{DodgerBlue3}‚द‚र्श‚ना‚{\tiny $_{lb}$}‚द‚भाव} एव । ‚{\color{DodgerBlue3}‚व्य‚क्तिग्र‚हे} त‚स्या व्य‚क्तेः । ‚{\color{DodgerBlue3}‚श‚ब्द‚स्य} गौरित्य‚स्य ‚{\color{DodgerBlue3}‚रूपाद‚न्य‚त्} सामान्यं ‚{\tiny $_{lb}$}‚‚{\color{DodgerBlue3}‚न दृश्य‚ते} । (४९)
	\pend% ending standard par
      \label{div_pvv.2.50}
	  
	% new div opening: depth here is 2
	
	  \bigskip
	  \begingroup
	
	    \large
	  
	    \begin{quote}
	  
	    
	    \stanza[\smallbreak]
	\label{pv.2.50}\flagstanza{\tiny\textenglish{...v.2.50}}ज्ञान‚मात्रार्थ‚क‚र‚णेप्य‚योग्य‚म‚त् एव त‚त् ।&त‚द‚योग्य‚त‚याऽरूपं त‚द्ध्य‚व‚स्तुषु ल‚क्ष‚ण‚म् ॥ ५० ॥\&[\smallbreak]


	
	    \end{quote}
	  
	  \endgroup
	

	  \pstart \leavevmode% starting standard par
	\hphantom{.}‚{\color{DodgerBlue3}‚अत एवा}‚दृश्य‚मान‚त्वात् ‚{\color{DodgerBlue3}‚ज्ञान‚मात्र}‚स्यार्थ‚स्यार्थ‚क्रियायाः ‚{\color{DodgerBlue3}‚क‚र‚णेप्य‚योग्य‚मेव त‚त्} । ‚{\tiny $_{lb}$}‚अन्त्या हीयं भावानाम‚र्थ‚क्रिया य‚दुत स्व‚ज्ञान‚ज‚न‚नं । त‚त्राप्य‚{\color{DodgerBlue3}‚योग्य‚त‚या} त‚त्सामान्य‚म‚{\tiny $_{lb}$}‚‚{\color{DodgerBlue3}‚रूपं} निःस्व‚भावं । हिर्य‚स्मात् स‚र्व्वार्थ‚क्रियायाम‚श‚क्त‚त्व‚{\color{DodgerBlue3}‚म‚व‚स्तुषु ल‚क्ष‚णं} स‚र्व‚सा‚{\tiny $_{lb}$}‚म‚र्थ्य‚र‚हितं‚{\tiny $_{3}$}‚ ह्य‚व‚स्त्विष्य‚ते । (५०)
	\pend% ending standard par
      \label{div_pvv.2.51}
	  
	% new div opening: depth here is 2
	

	  \pstart \leavevmode% starting standard par
	त‚था च सामान्य‚मिति न व‚स्तु\edtext{}{\edlabel{pvv.130-1}\label{pvv.130-1}\lemma{स्तु}\Bfootnote{रूपाहित‚वास‚नामाश्रित्योत्प‚त्तेः ।}} ।
	\pend% ending standard par
      
	  \bigskip
	  \begingroup
	
	    \large
	  
	    \begin{quote}
	  
	    
	    \stanza[\smallbreak]
	\label{pv.2.51a}\flagstanza{\tiny\textenglish{....2.51a}}य‚थोक्त‚विप‚रीतं य‚त् त‚त् स्व‚ल‚क्ष‚ण‚मिष्य‚ते ।\&[\smallbreak]


	
	    \end{quote}
	  
	  \endgroup
	

	  \pstart \leavevmode% starting standard par
	\hphantom{.}‚{\color{DodgerBlue3}‚य‚थो}‚क्तात्सामान्या‚{\color{DodgerBlue3}‚द्विप‚रीतं} य‚त्त‚{\color{DodgerBlue3}‚त्स्व‚ल‚क्ष‚ण}‚मुच्य‚ते । इद‚ञ्च वैप‚रीत्यं । अन‚{\tiny $_{lb}$}‚भिधेय‚त्वं । त‚त्वान्य‚त्वाभ्यां वाच्य‚त्वं । अस‚द‚र्थ‚प्र‚त्य‚याविशिष्ट‚प्र‚तिभास‚विष‚य‚त्वं (।) ‚{\tiny $_{lb}$}‚असाधार‚ण‚त्वं । संकेत‚स्म‚र‚णान‚पेक्ष‚प्र‚तिप‚त्तिक‚त्वं । अन्य‚रूप‚विविक्त‚स्व‚रूप‚प्र‚तिभास‚{\tiny $_{lb}$}‚व‚त्त्वं (।) अर्थ‚क्रियाक्ष‚म‚त्व‚ञ्च । एत‚द्युक्तं स्व‚ल‚क्ष‚ण‚मिष्य‚ते । एत‚द्विप‚र्य‚य‚स्याव‚स्तुत्व‚{\tiny $_{lb}$}‚साध‚न‚स्य सामान्य‚ल‚क्ष‚ण‚त्वात् ।
	\pend% ending standard par
      

	  \pstart \leavevmode% starting standard par
	य‚च्चा‚{\tiny $_{4}$}‚त‚त्कार्य‚व्य‚व‚च्छेद‚ल‚क्ष‚ण‚मुक्तं (।)
	\pend% ending standard par
      \textsuperscript{\textenglish{131/s}}
	  \bigskip
	  \begingroup
	
	    \large
	  
	    \begin{quote}
	  
	    
	    \stanza[\smallbreak]
	\label{pv.2.51b}\flagstanza{\tiny\textenglish{....2.51b}}सामान्यं त्रिविधं, त‚च्च भावाभावोभ‚याश्र‚यात् ॥ ५१ ॥\&[\smallbreak]


	
	    \end{quote}
	  
	  \endgroup
	

	  \pstart \leavevmode% starting standard par
	\hphantom{.}‚{\color{DodgerBlue3}‚सामान्यं त‚च्च त्रिविधं} बोद्ध‚व्यं ‚{\color{DodgerBlue3}‚भावाभावोभ‚याश्र‚यात्} । किञ्चिद् भावोपादानं ‚{\tiny $_{lb}$}‚सामान्यं य‚था रूपादीन् भावानाश्रित्य कृ\edtext{}{\edlabel{pvv.131-1}\label{pvv.131-1}\lemma{कृ}\Bfootnote{पंक्तिसेनादिव‚त् ।}}त‚क‚त्वादिश‚ब्द‚वाच्यं लिङ्गं । अभावो‚{\tiny $_{lb}$}‚पादानं य‚थोप‚ल‚ब्धिल‚क्ष‚ण‚प्राप्त‚स्यास‚तोऽनुप‚ल‚ब्धिर‚नुत्प‚त्तिम‚त्त्वादि च । त‚द्ध्य‚भावा‚{\tiny $_{lb}$}‚श्र‚यं सामान्यं लिङ्गं । उभ‚याश्र‚य‚म‚नुप‚ल‚ब्धिमात्रं\edtext{}{\edlabel{pvv.131-2}\label{pvv.131-2}\lemma{ब्धिमात्रं}\Bfootnote{किं पिशाचाद‚योऽत्य‚न्त‚म‚स‚न्तोऽथ स‚न्तोपि नेक्ष्य‚न्त इत्युभ‚योपादान‚त्वं प्र‚त्येतुं ‚{\tiny $_{lb}$}‚ज्ञान‚ध‚र्म‚मुक्त्वा विष‚य‚ध‚र्म‚माह ज्ञेय‚त्वादि ।}}ज्ञेय‚त्वादि च भावाभाव‚साध‚र‚ण‚{\tiny $_{lb}$}‚त्वात् । (५१)
	\pend% ending standard par
      \label{div_pvv.2.52}
	  
	% new div opening: depth here is 2
	
	  \bigskip
	  \begingroup
	
	    \large
	  
	    \begin{quote}
	  
	    
	    \stanza[\smallbreak]
	\label{pv.2.52}\flagstanza{\tiny\textenglish{...v.2.52}}य‚दि भावाश्र‚यं ज्ञानं भावे भावानुब‚न्ध‚तः ॥&नोक्तोत्त‚र‚त्वाद् दृष्ट‚त्वाद्; अतीतादिषु चान्य‚था ॥ ५२ ॥\&[\smallbreak]


	
	    \end{quote}
	  
	  \endgroup
	

	  \pstart \leavevmode% starting standard par
	\hphantom{.}‚{\color{DodgerBlue3}‚य‚दि} किञ्चित्सामान्यं ‚{\color{DodgerBlue3}‚भा‚{\tiny $_{5}$}‚वाश्र‚यं} त‚त् ‚{\color{DodgerBlue3}‚ज्ञानं त‚द्भावे} भाव‚विष‚यं प्राप्नोति । ‚{\tiny $_{lb}$}‚भावानुब‚न्ध‚तो भावान्व‚य‚व्य‚तिरेकानुविधानात् । नैत‚द्युक्तं न त‚द्व‚स्तु । अभि\edtext{}{\edlabel{pvv.131-3}\label{pvv.131-3}\lemma{अभि}\Bfootnote{य‚द्य‚पि भावानुविधानं पार‚म्प‚र्येण त‚थापि न साक्षाद्व‚स्तुविष‚य‚त्वं सिध्य‚ति ।}}धेय‚{\tiny $_{lb}$}‚त्वादिनोक्तोत्त‚र‚त्वात् । निषिद्धं हि सामान्य‚स्य व‚स्तुत्व‚म‚न‚न्त‚र‚मेव प्र‚प‚ञ्चेनेत्य‚{\tiny $_{lb}$}‚नुमान‚बाधित‚त्वं प्र‚तिज्ञा\edtext{}{\edlabel{pvv.131-4}\label{pvv.131-4}\lemma{तिज्ञा}\Bfootnote{भाव‚विष‚य‚त्व‚मिति प‚र‚प्र‚तिज्ञाऽभिधेय‚त्वादिनानुमानेन बाध्य‚ते । प‚रंप‚र‚या ‚{\tiny $_{lb}$}‚भावानुविधानेपि न साक्षाद्व‚स्तुविष‚य‚ता सिद्ध्य‚ति ।}}याः । दृष्ट‚त्वात् । ‚{\color{DodgerBlue3}‚अतीतादिषु चान्य‚था} व‚स्तुव्य‚तिरेकेणैवा‚{\tiny $_{lb}$}‚सीद् घ‚टो भ‚विष्य‚ति चेति सामान्य‚बुद्धिरुत्प‚द्य‚ते इत्य‚सिद्ध‚तापि भावानुब‚न्ध‚त ‚{\tiny $_{lb}$}‚इति हेतोः । (५२)
	\pend% ending standard par
      \label{div_pvv.2.53}
	  
	% new div opening: depth here is 2
	
	  \bigskip
	  \begingroup
	
	    \large
	  
	    \begin{quote}
	  
	    
	    \stanza[\smallbreak]
	\label{pv.2.53a}\flagstanza{\tiny\textenglish{....2.53a}}भाव‚ध‚र्म‚त्व‚हानिश्चेद् भाव‚ग्र‚ह‚ण‚पूर्व‚क‚म् ।&त‚ज्ज्ञान‚मित्य‚दोषोयं;\&[\smallbreak]


	
	    \end{quote}
	  
	  \endgroup
	

	  \pstart \leavevmode% starting standard par
	\hphantom{.}व‚स्तु विना सामान्य‚बुद्ध्युत्पादे ‚{\color{DodgerBlue3}‚भाव‚ध‚{\tiny $_{6}$}‚र्म‚त्व‚हानिः} सामान्य‚स्य प्र‚स‚ज्य‚ते चेत् । ‚{\tiny $_{lb}$}‚‚{\color{DodgerBlue3}‚भाव}‚स्य रूपादे‚{\color{DodgerBlue3}‚र्ग्र‚ह‚ण‚पूर्व्व‚कं}\edtext{}{\edlabel{pvv.131-5}\label{pvv.131-5}\lemma{रूपादे}\Bfootnote{रूपाहित‚वास‚नामाश्रित्योत्प‚त्तेः ।}} त‚स्य सामान्य‚स्य साधार‚ण‚बाह्य‚रूप‚त‚याऽध्य‚व‚सित‚{\tiny $_{lb}$}‚बुद्ध्याकार‚ल‚क्ष‚ण‚स्याध्य‚व‚सायेन ‚{\color{DodgerBlue3}‚ज्ञान‚मित्य}‚य‚म‚व‚स्तु\edtext{}{\edlabel{pvv.131-6}\label{pvv.131-6}\lemma{स्तु}\Bfootnote{व‚स्तुध‚र्म‚त्व‚हान्या ।}}ध‚र्म‚त्व‚ल‚क्ष‚णोऽदोषो ‚{\color{DodgerBlue3}‚दोषो} न ‚{\tiny $_{lb}$}‚भ‚व‚ति । न हि सामान्यं\edtext{}{\edlabel{pvv.131-7}\label{pvv.131-7}\lemma{सामान्यं}\Bfootnote{? भाव‚स्य स‚त्वे भ‚व‚त् । भावाश्र‚य‚त्वात् ।}} रूपादिरिव भाव‚रूप‚त‚या ज्ञान‚विष‚य इति भाव‚ध‚र्म इष्टं\edtext{}{\edlabel{pvv.131-8}\label{pvv.131-8}\lemma{इष्टं}\Bfootnote{? रूपादिरिव ।}} ।
	\pend% ending standard par
      \textsuperscript{\textenglish{132/s}}
	  \bigskip
	  \begingroup
	
	    \large
	  
	    \begin{quote}
	  
	    
	    \stanza[\smallbreak]
	\label{pv.2.53b}\flagstanza{\tiny\textenglish{....2.53b}}मेयं त्वेकं स्व‚ल‚क्ष‚णाम् ॥ ५३ ॥\&[\smallbreak]


	
	    \end{quote}
	  
	  \endgroup
	

	  \pstart \leavevmode% starting standard par
	किन्तु भाव‚वास‚नाप्र‚बोध‚प्र‚सूत‚विक‚ल्प‚क‚ल्पित‚त्वात्\edtext{}{\edlabel{pvv.132-1}\label{pvv.132-1}\lemma{त्वात्}\Bfootnote{य‚त एवं नान्य‚द‚र्थ‚क्रियाक्ष‚मं ।}} । प‚र‚मार्थ‚तो ‚{\color{DodgerBlue3}‚मेयं त्वेकं ‚{\tiny $_{lb}$}‚स्व‚ल‚क्ष‚णं} । त‚स्यैव रूप‚बुद्ध्युत्पाद‚क‚त्वात् । (५३)
	\pend% ending standard par
      \label{div_pvv.2.54}
	  
	% new div opening: depth here is 2
	
	  \bigskip
	  \begingroup
	
	    \large
	  
	    \begin{quote}
	  
	    
	    \stanza[\smallbreak]
	\label{pv.2.54a}\flagstanza{\tiny\textenglish{....2.54a}}त‚स्माद‚र्थ‚क्रियासिद्धेः स‚द‚स‚त्ताविचार‚णात् ॥\&[\smallbreak]


	
	    \end{quote}
	  
	  \endgroup
	\textsuperscript{\textenglish{25a/MA}}

	  \pstart \leavevmode% starting standard par
	\hphantom{.}‚{\color{DodgerBlue3}‚सामान्य‚स्य} तु क‚ल्पि‚{\tiny $_{7}$}‚त‚त्वात् । साम‚र्थ्याभावात्स्व‚ल‚क्ष‚ण‚मेकं प्र‚मेयं ‚{\color{DodgerBlue3}‚त‚स्माद‚{\tiny $_{lb}$}‚र्थ‚क्रियासिद्धेः} । अर्थ‚क्रियार्थिभिः ‚{\color{DodgerBlue3}‚स‚द‚स‚त्ताभ्यां} त‚स्यैव ‚{\color{DodgerBlue3}‚विचार‚णात्} । अस‚त्त्व‚म‚पि ‚{\tiny $_{lb}$}‚स्व‚ल‚क्ष‚ण\edtext{}{\edlabel{pvv.132-2}\label{pvv.132-2}\lemma{ण}\Bfootnote{व्याप्तिः प‚क्ष‚ध‚र्म‚ता चेत्य‚त्र नेदं वाच्यं । अन्व‚य‚व्य‚तिरेक‚प‚क्ष‚ध‚र्म‚ताकालेपि ‚{\tiny $_{lb}$}‚न वाच्यं । अन्व‚य‚व्य‚तिरेक‚मात्रे तु वाच्यं । अस्ति न वेति कृत्वा । दृष्ट‚त्वात}}स्यैव विचिन्त्य‚ते ।
	\pend% ending standard par
      

	  \pstart \leavevmode% starting standard par
	य‚द्येक‚मेव प्र‚मेयं त‚दा\edtext{}{\edlabel{pvv.132-3}\label{pvv.132-3}\lemma{दा}\Bfootnote{दिग्नागेन ।}}चार्येण प्र‚मेय‚द्वैविध्यं य‚दुक्तं न स्व‚सामान्य‚ल‚क्ष‚णाभ्या‚{\tiny $_{lb}$}‚म‚न्य‚त्प्र‚मेय‚म‚स्तीति त‚द् विरुध्य‚ते इत्याह ।
	\pend% ending standard par
      
	  \bigskip
	  \begingroup
	
	    \large
	  
	    \begin{quote}
	  
	    
	    \stanza[\smallbreak]
	\label{pv.2.54b}\flagstanza{\tiny\textenglish{....2.54b}}त‚स्य स्व‚प‚र‚रूपाभ्यां ग‚तेर्मेय‚द्व‚यं म‚त‚म् ॥ ५४ ॥\&[\smallbreak]


	
	    \end{quote}
	  
	  \endgroup
	

	  \pstart \leavevmode% starting standard par
	\hphantom{.}‚{\color{DodgerBlue3}‚त‚स्य} स्व‚ल‚क्ष‚ण‚स्य प्र‚त्य‚क्ष‚तः ‚{\color{DodgerBlue3}‚स्व}‚रूपेणानुमान‚तः ‚{\color{DodgerBlue3}‚प‚र‚रूपेण} सामान्याकारेण ‚{\tiny $_{lb}$}‚‚{\color{DodgerBlue3}‚ग‚तेर्मेय‚द्व‚यं म‚तं} न तु भूत‚सामान्य‚स्य स‚त्त्वात् (५४) ।
	\pend% ending standard par
      
	  
	% new div opening: depth here is 1
	
\chapter*[{४. अनुमान‚चिन्ता}]{४. अनुमान‚चिन्ता}

	  \begin{center}%% label @type='head'
	\textbf{(१) अनुमान‚सिद्धिः}
	\end{center}
	

	  \begin{center}%% label @type='head'
	\textbf{क. भ्रान्त‚म‚नुमानं प्र‚माणाम्}
	\end{center}
	\label{div_pvv.2.55}
	  
	% new div opening: depth here is 2
	
	  \bigskip
	  \begingroup
	
	    \large
	  
	    \begin{quote}
	  
	    
	    \stanza[\smallbreak]
	\label{pv.2.55a}\flagstanza{\tiny\textenglish{....2.55a}}अय‚थाभिनिवेशेन द्वितीया भ्रान्तिरिष्य‚ते ।\&[\smallbreak]


	
	    \end{quote}
	  
	  \endgroup
	

	  \pstart \leavevmode% starting standard par
	\hphantom{.}या च ‚{\color{DodgerBlue3}‚द्वितीया} प‚र‚रूपेण ग‚ति‚{\tiny $_{1}$}‚‚{\color{DodgerBlue3}‚र‚य‚थाभिनिवेशेन भ्रान्तिरिष्य‚ते} साऽनुमानं ‚{\tiny $_{lb}$}‚य‚थाऽर्थोस्ति य‚था वा स्वाकार‚स्त‚था नाभिनिविश‚ते किन्तु स्वाकारं बाह्यं साधार‚ण‚{\tiny $_{lb}$}‚त‚या म‚न्य‚ते ।
	\pend% ending standard par
      
	  \bigskip
	  \begingroup
	
	    \large
	  
	    \begin{quote}
	  
	    
	    \stanza[\smallbreak]
	\label{pv.2.55b}\flagstanza{\tiny\textenglish{....2.55b}}ग‚तिश्चेत् प‚र‚रूपेण न च भ्रान्तेः प्र‚माण‚ता ॥ ५५ ॥\&[\smallbreak]


	
	    \end{quote}
	  
	  \endgroup
	

	  \pstart \leavevmode% starting standard par
	\hphantom{.}न‚न्व‚नुमानं ‚{\color{DodgerBlue3}‚प‚र‚रूपेण ग‚तिश्चेत्} त‚दा भ्रान्तिरेव । ‚{\color{DodgerBlue3}‚न च भ्रान्तेः प्र‚माण‚ते}‚ष्य‚ते ‚{\tiny $_{lb}$}‚मृग‚तृष्णादेरिव । (५५)
	\pend% ending standard par
      \textsuperscript{\textenglish{133/s}}\label{div_pvv.2.56}
	  
	% new div opening: depth here is 2
	

	  \pstart \leavevmode% starting standard par
	अत्रोच्य‚ते ।
	\pend% ending standard par
      
	  \bigskip
	  \begingroup
	
	    \large
	  
	    \begin{quote}
	  
	    
	    \stanza[\smallbreak]
	\label{pv.2.56a}\flagstanza{\tiny\textenglish{....2.56a}}अभिप्रायाविसंवादाद‚पि भ्रान्तेः प्र‚माण‚ता ॥\edtext{\textsuperscript{*}}{\edlabel{pvv.133-asterisk}\label{pvv.133-asterisk}\lemma{*}\Bfootnote{द्र‚ष्ट‚व्यं प‚रिशिष्टं ।११}}\&[\smallbreak]


	
	    \end{quote}
	  
	  \endgroup
	
	  \bigskip
	  \begingroup
	
	    \large
	  
	    \begin{quote}
	  
	    
	    \stanza[\smallbreak]
	\label{pv.2.56b}\flagstanza{\tiny\textenglish{....2.56b}}ग‚तिर‚प्य‚न्य‚था दृष्टा;\&[\smallbreak]


	
	    \end{quote}
	  
	  \endgroup
	

	  \pstart \leavevmode% starting standard par
	\hphantom{.}‚{\color{DodgerBlue3}‚भ्रान्तेर‚पि प्र‚माण‚ता । अभिप्राय‚स्या}‚र्थ‚क्रियार्थिभिः ज्ञान‚गोच‚र‚त‚याऽभिप्राय‚{\tiny $_{lb}$}‚विष\edtext{}{\edlabel{pvv.133-1}\label{pvv.133-1}\lemma{विष}\Bfootnote{प्र‚वृत्तिविष‚य‚स्य ।}}यीकृत‚स्यार्थ‚क्रियास‚म‚र्थ‚स्यार्थ‚स्य ‚{\color{DodgerBlue3}‚संवादात्} । अर्थ‚क्रियार्थिनो हि त‚त्साध‚न‚स‚म‚{\tiny $_{lb}$}‚र्थार्थ‚प्राप‚कं‚{\tiny $_{2}$}‚ प्र‚माण‚मिच्छ‚न्ति । ‚{\color{DodgerBlue3}‚अन्य‚था} प‚र‚रूपेण ‚{\color{DodgerBlue3}‚ग‚तिर‚पि} काचिद‚भिप्रेतार्थ‚{\tiny $_{lb}$}‚स‚म्वादिका ‚{\color{DodgerBlue3}‚दृष्टे}‚ति प्र‚माण‚मेव ।
	\pend% ending standard par
      

	  \pstart \leavevmode% starting standard par
	न‚न्व‚नुमानं व‚स्त्वेव गृह्ण‚त् प्र‚माण‚म‚स्तु किं भ्रान्तिरिष्य‚ते इत्याह ।
	\pend% ending standard par
      
	  \bigskip
	  \begingroup
	
	    \large
	  
	    \begin{quote}
	  
	    
	    \stanza[\smallbreak]
	\label{pv.2.56c}\flagstanza{\tiny\textenglish{....2.56c}}प‚क्ष‚श्चायं कृतोत्त‚रः ॥ ५६ ॥\&[\smallbreak]


	
	    \end{quote}
	  
	  \endgroup
	

	  \pstart \leavevmode% starting standard par
	\hphantom{.}‚{\color{DodgerBlue3}‚प‚क्ष‚श्चायं} प्रागेव न त‚द्व‚स्तु अभिधेय‚त्वादित्यादिना (२।११) ‚{\color{DodgerBlue3}‚कृतोत्त‚रः} । (५६)
	\pend% ending standard par
      \label{div_pvv.2.57}
	  
	% new div opening: depth here is 2
	

	  \pstart \leavevmode% starting standard par
	न‚नु भ्रान्त‚म‚पि य‚द्य‚नुमानं प्र‚माणं त‚दा स‚र्व्वैव भ्रान्तिः प्र‚माणं स्यादित्याह ।
	\pend% ending standard par
      
	  \bigskip
	  \begingroup
	
	    \large
	  
	    \begin{quote}
	  
	    
	    \stanza[\smallbreak]
	\label{pv.2.57}\flagstanza{\tiny\textenglish{...v.2.57}}\edtext{\textsuperscript{*}}{\edlabel{pvv.133-asterisk-bis}\label{pvv.133-asterisk-bis}\lemma{*}\Bfootnote{द्र‚ष्ट‚व्यं प‚रिशिष्टं ।११}}म‚णिप्र‚दीप‚प्र‚भ‚योर्म‚णिबुद्ध्याभिधाव‚तोः ।&मिथ्याज्ञानाविशेषेपि विशेषोर्थ‚क्रियां प्राति ॥ ५७ ॥\&[\smallbreak]


	
	    \end{quote}
	  
	  \endgroup
	

	  \pstart \leavevmode% starting standard par
	\hphantom{.}‚{\color{DodgerBlue3}‚म‚णिप्र‚दीप‚यो}‚र्ये ‚{\color{DodgerBlue3}‚प्र‚भे} त‚यो‚{\color{DodgerBlue3}‚र्म‚णिबुद्ध्या} म‚णिरेवाय‚मित्य‚ध्य‚व‚सायेन त‚द्‏ग्र‚ह‚णार्थं ‚{\tiny $_{lb}$}‚‚{\color{DodgerBlue3}‚धाव‚तो मिथ्याज्ञान‚स्य} भ्रान्त‚त्व‚{\tiny $_{3}$}‚स्या‚{\color{DodgerBlue3}‚विशेषेपि विशेषोऽर्थ‚क्रियाम्प्र‚ति} । म‚णिप्र‚भायां‚{\tiny $_{lb}$}‚म‚प्य‚ध्य‚व‚सायी म‚णिसाध्याम‚र्थ‚क्रियां प्राप्नोति । (५७)
	\pend% ending standard par
      \label{div_pvv.2.58}
	  
	% new div opening: depth here is 2
	
	  \bigskip
	  \begingroup
	
	    \large
	  
	    \begin{quote}
	  
	    
	    \stanza[\smallbreak]
	\label{pv.2.58}\flagstanza{\tiny\textenglish{...v.2.58}}य‚था त‚थाऽय‚थार्थ‚त्वेप्य‚नुमान‚त‚दाभ‚योः ।&अर्थ‚क्रियानुरोधेन प्र‚माण‚त्वं व्य‚व‚स्थित‚म् ॥ ५८ ॥\&[\smallbreak]


	
	    \end{quote}
	  
	  \endgroup
	

	  \pstart \leavevmode% starting standard par
	\hphantom{.}दीप‚प्र‚भायान्तु म‚णिव्य‚व‚सायी त‚न्न प्राप्नोतीति ‚{\color{DodgerBlue3}‚य‚था त‚था} त्रिरूप‚लिङ्ग‚ज‚{\color{DodgerBlue3}‚म‚नु‚{\tiny $_{lb}$}‚मानं । त‚दा}‚भ‚ञ्च त‚न्न न त‚था । त‚योः स्वाकारे बाह्य‚ध्य‚व‚साय‚प्र‚वृत्त‚त्वात् अय‚था‚{\tiny $_{lb}$}‚र्थ‚त्वेपि ‚{\color{DodgerBlue3}‚प्र‚माण}‚त्वं ‚{\color{DodgerBlue3}‚व्य‚व‚स्थितं} । विशेषेणाव‚स्थित‚{\color{DodgerBlue3}‚म‚र्थ‚क्रियानुरोधेना}‚नुमान‚मेव ‚{\tiny $_{lb}$}‚प्र‚माणं । प‚र‚म्प‚र‚याऽर्थादुत्प‚त्तेः त‚त्प्राप‚क‚त्वात् । नेत‚र‚द्विप‚र्य‚या‚{\tiny $_{4}$}‚त् । त‚स्मात्प्र‚मेय‚{\tiny $_{lb}$}‚द्वित्वं ग‚तिभेदात् । (५८)
	\pend% ending standard par
      \label{div_pvv.2.59_2.60}
	  
	% new div opening: depth here is 2
	

	  \pstart \leavevmode% starting standard par
	त‚योर्ल‚क्ष‚णं ग्र‚ह‚ण‚ञ्चाख्यातुमाह ।
	\pend% ending standard par
      
	  \bigskip
	  \begingroup
	
	    \large
	  
	    \begin{quote}
	  
	    
	    \stanza[\smallbreak]
	\label{pv.2.59}\flagstanza{\tiny\textenglish{...v.2.59}}बुद्धिर्य‚त्रार्थ‚साम‚र्थ्याद‚न्व‚य‚व्य‚तिरेकिणी ।&त‚स्य स्व‚त‚त्रं ग्र‚ह‚ण‚म‚तोऽन्य‚द् व‚स्त्व‚तीन्द्रिय‚म् ॥ ५९ ॥\&[\smallbreak]


	
	    \end{quote}
	  
	  \endgroup
	\textsuperscript{\textenglish{134/s}}

	  \pstart \leavevmode% starting standard par
	अर्थ‚साम‚र्थ्याज्जाय‚माना बुद्धिर्य‚त्रान्व‚य‚व्य‚तिरेकिणी त‚त्स्व‚ल‚क्ष‚ण‚न्त‚स्य ग्र‚ह‚ण‚म‚{\tiny $_{lb}$}‚प‚राश्र‚यं स्व‚रूप‚ग्र‚ह‚णं स्व‚ज‚न्य‚याऽर्थ‚स्व‚रूप‚या बुद्ध्या साक्षात्त‚स्य ग्र‚ह‚णात् । अतः ‚{\tiny $_{lb}$}‚स्व‚ल‚क्ष‚ण‚त्वाद‚न्य‚द्व‚स्तु य‚द‚न्व‚य‚व्य‚तिरेकौ नानुक‚रोति बुद्धिः साक्षाद‚प्र‚तिभास‚मानं ‚{\tiny $_{lb}$}‚केव‚ल‚म‚ध्य‚व‚साय‚विष‚यः त‚त्सामान्य‚ल‚क्ष‚ण‚{\color{DodgerBlue3}‚म‚तीन्द्रियं} बुद्धिष्व‚प्र‚तिभा‚{\tiny $_{5}$}‚स‚नात् (५९) ।
	\pend% ending standard par
      

	  \pstart \leavevmode% starting standard par
	क‚थ‚न्त‚त्प्र‚त्येत‚व्य‚मित्याह (।)
	\pend% ending standard par
      
	  \bigskip
	  \begingroup
	
	    \large
	  
	    \begin{quote}
	  
	    
	    \stanza[\smallbreak]
	\label{pv.2.60}\flagstanza{\tiny\textenglish{...v.2.60}}त‚स्यादृष्टात्म‚रूप‚स्य ग‚तेर‚न्योर्थ आश्र‚यः ।&त‚दाश्र‚येण स‚म्ब‚न्धी य‚दि स्याद् ग‚म‚क‚स्त‚दा ॥ ६० ॥\&[\smallbreak]


	
	    \end{quote}
	  
	  \endgroup
	

	  \pstart \leavevmode% starting standard par
	\hphantom{.}‚{\color{DodgerBlue3}‚त‚स्य} सामान्य‚स्य स्व‚ल‚क्ष‚ण‚विशेषेण स्व‚ल‚क्ष‚ण‚विवेकेना‚{\color{DodgerBlue3}‚दृष्टात्म‚रूप‚स्य\edtext{}{\edlabel{pvv.134-1}\label{pvv.134-1}\lemma{स्य}\Bfootnote{साक्षान्न ग‚तिः । त‚त्स‚त्तामात्र‚स्य । ? त‚स्यैव क्ष‚ण‚ज्ञान‚स्य ।}}} ग‚तेः ‚{\tiny $_{lb}$}‚प्र‚तीतेः त‚स्माद‚{\color{DodgerBlue3}‚न्योऽर्थो} लिङ्ग‚भूत ‚{\color{DodgerBlue3}‚आश्र‚यः} सिद्धिनिमित्तं ‚{\color{DodgerBlue3}‚य‚दि} त‚तः प्र‚त्येत‚व्येना‚{\tiny $_{lb}$}‚न‚ग्निव्य‚व‚च्छेदादिना ‚{\color{DodgerBlue3}‚त‚दाश्र‚येण} च ध‚र्मिणा ‚{\color{DodgerBlue3}‚स‚म्ब‚न्धी} स‚म्ब‚न्ध‚वान् ‚{\color{DodgerBlue3}‚स्यात्त‚दा ‚{\tiny $_{lb}$}‚ग‚म‚को} नान्य‚था । अन‚ने साध्य‚प्र‚तिब‚न्धोऽन्व‚य‚व्य‚तिरेक‚रूपः प‚क्ष‚ध‚र्म‚ता च ‚{\tiny $_{lb}$}‚लिङ्ग‚स्योक्ता (६०) ।
	\pend% ending standard par
      \label{div_pvv.2.61}
	  
	% new div opening: depth here is 2
	

	  \pstart \leavevmode% starting standard par
	न‚न्व‚न्याश्र‚येण च स्व‚रूप‚प्र‚तीतिर्भ‚विष्य‚ति त‚थापि प‚रो‚{\tiny $_{6}$}‚क्ष‚ता क‚थ‚मित्याह (।)
	\pend% ending standard par
      
	  \bigskip
	  \begingroup
	
	    \large
	  
	    \begin{quote}
	  
	    
	    \stanza[\smallbreak]
	\label{pv.2.61}\flagstanza{\tiny\textenglish{...v.2.61}}ग‚म‚कानुग‚सामान्य‚रूपेणैव त‚दा ग‚तिः ।&त‚स्मात् स‚र्वः प‚रोक्षोर्थो विशेषेण न ग‚म्य‚ते ॥ ६१ ॥\&[\smallbreak]


	
	    \end{quote}
	  
	  \endgroup
	

	  \pstart \leavevmode% starting standard par
	\hphantom{.}‚{\color{DodgerBlue3}‚त‚दाऽर्था}‚न्त‚रात्प्र‚तीतिकाले ‚{\color{DodgerBlue3}‚ग‚म‚कं} प‚क्ष‚स‚प‚क्षानुयायि लिङ्गं ।\edtext{\textsuperscript{*}}{\edlabel{pvv.134-2}\label{pvv.134-2}\lemma{*}\Bfootnote{हेतुस‚त्त‚यैव साध्य‚स‚त्त्वात् ।}} त‚द‚नुयायिना ‚{\tiny $_{lb}$}‚त‚द्‏व्याप‚केन ‚{\color{DodgerBlue3}‚सामान्य‚रूपेणैव} प‚रोक्षोर्थो ग‚म्य‚ते । न तु स‚र्व्व‚तो व्यावृत्तेन विशिष्टेन ‚{\tiny $_{lb}$}‚रूपेण । ‚{\color{DodgerBlue3}‚त‚स्मात्स‚र्वः प‚रोक्षोऽर्थः} प्र‚तीय‚मानो ‚{\color{DodgerBlue3}‚न विशेषेण} क‚थ‚ञ्चिद् ‚{\color{DodgerBlue3}‚ग‚म्य‚ते} येन ‚{\tiny $_{lb}$}‚प‚रोक्ष‚ताहानिः स्यात् । (६१)
	\pend% ending standard par
      \label{div_pvv.2.62}
	  
	% new div opening: depth here is 2
	
	  \bigskip
	  \begingroup
	
	    \large
	  
	    \begin{quote}
	  
	    
	    \stanza[\smallbreak]
	\label{pv.2.62}\flagstanza{\tiny\textenglish{...v.2.62}}या च स‚म्ब‚न्धिनो ध‚र्माद् भूतिर्ध‚र्मिणि ज्ञाय‚ते ।&सानुमानं प‚रोक्षाणामेकान्तेनैव साध‚न‚म् ॥ ६२ ॥\&[\smallbreak]


	
	    \end{quote}
	  
	  \endgroup
	

	  \pstart \leavevmode% starting standard par
	\hphantom{.}‚{\color{DodgerBlue3}‚या च स‚म्ब‚न्धिनो ध‚र्माद}‚न्व‚य‚व्य‚तिरेक‚तो लिङ्गात् त‚दाश्र‚ये ‚{\color{DodgerBlue3}‚ध‚र्मिणि ज्ञाय‚ते} प‚रोक्षार्थ‚प्र‚तीतिः । ‚{\color{DodgerBlue3}‚सानुमानं} त्रिरूप‚लि‚{\tiny $_{7}$}‚ङ्ग‚प्र‚भ‚व‚त्त्वात् । त‚देवानुमानं ‚{\color{DodgerBlue3}‚प‚रोक्षाणा‚{\tiny $_{lb}$}‚मेकान्तेनैव साध‚नं} । प्र‚त्य‚क्ष‚स्य त‚त्रावृत्तेः । (६२)
	\pend% ending standard par
      \label{div_pvv.2.63}
	  
	% new div opening: depth here is 2
	
	  \bigskip
	  \begingroup
	
	    \large
	  
	    \begin{quote}
	  
	    
	    \stanza[\smallbreak]
	\label{pv.2.63}\flagstanza{\tiny\textenglish{...v.2.63}}न प्र‚त्य‚क्ष‚प‚रोक्षाभ्यां मेय‚स्यान्य‚स्य स‚म्भ‚वः ।&त‚स्मात् प्र‚मेय‚द्वित्वेन प्र‚माण‚द्वित्व‚मिष्य‚ते ॥ ६३ ॥\&[\smallbreak]


	
	    \end{quote}
	  
	  \endgroup
	

	  \pstart \leavevmode% starting standard par
	\hphantom{.}‚{\color{DodgerBlue3}‚न} च ‚{\color{DodgerBlue3}‚प्र‚त्य‚क्ष‚प‚रोक्षाभ्याम‚न्य‚स्य प्र‚मेय‚स्य स‚म्भ‚व} इति द‚र्शितं प्राक् । ‚{\color{DodgerBlue3}‚त‚स्मात्प्र‚मे‚{\tiny $_{lb}$}‚य‚स्य द्वित्वेन प्र‚माण‚द्वित्व‚मिष्य‚ते} । प्र‚मेय‚म‚धिग‚च्छ‚त् प्र‚माण‚मुच्य‚ते । (६३)
	\pend% ending standard par
      \textsuperscript{\textenglish{135/s}}\label{div_pvv.2.64}
	  
	% new div opening: depth here is 2
	

	  \begin{center}%% label @type='head'
	\textbf{ख. द्वितीयं प्र‚माण‚म‚नुमान‚म्}
	\end{center}
	
	  \bigskip
	  \begingroup
	
	    \large
	  
	    \begin{quote}
	  
	    
	    \stanza[\smallbreak]
	\label{pv.2.64a}\flagstanza{\tiny\textenglish{....2.64a}}त्र्येक‚संख्यानिरासो वा प्र‚मेय‚द्व‚य‚द‚र्श‚नात् ॥\&[\smallbreak]


	
	    \end{quote}
	  
	  \endgroup
	

	  \pstart \leavevmode% starting standard par
	\hphantom{.}त‚च्च द्विध‚मिति त‚द्‏ग्राह‚कं द्व‚य‚म‚पि प्र‚माण‚मेव ॥ ‚{\color{DodgerBlue3}‚प्र‚मेय‚द्व‚य‚स्य द‚र्श‚नात्} । ‚{\tiny $_{lb}$}‚‚{\color{DodgerBlue3}‚त्र्येक‚संख्यानिरासो वा} बोद्ध‚व्यः (।) तृतीयादिकं न प्र‚माणं तृतीयादिप्र‚मेयाभावात् । ‚{\tiny $_{lb}$}‚नाप्येकं‚{\tiny $_{8}$}‚ द्वितीय‚स्य प्र‚मेय‚स्य तेनान‚धिग‚तेः । न हि प्र‚त्य‚क्षं स्व‚ल‚क्ष‚ण‚साम‚र्थ्यात् त‚दा-\leavevmode\ledsidenote{\textenglish{25b/MA}} ‚{\tiny $_{lb}$}‚कार‚ग्राहि जातं सामान्यं प्र‚त्येति क‚ल्प‚नाग‚म्य‚त्वात्त‚स्य ॥
	\pend% ending standard par
      
	  \bigskip
	  \begingroup
	
	    \large
	  
	    \begin{quote}
	  
	    
	    \stanza[\smallbreak]
	\label{pv.2.64b}\flagstanza{\tiny\textenglish{....2.64b}}एक‚मेवाप्र‚मेय‚त्वाद‚स‚त‚श्चेन्म‚तं च नः ॥ ६४ ॥\&[\smallbreak]


	
	    \end{quote}
	  
	  \endgroup
	

	  \pstart \leavevmode% starting standard par
	\hphantom{.}न‚नु प्र‚त्य‚क्ष‚{\color{DodgerBlue3}‚मेक‚मेव} प्र‚माण‚म‚स‚तो‚{\color{DodgerBlue3}‚ऽप्र‚मेय‚त्वात्} । अस‚च्च सामान्य‚म‚त्राह (।) ‚{\tiny $_{lb}$}‚‚{\color{DodgerBlue3}‚अस‚तोऽप्र‚मेय‚त्वं म\edtext{}{\edlabel{pvv.135-1}\label{pvv.135-1}\lemma{म}\Bfootnote{चार्व्वाकं ‚{\color{DodgerBlue3}‚प्र‚ति सिद्ध‚साध‚न‚माह ? त‚त्स‚त्तामात्र‚स्य} ।}}त‚ञ्च नः} । (६४)
	\pend% ending standard par
      \label{div_pvv.2.65_2.66}
	  
	% new div opening: depth here is 2
	
	  \bigskip
	  \begingroup
	
	    \large
	  
	    \begin{quote}
	  
	    
	    \stanza[\smallbreak]
	\label{pv.2.65a}\flagstanza{\tiny\textenglish{....2.65a}}अनेकान्तोऽप्र‚मेय‚त्वेऽस‚द्भाव‚स्य विनिश्च‚यः ।&त‚न्निश्च‚य‚प्र‚माणां वा द्वितीयं;\&[\smallbreak]


	
	    \end{quote}
	  
	  \endgroup
	

	  \pstart \leavevmode% starting standard par
	किम‚निष्ट‚माप‚द्य‚ते । स्व‚ल‚क्ष‚ण‚मेव तु प‚र‚रूपेण ग‚तेः सामान्य‚ल‚क्ष‚ण‚मिष्टं । ‚{\tiny $_{lb}$}‚त‚च्च स‚देवेति क‚थ‚म‚प्र‚मेयं त‚त‚स्त‚त्साध‚न‚म‚पि प्र‚माण‚मेव ॥ त‚थाऽस‚तोऽ‚{\color{DodgerBlue3}‚प्र‚मेय‚त्वे} साध्ये‚{\tiny $_{lb}$}‚ऽस‚त्त्वादिति हेतुर‚{\color{DodgerBlue3}‚नेकान्तोपि} । त‚था‚{\tiny $_{1}$}‚ हि प‚र‚लोकादेर‚स‚त्त‚या चा र्व्वा के णापीष्य‚त ‚{\tiny $_{lb}$}‚एव केनापि प्र‚माणेन निश्च‚यः । अथ‚वा प्र‚मेय‚त्वाभाव‚स्याप्य‚स‚त्त्व‚हेतुनैव निश्च‚य ‚{\tiny $_{lb}$}‚इति व्य‚क्त‚म‚नैकान्तिक‚त्वं । अत‚श्च य‚त एव प्र‚माणात्त‚स्या‚{\color{DodgerBlue3}‚भाव‚स्य निश्च‚य}‚स्त‚देव‚{\tiny $_{lb}$}‚द्वितीय‚म्प्र‚माण‚म‚नुमानं नाध्य‚क्षं ।
	\pend% ending standard par
      

	  \pstart \leavevmode% starting standard par
	क‚स्मादेव‚मित्याह (।)
	\pend% ending standard par
      
	  \bigskip
	  \begingroup
	
	    \large
	  
	    \begin{quote}
	  
	    
	    \stanza[\smallbreak]
	\label{pv.2.65b}\flagstanza{\tiny\textenglish{....2.65b}}नाक्ष‚जा म‚तिः ॥ ३५ ॥\&[\smallbreak]


	
	    \end{quote}
	  
	  \endgroup
	
	  \bigskip
	  \begingroup
	
	    \large
	  
	    \begin{quote}
	  
	    
	    \stanza[\smallbreak]
	\label{pv.2.66}\flagstanza{\tiny\textenglish{...v.2.66}}अभावेऽर्थ‚ब‚लाज्जातेर‚र्थ‚श‚क्त्य‚न‚पेक्ष‚णो ।&व्य‚व‚धानादिभावेपि जायेतेन्द्रिय‚जा म‚तिः ॥ ६६ ॥\&[\smallbreak]


	
	    \end{quote}
	  
	  \endgroup
	

	  \pstart \leavevmode% starting standard par
	\hphantom{.}‚{\color{DodgerBlue3}‚ना\edtext{}{\edlabel{pvv.135-2}\label{pvv.135-2}\lemma{ना}\Bfootnote{प‚र‚लोकादेः प्र‚त्य‚क्षादेवाभाव‚निश्च‚योऽस्तीत्याह ।}}क्ष‚जा म‚ति}‚र‚भावे विष‚ये प्र‚व‚र्त‚तेऽर्थ‚स्य ग्राह्य‚स्य ब‚लाज्जातेः । य‚द्ब‚लेन प्र‚त्य‚क्षं ‚{\tiny $_{lb}$}‚प्र‚व‚र्त‚ते त‚देव\edtext{}{\edlabel{pvv.135-3}\label{pvv.135-3}\lemma{देव}\Bfootnote{प्र‚त्य‚क्षं निश्चिनोति ।}}प्र‚तिप‚द्य‚ते न चाभाव‚स्य साम‚र्थ्य‚{\tiny $_{2}$}‚ नाम । य‚दि पुन‚र‚र्थ‚साम‚र्थ्यान‚{\tiny $_{lb}$}‚पेक्ष‚ण‚म‚स्य त‚दा ग्राह्य‚स्यार्थ‚स्य श‚क्त्य‚न‚पेक्ष‚णे त‚द्‏व्य‚व‚धानादिभावेपीन्द्रिय‚जा म‚ति‚{\tiny $_{lb}$}‚र्जायेत । न चैत‚द‚स्ति । त‚तोऽर्थ‚साम‚र्थ्यापेक्षि नाभाव‚विष‚यं भ‚वितुम‚र्ह‚ति । (६५,६६)
	\pend% ending standard par
      \label{div_pvv.2.67}
	  
	% new div opening: depth here is 2
	

	  \pstart \leavevmode% starting standard par
	स्यादेत‚त् (।)
	\pend% ending standard par
      \textsuperscript{\textenglish{136/s}}
	  \bigskip
	  \begingroup
	
	    \large
	  
	    \begin{quote}
	  
	    
	    \stanza[\smallbreak]
	\label{pv.2.67}\flagstanza{\tiny\textenglish{...v.2.67}}अभावे विनिवृत्तिश्चेत् प्र‚त्य‚क्ष‚स्यैव निश्च‚यः ।&विरुद्धं सैव वा लिङ्ग‚म‚न्व‚य‚व्य‚तिरेकिणी ॥ ६७ ॥\&[\smallbreak]


	
	    \end{quote}
	  
	  \endgroup
	

	  \pstart \leavevmode% starting standard par
	\hphantom{.}‚{\color{DodgerBlue3}‚प्र‚त्य‚क्ष‚स्यैव} प्र‚वृत्तिर्भावे स‚त्त्व‚निश्च‚यो ‚{\color{DodgerBlue3}‚विनिवृत्तिश्चाभावे} नि\edtext{}{\edlabel{pvv.136-1}\label{pvv.136-1}\lemma{नि}\Bfootnote{प्र‚त्य‚क्ष‚निवृत्त्याभाव‚स्य निश्च‚य‚श्च स्यात् ।}}श्च‚यो न तु ‚{\tiny $_{lb}$}‚प्र‚माणान्त‚र‚वृत्तिरिति चेत् ।\edtext{\textsuperscript{*}}{\edlabel{pvv.136-2}\label{pvv.136-2}\lemma{*}\Bfootnote{स्व‚व‚च‚न‚विरुद्धं ।}} विरुद्ध‚मिदं यः प्र‚त्य‚क्षाभावान्निश्च‚यः स प्र‚त्य‚क्षादिति । ‚{\tiny $_{lb}$}‚न हि प्र‚त्य‚क्ष‚त‚निवृत्त्योरैका\edtext{}{\edlabel{pvv.136-3}\label{pvv.136-3}\lemma{निवृत्त्योरैका}\Bfootnote{ऐकात्म्य‚भाव एव स्यात् । ? अवैति । स्व‚पुत्रादौ ।}}त्म्यं‚{\tiny $_{3}$}‚ त‚था ह्य‚भावो भाव एव स्यात् । भावोपि चाभावः । ‚{\tiny $_{lb}$}‚न च प्र‚त्य‚क्ष‚त‚निवृत्ताव‚प्य‚व‚श्य‚म‚भावः । व्य‚व‚धानादिष्व‚र्थ‚स‚त्त्वेपि त‚स्याभावात् । अथा‚{\tiny $_{lb}$}‚र्थान्व‚य‚व्य‚तिरेकानुविधायिनी प्र‚त्य‚क्ष‚निवृत्तिरेक‚ज्ञान‚संस‚र्गिप‚दार्थान्त‚रोप‚ल‚ब्धि‚{\tiny $_{lb}$}‚रूपाऽभाव‚निश्च‚य‚हेतुस्त‚दा ‚{\color{DodgerBlue3}‚सैव} प्र‚त्य‚क्ष‚निवृत्तिर‚नुप‚ल‚ब्ध्याख्या‚{\color{DodgerBlue3}‚न्व‚य‚व्य‚तिरेकिणी ‚{\tiny $_{lb}$}‚लिङ्ग‚मिति} त‚ज्जा प्र‚तीतिर‚नुमान‚मेवेति क‚थं नाप्र‚त्य‚क्षं प्र‚माणं ।‚{\tiny $_{4}$}‚ (६७)
	\pend% ending standard par
      \label{div_pvv.2.68}
	  
	% new div opening: depth here is 2
	

	  \begin{center}%% label @type='head'
	\textbf{(ग. प्र‚माणाद्व‚य‚सिद्धिः)}
	\end{center}
	

	  \pstart \leavevmode% starting standard par
	किञ्च (।)
	\pend% ending standard par
      
	  \bigskip
	  \begingroup
	
	    \large
	  
	    \begin{quote}
	  
	    
	    \stanza[\smallbreak]
	\label{pv.2.68a}\flagstanza{\tiny\textenglish{....2.68a}}सिद्धं च प‚र‚चैत‚न्य‚प्र‚तिप‚त्तेः प्र‚माद्व‚य‚म् ।\&[\smallbreak]


	
	    \end{quote}
	  
	  \endgroup
	

	  \pstart \leavevmode% starting standard par
	\hphantom{.}‚{\color{DodgerBlue3}‚प‚र‚चैत‚न्य‚प्र‚तिप‚त्तेः प्र‚माण‚द्व‚यं सिद्धं} । न हि प्र‚त्य‚क्षाद‚र्व्वाग्द‚र्श‚नः प‚र‚चैत‚न्य‚म‚{\tiny $_{lb}$}‚वैति । किन्तु स्व‚स‚न्ताने बुद्धिपूर्व‚क‚त्वेनोप‚ल‚ब्ध‚चेष्टादिद‚र्श‚नात्त‚द‚नुमानं प‚र‚चित्त‚{\tiny $_{lb}$}‚निश्च‚यः किम‚स्तीत्याह (।)
	\pend% ending standard par
      
	  \bigskip
	  \begingroup
	
	    \large
	  
	    \begin{quote}
	  
	    
	    \stanza[\smallbreak]
	\label{pv.2.68b}\flagstanza{\tiny\textenglish{....2.68b}}व्य‚व‚हारादौ प्र‚वृत्तेश्च सिद्ध‚स्त‚द्भाव‚निश्च‚यः ॥ ६८ ॥\&[\smallbreak]


	
	    \end{quote}
	  
	  \endgroup
	

	  \pstart \leavevmode% starting standard par
	\hphantom{.}प‚र‚स्प‚र‚प्रेष‚णाध्येष‚ण‚व्य‚व‚हारादौ ‚{\color{DodgerBlue3}‚प्र‚वृत्तेश्च सिद्ध‚श्च} त‚स्याः प‚र‚चैत‚न्य‚प्र‚तिप‚त्ते‚{\tiny $_{lb}$}‚‚{\color{DodgerBlue3}‚र्भाव‚निश्च‚यः} । (६८)
	\pend% ending standard par
      \label{div_pvv.2.69}
	  
	% new div opening: depth here is 2
	

	  \begin{center}%% label @type='head'
	\textbf{(घ. अविसंवादाद‚नुमानं प्र‚माण‚म्)}
	\end{center}
	

	  \pstart \leavevmode% starting standard par
	न‚न्व‚नुमानाभिम‚ता प्र‚तीतिर्नास्त्येवेति न ब्रूमः । किन्तु प्रामाण्ये त‚स्याः विप्र‚ति‚{\tiny $_{lb}$}‚प‚{\tiny $_{5}$}‚द्याम‚ह इत्याह (।)
	\pend% ending standard par
      
	  \bigskip
	  \begingroup
	
	    \large
	  
	    \begin{quote}
	  
	    
	    \stanza[\smallbreak]
	\label{pv.2.69}\flagstanza{\tiny\textenglish{...v.2.69}}प्र‚माण‚म‚विसंवादात् त‚त् क्व‚चिद् व्य‚भिचार‚तः ।&नाश्वास इति चेल्लिङ्गं-दुर्दृष्टिर‚त‚दीदृश‚म् ॥ ६९ ॥\&[\smallbreak]


	
	    \end{quote}
	  
	  \endgroup
	

	  \pstart \leavevmode% starting standard par
	\hphantom{.}‚{\color{DodgerBlue3}‚प्र‚माण‚न्त}‚द‚नुमान‚म‚{\color{DodgerBlue3}‚विस‚म्वादात्}‚। प्र‚त्य‚क्ष‚म‚पि हि स‚म्वाद‚क‚त्वात् प्र‚माणं ‚{\tiny $_{lb}$}‚त‚च्चानुमान‚स्यापि स‚मानं । ‚{\color{DodgerBlue3}‚क्व‚चि}‚च्छ्याम‚तादिसाध‚नार्थ‚मुपात्ते त‚त्पुत्रादौ लिङ्गे ‚{\tiny $_{lb}$}‚\leavevmode\ledsidenote{\textenglish{137/s}} ‚{\color{DodgerBlue3}‚व्य‚भिचार‚तः} स‚म्वादेऽ‚{\color{DodgerBlue3}‚नाश्वास} इति चेत् । लिङ्ग‚स्य ‚{\color{DodgerBlue3}‚दुर्वृष्टि}‚र्भ्रान्तिर‚लिङ्ग एव ‚{\tiny $_{lb}$}‚लिङ्ग‚बुद्धिस्त‚स्या एत‚ल्लि‚{\color{DodgerBlue3}‚ङ्ग‚मीदृशं} व्य‚भिचारि भ‚व‚तः प्र‚तिभाति । न ख‚लु त्रिविधं ‚{\tiny $_{lb}$}‚लिङ्गं साध्य‚व्य‚भिचारि । य‚च्च व्य‚भिचारि त‚त्त्रिविध‚मे‚{\tiny $_{6}$}‚व न भ‚व‚ति । (६९)
	\pend% ending standard par
      \label{div_pvv.2.70}
	  
	% new div opening: depth here is 2
	

	  \pstart \leavevmode% starting standard par
	एत‚देवाह (।)
	\pend% ending standard par
      
	  \bigskip
	  \begingroup
	
	    \large
	  
	    \begin{quote}
	  
	    
	    \stanza[\smallbreak]
	\label{pv.2.70}\flagstanza{\tiny\textenglish{...v.2.70}}य‚तः क‚दाचित्सिद्धाऽस्य प्र‚तीतिर्व‚स्तुनः क्व‚चित्‚{\tiny $_{1}$}‚ ।&त‚द‚व‚श्यं त‚तो जातं त‚त्स्व‚भावोपि वा भ‚वेत् ॥ ७० ॥\&[\smallbreak]


	
	    \end{quote}
	  
	  \endgroup
	

	  \pstart \leavevmode% starting standard par
	\hphantom{.}‚{\color{DodgerBlue3}‚य‚तो व‚स्तुनो} धूम‚शिंश‚पादेः लिङ्गा‚{\color{DodgerBlue3}‚द‚स्य} साध्य‚स्य व‚ह्रिवृक्षादेः ‚{\color{DodgerBlue3}‚क्व‚चिद्ध‚र्मिणि ‚{\tiny $_{lb}$}‚क‚दाचिद}‚नुमान‚काले ‚{\color{DodgerBlue3}‚प्र‚तीतिः सिद्धा त‚द्} धूमादिकं ‚{\color{DodgerBlue3}‚त‚तो} व‚ह्न्यादेर‚{\color{DodgerBlue3}‚व‚श्यं जातं} । ‚{\tiny $_{lb}$}‚त‚च्छिंश‚पादिकं त‚स्य वृक्ष‚स्य ‚{\color{DodgerBlue3}‚स्व‚भावोपि वाऽव‚श्य‚म्भ‚वेत्} । (७०)
	\pend% ending standard par
      \label{div_pvv.2.71}
	  
	% new div opening: depth here is 2
	

	  \pstart \leavevmode% starting standard par
	न च कार्य‚स्व‚भाव‚योः कार‚ण‚व्याप‚क‚व्य‚भिचारो य‚स्मात् (।)
	\pend% ending standard par
      
	  \bigskip
	  \begingroup
	
	    \large
	  
	    \begin{quote}
	  
	    
	    \stanza[\smallbreak]
	\label{pv.2.71}\flagstanza{\tiny\textenglish{...v.2.71}}स्व‚निमित्तात् स्व‚भावाद् वा विना नार्थ‚स्य स‚म्भ‚वः ॥&य‚च्च रूपं त‚योर्दृष्टं तेद‚वान्य‚त्र ल‚क्ष‚ण‚म् ॥ ७१ ॥\&[\smallbreak]


	
	    \end{quote}
	  
	  \endgroup
	

	  \pstart \leavevmode% starting standard par
	\hphantom{.}‚{\color{DodgerBlue3}‚स्व‚स्य निमित्ता}‚त्कार‚णाद्विना, ‚{\color{DodgerBlue3}‚स्व‚भावाद्} व्याप‚काद्वा ‚{\color{DodgerBlue3}‚विनाऽर्थ‚स्य} कार्य‚{\tiny $_{7}$}‚स्य\leavevmode\ledsidenote{\textenglish{26a/MA}} ‚{\tiny $_{lb}$}‚व्याप्य‚स्य च न ‚{\color{DodgerBlue3}‚स‚म्भ‚वः} । त‚दुत्पाद्य‚त्वात्त‚द्रूप‚त्वाच्च । त‚त‚स्त‚दाश्र‚येणोत्प‚न्नाऽनुमान‚{\tiny $_{lb}$}‚प्र‚तीतिर‚व्य‚भिचारिण्येव ‚{\color{DodgerBlue3}‚य‚च्च त‚योर्धू}‚म‚शिंश‚प‚यो ‚{\color{DodgerBlue3}‚रूपं} साध्य‚कार्य‚स्व‚भाव\edtext{}{\edlabel{pvv.137-1}\label{pvv.137-1}\lemma{भाव}\Bfootnote{साध्येन स‚ह कार्य‚स्य स्व‚भाव‚स्य चाव्य‚भिचार‚निमित्तं ।}}त्व‚स्या‚{\tiny $_{lb}$}‚व्य‚भिचार‚निमित्तं ‚{\color{DodgerBlue3}‚दृष्टं त‚देवान्य}‚त्रापि हेतौ ‚{\color{DodgerBlue3}‚ल‚क्ष‚णं} बोद्ध‚व्यं । न च त‚त्पुत्र‚त्वं ‚{\tiny $_{lb}$}‚श्याम‚त्व‚स्य कार्यं स्व‚भावो वा । त‚तो य‚द् व्य‚भिचारि त‚द‚लिङ्ग‚मेव । (७१)
	\pend% ending standard par
      \label{div_pvv.2.72}
	  
	% new div opening: depth here is 2
	

	  \begin{center}%% label @type='head'
	\textbf{(ङ. अनुप‚ल‚ब्धेः प्र‚तिब‚न्धः)}
	\end{center}
	

	  \pstart \leavevmode% starting standard par
	अनुप‚ल‚ब्धेर‚व्य‚भिचारं द‚र्श‚यितुमाह (।)
	\pend% ending standard par
      
	  \bigskip
	  \begingroup
	
	    \large
	  
	    \begin{quote}
	  
	    
	    \stanza[\smallbreak]
	\label{pv.2.72}\flagstanza{\tiny\textenglish{...v.2.72}}स्व‚भावे स्व‚निमित्ते वा दृश्ये द‚र्श‚न‚हेतुषु ।&अन्येषु स‚त्स्व‚दृश्ये च स‚त्ता वा त‚द्व‚तः क‚थ‚म् ॥ ७२ ॥\&[\smallbreak]


	
	    \end{quote}
	  
	  \endgroup
	

	  \pstart \leavevmode% starting standard par
	\hphantom{.}प्र‚तिषेध‚स्य ‚{\color{DodgerBlue3}‚स्व‚{\tiny $_{1}$}‚भावे} स्व‚स्य निषेध्यात्म‚नो ‚{\color{DodgerBlue3}‚निमित्ते} का\edtext{}{\edlabel{pvv.137-2}\label{pvv.137-2}\lemma{का}\Bfootnote{स‚त्स्व‚प्य‚न्येषु ज्ञानानुत्पादात्}}र‚णे । ‚{\color{DodgerBlue3}‚दृश्ये} द‚र्श‚न‚योग्ये ‚{\color{DodgerBlue3}‚द‚र्श‚न}‚स्य हेतुष्विन्द्रिय‚म‚न‚स्कारादिष्व‚न्येषु स‚त्सु विद्य‚मानेषुदृश्ये‚{\tiny $_{lb}$}‚ऽनुप‚ल‚भ्य‚माने च ‚{\color{DodgerBlue3}‚त‚द्व‚तो} दृश्यानुप‚ल‚म्भ‚व‚तो भाव‚स्य ‚{\color{DodgerBlue3}‚स‚त्ता वा क‚थं\edtext{}{\edlabel{pvv.137-3}\label{pvv.137-3}\lemma{थं}\Bfootnote{स‚र्व‚था नैवेत्य‚र्थः ।}}} युज्य‚ते । ‚{\tiny $_{lb}$}‚न हि ‚{\color{DodgerBlue3}‚स‚त्स्व‚न्येषू}‚प‚ल‚म्भ‚प्र‚त्य‚येषु दृश्य‚स्य स‚तः क‚दाचिद‚नुप‚ल‚म्भ‚स‚म्भ‚वः । ‚{\tiny $_{lb}$}‚\leavevmode\ledsidenote{\textenglish{138/s}} त‚स्माद् दृश्यानुप‚ल‚ब्धिर‚र्थाभाव एव भ‚व‚तीति त‚त्प्र‚भ‚वाऽभाव‚प्र‚तीतिर‚विस‚म्वा‚{\tiny $_{lb}$}‚दिनी । त‚स्मात्सा‚{\tiny $_{2}$}‚ध्य‚प्र‚तिब‚द्ध‚लिङ्ग-प्र‚सूत‚त्वात् त्रिविध‚लिङ्ग‚जेप्य‚नुमाने नास्त्य‚{\tiny $_{lb}$}‚नाश्वासः । (७२)
	\pend% ending standard par
      \label{div_pvv.2.73}
	  
	% new div opening: depth here is 2
	

	  \begin{center}%% label @type='head'
	\textbf{(नः अनुमानं त्रिविध‚म्)}
	\end{center}
	

	  \begin{center}%% label @type='head'
	\textbf{(छः त‚त्र वार्वाक‚म‚त‚निरासः)}
	\end{center}
	
	  \bigskip
	  \begingroup
	
	    \large
	  
	    \begin{quote}
	  
	    
	    \stanza[\smallbreak]
	\label{pv.2.73a}\flagstanza{\tiny\textenglish{....2.73a}}अप्रामाण्ये च सामान्य‚बुद्धेस्त‚ल्लोप आग‚तः ।&प्रेत्य‚भाव‚व‚द्;\&[\smallbreak]


	
	    \end{quote}
	  
	  \endgroup
	

	  \pstart \leavevmode% starting standard par
	\hphantom{.}एव‚म‚पि ‚{\color{DodgerBlue3}‚त्व‚प्रामाण्ये} सामान्य‚बुद्धेरिष्य‚माणे त‚स्य प‚रोक्ष‚स्यार्थ‚स्यानुमान‚व्य‚व‚{\tiny $_{lb}$}‚स्थाप्य‚मान‚स्य ‚{\color{DodgerBlue3}‚लोपो}‚ऽभाव ‚{\color{DodgerBlue3}‚आग‚तः} । ‚{\color{DodgerBlue3}‚प्रेत्य‚भाव‚व‚त्} प‚र‚लोक‚स्येव । य‚दि हि प्र‚त्य‚क्ष‚{\tiny $_{lb}$}‚मेकं प्र‚माणं त‚दा य‚त्र त‚न्न प्र‚व‚र्त‚ते त‚स्याप्य\edtext{}{\edlabel{pvv.138-1}\label{pvv.138-1}\lemma{स्याप्य}\Bfootnote{एव‚ञ्च कार‚कं त‚त् ज्ञाप‚क‚ञ्चेष्य‚ते ।}}भाव एव स्यात् । न चैत‚द‚स्ति । न ‚{\tiny $_{lb}$}‚हि चार्व्वाको देशान्त‚र‚स्थं स्व‚पित‚र‚म‚नुप‚ल‚भ‚मान‚स्त‚भावं व्य‚व‚स्थाप‚यितुम‚{\tiny $_{3}$}‚र्ह‚ति ।
	\pend% ending standard par
      

	  \pstart \leavevmode% starting standard par
	स्यादेत‚त् (।)
	\pend% ending standard par
      
	  \bigskip
	  \begingroup
	
	    \large
	  
	    \begin{quote}
	  
	    
	    \stanza[\smallbreak]
	\label{pv.2.73b}\flagstanza{\tiny\textenglish{....2.73b}}अक्षैस्त‚त् प‚र्यायेण प्र‚तीय‚ते ॥ ७३ ॥\&[\smallbreak]


	
	    \end{quote}
	  
	  \endgroup
	

	  \pstart \leavevmode% starting standard par
	\hphantom{.}नोप‚ल‚भ्य‚मान‚मेवास्ति किन्तु ‚{\color{DodgerBlue3}‚प‚र्य्यायेण} प‚रिपाट्‏या‏ऽक्षैर्य‚त् ‚{\color{DodgerBlue3}‚प्र‚तीय‚ते} त‚द‚प्य‚स्ति ‚{\tiny $_{lb}$}‚चेत् । पित्राद‚य‚श्चोप‚ल‚ब्धा उप‚ल‚प्स्य‚न्ते चेति स‚न्त्येव । (७३)
	\pend% ending standard par
      \label{div_pvv.2.74}
	  
	% new div opening: depth here is 2
	
	  \bigskip
	  \begingroup
	
	    \large
	  
	    \begin{quote}
	  
	    
	    \stanza[\smallbreak]
	\label{pv.2.74}\flagstanza{\tiny\textenglish{...v.2.74}}त‚च्च नेन्द्रिय‚श‚क्त्यादाव‚क्ष‚बुद्धेर‚स‚म्भ‚वात् ।&अभाव‚प्र‚तिप‚त्तौ स्याद् बुद्धेर्ज‚न्मानिमित्त‚क‚म् ॥ ७४ ॥\&[\smallbreak]


	
	    \end{quote}
	  
	  \endgroup
	

	  \pstart \leavevmode% starting standard par
	\hphantom{.}‚{\color{DodgerBlue3}‚त‚च्च न युक्तं} इन्द्रियाख्यायां ‚{\color{DodgerBlue3}‚श‚क्ता}‚वादिग्र‚ह‚णादाहारादेः क्षुदुप‚घातादिसाम‚र्थ्ये ‚{\tiny $_{lb}$}‚च प‚र्य्यायेणापीन्द्रिय‚{\color{DodgerBlue3}‚बुद्धेर‚स‚म्भ‚वाद‚भाव‚प्र‚तीतौ} स‚त्यां प्र‚त्य‚क्षाया ‚{\color{DodgerBlue3}‚बुद्धेर्ज‚न्मानिमित्त‚कं ‚{\tiny $_{lb}$}‚स्यात्} । न हि दृष्ट‚मात्रेभ्यो विष‚यालोक‚म‚न‚स्कार‚च‚क्षुर्गोल‚{\tiny $_{4}$}‚केभ्योऽध्य‚क्ष‚ज‚न्म ‚{\tiny $_{lb}$}‚स‚त्स्व‚पि तेष्व‚भावात् व्य‚तिरे\edtext{}{\edlabel{pvv.138-2}\label{pvv.138-2}\lemma{तिरे}\Bfootnote{अनित्यः श‚ब्द इत्येकां बुद्धिं म‚न्य‚ते ।}}काद‚तिरिक्तं किञ्चिद‚दृश्यं कार‚ण‚मिष्य‚ते ‚{\tiny $_{lb}$}‚य‚स्येन्द्रिय‚मिति व्य‚प‚देशः । त‚स्य प‚र्य्यायेणापि नाध्य‚क्षं ग्राह‚क‚म‚स्तीत्य‚भावः ‚{\tiny $_{lb}$}‚स्यात् । त‚त‚श्चाकार‚ण\edtext{}{\edlabel{pvv.138-3}\label{pvv.138-3}\lemma{ण}\Bfootnote{अस‚क‚ल‚कार‚णं ।}}कं प्र‚त्य‚क्ष‚ज‚न्म प्राप्तं । अहेतोश्च नित्यं स‚त्व‚म‚स‚त्त्व‚म्वा ‚{\tiny $_{lb}$}‚स्यादि (१।८२)त्युक्तं ॥ (७४)
	\pend% ending standard par
      \label{div_pvv.2.75}
	  
	% new div opening: depth here is 2
	

	  \begin{center}%% label @type='head'
	\textbf{(जः प्र‚त्य‚क्षान्न सामान्य‚प्र‚तीतिः)}
	\end{center}
	

	  \pstart \leavevmode% starting standard par
	स्यादेत‚त् (।) प्र‚त्य‚क्षादेव सामान्य‚प्र‚तीतिर्भ‚विष्य‚ति किम‚नुमानेनेत्याह (।)
	\pend% ending standard par
      \textsuperscript{\textenglish{139/s}}
	  \bigskip
	  \begingroup
	
	    \large
	  
	    \begin{quote}
	  
	    
	    \stanza[\smallbreak]
	\label{pv.2.75}\flagstanza{\tiny\textenglish{...v.2.75}}स्व‚ल‚क्ष‚णे च प्र‚त्य‚क्ष‚म‚विक‚ल्प‚त‚या विना ।&विक‚ल्पेन न सामान्य‚ग्र‚ह‚स्त‚स्मिस्त‚तोऽनुमा ॥ ७५ ॥\&[\smallbreak]


	
	    \end{quote}
	  
	  \endgroup
	

	  \pstart \leavevmode% starting standard par
	\hphantom{.}‚{\color{DodgerBlue3}‚स्व‚ल‚क्ष‚णे च प्र‚त्य‚क्ष‚म‚विक‚ल्प‚त‚या प्र‚व‚र्त‚ते‚{\tiny $_{5}$}‚ सामान्य‚स्य तु ग्र‚हो विक‚ल्पेन विना न} भ‚व‚ति । ‚{\color{DodgerBlue3}‚त‚तः} कार‚णा‚{\color{DodgerBlue3}‚त्त‚स्मिन्} सामान्येऽनुमैव विक‚ल्पिका । नाध्य‚क्ष‚म‚विक‚ल्प‚कं ॥ (७५)
	\pend% ending standard par
      \label{div_pvv.2.76}
	  
	% new div opening: depth here is 2
	

	  \pstart \leavevmode% starting standard par
	न‚नु (।)
	\pend% ending standard par
      
	  \bigskip
	  \begingroup
	
	    \large
	  
	    \begin{quote}
	  
	    
	    \stanza[\smallbreak]
	\label{pv.2.76}\flagstanza{\tiny\textenglish{...v.2.76}}प्र‚मेय‚निय‚मे व‚र्णानित्य‚ता न प्र‚तीय‚ते ।&प्र‚माण‚म‚न्य‚त् त‚द्-बुद्धिर्विना लिङ्गेन संभ‚वात् ॥ ७६ ॥\&[\smallbreak]


	
	    \end{quote}
	  
	  \endgroup
	

	  \pstart \leavevmode% starting standard par
	\hphantom{.}प्र‚त्य‚क्षं स्व‚ल‚क्ष‚ण‚विष‚य‚म‚नुमानं सामान्य‚विष‚य‚मिति प्र‚मेय‚स्य ‚{\color{DodgerBlue3}‚निय‚मे} स्वी‚{\tiny $_{lb}$}‚क्रिय‚माणे ‚{\color{DodgerBlue3}‚व‚र्ण्ण}‚स्य नीलादे\edtext{}{\edlabel{pvv.139-1}\label{pvv.139-1}\lemma{नीलादे}\Bfootnote{अन्य‚क्ष‚णे नाशं दृष्ट्वा । आदिना श‚ब्द‚स्यान्त्य‚क्ष‚ण‚स्यानित्य‚ता । त‚था हि ‚{\tiny $_{lb}$}‚श‚ब्दादि स्व‚ल‚क्ष‚णं । अनित्य‚तादि सामान्यं । अन‚योः संक‚रेण ग्र‚ह‚णात् प्र‚मेयान्त‚र‚मेत‚त् ।}}र्व्विष‚य‚स्या‚{\color{DodgerBlue3}‚नित्य‚ता}‚ऽनित्य‚तासामान्यात्म‚ता ‚{\color{DodgerBlue3}‚न प्र‚तीय‚त} इति प्राप्तं । न हि सामान्य‚विशेषात्म‚कं प्र‚मेयं विशेष‚मात्र‚विष‚येणाध्य‚क्षेण सामा‚{\tiny $_{lb}$}‚न्य‚मात्र‚विष‚{\tiny $_{6}$}‚येणानुमानेन वा प्र‚त्येतुं श‚क्यं ।\edtext{\textsuperscript{*}}{\edlabel{pvv.139-2}\label{pvv.139-2}\lemma{*}\Bfootnote{अनित्यः श‚ब्द इत्येकां बुद्धिं म‚न्य‚ते ।}} प्र‚तीय‚ते चात‚{\color{DodgerBlue3}‚स्त‚द्‏बुद्धिर्प्र‚माण‚म‚न्य‚त्} स्यात् । न प्र‚त्य‚क्षं सामान्य‚स्य ग्र‚ह‚णात् । नाप्य‚नुमानं ‚{\color{DodgerBlue3}‚विना लिङ्गेन स‚म्भ‚वात्} । ‚{\tiny $_{lb}$}‚विशेष‚स्यापि ग्र‚ह‚णाच्च । (७६)
	\pend% ending standard par
      \label{div_pvv.2.77}
	  
	% new div opening: depth here is 2
	

	  \pstart \leavevmode% starting standard par
	त‚था (।)
	\pend% ending standard par
      
	  \bigskip
	  \begingroup
	
	    \large
	  
	    \begin{quote}
	  
	    
	    \stanza[\smallbreak]
	\label{pv.2.77}\flagstanza{\tiny\textenglish{...v.2.77}}विशेष‚दृष्टे लिङ्ग‚स्य स‚म्ब‚न्ध‚स्याप्र‚सिद्धितः ।&त‚त्प्र‚माणान्त‚रं मेय‚ब‚हुत्वाद् ब‚हुतापि वा ॥ ७७ ॥\&[\smallbreak]


	
	    \end{quote}
	  
	  \endgroup
	

	  \pstart \leavevmode% starting standard par
	\hphantom{.}‚{\color{DodgerBlue3}‚वि\edtext{}{\edlabel{pvv.139-3}\label{pvv.139-3}\lemma{वि}\Bfootnote{अस‚कृद्वेति [प्र‚माण]स‚मुच्च‚यं व्याच‚ष्टे । विशेष‚द‚ष्टे च य‚ज्ज्ञानं त‚द‚पि प्र‚मान्त‚र‚मित्याह ।}}शेष‚दृष्टे} प्र‚त्य‚क्षेणाग्निं दृष्ट्वा क्र‚मात्त‚मेव धूमाल्लिङ्गात् स एवायं ‚{\tiny $_{lb}$}‚व\edtext{}{\edlabel{pvv.139-4}\label{pvv.139-4}\lemma{व}\Bfootnote{न स‚कृद्वृत्तेन स‚माप्तिः किन्तु त‚त्र पुन‚र्मान‚वृत्तिः स‚म‚यान्त‚रे ।}}ह्निरिति निश्चिनोत्य‚नुमानेन । ‚{\color{DodgerBlue3}‚लिङ्ग‚स्य स‚म्ब‚न्धासिद्धितः त‚त्प्र‚माणान्त‚रं}\leavevmode\ledsidenote{\textenglish{26b/MA}} ‚{\tiny $_{lb}$}‚स्यात् । न हि त‚त्प्र‚त्य‚क्षं लिङ्गं-ब‚लादुत्प‚त्तेः । नाप्य‚नु‚{\tiny $_{7}$}‚मानं द‚ह‚न‚धूम\edtext{}{\edlabel{pvv.139-5}\label{pvv.139-5}\lemma{धूम}\Bfootnote{प‚र्व्व‚त‚व‚र्त्तिधूमेन व‚ह्नेः स‚म्ब‚न्धासिद्धेः ।}}विशेष‚योः ‚{\tiny $_{lb}$}‚स‚म्ब‚न्धाग्र‚ह‚णात् । त‚स्मात् ‚{\color{DodgerBlue3}‚मेया}‚नां प्र‚त्य‚क्षं ल‚क्ष‚णं सामान्यं सामान्य‚विशेषाणां ‚{\tiny $_{lb}$}‚‚{\color{DodgerBlue3}‚ब‚हुत्वात् ब‚हुतापि वा} (७७)
	\pend% ending standard par
      \label{div_pvv.2.78}
	  
	% new div opening: depth here is 2
	
	  \bigskip
	  \begingroup
	
	    \large
	  
	    \begin{quote}
	  
	    
	    \stanza[\smallbreak]
	\label{pv.2.78}\flagstanza{\tiny\textenglish{...v.2.78}}प्र‚माणानाम‚नेक‚स्य वृत्तेरेक‚त्र वा य‚था ।&विशेष‚दृष्टेरेक‚त्रिसंख्यापोहो न वा भ‚वेत् ॥ ७८ ॥\&[\smallbreak]


	
	    \end{quote}
	  
	  \endgroup
	

	  \pstart \leavevmode% starting standard par
	\hphantom{.}‚{\color{DodgerBlue3}‚प्र‚माणानां स्यात् । अनेक‚स्य} वा प्र‚माण‚स्यैक‚त्र विष‚ये वृत्तेः प्र‚माण‚ब‚हुता स्यात् । ‚{\tiny $_{lb}$}‚\leavevmode\ledsidenote{\textenglish{140/s}} य‚थैक‚मेव स्व‚ल‚क्ष‚णं प्र‚त्य‚क्षेण विशेष‚दृष्टेन चानुमानेन प्र‚तीय‚ते । प्र‚मेय‚स्य द्वित्व‚{\tiny $_{lb}$}‚ग्र‚ह‚ण‚योर्निय‚मे प्र‚माद्वित्वं स्यात् नान्य‚था । त्र्येक‚संख्याया अपोहो य उक्तः प्र‚मेय‚द्वि‚{\tiny $_{lb}$}‚त्वात्‚{\tiny $_{1}$}‚ स ‚{\color{DodgerBlue3}‚न वा भ‚वेत्} ॥ (७८)
	\pend% ending standard par
      \label{div_pvv.2.79}
	  
	% new div opening: depth here is 2
	
	  \bigskip
	  \begingroup
	
	    \large
	  
	    \begin{quote}
	  
	    
	    \stanza[\smallbreak]
	\label{pv.2.79a}\flagstanza{\tiny\textenglish{....2.79a}}विष‚यानिय‚माद‚न्य‚प्र‚मेय‚स्य च स‚म्भ‚वात् ।\&[\smallbreak]


	
	    \end{quote}
	  
	  \endgroup
	

	  \pstart \leavevmode% starting standard par
	\hphantom{.}‚{\color{DodgerBlue3}‚विष‚य‚स्य} ग्र‚ह‚णा‚{\color{DodgerBlue3}‚निय‚मात्} य‚दा ह्येकेनापि प्र‚माणेनानेकं गृह्य‚ते त‚दा प्र‚मेय‚{\tiny $_{lb}$}‚द्वित्वेप्येक‚मेव प्र‚माणं स्यात् । प्र‚मेय‚द्व‚या‚{\color{DodgerBlue3}‚द‚न्य‚स्य च} सामान्य‚विशेष‚स्य ‚{\color{DodgerBlue3}‚मेय‚स्य स‚म्भ‚{\tiny $_{lb}$}‚वात्} । त्र्यादिक‚म्वा त‚द्‏ग्राह‚कं मानं भ‚वेत् ।\edtext{\textsuperscript{*}}{\edlabel{pvv.140-1}\label{pvv.140-1}\lemma{*}\Bfootnote{अन्त्य‚म‚ध्य‚क्षं प्र‚वाह‚विच्छेदे त‚द्विविक्तोप‚ल‚म्भ‚श्चेत्य‚ध्य‚क्ष‚द्व‚य‚ज‚नितेन ।}}
	\pend% ending standard par
      

	  \pstart \leavevmode% starting standard par
	अत्रोच्य‚ते (।)
	\pend% ending standard par
      
	  \bigskip
	  \begingroup
	
	    \large
	  
	    \begin{quote}
	  
	    
	    \stanza[\smallbreak]
	\label{pv.2.79b}\flagstanza{\tiny\textenglish{....2.79b}}योज‚नाद् व‚र्ण‚सामान्ये नायं दोषः प्र‚स‚ज्य‚ते ॥ ७९ ॥\&[\smallbreak]


	
	    \end{quote}
	  
	  \endgroup
	

	  \pstart \leavevmode% starting standard par
	विक\edtext{}{\edlabel{pvv.140-2}\label{pvv.140-2}\lemma{विक}\Bfootnote{त‚देवं प्र‚त्य‚क्ष‚म‚नुमान‚ञ्च प्र‚माणे ल‚क्ष‚ण‚द्व‚यं प्र‚मेय‚मित्याख्याय त‚स्य स‚न्धानेन ‚{\tiny $_{lb}$}‚प्र‚माणान्त‚रं न च पुनः पुन‚र‚भिज्ञानेऽनिष्टास‚क्तेः स्मृतादिव‚त्य‚स्य वृत्ति‚{\tiny $_{lb}$}‚र्य‚त्त‚र्हीद‚म‚नित्यादिभिराकारैर्व‚र्ण्णादि गृह्य‚तेऽस‚कृद्वेति व्याख्याता ।}}ल्प‚केन ज्ञानेनानित्य‚ताया ‚{\color{DodgerBlue3}‚व‚र्ण्ण‚सा\edtext{}{\edlabel{pvv.140-3}\label{pvv.140-3}\lemma{सा}\Bfootnote{न तु विशेषे विशेष‚स्य विक‚ल्प‚केनाग्र‚हात् ।}}मान्ये योज‚नाद‚यं} सामान्य‚विशेषा‚{\tiny $_{lb}$}‚त्म‚क‚प्र‚मेय‚ग्राह‚क‚प्र‚माणान्त‚राभ्युप‚ग‚म‚ल‚क्ष‚णो ‚{\color{DodgerBlue3}‚दोषो न प्र‚सं‚{\tiny $_{2}$}‚ज्य‚ते} । न हि विशेषोऽ‚{\tiny $_{lb}$}‚नित्य‚त‚या योज्य‚ते । विक‚ल्पानाम‚त‚द्‏विष‚य‚त्व‚स्योक्तेर्व्व‚क्ष्य‚माण‚त्वाच्च । (७९)
	\pend% ending standard par
      \label{div_pvv.2.80}
	  
	% new div opening: depth here is 2
	

	  \begin{center}%% label @type='head'
	\textbf{(फ. अनित्याद‚यो नाव‚स्तुध‚र्माः)}
	\end{center}
	

	  \pstart \leavevmode% starting standard par
	न‚नु व‚र्ण्ण‚सामान्य‚स्याव‚स्तुत्वात्त‚द्योजिताऽनित्य‚ताद‚योऽव‚स्तु\edtext{}{\edlabel{pvv.140-4}\label{pvv.140-4}\lemma{स्तु}\Bfootnote{सामान्य‚ध‚र्माः । ? न स्व‚ल‚क्ष‚ण‚स्य}}ध‚र्माः स्युरि‚{\tiny $_{lb}$}‚त्याह (।)
	\pend% ending standard par
      
	  \bigskip
	  \begingroup
	
	    \large
	  
	    \begin{quote}
	  
	    
	    \stanza[\smallbreak]
	\label{pv.2.80}\flagstanza{\tiny\textenglish{...v.2.80}}नाव‚स्तुरूपं त‚स्यैव त‚था सिद्धे प्र‚साध‚नात् ।&अन्य‚त्र नान्य‚सिद्धिश्चेन्न त‚स्यैव प्र‚सिद्धितः ॥ ८० ॥\&[\smallbreak]


	
	    \end{quote}
	  
	  \endgroup
	

	  \pstart \leavevmode% starting standard par
	\hphantom{.}‚{\color{DodgerBlue3}‚नाव‚स्तुनो} रूप‚म‚नित्य‚त्वादि । ‚{\color{DodgerBlue3}‚त‚स्यैव} व‚स्तुन‚स्त‚थाऽनित्य‚त्वादिभिराकारैः ‚{\tiny $_{lb}$}‚‚{\color{DodgerBlue3}‚सिद्धे} र्निश्च‚य‚स्य प्राक् ‚{\color{DodgerBlue3}‚प्र‚साध‚नात्} । अध्य‚व‚सायानुरोधेन हि विक‚ल्पानां विष‚य‚{\tiny $_{lb}$}‚व्य‚व‚स्था (।) य‚द्य‚पि चैते स्वाकार‚ग्राहिण‚स्त‚थापि बा‚{\tiny $_{3}$}‚ह्य‚मेव विष‚य‚त‚या व्य‚व‚स्थाप्य‚न्ते ‚{\tiny $_{lb}$}‚अनाद्य‚भ्यास‚विशेषात् । अनुमान‚न्तु व‚स्तुप्र‚तिब‚द्ध‚लिङ्ग‚प्र‚भ‚व‚त्वाद्य‚थाव‚स्थित‚मेव ‚{\tiny $_{lb}$}‚व‚स्तु व्य‚व‚स्य‚तीति व‚स्त्वेव क्ष‚ण‚स्थितिध‚र्म‚क‚म‚स्मात्सिद्धं । त‚तो नाव‚स्तुरूप‚म‚{\tiny $_{lb}$}‚नित्य‚त्वादि ॥
	\pend% ending standard par
      \textsuperscript{\textenglish{141/s}}

	  \pstart \leavevmode% starting standard par
	स्यादेत‚द् (।) अ (न्य) त्राव‚स्तुनि सामान्येऽनित्य‚तादिस‚म्ब‚न्धिनि सिध्य‚त्य‚न्य‚स्य ‚{\tiny $_{lb}$}‚व‚स्तुनोऽनित्य‚रूप‚स्य न सिद्धिरिति चेत् । नैत‚द्युक्तं ‚{\color{DodgerBlue3}‚त\edtext{}{\edlabel{pvv.141-1}\label{pvv.141-1}\lemma{त}\Bfootnote{स्व‚यं विप्लुतोपि व‚स्तुप्र‚तिब‚द्ध‚ज‚न्म‚त‚या व‚स्तु साध‚य‚न् प्र‚माणं ।}}स्यैव} व‚स्तुनः त‚स्यैवा‚{\tiny $_{lb}$}‚नित्य‚रूप‚स्याध्य‚व‚साय‚{\tiny $_{4}$}‚व‚शेनानुमानात् ‚{\color{DodgerBlue3}‚प्र‚सिद्धितः} । (८०)
	\pend% ending standard par
      \label{div_pvv.2.81}
	  
	% new div opening: depth here is 2
	

	  \begin{center}%% label @type='head'
	\textbf{(ञ. लिङ्ग‚धीसंवाद‚क‚ता)}
	\end{center}
	

	  \pstart \leavevmode% starting standard par
	क‚थं पुन‚र्व्व‚स्त्व‚ध्य‚व‚सायेऽपि त‚त्स‚म्वाद\edtext{}{\edlabel{pvv.141-2}\label{pvv.141-2}\lemma{म्वाद}\Bfootnote{द‚र्प्प‚ण‚प्र‚तिबिम्बे तिल‚कादिसिद्ध्या मुखे तिल‚कादिसिद्धिव‚त् ।}}क‚तेत्याह (।)
	\pend% ending standard par
      
	  \bigskip
	  \begingroup
	
	    \large
	  
	    \begin{quote}
	  
	    
	    \stanza[\smallbreak]
	\label{pv.2.81}\flagstanza{\tiny\textenglish{...v.2.81}}यो हि भावो य‚थाभूतो स तादृग्लिङ्ग‚चेत‚सः ।&हेतुस्त‚ज्जा त‚थाभूते त‚स्माद् व‚स्तुनि लिङ्गिधीः ॥ ८१ ॥\&[\smallbreak]


	
	    \end{quote}
	  
	  \endgroup
	

	  \pstart \leavevmode% starting standard par
	\hphantom{.}‚{\color{DodgerBlue3}‚यो हि} साध्य‚ध‚र्मो ‚{\color{DodgerBlue3}‚य‚थाभूतः} का\edtext{}{\edlabel{pvv.141-3}\label{pvv.141-3}\lemma{का}\Bfootnote{व‚ह्निधूमादिज‚न‚कः । अनित्याकार‚वान् श‚ब्दः । त‚ल्लिङ्ग‚य‚तीति तादृग्‏लिङ्ग‚म् ।}}र‚ण‚व्याप‚क‚स्व‚भावः‚{\color{DodgerBlue3}‚स तादृशः} कार‚ण‚कार्य‚त‚या ‚{\tiny $_{lb}$}‚व्याप‚क‚व्याप्य‚त‚या गृहीत‚व्याप्तिक‚स्य ‚{\color{DodgerBlue3}‚लिङ्ग‚स्य चेत‚सः} प‚रंप‚राहेतुः ।\edtext{\textsuperscript{*}}{\edlabel{pvv.141-4}\label{pvv.141-4}\lemma{*}\Bfootnote{इति कृत्वा}} ‚{\color{DodgerBlue3}‚त‚ज्जा त‚स्मा}‚{\tiny $_{lb}$}‚ल्लिङ्ग‚चेत‚सो जाता । ‚{\color{DodgerBlue3}‚त‚थाभूते व‚स्तुनि} कार‚ण‚व्याप‚क‚रूपे साध्य‚ध‚र्मे ‚{\color{DodgerBlue3}‚लिङ्गिधीः} । ‚{\tiny $_{lb}$}‚त‚स्मात्प‚रंप‚र‚या साध्य‚प्र‚तिब‚न्धात् लिङ्गिधीः‚{\tiny $_{5}$}‚ स‚त्येव व‚स्तुनि भ‚व‚न्ती त‚त्स‚म्वादा‚{\tiny $_{lb}$}‚त्प्र‚माण‚मेव । (८१)
	\pend% ending standard par
      \label{div_pvv.2.82}
	  
	% new div opening: depth here is 2
	

	  \pstart \leavevmode% starting standard par
	न‚नु लिङ्ग‚म‚पि लिङ्गिव‚त्सामान्य‚मेव । त‚था धूमः कृत‚कं वेत्येव न लिङ्गं ‚{\tiny $_{lb}$}‚किन्त‚र्हि व‚ह्निकार्य‚त‚याऽनित्य‚त्व‚व्याप्य‚त‚या च गृहीतं । न च विशेषे व्याप्तिग्र‚हः । ‚{\tiny $_{lb}$}‚सामान्य‚ञ्च नाध्य‚क्ष‚ग‚म्यं । विक‚ल्प‚मात्रेण त‚त्प्र‚तीताव‚नाश्वासः । नैष दोषः । ‚{\tiny $_{lb}$}‚प्र‚त्य‚क्षेण कार‚ण‚कार्य‚योर्व्यावृत्तिद्व‚य‚विशिष्ट‚योर्गृ हीत‚योर्व्विजातीय‚व्यावृत्त्याश्र‚येणो‚{\tiny $_{lb}$}‚त्प‚न्न‚वि‚{\tiny $_{6}$}‚क‚ल्पेन क्व‚चिद‚नुमानेन व्याप्तिं गृहीत‚व‚तः प‚श्चाद् धूम‚कृत‚क‚त्वादि‚{\tiny $_{lb}$}‚द‚र्श‚नात्ताद्रूप्ये कार्य‚व्याप्य‚बुद्धिर्लिङ्ग‚बुद्धिः (।) सा च त‚त्प्र‚तिब‚न्धाद‚नुमान‚मेवेति ‚{\tiny $_{lb}$}‚नास्त्य‚नाश्वास (:।) एत‚देवाह (।)
	\pend% ending standard par
      
	  \bigskip
	  \begingroup
	
	    \large
	  
	    \begin{quote}
	  
	    
	    \stanza[\smallbreak]
	\label{pv.2.82}\flagstanza{\tiny\textenglish{...v.2.82}}लिङ्ग‚लिङ्गिधियोरेवं पार‚म्प‚र्येण व‚स्तुनि ।&प्र‚तिब‚न्धात् त‚दाभास‚शून्य‚योर‚प्य‚वंच‚न‚म् ॥ ८२ ॥\&[\smallbreak]


	
	    \end{quote}
	  
	  \endgroup
	

	  \pstart \leavevmode% starting standard par
	\hphantom{.}लिङ्ग‚लिङ्गि‚{\color{DodgerBlue3}‚धियोरेव‚मुक्त}‚क्र‚मात् ‚{\color{DodgerBlue3}‚पार‚म्प‚र्येण व‚स्तुनि प्र‚तिब‚न्धात्त}‚योर्लिङ्ग‚{\tiny $_{lb}$}‚\leavevmode\ledsidenote{\textenglish{142/s}} लिङ्गिनो‚{\color{DodgerBlue3}‚राभासः} साक्षात् स्व‚रूप‚प्र‚तिभासः । ‚{\color{DodgerBlue3}‚त‚च्छून्य\edtext{}{\edlabel{pvv.142-1a}\label{pvv.142-1a}\lemma{च्छून्य}\Bfootnote{1a प्र‚त्य‚क्षेण धूमादिः प्र‚तीय‚ते न त‚स्य लिङ्ग‚तापि निर्व्विक‚ल्पेन ।\begin{english} --- Placement of note uncertain; marked with a question mark in the edition (see encoding description for details).\end{english}}}\edtext{}{\edlabel{pvv.142-1}\label{pvv.142-1}\lemma{च्छून्य}\Bfootnote{वास‚नाप्र‚बोधात्त‚योः क‚ल्प‚नं व‚ह्निर्न भात्येव लिङ्ग‚त्वेनापि न भानं धूम‚स्यैव भानात् ।}}योर‚पि} लिङ्ग‚लिङ्गिव‚स्तुनि ‚{\tiny $_{lb}$}‚\leavevmode\ledsidenote{\textenglish{27a/MA}} ‚{\color{DodgerBlue3}‚अव‚ञ्च‚नं‚{\tiny $_{7}$}‚} स‚म्वाद‚नं । (८२)
	\pend% ending standard par
      \label{div_pvv.2.83}
	  
	% new div opening: depth here is 2
	

	  \pstart \leavevmode% starting standard par
	न‚नु लिङ्ग‚बुद्धेर‚पि लिङ्गिबुद्धित्वात् पृथ‚क् उपादान‚म‚न‚र्थ‚कं (।) नान‚र्थ‚कं । ‚{\tiny $_{lb}$}‚प्र‚त्य‚क्ष‚मुद्भूत‚विक‚ल्पो वा त‚द्बुद्धिरिति विप्र‚तिप‚त्तिनिरासार्थ‚त्वात् ।\edtext{\textsuperscript{*}}{\edlabel{pvv.142-2}\label{pvv.142-2}\lemma{*}\Bfootnote{प‚र‚म्प‚र‚या व‚स्त्व‚विसंवादात् । ? व‚स्तुनि ।}} य‚दि प्र‚माणे ‚{\tiny $_{lb}$}‚लिङ्ग‚लिङ्गिधियौ त‚दा प्र‚त्य‚क्ष‚व‚द‚भ्रान्ते स्यातामित्याह (।)
	\pend% ending standard par
      
	  \bigskip
	  \begingroup
	
	    \large
	  
	    \begin{quote}
	  
	    
	    \stanza[\smallbreak]
	\label{pv.2.83}\flagstanza{\tiny\textenglish{...v.2.83}}त‚द्रूपाध्य‚व‚सायाच्च त‚योस्त‚द्रूप‚शून्य‚योः ।&त‚द्रूपाव‚ञ्च‚क‚त्वेपि कृता भ्रान्तिव्य‚व‚स्थितिः ॥ ८३ ॥\&[\smallbreak]


	
	    \end{quote}
	  
	  \endgroup
	

	  \pstart \leavevmode% starting standard par
	\hphantom{.}‚{\color{DodgerBlue3}‚त‚यो}‚र्द्व‚योः ‚{\color{DodgerBlue3}‚त‚द्रूप}‚स्य लिङ्ग‚लिङ्गि‚{\color{DodgerBlue3}‚रूप}‚स्या‚{\color{DodgerBlue3}‚ध्य‚व‚सायात्} प्र‚व‚र्त‚ने स‚ति व‚स्तुनि ‚{\tiny $_{lb}$}‚प‚र‚म्प‚र‚या त‚त्प्र‚तिब‚द्ध‚त‚या च ‚{\color{DodgerBlue3}‚त‚द्रूप}‚स्या‚{\color{DodgerBlue3}‚व‚ञ्च‚क‚त्वे} स‚म्वाद‚क‚त्वे‚{\color{DodgerBlue3}‚पि भ्रान्तिव्य‚व‚स्थितिः} कृता । क‚स्मादित्याह‚{\tiny $_{1}$}‚ ‚{\color{DodgerBlue3}‚त‚द्रूप‚शून्य‚योः} । न हि लिङ्ग‚लिङ्गिस्व‚रूप‚प्र‚तिभासिन्यौ ‚{\tiny $_{lb}$}‚धियाविमे स्व‚प्र‚तिभासेऽन‚र्थेऽर्थाध्याव‚सायेन प्र‚वृत्त‚त्वात् । (८३)
	\pend% ending standard par
      \label{div_pvv.2.84}
	  
	% new div opening: depth here is 2
	
	  \bigskip
	  \begingroup
	
	    \large
	  
	    \begin{quote}
	  
	    
	    \stanza[\smallbreak]
	\label{pv.2.84}\flagstanza{\tiny\textenglish{...v.2.84}}त‚स्माद् व‚स्तुनि बोद्ध‚व्ये व्याप‚कं व्याप्य‚चेत‚सः&निमित्तं त‚त्स्व‚भावो वा कार‚णं, त‚च्च त‚द्धियः ॥ ८४ ॥\&[\smallbreak]


	
	    \end{quote}
	  
	  \endgroup
	

	  \pstart \leavevmode% starting standard par
	\hphantom{.}‚{\color{DodgerBlue3}‚त‚स्माद्व‚स्तुनि} विधि\edtext{}{\edlabel{pvv.142-3}\label{pvv.142-3}\lemma{विधि}\Bfootnote{विधिद्वारेण, न प्र‚तिषेध‚मुखेन ।}}ना ‚{\color{DodgerBlue3}‚बोद्ध‚व्ये व्याप‚कं} साध्यं । ‚{\color{DodgerBlue3}‚व्याप्य‚चेत‚सो} लिङ्ग‚बुद्धे‚{\tiny $_{lb}$}‚‚{\color{DodgerBlue3}‚र्निमित्ते\edtext{}{\edlabel{pvv.142-4}\label{pvv.142-4}\lemma{र्निमित्ते}\Bfootnote{स‚द्‏व्याप‚क‚ब‚लेन लिङ्ग‚बुद्ध्युत्प‚त्तेः । ? निमित्तं ।}}} प‚र‚म्प‚र‚या । य‚स्मा‚{\color{DodgerBlue3}‚त्स्व‚भावो} वा त‚द् व्याप‚कं । व्याप्य‚स्य य‚थानित्य‚त्वं ‚{\tiny $_{lb}$}‚कृत‚क‚त्व‚स्य । ‚{\color{DodgerBlue3}‚कार‚ण}‚म्वा द‚ह‚नो ‚{\color{DodgerBlue3}‚धूम‚स्य । त‚च्च} व्याप्य‚चेत‚स‚{\color{DodgerBlue3}‚स्त‚द्धियः} स्व‚भाव‚{\tiny $_{lb}$}‚कार‚ण‚धियो निमित्त‚मिति त‚याध्य‚व‚सित‚स्य स‚म्वाद‚निय‚मः‚{\tiny $_{2}$}‚ स‚म्वादित्व‚मेवं ‚{\tiny $_{lb}$}‚प्र\edtext{}{\edlabel{pvv.142-5}\label{pvv.142-5}\lemma{प्र}\Bfootnote{स्व‚ल‚क्ष‚णेन ज‚नितं । स्व‚ल‚क्ष‚ण‚व्य‚व‚स्थाप‚क‚ञ्च ।}}त्य‚क्ष‚स्यापि प्रामाण्य‚ल‚क्ष‚णं । (८४)
	\pend% ending standard par
      \label{div_pvv.2.85}
	  
	% new div opening: depth here is 2
	

	  \begin{center}%% label @type='head'
	\textbf{(२) अनुप‚ल‚व्धिचिन्ता\footnote{\label{pvv.142-asterisk}  द्र‚ष्ट‚व्यं प‚रिशिष्टं ।८}}
	\end{center}
	
	  \bigskip
	  \begingroup
	
	    \large
	  
	    \begin{quote}
	  
	    
	    \stanza[\smallbreak]
	\label{pv.2.85}\flagstanza{\tiny\textenglish{...v.2.85}}प्र‚तिषेध‚स्तु स‚र्व‚त्र साध्य‚तेऽनुप‚ल‚म्भ‚तः ।&सिद्धिं प्र‚माणैर्व‚द‚ताम‚र्थादेव विप‚र्य‚यात् ॥ ८५ ॥\&[\smallbreak]


	
	    \end{quote}
	  
	  \endgroup
	\textsuperscript{\textenglish{143/s}}

	  \pstart \leavevmode% starting standard par
	\hphantom{.}‚{\color{DodgerBlue3}‚प्र‚तिषेध}‚स्तु य‚त्र साक्षान्निषेध्यानुप‚ल‚ब्धिर्व्विरुद्धा ह्युप‚ल‚ब्धिर्व्वा द‚र्श्य‚ते ‚{\color{DodgerBlue3}‚त‚त्र ‚{\tiny $_{lb}$}‚स‚र्व्व‚त्रानुप‚ल‚म्भ‚तः साध्य‚ते} । य‚स्मा‚{\color{DodgerBlue3}‚त्प्र‚माणै}‚र‚र्थ‚स्य ‚{\color{DodgerBlue3}‚सिद्धिं व‚द‚ताम‚र्था}‚त्सामार्थ्या‚{\color{DodgerBlue3}‚देव ‚{\tiny $_{lb}$}‚विप‚र्य‚या}‚त्प्र‚माणाभावाद‚र्थाभावः सिध्य‚ति । (८५)
	\pend% ending standard par
      \label{div_pvv.2.86}
	  
	% new div opening: depth here is 2
	

	  \pstart \leavevmode% starting standard par
	न‚नु विरुद्धाद्युप‚ल‚ब्धाव‚नुप‚ल‚म्भः क‚थं निषेध‚साध‚क इत्याह (।)
	\pend% ending standard par
      
	  \bigskip
	  \begingroup
	
	    \large
	  
	    \begin{quote}
	  
	    
	    \stanza[\smallbreak]
	\label{pv.2.86}\flagstanza{\tiny\textenglish{...v.2.86}}दृष्टा विरुद्ध‚ध‚र्मोक्तिस्त‚स्य त‚त्कार‚ण‚स्य वा ।&निषेधे यापि त‚स्यैव साऽप्र‚माण‚त्व‚सूच‚ना ॥ ८६ ॥\&[\smallbreak]


	
	    \end{quote}
	  
	  \endgroup
	

	  \pstart \leavevmode% starting standard par
	\hphantom{.}दृष्टा विरुद्ध‚ध‚र्म‚स्योक्तिर्या ‚{\color{DodgerBlue3}‚त‚स्य} निषेध्य‚स्य ‚{\color{DodgerBlue3}‚त‚त्कार‚ण‚स्य‚{\tiny $_{3}$}‚ वा निषेधे} क‚र्त‚व्ये ‚{\tiny $_{lb}$}‚य‚था नात्र शीत‚स्प‚र्शो न वा रोम‚ह‚र्ष‚युक्त‚पुरुष‚वान‚यं प्र‚देशो व‚ह्नेरिति (।) ‚{\color{DodgerBlue3}‚सापि ‚{\tiny $_{lb}$}‚त‚स्यैव} निषेध्य‚स्यैवा‚{\color{DodgerBlue3}‚प्र‚माण‚त्व‚स्य} प्र‚माण‚र‚हित‚त्व‚स्य ‚{\color{DodgerBlue3}‚सूच‚ना} । विरुद्ध‚स्य कार‚ण‚वि‚{\tiny $_{lb}$}‚रुद्ध‚त्व‚स्य चोप‚ल‚म्भेन निषेध्यानुप‚ल‚म्भ एव ख्याप्य‚ते । (८६)
	\pend% ending standard par
      \label{div_pvv.2.87_2.88}
	  
	% new div opening: depth here is 2
	
	  \bigskip
	  \begingroup
	
	    \large
	  
	    \begin{quote}
	  
	    
	    \stanza[\smallbreak]
	\label{pv.2.87a}\flagstanza{\tiny\textenglish{....2.87a}}अन्य‚थैक‚स्य ध‚र्म‚स्य स्व‚भावोक्त्या प‚र‚स्य त‚त् ।&नास्तित्वं केन ग‚म्येत;\&[\smallbreak]


	
	    \end{quote}
	  
	  \endgroup
	

	  \pstart \leavevmode% starting standard par
	\hphantom{.}‚{\color{DodgerBlue3}‚अन्य‚थैक‚स्य} विरुद्धादे‚{\color{DodgerBlue3}‚र्ध‚र्म‚स्य स्व‚भावोक्त्या} स‚त्ताप्र‚तिपाद‚नेन ‚{\color{DodgerBlue3}‚प‚र‚स्य} निषेध्य‚स्य ‚{\tiny $_{lb}$}‚त‚न्नास्तित्वं सिसाध‚यि‚{\tiny $_{4}$}‚षितं ‚{\color{DodgerBlue3}‚केन} कार‚णेन ‚{\color{DodgerBlue3}‚ग‚म्येत} । न ह्य‚न्य‚स‚त्त्वेऽप‚र‚स्य स‚त्वाभावो‚{\tiny $_{lb}$}‚तिप्र‚स‚ङ्गात् । अनिषिद्धोप‚ल‚ब्धेर‚भावासिद्धेः ॥
	\pend% ending standard par
      
	  \bigskip
	  \begingroup
	
	    \large
	  
	    \begin{quote}
	  
	    
	    \stanza[\smallbreak]
	\label{pv.2.87b}\flagstanza{\tiny\textenglish{....2.87b}}विरोधाच्चेद‚साव‚पि ॥ ८७ ॥\&[\smallbreak]


	
	    \end{quote}
	  
	  \endgroup
	
	  \bigskip
	  \begingroup
	
	    \large
	  
	    \begin{quote}
	  
	    
	    \stanza[\smallbreak]
	\label{pv.2.88a}\flagstanza{\tiny\textenglish{....2.88a}}सिद्धः केन;\&[\smallbreak]


	
	    \end{quote}
	  
	  \endgroup
	

	  \pstart \leavevmode% starting standard par
	\hphantom{.}अथ य‚न्निषिध्य‚ते य‚च्चोप‚द‚र्श्य‚ते त‚यो‚{\color{DodgerBlue3}‚र्व्विरोधात्} स‚हान‚व‚स्थान‚ल‚क्ष‚णादेक‚भा‚{\tiny $_{lb}$}‚वेऽन्याभाव इति ‚{\color{DodgerBlue3}‚चेत्} । न‚न्व‚{\color{DodgerBlue3}‚साव‚पि} (८७) विरोधः ‚{\color{DodgerBlue3}‚केन} प्र‚का\edtext{}{\edlabel{pvv.143-1}\label{pvv.143-1}\lemma{का}\Bfootnote{प्र‚माणेन ।}}रेण ‚{\color{DodgerBlue3}‚सिद्धः} ॥
	\pend% ending standard par
      
	  \bigskip
	  \begingroup
	
	    \large
	  
	    \begin{quote}
	  
	    
	    \stanza[\smallbreak]
	\label{pv.2.88b}\flagstanza{\tiny\textenglish{....2.88b}}अस‚ह‚स्थानादिति चेत् त‚त् कुतो म‚त‚म् ॥&दृश्य‚स्य द‚र्श‚नाभावादिति चेत् साऽप्र‚माण‚ता ॥ ८८ ॥\&[\smallbreak]


	
	    \end{quote}
	  
	  \endgroup
	

	  \pstart \leavevmode% starting standard par
	द्व‚योः स‚हान‚व‚स्थानाद्विरोधः सिद्ध इति चेत् । त‚त्स‚हान‚व‚स्थानं कुतो हेतोर्म‚तं ‚{\tiny $_{lb}$}‚ये‚{\tiny $_{5}$}‚न विरोध‚व्य‚व‚स्था ॥ अविक‚ल‚कार‚ण‚स्य प्र‚व‚र्त‚मान‚स्य ‚{\color{DodgerBlue3}‚दृश्य‚स्या}‚न्य‚भावेऽ‚{\color{DodgerBlue3}‚द‚र्श‚नात्} स‚हान‚व‚स्थान‚ग‚ति‚{\color{DodgerBlue3}‚रिति चेत्} । न‚नु य‚देवाद‚र्श‚नं ‚{\color{DodgerBlue3}‚सा(ऽ)प्र‚माण‚ता} प्र‚माण‚र‚हित‚ता‚{\tiny $_{lb}$}‚ऽनुप‚ल‚ब्धिरित्य‚र्थः । (८८)
	\pend% ending standard par
      \label{div_pvv.2.89}
	  
	% new div opening: depth here is 2
	
	  \bigskip
	  \begingroup
	
	    \large
	  
	    \begin{quote}
	  
	    
	    \stanza[\smallbreak]
	\label{pv.2.89}\flagstanza{\tiny\textenglish{...v.2.89}}त‚स्मात् स्व‚श‚ब्देनोक्ताऽपि साऽभाव‚स्य प्र‚साधिका ।&य‚स्याप्र‚माणं साऽवाच्यो निषेध‚स्तेन स‚र्व‚था ॥ ८९ ॥\&[\smallbreak]


	
	    \end{quote}
	  
	  \endgroup
	\textsuperscript{\textenglish{144/s}}

	  \pstart \leavevmode% starting standard par
	\hphantom{.}य‚स्माद‚व‚श्यं प‚र‚म्प‚र‚यानुप‚ल‚ब्धेरेव प्र‚तिषेधः । ‚{\color{DodgerBlue3}‚त‚स्मात्स्व‚श‚ब्देना}‚नुप‚ल‚म्भ‚स्व‚रूप ‚{\tiny $_{lb}$}‚वाच‚क‚श‚ब्देनो‚{\color{DodgerBlue3}‚क्ता} स्व‚भावानुप‚ल‚म्भादा‚{\color{DodgerBlue3}‚व‚पि} श‚ब्दाद‚नुक्तापि विरुद्धोप‚ल‚ब्ध्या‚{\tiny $_{6}$}‚दौ ‚{\tiny $_{lb}$}‚‚{\color{DodgerBlue3}‚सानु}‚प‚ल‚ब्धिर‚{\color{DodgerBlue3}‚भाव‚साधिका} । त‚स्या एव प्र‚तिपाद्य‚त्वात् । ‚{\color{DodgerBlue3}‚य‚स्य} तु चा र्व्वा क स्य ‚{\tiny $_{lb}$}‚साऽनुप‚ल‚ब्धिर‚{\color{DodgerBlue3}‚प्र‚माणं} प्र‚तिषेधे साध्ये ‚{\color{DodgerBlue3}‚तेन स‚र्व्व‚था निषेधोऽवाच्यः} प्र‚त्य‚क्ष‚स्य भाव‚मात्र‚{\tiny $_{lb}$}‚विष‚य‚त्वात् । प्र‚माणान्त‚र‚स्य चाभावात् । (८९)
	\pend% ending standard par
      \label{div_pvv.2.90}
	  
	% new div opening: depth here is 2
	
	  \bigskip
	  \begingroup
	
	    \large
	  
	    \begin{quote}
	  
	    
	    \stanza[\smallbreak]
	\label{pv.2.90}\flagstanza{\tiny\textenglish{...v.2.90}}एतेन त‚द्विरुद्धार्थ‚कार्योक्तिरुप‚व‚र्णिता ।&प्र‚योगः केव‚लं भिन्नः स‚र्व‚त्रार्थो न भिद्य‚ते ॥ ९० ॥\&[\smallbreak]


	
	    \end{quote}
	  
	  \endgroup
	

	  \pstart \leavevmode% starting standard par
	\hphantom{.}‚{\color{DodgerBlue3}‚एतेन} स्व‚भाव\edtext{}{\edlabel{pvv.144-1}\label{pvv.144-1}\lemma{भाव}\Bfootnote{विरुद्धोप‚ल‚ब्धि ।}} ‚{\color{DodgerBlue3}‚त}‚त्कार‚ण‚विरुद्धोप‚ल‚ब्ध्योर‚नुप‚ल‚ब्धित्व‚प्र‚तिपाद‚नेन त‚योः ‚{\tiny $_{lb}$}‚\leavevmode\ledsidenote{\textenglish{27b/MA}} स्व‚भाव‚त‚त्कार\edtext{}{\edlabel{pvv.144-2}\label{pvv.144-2}\lemma{त्कार}\Bfootnote{स्व‚भाव‚विरुद्ध‚कार्योप‚ल‚ब्धिः । कार‚ण‚विरुद्ध‚कार्योप‚ल‚ब्धिश्च ।}}ण‚योर्व्विरुद्धार्थ‚स्य ‚{\color{DodgerBlue3}‚कार्योक्तिरुप‚व‚र्ण्णि‚{\tiny $_{7}$}‚ता} बोद्ध‚व्या । य‚था नात्र ‚{\tiny $_{lb}$}‚शीत‚स्प‚र्शो रोम‚ह‚र्ष‚वान्वा पुरुषो धूमादिति । अत्रापि निषेध्यानुप‚ल‚म्भ‚स्य प‚रंप‚र‚या ‚{\tiny $_{lb}$}‚प्र‚तिपाद‚नात् । उप‚ल‚क्ष‚ण‚ञ्चैत‚द् व्याप‚क‚विरुद्ध‚त‚त्कार्योप‚ल‚ब्धी अपि बोद्ध‚व्ये । य‚था ‚{\tiny $_{lb}$}‚नात्र तुषार‚स्प‚र्शो व‚ह्नेर्धूमादिति । एतासु स्व‚भावानुप‚ल‚ब्धिविरुद्धोप‚ल‚ब्ध्यादिषु ‚{\tiny $_{lb}$}‚‚{\color{DodgerBlue3}‚प्र‚योगः} श‚ब्दाभिधाव्यापारः प‚र‚म‚नुप‚ल‚म्भोप‚ल‚म्भ‚प्र‚तिपाद‚क‚त्वेन भिद्य‚ते । ‚{\color{DodgerBlue3}‚स‚र्व्व‚त्रार्थो} निषेध्यानुप‚ल‚म्भ‚ल‚क्ष‚णो ‚{\color{DodgerBlue3}‚न भि‚{\tiny $_{1}$}‚द्य‚ते} त‚स्यैव स‚र्व्व‚त्र प्र‚तिपाद्य‚त्वात् । (९०)
	\pend% ending standard par
      \label{div_pvv.2.91}
	  
	% new div opening: depth here is 2
	
	  \bigskip
	  \begingroup
	
	    \large
	  
	    \begin{quote}
	  
	    
	    \stanza[\smallbreak]
	\label{pv.2.91}\flagstanza{\tiny\textenglish{...v.2.91}}विरुद्धं त‚च्च सोपाय‚म‚विधायापिधाय च ।&प्र‚माणोक्तिर्निषेधे या न साम्नायानुसारिणी ॥ ९१ ॥\&[\smallbreak]


	
	    \end{quote}
	  
	  \endgroup
	

	  \pstart \leavevmode% starting standard par
	\edtext{\textsuperscript{*}}{\edlabel{pvv.144-3}\label{pvv.144-3}\lemma{*}\Bfootnote{सांख्येन प्र‚तिब‚न्धान्त‚रं विरोध इष्ट‚स्तादात्म्य‚त‚दुत्प‚त्तिभ्याम‚स्यासंग्र‚हा‚{\tiny $_{lb}$}‚दिति सोत्र स्व‚भावानुप‚ल‚म्भेऽन्त‚र्भावितो विरोध‚स्यापि दृश्यानुप‚ल‚म्भ‚ब‚लादेव सिद्धेः । ‚{\tiny $_{lb}$}‚तीर्थ्यास्तु विना विरुद्ध‚विधिम‚नुप‚ल‚म्भ‚ञ्च स्व‚तो निवृत्तेरित्याहुस्तानाह । या ‚{\tiny $_{lb}$}‚पुन‚रित्यादि ।}}या पुन‚र्द्विविधेनापि विरोधेन विरुद्ध‚म‚र्थ‚म‚भिधा\edtext{}{\edlabel{pvv.144-4}\label{pvv.144-4}\lemma{भिधा}\Bfootnote{क्रिय‚ते प्र‚माणोक्तिः । अकृत्वा ।}} य यं क‚ञ्चिद‚र्थ‚मुप‚द‚र्श्य‚ते ‚{\tiny $_{lb}$}‚त‚न्निषेध्य‚ञ्च स‚हा\edtext{}{\edlabel{pvv.144-5}\label{pvv.144-5}\lemma{हा}\Bfootnote{स‚हान‚व‚स्थान‚प‚र‚स्प‚र‚प‚रिहार‚विरोधाभ्यां ।}}भाव‚निश्च‚योपायेन कार‚ण‚व्याप‚क‚कार्यानुप‚ल‚म्भे व‚र्त‚मान‚म‚{\tiny $_{lb}$}‚‚{\color{DodgerBlue3}‚विधाया}‚त एवा‚{\color{DodgerBlue3}‚पिधाय} च पिधान‚मिव पिधानं स्व‚रूप‚प्र‚तीतिविरोधित्वा\edtext{}{\edlabel{pvv.144-6}\label{pvv.144-6}\lemma{तीतिविरोधित्वा}\Bfootnote{उप‚ल‚ब्धिल‚क्ष‚ण‚प्राप्त‚स्यानुप‚ल‚ब्धिम‚कृत्वाऽप्र‚द‚र्श्य ।}}द‚नुप‚ल‚म्भः। ‚{\tiny $_{lb}$}‚त‚म‚कृत्वा कार‚णाद्य‚नुप‚ल‚म्भाप्र‚द‚र्श‚नात्त‚दुप‚ल‚म्भ‚संभाव‚ना न व्याह‚तैव । एवं ‚{\tiny $_{lb}$}‚\leavevmode\ledsidenote{\textenglish{145/s}} निषेध्योप‚ल‚म्भ‚स‚म्भा‚{\tiny $_{2}$}‚व‚नाम‚निवार्यान्य‚विधिना ‚{\color{DodgerBlue3}‚निषेधे} क‚र्त‚व्ये ‚{\color{DodgerBlue3}‚प्र‚मा\edtext{}{\edlabel{pvv.145-1}\label{pvv.145-1}\lemma{मा}\Bfootnote{आग‚मे स‚र्व्व‚विदो व‚च‚न‚कौश‚ल‚दृष्टेः ।}}णोक्तिर्या ‚{\tiny $_{lb}$}‚न साम्नायानुसारिणी} । (९१)
	\pend% ending standard par
      \label{div_pvv.2.92}
	  
	% new div opening: depth here is 2
	
	  \bigskip
	  \begingroup
	
	    \large
	  
	    \begin{quote}
	  
	    
	    \stanza[\smallbreak]
	\label{pv.2.92a}\flagstanza{\tiny\textenglish{....2.92a}}उक्त्यादेः स‚र्व‚वित्प्रेत्य‚भावादिप्र‚तिषेध‚व‚त् ।\&[\smallbreak]


	
	    \end{quote}
	  
	  \endgroup
	

	  \pstart \leavevmode% starting standard par
	\hphantom{.}‚{\color{DodgerBlue3}‚उक्त्यादेर्हेतोः । स‚र्व्व‚विदः प्रेत्य‚भाव‚स्य} प‚र‚लोक‚स्य ‚{\color{DodgerBlue3}‚निषेध‚व‚त्} । य‚था न ‚{\tiny $_{lb}$}‚स‚र्व्व‚ज्ञ‚त्व‚म‚स्य पुरुष‚स्य व‚क्तृत्वात् प्रेत्य‚भावो पु\edtext{}{\edlabel{pvv.145-2}\label{pvv.145-2}\lemma{पु}\Bfootnote{अर्ह‚त्पुरुष‚व‚त् । स‚न्धानं ।}}रुष‚त्वादित्यादि । (।)
	\pend% ending standard par
      

	  \pstart \leavevmode% starting standard par
	क‚स्मात्पुन‚रियं विरुद्धोक्तिरेव नेत्याह (।)
	\pend% ending standard par
      
	  \bigskip
	  \begingroup
	
	    \large
	  
	    \begin{quote}
	  
	    
	    \stanza[\smallbreak]
	\label{pv.2.92b}\flagstanza{\tiny\textenglish{....2.92b}}अतीन्द्रियाणाम‚र्थानां विरोध‚स्याप्र‚सिद्धितः ॥ ९२ ॥\&[\smallbreak]


	
	    \end{quote}
	  
	  \endgroup
	

	  \pstart \leavevmode% starting standard par
	\hphantom{.}‚{\color{DodgerBlue3}‚अतीन्द्रियाणां} सार्व्व‚ज्ञ‚प‚र‚लोकादीना‚{\color{DodgerBlue3}‚म‚र्थानां} केन‚चित् व‚क्तृत्वादिना ‚{\color{DodgerBlue3}‚विरोध‚स्या‚{\tiny $_{lb}$}‚प्र‚सिद्धितः} । ज्ञानोत्क‚र्षाप‚{\tiny $_{3}$}‚क‚र्ष‚योर्व्व‚च‚नाप‚क‚र्षोत्क‚र्षाद‚र्श‚नाद्विप‚र्य‚य‚द‚र्श‚नाच्च । (९२)
	\pend% ending standard par
      \label{div_pvv.2.93}
	  
	% new div opening: depth here is 2
	
	  \bigskip
	  \begingroup
	
	    \large
	  
	    \begin{quote}
	  
	    
	    \stanza[\smallbreak]
	\label{pv.2.93}\flagstanza{\tiny\textenglish{...v.2.93}}बाध्य‚बाध‚क‚भावः कः स्यातां य‚द्युक्तिसंविदौ ।&तादृशोऽनुप‚ल‚ब्धेश्चेद् उच्य‚तां सैव साध‚न‚म् ॥ ९३ ॥\&[\smallbreak]


	
	    \end{quote}
	  
	  \endgroup
	

	  \pstart \leavevmode% starting standard par
	\hphantom{.}य‚दि चो‚{\color{DodgerBlue3}‚क्तिस‚म्विदौ} स‚ह ‚{\color{DodgerBlue3}‚स्यातां क}‚स्त‚योः प‚र‚स्प‚रं ‚{\color{DodgerBlue3}‚बाध्य‚बाध‚क‚भावः} । न हि ‚{\tiny $_{lb}$}‚ज्ञान‚व‚च‚न‚योर्व्विरोधः सिद्धः येनैक‚स‚त्वेऽप‚र‚स्याभावः । अविरोधोपि न सिद्ध इति‚{\tiny $_{lb}$}‚चेत् स‚त्यं किन्तु संश‚येपि न व‚क्तृत्व‚म‚नैकान्तिकं । ‚{\color{DodgerBlue3}‚तादृशः} स‚र्व्व‚ज्ञ‚स्य ‚{\color{DodgerBlue3}‚व‚क्तुर‚नुप‚{\tiny $_{lb}$}‚ल‚ब्धे}‚र‚भाव‚श्चेत् ‚{\color{DodgerBlue3}‚उच्य‚तां} निषेधे साध्ये सैवानुप‚ल‚ब्धिः‚{\tiny $_{4}$}‚ ‚{\color{DodgerBlue3}‚साध‚नं} नात्रान्य‚स्य श‚क्तिः । ‚{\tiny $_{lb}$}‚(९३)
	\pend% ending standard par
      \label{div_pvv.2.94}
	  
	% new div opening: depth here is 2
	
	  \bigskip
	  \begingroup
	
	    \large
	  
	    \begin{quote}
	  
	    
	    \stanza[\smallbreak]
	\label{pv.2.94}\flagstanza{\tiny\textenglish{...v.2.94}}अनिश्च‚य‚क‚रं प्रोक्त‚मीदृक् क्वानुप‚ल‚म्भ‚न‚म् ।&त‚न्नात्य‚न्त‚प‚रोक्षेषु स‚द‚स‚त्ताविनिश्च‚यौ ॥ ९४ ॥\&[\smallbreak]


	
	    \end{quote}
	  
	  \endgroup
	

	  \pstart \leavevmode% starting standard par
	\hphantom{.}किन्त्वीदृग‚तीन्द्रियार्थ‚विष‚य‚{\color{DodgerBlue3}‚म‚नुप‚ल‚म्भ‚न‚म‚निश्च‚य‚क‚रं प्रोक्तं} स‚त्य‚प्य‚र्थे ‚{\tiny $_{lb}$}‚स‚म्भ‚वात् । ‚{\color{DodgerBlue3}‚त‚त्} त‚स्मा‚{\color{DodgerBlue3}‚व‚त्य‚न्त‚प‚रोक्षेषु स‚द‚स‚त्तानिश्च‚यौ न स‚तः} । स‚त्य‚पि ‚{\tiny $_{lb}$}‚प्र‚माणावृ\edtext{}{\edlabel{pvv.145-3}\label{pvv.145-3}\lemma{माणावृ}\Bfootnote{सुमेर्व्वादौ ।}}त्तेः । प्र‚माण‚निवृत्ताव‚प्य\edtext{}{\edlabel{pvv.145-4}\label{pvv.145-4}\lemma{प्य}\Bfootnote{पिशाचादेः ।}}र्थाभावासिद्धेः । त‚स्मान्नानुप\edtext{}{\edlabel{pvv.145-5}\label{pvv.145-5}\lemma{स्मान्नानुप}\Bfootnote{अदृश्यानुप‚ल‚ब्धेः ।}}ल‚ब्धेः कुत‚श्चिद‚{\tiny $_{lb}$}‚न्य\edtext{}{\edlabel{pvv.145-6}\label{pvv.145-6}\lemma{न्य}\Bfootnote{अविरुद्धात् ।}}स्माद‚भाव‚सिद्धिः । किन्तु विरुद्धादेव । (९४)
	\pend% ending standard par
      \label{div_pvv.2.95}
	  
	% new div opening: depth here is 2
	
	  \bigskip
	  \begingroup
	
	    \large
	  
	    \begin{quote}
	  
	    
	    \stanza[\smallbreak]
	\label{pv.2.95}\flagstanza{\tiny\textenglish{...v.2.95}}भिन्नोऽभिन्नोपि वा ध‚र्मः स विरुद्धः प्र‚युज्य‚ते ।&य‚थाऽग्निर‚हिमे साध्ये स‚त्ता वा ज‚न्म‚बाध‚नी ॥ ९५ ॥\&[\smallbreak]


	
	    \end{quote}
	  
	  \endgroup
	

	  \pstart \leavevmode% starting standard par
	\hphantom{.}‚{\color{DodgerBlue3}‚स च विरुद्धो ध‚र्मः} क्व‚चिद् ‚{\color{DodgerBlue3}‚भिन्नो वा प्र‚युज्य‚ते य‚थाग्निर‚हिमे} हि‚{\tiny $_{5}$}‚माभावे ‚{\color{DodgerBlue3}‚साध्ये} (।) ‚{\color{DodgerBlue3}‚क्व‚चिद‚भिन्नो वा य‚था स‚त्ता ज‚न्म‚नो बाध‚नी} म‚ह‚दादीनां ज‚न्माभाव‚प्र‚स‚ङ्गः ‚{\tiny $_{lb}$}‚\leavevmode\ledsidenote{\textenglish{146/s}} स‚त्त्वादिति स‚त्ताज‚न्म‚नोश्चाभेदः सां-ख्या\edtext{}{\edlabel{pvv.146-1}\label{pvv.146-1}\lemma{ख्या}\Bfootnote{स‚दुत्प‚द्य‚ते आविर्भाव‚तिरोभावः प‚रं वैभाष्योपि स‚द्वादी ।}}भ्युप‚ग‚मादुच्य‚ते । न तु प‚र‚मार्थ‚तः ‚{\tiny $_{lb}$}‚तादात्म्यं विरुद्ध‚योर‚स्ति । विरुद्धे च स‚त्ताज‚न्म‚नी प्राग‚स‚त्त‚या हि अभिव्य‚क्ति‚{\tiny $_{lb}$}‚रूप‚ताय लाभो ज‚न्मोच्य‚ते । स‚त्ता तु विद्य‚मान‚ता । त‚तो येन रूपेण स‚त्त्वं न तेन ‚{\tiny $_{lb}$}‚ज‚न्म । येन च ज‚न्म तेन‚{\tiny $_{6}$}‚ न स‚त्त्व‚मिति व्य‚क्तः प‚र‚स्प‚र‚प‚रिहार‚स्थितिर्व्विरोधः । (९५)
	\pend% ending standard par
      \label{div_pvv.2.96}
	  
	% new div opening: depth here is 2
	
	  \bigskip
	  \begingroup
	
	    \large
	  
	    \begin{quote}
	  
	    
	    \stanza[\smallbreak]
	\label{pv.2.96}\flagstanza{\tiny\textenglish{...v.2.96}}य‚था व‚स्त्वेव व‚स्तूनां साध‚ने साध‚नं म‚त‚म् ।&त‚था व‚स्त्वेव व‚स्तूनां स्व‚निवृत्तौ निव‚र्त्त‚क‚म् ॥ ९६ ॥\&[\smallbreak]


	
	    \end{quote}
	  
	  \endgroup
	

	  \pstart \leavevmode% starting standard par
	\hphantom{.}‚{\color{DodgerBlue3}‚य‚था} च ‚{\color{DodgerBlue3}‚व‚स्तूनां} साध्यानां ‚{\color{DodgerBlue3}‚साध‚ने साध‚नं व‚स्त्वेव म‚तं} नाव‚स्तु । ‚{\color{DodgerBlue3}‚त‚था व‚स्त्वेव ‚{\tiny $_{lb}$}‚स्व‚स्य निवृत्तौ व‚स्तूनां निव‚र्त्त‚कं} नाव‚स्तु । (९६)
	\pend% ending standard par
      \label{div_pvv.2.97}
	  
	% new div opening: depth here is 2
	
	  \bigskip
	  \begingroup
	
	    \large
	  
	    \begin{quote}
	  
	    
	    \stanza[\smallbreak]
	\label{pv.2.97}\flagstanza{\tiny\textenglish{...v.2.97}}एतेन क‚ल्प‚नान्य‚स्तो य‚त्र क्व‚च‚न स‚म्भ‚वात् ।&ध‚र्मः प‚क्ष‚स‚प‚क्षान्य‚त‚र‚त्वादिर‚पोदितः ॥ ९७ ॥\&[\smallbreak]


	
	    \end{quote}
	  
	  \endgroup
	

	  \pstart \leavevmode% starting standard par
	\hphantom{.}‚{\color{DodgerBlue3}‚एतेन} व‚स्तुनः साध‚न‚त्व‚प्र‚तिपा\edtext{}{\edlabel{pvv.146-2}\label{pvv.146-2}\lemma{तिपा}\Bfootnote{अनित्यः श‚ब्दः प‚क्ष‚स‚प‚क्षान्य‚त‚र‚त्वाद् घ‚ट‚व‚दित्य‚नित्येन प्र‚त्यास‚त्तौ द‚र्शि‚{\tiny $_{lb}$}‚तायां । त‚दैव व‚क्तुरिच्छाव‚शान्नित्यः श‚ब्दः प‚क्ष‚स‚प‚क्षान्य‚त‚र‚त्वादाकाश‚व‚दिति ‚{\tiny $_{lb}$}‚नित्येन प्र‚त्यास‚त्तिरिति क‚म्ब‚लिप्रोक्तं हेतुम‚पाक‚र्त्तुमाह ।}}द‚नेन ‚{\color{DodgerBlue3}‚विक‚ल्प‚नान्य‚स्तो ध‚र्मः प‚क्ष‚स‚प‚क्षान्य‚त‚र‚{\tiny $_{lb}$}‚त्वादि} । आदिश‚ब्दान्म‚दुप‚ग‚मादिश्चा‚{\color{DodgerBlue3}‚पोदितः} प्र‚तिक्षिप्तो बोद्ध‚व्यः । तादृश‚स्य ‚{\tiny $_{lb}$}‚\leavevmode\ledsidenote{\textenglish{28a/MA}} साध‚न‚स्य ‚{\color{DodgerBlue3}‚य‚त्र क्व‚च‚न‚{\tiny $_{7}$}‚} साध्ये ‚{\color{DodgerBlue3}‚स‚म्भ‚वात्} । न हि प‚क्ष‚त्वं स‚प‚क्ष‚त्व‚म्वा जातिनिय‚तं ‚{\tiny $_{lb}$}‚किन्तु वाञ्छाधीन‚म‚व‚स्तु । अतः स‚मीहित‚सिद्धौ स‚र्व्व स‚र्व्व‚स्य सिध्येत् । (९७)
	\pend% ending standard par
      \label{div_pvv.2.98}
	  
	% new div opening: depth here is 2
	
	  \bigskip
	  \begingroup
	
	    \large
	  
	    \begin{quote}
	  
	    
	    \stanza[\smallbreak]
	\label{pv.2.98}\flagstanza{\tiny\textenglish{...v.2.98}}त‚त्रापि व्याप‚को ध‚र्मो निवृत्तेर्ग‚म‚को म‚तः ।&व्याप्य‚स्य स्व‚निवृत्तिश्चेत् प‚रिच्छिन्ना क‚थ‚ञ्च‚न ॥ ९८ ॥\&[\smallbreak]


	
	    \end{quote}
	  
	  \endgroup
	

	  \pstart \leavevmode% starting standard par
	\hphantom{.}‚{\color{DodgerBlue3}‚त‚त्र व‚स्तुन्य‚पि व्याप‚को ध‚र्मः} कार‚णं स्व‚भावो वा । ‚{\color{DodgerBlue3}‚व्याप्य‚स्य} कार्य‚स्य ‚{\color{DodgerBlue3}‚स्व‚भा}‚{\tiny $_{lb}$}‚व‚स्य ‚{\color{DodgerBlue3}‚निवृत्तेर्ग‚म‚को म‚तः} । व्याप‚क‚स्य कार‚ण‚स्व‚भाव‚स्य स्व‚निवृत्तिः ‚{\color{DodgerBlue3}‚क‚थ‚ञ्च‚न} साक्षात्प‚र‚म्प‚र‚या वा ‚{\color{DodgerBlue3}‚प‚रिच्छिन्ना} चेत् स्यात् । (९८)
	\pend% ending standard par
      \label{div_pvv.2.99}
	  
	% new div opening: depth here is 2
	
	  \bigskip
	  \begingroup
	
	    \large
	  
	    \begin{quote}
	  
	    
	    \stanza[\smallbreak]
	\label{pv.2.99a}\flagstanza{\tiny\textenglish{....2.99a}}य‚द‚प्र‚माण‚ताभावे लिङ्गं त‚स्यैव क‚थ्य‚ते ।&त‚द‚त्य‚न्त‚विमूढार्थं;\&[\smallbreak]


	
	    \end{quote}
	  
	  \endgroup
	

	  \pstart \leavevmode% starting standard par
	\hphantom{.}त‚था य‚स्यार्थ‚स्य एक‚ज्ञान‚संस‚र्ग्गिणोऽ‚{\color{DodgerBlue3}‚प्र‚माण‚ता}‚{\tiny $_{1}$}‚ प्र‚माण‚निवृत्तिस्त‚स्यैवा‚{\color{DodgerBlue3}‚भावे-} ऽभाव‚व्य‚व‚हारे च साध्ये सा प्र‚माण‚निवृत्तिः स्व‚भावानुप‚ल‚म्भाख्या ‚{\color{DodgerBlue3}‚लिङ्गं ‚{\tiny $_{lb}$}‚क‚थ्य‚ते । त‚त्स्व}‚भावानुप‚ल‚ब्धिलिङ्ग‚म‚{\color{DodgerBlue3}‚त्य‚न्त‚विमूढार्थ} । ये ह्य‚नुप‚ल‚भ्य‚मान‚म‚पि ‚{\tiny $_{lb}$}‚दृश्य‚म‚स‚त्त‚या मोहान्न व्य‚व‚ह‚र‚न्ति तान् प्र‚ति व्य‚व‚हार‚साध‚कं त‚ल्लिङ्गं ।
	\pend% ending standard par
      \textsuperscript{\textenglish{147/s}}

	  \pstart \leavevmode% starting standard par
	क‚स्मात्पुन‚र‚मूढार्थ‚म‚पि त‚न्नेत्याह (।)
	\pend% ending standard par
      
	  \bigskip
	  \begingroup
	
	    \large
	  
	    \begin{quote}
	  
	    
	    \stanza[\smallbreak]
	\label{pv.2.99b}\flagstanza{\tiny\textenglish{....2.99b}}आगोपाल‚म‚संवृत्तेः ॥ ९९ ॥\&[\smallbreak]


	
	    \end{quote}
	  
	  \endgroup
	

	  \pstart \leavevmode% starting standard par
	\hphantom{.}‚{\color{DodgerBlue3}‚आगोपाल‚म}‚स्यार्थ‚स्या‚{\color{DodgerBlue3}‚संवृ\edtext{}{\edlabel{pvv.147-1}\label{pvv.147-1}\lemma{संवृ}\Bfootnote{विवृतौ}}त्ते}‚र‚गूढ‚त्वात् प्र‚सिद्धेरित्य‚र्थः । (९९)
	\pend% ending standard par
      \label{div_pvv.2.100}
	  
	% new div opening: depth here is 2
	
	  \bigskip
	  \begingroup
	
	    \large
	  
	    \begin{quote}
	  
	    
	    \stanza[\smallbreak]
	\label{pv.2.100}\flagstanza{\tiny\textenglish{....2.100}}एताव‚न्निश्च‚य‚फ‚ल‚म‚भावेनुप‚ल‚म्भ‚न‚म् ।&त‚च्च हेतौ स्व‚भावे वाऽदृश्ये दृश्य‚त‚या म‚ते ॥ १०० ॥\&[\smallbreak]


	
	    \end{quote}
	  
	  \endgroup
	

	  \pstart \leavevmode% starting standard par
	\hphantom{.}‚{\color{DodgerBlue3}‚एताव}‚त्त्रिविध‚म् ।‚{\tiny $_{2}$}‚ कार‚ण‚व्याप‚क‚स्व‚भा‚{\color{DodgerBlue3}‚वानुप‚ल‚म्भ‚नं ।} अभावेऽभाव‚व्य‚व‚हारे ‚{\tiny $_{lb}$}‚च ‚{\color{DodgerBlue3}‚निश्च‚य‚फ‚लं} । कार‚ण‚विरुद्धोप‚ल‚म्भाद‚य‚स्तु कार‚णाद्य‚नुप‚ल‚म्भोप‚ल‚क्ष‚ण‚त्वात् ‚{\tiny $_{lb}$}‚त्रिविध एवान्त‚र्भूताः । ‚{\color{DodgerBlue3}‚त‚च्चा}‚भाव‚निश्च‚य‚फ‚ल‚त्व‚म‚नुप‚ल‚म्भ‚स्य ‚{\color{DodgerBlue3}‚हेतौ स्व‚भावे च} व्याप‚के निषेध्य‚रूपे च ‚{\color{DodgerBlue3}‚दृश्य‚त‚या} द‚र्श‚न‚योग्य‚त‚या म‚तेऽ‚{\color{DodgerBlue3}‚दृश्ये}‚ऽनुप‚ल‚भ्य‚माने स‚ति ‚{\tiny $_{lb}$}‚नान्य‚था ।
	\pend% ending standard par
      

	  \pstart \leavevmode% starting standard par
	त‚देवं त्रिविध‚लिङ्ग‚ज‚म‚नुमानं प्र‚माणं व‚स्तुस‚म्वादात्‚{\tiny $_{3}$}‚ । त‚त‚स्त‚त्प्र‚सिद्ध‚स्या‚{\tiny $_{lb}$}‚नित्य‚तादेर्न व‚स्तुध‚र्म‚तेति स्थितं ।\edtext{\textsuperscript{*}}{\edlabel{pvv.147-2}\label{pvv.147-2}\lemma{*}\Bfootnote{आनुष‚ङ्गिकं स‚माप्य प्र‚कृत‚माह ।}}(१००)
	\pend% ending standard par
      \label{div_pvv.2.101}
	  
	% new div opening: depth here is 2
	

	  \pstart \leavevmode% starting standard par
	न‚न्व‚नित्योऽयं व‚र्ण्ण इति स्व‚ल‚क्ष‚णे योज‚ना त‚त्क‚थं योज‚नाद्व‚र्ण्ण‚सामान्य‚{\tiny $_{lb}$}‚\cref{pv.2.79b} इत्युक्तं । अत्राह (।)
	\pend% ending standard par
      
	  \bigskip
	  \begingroup
	
	    \large
	  
	    \begin{quote}
	  
	    
	    \stanza[\smallbreak]
	\label{pv.2.101}\flagstanza{\tiny\textenglish{....2.101}}अनुमानाद‚नित्यादेर्ग्र‚ह‚णेऽयं क्र‚मो म‚तः ।&प्रामाण्य‚मेव नान्य‚त्र गृहीत‚ग्र‚ह‚णान्म‚त‚म् ॥ १०१ ॥\&[\smallbreak]


	
	    \end{quote}
	  
	  \endgroup
	

	  \pstart \leavevmode% starting standard par
	\hphantom{.}‚{\color{DodgerBlue3}‚अनुमाना}‚त्प‚रोक्ष‚स्या‚{\color{DodgerBlue3}‚नित्यादे}‚र्ध‚र्म‚स्य ‚{\color{DodgerBlue3}‚ग्र\edtext{}{\edlabel{pvv.147-3}\label{pvv.147-3}\lemma{ग्र}\Bfootnote{स‚ति नान्य‚दायं क्र‚मः ।}}ह‚णे} व‚स्तुसामान्य‚म‚नित्य‚त्वेन गृह्य‚त ‚{\tiny $_{lb}$}‚इत्य‚यं ‚{\color{DodgerBlue3}‚क्र‚मो म‚तः} । व‚र्ण्ण‚सामान्य‚ञ्च प्र‚त्य\edtext{}{\edlabel{pvv.147-4}\label{pvv.147-4}\lemma{त्य}\Bfootnote{स‚न्तानोप‚ल‚म्भाव‚स्थायाम‚नित्य‚मेत‚द्व‚व‚र्ण्णादिकं । विकारित्वात् । ज‚लादि‚{\tiny $_{lb}$}‚व‚दिति स्व‚भाव‚हेतुर‚तो व‚स्तुन एवात्र सिद्धिः ।}}क्ष‚तः सिद्धं त‚द्‏ब‚ल‚भाविना विक‚ल्पेन ‚{\tiny $_{lb}$}‚विजातीय‚व्यावृत्त्याश्र‚येण व्य‚व‚स्थाप‚नात् । य‚स्तु विन‚श्व‚रं‚{\tiny $_{4}$}‚ व‚र्ण्णं दृष्ट्वाऽनित्यो‚{\tiny $_{lb}$}‚ऽय‚मिति विशेष‚विष‚यः अनुमानाद‚न्य‚त्र । त‚त्र ‚{\color{DodgerBlue3}‚प्रामाण्य‚मेव न म‚तं गृहीत‚{\tiny $_{lb}$}‚ग्र‚ह‚णात्} (। १०१)
	\pend% ending standard par
      \label{div_pvv.2.102}
	  
	% new div opening: depth here is 2
	

	  \pstart \leavevmode% starting standard par
	क‚थं गृहीत‚ग्राहित्व‚मित्याह (।)
	\pend% ending standard par
      
	  \bigskip
	  \begingroup
	
	    \large
	  
	    \begin{quote}
	  
	    
	    \stanza[\smallbreak]
	\label{pv.2.102a}\flagstanza{\tiny\textenglish{...2.102a}}नान्यास्यानित्य‚ता भावात् पूर्व‚सिद्धः स चैन्द्रियात् ।&नानेक‚रूपो वाच्योऽसौ;\&[\smallbreak]


	
	    \end{quote}
	  
	  \endgroup
	\textsuperscript{\textenglish{148/s}}

	  \pstart \leavevmode% starting standard par
	\hphantom{.}‚{\color{DodgerBlue3}‚नान्या भावाद्व\edtext{}{\edlabel{pvv.148-1}\label{pvv.148-1}\lemma{भावाद्व}\Bfootnote{च‚लात् ।}}र्ण्णादेर‚नित्य‚ता} त‚स्यैव क्ष‚ण‚क्ष‚यिस्व‚भाव‚त्वात् । ‚{\color{DodgerBlue3}‚स च भाव ‚{\tiny $_{lb}$}‚ऐन्द्रियात्प्र‚त्य‚क्ष‚त्सिद्धः} । पूर्व्व विक‚ल्पात् । त‚त‚स्त‚मेव य‚थागृहीतं विक‚ल्प‚य‚न् ‚{\tiny $_{lb}$}‚विक‚ल्पो गृहीत‚ग्राही । य‚त एव नान्याभावाद‚नित्य‚ता । अत एवाभावोऽसौ ध‚र्मि‚{\tiny $_{lb}$}‚ध‚र्म‚{\tiny $_{5}$}‚रूप‚त‚या ‚{\color{DodgerBlue3}‚नानेक‚रूपः} । त‚था न ‚{\color{DodgerBlue3}‚वाच्योऽसौ} श‚ब्दानां ध‚र्मिध‚र्माभाव‚स्य तैर्व्व‚च‚नात् । ‚{\tiny $_{lb}$}‚त‚स्य च व‚स्तुन्य‚स‚म्भ‚वात् ।
	\pend% ending standard par
      

	  \pstart \leavevmode% starting standard par
	क‚स्त‚र्हि वाच्य इत्याह (।)
	\pend% ending standard par
      
	  \bigskip
	  \begingroup
	
	    \large
	  
	    \begin{quote}
	  
	    
	    \stanza[\smallbreak]
	\label{pv.2.102b}\flagstanza{\tiny\textenglish{...2.102b}}वाच्यो ध‚र्मो विक‚ल्प‚जः ॥ १०२ ॥\&[\smallbreak]


	
	    \end{quote}
	  
	  \endgroup
	

	  \pstart \leavevmode% starting standard par
	\hphantom{.}‚{\color{DodgerBlue3}‚विक‚ल्प‚जो} विजातीयाश्र‚येण विक‚ल्प‚क‚ल्पितो ‚{\color{DodgerBlue3}‚ध‚र्मः} प‚र‚स्प‚रं य‚थासंकेत‚म‚{\tiny $_{lb}$}‚संकीर्ण्णः श‚ब्दानां ‚{\color{DodgerBlue3}‚वाच्यः} । (१०२)
	\pend% ending standard par
      \label{div_pvv.2.103}
	  
	% new div opening: depth here is 2
	
	  \bigskip
	  \begingroup
	
	    \large
	  
	    \begin{quote}
	  
	    
	    \stanza[\smallbreak]
	\label{pv.2.103}\flagstanza{\tiny\textenglish{....2.103}}सामान्याश्र‚य‚संसिद्धौ सामान्यं सिद्ध‚मेव त‚त् ।&त‚द‚सिद्धौ त‚थास्यैव ह्य‚नुमानं प्र‚व‚र्त‚ते ॥ १०३ ॥\&[\smallbreak]


	
	    \end{quote}
	  
	  \endgroup
	

	  \pstart \leavevmode% starting standard par
	\hphantom{.}त‚स्मात्सामान्य‚स्यानित्यादेराश्र‚य‚स्य व‚र्ण्णादेः प्र‚त्य‚क्षात् ‚{\color{DodgerBlue3}‚संसिद्धौ सिद्ध‚मेव ‚{\tiny $_{lb}$}‚त‚त्सामान्यं} । य‚दि त्व‚भ्यासाद्य‚भावाद‚नुरूप‚निश्च‚याभावा‚{\tiny $_{6}$}‚द‚ज्ञानं त‚दा प्र‚त्य‚क्षा‚{\tiny $_{lb}$}‚‚{\color{DodgerBlue3}‚त्त‚द‚सिद्धाव‚स्यैव} हि व‚र्ण्णादेस्त‚था नित्य‚त्वेन त‚द‚व्य‚भिचारिलि‚{\color{DodgerBlue3}‚ङ्ग‚द‚नुमानं प्र‚व‚र्त‚ते} । ‚{\tiny $_{lb}$}‚(१०३)
	\pend% ending standard par
      \label{div_pvv.2.104}
	  
	% new div opening: depth here is 2
	

	  \pstart \leavevmode% starting standard par
	क‚स्मात्पुन‚र्गृहीतेप्य‚प‚रिज्ञान‚मित्याह (।)
	\pend% ending standard par
      
	  \bigskip
	  \begingroup
	
	    \large
	  
	    \begin{quote}
	  
	    
	    \stanza[\smallbreak]
	\label{pv.2.104}\flagstanza{\tiny\textenglish{....2.104}}क्व‚चित्त‚द‚प‚रिज्ञानं स‚दृशाप‚र‚स‚म्भ‚वात् ।&भ्रान्तेर‚प‚श्य‚तो भेदं मायागोल‚क‚भेद‚व‚त् ॥ १०४ ॥\&[\smallbreak]


	
	    \end{quote}
	  
	  \endgroup
	

	  \pstart \leavevmode% starting standard par
	\hphantom{.}‚{\color{DodgerBlue3}‚क्व‚चिद}‚नित्य‚तादौ गृहीतेप्य‚नुरूप‚निश्च‚याभावाद‚{\color{DodgerBlue3}‚प‚रिज्ञानं} । पूर्व्व‚क्ष‚ण‚विनाश‚{\tiny $_{lb}$}‚काले त‚त्स‚दृश‚स्याप‚र‚स्य ‚{\color{DodgerBlue3}‚स‚म्भ‚वात्} । भाव‚शून्यान्त‚रा ल‚क्ष‚णाभावात् । स एवाय‚मिति ‚{\tiny $_{lb}$}‚\leavevmode\ledsidenote{\textenglish{28b/MA}} भ्रान्तेः स्थैर्य‚ग्राहिण्याः क्ष‚णानां ‚{\color{DodgerBlue3}‚भेद‚म‚{\tiny $_{7}$}‚प‚श्य‚तो}‚ऽनिश्चिच‚न्व‚तः पुंसो दृष्टान्त‚माह (।) ‚{\tiny $_{lb}$}‚‚{\color{DodgerBlue3}‚मायागोल‚क‚भेद‚व‚त्} । मायाकार‚द‚र्शितौ गोल‚कौ भिन्नाव‚ध्य‚क्षेण गृहीत्वापि सादृश्या‚{\tiny $_{lb}$}‚ल्लाघ‚वाच्च विप्र‚ल‚ब्ध‚बुद्धिरेक‚त्वेनाध्य‚व‚स्य‚ति य‚था त‚था स्थिराध्य‚व‚सायः । (१०४)
	\pend% ending standard par
      \label{div_pvv.2.105}
	  
	% new div opening: depth here is 2
	

	  \pstart \leavevmode% starting standard par
	न‚नु य‚दि क्व‚चिन्नाशो दृश्येंत त‚दास्या\edtext{}{\edlabel{pvv.148-2}\label{pvv.148-2}\lemma{दास्या}\Bfootnote{दृश्येतैव दीपादिरिति नित्यं गौः भावी ।}}हेतुक‚स्य स्व‚भाव‚त्वात् क्ष‚ण‚क्ष‚यिषु भावेषु ‚{\tiny $_{lb}$}‚एक‚ताबुद्धेर्भ्रान्त‚त्वं भ‚वेदित्याह (।)
	\pend% ending standard par
      
	  \bigskip
	  \begingroup
	
	    \large
	  
	    \begin{quote}
	  
	    
	    \stanza[\smallbreak]
	\label{pv.2.105}\flagstanza{\tiny\textenglish{....2.105}}त‚था ह्य‚लिङ्ग‚माबाल‚म‚संश्लिष्टोत्त‚रोद‚य‚म् ।&प‚श्य‚न् प‚रिच्छिन‚त्त्येव दीपादिं नाशिनं ज‚नः ॥ १०५ ॥\&[\smallbreak]


	
	    \end{quote}
	  
	  \endgroup
	\textsuperscript{\textenglish{149/s}}

	  \pstart \leavevmode% starting standard par
	\hphantom{.}‚{\color{DodgerBlue3}‚त‚था} हि ‚{\color{DodgerBlue3}‚दीपादि}‚(।) आदिश‚ब्दात्तुर‚ङ्ग‚दिस‚न्तान‚{\tiny $_{1}$}‚विच्छेद‚कालेऽ‚{\color{DodgerBlue3}‚संश्लिष्टोत्त‚रो-} द‚य‚म‚घ‚टित उत्त‚र‚स्य उद‚यो ज‚न्म य‚स्मिन् त‚मेक‚स्य स्व‚र‚स‚निरोधित्वेऽप‚र‚स्य कार‚णा‚{\tiny $_{lb}$}‚भावात् । अनुत्प‚त्तौ ‚{\color{DodgerBlue3}‚नाशिन‚म}‚ध्य‚क्ष‚तः ‚{\color{DodgerBlue3}‚प‚श्य‚न्न}‚लिङ्ग‚ङ्ग‚म‚क‚लिङ्ग‚र‚हित‚माबालं बाल‚{\tiny $_{lb}$}‚प‚र्य‚न्तं ज‚नः ‚{\color{DodgerBlue3}‚प‚रिच्छिन‚त्त्येव} । त‚तः प्राग‚पि स‚तः स्व‚र‚स‚निरोधात् स‚दृशाप‚राप‚र‚क्ष‚ण‚{\tiny $_{lb}$}‚प्र‚च‚ये स एवाय‚मिति बुद्धिभ्रान्तिरेव । (१०५)
	\pend% ending standard par
      \label{div_pvv.2.106}
	  
	% new div opening: depth here is 2
	

	  \pstart \leavevmode% starting standard par
	य‚था (।)
	\pend% ending standard par
      
	  \bigskip
	  \begingroup
	
	    \large
	  
	    \begin{quote}
	  
	    
	    \stanza[\smallbreak]
	\label{pv.2.106}\flagstanza{\tiny\textenglish{....2.106}}भाव‚स्व‚भाव‚भूतायाम‚पि श‚क्तौ फ‚लेऽदृशः ।&अनान‚न्त‚र्य‚तो मोहो विनिश्चेतुर‚पाट‚वात् ॥ १०६ ॥\&[\smallbreak]


	
	    \end{quote}
	  
	  \endgroup
	

	  \pstart \leavevmode% starting standard par
	\hphantom{.}‚{\color{DodgerBlue3}‚भाव‚स्य}‚बीजादेः ‚{\color{DodgerBlue3}‚स्व‚भाव‚भूतायाम‚पि}‚{\tiny $_{2}$}‚ अङ्कुरादिज‚निकायां ‚{\color{DodgerBlue3}‚श‚क्तौ फ‚लेऽङ्कुरादौ ‚{\tiny $_{lb}$}‚अनान\edtext{}{\edlabel{pvv.149-1}\label{pvv.149-1}\lemma{अनान}\Bfootnote{फ‚ल‚स्यान‚न्त‚र‚म‚भावात्}}न्त‚र्य‚तोऽदृशो}‚ऽद‚र्श‚नात् ‚{\color{DodgerBlue3}‚विनिश्चेतुः} पुन्सोऽ‚{\color{DodgerBlue3}‚पाट‚वात् मोहो}‚ऽश‚क्त‚भ्र‚मः । त‚था ‚{\tiny $_{lb}$}‚क्ष‚णिकेषु भावेषु स‚दृशाप‚राप‚रोत्प‚त्तेर्भाव‚शून्य‚क्ष‚णाद‚र्श‚नाच्च स्थिर‚भ्र‚मः । (१०६)
	\pend% ending standard par
      \label{div_pvv.2.107}
	  
	% new div opening: depth here is 2
	
	  \bigskip
	  \begingroup
	
	    \large
	  
	    \begin{quote}
	  
	    
	    \stanza[\smallbreak]
	\label{pv.2.107}\flagstanza{\tiny\textenglish{....2.107}}त‚स्यैव विनिवृत्य‚र्थ‚म‚नुमानोप‚व‚र्ण‚न‚म् ।&व्य‚व‚स्य‚न्तीक्ष‚णादेव स‚र्वाकारान् म‚हाधियः ॥ १०७ ॥\&[\smallbreak]


	
	    \end{quote}
	  
	  \endgroup
	

	  \pstart \leavevmode% starting standard par
	\hphantom{.}त‚स्यैक‚त्व‚भ्र‚म‚स्यैव ‚{\color{DodgerBlue3}‚निवृत्त्य‚र्थ} स‚म्वादिलिङ्ग‚ज‚स्य क्ष‚णिक‚ता विष‚य‚स्यानु‚{\tiny $_{lb}$}‚‚{\color{DodgerBlue3}‚मान‚स्योप‚व‚र्ण्ण‚नं} निश्च‚यारोप‚म‚न‚सोर्ब्बाध्य‚बाध‚क‚भाव‚त (१।५०) इति न्यायात् । ‚{\tiny $_{lb}$}‚ये तु‚{\tiny $_{3}$}‚ ‚{\color{DodgerBlue3}‚म‚हाधियो} विप‚रीत‚व्य‚व‚सायानाक्रान्त‚प्र‚त्य‚क्षा योगिन‚स्ते प‚दार्थ‚स्य ‚{\color{DodgerBlue3}‚ईक्ष‚णादेव ‚{\tiny $_{lb}$}‚स‚र्व्वानाकार‚न् व्य‚व‚स्य‚न्ति} निश्चिन्व‚न्ति । (१०७)
	\pend% ending standard par
      \label{div_pvv.2.108}
	  
	% new div opening: depth here is 2
	

	  \pstart \leavevmode% starting standard par
	एव‚ञ्च (।)
	\pend% ending standard par
      
	  \bigskip
	  \begingroup
	
	    \large
	  
	    \begin{quote}
	  
	    
	    \stanza[\smallbreak]
	\label{pv.2.108}\flagstanza{\tiny\textenglish{....2.108}}व्यावृत्ते स‚र्व‚त‚स्त‚स्मिन् व्यावृत्तिविनिब‚न्ध‚नाः ।&बुद्ध‚योऽर्थे प्र‚व‚र्त्त‚न्तेऽभिन्ने भिन्नाश्र‚या इव ॥ १०८ ॥\&[\smallbreak]


	
	    \end{quote}
	  
	  \endgroup
	

	  \pstart \leavevmode% starting standard par
	\hphantom{.}‚{\color{DodgerBlue3}‚त‚स्मिन्न‚र्थे} व‚स्तुतोऽभिन्ने निरंशे ‚{\color{DodgerBlue3}‚स‚र्व्व‚तः} स‚जातीयाद् विजातीयाच्च ‚{\color{DodgerBlue3}‚व्यावृत्ते} याव‚{\tiny $_{lb}$}‚त्त्यो व्यावृत्त‚यः स‚न्ति ताव‚द् ‚{\color{DodgerBlue3}‚व्यावृत्तिविनिब‚न्ध‚ना} निश्च‚यात्मिका ‚{\color{DodgerBlue3}‚बुद्ध‚यो भिन्नाश्र‚या ‚{\tiny $_{lb}$}‚इव} त‚त्त‚द्‏व्यावृत्तिमात्र‚विष‚य‚त्वेन भिन्न‚ध‚र्मिध‚र्म्मादिगोच‚रा इव ‚{\color{DodgerBlue3}‚प्र‚व‚र्त‚न्ते} । (१०८)
	\pend% ending standard par
      \label{div_pvv.2.109}
	  
	% new div opening: depth here is 2
	
	  \bigskip
	  \begingroup
	
	    \large
	  
	    \begin{quote}
	  
	    
	    \stanza[\smallbreak]
	\label{pv.2.109}\flagstanza{\tiny\textenglish{....2.109}}य‚थाचोद‚न‚माख्याश्च । सोऽस‚ति भ्रान्तिकार‚णे ।&प्र‚तिभाः प्र‚तिस‚न्ध‚त्ते स्वानुरूपाः स्व‚भाव‚तः ॥ १०९ ॥\&[\smallbreak]


	
	    \end{quote}
	  
	  \endgroup
	

	  \pstart \leavevmode% starting standard par
	\hphantom{.}‚{\color{DodgerBlue3}‚य‚थाचो‚{\tiny $_{4}$}‚द‚नं} य‚थासंकेत‚{\color{DodgerBlue3}‚माख्याः} श‚ब्दा‚{\color{DodgerBlue3}‚श्च} प्र‚व‚र्त‚न्ते । य‚त्र व्यावृत्तौ यः श‚ब्दो ‚{\tiny $_{lb}$}‚विनिवेशितः । ‚{\color{DodgerBlue3}‚स} च त‚स्यां निश्चितायां प्र‚व‚र्त‚ते स भावोऽ‚{\color{DodgerBlue3}‚स‚ति भ्रान्तिकार‚णे} य‚था‚{\tiny $_{lb}$}‚भ्यासं ‚{\color{DodgerBlue3}‚स्वानुरूपाः} स्व‚व्यावृत्तिस‚मुचिताः ‚{\color{DodgerBlue3}‚प्र‚तिभा} निश्च‚य‚बुद्धीः ‚{\color{DodgerBlue3}‚स्व‚भाव‚तः प्र‚ति-} \leavevmode\ledsidenote{\textenglish{150/s}} ‚{\color{DodgerBlue3}‚संध‚त्ते} उत्पाद‚य‚ति । त‚स्माद‚भ्यास‚व‚तामीक्ष‚णादेवानित्यादिनिश्च‚यः अन्येषान्तु ‚{\tiny $_{lb}$}‚स्थिरारोप‚णार्थ‚म‚नुमान‚मिति स्थित‚मेत‚त् । (१०९)
	\pend% ending standard par
      \label{div_pvv.2.110}
	  
	% new div opening: depth here is 2
	

	  \pstart \leavevmode% starting standard par
	अन्ये त्वाचार्याः‚{\tiny $_{5}$}‚ प्राहुः ।\edtext{\textsuperscript{*}}{\edlabel{pvv.150-1}\label{pvv.150-1}\lemma{*}\Bfootnote{तेषां प्र‚ध्वंसाभावेऽनित्य‚ताभिम‚ता स य‚था सिध्य‚ति त‚द्द‚र्श‚य‚ति ।}}
	\pend% ending standard par
      
	  \bigskip
	  \begingroup
	
	    \large
	  
	    \begin{quote}
	  
	    
	    \stanza[\smallbreak]
	\label{pv.2.110a}\flagstanza{\tiny\textenglish{...2.110a}}सिद्धोऽत्राप्य‚थ‚वा ध्वंसो लिङ्गाद‚नुप‚ल‚म्भ‚नात् ।\&[\smallbreak]


	
	    \end{quote}
	  
	  \endgroup
	

	  \pstart \leavevmode% starting standard par
	\hphantom{.}‚{\color{DodgerBlue3}‚अत्र} व‚र्ण्ण‚स‚न्तान‚विच्छेद‚कालेपि स‚न्तान‚विच्छेद‚ल‚क्ष‚णो ध्वंसोऽनित्य‚ता न भाव‚{\tiny $_{lb}$}‚स्व‚भावात्मिका । स ‚{\color{DodgerBlue3}‚चानुप\edtext{}{\edlabel{pvv.150-2}\label{pvv.150-2}\lemma{चानुप}\Bfootnote{उप‚ल‚ब्धिल‚क्ष‚ण‚प्राप्त‚स्यान्त्य‚क्ष‚ण‚स्य ।}}ल‚म्भ‚नाल्लिङ्ग‚त्सिद्धः}‚। नाध्य‚क्षाद् भाव‚विष‚य‚त्वात्त‚स्य ।
	\pend% ending standard par
      

	  \pstart \leavevmode% starting standard par
	क‚स्मात्पुन‚र्ध्वंसोऽनित्य‚ता इत्याह (।)
	\pend% ending standard par
      
	  \bigskip
	  \begingroup
	
	    \large
	  
	    \begin{quote}
	  
	    
	    \stanza[\smallbreak]
	\label{pv.2.110b}\flagstanza{\tiny\textenglish{...2.110b}}प्राग् भूत्वा ह्य‚भ‚व‚द्भावोनित्य इत्य‚भिधीय‚ते ॥ ११० ॥\&[\smallbreak]


	
	    \end{quote}
	  
	  \endgroup
	

	  \pstart \leavevmode% starting standard par
	\hphantom{.}‚{\color{DodgerBlue3}‚प्राग् भूत्वा हि भावः} प‚श्चाद‚{\color{DodgerBlue3}‚भ‚व‚न्न‚नित्य इत्य‚भिधीय‚ते} न तु भाव इत्येव । ‚{\tiny $_{lb}$}‚त‚था ध्वंस एवानित्य‚ता सा चानुप‚ल‚ब्धिलिङ्ग‚जाऽनुमान‚ग‚म्या । (११०)
	\pend% ending standard par
      \label{div_pvv.2.111}
	  
	% new div opening: depth here is 2
	

	  \pstart \leavevmode% starting standard par
	न‚नु प्राक् प‚श्चाद‚भाव‚योर्व्य‚व‚धाय‚कः स‚त्ता‚{\tiny $_{6}$}‚स‚म्ब‚न्धोऽनित्य‚ता न प्र‚ध्वंसाभाव ‚{\tiny $_{lb}$}‚इत्याह (।)
	\pend% ending standard par
      
	  \bigskip
	  \begingroup
	
	    \large
	  
	    \begin{quote}
	  
	    
	    \stanza[\smallbreak]
	\label{pv.2.111}\flagstanza{\tiny\textenglish{....2.111}}य‚स्योभ‚यान्त‚व्य‚व‚धिस‚त्तास‚म्ब‚न्ध‚वाचिनी ।&अनित्य‚ता-श्रुतिस्तेन ताव‚न्ताविति कौ स्मृतौ ॥ १११ ॥\&[\smallbreak]


	
	    \end{quote}
	  
	  \endgroup
	

	  \pstart \leavevmode% starting standard par
	\hphantom{.}‚{\color{DodgerBlue3}‚य‚स्य} नै या यि का देरु‚{\color{DodgerBlue3}‚भ‚य}‚स्य प्राक् प‚श्चाद‚भा\edtext{}{\edlabel{pvv.150-3}\label{pvv.150-3}\lemma{भा}\Bfootnote{आकाश‚ञ्चास‚दि (ति) विशिन‚ष्टि द्विम‚ध्य‚स्थेति ।}}व‚स्या‚{\color{DodgerBlue3}‚न्त}‚र‚स्य यो ‚{\color{DodgerBlue3}‚व्य‚व‚धा}‚य‚कः ‚{\tiny $_{lb}$}‚स‚त्ता‚{\color{DodgerBlue3}‚स‚म्ब‚न्ध}‚स्त‚द्वाचिन्य‚नित्य‚ता ‚{\color{DodgerBlue3}‚श्रुति}‚रिष्टा ‚{\color{DodgerBlue3}‚तेन} वादिना ‚{\color{DodgerBlue3}‚ताव}‚न्ताविति कौ\edtext{}{\edlabel{pvv.150-4}\label{pvv.150-4}\lemma{कौ}\Bfootnote{पृच्छ‚त्य‚युक्त‚त्वात् ।}} ‚{\color{DodgerBlue3}‚स्मृतौ} । ‚{\tiny $_{lb}$}‚(१११)
	\pend% ending standard par
      \label{div_pvv.2.112}
	  
	% new div opening: depth here is 2
	
	  \bigskip
	  \begingroup
	
	    \large
	  
	    \begin{quote}
	  
	    
	    \stanza[\smallbreak]
	\label{pv.2.112}\flagstanza{\tiny\textenglish{....2.112}}प्राक् प‚श्चाद‚प्य‚भाव‚श्चेत् स एवानित्य‚ता न किम् ।&ष‚ष्ठ्याद्य‚योगादिति चेद् अन्त‚योः स क‚थं भ‚वेत् ॥ ११२ ॥\&[\smallbreak]


	
	    \end{quote}
	  
	  \endgroup
	

	  \pstart \leavevmode% starting standard par
	\hphantom{.}‚{\color{DodgerBlue3}‚प्राग्}‚भावः ‚{\color{DodgerBlue3}‚प‚श्चाद‚भावोपि चेत् स एव} प्राग‚भावः प‚श्चाद‚भावोऽ‚{\color{DodgerBlue3}‚नित्य‚ता कि-} न्नेष्य‚ते । स‚त्ताविशेष‚ण‚त्वेनापि त‚द‚भ्युप‚ग‚म‚स्येष्ट‚त्वात् ॥ प‚ट‚स्य घ‚टे चा\edtext{}{\edlabel{pvv.150-5}\label{pvv.150-5}\lemma{चा}\Bfootnote{भावाभाव‚योर्व्विरोधान्न स‚म्ब‚न्ध‚ष‚ष्ट्यादि ॥}}भाव इति ‚{\tiny $_{lb}$}‚\leavevmode\ledsidenote{\textenglish{29a/MA}} ‚{\color{DodgerBlue3}‚ष‚ष्ट्या}‚दिविभ‚क्त्य‚यो‚{\tiny $_{7}$}‚गात् प्र‚ध्वंसा\edtext{}{\edlabel{pvv.150-6}\label{pvv.150-6}\lemma{ध्वंसा}\Bfootnote{कार्य‚स्य स‚त्त‚या स‚हाशेषः ।}}भावो नानित्य‚ता । न हि भावाभाव‚योः संयो‚{\tiny $_{lb}$}‚\leavevmode\ledsidenote{\textenglish{151/s}} गादिः क‚श्चित्स‚म्ब‚न्धोस्ति य उच्येत ष‚ष्ट्यादिभिरिति ‚{\color{DodgerBlue3}‚चेत् एव‚न्त‚र्ह्य‚न्त‚योः} स ‚{\tiny $_{lb}$}‚ष‚ष्ट्यादियोगः ‚{\color{DodgerBlue3}‚क‚थं भ‚वेत्} । अभाव‚योर्व्य‚व‚धिभूता स‚त्तेत्यादि । (११२)
	\pend% ending standard par
      \label{div_pvv.2.113}
	  
	% new div opening: depth here is 2
	

	  \pstart \leavevmode% starting standard par
	किञ्च (।)
	\pend% ending standard par
      
	  \bigskip
	  \begingroup
	
	    \large
	  
	    \begin{quote}
	  
	    
	    \stanza[\smallbreak]
	\label{pv.2.113a}\flagstanza{\tiny\textenglish{...2.113a}}स‚त्तास‚म्ब‚न्ध‚योर्ध्रौव्याद‚न्ताभ्यां न विशेष‚ण‚म् ।\&[\smallbreak]


	
	    \end{quote}
	  
	  \endgroup
	

	  \pstart \leavevmode% starting standard par
	\hphantom{.}‚{\color{DodgerBlue3}‚स‚त्ता त‚त्स‚म्ब‚न्धे} प‚र‚म‚ते ‚{\color{DodgerBlue3}‚ध्रौव्याद‚न्ताभ्यां} प्राक्प्र‚ध्वंसाभावाभ्यां ‚{\color{DodgerBlue3}‚न विशेष‚णं} स्याद‚न्त‚द्व‚य‚विशिष्टा स‚त्ता त‚त्स‚म्ब‚न्धो वेति (।) न हि निंत्य‚स्य स‚र्व‚काल‚व्यापिनो‚{\tiny $_{lb}$}‚ऽन्त‚स‚म्भ‚वः ।\edtext{\textsuperscript{*}}{\edlabel{pvv.151-1}\label{pvv.151-1}\lemma{*}\Bfootnote{व‚क्त‚व्यः ।}}\edtext{\textsuperscript{*}}{\edlabel{pvv.151-1a}\label{pvv.151-1a}\lemma{*}\Bfootnote{1a भाव‚स्य प‚रं ।\begin{english} --- Placement of note uncertain; marked with a question mark in the edition (see encoding description for details).\end{english}}}
	\pend% ending standard par
      
	  \bigskip
	  \begingroup
	
	    \large
	  
	    \begin{quote}
	  
	    
	    \stanza[\smallbreak]
	\label{pv.2.113b}\flagstanza{\tiny\textenglish{...2.113b}}अविशेष‚ण‚मेव स्याद‚न्तौ चेत् कार्य‚कार‚णे ॥ ११३ ॥\&[\smallbreak]


	
	    \end{quote}
	  
	  \endgroup
	

	  \pstart \leavevmode% starting standard par
	\hphantom{.}‚{\color{DodgerBlue3}‚कार्य‚कार‚णे अन्ता}‚व‚भिम‚ते इति ‚{\color{DodgerBlue3}‚चेत्‚{\tiny $_{1}$}‚} त‚दा ताभ्यां स‚त्तास‚{\color{DodgerBlue3}‚म्ब‚न्ध‚योर‚विशेष‚ण‚{\tiny $_{lb}$}‚मेव स्यात्} । (११३)
	\pend% ending standard par
      \label{div_pvv.2.114}
	  
	% new div opening: depth here is 2
	
	  \bigskip
	  \begingroup
	
	    \large
	  
	    \begin{quote}
	  
	    
	    \stanza[\smallbreak]
	\label{pv.2.114}\flagstanza{\tiny\textenglish{....2.114}}अस‚म्ब‚न्धान्न भाव‚स्य प्राग‚भावं स वांछ‚ति ॥&त‚दुपाधिस‚माख्याने तेप्य‚स्य च न सिध्य‚तः ॥ ११४ ॥\&[\smallbreak]


	
	    \end{quote}
	  
	  \endgroup
	

	  \pstart \leavevmode% starting standard par
	\hphantom{.}य‚स्मात् स नै या यि का दिर्भाव‚स्या‚{\color{DodgerBlue3}‚स‚म्ब‚न्धात् प्राग}‚भावं स‚म्ब‚न्धिनं ‚{\color{DodgerBlue3}‚न वाञ्छ‚ति} । ‚{\tiny $_{lb}$}‚त‚था च ‚{\color{DodgerBlue3}‚त‚दुपाधिस‚माख्याने} प्राग‚भाव‚विशेष‚णं कार्य‚कार‚ण‚मिति स‚माख्यानं व्य‚प‚देशो ‚{\tiny $_{lb}$}‚य‚योस्ते त‚था ते कार्य‚कार‚णे‚{\color{DodgerBlue3}‚प्य‚स्य न सिध्य‚तः} । य‚स्य हि प्राग‚भावः स भ‚व‚न् कार्यं ‚{\tiny $_{lb}$}‚स्यात् । न च त‚त्स‚म्ब‚न्धो भाव‚स्येति कार्याभावः । कार्याभावाच्च कार‚णाभावः । ‚{\tiny $_{lb}$}‚त‚दुत्पाद‚क‚स्य‚{\tiny $_{2}$}‚ कार‚ण‚त्वात् । (११४)
	\pend% ending standard par
      \label{div_pvv.2.115}
	  
	% new div opening: depth here is 2
	

	  \pstart \leavevmode% starting standard par
	किञ्च (।)
	\pend% ending standard par
      
	  \bigskip
	  \begingroup
	
	    \large
	  
	    \begin{quote}
	  
	    
	    \stanza[\smallbreak]
	\label{pv.2.115}\flagstanza{\tiny\textenglish{....2.115}}स‚त्ता स्व‚कार‚ण‚श्लेष‚क‚र‚णात् कार‚णं किल ।&सा स‚त्ता स च स‚म्ब‚न्धो नित्यौ कार्य‚म‚थेह किम् ॥ ११५ ॥\&[\smallbreak]


	
	    \end{quote}
	  
	  \endgroup
	

	  \pstart \leavevmode% starting standard par
	\hphantom{.}ज‚न्म स‚त्ताश्लेषः ‚{\color{DodgerBlue3}‚स‚त्ता}‚स‚म‚वायः स्व‚कार‚णेन स‚म‚वायिना स‚ह नित्यः । त‚योः ‚{\tiny $_{lb}$}‚‚{\color{DodgerBlue3}‚क‚र‚णात् किल} त्व‚येष्टं ‚{\color{DodgerBlue3}‚कार‚णं । सा च स‚त्ता स च स‚म्ब‚न्धो नित्यौ} द्वाव‚पि ‚{\tiny $_{lb}$}‚‚{\color{DodgerBlue3}‚कार्य‚म\edtext{}{\edlabel{pvv.151-2}\label{pvv.151-2}\lemma{म}\Bfootnote{अतो य‚स्माद् भावः प्राग‚भाव‚र‚हित‚स्त‚तो न क्रिय‚ते ।}}थेह} द्व‚योर्म‚ध्ये ‚{\color{DodgerBlue3}‚किं} युक्तं । (११५)
	\pend% ending standard par
      \label{div_pvv.2.116}
	  
	% new div opening: depth here is 2
	

	  \pstart \leavevmode% starting standard par
	अपि च(।)
	\pend% ending standard par
      \textsuperscript{\textenglish{152/s}}
	  \bigskip
	  \begingroup
	
	    \large
	  
	    \begin{quote}
	  
	    
	    \stanza[\smallbreak]
	\label{pv.2.116}\flagstanza{\tiny\textenglish{....2.116}}य‚स्याभावः क्रियेतासौ न भावः प्राग‚भाव‚वान् ।&स‚म्ब‚न्धान‚भ्युप‚ग‚मान्नित्यं विश्व‚मिदं त‚तः ॥ ११६ ॥\&[\smallbreak]


	
	    \end{quote}
	  
	  \endgroup
	

	  \pstart \leavevmode% starting standard par
	\hphantom{.}‚{\color{DodgerBlue3}‚य‚स्याभाव\edtext{}{\edlabel{pvv.152-1}\label{pvv.152-1}\lemma{स्याभाव}\Bfootnote{प्र‚ध्वंसाभाव‚स्य मृत‚क‚र‚ण‚प्र‚स‚ङ्गः ।}}स्यो}‚त्प‚त्तेः प्राग‚भावः कार‚णैर‚सौ ‚{\color{DodgerBlue3}‚क्रिय‚ते । न\edtext{}{\edlabel{pvv.152-2}\label{pvv.152-2}\lemma{न}\Bfootnote{प्राग‚भाव‚श्चेत‚न्न अस‚म्ब‚न्धात् ।}}} च क‚श्चिद् ‚{\color{DodgerBlue3}‚भावः ‚{\tiny $_{lb}$}‚प्राग्भाव‚वान् स‚म्ब‚न्ध}‚वान‚न‚{\color{DodgerBlue3}‚भ्युप‚ग‚मात् । त‚तः} कार्याभावा‚{\color{DodgerBlue3}‚न्नित्यं विश्व‚मिदं} प्राप्तं । ‚{\tiny $_{lb}$}‚(११६)
	\pend% ending standard par
      \label{div_pvv.2.117}
	  
	% new div opening: depth here is 2
	

	  \pstart \leavevmode% starting standard par
	क‚थ‚न्त‚र्ह्य‚भावेन स‚म्ब‚न्धः । न क‚थ‚ञ्चित् । किन्तु बुद्धिप‚रिक‚ल्पि‚{\tiny $_{3}$}‚त एवा‚{\tiny $_{lb}$}‚सावित्याह (।)
	\pend% ending standard par
      
	  \bigskip
	  \begingroup
	
	    \large
	  
	    \begin{quote}
	  
	    
	    \stanza[\smallbreak]
	\label{pv.2.117}\flagstanza{\tiny\textenglish{....2.117}}त‚स्माद‚न‚र्थास्क‚न्दिन्योऽभिन्नार्थाभिम‚तेष्व‚पि ।&श‚ब्देषु वाच्य‚भेदिन्यो व्य‚तिरेकास्प‚दं धियः ॥ ११७ ॥\&[\smallbreak]


	
	    \end{quote}
	  
	  \endgroup
	

	  \pstart \leavevmode% starting standard par
	\hphantom{.}य‚स्माद्वास्त‚व‚स‚म्ब‚न्धाभ्युप‚ग‚मे दोष‚स्त‚स्माद् भाव‚स्य प्राग‚भाव इत्य‚{\color{DodgerBlue3}‚भिन्नार्थ}‚त्वे‚{\tiny $_{lb}$}‚नाभिम‚तेष्व‚पि श‚ब्दाद् भिन्नार्थाभिम‚तेषु च बीज‚स्याङ्कुर इत्यादिषु ‚{\color{DodgerBlue3}‚श‚ब्देषु धियो‚{\tiny $_{lb}$}‚ऽन‚र्थास्क‚न्दिन्यः} क‚ल्पित‚स‚म्ब‚न्ध‚विष‚या ‚{\color{DodgerBlue3}‚वाच्य‚भेदिन्यः} । संकेतानुरोधादुप‚क‚ल्पित‚{\tiny $_{lb}$}‚स‚म्ब‚न्धिस‚म्ब‚न्ध‚ल‚क्ष‚ण‚वाच्य‚भेद‚व‚त्यो ‚{\color{DodgerBlue3}‚व्य‚ति}‚रेक‚स्य स‚म्ब‚न्धिस‚म्ब‚न्ध‚स्या‚{\color{DodgerBlue3}‚स्प‚दं} निमित्तं\edtext{}{\edlabel{pvv.152-3}\label{pvv.152-3}\lemma{निमित्तं}\Bfootnote{धिय एव ।}} ‚{\tiny $_{lb}$}‚भ‚व‚न्ति । अलं वास्त‚व‚स‚म्ब‚{\tiny $_{4}$}‚न्धानुब‚न्धेन दोषाश्र‚येण । (११७)
	\pend% ending standard par
      \label{div_pvv.2.118}
	  
	% new div opening: depth here is 2
	

	  \pstart \leavevmode% starting standard par
	त‚देवं मेय‚ब‚हुत्वाद् (व) हुतापि चेत्य‚त्र सानुष‚ङ्गं प्र‚तिविहितं । अनेक‚स्य ‚{\tiny $_{lb}$}‚वृत्तेरेक‚त्र वा य‚था विशेष‚दृष्टेनेत्य‚त्राह (।)
	\pend% ending standard par
      
	  \bigskip
	  \begingroup
	
	    \large
	  
	    \begin{quote}
	  
	    
	    \stanza[\smallbreak]
	\label{pv.2.118a}\flagstanza{\tiny\textenglish{...2.118a}}विशेष‚प्र‚त्य‚भिज्ञानं न प्र‚तिक्ष‚ण‚भेद‚तः ।\&[\smallbreak]


	
	    \end{quote}
	  
	  \endgroup
	

	  \pstart \leavevmode% starting standard par
	\hphantom{.}अग्निं दृष्ट्वा क्र‚मेण धूमाल्लिङ्गात् त‚स्यैव ‚{\color{DodgerBlue3}‚विशेष}‚स्य स एवायं व‚ह्निरिति ‚{\tiny $_{lb}$}‚‚{\color{DodgerBlue3}‚य‚त्प्र‚त्य‚भिज्ञानं न} त‚त्प्र‚मा\edtext{}{\edlabel{pvv.152-4}\label{pvv.152-4}\lemma{मा}\Bfootnote{विशिष्ट‚त्वाभावात् पूर्व्व‚नाशाप‚रोत्प‚त्तेः स एवाय‚मिति मिथ्याज्ञानं । ‚{\tiny $_{lb}$}‚अव‚स्था ।}}णं । ‚{\color{DodgerBlue3}‚प्र‚तिक्ष‚णं} भाव‚स्य ‚{\color{DodgerBlue3}‚भेद‚तः} एकार्थ‚साध्यार्थ‚क्रियायाः ‚{\tiny $_{lb}$}‚स‚म्वादाभावात् ।
	\pend% ending standard par
      

	  \pstart \leavevmode% starting standard par
	स्यादेत‚न्न य‚था दृष्ट एव विशेषो गृह्य‚ते किन्तु त‚त्सामान्य‚मित्याह (।)
	\pend% ending standard par
      
	  \bigskip
	  \begingroup
	
	    \large
	  
	    \begin{quote}
	  
	    
	    \stanza[\smallbreak]
	\label{pv.2.118b}\flagstanza{\tiny\textenglish{...2.118b}}न वा विशेष‚विष‚यं दृष्ट‚साम्येन त‚द्ग्र‚हात् ॥ ११८ ॥\&[\smallbreak]


	
	    \end{quote}
	  
	  \endgroup
	

	  \pstart \leavevmode% starting standard par
	\hphantom{.}‚{\color{DodgerBlue3}‚न वा विशेष‚विष‚यं} त‚द्विशेष‚{\color{DodgerBlue3}‚दृष्ट}‚म‚नुमानं व‚क्त‚व्यं प्राग्दृष्ट‚स्य विशेष‚स्य ‚{\color{DodgerBlue3}‚साम्ये}‚{\tiny $_{lb}$}‚न त‚स्योत्त‚र‚स्य ‚{\color{DodgerBlue3}‚ग्र‚ह‚णात्} । (११८)
	\pend% ending standard par
      \label{div_pvv.2.119}
	  
	% new div opening: depth here is 2
	\textsuperscript{\textenglish{153/s}}

	  \pstart \leavevmode% starting standard par
	स्यादेत‚द‚त्र (।) य‚त्र दृष्टान्त‚दार्ष्टान्तिक‚योर्भेद‚स्त‚त्र सामान्य‚तो दृष्ट‚म‚नुमानं ‚{\tiny $_{lb}$}‚इह तु ।
	\pend% ending standard par
      
	  \bigskip
	  \begingroup
	
	    \large
	  
	    \begin{quote}
	  
	    
	    \stanza[\smallbreak]
	\label{pv.2.119}\flagstanza{\tiny\textenglish{....2.119}}निद‚र्श‚नं त‚देवेति सामान्याग्र‚ह‚णं य‚दि ।&निद‚र्श‚न‚त्वात् सिद्ध‚स्य प्र‚माणेनास्य किं पुनः ॥ ११९ ॥\&[\smallbreak]


	
	    \end{quote}
	  
	  \endgroup
	

	  \pstart \leavevmode% starting standard par
	\hphantom{.}‚{\color{DodgerBlue3}‚त‚देव} दार्ष्टान्तिकं ‚{\color{DodgerBlue3}‚निद‚र्श‚न‚मिति सामान्याग्र‚ह‚णं} य‚द्युच्य‚ते त‚दा ‚{\color{DodgerBlue3}‚निद‚र्श‚न‚त्वा‚{\tiny $_{lb}$}‚त्सिद्ध‚स्य} निश्चित‚स्या‚{\color{DodgerBlue3}‚स्य} दार्ष्टान्तिक‚स्य ‚{\color{DodgerBlue3}‚प्र‚माणेन किं} क‚र्त‚व्यं । (११९)
	\pend% ending standard par
      \label{div_pvv.2.120_2.121}
	  
	% new div opening: depth here is 2
	
	  \bigskip
	  \begingroup
	
	    \large
	  
	    \begin{quote}
	  
	    
	    \stanza[\smallbreak]
	\label{pv.2.120a}\flagstanza{\tiny\textenglish{...2.120a}}विस्मृत‚त्वाद‚दोष‚श्चेत् त‚त एवानिद‚र्श‚न‚म् ॥&दृष्टे त‚द्भाव‚सिद्धिश्चेत् प्र‚माणाद्;\&[\smallbreak]


	
	    \end{quote}
	  
	  \endgroup
	

	  \pstart \leavevmode% starting standard par
	नाप्र‚सिद्धो दृष्टान्तः स चेत् सिद्धः किम‚नुमानेन गृहीत‚स्यापि‚{\tiny $_{6}$}‚ ‚{\color{DodgerBlue3}‚विस्मृत‚त्वात्} । ‚{\tiny $_{lb}$}‚पुन‚र‚नुमान‚प्र‚तीताव‚{\color{DodgerBlue3}‚दोष‚श्चेत् । त‚तो} विस्मृत‚त्वा‚{\color{DodgerBlue3}‚देवानिद‚र्श‚नं} । न हि गृहीत‚विस्मृ‚{\tiny $_{lb}$}‚त‚स्य दृष्टान्त‚ता । पूर्व्व‚प्र‚त्य‚येन ‚{\color{DodgerBlue3}‚दृष्टे}‚ऽर्थ ‚{\color{DodgerBlue3}‚प्र‚माणाद्वि}‚शेष‚दृष्टानुमानाद्य एव प्राग्दृष्टः ‚{\tiny $_{lb}$}‚स एवाय‚मिति ‚{\color{DodgerBlue3}‚त‚द्भाव}‚स्य पूर्व्व‚स्य ‚{\color{DodgerBlue3}‚सिद्धिश्चेदि}‚ष्य‚ते (।)
	\pend% ending standard par
      

	  \pstart \leavevmode% starting standard par
	न‚न्व‚यं त‚द्भावः किम‚न्य‚व‚स्तुनि साध्य‚ते उत त‚त्रैव ।
	\pend% ending standard par
      
	  \bigskip
	  \begingroup
	
	    \large
	  
	    \begin{quote}
	  
	    
	    \stanza[\smallbreak]
	\label{pv.2.120b}\flagstanza{\tiny\textenglish{...2.120b}}अन्य‚व‚स्तुनि ॥ १२० ॥\&[\smallbreak]


	
	    \end{quote}
	  
	  \endgroup
	
	  \bigskip
	  \begingroup
	
	    \large
	  
	    \begin{quote}
	  
	    
	    \stanza[\smallbreak]
	\label{pv.2.121}\flagstanza{\tiny\textenglish{....2.121}}त‚त्त्वारोपे विप‚र्यास‚स्त‚त्सिद्ध‚रेप्र‚माण‚ता ।&प्र‚त्य‚क्षेत‚र‚योरैक्यादेक‚सिद्धिर्द्व‚योर‚पि ॥ १२१ ॥\&[\smallbreak]


	
	    \end{quote}
	  
	  \endgroup
	

	  \pstart \leavevmode% starting standard par
	\hphantom{.}त‚त्रान्य‚व‚स्तुनि (१२०) व‚र्त‚माने ‚{\color{DodgerBlue3}‚त‚त्त्व}‚स्यातीत‚व‚स्त्वात्म‚क‚{\color{DodgerBlue3}‚स्यारोपे} स्वीक्रिय‚{\tiny $_{lb}$}‚माणे ‚{\color{DodgerBlue3}‚विप‚र्यासो‚{\tiny $_{7}$}‚}‚ऽय‚थार्थ‚त्वं स्यात् । न ह्य‚न्य‚स्यान्यात्म‚त्व‚म‚स्ति त‚द‚श‚क्य‚प्राप‚ण\leavevmode\ledsidenote{\textenglish{29b/MA}}‚{\tiny $_{lb}$}‚मुप‚द‚र्श‚य‚द‚प्र‚माणं स्यात् । अथ त‚त्रैव त‚द्भाव‚सिद्धिरिति द्वितीयः प‚क्षः । त‚दा ‚{\tiny $_{lb}$}‚‚{\color{DodgerBlue3}‚त‚त्सिद्धे}‚रेक‚सिद्धेर्व्विशेष‚दृष्ट‚स्यानुमान‚स्या‚{\color{DodgerBlue3}‚प्र‚माण‚ता} गृहीत‚ग्राहित्वात् । न ह्येक‚स्य ‚{\tiny $_{lb}$}‚निर्भाग‚स्य किञ्चिद‚गृहीतं नाम । त‚था हि ‚{\color{DodgerBlue3}‚प्र‚त्य‚क्षेत‚र‚यो}‚र‚ध्य‚क्षानुमान‚विष‚य‚यौरै‚{\tiny $_{lb}$}‚‚{\color{DodgerBlue3}‚क्यात् द्व‚योर‚पि} प्र‚त्य‚क्षेऽनुमाने च ‚{\color{DodgerBlue3}‚एक‚स्यार्थ‚स्य सिद्धिः} । (१२१)
	\pend% ending standard par
      \label{div_pvv.2.122}
	  
	% new div opening: depth here is 2
	

	  \pstart \leavevmode% starting standard par
	त‚त‚श्च दृष्टान्त‚ग्राहिणैवा‚{\tiny $_{1}$}‚भ्र‚ष्ट‚स्मृतिसंस्कारेण प्र‚त्य‚क्षेण सिद्ध‚त्वात् विफ‚ल‚{\tiny $_{lb}$}‚म‚नुमानं । प्र‚त्य‚क्ष‚संस्कार‚भ्रंशे तु नादृष्टान्त‚म‚नुमान‚म‚स्ति (।) त‚स्माद् (।)
	\pend% ending standard par
      
	  \bigskip
	  \begingroup
	
	    \large
	  
	    \begin{quote}
	  
	    
	    \stanza[\smallbreak]
	\label{pv.2.122}\flagstanza{\tiny\textenglish{....2.122}}स‚न्धीय‚मानं चान्येन व्य‚व‚सायं स्मृतिं विदुः ।&त‚ल्लिङ्गापेक्ष‚णान्नो चेत् स्मृतिर्न व्य‚भिचार‚तः ॥ १२२ ॥\&[\smallbreak]


	
	    \end{quote}
	  
	  \endgroup
	

	  \pstart \leavevmode% starting standard par
	\hphantom{.}अन्येनातीत‚द‚र्श‚नेन एक‚विष‚य‚त‚या ‚{\color{DodgerBlue3}‚स‚न्धीय\edtext{}{\edlabel{pvv.153-1}\label{pvv.153-1}\lemma{न्धीय}\Bfootnote{अभ्र‚ष्ट‚द‚र्श‚न‚संस्कारं ।}} मानं} घ‚ट्य‚मानं प‚रं ‚{\color{DodgerBlue3}‚व्य‚व‚सायं}\edtext{}{\edlabel{pvv.153-2}\label{pvv.153-2}\lemma{रं}\Bfootnote{निश्च‚यं ।}} ‚{\tiny $_{lb}$}‚\leavevmode\ledsidenote{\textenglish{154/s}} ‚{\color{DodgerBlue3}‚स्मृतिं विदु}‚र्व्विद्वांसः । गृहीतार्थ‚विक‚ल्पेन स्मृतित्वं त‚च्चेहाविक‚लं । त‚स्य प्र‚तिप‚त्त‚व्य‚स्य ‚{\tiny $_{lb}$}‚चिह्न‚म‚व्य‚भिचारि ‚{\color{DodgerBlue3}‚लिङ्गं}\edtext{}{\edlabel{pvv.154-1}\label{pvv.154-1}\lemma{भिचारि}\Bfootnote{स्मृतेर्धूमाद्य‚पेक्षा नान्त्य‚स्यास्ति ।}} न्त‚{\color{DodgerBlue3}‚द‚पेक्ष‚णान्नोचे}‚द्विशेष‚दृष्ट‚म‚नुमानं ‚{\color{DodgerBlue3}‚स्मृतिः} सा तु लिङ्ग‚नि‚{\tiny $_{lb}$}‚र‚पेक्ष्या । नैत‚द्युक्तं‚{\tiny $_{2}$}‚ ‚{\color{DodgerBlue3}‚व्य‚भिचार\edtext{}{\edlabel{pvv.154-2}\label{pvv.154-2}\lemma{भिचार}\Bfootnote{न स्मृतिर‚निमित्तैव ।}} तः} ।\edtext{\textsuperscript{*}}{\edlabel{pvv.154-3}\label{pvv.154-3}\lemma{*}\Bfootnote{विशेष‚दृष्टे हि विशेषेणान्व‚याभावात् त्रैरूप्यं । त‚देवं लिङ्ग‚जं प्र‚त्य‚भिज्ञानं ‚{\tiny $_{lb}$}‚दूषितं प्र‚त्य‚क्ष‚जं दूष‚यिष्य‚ते ।}} त‚था हि य‚दि लिङ्ग‚न्त्रिरूप‚न्त‚दा व्याप्तिग्र‚ह‚{\tiny $_{lb}$}‚ण‚विष‚य‚त्वेनैव त‚त्सिद्धेर्व्य‚र्थ‚म‚नुमानं । अथ न त्रिरूप‚न्त‚दा नाव्य\edtext{}{\edlabel{pvv.154-4}\label{pvv.154-4}\lemma{नाव्य}\Bfootnote{देशादिक‚थ‚यापि स्यात् प्र‚त्य‚भिज्ञानं न त‚द‚व्य‚भिचारि ।}} भिचार‚निश्च‚यः । ‚{\tiny $_{lb}$}‚त‚स्माद्विशेष‚दृष्ट‚स्याप्र‚माण‚त्वादेक‚त्रानेक‚वृत्तेर‚पि त्र्येक‚संख्यापोह‚नाभावो निर‚स्तः । ‚{\tiny $_{lb}$}‚त‚स्मात्स्थित‚मेत‚न्मानं द्विविधं मेय‚द्वैविध्यादिति ॥ (१२२) ॥
	\pend% ending standard par
      
	  
	% new div opening: depth here is 1
	
\chapter*[{५. प्र‚त्य‚क्ष‚चिन्ता}]{५. प्र‚त्य‚क्ष‚चिन्ता}

	  \begin{center}%% label @type='head'
	\textbf{(१) प्र‚त्य‚क्ष‚ल‚क्ष‚ण‚विप्र‚तिप‚त्तिनिरासः}
	\end{center}
	

	  \begin{center}%% label @type='head'
	\textbf{क. क‚ल्प‚नापोढं प्र‚त्य‚क्ष‚म्}
	\end{center}
	\label{div_pvv.2.123}
	  
	% new div opening: depth here is 2
	

	  \pstart \leavevmode% starting standard par
	इदानीम‚व‚स‚र‚प्राप्तां प्र‚त्य‚क्ष‚स्य ल‚क्ष‚ण‚विप्र‚तिप‚त्तिं निराक‚र्त्तुमाह ।
	\pend% ending standard par
      
	  \bigskip
	  \begingroup
	
	    \large
	  
	    \begin{quote}
	  
	    
	    \stanza[\smallbreak]
	\label{pv.2.123}\flagstanza{\tiny\textenglish{....2.123}}प्र‚त्य‚क्षं क‚ल्प‚नापोढं प्र‚त्य‚क्षेणैव सिध्य‚ति ।&प्र‚त्यात्म‚वेद्यः स‚र्वेषां विक‚ल्पो नाम‚संश्र‚यः ॥ १२३ ॥\&[\smallbreak]


	
	    \end{quote}
	  
	  \endgroup
	

	  \pstart \leavevmode% starting standard par
	\hphantom{.}य‚त्त‚{\color{DodgerBlue3}‚त्प्र‚त्य‚क्ष}‚मिति प्र‚सिद्धं‚{\tiny $_{3}$}‚त‚त् ‚{\color{DodgerBlue3}‚क‚ल्प‚नाया अपोढं} द्र‚ष्ट‚व्यं क‚ल्प‚नार्थ‚र‚हित‚मित्य‚र्थः । ‚{\tiny $_{lb}$}‚त‚च्चैत‚दीदृशं ‚{\color{DodgerBlue3}‚प्र‚त्य‚क्षेणैव} स्व‚स‚म्वेद‚नेनैव ‚{\color{DodgerBlue3}‚सिध्य‚ति} । क‚ल्प‚नार‚हित‚स्यार्थ‚स्य रूप‚स्य ‚{\tiny $_{lb}$}‚स‚म्वेद‚न‚स्याप‚रोक्ष‚त्वात् । य‚दि तु क‚ल्प‚नास्व‚भाव‚त्व‚म‚स्य स्यात्त‚थैव प्र‚काशेत । ‚{\tiny $_{lb}$}‚विक‚ल्प‚स्याप‚रोक्ष‚त्वात् । त‚था हि ‚{\color{DodgerBlue3}‚प्र‚त्यात्म‚वेद्यः स‚र्व्वेषां} प्राणिनां ‚{\color{DodgerBlue3}‚विक‚ल्पो नाम‚{\tiny $_{lb}$}‚संश्र‚यः} श‚ब्द‚संस‚र्ग‚वान् । स य‚दि स्यादुप‚ल‚भ्य एव भ‚वेत्‚{\tiny $_{4}$}‚ (। १२३)
	\pend% ending standard par
      \label{div_pvv.2.124}
	  
	% new div opening: depth here is 2
	

	  \pstart \leavevmode% starting standard par
	त‚स्मात् (।)
	\pend% ending standard par
      
	  \bigskip
	  \begingroup
	
	    \large
	  
	    \begin{quote}
	  
	    
	    \stanza[\smallbreak]
	\label{pv.2.124}\flagstanza{\tiny\textenglish{....2.124}}संहृत्य स‚र्व‚त‚श्चिन्तां स्तिमितेनान्त‚रात्म‚ना ।&स्थितोपि च‚क्षुषा रूप‚मीक्ष‚ते साक्ष‚जा म‚तिः ॥ १२४ ॥\&[\smallbreak]


	
	    \end{quote}
	  
	  \endgroup
	

	  \pstart \leavevmode% starting standard par
	\hphantom{.}‚{\color{DodgerBlue3}‚संहृत्या}‚कृष्य ‚{\color{DodgerBlue3}‚स‚र्व्व‚तो} विक‚ल्प‚नीया‚{\color{DodgerBlue3}‚च्चिन्तां स्तिमितेन} स‚र्व्वाविक‚ल्प‚{\tiny $_{lb}$}‚विग‚मात् अधिक्षिप्तेना‚{\color{DodgerBlue3}‚न्त‚रात्म‚ना} चेत‚सा ‚{\color{DodgerBlue3}‚स्थितोऽपि} पुरुष‚{\color{DodgerBlue3}‚श्च‚क्षु}‚र्व्विज्ञानेन ‚{\color{DodgerBlue3}‚रूप‚{\tiny $_{lb}$}‚मीक्ष‚ते साक्ष}‚जा निर्व्विक‚ल्पा ‚{\color{DodgerBlue3}‚म‚तिः} स‚र्व स‚म्विदितैव । (१२४)
	\pend% ending standard par
      \label{div_pvv.2.125}
	  
	% new div opening: depth here is 2
	\textsuperscript{\textenglish{155/s}}

	  \pstart \leavevmode% starting standard par
	स‚न्त्येवेन्द्रिय‚धियः क‚ल्प‚नास्तास्तु नोप‚ल‚भ्य‚न्त इत्य‚प्य‚स‚त् । त‚था हि ।
	\pend% ending standard par
      
	  \bigskip
	  \begingroup
	
	    \large
	  
	    \begin{quote}
	  
	    
	    \stanza[\smallbreak]
	\label{pv.2.125}\flagstanza{\tiny\textenglish{....2.125}}पुन‚र्विक‚ल्प‚य‚न् किञ्चिदासीन्मे क‚ल्प‚नेदृशी ।&वेत्ति चेति न पूर्व्वोक्ताव‚स्थायामिन्द्रियाद् ग‚तौ ॥ १२५ ॥\&[\smallbreak]


	
	    \end{quote}
	  
	  \endgroup
	

	  \pstart \leavevmode% starting standard par
	\hphantom{.}विक‚ल्पाव‚स्थाया ऊर्ध्वं ‚{\color{DodgerBlue3}‚पुन‚र्व्विक‚ल्प‚य‚न्} पुमाना‚{\color{DodgerBlue3}‚सीन्मे क‚ल्प‚नेदृशीति} वेति ‚{\color{DodgerBlue3}‚नेन्द्रि‚{\tiny $_{lb}$}‚यादु‚{\tiny $_{5}$}‚}‚त्प‚न्नायां ‚{\color{DodgerBlue3}‚ग‚तौ} बुद्धौ संहृत्येत्यादिना ‚{\color{DodgerBlue3}‚पूर्व्व‚मुक्ताव‚स्था} य‚स्यास्त‚स्याः क‚ल्प‚नां वेति । ‚{\tiny $_{lb}$}‚य‚दि सा त‚त्र स्याद‚त‚त्संस्कार‚स्य स्मृतिर्जाय‚ते । त‚स्मान्नास्तीति निश्चीय‚ते (। १२५)
	\pend% ending standard par
      \label{div_pvv.2.126}
	  
	% new div opening: depth here is 2
	

	  \pstart \leavevmode% starting standard par
	किञ्च (।) वाच्य‚वाच‚काकार‚संस‚र्ग‚व‚ती प्र‚तीतिः क‚ल्प‚ना । न चेन्द्रिय‚विष‚ये‚{\tiny $_{lb}$}‚ऽन‚न्व‚यात् संकेताऽस‚म्भ‚वाच्च श‚ब्द‚योज‚नास्ति (।) त‚थाहि (।)
	\pend% ending standard par
      
	  \bigskip
	  \begingroup
	
	    \large
	  
	    \begin{quote}
	  
	    
	    \stanza[\smallbreak]
	\label{pv.2.126}\flagstanza{\tiny\textenglish{....2.126}}एक‚त्र दृष्टो भेदो हि क्व‚चिन्नान्य‚त्र दृश्य‚ते ॥&न त‚स्माद् भिन्न‚म‚स्त्य‚न्य‚त् सामान्यं बुद्ध‚य‚भेद‚तः॥ १२६ ॥\&[\smallbreak]


	
	    \end{quote}
	  
	  \endgroup
	

	  \pstart \leavevmode% starting standard par
	\hphantom{.}‚{\color{DodgerBlue3}‚एक‚त्र} दे\edtext{}{\edlabel{pvv.155-1}\label{pvv.155-1}\lemma{दे}\Bfootnote{श‚ब्दादेक‚त्र नियुक्तात्स‚र्व्व‚त्रार्थ‚प्र‚तीतिः स्यात् ।}}शादौ ‚{\color{DodgerBlue3}‚न दृश्य‚ते} न चा‚{\tiny $_{6}$}‚न‚नुयायिनि श‚ब्द‚संकेतः । सामान्य‚म‚नुयायीति ‚{\tiny $_{lb}$}‚चेत् । ‚{\color{DodgerBlue3}‚त‚स्मा}‚द् भेदाद‚न्य‚द् ‚{\color{DodgerBlue3}‚भिन्नं सामान्यं नास्ति बुद्धेर‚भेद‚तः} । (१२६)
	\pend% ending standard par
      \label{div_pvv.2.127}
	  
	% new div opening: depth here is 2
	

	  \pstart \leavevmode% starting standard par
	य‚दि हि सामान्यं सामान्यं स्यात् द्व्याकारा बुद्धिर्भ‚वेत् । विशेष‚मात्राकारैव ‚{\tiny $_{lb}$}‚तु प्र‚त्य‚क्ष‚बुद्धिरुप‚ल‚भ्य‚ते ।
	\pend% ending standard par
      
	  \bigskip
	  \begingroup
	
	    \large
	  
	    \begin{quote}
	  
	    
	    \stanza[\smallbreak]
	\label{pv.2.127}\flagstanza{\tiny\textenglish{....2.127}}त‚स्माद् विशेष‚विष‚या स‚र्वैवेन्द्रिय‚जा म‚तिः ॥&न विशेषेषु श‚ब्दानां प्र‚वृत्ताव‚स्ति स‚म्भ‚वः ॥ १२७ ॥\&[\smallbreak]


	
	    \end{quote}
	  
	  \endgroup
	

	  \pstart \leavevmode% starting standard par
	\hphantom{.}‚{\color{DodgerBlue3}‚त‚स्मात्स‚र्व्वैवेन्द्रिय‚जा म‚तिर्व्विशेष}‚मात्र‚{\color{DodgerBlue3}‚विष‚या}‚ऽन्य‚स्यानुप‚ल‚ब्धेः । ‚{\color{DodgerBlue3}‚न च विशेषेषु ‚{\tiny $_{lb}$}‚श‚ब्दानां प्र‚वृत्तौ संभ‚वोस्ति} (१२७)
	\pend% ending standard par
      \label{div_pvv.2.128}
	  
	% new div opening: depth here is 2
	
	  \bigskip
	  \begingroup
	
	    \large
	  
	    \begin{quote}
	  
	    
	    \stanza[\smallbreak]
	\label{pv.2.128}\flagstanza{\tiny\textenglish{....2.128}}अन‚न्व‚याद् विशेषाणां स‚ङ्केत‚स्याप्र‚वृत्तितः ॥&विष‚यो य‚श्च श‚ब्दानां संयोज्येत स एव तैः ॥ १२८ ॥\&[\smallbreak]


	
	    \end{quote}
	  
	  \endgroup
	

	  \pstart \leavevmode% starting standard par
	\hphantom{.}‚{\color{DodgerBlue3}‚विशेषाणाम‚न‚न्व‚यात्} त‚त्र ‚{\color{DodgerBlue3}‚संकेत‚स्या‚{\tiny $_{7}$}‚प्र‚वृत्तितः} । उत्त‚र‚कालं श‚ब्दार्थ‚प्र‚तिप‚त्त्य‚र्थं ‚{\tiny $_{lb}$}‚संकेत‚क्रिया । न च विशेषाः कालान्त‚र‚म‚नुव‚र्त‚न्ते । त‚स्माद्य एव ‚{\color{DodgerBlue3}‚श‚ब्दानां विष‚यो}\leavevmode\ledsidenote{\textenglish{30a/MA}} ‚{\tiny $_{lb}$}‚व्य‚व‚च्छेदः ‚{\color{DodgerBlue3}‚स एव तैः संयोज्येत} । न स्व‚ल‚क्ष‚णं ॥ (१२८)
	\pend% ending standard par
      \label{div_pvv.2.129}
	  
	% new div opening: depth here is 2
	
	  \bigskip
	  \begingroup
	
	    \large
	  
	    \begin{quote}
	  
	    
	    \stanza[\smallbreak]
	\label{pv.2.129}\flagstanza{\tiny\textenglish{....2.129}}अस्येद‚मिति स‚म्ब‚न्धे याव‚र्थौ प्र‚तिभासिनौ&त‚योरेव हि स‚म्ब‚न्धो न त‚देन्द्रिय‚गोच‚रः ॥ १२९ ॥\&[\smallbreak]


	
	    \end{quote}
	  
	  \endgroup
	

	  \pstart \leavevmode% starting standard par
	\hphantom{.}त‚स्माद‚स्या‚{\color{DodgerBlue3}‚र्थ‚स्येद}‚म्वाच‚क‚{\color{DodgerBlue3}‚मिति स‚म्ब‚न्धे} वाच्य‚वाच‚क‚भाव‚ल‚क्ष‚णे ‚{\color{DodgerBlue3}‚याव‚र्थौ प्र‚ति‚{\tiny $_{lb}$}‚भासिनौ त‚योरेव हि स‚म्ब‚न्धो} व‚क्त‚व्यः । य‚दा चार्थ‚न्दृष्ट्वा स‚केतं त‚त्र प्र‚व‚र्त‚य‚ति । ‚{\tiny $_{lb}$}‚‚{\color{DodgerBlue3}‚त‚देन्द्रिय‚गोच‚रोऽर्थो} नास्ति (। १२९)
	\pend% ending standard par
      \label{div_pvv.2.130}
	  
	% new div opening: depth here is 2
	\textsuperscript{\textenglish{156/s}}
	  \bigskip
	  \begingroup
	
	    \large
	  
	    \begin{quote}
	  
	    
	    \stanza[\smallbreak]
	\label{pv.2.130}\flagstanza{\tiny\textenglish{....2.130}}विश‚द‚प्र‚तिभास‚स्य त‚दार्थ‚स्याविभाव‚नात् ।&विज्ञानाभास‚भेद‚श्च प‚दार्थानां विशेष‚कः ॥ १३० ॥\&[\smallbreak]


	
	    \end{quote}
	  
	  \endgroup
	

	  \pstart \leavevmode% starting standard par
	\hphantom{.}संहृतेन्द्रिय‚व्यापार‚स्य त‚दा संकेत‚संक‚ल्प‚काले ‚{\color{DodgerBlue3}‚विष (श?)द‚प्र‚ति‚{\tiny $_{1}$}‚भास‚स्यार्थ‚स्या‚{\tiny $_{lb}$}‚विभाव‚नात्} । य‚दि त‚त्रार्थः प्र‚तिभाति त‚देन्द्रिय‚ज्ञान‚व‚त् स्फुटः प्र‚तीय‚ते । न च ‚{\tiny $_{lb}$}‚प्र‚तिभास‚भेदेपि श‚ब्देन्द्रिय‚ज्ञान‚योरेक‚विष‚य‚त्वं य‚स्मा‚{\color{DodgerBlue3}‚द्विज्ञान}‚स्या‚{\color{DodgerBlue3}‚भास‚भेद} आकार‚{\tiny $_{lb}$}‚भेदः ‚{\color{DodgerBlue3}‚प‚दार्थानां} ग्राह्यानां (? णां) ‚{\color{DodgerBlue3}‚विशेष‚को} भेद‚कः । य‚दि तु प्र‚तिभास‚भेदेप्य‚र्था‚{\tiny $_{lb}$}‚भेद‚स्त‚दा विश्व‚मेकं द्र‚व्यं स्यात् ॥(१३०)
	\pend% ending standard par
      \label{div_pvv.2.131}
	  
	% new div opening: depth here is 2
	

	  \pstart \leavevmode% starting standard par
	स्यादेत‚त् । य‚दा स्व‚ल‚क्ष‚ण‚मुप‚द‚र्श्य श‚ब्दो निवेश्य‚ते त‚दा स्व‚ल‚क्ष‚ण‚मेव वाच्य‚{\tiny $_{lb}$}‚वाच‚क‚मित्याह (।)
	\pend% ending standard par
      
	  \bigskip
	  \begingroup
	
	    \large
	  
	    \begin{quote}
	  
	    
	    \stanza[\smallbreak]
	\label{pv.2.131}\flagstanza{\tiny\textenglish{....2.131}}च‚क्षुषाऽर्थाव‚भासेऽपि यं प‚रोऽस्येति शंस‚ति ।&स एव योज्य‚ते श‚ब्दैर्न ख‚ल्विन्द्रिय‚गोच‚रः ॥ १३१ ॥\&[\smallbreak]


	
	    \end{quote}
	  
	  \endgroup
	

	  \pstart \leavevmode% starting standard par
	\hphantom{.}‚{\color{DodgerBlue3}‚च‚क्षुषार्थ‚व‚भासे‚{\tiny $_{2}$}‚पि} जाते त‚था श्रोत्राच्छाब्दाव‚भासेपि य‚म‚र्थ‚म‚न्य‚व्य‚व‚च्छेदं ‚{\tiny $_{lb}$}‚बुद्धिप‚रिव‚र्त्तिनं प‚रः प्र‚तिपाद‚कोऽस्यार्थ‚स्यायं वाच‚क इति ‚{\color{DodgerBlue3}‚शंस‚ति} क‚थ‚य‚ति ‚{\color{DodgerBlue3}‚स एवा}‚न्य‚{\tiny $_{lb}$}‚व्य‚व‚च्छेदः ‚{\color{DodgerBlue3}‚श‚ब्दैर्योज्य‚ते न ख‚ल्विन्द्रिय‚गोच‚रः} स्व‚ल‚क्ष‚ण‚म‚न‚न्व‚यात्त‚स्य । त‚द्द‚र्श‚नान्त‚रं ‚{\tiny $_{lb}$}‚संकेत‚संक‚ल्पे च विनाशाच्च । (१३१)
	\pend% ending standard par
      \label{div_pvv.2.132}
	  
	% new div opening: depth here is 2
	
	  \bigskip
	  \begingroup
	
	    \large
	  
	    \begin{quote}
	  
	    
	    \stanza[\smallbreak]
	\label{pv.2.132a}\flagstanza{\tiny\textenglish{...2.132a}}अव्यापृतेन्द्रिय‚स्यान्य‚वाङ्ग‚मात्रेणाविभाव‚नात् ।\&[\smallbreak]


	
	    \end{quote}
	  
	  \endgroup
	

	  \pstart \leavevmode% starting standard par
	\hphantom{.}‚{\color{DodgerBlue3}‚त‚थाऽव्यापृतेन्द्रिय‚स्यान्य‚वाङ‏्मात्रेण} स्व‚ल‚क्ष‚णा‚{\color{DodgerBlue3}‚विभाव‚नात्} प्र‚त्य‚क्ष इव ।
	\pend% ending standard par
      

	  \pstart \leavevmode% starting standard par
	स्यादेत‚त् (।) संकेताविष‚य‚त्वेपि श‚ब्द‚संसृष्ट‚मेव‚{\tiny $_{3}$}‚ स्व‚ल‚क्ष‚ण‚म‚ध्य‚क्षं प्र‚कृत्या ‚{\tiny $_{lb}$}‚प्र‚त्येष्य‚तीत्याह (।)
	\pend% ending standard par
      
	  \bigskip
	  \begingroup
	
	    \large
	  
	    \begin{quote}
	  
	    
	    \stanza[\smallbreak]
	\label{pv.2.132b}\flagstanza{\tiny\textenglish{...2.132b}}न चानुदित‚संब‚न्धः स्व‚यं ज्ञान‚प्र‚स‚ङ्ग‚तः ॥ १३२ ॥\&[\smallbreak]


	
	    \end{quote}
	  
	  \endgroup
	

	  \pstart \leavevmode% starting standard par
	\hphantom{.}‚{\color{DodgerBlue3}‚न चानुदित‚स‚म्ब‚न्धो} वाच्य‚वाच‚क‚भावो य‚स्य स श‚ब्दः प्र‚त्याय‚को दृष्ट ‚{\tiny $_{lb}$}‚इति शेषः (।) त‚थाभ्युप‚ग‚मे तु ‚{\color{DodgerBlue3}‚स्व‚यं} संकेत‚म‚न‚पेक्ष्यैव श्रुताच्छ‚ब्दाद‚र्थ‚स्य ‚{\color{DodgerBlue3}‚ज्ञान‚{\tiny $_{lb}$}‚प्र‚स‚ङ्ग‚तः} । (१३२)
	\pend% ending standard par
      \label{div_pvv.2.133}
	  
	% new div opening: depth here is 2
	

	  \pstart \leavevmode% starting standard par
	न‚न्विन्द्रिय‚व्यापार‚स‚म‚काल‚म‚हिर‚हिरिति धारावाहिस‚विक‚ल्प‚क‚म‚ध्य‚क्षं प्र‚व‚र्त‚ते । ‚{\tiny $_{lb}$}‚य‚दि तु त‚त्र विक‚ल्प‚क‚म‚विक‚ल्प‚क‚ञ्च द्व‚य‚मिष्य‚ते त‚दा विक‚ल्पेन निर्व्विक‚ल्प‚स्य ‚{\tiny $_{lb}$}‚व्य‚व‚धानाद् द‚र्श‚नं‚{\tiny $_{4}$}‚ विच्छिन्नं स्यात् । न चैत‚द‚स्ति । न युग‚प‚ज्ज्ञान‚स‚म्भ‚वः (।) ‚{\tiny $_{lb}$}‚अत्राह (।)
	\pend% ending standard par
      
	  \bigskip
	  \begingroup
	
	    \large
	  
	    \begin{quote}
	  
	    
	    \stanza[\smallbreak]
	\label{pv.2.133}\flagstanza{\tiny\textenglish{....2.133}}म‚न‚सो युग‚प‚द्वृत्तेः स‚विक‚ल्पाविक‚ल्प‚योः&विमूढो ल‚घुवृत्तेर्व्वा त‚योरैक्यं व्य‚व‚स्य‚ति ॥ १३३ ॥\&[\smallbreak]


	
	    \end{quote}
	  
	  \endgroup
	\textsuperscript{\textenglish{157/s}}

	  \pstart \leavevmode% starting standard par
	\hphantom{.}‚{\color{DodgerBlue3}‚म‚न‚सोः स‚विक‚ल्पाविक‚ल्प‚यो}‚रेक‚स्मात्स‚म‚न‚न्त‚रा‚{\color{DodgerBlue3}‚द्युग‚प‚द् वृत्तेः} कार‚णात्त‚योरैक्यं ‚{\tiny $_{lb}$}‚‚{\color{DodgerBlue3}‚विमूढः} प्र‚तिप‚त्ता व्य‚व‚स्य‚ति । निर्व्विक‚ल्प‚कं हि स्व‚ल‚क्ष‚ण‚विष‚यं विक‚ल्प‚श्च व‚स्तुतो‚{\tiny $_{lb}$}‚ऽत‚द्विष‚य‚त्वेप्य‚व‚सायानुरोधात्त‚द्विष‚यः\edtext{}{\edlabel{pvv.157-1}\label{pvv.157-1}\lemma{यः}\Bfootnote{अस्प‚ष्टः ।}} । स‚होत्प‚त्तिश्चानुभ‚व‚सिद्ध‚त्वात् दुर‚प‚ह्न‚वा । ‚{\tiny $_{lb}$}‚त‚तः स‚होत्प‚न्न‚योरेक‚विष‚य‚योरैक्य‚भ्र‚म एषः । प‚राभिम‚तायां‚{\tiny $_{5}$}‚ युग‚प‚द‚नुत्प‚त्ताव‚पि ‚{\tiny $_{lb}$}‚स‚विक‚ल्पाविक‚ल्प‚योर्ल‚घुवृत्तेः शीघ्र‚{\color{DodgerBlue3}‚वृत्ते}‚र्व्वा कार‚णात् ‚{\color{DodgerBlue3}‚त‚योर्मू}‚ढ‚म‚तिः प्र‚तिप‚त्ता ‚{\color{DodgerBlue3}‚ऐक्यं ‚{\tiny $_{lb}$}‚व्य‚व‚स्य‚ति} अलात‚भ्रान्तौ च‚क्र‚मिव । (१३३)
	\pend% ending standard par
      \label{div_pvv.2.134}
	  
	% new div opening: depth here is 2
	

	  \pstart \leavevmode% starting standard par
	य‚दि द‚र्श‚न‚म‚विक‚ल्पं त‚द‚न‚न्त‚र‚न्तु विक‚ल्पः पुन‚स्त‚द‚न‚न्त‚रं द‚र्श‚नं त‚दा (।)
	\pend% ending standard par
      
	  \bigskip
	  \begingroup
	
	    \large
	  
	    \begin{quote}
	  
	    
	    \stanza[\smallbreak]
	\label{pv.2.134}\flagstanza{\tiny\textenglish{....2.134}}विक‚ल्प‚व्य‚व‚धानेन विच्छिन्नं द‚र्श‚न‚म्भ‚वेत् ।&इति चेद् भिन्न‚जातीय‚विक‚ल्पेन्य‚स्य वा क‚थ‚म् ॥ १३४ ॥\&[\smallbreak]


	
	    \end{quote}
	  
	  \endgroup
	

	  \pstart \leavevmode% starting standard par
	\hphantom{.}‚{\color{DodgerBlue3}‚विक‚ल्पेन} व्य‚व‚धाना‚{\color{DodgerBlue3}‚द्विच्छिन्नं द‚र्श‚नं भ‚वेत्} । न धारावाहीति चेत् । ‚{\color{DodgerBlue3}‚न‚न्व‚स्य} प‚र‚स्यापि गां प‚श्य‚तो निर्व्विक‚ल्पेन प्र‚त्य‚क्षेण भिन्न‚जातीय‚स्याश्वादेर्व्विक‚ल्पे जा‚{\tiny $_{6}$}‚य‚{\tiny $_{lb}$}‚माने द‚र्श‚नं क‚थ‚म‚विच्छिन्नं । न ह्य‚श्व‚वाच‚क‚श‚ब्देन संयोज्य गौर्गृह्य‚ते । येनैक‚मेव ‚{\tiny $_{lb}$}‚स‚विक‚ल्पं त‚द‚ध्य‚क्षं भ‚वेत् । एक‚त्वे वाश्व‚प्र‚तीतिर्न स्यात् । गोविष‚य‚त्वात्त‚स्य । ‚{\tiny $_{lb}$}‚(१३४)
	\pend% ending standard par
      \label{div_pvv.2.135}
	  
	% new div opening: depth here is 2
	

	  \pstart \leavevmode% starting standard par
	स्यादेत‚त् ।
	\pend% ending standard par
      
	  \bigskip
	  \begingroup
	
	    \large
	  
	    \begin{quote}
	  
	    
	    \stanza[\smallbreak]
	\label{pv.2.135}\flagstanza{\tiny\textenglish{....2.135}}अलात‚दृष्टिव‚द् भाव‚प‚क्ष‚श्चेद् ब‚ल‚वान् म‚तः ।&अन्य‚त्रापि स‚मानं त‚द् व‚र्ण‚योर्व्वा स‚कृच्छृ तिः ॥ १३५ ॥\&[\smallbreak]


	
	    \end{quote}
	  
	  \endgroup
	

	  \pstart \leavevmode% starting standard par
	अलात‚स्य भ्र‚म्य‚मान‚स्य नानादेशेषु लाघ‚वाद् भाव‚प‚क्ष‚ब‚ल‚व‚त्वाच्च द‚र्श‚न‚{\tiny $_{lb}$}‚प्र‚तिस‚न्धानेन च‚क्र‚दृ‚{\color{DodgerBlue3}‚ष्टिव‚त्} विजातीय‚व्य‚व‚कीर्य‚माण‚स्य द‚र्श‚न‚स्यान्त‚रा अभावेपि\leavevmode\ledsidenote{\textenglish{30b/MA}} ‚{\tiny $_{lb}$}‚‚{\color{DodgerBlue3}‚भाव‚प‚क्षो ब‚ल‚वान् म‚त} इति द‚र्श‚ना‚{\tiny $_{7}$}‚विच्छेद‚बुद्धिर‚ति ‚{\color{DodgerBlue3}‚चेत् । अन्य‚त्राभाव‚प‚क्षेपि} त‚द्ब‚ल‚व‚त्त्वे लाघ‚व‚साम‚र्थ्यात् ‚{\color{DodgerBlue3}‚स‚मानं\edtext{}{\edlabel{pvv.157-2}\label{pvv.157-2}\lemma{मानं}\Bfootnote{दृष्टाविमृष्टादौ ।}}} । त‚तो द‚र्श‚न‚विच्छेद‚लाघ‚वाद्विच्छेद‚धीरे‚{\tiny $_{lb}$}‚वास्तु । स‚रो र‚स इत्यादौ ‚{\color{DodgerBlue3}‚व‚र्ण्ण‚यो}‚र्वा लाघ‚वा‚{\color{DodgerBlue3}‚त्स\edtext{}{\edlabel{pvv.157-3}\label{pvv.157-3}\lemma{त्स}\Bfootnote{त‚था क्र‚माभेदात्स‚कृत्श्रुतिभेवो न स्यात् ।}}कृत्श्रुतिः} प्राप्ता ॥(१३५)
	\pend% ending standard par
      \label{div_pvv.2.136}
	  
	% new div opening: depth here is 2
	

	  \begin{center}%% label @type='head'
	\textbf{ख. प‚र‚म‚त‚दूष‚ण‚म्}
	\end{center}
	

	  \pstart \leavevmode% starting standard par
	किञ्च (।)
	\pend% ending standard par
      
	  \bigskip
	  \begingroup
	
	    \large
	  
	    \begin{quote}
	  
	    
	    \stanza[\smallbreak]
	\label{pv.2.136}\flagstanza{\tiny\textenglish{....2.136}}स‚कृत् स‚ङ्ग‚त‚श‚ब्दार्थेष्विन्द्रियेष्विह स‚त्स्व‚पि ।&प‚ञ्च‚भिर्व्य‚व‚धानेपि भात्य‚व्य‚व‚हितेव या ॥ १३६ ॥\&[\smallbreak]


	
	    \end{quote}
	  
	  \endgroup
	\textsuperscript{\textenglish{158/s}}

	  \pstart \leavevmode% starting standard par
	\hphantom{.}‚{\color{DodgerBlue3}‚स‚कृत्} यु\edtext{}{\edlabel{pvv.158-1}\label{pvv.158-1}\lemma{यु}\Bfootnote{युग‚प‚द्विज्ञानानुत्प‚त्तिर्म‚न‚सो लिङ्ग‚मि\href{http://sarit.indology.info/?cref=nsū.1.1.16}{ (न्याय‚सूत्रे १।१।१६)} ति दूष‚य‚न्नाह ।}}ग‚प‚{\color{DodgerBlue3}‚त्संग‚ताः} स्व‚स्व‚गोच‚रीभूताः स‚र्व्वेऽ‚{\color{DodgerBlue3}‚र्था}\edtext{}{\edlabel{pvv.158-2}\label{pvv.158-2}\lemma{र्व्वेऽ}\Bfootnote{युग‚प‚द्विज्ञानानुत्प‚त्तिप‚क्षे ।}} येषु ‚{\tiny $_{lb}$}‚ते‚{\color{DodgerBlue3}‚ष्विन्द्रियेषु} च‚क्षुरादिषु ‚{\tiny $_{lb}$}‚म‚नःप‚र्य‚न्तेषु ‚{\color{DodgerBlue3}‚स‚त्स्व‚पीह}‚{\tiny $_{2}$}‚ स‚ङ्क्रान्त‚कान्ताव‚द‚न‚प्र‚तिबिम्ब‚स्य स‚ह‚कार‚सुग‚न्धिनः शीत‚स्य ‚{\tiny $_{lb}$}‚भ्र‚म‚द्भ्र‚म‚रोप‚गीत‚स्य स्वाद्रुनो म‚धुनः सार्व्व‚गुणानुभ‚व‚काले प्र‚स‚र‚त्संक‚ल्प‚ज‚न्म‚नां ‚{\tiny $_{lb}$}‚यूनां या म‚तिः ‚{\color{DodgerBlue3}‚प‚ञ्च‚भि}‚रिन्द्रिय‚बुद्धि‚{\color{DodgerBlue3}‚भिर्व्य‚व‚धानेपि} त्व‚त्प‚क्षे‚{\color{DodgerBlue3}‚ऽव्य‚व}‚हितेव स‚म‚कालेव ‚{\tiny $_{lb}$}‚‚{\color{DodgerBlue3}‚भाति} । (१३६)
	\pend% ending standard par
      \label{div_pvv.2.137}
	  
	% new div opening: depth here is 2
	
	  \bigskip
	  \begingroup
	
	    \large
	  
	    \begin{quote}
	  
	    
	    \stanza[\smallbreak]
	\label{pv.2.137}\flagstanza{\tiny\textenglish{....2.137}}सा म‚तिर्नाम‚प‚र्य‚न्त‚क्ष‚णिक‚ज्ञान‚मिश्र‚णात् ।&विच्छिन्नाभेति त‚च्चित्रं त‚स्मात् स‚न्तु स‚कृद्धियः ॥ १३७ ॥\&[\smallbreak]


	
	    \end{quote}
	  
	  \endgroup
	

	  \pstart \leavevmode% starting standard par
	\hphantom{.}‚{\color{DodgerBlue3}‚सेन्द्रिय‚म‚तिर्नाम्नः} श‚ब्द‚स्य ‚{\color{DodgerBlue3}‚प‚र्य‚न्तो} व‚र्ण्ण‚स्त‚स्य ‚{\color{DodgerBlue3}‚क्ष‚णिकं ज्ञानं} तेन ‚{\color{DodgerBlue3}‚मिश्र‚णात्} स‚रो र‚स इत्यादिष्व‚विच्छिन्ना प्राप्नोति । विजातीय\edtext{}{\edlabel{pvv.158-3}\label{pvv.158-3}\lemma{विजातीय}\Bfootnote{एक‚लोलीभूत‚व‚स्तुग्र‚ह‚णात्मिका स्यात् ।}}विज्ञानान्त‚राव्य‚व‚धानात् । ‚{\tiny $_{lb}$}‚त‚थापि ‚{\color{DodgerBlue3}‚विच्छिन्नाभा} क्र‚म‚व‚ती । य‚त्तु ‚{\color{DodgerBlue3}‚चित्र‚मा}‚श्च‚र्यं य‚दि लाघ‚व‚कृतः स‚कृद्ग्र‚हाभि‚{\tiny $_{lb}$}‚मानः त‚दा व‚र्ण्ण‚ज्ञाने स नित‚रां‚{\tiny $_{2}$}‚ युक्तो विजातीयाव्य‚व‚धानात् । इन्द्रिय‚ज्ञानेषु ‚{\tiny $_{lb}$}‚तु न युक्तः प‚ञ्च‚भिर्व्य‚व‚धानात् । त‚स्मा‚{\color{DodgerBlue3}‚त्स‚कृद्धियः स‚न्तु} य‚थोक्तं म‚न‚सो युग‚प‚द्वृत्ते‚{\tiny $_{lb}$}‚रि (२।१३३) ति ॥(१३७)
	\pend% ending standard par
      \label{div_pvv.2.138}
	  
	% new div opening: depth here is 2
	

	  \pstart \leavevmode% starting standard par
	अन्य\edtext{}{\edlabel{pvv.158-4}\label{pvv.158-4}\lemma{अन्य}\Bfootnote{युग‚प‚ज्ज्ञानोत्प‚त्त्य‚निष्टौ ।}}था (।)
	\pend% ending standard par
      
	  \bigskip
	  \begingroup
	
	    \large
	  
	    \begin{quote}
	  
	    
	    \stanza[\smallbreak]
	\label{pv.2.138}\flagstanza{\tiny\textenglish{....2.138}}प्र‚तिभासाविशेष‚श्च सान्त‚रान‚न्त‚रे क‚थ‚म् ।&शुद्धे म‚नोविक‚ल्पे च न क्र‚म‚ग्र‚ह‚ण‚म्भ‚वेत् ॥ १३८ ॥\&[\smallbreak]


	
	    \end{quote}
	  
	  \endgroup
	

	  \pstart \leavevmode% starting standard par
	\hphantom{.}‚{\color{DodgerBlue3}‚सान्त‚रे} प‚ञ्च‚भिरिन्द्रिय‚ज्ञानैर्व्य‚व‚हित‚त्वात् । ‚{\color{DodgerBlue3}‚अन‚न्त‚रे} स‚रो र‚स इत्यादिके ज्ञाने ‚{\tiny $_{lb}$}‚विजातीयाव्य‚व‚धानात् ‚{\color{DodgerBlue3}‚प्र‚तिभास}‚स्या‚{\color{DodgerBlue3}‚विशेष}‚श्चाप्र‚स‚क्तः (।) स चानुभ‚व‚बाधित‚त्वात् ‚{\tiny $_{lb}$}‚‚{\color{DodgerBlue3}‚क‚थ‚म}‚भ्युप‚ग‚म इति शेषः । य‚दि ल‚घुवृत्तित्वात् स‚कृद्ग्र‚ह‚स्त‚दा ‚{\color{DodgerBlue3}‚शु‚{\tiny $_{3}$}‚द्धे} विजातीया‚{\tiny $_{lb}$}‚व्य‚व‚कीर्ण्णे ‚{\color{DodgerBlue3}‚म‚नोविक‚ल्पे च} प्राब‚न्धिके ‚{\color{DodgerBlue3}‚क्र‚म‚ग्र‚ह‚ण‚म‚नु}‚भ‚व‚सिद्धं न ‚{\color{DodgerBlue3}‚भ‚वेत्} ।(१३८)
	\pend% ending standard par
      \label{div_pvv.2.139}
	  
	% new div opening: depth here is 2
	
	  \bigskip
	  \begingroup
	
	    \large
	  
	    \begin{quote}
	  
	    
	    \stanza[\smallbreak]
	\label{pv.2.139a}\flagstanza{\tiny\textenglish{...2.139a}}योऽग्र‚हः स‚ङ्ग‚तेप्य‚र्थे क्व‚चिदास‚क्त‚चेत‚सः ।\&[\smallbreak]


	
	    \end{quote}
	  
	  \endgroup
	

	  \pstart \leavevmode% starting standard par
	न‚नु य‚दि युग‚प‚त् ज्ञानोत्प‚त्तिस्त‚दैक‚त्रास‚क्तं पुन‚रुत्प‚त्तिध‚र्म‚कं चेतो य‚स्य त‚स्या‚{\tiny $_{lb}$}‚‚{\color{DodgerBlue3}‚स‚क्त‚चेत‚सः क्व‚चिद‚र्थे} स्वेन्द्रियेण ‚{\color{DodgerBlue3}‚स‚ङ्ग‚तेपि} योऽग्र‚हो विज्ञानानुत्प‚त्तिः स क‚थं ।
	\pend% ending standard par
      

	  \pstart \leavevmode% starting standard par
	आह (।)
	\pend% ending standard par
      
	  \bigskip
	  \begingroup
	
	    \large
	  
	    \begin{quote}
	  
	    
	    \stanza[\smallbreak]
	\label{pv.2.139b}\flagstanza{\tiny\textenglish{...2.139b}}स‚क्त्‏या नोत्प‚त्तिवैगुण्याच्चोद्यं वै त‚द्द्व‚योर‚पि ॥ १३९ ॥\&[\smallbreak]


	
	    \end{quote}
	  
	  \endgroup
	\textsuperscript{\textenglish{159/s}}

	  \pstart \leavevmode% starting standard par
	\hphantom{.}एक‚त्र ‚{\color{DodgerBlue3}‚स‚क्त्या} विष‚यास‚ञ्चार‚ल‚क्ष‚ण‚याऽन्य‚स्य भिन्न‚विष‚य‚स्य ज्ञान‚स्यो‚{\color{DodgerBlue3}‚त्प‚त्ति‚{\tiny $_{lb}$}‚वैगुण्यात्} । अविगुणो हि स‚{\tiny $_{4}$}‚म‚न‚न्त‚र‚प्र‚त्य‚यः स्व‚कार्य‚मार‚भ‚ते । न त्वास‚क्ति‚{\tiny $_{lb}$}‚विगुणः । ‚{\color{DodgerBlue3}‚य‚च्चैत‚च्चोद्यं} प‚रिहृत‚म‚स्माभिस्त‚द् ‚{\color{DodgerBlue3}‚द्व‚योर‚पि} स‚मानं युग‚प‚द्विज्ञाना‚{\tiny $_{lb}$}‚नुत्प‚त्तिवादिनोपि म‚ते स‚र्व्व‚त्रेवैन्द्रिय‚संग‚मे स‚माने क्व‚चिदेव ‚{\color{DodgerBlue3}‚ज्ञानं क्व‚चिन्नेति} कुतः। (१३९)
	\pend% ending standard par
      \label{div_pvv.2.140}
	  
	% new div opening: depth here is 2
	

	  \pstart \leavevmode% starting standard par
	त‚त्रास‚क्तिवैगुण्य‚मेव म‚न‚स उत्त‚रं त‚च्च स‚मान‚म‚स्माक‚म‚लात‚दृष्टिव‚दिति ‚{\tiny $_{lb}$}‚दृष्टान्त‚स्यासिद्धिमाह (।)
	\pend% ending standard par
      
	  \bigskip
	  \begingroup
	
	    \large
	  
	    \begin{quote}
	  
	    
	    \stanza[\smallbreak]
	\label{pv.2.140}\flagstanza{\tiny\textenglish{....2.140}}शीघ्र‚वृत्तेर‚लातादेर‚न्व‚य‚प्र‚तिधातिनी ।&च‚क्र‚भ्रान्तिं दृगाध‚त्ते न दृशां घ‚ट‚नेन सा ॥ १४० ॥\&[\smallbreak]


	
	    \end{quote}
	  
	  \endgroup
	

	  \pstart \leavevmode% starting standard par
	\hphantom{.}‚{\color{DodgerBlue3}‚शीघ्रा} प्र‚{\color{DodgerBlue3}‚वृत्तिर्भ्र}‚म‚णं य‚स्या‚{\color{DodgerBlue3}‚लाता}‚देस्त‚{\color{DodgerBlue3}‚स्या‚{\tiny $_{5}$}‚न्व‚येनानुग‚मेन प्र‚तिघात} उप‚ह‚त‚त्वं ‚{\tiny $_{lb}$}‚त‚द्व‚ती दृग् दृष्टि‚{\color{DodgerBlue3}‚श्च‚क्रा}‚कारां ‚{\color{DodgerBlue3}‚भ्रान्ति}‚मिन्द्रिय‚जां ‚{\color{DodgerBlue3}‚ध‚त्ते ।\edtext{\textsuperscript{*}}{\edlabel{pvv.159-1}\label{pvv.159-1}\lemma{*}\Bfootnote{स्व‚म‚त‚माख्याय प‚र‚म‚तं निषेध‚ति ।}}\edtext{\textsuperscript{*}}{\edlabel{pvv.1591a}\label{pvv.1591a}\lemma{*}\Bfootnote{1a त‚दाल‚म्ब‚नं ।\begin{english} --- Placement of note uncertain; marked with a question mark in the edition (see encoding description for details).\end{english}}} न दृशां} भिन्न‚भिन्न‚देशाला‚{\tiny $_{lb}$}‚त‚द‚र्श‚नानां ‚{\color{DodgerBlue3}‚घ‚ट‚नेन} योज‚न‚या ‚{\color{DodgerBlue3}‚सा} मान‚सी भ्रान्तिः स्फुट‚प्र‚तिभास‚त्वात् । ‚{\color{DodgerBlue3}‚मान‚स‚स्य} न विप‚र्य‚यात् । त‚स्माद‚हिर‚हिरिति विक‚ल्प‚स‚म‚काल‚म‚ध्य‚क्षं व‚स्तु स्फुट‚म‚वैति न ‚{\tiny $_{lb}$}‚तु विक‚ल्प इति स्थितं । (१४०)
	\pend% ending standard par
      \label{div_pvv.2.141}
	  
	% new div opening: depth here is 2
	
	  \bigskip
	  \begingroup
	
	    \large
	  
	    \begin{quote}
	  
	    
	    \stanza[\smallbreak]
	\label{pv.2.141}\flagstanza{\tiny\textenglish{....2.141}}केचिदिन्द्रिय‚ज‚त्वादेर्बाल‚धीव‚द‚क‚ल्प‚नाम् ।&आहुर्बालाविक‚ल्पे च हेतुं स‚ङ्केत‚म‚न्द‚ताम् ॥ १४१ ॥\&[\smallbreak]


	
	    \end{quote}
	  
	  \endgroup
	

	  \pstart \leavevmode% starting standard par
	\hphantom{.}‚{\color{DodgerBlue3}‚केचि}‚दाचार्यीयाः श ङ्क र स्वा मि प्र‚भृत‚यः ‚{\color{DodgerBlue3}‚इन्द्रिय‚ज‚त्वा}‚दादिश‚ब्दा‚{\tiny $_{6}$}‚द‚मान‚स‚त्वा‚{\tiny $_{lb}$}‚नुभ‚वाकार‚प्र‚वृत्त‚त्त्वादेर्हेतोः प्र‚त्य‚क्ष‚बुद्धि‚{\color{DodgerBlue3}‚म‚क‚ल्प‚नां बाल‚धीव‚दाहुः} । दृष्टान्त‚सिद्ध‚य‚र्थं । ‚{\tiny $_{lb}$}‚बाल‚स्या‚{\color{DodgerBlue3}‚विक‚ल्पे} विक‚ल्पाभावे च ‚{\color{DodgerBlue3}‚संकेत‚मंद‚तां} हेतुमाहुः । वाच्य‚वाच‚क‚योज‚ना ‚{\tiny $_{lb}$}‚हि विक‚ल्पः । सा च संकेत‚पूर्व्विका त‚द‚भावाद् बाल‚स्य क‚ल्प‚नाभावः । (१४१)
	\pend% ending standard par
      \label{div_pvv.2.142}
	  
	% new div opening: depth here is 2
	
	  \bigskip
	  \begingroup
	
	    \large
	  
	    \begin{quote}
	  
	    
	    \stanza[\smallbreak]
	\label{pv.2.142}\flagstanza{\tiny\textenglish{....2.142}}तेषां प्र‚त्य‚क्ष‚मेव स्याद् बालानाम‚विक‚ल्प‚नात् ।&स‚ङ्केतोपाय‚विग‚मात् प‚श्चाद‚पि भ‚वेन्न सः ॥ १४२ ॥\&[\smallbreak]


	
	    \end{quote}
	  
	  \endgroup
	

	  \pstart \leavevmode% starting standard par
	\hphantom{.}‚{\color{DodgerBlue3}‚तेषामेवं} वादिनां म‚ते ‚{\color{DodgerBlue3}‚बालानां प्र‚त्य‚क्ष‚मेव} ज्ञानं ‚{\color{DodgerBlue3}‚स्यान्न} विचार‚कं (।) किं ‚{\tiny $_{lb}$}‚कार‚ण‚म‚विक‚ल्प‚नात् । भ‚व‚तु को दोष‚{\tiny $_{7}$}‚ इति चेत् । आह ‚{\color{DodgerBlue3}‚संकेतोपाय‚स्य} विचार‚स्य ‚{\color{DodgerBlue3}‚विग-\leavevmode\ledsidenote{\textenglish{31a/MA}} ‚{\tiny $_{lb}$}‚मात्} बालानां ‚{\color{DodgerBlue3}‚प‚श्चाद‚पि} स स‚ङ्केतो ‚{\color{DodgerBlue3}‚न भ‚वेत्} । त‚द‚भावाद्विक‚ल्पाभाव‚श्च । (१४२)
	\pend% ending standard par
      \label{div_pvv.2.143}
	  
	% new div opening: depth here is 2
	
	  \bigskip
	  \begingroup
	
	    \large
	  
	    \begin{quote}
	  
	    
	    \stanza[\smallbreak]
	\label{pv.2.143}\flagstanza{\tiny\textenglish{....2.143}}म‚नो व्युत्प‚न्न‚स‚ङ्केत‚म‚स्ति तेन स चेन्म‚तः ।&एव‚मिन्द्रिय‚जेपि स्याद् शेष‚व‚च्चेद‚मीदृश‚म् ॥ १४३ ॥\&[\smallbreak]


	
	    \end{quote}
	  
	  \endgroup
	\textsuperscript{\textenglish{160/s}}

	  \pstart \leavevmode% starting standard par
	\hphantom{.}ज‚न्मान्त‚राग‚तं ‚{\color{DodgerBlue3}‚व्युत्प‚न्नं-संकेतं म‚नोस्ति} बालानां ‚{\color{DodgerBlue3}‚तेन} संकेतः तेषां म‚त‚श्चेत् । ‚{\tiny $_{lb}$}‚‚{\color{DodgerBlue3}‚एवं} स‚ती‚{\color{DodgerBlue3}‚न्द्रिय‚जेपि} ज्ञाने ‚{\color{DodgerBlue3}‚स्यात्} क‚ल्प‚ना त‚न्निव‚र्त‚क‚हेत्व‚भिधानात् त‚तो\edtext{}{\edlabel{pvv.160-1}\label{pvv.160-1}\lemma{तो}\Bfootnote{असिद्धिर्दृष्टे व‚स्तुनि ।}} दृष्टासिद्धि‚{\tiny $_{lb}$}‚रेव । ‚{\color{DodgerBlue3}‚इद}‚मिन्द्रिय‚ज‚त्वादि ‚{\color{DodgerBlue3}‚ईदृशं} निषेध्येन स‚हासिद्ध\edtext{}{\edlabel{pvv.160-2}\label{pvv.160-2}\lemma{हासिद्ध}\Bfootnote{विप‚क्षे बाध‚नाद‚र्श‚नात् ।}}विरोधं । शेष‚व‚च्चोक्तं । ‚{\tiny $_{lb}$}‚(१४३)
	\pend% ending standard par
      \label{div_pvv.2.144}
	  
	% new div opening: depth here is 2
	

	  \pstart \leavevmode% starting standard par
	अथान्येन लिङ्गेनाव्य‚भिचारिणा बाल‚ज्ञान‚म‚विक‚ल्प‚नं‚{\tiny $_{1}$}‚ प्र‚साध्य दृष्टान्ती‚{\tiny $_{lb}$}‚क्रिय‚ते त‚दा (।)
	\pend% ending standard par
      
	  \bigskip
	  \begingroup
	
	    \large
	  
	    \begin{quote}
	  
	    
	    \stanza[\smallbreak]
	\label{pv.2.144}\flagstanza{\tiny\textenglish{....2.144}}य‚देव साध‚नं बाले त‚देवात्रापि क‚थ्य‚ताम् ।&साम्याद‚क्ष‚धियामुक्त‚म‚नेनानुभ‚वादिक‚म् ॥ १४४ ॥\&[\smallbreak]


	
	    \end{quote}
	  
	  \endgroup
	

	  \pstart \leavevmode% starting standard par
	\hphantom{.}‚{\color{DodgerBlue3}‚य‚देवा}‚व्य‚भिचारि ‚{\color{DodgerBlue3}‚बाले} बाल‚स्येन्द्रिय‚ज्ञाने ‚{\color{DodgerBlue3}‚साध‚नं त‚देवात्र} व्युत्प‚न्न‚संकेता‚{\tiny $_{lb}$}‚नामिन्द्रिय‚ज्ञानेपि ‚{\color{DodgerBlue3}‚क‚थ्य‚तां} किमिन्द्रिज‚त्वादिनोप‚न्य‚स्तेन । व्युत्प‚न्नाव्युत्प‚न्न‚यो‚{\color{DodgerBlue3}‚र‚क्ष‚{\tiny $_{lb}$}‚धियाम}‚श‚ब्द‚संसृष्ट‚त्व‚मात्रानुक‚र‚णेन ‚{\color{DodgerBlue3}‚साम्यात् । अनेने}‚न्द्रिय‚ज‚त्व‚दूष‚णेना‚{\color{DodgerBlue3}‚नुभ‚व} आदि‚{\tiny $_{lb}$}‚र्य‚स्य मान‚स‚त्त्वादेस्त‚दुक्तं दोष‚व‚त्त‚या बोद्ध‚व्यं । (१४४)
	\pend% ending standard par
      \label{div_pvv.2.145}
	  
	% new div opening: depth here is 2
	

	  \begin{center}%% label @type='head'
	\textbf{(२) सामान्य‚निरासः}
	\end{center}
	

	  \begin{center}%% label @type='head'
	\textbf{क. व‚र्ण‚संस्थान‚राहित्याद‚सिद्धिः}
	\end{center}
	

	  \pstart \leavevmode% starting standard par
	अविक‚ल्प‚सिद्धौ प‚र\edtext{}{\edlabel{pvv.160-3}\label{pvv.160-3}\lemma{र}\Bfootnote{जातिगुण‚क्रियाद्र‚व्य‚स‚म्ब‚न्ध‚भेदाच्च‚तुष्ट‚यी श‚ब्दानां प्र‚वृत्तिरिति स‚विक‚ल्प‚वादिनं निषेध‚ति ।}}म‚तं दूष‚यित्वा स्व‚यं उप‚प‚त्त्य‚न्त‚र‚माह (।)
	\pend% ending standard par
      
	  \bigskip
	  \begingroup
	
	    \large
	  
	    \begin{quote}
	  
	    
	    \stanza[\smallbreak]
	\label{pv.2.145}\flagstanza{\tiny\textenglish{....2.145}}विशेष‚णं विशेष्य‚ञ्च स‚म्ब‚न्धं लौकिकीं स्थितिम् ।&गृहीत्वा स‚ङ्क‚ल‚य्यैत‚त् त‚था प्र‚त्येति नान्य‚था ॥ १४५ ॥\&[\smallbreak]


	
	    \end{quote}
	  
	  \endgroup
	

	  \pstart \leavevmode% starting standard par
	\hphantom{.}‚{\color{DodgerBlue3}‚विशेष‚णं‚{\tiny $_{2}$}‚} व्य‚व‚च्छेद‚कं ‚{\color{DodgerBlue3}‚विशेष्यं} व्य‚व‚च्छेद्यं त‚योः ‚{\color{DodgerBlue3}‚स‚म्ब‚न्धं} य‚थास‚म्भ‚वं स‚म‚वाया‚{\tiny $_{lb}$}‚दिकं ‚{\color{DodgerBlue3}‚लौकिकीं} लोक‚प्र‚सिद्धां ‚{\color{DodgerBlue3}‚स्थितिं} व्य‚व‚स्थाञ्च जात्यादिकं विशेष‚णं विशेष्यं ‚{\tiny $_{lb}$}‚द्र‚व्यादि विशेष‚ण‚विशेष्य‚श‚ब्द‚योश्च पूर्व्वाप‚र‚निय‚म इति पृथ‚क् प्र‚त्येकं स्व‚रूपेण ‚{\tiny $_{lb}$}‚‚{\color{DodgerBlue3}‚गृहीत्वा} त‚द‚न‚न्त‚र‚{\color{DodgerBlue3}‚मेत‚त्} स‚र्व्व ‚{\color{DodgerBlue3}‚स‚ङ्क‚ल‚य्य} संयोज्य ‚{\color{DodgerBlue3}‚त‚था} विशेष‚ण‚विशिष्ट‚त्वेन ‚{\color{DodgerBlue3}‚प्र‚त्येति} विशिष्ट‚बुद्धि‚{\color{DodgerBlue3}‚र्नान्य‚था} । विशेष‚णाद्य‚ग्र‚ह‚णे । (१४५)
	\pend% ending standard par
      \label{div_pvv.2.146}
	  
	% new div opening: depth here is 2
	\textsuperscript{\textenglish{161/s}}
	  \bigskip
	  \begingroup
	
	    \large
	  
	    \begin{quote}
	  
	    
	    \stanza[\smallbreak]
	\label{pv.2.146}\flagstanza{\tiny\textenglish{....2.146}}य‚था द‚ण्डिनि । जात्यादेर्व्विवेकेनानिरूप‚णात् ।&त‚द्व‚ता योज‚ना नास्ति क‚ल्प‚नाप्य‚त्र नास्त्य‚तः ॥ १४६ ॥\&[\smallbreak]


	
	    \end{quote}
	  
	  \endgroup
	

	  \pstart \leavevmode% starting standard par
	\hphantom{.}‚{\color{DodgerBlue3}‚य‚था द‚ण्डिनि} द‚ण्डीति‚{\tiny $_{3}$}‚ विशिष्ट‚बुद्धिः द‚ण्ड‚पुरुष‚त‚त्स‚म्ब‚न्धादि‚{\tiny $_{lb}$}‚ग्र‚ह‚ण‚पूर्व्विका त‚द‚ग्र‚हे च न भ‚व‚ति । जातिरादिर्य‚स्य गुण‚क‚र्मादेः ‚{\color{DodgerBlue3}‚स्व‚रूप‚स्य ‚{\tiny $_{lb}$}‚जात्या\edtext{}{\edlabel{pvv.161-1}\label{pvv.161-1}\lemma{जात्या}\Bfootnote{इयं जातिर‚यं जातिमानित्यादिविवेकेन न भाति य‚तो योज‚ना स्यात् द‚ण्डिव‚त् त‚तो न जात्याद‚यः स‚न्ति ।}}दिम‚तो विवेकेनानिरूप‚णात् त‚द्व‚ता} जातिम‚ता ‚{\color{DodgerBlue3}‚योज‚ना} विशेष‚ण‚विशेष्य‚भावो ‚{\tiny $_{lb}$}‚‚{\color{DodgerBlue3}‚नास्ति । अतो} योज‚नाविर‚हात् ‚{\color{DodgerBlue3}‚अत्र} जातिम‚दादौ ‚{\color{DodgerBlue3}‚क‚ल्प‚नापि नास्तीति} त‚स्मा‚{\tiny $_{lb}$}‚ज्जात्यादियोज‚नात्मिका क‚ल्प‚ना नास्ति । श‚ब्द‚योज‚नात्मिका तु ‚{\color{DodgerBlue3}‚स‚म्भाव्येत} । ‚{\tiny $_{lb}$}‚सापि स्व‚ल‚क्ष‚णे संकेताभावान्निर‚{\tiny $_{4}$}‚स्ता प्राक्॥ (१४६)
	\pend% ending standard par
      \label{div_pvv.2.147}
	  
	% new div opening: depth here is 2
	

	  \pstart \leavevmode% starting standard par
	न‚नु य‚दि सामान्याभाव‚स्त‚दा विभिन्नासु व्य‚क्तिषु क‚थ‚म‚न्व‚यिप्र‚त्य‚य ‚{\tiny $_{lb}$}‚इत्याह (।)
	\pend% ending standard par
      
	  \bigskip
	  \begingroup
	
	    \large
	  
	    \begin{quote}
	  
	    
	    \stanza[\smallbreak]
	\label{pv.2.147a}\flagstanza{\tiny\textenglish{...2.147a}}य‚द‚प्य‚न्व‚यिविज्ञानं श‚ब्द‚व्य‚क्त्य‚व‚भासि त‚त् ।\&[\smallbreak]


	
	    \end{quote}
	  
	  \endgroup
	

	  \pstart \leavevmode% starting standard par
	\hphantom{.}‚{\color{DodgerBlue3}‚य‚द‚प्य‚न्व‚यिविज्ञान‚मु}‚त्प‚द्य‚ते, त‚च्च ‚{\color{DodgerBlue3}‚श‚ब्द}‚स्य गौरित्यादे‚{\color{DodgerBlue3}‚र्व्य‚क्ते}‚श्च व‚र्ण्ण‚संस्थान‚{\tiny $_{lb}$}‚विशेष‚स्य आभास आकार‚स्त‚द्व‚त्प्र‚तीय‚ते न जात्याभास‚व‚त् ।
	\pend% ending standard par
      

	  \pstart \leavevmode% starting standard par
	किं पुनः सामान्याभास‚मेव नेत्याह (।)
	\pend% ending standard par
      
	  \bigskip
	  \begingroup
	
	    \large
	  
	    \begin{quote}
	  
	    
	    \stanza[\smallbreak]
	\label{pv.2.147b}\flagstanza{\tiny\textenglish{...2.147b}}व‚र्ण्णाकृत्य‚क्ष‚राकार‚शून्यं गोत्वं हि व‚र्ण्ण्य‚ते ॥ १४७ ॥\&[\smallbreak]


	
	    \end{quote}
	  
	  \endgroup
	

	  \pstart \leavevmode% starting standard par
	\hphantom{.}‚{\color{DodgerBlue3}‚व‚र्ण्णो} नीलादिरा‚{\color{DodgerBlue3}‚कृतिः} संस्थान‚{\color{DodgerBlue3}‚म‚क्ष‚रं} ग‚वादिश‚ब्दः । तेषामाकारो य‚था ‚{\tiny $_{lb}$}‚प्र‚तीतः तेन ‚{\color{DodgerBlue3}‚शून्यं गोत्वं हि} सामान्य‚वा‚{\tiny $_{5}$}‚दिभि‚{\color{DodgerBlue3}‚र्व्व‚र्ण्य‚ते} (१४७)
	\pend% ending standard par
      \label{div_pvv.2.148}
	  
	% new div opening: depth here is 2
	

	  \pstart \leavevmode% starting standard par
	अतोऽन्व‚यिविज्ञाने य‚द्व‚र्ण्ण‚संस्थानादि प्र‚तिभास‚ते न त‚त्सामान्यं (।)
	\pend% ending standard par
      
	  \bigskip
	  \begingroup
	
	    \large
	  
	    \begin{quote}
	  
	    
	    \stanza[\smallbreak]
	\label{pv.2.148}\flagstanza{\tiny\textenglish{....2.148}}स‚मान‚त्वेपि त‚स्यैव नेक्ष‚णं नेत्र‚गोच‚रे ।&प्र‚तिभास‚द्व‚याभावात् बुद्धे र्भेद‚श्च दुर्ल‚भः ॥ १४८ ॥\&[\smallbreak]


	
	    \end{quote}
	  
	  \endgroup
	

	  \pstart \leavevmode% starting standard par
	\hphantom{.}‚{\color{DodgerBlue3}‚त‚स्यैव स‚मान‚त्वे} वा स्वीक्रिय‚माणे ‚{\color{DodgerBlue3}‚नेत्र‚गोच}‚रेऽर्थे नाङ्गीक‚र्त‚व्य‚मीक्ष‚णं । ‚{\tiny $_{lb}$}‚[विक‚ल्प‚प्र‚तिभासिनः] ‚{\color{DodgerBlue3}‚प्र‚तिभास‚द्व‚य‚स्य} स्फुटास्फुट‚व‚र्ण्ण‚संस्थान‚व‚तो‚{\color{DodgerBlue3}‚ऽभावात्} । एका‚{\tiny $_{lb}$}‚कार‚मेव ज्ञानं य‚द्युभ‚याभास‚म‚ङ्गीक्रिय‚ते त‚दा ‚{\color{DodgerBlue3}‚बुद्धे}‚(र्भे)दः प्र‚त्य‚क्ष‚त्वाप्र‚त्य‚क्ष‚त्वादिना ‚{\tiny $_{lb}$}‚दुर्ल‚भः । य‚द‚पि स्प‚ष्ट‚प्र‚तिभास‚म‚ध्य‚क्षं त‚द‚प्य‚न‚क्ष‚जं स्यात् । अस्प‚ष्ट‚प्र‚तिभास‚त्वात्‚{\tiny $_{6}$}‚ । ‚{\tiny $_{lb}$}‚एव‚म‚न‚क्ष‚ज‚म‚ध्य‚क्षं स्यात् । स्प‚ष्ट‚प्र‚तिभास‚त्वात् । त‚स्माद‚दृष्टेन सामान्यादिना न ‚{\tiny $_{lb}$}‚योज‚य‚तीत्य‚क‚ल्प‚न‚म‚ध्य‚क्षं (। १४८)
	\pend% ending standard par
      \label{div_pvv.2.149}
	  
	% new div opening: depth here is 2
	\textsuperscript{\textenglish{162/s}}

	  \begin{center}%% label @type='head'
	\textbf{ख. स‚म‚वाय‚स्यातीन्द्रिय‚त्वाद‚सिद्धिः}
	\end{center}
	

	  \pstart \leavevmode% starting standard par
	किञ्च (।)
	\pend% ending standard par
      
	  \bigskip
	  \begingroup
	
	    \large
	  
	    \begin{quote}
	  
	    
	    \stanza[\smallbreak]
	\label{pv.2.149a}\flagstanza{\tiny\textenglish{...2.149a}}स‚म‚वायाग्र‚हाद‚क्षैः स‚म्ब‚न्धाद‚र्श‚नं स्थित‚म् ।\&[\smallbreak]


	
	    \end{quote}
	  
	  \endgroup
	

	  \pstart \leavevmode% starting standard par
	\hphantom{.}‚{\color{DodgerBlue3}‚स‚म‚वा\edtext{}{\edlabel{pvv.162-1}\label{pvv.162-1}\lemma{वा}\Bfootnote{इह बुद्धिनिब‚न्ध‚नोनुमेय इष्टः ।}}य}‚स्यातीन्द्रिय‚स्या‚{\color{DodgerBlue3}‚ग्र‚हाद‚क्षै}‚र‚क्ष‚भ‚वैर्व्विज्ञानैर्जातित‚द्व‚तोः ‚{\color{DodgerBlue3}‚स‚म्ब‚न्ध}‚स्या\edtext{}{\edlabel{pvv.162-2}\label{pvv.162-2}\lemma{स्या}\Bfootnote{स‚म्ब‚न्धाग्र‚हे त‚द्विशिष्टाग्र‚हात् ।}}‚{\tiny $_{lb}$}‚विशिष्ट‚प्र‚तीत्या‚{\color{DodgerBlue3}‚ऽद‚र्श‚नं स्थितं} निश्चितं य‚द्ब‚लेन य‚त्प्र‚तीतिस्त‚द‚ग्र‚हे न युक्ता सा ।
	\pend% ending standard par
      

	  \pstart \leavevmode% starting standard par
	य\edtext{}{\edlabel{pvv.162-3}\label{pvv.162-3}\lemma{य}\Bfootnote{स‚म‚वायास्तित्व‚माह ।}}दि नास्ति स‚म‚वाय‚स्त‚देह त‚न्तुषु प‚ट इत्याद‚यो बुद्ध‚यो न स्युरित्याह (।)
	\pend% ending standard par
      
	  \bigskip
	  \begingroup
	
	    \large
	  
	    \begin{quote}
	  
	    
	    \stanza[\smallbreak]
	\label{pv.2.149b}\flagstanza{\tiny\textenglish{...2.149b}}प‚ट‚स्त‚न्तुष्विहेत्यादिश‚ब्दाश्चेमे स्व‚यं कृताः ॥ १४९ ॥\&[\smallbreak]


	
	    \end{quote}
	  
	  \endgroup
	\textsuperscript{\textenglish{31b/MA}}

	  \pstart \leavevmode% starting standard par
	\hphantom{.}इह ‚{\color{DodgerBlue3}‚त‚न्तुषु प‚ट इत्यादि श‚ब्दाः‚{\tiny $_{7}$}‚ इमे स्व‚यं} स‚म‚यानुलोच‚नैः ‚{\color{DodgerBlue3}‚कृता} न व‚स्तुप‚रा‚{\tiny $_{lb}$}‚धीनाः । (१४९)
	\pend% ending standard par
      \label{div_pvv.2.150}
	  
	% new div opening: depth here is 2
	

	  \pstart \leavevmode% starting standard par
	त‚था\edtext{}{\edlabel{pvv.162-4}\label{pvv.162-4}\lemma{था}\Bfootnote{नापि लौकिका इत्याह ।}} हि (।)
	\pend% ending standard par
      
	  \bigskip
	  \begingroup
	
	    \large
	  
	    \begin{quote}
	  
	    
	    \stanza[\smallbreak]
	\label{pv.2.150a}\flagstanza{\tiny\textenglish{...2.150a}}शृङ्गं ग‚वीति लोके स्यात् शृङ्गे गौरित्य‚लौकिक‚म् ।\&[\smallbreak]


	
	    \end{quote}
	  
	  \endgroup
	

	  \pstart \leavevmode% starting standard par
	\hphantom{.}‚{\color{DodgerBlue3}‚शृङ्गं ग‚वि} तिष्ट‚{\color{DodgerBlue3}‚तीति लोके स्यात्} प्र‚माण‚प्र‚सिद्ध्योर‚नुरोधात् । ‚{\color{DodgerBlue3}‚शृङ्गे गौरिति ‚{\tiny $_{lb}$}‚तु} त‚दुप‚क‚ल्पित‚{\color{DodgerBlue3}‚म‚लौकिकं} प्र‚माण‚प्र‚सिद्धिब‚हिर्भावात् ।
	\pend% ending standard par
      

	  \begin{center}%% label @type='head'
	\textbf{(३) अव‚य‚विनिरासः}
	\end{center}
	

	  \pstart \leavevmode% starting standard par
	य‚द्य‚व‚य‚वेभ्यो न गौर्भिन्न‚स्त‚दा ग‚वि शृङ्ग‚मित्य‚पि न स्यादित्याह\edtext{}{\edlabel{pvv.162-5}\label{pvv.162-5}\lemma{स्यादित्याह}\Bfootnote{अतीन्द्रिय‚त्वान्न स‚म‚वायाद‚यं व्य‚प‚देशः किन्तु ।}} (।)
	\pend% ending standard par
      
	  \bigskip
	  \begingroup
	
	    \large
	  
	    \begin{quote}
	  
	    
	    \stanza[\smallbreak]
	\label{pv.2.150b}\flagstanza{\tiny\textenglish{...2.150b}}ग‚वाख्य‚प‚रिशिष्टाङ्ग‚विच्छेदानुप‚ल‚म्भ‚नात् ॥ १५० ॥\&[\smallbreak]


	
	    \end{quote}
	  
	  \endgroup
	

	  \pstart \leavevmode% starting standard par
	\hphantom{.}शृङ्ग‚स्य ‚{\color{DodgerBlue3}‚ग‚वाख्यैः प‚रिशिष्टा\edtext{}{\edlabel{pvv.162-6}\label{pvv.162-6}\lemma{रिशिष्टा}\Bfootnote{सास्नाद्यैः ।}}ङ्गैर्व्विच्छेदेस्य} वियोग‚स्या‚{\color{DodgerBlue3}‚नुप‚ल‚म्भ‚नात्} शृङ्गं ‚{\tiny $_{lb}$}‚ग‚वीत्युच्य‚ते न त्व‚व‚य‚वातिरिक्त‚गोस‚द्भावात् । (१५०)
	\pend% ending standard par
      \label{div_pvv.2.151}
	  
	% new div opening: depth here is 2
	

	  \pstart \leavevmode% starting standard par
	न‚नु त‚न्तुषु प‚ट‚{\tiny $_{1}$}‚ इति भ‚व‚त्येव प्र‚तीतिरिति चेत् । आह (।)
	\pend% ending standard par
      
	  \bigskip
	  \begingroup
	
	    \large
	  
	    \begin{quote}
	  
	    
	    \stanza[\smallbreak]
	\label{pv.2.151}\flagstanza{\tiny\textenglish{....2.151}}तैस्त‚न्तुभिरियं शाटीत्युत्त‚रं कार्य‚मुच्य‚ते ।&त‚न्तुसंस्कार‚स‚म्भूतं नैक‚कालं क‚थ‚ञ्च‚न ॥ १५१ ॥\&[\smallbreak]


	
	    \end{quote}
	  
	  \endgroup
	

	  \pstart \leavevmode% starting standard par
	\hphantom{.}‚{\color{DodgerBlue3}‚तैस्त‚न्तुभिः} प‚टाव‚स्थाप्राग्भाविभि‚{\color{DodgerBlue3}‚रियं शाटीति} कार‚ण‚भूत‚त‚न्तूत्त‚र‚काल‚भावि‚{\tiny $_{lb}$}‚‚{\color{DodgerBlue3}‚त‚न्तू}‚नां ‚{\color{DodgerBlue3}‚सँस्कार}‚स्तुरीवेम‚कुविन्द‚क‚रादिस‚ह‚कारी प्र‚भ‚वाव‚स्थाविशेष‚लाभः । ‚{\tiny $_{lb}$}‚त‚स्मात्स्व‚र‚सेन निरुध्य‚मानात्सं‚{\color{DodgerBlue3}‚भूतंकार्य‚मुच्य‚ते} । न तु त‚न्तुभिः स‚है‚{\color{DodgerBlue3}‚क‚कालं} कार्यं ‚{\tiny $_{lb}$}‚\leavevmode\ledsidenote{\textenglish{163/s}} त‚न्तुप‚ट इति ‚{\color{DodgerBlue3}‚क‚थ‚ञ्च‚न क‚थ्य‚ते} कार्य‚कार‚ण‚योः स‚म\edtext{}{\edlabel{pvv.163-1}\label{pvv.163-1}\lemma{म}\Bfootnote{स‚म‚वाय‚स्तु स‚म‚काल‚योरेव ।}}काल‚त्वाभावात् (। १५१)
	\pend% ending standard par
      \label{div_pvv.2.152}
	  
	% new div opening: depth here is 2
	

	  \pstart \leavevmode% starting standard par
	य‚दि नास्ति त‚न्तुप‚ट‚योर्भेद‚स्त‚दा क‚थ‚मे‚{\tiny $_{2}$}‚ते त‚न्त‚वः प‚ट‚श्चाय‚मिति व्य‚प‚देश ‚{\tiny $_{lb}$}‚इत्याह (।)
	\pend% ending standard par
      
	  \bigskip
	  \begingroup
	
	    \large
	  
	    \begin{quote}
	  
	    
	    \stanza[\smallbreak]
	\label{pv.2.152}\flagstanza{\tiny\textenglish{....2.152}}कार‚णारोप‚तः क‚श्चित् एकापोद्धार‚तोपि वा ।&त‚न्त्वाख्यां व‚र्त्त‚येत् कार्ये द‚र्श‚य‚न् नाश्र‚यं श्रुतेः ॥ १५२ ॥\&[\smallbreak]


	
	    \end{quote}
	  
	  \endgroup
	

	  \pstart \leavevmode% starting standard par
	\hphantom{.}‚{\color{DodgerBlue3}‚कार‚णा}‚नान्त‚न्तूनामा‚{\color{DodgerBlue3}‚रोप‚तः} । ‚{\color{DodgerBlue3}‚एक}‚स्य त‚न्तोर‚{\color{DodgerBlue3}‚पोद्धार‚तो} बुद्ध्या निःक‚र्ष‚णात् ‚{\tiny $_{lb}$}‚वा ‚{\color{DodgerBlue3}‚क‚श्चिद्} व्य‚व‚ह‚र्त्ता कार्ये प‚टे ‚{\color{DodgerBlue3}‚त‚न्त्वाख्यां} त‚न्तुश्रुतिं ‚{\color{DodgerBlue3}‚व‚र्त‚येत्} । प‚ट‚श्रुतेराश्र‚यं ‚{\tiny $_{lb}$}‚कार‚णं ‚{\color{DodgerBlue3}‚द‚र्श‚येन् न} त‚न्तुभ्यो व्य‚तिरिक्तः ‚{\color{DodgerBlue3}‚प‚टोस्ति} एव‚माकार‚प‚रिण‚तास्त‚न्त‚वः प‚ट ‚{\tiny $_{lb}$}‚इत्य‚र्थः । (१५२)
	\pend% ending standard par
      \label{div_pvv.2.153}
	  
	% new div opening: depth here is 2
	

	  \pstart \leavevmode% starting standard par
	य‚दि त‚न्त‚वः केव‚ला न तेभ्यः प‚टोऽन्य‚स्त‚दा प‚ट‚व्य‚प‚देशो निर्निब‚न्ध‚नः ‚{\tiny $_{lb}$}‚स्यादित्याह (।)
	\pend% ending standard par
      
	  \bigskip
	  \begingroup
	
	    \large
	  
	    \begin{quote}
	  
	    
	    \stanza[\smallbreak]
	\label{pv.2.153a}\flagstanza{\tiny\textenglish{...2.153a}}उप‚कार्योप‚कारित्वं विच्छेदाद् दृष्टिरेव वा ।\&[\smallbreak]


	
	    \end{quote}
	  
	  \endgroup
	

	  \pstart \leavevmode% starting standard par
	\hphantom{.}त‚न्तूनां प‚र‚स्प‚रं शीताद्य‚प‚नोद‚क्ष‚मं साहित्य‚{\color{DodgerBlue3}‚मुप‚कार्योप‚कारित्वं} । अन्योन्य‚स्य ‚{\tiny $_{lb}$}‚‚{\color{DodgerBlue3}‚विच्छेदाद् दृष्टिरेव वा} प‚ट‚व्य‚प‚देश‚निब\edtext{}{\edlabel{pvv.163-2}\label{pvv.163-2}\lemma{निब}\Bfootnote{न स‚म‚वायः ।}}न्ध‚न‚मिति शेषः ॥
	\pend% ending standard par
      

	  \pstart \leavevmode% starting standard par
	य‚दि व्य‚तिरिक्तं व्य‚प‚देश‚निब‚न्ध‚नं नास्ति त‚दा प‚ट इति व्य‚प‚देशो न मुख्यः ‚{\tiny $_{lb}$}‚स्यात् । बा ही के गोव्य‚प‚देश‚व‚दित्याह (।)
	\pend% ending standard par
      
	  \bigskip
	  \begingroup
	
	    \large
	  
	    \begin{quote}
	  
	    
	    \stanza[\smallbreak]
	\label{pv.2.153b}\flagstanza{\tiny\textenglish{...2.153b}}मुख्यं य‚द‚स्ख‚ल‚ज्ज्ञान‚मादिसंकेत‚गोच‚रः ॥ १५३ ॥\&[\smallbreak]


	
	    \end{quote}
	  
	  \endgroup
	

	  \pstart \leavevmode% starting standard par
	\hphantom{.}‚{\color{DodgerBlue3}‚मुख्य}‚न्त‚दुच्य‚ते ‚{\color{DodgerBlue3}‚य‚दादिसंकेत}‚स्य ‚{\color{DodgerBlue3}‚गोच‚रो} न व्य‚तिरिक्त इत्येव प‚र‚स्प‚राविच्छेदा‚{\tiny $_{lb}$}‚व‚स्थेषु च प‚ट‚श्रुतेः संकेता‚{\color{DodgerBlue3}‚द‚स्ख‚ल}‚द्ग‚तिगोच‚र‚त्वान्मुख्य‚त्वं । \edtext{\textsuperscript{*}}{\edlabel{pvv.163-3}\label{pvv.163-3}\lemma{*}\Bfootnote{य‚था सास्नादिमान् गौः ।}}त‚स्मात्प्र‚त्य‚क्ष‚तो जाते‚{\tiny $_{lb}$}‚र‚नुप‚ल‚म्भाद्योज‚नाविर‚हः । (१५३)
	\pend% ending standard par
      \label{div_pvv.2.154}
	  
	% new div opening: depth here is 2
	

	  \begin{center}%% label @type='head'
	\textbf{(४) नानुमान‚तः सामान्य‚सिद्धिः}
	\end{center}
	

	  \pstart \leavevmode% starting standard par
	स्यादेत‚त् (।) विशिष्ट‚प्र‚तीतिर्व्विशेष‚ण‚प्र‚तीतिपूर्व्विका य‚था द‚ण्डिप्र‚तीतिः । ‚{\tiny $_{lb}$}‚विशिष्ट‚प्र‚तीतिश्च शाव‚लेयादिषु गौरिति विशेष‚ण‚ञ्च शाव‚लेयादिषु गोत्व‚मे‚{\tiny $_{lb}$}‚वेत्य‚नुमान‚तो जातिसिद्धिरित्याह (।)
	\pend% ending standard par
      \textsuperscript{\textenglish{164/s}}
	  \bigskip
	  \begingroup
	
	    \large
	  
	    \begin{quote}
	  
	    
	    \stanza[\smallbreak]
	\label{pv.2.154a}\flagstanza{\tiny\textenglish{...2.154a}}अनुमान‚ञ्च जात्यादौ व‚स्तुनो नास्ति भेदिनि ।\&[\smallbreak]


	
	    \end{quote}
	  
	  \endgroup
	

	  \pstart \leavevmode% starting standard par
	\hphantom{.}‚{\color{DodgerBlue3}‚अनुमानं व‚स्तुनः} शाब‚लेयादे‚{\color{DodgerBlue3}‚र्भेदिनि जात्यादौ नास्ति} । न हि व्य‚क्तिव्य‚तिरिक्तं ‚{\tiny $_{lb}$}‚विशेष‚ण‚मुप‚ल‚भ्य‚ते शृङ्गा‚{\tiny $_{5}$}‚द्य‚व‚य‚व‚स‚न्निवेश एव त्व‚भिन्नो विशेष‚ण‚म‚स्तु । त‚था च ‚{\tiny $_{lb}$}‚नाभिम‚त‚सिद्धिः ।
	\pend% ending standard par
      

	  \pstart \leavevmode% starting standard par
	दृष्टान्तासिद्धिम‚प्याह (।)
	\pend% ending standard par
      
	  \bigskip
	  \begingroup
	
	    \large
	  
	    \begin{quote}
	  
	    
	    \stanza[\smallbreak]
	\label{pv.2.154b}\flagstanza{\tiny\textenglish{...2.154b}}स‚र्व्व‚त्र व्य‚प‚देशो हि द‚ण्डादेर‚पि सांवृतात् ॥ १५४ ॥\&[\smallbreak]


	
	    \end{quote}
	  
	  \endgroup
	

	  \pstart \leavevmode% starting standard par
	\hphantom{.}‚{\color{DodgerBlue3}‚स‚र्व्व‚त्र पुरुषादौ} द‚ण्डीत्यादि‚{\color{DodgerBlue3}‚व्य‚प‚देशो}‚पि ‚{\color{DodgerBlue3}‚हि} न ‚{\color{DodgerBlue3}‚द‚ण्डादे}‚र्व्व‚स्तुनः किन्तु ‚{\color{DodgerBlue3}‚सांवृतात्\edtext{}{\edlabel{pvv.164-1}\label{pvv.164-1}\lemma{सांवृतात्}\Bfootnote{द‚ण्ड‚द‚ण्डिनोः प‚र‚स्प‚रोप‚कार्योप‚कार‚क‚भावोप‚क‚ल्पिताद‚व‚स्थाविशेषाद‚र्थान्त‚र‚भूतात् ।}}} । ‚{\tiny $_{lb}$}‚द‚ण्ड‚स्व‚ल‚क्ष‚ण‚स्य व्युप‚देशे हेतुत्वे स‚र्व्व‚त्र पुरुषे स्यात् । स‚म्ब‚न्धिन्येवान्य‚त्रेति चेत् । ‚{\tiny $_{lb}$}‚त‚र्हि स‚म्ब‚न्धो द‚ण्डः कार‚णं द‚ण्डि व्य‚प‚देश‚स्य स‚म्ब‚न्ध‚श्च संयोगादिर्नास्तीति प‚रं ‚{\tiny $_{lb}$}‚कार्य‚कार‚ण‚भावः प‚रिशिष्य‚ते । त‚स्य निमित्त‚त्वे य‚था‚{\tiny $_{6}$}‚ द‚ण्डी पुरुष‚स्त‚था पुरुषी द‚ण्ड ‚{\tiny $_{lb}$}‚इत्य‚पि स्यात् । स‚मान‚त्वान्निमित्त‚स्य (।) त‚स्मात् क‚ल्पित‚विशेष‚ण‚भाव‚निय‚मो ‚{\tiny $_{lb}$}‚द‚ण्डः स‚म्ब‚न्धी सांवृत एव विशेष‚णं । (१५४)
	\pend% ending standard par
      \label{div_pvv.2.155}
	  
	% new div opening: depth here is 2
	

	  \pstart \leavevmode% starting standard par
	किञ्च (।)
	\pend% ending standard par
      
	  \bigskip
	  \begingroup
	
	    \large
	  
	    \begin{quote}
	  
	    
	    \stanza[\smallbreak]
	\label{pv.2.155}\flagstanza{\tiny\textenglish{....2.155}}व‚स्तुप्रासाद‚मालादिश‚ब्दाश्चान्यान‚पेक्षिणः ।&गेहो य‚द्य‚पि संयोग‚स्त‚न्माला किन्तु त‚द्भ‚वेत् ॥ १५५ ॥\&[\smallbreak]


	
	    \end{quote}
	  
	  \endgroup
	

	  \pstart \leavevmode% starting standard par
	\hphantom{.}ष‚ट्सु प‚दार्थेषु ‚{\color{DodgerBlue3}‚व‚स्तु} व‚स्त्विति स‚र्व्वानुयायी श‚ब्दः प्रासादेषु ‚{\color{DodgerBlue3}‚प्रासाद‚माले}‚ति ‚{\tiny $_{lb}$}‚श‚ब्दो गृहे\edtext{}{\edlabel{pvv.164-2}\label{pvv.164-2}\lemma{गृहे}\Bfootnote{आदिना ।}}षु ब‚हुषु न‚ग‚र‚मित्या‚{\color{DodgerBlue3}‚दिश‚ब्दाश्चान्यान‚पेक्षिणो}‚ऽर्थान्त‚र‚भूत‚विशेष‚ण‚र‚हिता ‚{\tiny $_{lb}$}‚\leavevmode\ledsidenote{\textenglish{32a/MA}} इति व्य‚भिचारिता\edtext{}{\edlabel{pvv.164-3}\label{pvv.164-3}\lemma{भिचारिता}\Bfootnote{अनैकान्तिक‚ता ।}} हेतोः । न हि प‚दार्थेषु व्य‚तिरिक्तं सामा\edtext{}{\edlabel{pvv.164-4}\label{pvv.164-4}\lemma{सामा}\Bfootnote{व‚स्तुव‚स्त्विति ।}}न्यं व‚स्तुतात्व‚{\tiny $_{7}$}‚म‚भ्युप‚{\tiny $_{lb}$}‚ग‚म्य‚ते वै शे षि कैः । न च प्रासादो द्र‚व्यं विजातीयानां द्र‚व्यानार‚म्भात् । त‚त‚श्च ‚{\tiny $_{lb}$}‚मालागु\edtext{}{\edlabel{pvv.164-5}\label{pvv.164-5}\lemma{मालागु}\Bfootnote{न द्र‚व्यं ।}}णोपि ‚{\color{DodgerBlue3}‚गेहो\edtext{}{\edlabel{pvv.164-6}\label{pvv.164-6}\lemma{गेहो}\Bfootnote{अर्थान्त‚र‚संयोगाभावाद‚भ्युप‚ग‚म्योच्य‚ते ।}} य‚द्य‚पि संयोग‚स्त}‚स्य ‚{\color{DodgerBlue3}‚माला किन्तु त‚द् भ‚वे\edtext{}{\edlabel{pvv.164-7}\label{pvv.164-7}\lemma{वे}\Bfootnote{व‚स्त्व‚न‚न्त‚र्भा(वा)न्न किञ्चित् ।}}त्} । न भाव‚गुणो ‚{\tiny $_{lb}$}‚निर्गुण‚त्वात् गुणानां । (१५५)
	\pend% ending standard par
      \label{div_pvv.2.156}
	  
	% new div opening: depth here is 2
	\textsuperscript{\textenglish{165/s}}

	  \begin{center}%% label @type='head'
	\textbf{क. सामान्य‚स्वीकारे दोषः}
	\end{center}
	
	  \bigskip
	  \begingroup
	
	    \large
	  
	    \begin{quote}
	  
	    
	    \stanza[\smallbreak]
	\label{pv.2.156}\flagstanza{\tiny\textenglish{....2.156}}जातिश्चेद् गेह एकोपि मालेत्युच्येत वृक्ष‚व‚त् ।&मालाब‚हुत्वे त‚च्छ‚ब्दः क‚थं जातेर‚जातितः ॥ १५६ ॥\&[\smallbreak]


	
	    \end{quote}
	  
	  \endgroup
	

	  \pstart \leavevmode% starting standard par
	\hphantom{.}‚{\color{DodgerBlue3}‚जातिश्चेद}‚भ्युप‚ग‚म्य‚ते ‚{\color{DodgerBlue3}‚एको गेहो मालेत्युच्येत वृक्ष‚व‚त्} । य‚था वृक्ष‚त्व‚जाति‚{\tiny $_{lb}$}‚योगादेको वृक्षो वृक्ष इत्युच्य‚ते एव‚मेकोपि गेहो माला स्यात् । त‚था गेह‚मालानां ‚{\tiny $_{lb}$}‚ब‚हुत्वे ‚{\color{DodgerBlue3}‚त‚च्छ‚ब्दो} मालेत्य‚नुगा‚{\tiny $_{1}$}‚मिश‚ब्दः ‚{\color{DodgerBlue3}‚क‚थं जाते\edtext{}{\edlabel{pvv.165-1}\label{pvv.165-1}\lemma{जाते}\Bfootnote{निःसामान्यानि सामान्यानीति व‚च‚नात् ।}}} र्मालायां ‚{\color{DodgerBlue3}‚अजातितो} जात्य‚न्त‚र‚{\tiny $_{lb}$}‚विर‚हात् । (१५६)
	\pend% ending standard par
      \label{div_pvv.2.157}
	  
	% new div opening: depth here is 2
	

	  \pstart \leavevmode% starting standard par
	किञ्च (।)
	\pend% ending standard par
      
	  \bigskip
	  \begingroup
	
	    \large
	  
	    \begin{quote}
	  
	    
	    \stanza[\smallbreak]
	\label{pv.2.157}\flagstanza{\tiny\textenglish{....2.157}}मालादौ च म‚ह‚त्वादिरिष्टो य‚श्चौप‚चारिकः ।&मुख्याविशिष्ट‚विज्ञान‚ग्राह्य‚त्वान्नौप‚चारिकः ॥ १५७ ॥\&[\smallbreak]


	
	    \end{quote}
	  
	  \endgroup
	

	  \pstart \leavevmode% starting standard par
	म‚ह‚ती प्रासाद‚मालेति क‚थं व्य‚प‚देशः । म‚ह‚त्वं प‚रिमाणं गुणः । त‚स्य मालायां ‚{\tiny $_{lb}$}‚न स‚त्त्वं(।) न हि प्रासाद‚माला किञ्चिदित्युक्तं । नापि संयोग‚ल‚क्ष‚णे प्रासादे ‚{\tiny $_{lb}$}‚म‚ह‚त्त्वं निर्गुण‚त्वा(त्) गुणानां । काष्ठादिषु द्र‚व्येषु प्रासादार‚म्भ‚केषु म‚ह‚त्त्व‚स‚त्त्वा‚{\tiny $_{lb}$}‚द्वृक्षेषु कुसुम‚स‚म्भ‚वाद्\edtext{}{\edlabel{pvv.165-2}\label{pvv.165-2}\lemma{वाद्}\Bfootnote{अव‚य‚व्य‚भावाद्व‚न‚संख्या ।}} व‚न‚संख्याल‚क्ष‚णं कुसुमित‚मिति य‚थोच्य‚ते । त‚था प्रासाद‚{\tiny $_{lb}$}‚‚{\color{DodgerBlue3}‚माला‚{\tiny $_{2}$}‚दौ म‚ह‚त्वादिरौप‚चारिको} य‚श्चेष्टः स चायुक्तः काष्ठादिष्व‚पि तादृश‚स्य ‚{\tiny $_{lb}$}‚म‚ह\edtext{}{\edlabel{pvv.165-3}\label{pvv.165-3}\lemma{ह}\Bfootnote{प्र‚त्य‚व‚य‚देषु ।}}त्त्व‚स्याभावात् ।
	\pend% ending standard par
      

	  \pstart \leavevmode% starting standard par
	किञ्च(।) म‚हान् प‚र्व्व‚त इति\edtext{}{\edlabel{pvv.165-4}\label{pvv.165-4}\lemma{इति}\Bfootnote{अस्म‚न्म‚ते ।}} ‚{\color{DodgerBlue3}‚मु}‚ख्य‚म‚ह‚त्त्व‚ग्राह‚क‚ज्ञानेनास्ख‚ल\edtext{}{\edlabel{pvv.165-5}\label{pvv.165-5}\lemma{ल}\Bfootnote{मुख्य‚विष‚यौ यौ श‚ब्द‚प्र‚त्य‚यौ ताभ्याम‚विशिष्टो यः प्र‚त्य‚य‚स्तेन ग्राह्य एताव‚न्मात्र‚निमित्त‚त्वान्मुख्य‚त्व‚व्य‚व‚स्थायाः ।}}द्वृत्तित्वाद‚वि‚{\tiny $_{lb}$}‚शिष्टेन ज्ञानेन ग्राह्य‚त्वात् प्रासाद‚मालाम‚ह‚त्वादि‚{\color{DodgerBlue3}‚रौप‚चारिको} न युक्तः (।) ‚{\tiny $_{lb}$}‚न हि माण‚व‚क इव सिंह‚बुद्धिर्म‚ह‚त्त्व‚बुद्धिश्च मालायां स्ख‚ल‚ति । (१५७)
	\pend% ending standard par
      \label{div_pvv.2.158}
	  
	% new div opening: depth here is 2
	

	  \pstart \leavevmode% starting standard par
	\hphantom{.}किञ्च (।) भिन्न‚विशेष‚णं ‚{\color{DodgerBlue3}‚मुख्य}‚म‚भिन्न‚विशेष‚ण‚ञ्चामुख्य‚मिति य‚दु‚{\tiny $_{3}$}‚च्य‚ते ‚{\tiny $_{lb}$}‚त‚द‚प्य‚युक्त‚मित्याह (।)
	\pend% ending standard par
      
	  \bigskip
	  \begingroup
	
	    \large
	  
	    \begin{quote}
	  
	    
	    \stanza[\smallbreak]
	\label{pv.2.158}\flagstanza{\tiny\textenglish{....2.158}}अन‚न्य‚हेतुता तुल्या सा मुख्याभिम‚तेष्व‚पि ॥&प‚दार्थ‚श‚ब्दः कं हेतुम‚न्यं ष‚ट्कं स‚मीक्ष‚ते ॥ १५८ ॥\&[\smallbreak]


	
	    \end{quote}
	  
	  \endgroup
	

	  \pstart \leavevmode% starting standard par
	\hphantom{.}‚{\color{DodgerBlue3}‚मुख्याभिम‚तेष्व‚पि} द‚ण्ड्यादिष्व‚{\color{DodgerBlue3}‚न‚न्य‚हेतुता} भिन्न‚विशेष‚ण‚निमित्त‚र‚हित‚ता ‚{\color{DodgerBlue3}‚तुल्या} गौणेन \edtext{}{\edlabel{pvv.165-6}\label{pvv.165-6}\lemma{गौणेन}\Bfootnote{स‚ह मुख्येषु ।}}स‚र्व्व‚त्र व्य‚प‚देशो हि द‚ण्डा\edtext{}{\edlabel{pvv.165-7}\label{pvv.165-7}\lemma{ण्डा}\Bfootnote{अस‚त एव विशेष‚ण‚त्वेनोप‚न‚यात् ।}}देर‚पि सांवृतादि(२।१५४)त्युक्तेः । किञ्चा‚{\tiny $_{lb}$}‚\leavevmode\ledsidenote{\textenglish{166/s}} नुयायी ‚{\color{DodgerBlue3}‚प‚दार्थ‚श‚ब्दः} ष‚ष्ठ्या प‚दार्थेषु ‚{\color{DodgerBlue3}‚क‚म‚न्यं हेतुं} निमित्तं प्र‚वृत्तौ ‚{\color{DodgerBlue3}‚स‚मीक्ष्य‚ते} (? क्ष‚ते)\edtext{\textsuperscript{*}}{\edlabel{pvv.166-1}\label{pvv.166-1}\lemma{*}\Bfootnote{पूर्व्व‚म‚नुमादूष‚णे उक्तेधुना मुख्योप‚चार‚ल‚क्ष‚ण‚दूष‚णे ।}}। ‚{\tiny $_{lb}$}‚न हि ष‚ट्प‚दार्थातिरिक्तं किञ्चिद‚स्ति (।) (१५८)
	\pend% ending standard par
      \label{div_pvv.2.159}
	  
	% new div opening: depth here is 2
	
	  \bigskip
	  \begingroup
	
	    \large
	  
	    \begin{quote}
	  
	    
	    \stanza[\smallbreak]
	\label{pv.2.159a}\flagstanza{\tiny\textenglish{...2.159a}}यो य‚था रूढितः सिद्धः त‚त्साम्याद्य‚स्त‚थोच्य‚ते ।\&[\smallbreak]


	
	    \end{quote}
	  
	  \endgroup
	

	  \pstart \leavevmode% starting standard par
	त‚स्माद्योऽर्थो येन प्र‚कारेण रूढितः आदिसंकेतानुसारेण सिद्धः स मुख्य‚{\tiny $_{4}$}‚ः । ‚{\tiny $_{lb}$}‚य‚श्च त‚स्य मुख्य‚स्य साम्यात्त‚थामुख्य‚वाच‚क‚श‚ब्देनोच्य‚ते स गौणः (।)
	\pend% ending standard par
      

	  \pstart \leavevmode% starting standard par
	कुत एत‚दित्याह (।)
	\pend% ending standard par
      
	  \bigskip
	  \begingroup
	
	    \large
	  
	    \begin{quote}
	  
	    
	    \stanza[\smallbreak]
	\label{pv.2.159b}\flagstanza{\tiny\textenglish{...2.159b}}य‚त्र गौण‚श्च भावेष्य‚प्य‚भाव‚स्योप‚चार‚तः ॥ १५९ ॥\&[\smallbreak]


	
	    \end{quote}
	  
	  \endgroup
	

	  \pstart \leavevmode% starting standard par
	\hphantom{.}‚{\color{DodgerBlue3}‚भावेष्व‚पि} कुपुत्रादिषु पुत्रादिरित्य‚भावो‚{\color{DodgerBlue3}‚प‚चार‚तः} । प‚र‚म‚ते तु भावे वृत्त‚त्वा‚{\tiny $_{lb}$}‚न्मुख्य‚त्वं भ‚वेत् । (१५९)
	\pend% ending standard par
      \label{div_pvv.2.160}
	  
	% new div opening: depth here is 2
	

	  \begin{center}%% label @type='head'
	\textbf{ख. विव‚क्षान्व‚यिसंकेतानुग‚म‚त्वाद् रूढेः}
	\end{center}
	

	  \pstart \leavevmode% starting standard par
	स्यादेत‚द् (।) रूढ्यैव मुख्य‚ता किन्तु सापि भिन्ने विशेष‚णे स‚तीत्याह (।) ‚{\tiny $_{lb}$}‚न (।)
	\pend% ending standard par
      
	  \bigskip
	  \begingroup
	
	    \large
	  
	    \begin{quote}
	  
	    
	    \stanza[\smallbreak]
	\label{pv.2.160}\flagstanza{\tiny\textenglish{....2.160}}संकेतान्व‚यिनी रूढिर्व‚क्तुरिच्छान्व‚यी च सः ।&क्रिय‚ते व्य‚व‚हारार्थ छ‚न्दः श‚ब्दांश‚नाभ‚व‚त् ॥ १६० ॥\&[\smallbreak]


	
	    \end{quote}
	  
	  \endgroup
	

	  \pstart \leavevmode% starting standard par
	\hphantom{.}‚{\color{DodgerBlue3}‚संकेतान्व‚यिनी} य‚थासंकेतं ‚{\color{DodgerBlue3}‚रूढिः स च} संकेतो ‚{\color{DodgerBlue3}‚व‚क्तुः\edtext{}{\edlabel{pvv.166-2}\label{pvv.166-2}\lemma{क्तुः}\Bfootnote{नान्त‚र‚निमित्तः ।}}} संकेत‚यितु‚{\color{DodgerBlue3}‚रिच्छान्व‚यी ‚{\tiny $_{lb}$}‚व्य\edtext{}{\edlabel{pvv.166-3}\label{pvv.166-3}\lemma{व्य}\Bfootnote{एवं त‚र्हि किम‚र्थं क्रिय‚ते आह ।}}व‚हारार्थं क्रिय‚ते । छ‚न्द‚सो} गाय‚त्र्या\edtext{}{\edlabel{pvv.166-4}\label{pvv.166-4}\lemma{त्र्या}\Bfootnote{अर्थान्त‚र‚विशेष‚ण‚म्विना वृत्तेष्व‚न‚ष्ट‚वादिनाम‚व‚त् ।}}देः ‚{\color{DodgerBlue3}‚श‚ब्दां\edtext{}{\edlabel{pvv.166-5}\label{pvv.166-5}\lemma{ब्दां}\Bfootnote{अव‚य‚व ।}}}‚श‚स्य प्र‚कृतिप्र‚त्य‚{\tiny $_{5}$}‚यादे‚{\tiny $_{lb}$}‚‚{\color{DodgerBlue3}‚र्नाम‚व‚त्} । न हि विशिष्टानुपूर्व्वीकेषु व‚र्ण्णेषु पृथ‚ग्भूतं गाय‚त्र्यादिशूब्द‚निमित्तं ‚{\tiny $_{lb}$}‚किञ्चिद‚स्ति । श‚ब्दांशेषु वाऽपि तु संकेत‚यितुरिच्छानुरोधादेव त‚था व्य‚प‚देशः । ‚{\tiny $_{lb}$}‚(१६०)
	\pend% ending standard par
      \label{div_pvv.2.161}
	  
	% new div opening: depth here is 2
	

	  \pstart \leavevmode% starting standard par
	य‚दि व्य‚क्तिभ्यो न भिन्नं सामान्यं त‚दा क‚थ‚म‚नुगामी प्र‚त्य‚य इत्याह (।)
	\pend% ending standard par
      
	  \bigskip
	  \begingroup
	
	    \large
	  
	    \begin{quote}
	  
	    
	    \stanza[\smallbreak]
	\label{pv.2.161}\flagstanza{\tiny\textenglish{....2.161}}व‚स्तुध‚र्म‚त‚यैवार्थास्तादृग्विज्ञान‚कार‚ण‚म् ।&भेदेपि य‚त्र त‚ज्ज्ञानांत्तान्त‚था प्र‚तिप‚द्य‚ते ॥ १६१ ॥\&[\smallbreak]


	
	    \end{quote}
	  
	  \endgroup
	\textsuperscript{\textenglish{167/s}}

	  \pstart \leavevmode% starting standard par
	\hphantom{.}‚{\color{DodgerBlue3}‚व‚स्तुध‚र्म‚त‚या} प्र‚कृ‚{\color{DodgerBlue3}‚त्यैव} केचि‚{\color{DodgerBlue3}‚द‚र्थाः} प‚र‚स्प‚रं ‚{\color{DodgerBlue3}‚भेदेपि तादृ}‚श‚स्यानुगामिनोऽत‚त्कार्यं‚{\tiny $_{lb}$}‚व्यावृत्तिविष‚य‚स्य ‚{\color{DodgerBlue3}‚विज्ञान‚स्य कार‚णं} । य‚त्र येष्व‚र्थेष्व‚नुगामि ज्ञानं तान‚भेदिनो‚{\tiny $_{5}$}‚‚{\tiny $_{lb}$}‚ऽर्थान् ‚{\color{DodgerBlue3}‚त‚था} एक‚त्वे‚{\color{DodgerBlue3}‚न प्र‚तिप‚द्य‚ते} न त्वेक‚सामान्यं (? न्य-) ब‚लात्त‚था ज्ञानं । (१६१)
	\pend% ending standard par
      \label{div_pvv.2.162}
	  
	% new div opening: depth here is 2
	

	  \pstart \leavevmode% starting standard par
	स्यादेत‚त् । प्र‚तिव्य‚क्ति ज्ञानान्य‚पि भिन्नानीति क‚थ‚म‚नुगामि ज्ञान‚मित्याह (।)
	\pend% ending standard par
      
	  \bigskip
	  \begingroup
	
	    \large
	  
	    \begin{quote}
	  
	    
	    \stanza[\smallbreak]
	\label{pv.2.162}\flagstanza{\tiny\textenglish{....2.162}}ज्ञानान्य‚पि त‚था भेदेऽभेद‚प्र‚त्य‚व‚म‚र्श‚ने ।&इत्य‚त‚त्कार्य‚विश्लेष‚स्यान्व‚यो नैक‚व‚स्तुनः ॥ १६२ ॥\&[\smallbreak]


	
	    \end{quote}
	  
	  \endgroup
	

	  \pstart \leavevmode% starting standard par
	\hphantom{.}‚{\color{DodgerBlue3}‚ज्ञानान्य‚पि} प‚र‚स्प‚र‚तो ‚{\color{DodgerBlue3}‚भेदे त‚था}‚ऽर्थ‚व‚द्व‚स्तुध‚र्मित‚या‚{\color{DodgerBlue3}‚ऽभेद‚प्र‚त्य‚व‚म‚र्श‚ने} निमित्तं ‚{\tiny $_{lb}$}‚त‚तो ज्ञानान्य‚पि त‚देक‚प‚राम‚र्श‚गोच‚र‚त‚याऽनुगामिप्र‚त्य‚य उच्य‚न्ते इत्य‚नेन प्र‚कारेण ‚{\tiny $_{lb}$}‚भेदिष्व‚र्थेष्व‚त‚{\color{DodgerBlue3}‚त्कार्या}‚द‚र्थ‚स्य ‚{\color{DodgerBlue3}‚विश्लेषो} व्य‚व‚च्छेद‚स्त‚स्या‚{\color{DodgerBlue3}‚न्व‚यो} विद्य‚{\tiny $_{7}$}‚ते। ‚{\color{DodgerBlue3}‚न त्वेक‚स्य}\leavevmode\ledsidenote{\textenglish{32b/MA}} ‚{\tiny $_{lb}$}‚‚{\color{DodgerBlue3}‚व‚स्तुनः} सामान्य‚स्य (। १६२)
	\pend% ending standard par
      \label{div_pvv.2.163}
	  
	% new div opening: depth here is 2
	
	  \bigskip
	  \begingroup
	
	    \large
	  
	    \begin{quote}
	  
	    
	    \stanza[\smallbreak]
	\label{pv.2.163a}\flagstanza{\tiny\textenglish{...2.163a}}व‚स्तूनां विद्य‚ते त‚स्मात्त‚न्निष्ठा व‚स्तुनि श्रुतिः ।\&[\smallbreak]


	
	    \end{quote}
	  
	  \endgroup
	

	  \pstart \leavevmode% starting standard par
	\hphantom{.}‚{\color{DodgerBlue3}‚व\edtext{}{\edlabel{pvv.167-1}\label{pvv.167-1}\lemma{व}\Bfootnote{व‚स्तुषु ।}}स्तूनां} विशेषाणाम‚न्व‚यो ‚{\color{DodgerBlue3}‚विद्य‚ते}‚ऽनुप‚ल‚म्भ‚बाधित‚त्वात्त‚स्य । ‚{\color{DodgerBlue3}‚त‚स्मात् त‚न्निष्ठा} व्यावृत्तिविष‚या ‚{\color{DodgerBlue3}‚व‚स्तु}‚नि ‚{\color{DodgerBlue3}‚श्रुतिः} प्र‚व‚र्त‚ते ।
	\pend% ending standard par
      

	  \begin{center}%% label @type='head'
	\textbf{(५) अन्यापोह‚चिन्ता}
	\end{center}
	

	  \begin{center}%% label @type='head'
	\textbf{क. अत‚त्कार्य‚व्यावृत्तिः}
	\end{center}
	

	  \pstart \leavevmode% starting standard par
	\hphantom{.}न‚न्व‚त‚त्कार्य‚व्यावृत्तिर्व‚स्तुनः स्व‚भाव‚भूता त‚तो व्यावृत्तिविष‚य‚त्वे ‚{\color{DodgerBlue3}‚व‚स्तुविष‚य‚{\tiny $_{lb}$}‚तैव श‚ब्द‚स्य} स्यादित्याह\edtext{}{\edlabel{pvv.167-2}\label{pvv.167-2}\lemma{स्यादित्याह}\Bfootnote{न च श‚ब्दो व‚स्तुस्प‚र्शीत्य‚नेनास्य विरोधं म‚न्य‚ते ।}} (।)
	\pend% ending standard par
      
	  \bigskip
	  \begingroup
	
	    \large
	  
	    \begin{quote}
	  
	    
	    \stanza[\smallbreak]
	\label{pv.2.163b}\flagstanza{\tiny\textenglish{...2.163b}}बाह्य‚श‚क्तिव्य‚व‚च्छेद‚निष्ठाभावेपि त‚च्छ्रुतिः ॥ १६३ ॥\&[\smallbreak]


	
	    \end{quote}
	  
	  \endgroup
	

	  \pstart \leavevmode% starting standard par
	\hphantom{.}‚{\color{DodgerBlue3}‚बाह्य}‚स्य व‚स्तुनः ‚{\color{DodgerBlue3}‚श‚क्ते}‚र‚त‚त्कार्याद्यो ‚{\color{DodgerBlue3}‚व्य‚व‚च्छेद}‚स्त‚त्र ‚{\color{DodgerBlue3}‚निष्ठा} विष‚यित्वं त‚स्या‚{\tiny $_{lb}$}‚भावेपि ‚{\color{DodgerBlue3}‚त‚च्छ्रुति}‚र्व्य‚व‚च्छेद‚वाचिनी श्रु\edtext{}{\edlabel{pvv.167-3}\label{pvv.167-3}\lemma{श्रु}\Bfootnote{क‚थ‚मिति न वृत्तेन स‚म्ब‚न्ध‚नीयः ।}}तिः । (१६३)
	\pend% ending standard par
      \label{div_pvv.2.164}
	  
	% new div opening: depth here is 2
	
	  \bigskip
	  \begingroup
	
	    \large
	  
	    \begin{quote}
	  
	    
	    \stanza[\smallbreak]
	\label{pv.2.164}\flagstanza{\tiny\textenglish{....2.164}}विक‚ल्प‚प्र‚तिबिम्बेषु त‚न्निष्ठेषु निब‚ध्य‚ते ।&त‚तोन्यापोह‚निष्ठ‚त्वादुक्तान्यापोह‚कृच्छ्रुतिः ॥ १६४ ॥\&[\smallbreak]


	
	    \end{quote}
	  
	  \endgroup
	

	  \pstart \leavevmode% starting standard par
	\hphantom{.}विक‚{\color{DodgerBlue3}‚ल्पानां प्र‚तिबिम्बे}‚ष्वाकारेषु ‚{\color{DodgerBlue3}‚त‚न्निष्ठेषु} त‚द्व्यावृत्तिव‚स्तुत्वेन‚{\tiny $_{1}$}‚ व्य‚व‚स्था‚{\tiny $_{lb}$}‚विष‚य‚त‚या त‚द्व्य‚व‚हार‚व्य‚व‚स्थितिषु संके\edtext{}{\edlabel{pvv.167-4}\label{pvv.167-4}\lemma{संके}\Bfootnote{स्व‚ल‚क्ष‚ण‚विक‚ल्प‚विष‚य‚योरैक्य‚भ्रान्त्या ।}}त‚काले ‚{\color{DodgerBlue3}‚निब‚ध्य‚ते । त‚तो} विक‚ल्प‚प्र‚ति‚{\tiny $_{lb}$}‚\leavevmode\ledsidenote{\textenglish{168/s}} बिम्बानां बाह्य\edtext{}{\edlabel{pvv.168-1}\label{pvv.168-1}\lemma{बाह्य}\Bfootnote{व‚स्तुविवेक‚निष्ठ‚त्वात् ।}}व्यावृत्तात्म‚त्वेन व्य‚व‚हार‚विष‚य‚त्वाद‚न्या‚{\color{DodgerBlue3}‚पोह‚निष्ठ‚त्वा}‚त्कार‚णा‚{\tiny $_{lb}$}‚‚{\color{DodgerBlue3}‚दुक्ता श्रुतिर‚न्यापोह‚कृत्} । अन्य‚व्यावृत्ताकार‚विक‚ल्प‚ज‚न‚नात् अन्य‚व्यावृत्तेषु प्र‚व\edtext{}{\edlabel{pvv.168-2}\label{pvv.168-2}\lemma{व}\Bfootnote{दृश्य‚विक‚ल्प‚योरेकीक‚र‚णात् ।}}र्त‚{\tiny $_{lb}$}‚नाच्च श‚ब्दोऽन्यापोह‚कृदुक्तः । (१६४)
	\pend% ending standard par
      \label{div_pvv.2.165}
	  
	% new div opening: depth here is 2
	

	  \pstart \leavevmode% starting standard par
	न‚नु शाब्दे ज्ञाने ग्राह्यं बाह्य‚त‚यैव प्र‚तीय‚ते न ज्ञानाकार‚त‚येत्याह (।)
	\pend% ending standard par
      
	  \bigskip
	  \begingroup
	
	    \large
	  
	    \begin{quote}
	  
	    
	    \stanza[\smallbreak]
	\label{pv.2.165}\flagstanza{\tiny\textenglish{....2.165}}व्य‚तिरेकीव य‚ज्ज्ञाने भात्य‚र्थ‚प्र‚तिबिम्ब‚क‚म् ।&श‚ब्दात्त‚द‚पि नार्थात्मा भ्रान्तिः सा वास‚नोद्भ‚वा ॥ १६५ ॥\&[\smallbreak]


	
	    \end{quote}
	  
	  \endgroup
	

	  \pstart \leavevmode% starting standard par
	\hphantom{.}‚{\color{DodgerBlue3}‚श‚ब्दादु}‚त्प‚न्न‚{\color{DodgerBlue3}‚ज्ञानेऽर्थ‚प्र‚तिबिम्ब‚कं व्य‚तिरेकीव} भिन्नं बाह्य‚मिव ‚{\color{DodgerBlue3}‚य‚दाभा‚{\tiny $_{2}$}‚ति त‚द‚पि ‚{\tiny $_{lb}$}‚नार्थात्मा} ब‚हिर‚र्थ‚स्व‚रूपं किन्तु ‚{\color{DodgerBlue3}‚भ्रान्तिः सा वास‚ना\edtext{}{\edlabel{pvv.168-3}\label{pvv.168-3}\lemma{ना}\Bfootnote{स्व‚ल‚क्ष‚णानुभ‚व‚वास‚नाहेतु ।}}निर्मिता} । य‚था तैमिरिक‚{\tiny $_{lb}$}‚दृष्टेषु केशादिषु बाह्य‚भ्र‚मः । एवं विक‚ल्पाकारेपि बाह्य‚व्य‚व‚हारोऽविद्याव‚शा‚{\tiny $_{lb}$}‚दित्य‚र्थः । (१६५)
	\pend% ending standard par
      \label{div_pvv.2.166}
	  
	% new div opening: depth here is 2
	

	  \pstart \leavevmode% starting standard par
	ज्ञाना\edtext{}{\edlabel{pvv.168-4}\label{pvv.168-4}\lemma{ज्ञाना}\Bfootnote{साव‚तारः श्लोकः पूर्व्व‚प‚क्षः ।}}कार‚स्त‚र्हि व‚स्तुभूतो वाच्यः स्यादित्याह (।)
	\pend% ending standard par
      
	  \bigskip
	  \begingroup
	
	    \large
	  
	    \begin{quote}
	  
	    
	    \stanza[\smallbreak]
	\label{pv.2.166}\flagstanza{\tiny\textenglish{....2.166}}त‚स्याभिधाने श्रुतिभिर‚र्थे कोंशोव‚ग‚म्य‚ते ।&त‚स्याग‚तौ च संकेत‚क्रिया व्य‚र्था त‚द‚र्थिका ॥ १६६ ॥\&[\smallbreak]


	
	    \end{quote}
	  
	  \endgroup
	

	  \pstart \leavevmode% starting standard par
	\hphantom{.}‚{\color{DodgerBlue3}‚त‚स्य} ज्ञानाकार‚स्य ‚{\color{DodgerBlue3}‚श्रुतिभिर‚भिधानेऽर्थे}‚ऽत‚त्कार्य‚व्यावृत्ते श‚ब्देनाचोदिते कोंशो‚{\tiny $_{lb}$}‚व‚ग‚म्य‚ते न क‚श्चित्त‚द‚कार्य‚व्यावृ‚{\color{DodgerBlue3}‚त्त‚स्या}‚र्थ‚स्य श‚ब्दा‚{\color{DodgerBlue3}‚द‚ग‚तौ च} स‚त्यां ‚{\color{DodgerBlue3}‚संकेत‚क्रिया व्य‚र्था} य\edtext{}{\edlabel{pvv.168-5}\label{pvv.168-5}\lemma{य}\Bfootnote{कुत इत्याह ।}}स्मा‚{\color{DodgerBlue3}‚त्त‚द‚र्थिका} अत‚त्कार्य‚व्यावृत्तार्थ‚प्र‚तीतिफ‚ला से‚{\tiny $_{3}$}‚ष्य‚ते । (१६६)
	\pend% ending standard par
      \label{div_pvv.2.167}
	  
	% new div opening: depth here is 2
	

	  \pstart \leavevmode% starting standard par
	एव‚न्त‚र्ह्य‚न्यापोहेपि संकेते कृते प्र‚वृत्तिर‚र्थेषु न स्यात् । त‚स्यार्थात्म‚त्वाभा‚{\tiny $_{lb}$}‚वादित्याह ।\edtext{\textsuperscript{*}}{\edlabel{pvv.168-6}\label{pvv.168-6}\lemma{*}\Bfootnote{अत्राह सिद्धान्ती ।}}
	\pend% ending standard par
      
	  \bigskip
	  \begingroup
	
	    \large
	  
	    \begin{quote}
	  
	    
	    \stanza[\smallbreak]
	\label{pv.2.167}\flagstanza{\tiny\textenglish{....2.167}}श‚ब्दोर्थांशंक‚माहेति त‚त्रान्यापोह उच्य‚ते ।&आकारः स च नार्थेस्ति तं व‚द‚न्न‚र्थ‚भाक् क‚थ‚म् ॥ १६७ ॥\&[\smallbreak]


	
	    \end{quote}
	  
	  \endgroup
	

	  \pstart \leavevmode% starting standard par
	\hphantom{.}‚{\color{DodgerBlue3}‚श‚ब्दोऽर्थांशंक‚माहेति} प्र‚श्ने ‚{\color{DodgerBlue3}‚त‚त्रान्यापोहो}‚ऽत‚त्कार्य‚व्यावृत्तिः स‚र्व्व‚विशेष‚संभ‚{\tiny $_{lb}$}‚विनी वाच्य‚त‚यो‚{\color{DodgerBlue3}‚च्य‚ते} । अतोऽर्थांशात्म‚न्य‚न्यापोहे गृहीत‚संकेतः श‚ब्दादुच्च‚रितार्थं ‚{\tiny $_{lb}$}‚प्र‚तीत्य त‚त्र प्र‚व‚र्त्तित इति युक्तं । य‚स्त‚त्रा\edtext{}{\edlabel{pvv.168-7}\label{pvv.168-7}\lemma{त्रा}\Bfootnote{बाह्य‚नीलादिद‚र्श‚नाभ्यासायातो ज्ञानीयः ।}} ‚{\color{DodgerBlue3}‚क्षाकारः स चार्थे नास्ति तं} बुद्ध्याकारं ‚{\tiny $_{lb}$}‚‚{\color{DodgerBlue3}‚व‚द‚न् श‚ब्दोऽर्थ‚भाक्} बाह्यार्थाभिधायी क‚थ‚म‚स्तु । (१६७)
	\pend% ending standard par
      ‚{\tiny $_{lb}$}‚\label{div_pvv.2.168}
	  
	% new div opening: depth here is 2
	\textsuperscript{\textenglish{169/s}}

	  \pstart \leavevmode% starting standard par
	किञ्च (।)
	\pend% ending standard par
      
	  \bigskip
	  \begingroup
	
	    \large
	  
	    \begin{quote}
	  
	    
	    \stanza[\smallbreak]
	\label{pv.2.168}\flagstanza{\tiny\textenglish{....2.168}}श‚ब्द‚स्यान्व‚यिनः कार्य‚म‚र्थेनान्व‚यिना स च ।&अन‚न्व‚यी धियोऽभेदाद् द‚र्श‚नाभ्यास‚निर्म्म‚तः ॥ १६८ ॥\&[\smallbreak]


	
	    \end{quote}
	  
	  \endgroup
	

	  \pstart \leavevmode% starting standard par
	\hphantom{.}‚{\color{DodgerBlue3}‚श‚ब्द‚स्या\edtext{}{\edlabel{pvv.169-1}\label{pvv.169-1}\lemma{स्या}\Bfootnote{गौगौ/?/रिति ।}}न्व‚यि‚{\tiny $_{4}$}‚नो}‚ऽन्व‚यिना‚{\color{DodgerBlue3}‚र्थेन कार्यं} व्य‚व‚हार‚काले प्र‚तीतिल‚क्ष‚णं प्र‚योज‚नं । ‚{\tiny $_{lb}$}‚‚{\color{DodgerBlue3}‚स च} बुद्ध्याकारः स्व‚ल‚क्ष‚ण‚{\color{DodgerBlue3}‚द‚र्श‚नाभ्यासेन} वास‚ना‚{\color{DodgerBlue3}‚निर्म्मितोऽन\edtext{}{\edlabel{pvv.169-2}\label{pvv.169-2}\lemma{निर्म्मितोऽन}\Bfootnote{य‚त्र बुद्धौ भास‚ते त‚तोऽभिन्नः ज्ञान‚व‚त् ।}}न्व‚यी धियो}‚ऽन‚न्व‚{\tiny $_{lb}$}‚यिन्या ‚{\color{DodgerBlue3}‚अभेदात्} । (१६८)
	\pend% ending standard par
      \label{div_pvv.2.169}
	  
	% new div opening: depth here is 2
	

	  \pstart \leavevmode% starting standard par
	न‚नु य‚द्य‚र्थः श‚ब्द‚स्य न विष‚य‚स्त‚दा त‚दं श‚रूपोप्य‚न्यापोहः क‚थं वाच्य इत्याह (।)
	\pend% ending standard par
      
	  \bigskip
	  \begingroup
	
	    \large
	  
	    \begin{quote}
	  
	    
	    \stanza[\smallbreak]
	\label{pv.2.169}\flagstanza{\tiny\textenglish{....2.169}}त‚द्रूपारोप‚ग‚त्यान्य‚व्यावृत्ताधिग‚तेः पुनः ।&श‚ब्दार्थार्थः स एवेति व‚च‚ने न विरुध्य‚ते ॥ १६९ ॥\&[\smallbreak]


	
	    \end{quote}
	  
	  \endgroup
	

	  \pstart \leavevmode% starting standard par
	बुद्ध्या\edtext{}{\edlabel{pvv.169-3}\label{pvv.169-3}\lemma{बुद्ध्या}\Bfootnote{विनारोपं व्य‚व‚हाराभावात् य‚था संग‚तिस्त‚स्य त‚थाह ।}}कारे त‚द्रूप‚स्याप‚र्थांशापोह‚स्या‚{\color{DodgerBlue3}‚रोप‚ग‚त्या} एक‚त्वाध्य‚व‚सायेना‚{\color{DodgerBlue3}‚न्य‚व्यावृ‚{\tiny $_{lb}$}‚त्त‚स्या}‚र्थ\edtext{}{\edlabel{pvv.169-4}\label{pvv.169-4}\lemma{र्थ}\Bfootnote{स्व‚ल‚क्ष‚ण‚स्य ।}}स्या‚{\color{DodgerBlue3}‚धिग‚तेः श‚ब्दार्थां}‚शापोहः श‚ब्दार्थ उच्य‚ते । न तु सामान्याच्छ‚ब्दाद‚{\tiny $_{5}$}‚र्थ‚{\tiny $_{lb}$}‚प्र‚तीतेः । य‚दि पुन‚र्बुद्ध्याकार‚स्य व्यावृत्तार्थ‚त्वेन प्र‚तीतेः स बुद्ध्याकार एव श‚ब्दार्थ ‚{\tiny $_{lb}$}‚इत्युप‚चारादुच्य‚ते बुद्ध्याकार‚श‚ब्दार्थ‚वादिना (।) त‚दैवं ‚{\color{DodgerBlue3}‚व‚च‚ने} किञ्चिद‚पि ‚{\color{DodgerBlue3}‚न ‚{\tiny $_{lb}$}‚विरुध्य‚ते} । बुद्ध्याकार‚स्यान्व‚यिनः श‚ब्दार्थ‚त्वानिष्टेः । (१६९)
	\pend% ending standard par
      \label{div_pvv.2.170}
	  
	% new div opening: depth here is 2
	

	  \begin{center}%% label @type='head'
	\textbf{ख. अन्यापोह‚कृच्छ‚ब्दः}
	\end{center}
	
	  \bigskip
	  \begingroup
	
	    \large
	  
	    \begin{quote}
	  
	    
	    \stanza[\smallbreak]
	\label{pv.2.170a}\flagstanza{\tiny\textenglish{...2.170a}}मिथ्याव‚भासिनो वैते प्र‚त्य‚याः श‚ब्द‚निर्म्मिताः ।\&[\smallbreak]


	
	    \end{quote}
	  
	  \endgroup
	

	  \pstart \leavevmode% starting standard par
	\hphantom{.}मिथ्या‚{\color{DodgerBlue3}‚व‚भासिनो वा श‚ब्द‚निर्मिता एते प्र‚त्य‚याः} । त‚था हि न ताव‚द‚र्थः श‚ब्द‚{\tiny $_{lb}$}‚बुद्धेर्विष‚यः । त‚त्स्व‚रूपान‚व‚भास‚तः । त‚त्र श‚ब्द‚संकेताभावाच्च । नापि ‚{\tiny $_{lb}$}‚बुद्ध्याकार‚स्त‚स्य वेद‚नेपि विष‚य‚त्वेनान‚ध्य‚व‚सायात् स्व‚ल‚{\tiny $_{6}$}‚क्ष‚ण‚त्वात् संकेता‚{\tiny $_{lb}$}‚भावाच्च । न हि बुद्ध्याकार‚स्य ब‚हिष्ट्वं बाह्य‚स्य वा बुद्ध्याक‚र‚त्व‚म‚स्ति येन ‚{\tiny $_{lb}$}‚त‚थेति भासः स‚त्य‚प्र‚तिभासः स्यात् । त‚स्माद्व‚स्तुतोऽव‚स्तुप्र‚तिभासिनः शाब्दाः ‚{\tiny $_{lb}$}‚प्र‚त्य‚याः ।
	\pend% ending standard par
      

	  \pstart \leavevmode% starting standard par
	क‚थ‚न्त‚र्हीदानीम‚र्थांशापोह‚कृच्छ्रुतिरुक्तेत्याह\edtext{}{\edlabel{pvv.169-5}\label{pvv.169-5}\lemma{कृच्छ्रुतिरुक्तेत्याह}\Bfootnote{अनिष्टं प‚रित्य‚ज्य इष्टे प्र‚व‚र्त‚नात् श‚ब्दाः ।}} (।)
	\pend% ending standard par
      
	  \bigskip
	  \begingroup
	
	    \large
	  
	    \begin{quote}
	  
	    
	    \stanza[\smallbreak]
	\label{pv.2.170b}\flagstanza{\tiny\textenglish{...2.170b}}अनुयान्तीम‚म‚र्थांश‚मिति वापोह‚कृच्छ्रुतिः ॥ १७० ॥\&[\smallbreak]


	
	    \end{quote}
	  
	  \endgroup
	\textsuperscript{\textenglish{170/s}}

	  \pstart \leavevmode% starting standard par
	\hphantom{.}‚{\color{DodgerBlue3}‚इम‚म}‚न्यापोह‚म‚{\color{DodgerBlue3}‚र्थांशं} श‚ब्दा अत‚त्प्र‚तिभासित्वेपि ‚{\color{DodgerBlue3}‚अनुयान्ति} वृत्तिविष‚य‚त्वेन ‚{\tiny $_{lb}$}‚व्य‚व‚स्थाप‚य‚न्ति । अर्थ‚द‚र्श‚नायात‚त्वेन प‚र‚स्प‚रं या त‚त्प्र‚तिब‚न्ध‚नादिति चान्या‚{\color{DodgerBlue3}‚पोह‚{\tiny $_{lb}$}‚कृच्छ्रुति}‚रु\edtext{}{\edlabel{pvv.170-1}\label{pvv.170-1}\lemma{रु}\Bfootnote{नैतेषु विष‚ये प्र‚व‚र्त‚येयुरिति (दिग्)नागेन ।}}क्ता । (१७०)
	\pend% ending standard par
      \label{div_pvv.2.171}
	  
	% new div opening: depth here is 2
	
	  \bigskip
	  \begingroup
	
	    \large
	  
	    \begin{quote}
	  
	    
	    \stanza[\smallbreak]
	\label{pv.2.171a}\flagstanza{\tiny\textenglish{...2.171a}}त‚स्मात् संकेत‚कालेपि;\&[\smallbreak]


	
	    \end{quote}
	  
	  \endgroup
	\textsuperscript{\textenglish{33a/MA}}

	  \pstart \leavevmode% starting standard par
	य‚स्माद् व्य‚व‚हा‚{\tiny $_{7}$}‚र‚कालेऽन्य‚व्य‚व‚च्छेद‚प्र‚तीतिः श‚ब्दात् ‚{\color{DodgerBlue3}‚त‚स्मात्संकेत‚कालेप्य}‚न्या‚{\tiny $_{lb}$}‚पोहः श्रुतौ वाच्य‚त‚या स‚म्ब‚ध्य‚ते नान्य‚त् ।
	\pend% ending standard par
      

	  \pstart \leavevmode% starting standard par
	न‚न्व‚र्थ‚मुप‚द‚र्श्य संकेतः क्रिय‚ते त‚त्क‚थ‚म‚पोह उच्य‚त इत्याह (।)
	\pend% ending standard par
      
	  \bigskip
	  \begingroup
	
	    \large
	  
	    \begin{quote}
	  
	    
	    \stanza[\smallbreak]
	\label{pv.2.171b}\flagstanza{\tiny\textenglish{...2.171b}}निर्दिष्टार्थेन संयुतः ।&स्व‚प्र‚तीतिफ‚लेनान्यापोहः संब‚ध्य‚ते श्रुतौ ॥ १७१ ॥\&[\smallbreak]


	
	    \end{quote}
	  
	  \endgroup
	

	  \pstart \leavevmode% starting standard par
	\hphantom{.}‚{\color{DodgerBlue3}‚निर्दिष्टेनार्थेना}‚न्य‚व्यावृत्तेन व्य‚व‚हार‚काले ‚{\color{DodgerBlue3}‚स्व}‚स्य ‚{\color{DodgerBlue3}‚प्र‚तीतिः फ‚लं} प्र‚योज‚नं य‚स्य ‚{\tiny $_{lb}$}‚तेन ‚{\color{DodgerBlue3}‚संयुतो}‚ऽभेदाध्य‚व‚सायादेक‚त्व‚मुप‚नीतो‚{\color{DodgerBlue3}‚ऽन्यापोहो} बुद्ध्याकार‚स्व‚भावः ‚{\color{DodgerBlue3}‚श्रुतौ स‚म्ब‚{\tiny $_{lb}$}‚ध्य‚ते} न त्व‚र्थ एव । (१७१)
	\pend% ending standard par
      \label{div_pvv.2.172}
	  
	% new div opening: depth here is 2
	

	  \pstart \leavevmode% starting standard par
	\edtext{\textsuperscript{*}}{\edlabel{pvv.170-2}\label{pvv.170-2}\lemma{*}\Bfootnote{विधिमुखेन सामान्य‚निमित्त‚श‚ब्द‚प्र‚वृत्तौ दोष‚माह ।}}त‚था हि (।)
	\pend% ending standard par
      
	  \bigskip
	  \begingroup
	
	    \large
	  
	    \begin{quote}
	  
	    
	    \stanza[\smallbreak]
	\label{pv.2.172}\flagstanza{\tiny\textenglish{....2.172}}अन्यात्रादृष्ट्य‚पेक्ष‚त्वात् क्व‚चित्त‚द्दृष्ट्य‚पेक्ष‚णात् ।&श्रुतौ संब‚ध्य‚तेपोहो नैत‚द् व‚स्तुनि युज्य‚ते ॥ १७२ ॥\&[\smallbreak]


	
	    \end{quote}
	  
	  \endgroup
	

	  \pstart \leavevmode% starting standard par
	\hphantom{.}संकेत‚स्या‚{\color{DodgerBlue3}‚न्य‚त्र} व्य‚व‚च्छेद्ये‚{\color{DodgerBlue3}‚ऽवृक्षेऽद‚र्श‚नापेक्ष‚त्वात् । क्व‚चि}‚द‚व्य‚{\tiny $_{1}$}‚व‚च्छेद्ये वृक्षैक‚देशे ‚{\tiny $_{lb}$}‚‚{\color{DodgerBlue3}‚दृष्ट्य‚पेक्ष‚णात् श्रुता}‚व‚पोहः ‚{\color{DodgerBlue3}‚स‚म्ब‚ध्य‚त} इति निश्चीय‚ते । ‚{\color{DodgerBlue3}‚व‚स्तुनि}‚सामान्यादौ संकेत‚विष‚ये ‚{\tiny $_{lb}$}‚एत‚त् व्य‚व‚च्छेद्याव्य‚व‚च्छेद्य‚योर्द‚र्श‚नाद‚र्श‚नापेक्ष‚णं ‚{\color{DodgerBlue3}‚न युज्य‚ते} ।\edtext{\textsuperscript{*}}{\edlabel{pvv.170-3}\label{pvv.170-3}\lemma{*}\Bfootnote{त‚था हि सामान्ये ।}} व‚स्तुनि विधिमुखेन ‚{\tiny $_{lb}$}‚प्र‚तिपाद्ये किम‚न्य‚त्राद‚र्श‚नापेक्ष‚या । अपेक्ष्य‚ते च त‚तोऽन्य‚व्य‚व‚च्छेद एव प्र‚तिपाद्य‚त इति ‚{\tiny $_{lb}$}‚ग‚म्य‚ते । अन्य‚व्य‚व‚च्छेदः सामान्यादिकं चापेक्ष्य‚ते विप‚क्ष‚प‚रिहारेण प्र‚तिप‚त्त्य‚र्थ‚मिति ‚{\tiny $_{lb}$}‚चेत् ।\edtext{\textsuperscript{*}}{\edlabel{pvv.170-4}\label{pvv.170-4}\lemma{*}\Bfootnote{क‚ल्प‚नानुविद्धार्थ‚ग्र‚ह‚णं ।}}अलं त‚दा‚{\tiny $_{2}$}‚ सामान्येन । अन्य‚व्य‚व‚च्छेदेनैव व्य‚व‚हार‚प‚रिस‚माप्तेः । (१७२)
	\pend% ending standard par
      \label{div_pvv.2.173}
	  
	% new div opening: depth here is 2
	

	  \pstart \leavevmode% starting standard par
	य‚त‚श्च जातिगुण‚क्रियादीनि विशेष‚णानि व‚स्तुग्राहिणि ज्ञाने नाभास‚न्ते (।)
	\pend% ending standard par
      
	  \bigskip
	  \begingroup
	
	    \large
	  
	    \begin{quote}
	  
	    
	    \stanza[\smallbreak]
	\label{pv.2.173}\flagstanza{\tiny\textenglish{....2.173}}त‚स्माद् जात्यादित‚द्योगा नार्थे तेषु च न श्रुतिः ।&संयोज्य‚तेन्य‚व्यावृत्तौ श‚ब्दानामेव योज‚नात् ॥ १७३ ॥\&[\smallbreak]


	
	    \end{quote}
	  
	  \endgroup
	

	  \pstart \leavevmode% starting standard par
	\hphantom{.}‚{\color{DodgerBlue3}‚त‚स्माज्जात्याद‚य‚स्तेषां योगा\edtext{}{\edlabel{pvv.170-5}\label{pvv.170-5}\lemma{योगा}\Bfootnote{स‚म्ब‚न्धाः ।}}श्चार्थे\edtext{}{\edlabel{pvv.170-6}\label{pvv.170-6}\lemma{श्चार्थे}\Bfootnote{इन्द्रिय‚विष‚ये ।}} न} स‚न्ति । ‚{\color{DodgerBlue3}‚अत}‚स्तेषु ‚{\color{DodgerBlue3}‚श्रुतिश्च नोप}‚यु‚{\tiny $_{lb}$}‚ज्य‚ते । ‚{\color{DodgerBlue3}‚अन्य‚व्यावृत्तावेव} प्र‚तीतिसिद्धायां ‚{\color{DodgerBlue3}‚श‚ब्दानां योज‚नात्} । (१७३)
	\pend% ending standard par
      \label{div_pvv.2.174}
	  
	% new div opening: depth here is 2
	\textsuperscript{\textenglish{171/s}}

	  \begin{center}%% label @type='head'
	\textbf{(६) क. प्र‚त्य‚क्षे श‚ब्द‚क‚ल्प‚नानिरासः}
	\end{center}
	

	  \pstart \leavevmode% starting standard par
	त‚देवं जात्यादिक‚ल्प‚ना त‚त्स‚म्ब‚न्ध‚क‚ल्प‚ना च नास्तीत्युक्तं । श‚ब्द‚क‚ल्प‚नापि ‚{\tiny $_{lb}$}‚न स‚म्भ‚व‚तीत्याह (।)
	\pend% ending standard par
      
	  \bigskip
	  \begingroup
	
	    \large
	  
	    \begin{quote}
	  
	    
	    \stanza[\smallbreak]
	\label{pv.2.174}\flagstanza{\tiny\textenglish{....2.174}}संकेत‚स्म‚र‚णोपायं दृष्ट‚संक‚ल‚नात्म‚क‚म् ।&पूर्वाप‚र‚प‚राम‚र्श‚शून्ये त‚च्चाक्षुषे क‚थ‚म् ॥ १७४ ॥\&[\smallbreak]


	
	    \end{quote}
	  
	  \endgroup
	

	  \pstart \leavevmode% starting standard par
	\hphantom{.}श‚ब्द‚क‚ल्प‚नं हि पूर्व्व‚गृहीत‚स्य ‚{\color{DodgerBlue3}‚संकेत}‚{\tiny $_{3}$}‚स्य ‚{\color{DodgerBlue3}‚स्म‚र‚ण‚मुपायो} य‚स्य संकेत‚स्म‚र‚णो‚{\tiny $_{lb}$}‚पायं वाच‚क‚त्वेन ‚{\color{DodgerBlue3}‚दृष्ट}‚स्य श‚ब्द‚स्य ‚{\color{DodgerBlue3}‚संक‚ल‚नं} त‚थायोज‚न\edtext{}{\edlabel{pvv.171-1}\label{pvv.171-1}\lemma{न}\Bfootnote{व‚र्त्त‚मानार्थेन ।}} ‚{\color{DodgerBlue3}‚मात्मा} य‚स्य त‚त् दृष्ट‚संक‚{\tiny $_{lb}$}‚ल‚नात्म‚कं प्र‚सिद्धं । त‚च्च ‚{\color{DodgerBlue3}‚पूर्व्व}‚स्य संकेत‚काल‚दृष्ट‚वाच‚क‚श‚ब्द‚स्या‚{\color{DodgerBlue3}‚प‚र‚स्य} दृश्य‚मानार्थ‚स्य ‚{\tiny $_{lb}$}‚‚{\color{DodgerBlue3}‚प‚राम‚र्शो} वाच्य‚वाच‚क‚तायोज‚न‚न्तेन शून्या श‚ब्दामिश्र‚व‚स्तुस्व‚रूप‚ग्राहिणि ‚{\color{DodgerBlue3}‚चाक्षुषे} ज्ञाने ‚{\color{DodgerBlue3}‚क‚थं} संभाव्य‚ते । चाक्षुषं चाक्ष‚ज‚मात्रोप‚ल‚क्ष‚ण‚मिन्द्रिय‚ज्ञान‚मित्य‚र्थः । (१७४)
	\pend% ending standard par
      \label{div_pvv.2.175}
	  
	% new div opening: depth here is 2
	

	  \pstart \leavevmode% starting standard par
	किञ्च‚{\tiny $_{4}$}‚ (।)
	\pend% ending standard par
      
	  \bigskip
	  \begingroup
	
	    \large
	  
	    \begin{quote}
	  
	    
	    \stanza[\smallbreak]
	\label{pv.2.175}\flagstanza{\tiny\textenglish{....2.175}}अन्य‚त्र ग‚त‚चित्तोपि च‚क्षुषा रूप‚मीक्ष‚ते ।&त‚त्संकेताग्र‚ह‚स्त‚त्र स्प‚ष्ट‚स्त‚ज्जा च क‚ल्प‚ना ॥ १७५ ॥\&[\smallbreak]


	
	    \end{quote}
	  
	  \endgroup
	

	  \pstart \leavevmode% starting standard par
	\hphantom{.}दृश्य‚मानाद‚र्थाद‚{\color{DodgerBlue3}‚न्य‚त्रा}‚तीतादौ विक‚ल्प‚नीये ‚{\color{DodgerBlue3}‚ग‚त‚चित्तः} प्र‚वृत्त‚विक‚ल्पोपि द्र‚ष्टा ‚{\tiny $_{lb}$}‚‚{\color{DodgerBlue3}‚च‚क्षुषा} च‚क्षुर्व्विज्ञानेन ‚{\color{DodgerBlue3}‚रूप‚मीक्ष‚ते} । त‚स्य दृश्य‚मानार्थ‚स्य ‚{\color{DodgerBlue3}‚संकेतः} संकेत‚विष‚यो ‚{\tiny $_{lb}$}‚वाच‚कं नाम त‚स्या‚{\color{DodgerBlue3}‚ग्र‚हो}‚ऽस्म‚र‚णं ‚{\color{DodgerBlue3}‚त‚त्र} चाक्षुषे ज्ञाने ‚{\color{DodgerBlue3}‚स्प‚ष्टः} । त‚त‚{\color{DodgerBlue3}‚स्त‚ज्जा} वाच‚क‚नाम‚{\tiny $_{lb}$}‚स्म‚र‚ण‚प्र‚भ‚वा ‚{\color{DodgerBlue3}‚क‚ल्प‚ना च} चाक्षुषे ज्ञाने नास्तीति शेषः । (१७५)
	\pend% ending standard par
      \label{div_pvv.2.176}
	  
	% new div opening: depth here is 2
	

	  \pstart \leavevmode% starting standard par
	किञ्च (।)
	\pend% ending standard par
      
	  \bigskip
	  \begingroup
	
	    \large
	  
	    \begin{quote}
	  
	    
	    \stanza[\smallbreak]
	\label{pv.2.176}\flagstanza{\tiny\textenglish{....2.176}}जाय‚न्ते क‚ल्प‚नास्त‚त्र य‚त्र श‚ब्दो निवेशितः ।&तेनेच्छातः प्र‚व‚र्त्तेर‚न् नेक्षेर‚न् बाह्य‚म‚क्ष‚जाः ॥ १७६ ॥\&[\smallbreak]


	
	    \end{quote}
	  
	  \endgroup
	

	  \pstart \leavevmode% starting standard par
	\hphantom{.}त‚त्र विष‚ये श‚ब्द‚योज‚नात्मिकाः ‚{\color{DodgerBlue3}‚क‚ल्प‚ना जाय‚न्ते । य‚त्र} संकेत‚काले ‚{\color{DodgerBlue3}‚श‚ब्दो ‚{\tiny $_{lb}$}‚निवेशितः} । न चेन्द्रिय‚{\tiny $_{5}$}‚विष‚ये श‚ब्द‚संकेत इति न त‚त् ज्ञानं श‚ब्द‚योज‚नात्म‚कं ‚{\tiny $_{lb}$}‚(प्राक्) । अथेन्द्रिय‚विष‚य एव श‚ब्द‚निवेश‚स्त‚दा ‚{\color{DodgerBlue3}‚तेन} श‚ब्द‚विष‚य‚त्वेन कार‚णे‚{\tiny $_{lb}$}‚‚{\color{DodgerBlue3}‚नेच्छातः प्र‚व‚र्त्तेर‚न्न‚क्ष‚जाः} प्र‚त्य‚या विक‚ल्प‚व‚त् । न चैत‚द‚स्ति । इच्छाप्र‚भ‚व‚त्वे वा ‚{\tiny $_{lb}$}‚बाह्यार्थ‚संनिधानान‚पेक्ष‚त्वात् । ‚{\color{DodgerBlue3}‚बाह्य‚म‚र्थं नेक्षेर‚न्न‚क्ष‚जाः} प्र‚त्य‚य‚विक‚ल्प‚व‚त् । (१७६)
	\pend% ending standard par
      \label{div_pvv.2.177}
	  
	% new div opening: depth here is 2
	

	  \pstart \leavevmode% starting standard par
	अपि च (।)
	\pend% ending standard par
      
	  \bigskip
	  \begingroup
	
	    \large
	  
	    \begin{quote}
	  
	    
	    \stanza[\smallbreak]
	\label{pv.2.177}\flagstanza{\tiny\textenglish{....2.177}}रूपं रूप‚मितीक्षेत त‚द्धियं किमितीक्ष‚ते ॥&अस्ति चानुभ‚व‚स्त‚स्याःसोविक‚ल्पः क‚थं भ‚वेत् ॥ १७७ ॥\&[\smallbreak]


	
	    \end{quote}
	  
	  \endgroup
	\textsuperscript{\textenglish{172/s}}\textsuperscript{\textenglish{33b/MA}}

	  \pstart \leavevmode% starting standard par
	\hphantom{.}स‚र्व्व‚विक‚ल्प‚वादिनो म‚ते ‚{\color{DodgerBlue3}‚रूप‚मिति} प्र‚वृत्त‚विक‚ल्प‚बुद्धी ‚{\color{DodgerBlue3}‚रूप‚मीक्षेत ।\edtext{\textsuperscript{*}}{\edlabel{pvv.172-1}\label{pvv.172-1}\lemma{*}\Bfootnote{निर्व्विक‚ल्प एवैत‚द्युक्तं प्राक् रूप‚बुद्धिस्त‚तोस्याविक‚ल्पो नात्र ।}} त‚द्धियं} रूप‚धिय‚म‚पि क‚ल्प्य‚मानां ‚{\color{DodgerBlue3}‚किमितीक्ष‚ते} ज्ञाता‚{\tiny $_{6}$}‚ रूप‚बुद्ध्य‚{\color{DodgerBlue3}‚नुभ‚वो} नास्तीति न युक्तं । ‚{\tiny $_{lb}$}‚य‚स्माद‚स्ति चानुभ‚व‚स्त‚स्याः \edtext{}{\edlabel{pvv.172-2}\label{pvv.172-2}\lemma{स्याः}\Bfootnote{य‚दि रूप‚ग्र‚ह‚णे बुद्धेर‚न‚नुभ‚व‚स्त‚दोत्त‚रा स्मृतिर्न स्याद‚स्ति च त‚द‚प्र‚त्य‚क्ष‚त्वेऽर्थाप्र‚त्य‚क्ष‚त्वाच्च ।}}स‚र्व्वेषां प्र‚तिप‚त्तृणां । न च रूप इव त‚द्बुद्धाव‚पि ‚{\tiny $_{lb}$}‚क‚ल्प‚नाऽनुभूय‚ते । त‚त‚श्च रूप‚बुद्धेर‚नुभ‚वो‚{\color{DodgerBlue3}‚ऽविक‚ल्पः क‚थ‚म्भ‚वेत्\edtext{}{\edlabel{pvv.172-3}\label{pvv.172-3}\lemma{वेत्}\Bfootnote{त्व‚न्म‚तेन ।}}} । (१७७)
	\pend% ending standard par
      \label{div_pvv.2.178}
	  
	% new div opening: depth here is 2
	
	  \bigskip
	  \begingroup
	
	    \large
	  
	    \begin{quote}
	  
	    
	    \stanza[\smallbreak]
	\label{pv.2.178}\flagstanza{\tiny\textenglish{....2.178}}त‚यैवानुभ‚वे दृष्टं न विक‚ल्प‚द्व‚यं स‚कृत् ।&एतेन तुल्य‚कालान्य‚विज्ञानानुभ‚वो ग‚तः ॥ १७८ ॥\&[\smallbreak]


	
	    \end{quote}
	  
	  \endgroup
	

	  \pstart \leavevmode% starting standard par
	\hphantom{.}‚{\color{DodgerBlue3}‚त‚यै}‚व रूप‚बुद्ध्या रूप‚स्य स्वात्म‚न‚श्चाऽ‚{\color{DodgerBlue3}‚नुभ‚वे}‚भ्युप‚ग‚म्य‚माने रूप‚मिति रूपानुभ‚व ‚{\tiny $_{lb}$}‚इति च ‚{\color{DodgerBlue3}‚विक‚ल्प‚द्व‚यं स‚कृत्} स्यात् । त‚च्च नास्त्य\edtext{}{\edlabel{pvv.172-4}\label{pvv.172-4}\lemma{नास्त्य}\Bfootnote{अनुप‚ल‚ब्धेः ।}}नुभ‚व‚वाधित‚त्वात् । ‚{\color{DodgerBlue3}‚एतेन} स‚कृत्क‚{\tiny $_{lb}$}‚ल्प‚नाद्व‚य‚निषेधेन ‚{\color{DodgerBlue3}‚तुल्य‚कालेनान्येन}\edtext{}{\edlabel{pvv.172-5}\label{pvv.172-5}\lemma{निषेधेन}\Bfootnote{इन्द्रिय‚ज्ञान‚कालेन विक‚ल्पेन रूप‚बुद्ध्य‚नुभ‚वो निर‚स्तः इति वृत्तिः ।}} निर्व्विक‚ल्प‚ज्ञानेना‚{\color{DodgerBlue3}‚नुभ‚वो} रूप‚बु्‚{\tiny $_{1}$}‚द्धे‚{\color{DodgerBlue3}‚र्ग‚तो} निर्ण्णी‚{\tiny $_{lb}$}‚तोत्त‚रो बोद्ध‚व्यः । (१७८)
	\pend% ending standard par
      \label{div_pvv.2.179}
	  
	% new div opening: depth here is 2
	

	  \begin{center}%% label @type='head'
	\textbf{ख. उत्त‚र‚काल‚भाविविक‚ल्प‚ज्ञानेनानुभ‚वः}
	\end{center}
	

	  \pstart \leavevmode% starting standard par
	उत्त‚र‚काल‚भाविना विक‚ल्प‚ज्ञानेनानुभ‚व इति चेदाह (।)
	\pend% ending standard par
      
	  \bigskip
	  \begingroup
	
	    \large
	  
	    \begin{quote}
	  
	    
	    \stanza[\smallbreak]
	\label{pv.2.179}\flagstanza{\tiny\textenglish{....2.179}}स्मृतिर्भ‚वेद‚तीते च साऽगृहीते क‚थं भ‚वेत् ।&स्याच्चान्य‚धीप‚रिच्छेदाभिन्न‚रूपा स्व‚बुद्धिधीः ॥ १७९ ॥\&[\smallbreak]


	
	    \end{quote}
	  
	  \endgroup
	

	  \pstart \leavevmode% starting standard par
	\hphantom{.}‚{\color{DodgerBlue3}‚अतीते च} रूपानुभ‚वे ‚{\color{DodgerBlue3}‚स्मृतिः} प‚श्चात्त‚नेन विक‚ल्पेन न भ‚वेन्नानुभ‚वः । ‚{\color{DodgerBlue3}‚सा} स्मृतिर‚{\color{DodgerBlue3}‚गृहीते}‚ऽनुभ‚वे ‚{\color{DodgerBlue3}‚क‚थ‚म्भ‚वेत्} । य‚दि चातीत‚बुद्धिर्व्विक‚ल्प्य‚ते त‚दाऽन्य‚स्य पुंसो ‚{\tiny $_{lb}$}‚धियः ‚{\color{DodgerBlue3}‚प‚रिच्छेदेन} प‚रोक्ष‚बुद्धिविक‚ल्पात्म‚केना‚{\color{DodgerBlue3}‚भिन्न‚रूपा} त‚थात्वेन ‚{\color{DodgerBlue3}‚स्व‚बुद्धिधीः} स्यात् । ‚{\tiny $_{lb}$}‚अस्ति च प‚र‚बुद्धिप्र‚तीतिविल‚क्ष‚ण‚स्व‚बुद्ध्य‚नुभ‚वः । त‚स्माद‚वि‚{\tiny $_{2}$}‚क‚ल्प एवासौ । (१७९)
	\pend% ending standard par
      \label{div_pvv.2.180}
	  
	% new div opening: depth here is 2
	

	  \pstart \leavevmode% starting standard par
	किञ्च (।)
	\pend% ending standard par
      
	  \bigskip
	  \begingroup
	
	    \large
	  
	    \begin{quote}
	  
	    
	    \stanza[\smallbreak]
	\label{pv.2.180}\flagstanza{\tiny\textenglish{....2.180}}अतीत‚म‚प‚दृष्टान्त‚म‚लिङ्ग‚ञ्चार्थ‚वेद‚न‚म् ।&सिद्धं त‚त्केन त‚स्मिन् हि न प्र‚त्य‚क्षं न लैङ्गिक‚म् ॥ १८० ॥\&[\smallbreak]


	
	    \end{quote}
	  
	  \endgroup
	

	  \pstart \leavevmode% starting standard par
	स‚विक‚ल्प‚क‚प्र‚त्य‚क्ष‚वादिना निर्व्विक‚ल्प‚स्याप्य‚स्व‚संवेद‚न‚वा\edtext{}{\edlabel{pvv.172-6}\label{pvv.172-6}\lemma{वा}\Bfootnote{वैशेषिक‚स्य ।}}दिनो म‚तेऽ‚{\color{DodgerBlue3}‚तीत‚{\tiny $_{lb}$}‚म‚र्थ‚वेद‚नं} न केव‚ल‚म‚ध्य‚क्ष‚तो व‚र्त‚मान‚विष‚य\edtext{}{\edlabel{pvv.172-7}\label{pvv.172-7}\lemma{य}\Bfootnote{हेतुः ।}}त्वान्न सिध्य‚ति । किन्त्व‚नुमानाद‚पि ‚{\tiny $_{lb}$}‚य‚स्माद‚{\color{DodgerBlue3}‚लिङ्गं} लिङ्ग‚र‚हितं । त‚था हि ध‚र्मिणो ज्ञान‚स्यासिद्ध‚त्वात् लिङ्ग‚माश्र‚या‚{\tiny $_{lb}$}‚\leavevmode\ledsidenote{\textenglish{173/s}} ‚{\color{DodgerBlue3}‚सिद्धं । अनुमानात् ज्ञान‚सिद्धिवादिनः क‚स्य‚चिज्ज्ञान‚स्याध्य‚क्षासिद्ध‚त्वात्} अनुमान‚सिद्धाव‚न‚व‚स्थानात् ‚{\color{DodgerBlue3}‚अप‚दृष्टान्तं} । दृष्टान्तासिद्धौ न‚{\tiny $_{5}$}‚ ‚{\color{DodgerBlue3}‚व्याप्तिसिद्धिरिति} लिङ्ग‚र‚हित‚मेवातीतं रूपादिद‚र्श‚नं । ‚{\color{DodgerBlue3}‚त‚त्त‚स्मात्केन प्र‚माणेन सिद्धं त‚स्मिन् हि} रूपादिद‚र्श‚ने ‚{\color{DodgerBlue3}‚न प्र‚त्य‚क्ष}‚म‚भिम‚त‚त्वाद‚स्ति । ‚{\color{DodgerBlue3}‚न च लैङ्गिक‚म‚नुमान‚मुक्त‚क्र‚मादिति} स‚क‚ल‚म‚प्र‚तिप‚त्तिक‚म‚न्ध‚मू\edtext{}{\edlabel{pvv.173-1}\label{pvv.173-1}\lemma{मू}\Bfootnote{प्र‚त्य‚क्षासिद्ध्या । व‚च‚नाभावात् अन‚नुभूतेनाविक‚ल्प्य व‚च‚नं ।}}कं ज‚ग‚त्प्राप्त‚मिति ॥ (१८०)
	\pend% ending standard par
      \label{div_pvv.2.181_2.182_2.183_2.184}
	  
	% new div opening: depth here is 2
	

	  \pstart \leavevmode% starting standard par
	अथ (।)
	\pend% ending standard par
      
	  \bigskip
	  \begingroup
	
	    \large
	  
	    \begin{quote}
	  
	    
	    \stanza[\smallbreak]
	\label{pv.2.181a}\flagstanza{\tiny\textenglish{...2.181a}}त‚त्स्व‚रूपाव‚भासिन्या बुद्ध्यान‚न्त‚र‚या य‚दि ।&रूपादिरिव गृह्येत;\&[\smallbreak]


	
	    \end{quote}
	  
	  \endgroup
	

	  \pstart \leavevmode% starting standard par
	\hphantom{.}‚{\color{DodgerBlue3}‚त‚त्स्व‚रूपाव‚भासिन्या}‚ऽतीत‚रूपादिबुद्धिरूप‚प्र‚तिभासिन्या ‚{\color{DodgerBlue3}‚त‚ज्ज‚न्य‚याऽन‚न्त‚या} धियाऽतीत‚बुद्धि‚{\color{DodgerBlue3}‚र्य‚दि गृह्य‚ते} सौ त्रा न्ति क म‚ते ‚{\color{DodgerBlue3}‚रूपादिरिव} त‚द‚नुकारिण्या ‚{\color{DodgerBlue3}‚त‚द‚{\tiny $_{lb}$}‚न‚न्त‚र‚{\tiny $_{4}$}‚या} धिया त‚दा को ‚{\color{DodgerBlue3}‚दोष} इत्याह (।)
	\pend% ending standard par
      

	  \pstart \leavevmode% starting standard par
	य‚दा चिरं ब‚हुषु विष‚येषु ज्ञा\edtext{}{\edlabel{pvv.173-2}\label{pvv.173-2}\lemma{ज्ञा}\Bfootnote{इन्द्रिय‚ज्ञानान्य‚स‚न्तानेन ।}}नानि प्र‚व‚र्त‚न्ते त‚दा (।)
	\pend% ending standard par
      
	  \bigskip
	  \begingroup
	
	    \large
	  
	    \begin{quote}
	  
	    
	    \stanza[\smallbreak]
	\label{pv.2.181b}\flagstanza{\tiny\textenglish{...2.181b}}न स्यात् त‚त्पूर्व‚धीग्र‚हः ॥ १८१ ॥\&[\smallbreak]


	
	    \end{quote}
	  
	  \endgroup
	
	  \bigskip
	  \begingroup
	
	    \large
	  
	    \begin{quote}
	  
	    
	    \stanza[\smallbreak]
	\label{pv.2.182}\flagstanza{\tiny\textenglish{....2.182}}सोविक‚ल्पः स्व‚विष‚यो विज्ञानानुभ‚वो य‚था ।&अश‚क्य‚स‚म‚यं त‚द्व‚द‚न्य‚द‚प्य‚विक‚ल्प‚क‚म् ॥ १८२ ॥\&[\smallbreak]


	
	    \end{quote}
	  
	  \endgroup
	
	  \bigskip
	  \begingroup
	
	    \large
	  
	    \begin{quote}
	  
	    
	    \stanza[\smallbreak]
	\label{pv.2.183}\flagstanza{\tiny\textenglish{....2.183}}सामान्य‚वाचिनः श‚ब्दास्त‚देकार्था च क‚ल्प‚ना ।&अभावे निर्विक‚ल्प‚स्य विशेषाधिग‚मः क‚थ‚म् ॥ १८३ ॥\&[\smallbreak]


	
	    \end{quote}
	  
	  \endgroup
	
	  \bigskip
	  \begingroup
	
	    \large
	  
	    \begin{quote}
	  
	    
	    \stanza[\smallbreak]
	\label{pv.2.184}\flagstanza{\tiny\textenglish{....2.184}}अस्ति चेन्निर्विक‚ल्प‚ञ्च किञ्चित्त‚त्तुल्य‚हेतुक‚म् ।&स‚र्वं त‚थैव हेतोर्हि भेदाद् भेदः फ‚लात्म‚नाम् ॥ १८४ ॥\&[\smallbreak]


	
	    \end{quote}
	  
	  \endgroup
	

	  \pstart \leavevmode% starting standard par
	\hphantom{.}त‚स्माद‚न्त्यात् ज्ञानात् याः ‚{\color{DodgerBlue3}‚पूर्व्वा} धिय‚स्तासां ‚{\color{DodgerBlue3}‚ग्र‚हो न स्यादि}‚ति दोषः । ‚{\tiny $_{lb}$}‚अन्त्य‚बुद्धिज‚नित‚या हि धिया सैव गृह्य‚ते न त्व‚न्या इति स्यात् । अस्ति चानु‚{\tiny $_{lb}$}‚भ‚व‚स्तासां य‚द्व‚लेन चिर‚म‚ह‚म‚द्राक्ष‚मिति भ‚व‚ति द्र‚ष्टुः । त‚स्मा‚{\color{DodgerBlue3}‚त्स्व‚विष‚यः} स्व‚रूपा‚{\tiny $_{lb}$}‚ल‚म्ब‚नो ‚{\color{DodgerBlue3}‚विज्ञाना}‚नां पूर्व्व‚भाविना‚{\color{DodgerBlue3}‚म‚नुभ‚वोऽविक‚ल्पः} । स य‚था ‚{\color{DodgerBlue3}‚त‚द्व‚द‚न्य‚द‚पि} ज्ञान‚म‚न्त्य‚म‚प्र‚तिब‚{\tiny $_{5}$}‚द्ध‚वृत्ति ‚{\color{DodgerBlue3}‚चाविक‚ल्प‚कं} बोद्ध‚व्यं । य‚स्मात्स‚क‚ल‚मेव स्व‚रूप‚{\color{DodgerBlue3}‚म‚श‚क्य‚{\tiny $_{lb}$}‚स‚म‚यं} श‚ब्द‚संकेताविष‚यः । त‚त‚श्च न विक‚ल्प‚ग्राह्यं (।) किञ्च विशेष‚संकेताभावात् ‚{\tiny $_{lb}$}‚व्य‚व‚हार‚कालानुयायित्वाच्च ‚{\color{DodgerBlue3}‚सामान्य‚वाचिनः श‚ब्दास्तैः} श‚ब्दै‚{\color{DodgerBlue3}‚रेकार्था} एक‚{\tiny $_{lb}$}‚विष‚या ‚{\color{DodgerBlue3}‚च क‚ल्प‚ना} श‚ब्द‚योज‚न‚या श‚ब्दार्थ एव क‚ल्प‚ना । प‚र‚म‚ते च ‚{\color{DodgerBlue3}‚निर्व्विक‚ल‚{\tiny $_{lb}$}‚प‚स्य} ज्ञान‚स्या‚{\color{DodgerBlue3}‚भावे} विशेष‚स्य ‚{\color{DodgerBlue3}‚विक‚ल्पा}‚विष‚य‚स्या‚{\color{DodgerBlue3}‚धिग‚मः क‚थं} न क‚थ‚ञ्चि-\leavevmode\ledsidenote{\textenglish{34a/MA}} ‚{\tiny $_{lb}$}‚दित्य‚र्थः । विशेषानुभ‚व‚द‚र्श‚ना‚{\color{DodgerBlue3}‚द‚स्ति किञ्चि‚{\tiny $_{6}$}‚न्निर्व्विक‚ल्पं च} ज्ञानं य‚थाहुर्मी मां स ‚{\tiny $_{lb}$}‚\leavevmode\ledsidenote{\textenglish{174/s}} ‚{\color{DodgerBlue3}‚काद‚य इति चेत्} । एव‚न्त‚र्हि तेन निर्व्विक‚ल्पेन ‚{\color{DodgerBlue3}‚तुल्य‚हेतुकं} च‚क्षूरूप‚म‚न‚स्कारादिस‚मान‚{\tiny $_{lb}$}‚हेतुकं विशेष‚विष‚यं ‚{\color{DodgerBlue3}‚स‚र्व्वं} ज्ञानं ‚{\color{DodgerBlue3}‚त‚थैवा}‚विक‚ल्प‚क‚म‚स्तु (।) न तु स्व‚ल‚क्ष‚ण‚विष‚य‚म‚पि ‚{\tiny $_{lb}$}‚किञ्चित्स‚विक‚ल्प‚कं । हिर्य‚स्मा‚{\color{DodgerBlue3}‚द्ध‚तोर्भेदात्फ‚लात्म‚नां भेदो} भ‚व‚ति । हेत्व‚भेदे तु ‚{\tiny $_{lb}$}‚फ‚लाभेद एव युक्तः । नान्य‚था क्व‚चिद‚प्येक‚जातीय‚ता स्यात् ॥\edtext{\textsuperscript{*}}{\edlabel{pvv.174-1}\label{pvv.174-1}\lemma{*}\Bfootnote{प्र‚त्य‚क्षं क‚ल्प‚नापोढ‚मि त्यादि य‚त्रैषा क‚ल्प‚ना नास्तीत्य‚न्तः \href{http://sarit.indology.info/?cref=psv.1.3}{[प्र‚माण] स‚मुच्च‚यो व्याख्यातः} ।}}(१८१-८४)
	\pend% ending standard par
      \label{div_pvv.2.185}
	  
	% new div opening: depth here is 2
	

	  \pstart \leavevmode% starting standard par
	किञ्च (।)
	\pend% ending standard par
      
	  \bigskip
	  \begingroup
	
	    \large
	  
	    \begin{quote}
	  
	    
	    \stanza[\smallbreak]
	\label{pv.2.185}\flagstanza{\tiny\textenglish{....2.185}}अन‚पेक्षित‚बाह्यार्था योज‚ना स‚म‚य‚स्मृतेः ।&त‚थान‚पेक्ष्य स‚म‚यं व‚स्तुश‚क्त्यैव नेत्र‚धीः ॥ १८५ ॥\&[\smallbreak]


	
	    \end{quote}
	  
	  \endgroup
	

	  \pstart \leavevmode% starting standard par
	\hphantom{.}‚{\color{DodgerBlue3}‚योज\edtext{}{\edlabel{pvv.174-2}\label{pvv.174-2}\lemma{योज}\Bfootnote{श‚ब्दार्थ‚योः ।}}ना} क‚ल्प‚नाऽन‚{\color{DodgerBlue3}‚पेक्षित‚बाह्यार्था} ब‚हिर‚र्थ‚संकेत‚विष‚य‚म‚न‚पेक्ष्यै‚{\tiny $_{1}$}‚व स‚म‚य‚स्य ‚{\tiny $_{lb}$}‚प्राग्गृहीत‚स्य ‚{\color{DodgerBlue3}‚स्मृतेः} स‚काशाद् भ‚व‚ति ताव‚त् । ‚{\color{DodgerBlue3}‚त‚था स‚म‚य‚म‚न‚पेक्ष्य व‚स्तुनः} स्व‚ल‚क्ष‚ण‚स्य ‚{\color{DodgerBlue3}‚श‚क्त्या} स्वाकारानुकारिविज्ञान‚ज‚न‚न‚साम‚र्थ्येनैव ‚{\color{DodgerBlue3}‚नेत्र‚धीर्जा}‚य‚ते ‚{\color{DodgerBlue3}‚य‚दि} त‚दा को विरोधः । (१८५)
	\pend% ending standard par
      \label{div_pvv.2.186}
	  
	% new div opening: depth here is 2
	

	  \pstart \leavevmode% starting standard par
	स्यादेत‚द् (।)
	\pend% ending standard par
      
	  \bigskip
	  \begingroup
	
	    \large
	  
	    \begin{quote}
	  
	    
	    \stanza[\smallbreak]
	\label{pv.2.186}\flagstanza{\tiny\textenglish{....2.186}}संकेत‚स्म‚र‚णापेक्ष रूपं य‚द्य‚क्ष‚चेत‚सि ।&अन‚पेक्ष्य न चेच्छ‚क्तं स्यात् स्मृतावेव लिङ्ग‚व‚त् ॥ १८६ ॥\&[\smallbreak]


	
	    \end{quote}
	  
	  \endgroup
	

	  \pstart \leavevmode% starting standard par
	\hphantom{.}रूप‚म‚{\color{DodgerBlue3}‚क्ष‚चेत‚सि} क‚र्त‚व्ये ‚{\color{DodgerBlue3}‚संकेत‚स्म‚र‚णापेक्षं} त‚द‚न‚पेक्षं पुन‚र्नाश‚क्त‚मिति ‚{\tiny $_{lb}$}‚एवं त‚र्हि श्रुतावेव रूपं श‚क्त‚मिति स्यात् । न त्विन्द्रिय‚बुद्धौ ‚{\color{DodgerBlue3}‚लिङ्ग‚व‚त्} । य‚था ‚{\tiny $_{lb}$}‚हि लिंगं ‚{\color{DodgerBlue3}‚न} लिंग‚बुद्धौ साक्षाच्छ‚क्तं किन्तु लिङ्ग‚लिङ्गिनोः स‚म्म्ब‚न्धिस्मृतावेव । ‚{\tiny $_{lb}$}‚त‚था संकेत‚स्म‚र‚णे रूपं नि‚{\tiny $_{2}$}‚मित्तं स्यात् । न चागृहीतं स्मृतिप्र‚तिबोध‚क‚मिति ‚{\tiny $_{lb}$}‚निर्व्विक‚ल्प‚क‚म‚स्य ग्र‚ह‚णं प्राक् त‚तः स्मृतिः । त‚त‚श्च योज‚नेति क्र‚मः । (१८६)
	\pend% ending standard par
      \label{div_pvv.2.187}
	  
	% new div opening: depth here is 2
	

	  \pstart \leavevmode% starting standard par
	क‚थं पुन‚र‚र्थ‚स‚म्मुखीभावात् स्मृतिज‚न्मेत्याह (।)
	\pend% ending standard par
      
	  \bigskip
	  \begingroup
	
	    \large
	  
	    \begin{quote}
	  
	    
	    \stanza[\smallbreak]
	\label{pv.2.187}\flagstanza{\tiny\textenglish{....2.187}}त‚स्यास्त‚त्स‚ङ्ग‚मोत्प‚त्तेर‚क्ष‚धीः स्यात् स्मृतेर्न्न वा ।&त‚तः कालान्त‚रेपि स्यात् क्व‚चिद् व्याक्षेप‚स‚म्भ‚वात् ॥ १८७ ॥\&[\smallbreak]


	
	    \end{quote}
	  
	  \endgroup
	

	  \pstart \leavevmode% starting standard par
	\hphantom{.}‚{\color{DodgerBlue3}‚त‚स्याः स्मृतेस्त}‚स्यार्थ‚स्य ‚{\color{DodgerBlue3}‚संग‚मेन} स‚म्मुखीभावे‚{\color{DodgerBlue3}‚नोत्प‚त्तेः\edtext{}{\edlabel{pvv.174-3}\label{pvv.174-3}\lemma{त्तेः}\Bfootnote{अर्थात्सैव स्मृतिः स्यात् ।}} । न त्व‚क्ष‚धी}‚र‚र्था‚{\color{DodgerBlue3}‚त्स्यात्} । ‚{\tiny $_{lb}$}‚सा तु स्मृतेर‚र्थ‚ज‚नितायाः स्यात् । न वा स्मृतेर‚पि भ‚वेत् स्मृत्य‚धीन‚तायां नार्था‚{\tiny $_{lb}$}‚धीन‚ता । य‚च्च स्म‚र‚णं भावि त‚न्नाव‚श्यं\edtext{}{\edlabel{pvv.174-4}\label{pvv.174-4}\lemma{श्यं}\Bfootnote{इच्छाव‚शात् ।}} भ‚व‚तीति क‚दाचिन्न भ‚वेद‚पि । ‚{\color{DodgerBlue3}‚त‚तः} स्मृतेः ‚{\tiny $_{lb}$}‚‚{\color{DodgerBlue3}‚कालान्त‚{\tiny $_{3}$}‚रेणापि स्याद}‚ध्य‚क्ष‚धीः स्मृत्य‚न‚न्त‚रं ‚{\color{DodgerBlue3}‚क्व‚चि}‚द्विष‚यान्त‚रे ‚{\color{DodgerBlue3}‚व्याक्षेप‚स्या-} श‚क्तिल‚क्ष‚ण‚स्य ‚{\color{DodgerBlue3}‚संभ‚वात्} । त‚न्निवृत्तौ स‚त्यां क्र‚मेण भ‚वेत् । (१८७)
	\pend% ending standard par
      \label{div_pvv.2.188}
	  
	% new div opening: depth here is 2
	\textsuperscript{\textenglish{175/s}}

	  \pstart \leavevmode% starting standard par
	स्यादेत‚त्प्र‚थ‚म‚मुभिमुखीभ‚व‚न्न‚र्थः स्मृ\edtext{}{\edlabel{pvv.175-1}\label{pvv.175-1}\lemma{स्मृ}\Bfootnote{इन्द्रिय‚ज्ञान‚हेतोः ।}}तेर्हेतुस्त‚त इन्द्रिय‚ज्ञान‚स्येति (।)
	\pend% ending standard par
      
	  \bigskip
	  \begingroup
	
	    \large
	  
	    \begin{quote}
	  
	    
	    \stanza[\smallbreak]
	\label{pv.2.188}\flagstanza{\tiny\textenglish{....2.188}}क्र‚मेणोभ‚य‚हेतुश्चेत् प्रागेव स्याद‚भेद‚तः ॥&अन्योक्ष‚बुद्धिहेतुश्चेत् स्मृतिस्त‚त्राप्य‚न‚र्थिका ॥ १८८ ॥\&[\smallbreak]


	
	    \end{quote}
	  
	  \endgroup
	

	  \pstart \leavevmode% starting standard par
	\hphantom{.}‚{\color{DodgerBlue3}‚क्र‚मेणोभ‚य‚हेतु}‚र‚भिम‚त‚{\color{DodgerBlue3}‚श्चेत्} । य‚द्येवं पूर्व्वाप‚रै‚{\color{DodgerBlue3}‚क‚स्व‚भाव‚साम‚र्थ्य‚स्य} भाव‚स्या‚{\color{DodgerBlue3}‚भेद‚त-} स्त‚ज्ज‚न्यं द्व‚य‚म‚पि ‚{\color{DodgerBlue3}‚प्रागेव स्यात्} न क्र‚म‚तः ॥\edtext{\textsuperscript{*}}{\edlabel{pvv.175-2}\label{pvv.175-2}\lemma{*}\Bfootnote{अत‚त्स्व‚भाव‚त्वेन वा प‚श्चाद‚पि अक्ष‚णिक‚प‚क्षे ।}} स्यादेत‚द्(।) भावानां क्ष‚णिक‚त्वा‚{\tiny $_{lb}$}‚च्चान्यः स्मृतिप्र‚बोध‚कः क्ष‚णोऽन्य‚श्चाक्ष‚{\tiny $_{4}$}‚बुद्धेर्हेतुश्चेत् । ‚{\color{DodgerBlue3}‚त‚त्राद्येपि} क्ष‚णे वाच‚क‚{\tiny $_{lb}$}‚‚{\color{DodgerBlue3}‚श‚ब्द‚स्मृतिर‚न‚र्थिका} (। १८८)
	\pend% ending standard par
      \label{div_pvv.2.189}
	  
	% new div opening: depth here is 2
	

	  \pstart \leavevmode% starting standard par
	य‚स्माद् (।)
	\pend% ending standard par
      
	  \bigskip
	  \begingroup
	
	    \large
	  
	    \begin{quote}
	  
	    
	    \stanza[\smallbreak]
	\label{pv.2.189}\flagstanza{\tiny\textenglish{....2.189}}य‚था स‚मित‚सिध्य‚र्थ‚मिष्य‚ते स‚म‚य‚स्मृतिः ।&भेद‚श्चास‚मितो ग्राह्यः स्मृतिस्त‚त्र किम‚र्थिका ॥ १८९ ॥\&[\smallbreak]


	
	    \end{quote}
	  
	  \endgroup
	

	  \pstart \leavevmode% starting standard par
	\hphantom{.}‚{\color{DodgerBlue3}‚य‚था स‚मित}‚स्य श‚ब्द‚वाच्य‚त‚याऽ\edtext{}{\edlabel{pvv.175-3}\label{pvv.175-3}\lemma{याऽ}\Bfootnote{स्मृतिविष‚य‚स्याक्ष‚ज्ञानेन ग्र‚हार्थं हि स्मृतिः ।}}र्थ‚स्य ‚{\color{DodgerBlue3}‚सिद्ध्य‚र्थं स‚म‚य‚स्मृतिरिष्य‚ते ।\edtext{\textsuperscript{*}}{\edlabel{pvv.175-4}\label{pvv.175-4}\lemma{*}\Bfootnote{य‚त्र स्मृतिविष‚याद‚न्यो ।}} भेदो} विशेषो‚{\color{DodgerBlue3}‚ऽस‚मितः} संकेताविष‚य‚श्चाक्ष‚धिया ‚{\color{DodgerBlue3}‚ग्राह्यः} । त‚त्र स्मृतिः स‚म‚य‚स्य ‚{\color{DodgerBlue3}‚किम‚र्थिका} निष्प्र‚योज‚ना (। १८९)
	\pend% ending standard par
      \label{div_pvv.2.190}
	  
	% new div opening: depth here is 2
	

	  \pstart \leavevmode% starting standard par
	सामान्ये कालान्त‚रानुव‚र्त्तिनि संकेतः स एव स्म‚र्य‚त इति चेदाह (।) न (।)
	\pend% ending standard par
      
	  \bigskip
	  \begingroup
	
	    \large
	  
	    \begin{quote}
	  
	    
	    \stanza[\smallbreak]
	\label{pv.2.190}\flagstanza{\tiny\textenglish{....2.190}}सामान्य‚मात्र‚ग्र‚ह‚णे भेदापेक्षा न युज्य‚ते ॥&त‚स्माच्च‚क्षुश्च रूप‚ञ्च प्र‚तीत्योदेति नेत्र‚धीः ॥ १९० ॥\&[\smallbreak]


	
	    \end{quote}
	  
	  \endgroup
	

	  \pstart \leavevmode% starting standard par
	\hphantom{.}‚{\color{DodgerBlue3}‚सामान्य‚मात्र‚स्य ग्र‚ह‚णे}‚ऽभ्युप‚ग‚म्य‚माने ‚{\color{DodgerBlue3}‚भेद}‚स्य विशेष‚स्य संकेत‚विष‚य‚स्या‚{\color{DodgerBlue3}‚पेक्षा} न युज्य‚ते‚{\tiny $_{5}$}‚ य‚था गौरित्युक्ते कीदृशो गौरिति । त‚स्मात्सामान्य‚व‚ति विशेषे संकेत‚{\tiny $_{lb}$}‚स्तेनैवार्थित्वाद् व्य‚व‚हारिणां । त‚था च भेद‚श्चास‚मितो ग्राह्य इत्युक्तं । ‚{\color{DodgerBlue3}‚त‚स्मा‚{\tiny $_{lb}$}‚च्च‚क्षुश्च रूप‚ञ्च\edtext{}{\edlabel{pvv.175-5}\label{pvv.175-5}\lemma{ञ्च}\Bfootnote{अथ क‚स्माद् द्व‚याधीनायामुत्प‚त्तौ प्र‚त्य‚क्ष‚मुच्य‚ते न प्र‚तिविष‚य‚मिति[प्र‚माण] स‚मुच्च‚यं व्याख्यातुमुप‚क्र‚म‚ते ॥}} प्र‚तीत्यासाद्योदेति नेत्र‚धी}‚रित्य‚भ्युप‚ग‚न्त‚व्यं ॥ (१९०)
	\pend% ending standard par
      \label{div_pvv.2.191ab}
	  
	% new div opening: depth here is 2
	

	  \pstart \leavevmode% starting standard par
	व्य‚व‚धानेनापि रूप‚कार‚ण‚ता स्यादिति चेत् । आह (।)
	\pend% ending standard par
      
	  \bigskip
	  \begingroup
	
	    \large
	  
	    \begin{quote}
	  
	    
	    \stanza[\smallbreak]
	\label{pv.2.191a}\flagstanza{\tiny\textenglish{...2.191a}}साक्षाच्चेत् ज्ञान‚ज‚न‚ने स‚म‚र्थो विष‚योक्ष‚व‚त् ।\&[\smallbreak]


	
	    \end{quote}
	  
	  \endgroup
	

	  \pstart \leavevmode% starting standard par
	\hphantom{.}‚{\color{DodgerBlue3}‚साक्षाच्च विष‚यो} रूपादिः स्व‚ग्राह‚क‚{\color{DodgerBlue3}‚ज्ञान‚ज‚न‚ने स‚म‚र्थोऽक्ष‚व‚त्} । न व्य‚व‚धानेन । ‚{\tiny $_{lb}$}‚स्मृत्य‚धीन‚तायां दोष‚स्योक्त‚त्वात्‚{\tiny $_{6}$}‚ (।) त‚स्माद‚श‚ब्द‚संसृ/?/ष्टार्थ‚ब‚ल‚भावित‚द्रूपानुकारि\leavevmode\ledsidenote{\textenglish{34b/MA}} ‚{\tiny $_{lb}$}‚प्र‚त्य‚क्ष‚म‚नाविष्टाभिलाप‚म‚विक‚ल्प‚क‚मेव युक्तं ॥ X X ॥
	\pend% ending standard par
      
	  
	% new div opening: depth here is 1
	
\chapter*[{(६. प्र‚त्य‚क्ष‚भेदाः)}]{(६. प्र‚त्य‚क्ष‚भेदाः)}

	  \begin{center}%% label @type='head'
	\textbf{(१) इन्द्रिय‚प्र‚त्य‚क्ष‚म्}
	\end{center}
	\label{div_pvv.2.191cd}
	  
	% new div opening: depth here is 2
	\textsuperscript{\textenglish{176/s}}
	  \bigskip
	  \begingroup
	
	    \large
	  
	    \begin{quote}
	  
	    
	    \stanza[\smallbreak]
	\label{pv.2.191b}\flagstanza{\tiny\textenglish{...2.191b}}अथ क‚स्माद् द्व‚याधीन‚ज‚न्म त‚त्तेन नोच्य‚ते ॥ १९१ ॥\&[\smallbreak]


	
	    \end{quote}
	  
	  \endgroup
	

	  \pstart \leavevmode% starting standard par
	\hphantom{.}‚{\color{DodgerBlue3}‚अथ द्व‚याधीन‚ज‚न्म}‚विष‚येन्द्रियोत्प‚त्ति‚{\color{DodgerBlue3}‚त‚दि}‚न्द्रिय‚ज्ञान‚मिन्द्रियेणो‚{\color{DodgerBlue3}‚च्य‚ते} व्य‚प‚दिश्य‚ते ‚{\tiny $_{lb}$}‚प्र‚त्य‚क्ष‚मिति प्र‚तिग‚त‚म‚क्ष‚म्प्र‚त्य‚क्ष‚मिन्द्रियाश्रित‚मित्य‚र्थः ः (।) क‚स्मात्पुन‚र्व्विष‚येण ‚{\tiny $_{lb}$}‚‚{\color{DodgerBlue3}‚नोच्य‚ते} प्र‚तिविष‚य‚मिति ॥ (१९१)
	\pend% ending standard par
      \label{div_pvv.2.192}
	  
	% new div opening: depth here is 2
	

	  \pstart \leavevmode% starting standard par
	न ख‚लु व्य‚स‚नित‚या व्य‚प‚देशो नियुज्य‚ते । अपि तु (।)
	\pend% ending standard par
      
	  \bigskip
	  \begingroup
	
	    \large
	  
	    \begin{quote}
	  
	    
	    \stanza[\smallbreak]
	\label{pv.2.192}\flagstanza{\tiny\textenglish{....2.192}}स‚मीक्ष्य ग‚म‚क‚त्वं हि व्य‚प‚देशो न गृह्य‚ते ।&त‚च्चाक्ष‚व्य‚प‚देशेस्ति त‚द्ध‚र्म‚श्च नियोज्य‚ताम् ॥ १९२ ॥\&[\smallbreak]


	
	    \end{quote}
	  
	  \endgroup
	

	  \pstart \leavevmode% starting standard par
	\hphantom{.}‚{\color{DodgerBlue3}‚ग‚म‚क‚त्वं स‚मीक्ष्य} प‚रिभाव्य । ‚{\color{DodgerBlue3}‚त‚च्च ग‚म‚क‚त्व‚म‚क्षेण व्य‚प‚{\tiny $_{1}$}‚देशे} प्र‚त्य‚क्ष‚मित्य‚{\tiny $_{lb}$}‚त्रास्ति\edtext{}{\edlabel{pvv.176-1}\label{pvv.176-1}\lemma{त्रास्ति}\Bfootnote{रूप‚श‚ब्दादेः ।}} त‚स्य ग‚म‚क‚त्व‚स्य व्याप‚क‚स्य ‚{\color{DodgerBlue3}‚ध‚र्म्मो} व्याप्य‚भूतो ‚{\color{DodgerBlue3}‚नियोज्य‚तां ।}\edtext{}{\edlabel{pvv.176-2}\label{pvv.176-2}\lemma{भूतो}\Bfootnote{अर्ह‚ता ।}}(१९२)
	\pend% ending standard par
      \label{div_pvv.2.193}
	  
	% new div opening: depth here is 2
	
	  \bigskip
	  \begingroup
	
	    \large
	  
	    \begin{quote}
	  
	    
	    \stanza[\smallbreak]
	\label{pv.2.193}\flagstanza{\tiny\textenglish{....2.193}}त‚तो लिङ्ग‚स्व‚भावोत्र व्य‚प‚देशे नियोज्य‚ताम् ।&निव‚र्त्त‚ते व्याप‚क‚स्य स्व‚भाव‚स्य निवृत्तितः ॥ १९३ ॥\&[\smallbreak]


	
	    \end{quote}
	  
	  \endgroup
	

	  \pstart \leavevmode% starting standard par
	\hphantom{.}‚{\color{DodgerBlue3}‚त‚तो} व्याप\edtext{}{\edlabel{pvv.176-3}\label{pvv.176-3}\lemma{व्याप}\Bfootnote{पूर्व्वार्द्धेनान्व‚यो द्वितीयेन व्य‚तिरेकः ।}}काभावात् ‚{\color{DodgerBlue3}‚व्य‚प‚देशे} ध‚र्म्मिणि ‚{\color{DodgerBlue3}‚अत्र} ग‚म‚क‚त्वे साध्ये ‚{\color{DodgerBlue3}‚नियोज्य‚तां} लिङ्गं । प्र‚तिविष‚य‚मिति व्य‚प‚देशात् ‚{\color{DodgerBlue3}‚व्याप‚क‚स्य} ग‚म‚क‚त्व‚स्य ‚{\color{DodgerBlue3}‚निवृत्तितो निव‚र्त‚ते} नियोज्य‚तेति व्याप‚कानुप‚ल‚ब्ध्या त‚त्र नियोज्य‚त्वाभावः सिद्धः । (१९३)
	\pend% ending standard par
      \label{div_pvv.2.194}
	  
	% new div opening: depth here is 2
	

	  \begin{center}%% label @type='head'
	\textbf{क. अक्षाणां ग‚म‚क‚त्वात् प्र‚त्य‚क्ष‚म्}
	\end{center}
	
	  \bigskip
	  \begingroup
	
	    \large
	  
	    \begin{quote}
	  
	    
	    \stanza[\smallbreak]
	\label{pv.2.194}\flagstanza{\tiny\textenglish{....2.194}}स‚ञ्चितः स‚मुदायः स सामान्यं त‚त्र चाक्ष‚धीः ।&सामान्य‚बुद्धिश्चाव‚श्यं विक‚ल्पेनानुब‚ध्य‚ते ॥ १९४ ॥\&[\smallbreak]


	
	    \end{quote}
	  
	  \endgroup
	

	  \pstart \leavevmode% starting standard par
	\hphantom{.}न‚नु स‚ञ्चिताल‚म्ब‚नाः प‚ञ्च विज्ञान‚काया इति सिद्धान्तः । त‚त्रा\edtext{}{\edlabel{pvv.176-4}\label{pvv.176-4}\lemma{त्रा}\Bfootnote{य‚च्च व‚सुब‚न्धुनोक्तं । आय‚त‚न‚स्व‚ल‚क्ष‚णं च‚क्षुर्ग्राह्य‚त्वादि त‚त्प्र‚तिज्ञानानि स्व‚ल‚क्ष‚ण‚विष‚याणि, न द्र‚व्यं स्व‚ल‚क्ष‚णं प्र‚ति । एक‚प‚र‚माणु ।}}नेकार्थ‚{\tiny $_{lb}$}‚ज‚न्य‚त्वात् स्वार्थे सामान्य‚गोत‚र मिति चोक्तं\edtext{}{\edlabel{pvv.176-5}\label{pvv.176-5}\lemma{चोक्तं}\Bfootnote{\href{http://sarit.indology.info/?cref=ps.1.4cd}{[प्र‚माण] स‚मुच्च‚ये} ।}} । त‚था च प‚र‚माणूनां ‚{\color{DodgerBlue3}‚स‚मुदा\edtext{}{\edlabel{pvv.176-6}\label{pvv.176-6}\lemma{मुदा}\Bfootnote{रूप‚श‚ब्दादेः अष्ट‚द्र‚व्य‚त्वात् ।}}यः} \leavevmode\ledsidenote{\textenglish{177/s}} स‚ञ्चित इत्युच्य‚ते । ‚{\color{DodgerBlue3}‚स} एव च सामान्ये म‚तः ‚{\color{DodgerBlue3}‚त‚त्र च} सामान्येऽ‚{\color{DodgerBlue3}‚क्ष‚धी}‚र्ज्जाय‚ते (।) ‚{\tiny $_{lb}$}‚‚{\color{DodgerBlue3}‚सामान्य‚बुद्धिश्चाव‚श्यं विक\edtext{}{\edlabel{pvv.177-1}\label{pvv.177-1}\lemma{विक}\Bfootnote{सामान्य‚विष‚याऽक्ष‚धीः स‚विक‚ल्पा प‚र‚स्य ।}}ल्पेनानुब‚ध्य‚ते} अनुसीव्य‚ते । (१९४)
	\pend% ending standard par
      \label{div_pvv.2.195}
	  
	% new div opening: depth here is 2
	

	  \pstart \leavevmode% starting standard par
	त‚त्क‚थ\edtext{}{\edlabel{pvv.177-2}\label{pvv.177-2}\lemma{थ}\Bfootnote{य‚दि रूप‚श‚ब्दादिस‚मुदायाल‚म्ब‚ना अपि प‚ञ्च विज्ञान‚कायाः क‚थ‚मेषां स्व‚ल‚क्ष‚ण‚विष‚य‚त्वं न व्याहृतं क‚ल्प‚नापोह‚त्व‚ञ्च ।}}म‚विक‚ल्पं प्र‚त्य‚क्ष‚मुच्य‚ते ॥
	\pend% ending standard par
      

	  \pstart \leavevmode% starting standard par
	अत्रा\edtext{}{\edlabel{pvv.177-3}\label{pvv.177-3}\lemma{अत्रा}\Bfootnote{पूर्व्व‚प‚क्ष‚द्व‚ये बौद्धः ।}}ह (।)
	\pend% ending standard par
      
	  \bigskip
	  \begingroup
	
	    \large
	  
	    \begin{quote}
	  
	    
	    \stanza[\smallbreak]
	\label{pv.2.195}\flagstanza{\tiny\textenglish{....2.195}}अर्थान्त‚राभिस‚म्ब‚न्धाज्जाय‚न्ते येऽण‚वोऽप‚रे ।&उक्तास्ते स‚ञ्चितास्ते हि निमित्तं ज्ञान‚ज‚न्म‚नः ॥ १९५ ॥\&[\smallbreak]


	
	    \end{quote}
	  
	  \endgroup
	

	  \pstart \leavevmode% starting standard par
	\hphantom{.}‚{\color{DodgerBlue3}‚अर्थान्त‚राणां प‚र‚माण्व‚न्त‚राणाम‚भिस‚म्ब‚न्धात्\edtext{}{\edlabel{pvv.177-4}\label{pvv.177-4}\lemma{न्धात्}\Bfootnote{विज्ञान‚ज‚न‚न‚स‚म‚र्थ‚स्व‚भावोत्पाद‚न‚प्र‚त्य‚य‚स‚न्निधानात् ।}}} स‚न्निधान‚विशेषेणोप‚स‚र्प‚ण‚{\tiny $_{lb}$}‚प्र‚त्य‚येभ्यः पूर्व्व‚केभ्यः\edtext{}{\edlabel{pvv.177-5}\label{pvv.177-5}\lemma{केभ्यः}\Bfootnote{अस‚म‚र्थेभ्यः ।}}प‚र‚म‚स‚न्निहितेभ्योऽप‚रेन्ये‚{\color{DodgerBlue3}‚येऽण\edtext{}{\edlabel{pvv.177-6}\label{pvv.177-6}\lemma{येऽण}\Bfootnote{स‚म‚र्थाः प्र‚त्येकं । नान्य‚देव सामान्यं ।}}वो जाय‚न्ते} ते स‚ञ्चिता ‚{\color{DodgerBlue3}‚उक्ताः ‚{\tiny $_{lb}$}‚स‚ञ्चि}‚ताल‚म्ब‚ना‚{\tiny $_{3}$}‚ विज्ञान‚काय इत्यादौ । ‚{\color{DodgerBlue3}‚ज्ञान‚ज‚न्म\edtext{}{\edlabel{pvv.177-7}\label{pvv.177-7}\lemma{न्म}\Bfootnote{द्वितीयं प‚रिह‚र‚ति ।}}}‚न‚स्त एव हि ‚{\color{DodgerBlue3}‚निमित्त}‚मुक्ताः ‚{\tiny $_{lb}$}‚त‚त्रानेकार्थ‚ज‚न्य‚त्वादित्यादिना । (१९५)
	\pend% ending standard par
      \label{div_pvv.2.196}
	  
	% new div opening: depth here is 2
	
	  \bigskip
	  \begingroup
	
	    \large
	  
	    \begin{quote}
	  
	    
	    \stanza[\smallbreak]
	\label{pv.2.196}\flagstanza{\tiny\textenglish{....2.196}}अणूनां स विशेष‚श्च नान्त‚रेणाप‚रान‚णून ।&त‚देकानिय‚माज्ज्ञान‚मुक्तं सामान्य‚गोच‚र‚म् ॥ १९६ ॥\&[\smallbreak]


	
	    \end{quote}
	  
	  \endgroup
	

	  \pstart \leavevmode% starting standard par
	\hphantom{.}‚{\color{DodgerBlue3}‚अणूनां} स च ज्ञान‚ज‚न‚न‚साम‚र्थ्य‚ल‚क्ष‚णो ‚{\color{DodgerBlue3}‚विशेषोऽप‚रान‚णून}‚व्य‚व‚धान‚व‚र्त्तिनो\edtext{}{\edlabel{pvv.177-8}\label{pvv.177-8}\lemma{र्त्तिनो}\Bfootnote{स‚र्वेषां त‚त्साधार‚णं कार्य‚मित्य‚र्थः । न पुन‚राय‚त‚न‚सामान्य‚स्य ग्र‚ह‚णात् ।}}ऽन्त‚{\tiny $_{lb}$}‚रेण विना न भ‚व‚ति । न हि प्र‚त्येक‚म‚ण‚वो दृश्याः किं तु स‚हिता एव । ‚{\color{DodgerBlue3}‚त‚त्त‚स्मा‚{\tiny $_{lb}$}‚देक}‚स्मिन्न‚र्थे प‚र‚माणौ ज्ञान‚स्या‚{\color{DodgerBlue3}‚निय‚मात्\edtext{}{\edlabel{pvv.177-9}\label{pvv.177-9}\lemma{मात्}\Bfootnote{न द्र‚व्य‚स्व‚ल‚क्ष‚ण‚मिति व्याच‚ष्टे । निय‚मेनानुत्प‚त्तेः ।}} सामान्य‚गोच‚रं} संचित‚प‚र‚माणुसंघात‚विष‚यं ‚{\tiny $_{lb}$}‚‚{\color{DodgerBlue3}‚ज्ञान‚मुक्तं} त‚त्त्व‚वादिना । न तु प‚र‚माण्व‚तिरिक्त‚सामान्य‚{\tiny $_{4}$}‚विष‚यं । त‚त्क‚थं सामान्य‚{\tiny $_{lb}$}‚विष‚य‚त्वात् स‚विक‚ल्प‚त्व‚प्र‚स‚ङ्गः ॥ (१९६)
	\pend% ending standard par
      \label{div_pvv.2.197}
	  
	% new div opening: depth here is 2
	
	  \bigskip
	  \begingroup
	
	    \large
	  
	    \begin{quote}
	  
	    
	    \stanza[\smallbreak]
	\label{pv.2.197}\flagstanza{\tiny\textenglish{....2.197}}अथैकाय‚त‚न‚त्वेपि नानेकं दृश्य‚ते स‚कृत् ।&स‚कृद्ग्र‚हाव‚भासः किं वियुक्तेषु तिलादिषु ॥ १९७ ॥\&[\smallbreak]


	
	    \end{quote}
	  
	  \endgroup
	

	  \pstart \leavevmode% starting standard par
	अथैकेन्द्रिय‚ज्ञान‚ज\edtext{}{\edlabel{pvv.177-10}\label{pvv.177-10}\lemma{ज}\Bfootnote{न विष‚य‚व्य‚प‚देशि म‚नोज्ञान‚स्यापि विष‚य‚ज्ञान‚त्वात् । असाधार‚ण‚हेतुत्त्वाद‚क्षैस्त‚द् व्य‚प‚दिश्य‚ते इति दिग्नागः । अव‚य‚विद्र‚व्य‚मेकं ज‚न‚कं न ब‚ह‚वः ।}}न‚क‚त्वात् नील‚पीतादीना‚{\color{DodgerBlue3}‚मेकाय‚त‚न‚त्वे} रूपाय‚त‚न‚त्व‚{\tiny $_{lb}$}‚संग्र‚हेपि ‚{\color{DodgerBlue3}‚नानेकं} नीलादि ‚{\color{DodgerBlue3}‚स‚कृद् दृश्य‚ते} किन्तु क्र‚मेण त‚त्क‚थ‚म‚णूना ‚{\color{DodgerBlue3}‚ब‚हूनामेक‚दा} \leavevmode\ledsidenote{\textenglish{178/s}} ‚{\color{DodgerBlue3}‚ग्र‚ह‚णं ।\edtext{\textsuperscript{*}}{\edlabel{pvv.178-1}\label{pvv.178-1}\lemma{*}\Bfootnote{येन सामान्य‚विष‚य‚त्वं स्यात् ।}} अत्रोच्य‚ते} । य‚दि नानेक‚मेक‚दा गृह्य‚ते त‚दा ‚{\color{DodgerBlue3}‚तिलादिषु वियु\edtext{}{\edlabel{pvv.178-2}\label{pvv.178-2}\lemma{वियु}\Bfootnote{माष‚मुद्गादेः संयोग‚स्य स‚ह‚कारिणो नियुक्त‚त्वेऽभावान्नाव‚य‚वी ।}}क्तेषु} विभिन्न‚{\tiny $_{lb}$}‚‚{\color{DodgerBlue3}‚देशेषु स‚कृद्ग्र}‚हाव‚{\color{DodgerBlue3}‚भासो} युग‚प‚द् ग्र‚ह‚णानुभ‚वः किं क‚स्माद्धे तोः । (१९७)
	\pend% ending standard par
      \label{div_pvv.2.198}
	  
	% new div opening: depth here is 2
	

	  \pstart \leavevmode% starting standard par
	ज्ञानानां ल‚घुवृत्तित्वात् स‚कृद् ग्र‚ह‚{\tiny $_{5}$}‚ण‚भ्र‚म‚श्चेत् । आह (।)
	\pend% ending standard par
      
	  \bigskip
	  \begingroup
	
	    \large
	  
	    \begin{quote}
	  
	    
	    \stanza[\smallbreak]
	\label{pv.2.198}\flagstanza{\tiny\textenglish{....2.198}}प्र‚त्युक्तं लाघ‚व‚ञ्चात्र तेष्वेव क्र‚म‚पातिषु ।&किं नाक्र‚म‚ग्र‚ह‚स्तुल्य‚कालाः स‚र्व्वाश्च बुद्ध‚यः ॥ १९८ ॥\&[\smallbreak]


	
	    \end{quote}
	  
	  \endgroup
	

	  \pstart \leavevmode% starting standard par
	\hphantom{.}‚{\color{DodgerBlue3}‚प्र‚त्युक्तं} प्र‚तिक्षिप्तं ‚{\color{DodgerBlue3}‚चात्र} स‚कृद्ग्र‚हाव‚भासे ‚{\color{DodgerBlue3}‚लाघ‚वं} बुद्धीनां । अन्य‚त्रापि स‚मानं ‚{\tiny $_{lb}$}‚त‚द्व‚र्ण्ण‚योर्व्वा स‚कृच्छ्रुति\edtext{}{\edlabel{pvv.178-3}\label{pvv.178-3}\lemma{कृच्छ्रुति}\Bfootnote{युग‚प‚त्प‚ञ्च‚ज्ञान‚साध‚ने ।}}(२।१३५)रित्यादिना । त‚थाप्युच्य‚ते । ‚{\color{DodgerBlue3}‚तेष्वेव} तिलादिषु ‚{\tiny $_{lb}$}‚ह‚स्तादिभ्यः ‚{\color{DodgerBlue3}‚क्र‚म‚पातिषु किं} क‚स्मा‚{\color{DodgerBlue3}‚न्नाक्र‚म‚ग्र‚ह‚णं} भ‚व‚ति । ‚{\color{DodgerBlue3}‚स‚र्व्वाश्च बुद्ध‚यः\edtext{}{\edlabel{pvv.178-4}\label{pvv.178-4}\lemma{यः}\Bfootnote{क्ष‚णिक‚त्वात् ।}}} स‚हा‚{\tiny $_{lb}$}‚व‚स्थितेषु स‚म्भ‚व‚न्त्य‚{\color{DodgerBlue3}‚स्तुल्य‚कालाः} । (१९८)
	\pend% ending standard par
      \label{div_pvv.2.199}
	  
	% new div opening: depth here is 2
	

	  \pstart \leavevmode% starting standard par
	त‚तः (।)
	\pend% ending standard par
      
	  \bigskip
	  \begingroup
	
	    \large
	  
	    \begin{quote}
	  
	    
	    \stanza[\smallbreak]
	\label{pv.2.199}\flagstanza{\tiny\textenglish{....2.199}}काश्चित्तास्व‚क्र‚माभासाः क्र‚म‚व‚त्योप‚राश्च किम् ।&स‚र्वार्थ‚ग्र‚ह‚णे त‚स्माद‚क्र‚मोयं प्र‚स‚ज्य‚ते ॥ १९९ ॥\&[\smallbreak]


	
	    \end{quote}
	  
	  \endgroup
	

	  \pstart \leavevmode% starting standard par
	\hphantom{.}‚{\color{DodgerBlue3}‚तासु काश्चिद‚क्र‚माभासायाः} स‚ह‚स्थित‚व‚स्तुविष‚याया अप‚राश्च बुद्ध‚यः ‚{\tiny $_{lb}$}‚\leavevmode\ledsidenote{\textenglish{35a/MA}} ‚{\color{DodgerBlue3}‚क्र‚म‚व‚त्यो}‚ऽयुग‚प‚त्प्र‚तिभासाः ‚{\color{DodgerBlue3}‚किं} भ‚व‚न्ति याः क्र‚म‚पातिव‚स्तुविष‚याः‚{\tiny $_{6}$}‚ (।)अस्ति चायं ‚{\tiny $_{lb}$}‚भेदः । ‚{\color{DodgerBlue3}‚त‚स्मात्स‚र्व्व}‚स्यार्थ‚स्य क्र‚मिणोऽक्र‚मिण‚श्च ‚{\color{DodgerBlue3}‚ग्र‚ह‚णेऽक्र‚मोऽयं} लाघ‚वाविशेषा‚{\tiny $_{lb}$}‚‚{\color{DodgerBlue3}‚त्प्र‚स‚ज्य‚ते} । (१९९)
	\pend% ending standard par
      \label{div_pvv.2.200}
	  
	% new div opening: depth here is 2
	

	  \pstart \leavevmode% starting standard par
	किञ्च (।)
	\pend% ending standard par
      
	  \bigskip
	  \begingroup
	
	    \large
	  
	    \begin{quote}
	  
	    
	    \stanza[\smallbreak]
	\label{pv.2.200}\flagstanza{\tiny\textenglish{....2.200}}नैकं चित्र‚प‚त‚ङ्गादि रूपं वा दृश्य‚ते क‚थ‚म् ।&चित्रं त‚देक‚मिति चेदिदं चित्र‚त‚र‚न्त‚तः ॥ २०० ॥\&[\smallbreak]


	
	    \end{quote}
	  
	  \endgroup
	

	  \pstart \leavevmode% starting standard par
	\hphantom{.}‚{\color{DodgerBlue3}‚चित्र‚प‚त‚ङ्गादि नैक‚म}‚नेकं नीलादिरूपं ‚{\color{DodgerBlue3}‚वा दृश्य‚ते क‚थं} य‚दि नानेक‚मेकेन गृह्य‚ते । ‚{\tiny $_{lb}$}‚‚{\color{DodgerBlue3}‚चित्रं} नील‚पीताद्यात्म‚कं त‚त्प‚त‚ङ्गादिक‚{\color{DodgerBlue3}‚मेक‚मिति\edtext{}{\edlabel{pvv.178-5}\label{pvv.178-5}\lemma{मिति}\Bfootnote{स‚कृद्द‚र्श‚न‚म‚विरुद्धं ।}} चेत्} इद‚ञ्चित्र‚{\color{DodgerBlue3}‚मेकं} य‚दुच्य‚ते त‚{\color{DodgerBlue3}‚त्त‚त}‚{\tiny $_{lb}$}‚श्चित्र‚प‚त‚ङ्गाद‚पि ‚{\color{DodgerBlue3}‚चित्र‚त‚र‚मा}‚श्च‚र्य‚त‚रं । चित्र‚मिति नानारूपाणि त‚देव पुन‚रेक‚{\tiny $_{lb}$}‚मुच्य‚त इत्युप‚ह‚स‚ति (॥ २००)
	\pend% ending standard par
      \label{div_pvv.2.201}
	  
	% new div opening: depth here is 2
	

	  \pstart \leavevmode% starting standard par
	त‚था च (।)
	\pend% ending standard par
      
	  \bigskip
	  \begingroup
	
	    \large
	  
	    \begin{quote}
	  
	    
	    \stanza[\smallbreak]
	\label{pv.2.201}\flagstanza{\tiny\textenglish{....2.201}}नैकं स्व‚भावं चित्रं हि म‚णिरूपं य‚थैव त‚त् ।&नीलादि प्र‚तिभास‚श्च तुल्य‚श्चित्र‚प‚टादिषु ॥ २०१ ॥\&[\smallbreak]


	
	    \end{quote}
	  
	  \endgroup
	\textsuperscript{\textenglish{179/s}}

	  \pstart \leavevmode% starting standard par
	\hphantom{.}‚{\color{DodgerBlue3}‚चित्र‚म}‚नेक‚रूपं ‚{\color{DodgerBlue3}‚हि}‚{\tiny $_{1}$}‚ य‚स्मात्त‚स्मान्नैकं प‚त‚ङ्गादि । ‚{\color{DodgerBlue3}‚य‚थैव} संस्थान‚विशेषेण स‚न्नि\edtext{}{\edlabel{pvv.179-1}\label{pvv.179-1}\lemma{न्नि}\Bfootnote{द्र‚व्याणि द्र‚व्यान्त‚र‚मार‚भ‚न्ते गुणा गुणान्त‚र‚म‚त्र तु नानारूपा म‚ण‚यः ।}} ‚{\tiny $_{lb}$}‚विष्टानां ब‚हूनां ‚{\color{DodgerBlue3}‚म‚णी}‚नां ‚{\color{DodgerBlue3}‚रूपं त}‚च्चित्र‚म‚नेकं नैक‚म‚व‚य‚वि द्र‚व्यं विजातीयानां द्र‚व्याना‚{\tiny $_{lb}$}‚र‚म्भात् । चित्र‚बुद्धिरेक‚त्वान्मुख्या प‚त‚ङ्गे\edtext{}{\edlabel{pvv.179-2}\label{pvv.179-2}\lemma{ङ्गे}\Bfootnote{एकोत्राव‚य‚वी । अव‚य‚वास्तु विभिन्नाः दृश्य‚न्ते ।}}म‚णिरूपादिषु पुन‚रुप‚च‚रितेति चेत् । आह(।) ‚{\tiny $_{lb}$}‚नीलादिप्र‚तिभास‚श्चित्र‚प्र‚तिभासः स ‚{\color{DodgerBlue3}‚चित्र‚प‚ट आदि}‚र्येषां म‚णिरूपादीनां तेषु चित्र‚प‚त‚ङ्गे ‚{\tiny $_{lb}$}‚‚{\color{DodgerBlue3}‚च तुल्यः} । न त्व‚ग्निमाण‚व‚क‚योर्द‚ह‚न‚बुद्धिरिव स्ख‚ल‚द‚स्ख‚ल‚द्वृत्तिर्ल‚क्ष्य‚ते । (२०१)
	\pend% ending standard par
      \label{div_pvv.2.202}
	  
	% new div opening: depth here is 2
	
	  \bigskip
	  \begingroup
	
	    \large
	  
	    \begin{quote}
	  
	    
	    \stanza[\smallbreak]
	\label{pv.2.202}\flagstanza{\tiny\textenglish{....2.202}}त‚त्राव‚य‚व‚रूप‚ञ्चैत् केव‚लं दृश्य‚ते त‚था ।&नीलादीनि निर‚स्यान्य‚च्चित्रं चित्रं य‚दीक्ष‚से ॥ २०२ ॥\&[\smallbreak]


	
	    \end{quote}
	  
	  \endgroup
	

	  \pstart \leavevmode% starting standard par
	\hphantom{.}‚{\color{DodgerBlue3}‚त‚त्र} चित्र‚प‚{\tiny $_{2}$}‚टादिषु केव‚ल‚म‚व‚{\color{DodgerBlue3}‚य‚व‚रूपं त‚था चित्र‚त‚या दृश्य‚ते} नाव‚य‚वी विजातीया‚{\tiny $_{lb}$}‚नां द्र‚व्यानार‚म्भादिति चेत् । चित्र‚प‚त‚ङ्गादाव‚पि ‚{\color{DodgerBlue3}‚नीलादीनि निर‚स्य} पृथ‚क्कृत्य ‚{\color{DodgerBlue3}‚तेभ्यो‚{\tiny $_{lb}$}‚ऽन्य‚च्चित्र}‚म‚व‚य‚विरूपं ‚{\color{DodgerBlue3}‚य‚दीक्ष‚से} त्वं त‚{\color{DodgerBlue3}‚च्चित्र}‚माश्च‚र्यं । स्व‚सिद्धान्तानुराग‚भेष‚ज‚विशो‚{\tiny $_{lb}$}‚धित‚च‚क्षुरीक्ष‚से त्व‚मेव य‚दीदृश‚म‚व‚य‚विनं प‚रं नात्रान्येषाम‚धिका\edtext{}{\edlabel{pvv.179-3}\label{pvv.179-3}\lemma{धिका}\Bfootnote{इति स्व‚भावानुप‚ल‚म्भ उक्तः ।}}रः । (२०२)
	\pend% ending standard par
      \label{div_pvv.2.203}
	  
	% new div opening: depth here is 2
	

	  \pstart \leavevmode% starting standard par
	अपि च(।)
	\pend% ending standard par
      
	  \bigskip
	  \begingroup
	
	    \large
	  
	    \begin{quote}
	  
	    
	    \stanza[\smallbreak]
	\label{pv.2.203}\flagstanza{\tiny\textenglish{....2.203}}तुल्यार्थाकार‚काल‚त्वेनोप‚ल‚क्षित‚योर्द्व‚योः ।&नानार्था क्र‚म‚व‚त्येका किमेकार्थाक्र‚माप‚रा ॥ २०३ ॥\&[\smallbreak]


	
	    \end{quote}
	  
	  \endgroup
	

	  \pstart \leavevmode% starting standard par
	तुल्या\edtext{}{\edlabel{pvv.179-4}\label{pvv.179-4}\lemma{तुल्या}\Bfootnote{अर्थाकार‚श्च काल‚श्च ।}}र्थाकार‚त्वेन तुल्य‚काल‚त्त्वेन ‚{\color{DodgerBlue3}‚चोप‚ल‚क्षित‚योः} कृत्रिमाकृत्रिम‚प‚त‚ङ्ग‚{\tiny $_{lb}$}‚विशेष‚ण‚योर्द्व‚योर्म्म‚ध्ये ‚{\color{DodgerBlue3}‚एका}‚{\tiny $_{3}$}‚ कृत्रिम‚प‚त‚ङ्ग‚विष‚या ‚{\color{DodgerBlue3}‚धीर्नानार्था} विजाती‚{\tiny $_{lb}$}‚यात्म‚क‚द्र‚व्यानार‚म्भात् ‚{\color{DodgerBlue3}‚क्र‚म‚व‚ती} च नीलानां ब‚हूनां क्र‚मेण ग्र‚ह‚णात् । ‚{\color{DodgerBlue3}‚अप‚रा} अकृत्रिम‚प‚त‚ङ्ग‚विष‚या एकार्थाविय‚विविष‚या अत एवाक्र‚मा च किं क‚स्मादिष्य‚ते । ‚{\tiny $_{lb}$}‚द्व‚योर‚पि स‚मान‚ता युक्ता निमित्त‚स्य साम्यात् । (२०३)
	\pend% ending standard par
      \label{div_pvv.2.204}
	  
	% new div opening: depth here is 2
	

	  \pstart \leavevmode% starting standard par
	किञ्च ।
	\pend% ending standard par
      
	  \bigskip
	  \begingroup
	
	    \large
	  
	    \begin{quote}
	  
	    
	    \stanza[\smallbreak]
	\label{pv.2.204}\flagstanza{\tiny\textenglish{....2.204}}वैश्व‚रूप्याद्धियामेव भावानां विश्व‚रूप‚ता ॥&त‚च्चेद‚न‚ङ्गं केनेयं सिद्धा भेद‚व्य‚व‚स्थितिः ॥ २०४ ॥\&[\smallbreak]


	
	    \end{quote}
	  
	  \endgroup
	

	  \pstart \leavevmode% starting standard par
	\hphantom{.}‚{\color{DodgerBlue3}‚धियामेव वैश्व‚रू\edtext{}{\edlabel{pvv.179-5}\label{pvv.179-5}\lemma{रू}\Bfootnote{स‚म्विन‚ष्ट‚त्वाद्विष‚य‚स्थितेः ।}}प्यान्ना}‚नाकार‚त्वाद् ‚{\color{DodgerBlue3}‚भावानां} ग्राह्यानां (? णां) ‚{\color{DodgerBlue3}‚विश्व‚रूप‚ता} व्य‚व‚स्थाप्य‚ते । चित्र‚स्याव‚य‚विन\edtext{}{\edlabel{pvv.179-6}\label{pvv.179-6}\lemma{विन}\Bfootnote{चित्र‚प‚त‚ङ्गे ।}} एक‚तास्वीकारे त‚द्बुद्धिषु प्र‚तिभास‚नानात्वं ‚{\tiny $_{lb}$}‚भेद‚व्य‚{\tiny $_{4}$}‚व‚स्थिता‚{\color{DodgerBlue3}‚व‚न‚ङ्गं चेत्} । त‚दा भावानां ‚{\color{DodgerBlue3}‚भेद‚व्य‚व‚स्थिति}‚र‚प‚ह‚नूयेतैव । ‚{\color{DodgerBlue3}‚केना-} न्येन निब‚न्धेन ‚{\color{DodgerBlue3}‚सिद्धा} भ‚विष्य‚ति । (२०४)
	\pend% ending standard par
      \label{div_pvv.2.205}
	  
	% new div opening: depth here is 2
	\textsuperscript{\textenglish{180/s}}

	  \pstart \leavevmode% starting standard par
	अपि च\edtext{}{\edlabel{pvv.180-1}\label{pvv.180-1}\lemma{च}\Bfootnote{एकोऽव‚य‚वी य‚दि ज्ञान‚ज‚न‚को न नीलाद‚य‚स्त‚दा ।}} (।)
	\pend% ending standard par
      
	  \bigskip
	  \begingroup
	
	    \large
	  
	    \begin{quote}
	  
	    
	    \stanza[\smallbreak]
	\label{pv.2.205}\flagstanza{\tiny\textenglish{....2.205}}विजातीनाम‚नार‚म्भादालेख्यादौ न चित्र‚धीः ।&अरूप‚त्वान्न संयोग‚श्चित्रो भ‚क्तेश्च नाश्र‚यः ॥ २०५ ॥\&[\smallbreak]


	
	    \end{quote}
	  
	  \endgroup
	

	  \pstart \leavevmode% starting standard par
	\hphantom{.}‚{\color{DodgerBlue3}‚विजातीनां} भिन्न‚जातीनां राग‚द्र‚व्याणां कार्य‚द्र‚व्या‚{\color{DodgerBlue3}‚नार‚म्भात् आलेख्यादौ ‚{\tiny $_{lb}$}‚चित्र‚धीर्न} स्यात् ॥ आलेख्यं संयोग‚स्त‚स्य चित्रं रूप‚मिति चेत् (।)‚{\color{DodgerBlue3}‚न} (।)‚{\color{DodgerBlue3}‚संयोगो}‚पि ‚{\tiny $_{lb}$}‚चित्र‚स्त‚स्या‚{\color{DodgerBlue3}‚रूप‚त्वात्} ।\edtext{\textsuperscript{*}}{\edlabel{pvv.180-2}\label{pvv.180-2}\lemma{*}\Bfootnote{संयोग‚स्य रूप‚त्वे रूपे रूपं स्यात् ।}} संयोगो गुण‚स्त‚था रूप‚ञ्च । न च गुणे गुणान्त‚र‚म‚स्ति ॥
	\pend% ending standard par
      

	  \pstart \leavevmode% starting standard par
	स्यादेत‚त्(।)य‚था त‚रुषु संख्याल‚क्ष‚णं व‚नं कुसुमित‚त्वं चात एकार्थ‚स‚म‚वा‚{\tiny $_{lb}$}‚यात् व‚ने कुसुमित‚बुद्धिः । त‚थाव‚य‚वेषु चित्र‚संयोग‚योः स‚म‚वायाच्चित्रं चित्र‚मिति ‚{\tiny $_{lb}$}‚बुद्धिरुप‚चारादित्यादि भ‚क्तेरुप‚चार‚स्य च संयोग ‚{\color{DodgerBlue3}‚आश्र‚यो न} युक्तः । (२०५)
	\pend% ending standard par
      \label{div_pvv.2.206}
	  
	% new div opening: depth here is 2
	
	  \bigskip
	  \begingroup
	
	    \large
	  
	    \begin{quote}
	  
	    
	    \stanza[\smallbreak]
	\label{pv.2.206}\flagstanza{\tiny\textenglish{....2.206}}प्र‚त्येक‚म‚विचित्र‚त्वाद् गृहीतेषु क्र‚मेण च ।&न चित्र‚धीस‚ङ्क‚ल‚न‚म‚नेक‚स्यैक‚याऽग्र‚हात् ॥ २०६ ॥\&[\smallbreak]


	
	    \end{quote}
	  
	  \endgroup
	

	  \pstart \leavevmode% starting standard par
	\hphantom{.}‚{\color{DodgerBlue3}‚प्र‚त्येक}‚म‚व‚य(व)ानाम‚{\color{DodgerBlue3}‚विचित्र‚त्वात्} । नीलादिषु ‚{\color{DodgerBlue3}‚क्र‚मेण} स्व‚बुद्धिभि‚{\color{DodgerBlue3}‚र्गृहीतेषु}\edtext{}{\edlabel{pvv.180-3}\label{pvv.180-3}\lemma{बुद्धिभि}\Bfootnote{प‚श्चात् ।}} ‚{\tiny $_{lb}$}‚बुद्ध्य‚न्त‚रेण चित्र‚संक‚ल‚नं संक्षिप्य ग्र‚ह‚ण‚ञ्च न युक्तं । ‚{\color{DodgerBlue3}‚एक‚या} धियाऽ‚{\color{DodgerBlue3}‚नेक‚स्याग्र‚{\tiny $_{lb}$}‚हात्} । (२०६)
	\pend% ending standard par
      \label{div_pvv.2.207}
	  
	% new div opening: depth here is 2
	
	  \bigskip
	  \begingroup
	
	    \large
	  
	    \begin{quote}
	  
	    
	    \stanza[\smallbreak]
	\label{pv.2.207}\flagstanza{\tiny\textenglish{....2.207}}नानार्थैका भ‚वेत्त‚स्मात् सिद्धातोप्य‚विक‚ल्पिका ।&विक‚ल्प‚य‚न्नायेकार्थ य‚तोन्य‚द‚पि प‚श्य‚ति ॥ २०७ ॥\&[\smallbreak]


	
	    \end{quote}
	  
	  \endgroup
	

	  \pstart \leavevmode% starting standard par
	\hphantom{.}क‚थं नीलानामेक‚बुद्ध्या ‚{\color{DodgerBlue3}‚संक‚ल‚नं} । इष्टौ वा त‚स्मात्संक‚ल‚न‚स्वीकारादेवैका ‚{\tiny $_{lb}$}‚बुद्धिर्नाना‚{\tiny $_{6}$}‚र्थाऽनेक‚विष‚या भ‚वेत्\edtext{}{\edlabel{pvv.180-4}\label{pvv.180-4}\lemma{वेत्}\Bfootnote{य‚दुक्तमेकाय‚त‚न‚त्वेपि नानेकं दृश्य‚ते स‚कृदिति\cref{pv.2.197} त‚द् ध्व‚स्तं स्यात् ।}} । अतोऽनेकार्थ‚वेद‚नाद‚पि बुद्धिर‚{\color{DodgerBlue3}‚विक‚ल्पिका ‚{\tiny $_{lb}$}‚सिद्धा} । य‚तः श‚ब्द‚योजित‚{\color{DodgerBlue3}‚मेक‚म‚र्थं विक‚ल्प}‚य‚न्ना‚{\color{DodgerBlue3}‚न्य‚द}‚संयोजित‚म‚र्थान्त‚र‚{\color{DodgerBlue3}‚म‚पि प‚श्य‚ति} द्र‚ष्टा । न ह्येक‚दानेक‚श‚ब्द‚योज‚ना त‚स्माद‚र्थ‚स‚ञ्च‚य‚विष‚य‚त्वात् सामान्य‚विष‚य‚त्वं (।) ‚{\tiny $_{lb}$}‚त‚थापि त्व‚विक‚ल्पिते तेन विरोधः । (२०७)
	\pend% ending standard par
      \label{div_pvv.2.208}
	  
	% new div opening: depth here is 2
	

	  \begin{center}%% label @type='head'
	\textbf{ख. चित्रैक‚त्व‚चिन्ता}
	\end{center}
	

	  \pstart \leavevmode% starting standard par
	न‚नु (।)
	\pend% ending standard par
      
	  \bigskip
	  \begingroup
	
	    \large
	  
	    \begin{quote}
	  
	    
	    \stanza[\smallbreak]
	\label{pv.2.208}\flagstanza{\tiny\textenglish{....2.208}}चित्राव‚भासेष्व‚र्थेषु य‚द्यैक‚त्वं न युज्य‚ते ।&सैव ताव‚त् क‚थं बुद्धिरेका चित्राव‚भासिनी ॥ २०८ ॥\&[\smallbreak]


	
	    \end{quote}
	  
	  \endgroup
	\textsuperscript{\textenglish{181/s}}

	  \pstart \leavevmode% starting standard par
	\hphantom{.}‚{\color{DodgerBlue3}‚चित्रं} नानाकारोऽ\edtext{}{\edlabel{pvv.181-1}\label{pvv.181-1}\lemma{नानाकारोऽ}\Bfootnote{प‚रो दूष‚य‚ति पूर्व्वोक्तं ।}} ‚{\color{DodgerBlue3}‚व‚भासो} येषां प‚त‚ङ्गादीनां तेष्व‚र्थेष्वे‚{\color{DodgerBlue3}‚क‚त्वं} न युज्य‚ते । ‚{\color{DodgerBlue3}‚य‚दि ‚{\tiny $_{lb}$}‚सैव} चित्रार्थ‚ग्राहिणी बुद्धिश्चित्राव‚भासिनी चि‚{\tiny $_{7}$}‚त्रा\edtext{}{\edlabel{pvv.181-2}\label{pvv.181-2}\lemma{त्रा}\Bfootnote{प‚र‚मुखेनोप‚न्यासो ध‚र्म‚नैरात्म्य‚सूच‚नाय ।}}कारा ‚{\color{DodgerBlue3}‚क‚थ‚मेका} संम‚ता । य‚था ‚{\tiny $_{lb}$}‚चित्र‚त्वेपि बुद्धिरेका त‚था कार्य‚द्र‚व्य‚ञ्चैकं स्यात् । (२०८)\leavevmode\ledsidenote{\textenglish{35b/MA}}
	\pend% ending standard par
      \label{div_pvv.2.209}
	  
	% new div opening: depth here is 2
	

	  \pstart \leavevmode% starting standard par
	अत्राह (।)
	\pend% ending standard par
      
	  \bigskip
	  \begingroup
	
	    \large
	  
	    \begin{quote}
	  
	    
	    \stanza[\smallbreak]
	\label{pv.2.209a}\flagstanza{\tiny\textenglish{...2.209a}}इदं व‚स्तुब‚लायातं य‚द् व‚द‚न्ति विप‚श्चितः ।\&[\smallbreak]


	
	    \end{quote}
	  
	  \endgroup
	

	  \pstart \leavevmode% starting standard par
	इदं व‚स्तु\edtext{}{\edlabel{pvv.181-3}\label{pvv.181-3}\lemma{स्तु}\Bfootnote{तात्विकाकार‚वादिनोयं दोषो न म‚मेत्याह ।}}नोऽव्य‚भिचारि लिङ्ग‚स्य ‚{\color{DodgerBlue3}‚ब‚लादायातं य‚द्व\edtext{}{\edlabel{pvv.181-4}\label{pvv.181-4}\lemma{द्व}\Bfootnote{ध‚र्म‚नैरात्म्यं ।}}द‚न्ति विप‚श्चितो} बु द्धा ‚{\tiny $_{lb}$}‚भ‚ग‚व‚न्तः ।
	\pend% ending standard par
      

	  \pstart \leavevmode% starting standard par
	किं त‚दित्याह (।)
	\pend% ending standard par
      
	  \bigskip
	  \begingroup
	
	    \large
	  
	    \begin{quote}
	  
	    
	    \stanza[\smallbreak]
	\label{pv.2.209b}\flagstanza{\tiny\textenglish{...2.209b}}य‚था य‚थार्था चिन्त्य‚न्ते विशीर्य‚न्ते त‚था त‚था ॥ २०९ ॥\&[\smallbreak]


	
	    \end{quote}
	  
	  \endgroup
	

	  \pstart \leavevmode% starting standard par
	\hphantom{.}‚{\color{DodgerBlue3}‚य‚था य‚था} येन प्र‚कारेण एक‚त्वेनानेक‚त्वेन ‚{\color{DodgerBlue3}‚वाऽर्था} नीलाद‚यो बाह्य‚ज्ञानात्मानो ‚{\tiny $_{lb}$}‚वा ‚{\color{DodgerBlue3}‚विचिन्त्य‚न्ते त‚था} विशीर्य‚न्ते क्व‚चिद‚पि ‚{\color{DodgerBlue3}‚न} व्य‚व‚तिष्ठ‚न्त इति याव‚त् । न हि ‚{\tiny $_{lb}$}‚ज्ञान‚मेकं नानाकार‚त्वात् । त‚ल्ल‚क्ष‚ण‚त्वाच्च भेद‚स्य । ना‚{\tiny $_{1}$}‚प्य‚नेकं चित्र‚प्र‚ति‚{\tiny $_{lb}$}‚भासानुप‚प‚त्तेः । प‚र‚स्प‚र‚म‚वेद‚नात् । अन्य‚स्य च ग्राह‚क‚स्याभावात् । (२०९)
	\pend% ending standard par
      \label{div_pvv.2.210}
	  
	% new div opening: depth here is 2
	
	  \bigskip
	  \begingroup
	
	    \large
	  
	    \begin{quote}
	  
	    
	    \stanza[\smallbreak]
	\label{pv.2.210a}\flagstanza{\tiny\textenglish{...2.210a}}किं स्यात् सा चित्र‚तैक‚स्यां;\&[\smallbreak]


	
	    \end{quote}
	  
	  \endgroup
	

	  \pstart \leavevmode% starting standard par
	\hphantom{.}न‚नु य‚दि ‚{\color{DodgerBlue3}‚सा चित्र‚ता} बुद्धा‚{\color{DodgerBlue3}‚वेक‚स्यां} स्यात् त‚या च चित्र‚मे\edtext{}{\edlabel{pvv.181-5}\label{pvv.181-5}\lemma{मे}\Bfootnote{प्र‚ष्टुः साध्यं ।}}कं द्र‚व्यं ‚{\tiny $_{lb}$}‚व्य‚व‚स्थाप्येत (।) त‚दा ‚{\color{DodgerBlue3}‚कि} दूष‚णं ‚{\color{DodgerBlue3}‚स्यात्} ।
	\pend% ending standard par
      

	  \pstart \leavevmode% starting standard par
	आह (।)
	\pend% ending standard par
      
	  \bigskip
	  \begingroup
	
	    \large
	  
	    \begin{quote}
	  
	    
	    \stanza[\smallbreak]
	\label{pv.2.210b}\flagstanza{\tiny\textenglish{...2.210b}}न स्यात्त‚स्यां म‚ताव‚पि ।\&[\smallbreak]


	
	    \end{quote}
	  
	  \endgroup
	

	  \pstart \leavevmode% starting standard par
	\hphantom{.}‚{\color{DodgerBlue3}‚न} केव‚लं द्र‚व्ये त‚स्यां म‚ताव‚प्ये‚{\color{DodgerBlue3}‚क‚स्यां न स्या च्चित्र‚ता} । आकार‚नानात्व‚ल‚क्ष‚ण‚{\tiny $_{lb}$}‚त्वाद् भेद‚स्य । नानात्वेपि चित्र‚ता क‚थ‚म् (।) अनेक‚पुरुष‚प्र‚तीतिव‚त् ।
	\pend% ending standard par
      

	  \pstart \leavevmode% starting standard par
	क‚थ‚न्त‚र्हि प्र‚तीतिरित्याह (।)
	\pend% ending standard par
      
	  \bigskip
	  \begingroup
	
	    \large
	  
	    \begin{quote}
	  
	    
	    \stanza[\smallbreak]
	\label{pv.2.210c}\flagstanza{\tiny\textenglish{...2.210c}}य‚दीदं स्व‚य‚म‚र्थानां रोच‚ते त‚त्र के व‚य‚म् ॥ २१० ॥\&[\smallbreak]


	
	    \end{quote}
	  
	  \endgroup
	

	  \pstart \leavevmode% starting standard par
	\hphantom{.}‚{\color{DodgerBlue3}‚य‚दीद}‚म‚ताद्रूप्येपि ताद्रूप्य‚प्र‚थ‚{\tiny $_{2}$}‚न‚{\color{DodgerBlue3}‚म‚र्था}‚नां भास‚मानानां नीलादीनां ‚{\color{DodgerBlue3}‚स्व‚य‚म}‚प‚र‚{\tiny $_{lb}$}‚प्रेर‚ण‚या ‚{\color{DodgerBlue3}‚रोच‚ते । त‚त्र} त‚थाप्र‚तिभासे ‚{\color{DodgerBlue3}‚के व‚य‚म}‚स‚ह‚मानाऽ(? अ) पि निषेद्धुं । ‚{\tiny $_{lb}$}‚अव‚स्तु च प्र‚ति\edtext{}{\edlabel{pvv.181-6}\label{pvv.181-6}\lemma{ति}\Bfootnote{अविद्याव‚शादुप‚ल‚म्भः ।}}भास‚ते चेति व्य‚क्त‚मालीक्यं । (२१०)
	\pend% ending standard par
      \label{div_pvv.2.211}
	  
	% new div opening: depth here is 2
	\textsuperscript{\textenglish{182/s}}
	  \bigskip
	  \begingroup
	
	    \large
	  
	    \begin{quote}
	  
	    
	    \stanza[\smallbreak]
	\label{pv.2.211}\flagstanza{\tiny\textenglish{....2.211}}त‚स्मान्नार्थेषु न ज्ञाने स्थूलाभास‚स्त‚दात्म‚नः ।&एक‚त्र प्र‚तिषिद्ध‚त्वाद् ब‚हुष्व‚पि न स‚म्भ‚वः ॥ २११ ॥\&[\smallbreak]


	
	    \end{quote}
	  
	  \endgroup
	

	  \pstart \leavevmode% starting standard par
	\hphantom{.}‚{\color{DodgerBlue3}‚त‚स्मान्ना\edtext{}{\edlabel{pvv.182-1}\label{pvv.182-1}\lemma{स्मान्ना}\Bfootnote{एकानेकोभ‚य‚रूप‚र‚हिता स्व‚स‚म्वित्तिर‚विरुद्धैव ।}}र्थेषु बाह्येषु न ज्ञाने} त‚द्ग्राह‚के ‚{\color{DodgerBlue3}‚स्थूलाभासः} स्थूल आकारः संग‚च्छ‚ते । ‚{\tiny $_{lb}$}‚‚{\color{DodgerBlue3}‚त‚दात्म‚नः} स्थूल‚स्व‚रूप‚{\color{DodgerBlue3}‚स्यैक‚त्रा}‚व‚य‚वे प‚र‚माणौ वा ‚{\color{DodgerBlue3}‚प्र‚तिषिद्ध‚त्वात् । ब‚हुष्व‚पि} तेषु ‚{\tiny $_{lb}$}‚‚{\color{DodgerBlue3}‚स‚म्भ‚वो ना}‚स्ति मिलिता अपि हि त\edtext{}{\edlabel{pvv.182-2}\label{pvv.182-2}\lemma{त}\Bfootnote{प‚र‚माण‚व एव न स्थूलाः स्व‚रूप‚हानेः ।}} ‚{\color{DodgerBlue3}‚एव} । ते च प्र‚त्येकं स्थौल्य‚वि‚{\tiny $_{3}$}‚क‚ला इति ‚{\tiny $_{lb}$}‚स‚मुदिता अपि त‚थैव स्युः । त‚था नीलाद्याकारेषु प्र‚त्येकं चित्र‚स्य स्थौल्य‚स्याभावात् ‚{\tiny $_{lb}$}‚स‚मुदायेप्य‚भावः । (२११)
	\pend% ending standard par
      \label{div_pvv.2.212}
	  
	% new div opening: depth here is 2
	

	  \pstart \leavevmode% starting standard par
	न‚नु सुखाद्यात्म‚कं\edtext{}{\edlabel{pvv.182-3}\label{pvv.182-3}\lemma{कं}\Bfootnote{त‚त्त्व‚तो द्व‚य‚प्र‚तिभासि ।}} स्व‚प्र‚काशं विज्ञान‚मेक‚मिद‚मिति यो गा चा र म‚त‚म‚व्याह‚त‚{\tiny $_{lb}$}‚‚{\color{DodgerBlue3}‚मित्याह} (।)
	\pend% ending standard par
      
	  \bigskip
	  \begingroup
	
	    \large
	  
	    \begin{quote}
	  
	    
	    \stanza[\smallbreak]
	\label{pv.2.212}\flagstanza{\tiny\textenglish{....2.212}}प‚रिच्छेदोन्त‚र‚न्योऽयं भागो ब‚हिरिव स्थितः ।&ज्ञान‚स्याभेदिनौ भिन्नौ प्र‚तिभासो ह्युप‚प्ल‚वः ॥ २१२ ॥\&[\smallbreak]


	
	    \end{quote}
	  
	  \endgroup
	

	  \pstart \leavevmode% starting standard par
	\hphantom{.}‚{\color{DodgerBlue3}‚प‚रिच्छेदो} ग्राह‚काकारः सुखादे‚{\color{DodgerBlue3}‚र‚न्त‚र}‚ब‚हिर्देशे प‚रिच्छेदा‚{\color{DodgerBlue3}‚द‚न्योयं} भागो ग्राह्यो ‚{\tiny $_{lb}$}‚नीलादिर्ब्ब‚हिःस्थित ‚{\color{DodgerBlue3}‚इवा}‚भाति स‚र्व्वेषां । ‚{\color{DodgerBlue3}‚हि}‚र्य‚स्मात् ‚{\color{DodgerBlue3}‚ज्ञान‚स्याभेदिनौ भि}‚न्नावाकारौ ‚{\tiny $_{lb}$}‚त‚त्त्व‚तो ‚{\color{DodgerBlue3}‚न} युक्तौ (।) त‚स्माद‚न्त‚र्ब्ब‚हिर्देश‚{\tiny $_{4}$}‚स‚म्ब‚न्ध‚त‚या ‚{\color{DodgerBlue3}‚प्र‚तिभास उप‚प्ल‚वो न ‚{\tiny $_{lb}$}‚स‚त्यः} । (२१२)
	\pend% ending standard par
      \label{div_pvv.2.213}
	  
	% new div opening: depth here is 2
	
	  \bigskip
	  \begingroup
	
	    \large
	  
	    \begin{quote}
	  
	    
	    \stanza[\smallbreak]
	\label{pv.2.213}\flagstanza{\tiny\textenglish{....2.213}}त‚त्रैक‚स्याप्य‚भावेन द्व‚य‚म‚प्य‚व‚हीय‚ते ।&त‚स्मात्त‚देव त‚स्यापि त‚त्त्वं या द्व‚य‚शून्य‚ता ॥ २१३ ॥\&[\smallbreak]


	
	    \end{quote}
	  
	  \endgroup
	

	  \pstart \leavevmode% starting standard par
	\hphantom{.}‚{\color{DodgerBlue3}‚त‚त्र} एक‚ज्ञानात्म‚नि विरुद्धं द्व‚यं न युक्त‚मि‚{\color{DodgerBlue3}‚त्येक‚स्य} ग्राह्य‚त्व‚स्य ग्राह‚क‚त्व‚स्य ‚{\tiny $_{lb}$}‚वाव‚श्याभ्युप‚ग‚न्त‚व्ये‚{\color{DodgerBlue3}‚नाभावेन द्व‚य‚म‚प्य‚व‚हीय‚ते} । अन्योन्य‚सापेक्ष‚योरेकाभावेऽप‚रा‚{\tiny $_{lb}$}‚भाव‚स्य न्याय‚प्राप्त‚त्वात् । ‚{\color{DodgerBlue3}‚त‚स्मात्त‚स्य} ज्ञान‚स्या‚{\color{DodgerBlue3}‚पि त‚त्त्वं त‚देव या} द्व‚येन ग्राह्य‚ग्राह‚का‚{\tiny $_{lb}$}‚कारेण शून्य‚ता नाम । (२१३)
	\pend% ending standard par
      \label{div_pvv.2.214}
	  
	% new div opening: depth here is 2
	
	  \bigskip
	  \begingroup
	
	    \large
	  
	    \begin{quote}
	  
	    
	    \stanza[\smallbreak]
	\label{pv.2.214}\flagstanza{\tiny\textenglish{....2.214}}त‚द्भेदाश्र‚यिणी चेयं भावानां भेद‚संस्थितिः ।&त‚दुप‚प्ल‚व‚भावे च तेषां भेदोप्युप‚प्ल‚वः ॥ २१४ ॥\&[\smallbreak]


	
	    \end{quote}
	  
	  \endgroup
	

	  \pstart \leavevmode% starting standard par
	\hphantom{.}‚{\color{DodgerBlue3}‚इयं च भावानां} रूप‚वेद‚नादीनां ‚{\color{DodgerBlue3}‚भेद‚सं}‚स्थितिः (।) त‚स्य ग्राह्य‚ग्राह‚क‚स्य ‚{\color{DodgerBlue3}‚भेदः} स ‚{\tiny $_{lb}$}‚‚{\color{DodgerBlue3}‚आश्र‚यो}‚{\tiny $_{5}$}‚ य‚स्याः सा त‚था । त‚स्य ग्राह्य‚ग्राह‚क‚भाव‚स्य भेद‚व्य‚व‚स्थानिब‚न्ध‚न‚स्यो‚{\tiny $_{lb}$}‚‚{\color{DodgerBlue3}‚प‚प्ल‚व‚भावे} मिथ्यात्वे च तेषां रूपादीनां ‚{\color{DodgerBlue3}‚भेदो}‚ऽपि त‚द्व्य‚व‚स्थापित उप‚प्ल‚वः । ‚{\tiny $_{lb}$}‚न केव‚लं रूपादीनां भेदाभेदावुप‚प्ल‚वः । (२१४)
	\pend% ending standard par
      \label{div_pvv.2.215}
	  
	% new div opening: depth here is 2
	\textsuperscript{\textenglish{183/s}}

	  \pstart \leavevmode% starting standard par
	ल‚क्ष‚ण‚शून्य‚त्वान्निःस्व‚भाव‚त्व‚म‚पीत्याह (।)
	\pend% ending standard par
      
	  \bigskip
	  \begingroup
	
	    \large
	  
	    \begin{quote}
	  
	    
	    \stanza[\smallbreak]
	\label{pv.2.215}\flagstanza{\tiny\textenglish{....2.215}}न ग्राह्य‚ग्राह‚काकार‚बाह्य‚म‚स्ति च ल‚क्ष‚ण‚म् ।&अतो ल‚क्ष‚ण‚शून्य‚त्वान्निःस्व‚भावाः प्र‚काशिताः ॥ २१५ ॥\&[\smallbreak]


	
	    \end{quote}
	  
	  \endgroup
	

	  \pstart \leavevmode% starting standard par
	\hphantom{.}रूपादीनां ‚{\color{DodgerBlue3}‚ग्राह्य‚ग्राह‚काकारा}‚भ्यां ‚{\color{DodgerBlue3}‚बाह्यं} भिन्नं ‚{\color{DodgerBlue3}‚ल‚क्ष‚णं न चास्ति} । त‚था हि ‚{\tiny $_{lb}$}‚विष‚य‚त‚या किञ्चिन्निर्द्दिश्य‚ते । य‚था रूप्य‚त इति कृत्वा रूपं । किञ्चिद्विष‚यित‚या ‚{\tiny $_{lb}$}‚विजानातीत्यादि\edtext{}{\edlabel{pvv.183-1}\label{pvv.183-1}\lemma{विजानातीत्यादि}\Bfootnote{संजानातीति संज्ञा, अनुभ‚व‚तीत्य‚नुभ‚व इत्यादि ।}} व्युत्प‚त्त्या ‚{\tiny $_{6}$}‚य‚था रूपिणः स्क‚न्धाः । न त्वेत‚द्व्य‚तिरिक्तं ‚{\tiny $_{lb}$}‚किञ्चिल्ल‚क्ष‚ण‚म‚स्ति । ‚{\color{DodgerBlue3}‚अतो ल‚क्ष‚णेन} ग्राह्य‚ग्राह‚क‚त्वे\edtext{}{\edlabel{pvv.183-2}\label{pvv.183-2}\lemma{त्वे}\Bfootnote{न तु स‚म्वित्त्या ।}}न ‚{\color{DodgerBlue3}‚शून्य‚त्वान्निःस्व‚भावाः} स‚र्व्व‚ध‚र्म्माः ‚{\color{DodgerBlue3}‚प्र‚काशिता} भ‚ग‚व‚द््भिर्बु द्धैः । (२१५)
	\pend% ending standard par
      \label{div_pvv.2.216}
	  
	% new div opening: depth here is 2
	

	  \pstart \leavevmode% starting standard par
	किञ्च (।) ब‚हिर‚र्थ‚वादेपि ल‚क्ष‚ण‚शून्य‚त्वात् निःस्व‚भाव‚तां ध‚र्म्माणामाख्यातु‚{\tiny $_{lb}$}‚माह (।)
	\pend% ending standard par
      
	  \bigskip
	  \begingroup
	
	    \large
	  
	    \begin{quote}
	  
	    
	    \stanza[\smallbreak]
	\label{pv.2.216}\flagstanza{\tiny\textenglish{....2.216}}व्यापारोपाधिकं स‚र्वं स्क‚न्धादीनां विशेष‚तः ।&ल‚क्ष‚णं स च त‚त्त्व‚न्न तेनाप्येते विल‚क्ष‚णाः ॥ २१६ ॥\&[\smallbreak]


	
	    \end{quote}
	  
	  \endgroup
	

	  \pstart \leavevmode% starting standard par
	\hphantom{.}‚{\color{DodgerBlue3}‚स्क‚न्ध} आदिर्येषां धात्वाय‚त‚ना‚{\color{DodgerBlue3}‚नां} तेषां य‚ल्ल‚क्ष‚णं राशीभ‚व‚न्तीति राश्य‚र्थः ‚{\tiny $_{lb}$}‚स्क‚न्धानां (।) कार्योत्पाद‚क‚त्वेनाकारार्थो धातूनां । आय‚न्त‚न्व\edtext{}{\edlabel{pvv.183-3}\label{pvv.183-3}\lemma{न्व}\Bfootnote{चित्त‚चैत्तानामुत्प‚त्तिम्विस्तृण्व‚न्ति ।}}न्तीत्याय‚द्वारार्थ ‚{\tiny $_{lb}$}‚आय‚त‚नानां‚{\tiny $_{7}$}‚ (।) ‚{\color{DodgerBlue3}‚त‚त्स‚र्वं ल‚क्ष‚णं व्यापारोपाधिकं} व्यापार‚विशेष‚णं । ‚{\color{DodgerBlue3}‚स च} व्यापारो-\leavevmode\ledsidenote{\textenglish{36a/MA}} ‚{\tiny $_{lb}$}‚ऽप्र‚तीतेर्भेद‚प‚क्षे न त‚त्त्वं । अभेदेपि प्राग‚भावी भाव एव स्यात् । स च त‚त्त्वं न ‚{\tiny $_{lb}$}‚भ‚व‚ति । अश‚क्तं स‚र्व‚मिति चेदि \href{http://sarit.indology.info/?cref=pv.2.4}{(२।४)} त्य‚त्रोक्त‚क्र‚मात् । ‚{\color{DodgerBlue3}‚तेन} व्यापारोपा‚{\tiny $_{lb}$}‚धिक‚ल‚क्ष‚णायोगेनापि ‚{\color{DodgerBlue3}‚विल‚क्ष‚णा} निःस्व‚{\color{DodgerBlue3}‚भावा एते} स्क‚न्धाद‚यः । (२१६)
	\pend% ending standard par
      \label{div_pvv.2.217}
	  
	% new div opening: depth here is 2
	

	  \pstart \leavevmode% starting standard par
	क‚थं त‚र्हि बाह्य‚स्क‚न्धादिदेश‚ना भ‚ग‚व‚तामित्याह (।)
	\pend% ending standard par
      
	  \bigskip
	  \begingroup
	
	    \large
	  
	    \begin{quote}
	  
	    
	    \stanza[\smallbreak]
	\label{pv.2.217}\flagstanza{\tiny\textenglish{....2.217}}य‚थास्वंप्र‚त्य‚यापेक्षाद‚विद्योप‚प्लुतात्म‚नाम् ।&विज्ञ‚प्तिर्वित‚थाकारा जाय‚ते तिमिरादिव‚त् ॥ २१७ ॥\&[\smallbreak]


	
	    \end{quote}
	  
	  \endgroup
	

	  \pstart \leavevmode% starting standard par
	\hphantom{.}अनाद्य‚{\color{DodgerBlue3}‚विद्योप‚प्लुतात्म}‚नाम‚प्र‚हीणाक्लिष्ट‚ज्ञा\edtext{}{\edlabel{pvv.183-4}\label{pvv.183-4}\lemma{ज्ञा}\Bfootnote{ग्राह्य‚ग्राह‚काभिनिविष्टानां ।}}नानां पुंसां य\edtext{}{\edlabel{pvv.183-5}\label{pvv.183-5}\lemma{य}\Bfootnote{बाह्यान‚पेक्ष‚त्वे स‚दा स्यादित्याह ।}}थास्वं य‚स्य ‚{\tiny $_{lb}$}‚भ्र‚म‚स्य य अत्मीयः\edtext{}{\edlabel{pvv.183-6}\label{pvv.183-6}\lemma{अत्मीयः}\Bfootnote{स‚न्तान‚प‚रिणाम उत्प‚त्तिनिमित्तं ।}} प्र‚त्य‚यो ‚{\color{DodgerBlue3}‚य‚थास्वंप्र‚त्य}‚य‚स्त‚{\tiny $_{1}$}‚स्या‚{\color{DodgerBlue3}‚पेक्ष‚ण}‚म‚पेक्षः । त‚स्माद्वित‚थौ ‚{\tiny $_{lb}$}‚ग्राह्य‚ग्राह‚काकारौ य‚स्याः सा तादृशी ‚{\color{DodgerBlue3}‚विज्ञ‚प्तिर्जाय‚ते । तिमिरादिव‚त्} तिमिरा‚{\tiny $_{lb}$}‚दाविव, ‚{\color{DodgerBlue3}‚वित‚थाकार}‚च‚न्द्र‚द्व‚यादिविज्ञाप्तिः । (२१७)
	\pend% ending standard par
      \label{div_pvv.2.218}
	  
	% new div opening: depth here is 2
	\textsuperscript{\textenglish{184/s}}
	  \bigskip
	  \begingroup
	
	    \large
	  
	    \begin{quote}
	  
	    
	    \stanza[\smallbreak]
	\label{pv.2.218a}\flagstanza{\tiny\textenglish{...2.218a}}असंविदित‚त‚त्त्वा च सा स‚र्व्वाप‚र‚द‚र्श‚नैः ।\&[\smallbreak]


	
	    \end{quote}
	  
	  \endgroup
	

	  \pstart \leavevmode% starting standard par
	सा च विज्ञ‚प्तिः स‚र्व्वैर‚प‚र‚द\edtext{}{\edlabel{pvv.184-1}\label{pvv.184-1}\lemma{द}\Bfootnote{बुद्धाद‚न्यैः ।}}र्श‚नैर‚नुत्कृष्ट‚द‚र्श‚{\color{DodgerBlue3}‚नैर‚स‚म्विदिता} द्व‚य‚शून्य‚तात‚त्त्वं ‚{\tiny $_{lb}$}‚य‚स्याः साऽसंविदित‚त‚त्वा (।)
	\pend% ending standard par
      

	  \pstart \leavevmode% starting standard par
	क‚स्मादित्याह (।)
	\pend% ending standard par
      
	  \bigskip
	  \begingroup
	
	    \large
	  
	    \begin{quote}
	  
	    
	    \stanza[\smallbreak]
	\label{pv.2.218b}\flagstanza{\tiny\textenglish{...2.218b}}असंभ‚वाद्विना तेषां ग्राह्य‚ग्राह‚क‚विप्ल‚वैः ॥ २१८ ॥\&[\smallbreak]


	
	    \end{quote}
	  
	  \endgroup
	

	  \pstart \leavevmode% starting standard par
	\hphantom{.}‚{\color{DodgerBlue3}‚ग्राह्य‚ग्राह‚क‚विप्ल‚वैर्व्विना तेषा}‚म‚प‚र‚द‚र्श‚नानां विज्ञ‚प्तेर‚{\color{DodgerBlue3}‚स‚म्भ‚वात्} । ग्राह्य‚ग्राह‚को‚{\tiny $_{lb}$}‚प‚प्लुत‚द‚र्श‚न‚त्व‚मेव चानुत्कृष्ट‚द‚र्श‚न‚त्वं । (२१८)
	\pend% ending standard par
      \label{div_pvv.2.219}
	  
	% new div opening: depth here is 2
	
	  \bigskip
	  \begingroup
	
	    \large
	  
	    \begin{quote}
	  
	    
	    \stanza[\smallbreak]
	\label{pv.2.219}\flagstanza{\tiny\textenglish{....2.219}}त‚दुपेक्षित‚त‚त्त्वार्थैः कृत्वा ग‚ज‚निमील‚न‚म् ।&केव‚लं लोक‚बुध्यैव बाह्य‚चिन्ता प्र‚य‚न्य‚ते ॥ २१९ ॥\&[\smallbreak]


	
	    \end{quote}
	  
	  \endgroup
	

	  \pstart \leavevmode% starting standard par
	\hphantom{.}‚{\color{DodgerBlue3}‚त‚द् भ्रा}‚न्त‚द‚र्श‚नानु‚{\tiny $_{2}$}‚रोधा‚{\color{DodgerBlue3}‚दुपेक्षित‚त‚त्वार्थै}‚र‚व‚धारित‚त‚द‚द्व‚य‚विवेकै‚{\color{DodgerBlue3}‚र्भ‚ग‚व‚द््भिर्बुद्धै‚{\tiny $_{lb}$}‚र्ग्ग‚ज‚निमील‚नं कृत्वा} ग‚ज‚स्येव पार्श्व‚द्व‚यं प‚श्य‚तोपि न‚य‚न‚निमील‚न‚क‚ल‚या त‚त्त्व‚म‚{\tiny $_{lb}$}‚प‚श्य‚न्त‚मिवात्मानं द‚र्श‚य‚द््भि\edtext{}{\edlabel{pvv.184-2}\label{pvv.184-2}\lemma{द््भि}\Bfootnote{केव‚लं ।}} ‚{\color{DodgerBlue3}‚र्लोक}‚स्याविद्योप‚ह‚त‚स्य ‚{\color{DodgerBlue3}‚बुद्ध्या} द्व‚य‚ग्राहिण्या ‚{\color{DodgerBlue3}‚बाह्य}‚स्य ‚{\tiny $_{lb}$}‚‚{\color{DodgerBlue3}‚चिन्ता}‚स्क‚न्धाय‚त‚नादित्वेन ‚{\color{DodgerBlue3}‚प्र‚त‚न्य‚ते} । स्क‚न्धादिदेश‚न‚या हि लोक‚बुद्ध्य‚नुरोध‚प्र‚व‚र्तित‚{\tiny $_{lb}$}‚पाष‚ण्ड‚ग्रा\edtext{}{\edlabel{pvv.184-3}\label{pvv.184-3}\lemma{ग्रा}\Bfootnote{आत्म‚ग्र‚ह ।}}ह‚निवृत्तौ मुख्य‚या येषु मुमुक्ष‚वोऽव‚तार्य‚न्त इति देश‚नाक्र‚मः । (२१९)
	\pend% ending standard par
      \label{div_pvv.2.220}
	  
	% new div opening: depth here is 2
	

	  \pstart \leavevmode% starting standard par
	अथ‚वा चित्र‚त्वेपि‚{\tiny $_{3}$}‚ बाह्य‚मेकं न युक्तं बुद्धिस्तु चित्राप्येकैवेति द‚र्श‚यितु\edtext{}{\edlabel{pvv.184-4}\label{pvv.184-4}\lemma{यितु}\Bfootnote{प्र‚तिबिम्ब‚व‚दाकाराः पार‚मार्थिक‚त्वेऽनेकान्तः स्यात् ।}}माह (।)
	\pend% ending standard par
      
	  \bigskip
	  \begingroup
	
	    \large
	  
	    \begin{quote}
	  
	    
	    \stanza[\smallbreak]
	\label{pv.2.220a}\flagstanza{\tiny\textenglish{...2.220a}}नीलादिश्चित्र‚विज्ञाने ज्ञानोपाधिर‚न‚न्य‚भाक् ॥&अश‚क्य‚द‚र्श‚नः;\&[\smallbreak]


	
	    \end{quote}
	  
	  \endgroup
	

	  \pstart \leavevmode% starting standard par
	\hphantom{.}‚{\color{DodgerBlue3}‚नीलादिश्चित्रे ज्ञाने ज्ञानोपाधिर}‚नुभ‚व‚स्यात्म‚भूतः ‚{\color{DodgerBlue3}‚अन‚न्य‚भाक्} आकारान्त‚रा‚{\tiny $_{lb}$}‚स‚ह‚च‚रः\edtext{}{\edlabel{pvv.184-5}\label{pvv.184-5}\lemma{रः}\Bfootnote{त‚ज्ज्ञान‚व‚त् ।}} केव‚ल इत्य‚र्थः । तादृशोऽ‚{\color{DodgerBlue3}‚श‚क्य‚द\edtext{}{\edlabel{pvv.184-6}\label{pvv.184-6}\lemma{द}\Bfootnote{स नीलादिरेकैकं ।}}र्श‚नः} स‚हैवाकारान्त‚र‚वेद‚न‚निय‚मात् । न ‚{\tiny $_{lb}$}‚हि चित्रे विज्ञाने स‚मुत्प‚न्ने नीलं निर‚स्य पीतं श‚क्य‚द‚र्श‚नं । त‚स्माद‚श‚क्य‚विवेच‚न‚त्वं ‚{\tiny $_{lb}$}‚तुल्य‚योग‚क्षेम‚त्वं स‚ह‚प्र‚तिभास‚निय‚त‚त्वं ज्ञानात्म‚नां नीलादीनामेक‚त्वं । बाह्यात्म‚नां ‚{\tiny $_{lb}$}‚तु नैत‚त्संभ‚व‚ति । एकं पिधायापि द्र‚ष्टु‚{\tiny $_{4}$}‚म‚न्य‚स्य श‚क्य‚त्वात् ।
	\pend% ending standard par
      

	  \pstart \leavevmode% starting standard par
	न‚नु ज्ञानाकारोपि नीलः पीतानुभ‚व‚काले त‚दा नानुभूय‚ते त‚दा श‚क्य‚विवेच‚न ‚{\tiny $_{lb}$}‚एवेत्याह (।)
	\pend% ending standard par
      
	  \bigskip
	  \begingroup
	
	    \large
	  
	    \begin{quote}
	  
	    
	    \stanza[\smallbreak]
	\label{pv.2.220b}\flagstanza{\tiny\textenglish{...2.220b}}तं हि प‚त‚त्य‚र्थे विवेच‚य‚न् ॥ २२० ॥\&[\smallbreak]


	
	    \end{quote}
	  
	  \endgroup
	\textsuperscript{\textenglish{185/s}}

	  \pstart \leavevmode% starting standard par
	\hphantom{.}त‚म‚नुभूय‚मानात् पीतात् ‚{\color{DodgerBlue3}‚वि\edtext{}{\edlabel{pvv.185-1}\label{pvv.185-1}\lemma{वि}\Bfootnote{उत्त‚र‚ज्ञानेन पूर्व्व‚कं ।}}वेच‚य‚न्} भेदेन व्य‚व‚स्थाप‚य‚न् प्र‚माता ‚{\color{DodgerBlue3}‚अर्थ} एव ‚{\tiny $_{lb}$}‚नीले ‚{\color{DodgerBlue3}‚प‚त‚ति} विवेच‚क‚त्वेन । प‚रो\edtext{}{\edlabel{pvv.185-2}\label{pvv.185-2}\lemma{रो}\Bfootnote{अतीत‚त्वात् ।}}क्षं त‚दा नील‚म‚र्थ एव । अप‚रोक्ष‚तैव तु ज्ञान‚स्व‚{\tiny $_{lb}$}‚भावः । अतो य‚द् विविच्य‚ते त‚द‚ज्ञानं । य‚ज्ज्ञानं त‚न्न विवेच्य‚त एव । (२२०)
	\pend% ending standard par
      \label{div_pvv.2.221}
	  
	% new div opening: depth here is 2
	

	  \pstart \leavevmode% starting standard par
	त‚स्माद् (।)
	\pend% ending standard par
      
	  \bigskip
	  \begingroup
	
	    \large
	  
	    \begin{quote}
	  
	    
	    \stanza[\smallbreak]
	\label{pv.2.221}\flagstanza{\tiny\textenglish{....2.221}}य‚द् य‚था भास‚ते ज्ञानं त‚त्त‚थैव प्र‚काश‚ते ।&इति नामैक‚भावः स्याच्चित्राकार‚स्य चेत‚सि ॥ २२१ ॥\&[\smallbreak]


	
	    \end{quote}
	  
	  \endgroup
	

	  \pstart \leavevmode% starting standard par
	\hphantom{.}‚{\color{DodgerBlue3}‚य‚ज्ज्ञानं य‚था} नीलाद्यात्म‚त‚या जातं ‚{\color{DodgerBlue3}‚स‚त्त‚था} भास‚ते ‚{\color{DodgerBlue3}‚प्र‚काश‚ते} । भास‚{\tiny $_{lb}$}‚मान‚{\tiny $_{5}$}‚स्व‚भाव‚त्वात् ‚{\color{DodgerBlue3}‚ज्ञानं} त‚था तेनैव स्व‚रूपेणानुभूय‚ते स‚र्वेः प्र‚तिप‚त्तृभिः । न ‚{\tiny $_{lb}$}‚चोत्प‚न्न‚स्यापि ज्ञान‚स्य स्व‚प्र‚काश‚क‚स्याविदितः क‚श्चिदाकारोस्ति ‚{\color{DodgerBlue3}‚इत्य}‚श‚क्य‚विवे‚{\tiny $_{lb}$}‚च‚न‚त्वात् तुल्य‚योग‚क्षेम‚त्वात् स‚ह‚प्र‚तिभास‚निय‚मात् । ‚{\color{DodgerBlue3}‚चित्र}‚स्य नील‚पीताद्या‚{\color{DodgerBlue3}‚कार‚स्य ‚{\tiny $_{lb}$}‚चे}‚त‚सि बुद्धावेक‚भावो नाम भ‚वेत्त‚दा को दोषः ।\edtext{\textsuperscript{*}}{\edlabel{pvv.185-3}\label{pvv.185-3}\lemma{*}\Bfootnote{न ह्येवं बाह्य‚श्चित्रोर्थ‚स्त‚थाहि ।}}(२२१)
	\pend% ending standard par
      \label{div_pvv.2.222}
	  
	% new div opening: depth here is 2
	
	  \bigskip
	  \begingroup
	
	    \large
	  
	    \begin{quote}
	  
	    
	    \stanza[\smallbreak]
	\label{pv.2.222a}\flagstanza{\tiny\textenglish{...2.222a}}प‚टादिरूप‚स्यैक‚त्वे त‚था स्याद‚विवेकिता ।\&[\smallbreak]


	
	    \end{quote}
	  
	  \endgroup
	

	  \pstart \leavevmode% starting standard par
	ज्ञान‚व‚त् प‚टादिरूप‚स्य एक‚त्वेऽभ्युप‚ग‚म्य‚माने त‚था ज्ञान‚स्येवाविवेकिता‚{\tiny $_{lb}$}‚ऽश‚क्य‚विवेच‚न‚त्वं स्यात् । न चास्ति (।) नीलादीनां‚{\tiny $_{6}$}‚ ग्र‚ह‚णाग्र‚ह‚ण‚भेद‚स्य द‚र्श‚{\tiny $_{lb}$}‚नात् ।
	\pend% ending standard par
      

	  \pstart \leavevmode% starting standard par
	स्यादेत‚द् (।) अव‚य‚वा नीलाद्याः प‚र‚स्प‚र‚तोऽव‚य‚विन‚श्च भिन्नास्तेषां भेदाद्वि‚{\tiny $_{lb}$}‚वेकेन ग्र‚ह‚णं । य‚स्त्व‚भिन्नोऽव‚य‚वी न त‚स्य विवेकेन ग्र‚ह‚ण‚मित्याह (।)
	\pend% ending standard par
      
	  \bigskip
	  \begingroup
	
	    \large
	  
	    \begin{quote}
	  
	    
	    \stanza[\smallbreak]
	\label{pv.2.222b}\flagstanza{\tiny\textenglish{...2.222b}}विवेकीनि निर‚स्यान्य‚दा विवेकि च नेक्ष‚ते ॥ २२२ ॥\&[\smallbreak]


	
	    \end{quote}
	  
	  \endgroup
	

	  \pstart \leavevmode% starting standard par
	\hphantom{.}विवेकीनि नीलाद्य‚व‚य‚व‚रूपाणि निर‚स्य पृथ‚क् कृत्वा ‚{\color{DodgerBlue3}‚अन्य‚दा-विवेकिरूप‚ञ्च ‚{\tiny $_{lb}$}‚नेक्ष‚ते} ।\edtext{\textsuperscript{*}}{\edlabel{pvv.185-4}\label{pvv.185-4}\lemma{*}\Bfootnote{पिहितेष्व‚प्य‚व‚य‚वेषु निर‚व‚य‚त्वाद‚व‚य‚वी दृश्य‚ते ।}}दृश्य‚स‚म्म‚त‚म‚नुप‚ल‚भ्य‚मानं क‚थ‚म‚भ्युप‚ग‚मार्ह । (२२२)
	\pend% ending standard par
      \label{div_pvv.2.223}
	  
	% new div opening: depth here is 2
	

	  \pstart \leavevmode% starting standard par
	य‚च्चोच्य‚ते प‚र‚माण‚वः प्र‚त्येक‚म\edtext{}{\edlabel{pvv.185-5}\label{pvv.185-5}\lemma{म}\Bfootnote{उद्योत‚क‚राद्यैः ।}}तीन्द्रिय‚त्वात् स‚ञ्चिता अपि न ज्ञान‚गोच‚र ‚{\tiny $_{lb}$}‚इ\edtext{}{\edlabel{pvv.185-6}\label{pvv.185-6}\lemma{इ}\Bfootnote{नाप्युद‚काभ्याह‚र‚णं विनाव‚य‚विन‚मिति भावः ।}}ति त‚त्राह(।)
	\pend% ending standard par
      
	  \bigskip
	  \begingroup
	
	    \large
	  
	    \begin{quote}
	  
	    
	    \stanza[\smallbreak]
	\label{pv.2.223}\flagstanza{\tiny\textenglish{....2.223}}को वा विरोधो ब‚ह‚वः संजातातिश‚याः (पृथ‚क्)।&भ‚वेयुः कार‚णं बुद्धेर्य‚दि नात्मेन्द्रियादिव‚त् ॥ २२३ ॥\&[\smallbreak]


	
	    \end{quote}
	  
	  \endgroup
	\textsuperscript{\textenglish{186/s}}\textsuperscript{\textenglish{36b/MA}}

	  \pstart \leavevmode% starting standard par
	\hphantom{.}‚{\color{DodgerBlue3}‚य‚दि ब‚ह‚वः} प‚र‚माण‚व उ‚{\tiny $_{7}$}‚प‚स‚र्प‚ण‚प्र‚त्य‚यात् ‚{\color{DodgerBlue3}‚संजा\edtext{}{\edlabel{pvv.186-1}\label{pvv.186-1}\lemma{संजा}\Bfootnote{योग्य‚देश‚त्वाव्य‚व‚हित‚त्वादिना ।}} तातिश‚या} विज्ञान‚ज‚न‚न‚योग्याः ‚{\tiny $_{lb}$}‚संह‚ता उत्प‚न्ना स्व‚ग्राहि कार्य‚ञ्च ‚{\color{DodgerBlue3}‚बुद्धेः कार‚णं भ‚वे}‚युस्त‚दा ‚{\color{DodgerBlue3}‚को विरोध इन्द्रिया}\edtext{}{\edlabel{pvv.186-2}\label{pvv.186-2}\lemma{दा}\Bfootnote{अत्मेन्द्रियार्थ‚स‚न्निक‚र्षाद् वैशेषिकाः । इन्द्रियार्थ‚स‚न्निक‚र्षान्नैयायिकाः ।}} दिव‚त् । ‚{\tiny $_{lb}$}‚इन्द्रियाद‚यः प्र‚त्ये\edtext{}{\edlabel{pvv.186-3}\label{pvv.186-3}\lemma{त्ये}\Bfootnote{पृथ‚ग‚र्थः ।}}कं न बुद्धेर्हेतुर्मिलितास्तु भ‚व‚न्ति त‚द्व‚द‚ण‚वोपि स्युः । न हि प‚रेषा‚{\tiny $_{lb}$}‚मिवास्माक‚ञ्च नित्यैक‚स्व‚भावा अण‚वः । ते हि य‚थाप्र‚त्य‚य‚म‚तीन्द्रियाः स‚न्त ऐन्द्रिया ‚{\tiny $_{lb}$}‚अपि स्युः ॥ (२२३)
	\pend% ending standard par
      \label{div_pvv.2.224}
	  
	% new div opening: depth here is 2
	

	  \pstart \leavevmode% starting standard par
	न‚नु हेतुत्वेपि क‚थ‚म‚ण‚वो ग्राह्या इत्याह (।)
	\pend% ending standard par
      
	  \bigskip
	  \begingroup
	
	    \large
	  
	    \begin{quote}
	  
	    
	    \stanza[\smallbreak]
	\label{pv.2.224a}\flagstanza{\tiny\textenglish{...2.224a}}हेतुभावादृते नान्या ग्राह्य‚ता नाम काच‚न ॥\&[\smallbreak]


	
	    \end{quote}
	  
	  \endgroup
	

	  \pstart \leavevmode% starting standard par
	\hphantom{.}‚{\color{DodgerBlue3}‚हेतुभाव‚दृते विना ग्राह्य‚ता नाम} या प्र‚सिद्ध‚{\tiny $_{1}$}‚ सा नान्या ‚{\color{DodgerBlue3}‚काचित्} । अपि तु ‚{\tiny $_{lb}$}‚हेतुतैन ग्राह्य‚ता ।
	\pend% ending standard par
      

	  \pstart \leavevmode% starting standard par
	एव‚न्त‚र्हीन्द्रियादिक‚म‚पि हेतुत्वाद् ग्राह्यं स्यादित्याह (।)
	\pend% ending standard par
      
	  \bigskip
	  \begingroup
	
	    \large
	  
	    \begin{quote}
	  
	    
	    \stanza[\smallbreak]
	\label{pv.2.224b}\flagstanza{\tiny\textenglish{...2.224b}}त‚त्र बुद्धिर्य‚दाकारा त‚स्यास्त‚द् ग्राह्य‚मुच्य‚ते ॥ २२४ ॥\&[\smallbreak]


	
	    \end{quote}
	  
	  \endgroup
	

	  \pstart \leavevmode% starting standard par
	\hphantom{.}‚{\color{DodgerBlue3}‚त‚त्र} तेषु हेतुषु ‚{\color{DodgerBlue3}‚बुद्धिर्य‚दाकारा} भ‚व‚ति ‚{\color{DodgerBlue3}‚त‚स्या} बुद्धेस्त‚द् ग्राह्य‚{\color{DodgerBlue3}‚मुच्य‚ते} अणुस‚ञ्च‚यः । ‚{\tiny $_{lb}$}‚सैव च बुद्धिराकार‚म‚नुक‚रोति नेन्द्रियादेः । (२२४)
	\pend% ending standard par
      \label{div_pvv.2.225}
	  
	% new div opening: depth here is 2
	
	  \bigskip
	  \begingroup
	
	    \large
	  
	    \begin{quote}
	  
	    
	    \stanza[\smallbreak]
	\label{pv.2.225}\flagstanza{\tiny\textenglish{....2.225}}क‚थं वाऽव‚य‚वी ग्राह्यः स‚कृत् स्वाव‚य‚वैः स‚ह ।&न हि गोप्र‚त्य‚यो दृष्टः सास्नादीनाम‚द‚र्श‚ने ॥ २२५ ॥\&[\smallbreak]


	
	    \end{quote}
	  
	  \endgroup
	

	  \pstart \leavevmode% starting standard par
	\hphantom{.}‚{\color{DodgerBlue3}‚त‚त्क‚थं} त‚द्ग्राह्यं । योप्याह (।) नानेकं ‚{\color{DodgerBlue3}‚स‚कृ}‚द् गृह्य‚त इति । त‚न्म‚ते‚{\color{DodgerBlue3}‚ऽव‚य‚वी ‚{\tiny $_{lb}$}‚स्वाव‚य‚वैः} सास्नाक‚कुद्लाङ्गूलादिभिः स‚ह क‚थ‚म्वा ग्राह्यो युक्तः । न गृह्य‚त ‚{\tiny $_{lb}$}‚एवेति चेत् । ‚{\color{DodgerBlue3}‚न हि गो}‚र‚व‚य‚विनः ‚{\color{DodgerBlue3}‚प्र‚त्य‚यः‚{\tiny $_{2}$}‚ सास्नादीनाम}‚व‚य‚वा‚{\color{DodgerBlue3}‚नाम‚द‚र्श‚ने} क्वापि ‚{\tiny $_{lb}$}‚‚{\color{DodgerBlue3}‚दृष्टः}\edtext{}{\edlabel{pvv.186-4}\label{pvv.186-4}\lemma{क्वापि}\Bfootnote{प‚र‚स्प‚र‚म‚नुपाधिभूतानेक‚द्र‚व्य‚ग्र‚ह‚णं नेष्य‚ते न पुन‚र्गुण‚प्र‚धान‚भूत‚स्य । विशे‚{\tiny $_{lb}$}‚ष‚ण‚भूताश्च सास्नाद‚योऽव‚य‚वा गोर्द्र‚व्य‚स्येति तेषां स‚ह‚ग्र‚हो युक्तः । न त्व‚णूनाम‚ने ‚{\tiny $_{lb}$}‚वंत्वात् ।}}। (२२५)
	\pend% ending standard par
      \label{div_pvv.2.226}
	  
	% new div opening: depth here is 2
	
	  \bigskip
	  \begingroup
	
	    \large
	  
	    \begin{quote}
	  
	    
	    \stanza[\smallbreak]
	\label{pv.2.226}\flagstanza{\tiny\textenglish{....2.226}}गुण‚प्र‚धानाधिग‚मः स‚हाप्य‚भिम‚तो य‚दि ।&स‚म्पूर्णाङ्गो न गृह्येत स‚कृन्नापि गुणादिमान् ॥ २२६ ॥\&[\smallbreak]


	
	    \end{quote}
	  
	  \endgroup
	

	  \pstart \leavevmode% starting standard par
	\hphantom{.}‚{\color{DodgerBlue3}‚य‚दि स‚हाप्य‚भिम‚तो गुण}‚प्र‚धान‚योर्विशेष‚ण‚विशेष्य‚योर‚{\color{DodgerBlue3}‚धिग‚मो} नोपाधीनाम‚न्योन्यं ‚{\tiny $_{lb}$}‚विशेष‚ण‚विशेष्य‚भूतानां द्र‚व्यानां (? णां) वा तादृशानां (? णां) । एव‚न्त‚र्हि विषाणी ‚{\tiny $_{lb}$}‚\leavevmode\ledsidenote{\textenglish{187/s}} सास्नादिमानिति वा य‚दा गृह्य‚ते त‚दा तेनैवाव‚य‚वेन स‚म्ब‚न्ध‚व्य‚व‚सायादित‚राव‚य‚व‚{\tiny $_{lb}$}‚स‚म्ब‚न्धान‚व‚सायात् ‚{\color{DodgerBlue3}‚संपूर्णाङ्गो}‚ऽव‚य\edtext{}{\edlabel{pvv.187-1}\label{pvv.187-1}\lemma{य}\Bfootnote{न हि विषाणेनान्येप्युपाध‚योऽव‚च्छिद्य‚न्ते तेषां विशेष्य‚त्व‚प्र‚स‚ङ्गात् ।}}वी ‚{\color{DodgerBlue3}‚न गृह्येत । स‚कृद्} गृह्य‚ते च । ‚{\color{DodgerBlue3}‚नापि गुणा‚{\tiny $_{lb}$}‚दिमान्} गृह्येत । विषाणी‚{\tiny $_{3}$}‚ गौरिति बुद्ध्या विषाण‚विशिष्टो गौर्विष‚यीकृतो न त्व‚न्ये ‚{\tiny $_{lb}$}‚गुण‚क‚र्म‚सामान्याद‚यः । त‚त‚श्च नीलादिरूपं गुणः प‚रिस्प‚न्दादि च क‚र्म्म । व‚स्तुत्वादि ‚{\tiny $_{lb}$}‚च सामान्यं न गृह्येत । दृष्ट‚विरुद्ध‚ञ्चैत‚त् । (२२६)
	\pend% ending standard par
      \label{div_pvv.2.227}
	  
	% new div opening: depth here is 2
	

	  \pstart \leavevmode% starting standard par
	\hphantom{.}स‚र्व्वेषां गुण‚क‚र्म्म‚सामान्याव‚य‚वादीनां व‚स्तुतो विशेष‚ण‚त्वात् ‚{\color{DodgerBlue3}‚स‚र्व्व‚ग्र‚ह‚ण‚मिति} चेत् । आह (।)
	\pend% ending standard par
      
	  \bigskip
	  \begingroup
	
	    \large
	  
	    \begin{quote}
	  
	    
	    \stanza[\smallbreak]
	\label{pv.2.227}\flagstanza{\tiny\textenglish{....2.227}}विव‚क्षा प‚र‚त‚न्त्र‚त्वात् विशेष‚ण‚विशेष्य‚योः ।&य‚द‚ङ्ग‚भावेनोपात्त‚न्त‚त्तेनैव हि गृह्य‚ते ॥ २२७ ॥\&[\smallbreak]


	
	    \end{quote}
	  
	  \endgroup
	

	  \pstart \leavevmode% starting standard par
	\hphantom{.}‚{\color{DodgerBlue3}‚विशेष‚ण‚विशेष्य‚योर्व्विव‚क्षाप‚र‚त‚न्त्र‚त्वान्} पुरुषेच्छानुरोधात् न पार‚मार्थिक‚त्वं । ‚{\tiny $_{lb}$}‚त‚था विषाणी गौरिति गोर्विषा‚{\tiny $_{4}$}‚ण‚मित्यादौ विप‚र्य‚यो विशेष‚ण‚विशेष्य‚योः प्र‚योक्तु‚{\tiny $_{lb}$}‚रिच्छाव‚शेन दृश्य‚ते । त‚स्माद्य‚देव ‚{\color{DodgerBlue3}‚ह्य‚ङ्ग‚भावेन} विशेष‚ण‚भावेन स्व‚म‚नीषिक‚या ‚{\tiny $_{lb}$}‚प्र‚तिपाद‚यित्रोपात्तं ‚{\color{DodgerBlue3}‚तेनैव} विशेष‚णेन विशिष्टं त‚द्विव‚क्षितं ‚{\color{DodgerBlue3}‚गृह्य‚ते} न त‚दित‚रैः । ‚{\tiny $_{lb}$}‚तेषाम‚विव‚क्षित‚त्वेनाविव‚क्षित‚त्वात् । त‚त‚श्च न संपूर्ण्णाङ्गो नापि गुणादिमान् ‚{\tiny $_{lb}$}‚गृह्येत । (२२७)
	\pend% ending standard par
      \label{div_pvv.2.228_2.229}
	  
	% new div opening: depth here is 2
	

	  \pstart \leavevmode% starting standard par
	किञ्च (।)
	\pend% ending standard par
      
	  \bigskip
	  \begingroup
	
	    \large
	  
	    \begin{quote}
	  
	    
	    \stanza[\smallbreak]
	\label{pv.2.228}\flagstanza{\tiny\textenglish{....2.228}}स्व‚तो व‚स्त्व‚न्त‚राभेदाद् गुणादेर्भेद‚क‚स्य च ।&अग्र‚हादेक‚बुद्धिः स्यात् प‚श्य‚तोपि प‚राप‚र‚म् ॥ २२८ ॥\&[\smallbreak]


	
	    \end{quote}
	  
	  \endgroup
	
	  \bigskip
	  \begingroup
	
	    \large
	  
	    \begin{quote}
	  
	    
	    \stanza[\smallbreak]
	\label{pv.2.229}\flagstanza{\tiny\textenglish{....2.229}}गुणादिभेद‚ग्र‚ह‚णान्नानात्व‚प्र‚तिप‚द्य‚दि ।&अस्तु नाम त‚थाप्येषां भ‚वेत् स‚म्ब‚न्धिसंक‚रः ॥ २२९ ॥\&[\smallbreak]


	
	    \end{quote}
	  
	  \endgroup
	

	  \pstart \leavevmode% starting standard par
	\hphantom{.}व‚स्तुनः ‚{\color{DodgerBlue3}‚स्व‚तो व‚स्त्व‚न्त‚राद् भेदाभावात्} । व‚द‚न्ति हि स्व‚तो हि न ‚{\tiny $_{lb}$}‚गौर्नागौर्ग्गोत्व‚योगात् गौः । त‚था स्व‚तो हि शुक्लो नाशुक्लः शुक्ल‚{\tiny $_{5}$}‚त्व‚योगात् शुक्ल ‚{\tiny $_{lb}$}‚इत्यादि । ‚{\color{DodgerBlue3}‚भेद‚क‚स्य च गुणा\edtext{}{\edlabel{pvv.187-2}\label{pvv.187-2}\lemma{गुणा}\Bfootnote{स‚कृद‚नेकार्थ‚प्र‚तीतिप्र‚स‚ङ्गात् ।}}}‚देर‚ग्र‚हात् प‚दार्थैः स‚ह । त‚त‚श्च ‚{\color{DodgerBlue3}‚प‚राप‚र‚म}‚र्थ‚जातं ‚{\color{DodgerBlue3}‚प‚श्य‚{\tiny $_{lb}$}‚तोप्येक}‚प‚दार्थ‚त्व‚{\color{DodgerBlue3}‚बुद्धिः स्यात्} । न हि द्र‚व्याणाम‚न्योन्य‚स्य भेद‚का गुण\edtext{}{\edlabel{pvv.187-3}\label{pvv.187-3}\lemma{गुण}\Bfootnote{घ‚ट‚द‚र्श‚नेपि प‚ट‚बोध एव विनोपाधिं प‚दार्थाग्र‚हात् ।}}जात्याद‚य‚स्तैः\edtext{}{\edlabel{pvv.187-4}\label{pvv.187-4}\lemma{स्तैः}\Bfootnote{दीर्घ‚त्वादि ।}} ‚{\tiny $_{lb}$}‚स‚ह गृह्य‚न्ते येन भेद‚बुद्धिः स्यात् । द्र‚व्ये गृहीते प‚श्चाद् गुणादीनां भेदानां विशेषाणां ‚{\tiny $_{lb}$}‚नानात्व‚ग्र‚ह‚णं द्र‚व्या\edtext{}{\edlabel{pvv.187-5}\label{pvv.187-5}\lemma{व्या}\Bfootnote{नाना सामान्यादीनि नैक‚स्येति त‚द्विशिष्टानां ।}}णां य‚दीष्य‚ते । ‚{\color{DodgerBlue3}‚अस्तु नाम} प‚श्चाद् गुणादिग्र‚ह‚णं । ‚{\color{DodgerBlue3}‚त‚थाप्येषां} \leavevmode\ledsidenote{\textenglish{188/s}} गुण‚क‚र्म्म‚सामान्यादीनां ‚{\color{DodgerBlue3}‚स‚म्ब‚न्धिनः सांक‚र्यं} स्यात् ।‚{\tiny $_{6}$}‚ स्व‚तो हि द्र‚व्यं भेदान्नोप‚ल‚भ्य‚ते । ‚{\tiny $_{lb}$}‚प‚श्चादुप‚ल‚भ्य‚मानैस्तु गुणादिभिः संब‚न्धित‚याव‚ग‚तैर्भेदेन त‚द् व्य‚व‚स्थाप‚नीयं । त‚त‚श्चै‚{\tiny $_{lb}$}‚क‚त्रैक\edtext{}{\edlabel{pvv.188-1}\label{pvv.188-1}\lemma{त्रैक}\Bfootnote{भिन्न‚द्र‚व्य‚स‚म‚वायान्य‚पि संक‚ल‚नात् स‚म्भिन्न‚ग्र‚ह‚तः ।}} द्र‚व्ये गुणाद‚यः स‚म्ब‚न्धित‚या प्र‚तीयेर‚न् ॥ भिन्न‚देशानां गुणानां भिन्न‚देशेषु ‚{\tiny $_{lb}$}‚योज‚नान्न स‚म्ब‚न्धिसांक‚र्य‚मिति चेत् । न (।) देश‚भेद‚स्यापि विशेष‚ण‚त्वात् सोप्ये‚{\tiny $_{lb}$}‚क‚स्येति स्यात् । एक‚स्य विरुद्ध‚ध‚र्मायोगात् प्र‚तीय‚मानार्थाः प‚श्चात् प्र‚तीय‚मानै‚{\tiny $_{lb}$}‚\leavevmode\ledsidenote{\textenglish{37a/MA}} रूपाधिभिर्भिन्ना व्य‚व‚स्थाप्य‚न्त इति चे‚{\tiny $_{7}$}‚त् । न‚न्वेव‚म‚पि स्व‚तोऽर्थानां भेदाभावात् । ‚{\tiny $_{lb}$}‚भेद‚निब‚न्ध‚नैश्चोपाधिभिः स‚ह वेद‚नाभावान्नाध्य‚क्ष‚सिद्धा भेद‚व्य‚व‚स्थितिः स्यात् । ‚{\tiny $_{lb}$}‚किन्तु विरुद्धोपाधिस‚म्ब‚न्धान्य‚थानुप‚प‚त्त्या\edtext{}{\edlabel{pvv.188-2}\label{pvv.188-2}\lemma{त्त्या}\Bfootnote{भेद‚स्थितिः ।}} क‚ल्प‚नीया । त‚था च प्र‚त्य‚क्ष‚विरोधः\edtext{}{\edlabel{pvv.188-3}\label{pvv.188-3}\lemma{विरोधः}\Bfootnote{अन‚ध्य‚क्ष‚त्व‚स्य ।}} ।
	\pend% ending standard par
      

	  \pstart \leavevmode% starting standard par
	अथ दृष्टेर्थे प‚श्चात्त‚न‚म‚ध्य‚क्ष‚मुपाधिग्राह‚कं तान् योज‚येद् भेद‚ग्राह‚कं । क‚थ‚म‚दृश्य‚{\tiny $_{lb}$}‚माने योज‚नोपाधीनां प्र‚त्य‚क्ष‚कृता विक‚ल्प‚कृतैव तु स्यात् । त‚था च व्य‚क्तं न भेद‚द‚र्श‚नं ‚{\tiny $_{lb}$}‚स्यात् । विक‚ल्प‚विष‚य‚स्यास्फुट‚{\tiny $_{1}$}‚त्वात् । ृष्टेऽर्थे प‚श्चादुपाधिः प्र‚तीय‚मान‚स्त‚स्यैव दृष्ट ‚{\tiny $_{lb}$}‚इति चेत् । त‚स्येति किमुच्य‚ते । न ताव‚द‚नेकासूपाधिग्र‚ह‚णात् पूर्व्व‚म‚नेक‚त्व‚स्याप्र‚तीतेः । ‚{\tiny $_{lb}$}‚एकं चेत् क‚थ‚मुपाधिभिर्भेद‚नीयं । अन‚व‚धृतैकानेक‚भावं व‚स्तुमात्रं त‚दिति चेत् । त‚त्र ‚{\tiny $_{lb}$}‚त‚र्हि देश‚भेदाद‚योप्युपाध‚यः प‚श्चादुप‚ल‚भ्य‚माना योज्य‚न्ते । त‚द‚न‚न्त‚रं दृष्ट‚त्वात् ‚{\tiny $_{lb}$}‚त‚त\edtext{}{\edlabel{pvv.188-4}\label{pvv.188-4}\lemma{त}\Bfootnote{उपाधीन् । पूर्व्व‚के न‚ष्टे । देश‚भेदादियोगे स‚ति ।}}श्चानिवार्यः स‚म्ब‚न्धिसंक‚र‚प्र‚स\edtext{}{\edlabel{pvv.188-5}\label{pvv.188-5}\lemma{स}\Bfootnote{अनेन क्र‚मेण ।}}ङ्गः । (२२८, २२९)
	\pend% ending standard par
      \label{div_pvv.2.230}
	  
	% new div opening: depth here is 2
	

	  \pstart \leavevmode% starting standard par
	सां ख्य\edtext{}{\edlabel{pvv.188-6}\label{pvv.188-6}\lemma{ख्य}\Bfootnote{येनैक‚रूप‚त (ा) इष्टा ।}} म‚तेपि (।)
	\pend% ending standard par
      
	  \bigskip
	  \begingroup
	
	    \large
	  
	    \begin{quote}
	  
	    
	    \stanza[\smallbreak]
	\label{pv.2.230}\flagstanza{\tiny\textenglish{....2.230}}श‚ब्दादीनाम‚नेक‚त्वात् सिद्धोनेक‚ग्र‚हः स‚कृत् ।&स‚न्निवेश‚ग्र‚हायोगाद‚ग्र‚हे स‚न्निवेशिनाम् ॥ २३० ॥\&[\smallbreak]


	
	    \end{quote}
	  
	  \endgroup
	

	  \pstart \leavevmode% starting standard par
	\hphantom{.}‚{\color{DodgerBlue3}‚श‚ब्दादीनां} सु\edtext{}{\edlabel{pvv.188-7}\label{pvv.188-7}\lemma{सु}\Bfootnote{श्रोत्रादीनां । स‚त्त्व‚र‚ज‚स्त‚मः ।}}ख‚दुः‚{\tiny $_{2}$}‚ख‚मोहात्म‚क‚त‚या अनेक‚त्वात् श‚ब्दादिग्र‚हे ‚{\color{DodgerBlue3}‚स‚कृद‚नेक‚ग्र‚हः ‚{\tiny $_{lb}$}‚सिद्धः । संन्निवेशिना\edtext{}{\edlabel{pvv.188-8}\label{pvv.188-8}\lemma{संन्निवेशिना}\Bfootnote{सुखादीनां ।}}म‚ग्र‚हे स‚न्निवे\edtext{}{\edlabel{pvv.188-9}\label{pvv.188-9}\lemma{न्निवे}\Bfootnote{श‚ब्दादेः ।}}श‚स्य ग्र‚हायोगात्} ।\edtext{\textsuperscript{*}}{\edlabel{pvv.188-10}\label{pvv.188-10}\lemma{*}\Bfootnote{दृष्टान्तः ।}}न ह्य‚ङ्गुल्य‚ग्र‚ह‚णे ‚{\tiny $_{lb}$}‚मुष्टिग्र‚ह‚णं । (२३०)
	\pend% ending standard par
      \label{div_pvv.2.231}
	  
	% new div opening: depth here is 2
	

	  \begin{center}%% label @type='head'
	\textbf{(ग. (क) क‚ल्प‚नापोढ‚त्वे ध‚र्म‚ध‚र्म्यादिसंग‚तिः)}
	\end{center}
	‚{\tiny $_{lb}$}‚

	  \pstart \leavevmode% starting standard par
	य‚दि प्र‚त्य‚क्ष‚म‚विक‚ल्पं त‚दा क‚थं ध‚र्म्म‚ध‚र्म्म्यादिग्र‚ह‚ण‚मित्याह (।)
	\pend% ending standard par
      
	  \bigskip
	  \begingroup
	
	    \large
	  
	    \begin{quote}
	  
	    
	    \stanza[\smallbreak]
	\label{pv.2-231}\flagstanza{\tiny\textenglish{....2-231}}स‚र्व‚तो विनिवृत्त‚स्य विनिवृत्तिर्य‚तो य‚तः ।&त‚द् भेदोन्नीत‚भेदा सा ध‚र्म्मिणोऽनेक‚रूप‚ता ॥ २३१ ॥\&[\smallbreak]


	
	    \end{quote}
	  
	  \endgroup
	\textsuperscript{\textenglish{189/s}}

	  \pstart \leavevmode% starting standard par
	\hphantom{.}‚{\color{DodgerBlue3}‚स‚र्व्व‚तः\edtext{}{\edlabel{pvv.189-1}\label{pvv.189-1}\lemma{तः}\Bfootnote{स्वार्थे सामान्य‚गोच‚रं व्याख्याय \href{http://sarit.indology.info/?cref=ps.1.5}{[प्र‚माण] स‚मुच्च‚ये} । ध‚र्म्मिणो नैक‚रूप‚स्य नेन्द्रियात् स‚र्व्व‚था ग‚तिः । स्व‚स‚म्वेद्य‚न्त्वानिर्द्देश्यं रूप‚मिन्द्रिय‚गोच‚र  इति व्याच‚ष्टे ।}}} प‚र‚स्मा‚{\color{DodgerBlue3}‚द्विनिवृत्त‚स्या}‚र्थ‚स्य ‚{\color{DodgerBlue3}‚य‚तो य‚तः} प‚र‚स्मा‚{\color{DodgerBlue3}‚द्विनिवृत्त‚स्तैर्भेदै}‚र्व्यावृत्ति‚{\tiny $_{lb}$}‚भिरुन्नीता साऽ‚{\color{DodgerBlue3}‚नेक‚रूप‚ता} ध‚र्म्मिध‚र्म्मात्म‚क‚त‚या ‚{\color{DodgerBlue3}‚ध‚र्म्मिणो}‚ऽर्थ‚स्य । स‚र्व्वंतो व्यावृत्ते ‚{\tiny $_{lb}$}‚प्र‚त्य‚क्षेण गृहीते व‚स्तुनि त‚द्व्यावृत्त्य‚नुकारिणो‚{\tiny $_{3}$}‚ विक‚ल्पा (:) प्र‚त्य‚क्ष‚दृष्ट‚त्वेन ‚{\tiny $_{lb}$}‚ध‚र्म्मिध‚र्म्म‚भावं व्य‚व‚स्थाप‚य‚न्तीति प्र‚त्य‚क्ष‚कृतः स उच्य‚ते । न तु प्र‚त्य‚क्ष‚प्र‚ति‚{\tiny $_{lb}$}‚भास‚मान‚त्वात् । (२३१)
	\pend% ending standard par
      \label{div_pvv.2.232}
	  
	% new div opening: depth here is 2
	

	  \pstart \leavevmode% starting standard par
	त‚था हि (।)
	\pend% ending standard par
      
	  \bigskip
	  \begingroup
	
	    \large
	  
	    \begin{quote}
	  
	    
	    \stanza[\smallbreak]
	\label{pv.2.232}\flagstanza{\tiny\textenglish{....2.232}}ते क‚ल्पिता रूप‚भेदाद् निर्व्विक‚ल्प‚स्य चेत‚सः ।&न विचित्र‚स्य चित्राभाः कादाचित्क‚स्य गोच‚रः ॥ २३२ ॥\&[\smallbreak]


	
	    \end{quote}
	  
	  \endgroup
	

	  \pstart \leavevmode% starting standard par
	\hphantom{.}‚{\color{DodgerBlue3}‚ते} ध‚र्मिध‚र्माद‚यो ‚{\color{DodgerBlue3}‚रूप‚भेदाद्} बुद्ध्याकार‚विशेषा विजातीय‚व्यावृत्त्याश्र‚येण ‚{\tiny $_{lb}$}‚‚{\color{DodgerBlue3}‚क‚ल्पिता} विचित्राभा व‚स्तुब‚ल‚भावित्वात् ‚{\color{DodgerBlue3}‚का\edtext{}{\edlabel{pvv.189-2}\label{pvv.189-2}\lemma{का}\Bfootnote{कादाचित्क‚त्वाद‚र्थ‚स‚न्निधिसापेक्ष‚त्व‚माहान्य‚था क‚ल्पित‚ध‚र्म‚भेद‚ज्ञेय‚त्वादि‚{\tiny $_{lb}$}‚विष‚य‚त्वे स‚र्व्व‚दा स्यात् । एतेन सामान्यं व‚स्तुस‚त्प्र‚तिषिद्धं ।}}दाचित्क‚स्य विचित्र‚स्य} ध‚र्म्मि‚{\tiny $_{lb}$}‚ध‚र्म्म‚भेद‚प्र‚तिभास‚र‚हित‚स्य ‚{\color{DodgerBlue3}‚निर्व्विक‚ल्प‚स्य चेत‚सः} प्र‚त्य‚क्ष‚स्य ‚{\color{DodgerBlue3}‚न गोच‚रः} । प्र‚त्य‚क्षं ‚{\tiny $_{lb}$}‚हि व‚स्तुसाम‚र्थ्योत्प‚न्नं त‚दाकार‚म‚नुकुर्यात् न विक‚ल्पितं । (२३२)
	\pend% ending standard par
      \label{div_pvv.2.233}
	  
	% new div opening: depth here is 2
	

	  \begin{center}%% label @type='head'
	\textbf{(ख) श‚ब्द‚विक‚ल्प‚विष‚यः सामान्य‚म्}
	\end{center}
	‚{\tiny $_{lb}$}‚

	  \pstart \leavevmode% starting standard par
	भ‚व‚तु‚{\tiny $_{4}$}‚ वा व‚स्त्वेव सामान्यं त‚थापि नासौ श‚ब्द‚विक‚ल्प‚विष‚य इत्याह (।)
	\pend% ending standard par
      
	  \bigskip
	  \begingroup
	
	    \large
	  
	    \begin{quote}
	  
	    
	    \stanza[\smallbreak]
	\label{pv.2.233}\flagstanza{\tiny\textenglish{....2.233}}य‚द्य‚प्य‚स्ति सित‚त्वादि यादृगिन्द्रिय‚गोच‚रः ।&न सोभिधीय‚ते श‚ब्दैर्ज्ञान‚यो रूप‚भेद‚तः ॥ २३३ ॥\&[\smallbreak]


	
	    \end{quote}
	  
	  \endgroup
	

	  \pstart \leavevmode% starting standard par
	\hphantom{.}‚{\color{DodgerBlue3}‚य‚द्य‚पि सित‚त्वादि} सामान्य‚म‚स्ति प‚टादौ ध‚र्म्मिणि ‚{\color{DodgerBlue3}‚यादृक्} सित‚त्वादिविश‚दा‚{\tiny $_{lb}$}‚कार ‚{\color{DodgerBlue3}‚इन्द्रिय‚स्य गोच‚रः । स} इन्द्रिय‚ज्ञान‚गोच‚रोऽर्थोऽ‚{\color{DodgerBlue3}‚भिधीय‚ते न श‚ब्दैः । ज्ञान‚यो-} रिन्द्रिय‚श‚ब्द‚ज‚नित‚यो ‚{\color{DodgerBlue3}‚रूप}‚स्याकार‚स्य स्फुटास्फुट‚त्वेन ‚{\color{DodgerBlue3}‚भेद‚तः} (२३३)
	\pend% ending standard par
      \label{div_pvv.2.234}
	  
	% new div opening: depth here is 2
	
	  \bigskip
	  \begingroup
	
	    \large
	  
	    \begin{quote}
	  
	    
	    \stanza[\smallbreak]
	\label{pv.2.234}\flagstanza{\tiny\textenglish{....2.234}}एकार्थ‚त्वेपि बुद्धीनां नानाश्र‚य‚त‚या स चेत् ।&श्रोत्रादिचित्तानीदानीं भिन्नार्थानीति त‚त्कुतः ॥ २३४ ॥\&[\smallbreak]


	
	    \end{quote}
	  
	  \endgroup
	

	  \pstart \leavevmode% starting standard par
	\hphantom{.}‚{\color{DodgerBlue3}‚बुद्धीना}‚मिन्द्रिय‚श‚ब्द‚ज‚निताना‚{\color{DodgerBlue3}‚मेकार्थ}‚त्वे एक‚वि\edtext{}{\edlabel{pvv.189-3}\label{pvv.189-3}\lemma{वि}\Bfootnote{सामान्य‚मात्र ।}}ष‚य‚त्वे‚{\color{DodgerBlue3}‚पि नानाश्र\edtext{}{\edlabel{pvv.189-4}\label{pvv.189-4}\lemma{नानाश्र}\Bfootnote{च‚क्षुरादिम‚न‚सी ।}}य‚त‚या} कार‚ण‚भेदा‚{\color{DodgerBlue3}‚त्म}‚क‚स्य आकार‚भेद‚{\color{DodgerBlue3}‚श्चेत् । इ‚{\tiny $_{5}$}‚दानी}‚मेवं स्थितौ ‚{\color{DodgerBlue3}‚श्रोत्रा}‚दीन्द्रिय‚{\color{DodgerBlue3}‚चित्तानि} \leavevmode\ledsidenote{\textenglish{190/s}} ‚{\color{DodgerBlue3}‚भिन्नार्थानि} श‚ब्द‚रूप‚ग‚न्धादिभिन्न‚विष‚याणीति व्य‚प‚दिश्य‚के (।) ‚{\color{DodgerBlue3}‚त‚त्कुतः} प्र‚माणाद‚व‚{\tiny $_{lb}$}‚धारितं । तान्य‚पि चितान्य‚भिन्न‚विष‚य‚त्वेपीन्द्रियाणामाश्र‚य‚भूतानां भेदाद् भिन्ना‚{\tiny $_{lb}$}‚काराणीति किन्न क‚ल्प्य‚ते । (२३४)
	\pend% ending standard par
      \label{div_pvv.2.235}
	  
	% new div opening: depth here is 2
	

	  \pstart \leavevmode% starting standard par
	किञ्च (।)
	\pend% ending standard par
      
	  \bigskip
	  \begingroup
	
	    \large
	  
	    \begin{quote}
	  
	    
	    \stanza[\smallbreak]
	\label{pv.2.235}\flagstanza{\tiny\textenglish{....2.235}}जातो नामाश्र‚योन्योन्यः चेत‚सां त‚स्य व‚स्तुनः ।&एक‚स्यैव कुतो रूपं भिन्नाकाराव‚भासि त‚त् ॥ २३५ ॥\&[\smallbreak]


	
	    \end{quote}
	  
	  \endgroup
	

	  \pstart \leavevmode% starting standard par
	\hphantom{.}‚{\color{DodgerBlue3}‚सामान्यादिचेत‚सामाश्र‚यः} कार‚ण‚मिन्द्रियं श‚ब्द‚श्चेति ‚{\color{DodgerBlue3}‚अन्योन्यो नाम जातः} । त‚था ‚{\tiny $_{lb}$}‚पि सामान्यादे‚{\color{DodgerBlue3}‚र्व्व‚स्तुन एक‚स्यैव रूपं त‚त्कुतो भिन्नाकाराव‚भासि} स्फुटास्फुटाव‚भासि ‚{\tiny $_{lb}$}‚स्फुटा‚{\tiny $_{6}$}‚स्फुट‚प्र‚तिभासं । न हि स्व‚रूपेण भास‚मान‚मेकं भिन्न‚प्र‚तिभासं युक्त‚म् । (२३५)
	\pend% ending standard par
      \label{div_pvv.2.236}
	  
	% new div opening: depth here is 2
	

	  \pstart \leavevmode% starting standard par
	य‚द‚प्युच्य‚ते प‚रैः (।) शाब्देन्द्रिय‚ज्ञान‚योर्यादि नैक‚विष‚य‚त्वं त‚दा विषाणादि‚{\tiny $_{lb}$}‚म‚न्त‚म‚र्थं गौरिति श‚ब्दात्प्र‚तीत्य कालान्त‚रे व्य‚क्तिविशेषं दृष्ट‚व‚तोऽय‚म‚सौ श‚ब्दात् ‚{\tiny $_{lb}$}‚प्राङ्म‚या प्र‚तीतो गौरिति प्र‚त्य‚भिज्ञान‚मेक‚ताध्य‚व‚सायि य‚दुत्प‚द्य‚ते त‚न्न स्यादिति । ‚{\tiny $_{lb}$}‚त‚त्राह (।)
	\pend% ending standard par
      
	  \bigskip
	  \begingroup
	
	    \large
	  
	    \begin{quote}
	  
	    
	    \stanza[\smallbreak]
	\label{pv.2.236}\flagstanza{\tiny\textenglish{....2.236}}वृत्तेर्दृश्याप‚राम‚र्शेनाभिधान‚विक‚ल्प‚योः ।&द‚र्श‚नात् प्र‚त्य‚भिज्ञानं ग‚वादीनां निवारित‚म् ॥ २३६ ॥\&[\smallbreak]


	
	    \end{quote}
	  
	  \endgroup
	\textsuperscript{\textenglish{37b/MA}}

	  \pstart \leavevmode% starting standard par
	\hphantom{.}‚{\color{DodgerBlue3}‚अभिधान‚विक‚ल्प}‚योर्दृ श्य‚स्याध्य‚क्ष‚विष‚य‚स्या‚{\color{DodgerBlue3}‚प‚राम‚र्शेन} विष‚यीक‚र‚णेन ‚{\color{DodgerBlue3}‚वृत्तेः}‚{\tiny $_{7}$}‚ ‚{\tiny $_{lb}$}‚प‚श्चाद् ‚{\color{DodgerBlue3}‚द‚र्श‚नात्} स एवायं श‚ब्द‚निर्दिष्टो गौरिति ‚{\color{DodgerBlue3}‚प्र‚त्य\edtext{}{\edlabel{pvv.190-1}\label{pvv.190-1}\lemma{त्य}\Bfootnote{मिथ्याव‚त् ।}} भिज्ञानं ग‚वादीनां} य‚दिष्य‚ते ‚{\color{DodgerBlue3}‚त‚न्निवारितं} बोद्ध‚व्यं । इन्द्रिय‚श‚ब्द‚ज्ञान‚योर्भिन्नाकार‚त्वेनैव विष‚य‚त्त्वा‚{\tiny $_{lb}$}‚भावात् क‚थं त\edtext{}{\edlabel{pvv.190-2}\label{pvv.190-2}\lemma{त}\Bfootnote{पूर्व्वाप‚रैक‚ता इन्द्रिय‚ज्ञानोत्त‚र‚भावि ।}}देक‚ताव्य‚व‚सायि व‚स्तुविष‚यं स्यात् ।(२३६)
	\pend% ending standard par
      \label{div_pvv.2.237}
	  
	% new div opening: depth here is 2
	
	  \bigskip
	  \begingroup
	
	    \large
	  
	    \begin{quote}
	  
	    
	    \stanza[\smallbreak]
	\label{pv.2.237}\flagstanza{\tiny\textenglish{....2.237}}अन्व‚याच्चानुमानं य‚द‚भिधान‚विक‚ल्प‚योः ।&दृश्ये ग‚वादौ जात्यादेस्त‚द‚प्येतेन दूषित‚म् ॥ २३७ ॥\&[\smallbreak]


	
	    \end{quote}
	  
	  \endgroup
	

	  \pstart \leavevmode% starting standard par
	\hphantom{.}य‚च्च ‚{\color{DodgerBlue3}‚दृश्ये ग‚वादा}‚व‚नेक‚त्राभिन्नाकार‚योर‚{\color{DodgerBlue3}‚भिधान‚विक‚ल्प‚योर‚न्व‚याद‚नुवृत्ते‚{\tiny $_{lb}$}‚र्जात्यादेर‚नुमानं प‚रैरुच्य‚ते त‚द‚प्येते}‚नाभिधान‚विक‚ल्प‚योर्दृश्य‚त्वाप‚राम‚र्शेन हेतुना ‚{\tiny $_{lb}$}‚‚{\color{DodgerBlue3}‚दूषितं} बोद्ध‚व्यं । न ह्य‚न्व‚यिनाव‚पि श‚ब्द‚विक‚ल्पौ व‚स्तु स्पृश‚तः । त‚त्क‚थं ताभ्यां ‚{\tiny $_{lb}$}‚व‚स्तुनः सामान्य‚स्य सिद्धिः । (२३७)
	\pend% ending standard par
      \label{div_pvv.2.238}
	  
	% new div opening: depth here is 2
	

	  \pstart \leavevmode% starting standard par
	य‚दि नास्ति सामान्यं त‚था क‚थ‚म‚स्तु प्र‚त्य‚भिज्ञान‚मित्याह (।)
	\pend% ending standard par
      \textsuperscript{\textenglish{191/s}}
	  \bigskip
	  \begingroup
	
	    \large
	  
	    \begin{quote}
	  
	    
	    \stanza[\smallbreak]
	\label{pv.2.238}\flagstanza{\tiny\textenglish{....2.238}}द‚र्श‚नान्येव भिन्नान्य‚प्येकां कुर्व‚न्ति क‚ल्प‚नाम् ।&प्र‚त्य‚भिज्ञान‚संख्यातां स्व‚भावेनेति व‚र्णित‚म् ॥ २३८ ॥\&[\smallbreak]


	
	    \end{quote}
	  
	  \endgroup
	

	  \pstart \leavevmode% starting standard par
	\hphantom{.}‚{\color{DodgerBlue3}‚द‚र्श‚नानी}‚न्द्रिय‚ज्ञा‚{\color{DodgerBlue3}‚नान्येव भिन्नानि} नानाक्ष‚ण‚विष‚याण्य‚नेका‚{\color{DodgerBlue3}‚न्य‚पि स्व‚भावेन} प्र‚त्य‚भिज्ञान‚कार‚ण‚स्व‚रूपेण स्व‚कार‚ण‚प्र‚सूतेन क‚ल्प‚नात्मेक‚त्वाध्य‚व‚सायिनीं ‚{\color{DodgerBlue3}‚प्र‚त्य‚{\tiny $_{lb}$}‚भिज्ञान‚संख्यातां} प्र‚त्य‚भिज्ञान‚नाम्ना प्र‚सिद्धां ‚{\color{DodgerBlue3}‚कुर्व‚न्तीति व‚र्ण्णितं} प्राक्\edtext{}{\edlabel{pvv.191-1}\label{pvv.191-1}\lemma{प्राक्}\Bfootnote{प्र‚थ‚म‚प‚रिच्छेदे ।}} ।
	\pend% ending standard par
      

	  \pstart \leavevmode% starting standard par
	त‚स्मात्स्थित‚मेत‚त् प्र‚त्य‚क्ष‚म‚निर्देश्य‚त्वाद‚विक‚ल्प‚{\tiny $_{2}$}‚मिति । उक्त‚मिन्द्रिय‚प्र‚त्य‚क्षं ॥ ‚{\tiny $_{lb}$}‚XX ॥ (२३८)
	\pend% ending standard par
      \label{div_pvv.2.239}
	  
	% new div opening: depth here is 2
	

	  \begin{center}%% label @type='head'
	\textbf{(२) मान‚स‚प्र‚त्य‚क्ष‚म्}
	\end{center}
	

	  \pstart \leavevmode% starting standard par
	मान‚स‚माख्यातुमाह (।) त‚च्चेन्द्रिय‚ज्ञानान‚न्त‚र‚मिष्टं । तेन स‚हैक‚विष‚यं भिन्न‚{\tiny $_{lb}$}‚विष‚यं वा स्यात् । उभ‚य‚थापि तु दोष इत्याह (।)
	\pend% ending standard par
      
	  \bigskip
	  \begingroup
	
	    \large
	  
	    \begin{quote}
	  
	    
	    \stanza[\smallbreak]
	\label{pv.2.239}\flagstanza{\tiny\textenglish{....2.239}}पूर्व्वानुभूत‚ग्र‚ह‚णे मान‚स‚स्याप्र‚माण‚ता ।&अदृष्ट‚ग्र‚ह‚णेन्धादेर‚पि स्याद‚र्थ‚द‚र्श‚न‚म् ॥ २३९ ॥\&[\smallbreak]


	
	    \end{quote}
	  
	  \endgroup
	

	  \pstart \leavevmode% starting standard par
	\hphantom{.}‚{\color{DodgerBlue3}‚पूर्व्वानुभूत}‚स्येन्द्रिय‚ज्ञान‚गृहीत‚स्य ‚{\color{DodgerBlue3}‚ग्र‚ह‚णे मान‚स‚स्य} स्वीक्रिय‚माणे‚{\color{DodgerBlue3}‚ऽप्र‚माण‚ता} स्यात् । \edlabel{pvv-sankrtyayana-text__36r1PD20IN0G379JB49SEV63HIA}\label{pvv-sankrtyayana-text__36r1PD20IN0G379JB49SEV63HIA}अज्ञाता(र्थ) प्र‚काश‚स्य प्र‚माण‚त्वात् । इन्द्रिय‚ज्ञाना‚{\color{DodgerBlue3}‚दृष्ट‚स्य ग्र‚ह‚णे} पुन‚रिष्य‚{\tiny $_{lb}$}‚माणे‚{\color{DodgerBlue3}‚ऽन्धादेर‚पि स्याद‚र्थ}‚स्य रूपादे‚{\color{DodgerBlue3}‚र्द‚र्श‚नं} । (२३९)
	\pend% ending standard par
      \label{div_pvv.2.240}
	  
	% new div opening: depth here is 2
	

	  \pstart \leavevmode% starting standard par
	इन्द्रिय‚ज्ञानानुभूत‚विष‚य‚त्वे‚{\tiny $_{3}$}‚पि द्वौ विक‚ल्पौ सोऽर्थः क्ष‚णिको न वा ।
	\pend% ending standard par
      
	  \bigskip
	  \begingroup
	
	    \large
	  
	    \begin{quote}
	  
	    
	    \stanza[\smallbreak]
	\label{pv.2.240}\flagstanza{\tiny\textenglish{....2.240}}क्ष‚णिक‚त्वाद‚तीत‚स्य द‚र्श‚ने च न स‚म्भ‚वः ।&वाच्य‚म‚क्ष‚णिक‚त्वे स्याल्ल‚क्ष‚णं स‚विशेष‚ण‚म् ॥ २४० ॥\&[\smallbreak]


	
	    \end{quote}
	  
	  \endgroup
	

	  \pstart \leavevmode% starting standard par
	प्र‚थ‚म‚प‚क्ष इन्द्रिय‚ज्ञान‚विष‚य‚स्यार्थ\edtext{}{\edlabel{pvv.191-2}\label{pvv.191-2}\lemma{स्यार्थ}\Bfootnote{. . . . .... सौत्रान्तिक‚स्य ।}}स्य \edtext{}{\edlabel{pvv.191-3}\label{pvv.191-3}\lemma{स्य}\Bfootnote{मान‚स‚ञ्चार्थ‚रागादि स्व‚स‚म्वित्तिर‚क‚ल्पिका ‚{\tiny $_{lb}$}‚योगिनां गुरुनिर्देशाव्य‚तिभिन्नार्थ‚मात्र‚दृगि तिस‚वृत्ति व्याख्यातुं प‚रोक्त‚दूष‚णं ‚{\tiny $_{lb}$}‚च‚तुःश्लोक्याह । वृत्तिर्मान‚स‚म‚पि रूपादिविष‚य‚म‚विक‚ल्प‚क‚म‚नुभ‚वाकार‚प्र‚वृत्त‚मिति ।}}पूर्व्व‚काल‚स्य स‚ह‚भु\edtext{}{\edlabel{pvv.191-4}\label{pvv.191-4}\lemma{भु}\Bfootnote{वैशेषिक‚स्तुल्यं क‚राम‚ल‚व‚द्विष‚य‚विष‚यित्व‚माह ।}}वो वा ‚{\color{DodgerBlue3}‚क्ष‚णिक‚{\tiny $_{lb}$}‚त्वाद‚तीत‚स्य} न\edtext{}{\edlabel{pvv.191-5}\label{pvv.191-5}\lemma{न}\Bfootnote{वैभाष्य‚स्य स ज्ञान‚विष‚य‚निरोधः ।}}ष्ट‚स्य मान‚सेन ‚{\color{DodgerBlue3}‚द‚र्श‚ने} द‚र्श‚न‚स्य ‚{\color{DodgerBlue3}‚च न स‚म्भ‚वो}‚स्ति । ‚{\color{DodgerBlue3}‚अक्ष‚णिक‚त्वे} वा ‚{\tiny $_{lb}$}‚अधिग‚तार्थाधिग‚न्तृत्वाद‚प्र‚माणं स्यात् । अथ मान‚स‚म‚धिग‚तार्थाधिग‚न्तृप्र‚माण‚{\tiny $_{lb}$}‚मिष्य‚ते त‚दापि मान‚स‚स्य ‚{\color{DodgerBlue3}‚स‚विशेष‚णं ल‚क्ष‚णं वाच्यं स्यात्} । य‚थाधिग‚त‚विष‚य‚त्वेपि ‚{\tiny $_{lb}$}‚क‚ल्प‚नापोढ‚म‚भ्रान्त मान‚सं प्र‚त्य‚क्ष‚मिति‚{\tiny $_{4}$}‚ (। २४०)
	\pend% ending standard par
      \label{div_pvv.2.241}
	  
	% new div opening: depth here is 2
	\textsuperscript{\textenglish{192/s}}

	  \pstart \leavevmode% starting standard par
	किञ्च (।)
	\pend% ending standard par
      
	  \bigskip
	  \begingroup
	
	    \large
	  
	    \begin{quote}
	  
	    
	    \stanza[\smallbreak]
	\label{pv.2.241}\flagstanza{\tiny\textenglish{....2.241}}निष्पादित‚क्रिये किञ्चिद् विशेष‚म‚स‚माद‚ध‚त् ।&क‚र्म्म‚ण्यैन्द्रिय‚म‚न्य‚द् वा साध‚नं किमितीष्य‚ते ॥ २४१ ॥\&[\smallbreak]


	
	    \end{quote}
	  
	  \endgroup
	

	  \pstart \leavevmode% starting standard par
	\hphantom{.}‚{\color{DodgerBlue3}‚निष्पादिता क्रिया} य‚स्मिन् त‚त्र ‚{\color{DodgerBlue3}‚क‚र्म‚णि विशेषं क‚ञ्चिद‚स‚माद‚ध‚त् ऐन्द्रिय‚मि}‚{\tiny $_{lb}$}‚न्द्रिय‚ज्ञान‚{\color{DodgerBlue3}‚म‚न्य‚द्वा} प\edtext{}{\edlabel{pvv.192-1}\label{pvv.192-1}\lemma{प}\Bfootnote{मान‚सादि ।}}र‚श्वादि ‚{\color{DodgerBlue3}‚साध‚नं किमितीष्य‚ते} (।) क्रियानिर्व्व‚र्त‚नं हि साध‚न‚व्या‚{\tiny $_{lb}$}‚पारः । त‚च्चेन्निष्प‚न्नं किम‚न्य‚त् कुर्व्व‚त्त‚त् साध‚नं स्यात् । (२४१)
	\pend% ending standard par
      \label{div_pvv.2.242}
	  
	% new div opening: depth here is 2
	

	  \pstart \leavevmode% starting standard par
	अपि च (।)
	\pend% ending standard par
      
	  \bigskip
	  \begingroup
	
	    \large
	  
	    \begin{quote}
	  
	    
	    \stanza[\smallbreak]
	\label{pv.2.242}\flagstanza{\tiny\textenglish{....2.242}}स‚कृद् भाव‚श्च स‚र्व्वासां धियां त‚द्भाव‚ज‚न्म‚नाम् ।&अन्यैर‚कार्य‚भेद‚स्य त‚द‚पेक्षाविरोध‚तः ॥ २४२ ॥\&[\smallbreak]


	
	    \end{quote}
	  
	  \endgroup
	

	  \pstart \leavevmode% starting standard par
	\hphantom{.}त‚स्मात् स्थिराद् ‚{\color{DodgerBlue3}‚भावाज्ज‚न्म} यासान्तासां ‚{\color{DodgerBlue3}‚स‚र्व्वासां धियां स‚कृद् भाव‚श्च ‚{\tiny $_{lb}$}‚स्यात्} । न क्र‚म‚भावः । क्र‚मिस‚ह‚कार्य‚पेक्ष‚या क्र‚मेण स्थिरोप्य‚र्थः क‚रोति बुद्धीरिति ‚{\tiny $_{lb}$}‚चेत् । अतो‚{\color{DodgerBlue3}‚ऽन्यैः} स‚ह‚कारिभिर‚{\tiny $_{5}$}‚‚{\color{DodgerBlue3}‚कार्यो भेदो} विशेषो य‚स्य स्थिरैक‚रूप‚स्य त‚स्य ‚{\color{DodgerBlue3}‚त‚द‚{\tiny $_{lb}$}‚पेक्षाया विरोध‚तः} ॥ (२४२)
	\pend% ending standard par
      \label{div_pvv.2.243}
	  
	% new div opening: depth here is 2
	

	  \pstart \leavevmode% starting standard par
	य‚त एवं (।)
	\pend% ending standard par
      
	  \bigskip
	  \begingroup
	
	    \large
	  
	    \begin{quote}
	  
	    
	    \stanza[\smallbreak]
	\label{pv.2.243}\flagstanza{\tiny\textenglish{....2.243}}त‚स्मादिन्द्रिय‚विज्ञानान्त‚र‚प्र‚त्य‚योद्भ‚वः ।&म‚नोन्य‚मेव गृह्णाति विष‚यं नान्ध‚दृक् त‚तः ॥ २४३ ॥\&[\smallbreak]


	
	    \end{quote}
	  
	  \endgroup
	

	  \pstart \leavevmode% starting standard par
	\hphantom{.}‚{\color{DodgerBlue3}‚त‚स्मादिन्द्रिय‚विज्ञान}‚मेवान‚न्त‚र‚{\color{DodgerBlue3}‚प्र‚त्य‚य}‚स्त‚स्मा‚{\color{DodgerBlue3}‚दुद्भ‚वो} य‚स्य त‚{\color{DodgerBlue3}‚न्म‚नो} मान‚सं ‚{\tiny $_{lb}$}‚प्र‚त्य‚क्षं । इन्द्रिय‚प्र‚त्य‚क्ष‚ग्राह्य (ा) द्विष‚या‚{\color{DodgerBlue3}‚द‚न्य‚मेव विष‚यं गृह्णाति । त‚त} इन्द्रिय‚{\tiny $_{lb}$}‚ज्ञान‚ज‚न्य‚त्वात् मान‚स‚स्यान्धानां च‚क्षुर्व्विज्ञान‚विक‚लानां दृग् द‚र्श‚नं रूप‚स्य न ‚{\tiny $_{lb}$}‚भ‚व‚ति । गृहीत‚ग्राहित्वं च विष‚यान्त‚र‚ग्र‚ह‚णाद‚पास्तं (। २४३)
	\pend% ending standard par
      \label{div_pvv.2.244}
	  
	% new div opening: depth here is 2
	

	  \pstart \leavevmode% starting standard par
	य‚द्य‚न्य‚विष‚य‚ग्राह‚कं मा‚{\tiny $_{6}$}‚न‚सं त‚दा भूत‚भ‚विष्य‚द्ग्राह‚क‚म‚पि स्यादित्याह (।)
	\pend% ending standard par
      
	  \bigskip
	  \begingroup
	
	    \large
	  
	    \begin{quote}
	  
	    
	    \stanza[\smallbreak]
	\label{pv.2.244}\flagstanza{\tiny\textenglish{....2.244}}स्वार्थान्व‚यार्थापेक्षैव हेतुरिन्द्रिय‚जा म‚तिः ।&त‚तोन्य‚ग्र‚ह‚णेप्य‚स्य निय‚त‚ग्राह्य‚ता म‚ता ॥ २४४ ॥\&[\smallbreak]


	
	    \end{quote}
	  
	  \endgroup
	

	  \pstart \leavevmode% starting standard par
	\hphantom{.}स्वार्थः स्व‚कीयो विष‚य‚स्त‚स्माद‚{\color{DodgerBlue3}‚न्व‚य} उत्पादो य‚स्यार्थ‚स्य त‚द‚{\color{DodgerBlue3}‚पेक्षैव इन्द्रिय‚{\tiny $_{lb}$}‚जा म‚ति}‚र्म‚नोविज्ञान‚स्य ‚{\color{DodgerBlue3}‚हेतु}‚रिष्य‚ते । ‚{\color{DodgerBlue3}‚त‚तोऽन्य}‚स्य विष‚य‚स्य ‚{\color{DodgerBlue3}‚ग्र‚ह‚णेपि} म‚नोविज्ञान‚स्ये‚{\tiny $_{lb}$}‚ष्य‚माणे निय‚त इन्द्रिय‚ज्ञान‚ग्राह्योपादेय‚क्ष‚ण एव ग्राह्यो य‚स्य त‚द्भावो ‚{\color{DodgerBlue3}‚निय‚त‚ग्राह्य‚ता ‚{\tiny $_{lb}$}‚सा म‚ता}\edtext{}{\edlabel{pvv.192-2}\label{pvv.192-2}\lemma{द्भावो}\Bfootnote{अभिध‚र्मेऽस्ति म‚नोविज्ञान‚स‚म‚ङ्गी तु नील‚मिद‚मिति च, म‚नो द्विधा । ‚{\tiny $_{lb}$}‚स‚विक‚ल्पं निर्व्विक‚ल्प‚ञ्च तु त‚न्मान‚सं प्र‚त्य‚क्षं विम‚तिनिरासाय व्युत्पादितं ।}}। (२४४)
	\pend% ending standard par
      \label{div_pvv.2.245}
	  
	% new div opening: depth here is 2
	\textsuperscript{\textenglish{193/s}}

	  \pstart \leavevmode% starting standard par
	न‚नु स्व‚ज्ञानेन स्वाल‚म्ब‚न‚ज्ञानेन एक‚कालिक‚स्तुल्य‚कालिकोर्थः ।
	\pend% ending standard par
      
	  \bigskip
	  \begingroup
	
	    \large
	  
	    \begin{quote}
	  
	    
	    \stanza[\smallbreak]
	\label{pv.2.245}\flagstanza{\tiny\textenglish{....2.245}}त‚द‚तुल्य‚क्रियाकालः क‚थं स्व‚ज्ञान‚कालिकः ।&स‚ह‚कारी भ‚वेद‚र्थ इति चेद‚क्ष‚चेत‚सः ॥ २४५ ॥\&[\smallbreak]


	
	    \end{quote}
	  
	  \endgroup
	

	  \pstart \leavevmode% starting standard par
	तेन स‚{\tiny $_{7}$}‚ह‚कारिस‚म्म‚तेन्द्रिय‚ज्ञाने‚{\color{DodgerBlue3}‚नातुल्यः क्रिया\edtext{}{\edlabel{pvv.193-1}\label{pvv.193-1}\lemma{क्रिया}\Bfootnote{स्व‚स‚त्ताकालः ।}}कालो} य‚स्य भिन्न‚काल‚त्वात्\leavevmode\ledsidenote{\textenglish{38a/MA}} ‚{\tiny $_{lb}$}‚सोऽर्थः ‚{\color{DodgerBlue3}‚स‚ह‚कारी} क‚थ‚{\color{DodgerBlue3}‚म‚क्ष‚चेत‚सो भ‚वेदिति चेत्} । (२४५)
	\pend% ending standard par
      \label{div_pvv.2.246}
	  
	% new div opening: depth here is 2
	

	  \pstart \leavevmode% starting standard par
	अत्राह (।)
	\pend% ending standard par
      
	  \bigskip
	  \begingroup
	
	    \large
	  
	    \begin{quote}
	  
	    
	    \stanza[\smallbreak]
	\label{pv.2.246}\flagstanza{\tiny\textenglish{....2.246}}अस‚तः प्राग‚साम‚र्थ्यात् प‚श्चाच्चानुप‚योग‚तः ।&प्राग्भावः स‚र्व्व‚हेतूनां नातोर्थः स्व‚धिया स‚ह ॥ २४६ ॥\&[\smallbreak]


	
	    \end{quote}
	  
	  \endgroup
	

	  \pstart \leavevmode% starting standard par
	\hphantom{.}कार्योत्प‚त्तेः प्राग‚स‚त‚स्त‚त्रासा‚{\color{DodgerBlue3}‚म‚र्थ्यात्} । स‚द‚धिष्ठानं हि ‚{\color{DodgerBlue3}‚साम‚र्थ्य‚म‚स‚तः क‚थं} स्यात् । कार्योत्प‚त्तेः ‚{\color{DodgerBlue3}‚प‚श्चात्} स‚तः कार‚ण‚व्यापाराद्वा प‚श्चात् कार्य‚स‚म‚काल‚स्य स‚तो ‚{\tiny $_{lb}$}‚वा त‚त्रा‚{\color{DodgerBlue3}‚नुप‚योग‚तो} व्यापाराभावात् । कार्या‚{\color{DodgerBlue3}‚त्प्राग्भावः स‚र्व्व‚हेतूनामि}‚ति स्थितं । ‚{\tiny $_{lb}$}‚विष‚य‚श्च ज्ञानानां नाकार‚ण‚म‚तिप्र‚स‚ङ्गात् । ‚{\color{DodgerBlue3}‚अतो}‚{\tiny $_{1}$}‚ विष‚यः कार‚णात्म‚कः ‚{\color{DodgerBlue3}‚स्व‚धिया} स्वाल‚म्ब‚न‚धिया ‚{\color{DodgerBlue3}‚स‚ह न} भ‚व‚ति । पूर्व्व‚भावित्वे च विष‚य‚स्य त‚त्कालेन्द्रिय‚ज्ञान‚{\tiny $_{lb}$}‚स‚ह‚कारिता युक्तिम‚ती । (२४६)
	\pend% ending standard par
      \label{div_pvv.2.247}
	  
	% new div opening: depth here is 2
	

	  \pstart \leavevmode% starting standard par
	न‚नु (।)
	\pend% ending standard par
      
	  \bigskip
	  \begingroup
	
	    \large
	  
	    \begin{quote}
	  
	    
	    \stanza[\smallbreak]
	\label{pv.2.247}\flagstanza{\tiny\textenglish{....2.247}}भिन्न‚कालं क‚थं ग्राह्य‚मिति चेद् ग्राह्य‚तां विदुः ।&हेतुत्व‚मेव युक्तिज्ञा ज्ञानाकारार्प‚ण‚क्ष‚म‚म् ॥ २४७ ॥\&[\smallbreak]


	
	    \end{quote}
	  
	  \endgroup
	

	  \pstart \leavevmode% starting standard par
	\hphantom{.}प्राग्भाव‚भावित्वाद् ‚{\color{DodgerBlue3}‚भिन्न‚कालं} व‚स्तु ‚{\color{DodgerBlue3}‚क‚थं ग्राह्य}‚मिति ‚{\color{DodgerBlue3}‚चेत् । हेतुत्व‚मेव ‚{\tiny $_{lb}$}‚ज्ञानेऽकार}‚स्य स्वानुरूप‚स्या‚{\color{DodgerBlue3}‚र्प‚ण‚क्ष‚णं ग्राह्य‚तां युक्तिज्ञा विदुः} (।) न हि संदंशा‚{\tiny $_{lb}$}‚योगोल‚योरिव ज्ञान‚प‚दार्थ‚योर्ग्राह्य‚ग्राह‚क‚भावः । क‚थ‚न्त‚र्हि य‚दाकार‚म‚नुक‚रोति त‚त् ‚{\tiny $_{lb}$}‚ग्राह्य‚स्य ग्राह‚क‚मित्युच्य‚ते । (२४७)
	\pend% ending standard par
      \label{div_pvv.2.248}
	  
	% new div opening: depth here is 2
	

	  \pstart \leavevmode% starting standard par
	न‚नु य‚दि कार‚णं ग्राह्यंत‚{\tiny $_{2}$}‚दा स‚म‚न‚न्त‚र‚प्र‚त्य‚यादिकं च त‚था स्यात् । इत्याहं (।)
	\pend% ending standard par
      
	  \bigskip
	  \begingroup
	
	    \large
	  
	    \begin{quote}
	  
	    
	    \stanza[\smallbreak]
	\label{pv.2.248}\flagstanza{\tiny\textenglish{....2.248}}कार्यं ह्य‚नेक‚हेतुत्वेऽप्य‚नुकुर्व्व‚दुदेति य‚त् ।&त‚त्तेनाप्य‚त्र त‚द्रूपं गृहीत‚मिति चोच्य‚ते ॥ २४८ ॥\&[\smallbreak]


	
	    \end{quote}
	  
	  \endgroup
	

	  \pstart \leavevmode% starting standard par
	\hphantom{.}‚{\color{DodgerBlue3}‚कार्यं} हि ज्ञान‚म‚{\color{DodgerBlue3}‚नेक‚हेतुत्वेऽपि} य‚त्कार‚ण‚माकार‚द्वारेणा‚{\color{DodgerBlue3}‚नुकुर्व्व‚दुदेति} त‚त्कार‚ण‚{\tiny $_{lb}$}‚म‚र्पि त‚द्रूप‚मुप‚रोपित‚स्वाकारं ‚{\color{DodgerBlue3}‚तेन} कार‚णाकारेण ‚{\color{DodgerBlue3}‚गृहीत‚मिति चोच्य‚ते} । य‚था पितृ‚{\tiny $_{lb}$}‚रूपं पुत्रेण गृहीत‚मिति क‚थ्य‚ते । उक्तं ‚{\color{DodgerBlue3}‚मान‚सं} ॥ X X ॥ (२४८)
	\pend% ending standard par
      \label{div_pvv.2.249}
	  
	% new div opening: depth here is 2
	\textsuperscript{\textenglish{194/s}}

	  \begin{center}%% label @type='head'
	\textbf{(३) क. स्व‚संवेद‚न‚प्र‚त्य‚क्ष‚म्}
	\end{center}
	

	  \pstart \leavevmode% starting standard par
	स्व‚स‚म्वेद‚न‚माख्यातुमाह (।)
	\pend% ending standard par
      
	  \bigskip
	  \begingroup
	
	    \large
	  
	    \begin{quote}
	  
	    
	    \stanza[\smallbreak]
	\label{pv.2.249}\flagstanza{\tiny\textenglish{....2.249}}अश‚क्य‚स‚म‚यो ह्यात्मा रागादीनाम‚न‚न्य‚भाक् ।&तेषाम‚तः स्व‚संवित्तिर्न्नाभिज‚ल्पानुष‚ङ्गिणी ॥ २४९ ॥\&[\smallbreak]


	
	    \end{quote}
	  
	  \endgroup
	

	  \pstart \leavevmode% starting standard par
	राग‚द्वेष‚सुख‚दुःखादीनां स‚र्व्व‚चित्त‚चैत्तानामात्म‚संवेद‚नं प्र‚त्य‚क्ष‚म‚विक‚ल्प‚त्वात् ।
	\pend% ending standard par
      

	  \pstart \leavevmode% starting standard par
	त‚था‚{\tiny $_{3}$}‚ हि । ‚{\color{DodgerBlue3}‚रागादीनामात्मा स्व‚रूप‚म‚न‚न्य‚भाक्} ना\edtext{}{\edlabel{pvv.194-1}\label{pvv.194-1}\lemma{ना}\Bfootnote{रागादिसुखादिषु स्व‚स‚म्वेद‚न‚मिन्द्रियान‚पेक्ष‚त्वान्मान‚सं प्र‚त्य‚क्ष‚मिति वृद्धिः । ‚{\tiny $_{lb}$}‚अत्र च नात्म‚संयोग‚मात्र‚भावित्वं सुखादेर्दृष्ट‚मिति दृष्ट‚ग्र‚ह‚णात् ।}}न्यं भ‚ज‚ते । स्व‚रूप‚{\tiny $_{lb}$}‚मात्राव‚स्थितेः । त‚स्मा‚{\color{DodgerBlue3}‚द‚श‚क्यः स‚म‚यः} संकेतोऽस्मिन् ‚{\color{DodgerBlue3}‚अतः} संकेताविष‚य‚त्वात् ‚{\color{DodgerBlue3}‚तेषां} रागादीनां ‚{\color{DodgerBlue3}‚स्व}‚स्यात्म‚नः ‚{\color{DodgerBlue3}‚स‚म्वित्तिः} प्र‚काशोऽ‚{\color{DodgerBlue3}‚भिज‚ल्पो} वाच‚क‚श‚ब्दोल्लेख‚स्त‚{\color{DodgerBlue3}‚द‚नु}‚ष‚ङ्गो ‚{\tiny $_{lb}$}‚य‚स्यास्ति सा त‚था न भ‚व‚ति । वा\edtext{}{\edlabel{pvv.194-2}\label{pvv.194-2}\lemma{वा}\Bfootnote{वैशेषिका ज्ञानाद् व्य‚तिरिक्तं सुखादिक‚मात्म‚गुण‚माहुः । सुखादिज्ञान‚{\tiny $_{lb}$}‚बाह्य‚मिति सांख्या एत‚द‚पाक‚र‚णाद्य‚र्थं मान‚स‚ग्र‚ह‚णं । सात‚तिकं स्व‚वेद‚नं स‚र्व्वेन्द्रिय‚{\tiny $_{lb}$}‚कालिक‚त्वात् । धार‚वाहि च प्र‚त्य‚क्ष‚सिद्ध‚त्वात् । अविक‚ल्प‚ञ्च स्व‚प्ने स्प‚ष्ट‚भासात् । ‚{\tiny $_{lb}$}‚काम‚शोकादिष्व‚पि स‚प्र‚योज‚न‚ञ्च । विनानेन स्मृतेर‚योगात् । स्व‚स‚म्वित्तेर्निर्व्वि‚{\tiny $_{lb}$}‚क‚ल्प‚त्वं साध्यं सा च ज्ञान‚स्यापि नास्ति कुतः सुखादीनां ।}}च्यं हि वाच‚केन संयोज्येत । न च रागाद्यात्मा ‚{\tiny $_{lb}$}‚वा\edtext{}{\edlabel{pvv.194-3}\label{pvv.194-3}\lemma{वा}\Bfootnote{यादृशो यादृशादुत्प‚न्नो दृष्ट‚स्तादृशोन्योपि तादृशादेव अन्यादृश‚स्तु यादृशा ‚{\tiny $_{lb}$}‚इति स्थिते ।}}च्य‚स्त‚त‚स्त‚त्प्र‚काशो न श‚ब्द‚संग‚तः । (२४९)
	\pend% ending standard par
      \label{div_pvv.2.250}
	  
	% new div opening: depth here is 2
	
	  \bigskip
	  \begingroup
	
	    \large
	  
	    \begin{quote}
	  
	    
	    \stanza[\smallbreak]
	\label{pv.2.250a}\flagstanza{\tiny\textenglish{...2.250a}}अवेद‚काः प‚र‚स्यापि ते स्व‚रूपं क‚थं विदुः ।\&[\smallbreak]


	
	    \end{quote}
	  
	  \endgroup
	

	  \pstart \leavevmode% starting standard par
	न‚नु राग‚सुखाद‚य आत्म‚गुणाः‚{\tiny $_{4}$}‚ ‚{\color{DodgerBlue3}‚प‚र‚स्य} बाह्य‚स्या‚{\color{DodgerBlue3}‚प्य‚वेद‚ना}‚ङ्ग‚ज्ञात‚त्वात् ‚{\color{DodgerBlue3}‚ते ‚{\tiny $_{lb}$}‚स्व‚रूपं क‚थं विदुः} । वेद‚क‚स्य क‚दाचित् स्व‚वेद‚नं संभाव्येत । अवेद‚कं च न ‚{\tiny $_{lb}$}‚क्व‚चिदुप‚योगि ॥
	\pend% ending standard par
      

	  \pstart \leavevmode% starting standard par
	य‚दि न स्व‚वेद‚ना रागाद‚य‚स्त‚दा क‚थं वेद‚य‚न्त इत्याह (।)
	\pend% ending standard par
      
	  \bigskip
	  \begingroup
	
	    \large
	  
	    \begin{quote}
	  
	    
	    \stanza[\smallbreak]
	\label{pv.2.250b}\flagstanza{\tiny\textenglish{...2.250b}}एकार्थाश्र‚यिणा वेद्या विज्ञानेनेति केच‚न ॥ २५० ॥\&[\smallbreak]


	
	    \end{quote}
	  
	  \endgroup
	

	  \pstart \leavevmode% starting standard par
	\hphantom{.}‚{\color{DodgerBlue3}‚एकोर्थ} आत्मा ‚{\color{DodgerBlue3}‚आश्र‚यो} रागादिभिः स‚ह य‚स्यास्ति ‚{\color{DodgerBlue3}‚तेनै}‚कार्थाश्र‚यिणा ‚{\color{DodgerBlue3}‚ज्ञानेन वेद्या} रागाद‚य ‚{\color{DodgerBlue3}‚इति केच‚न नै} या यि का द‚यः । (२५०)
	\pend% ending standard par
      \label{div_pvv.2.251}
	  
	% new div opening: depth here is 2
	

	  \pstart \leavevmode% starting standard par
	अत्राह (।)
	\pend% ending standard par
      \textsuperscript{\textenglish{195/s}}
	  \bigskip
	  \begingroup
	
	    \large
	  
	    \begin{quote}
	  
	    
	    \stanza[\smallbreak]
	\label{pv.2.251}\flagstanza{\tiny\textenglish{....2.251}}त‚द‚त‚द्रूपिणो भाव‚स्त‚द‚त‚द्रूप‚हेतुजाः ।&त‚त्सुखादि किम‚ज्ञानं विज्ञानाभिन्न‚हेतुज‚म् ॥ २५१ ॥\&[\smallbreak]


	
	    \end{quote}
	  
	  \endgroup
	

	  \pstart \leavevmode% starting standard par
	\hphantom{.}‚{\color{DodgerBlue3}‚त‚द्रूपिणो} विव‚क्षितैक‚रूप‚व‚न्तोऽ‚{\color{DodgerBlue3}‚त‚द्रू}‚पिण इत‚र‚रूप‚व‚न्तो भावा य‚थाक्र‚मं त‚द्रू‚{\tiny $_{5}$}‚‚{\tiny $_{lb}$}‚पाद् दृष्टैक‚रूपाद्धेतोः साम‚ग्रील‚क्ष‚णाज्जाता ‚{\color{DodgerBlue3}‚अत‚द्रूप‚हेतुजाता} विल‚क्ष‚ण‚साम‚ग्री‚{\tiny $_{lb}$}‚जाता भ‚व‚न्तीति ताव‚त् स्थितं । त‚त् त‚स्मादिमं न्याय‚मुल्ल‚ङ्ध्य ‚{\color{DodgerBlue3}‚विज्ञानेन स‚हाभिन्न} एको ‚{\color{DodgerBlue3}‚हेतु}‚रिन्द्रिय‚विष‚य‚म‚न‚स्कारादिसाम‚ग्रील‚क्ष‚ण‚स्त‚स्मा (त्) जातं ‚{\color{DodgerBlue3}‚सुखादिकं} क‚स्माद‚ज्ञानं स‚मान‚साम‚ग्रीप्र‚सूत‚त्वात् द्व‚य‚म‚पि ज्ञानं स्यान्न वा किञ्चित् । (२५१)
	\pend% ending standard par
      \label{div_pvv.2.252}
	  
	% new div opening: depth here is 2
	

	  \pstart \leavevmode% starting standard par
	अभिन्न‚हेतुक‚तामेव स‚म‚र्थ‚यितुमाह (।)
	\pend% ending standard par
      
	  \bigskip
	  \begingroup
	
	    \large
	  
	    \begin{quote}
	  
	    
	    \stanza[\smallbreak]
	\label{pv.2.252}\flagstanza{\tiny\textenglish{....2.252}}सार्थे स‚तीन्द्रिये योग्ये य‚थास्व‚म‚पि चेत‚सि ।&दृष्टं ज‚न्म सुखादीनां त‚त्तुल्यं म‚न‚साम‚पि ॥ २५२ ॥\&[\smallbreak]


	
	    \end{quote}
	  
	  \endgroup
	

	  \pstart \leavevmode% starting standard par
	\hphantom{.}य‚स्य य‚दात्मीयं ज‚न‚कं त‚स्मि‚{\color{DodgerBlue3}‚न्निन्द्रिये सार्थे} स‚विष‚ये योग्ये कार्योत्प‚{\tiny $_{6}$}‚ाद‚नं ‚{\tiny $_{lb}$}‚प्र‚ति ‚{\color{DodgerBlue3}‚चेत‚सि} स‚म‚न‚न्त‚र‚प्र‚त्य‚ये स‚ति ‚{\color{DodgerBlue3}‚सुखादीना}‚म‚पि ‚{\color{DodgerBlue3}‚ज‚न्म दृष्टं} । त‚द्य‚थास्व‚मिन्द्रिया‚{\tiny $_{lb}$}‚दिषु योग्येषु स‚त्सु ज‚न्म द‚र्श‚नं ‚{\color{DodgerBlue3}‚म‚न}‚सां ज्ञानाना‚{\color{DodgerBlue3}‚म‚पि तुल्यं} । अत‚स्तुल्य‚हेतुक‚त्वात् ‚{\tiny $_{lb}$}‚तुल्य‚जातीय‚तैव युक्ता नान्य‚था क्व‚चिदेक‚ता स्यात् । (२५२)
	\pend% ending standard par
      \label{div_pvv.2.253}
	  
	% new div opening: depth here is 2
	

	  \pstart \leavevmode% starting standard par
	तुल्य‚हेतुक‚त्वेपि संस्कारादेर्नियाम‚क‚त्वात् सुखादिक‚म‚ज्ञानं स्यादित्याह (।)
	\pend% ending standard par
      
	  \bigskip
	  \begingroup
	
	    \large
	  
	    \begin{quote}
	  
	    
	    \stanza[\smallbreak]
	\label{pv.2.253}\flagstanza{\tiny\textenglish{....2.253}}अस‚त्सु स‚त्सु चैतेषु न ज‚न्माज‚न्म वा क्व‚चित् ।&दृष्टं सुखादेर्बुद्धेर्वा त‚त्त‚तो नान्य‚त‚श्च ते ॥ २५३ ॥\&[\smallbreak]


	
	    \end{quote}
	  
	  \endgroup
	

	  \pstart \leavevmode% starting standard par
	\hphantom{.}‚{\color{DodgerBlue3}‚अस}‚त्स्विन्द्रियादिषु दुःखादेर्बुद्धेर्व्वा ज‚न्म न क्व‚चिदिष्टं । ‚{\color{DodgerBlue3}‚स‚त्सु} च एतेष्वि‚{\color{DodgerBlue3}‚न्द्रिया}‚{\tiny $_{lb}$}‚दिषु अज‚न्म वा ‚{\color{DodgerBlue3}‚सुखादेर्ब्बुद्धेर्व्वा न क्व‚चिद् दृष्टं}‚{\tiny $_{7}$}‚ (।) त‚त् त‚स्मात् त‚त इन्द्रियादे-\leavevmode\ledsidenote{\textenglish{38b/MA}} ‚{\tiny $_{lb}$}‚र्दृष्ट‚साम‚र्थ्यात् कार‚णात् ते सुख‚बुद्धी जायेते । ‚{\color{DodgerBlue3}‚अन्य‚तः} संस्कारादे‚{\color{DodgerBlue3}‚र्न ते} त‚स्य ‚{\tiny $_{lb}$}‚साम‚र्थ्याद‚र्श‚नात् । (२५३)
	\pend% ending standard par
      \label{div_pvv.2.254}
	  
	% new div opening: depth here is 2
	

	  \begin{center}%% label @type='head'
	\textbf{(ख. सुखादिप‚र‚वेद्य‚तानिर‚स)}
	\end{center}
	‚{\tiny $_{lb}$}‚

	  \pstart \leavevmode% starting standard par
	न‚न्व‚भिन्न‚हेतुक‚त्वे सुख‚दुःखादिभेद‚श्च न स्यादित्याह (।)
	\pend% ending standard par
      
	  \bigskip
	  \begingroup
	
	    \large
	  
	    \begin{quote}
	  
	    
	    \stanza[\smallbreak]
	\label{pv.2.254}\flagstanza{\tiny\textenglish{....2.254}}सुख‚दुःखादिभेद‚श्च तेषामेव विशेष‚तः ॥&त‚स्या एव य‚था बुद्धेर्म्मान्द्य‚पाट‚व‚संश्र‚याः ॥ २५४ ॥\&[\smallbreak]


	
	    \end{quote}
	  
	  \endgroup
	

	  \pstart \leavevmode% starting standard par
	\hphantom{.}‚{\color{DodgerBlue3}‚सुख‚दुःखादिभेद‚श्चा}‚वान्त‚र‚{\color{DodgerBlue3}‚स्तेषामि}‚न्द्रियादीना‚{\color{DodgerBlue3}‚मेव} हेतूनां ‚{\color{DodgerBlue3}‚विशेष‚तो}‚ऽवान्त‚{\tiny $_{lb}$}‚रात् । ‚{\color{DodgerBlue3}‚य‚था त‚स्या बुद्धे}‚रेवान्त‚र‚हेतुविशेषात् ‚{\color{DodgerBlue3}‚मान्द्य‚पाट‚व‚संश्र‚याः} प‚र‚स्प‚र‚संभ‚विनो ‚{\tiny $_{lb}$}‚भ‚व‚न्ति (।) त‚स्मात् सुखाद‚यो विज्ञानेनाभिन्न‚हेतुक‚त्वाद्विज्ञान‚स्व‚भावा एवेति‚{\tiny $_{1}$}‚ ‚{\tiny $_{lb}$}‚स्थितं । (२५४)
	\pend% ending standard par
      \label{div_pvv.2.255}
	  
	% new div opening: depth here is 2
	\textsuperscript{\textenglish{196/s}}

	  \pstart \leavevmode% starting standard par
	इदानीं प‚र‚वेद्य‚तामेषां निषेद्ध्ुमाह (।)
	\pend% ending standard par
      
	  \bigskip
	  \begingroup
	
	    \large
	  
	    \begin{quote}
	  
	    
	    \stanza[\smallbreak]
	\label{pv.2.255}\flagstanza{\tiny\textenglish{....2.255}}य‚स्यार्थ‚स्य निपातेन ते जाता धीसुखाद‚यः ।&मुक्त्वा तं प्र‚तिप‚द्येत सुखादीनेव सा क‚थ‚म् ॥ २५५ ॥\&[\smallbreak]


	
	    \end{quote}
	  
	  \endgroup
	

	  \pstart \leavevmode% starting standard par
	\hphantom{.}‚{\color{DodgerBlue3}‚य‚स्यार्थ‚स्य} स्त्र्यादे‚{\color{DodgerBlue3}‚र्निपातेन} स‚न्निधानेन ‚{\color{DodgerBlue3}‚ते धीसुखाद‚यो जाता}‚स्तं कार‚ण‚भूत‚{\tiny $_{lb}$}‚माकारार्प‚ण‚क्ष‚मं ‚{\color{DodgerBlue3}‚मुक्त्वा} सुखादीनेव स‚ह‚भाविनो विष‚य‚ल‚क्ष‚ण‚र‚हितान् सा बुद्धिः ‚{\tiny $_{lb}$}‚‚{\color{DodgerBlue3}‚क‚थं प्र‚तिप‚द्येत} । अकार‚ण‚स्य विष‚य‚त्वेतिप्र‚स‚ङ्गात् । (२५५)
	\pend% ending standard par
      \label{div_pvv.2.256}
	  
	% new div opening: depth here is 2
	
	  \bigskip
	  \begingroup
	
	    \large
	  
	    \begin{quote}
	  
	    
	    \stanza[\smallbreak]
	\label{pv.2.256}\flagstanza{\tiny\textenglish{....2.256}}अविछिन्नाथ भासेत त‚त्संवित्तिः क्र‚म‚ग्र‚हे ।&त‚ल्लाघ‚वाच्चेत्त‚त्तुल्य‚भित्य‚संवेद‚नं न किम् ॥ २५६ ॥\&[\smallbreak]


	
	    \end{quote}
	  
	  \endgroup
	

	  \pstart \leavevmode% starting standard par
	\hphantom{.}‚{\color{DodgerBlue3}‚अथे}‚न्द्रिय‚ज्ञानेन विष‚य‚ग्र‚ह‚स्त‚द‚न‚न्त‚रं मान‚साध्य‚क्षेण सुखादिग्र‚ह इति ‚{\color{DodgerBlue3}‚क्र‚म‚ग्र‚हे} स्वीक्रिय‚माणे त‚योर्व्विष‚य‚सुख‚योः ‚{\color{DodgerBlue3}‚संवित्तिर‚विच्छिन्ना} यौग‚प‚{\tiny $_{2}$}‚द्येन न ‚{\color{DodgerBlue3}‚भासेत} । ‚{\tiny $_{lb}$}‚अस्ति च युग‚प‚त्प्र‚तिभासः प‚रिस्फुटः । त‚योर्ब्बाह्य‚सुख‚ग्राहिम‚न‚सोर्लाघ‚वात् ‚{\tiny $_{lb}$}‚यौग‚प‚द्य‚प्र‚तिभास‚{\color{DodgerBlue3}‚श्चेत् । त}‚ल्लाघ‚वं प्र‚तिक्ष‚ण‚मेकैक‚त्व‚संवेद‚नाभाव‚स्यापि ‚{\color{DodgerBlue3}‚तुल्य‚मि‚{\tiny $_{lb}$}‚त्य‚स‚म्वेद‚न‚मेव} साहित्य‚स्य ‚{\color{DodgerBlue3}‚किं न भ‚व‚ति} । (२५६)
	\pend% ending standard par
      \label{div_pvv.2.257}
	  
	% new div opening: depth here is 2
	

	  \pstart \leavevmode% starting standard par
	इन्द्रिय‚बुद्ध्यैवैक‚या बाह्य‚सुख‚योः स‚कृद्ग्र‚ह‚ण‚मिति चेत् । आह ।
	\pend% ending standard par
      
	  \bigskip
	  \begingroup
	
	    \large
	  
	    \begin{quote}
	  
	    
	    \stanza[\smallbreak]
	\label{pv.2.257a}\flagstanza{\tiny\textenglish{...2.257a}}न चैक‚या द्व‚य‚ज्ञानं निय‚माद‚क्ष‚चेत‚सः ।\&[\smallbreak]


	
	    \end{quote}
	  
	  \endgroup
	

	  \pstart \leavevmode% starting standard par
	\hphantom{.}‚{\color{DodgerBlue3}‚न चैक‚या द्व‚य‚स्य ज्ञानं} संभ‚व‚ति । ‚{\color{DodgerBlue3}‚अक्ष‚चेत‚सो}‚र्ब्बाह्य‚रूपादिग्र‚ह‚ण एव ‚{\color{DodgerBlue3}‚निय‚मा ‚{\tiny $_{lb}$}‚न्नि}‚य‚त‚त्वात् ‚{\color{DodgerBlue3}‚सुखा\edtext{}{\edlabel{pvv.196-1}\label{pvv.196-1}\lemma{सुखा}\Bfootnote{सुखाद्यात्म‚नः संयोग‚ज‚म‚न‚सा गृह्य‚ते इत्युप‚ग‚मात् ।}} द्या}‚त्म‚गुण‚ग्र‚ह‚णा‚{\tiny $_{3}$}‚‚{\color{DodgerBlue3}‚भा}‚वात् ।
	\pend% ending standard par
      

	  \pstart \leavevmode% starting standard par
	किञ्च (।)
	\pend% ending standard par
      
	  \bigskip
	  \begingroup
	
	    \large
	  
	    \begin{quote}
	  
	    
	    \stanza[\smallbreak]
	\label{pv.2.257b}\flagstanza{\tiny\textenglish{...2.257b}}सुखाद्य‚भावेप्य‚र्थाच्च जातेस्त‚च्छ‚क्त्य‚सिद्धितः ॥ २५७ ॥\&[\smallbreak]


	
	    \end{quote}
	  
	  \endgroup
	

	  \pstart \leavevmode% starting standard par
	\hphantom{.}नाकार‚णं विष‚यः ‚{\color{DodgerBlue3}‚सुखाद्य‚भावेपि} केव‚ला‚{\color{DodgerBlue3}‚द‚र्थादि}‚न्द्रिय‚बुद्धेर्जातेरुत्पादात् । त‚स्य ‚{\tiny $_{lb}$}‚सुखादेरिन्द्रिय‚बुद्धिज‚न‚नं प्र‚ति ‚{\color{DodgerBlue3}‚श‚क्त्य‚सि}‚द्धेर‚विष‚य‚त्वं ।\edtext{\textsuperscript{*}}{\edlabel{pvv.196-2}\label{pvv.196-2}\lemma{*}\Bfootnote{सुखाद‚यो नात्मानं विदुरिति बुद्ध्या वेद्येर‚न्निति किं स‚ह‚ज‚या उत्त‚र‚काल‚या ‚{\tiny $_{lb}$}‚वा त‚त्र । स‚हिता व्य‚स्ताश्च य‚था नीलाद‚यो ज्ञान‚हेतुस्त‚द्व‚त् सुखादिर्व्विष‚य‚श्च ‚{\tiny $_{lb}$}‚स्यादिति द्व‚यं ग्र‚ह‚ण‚म‚विरुद्ध‚मित्याह ।}}(२५७)
	\pend% ending standard par
      \label{div_pvv.2.258}
	  
	% new div opening: depth here is 2
	
	  \bigskip
	  \begingroup
	
	    \large
	  
	    \begin{quote}
	  
	    
	    \stanza[\smallbreak]
	\label{pv.2.258a}\flagstanza{\tiny\textenglish{...2.258a}}पृथ‚क् पृथ‚क् च साम‚र्थ्ये द्व‚योर्नीलादिव‚त् सुख‚म् ।&गृह्येत केव‚लं;\&[\smallbreak]


	
	    \end{quote}
	  
	  \endgroup
	

	  \pstart \leavevmode% starting standard par
	\hphantom{.}‚{\color{DodgerBlue3}‚द्व‚यो} रूपिसुखाद्योः ‚{\color{DodgerBlue3}‚पृथ‚क् पृथ‚ग्ज्ञा}‚नोप‚प‚त्तौ ‚{\color{DodgerBlue3}‚साम‚र्थ्ये} वाभ्युप‚ग‚म्य‚माने ‚{\color{DodgerBlue3}‚नीला‚{\tiny $_{lb}$}‚दिव‚त् केव‚लं सुखं गृह्येत} ।
	\pend% ending standard par
      \textsuperscript{\textenglish{197/s}}

	  \pstart \leavevmode% starting standard par
	अयुक्तं चैत‚त् ।
	\pend% ending standard par
      
	  \bigskip
	  \begingroup
	
	    \large
	  
	    \begin{quote}
	  
	    
	    \stanza[\smallbreak]
	\label{pv.2.258b}\flagstanza{\tiny\textenglish{...2.258b}}त‚स्य त‚द्धेत्व‚र्थ‚म‚गृह्ण‚तः ॥ २५८ ॥\&[\smallbreak]


	
	    \end{quote}
	  
	  \endgroup
	

	  \pstart \leavevmode% starting standard par
	\hphantom{.}‚{\color{DodgerBlue3}‚न हि त‚स्य} सुखादे‚{\color{DodgerBlue3}‚र्हेतुम‚र्थं} स्त्र्यादि‚{\color{DodgerBlue3}‚म‚गृह्ण‚तः}\edtext{}{\edlabel{pvv.197-1}\label{pvv.197-1}\lemma{स्त्र्यादि}\Bfootnote{अर्थ‚ग्र‚ह‚ण‚निर‚पेक्षं न त‚त् सिद्धिक‚ल्पं ॥}}(। २५८)
	\pend% ending standard par
      \label{div_pvv.2.259}
	  
	% new div opening: depth here is 2
	
	  \bigskip
	  \begingroup
	
	    \large
	  
	    \begin{quote}
	  
	    
	    \stanza[\smallbreak]
	\label{pv.2.259a}\flagstanza{\tiny\textenglish{...2.259a}}न हि संवेद‚नं युक्त‚म‚र्थेनैव स‚ह ग्र‚हे ।\&[\smallbreak]


	
	    \end{quote}
	  
	  \endgroup
	

	  \pstart \leavevmode% starting standard par
	\hphantom{.}सुख‚स्य ‚{\color{DodgerBlue3}‚संवेद‚नं युक्तं । अर्थेनैव स‚ह} सुख‚स्येन्द्रिय‚धिया ‚{\color{DodgerBlue3}‚ग्र‚हे} नास्ति ‚{\color{DodgerBlue3}‚दोष}‚{\tiny $_{4}$}‚ इति ‚{\tiny $_{lb}$}‚चेत् । न (।)
	\pend% ending standard par
      
	  \bigskip
	  \begingroup
	
	    \large
	  
	    \begin{quote}
	  
	    
	    \stanza[\smallbreak]
	\label{pv.2.259b}\flagstanza{\tiny\textenglish{...2.259b}}किं साम‚र्थ्यं सुखादीनां नेष्टा धीर्य‚त्त‚दुद्भ‚वा ॥ २५९ ॥\&[\smallbreak]


	
	    \end{quote}
	  
	  \endgroup
	

	  \pstart \leavevmode% starting standard par
	\hphantom{.}‚{\color{DodgerBlue3}‚सुखा}‚देरिन्द्रिय‚बुद्धिज‚न‚ने ‚{\color{DodgerBlue3}‚किम}‚स्ति ‚{\color{DodgerBlue3}‚साम‚र्थ्यं} । येन त‚स्याः स विष‚यः स‚न् स‚ह ‚{\tiny $_{lb}$}‚गृह्येत । ‚{\color{DodgerBlue3}‚य‚द्य}‚स्मा‚{\color{DodgerBlue3}‚त्त‚दुद्भ‚वा} सुखादुद्भ‚वा नेन्द्रिय‚{\color{DodgerBlue3}‚धीरिष्टा} त‚स्मान्न त‚द्ग्राहिका । ‚{\tiny $_{lb}$}‚(२५९)
	\pend% ending standard par
      \label{div_pvv.2.260}
	  
	% new div opening: depth here is 2
	
	  \bigskip
	  \begingroup
	
	    \large
	  
	    \begin{quote}
	  
	    
	    \stanza[\smallbreak]
	\label{pv.2.260}\flagstanza{\tiny\textenglish{....2.260}}विनार्थेन सुखादीनां वेद‚ने च‚क्षुरादिभिः ।&रूपादिः स्त्र्यादिभेदोऽक्ष्णा न गृह्येत क‚दाच‚न ॥ २६० ॥\&[\smallbreak]


	
	    \end{quote}
	  
	  \endgroup
	

	  \pstart \leavevmode% starting standard par
	\hphantom{.}त‚त‚श्च क‚थ‚म‚र्थ‚सुख‚योः स‚ह‚वेद‚नं । ‚{\color{DodgerBlue3}‚अर्थेन विनैव च‚क्षुरादि}‚भिश्च‚क्षुरादिज्ञानैः ‚{\tiny $_{lb}$}‚‚{\color{DodgerBlue3}‚सुखादीनां} वेद‚ने चाभ्युप‚ग‚म्य‚माने ‚{\color{DodgerBlue3}‚स्त्र्यादेर्भेदो} विशेषो ‚{\color{DodgerBlue3}‚रूपादिः} सुख‚हेतुर‚क्ष्णा च‚क्षुर्ज्ञा‚{\tiny $_{lb}$}‚नेन (न) ‚{\color{DodgerBlue3}‚क‚दाच‚न गृह्येत} । (२६०)
	\pend% ending standard par
      \label{div_pvv.2.261}
	  
	% new div opening: depth here is 2
	

	  \pstart \leavevmode% starting standard par
	क‚स्मादेव‚मित्याह (।)
	\pend% ending standard par
      
	  \bigskip
	  \begingroup
	
	    \large
	  
	    \begin{quote}
	  
	    
	    \stanza[\smallbreak]
	\label{pv.2.261}\flagstanza{\tiny\textenglish{....2.261}}न हि स‚त्य‚न्त‚र‚ङ्गेर्थे श‚क्ते धीर्बाह्य‚द‚र्श‚नो ।&अर्थ‚ग्र‚हे सुखादीनां त‚ज्जानां स्याद‚वेद‚न‚म् ॥ २६१ ॥\&[\smallbreak]


	
	    \end{quote}
	  
	  \endgroup
	

	  \pstart \leavevmode% starting standard par
	\hphantom{.}‚{\color{DodgerBlue3}‚अन्त}‚र‚ङ्गे स‚न्निहि‚{\tiny $_{5}$}‚ते ‚{\color{DodgerBlue3}‚अर्थे} क्षेत्र‚ज्ञ‚स‚म‚वेते सुखादौ ‚{\color{DodgerBlue3}‚श‚क्ते} स्व‚ग्राहिज्ञानो‚{\tiny $_{lb}$}‚त्पाद‚न‚क्ष‚मे स‚ति न हि ‚{\color{DodgerBlue3}‚बाह्य‚द‚र्श‚नी धी}‚र‚र्थ‚ग्राहिणी युक्ता । ब‚हिर‚ङ्गो बाह्योर्थ ‚{\tiny $_{lb}$}‚इन्द्रियालोकादिस‚ह‚कार्य‚पेक्ष‚णात् । सुखादिस्तु न त‚त्सापेक्ष इति त‚स्माज्जाता ‚{\tiny $_{lb}$}‚धीस्त‚मेव गृह्णीयात् । इन्द्रिय‚धिया‚{\color{DodgerBlue3}‚ऽर्थ‚ग्र‚हे} स्वीक्रिय‚माणे ‚{\color{DodgerBlue3}‚सुखादीनां} त‚ज्जानाम‚र्थ‚{\tiny $_{lb}$}‚द‚र्श‚न‚प्र‚सूतानाम‚{\color{DodgerBlue3}‚र्थ‚म‚वेद‚नं स्यात्} । इन्द्रिय‚बुद्धिर‚र्थ‚ग्र‚ह एवोप‚युक्ता त‚त्काल‚म‚न्या ‚{\tiny $_{lb}$}‚च धीर्नास्ति प‚श्चात् ज्ञाना‚{\tiny $_{6}$}‚न्त‚र‚काले विष‚य‚ब‚ल‚भाविनः सुखादे\edtext{}{\edlabel{pvv.197-2}\label{pvv.197-2}\lemma{सुखादे}\Bfootnote{विष‚याभावादेव ।}}र‚भाव इति क‚थं ‚{\tiny $_{lb}$}‚वेद‚नं । (२६१)
	\pend% ending standard par
      \label{div_pvv.2.262}
	  
	% new div opening: depth here is 2
	
	  \bigskip
	  \begingroup
	
	    \large
	  
	    \begin{quote}
	  
	    
	    \stanza[\smallbreak]
	\label{pv.2.262}\flagstanza{\tiny\textenglish{....2.262}}धियोर्युग‚प‚दुत्प‚त्तौ त‚त्त‚द्-विष‚य‚स‚म्भ‚वात् ।&सुख‚दुःख‚विदौ स्यातां स‚कृद‚र्थ‚स्य स‚म्भ‚वे ॥ २६२ ॥\&[\smallbreak]


	
	    \end{quote}
	  
	  \endgroup
	\textsuperscript{\textenglish{198/s}}

	  \pstart \leavevmode% starting standard par
	विष‚य‚सुख‚ग्राहिण्योर्द्व‚योरिन्द्रिय‚बुद्धिम‚नोबुद्ध्यात्मिक‚योः । त‚स्य सुख‚हेतो‚{\tiny $_{lb}$}‚र्दुःख‚हेतोश्च ‚{\color{DodgerBlue3}‚विष‚य‚स्य स‚म्भ‚वात् युग‚प‚दुत्प‚त्ता}‚व‚भिम‚तायां ‚{\color{DodgerBlue3}‚स‚कृद‚र्थ‚स्य} सुख‚दुःख‚हेतोः ‚{\tiny $_{lb}$}‚‚{\color{DodgerBlue3}‚स‚म्भ‚वे} स‚ति ‚{\color{DodgerBlue3}‚सुख‚दुःख‚विदौ स्यातां} । (२६२)
	\pend% ending standard par
      \label{div_pvv.2.263}
	  
	% new div opening: depth here is 2
	

	  \pstart \leavevmode% starting standard par
	स्यादेत‚त् (।)
	\pend% ending standard par
      
	  \bigskip
	  \begingroup
	
	    \large
	  
	    \begin{quote}
	  
	    
	    \stanza[\smallbreak]
	\label{pv.2.263}\flagstanza{\tiny\textenglish{....2.263}}स‚त्यान्त‚रेप्युपादाने ज्ञाने दुःखादिस‚म्भ‚वः ।&नोपादानं विरुद्ध‚स्य त‚च्चैक‚मिति चेन्म‚त‚म् ॥ २६३ ॥\&[\smallbreak]


	
	    \end{quote}
	  
	  \endgroup
	

	  \pstart \leavevmode% starting standard par
	\hphantom{.}न केव‚ल‚म‚र्थे स‚ति किन्त‚र्ह्या‚{\color{DodgerBlue3}‚न्त‚रे} स‚म‚न‚न्त‚र‚प्र‚त्य‚ये ‚{\color{DodgerBlue3}‚ज्ञान उपादाने स‚ति} \leavevmode\ledsidenote{\textenglish{39a/MA}} ‚{\color{DodgerBlue3}‚दुःखादिसंभ‚वः} । त‚च्चोपादानापेक्षं ज्ञानं ‚{\color{DodgerBlue3}‚विरुद्ध‚स्य} सुख‚{\tiny $_{7}$}‚दुःखादेर्न कार‚णं भ‚वितु‚{\tiny $_{lb}$}‚म‚र्ह‚ती‚{\color{DodgerBlue3}‚ति म‚तं चेत्} (। २६३)
	\pend% ending standard par
      \label{div_pvv.2.264}
	  
	% new div opening: depth here is 2
	

	  \pstart \leavevmode% starting standard par
	अत्राह (।)
	\pend% ending standard par
      
	  \bigskip
	  \begingroup
	
	    \large
	  
	    \begin{quote}
	  
	    
	    \stanza[\smallbreak]
	\label{pv.2.264}\flagstanza{\tiny\textenglish{....2.264}}त‚द‚ज्ञान‚स्य विज्ञानं केनोपादान‚कार‚णं ।&आधिप‚त्यं तु कुर्वीत त‚द्‏्विरुद्धेपि दृश्य‚ते ॥ २६४ ॥\&[\smallbreak]


	
	    \end{quote}
	  
	  \endgroup
	

	  \pstart \leavevmode% starting standard par
	\hphantom{.}‚{\color{DodgerBlue3}‚त‚द्विज्ञान‚म‚ज्ञान‚स्य} सुख‚दुःखादेरु‚{\color{DodgerBlue3}‚पादान‚कार‚णं} केन हेतुना स‚म्म‚तं । स‚मान‚{\tiny $_{lb}$}‚जातीयं कार‚ण‚मुपादानं नान्य‚त् । न च सुखादिज्ञानामिष्टं । ज्ञान‚म‚ज्ञान‚कार्ये आधि‚{\tiny $_{lb}$}‚प\edtext{}{\edlabel{pvv.198-1}\label{pvv.198-1}\lemma{प}\Bfootnote{सिद्धान्त्येवाह (।) आधिप‚त्य‚मानं स्यात् ।}}त्यं स‚ह‚कारित्वं कुर्व्वीत । ‚{\color{DodgerBlue3}‚त‚दा}‚धिप‚त्यं ‚{\color{DodgerBlue3}‚विरुद्धेपि} का\edtext{}{\edlabel{pvv.198-2}\label{pvv.198-2}\lemma{का}\Bfootnote{न च ताव‚तोपादान‚त्त्वात् ।}}र्ये दृश्य‚ते । (२६४)
	\pend% ending standard par
      \label{div_pvv.2.265}
	  
	% new div opening: depth here is 2
	
	  \bigskip
	  \begingroup
	
	    \large
	  
	    \begin{quote}
	  
	    
	    \stanza[\smallbreak]
	\label{pv.2.265}\flagstanza{\tiny\textenglish{....2.265}}अक्ष्णोर्य‚थैक आलोको न‚क्त‚ञ्च‚र‚त‚द‚न्य‚योः ।&रूप‚द‚र्श‚न‚वैगुण्यावैगुण्ये कुरुते स‚कृत् ॥ २६५ ॥\&[\smallbreak]


	
	    \end{quote}
	  
	  \endgroup
	

	  \pstart \leavevmode% starting standard par
	\hphantom{.}‚{\color{DodgerBlue3}‚य‚था एक आलोको न‚क्तंच}‚र‚स्य म‚नुष्यादेर‚क्ष्णो रूप‚द‚र्श‚न‚स्य वैगुण्यावैगुण्ये ‚{\tiny $_{lb}$}‚य‚थाक्र‚मं ‚{\color{DodgerBlue3}‚स‚कृत् कुरुते} । (२६५)
	\pend% ending standard par
      \label{div_pvv.2.266}
	  
	% new div opening: depth here is 2
	
	  \bigskip
	  \begingroup
	
	    \large
	  
	    \begin{quote}
	  
	    
	    \stanza[\smallbreak]
	\label{pv.2.266}\flagstanza{\tiny\textenglish{....2.266}}त‚स्मात् सुखाद‚योर्थानां स्व‚संक्रान्ताव‚भासिनाम् ।&वेद‚काः स्वात्म‚न‚श्चैषाम‚र्थेभ्यो ज‚न्म केव‚ल‚म् ॥ २६६ ॥\&[\smallbreak]


	
	    \end{quote}
	  
	  \endgroup
	

	  \pstart \leavevmode% starting standard par
	य‚स्माद\edtext{}{\edlabel{pvv.198-3}\label{pvv.198-3}\lemma{स्माद}\Bfootnote{म‚ह‚त्त‚त्त्वादिना ।}}न्येन सुखादीनां‚{\tiny $_{1}$}‚ वेद‚नं न घ‚ट‚ते । ‚{\color{DodgerBlue3}‚त‚स्मात्सुखाद‚यो} ज्ञानात्मानः ‚{\tiny $_{lb}$}‚स्व‚स्मिन् स्व‚रूपे प्र‚तिब‚न्ध‚द्वारेण ‚{\color{DodgerBlue3}‚संक्रान्ताव‚भास‚न}‚शीलाश्च ये‚{\color{DodgerBlue3}‚ऽर्थानां} तेषां ‚{\color{DodgerBlue3}‚वेद‚काः} । ‚{\tiny $_{lb}$}‚‚{\color{DodgerBlue3}‚आत्म‚न‚श्चा}‚प‚रोक्ष‚त्वाद्वेद‚काः । अर्थ‚स‚रूप‚स्यात्म‚नोऽप‚रोक्ष‚तैव अर्थ‚वेद‚नं स्व‚वेद‚न‚ञ्च । ‚{\tiny $_{lb}$}‚न त्व‚न्यः क‚श्चिद् ग्र‚ह‚ण‚प्र‚कारः । त‚त‚{\color{DodgerBlue3}‚श्चैषां} सुखादीनाम‚र्थ‚स‚रूपाणा‚{\color{DodgerBlue3}‚म‚र्थेभ्यो ज‚न्मैव ‚{\tiny $_{lb}$}‚केव‚लं} ग्र‚हीतृत्वं नाप‚रः क‚श्चिद् व्यापारः । (२६६)
	\pend% ending standard par
      \label{div_pvv.2.267}
	  
	% new div opening: depth here is 2
	\textsuperscript{\textenglish{199/s}}
	  \bigskip
	  \begingroup
	
	    \large
	  
	    \begin{quote}
	  
	    
	    \stanza[\smallbreak]
	\label{pv.2.267}\flagstanza{\tiny\textenglish{....2.267}}अर्थात्मा स्वात्म‚भूतो हि तेषां तैर‚नुभूय‚ते ।&तेनार्थानुभ‚व‚ख्यातिराल‚म्ब‚स्तु त‚दाभ‚ता ॥ २६७ ॥\&[\smallbreak]


	
	    \end{quote}
	  
	  \endgroup
	

	  \pstart \leavevmode% starting standard par
	\hphantom{.}अत एवा‚{\color{DodgerBlue3}‚र्थानुभ‚वः} क‚थं स्व‚वेद‚न‚म‚र्थ‚ज्ञान‚योर्भेदा‚{\tiny $_{2}$}‚त् । त‚द्वेद‚न‚योश्च भेद‚स्य ‚{\tiny $_{lb}$}‚न्याय‚प्राप्त‚त्वादिति ब्रुवाणः प्र‚तिक्षिप्तः । त‚था हि तेषां सुखादीना‚{\color{DodgerBlue3}‚म‚र्थात्मा} अर्थाकारः ‚{\tiny $_{lb}$}‚प्र‚तिबिम्ब‚संक्रान्त्या ‚{\color{DodgerBlue3}‚स्वात्म‚भूतः} स्व‚भाव‚भूत‚{\color{DodgerBlue3}‚स्तैः} सुखादिभि‚{\color{DodgerBlue3}‚र‚नुभूय‚ते}‚ऽप‚रोक्ष‚त्वात् । ‚{\tiny $_{lb}$}‚तेनोप‚चारेणार्थ‚स्य प‚र‚भूत‚स्य प्र‚तिबिम्ब‚हेतोर‚नुभ‚व‚स्य ख्यातिः प्र‚सिद्धिः । त‚त‚श्च ‚{\tiny $_{lb}$}‚ज्ञान‚स्याल‚म्बोऽर्थाल‚म्ब‚नं त‚दाभ‚ता अर्थाकार‚त्वं विचार्य‚माण‚म‚व‚शिष्य‚ते न तु पार‚{\tiny $_{lb}$}‚मार्थिक‚माल‚म्व्याल‚म्ब‚क‚त्वं नाम । (२६७)
	\pend% ending standard par
      \label{div_pvv.2.268}
	  
	% new div opening: depth here is 2
	

	  \begin{center}%% label @type='head'
	\textbf{ग. स्व‚संवेद‚ने सांख्य‚म‚त‚निरासः}
	\end{center}
	

	  \pstart \leavevmode% starting standard par
	इदा‚{\tiny $_{3}$}‚नीं सां ख्य म त‚मुत्थाप‚य‚न्नाह (।)
	\pend% ending standard par
      
	  \bigskip
	  \begingroup
	
	    \large
	  
	    \begin{quote}
	  
	    
	    \stanza[\smallbreak]
	\label{pv.2.268}\flagstanza{\tiny\textenglish{....2.268}}क‚श्चिद् ब‚हिःस्थितानेव सुखादीन‚प्र‚चेत‚नान् ।&ग्राह्यानाह न त‚स्यापि स‚कृद्युक्तो द्व‚य‚ग्र‚हः ॥ २६८ ॥\&[\smallbreak]


	
	    \end{quote}
	  
	  \endgroup
	

	  \pstart \leavevmode% starting standard par
	\hphantom{.}‚{\color{DodgerBlue3}‚क‚श्चिद् ब‚हिःस्थितानेव} न त्वात्म‚स‚म‚वायिनः । प्र‚धान‚प‚रिणाम‚ज‚त्वेन सुख‚दुःख‚{\tiny $_{lb}$}‚मोह‚स्व‚भाव‚त्वात् अर्थान‚{\color{DodgerBlue3}‚प्र‚चेत‚नात्} आत्म‚न एव चेत‚न‚त्वाद् ्। बुद्धिद‚र्प‚णे ज‚डा‚{\tiny $_{3}$}‚त्म‚नि ‚{\tiny $_{lb}$}‚स्व‚च्छेऽर्थ‚चेत‚न‚योः प्र‚तिबिम्ब‚संक्रान्तिद्वारेण च्छायाप‚त्त्या चेत‚न‚स्य ‚{\color{DodgerBlue3}‚ग्राह्या‚{\tiny $_{lb}$}‚नाह (।) त‚स्यापि} म‚ते ‚{\color{DodgerBlue3}‚द्व‚य‚स्य} सुख‚स्य नीलादेश्च ‚{\color{DodgerBlue3}‚स‚कृद् ग्र‚हो} निय‚मेन ‚{\color{DodgerBlue3}‚न युक्तः} । ‚{\tiny $_{lb}$}‚क‚दाचिद्रूप‚निर‚पेक्ष‚म‚पि रूप‚सुख‚म‚नुभूयेत । (२६८)
	\pend% ending standard par
      \label{div_pvv.2.269}
	  
	% new div opening: depth here is 2
	
	  \bigskip
	  \begingroup
	
	    \large
	  
	    \begin{quote}
	  
	    
	    \stanza[\smallbreak]
	\label{pv.2.269}\flagstanza{\tiny\textenglish{....2.269}}सुखाद्य‚भिन्न‚रूप‚त्वान्नोलादेश्चेत् स‚कृद् ग्र‚हः ।&भिन्नाव‚भासिनोर्ग्राह्यं चेत‚सोस्त‚द‚भेदि किम् ॥ २६९ ॥\&[\smallbreak]


	
	    \end{quote}
	  
	  \endgroup
	

	  \pstart \leavevmode% starting standard par
	\hphantom{.}‚{\color{DodgerBlue3}‚सुखादेर‚भिन्न‚{\tiny $_{4}$}‚रूप‚त्वान्नीलादेः स‚कृन्निय‚मेन ग्र‚ह‚श्चेत् । भिन्नाव‚भासिनो}‚{\tiny $_{lb}$}‚र्भिन्नाकार‚योः सुख‚नील‚{\color{DodgerBlue3}‚चेत‚सोर्ग्राह्यं त‚त् किम‚भेदि} इष्य‚ते । सुख‚दुःखादीनां नील‚{\tiny $_{lb}$}‚पीतादीनाञ्च भिन्न‚म‚नोग्राह्यानामेक‚त्व‚मेवं स्यात् । (२६९)
	\pend% ending standard par
      \label{div_pvv.2.270}
	  
	% new div opening: depth here is 2
	

	  \pstart \leavevmode% starting standard par
	किञ्च (।)
	\pend% ending standard par
      
	  \bigskip
	  \begingroup
	
	    \large
	  
	    \begin{quote}
	  
	    
	    \stanza[\smallbreak]
	\label{pv.2.270}\flagstanza{\tiny\textenglish{....2.270}}त‚स्याविशेषे बाह्य‚स्य भाव‚नातार‚त‚म्य‚तः ।&तार‚त‚म्य‚ञ्च बुद्धौ स्यान्न प्रीतिप‚रिताप‚योः ॥ २७० ॥\&[\smallbreak]


	
	    \end{quote}
	  
	  \endgroup
	

	  \pstart \leavevmode% starting standard par
	\hphantom{.}य‚दि नीलाद्येव सुखाद्यात्म‚कं त‚दा ‚{\color{DodgerBlue3}‚त‚स्य बाह्य‚स्य} सुख‚दुःखाद्यात्म‚त‚याऽ‚{\color{DodgerBlue3}‚विशेषे} विशेषाभावे रुच्य‚रुचिविष‚य‚त‚या ‚{\color{DodgerBlue3}‚भाव‚ना}‚यास्ता‚{\color{DodgerBlue3}‚र‚त‚म्य‚तः । तार‚त‚म्य‚ञ्च बुद्धौ ‚{\tiny $_{lb}$}‚प्रीतिप‚रिताप‚योर्न} स्यात् । न हि ग्राह्याविशेषे त‚ज्ज्ञानं विशि‚{\tiny $_{5}$}‚ष्य‚ते । (२७०)
	\pend% ending standard par
      \label{div_pvv.2.271}
	  
	% new div opening: depth here is 2
	
	  \bigskip
	  \begingroup
	
	    \large
	  
	    \begin{quote}
	  
	    
	    \stanza[\smallbreak]
	\label{pv.2.271}\flagstanza{\tiny\textenglish{....2.271}}सुखाद्यात्म‚त‚या बुद्धेर‚पि य‚द्य‚विरोधिता ।&स इदानीं क‚थं बाह्यः सुखाद्यात्मेति ग‚म्य‚ते ॥ २७१ ॥\&[\smallbreak]


	
	    \end{quote}
	  
	  \endgroup
	\textsuperscript{\textenglish{200/s}}

	  \pstart \leavevmode% starting standard par
	बुद्धेर‚पि प्र‚धान‚प‚रिणाम‚रूपाया भाव‚नातार‚त‚म्यात् सुखाद्यात्म\edtext{}{\edlabel{pvv.200-1}\label{pvv.200-1}\lemma{सुखाद्यात्म}\Bfootnote{नीलादि सुखादि च भिन्न‚मेव । नील‚संप्र‚ह‚र्ष‚णाकार‚योर्भेदात् ।}}ताविशेषाद् ‚{\tiny $_{lb}$}‚बाह्याविशेषेपि प्रीतिप‚रिताप‚तार‚त‚म्य‚स्याविरोधितेष्य‚ते । ‚{\color{DodgerBlue3}‚य‚दि इदानी}‚म‚न्याभ्युप‚ग‚मे ‚{\tiny $_{lb}$}‚‚{\color{DodgerBlue3}‚बाह्योऽर्थः} स सुखाद्यात्मेति क‚थं ग‚म्य‚ते । अनुभूय‚मानं सुख‚म‚न्य‚त्रासंभ‚व‚द् बाह्ये ‚{\tiny $_{lb}$}‚व्य‚व‚स्थाप‚नीयं । (२७१)
	\pend% ending standard par
      \label{div_pvv.2.272}
	  
	% new div opening: depth here is 2
	
	  \bigskip
	  \begingroup
	
	    \large
	  
	    \begin{quote}
	  
	    
	    \stanza[\smallbreak]
	\label{pv.2.272}\flagstanza{\tiny\textenglish{....2.272}}अग्राह्य‚ग्राह‚क‚त्वाच्चेद् भिन्न‚जातीय‚योः पुमान् ।&अग्राह‚कः स्यात् स‚र्व्व‚स्य त‚तो हीयेत भोक्तृता ॥ २७२ ॥\&[\smallbreak]


	
	    \end{quote}
	  
	  \endgroup
	

	  \pstart \leavevmode% starting standard par
	य‚दा तु बुद्धिर‚पि सुखात्मिका त‚दा किं बाह्य‚सुख‚क‚ल्प‚न‚या । बाह्य‚म‚ह\edtext{}{\edlabel{pvv.200-2}\label{pvv.200-2}\lemma{ह}\Bfootnote{प्र‚धानान्म‚हान्म‚ह‚तोहंकार‚स्त‚तः प‚ञ्च श‚ब्दादीनि त‚तः प‚ञ्चाकाशादीनि ‚{\tiny $_{lb}$}‚प‚ञ्च क‚र्म्मेन्द्रियाणि वाक्पाणिपाद‚पायूप‚स्थानि प‚ञ्च बुद्धीन्द्रियाणि म‚न‚श्च ... ‚{\tiny $_{lb}$}‚सांख्य‚स्य । ग्राह्य‚ग्राह‚क‚त्वाभावः प्र‚स‚ज्येद(? ना) स्ति स न स्यात् ।}}तोर‚{\tiny $_{lb}$}‚\leavevmode\ledsidenote{\textenglish{39b/MA}} सुख‚रूप‚त्वाच्च ‚{\color{DodgerBlue3}‚भिन्न‚जातीय‚योर्ग्राह्य‚ग्राह‚क‚त्वा}‚भावात् सु‚{\tiny $_{6}$}‚खाद्यात्म‚ता बाह्य‚स्य ‚{\tiny $_{lb}$}‚ग‚म्य‚त इति चेत्\edtext{}{\edlabel{pvv.200-3}\label{pvv.200-3}\lemma{चेत्}\Bfootnote{विष‚य‚बुद्ध‚योः ।}} । ‚{\color{DodgerBlue3}‚पुमान्} चेत‚नो निर्गुण‚त्वाद‚सुखादिरूपो ग्राह्य‚स्याचेत‚न‚स्य सुखा‚{\tiny $_{lb}$}‚द्यात्म‚क‚स्य स‚र्व्व‚स्य भिन्न‚जातीय‚त्वात् ‚{\color{DodgerBlue3}‚अग्राह‚कः} स्यात् । ‚{\color{DodgerBlue3}‚त‚तो}‚ऽग्राह\edtext{}{\edlabel{pvv.200-4}\label{pvv.200-4}\lemma{ऽग्राह}\Bfootnote{न बाह्यं ज्ञाने संक्राम‚ति स्व‚रूपेण वेद्य‚ते वाऽग्न्यादाह‚प्र‚स‚ङ्गात् ।}}क‚त्वात् ‚{\color{DodgerBlue3}‚भोक्तृ‚{\tiny $_{lb}$}‚तास्य हीय‚ते} । अनुभ‚विता हि भोक्तोच्य‚तेऽनुभ‚वाभावे क‚थं भोक्ता । (२७२)
	\pend% ending standard par
      \label{div_pvv.2.273}
	  
	% new div opening: depth here is 2
	
	  \bigskip
	  \begingroup
	
	    \large
	  
	    \begin{quote}
	  
	    
	    \stanza[\smallbreak]
	\label{pv.2.273}\flagstanza{\tiny\textenglish{....2.273}}कार्य‚कार‚ण‚तानेन प्र‚त्युक्ताऽकार्य‚कार‚णे ।&ग्राह्य‚ग्राह‚क‚ताभावाद् भावेन्य‚त्रापि सा भ‚वेत् ॥ २७३ ॥\&[\smallbreak]


	
	    \end{quote}
	  
	  \endgroup
	

	  \pstart \leavevmode% starting standard par
	बुद्धिसुख‚योर्ग्राह्य‚ग्राह‚काभावात् । नाकार‚णं विष‚य इति कार्य‚कार‚ण‚ता । ‚{\tiny $_{lb}$}‚त‚तो ग्राह्य\edtext{}{\edlabel{pvv.200-5}\label{pvv.200-5}\lemma{ग्राह्य}\Bfootnote{बुद्ध्या क‚टो विष‚याकारः पुमांस‚म‚ध्यारोह‚ति त‚तोस्य भोक्तृत्वं ।}}स्य सुखाद्यात्म‚क‚त्वात् त‚त्कार्य‚या बुद्ध्यापि सुखाद्यात्मिक‚या‚{\tiny $_{7}$}‚ ‚{\tiny $_{lb}$}‚भ‚वित‚व्य‚मिति प‚रैरुक्ता ‚{\color{DodgerBlue3}‚कार्य‚कार‚ण‚ता} सानेनातिप्र‚स‚ङ्गेन ‚{\color{DodgerBlue3}‚प्र‚त्युक्ता} । त‚था हि ‚{\tiny $_{lb}$}‚भोक्तापि ‚{\color{DodgerBlue3}‚ग्राह‚क} इति स च कार्य‚सुखाद्यात्म‚कः क‚थं ‚{\color{DodgerBlue3}‚स्यात्} । अपि चाकार्य‚कार‚णे ‚{\tiny $_{lb}$}‚बुद्धिसुखे । ग्राह्य‚ग्राह‚क‚तायाः कार्य‚कार‚ण‚तानिब‚न्ध‚नाया अभावात् । ‚{\color{DodgerBlue3}‚ग्राह्य‚ग्राह‚क‚{\tiny $_{lb}$}‚त्व}‚भावात् कार्य‚कार‚ण‚भाव‚स्य च ‚{\color{DodgerBlue3}‚भावे}‚ऽभ्युप‚ग‚म्य‚माने‚{\color{DodgerBlue3}‚ऽन्य}‚त्रात्म‚विष‚य\edtext{}{\edlabel{pvv.200-6}\label{pvv.200-6}\lemma{य}\Bfootnote{विजातीय‚योर‚पि ।}}योर‚पि ‚{\color{DodgerBlue3}‚सा} कार्य‚कार‚ण‚ता स्यात् । ग्राह्य‚ग्राह‚क‚ताया भावात् । (२७३)
	\pend% ending standard par
      \label{div_pvv.2.274}
	  
	% new div opening: depth here is 2
	\textsuperscript{\textenglish{201/s}}

	  \pstart \leavevmode% starting standard par
	य‚स्मात्सुखादेर्ब्बाह्य‚स्य भाव‚नात‚स्ता‚{\tiny $_{1}$}‚र‚त‚म्यानुयोग‚योगः ।
	\pend% ending standard par
      
	  \bigskip
	  \begingroup
	
	    \large
	  
	    \begin{quote}
	  
	    
	    \stanza[\smallbreak]
	\label{pv.2.274a}\flagstanza{\tiny\textenglish{...2.274a}}त‚स्मात्त आन्त‚रा एव,\&[\smallbreak]


	
	    \end{quote}
	  
	  \endgroup
	

	  \pstart \leavevmode% starting standard par
	\hphantom{.}‚{\color{DodgerBlue3}‚त‚स्माद}‚सुखाद‚य ‚{\color{DodgerBlue3}‚आन्त‚रा} ज्ञान‚स्व‚भावा अबाह्या अभ्युप‚ग‚न्त‚व्याः ।
	\pend% ending standard par
      

	  \pstart \leavevmode% starting standard par
	किञ्च (।)
	\pend% ending standard par
      
	  \bigskip
	  \begingroup
	
	    \large
	  
	    \begin{quote}
	  
	    
	    \stanza[\smallbreak]
	\label{pv.2.274b}\flagstanza{\tiny\textenglish{...2.274b}}स‚म्वेद्य‚त्वाच्च चेत‚नाः ।&संवेद‚नं न य‚द् रूपं न हि त‚त्त‚स्य वेद‚न‚म् ॥ २७४ ॥\&[\smallbreak]


	
	    \end{quote}
	  
	  \endgroup
	

	  \pstart \leavevmode% starting standard par
	\hphantom{.}‚{\color{DodgerBlue3}‚स‚म्वेद्य‚त्वाच्च हेतोस्ते चेत‚नाः} । त‚था हि स‚म्वेद‚नं त‚द्रूपं य‚द्विष‚य‚स्व‚रूपं ‚{\tiny $_{lb}$}‚न भ‚व‚ति ‚{\color{DodgerBlue3}‚त‚त्संवेद‚नं त‚स्य} विष‚य‚स्य ‚{\color{DodgerBlue3}‚वेद‚नं} य‚स्मा‚{\color{DodgerBlue3}‚न्न} भ‚व‚ति । न ह्य‚विष‚य‚{\tiny $_{lb}$}‚स्व‚रूपं प्र‚तिविष‚यं भेद‚व्य‚व‚स्थां क‚र्त्तुम‚र्ह‚तीति साकारं ग्राह‚कं । त‚था च साकारं ‚{\tiny $_{lb}$}‚ज्ञान‚मेव सुख‚म‚स्तु त‚दुप‚धाय‚कं तु बाह्य‚म‚युक्तं । भाव‚नाविशेषेण सुखादिवेद‚न‚स्या‚{\tiny $_{lb}$}‚वि‚{\tiny $_{2}$}‚शेष‚द‚र्श‚नात् । बाह्याधीन‚त्वे च त‚द‚योगात् । (२७४)
	\pend% ending standard par
      \label{div_pvv.2.275}
	  
	% new div opening: depth here is 2
	

	  \pstart \leavevmode% starting standard par
	स्यादेत‚द् (।)
	\pend% ending standard par
      
	  \bigskip
	  \begingroup
	
	    \large
	  
	    \begin{quote}
	  
	    
	    \stanza[\smallbreak]
	\label{pv.2.275}\flagstanza{\tiny\textenglish{....2.275}}अत‚त्स्व‚भावोऽनुभ‚वो बौद्धांस्तान् स‚न्न‚वैति चेत् ।&मुक्त्‏वाध्य‚क्ष‚स्मृताकारां संवित्तिं बुद्धिर‚त्र का ॥ २७५ ॥\&[\smallbreak]


	
	    \end{quote}
	  
	  \endgroup
	

	  \pstart \leavevmode% starting standard par
	\hphantom{.}भाव‚नात‚स्तार‚त‚म्य‚योगिन‚स्तान् बुद्ध्यात्म‚क‚सुखादीन् ‚{\color{DodgerBlue3}‚अत‚त्स्व‚भावो}‚ऽसुख्याद्या‚{\tiny $_{lb}$}‚कारोऽ‚{\color{DodgerBlue3}‚नुभ‚वः स‚न्न‚वैति} प्र‚त्येतीति‚{\tiny $_{3}$}‚ ‚{\color{DodgerBlue3}‚चेत् ।\edtext{\textsuperscript{*}}{\edlabel{pvv.201-1}\label{pvv.201-1}\lemma{*}\Bfootnote{स्व‚प्र‚क्रियामात्र‚दीप‚न‚मेत‚त् ।}} संवित्तिम‚ध्य‚क्षा}‚कारां ह‚र्ष‚विषादाकारां ‚{\tiny $_{lb}$}‚‚{\color{DodgerBlue3}‚स्मृताकाराम}‚तीत‚सुखादिक‚ल्प‚नारूपां सौम‚न‚स्य‚ल‚क्ष‚णां मुक्त्वा अत्र संवेद‚नाव‚स‚रे ‚{\color{DodgerBlue3}‚का} प‚रा बुद्धिर‚नुभूय‚ते । य‚स्याः सुखाद्यात्म‚क‚त्व‚म‚ध्य\edtext{}{\edlabel{pvv.201-2}\label{pvv.201-2}\lemma{ध्य}\Bfootnote{बुद्धिः ।}}व‚सायात्म‚क‚त्व‚ञ्चेष्य‚ते । (२७५)
	\pend% ending standard par
      \label{div_pvv.2.276}
	  
	% new div opening: depth here is 2
	
	  \bigskip
	  \begingroup
	
	    \large
	  
	    \begin{quote}
	  
	    
	    \stanza[\smallbreak]
	\label{pv.2.276}\flagstanza{\tiny\textenglish{....2.276}}तांस्तान‚र्थानुपादाय सुख‚दुःखादिवेद‚न‚म् ।&एक‚माविर्भ‚व‚द् दृष्टं न दृष्टं त्व‚न्य‚द‚न्त‚रा ॥ २७६ ॥\&[\smallbreak]


	
	    \end{quote}
	  
	  \endgroup
	

	  \pstart \leavevmode% starting standard par
	\hphantom{.}‚{\color{DodgerBlue3}‚ताँस्तान‚र्थानि}‚ष्टा‚{\tiny $_{4}$}‚न‚निष्टा‚{\color{DodgerBlue3}‚नुपादाया}‚श्रित्यैकं ‚{\color{DodgerBlue3}‚सुख‚दुःखादिवेद‚न‚माविर्भ‚व‚द् दृष्टं} अन्य‚द् बुद्धिरूप‚{\color{DodgerBlue3}‚म‚न्त‚रा} विष‚य‚संवेद‚न‚योर‚न्त‚राले ‚{\color{DodgerBlue3}‚न दृष्टं} । (२७६)
	\pend% ending standard par
      \label{div_pvv.2.277}
	  
	% new div opening: depth here is 2
	
	  \bigskip
	  \begingroup
	
	    \large
	  
	    \begin{quote}
	  
	    
	    \stanza[\smallbreak]
	\label{pv.2.277}\flagstanza{\tiny\textenglish{....2.277}}संस‚र्गाद‚विभाग‚श्चेद‚योगोल‚क‚व‚ह्निव‚त् ।&भेदाभेद‚व्य‚व‚स्थैव‚मुच्छिन्ना स‚र्व‚व‚स्तुषु ॥ २७७ ॥\&[\smallbreak]


	
	    \end{quote}
	  
	  \endgroup
	

	  \pstart \leavevmode% starting standard par
	\hphantom{.}बुद्धिचेत‚न‚योः ‚{\color{DodgerBlue3}‚संस‚र्ग्गाद‚योगो\edtext{}{\edlabel{pvv.201-3}\label{pvv.201-3}\lemma{योगो}\Bfootnote{अयःपिण्ड‚म‚ग्निदीप्तं ।}}ल‚क‚व‚ह्न्यो}‚रिवाविभागो भेदानुप‚ल‚ब्धिरिति ‚{\tiny $_{lb}$}‚‚{\color{DodgerBlue3}‚चेत्} । एवं भिन्नाकारानुभ‚वेप्य‚भेद‚क‚ल्प‚नायां\edtext{}{\edlabel{pvv.201-4}\label{pvv.201-4}\lemma{नायां}\Bfootnote{त‚था एकाकारेप्य‚योगोल‚क‚व‚त्संस‚र्ग‚क‚ल्प‚नायां ।}} ‚{\color{DodgerBlue3}‚स‚र्व्व‚व‚स्तुषु भेदाभेद‚व्य‚व‚स्थोच्छिन्ना} स्यात् । (२७७)
	\pend% ending standard par
      \label{div_pvv.2.278}
	  
	% new div opening: depth here is 2
	\textsuperscript{\textenglish{202/s}}

	  \pstart \leavevmode% starting standard par
	त‚था हि (।)
	\pend% ending standard par
      
	  \bigskip
	  \begingroup
	
	    \large
	  
	    \begin{quote}
	  
	    
	    \stanza[\smallbreak]
	\label{pv.2.278}\flagstanza{\tiny\textenglish{....2.278}}अभिन्न‚वेद‚न‚स्यैक्यं य‚न्नैवं त‚द् विभेद‚व‚त् ।&सिध्येद‚साध‚न‚त्वेस्य न सिद्धं भेद‚साध‚न‚म् ॥ २७८ ॥\&[\smallbreak]


	
	    \end{quote}
	  
	  \endgroup
	

	  \pstart \leavevmode% starting standard par
	\hphantom{.}‚{\color{DodgerBlue3}‚अभिन्न}‚मेकाकारं ‚{\color{DodgerBlue3}‚वेद‚नं} य‚स्य त‚स्यैक्यं । ‚{\color{DodgerBlue3}‚य‚न्नैवं} भिन्न‚वेद‚नं ‚{\color{DodgerBlue3}‚त‚द्विभेद\edtext{}{\edlabel{pvv.202-1}\label{pvv.202-1}\lemma{द्विभेद}\Bfootnote{विभेदोस्यास्तीति कृत्वा ।}} व‚त् ‚{\tiny $_{lb}$}‚सिध्येत्(।)} अस्यैकाकार‚ज्ञा‚{\tiny $_{5}$}‚न‚स्याभेद‚म्प्र‚{\color{DodgerBlue3}‚त्य‚साध‚न‚त्वे} त‚द्विप‚रीत‚भिन्नाकार‚वेद‚नं ‚{\color{DodgerBlue3}‚भेद‚{\tiny $_{lb}$}‚साध‚न‚म‚सिद्धं} ॥ (२७८)
	\pend% ending standard par
      \label{div_pvv.2.279}
	  
	% new div opening: depth here is 2
	
	  \bigskip
	  \begingroup
	
	    \large
	  
	    \begin{quote}
	  
	    
	    \stanza[\smallbreak]
	\label{pv.2.279}\flagstanza{\tiny\textenglish{....2.279}}भिन्नाभः सित‚दुःखादिर‚भिन्नो बुद्धिवेद‚ने ।&अभिन्नाभे विभिन्ने चेत् भेदाभेदौ किमाश्र‚यौ ॥ २७९ ॥\&[\smallbreak]


	
	    \end{quote}
	  
	  \endgroup
	

	  \pstart \leavevmode% starting standard par
	सां\edtext{}{\edlabel{pvv.202-2}\label{pvv.202-2}\lemma{सां}\Bfootnote{भेदाभेद‚निब‚न्ध‚नाभाव‚माह ।}} ख्य स्य तु सित‚दुःखादिर्भिन्नाकारोऽभिन्न इष्टः । ‚{\color{DodgerBlue3}‚बुद्धिवेद‚ने त्व‚भिन्ना‚{\tiny $_{lb}$}‚भे विभिन्ने इष्टे चेत् । भेदाभेदौ किमाश्र‚यौ} किं निमित्तौ ते व्य‚व‚स्थाप‚नीयौ । ‚{\tiny $_{lb}$}‚(२७९)
	\pend% ending standard par
      \label{div_pvv.2.280}
	  
	% new div opening: depth here is 2
	

	  \pstart \leavevmode% starting standard par
	न‚नु वे\edtext{}{\edlabel{pvv.202-3}\label{pvv.202-3}\lemma{वे}\Bfootnote{वैभा ष्या वेद‚नादिसंप्र‚युक्त‚विप्र‚युक्त‚वादिनः प्र‚त्याहुः ।}}द‚नाचेत‚नासंज्ञादीनां चैत्तानां म‚हाभूमिकादीनां स‚कृदुत्प‚न्नानां भेदः ‚{\tiny $_{lb}$}‚प‚र‚स्प‚रं\edtext{}{\edlabel{pvv.202-4}\label{pvv.202-4}\lemma{रं}\Bfootnote{एकैक‚स्योप‚ल‚ब्धेः ।}} प्र‚तीय‚ते । अथ चास्ति । त‚द्व‚द्ब्ुद्धिसुख‚योर‚पि स्यादित्याह ।\edtext{\textsuperscript{*}}{\edlabel{pvv.202-5}\label{pvv.202-5}\lemma{*}\Bfootnote{नैष दोषः ।}}
	\pend% ending standard par
      
	  \bigskip
	  \begingroup
	
	    \large
	  
	    \begin{quote}
	  
	    
	    \stanza[\smallbreak]
	\label{pv.2.280}\flagstanza{\tiny\textenglish{....2.280}}तिर‚स्कृतानां प‚टुनाप्येक‚दाऽभेद‚द‚र्श‚नात् ।&प्र‚वाहे वित्तिभेदानां सिद्धा भेद‚व्य‚व‚स्थितिः ॥ २८० ॥\&[\smallbreak]


	
	    \end{quote}
	  
	  \endgroup
	

	  \pstart \leavevmode% starting standard par
	\hphantom{.}‚{\color{DodgerBlue3}‚प‚टुना} सुख‚दुःखा‚{\tiny $_{6}$}‚दीनां वेद‚नास्क‚न्ध‚संगृहीतेन चैत्तेन ‚{\color{DodgerBlue3}‚तिर‚स्कृतानाम}‚भिभूतानां ‚{\tiny $_{lb}$}‚संज्ञादी‚{\color{DodgerBlue3}‚नामे}‚क‚दाऽ‚{\color{DodgerBlue3}‚भेद‚द‚र्श‚नात् । प्र‚वाहे} चित्त\edtext{}{\edlabel{pvv.202-6}\label{pvv.202-6}\lemma{चित्त}\Bfootnote{य‚दा क‚म‚नीये तृष्य‚ति अन्य‚त्र द्वेष्टि र‚ज्य‚ते विर‚ज्य‚ते ।}} स‚न्ताने सुखाद्य‚भाव‚काले स्व\edtext{}{\edlabel{pvv.202-7}\label{pvv.202-7}\lemma{स्व}\Bfootnote{क‚दाचित्तिर‚स्कारेपि चित्ताभिसंस्कार‚काले चेत‚ना वेद्य‚त एव य‚तः ।}}रू‚{\tiny $_{lb}$}‚पेणोप‚ल‚क्षितानां ‚{\color{DodgerBlue3}‚भेद‚व्य‚व‚स्थितिः सिद्धा} । र‚वेरुद‚ये तार‚का अनुप‚ल‚क्षिता अपि ‚{\tiny $_{lb}$}‚निशायामुप‚ल‚क्ष्य‚माणा भेदेन व्य‚व‚स्थाप्य‚न्त एव ।\edtext{\textsuperscript{*}}{\edlabel{pvv.202-8}\label{pvv.202-8}\lemma{*}\Bfootnote{नैवं प्र‚वाहेपि बुद्धिसंवेद‚न‚योर्भेद‚वेद‚न‚म‚स्ति ।}}त‚स्माज्‏ज्ञानात्मानः स्व‚वेद‚नाश्च ‚{\tiny $_{lb}$}‚सुखाद‚य इत्याख्यातं स्व‚वेद‚नं ॥ X X ॥ (२८०)
	\pend% ending standard par
      \label{div_pvv.2.281}
	  
	% new div opening: depth here is 2
	

	  \begin{center}%% label @type='head'
	\textbf{(४) योगिज्ञान‚प्र‚त्य‚क्ष‚म्}
	\end{center}
	

	  \pstart \leavevmode% starting standard par
	योगिज्ञान‚माख्यातुमाह (।)
	\pend% ending standard par
      
	  \bigskip
	  \begingroup
	
	    \large
	  
	    \begin{quote}
	  
	    
	    \stanza[\smallbreak]
	\label{pv.2.281}\flagstanza{\tiny\textenglish{....2.281}}प्रागुक्तं योगिनां ज्ञानं तेषां त‚द् भाव‚नाम‚य‚म् ।&विधूत‚क‚ल्प‚नाजालं स्प‚ष्ट‚मेवाव‚भास‚ते ॥ २८१ ॥\&[\smallbreak]


	
	    \end{quote}
	  
	  \endgroup
	\textsuperscript{\textenglish{203/s}}

	  \pstart \leavevmode% starting standard par
	\hphantom{.}‚{\color{DodgerBlue3}‚प्राक्} प्र‚थ‚म‚प‚रिच्छेदे\edtext{}{\edlabel{pvv.203-1}\label{pvv.203-1}\lemma{रिच्छेदे}\Bfootnote{योगिनाम‚प्याग‚म‚विक‚ल्पाव्य‚व‚कीर्ण‚म‚र्थ‚मात्र‚द‚र्श‚नं प्र‚त्य‚क्ष‚मिति व्याच‚ष्टे ।}} ‚{\tiny $_{7}$}‚ ‚{\color{DodgerBlue3}‚योगिनां ज्ञानं} स‚त्त्य‚विष‚य‚मुक्तं । ‚{\color{DodgerBlue3}‚तेषां} योगिनां ‚{\color{DodgerBlue3}‚भाव‚ना‚{\tiny $_{lb}$}‚म‚यं} भाव‚नाहेतुनिष्प‚त्तिकं ‚{\color{DodgerBlue3}‚त‚त्} ज्ञानं स‚त्त्य‚स्व‚रूप‚विष‚य‚त्वेन ‚{\color{DodgerBlue3}‚विधूत‚क‚ल्प‚नाजाल‚म-} विक‚ल्प‚त्वाच्च ‚{\color{DodgerBlue3}‚स्प‚ष्टं} विश‚द‚ज्ञेयाकार‚{\color{DodgerBlue3}‚मेवाव‚भास‚ते} । (२८१)
	\pend% ending standard par
      \label{div_pvv.2.282}
	  
	% new div opening: depth here is 2
	

	  \pstart \leavevmode% starting standard par
	भाव‚नाभ‚वं क‚थं स्प‚ष्ट‚मित्याह (।)
	\pend% ending standard par
      
	  \bigskip
	  \begingroup
	
	    \large
	  
	    \begin{quote}
	  
	    
	    \stanza[\smallbreak]
	\label{pv.2.282}\flagstanza{\tiny\textenglish{....2.282}}काम‚शोक‚भ‚योन्माद‚चौर‚स्व‚प्राद्युप‚प्लुताः ।&अभूतान‚पि प‚श्य‚न्ति पुर‚तोव‚स्थितानिव ॥ २८२ ॥\&[\smallbreak]


	
	    \end{quote}
	  
	  \endgroup
	

	  \pstart \leavevmode% starting standard par
	\hphantom{.}‚{\color{DodgerBlue3}‚काम‚श्च} शोक‚श्च भ‚य‚ञ्च तैरुन्मादाश्चौर‚स्व‚प्नाद‚य‚श्चेति ‚{\color{DodgerBlue3}‚काम‚शोक‚भ‚योन्माद‚{\tiny $_{lb}$}‚चौर‚स्व‚प्नादिभिरुप‚प्लुता} भ्रान्तास्ते‚{\color{DodgerBlue3}‚ऽभूतान‚प्य}‚र्थान् भाव‚ना\edtext{}{\edlabel{pvv.203-2}\label{pvv.203-2}\lemma{ना}\Bfootnote{य‚द् भाव्य‚ते त‚त् स्फुटं स्यादित्य‚स्य दृष्टान्तोयं श्लोकः ।}}व‚शात् ‚{\color{DodgerBlue3}‚पुर‚तोऽव‚स्थिता-\leavevmode\ledsidenote{\textenglish{40a/MA}} ‚{\tiny $_{lb}$}‚निव प‚{\tiny $_{8}$}‚श्य\edtext{}{\edlabel{pvv.203-3}\label{pvv.203-3}\lemma{श्य}\Bfootnote{अनुमेयं ।}}न्ति} । य‚स्मात्त‚द‚नुरूपां प्र‚वृत्तिं चेष्ट‚न्ते ॥ (२८२)
	\pend% ending standard par
      \label{div_pvv.2.283}
	  
	% new div opening: depth here is 2
	

	  \pstart \leavevmode% starting standard par
	भ‚व‚तु भाव‚नाजं स्प‚ष्ट‚म‚विक‚ल्पं तु क‚थ‚मित्याह (।)
	\pend% ending standard par
      
	  \bigskip
	  \begingroup
	
	    \large
	  
	    \begin{quote}
	  
	    
	    \stanza[\smallbreak]
	\label{pv.2.283a}\flagstanza{\tiny\textenglish{...2.283a}}न विक‚ल्पानुब‚द्ध‚स्यास्ति स्फुटार्थाव‚भासिता ॥\&[\smallbreak]


	
	    \end{quote}
	  
	  \endgroup
	

	  \pstart \leavevmode% starting standard par
	\hphantom{.}‚{\color{DodgerBlue3}‚न विक‚ल्पेनानुब‚द्ध‚स्य} संस्तुत‚स्य ज्ञान‚स्य ‚{\color{DodgerBlue3}‚स्फुटार्थाव‚भासितास्ति} ॥
	\pend% ending standard par
      

	  \pstart \leavevmode% starting standard par
	न‚नु विप्ल‚व‚व‚शात् विक‚ल्प‚क‚म‚पि स्व‚प्ने स्प‚ष्टाभं ज्ञानं भ‚व‚तीत्याह (।)
	\pend% ending standard par
      
	  \bigskip
	  \begingroup
	
	    \large
	  
	    \begin{quote}
	  
	    
	    \stanza[\smallbreak]
	\label{pv.2.283b}\flagstanza{\tiny\textenglish{...2.283b}}स्व‚प्नेपि स्म‚र्य‚ते स्मार्त्तं न च त‚त्तादृग‚र्थ‚व‚त् ॥ २८३ ॥\&[\smallbreak]


	
	    \end{quote}
	  
	  \endgroup
	

	  \pstart \leavevmode% starting standard par
	\hphantom{.}‚{\color{DodgerBlue3}‚स्व‚प्नेपि स्मार्त्तं} स्म‚र‚णं किञ्चिदुत्प‚द्य‚ते । ‚{\color{DodgerBlue3}‚न च} त‚त्प्र‚बोधाव‚स्थायां ‚{\color{DodgerBlue3}‚तादृग‚र्थ‚{\tiny $_{lb}$}‚व‚द्या}‚दृशो निर्व्विक‚ल्पेनानुभूतोऽर्थ‚स्तादृशार्थेन युक्तं स्म‚र्य‚ते । किन्त‚र्हि अस्प‚ष्टार्थ‚{\tiny $_{lb}$}‚मेव स्व‚प्न‚स्म‚र‚णं स्म‚र्य‚ते । (२८३)
	\pend% ending standard par
      \label{div_pvv.2.284}
	  
	% new div opening: depth here is 2
	

	  \pstart \leavevmode% starting standard par
	न‚न्व‚भूतार्थ‚भाव‚नाब‚{\tiny $_{1}$}‚ल‚जं भ‚व‚तापि स्प‚ष्टाभ‚म‚विक‚ल्प‚ञ्च सिद्धान्ते नेष्य‚ते ‚{\tiny $_{lb}$}‚इति तेन स‚ह विरोध इत्या\edtext{}{\edlabel{pvv.203-4}\label{pvv.203-4}\lemma{इत्या}\Bfootnote{द्विधा स्व‚प्न‚ज्ञानं स्प‚ष्ट‚मेक‚म‚न्य‚द‚तीत‚स्व‚प्नाकारं ।}} ह (।)
	\pend% ending standard par
      
	  \bigskip
	  \begingroup
	
	    \large
	  
	    \begin{quote}
	  
	    
	    \stanza[\smallbreak]
	\label{pv.2.284}\flagstanza{\tiny\textenglish{....2.284}}अशुभा पृथिवी-कृत्स्नाद्य‚भूत‚म‚पि व‚र्ण्य‚ते ।&स्प‚ष्टाभं निर्विक‚ल्प‚ञ्च भाव‚नाब‚ल‚निर्मित‚म् ॥ २८४ ॥\&[\smallbreak]


	
	    \end{quote}
	  
	  \endgroup
	

	  \pstart \leavevmode% starting standard par
	\hphantom{.}‚{\color{DodgerBlue3}‚अशुभा} विनील‚क‚विपूय‚कास्थिसंक‚लादिका ‚{\color{DodgerBlue3}‚पृथ्वीकृत्स्नादि} भूम‚य‚त्वादि ‚{\tiny $_{lb}$}‚‚{\color{DodgerBlue3}‚अभूत}‚म‚स‚त्य‚म‚पि ‚{\color{DodgerBlue3}‚भाव‚नाब‚लेन निर्मितं स्प‚ष्टाभं निर्व्विक‚ल्प‚क‚ञ्चा}‚स्मा‚{\color{DodgerBlue3}‚भिर्व्व‚र्ण्ण्य‚त} इति नास्ति सिद्धान्त‚विरोधः । (२८४)
	\pend% ending standard par
      \label{div_pvv.2.285}
	  
	% new div opening: depth here is 2
	\textsuperscript{\textenglish{204/s}}
	  \bigskip
	  \begingroup
	
	    \large
	  
	    \begin{quote}
	  
	    
	    \stanza[\smallbreak]
	\label{pv.2.285}\flagstanza{\tiny\textenglish{....2.285}}त‚स्माद् भूत‚म‚भूतं वा य‚द् य‚देवाभिभाव्य‚ते ।&भाव‚नाप‚रिनिष्प‚त्तौ त‚त् स्फुटाक‚ल्प‚धीफ‚लं ॥ २८५ ॥\&[\smallbreak]


	
	    \end{quote}
	  
	  \endgroup
	

	  \pstart \leavevmode% starting standard par
	\hphantom{.}य‚तो भाव‚नाया भाव्य‚स्प‚ष्ट‚तायामाधिप‚त्यं । ‚{\color{DodgerBlue3}‚त‚स्माद् भूत}‚मार्य‚स‚त्यादि ‚{\color{DodgerBlue3}‚अभू‚{\tiny $_{lb}$}‚त‚म}‚शुभादि ‚{\color{DodgerBlue3}‚य‚द्य‚देवा}‚त्य‚न्तं ‚{\color{DodgerBlue3}‚भाव्य‚ते} । त‚द् भाव्य‚मानं ‚{\color{DodgerBlue3}‚भाव‚नायाः} साद‚र‚निर‚न्त‚र‚दीर्घ‚{\tiny $_{lb}$}‚काल‚प्र‚व‚र्त्तितायाः ‚{\color{DodgerBlue3}‚प‚रिनिष्प‚त्तौ‚{\tiny $_{2}$}‚ स्फुटा क‚ल्प‚धीः} सा फ‚लं य‚स्य त‚त्त‚था । ‚{\tiny $_{lb}$}‚(२८५)
	\pend% ending standard par
      \label{div_pvv.2.286}
	  
	% new div opening: depth here is 2
	
	  \bigskip
	  \begingroup
	
	    \large
	  
	    \begin{quote}
	  
	    
	    \stanza[\smallbreak]
	\label{pv.2.286}\flagstanza{\tiny\textenglish{....2.286}}त‚त्र प्र‚माणां स‚म्वादि य‚त्; प्राङ् निर्णीत‚व‚स्तुव‚त् ।&त‚द् भाव‚नाजं प्र‚त्य‚क्ष‚मिष्टं शेषा उप‚प्ल‚वाः ॥ २८६ ॥\&[\smallbreak]


	
	    \end{quote}
	  
	  \endgroup
	

	  \pstart \leavevmode% starting standard par
	\hphantom{.}‚{\color{DodgerBlue3}‚त‚त्र} भाव‚नाब‚ल‚भाविषु स्प‚ष्टानिर्व्विक‚ल्पेषु ‚{\color{DodgerBlue3}‚य‚त्संवादि} उप‚द‚र्शितार्थ‚प्राप‚कं ‚{\tiny $_{lb}$}‚त‚द् भाव‚नाजं प्र‚त्य‚क्षं ‚{\color{DodgerBlue3}‚प्र‚माण‚मिष्टं} ।
	\pend% ending standard par
      

	  \pstart \leavevmode% starting standard par
	किमिवेत्याह (।)
	\pend% ending standard par
      

	  \pstart \leavevmode% starting standard par
	\hphantom{.}‚{\color{DodgerBlue3}‚प्राक्} प्र‚थ‚म‚प‚रिच्छेदे ‚{\color{DodgerBlue3}‚निर्ण्णीतं व‚स्तु} स‚त्य‚च‚तुष्ट‚यं त‚स्मिन्निव । य‚था आर्य‚{\tiny $_{lb}$}‚स\edtext{}{\edlabel{pvv.204-1}\label{pvv.204-1}\lemma{स}\Bfootnote{अव्य‚तिभिन्नाथो न‚र‚व्यापितार्थ उक्तः ।}}त्त्य‚विष‚यं भाव‚नाब‚ल‚जं संवादित्वात्प्र‚त्य‚क्षं प्र‚माण‚मेव‚म‚न्य‚द‚पीदृशं । ‚{\color{DodgerBlue3}‚शेषा\edtext{}{\edlabel{pvv.204-2}\label{pvv.204-2}\lemma{शेषा}\Bfootnote{मात्र‚श‚ब्द‚व्य‚व‚च्छेद्याः ।}} ‚{\tiny $_{lb}$}‚अय‚थार्था उप‚प्ल‚वा} भ्र‚मा य‚था अशुभा पृथ्वीकृत्स्नादिप्र‚त्य‚याः । (२८६)
	\pend% ending standard par
      \label{div_pvv.2.287}
	  
	% new div opening: depth here is 2
	

	  \pstart \leavevmode% starting standard par
	क‚ल्प‚नापि स्व‚संवित्ताविष्टा ना‚{\tiny $_{3}$}‚र्थे विक‚ल्प‚नादिति व्याख्यातुमाह\edtext{}{\edlabel{pvv.204-3}\label{pvv.204-3}\lemma{व्याख्यातुमाह}\Bfootnote{अत्रापि स्मृतिर‚स्तीति निर्व्विष‚य‚त्वं स्फुट‚य‚ति ।}} (।)
	\pend% ending standard par
      
	  \bigskip
	  \begingroup
	
	    \large
	  
	    \begin{quote}
	  
	    
	    \stanza[\smallbreak]
	\label{pv.2.287}\flagstanza{\tiny\textenglish{....2.287}}श‚ब्दार्थ‚ग्राहि य‚द् य‚त्र त‚ज्ज्ञानं त‚त्र क‚ल्प‚ना ।&स्व‚रूप‚ञ्च न श‚ब्दार्थ‚स्त‚त्राध्य‚क्ष‚म‚तोऽखिल‚म् ॥ २८७ ॥\&[\smallbreak]


	
	    \end{quote}
	  
	  \endgroup
	

	  \pstart \leavevmode% starting standard par
	\hphantom{.}‚{\color{DodgerBlue3}‚श‚ब्दार्थ‚ग्राहि} अन्य‚व्य‚व‚च्छेद‚स्य ग्राह‚कं ‚{\color{DodgerBlue3}‚य‚ज्ज्ञानं य‚त्र} विष‚ये ‚{\color{DodgerBlue3}‚त‚ज्ज्ञानं त‚त्र क‚ल्प‚{\tiny $_{lb}$}‚नोच्य‚ते} । ज्ञानानां ‚{\color{DodgerBlue3}‚स्व‚रूप‚ञ्च} स्व‚ल‚क्ष‚णात्म‚कं ‚{\color{DodgerBlue3}‚न श‚ब्द}‚स्या‚{\color{DodgerBlue3}‚र्थो} विष‚यो‚{\color{DodgerBlue3}‚ऽतो} वाच्य‚त्वा‚{\tiny $_{lb}$}‚‚{\color{DodgerBlue3}‚त्त‚त्र स्व‚रूपेऽखिलं} ज्ञान‚म‚विक‚ल्प‚त्वा‚{\color{DodgerBlue3}‚द‚ध्य‚क्षं} । उक्तं च‚तुर्व्विधं प्र‚त्य‚क्षं ॥ X X ॥ ‚{\tiny $_{lb}$}‚(२८७)
	\pend% ending standard par
      
	  
	% new div opening: depth here is 1
	
\chapter*[{७. प्र‚त्य‚क्षाभास‚चिन्ता}]{७. प्र‚त्य‚क्षाभास‚चिन्ता}\label{div_pvv.2.288}
	  
	% new div opening: depth here is 2
	

	  \pstart \leavevmode% starting standard par
	प्र‚त्य‚क्षाभास‚मिदानीं व‚क्त‚व्यं ।
	\pend% ending standard par
      
	  \bigskip
	  \begingroup
	
	    \large
	  
	    \begin{quote}
	  
	    
	    \stanza[\smallbreak]
	\label{pv.2.288}\flagstanza{\tiny\textenglish{....2.288}}त्रिविधं क‚ल्प‚नाज्ञान‚माश्र‚योप‚प्ल‚वोद्‏भ‚व‚म् ।&अविक‚ल्प‚क‚मेक‚ञ्च प्र‚त्य‚ज्ञाभं च‚तुर्विध‚म् ॥ २८८ ॥\&[\smallbreak]


	
	    \end{quote}
	  
	  \endgroup
	\textsuperscript{\textenglish{205/s}}

	  \pstart \leavevmode% starting standard par
	\hphantom{.}‚{\color{DodgerBlue3}‚त्रिविधं क‚ल्प‚नाज्ञानं\edtext{}{\edlabel{pvv.205-1}\label{pvv.205-1}\lemma{नाज्ञानं}\Bfootnote{भ्रान्तिसंवृतिस‚त् ज्ञान‚म‚नुमा (ना) नुमानिकं । ‚{\tiny $_{lb}$}‚स्मार्त्ताभिलापिक‚ञ्तेति प्र‚त्य‚क्षाभं स‚तैमिरं ॥ व्याच‚ष्टे ॥ ‚{\tiny $_{lb}$}‚संकेताश्र‚य‚क‚ल्प‚ना घ‚टादि । अर्थान्त‚रारोप‚क‚ल्प‚ना म‚रीचिषु ज‚लं । प‚रोक्षार्थ‚{\tiny $_{lb}$}‚क‚ल्प‚नाऽनुमानानुमानिकादिषु स‚म्ब‚न्ध‚काल‚दृष्टैक‚त्वेन वृत्ता ॥}}} प्र‚त्य‚क्षाभं म‚रीचिकायां ज‚लाध्य‚व‚सायि भ्रान्तिज्ञानं ‚{\tiny $_{lb}$}‚संवृतौ वि\edtext{}{\edlabel{pvv.205-2}\label{pvv.205-2}\lemma{वि}\Bfootnote{स‚त्त्वं द्र‚व्यं घ‚ट‚संख्याक्षेप‚ण‚स‚त्ता घ‚ट‚त्वादिषु ।}} संवादिव्य‚व‚साय‚सां‚{\tiny $_{4}$}‚वृत‚ज्ञानं । पूर्व्व‚दृष्टैक‚त्व‚क‚ल्प‚नाप्र‚वृत्तं लिङ्गानुमे‚{\tiny $_{lb}$}‚यादिज्ञानं । ‚{\color{DodgerBlue3}‚अविक‚ल्प‚क‚ञ्च एकं प्र‚त्य‚क्षाभं} । कीदृश‚मा‚{\color{DodgerBlue3}‚श्र‚य‚स्ये}‚न्द्रिय‚स्यो‚{\color{DodgerBlue3}‚प‚प्ल‚व‚स्ति}‚मिरा ‚{\tiny $_{lb}$}‚द्युप‚घात‚स्त‚स्मा‚{\color{DodgerBlue3}‚द् भ‚वो} य‚स्य त‚त्त‚था । एव‚ञ्च ‚{\color{DodgerBlue3}‚च‚तुर्विध‚ञ्च} प्र‚त्य‚क्षाभासं ॥ (२८८)
	\pend% ending standard par
      \label{div_pvv.2.289}
	  
	% new div opening: depth here is 2
	

	  \pstart \leavevmode% starting standard par
	न‚न्व‚विक‚ल्प‚कं प्र‚त्य‚क्षं । त‚त‚स्त्र‚य‚म‚पीदं स‚विक‚ल्प‚क‚त्वादेकः प्र‚त्य‚क्षाभासः । त‚त् ‚{\tiny $_{lb}$}‚किं भ्रान्तिज्ञानं मृग‚तृष्णिकायां ज‚लाव‚सायि । संवृतिस‚तो द्र‚व्यादेर्ज्ञानं । अनुमानं ‚{\tiny $_{lb}$}‚लिङ्ग‚ज्ञानं आनुमानिकं‚{\tiny $_{5}$}‚ लिङ्गिज्ञानं स्मार्त्त स्मृतिः आभिलापिकं चेति विक‚ल्प‚{\tiny $_{lb}$}‚प्र‚भेद आचार्य दि ग्ना गे नोक्त इत्याह (।)
	\pend% ending standard par
      
	  \bigskip
	  \begingroup
	
	    \large
	  
	    \begin{quote}
	  
	    
	    \stanza[\smallbreak]
	\label{pv.2.289}\flagstanza{\tiny\textenglish{....2.289}}अन‚क्ष‚ज‚त्व‚सिद्ध्य‚र्थ‚मुक्ते द्वे भ्रान्तिद‚श‚नात् ॥&सिद्धानुमादिव‚च‚नं साध‚नायैव पूर्व्व‚योः ॥ २८९ ॥\&[\smallbreak]


	
	    \end{quote}
	  
	  \endgroup
	

	  \pstart \leavevmode% starting standard par
	\hphantom{.}द्वे सांवृतारोपित‚योः क‚ल्प‚नाज्ञानेऽ‚{\color{DodgerBlue3}‚न‚क्ष‚ज‚त्व}‚स्यानिन्द्रिय‚त्व‚स्य ‚{\color{DodgerBlue3}‚सिद्ध्य‚र्थं} भेदे‚{\tiny $_{lb}$}‚‚{\color{DodgerBlue3}‚नोक्ते} । प‚रेषां त‚योर‚क्ष‚ज‚त्व‚भ्रान्तिद‚र्श‚नाद् । घ‚टोयं द्वौ\edtext{}{\edlabel{pvv.205-3}\label{pvv.205-3}\lemma{द्वौ}\Bfootnote{क‚ल्प‚न‚म‚पि स्व‚स‚म्वित्ताव‚ध्य‚क्षं स्व‚स‚म्वेद्य‚त्वात् सुखादिव‚त् ।}} क‚म्प‚त इत्यादि । ज‚ल‚{\tiny $_{lb}$}‚मिद‚मिति च व्य‚व‚सायात्म‚क‚{\color{DodgerBlue3}‚मिन्द्रिय‚प्र‚त्य‚क्ष‚मेव} प्र‚तिप‚द्य‚त इति प‚रो म‚न्य‚ते । त‚न्निरा ‚{\tiny $_{lb}$}‚सार्थं द्व‚योरुपादानं । अनिन्द्रिय‚त्वेन स्मृतिब‚ल‚भावित्वेन सिद्धं च त‚द‚{\color{DodgerBlue3}‚नुमादि} च ‚{\tiny $_{lb}$}‚त‚स्य ‚{\color{DodgerBlue3}‚व‚च‚नं पूर्व्व‚योः} संवृतारोपित‚क‚ल्प‚न‚योरेवा‚{\color{DodgerBlue3}‚न‚क्ष‚ज‚त्व‚साध‚नाय} । त‚था हि ‚{\tiny $_{lb}$}‚य‚त्पूर्व्वानुभूत‚स‚म‚य‚स्मृतिभावि न त‚त्प्र‚त्य‚क्षं । य‚थाऽनुमानादि । अनुभूत‚स‚म‚य‚स्मृति‚{\tiny $_{lb}$}‚सापेक्षा चाव‚य‚विज‚लादिक‚ल्प‚ना इति विरुद्धोप‚ल‚ब्धिरुक्ता । (२८९)
	\pend% ending standard par
      \label{div_pvv.2.290}
	  
	% new div opening: depth here is 2
	

	  \pstart \leavevmode% starting standard par
	न‚नु सांवृतारोपित‚क‚ल्प‚नायाः क‚थं‚{\tiny $_{6}$}‚ प्र‚त्य‚क्ष‚ताश‚ङ्केत्याह (।)
	\pend% ending standard par
      
	  \bigskip
	  \begingroup
	
	    \large
	  
	    \begin{quote}
	  
	    
	    \stanza[\smallbreak]
	\label{pv.2.290}\flagstanza{\tiny\textenglish{....2.290}}संकेत‚संश्र‚यान्यार्थ‚स‚मारोप‚विक‚ल्प‚ने ।&न प्र‚त्य‚क्षानुवृत्तित्वात् क‚दाचिद् भ्रान्तिकार‚ण‚म् ॥ २९० ॥\&[\smallbreak]


	
	    \end{quote}
	  
	  \endgroup
	

	  \pstart \leavevmode% starting standard par
	ब‚हूनां रूपादीनामेकार्थ‚कारित्व‚ख्याप‚नार्थं घ‚ट इत्यादिश‚ब्द‚निवेशः स संश्र‚यो ‚{\tiny $_{lb}$}‚हेतुर्य‚स्याः सा ‚{\color{DodgerBlue3}‚संकेत\edtext{}{\edlabel{pvv.205-4}\label{pvv.205-4}\lemma{संकेत}\Bfootnote{संकेतः प‚र‚माणुषु । संख्यास‚म‚च्च‚य‚व्य‚व‚च्छेदेन ।}}संश्र‚या}‚क‚ल्प‚ना दृश्य‚मानान्म‚रीचिनिच‚यादेर‚न्य‚स्य ज‚लादेरा‚{\tiny $_{lb}$}‚\leavevmode\ledsidenote{\textenglish{206/s}} रोपित‚स्य क‚ल्प‚ना । ‚{\color{DodgerBlue3}‚अन्यार्थ}‚क‚ल्प‚ना । ते एते क‚ल्प‚ने प्र‚त्य‚क्षान‚न्त‚र‚भावित्वेन ‚{\tiny $_{lb}$}‚‚{\color{DodgerBlue3}‚प्र‚त्य‚क्षानुवृत्तित्वादा}‚त्म‚न्य‚पि ‚{\color{DodgerBlue3}‚क‚दाचिद}‚विम‚र्श‚कानां प्र‚त्य‚क्ष‚ता---‚{\color{DodgerBlue3}‚भ्रान्तिकार‚णं} भ‚व‚तः । (२९०)
	\pend% ending standard par
      \label{div_pvv.2.291}
	  
	% new div opening: depth here is 2
	
	  \bigskip
	  \begingroup
	
	    \large
	  
	    \begin{quote}
	  
	    
	    \stanza[\smallbreak]
	\label{pv.2.291}\flagstanza{\tiny\textenglish{....2.291}}य‚थैवेयं प‚रोक्षार्थ‚क‚ल्प‚ना स्म‚र‚णात्मिका ।&स‚म‚यापेक्षिणी नार्थं प्र‚त्य‚क्ष‚म‚ध्य‚व‚स्य‚ति ॥ २९१ ॥\&[\smallbreak]


	
	    \end{quote}
	  
	  \endgroup
	\textsuperscript{\textenglish{40b/MA}}

	  \pstart \leavevmode% starting standard par
	\hphantom{.}अतो ‚{\color{DodgerBlue3}‚य‚थैवेय‚म}‚नुमानानुमानिकादि‚{\color{DodgerBlue3}‚प‚रो‚{\tiny $_{7}$}‚क्षार्थ‚क‚ल्प‚ना} पूर्व्वं गृहीत‚{\color{DodgerBlue3}‚स‚म‚यापेक्षिणी} स‚ती ‚{\color{DodgerBlue3}‚स्म‚र‚णा}‚दिरूपा ‚{\color{DodgerBlue3}‚प्र‚त्य‚क्षं} प्र‚त्य‚क्ष‚विष‚य‚{\color{DodgerBlue3}‚म‚र्थं ना} (‚{\color{DodgerBlue3}‚ध्य}‚) ‚{\color{DodgerBlue3}‚व‚स्य‚ति} । किन्तु क‚ल्पित‚मेव ‚{\tiny $_{lb}$}‚गृह्णाति (। २९१)
	\pend% ending standard par
      \label{div_pvv.2.292}
	  
	% new div opening: depth here is 2
	
	  \bigskip
	  \begingroup
	
	    \large
	  
	    \begin{quote}
	  
	    
	    \stanza[\smallbreak]
	\label{pv.2.292}\flagstanza{\tiny\textenglish{....2.292}}त‚थानुभूत‚स्म‚र‚ण‚म‚न्त‚रेण घ‚टादिषु ।&न प्र‚त्य‚योनुयँस्त‚च्च प्र‚त्य‚क्षात् प‚रिहीय‚ते ॥ २९२ ॥\&[\smallbreak]


	
	    \end{quote}
	  
	  \endgroup
	

	  \pstart \leavevmode% starting standard par
	\hphantom{.}‚{\color{DodgerBlue3}‚त‚था} संकेत‚काले व्य‚व‚हार‚ल‚घ‚वार्थं क‚ल्पित‚स्य द्र‚व्यादेर‚नु‚{\color{DodgerBlue3}‚मान‚भूत}‚स्य प‚श्चा‚{\tiny $_{lb}$}‚‚{\color{DodgerBlue3}‚त्स्म‚र‚ण‚म‚न्त‚रेण घ‚टादिषु} घ‚टोऽय‚मित्याद्येक‚व‚स्त्व‚व‚सायी ‚{\color{DodgerBlue3}‚प्र‚त्य‚यो} न भ‚व‚ति । ‚{\color{DodgerBlue3}‚त‚च्च} संकेतित‚मारोपितं चार्थ‚म‚नुय‚न् घ‚टोऽयं ज‚ल‚मिति च प्र‚त्य‚यः ‚{\color{DodgerBlue3}‚प्र‚त्य‚क्षात्} प्र‚त्य‚क्ष‚{\color{DodgerBlue3}‚त्वात् ‚{\tiny $_{lb}$}‚प‚रिहीय‚ते} । त‚स्य व‚स्तुविष‚य‚त्वात् संकेतापेक्ष‚स्य च वि‚{\tiny $_{1}$}‚प‚र्य‚यात् । (२९२)
	\pend% ending standard par
      \label{div_pvv.2.293}
	  
	% new div opening: depth here is 2
	
	  \bigskip
	  \begingroup
	
	    \large
	  
	    \begin{quote}
	  
	    
	    \stanza[\smallbreak]
	\label{pv.2.293}\flagstanza{\tiny\textenglish{....2.293}}अप‚वाद‚श्च‚तुर्थोत्र तेनोक्त‚मुप‚घात‚ज‚म् ।&केव‚ल‚न्त‚त्र तिमिर‚मुप‚घातोप‚ल‚क्ष‚ण‚म् ॥ २९३ ॥\&[\smallbreak]


	
	    \end{quote}
	  
	  \endgroup
	

	  \pstart \leavevmode% starting standard par
	\hphantom{.}स‚तैमिर‚मिति ‚{\color{DodgerBlue3}‚च‚तुर्थो} हेत्वाभासः क‚ल्प‚नार‚हित‚त्वेनातिप्र‚स‚क्तायाः प्र‚त्य‚क्ष‚ताया ‚{\tiny $_{lb}$}‚‚{\color{DodgerBlue3}‚अप‚वादः} । अभ्रान्त‚त्व‚स्य ल‚क्ष‚णैक‚देश‚त्वोप‚ल‚क्ष‚ण‚त्वं च न तु क‚ल्प‚नापोढ‚त्व‚निरा‚{\tiny $_{lb}$}‚कृत‚स्योदाह‚र‚ण‚मिदं । ‚{\color{DodgerBlue3}‚केव‚लं स‚तैमिर‚मि}‚त्य‚त्र तिमिरं स‚बाह्यो‚{\color{DodgerBlue3}‚प‚घात‚स्य} ज्ञान‚विकृति‚{\tiny $_{lb}$}‚हेतो‚{\color{DodgerBlue3}‚रुप‚ल‚क्ष‚णं} । तेनान्ये\edtext{}{\edlabel{pvv.206-1}\label{pvv.206-1}\lemma{तेनान्ये}\Bfootnote{श्रोत्रादि ।}}न्द्रिय‚विकार‚ज‚मिन्द्रिय‚ज‚ञ्च निर्व्विक‚ल्पं प्र‚त्य‚क्षाभासं ‚{\tiny $_{lb}$}‚सि\edtext{}{\edlabel{pvv.206-2}\label{pvv.206-2}\lemma{सि}\Bfootnote{उक्तं ।}}ध्य‚ति (॥ २९३)
	\pend% ending standard par
      \label{div_pvv.2.294}
	  
	% new div opening: depth here is 2
	
	  \bigskip
	  \begingroup
	
	    \large
	  
	    \begin{quote}
	  
	    
	    \stanza[\smallbreak]
	\label{pv.2.294}\flagstanza{\tiny\textenglish{....2.294}}मान‚सं त‚द‚पीत्येके तेषां ग्र‚न्थो विरुध्य‚ते ।&नील‚द्विच‚न्द्रादिधियां हेतुर‚क्षाण्य‚पीत्य‚य‚म् ॥ २९४ ॥\&[\smallbreak]


	
	    \end{quote}
	  
	  \endgroup
	

	  \pstart \leavevmode% starting standard par
	\hphantom{.}‚{\color{DodgerBlue3}‚त‚द्} द्विच‚न्द्रादिज्ञान‚{\color{DodgerBlue3}‚म‚पि मान‚सं} म‚नोभ्र‚म ‚{\color{DodgerBlue3}‚इत्येके}\edtext{}{\edlabel{pvv.206-3}\label{pvv.206-3}\lemma{म}\Bfootnote{क‚णादाद‚यः ।}} आ‚{\tiny $_{2}$}‚ चा र्याः । ‚{\color{DodgerBlue3}‚तेषा}‚मेवं ‚{\tiny $_{lb}$}‚वादिनां ‚{\color{DodgerBlue3}‚नील‚द्विच‚न्द्रादिधियाम‚क्षाण्य‚पि हेतु}‚रित्येत‚द‚र्थ‚वाच‚को ‚{\color{DodgerBlue3}‚ग्र‚न्थो विरुध्य‚ते । ‚{\tiny $_{lb}$}‚ग्र‚न्थः} पुन‚र‚यं याव‚च्च‚क्षुरादीनाम‚प्या‚{\color{DodgerBlue3}‚ल‚म्ब‚न‚त्व‚प्र‚स‚ङ्गः} । तेपि हि प‚र‚मार्थ‚{\tiny $_{lb}$}‚\leavevmode\ledsidenote{\textenglish{207/s}} तोऽन्य‚था विद्य\edtext{}{\edlabel{pvv.207-1}\label{pvv.207-1}\lemma{विद्य}\Bfootnote{अजाजीपुष्प‚तुल्येनास‚रूप‚क‚त्वेन ।}}माना नीलाद्याभास‚स्य द्विच‚न्द्राद्याभास‚स्य च ज्ञान‚स्य कार‚णी‚{\tiny $_{lb}$}‚भ‚व‚न्तीति । (२९४)
	\pend% ending standard par
      \label{div_pvv.2.295}
	  
	% new div opening: depth here is 2
	

	  \pstart \leavevmode% starting standard par
	स्यादेत‚त् (।)
	\pend% ending standard par
      
	  \bigskip
	  \begingroup
	
	    \large
	  
	    \begin{quote}
	  
	    
	    \stanza[\smallbreak]
	\label{pv.2.295}\flagstanza{\tiny\textenglish{....2.295}}पार‚म्प‚र्येण हेतुश्चेदिन्द्रिय‚ज्ञान‚गोच‚रे ।&विचार्य‚माणे प्र‚स्तावो मान‚स‚स्येह कीदृशः ॥ २९५ ॥\&[\smallbreak]


	
	    \end{quote}
	  
	  \endgroup
	

	  \pstart \leavevmode% starting standard par
	मान‚स‚स्य प्र‚त्य\edtext{}{\edlabel{pvv.207-2}\label{pvv.207-2}\lemma{त्य}\Bfootnote{त‚द‚न‚न्त‚र‚ज‚त्वात् ।}}क्ष‚स्येन्द्रियं ‚{\color{DodgerBlue3}‚पार‚म्प‚र्येण हेतुः} । तेन विरोधाभाव‚{\color{DodgerBlue3}‚श्चेत्\edtext{}{\edlabel{pvv.207-3}\label{pvv.207-3}\lemma{श्चेत्}\Bfootnote{य‚दुक्तं दिग्नागेन बौद्धं प्र‚त्येत‚त् ।}}} वा द ‚{\tiny $_{lb}$}‚वि धि प्र क‚र‚णे ‚{\color{DodgerBlue3}‚इन्द्रिय‚ज्ञान}‚स्य प्र‚त्य‚क्ष‚स्य ‚{\color{DodgerBlue3}‚गोच‚रे‚{\tiny $_{3}$}‚विचार्य‚माणे मान‚स‚स्य} विक‚ल्प‚स्य ‚{\tiny $_{lb}$}‚इहाव‚स‚रे कीदृशः ‚{\color{DodgerBlue3}‚प्र‚स्तावः} । येन प‚र‚म्प‚र‚या त‚द्धेतुरिन्द्रिय‚मुच्य‚ते । (२९५)
	\pend% ending standard par
      \label{div_pvv.2.296}
	  
	% new div opening: depth here is 2
	
	  \bigskip
	  \begingroup
	
	    \large
	  
	    \begin{quote}
	  
	    
	    \stanza[\smallbreak]
	\label{pv.2.296}\flagstanza{\tiny\textenglish{....2.296}}किम्वैन्द्रियं य‚द‚क्षाणां भावाभावानुरोधि चेत् ।&त‚त्तुल्यं विक्रियाव‚च्चेत् सैवेयं किं निषिध्य‚ते ॥ २९६ ॥\&[\smallbreak]


	
	    \end{quote}
	  
	  \endgroup
	

	  \pstart \leavevmode% starting standard par
	\hphantom{.}तिमिर‚ज्ञानं भ्र‚म‚म‚निच्छ‚तोपि ‚{\color{DodgerBlue3}‚किं} कीदृश‚{\color{DodgerBlue3}‚मैन्द्रिय}‚ज्ञान‚मिष्य‚ते । ‚{\color{DodgerBlue3}‚य‚द‚क्षाणां भावा‚{\tiny $_{lb}$}‚भाव‚योर‚नुरोधि} स्व‚भावाभ्याम‚नुव‚र्त‚कं त‚दैन्द्रिय‚मिति ‚{\color{DodgerBlue3}‚चेत्} । त‚त् इन्द्रिय‚{\color{DodgerBlue3}‚भावाभा}‚{\tiny $_{lb}$}‚वानुरोधित्वं तैमिरिक‚ज्ञान‚स्यापि ‚{\color{DodgerBlue3}‚तुल्यं} (।) न हीन्द्रिय‚व्यापार‚म‚न्त‚रेण तैमिरि‚{\tiny $_{lb}$}‚क‚ज्ञान‚मुत्प‚द्य‚ते । इन्द्रि\edtext{}{\edlabel{pvv.207-4}\label{pvv.207-4}\lemma{इन्द्रि}\Bfootnote{अथ पूर्व्व‚कं हित्वा ।}}य‚विकारेण विक्रियाव‚त् ज्ञान‚मैन्द्रियं चेत् द्वि‚{\tiny $_{4}$}‚च‚न्द्रादिज्ञानानां ‚{\tiny $_{lb}$}‚तिमिरादीन्द्रिय‚विकारेण ‚{\color{DodgerBlue3}‚विक्रिया} (।) ‚{\color{DodgerBlue3}‚सैवेय}‚म‚भूतार्थोप‚द‚र्श‚नान्मिका भ्रान्त‚त्व‚निमि‚{\tiny $_{lb}$}‚त्त‚मुक्ताऽस्माभिः ‚{\color{DodgerBlue3}‚किं निषिध्य‚ते} । (२९६)
	\pend% ending standard par
      \label{div_pvv.2.297}
	  
	% new div opening: depth here is 2
	
	  \bigskip
	  \begingroup
	
	    \large
	  
	    \begin{quote}
	  
	    
	    \stanza[\smallbreak]
	\label{pv.2.297}\flagstanza{\tiny\textenglish{....2.297}}स‚र्प्पादिभ्रान्तिव‚च्चास्याः स्याद‚क्ष‚विकृताव‚पि ।&निवृत्तिर्न निव‚र्त्तेत निवृत्तेप्य‚क्ष‚विप्ल‚वे ॥ २९७ ॥\&[\smallbreak]


	
	    \end{quote}
	  
	  \endgroup
	

	  \pstart \leavevmode% starting standard par
	य‚दि च द्विच‚न्द्रादिधीर्मान‚सी भ्रान्तिस्त‚दा स‚र्पोचित‚देशे म‚न्द‚म‚न्दालोके र‚ज्वादौ ‚{\tiny $_{lb}$}‚संस्थान‚साम्य‚ग्र‚हात् उत्प‚न्नायाः स‚र्पादिभ्रान्तेरिव म‚रीचिषु ज‚ल‚भ्रान्तेरिव य‚था ‚{\tiny $_{lb}$}‚प्र‚त्य‚क्षं विचारात् । अस्य(ा) द्विच‚न्द्रादिभ्रान्तेर‚क्ष‚{\color{DodgerBlue3}‚विकृताव‚पि} सा निवृत्तिः स्यात् । ‚{\tiny $_{lb}$}‚अक्ष‚विप्ल‚वो हि न त‚स्या हेतुः‚{\tiny $_{5}$}‚ (।) त‚त‚श्च स‚त्य‚पि त‚स्मिन् विम‚र्शान्निव‚र्त‚ते ‚{\tiny $_{lb}$}‚स‚र्प‚बुद्धिरिव । त‚था ह्य‚क्ष‚{\color{DodgerBlue3}‚विप्ल‚वेपि} तिमिरादौ ‚{\color{DodgerBlue3}‚निवृत्तेपि} त‚त्त्व‚म‚विचार‚य‚तो ‚{\color{DodgerBlue3}‚न ‚{\tiny $_{lb}$}‚निव‚र्त‚ते} त‚द्‏द्विच‚न्द्र‚बुद्धिः । अक्ष‚विप्ल‚व‚स्य त‚त्कार‚ण‚त्वाभावात् । अकार‚ण‚निवृत्तौ ‚{\tiny $_{lb}$}‚च निवृत्त्य‚योगात् । (२९७)
	\pend% ending standard par
      \label{div_pvv.2.298}
	  
	% new div opening: depth here is 2
	

	  \pstart \leavevmode% starting standard par
	किञ्च (।)
	\pend% ending standard par
      
	  \bigskip
	  \begingroup
	
	    \large
	  
	    \begin{quote}
	  
	    
	    \stanza[\smallbreak]
	\label{pv.2.298}\flagstanza{\tiny\textenglish{....2.298}}क‚दाचिद‚न्य‚स‚न्ताने त‚थैवार्प्येत वाच‚कैः ।&दृष्ट‚स्मृतिम‚पेक्षेत न भासेत प‚रिस्फुट‚म् ॥ २९८ ॥\&[\smallbreak]


	
	    \end{quote}
	  
	  \endgroup
	\textsuperscript{\textenglish{208/s}}

	  \pstart \leavevmode% starting standard par
	य‚थानुभ‚वं स‚म‚देश‚काल‚स्यान्य‚स्य चित्त‚स‚न्ताने य‚था अहिर‚हि‚{\tiny $_{lb}$}‚रित्युप‚द‚र्श‚नेन स‚र्प‚भ्रान्तिर‚र्प्य‚ते । ‚{\color{DodgerBlue3}‚त‚था} द्विच‚न्द्रादिभ्रान्तिर‚पि त‚स्या वाच‚कैः ‚{\tiny $_{lb}$}‚श‚ब्दैर‚{\color{DodgerBlue3}‚र्प्येत} साम‚ग्रीतुल्य‚त्वात् । तिमिर‚{\tiny $_{6}$}‚स्य चाहेतुक‚त्वात् । अपि च(।) य‚था म‚रीचिषु ‚{\tiny $_{lb}$}‚त‚र‚ङ्ग‚ज‚ल‚स‚मासु पूर्व्व‚दृष्ट‚ज‚ल‚स्म‚र‚ण‚सापेक्षा ज‚ल‚भ्रान्तिः । त‚त्र द्विच‚न्द्रादिभ्रान्तिर‚पि ‚{\tiny $_{lb}$}‚मान‚सीत्वात् ‚{\color{DodgerBlue3}‚दृष्ट}‚च‚न्द्र‚द्व‚य‚{\color{DodgerBlue3}‚स्मृतिम‚पेक्षेत} । न चेयं स्म‚र‚ण‚सापेक्षा च‚क्षुर्व्विस्फार‚ण‚{\tiny $_{lb}$}‚मात्रेण स्फ‚र‚णात् । \edtext{\textsuperscript{*}}{\edlabel{pvv.208-1}\label{pvv.208-1}\lemma{*}\Bfootnote{न विक‚ल्पानुब‚द्ध‚स्य स्प‚ष्टार्थ‚प्र‚तिभासिता ।}} त‚था मान‚सीत्वात् ‚{\color{DodgerBlue3}‚प‚रिस्फुटं} सुव्य‚क्त‚ग्राह्याकारा ‚{\color{DodgerBlue3}‚न भासेत} ।\edtext{\textsuperscript{*}}{\edlabel{pvv.208-2}\label{pvv.208-2}\lemma{*}\Bfootnote{क‚ल्प‚नापोढ‚व‚च‚नादेव निवृत्तेः । अभ्रान्त‚त्वे य‚त्न‚वैर्थ्यं स्यात् ।}} ‚{\tiny $_{lb}$}‚ज‚लादिभ्रान्तिरिव । (२९८)
	\pend% ending standard par
      \label{div_pvv.2.299}
	  
	% new div opening: depth here is 2
	
	  \bigskip
	  \begingroup
	
	    \large
	  
	    \begin{quote}
	  
	    
	    \stanza[\smallbreak]
	\label{pv.2.299}\flagstanza{\tiny\textenglish{....2.299}}सुप्त‚स्य जाग्र‚तो वापि यैव धीः स्फुट‚भासिनी ।&सा निर्व्विक‚ल्पोभ‚य‚थाप्य‚न्य‚थैव विक‚ल्पिका ॥ २९९ ॥\&[\smallbreak]


	
	    \end{quote}
	  
	  \endgroup
	

	  \pstart \leavevmode% starting standard par
	\hphantom{.}त‚स्मा‚{\color{DodgerBlue3}‚त्सुप्त‚स्य जाग्र‚तोपि वा यैव धीः स्फुटाव‚भासिनी} व्य‚क्त‚ग्राह्याकारा सा ‚{\tiny $_{lb}$}‚\leavevmode\ledsidenote{\textenglish{41a/MA}} ‚{\color{DodgerBlue3}‚निर्व्विक‚ल्पा}‚ऽभ्युप‚ग‚न्त‚व्या‚{\tiny $_{7}$}‚ ‚{\color{DodgerBlue3}‚अन्य‚थैव} ह्य‚स्फुटाव‚भासिनी धीरुभ‚य‚था सुप्त‚स्य जाग्र‚तो ‚{\tiny $_{lb}$}‚पि वा ‚{\color{DodgerBlue3}‚क‚ल्पिका} युक्ता । (२९९)
	\pend% ending standard par
      \label{div_pvv.2.300}
	  
	% new div opening: depth here is 2
	

	  \pstart \leavevmode% starting standard par
	य‚त एवं (।)
	\pend% ending standard par
      
	  \bigskip
	  \begingroup
	
	    \large
	  
	    \begin{quote}
	  
	    
	    \stanza[\smallbreak]
	\label{pv.2.300}\flagstanza{\tiny\textenglish{....2.300}}त‚स्मात्त‚स्याविक‚ल्पेपि प्रामाण्यं प्र‚तिषिध्य‚ते ।&विसंवादात्त‚द‚र्थं च प्र‚त्य‚क्षाभं द्विधोदित‚म् ॥ ३०० ॥\&[\smallbreak]


	
	    \end{quote}
	  
	  \endgroup
	

	  \pstart \leavevmode% starting standard par
	\hphantom{.}‚{\color{DodgerBlue3}‚त‚स्मात्} क‚ल्प‚नापोढ‚मित्युक्ते तिमिर‚ज्ञान‚स्या‚{\color{DodgerBlue3}‚विक‚ल्पे} निर्व्विक‚ल्प‚त्वेपि स‚ति ‚{\tiny $_{lb}$}‚‚{\color{DodgerBlue3}‚प्रामाण्यं} प्र‚त्य‚क्षात्म‚कं प्राप्तं स‚तैमिर‚मित्य‚प‚वादेन ‚{\color{DodgerBlue3}‚विस‚म्वादात् प्र‚तिषिध्य‚ते} । ‚{\tiny $_{lb}$}‚संवाद‚ल‚क्ष‚ण‚त्वात्प्रामाण्य‚स्य । ‚{\color{DodgerBlue3}‚त‚द‚र्थ‚मु}‚प‚प्लुत‚ज्ञान‚निवृत्त्य‚र्थ चाचार्य दिङ् ना गे न प्र‚त्य ‚{\tiny $_{lb}$}‚‚{\color{DodgerBlue3}‚क्षाभं} संक्षेप‚तो ‚{\color{DodgerBlue3}‚द्विधोक्तं} स‚विक‚ल्प‚म‚विक‚ल्प‚ञ्च । उक्तं प्र‚त्य‚क्षाभं ॥ XX ॥(३००)
	\pend% ending standard par
      
	  
	% new div opening: depth here is 1
	
\chapter*[{८. प्र‚माण‚फ‚ल‚चिन्ता}]{८. प्र‚माण‚फ‚ल‚चिन्ता}\label{div_pvv.2.301}
	  
	% new div opening: depth here is 2
	

	  \pstart \leavevmode% starting standard par
	प्र‚माण‚{\tiny $_{1}$}‚फ‚ल‚व्य‚व‚स्थां क‚र्तुमाह ।
	\pend% ending standard par
      
	  \bigskip
	  \begingroup
	
	    \large
	  
	    \begin{quote}
	  
	    
	    \stanza[\smallbreak]
	\label{pv.2.301}\flagstanza{\tiny\textenglish{....2.301}}क्रियासाध‚न‚मित्येव स‚र्व्वं स‚र्व्व‚स्य क‚र्म्म‚णः ।&साध‚नं न हि त‚त्त‚स्य साध‚नं या क्रिया य‚तः ॥ ३०१ ॥\&[\smallbreak]


	
	    \end{quote}
	  
	  \endgroup
	

	  \pstart \leavevmode% starting standard par
	\hphantom{.}‚{\color{DodgerBlue3}‚क्रियायाः साध‚नं} हेतुरि‚{\color{DodgerBlue3}‚त्येव न} हि ‚{\color{DodgerBlue3}‚स‚र्व्वं} कार‚णं ‚{\color{DodgerBlue3}‚स‚र्व्व‚स्य क‚र्म्म‚णः} क्रियायाः ‚{\tiny $_{lb}$}‚‚{\color{DodgerBlue3}‚साध‚नं} क‚र‚णं (।) किन्त‚र्हि त‚द्व‚स्तु ‚{\color{DodgerBlue3}‚त‚स्य} क‚र्म‚णः ‚{\color{DodgerBlue3}‚साध‚नं} क‚र‚णं ‚{\color{DodgerBlue3}‚या क्रिया य‚तः} प‚दार्थाद‚{\tiny $_{lb}$}‚\leavevmode\ledsidenote{\textenglish{209/s}} व्य‚व‚धानेन भ‚व‚ति स त‚स्याः कार‚ण‚मुच्य‚ते । त‚त‚श्चेन्द्रियादेः प्र‚मितिं प्र‚त्य‚व्य‚व‚हित‚{\tiny $_{lb}$}‚साध‚क‚त्वाभावात् न प्र‚माणं । (३०१)
	\pend% ending standard par
      \label{div_pvv.2.302}
	  
	% new div opening: depth here is 2
	

	  \pstart \leavevmode% starting standard par
	किन्त‚र्हि प्र‚माण‚म‚स्तीत्याह (।)
	\pend% ending standard par
      
	  \bigskip
	  \begingroup
	
	    \large
	  
	    \begin{quote}
	  
	    
	    \stanza[\smallbreak]
	\label{pv.2.302}\flagstanza{\tiny\textenglish{....2.302}}त‚त्रानुभ‚व‚मात्रेण ज्ञान‚स्य स‚दृशात्म‚नः ।&भाव्य‚न्तेनात्म‚ना येन प्र‚तिक‚र्म्म विभ‚ज्य‚ते ॥ ३०२ ॥\&[\smallbreak]


	
	    \end{quote}
	  
	  \endgroup
	

	  \pstart \leavevmode% starting standard par
	\hphantom{.}‚{\color{DodgerBlue3}‚त‚त्र} रूपादौ क‚र्म‚णि ‚{\color{DodgerBlue3}‚ज्ञान‚स्यानुभ‚व‚मात्रेणा}‚नुभ‚वात्म‚नो ‚{\color{DodgerBlue3}‚स‚दृशात्म}‚न‚स्तुल्य‚रूप‚स्य‚{\tiny $_{2}$}‚ ‚{\tiny $_{lb}$}‚‚{\color{DodgerBlue3}‚तेनात्म‚ना} स्व‚रूपेण प्र‚तिविष‚यं व्य‚तिरेकिणा ‚{\color{DodgerBlue3}‚भाव्यं येन प्र‚तिक‚र्म्म} प्र‚तिविष‚य‚ज्ञानं ‚{\tiny $_{lb}$}‚‚{\color{DodgerBlue3}‚विभ‚ज्य‚ते} । नील‚स्येदं पीत‚स्येद‚मिति । अन्य‚थानुभ‚व‚मात्र‚त‚या स‚र्व्व‚त्र विष‚ये स‚दृशं ‚{\tiny $_{lb}$}‚ज्ञानं प्र‚तिविष‚यं क‚थं भेदेन व्य‚व‚स्थाप‚यितुं श‚क्येत । (३०२)
	\pend% ending standard par
      \label{div_pvv.2.303}
	  
	% new div opening: depth here is 2
	

	  \pstart \leavevmode% starting standard par
	स्यादेत‚दिन्द्रियादेर्हेतोः स‚व्यापा\edtext{}{\edlabel{pvv.209-1}\label{pvv.209-1}\lemma{व्यापा}\Bfootnote{आविल‚तादि ।}} र‚तादिल‚क्ष‚णो विशेषो ज्ञानानां भेदेन निया‚{\tiny $_{lb}$}‚म‚क‚मित्याह (।)
	\pend% ending standard par
      
	  \bigskip
	  \begingroup
	
	    \large
	  
	    \begin{quote}
	  
	    
	    \stanza[\smallbreak]
	\label{pv.2.303}\flagstanza{\tiny\textenglish{....2.303}}अनात्म‚भूतो भेदोस्य विद्य‚मानोपि हेतुषु ।&भिन्ने क‚र्म‚ण्य‚भिन्न‚स्य न भेदेन नियाम‚कः ॥ ३०३ ॥\&[\smallbreak]


	
	    \end{quote}
	  
	  \endgroup
	

	  \pstart \leavevmode% starting standard par
	\hphantom{.}‚{\color{DodgerBlue3}‚हेतु}‚ष्विन्द्रियादिषु ‚{\color{DodgerBlue3}‚भेदो} विशेषः स‚व्यापार‚तादिल‚क्ष‚णो ‚{\color{DodgerBlue3}‚विद्य‚{\tiny $_{3}$}‚मानोप्य‚स्य} ज्ञान\edtext{}{\edlabel{pvv.209-2}\label{pvv.209-2}\lemma{ज्ञान}\Bfootnote{विष‚य‚सारूप्यान‚धिग‚मे निराकार‚वादिनः स‚र्व्वं ज्ञानं ‚{\tiny $_{lb}$}‚बोध‚रूप‚तामात्रेणाव‚शिष्टं विष‚याधिग‚ते हेतुरिति स्थितिः ।}}स्यानुभ‚व‚मात्रात्म‚त‚या भिन्न‚स्य ‚{\color{DodgerBlue3}‚भिन्ने क‚र्म्म‚णि} नीलादौ ग्राह्ये ‚{\color{DodgerBlue3}‚भेदेन नियाम‚को} न युक्तः । क‚स्मादित्याह (।) ‚{\color{DodgerBlue3}‚अनात्म‚भूतः} । ज्ञानास्व‚रूप‚त्वादिन्द्रिय‚स्य विशेषः ‚{\tiny $_{lb}$}‚प्र‚तिक‚र्म्म न भेत्तुम‚र्ह‚ति । (३०३)
	\pend% ending standard par
      \label{div_pvv.2.304}
	  
	% new div opening: depth here is 2
	
	  \bigskip
	  \begingroup
	
	    \large
	  
	    \begin{quote}
	  
	    
	    \stanza[\smallbreak]
	\label{pv.2.304}\flagstanza{\tiny\textenglish{....2.304}}त‚स्माद् य‚तोस्यात्म‚भेदाद‚स्याधिग‚तिरित्य‚य‚म् ।&क्रियायाः क‚र्म‚निय‚मः सिद्धा सा त‚त्प्र‚साध‚ना ॥ ३०४ ॥\&[\smallbreak]


	
	    \end{quote}
	  
	  \endgroup
	

	  \pstart \leavevmode% starting standard par
	\hphantom{.}‚{\color{DodgerBlue3}‚त‚स्माद‚स्य} ज्ञान‚स्या‚{\color{DodgerBlue3}‚त्म‚भेदात् य‚तोऽस्या}‚र्थ‚स्येय‚म‚{\color{DodgerBlue3}‚धिग‚तिरिति} क्रियाया अधि‚{\tiny $_{lb}$}‚ग‚तेः ‚{\color{DodgerBlue3}‚क‚र्म‚णि} वेद्ये ‚{\color{DodgerBlue3}‚निय‚मः सा क्रिया त‚त्प्र‚साध‚ना} त‚त्क‚र‚णाऽभ्युप‚ग‚न्त‚व्या । ‚{\tiny $_{lb}$}‚(३०४)
	\pend% ending standard par
      \label{div_pvv.2.305}
	  
	% new div opening: depth here is 2
	

	  \pstart \leavevmode% starting standard par
	स्यादेत‚द् (।) इन्द्रियादिरेव स्व‚भेदाद् भेद‚{\tiny $_{4}$}‚को ज्ञान‚स्य प्र‚तिविष‚य‚म‚धिग‚ते‚{\tiny $_{lb}$}‚र्नियाम‚कः । त‚त‚श्चानुभ‚वात्म‚त्वाद‚विष‚य एवासिद्ध इत्याह (।)
	\pend% ending standard par
      
	  \bigskip
	  \begingroup
	
	    \large
	  
	    \begin{quote}
	  
	    
	    \stanza[\smallbreak]
	\label{pv.2.305}\flagstanza{\tiny\textenglish{....2.305}}अर्थेन घ‚ट‚य‚त्येनां न हि मुक्त्‏वार्थ‚रूप‚ताम् ।&अन्यः स्व‚भेदाज्ज्ञान‚स्य भेद‚कोपि क‚थ‚ञ्च‚न ॥ ३०५ ॥\&[\smallbreak]


	
	    \end{quote}
	  
	  \endgroup
	\textsuperscript{\textenglish{210/s}}

	  \pstart \leavevmode% starting standard par
	\hphantom{.}‚{\color{DodgerBlue3}‚एनाम}‚धिग‚तिम‚{\color{DodgerBlue3}‚र्थ‚रूप‚ता}‚म‚र्थ‚स‚रू\edtext{}{\edlabel{pvv.210-1}\label{pvv.210-1}\lemma{रू}\Bfootnote{अत्रैव तादात्म्यात् ।}} प‚तां ‚{\color{DodgerBlue3}‚मुक्त्वा न} ह्य‚न्यः क‚श्चिन्द्रियादिः ‚{\color{DodgerBlue3}‚स्व‚भेदात् ‚{\tiny $_{lb}$}‚क‚थ‚ञ्च‚न} केनापि प्र‚कारेण ‚{\color{DodgerBlue3}‚ज्ञान‚स्य भेद\edtext{}{\edlabel{pvv.210-2}\label{pvv.210-2}\lemma{भेद}\Bfootnote{इन्द्रिय‚माविलं ज्ञान‚ञ्च त‚था ।}}कोप्य‚र्थेव} ज्ञेयेन ‚{\color{DodgerBlue3}‚घ‚ट‚य‚ति} योज‚य‚ति नील‚स्ये‚{\tiny $_{lb}$}‚य‚म‚धिग‚तिः पीत‚स्य चेय‚मित्यादि । त‚था हि य‚द्य‚पि प्र‚त्य‚र्थं प्र‚तीन्द्रिय‚ञ्च ‚{\tiny $_{lb}$}‚ज्ञानानाम‚स्ति भेदः त‚थापि विष‚य‚सारूप्याभावे स एव विशेषोऽश‚{\tiny $_{lb}$}‚क्य‚निर्दे‚{\tiny $_{5}$}‚शः । अथ विशेषोप्य‚धिग‚तेरेव व्य‚व‚स्थाप्य‚त इति चेत् । न‚नु त‚स्या ‚{\tiny $_{lb}$}‚एव\edtext{}{\edlabel{pvv.210-3}\label{pvv.210-3}\lemma{एव}\Bfootnote{अधिग‚तेरेव व्य‚व‚स्थाप‚कं सारूप्य‚स्य ।}} व्य‚व‚स्थाप‚क‚मिष्य‚ते । न चाव्य‚स्थिते व्य‚व‚स्थाप‚के व्य‚व‚स्थाप्य‚{\tiny $_{lb}$}‚सिद्धिः । न‚नु न स‚र्व‚त्र प‚र‚शुव्यापार‚दृष्टिपूर्व्विकाच्छिदासिद्धिः । छिदाद‚र्श‚नाद‚पि ‚{\tiny $_{lb}$}‚प‚र‚शुव्यापार‚व्य‚व‚स्थितेः । एव‚मिहाप्य‚धिग‚तिपूर्व्विकायां प्र‚माण‚स्थितौ न दोषः । ‚{\tiny $_{lb}$}‚अस‚मान‚मेत‚त् । त‚था हि प‚र‚शुच्छिद‚योः कार्य‚कार‚ण‚भाव‚विशेषः क्रियाक‚र‚ण‚भावः । ‚{\tiny $_{lb}$}‚अधिग‚तिज्ञानात्म‚भूत‚विशेष\edtext{}{\edlabel{pvv.210-4}\label{pvv.210-4}\lemma{विशेष}\Bfootnote{सारूप्यं ।}}योस्तु व्य‚व‚स्थाप्य‚व्य‚व‚स्थाक‚भाव एषित‚व्यः । उभ‚यो‚{\tiny $_{lb}$}‚र‚पि ज्ञान‚स्व‚रूपा‚{\tiny $_{6}$}‚त्म‚त्वात् । त‚त्र कार‚ण‚म‚ज्ञात‚म‚पि स्व‚कार्यं निर्व‚र्त‚य‚तीति कार्य‚द‚र्श--‚{\tiny $_{lb}$}‚‚{\color{DodgerBlue3}‚नाच्च त‚द्‏व्य‚व‚स्था} युक्तैव । व्य‚व‚स्थाप‚क‚स्तु नानुप‚ल‚क्षितो व्य‚व‚स्थाप्य व्य‚व‚स्थायां ‚{\tiny $_{lb}$}‚क्ष‚म‚ते । न ह्य‚प्र‚तीतं क‚ल्प्य‚मानं आश्व‚स्त्यं\edtext{}{\edlabel{pvv.210-5}\label{pvv.210-5}\lemma{स्त्यं}\Bfootnote{आश्व‚त‚स्त‚स्य भावः ।}}व्य‚व‚स्थाप‚य‚ति । (३०५)
	\pend% ending standard par
      \label{div_pvv.2.306}
	  
	% new div opening: depth here is 2
	
	  \bigskip
	  \begingroup
	
	    \large
	  
	    \begin{quote}
	  
	    
	    \stanza[\smallbreak]
	\label{pv.2.306}\flagstanza{\tiny\textenglish{....2.306}}त‚स्मात्प्र‚मेयाधिग‚तेः साध‚नं मेय‚रूप‚ता ।&साध‚नेऽन्य‚त्र त‚त्क‚र्म‚स‚म्ब‚न्धो न प्र‚सिध्य‚ति ॥ ३०६ ॥\&[\smallbreak]


	
	    \end{quote}
	  
	  \endgroup
	

	  \pstart \leavevmode% starting standard par
	\hphantom{.}‚{\color{DodgerBlue3}‚त‚स्मात्प्र‚मेयाधिग‚तेः} फ‚ल‚भूतायाः व्य‚व‚स्थाप्यायाः ‚{\color{DodgerBlue3}‚साध‚नं} प्र‚माणं ‚{\color{DodgerBlue3}‚मेय‚रूप‚ता} । ‚{\tiny $_{lb}$}‚\leavevmode\ledsidenote{\textenglish{41b/MA}} अथ न सारूप्यं त‚स्य प्र‚तिविष‚यं भिन्न‚स्य सूप‚ल‚क्ष‚ण‚त्वात् सारूप्यात्पुन‚र‚{\tiny $_{7}$}‚ ‚{\color{DodgerBlue3}‚न्य‚त्र साध‚ने ‚{\tiny $_{lb}$}‚त‚स्याः} क्रियायाः ‚{\color{DodgerBlue3}‚क‚र्म्म‚स‚म्ब‚न्धो} नील‚स्येय‚म‚धिग‚तिः पीत‚स्य चेत्यादि ‚{\color{DodgerBlue3}‚न सिध्य‚ति} । ‚{\tiny $_{lb}$}‚इन्द्रियाधिग‚तिविशेष‚स्य स‚म्भ‚वेप्य‚नुभ‚व‚मात्रात्म‚क‚ज्ञान‚स्याविशेष‚क‚त्वायोगात् । ‚{\tiny $_{lb}$}‚ज्ञान‚ग‚त‚स्याप‚र‚विशेष‚स्य ल‚क्ष‚ण‚भेदेनानुप‚ल‚क्ष‚णात् । (३०६)
	\pend% ending standard par
      \label{div_pvv.2.307}
	  
	% new div opening: depth here is 2
	
	  \bigskip
	  \begingroup
	
	    \large
	  
	    \begin{quote}
	  
	    
	    \stanza[\smallbreak]
	\label{pv.2.307a}\flagstanza{\tiny\textenglish{...2.307a}}सा च त‚स्यात्म‚भूतैव तेन नार्थान्त‚रं फ‚ल‚म् ।\&[\smallbreak]


	
	    \end{quote}
	  
	  \endgroup
	

	  \pstart \leavevmode% starting standard par
	\hphantom{.}‚{\color{DodgerBlue3}‚सा चा}‚धिग‚तिर‚नुभ‚व‚स्व‚भावा ज्ञान‚स्या‚{\color{DodgerBlue3}‚त्म‚भूतैव । तेन} प्र‚माणान्ना‚{\color{DodgerBlue3}‚र्थान्त‚रं फ‚लं} । ‚{\tiny $_{lb}$}‚प्र‚मेय‚मेव फ‚ल‚मित्य‚र्थः ।
	\pend% ending standard par
      

	  \pstart \leavevmode% starting standard par
	ग्राह्य‚ग्राह‚क‚भावोपि भाक्त एवेति द‚र्श‚यितुमाह (।)
	\pend% ending standard par
      
	  \bigskip
	  \begingroup
	
	    \large
	  
	    \begin{quote}
	  
	    
	    \stanza[\smallbreak]
	\label{pv.2.307b}\flagstanza{\tiny\textenglish{...2.307b}}द‚धानं त‚च्च तामात्म‚न्य‚र्थाधिग‚म‚नात्म‚ना ॥ ३०७ ॥\&[\smallbreak]


	
	    \end{quote}
	  
	  \endgroup
	\textsuperscript{\textenglish{211/s}}

	  \pstart \leavevmode% starting standard par
	\hphantom{.}‚{\color{DodgerBlue3}‚त‚च्च} ज्ञान‚मा‚{\color{DodgerBlue3}‚त्म‚नि} ताम‚र्थ‚{\tiny $_{1}$}‚‚{\color{DodgerBlue3}‚स‚रूप‚तां द‚धानं} बिभ्र‚द् अर्थ‚स्या‚{\color{DodgerBlue3}‚धिग\edtext{}{\edlabel{pvv.211-1}\label{pvv.211-1}\lemma{धिग}\Bfootnote{य‚दाकार‚मुत्प‚द्य‚ते त‚त्र प्र‚माण‚मुप‚च‚र्य‚ते ।}} म‚नात्म‚ना}‚{\tiny $_{lb}$}‚ऽधिग‚म‚ल‚क्ष‚णेन (३०७)
	\pend% ending standard par
      \label{div_pvv.2.308}
	  
	% new div opening: depth here is 2
	
	  \bigskip
	  \begingroup
	
	    \large
	  
	    \begin{quote}
	  
	    
	    \stanza[\smallbreak]
	\label{pv.2.308}\flagstanza{\tiny\textenglish{....2.308}}स‚व्यापार‚मिवाभाति व्यापारेण स्व‚क‚र्म‚णि ।&त‚द्व‚शात्त‚द्व्य‚व‚स्थानाद‚कार‚क‚म‚पि स्व‚य‚म् ॥ ३०८ ॥\&[\smallbreak]


	
	    \end{quote}
	  
	  \endgroup
	

	  \pstart \leavevmode% starting standard par
	\hphantom{.}व्यापारेण स‚व्यापार‚{\color{DodgerBlue3}‚मिव} स्व‚क‚र्म्म‚णि ग्राह्ये ‚{\color{DodgerBlue3}‚आभाति\edtext{}{\edlabel{pvv.211-2}\label{pvv.211-2}\lemma{आभाति}\Bfootnote{विना हि क्रियां न क‚र‚णं । अथैत‚त् क्रियाव्याप्यं क‚र्म वेति विष‚याधिग‚तिः ‚{\tiny $_{lb}$}‚क‚र्म‚व्यापारः प्र‚तिप‚त्तृध‚र्म‚त्वाद‚स्या एत‚या हि विष‚यो व्याप्य‚ते । व्य‚व‚साय‚व‚शाच्च ‚{\tiny $_{lb}$}‚क‚र्तृग‚तापि विष‚य‚ग‚ता व्य‚व‚स्थाप्य‚ते य‚थाध्य‚व‚सायं स‚र्व‚व्य‚व‚हार‚वृत्तेः (।) त‚त्र च ‚{\tiny $_{lb}$}‚साध‚क‚त‚म‚स्व‚व्यापारेण त‚दानुकूल्योत्प‚त्त्या अर्थ‚सारूप्यं क‚र‚ण‚मिति क‚र्त्ता क‚र्म ‚{\tiny $_{lb}$}‚क‚र‚णं क्रिया चेति च‚तुष्ट‚य‚मुक्तं ज्ञेयं ।}}} । व्यापार‚म‚पि व‚स्तुतो‚{\tiny $_{lb}$}‚ऽकार‚क‚म‚पि स्व‚यं त‚द्व‚शान्मेय‚सारूप्य‚व‚शात्त‚स्याधिग‚म‚स्य ‚{\color{DodgerBlue3}‚व्य‚व‚स्थानात्} । (३०८)
	\pend% ending standard par
      \label{div_pvv.2.309}
	  
	% new div opening: depth here is 2
	
	  \bigskip
	  \begingroup
	
	    \large
	  
	    \begin{quote}
	  
	    
	    \stanza[\smallbreak]
	\label{pv.2.309}\flagstanza{\tiny\textenglish{....2.309}}य‚था फ‚ल‚स्य हेतूनां स‚दृशात्म‚त‚योद्भ‚वाद् ।&हेतुरूप‚ग्र‚हो लोकेऽक्रियाव‚त्वेपि क‚थ्य‚ते ॥ ३०९ ॥\&[\smallbreak]


	
	    \end{quote}
	  
	  \endgroup
	

	  \pstart \leavevmode% starting standard par
	\hphantom{.}‚{\color{DodgerBlue3}‚य‚था लोकेपि हेतूनां स‚दृशात्म‚त‚या} स‚दृश‚रूप‚त‚यो‚{\color{DodgerBlue3}‚द्भ‚वात् । फ‚ल‚स्याक्रियाव‚त्त्वेपि} हेतुरूप‚ग्र‚ह‚ण‚व्यापाराभावेपि ‚{\color{DodgerBlue3}‚हेतुरूप‚ग्र‚हः क‚थ्य‚ते} पितू रूपं गृहीतं सुतेनेत्यादि । ‚{\tiny $_{lb}$}‚अतोऽ‚{\tiny $_{2}$}‚र्थ‚रूप‚तां मुक्त्वाऽधिग‚तिसाध‚न‚म‚न्य‚द‚युक्तं । (३०९)
	\pend% ending standard par
      \label{div_pvv.2.310}
	  
	% new div opening: depth here is 2
	
	  \bigskip
	  \begingroup
	
	    \large
	  
	    \begin{quote}
	  
	    
	    \stanza[\smallbreak]
	\label{pv.2.310}\flagstanza{\tiny\textenglish{....2.310}}आलोच‚नाक्ष‚स‚म्ब‚न्ध‚विशेष‚ण‚धियाम‚तः ।&नेष्ट‚म्प्रामाण्य‚मेतेषां व्य‚व‚धानात् क्रियाम्प्र‚ति ॥ ३१० ॥\&[\smallbreak]


	
	    \end{quote}
	  
	  \endgroup
	

	  \pstart \leavevmode% starting standard par
	\hphantom{.}अत आ‚{\color{DodgerBlue3}‚लोच‚न}‚स्यार्थालोच‚न‚मा\edtext{}{\edlabel{pvv.211-3}\label{pvv.211-3}\lemma{मा}\Bfootnote{प्र‚थ‚मं ज‚ग‚ति किम‚प्येत‚दित्यालोच‚नं । नागृहीत‚विशेष‚ण विशेष्ये धीः ‚{\tiny $_{lb}$}‚(।) विशेष‚णं जातिगुण‚क्रियाद‚यः ।}}त्र‚स्य जात्यादिविशिष्ट‚निश्च‚य‚फ‚लं प्र‚ति । ‚{\tiny $_{lb}$}‚‚{\color{DodgerBlue3}‚अक्ष‚स‚म्ब‚न्ध‚स्य} इन्द्रियार्थ‚स‚न्निक‚र्ष‚स्यालोच‚न‚फ‚लं प्र‚ति ‚{\color{DodgerBlue3}‚विशेष‚ण}‚(स्य) वेद्य‚विशेष्य‚बुद्धि‚{\tiny $_{lb}$}‚फ‚लं प्र‚ति ‚{\color{DodgerBlue3}‚प्रामाण्यं} साध‚न‚त्वं ‚{\color{DodgerBlue3}‚नेष्टं । एतेषा}‚मालोच‚नार्थ‚स‚म्ब‚न्ध‚विशेष‚ण‚ज्ञानानां ‚{\tiny $_{lb}$}‚‚{\color{DodgerBlue3}‚क्रिया}‚याम‚धिग‚तिविष‚य‚सारूप्येण ‚{\color{DodgerBlue3}‚व्य‚व‚धानात्} । (३१०)
	\pend% ending standard par
      \label{div_pvv.2.311}
	  
	% new div opening: depth here is 2
	

	  \pstart \leavevmode% starting standard par
	व्य‚व‚स्थानेपि साध‚क‚त‚म‚त्वं स्यादिति चेत् । आह ।
	\pend% ending standard par
      
	  \bigskip
	  \begingroup
	
	    \large
	  
	    \begin{quote}
	  
	    
	    \stanza[\smallbreak]
	\label{pv.2.311}\flagstanza{\tiny\textenglish{....2.311}}स‚र्वेषामुप‚योगेपि कार‚काणां क्रियाम्प्र‚ति ।&य‚द‚न्त्यं भेद‚कं त‚स्यास्त‚त् साध‚क‚त‚मं म‚त‚म् ॥ ३११ ॥\&[\smallbreak]


	
	    \end{quote}
	  
	  \endgroup
	\textsuperscript{\textenglish{212/s}}

	  \pstart \leavevmode% starting standard par
	\hphantom{.}‚{\color{DodgerBlue3}‚स‚र्व्वेषां‚{\tiny $_{3}$}‚ कार‚काणां} साक्षात्पार‚म्प‚र्येण ‚{\color{DodgerBlue3}‚क्रियां प्र‚त्युप‚योगेपि} तेषु म‚ध्ये ‚{\color{DodgerBlue3}‚य‚त्कार‚क‚{\tiny $_{lb}$}‚म‚न्त्यं} कार‚कान्त‚रेणाव्य‚व‚हित‚व्याप्यारं स‚त् क्रिया‚{\color{DodgerBlue3}‚भेद‚कं त‚त्त‚स्याः साध‚क‚त‚मं म‚तं} नान्य‚त् । (३११)
	\pend% ending standard par
      \label{div_pvv.2.312}
	  
	% new div opening: depth here is 2
	

	  \pstart \leavevmode% starting standard par
	त‚था हि (।)
	\pend% ending standard par
      
	  \bigskip
	  \begingroup
	
	    \large
	  
	    \begin{quote}
	  
	    
	    \stanza[\smallbreak]
	\label{pv.2.312}\flagstanza{\tiny\textenglish{....2.312}}स‚र्व-सामान्य‚हेतुत्वाद‚क्षाणाम‚स्ति नेदृश‚म् ।&त‚द्भेदेपि ह्य‚त‚द्रूप‚स्यास्येद‚मिति त‚त्कुतः ॥ ३१२ ॥\&[\smallbreak]


	
	    \end{quote}
	  
	  \endgroup
	

	  \pstart \leavevmode% starting standard par
	\hphantom{.}‚{\color{DodgerBlue3}‚अक्षाणां} ताव‚दी‚{\color{DodgerBlue3}‚दृशं} साध‚क‚त‚म‚त्वं ‚{\color{DodgerBlue3}‚नास्ति स‚र्व्वं---सामान्य‚हेतुत्वात्} । ‚{\tiny $_{lb}$}‚\edtext{\textsuperscript{*}}{\edlabel{pvv.212-1}\label{pvv.212-1}\lemma{*}\Bfootnote{नीलादेः ।}}स‚र्व्व‚ज्ञान‚साधार‚ण‚हेतुत्वात् । अवान्त‚राधिग‚तिभेद‚क‚त्वानुप‚प‚त्तेः । तेषामिन्द्रियाणां ‚{\tiny $_{lb}$}‚प्र‚मादाविल‚त्वादि‚{\color{DodgerBlue3}‚भेदेपि} ज्ञान‚स्यात‚द्रूप‚स्य विष‚य‚सारू‚{\tiny $_{4}$}‚प्य‚र‚हित‚स्य इद‚म‚स्य ‚{\tiny $_{lb}$}‚ग्राह‚क‚मिति ग्राह‚क‚त्वं य‚दिष्य‚ते ‚{\color{DodgerBlue3}‚त‚त्कुतः} । (३१२)
	\pend% ending standard par
      \label{div_pvv.2.313}
	  
	% new div opening: depth here is 2
	
	  \bigskip
	  \begingroup
	
	    \large
	  
	    \begin{quote}
	  
	    
	    \stanza[\smallbreak]
	\label{pv.2.313}\flagstanza{\tiny\textenglish{....2.313}}एतेन शेषं व्याख्यातं विशेष‚ण‚धियां पुनः ।&अताद्रुप्ये न भेदोपि त‚द्व‚द‚न्य‚धियोपि वा ॥ ३१३ ॥\&[\smallbreak]


	
	    \end{quote}
	  
	  \endgroup
	

	  \pstart \leavevmode% starting standard par
	\hphantom{.}‚{\color{DodgerBlue3}‚एते}‚नाक्षाणाम‚साध‚न‚त्व‚द‚र्श‚नेन ‚{\color{DodgerBlue3}‚शेष}‚मालोच‚नार्थ‚स‚म्ब‚न्धादि ‚{\color{DodgerBlue3}‚व्याख्यातं} । त‚द‚प्य‚र्थ‚{\tiny $_{lb}$}‚सारूप्य‚र‚हित‚म‚साध‚नं ।\edtext{\textsuperscript{*}}{\edlabel{pvv.212-2}\label{pvv.212-2}\lemma{*}\Bfootnote{अस्येद‚मित्यालोच‚न‚मित्य‚सिद्धेः ।}} सारूप्ये च स्वीक‚र्त‚व्ये त‚देवाव्य‚व‚हित‚त्वात् साध‚न‚म‚स्तु । ‚{\tiny $_{lb}$}‚‚{\color{DodgerBlue3}‚विशेष‚ण‚धियां पुन‚र}‚य‚म‚धिको दोषः । ‚{\color{DodgerBlue3}‚अताद्रूप्ये} विशेष‚ण‚सारूप्याभावे विशेष्य‚धियः ‚{\tiny $_{lb}$}‚स‚काशा‚{\color{DodgerBlue3}‚द्भेदोपि} न स्यात् । द्व‚योर‚प्याकार‚र‚हित‚त्वेन भेद‚स्थित्य‚नुप‚प‚{\tiny $_{5}$}‚त्तेः । त‚त‚श्चैका ‚{\tiny $_{lb}$}‚प्र‚माण‚म‚न्या च फ‚ल‚मिति कुतः । अथ सारूप्यं विशेष‚ण‚धियो भेद‚क‚मिष्य‚ते । एवं च ‚{\tiny $_{lb}$}‚स‚ति त‚देव प्र‚माणं फ‚लं च स्यात् । अर्थ‚स‚रूप‚त्वाद‚धिग‚तिरूप‚त्वाच्च । ‚{\color{DodgerBlue3}‚त‚द्व‚द्} विशेष‚ण‚{\tiny $_{lb}$}‚बुद्धेरिव ‚{\color{DodgerBlue3}‚अन्य\edtext{}{\edlabel{pvv.212-3}\label{pvv.212-3}\lemma{अन्य}\Bfootnote{विशेष्य‚बुद्धेः ।}}} बुद्धेर्विष‚य‚सारूप्यं फ‚लात्म‚क‚ञ्चेष्य‚तां । (३१३)
	\pend% ending standard par
      \label{div_pvv.2.314}
	  
	% new div opening: depth here is 2
	

	  \pstart \leavevmode% starting standard par
	किञ्च (।)
	\pend% ending standard par
      
	  \bigskip
	  \begingroup
	
	    \large
	  
	    \begin{quote}
	  
	    
	    \stanza[\smallbreak]
	\label{pv.2.314}\flagstanza{\tiny\textenglish{....2.314}}नेष्टो विष‚य‚भेदोपि क्रियासाध‚न‚योर्द्व‚योः ।&एकार्थ‚त्वे द्व‚यं व्य‚र्थं न च स्यात् क्र‚म‚भाविता ॥ ३१४ ॥\&[\smallbreak]


	
	    \end{quote}
	  
	  \endgroup
	

	  \pstart \leavevmode% starting standard par
	\hphantom{.}‚{\color{DodgerBlue3}‚क्रियासाध‚न‚योर्विष‚य‚भेदोपि नेष्टः} स‚र्व्व‚स्य । न ह्य‚न्य\edtext{}{\edlabel{pvv.212-4}\label{pvv.212-4}\lemma{न्य}\Bfootnote{द्वे ज्ञाने विशेष‚ण‚विशेष्य‚भिन्न‚विष‚ये इति विशेष‚णे त‚द‚धिग‚म‚रूप‚त‚या ‚{\tiny $_{lb}$}‚व्यापृतं विशेष्ये ।}}त्र प‚र‚शुव्यापार‚श्छिदा ‚{\tiny $_{lb}$}‚चान्य‚त्र । इह तु विशेष‚णे प्र‚माण‚व्यापारः क्रिया च विशेष्य इति भि‚{\tiny $_{6}$}‚न्न‚विष‚य‚ता ‚{\tiny $_{lb}$}‚क‚थ‚मिष्टा ।
	\pend% ending standard par
      \textsuperscript{\textenglish{213/s}}

	  \pstart \leavevmode% starting standard par
	\hphantom{.}अथैत‚द्दोष‚त‚या ‚{\color{DodgerBlue3}‚द्व‚योः} क्रियासाध‚न‚योः क्र‚म‚भाविनो‚{\color{DodgerBlue3}‚रेकार्थ‚त्वे} एक‚विष‚य‚त्वे ‚{\tiny $_{lb}$}‚च स्वीक्रिय‚माणे ‚{\color{DodgerBlue3}‚द्व‚यं} भिन्नं प्र‚माणं फ‚लं च ‚{\color{DodgerBlue3}‚व्य‚र्थं} विशेष‚ण‚ज्ञानं प्र‚माणं फ‚लं चास्तु ‚{\tiny $_{lb}$}‚विष‚य‚स्व‚रूप‚त्वाद‚धिग‚म‚स्व‚भावाच्च । त‚तः प‚रं तु विशेष्य‚ज्ञान‚म‚धिग‚ताधिग‚न्तृत्वाद‚{\tiny $_{lb}$}‚नुप‚युक्तं । ‚{\color{DodgerBlue3}‚न चै}‚क‚विष‚य‚योर्ज्ञान‚योः ‚{\color{DodgerBlue3}‚क्र‚म‚भाविता}‚स्ति विष‚य‚स्य त‚ज्ज‚न‚न‚श‚क्त‚स्य ‚{\tiny $_{lb}$}‚क्र‚मेण स्व‚कार्य‚ज‚न‚न‚विरोधात् । (३१४)
	\pend% ending standard par
      \label{div_pvv.2.315}
	  
	% new div opening: depth here is 2
	

	  \pstart \leavevmode% starting standard par
	स‚कृदुत्प‚त्ता‚{\tiny $_{7}$}‚वेव विशेष‚ण‚विशेष्य‚धियोः प्र‚माण‚फ‚ल‚ता भ‚विष्य‚तीति चेत् ।\leavevmode\ledsidenote{\textenglish{42a/MA}} ‚{\tiny $_{lb}$}‚आह (।)
	\pend% ending standard par
      
	  \bigskip
	  \begingroup
	
	    \large
	  
	    \begin{quote}
	  
	    
	    \stanza[\smallbreak]
	\label{pv.2.315a}\flagstanza{\tiny\textenglish{...2.315a}}साध्य‚साध‚न‚ताभावः स‚कृद्भावे;\&[\smallbreak]


	
	    \end{quote}
	  
	  \endgroup
	

	  \pstart \leavevmode% starting standard par
	\hphantom{.}‚{\color{DodgerBlue3}‚स‚कृद् भावे साध्य‚साध‚न‚तायाः अभावः} । कार्य‚कार‚ण‚भाव‚विशेष‚त्वेन त‚स्या ‚{\tiny $_{lb}$}‚इ\edtext{}{\edlabel{pvv.213-1}\label{pvv.213-1}\lemma{इ}\Bfootnote{प्र‚माणाद‚न्य‚फ‚ल‚वादिना ।}}ष्ट‚त्वात् ।
	\pend% ending standard par
      

	  \pstart \leavevmode% starting standard par
	न‚न्वेव‚माकाराधिग‚म‚योरेक‚ज्ञानात्म‚त्वेपि प्र‚माण‚फ‚ल‚ताऽनुप‚प‚न्नेत्याह (।)
	\pend% ending standard par
      
	  \bigskip
	  \begingroup
	
	    \large
	  
	    \begin{quote}
	  
	    
	    \stanza[\smallbreak]
	\label{pv.2.315b}\flagstanza{\tiny\textenglish{...2.315b}}धियोंश‚योः ।&त‚द्‏व्य‚व‚स्थाश्र‚य‚त्वेन साध्य‚साध‚न‚संस्थितिः ॥ ३१५ ॥\&[\smallbreak]


	
	    \end{quote}
	  
	  \endgroup
	

	  \pstart \leavevmode% starting standard par
	\hphantom{.}‚{\color{DodgerBlue3}‚धियोऽशंयो}‚राकाराधिग‚म‚ल‚क्ष‚ण‚योः ‚{\color{DodgerBlue3}‚साध्य‚साध‚न‚संस्थितिः} क्रियाक‚र‚ण‚व्य‚{\tiny $_{lb}$}‚व‚स्था । ‚{\color{DodgerBlue3}‚त‚द्‏व्य‚व‚स्थाश्र‚य‚त्वे}‚नाकार‚व‚शेनाधिग‚तिविशेष‚व्य‚व‚स्थानात् । नास्त्य‚त्र ‚{\tiny $_{lb}$}‚कार्य‚कार‚णात्म‚कः क्रियाक‚र‚ण‚भावः किं तु‚{\tiny $_{1}$}‚ व्य‚व‚स्थाप्य‚व्य‚व‚स्थाप‚क‚भावः । स च ‚{\tiny $_{lb}$}‚तादात्म्येप्य‚विरुद्धः । (३१५)
	\pend% ending standard par
      \label{div_pvv.2.316}
	  
	% new div opening: depth here is 2
	

	  \pstart \leavevmode% starting standard par
	अर्थ‚स‚न्निक‚र्षोपि न प्र‚माण‚मित्याह (।)
	\pend% ending standard par
      
	  \bigskip
	  \begingroup
	
	    \large
	  
	    \begin{quote}
	  
	    
	    \stanza[\smallbreak]
	\label{pv.2.316}\flagstanza{\tiny\textenglish{....2.316}}स‚र्वात्म‚नापि स‚म्ब‚द्धं कैश्चिदेवाव‚ग‚म्य‚ते ।&ध‚र्मैः स निय‚मो न स्यात् स‚म्ब‚न्ध‚स्याविशेष‚तः ॥ ३१६ ॥\&[\smallbreak]


	
	    \end{quote}
	  
	  \endgroup
	

	  \pstart \leavevmode% starting standard par
	\hphantom{.}बाह्यं ‚{\color{DodgerBlue3}‚स‚र्व्वात्म‚ना} स‚र्व्वैराकारैरिन्द्रियादिभिः ‚{\color{DodgerBlue3}‚स‚म्ब‚द्ध‚म‚पि कैश्चिदेय ध‚र्म्मै}‚र्नील‚{\tiny $_{lb}$}‚त्वादिभि‚{\color{DodgerBlue3}‚र्ग्ग‚म्य‚ते} न त्प‚णुपुञ्ज‚त्वादिभिः\edtext{}{\edlabel{pvv.213-2}\label{pvv.213-2}\lemma{त्वादिभिः}\Bfootnote{विज्ञान‚मेव हि विष‚य‚प्र‚तिभास‚मुत्प‚द्य‚मानं ग्राह्य‚ग्राह‚क‚भेद‚व‚दुप‚ल‚क्ष्य‚ते तेन ‚{\tiny $_{lb}$}‚बाह्योर्थोस्तीति नार्थाधिग‚तिरूपा काचित् क्रियास्ति, या प्र‚माण‚फ‚ल‚त्वेन ‚{\tiny $_{lb}$}‚व्य‚व‚स्थाप्य‚त इति वृत्तिः । प्राप्य‚कारीन्द्रिय‚प‚क्षे स‚र्व्वात्म‚ना विष‚य‚स्प‚र्शात् स‚र्व्व‚था‚{\tiny $_{lb}$}‚व‚सायः स्यात् सारूप्ये तु याव‚न्मात्रेण प्र‚तिबिम्ब‚स्ताव‚तो ग्र‚हान्न दोषः ।}} । ‚{\color{DodgerBlue3}‚स एष} ग्र‚ह‚ण‚स्य ‚{\color{DodgerBlue3}‚निय\edtext{}{\edlabel{pvv.213-3}\label{pvv.213-3}\lemma{निय}\Bfootnote{स‚न्निक‚र्ष‚प्र‚माण‚त्वे ।}} मो न स्यात् । ‚{\tiny $_{lb}$}‚स‚म्ब}\edtext{}{\edlabel{pvv.213-4}\label{pvv.213-4}\lemma{स्य}\Bfootnote{स‚न्निक‚र्ष‚स्य ।}} ‚{\color{DodgerBlue3}‚न्ध}‚स्यापि ग‚तिस्थितिहेतोर‚{\color{DodgerBlue3}‚विशेष‚तः} । (३१६)
	\pend% ending standard par
      \label{div_pvv.2.317}
	  
	% new div opening: depth here is 2
	\textsuperscript{\textenglish{214/s}}
	  \bigskip
	  \begingroup
	
	    \large
	  
	    \begin{quote}
	  
	    
	    \stanza[\smallbreak]
	\label{pv.2.317}\flagstanza{\tiny\textenglish{....2.317}}त‚द‚भेदेपि भेदोयं य‚स्मात्त‚स्य प्र‚माण‚ता ।&संस्काराच्चेद‚ताद्रूप्ये न त‚स्याप्य‚व्य‚व‚स्थितेः ॥ ३१७ ॥\&[\smallbreak]


	
	    \end{quote}
	  
	  \endgroup
	

	  \pstart \leavevmode% starting standard par
	\hphantom{.}‚{\color{DodgerBlue3}‚य‚स्मा}‚दिन्द्रिय‚स‚न्निक‚र्षादेः प्रामाण्य‚म‚युक्तं ‚{\color{DodgerBlue3}‚त‚त्त}‚स्माद‚{\color{DodgerBlue3}‚भेदेपि} स‚न्निक‚र्षाद्य‚विशेषेपि ‚{\tiny $_{lb}$}‚य‚स्मान्नियाम‚कात् त‚ज्ज्ञान‚स्या‚{\color{DodgerBlue3}‚यं भेदो} नील‚स्येदं ज्ञानं पीत‚स्य चेद‚मित्या‚{\tiny $_{2}$}‚दि ‚{\color{DodgerBlue3}‚त‚स्य ‚{\tiny $_{lb}$}‚प्र‚माण}‚ता युक्ता । स च आकार एवेति स एव प्र‚माणं ज्ञान‚जात‚ज्ञान‚हेतोः ‚{\tiny $_{lb}$}‚‚{\color{DodgerBlue3}‚संस्कारा}‚ज्ज्ञान‚स्य कैश्चिदेव ध‚र्मैर्ग्र‚ह‚ण‚निय‚म इति चेत् ‚{\color{DodgerBlue3}‚न} (।) ‚{\color{DodgerBlue3}‚त‚स्य} संस्कार‚स्याप्य‚{\tiny $_{lb}$}‚‚{\color{DodgerBlue3}‚ताद्रूप्ये} विष‚यास‚रूप‚त्वे‚{\color{DodgerBlue3}‚ऽव्य‚व‚स्थितेः} । (३१७)
	\pend% ending standard par
      \label{div_pvv.2.318_2.319}
	  
	% new div opening: depth here is 2
	

	  \pstart \leavevmode% starting standard par
	य‚दाकारं ज्ञानं स्यात्त‚स्यानुभ‚वः संस्कार‚श्च भ‚वेत् । त‚स्माच्च ग्र‚ह‚ण‚प्र‚ति‚{\tiny $_{lb}$}‚निय‚मः । अनुभ‚व‚स्थित्य‚भावे तु स‚र्व्व‚म‚व्य‚व‚स्थितं ।
	\pend% ending standard par
      
	  \bigskip
	  \begingroup
	
	    \large
	  
	    \begin{quote}
	  
	    
	    \stanza[\smallbreak]
	\label{pv.2.318a}\flagstanza{\tiny\textenglish{...2.318a}}क्रियाक‚र‚ण‚योरैक्य‚विरोध इति चेद‚स‚त् ।&ध‚र्म‚भेदाभ्युप‚ग‚माद्;\&[\smallbreak]


	
	    \end{quote}
	  
	  \endgroup
	

	  \pstart \leavevmode% starting standard par
	\hphantom{.}‚{\color{DodgerBlue3}‚क्रियाक‚र‚ण‚यो}‚र‚धिग‚माकार‚योरैकात्म्ये विरोध ‚{\color{DodgerBlue3}‚इति चेत् । अस‚दे}‚त‚त् । ‚{\color{DodgerBlue3}‚ध‚र्म‚{\tiny $_{lb}$}‚भेद}‚स्य व्यावृत्त्युप‚क‚{\tiny $_{3}$}‚ल्पित‚स्या‚{\color{DodgerBlue3}‚भ्युप‚ग‚मात्} । अनाकार‚व्यावृत्तिः प्र‚माणं । अन‚धिग‚ति‚{\tiny $_{lb}$}‚व्यावृत्तिश्च फ‚ल‚मिति नान‚योरैक्यं ।
	\pend% ending standard par
      

	  \pstart \leavevmode% starting standard par
	\hphantom{.}क‚थ‚न्त‚र्ह्युक्तं (।) 
	    \pend% close preceding par
	  
	    \begin{quote}
	  
	    
	    \stanza[\smallbreak]
	सा च त‚स्यात्म‚भूतैव तेन नार्थान्त‚रं फ‚ल‚मि\&[\smallbreak]


	
	    \end{quote}
	  
	    \pstart  \leavevmode% new par for following
	    \hphantom{.}
	   \href{http://sarit.indology.info/?cref=pv.2.307}{(२।३०७)} ‚{\tiny $_{lb}$}‚त्याह (।)
	\pend% ending standard par
      
	  \bigskip
	  \begingroup
	
	    \large
	  
	    \begin{quote}
	  
	    
	    \stanza[\smallbreak]
	\label{pv.2.318b}\flagstanza{\tiny\textenglish{...2.318b}}व‚स्त्व‚भिन्न‚मितीष्य‚ते ॥ ३१८ ॥\&[\smallbreak]


	
	    \end{quote}
	  
	  \endgroup
	

	  \pstart \leavevmode% starting standard par
	\hphantom{.}प‚र‚मार्थ‚तो ज्ञानात्म‚कं ‚{\color{DodgerBlue3}‚व‚स्त्व‚भिन्न‚मितीष्य‚ते} । (३१८) न तु क‚ल्पित‚ध‚र्म‚द्वारेण‚पि । ‚{\tiny $_{lb}$}‚किं पुन‚स्त‚त्त्व‚त एव साध‚नाद् भिन्नाऽधिग‚तिश्छि दादिव‚न्नेष्य‚त इत्याह (।)
	\pend% ending standard par
      
	  \bigskip
	  \begingroup
	
	    \large
	  
	    \begin{quote}
	  
	    
	    \stanza[\smallbreak]
	\label{pv.2.319}\flagstanza{\tiny\textenglish{....2.319}}एवं प्र‚कारा स‚र्व्वैव क्रियाकार‚क‚संस्थितिः ।&भाव‚स्य भिन्नाभिम‚तेष्व‚प्यारोपेण वृत्तितः ॥ ३१९ ॥\&[\smallbreak]


	
	    \end{quote}
	  
	  \endgroup
	

	  \pstart \leavevmode% starting standard par
	\hphantom{.}‚{\color{DodgerBlue3}‚स‚र्वैव क्रियाकार‚क}‚योः ‚{\color{DodgerBlue3}‚संस्थिति}‚र्व्य‚व‚स्था ‚{\color{DodgerBlue3}‚एवं-प्र‚कारा} क‚ल्पितैव ‚{\color{DodgerBlue3}‚भिन्नाभि‚{\tiny $_{lb}$}‚म‚तेष्व‚पि} दारुप‚र‚श्वादिषु क्रियाक‚र‚ण‚भाव‚{\tiny $_{4}$}‚स्या‚{\color{DodgerBlue3}‚रोपेण वृत्तितः} । न हि त‚त्रापि कार्य‚{\tiny $_{lb}$}‚कार‚ण‚व‚स्तुद्व‚य‚व्य‚तिरिक्ता क्रियास्ति । द्विधाभूतं काष्ठ‚मेवात‚द्‏व्यावृत्त्या भेदान्त‚र‚{\tiny $_{lb}$}‚प्र‚तिक्षेपेण छिदेत्युच्य‚ते । त‚त्कार‚णेषु च पुरुष‚क‚र‚प‚र‚श्वादिषु साम‚ग्र्‏य‚न्त‚र‚व‚र्त्तिनः ‚{\tiny $_{lb}$}‚क‚राच्छिदायाः अनुत्प‚त्तेः प‚र‚शोर‚साधार‚णं स‚ह‚कारित्व‚मुप‚द‚र्श‚यितुम‚त‚द्‏व्यावृत्त्या ‚{\tiny $_{lb}$}‚क‚र‚ण‚व्य‚प‚देशः । न तु क्रियाक‚र‚ण‚त्व‚म‚न्य‚देव कार्य‚क‚र‚णाभ्यां । अधिग‚माकार‚{\tiny $_{lb}$}‚योस्तु व्य‚व‚स्थाप्य‚व्य‚व‚स्थाप‚क‚भावः क्रियाक‚र‚ण‚{\tiny $_{5}$}‚भावः (।) य‚था यः क‚म्प‚ते सोऽश्व ‚{\tiny $_{lb}$}‚इति (।) उक्ता प्र‚माण‚फ‚ल‚व्य‚व‚स्था ॥ X X ॥ (३१९)
	\pend% ending standard par
      
	  
	% new div opening: depth here is 1
	
\chapter*[{९. विक्ष‚प्तिमात्र‚ताचिन्ता}]{९. विक्ष‚प्तिमात्र‚ताचिन्ता}

	  \begin{center}%% label @type='head'
	\textbf{(१) अर्थ‚संवेद‚न‚चिन्ता}
	\end{center}
	

	  \begin{center}%% label @type='head'
	\textbf{क. अर्थ‚संविद्}
	\end{center}
	\label{div_pvv.2.320}
	  
	% new div opening: depth here is 2
	

	  \pstart \leavevmode% starting standard par
	\leavevmode\ledsidenote{\textenglish{215/s}}इदानीं यो गा चा रो वेद्य‚वेद‚क‚भाव‚म‚प‚श्य‚न् सौ त्रा न्ति कं पृ\edtext{}{\edlabel{pvv.215-1}\label{pvv.215-1}\lemma{पृ}\Bfootnote{अन्त‚र्ब्ब‚हिस्तीर्थ्या दूषिता(ः।) प्र‚माद्व‚यं प्र‚मेय‚द्वैविध्यादुक्तं प्र‚त्य‚क्ष‚ल‚क्ष‚ण‚ञ्च । ‚{\tiny $_{lb}$}‚सौत्रान्तिक‚प्र‚माणं सारूप्यं बाह्योर्थः प्र‚मेयोधिग‚तिः फ‚लं व्य‚व‚स्थाप्याधुना विज्ञ‚प्तौ ‚{\tiny $_{lb}$}‚प्र‚माण‚फ‚ल‚व्य‚व‚स्थां निर्दिदिक्षुः स्व‚स‚म्वित्तिः फ‚ल‚ञ्चात्र त‚द्रूपो ह्य‚र्थ‚निश्च‚य इति ‚{\tiny $_{lb}$}‚स‚म्वित्तिं व्याख्यातुमाह सौत्रान्तिकं ।}} च्छ‚ति ।
	\pend% ending standard par
      
	  \bigskip
	  \begingroup
	
	    \large
	  
	    \begin{quote}
	  
	    
	    \stanza[\smallbreak]
	\label{pv.2.320a}\flagstanza{\tiny\textenglish{...2.320a}}कार्थ‚संविद्;\&[\smallbreak]


	
	    \end{quote}
	  
	  \endgroup
	

	  \pstart \leavevmode% starting standard par
	\hphantom{.}‚{\color{DodgerBlue3}‚काऽर्थ‚स्य संवित्} ज्ञान‚मुच्य‚ते ।
	\pend% ending standard par
      

	  \pstart \leavevmode% starting standard par
	अत आह (।)
	\pend% ending standard par
      
	  \bigskip
	  \begingroup
	
	    \large
	  
	    \begin{quote}
	  
	    
	    \stanza[\smallbreak]
	\label{pv.2.320b}\flagstanza{\tiny\textenglish{...2.320b}}य‚देवेदं प्र‚त्य‚क्षं प्र‚तिवेद‚न‚म् ।\&[\smallbreak]


	
	    \end{quote}
	  
	  \endgroup
	

	  \pstart \leavevmode% starting standard par
	\hphantom{.}‚{\color{DodgerBlue3}‚य‚देवेदं प्र‚त्य‚क्ष}‚म‚नुभ‚व‚सिद्धं ‚{\color{DodgerBlue3}‚प्र‚तिवेद‚नं} नीलाद्याकारेण प्र‚तिनिय‚तं वेद‚नं प्र‚ति‚{\tiny $_{lb}$}‚स‚न्तान‚निय‚तं वा सैवार्थ‚संविदुच्य‚ते ।
	\pend% ending standard par
      
	  \bigskip
	  \begingroup
	
	    \large
	  
	    \begin{quote}
	  
	    
	    \stanza[\smallbreak]
	\label{pv.2.320c}\flagstanza{\tiny\textenglish{...2.320c}}त‚द‚र्थ‚वेद‚नं केन ताद्रूप्याद् व्य‚भिचारि त‚त् ॥ ३२० ॥\&[\smallbreak]


	
	    \end{quote}
	  
	  \endgroup
	

	  \pstart \leavevmode% starting standard par
	\hphantom{.}न‚नु ‚{\color{DodgerBlue3}‚त‚त्} प्र‚तिनिय‚तं ‚{\color{DodgerBlue3}‚वेद‚न}‚म‚नुभूय‚मान‚म‚र्थंस्य वेद‚नं ‚{\color{DodgerBlue3}‚केन} हेतुनोच्य‚ते स्व‚प्र‚का‚{\tiny $_{lb}$}‚शात्म‚क‚त्वात् स्व‚{\tiny $_{6}$}‚वेद‚न‚मेव त‚द्युक्तं नार्थ‚वेद‚नं । त‚स्य स‚र्व्व‚दा प‚रोक्ष‚त्वात् । ‚{\color{DodgerBlue3}‚ताद्रू‚{\tiny $_{lb}$}‚प्या}‚द‚र्थ‚स‚रूप‚त्वात् ज्ञान‚म‚र्थ‚वेद‚न‚मिति चेत् । त‚द‚र्थ‚सारूप्यं‚{\color{DodgerBlue3}‚व्य‚भि\edtext{}{\edlabel{pvv.215-2}\label{pvv.215-2}\lemma{भि}\Bfootnote{अतिव्याप्त्या ।}} चारि}‚, द्विच‚न्द्र‚{\tiny $_{lb}$}‚केशोण्डूक‚ज्ञानाद्याकार‚स्यार्थ‚म‚न्त‚रेणापि भावात् । (३२०)
	\pend% ending standard par
      \label{div_pvv.2.321}
	  
	% new div opening: depth here is 2
	
	  \bigskip
	  \begingroup
	
	    \large
	  
	    \begin{quote}
	  
	    
	    \stanza[\smallbreak]
	\label{pv.2.321a}\flagstanza{\tiny\textenglish{...2.321a}}अथ सोनुभ‚वः क्वास्य\&[\smallbreak]


	
	    \end{quote}
	  
	  \endgroup
	

	  \pstart \leavevmode% starting standard par
	\hphantom{.}अथानुभाव्याभावे ‚{\color{DodgerBlue3}‚सोऽनुभ‚वः} प्र‚त्यात्म‚वेद्यो‚{\color{DodgerBlue3}‚ऽस्य} ज्ञान‚स्य न क‚र्म्म‚णि व्य‚व‚स्थाप‚{\tiny $_{lb}$}‚नीयः । न हि क‚र्म‚र‚हिता क्रिया ‚{\color{DodgerBlue3}‚क्व}‚चिद‚स्ति ।
	\pend% ending standard par
      

	  \pstart \leavevmode% starting standard par
	अत्राह (।)
	\pend% ending standard par
      
	  \bigskip
	  \begingroup
	
	    \large
	  
	    \begin{quote}
	  
	    
	    \stanza[\smallbreak]
	\label{pv.2.321b}\flagstanza{\tiny\textenglish{...2.321b}}त‚देवेदं विचार्य‚ते ।\&[\smallbreak]


	
	    \end{quote}
	  
	  \endgroup
	

	  \pstart \leavevmode% starting standard par
	\hphantom{.}य‚दुच्य‚ते व्य‚व‚ह‚र्तृ भिरिद‚म‚नेनानुभूय‚ते इति ‚{\color{DodgerBlue3}‚त‚देवेद}‚म‚स्माभि‚{\color{DodgerBlue3}‚र्व्विचार्य‚ते} । ज्ञानं ‚{\tiny $_{lb}$}‚स्व‚प्र‚{\tiny $_{7}$}‚काश‚मुप‚ल‚भ्य‚ते प्र‚काश‚श्चोच्य‚त इत्य‚युक्तं ।
	\pend% ending standard par
      \textsuperscript{\textenglish{42b/MA}}‚{\tiny $_{lb}$}‚\textsuperscript{\textenglish{216/s}}

	  \pstart \leavevmode% starting standard par
	य‚च्चार्थ‚सारूप्य‚म‚नुभ‚व‚निब‚न्ध‚न‚मुक्तं त‚द‚प्य‚स‚म्भ‚वीति द‚र्श‚य‚न्नाह (।)
	\pend% ending standard par
      
	  \bigskip
	  \begingroup
	
	    \large
	  
	    \begin{quote}
	  
	    
	    \stanza[\smallbreak]
	\label{pv.2.321c}\flagstanza{\tiny\textenglish{...2.321c}}स‚रूप‚य‚न्ति त‚त् केन स्थूलाभास‚ञ्च तेऽण‚वः ॥ ३२१ ॥\&[\smallbreak]


	
	    \end{quote}
	  
	  \endgroup
	

	  \pstart \leavevmode% starting standard par
	\hphantom{.}‚{\color{DodgerBlue3}‚ते} प‚र‚स्प‚रं भिन्ना ‚{\color{DodgerBlue3}‚अण‚वः} त‚ज्ज्ञानं ‚{\color{DodgerBlue3}‚स्थूलाभासं} स्थूलाकारं केन रूपेण स‚{\color{DodgerBlue3}‚रूप‚{\tiny $_{lb}$}‚य‚न्ति\edtext{}{\edlabel{pvv.216-1}\label{pvv.216-1}\lemma{न्ति}\Bfootnote{अव‚य‚वी सौन्त्रान्तिकेनैव निर‚स्तः ।}}} । य‚द‚णुस्व‚रूप‚म‚स्थूल‚म‚स्ति न त‚त् ज्ञानारूढं । य‚च्च ज्ञानारूढं स्थौल्यं नाणुषु ‚{\tiny $_{lb}$}‚त‚द‚स्ति । (३२१)
	\pend% ending standard par
      \label{div_pvv.2.322}
	  
	% new div opening: depth here is 2
	
	  \bigskip
	  \begingroup
	
	    \large
	  
	    \begin{quote}
	  
	    
	    \stanza[\smallbreak]
	\label{pv.2.322}\flagstanza{\tiny\textenglish{....2.322}}त‚न्नार्थ‚रूप‚ता त‚स्य स‚त्यांर्थाव्य‚भिचारिणी ।&त‚त्स‚म्वेद‚न‚भाव‚स्य न स‚म‚र्था प्र‚साध‚ने ॥ ३२२ ॥\&[\smallbreak]


	
	    \end{quote}
	  
	  \endgroup
	

	  \pstart \leavevmode% starting standard par
	\hphantom{.}त‚स्मात्तुल्य‚ज्ञान‚स्य ‚{\color{DodgerBlue3}‚नार्थ‚रूप‚ता}‚ऽस्ति । स‚त्यां वार्थ‚रूप‚तायां व्य‚भिचारिणी सा ‚{\tiny $_{lb}$}‚द्विच‚न्द्र‚ज्ञानादिषु । त‚त‚श्च ‚{\color{DodgerBlue3}‚त‚त्संवेद‚न‚भाव‚स्या}‚र्थ‚संवेद‚न‚त्व‚स्य ‚{\color{DodgerBlue3}‚प्र‚{\tiny $_{1}$}‚साध‚ने}‚षु साऽर्थ‚{\tiny $_{lb}$}‚रूप‚ता न ‚{\color{DodgerBlue3}‚स‚म‚र्था । न} केव‚लाद‚र्थ‚सारूप्याद‚र्थ‚संवेद‚न‚त्वं येन व्य‚भिचारः स्यात्(।) ‚{\tiny $_{lb}$}‚किन्त‚र्हि सारूप्य‚त‚दुत्प‚त्तिभ्यां(।) ते च द्विच‚न्द्र‚ज्ञानादीनां न स्तः । च‚न्द्र‚द्व‚य‚स्याभावात् ‚{\tiny $_{lb}$}‚त‚दुत्प‚त्तेर‚योगात् । (३२२)
	\pend% ending standard par
      \label{div_pvv.2.323}
	  
	% new div opening: depth here is 2
	

	  \pstart \leavevmode% starting standard par
	एत‚देवाह (।)
	\pend% ending standard par
      
	  \bigskip
	  \begingroup
	
	    \large
	  
	    \begin{quote}
	  
	    
	    \stanza[\smallbreak]
	\label{pv.2.323}\flagstanza{\tiny\textenglish{....2.323}}त‚त्सारूप्य‚त‚दुत्प‚त्ती य‚दि स‚म्वेद्य‚ल‚क्ष‚ण‚म् ।&स‚म्वेद्यं स्यात् स‚मानार्थं विज्ञानं स‚म‚न‚न्त‚र‚म् ॥ ३२३ ॥\&[\smallbreak]


	
	    \end{quote}
	  
	  \endgroup
	

	  \pstart \leavevmode% starting standard par
	\hphantom{.}‚{\color{DodgerBlue3}‚तेन} ग्राह्येण ‚{\color{DodgerBlue3}‚सारूप्यं त‚स्मादुत्प‚त्तिः} स्व‚{\color{DodgerBlue3}‚स‚म्वेद्य}‚स्य ‚{\color{DodgerBlue3}‚ल‚क्ष‚णं} य‚दि संम‚तं त‚दापि ‚{\tiny $_{lb}$}‚‚{\color{DodgerBlue3}‚स‚म‚न\edtext{}{\edlabel{pvv.216-2}\label{pvv.216-2}\lemma{न}\Bfootnote{पूर्व्वानुभूत‚स्म‚र‚णं ।}}न्त‚रं ज्ञान}‚मुत्त‚र‚ज्ञानेन ‚{\color{DodgerBlue3}‚स‚मानार्थं} स‚मान‚ग्राह्यं ‚{\color{DodgerBlue3}‚संवेद्यं स्यात्} । त‚त्स‚रूप‚{\tiny $_{lb}$}‚त‚दुत्प‚त्त्योः संभ‚वात् । (३२३)
	\pend% ending standard par
      \label{div_pvv.2.324}
	  
	% new div opening: depth here is 2
	

	  \begin{center}%% label @type='head'
	\textbf{ख. दृश्य‚द‚र्श‚ने प्र‚त्यास‚त्तिविचारः}
	\end{center}
	

	  \pstart \leavevmode% starting standard par
	स्यादेत‚त् (।)
	\pend% ending standard par
      
	  \bigskip
	  \begingroup
	
	    \large
	  
	    \begin{quote}
	  
	    
	    \stanza[\smallbreak]
	\label{pv.2.324a}\flagstanza{\tiny\textenglish{...2.324a}}इदं दृष्टं श्रुत‚म्वेद‚मिति य‚त्राव‚साय‚धीः ।&स त‚स्यानुभ‚वः;\&[\smallbreak]


	
	    \end{quote}
	  
	  \endgroup
	

	  \pstart \leavevmode% starting standard par
	सा‚{\tiny $_{2}$}‚रूप्य‚त‚दुत्प‚त्तिम‚त्त्वेपीदं ‚{\color{DodgerBlue3}‚दृष्टं श्रुत‚म्वे\edtext{}{\edlabel{pvv.216-3}\label{pvv.216-3}\lemma{म्वे}\Bfootnote{स्मार्त्तेनेदं प्र‚योगः ।}}द‚मिति य‚त्राव‚साय‚धी}‚रुत्प‚द्य‚ते ‚{\color{DodgerBlue3}‚त‚स्य ‚{\tiny $_{lb}$}‚सोऽनुभ‚वो} नान्य‚स्य । न च स‚म‚न‚न्त‚र‚प्र‚त्य‚ये दृष्ट‚श्रुताद्य‚व‚सायो भ‚व‚ति त‚न्न ‚{\tiny $_{lb}$}‚ग्राह्योऽसौ ।
	\pend% ending standard par
      

	  \pstart \leavevmode% starting standard par
	अत्राह (।)
	\pend% ending standard par
      
	  \bigskip
	  \begingroup
	
	    \large
	  
	    \begin{quote}
	  
	    
	    \stanza[\smallbreak]
	\label{pv.2.324b}\flagstanza{\tiny\textenglish{...2.324b}}सैव प्र‚त्यास‚त्तिर्व्विचार्य‚ते ॥ ३२४ ॥\&[\smallbreak]


	
	    \end{quote}
	  
	  \endgroup
	\textsuperscript{\textenglish{217/s}}

	  \pstart \leavevmode% starting standard par
	\hphantom{.}दृश्य‚द‚र्श‚न‚योः ‚{\color{DodgerBlue3}‚सैव} प्र‚त्यास‚त्ति‚{\color{DodgerBlue3}‚र्व्विचार्य‚ते}‚ऽस्मा\edtext{}{\edlabel{pvv.217-1}\label{pvv.217-1}\lemma{ऽस्मा}\Bfootnote{ज्ञान‚ज्ञेय‚योः पृथ‚गिष्टे न तादात्म्यं त‚द्रूपेर्थेऽस‚त्य‚पि त‚द्रूप‚ज्ञानात् न त‚दुत्प‚त्तिः ‚{\tiny $_{lb}$}‚विष‚य‚व्य‚व‚स्थाश्र‚य इदं प‚राम‚र्शः प‚राम‚र्शाच्च त‚द्विष‚य‚व्य‚व‚स्था ।}}भिः । (३२४)
	\pend% ending standard par
      \label{div_pvv.2.325}
	  
	% new div opening: depth here is 2
	
	  \bigskip
	  \begingroup
	
	    \large
	  
	    \begin{quote}
	  
	    
	    \stanza[\smallbreak]
	\label{pv.2.325}\flagstanza{\tiny\textenglish{....2.325}}दृश्य‚द‚र्श‚न‚योर्येन त‚स्य त‚द् द‚र्श‚न‚म्म‚त‚म् ।&त‚योः स‚म्ब‚न्ध‚माश्रित्य द्र‚ष्टुरेष विनिश्च‚यः ॥ ३२५ ॥\&[\smallbreak]


	
	    \end{quote}
	  
	  \endgroup
	

	  \pstart \leavevmode% starting standard par
	\hphantom{.}‚{\color{DodgerBlue3}‚येन} प्र‚त्यास‚त्तिस‚म्भ‚वेन ‚{\color{DodgerBlue3}‚त‚स्य} बाह्य‚स्य त‚ज्ज्ञानं ‚{\color{DodgerBlue3}‚द‚र्श‚नं म‚तं । त‚योर्दृ श्य‚द‚र्श‚न‚योः ‚{\tiny $_{lb}$}‚स‚म्ब‚न्धं} प्र‚त्यास‚त्तिमा‚{\color{DodgerBlue3}‚श्रित्य द्र‚ष्टुः} पुरुष‚{\color{DodgerBlue3}‚स्येदं दृष्टं श्रुतं वेद‚मित्य‚र्थ‚निश्च‚यः} । त‚त्सा‚{\tiny $_{lb}$}‚रूप्य‚त‚दु‚{\tiny $_{2}$}‚त्प‚त्तिल‚क्ष‚णा च प्र‚त्यास‚त्तिः स‚म‚न‚न्त‚र‚प्र‚त्य‚येपि स‚मानेति त‚न्निब‚न्ध‚नो ‚{\tiny $_{lb}$}‚दृष्ट‚श्रुताध्य‚व‚साय(ो) पि त‚त्र स्यादित्य‚र्थः । (३२५)
	\pend% ending standard par
      \label{div_pvv.2.326}
	  
	% new div opening: depth here is 2
	
	  \bigskip
	  \begingroup
	
	    \large
	  
	    \begin{quote}
	  
	    
	    \stanza[\smallbreak]
	\label{pv.2.326}\flagstanza{\tiny\textenglish{....2.326}}आत्मा स त‚स्यानुभ‚वः स च नान्य‚स्य क‚स्य‚चित् ।&प्र‚त्य‚क्ष‚प्र‚तिवेद्य‚त्व‚म‚पि त‚स्य त‚दात्म‚ता ॥ ३२६ ॥\&[\smallbreak]


	
	    \end{quote}
	  
	  \endgroup
	

	  \pstart \leavevmode% starting standard par
	\hphantom{.}त‚स्माद्वेद्य‚र‚हित‚स्त‚स्य ज्ञान‚स्य स नीलादिरूप आत्मा‚{\color{DodgerBlue3}‚नुभ‚वः (।) स चानुभ‚वो ‚{\tiny $_{lb}$}‚नान्य‚स्य क‚स्य‚चिद्} बाह्य‚स्य । ‚{\color{DodgerBlue3}‚त‚स्य} ज्ञान‚स्य ‚{\color{DodgerBlue3}‚प्र‚त्य‚क्ष‚प्र‚तिवेद्य‚त्व‚म‚पि} य‚दुच्य‚ते । सा ‚{\tiny $_{lb}$}‚त‚दात्म‚ताऽप‚रोक्षानुभ‚वात्म‚ता । (३२६)
	\pend% ending standard par
      \label{div_pvv.2.327}
	  
	% new div opening: depth here is 2
	
	  \bigskip
	  \begingroup
	
	    \large
	  
	    \begin{quote}
	  
	    
	    \stanza[\smallbreak]
	\label{pv.2.327}\flagstanza{\tiny\textenglish{....2.327}}नान्योनुभाव्य‚स्तेनास्ति त‚स्य नानुभ‚वोप‚रः ।&त‚स्यापि तुल्य‚चोद्य‚त्वात् स्व‚यं सैव प्र‚काश‚ते ॥ ३२७ ॥\&[\smallbreak]


	
	    \end{quote}
	  
	  \endgroup
	

	  \pstart \leavevmode% starting standard par
	य‚था च स्व‚रू\edtext{}{\edlabel{pvv.217-2}\label{pvv.217-2}\lemma{रू}\Bfootnote{ग्राह्य‚ग्राह‚क‚त्वायोगात् ।}}पाद‚{\color{DodgerBlue3}‚न्यो} बुद्ध्या ‚{\color{DodgerBlue3}‚अनुभाव्यो नास्ति} । त‚था ‚{\color{DodgerBlue3}‚त‚स्य} ज्ञान‚स्य ‚{\tiny $_{lb}$}‚चाप‚रोऽ‚{\color{DodgerBlue3}‚नुभ‚वो नास्ति} । त‚स्य‚{\tiny $_{3}$}‚ \edtext{\textsuperscript{*}}{\edlabel{pvv.217-3}\label{pvv.217-3}\lemma{*}\Bfootnote{ग्राह्य‚त्व‚स्य ।}}ज्ञान‚ग्र‚ह‚ण\edtext{}{\edlabel{pvv.217-4}\label{pvv.217-4}\lemma{ण}\Bfootnote{य‚त् स्व‚स‚म्विद्रूपं त‚न्निराल‚म्ब‚नं य‚था स्व‚प्न‚ज्ञानं स्व‚विद्रूप‚ञ्च जाग‚रेपीति ‚{\tiny $_{lb}$}‚त‚न्मात्रानुब‚न्धित्वा (त्) स्व‚भाव‚हेतुरुक्त (ः।) एतेन अप्र‚त्य‚क्षोप‚ल‚म्भ‚स्य नार्थ‚दृष्टिः ‚{\tiny $_{lb}$}‚प्र‚सिध्य‚तीति नासिद्ध‚त्वं ।}} स्यापि ‚{\color{DodgerBlue3}‚तुल्यार्थ‚चोद्य‚त्वात्} । स ह्य‚न्य‚त्व‚{\tiny $_{lb}$}‚निब‚न्ध‚नो ग्राह्य‚ग्राह‚क‚भावः । त‚च्चानुप‚प‚न्न‚मित्युक्तं । त‚त्त‚स्मात्त‚त् ज्ञान‚म‚प‚रोक्ष‚{\tiny $_{lb}$}‚त‚योत्प‚न्नं ‚{\color{DodgerBlue3}‚स्व‚यं प्र‚काश‚ते} । नान्येन प्र‚काश्य‚ते । (३२७)
	\pend% ending standard par
      \label{div_pvv.2.328}
	  
	% new div opening: depth here is 2
	

	  \begin{center}%% label @type='head'
	\textbf{ग. नीलाद्य‚नुभ‚व‚प्र‚सिद्धिः}
	\end{center}
	

	  \pstart \leavevmode% starting standard par
	क‚थं त‚र्हि नीलाद्य‚नुभ‚व‚प्र‚सिद्धिरित्याह (।)
	\pend% ending standard par
      
	  \bigskip
	  \begingroup
	
	    \large
	  
	    \begin{quote}
	  
	    
	    \stanza[\smallbreak]
	\label{pv.2.328}\flagstanza{\tiny\textenglish{....2.328}}नीलादिरूप‚स्त‚स्यासौ स्व‚भावोनुभ‚व‚श्च सः ॥&नीलाद्य‚नुभ‚वात् ख्यातः स्व‚रूपानुभ‚वोपि स‚न् ॥ ३२८ ॥\&[\smallbreak]


	
	    \end{quote}
	  
	  \endgroup
	\textsuperscript{\textenglish{218/s}}

	  \pstart \leavevmode% starting standard par
	त‚स्य ज्ञान‚स्य नीला\edtext{}{\edlabel{pvv.218-1}\label{pvv.218-1}\lemma{नीला}\Bfootnote{क‚ल्पितः क‚र्म‚क‚र्त्रादीति वाच्यं ।}}दिरूपोऽसौ स्व‚भावोऽनुभ‚वः प्र‚काशात्म‚क‚श्च सः । तेन ‚{\tiny $_{lb}$}‚स्व‚रूपानुभ‚वोपि स‚न्नीलाद्य‚नुभ‚वोत् त‚था संप्र‚सिद्धिः । (३२८)
	\pend% ending standard par
      \label{div_pvv.2.329}
	  
	% new div opening: depth here is 2
	
	  \bigskip
	  \begingroup
	
	    \large
	  
	    \begin{quote}
	  
	    
	    \stanza[\smallbreak]
	\label{pv.2.329}\flagstanza{\tiny\textenglish{....2.329}}प्र‚काश‚मान‚स्तादात्म्यात् स्व‚रूप‚स्य प्र‚काश‚कः ।&य‚था प्र‚काशोभिम‚त‚स्त‚था धीरात्म‚वेदिनी ॥ ३२९ ॥\&[\smallbreak]


	
	    \end{quote}
	  
	  \endgroup
	

	  \pstart \leavevmode% starting standard par
	\hphantom{.}‚{\color{DodgerBlue3}‚य‚था प्र‚काश‚स्तादात्म्यात्} प्र‚काशात्म‚क‚त्वात्‚{\tiny $_{4}$}‚ प‚र‚निर‚पेक्षः ‚{\color{DodgerBlue3}‚प्र‚काश‚मानः ‚{\tiny $_{lb}$}‚स्व‚रूप‚स्य प्र‚काश‚कोऽभिम‚तः । त‚था धीः} प‚र‚निर‚पेक्षा प्र‚काशात्म‚नोत्प‚न्ना प्र‚काश‚{\tiny $_{lb}$}‚मानाऽ‚{\color{DodgerBlue3}‚त्म‚वेदिनीति} उप‚चारादुच्य‚ते । (३२९)
	\pend% ending standard par
      \label{div_pvv.2.330}
	  
	% new div opening: depth here is 2
	
	  \bigskip
	  \begingroup
	
	    \large
	  
	    \begin{quote}
	  
	    
	    \stanza[\smallbreak]
	\label{pv.2.330}\flagstanza{\tiny\textenglish{....2.330}}त‚स्याश्चार्थान्त‚रे वेद्ये दुर्घ‚टौ वेद्य‚वेद‚कौ ।&अवेद्य‚वेद‚काकारा; य‚था भ्रान्तैर्निरीक्ष्य‚ते ॥ ३३० ॥\&[\smallbreak]


	
	    \end{quote}
	  
	  \endgroup
	
	  \bigskip
	  \begingroup
	
	    \large
	  
	    \begin{quote}
	  
	    
	    \stanza[\smallbreak]
	\label{pv.2.331a}\flagstanza{\tiny\textenglish{...2.331a}}विभ‚क्त‚ल‚क्ष‚ण‚ग्राह्य‚ग्राह‚काकार‚विप्ल‚वा ।\&[\smallbreak]


	
	    \end{quote}
	  
	  \endgroup
	

	  \pstart \leavevmode% starting standard par
	\hphantom{.}च‚कारो हेतौ । य‚स्मा‚{\color{DodgerBlue3}‚त्त‚स्या} धियोऽ‚{\color{DodgerBlue3}‚र्थान्त‚रे वेद्ये वेद्य‚वे (द) काकारौ दुर्घ‚टौ} (।) ‚{\tiny $_{lb}$}‚त‚स्माद् व‚स्तुतोऽ‚{\color{DodgerBlue3}‚वेद्य‚वेद‚काकारा} सा बुद्धिर्नील‚प्र‚काशात्म‚नोत्प‚न्ना त‚था प्र‚काश‚ते । ‚{\tiny $_{lb}$}‚न तु त‚त्र क‚श्चिद् ग्राह्य‚स्य ग्राह‚क‚स्य चाकारः स‚म‚स्ति । क‚थं त‚र्हि ग्राह्य‚ग्राह‚क‚{\tiny $_{lb}$}‚प्र‚तिभास‚व्य‚व‚{\tiny $_{6}$}‚सायावित्याह (।) ‚{\color{DodgerBlue3}‚भ्रान्तै}‚र‚प्र‚हीण‚द्व‚य‚वास‚नाविप्ल‚वै‚{\color{DodgerBlue3}‚र्य‚था विभ‚क्त‚{\tiny $_{lb}$}‚ल‚क्ष‚णौ ग्राह्य‚ग्राह‚काकारावेव विप्ल‚वौ} य‚स्याः सा तादृशी ‚{\color{DodgerBlue3}‚निरीक्ष्य‚ते ‚{\tiny $_{lb}$}‚विभाव्य‚ते भ्रान्त‚द‚र्श‚नानुरोधेन} (। ३३०)
	\pend% ending standard par
      \label{div_pvv.2.331}
	  
	% new div opening: depth here is 2
	

	  \begin{center}%% label @type='head'
	\textbf{घ. ग्राह्य‚ग्राह‚क‚प्र‚तिभास‚व्य‚व‚साय:}
	\end{center}
	
	  \bigskip
	  \begingroup
	
	    \large
	  
	    \begin{quote}
	  
	    
	    \stanza[\smallbreak]
	\label{pv.2.331b}\flagstanza{\tiny\textenglish{...2.331b}}त‚था कृत‚व्य‚व‚स्थेयं केशादिज्ञान‚भेद‚व‚त् ॥ ३३१ ॥\&[\smallbreak]


	
	    \end{quote}
	  
	  \endgroup
	

	  \pstart \leavevmode% starting standard par
	\hphantom{.}‚{\color{DodgerBlue3}‚त‚था} ग्राह्य‚ग्राह‚क‚भेदेन ‚{\color{DodgerBlue3}‚कृत‚व्य‚व‚स्था} स नीलादिब‚हिर्देशं ग्राह्य‚मान्त‚र‚ञ्च संवेद‚नं ‚{\tiny $_{lb}$}‚ग्राह‚क‚मिति विप्ल‚व एष ‚{\color{DodgerBlue3}‚केशादिज्ञान‚भेद‚व‚त्} । न हि केशोण्डूक‚ज्ञान‚विशेष‚स्य ‚{\tiny $_{lb}$}‚ग्राह‚क‚व‚द् ग्राह्यः केशाव‚य‚वोस्ति (।) किं त‚र्हि केशाभासः प्र‚काश एव ‚{\tiny $_{lb}$}‚\leavevmode\ledsidenote{\textenglish{43a/MA}} के‚{\tiny $_{7}$}‚व‚लः । (३३१)
	\pend% ending standard par
      \label{div_pvv.2.332}
	  
	% new div opening: depth here is 2
	
	  \bigskip
	  \begingroup
	
	    \large
	  
	    \begin{quote}
	  
	    
	    \stanza[\smallbreak]
	\label{pv.2.332}\flagstanza{\tiny\textenglish{....2.332}}य‚दा त‚दा न संचोद्य‚ग्राह्य‚ग्राह‚क‚ल‚क्ष‚णी ।&त‚दान्य‚स‚म्विदोभावात् स्व‚स‚म्वित् फ‚ल‚मिष्य‚ते ॥ ३३२ ॥\&[\smallbreak]


	
	    \end{quote}
	  
	  \endgroup
	

	  \pstart \leavevmode% starting standard par
	\hphantom{.}विप्ल‚व‚व‚शाच्च ग्राह्य‚ग्राह‚क‚भेदेन बुद्धि‚{\color{DodgerBlue3}‚र्य‚दा} व्य‚व‚स्थाप्य‚ते ‚{\color{DodgerBlue3}‚त‚दा न सं\edtext{}{\edlabel{pvv.218-2}\label{pvv.218-2}\lemma{सं}\Bfootnote{न संचोद्ये ग्राह्य‚ग्राह‚क‚ल‚क्ष‚णे य‚स्याः ।}} चोद्य-} \leavevmode\ledsidenote{\textenglish{219/s}} ‚{\color{DodgerBlue3}‚ग्राह्य‚ग्राह‚क‚ल‚क्ष‚णा सा} । न हि व‚स्तुतो ग्राह्य‚ग्राह‚क‚भावः स‚म्भ‚व‚ति । न च विप्ल‚व‚{\tiny $_{lb}$}‚व‚शाद्व‚स्तुव्य‚व‚स्था केश‚द्विच‚न्द्रादेर‚पि त‚त्त्व‚प्र‚स‚ङ्गात्(।) य‚दा च न ग्राह्य‚ग्राह‚क‚ता ‚{\tiny $_{lb}$}‚विज्ञ‚प्तिमात्र‚तायां ‚{\color{DodgerBlue3}‚त‚दान्य}‚स्य ग्राह्य‚स्य ‚{\color{DodgerBlue3}‚संविदो} ज्ञान‚स्या‚{\color{DodgerBlue3}‚भावात् स्व‚संवित् फ‚ल‚मि‚{\tiny $_{lb}$}‚ष्य‚ते} । (३३२)
	\pend% ending standard par
      \label{div_pvv.2.333}
	  
	% new div opening: depth here is 2
	

	  \begin{center}%% label @type='head'
	\textbf{ङ. बाह्यार्थ‚निरासः}
	\end{center}
	
	  \bigskip
	  \begingroup
	
	    \large
	  
	    \begin{quote}
	  
	    
	    \stanza[\smallbreak]
	\label{pv.2.333}\flagstanza{\tiny\textenglish{....2.333}}य‚दि बाह्योनुभूयेत को दोषो नैव क‚श्च‚न ।&इद‚मेव किमुक्तं स्यात् स बाह्योर्थोनुभूय‚ते ॥ ३३३ ॥\&[\smallbreak]


	
	    \end{quote}
	  
	  \endgroup
	

	  \pstart \leavevmode% starting standard par
	\hphantom{.}न‚नु ‚{\color{DodgerBlue3}‚य‚दि बाह्योऽर्थो} ज्ञानेना‚{\color{DodgerBlue3}‚नुभूय‚ते} त‚दा ‚{\color{DodgerBlue3}‚को दोषः} । येन स्व‚संवित् फ‚ल‚मिष्य‚ते ।
	\pend% ending standard par
      

	  \pstart \leavevmode% starting standard par
	आह (।) य‚द्य‚नुभूय‚ते त‚{\tiny $_{1}$}‚दा ‚{\color{DodgerBlue3}‚नैव क‚श्च‚न} दोषः । अनुभ‚व एव तु बाह्य‚स्य ‚{\tiny $_{lb}$}‚नास्तीत्युच्य‚ते । त‚था ‚{\color{DodgerBlue3}‚इद‚मेव किमुक्तं स्या\edtext{}{\edlabel{pvv.219-1}\label{pvv.219-1}\lemma{स्या}\Bfootnote{रिक्तं प‚श्य‚न् पृच्छ‚ति प‚रं ।}}त् बाह्योऽर्थो ज्ञानेना‚{\color{DodgerBlue3}‚नुभूय‚त}} इति\edtext{}{\edlabel{pvv.219-2}\label{pvv.219-2}\lemma{इति}\Bfootnote{य‚दि निराकारा बुद्धिर्व्विष‚योऽर्थान्त‚रं त‚दा स‚म्ब‚न्धासिद्धेः ।}} ‚{\tiny $_{lb}$}‚(३३३ ।)
	\pend% ending standard par
      \label{div_pvv.2.334}
	  
	% new div opening: depth here is 2
	
	  \bigskip
	  \begingroup
	
	    \large
	  
	    \begin{quote}
	  
	    
	    \stanza[\smallbreak]
	\label{pv.2.334}\flagstanza{\tiny\textenglish{....2.334}}य‚दि बुद्धिस्त‚दाकारा सास्त्याकार‚विशेषिणी ।&सा बाह्याद‚न्य‚तो वेति विचार‚मिद‚म‚र्ह‚ति ॥ ३३४ ॥\&[\smallbreak]


	
	    \end{quote}
	  
	  \endgroup
	

	  \pstart \leavevmode% starting standard par
	\hphantom{.}‚{\color{DodgerBlue3}‚य‚दि बुद्धिस्त‚दाकारा} वा बाह्य‚स‚रूपेत्युच्य‚ते । स‚त्य‚म‚स्ति ‚{\color{DodgerBlue3}‚सा बुद्धिराकार‚{\tiny $_{lb}$}‚विशेषिणी} नीलानीलाद्याकार‚विशेष‚युक्ता । किंतु सा बुद्धि‚{\color{DodgerBlue3}‚र्ब्बाह्याद}‚र्थाज्जायेता‚{\color{DodgerBlue3}‚न्य‚तो} वास‚नाप्र‚तिनिय‚मा‚{\color{DodgerBlue3}‚द्वा इति विचार‚मिद‚म‚र्ह‚ति} । न ताव‚द् बुद्धिव्य‚तिरेकिणाऽर्थः ‚{\tiny $_{lb}$}‚क‚श्चिद्धेतुत‚योप‚ल‚भ्य‚ते । बुद्धिस्व‚रूप‚मात्र‚वेद‚नात् । कादाचि‚{\tiny $_{2}$}‚त्क‚त‚या तु कार‚णं ‚{\tiny $_{lb}$}‚त‚स्याः किञ्चिद् व्य‚व‚स्थाप‚नीयं त‚च्च बाह्यं वास‚ना वा स्यात् उभ‚य‚थाप्युप‚प‚त्तेः । ‚{\tiny $_{lb}$}‚(३३४)
	\pend% ending standard par
      \label{div_pvv.2.335}
	  
	% new div opening: depth here is 2
	
	  \bigskip
	  \begingroup
	
	    \large
	  
	    \begin{quote}
	  
	    
	    \stanza[\smallbreak]
	\label{pv.2.335a}\flagstanza{\tiny\textenglish{...2.335a}}द‚र्श‚नोपाधिर‚हित‚स्याग्र‚हात्त‚द्‏ग्र‚हे ग्र‚हाद् ।&द‚र्श‚नं नील‚निर्भासं;\&[\smallbreak]


	
	    \end{quote}
	  
	  \endgroup
	

	  \pstart \leavevmode% starting standard par
	\hphantom{.}त‚त्र ‚{\color{DodgerBlue3}‚द‚र्श‚नेन} ज्ञानेनो‚{\color{DodgerBlue3}‚पाधि}‚ना विशेष‚णेन ‚{\color{DodgerBlue3}‚र‚हित‚स्य} नीलादेर‚{\color{DodgerBlue3}‚ग्र‚हात्}\edtext{}{\edlabel{pvv.219-3}\label{pvv.219-3}\lemma{नीलादेर}\Bfootnote{त‚था नीलाकार‚स्याग्र‚हे ज्ञानाग्र‚हात् ।}} त‚स्य ग्र‚हे ‚{\tiny $_{lb}$}‚च नील‚स्य ‚{\color{DodgerBlue3}‚ग्र\edtext{}{\edlabel{pvv.219-4}\label{pvv.219-4}\lemma{ग्र}\Bfootnote{नील‚ग्र‚हे द‚र्श‚नाच्च ।}}हात्} स‚हैव नील‚धियोर्व्वेद‚नात् ‚{\color{DodgerBlue3}‚द\edtext{}{\edlabel{pvv.219-5}\label{pvv.219-5}\lemma{द}\Bfootnote{न हि ज्ञाने द्वाव‚नुभ‚वौ किन्तु स्वोप‚ल‚म्भ एव ज्ञान‚स्यापि ।}}र्श‚नं नीलादिनिर्भासं} नीलाकारं ‚{\tiny $_{lb}$}‚\leavevmode\ledsidenote{\textenglish{220/s}} ‚{\color{DodgerBlue3}‚व्य‚व‚स्थितं} । य‚त्ताव‚त् नीलादिकं बाह्य‚मित्युच्य‚ते । त‚ज्ज्ञानेन स‚होप‚ल‚म्भ‚निय‚मात् ‚{\tiny $_{lb}$}‚त‚द‚भिन्न‚स्व‚भावं द्विच\edtext{}{\edlabel{pvv.220-1}\label{pvv.220-1}\lemma{द्विच}\Bfootnote{भ्रान्त्या ।}} न्द्रादिव‚त् ।
	\pend% ending standard par
      

	  \pstart \leavevmode% starting standard par
	क‚स्त‚र्हि नास्तीत्याह (।)
	\pend% ending standard par
      
	  \bigskip
	  \begingroup
	
	    \large
	  
	    \begin{quote}
	  
	    
	    \stanza[\smallbreak]
	\label{pv.2.335b}\flagstanza{\tiny\textenglish{...2.335b}}नार्थो बाह्योस्ति केव‚ल‚म् ॥ ३३५ ॥\&[\smallbreak]


	
	    \end{quote}
	  
	  \endgroup
	

	  \pstart \leavevmode% starting standard par
	\hphantom{.}‚{\color{DodgerBlue3}‚बाह्यो} नीलादि‚{\tiny $_{3}$}‚‚{\color{DodgerBlue3}‚र‚र्थः केव‚लं नास्ति} । त‚त्साध‚क‚त्वेनाभिम‚त‚स्याध्य‚क्ष‚स्या‚{\tiny $_{lb}$}‚साम‚र्थ्यात् । (३३५)
	\pend% ending standard par
      \label{div_pvv.2.336}
	  
	% new div opening: depth here is 2
	

	  \pstart \leavevmode% starting standard par
	एव‚न्त‚र्हि ज्ञान‚स्य ग‚जाद्याकार‚स्यालोकादिनिमित्तान्त‚र‚स‚द्भावेपि देश‚का‚{\tiny $_{lb}$}‚लादिप्र‚तिनिय‚म‚द‚र्श‚नात् अर्थो \edtext{}{\edlabel{pvv.220-2}\label{pvv.220-2}\lemma{अर्थो}\Bfootnote{प्य‚स्तीति । अन्य‚था स‚दा स‚र्व‚त्र ज्ञानं स्यात् । य‚था ‚{\tiny $_{lb}$}‚बीजादि स‚दा स्थितोपि कार्य‚कारी ।}}व्य‚व‚स्य‚तीत्याह (।)
	\pend% ending standard par
      
	  \bigskip
	  \begingroup
	
	    \large
	  
	    \begin{quote}
	  
	    
	    \stanza[\smallbreak]
	\label{pv.2.336}\flagstanza{\tiny\textenglish{....2.336}}क‚स्य‚चित् किञ्चिदेवान्त‚र्व्वास‚नायाः प्र‚बोध‚क‚म् ।&त‚तो धियाम्विनिय‚मो न बाह्यार्थ‚व्य‚पेक्ष‚या ॥ ३३६ ॥\&[\smallbreak]


	
	    \end{quote}
	  
	  \endgroup
	

	  \pstart \leavevmode% starting standard par
	\hphantom{.}‚{\color{DodgerBlue3}‚क‚स्य‚चिज्ज्ञा}‚न‚स्य ग‚जाद्याकार‚स्य ‚{\color{DodgerBlue3}‚किञ्चिदेव} ज्ञान‚म‚{\color{DodgerBlue3}‚न्त‚र्व्वास‚नायाः} स‚म‚न‚न्त‚र‚{\tiny $_{lb}$}‚प्र‚त्य‚या\edtext{}{\edlabel{pvv.220-3}\label{pvv.220-3}\lemma{या}\Bfootnote{आल‚याख्य ।}}न्त‚र‚{\color{DodgerBlue3}‚व‚र्त्तिन्या} निय‚त‚ज्ञान‚ज‚न‚न‚योग्य‚ताल‚क्ष‚णायाः ‚{\color{DodgerBlue3}‚प्र‚बोध\edtext{}{\edlabel{pvv.220-4}\label{pvv.220-4}\lemma{बोध}\Bfootnote{आल‚य‚प‚रिण‚तेः स‚ह‚कारिज्ञानं ।}}कं} कार्योत्पाद‚ना‚{\tiny $_{lb}$}‚भिमुख्य‚कार‚कं । ‚{\color{DodgerBlue3}‚त‚तः} प्र‚बोध‚क‚व‚शात् ‚{\color{DodgerBlue3}‚धियां} निय‚ताकार‚{\tiny $_{4}$}‚त‚या ‚{\color{DodgerBlue3}‚विनिय‚मः। न बाह्यार्थ‚{\tiny $_{lb}$}‚व्य‚पेक्ष‚या} (।) को हि विशेषो बाह्यो वा नियाम‚कः प्र‚तिभास‚स्य प्र‚बुद्ध‚वास‚ना‚{\tiny $_{lb}$}‚विशेषः स‚म‚न‚न्त‚र‚प्र‚त्य‚यो वा । त‚त्र वास‚नायाः साम‚र्थ्यं स्व‚प्नादावुप‚ल‚ब्धं । ‚{\tiny $_{lb}$}‚न तु बाह्य‚स्य नित्य‚प‚रोक्ष‚त्वात् । न त‚थापि प‚रोक्ष‚स्य बाह्य‚स्य साध‚क‚स्याभावेपि ‚{\tiny $_{lb}$}‚नाभाव‚स्थितिरिति चेत् । प्र‚तिभास‚मानं ज्ञानं बाह्यं तु न प्र‚तिभास‚त एवेति ताव‚{\tiny $_{lb}$}‚तैवाभिम‚त‚सिद्धेः । साध‚क‚प्र‚माण‚र‚हित‚पिशाचाय‚मान‚ब‚हिर‚र्थ‚निषेधेनास्माक‚माद‚रः । ‚{\tiny $_{lb}$}‚य‚दि तु‚{\tiny $_{5}$}‚ त‚न्निषेध‚निर्ब्ब‚न्धो ग‚रीयान् सांश‚त्वानंश‚त्व‚क‚ल्प‚न‚या प‚र‚माणुप्र‚तिषेधे ‚{\tiny $_{lb}$}‚आ चा र्यी यः प‚र्येशि (? ।षि)त‚व्यः । (३३६)
	\pend% ending standard par
      \label{div_pvv.2.337}
	  
	% new div opening: depth here is 2
	

	  \begin{center}%% label @type='head'
	\textbf{(च. विज्ञान‚द्वैरूप्य‚म्)}
	\end{center}
	‚{\tiny $_{lb}$}‚
	  \bigskip
	  \begingroup
	
	    \large
	  
	    \begin{quote}
	  
	    
	    \stanza[\smallbreak]
	\label{pv.2.337}\flagstanza{\tiny\textenglish{....2.337}}त‚स्माद् द्विरूप‚म‚स्त्येकं य‚देव‚म‚नुभूय‚ते ।&स्म‚र्य‚ते चोभ‚याकार‚स्यास्य स‚म्वेद‚नं फ‚ल‚म् ॥ ३३७ ॥\&[\smallbreak]


	
	    \end{quote}
	  
	  \endgroup
	

	  \pstart \leavevmode% starting standard par
	\hphantom{.}य‚स्माद् बाह्योऽर्थो ‚{\color{DodgerBlue3}‚नानुभूय‚ते त‚स्मा}‚देकं विज्ञानं अविद्योप‚प्लुत‚त्वात् ‚{\color{DodgerBlue3}‚द्विरूपं} बोध‚{\tiny $_{lb}$}‚रूपं नीलादिरूप‚ञ्चा‚{\color{DodgerBlue3}‚स्ति} । य‚त् य‚स्माज्ज्ञान‚मेवं द्व्याकार‚त‚याऽनुभूय‚ते स्व‚वेद‚नेन य‚था‚{\tiny $_{lb}$}‚\leavevmode\ledsidenote{\textenglish{221/s}} नुभ‚वं कालान्त‚रे ‚{\color{DodgerBlue3}‚स्म‚र्य‚ते च} (।) य‚था चान्य‚स्य संवेद‚नाभावात् । ‚{\color{DodgerBlue3}‚उभ‚याद्याकार‚स्य} नीलाद्य‚नुभ‚व‚रूप‚स्य ‚{\color{DodgerBlue3}‚संवेद‚नं फ‚लं} । त‚देवं प्र‚मेयो ग्राह्याकारः प्र‚माणं ग्राह‚काका\edtext{}{\edlabel{pvv.221-1}\label{pvv.221-1}\lemma{काका}\Bfootnote{स‚व्यापार‚प्र‚तीत‚त्वात् प्र‚माणं फ‚ल‚मेव स‚त् । स्व‚स‚म्वित्तिः फ‚ल‚म्वात्र त‚द्रूपो ह्य‚र्थ‚निश्च‚यः । विष‚याभास‚तैवास्य प्र‚माण‚न्तेन मीय‚ते । य‚दाभासं प्र‚मेय‚न्त‚त् प्र‚माण‚फ‚ल‚ते पुनः ॥ ग्राह्य‚ग्राह‚क‚स‚म्वित्ती त्र‚य‚न्नातः पृथ‚क्‏कृत‚मि ति ‚{\tiny $_{lb}$}‚सूत्र‚च‚तुष्ट‚यानुरोधाच्च‚तुराव‚र्त्तितः फ‚ल‚विक‚ल्पः । स‚म्भ‚व‚मात्रेणामी विक‚ल्पाः ‚{\tiny $_{lb}$}‚स‚माप्तिस्तु विज्ञान‚वाद एव कृता ॥}}रः‚{\tiny $_{6}$}‚ ‚{\tiny $_{lb}$}‚फ‚लं स्व‚संविदिति द‚र्शितं भ‚व‚ति ।\edtext{\textsuperscript{*}}{\edlabel{pvv.221-2}\label{pvv.221-2}\lemma{*}\Bfootnote{अस‚म‚र्थोंय‚मेषः फ‚ल‚विक‚ल्पः ।}}(३३७)
	\pend% ending standard par
      \label{div_pvv.2.338}
	  
	% new div opening: depth here is 2
	

	  \begin{center}%% label @type='head'
	\textbf{(छ. अर्थ‚संवित्फ‚ल‚म्)}
	\end{center}
	
	  \bigskip
	  \begingroup
	
	    \large
	  
	    \begin{quote}
	  
	    
	    \stanza[\smallbreak]
	\label{pv.2.338}\flagstanza{\tiny\textenglish{....2.338}}य‚दा निष्प‚न्न‚त‚द्भाव इष्टोनिष्टोपि वा प‚रः ।&विज्ञ‚प्तिहेतुर्व्विष‚य‚स्त‚स्याश्चानुभ‚व‚स्त‚था ॥ ३३८ ॥\&[\smallbreak]


	
	    \end{quote}
	  
	  \endgroup
	

	  \pstart \leavevmode% starting standard par
	\hphantom{.}‚{\color{DodgerBlue3}‚य‚दा} ब‚हिर‚र्थ‚वादेपि ‚{\color{DodgerBlue3}‚प‚रो} वाह्योऽर्थ ‚{\color{DodgerBlue3}‚इष्टोऽनिष्टोऽ\edtext{}{\edlabel{pvv.221-3}\label{pvv.221-3}\lemma{इष्टोऽनिष्टोऽ}\Bfootnote{राग‚द्वेषाभ्याम् ।}}पि वा निष्य‚न्न‚त‚द्भावो} भाव‚नाव‚शाद् व्य‚व‚स्थितेष्टानिष्ट‚भावः स‚रूपाया ‚{\color{DodgerBlue3}‚विज्ञ‚प्तेर्हेतुः} स‚न् ‚{\color{DodgerBlue3}‚विष‚यो} भ‚व‚ति । ‚{\tiny $_{lb}$}‚त‚दा ‚{\color{DodgerBlue3}‚त‚स्या} विज्ञ‚प्तेस्त‚था इष्टानिष्टाकारेणानुभ‚वो विष‚य‚स्य ‚{\color{DodgerBlue3}‚चानुभ‚व} उच्य‚ते । ‚{\tiny $_{lb}$}‚तेन विष‚य‚सारूप्यं प्र‚माण‚म‚र्थ‚संवित् फ‚ल‚मु\edtext{}{\edlabel{pvv.221-4}\label{pvv.221-4}\lemma{मु}\Bfootnote{इति द्वितीयः फ‚ल‚विक‚ल्पः}}क्तं । (३३८)
	\pend% ending standard par
      \label{div_pvv.2.339}
	  
	% new div opening: depth here is 2
	

	  \pstart \leavevmode% starting standard par
	अथ‚वा विज्ञान‚वादेप्य‚विरुद्ध‚मित्याह (।)
	\pend% ending standard par
      
	  \bigskip
	  \begingroup
	
	    \large
	  
	    \begin{quote}
	  
	    
	    \stanza[\smallbreak]
	\label{pv.2.339}\flagstanza{\tiny\textenglish{....2.339}}य‚दा स‚विष‚यं ज्ञानं ज्ञानांशेर्थ‚व्य‚व‚स्थितेः ।&त‚दा य आत्मानुभ‚वः स एवार्थ‚विनिश्च‚यः ॥ ३३९ ॥\&[\smallbreak]


	
	    \end{quote}
	  
	  \endgroup
	

	  \pstart \leavevmode% starting standard par
	\hphantom{.}‚{\color{DodgerBlue3}‚य‚दा} ज्ञान‚स्यांशे आकारे विप्ल‚व‚व‚शात् ‚{\color{DodgerBlue3}‚अर्थ\edtext{}{\edlabel{pvv.221-5}\label{pvv.221-5}\lemma{अर्थ}\Bfootnote{ग्राह्य‚स्य ।}}स्य व्य‚व‚{\tiny $_{7}$}‚स्थितेर्ज्ञानं स‚विष‚य-} मि\edtext{}{\edlabel{pvv.221-6}\label{pvv.221-6}\lemma{मि}\Bfootnote{अर्थ‚शून्यं ।}}ष्टं । ‚{\color{DodgerBlue3}‚त‚दा य आत्म‚नो} ज्ञानाकार‚स्या‚{\color{DodgerBlue3}‚नुभ‚वः स एवार्थ‚स्य निश्च‚यः} संवेद‚न‚मिष्य‚ते\leavevmode\ledsidenote{\textenglish{43b/MA}} ‚{\tiny $_{lb}$}‚(।) त‚त‚श्च वि ज्ञा न वा दे प्य\edtext{}{\edlabel{pvv.221-7}\label{pvv.221-7}\lemma{प्य}\Bfootnote{ग्राह्याकारः प्र‚मेयः ।}}र्थाकारः प्र‚माण‚म‚र्थ‚संवित्फ‚ल‚म‚विरुद्धं । (३३९)
	\pend% ending standard par
      \label{div_pvv.2.340}
	  
	% new div opening: depth here is 2
	
	  \bigskip
	  \begingroup
	
	    \large
	  
	    \begin{quote}
	  
	    
	    \stanza[\smallbreak]
	\label{pv.2.340}\flagstanza{\tiny\textenglish{....2.340}}य‚दीष्टाकार आत्मा स्याद‚न्य‚था वानुभूय‚ते ।&इष्टोनिष्टोपि वा तेन भ‚व‚त्य‚र्थः प्र‚वेदितः ॥ ३४० ॥\&[\smallbreak]


	
	    \end{quote}
	  
	  \endgroup
	\textsuperscript{\textenglish{222/s}}

	  \pstart \leavevmode% starting standard par
	\hphantom{.}ब‚हिर‚र्थ‚न‚येपि बुद्धिवेद‚न‚स्यैवार्थ‚वेद‚न‚त्वात् त‚था ‚{\color{DodgerBlue3}‚य‚दीष्टाका}‚रोऽस्या बुद्धेरा‚{\color{DodgerBlue3}‚त्मा‚{\tiny $_{lb}$}‚ऽनुभूय‚तेऽन्य‚था}‚निष्टाकारो वा । त‚दा ‚{\color{DodgerBlue3}‚तेन} ज्ञानेने‚{\color{DodgerBlue3}‚ष्टोऽनिष्टो वार्थः प्र‚वेदितो भ‚व}‚ति ‚{\tiny $_{lb}$}‚नान्य‚था । (३४०)
	\pend% ending standard par
      \label{div_pvv.2.341}
	  
	% new div opening: depth here is 2
	
	  \bigskip
	  \begingroup
	
	    \large
	  
	    \begin{quote}
	  
	    
	    \stanza[\smallbreak]
	\label{pv.2.341}\flagstanza{\tiny\textenglish{....2.341}}विद्य‚मानेपि बाह्येर्थे य‚थानुभ‚व‚मेव सः ।&निश्चितात्मा स्व‚रूपेण नानेकात्म‚त्व‚दोष‚तः ॥ ३४१ ॥\&[\smallbreak]


	
	    \end{quote}
	  
	  \endgroup
	

	  \pstart \leavevmode% starting standard par
	\hphantom{.}य‚स्मा‚{\color{DodgerBlue3}‚द्विद्य‚मानेपि बाह्येऽर्थे य‚थानुभ‚व}‚म‚नुभ‚वाकारान‚तिक्र‚मेण स‚{\tiny $_{1}$}‚ बाह्योऽर्थो ‚{\tiny $_{lb}$}‚‚{\color{DodgerBlue3}‚निश्चितात्मा} व्य‚व‚स्थाप्य‚ते नार्थ‚{\color{DodgerBlue3}‚स्व‚रूपेण} । य‚था स‚म्भ‚विना निश्चितात्मा व्य‚व‚{\tiny $_{lb}$}‚तिष्ठ‚ते । इष्टानिष्ट‚त्वेन पुरुषाभ्यामेक‚स्यार्थ‚स्य ग्र‚ह‚णाद‚{\color{DodgerBlue3}‚नेकात्म‚त्व‚दोषः} प्र‚स‚ज्य‚ते ।\edtext{\textsuperscript{*}}{\edlabel{pvv.222-1}\label{pvv.222-1}\lemma{*}\Bfootnote{अर्थ‚व‚शाद् बुद्धेर्व्य‚व‚स्थाने याव‚न्त आकारा ज्ञाने तेर्थ‚ग‚ताः प्र‚स‚ज‚न्तीत्य‚{\tiny $_{lb}$}‚नेकाकारोर्थः स्यात् ।}} ‚{\tiny $_{lb}$}‚(३४१)
	\pend% ending standard par
      \label{div_pvv.2.342_2.343}
	  
	% new div opening: depth here is 2
	
	  \bigskip
	  \begingroup
	
	    \large
	  
	    \begin{quote}
	  
	    
	    \stanza[\smallbreak]
	\label{pv.2.342}\flagstanza{\tiny\textenglish{...३४२}}य‚दि बाह्यं न विद्येत क‚स्य संवेद‚नं भ‚वेत् ।&य‚द्य‚ग‚त्या स्व‚रूप‚स्य बाह्य‚स्यैव न किं म‚त‚म्\edtext{}{\edlabel{pvv.222-asterisk}\label{pvv.222-asterisk}\lemma{म्}\Bfootnote{वृत्तिकृता त्व‚विवृतैषा ॥}}  ॥\&[\smallbreak]


	
	    \end{quote}
	   
	    \begin{quote}
	  
	    
	    \stanza[\smallbreak]
	\label{pv.2.343}\flagstanza{\tiny\textenglish{....2.343}}अभ्युपायेपि भेदेन न स्याद‚नुभ‚वो द्व‚योः ।&अदृष्टाव‚र‚णात्स्यात् चेन्न नामार्थ‚व‚शा ग‚तिः ॥ ३४३ ॥\&[\smallbreak]


	
	    \end{quote}
	  
	  \endgroup
	

	  \pstart \leavevmode% starting standard par
	अनेका\edtext{}{\edlabel{pvv.222-2}\label{pvv.222-2}\lemma{अनेका}\Bfootnote{सांख्य‚स्य बाह्य‚रूपाः सुखाद‚यः त्रिगुणात्म‚क‚त्वाज्ज‚ग‚तः ।}}त्म‚क‚त्व‚स्या‚{\color{DodgerBlue3}‚भ्युपाये} स्वीकारेपि\edtext{}{\edlabel{pvv.222-3}\label{pvv.222-3}\lemma{स्वीकारेपि}\Bfootnote{बाह्य‚वादिभिर‚पि य‚थानुभ‚व‚मेवार्थ‚निश्च‚योऽभ्युप‚ग‚त इत्यात्मानुभ‚व एवार्थ‚स्य ।}} ‚{\color{DodgerBlue3}‚द्व‚योः} पुरुष‚यो‚{\color{DodgerBlue3}‚र्भेदेने}‚ष्ट‚त्वेनैक‚स्यान्य‚स्य ‚{\tiny $_{lb}$}‚चानिष्ट‚त्वे‚{\color{DodgerBlue3}‚नानुभ‚वो न स्यात्} । द्व्याकार‚त्वाद्व‚स्तुन‚स्त‚थै\edtext{}{\edlabel{pvv.222-4}\label{pvv.222-4}\lemma{थै}\Bfootnote{द्व‚योर‚पीष्टानिष्ट‚स‚ङ्क‚र‚प्र‚तीतिः स्यात् ।}}व प्र‚तीतिप्र‚स‚क्तिः । अदृष्टे‚{\tiny $_{lb}$}‚न सुख‚दुःख‚वेद‚नीयेन क‚र्म‚णाव‚र‚णात् । द्वितीय‚स्याका‚{\tiny $_{2}$}‚र‚स्य एकात्म‚त्वेन प्र‚तिभासः ‚{\tiny $_{lb}$}‚स्यादिति चेत् । एवं स‚त्य‚र्थ‚{\color{DodgerBlue3}‚व‚शा ग\edtext{}{\edlabel{pvv.222-5}\label{pvv.222-5}\lemma{ग}\Bfootnote{आलोकं चिकीर्ष‚ता स‚र्व्व‚मेवान्धीकृतं त‚र्हि ।}}ति}‚र्ज्ञान‚मिति नाम प्र‚सिद्धं न स्यात् । अदृष्ट‚{\tiny $_{lb}$}‚व‚शेनेष्टानिष्टाकार‚योः प्र‚तीतेरुप‚द‚र्श‚नात् । (३४२, ३४३)
	\pend% ending standard par
      \label{div_pvv.2.344}
	  
	% new div opening: depth here is 2
	

	  \pstart \leavevmode% starting standard par
	किञ्च (।)
	\pend% ending standard par
      
	  \bigskip
	  \begingroup
	
	    \large
	  
	    \begin{quote}
	  
	    
	    \stanza[\smallbreak]
	\label{pv.2.344}\flagstanza{\tiny\textenglish{....2.344}}त‚म‚नेकात्म‚कं भाव‚मेकात्म‚त्वेन द‚र्श‚य‚त् ।&त‚द‚दृष्टं क‚थं नाम भ‚वेद‚र्थ‚स्य द‚र्श‚क‚म् ॥ ३४४ ॥\&[\smallbreak]


	
	    \end{quote}
	  
	  \endgroup
	

	  \pstart \leavevmode% starting standard par
	\hphantom{.}भाव‚म‚{\color{DodgerBlue3}‚नेकात्म‚क‚मेकात्म‚क‚त्वे}‚नैकाकार‚त‚या द‚र्श‚{\color{DodgerBlue3}‚य‚द‚दृष्टं त‚त्क‚थ‚म‚र्थ‚स्य द‚र्श‚कं नाम ‚{\tiny $_{lb}$}‚भ‚वेत्} । त‚देव हि द‚र्श‚क‚म‚स्य य‚त् त‚त्स्व‚रूपं प्र‚तिभास‚य‚ति । न चैकाकारोऽर्थः । ‚{\tiny $_{lb}$}‚त‚त‚स्त‚त्प्र‚तीतिर्नार्थ‚प्र‚तीतिः । (३४४)
	\pend% ending standard par
      \label{div_pvv.2.345}
	  
	% new div opening: depth here is 2
	\textsuperscript{\textenglish{223/s}}
	  \bigskip
	  \begingroup
	
	    \large
	  
	    \begin{quote}
	  
	    
	    \stanza[\smallbreak]
	\label{pv.2.345}\flagstanza{\tiny\textenglish{....2.345}}इष्टानिष्टाव‚भासिन्यः क‚ल्प‚ना नाक्ष‚धीर्य‚दि ।&अनिष्टादाव‚स‚न्धानं दृष्टं त‚त्रापि चेत‚साम् ॥ ३४५ ॥\&[\smallbreak]


	
	    \end{quote}
	  
	  \endgroup
	

	  \pstart \leavevmode% starting standard par
	\hphantom{.}य‚थाव‚स्थित‚व‚स्तुग्राहिण्य‚{\color{DodgerBlue3}‚क्ष‚धी}‚स्त‚द‚न‚न्त‚र‚{\color{DodgerBlue3}‚मिष्टानिष्टाव‚भासिन्यः क‚ल्प‚ना‚{\tiny $_{3}$}‚ऽय‚{\tiny $_{lb}$}‚थार्था} । तेनानेकात्म‚क‚त्व‚दोष‚प्र‚स‚ङ्ग इति ‚{\color{DodgerBlue3}‚य‚दी}‚ष्य‚ते । ‚{\color{DodgerBlue3}‚त‚त्रैन्द्रिय‚त्वेप्य‚निष्टादावादि}‚{\tiny $_{lb}$}‚श‚ब्दात्काम‚लादौ ‚{\color{DodgerBlue3}‚चेत‚सामि}‚न्द्रिय‚ज्ञानानाम‚{\color{DodgerBlue3}‚स‚न्धा\edtext{}{\edlabel{pvv.223-1}\label{pvv.223-1}\lemma{न्धा}\Bfootnote{अन्याकार‚त्वं ।}}न‚म‚र्था}‚कारान‚नुविधानं ‚{\color{DodgerBlue3}‚दृष्टं} । ‚{\tiny $_{lb}$}‚त‚स्मादिन्द्रिय‚बुद्धिर‚य‚थार्थाकारा न भ‚व‚तीति नास्ति । (३४५)
	\pend% ending standard par
      \label{div_pvv.2.346}
	  
	% new div opening: depth here is 2
	
	  \bigskip
	  \begingroup
	
	    \large
	  
	    \begin{quote}
	  
	    
	    \stanza[\smallbreak]
	\label{pv.2.346}\flagstanza{\tiny\textenglish{....2.346}}त‚स्मात् प्र‚मेये बाह्येपि युक्तं स्वानुभ‚वः फ‚ल‚म् ।&य‚तः स्व‚भावोस्य य‚था त‚थैवार्थ‚विनिश्च‚यः ॥ ३४६ ॥\&[\smallbreak]


	
	    \end{quote}
	  
	  \endgroup
	

	  \pstart \leavevmode% starting standard par
	\hphantom{.}‚{\color{DodgerBlue3}‚त‚स्मान्न} केव‚लं स्व‚रूपे ‚{\color{DodgerBlue3}‚बाह्येपि प्र‚मेये स्वानुभ‚वः फ‚लं युक्तं । य‚तः कार‚णात् ‚{\tiny $_{lb}$}‚स्व‚भावोस्य} ज्ञान‚स्य ‚{\color{DodgerBlue3}‚य‚था} प्र‚तिभाति ‚{\color{DodgerBlue3}‚त‚थैव नार्थ‚स्य विनिश्च‚यः} सिध्य‚ति । (३४६)
	\pend% ending standard par
      \label{div_pvv.2.347}
	  
	% new div opening: depth here is 2
	
	  \bigskip
	  \begingroup
	
	    \large
	  
	    \begin{quote}
	  
	    
	    \stanza[\smallbreak]
	\label{pv.2.347}\flagstanza{\tiny\textenglish{....2.347}}त‚द‚र्थाभास‚तैवास्य प्र‚माणं न तु स‚न्न‚पि ।&ग्राह‚काऽत्माऽप‚रार्थ‚त्वाद् बाह्येष्व‚र्थेष्व‚पेक्ष‚ते ॥ ३४७ ॥\&[\smallbreak]


	
	    \end{quote}
	  
	  \endgroup
	

	  \pstart \leavevmode% starting standard par
	\hphantom{.}‚{\color{DodgerBlue3}‚त‚त्} त‚स्माद् ‚{\color{DodgerBlue3}‚बाह्येष्व‚र्थेषु} ग्राह्येष्व‚स्य ज्ञान‚स्या‚{\color{DodgerBlue3}‚र्थाभास‚{\tiny $_{4}$}‚ताऽ}‚र्थाकार‚तैव ‚{\color{DodgerBlue3}‚प्र‚माण‚{\tiny $_{lb}$}‚म\edtext{}{\edlabel{pvv.223-2}\label{pvv.223-2}\lemma{म}\Bfootnote{ज्ञानारूढाकार‚स‚म्वेद‚नानुरूपेणार्थाव‚साय‚व्य‚व‚स्थानात् ।}}पेक्ष‚ते} न त्व‚न्व‚यि‚{\color{DodgerBlue3}‚ग्राह‚कात्मा} ग्राह‚काकारो‚{\color{DodgerBlue3}‚ऽप‚रार्थ‚त्वात्}\edtext{}{\edlabel{pvv.223-3}\label{pvv.223-3}\lemma{काकारो}\Bfootnote{प‚रार्थं प्र‚माण‚न्न च ग्राह्याकार‚व‚द् ग्राह‚काकारो व्य‚व‚स्थाकारीति भावः ।}}आत्म‚विष‚य‚त्वात्त‚स्य । ‚{\tiny $_{lb}$}‚(३४७)
	\pend% ending standard par
      \label{div_pvv.2.348}
	  
	% new div opening: depth here is 2
	

	  \begin{center}%% label @type='head'
	\textbf{(ज. आत्म‚संविदेवार्थ‚संचिद्)}
	\end{center}
	
	  \bigskip
	  \begingroup
	
	    \large
	  
	    \begin{quote}
	  
	    
	    \stanza[\smallbreak]
	\label{pv.2.348}\flagstanza{\tiny\textenglish{....2.348}}य‚स्माद् य‚था निविष्टोसाव‚र्थात्मा प्र‚त्य‚ये त‚था ।&निश्चीय‚ते निविष्टोसावेव‚मित्यात्म‚संविदः ॥ ३४८ ॥\&[\smallbreak]


	
	    \end{quote}
	  
	  \endgroup
	

	  \pstart \leavevmode% starting standard par
	\hphantom{.}‚{\color{DodgerBlue3}‚य‚स्मा}‚त्कार‚ण‚{\color{DodgerBlue3}‚द्य‚था} इष्ट‚त्वेनानिष्ट‚त्वेन वार्थ‚स्यात्माकारः ‚{\color{DodgerBlue3}‚प्र‚त्य‚ये निविष्ट}‚स्त‚था‚{\tiny $_{lb}$}‚ऽर्थों निश्चीय‚ते त‚स्माद‚र्थाकारः प्र‚माणं । ‚{\color{DodgerBlue3}‚असा}‚व‚र्थाकार एव‚मिष्टानिष्ट‚त्वेन बुद्धौ ‚{\tiny $_{lb}$}‚‚{\color{DodgerBlue3}‚निविष्ट} इत्यात्म‚संविदः स्व‚संवेद‚ना‚{\color{DodgerBlue3}‚न्निश्चीय‚ते} । (३४८)
	\pend% ending standard par
      \label{div_pvv.2.349}
	  
	% new div opening: depth here is 2
	
	  \bigskip
	  \begingroup
	
	    \large
	  
	    \begin{quote}
	  
	    
	    \stanza[\smallbreak]
	\label{pv.2.349}\flagstanza{\tiny\textenglish{....2.349}}इत्य‚र्थ‚संवित् सैवेष्टा य‚तोर्थात्मा न दृश्य‚ते ।&त‚स्माद् बुद्धिनिवेश्यार्थः साध‚नं त‚स्य सा क्रिया ॥ ३४९ ॥\&[\smallbreak]


	
	    \end{quote}
	  
	  \endgroup
	

	  \pstart \leavevmode% starting standard par
	\hphantom{.}इति त‚स्मात् सैवात्म‚संविद‚{\color{DodgerBlue3}‚र्थ‚संविदिष्टा । य‚तः} स्व‚रूपाद् ब‚हिर्भूतो‚{\color{DodgerBlue3}‚ऽर्थात्मा न ‚{\tiny $_{lb}$}‚दृश्य‚ते‚{\tiny $_{5}$}‚} बुद्ध्याकार एव तु वेद्य‚ते । अत‚स्त‚द्वेद‚न‚द‚र्श‚न‚मेवार्थ‚वेद‚नं । य‚स्माच्चार्थ‚{\tiny $_{lb}$}‚\leavevmode\ledsidenote{\textenglish{224/s}} सारूप्य‚व‚शेनार्थाधिग‚तिव्य‚व‚स्था ‚{\color{DodgerBlue3}‚त‚स्मा}‚द् ‚{\color{DodgerBlue3}‚बुद्धि}‚निवे‚{\color{DodgerBlue3}‚श्यार्थो}‚ऽर्थ‚प्र‚तिबिम्बं ‚{\color{DodgerBlue3}‚साध‚नं} प्र‚माणं ‚{\tiny $_{lb}$}‚‚{\color{DodgerBlue3}‚त‚स्य} प्र‚माण‚स्य ‚{\color{DodgerBlue3}‚सा}‚धिग‚तिः ‚{\color{DodgerBlue3}‚क्रिया} फ‚ल‚मिष्य‚ते । (३४९)
	\pend% ending standard par
      \label{div_pvv.2.350}
	  
	% new div opening: depth here is 2
	
	  \bigskip
	  \begingroup
	
	    \large
	  
	    \begin{quote}
	  
	    
	    \stanza[\smallbreak]
	\label{pv.2.350}\flagstanza{\tiny\textenglish{....2.350}}य‚था निविश‚ते सोर्थो य‚तः सा प्र‚थ‚ते त‚था ।&अर्थ‚स्थितेस्त‚दात्म‚त्वात् स्व‚विद‚प्य‚र्थ‚विन्म‚ता ॥ ३५० ॥\&[\smallbreak]


	
	    \end{quote}
	  
	  \endgroup
	

	  \pstart \leavevmode% starting standard par
	\hphantom{.}य‚तः कार‚णात् ‚{\color{DodgerBlue3}‚य‚थार्थो} ज्ञानात्म‚नि ‚{\color{DodgerBlue3}‚निविश‚ते} त‚था सा स्व‚संवित्तिः ‚{\tiny $_{lb}$}‚प्र‚थ‚ते ख्याति (।) त‚स्माद‚र्थ‚स्य स्थितेर‚धिग‚तेस्त‚दात्म‚त्वात् प‚र‚मार्थ‚तः\edtext{}{\edlabel{pvv.224-1}\label{pvv.224-1}\lemma{तः}\Bfootnote{त‚द्य‚द्य‚र्थ‚सारूप्यं मानं स्व‚वित्फ‚लं त‚दा विष‚य‚भेदः । आत्म‚विष‚या वित्तिर्ब्बाह्य‚विष‚यं सारूप्य‚मित्याह । प्र‚तिभास‚मात्रेण त‚द्व्य‚व‚स्थाहेतुत्वात् ।}} ‚{\color{DodgerBlue3}‚स्व‚वि‚{\tiny $_{lb}$}‚द‚पि} स‚ती ‚{\color{DodgerBlue3}‚अर्थ‚विद् म‚ता} । स्व‚संवेद‚न‚मेवार्थ‚वेद‚न‚मुप‚चारादुच्य‚{\tiny $_{6}$}‚त इति तादात्म्य‚{\tiny $_{lb}$}‚म‚न‚योः (।) (३५०)
	\pend% ending standard par
      \label{div_pvv.2.351}
	  
	% new div opening: depth here is 2
	
	  \bigskip
	  \begingroup
	
	    \large
	  
	    \begin{quote}
	  
	    
	    \stanza[\smallbreak]
	\label{pv.2.351a}\flagstanza{\tiny\textenglish{...2.351a}}त‚स्माद् विष‚य‚भेदोपि न;\&[\smallbreak]


	
	    \end{quote}
	  
	  \endgroup
	

	  \pstart \leavevmode% starting standard par
	\hphantom{.}य‚स्माद‚र्थाकार एव प्र‚माणं प्र‚तीतिसाध‚न‚त्वात् फ‚ल‚ञ्च प्र‚तीय‚मानं ‚{\color{DodgerBlue3}‚त‚स्मा-} त्प्र‚माण‚फ‚ल‚यो‚{\color{DodgerBlue3}‚र्व्विष‚य‚भेदोपि} नास्ति । य‚था प‚र‚म‚ते आलोच‚न‚विशेष‚ण‚ज्ञानादीना‚{\tiny $_{lb}$}‚मेकार्थापेक्षिण्यावेव हि सारूप्य‚प्र‚तिप‚त्ती प्र‚माण‚फ‚ल‚त्वं प्र‚तिल‚भेते ।
	\pend% ending standard par
      

	  \pstart \leavevmode% starting standard par
	य‚द्य‚र्थ‚संवेद‚नं फ‚लं त‚दा स्व‚वित्फ‚लं क‚थ‚मुक्त‚मित्याह (।)
	\pend% ending standard par
      
	  \bigskip
	  \begingroup
	
	    \large
	  
	    \begin{quote}
	  
	    
	    \stanza[\smallbreak]
	\label{pv.2.351b}\flagstanza{\tiny\textenglish{...2.351b}}स्व‚संवेद‚नं फ‚ल‚म् ।&उक्तं स्व‚भाव‚चिन्तायां तादात्म्याद‚र्थ‚संविदः ॥ ३५१ ॥\&[\smallbreak]


	
	    \end{quote}
	  
	  \endgroup
	

	  \pstart \leavevmode% starting standard par
	\hphantom{.}अर्थ‚संवेदंन‚स्य व‚स्तुतः ‚{\color{DodgerBlue3}‚स्व‚भाव‚चिन्तायां} स्व‚संवेद‚नं फ‚ल‚{\color{DodgerBlue3}‚मुक्तं । अर्थ‚संविद‚स्ता}‚{\tiny $_{lb}$}‚\leavevmode\ledsidenote{\textenglish{44a/MA}} ‚{\color{DodgerBlue3}‚दात्म्यात्}‚{\tiny $_{7}$}‚ \edtext{\textsuperscript{*}}{\edlabel{pvv.224-2}\label{pvv.224-2}\lemma{*}\Bfootnote{विद्य‚मानेपि बाह्येर्थे य‚थानुभ‚व‚मेव वे \cref{pv.2.341}त्यादिना । य‚दा त्व‚बाह्य एवार्थः प्र‚मेय इत्यादि \href{http://sarit.indology.info/?cref=psv.1.9}{स‚मुच्च‚य‚स्य तृतीयः फ‚ल‚विक‚ल्पो व्याख्यातः} ।}}स्व‚संवेद‚नात्म‚त्वात् । न स्वाकार‚प्र‚तीतेर‚न्यास्त्य‚र्थ‚प्र‚तीतिः स्व‚रूपेण ‚{\tiny $_{lb}$}‚त‚स्याः प्र‚तीतेः । (३५१)
	\pend% ending standard par
      \label{div_pvv.2.352}
	  
	% new div opening: depth here is 2
	

	  \pstart \leavevmode% starting standard par
	न‚नु य‚दि य‚थार्थं नानुभ‚व‚स्त‚दा स्व‚वास‚नाप्र‚बोध‚क‚म‚नुव‚र्त‚मान‚स्य ज्ञान‚स्यार्थों\edtext{}{\edlabel{pvv.224-3}\label{pvv.224-3}\lemma{स्यार्थों}\Bfootnote{विना च बाह्यं त‚त्स्व‚रूपाग्र‚ह‚णे क‚थ‚म्बाह्यार्थ‚स्तृतीयः फ‚ल‚विक‚ल्पः ।}} ‚{\tiny $_{lb}$}‚बाह्योस्तीत्येत‚देव कुत इत्याह (।)
	\pend% ending standard par
      
	  \bigskip
	  \begingroup
	
	    \large
	  
	    \begin{quote}
	  
	    
	    \stanza[\smallbreak]
	\label{pv.2.352}\flagstanza{\tiny\textenglish{....2.352}}त‚थाव‚भास‚मान‚स्य तादृशोऽन्यादृशोपि वा ।&ज्ञान‚स्य हेतुर‚र्थोपीत्य‚र्थ‚स्येष्टा प्र‚मेय‚ता ॥ ३५२ ॥\&[\smallbreak]


	
	    \end{quote}
	  
	  \endgroup
	

	  \pstart \leavevmode% starting standard par
	\hphantom{.}‚{\color{DodgerBlue3}‚त‚था} इष्टानिष्टाकारेणा‚{\color{DodgerBlue3}‚व\edtext{}{\edlabel{pvv.224-4}\label{pvv.224-4}\lemma{व}\Bfootnote{अर्थ‚निर‚पेक्ष‚त्वं ।}}भास‚मान‚स्य ज्ञान‚स्या}‚र्थो‚{\color{DodgerBlue3}‚पि ता\edtext{}{\edlabel{pvv.224-5}\label{pvv.224-5}\lemma{ता}\Bfootnote{य‚थाव‚भासोऽन्यादृशो आलोकादिभावेपि नीलादिर‚हित‚देशेऽनुत्प‚त्तेरिन्द्रिय‚सिद्धिव‚त् ।}}दृशो}‚न्यादृशो वा ‚{\tiny $_{lb}$}‚\leavevmode\ledsidenote{\textenglish{225/s}} हेतुरित्य‚र्थ‚स्य ‚{\color{DodgerBlue3}‚प्र‚मेय‚तेष्टा} सौ त्रा न्ति क म‚ते । सारूप्य‚स्य प‚र‚मार्थ‚तः ‚{\color{DodgerBlue3}‚स‚रूप}‚य‚न्ति त‚त् केन स्थूलाभास‚ञ्च तेण‚व \href{http://sarit.indology.info/?cref=pv.2.321}{(२।३२१)} इति प्र‚तिषेधात्\edtext{}{\edlabel{pvv.225-1}\label{pvv.225-1}\lemma{तिषेधात्}\Bfootnote{न‚नु तादृशोऽन्यादृशोपि वार्थः प्र‚मेयो नेष्टः स्व‚प‚र‚योस्त‚त् किमेव‚मुच्य‚ते । प‚रेण स्व‚वाचा बाह्यं निषेध‚यितुमुक्तं त‚दाह ।}}। (३५२)
	\pend% ending standard par
      \label{div_pvv.2.353}
	  
	% new div opening: depth here is 2
	
	  \bigskip
	  \begingroup
	
	    \large
	  
	    \begin{quote}
	  
	    
	    \stanza[\smallbreak]
	\label{pv.2.353a}\flagstanza{\tiny\textenglish{...2.353a}}य‚था क‚थ‚ञ्चित्त‚स्यार्थ‚रूपं मुक्त्वाव‚भासिनः ।&अर्थ‚ग्र‚हः क‚थं;\&[\smallbreak]


	
	    \end{quote}
	  
	  \endgroup
	

	  \pstart \leavevmode% starting standard par
	य‚दि सारूप्याभाव‚स्त‚दा‚{\tiny $_{1}$}‚ त‚स्य ज्ञान‚स्याव‚भासिनो ‚{\color{DodgerBlue3}‚य‚था क‚थ‚ञ्चि}‚दिष्टा‚{\tiny $_{lb}$}‚निष््टादिना भास‚मान‚म‚र्थ‚रूप‚म‚र्थाकारं मुक्त्वा ‚{\color{DodgerBlue3}‚क‚थं} केन प्र‚कारेणार्थ‚स्य ‚{\color{DodgerBlue3}‚ग्र‚हः} स्यात् । ‚{\tiny $_{lb}$}‚न ह्य‚र्थः स्व‚रूपेण दृश्य‚ते । त‚त् स्व‚रूप‚बुद्धिवेद‚नाद‚र्थ‚ग्र‚ह‚व्य‚व‚स्था । सारूप्य‚मेव ‚{\tiny $_{lb}$}‚चेन्न संभ‚व‚ति क‚थ‚म‚र्थ‚ग्र‚ह इति म‚न्य‚ते सौ त्रा न्ति कः
	\pend% ending standard par
      

	  \pstart \leavevmode% starting standard par
	यो गा चा र स्तु त‚स्य साहाय्य‚कं म‚न्य‚मान आह (।)
	\pend% ending standard par
      
	  \bigskip
	  \begingroup
	
	    \large
	  
	    \begin{quote}
	  
	    
	    \stanza[\smallbreak]
	\label{pv.2.353b}\flagstanza{\tiny\textenglish{...2.353b}}स‚त्यं न जानेह‚म‚पोदृश‚म् ॥ ३५३ ॥\&[\smallbreak]


	
	    \end{quote}
	  
	  \endgroup
	

	  \pstart \leavevmode% starting standard par
	\hphantom{.}‚{\color{DodgerBlue3}‚स\edtext{}{\edlabel{pvv.225-2}\label{pvv.225-2}\lemma{स}\Bfootnote{य एव‚भिच्छ‚ति स पृच्छ‚य‚तां ।}}त्यं न जानेऽह‚म‚पीदृश‚म‚र्थ‚ग्र‚हः} क‚थ‚मिति । (३५३)
	\pend% ending standard par
      \label{div_pvv.2.354}
	  
	% new div opening: depth here is 2
	

	  \pstart \leavevmode% starting standard par
	क‚थं त‚र्ह्य‚स‚त्य‚र्थे ग्राह्य‚ग्राह‚क‚फ‚ल‚भेद इत्याह (।)
	\pend% ending standard par
      
	  \bigskip
	  \begingroup
	
	    \large
	  
	    \begin{quote}
	  
	    
	    \stanza[\smallbreak]
	\label{pv.2.354}\flagstanza{\tiny\textenglish{....2.354}}अविभागोपि बुद्ध्यात्म‚विप‚र्यासित‚द‚र्श‚नैः ।&ग्राह्य‚ग्राह‚क‚संवित्तिभेद‚वानिव ल‚क्ष्य‚ते ॥ ३५४ ॥\&[\smallbreak]


	
	    \end{quote}
	  
	  \endgroup
	

	  \pstart \leavevmode% starting standard par
	प‚र\edtext{}{\edlabel{pvv.225-3}\label{pvv.225-3}\lemma{र}\Bfootnote{आनुष‚ङ्गिक‚मुक्त्वा विज्ञ‚प्तौ प्र‚मादिमाह ।}} ‚{\tiny $_{2}$}‚मार्थ‚तो‚{\color{DodgerBlue3}‚ऽविभा}‚गो भेद‚र‚हितो‚{\color{DodgerBlue3}‚पि बुद्ध्यात्म‚द्व}‚य‚वास‚न‚या ‚{\color{DodgerBlue3}‚विप‚र्यासितं} विभागेनोप‚{\color{DodgerBlue3}‚द‚र्शितं} द‚र्श‚नं येषां तैर‚त‚त्त्व‚द‚र्शिपुरुषै‚{\color{DodgerBlue3}‚र्ग्राह्य‚ग्राह‚क‚संवित्ती}‚नां प‚र‚स्प‚रं ‚{\tiny $_{lb}$}‚भेद‚स्त‚द्वानिव ल‚क्ष्य‚ते । (३५४)
	\pend% ending standard par
      \label{div_pvv.2.355}
	  
	% new div opening: depth here is 2
	

	  \pstart \leavevmode% starting standard par
	अत्र दृष्टान्त‚माह (।)
	\pend% ending standard par
      
	  \bigskip
	  \begingroup
	
	    \large
	  
	    \begin{quote}
	  
	    
	    \stanza[\smallbreak]
	\label{pv.2.355}\flagstanza{\tiny\textenglish{....2.355}}म‚न्त्राद्युप‚प्लुताक्षाणां य‚था मृच्छ‚क‚लाद‚यः ।&अन्य‚थैवाव‚भास‚न्ते त‚द्रूप‚र‚हिता अपि ॥ ३५५ ॥\&[\smallbreak]


	
	    \end{quote}
	  
	  \endgroup
	

	  \pstart \leavevmode% starting standard par
	\hphantom{.}‚{\color{DodgerBlue3}‚य‚था मूच्छ‚क‚लाद‚यो म‚न्त्रादिभिरुप‚प्लुत}‚य‚थार्थ‚ज्ञान‚हेतुकृत‚{\color{DodgerBlue3}‚म‚क्षं} येषां तेषा‚{\tiny $_{lb}$}‚म‚न्य‚था सुव‚र्ण्णादित्वे‚{\color{DodgerBlue3}‚नैवाव‚भास‚न्ते} । तेन सुव‚र्ण्णादिना रूपेण ‚{\color{DodgerBlue3}‚र‚हिता अपि} व‚स्तुतः । (३५५)
	\pend% ending standard par
      \textsuperscript{\textenglish{226/s}}\label{div_pvv.2.356}
	  
	% new div opening: depth here is 2
	

	  \pstart \leavevmode% starting standard par
	क‚स्मात्पुन‚र्म्म‚न्त्रादिसाम‚र्थ्यात्‚{\tiny $_{3}$}‚ सुव‚र्ण्णादितामेव यातं मृत्ख‚ण्ड‚मिति नाभ्यु‚{\tiny $_{lb}$}‚प(ग)म्य‚ते इत्याह (।)
	\pend% ending standard par
      
	  \bigskip
	  \begingroup
	
	    \large
	  
	    \begin{quote}
	  
	    
	    \stanza[\smallbreak]
	\label{pv.2.356}\flagstanza{\tiny\textenglish{....2.356}}त‚थैवाद‚र्श‚नात्तेषाम‚नुप‚प्लुत‚च‚क्षुषाम् ।&दूरे य‚था वा म‚रुषु म‚हान‚ल्पोपि दृश्य‚ते ॥ ३५६ ॥\&[\smallbreak]


	
	    \end{quote}
	  
	  \endgroup
	

	  \pstart \leavevmode% starting standard par
	\hphantom{.}‚{\color{DodgerBlue3}‚म‚न्त्रादि}‚नाऽनुप‚प्लुत‚च‚क्षुषान्येन पुंसा तेषां मृच्छ‚क‚लादीनां य‚थोप‚ह‚ताक्षेण ‚{\tiny $_{lb}$}‚ते दृश्य‚न्ते ‚{\color{DodgerBlue3}‚त‚थैव द‚र्श‚नात्} मृदात्म‚त्वेनैव द‚र्श‚नात् सुव‚र्ण्णादित्वेन निष्प‚त्तिक‚ल्प‚न‚म‚{\tiny $_{lb}$}‚युक्तं । ‚{\color{DodgerBlue3}‚य‚था म‚रुषु दूरेऽल्पोपि म‚हान् दृश्य‚ते} त‚द्देश‚स्थैर‚ल्प‚त्वेनैव भात‚स्य द‚र्श‚{\tiny $_{lb}$}‚नात् । (३५६)
	\pend% ending standard par
      \label{div_pvv.2.357}
	  
	% new div opening: depth here is 2
	
	  \bigskip
	  \begingroup
	
	    \large
	  
	    \begin{quote}
	  
	    
	    \stanza[\smallbreak]
	\label{pv.2.357}\flagstanza{\tiny\textenglish{....2.357}}य‚थानुद‚र्श‚न‚ञ्चेयं मेय‚मान‚फ‚ल‚स्थितिः ।&क्रिय‚तेऽविद्य‚मानापि ग्राह्य‚ग्राह‚क‚संविदाम् ॥ ३५७ ॥\&[\smallbreak]


	
	    \end{quote}
	  
	  \endgroup
	

	  \pstart \leavevmode% starting standard par
	\hphantom{.}त‚स्माद् ‚{\color{DodgerBlue3}‚ग्राह्य‚ग्राह‚क‚संविदां} प‚र‚मार्थ‚तो‚{\color{DodgerBlue3}‚ऽविद्य‚मानापि मेय‚मान‚फ‚ल‚स्थितिर्य‚था‚{\tiny $_{lb}$}‚द‚र्श‚नं क्रि‚{\tiny $_{4}$}‚य}‚ते । ग्राह्याका\edtext{}{\edlabel{pvv.226-1}\label{pvv.226-1}\lemma{ग्राह्याका}\Bfootnote{प्र‚थ‚मे विक‚ल्पे नीलादिप‚रिच्छेद‚द्वारेव त‚ज्ज्ञानं फ‚लं आकारः प्र‚मा । द्वितीये विज्ञ‚प्तौ ग्राह्यांशः प्र‚मेयो ग्राह‚कांशः प्र‚मा । त‚दुभ‚य‚वेद‚नं फ‚लं । तृतीये तु प्र‚तिप‚त्तृव्य‚व‚साये नेष्टानिष्टादिरूपेण गृह्य‚मानोऽर्थः प्र‚मेयः त‚था भानं मानं । त‚त्संवित् फ‚लं । त‚दार्थाभास‚तैवास‚त्तैवास्य प्र‚माण‚मित्युक्तः च‚तुर्थः ।}}रो मेयः ग्राह‚काकारो मानं संवित्तिः फ‚ल‚मिति ‚{\tiny $_{lb}$}‚व्य‚व‚स्थाप्य‚ते । (३५७)
	\pend% ending standard par
      \label{div_pvv.2.358}
	  
	% new div opening: depth here is 2
	
	  \bigskip
	  \begingroup
	
	    \large
	  
	    \begin{quote}
	  
	    
	    \stanza[\smallbreak]
	\label{pv.2.358}\flagstanza{\tiny\textenglish{....2.358}}अन्य‚थैक‚स्य भाव‚स्य नानारूपाव‚भासिनः ।&स‚त्यं क‚थं स्युराकारास्त‚देक‚त्व‚स्य हानितः ॥ ३५८ ॥\&[\smallbreak]


	
	    \end{quote}
	  
	  \endgroup
	

	  \pstart \leavevmode% starting standard par
	\hphantom{.}‚{\color{DodgerBlue3}‚अन्य‚था} य‚दि व‚स्तुतो ग्राह‚कादिविभाग इष्य‚ते त‚{\color{DodgerBlue3}‚दैक‚स्य भाव‚स्य} ज्ञानात्म‚नो ‚{\tiny $_{lb}$}‚‚{\color{DodgerBlue3}‚नानारूपा(व)भासिनो}‚ऽनेकाकार‚प्र‚तिभास‚व‚न्त आकारा ग्राह्यादिकाः ‚{\color{DodgerBlue3}‚क‚थं स‚त्यं ‚{\tiny $_{lb}$}‚स्युः} । त‚स्यैक‚ज्ञानात्म‚न ‚{\color{DodgerBlue3}‚एक‚त्व‚स्य हानितः} । न ह्येकं प्र‚तिभास‚मानानेकाकारात्म‚कं ‚{\tiny $_{lb}$}‚भ‚वितुम‚र्ह‚ति । प्र‚तिभासेन स‚र्व्वेषां भेदेन व्य‚व‚स्थाप‚नात् । (३५८)
	\pend% ending standard par
      \label{div_pvv.2.359}
	  
	% new div opening: depth here is 2
	

	  \pstart \leavevmode% starting standard par
	अनेक‚मेव त‚र्हि ज्ञानं प‚रिच्छेदादित्वेन स्यादि‚{\tiny $_{5}$}‚त्याह ।
	\pend% ending standard par
      
	  \bigskip
	  \begingroup
	
	    \large
	  
	    \begin{quote}
	  
	    
	    \stanza[\smallbreak]
	\label{pv.2.359a}\flagstanza{\tiny\textenglish{...2.359a}}अन्य‚स्यान्य‚त्व‚हानेश्च ;\&[\smallbreak]


	
	    \end{quote}
	  
	  \endgroup
	

	  \pstart \leavevmode% starting standard par
	न चानेकं ज्ञानं एक‚मेवेष्ट‚व्यं । व‚स्तुतो भिन्न‚स्याभेदे इष्य‚माणे ग्राह्यादेः ‚{\tiny $_{lb}$}‚प‚र‚स्प‚र‚तो‚{\color{DodgerBlue3}‚ऽन्य‚स्यान्य‚त्व‚हानेः} । य‚दा हि भेद‚प्र‚तिभासेप्येक‚त्व‚मिष्टं त‚दा भेद एव ‚{\tiny $_{lb}$}‚न व्य‚व‚स्थितः । त‚त‚श्च सुख‚दुःखाद‚यो नील‚पीताद‚य‚श्च प‚र‚स्प‚रं न भिद्येर‚न् । ‚{\tiny $_{lb}$}‚प्र‚तिभास‚भेद‚स्य त‚त्साध‚न‚त्वात् । अन्य‚स्य च त‚त्साध‚न‚स्याभावात् ।
	\pend% ending standard par
      \textsuperscript{\textenglish{227/s}}

	  \pstart \leavevmode% starting standard par
	एव‚न्त‚र्ह्येकं ताव‚त् ज्ञानं नानाकारं स्यादिति श‚ङ्कायां न‚कारं काकाक्षिव‚त् ‚{\tiny $_{lb}$}‚संब‚न्ध‚य‚न्नाह (।)
	\pend% ending standard par
      
	  \bigskip
	  \begingroup
	
	    \large
	  
	    \begin{quote}
	  
	    
	    \stanza[\smallbreak]
	\label{pv.2.359b}\flagstanza{\tiny\textenglish{...2.359b}}नाभेदो रूप‚द‚र्श‚नात् ।&रूपाभेदेपि प‚श्य‚न्ती धीर‚भेदं व्य‚व‚स्य‚ति ॥ ३५९ ॥\&[\smallbreak]


	
	    \end{quote}
	  
	  \endgroup
	

	  \pstart \leavevmode% starting standard par
	\hphantom{.}‚{\color{DodgerBlue3}‚ना\edtext{}{\edlabel{pvv.227-1}\label{pvv.227-1}\lemma{ना}\Bfootnote{अभिन्न‚रूप‚द‚र्श‚ने हि अभेदः स्यात् ।}}भेदो} ज्ञान‚स्याभिबुद्ध‚स्य ‚{\color{DodgerBlue3}‚रूप}\edtext{}{\edlabel{pvv.227-2}\label{pvv.227-2}\lemma{स्य}\Bfootnote{बुद्धेन रूप‚न्न दृश्य‚ते ।}}स्याद‚{\color{DodgerBlue3}‚र्श‚नात्} । रूपाभेद‚ञ्चाद्र‚यादौ ‚{\color{DodgerBlue3}‚प‚श्य‚{\tiny $_{6}$}‚न्ती} धीर‚भेदं ‚{\color{DodgerBlue3}‚व्य‚व‚स्य‚ति} । न च ग्राह्य‚ग्राह‚कादिषु भिन्नाभासेष्व‚भेद‚प्र‚तिभासः । (३५९)
	\pend% ending standard par
      \label{div_pvv.2.360}
	  
	% new div opening: depth here is 2
	

	  \pstart \leavevmode% starting standard par
	त‚स्याद् (।)
	\pend% ending standard par
      
	  \bigskip
	  \begingroup
	
	    \large
	  
	    \begin{quote}
	  
	    
	    \stanza[\smallbreak]
	\label{pv.2.360}\flagstanza{\tiny\textenglish{....2.360}}भावा येन निरूप्य‚न्ते त‚द्रूपं नास्ति त‚त्त्व‚तः ।&य‚स्मादेक‚म‚नेकं च रूपं तेषां न विद्य‚ते ॥ ३६० ॥\&[\smallbreak]


	
	    \end{quote}
	  
	  \endgroup
	

	  \pstart \leavevmode% starting standard par
	\hphantom{.}‚{\color{DodgerBlue3}‚भावा\edtext{}{\edlabel{pvv.227-3}\label{pvv.227-3}\lemma{भावा}\Bfootnote{चित्त‚ग‚ताः ।}}} ग्राह्याद‚यो ‚{\color{DodgerBlue3}‚येन} रूपेण ग्राह्य‚त्वादिना ‚{\color{DodgerBlue3}‚निरूप्य‚न्ते} अनुभूय‚न्ते ‚{\color{DodgerBlue3}‚त‚द्रूपं त‚त्त्व‚त-} स्तेषां ‚{\color{DodgerBlue3}‚नास्ति । य‚स्मादेकं} रूप‚{\color{DodgerBlue3}‚म‚नेक‚ञ्च रूपं तेषां न विद्य‚ते} । व‚स्तुभ‚व‚देक‚म‚नेकं वा ‚{\tiny $_{lb}$}‚स्यात् । न च ग्राह्याद्याभा एकोऽनेको वा युक्तः । त‚स्मादुप्प‚ल‚व एवायं । (३६०)
	\pend% ending standard par
      \label{div_pvv.2.361_2.362_2.363}
	  
	% new div opening: depth here is 2
	

	  \pstart \leavevmode% starting standard par
	न‚नु (।)
	\pend% ending standard par
      
	  \bigskip
	  \begingroup
	
	    \large
	  
	    \begin{quote}
	  
	    
	    \stanza[\smallbreak]
	\label{pv.2.361}\flagstanza{\tiny\textenglish{....2.361}}साध‚र्म्य‚द‚र्श‚नाल्लोके भ्रान्तिर्नामोप‚जाय‚ते ।&अत‚दात्म‚नि तादात्म्य‚व्य‚व‚सायेना नेह त‚त् ॥ ३६१ ॥\&[\smallbreak]


	
	    \end{quote}
	  
	  \endgroup
	
	  \bigskip
	  \begingroup
	
	    \large
	  
	    \begin{quote}
	  
	    
	    \stanza[\smallbreak]
	\label{pv.2.362a}\flagstanza{\tiny\textenglish{...2.362a}}अद‚र्श‚नाज्ज‚ग‚त्य‚स्मिन्नेक‚स्यापि त‚दात्म‚नः ॥\&[\smallbreak]


	
	    \end{quote}
	  
	  \endgroup
	

	  \pstart \leavevmode% starting standard par
	\hphantom{.}‚{\color{DodgerBlue3}‚लोके साध‚र्म्य‚द‚र्श‚नाद‚त‚दात्म\edtext{}{\edlabel{pvv.227-4}\label{pvv.227-4}\lemma{दात्म}\Bfootnote{म‚रीचिषु ज‚ल‚व‚त् ।}}नि तादात्म्य‚व्य‚व‚सायेन भ्रान्तिर्व्वि}‚त‚था-\leavevmode\ledsidenote{\textenglish{44b/MA}} ‚{\tiny $_{lb}$}‚कारा बुद्धि‚{\color{DodgerBlue3}‚र्जाय‚ते} इति नाम प्र‚सिद्धं । ‚{\color{DodgerBlue3}‚इह}‚{\tiny $_{7}$}‚ विज्ञ‚प्तिन‚ये त‚त् साध‚र्म्य‚द‚र्श‚नं ‚{\color{DodgerBlue3}‚ना}‚स्ति । ‚{\tiny $_{lb}$}‚(३६१) ‚{\color{DodgerBlue3}‚एक‚स्यापि त‚दात्म‚नो}‚ऽभूताकार‚स्य ‚{\color{DodgerBlue3}‚ज‚ग‚त्य‚द‚र्श‚नात्} ।
	\pend% ending standard par
      

	  \pstart \leavevmode% starting standard par
	अत्राह (।)
	\pend% ending standard par
      
	  \bigskip
	  \begingroup
	
	    \large
	  
	    \begin{quote}
	  
	    
	    \stanza[\smallbreak]
	\label{pv.2.362b}\flagstanza{\tiny\textenglish{...2.362b}}अस्तीय‚म‚पि या त्व‚न्त‚रुप‚प्ल‚व‚स‚मुद्भ‚वा ॥ ३६२ ॥\&[\smallbreak]


	
	    \end{quote}
	  
	  \endgroup
	
	  \bigskip
	  \begingroup
	
	    \large
	  
	    \begin{quote}
	  
	    
	    \stanza[\smallbreak]
	\label{pv.2.363}\flagstanza{\tiny\textenglish{....2.363}}दोषोद्भ‚वा प्र‚कृत्या सा वित‚थ‚प्र‚तिभासिनी ।&अन‚पेक्षित‚साध‚र्म्य‚दृगादिस्तैमिरादिव‚त् ॥ ३६३ ॥\&[\smallbreak]


	
	    \end{quote}
	  
	  \endgroup
	

	  \pstart \leavevmode% starting standard par
	\hphantom{.}साध‚र्म्य‚द‚र्श‚नाद्या मान‚सी भ्रान्ति‚{\color{DodgerBlue3}‚रिय‚म‚प्य‚स्ति ।\edtext{\textsuperscript{*}}{\edlabel{pvv.227-5}\label{pvv.227-5}\lemma{*}\Bfootnote{इय‚मेवेति तु न निय‚मः । अन‚पेक्षित‚साध‚र्म्याप्य‚स्ति ।}} या पुन‚र‚न्त‚रुप‚प्ल‚व‚स‚मुद्भ‚वा} अविद्याप्र‚भ‚वा सा प्र‚कृत्या दोषोद्भ‚वा । अविद्या तिमिरादिहेतुका वित‚थ‚प्र‚ति‚{\tiny $_{lb}$}‚भासिन्य‚भूताकारा अन‚पेक्षित‚साध‚र्म्य‚दृगादिस्तैमिरादिव‚त् । य‚था तिमिर‚ज्ञान‚{\tiny $_{lb}$}‚\leavevmode\ledsidenote{\textenglish{228/s}} मिन्द्रिय‚दोषाद‚न‚पेक्षित‚साध‚र्म्य‚मेव जाय‚ते त‚था ग्राह्यादिभ्रान्तिर‚पीत्य‚र्थः, ‚{\tiny $_{lb}$}‚जाय‚मान‚मेव हि ज्ञानं वास‚नासा‚{\tiny $_{1}$}‚म‚र्थ्याद‚नुभ‚वान‚नुभ‚व‚त्वेनान्त‚र्ब्ब‚हिर्देश‚त्वेन सुखादि ‚{\tiny $_{lb}$}‚नीलादि द‚र्श‚य‚ति । एवं त‚र्हि ब‚हिर‚र्थोप‚द‚र्श‚न‚श‚क्त‚मेव स्व‚द‚र्श‚न‚हेतुब‚लात् क‚स्मा‚{\tiny $_{lb}$}‚न्नेष्य‚त ज्ञान‚मिति चेत् । न (।) स्व‚प्नादाव‚विद्याया अस‚द्ग्राह्याकारोप‚द‚र्श‚न‚साम‚{\tiny $_{lb}$}‚र्थ्योप‚ल‚ब्धेः । आकारातिरिक्त‚ब‚हिर‚र्थ‚स्याद‚र्श‚नान्न ज्ञान‚स्य त‚दुप‚द‚र्श‚न‚साम‚र्थ्य‚{\tiny $_{lb}$}‚क‚ल्प‚ना न्याय्येत्यास्तां ताव‚दिदं ॥ (३६२, ३६३)
	\pend% ending standard par
      \label{div_pvv.2.364}
	  
	% new div opening: depth here is 2
	

	  \begin{center}%% label @type='head'
	\textbf{भ. विज्ञ‚प्तिमात्र‚तायां प्र‚माण‚फ‚ल‚व्य‚व‚स्था}
	\end{center}
	

	  \pstart \leavevmode% starting standard par
	विज्ञ‚प्तिमात्र‚तायां प्र‚माण‚फ‚ल‚व्य‚व‚स्थाप(ना)र्थ‚मा\edtext{}{\edlabel{pvv.228-1}\label{pvv.228-1}\lemma{मा}\Bfootnote{बाह्य‚विज्ञान‚वादौ द्विराव‚र्त्तितौ भेद‚क‚थ‚नार्थ ।}}ह (।)
	\pend% ending standard par
      
	  \bigskip
	  \begingroup
	
	    \large
	  
	    \begin{quote}
	  
	    
	    \stanza[\smallbreak]
	\label{pv.2.364}\flagstanza{\tiny\textenglish{....2.364}}त‚त्र बुद्धेः प‚रिच्छेदो ग्राह‚काकार‚स‚म्म‚तः ।&त‚दात्म्यादात्म‚वित्त‚स्य स त‚स्य साध‚नं त‚तः ॥ ३६४ ॥\&[\smallbreak]


	
	    \end{quote}
	  
	  \endgroup
	

	  \pstart \leavevmode% starting standard par
	\hphantom{.}‚{\color{DodgerBlue3}‚त‚त्र} विज्ञ‚प्तिमात्र‚तायां ‚{\color{DodgerBlue3}‚बुद्धेः} स‚{\tiny $_{2}$}‚ग्राह‚काकारः ‚{\color{DodgerBlue3}‚संम‚तः प‚रिच्छेद} आकार ‚{\color{DodgerBlue3}‚आत्म‚{\tiny $_{lb}$}‚वित्} त‚स्य ग्राह‚काकार‚स्य फ‚लं । ‚{\color{DodgerBlue3}‚तादात्म्यात्} प‚रिच्छेद‚स्व‚भाव‚त्वात् ‚{\color{DodgerBlue3}‚स्व‚संवित्तिः ‚{\tiny $_{lb}$}‚फ‚ल‚म्वेति सूत्रे स्वाभासं विष‚याभास‚ञ्च ज्ञान‚मुत्प‚द्य‚ते (।) त‚त्र य‚त् स्व‚संवेद‚न‚न्त‚त् ‚{\tiny $_{lb}$}‚फ‚लं विष‚य (:)} प‚रिच्छेदात्म‚क एव हि ग्राह‚काकारः स एव चात्म‚वेद‚नं । ‚{\color{DodgerBlue3}‚त‚तः} प‚रि‚{\tiny $_{lb}$}‚च्छेद‚व‚शेनात्म‚वेद‚न‚व्य‚व‚स्थानात् (।) स प‚रिच्छेद‚{\color{DodgerBlue3}‚स्त‚स्या}‚त्म‚विदः\edtext{}{\edlabel{pvv.228-2}\label{pvv.228-2}\lemma{विदः}\Bfootnote{विज्ञ‚प्तौ ग्राह्यांशो मेयो यः पूर्व्व प्र‚माण‚मुक्तः । ग्राह‚कांशः प्र‚मा त‚दुभ‚य‚संवेद‚नं फ‚लं ।}} फ‚ल‚भूतायाः ‚{\tiny $_{lb}$}‚‚{\color{DodgerBlue3}‚साध‚नं} प्र‚माणं । (३६४)
	\pend% ending standard par
      \label{div_pvv.2.365}
	  
	% new div opening: depth here is 2
	
	  \bigskip
	  \begingroup
	
	    \large
	  
	    \begin{quote}
	  
	    
	    \stanza[\smallbreak]
	\label{pv.2.365}\flagstanza{\tiny\textenglish{....2.365}}त‚त्रात्म‚विष‚ये माने य‚था रागादिवेद‚न‚म् ।&इयं स‚र्व्व‚त्र संयोज्या मान‚मेय‚फ‚ल‚स्थितिः ॥ ३६५ ॥\&[\smallbreak]


	
	    \end{quote}
	  
	  \endgroup
	

	  \pstart \leavevmode% starting standard par
	\hphantom{.}‚{\color{DodgerBlue3}‚त‚त्र} एवं स‚त्या‚{\color{DodgerBlue3}‚त्म‚विष‚ये माने} स्व‚संवेद‚ने प्र‚माणे ‚{\color{DodgerBlue3}‚य‚था रागादिवेद‚नं\edtext{}{\edlabel{pvv.228-3}\label{pvv.228-3}\lemma{नं}\Bfootnote{स्व‚विन्निष्ठं ।}}} मान‚मेय‚{\tiny $_{lb}$}‚फ‚लात्म‚कं । ‚{\color{DodgerBlue3}‚इयं मान‚मेय‚फ‚ल‚स्थितिः} स‚र्व्व‚त्र विज्ञ‚प्तिन‚येपि सं‚{\tiny $_{3}$}‚योज्या । (३६५)
	\pend% ending standard par
      \label{div_pvv.2.366}
	  
	% new div opening: depth here is 2
	

	  \pstart \leavevmode% starting standard par
	त‚था हि (।)
	\pend% ending standard par
      
	  \bigskip
	  \begingroup
	
	    \large
	  
	    \begin{quote}
	  
	    
	    \stanza[\smallbreak]
	\label{pv.2.366}\flagstanza{\tiny\textenglish{....2.366}}त‚त्राप्य‚नुभ‚वात्म‚त्वात्ते योग्या स्वात्म‚संविदि ।&इति सा योग्य‚ता मान‚मात्मा मेयः फ‚लं स्व‚वित् ॥ ३६६ ॥\&[\smallbreak]


	
	    \end{quote}
	  
	  \endgroup
	

	  \pstart \leavevmode% starting standard par
	\hphantom{.}‚{\color{DodgerBlue3}‚त‚त्रापि} रागादिवेद‚नेपि ‚{\color{DodgerBlue3}‚ते} रागाद‚योऽ‚{\color{DodgerBlue3}‚नुभ‚वात्म‚त्वात् स्वा}‚त्म‚नः ‚{\color{DodgerBlue3}‚संविदि ‚{\tiny $_{lb}$}‚योग्या इति} त‚स्मात् ‚{\color{DodgerBlue3}‚सा} स्व‚वेद‚न‚{\color{DodgerBlue3}‚योग्य‚ता} रागादीनां ‚{\color{DodgerBlue3}‚मान‚मा\edtext{}{\edlabel{pvv.228-4}\label{pvv.228-4}\lemma{मा}\Bfootnote{अनुभ‚व‚नीयः ।}}त्मा मेयः फ‚लं स्व‚वित्} । ‚{\tiny $_{lb}$}‚(३६६)
	\pend% ending standard par
      \textsuperscript{\textenglish{229/s}}\label{div_pvv.2.367}
	  
	% new div opening: depth here is 2
	

	  \pstart \leavevmode% starting standard par
	न‚नु रागादिष्टात्म‚संवेद‚नं फ‚ल‚भूतं मान‚युक्त‚मिह तु ग्राह‚काकारो योग्य‚ता‚{\tiny $_{lb}$}‚ल‚क्ष‚णं प्र‚माण‚मुच्य‚त इति व्या\edtext{}{\edlabel{pvv.229-1}\label{pvv.229-1}\lemma{व्या}\Bfootnote{दिग्नागेनोक्तं रागादिषु च स्व‚स‚म्वेद‚न‚मिन्द्रियान‚पेक्ष‚त्वात् मान‚सं प्र‚त्य‚क्षं । पुनः ग्राह‚काकार‚स‚म्वित्ती प्र‚माण‚फ‚ल‚ते इति ग्राह‚काकारः प्र‚माणं । वार्त्तिके तु योग्य‚तोक्ता । आह एकाभिप्राय‚तां ।}}ह‚त‚मिति श‚ङ्कायामुत्त‚रं (।)
	\pend% ending standard par
      
	  \bigskip
	  \begingroup
	
	    \large
	  
	    \begin{quote}
	  
	    
	    \stanza[\smallbreak]
	\label{pv.2.367}\flagstanza{\tiny\textenglish{....2.367}}ग्राह‚काकार‚संख्याता प‚रिच्छेदात्म‚तात्म‚नि ।&सा योग्य‚तेति च प्रोक्तं प्र‚माणं स्वात्म‚वेद‚न‚म् ॥ ३६७ ॥\&[\smallbreak]


	
	    \end{quote}
	  
	  \endgroup
	

	  \pstart \leavevmode% starting standard par
	\hphantom{.}या ‚{\color{DodgerBlue3}‚ग्राह‚काकार‚संख्याता प‚रिच्छेदात्म‚ता साऽत्म‚नि} संवेद‚ने ‚{\color{DodgerBlue3}‚योग्य‚ता चेति} त‚स्माद्रागादिषु ‚{\color{DodgerBlue3}‚स्वा‚{\tiny $_{4}$}‚त्म‚संवेद‚नं प्र‚माणं प्रोक्तं । न हि स्व‚वेद‚नं} फ‚ल‚भूत‚मिह विव‚क्षितं ‚{\tiny $_{lb}$}‚किन्तु योग्य‚ता । त‚स्माद‚भिन्नार्थ‚तैव । (३६७)
	\pend% ending standard par
      \label{div_pvv.2.368}
	  
	% new div opening: depth here is 2
	

	  \pstart \leavevmode% starting standard par
	न‚नु स्वाभासं ताव‚द‚स्तु ज्ञानं प‚रिच्छेद‚स्व\edtext{}{\edlabel{pvv.229-2}\label{pvv.229-2}\lemma{स्व}\Bfootnote{ज्ञानं त‚ज्‏ज्ञान‚विशेषात्तु द्विरूप‚ता ज्ञान‚स्य ।}}भाव‚त्वात् । विष‚याभासं तु क‚थं । ‚{\tiny $_{lb}$}‚य‚था विष‚यं कार‚णं त‚था च‚क्षुरादिर‚पीति त‚दाकार‚तापि स्यादित्याह (।)
	\pend% ending standard par
      
	  \bigskip
	  \begingroup
	
	    \large
	  
	    \begin{quote}
	  
	    
	    \stanza[\smallbreak]
	\label{pv.2.368}\flagstanza{\tiny\textenglish{....2.368}}स‚र्व्व‚मेव हि विज्ञानं विष‚येभ्यः स‚मुद्भ‚व‚द् ।&त‚द‚न्य‚स्यापि हेतुत्वे क‚थ‚ञ्चिद् विष‚याकृति ॥ ३६८ ॥\&[\smallbreak]


	
	    \end{quote}
	  
	  \endgroup
	

	  \pstart \leavevmode% starting standard par
	\hphantom{.}‚{\color{DodgerBlue3}‚स‚र्व‚मेव हि ज्ञानं विष‚येभ्यः स‚मुद्भ‚व‚दु}‚त्प‚द्य‚मानं तेभ्यो विष‚येभ्योऽन्य‚स्येन्द्रिया‚{\tiny $_{lb}$}‚देर‚पि हेतुत्वे क‚थ‚ञ्चित् स्व‚कार‚णायात‚विष‚य‚ग‚त‚श‚क्ति‚{\tiny $_{5}$}‚भेदा‚{\color{DodgerBlue3}‚द्विष‚याकृति} भ\edtext{}{\edlabel{pvv.229-3}\label{pvv.229-3}\lemma{भ}\Bfootnote{आल‚म्ब‚न‚च्छाय‚मिति व‚स्तुध‚र्म‚ता ।}}व‚ति ‚{\tiny $_{lb}$}‚नेन्द्रियाद्याकारं । (३६८)
	\pend% ending standard par
      \label{div_pvv.2.369}
	  
	% new div opening: depth here is 2
	
	  \bigskip
	  \begingroup
	
	    \large
	  
	    \begin{quote}
	  
	    
	    \stanza[\smallbreak]
	\label{pv.2.369}\flagstanza{\tiny\textenglish{....2.369}}य‚थैवाहार‚कालादेर्हेतुत्वेऽप‚त्य‚ज‚न्म‚नि ।&पित्रोस्त‚देक‚स्याकारं ध‚त्ते नान्य‚स्य क‚स्य‚चित् ॥ ३६९ ॥\&[\smallbreak]


	
	    \end{quote}
	  
	  \endgroup
	

	  \pstart \leavevmode% starting standard par
	\hphantom{.}‚{\color{DodgerBlue3}‚य‚थैवाप‚त्य‚ज‚न्म‚नि} पित्रोर्माता‚{\color{DodgerBlue3}‚पित्रोराहार‚कालादे}\edtext{}{\edlabel{pvv.229-4}\label{pvv.229-4}\lemma{पित्रोर्माता}\Bfootnote{आदिनाऽदृष्टादिः ।}}श्च हेतुत्वेपि त‚देक‚स्य ‚{\tiny $_{lb}$}‚पित्रोर्म्म‚ध्ये एक‚स्य पितुर्म्मातुर्व्वा‚{\color{DodgerBlue3}‚कार}‚म‚प‚त्यं ‚{\color{DodgerBlue3}‚ध‚त्तेऽन्य‚स्या}‚हार‚कालादेश्च ‚{\color{DodgerBlue3}‚क‚स्य‚{\tiny $_{lb}$}‚चित्} (न) । (३६९)
	\pend% ending standard par
      \label{div_pvv.2.370}
	  
	% new div opening: depth here is 2
	
	  \bigskip
	  \begingroup
	
	    \large
	  
	    \begin{quote}
	  
	    
	    \stanza[\smallbreak]
	\label{pv.2.370}\flagstanza{\tiny\textenglish{....2.370}}त‚द्धेतुत्वेन तुल्येपि त‚द‚न्यैर्विष‚ये म‚त‚म् ।&विष‚य‚त्वं त‚दंशेन त‚द‚भावे न त‚द् भ‚वेत् ॥ ३७० ॥\&[\smallbreak]


	
	    \end{quote}
	  
	  \endgroup
	

	  \pstart \leavevmode% starting standard par
	\hphantom{.}य‚त एवं ‚{\color{DodgerBlue3}‚त‚त्} त‚स्मात् त‚तो विष‚याद‚न्यैरिन्द्रियादिभि‚{\color{DodgerBlue3}‚र्हेतुत्वेन तुल्ये} स‚दृशेपि ‚{\tiny $_{lb}$}‚विष‚ये रूपादौ ‚{\color{DodgerBlue3}‚विष‚य‚त्वं} ग्राह्य‚त्वं तेन स्वाकारार्प‚क‚त्वेनां‚{\color{DodgerBlue3}‚शेन} विशेष‚णं ‚{\color{DodgerBlue3}‚म‚तं} । य‚स्मादा‚{\tiny $_{lb}$}‚\leavevmode\ledsidenote{\textenglish{230/s}} कारार्प‚{\tiny $_{6}$}‚क‚त्वं विष‚य‚ल‚क्ष‚ण‚म‚व‚स्थितं । त‚स्मात्त‚स्याकारार्प‚क‚त्व‚स्याभावे त‚द्विष‚य‚त्व‚{\tiny $_{lb}$}‚मिन्द्रियादिषु ‚{\color{DodgerBlue3}‚न भ‚वेत्} । (३७०)
	\pend% ending standard par
      \label{div_pvv.2.371}
	  
	% new div opening: depth here is 2
	

	  \pstart \leavevmode% starting standard par
	किञ्च (।)
	\pend% ending standard par
      
	  \bigskip
	  \begingroup
	
	    \large
	  
	    \begin{quote}
	  
	    
	    \stanza[\smallbreak]
	\label{pv.2.371}\flagstanza{\tiny\textenglish{....2.371}}अन‚र्थाकार‚श‚ङ्का स्याद‚प्य‚र्थ‚व‚ति चेत‚सि ।&अतीतार्थ‚ग्र‚हे सिद्धे द्विरूप‚त्वात्म‚वेद‚ने ॥ ३७१ ॥\&[\smallbreak]


	
	    \end{quote}
	  
	  \endgroup
	

	  \pstart \leavevmode% starting standard par
	\hphantom{.}‚{\color{DodgerBlue3}‚अर्थ‚व‚ति चेत\edtext{}{\edlabel{pvv.230-1}\label{pvv.230-1}\lemma{चेत}\Bfootnote{साकार‚त्वात्त‚तोन्यं भाविक‚म‚र्थं क‚ल्प‚य‚तः किम‚यं ज्ञानीय एवार्थो ब‚हिर्व्वा येन ज्ञान‚म‚न‚र्थ‚कं स्यात् ।}}स्य‚न‚र्थाकार}‚ता अर्थाकार‚र‚हित‚ता श‚ङ्का ‚{\color{DodgerBlue3}‚स्याद‚पि} दृश्य‚मान‚{\tiny $_{lb}$}‚स्याकार‚स्यार्थ‚त्वेनैव स‚म्भाव्य‚मान‚त्वात् । ‚{\color{DodgerBlue3}‚अतीत}‚स्या‚{\color{DodgerBlue3}‚र्थ‚स्य ग्र‚हे} विक‚ल्पात्म‚के त्व‚र्थ‚स्या\leavevmode\ledsidenote{\textenglish{45a/MA}}‚{\tiny $_{lb}$}‚भा\edtext{}{\edlabel{pvv.230-2}\label{pvv.230-2}\lemma{भा}\Bfootnote{सिद्धा ज्ञान‚स्यैव द्व्याभास‚ता ।}} वात् । अर्थाभास‚त‚या च ‚{\color{DodgerBlue3}‚द्विरूप‚त्वं} त‚था चाप‚रोक्ष‚त्वादा‚{\color{DodgerBlue3}‚त्म‚संवेद‚नं} चेति द्वे अप्य‚स्तु ‚{\tiny $_{lb}$}‚सिद्धे । न ह्य‚स‚न्नेवार्थो दृश्य‚ते । त‚तोऽर्थाकार‚ता सा ज्ञान‚स्यैव‚{\tiny $_{7}$}‚ त‚था\edtext{}{\edlabel{pvv.230-3}\label{pvv.230-3}\lemma{था}\Bfootnote{बुद्धेद्विरूप‚त्व‚मात्म‚वेद‚न‚ञ्च ते सिद्धे ।}}र्थेन च प‚रो‚{\tiny $_{lb}$}‚क्ष‚त्वाभावात् स्वाभास‚ञ्च त‚त् । न चान्येन ज्ञानेन त‚द् वेद्य‚ते । वेद्य‚वेद‚क‚भाव‚स्य ‚{\tiny $_{lb}$}‚निषिद्ध‚त्वात् । अतः स्व‚वेद‚न‚मेव त‚त् । (३७१)
	\pend% ending standard par
      \label{div_pvv.2.372}
	  
	% new div opening: depth here is 2
	

	  \begin{center}%% label @type='head'
	\textbf{ब. सामान्य‚स्य नित्यानित्य‚त्व‚प्र‚तिषेधः}
	\end{center}
	

	  \pstart \leavevmode% starting standard par
	स्यादेत‚द् (।) अतीता व्य‚क्तिर‚स‚त्त्वान्न विक‚ल्प‚विष‚यो जातिस्यु स‚त्त्वात् ‚{\tiny $_{lb}$}‚स्यादित्याह (।)
	\pend% ending standard par
      
	  \bigskip
	  \begingroup
	
	    \large
	  
	    \begin{quote}
	  
	    
	    \stanza[\smallbreak]
	\label{pv.2.372}\flagstanza{\tiny\textenglish{....2.372}}नीलाद्याभास‚भेदित्वान्नार्थो जातिर‚त‚द्व‚ती ।&सा वा-नित्या न जातिः स्यान्नित्या वा ज‚निका क‚थ‚म् ॥ ३७२ ॥\&[\smallbreak]


	
	    \end{quote}
	  
	  \endgroup
	

	  \pstart \leavevmode% starting standard par
	\hphantom{.}अतीत‚विक‚ल्प‚स्य ‚{\color{DodgerBlue3}‚नीलाद्याभास‚भेदित्वात्} व‚र्ण्ण‚संस्थानाद्याकार‚विशेष‚व‚त्त्वात् ‚{\tiny $_{lb}$}‚‚{\color{DodgerBlue3}‚न जातिर‚त‚द्व‚ती} व‚र्ण्णाद्याकार‚र‚हितार्थेर्विष‚यः । किञ्च (।) जातिर्व्विष‚यीभ‚व‚न्ती ‚{\tiny $_{lb}$}‚अनित्या वा भ‚वेत् नित्या वा । ‚{\color{DodgerBlue3}‚सा चानित्या जातिर्न स्यात्} नि‚{\tiny $_{1}$}‚त्य‚ल‚क्ष‚ण‚त्वा‚{\tiny $_{lb}$}‚त्त‚स्याः ‚{\color{DodgerBlue3}‚नित्या} वा बुद्धे‚{\color{DodgerBlue3}‚र्ज‚निका क‚थं} स्यात् । (३७२)
	\pend% ending standard par
      \label{div_pvv.2.373}
	  
	% new div opening: depth here is 2
	

	  \pstart \leavevmode% starting standard par
	नित्य‚स्य क्र‚म‚यौग‚प‚द्याभ्यां विर‚हात् ॥ ना\edtext{}{\edlabel{pvv.230-4}\label{pvv.230-4}\lemma{ना}\Bfootnote{संज्ञा नाम । अर्थ‚स‚रूप‚न्निमित्तं ।}}म‚निमित्ते विप्र‚युक्तः संस्कारो विष‚{\tiny $_{lb}$}‚य‚श्चेदित्याह (।)
	\pend% ending standard par
      
	  \bigskip
	  \begingroup
	
	    \large
	  
	    \begin{quote}
	  
	    
	    \stanza[\smallbreak]
	\label{pv.2.373}\flagstanza{\tiny\textenglish{....2.373}}नामादिकं निषिद्धं प्राङ् नाय‚म‚र्थ‚व‚तां क्र‚मः ।&इच्छामात्रानुरोधित्वाद‚र्थ‚श‚क्तिर्न सिध्य‚ति ॥ ३७३ ॥\&[\smallbreak]


	
	    \end{quote}
	  
	  \endgroup
	\textsuperscript{\textenglish{231/s}}

	  \pstart \leavevmode% starting standard par
	\hphantom{.}‚{\color{DodgerBlue3}‚नामादिकं} विष‚य‚त्वेन ‚{\color{DodgerBlue3}‚निषिद्धं प्राक्} \href{http://sarit.indology.info/?cref=pv.2.11}{(२।११)} नामादिव‚च‚ने व‚क्तृश्रोतृवाच्यानु‚{\tiny $_{lb}$}‚ब‚न्धिनी \href{http://sarit.indology.info/?cref=pv.2.11}{(२।११)} त्यादिना । किञ्चा‚{\color{DodgerBlue3}‚र्थ‚व‚तां} स‚विष‚याणां च‚क्षुर्व्विज्ञानादीना‚{\color{DodgerBlue3}‚म‚यं ‚{\tiny $_{lb}$}‚क्र‚मो}‚ऽर्थ‚साम‚र्थ्येन विनोत्पाद इति नास्ति । म‚नोविज्ञानानां त्व‚र्थ‚स‚न्निधानान‚पेक्षाणा‚{\tiny $_{lb}$}‚‚{\color{DodgerBlue3}‚मिच्छामात्रानुरोधित्वात्} ज‚निका‚{\color{DodgerBlue3}‚र्थ‚{\tiny $_{2}$}‚श‚क्तिर्न सिध्य‚ति} । न चाहेतुर‚र्थी ग्राह्यो‚{\tiny $_{lb}$}‚ऽतिप्र‚स‚ङ्गात् । (३७३)
	\pend% ending standard par
      \label{div_pvv.2.374}
	  
	% new div opening: depth here is 2
	

	  \begin{center}%% label @type='head'
	\textbf{(२) स्तृतिचिन्ता}
	\end{center}
	‚{\tiny $_{lb}$}‚

	  \pstart \leavevmode% starting standard par
	भ‚व‚त्व‚तीतार्थ‚ल‚म्ब‚नं विज्ञानं विष‚याकार‚म‚नुभ‚व‚स्तु मा भूदित्याह (।)
	\pend% ending standard par
      
	  \bigskip
	  \begingroup
	
	    \large
	  
	    \begin{quote}
	  
	    
	    \stanza[\smallbreak]
	\label{pv.2.374}\flagstanza{\tiny\textenglish{....2.374}}स्मृतिश्चेदृग्‏विधं ज्ञानं त‚स्याश्चानुभ‚वाद्भ‚वः ।&स चार्थाकार‚र‚हितः सेदानीं त‚द्व‚ती क‚थ‚म् ॥ ३७४ ॥\&[\smallbreak]


	
	    \end{quote}
	  
	  \endgroup
	

	  \pstart \leavevmode% starting standard par
	\hphantom{.}‚{\color{DodgerBlue3}‚ईदृग्‏विध‚म}‚तीत‚विक‚ल्प‚नात्म‚कं ‚{\color{DodgerBlue3}‚ज्ञान‚ञ्च} स्मृतिः । ‚{\color{DodgerBlue3}‚त‚स्याश्चानुभ‚वाद् भ‚व} उत्पादः । न ह्य‚नुभ‚व‚म‚न्त‚रेण स्मृतिः । ‚{\color{DodgerBlue3}‚स चा}‚नुभ‚वो भ‚व‚न्म‚ते‚{\color{DodgerBlue3}‚ऽर्था\edtext{}{\edlabel{pvv.231-1}\label{pvv.231-1}\lemma{ऽर्था}\Bfootnote{निराकार‚वादित्वात् ।}}कार‚र‚हितः । ‚{\tiny $_{lb}$}‚इदानी}‚म‚स्मिन्न‚भ्युप‚ग‚मे सा स्मृति‚{\color{DodgerBlue3}‚स्त‚द्व‚त्य}‚र्थाकार‚व‚ती ‚{\color{DodgerBlue3}‚क‚थ‚म}‚स्तु । य‚द्य‚नुभ‚वारूढो ‚{\tiny $_{lb}$}‚नार्थाकारः क‚थ‚म‚सौ विदितः । अविदित‚स्य का स्मृतिः । (३७४)
	\pend% ending standard par
      \label{div_pvv.2.375}
	  
	% new div opening: depth here is 2
	

	  \begin{center}%% label @type='head'
	\textbf{क. नार्थात् स्मृतिः}
	\end{center}
	

	  \pstart \leavevmode% starting standard par
	अर्था‚{\tiny $_{3}$}‚त्स्मृतिरुत्प‚द्य‚त इति चेत् । आह (।)
	\pend% ending standard par
      
	  \bigskip
	  \begingroup
	
	    \large
	  
	    \begin{quote}
	  
	    
	    \stanza[\smallbreak]
	\label{pv.2.375}\flagstanza{\tiny\textenglish{....2.375}}नार्थाद् भाव‚स्त‚दाभावात् स्यात्त‚थानुभ‚वेपि सः ।&आकारः स च नार्थ‚स्य स्प‚ष्टाकार‚विवेक‚तः ॥ ३७५ ॥\&[\smallbreak]


	
	    \end{quote}
	  
	  \endgroup
	

	  \pstart \leavevmode% starting standard par
	\hphantom{.}‚{\color{DodgerBlue3}‚नार्थाद् भाव}‚स्त‚स्या‚{\color{DodgerBlue3}‚स्त‚दा}\edtext{}{\edlabel{pvv.231-2}\label{pvv.231-2}\lemma{स्या}\Bfootnote{सिद्ध‚म‚तः साकारं स्म‚र‚णं ।}} स्मृतिकाले‚{\color{DodgerBlue3}‚ऽभावा}‚द‚तीतार्थ‚स्य ।\edtext{\textsuperscript{*}}{\edlabel{pvv.231-3}\label{pvv.231-3}\lemma{*}\Bfootnote{अभ्युप‚ग‚म्य निराकार‚वादिनि प्र‚स‚ङ्ग‚माह ।}} य‚था चार्थादुत्प‚द्य‚{\tiny $_{lb}$}‚मानायाः स्मृतेर‚र्थाकारो भ‚व‚ति । ‚{\color{DodgerBlue3}‚त‚थानुभ‚वेप्य}‚र्थादुत्प‚द्य‚माने ‚{\color{DodgerBlue3}‚सोऽर्थाकारः} स्यात्(।) ‚{\tiny $_{lb}$}‚अपि च स्मृत्यारूढः ‚{\color{DodgerBlue3}‚सोऽ}‚स्प‚ष्ट‚{\color{DodgerBlue3}‚श्चा}‚कारो ‚{\color{DodgerBlue3}‚नार्थ}‚स्या\edtext{}{\edlabel{pvv.231-4}\label{pvv.231-4}\lemma{स्या}\Bfootnote{युक्तो (?) स्मृतौ नीलाद्यः ।}}नुभ‚वारूढात् ‚{\color{DodgerBlue3}‚स्प‚ष्टाद‚र्थाकाराद् ‚{\tiny $_{lb}$}‚विवेक‚तो} भेदात् । (३७५)
	\pend% ending standard par
      \label{div_pvv.2.376}
	  
	% new div opening: depth here is 2
	
	  \bigskip
	  \begingroup
	
	    \large
	  
	    \begin{quote}
	  
	    
	    \stanza[\smallbreak]
	\label{pv.2.376}\flagstanza{\tiny\textenglish{....2.376}}व्य‚तिरिक्तं त‚दाकारं प्र‚तीयाद‚प‚र‚स्त‚था ।&नित्य‚मात्म‚नि स‚म्ब‚न्धे प्र‚तीयात् क‚थित‚ञ्च न ॥ ३७६ ॥\&[\smallbreak]


	
	    \end{quote}
	  
	  \endgroup
	

	  \pstart \leavevmode% starting standard par
	\hphantom{.}य‚दि च बुद्धिव्य‚तिरिक्तोऽर्थ एव म‚नोविज्ञान‚ग्राह्य‚स्त‚दा बुद्धे‚{\color{DodgerBlue3}‚र्व्य‚तिरिक्त‚मे}‚केन ‚{\tiny $_{lb}$}‚विक‚ल्प्य‚मान‚{\color{DodgerBlue3}‚न्त‚दाकार‚म‚प‚रो}‚पि‚{\tiny $_{4}$}‚ योग्य‚देश‚स्थः प्र‚मा न प्र‚तीयात् । प‚रोप‚ल‚भ्य‚तां ‚{\tiny $_{lb}$}‚\leavevmode\ledsidenote{\textenglish{232/s}} निषेद्धुं ‚{\color{DodgerBlue3}‚नित्य‚मात्म‚न्य‚स्य} विक‚ल्प्य‚स्यार्थ‚स्य ‚{\color{DodgerBlue3}‚स‚म्ब‚न्धे} वा\edtext{}{\edlabel{pvv.232-1}\label{pvv.232-1}\lemma{वा}\Bfootnote{प्राप्य‚कारित्वात्कार्य‚स्य ।}} स्वीक्रिय‚माणे य‚दा ‚{\tiny $_{lb}$}‚स्व‚य‚म‚नुचिन्त‚यितुं प‚र\edtext{}{\edlabel{pvv.232-2}\label{pvv.232-2}\lemma{र}\Bfootnote{किञ्चिन्त‚य‚सीति पृष्टे ।}}स्मै क‚थ्य‚ते त‚दा क‚थ‚ञ्चित् प‚रो न ‚{\color{DodgerBlue3}‚प्र‚तीयात्} । प्र‚त्येति च ‚{\tiny $_{lb}$}‚‚{\color{DodgerBlue3}‚क‚थितं} (३७६)
	\pend% ending standard par
      \label{div_pvv.2.377}
	  
	% new div opening: depth here is 2
	
	  \bigskip
	  \begingroup
	
	    \large
	  
	    \begin{quote}
	  
	    
	    \stanza[\smallbreak]
	\label{pv.2.377}\flagstanza{\tiny\textenglish{....2.377}}एकैकेनाभिस‚म्ब‚न्धे प्र‚तिस‚न्धिर्न युज्य‚ते ।&एकार्थाभिनिवेशात्मा प्र‚व‚क्तृश्रोतृचेत‚सोः ॥ ३७७ ॥\&[\smallbreak]


	
	    \end{quote}
	  
	  \endgroup
	

	  \pstart \leavevmode% starting standard par
	\hphantom{.}प्र‚तीत्य‚र्थ‚{\color{DodgerBlue3}‚मेकेनैकेन} व‚क्त्रा श्रोत्रा च भिन्न‚स्य भिन्न‚स्य अर्थ‚स्या‚{\color{DodgerBlue3}‚भिस‚म्ब‚न्धे-} ऽत्रा\edtext{}{\edlabel{pvv.232-3}\label{pvv.232-3}\lemma{ऽत्रा}\Bfootnote{स‚र्वः पुरुषः प्र‚त्यात्मैकैक‚संग‚तः प‚र‚तः श्र‚व‚णेपि स्वार्थाबाधी ।}}पि स्वीक्रिमाणे ‚{\color{DodgerBlue3}‚प्र‚व‚क्तृश्रोतृचेत‚सोः प्र‚तिस‚न्धिरेकार्थाभिनिवे\edtext{}{\edlabel{pvv.232-4}\label{pvv.232-4}\lemma{न्धिरेकार्थाभिनिवे}\Bfootnote{य‚म‚ल‚क‚योरिव ।}}शात्} य‚देवानेन ‚{\tiny $_{lb}$}‚क‚थितं त‚देव म‚या प्र‚तीतं य‚देव म‚या क‚थि‚{\tiny $_{5}$}‚तं\edtext{}{\edlabel{pvv.232-5}\label{pvv.232-5}\lemma{तं}\Bfootnote{किं चिन्त‚य‚सीति पृष्टे ।}} त‚देवानेन ज्ञात‚मित्य‚भिन्नार्थाध्य‚{\tiny $_{lb}$}‚व‚साय‚रूपो ‚{\color{DodgerBlue3}‚न युज्य‚ते} । (३७७)
	\pend% ending standard par
      \label{div_pvv.2.378}
	  
	% new div opening: depth here is 2
	
	  \bigskip
	  \begingroup
	
	    \large
	  
	    \begin{quote}
	  
	    
	    \stanza[\smallbreak]
	\label{pv.2.378}\flagstanza{\tiny\textenglish{....2.378}}त‚देक‚व्य‚व‚हार‚श्चेत् सादृश्याद‚त‚दाभ‚योः ।&भिन्नात्मार्थः क‚थं ग्राह्य‚स्त‚दा स्याद्धीर‚न‚र्थिका ॥ ३७८ ॥\&[\smallbreak]


	
	    \end{quote}
	  
	  \endgroup
	

	  \pstart \leavevmode% starting standard par
	\hphantom{.}व‚क्तृश्रोतृस‚म्ब‚न्धिनोस्त‚योर‚र्थ‚योरे‚{\color{DodgerBlue3}‚क‚व्य‚व‚हार} एक‚त्वाव‚सायः ‚{\color{DodgerBlue3}‚सादृश्याच्चेत्} । ‚{\tiny $_{lb}$}‚त‚दुभ‚य‚द‚र्श‚ने सादृश्यं न चात्र द‚र्श‚नं । भ‚व‚तु वा त‚थापि अत‚दाभ‚यो‚{\color{DodgerBlue3}‚र्भिन्नात्म-} स‚म्ब‚न्धार्थाप्र‚तिभासिनोर्व्व‚क्तृश्रोतृ चेत‚सो‚{\color{DodgerBlue3}‚र्भिन्नात्माऽर्थः \edtext{}{\edlabel{pvv.232-6}\label{pvv.232-6}\lemma{र्भिन्नात्माऽर्थः}\Bfootnote{भेदेनाभानात् ।}}क‚थं ग्राह्यः} । य‚दा चार्थाभेदो ‚{\tiny $_{lb}$}‚नास्ति ‚{\color{DodgerBlue3}‚त‚दा}‚ऽभिन्नार्थाध्य‚व‚सायिनी ‚{\color{DodgerBlue3}‚धी}‚र्व्व‚क्तृश्रोत्रो‚{\color{DodgerBlue3}‚र‚न‚र्थिका स्यात्} । य‚थार्थं बुद्धेर‚{\tiny $_{lb}$}‚भावात् । य‚थाबुद्धि चार्थाभावात् । (३७८)
	\pend% ending standard par
      \label{div_pvv.2.379}
	  
	% new div opening: depth here is 2
	

	  \pstart \leavevmode% starting standard par
	भ‚व‚तु ताव‚देवं स्मृतिर्व्विष‚याकाराऽनुभ‚व‚ज्ञानं‚{\tiny $_{6}$}‚ त्व‚नाकृति स्यादित्या\edtext{}{\edlabel{pvv.232-7}\label{pvv.232-7}\lemma{स्यादित्या}\Bfootnote{विष‚य‚ज्ञाने त‚ज्ज्ञानाद्याच‚ष्टेनुभ‚व‚विज्ञान‚म् ।}}ह (।)
	\pend% ending standard par
      
	  \bigskip
	  \begingroup
	
	    \large
	  
	    \begin{quote}
	  
	    
	    \stanza[\smallbreak]
	\label{pv.2.379}\flagstanza{\tiny\textenglish{....2.379}}त‚च्चानुभ‚व‚विज्ञानेनुोभ‚यांशाव‚लंबिना ।&एकाकार‚विशेषेण त‚ज्ज्ञानेनानुबुध्य‚ते ॥ ३७९ ॥\&[\smallbreak]


	
	    \end{quote}
	  
	  \endgroup
	

	  \pstart \leavevmode% starting standard par
	\hphantom{.}‚{\color{DodgerBlue3}‚त‚च्चानुभ\edtext{}{\edlabel{pvv.232-8}\label{pvv.232-8}\lemma{च्चानुभ}\Bfootnote{प्राचीन ।}}}‚विष‚येण स्म‚र‚णेनानुब \edtext{}{\edlabel{pvv.232-9}\label{pvv.232-9}\lemma{णेनानुब}\Bfootnote{विष‚याकार‚ता नेष्टा प‚रेणेति द्विरूप‚ता साध्य‚ते । श‚रीरान्त‚र्व‚र्त्तिनो य‚था ।}}ध्य‚तेऽनुग‚म्य‚ते स्म‚र्य‚ते इति याव‚त् ।
	\pend% ending standard par
      

	  \pstart \leavevmode% starting standard par
	\hphantom{.}कीदृशेनो‚{\color{DodgerBlue3}‚भ‚यांशाव‚ल‚म्बिना} ज्ञेय‚प्राचीन‚ज्ञान‚ग‚त‚विष‚याकारानुभ‚व‚रूप‚ध‚र्म‚द्व‚य‚{\tiny $_{lb}$}‚विष‚येण । क‚थ‚{\color{DodgerBlue3}‚म‚नुब‚ध्य‚त} इत्याह (।)
	\pend% ending standard par
      

	  \pstart \leavevmode% starting standard par
	\hphantom{.}‚{\color{DodgerBlue3}‚एक}‚स्माद‚विवाद‚सिद्धाद‚नुभ‚वा‚{\color{DodgerBlue3}‚काराद्विशेंषो} विष‚याकार‚स्तेन स्म‚र‚ण‚ज्ञानेन ‚{\tiny $_{lb}$}‚हि विज्ञान‚म‚र्याकारानुभ‚वाकार‚विशिष्ट‚मेव स्म‚र्य‚ते । त‚त‚स्तादृश‚मेव त‚त् । (३७९)
	\pend% ending standard par
      \label{div_pvv.2.380}
	  
	% new div opening: depth here is 2
	\textsuperscript{\textenglish{233/s}}
	  \bigskip
	  \begingroup
	
	    \large
	  
	    \begin{quote}
	  
	    
	    \stanza[\smallbreak]
	\label{pv.2.380}\flagstanza{\tiny\textenglish{....2.380}}अन्य‚था ह्य‚त‚थारूपं क‚थं ज्ञानेधिरोह‚ति ।&एकाकारोत्त‚रं ज्ञानं त‚था ह्युत्त‚र‚मुत्त‚र‚म् ॥ ३८० ॥\&[\smallbreak]


	
	    \end{quote}
	  
	  \endgroup
	

	  \pstart \leavevmode% starting standard par
	\hphantom{.}अन्य‚थाऽत‚थारूप‚माकार‚द्व‚य‚र‚हित‚म‚नुभ‚व‚ज्ञानं स्व‚ग्राहिणि स्म‚र‚ण‚ज्ञाने ‚{\color{DodgerBlue3}‚क‚थं} द्व्याकार‚म‚धिरोह‚ति । अधिरूढं च । त‚तो द्व्याकारं त‚त् । त‚था ‚{\color{DodgerBlue3}‚ज्ञान‚मुत्त‚र‚मुत्त‚रं} बुद्धिपूर्व्व‚ज्ञानाल‚म्ब‚न‚{\color{DodgerBlue3}‚मेकाकारोत्त‚र}‚मेकेनैकेनाकारेणाधिकं प्र‚तीय‚ते । अर्थेज्ञानेन ‚{\tiny $_{lb}$}‚त‚दाल‚म्ब‚क एकोऽर्थाकारः प्र‚तीय‚ते । त‚दाल‚म्ब‚नेन\edtext{}{\edlabel{pvv.233-1}\label{pvv.233-1}\lemma{नेन}\Bfootnote{प्र‚थ‚म‚ज्ञान‚स्य यौ बाह्य‚ग्राह‚काकारौ । तौ द्वितीय‚स्य ज्ञान‚स्य ग्र‚ह्याकारौ जातौ ।}}तु विष‚य‚{\tiny $_{7}$}‚भूत‚ज्ञानाकार‚स्त‚दा-\leavevmode\ledsidenote{\textenglish{45b/MA}} ‚{\tiny $_{lb}$}‚ल‚म्ब‚क‚श्च स्वाकारः । (३८०)
	\pend% ending standard par
      \label{div_pvv.2.381}
	  
	% new div opening: depth here is 2
	
	  \bigskip
	  \begingroup
	
	    \large
	  
	    \begin{quote}
	  
	    
	    \stanza[\smallbreak]
	\label{pv.2.381}\flagstanza{\tiny\textenglish{....2.381}}त‚स्यार्थ‚रूपेणाकारावात्माकार‚श्च क‚श्च‚न ।&द्वितीय‚स्य तृतीयेन ज्ञानेन हि विविच्य‚ते ॥ ३८१ ॥\&[\smallbreak]


	
	    \end{quote}
	  
	  \endgroup
	

	  \pstart \leavevmode% starting standard par
	\hphantom{.}य‚स्मा‚{\color{DodgerBlue3}‚त्त‚स्य} द्वितीय‚स्य ज्ञान‚स्य तौ द्वावाकाराव‚{\color{DodgerBlue3}‚र्थ‚रूपेण} विष‚य‚भावेन त‚दा‚{\tiny $_{lb}$}‚ल‚म्ब‚क ‚{\color{DodgerBlue3}‚आत्माकार‚श्च क‚श्च‚न} तृतीयेन द्वितीय‚ज्ञानाल‚म्ब‚केन हि य‚स्माद् विवेच्य‚ते ‚{\tiny $_{lb}$}‚अव‚धार्य‚ते । त‚स्माद‚र्थाकारं स्व‚वेद‚न‚ञ्च ज्ञान‚म‚भ्युप‚ग‚न्त‚व्यं । (३८१)
	\pend% ending standard par
      \label{div_pvv.2.382}
	  
	% new div opening: depth here is 2
	
	  \bigskip
	  \begingroup
	
	    \large
	  
	    \begin{quote}
	  
	    
	    \stanza[\smallbreak]
	\label{pv.2.382a}\flagstanza{\tiny\textenglish{...2.382a}}अर्थ‚कार्य‚त‚या ज्ञान‚स्मृताव‚र्थ‚स्मृतेर्य‚दि ।&भ्रान्त्या स‚ङ्क‚ल‚नं;\&[\smallbreak]


	
	    \end{quote}
	  
	  \endgroup
	

	  \pstart \leavevmode% starting standard par
	अ\edtext{}{\edlabel{pvv.233-2}\label{pvv.233-2}\lemma{अ}\Bfootnote{स्प‚ष्टेप्य‚र्थे बाह्याभिनिवेश‚त्यागार्थ प‚र‚प्र‚क्रियां दूष‚यितुं प‚र‚म‚त‚{\tiny $_{lb}$}‚मुप‚भिप‚ति ।}}थ निराकार‚मेव ‚{\color{DodgerBlue3}‚ज्ञान‚म}‚र्थ‚स्य कार्य‚म‚नुभ‚व‚रूप‚{\color{DodgerBlue3}‚म‚र्थ‚कार्य‚त‚या} ज्ञात\edtext{}{\edlabel{pvv.233-3}\label{pvv.233-3}\lemma{ज्ञात}\Bfootnote{कार्य‚भूते विष‚य‚ज्ञाने या स्मृतिः ।}} ‚{\color{DodgerBlue3}‚स्मृतौ} विशेष‚ण‚त्वेना\edtext{}{\edlabel{pvv.233-4}\label{pvv.233-4}\lemma{त्वेना}\Bfootnote{त‚त्कार‚ण‚स्य स्मूतिर्भ‚व‚ति त‚तः ।}} ‚{\color{DodgerBlue3}‚र्थ‚स्य स्मृतेः} स्म‚र्य‚माण‚स्यार्थ‚स्य ‚{\color{DodgerBlue3}‚भ्रान्त्या} ज्ञानात्म\edtext{}{\edlabel{pvv.233-5}\label{pvv.233-5}\lemma{ज्ञानात्म}\Bfootnote{निराकारेपि साकार‚व‚तः ।}}नि ‚{\color{DodgerBlue3}‚संक‚{\tiny $_{1}$}‚ल‚नं} स‚म्ब(न्ध)नं य‚दि स्यात्त‚दा को दोषः ।
	\pend% ending standard par
      

	  \pstart \leavevmode% starting standard par
	आह ।
	\pend% ending standard par
      
	  \bigskip
	  \begingroup
	
	    \large
	  
	    \begin{quote}
	  
	    
	    \stanza[\smallbreak]
	\label{pv.2.382b}\flagstanza{\tiny\textenglish{...2.382b}}ज्योतिर्म‚न‚स्कारे च सा भ‚वेत् ॥ ३८२ ॥\&[\smallbreak]


	
	    \end{quote}
	  
	  \endgroup
	

	  \pstart \leavevmode% starting standard par
	अर्थ‚व‚त् ज्योतिष आलोक‚स्य म‚न‚स्कार‚स्य च ज्ञान‚हेतुत्वात् । त‚त्कार्य‚ज्ञाने ‚{\tiny $_{lb}$}‚स्म‚र्य‚माणे सा स्मृति‚{\color{DodgerBlue3}‚र्ज्योतिर्म‚न‚स्कारे} आलोक‚स‚म‚न‚न्त‚र‚प्र‚त्य‚योर‚पि स्यात् । त‚त्र ‚{\tiny $_{lb}$}‚भ्रान्त्या अर्थाकार‚मिवालोक‚म‚न‚स्काराकार‚म‚पि संक‚ल‚नीयं । न त्व‚र्थाकार‚मेव निय‚{\tiny $_{lb}$}‚मेन संक‚ल‚यितुं युक्तं । (३८२)
	\pend% ending standard par
      \label{div_pvv.2.383}
	  
	% new div opening: depth here is 2
	
	  \bigskip
	  \begingroup
	
	    \large
	  
	    \begin{quote}
	  
	    
	    \stanza[\smallbreak]
	\label{pv.2.383}\flagstanza{\tiny\textenglish{....2.383}}स‚र्व्वेषाम‚पि कार्य्याणां कार‚णैः स्यात्त‚था ग्र‚हः ।&कुलालादिविवेकेन न स्म‚र्येत घ‚ट‚स्त‚तः ॥ ३८३ ॥\&[\smallbreak]


	
	    \end{quote}
	  
	  \endgroup
	\textsuperscript{\textenglish{234/s}}

	  \pstart \leavevmode% starting standard par
	य‚दि चा\edtext{}{\edlabel{pvv.234-1}\label{pvv.234-1}\lemma{चा}\Bfootnote{इत्य‚र्थाकार‚स‚मावेशित‚ज्ञान‚म् ।}}र्थ‚कार्यं भ्रान्त्या स्म‚र्य‚ते त‚दा ‚{\color{DodgerBlue3}‚स‚र्व्वेषाम‚पि कार्य्याणां कार‚णैः} स‚ह ‚{\tiny $_{lb}$}‚‚{\color{DodgerBlue3}‚त‚था ग्र‚हः} कार‚{\tiny $_{2}$}‚णात्म‚त्वेन ग्र‚ह‚णं ‚{\color{DodgerBlue3}‚स्यात्} । त‚त‚श्च ध‚ट‚कुलालादिकार्यं ‚{\color{DodgerBlue3}‚कुलाला‚{\tiny $_{lb}$}‚देर्व्विवेकेन} भेदेन ‚{\color{DodgerBlue3}‚न स्म‚र्येत} । (३८३)
	\pend% ending standard par
      \label{div_pvv.2.384}
	  
	% new div opening: depth here is 2
	

	  \pstart \leavevmode% starting standard par
	अथास्त्येव क‚श्चिदालोकादिभ्यो विष‚य‚स्य ज्ञानात्म‚न्यारोप‚णीयो विशेष‚स्त‚{\tiny $_{lb}$}‚त‚स्त‚दाकाराव‚ग्र‚हेण स्म‚र्य‚ते । नान्य‚थेत्याह (।)
	\pend% ending standard par
      
	  \bigskip
	  \begingroup
	
	    \large
	  
	    \begin{quote}
	  
	    
	    \stanza[\smallbreak]
	\label{pv.2.384}\flagstanza{\tiny\textenglish{....2.384}}य‚स्माद‚तिश‚याज्‏ज्ञान‚म‚र्थ‚संस‚र्ग‚भाज‚न‚म् ।&सारूप्यात्त‚त् किम‚न्य‚त् स्याद् दृष्टेश्च य‚म‚लादिषु ॥ ३८४ ॥\&[\smallbreak]


	
	    \end{quote}
	  
	  \endgroup
	

	  \pstart \leavevmode% starting standard par
	\hphantom{.}‚{\color{DodgerBlue3}‚य‚स्माद}‚र्थारोपिताद‚{\color{DodgerBlue3}‚तिश‚यात् ज्ञान‚म‚र्थ‚संस‚र्ग}‚स्यार्थ‚संश्लेष‚स्य ‚{\color{DodgerBlue3}‚भाज‚नं} पात्रं भ‚व‚ति । ‚{\tiny $_{lb}$}‚अर्थेन सारूप्या‚{\color{DodgerBlue3}‚द‚न्य‚त् किन्त‚त् स्यात्} । सारूप्याद‚न्य‚स्यातिश‚य‚स्योप‚ल‚क्ष‚ण‚त्वायो‚{\tiny $_{3}$}‚‚{\tiny $_{lb}$}‚गात् । ‚{\color{DodgerBlue3}‚य‚म‚लादिषु} सारूप्याद् भ्रान्त्या त‚थार्थ‚निश्च‚य‚स्य ‚{\color{DodgerBlue3}‚दृष्टेश्च} । (३८४)
	\pend% ending standard par
      \label{div_pvv.2.385}
	  
	% new div opening: depth here is 2
	

	  \pstart \leavevmode% starting standard par
	किञ्च ।
	\pend% ending standard par
      
	  \bigskip
	  \begingroup
	
	    \large
	  
	    \begin{quote}
	  
	    
	    \stanza[\smallbreak]
	\label{pv.2.385}\flagstanza{\tiny\textenglish{....2.385}}आद्यानुभ‚य‚रूप‚त्वे ह्येक‚रूपं व्य‚व‚स्थित‚म् ।&द्वितीयं व्य‚तिरिच्येत न प‚राम‚र्श‚चेत‚सा ॥३८५ ॥\&[\smallbreak]


	
	    \end{quote}
	  
	  \endgroup
	

	  \pstart \leavevmode% starting standard par
	\hphantom{.}‚{\color{DodgerBlue3}‚आद्य}‚स्यार्थ‚ज्ञान‚स्या‚{\color{DodgerBlue3}‚नुभ‚य‚रूप‚त्वे}‚ऽर्थाकार‚र‚हित‚त्वात् अनुभ‚वैक‚रूप‚त्वे त‚दाल‚म्ब‚कं ‚{\tiny $_{lb}$}‚‚{\color{DodgerBlue3}‚द्वितीयं} ज्ञान‚{\color{DodgerBlue3}‚मेक}‚स्मिन् ‚{\color{DodgerBlue3}‚रू\edtext{}{\edlabel{pvv.234-2}\label{pvv.234-2}\lemma{रू}\Bfootnote{विष‚य‚ज्ञान‚त‚ज्ज्ञान‚विशेषात्तु द्विरूप‚तेति व्याख्याय स्व‚य‚मुप‚प‚त्तिमाह (।) ‚{\tiny $_{lb}$}‚अद्विरूपं विष‚याभासाभावात् ।}}पे}‚ऽनुभ‚वात्म‚नि ‚{\color{DodgerBlue3}‚व्य‚व‚स्थितं प‚राम‚र्श‚चेत‚सा} ज्ञान‚ज्ञाना‚{\tiny $_{lb}$}‚ल‚म्ब‚केन तृतीय‚ज्ञानेन न व्य‚तिरिच्येत । न विष‚य‚ज्ञान‚ग्राह‚क‚त‚या भेदेन गृह्येत । ‚{\tiny $_{lb}$}‚स्वाभास‚मात्र‚त्वेन स‚र्व्व‚स्याविशेषात् । (३८५)
	\pend% ending standard par
      \label{div_pvv.2.386}
	  
	% new div opening: depth here is 2
	
	  \bigskip
	  \begingroup
	
	    \large
	  
	    \begin{quote}
	  
	    
	    \stanza[\smallbreak]
	\label{pv.2.386}\flagstanza{\tiny\textenglish{....2.386}}अर्थ‚संक‚ल‚नाश्लेषा धीर्द्वितीयाव‚ल‚म्ब‚ते ।&नीलादिरूपेण धियं भास‚मानां पुर‚स्त‚तः ॥ ३८६ ॥\&[\smallbreak]


	
	    \end{quote}
	  
	  \endgroup
	

	  \pstart \leavevmode% starting standard par
	\hphantom{.}य‚तो बुद्धेर‚नाकार‚त्वे दोषोऽयं ‚{\color{DodgerBlue3}‚त‚तः} पु‚{\tiny $_{4}$}‚‚{\color{DodgerBlue3}‚रोऽर्थ‚स्य} धियं ‚{\color{DodgerBlue3}‚नीलादिरूपेण भास‚मानां ‚{\tiny $_{lb}$}‚द्वितीया धीर‚र्थ‚संक‚ल}‚न‚स्यार्थाकाराव‚ग्र‚ह‚स्या‚{\color{DodgerBlue3}‚श्लेषः} संस‚र्ग्गो य‚स्याः सा ताम‚{\color{DodgerBlue3}‚व‚ल‚म्ब}‚ते । ‚{\tiny $_{lb}$}‚(३८६)
	\pend% ending standard par
      \label{div_pvv.2.387}
	  
	% new div opening: depth here is 2
	
	  \bigskip
	  \begingroup
	
	    \large
	  
	    \begin{quote}
	  
	    
	    \stanza[\smallbreak]
	\label{pv.2.387}\flagstanza{\tiny\textenglish{....2.387}}अन्य‚था ह्याद्य‚मेवैकं संयोज्येतार्थ‚स‚म्भ‚वात् ।&ज्ञानं नादृष्ट‚स‚म्ब‚न्धं पूर्व्वार्थेनोत्त‚रोत्त‚र‚म् ॥ ३८७ ॥\&[\smallbreak]


	
	    \end{quote}
	  
	  \endgroup
	

	  \pstart \leavevmode% starting standard par
	\hphantom{.}‚{\color{DodgerBlue3}‚अन्य‚था} य‚द्य‚नाकारं ज्ञान‚म‚र्थ‚कार्य‚त‚या भ्रान्त्याऽर्थाकारं स्म‚र्य‚त इत्याश्रीय‚ते ‚{\tiny $_{lb}$}‚त\edtext{}{\edlabel{pvv.234-3}\label{pvv.234-3}\lemma{त}\Bfootnote{न विष‚याकारोप‚धानात् ।}} ‚{\color{DodgerBlue3}‚दाद्य‚मेवैक}‚म‚र्थ‚ज्ञानं स्मृत्याऽर्थेन ‚{\color{DodgerBlue3}‚संयोज्येत । अर्थात्संभ‚वात्} त‚स्य । न ‚{\color{DodgerBlue3}‚उत्त‚रो‚{\tiny $_{lb}$}‚त्त‚रं ज्ञानं पूर्व्व‚स्य} ज्ञान‚स्या‚{\color{DodgerBlue3}‚र्थेना}‚कार‚ण‚भूतेना‚{\color{DodgerBlue3}‚दृष्ट‚स‚म्ब‚न्धं} संयोज्येत ।
	\pend% ending standard par
      \textsuperscript{\textenglish{235/s}}

	  \pstart \leavevmode% starting standard par
	त‚स्मात्स्थित‚मेत‚त् ज्ञानानां विष‚य‚सा‚{\tiny $_{5}$}‚रूप्यानुभ‚व‚रूप‚त्वाभ्यां द्व्याकार‚त्वं । ‚{\tiny $_{lb}$}‚त‚त‚श्च स‚होप‚ल‚म्भ‚निय‚मोऽर्थ‚विज्ञान‚योः । अर्थाकार‚ताया अर्थोप‚ल‚म्भात् । अनु‚{\tiny $_{lb}$}‚भ‚व‚रूप‚तायाः स्व‚{\color{DodgerBlue3}‚भे} (?वे) द‚न‚त्वात् ।(३८७)
	\pend% ending standard par
      \label{div_pvv.2.388}
	  
	% new div opening: depth here is 2
	

	  \pstart \leavevmode% starting standard par
	त‚था च (।)
	\pend% ending standard par
      
	  \bigskip
	  \begingroup
	
	    \large
	  
	    \begin{quote}
	  
	    
	    \stanza[\smallbreak]
	\label{pv.2.388}\flagstanza{\tiny\textenglish{....2.388}}स‚कृत् स‚म्वेद्य‚मान‚स्य निय‚मेन धिया स‚ह ।&विष‚य‚स्य त‚तोऽन्य‚त्वं केनाकारेण सिध्य‚ति ॥ ३८८ ॥\&[\smallbreak]


	
	    \end{quote}
	  
	  \endgroup
	

	  \pstart \leavevmode% starting standard par
	\hphantom{.}‚{\color{DodgerBlue3}‚धिया स‚ह निय‚मेन स‚कृत्संवेद्य‚मान‚स्य विष‚य‚स्य त‚तो} धियो‚{\color{DodgerBlue3}‚ऽन्य‚त्वं} भेदः ‚{\color{DodgerBlue3}‚केना‚{\tiny $_{lb}$}‚कारे}‚ण प्र‚कारेण ‚{\color{DodgerBlue3}‚सिध्य‚ति} ।\edtext{\textsuperscript{*}}{\edlabel{pvv.235-1}\label{pvv.235-1}\lemma{*}\Bfootnote{विष‚य‚स्याभावात्त‚द‚भेदो न साध्यः किन्तु बुद्धिरेव त‚दात्मिका साध्य‚ते ।}}भिन्न‚योः स‚होप‚ल‚म्भ‚निय‚मायोगात् । (३८८)
	\pend% ending standard par
      \label{div_pvv.2.389}
	  
	% new div opening: depth here is 2
	

	  \pstart \leavevmode% starting standard par
	य‚दि विष‚य‚ज्ञान‚योर‚भेद‚स्त‚दा ग्राह्य‚ग्राह‚क‚त‚या भेदः क‚थं प्र‚तिभातीत्याह (।)
	\pend% ending standard par
      
	  \bigskip
	  \begingroup
	
	    \large
	  
	    \begin{quote}
	  
	    
	    \stanza[\smallbreak]
	\label{pv.2.389a}\flagstanza{\tiny\textenglish{...2.389a}}भेद‚श्च भ्रान्त‚विज्ञानैर्दृश्येतेन्दाविवाद्व‚ये ।\&[\smallbreak]


	
	    \end{quote}
	  
	  \endgroup
	

	  \pstart \leavevmode% starting standard par
	\hphantom{.}‚{\color{DodgerBlue3}‚भेद‚श्च} वा‚{\tiny $_{6}$}‚स‚नाव‚शात् ‚{\color{DodgerBlue3}‚भ्रान्त}‚मुप‚प्लुताकारं ‚{\color{DodgerBlue3}‚ज्ञानं} येषां तैर‚र्वाग्‏द‚र्शिभि‚{\tiny $_{lb}$}‚‚{\color{DodgerBlue3}‚र्दृश्येत । इन्दाविवाद्व}‚ये एक‚रूपे द्वैतं तिमिरोप‚ह‚त‚बुद्धिभिः ।
	\pend% ending standard par
      

	  \pstart \leavevmode% starting standard par
	भेदेपि क‚स्भान्न स‚होप‚ल‚म्भ‚निय‚म इत्याह (।)
	\pend% ending standard par
      
	  \bigskip
	  \begingroup
	
	    \large
	  
	    \begin{quote}
	  
	    
	    \stanza[\smallbreak]
	\label{pv.2.389b}\flagstanza{\tiny\textenglish{...2.389b}}संवित्तिनिय‚मो नास्ति भिन्न‚योर्नील‚पीत‚योः ॥ ३८९ ॥\&[\smallbreak]


	
	    \end{quote}
	  
	  \endgroup
	

	  \pstart \leavevmode% starting standard par
	\hphantom{.}भिन्न‚योर्नील‚पीत‚योः ‚{\color{DodgerBlue3}‚संवित्तिनिय‚मो नास्ति} । त‚तो य‚त्रास्ति त‚त्राभेदएव (। ३८९)
	\pend% ending standard par
      \label{div_pvv.2.390_2.391_2.392}
	  
	% new div opening: depth here is 2
	

	  \pstart \leavevmode% starting standard par
	त‚था हि (।)
	\pend% ending standard par
      
	  \bigskip
	  \begingroup
	
	    \large
	  
	    \begin{quote}
	  
	    
	    \stanza[\smallbreak]
	\label{pv.2.390}\flagstanza{\tiny\textenglish{....2.390}}नार्थोऽस‚म्वेद‚नः क‚श्चिद‚न‚र्थ‚म्वापि वेद‚न‚म् ।&दृष्टं स‚म्वेद्य‚मान‚न्त‚त् त‚योर्नास्ति विवेकिता ॥ ३९० ॥\&[\smallbreak]


	
	    \end{quote}
	  
	  \endgroup
	
	  \bigskip
	  \begingroup
	
	    \large
	  
	    \begin{quote}
	  
	    
	    \stanza[\smallbreak]
	\label{pv.2.391a}\flagstanza{\tiny\textenglish{...2.391a}}त‚स्माद‚र्थ‚स्य दुर्वारं ज्ञान‚कालाव‚भासिनः ।&ज्ञानाद‚व्य‚तिरेकित्वं;\&[\smallbreak]


	
	    \end{quote}
	  
	  \endgroup
	

	  \pstart \leavevmode% starting standard par
	नार्थोनुभ‚व‚म‚न्त‚रेण क‚श्चिद् दृष्टः (।) वेद‚न‚ञ्चार्थाकार‚म्विना न दृष्टं ‚{\tiny $_{lb}$}‚स‚म्वेद्य‚मानं । त‚त्त‚स्मात्त‚योर‚र्थ‚त‚दुप‚ल‚म्भ‚योर्नास्ति विवेधिता । (३९०)त‚स्मात् ‚{\tiny $_{lb}$}‚ज्ञान‚कालाव‚भासिनोर्थ‚स्य । अर्थ‚ज्ञान‚योर‚पि स‚ह‚संवित्तिज्ञानाद‚व्य‚तिरेकित्व‚म‚भिन्न\edtext{}{\edlabel{pvv.235-2}\label{pvv.235-2}\lemma{भिन्न}\Bfootnote{तादात्म्य‚प्र‚तिब‚न्ध उक्तः ।}}त्वं ‚{\tiny $_{lb}$}‚दुर्व्वार‚मित्युप‚संहारः ।
	\pend% ending standard par
      

	  \pstart \leavevmode% starting standard par
	स्यादेत‚त् ।
	\pend% ending standard par
      
	  \bigskip
	  \begingroup
	
	    \large
	  
	    \begin{quote}
	  
	    
	    \stanza[\smallbreak]
	\label{pv.2.391b}\flagstanza{\tiny\textenglish{...2.391b}}हेतुभेदानुमा भ‚वेत् ॥ ३९१ ॥\&[\smallbreak]


	
	    \end{quote}
	  
	  \endgroup
	\textsuperscript{\textenglish{236/s}}
	  \bigskip
	  \begingroup
	
	    \large
	  
	    \begin{quote}
	  
	    
	    \stanza[\smallbreak]
	\label{pv.2.392}\flagstanza{\tiny\textenglish{....2.392}}अभावाद‚क्ष‚बुद्धीनां स‚त्स्व‚प्य‚न्येषु हेतुषु ।&निय‚मं य‚दि न ब्रूयाद प्र‚त्य‚यात् स‚म‚न‚न्त‚रात् ॥ ३९२ ॥\&[\smallbreak]


	
	    \end{quote}
	  
	  \endgroup
	\textsuperscript{\textenglish{46a/MA}}

	  \pstart \leavevmode% starting standard par
	\hphantom{.}‚{\color{DodgerBlue3}‚स‚त्स्व‚प्य‚न्येष्वि}‚न्द्रियादिषु ‚{\color{DodgerBlue3}‚हेतुष्व‚क्ष‚बुद्धीनाम‚भावात्} । हेतुभेद‚{\tiny $_{7}$}‚स्य त‚द‚तिरिक्त‚स्य ‚{\tiny $_{lb}$}‚कार‚ण‚विशेष‚स्यानुमा भ‚वेत् । कार‚ण‚साक‚ल्ये स‚ति त‚न्मात्र‚साध्य‚स्य कार्य‚स्यानुत्पादा‚{\tiny $_{lb}$}‚योगात् । य‚श्चासौ कार‚ण‚भेदः स बाह्योऽर्थो भ‚विष्य‚यीति म‚न्य‚ते प‚रः । एव‚म‚प्य‚नुमान‚{\tiny $_{lb}$}‚ग‚म्यो ब‚हिर‚र्थो भ‚वेन्न प्र‚त्य‚क्षो य‚थेष्य‚ते । किन्तु व्य‚तिरेक‚साम‚र्थ्याद‚पि त‚दाऽर्थः‚{\tiny $_{lb}$}‚सिध्येत । ‚{\color{DodgerBlue3}‚य‚दि} यो गा चा रो नानार्थ‚प्र‚तिभासिनीनां धियामुत्पाद‚क्र‚म‚स्य ‚{\color{DodgerBlue3}‚निय‚मं}‚य‚था‚{\tiny $_{lb}$}‚प्र‚त्य‚यं प्र‚बुद्ध‚वास‚नाग‚र्भात् ‚{\color{DodgerBlue3}‚स‚म‚न‚न्त‚र‚प्र‚त्य‚यान्न ब्रूयात्} । (३९१, ३९२)
	\pend% ending standard par
      \label{div_pvv.2.393}
	  
	% new div opening: depth here is 2
	
	  \bigskip
	  \begingroup
	
	    \large
	  
	    \begin{quote}
	  
	    
	    \stanza[\smallbreak]
	\label{pv.2.393}\flagstanza{\tiny\textenglish{....2.393}}बीजादंकुर‚ज‚न्मान्नेर्धूमात् सिद्धिरितीदृशी ।&बाह्यार्थाश्र‚यिणी यापि कार‚क‚ज्ञाप‚क‚स्थितिः ॥ ३९३ ॥\&[\smallbreak]


	
	    \end{quote}
	  
	  \endgroup
	

	  \pstart \leavevmode% starting standard par
	न‚नु बाह्या‚{\tiny $_{1}$}‚र्था\edtext{}{\edlabel{pvv.236-1}\label{pvv.236-1}\lemma{र्था}\Bfootnote{विद्य‚मानेपि बाह्येर्थे य‚थानुभ‚व‚मेव वे \cref{pv.2.341} त्यादिना य‚दा तु बाह्य ‚{\tiny $_{lb}$}‚एवार्थः प्र‚मेय इत्यादि \href{http://sarit.indology.info/?cref=psv.1.9}{[प्र‚माण] स‚मुच्च‚य‚स्य तृतीयः फ‚ल‚विक‚ल्पो व्य‚ख्यातः} ।}} भावे ‚{\color{DodgerBlue3}‚बीजाद‚ङ्कुर‚स्य ज‚न्मे}‚तीदृशी एवंजातीया यापि प्र‚तीति‚{\tiny $_{lb}$}‚सिद्धा कार‚क‚स्थितिः । ‚{\color{DodgerBlue3}‚धूमात्} कार्यात्कार‚ण‚स्याग्नेः ‚{\color{DodgerBlue3}‚सिद्धिरितीदृशी} यापि ‚{\color{DodgerBlue3}‚ज्ञाप‚क‚{\tiny $_{lb}$}‚हेतुस्थितिः} त‚दुच्छेदः स्यात् । हेतुफ‚ल‚भावाश्र‚य‚स्य बाह्य‚स्यैवाभावात् । (३९३)
	\pend% ending standard par
      \label{div_pvv.2.394_2.395_2.396}
	  
	% new div opening: depth here is 2
	

	  \pstart \leavevmode% starting standard par
	अत्राह\edtext{}{\edlabel{pvv.236-2}\label{pvv.236-2}\lemma{अत्राह}\Bfootnote{विज्ञ‚प्तौ कार्य‚कार‚ण‚त्व‚स्थाप‚नाय प्र‚त्य‚क्षानुप‚ल‚म्भ‚साध‚नं हेतुफ‚ल‚भावं प‚र‚{\tiny $_{lb}$}‚माश‚ङ्क‚ते ।}} (।)
	\pend% ending standard par
      
	  \bigskip
	  \begingroup
	
	    \large
	  
	    \begin{quote}
	  
	    
	    \stanza[\smallbreak]
	\label{pv.2.394}\flagstanza{\tiny\textenglish{....2.394}}सापि त‚द्रूप‚निर्भासा त‚था निय‚त‚स‚ङ्ग‚माः ।&बुद्धीराश्रित्य क‚ल्प्येत य‚दि किं वा विरुध्य‚ते ॥ ३९४ ॥\&[\smallbreak]


	
	    \end{quote}
	  
	  \endgroup
	

	  \pstart \leavevmode% starting standard par
	\hphantom{.}‚{\color{DodgerBlue3}‚सा} कार‚क‚ज्ञाप‚क‚स्थिति‚{\color{DodgerBlue3}‚र‚पि त‚द्रूप‚निर्भा\edtext{}{\edlabel{pvv.236-3}\label{pvv.236-3}\lemma{निर्भा}\Bfootnote{बुद्ध्योर्व‚स्तुत्वान्नाव‚स्तुकं कार्य‚कार‚ण‚त्वं ।}}सा} बीजाङ्कुर‚धूमाग्निप्र‚तिभास‚{\tiny $_{lb}$}‚वास‚नाप्र‚तिनिय‚मात् । ‚{\color{DodgerBlue3}‚त‚था} क्र‚म\edtext{}{\edlabel{pvv.236-4}\label{pvv.236-4}\lemma{म}\Bfootnote{य‚था बीजाद‚ङ्कुर‚ज‚न्म‚नि एव‚मेव बीज‚बुद्ध्य‚न‚न्त‚र‚म‚ङ्कुर‚बुद्धिः ।}}विशेषेण ‚{\color{DodgerBlue3}‚निय‚तः स‚ङ्ग‚म} उत्पादो यासां ता ‚{\tiny $_{lb}$}‚‚{\color{DodgerBlue3}‚बुद्धीरा\edtext{}{\edlabel{pvv.236-5}\label{pvv.236-5}\lemma{बुद्धीरा}\Bfootnote{न हि व‚स्तु केन‚चिद् ग‚म्य‚ते स्व‚य‚मात्मानं ग‚म‚य‚ति ज्ञान‚प्र‚तिभास‚स्यैव ‚{\tiny $_{lb}$}‚वेद‚नात् ।}}श्रित्य} य‚दि ‚{\color{DodgerBlue3}‚क‚ल्प्य‚ते} त‚दा ‚{\color{DodgerBlue3}‚कि‚{\tiny $_{2}$}‚म्वा विरुध्येत} न किञ्चित् । हि बीज‚प्र‚तिभासं ‚{\tiny $_{lb}$}‚ज्ञानं स्व‚हेतोः प्र‚बुद्धाङ्कुर‚ज्ञान‚वास‚नापाट‚व‚म‚ङ्कुर‚ज्ञानं ज‚न‚य‚ति । एवं धूम‚ज्ञान‚{\tiny $_{lb}$}‚म‚ग्निज्ञान‚मुत्पाद‚य‚ति । ताव‚तैव च ज्ञाप‚क‚व्य‚व‚स्थाया अविरोधः । (३९४)
	\pend% ending standard par
      \textsuperscript{\textenglish{237/s}}

	  \pstart \leavevmode% starting standard par
	न‚न्व‚स्ति विरोधः । त‚था हि (।)
	\pend% ending standard par
      
	  \bigskip
	  \begingroup
	
	    \large
	  
	    \begin{quote}
	  
	    
	    \stanza[\smallbreak]
	\label{pv.2.395}\flagstanza{\tiny\textenglish{....2.395}}अन‚ग्निज‚न्यो धूमः स्यात् त‚त्कार्यात् कार‚णेऽग‚तिः ।&न स्यात् कार‚ण‚तायां वा कुत एकान्त‚तो ग‚तिः ॥ ३९५ ॥\&[\smallbreak]


	
	    \end{quote}
	  
	  \endgroup
	

	  \pstart \leavevmode% starting standard par
	\hphantom{.}धूम‚ज्ञानाद‚ग्निज्ञानोत्पादे‚{\color{DodgerBlue3}‚ऽन‚ग्निज‚न्यो धूमः स्यात्} । अग्निप्र‚तिभास‚स्य प्राग‚वि‚{\tiny $_{lb}$}‚द्य‚मान‚त्वात् विप‚र्य‚यः स्यात् । ‚{\color{DodgerBlue3}‚त‚त्} त‚स्मा‚{\color{DodgerBlue3}‚त्कार्यात् कार‚णे ग‚तिर्न स्यात्} । अग्निज्ञानं ‚{\tiny $_{lb}$}‚प्र‚ति धूम‚ज्ञान‚स्य ‚{\color{DodgerBlue3}‚का‚{\tiny $_{3}$}‚र‚ण‚तायां वा} कार‚णात् कार्य\edtext{}{\edlabel{pvv.237-1}\label{pvv.237-1}\lemma{कार्य}\Bfootnote{अनुमात‚व्ये ।}} ‚{\color{DodgerBlue3}‚एकान्त\edtext{}{\edlabel{pvv.237-2}\label{pvv.237-2}\lemma{एकान्त}\Bfootnote{नाव‚श्यं कार‚णानि कार्य‚व‚न्ति स्युः ।}}तो}‚ऽसंदिग्धा ‚{\color{DodgerBlue3}‚कुतो ‚{\tiny $_{lb}$}‚ग‚ति}‚रिति (३९५)
	\pend% ending standard par
      

	  \pstart \leavevmode% starting standard par
	अत्राह (।)
	\pend% ending standard par
      
	  \bigskip
	  \begingroup
	
	    \large
	  
	    \begin{quote}
	  
	    
	    \stanza[\smallbreak]
	\label{pv.2.396}\flagstanza{\tiny\textenglish{....2.396}}त‚त्रापि धूमाभासा धीः प्र‚बोध‚प‚टुवास‚नाम् ।&ग‚म‚येद‚ग्निनिर्भासान्धिय‚मेव न पाव‚क‚म् ॥३९६ ॥\&[\smallbreak]


	
	    \end{quote}
	  
	  \endgroup
	

	  \pstart \leavevmode% starting standard par
	\hphantom{.}‚{\color{DodgerBlue3}‚त‚त्र} धूमाद‚ग्न्य‚नुमा‚{\color{DodgerBlue3}‚नेपि धूमाभासा धीर}‚ग्निवास‚नाप्र‚तिब‚द्धा ‚{\color{DodgerBlue3}‚एक‚साम‚ग‚ग्र‚य-} धीन‚त‚या‚{\color{DodgerBlue3}‚ऽग्निनिर्भासान्धिय‚मेव} धूम‚ज्ञानादेव ‚{\color{DodgerBlue3}‚प्र‚बोधेन प‚टुज}‚न‚नोन्मुखा ‚{\color{DodgerBlue3}‚वास‚ना} श‚क्तिर्य‚स्यास्ता‚{\color{DodgerBlue3}‚ङ्ग‚म‚येत् । न पा\edtext{}{\edlabel{pvv.237-3}\label{pvv.237-3}\lemma{पा}\Bfootnote{येन लिङ्ग‚बोध‚काले धूम‚भास‚ज्ञान‚स्यान‚ग्निज‚न्य‚त्वं स्यात् ।}}व‚कं बाह्य‚रूपं} स‚र्व्व‚दाऽद‚र्श‚नात् । (३९६)
	\pend% ending standard par
      \label{div_pvv.2.397}
	  
	% new div opening: depth here is 2
	

	  \pstart \leavevmode% starting standard par
	अग्निवास‚नाधूम‚ज्ञान‚योर्हेतुफ‚ल‚तामाख्यातुमाह (।)
	\pend% ending standard par
      
	  \bigskip
	  \begingroup
	
	    \large
	  
	    \begin{quote}
	  
	    
	    \stanza[\smallbreak]
	\label{pv.2.397}\flagstanza{\tiny\textenglish{....2.397}}त‚द्योग्य‚वास‚नाग‚र्भ एव धूमाव‚भासिनीम् ।&व्य‚न‚क्ति चित्त‚स‚न्तानो धियं धूमोग्नित‚स्त‚तः ॥ ३९७ ॥\&[\smallbreak]


	
	    \end{quote}
	  
	  \endgroup
	

	  \pstart \leavevmode% starting standard par
	\hphantom{.}‚{\color{DodgerBlue3}‚त}‚स्याग्निप्र‚तिभास‚स्य ‚{\color{DodgerBlue3}‚योग्या‚{\tiny $_{4}$}‚} ज‚न‚न‚स‚म‚र्था ‚{\color{DodgerBlue3}‚वास‚नाग‚र्भे} स्व‚भाव‚भूता य‚स्य ‚{\tiny $_{lb}$}‚चित्त‚स‚न्तान‚स्य स ‚{\color{DodgerBlue3}‚चित्तंस‚न्तानो धूमाव‚भासिनीं धियं व्य‚न‚क्ति} उत्पाद‚य‚ति(।) ‚{\color{DodgerBlue3}‚त\edtext{}{\edlabel{pvv.237-4}\label{pvv.237-4}\lemma{त}\Bfootnote{बाह्य‚वादिनोपि तुल्य‚मेत‚द्येन व‚ह्निना धूमो ज‚नितः क‚थं त‚द‚नुमानं य‚श्च ‚{\tiny $_{lb}$}‚भावी तेनासौ न ज‚नित इति भाव्य‚नुमीय‚ते ।}}तो‚{\tiny $_{lb}$}‚ऽग्नित} एव ‚{\color{DodgerBlue3}‚धूमो} भ‚व‚तीति न कार्य‚कार‚ण‚ताविप‚र्य‚यः । न च कार‚णात्कार्यानुमान‚{\tiny $_{lb}$}‚म‚ग्निवास‚नाप्र‚भ‚व‚त्वात् धूमाग्निज्ञान‚योः । धूम‚ज्ञानात् प्र‚बुद्धाग्निवास‚नाद्वारेणा‚{\tiny $_{lb}$}‚ग्निज्ञानानुमितिरेक‚साम‚ग्र‚य‚धीना । (३९७)
	\pend% ending standard par
      \label{div_pvv.2.398}
	  
	% new div opening: depth here is 2
	

	  \begin{center}%% label @type='head'
	\textbf{(ख. ज्ञान‚द्व‚य‚रूप‚तासिद्धि)}
	\end{center}
	

	  \pstart \leavevmode% starting standard par
	a. एव‚न्त‚र्हि विज्ञान‚न‚य एव स‚र्व्व‚व्य‚व‚स्था‚{\tiny $_{5}$}‚ न‚म‚विरोधात् क‚थ‚माचार्येण ब‚हि‚{\tiny $_{lb}$}‚र‚र्थापेक्ष‚या ज्ञान‚द्विरूप‚तोक्तेत्याह (।)
	\pend% ending standard par
      \textsuperscript{\textenglish{238/s}}
	  \bigskip
	  \begingroup
	
	    \large
	  
	    \begin{quote}
	  
	    
	    \stanza[\smallbreak]
	\label{pv.2.398}\flagstanza{\tiny\textenglish{....2.398}}अस्त्येष विदुषां वादो बाह्य‚न्त्वाश्रित्य व‚र्ण्य‚ते ।&द्वैरूप्यं स‚ह‚संवित्तिनिय‚मात्त‚च्च सिध्य‚ति ॥ ३९८ ॥\&[\smallbreak]


	
	    \end{quote}
	  
	  \endgroup
	

	  \pstart \leavevmode% starting standard par
	\hphantom{.}‚{\color{DodgerBlue3}‚अस्त्येष} स‚र्व्व‚व्य‚व‚स्थासु विज्ञ‚प्तिमात्र‚ताप्र‚तिपाद‚को ‚{\color{DodgerBlue3}‚विदुषां} न्याय‚द‚र्शिनां ‚{\tiny $_{lb}$}‚‚{\color{DodgerBlue3}‚यो गा} चा रा णां ‚{\color{DodgerBlue3}‚वादः} । सौ त्रा न्ति कै रिष्टं ‚{\color{DodgerBlue3}‚बाह्य}‚म‚र्थ‚मा‚{\color{DodgerBlue3}‚श्रित्य} ज्ञान‚स्य द्वैरूप्य‚मा‚{\tiny $_{lb}$}‚‚{\color{DodgerBlue3}‚चार्येण व‚र्ण्य‚ते} । त‚च्च ‚{\color{DodgerBlue3}‚द्वैरूप्यं} स‚ह‚स‚म्वेद‚न‚{\color{DodgerBlue3}‚निय‚मात्} स‚होप‚ल‚म्भ‚निय‚मात् ‚{\color{DodgerBlue3}‚सिध्य‚ति} । ‚{\tiny $_{lb}$}‚(३९८) भेदेपि स‚ति त‚द‚भावात् ।
	\pend% ending standard par
      \label{div_pvv.2.399}
	  
	% new div opening: depth here is 2
	

	  \pstart \leavevmode% starting standard par
	b. द्वैरूप्य‚सिद्धावुप‚प‚त्त्य‚न्त‚रं व‚क्तुमाह (।)
	\pend% ending standard par
      
	  \bigskip
	  \begingroup
	
	    \large
	  
	    \begin{quote}
	  
	    
	    \stanza[\smallbreak]
	\label{pv.2.399}\flagstanza{\tiny\textenglish{....2.399}}ज्ञान‚मिन्द्रिय‚भेदेन प‚टुम‚न्दाविलादिकाम् ।&प्र‚तिभास‚भिदाम‚र्थे विभ्र‚देक‚त्र दृश्य‚ते ॥ ३९९ ॥\&[\smallbreak]


	
	    \end{quote}
	  
	  \endgroup
	

	  \pstart \leavevmode% starting standard par
	\hphantom{.}इन्द्रिय‚स्य ‚{\color{DodgerBlue3}‚भेदेन} प्र‚सादोप‚घातादिना विशेषेण पुरु‚{\color{DodgerBlue3}‚षार्थे ज्ञानं प‚टुम‚न्दा}‚विलादिकां ‚{\tiny $_{lb}$}‚‚{\color{DodgerBlue3}‚प्र‚तिभास‚भिदां विभ्र‚त्} द‚ध‚त् ‚{\color{DodgerBlue3}‚दृश्य‚ते} । (३९९)
	\pend% ending standard par
      \label{div_pvv.2.400}
	  
	% new div opening: depth here is 2
	
	  \bigskip
	  \begingroup
	
	    \large
	  
	    \begin{quote}
	  
	    
	    \stanza[\smallbreak]
	\label{pv.2.400}\flagstanza{\tiny\textenglish{....2.400}}अर्थ‚स्याभिन्न‚रूप‚त्वादेक‚रूपं भ‚वेन्म‚नः ।&स‚र्वं त‚द‚र्थ‚म‚र्थाच्चेत् त‚स्य नास्ति त‚दाभ‚ता ॥ ४०० ॥\&[\smallbreak]


	
	    \end{quote}
	  
	  \endgroup
	

	  \pstart \leavevmode% starting standard par
	\hphantom{.}‚{\color{DodgerBlue3}‚त‚स्य} ज्ञान‚स्यार्था‚{\color{DodgerBlue3}‚च्चेन्नास्ति त‚दाभ‚ताऽ}\edtext{}{\edlabel{pvv.238-1}\label{pvv.238-1}\lemma{स्यार्था}\Bfootnote{निराकार‚त्वात् ।}} र्थाकार‚ता त‚दा‚{\color{DodgerBlue3}‚र्थ‚स्याभिन्न‚रूप\edtext{}{\edlabel{pvv.238-2}\label{pvv.238-2}\lemma{रूप}\Bfootnote{एक‚रूप एवार्थः साकारः ।}} त्वात् । ‚{\tiny $_{lb}$}‚त‚द‚र्थ} त‚द्विष‚यं ‚{\color{DodgerBlue3}‚स‚र्व्वं म‚नो}‚वेद‚न‚{\color{DodgerBlue3}‚मेक‚रूप‚म्भ‚वेत्} । न प‚टुम‚न्दाविल‚तादिभिन्नं ज्ञान‚स्य ‚{\tiny $_{lb}$}‚स्व‚ग‚ताकार‚भेदान‚भ्युप‚ग‚मात् । अर्थ‚स्यैक‚रूप‚त्वात् प्र‚तिभास\edtext{}{\edlabel{pvv.238-3}\label{pvv.238-3}\lemma{तिभास}\Bfootnote{एक‚स्मिन्न‚र्थे नानाकारः ।}}भेद‚विरोधा‚{\tiny $_{7}$}‚त् । अस्ति ‚{\tiny $_{lb}$}‚\leavevmode\ledsidenote{\textenglish{46b/MA}} चायं त‚स्माद‚र्थ‚रूप‚ताऽनुभ‚व‚रूप‚ता चेति द्वैरूप्य‚सिद्धिः । (४००)
	\pend% ending standard par
      \label{div_pvv.2.401}
	  
	% new div opening: depth here is 2
	

	  \pstart \leavevmode% starting standard par
	न‚न्व‚र्थ‚रूप‚तायाम‚प्य‚र्थ‚स्यैक‚रूप‚त्वात् त‚त्स‚रूपं ज्ञान‚मेकाकारं स्यात् । न प्र‚स‚न्ना‚{\tiny $_{lb}$}‚विलादिरूप‚मित्याह (।)
	\pend% ending standard par
      
	  \bigskip
	  \begingroup
	
	    \large
	  
	    \begin{quote}
	  
	    
	    \stanza[\smallbreak]
	\label{pv.2.401}\flagstanza{\tiny\textenglish{....2.401}}अर्थाश्र‚येणोद्भ‚व‚त‚स्त‚द्रूप‚म‚नुकुर्व्व‚तः ।&त‚स्य केन‚चिदंशेन प‚र‚तोपि भिदा भ‚वेत् ॥ ४०१ ॥\&[\smallbreak]


	
	    \end{quote}
	  
	  \endgroup
	

	  \pstart \leavevmode% starting standard par
	\hphantom{.}‚{\color{DodgerBlue3}‚अर्थ}‚स्य स‚रूप‚स्या‚{\color{DodgerBlue3}‚श्र‚येणोद्भ‚व‚त‚स्त‚स्य} ज्ञान‚स्य ‚{\color{DodgerBlue3}‚त‚द्रूप‚म}‚र्थाकार‚{\color{DodgerBlue3}‚म‚नुकुर्व्व‚तः केन‚{\tiny $_{lb}$}‚चिदंशेना}‚कारेण प‚टुम‚न्द‚त्वादिना ‚{\color{DodgerBlue3}‚प‚र‚तो} वास‚नादेर‚पि कार‚णाद्\edtext{}{\edlabel{pvv.238-4}\label{pvv.238-4}\lemma{णाद्}\Bfootnote{य‚स्मिन् स‚ति य‚त्स्यात् येन विना न भ‚व‚ति त‚दान्येषु स‚त्स्व‚पि त‚त्त‚स्य कार‚णं ।}} ‚{\color{DodgerBlue3}‚भिदा ‚{\tiny $_{lb}$}‚भ‚वेत्} । (४०१)
	\pend% ending standard par
      \label{div_pvv.2.402}
	  
	% new div opening: depth here is 2
	\textsuperscript{\textenglish{239/s}}
	  \bigskip
	  \begingroup
	
	    \large
	  
	    \begin{quote}
	  
	    
	    \stanza[\smallbreak]
	\label{pv.2.402}\flagstanza{\tiny\textenglish{....2.402}}त‚था ह्याश्रित्य पित‚रं त‚द्रूपो हि सुतः पितुः ।&भेदं केन‚चिदंशेन कुत‚श्चिद‚व‚ल‚म्ब‚ते ॥ ४०२ ॥\&[\smallbreak]


	
	    \end{quote}
	  
	  \endgroup
	

	  \pstart \leavevmode% starting standard par
	\hphantom{.}‚{\color{DodgerBlue3}‚त‚था हि पित‚र‚माश्रित्य त‚द्रूपः} पित्राकारोपि ‚{\color{DodgerBlue3}‚सुत} उत्प‚न्नः ‚{\color{DodgerBlue3}‚के‚{\tiny $_{1}$}‚न‚चिदंशेनाकारेण ‚{\tiny $_{lb}$}‚कुत‚श्चित्} क‚र्म्मादेर्हेतोः ‚{\color{DodgerBlue3}‚पितुः} श‚कासात् (? स‚काशात्) ‚{\color{DodgerBlue3}‚भेद‚म}‚न्यादृश‚त्व‚{\color{DodgerBlue3}‚म‚व‚ल‚म्व‚ते ‚{\tiny $_{lb}$}‚पि}‚तापुत्र‚योः स‚र्व्व‚था साम्याभावात् । (४०२)
	\pend% ending standard par
      \label{div_pvv.2.403}
	  
	% new div opening: depth here is 2
	

	  \pstart \leavevmode% starting standard par
	C. द्वैरूप्य‚सिद्धावुप‚प‚त्त्य‚न्त‚र‚माह (।)
	\pend% ending standard par
      
	  \bigskip
	  \begingroup
	
	    \large
	  
	    \begin{quote}
	  
	    
	    \stanza[\smallbreak]
	\label{pv.2.403}\flagstanza{\tiny\textenglish{....2.403}}म‚यूर‚च‚न्द्र‚काकारं नील‚लोहित‚भास्व‚र‚म् ।&स‚म्प‚श्य‚न्ति प्र‚दीपादेर्म्म‚ण्ड‚लं म‚न्द‚च‚क्षुषः ॥ ४०३ ॥\&[\smallbreak]


	
	    \end{quote}
	  
	  \endgroup
	

	  \pstart \leavevmode% starting standard par
	\hphantom{.}‚{\color{DodgerBlue3}‚म‚यूर‚च‚न्द्र‚काकार‚म‚न्त‚रान्त‚रा नील‚लोहित‚भास्व‚रं दीप्तं प्र‚दीपादेर्म‚ण्ड‚ल‚म‚विद्य-} मान‚मेव ‚{\color{DodgerBlue3}‚म‚न्द‚च‚क्षुषः\edtext{}{\edlabel{pvv.1a}\label{pvv.1a}\lemma{क्षुषः}\Bfootnote{1a च‚त्त‚ग‚ताः ।\begin{english} --- Placement of note uncertain; marked with a question mark in the edition (see encoding description for details)\end{english}}}\edtext{}{\edlabel{pvv.1b}\label{pvv.1b}\lemma{क्षुषः}\Bfootnote{1b रूपेण ।\begin{english} --- Placement of note uncertain; marked with a question mark in the edition (see encoding description for details)\end{english}}}\edtext{}{\edlabel{pvv.239-1}\label{pvv.239-1}\lemma{क्षुषः}\Bfootnote{विप्लुताक्षः ।}} संप‚श्य‚न्ति} । दीप‚स्य तादृश‚स्व‚रूपाभावात् । ‚{\color{DodgerBlue3}‚ज्ञान‚स्यानु-} भ‚वात्म‚नः स आकार इति द्वैरूप्य‚सिद्धिः । (४०३)
	\pend% ending standard par
      \label{div_pvv.2.404_2.405}
	  
	% new div opening: depth here is 2
	

	  \pstart \leavevmode% starting standard par
	अथ तादृशं व‚स्त्वेवोत्प‚न्नं दृश्य‚त इति न ज्ञानाकार इत्याह (।)
	\pend% ending standard par
      
	  \bigskip
	  \begingroup
	
	    \large
	  
	    \begin{quote}
	  
	    
	    \stanza[\smallbreak]
	\label{pv.2.404}\flagstanza{\tiny\textenglish{....2.404}}त‚स्य त‚द्बाह्य‚रूप‚त्वे का प्र‚स‚न्नेक्ष‚णेऽक्ष‚मा ।&भूतं प‚श्यँश्च त‚द्द‚र्शी क‚थ‚ञ्चोप‚ह‚तेन्द्रियः ॥ ४०४ ॥\&[\smallbreak]


	
	    \end{quote}
	  
	  \endgroup
	
	  \bigskip
	  \begingroup
	
	    \large
	  
	    \begin{quote}
	  
	    
	    \stanza[\smallbreak]
	\label{pv.2.405}\flagstanza{\tiny\textenglish{....2.405}}शोधितं तिमिरेणास्य व्य‚क्तं च‚क्षुर‚तीन्द्रिय‚म् ।&प‚श्य‚तोऽन्याक्ष‚दृश्येर्थे त‚द‚व्य‚क्तं क‚थं पुनः ॥ ४०५ ॥\&[\smallbreak]


	
	    \end{quote}
	  
	  \endgroup
	

	  \pstart \leavevmode% starting standard par
	\hphantom{.}‚{\color{DodgerBlue3}‚त‚स्य} म‚ण्ड‚ल‚स्य ‚{\color{DodgerBlue3}‚त‚द्बाह्य‚रूप‚त्वे}‚ऽभ्युप‚ग‚म्य‚माने ‚{\color{DodgerBlue3}‚प्र‚स‚न्नेक्ष‚णे} द्र‚ष्ट‚रि ‚{\color{DodgerBlue3}‚काऽक्ष‚मा द्वेषो} येनास्मै नात्मान‚मुप‚द‚र्श‚येत् । य‚द्व‚स्तुप‚ह‚तेन्द्रियेण दृश्य‚ते । त‚द‚नुप‚ह‚तेन्द्रियेण सुत‚रां ‚{\tiny $_{lb}$}‚दृश्य‚ते । ‚{\color{DodgerBlue3}‚भूतं} स‚त्यं ‚{\color{DodgerBlue3}‚च प‚श्य‚न् त‚द्द‚र्शी} म‚ण्ड‚ल‚द‚र्शी ‚{\color{DodgerBlue3}‚क‚थ‚मुप‚ह‚तेन्द्रियैः} (४०४) ‚{\color{DodgerBlue3}‚अप‚रैर‚दृश्यं} म‚ण्ड‚लं ‚{\color{DodgerBlue3}‚प‚श्य}‚तोऽस्य म‚न्द‚च‚{\tiny $_{3}$}‚क्षुष्ट्वेन न‚ष्ट‚स्य ‚{\color{DodgerBlue3}‚तिमिरेण व्य‚क्तं च‚क्षुः शोधित}‚मित्युप‚{\tiny $_{lb}$}‚ह‚स‚ति । किन्तु तैमिरिक‚स्यातीन्द्रियार्थ‚द‚र्श‚न‚क्ष‚मं त‚च्च‚क्षुर‚न्य‚स्यातैमिरिक‚स्या‚{\tiny $_{lb}$}‚क्ष‚दृश्येऽर्थे प्र‚दीपे ‚{\color{DodgerBlue3}‚क‚थं पुन‚र‚व्य‚क्त‚म}‚स्फुटं य‚द‚तीन्द्रियं प‚श्य‚ति त‚त् स‚र्व्वं दृश्यं सुत‚रां ‚{\tiny $_{lb}$}‚प‚श्य‚ति । (४०५)
	\pend% ending standard par
      \label{div_pvv.2.406}
	  
	% new div opening: depth here is 2
	

	  \pstart \leavevmode% starting standard par
	किञ्च (।)
	\pend% ending standard par
      
	  \bigskip
	  \begingroup
	
	    \large
	  
	    \begin{quote}
	  
	    
	    \stanza[\smallbreak]
	\label{pv.2.406}\flagstanza{\tiny\textenglish{....2.406}}आलोकाक्ष‚म‚न‚स्काराद‚न्य‚स्यैक‚स्य ग‚म्य‚ते ।&श‚क्तिर्हेतुस्त‚तो नान्योऽहेतुश्च विष‚यः क‚थ‚म् ॥ ४०६ ॥\&[\smallbreak]


	
	    \end{quote}
	  
	  \endgroup
	\textsuperscript{\textenglish{240/s}}

	  \pstart \leavevmode% starting standard par
	\hphantom{.}‚{\color{DodgerBlue3}‚आलोकाक्ष‚म‚न‚स्काराद‚न्य‚स्य दीप\edtext{}{\edlabel{pvv.240-1}\label{pvv.240-1}\lemma{दीप}\Bfootnote{एक‚मेष्ट‚व्यं व‚स्तुतो भिन्न‚स्याभेदे इष्य‚माणे । म‚रीचिषु ज‚ल‚व‚त् ।}}स्य एक‚स्य} म‚ण्ड‚{\tiny $_{4}$}‚ल‚ज्ञान‚ज‚न‚ने ‚{\color{DodgerBlue3}‚श‚क्तिर्ग्ग‚म्य‚ते} त‚न्मात्र‚भावेन भावात् । त‚तो दीपाद‚न्यो म‚ण्ड‚लो न हेतुर‚हेतुश्चासौ म‚ण्ड‚ल‚ज्ञान‚स्य ‚{\tiny $_{lb}$}‚‚{\color{DodgerBlue3}‚क‚थं विष‚यो}‚तिप्र‚स‚ङ्गात् । (४०६)
	\pend% ending standard par
      \label{div_pvv.2.407}
	  
	% new div opening: depth here is 2
	
	  \bigskip
	  \begingroup
	
	    \large
	  
	    \begin{quote}
	  
	    
	    \stanza[\smallbreak]
	\label{pv.2.407a}\flagstanza{\tiny\textenglish{...2.407a}}स एव य‚दि धीहेतुः किम्प्र‚दीप‚म‚पेक्ष‚ते ।\&[\smallbreak]


	
	    \end{quote}
	  
	  \endgroup
	

	  \pstart \leavevmode% starting standard par
	\hphantom{.}‚{\color{DodgerBlue3}‚स} म‚ण्ड‚ल ‚{\color{DodgerBlue3}‚एव} म‚ण्ड‚ल‚ग्राहिण्या ‚{\color{DodgerBlue3}‚धियो हेतु}‚र्न दीपो ‚{\color{DodgerBlue3}‚य‚दी}‚ष्य‚ते त‚दा म‚ण्ड‚लं ‚{\color{DodgerBlue3}‚किं} क‚स्मात् ‚{\color{DodgerBlue3}‚प्र‚दीप‚म‚पेक्ष‚ते} न ह्य‚हेतोर‚पेक्षाऽतिप्र‚स‚ङ्गात् ।
	\pend% ending standard par
      

	  \pstart \leavevmode% starting standard par
	दीपो म‚ण्ड‚ल‚ञ्च म‚ण्ड‚ल‚बुद्धिहेतुरित्याह (।)
	\pend% ending standard par
      
	  \bigskip
	  \begingroup
	
	    \large
	  
	    \begin{quote}
	  
	    
	    \stanza[\smallbreak]
	\label{pv.2.407b}\flagstanza{\tiny\textenglish{...2.407b}}दीप‚मात्रेण धीभावादुभ‚य‚न्नापि कार‚ण‚म् ॥ ४०७ ॥\&[\smallbreak]


	
	    \end{quote}
	  
	  \endgroup
	

	  \pstart \leavevmode% starting standard par
	\hphantom{.}‚{\color{DodgerBlue3}‚उभ‚यं न कार‚णं दीप‚मात्रे‚{\tiny $_{5}$}‚ण} म‚ण्ड‚ल‚धियो ‚{\color{DodgerBlue3}‚भावात्} ।\edtext{\textsuperscript{*}}{\edlabel{pvv.240-2}\label{pvv.240-2}\lemma{*}\Bfootnote{बाह्य‚म‚भ्युपेत्य द्वैरूप्य‚मुक्त्वाधुना बाह्याभाव‚माह ।}}(४०७)
	\pend% ending standard par
      \label{div_pvv.2.408}
	  
	% new div opening: depth here is 2
	
	  \bigskip
	  \begingroup
	
	    \large
	  
	    \begin{quote}
	  
	    
	    \stanza[\smallbreak]
	\label{pv.2.408}\flagstanza{\tiny\textenglish{....2.408}}दूरास‚न्नादिभेदेन व्य‚क्ताव्य‚क्तं न युज्य‚ते ।&त‚त्स्यादालोक‚भेदाच्चेत्त‚त्पिधानापिधान‚योः ॥ ४०८ ॥\&[\smallbreak]


	
	    \end{quote}
	  
	  \endgroup
	

	  \pstart \leavevmode% starting standard par
	\hphantom{.}य‚दि चार्थ एव साकारो ग्राह्यो ज्ञान‚न्त्व‚नाकारं त‚दार्थ‚स्य ‚{\color{DodgerBlue3}‚दूरास‚न्नादिना भेदेन ‚{\tiny $_{lb}$}‚विशेषेण व्य‚क्ताव्य‚क्तं न युज्य‚ते} एकात्म‚नः प‚दार्थ‚स्य स्व‚रूपेण दृश्य‚मान‚त्वात् । ‚{\tiny $_{lb}$}‚दूरास‚न्न‚स्थाभ्यां स‚मानः प्र‚तीयेत ।\edtext{\textsuperscript{*}}{\edlabel{pvv.240-3}\label{pvv.240-3}\lemma{*}\Bfootnote{एक‚स्य नानात्व‚विरोधात् ।}} ज्ञान‚स्व‚ग‚ताकार‚भेदान‚भ्युप‚ग‚मात् । व्य‚व‚{\tiny $_{lb}$}‚घानाव्य‚व‚धान‚योरा‚{\color{DodgerBlue3}‚लोक}‚स्य मान्द्यामान्द्य‚{\color{DodgerBlue3}‚भेदात् । त‚द्} व्य‚क्ताव्य‚क्तं व‚स्तु ‚{\color{DodgerBlue3}‚स्यादि}‚ति ‚{\tiny $_{lb}$}‚चेत् (।) दूर‚स्थितौ\edtext{}{\edlabel{pvv.240-4}\label{pvv.240-4}\lemma{स्थितौ}\Bfootnote{अत्र द्वौ विक‚ल्पौ ।}} त‚स्यालोक‚स्य ‚{\color{DodgerBlue3}‚पिधान‚म‚पिधा}‚न‚ञ्चाऽभ्युप‚ग‚न्त‚व्यं त‚यो‚{\tiny $_{lb}$}‚\leavevmode\ledsidenote{\textenglish{47a/MA}} ‚{\color{DodgerBlue3}‚स्तुल्य‚योः} (।) (४०८)
	\pend% ending standard par
      \label{div_pvv.2.409}
	  
	% new div opening: depth here is 2
	
	  \bigskip
	  \begingroup
	
	    \large
	  
	    \begin{quote}
	  
	    
	    \stanza[\smallbreak]
	\label{pv.2.409}\flagstanza{\tiny\textenglish{....2.409}}तुल्या दृष्टिर‚दृष्टिर्वा सूक्ष्मोंश‚स्त‚स्य क‚श्च‚न ।&आलोकेन च म‚न्देन दृश्य‚तेतो भिदा य‚दि ॥ ४०९ ॥\&[\smallbreak]


	
	    \end{quote}
	  
	  \endgroup
	

	  \pstart \leavevmode% starting standard par
	\hphantom{.}स‚र्व्व‚स्य प्र‚तिप‚त्तु‚{\color{DodgerBlue3}‚र्दृष्टिर‚दृष्टि}‚र्व्वा ‚{\color{DodgerBlue3}‚तुल्या}‚र्थ‚स्य स्यात् । \edtext{\textsuperscript{*}}{\edlabel{pvv.240-5}\label{pvv.240-5}\lemma{*}\Bfootnote{साव्य‚क्ताव्य‚क्त‚त्वं ।}}दूर‚स्थ‚स्य र‚जोनी‚{\tiny $_{lb}$}‚हारादिभिरुप‚ह‚त‚त्वात् ‚{\color{DodgerBlue3}‚म‚न्देनालोकेन त‚स्य} दृश्यार्थ‚स्य ‚{\color{DodgerBlue3}‚सूक्ष्मोऽशो}‚ऽव‚य‚वः ‚{\color{DodgerBlue3}‚क‚श्च‚न} \leavevmode\ledsidenote{\textenglish{241/s}} न दृश्य‚तेऽतः । स‚न्निकृष्टाद् व्य‚क्तं दृश्य‚मानाद‚र्थाद‚व्य‚क्त‚त्वेन ‚{\color{DodgerBlue3}‚भिदा य‚द्युच्य‚ते} । ‚{\tiny $_{lb}$}‚त‚दापि द्व‚यी क‚ल्प‚ना । (४०९)
	\pend% ending standard par
      \label{div_pvv.2.410}
	  
	% new div opening: depth here is 2
	

	  \pstart \leavevmode% starting standard par
	योसौ स्थ‚वीयान् दृश्य‚ते स एकोऽनेको वा ।
	\pend% ending standard par
      
	  \bigskip
	  \begingroup
	
	    \large
	  
	    \begin{quote}
	  
	    
	    \stanza[\smallbreak]
	\label{pv.2.410}\flagstanza{\tiny\textenglish{....2.410}}एक‚त्वेर्थ‚स्य बाह्य‚स्य दृश्यादृश्य‚भिदा कुतः ॥&अनेक‚त्वेऽणुशो भिन्ने दृश्यादृश्य‚भिदा कुतः ॥ ४१० ॥\&[\smallbreak]


	
	    \end{quote}
	  
	  \endgroup
	

	  \pstart \leavevmode% starting standard par
	\hphantom{.}त‚त्रै‚{\color{DodgerBlue3}‚क‚त्वेऽर्थ‚स्य बाह्य‚{\tiny $_{1}$}‚स्या}‚भ्युप‚ग‚म्य‚माने ‚{\color{DodgerBlue3}‚दृश्यादृश्य‚भिदा कुतः} श‚ङ्किता(।) एवं ‚{\tiny $_{lb}$}‚दृश्य‚म‚दृश्य‚मेव वा स्यादेकान्तेन । अनेक‚त्वे दृश्य‚स्यार्थ‚स्याभ्युप‚ग‚म्य‚मानेऽणुशो ‚{\tiny $_{lb}$}‚भिन्नेऽस्मिन् दृश्यादृश्य‚भिदा कुतः । न ह्य‚णुष्व‚पि स्थूल‚सूक्ष्म‚भेदः । येन किञ्चिदुप‚{\tiny $_{lb}$}‚ल‚भ्येत किञ्चिन्नेति विभागः । (४१०)
	\pend% ending standard par
      \label{div_pvv.2.411}
	  
	% new div opening: depth here is 2
	
	  \bigskip
	  \begingroup
	
	    \large
	  
	    \begin{quote}
	  
	    
	    \stanza[\smallbreak]
	\label{pv.2.411}\flagstanza{\tiny\textenglish{....2.411}}मान्द्य‚पाट‚व‚भेदेन भासो बुद्धिभिदा य‚दि ।&भिन्नेऽन्य‚स्मिन्न‚भिन्न‚स्य कुतो भेदेन भास‚न‚म् ॥ ४११ ॥\&[\smallbreak]


	
	    \end{quote}
	  
	  \endgroup
	

	  \pstart \leavevmode% starting standard par
	\hphantom{.}अथैक‚रूपेप्य‚र्थे ‚{\color{DodgerBlue3}‚भास} आलोक‚स्य ‚{\color{DodgerBlue3}‚मान्द्य‚पाट‚व‚भेदेन} स्प‚ष्टास्प‚ष्ट‚त‚या ‚{\color{DodgerBlue3}‚भेदो ‚{\tiny $_{lb}$}‚बुद्धेर्य‚दी}‚ष्य‚ते ।‚{\tiny $_{3}$}‚ त‚दाप्य‚{\color{DodgerBlue3}‚न्य‚स्मिन्ना}‚लोके प‚टुम‚न्द‚त‚या ‚{\color{DodgerBlue3}‚भिन्नेऽभिन्न‚स्या}‚र्थ‚स्य स्व‚रूपेण ‚{\tiny $_{lb}$}‚दृश्य‚मान‚स्य ‚{\color{DodgerBlue3}‚कुतो भेदेन} स्प‚ष्टास्प‚ष्ट‚त‚या ‚{\color{DodgerBlue3}‚भास‚नं} युक्तं । (४११)
	\pend% ending standard par
      \label{div_pvv.2.412}
	  
	% new div opening: depth here is 2
	

	  \pstart \leavevmode% starting standard par
	किञ्च (।)
	\pend% ending standard par
      
	  \bigskip
	  \begingroup
	
	    \large
	  
	    \begin{quote}
	  
	    
	    \stanza[\smallbreak]
	\label{pv.2.412}\flagstanza{\tiny\textenglish{....2.412}}म‚न्द‚न्त‚द‚पि तेजः किमावृत्तेरिह सा न किम् ।&त‚नुत्व‚न्तेज‚सोप्येत‚द‚स्त्य‚न्य‚त्राप्य‚तान‚व‚म् ॥ ४१२ ॥\&[\smallbreak]


	
	    \end{quote}
	  
	  \endgroup
	

	  \pstart \leavevmode% starting standard par
	\hphantom{.}व्य‚व‚हित‚व‚स्त्व‚न्त‚राल‚व‚र्त्ति ‚{\color{DodgerBlue3}‚तेजो म‚न्दं किं} क‚स्माद्धेतो र‚जोनीहारादिभिस्त‚{\tiny $_{lb}$}‚द्देश‚व‚र्त्तिभिरा‚{\color{DodgerBlue3}‚वृत्ते}‚रालोको म‚न्द इति चेत् । इह स‚न्निहित‚व‚स्त्व‚न्त‚राल‚व‚र्त्तिन्यालोके ‚{\color{DodgerBlue3}‚सा} र‚जोनीहारादिभिरावृत्तिः ‚{\color{DodgerBlue3}‚किन्न} भ‚व‚ति ।‚{\tiny $_{3}$}‚ ‚{\color{DodgerBlue3}‚त‚नुत्वा}‚दावार‚क‚स्य नीहारादेर्न्नावृत्तिश्चेत् । ‚{\tiny $_{lb}$}‚‚{\color{DodgerBlue3}‚त‚त्त‚नुत्वं तेज‚सोपि} स‚न्निहित‚व‚स्त्व‚न्त‚राल‚व‚र्त्तिनो‚{\color{DodgerBlue3}‚ऽस्तीति} स‚न्निहित‚ञ्च व‚स्तु न स्फुटं ‚{\tiny $_{lb}$}‚प्र‚तीय‚ते त‚था‚{\color{DodgerBlue3}‚ऽन्य‚त्र} दूर‚स्थं व‚स्तु स्फुटं प्र‚तीयेत् । दूर‚स्थे व‚स्तुन्यावार‚कालोक‚योः ‚{\tiny $_{lb}$}‚स‚मान‚म‚तान‚वं, स‚मीप‚स्थे च स‚मं तान‚व‚मिति न स्यात् प्र‚तीतिभेदः । (४१२)
	\pend% ending standard par
      \label{div_pvv.2.413}
	  
	% new div opening: depth here is 2
	

	  \pstart \leavevmode% starting standard par
	किञ्च (।)
	\pend% ending standard par
      
	  \bigskip
	  \begingroup
	
	    \large
	  
	    \begin{quote}
	  
	    
	    \stanza[\smallbreak]
	\label{pv.2.413}\flagstanza{\tiny\textenglish{....2.413}}अत्यास‚न्ने च सुव्य‚क्तं तेज‚स्त‚त्स्याद‚तिस्फुट‚म् ।&त‚त्राप्य‚दृष्ट‚माश्रित्य भ‚वेद्रूपान्त‚रं य‚दि ॥ ४१३ ॥\&[\smallbreak]


	
	    \end{quote}
	  
	  \endgroup
	

	  \pstart \leavevmode% starting standard par
	\hphantom{.}लोच‚न‚स्या‚{\color{DodgerBlue3}‚त्यास‚न्ने} श‚लाकादौ ‚{\color{DodgerBlue3}‚सुव्य‚क्तं तेज} आवार‚क‚स्य त‚नुत्वादिति त‚द‚त्या‚{\tiny $_{3}$}‚‚{\tiny $_{lb}$}‚स‚न्नं श‚लाकादिक‚{\color{DodgerBlue3}‚म‚तिस्फुटं स्यात्} । न च म‚नाग‚व्य‚व‚हित‚मिवात्यास‚न्नं व‚स्तु स्फुट‚{\tiny $_{lb}$}‚\leavevmode\ledsidenote{\textenglish{242/s}} मीक्ष्य‚ते । ‚{\color{DodgerBlue3}‚त‚त्र} दूरात्यास‚त्तिभेदेन व्य‚क्ताव्य‚क्त‚द‚र्श‚ने‚{\color{DodgerBlue3}‚प्य‚दृष्टं} ध‚र्माध‚र्म‚{\color{DodgerBlue3}‚माश्रि}‚त्यापेक्ष्य ‚{\tiny $_{lb}$}‚‚{\color{DodgerBlue3}‚रूपान्त‚रं} व्य‚क्ताव्य‚क्तं जाय‚त इति य‚द्युच्य‚ते (। ४१३)
	\pend% ending standard par
      \label{div_pvv.2.414}
	  
	% new div opening: depth here is 2
	
	  \bigskip
	  \begingroup
	
	    \large
	  
	    \begin{quote}
	  
	    
	    \stanza[\smallbreak]
	\label{pv.2.414}\flagstanza{\tiny\textenglish{....2.414}}अन्योन्याव‚र‚णात्तेषां स्यात्तेजोविह‚तिस्त‚तः ।&त‚त्रैक‚मेव दृश्येत त‚स्यानाव‚र‚णे स‚कृत् ॥ ४१४ ॥\&[\smallbreak]


	
	    \end{quote}
	  
	  \endgroup
	

	  \pstart \leavevmode% starting standard par
	\hphantom{.}त‚दा ‚{\color{DodgerBlue3}‚तेषां} व्य‚क्ताव्य‚क्तानां रूपाणा‚{\color{DodgerBlue3}‚म‚न्योन्य}‚स्या‚{\color{DodgerBlue3}‚व‚र‚णात्} क‚दाचित्क‚स्य‚चिदुप‚{\tiny $_{lb}$}‚ल‚म्भो भ‚व‚तीति व‚क्त‚व्यं । ‚{\color{DodgerBlue3}‚त‚त} एकोप‚ल‚म्भ‚कालेऽप‚रोप‚ल‚म्भ‚हेतो‚{\color{DodgerBlue3}‚स्तेज‚सो‚{\tiny $_{5}$}‚विह‚ति}‚रा‚{\tiny $_{lb}$}‚वृतिरिति च ‚{\color{DodgerBlue3}‚स्यात्} । अन्य‚था नावृत्ते त‚स्मिन्न‚प‚र‚स्याप्युप‚ल‚ब्धिः स्याद‚वैक‚ल्यात्साम‚{\tiny $_{lb}$}‚ग्र्याः । ‚{\color{DodgerBlue3}‚त‚त्रा}‚प‚रोप‚ल‚म्भ‚हेतोरालोक‚स्याव‚र‚णे स‚ति ‚{\color{DodgerBlue3}‚एक‚मेव} व्य‚क्त‚म‚व्य‚क्त‚स्व‚रूपं दूरा‚{\tiny $_{lb}$}‚स‚न्नादिदेश‚स्थितैः प्र‚तिप‚त्तृभिः स‚र्व्वैर्दृश्येत । न त्वेकेन व्य‚क्त‚मित‚रेण चाव्य‚क्त‚मिति ‚{\tiny $_{lb}$}‚स्यात् । ‚{\color{DodgerBlue3}‚त‚स्या}‚दृष्टोत्प‚न्न‚रूप‚स्य प‚र‚स्प‚र‚म‚नाव‚र‚णे तेज‚स‚श्चा‚{\color{DodgerBlue3}‚नाव‚र‚णे स‚कृत्} (। ४१४)
	\pend% ending standard par
      \label{div_pvv.2.415}
	  
	% new div opening: depth here is 2
	
	  \bigskip
	  \begingroup
	
	    \large
	  
	    \begin{quote}
	  
	    
	    \stanza[\smallbreak]
	\label{pv.2.415}\flagstanza{\tiny\textenglish{....2.415}}प‚श्येत् स्फुटास्फुटं रूप‚मेकोऽदृष्टेन वार‚णे ।&अर्थान‚र्थौ न येन स्त‚स्त‚द‚दृष्टं क‚रोति किम् ॥ ४१५ ॥\&[\smallbreak]


	
	    \end{quote}
	  
	  \endgroup
	\textsuperscript{\textenglish{47b/MA}}

	  \pstart \leavevmode% starting standard par
	\hphantom{.}‚{\color{DodgerBlue3}‚स्फुटास्फुटं रू‚{\tiny $_{6}$}‚पं प‚श्येदे}‚कः\edtext{}{\edlabel{pvv.242-1}\label{pvv.242-1}\lemma{कः}\Bfootnote{एकैको दूरास‚न्न‚स्थः स‚र्व्वः ।}} प्र‚तिप‚त्ता । दृश्य‚रूप‚द्व\edtext{}{\edlabel{pvv.242-2}\label{pvv.242-2}\lemma{द्व}\Bfootnote{युग‚प‚त् स्फुटानि भ‚व‚न्ति रूपाणि ॥}}य‚स्य द‚र्श‚न‚हेतोश्चालोक‚{\tiny $_{lb}$}‚स्यानाव‚र‚णाद‚दृष्टेन द्वितीय‚स्य रूप‚स्य । व‚र‚णे व्य‚क्त‚म‚व्य‚क्त‚मेव वा रूप‚मेकं दृश्य‚त ‚{\tiny $_{lb}$}‚इति\edtext{}{\edlabel{pvv.242-3}\label{pvv.242-3}\lemma{इति}\Bfootnote{सौम‚न‚स्योत्पाद‚नेनानुग्राह‚कं ध‚र्मी द‚र्श‚य‚ति । दुःखोत्पाद‚नेने(?नो)प‚पात‚क‚मावृणोति विप‚र्य्य‚याद‚ध‚र्मः ।}} चेत् । ‚{\color{DodgerBlue3}‚येन} द्वितीय‚रूपाव‚र‚णेन कृतेन पुंसो‚{\color{DodgerBlue3}‚ऽर्थान‚र्थाव}‚दृष्ट‚कार्यौ ‚{\color{DodgerBlue3}‚न स्तः} स‚म्भ‚व‚त‚{\tiny $_{lb}$}‚‚{\color{DodgerBlue3}‚स्त‚दाव‚र‚ण‚म‚दृष्टं} क‚र्त्तृ ‚{\color{DodgerBlue3}‚किं} क‚स्मात् ‚{\color{DodgerBlue3}‚क‚रोति} शुभाशुभ‚ल‚क्ष‚णं ह्य‚दृष्ट‚म‚र्थान‚र्थ‚फ‚लं(।) ‚{\tiny $_{lb}$}‚य‚त्पुन‚र‚नुभ‚व‚स्व‚भावं त‚त्‚{\tiny $_{1}$}‚ दृष्ट‚फ‚ल‚मेव भ‚व‚ति (। ४१५)
	\pend% ending standard par
      \label{div_pvv.2.416}
	  
	% new div opening: depth here is 2
	

	  \pstart \leavevmode% starting standard par
	य‚स्मात् स‚र्व्व‚म‚न‚न्त‚रोक्त‚म‚संग‚तं ।
	\pend% ending standard par
      
	  \bigskip
	  \begingroup
	
	    \large
	  
	    \begin{quote}
	  
	    
	    \stanza[\smallbreak]
	\label{pv.2.416}\flagstanza{\tiny\textenglish{....2.416}}त‚स्मात् संविद् य‚थाहेतु जाय‚मानार्थ‚संश्र‚यात् ।&प्र‚तिभास‚भिदां ध‚त्ते शेषाः कुम‚ति-दुर्न्न‚याः ॥ ४१६ ॥\&[\smallbreak]


	
	    \end{quote}
	  
	  \endgroup
	

	  \pstart \leavevmode% starting standard par
	\hphantom{.}‚{\color{DodgerBlue3}‚त‚स्माद‚र्थ‚संश्र‚या\edtext{}{\edlabel{pvv.242-4}\label{pvv.242-4}\lemma{या}\Bfootnote{अभ्युप‚ग‚मेप्य‚र्थ‚स्य ।}}ज्‏जाय‚माना संवित्} बुद्धिर्य‚थाहेतु वास‚नाप्र‚बोध‚हेत्व‚न‚तिक्र‚मेण ‚{\tiny $_{lb}$}‚‚{\color{DodgerBlue3}‚प्र‚तिभास}‚स्याकार‚स्य व्य‚क्ताव्य‚क्तादे‚{\color{DodgerBlue3}‚र्भिदां ध‚त्ते} विभ‚र्त्तीति\edtext{}{\edlabel{pvv.242-5}\label{pvv.242-5}\lemma{र्त्तीति}\Bfootnote{द्वैरूप्यं}} न्याय्यं । त‚दित‚रे पुन‚रा‚{\tiny $_{lb}$}‚लोक‚भेदोप‚न्यासाद्याः ‚{\color{DodgerBlue3}‚शेषाः कुम‚तिर्दुर्न‚याः} प‚र‚वादिनां कुम‚तीनां दुर्व्विम‚र्शाः\edtext{}{\edlabel{pvv.242-6}\label{pvv.242-6}\lemma{र्शाः}\Bfootnote{अर्थ‚व्य‚क्त्य‚संभ‚व‚न्द‚र्श‚य‚न् बुद्धेर्द्वैरूप्य‚माह ।}}(। ४१६)
	\pend% ending standard par
      \label{div_pvv.2.417}
	  
	% new div opening: depth here is 2
	\textsuperscript{\textenglish{243/s}}

	  \pstart \leavevmode% starting standard par
	किञ्च(।) अनाकारेण ज्ञाने‚{\tiny $_{2}$}‚नार्थःक्ष‚णि\edtext{}{\edlabel{pvv.243-1}\label{pvv.243-1}\lemma{णि}\Bfootnote{नित्यो दीपः क‚स्य‚चित् । श‚ब्दो नित्यो वैयाक‚र‚णादेः बुद्धिर्नित्या सांख्य‚स्य ।}}कोऽक्ष‚णिको वा व्य‚ज्येत(।) त‚त्र(।)
	\pend% ending standard par
      
	  \bigskip
	  \begingroup
	
	    \large
	  
	    \begin{quote}
	  
	    
	    \stanza[\smallbreak]
	\label{pv.2.417}\flagstanza{\tiny\textenglish{....2.417}}ज्ञान‚श‚ब्द‚प्र‚दीपानां प्र‚त्य‚क्ष‚स्येत‚र‚स्य वा ।&ज‚न‚क‚त्वेन पूर्वेषां क्ष‚णिकानां विनाश‚तः ॥ ४१७ ॥\&[\smallbreak]


	
	    \end{quote}
	  
	  \endgroup
	

	  \pstart \leavevmode% starting standard par
	\hphantom{.}‚{\color{DodgerBlue3}‚क्ष‚णिकानां ज्ञान‚श‚ब्द‚प्र‚दीपादीनां} स्व‚विष‚य‚स्य ‚{\color{DodgerBlue3}‚प्र\edtext{}{\edlabel{pvv.243-2}\label{pvv.243-2}\lemma{प्र}\Bfootnote{य‚दा स्व‚रूपानुकारि ज्ञान‚म‚व्य‚व‚हितं विक‚ल्प‚ज्ञान‚ज‚न‚ने ।}}त्य‚क्ष}‚स्याप्र‚त्य‚क्ष‚स्य ‚{\color{DodgerBlue3}‚वा} ज्ञान‚स्य ‚{\tiny $_{lb}$}‚‚{\color{DodgerBlue3}‚ज‚न‚क‚त्वेन} हे\edtext{}{\edlabel{pvv.243-3}\label{pvv.243-3}\lemma{हे}\Bfootnote{ज्ञानादीनां ।}}तूनां ‚{\color{DodgerBlue3}‚पू\edtext{}{\edlabel{pvv.243-4}\label{pvv.243-4}\lemma{पू}\Bfootnote{नाकार‚णं विष‚यः ।}}र्व्वेषां} ज्ञान‚काले ‚{\color{DodgerBlue3}‚विनाश‚तः} (४१७)
	\pend% ending standard par
      \label{div_pvv.2.418}
	  
	% new div opening: depth here is 2
	
	  \bigskip
	  \begingroup
	
	    \large
	  
	    \begin{quote}
	  
	    
	    \stanza[\smallbreak]
	\label{pv.2.418a}\flagstanza{\tiny\textenglish{...2.418a}}व्य‚क्तिः कुतोऽस‚तां ज्ञानाद्;\&[\smallbreak]


	
	    \end{quote}
	  
	  \endgroup
	

	  \pstart \leavevmode% starting standard par
	\hphantom{.}‚{\color{DodgerBlue3}‚अस‚तां ज्ञानात् कुतो व्य‚क्तिः} । य‚दाऽर्थ‚स्त‚दा न ज्ञानं य‚दा ज्ञानं ‚{\color{DodgerBlue3}‚त‚दा नार्थ इति} कुतो व्य\edtext{}{\edlabel{pvv.243-5}\label{pvv.243-5}\lemma{व्य}\Bfootnote{ज्ञान‚स्यैव विष‚याकारः सिद्धः ।}}ङ्ग्य‚व्य‚ञ्ज‚क‚भाव‚स्त‚योः । अथ श‚ब्दाद‚यो ज्ञानेन स‚ह द्वितीयं स्वोपादेय‚{\tiny $_{lb}$}‚क्ष‚णं ज‚न‚य‚न्ति स एव तेन ज्ञानेन व्य‚ज्य‚{\tiny $_{3}$}‚ते नेत‚र‚दित्याह (।)
	\pend% ending standard par
      
	  \bigskip
	  \begingroup
	
	    \large
	  
	    \begin{quote}
	  
	    
	    \stanza[\smallbreak]
	\label{pv.2.418b}\flagstanza{\tiny\textenglish{...2.418b}}अन्य‚स्यानुप‚कारिणः ।&व्य‚क्तौ व्य‚ज्येत स‚र्व्वोर्थ‚स्त‚द्धेतोर्निय‚मो य‚दि ॥ ४१८ ॥\&[\smallbreak]


	
	    \end{quote}
	  
	  \endgroup
	

	  \pstart \leavevmode% starting standard par
	\hphantom{.}स्व‚कार‚णाद‚{\color{DodgerBlue3}‚न्य‚स्य} स‚होत्प‚न्न‚स्या‚{\color{DodgerBlue3}‚नुप‚कारिणो व्य‚क्ति}‚विव‚क्षायां स‚र्व्वोऽर्थः स‚मान‚{\tiny $_{lb}$}‚कालेन ज्ञानेन ‚{\color{DodgerBlue3}‚व्य‚ज्य}‚ताम‚नुप‚कार‚क‚त्वाविशेषात् । ‚{\color{DodgerBlue3}‚त‚स्मा}‚त्स‚होत्पाद‚कात् ‚{\color{DodgerBlue3}‚हेतोर}‚नुप‚कार‚{\tiny $_{lb}$}‚क‚त्व‚विशेषेष्व‚पि स‚होत्प‚न्न एवार्थो ‚{\color{DodgerBlue3}‚ज्ञानेन} व्य‚ज्य‚त इति ‚{\color{DodgerBlue3}‚निय‚मो} य‚दि क‚ल्प्य‚ते(४१८)
	\pend% ending standard par
      \label{div_pvv.2.419}
	  
	% new div opening: depth here is 2
	
	  \bigskip
	  \begingroup
	
	    \large
	  
	    \begin{quote}
	  
	    
	    \stanza[\smallbreak]
	\label{pv.2.419}\flagstanza{\tiny\textenglish{....2.419}}नैषापि क‚ल्प‚ना ज्ञाने ज्ञान‚न्त्व‚र्थाव‚भास‚तः ।&तं व्य‚न‚क्तीतिं क‚थ्येत त‚द‚भावेपि त‚त्कृत‚म् ॥ ४१९ ॥\&[\smallbreak]


	
	    \end{quote}
	  
	  \endgroup
	

	  \pstart \leavevmode% starting standard par
	\hphantom{.}त‚दैषापि ‚{\color{DodgerBlue3}‚क‚ल्प‚ना न} युक्ता ज्ञाने व्य‚ञ्ज‚क‚त्व‚स्य स\edtext{}{\edlabel{pvv.243-6}\label{pvv.243-6}\lemma{स}\Bfootnote{इन्द्रिंयेणापि ज्ञान‚हेतुना य‚ज्ज‚नित‚मिन्द्रिय‚न्त‚द्विष‚यः स्यात् ।}}होत्प‚न्नेन्द्रियादिव्य‚ञ्ज‚क‚त्व‚{\tiny $_{lb}$}‚प्र‚स‚ङ्गा‚{\tiny $_{4}$}‚त् । अस्माकं म‚ते तु साकारं तेनार्थेन कृतं ‚{\color{DodgerBlue3}‚ज्ञान‚मिति} ज्ञान‚स्य काले ‚{\tiny $_{lb}$}‚‚{\color{DodgerBlue3}‚त‚स्यार्थ}‚स्या‚{\color{DodgerBlue3}‚भावे}‚प्य‚र्था‚{\color{DodgerBlue3}‚व‚भास‚तो}‚र्थाकारात्संवेद्य‚मानात् ‚{\color{DodgerBlue3}‚त‚म‚र्थं व्य‚न‚क्तीति क‚थ्य‚ते}‚ऽन्य‚{\tiny $_{lb}$}‚स्यार्थ‚व्य‚क्तिप्र‚कार‚स्यायोगात् (। ४१९)
	\pend% ending standard par
      \label{div_pvv.2.420}
	  
	% new div opening: depth here is 2
	

	  \pstart \leavevmode% starting standard par
	स‚होत्प‚न्न‚स्यापि त‚र्हि स्वाकार‚ज्ञानेन व्य‚क्तिः स्यादित्याह (।)
	\pend% ending standard par
      
	  \bigskip
	  \begingroup
	
	    \large
	  
	    \begin{quote}
	  
	    
	    \stanza[\smallbreak]
	\label{pv.2.420}\flagstanza{\tiny\textenglish{....2.420}}नाकार‚य‚ति चान्योर्थोऽनुप‚कारात् स‚होदितः ।&व्य‚क्तोनाकार‚य‚ज्ज्ञानं स्वाकारेण क‚थं भ‚वेत् ॥ ४२० ॥\&[\smallbreak]


	
	    \end{quote}
	  
	  \endgroup
	

	  \pstart \leavevmode% starting standard par
	\hphantom{.}कार‚णाद‚न्य‚स्या‚{\color{DodgerBlue3}‚न्य‚श्चार्थः स‚हो}‚त्प‚न्नो ज्ञानं ‚{\color{DodgerBlue3}‚नाकार‚य‚ति स्वाकारेण} विशेष‚य‚ति ‚{\tiny $_{lb}$}‚‚{\color{DodgerBlue3}‚अनुप‚कारात्} । न ह्य‚नुप‚कार(का)कारे‚{\tiny $_{5}$}‚ण विशिष्य‚तेऽतिप्र‚स‚ङ्गात् । य‚श्चार्थो ‚{\tiny $_{lb}$}‚‚{\color{DodgerBlue3}‚ज्ञानं नाकार‚य‚ति स क‚थं व्य‚क्तो भ‚वेत्} । (४२०)
	\pend% ending standard par
      \label{div_pvv.2.421}
	  
	% new div opening: depth here is 2
	\textsuperscript{\textenglish{244/s}}

	  \begin{center}%% label @type='head'
	\textbf{ग. अक्ष‚णिक‚स्य व्य‚क्तिर‚स‚म्भ‚वा}
	\end{center}
	

	  \pstart \leavevmode% starting standard par
	अक्ष‚णिक‚स्याप्य‚र्थ‚स्य व्य‚क्तिं निषेद्ध्ुमाह (।)
	\pend% ending standard par
      
	  \bigskip
	  \begingroup
	
	    \large
	  
	    \begin{quote}
	  
	    
	    \stanza[\smallbreak]
	\label{pv.2.421}\flagstanza{\tiny\textenglish{....2.421}}व‚ज्रोप‚लादिर‚प्य‚र्थः स्थिरः सोन्यान‚पेक्ष‚णात् ।&स‚कृत् स‚र्व‚स्य ज‚न‚येज्ज्ञानानि ज‚ग‚तः स‚म‚म् ॥ ४२१ ॥\&[\smallbreak]


	
	    \end{quote}
	  
	  \endgroup
	

	  \pstart \leavevmode% starting standard par
	\hphantom{.}यो ‚{\color{DodgerBlue3}‚व‚ज्रोप‚लादिः} स्थिरो‚{\color{DodgerBlue3}‚ऽर्थः सोऽपि} ज्ञानोत्पाद‚न‚स्व‚भाव‚त्वे‚{\color{DodgerBlue3}‚नान्य‚स्य} स‚ह‚कारि‚{\tiny $_{lb}$}‚णोऽनुप‚कार‚क‚स्या‚{\color{DodgerBlue3}‚न‚पेक्ष‚णात् स‚कृत्स‚र्व्व‚स्य ज‚ग‚तः} स्व‚ग्राह‚काणि ‚{\color{DodgerBlue3}‚ज्ञाना}‚नि ‚{\color{DodgerBlue3}‚स‚म‚मे}‚क‚{\tiny $_{lb}$}‚कालं ‚{\color{DodgerBlue3}‚ज‚न‚येत्} । (४२१)
	\pend% ending standard par
      \label{div_pvv.2.422}
	  
	% new div opening: depth here is 2
	

	  \pstart \leavevmode% starting standard par
	न चैत‚द‚स्ति (।)
	\pend% ending standard par
      
	  \bigskip
	  \begingroup
	
	    \large
	  
	    \begin{quote}
	  
	    
	    \stanza[\smallbreak]
	\label{pv.2.422}\flagstanza{\tiny\textenglish{....2.422}}क्र‚माद् भ‚व‚न्ति तान्य‚स्य स‚ह‚कार्युप‚कार्य‚तः ।&आहुः प्र‚तिक्ष‚णं भेदं स दोषोऽत्रापि पूर्व‚व‚त् ॥ ४२२ ॥\&[\smallbreak]


	
	    \end{quote}
	  
	  \endgroup
	\textsuperscript{\textenglish{48a/MA}}

	  \pstart \leavevmode% starting standard par
	\hphantom{.}क्र‚मेणोत्पादात् ‚{\color{DodgerBlue3}‚क्र‚माद्} भ‚व‚न्ति जाय‚मानानि‚{\tiny $_{6}$}‚ ‚{\color{DodgerBlue3}‚तानि} व‚ज्रादिज्ञानान्य‚स्य व‚ज्रोप‚{\tiny $_{lb}$}‚लादेः । ‚{\color{DodgerBlue3}‚स‚ह‚कारि}‚णामुप‚कारात् स्व‚भावान्त‚र‚ल‚क्ष‚णात् ‚{\color{DodgerBlue3}‚प्र‚तिक्ष‚णं भेद}\edtext{}{\edlabel{pvv.244-1}\label{pvv.244-1}\lemma{णात्}\Bfootnote{कार‚काधार‚क‚त्वेन ।}} म‚न्य‚स्व‚भाव‚ता\edtext{}{\edlabel{pvv.244-2}\label{pvv.244-2}\lemma{ता}\Bfootnote{क्ष‚णिक‚त्वं ।}}‚{\tiny $_{lb}$}‚‚{\color{DodgerBlue3}‚माहुः}‚, य‚था च ‚{\color{DodgerBlue3}‚पू\edtext{}{\edlabel{pvv.244-3}\label{pvv.244-3}\lemma{पू}\Bfootnote{विनाशाद‚स‚तां कुतो ज्ञानाद् व्य‚क्तिरिति ग्राहिक‚या ।}}र्व्व‚व‚त्} श‚ब्दादिष्वि‚{\color{DodgerBlue3}‚वात्र} व‚ज्रादिष्व‚पि क्ष‚णिकेषु विज्ञानात् पूर्व्व‚{\tiny $_{lb}$}‚काल‚भाविषु ‚{\color{DodgerBlue3}‚सो}‚ऽव्य‚क्ति‚{\color{DodgerBlue3}‚दोष}‚प्र‚स‚ङ्ग‚स्त‚द‚व‚स्थः । (४२२)
	\pend% ending standard par
      \label{div_pvv.2.423}
	  
	% new div opening: depth here is 2
	

	  \begin{center}%% label @type='head'
	\textbf{(३) a. स्व‚संवेद‚न‚चिन्ता}
	\end{center}
	

	  \begin{center}%% label @type='head'
	\textbf{क. बुद्धिर‚र्थाकारा}
	\end{center}
	

	  \pstart \leavevmode% starting standard par
	पुन‚र्बुद्धेर‚र्थाकार‚सिद्ध्य‚र्थ‚माह (।)
	\pend% ending standard par
      
	  \bigskip
	  \begingroup
	
	    \large
	  
	    \begin{quote}
	  
	    
	    \stanza[\smallbreak]
	\label{pv.2.423}\flagstanza{\tiny\textenglish{....2.423}}संवेद‚न‚स्य तादात्म्ये न विवादोस्ति क‚स्य‚चित् ।&त‚स्यार्थ‚रूप‚ताऽसिद्धा सापि सिध्य‚ते संस्मृतेः ॥ ४२३ ॥\&[\smallbreak]


	
	    \end{quote}
	  
	  \endgroup
	

	  \pstart \leavevmode% starting standard par
	\hphantom{.}‚{\color{DodgerBlue3}‚स‚म्वेद\edtext{}{\edlabel{pvv.244-4}\label{pvv.244-4}\lemma{म्वेद}\Bfootnote{य‚स्माच्चानुभ‚वोत्त‚र‚कालं विष‚य इव ज्ञाने स्मृतिरुत्प‚द्य‚ते, त‚स्माद‚स्ति द्विरूप‚ता ज्ञान‚स्येत्यादि व्याच‚ष्टे ॥}}न‚स्य तादात्म्ये}‚ऽनुभ‚व‚रूप‚त्वे ‚{\color{DodgerBlue3}‚क‚स्य‚चि}‚द्विदुषो ‚{\color{DodgerBlue3}‚वा\edtext{}{\edlabel{pvv.244-5}\label{pvv.244-5}\lemma{वा}\Bfootnote{ज्ञानं न ज्ञान‚रूप‚मितिव‚त् ।}}दो नास्ति । त‚स्य} संवेद‚न‚स्या‚{\color{DodgerBlue3}‚र्थ‚रूप‚ता} विवादाद‚{\color{DodgerBlue3}‚सिद्धा}\edtext{}{\edlabel{pvv.244-6}\label{pvv.244-6}\lemma{विवादाद}\Bfootnote{विष‚य‚ज्ञान‚त‚ज्ज्ञानेत्यादौ एक‚त्र ज्ञाने विष‚याकारोल्लेखेन द्वैरूप्य‚मुक्त‚म‚त्र ।}} ‚{\color{DodgerBlue3}‚साऽर्था}‚{\tiny $_{1}$}‚कार‚ता‚{\color{DodgerBlue3}‚पि संस्मृतेर}‚र्थाभासानुभ‚व‚स्य ‚{\tiny $_{lb}$}‚स‚म्य\edtext{}{\edlabel{pvv.244-7}\label{pvv.244-7}\lemma{म्य}\Bfootnote{य‚था प‚र‚स्प‚र‚विल‚क्ष‚णेषु रूपादिष्व‚नुभूतेष्व‚न्योन्य‚विवेकेन स्मृतिः स्यात्त‚था ज्ञानेपि स्मृतिरुत्प‚द्य‚ते त‚द‚स्ति द्वैरूप्य‚म‚र्थ‚वेद‚न‚म्विनार्थ‚स्मृतेर‚योगाद‚स्ति च सात्र ।}}क् स्म‚र‚णात् ‚{\color{DodgerBlue3}‚सिध्य\edtext{}{\edlabel{pvv.244-8}\label{pvv.244-8}\lemma{सिध्य}\Bfootnote{कुत एत‚द्य‚स्मात् ।}}ति} । (४२३)
	\pend% ending standard par
      \label{div_pvv.2.424}
	  
	% new div opening: depth here is 2
	\textsuperscript{\textenglish{245/s}}
	  \bigskip
	  \begingroup
	
	    \large
	  
	    \begin{quote}
	  
	    
	    \stanza[\smallbreak]
	\label{pv.2.424}\flagstanza{\tiny\textenglish{....2.424}}भेदेनान‚नुभूतेस्मिन्न‚विभ‚क्ते स्व‚गोच‚रैः ।&एव‚मेत‚न्न ख‚ल्वेव‚मिति सा स्यान्न भेदिनि ॥ ४२४ ॥\&[\smallbreak]


	
	    \end{quote}
	  
	  \endgroup
	

	  \pstart \leavevmode% starting standard par
	\hphantom{.}‚{\color{DodgerBlue3}‚अस्मिन्न}‚र्थ‚संवेद‚नेंन ‚{\color{DodgerBlue3}‚भेदेनानुभूते} स्व स्य ‚{\color{DodgerBlue3}‚गोच‚रैर‚र्थेः} स्वाकार‚स‚म‚र्प‚ण‚द्वारेणा‚{\tiny $_{lb}$}‚‚{\color{DodgerBlue3}‚विभ‚क्ते} प‚र‚स्प‚र‚तो भेदेन व्य‚व‚स्थापिते ‚{\color{DodgerBlue3}‚एत‚ज्ज्ञा}‚न‚मेवं घ‚ट‚ग्रा\edtext{}{\edlabel{pvv.245-1}\label{pvv.245-1}\lemma{ग्रा}\Bfootnote{स्व‚स‚त्ताकाले}}ह‚कं । ‚{\color{DodgerBlue3}‚न ख‚ल्वेवं नैव} घ‚ट‚ग्राह‚क‚{\color{DodgerBlue3}‚मिति । सा} स‚म्वेद‚न‚स्मृति‚{\color{DodgerBlue3}‚र्भेदिनी} विभाग‚व‚ती ‚{\color{DodgerBlue3}‚न स्यात्} ज्ञानेन स‚ह ‚{\tiny $_{lb}$}‚(। ४२४)
	\pend% ending standard par
      \label{div_pvv.2.425}
	  
	% new div opening: depth here is 2
	
	  \bigskip
	  \begingroup
	
	    \large
	  
	    \begin{quote}
	  
	    
	    \stanza[\smallbreak]
	\label{pv.2.425}\flagstanza{\tiny\textenglish{....2.425}}न चानुभ‚व‚मात्रेण क‚श्चिद् भेदो विवेच‚कः ।&विवेकिनी न चास्प‚ष्ट‚भेदे धीर्य‚म‚लादिव‚त् ॥ ४२५ ॥\&[\smallbreak]


	
	    \end{quote}
	  
	  \endgroup
	

	  \pstart \leavevmode% starting standard par
	\hphantom{.}‚{\color{DodgerBlue3}‚न चानुभ‚व‚मात्रेणा}‚वान्त‚र‚भिन्नेन प्र‚तिज्ञानं ‚{\color{DodgerBlue3}‚क‚श्चिद् भेदो} विद्य‚मानोपि प‚र‚स्प‚रं ‚{\tiny $_{lb}$}‚‚{\color{DodgerBlue3}‚विवेच‚को} भेद‚व्य‚{\tiny $_{2}$}‚व‚स्थाप‚न‚हेतुः । त‚थाविधा‚{\color{DodgerBlue3}‚ऽस्प‚ष्टे भेदे} स‚त्य‚पि ‚{\color{DodgerBlue3}‚धीः} स्मृतिरूपा ‚{\tiny $_{lb}$}‚‚{\color{DodgerBlue3}‚विवेकिनी} न भ‚व‚ति । किन्त्वे\edtext{}{\edlabel{pvv.245-2}\label{pvv.245-2}\lemma{किन्त्वे}\Bfootnote{सारूप्याद‚न्य इन्द्रियादिभेदात् ।}}क‚बोधाध्य‚व‚सायिनी प्र‚त्य‚भिज्ञैव स्या‚{\color{DodgerBlue3}‚द्य‚म‚लादि\edtext{}{\edlabel{pvv.245-3}\label{pvv.245-3}\lemma{लादि}\Bfootnote{साम‚ग्रीभेदात् सुखादिभेद‚व‚द्विवेकेन स्मृतिः सेत्स्य‚तीत्य‚पि न । य‚तः स‚दार्था स्प‚ष्टानुभ‚व‚पूर्व्वा साकारा नैवं सुखादिर‚न्तः प्रीत्यादिरूपः ।}}व‚द्य-} म‚ल‚योर‚र्थान्त‚र‚भेद‚स‚द्‏भावेप्येको दृश्य‚मानो नाप‚र‚स्माद् भेदेनाव‚सीय‚ते किन्त्वेक‚त्वेनैव ‚{\tiny $_{lb}$}‚प्र‚त्य‚भिज्ञाय‚ते । (४२५)
	\pend% ending standard par
      \label{div_pvv.2.426}
	  
	% new div opening: depth here is 2
	

	  \pstart \leavevmode% starting standard par
	त‚स्माद‚र्थाकारानुभ‚वाकार‚त‚या बुद्धिर्द्विरूपैव ॥
	\pend% ending standard par
      
	  \bigskip
	  \begingroup
	
	    \large
	  
	    \begin{quote}
	  
	    
	    \stanza[\smallbreak]
	\label{pv.2.426}\flagstanza{\tiny\textenglish{....2.426}}द्वैरूप्य‚साध‚नेनापि प्रायः सिद्धं स्व‚वेद‚न‚म् ।&स्व‚रूप‚भूताभास‚स्य त‚दा संवेद‚नेक्ष‚णात् ॥ ४२६ ॥\&[\smallbreak]


	
	    \end{quote}
	  
	  \endgroup
	

	  \pstart \leavevmode% starting standard par
	\hphantom{.}ज्ञानानां ‚{\color{DodgerBlue3}‚द्वैरूप्य‚साध‚नेनापि \edtext{}{\edlabel{pvv.245-4}\label{pvv.245-4}\lemma{नेनापि}\Bfootnote{य‚थाप‚र‚स्प‚र‚विल‚क्ष‚णेषु रूपादिष्व‚नुभूतेष्व‚न्योन्य‚विवेकेन ।}}प्रायो} बाहुल्येन\edtext{}{\edlabel{pvv.245-5}\label{pvv.245-5}\lemma{बाहुल्येन}\Bfootnote{ग्राहिक‚या }} ‚{\color{DodgerBlue3}‚स्व‚वे\edtext{}{\edlabel{pvv.245-6}\label{pvv.245-6}\lemma{वे}\Bfootnote{त‚च्च वेद्य‚त इत्य‚र्थादात्म‚वेद‚नं न साक्षात्प्रायः श‚ब्दाः ।}}द}‚नं ज्ञानं ‚{\color{DodgerBlue3}‚सिद्धं । त‚था}‚{\tiny $_{3}$}‚ ‚{\tiny $_{lb}$}‚ज्ञान‚स्य ‚{\color{DodgerBlue3}‚स्व‚रूप‚भूत‚स्याभास‚स्या}‚कार‚स्य ‚{\color{DodgerBlue3}‚त‚दा} द्विरूप‚ज्ञानोत्प‚त्तिकाले ‚{\color{DodgerBlue3}‚संवेद‚नाद‚नु}‚{\tiny $_{lb}$}‚भूतेरीक्ष‚णात् (। ४२६)
	\pend% ending standard par
      \label{div_pvv.2.427}
	  
	% new div opening: depth here is 2
	

	  \begin{center}%% label @type='head'
	\textbf{ख. अर्थानुभ‚वाकारा}
	\end{center}
	

	  \pstart \leavevmode% starting standard par
	ज्ञानान्त‚रेण स‚रूपेण ज्ञान‚म‚र्थ‚व‚द्वेद्य‚ते इति चेत् । त‚दा (।)
	\pend% ending standard par
      
	  \bigskip
	  \begingroup
	
	    \large
	  
	    \begin{quote}
	  
	    
	    \stanza[\smallbreak]
	\label{pv.2.427}\flagstanza{\tiny\textenglish{....2.427}}धियाऽत‚द्रूप‚या ज्ञाने निरुद्धेऽनुभ‚वः क‚थ‚म् ॥&स्व‚ञ्च रूपं न सा वेत्तीत्युत्स‚न्नोनुभ‚वोऽखिलः ॥ ४२७ ॥\&[\smallbreak]


	
	    \end{quote}
	  
	  \endgroup
	

	  \pstart \leavevmode% starting standard par
	\hphantom{.}‚{\color{DodgerBlue3}‚धि\edtext{}{\edlabel{pvv.245-7}\label{pvv.245-7}\lemma{धि}\Bfootnote{अनुत्त‚रेण द्वैरूप्ये विष‚य‚सारूप्य‚मात्म‚भूतं ज्ञान‚स्य सिद्धं ।}}याऽत‚द्रूप‚या}‚ऽग्राह्य‚ज्ञान‚स्व‚रूप‚या ‚{\color{DodgerBlue3}‚निरुद्धे} ग्राह्ये ‚{\color{DodgerBlue3}‚ज्ञाने क‚थ‚म‚नुभ‚वः । स्व‚काले}‚{\tiny $_{lb}$}‚\leavevmode\ledsidenote{\textenglish{246/s}} ज्ञानं न वे\edtext{}{\edlabel{pvv.246-1}\label{pvv.246-1}\lemma{वे}\Bfootnote{तुल्य‚काल‚योर्न ग्राह्य‚ग्राह‚क‚त्वं स्व‚स‚म्वेद‚नं नाभ्युपेतं ।}}द्य‚ते ग्राह‚क‚काले ग्राह्य‚स्यैवाभाव इति क‚थं बुद्धिवेद‚नं । स्व‚ञ्च रूपं ‚{\tiny $_{lb}$}‚त्व‚न्म‚ते सा बुद्धिर्न ‚{\color{DodgerBlue3}‚वेत्तीत्य‚नुभ‚वो‚{\tiny $_{4}$}‚ऽखिलो}‚ऽर्थ‚स्य ज्ञान‚स्य चो‚{\color{DodgerBlue3}‚त्स‚न्नः} स्यात् । ज्ञान‚प्र‚काशो ‚{\tiny $_{lb}$}‚ह्य‚र्थ‚प्र‚काशः । स च स्व‚प‚र‚काल‚योर्नास्तीति प्र‚काशो न स्यात् स‚र्व्व‚स्य । (४२७)
	\pend% ending standard par
      \label{div_pvv.2.428}
	  
	% new div opening: depth here is 2
	

	  \pstart \leavevmode% starting standard par
	किञ्च (।)
	\pend% ending standard par
      
	  \bigskip
	  \begingroup
	
	    \large
	  
	    \begin{quote}
	  
	    
	    \stanza[\smallbreak]
	\label{pv.2.428}\flagstanza{\tiny\textenglish{....2.428}}ब‚हिर्मुख‚ञ्च त‚ज्ज्ञानं भात्य‚र्थ‚प्र‚तिभास‚व‚त् ।&बुद्धेश्च ग्राहिका वित्तिर्न्नित्य‚म‚न्त‚र्मुखात्म‚नि ॥ ४२८ ॥\&[\smallbreak]


	
	    \end{quote}
	  
	  \endgroup
	

	  \pstart \leavevmode% starting standard par
	\hphantom{.}अस्यार्थ‚स्य ग्राह्य‚स्य प्र‚तिभास‚व‚दाकार‚व‚त्त‚द् बाह्य‚ग्राह‚कं ज्ञानं ‚{\color{DodgerBlue3}‚ब‚हिर्मुखं} बाह्य‚{\tiny $_{lb}$}‚त‚या प्र‚तिभाति य‚था नीलादिज्ञानं । ‚{\color{DodgerBlue3}‚बुद्धेश्चात्म‚नि ग्राहिका} वित्ति\edtext{}{\edlabel{pvv.246-2}\label{pvv.246-2}\lemma{वित्ति}\Bfootnote{प्र‚त्य‚क्ष‚विरुद्ध‚त्वं स्व‚वेद‚नाभाव‚स्याह एतेन ।}} ‚{\color{DodgerBlue3}‚र्नित्यं} स‚र्व्व‚काल‚{\tiny $_{lb}$}‚‚{\color{DodgerBlue3}‚म‚न्त‚र्मुखा}‚ऽबाह्य‚त‚या ग्राह‚क‚त्वेन प्र‚तिभाति । त‚देत‚त् स्व‚वेद‚न‚ताया‚{\tiny $_{5}$}‚मेवोप‚प‚न्नं । ‚{\tiny $_{lb}$}‚(४२८)-
	\pend% ending standard par
      \label{div_pvv.2.429}
	  
	% new div opening: depth here is 2
	

	  \pstart \leavevmode% starting standard par
	--- य‚दि तु बुद्ध्य‚न्त‚र‚ग्राहिका बुद्धिः स्यात्त‚दा नीलादिव‚त् स्म‚र्य‚माणाऽतीत‚स्व‚{\tiny $_{lb}$}‚बुद्धि\edtext{}{\edlabel{pvv.246-3}\label{pvv.246-3}\lemma{बुद्धि}\Bfootnote{या विष‚य‚ग्राहिका पूर्व्वा ।}}व‚च्च ग्राह‚काकार‚त्वाद् ब‚हिष्ट्वेनाव‚भासेत त‚था स्व‚वेद‚न‚ताऽभावे (।)
	\pend% ending standard par
      
	  \bigskip
	  \begingroup
	
	    \large
	  
	    \begin{quote}
	  
	    
	    \stanza[\smallbreak]
	\label{pv.2.429}\flagstanza{\tiny\textenglish{....2.429}}यो य‚स्य विष‚याभास‚स्तं वेत्ति न त‚दित्य‚पि ।&प्राप्तां का संविद‚न्यास्ति ताद्रूप्यादिति चेन्म‚त‚म् ॥ ४२९ ॥\&[\smallbreak]


	
	    \end{quote}
	  
	  \endgroup
	

	  \pstart \leavevmode% starting standard par
	\hphantom{.}‚{\color{DodgerBlue3}‚यो विष‚य‚स्याभास} आकारो ‚{\color{DodgerBlue3}‚य‚स्य} ज्ञान‚स्य ‚{\color{DodgerBlue3}‚तं} स्वाकारार्प‚कं विष‚यं ‚{\color{DodgerBlue3}‚त‚दा}‚कार‚व‚त् ‚{\tiny $_{lb}$}‚ज्ञानं न ‚{\color{DodgerBlue3}‚वेत्तीति} प्राप्तं विष‚य‚स्व‚रूप‚स्यात्म‚नो वेद‚ने हि विष‚य‚वेद‚नं त‚त्प‚रोक्ष‚त‚या ‚{\tiny $_{lb}$}‚अर्थोपि प‚रोक्षः स्यात् । य‚तोऽर्थ‚स्व‚रूप‚धी‚{\tiny $_{6}$}‚वेद‚नाद‚न्या ‚{\color{DodgerBlue3}‚का संविद‚र्थ}‚स्यास्ति । ‚{\tiny $_{lb}$}‚‚{\color{DodgerBlue3}‚ताद्रूप्याद्वि}‚ष‚य‚सारूप्या\edtext{}{\edlabel{pvv.246-4}\label{pvv.246-4}\lemma{सारूप्या}\Bfootnote{नैयायिकः न स्व‚रूप‚बोधं म‚न्य‚ते ।}}द‚स्व‚संवेद‚नाद‚र्थ‚स्य संविदि‚{\color{DodgerBlue3}‚ति चेन्म‚तं} । (४२९)
	\pend% ending standard par
      \label{div_pvv.2.430}
	  
	% new div opening: depth here is 2
	

	  \pstart \leavevmode% starting standard par
	एवं स‚ति (।)
	\pend% ending standard par
      
	  \bigskip
	  \begingroup
	
	    \large
	  
	    \begin{quote}
	  
	    
	    \stanza[\smallbreak]
	\label{pv.2.430}\flagstanza{\tiny\textenglish{....2.430}}प्राप्तं संवेद‚नं स‚र्व्व‚स‚दृशानां प‚र‚स्प‚र‚म् ।&बुद्धिः स‚रूपा त‚द्विच्चेत् नेदानीं वित् स‚रूपिका ॥ ४३० ॥\&[\smallbreak]


	
	    \end{quote}
	  
	  \endgroup
	

	  \pstart \leavevmode% starting standard par
	\hphantom{.}‚{\color{DodgerBlue3}‚स‚र्व्वे}‚षां य‚म‚ल‚कादीनां ‚{\color{DodgerBlue3}‚स‚दृशानां प‚र‚स्प‚रं स‚म्वेद‚नं प्राप्तं} न स‚दृश इत्येवानुभ‚वः ‚{\tiny $_{lb}$}‚‚{\color{DodgerBlue3}‚स‚रूपा त‚द्विद}‚र्थ‚स्य संवेद‚न‚{\color{DodgerBlue3}‚ञ्चेत्} । ‚{\color{DodgerBlue3}‚इदानीम}‚स्मिन्न‚भ्युप‚ग‚मे ‚{\color{DodgerBlue3}‚स‚रूपेका} विन्न भ‚व‚ति । ‚{\tiny $_{lb}$}‚सारूप्यं वेद‚न‚ल‚क्ष‚णं न भ‚व‚ति । किन्त्व‚नुभ‚व‚रूप‚ता । स‚त्य‚पि सारूप्ये य‚म‚ल‚{\tiny $_{lb}$}‚कादी‚{\tiny $_{7}$}‚नाम‚नुभ‚व‚त्वात् । नापि सारूप्य‚वाहितानुभ‚व‚मात्रं वेद‚नं, किन्तु स्व‚संवेद‚नं ‚{\tiny $_{lb}$}‚\leavevmode\ledsidenote{\textenglish{48b/MA}} सारूप्यं । (४३०)
	\pend% ending standard par
      \label{div_pvv.2.431}
	  
	% new div opening: depth here is 2
	\textsuperscript{\textenglish{247/s}}

	  \pstart \leavevmode% starting standard par
	य‚दि सारूप्य‚व‚शाद्वेद‚नं त‚दा बुद्ध्यात्म‚नापि सारूप्याद्वेद‚नं स्यात् । त‚था ग्राह्य‚{\tiny $_{lb}$}‚ग्राह‚क‚योर्भेद एव प्राप्त इत्याह (।)
	\pend% ending standard par
      
	  \bigskip
	  \begingroup
	
	    \large
	  
	    \begin{quote}
	  
	    
	    \stanza[\smallbreak]
	\label{pv.2.431a}\flagstanza{\tiny\textenglish{...2.431a}}स्व‚यं सोनुभ‚व‚स्त‚स्या न स सारूप्य‚कार‚णः ।\&[\smallbreak]


	
	    \end{quote}
	  
	  \endgroup
	

	  \pstart \leavevmode% starting standard par
	\hphantom{.}‚{\color{DodgerBlue3}‚त‚स्या} बुद्धेः ‚{\color{DodgerBlue3}‚सोऽनुभ‚वो}‚ऽप‚रोक्ष‚त्वं स्व‚यं स्व‚रूपेण त‚थोत्प‚त्तेर्न ‚{\color{DodgerBlue3}‚सारूप्य\edtext{}{\edlabel{pvv.247-1}\label{pvv.247-1}\lemma{सारूप्य}\Bfootnote{सारूप्य‚कृतः ।}}कार‚णः} सोऽनुभ‚वो बुद्धेः । एवं त‚र्हि बाह्येप्य‚र्थे बुद्धिसारूप्यं निष्फ‚ल‚मित्याह (।)
	\pend% ending standard par
      
	  \bigskip
	  \begingroup
	
	    \large
	  
	    \begin{quote}
	  
	    
	    \stanza[\smallbreak]
	\label{pv.2.431b}\flagstanza{\tiny\textenglish{...2.431b}}क्रियाक‚र्म्म‚व्य‚व‚स्थायास्त‚ल्लोके स्यान्निब‚न्ध‚न‚म् ॥ ४३१ ॥\&[\smallbreak]


	
	    \end{quote}
	  
	  \endgroup
	

	  \pstart \leavevmode% starting standard par
	\hphantom{.}‚{\color{DodgerBlue3}‚त‚द‚र्थ}‚सारूप्यं ‚{\color{DodgerBlue3}‚क्रियाया} अर्थाद‚नुभूतेः । ‚{\color{DodgerBlue3}‚क‚र्म‚णो} बाह्य‚स्य ‚{\color{DodgerBlue3}‚व्य‚व‚स्थाया निब‚न्ध‚नं ‚{\tiny $_{lb}$}‚लोके} ब‚हिर‚ध्य‚व‚सायिनि स्यात् । (४३१)
	\pend% ending standard par
      \label{div_pvv.2.432}
	  
	% new div opening: depth here is 2
	

	  \pstart \leavevmode% starting standard par
	न ह्य‚स्य सारूप्य‚म‚न्त‚रेणेंय‚म‚स्य संवित्तिरिति श‚क्यं व्य‚व‚स्थाप‚यितुं\edtext{}{\edlabel{pvv.247-2}\label{pvv.247-2}\lemma{यितुं}\Bfootnote{य‚दि स्वानुभ‚वात्म‚त‚यैव प्र‚काशो नार्थानुभ‚वात्म‚त‚या त‚दा स‚म्ब‚न्धाभावाद‚र्थानुभ‚व‚व्य‚प‚देशो न युक्त इत्याह(।)}} (॥)
	\pend% ending standard par
      
	  \bigskip
	  \begingroup
	
	    \large
	  
	    \begin{quote}
	  
	    
	    \stanza[\smallbreak]
	\label{pv.2.432}\flagstanza{\tiny\textenglish{....2.432}}स्व‚भाव‚भूत‚त‚द्रूप‚संविदारोप‚विप्ल‚वात् ।&नीलादेर‚नुभूताख्या नानुभूतेः प‚रात्म‚नः ॥ ४३२ ॥\&[\smallbreak]


	
	    \end{quote}
	  
	  \endgroup
	

	  \pstart \leavevmode% starting standard par
	\hphantom{.}य‚स्या बुद्धेः ‚{\color{DodgerBlue3}‚स्व‚भाव‚भूत‚स्य} रूप‚स्य विष‚याकार‚स्य ‚{\color{DodgerBlue3}‚संविदो} ब‚हिर‚र्थेष्वा\edtext{}{\edlabel{pvv.247-3}\label{pvv.247-3}\lemma{र्थेष्वा}\Bfootnote{अस‚तोपि ।}} ‚{\color{DodgerBlue3}‚रोपः} स ‚{\tiny $_{lb}$}‚एव ‚{\color{DodgerBlue3}‚विप्ल‚वो} भ्रान्त्युप‚नीत‚त्वात् त‚स्मा‚{\color{DodgerBlue3}‚न्नीलादे}‚र्व्व‚स्तुतोऽदृश्य‚मान‚स्याप्य‚{\color{DodgerBlue3}‚नुभूताख्या}‚{\tiny $_{lb}$}‚ऽनुभ‚व‚व्य‚व‚हारो लोक‚स्य ‚{\color{DodgerBlue3}‚न} पुन‚{\tiny $_{2}$}‚र्ज्ञाना‚{\color{DodgerBlue3}‚त्प‚रात्म‚नो}‚र्थ‚स्य साक्षाद‚{\color{DodgerBlue3}‚नुभूते}‚र‚र्थानुभ‚व‚{\tiny $_{lb}$}‚व्य‚व‚हारः । (४३२)
	\pend% ending standard par
      \label{div_pvv.2.433}
	  
	% new div opening: depth here is 2
	
	  \bigskip
	  \begingroup
	
	    \large
	  
	    \begin{quote}
	  
	    
	    \stanza[\smallbreak]
	\label{pv.2.433a}\flagstanza{\tiny\textenglish{...2.433a}}धियो नीलादिरूप‚त्वे बाह्योर्थः किं प्र‚माण‚कः ।\&[\smallbreak]


	
	    \end{quote}
	  
	  \endgroup
	

	  \pstart \leavevmode% starting standard par
	\hphantom{.}प‚र‚मार्थ‚त‚स्तु ‚{\color{DodgerBlue3}‚धियो} नीला‚{\color{DodgerBlue3}‚दिरूप‚त्वे} स्व‚स‚म्वेद्ये त‚दाकारार्प्प‚को ‚{\color{DodgerBlue3}‚बाह्योऽर्थः} स्व‚रू‚{\tiny $_{lb}$}‚पेणादृश्य‚मानः ‚{\color{DodgerBlue3}‚किं प्र‚माण‚कः} । न ह्याकार‚द्व‚यं वेद्य‚ते येनैको बाह्य‚स्याप‚रो ज्ञान‚स्येति ‚{\tiny $_{lb}$}‚स्यात् ।
	\pend% ending standard par
      

	  \pstart \leavevmode% starting standard par
	बाह्य एवाकार‚वान् धीस्तु निराकारेति प्र‚त्य‚क्ष‚सिद्धोऽर्थः स्यादित्याह (।)
	\pend% ending standard par
      
	  \bigskip
	  \begingroup
	
	    \large
	  
	    \begin{quote}
	  
	    
	    \stanza[\smallbreak]
	\label{pv.2.433b}\flagstanza{\tiny\textenglish{...2.433b}}धियोऽनीलादिरूप‚त्वे स त‚स्यानुभ‚वः क‚थ‚म् ॥ ४३३ ॥\&[\smallbreak]


	
	    \end{quote}
	  
	  \endgroup
	

	  \pstart \leavevmode% starting standard par
	\hphantom{.}‚{\color{DodgerBlue3}‚धियोऽनीलादिरूप‚त्वे सो}‚ऽर्थाकार‚र‚हितो‚{\color{DodgerBlue3}‚नु‚{\tiny $_{3}$}‚भ‚व‚स्त‚स्य} नील‚स्य ग्राह‚क इति ‚{\tiny $_{lb}$}‚‚{\color{DodgerBlue3}‚क‚थं} श‚क्य‚व्य‚व‚स्थाप‚नः । (४३३)
	\pend% ending standard par
      \label{div_pvv.2.434}
	  
	% new div opening: depth here is 2
	
	  \bigskip
	  \begingroup
	
	    \large
	  
	    \begin{quote}
	  
	    
	    \stanza[\smallbreak]
	\label{pv.2.434}\flagstanza{\tiny\textenglish{....2.434}}य‚दा संवेद‚नात्म‚त्वं न सारूप्य‚निब‚न्ध‚न‚म् ।&सिद्धं त‚त् स्व‚त एवास्य किम‚र्थेनोप‚नीय‚ते ॥ ४३४ ॥\&[\smallbreak]


	
	    \end{quote}
	  
	  \endgroup
	

	  \pstart \leavevmode% starting standard par
	अनुभ‚व‚मात्रात्म‚त‚या स‚र्व्व‚त्र ज्ञानेऽविशेषात् विेशेष‚व्य‚व‚स्थान‚श‚क्त‚या ।
	\pend% ending standard par
      \textsuperscript{\textenglish{248/s}}

	  \pstart \leavevmode% starting standard par
	\hphantom{.}‚{\color{DodgerBlue3}‚य‚दा स‚म्वेद‚नात्म‚क‚त्व}‚म‚प‚रोक्ष‚त्वं ‚{\color{DodgerBlue3}‚न सारूप्य‚निब‚न्ध‚नं त‚दा स्व‚त एव} प्र‚काशात्म‚{\tiny $_{lb}$}‚त‚योत्प‚त्तेस्त‚त्संवेद‚नात्म‚त्वं ‚{\color{DodgerBlue3}‚सिद्धं} । त‚त‚श्चार्थेनास्य ज्ञान‚स्य ‚{\color{DodgerBlue3}‚किमुन्नीय‚ते} येन त‚स्य ‚{\tiny $_{lb}$}‚त‚द्वेद‚न‚मित्युच्य‚ते(।) न हि ज्ञान‚स्य स्व‚प्र‚काशेऽर्थापेक्षा । न च स्व‚स्मा‚{\tiny $_{4}$}‚द् व्य‚तिरिक्तं ‚{\tiny $_{lb}$}‚तेन वेद्य‚ते त‚त्क‚थ‚म‚र्थ‚वेद‚न‚म‚नेनेत्युच्य‚ते (। ४३४)
	\pend% ending standard par
      \label{div_pvv.2.435}
	  
	% new div opening: depth here is 2
	

	  \pstart \leavevmode% starting standard par
	किञ्च (।)
	\pend% ending standard par
      
	  \bigskip
	  \begingroup
	
	    \large
	  
	    \begin{quote}
	  
	    
	    \stanza[\smallbreak]
	\label{pv.2.435}\flagstanza{\tiny\textenglish{....2.435}}न‚च स‚र्वात्म‚ना साम्य‚म‚ज्ञान‚त्व‚प्र‚स‚ङ्ग‚तः ।&न च केन‚चिदंशेन स‚र्वं स‚र्व‚स्य वेद‚न‚म् ॥ ४३५ ॥\&[\smallbreak]


	
	    \end{quote}
	  
	  \endgroup
	

	  \pstart \leavevmode% starting standard par
	\hphantom{.}ज्ञानं ‚{\color{DodgerBlue3}‚स‚र्व्वात्म‚ना} वा एक‚देशेन वाऽर्थ‚स्य स‚रूपं य‚त्त‚द्‏ग्राह‚कं स्यात् । त‚त्र न ‚{\tiny $_{lb}$}‚ताव‚त्स‚र्व्वेण ज‚ड‚त्वादिना ‚{\color{DodgerBlue3}‚साम्य‚म‚ज्ञान‚त्व‚प्र‚स‚ङ्ग‚तः । न च} ज‚ड‚योर्ग्राह्य‚ग्राह‚क‚भावः ‚{\tiny $_{lb}$}‚‚{\color{DodgerBlue3}‚केन‚चिदंशेन} व‚स्तुत्व‚नील‚त्वादीना ‚{\color{DodgerBlue3}‚स‚र्व्वं} ज्ञानं स‚र्व्व‚स्यार्थ‚स्य संवेद‚नं स्यात् । ‚{\color{DodgerBlue3}‚स‚र्व्वं} वा नील‚ज्ञानं स‚र्व्व‚स्य\edtext{}{\edlabel{pvv.248-1}\label{pvv.248-1}\lemma{स्य}\Bfootnote{भिन्नाकार‚स्य न वेद‚न‚मिति श‚ङ्का स्यात् नीलाकार‚स्तु स‚र्व्व‚नीलानुकारीति स‚र्व्व‚ग्र‚हः ।}} नील‚स्य ‚{\color{DodgerBlue3}‚वेद‚नं स्यात्} । (४३५)
	\pend% ending standard par
      \label{div_pvv.2.436}
	  
	% new div opening: depth here is 2
	
	  \bigskip
	  \begingroup
	
	    \large
	  
	    \begin{quote}
	  
	    
	    \stanza[\smallbreak]
	\label{pv.2.436}\flagstanza{\tiny\textenglish{....2.436}}य‚था नीलादिरूप‚त्वान्नीलाद्य‚नुभ‚वो म‚तः ।&त‚थानुभ‚व‚रूप‚त्वात्त‚स्याप्य‚नुभ‚वो भ‚वेत् ॥ ४३६ ॥\&[\smallbreak]


	
	    \end{quote}
	  
	  \endgroup
	

	  \pstart \leavevmode% starting standard par
	\hphantom{.}‚{\color{DodgerBlue3}‚य‚था नीलादि‚{\tiny $_{5}$}‚रूप‚त्वात्} ज्ञानं ‚{\color{DodgerBlue3}‚नीलादी}‚नाम‚{\color{DodgerBlue3}‚नुभ‚वो म‚तः} त‚था किम‚नुभ‚व‚{\tiny $_{lb}$}‚‚{\color{DodgerBlue3}‚रूप‚त्वात्} त‚स्यानुभ‚व‚स्यार्थ‚विष‚य‚स्यापि पूर्व्व‚क‚स्योत्त‚रं ज्ञान‚म‚{\color{DodgerBlue3}‚नुभ‚वो न\edtext{}{\edlabel{pvv.248-2}\label{pvv.248-2}\lemma{न}\Bfootnote{ताद्रूप्य‚म‚भिव्यापित्वाद्विषाणित्व‚मिव गौर्नानुभ‚व‚ल‚क्ष‚णः ।}} भ‚वेत्} । ‚{\tiny $_{lb}$}‚(४३६)
	\pend% ending standard par
      \label{div_pvv.2.437}
	  
	% new div opening: depth here is 2
	
	  \bigskip
	  \begingroup
	
	    \large
	  
	    \begin{quote}
	  
	    
	    \stanza[\smallbreak]
	\label{pv.2.437}\flagstanza{\tiny\textenglish{....2.437}}नानुभूतोनुभ‚व इत्य‚र्थ‚व‚द् (हि) विनिश्च‚यः ।&त‚स्माद‚दोष इति चेत् नार्थेप्य‚स्त्येष स‚र्व‚दा ॥ ४३७ ॥\&[\smallbreak]


	
	    \end{quote}
	  
	  \endgroup
	

	  \pstart \leavevmode% starting standard par
	\hphantom{.}‚{\color{DodgerBlue3}‚ना}\edtext{\textsuperscript{*}}{\edlabel{pvv.248-3}\label{pvv.248-3}\lemma{*}\Bfootnote{न चेष्य‚ते ।}}नुभ‚वेऽ‚{\color{DodgerBlue3}‚नुभूतोऽनुभ‚व इत्य‚र्थ‚व‚द‚र्थ} इव गृहीते ‚{\color{DodgerBlue3}‚विनिश्च‚यो} भ‚व‚ति । ‚{\color{DodgerBlue3}‚त‚स्मा}‚द‚नु‚{\tiny $_{lb}$}‚भ‚व‚स्याप्य‚नुभ‚वो ग्राह्यः कः स्यादित्य‚य‚{\color{DodgerBlue3}‚म‚दोष इति चेत्} ।\edtext{\textsuperscript{*}}{\edlabel{pvv.248-4}\label{pvv.248-4}\lemma{*}\Bfootnote{त‚दुत्प‚त्तिसारूप्य‚योः स‚तोर‚पीदं म‚यानुभूत‚मिति निश्च‚योऽर्थानुभ‚वः । नैव‚म‚नुभ‚व‚विष‚योऽनुभ‚वः स्मृतिरेव तु स्यात् ।}} न केव‚ल‚म‚नुभ‚वे‚{\tiny $_{lb}$}‚‚{\color{DodgerBlue3}‚ऽर्थेऽप्येषो}\edtext{}{\edlabel{pvv.248-5}\label{pvv.248-5}\lemma{वे}\Bfootnote{प‚राम‚र्श‚योगी ।}}नुभूत‚त्व‚निश्च‚यः ‚{\color{DodgerBlue3}‚स‚र्व्व‚दा ना}‚स्ति । न हि धारावाहिन्य‚र्थ‚ज्ञानेऽर्थेषु ‚{\tiny $_{lb}$}‚प्र‚तिक्ष‚ण‚म‚नुभूत‚नि‚{\tiny $_{6}$}‚श्च‚यः\edtext{}{\edlabel{pvv.248-6}\label{pvv.248-6}\lemma{यः}\Bfootnote{प‚श्य‚त एवार्थं विष‚यान्त‚र‚व्याक्षेपान्नानुभूत‚निश्च‚यः ।}}। त‚त‚श्चार्थोपि नानुभूतः स्यात् । (४३७)
	\pend% ending standard par
      \label{div_pvv.2.438_2.439}
	  
	% new div opening: depth here is 2
	
	  \bigskip
	  \begingroup
	
	    \large
	  
	    \begin{quote}
	  
	    
	    \stanza[\smallbreak]
	\label{pv.2.438}\flagstanza{\tiny\textenglish{....2.438}}क‚स्माद्वाऽनुभ‚वे नास्ति स‚ति स‚त्तानिब‚न्ध‚ने ।&अपि चेदं य‚दाभाति दृश्य‚माने सितादिके ॥ ४३८ ॥\&[\smallbreak]


	
	    \end{quote}
	  
	  \endgroup
	\textsuperscript{\textenglish{249/s}}
	  \bigskip
	  \begingroup
	
	    \large
	  
	    \begin{quote}
	  
	    
	    \stanza[\smallbreak]
	\label{pv.2.439}\flagstanza{\tiny\textenglish{....2.439}}पुंसः सिताद्य‚भिव्य‚क्तिरूपं संवेद‚नं स्फुट‚म् ।&त‚त्किं सिताद्य‚भिव्य‚क्तेः प‚र‚रूप‚म‚थात्म‚नः ॥ ४३९ ॥\&[\smallbreak]


	
	    \end{quote}
	  
	  \endgroup
	

	  \pstart \leavevmode% starting standard par
	\hphantom{.}‚{\color{DodgerBlue3}‚क‚स्माद्वाऽनुभ‚वे}‚ऽनुभूत‚निश्च‚य‚स्य ‚{\color{DodgerBlue3}‚स‚त्तानिब‚न्ध‚ने} सारूप्ये त‚दुत्पादे च ‚{\color{DodgerBlue3}‚स‚ति ‚{\tiny $_{lb}$}‚नास्ति} स‚त्तानिश्च‚यः । अर्थेप्य‚नुभूत‚निब‚न्ध‚ने सारूप्य‚त‚दुत्प‚त्ती एव ते त‚ज्ज्ञानेपि ‚{\tiny $_{lb}$}‚स‚माने । ‚{\color{DodgerBlue3}‚अपि च} स्व‚संवेद‚नान‚भ्युप‚ग‚मे ‚{\color{DodgerBlue3}‚सितादिके \edtext{}{\edlabel{pvv.249-1}\label{pvv.249-1}\lemma{सितादिके}\Bfootnote{विच्छिन्ने ।}}दृश्य‚माने, य‚दिदं\edtext{}{\edlabel{pvv.249-2}\label{pvv.249-2}\lemma{दिदं}\Bfootnote{त‚दैवावेद‚नात् ।}}} (४३८) ‚{\tiny $_{lb}$}‚‚{\color{DodgerBlue3}‚सिताद्य‚भिव्य‚क्तिरूप‚म‚न्तः} प्र‚काश‚मानं ‚{\color{DodgerBlue3}‚संवेद‚नं स्फुटं पुंसः} प्र‚तिप‚त्तु‚{\color{DodgerBlue3}‚राभाति । ‚{\tiny $_{lb}$}‚त‚त् किं सिताद्य‚भिव्य‚क्तेः} प‚र‚रूप‚{\tiny $_{7}$}‚‚{\color{DodgerBlue3}‚म‚थात्म}‚भूत‚मिति विक‚ल्पौ, स‚त‚स्त‚त्त्वान्य‚त्वा‚{\tiny $_{lb}$}‚व्य‚तिक्र‚मात् । (४३९)\leavevmode\ledsidenote{\textenglish{49a/MA}}
	\pend% ending standard par
      \label{div_pvv.2.440}
	  
	% new div opening: depth here is 2
	
	  \bigskip
	  \begingroup
	
	    \large
	  
	    \begin{quote}
	  
	    
	    \stanza[\smallbreak]
	\label{pv.2.440}\flagstanza{\tiny\textenglish{....2.440}}प‚र‚रूपेऽप्र‚काशायां व्य‚क्तौ व्य‚क्तं क‚थं सित‚म् ।&ज्ञानं व्य‚क्तिर्न सा व्य‚क्तेत्य‚व्य‚क्त‚म‚खिलं ज‚ग‚त् ॥ ४४० ॥\&[\smallbreak]


	
	    \end{quote}
	  
	  \endgroup
	

	  \pstart \leavevmode% starting standard par
	\hphantom{.}‚{\color{DodgerBlue3}‚प‚र‚रूपे}‚ऽभ्युप‚ग‚म्य‚माने स‚म्वेद‚नं य‚त्प्र‚काश‚ते त‚न्न सितादिव्य‚क्तिरूप‚मित्य‚{\color{DodgerBlue3}‚प्र‚का-} शायां शुक्लादि‚{\color{DodgerBlue3}‚व्य‚क्तौ सितं क‚थं व्य‚क्तं । ज्ञानं} हि ‚{\color{DodgerBlue3}‚व्य‚क्तिर्न च सा व्य‚क्ता} सितादिके दृश्य‚माने इष्य‚त ‚{\color{DodgerBlue3}‚इत्य‚व्य‚क्त‚म‚खिलं ज‚ग}‚त्प्राप्तं । सिताद्य‚भिव्य‚क्ति‚{\tiny $_{lb}$}‚रेषित‚व्या । (४४०)
	\pend% ending standard par
      \label{div_pvv.2.441}
	  
	% new div opening: depth here is 2
	

	  \pstart \leavevmode% starting standard par
	त‚था हि (।)
	\pend% ending standard par
      
	  \bigskip
	  \begingroup
	
	    \large
	  
	    \begin{quote}
	  
	    
	    \stanza[\smallbreak]
	\label{pv.2.441a}\flagstanza{\tiny\textenglish{...2.441a}}व्य‚क्तेर्व्य‚क्त्य‚न्त‚र‚व्य‚क्ताव‚पि दोष‚प्र‚स‚ङ्ग‚तः ।\&[\smallbreak]


	
	    \end{quote}
	  
	  \endgroup
	

	  \pstart \leavevmode% starting standard par
	\hphantom{.}अर्थ‚{\color{DodgerBlue3}‚व्य‚क्तेर्व्य‚क्तिरा}‚प‚द्य‚माना न व्य‚क्ता स्यात् । अथ व्य‚क्तिरेव न सिध्येत् । ‚{\tiny $_{lb}$}‚त‚स्याः स्व‚प्र‚काश‚त्वेऽर्थ‚व्य‚क्तिर‚पि त‚थास्तु । अथार्थ‚व्य‚क्तिर्व्य‚क्तेर्व्य‚क्त्य‚न्त‚राद् ‚{\tiny $_{lb}$}‚उत्त‚र‚काल‚भाविवेद‚नाद् व्य‚क्तिरेवं त‚स्याश्चा\edtext{}{\edlabel{pvv.249-3}\label{pvv.249-3}\lemma{स्याश्चा}\Bfootnote{विष‚य‚ज्ञान‚स्यापि स्व‚य‚म‚व्य‚क्तेः ।}}न्य‚त इत्य‚न‚व‚स्था दुर्व्वारा ।
	\pend% ending standard par
      

	  \pstart \leavevmode% starting standard par
	किञ्च (।)
	\pend% ending standard par
      
	  \bigskip
	  \begingroup
	
	    \large
	  
	    \begin{quote}
	  
	    
	    \stanza[\smallbreak]
	\label{pv.2.441b}\flagstanza{\tiny\textenglish{...2.441b}}दृष्ट्या वाज्ञात‚स‚म्ब‚न्धं विशिन‚ष्टि त‚या क‚थं ॥ ४४१ ॥\&[\smallbreak]


	
	    \end{quote}
	  
	  \endgroup
	

	  \pstart \leavevmode% starting standard par
	\hphantom{.}‚{\color{DodgerBlue3}‚दृष्ट्या} धि याऽ\edtext{}{\edlabel{pvv.249-4}\label{pvv.249-4}\lemma{याऽ}\Bfootnote{त‚त्स‚म‚कालं, आह}}स‚म्विदित‚या स‚हा‚{\color{DodgerBlue3}‚ज्ञात‚स‚म्ब‚न्ध‚म‚र्थं क‚थं विशिन‚ष्टि} प्र‚तिप‚त्ता ‚{\tiny $_{lb}$}‚दृष्टोय‚मिति । (४४१)
	\pend% ending standard par
      \label{div_pvv.2.442}
	  
	% new div opening: depth here is 2
	

	  \pstart \leavevmode% starting standard par
	क‚थ‚म‚र्थो दृष्ट्याऽज्ञात‚स‚म्ब‚न्ध इत्याह (।)
	\pend% ending standard par
      
	  \bigskip
	  \begingroup
	
	    \large
	  
	    \begin{quote}
	  
	    
	    \stanza[\smallbreak]
	\label{pv.2.442}\flagstanza{\tiny\textenglish{....2.442}}य‚स्माद् द्व‚योरेक‚ग‚तौ न द्वितीय‚स्य द‚र्श‚न‚म् ।&द्व‚योः संसृष्ट‚योर्दृष्टौ स्याद् दृष्ट‚मिति निश्च‚यः ॥ ४४२ ॥\&[\smallbreak]


	
	    \end{quote}
	  
	  \endgroup
	

	  \pstart \leavevmode% starting standard par
	\hphantom{.}‚{\color{DodgerBlue3}‚य‚स्माद्} द्व‚योर‚र्थ‚ज्ञान‚योर्म‚ध्ये ‚{\color{DodgerBlue3}‚एक‚स्य ग‚तौ} द‚र्श‚न‚{\tiny $_{1}$}‚काले ‚{\color{DodgerBlue3}‚न द्वितीय‚स्य द‚र्श‚न‚म‚स्ति} । ‚{\tiny $_{lb}$}‚त‚था हि न प‚दार्थो दृश्य‚ते न त‚दा बुद्धिरुप‚ल‚भ्य‚ते त‚दुप‚ल‚म्भ‚स्य भावित्वात् ।
	\pend% ending standard par
      \textsuperscript{\textenglish{250/s}}

	  \pstart \leavevmode% starting standard par
	\hphantom{.}य‚दा च बुद्धिरुप‚ल‚भ्य‚ते न त‚दाऽन्योऽतीत‚त्वात् । त‚स्माद् ‚{\color{DodgerBlue3}‚द्व‚यो}‚र‚र्थ‚ज्ञान‚योः ‚{\tiny $_{lb}$}‚‚{\color{DodgerBlue3}‚संसृष्ट‚योरे}‚कोप‚ल‚म्भात् ‚{\color{DodgerBlue3}‚दृष्टौ} स‚त्यां ‚{\color{DodgerBlue3}‚दृष्ट‚मिद‚मिति निश्च‚यः} त‚तोऽन्योप‚ल‚ब्धिः ‚{\tiny $_{lb}$}‚स्वोप‚ल‚ब्धिरूपेव ‚{\color{DodgerBlue3}‚स्यादे}‚त‚त् । (४४२)
	\pend% ending standard par
      \label{div_pvv.2.443}
	  
	% new div opening: depth here is 2
	
	  \bigskip
	  \begingroup
	
	    \large
	  
	    \begin{quote}
	  
	    
	    \stanza[\smallbreak]
	\label{pv.2.443a}\flagstanza{\tiny\textenglish{...2.443a}}स‚रूपं द‚र्श‚नं य‚स्य दृश्य‚तेन्येन चेत‚सा ।&दृष्टाख्येति न चेत्;\&[\smallbreak]


	
	    \end{quote}
	  
	  \endgroup
	

	  \pstart \leavevmode% starting standard par
	\hphantom{.}‚{\color{DodgerBlue3}‚य‚स्यार्थ‚स्य स‚रूपं} स‚मानाकारं ‚{\color{DodgerBlue3}‚द‚र्श‚नं} ज्ञान‚{\color{DodgerBlue3}‚म‚न्येन चेत‚सा दृश्य‚ते} त‚त्रार्थे ‚{\tiny $_{lb}$}‚‚{\color{DodgerBlue3}‚दृष्टाख्या} दृष्ट‚व्य‚व‚हा‚{\tiny $_{2}$}‚र ‚{\color{DodgerBlue3}‚इति चेत्} ।
	\pend% ending standard par
      

	  \pstart \leavevmode% starting standard par
	अत्राह (।)
	\pend% ending standard par
      
	  \bigskip
	  \begingroup
	
	    \large
	  
	    \begin{quote}
	  
	    
	    \stanza[\smallbreak]
	\label{pv.2.443b}\flagstanza{\tiny\textenglish{...2.443b}}सिद्धं सारूप्येऽस्य स्व‚वेद‚न‚म् ॥ ४४३ ॥\&[\smallbreak]


	
	    \end{quote}
	  
	  \endgroup
	

	  \pstart \leavevmode% starting standard par
	\hphantom{.}अस्य ज्ञान‚स्यार्थेन स‚ह ‚{\color{DodgerBlue3}‚सारूप्ये} सिद्धे ‚{\color{DodgerBlue3}‚स्व‚संवेद‚नं} ज्ञानं सिद्धं । त‚था ह्य‚र्था‚{\tiny $_{lb}$}‚कार‚स्ताव‚त् ज्ञान‚काले प‚रिस्फुटं वेद्य‚मानो ज्ञान‚स्यात्मा चेत् ज्ञान‚म‚प्य‚र्थाकार‚{\tiny $_{lb}$}‚व‚द‚प‚रोक्ष‚मेव स्व‚भाव‚त इति नान्य‚वेद्यं । (४४३)
	\pend% ending standard par
      \label{div_pvv.2.444}
	  
	% new div opening: depth here is 2
	
	  \bigskip
	  \begingroup
	
	    \large
	  
	    \begin{quote}
	  
	    
	    \stanza[\smallbreak]
	\label{pv.2.444a}\flagstanza{\tiny\textenglish{...2.444a}}अथात्म‚रूपं नो वेत्ति प‚र‚रूप‚स्य वित् क‚थ‚म् ।\&[\smallbreak]


	
	    \end{quote}
	  
	  \endgroup
	

	  \pstart \leavevmode% starting standard par
	\hphantom{.}‚{\color{DodgerBlue3}‚अथ} ज्ञान‚मात्म‚रूपं ‚{\color{DodgerBlue3}‚न वेत्ति प‚र‚रूप‚स्य} बाह्य‚रूप‚स्य ‚{\color{DodgerBlue3}‚वित्\edtext{}{\edlabel{pvv.250-1}\label{pvv.250-1}\lemma{वित्}\Bfootnote{वेद‚क‚म‚प्र‚त्य‚क्षोप‚ल‚म्भात् ।}} क‚थं} । न ह्य‚र्थाकार‚{\tiny $_{lb}$}‚ज्ञान‚वेद‚न‚म‚न्त‚रेणार्थ‚वेद‚न‚मित्युक्तं ।
	\pend% ending standard par
      

	  \pstart \leavevmode% starting standard par
	सारूप्य‚मात्रेणार्थ‚वित्तिर्भ‚विष्य‚तीति चेत् (।)
	\pend% ending standard par
      
	  \bigskip
	  \begingroup
	
	    \large
	  
	    \begin{quote}
	  
	    
	    \stanza[\smallbreak]
	\label{pv.2.444b}\flagstanza{\tiny\textenglish{...2.444b}}सारूप्याद् वेद‚नाख्या च प्रागेव प्र‚तिव‚र्ण्णिता ॥ ४४४ ॥\&[\smallbreak]


	
	    \end{quote}
	  
	  \endgroup
	

	  \pstart \leavevmode% starting standard par
	\hphantom{.}‚{\color{DodgerBlue3}‚सारूप्याद‚नु}‚भ‚वात्म‚तार‚हिता‚{\color{DodgerBlue3}‚द्वेद‚नाख्या} वेद‚न‚व्य‚व‚हृतिश्च ‚{\color{DodgerBlue3}‚प्रागेव} \cref{pv.2.430} ‚{\tiny $_{lb}$}‚प्राप्तं संवेद‚नं स‚र्व‚स‚दृशानां प‚र‚स्प‚र‚भित्य‚नेन प्र‚तिव‚र्ण्णिता ‚{\color{DodgerBlue3}‚प्र‚त्युक्ता} । (४४४)
	\pend% ending standard par
      \label{div_pvv.2.445}
	  
	% new div opening: depth here is 2
	

	  \pstart \leavevmode% starting standard par
	किञ्च\edtext{}{\edlabel{pvv.250-2}\label{pvv.250-2}\lemma{किञ्च}\Bfootnote{सारूप्य‚मेव नास्तीत्याह(।)}} (।)
	\pend% ending standard par
      
	  \bigskip
	  \begingroup
	
	    \large
	  
	    \begin{quote}
	  
	    
	    \stanza[\smallbreak]
	\label{pv.2.445}\flagstanza{\tiny\textenglish{....2.445}}दृष्ट‚योरेव सारूप्य‚ग्र‚होर्थ‚ञ्च न दृष्ट‚वान् ।&प्राक् क‚थं द‚र्श‚नेनास्य सारूप्यं सोध्य‚व‚स्य‚ति ॥ ४४५ ॥\&[\smallbreak]


	
	    \end{quote}
	  
	  \endgroup
	

	  \pstart \leavevmode% starting standard par
	\hphantom{.}‚{\color{DodgerBlue3}‚दृष्ट‚योरेव} क‚योश्चित् ‚{\color{DodgerBlue3}‚सारूप्य‚ग्र‚हो} दृष्टो य‚था य‚म‚ल‚क‚योः । न च क‚श्चिद् ‚{\tiny $_{lb}$}‚द्र‚ष्टा ज्ञानात्\edtext{}{\edlabel{pvv.250-3}\label{pvv.250-3}\lemma{ज्ञानात्}\Bfootnote{प्र‚थ‚म‚ज्ञाने ।}} प्राग‚र्थं दृष्ट‚वान् । त‚त्क‚थ‚न्द‚र्श‚नेन स‚हास्यादृष्ट‚स्यार्थ‚स्य स द्र‚ष्टा ‚{\tiny $_{lb}$}‚‚{\color{DodgerBlue3}‚सारूप्य‚म‚ध्य‚व‚स्य‚ति} निश्चिनोति । (४४५)
	\pend% ending standard par
      \label{div_pvv.2.446}
	  
	% new div opening: depth here is 2
	

	  \pstart \leavevmode% starting standard par
	किञ्च (।)
	\pend% ending standard par
      \textsuperscript{\textenglish{251/s}}
	  \bigskip
	  \begingroup
	
	    \large
	  
	    \begin{quote}
	  
	    
	    \stanza[\smallbreak]
	\label{pv.2.446}\flagstanza{\tiny\textenglish{....2.446}}सारूप्य‚भ‚पि नेच्छेद्यः त‚स्य नोभ‚य‚द‚र्श‚न‚म् ।&त‚दार्थो ज्ञान‚मिति च ज्ञाते चेति ग‚ता क‚था ॥ ४४६ ॥\&[\smallbreak]


	
	    \end{quote}
	  
	  \endgroup
	

	  \pstart \leavevmode% starting standard par
	\hphantom{.}‚{\color{DodgerBlue3}‚सारूप्य‚म‚पि}\edtext{\textsuperscript{*}}{\edlabel{pvv.251-1}\label{pvv.251-1}\lemma{*}\Bfootnote{अनाकार‚वादी वैभाष्यादिः । साकार‚वाद्य‚पि नैयायिकादिर्यः स‚म्वेद‚नं ‚{\tiny $_{lb}$}‚नेच्छ‚ति ।}} श‚ब्दात् स्व‚स‚म्वेद‚{\tiny $_{4}$}‚‚{\color{DodgerBlue3}‚नं यो} वादी ‚{\color{DodgerBlue3}‚नेच्छेत् न त‚स्योभ‚य}‚स्यार्थ‚स्य ‚{\tiny $_{lb}$}‚ज्ञान‚स्य च ‚{\color{DodgerBlue3}‚द‚र्श‚नं} संग‚च्छ‚ते । सारूप्याभावेऽर्थ‚वेद‚नाऽयोगात् । स्व‚वेद‚नाभावे च न ‚{\tiny $_{lb}$}‚ज्ञान‚स‚म्वेद‚नं । अन्येन त‚द्ग्र‚ह‚स्य निषिद्ध‚त्वात्\edtext{}{\edlabel{pvv.251-2}\label{pvv.251-2}\lemma{त्वात्}\Bfootnote{ज्ञानान्त‚रेण स्व‚य‚म‚विदिते नान्य‚ग्र‚हायोगात् ।}} । य‚दा चैवं ‚{\color{DodgerBlue3}‚त‚दार्थो ज्ञान‚मिति} भेदः । ‚{\tiny $_{lb}$}‚‚{\color{DodgerBlue3}‚ते च ज्ञाते चोति क‚थापि ग‚तेति} कृत्स्नं ज‚ग‚द‚न्ध‚मूकं भ‚वेत् । प्र‚तीतिनिब‚न्ध‚न‚त्वाद‚स्य ‚{\tiny $_{lb}$}‚व्य‚व‚हार‚स्य । (४४६)
	\pend% ending standard par
      \label{div_pvv.2.447}
	  
	% new div opening: depth here is 2
	
	  \bigskip
	  \begingroup
	
	    \large
	  
	    \begin{quote}
	  
	    
	    \stanza[\smallbreak]
	\label{pv.2.447}\flagstanza{\tiny\textenglish{....2.447}}अथ स्व‚रूपं सा त‚र्हि स्व‚य‚मेव प्र‚काश‚ते ।&य‚त्त‚स्याम‚प्र‚काशायाम‚र्थः स्याद‚प्र‚काशितः ॥ ४४७ ॥\&[\smallbreak]


	
	    \end{quote}
	  
	  \endgroup
	

	  \pstart \leavevmode% starting standard par
	\hphantom{.}‚{\color{DodgerBlue3}‚अथ} य‚देत‚त्सिताद्य‚भिव्य‚क्तिरूपं स्फृट‚स‚म्वेद‚न‚मा‚{\tiny $_{5}$}‚भाति त‚द् बुद्धेः ‚{\color{DodgerBlue3}‚स्व‚रूपं} (।) ‚{\tiny $_{lb}$}‚सा बुद्धि‚{\color{DodgerBlue3}‚स्त‚र्हि स्व‚य‚मेवा}‚प‚रोक्ष‚त‚या ‚{\color{DodgerBlue3}‚प्र‚काश‚ते य‚द्य‚स्माद‚स्यां} बुद्धाव‚{\color{DodgerBlue3}‚प्र‚काशायां} प‚रोक्षाया‚{\color{DodgerBlue3}‚म‚र्थोऽप्र‚काशितः स्यात्} । प्र‚काश‚ते चार्थ इति बुद्धिर‚प्य‚प‚रोक्ष‚स्व‚भावेति ‚{\tiny $_{lb}$}‚स्व‚स‚म्वेद‚न‚सिद्धिः । (४४७)
	\pend% ending standard par
      \label{div_pvv.2.448}
	  
	% new div opening: depth here is 2
	
	  \bigskip
	  \begingroup
	
	    \large
	  
	    \begin{quote}
	  
	    
	    \stanza[\smallbreak]
	\label{pv.2.448}\flagstanza{\tiny\textenglish{....2.448}}एतेनानात्म‚वित्‏प‚क्षे स‚र्व्वार्थाद‚र्श‚नेन ये ।&अप्र‚त्य‚क्षां धियं प्राहुस्तेपि निर्व्व‚र्णितोत्त‚राः ॥ ४४८ ॥\&[\smallbreak]


	
	    \end{quote}
	  
	  \endgroup
	

	  \pstart \leavevmode% starting standard par
	\hphantom{.}‚{\color{DodgerBlue3}‚अनात्म‚वित्प‚क्षे} स्व‚संवेद‚नाभावे\edtext{}{\edlabel{pvv.251-3}\label{pvv.251-3}\lemma{नाभावे}\Bfootnote{य‚त्स‚र्व्वार्थाद‚र्श‚न‚मुक्तं पूर्व्व अर्थो ज्ञान‚मिति ते ।}} एतेनान‚न्त‚र‚मुप‚द‚र्शितेन ‚{\color{DodgerBlue3}‚स‚र्व्व}‚स्या‚{\color{DodgerBlue3}‚र्थ}‚स्या‚{\color{DodgerBlue3}‚द‚र्श‚{\tiny $_{lb}$}‚नेन} द‚र्श‚नाभाव‚प्र‚स‚ङ्गेन ‚{\color{DodgerBlue3}‚ये} जै मि नी या ‚{\color{DodgerBlue3}‚अप्र‚त्य‚क्षां धिय}‚म‚र्थाप‚त्तिग‚म्यामा‚{\tiny $_{6}$}‚‚{\color{DodgerBlue3}‚हुः तेपि ‚{\tiny $_{lb}$}‚निर्व्व‚र्ण्णितोत्त‚रा} द‚त्तोत्त‚रा बोद्ध‚व्याः । त‚था ह्य‚र्थ‚द‚र्श‚नान्य‚थानुप‚प‚त्त्या बुद्धिर्व्य‚व‚{\tiny $_{lb}$}‚स्थाप‚नीया । अर्थ‚द‚र्श‚न‚मेव तु बुद्धिप‚रोक्ष‚तायाम‚स‚ङ्ग‚त‚मिति न त‚द‚न्य‚थानुप‚प‚द्य‚मानं ‚{\tiny $_{lb}$}‚बृद्धिं क‚ल्प‚यितुम‚लं । (४४८)
	\pend% ending standard par
      \label{div_pvv.2.449}
	  
	% new div opening: depth here is 2
	

	  \pstart \leavevmode% starting standard par
	अपि च (।)
	\pend% ending standard par
      
	  \bigskip
	  \begingroup
	
	    \large
	  
	    \begin{quote}
	  
	    
	    \stanza[\smallbreak]
	\label{pv.2.449}\flagstanza{\tiny\textenglish{....2.449}}आश्र‚याल‚म्ब‚नाभ्यास‚भेदाद् भिन्न‚प्र‚वृत्त‚यः ।&सुख‚दुःखाभिलाषादिभेदा बुद्ध्य एव ताः ॥ ४४९ ॥\&[\smallbreak]


	
	    \end{quote}
	  
	  \endgroup
	

	  \pstart \leavevmode% starting standard par
	\hphantom{.}‚{\color{DodgerBlue3}‚आश्र‚य‚स्ये}‚न्द्रिय‚स्या‚{\color{DodgerBlue3}‚ल‚म्ब‚न}‚स्य सुखादिवेद‚नीय‚स्या‚{\color{DodgerBlue3}‚भ्यास}‚स्य च य‚था वृत्त‚स्य ‚{\tiny $_{lb}$}‚‚{\color{DodgerBlue3}‚भेदा}‚द्विशेषात् ‚{\color{DodgerBlue3}‚सुख‚दुःख‚भिलाषादि\edtext{}{\edlabel{pvv.251-4}\label{pvv.251-4}\lemma{भिलाषादि}\Bfootnote{द्वेष‚प्र‚त्य‚त्नादिरादिना ।}}भेदा भिन्न‚प्र‚वृत्त‚यो} नानाकाराः स‚म्वि‚{\tiny $_{lb}$}‚\leavevmode\ledsidenote{\textenglish{252/s}} \leavevmode\ledsidenote{\textenglish{49b/MA}} दित‚रूपा जाय‚{\tiny $_{7}$}‚न्ते ‚{\color{DodgerBlue3}‚बुद्ध्य\edtext{}{\edlabel{pvv.252-1}\label{pvv.252-1}\lemma{बुद्ध्य}\Bfootnote{सुखादिजातं प्र‚तिभासे ।}} एव} च ता बोध‚स्व‚भाव‚त्वात् । ज्ञानेनाभिन्न‚हेतुक‚त्वा\edtext{}{\edlabel{pvv.252-2}\label{pvv.252-2}\lemma{त्वा}\Bfootnote{पूर्व्वं बृद्धिरूपाः स्थापिताः ।}}च्च । ‚{\tiny $_{lb}$}‚(४४९)
	\pend% ending standard par
      \label{div_pvv.2.450}
	  
	% new div opening: depth here is 2
	
	  \bigskip
	  \begingroup
	
	    \large
	  
	    \begin{quote}
	  
	    
	    \stanza[\smallbreak]
	\label{pv.2.450}\flagstanza{\tiny\textenglish{....2.450}}प्र‚त्य‚क्षास्त‚द्‏विविक्त‚ञ्च नान्य‚त् किञ्चिद् विभाव्य‚ते ।&य‚त्त‚ज्ज्ञानं । प‚रोप्येतान् भुञ्चीतांन्येन विद्य‚दि ॥ ४५० ॥\&[\smallbreak]


	
	    \end{quote}
	  
	  \endgroup
	

	  \pstart \leavevmode% starting standard par
	\hphantom{.}त‚तः ‚{\color{DodgerBlue3}‚प्र‚त्य‚क्षाः} । न च सामान्येन वेद‚न‚मिति स्व‚वेद‚न‚तैव । ‚{\color{DodgerBlue3}‚न च ते}‚भ्यः सुखा‚{\tiny $_{lb}$}‚दिभ्यो ‚{\color{DodgerBlue3}‚विविक्तं} भिन्न‚{\color{DodgerBlue3}‚म‚न्य‚त् किञ्चिद्} बृद्धिस्व‚रूपं ‚{\color{DodgerBlue3}‚विभाव्य‚ते} उप‚ल‚भ्य‚ते ‚{\color{DodgerBlue3}‚य‚त्त‚त्} प्र‚त्य‚क्षं ‚{\color{DodgerBlue3}‚ज्ञानं} स्यात् । कि‚{\color{DodgerBlue3}‚ञ्चान्येन} ज्ञानेनान्य‚स्य ज्ञान‚स्य ‚{\color{DodgerBlue3}‚वित्} वेद‚नं य‚दीष्य‚ते त‚दा ‚{\tiny $_{lb}$}‚भोक्तृस‚न्तान‚व‚र्त्तिन ‚{\color{DodgerBlue3}‚एता}‚न् सुखादीन् ‚{\color{DodgerBlue3}‚प‚रः} प्र‚तिप‚त्ताऽल\edtext{}{\edlabel{pvv.252-3}\label{pvv.252-3}\lemma{त्ताऽल}\Bfootnote{पाश्चात्य‚ज्ञानं त‚दाल‚म्ब‚ते न स्व‚वेद‚नं त‚द‚नुमातृस‚न्तानेप्य‚स्ति ।}} भ्य‚मानो ‚{\color{DodgerBlue3}‚भुञ्जीत} सुखाद्युप‚{\tiny $_{lb}$}‚भोग‚वान्भ‚वे‚{\tiny $_{1}$}‚त् भोक्तृपुरुष‚व‚त् । (४५०)
	\pend% ending standard par
      \label{div_pvv.2.451}
	  
	% new div opening: depth here is 2
	
	  \bigskip
	  \begingroup
	
	    \large
	  
	    \begin{quote}
	  
	    
	    \stanza[\smallbreak]
	\label{pv.2.451}\flagstanza{\tiny\textenglish{....2.451}}त‚ज्जा त‚त्प्र‚तिभासा वा य‚दि धीर्वेत्ति नाप‚रा ।&आल‚म्ब‚मान‚स्यान्य‚स्याप्य‚स्त्य‚व‚श्य‚मिदं द्व‚य‚म् ॥ ४५१ ॥\&[\smallbreak]


	
	    \end{quote}
	  
	  \endgroup
	

	  \pstart \leavevmode% starting standard par
	\hphantom{.}‚{\color{DodgerBlue3}‚त‚स्मा}‚त्सुखादेर्जाता ‚{\color{DodgerBlue3}‚त‚त्प्र‚तिभासा} सुखादिप्र‚तिभासा ‚{\color{DodgerBlue3}‚वा} तान् सुखादीन् ‚{\tiny $_{lb}$}‚‚{\color{DodgerBlue3}‚वेत्ति} भोक्तृत्वेन ‚{\color{DodgerBlue3}‚नाप}‚रायाः काश्िच‚द् बु\edtext{}{\edlabel{pvv.252-4}\label{pvv.252-4}\lemma{बु}\Bfootnote{त‚न्नान्य‚स्य भोगः ।}}द्धिरिति य‚दीष्य‚ते त‚दा भोक्तृस‚न्तान‚व‚र्त्तिनः ‚{\tiny $_{lb}$}‚सुखादीना‚{\color{DodgerBlue3}‚ल‚म्ब‚मान‚स्यान्य‚स्य} पुरुषान्त‚र‚ज्ञान‚{\color{DodgerBlue3}‚स्येदं} त‚ज्ज‚त्वं त‚त्प्र‚तिभासि\edtext{}{\edlabel{pvv.252-5}\label{pvv.252-5}\lemma{तिभासि}\Bfootnote{उत्प‚त्तिसारूप्याभ्यामाल‚म्ब‚न‚व्य‚व‚स्थानात् ।}}त्वं ‚{\color{DodgerBlue3}‚द्व‚य}‚मा‚{\tiny $_{lb}$}‚ल‚म्ब‚नीय‚सुखाद्य‚पेक्ष‚याप्य‚{\color{DodgerBlue3}‚द‚श्य‚म‚स्ति} त‚तः सोपि भोक्ता स्यात् । (४५१)
	\pend% ending standard par
      \label{div_pvv.2.452_2.453}
	  
	% new div opening: depth here is 2
	
	  \bigskip
	  \begingroup
	
	    \large
	  
	    \begin{quote}
	  
	    
	    \stanza[\smallbreak]
	\label{pv.2.452a}\flagstanza{\tiny\textenglish{...2.452a}}अथ नोत्प‚द्य‚ते त‚स्मान्न च त‚त्प्र‚तिभासिनी ।&सा धोर्निर्विष‚या प्राप्ता;\&[\smallbreak]


	
	    \end{quote}
	  
	  \endgroup
	

	  \pstart \leavevmode% starting standard par
	\hphantom{.}‚{\color{DodgerBlue3}‚अथा}‚न्य‚स्य धीर्भोक्तृसुखादेः स‚काशा‚{\tiny $_{2}$}‚‚{\color{DodgerBlue3}‚न्नोत्प‚द्य‚ते नापि त‚त्प्र‚तिभासिनी}‚ष्य\edtext{}{\edlabel{pvv.252-6}\label{pvv.252-6}\lemma{ष्य}\Bfootnote{सुखाद्य‚नाल‚म्ब‚न‚त्वात् ।}}ते ‚{\tiny $_{lb}$}‚त‚दा ‚{\color{DodgerBlue3}‚सा धीर्निर्विष‚या प्राप्ता} । ग्राह्य‚स्य प‚र‚रूप‚स्याभावेपि प्र‚काश‚माना स्व‚प्र‚काशैव ‚{\tiny $_{lb}$}‚स्यात् ।
	\pend% ending standard par
      

	  \pstart \leavevmode% starting standard par
	स्यादेत‚त् (।) भोक्तुः सुखं य‚द्य‚पि स्व‚रूपेण प‚र‚बुद्ध्या न गृह्य‚ते त‚त्सामान्य‚मात्रं ‚{\tiny $_{lb}$}‚तु गृह्य‚ते इति भोक्तृत्व‚निराल‚म्ब‚न‚त्व‚योर‚भाव इत्याह (।)
	\pend% ending standard par
      
	  \bigskip
	  \begingroup
	
	    \large
	  
	    \begin{quote}
	  
	    
	    \stanza[\smallbreak]
	\label{pv.2.452b}\flagstanza{\tiny\textenglish{...2.452b}}सामान्यं च त‚द‚ग्र‚हे ॥ ४५२ ॥\&[\smallbreak]


	
	    \end{quote}
	  
	  \endgroup
	
	  \bigskip
	  \begingroup
	
	    \large
	  
	    \begin{quote}
	  
	    
	    \stanza[\smallbreak]
	\label{pv.2.453a}\flagstanza{\tiny\textenglish{...2.453a}}न गृह्य‚त इति प्रोक्तं;\&[\smallbreak]


	
	    \end{quote}
	  
	  \endgroup
	\textsuperscript{\textenglish{253/s}}

	  \pstart \leavevmode% starting standard par
	\hphantom{.}‚{\color{DodgerBlue3}‚त‚स्य} भोक्तृसुख‚विशेष‚स्या‚{\color{DodgerBlue3}‚ग्र‚हे} (४५२) त‚त्स‚म‚वायि ‚{\color{DodgerBlue3}‚सामान्यं न गृह्य‚त इति ‚{\tiny $_{lb}$}‚प्रोक्तं} । अत‚त्स‚मान‚ता व्य‚क्ती तेन नि‚{\tiny $_{3}$}‚त्योप‚ल‚म्भ‚न‚मि \cref{pv.2.20} त्यादिना ।
	\pend% ending standard par
      
	  \bigskip
	  \begingroup
	
	    \large
	  
	    \begin{quote}
	  
	    
	    \stanza[\smallbreak]
	\label{pv.2.453b}\flagstanza{\tiny\textenglish{...2.453b}}न च त‚द्व‚स्तु किञ्च‚न ।&त‚स्माद‚र्थाव‚भासोसौ नान्य‚स्त‚स्या धिय‚स्त‚तः ॥ ४५३ ॥\&[\smallbreak]


	
	    \end{quote}
	  
	  \endgroup
	

	  \pstart \leavevmode% starting standard par
	\hphantom{.}न च त‚त्सामान्यं ‚{\color{DodgerBlue3}‚किञ्च‚न \edtext{}{\edlabel{pvv.253-1}\label{pvv.253-1}\lemma{न}\Bfootnote{त‚दा व्य‚तिरेकाव्य‚तिरेकोभ‚य‚रूपेणायोगः प्रागृक्त एव ।}}व‚स्तु} (।)य‚था ह्युप‚ग‚त‚स्यानुप‚ल‚म्भ‚बाधित‚त्वात् ।
	\pend% ending standard par
      

	  \pstart \leavevmode% starting standard par
	\hphantom{.}‚{\color{DodgerBlue3}‚त‚स्माद}‚न‚न्त‚रोक्ताद् युक्तिक‚लापात् ‚{\color{DodgerBlue3}‚अर्थाव‚भासोसौ} स्फृटं प्र‚काश‚मान‚{\color{DodgerBlue3}‚स्त‚स्याः} प‚रोक्ष‚त्वेनेष्टाया ‚{\color{DodgerBlue3}‚धियो नान्यः} किन्तु त‚द्रूप एव (। ४५३)
	\pend% ending standard par
      \label{div_pvv.2.454}
	  
	% new div opening: depth here is 2
	
	  \bigskip
	  \begingroup
	
	    \large
	  
	    \begin{quote}
	  
	    
	    \stanza[\smallbreak]
	\label{pv.2.454}\flagstanza{\tiny\textenglish{....2.454}}सिद्धे प्र‚त्य‚क्ष‚भावात्म‚विदौ । गृह्णाति त‚त्पुनः ।&नाध्य‚क्ष‚मिति चेदेष कुतो भेदः स‚मार्थ‚योः ॥ ४५४ ॥\&[\smallbreak]


	
	    \end{quote}
	  
	  \endgroup
	

	  \pstart \leavevmode% starting standard par
	\hphantom{.}त‚तोऽर्थाभास‚ज्ञान‚योस्तादात्म्यात् ‚{\color{DodgerBlue3}‚प्र‚त्य‚क्ष‚भावात्म‚विदौ} प्र‚त्य‚क्ष‚त्व‚स्व‚सं‚{\color{DodgerBlue3}‚वित्ती ‚{\tiny $_{lb}$}‚सिद्धे} । ज्ञान‚स्याप‚रोक्ष‚त‚या प‚र‚निर‚पेक्ष‚प्र‚काश‚त्वाच्च ।
	\pend% ending standard par
      

	  \pstart \leavevmode% starting standard par
	स्यादेत‚त् (।) स्व‚स‚न्तान‚व‚र्त्तिनः‚{\tiny $_{4}$}‚ सुखादीन‚ध्य‚{\color{DodgerBlue3}‚क्ष‚माल‚म्ब‚ते} त‚तः प्रीतिप‚रिता‚{\tiny $_{lb}$}‚पादियोगाद् भोक्तृता (।)
	\pend% ending standard par
      

	  \pstart \leavevmode% starting standard par
	\hphantom{.}अन्य‚स्य ‚{\color{DodgerBlue3}‚पुन}‚स्तान् सुखादीन् ‚{\color{DodgerBlue3}‚नाध्य‚क्षं\edtext{}{\edlabel{pvv.253-2}\label{pvv.253-2}\lemma{क्षं}\Bfootnote{अन्य‚स्य ।}} गृह्णाति} किन्तु बुद्धिमात्रं\edtext{}{\edlabel{pvv.253-3}\label{pvv.253-3}\lemma{बुद्धिमात्रं}\Bfootnote{अनुमा ।}} त‚तः प्रीति‚{\tiny $_{lb}$}‚प‚रितापाद्य‚भावात् न भोक्तृत्त्व‚मिति चेत् । ‚{\color{DodgerBlue3}‚स‚मार्थ‚यो}‚रेक‚विष‚य‚योः स्व‚प‚र‚स‚न्तान‚{\tiny $_{lb}$}‚व‚र्त्तिनोर्ज्ञान‚योरेष प्र‚त्य‚क्षाप्र‚त्य‚क्ष‚ल‚क्ष‚णः ‚{\color{DodgerBlue3}‚कुतो\edtext{}{\edlabel{pvv.253-4}\label{pvv.253-4}\lemma{कुतो}\Bfootnote{अस्व‚वेद‚ने ।}} भेदः}‚। सुख‚स्व‚रूप‚विष‚य‚त्वात् ‚{\tiny $_{lb}$}‚द्व‚य‚म‚पि प्र‚त्य‚क्ष‚म‚प्र‚त्य‚क्ष‚म्वा स्यात् । (४५४)
	\pend% ending standard par
      \label{div_pvv.2.455}
	  
	% new div opening: depth here is 2
	
	  \bigskip
	  \begingroup
	
	    \large
	  
	    \begin{quote}
	  
	    
	    \stanza[\smallbreak]
	\label{pv.2.455}\flagstanza{\tiny\textenglish{....2.455}}अदृष्टैकार्थ‚योगादेः स‚म्विदो निय‚मो य‚दि ।&स‚र्व‚थान्यो न गृह्णीयात्स‚म्विद्भेदोप्य‚पोदितः ॥ ४५५ ॥\&[\smallbreak]


	
	    \end{quote}
	  
	  \endgroup
	

	  \pstart \leavevmode% starting standard par
	\hphantom{.}‚{\color{DodgerBlue3}‚अदृष्टा}‚च्छुभाशुभादिल‚क्ष‚णादे‚{\color{DodgerBlue3}‚कार्थ}‚स‚म‚{\tiny $_{5}$}‚वायादेर्व्वा निमित्तात्स्व‚स‚न्तान‚व‚र्त्ति‚{\tiny $_{lb}$}‚सुख‚ग्राहिकायाः ‚{\color{DodgerBlue3}‚स‚म्विदो निय‚मो}\edtext{}{\edlabel{pvv.253-5}\label{pvv.253-5}\lemma{ग्राहिकायाः}\Bfootnote{नान्य‚स्यैकात्म‚स‚म‚वायः । ईश्व‚र‚प्र‚साद आदिना ।}} भोग‚रूप‚त्वाव‚धार‚णं । तेनान्य‚स्य न भोक्तृतेति ‚{\tiny $_{lb}$}‚य‚दीष्य‚ते त‚दाऽदृष्टेनैकार्थ‚स‚म‚वायेन वा निय‚मितं सुखाद्य‚{\color{DodgerBlue3}‚न्यो न गृह्णीया}‚देवेत्य‚स्तु ‚{\tiny $_{lb}$}‚स्व‚रूप‚प्र‚तिभासे\edtext{}{\edlabel{pvv.253-6}\label{pvv.253-6}\lemma{तिभासे}\Bfootnote{प‚र‚स्य स्वीकृते ।}} भोक्तृत्व‚स्याप्र‚तिषेधात् । न ह्यात्म‚स‚म‚वायितामात्रेण सुखादेर्भोगः ‚{\tiny $_{lb}$}‚किन्त‚र्ह्युप‚ल‚म्भेन । स च प‚र‚स्याप्य‚स्तीति भोक्ता स्यात् । एक‚स्य विष‚य‚स्य ‚{\color{DodgerBlue3}‚स‚म्वि‚{\tiny $_{lb}$}‚द्भे‚{\tiny $_{6}$}‚दो} ग्र‚ह‚ण‚भेदे‚{\color{DodgerBlue3}‚पि} व्य‚क्ताव्य‚क्त‚त‚या ‚{\color{DodgerBlue3}‚उदितो} निराकृतः । त‚तः स्व‚रूप‚प्र‚तिभास‚स्यै‚{\tiny $_{lb}$}‚क‚प्र‚कार‚त्वात् । (४५५)
	\pend% ending standard par
      \textsuperscript{\textenglish{254/s}}\label{div_pvv.2.456}
	  
	% new div opening: depth here is 2
	
	  \bigskip
	  \begingroup
	
	    \large
	  
	    \begin{quote}
	  
	    
	    \stanza[\smallbreak]
	\label{pv.2.456}\flagstanza{\tiny\textenglish{....2.456}}येषाञ्च योगिनोन्य‚स्य प्र‚त्य‚क्षेण सुखादिक‚म् ।&विद‚न्ति तुल्यानुभ‚वास्त‚द्व‚त्तेपि स्युरातुराः ॥ ४५६ ॥\&[\smallbreak]


	
	    \end{quote}
	  
	  \endgroup
	

	  \pstart \leavevmode% starting standard par
	\hphantom{.}‚{\color{DodgerBlue3}‚येषाञ्च}\edtext{\textsuperscript{*}}{\edlabel{pvv.254-1}\label{pvv.254-1}\lemma{*}\Bfootnote{वैभाष्यादीनां ।}} प‚रेषां कार‚णादीनां ‚{\color{DodgerBlue3}‚योगिनोन्य‚स्य सुखादिकं प्र‚त्य‚क्षेण} योग‚ब‚लोत्प‚न्नेन ‚{\tiny $_{lb}$}‚‚{\color{DodgerBlue3}‚विद‚न्ती}‚ति म‚तं । तेषां म‚ते प‚रेण सुखिना दुःखिना च स‚ह ‚{\color{DodgerBlue3}‚तुल्यानुभ‚वा} योगिन ‚{\tiny $_{lb}$}‚इति ‚{\color{DodgerBlue3}‚त‚द्व‚त्} दुःखिपुरुष‚व‚त् योगिनो‚{\color{DodgerBlue3}‚प्यातुरा} दुःख‚पीडिताः स्युः । (४५६)
	\pend% ending standard par
      \label{div_pvv.2.457}
	  
	% new div opening: depth here is 2
	
	  \bigskip
	  \begingroup
	
	    \large
	  
	    \begin{quote}
	  
	    
	    \stanza[\smallbreak]
	\label{pv.2.457}\flagstanza{\tiny\textenglish{....2.457}}विष‚येन्द्रिय‚स‚म्पाताभावात्तेषां त‚दुद्भ‚व‚म् ।&नोदेति दुःख‚मिति चेत् न वै दुःख‚स‚मुद्भ‚वः ॥ ४५७ ॥\&[\smallbreak]


	
	    \end{quote}
	  
	  \endgroup
	\textsuperscript{\textenglish{50a/MA}}

	  \pstart \leavevmode% starting standard par
	\hphantom{.}‚{\color{DodgerBlue3}‚विष‚येन्द्रिय‚योः स‚म्पात}‚स्य संस‚र्ग‚स्या‚{\color{DodgerBlue3}‚भावात् । त‚दुद्भ‚{\tiny $_{7}$}‚वं} विष‚येन्द्रिय‚संस‚र्ग‚जं ‚{\tiny $_{lb}$}‚‚{\color{DodgerBlue3}‚दुःखं तेषां} योगिनां ‚{\color{DodgerBlue3}‚नोदेतीति चेत् । न वै} नैव ‚{\color{DodgerBlue3}‚दुःख}‚स्य ‚{\color{DodgerBlue3}‚स‚मुद्भ‚व} उत्प‚त्तिः (४५७)
	\pend% ending standard par
      \label{div_pvv.2.458}
	  
	% new div opening: depth here is 2
	
	  \bigskip
	  \begingroup
	
	    \large
	  
	    \begin{quote}
	  
	    
	    \stanza[\smallbreak]
	\label{pv.2.458}\flagstanza{\tiny\textenglish{....2.458}}दुःख‚स्य वेद‚नं किन्तु दुःख‚ज्ञान‚स‚मुद्भ‚वः ।&न हि दुःखाद्य‚संवेद्यं पीडानुग्र‚ह‚कार‚ण‚म् ॥ ४५८ ॥\&[\smallbreak]


	
	    \end{quote}
	  
	  \endgroup
	

	  \pstart \leavevmode% starting standard par
	\hphantom{.}‚{\color{DodgerBlue3}‚दुःख‚स्य वेद‚नं} दुःखित्वं । ‚{\color{DodgerBlue3}‚किन्तु दुःख}‚विष‚य‚{\color{DodgerBlue3}‚ज्ञान‚स‚मुद्भ‚वो} दुःखिता (।) न हि ‚{\tiny $_{lb}$}‚‚{\color{DodgerBlue3}‚दुःखं आदि}‚श‚ब्दात् सुख‚{\color{DodgerBlue3}‚म‚स‚म्वेद्यं} अज्ञाय‚मानं ‚{\color{DodgerBlue3}‚पीडानुग्र‚ह‚योः कार‚णं} भ‚व‚ति येन ‚{\tiny $_{lb}$}‚दुःख‚सुख‚योरूत्प‚त्ती दुःखितासुखिते स्यातां । (४५८)
	\pend% ending standard par
      \label{div_pvv.2.459}
	  
	% new div opening: depth here is 2
	

	  \begin{center}%% label @type='head'
	\textbf{(ग. स्व‚संवेद‚न‚न‚ये योगिनाम‚नातुर‚ता)}
	\end{center}
	

	  \pstart \leavevmode% starting standard par
	न‚नु बौद्ध‚स्यापि म‚ते योगिनः सुखाद्याकारेण ज्ञानेन प‚र‚दुःख‚माल‚म्ब‚मानाः ‚{\tiny $_{lb}$}‚क‚स्मादा‚{\tiny $_{1}$}‚तुरा न भ‚व‚न्ति दुःखिन इव योगिनोपि दुःखा\edtext{}{\edlabel{pvv.254-2}\label{pvv.254-2}\lemma{दुःखा}\Bfootnote{प‚र‚चित्ताभिज्ञ‚या म‚न‚सा सुखादियुतं म‚नो वेत्ति योगी य‚थाग्न्याद‚यः स्व‚य‚म‚{\tiny $_{lb}$}‚स‚ञ्च‚र‚न्तः स्व‚ज्ञानेध्य‚क्षास्त‚दाभास‚मात्रेण । न हि दुःखादीति वृत्तौ सिद्धान्ते ‚{\tiny $_{lb}$}‚योजितं ।}}कारं स्व‚स‚म्वेद‚न‚ञ्च ज्ञान‚{\tiny $_{lb}$}‚मिति न क‚श्चिद्विशेष इत्याह (।)
	\pend% ending standard par
      
	  \bigskip
	  \begingroup
	
	    \large
	  
	    \begin{quote}
	  
	    
	    \stanza[\smallbreak]
	\label{pv.2.459}\flagstanza{\tiny\textenglish{....2.459}}भास‚मानं स्व‚रूपेण पीडा दुःखं स्व‚यं य‚दा ।&न त‚दाल‚म्ब‚नं ज्ञानं न त‚दैवं प्र‚युज्य‚ते ॥ ४५९ ॥\&[\smallbreak]


	
	    \end{quote}
	  
	  \endgroup
	

	  \pstart \leavevmode% starting standard par
	\hphantom{.}‚{\color{DodgerBlue3}‚दुःखं स्व‚यं}\edtext{\textsuperscript{*}}{\edlabel{pvv.254-3}\label{pvv.254-3}\lemma{*}\Bfootnote{स्व‚स‚न्त‚तिजं ।}} प‚र‚निर‚पेक्ष‚प्र‚काशं ‚{\color{DodgerBlue3}‚स्व‚रूपेण} प्र‚काश‚स्व‚भावेन ‚{\color{DodgerBlue3}‚भास‚मानं पीडा} (।) ‚{\tiny $_{lb}$}‚त‚त्त‚स्मादुत्प‚न्नं त‚त्स‚रूपं ‚{\color{DodgerBlue3}‚त‚दाल‚म्ब‚नं} योगिनो ‚{\color{DodgerBlue3}‚ज्ञान}‚न्न पीडेति य‚दा बौद्धैरिष्य‚ते ‚{\color{DodgerBlue3}‚त‚दैवं} योगिनोपि प‚र‚दुःखाल‚म्ब‚का दुःखिनः स्युरिति ‚{\color{DodgerBlue3}‚न युज्य‚ते} येनायं भेदः (४५९)
	\pend% ending standard par
      \textsuperscript{\textenglish{255/s}}\label{div_pvv.2.460}
	  
	% new div opening: depth here is 2
	
	  \bigskip
	  \begingroup
	
	    \large
	  
	    \begin{quote}
	  
	    
	    \stanza[\smallbreak]
	\label{pv.2.460}\flagstanza{\tiny\textenglish{....2.460}}भिन्ने ज्ञान‚स्य स‚र्व्व‚स्य तेनाल‚म्ब‚न‚वेद‚ने ।&अर्थ‚सारूप्य‚माल‚म्ब आत्मा वित्तिः स्व‚यं स्फुटा ॥ ४६० ॥\&[\smallbreak]


	
	    \end{quote}
	  
	  \endgroup
	

	  \pstart \leavevmode% starting standard par
	\hphantom{.}तेन ‚{\color{DodgerBlue3}‚स‚र्व्व‚स्य ज्ञान‚स्याल‚म्ब‚{\tiny $_{2}$}‚न‚वेद‚ने भिन्ने} भिन्न‚ल‚क्ष‚णे त‚था ‚{\color{DodgerBlue3}‚ह्य‚र्थ\edtext{}{\edlabel{pvv.255-1}\label{pvv.255-1}\lemma{र्थ}\Bfootnote{अर्थान्त‚रात्सारूप्येणोत्प‚त्तिः ।}}सारूप्य‚मा‚{\tiny $_{lb}$}‚ल‚म्ब} आल‚म्ब‚नार्थः । ‚{\color{DodgerBlue3}‚आत्मा स्व‚यं} प\edtext{}{\edlabel{pvv.255-2}\label{pvv.255-2}\lemma{प}\Bfootnote{ग्राह्य‚ग्राह‚क‚त्वादिजात्यादिर‚हितः ।}}र‚निर‚पेक्षः ‚{\color{DodgerBlue3}‚स्फुटा वित्ति}\edtext{}{\edlabel{pvv.255-3}\label{pvv.255-3}\lemma{पेक्षः}\Bfootnote{स्व‚स‚म्वेद‚न‚रूपोत्प‚त्तिः}}र्व्वेद‚नार्थः । (४६०)
	\pend% ending standard par
      \label{div_pvv.2.461}
	  
	% new div opening: depth here is 2
	
	  \bigskip
	  \begingroup
	
	    \large
	  
	    \begin{quote}
	  
	    
	    \stanza[\smallbreak]
	\label{pv.2.461}\flagstanza{\tiny\textenglish{....2.461}}अपि चाध्य‚क्ष‚ताऽभावे धियः स्याल्लिङ्ग‚तो ग‚तिः ॥&त‚च्चाक्ष‚म‚र्थो धीः पूर्व्वो म‚न‚स्कारोपि वा भ‚वेत् ॥ ४६१ ॥\&[\smallbreak]


	
	    \end{quote}
	  
	  \endgroup
	

	  \pstart \leavevmode% starting standard par
	\hphantom{.}‚{\color{DodgerBlue3}‚अपि च} धियो‚{\color{DodgerBlue3}‚ऽध्य‚क्ष‚ताऽभावे लिङ्ग‚तो ग‚तिः स्यात् । त‚च्च} लिङ्ग‚म्भ‚व‚द‚क्ष‚{\tiny $_{lb}$}‚मिन्द्रिय‚{\color{DodgerBlue3}‚म‚र्थो} विष‚यो ‚{\color{DodgerBlue3}‚धी}\edtext{}{\edlabel{pvv.255-4}\label{pvv.255-4}\lemma{यो}\Bfootnote{धीः स्व‚य‚मिति वृत्तिः । यानुमेया सैव प्र‚कारान्त‚रेण लिङ्गं क‚ल्प्येत य‚था ‚{\tiny $_{lb}$}‚य‚त्कृत‚कं त‚देवानित्यं । अन‚न्त‚रानुमान‚मेव युक्तं ।}}र‚न‚न्त‚रा ‚{\color{DodgerBlue3}‚पूर्व्व}‚को ‚{\color{DodgerBlue3}‚म‚न‚स्कारो वा भ‚वेत् ।} (४६१)
	\pend% ending standard par
      \label{div_pvv.2.462_2.463}
	  
	% new div opening: depth here is 2
	
	  \bigskip
	  \begingroup
	
	    \large
	  
	    \begin{quote}
	  
	    
	    \stanza[\smallbreak]
	\label{pv.2.462}\flagstanza{\tiny\textenglish{....2.462}}कार्य‚कार‚ण‚साम‚ग्र‚याम‚स्यां स‚म्ब‚न्धि नाप‚राम् ।&साम‚र्थ्याद‚र्श‚नात्त‚त्र नेन्द्रियं व्य‚भिचार‚तः ॥ ४६२ ॥\&[\smallbreak]


	
	    \end{quote}
	  
	  \endgroup
	
	  \bigskip
	  \begingroup
	
	    \large
	  
	    \begin{quote}
	  
	    
	    \stanza[\smallbreak]
	\label{pv.2.463}\flagstanza{\tiny\textenglish{....2.463}}त‚थार्थो धीम‚न‚स्कारौ ज्ञानं तौ च न सिध्य‚तः ।&नाप्र‚सिद्ध‚स्य लिङ्ग‚त्वं व्य‚क्तिर‚र्थ‚स्य चेन्म‚ता ॥ ४६३ ॥\&[\smallbreak]


	
	    \end{quote}
	  
	  \endgroup
	
	  \bigskip
	  \begingroup
	
	    \large
	  
	    \begin{quote}
	  
	    
	    \stanza[\smallbreak]
	\label{pv.2.464a}\flagstanza{\tiny\textenglish{...2.464a}}लिङ्गं;\&[\smallbreak]


	
	    \end{quote}
	  
	  \endgroup
	

	  \pstart \leavevmode% starting standard par
	\hphantom{.}य‚स्मा‚{\color{DodgerBlue3}‚त्कार्य‚कार‚ण‚साम‚ग्र‚याम‚स्या}‚मेभ्यो‚{\color{DodgerBlue3}‚ऽप‚र‚मा}‚त्म‚नः संयोगादि‚{\color{DodgerBlue3}‚स‚म्ब‚न्धि नास्ति ‚{\tiny $_{lb}$}‚साम‚र्थ्याद‚र्श‚नात् । त‚त्र}‚{\tiny $_{3}$}‚ तेष्वि‚{\color{DodgerBlue3}‚\edtext{\textsuperscript{*}}{\edlabel{pvv.255-5}\label{pvv.255-5}\lemma{*}\Bfootnote{सुप्त‚मूर्च्छादौ ।}}न्द्रियं} ताव‚न्न लिङ्गं ‚{\color{DodgerBlue3}‚व्य‚भिचार‚तः} स‚त्य‚पि त‚स्मिन् ‚{\tiny $_{lb}$}‚ज्ञानाभावात् । (४६२) ‚{\color{DodgerBlue3}‚त‚थार्थोपि} ज्ञान‚व्य‚भिचारान्न लिङ्ग‚म् । ‚{\color{DodgerBlue3}‚धीम‚न‚स्कारौ} बोध‚स्व‚भाव‚त्वात् ‚{\color{DodgerBlue3}‚ज्ञानं तौ च} लीङ्ग‚ज्ञाना\edtext{}{\edlabel{pvv.255-6}\label{pvv.255-6}\lemma{ज्ञाना}\Bfootnote{स‚र्व्वं लिङ्ग‚मात्म‚नि ज्ञानापेक्षं लिङ्गिबोधोपायो भ‚व‚ति ।}}त्प्राङ् ‚{\color{DodgerBlue3}‚न सिध्य‚तः} ज्ञान‚स्यानुमेय‚{\tiny $_{lb}$}‚त्वात् (।) न चा‚{\color{DodgerBlue3}‚प्र‚सिद्ध‚स्य} निश्चित‚स्य ‚{\color{DodgerBlue3}‚लिङ्ग‚त्वं} स‚त्तामात्रेण लिङ्ग‚त्वेऽतिप्र‚स‚ङ्गात् । ‚{\tiny $_{lb}$}‚‚{\color{DodgerBlue3}‚अर्थ‚स्य व्य‚क्तिः} स्फुट‚ता बुर्द्ध‚र्लिङ्गं ‚{\color{DodgerBlue3}‚म‚ता चेत्} (४६३)
	\pend% ending standard par
      \label{div_pvv.2.464}
	  
	% new div opening: depth here is 2
	
	  \bigskip
	  \begingroup
	
	    \large
	  
	    \begin{quote}
	  
	    
	    \stanza[\smallbreak]
	\label{pv.2.464b}\flagstanza{\tiny\textenglish{...2.464b}}सैव न‚नु ज्ञानं व्य‚क्तोर्थोनेन व‚र्ण्णितः ।&व्य‚क्ताव‚न‚नुभूतायां त‚द्व्य‚क्त‚त्वाविनिश्च‚यात् ॥ ४६४ ॥\&[\smallbreak]


	
	    \end{quote}
	  
	  \endgroup
	

	  \pstart \leavevmode% starting standard par
	\hphantom{.}‚{\color{DodgerBlue3}‚न‚नु सैव} व्य‚क्तिर्ज्ञान‚म‚र्थ‚प्र‚काश‚ल‚क्ष‚ण‚{\tiny $_{4}$}‚त्वात् । न च त‚देव लिङ्गि चेति युक्तं । ‚{\tiny $_{lb}$}‚अनेन व्य‚क्तेर्लिङ्ग‚त्व‚क‚थ‚नेन ‚{\color{DodgerBlue3}‚व्य‚क्तोऽर्थो व‚र्ण्णितः} प्र‚तिक्षिप्तः । त‚था हि ‚{\color{DodgerBlue3}‚व्य‚क्तौ} \leavevmode\ledsidenote{\textenglish{256/s}} बृद्धिरूपायाम‚{\color{DodgerBlue3}‚न‚नुभूताया}‚म‚र्थ‚स‚म्ब‚न्धिन‚{\color{DodgerBlue3}‚स्त‚द्ब्य‚क्त‚त्व‚स्य} व्य‚क्तिव्य‚क्त‚त्व‚स्या‚{\color{DodgerBlue3}‚विनिश्च‚{\tiny $_{lb}$}‚यान्न} लि\edtext{}{\edlabel{pvv.256-1}\label{pvv.256-1}\lemma{लि}\Bfootnote{व्य‚क्तोऽर्थो लिङ्ग‚न्नार्थ‚मात्र‚मिति य‚त्क‚ल्प्य‚ते त‚स्य ।}}ङ्ग‚त्वं स‚त्तामात्रेण लिङ्ग‚त्वेऽतिप्र‚स‚ङ्गात् । (४६४)
	\pend% ending standard par
      \label{div_pvv.2.465}
	  
	% new div opening: depth here is 2
	
	  \bigskip
	  \begingroup
	
	    \large
	  
	    \begin{quote}
	  
	    
	    \stanza[\smallbreak]
	\label{pv.2.465a}\flagstanza{\tiny\textenglish{...2.465a}}अथार्थ‚स्यैव क‚श्चित्स विशेषो व्य‚क्तिरिष्य‚ते ।\&[\smallbreak]


	
	    \end{quote}
	  
	  \endgroup
	

	  \pstart \leavevmode% starting standard par
	\hphantom{.}‚{\color{DodgerBlue3}‚अथार्थ‚स्यैव} स्व‚भाव‚भूतः ‚{\color{DodgerBlue3}‚स क‚श्चित्} स्व‚भाव‚{\color{DodgerBlue3}‚विशेषो व्य‚क्तिरिष्य‚ते} न ज्ञानं ।
	\pend% ending standard par
      
	  \bigskip
	  \begingroup
	
	    \large
	  
	    \begin{quote}
	  
	    
	    \stanza[\smallbreak]
	\label{pv.2.465b}\flagstanza{\tiny\textenglish{...2.465b}}नानुत्पाद‚व्य‚य‚व‚तो विशेषोऽर्थ‚स्य क‚श्च‚न ॥ ४६५ ॥\&[\smallbreak]


	
	    \end{quote}
	  
	  \endgroup
	

	  \pstart \leavevmode% starting standard par
	\hphantom{.}त‚द‚{\color{DodgerBlue3}‚प्य‚र्थ‚स्य} स्थिरैक‚रूप‚त्वाद‚{\color{DodgerBlue3}‚नुत्पाद‚व्य‚य‚व‚तो‚{\tiny $_{25}$}‚\edtext{}{\edlabel{pvv.256-2}\label{pvv.256-2}\lemma{तो}\Bfootnote{य‚त् केन‚चिज्ज्ञानेन गृह्य‚माणेस्य व्य‚क्तिर्व्य‚व‚स्थितिः ।}} विशेषो} व्य‚क्तिरूपः ‚{\color{DodgerBlue3}‚क‚श्च‚न} न ‚{\tiny $_{lb}$}‚स‚ङ्ग‚तः । (४६५)
	\pend% ending standard par
      \label{div_pvv.2.466}
	  
	% new div opening: depth here is 2
	
	  \bigskip
	  \begingroup
	
	    \large
	  
	    \begin{quote}
	  
	    
	    \stanza[\smallbreak]
	\label{pv.2.466}\flagstanza{\tiny\textenglish{....2.466}}त‚दिष्टौ वा प्र‚तिज्ञानं क्ष‚ण‚भ‚ङ्गः प्र‚स‚ज्य‚ते ।&स च ज्ञातोऽथ‚वाऽज्ञातो भ‚वेज्ज्ञात‚स्य लिङ्ग‚ता ॥ ४६६ ॥\&[\smallbreak]


	
	    \end{quote}
	  
	  \endgroup
	

	  \pstart \leavevmode% starting standard par
	\hphantom{.}त‚स्य विशेष‚{\color{DodgerBlue3}‚स्येष्टौ वा प्र‚तिज्ञान}‚म‚र्थ‚स्य पूर्व्व‚स्व‚भाव‚नाशे स‚ति स्व‚भावान्त‚रो‚{\tiny $_{lb}$}‚त्पादात् ‚{\color{DodgerBlue3}‚क्ष‚ण‚भ‚ङ्गः प्र‚स‚ज्य‚ते} (।) ‚{\color{DodgerBlue3}‚स चा}‚र्थ‚स्व‚भाव‚विशेषो ‚{\color{DodgerBlue3}‚ज्ञातोऽज्ञातो वा भ‚वेत्} लिङ्गं ज्ञान‚स्य त‚त्र ‚{\color{DodgerBlue3}‚ज्ञात‚स्य य‚दि लीङ्ग‚ते}‚ष्य‚ते (। ४६६)
	\pend% ending standard par
      \label{div_pvv.2.467}
	  
	% new div opening: depth here is 2
	

	  \pstart \leavevmode% starting standard par
	त‚दा (।)
	\pend% ending standard par
      
	  \bigskip
	  \begingroup
	
	    \large
	  
	    \begin{quote}
	  
	    
	    \stanza[\smallbreak]
	\label{pv.2.467}\flagstanza{\tiny\textenglish{....2.467}}य‚दि ज्ञानेऽप‚रिच्छिन्ने ज्ञातोसाविति त‚त्कुतः ।&ज्ञात‚त्वेनाप‚रिच्छिन्न‚म‚पि त‚द् ग‚म‚कं क‚थ‚म् ॥ ४६७ ॥\&[\smallbreak]


	
	    \end{quote}
	  
	  \endgroup
	

	  \pstart \leavevmode% starting standard par
	\hphantom{.}‚{\color{DodgerBlue3}‚ज्ञानेऽप‚रिच्छिन्ने} त‚दुपाधिर्ज्ञातोसाव‚र्थों लिङ्ग\edtext{}{\edlabel{pvv.256-3}\label{pvv.256-3}\lemma{लिङ्ग}\Bfootnote{न कार‚क‚व‚ज्‏ज्ञाप‚कोऽज्ञातोपि कार्य‚कारी ।}}मिति य‚दिष्टं त‚त्कुत ‚{\tiny $_{lb}$}‚उप‚प‚द्य‚ते (।) अथाज्ञात‚स्य लिङ्ग‚ता त‚दा ‚{\color{DodgerBlue3}‚ज्ञात‚त्वे‚{\tiny $_{6}$}‚नाप‚रिच्छिन्न‚म‚पि} त‚द्व‚स्तु ‚{\color{DodgerBlue3}‚क‚थं ‚{\tiny $_{lb}$}‚ग‚म‚कं} लिङ्गं स‚त्तामात्रेण ग‚म‚क‚त्वेऽतिप्र‚स‚ङ्गादित्युक्तं । न च ज्ञानाद‚र्श‚ने दृष्ट‚ता ‚{\tiny $_{lb}$}‚युक्ता । (४६७)
	\pend% ending standard par
      \label{div_pvv.2.468_2.469}
	  
	% new div opening: depth here is 2
	
	  \bigskip
	  \begingroup
	
	    \large
	  
	    \begin{quote}
	  
	    
	    \stanza[\smallbreak]
	\label{pv.2.468a}\flagstanza{\tiny\textenglish{...2.468a}}अदृष्टादृष्ट‚योन्येन द्र‚ष्ट्रा दृष्टा न हि क्व‚चित् ।\&[\smallbreak]


	
	    \end{quote}
	  
	  \endgroup
	

	  \pstart \leavevmode% starting standard par
	\hphantom{.}‚{\color{DodgerBlue3}‚हिर्य}‚स्माद‚{\color{DodgerBlue3}‚दृष्टा दृष्टि}‚र्ज्ञानं येषां तेऽर्थाः ‚{\color{DodgerBlue3}‚क्व‚चि}‚द‚न्येन ‚{\color{DodgerBlue3}‚द्र‚ष्ट्रा दृष्टा} इति न दृष्टा ‚{\tiny $_{lb}$}‚निश्च‚य‚विष‚याः स्युः ।
	\pend% ending standard par
      

	  \pstart \leavevmode% starting standard par
	अथार्थ‚स्यै (व) वि\edtext{}{\edlabel{pvv.256-4}\label{pvv.256-4}\lemma{वि}\Bfootnote{एक‚दृष्टाव‚प्य‚न्य‚स्यास्त्य‚र्थ‚स्याज्ञातो विशेषः ।}}शेषः क‚श्चिद् बुद्धिकृत आस्ते तेन बुद्ध्य‚नुमान‚मित्याह (।)
	\pend% ending standard par
      
	  \bigskip
	  \begingroup
	
	    \large
	  
	    \begin{quote}
	  
	    
	    \stanza[\smallbreak]
	\label{pv.2.468b}\flagstanza{\tiny\textenglish{...2.468b}}विशेषः सोन्य‚दृष्टाव‚प्य‚स्तीति स्यात्स्व‚धीग‚तिः ॥ ४६८ ॥\&[\smallbreak]


	
	    \end{quote}
	  
	  \endgroup
	\textsuperscript{\textenglish{257/s}}

	  \pstart \leavevmode% starting standard par
	\hphantom{.}‚{\color{DodgerBlue3}‚स विशेषो}‚ऽर्थ‚स्यान्येन पुरुषेण ‚{\color{DodgerBlue3}‚दृष्टाव‚प्य‚स्तीति} पुरुषान्त‚र‚स्यात‚द्व्यापृतेन्द्रि‚{\tiny $_{lb}$}‚य‚स्य‚{\tiny $_{7}$}‚ त‚स्माद‚र्थ‚ग‚त‚विशेषात् ‚{\color{DodgerBlue3}‚स्व‚धीग\edtext{}{\edlabel{pvv.257-1}\label{pvv.257-1}\lemma{धीग}\Bfootnote{त‚म‚र्थ‚म‚प‚श्य‚तोपि स्व‚बुद्ध्य‚नुमानं स्यान्न च युक्तं ।}}तिः स्यात्} । (४६८)
	\pend% ending standard par
      \textsuperscript{\textenglish{50b/MA}}‚{\tiny $_{lb}$}‚

	  \pstart \leavevmode% starting standard par
	अथ ध‚र्म‚स्य साधार‚ण‚त्वात् त‚स्य बुद्ध्य‚व्य‚भिचारात् ।
	\pend% ending standard par
      
	  \bigskip
	  \begingroup
	
	    \large
	  
	    \begin{quote}
	  
	    
	    \stanza[\smallbreak]
	\label{pv.2.469}\flagstanza{\tiny\textenglish{....2.469}}त‚स्माद‚नुमितिर्बुद्धेः स्व‚ध‚र्म‚निर‚पेक्षिणः ।&केव‚लान्नार्थ‚ध‚र्मात्कः; स्व‚ध‚र्मः स्व‚धियो प‚रः ॥ ४६९ ॥\&[\smallbreak]


	
	    \end{quote}
	  
	  \endgroup
	

	  \pstart \leavevmode% starting standard par
	\hphantom{.}‚{\color{DodgerBlue3}‚त‚स्मात् केव‚लाद‚र्थ‚ध‚र्मात् स्व‚ध‚र्म‚निर‚पेक्षिणोऽ}‚नुमातृपुरुषात्म‚भूत‚ज्ञान‚निर‚पेक्षा‚{\tiny $_{lb}$}‚द्व्ुद्धेर‚नुमितिर्न स‚म्भ‚व‚ति स‚र्व्व‚स्यैव त‚स्मात् स्व‚बुद्ध्य‚नुमान‚प्र‚स‚ङ्गात् । अथात्म‚ध‚र्म ‚{\tiny $_{lb}$}‚ए\edtext{}{\edlabel{pvv.257-2}\label{pvv.257-2}\lemma{ए}\Bfootnote{पुरुष‚स्य ।}}व स क‚श्चिद् बुद्धेर्ग‚म‚क इति चेत् । आह (।) ‚{\color{DodgerBlue3}‚स्व‚स्या}‚त्म‚नो ‚{\color{DodgerBlue3}‚ध‚र्मः स्व}‚बुद्धेरात्म‚{\tiny $_{lb}$}‚स‚म्ब‚न्धिन्या बुद्धेर‚प‚रोन्यः कः (। ४६९)
	\pend% ending standard par
      \label{div_pvv.2.470}
	  
	% new div opening: depth here is 2
	
	  \bigskip
	  \begingroup
	
	    \large
	  
	    \begin{quote}
	  
	    
	    \stanza[\smallbreak]
	\label{pv.2.470}\flagstanza{\tiny\textenglish{....2.470}}प्र‚त्य‚क्षाधिग‚तो हेतुः तुल्य‚कार‚ण‚ज‚न्म‚नः ।&त‚स्य भेदः कुतो बुद्धेर्व्य‚भिचार्य‚न्य‚ज‚श्च सः ॥ ४७० ॥\&[\smallbreak]


	
	    \end{quote}
	  
	  \endgroup
	

	  \pstart \leavevmode% starting standard par
	\hphantom{.}‚{\color{DodgerBlue3}‚प्र‚त्य‚क्षाधिग‚{\tiny $_{1}$}‚तो हेतुः} स्यात् । न ह्य‚प्र‚तीत‚स्य हेतुता । न च बुद्धेः प्र‚त्य‚क्ष‚{\tiny $_{lb}$}‚तेष्य‚ते त‚द्व्य‚तिरिक्तिश्च क‚श्चिदात्म‚ध‚र्मो न प्र‚त्य‚क्ष इति न स्याद् बुद्ध्य‚नुमानं (।) ‚{\tiny $_{lb}$}‚किञ्चात्म‚ध‚र्मोसौ बुद्ध्या स‚म‚मेक‚कार‚णो वा स्यात् भिन्न‚कार‚णो वा (।) त‚त्र ‚{\tiny $_{lb}$}‚बुद्ध्या स‚ह ‚{\color{DodgerBlue3}‚तुल्यात् कार‚णाज्ज‚न्म} य‚स्य ‚{\color{DodgerBlue3}‚त‚स्या}‚त्म‚ध‚र्म‚स्य ‚{\color{DodgerBlue3}‚बुद्धेः} स‚काशात् ‚{\tiny $_{lb}$}‚‚{\color{DodgerBlue3}‚कुतो भेदः} । अभिन्न‚हेतुक‚त्वेऽभिन्न‚तैव युक्ता । अथान्य‚हेतुकोसौ त‚दा‚{\color{DodgerBlue3}‚न्य‚ज‚श्च ‚{\tiny $_{lb}$}‚स व्य\edtext{}{\edlabel{pvv.257-3}\label{pvv.257-3}\lemma{व्य}\Bfootnote{बुद्ध्य‚सिद्धेर्न तादात्म्यं नापि त‚दुत्प‚त्तिर‚नुमेय‚धिया ।}}भिचारी} स्यात् । एक‚साम‚ग्र‚य‚धीन‚योरेक‚द‚र्श‚नाद‚प‚रानुमान‚म‚व्य‚भिचारि ‚{\tiny $_{lb}$}‚नान्य‚था ।\edtext{\textsuperscript{*}}{\edlabel{pvv.257-4}\label{pvv.257-4}\lemma{*}\Bfootnote{बुद्धेर्लिङ्गाद‚नुमाने त‚ल्लिङ्गेन कार्येण भाव्य‚मिति ।}}(४७०)
	\pend% ending standard par
      \label{div_pvv.2.471}
	  
	% new div opening: depth here is 2
	

	  \pstart \leavevmode% starting standard par
	उक्त‚मेवार्थं संगृह्ण‚न्नाह (।)
	\pend% ending standard par
      
	  \bigskip
	  \begingroup
	
	    \large
	  
	    \begin{quote}
	  
	    
	    \stanza[\smallbreak]
	\label{pv.2.471}\flagstanza{\tiny\textenglish{....2.471}}रूपादीन् प‚ञ्च‚विष‚यानिन्द्रियाण्युप‚ल‚म्भ‚न‚म् ।&मुक्त्त्वा न कार्य‚म‚प‚रं त‚स्याः स‚मुप‚ल‚भ्य‚ते ॥ ४७१ ॥\&[\smallbreak]


	
	    \end{quote}
	  
	  \endgroup
	

	  \pstart \leavevmode% starting standard par
	\hphantom{.}‚{\color{DodgerBlue3}‚रूप}‚मादिर्येषां तान् श‚ब्द‚ग‚न्ध‚र‚स‚स्प‚र्श‚न् ‚{\color{DodgerBlue3}‚प‚ञ्च‚विष‚यान्} प‚ञ्चे‚{\color{DodgerBlue3}‚न्द्रियाणि} च‚क्षुः‚{\tiny $_{lb}$}‚श्रोत्रादीनि ‚{\color{DodgerBlue3}‚उप‚ल‚म्भ‚नं} ज्ञानं ‚{\color{DodgerBlue3}‚मुक्त्वा त‚स्या} बुद्धेर्न ‚{\color{DodgerBlue3}‚कार्य‚म‚प‚रं स‚मुप‚ल‚भ्य‚ते ।} इय‚तैव ‚{\tiny $_{lb}$}‚स‚र्व्व‚स्य संग्र‚हात् । (४७१)
	\pend% ending standard par
      \textsuperscript{\textenglish{258/s}}\label{div_pvv.2.472}
	  
	% new div opening: depth here is 2
	
	  \bigskip
	  \begingroup
	
	    \large
	  
	    \begin{quote}
	  
	    
	    \stanza[\smallbreak]
	\label{pv.2.472}\flagstanza{\tiny\textenglish{....2.472}}त‚त्रात्य‚क्षं द्व‚यं । प‚ञ्च‚स्व‚र्थेष्वेकोपि नेक्ष्य‚ते ।&रूप‚द‚र्श‚न‚तो जातो योन्य‚था-व्य‚स्त‚स‚म्भ‚वः ॥ ४७२ ॥\&[\smallbreak]


	
	    \end{quote}
	  
	  \endgroup
	

	  \pstart \leavevmode% starting standard par
	\hphantom{.}‚{\color{DodgerBlue3}‚त‚त्र} तेषु म‚ध्ये ‚{\color{DodgerBlue3}‚द्व‚य}‚मिन्द्रियं ज्ञान‚ञ्चा‚{\color{DodgerBlue3}‚त्य‚क्ष}‚म‚तीन्द्रि‚{\tiny $_{3}$}‚यं इन्द्रिय‚स्य ज्ञानान्य‚थानुप‚{\tiny $_{lb}$}‚प‚त्त्या व्य‚व‚स्थाप‚नात् । ज्ञान‚स्य त्व‚न्म‚तेऽप्र‚त्य‚क्ष‚त्वात् । रूपादिषु ‚{\color{DodgerBlue3}‚प‚ञ्च‚स्व‚र्थेषु एकोपि ‚{\tiny $_{lb}$}‚नेक्ष्य‚ते} बुद्धेर‚प्र‚त्य‚क्ष‚त्वात् । ‚{\color{DodgerBlue3}‚यो या}‚व‚द् दृश्य‚मानोऽ‚{\color{DodgerBlue3}‚न्य‚था} ज्ञान‚म‚न्त‚रेण व्य‚{\color{DodgerBlue3}‚स्त‚स‚म्भ‚वः} प्र‚तिक्षिप्त‚स‚त्त्वो विष‚य‚स्य ‚{\color{DodgerBlue3}‚रूप‚द‚र्श‚न‚तो} बुद्धे‚{\color{DodgerBlue3}‚र्जातो}‚ऽभ्युप‚ग‚म्य‚ते । (४७२)
	\pend% ending standard par
      \label{div_pvv.2.473}
	  
	% new div opening: depth here is 2
	
	  \bigskip
	  \begingroup
	
	    \large
	  
	    \begin{quote}
	  
	    
	    \stanza[\smallbreak]
	\label{pv.2.473}\flagstanza{\tiny\textenglish{....2.473}}य‚देव‚म‚प्र‚तीतं त‚ल्लिङ्ग‚मित्य‚तिलौकिक‚म् ।&विद्य‚मानेपि लिङ्गे तान्तेन सार्द्ध‚म‚प‚श्य‚तः ॥ ४७३ ॥\&[\smallbreak]


	
	    \end{quote}
	  
	  \endgroup
	

	  \pstart \leavevmode% starting standard par
	\hphantom{.}‚{\color{DodgerBlue3}‚य‚च्चैवं} बुद्धिनान्त‚र‚यीक‚त‚या‚{\color{DodgerBlue3}‚ऽप्र‚तीतं} त‚द् बुद्धे‚{\color{DodgerBlue3}‚र्लिङ्ग‚मि}‚त्य‚ति‚{\color{DodgerBlue3}‚लौकिकं} लो \edtext{}{\edlabel{pvv.258-1}\label{pvv.258-1}\lemma{लो}\Bfootnote{बालोपि स‚म्ब‚द्ध‚मेव लिङ्ग‚माह व‚ह्नेरिव धूमः ।}} काति‚{\tiny $_{4}$}‚‚{\tiny $_{lb}$}‚क्रान्तं । किञ्चाभ्युप‚ग‚म्योच्य‚ते । ‚{\color{DodgerBlue3}‚विद्य‚मानेपि} क‚स्मिंश्चिल्लि‚{\color{DodgerBlue3}‚ङ्गे} क‚दाचिद् बुद्धिं ‚{\color{DodgerBlue3}‚तां ‚{\tiny $_{lb}$}‚तेन} लि\edtext{}{\edlabel{pvv.258-2}\label{pvv.258-2}\lemma{लि}\Bfootnote{आत्म‚नि बुद्धेर‚न्व‚यादृष्टेः ।}}ङ्गेन ‚{\color{DodgerBlue3}‚सार्ध‚म‚प‚श्य‚तो}‚ऽप्र‚तिप‚त्तेः । (४७३)
	\pend% ending standard par
      \label{div_pvv.2.474}
	  
	% new div opening: depth here is 2
	
	  \bigskip
	  \begingroup
	
	    \large
	  
	    \begin{quote}
	  
	    
	    \stanza[\smallbreak]
	\label{pv.2.474}\flagstanza{\tiny\textenglish{....2.474}}क‚थं प्र‚तीतिर्लिङ्गं हि नादृष्ट‚स्य प्र‚काश‚क‚म् ।&त‚त एवास्य लिङ्गात्प्राक् प्र‚सिद्धेरुप‚व‚र्ण्ण‚ने ॥ ४७४ ॥\&[\smallbreak]


	
	    \end{quote}
	  
	  \endgroup
	

	  \pstart \leavevmode% starting standard par
	\hphantom{.}त‚स्माल्लिङ्गात् ‚{\color{DodgerBlue3}‚क‚थं} बुद्धि‚{\color{DodgerBlue3}‚प्र‚तीतिः । लिङ्गं ह्य‚न्व}‚य‚र‚हित‚म‚{\color{DodgerBlue3}‚दृष्ट}‚स्यार्थ‚स्य ‚{\color{DodgerBlue3}‚न प्र‚का‚{\tiny $_{lb}$}‚श}‚कं यु\edtext{}{\edlabel{pvv.258-3}\label{pvv.258-3}\lemma{यु}\Bfootnote{स्यादेत‚द् (।) येन काय‚स्प‚न्दादिनात्म‚नि बुर्द्धि साध‚यितुमिच्छ‚ति प्र‚माता त‚त ‚{\tiny $_{lb}$}‚एव लिङ्गात् स‚प‚क्षे बृद्धिः सेत्स्य‚ति त‚या सिद्ध्यात्म‚न्य‚नुमान‚मिति स‚प‚क्षेऽनुमानं ‚{\tiny $_{lb}$}‚विना दृष्टान्त‚ब‚लेनेत्याह ।}}क्तं (।) ‚{\color{DodgerBlue3}‚त‚त एव लिङ्गाद‚स्य} ज्ञान‚स्यान्व‚य‚सिद्ध्य‚र्थ‚मात्म‚न्य‚नुमाना‚{\color{DodgerBlue3}‚त्प्राक् ‚{\tiny $_{lb}$}‚सिद्धेर्निश्च‚य‚स्योप‚व‚र्ण्ण‚ने} वाभिधीय‚माने (४७४)
	\pend% ending standard par
      \label{div_pvv.2.475}
	  
	% new div opening: depth here is 2
	
	  \bigskip
	  \begingroup
	
	    \large
	  
	    \begin{quote}
	  
	    
	    \stanza[\smallbreak]
	\label{pv.2.475}\flagstanza{\tiny\textenglish{....2.475}}दृष्टान्तान्त‚र‚साध्य‚त्वं त‚स्यापीत्य‚न‚व‚स्थितिः ।&इत्य‚र्थ‚स्य धियः सिद्धिः नार्थात्त‚स्याः क‚थ‚ञ्च‚न ॥ ४७५ ॥\&[\smallbreak]


	
	    \end{quote}
	  
	  \endgroup
	

	  \pstart \leavevmode% starting standard par
	\hphantom{.}‚{\color{DodgerBlue3}‚त‚स्या}‚न्व‚य‚साध‚क‚स्याप्य‚नुमान‚स्य ‚{\color{DodgerBlue3}‚दृष्टा‚{\tiny $_{5}$}‚न्तान्त‚रे}‚णानुमान‚साध्येन ‚{\color{DodgerBlue3}‚साध्य‚त्व\edtext{}{\edlabel{pvv.258-4}\label{pvv.258-4}\lemma{त्व}\Bfootnote{अन्य‚था स‚प‚क्षे य‚दि दृष्टान्तं विना सिद्धिरात्म‚न्य‚पि किन्न सिद्धिः ।}}‚{\tiny $_{lb}$}‚मित्य‚न‚व‚स्थितिः} स्यात् । त‚था चैक‚स्यासिद्धौ स‚र्व्व‚स्यासिद्धिः प्र‚स‚ज्य‚ते ।
	\pend% ending standard par
      

	  \pstart \leavevmode% starting standard par
	\hphantom{.}‚{\color{DodgerBlue3}‚इति} त‚स्मा‚{\color{DodgerBlue3}‚द‚र्थ‚स्य धियः} स‚काशात् ‚{\color{DodgerBlue3}‚सिद्धिर्नार्थात् त‚स्या} धियः ‚{\color{DodgerBlue3}‚क‚थं च न} सिद्धि‚{\tiny $_{lb}$}‚रिति न्याय्यं (। ४७५)
	\pend% ending standard par
      \label{div_pvv.2.476}
	  
	% new div opening: depth here is 2
	
	  \bigskip
	  \begingroup
	
	    \large
	  
	    \begin{quote}
	  
	    
	    \stanza[\smallbreak]
	\label{pv.2.476}\flagstanza{\tiny\textenglish{....2.476}}त‚द‚प्र‚सिद्धाव‚र्थ‚स्य स्व‚य‚मेवाप्र‚सिद्धितः ।&प्र‚त्य‚क्षाञ्च धियं दृष्ट्वा त‚स्याश्चेष्टाभिधादिक‚म् ॥ ४७६ ॥\&[\smallbreak]


	
	    \end{quote}
	  
	  \endgroup
	\textsuperscript{\textenglish{259/s}}

	  \pstart \leavevmode% starting standard par
	\hphantom{.}य‚स्मात्त‚स्या धियो‚{\color{DodgerBlue3}‚ऽसिद्धाव‚र्थ‚स्य स्व‚य‚मेवाप्र‚सिद्धितः} क‚थं लिङ्ग‚ता ।
	\pend% ending standard par
      

	  \pstart \leavevmode% starting standard par
	\hphantom{.}स्व‚प्र‚काश‚त्वात् ‚{\color{DodgerBlue3}‚प्र‚त्य‚क्षां धियं त‚स्याश्च चेष्टाऽभिधाऽदिर्य‚स्य} सुख‚प्र‚साद‚{\tiny $_{lb}$}‚वैव‚र्ण्ण्यादे‚{\tiny $_{6}$}‚स्तं ‚{\color{DodgerBlue3}‚दृष्ट्वा} गृहीत‚व्याप्तिक‚स्यान्य‚स‚म्ब‚न्धिचेष्टादिद‚र्श‚नात् । (४७६)
	\pend% ending standard par
      \label{div_pvv.2.477}
	  
	% new div opening: depth here is 2
	
	  \bigskip
	  \begingroup
	
	    \large
	  
	    \begin{quote}
	  
	    
	    \stanza[\smallbreak]
	\label{pv.2.477}\flagstanza{\tiny\textenglish{....2.477}}प‚र‚चित्तानुमान‚ञ्च न स्यादात्म‚न्य‚द‚र्श‚नात् ।&स‚ब‚न्ध‚स्य म‚नोबुद्धाव‚र्थ‚लिङ्गाप्र‚सिद्धितः ॥ ४७७ ॥\&[\smallbreak]


	
	    \end{quote}
	  
	  \endgroup
	

	  \pstart \leavevmode% starting standard par
	\hphantom{.}‚{\color{DodgerBlue3}‚प‚र‚चित्तानुमान‚ञ्चेष्टं न स्यात्} । बुद्धेरा‚{\color{DodgerBlue3}‚त्म‚नि} स्व‚स‚न्त‚तौ चेष्टादिभिः स‚ह ‚{\color{DodgerBlue3}‚स‚म्ब‚न्ध}‚{\tiny $_{lb}$}‚स्या‚{\color{DodgerBlue3}‚द‚र्श‚नात्} । अपि च वास‚नामात्र‚ब‚ल‚भाविन्या ‚{\color{DodgerBlue3}‚म‚नोबृद्धौ} विक‚ल्प‚बुद्धौ विष‚य‚{\tiny $_{lb}$}‚भूत‚स्यार्थ‚स्याभावात्\edtext{}{\edlabel{pvv.259-1}\label{pvv.259-1}\lemma{स्याभावात्}\Bfootnote{न विक‚ल्पोऽर्थापेक्षः ।}} ‚{\color{DodgerBlue3}‚अर्थ‚स्य लिङ्ग‚स्यासिद्धितो}\edtext{\textsuperscript{*}}{\edlabel{pvv.259-2}\label{pvv.259-2}\lemma{*}\Bfootnote{अनुमानं न स्यात् । ज्ञानान्त‚र‚वेद्य‚प‚क्षेपि ।}} बुद्ध्य‚न्त‚राल्लिङ्गाद‚नुमानं ‚{\tiny $_{lb}$}‚स्यात् ।\edtext{\textsuperscript{*}}{\edlabel{pvv.259-3}\label{pvv.259-3}\lemma{*}\Bfootnote{त‚त्रापि}}(४७७)
	\pend% ending standard par
      \label{div_pvv.2.478}
	  
	% new div opening: depth here is 2
	
	  \bigskip
	  \begingroup
	
	    \large
	  
	    \begin{quote}
	  
	    
	    \stanza[\smallbreak]
	\label{pv.2.478}\flagstanza{\tiny\textenglish{....2.478}}प्र‚काशिता क‚थं वा स्यात् बुद्धिर्बुद्ध्य‚न्त‚रेण वः ।&अप्र‚काशात्म‚नोः साम्याद् व्य‚ङ्ग्य‚व्य‚ञ्ज‚क‚ता कुतः ॥ ४७८ ॥\&[\smallbreak]


	
	    \end{quote}
	  
	  \endgroup
	

	  \pstart \leavevmode% starting standard par
	\hphantom{.}‚{\color{DodgerBlue3}‚क‚थ‚म्वा बुद्ध्य‚न्त‚रेणा}‚प्र‚त्य‚क्षेण ‚{\color{DodgerBlue3}‚बुद्धिः प्र‚काशिता स्या‚{\tiny $_{7}$}‚त् । वो} युष्माकं द‚र्श‚ने\edtext{}{\edlabel{pvv.259-4}\label{pvv.259-4}\lemma{ने}\Bfootnote{पूर्व्वात्म‚रूप‚योत्त‚र‚बुद्ध्येति चेदाह । एक‚स्य विक‚ल्पाविक‚ल्प‚ज‚न‚न‚विरो‚{\tiny $_{lb}$}‚धात् पूर्व्व‚धिया प‚र‚धीबोध‚ज‚न‚ने स्मृतेर‚ज‚न‚नात्तु द्वितीय‚त‚या धिया स्व‚विष‚या‚{\tiny $_{lb}$}‚(व) बोधाद्य‚व‚धानं ।}} ।\leavevmode\ledsidenote{\textenglish{51a/MA}} ‚{\tiny $_{lb}$}‚य‚स्माद‚{\color{DodgerBlue3}‚प्र‚काशात्म‚नो}‚र्ल्लिङ्ग‚लिङ्गिनोर‚सिद्ध‚त्वेन ‚{\color{DodgerBlue3}‚साम्यात् व्य‚ङ्ग्य‚व्य‚ञ्ज‚क‚ता कुतः} । ‚{\tiny $_{lb}$}‚य‚द्य‚प्र‚काशात्म‚नोर्न व्य‚ङ्ग्य‚व्य‚ञ्ज‚क‚ता त‚दार्थ‚ज्ञान‚योर‚पि क‚थं व्य‚ङ्ग्य‚व्य‚ञ्ज‚क‚ता‚{\tiny $_{lb}$}‚भाव इति । (४७८)
	\pend% ending standard par
      \label{div_pvv.2.479}
	  
	% new div opening: depth here is 2
	
	  \bigskip
	  \begingroup
	
	    \large
	  
	    \begin{quote}
	  
	    
	    \stanza[\smallbreak]
	\label{pv.2.479a}\flagstanza{\tiny\textenglish{...2.479a}}विष‚य‚स्य क‚थं व्य‚क्तिः;\&[\smallbreak]


	
	    \end{quote}
	  
	  \endgroup
	

	  \pstart \leavevmode% starting standard par
	\hphantom{.}‚{\color{DodgerBlue3}‚विष‚य‚स्य क‚थं व्य‚क्ति}‚रिति (।)
	\pend% ending standard par
      

	  \pstart \leavevmode% starting standard par
	उत्त‚र‚माह (।)
	\pend% ending standard par
      
	  \bigskip
	  \begingroup
	
	    \large
	  
	    \begin{quote}
	  
	    
	    \stanza[\smallbreak]
	\label{pv.2.479b}\flagstanza{\tiny\textenglish{...2.479b}}प्र‚काशे रूप‚संक्र‚मात् ।&स च प्र‚काश‚स्त‚द्रूपः स्व‚य‚मेव प्र‚काश‚ते ॥ ४७९ ॥\&[\smallbreak]


	
	    \end{quote}
	  
	  \endgroup
	

	  \pstart \leavevmode% starting standard par
	\hphantom{.}‚{\color{DodgerBlue3}‚प्र‚काशे} स्व‚स‚म्विदिते ज्ञाने विष‚य‚स्य ‚{\color{DodgerBlue3}‚रूप‚संक्र‚मात्} सारूप्य‚संभ‚वात् ज्ञानेनार्थ‚{\tiny $_{lb}$}‚प्र‚काशित इत्युच्य‚ते । ‚{\color{DodgerBlue3}‚स च प्र‚काश‚स्त‚द्रूपो} विष‚य‚स्व‚रुपः ‚{\color{DodgerBlue3}‚स्व‚य‚मेवा}‚{\tiny $_{1}$}‚प‚रोक्ष‚प्र‚काशा‚{\tiny $_{lb}$}‚त्म‚नोत्प‚न्नः ‚{\color{DodgerBlue3}‚प्र‚काश‚ते} न त्व‚न्येन प्र‚काश्य‚ते । (४७९)
	\pend% ending standard par
      \label{div_pvv.2.480}
	  
	% new div opening: depth here is 2
	\textsuperscript{\textenglish{260/s}}

	  \pstart \leavevmode% starting standard par
	स्यादेत‚द् (।) बुद्धिर‚पि बुद्ध्य‚न्त‚र‚स‚रूपोत्प‚न्ना प्र‚काश‚माना बुद्धेर्व्य‚ञ्जिका ‚{\tiny $_{lb}$}‚म‚तेति चेत् । आह (।)
	\pend% ending standard par
      
	  \bigskip
	  \begingroup
	
	    \large
	  
	    \begin{quote}
	  
	    
	    \stanza[\smallbreak]
	\label{pv.2.480}\flagstanza{\tiny\textenglish{....2.480}}त‚थाभ्युप‚ग‚मे बुद्धेः बुद्धौ बुद्धिः स्व‚वेदिका ।&सिद्धान्य‚था तुल्य‚ध‚र्मा विष‚योपि धिया स‚ह ॥ ४८० ॥\&[\smallbreak]


	
	    \end{quote}
	  
	  \endgroup
	

	  \pstart \leavevmode% starting standard par
	\hphantom{.}‚{\color{DodgerBlue3}‚बुद्धेः} प्र‚काश्यायाः ‚{\color{DodgerBlue3}‚बुद्धौ} व्य‚ञ्जिकायां ‚{\color{DodgerBlue3}‚त‚था} संक्रान्त‚सारूप्य‚प्र‚काश‚द्वारेण ‚{\tiny $_{lb}$}‚‚{\color{DodgerBlue3}‚वेद‚नाभ्युप‚ग‚मे} व्य‚ञ्जिका ‚{\color{DodgerBlue3}‚बुद्धिः स्व‚संवेदिका सिद्धा} (।) धीस‚रूपाया ‚{\color{DodgerBlue3}‚बुद्धेः} स्व‚प्र‚काश‚त्वे पूर्व्व‚बुद्धिः प्र‚काशिता स्यात् । ‚{\color{DodgerBlue3}‚अन्य‚था}‚{\tiny $_{2}$}‚ स्व‚प्र‚काश‚त्वान‚भ्युप‚ग‚मे ‚{\tiny $_{lb}$}‚‚{\color{DodgerBlue3}‚विष‚यो}‚प्य‚प्र‚काश‚स्व‚भाव‚त‚या ‚{\color{DodgerBlue3}‚धिया स‚ह तुल्य‚ध‚र्मेति} सोपि बुद्धेर्व्य‚ञ्ज‚कः स्यात् । ‚{\tiny $_{lb}$}‚स‚रूप‚योर्धीविष‚य‚योर‚न्योन्यं व्य‚ञ्ज‚क‚ता भ‚वेत् । (४८०)
	\pend% ending standard par
      \label{div_pvv.2.481}
	  
	% new div opening: depth here is 2
	
	  \bigskip
	  \begingroup
	
	    \large
	  
	    \begin{quote}
	  
	    
	    \stanza[\smallbreak]
	\label{pv.2.481}\flagstanza{\tiny\textenglish{....2.481}}इति प्र‚काश‚रूपा नः स्व‚यं धीः संप्र‚काश‚ते ।&अन्योस्यां रूप‚संक्रान्त्या प्र‚काशः स‚न् प्र‚काश‚ते ॥ ४८१ ॥\&[\smallbreak]


	
	    \end{quote}
	  
	  \endgroup
	

	  \pstart \leavevmode% starting standard par
	\hphantom{.}‚{\color{DodgerBlue3}‚इति} त‚स्मान्नोऽस्माकं म‚ते ‚{\color{DodgerBlue3}‚धीः स्व‚य}‚मात्म‚ना ‚{\color{DodgerBlue3}‚प्र‚काश‚रूपो}‚त्प‚न्ना स‚ती ‚{\color{DodgerBlue3}‚प्र‚काश‚ते} । ‚{\tiny $_{lb}$}‚न त्व‚न्येन प्र‚काश्य‚ते इति युक्तं । अन्यः पुन‚र‚{\color{DodgerBlue3}‚न्योस्यां} बुद्धौ प्र‚काशायां ‚{\color{DodgerBlue3}‚रूप‚सं}\edtext{}{\edlabel{pvv.260-1}\label{pvv.260-1}\lemma{काशायां}\Bfootnote{स्फ‚टिक‚म‚णाविव ज‚वा (कु) सुमं ।}}क्रान्त्या ‚{\tiny $_{lb}$}‚‚{\color{DodgerBlue3}‚प्र‚काशः स‚न् प्र‚काश‚ते} । त‚तोऽर्थ‚वेद‚न‚व्य‚व‚हारः । (४८१)
	\pend% ending standard par
      \label{div_pvv.2.482}
	  
	% new div opening: depth here is 2
	
	  \bigskip
	  \begingroup
	
	    \large
	  
	    \begin{quote}
	  
	    
	    \stanza[\smallbreak]
	\label{pv.2.482a}\flagstanza{\tiny\textenglish{...2.482a}}सादृश्येपि हि धीर‚न्या प्र‚काश्या न त‚या म‚ता ।&स्व‚यं प्र‚काश‚माना ;\&[\smallbreak]


	
	    \end{quote}
	  
	  \endgroup
	

	  \pstart \leavevmode% starting standard par
	\hphantom{.}‚{\color{DodgerBlue3}‚सादृश्ये} सारूप्येपि स‚ति ‚{\color{DodgerBlue3}‚त‚या} स‚रूप‚या धियाऽ‚{\color{DodgerBlue3}‚न्या} पूर्व्विका ‚{\color{DodgerBlue3}‚धीर्न प्र‚काश्या ‚{\tiny $_{lb}$}‚म‚ता} प्र‚काश‚स्व‚भाव‚स्य प‚रेण प्र‚काशायोगात् । किन्तु ‚{\color{DodgerBlue3}‚स्व‚यं प्र‚काश}‚स्व‚भाव‚त‚या ‚{\tiny $_{lb}$}‚प्र‚काश‚माना प्र‚काश‚त इत्य‚भ्युपेयं ।
	\pend% ending standard par
      

	  \begin{center}%% label @type='head'
	\textbf{घ. अर्थ‚स्य ज्ञान‚रूपेण प्र‚काश‚क‚ता}
	\end{center}
	

	  \pstart \leavevmode% starting standard par
	एव‚न्त‚र्ह्य‚र्थ‚स्याप्र‚काशात्म‚नः क‚थं प्र‚काश‚त इत्याह (।)
	\pend% ending standard par
      
	  \bigskip
	  \begingroup
	
	    \large
	  
	    \begin{quote}
	  
	    
	    \stanza[\smallbreak]
	\label{pv.2.482b}\flagstanza{\tiny\textenglish{...2.482b}}अर्थ‚स्त‚द्रूपेण प्र‚काश‚ते ॥ ४८२ ॥\&[\smallbreak]


	
	    \end{quote}
	  
	  \endgroup
	

	  \pstart \leavevmode% starting standard par
	\hphantom{.}‚{\color{DodgerBlue3}‚अर्थः} सारूप्य‚संक्रान्ते‚{\color{DodgerBlue3}‚स्त‚द्रूपे}‚ण ज्ञान‚रूपेण ‚{\color{DodgerBlue3}‚प्र‚काश‚ते} न तु साक्षात् ‚{\tiny $_{4}$}‚स्व‚रूपेण । ‚{\tiny $_{lb}$}‚(४८२)
	\pend% ending standard par
      \label{div_pvv.2.483}
	  
	% new div opening: depth here is 2
	

	  \pstart \leavevmode% starting standard par
	दृष्टान्त‚माह (।)
	\pend% ending standard par
      
	  \bigskip
	  \begingroup
	
	    \large
	  
	    \begin{quote}
	  
	    
	    \stanza[\smallbreak]
	\label{pv.2.483}\flagstanza{\tiny\textenglish{....2.483}}य‚था प्र‚दीप‚योर्द्दीप‚घ‚ट‚योश्च त‚दाश्र‚यः ।&व्य‚ङ्ग्य‚व्य‚ञ्ज‚क‚भेदेन व्य‚व‚हारः प्र‚त‚न्य‚ते ॥ ४८३ ॥\&[\smallbreak]


	
	    \end{quote}
	  
	  \endgroup
	\textsuperscript{\textenglish{261/s}}

	  \pstart \leavevmode% starting standard par
	\hphantom{.}‚{\color{DodgerBlue3}‚य‚था प्र‚दीप‚योः} प्र‚काशात्म‚नोर्न प्र‚क‚श्य‚प्र‚काश‚क‚भावः । त‚था बुद्ध्योर‚पि । ‚{\tiny $_{lb}$}‚य‚था ‚{\color{DodgerBlue3}‚दीप‚घ‚ट‚योः} प्र‚काशाप्र‚काश‚स्व‚भाव‚योरेकः प्र‚काश‚कोऽन्यः प्र‚काश्यः । त‚था ‚{\tiny $_{lb}$}‚ज्ञानार्थ‚योर‚पि । ‚{\color{DodgerBlue3}‚त‚दाश्र‚यो}‚ऽप्र‚काश‚प्र‚काशात्म‚निष्ठो ‚{\color{DodgerBlue3}‚व्य‚ङ्ग्य‚व्य‚ञ्ज‚क‚भेदेन व्य‚व‚हारो} लोके ‚{\color{DodgerBlue3}‚प्र‚त‚न्य}‚ते (। ४८३)
	\pend% ending standard par
      \label{div_pvv.2.484}
	  
	% new div opening: depth here is 2
	
	  \bigskip
	  \begingroup
	
	    \large
	  
	    \begin{quote}
	  
	    
	    \stanza[\smallbreak]
	\label{pv.2.484}\flagstanza{\tiny\textenglish{....2.484}}विष‚येन्द्रिय‚मात्रेण न दृष्ट‚मिति निश्च‚यः ।&त‚स्माद्य‚तोयं त‚स्यापि वाच्य‚म‚न्य‚स्य द‚र्श‚न‚म् ॥ ४८४ ॥\&[\smallbreak]


	
	    \end{quote}
	  
	  \endgroup
	

	  \pstart \leavevmode% starting standard par
	\hphantom{.}य‚त‚श्च ‚{\color{DodgerBlue3}‚विष‚य}‚मात्रेण ‚{\color{DodgerBlue3}‚इन्द्रिय‚मात्रेण} वे\edtext{}{\edlabel{pvv.261-1}\label{pvv.261-1}\lemma{वे}\Bfootnote{नैयायिक‚जैमिनीयादेर्बुद्धिप‚रोक्ष‚त्वात् न स्यादेव ।}} दं ‚{\color{DodgerBlue3}‚दृष्ट‚मिति न निश्य‚स्त}‚{\tiny $_{5}$}‚ स्माद्य‚त‚स्त‚द्‚{\tiny $_{lb}$}‚व्य‚तिरिक्ताज्ज्ञानादिदं दृष्ट‚मिति निश्च‚य‚स्त‚स्यापि विष‚येन्द्रियाभ्या‚{\color{DodgerBlue3}‚म‚न्य‚स्य} ज्ञान‚स्य ‚{\tiny $_{lb}$}‚‚{\color{DodgerBlue3}‚द‚र्श‚न‚म}‚प‚रोक्ष‚त्वं ‚{\color{DodgerBlue3}‚वाच्य}‚मिति बुद्धिप‚रोक्ष‚तावादो न युक्तः । (४८४)
	\pend% ending standard par
      \label{div_pvv.2.485}
	  
	% new div opening: depth here is 2
	

	  \begin{center}%% label @type='head'
	\textbf{(३) व. स्व‚संवित्तिसिद्धिः}
	\end{center}
	

	  \begin{center}%% label @type='head'
	\textbf{क. स्मृतेः स्व‚संवित्तिः}
	\end{center}
	

	  \pstart \leavevmode% starting standard par
	स्व\edtext{}{\edlabel{pvv.261-2}\label{pvv.261-2}\lemma{स्व}\Bfootnote{स्वोप‚प‚त्तिभिः स्व‚वेदं प्र‚साध्याचार्योप‚प‚त्तिमाह ।}}संवित्तिसिद्ध्य‚र्थ‚मुप‚प‚त्त्य‚न्त‚र‚माह (।)
	\pend% ending standard par
      
	  \bigskip
	  \begingroup
	
	    \large
	  
	    \begin{quote}
	  
	    
	    \stanza[\smallbreak]
	\label{pv.2.485a}\flagstanza{\tiny\textenglish{...2.485a}}स्मृतेर‚प्यात्म‚वित्सिद्धा ज्ञान‚स्य;\&[\smallbreak]


	
	    \end{quote}
	  
	  \endgroup
	

	  \pstart \leavevmode% starting standard par
	\hphantom{.}ज्ञान‚स्यातीत‚स्य ‚{\color{DodgerBlue3}‚स्मृतेर‚प्यात्म‚वित्} सुसंवित्तिः ‚{\color{DodgerBlue3}‚सिद्धा} । प्र‚तीत‚मेव हि स्म‚र्य‚ते ‚{\tiny $_{lb}$}‚य‚थार्थः ।\edtext{\textsuperscript{*}}{\edlabel{pvv.261-3}\label{pvv.261-3}\lemma{*}\Bfootnote{स्म‚र्य‚ते च बुद्धिः ।}}
	\pend% ending standard par
      

	  \pstart \leavevmode% starting standard par
	स्यादेत‚त् ज्ञानं प्र‚तीत‚म‚न्येन चेत‚सा न स्व‚संवेद‚नेनेत्याह (।)
	\pend% ending standard par
      
	  \bigskip
	  \begingroup
	
	    \large
	  
	    \begin{quote}
	  
	    
	    \stanza[\smallbreak]
	\label{pv.2.485b}\flagstanza{\tiny\textenglish{...2.485b}}अन्येन वेद‚ने ।&दीर्घादिग्र‚ह‚ण‚न्न स्याद् ब‚हुमात्रान‚व‚स्थितेः ॥ ४८५ ॥\&[\smallbreak]


	
	    \end{quote}
	  
	  \endgroup
	

	  \pstart \leavevmode% starting standard par
	अन्येन‚{\tiny $_{6}$}‚ ज्ञानेन पूर्व्व‚क‚स्य ज्ञान‚स्य\edtext{}{\edlabel{pvv.261-4}\label{pvv.261-4}\lemma{स्य}\Bfootnote{प‚रोक्तं सिद्ध‚साध‚न‚त्वं स्व‚यं प‚रिह (र)ति ज्ञानान्त‚रेणानुभ‚वेऽनिष्टा ।}} ‚{\color{DodgerBlue3}‚वेद‚ने}‚ऽभिधीय‚माने \edtext{}{\edlabel{pvv.261-5}\label{pvv.261-5}\lemma{माने}\Bfootnote{त‚त्रापि हि स्मृतिविष‚यान्त‚र‚स‚ञ्चार‚स्त‚था न स्यात् स चेक्ष‚ते इत्याद्याचार्य‚{\tiny $_{lb}$}‚सिद्धान्तं मुक्त्वाधिक‚दोषाभिधानाय ।}} ‚{\color{DodgerBlue3}‚दीर्घ\edtext{}{\edlabel{pvv.261-6}\label{pvv.261-6}\lemma{दीर्घ}\Bfootnote{ह्र‚स्व‚प्लुतादेः ।}}देः} स्व‚र‚स्य ‚{\tiny $_{lb}$}‚\leavevmode\ledsidenote{\textenglish{262/s}} ‚{\color{DodgerBlue3}‚ग्र‚ह‚णं न स्यात्} । क्ष‚णिक‚स्य ज्ञान‚स्य एका\edtext{}{\edlabel{pvv.262-1}\label{pvv.262-1}\lemma{एका}\Bfootnote{याव‚ता कालेन प‚र‚माणुरिष्टः प‚र‚माण्व‚न्त‚र‚मेक‚म‚तिक्राम‚ति ताव‚त्कालः ‚{\tiny $_{lb}$}‚क्ष‚णः (।) क्ष‚णिका श्रोत्र‚धीः स‚र्व्वेषाम‚नेक‚क्ष‚णात्म‚क‚मेक‚व‚र्प्ण‚निष्प‚त्तिकालं न ‚{\tiny $_{lb}$}‚तिष्ठ‚ति ।}}ण्व‚त्य‚य‚काल‚मात्र‚स्थायि\edtext{}{\edlabel{pvv.262-2}\label{pvv.262-2}\lemma{स्थायि}\Bfootnote{एक‚व‚र्ण‚भाग‚ग्राहिबुद्धौ न‚ष्टायाम‚न‚न्त‚र‚न्त‚द‚नुभ‚व‚बुद्धिरिति व्य‚व‚हितं त‚द‚{\tiny $_{lb}$}‚प‚र‚व‚र्ण्ण‚भाग‚ज्ञानं त‚द्धिया त‚द‚ज्ञानात् दीर्घ‚ग्र‚हो न स्यात् स्मृतिर‚पि न विक‚ल्पावि‚{\tiny $_{lb}$}‚क‚ल्प‚योरेकेन ज‚न‚न‚विरोधात् ।}}नोऽ\edtext{}{\edlabel{pvv.262-3}\label{pvv.262-3}\lemma{नोऽ}\Bfootnote{स‚र्व्वेषु व‚र्ण्ण‚भागेषु बुद्ध‚यः स्व‚संविदिता इति स्मृतिसंक‚लं स्यात् न च ‚{\tiny $_{lb}$}‚स्व‚वेद‚नं म‚न्य‚ते ।}}नेक‚क्ष‚ण‚{\tiny $_{lb}$}‚निर्व्व‚र्त्त्यासु ब‚ह्वीषु ‚{\color{DodgerBlue3}‚मात्रासु} दीर्घादिनिर्व्व‚र्त्त‚निकासु ग्राह‚क‚त्वेना‚{\color{DodgerBlue3}‚न‚व‚स्थितेः} । ‚{\tiny $_{lb}$}‚न ह्येक‚क्ष‚ण‚मात्र‚स्थापि ज्ञान‚म‚नेक‚क्ष‚ण‚क‚लाप‚निर्व्व‚र्त‚नीय‚मात्रास‚ञ्च‚यात्म‚कं दीर्घा‚{\tiny $_{lb}$}‚\leavevmode\ledsidenote{\textenglish{51b/MA}} दिकं श‚क्नोति ग्र‚हीतुं । किन्त्व‚काराद्येकैक‚मा‚{\tiny $_{7}$}‚त्राव‚य‚व‚लेशं गृह्णाति । त‚द्‏ग्राह‚{\tiny $_{lb}$}‚क‚त्वेन द्वितीय‚ज्ञानेन त‚त् प्र‚तीय‚ते एव‚म‚प‚राप‚रैर्ज्ञानैर‚व‚य‚व‚लेश‚ग्राह‚कैः स्व‚स्व\edtext{}{\edlabel{pvv.262-4}\label{pvv.262-4}\lemma{स्व}\Bfootnote{पूर्व्व‚पूर्व्वेण ।}}‚{\tiny $_{lb}$}‚ग्राह‚क‚ज्ञानान्त‚रितैर्ग्र‚ह‚ण‚क्र‚मे केन मात्राग्र‚ह‚णं । मात्राप्र‚च‚य‚दीर्घादिग्र‚ह‚णं वा ‚{\tiny $_{lb}$}‚स्यात् । (४८५)
	\pend% ending standard par
      \label{div_pvv.2.486}
	  
	% new div opening: depth here is 2
	
	  \bigskip
	  \begingroup
	
	    \large
	  
	    \begin{quote}
	  
	    
	    \stanza[\smallbreak]
	\label{pv.2.486a}\flagstanza{\tiny\textenglish{...2.486a}}अव‚स्थिताव‚क्र‚मायां स‚कृदाभास‚नान्म‚तौ ।&व‚र्ण्णाः स्याद‚क्र‚मोऽदीर्घः;\&[\smallbreak]


	
	    \end{quote}
	  
	  \endgroup
	

	  \pstart \leavevmode% starting standard par
	अथा\edtext{}{\edlabel{pvv.262-5}\label{pvv.262-5}\lemma{अथा}\Bfootnote{आस‚र्ग‚प्र‚ल‚यादेरेकैव बुद्धिः सांख्य‚स्य त‚माह तां ।}}नेक‚मात्राकाल‚मेकैव बुद्धिर‚स्तीत्युच्य‚ते त‚दानेक‚काल‚{\color{DodgerBlue3}‚म‚व‚स्थितौ} स‚त्या‚{\tiny $_{lb}$}‚‚{\color{DodgerBlue3}‚म‚क्र‚मायां म‚तौ} स‚र्व्व‚मात्राणां ‚{\color{DodgerBlue3}‚स‚कृदाभास‚नाद्दीर्घादिर्व्व‚र्ण्णोऽक्र‚मः स्यात्} ।‚{\tiny $_{1}$}‚ प्र‚तीति‚{\tiny $_{lb}$}‚निब‚न्ध‚न‚त्वाद्व‚स्तुव्य‚व‚स्थायाः । त‚था चादीर्घो \edtext{}{\edlabel{pvv.262-6}\label{pvv.262-6}\lemma{चादीर्घो}\Bfootnote{अप‚राप‚र‚बुद्धाव‚प‚राप‚र‚मात्राभाग‚प्र‚तिभास‚क्र‚मेण दीर्घ‚भानं य‚तः ।}}भ‚वेत् । न ह्येक‚काल‚मुच्चैरुच्चार्य‚{\tiny $_{lb}$}‚माणोप्येक‚मात्रिको दीर्घः ।
	\pend% ending standard par
      

	  \pstart \leavevmode% starting standard par
	न‚नु क्र‚म‚व‚न्तो व‚र्ण्णा अव‚य‚व‚क्र‚मेणोत्प‚द्य‚माना दीर्घादिबुद्धिमुत्पाद‚यिष्य‚न्ती‚{\tiny $_{lb}$}‚त्याह (।)
	\pend% ending standard par
      
	  \bigskip
	  \begingroup
	
	    \large
	  
	    \begin{quote}
	  
	    
	    \stanza[\smallbreak]
	\label{pv.2.486b}\flagstanza{\tiny\textenglish{...2.486b}}क्र‚म‚वान‚क्र‚मां क‚थ‚म् ॥ ४८६ ॥\&[\smallbreak]


	
	    \end{quote}
	  
	  \endgroup
	
	  \bigskip
	  \begingroup
	
	    \large
	  
	    \begin{quote}
	  
	    
	    \stanza[\smallbreak]
	\label{pv.2.487a}\flagstanza{\tiny\textenglish{...2.487a}}उप‚कुर्याद‚संश्लिष्य‚न्व‚र्ण्ण‚भागः प‚र‚स्प‚र‚म् (।)\&[\smallbreak]


	
	    \end{quote}
	  
	  \endgroup
	\textsuperscript{\textenglish{263/s}}

	  \pstart \leavevmode% starting standard par
	\hphantom{.}‚{\color{DodgerBlue3}‚क्र‚म‚वान् व‚र्ण्ण‚भागः} स्व‚स्व‚काल‚स्थायी ‚{\color{DodgerBlue3}‚प\edtext{}{\edlabel{pvv.263-1}\label{pvv.263-1}\lemma{प}\Bfootnote{उच्चारितैक‚व‚र्ण‚भाग‚नाशेऽप‚रोच्चार‚ण‚मिति साहित्यं नास्ति ।}}र‚स्प‚र‚म‚संश्लिष्य‚न्न}‚स‚म्ब‚ध्य‚मानो दीर्घ‚{\tiny $_{lb}$}‚बुद्धि‚{\color{DodgerBlue3}‚म‚क्र‚मां क‚थ‚मुप‚कुर्य्या}‚दुत्पाद‚ये\edtext{}{\edlabel{pvv.263-2}\label{pvv.263-2}\lemma{ये}\Bfootnote{पूर्व‚भाग‚स्य बुद्धिज‚न‚क‚त्वे प‚र‚भागानाम‚नुप‚योगात् ।}}त् ।\edtext{\textsuperscript{*}}{\edlabel{pvv.263-3}\label{pvv.263-3}\lemma{*}\Bfootnote{स्वाकार‚बुद्धिज‚न‚नेन ।}}क्र‚म‚व‚ति ज्ञेये ज्ञान‚म‚पि त‚थैव यु‚{\tiny $_{2}$}‚क्तं । ‚{\tiny $_{lb}$}‚(४८६)
	\pend% ending standard par
      \label{div_pvv.2.487}
	  
	% new div opening: depth here is 2
	

	  \begin{center}%% label @type='head'
	\textbf{ख. क्र‚म‚भाविनां व‚र्णानां स्फोटेनासंग‚तिः}
	\end{center}
	

	  \pstart \leavevmode% starting standard par
	अथोत्प‚न्ना व‚र्ण्णाव‚य‚वा अन्त्याव‚य‚वोत्प‚त्तिप‚र्य‚न्त‚म‚नुव‚र्त‚न्ते । त‚तः क्र‚म‚ग्र‚ह‚णं ‚{\tiny $_{lb}$}‚स‚र्व्व‚ग्र‚ह‚ण‚ञ्चास्तीति युक्तं दीर्घादिग्र‚ह‚ण‚मित्याह (।)
	\pend% ending standard par
      
	  \bigskip
	  \begingroup
	
	    \large
	  
	    \begin{quote}
	  
	    
	    \stanza[\smallbreak]
	\label{pv.2.487b}\flagstanza{\tiny\textenglish{...2.487b}}आन्त्यं पूर्व‚स्थितादूर्ध्वं व‚र्ध‚मानो ध्व‚निर्भ‚वेत् ।\&[\smallbreak]


	
	    \end{quote}
	  
	  \endgroup
	

	  \pstart \leavevmode% starting standard par
	\hphantom{.}‚{\color{DodgerBlue3}‚आ} अन्त्य‚{\color{DodgerBlue3}‚म‚न्त्यं} व‚र्ण्णाव‚य‚वं याव‚त् ‚{\color{DodgerBlue3}‚पूर्व्व}‚पूर्व्वेषां व‚र्ण्णाव‚य‚वानां क्र‚मोत्प‚न्नानां ‚{\tiny $_{lb}$}‚‚{\color{DodgerBlue3}‚स्थितौ} स‚त्यां प्र‚थ‚म‚व‚र्ण्णाव‚य‚वा‚{\color{DodgerBlue3}‚दूर्ध्व} पूर्व्वोत्प‚न्न‚स्यानुवृत्ताव‚पूर्व्व‚स्य चाप‚र‚स्योत्प‚त्तौ ‚{\tiny $_{lb}$}‚‚{\color{DodgerBlue3}‚व‚र्द्ध‚मानो ध्व‚निर्भ‚वेत्}\edtext{}{\edlabel{pvv.263-4}\label{pvv.263-4}\lemma{त्तौ}\Bfootnote{न द्विमात्र इति प्र‚त्य‚क्ष‚विरोधः ।}} । न चै‚{\tiny $_{3}$}‚त‚द‚स्ति । अव‚य‚व‚क्र‚म‚ग्र‚ह‚णेन दीर्घ‚बुद्धेरुत्पा ‚{\tiny $_{lb}$}‚दात् ।
	\pend% ending standard par
      

	  \pstart \leavevmode% starting standard par
	स्यादेत‚त् (।) क्र‚मेणोत्प‚न्नानाम‚व‚य‚वानाम‚न्त्याव‚य‚व\edtext{}{\edlabel{pvv.263-5}\label{pvv.263-5}\lemma{व}\Bfootnote{नादैराहित‚बीजायाम‚न्त्येन ध्व‚निना स‚ह आवृत्त‚प‚रिपाकायां बुद्धौ श‚ब्दोव‚धार्य‚त  इति ‚{\tiny $_{lb}$}‚वै या क र णा स्तानाह ।}}काले ग्र‚ह‚ण‚मिति न ‚{\tiny $_{lb}$}‚व‚र्द्ध‚मान‚ध्व‚निर्भ‚व‚ति पूर्व्वं क‚स्य‚चिद् ग्र‚ह‚णाभावादित्याह (।)
	\pend% ending standard par
      
	  \bigskip
	  \begingroup
	
	    \large
	  
	    \begin{quote}
	  
	    
	    \stanza[\smallbreak]
	\label{pv.2.487c}\flagstanza{\tiny\textenglish{...2.487c}}अक्र‚मेण ग्र‚हाद‚न्ते क्र‚म‚व‚द्धीश्च नो भ‚वेत् ॥ ४८७ ॥\&[\smallbreak]


	
	    \end{quote}
	  
	  \endgroup
	

	  \pstart \leavevmode% starting standard par
	\hphantom{.}‚{\color{DodgerBlue3}‚अक्र‚मेण} ग्र‚ह‚णाद‚न्त्य‚व‚र्ण‚निष्प‚त्तिकाले च त‚द्‏ग्राहिका ‚{\color{DodgerBlue3}‚क्र‚म‚व‚ती धीर्नो भ‚वेत्} । ‚{\tiny $_{lb}$}‚त‚त‚श्च न दीर्घ‚ग्र‚ह‚णं । न ह्येक‚काल‚म‚नेकैरुच्चार्य‚माणेऽनेक‚स्मिन्न‚कारादौ दीर्घ‚बु‚{\tiny $_{4}$}‚द्धि‚{\tiny $_{lb}$}‚र्भ‚वेत् । (४८७)
	\pend% ending standard par
      \label{div_pvv.2.488}
	  
	% new div opening: depth here is 2
	

	  \pstart \leavevmode% starting standard par
	एकैक‚बुद्धिः स‚कृदुत्प‚न्ना क्र‚मेणाव‚य‚वान् गृह्णात्य‚न्त्याव‚य‚व‚ग्र‚ह‚ण‚काले दीर्घ‚{\tiny $_{lb}$}‚ग्राहिकेति चेत् आह (।)
	\pend% ending standard par
      
	  \bigskip
	  \begingroup
	
	    \large
	  
	    \begin{quote}
	  
	    
	    \stanza[\smallbreak]
	\label{pv.2.488}\flagstanza{\tiny\textenglish{....2.488}}धियः स्व‚य‚ञ्च न स्थानं त‚दूर्ध्व‚विष‚यास्थितेः ॥ ४८८ ॥\&[\smallbreak]


	
	    \end{quote}
	  
	  \endgroup
	\textsuperscript{\textenglish{264/s}}

	  \pstart \leavevmode% starting standard par
	\hphantom{.}‚{\color{DodgerBlue3}‚धियो}‚ऽव‚य‚व‚ग्राहिकाया‚{\color{DodgerBlue3}‚स्त‚स्मा}‚देकाय‚व‚ग्र‚ह‚णा‚{\color{DodgerBlue3}‚दूर्ध्वं} पूर्व‚गृहीत‚स्य विष‚य\edtext{}{\edlabel{pvv.264-1}\label{pvv.264-1}\lemma{य}\Bfootnote{भागानाम‚स्थितेरिति तुल्य‚काल‚म्विष (य) विष‚यित्वेऽयं येन दीर्घादिबुद्धि‚{\tiny $_{lb}$}‚काले न स्यात् ।}}स्या‚{\tiny $_{lb}$}‚‚{\color{DodgerBlue3}‚स्थितेर्हेतोर्न स्थानं} स्थितिर्युक्ता\edtext{}{\edlabel{pvv.264-2}\label{pvv.264-2}\lemma{स्थितिर्युक्ता}\Bfootnote{श‚ब्द‚प‚र‚माणून् त‚द्देश‚विभागेनाप‚राप‚र‚देश‚संयोगेन श्रोत्र‚प‚थ‚मान‚य‚त्य‚तो ‚{\tiny $_{lb}$}‚व‚र्ण्णाभिव्य‚क्तिरिति प्र‚त्य‚भिज्ञा च ।}}। (४८८)
	\pend% ending standard par
      \label{div_pvv.2.489}
	  
	% new div opening: depth here is 2
	
	  \bigskip
	  \begingroup
	
	    \large
	  
	    \begin{quote}
	  
	    
	    \stanza[\smallbreak]
	\label{pv.2.489}\flagstanza{\tiny\textenglish{....2.489}}स्थाने स्व‚य‚न्न न‚श्येत् सा प‚श्चाद‚प्य‚विशेष‚तः ।&दोषोयं स‚कृदुत्प‚न्नाक्र‚म‚व‚र्ण्ण‚स्थिताव‚पि ॥ ४८९ ॥\&[\smallbreak]


	
	    \end{quote}
	  
	  \endgroup
	

	  \pstart \leavevmode% starting standard par
	\hphantom{.}अथ विष‚यान‚व‚स्थानेपि ‚{\color{DodgerBlue3}‚स्व‚य‚म}‚विन‚श्व‚र‚स्व‚भाव‚त‚या बुद्धेः ‚{\color{DodgerBlue3}‚स्थाने} वा स्वीक्रिय‚{\tiny $_{lb}$}‚माणे ‚{\color{DodgerBlue3}‚प‚श्चाद}‚न्ताव‚य‚व‚ग्र‚ह‚णान‚न्त‚र‚म‚प्य‚विन‚श्व‚र‚स्व‚भाव‚त‚या‚{\tiny $_{5}$}‚‚{\color{DodgerBlue3}‚ऽविशेष‚तो न न‚श्येत्} ॥ ‚{\tiny $_{lb}$}‚\edtext{\textsuperscript{*}}{\edlabel{pvv.264-3}\label{pvv.264-3}\lemma{*}\Bfootnote{नैयायिक‚स्य ।}}बुद्धेः स‚कृदुत्प‚न्नायाश्चिराव‚स्थाने यो ‚{\color{DodgerBlue3}‚दोषः} स‚र्व्व‚दाऽविनाश‚प्र‚स‚ङ्ग उक्तोऽयं ‚{\color{DodgerBlue3}‚स‚कृ‚{\tiny $_{lb}$}‚दुत्प‚न्ना}‚नाम‚क्र‚माणां ‚{\color{DodgerBlue3}‚व‚र्ण्णानां} क्र‚म‚ग्राहिविज्ञानोत्प‚त्तिकालं याव‚द‚{\color{DodgerBlue3}‚व‚स्थिताव}‚प्युच्य‚{\tiny $_{lb}$}‚मानायां बोद्ध‚व्यः । य‚दि स‚कृदुत्प‚न्ना अप्य‚विन‚श्व‚र‚स्व‚भाव‚त‚या क‚ञ्चित्काल‚म‚नु‚{\tiny $_{lb}$}‚व‚र्त‚ते त‚दा चिर‚म‚पि त‚त्स्व‚भावाप्र‚च्य‚वाद‚नुव‚र्तेर‚न् । (४८९)
	\pend% ending standard par
      \label{div_pvv.2.490}
	  
	% new div opening: depth here is 2
	

	  \pstart \leavevmode% starting standard par
	किञ्च (।)
	\pend% ending standard par
      
	  \bigskip
	  \begingroup
	
	    \large
	  
	    \begin{quote}
	  
	    
	    \stanza[\smallbreak]
	\label{pv.2.490a}\flagstanza{\tiny\textenglish{...2.490a}}स‚कृद्य‚त्नोद्भ‚वाद् व्य‚र्थः स्याद्य‚त्न‚श्चोत्त‚रोत्त‚रः ।\&[\smallbreak]


	
	    \end{quote}
	  
	  \endgroup
	

	  \pstart \leavevmode% starting standard par
	\hphantom{.}‚{\color{DodgerBlue3}‚स‚कृत् कृताद्य‚त्ना‚{\tiny $_{6}$}‚}‚त्ताल्वादिव्यापारात् व‚र्ण्णा\edtext{}{\edlabel{pvv.264-4}\label{pvv.264-4}\lemma{र्ण्णा}\Bfootnote{मीमांस‚को व‚र्ण‚स्फोट‚वादी न व‚र्ण्णातिरिक्तं प‚दं वाक्यं वाच‚क‚म‚स्तीत्याह ‚{\tiny $_{lb}$}‚श्रोत्र‚प्राप्य‚कारि । व‚र्ण्णाश्च नित्या देश‚काल‚न‚रान्त‚रेष्वेकाकार‚बुद्धिग्राह्य‚त्वात् ‚{\tiny $_{lb}$}‚य‚द्देशः श‚ब्द‚स्त‚द्देश‚स‚न्निहितो वायुर‚दुष्ट‚चोदितः ।}}ना‚{\color{DodgerBlue3}‚मुत्प‚न्न}‚त्वादुत्त‚रोत्त‚रो य‚त्न‚श्च ‚{\tiny $_{lb}$}‚‚{\color{DodgerBlue3}‚व्य‚र्थः} स्यात् ।
	\pend% ending standard par
      

	  \begin{center}%% label @type='head'
	\textbf{ग. न संयोग‚विभाग‚द्वारेण श‚ब्दाभिव्य‚क्तिः}
	\end{center}
	

	  \pstart \leavevmode% starting standard par
	योपि म‚न्य‚ते (।) प्र‚य‚त्न‚प्रेरितेन वायुना स्तिमित‚स्य वायोराकाश‚संयुक्त‚स्य ‚{\tiny $_{lb}$}‚संयोग‚विभाग‚कृता श‚ब्द‚स्याभिव्य‚क्तिर्भ‚व‚तीति तं प्र‚त्याह (।)
	\pend% ending standard par
      
	  \bigskip
	  \begingroup
	
	    \large
	  
	    \begin{quote}
	  
	    
	    \stanza[\smallbreak]
	\label{pv.2.490b}\flagstanza{\tiny\textenglish{...2.490b}}व्य‚क्ताव‚प्येष व‚र्ण्णानां दोषः स‚म‚नुष‚ज्य‚ते ॥ ४९० ॥\&[\smallbreak]


	
	    \end{quote}
	  
	  \endgroup
	\textsuperscript{\textenglish{265/s}}

	  \pstart \leavevmode% starting standard par
	\hphantom{.}‚{\color{DodgerBlue3}‚व‚र्ण्णाना}‚म्वाय‚वीय‚संयोग‚विभाग‚द्वारेणाभि‚{\color{DodgerBlue3}‚व्य‚क्ताव}‚पीष्य‚माणायाम‚य‚म‚न‚न्त‚रोक्‚{\tiny $_{lb}$}‚तोऽक्र‚माणां स‚कृद‚भिव्य‚क्तेरुत्त‚रो य‚त्नः प्राण‚प्रेर‚णादि\edtext{}{\edlabel{pvv.265-1}\label{pvv.265-1}\lemma{णादि}\Bfootnote{स‚कृद‚भिव्य‚क्त‚स्य स‚दा स्थितेर्नित्य‚त्वात्}}को व्य‚{\tiny $_{7}$}‚र्थः स्यादिति ‚{\color{DodgerBlue3}‚दोषः}\leavevmode\ledsidenote{\textenglish{52a/MA}} ‚{\tiny $_{lb}$}‚‚{\color{DodgerBlue3}‚स‚म‚नु}\edtext{\textsuperscript{*}}{\edlabel{pvv.265-2}\label{pvv.265-2}\lemma{*}\Bfootnote{उक्त‚म‚प्युत्त‚र‚प्र‚ब‚न्धाव‚तारार्थ‚माह ।}}ष‚ज्य‚ते । (४९०)
	\pend% ending standard par
      \label{div_pvv.2.491}
	  
	% new div opening: depth here is 2
	

	  \pstart \leavevmode% starting standard par
	स्या\edtext{}{\edlabel{pvv.265-3}\label{pvv.265-3}\lemma{स्या}\Bfootnote{साव‚य‚व‚वादी मीमांस‚क आह ।}}देत‚द् (।)
	\pend% ending standard par
      
	  \bigskip
	  \begingroup
	
	    \large
	  
	    \begin{quote}
	  
	    
	    \stanza[\smallbreak]
	\label{pv.2.491}\flagstanza{\tiny\textenglish{....2.491}}अनेक‚या त‚द्ग्र‚ह‚णे यान्त्या धीः सानुभूय‚ते ।&न दीर्घ‚ग्राहिका सा च त‚न्न स्याद्दीर्घ‚धीस्मृतिः ॥ ४९१ ॥\&[\smallbreak]


	
	    \end{quote}
	  
	  \endgroup
	

	  \pstart \leavevmode% starting standard par
	\hphantom{.}‚{\color{DodgerBlue3}‚अनेक\edtext{}{\edlabel{pvv.265-4}\label{pvv.265-4}\lemma{अनेक}\Bfootnote{निर‚न्त‚र‚या इति पूर्व्व‚तो विशेषः ।}}या} धिया त‚स्यानेकाश्र‚यात्म‚क‚स्य व‚र्ण्ण‚स्य ‚{\color{DodgerBlue3}‚ग्र‚ह‚णे} कृते प‚श्चाद् ‚{\tiny $_{lb}$}‚‚{\color{DodgerBlue3}‚यान्त्या धी}‚र‚न्त्याव‚य‚व‚ग्राहिका ‚{\color{DodgerBlue3}‚सा} बुद्ध्य‚न्त‚रेणा‚{\color{DodgerBlue3}‚नुभूय‚ते} त‚या दीर्घादिबुंद्धि‚{\tiny $_{lb}$}‚र्भ‚विष्य‚तीति चेत् । न‚न्वेवं पूर्व्वासां बुद्धीनाम‚नेकाव‚य‚व‚ग्राहिकाणां न बुद्ध्य‚न्त‚रेण ‚{\tiny $_{lb}$}‚ग्र‚हो नापि स्व‚स‚म्वेद‚नेन वित्तिः । या चानुभूय‚ते साऽन्त्याव‚य‚व‚ग्राहिका धीः ‚{\color{DodgerBlue3}‚सापि} हि ‚{\color{DodgerBlue3}‚न दीर्घा}\edtext{}{\edlabel{pvv.265-5}\label{pvv.265-5}\lemma{हि}\Bfootnote{पूर्व्वाग्र‚हादेव ।}}दिव‚र्ण्ण‚{\color{DodgerBlue3}‚ग्राहिका त}‚त्त‚स्मा‚{\color{DodgerBlue3}‚द्दीर्घ‚धीस्मृतिर्न स्या‚{\tiny $_{1}$}‚त्} । (४९१)
	\pend% ending standard par
      \label{div_pvv.2.492}
	  
	% new div opening: depth here is 2
	
	  \bigskip
	  \begingroup
	
	    \large
	  
	    \begin{quote}
	  
	    
	    \stanza[\smallbreak]
	\label{pv.2.492}\flagstanza{\tiny\textenglish{....2.492}}पृथ‚क् पृथ‚क् च बुद्धीनां स‚म्वित्तौ त‚द् ध्व‚निः श्रुतेः ।&अविच्छिन्नाभ‚ता न स्याद् घ‚ट‚न‚ञ्च निराकृत‚म् ॥ ४९२ ॥\&[\smallbreak]


	
	    \end{quote}
	  
	  \endgroup
	

	  \pstart \leavevmode% starting standard par
	\hphantom{.}न ह्य‚गृहीत‚मेव स्म‚र्य‚ते व‚र्ण्णाव‚य‚व‚ग्राहिकाणां ‚{\color{DodgerBlue3}‚बुद्धीनां पृथ‚क् पृथ‚क्} स्व\edtext{}{\edlabel{pvv.265-6}\label{pvv.265-6}\lemma{स्व}\Bfootnote{पूर्व्वानुभ‚वाभिः ।}}स्व‚ग्रा‚{\tiny $_{lb}$}‚हिणीभिः बुद्धिभिः ‚{\color{DodgerBlue3}‚संवित्ता}‚विष्य‚माणायाञ्च त‚द् ‚{\color{DodgerBlue3}‚ध्व‚निश्रुते}‚र्दीर्घादिव‚र्ण्ण‚बुद्धेर‚{\tiny $_{lb}$}‚‚{\color{DodgerBlue3}‚विच्छिन्ना}‚भ‚ता निर‚न्त‚र‚प्र‚तिभास‚ता ‚{\color{DodgerBlue3}‚न स्यात्} । अन्त‚रान्त‚रा बुद्धिग्राहिणीभिर्बुद्धि‚{\tiny $_{lb}$}‚भिर‚व‚य‚व‚ग्राहिबुद्धीनां व्य‚व‚धानात् । स्व‚ग्राहिकाभिर्बुद्धिभिर्व्य‚व‚धानेपि बुद्धीनां ल‚घु‚{\tiny $_{lb}$}‚वृत्तित्वात् ‚{\color{DodgerBlue3}‚घ‚ट‚न}‚म‚व्य‚व‚धानाध्य‚व‚{\tiny $_{2}$}‚सानं ‚{\color{DodgerBlue3}‚निराकृत}‚म‚न्य‚त्रापि स‚मानं त‚दि\edtext{}{\edlabel{pvv.265-7}\label{pvv.265-7}\lemma{दि}\Bfootnote{त‚द्व‚र्ण्ण‚योर्व्वा स‚कृत् श्रुतिः ।}}त्यादिना ‚{\tiny $_{lb}$}‚(२।१३५) य‚थाव‚य‚व‚ग्राहिका बुद्ध‚यो ल‚घुवृत्त‚यः त‚था त‚द्‏ग्राहिबुद्धिकाले त‚द‚{\tiny $_{lb}$}‚भावोंपि ल‚घुवृत्तिरिति बुद्ध्य‚व‚च्छेद‚व‚र्ण्ण‚विच्छेदोऽपि म‚न्येतेत्युक्तं प्राक् । (४९२)
	\pend% ending standard par
      \label{div_pvv.2.493}
	  
	% new div opening: depth here is 2
	

	  \pstart \leavevmode% starting standard par
	किञ्च (।)
	\pend% ending standard par
      
	  \bigskip
	  \begingroup
	
	    \large
	  
	    \begin{quote}
	  
	    
	    \stanza[\smallbreak]
	\label{pv.2.493}\flagstanza{\tiny\textenglish{....2.493}}विच्छिन्नं श्रृण्व‚तोप्य‚स्य य‚द्य‚विच्छिन्न‚विभ्र‚मः ।&ह्र‚स्व‚द्व‚योच्चार‚णेपि स्याद‚विच्छिन्न‚विभ्र‚मः ॥ ४९३ ॥\&[\smallbreak]


	
	    \end{quote}
	  
	  \endgroup
	

	  \pstart \leavevmode% starting standard par
	\hphantom{.}‚{\color{DodgerBlue3}‚विच्छिन्नं शृण्व‚तोपि} ल‚घुवृत्ते‚{\color{DodgerBlue3}‚र्य‚द्य‚विच्छिन्न‚भ्र‚म}‚स्त‚दा ‚{\color{DodgerBlue3}‚ह्र‚स्व‚द्व‚योच्चार‚णेपि} ल‚घु‚{\tiny $_{lb}$}‚वृत्तेर‚{\color{DodgerBlue3}‚विच्छिन्न‚विभ्र‚मः स्यादि}‚ति दीर्घ‚बुद्धिर्भ‚वेत् । (४९३)
	\pend% ending standard par
      \textsuperscript{\textenglish{266/s}}\label{div_pvv.2.494}
	  
	% new div opening: depth here is 2
	
	  \bigskip
	  \begingroup
	
	    \large
	  
	    \begin{quote}
	  
	    
	    \stanza[\smallbreak]
	\label{pv.2.494}\flagstanza{\tiny\textenglish{....2.494}}विच्छिन्ने द‚र्श‚ने चाक्षाद‚विच्छ‚न्नाधिरोप‚ण‚म् ।&नाक्षात्स‚र्व्वाक्ष‚बुद्धीनां वित‚थ‚त्व‚प्र‚स‚ङ्ग‚तः ॥ ४९४ ॥\&[\smallbreak]


	
	    \end{quote}
	  
	  \endgroup
	

	  \pstart \leavevmode% starting standard par
	व‚र्ण्णाव‚य‚वानाम\edtext{}{\edlabel{pvv.266-1}\label{pvv.266-1}\lemma{वानाम}\Bfootnote{श्रोत्रात् श्र‚व‚णे ।}}क्षादिन्द्रिया‚{\tiny $_{3}$}‚‚{\color{DodgerBlue3}‚द्विच्छिन्ने} स्व‚बुद्धिभिर्व्य‚व‚हिते च व‚स्तुतो ‚{\color{DodgerBlue3}‚द‚र्श‚{\tiny $_{lb}$}‚नेऽक्षाद‚विच्छिन्न}‚स्य द‚र्श‚न‚स्या‚{\color{DodgerBlue3}‚धिरोप‚ण‚मारोप} इति न युक्तं । एव ‚{\color{DodgerBlue3}‚स‚र्व्वा}‚सा‚{\color{DodgerBlue3}‚म‚क्ष‚{\tiny $_{lb}$}‚बुद्धीनां} रूपादिग्राहिणीनां ‚{\color{DodgerBlue3}‚वित‚थ‚त्व‚प्र‚स‚ङ्ग\edtext{}{\edlabel{pvv.266-2}\label{pvv.266-2}\lemma{ङ्ग}\Bfootnote{व्य‚व‚धानेप्य‚व्य‚व‚हित‚भानात् ।}}तो} न किञ्चिद‚भ्रान्तं स्यात् । ‚{\tiny $_{lb}$}‚(४९४)
	\pend% ending standard par
      \label{div_pvv.2.495}
	  
	% new div opening: depth here is 2
	

	  \begin{center}%% label @type='head'
	\textbf{घ. बुद्धेर्बुद्ध्य‚न्त‚रेण गृहीतो नाविच्छेद‚प्र‚तिभासः}
	\end{center}
	

	  \pstart \leavevmode% starting standard par
	बुद्धेर्बुद्ध्य‚न्त‚रेण गृहीताव‚विच्छेद‚प्र‚तिभासो नास्तीति सोप‚प‚त्तिक‚माख्यातु‚{\tiny $_{lb}$}‚माह (।)
	\pend% ending standard par
      
	  \bigskip
	  \begingroup
	
	    \large
	  
	    \begin{quote}
	  
	    
	    \stanza[\smallbreak]
	\label{pv.2.495}\flagstanza{\tiny\textenglish{....2.495}}स‚र्व्वान्त्योपि हि व‚र्ण्णात्मा निमेष‚तुलित‚स्थितिः ।&स च क्र‚माद‚नेकाणुस‚म्ब‚न्धेन नितिष्ठ‚ति ॥ ४९५ ॥\&[\smallbreak]


	
	    \end{quote}
	  
	  \endgroup
	

	  \pstart \leavevmode% starting standard par
	\hphantom{.}‚{\color{DodgerBlue3}‚स‚र्व्वे}‚षां दीर्घादीनामेक‚मात्रिक‚त्वा‚{\color{DodgerBlue3}‚द‚न्त्यो व‚र्ण्णात्मा}‚ऽकारादि‚{\color{DodgerBlue3}‚र्निमेषे}‚ण न‚य‚न‚नि‚{\tiny $_{4}$}‚ ‚{\tiny $_{lb}$}‚मील‚नेन ‚{\color{DodgerBlue3}‚तुलिता} \edtext{\textsuperscript{*}}{\edlabel{pvv.266-3}\label{pvv.266-3}\lemma{*}\Bfootnote{निमेष‚मात्रेणाभिनिर्वृत्तिध‚र्म्मेति याव‚त् ।}}प‚रिमिता ‚{\color{DodgerBlue3}‚स्थिति}‚र्य‚स्य स त‚था । स चाक्षिनिमेष‚काल‚स्थाय्य‚{\tiny $_{lb}$}‚कारादि‚{\color{DodgerBlue3}‚र‚नेके}‚षामाकाश‚त‚मःसंज्ञितानां प‚क्ष्म‚मालाद्व‚य‚व‚र्त्तिना‚{\color{DodgerBlue3}‚म‚णू}‚नां ‚{\color{DodgerBlue3}‚स‚म्ब‚न्धे}‚ना‚{\tiny $_{lb}$}‚क्र‚म‚णेन निमेष‚काल‚संज्ञितेन ‚{\color{DodgerBlue3}‚क्र‚माद् भूयः} क्ष‚ण‚व्य‚व‚धाना‚{\color{DodgerBlue3}‚न्नितिष्ठ‚ति} प‚रिस‚माप्य‚ते । ‚{\tiny $_{lb}$}‚एतेनान्त्योपि व‚र्ण्णोऽनेक‚क्ष‚ण‚स्थायीत्युक्तं । (४९५)
	\pend% ending standard par
      \label{div_pvv.2.496}
	  
	% new div opening: depth here is 2
	
	  \bigskip
	  \begingroup
	
	    \large
	  
	    \begin{quote}
	  
	    
	    \stanza[\smallbreak]
	\label{pv.2.496}\flagstanza{\tiny\textenglish{....2.496}}एकाण्व‚त्य‚य‚काल‚श्च कालोल्पोयान् क्ष‚णो म‚तः ।&बुद्धिश्च क्ष‚णिका त‚स्मात्क्र‚माद्व‚र्ण्णान्प्र‚प‚द्य‚ते ॥ ४९६ ॥\&[\smallbreak]


	
	    \end{quote}
	  
	  \endgroup
	

	  \pstart \leavevmode% starting standard par
	\hphantom{.}‚{\color{DodgerBlue3}‚एक‚स्याणोर‚त्य‚य} आक्र‚म‚णं स ‚{\color{DodgerBlue3}‚कालः} प‚रिमाणं य‚स्य‚{\tiny $_{5}$}‚ तादृश‚श्च ‚{\color{DodgerBlue3}‚कालो}‚ल्पीयान् ‚{\tiny $_{lb}$}‚ल‚घुत‚मो भेत्तुम‚श‚क्यः ‚{\color{DodgerBlue3}‚क्ष\edtext{}{\edlabel{pvv.266-4}\label{pvv.266-4}\lemma{क्ष}\Bfootnote{अन्य‚स्य द्वित्रिप‚र‚माणुव्य‚तिक्र‚मादिभेदेन कालो भेत्तुं श‚क्यः ।}}णो} भ‚वेत् । ताव‚त्काल‚मात्र‚स्थाष्णुत‚या ‚{\color{DodgerBlue3}‚बुद्धिश्च क्ष‚णिका\edtext{}{\edlabel{pvv.266-5}\label{pvv.266-5}\lemma{णिका}\Bfootnote{अनेक‚क्ष‚णाभिनिर्वृत्तिध‚र्म्माश्च व‚र्ण्णाः ।}} ‚{\tiny $_{lb}$}‚त‚स्मा}‚त्क्ष‚णिक‚त्वाद् बुद्धिः ‚{\color{DodgerBlue3}‚क्र‚मा}‚त्पुन\edtext{}{\edlabel{pvv.266-6}\label{pvv.266-6}\lemma{त्पुन}\Bfootnote{क्र‚म‚ग्राहिकापि स्व‚ग्राहिक‚यान्त‚रितेत्य‚विच्छिन्न‚ग्र‚हः कुतः ।}} र्धीर्व्य‚व‚हिता भूय‚सी भ‚व‚न्ती ‚{\color{DodgerBlue3}‚व‚र्ण्णान् प्र‚तिप‚द्य‚ते} त‚त्क‚थ‚म‚विच्छिन्न‚व‚र्ण्ण‚ग्र‚हः स‚ङ्ग‚तः । (४९६)
	\pend% ending standard par
      \textsuperscript{\textenglish{267/s}}\label{div_pvv.2.497}
	  
	% new div opening: depth here is 2
	
	  \bigskip
	  \begingroup
	
	    \large
	  
	    \begin{quote}
	  
	    
	    \stanza[\smallbreak]
	\label{pv.2.497}\flagstanza{\tiny\textenglish{....2.497}}इति व‚र्ण्णेपि रूपादाव‚विच्छिन्नाव‚भासिनी ।&विच्छिन्नाप्य‚न्य‚या बुद्धिः स‚र्व्वा स्याद्वित‚थार्थिका ॥ ४९७ ॥\&[\smallbreak]


	
	    \end{quote}
	  
	  \endgroup
	

	  \pstart \leavevmode% starting standard par
	\hphantom{.}‚{\color{DodgerBlue3}‚इति} त‚स्माद् ‚{\color{DodgerBlue3}‚व‚र्ण्णेपि रू\edtext{}{\edlabel{pvv.267-1}\label{pvv.267-1}\lemma{रू}\Bfootnote{य‚था व‚र्ण्णेषु प्र‚त्येकं विभ्र‚म एवं रूपादाव‚पीति दार्ष्टान्तिक‚क‚थ‚नं ।}}पादा}‚व‚पि बुद्धिस्त‚द्‏ग्राहिक‚याऽन्य‚या धियाऽन्त‚रिता ‚{\tiny $_{lb}$}‚व‚स्तुतो ‚{\color{DodgerBlue3}‚विच्छिन्नाप्य‚विच्छिन्नाव‚भासिनी} कु\edtext{}{\edlabel{pvv.267-2}\label{pvv.267-2}\lemma{कु}\Bfootnote{त‚न्माभूत् स‚र्व्वेन्द्रिय‚ज्ञान‚वित‚थ‚त्व‚मिति नाविच्छिन्न‚भ्र‚मः स्वीक‚र्त‚व्य‚स्त‚दा ‚{\tiny $_{lb}$}‚च निर‚न्त‚र‚भान‚न्न स्याद‚स्ति चातः स्व‚वित्तिर्ज्ञानानां सिद्धा ।}}त‚{\tiny $_{6}$}‚श्चिन्निमिताद् भ‚व‚न्ती ‚{\color{DodgerBlue3}‚स‚र्व्वा\leavevmode\ledsidenote{\textenglish{52b/MA}} ‚{\tiny $_{lb}$}‚वित‚थार्थिका} शून्या भ्रान्तिः ‚{\color{DodgerBlue3}‚स्यात्} । (४९७)
	\pend% ending standard par
      \label{div_pvv.2.498}
	  
	% new div opening: depth here is 2
	
	  \bigskip
	  \begingroup
	
	    \large
	  
	    \begin{quote}
	  
	    
	    \stanza[\smallbreak]
	\label{pv.2.498}\flagstanza{\tiny\textenglish{....2.498}}घ‚ट‚नं य‚च्च भावानाम‚न्य‚त्रेन्द्रिय‚विभ्र‚मात् ।&भेदाल‚क्ष‚ण‚विभ्रान्तं स्म‚र‚ण‚न्त‚द्विक‚ल्प‚कं ॥ ४९८ ॥\&[\smallbreak]


	
	    \end{quote}
	  
	  \endgroup
	

	  \pstart \leavevmode% starting standard par
	\hphantom{.}‚{\color{DodgerBlue3}‚य‚च्च भावानां\edtext{}{\edlabel{pvv.267-3}\label{pvv.267-3}\lemma{भावानां}\Bfootnote{क्ष‚णिकानां निर‚न्व‚य‚विनाशिनां । अस‚ति बाध‚के भानादेवान‚योः प्रामाण्यं ।}}} स‚दृशाप‚राप‚रेषामुत्प‚द्य‚मानानां स एवाय‚मित्येक‚त्व‚{\color{DodgerBlue3}‚घ‚ट‚नं} त‚द् भावानां प‚र‚स्प‚र‚तो ‚{\color{DodgerBlue3}‚भेदाल‚क्ष‚णेन विभ्रान्तं विक‚ल्प‚कं} ज्ञानं ‚{\color{DodgerBlue3}‚स्म‚र‚णं । अ\edtext{}{\edlabel{pvv.267-4}\label{pvv.267-4}\lemma{अ}\Bfootnote{इन्द्रिय‚वैगुण्ये तु न घ‚ट‚नं स‚कृद‚स‚द‚र्थ‚प्र‚तिभासं स्प‚ष्ट‚मेव ज्ञान‚म‚त्र ।}}न्य‚त्रे‚{\tiny $_{lb}$}‚न्द्रिय‚विभ्र‚मात्} । अतः पुन‚र‚लातादिषु भ्राम्य‚माणेषु च‚क्राद्याकारं ज्ञान‚मुत्प‚द्य‚ते‚{\tiny $_{lb}$}‚ऽसाविन्द्रिय‚क्र‚मो निर्व्विक‚ल्प‚क एव । (४९८)
	\pend% ending standard par
      \label{div_pvv.2.499}
	  
	% new div opening: depth here is 2
	
	  \bigskip
	  \begingroup
	
	    \large
	  
	    \begin{quote}
	  
	    
	    \stanza[\smallbreak]
	\label{pv.2.499}\flagstanza{\tiny\textenglish{....2.499}}त‚स्य स्प‚ष्टाव‚भासित्वं ज‚ल्प‚संस‚र्गिणः कुतः ॥&नाक्ष‚ग्राह्येस्ति श‚ब्दानां योज‚नेति विवेचित‚म् ॥ ४९९ ॥\&[\smallbreak]


	
	    \end{quote}
	  
	  \endgroup
	

	  \pstart \leavevmode% starting standard par
	\hphantom{.}‚{\color{DodgerBlue3}‚त‚स्य} विक‚ल्प‚स्य ‚{\color{DodgerBlue3}‚ज‚ल्प‚संस‚{\tiny $_{1}$}‚र्गिणः} श‚ब्द‚संस‚र्ग‚व‚तः ‚{\color{DodgerBlue3}‚स्प‚ष्टाव\edtext{}{\edlabel{pvv.267-5}\label{pvv.267-5}\lemma{ष्टाव}\Bfootnote{य‚तोनेन दीर्घादिव‚र्ण्ण‚स्य प्र‚तिभासो गृह्येत ।}}भासित्वं} ज्ञेयाकार‚{\tiny $_{lb}$}‚वैश‚द्यं ‚{\color{DodgerBlue3}‚कुतः} ।\edtext{\textsuperscript{*}}{\edlabel{pvv.267-6}\label{pvv.267-6}\lemma{*}\Bfootnote{य‚श्चाक्ष‚प्र‚त्य‚यः स्प‚ष्ट‚प्र‚तिभास‚स्त‚स्य व‚र्ण‚भागेष्व‚नुस‚न्धायी दीर्घादिप्र‚त्य‚यो नास्ति ।}} य‚स्मा‚{\color{DodgerBlue3}‚द‚क्ष‚ग्राह्ये} स्व‚ल‚क्ष‚णे ‚{\color{DodgerBlue3}‚श‚ब्दानां} संकेताग्र‚ह‚णात् वाच‚क‚त्वेन ‚{\tiny $_{lb}$}‚‚{\color{DodgerBlue3}‚योज‚ना} नास्तीति ‚{\color{DodgerBlue3}‚विवेचितं} प्राक् । (४९९)
	\pend% ending standard par
      \label{div_pvv.2.500}
	  
	% new div opening: depth here is 2
	

	  \begin{center}%% label @type='head'
	\textbf{(ङ. त‚दा न विच्छिन्नं द‚र्श‚न‚म्)}
	\end{center}
	
	  \bigskip
	  \begingroup
	
	    \large
	  
	    \begin{quote}
	  
	    
	    \stanza[\smallbreak]
	\label{pv.2.500a}\flagstanza{\tiny\textenglish{...2.500a}}विच्छिन्नं प‚श्य‚तोप्य‚क्षैर्घ‚ट‚येद्य‚दि क‚ल्प‚ना ।\&[\smallbreak]


	
	    \end{quote}
	  
	  \endgroup
	

	  \pstart \leavevmode% starting standard par
	\hphantom{.}अथाक्षैरिन्द्रिय‚ज्ञानैः स्व‚ज्ञान‚व्य‚व‚हितैर‚प‚राप‚र‚म‚र्थं ‚{\color{DodgerBlue3}‚विच्छिन्नं प‚श्य‚तोपि} ज्ञाना‚{\tiny $_{lb}$}‚नुच‚री ‚{\color{DodgerBlue3}‚क‚ल्प\edtext{}{\edlabel{pvv.267-7}\label{pvv.267-7}\lemma{ल्प}\Bfootnote{ग्राह‚क‚ज्ञानानुभ‚व‚स्यान‚न्त‚र‚जा तृतीया ।}}ना} त‚देवेद‚मित्येक‚त्वेन ‚{\color{DodgerBlue3}‚घ‚ट‚येदिति य‚द्यु}‚च्य‚ते (।)
	\pend% ending standard par
      \textsuperscript{\textenglish{268/s}}

	  \pstart \leavevmode% starting standard par
	त‚दा विच्छिन्न‚द‚र्श‚न‚मेव नास्तीति व‚{\tiny $_{2}$}‚क्तुमाह (।)
	\pend% ending standard par
      
	  \bigskip
	  \begingroup
	
	    \large
	  
	    \begin{quote}
	  
	    
	    \stanza[\smallbreak]
	\label{pv.2.500b}\flagstanza{\tiny\textenglish{...2.500b}}अर्थ‚स्य त‚त्संवित्तेश्च स‚त‚तं भास‚मान‚योः ॥ ५०० ॥\&[\smallbreak]


	
	    \end{quote}
	  
	  \endgroup
	

	  \pstart \leavevmode% starting standard par
	\hphantom{.}‚{\color{DodgerBlue3}‚अर्थ‚स्य} ग्राह्य‚स्य ‚{\color{DodgerBlue3}‚त‚त्संवित्तेश्च} विजातीयाव्य‚व\edtext{}{\edlabel{pvv.268-1}\label{pvv.268-1}\lemma{व}\Bfootnote{इन्द्रिय‚ज्ञाने अर्थोपि निर‚न्त‚रं भास‚ते त‚ज्ज्ञान‚म‚पि त‚दाकारं स‚म्वेद्य‚तेऽक्ष‚सिद्धं ।}}कीर्ण्ण‚त्वात् । ‚{\color{DodgerBlue3}‚स‚त‚तं भास‚{\tiny $_{lb}$}‚मान‚योर}‚न‚योर‚विच्छेद\edtext{}{\edlabel{pvv.268-2}\label{pvv.268-2}\lemma{विच्छेद}\Bfootnote{प्र‚त्य‚भिज्ञा नेह विच्छेदाव‚साय‚साध‚कं किञ्चित्प्र‚माण‚म‚स्ति ।}}भास‚न‚स्य (। ५००)
	\pend% ending standard par
      \label{div_pvv.2.501}
	  
	% new div opening: depth here is 2
	
	  \bigskip
	  \begingroup
	
	    \large
	  
	    \begin{quote}
	  
	    
	    \stanza[\smallbreak]
	\label{pv.2.501}\flagstanza{\tiny\textenglish{....2.501}}बाध‚केऽस‚ति स‚न्न्याये विच्छिन्ने इति त‚त्कुतः ।&बुद्धीनां श‚क्तिनिय‚मादिति चेत्स कुतो म‚तः ॥ ५०१ ॥\&[\smallbreak]


	
	    \end{quote}
	  
	  \endgroup
	

	  \pstart \leavevmode% starting standard par
	\hphantom{.}‚{\color{DodgerBlue3}‚बाध‚के} संश्चासौ ‚{\color{DodgerBlue3}‚न्याय‚श्च} त‚स्मिन् प्र‚माण‚भूते बाध‚के‚{\color{DodgerBlue3}‚ऽस‚ती}‚त्य‚र्थः । तेऽर्थ‚संवित्ती ‚{\tiny $_{lb}$}‚ज्ञान‚ज्ञानेन ‚{\color{DodgerBlue3}‚विच्छिन्ने इति} य‚दुच्य‚ते ‚{\color{DodgerBlue3}‚त‚त्कुतः\edtext{}{\edlabel{pvv.268-3}\label{pvv.268-3}\lemma{त्कुतः}\Bfootnote{य‚स्मिन्विच्छेदे स‚ति क‚ल्पिक‚या घ‚ट‚नं ।}} । बुद्धीनामे}‚कैक‚बुद्धिज‚न‚न‚{\color{DodgerBlue3}‚श‚क्ति‚{\tiny $_{lb}$}‚निय‚मात्} न ग्राह्य‚बुद्ध्या स‚ह त‚द्‏ग्राहिका बुद्धिर्भ‚व‚ति । त‚तो विच्छिन्न‚{\tiny $_{3}$}‚त‚याऽर्थ‚{\tiny $_{lb}$}‚द‚र्श‚न‚{\color{DodgerBlue3}‚मिति चेत् । स} युग‚प‚त् ज्ञानानुत्पादः ‚{\color{DodgerBlue3}‚कुतो} हेतोर्म‚तः । (५०१)
	\pend% ending standard par
      \label{div_pvv.2.502}
	  
	% new div opening: depth here is 2
	
	  \bigskip
	  \begingroup
	
	    \large
	  
	    \begin{quote}
	  
	    
	    \stanza[\smallbreak]
	\label{pv.2.502a}\flagstanza{\tiny\textenglish{...2.502a}}युग‚प‚द् बुद्ध्य‚दृष्टेश्चेत् त‚देवेदं विचार्य‚ते ।\&[\smallbreak]


	
	    \end{quote}
	  
	  \endgroup
	

	  \pstart \leavevmode% starting standard par
	\hphantom{.}‚{\color{DodgerBlue3}‚युग‚प‚द् बु (द्) ध्यो}‚रुत्प‚न्न‚योर‚{\color{DodgerBlue3}‚दृष्टेश्चेत्} य‚द्य‚द‚र्श‚नं प्र‚माणं विच्छेद‚स्यापि त‚द‚{\tiny $_{lb}$}‚स्तीति सोप्य‚युक्ताभ्युप‚ग‚मः । अथ विच्छेदाद‚र्श‚नेपि विचारात् सा सिध्य‚ति त‚दा ‚{\tiny $_{lb}$}‚‚{\color{DodgerBlue3}‚त‚देवेदं युग‚प‚द् बुद्ध्य}‚द‚र्श‚न‚म‚पि बाध्य‚त‚या न युग‚प‚द् बुद्ध्युत्पाद‚बाध‚नास‚म‚र्थ‚मिति ‚{\tiny $_{lb}$}‚‚{\color{DodgerBlue3}‚विचार्य‚ते} । त‚त् किम‚स्योप‚न्यासेन ।
	\pend% ending standard par
      

	  \begin{center}%% label @type='head'
	\textbf{च. न युग‚प‚त् चित्त‚द्व‚य‚संप्र‚तिप‚त्तिः}
	\end{center}
	

	  \pstart \leavevmode% starting standard par
	\hphantom{.}न‚नूक्तं भ ग व ता ऽस्था\edtext{}{\edlabel{pvv.268-4}\label{pvv.268-4}\lemma{ऽस्था}\Bfootnote{वैभाष्यः स्व‚वित्तिं नेच्छ‚ति ।}} ‚{\tiny $_{4}$}‚न‚मेत‚त् य‚त् द्वे चित्ते युग‚प‚त् संप्र‚तिप‚द्येयाता‚{\tiny $_{lb}$}‚मि ति । त‚त्क‚थ‚मित्याह (।)
	\pend% ending standard par
      
	  \bigskip
	  \begingroup
	
	    \large
	  
	    \begin{quote}
	  
	    
	    \stanza[\smallbreak]
	\label{pv.2.502b}\flagstanza{\tiny\textenglish{...2.502b}}तासां स‚मान‚जातीये साम‚र्थ्य‚निय‚मो भ‚वेत् ॥ ५०२ ॥\&[\smallbreak]


	
	    \end{quote}
	  
	  \endgroup
	

	  \pstart \leavevmode% starting standard par
	\hphantom{.}‚{\color{DodgerBlue3}‚ता\edtext{}{\edlabel{pvv.268-5}\label{pvv.268-5}\lemma{ता}\Bfootnote{आभिप्रायिक‚मेत‚त् ।}}सां} बुद्धीनां विक‚ल्पिकानामिन्द्रिय‚जानाञ्च ‚{\color{DodgerBlue3}‚स‚मान‚जातीय} एक‚स्मिन् ज्ञाने ‚{\tiny $_{lb}$}‚क‚र्त्त‚व्ये ‚{\color{DodgerBlue3}‚साम‚र्थ्य‚निय‚म}‚स्त‚था भ‚ग‚व‚ता ‚{\color{DodgerBlue3}‚प्र‚तिपादितो भ‚वेत्} । त‚तो ग्राह्य\edtext{}{\edlabel{pvv.268-6}\label{pvv.268-6}\lemma{ग्राह्य}\Bfootnote{विष‚य‚ज्ञान‚न्त‚द‚नुभ‚व‚ज्ञान‚ञ्च ।}}ग्राह‚क‚म‚{\tiny $_{lb}$}‚स‚मान‚जातीयं बुद्धिद्व‚य‚म‚पि स‚ह जायेत । स‚मान‚जातीय‚न्तु न स‚ह जाय‚ते । (५०२)
	\pend% ending standard par
      \textsuperscript{\textenglish{269/s}}\label{div_pvv.2.503_2.504}
	  
	% new div opening: depth here is 2
	
	  \bigskip
	  \begingroup
	
	    \large
	  
	    \begin{quote}
	  
	    
	    \stanza[\smallbreak]
	\label{pv.2.503a}\flagstanza{\tiny\textenglish{...2.503a}}त‚था हि स‚म्य‚क् ल‚क्ष्य‚न्ते विक‚ल्पाः क्र‚म‚भाविनः ।\&[\smallbreak]


	
	    \end{quote}
	  
	  \endgroup
	

	  \pstart \leavevmode% starting standard par
	\hphantom{.}‚{\color{DodgerBlue3}‚त‚था हि विक‚ल्पाः क्र‚म‚भाविनः स‚म्य‚ग् ल‚क्ष्य‚न्ते} । इन्द्रिय‚ज्ञानानि‚{\tiny $_{5}$}‚ च स‚मान‚{\tiny $_{lb}$}‚जातीयानि क्र‚म‚व‚न्ति च दृश्य‚न्ते । विक‚ल्पेन्द्रिय‚ज्ञाने च‚क्षुःश्रोत्रादिज्ञाने च ‚{\tiny $_{lb}$}‚स‚होत्प‚द्य‚ते (न्ते) ।
	\pend% ending standard par
      

	  \begin{center}%% label @type='head'
	\textbf{(४) प्र‚त्य‚भिज्ञाचिन्ता}
	\end{center}
	

	  \begin{center}%% label @type='head'
	\textbf{क. एक‚त्व‚म‚न्त‚रेण प्र‚त्य‚भिज्ञान‚म्}
	\end{center}
	
	  \bigskip
	  \begingroup
	
	    \large
	  
	    \begin{quote}
	  
	    
	    \stanza[\smallbreak]
	\label{pv.2.503b}\flagstanza{\tiny\textenglish{...2.503b}}एतेन यः स‚म‚क्षेर्थे प्र‚त्य‚भिज्ञान‚क‚ल्प‚नाम् ॥ ५०३ ॥\&[\smallbreak]


	
	    \end{quote}
	  
	  \endgroup
	
	  \bigskip
	  \begingroup
	
	    \large
	  
	    \begin{quote}
	  
	    
	    \stanza[\smallbreak]
	\label{pv.2.504a}\flagstanza{\tiny\textenglish{...2.504a}}स्प‚ष्टाव‚भासां प्र‚त्य‚क्षां क‚ल्प‚येत् सोपि वारितः ॥\&[\smallbreak]


	
	    \end{quote}
	  
	  \endgroup
	

	  \pstart \leavevmode% starting standard par
	\hphantom{.}‚{\color{DodgerBlue3}‚एतेन} विक‚ल्पाविक‚ल्प‚योः स‚होत्पादेन यो ‚{\color{DodgerBlue3}‚मीमांस\edtext{}{\edlabel{pvv.269-1}\label{pvv.269-1}\lemma{मीमांस}\Bfootnote{जैमिनीयः ।}}कः स‚म‚क्षे} प्र‚त्य‚क्षे‚{\color{DodgerBlue3}‚ऽर्थे स} एवाय‚मित्येक\edtext{}{\edlabel{pvv.269-2}\label{pvv.269-2}\lemma{मित्येक}\Bfootnote{विक‚ल्प एव प्र‚त्य‚भिज्ञा ।}}त्व‚विष‚यां ‚{\color{DodgerBlue3}‚प्र‚त्य‚भिज्ञानाख्यां क‚ल्प‚नां} (५०३) ‚{\color{DodgerBlue3}‚स्प‚ष्ट‚प्र‚तिभा\edtext{}{\edlabel{pvv.269-3}\label{pvv.269-3}\lemma{तिभा}\Bfootnote{विष‚य‚ज्ञान‚न्त‚द‚नुभ‚व‚ज्ञान‚ञ्च प‚रो भ्रान्त्या ।}}सां प्र‚त्य‚क्षां} प्र‚माणं ‚{\color{DodgerBlue3}‚क‚ल्प‚येत् सोपि निवारितः} । विक‚ल्पो हि विक‚ल्पितार्थ‚गोच‚र‚त्वाद‚स्प‚ष्ट‚{\tiny $_{lb}$}‚प्र‚तिभास एव । विष (?श) द‚प्र‚तिभा‚{\tiny $_{6}$}‚स‚नेन्द्रिय‚ज्ञानेन स‚ह‚भावित्वात् स्प‚ष्ट-\leavevmode\ledsidenote{\textenglish{53a/MA}} ‚{\tiny $_{lb}$}‚ग्राह्यारोपोऽध्य‚व‚सीय‚ते ।
	\pend% ending standard par
      

	  \pstart \leavevmode% starting standard par
	किञ्चैक‚त्व‚म‚न्त‚रेणापि प्र‚त्य‚भिज्ञानं दृश्य‚ते इति द‚र्श‚य‚न्नाह (।)
	\pend% ending standard par
      
	  \bigskip
	  \begingroup
	
	    \large
	  
	    \begin{quote}
	  
	    
	    \stanza[\smallbreak]
	\label{pv.2.504b}\flagstanza{\tiny\textenglish{...2.504b}}केश‚गोल‚क‚दीपादाव‚पि स्व‚ष्टाव‚भास‚नात् ॥ ५०४ ॥\&[\smallbreak]


	
	    \end{quote}
	  
	  \endgroup
	

	  \pstart \leavevmode% starting standard par
	\hphantom{.}लून‚पुन‚र्जाते ‚{\color{DodgerBlue3}‚केशादौ} मायाकार‚द‚र्शिते ‚{\color{DodgerBlue3}‚गोल‚कादौ} क्ष‚ण‚विनाशि‚{\color{DodgerBlue3}‚दीपादौ । ‚{\tiny $_{lb}$}‚स्प‚ष्टाव‚भास‚नात्} । (५०४)
	\pend% ending standard par
      \label{div_pvv.2.505}
	  
	% new div opening: depth here is 2
	
	  \bigskip
	  \begingroup
	
	    \large
	  
	    \begin{quote}
	  
	    
	    \stanza[\smallbreak]
	\label{pv.2.505}\flagstanza{\tiny\textenglish{....2.505}}प्र‚तीत‚भेदेप्य‚ध्य‚क्षा धीः क‚थ‚न्तादृशी भ‚वेत् ॥&त‚स्मान्न प्र‚त्य‚भिज्ञानाद्व‚र्ण्णाद्येक‚त्व‚निश्च‚यः ॥ ५०५ ॥\&[\smallbreak]


	
	    \end{quote}
	  
	  \endgroup
	

	  \pstart \leavevmode% starting standard par
	\hphantom{.}प्र‚त्य‚क्षात् ‚{\color{DodgerBlue3}‚प्र‚तीते भेदे} सा प्र‚त्य‚भिज्ञा ‚{\color{DodgerBlue3}‚तादृशी} एक‚त्वाध्य‚व‚सायिनी दृश्य‚{\tiny $_{lb}$}‚माना‚{\color{DodgerBlue3}‚ध्य‚क्षा धीः क‚थं\edtext{}{\edlabel{pvv.269-4}\label{pvv.269-4}\lemma{थं}\Bfootnote{एक‚त्व‚साध‚नाभिम‚त‚स्यानेक‚त्रापि द‚र्श‚नात् ।}} भ‚वेत्} । प्राप्नोति च प्र‚त्य‚भिज्ञा प्र‚त्य‚क्ष‚वादिनो म\edtext{}{\edlabel{pvv.269-5}\label{pvv.269-5}\lemma{म}\Bfootnote{इति साध्यं हेतुः प‚रं ।}}ते । ‚{\tiny $_{lb}$}‚‚{\color{DodgerBlue3}‚त‚स्मान्नास्ति प्र‚त्य‚भिज्ञानाद् व‚र्ण्णाद्ये \edtext{}{\edlabel{pvv.269-6}\label{pvv.269-6}\lemma{र्ण्णाद्ये}\Bfootnote{रूपादिः ।}} ‚{\tiny $_{1}$}‚क‚त्व‚निश्च‚यः} । (५०५)
	\pend% ending standard par
      \textsuperscript{\textenglish{270/s}}\label{div_pvv.2.506}
	  
	% new div opening: depth here is 2
	
	  \bigskip
	  \begingroup
	
	    \large
	  
	    \begin{quote}
	  
	    
	    \stanza[\smallbreak]
	\label{pv.2.506}\flagstanza{\tiny\textenglish{....2.506}}पूर्व्वानुभूत‚स्म‚र‚णात्त‚द्ध‚र्मारोप‚णाद्विना ।&स एवाय‚मिति ज्ञानं नास्ति त‚च्चाक्ष‚जे कुतः ॥ ५०६ ॥\&[\smallbreak]


	
	    \end{quote}
	  
	  \endgroup
	

	  \pstart \leavevmode% starting standard par
	य‚स\edtext{}{\edlabel{pvv.270-1}\label{pvv.270-1}\lemma{स}\Bfootnote{स्वीकृत्येन्द्रिय‚त्वं दूष‚यित्वा ऐन्द्रिय‚स्व‚भाव‚म‚धुनाह ।}}मा‚{\color{DodgerBlue3}‚त्पूर्व्वानुभूत‚स्यार्थ‚स्य} स्म‚र‚णा‚{\color{DodgerBlue3}‚त्त‚द्ध‚र्म‚स्य} विद्य‚मान‚त्वे‚{\color{DodgerBlue3}‚नारोप\edtext{}{\edlabel{pvv.270-2}\label{pvv.270-2}\lemma{नारोप}\Bfootnote{अतीतारोपं विना स एवाय‚मिति नास्ति ।}}णात् विना स ‚{\tiny $_{lb}$}‚एवाय‚मिति} ज्ञान‚मेक‚त्व‚विष‚यं नास्ति(।) त‚च्च पूर्व्वानुभूत‚स्म‚र‚णं त‚द्ध‚र्मारोप‚ण‚{\tiny $_{lb}$}‚ञ्चाक्ष‚जे व‚र्त‚मान‚व‚स्तुब‚ल‚भाविनि कुतः स‚म्भ‚व‚ति । (५०६)
	\pend% ending standard par
      \label{div_pvv.2.507_2.508}
	  
	% new div opening: depth here is 2
	

	  \begin{center}%% label @type='head'
	\textbf{(ख. अविच्छिन्नं व‚र्णादिद‚र्श‚न‚म्)}
	\end{center}
	

	  \pstart \leavevmode% starting standard par
	स्यादेत‚द् (।) अर्थ‚ज्ञान‚योर्युग‚प‚त्स\edtext{}{\edlabel{pvv.270-3}\label{pvv.270-3}\lemma{त्स}\Bfootnote{बौद्धेन स्वीकृत‚युग‚प‚ज्ज्ञान‚स‚म्भ‚वे स‚ति ।}}म्भ‚वे स‚त्य‚विच्छिन्नं व‚र्ण्णादिद‚र्श‚नं स्यादि‚{\tiny $_{lb}$}‚त्याह (।)
	\pend% ending standard par
      
	  \bigskip
	  \begingroup
	
	    \large
	  
	    \begin{quote}
	  
	    
	    \stanza[\smallbreak]
	\label{pv.2.507}\flagstanza{\tiny\textenglish{....2.507}}न चार्थ‚ज्ञान‚स‚म्वित्योर्युग‚प‚त्स‚म्भ‚वो य‚तः ।&ल‚क्ष्येते प्र‚तिभासौ द्वौ नार्थार्थ‚ज्ञान‚योः पृथ‚क् ॥ ५०७ ॥\&[\smallbreak]


	
	    \end{quote}
	  
	  \endgroup
	
	  \bigskip
	  \begingroup
	
	    \large
	  
	    \begin{quote}
	  
	    
	    \stanza[\smallbreak]
	\label{pv.2.508a}\flagstanza{\tiny\textenglish{...2.508a}}न ह्य‚र्थाभासि च ज्ञान‚म‚र्थो बाह्य‚श्च केव‚लः ।\&[\smallbreak]


	
	    \end{quote}
	  
	  \endgroup
	

	  \pstart \leavevmode% starting standard par
	\hphantom{.}न चार्थ‚ज्ञान‚योर्ये स‚म्वित्ती त‚यो‚{\color{DodgerBlue3}‚र्युग‚प‚त्स‚म्भ‚वो}‚स्ति (।) य‚तोर्थ‚स्या‚{\color{DodgerBlue3}‚र्थ‚ज्ञान}‚{\tiny $_{2}$}‚स्य ‚{\tiny $_{lb}$}‚द्वौ ‚{\color{DodgerBlue3}‚प्र‚तिभा\edtext{}{\edlabel{pvv.270-4}\label{pvv.270-4}\lemma{तिभा}\Bfootnote{बौद्धेन स्वीकृत‚युग‚प‚ज्ज्ञान‚स‚म्भ‚वे स‚ति ।}}सौ पृथ‚ग्न} भेद‚न ‚{\color{DodgerBlue3}‚ल\edtext{}{\edlabel{pvv.270-5}\label{pvv.270-5}\lemma{ल}\Bfootnote{उप‚ल‚ब्धिल‚क्ष‚ण‚प्राप्त‚स्यानुप‚ल‚ब्धेः ।}}क्ष्य‚ते} । (५०७) न ह्य‚{\color{DodgerBlue3}‚र्थाभासि} त‚त् ‚{\color{DodgerBlue3}‚ज्ञानं । अर्थो ‚{\tiny $_{lb}$}‚बाह्य‚श्च केवेलो} बुद्धिव्य‚तिरिक्त इति द्वौ प्र‚तिभासौ स‚म्भ‚व‚तः ।
	\pend% ending standard par
      

	  \pstart \leavevmode% starting standard par
	स्यादेत‚त् । अर्थ‚ज्ञान‚ज्ञान‚योर्भिन्नावेव प्र‚तिभासौ केव‚लं त‚दुत्त‚र‚या विक‚ल्प‚{\tiny $_{lb}$}‚बुद्ध्या एक‚त्वेन गृह्येते अत्राह (।)
	\pend% ending standard par
      
	  \bigskip
	  \begingroup
	
	    \large
	  
	    \begin{quote}
	  
	    
	    \stanza[\smallbreak]
	\label{pv.2.508b}\flagstanza{\tiny\textenglish{...2.508b}}एकाकार‚म‚तिग्राह्ये भेदाभाव‚प्र‚स‚ङ्ग‚तः ॥ ५०८ ॥\&[\smallbreak]


	
	    \end{quote}
	  
	  \endgroup
	

	  \pstart \leavevmode% starting standard par
	\hphantom{.}‚{\color{DodgerBlue3}‚एकाकार‚या म‚त्या} विक‚ल्पिक‚या ‚{\color{DodgerBlue3}‚ग्राह्ये}‚ऽर्थ‚ज्ञान‚योर्ज्ञाने य‚दि त‚दा त‚योर्भेद\edtext{}{\edlabel{pvv.270-6}\label{pvv.270-6}\lemma{योर्भेद}\Bfootnote{बुद्ध्याकाराभेदाद‚र्थाभेदेन ।}}स्या‚{\tiny $_{lb}$}‚‚{\color{DodgerBlue3}‚भाव‚प्र‚स‚ङ्ग‚त} एक‚त्व‚मेवाभ्युप‚ग‚न्त‚व्यं ।‚{\tiny $_{3}$}‚ अर्थ‚ज्ञान‚ज्ञान‚योः प्र‚तिभास‚{\color{DodgerBlue3}‚भेदाभावात्} । ‚{\tiny $_{lb}$}‚(५०८)
	\pend% ending standard par
      \label{div_pvv.2.509}
	  
	% new div opening: depth here is 2
	
	  \bigskip
	  \begingroup
	
	    \large
	  
	    \begin{quote}
	  
	    
	    \stanza[\smallbreak]
	\label{pv.2.509}\flagstanza{\tiny\textenglish{....2.509}}सूप‚ल‚क्षेण भेदेन यौ स‚म्वित्तौ न ल‚क्षितौ ।&अर्थार्थ‚प्र‚त्य‚यौ प‚श्चात् स्म‚र्येते तौ पृथ‚क् क‚थ‚म् ॥ ५०९ ॥\&[\smallbreak]


	
	    \end{quote}
	  
	  \endgroup
	\textsuperscript{\textenglish{271/s}}

	  \pstart \leavevmode% starting standard par
	\hphantom{.}अर्थ‚श्चार्थ‚प्र‚त्य‚य‚श्चा‚{\color{DodgerBlue3}‚र्थार्थ‚प्र‚त्य‚यौ सूप‚ल‚क्षेण भेदेन संवित्ताव‚नुभ‚वे, न ल‚क्षितौ । ‚{\tiny $_{lb}$}‚तौ प‚श्चात् कालान्त‚रे पृथ‚ग्} भेदेनाय‚म‚र्थोऽर्थ‚ज्ञा\edtext{}{\edlabel{pvv.271-1}\label{pvv.271-1}\lemma{ज्ञा}\Bfootnote{स्म‚र्येते च त‚स्मान्न युग‚प‚द् द्व‚यं ।}} न‚ञ्चेद‚मिति ‚{\color{DodgerBlue3}‚क‚थं स्म‚र्येते} युग‚प‚द‚र्थं‚{\tiny $_{lb}$}‚ज्ञान‚त‚ज्ज्ञान‚योरुत्पादे प्र‚तिभास‚नानात्व‚प्र‚स‚ङ्गात् । (५०९)
	\pend% ending standard par
      \label{div_pvv.2.510}
	  
	% new div opening: depth here is 2
	
	  \bigskip
	  \begingroup
	
	    \large
	  
	    \begin{quote}
	  
	    
	    \stanza[\smallbreak]
	\label{pv.2.510}\flagstanza{\tiny\textenglish{....2.510}}क्र‚मेणानुभ‚वोत्पादेप्य‚र्थार्थ‚म‚न‚सोर‚य‚म् ।&प्र‚तिभास‚स्य नानात्व‚चोद्य‚दोषो दुरुद्ध‚रः ॥ ५१० ॥\&[\smallbreak]


	
	    \end{quote}
	  
	  \endgroup
	

	  \pstart \leavevmode% starting standard par
	\hphantom{.}अर्थार्थ‚म‚न‚सोः ‚{\color{DodgerBlue3}‚क्र‚मेणानुभ‚वोत्पादेपीष्य‚माणेऽयं प्र‚तिभास‚स्य नाना\edtext{}{\edlabel{pvv.271-2}\label{pvv.271-2}\lemma{नाना}\Bfootnote{पूर्व्वेन्द्रिय‚ज्ञान‚स्य म‚न‚सानुभ‚वेपि युग‚प‚द् भान‚म्मा भूद्विच्छिन्न‚मुत्प‚द्येत ।}}त्व‚चोद्य}‚{\tiny $_{4}$}‚‚{\tiny $_{lb}$}‚ल‚क्ष‚णो ‚{\color{DodgerBlue3}‚दोषो दुरुद्ध‚रः} । द्व‚य‚प्र‚तिभास‚स्य क्र‚मेणाभ्युप‚ग‚मात् । (५१०)
	\pend% ending standard par
      \label{div_pvv.2.511_2.512}
	  
	% new div opening: depth here is 2
	
	  \bigskip
	  \begingroup
	
	    \large
	  
	    \begin{quote}
	  
	    
	    \stanza[\smallbreak]
	\label{pv.2.511}\flagstanza{\tiny\textenglish{...५११}}अर्थ‚संवेद‚नं ताव‚त्त‚तोर्थाभास‚वेद‚न‚म् ।&न हि संवेद‚नं शुद्धं भ‚वेद‚र्थ‚स्य वेद‚न‚म्\edtext{}{\edlabel{pvv.271-asterisk}\label{pvv.271-asterisk}\lemma{म्}\Bfootnote{व्त्तिकृतात्त्व‚विवृतैषा ।}} ॥  ॥\&[\smallbreak]


	
	    \end{quote}
	  
	  \endgroup
	
	  \bigskip
	  \begingroup
	
	    \large
	  
	    \begin{quote}
	  
	    
	    \stanza[\smallbreak]
	\label{pv.2.512}\flagstanza{\tiny\textenglish{....2.512}}त‚था हि नीलाद्याकार एक एकं च वेद‚न‚म् ।&ल‚क्ष्य‚ते न तु नीलाभे वेद‚ने वेद‚नं प‚र‚म् ॥ ५१२ ॥\&[\smallbreak]


	
	    \end{quote}
	  
	  \endgroup
	

	  \pstart \leavevmode% starting standard par
	न चार्थ‚ज्ञानं ज्ञान‚ञ्च क‚दाचिद् भेदेन प्र‚तीय‚ते\edtext{}{\edlabel{pvv.271-3}\label{pvv.271-3}\lemma{ते}\Bfootnote{अर्थ‚ज्ञानं ज्ञान‚ज्ञानारूढ‚मिति ।}}। (५११) ‚{\color{DodgerBlue3}‚त‚था हि नीला‚{\tiny $_{lb}$}‚द्याकार एको} ग्राह्य‚त‚या ‚{\color{DodgerBlue3}‚एक‚ञ्च} त‚द्ग्राह‚कं ‚{\color{DodgerBlue3}‚वेद‚नं ल‚क्ष्य‚ते न तु नीलाभे वेद‚ने} पृथ‚ग् ‚{\tiny $_{lb}$}‚ग्राह‚क‚{\color{DodgerBlue3}‚म‚प‚रं वेद‚नं} ल‚क्ष्य‚ते । त‚स्माज्‏ज्ञान‚स्य विदित‚स्यान्येनानुभ‚वास‚म्भ‚वे स्व‚वेद\edtext{}{\edlabel{pvv.271-4}\label{pvv.271-4}\lemma{वेद}\Bfootnote{एकोऽर्थाकारः स‚म्वेद‚नाकार‚श्चेत्य‚प्र‚स‚ङ्गोऽत्र ।}}‚{\tiny $_{lb}$}‚न‚मेव त‚दिति व्य‚व‚तिष्ठ‚ते ॥ (५१२)
	\pend% ending standard par
      \label{div_pvv.2.513_2.514}
	  
	% new div opening: depth here is 2
	

	  \pstart \leavevmode% starting standard par
	किञ्च (।)
	\pend% ending standard par
      
	  \bigskip
	  \begingroup
	
	    \large
	  
	    \begin{quote}
	  
	    
	    \stanza[\smallbreak]
	\label{pv.2.513a}\flagstanza{\tiny\textenglish{...2.513a}}ज्ञानान्त‚रेणानुभ‚वो भ‚वेत्त‚त्रापि च स्मृतिः ।&दृष्टा; त‚द्वेद‚नं केन त‚स्याप्य‚न्येन चेत्;\&[\smallbreak]


	
	    \end{quote}
	  
	  \endgroup
	

	  \pstart \leavevmode% starting standard par
	नीलादिविष‚य‚स्य‚{\tiny $_{5}$}‚ ज्ञान‚स्य ‚{\color{DodgerBlue3}‚ज्ञानान्त‚रेणानुभ‚वो भ‚वेत्\edtext{}{\edlabel{pvv.271-5}\label{pvv.271-5}\lemma{वेत्}\Bfootnote{शास्त्र‚कृत् स्व‚यं प‚रित्य‚ज्याचार्यीय‚माह (।) न ह्य‚सौ स्मृतिर‚भावितेषु ‚{\tiny $_{lb}$}‚जात इत्याचार्येण स्व‚स‚म्वित्तिसाध‚नायोक्तं । प‚रेणात्र सिद्ध‚साध‚न‚मुक्तं विनापि ‚{\tiny $_{lb}$}‚स्व‚संवित्तिं ज्ञानान्त‚रेण संविद् भ‚विष्य‚तीति सिद्धान्तित‚म‚त्र । ज्ञानान्त‚रेणानु‚{\tiny $_{lb}$}‚भ‚वेऽनिष्टा त‚त्रापि हि स्मृतिः विष‚यान्त‚र‚स‚ञ्चार‚स्त‚था स्यात् स चेत् स इति ‚{\tiny $_{lb}$}‚व्याच‚ष्टे ।}} । त‚त्रापि ज्ञान-} ज्ञानेपि हि\edtext{}{\edlabel{pvv.271-6}\label{pvv.271-6}\lemma{हि}\Bfootnote{भ‚व‚तु नाम, त‚थापि चात्र स्मृतिर्द‚ष्टासीन् मे ज्ञान‚ज्ञान‚मिति ‚{\tiny $_{lb}$}‚सा न स्यात् स्व‚य‚न्त‚स्यान‚नुभ‚वात् ।}} ‚{\color{DodgerBlue3}‚स्मृति}‚र्दृष्ट । य‚दा ज्ञानान्त‚राल‚म्ब‚कं ज्ञानं क्र‚मेण स्म‚र्य‚ते न चागु‚{\tiny $_{lb}$}‚\leavevmode\ledsidenote{\textenglish{272/s}} हीतं स्म‚र्य‚ते इति त‚स्य ज्ञान‚ज्ञान‚स्य वेद‚नं व‚क्त‚व्यं । ‚{\color{DodgerBlue3}‚त‚त् केनास्तु वेद‚नं} य‚दि स्व‚स‚म्वे‚{\tiny $_{lb}$}‚द‚नेन ‚{\color{DodgerBlue3}‚त‚दा पूर्व्व‚क‚स्यापि} त‚था स्थितिर्व्य‚र्थ‚म‚न्येन वेद‚नाङ्गीक‚र‚णं (।) ‚{\color{DodgerBlue3}‚त‚स्याप्य‚{\tiny $_{lb}$}‚न्येन\edtext{}{\edlabel{pvv.272-1}\label{pvv.272-1}\lemma{न्येन}\Bfootnote{अन‚व‚स्था स्यादेवं ।}} चेद्वेद‚नं} त‚दा (।)
	\pend% ending standard par
      
	  \bigskip
	  \begingroup
	
	    \large
	  
	    \begin{quote}
	  
	    
	    \stanza[\smallbreak]
	\label{pv.2.513b}\flagstanza{\tiny\textenglish{...2.513b}}इमाम् ॥ ५१३ ॥\&[\smallbreak]


	
	    \end{quote}
	  
	  \endgroup
	
	  \bigskip
	  \begingroup
	
	    \large
	  
	    \begin{quote}
	  
	    
	    \stanza[\smallbreak]
	\label{pv.2.514}\flagstanza{\tiny\textenglish{....2.514}}मालां ज्ञान‚विदां कोयं ज‚न‚य‚त्य‚नुब‚न्धिनीम् ॥&पूर्व्वा धीः सैव चेन्न स्यात्स‚ञ्चारो विष‚यान्त‚रे ॥ ५१४ ॥\&[\smallbreak]


	
	    \end{quote}
	  
	  \endgroup
	

	  \pstart \leavevmode% starting standard par
	\hphantom{.}ज्ञान‚वृत्तीनामि‚{\color{DodgerBlue3}‚मां} (५१३) ‚{\color{DodgerBlue3}‚मा\edtext{}{\edlabel{pvv.272-2}\label{pvv.272-2}\lemma{मा}\Bfootnote{स्व‚कं विष‚यान्त‚रास‚ञ्चार‚न्ताव‚दाह त्य‚क्त्वाचार्यीयं ।}}लाम‚नुब‚न्धिनीं} प्र‚ब‚न्ध‚{\tiny $_{6}$}‚प्र‚वृत्तां ‚{\color{DodgerBlue3}‚को ज‚न‚य‚ति । ‚{\tiny $_{lb}$}‚न} ताव‚द‚र्थेन्द्रियादिसाम‚ग्री त‚स्या अर्थ‚ज्ञान‚मात्र‚ज‚न‚न‚व्यापार‚त्वात् । ‚{\color{DodgerBlue3}‚सैव} ग्राह्या ‚{\color{DodgerBlue3}‚धीः} पूर्व्विका स्व‚ग्राहिकां धियं ज‚न‚य‚ति सा च स्व‚ग्राहिकामिति । प्र‚ब‚न्ध‚प्र‚वृत्तिरिति ‚{\tiny $_{lb}$}‚‚{\color{DodgerBlue3}‚चेत्} । एवं स‚ति स्व‚कार‚ण‚ग्र‚ह‚ण‚प्र‚व‚ण‚त्वा‚{\color{DodgerBlue3}‚द्विष‚यान्त‚रे} ज्ञानाद‚न्य‚स्मिन्न‚र्थे ‚{\color{DodgerBlue3}‚स‚ञ्चारो} ग्राह‚क‚त्वेन प्र‚वृत्तिर्न ‚{\color{DodgerBlue3}‚स्यात्\edtext{}{\edlabel{pvv.272-3}\label{pvv.272-3}\lemma{स्यात्}\Bfootnote{एकार्थ‚विष‚यैव ज्ञान‚प‚र‚म्प‚राऽस‚ञ्चारं स्यात् ।}}} । (५१४)
	\pend% ending standard par
      \label{div_pvv.2.515}
	  
	% new div opening: depth here is 2
	

	  \pstart \leavevmode% starting standard par
	त‚था हि (।)
	\pend% ending standard par
      
	  \bigskip
	  \begingroup
	
	    \large
	  
	    \begin{quote}
	  
	    
	    \stanza[\smallbreak]
	\label{pv.2.515}\flagstanza{\tiny\textenglish{....2.515}}तां ग्राह्य‚ल‚क्ष‚ण‚प्राप्तामास‚न्नां ज‚निकां धिय‚म् ।&अगृहीत्वोत्त‚रं ज्ञानं गृह्णीयाद‚प‚रं क‚थ‚म् ॥ ५१५ ॥\&[\smallbreak]


	
	    \end{quote}
	  
	  \endgroup
	\textsuperscript{\textenglish{53b/MA}}

	  \pstart \leavevmode% starting standard par
	\hphantom{.}‚{\color{DodgerBlue3}‚ज‚निकां धिय‚मास‚न्नां ग्राह्य‚ल‚क्ष‚ण‚प्राप्ता}‚म‚गृहीत्वोत्त‚रं ‚{\color{DodgerBlue3}‚ज्ञान‚म‚प‚रं} विष‚यं‚{\tiny $_{7}$}‚ ‚{\color{DodgerBlue3}‚क‚थं ‚{\tiny $_{lb}$}‚गृह्णीयात्} । (५१५)
	\pend% ending standard par
      \label{div_pvv.2.516}
	  
	% new div opening: depth here is 2
	

	  \pstart \leavevmode% starting standard par
	प्र‚त्यास‚न्नेनार्थेन प्र‚तिब‚द्ध‚श‚क्तित्वात् पूर्व्वा धीर‚र्थ‚ग्राह‚क‚मेव ज्ञानं ज‚न‚य‚ति ‚{\tiny $_{lb}$}‚न स्व‚ग्राह‚क‚मिति चेत् । आह (।)
	\pend% ending standard par
      
	  \bigskip
	  \begingroup
	
	    \large
	  
	    \begin{quote}
	  
	    
	    \stanza[\smallbreak]
	\label{pv.2.516}\flagstanza{\tiny\textenglish{....2.516}}आत्म‚नि ज्ञान‚ज‚न‚ने स्व‚भावे निय‚ताञ्च ताम् ।&को नामान्यो विब‚ध्नीयाद् ब‚हिर‚ङ्गेऽन्त‚र‚ङ्‏गिकाम् ॥ ५१६ ॥\&[\smallbreak]


	
	    \end{quote}
	  
	  \endgroup
	

	  \pstart \leavevmode% starting standard par
	\hphantom{.}‚{\color{DodgerBlue3}‚आत्म‚नि विज्ञान‚ज‚न‚ने} स्व‚ग्राह‚क‚ज्ञानोत्पाद‚के ‚{\color{DodgerBlue3}‚स्व‚भावो निय‚तां} व्य‚व‚स्थितां च ‚{\tiny $_{lb}$}‚तामिमाम‚{\color{DodgerBlue3}‚न्त‚र‚ङ्गिकाम‚न्य}‚निर‚पेक्ष‚त्वात् ‚{\color{DodgerBlue3}‚को नामान्योर्थो ब‚हिर‚ङ्गः} स्व‚ज्ञान‚ज‚न‚ने ‚{\tiny $_{lb}$}‚च‚क्षुर्म‚न‚स्काराद्य‚पेक्षित्वा‚{\color{DodgerBlue3}‚द्विब‚ध्नीयात्} । स्व‚ग्राह‚क‚ज्ञान‚ज‚न‚न‚प्र‚वृत्तां तिर‚स्कुर्य्यात् । ‚{\tiny $_{lb}$}‚येन विष‚यान्त‚र‚स‚ञ्चारो धियः स्यात् । (५१६)
	\pend% ending standard par
      \label{div_pvv.2.517}
	  
	% new div opening: depth here is 2
	
	  \bigskip
	  \begingroup
	
	    \large
	  
	    \begin{quote}
	  
	    
	    \stanza[\smallbreak]
	\label{pv.2.517}\flagstanza{\tiny\textenglish{....2.517}}बाह्यः स‚न्निहितोप्य‚र्थः तां विब‚ध्न‚न् हि न प्र‚भुः ।&धियं नानुभ‚वेत् क‚श्चिद‚न्य‚थार्थ‚स्य स‚न्निधौ ॥ ५१७ ॥\&[\smallbreak]


	
	    \end{quote}
	  
	  \endgroup
	\textsuperscript{\textenglish{273/s}}

	  \pstart \leavevmode% starting standard par
	\hphantom{.}त‚स्माद्वा‚{\color{DodgerBlue3}‚ह्यः स‚न्निहितोप्य‚र्थ‚स्तां} स्व‚ग्राह‚क‚ज्ञान‚ज‚न‚ने श‚क्तां ‚{\color{DodgerBlue3}‚धियं विब‚ध्न‚न्} प्र‚तिहातुं ‚{\color{DodgerBlue3}‚न} प्र‚भुः श‚क्तः । ‚{\color{DodgerBlue3}‚अन्य‚था} य‚द्येवं नाभ्युप‚ग‚म्य‚ते ‚{\color{DodgerBlue3}‚त‚दार्थ‚स्य स‚न्निधौ} त‚त् ‚{\tiny $_{lb}$}‚ज्ञान‚स्यैवोत्पाद‚नात् ज्ञान‚ज्ञान‚स्यानुत्पादाद्धियं क‚श्चिन्नानुभ‚वेत् । (५१७)
	\pend% ending standard par
      \label{div_pvv.2.518}
	  
	% new div opening: depth here is 2
	
	  \bigskip
	  \begingroup
	
	    \large
	  
	    \begin{quote}
	  
	    
	    \stanza[\smallbreak]
	\label{pv.2.518}\flagstanza{\tiny\textenglish{....2.518}}न चास‚न्निहितार्थास्ति द‚शा काचिद‚तो धियः ।&उत्खात‚मूला स्मृतिर‚प्युत्स‚न्नेत्युज्ज्व‚लं म‚त‚म् ॥ ५१८ ॥\&[\smallbreak]


	
	    \end{quote}
	  
	  \endgroup
	

	  \pstart \leavevmode% starting standard par
	\hphantom{.}‚{\color{DodgerBlue3}‚न च काचिद्द‚शा स‚न्निहितार्था/?/\edtext{\textsuperscript{*}}{\edlabel{pvv.273-1}\label{pvv.273-1}\lemma{*}\Bfootnote{प‚ञ्च‚स्क‚न्ध‚के भ‚वे स‚दैवार्थः ।}} /?/स्ति} य‚त्र ज्ञान‚ज्ञान‚मुत्प‚द्येत नार्थ‚ज्ञान‚मिति ‚{\tiny $_{lb}$}‚न स्याद‚नुभ‚वो बुद्धेः । त‚स्माद् ‚{\color{DodgerBlue3}‚धियः स्मृतिर‚प्यु‚{\tiny $_{2}$}‚ त्खात‚मूला उत्स‚न्ना} बुद्धिस्म‚र‚ण‚{\tiny $_{lb}$}‚स्य हि मूलं बुद्ध्य‚नुभ‚व ‚{\color{DodgerBlue3}‚इति} ज्ञानान‚नुभ‚वे कुतः स्म‚र‚ण‚मिति ज्ञानान्त‚रेण ‚{\tiny $_{lb}$}‚ज्ञानानुभ‚व‚वादिना‚{\color{DodgerBlue3}‚मुज्ज्व‚लं म‚त‚मि}‚त्युप‚ह‚स‚ति । (५१८) किञ्च (।)
	\pend% ending standard par
      \label{div_pvv.2.519}
	  
	% new div opening: depth here is 2
	
	  \bigskip
	  \begingroup
	
	    \large
	  
	    \begin{quote}
	  
	    
	    \stanza[\smallbreak]
	\label{pv.2.519}\flagstanza{\tiny\textenglish{....2.519}}अतीतादिविक‚ल्पानां येषां नार्थ‚स्य स‚न्निधिः ।&स‚ञ्चार‚क‚र‚णाभावादुत्सीदेद‚र्थ‚चिन्त‚न‚म् ॥ ५१९ ॥\&[\smallbreak]


	
	    \end{quote}
	  
	  \endgroup
	

	  \pstart \leavevmode% starting standard par
	येषाम‚तीताद्य‚र्थं\edtext{}{\edlabel{pvv.273-2}\label{pvv.273-2}\lemma{र्थं}\Bfootnote{भ‚व‚तु नामाध्य‚क्षेषु विष‚येषु स‚ञ्चारोऽतीतानाग‚त‚विक‚ल्पे तु नार्थो यः ।}}विष‚याणां विक‚ल्पानाम‚र्थ‚स्य स‚न्निधिर्नास्ति यः पूर्व्व‚क‚स्य ‚{\tiny $_{lb}$}‚ज्ञान‚स्य स्व‚ग्राह‚क‚ज्ञान‚ज‚न‚न‚श‚क्तिं प्र‚तिब‚ध्नीयात् येन विष‚यान्त‚र‚ग्राहिणो विक‚ल्पाः ‚{\tiny $_{lb}$}‚स्युः । त‚त्र स‚ञ्चार‚कार‚ण‚स्या‚{\tiny $_{3}$}‚र्थ‚स्य विक‚ल्प‚विष‚य‚स्याभावात् क‚स्य‚चि‚{\color{DodgerBlue3}‚द‚तीतादेर‚र्थ}‚स्य ‚{\tiny $_{lb}$}‚चिन्त‚नं ‚{\color{DodgerBlue3}‚विक‚ल्प‚न‚मुत्सीदे}‚त् । (५१९)
	\pend% ending standard par
      \label{div_pvv.2.520_2.521}
	  
	% new div opening: depth here is 2
	

	  \pstart \leavevmode% starting standard par
	स्यादेत‚त् (।)
	\pend% ending standard par
      
	  \bigskip
	  \begingroup
	
	    \large
	  
	    \begin{quote}
	  
	    
	    \stanza[\smallbreak]
	\label{pv.2.520a}\flagstanza{\tiny\textenglish{...2.520a}}आत्म‚विज्ञान‚ज‚न‚ने श‚क्तिसंक्ष‚य‚तः श‚नैः ।&विष‚यान्त‚र‚स‚ञ्चारो य‚दि;\&[\smallbreak]


	
	    \end{quote}
	  
	  \endgroup
	

	  \pstart \leavevmode% starting standard par
	\hphantom{.}त‚ज्ज्ञानानामा‚{\color{DodgerBlue3}‚त्म‚नि} ग्राह‚क‚{\color{DodgerBlue3}‚ज्ञान‚ज‚न‚ने श‚नैः} क्र‚मेण स्व‚ग्राह‚क‚ज्ञान‚ज‚निकायाः ‚{\tiny $_{lb}$}‚‚{\color{DodgerBlue3}‚श‚क्तेः संक्ष‚य‚तः} त‚थाभूत‚ज्ञानानुत्प‚त्तौ ‚{\color{DodgerBlue3}‚विष‚यान्त‚रे स‚ञ्चारो} ज्ञान‚प्र‚वृत्ति‚{\color{DodgerBlue3}‚र्य‚दि} क‚थ्य‚ते ।
	\pend% ending standard par
      
	  \bigskip
	  \begingroup
	
	    \large
	  
	    \begin{quote}
	  
	    
	    \stanza[\smallbreak]
	\label{pv.2.520b}\flagstanza{\tiny\textenglish{...2.520b}}सैवार्थ‚धीः कुतः ॥ ५२० ॥\&[\smallbreak]


	
	    \end{quote}
	  
	  \endgroup
	
	  \bigskip
	  \begingroup
	
	    \large
	  
	    \begin{quote}
	  
	    
	    \stanza[\smallbreak]
	\label{pv.2.521a}\flagstanza{\tiny\textenglish{...2.521a}}श‚क्तिक्ष‚ये पूर्व‚धियो न हि धीः प्राग्धियां विना ।\&[\smallbreak]


	
	    \end{quote}
	  
	  \endgroup
	

	  \pstart \leavevmode% starting standard par
	\hphantom{.}त‚दा पूर्व्व‚धियः श‚क्तिक्ष‚ये स‚ति ‚{\color{DodgerBlue3}‚सैवार्थ}‚ग्राहिका ‚{\color{DodgerBlue3}‚धीः कुतो} जा\edtext{}{\edlabel{pvv.273-3}\label{pvv.273-3}\lemma{जा}\Bfootnote{अर्थ‚बुद्धिज‚न‚नापेक्ष‚या पूर्व्वाऽन्त्या त‚स्याः क्ष‚ये स‚र्व्वा बुद्धिं न ज‚न‚येद‚विशेषात् ।}}य‚ते । (५२०) ‚{\tiny $_{lb}$}‚न हि प्रा‚{\tiny $_{4}$}‚ग्धिया पूर्व्विकां धियं स‚म‚र्थां विनोत्त‚र‚म‚र्थ‚ज्ञान‚मुत्प‚द्य‚ते ।
	\pend% ending standard par
      
	  \bigskip
	  \begingroup
	
	    \large
	  
	    \begin{quote}
	  
	    
	    \stanza[\smallbreak]
	\label{pv.2.521b}\flagstanza{\tiny\textenglish{...2.521b}}अन्यार्थाश‚क्तिविगुणे ज्ञाने ज्ञानोद‚याग‚तेः ॥ ५२१ ॥\&[\smallbreak]


	
	    \end{quote}
	  
	  \endgroup
	\textsuperscript{\textenglish{274/s}}

	  \pstart \leavevmode% starting standard par
	य‚दि च सा-श‚क्त‚त्वात् स्व‚ग्राहिकाम्धियं क‚र्त्तुम‚स‚म‚र्थार्थ‚धिय‚म‚पि न कुर्य्यात् ।\edtext{\textsuperscript{*}}{\edlabel{pvv.274-1}\label{pvv.274-1}\lemma{*}\Bfootnote{कुत एत‚दिति युक्तिमाह । स‚न्निहितेपि विष‚ये ज्ञानोत्पादानुप‚ल‚ब्धेः ।}} ‚{\tiny $_{lb}$}‚त‚था ह्य‚न्य‚स्मि‚{\color{DodgerBlue3}‚न्न‚र्थे आस‚क्ति}‚र‚भिष्व‚ङ्ग‚स्त‚या ‚{\color{DodgerBlue3}‚विगुणे} पूर्व्वे ‚{\color{DodgerBlue3}‚विज्ञाने} त‚दुत्त‚र‚स्य ‚{\tiny $_{lb}$}‚‚{\color{DodgerBlue3}‚ज्ञान‚स्योद‚याग‚तेः} ज‚न्माप्र‚तीतेः पूर्व्व‚बुद्धेः साम‚र्थ्यादुत्त‚र‚बुद्धेर्ज‚न्मेति निश्चीय‚ते (।) ‚{\tiny $_{lb}$}‚त‚तः स‚म‚र्था पूर्व्व‚बुद्धिः स्व‚ग्राहिणीमेव धि‚{\tiny $_{5}$}‚यं ज‚न‚येत् । त‚त्र प‚रापेक्षाविर‚हात् । ‚{\tiny $_{lb}$}‚(५२१)
	\pend% ending standard par
      \label{div_pvv.2.522}
	  
	% new div opening: depth here is 2
	

	  \pstart \leavevmode% starting standard par
	न\edtext{}{\edlabel{pvv.274-2}\label{pvv.274-2}\lemma{न}\Bfootnote{विज्ञान‚वादिन आल‚य (विज्ञानं) विविध‚वास‚नाभावितं न‚दीस्योतोव‚द‚विर‚तं ‚{\tiny $_{lb}$}‚इन्द्रिय‚ज्ञानानां प्र‚व‚र्त‚कं ।}}न्वा ल य वि ज्ञा नात् स‚कृत् ष‚ट् प्र‚वृत्तिविज्ञानानि जाय‚न्त इतीष्य‚ते ‚{\tiny $_{lb}$}‚त‚त‚त्सान्य‚र्थाश‚क्तिवैगुण्येपि पू\edtext{}{\edlabel{pvv.274-3}\label{pvv.274-3}\lemma{पू}\Bfootnote{म‚न‚सः ।}}र्व्व‚चेत‚स आल‚य‚ज्ञानान्त‚रं जायेर‚न् आल‚य‚ज्ञान‚{\tiny $_{lb}$}‚स्याव्याह‚त‚श‚क्तित्वादित्याह\edtext{}{\edlabel{pvv.274-4}\label{pvv.274-4}\lemma{क्तित्वादित्याह}\Bfootnote{न च प्र‚व‚र्त‚न्ते इति म‚न एवोपादानं नाल‚यं ।}} (।)
	\pend% ending standard par
      
	  \bigskip
	  \begingroup
	
	    \large
	  
	    \begin{quote}
	  
	    
	    \stanza[\smallbreak]
	\label{pv.2.522}\flagstanza{\tiny\textenglish{....2.522}}स‚कृद्विजातीय‚जाताव‚प्येंकेन प‚टीय‚सा ।&चित्तेनाहित‚वैगुण्यादाल‚यान्नान्य‚स‚म्भ‚वः ॥ ५२२ ॥\&[\smallbreak]


	
	    \end{quote}
	  
	  \endgroup
	

	  \pstart \leavevmode% starting standard par
	स‚कृद्विजातीया\edtext{}{\edlabel{pvv.274-5}\label{pvv.274-5}\lemma{कृद्विजातीया}\Bfootnote{आल‚याद्विजातीय‚त्वं । म‚न‚सो व्याकृत‚त्वेन, ऐन्द्रियाणां च कादाचित्क‚त्वेन ।}}नां प्र‚वृतिज्ञानानां जाताव‚प्याल‚य‚ज्ञानादिष्टायामेकेन चित्तेन ‚{\tiny $_{lb}$}‚‚{\color{DodgerBlue3}‚स्व‚विष‚यास‚क्तेन} प‚टीय‚सा विष‚या‚{\tiny $_{6}$}‚न्त‚र‚ज्ञान‚ज‚न‚नं प्र‚त्याहित‚मारोपितं ‚{\color{DodgerBlue3}‚वैगुण्यं ‚{\tiny $_{lb}$}‚य‚स्य त‚स्मादाल‚याद‚न्य}‚स्य विष‚यान्त‚र‚ग्राहिज्ञान‚स्य ‚{\color{DodgerBlue3}‚स‚म्भ‚वो न} भ‚व‚ति । (५२२)
	\pend% ending standard par
      \label{div_pvv.2.523}
	  
	% new div opening: depth here is 2
	

	  \pstart \leavevmode% starting standard par
	न ह्याल‚य‚ज्ञान‚मित्येव प्र‚वृत्तिज्ञानानि भ‚व‚न्ति (।) किन्तु म‚न‚स्कार‚साद्गुण्य‚म‚{\tiny $_{lb}$}‚‚{\color{DodgerBlue3}‚पेक्ष‚न्तेऽन्य‚था} य‚द्येवं नेष्य‚ते त‚दा (।)
	\pend% ending standard par
      
	  \bigskip
	  \begingroup
	
	    \large
	  
	    \begin{quote}
	  
	    
	    \stanza[\smallbreak]
	\label{pv.2.523}\flagstanza{\tiny\textenglish{....2.523}}नापेक्षेतान्य‚था साम्यं म‚नोवृत्तेर्म‚नोन्त‚र‚म् ।&म‚नोज्ञान‚क्र‚मोत्प‚त्तिर‚प्य‚पेक्षा-प्र‚साध‚नी ॥ ५२३ ॥\&[\smallbreak]


	
	    \end{quote}
	  
	  \endgroup
	\textsuperscript{\textenglish{54a/MA}}

	  \pstart \leavevmode% starting standard par
	\hphantom{.}‚{\color{DodgerBlue3}‚म‚न‚सः स‚म‚न‚न्त‚र}‚प्र‚त्य‚य‚स्य वृत्तेः ‚{\color{DodgerBlue3}‚साम्य\edtext{}{\edlabel{pvv.274-6}\label{pvv.274-6}\lemma{साम्य}\Bfootnote{आल‚याद्विजातीय‚त्वं । अवैगुण्यं साम्यं ।}}म}‚र्थान्त‚रानास‚क्त‚त्वेनानुकूल्य‚मुत्पित्सु ‚{\tiny $_{lb}$}‚‚{\color{DodgerBlue3}‚म‚नोऽन्त‚रं नापेक्षे}‚{\tiny $_{7}$}‚त । अपेक्ष‚ते च । त‚तो म‚न‚स्कारानुकूल‚ता-सापेक्षादाल‚यात् ‚{\tiny $_{lb}$}‚ज्ञानोत्प‚त्तिः । त‚था ‚{\color{DodgerBlue3}‚म‚नोज्ञानानां} विक‚ल्पानां ‚{\color{DodgerBlue3}‚क्र‚मेणोत्प‚त्ति}‚र्युग‚प‚द‚नुत्पादो‚{\color{DodgerBlue3}‚प्यु}‚त्त‚र‚ज्ञा‚{\tiny $_{lb}$}‚नानां ‚{\color{DodgerBlue3}‚पूर्व्व‚ज्ञानापेक्षा प्र‚साध‚नी} । (५२३)
	\pend% ending standard par
      \label{div_pvv.2.524}
	  
	% new div opening: depth here is 2
	
	  \bigskip
	  \begingroup
	
	    \large
	  
	    \begin{quote}
	  
	    
	    \stanza[\smallbreak]
	\label{pv.2.524}\flagstanza{\tiny\textenglish{....2.524}}एक‚त्वान्म‚न‚सोन्य‚स्मिन्स‚क्त‚स्यान्याग‚तेर्य‚दि ।&ज्ञानान्त‚र‚स्यानुद‚यो न क‚दाचित्स‚होद‚यात् ॥ ५२४ ॥\&[\smallbreak]


	
	    \end{quote}
	  
	  \endgroup
	\textsuperscript{\textenglish{275/s}}

	  \pstart \leavevmode% starting standard par
	अविगुण‚पूर्व्व‚ज्ञानान‚पेक्षायामाल‚यादिन्द्रिय‚ज्ञानानीव विक‚ल्प‚ज्ञानान्य‚पि स‚ह ‚{\tiny $_{lb}$}‚जायेर‚न् ।\edtext{\textsuperscript{*}}{\edlabel{pvv.275-1}\label{pvv.275-1}\lemma{*}\Bfootnote{य‚तो विष‚यान्त‚रादिदुक्षुः स्यात् ।}} अणुप‚रिमाण‚स्य नित्य‚स्यैक‚स्य ‚{\color{DodgerBlue3}‚म‚न‚सोऽन्य‚स्मिन्नि}‚न्द्रिये ‚{\color{DodgerBlue3}‚स‚क्त‚स्यान्य‚{\tiny $_{lb}$}‚त्रेन्द्रियान्त‚{\tiny $_{1}$}‚}‚रेऽग‚तेर‚ग‚म‚नादिन्द्रियान्त‚र‚ज‚स्य ‚{\color{DodgerBlue3}‚ज्ञानान्त‚र‚स्यानुद‚यो} य‚दि स‚म्म‚तः । ‚{\tiny $_{lb}$}‚सो पि न युक्तः । ‚{\color{DodgerBlue3}‚क‚दाचिद्} न‚र्त‚कीदृष्ट्‚{\tiny $_{2}$}‚य‚व‚स्थादिष्व‚नेक‚विष‚य‚स‚न्निपाते च‚क्षुरादि\edtext{}{\edlabel{pvv.275-2}\label{pvv.275-2}\lemma{क्षुरादि}\Bfootnote{स‚कृद‚पि दीर्घ‚श‚ष्कुलीस‚म‚र्म‚र‚र्ध्वानं स‚मुद्व‚ह‚द्ब‚ह‚लामोदां ध‚व‚लादिर‚सान्वितां ‚{\tiny $_{lb}$}‚अतिस्प‚र्श‚व‚तीं कुड्यां मिश्र‚तो (? ता) नेकेन्द्रिय‚ज्ञान‚स्य स‚म्वेद‚नात् स्फ‚टिक‚तुल्ये ‚{\tiny $_{lb}$}‚स‚म‚न‚न्त‚रे स‚कृत्संग‚त‚स‚र्व्वार्थेष्विन्द्रियेष्व‚स‚त्स्व‚पीत्यादौ साधितं प्राक् ।}}‚{\tiny $_{lb}$}‚ज्ञानानां स‚होद‚यात् ।(५२४)
	\pend% ending standard par
      \label{div_pvv.2.525}
	  
	% new div opening: depth here is 2
	

	  \pstart \leavevmode% starting standard par
	य‚दा च न प‚टीयान् क‚श्चिद्विष‚यः प्र‚त्युप‚तिष्ठ‚ते पुरुष‚स्य च न क्व‚चिद्वि‚{\tiny $_{lb}$}‚शेषेणेच्छा भ‚व‚ति त‚दा\edtext{}{\edlabel{pvv.275-3}\label{pvv.275-3}\lemma{दा}\Bfootnote{वैशेषिक आह (।) अणु म‚नोद्र‚व्यं पृथिव्यादिद्र‚व्येषु प‚तितं । ज्ञान‚न्त्वा‚{\tiny $_{lb}$}‚त्म‚गुणो न‚व‚स्वात्म‚गुणेषु म‚ध्ये पाठात् ।}}(।)
	\pend% ending standard par
      
	  \bigskip
	  \begingroup
	
	    \large
	  
	    \begin{quote}
	  
	    
	    \stanza[\smallbreak]
	\label{pv.2.525}\flagstanza{\tiny\textenglish{....2.525}}स‚म‚वृत्तौ च तुल्य‚त्त्वात्स‚र्व्व‚दान्याग‚तिर्भ‚वेत् ।&ज‚न्म चात्म‚म‚नोयोग‚मात्र‚जानां स‚कृद् भ‚वेत् ॥ ५२५ ॥\&[\smallbreak]


	
	    \end{quote}
	  
	  \endgroup
	

	  \pstart \leavevmode% starting standard par
	\hphantom{.}अर्थ‚पुरुषेच्छायाः ‚{\color{DodgerBlue3}‚स‚मा}‚यां साधार‚णायां ‚{\color{DodgerBlue3}‚वृत्तौ च} म‚न‚स एकेन्द्रिय‚स‚म्ब‚द्ध‚स्य ‚{\tiny $_{lb}$}‚तुल्य‚{\tiny $_{2}$}‚त्वाद्विष‚यान्त‚रे प्रेर‚काभावात् उदासीन‚त्वात् ‚{\color{DodgerBlue3}‚स‚र्व्व‚दा} त‚दिन्द्रिय‚ज्ञानोत्प‚त्तौ ‚{\tiny $_{lb}$}‚स‚त्या‚{\color{DodgerBlue3}‚म‚न्य}‚स्येन्द्रियान्त‚र‚ज्ञान‚ज्ञेय‚स्या‚{\color{DodgerBlue3}‚ग‚तिर्भ‚वेत्} प्र‚तीतिर्न स्यात् । अस्ति च क्व‚चिद‚{\tiny $_{lb}$}‚नास‚क्त‚स्य विष‚यान्त‚रेष्वेक‚स्याप्य‚नेकार्थ‚द‚र्श‚नं ।\edtext{\textsuperscript{*}}{\edlabel{pvv.275-4}\label{pvv.275-4}\lemma{*}\Bfootnote{य‚त्रात्मा म‚न‚सा त‚दिन्द्रियेण त‚द्विष‚येण ‚{\tiny $_{lb}$}‚त‚त्र क्र‚मोत्प‚त्तिक‚ल्प‚नैवं स्याद‚पि । आत्म‚म‚न‚सोर्नित्य‚त्वाद‚विशेषात्त‚त्प्र‚तिब‚द्ध‚स‚न्नि‚{\tiny $_{lb}$}‚क‚र्षोप्य‚विशिष्टः ।}} ‚{\color{DodgerBlue3}‚आत्म‚म‚नोयोग‚मात्र‚जानां} विष‚या‚{\tiny $_{lb}$}‚निर‚पेक्षाणां सुखादिज्ञानानामिन्द्रिय‚म‚नोयोग‚विशेष‚स्य निया\edtext{}{\edlabel{pvv.275-5}\label{pvv.275-5}\lemma{निया}\Bfootnote{न क्र‚म‚नियाम‚कं म‚न एक‚त्वात् स‚कृदात्म‚संयोगात् ।}}म‚क‚स्याभावात् ‚{\tiny $_{lb}$}‚‚{\color{DodgerBlue3}‚स‚कृद्} वा ‚{\color{DodgerBlue3}‚ज‚न्म} स्या‚{\tiny $_{3}$}‚ त् । (२२५)
	\pend% ending standard par
      \label{div_pvv.2.526}
	  
	% new div opening: depth here is 2
	
	  \bigskip
	  \begingroup
	
	    \large
	  
	    \begin{quote}
	  
	    
	    \stanza[\smallbreak]
	\label{pv.2.526}\flagstanza{\tiny\textenglish{....2.526}}एकैव चैत्क्रियैकः स्यात् किन्दीपोऽनेक‚द‚र्श‚नः ।&क्र‚मेणापि न श‚क्तं स्यात्प‚श्चाद‚प्य‚विशेष‚तः ॥ ५२६ ॥\&[\smallbreak]


	
	    \end{quote}
	  
	  \endgroup
	

	  \pstart \leavevmode% starting standard par
	न हि रूपादिबुद्धीनामेव सुखादिबुद्धीनामिन्द्रिय‚म‚नोयोग‚विशेषात् प्र‚तिनिय‚मः ‚{\tiny $_{lb}$}‚श‚क्यो व‚क्तुं एक‚स्मान्म‚न\edtext{}{\edlabel{pvv.275-6}\label{pvv.275-6}\lemma{न}\Bfootnote{आत्मानं क‚र्त्तार‚म‚पेक्ष्य कार‚ण‚भूतात् ।}}स ‚{\color{DodgerBlue3}‚एकैव} सुखादिबुद्धिल‚क्ष‚णा ‚{\color{DodgerBlue3}‚क्रिया} जाय‚ते न च द्वे ‚{\tiny $_{lb}$}‚इति ‚{\color{DodgerBlue3}‚चेत्} । य‚द्येवं ‚{\color{DodgerBlue3}‚किं} क‚स्मा‚{\color{DodgerBlue3}‚द्दीप} एकोऽ‚{\color{DodgerBlue3}‚नेक‚द‚र्श‚नो}‚ऽनेक‚द्र‚ष्टृज्ञान‚ज‚न‚कः । द्र‚ष्टुर‚{\tiny $_{lb}$}‚\leavevmode\ledsidenote{\textenglish{276/s}} नेक‚त्वादेको\edtext{}{\edlabel{pvv.276-1}\label{pvv.276-1}\lemma{त्वादेको}\Bfootnote{म‚न‚सः क‚र‚णानां न स्वात‚न्त्र्यं क‚र्तृव‚श‚वृत्तेः क‚र्त्रोरैक्यै (इ) न्द्रियैक्य‚म‚नेक‚त्वे ‚{\tiny $_{lb}$}‚तु एक‚क‚र‚णेनाप्य‚नेका क्रिया प‚र‚स्य ।}}प्य‚नेक‚ज्ञान‚ज‚न‚क इति चेत् । ज्ञेय‚स्यानेक‚त्वात्त‚द्ग्राह‚कानेक\edtext{}{\edlabel{pvv.276-2}\label{pvv.276-2}\lemma{कानेक}\Bfootnote{पूर्व्वानुवाद (:) । न त‚र्हि म‚नो नियाम‚कं ज्ञानानामात्मैव स‚वास‚नो ‚{\tiny $_{lb}$}‚नियाम‚कः स्यात् ।}}‚{\tiny $_{lb}$}‚ज्ञान ‚{\tiny $_{4}$}‚ज‚न\edtext{}{\edlabel{pvv.276-3}\label{pvv.276-3}\lemma{न}\Bfootnote{युग‚प‚त्}}कोपि स्यादिति स‚मानं । किञ्च (।) म‚नो नित्य‚मेक‚दापि स‚र्व्व‚ज्ञान‚ज‚न‚न‚{\tiny $_{lb}$}‚श‚क्तं न वा । श‚क्त‚ञ्चेत् स‚र्व्व‚ज्ञानानि स‚कृत् कुर्यात् । अश‚क्त‚ञ्चेत् । ‚{\color{DodgerBlue3}‚क्र‚मेणापि} त‚ज्ज‚न‚ने ‚{\color{DodgerBlue3}‚न श‚क्तं} स्यात् । पूर्व्वाव‚स्थित‚स्याश‚क्त‚स्य रूप‚स्य ‚{\color{DodgerBlue3}‚प‚श्चाद‚प्य‚विशेष‚तो ‚{\tiny $_{lb}$}‚विशेषाभावात्\edtext{}{\edlabel{pvv.276-4}\label{pvv.276-4}\lemma{विशेषाभावात्}\Bfootnote{न स‚ह‚कार्य‚पेक्षाप्य‚नाधेयातिश‚य‚स्य ।}}} । (५२६)
	\pend% ending standard par
      \label{div_pvv.2.527}
	  
	% new div opening: depth here is 2
	
	  \bigskip
	  \begingroup
	
	    \large
	  
	    \begin{quote}
	  
	    
	    \stanza[\smallbreak]
	\label{pv.2.527}\flagstanza{\tiny\textenglish{....2.527}}अनेन देह‚पुरुषावुक्तौ संस्कार‚तो य‚दि ।&निय‚मः स कुतः प‚श्चात् बुद्धेश्चेद‚स्तु स‚म्म‚त‚म् ॥ ५२७ ॥\&[\smallbreak]


	
	    \end{quote}
	  
	  \endgroup
	

	  \pstart \leavevmode% starting standard par
	\hphantom{.}‚{\color{DodgerBlue3}‚अनेन} म‚न‚सः श‚क्त‚त्वाश‚क्त‚त्व‚विक‚ल्पाद्दोष‚द्व‚येन ‚{\color{DodgerBlue3}‚देह‚पुरुषौ} श‚रीरात्माना‚{\tiny $_{lb}$}‚‚{\color{DodgerBlue3}‚वुक्तावु}‚क्तोत्त‚रौ । त‚यो‚{\tiny $_{5}$}‚र‚पि त‚था\edtext{}{\edlabel{pvv.276-5}\label{pvv.276-5}\lemma{था}\Bfootnote{एक‚देहादिक‚ज्ञानं वैभाष्य‚स्य पुद्ग‚ल‚नामा पुरुषोपि । स सांख्य‚स्यात्मा इष्टः ।}}विक‚ल्पे त‚थादोष‚स‚द्भावात् ज्ञान‚जात‚{\tiny $_{lb}$}‚ज्ञान‚हेतोः ‚{\color{DodgerBlue3}‚संस्कार‚तो} बुद्धीनाम‚स‚कृद्-‚{\color{DodgerBlue3}‚भाव‚निय‚मो य‚दीष्ट}‚स्त‚दा संस्कार‚स्य ‚{\color{DodgerBlue3}‚कुतः ‚{\tiny $_{lb}$}‚प‚श्चा\edtext{}{\edlabel{pvv.276-6}\label{pvv.276-6}\lemma{श्चा}\Bfootnote{नित्य‚त्वे स‚कृद्भाव एव अनित्य‚त्वे तु प्र‚श्नः ।}} दु}‚त्प‚न्नो बुद्धीर्निय‚म‚येत् । बुद्धेः पूर्व्विकाया उप‚प‚द्य‚त इति चेत् (।) अस्तु ‚{\tiny $_{lb}$}‚पूर्व्व‚स्या ‚{\color{DodgerBlue3}‚बुद्धे}‚र्व्वास‚नासंज्ञ‚क‚स्य संस्कारः स बुद्ध्यात्म‚नो ज‚न्म ‚{\color{DodgerBlue3}‚स‚म्म‚त}‚म‚स्माकं ‚{\tiny $_{lb}$}‚(।) त‚था च य‚दुक्तं न हि धीः प्राग् धिया विने\href{http://sarit.indology.info/?cref=pv.2.521}{(२।५२१)}ति त‚देव प्र‚साधितं ‚{\tiny $_{lb}$}‚स्यात् । त‚त‚श्च बुद्धेर्बुद्ध्य‚{\tiny $_{6}$}‚न्त‚र‚ज‚न्म‚नः साम‚र्थ्यादुत्त‚र‚या धिया पूर्व्व‚बुद्धिर्गृह्येतेति न ‚{\tiny $_{lb}$}‚स्याद्विष‚यान्त‚र‚स‚ञ्चारोस्याः । (५२७)
	\pend% ending standard par
      \label{div_pvv.2.528}
	  
	% new div opening: depth here is 2
	

	  \pstart \leavevmode% starting standard par
	स्यादेत‚त् । बुद्धेरुपादान‚तामात्रं बुद्ध्य‚न्त‚रं प्र‚ति न त‚द्ग्राह्य‚ता त‚तो ‚{\tiny $_{lb}$}‚विष‚यान्त‚र‚स‚ञ्चारो भ‚वेदित्याह (।)
	\pend% ending standard par
      
	  \bigskip
	  \begingroup
	
	    \large
	  
	    \begin{quote}
	  
	    
	    \stanza[\smallbreak]
	\label{pv.2.528a}\flagstanza{\tiny\textenglish{...2.528a}}न ग्राह्य‚तान्या ज‚न‚नाज्ज‚न‚नं ग्राह्य‚ल‚क्ष‚ण‚म् ।&अग्राह्यं न हि तेजोस्ति;\&[\smallbreak]


	
	    \end{quote}
	  
	  \endgroup
	

	  \pstart \leavevmode% starting standard par
	\hphantom{.}‚{\color{DodgerBlue3}‚न} बुद्ध्य‚न्त‚र‚{\color{DodgerBlue3}‚ज‚न‚नाद‚न्या} त‚द्ग्राह्य‚ता । किन्त‚र्हि ज‚न‚न‚मेव ‚{\color{DodgerBlue3}‚ग्राह्य‚स्य ल‚क्ष‚णं} । ‚{\tiny $_{lb}$}‚त‚च्चेद‚स्ति कुतो ग्राह्य‚ताया अभावः (।) ‚{\color{DodgerBlue3}‚न हि तेजो} ज्योतिर्बुद्धेर्ज‚न‚क‚त‚याऽ‚{\tiny $_{lb}$}‚\leavevmode\ledsidenote{\textenglish{277/s}} ‚{\color{DodgerBlue3}‚ग्राह्य‚म‚स्ति} । त‚स्मा‚{\tiny $_{7}$}‚\edtext{}{\edlabel{pvv.277-1}\label{pvv.277-1}\lemma{स्मा}\Bfootnote{आलोको य‚दि न ग्राह्यो न ज‚न‚येदेव बुद्धिं ।}}दिद‚मेव बुद्धिं प्र‚ति ग्राह्य‚त्वं ग्राह्य‚स्य य‚त्त‚त् ज्ञानं नाम ॥
	\pend% ending standard par
      \textsuperscript{\textenglish{54b/MA}}‚{\tiny $_{lb}$}‚

	  \pstart \leavevmode% starting standard par
	न‚नु च तेज‚सो बुद्धिज‚न‚क‚स्य सूक्ष्मो व्य‚व‚हितः स‚न्निहित‚त‚र‚श्च क‚श्चिद‚व‚य‚वो ‚{\tiny $_{lb}$}‚य‚था न गृह्य‚ते त‚था ज‚निकापि बुद्धिर्न गृह्येतेत्याह\edtext{}{\edlabel{pvv.277-2}\label{pvv.277-2}\lemma{गृह्येतेत्याह}\Bfootnote{आलोकेप्य‚ग्राह्य‚त्वं ज‚न‚क‚त्वेन दृष्ट‚मेव । अभ्युप‚ग‚म्याह । आलोकः ‚{\tiny $_{lb}$}‚साव‚य‚व इति स्याद‚प्य‚ग्राह्य‚ता न त्व‚मूर्त्ते ज्ञाने ।}} ।
	\pend% ending standard par
      
	  \bigskip
	  \begingroup
	
	    \large
	  
	    \begin{quote}
	  
	    
	    \stanza[\smallbreak]
	\label{pv.2.528b}\flagstanza{\tiny\textenglish{...2.528b}}न च सौक्ष्म्याद्य‚नंश‚के ॥ ५२८ ॥\&[\smallbreak]


	
	    \end{quote}
	  
	  \endgroup
	

	  \pstart \leavevmode% starting standard par
	\hphantom{.}‚{\color{DodgerBlue3}‚अनंश‚के} निर‚व‚य‚वे ज्ञाने ‚{\color{DodgerBlue3}‚सौक्ष्म्यं} सूक्ष्म‚त्वं नास्ति । आदिश‚ब्दात् स्व‚स‚न्तान‚{\tiny $_{lb}$}‚व‚र्तित्वाव्य‚व‚धानात्यास‚त्त्याद‚य‚श्चानुप‚ल (म्भ) हेत‚वो न स‚न्ति । (५२८)
	\pend% ending standard par
      \label{div_pvv.2.529}
	  
	% new div opening: depth here is 2
	

	  \pstart \leavevmode% starting standard par
	स्यादेत‚त् (।) ज्ञान‚स्य द्वे श‚क्ती ज्ञान‚ज‚न‚न‚श‚क्तिर्ग्राह्य‚ताश‚क्तिश्च‚{\tiny $_{1}$}‚त‚त्र क्र‚मेण ‚{\tiny $_{lb}$}‚ग्राह्य‚ताश‚क्तिहानौ ज‚न‚न‚श‚क्तिमात्र‚म‚व‚तिष्ठ‚त इत्याह (।)
	\pend% ending standard par
      
	  \bigskip
	  \begingroup
	
	    \large
	  
	    \begin{quote}
	  
	    
	    \stanza[\smallbreak]
	\label{pv.2.529}\flagstanza{\tiny\textenglish{....2.529}}ग्राह्य‚ताश‚क्तिहानिः स्यात् नान्य‚स्य ज‚न‚नात्म‚नः ।&ग्राह्य‚ताया न ख‚ल्व‚न्य‚ज्ज‚न‚नं ग्राह्य‚ल‚क्ष‚णे ॥ ५२९ ॥\&[\smallbreak]


	
	    \end{quote}
	  
	  \endgroup
	

	  \pstart \leavevmode% starting standard par
	\hphantom{.}‚{\color{DodgerBlue3}‚अन्य‚स्य} ज्ञान‚कार्य‚भूत‚{\color{DodgerBlue3}‚ज‚न‚नात्म‚नो} ज‚न‚क‚स्व‚भाव‚स्य पूर्व्व‚ज्ञान‚स्य ‚{\color{DodgerBlue3}‚ग्राह्य‚ता‚{\tiny $_{lb}$}‚श‚क्तिहानिर्न स्यात्} । ज‚न‚न‚स्य ग्राह्य‚ल‚क्ष‚ण‚त्वात्त‚स्य च स‚त्त्वात् । त‚थाहि ‚{\tiny $_{lb}$}‚‚{\color{DodgerBlue3}‚ग्राह्य‚ल‚क्ष‚णे}‚ऽक्ष‚वि\edtext{}{\edlabel{pvv.277-3}\label{pvv.277-3}\lemma{वि}\Bfootnote{ग्राह‚के रूप‚ज्ञानादौ स‚ति ।}}द्युप‚ल‚भ्य‚माने\edtext{}{\edlabel{pvv.277-4}\label{pvv.277-4}\lemma{माने}\Bfootnote{व्य‚व‚धानाभावात् स्व‚भाव‚विशेष‚संमुखीभावात् कार‚णान्त‚र‚साक‚ल्याच्च ।}} ‚{\color{DodgerBlue3}‚न ख‚लु ग्राह्य‚ताया अन्य‚त्} साक्षा‚{\color{DodgerBlue3}‚ज्ज्ज‚न‚नं} । ‚{\tiny $_{lb}$}‚य‚देवोत्त‚रोत्त‚र‚ज्ञान‚स्य साक्षाज्ज‚न‚नं त‚देव त‚द्ग्राह्य‚त्वं (।) त‚तो‚{\tiny $_{2}$}‚ ज‚न‚न‚स‚त्त्वे नास्ति ‚{\tiny $_{lb}$}‚ग्राह्य‚त्व‚हानिः । (५२९)
	\pend% ending standard par
      \label{div_pvv.2.530}
	  
	% new div opening: depth here is 2
	
	  \bigskip
	  \begingroup
	
	    \large
	  
	    \begin{quote}
	  
	    
	    \stanza[\smallbreak]
	\label{pv.2.530}\flagstanza{\tiny\textenglish{....2.530}}साक्षान्न ह्य‚न्य‚था बुद्धे रूपादिरुप‚कार‚कः ।&ग्राह्य‚ताल‚क्ष‚णाद‚न्य‚स्त‚द्भाव‚निय‚मोस्य कः ॥ ५३० ॥\&[\smallbreak]


	
	    \end{quote}
	  
	  \endgroup
	

	  \pstart \leavevmode% starting standard par
	\hphantom{.}न ‚{\color{DodgerBlue3}‚ह्य‚न्य‚था साक्षा}‚ज्ज‚न‚नाद‚न्येन प्र‚कारेण ‚{\color{DodgerBlue3}‚रूपादिर्दृ}‚श्य‚मानो ‚{\color{DodgerBlue3}‚बुद्धे}‚र्ग्राहिकाया ‚{\tiny $_{lb}$}‚‚{\color{DodgerBlue3}‚उप‚कार‚कः} । प‚र‚स्प‚रोप‚कारिणोपि ग्राह्य‚त्वे रूप‚कार‚ण‚ग्र‚ह‚ण‚म‚पि स्यात् । अनु‚{\tiny $_{lb}$}‚प‚कार‚क‚स्य ग्राह्य‚त्वे स‚र्व्व‚ग्र‚ह‚ण‚प्र‚स‚ङ्गः । त‚स्मा‚{\color{DodgerBlue3}‚द‚स्य} रूपादे‚{\color{DodgerBlue3}‚र्ग्राह्य‚ताल‚क्ष‚णाद्दे}‚शाद्य‚{\tiny $_{lb}$}‚वि\edtext{}{\edlabel{pvv.277-5}\label{pvv.277-5}\lemma{वि}\Bfootnote{न बुद्धिमात्रं प‚र‚बुद्धिस‚त्त्वात् । नोपादान‚मात्र‚मिन्द्रिय‚त्वात् प‚र‚म‚ते । ‚{\tiny $_{lb}$}‚त‚स्माद् द्व‚यं ।}}प्र‚क‚र्षिणः साक्षाज्ज‚न‚क‚त्वाद‚न्यः ‚{\color{DodgerBlue3}‚क‚स्त‚द्भाव‚निय‚मो} ग्राह्य‚त्व‚निय‚म‚{\tiny $_{3}$}‚: । (५३०)
	\pend% ending standard par
      \label{div_pvv.2.531}
	  
	% new div opening: depth here is 2
	
	  \bigskip
	  \begingroup
	
	    \large
	  
	    \begin{quote}
	  
	    
	    \stanza[\smallbreak]
	\label{pv.2.531}\flagstanza{\tiny\textenglish{....2.531}}बुद्धेर‚पि त‚द‚स्तीति सापि स‚त्त्वे व्य‚व‚स्थिता ।&ग्राह्युोपादान‚संवित्ती चेत‚सो ग्राह्य‚ल‚क्ष‚ण‚म् ॥ ५३१ ॥\&[\smallbreak]


	
	    \end{quote}
	  
	  \endgroup
	\textsuperscript{\textenglish{278/s}}

	  \pstart \leavevmode% starting standard par
	\hphantom{.}‚{\color{DodgerBlue3}‚त‚त्साक्षा}‚ज्ज‚न‚क‚त्व‚ल‚क्ष‚णं ग्राह्य‚त्वं ‚{\color{DodgerBlue3}‚बुद्धेर‚प्य‚स्तीति सा}‚प्युत्त‚र‚बुद्धिज‚निका ‚{\tiny $_{lb}$}‚‚{\color{DodgerBlue3}‚त‚थैव} त‚द्ग्राह्य‚त्वे ‚{\color{DodgerBlue3}‚व्य‚व‚स्थिता} रूपादिषु कार‚ण‚त्वेपि देशाद्य‚विप्र‚क‚र्ष‚श्च‚क्षुराद्यु‚{\tiny $_{lb}$}‚प‚योग‚ञ्चापेक्ष्य त‚द्ग्राह्य‚तेष्य‚ते । ‚{\color{DodgerBlue3}‚चेत}‚स‚स्तु देशादिविप्र‚क‚र्षाभावाच्च‚श्रुराद्युप‚{\tiny $_{lb}$}‚योगाभावाच्च ग्राहिणो ज्ञान‚स्योपादान‚मुपादान‚कार‚ण‚ता स‚म्वित्तिर‚नुभ‚वात्म‚ता ‚{\tiny $_{lb}$}‚ते ‚{\color{DodgerBlue3}‚ग्राह्य‚ल‚क्ष‚णं} । स्व‚स‚म्वे‚{\tiny $_{4}$}‚द‚न‚स्व‚भाव‚तायां स‚त्यामुपादान‚कार‚ण‚तोत्त‚र‚बुद्धिग्राह्य‚{\tiny $_{lb}$}‚तेत्य‚र्थः । (५३१)
	\pend% ending standard par
      \label{div_pvv.2.532}
	  
	% new div opening: depth here is 2
	

	  \begin{center}%% label @type='head'
	\textbf{(५) क. योगिनां ज्ञान‚म्}
	\end{center}
	‚{\tiny $_{lb}$}‚

	  \pstart \leavevmode% starting standard par
	न‚नु सूक्ष्म‚व्य‚व‚हितास‚न्न‚रूपाद्य‚नुपादान‚ञ्च ज्ञान‚ञ्चेन्न ग्राह्यं त‚दा क‚थं यो\edtext{}{\edlabel{pvv.278-1}\label{pvv.278-1}\lemma{यो}\Bfootnote{य‚दि सौक्ष्म्यादेर‚ग्राह्यं रूपादिर‚चित्त‚ञ्चानुपादान‚भूत‚त्त्वात्त‚दा योगिनां त‚दु‚{\tiny $_{lb}$}‚भ‚यं न ग्राह्यं स्यात् (।) भ‚व‚ति च त‚द् व्य‚भिचारि ल‚क्ष‚णं ।}}‚{\tiny $_{lb}$}‚गिनां तेषां ग्र‚ह‚ण‚मित्याह (।)्
	\pend% ending standard par
      
	  \bigskip
	  \begingroup
	
	    \large
	  
	    \begin{quote}
	  
	    
	    \stanza[\smallbreak]
	\label{pv.2.532}\flagstanza{\tiny\textenglish{....2.532}}रूपादेश्चेत‚स‚श्चैव‚म‚विशुद्ध‚धियं प्र‚ति ।&ग्राह्य‚ल‚क्ष‚ण‚चिन्तेय‚म‚चिन्त्या योगिनां ग‚तिः ॥ ५३२ ॥\&[\smallbreak]


	
	    \end{quote}
	  
	  \endgroup
	

	  \pstart \leavevmode% starting standard par
	\hphantom{.}रूपादेर‚सूक्ष्म‚स्याव्य‚व‚धानादिविशिष्ट‚स्य बुद्धिकार‚ण‚त्वं । ‚{\color{DodgerBlue3}‚चेत‚स‚श्चानुभ}‚वात्म‚त्वे ‚{\tiny $_{lb}$}‚स‚त्युपादान‚ता ग्राह्य‚त्व‚मिति । एव‚म‚नेन प्र‚कारेण ‚{\color{DodgerBlue3}‚ग्राह्य‚ल‚क्ष‚ण‚चिन्तेयं} प्र‚क्रान्ता ‚{\tiny $_{lb}$}‚‚{\color{DodgerBlue3}‚अशुद्ध‚धियं} वास‚नोप‚{\tiny $_{5}$}‚प्लुत‚बुद्धिम‚द‚र्व्वाग्द‚र्शिनं ज‚नं प्र‚ति न तु विशिष्ट‚बुद्धीन् ‚{\tiny $_{lb}$}‚योगिनः प्र‚ति (।) य‚स्मा\edtext{}{\edlabel{pvv.278-2}\label{pvv.278-2}\lemma{स्मा}\Bfootnote{स‚माधिब‚लेनैक‚ज‚न्म‚कुश‚लेनापि देवेष्व‚ष्ट‚गुणैश्व‚र्यादिः किं पुन‚श्चिरेण वीता‚{\tiny $_{lb}$}‚व‚र‚णानां ।}} ‚{\color{DodgerBlue3}‚द्योगिनां} सूक्ष्म‚व्य‚व‚हित‚प‚र‚चित्त‚दिग‚तिर‚चिन्त्या । ‚{\tiny $_{lb}$}‚भाव‚स‚त्तामात्रं योगिज्ञानापेक्ष्यं ग्राह्य‚ल‚क्ष‚ण‚मित्य‚र्थः । (५३२)
	\pend% ending standard par
      \label{div_pvv.2.533}
	  
	% new div opening: depth here is 2
	
	  \bigskip
	  \begingroup
	
	    \large
	  
	    \begin{quote}
	  
	    
	    \stanza[\smallbreak]
	\label{pv.2.533}\flagstanza{\tiny\textenglish{....2.533}}त‚त्र सूक्ष्मादिभावेन ग्राह्य‚म‚ग्राह्य‚तां व्र‚जेत् ।&रूपादि बुद्धेः कि जातं प‚श्चाद् य‚त् प्राङ् न विद्य‚ते ॥ ५३३ ॥\&[\smallbreak]


	
	    \end{quote}
	  
	  \endgroup
	

	  \pstart \leavevmode% starting standard par
	\hphantom{.}‚{\color{DodgerBlue3}‚त‚त्रै}‚तादृशे ग्राह्य‚ल‚क्ष‚ण‚द्व‚ये स्थिते स‚ति रूपं स्थूल‚म‚व्य‚व‚धानादिभावे स‚ति ‚{\tiny $_{lb}$}‚‚{\color{DodgerBlue3}‚ग्राह्यं} स‚त्प‚श्चात् ‚{\color{DodgerBlue3}‚सूक्ष्मादिभावेनाग्राह्य‚तां व्र}‚जेदिति युज्य‚त एवैत‚त् ।\edtext{\textsuperscript{*}}{\edlabel{pvv.278-3}\label{pvv.278-3}\lemma{*}\Bfootnote{रूप‚म‚ग्राह्यं स्याद‚पि बुद्धेस्त्वेत‚त् स‚र्व्वं नास्तीति क‚थ‚म‚ग्राह्य‚ता ।}} बुद्धेरेक‚दा ‚{\tiny $_{lb}$}‚ग्राह्यायाः ‚{\color{DodgerBlue3}‚प‚श्चाद}‚ग्र‚ह‚ण‚काले ‚{\color{DodgerBlue3}‚किं} सूक्ष्म‚त्वाद्य‚{\tiny $_{6}$}‚ग्राह्य‚ताकार‚णं ‚{\color{DodgerBlue3}‚जातं य‚त् प्रागु}‚प‚ल‚म्भ‚{\tiny $_{lb}$}‚काले ‚{\color{DodgerBlue3}‚न विद्य‚ते} । न हि स्व‚स‚न्तान‚व‚र्तित‚न‚श्चेसो निर‚व‚य‚व‚स्य स‚र्व्व‚दा किञ्चि‚{\tiny $_{lb}$}‚\leavevmode\ledsidenote{\textenglish{279/s}} दुप‚ल‚म्भ‚कार‚ण‚म‚स्ति । तादुश‚स्य य‚द्य‚नुप‚ल‚म्भः क‚दाचिन्नोप‚ल‚भ्येत । उप‚ल‚भ्य‚ते ‚{\tiny $_{lb}$}‚चेह क‚दाचित्स‚र्व्व‚दोप‚ल‚म्भ‚प्र‚स‚ङ्गः ॥ (५३३)
	\pend% ending standard par
      \label{div_pvv.2.534}
	  
	% new div opening: depth here is 2
	
	  \bigskip
	  \begingroup
	
	    \large
	  
	    \begin{quote}
	  
	    
	    \stanza[\smallbreak]
	\label{pv.2.534}\flagstanza{\tiny\textenglish{....2.534}}स‚ति स्व‚धीग्र‚हे त‚स्मात्सैवान‚न्त‚र‚हेतुता ।&चेत‚सो ग्राह्य‚ता सैव त‚तो नार्थान्त‚रे ग‚तिः ॥ ५३४ ॥\&[\smallbreak]


	
	    \end{quote}
	  
	  \endgroup
	

	  \pstart \leavevmode% starting standard par
	\hphantom{.}‚{\color{DodgerBlue3}‚त‚स्मा}‚च्चित्तान्त‚रेण ‚{\color{DodgerBlue3}‚स्व‚धीग्र‚हे स‚ति चेत‚सो} यैवान‚{\color{DodgerBlue3}‚न्त‚र‚हेतुता} बुद्ध्य‚न्त‚रं प्र‚ति ‚{\tiny $_{lb}$}‚‚{\color{DodgerBlue3}‚सैव ग्राह्य‚ता} सा च नापैति ‚{\color{DodgerBlue3}‚त‚त} उत्त‚रोत्त‚र‚बुद्धेः पूर्व्व‚पू‚{\tiny $_{7}$}‚र्व्व‚बुद्धिग्र‚ह‚ण‚मेव व्यापार\leavevmode\ledsidenote{\textenglish{55a/MA}} ‚{\tiny $_{lb}$}‚इत्य‚र्था‚{\color{DodgerBlue3}‚न्त‚रे} न स्या‚{\color{DodgerBlue3}‚द् ग\edtext{}{\edlabel{pvv.279-1}\label{pvv.279-1}\lemma{ग}\Bfootnote{न विष‚यान्त‚स‚ञ्चार इत्युप‚संहारः ।}}तिः} ॥ (५३४)
	\pend% ending standard par
      \label{div_pvv.2.535}
	  
	% new div opening: depth here is 2
	

	  \begin{center}%% label @type='head'
	\textbf{ख. ग्राह्य‚ताश‚क्तिहानाद‚पि न ज‚न‚न‚श‚क्तिः}
	\end{center}
	

	  \pstart \leavevmode% starting standard par
	न‚नु भावा अनेक‚श‚क्तियो\edtext{}{\edlabel{pvv.279-2}\label{pvv.279-2}\lemma{क्तियो}\Bfootnote{द‚र्श‚क‚वादी ।}}गाद‚नेक‚कार्य‚कारिण एक‚श‚क्तियोगादेक‚कार्य‚{\tiny $_{lb}$}‚कारिणः त‚त् ज्ञान‚स्य ग्राह्य‚ताश\edtext{}{\edlabel{pvv.279-3}\label{pvv.279-3}\lemma{ताश}\Bfootnote{क‚र्म‚रूपा ।}}क्तिर‚प्य‚न्या । अन्या च ज‚न‚न\edtext{}{\edlabel{pvv.279-4}\label{pvv.279-4}\lemma{न}\Bfootnote{क‚र्तृरूपा ।}}श‚क्तिः । त‚तो ‚{\tiny $_{lb}$}‚ग्राह्य‚ताश‚क्तिहानाद‚पि ज‚न‚न‚श‚क्तिर्भ‚विष्य‚तीत्याह (।)
	\pend% ending standard par
      
	  \bigskip
	  \begingroup
	
	    \large
	  
	    \begin{quote}
	  
	    
	    \stanza[\smallbreak]
	\label{pv.2.535}\flagstanza{\tiny\textenglish{....2.535}}नानैक‚श‚क्त्य‚भावेपि भावो नानैक‚कार्य‚कृत् ।&प्र‚कृत्यैवेति ग‚दितं; नानैक‚स्मान्न चेद्भ‚वेत् ॥ ५३५ ॥\&[\smallbreak]


	
	    \end{quote}
	  
	  \endgroup
	

	  \pstart \leavevmode% starting standard par
	\hphantom{.}‚{\color{DodgerBlue3}‚नानैक‚श‚क्त्य‚भा\edtext{}{\edlabel{pvv.279-5}\label{pvv.279-5}\lemma{भा}\Bfootnote{भिन्ना श‚क्तिर‚युक्ता ।}}वेपि} भाव‚व्य‚तिरिक्ताऽनेकासां श‚क्तीनामेक‚स्याश्च श‚क्तेर‚{\tiny $_{lb}$}‚भावेपि भावः स्व‚हेतो‚{\tiny $_{1}$}‚र्नानैक‚कार्य‚कार‚ण‚स्व‚भाव‚त‚योत्प‚न्नो ‚{\color{DodgerBlue3}‚नानैक‚कार्य‚कृत् प्र‚कृत्यैव} भ‚व‚ती‚{\color{DodgerBlue3}‚ति ग‚दि\edtext{}{\edlabel{pvv.279-6}\label{pvv.279-6}\lemma{दि}\Bfootnote{प्राक् ।}} तं} ।
	\pend% ending standard par
      

	  \pstart \leavevmode% starting standard par
	न‚नु नानैकं कार्य‚मेक‚स्मात्कार‚णान्न भ‚वेत्सिद्धा\edtext{}{\edlabel{pvv.279-7}\label{pvv.279-7}\lemma{वेत्सिद्धा}\Bfootnote{स्व‚यूथ्याः ।}}न्त‚विरोधादिति चेत् । (५३५)
	\pend% ending standard par
      \label{div_pvv.2.536}
	  
	% new div opening: depth here is 2
	

	  \begin{center}%% label @type='head'
	\textbf{(६) हेतुसाम‚ग्र्‏याः स‚र्व‚स‚म्भ‚वः}
	\end{center}
	
	  \bigskip
	  \begingroup
	
	    \large
	  
	    \begin{quote}
	  
	    
	    \stanza[\smallbreak]
	\label{pv.2.536a}\flagstanza{\tiny\textenglish{...2.536a}}न किञ्चिदेक‚मेक‚स्मात्साम‚ग्र‚याः स‚र्व्व‚स‚म्भ‚वः ।\&[\smallbreak]


	
	    \end{quote}
	  
	  \endgroup
	

	  \pstart \leavevmode% starting standard par
	\hphantom{.}स‚त्त्य‚मेत‚न्न किञ्चि‚{\color{DodgerBlue3}‚त्कार्य}‚मेक‚स्मात्कार‚णाज्जाय‚ते किन्तु साम‚ग्र्‏या ज‚न्मानेको‚{\tiny $_{lb}$}‚पादान‚स‚ह‚कारिभाव‚स‚ञ्च‚यात्मिकायाः स‚र्व्व‚स्यैक‚स्यानेक‚स्य च कार्य‚स्य स‚म्भ‚वः ।
	\pend% ending standard par
      

	  \pstart \leavevmode% starting standard par
	क‚थ‚न्त‚र्ह्येक‚म‚नेक‚स्यं\edtext{}{\edlabel{pvv.279-8}\label{pvv.279-8}\lemma{स्यं}\Bfootnote{त‚थानेक‚कृदेकोपि त‚द्भाव‚प‚रिदीप‚न इत्य‚त्र ।}} हेतुरुच्य‚त इत्याह (।)
	\pend% ending standard par
      
	  \bigskip
	  \begingroup
	
	    \large
	  
	    \begin{quote}
	  
	    
	    \stanza[\smallbreak]
	\label{pv.2.536b}\flagstanza{\tiny\textenglish{...2.536b}}एकं स्याद‚पि साम‚ग्र्योरित्युक्तं त‚द‚नेक‚कृत् ॥ ५३६ ॥\&[\smallbreak]


	
	    \end{quote}
	  
	  \endgroup
	\textsuperscript{\textenglish{280/s}}

	  \pstart \leavevmode% starting standard par
	च‚क्षुरूपालोक‚म‚न‚स्कारादिषु जाय‚मान‚रूप‚क्ष‚णेन्द्रिय‚ज्ञान‚कार्य‚द्व‚यापेक्ष‚याऽवान्त‚र‚{\tiny $_{lb}$}‚भिन्त‚साम‚ग्र्‏योरेक‚स्य रूप‚स्योपादान‚स‚ह‚कारिभावेनोप‚योगात् एक‚स्माद‚प्य‚नेकं कार्यं ‚{\tiny $_{lb}$}‚जाय‚त इति त‚त एक‚म‚नेक‚कार्य‚कृदित्युच्य‚ते (।) न त्वेक‚स्माद‚स‚हायाद‚नेकं कार्यं ‚{\tiny $_{lb}$}‚जाय‚ते इत्य‚भिप्रायात् (५३६)
	\pend% ending standard par
      \label{div_pvv.2.537_2.538}
	  
	% new div opening: depth here is 2
	

	  \pstart \leavevmode% starting standard par
	स्यादेत‚त् (।)
	\pend% ending standard par
      
	  \bigskip
	  \begingroup
	
	    \large
	  
	    \begin{quote}
	  
	    
	    \stanza[\smallbreak]
	\label{pv.2.537a}\flagstanza{\tiny\textenglish{...2.537a}}अर्थं पूर्व्व‚ञ्च विज्ञानं गृह्णीयाद् य‚दि धीः प‚रा ।\&[\smallbreak]


	
	    \end{quote}
	  
	  \endgroup
	

	  \pstart \leavevmode% starting standard par
	\hphantom{.}‚{\color{DodgerBlue3}‚धी}‚र्जाय‚माना ‚{\color{DodgerBlue3}‚पूर्व्व‚ञ्च‚{\tiny $_{3}$}‚ विज्ञान‚म‚र्थ‚ञ्च} त‚त्काल‚स‚न्निप‚तितं य‚दि ‚{\color{DodgerBlue3}‚गृह्णीयात्} त‚दा विष‚यान्त‚र‚स‚ञ्चारः स‚म्भ‚वेत् । अत्राह (।)
	\pend% ending standard par
      
	  \bigskip
	  \begingroup
	
	    \large
	  
	    \begin{quote}
	  
	    
	    \stanza[\smallbreak]
	\label{pv.2.537b}\flagstanza{\tiny\textenglish{...2.537b}}अभिलाप‚द्व‚यं नित्यं स्याद् दृष्ट‚क्र‚म‚म‚क्र‚म‚म् ॥ ५३७ ॥\&[\smallbreak]


	
	    \end{quote}
	  
	  \endgroup
	

	  \pstart \leavevmode% starting standard par
	\hphantom{.}पूर्व्व‚गृहीत‚स्य व‚र्ण्ण‚स्य चिन्तादावादिश‚ब्दात् प्र‚त्य‚क्षेप्येक‚स्मिन् ‚{\color{DodgerBlue3}‚चेत‚सि} दृष्टः ‚{\tiny $_{lb}$}‚क्र‚मो य‚स्य त‚त् दृष्ट‚क्र‚म‚{\color{DodgerBlue3}‚म‚भिलाप‚द्व‚यं} व‚र्ण्ण‚द्व‚यं ‚{\color{DodgerBlue3}‚नित्य‚म}‚क्र‚मं ‚{\color{DodgerBlue3}‚स्यात्} युग‚प‚त्प्र‚तीयेतेत्य‚र्थः । ‚{\tiny $_{lb}$}‚(५३७) किं कार‚ण‚मित्याह (।) ‚{\color{DodgerBlue3}‚पूर्व्वाप‚रार्थ‚भासित्वा\edtext{}{\edlabel{pvv.280-1}\label{pvv.280-1}\lemma{भासित्वा}\Bfootnote{प्र‚स‚ङ्ग‚साध‚न‚मिदं य‚दि पूर्व्वाप‚राव‚र्थावेक‚त्र भासेते इत्य‚र्थः ।}}त् चिन्तादि}‚ज्ञानेन हि ‚{\tiny $_{lb}$}‚पूर्व्व‚व‚र्ण्ण‚ग्राह‚क‚ज्ञानं गृह्येत इति‚{\tiny $_{5}$}‚ त‚त्प्र‚तिभासी व‚र्ण्णो गृह्येत । त‚त्का\edtext{}{\edlabel{pvv.280-2}\label{pvv.280-2}\lemma{त्का}\Bfootnote{अव‚र्ण्ण‚चिन्त‚ने अग्राहिज्ञान‚ग्राह‚कं य‚दुत्त‚र‚मिव‚र्ण्ण‚ग्राह‚कं त‚त्रेकार‚श्च पूर्व्व‚ज्ञाना‚{\tiny $_{lb}$}‚रूढाकार‚श्च भास‚त इत्य‚भिलाप‚द्व‚य‚मेक‚चित्तं स्यात् । न चास्ति ।}}लोप‚निप‚ति‚{\tiny $_{lb}$}‚त‚श्चान्यो व‚र्ण्ण इति क्र‚मोप‚ल‚भ्य‚योर्युग‚प‚द् ग्र‚ह‚ण‚प्र‚स‚ङ्गः ।
	\pend% ending standard par
      
	  \bigskip
	  \begingroup
	
	    \large
	  
	    \begin{quote}
	  
	    
	    \stanza[\smallbreak]
	\label{pv.2.538}\flagstanza{\tiny\textenglish{....2.538}}द्विर्द्विरेकं च भासेत भास‚नादात्म‚त‚द्धियोः ॥ ५३८ ॥\&[\smallbreak]


	
	    \end{quote}
	  
	  \endgroup
	

	  \pstart \leavevmode% starting standard par
	\hphantom{.}एक‚ञ्च व‚र्ण्णादि‚{\color{DodgerBlue3}‚र्द्विर्द्विर्भासेत} । आ\edtext{}{\edlabel{pvv.280-3}\label{pvv.280-3}\lemma{आ}\Bfootnote{विष‚य‚रूप‚मात्मा, त‚दिति त‚दाल‚म्व‚नं ज्ञानं त‚योः । एक‚दा ह्य‚कारः ‚{\tiny $_{lb}$}‚स्व‚ग्राहिज्ञाने भाति पुन‚रिकार‚ज्ञाने स्व‚ज्ञानारूढः इत्येक‚व‚र्ण्ण‚चिन्तैव न स्यात् ।}}त्म‚त‚द्धियोर्भास‚नात् । त‚द्द‚र्श‚ने त‚त् ‚{\tiny $_{lb}$}‚ज्ञान‚द‚र्श‚ने तु त‚दृष्ट‚मिति द्विधा द‚र्श‚न‚प्र‚स‚ङ्गः\edtext{}{\edlabel{pvv.280-4}\label{pvv.280-4}\lemma{ङ्गः}\Bfootnote{स्व‚यं सिद्ध‚साध‚नं प‚रिहृत्याचार्यीयं स‚म‚न्व‚य‚ते प‚रिहारं ।}}। (५३८)
	\pend% ending standard par
      \label{div_pvv.2.539}
	  
	% new div opening: depth here is 2
	

	  \begin{center}%% label @type='head'
	\textbf{(७) आत्मानुभूतं प्र‚त्य‚क्ष‚म्}
	\end{center}
	
	  \bigskip
	  \begingroup
	
	    \large
	  
	    \begin{quote}
	  
	    
	    \stanza[\smallbreak]
	\label{pv.2.539}\flagstanza{\tiny\textenglish{....2.539}}विष‚यान्त‚र‚स‚ञ्चारे य‚द्य‚न्त्य‚न्नानुभूय‚ते ।&प‚रानुभूत‚व‚त्स‚र्व्वान‚नुभूतिः प्र‚स‚ज्य‚ते ॥ ५३९ ॥\&[\smallbreak]


	
	    \end{quote}
	  
	  \endgroup
	

	  \pstart \leavevmode% starting standard par
	अथ विष‚य‚ज्ञाने वृत्ते त‚ज्ज्ञाने चोत्प‚न्ने विष‚य‚द‚र्श‚न‚स्य निष्प‚न्न‚त्वात् ‚{\tiny $_{lb}$}‚\leavevmode\ledsidenote{\textenglish{281/s}} ज्ञानान्त‚रं विष‚यान्त‚र‚ग्राह‚क‚मिति स्या‚{\tiny $_{5}$}‚‚{\color{DodgerBlue3}‚द्विष‚या\edtext{}{\edlabel{pvv.281-1}\label{pvv.281-1}\lemma{या}\Bfootnote{ज्ञान‚ज्ञानाद‚न्त्य‚व‚र्तिनो ज्ञानान्त‚रेणान‚नुभूताद्विष‚यान्त‚र‚स‚ञ्चार‚स्त‚स्य स्व‚वेद‚न‚म‚व‚श्याभ्युपेयं । त‚स्यान‚नुभ‚वे प्राक्त‚न‚म‚गृहीत‚मिति स‚र्व्व‚लोपः ।}}न्त‚र‚स‚ञ्चार} इत्य‚भिम‚ते ‚{\color{DodgerBlue3}‚य‚द्य‚न्त्यं} ज्ञानं ज्ञानानुभ‚वितृ ‚{\color{DodgerBlue3}‚नानुभूय‚ते} त‚दा ‚{\color{DodgerBlue3}‚स‚र्व्व}‚स्यार्थ‚स्य त‚त् ज्ञान‚स्य ‚{\color{DodgerBlue3}‚चान‚नुभूतिः प्र‚स‚ज्य‚ते} प‚रानुभूत‚व‚त् । पुरुषान्त‚रानुभूतार्थ‚त‚ज्ज्ञान‚योरिव (५३९) ।
	\pend% ending standard par
      \label{div_pvv.2.540}
	  
	% new div opening: depth here is 2
	

	  \pstart \leavevmode% starting standard par
	न‚नु आत्म‚नाऽनुभूत‚म‚र्थं-त‚त्-ज्ञानं-प्र‚त्य‚क्षं न प‚रैर‚नुभूत‚मिति य‚द्युच्य‚ते ।
	\pend% ending standard par
      
	  \bigskip
	  \begingroup
	
	    \large
	  
	    \begin{quote}
	  
	    
	    \stanza[\smallbreak]
	\label{pv.2.540}\flagstanza{\tiny\textenglish{....2.540}}आत्मानुभूतं प्र‚त्य‚क्षं नानुभूतं प‚रैः य‚दि ।&आत्मानुभूतिः सा सिद्धा कुतो येनैव‚मुच्य‚ते ॥ ५४० ॥\&[\smallbreak]


	
	    \end{quote}
	  
	  \endgroup
	

	  \pstart \leavevmode% starting standard par
	\hphantom{.}न‚न्व‚न्त्य‚ज्ञानान‚नुभ‚वेऽर्थ‚ज्ञान‚स्यान‚नुभ‚वात् । त‚द‚नुभ‚वाभावे ‚{\color{DodgerBlue3}‚चार्थानुभ‚वासिद्धे‚{\tiny $_{lb}$}‚रात्मानुभूतिः सा कुतः सिद्धा । येनैव‚{\tiny $_{7}$}‚मुच्य‚ते । आत्मानुभूतं प्र‚त्य\edtext{}{\edlabel{pvv.281-2}\label{pvv.281-2}\lemma{त्य}\Bfootnote{अनुभ‚व‚सिद्धौ ह्यात्म‚प‚र‚विभागः ।}}क्षं} न प‚रानु-\leavevmode\ledsidenote{\textenglish{55b/MA}} ‚{\tiny $_{lb}$}‚भूत‚मिति । (५४०)
	\pend% ending standard par
      \label{div_pvv.2.541}
	  
	% new div opening: depth here is 2
	

	  \pstart \leavevmode% starting standard par
	न‚नु च‚क्षुरादाव‚न‚र्थ‚भूते च‚क्षुरादिना रूपाद्य‚नुभूत‚मिति य‚था त‚था ज्ञाना‚{\tiny $_{lb}$}‚न‚नुभ‚वेप्य‚र्थो ज्ञात इति भ‚विष्य‚तीत्याह (।)
	\pend% ending standard par
      
	  \bigskip
	  \begingroup
	
	    \large
	  
	    \begin{quote}
	  
	    
	    \stanza[\smallbreak]
	\label{pv.2.541}\flagstanza{\tiny\textenglish{....2.541}}व्य‚क्तिहेत्व‚प्र‚सिद्धिः स्यात् न व्य‚क्तेर्व्य‚क्त‚मिच्छ‚तः ।&व्य‚क्त्य‚सिद्धाव‚पि व्य‚क्तं य‚दि व्य‚क्त‚मिदं ज‚ग‚त् ॥ ५४१ ॥\&[\smallbreak]


	
	    \end{quote}
	  
	  \endgroup
	

	  \pstart \leavevmode% starting standard par
	\hphantom{.}अर्थ‚व्य‚{\color{DodgerBlue3}‚क्तिहेतो}‚श्च‚क्षुरादेर‚र्थ‚द‚र्श‚नेप्य‚{\color{DodgerBlue3}‚प्र‚सिद्धिर}‚व्य‚क्तिः स्यात्\edtext{}{\edlabel{pvv.281-3}\label{pvv.281-3}\lemma{स्यात्}\Bfootnote{अज्ञात‚स्यापि बीज‚स्याङ‚कुर‚ज‚न‚न‚दृष्टेः ।}} य‚तो न कार‚ण‚{\tiny $_{lb}$}‚द‚र्श‚न‚पूर्व्व‚कं कार्य‚द‚र्श‚नं । ‚{\color{DodgerBlue3}‚न तु व्य‚क्ते}‚रूप‚ल‚ब्धेर्व्य‚{\color{DodgerBlue3}‚क्त‚म‚र्थ‚मिच्छ‚तो} व्य‚क्त्य‚सिद्धि‚{\tiny $_{lb}$}‚र्युक्ता । ‚{\color{DodgerBlue3}‚य‚दि} पुन‚{\color{DodgerBlue3}‚र्व्य‚क्तेर‚सिद्धाव‚पि व्य‚क्तं} व‚स्तूच्य‚ते त‚दा स‚र्व्व‚{\tiny $_{1}$}‚‚{\color{DodgerBlue3}‚मिदं ज‚ग‚द् ‚{\tiny $_{lb}$}‚व्य‚क्तं्}\edtext{}{\edlabel{pvv.281-4}\label{pvv.281-4}\lemma{र्व्व}\Bfootnote{स‚र्व्वे स‚र्व्व‚ज्ञाः स्युः ।}} स्यात् । अव्य‚क्त‚व्य‚क्तिक‚त्वेन विशेषाभावात्\edtext{}{\edlabel{pvv.281-5}\label{pvv.281-5}\lemma{विशेषाभावात्}\Bfootnote{न च भ‚व‚ति त‚स्मात् स्व‚वेद‚न‚मेष्ट‚व्य‚म् ।}}त‚था ह्य‚र्थो न स‚त्तामात्रेण ‚{\tiny $_{lb}$}‚प्र‚तीत उच्य‚ते । प्र‚तिप‚त्तिसंयोगात् (।) नापि ज्ञान‚स‚त्तामात्रेण प‚र‚चित्त‚ज्ञानेनापि ‚{\tiny $_{lb}$}‚संवेद‚न‚प्र‚स‚ङ्गात् किन्तु ज्ञानानुभ‚वेनार्थ‚प्र‚तीतिर्व्वाच्या । न चान्येन स‚म्वेद‚नं ज्ञान‚{\tiny $_{lb}$}‚स्योप‚प‚प‚द्य‚त इति स्व‚प्र‚काश‚मेव स्व‚भाव‚त‚स्त‚द्व‚क्त‚व्य‚मिति स्व‚स‚म्वेद‚न‚सिद्धिरिति ।‚{\tiny $_{1}$}‚ ‚{\tiny $_{lb}$}‚(५४१)
	\pend% ending standard par
      
	    
	    \pstart
	    \begin{center}
	  आचार्य‚श्री म नो र थ न न्दि कृतायां वार्त्तिक‚{\tiny $_{lb}$}‚वृत्तौ प्र‚त्य‚क्ष‚प‚रिच्छेदो द्वितीयः ॥
	    \end{center}
	    \pend
	  
	  \textsuperscript{\textenglish{282/s}}
	    
	    \endnumbering% ending numbering from div
	    \endgroup
	    
	  
	  
	% new div opening: depth here is 0
	
	    
	    \begingroup
	    \beginnumbering% beginning numbering from div depth=0
	    
	  
\chapter*[{तृतीयः प‚रिच्छेदः: स्वार्थानुमानं}]{तृतीयः प‚रिच्छेदः: स्वार्थानुमानं}
	  
	% new div opening: depth here is 1
	

	  \pstart \leavevmode% starting standard par
	प्र‚त्य‚क्ष‚माख्यायाव‚स‚र‚प्राप्त‚म‚नु\edtext{}{\edlabel{pvv.282-1}\label{pvv.282-1}\lemma{नु}\Bfootnote{निरोध‚मार्ग‚प्राप्तिदुःख‚स‚मुद‚य‚प‚रिहारार्थ‚त्वाद‚स्यैव ।}}मान‚मिदानीम्व‚क्त‚व्यं (।) त‚च्च द्विविधं (।) ‚{\tiny $_{lb}$}‚स्वार्थं प‚रार्थ‚ञ्च । त‚त्र स्वार्थ‚मिदानीं व‚क्त‚व्यं एत‚त्पूर्व‚क‚त्वात् प‚रार्थ‚स्य ।
	\pend% ending standard par
      
	  
	% new div opening: depth here is 1
	
\chapter*[{१---हेतुचिन्ता}]{१---हेतुचिन्ता}

	  \begin{center}%% label @type='head'
	\textbf{(१) हेतुल‚क्ष‚ण‚म्}
	\end{center}
	\label{div_pvv.3.1}
	  
	% new div opening: depth here is 2
	

	  \pstart \leavevmode% starting standard par
	त‚ल्लिङ्गे विप्र‚तिप‚त्त‚यः स‚न्तीति तासां निराक‚र‚णेन त‚द्व्य‚व‚स्थाप‚नार्थ‚माह ।
	\pend% ending standard par
      
	  \bigskip
	  \begingroup
	
	    \large
	  
	    \begin{quote}
	  
	    
	    \stanza[\smallbreak]
	\label{pv.3.1a}\flagstanza{\tiny\textenglish{...v.3.1a}}प‚क्ष\edtext{}{\edlabel{pvv.282-2}\label{pvv.282-2}\lemma{क्ष}\Bfootnote{श्लोकेत्र लिङ्ग‚ल‚क्ष‚णं संख्या(त्रिधैव)निय‚मः (अविनाभाव‚निय‚मात्) संख्यानिय‚म‚कार‚णं विप‚क्ष‚निवृत्ति(हेत्वाभास)श्चोक्तो}}ध‚र्म‚स्त‚दंशेन व्याप्तो हेतुः ;\&[\smallbreak]


	
	    \end{quote}
	  
	  \endgroup
	

	  \pstart \leavevmode% starting standard par
	प‚क्षो ध‚र्मिध‚र्म‚स‚मुदायोनुमेयः । त‚देक‚देश‚त्वा‚{\tiny $_{3}$}‚दुप‚चारेण\edtext{}{\edlabel{pvv.282-3}\label{pvv.282-3}\lemma{चारेण}\Bfootnote{त‚दंश‚त्वेन स‚म्ब‚न्ध उक्तः ।}} ध‚र्मी प‚क्ष उक्तः । ‚{\tiny $_{lb}$}‚त‚स्य ध‚र्मः (।) अनेन प‚क्ष‚ध‚र्म‚त्व‚मुक्तं । ध‚र्मिध‚र्मो हेतुरित्य‚नुमाने स‚र्व्व‚स्य ध‚र्म्मि‚{\tiny $_{lb}$}‚ध‚र्म्मो हेतुः स्यादित्युप‚चाराश्र‚य‚णं । त‚था चाक्षु\edtext{}{\edlabel{pvv.282-4}\label{pvv.282-4}\lemma{चाक्षु}\Bfootnote{य‚था रूप‚मिति दृष्टान्त‚ध‚र्मिध‚र्म एव हेतुः स्याद‚न्य‚था ।}}ष‚त्वादिः श‚ब्दे ध‚र्मिणि न हेतुः । ‚{\tiny $_{lb}$}‚त‚दंशेन प‚क्ष‚स्य ध‚र्मेण\edtext{}{\edlabel{pvv.282-5}\label{pvv.282-5}\lemma{र्मेण}\Bfootnote{ध‚र्मिमात्र‚स्य प‚क्ष‚त्वाद् ध‚र्मी च व‚न्ह्यादिः साध्य एव त‚दंशो विव‚क्षाव‚शाद् य‚दा तु स‚मुदायः प‚क्ष‚स्त‚दा त‚दंश एक‚देश एव ।}} सिसाध‚यिषितेन व्याप्तो हेतुर्बोद्ध‚व्यः । द्विविधा चेयं ‚{\tiny $_{lb}$}‚व्याप्तिर्व्याप‚क‚व्याप्य‚ध‚र्म‚त‚या । त‚त्र व्याप्ये स‚ति व्याप‚क‚स्याव‚श्य‚म्भाव\edtext{}{\edlabel{pvv.282-6}\label{pvv.282-6}\lemma{म्भाव}\Bfootnote{इति व्याप्य‚ध‚र्म उक्तो व्याप्ये ध‚र्मिणि व्याप्तिर्द्ध‚र्मः ।}}स्त‚स्य ‚{\tiny $_{lb}$}‚व्याप्तिः । व्याप्य‚स्य च व्याप‚{\tiny $_{4}$}‚क एव\edtext{}{\edlabel{pvv.282-7}\label{pvv.282-7}\lemma{एव}\Bfootnote{व्याप‚क‚ध‚र्मोऽत्र व्याप्तिम‚त्वात् ध‚र्म‚त्वं व्याप्तेः ।}} स‚ति भावो नाम त‚स्य व्याप्तिः । आभ्यां ‚{\tiny $_{lb}$}‚\leavevmode\ledsidenote{\textenglish{283/s}} य‚थाक्र‚म‚म‚न्व‚य‚व्य‚तिरेकावुक्तौ । व्याप्य‚स‚द्‏भावे व्याप‚क‚स्य स‚त्व‚निय‚म‚स्यान्व‚य‚रूप‚{\tiny $_{lb}$}‚त्वात् । व्याप‚काभावे व्याप्याभाव‚स्य च व्य‚तिरेक‚रूप‚त्वात् (।)
	\pend% ending standard par
      

	  \begin{center}%% label @type='head'
	\textbf{(२) हेतुस्त्रिधा ‚{\tiny $_{lb}$}‚एतेन त्रिरूप‚त्वं हेतोर्ल‚क्ष‚ण‚मुक्तं ॥}
	\end{center}
	
	  \bigskip
	  \begingroup
	
	    \large
	  
	    \begin{quote}
	  
	    
	    \stanza[\smallbreak]
	\label{pv.3.1b}\flagstanza{\tiny\textenglish{...v.3.1b}}त्रिधैव सः ।\&[\smallbreak]


	
	    \end{quote}
	  
	  \endgroup
	

	  \pstart \leavevmode% starting standard par
	\hphantom{.}‚{\color{DodgerBlue3}‚त्रिधैव} त्रिप्र‚कार एव कार्य‚स्व‚भावानुप‚ल‚म्भ‚भेदेन ‚{\color{DodgerBlue3}‚स\edtext{}{\edlabel{pvv.283-1}\label{pvv.283-1}\lemma{स}\Bfootnote{प‚क्ष‚ध‚र्म‚त्त्वेन य‚थोक्त‚व्याप्त्या च युक्तः ।}} हेतुः । य‚थाग्निर‚त्र} ध‚मात् । वृक्ष‚व्य‚व‚हार‚योग्योयं शिंश‚पात्वात् ।‚{\tiny $_{5}$}‚ नेह प्र‚देशे घ‚ट उप‚ल‚ब्धिल‚क्ष‚ण‚{\tiny $_{lb}$}‚प्राप्त‚स्यानुप‚ल‚ब्धेरिति संख्यानिय‚म उक्तः ।
	\pend% ending standard par
      

	  \pstart \leavevmode% starting standard par
	क‚स्मात् पुन‚स्त्रिविध एव हेतुरित्याह ।
	\pend% ending standard par
      
	  \bigskip
	  \begingroup
	
	    \large
	  
	    \begin{quote}
	  
	    
	    \stanza[\smallbreak]
	\label{pv.3.1c}\flagstanza{\tiny\textenglish{...v.3.1c}}अविनाभाव‚निय‚मात्;\&[\smallbreak]


	
	    \end{quote}
	  
	  \endgroup
	

	  \pstart \leavevmode% starting standard par
	\hphantom{.}‚{\color{DodgerBlue3}‚अविनाभाव‚स्य} साध्याव्य‚भिचारित्व‚स्य त्रिविध एव हेतौ ‚{\color{DodgerBlue3}‚निय‚मात्} निय‚{\tiny $_{lb}$}‚त‚त्वात् । संयोग्यादिषु चाभावात् स‚त्येवाविनाभावे हेतुत्वं स च व्याप्त्या क‚थितः । ‚{\tiny $_{lb}$}‚अनेन संख्यानिय‚म‚कार‚ण‚मुक्तं ।
	\pend% ending standard par
      

	  \begin{center}%% label @type='head'
	\textbf{(३) हेत्वाभासाः}
	\end{center}
	

	  \pstart \leavevmode% starting standard par
	हेत‚व उक्ताः । के पुन‚र्हेत्वाभासा इत्याह (।)
	\pend% ending standard par
      
	  \bigskip
	  \begingroup
	
	    \large
	  
	    \begin{quote}
	  
	    
	    \stanza[\smallbreak]
	\label{pv.3.1d}\flagstanza{\tiny\textenglish{...v.3.1d}}हेत्वाभासास्त‚तोऽप‚रे ॥ १ ॥ \leavevmode\ledsidenote{\textenglish{56a/MA}}\&[\smallbreak]


	
	    \end{quote}
	  
	  \endgroup
	

	  \pstart \leavevmode% starting standard par
	हेतुव‚दा‚{\tiny $_{6}$}‚भास‚न्त इति ‚{\color{DodgerBlue3}‚हेत्वाभासा} हेतुप्र‚तिरूप‚काः (।) ‚{\color{DodgerBlue3}‚त‚त}‚स्त्रिविधाद्धेतो‚{\tiny $_{lb}$}‚र‚प‚रेऽन्ये संयोग्याद‚यः । अविनाभावे स‚ति हेतुत्वं स च तादात्म्यात् त‚दुत्प‚त्तेश्च । ‚{\tiny $_{lb}$}‚ये तु त‚द्विक‚लास्तेऽविनाभाव‚विर‚हात् हेत्वाभासा इत्य‚र्थः (। १)
	\pend% ending standard par
      \label{div_pvv.3.2ab}
	  
	% new div opening: depth here is 2
	

	  \pstart \leavevmode% starting standard par
	य‚दि\edtext{}{\edlabel{pvv.283-2}\label{pvv.283-2}\lemma{दि}\Bfootnote{दिङ्नागोक्तं प‚रं प्र‚ति स‚र्व्व‚था ग‚म्य‚ग‚म‚क‚त्वं कार्य‚हेतौ श‚ङ्क‚ते ।}} त‚दुत्प‚त्त्या ग‚म्य‚ग‚म‚क‚भाव‚स्त‚दा धूम‚त्व‚विशेष‚व‚त् सामान्य‚ध‚र्माः पार्थि‚{\tiny $_{lb}$}‚व‚त्वाद‚योपि ग‚म‚काः स्युः । त‚थाऽग्नेः सामान्य‚ध‚र्म‚व‚च्चान्द‚न‚त्वाद‚योपि विशेष‚ध‚र्म्मा‚{\tiny $_{1}$}‚ ‚{\tiny $_{lb}$}‚ग‚म्याः स्युः । स‚र्व्व‚था कार्य‚कार‚ण‚भावादित्याह ।
	\pend% ending standard par
      
	  \bigskip
	  \begingroup
	
	    \large
	  
	    \begin{quote}
	  
	    
	    \stanza[\smallbreak]
	\label{pv.3.2a}\flagstanza{\tiny\textenglish{...v.3.2a}}कार्यं स्व‚भावैर्याव‚द्भिर‚विनाभावि कार्य‚व‚त् ।\&[\smallbreak]


	
	    \end{quote}
	  
	  \endgroup
	\textsuperscript{\textenglish{284/s}}

	  \pstart \leavevmode% starting standard par
	\hphantom{.}‚{\color{DodgerBlue3}‚कार्य स्व‚भावैर्याव}‚द्भिर्व्विशिष्टैर्द्‏धूम\edtext{}{\edlabel{pvv.284-1}\label{pvv.284-1}\lemma{धूम}\Bfootnote{इत्थंभूत‚ल‚क्ष‚णा तृतीया । स्व‚ग‚तैः सामान्यैः ध‚र्मेः कार्य्य‚स्यास्तु विशेष‚ध‚र्मा ग‚म‚काः स्व‚भावे कृत‚क‚त्वेन प्र‚मेय‚त्वं ग‚म‚कं ।}}व‚त्वादिभि‚{\color{DodgerBlue3}‚र‚विनाभावि} विना न भ‚व‚तीति ‚{\tiny $_{lb}$}‚कार‚णे कार‚ण‚विष‚येऽव‚धारितं तैरेव विशिष्टैः स्व‚भावैर्हेतुर्न साधार‚णैः पार्थिव‚त्वा‚{\tiny $_{lb}$}‚दिभिः । अग्निम‚न्त‚रेणापि तेषां भावात् । त‚था कार‚णेऽधिक‚र‚णे याव\edtext{}{\edlabel{pvv.284-2}\label{pvv.284-2}\lemma{याव}\Bfootnote{कार‚ण‚स्थैः ।}}द्‏भिः ‚{\tiny $_{lb}$}‚स्व‚भावैर‚ग्नित्वादिभिः सामान्य‚ध‚र्मैर‚विनाभावि कार्य‚{\tiny $_{2}$}‚ धूमादिनिश्चितं तेषां ‚{\tiny $_{lb}$}‚साम‚न्य‚ध‚र्माणां हेतुर्न विशेष‚ध‚र्माणां चान्द‚न‚त्वादीनां । तान्य‚न्त‚रेणापि धूमा‚{\tiny $_{lb}$}‚देर्द‚र्श‚नात् । य‚दि धूम‚त्व‚विशेषितं पार्थिव‚त्वं हेतुः क्रिय‚ते त‚देष्ट‚मेव व्य‚भिचारा‚{\tiny $_{lb}$}‚भावात् (।) य‚दि च कार‚ण\edtext{}{\edlabel{pvv.284-3}\label{pvv.284-3}\lemma{ण}\Bfootnote{एव‚म‚ङ्गेन ज‚न्य‚ज‚न(क)त्वं स्यादिष्य‚ते च स‚र्व्व‚था । आह ज्ञाप‚क‚हेतुर‚यं ।}}ग‚त‚चान्द‚न‚त्वादिविशेष‚ज‚नितो धूम‚स्य विशेषः श‚क्यो ‚{\tiny $_{lb}$}‚निश्चेतुं त‚दा चान्द‚न‚त्वाद‚यो ग‚म्या इष्य‚न्ते । न हि कार्य‚कार‚ण‚भावः स‚त्तामात्रेण ‚{\tiny $_{lb}$}‚ग‚भ्य‚ग‚म‚क‚भाव‚निमित्तं किन्तु निश्च‚यापेक्षः । स च याव‚न्निश्चीय‚ते त‚था‚{\tiny $_{lb}$}‚ग‚म्य‚ग‚म‚क‚भावः ।
	\pend% ending standard par
      

	  \pstart \leavevmode% starting standard par
	\hphantom{.}‚{\color{DodgerBlue3}‚भावोपि} स्व‚भावोपि हेतुः स्व‚भावे साध्ये कीदृशे हेतोर्भावः केव‚लो भाव‚मात्रं ‚{\tiny $_{lb}$}‚त‚द‚नुरोद्‏धुम‚नुव‚र्त्तितुं शील‚म‚स्येति भाव‚मात्रानुरोध त‚स्मिन् । य‚स्य स‚त्तामात्रेण ‚{\tiny $_{lb}$}‚यो ध‚र्मोऽव‚श्यं भ‚व‚ति न हेत्व‚न्त‚र‚म‚पेक्ष‚ते । त‚स्मिन् साध्ये स्व‚भावाख्यो‚{\tiny $_{4}$}‚ हेतुर्नान्य‚त्र ‚{\tiny $_{lb}$}‚य‚था व‚स्त्र‚त्वं रागे(।) एव‚ञ्च विधिसाध‚न‚त्वं कार्य‚स्व‚भाव‚योर्द‚र्शितं ।
	\pend% ending standard par
      
	  
	% new div opening: depth here is 1
	
\chapter*[{२ अनुप‚ल‚ब्धिचिन्ता}]{२ अनुप‚ल‚ब्धिचिन्ता}\label{div_pvv.3.2cd}
	  
	% new div opening: depth here is 2
	

	  \begin{center}%% label @type='head'
	\textbf{(१) दृश्यानुप‚ल‚ब्धिफ‚ल‚म्}
	\end{center}
	

	  \pstart \leavevmode% starting standard par
	इदानीं तृतीय‚हेतोः प्र‚तिषेध‚फ‚ल‚त्व‚माह ।
	\pend% ending standard par
      
	  \bigskip
	  \begingroup
	
	    \large
	  
	    \begin{quote}
	  
	    
	    \stanza[\smallbreak]
	\label{pv.3.2b}\flagstanza{\tiny\textenglish{...v.3.2b}}अप्र‚वृत्तिः प्र‚माणानां\edtext{}{\edlabel{pvv.284-4}\label{pvv.284-4}\lemma{माणानां}\Bfootnote{प्र‚त्य‚क्ष‚पृष्ठ‚भाविना निश्च‚येनाधूम‚व्यावृत्तिरूपाव‚धार‚णेन धूमादिस्व‚ल‚क्ष‚णं भास‚मानं तार्ण्णादिविशेषान‚व‚धार‚णेन चानेक‚स्व‚ल‚क्ष‚ण‚रूपं सामान्य‚ल‚क्ष‚णं लिङ्गं प्र‚त्य‚क्ष‚विष‚यो व्य‚व‚स्थाप्येत लिङ्गि च ।  --- व्य‚क्तिभेदेन मान‚ब‚हुत्वात् आग‚मापेक्ष‚या वा आग‚म‚स्यापि निवृत्तिर्नाभावं साध‚य‚तीति व‚क्ष्य‚ते ।}};\&[\smallbreak]


	
	    \end{quote}
	  
	  \endgroup
	\textsuperscript{\textenglish{285/s}}

	  \pstart \leavevmode% starting standard par
	प्र‚माण‚निवृत्तिरूपाऽनुप‚ल‚ब्धिर‚स‚ति स‚द्व्य‚व‚हारातिक्रान्त‚त्वाद‚स‚द‚विशिष्टे देश‚{\tiny $_{lb}$}‚काल‚स्व‚भाव‚विप्र‚कृष्टे सु-मे-र्व्वा-दौ (।) स‚ज्ज्ञान\edtext{}{\edlabel{pvv.285-1}\label{pvv.285-1}\lemma{ज्ज्ञान}\Bfootnote{उप‚ल‚ब्धिः क‚र्म‚ध‚र्म‚श्चेद्व‚स्तुयोग्य‚ता । क‚र्तृध‚र्म‚श्चेज्ज्ञानं । उप‚लि(? ल‚ब्धि): स‚त्व‚मेव । अस‚ताञ्चानुप‚ल‚ब्धिर‚स‚त्वं । मुख्यं स‚त्व‚म‚नेन दृश्यानुप‚ल‚ब्ध्याऽक्षेपः । अदृश्यानुप‚ल‚ब्धिः । दृश्यानुप‚ल‚ब्धौ व‚स्तुव‚शान्निवृत्तं स‚त्वं स्व‚निमित्तं ज्ञानादि निव‚र्त्त‚य‚ति । अदृश्येप्य‚भीष्ट‚कार्याक‚र‚णाद‚स‚त्क‚ल्पे प्र‚तिप‚त्त्य‚ध्य‚व‚सायान्निवृत्तं स‚त्वं स्व‚निबंध‚नं निव‚र्त‚य‚ति (।)}}श‚ब्द‚व्य‚व‚हाराणां प्र‚वृत्तिप्र‚तिषेधो‚{\tiny $_{lb}$}‚ऽप्र‚वृ‚{\tiny $_{5}$}‚त्तिस्त‚त्फ‚ला (।) एत‚च्चाप्र‚वृत्तिफ‚ल‚त्वं दृश्यादृश्यानुप‚ल‚ब्ध्योः साधार‚णंज्ञान‚{\tiny $_{lb}$}‚पूर्व‚क‚त्वात्\edtext{}{\edlabel{pvv.285-2}\label{pvv.285-2}\lemma{त्वात्}\Bfootnote{उप‚ल‚ब्धिः कार‚णं स‚द्व्य‚व‚हार‚स्य । त‚द‚भावे कार्याभाव उक्तः ।}} (।) स‚त् ज्ञान‚श‚ब्द‚व्य‚व‚हार‚स्य ज्ञानाभावे त‚द‚भाव‚स्य न्याय‚प्राप्त‚त्वात् ।
	\pend% ending standard par
      

	  \pstart \leavevmode% starting standard par
	दृश्यानुप‚ल‚ब्धेः फ‚लान्त‚र‚माह । काचित् प्र‚माण‚निवृत्तिर‚स‚त्‏ज्ञान‚म‚भाव‚ज्ञानं ‚{\tiny $_{lb}$}‚त‚त्फ‚ला । क‚थ‚मित्याह ।
	\pend% ending standard par
      \textsuperscript{\textenglish{56b/MA}}‚{\tiny $_{lb}$}‚
	  \bigskip
	  \begingroup
	
	    \large
	  
	    \begin{quote}
	  
	    
	    \stanza[\smallbreak]
	\label{pv.3.2c}\flagstanza{\tiny\textenglish{...v.3.2c}}हेतुभेद‚व्य‚पेक्ष‚या ॥ २ ॥\&[\smallbreak]


	
	    \end{quote}
	  
	  \endgroup
	

	  \pstart \leavevmode% starting standard par
	\hphantom{.}‚{\color{DodgerBlue3}‚हेतु}‚र‚नुप‚ल‚म्भ‚स्त‚स्य ‚{\color{DodgerBlue3}‚भेदो} विशेष‚ण‚मुप‚ल‚ब्धिल‚क्ष‚ण‚प्राप्त‚विष‚य‚त्वं त‚स्य ‚{\tiny $_{lb}$}‚‚{\color{DodgerBlue3}‚व्य‚पेक्ष‚या} कांक्ष‚या उप‚ल‚ब्धिल‚क्ष‚ण‚प्रा‚{\tiny $_{6}$}‚प्तानुप‚ल‚ब्धिरित्य‚र्थः ॥
	\pend% ending standard par
      

	  \pstart \leavevmode% starting standard par
	न‚नु\edtext{}{\edlabel{pvv.285-3}\label{pvv.285-3}\lemma{नु}\Bfootnote{अभाव‚प्र‚माण‚स्वीकारे य‚द् दूष‚णं त‚दिहातिदिश्य प‚रिह(र)ति ।}} य‚द्य‚नुप‚ल‚ब्ध्याऽभावः साध्य‚ते त‚दोप‚ल‚ब्ध्य‚भावोपि त‚त एवेत्य‚न\edtext{}{\edlabel{pvv.285-4}\label{pvv.285-4}\lemma{न}\Bfootnote{य‚स्य विष‚य‚स्याभावः साध्य‚ते त‚दुप‚ल‚ब्धेर‚प्य‚भावोन्य‚यानुप‚ल‚ब्ध्या त‚स्या अप्य‚न्य‚येति ।}}व‚स्था‚{\tiny $_{lb}$}‚नाद‚भावाप्र‚तिप‚त्तिः स्यात् । दृष्टान्त‚श्च न स्याद‚न्य‚स्याभाव‚साध‚न‚स्याभावाद‚नुप‚{\tiny $_{lb}$}‚ल‚ब्धेश्चान\edtext{}{\edlabel{pvv.285-5}\label{pvv.285-5}\lemma{ब्धेश्चान}\Bfootnote{येनैव प‚क्ष‚ध‚र्मेण साध्य‚ध‚र्मिण्य‚भावः साध्य‚ते । तेनैव दृष्टान्त‚ध‚र्मिण्य‚पि । त‚त्राप्य‚प‚र(ो) दृष्टान्त इति ।}}व‚स्थाप्राप्त‚त्वात् । अथ प‚दार्थान्त‚रोप‚ल‚ब्धिरेवानुप‚ल‚ब्धिस्त‚याऽध्य‚क्ष‚{\tiny $_{lb}$}‚सिद्ध्याऽभावः साध्य‚ते त‚स्मान्न दोषः । य‚द्येव‚म‚र्थान्त‚राद‚पि किन्नाभावः साध्य‚ते (।) ‚{\tiny $_{lb}$}‚त‚द‚पि ह्य‚नुप‚{\tiny $_{1}$}‚ल‚ब्धिरूपं प्र‚त्य‚क्ष‚सिद्ध‚ञ्च । को वान्योप‚ल‚म्भाभावेन स‚म्ब‚न्धः ।
	\pend% ending standard par
      

	  \pstart \leavevmode% starting standard par
	उच्य‚ते । एक‚ज्ञान‚संस‚र्ग्गि व‚स्त्व‚न्त‚रं त‚दुप‚ल‚ब्धिश्चानुप‚ल‚ब्धिर्व्विव‚क्षितोप‚{\tiny $_{lb}$}‚ल‚ब्धेर‚न्य‚त्वाद‚भ‚क्षास्प‚र्श‚नीय‚व‚त् । स एवाभावः त‚द‚तिरिक्त‚स्य विग्र‚ह‚व‚तोऽभाव‚स्या‚{\tiny $_{lb}$}‚भावात् । त‚स्मात्तादात्म्य‚म‚भावानुप‚ल‚म्भ‚योः स‚म्ब‚न्धः ।
	\pend% ending standard par
      

	  \pstart \leavevmode% starting standard par
	एवं त‚र्ह्य‚नुप‚ल‚ब्धिसिद्धिरेवाभाव‚सिद्धिः किं साध्य‚ते ॥
	\pend% ending standard par
      \textsuperscript{\textenglish{286/s}}

	  \pstart \leavevmode% starting standard par
	स‚त्यं (।) नाभावः‚{\tiny $_{2}$}‚ साध्यः सिद्ध‚त्वाद‚स्य । किन्त‚र्हि (।) विष‚योप‚द‚र्श‚नेन ‚{\tiny $_{lb}$}‚विष‚यी व्य‚व‚हारः साध्य‚ते । य‚था गोव्य‚व‚हार‚विष‚योऽयं सास्नादिस‚मुदाया\edtext{}{\edlabel{pvv.286-1}\label{pvv.286-1}\lemma{मुदाया}\Bfootnote{सिद्ध‚मेत‚द‚स्यैव गोत्वात् ।}}त्म‚{\tiny $_{lb}$}‚क‚त्वात् । (२)
	\pend% ending standard par
      \label{div_pvv.3.3}
	  
	% new div opening: depth here is 2
	

	  \begin{center}%% label @type='head'
	\textbf{(२) अनुप‚ल‚ब्धिश्च‚तुर्विधा}
	\end{center}
	

	  \pstart \leavevmode% starting standard par
	\hphantom{.}सा च प्र‚योग‚भेदाद‚{\color{DodgerBlue3}‚नुप‚ल‚ब्धिश्च‚तुर्व्विधा} क‚थ‚मित्याह ।
	\pend% ending standard par
      
	  \bigskip
	  \begingroup
	
	    \large
	  
	    \begin{quote}
	  
	    
	    \stanza[\smallbreak]
	\label{pv.3.3}\flagstanza{\tiny\textenglish{...pv.3.3}}विरुद्ध‚कार्य‚योः सिद्धिर/?/ हेतु/?/भायोः ।&दृश्यात्म‚नोर‚भावार्थानुप‚ल‚ब्धिश्च‚तुर्व्विधा ॥ ३ ॥\&[\smallbreak]


	
	    \end{quote}
	  
	  \endgroup
	

	  \pstart \leavevmode% starting standard par
	विरुद्ध‚ञ्च कार्य‚ञ्च विरुद्ध‚कार्ये कार्य‚ञ्च । प्र‚त्यास‚त्तेर्व्विरुद्ध‚स्यैव बोद्ध‚व्यं । ‚{\tiny $_{lb}$}‚‚{\color{DodgerBlue3}‚त‚योर्दृश्यात्म‚नोर्हेतुभाव‚योः} कार‚ण‚स्व‚भाव‚योश्च दृश्यात्म‚नोर्य‚थाक्र‚मं सिद्धिरुप‚{\tiny $_{lb}$}‚ल‚ब्धिर‚सिद्धि‚{\tiny $_{3}$}‚र‚नुप‚ल‚ब्धिश्च ।
	\pend% ending standard par
      

	  \pstart \leavevmode% starting standard par
	\hphantom{.}‚{\color{DodgerBlue3}‚विरुद्धो}‚प‚ल‚ब्धिर्य‚था नात्र शीत‚स्प‚र्शोऽग्नेः । व्याप्य‚व्याप‚क‚योर्व‚स्तुत‚स्तादात्म्यात् । ‚{\tiny $_{lb}$}‚\edtext{\textsuperscript{*}}{\edlabel{pvv.286-2}\label{pvv.286-2}\lemma{*}\Bfootnote{कार‚णाभाव‚स्यैव ख्याप‚नात् ।}} ‚{\color{DodgerBlue3}‚व्याप‚क}‚विरुद्धोप‚ल‚ब्धिर‚प्य‚नेनैवोक्ता भ‚व‚ति । य‚था नात्र तु\edtext{}{\edlabel{pvv.286-3}\label{pvv.286-3}\lemma{तु}\Bfootnote{अस्य व्याप‚कं शीतं ।}}षार‚स्प‚र‚र्शोऽग्नेः । ‚{\tiny $_{lb}$}‚‚{\color{DodgerBlue3}‚विरु\edtext{}{\edlabel{pvv.286-4}\label{pvv.286-4}\lemma{विरु}\Bfootnote{एक‚प्र‚कारैवेयं ।}}द्ध‚कार्यो}‚प‚ल‚ब्धिर्य‚था नात्र शीत‚स्प‚र्शो धूमात् । कार‚णा\edtext{}{\edlabel{pvv.286-5}\label{pvv.286-5}\lemma{णा}\Bfootnote{प्र‚भेदोस्याः कार‚ण‚विरुद्धोप‚ल‚ब्धिः कार‚ण‚विरुद्ध‚कार्य्योप‚ल‚ब्धिश्च प्र‚तिषेध्य‚कार‚णानुप‚ल‚म्भ‚साध‚नात् । व‚ह्निर्विरुद्धं शीतं निव‚र्त्त‚य‚न् त‚त्कार्यं निव‚र्त्त‚य‚ति (।)}}नुप‚ल‚ब्धिर्य‚था नात्र ‚{\tiny $_{lb}$}‚‚{\color{DodgerBlue3}‚धूमो}‚ऽन‚ग्नेः । ‚{\color{DodgerBlue3}‚स्व‚भावा}‚नुप‚ल‚ब्धिर्य‚था नात्र धूमोनुप‚ल‚ब्धेः । अनेन ‚{\color{DodgerBlue3}‚व्याप‚कानु}‚प‚ल‚ब्धिर‚{\tiny $_{lb}$}‚प्युक्ता\edtext{}{\edlabel{pvv.286-6}\label{pvv.286-6}\lemma{प्युक्ता}\Bfootnote{स्व‚भावाभाव‚स्यैव प्र‚तिपाद‚नात्}} य‚था नात्र शिंश‚पा वृक्षाभावात् ।
	\pend% ending standard par
      

	  \pstart \leavevmode% starting standard par
	च‚तुर्व्विधाप्य‚नुप‚ल‚ब्धिर‚भावार्था प्र‚तिषेध‚फ‚ला (।) त‚त्र स्व‚भावानुप‚ल‚ब्धिः ‚{\tiny $_{lb}$}‚स्व‚य‚मेव प्र‚तिषेध्याभाव‚रूप‚त‚या सिद्धाऽभाव‚व्य‚व‚हार‚साध‚नी । इत‚रास्तु निषेध्या‚{\tiny $_{lb}$}‚भावाव्य‚भिचारिण्यःत‚द‚भाव‚म‚भाव‚व्य‚व‚हार‚ञ्च साध‚य‚न्ति । त‚स्यासिद्ध‚त्वात् । (३)
	\pend% ending standard par
      \label{div_pvv.3.4}
	  
	% new div opening: depth here is 2
	

	  \pstart \leavevmode% starting standard par
	य‚दि विरुद्ध‚कार्योप‚ल‚ब्ध्याऽभाव‚सिद्धिस्त‚दा विरुद्ध‚कार‚णोप‚ल‚{\tiny $_{5}$}‚ब्ध्यापि किं न ‚{\tiny $_{lb}$}‚साध्य‚त इत्याह ।
	\pend% ending standard par
      
	  \bigskip
	  \begingroup
	
	    \large
	  
	    \begin{quote}
	  
	    
	    \stanza[\smallbreak]
	\label{pv.3.4}\flagstanza{\tiny\textenglish{...pv.3.4}}त‚द्विरुद्ध‚निमित्त‚स्य योप‚ल‚ब्धिः प्र‚युज्य‚ते ।&निमित्त‚योर्विरुद्ध‚त्वाभावो हि व्य‚भिचार‚वान् ॥ ४ ॥\&[\smallbreak]


	
	    \end{quote}
	  
	  \endgroup
	\textsuperscript{\textenglish{287/s}}

	  \pstart \leavevmode% starting standard par
	\hphantom{.}‚{\color{DodgerBlue3}‚त‚द्विरुद्ध‚निमित्त‚स्य} निषेध्य(=शीत‚स्प‚र्श‚व‚ह्नि)विरुद्ध‚कार‚ण(=काष्ठ)स्य ‚{\tiny $_{lb}$}‚‚{\color{DodgerBlue3}‚योप‚ल‚ब्धिः प्र‚युज्य‚ते} (।) य‚था न शीत‚स्प‚र्शोत्र काष्ठादिति सा निषेध्य‚स्य विरुद्ध‚स्य ‚{\tiny $_{lb}$}‚च ये निमित्ते त‚यो\edtext{}{\edlabel{pvv.287-1}\label{pvv.287-1}\lemma{यो}\Bfootnote{काष्ठ‚तुषार‚योः}}र्व्विरुद्ध‚त्वाभावे व्य‚भिचारिणी अनै\edtext{}{\edlabel{pvv.287-2}\label{pvv.287-2}\lemma{अनै}\Bfootnote{य‚द्य‚प्य‚ग्निज‚न‚कान्त्य‚काष्ठेन शीत‚निमित्त‚स्य विरोध‚स्त‚थापि त‚स्य कार्य‚द‚र्श‚नादेव निश्च‚य इति कार्य‚विरोध एव ।}}कान्तिकी । विरोधे तु ‚{\tiny $_{lb}$}‚निमित्त‚योरिष्य‚त एव\edtext{}{\edlabel{pvv.287-3}\label{pvv.287-3}\lemma{एव}\Bfootnote{कार‚ण‚विरुद्धोप‚ल‚ब्धिः ।}} (।) य‚था नास्य\edtext{}{\edlabel{pvv.287-4}\label{pvv.287-4}\lemma{नास्य}\Bfootnote{द‚न्त‚वीणादि ।}} रोम‚ह‚र्षादि\edtext{}{\edlabel{pvv.287-5}\label{pvv.287-5}\lemma{र्षादि}\Bfootnote{शीत‚कार्या न पिशाचादिकृताः ।}}विशेषाः\edtext{}{\edlabel{pvv.287-6}\label{pvv.287-6}\lemma{विशेषाः}\Bfootnote{रोम‚ह‚र्षाप‚न‚य‚क्ष‚म ।}} स‚न्ति स‚न्नि‚{\tiny $_{lb}$}‚हित‚द‚ह‚न‚विशेष\edtext{}{\edlabel{pvv.287-7}\label{pvv.287-7}\lemma{विशेष}\Bfootnote{संताप ।}}त्वात् (।) रोम‚ह‚र्ष‚त‚द‚भाव‚योर्व्विरो‚{\tiny $_{6}$}‚धः । त‚त्कार‚ण‚योश्च शीत\edtext{}{\edlabel{pvv.287-8}\label{pvv.287-8}\lemma{शीत}\Bfootnote{रोमाञ्च‚ताप‚निमित्त‚योः}}-\leavevmode\ledsidenote{\textenglish{57a/MA}} ‚{\tiny $_{lb}$}‚द‚ह‚न‚योरिति युक्तेय‚म‚नुप‚ल‚ब्धिः । त‚था निषेध्य(=रोम‚ह‚र्ष)विरुद्ध(=ताप)‚{\tiny $_{lb}$}‚कार‚ण(=द‚ह‚न)काय(=धूर्मो/?/)प‚ल‚ब्धिर्य‚था । ‚{\color{DodgerBlue3}‚न रोम\edtext{}{\edlabel{pvv.287-9}\label{pvv.287-9}\lemma{रोम}\Bfootnote{कार‚ण‚विरुद्ध‚कार्योप‚ल‚ब्धिरियं ।}}ह‚र्ष‚युक्त‚पुरुष‚वान‚यं} प्र‚देशो\edtext{}{\edlabel{pvv.287-10}\label{pvv.287-10}\lemma{देशो}\Bfootnote{धूमादित्य‚स्य प‚क्ष‚ध‚र्म‚त्व‚द‚र्श‚नाय पुरुषे ध‚र्मिणि न स्यात् । धूम‚स्य प्र‚देश‚ध‚र्म‚त्वात् । अत्र धूमोप‚ल‚ब्ध्याऽग्निविरुद्ध‚शीत‚कार्य‚रोम‚ह‚र्षाद्य‚भावः ।}}धूमादित्युक्ताष्ट‚विधानुप‚ल‚ब्धिः । (४)
	\pend% ending standard par
      \label{div_pvv.3.5}
	  
	% new div opening: depth here is 2
	

	  \pstart \leavevmode% starting standard par
	अन‚या दिशाऽप‚रा अपि बोद्ध‚व्याः ।
	\pend% ending standard par
      
	  \bigskip
	  \begingroup
	
	    \large
	  
	    \begin{quote}
	  
	    
	    \stanza[\smallbreak]
	\label{pv.3.5}\flagstanza{\tiny\textenglish{...pv.3.5}}इष्टं विरुद्ध‚कार्येपि देश‚कालाद्य‚पेक्ष‚णाम् ।&अन्य‚था व्य‚भिचारी स्यात् भ‚स्मेवाशीत‚साध‚ने ॥ ५ ॥\&[\smallbreak]


	
	    \end{quote}
	  
	  \endgroup
	

	  \pstart \leavevmode% starting standard par
	\hphantom{.}‚{\color{DodgerBlue3}‚वि\edtext{}{\edlabel{pvv.287-11}\label{pvv.287-11}\lemma{वि}\Bfootnote{चिर‚विन‚ष्टेऽप्य‚ग्नौ धूम‚संभ‚वात् क‚थं न व्य‚भिचार इत्याह ।}}रुप‚द्ध‚कार्ये} विरुद्ध‚कार्योप‚ल‚ब्धाव‚पि\edtext{}{\edlabel{pvv.287-12}\label{pvv.287-12}\lemma{पि}\Bfootnote{अपिश‚ब्दात् कार्य‚हेताव‚पि (।)}}विष‚ये निर्देशात् (।) देश‚स्य स‚न्नि‚{\tiny $_{lb}$}‚हित‚स्य काल‚स्य व‚र्त्त‚मान‚स्य क्व‚चिद‚तीतापेक्ष‚ण‚{\color{DodgerBlue3}‚मिष्टं}\edtext{}{\edlabel{pvv.287-13}\label{pvv.287-13}\lemma{ण}\Bfootnote{आदिश‚ब्दाद‚व‚स्थाविशेषापेक्षा ।}}(।) अन्य‚था व्य‚भिचारि ‚{\tiny $_{lb}$}‚स्या‚{\tiny $_{1}$}‚ल्लिङ्गं (।) उदाह‚र‚ण‚माह (।) ‚{\color{DodgerBlue3}‚भ‚स्मेवाशीत‚स्य} शीताभाव‚स्य ‚{\color{DodgerBlue3}‚साध‚ने} (।) ‚{\tiny $_{lb}$}‚य‚था नासीत् क्व‚चित्\edtext{}{\edlabel{pvv.287-14}\label{pvv.287-14}\lemma{चित्}\Bfootnote{देशान‚पेक्षा ।}}शीत‚स्प‚र्शो, नास्ति वात्र भ‚स्म‚न इति भ‚स्मानैकान्तिकं। ‚{\tiny $_{lb}$}‚देश‚विशेष्येऽतीत‚कालापेक्ष‚या तु स्याद्धेतुर्नासीद‚त्र शीत‚स्प‚र्शो भ‚स्म‚न ‚{\tiny $_{lb}$}‚इति ॥ (५)
	\pend% ending standard par
      \label{div_pvv.3.6}
	  
	% new div opening: depth here is 2
	

	  \pstart \leavevmode% starting standard par
	न‚नु (।)
	\pend% ending standard par
      \textsuperscript{\textenglish{288/s}}
	  \bigskip
	  \begingroup
	
	    \large
	  
	    \begin{quote}
	  
	    
	    \stanza[\smallbreak]
	\label{pv.3.6}\flagstanza{\tiny\textenglish{...pv.3.6}}हेतुना यः स‚म‚ग्रेण कार्योत्पादोनुमीय‚ते ।&अर्थान्त‚रान‚पेक्ष‚त्वात् स स्व‚भावोनुव‚र्ण्णितः ॥ ६ ॥\&[\smallbreak]


	
	    \end{quote}
	  
	  \endgroup
	

	  \pstart \leavevmode% starting standard par
	\hphantom{.}‚{\color{DodgerBlue3}‚हेतुना स‚म‚ग्रेण यः कार्योत्पादोनुमीय‚ते} (।) स क‚स्मिन् हेताव\edtext{}{\edlabel{pvv.288-1}\label{pvv.288-1}\lemma{हेताव}\Bfootnote{अप्र‚तिब‚द्ध‚साम‚र्थ्याच्चेत् कार्य‚मुत्प‚न्नं दुश्येतैव । कार‚ण‚कार‚णात्तु व्य‚भिचार इत्याह ।}}न्त‚र्भ‚व‚ती‚{\tiny $_{lb}$}‚त्याह (।) ‚{\color{DodgerBlue3}‚स स्व‚भाव}‚हेतुर‚{\color{DodgerBlue3}‚नुव‚र्ण्णितः} । न हि स‚म‚ग्रा‚{\tiny $_{2}$}‚द्धेतोः कार्य‚स‚म्भ‚वोनुमीय‚ते (।) ‚{\tiny $_{lb}$}‚किन्त‚र्हि (।) कार्यार्ज‚न‚योग्य‚त्वं (।) त‚च्चार्थान्त‚रान‚पेक्ष‚त्वाद्धेतुसाक‚ल्य‚मात्रानु‚{\tiny $_{lb}$}‚ब‚न्ध्येवेति त‚स्मिन् साध्ये हेतुसाक‚ल्यं स्व‚भाव‚हेतुरेव । (६)
	\pend% ending standard par
      \label{div_pvv.3.7}
	  
	% new div opening: depth here is 2
	

	  \pstart \leavevmode% starting standard par
	क‚स्मात् पुनः कार्य‚मेव नानुमीय‚त इत्याह ।
	\pend% ending standard par
      
	  \bigskip
	  \begingroup
	
	    \large
	  
	    \begin{quote}
	  
	    
	    \stanza[\smallbreak]
	\label{pv.3.7}\flagstanza{\tiny\textenglish{...pv.3.7}}साम‚ग्रोफ‚ल‚श‚क्तीनां प‚रिणामानुब‚न्धिनि ।&अनैकान्तिक‚ता कार्ये प्र‚तिब‚न्ध‚स्य स‚म्भ‚वात् ॥ ७ ॥\&[\smallbreak]


	
	    \end{quote}
	  
	  \endgroup
	

	  \pstart \leavevmode% starting standard par
	\hphantom{.}कार्येऽनुमेयेऽनैकान्तिक‚ता हेतुसाक‚ल्य‚स्य । कीदृशे (।) ‚{\color{DodgerBlue3}‚साम‚ग्र्‏याः} फ‚ल‚ञ्च ताः ‚{\tiny $_{lb}$}‚[उत्त‚र‚क्ष‚णं] (।) ‚{\color{DodgerBlue3}‚श‚क्त}‚य‚श्च तासां ‚{\color{DodgerBlue3}‚प‚रिणामः}\edtext{}{\edlabel{pvv.288-2}\label{pvv.288-2}\lemma{तासां}\Bfootnote{य‚था तान‚वितान‚व‚त‚स्त(न्तू)न् व्यापार‚व‚त्पुंसाधिष्ठितान् दृष्ट्‏वा प‚टानुमानं । कार‚णाख्यं नानुप‚ल‚ब्धि(त्रिधैवेति इष्य‚ते)र्विधिसाध‚नात् । कार‚ण‚स्व‚भाव‚त्वान्न कार्यं । अन्येनान्य(ानु)मानान्न स्व‚भावोपि हेतुरित्याह । नात्र कार्योत्पादानुमानं ‚{\tiny $_{lb}$}‚किन्तु स‚म‚ग्राणां कार्योत्पाद‚न‚योग्य‚ता । स्व‚भाव‚विशेषः कार्य‚मुत्प‚द्य‚तेऽस्मादिति ‚{\tiny $_{lb}$}‚श्लोकेप्य‚र्थः प्र‚योगो या स‚म‚ग्रा सोत्त‚र‚प‚रिणामात् कार्योत्पाद‚न‚योग्या य‚थोत्पादित‚{\tiny $_{lb}$}‚कार्य‚साम‚ग्री (।)}} कार्योत्पादानुगुण‚स्तार‚{\tiny $_{3}$}‚त‚म्य‚योगी ‚{\tiny $_{lb}$}‚क्ष‚ण‚प्र‚ब‚न्धः । ‚{\color{DodgerBlue3}‚त‚द‚नुब‚न्धिनि} त‚द‚पेक्षिणि । अत एव प्र‚तिब‚न्ध‚स्य स‚म्भ‚वादित्युक्तं । (७)
	\pend% ending standard par
      \label{div_pvv.3.8}
	  
	% new div opening: depth here is 2
	

	  \pstart \leavevmode% starting standard par
	न ख‚लु कार्यं हेत‚वो मिलिता इत्येव भ‚व‚ति (।) किं त्वेषां विशेष‚म‚पेक्ष‚ते‚{\tiny $_{lb}$}‚ऽत्रान्त‚रे च श‚क्तिव्याघातो म‚न्त्र‚त‚न्त्रादिना वैक‚ल्य‚ञ्च स‚म्भ‚व‚तीति कार्य‚स्याव‚श्य‚{\tiny $_{lb}$}‚भावान्न त‚द‚नुमानं ।
	\pend% ending standard par
      
	  \bigskip
	  \begingroup
	
	    \large
	  
	    \begin{quote}
	  
	    
	    \stanza[\smallbreak]
	\label{pv.3.8}\flagstanza{\tiny\textenglish{...pv.3.8}}एक‚साम‚ग्र्य‚धीन‚स्य रूपादे र‚स‚तो ग‚तिः ।&हेतुध‚र्मानुमानेन धूमेन्ध‚न‚विकार‚व‚त् ॥ ८ ॥\&[\smallbreak]


	
	    \end{quote}
	  
	  \endgroup
	

	  \pstart \leavevmode% starting standard par
	\hphantom{.}या च ‚{\color{DodgerBlue3}‚र‚स‚तो} म‚धुरादिकात् (।) ‚{\color{DodgerBlue3}‚रूपादे}‚रादिश‚ब्दाद् ग‚न्ध‚स्य स्प‚र्श‚स्य च ‚{\tiny $_{lb}$}‚‚{\color{DodgerBlue3}‚एक‚साम‚ग्र्‏य‚{\tiny $_{4}$}‚धीन‚स्य} र‚सादिना \edtext{}{\edlabel{pvv.288-3}\label{pvv.288-3}\lemma{सादिना}\Bfootnote{पुन‚स्त्रिधैवेति त्र‚स्य‚ति (।) अकार्य‚कार‚ण‚भूते नानुमेयाद‚न्येनान्ध‚कारे मातुलुङ्गादिर‚स‚मास्वाद्य च‚म्प‚क‚ग‚न्ध‚माघ्राय व‚ह्निस्प‚र्श‚म‚नुभूय तेषां रूप‚{\tiny $_{lb}$}‚सामान्य‚म‚नुमीय‚ते । सा क‚थ‚म‚नुमेयाद‚न्येनानुमानान्न स्व‚भाव‚हेतुः ।}}स‚हैक‚साम‚ग्र‚याय‚त्त‚स्य ग‚तिः सा क‚थ‚मित्याह(।) ‚{\tiny $_{lb}$}‚\leavevmode\ledsidenote{\textenglish{289/s}} ‚{\color{DodgerBlue3}‚हेतुध‚र्मानुमानेन} र‚स‚कार‚ण‚स्य ‚{\color{DodgerBlue3}‚ध‚र्मो र‚सादिस‚ह‚च‚र‚रूप‚ज‚न‚क‚त्वं (।)त‚द‚नुमानेन र‚साद्} रूपादिग‚तिः । न हि कार्यं र‚सः कार‚ण‚म‚न्त‚रेण । कार‚ण‚ञ्चास्य\edtext{}{\edlabel{pvv.289-1}\label{pvv.289-1}\lemma{ञ्चास्य}\Bfootnote{इति कार्यात् कार‚ण‚विशेषानुमानं ।}} र‚स‚स‚ह‚कारि रूप‚{\tiny $_{lb}$}‚ज‚न‚कं पुञ्जात् पुञ्जोत्प‚त्तेः । अतः त‚स्मिन्न‚नुमितेऽनुमित‚मेव रूपं ‚{\color{DodgerBlue3}‚धूमेन्ध‚न‚विकार-} व‚त् ।‚{\tiny $_{5}$}‚ धूमाद्धेतुध‚र्मानुमानेनेन्ध‚न‚विकार‚स्याङ्गारादेर्द्धूम‚स‚ह‚च‚र‚स्योवानुमानं । (८)
	\pend% ending standard par
      \label{div_pvv.3.9}
	  
	% new div opening: depth here is 2
	

	  \pstart \leavevmode% starting standard par
	एत‚देव स्फुट‚य‚न्ना (।)
	\pend% ending standard par
      
	  \bigskip
	  \begingroup
	
	    \large
	  
	    \begin{quote}
	  
	    
	    \stanza[\smallbreak]
	\label{pv.3.9}\flagstanza{\tiny\textenglish{...pv.3.9}}श‚क्तिप्र‚वृत्त्या न विना र‚सः सैवान्य‚कार‚ण‚म् ।&इत्य‚तीतैक‚कालानां जाय‚ते कार्य‚लिङ्ग‚तः ॥ ९ ॥\&[\smallbreak]


	
	    \end{quote}
	  
	  \endgroup
	

	  \pstart \leavevmode% starting standard par
	\hphantom{.}र‚स‚हेतोः ‚{\color{DodgerBlue3}‚श‚क्तिप्र‚वृत्त्या} श‚क्त्याभिमुख्येन ‚{\color{DodgerBlue3}‚विना न र‚स} उत्प‚द्य‚ते (।) सैव ‚{\tiny $_{lb}$}‚र‚स‚हेतोः श‚क्तिप्र‚वृत्तिर‚न्य‚स्य रूपादेः ‚{\color{DodgerBlue3}‚कार‚णं} इत‚र‚स‚मान‚काल‚रूपादिज‚न‚क‚त्वेन ‚{\color{DodgerBlue3}‚र‚स}‚{\tiny $_{lb}$}‚हेतोर\edtext{}{\edlabel{pvv.289-2}\label{pvv.289-2}\lemma{हेतोर}\Bfootnote{र‚सोपादान‚स‚मान‚काल‚भाविनोतीताः । लिङ्ग‚भूत‚र‚स‚स‚ह‚भाविन एक‚कालाः । इति अनेन द्वारेण । नानाग‚त‚ग‚तिः(।)}} ‚{\color{DodgerBlue3}‚तीत‚स्य} हेतो‚{\color{DodgerBlue3}‚रेक‚कालानां} र‚स‚स‚ह‚च‚राणां च रूपादीनां ‚{\color{DodgerBlue3}‚ग‚तिः} त‚{\tiny $_{6}$}‚स्य र‚स‚{\tiny $_{lb}$}‚रूप‚हेतोः ‚{\color{DodgerBlue3}‚कार्याद्} र‚साल्लिङ्गाज्जाता ॥
	\pend% ending standard par
      

	  \pstart \leavevmode% starting standard par
	एवं पिपीलिकोत्स‚र‚णादि\edtext{}{\edlabel{pvv.289-3}\label{pvv.289-3}\lemma{णादि}\Bfootnote{योग्य‚तानुमानं ।}}द‚र्श‚नात् व‚र्ष(ा) पिपी\edtext{}{\edlabel{pvv.289-4}\label{pvv.289-4}\lemma{पिपी}\Bfootnote{म‚त्स्योत्प‚त‚नात् । गृहीताण्ड‚स्यान्य‚त्र ग‚तिः ब‚लाकात‚स्तोयं (।)}}लिकादिक्षोभ‚हेतोर्भूत‚प‚रि‚{\tiny $_{lb}$}‚णाम‚विशेष‚स्यानुमानात् व‚र्षानुमानं कार्य‚लिङ्ग‚जं वेदित‚व्यं । न‚दीपूरादेर्व्व‚र्ष‚कार्य‚त्वं ‚{\tiny $_{lb}$}‚व्य‚क्त‚मेव ॥ (९)
	\pend% ending standard par
      \label{div_pvv.3.10}
	  
	% new div opening: depth here is 2
	
	  \bigskip
	  \begingroup
	
	    \large
	  
	    \begin{quote}
	  
	    
	    \stanza[\smallbreak]
	\label{pv.3.10}\flagstanza{\tiny\textenglish{...v.3.10}}हेतुना योऽस‚म‚ग्रेण कार्योत्पादोऽनुमीय‚ते ।&त‚च्छेष‚व‚द‚साम‚र्थ्याद् देहाद् रागानुमान‚व‚त् ॥ १० ॥\&[\smallbreak]


	
	    \end{quote}
	  
	  \endgroup
	

	  \pstart \leavevmode% starting standard par
	\hphantom{.}‚{\color{DodgerBlue3}‚यः} पुन‚{\color{DodgerBlue3}‚र्हेतुना}‚ऽस‚म‚ग्रेण कार्योत्पादः कार्योत्पाद‚न‚योग्य‚त्व‚म‚नुमीय‚ते\edtext{}{\edlabel{pvv.289-5}\label{pvv.289-5}\lemma{ते}\Bfootnote{मीमांसाकादिभिः ।}} त‚च्छेष‚{\tiny $_{lb}$}‚व‚द‚नैकान्तिक‚म‚साम‚र्थ्यात् । स‚म‚र्थं‚{\tiny $_{7}$}‚ हि कार्योत्पाद‚न‚योग्यं न च व्य‚ग्राणां स‚म‚र्थ‚ता ।\leavevmode\ledsidenote{\textenglish{57b/MA}} ‚{\tiny $_{lb}$}‚देहात्\edtext{}{\edlabel{pvv.289-6}\label{pvv.289-6}\lemma{देहात्}\Bfootnote{कार्यं स‚र्व‚थाऽस‚म्ब‚द्धं योग्य‚तापि नास्ति क्ष‚ण‚प‚रिणामेनापि रागादिमान‚यं देह इन्द्रिय‚बुद्धिम‚त्वात् ।}} बुद्धीन्द्रियादेश्च हेतो रागानुमान‚व‚त् । देहादीनां क‚थ‚ञ्चिद्\edtext{}{\edlabel{pvv.289-7}\label{pvv.289-7}\lemma{ञ्चिद्}\Bfootnote{अदेहादे रागादृष्टेः}} राग‚हेतुत्वेपि\edtext{}{\edlabel{pvv.289-8}\label{pvv.289-8}\lemma{हेतुत्वेपि}\Bfootnote{अहं म‚मेत्यात्मात्मीयाभिनिवेश‚पूर्व्व‚क‚त्वात् (।)}} ‚{\tiny $_{lb}$}‚\leavevmode\ledsidenote{\textenglish{290/s}} नायोनिशोम‚न‚स्कार\edtext{}{\edlabel{pvv.290-1}\label{pvv.290-1}\lemma{स्कार}\Bfootnote{योनिशो म‚नो नैरात्म्य‚ज्ञानं (।) त‚द्विरुद्ध‚मात्मादिज्ञानं ।}} र‚हितानां कार‚ण‚त्वं स‚हितानामेव\edtext{}{\edlabel{pvv.290-2}\label{pvv.290-2}\lemma{हितानामेव}\Bfootnote{रागे साध्ये त‚त्र हितोप‚लंबिप‚क्ष‚स्त‚त्र देहाद्य‚दृष्टेर‚स्ति (।)}} हेतुत्वात् । त‚तःकेव‚लाद् ‚{\tiny $_{lb}$}‚देहाद् रागानुमान‚म‚नैकान्तिकं । (१०)
	\pend% ending standard par
      \label{div_pvv.3.11}
	  
	% new div opening: depth here is 2
	

	  \pstart \leavevmode% starting standard par
	त‚था (।)
	\pend% ending standard par
      
	  \bigskip
	  \begingroup
	
	    \large
	  
	    \begin{quote}
	  
	    
	    \stanza[\smallbreak]
	\label{pv.3.11}\flagstanza{\tiny\textenglish{...v.3.11}}विप‚क्षेऽदृष्टिमात्रेण कार्य‚सामान्य‚द‚र्श‚नात् ।&हेतुज्ञान‚प्र‚माणामं व‚च‚नाद् रागितादिव‚त् ॥ ११ ॥\&[\smallbreak]


	
	    \end{quote}
	  
	  \endgroup
	

	  \pstart \leavevmode% starting standard par
	\hphantom{.}‚{\color{DodgerBlue3}‚विप‚क्षेऽदृष्टिमात्रेण कार्य‚सामान्य‚स्य\edtext{}{\edlabel{pvv.290-3}\label{pvv.290-3}\lemma{स्य}\Bfootnote{व‚क्तुकाम‚ताज‚स्य राग्य‚रागिग‚त‚व‚च‚न‚स्य (।) व‚क्तुकाम‚तैव राग‚श्चेदिष्ट‚न्त‚त् । अभिष्व‚ङ्गो नो रागः ।}}} क्व‚चिद् ध‚र्मिणि ‚{\color{DodgerBlue3}‚द‚र्श‚नात् । हेतुज्ञानं}\edtext{}{\edlabel{pvv.290-4}\label{pvv.290-4}\lemma{र्मिणि}\Bfootnote{नैवं क‚रुणाध‚र्माल‚म्ब‚न‚त्वात् । न वाग्विशेषात् स‚रागोपि वीत‚राग‚व‚दित्य‚निश्च‚यात् ।}} ‚{\tiny $_{lb}$}‚य‚त् ‚{\color{DodgerBlue3}‚प्र‚माणाभं} स‚न्दिग्ध‚विप‚क्ष‚व्यावृत्तिकं‚{\tiny $_{1}$}‚ त‚त् ‚{\color{DodgerBlue3}‚व‚च‚नाद्} रागितादिव‚त्\edtext{}{\edlabel{pvv.290-5}\label{pvv.290-5}\lemma{त्}\Bfootnote{द्वेष‚मोहादि ।}} । य‚था ‚{\tiny $_{lb}$}‚व‚च‚न‚स्य वीत‚रागेऽद‚र्श‚नात् क्व‚चिद् द‚र्श‚नाद् रागानुमान‚म‚नैकान्तिकं (।) ‚{\tiny $_{lb}$}‚न हि राग‚व‚च‚न‚योः कार्य‚कार‚ण‚भावः सिद्धः । व‚क्तुकाम‚ता\edtext{}{\edlabel{pvv.290-6}\label{pvv.290-6}\lemma{ता}\Bfootnote{अविशिष्टं विव‚क्षामात्रं ।}}हेतुत्वात् त‚स्य । ‚{\tiny $_{lb}$}‚सा च वीत‚राग‚स्य क‚रुण‚या\edtext{}{\edlabel{pvv.290-7}\label{pvv.290-7}\lemma{या}\Bfootnote{नेय‚म‚भिष्व‚ङ्गो ध‚र्माल‚म्ब‚न‚त्वात् ।}}संभ‚व‚ति । य‚द्य‚पि व‚क्त‚रि रागो दृश्य‚ते त‚थापि ‚{\tiny $_{lb}$}‚क्व‚चिद् व‚क्तुकाम‚तास‚त्वेपि रागाभावे व‚च‚नाभावासिद्धेर्न त‚त्कार्य‚तासिद्धिरुप‚ल‚{\tiny $_{lb}$}‚ख‚ण्डाद् व‚क्तुकाम‚ता‚{\tiny $_{2}$}‚निवृत्तेरेव व‚च‚नाभावो न तु रागाभावात् (।) त‚तो वीत‚{\tiny $_{lb}$}‚रागात् स‚न्दिग्धोस्य व्य‚तिरेकः ॥ (११)
	\pend% ending standard par
      \label{div_pvv.3.12}
	  
	% new div opening: depth here is 2
	
	  \bigskip
	  \begingroup
	
	    \large
	  
	    \begin{quote}
	  
	    
	    \stanza[\smallbreak]
	\label{pv.3.12}\flagstanza{\tiny\textenglish{...v.3.12}}न चाद‚र्श‚न‚मात्रेण विप‚क्षेऽव्य‚भिचारिता ।&संभाव्य व्य‚भिचारित्वात् स्थालीतंडुल‚पाक‚व‚त् ॥ १२ ॥\&[\smallbreak]


	
	    \end{quote}
	  
	  \endgroup
	

	  \pstart \leavevmode% starting standard par
	\hphantom{.}‚{\color{DodgerBlue3}‚न च विप‚क्षे}\edtext{\textsuperscript{*}}{\edlabel{pvv.290-8}\label{pvv.290-8}\lemma{*}\Bfootnote{विप‚क्षे दृष्ट्या व्य‚भिचारी न च विरागे व‚च‚नं दृष्ट‚मित्याह । अनैकान्तिक‚स्याद‚र्श‚न‚मात्रेणानिरासाद‚नैकान्तिक‚विप‚क्षेण य‚द्व‚च‚नं तेन वैशेषिकेण वायोः स‚त्व‚साध‚नार्थं स्प‚र्श‚श्च न च दृष्टानामिति (वैशे॰ सू १० । अ॰ २ । आ॰ १)}} हेतोर‚द‚र्श‚न‚{\color{DodgerBlue3}‚मात्रेण} साध्याभाव‚प्र‚युक्त‚साध‚नाभाव‚निश्च‚य‚र‚हितेन ‚{\tiny $_{lb}$}‚साध्या‚{\color{DodgerBlue3}‚व्य‚भिचारिता} साध्य‚ते स‚म्भाव्य‚व्य‚भिचा(ि)र‚त्वात् । य‚दि विप‚क्षाद्धेतु‚{\tiny $_{lb}$}‚निवृत्तिनिश्च‚य एवं व्य‚भिचार‚श‚ङ्कानिरासः । स तु नास्तीति त‚स्याप्य‚भावः । ‚{\tiny $_{lb}$}‚\leavevmode\ledsidenote{\textenglish{291/s}} ‚{\color{DodgerBlue3}‚स्थालीत‚ण्डुल‚पाक‚व‚त्} । स्थाल्य‚न्त‚र्ग‚ता‚{\tiny $_{3}$}‚नां त‚ण्डुलानां\edtext{}{\edlabel{pvv.291-1}\label{pvv.291-1}\lemma{ण्डुलानां}\Bfootnote{बाहुल्येन प‚क्ष‚द‚र्श‚नेपि प‚क्ष‚स‚म‚हेत‚वः प‚क्षा इति युक्तं वैध‚र्म्ये ।}}पाक‚स्येवानुमानं । त‚दित‚{\tiny $_{lb}$}‚रेषां पाकाद‚र्श‚नात् शेष‚व‚त् ॥ (१२)
	\pend% ending standard par
      \label{div_pvv.3.13}
	  
	% new div opening: depth here is 2
	

	  \begin{center}%% label @type='head'
	\textbf{(क) शेष‚व‚द‚नुमान‚निरासः}
	\end{center}
	

	  \pstart \leavevmode% starting standard par
	किं पुनः शेष‚व‚दित्युच्य‚ते\edtext{}{\edlabel{pvv.291-2}\label{pvv.291-2}\lemma{ते}\Bfootnote{पूर्व्व‚व‚च्छेष‚व‚त्सामान्य‚तो दृष्ट‚मिति (न्याय‚सू॰१।१।५) शेष‚व‚ल्ल‚क्ष‚णं ‚{\tiny $_{lb}$}‚नैयायिक‚स्य विरुद्धं कार‚णात् कार्यानुमानं पूर्व्व‚व‚त् । शेषः कार्यं य‚स्यास्ति त‚च्छेष‚व‚त् ‚{\tiny $_{lb}$}‚(।) कार्यात् कार‚णानुमानं । अतः पृच्छ‚ति किमिति (।)}} । इत्याह ।
	\pend% ending standard par
      
	  \bigskip
	  \begingroup
	
	    \large
	  
	    \begin{quote}
	  
	    
	    \stanza[\smallbreak]
	\label{pv.3.13}\flagstanza{\tiny\textenglish{...v.3.13}}य‚स्याद‚र्श‚न‚मात्रेण व्य‚तिरेकः प्र‚द‚र्श्य‚ते ।&त‚स्य संश‚य‚हेतुत्वाच्छेष‚व‚त्त‚दुदाहृत‚म् ॥ १३ ॥\&[\smallbreak]


	
	    \end{quote}
	  
	  \endgroup
	

	  \pstart \leavevmode% starting standard par
	\hphantom{.}‚{\color{DodgerBlue3}‚य‚स्य} लिङ्ग‚स्या‚{\color{DodgerBlue3}‚द‚र्श‚न‚मा}‚त्रेण विप‚क्षे\edtext{}{\edlabel{pvv.291-3}\label{pvv.291-3}\lemma{क्षे}\Bfootnote{क्व‚चित् ।}} ‚{\color{DodgerBlue3}‚व्य‚तिरेकः प्र‚द‚र्श्य‚ते} । न तु कार‚ण‚व्याप‚क‚{\tiny $_{lb}$}‚निवृत्त्या त‚स्य संश‚य‚हेतुत्वात् साध्यानिश्चाय‚क‚त्वात् शेष‚व‚त् त‚दुदाहृतं । स‚न्दिग्ध‚{\tiny $_{lb}$}‚विप‚क्ष‚व्यावृत्तिक‚त्वं शेष‚व‚दुच्य‚ते प्र‚तिब‚न्धाभावादित्य‚र्थः ।‚{\tiny $_{4}$}‚(१३)
	\pend% ending standard par
      \label{div_pvv.3.14}
	  
	% new div opening: depth here is 2
	

	  \begin{center}%% label @type='head'
	\textbf{(ख) त्रिषु हेतुरूपेषु निश्च‚यः}
	\end{center}
	
	  \bigskip
	  \begingroup
	
	    \large
	  
	    \begin{quote}
	  
	    
	    \stanza[\smallbreak]
	\label{pv.3.14}\flagstanza{\tiny\textenglish{...v.3.14}}हेतोस्त्रिष्व‚पि रूपेषु निश्च‚य‚स्तेन व‚र्णितः ।&असिद्ध‚विप‚रीतार्थ‚व्य‚भिचारिविप‚क्ष‚तः ॥ १४ ॥\&[\smallbreak]


	
	    \end{quote}
	  
	  \endgroup
	

	  \pstart \leavevmode% starting standard par
	सूत्र‚मुक्तं (।) अस्याय‚म‚र्थः (।) यो गुणः स द्र‚व्याश्र‚यी त‚द्य‚था रूपादिः । अपाक‚{\tiny $_{lb}$}‚जानुष्णाशीत‚स्प‚र्श‚श्य (? स्य) गुणः त‚स्मात् त‚स्याश्र‚य‚भूतेन द्र‚व्येण भ‚वित‚व्यं‚{\tiny $_{1}$}‚ । (।) ‚{\tiny $_{lb}$}‚न चायं दृष्टानां पृथिव्यादीनां गुणः (।) तेषां पाक‚जानुष्णाशीत‚स्प‚र्शादिगुण‚त्वात् ‚{\tiny $_{lb}$}‚(।) त‚तो य‚स्यायं गुणः स वायुर्भ‚विष्य‚तीत्युक्ते वैशेषिकेण (।) त‚त्राचार्य दि ग्ना गे‚{\tiny $_{lb}$}‚नोक्तं (।) य‚द्य‚द‚र्श‚न‚मात्रेण दृष्टेभ्यः‚{\tiny $_{2}$}‚ (स्प‚र्श‚स्य) प्र‚तिषेधः क्रिय‚ते न च सोपि ‚{\tiny $_{lb}$}‚युक्त इति दृश्यादृश्य‚स‚मुदाय‚स्य सामान्येन निषेधात् (।) य‚देत‚दुक्तं त‚द्विरुध्य‚त ‚{\tiny $_{lb}$}‚इति वा र्ति क का रो द‚र्श‚य‚न्नाह दृष्टेत्यादि । य‚द्य‚दृष्ट्या निवृतिः स्यात् ‚{\tiny $_{lb}$}‚त‚दाऽदृष्टेन द‚र्श‚नात् कार‚णाद‚पाक‚ज‚स्यानुष्णाशीत‚{\tiny $_{3}$}‚स्प‚र्श‚स्य दृष्टाऽयुक्तिः (।) दृष्टेषु ‚{\tiny $_{lb}$}‚पृथिव्यादिष्व‚स‚ङ्ग‚तिर्यां व‚र्णिता वै शे षि कैः य‚स्या आचार्येणायुक्त‚त्व‚मुक्तं सा ‚{\tiny $_{lb}$}‚स्याद‚विरोधिनी युक्तैव स्यादित्य‚र्थः । वायुप्र‚क‚र‚णे पृथिव्यादिभ्यः‚{\tiny $_{4}$}‚ य‚दि स्प‚र्शा‚{\tiny $_{lb}$}‚देर्गुण‚त्वं सिद्धं स्यात् त‚तो वायुद्र‚व्यानुमानं स्यात् । सैव त्व‚सिद्धा स्वात‚न्त्र्येण ‚{\tiny $_{lb}$}‚प्र‚तीतेः स्प‚र्शाविशेषोस्माकं वायुः (।) आचार्येण तु स्प‚र्श‚व्य‚तिरिक्तं वायुम‚भ्युप‚{\tiny $_{lb}$}‚ग‚म्यायुक्त‚त्व‚मुक्तं प‚रे ।
	\pend% ending standard par
      \textsuperscript{\textenglish{292/s}}

	  \pstart \leavevmode% starting standard par
	\hphantom{.}‚{\color{DodgerBlue3}‚तेन} प्र‚तिब‚न्ध‚स्याव‚श्याभ्युप‚ग‚न्त‚व्य‚त्वेन ‚{\color{DodgerBlue3}‚हेतोस्त्रिष्व‚पि} प‚क्ष‚ध‚र्मा\edtext{}{\edlabel{pvv.292-1}\label{pvv.292-1}\lemma{र्मा}\Bfootnote{प्र‚सिद्ध‚स्तु द्व‚योर‚पि साध‚न‚मित्यादिना वादिविवादिनोः ।}}न्व‚य‚व्य‚तिरेकेषु ‚{\tiny $_{lb}$}‚‚{\color{DodgerBlue3}‚निश्च‚य} आचार्य ‚{\color{DodgerBlue3}‚दि ग्ना गे न} वार्ण्णितः । क‚थ‚मुक्त इत्याह । ‚{\color{DodgerBlue3}‚असिद्ध‚स्य विप‚रीता}‚{\tiny $_{lb}$}‚र्थ‚स्य ‚{\color{DodgerBlue3}‚विरुद्ध‚स्य व्य‚भिचारिणो}‚ऽनैकान्तिक‚स्य ‚{\color{DodgerBlue3}‚विप‚क्षेण}\edtext{}{\edlabel{pvv.292-2}\label{pvv.292-2}\lemma{स्य}\Bfootnote{तृतीया आद्य‚दिग्धातु सिविप‚क्ष‚त (?) इत्य‚त्र ।}}। त‚त्रासिद्ध‚त्व‚विप‚क्षेण ‚{\tiny $_{lb}$}‚प‚क्ष‚ध‚र्म‚त्व‚स्य निश्च‚य उक्तः । विरुद्ध‚विप‚क्षेणान्व‚य‚स्य । अनैकान्तिक‚विप‚क्षेण व्य‚ति‚{\tiny $_{lb}$}‚रेक‚स्य ॥ (१४)
	\pend% ending standard par
      \label{div_pvv.3.15}
	  
	% new div opening: depth here is 2
	

	  \begin{center}%% label @type='head'
	\textbf{(३) व्याप्तिचिन्ता}
	\end{center}
	

	  \begin{center}%% label @type='head'
	\textbf{(क) प्र‚तिब‚न्धो दिग्नागेष्टः}
	\end{center}
	

	  \pstart \leavevmode% starting standard par
	न ह्य‚{\tiny $_{5}$}‚ स‚ति प्र‚तिब‚न्धे विप‚क्षाद् व्य‚तिरेकः श‚क्य‚निश्च‚यः । उक्त‚श्च निश्च‚य‚{\tiny $_{lb}$}‚स्त‚तः प्र‚तिब‚न्धोप्याचार्येणेष्ट इति प्र‚तीय‚ते । अन्य‚था (।)
	\pend% ending standard par
      
	  \bigskip
	  \begingroup
	
	    \large
	  
	    \begin{quote}
	  
	    
	    \stanza[\smallbreak]
	\label{pv.3.15}\flagstanza{\tiny\textenglish{...v.3.15}}व्य‚भिचारिविप‚क्षेण वैध‚र्म्य‚व‚च‚नं च य‚त् ।&य‚द्य‚दृष्टिफ‚लं त‚च्च त‚द‚नुक्तेपि ग‚म्य‚ते ॥ १५ ॥\&[\smallbreak]


	
	    \end{quote}
	  
	  \endgroup
	

	  \pstart \leavevmode% starting standard par
	प्र‚तिब‚न्धानिष्टौ व्य‚भिचारिणोऽनैकान्तिक‚स्य विप‚क्षेण वैध‚र्म्य‚स्य व्य‚तिरेक‚स्य ‚{\tiny $_{lb}$}‚व‚च‚नं य‚दाचार्य‚स्य\edtext{}{\edlabel{pvv.292-3}\label{pvv.292-3}\lemma{स्य}\Bfootnote{त‚द्व्य‚र्थ‚मित्याकूतं}} (।) एष ताव‚न्न्यायः य‚दुभ‚यं व‚क्त‚व्यं\edtext{}{\edlabel{pvv.292-4}\label{pvv.292-4}\lemma{व्यं}\Bfootnote{न्याय‚मुख आचार्येणोक्तं साध‚र्म्यं वैध‚र्म्यं चोभ‚यं ।}} विरुद्धानैकान्तिक‚{\tiny $_{lb}$}‚प्र‚तिप‚क्षेणेत्युभ‚य‚म‚न्व‚य‚व्य‚तिरेकौ । त‚च्च य‚द्य‚दृष्टिफ‚ल‚म‚द‚र्श‚न‚मात्र‚फ‚लं त‚द‚द‚र्श‚{\tiny $_{6}$}‚‚{\tiny $_{lb}$}‚न‚मात्र‚{\color{DodgerBlue3}‚म‚नुक्तेपि}\edtext{}{\edlabel{pvv.292-5}\label{pvv.292-5}\lemma{मात्र}\Bfootnote{वैध‚र्म्य‚स्य ।}} व्य‚भिचारिविप‚क्षेण व्य‚तिरेको ग‚म्य‚ते (।) हेतुत्रैरूप्य‚निर्द्देशादेव ‚{\tiny $_{lb}$}‚विप‚क्षेऽद‚र्श‚न‚मात्र‚स्य ग‚त‚त्वात् । त‚स्माद् ‚{\color{DodgerBlue3}‚वैध‚र्म्य‚व‚च‚नेन}\edtext{}{\edlabel{pvv.292-6}\label{pvv.292-6}\lemma{स्माद्}\Bfootnote{अनैकान्तिक‚स्याद‚र्श‚न‚मात्रेणानिरासाद‚नैकान्तिक‚विप‚क्षेण य‚द्व‚च‚नं तेन ।}}विप‚क्षे हेत्व‚भावः क‚थ्य‚ते । ‚{\tiny $_{lb}$}‚स चाद‚र्श‚न‚मात्रेण न सिध्य‚ति ॥ (१५)
	\pend% ending standard par
      \label{div_pvv.3.16}
	  
	% new div opening: depth here is 2
	

	  \pstart \leavevmode% starting standard par
	नास्तीति व‚च‚नादेव सिध्य‚तीति चेत् ।
	\pend% ending standard par
      
	  \bigskip
	  \begingroup
	
	    \large
	  
	    \begin{quote}
	  
	    
	    \stanza[\smallbreak]
	\label{pv.3.16}\flagstanza{\tiny\textenglish{...v.3.16}}न च नास्तीति व‚च‚नात् त‚न्नास्त्येव य‚था य‚दि ।&नास्ति स ख्याप्य‚ते चाय‚मुक्तौ नेति ग‚तिस्त‚दा ॥ १६ ॥\&[\smallbreak]


	
	    \end{quote}
	  
	  \endgroup
	\textsuperscript{\textenglish{293/s}}

	  \pstart \leavevmode% starting standard par
	\hphantom{.}‚{\color{DodgerBlue3}‚न च नास्तीति व‚च‚नादेव} त‚त्साध‚नं विप‚क्षे ‚{\color{DodgerBlue3}‚नास्त्येवेति} युक्तं । त‚द‚पि ह्य‚नुप‚{\tiny $_{lb}$}‚ल‚म्भ‚मेव ख्याप‚य‚ति\edtext{}{\edlabel{pvv.293-1}\label{pvv.293-1}\lemma{ति}\Bfootnote{विशिष्ट‚कार‚णानुमानं ।}}(।) स च स‚र्व्व‚स्माद्विप‚क्षाद्धेत्व‚भाव‚प्र‚ती‚{\tiny $_{7}$}‚ताव‚श‚क्तः ।\leavevmode\ledsidenote{\textenglish{58a/MA}} ‚{\tiny $_{lb}$}‚(क‚थ‚न्त‚र्हीत्याह) य‚था येन प्र‚कारेण साध्य‚साध‚न‚योः प्र‚तिब‚न्धो स‚ति साध्याभावेन ‚{\tiny $_{lb}$}‚साध‚नं विप‚क्षे नास्ति स न्यायो य‚दि ख्याप्य‚ते व्य‚भिचारिविप‚क्षेण वैध‚र्म्योक्त्या ‚{\tiny $_{lb}$}‚त‚दा नास्तीति ग‚म्य‚ते नान्य‚था ॥ (१६)
	\pend% ending standard par
      \label{div_pvv.3.17}
	  
	% new div opening: depth here is 2
	

	  \pstart \leavevmode% starting standard par
	किञ्च (।)
	\pend% ending standard par
      
	  \bigskip
	  \begingroup
	
	    \large
	  
	    \begin{quote}
	  
	    
	    \stanza[\smallbreak]
	\label{pv.3.17}\flagstanza{\tiny\textenglish{...v.3.17}}य‚द्य‚दृष्टौ निवृत्तिः स्याच्छेष‚व‚द् व्य‚भिचारि किम् ।&व्य‚तिरेक्य‚पि हेतुः स्यान्न वाच्याऽसिद्धियोज‚ना ॥ १७ ॥\&[\smallbreak]


	
	    \end{quote}
	  
	  \endgroup
	

	  \pstart \leavevmode% starting standard par
	\hphantom{.}‚{\color{DodgerBlue3}‚य‚द्य‚दृष्टौ} विप‚क्षाद्धेतो‚{\color{DodgerBlue3}‚र्निवृत्तिः} स्यात् ‚{\color{DodgerBlue3}‚शेष‚व‚त्} स‚न्दिग्ध‚विप‚क्ष‚व्य‚तिरेकं\edtext{}{\edlabel{pvv.293-2}\label{pvv.293-2}\lemma{तिरेकं}\Bfootnote{उप‚भुक्तान्य‚फ‚लानि प‚क्वानि म‚धुरादिर‚सानि वा । र‚क्तादिरूपा‚{\tiny $_{lb}$}‚विशेषादुप‚भुक्त‚फ‚ल‚व‚त् । एक‚शाखाप्र‚भ‚व‚त्वाद्वा ।}} ‚{\color{DodgerBlue3}‚व्य‚भि‚{\tiny $_{lb}$}‚चारि} किमिष्ट‚मा चा र्ये ण । एव‚ञ्च स‚र‚सान्येतानि फ‚लानि रूपाविशेषादिति । ‚{\tiny $_{lb}$}‚स‚र्व्व‚स्य तादृग्रू‚{\tiny $_{1}$}‚प‚स्य प‚क्षीकृत‚त्वात् न व्य‚भिचार‚द‚र्श‚नं विप‚क्षे चाद‚र्श‚न‚म‚स्तीति ‚{\tiny $_{lb}$}‚प्राप्त‚म‚द‚र्श‚न‚मात्र‚द् व्य‚तिरेक‚वादिनो हेतुत्व‚म‚स्य ॥
	\pend% ending standard par
      

	  \pstart \leavevmode% starting standard par
	\hphantom{.}‚{\color{DodgerBlue3}‚व्य‚तिरेक्य‚पि हेतुः स्यात्} । नेदं निरात्म‚कं जीव‚च्छ‚रीर‚म‚प्राणादिम‚त्व‚प्र‚स‚ङ्गादि ‚{\tiny $_{lb}$}‚ति । बौ द्धं प्र‚ति सात्म‚क‚स्य क‚स्य‚चिद‚सिद्धेर‚न्व‚याभावात् । निरात्म‚केभ्य‚श्च प‚टा‚{\tiny $_{lb}$}‚दिभ्यो निवृत्तेः प्राणादि\edtext{}{\edlabel{pvv.293-3}\label{pvv.293-3}\lemma{प्राणादि}\Bfootnote{अपानोन्मेष‚निमेषादि ।}}र्व्य‚तिरेकी स हेतुः स्यात् । अद‚र्श‚नाद् व्य‚तिरेक‚सिद्धेर‚{\tiny $_{lb}$}‚निष्ट\edtext{}{\edlabel{pvv.293-4}\label{pvv.293-4}\lemma{निष्ट}\Bfootnote{घ‚टादाव‚पि नैरात्म्यासिद्धेर्बौद्धाभ्युप‚ग‚माच्चेत् आत्मा न स्यात् ।}}श्चा‚{\tiny $_{2}$}‚चार्येण ॥
	\pend% ending standard par
      

	  \pstart \leavevmode% starting standard par
	किञ्च (।) वादिप्र‚तिवादिनोर‚सिद्ध‚स्यान्य‚त‚रासिद्ध‚स्य स‚न्दिग्ध‚स्याश्र‚या\edtext{}{\edlabel{pvv.293-5}\label{pvv.293-5}\lemma{या}\Bfootnote{आचार्येण ह्युक्तं प‚क्ष‚ध‚र्मों वादिप्र‚तिवादिनिश्चितो गृह्य‚ते । तेनैषां ‚{\tiny $_{lb}$}‚निरासः ।}} ‚{\tiny $_{lb}$}‚सिद्ध‚स्य स‚न्दिग्धाश्र‚यासिद्ध‚स्य असिद्ध‚स्य च निरासं प‚क्ष‚ध‚र्म‚त्व‚निश्च‚येन प्र‚तिपाद्यां‚{\tiny $_{lb}$}‚न्व‚य‚व्य‚तिरेक‚निश्च‚य‚व‚च‚नेन स‚प‚क्ष‚विप‚क्ष‚योर्हेतोर‚न्व‚य‚स्य व्य‚तिरेक‚स्य च विप‚रीताया ‚{\tiny $_{lb}$}‚असिद्धेश्च‚तुर्व्विधाया योज‚ना आचार्येण स‚प‚क्षे स‚न्न‚स‚न्नित्येव‚मादिष्व‚पि य‚थायो\edtext{}{\edlabel{pvv.293-6}\label{pvv.293-6}\lemma{थायो}\Bfootnote{अन्य‚त‚रासिद्धादीनां स‚प‚क्षादिष्व‚स‚म्भ‚वात् य‚थायोगं ।}}ग‚{\tiny $_{lb}$}‚मुदाहार्य्य‚मित्यादि‚{\tiny $_{3}$}‚ना निर्दिष्टा (।) सापि न वाच्या । अद‚र्श‚न‚स्य व्य‚तिरेक‚निश्च‚य‚{\tiny $_{lb}$}‚हेतुत्वे स‚न्दिग्ध‚व्य‚तिरेक‚स्य हेतुत्वात् ॥ (१७)
	\pend% ending standard par
      \label{div_pvv.3.18}
	  
	% new div opening: depth here is 2
	

	  \pstart \leavevmode% starting standard par
	अपि च (।)
	\pend% ending standard par
      \textsuperscript{\textenglish{294/s}}
	  \bigskip
	  \begingroup
	
	    \large
	  
	    \begin{quote}
	  
	    
	    \stanza[\smallbreak]
	\label{pv.3.18}\flagstanza{\tiny\textenglish{...v.3.18}}विशेष‚स्य व्य‚व‚च्छेद‚हेतुता स्याद‚द‚र्श‚नात् ।&प्र‚माणान्त‚र‚वाधा चेन्नेदानीन्नास्ति-तादृशः ॥ १८ ॥\&[\smallbreak]


	
	    \end{quote}
	  
	  \endgroup
	

	  \pstart \leavevmode% starting standard par
	विशेष‚स्यासाधार‚ण‚स्य\edtext{}{\edlabel{pvv.294-1}\label{pvv.294-1}\lemma{स्य}\Bfootnote{अद‚र्श‚न‚मात्राद् व्यावृत्तिरिष्टाऽस्ति च नित्यानित्य‚योर‚द‚र्श‚नं ।}} श्राव‚ण‚त्वादेर्नित्येऽनित्ये चाद‚र्श‚नादुभ‚य‚तो व्यावृत्तेः ‚{\tiny $_{lb}$}‚श‚ब्दे ध‚र्मिणि द्व‚य‚व्य‚व‚च्छेद‚न\edtext{}{\edlabel{pvv.294-2}\label{pvv.294-2}\lemma{न}\Bfootnote{प्र‚तिषेध‚हेतुत्वं ।}} हेतुता स्यात् । ‚{\color{DodgerBlue3}‚प्र‚माणान्त‚र‚बाधा चेत्}\edtext{\textsuperscript{*}}{\edlabel{pvv.294-3}\label{pvv.294-3}\lemma{*}\Bfootnote{श्राव‚ण‚त्व‚स्य कृत‚क‚त्व‚व‚द् व‚स्तुध‚र्म‚त्वान्नित्याद‚स्त्येव व्य‚तिरेको बाध‚कात् । ‚{\tiny $_{lb}$}‚प‚रे तु नित्य‚म‚पि व‚स्त्विच्छ‚न्ति त‚द‚पेक्ष‚योच्य‚ते नित्यानित्य‚व्यावृत्तिः ।}} अन्योन्य‚{\tiny $_{lb}$}‚व्य‚व‚च्छेद‚रूप‚योरेक‚प्र‚तिषेध‚स्याप‚र‚विधिनान्त‚रीय‚क‚त्वात् उभ‚य‚व्य‚व‚च्छेदो\edtext{}{\edlabel{pvv.294-4}\label{pvv.294-4}\lemma{च्छेदो}\Bfootnote{व‚स्तु भ‚व‚त्त‚त्व‚म‚न्य‚त्व‚म्वा नातिक्राम‚ति (।)}}नुमान‚{\tiny $_{lb}$}‚बाधित‚{\tiny $_{4}$}‚ः । ‚{\color{DodgerBlue3}‚नेदानीं}\edtext{\textsuperscript{*}}{\edlabel{pvv.294-5}\label{pvv.294-5}\lemma{*}\Bfootnote{य‚द् विरुद्धं न त‚स्यैक‚त्र युग‚प‚त्स‚म्भ‚वो य‚था शीतोष्ण‚योर्विरुद्ध‚श्च नित्या‚{\tiny $_{lb}$}‚नित्य‚यो र्युग‚प‚देक‚त्र ध‚र्मिणि व्य‚व‚च्छेदो य‚थोक्त‚विधानेनेति व्याप‚क‚विरुद्धं ।}}(बाध‚संभ‚वे) नास्ति तादृशः । एवं त‚र्ह्य‚दृष्टेर‚भाव‚निश्च‚यो ‚{\tiny $_{lb}$}‚नास्तीति व‚क्त‚व्यं । य‚था श्राव‚ण‚त्वेनोभ‚य‚व्यावृत्त‚त‚या निश्चितेन प्र‚साध्य‚मान‚स्यो‚{\tiny $_{lb}$}‚भ‚य‚व्य‚व‚च्छेद‚स्य प्र‚मा (णा) न्त‚रेण बाधा । (१८)
	\pend% ending standard par
      \label{div_pvv.3.19}
	  
	% new div opening: depth here is 2
	

	  \begin{center}%% label @type='head'
	\textbf{(ख) आचार्योय‚म‚त‚निरासः}
	\end{center}
	
	  \bigskip
	  \begingroup
	
	    \large
	  
	    \begin{quote}
	  
	    
	    \stanza[\smallbreak]
	\label{pv.3.19}\flagstanza{\tiny\textenglish{...v.3.19}}त‚थान्य‚त्रापि संभाव्यं प्र‚माणान्त‚र‚बाध‚न‚म् ।&दृष्टाऽयुक्तिर‚दृष्टेश्च स्यात् स्प‚र्श‚स्याविरोध‚नी ॥ १९ ॥\&[\smallbreak]


	
	    \end{quote}
	  
	  \endgroup
	

	  \pstart \leavevmode% starting standard par
	त‚थान्य‚त्रापि विप‚क्षाद्धेतो\edtext{}{\edlabel{pvv.294-6}\label{pvv.294-6}\lemma{क्षाद्धेतो}\Bfootnote{ल‚क्ष‚ण‚युक्ते बाधास‚म्भ‚वे त‚ल्ल‚क्ष‚ण‚मेव दूषितं स्यादिति स‚र्व्व‚त्रानाश्वासः । ‚{\tiny $_{lb}$}‚नैवं मान‚द्व‚ये ।}}र्व्य‚तिरिकेपि ‚{\color{DodgerBlue3}‚संभाव्यं प्र‚माणान्त‚र‚बाध‚नं} । अद‚र्श‚न‚{\tiny $_{lb}$}‚मात्र‚स्य निमित्त‚स्य स‚मान‚त्वात् । त‚था दृष्टाऽयुक्तिर‚दृष्टेश्च\edtext{}{\edlabel{pvv.294-7}\label{pvv.294-7}\lemma{दृष्टेश्च}\Bfootnote{अध्य‚क्षं किञ्चिद् दृष्ट्वा य‚त् किञ्चिद् पृथिव्यादि त‚त् स‚र्व्व‚म‚नुष्णा‚{\tiny $_{lb}$}‚शीत‚र‚हितं त‚न्न तूलोप‚ल‚प‚र्ण्ण‚वादिव‚द् भेद‚स‚म्भ‚वात् । अद‚र्श‚न‚मात्रे प्र‚त्य‚क्ष‚वाधा ।}} स्यात् स्प‚र्श‚स्याविरो‚{\tiny $_{lb}$}‚धिनी । दृष्टेषु पृथि‚{\tiny $_{5}$}‚व्यादिष्व‚पाक‚ज‚स्यानुष्णाशीत‚स्प‚र्श‚स्यायुक्ति\edtext{}{\edlabel{pvv.294-8}\label{pvv.294-8}\lemma{स्यायुक्ति}\Bfootnote{आस‚ङ्ग‚तिः ।}}र्योगाभावः (।) ‚{\tiny $_{lb}$}‚अदृष्टेर‚द‚र्श‚नात् वै शे षि क स्येष्टाऽनुप‚ल‚म्भाद् व्याप्त्या निवृत्त्या निश्च‚यादा चार्येण ‚{\tiny $_{lb}$}‚प्र‚तिक्षिप्ता च । \edtext{\textsuperscript{*}}{\edlabel{pvv.294-9}\label{pvv.294-9}\lemma{*}\Bfootnote{य‚द्य‚दृष्ट्या निवृत्तिस्त‚दाऽदृष्टेर‚द‚र्श‚नाद‚नुष्णाशीत‚स्य स्प‚र्शंस्य दृष्टो युक्तिः ‚{\tiny $_{lb}$}‚दृष्टेषु पृथिव्यादिष्व‚स‚ङ्ग‚तिर्या वैशेषिकोक्ता । य‚स्याचार्येणायुक्त‚त्व‚मुक्तं स्याद‚{\tiny $_{lb}$}‚विरोधिनी, युक्तैव ।}} य‚द्य‚द‚र्श‚न‚मात्र‚णे दृष्टेभ्यः\edtext{}{\edlabel{pvv.294-10}\label{pvv.294-10}\lemma{दृष्टेभ्यः}\Bfootnote{दृष्ट‚मेव स्वीकृत्यादृष्टेपि ।}}प्र‚तिषेधः क्रिय‚ते न च सोपि युक्त ‚{\tiny $_{lb}$}‚\leavevmode\ledsidenote{\textenglish{295/s}} इत्यादिनाऽ‚{\color{DodgerBlue3}‚विरोधीनो} विरोध‚र‚हिता स्यात् । एव‚मा चा र्यी यः\edtext{}{\edlabel{pvv.295-1}\label{pvv.295-1}\lemma{यः}\Bfootnote{शिष्योऽज्ञोऽद‚र्श‚नेनाभाव‚वादी ।}}क‚श्च‚न प्र‚ति‚{\tiny $_{lb}$}‚ब‚न्ध‚म‚न‚भ्युप‚ग‚च्छ‚न् विरोधादुपाल‚ब्धः ॥ (१९)
	\pend% ending standard par
      \label{div_pvv.3.20}
	  
	% new div opening: depth here is 2
	

	  \pstart \leavevmode% starting standard par
	संप्र‚ति प‚रान् प्र‚त्याह‚{\tiny $_{6}$}‚ (।)
	\pend% ending standard par
      
	  \bigskip
	  \begingroup
	
	    \large
	  
	    \begin{quote}
	  
	    
	    \stanza[\smallbreak]
	\label{pv.3.20}\flagstanza{\tiny\textenglish{...v.3.20}}देशादिभेदाद् दृश्य‚न्ते भिन्ना द्र‚व्येषु श‚क्त‚यः ।&त‚त्रैक‚दृष्ट्या नान्य‚त्र युक्त‚स्त‚द्भाव‚निश्च‚यः ॥ २० ॥\&[\smallbreak]


	
	    \end{quote}
	  
	  \endgroup
	

	  \pstart \leavevmode% starting standard par
	\hphantom{.}‚{\color{DodgerBlue3}‚देशादिभेदात्} देश‚काल‚स‚ह‚कारिभेदात् ‚{\color{DodgerBlue3}‚दृश्य‚न्ते भिन्ना} नानाप्र‚कारा ‚{\color{DodgerBlue3}‚द्र‚व्येषु ‚{\tiny $_{lb}$}‚श‚क्त‚यः}\edtext{}{\edlabel{pvv.295-2}\label{pvv.295-2}\lemma{कारा}\Bfootnote{अनेक‚श‚क्तिद्र‚व्येषु त‚त्र ।}} (।) ‚{\color{DodgerBlue3}‚त‚त्रैक}‚स्मिन् देशादावेक‚स्य द्र‚व्य‚स्य श‚क्तिविशेष‚व‚तो ‚{\color{DodgerBlue3}‚दृष्ट्याऽन्य‚त्र} देशादौ\edtext{}{\edlabel{pvv.295-3}\label{pvv.295-3}\lemma{देशादौ}\Bfootnote{हेतोर्व्विरोधाद‚नुयुक्ते ।}} ‚{\color{DodgerBlue3}‚त‚स्य} श‚क्तिविशेष‚व‚तो द्र‚व्य‚स्य ‚{\color{DodgerBlue3}‚भाव‚निश्च‚यो न युक्तः} प्र‚तिब‚न्ध‚म‚न्त‚रेण । ‚{\tiny $_{lb}$}‚न ह्येक‚त्र य‚थौष‚ध‚यो दृष्टास्त‚थैवान्य‚त्रापि ताः श‚क्य‚न्ते व्य‚व‚स्थाप‚यितुं क्षेत्रादि‚{\tiny $_{lb}$}‚विशेषाद् विशिष्ट‚त‚र‚र‚स\edtext{}{\edlabel{pvv.295-4}\label{pvv.295-4}\lemma{स}\Bfootnote{वीर्य‚दोषाप‚न‚य‚श‚क्तिः प‚रिणामो विपाकः संस्कारः क्षीर‚सेकादि ।}}वीर्यादि‚{\tiny $_{7}$}‚द‚र्श‚नात् । (२०)\leavevmode\ledsidenote{\textenglish{58b/MA}}
	\pend% ending standard par
      \label{div_pvv.3.21}
	  
	% new div opening: depth here is 2
	

	  \begin{center}%% label @type='head'
	\textbf{(क) वैशेषिक‚म‚त‚निरासः}
	\end{center}
	

	  \pstart \leavevmode% starting standard par
	किञ्च ।
	\pend% ending standard par
      
	  \bigskip
	  \begingroup
	
	    \large
	  
	    \begin{quote}
	  
	    
	    \stanza[\smallbreak]
	\label{pv.3.21}\flagstanza{\tiny\textenglish{...v.3.21}}आत्म‚मृच्चेत‚नादीनां योऽभाव‚स्याप्र‚साध‚कः ।&स एवानुप‚ल‚म्भः कि हेत्व‚भाव‚स्य साध‚कः ॥ २१ ॥\&[\smallbreak]


	
	    \end{quote}
	  
	  \endgroup
	

	  \pstart \leavevmode% starting standard par
	\hphantom{.}‚{\color{DodgerBlue3}‚आत्म‚मृच्चेत‚नादीनां योऽभाव‚स्याप्र‚साध‚कः} । आत्म‚नोनुप‚ल‚भ्य‚मान‚स्य च ‚{\tiny $_{lb}$}‚योऽनुप‚ल‚म्भोऽभाव‚स्याप्र‚साध‚को वै शे षि क स्ये ष्टः (।) त‚था मुद‚श्चेत‚नायाः ‚{\tiny $_{lb}$}‚स्व‚भाव‚भूताया योऽनुप‚ल‚म्भोऽभावाप्र‚साध‚क‚श्चा र्व्वा क स्येष्टः ।\edtext{\textsuperscript{*}}{\edlabel{pvv.295-5}\label{pvv.295-5}\lemma{*}\Bfootnote{लोकाय‚तिक‚स्य ।}} \edtext{\textsuperscript{*}}{\edlabel{pvv.295-6}\label{pvv.295-6}\lemma{*}\Bfootnote{आदिना ।}}त‚था क्षीरादौ ‚{\tiny $_{lb}$}‚द‚ध्याद्य‚भाव‚स्य योनुप‚ल‚म्भोऽप्र‚साध‚क इष्टः सां ख्य स्य । स एवानुप‚ल‚म्भः किं ‚{\tiny $_{lb}$}‚क‚स्माद् विप‚क्षे हेत्व‚भाव‚स्येष्टो वै शे षि‚{\tiny $_{1}$}‚का दिभिः (।) य‚था घ‚टादौ प्राणादि‚{\tiny $_{lb}$}‚म‚त्वाभावो वैशेषिक‚स्य । व‚क्तृत्वाभावो विप‚क्षे\edtext{}{\edlabel{pvv.295-7}\label{pvv.295-7}\lemma{क्षे}\Bfootnote{अस‚र्व्व‚ज्ञाविराग‚योः ।}} चा र्व्वा क स्य । संघात\edtext{}{\edlabel{pvv.295-8}\label{pvv.295-8}\lemma{संघात}\Bfootnote{अप‚रार्थेषु च श‚श‚विषाणादिषु (प‚रार्थाश्च‚क्षुराद‚य इत्य‚भिधाय) ‚{\tiny $_{lb}$}‚संघात‚त्व‚स्याद‚र्श‚नाद् व्य‚तिरेकः को हि संह‚त‚स्य प‚रार्थ‚त्वे निय‚मः ।}} त्वाभावो ‚{\tiny $_{lb}$}‚वा विप‚क्षे सां ख्य स्यानुप‚ल‚म्भादिष्टः । ‚{\tiny $_{lb}$}‚\leavevmode\ledsidenote{\textenglish{296/s}} उक्तः प‚रेषाञ्च न्याय‚व्याघातः\edtext{}{\edlabel{pvv.296-1}\label{pvv.296-1}\lemma{व्याघातः}\Bfootnote{भिन्ना द्र‚व्येष्विति ।}} प‚र‚स्प‚र‚विरोध‚श्च\edtext{}{\edlabel{pvv.296-2}\label{pvv.296-2}\lemma{श्च}\Bfootnote{आत्म‚न्य‚स्वीकृतिः ।}}(।) (२१)
	\pend% ending standard par
      \label{div_pvv.3.22}
	  
	% new div opening: depth here is 2
	

	  \pstart \leavevmode% starting standard par
	य‚स्माद‚द‚र्श‚न‚मात्र‚त् न व्य‚तिरेक‚सिद्धिः ।
	\pend% ending standard par
      
	  \bigskip
	  \begingroup
	
	    \large
	  
	    \begin{quote}
	  
	    
	    \stanza[\smallbreak]
	\label{pv.3.22}\flagstanza{\tiny\textenglish{...v.3.22}}त‚स्मात् त‚न्मात्र‚स‚म्ब‚न्धः स्व‚भावो भाव‚मेव वा ।&निव‚र्त‚येत् कार‚णं वा कार्य‚म‚व्य‚भिचार‚तः ॥ २२ ॥\&[\smallbreak]


	
	    \end{quote}
	  
	  \endgroup
	

	  \pstart \leavevmode% starting standard par
	\hphantom{.}‚{\color{DodgerBlue3}‚त‚स्मात् त‚न्मात्र‚स‚म्ब‚न्ध}‚स्त‚त्साध‚नं केव‚लं त‚न्मात्रं तेन स‚म्ब‚द्धः । हेतु\edtext{}{\edlabel{pvv.296-3}\label{pvv.296-3}\lemma{हेतु}\Bfootnote{ज्ञाप‚को लिङ्गं ।}}स‚त्ता‚{\tiny $_{lb}$}‚मात्रेण कार‚णान्त‚रापेक्षार‚{\tiny $_{2}$}‚हितेन स‚म्ब‚न्धो य‚स्येत्य‚र्थः\edtext{}{\edlabel{pvv.296-4}\label{pvv.296-4}\lemma{र्थः}\Bfootnote{साध्य‚स्य ।}} । ‚{\color{DodgerBlue3}‚स्व‚भावो} व्याप‚को ‚{\tiny $_{lb}$}‚ध‚र्मो निव‚र्त्त‚मानो भावं व्याप्य‚मेव वा निव‚र्त्त‚येत् । य‚था वृक्ष‚त्वं शिंश‚पां वृक्ष‚{\tiny $_{lb}$}‚विशेष‚त्वाच्छिंश‚पायाः । ‚{\color{DodgerBlue3}‚कार‚णं वा} निव‚र्त्त‚मानं ‚{\color{DodgerBlue3}‚कार्यं निव‚र्त्त‚य‚ति अव्य‚भिचार‚तः} (।) ‚{\tiny $_{lb}$}‚न हि व्याप्यं कार्यं च व्याप‚केन कार‚णेन विना भ‚व‚तः त‚त्स्व‚भाव‚त्वात् त‚द‚धीन‚त्वा‚{\tiny $_{lb}$}‚च्चेति । (२२)
	\pend% ending standard par
      \label{div_pvv.3.23}
	  
	% new div opening: depth here is 2
	

	  \pstart \leavevmode% starting standard par
	तादात्म्य‚त‚दुत्प‚त्ती अव‚श्याश्र‚य‚णीये ॥
	\pend% ending standard par
      
	  \bigskip
	  \begingroup
	
	    \large
	  
	    \begin{quote}
	  
	    
	    \stanza[\smallbreak]
	\label{pv.3.23}\flagstanza{\tiny\textenglish{...v.3.23}}अन्य‚थैक‚निवृत्यान्य‚विनिवृत्तिः क‚थं भ‚वेत् ।&नाश्व‚वानिति संदेहे किं न वा प‚शुसंश‚यः ॥ २३ ॥\&[\smallbreak]


	
	    \end{quote}
	  
	  \endgroup
	

	  \pstart \leavevmode% starting standard par
	\hphantom{.}‚{\color{DodgerBlue3}‚अन्य‚{\tiny $_{3}$}‚थैक‚स्य} साध्य‚स्य ‚{\color{DodgerBlue3}‚निवृत्याऽन्य‚स्य} साध‚न‚स्य ‚{\color{DodgerBlue3}‚विनिवृत्तिः क‚थ‚म्भ‚वेत्} स्व‚त‚न्त्र‚त्वात् । न हि म‚र्त्योऽश्व‚र‚हित इति गोर‚हितोपि भ‚व‚ति । (२३)
	\pend% ending standard par
      \label{div_pvv.3.24}
	  
	% new div opening: depth here is 2
	

	  \pstart \leavevmode% starting standard par
	त‚था (।)
	\pend% ending standard par
      
	  \bigskip
	  \begingroup
	
	    \large
	  
	    \begin{quote}
	  
	    
	    \stanza[\smallbreak]
	\label{pv.3.24}\flagstanza{\tiny\textenglish{...v.3.24}}स‚न्निधानात् त‚थैक‚स्य क‚थ‚म‚न्य‚स्य स‚न्निधिः ।&गोमानित्येव म‚र्त्येन भाव्य‚म‚श्व‚व‚तापि किम् ॥ २४ ॥\&[\smallbreak]


	
	    \end{quote}
	  
	  \endgroup
	

	  \pstart \leavevmode% starting standard par
	\hphantom{.}त‚था ‚{\color{DodgerBlue3}‚स‚न्निधानादे}‚क‚स्य हेतोः\edtext{}{\edlabel{pvv.296-5}\label{pvv.296-5}\lemma{हेतोः}\Bfootnote{स्व‚भावेनास‚म्ब‚द्ध‚स्य ।}} ‚{\color{DodgerBlue3}‚क‚थ‚म‚न्य‚स्य} साध्य‚स्य ‚{\color{DodgerBlue3}‚स‚न्निधिः । गोमा\edtext{}{\edlabel{pvv.296-6}\label{pvv.296-6}\lemma{गोमा}\Bfootnote{य‚स्मात् ।}}नित्येव ‚{\tiny $_{lb}$}‚म‚र्त्येन भाव्य‚म‚श्व‚व‚तापि किं} । ग‚वाश्व‚स्य प‚र‚स्प‚र‚म‚प्र‚तिब‚द्ध‚त्वादेक‚भावे नान्य‚भावः । ‚{\tiny $_{lb}$}‚त‚था शिं‚{\tiny $_{4}$}‚श‚पाऽवृक्षादाव‚पि स्यात् (।) य‚स्मात् कार‚ण‚व्याप‚क‚निवृत्त्या कार्य‚{\tiny $_{lb}$}‚व्याप‚क‚निवृत्तिर्द‚र्श‚नीया वि\edtext{}{\edlabel{pvv.296-7}\label{pvv.296-7}\lemma{वि}\Bfootnote{य‚स्माद् व्याप्तिग्राह‚कं प्र‚माणं नान्य‚त् स्व‚भाव‚प्र‚तिब‚न्ध‚ग्राह‚कात् । तेनैव ‚{\tiny $_{lb}$}‚साध‚न‚स्य साध्याय‚त्त‚त्व‚ग्र‚हात् । साध्याभावेऽभावो गृहीत एव । केव‚लं त‚द‚{\tiny $_{lb}$}‚विनाभाव‚ग्राह‚कं प्र‚माणं विस्मृत‚त्वाद् दृष्टान्ताभ्याँ साध‚र्म्य‚वैध‚र्म्य‚योः प्र‚द‚र्श्य‚ते ।}}प‚क्षे (। २४)
	\pend% ending standard par
      \textsuperscript{\textenglish{297/s}}\label{div_pvv.3.25}
	  
	% new div opening: depth here is 2
	
	  \bigskip
	  \begingroup
	
	    \large
	  
	    \begin{quote}
	  
	    
	    \stanza[\smallbreak]
	\label{pv.3.25}\flagstanza{\tiny\textenglish{...v.3.25}}त‚स्माद् वैध‚र्म्य‚दृष्टान्ते नेष्टोव‚श्य‚मिहाश्र‚यः&त‚द‚भावे च त‚न्नेति व‚च‚नाद‚पि त‚द्ग‚तिः ॥ २५ ॥\&[\smallbreak]


	
	    \end{quote}
	  
	  \endgroup
	

	  \pstart \leavevmode% starting standard par
	\hphantom{.}‚{\color{DodgerBlue3}‚त‚स्माद् वैध‚र्म्य‚दृष्टान्ते नेष्टो\edtext{}{\edlabel{pvv.297-1}\label{pvv.297-1}\lemma{नेष्टो}\Bfootnote{निय‚मेन कार्य‚स्व‚भावे हेतौ ।}}ऽव‚श्य‚मिहाश्र‚यः} व‚स्तुभूतो ध‚र्मी । त‚स्य ‚{\tiny $_{lb}$}‚कार‚ण‚व्याप‚क‚स्य साध्य‚स्याभावे च त‚त्कार्य‚वाय‚प्यं ‚{\color{DodgerBlue3}‚साध‚नं नेति व‚च‚नाद‚पि} त‚स्य ‚{\tiny $_{lb}$}‚साध्याभावे साध‚नाभाव‚स्य ग‚तिः । व‚स्तुनि व‚स्तुनिवृत्तिः स‚न्दिग्‏धा व‚स्तुस‚म्ब‚न्धा‚{\tiny $_{lb}$}‚विरोधात् । अव‚स्तु‚{\tiny $_{5}$}‚नि तु व‚स्तुस‚त्ता विरुध्य‚ते निवृत्तिस्तु युक्ता । अतः स ‚{\tiny $_{lb}$}‚ध‚र्मिण‚म‚न्त‚रेणापि वोङ्मात्र‚तोपि ग‚म्य‚त एव (। २५)
	\pend% ending standard par
      \label{div_pvv.3.26}
	  
	% new div opening: depth here is 2
	

	  \pstart \leavevmode% starting standard par
	य‚तः ।
	\pend% ending standard par
      
	  \bigskip
	  \begingroup
	
	    \large
	  
	    \begin{quote}
	  
	    
	    \stanza[\smallbreak]
	\label{pv.3.26}\flagstanza{\tiny\textenglish{...v.3.26}}त‚द्‏भाव‚हेतुभावौ हि दृष्टान्ते त‚द‚वेदिनः ।&ख्याप्येते विदुषां वाच्यो हेतुरेव हि केव‚लः ॥ २६ ॥\&[\smallbreak]


	
	    \end{quote}
	  
	  \endgroup
	

	  \pstart \leavevmode% starting standard par
	\hphantom{.}त‚द्‏भाव‚हेतुभावौ ‚{\color{DodgerBlue3}‚दृष्टान्ते ख्याप्येते} (।) त‚स्य साध‚न‚स्य भाव‚स्तादात्म्यं ‚{\tiny $_{lb}$}‚साध्य‚स्य हेतुभा\edtext{}{\edlabel{pvv.297-2}\label{pvv.297-2}\lemma{हेतुभा}\Bfootnote{स्व‚भाव‚हेतौ साध्य‚स्य त‚द्‏भावः साध‚न‚व्याप‚क‚त्वं । कार्य‚हेतौ साध्य‚स्य ‚{\tiny $_{lb}$}‚हेतुभावः कार‚ण‚त्वं ।}}व‚श्च । ख्याप्य‚ते वैध‚र्म्य‚दृष्टान्ते साध्य‚निवृत्त्या ‚{\color{DodgerBlue3}‚साध‚न‚निवृत्ति}‚{\tiny $_{lb}$}‚क‚थ‚नेन त‚योस्तादात्म्य‚त‚दुत्प‚त्तिब‚न्ध‚म‚जान‚तो याह्य‚न्य‚निवृत्त्येत‚र‚निवृत्तिः । सा ‚{\tiny $_{lb}$}‚व‚स्तुभूत‚मा‚{\tiny $_{6}$}‚श्र‚य‚म‚न्त‚रेणापि श‚क्य‚ते निर्देष्टुं । ये तु प्र‚तिब‚न्धं विद‚न्ति\edtext{}{\edlabel{pvv.297-3}\label{pvv.297-3}\lemma{न्ति}\Bfootnote{अविस्मृतेः ।}} तेषां ‚{\color{DodgerBlue3}‚विदुषां ‚{\tiny $_{lb}$}‚हेतुरेव केव‚लो वाच्यः}\edtext{}{\edlabel{pvv.297-4}\label{pvv.297-4}\lemma{तेषां}\Bfootnote{प‚क्ष‚ध‚र्म‚मात्र‚निश्च‚यार्थं ।}} । न दृष्टान्तः त‚त्र द‚र्श‚नीय‚स्य\edtext{}{\edlabel{pvv.297-5}\label{pvv.297-5}\lemma{स्य}\Bfootnote{अन्व‚य‚व्य‚तिरेक‚निश्चाय‚क‚स्य ।}}प्र‚तिब‚न्ध‚स्य सिद्ध‚त्वात् । ‚{\tiny $_{lb}$}‚य‚स्माद् दृष्टान्ते प्र‚तिब‚न्धः क‚थ्य‚ते । (२६)
	\pend% ending standard par
      \label{div_pvv.3.27}
	  
	% new div opening: depth here is 2
	
	  \bigskip
	  \begingroup
	
	    \large
	  
	    \begin{quote}
	  
	    
	    \stanza[\smallbreak]
	\label{pv.3.27}\flagstanza{\tiny\textenglish{...v.3.27}}तेनैव ज्ञात‚स‚म्ब‚न्धे द्व‚योर‚न्य‚त‚रोक्तितः ।&अर्थाप‚त्या द्वितीयेपि स्मृतिः स‚मुप‚जाय‚ते ॥ २७ ॥\&[\smallbreak]


	
	    \end{quote}
	  
	  \endgroup
	

	  \pstart \leavevmode% starting standard par
	\hphantom{.}‚{\color{DodgerBlue3}‚तेनैव} कार‚णेन ‚{\color{DodgerBlue3}‚ज्ञात‚स‚म्ब‚न्धे} हेतौ साध‚र्म्य‚वैध‚र्म्य‚दृष्टान्त‚यो‚{\color{DodgerBlue3}‚र‚न्य‚त‚र‚स्योक्तितोऽर्था‚{\tiny $_{lb}$}‚प‚त्त्या} साम‚र्थ्येन ‚{\color{DodgerBlue3}‚द्वितीयेपि स्मृतिः स‚मुप‚जा}‚य‚ते इति न त‚स्य निर्देशः क‚र्त्त‚व्यः‚{\tiny $_{7}$}‚ (।) ‚{\tiny $_{lb}$}‚त‚था हि य‚त् कृत‚कं त‚द‚नित्यं\edtext{}{\edlabel{pvv.297-6}\label{pvv.297-6}\lemma{नित्यं}\Bfootnote{य‚दैक‚कार्य‚क‚र‚णं प्र‚ति साम‚र्थ्यं त‚त् त‚दैव न पूर्वं न प‚श्चात् त‚त्कार्याभावात् ‚{\tiny $_{lb}$}‚(।) साम‚र्थ्य‚ञ्च त‚द‚व्य‚तिरिक्त‚मेव‚मुत्त‚रोत्त‚र‚कार्योत्प‚त्ताव‚पि साम‚र्थ्य‚भेदेन प‚दार्थ‚{\tiny $_{lb}$}‚भेदात् क्ष‚णिक एव क्र‚माक्र‚म‚निय‚मः । अन्य‚था एक‚दैक‚कार्य‚क‚र‚णेऽनेक‚कृतौ वान्य‚{\tiny $_{lb}$}‚दाऽव‚स्तुत्वं कार्याभावात् पुनः कृतोक्र‚म एव । अथ प्र‚कारान्त‚रेण नैक‚दा नापि पुनः ‚{\tiny $_{lb}$}‚पुनः क‚रोति त‚दास्याव‚स्तुत्वं नित्य‚स्याक‚र्तृत्वात् ।}} य‚था घ‚ट इति (।) तादात्म्ये निर्ज्ञाते नित्य‚त्वाभावे\leavevmode\ledsidenote{\textenglish{59a/MA}} ‚{\tiny $_{lb}$}‚\leavevmode\ledsidenote{\textenglish{298/s}} कृत‚क‚त्वाभावः साम‚र्थ्यादाकाशादौ ग‚म्य‚ते । त‚था वैध‚र्म्य‚दृष्टान्तेन तादात्म्ये ‚{\tiny $_{lb}$}‚क‚थिते साम‚र्थ्याद‚न्व‚यो घ‚टादौ ग‚म्य‚ते ॥
	\pend% ending standard par
      

	  \pstart \leavevmode% starting standard par
	एवं साध‚न‚भावे साध्य‚स्याव‚श्यं भावो य‚दि साध्याभावे हेत्व‚भावः । त‚था ‚{\tiny $_{lb}$}‚साध्याभावे हेत्व‚भाव‚स्त‚दा भ‚व‚ति । (२७)
	\pend% ending standard par
      \label{div_pvv.3.28}
	  
	% new div opening: depth here is 2
	

	  \pstart \leavevmode% starting standard par
	य‚दि साध‚न‚भावेऽव‚श्यं साध्य‚भावः । एवं कार्यानु‚{\tiny $_{1}$}‚प‚ल‚म्भ‚योर‚पि योज्यं ॥
	\pend% ending standard par
      
	  \bigskip
	  \begingroup
	
	    \large
	  
	    \begin{quote}
	  
	    
	    \stanza[\smallbreak]
	\label{pv.3.28}\flagstanza{\tiny\textenglish{...v.3.28}}हेतुस्व‚भावाभावोतः प्र‚तिषेधे च क‚स्य‚चित् ।&हेतुर्युक्तोप‚ल‚म्भ‚स्य त‚स्य चानुप‚ल‚म्भ‚न‚म् ॥ २८ ॥\&[\smallbreak]


	
	    \end{quote}
	  
	  \endgroup
	

	  \pstart \leavevmode% starting standard par
	य‚तः कार‚ण‚व्याप‚क‚निवृत्तिभ्यां कार्य‚व्याप्य‚निवृत्तिः अतो हेतोः\edtext{}{\edlabel{pvv.298-1}\label{pvv.298-1}\lemma{हेतोः}\Bfootnote{कार‚ण‚स्य ।}} ‚{\color{DodgerBlue3}‚स्व‚भाव‚स्य} व्याप‚क‚स्या‚{\color{DodgerBlue3}‚भावः । क‚स्य‚चित्} कार्य‚स्य व्याप्य‚स्य च ‚{\color{DodgerBlue3}‚प्र‚तिषेधे}‚ऽभावे\edtext{}{\edlabel{pvv.298-2}\label{pvv.298-2}\lemma{ऽभावे}\Bfootnote{च श‚ब्दात् ।}}ऽभाव‚व्य‚व‚हारे ‚{\tiny $_{lb}$}‚च हेतुः । त‚था ‚{\color{DodgerBlue3}‚त‚स्य} प्र‚तिषेध्य‚स्य\edtext{}{\edlabel{pvv.298-3}\label{pvv.298-3}\lemma{स्य}\Bfootnote{स्व‚भाव‚स्य ।}} ‚{\color{DodgerBlue3}‚युक्तो\edtext{}{\edlabel{pvv.298-4}\label{pvv.298-4}\lemma{युक्तो}\Bfootnote{न्याय्य ।}}प‚ल‚म्भ‚स्य} उप‚ल‚ब्धिल‚क्ष‚ण‚प्राप्त‚स्या‚{\color{DodgerBlue3}‚नुप‚{\tiny $_{lb}$}‚ल‚म्भ‚नं} प्र‚ति\edtext{}{\edlabel{pvv.298-5}\label{pvv.298-5}\lemma{ति}\Bfootnote{कार‚ण‚व्याप‚कानुप‚ल‚ब्धी तु प्र‚तिषेध‚त‚द्‏व्य‚व‚हार‚योः ।}}षेधे\edtext{}{\edlabel{pvv.298-6}\label{pvv.298-6}\lemma{षेधे}\Bfootnote{इत्य‚र्थः प‚रः ।}} प्र‚तिषेध‚व्य‚व‚हारे हेतुः\edtext{}{\edlabel{pvv.298-7}\label{pvv.298-7}\lemma{हेतुः}\Bfootnote{किं कार‚णं प्र‚तिषेध‚स्य न हेतुरित्याह ।}}। त‚स्य स्व‚य‚मेवाभा‚{\tiny $_{2}$}‚व‚रूप‚त्वात् ॥ ‚{\tiny $_{lb}$}‚(२८)
	\pend% ending standard par
      \label{div_pvv.3.29}
	  
	% new div opening: depth here is 2
	
	  \bigskip
	  \begingroup
	
	    \large
	  
	    \begin{quote}
	  
	    
	    \stanza[\smallbreak]
	\label{pv.3.29}\flagstanza{\tiny\textenglish{...v.3.29}}इतीय‚न्त्रिविधोक्ताप्य‚नुप‚ल‚ब्धिर‚नेक‚धा ।&त‚त्त‚द्‏विरुद्धाद्य‚ग‚तिग‚तिभेद‚प्र‚योग‚तः ॥ २९ ॥\&[\smallbreak]


	
	    \end{quote}
	  
	  \endgroup
	

	  \pstart \leavevmode% starting standard par
	\hphantom{.}‚{\color{DodgerBlue3}‚इति} निर्दिष्ट‚क्र‚मेणे‚{\color{DodgerBlue3}‚य‚म}‚नुप‚ल‚ब्धिः कार‚ण‚व्याप‚क‚स्व‚भावानुप‚ल‚ब्धिभेदेन ‚{\color{DodgerBlue3}‚त्रिवि‚{\tiny $_{lb}$}‚धा\edtext{}{\edlabel{pvv.298-8}\label{pvv.298-8}\lemma{धा}\Bfootnote{स‚मासात् ।}}प्य‚नेक‚धा} ब‚हुप्र‚कारा ‚{\color{DodgerBlue3}‚त‚त्त‚द्विरुद्धाद्य‚ग‚तिग‚तिभेद‚प्र‚योग‚तः । ते च} कार‚ण‚व्या‚{\tiny $_{lb}$}‚प‚क‚स्व‚भावा‚{\color{DodgerBlue3}‚स्तेषां} विरुद्धाद‚य‚श्च त‚त्त‚द्विरुद्धाद‚य आदि\edtext{}{\edlabel{pvv.298-9}\label{pvv.298-9}\lemma{आदि}\Bfootnote{कार‚ण‚विरुद्ध‚कार्योप‚ल‚ब्धिरादिना ।}}श‚ब्दाद् विरुद्ध‚कार्या‚{\tiny $_{lb}$}‚द‚य‚श्च तेषां य‚थाक्र‚म‚म‚ग‚तिग‚त‚यः कार‚ण‚व्याप‚क‚स्व‚भावानाम‚नुप‚ल‚ब्ध‚य‚{\tiny $_{3}$}‚ (:।) ‚{\tiny $_{lb}$}‚तेषां विरुद्धादीनामुप‚ल‚ब्ध‚य‚श्च तासाम‚न्योन्यं भेदो नानात्वं त‚स्य प्र‚योग‚तः । ‚{\tiny $_{lb}$}‚श‚ब्द‚स्याभिधाव्यापारात् । य‚थोक्तं\edtext{}{\edlabel{pvv.298-10}\label{pvv.298-10}\lemma{थोक्तं}\Bfootnote{प्राग‚त्राष्ट‚धोक्त‚व‚त् ।}}प्राक् । सैषाऽनेक‚प्र‚कारापि त्रिविधानुप‚ल‚ब्धिः ‚{\tiny $_{lb}$}‚संगृहीतेत्य‚र्थः ॥ (२९)
	\pend% ending standard par
      \label{div_pvv.3.30}
	  
	% new div opening: depth here is 2
	

	  \pstart \leavevmode% starting standard par
	उक्त‚म‚र्थ श्लोकाभ्यां संगृह्ण‚न्नाह ।
	\pend% ending standard par
      
	  \bigskip
	  \begingroup
	
	    \large
	  
	    \begin{quote}
	  
	    
	    \stanza[\smallbreak]
	\label{pv.3.30}\flagstanza{\tiny\textenglish{...v.3.30}}कार्य‚कार‚ण‚भावाद्वा स्व‚भावाद्वा नियाम‚कात् ।&अविनाभाव‚निय‚मोऽद‚र्श‚नान्न न द‚र्श‚नात् ॥ ३० ॥\&[\smallbreak]


	
	    \end{quote}
	  
	  \endgroup
	\textsuperscript{\textenglish{299/s}}

	  \pstart \leavevmode% starting standard par
	\hphantom{.}‚{\color{DodgerBlue3}‚कार्य‚कार‚ण‚भावात्} त‚दुत्प‚त्तेर्व्वा ‚{\color{DodgerBlue3}‚नियाम‚कात्} साध‚न‚स्य साध्याव्य‚भिचार‚{\tiny $_{lb}$}‚कार‚णात् ‚{\color{DodgerBlue3}‚स्व‚भावात्} तादात्म्याद्वा नियाम‚{\tiny $_{4}$}‚‚{\color{DodgerBlue3}‚काद‚वि\edtext{}{\edlabel{pvv.299-1}\label{pvv.299-1}\lemma{वि}\Bfootnote{अविनाभावेनैव सिद्धे पुन‚र्निय‚म‚ग्र‚ह‚णं प‚र‚निरासाय । स ह्य‚न्यं ‚{\tiny $_{lb}$}‚निय‚म‚मिच्छ‚ति ।}}नाभाव‚निय‚मः} साध्याव्य‚भि‚{\tiny $_{lb}$}‚चारित्व‚निय‚मः साध‚न‚स्य । विप‚क्षे हेतो‚{\color{DodgerBlue3}‚र‚द‚र्श‚नात्} न स‚प‚क्षे ‚{\color{DodgerBlue3}‚द‚र्श‚नात्} । द‚र्श‚नाद‚र्श‚न‚{\tiny $_{lb}$}‚योर्व्य‚भिचारिण्य‚पि हेतौ स‚म्भ‚वात् निय‚म‚हेत्व‚भावाच्च ॥ (३०)
	\pend% ending standard par
      \label{div_pvv.3.31}
	  
	% new div opening: depth here is 2
	
	  \bigskip
	  \begingroup
	
	    \large
	  
	    \begin{quote}
	  
	    
	    \stanza[\smallbreak]
	\label{pv.3.31}\flagstanza{\tiny\textenglish{...v.3.31}}अव‚श्यंभाव‚निय‚मोऽन्य‚था प‚र‚स्य कः प‚रैः ।&अर्थान्त‚र‚निमित्ते वा ध‚र्मे वास‚सि राग‚व‚त् ॥ ३१ ॥\&[\smallbreak]


	
	    \end{quote}
	  
	  \endgroup
	

	  \pstart \leavevmode% starting standard par
	\hphantom{.}‚{\color{DodgerBlue3}‚अन्य‚था} तादात्म्य‚त‚दुत्प‚त्त्योर‚व्य‚भिचार‚निब‚न्ध‚न‚योर‚स्वीकारे ‚{\color{DodgerBlue3}‚प‚र‚स्य} साध्य‚स्य ‚{\tiny $_{lb}$}‚‚{\color{DodgerBlue3}‚प‚रैः} साध‚नैः ‚{\color{DodgerBlue3}‚कोऽव‚श्यंभाव‚निय‚मो} न क‚श्चित् ।‚{\tiny $_{5}$}‚उत्पाद‚काद‚न्योर्थो‚{\color{DodgerBlue3}‚ऽर्थान्त‚रं त‚न्निमित्ते ‚{\tiny $_{lb}$}‚वा ध‚र्मे}\edtext{}{\edlabel{pvv.299-2}\label{pvv.299-2}\lemma{न्योर्थो}\Bfootnote{कृत‚क‚स्य हेतोर‚र्थान्त‚र‚स्य मृद्गादेर्निमित्त‚त्व‚म‚नित्यं प्र‚तीष्य‚ते य‚त् ‚{\tiny $_{lb}$}‚त‚स्य कुतो निय‚मः ।}}ऽस्व‚भाव‚भूते कोऽव‚श्य‚म्भाव‚निय‚मः । ‚{\color{DodgerBlue3}‚वास‚सि राग‚व‚त्}‚। य‚था\edtext{}{\edlabel{pvv.299-3}\label{pvv.299-3}\lemma{था}\Bfootnote{कुसुम्भादि ।}}र्थान्त‚र‚{\tiny $_{lb}$}‚निमित्त‚स्य राग‚स्य व‚स्त्रे\edtext{}{\edlabel{pvv.299-4}\label{pvv.299-4}\lemma{स्त्रे}\Bfootnote{पूर्व्व‚निष्प‚न्ने ।}} नाव‚श्यंभाव‚नि\edtext{}{\edlabel{pvv.299-5}\label{pvv.299-5}\lemma{नि}\Bfootnote{दूष‚णान्त‚र‚माह । अर्थान्त‚र‚निमित्त‚म‚नित्य‚त्वं कृत‚काद‚न्य‚देव स्यात् ।}}य‚मः ॥ (३१)
	\pend% ending standard par
      \label{div_pvv.3.32}
	  
	% new div opening: depth here is 2
	

	  \begin{center}%% label @type='head'
	\textbf{(ख) अविनाभाव‚निय‚मः}
	\end{center}
	
	  \bigskip
	  \begingroup
	
	    \large
	  
	    \begin{quote}
	  
	    
	    \stanza[\smallbreak]
	\label{pv.3.32}\flagstanza{\tiny\textenglish{...v.3.32}}अर्थान्त‚र‚निमित्तो हे ध‚र्मः स्याद‚न्य एव सः ।&प‚श्चाद्भावान्न हेतुत्वं फ‚लेप्येकान्त‚ता कुतः ॥ ३२ ॥\&[\smallbreak]


	
	    \end{quote}
	  
	  \endgroup
	

	  \pstart \leavevmode% starting standard par
	\hphantom{.}व‚स्त्रोत्पाद‚का‚{\color{DodgerBlue3}‚द‚र्थान्त‚र‚निमित्तो} राग‚द्र‚व्य‚कार‚णो ‚{\color{DodgerBlue3}‚हि ध‚र्मो} रागः\edtext{}{\edlabel{pvv.299-6}\label{pvv.299-6}\lemma{रागः}\Bfootnote{अय‚मेव हि भेदो भेद‚हेतुर्व्वा य‚दुत विरुद्ध‚ध‚र्माध्यासः कार‚ण‚भेद‚श्च ।}} प्रागुत्प‚न्नाद् ‚{\tiny $_{lb}$}‚व‚स्त्रा‚{\color{DodgerBlue3}‚द‚न्य एव स्याद्} भिन्न‚काल‚त्वाद् भिन्न‚हेतुक‚त्वाच्च (।) इत्थ\edtext{}{\edlabel{pvv.299-7}\label{pvv.299-7}\lemma{इत्थ}\Bfootnote{त्रैगुण्यादैक्य‚मिति चेदाह ।}}म‚पि य‚द्येक‚त्वं\leavevmode\ledsidenote{\textenglish{59b/MA}} ‚{\tiny $_{lb}$}‚विश्व‚{\tiny $_{6}$}‚मेकं भ‚वेत् । त‚त‚श्च स‚होत्प‚त्तिनाशौ स्यातां । स‚र्व‚त्र स‚र्व्वं चोप‚युज्य‚ते । ‚{\tiny $_{lb}$}‚एक‚त्वाभिमान‚स्तु स‚दृशाप‚राप‚रोत्प‚त्तेर्भ्रान्त्या । त‚स्य\edtext{}{\edlabel{pvv.299-8}\label{pvv.299-8}\lemma{स्य}\Bfootnote{अहेतुफ‚ल‚स्य साध्य‚स्यासंब‚न्धाद्धेतुः फ‚ल‚म्वा स्यादित्याह ।}} चा\edtext{}{\edlabel{pvv.299-9}\label{pvv.299-9}\lemma{चा}\Bfootnote{कृत‚काद‚र्थान्त‚र‚निमित्त‚स्यानित्य‚स्य न हेतुत्वं ।}}र्थान्त‚र‚निमित्त\edtext{}{\edlabel{pvv.299-10}\label{pvv.299-10}\lemma{निमित्त}\Bfootnote{फ‚ले नित्ये प‚श्चाद्भाविनि च नैकान्त‚ता कृत‚क‚स्य ज्ञाप‚क‚हेतोः ।}}स्य ‚{\tiny $_{lb}$}‚ध‚र्म‚स्य व‚स्त्रोत्पादात् ‚{\color{DodgerBlue3}‚प‚श्चाद्भावान्न हेतुत्वं} व‚स्त्रं प्र‚ति । अतः कार‚ण‚त‚यापि ‚{\tiny $_{lb}$}‚नास्यानुमानं रागोत्प‚त्तौ व‚स्त्रं स‚ह‚कारिकार‚णं । अतः कार्य‚त‚या रागानुमानं चेत् ‚{\tiny $_{lb}$}‚\leavevmode\ledsidenote{\textenglish{300/s}} ‚{\color{DodgerBlue3}‚फ‚लेपि} कार‚णाद‚नुमीय‚माने ‚{\color{DodgerBlue3}‚एकान्त‚ता कु‚{\tiny $_{1}$}‚तः} । न ह्य‚व‚श्यं कार‚णानि कार्य‚व‚न्ति ‚{\tiny $_{lb}$}‚भ‚व‚न्ति ॥ (३२)
	\pend% ending standard par
      \label{div_pvv.3.33}
	  
	% new div opening: depth here is 2
	

	  \pstart \leavevmode% starting standard par
	य‚दि द‚र्श‚नाद‚र्श‚ने नान्व‚य‚व्य‚तिरेक‚बुद्धिहेतुर्द्धूमोग्निन्न व्य‚भिच‚र‚तीति न स्यात ‚{\tiny $_{lb}$}‚प्र‚तिप‚त्तिरित्याह ।
	\pend% ending standard par
      
	  \bigskip
	  \begingroup
	
	    \large
	  
	    \begin{quote}
	  
	    
	    \stanza[\smallbreak]
	\label{pv.3.33}\flagstanza{\tiny\textenglish{...v.3.33}}कार्यं धूमो हुत‚भुजः कार्य‚ध‚र्मानुवुत्तितः ।&त‚स्याभावे तु स भ‚व‚न् हेतुम‚त्तां विलंघ‚येत् ॥ ३३ ॥\&[\smallbreak]


	
	    \end{quote}
	  
	  \endgroup
	

	  \pstart \leavevmode% starting standard par
	\hphantom{.}‚{\color{DodgerBlue3}‚कार्यं धूमो हुत‚भुजः} कार्य‚ध‚र्म‚स्य कार‚णान्व‚य‚व्य‚तिरेकानुविधायित्व‚स्य त्रिविध\edtext{}{\edlabel{pvv.300-1}\label{pvv.300-1}\lemma{त्रिविध}\Bfootnote{प्राग‚दृष्टौ क्र‚मात् प‚श्य‚न् वेत्ति हेतुफ‚ल‚स्थितिं । ‚{\tiny $_{lb}$}‚दृष्टौ वा क्र‚म‚शोऽप‚श्य‚न्न‚न्य‚था त्व‚न‚व‚स्थितिः ॥}}‚{\tiny $_{lb}$}‚द‚र्श‚नाद‚र्श‚न‚निश्चित‚स्यानुवृत्तितः स\edtext{}{\edlabel{pvv.300-2}\label{pvv.300-2}\lemma{स}\Bfootnote{धूम‚स्यान्य‚त्राग्निज‚न्य‚त्वे न किञ्चिद् बाध‚क‚म‚स्ति । त‚देवेद‚मिति च ‚{\tiny $_{lb}$}‚प्र‚तीतेस्त‚त्सामान्यं प्र‚तीत‚मुच्य‚ते ।}}धूम‚स्त‚स्याग्नेर‚{\color{DodgerBlue3}‚भावे तु} भ‚व‚न् ‚{\color{DodgerBlue3}‚हेतुम‚त्ताम्वि\edtext{}{\edlabel{pvv.300-3}\label{pvv.300-3}\lemma{त्ताम्वि}\Bfootnote{न वाग्निस‚म्ब‚न्धित‚यात्य‚क्ष‚मीक्षेताकार‚णान् स‚कृद‚प्य‚नुद‚यात् ।}}लंघ‚{\tiny $_{lb}$}‚येद‚ति}‚क्राम‚येत् ॥ (३३)
	\pend% ending standard par
      \label{div_pvv.3.34}
	  
	% new div opening: depth here is 2
	
	  \bigskip
	  \begingroup
	
	    \large
	  
	    \begin{quote}
	  
	    
	    \stanza[\smallbreak]
	\label{pv.3.34}\flagstanza{\tiny\textenglish{...v.3.34}}नित्यं स‚त्त्व‚म‚स‚त्त्वं वाऽहेतोर‚न्यान‚पेक्ष‚णात् ।&अपेक्षात‚श्च भावानां कादाचित्क‚स्य स‚म्भ‚वः ॥ ३४ ॥\&[\smallbreak]


	
	    \end{quote}
	  
	  \endgroup
	

	  \pstart \leavevmode% starting standard par
	\hphantom{.}अहेतुत्वे च धूम‚स्य ‚{\color{DodgerBlue3}‚नित्यं स‚त्त्व‚मा}‚काश‚स्येव स्यात् । ‚{\color{DodgerBlue3}‚अस‚त्त्वं} श‚श‚विषाणादेरिव । ‚{\tiny $_{lb}$}‚‚{\color{DodgerBlue3}‚अहेतोर‚न्यापेक्ष‚णा}‚भावात् ‚{\color{DodgerBlue3}‚अपेक्षात‚श्च भावानां कादाचित्क‚स्य स‚म्भ‚वः} । त‚तो ‚{\tiny $_{lb}$}‚य‚द्य‚हेतुर्भाव‚स्त‚दा नित्यं स्यात् । अस‚देव वा स्याद्धेत्व‚भावात् (।) त‚स्माद् धूम‚स्य\edtext{}{\edlabel{pvv.300-4}\label{pvv.300-4}\lemma{स्य}\Bfootnote{नापि स्व‚भाव‚तो भ‚व‚ति । अनिष्प‚न्न‚स्यास‚त्वादेव ।}} ‚{\tiny $_{lb}$}‚कादाचित्क‚त्व‚द‚र्श‚नात् हेतुम‚त्वं । अग्न्य‚न्व‚य‚व्य‚तिरेकानुविधान‚द‚र्श‚नात् त‚त्कार्य‚{\tiny $_{lb}$}‚त्व‚ञ्च । य‚श्च य‚स्य कार्यं स तं न‚{\tiny $_{3}$}‚ व्य‚भिच‚र‚ति । त‚द‚धीन‚स्व‚रूप‚त्वात्\edtext{}{\edlabel{pvv.300-5}\label{pvv.300-5}\lemma{त्वात्}\Bfootnote{त‚द्विकार‚ण‚म‚कार‚णं ।}}॥ (३४)
	\pend% ending standard par
      \label{div_pvv.3.35}
	  
	% new div opening: depth here is 2
	

	  \pstart \leavevmode% starting standard par
	अत‚श्च (।)
	\pend% ending standard par
      
	  \bigskip
	  \begingroup
	
	    \large
	  
	    \begin{quote}
	  
	    
	    \stanza[\smallbreak]
	\label{pv.3.35}\flagstanza{\tiny\textenglish{...v.3.35}}अग्निस्व‚भावः श‚क्र‚स्य मूर्धा य‚द्य‚ग्निरेव सः ।&अथान‚ग्निस्व‚भावोसौ धूम‚स्त‚त्र क‚थं भ‚वेत् ॥ ३५ ॥\&[\smallbreak]


	
	    \end{quote}
	  
	  \endgroup
	

	  \pstart \leavevmode% starting standard par
	\hphantom{.}‚{\color{DodgerBlue3}‚अग्निस्व‚भावः श‚क्र‚स्य मूर्द्धा} व‚ल्मीको ‚{\color{DodgerBlue3}‚य‚द्य‚ग्निरेव स} त‚दा न हि व‚ह्निस्व‚रूप‚तां ‚{\tiny $_{lb}$}‚विहायान्य‚द् व‚ह्ने रूपं । ‚{\color{DodgerBlue3}‚अथा}‚न्य‚था प्र‚तीय‚मान‚त्वाद‚{\color{DodgerBlue3}‚न‚ग्निस्व‚भावोसौ} त‚दा ‚{\color{DodgerBlue3}‚धूमो} व‚ह्नेर्ज‚न्य‚स्व‚भाव‚{\color{DodgerBlue3}‚स्त‚त्र श‚क्र‚मूर्ध्नि क‚थं भ‚वेत्} । न हि व‚ह्निज‚न्योन्य‚स्माद् भ‚वितु‚{\tiny $_{lb}$}‚\leavevmode\ledsidenote{\textenglish{301/s}} म‚र्ह‚ति त‚द‚धीन‚त्वात् । त‚तः श‚क्र‚मूर्ध्नो धूमोत्प‚त्तिरिति भ्रान्तिरे‚{\tiny $_{4}$}‚षा\edtext{}{\edlabel{pvv.301-1}\label{pvv.301-1}\lemma{षा}\Bfootnote{वाष्पे व‚र्षासु । ..........}}। व‚ह्नेरेव ‚{\tiny $_{lb}$}‚त‚द्देश‚व‚र्तिनोऽनुप‚ल‚क्षितादुत्प‚त्तिः ॥(३५)
	\pend% ending standard par
      \label{div_pvv.3.36}
	  
	% new div opening: depth here is 2
	

	  \pstart \leavevmode% starting standard par
	किञ्च (।)
	\pend% ending standard par
      
	  \bigskip
	  \begingroup
	
	    \large
	  
	    \begin{quote}
	  
	    
	    \stanza[\smallbreak]
	\label{pv.3.36}\flagstanza{\tiny\textenglish{...v.3.36}}धूम‚हेतुस्व‚भावो हि व‚ह्निस्त‚च्छ‚क्तिभेद‚वान् ।&अधूम‚हेतोर्धूम‚स्य भावे स स्याद‚हेतुकः ॥ ३६ ॥\&[\smallbreak]


	
	    \end{quote}
	  
	  \endgroup
	

	  \pstart \leavevmode% starting standard par
	\hphantom{.}‚{\color{DodgerBlue3}‚धूम‚हेतुस्व‚भाव‚स्त‚च्छ‚क्ति\edtext{}{\edlabel{pvv.301-2}\label{pvv.301-2}\lemma{क्ति}\Bfootnote{ख‚द्योतादिस‚काशात् ।}}भेद‚वान्} धूम‚ज‚न‚न‚श‚क्तिविशेष‚युक्तो ‚{\color{DodgerBlue3}‚व‚ह्निः} प्र‚तीतः । ‚{\tiny $_{lb}$}‚‚{\color{DodgerBlue3}‚अधूम‚हेतो}‚र‚द‚ह‚नात् ‚{\color{DodgerBlue3}‚धूम‚स्य भावे स} धूमो‚{\color{DodgerBlue3}‚ऽहेतुकः स्यात्} । हेतुं प्र‚माण‚निश्चित‚म‚न्त‚{\tiny $_{lb}$}‚रेणैवोत्पादात् । अहेतुत्वे च नित्यं स‚त्त्व‚म‚स‚त्त्व‚म्वां स्यादित्युक्तं ॥(३६)
	\pend% ending standard par
      \label{div_pvv.3.37}
	  
	% new div opening: depth here is 2
	

	  \pstart \leavevmode% starting standard par
	स्यादेत‚त्(।) शालूकादि स्व‚बीजाद् विजाती‚{\tiny $_{5}$}‚‚{\color{DodgerBlue3}‚याच्च} गोम‚यादेर्दृश्य‚ते (।) ‚{\tiny $_{lb}$}‚त‚त्क‚थ‚म‚व‚ह्नेर्द्धूमोत्पादेऽहेतुत्व‚प्र‚स‚ङ्ग इत्याह (।)
	\pend% ending standard par
      
	  \bigskip
	  \begingroup
	
	    \large
	  
	    \begin{quote}
	  
	    
	    \stanza[\smallbreak]
	\label{pv.3.37}\flagstanza{\tiny\textenglish{...v.3.37}}अन्व‚य‚व्य‚तिरेकाद्यः य‚स्य दृष्टोनुव‚र्त‚कः ।&स्व‚भाव‚स्त‚स्य त‚द्धेतुर‚तो भिन्नान्न स‚म्भ‚वः ॥ ३७ ॥\&[\smallbreak]


	
	    \end{quote}
	  
	  \endgroup
	

	  \pstart \leavevmode% starting standard par
	\hphantom{.}‚{\color{DodgerBlue3}‚अन्व‚य\edtext{}{\edlabel{pvv.301-3}\label{pvv.301-3}\lemma{य}\Bfootnote{कार्य‚हेतुम‚धि ।}}व्य‚तिरेकाद् यः स्व‚भावो य‚स्यानुव‚र्त्त‚को}‚ऽपेक्ष‚को दृष्ट‚स्तंस्य स्व‚भाव‚स्य ‚{\tiny $_{lb}$}‚त‚द‚नुव‚र्त्त्य‚मानं हेतुर‚तो भिन्नाद् विजातीयान्न संभ‚वः क‚स्य‚चि\edtext{}{\edlabel{pvv.301-4}\label{pvv.301-4}\lemma{चि}\Bfootnote{उत्प‚द्य‚मान‚स्य पूर्व्वाप‚र‚रूप‚विविक्त‚स्य प्र‚त्य‚क्षेण ग्र‚हात् क्ष‚णिक‚ग्र‚ह एव । ‚{\tiny $_{lb}$}‚न त्व‚क्ष‚णिक‚ग्र‚हः पूर्व्वाप‚र‚काल‚योर‚भासात् त‚त्स‚म्ब‚न्धित‚याऽभान‚मिदानीं पूर्व्व‚स्य ‚{\tiny $_{lb}$}‚विनाशः । अन्य‚स्व‚भाव‚स्य भान‚मेवोत्पाद इति क‚थ‚मुच्य‚ते पूर्वोत्त‚र‚क्ष‚णानां विनाशो‚{\tiny $_{lb}$}‚त्पादादृष्टेर‚क्ष‚णिक इति । नाप्य‚नेक‚क्ष‚ण‚रूप इदानीन्त‚नः त‚त्त्वे ह्य‚तीतादित्व‚म‚स्य ‚{\tiny $_{lb}$}‚स्यात् । त‚त्र चास‚त्त्वादेवाभानं । भास‚श्च व‚र्त्त‚मान‚स्यैव स‚त्त्वात् । तेन स्प‚ष्ट‚व‚स्तुभान‚{\tiny $_{lb}$}‚मेव क्ष‚णिक‚त्वं भ्रान्तेस्तु नाध्य‚व‚सायः प्र‚थ‚म‚म‚नुप‚ल‚ब्ध्या प्राक् स‚त्त्वं । अन्य‚त आग‚म‚नं ‚{\tiny $_{lb}$}‚त‚त्सिन्निहि (त) कुड्यादेर्हेतुत्व‚निषेधः । एताव‚द्भिर्धूमो नाग्निज‚न्यः स्यात् । ‚{\tiny $_{lb}$}‚प्र‚थ‚म‚प्र‚त्य‚क्ष‚प‚क्षे य‚त्संनिधानात् कार्य‚प्र‚वृत्तिस्त‚न्म‚ध्ये य‚द‚भावे कार्याभाव‚स्त‚त्क‚र‚णं ‚{\tiny $_{lb}$}‚ज्ञेयं ग‚र्द‚भादेः ।}}द् । य‚द्धि शालूकादि ‚{\tiny $_{lb}$}‚स्व‚बीज‚प्र‚भ‚वं य‚च्च विजातीय‚प्र‚भ‚वं त‚योः स‚दृश‚त्व‚प्र‚तीताव‚पि न तादृश‚त्वं र‚स‚{\tiny $_{lb}$}‚वीर्य‚विपा‚{\tiny $_{6}$}‚क\edtext{}{\edlabel{pvv.301-5}\label{pvv.301-5}\lemma{क}\Bfootnote{य‚दा क‚द‚लीबीज‚क‚न्दोद्भ‚वा स्फुट‚मेव तादृशो लोको विवेच‚य‚त्याकार‚भेदात् ।}}भेदात् । य‚दि तु हेतुभेदेप्य‚भेदो विश्वात्म‚कं द्र‚व्यं स्यादित्याद्युक्तं\leavevmode\ledsidenote{\textenglish{60a/MA}} ‚{\tiny $_{lb}$}‚प्र‚स‚ज्येत ॥(३७)
	\pend% ending standard par
      \label{div_pvv.3.38}
	  
	% new div opening: depth here is 2
	

	  \begin{center}%% label @type='head'
	\textbf{(४) सामान्य‚चिन्ता}
	\end{center}
	
	  \bigskip
	  \begingroup
	
	    \large
	  
	    \begin{quote}
	  
	    
	    \stanza[\smallbreak]
	\label{pv.3.38}\flagstanza{\tiny\textenglish{...v.3.38}}स्व‚भावेप्य‚विनाभावो भाव‚मात्रानुरोधिनि ।&त‚द‚भावे स्व‚य‚म्भाव‚स्याभावः स्याद‚भेद‚तः ॥ ३८ ॥\&[\smallbreak]


	
	    \end{quote}
	  
	  \endgroup
	\textsuperscript{\textenglish{302/s}}

	  \pstart \leavevmode% starting standard par
	स्व‚भावे\edtext{}{\edlabel{pvv.302-1}\label{pvv.302-1}\lemma{भावे}\Bfootnote{स्व‚भावे भावोपीत्युक्तेपि प्राक् त‚द‚नूद्य प‚क्षैक‚देशासिद्धिनिवृत्त्य‚र्थं पुन‚राह ।}}पि हेताव‚विनाभावः । क्व (।) साध्ये ‚{\color{DodgerBlue3}‚भाव‚मात्रानुरोधिनि} साध‚न‚{\tiny $_{lb}$}‚स्व‚रूप‚मात्रानुव‚र्त्तिनि व्याप‚के य‚था कृत‚क‚त्व‚स्यानित्य‚त्वे (।) य‚त‚{\color{DodgerBlue3}‚स्त}‚स्य व्याप‚क‚स्या‚{\tiny $_{lb}$}‚‚{\color{DodgerBlue3}‚भावे भाव‚स्य} व्याप्य‚स्य कृत‚क‚त्वादेः ‚{\color{DodgerBlue3}‚स्व‚य‚म‚भावः स्यात् । अभेद‚त} ऐकात्म्याद् ‚{\tiny $_{lb}$}‚व‚स्तुतः ॥ (३८)
	\pend% ending standard par
      \label{div_pvv.3.39}
	  
	% new div opening: depth here is 2
	

	  \pstart \leavevmode% starting standard par
	य‚दि य एव कृत‚कः स एवानित्यो भेदाभावात् त‚दा\edtext{}{\edlabel{pvv.302-2}\label{pvv.302-2}\lemma{दा}\Bfootnote{एकं स‚न्धित्स‚तोऽप‚रं प्र‚बाध‚ते ।}} ‚{\tiny $_{1}$}‚ प्र‚तिज्ञा\edtext{}{\edlabel{pvv.302-3}\label{pvv.302-3}\lemma{तिज्ञा}\Bfootnote{प‚क्ष‚निर्देशः प्र‚तिज्ञा त‚स्या अर्थो ध‚र्म्म‚र्धाम्म‚स‚मुदायः । त‚दैक‚देशः ‚{\tiny $_{lb}$}‚साध्य‚ध‚र्मात्म‚को हेतुः स्याद‚सिद्धः (।) अत्र च पूर्व‚वृत्तेन ध‚र्म‚क‚ल्प‚नाबीजं द्वितीयेन ‚{\tiny $_{lb}$}‚ध‚र्म‚क‚ल्प‚ना तृतीयेन प्र‚तिज्ञार्थैक‚देश‚ग‚ताप‚रिहार इति स‚मासार्थः ।}}र्थैक‚देशो ‚{\tiny $_{lb}$}‚हेतुः स्यादित्याह ।
	\pend% ending standard par
      
	  \bigskip
	  \begingroup
	
	    \large
	  
	    \begin{quote}
	  
	    
	    \stanza[\smallbreak]
	\label{pv.3.39}\flagstanza{\tiny\textenglish{...v.3.39}}स‚र्वे भावाः स्व‚भावेन स्व‚स्व‚भाव‚व्य‚व‚स्थितेः ।&स्व‚भाव‚प‚र‚भावाभ्यां व्यावृत्तिभागिनो य‚तः ॥ ३९ ॥\&[\smallbreak]


	
	    \end{quote}
	  
	  \endgroup
	

	  \pstart \leavevmode% starting standard par
	\hphantom{.}‚{\color{DodgerBlue3}‚स‚र्व्वे भावाः स्व‚भावेन\edtext{}{\edlabel{pvv.302-4}\label{pvv.302-4}\lemma{भावेन}\Bfootnote{जात्यादिभिर्भावा भिद्य‚न्ते न स्व‚भावेनेति प‚रः (।) त‚न्न (।) भावा ‚{\tiny $_{lb}$}‚भिन्नाभिन्न‚भेद‚कार‚णे भाव‚स्य न कि (ञ्चि)त् ॥ स‚र्व्व‚भावा भिन्ना इति नाध्य‚क्षानु‚{\tiny $_{lb}$}‚ग‚म्य‚मिति ब‚ह‚वः । त‚न्न । त‚त्व‚मुक्तं प्र‚तिद्र‚व्यं भिन्न‚रूपोप‚ल‚म्भ‚नात् । न ह्याख्यातुम‚श‚क्य‚त्वाद् भेदो नास्तीति ग‚म्य‚ते ॥ ग‚वाश्वादीनां स्व‚रूपेणोत्प‚त्तिरेव भेदः । }} स्व‚स्व‚भाव‚व्य‚व‚स्थितेः} । आत्मात्मीय‚रूप‚व‚स्थित‚{\tiny $_{lb}$}‚त्वात् । स्व‚भाव‚प‚र‚भावाभ्यां स‚जातीयाद् विजातीयाच्च व्यावृत्ति\edtext{}{\edlabel{pvv.302-5}\label{pvv.302-5}\lemma{व्यावृत्ति}\Bfootnote{य‚दि न व‚स्तुतो ध‚र्म‚ध‚र्मिभावः क‚थ‚म्बुद्धिभेद इत्याह ।}}भागिनो य‚स्मा\edtext{}{\edlabel{pvv.302-6}\label{pvv.302-6}\lemma{स्मा}\Bfootnote{नित्याकृत‚कादेः ।}}न्न ‚{\tiny $_{lb}$}‚केन‚चिन्मिश्राः । (३९)
	\pend% ending standard par
      \label{div_pvv.3.40}
	  
	% new div opening: depth here is 2
	
	  \bigskip
	  \begingroup
	
	    \large
	  
	    \begin{quote}
	  
	    
	    \stanza[\smallbreak]
	\label{pv.3.40}\flagstanza{\tiny\textenglish{...v.3.40}}त‚स्माद् व्यावृत्तिर‚र्थानां य‚त‚श्च त‚न्निब‚न्ध‚नाः ।&जातिभेदाः प्र‚क‚ल्प्य‚न्ते त‚द्विशेषाव‚गाहिनः ॥ ४० ॥\&[\smallbreak]


	
	    \end{quote}
	  
	  \endgroup
	

	  \pstart \leavevmode% starting standard par
	\hphantom{.}त‚स्मात् ‚{\color{DodgerBlue3}‚य‚तो} य‚तोऽस्व‚रूपाद‚र्था\edtext{}{\edlabel{pvv.302-7}\label{pvv.302-7}\lemma{र्था}\Bfootnote{श‚ब्दाद‚यः ।}}द् ‚{\color{DodgerBlue3}‚व्यावृत्तिर‚र्थानां त‚न्निब‚न्ध‚नाः}\edtext{}{\edlabel{pvv.302-8}\label{pvv.302-8}\lemma{द्}\Bfootnote{नित्यो नेति वास‚नातोऽनित्य‚बुद्धिरेवेति बुद्धिभेदः ।}} त‚त्त‚द्व्या-
	\pend% ending standard par
      \textsuperscript{\textenglish{303/s}}

	  \pstart \leavevmode% starting standard par
	\hphantom{.}वृत्तिनिमित्ता ‚{\color{DodgerBlue3}‚जाति\edtext{}{\edlabel{pvv.303-1}\label{pvv.303-1}\lemma{जाति}\Bfootnote{अनित्य‚कृत‚काद‚यः ।}}भेदास्त\edtext{}{\edlabel{pvv.303-2}\label{pvv.303-2}\lemma{भेदास्त}\Bfootnote{स्व‚ल‚क्ष‚ण‚स्य ये विशेषा अकृत‚कादिव्यावृत्ताः ।}}द्विशेषाव‚गाहिनः} त‚त्स्व‚ल‚क्ष‚णाश्र‚{\tiny $_{2}$}‚याः ‚{\color{DodgerBlue3}‚क‚ल्प्य‚न्ते} श‚ब्दैः ‚{\tiny $_{lb}$}‚स्व‚वाच्य‚त‚या ॥ (४०)
	\pend% ending standard par
      \label{div_pvv.3.41}
	  
	% new div opening: depth here is 2
	

	  \pstart \leavevmode% starting standard par
	य‚तः श‚ब्दः काश्चिद् अर्थ‚क्रियाः कुर्व‚न्न‚त‚त्कारिणोऽश‚ब्दाद‚कृत‚कान्नित्यादेश्च ‚{\tiny $_{lb}$}‚व्यावृत्त इति त‚त्त‚द्व‚य‚व‚च्छेद‚प्र‚तिपाद‚नार्थं श‚ब्द‚कृत‚कानित्य‚त्वाद्या जात‚योऽन‚न्य‚{\tiny $_{lb}$}‚श‚ब्द‚वाच्य‚त‚या क‚ल्प्य‚न्ते ।
	\pend% ending standard par
      
	  \bigskip
	  \begingroup
	
	    \large
	  
	    \begin{quote}
	  
	    
	    \stanza[\smallbreak]
	\label{pv.3.41}\flagstanza{\tiny\textenglish{...v.3.41}}त‚स्माद् विशेषो यो येन ध‚र्मेण संप्र‚तीय‚ते ।&न स श‚क्य‚स्त‚तोन्येन तेन भिन्ना व्य‚व‚स्थितिः ॥ ४१ ॥\&[\smallbreak]


	
	    \end{quote}
	  
	  \endgroup
	

	  \pstart \leavevmode% starting standard par
	\hphantom{.}‚{\color{DodgerBlue3}‚त‚स्मात्} स्व‚भावाभेदेपि कृत‚क‚त्वानित्य‚त्वादीनां ‚{\color{DodgerBlue3}‚येन} ध‚र्मेण नाम्नाऽनित्य ‚{\tiny $_{lb}$}‚इत्य‚नेन ‚{\color{DodgerBlue3}‚यो विशेषो} नित्याद् व्यावृत्तिः ‚{\color{DodgerBlue3}‚संप्र‚{\tiny $_{3}$}‚तीय‚ते न} स विशेषः ‚{\color{DodgerBlue3}‚श‚क्यः त‚तो}‚ऽनित्य‚{\tiny $_{lb}$}‚श‚ब्दा‚{\color{DodgerBlue3}‚द‚न्येन} कृत‚क‚श‚ब्देन प्र‚त्येतुं (।)। ‚{\color{DodgerBlue3}‚तेन} कार‚णेन साध्य‚साध‚न‚यो‚{\color{DodgerBlue3}‚र्भिन्ना व्य‚व‚स्थितिः} । ‚{\tiny $_{lb}$}‚य‚द्य‚पि श‚ब्द एव कृत‚कोऽनित्य‚श्च त‚थाप्य‚श‚ब्द‚व्यावृत्त‚त‚या निश्चितो ध‚र्मी कृत‚{\tiny $_{lb}$}‚क‚त‚या च हेतुः । अनित्य‚त‚या चासिद्धः साध्यः\edtext{}{\edlabel{pvv.303-3}\label{pvv.303-3}\lemma{साध्यः}\Bfootnote{अतो भिन्न‚विष‚या विक‚ल्पाः श‚ब्दाश्च त‚त्स‚म‚विष‚या अप‚र्यायाः ।}} । त‚तो ध‚र्मिसाध्य‚साध‚नानां भेदः ‚{\tiny $_{lb}$}‚क‚ल्पितो न तु तान्येव क‚ल्पितानि तेषां स‚त्त्वात् ।‚{\tiny $_{4}$}‚ त‚तः क‚ल्पित‚ध‚र्मिणि क‚ल्पितात् ‚{\tiny $_{lb}$}‚साध‚नात् क‚ल्पित‚स्य साध्य‚स्य\edtext{}{\edlabel{pvv.303-4}\label{pvv.303-4}\lemma{स्य}\Bfootnote{स‚र्व्व‚ग‚त आत्मा स‚र्व्व‚त्रोप‚ल‚भ्य‚मान‚गुण‚त्वादित्यादि ।}} सिद्धिमाच‚क्षाणः प्र‚त्याख्यातः (।) ‚{\color{DodgerBlue3}‚त‚स्मान्न} प्र‚तिज्ञार्थैक‚देशो हेतुः । साध्य‚साध‚न‚योर्भिन्न‚व्य‚व‚च्छेद‚रूप‚त्वात् ॥ (४१)
	\pend% ending standard par
      \label{div_pvv.3.42}
	  
	% new div opening: depth here is 2
	

	  \pstart \leavevmode% starting standard par
	\hphantom{.}क‚स्मात् पुन‚र्व्य‚व‚च्छेदः श‚ब्द‚लिङ्गाभ्यां प्र‚तिपाद्य‚त इतीष्य‚ते न तु ‚{\color{DodgerBlue3}‚व‚स्त्वेव} विधिरूपेणेत्याह\edtext{}{\edlabel{pvv.303-5}\label{pvv.303-5}\lemma{विधिरूपेणेत्याह}\Bfootnote{सामान्यं ख‚लु नात्माधि (ष्ठि)तोपि क‚ल्पितो ध‚र्म‚ध‚र्मिभावोन्य‚थापि साध्य‚ते ।}} (।)
	\pend% ending standard par
      
	  \bigskip
	  \begingroup
	
	    \large
	  
	    \begin{quote}
	  
	    
	    \stanza[\smallbreak]
	\label{pv.3.42}\flagstanza{\tiny\textenglish{...v.3.42}}एक‚स्यार्थ‚स्व‚भाव‚स्य प्र‚त्य‚क्ष‚स्य स‚तः स्व‚य‚म् ।&कोन्यो भागो न दृष्टः स्यात् यः प्र‚माणैः प‚रीक्ष्य‚ते ॥ ४२ ॥\&[\smallbreak]


	
	    \end{quote}
	  
	  \endgroup
	

	  \pstart \leavevmode% starting standard par
	\hphantom{.}‚{\color{DodgerBlue3}‚एक‚स्थार्थ‚स्व‚भाव‚स्य स्व‚य‚मात्म‚ना प्र‚त्य‚क्ष‚स्य स‚तः कोन्यो भागो न दृ‚{\tiny $_{5}$}‚ष्टः ‚{\tiny $_{lb}$}‚स्यात् यः प्र‚माणैः\edtext{}{\edlabel{pvv.303-6}\label{pvv.303-6}\lemma{माणैः}\Bfootnote{ध‚र्म‚ध‚र्मिनिरंश‚स्य दृष्ट‚स्यानुमानैर्व्य‚क्त्य‚पेक्ष‚या ब‚हुव‚च‚नं ।}} प‚रीक्ष्य‚ते} निर्ण्णीय‚ते । य‚दि प्र‚माणान्त‚रैः श‚ब्दान्त‚रैश्च ‚{\tiny $_{lb}$}‚व‚स्त्वेव विष‚यीक‚र्त्त‚व्यं\edtext{}{\edlabel{pvv.303-7}\label{pvv.303-7}\lemma{व्यं}\Bfootnote{य‚थाऽनित्ये साध्ये श‚ब्दो ध‚र्मी प्र‚त्य‚क्ष‚तः स‚र्व्वांकारं सिद्धः स्यात् ।}}त‚दा त‚त् प्र‚त्य‚क्षेण श‚ब्दान्त‚रेण च दृष्ट‚मेवेति व्य‚र्थानि ‚{\tiny $_{lb}$}‚प्र‚माणानि स्युः । (४२)
	\pend% ending standard par
      \textsuperscript{\textenglish{304/s}}\label{div_pvv.3.43}
	  
	% new div opening: depth here is 2
	

	  \pstart \leavevmode% starting standard par
	त्व‚न्म‚तेपि प्र‚त्य‚क्षेण दृष्टे ध‚र्मिणि प्र‚माणान्त‚र‚वैफ‚ल्यं स्यात् (।) नेत्याह ।
	\pend% ending standard par
      
	  \bigskip
	  \begingroup
	
	    \large
	  
	    \begin{quote}
	  
	    
	    \stanza[\smallbreak]
	\label{pv.3.43}\flagstanza{\tiny\textenglish{...v.3.43}}नो चेद् भ्रान्तिनिमित्तेन संयोज्येत गुणान्त‚र‚म् ।&शुक्तौ वा र‚ज‚ताकारो रूप‚साध‚र्म्य‚द‚र्श‚नात् ॥ ४३ ॥\&[\smallbreak]


	
	    \end{quote}
	  
	  \endgroup
	

	  \pstart \leavevmode% starting standard par
	\hphantom{.}दृष्टे स‚र्व‚था व‚स्तुनि ‚{\color{DodgerBlue3}‚नो चेद् भ्रान्तिनिमित्तेन} सादृश्यादिना ‚{\color{DodgerBlue3}‚संयोज्ये}‚तारोप्येत ‚{\tiny $_{lb}$}‚‚{\color{DodgerBlue3}‚गुणान्त‚रं} ध‚र्म्मान्त‚र‚म‚स‚दे‚{\tiny $_{6}$}‚व\edtext{}{\edlabel{pvv.304-1}\label{pvv.304-1}\lemma{व}\Bfootnote{विक‚ल्पेन स‚दृशाप‚रोत्प‚त्या भ्रान्त्या नारोप्येत स्थिर‚त्वादि (ति) चेत् । ‚{\tiny $_{lb}$}‚अनुभूतानिश्चिते तु प्र‚माणान्त‚रं प्र‚व‚र्त्त‚त एवारोप‚निवृत्त‚ये । त‚च्च लिङ्गं स्व‚व्याप‚कं ‚{\tiny $_{lb}$}‚विधिरूपेण निश्चित‚त्वाद‚न्य‚स‚मारोपं निषेध‚ति । अन्य‚था स‚मारोप एव व्यावृते‚{\tiny $_{lb}$}‚र‚ल्प‚वृत्तिर्न स्यात् ।}} शुक्तौ वा शुक्ताविव ‚{\color{DodgerBlue3}‚रूप‚साध‚र्म्य‚द‚र्श‚नात्} भ्रान्ति‚{\tiny $_{lb}$}‚\leavevmode\ledsidenote{\textenglish{60b/MA}} निमित्ताद् र‚ज‚ताकार‚स्त‚दा स‚र्व्व‚था व‚स्तुनिश्च‚यात् प्र‚माणान्त‚र‚श‚ब्दान्त‚र‚वैफ‚ल्यं ‚{\tiny $_{lb}$}‚स्यात् । (४३)
	\pend% ending standard par
      \label{div_pvv.3.44}
	  
	% new div opening: depth here is 2
	

	  \pstart \leavevmode% starting standard par
	एत‚देवाह (।)
	\pend% ending standard par
      
	  \bigskip
	  \begingroup
	
	    \large
	  
	    \begin{quote}
	  
	    
	    \stanza[\smallbreak]
	\label{pv.3.44}\flagstanza{\tiny\textenglish{...v.3.44}}त‚स्माद् दृष्ट‚स्य भाव‚स्य दृष्ट एवाखिलो गुणः ।&भ्रान्तेर्न निश्च‚य इति साध‚नं संप्र‚व‚र्त्त‚ते ॥ ४४ ॥\&[\smallbreak]


	
	    \end{quote}
	  
	  \endgroup
	

	  \pstart \leavevmode% starting standard par
	\hphantom{.}‚{\color{DodgerBlue3}‚त‚स्माद्\edtext{}{\edlabel{pvv.304-2}\label{pvv.304-2}\lemma{स्माद्}\Bfootnote{अनंश‚स्यैक‚देश‚द‚र्श‚नायोगात् ।}} दृष्ट‚स्य भाव‚स्य दृष्ट एवाखिलो गुणः} ध‚र्मः । त‚थापि भ्रान्तेर्व्विप‚{\tiny $_{lb}$}‚रीताकारारोपिकाया न निश्ची\edtext{}{\edlabel{pvv.304-3}\label{pvv.304-3}\lemma{निश्ची}\Bfootnote{दृष्टोपि ।}}य‚त इत्यारोपित‚व्य‚व‚च्छेदार्थं ‚{\color{DodgerBlue3}‚साध‚नं} श‚ब्दान्त‚{\tiny $_{lb}$}‚र‚ञ्च ‚{\color{DodgerBlue3}‚संप्र‚व‚र्त्त‚ते ॥} (४४)
	\pend% ending standard par
      \label{div_pvv.3.45}
	  
	% new div opening: depth here is 2
	

	  \pstart \leavevmode% starting standard par
	न केव‚लं प्र‚त्य‚क्षाद् व‚स्तुग्र‚हे साध‚नान्त‚र‚श‚ब्दान्त‚र‚{\tiny $_{1}$}‚वैफ‚ल्यं किन्त्व‚नुमाना‚{\tiny $_{lb}$}‚द‚पीत्याह ।
	\pend% ending standard par
      
	  \bigskip
	  \begingroup
	
	    \large
	  
	    \begin{quote}
	  
	    
	    \stanza[\smallbreak]
	\label{pv.3.45}\flagstanza{\tiny\textenglish{...v.3.45}}व‚स्तुग्र‚हेनुमानाच्च ध‚र्म‚स्यैक‚स्य निश्च‚ये ।&स‚र्व‚ग्र‚हो ह्य‚पोहे तु नायं दोषः प्र‚स‚ज्य‚ते ॥ ४५ ॥\&[\smallbreak]


	
	    \end{quote}
	  
	  \endgroup
	

	  \pstart \leavevmode% starting standard par
	\hphantom{.}‚{\color{DodgerBlue3}‚व‚स्तुग्र‚हेनुमानाच्च ध‚र्म‚स्यैक‚स्य} कृत‚क‚त्वादेः प्र‚त्य‚य‚भेदाभेदित्वादिकृते ‚{\color{DodgerBlue3}‚निश्च‚ये ‚{\tiny $_{lb}$}‚स‚र्व}‚स्यानित्य‚त्वादे‚{\color{DodgerBlue3}‚र्ग्र‚हः} स्यात् । एक‚रूप‚त्वात् । कृत‚क‚त्वादिसाध‚नान्त‚र‚वैय‚र्थ्यं । ‚{\tiny $_{lb}$}‚‚{\color{DodgerBlue3}‚अपोहे\edtext{}{\edlabel{pvv.304-4}\label{pvv.304-4}\lemma{अपोहे}\Bfootnote{अपोह्य‚तेऽनेनेति बाह्य‚त‚या आरोपित आकारः । अपोह्य‚तेस्मिन् स्व‚ल‚{\tiny $_{lb}$}‚क्ष‚ण‚म्वाऽपोहः ।}} त्व}‚न्य‚व्य‚व‚च्छेदे श‚ब्द‚लिङ्ग‚विष‚ये स्वीक्रिय‚माणे ‚{\color{DodgerBlue3}‚नायं} प्र‚माणान्त‚रादिवैफ‚ल्य‚{\tiny $_{lb}$}‚\leavevmode\ledsidenote{\textenglish{305/s}} ‚{\color{DodgerBlue3}‚षः प्र‚स‚ज्य‚ते}\edtext{\textsuperscript{*}}{\edlabel{pvv.305-1}\label{pvv.305-1}\lemma{*}\Bfootnote{न विक‚ल्पानां स्व‚रूपेण बाह्यो ग्राह्योपि तु स्वाकारेण स‚हैकीकृत एव ‚{\tiny $_{lb}$}‚बाह्यो विष‚यः स चास‚त्योऽपोह्य‚तेऽनेनान्य‚दित्य‚पोहः ।}} । न ख‚लु ‚{\color{DodgerBlue3}‚प्र‚त्य‚ग्र‚दृष्टं व‚स्त्व‚न्त‚रं श‚ब्द‚लिङ्गा‚{\tiny $_{2}$}‚भ्यां विष‚यीक्रिय‚ते\edtext{}{\edlabel{pvv.305-2}\label{pvv.305-2}\lemma{ते}\Bfootnote{अक‚स्माद् धूमाद‚ग्निप्र‚तिप‚त्ताव‚पि विप‚र्यासोस्ति । अग्निम‚न्तं प्र‚देश‚म‚न‚ग्नि‚{\tiny $_{lb}$}‚म‚त्वेन (धूम‚द‚र्श‚नात् प्राक् द‚र्श‚ने साग्निदेशाद‚न्योय‚मिति) ग्र‚हात् त‚स्मान्न ‚{\tiny $_{lb}$}‚ध‚र्म्म‚बुद्धिर्व‚स्तुग्राहिणीत्य‚पोह‚विष‚या लिङ्गाद‚नुमेय‚म‚नुस‚र‚न्न‚व‚श्यं संश‚यितो विप‚र्य‚स्तो ‚{\tiny $_{lb}$}‚वा स्यात् ।}}} किन्त्व‚न्य‚व्य‚व‚च्छेदः । एक‚स्मिँश्च व्य‚व‚च्छेदे सिद्धेप्य‚सिद्धं व्य‚व‚च्छेदान्त‚रं लिङ्गा‚{\tiny $_{lb}$}‚न्त‚र‚तः श‚ब्दान्त‚र‚त‚श्च साध्य‚त इति न क‚श्चिद् दोषः ॥ (४५)
	\pend% ending standard par
      \label{div_pvv.3.46}
	  
	% new div opening: depth here is 2
	
	  \bigskip
	  \begingroup
	
	    \large
	  
	    \begin{quote}
	  
	    
	    \stanza[\smallbreak]
	\label{pv.3.46}\flagstanza{\tiny\textenglish{...v.3.46}}त‚स्माद‚पोह‚विष‚यं लिङ्ग‚मिति प्र‚कीर्तित‚म् ।&अन्य‚था ध‚र्मिणः सिद्धौ किम‚तः साध‚कं प‚र‚म् ॥ ४६ ॥\&[\smallbreak]


	
	    \end{quote}
	  
	  \endgroup
	

	  \pstart \leavevmode% starting standard par
	\hphantom{.}‚{\color{DodgerBlue3}‚त‚स्मा}‚दाचार्येणा‚{\color{DodgerBlue3}‚पोह‚विष‚यं लिङ्ग‚मि\edtext{}{\edlabel{pvv.305-3}\label{pvv.305-3}\lemma{मि}\Bfootnote{उक्तेन प्र‚कारेण न साक्षात् ।}}ति प्र‚कीर्तितं} । लिङ्ग‚मुप‚ल‚क्ष‚णं श‚ब्द‚श्च । ‚{\tiny $_{lb}$}‚‚{\color{DodgerBlue3}‚अन्य‚था} य‚दि व‚स्तुविष‚यं\edtext{}{\edlabel{pvv.305-4}\label{pvv.305-4}\lemma{यं}\Bfootnote{नापोह‚विष‚यं ।}} लिङ्गं त‚दा ‚{\color{DodgerBlue3}‚ध‚र्म्मिणः} स‚र्वात्म‚ना प्र‚त्य‚क्ष‚तः ‚{\color{DodgerBlue3}‚सिद्धौ किम‚तो} ध‚र्मिणः ‚{\color{DodgerBlue3}‚प‚र‚म}‚सिद्ध‚म‚स्ति य‚त्प्र‚{\color{DodgerBlue3}‚साध‚कं} लिङ्ग‚म्प्र‚{\tiny $_{3}$}‚माणं स्यात् ॥ (४६)
	\pend% ending standard par
      \label{div_pvv.3.47}
	  
	% new div opening: depth here is 2
	
	  \bigskip
	  \begingroup
	
	    \large
	  
	    \begin{quote}
	  
	    
	    \stanza[\smallbreak]
	\label{pv.3.47}\flagstanza{\tiny\textenglish{...v.3.47}}क्व‚चित् सामान्य‚विष‚यं दृष्टे ज्ञान‚म‚लिङ्ग‚ज‚म् ।&क‚थ‚म‚न्यापोह‚विष‚यं त‚न्मात्रापोह‚गोच‚रः ॥ ४७ ॥\&[\smallbreak]


	
	    \end{quote}
	  
	  \endgroup
	

	  \pstart \leavevmode% starting standard par
	\hphantom{.}न‚नु ‚{\color{DodgerBlue3}‚क्व‚चि}‚न्नीलादाव‚स‚मारोपितोऽन्यो विप‚रीतांशो य‚स्मिन् । त‚स्मिन् प्र‚त्य‚क्षेण ‚{\tiny $_{lb}$}‚‚{\color{DodgerBlue3}‚दृष्टे य‚ज्ज्ञान‚म‚लिङ्ग‚जं} विक‚ल्प‚कं ‚{\color{DodgerBlue3}‚सामान्य‚विष‚यं} भ‚व‚ति । त‚दारोपाभावात् । क‚थ‚{\tiny $_{lb}$}‚म‚न्यापोह‚विष‚यं । आह\edtext{}{\edlabel{pvv.305-5}\label{pvv.305-5}\lemma{आह}\Bfootnote{स्व‚ल‚क्ष‚णं स्वाकाराभेदेन गृह्ण‚त् प्र‚त्य‚क्ष‚व्यापार‚मात्म‚न्यारोप‚य‚ति । ‚{\tiny $_{lb}$}‚बाह्याध्य‚व‚साय एव योज्य‚ते तेन नील‚मिति नील‚स‚जातीये विहितं स‚र्व्व‚म‚न्य‚द् ‚{\tiny $_{lb}$}‚व्य‚व‚च्छिन‚त्ति । रूप‚निश्च‚येपि क्ष‚णि (क‚त्वा) निश्च‚यात् ।}} ‚{\color{DodgerBlue3}‚त‚न्मात्र}‚स्यानील‚मात्र‚स्या‚{\color{DodgerBlue3}‚पोहो} विजातीयाद् व्यावृत्ति‚{\tiny $_{lb}$}‚र्व्य‚व‚च्छेदः स ‚{\color{DodgerBlue3}‚गोच‚रो} य‚स्य त‚त्त‚था । नील‚विक‚ल्प‚स्यानील‚व्य‚व‚च्छेद एव विष‚य ‚{\tiny $_{lb}$}‚इत्य‚र्थः ॥ (४७)
	\pend% ending standard par
      \label{div_pvv.3.48}
	  
	% new div opening: depth here is 2
	

	  \pstart \leavevmode% starting standard par
	क‚स्मा‚{\tiny $_{4}$}‚देव‚मित्याह (।)
	\pend% ending standard par
      
	  \bigskip
	  \begingroup
	
	    \large
	  
	    \begin{quote}
	  
	    
	    \stanza[\smallbreak]
	\label{pv.3.48}\flagstanza{\tiny\textenglish{...v.3.48}}निश्च‚यारोप‚म‚न‚सोर्बाध्य‚बाध‚क‚भाव‚तः ।&स‚मारोप‚विवेकेऽस्य प्र‚वृत्तिरिति ग‚म्य‚ते ॥ ४८ ॥\&[\smallbreak]


	
	    \end{quote}
	  
	  \endgroup
	\textsuperscript{\textenglish{306/s}}

	  \pstart \leavevmode% starting standard par
	\hphantom{.}‚{\color{DodgerBlue3}‚निश्च‚यारो\edtext{}{\edlabel{pvv.306-1}\label{pvv.306-1}\lemma{यारो}\Bfootnote{प्र‚त्य‚क्ष‚पृष्ठ‚भाविन्यारोपि ।}}प‚म‚न‚सोर्बाष्य‚बाध‚क‚भाव‚तः ।} स‚मारोप‚स्य ‚{\color{DodgerBlue3}‚विवेके} व्य‚व‚च्छेदेऽस्य ‚{\tiny $_{lb}$}‚निश्च‚य‚स्य ‚{\color{DodgerBlue3}‚प्र‚वृत्तिरिति ग‚म्य‚ते} ॥ (४८)
	\pend% ending standard par
      \label{div_pvv.3.49}
	  
	% new div opening: depth here is 2
	

	  \pstart \leavevmode% starting standard par
	त‚स्माद् (।)
	\pend% ending standard par
      
	  \bigskip
	  \begingroup
	
	    \large
	  
	    \begin{quote}
	  
	    
	    \stanza[\smallbreak]
	\label{pv.3.49}\flagstanza{\tiny\textenglish{...v.3.49}}याव‚न्तोंऽश‚स‚मारोपा निश्च‚यास्त‚न्निरास‚तः ।&ताव‚न्त एव श‚ब्दा भिन्न‚व्य‚व‚च्छेद‚गोच‚राः ॥ ४९ ॥\&[\smallbreak]


	
	    \end{quote}
	  
	  \endgroup
	

	  \pstart \leavevmode% starting standard par
	\hphantom{.}‚{\color{DodgerBlue3}‚याव‚न्तोऽ}‚श‚स्य\edtext{}{\edlabel{pvv.306-2}\label{pvv.306-2}\lemma{स्य}\Bfootnote{य‚तो व‚स्त्व‚ध्य‚व‚सायेन श‚ब्दादेर्वुत्तः न व‚स्तुस्व‚रूप‚ग्र‚हेण त‚तः ।}} ध‚र्म‚स्य ‚{\color{DodgerBlue3}‚स‚मारोपास्त‚स्य निरास}‚निमित्तं वि‚{\color{DodgerBlue3}‚निश्च‚याः} श‚ब्दाश्च ‚{\tiny $_{lb}$}‚‚{\color{DodgerBlue3}‚ताव‚न्त} एव (।) तेन ते निश्च‚याः श‚ब्दाश्च ‚{\color{DodgerBlue3}‚भिन्न‚व्य‚व‚च्छेद‚गोच‚राः} ॥ (४९)
	\pend% ending standard par
      \label{div_pvv.3.50}
	  
	% new div opening: depth here is 2
	
	  \bigskip
	  \begingroup
	
	    \large
	  
	    \begin{quote}
	  
	    
	    \stanza[\smallbreak]
	\label{pv.3.50}\flagstanza{\tiny\textenglish{...v.3.50}}अन्य‚था व‚स्तु विष‚यीकुरुत एकेन वा धिया ।&एक‚त्र नान्यो विष‚योस्तीति स्यात् प‚र्याय‚ता ॥ ५० ॥\&[\smallbreak]


	
	    \end{quote}
	  
	  \endgroup
	

	  \pstart \leavevmode% starting standard par
	\hphantom{.}‚{\color{DodgerBlue3}‚अन्य‚था} य‚दि श‚ब्द‚निश्च‚यौ स‚र्वात्म‚ना व‚स्तु विष‚यीकुरुतः त‚दे‚{\tiny $_{5}$}‚केन श‚ब्देन ‚{\tiny $_{lb}$}‚‚{\color{DodgerBlue3}‚बुद्ध्या वा} विक‚ल्पिक‚या व्याप्ते स‚र्वात्म‚ना विष‚यीकृते व‚स्तु‚{\color{DodgerBlue3}‚न्येक‚त्र नान्यो}‚ऽप्र‚तिप‚न्नो ‚{\tiny $_{lb}$}‚‚{\color{DodgerBlue3}‚विष}‚यो‚{\color{DodgerBlue3}‚स्ती}‚ति ‚{\color{DodgerBlue3}‚प‚र्याय‚ता} श‚ब्दानां ‚{\color{DodgerBlue3}‚स्यात्} विक‚ल्पानाञ्चैक‚विष‚य‚ता भ‚वेत् ॥ (५०)
	\pend% ending standard par
      \label{div_pvv.3.51}
	  
	% new div opening: depth here is 2
	

	  \pstart \leavevmode% starting standard par
	न‚नु द्र‚व्यादुपाध‚यः प‚र‚स्प‚र‚ञ्च भिन्नास्त‚न्निमित्ता विक‚ल्पाः श‚ब्दाश्च तेषु ‚{\tiny $_{lb}$}‚त‚दाधारे वा द्र‚व्ये व‚र्त्त‚न्त इत्याह ।
	\pend% ending standard par
      
	  \bigskip
	  \begingroup
	
	    \large
	  
	    \begin{quote}
	  
	    
	    \stanza[\smallbreak]
	\label{pv.3.51}\flagstanza{\tiny\textenglish{...v.3.51}}नानोपाधिविशिष्ट‚स्य भेदिनोर्थ‚स्य ग्राहिका ।&उप‚काराङ्ग‚श‚क्तिभ्योऽभिन्नात्म‚निश्च‚य‚ग्र‚हे ॥ ५१ ॥\&[\smallbreak]


	
	    \end{quote}
	  
	  \endgroup
	

	  \pstart \leavevmode% starting standard par
	\hphantom{.}‚{\color{DodgerBlue3}‚य‚स्यापि नै या यि का दे}\edtext{\textsuperscript{*}}{\edlabel{pvv.306-3}\label{pvv.306-3}\lemma{*}\Bfootnote{वैशेषिकादेः भिन्नं पार‚मार्थिकं ध‚र्म‚ध‚र्मित्वं निन्द‚ति ।}}र्म‚ते ‚{\color{DodgerBlue3}‚नानोपाधे}‚र्द्र‚व्य‚त्वाद्य‚नेक‚ध‚र्म‚{\color{DodgerBlue3}‚विशिष्ठ}\edtext{}{\edlabel{pvv.306-4}\label{pvv.306-4}\lemma{र्म}\Bfootnote{घ‚टादेः ।}} स्यात ‚{\tiny $_{lb}$}‚\leavevmode\ledsidenote{\textenglish{61a/MA}} एवोपाघि\edtext{}{\edlabel{pvv.306-5}\label{pvv.306-5}\lemma{एवोपाघि}\Bfootnote{उपाधिषु वा श‚ब्द‚धियौ व‚र्त्तेते इति न प‚र्याय‚ताप्र‚स‚ङ्गः ।}} भेदाद् ‚{\color{DodgerBlue3}‚भेदिनोर्थ‚स्य‚{\tiny $_{6}$}‚\edtext{}{\edlabel{pvv.306-6}\label{pvv.306-6}\lemma{स्य}\Bfootnote{नानाभेद‚योगात् स्व‚य‚म‚भिन्न‚स्य ।}} ग्राहिका} विधिरूपेण\edtext{}{\edlabel{pvv.306-7}\label{pvv.306-7}\lemma{विधिरूपेण}\Bfootnote{प्र‚त्युपाधिभिन्ना ।}} धीः (।) त‚स्यापि\edtext{}{\edlabel{pvv.306-8}\label{pvv.306-8}\lemma{स्यापि}\Bfootnote{दूष‚ण‚माश‚ङ्क्य प‚रिह‚र‚ति}} म‚तेन ‚{\tiny $_{lb}$}‚नानाप्र‚काराणामुपाधी‚{\color{DodgerBlue3}‚नामुप‚कार‚स्याङ्गं\edtext{}{\edlabel{pvv.306-9}\label{pvv.306-9}\lemma{स्याङ्गं}\Bfootnote{कार‚णं ।}}} याः ‚{\color{DodgerBlue3}‚श‚क्त}‚य‚स्ताभ्यो‚{\color{DodgerBlue3}‚ऽभिन्नात्म‚न} उपाधिम‚तो ‚{\tiny $_{lb}$}‚द्र‚व्य‚स्य ‚{\color{DodgerBlue3}‚निश्च‚य}‚ज्ञानेन ‚{\color{DodgerBlue3}‚ग्र‚हे} ॥ (५१)
	\pend% ending standard par
      \textsuperscript{\textenglish{307/s}}\label{div_pvv.3.52}
	  
	% new div opening: depth here is 2
	
	  \bigskip
	  \begingroup
	
	    \large
	  
	    \begin{quote}
	  
	    
	    \stanza[\smallbreak]
	\label{pv.3.52}\flagstanza{\tiny\textenglish{...v.3.52}}स‚र्वात्म‚ना कृते ग्राहे को भेदः स्याद‚निश्चितः ।&त‚योरात्म‚नि स‚म्ब‚न्धादेक‚ज्ञाने द्व‚य‚ग्र‚हः ॥ ५२ ॥\&[\smallbreak]


	
	    \end{quote}
	  
	  \endgroup
	

	  \pstart \leavevmode% starting standard par
	\hphantom{.}‚{\color{DodgerBlue3}‚स‚र्व्वात्म‚ना} कृते स‚त्युप‚कार्य‚स्योपाधिक‚लाप‚स्य म‚ध्ये को ‚{\color{DodgerBlue3}‚भेद} उपाधिविशेषः ‚{\tiny $_{lb}$}‚‚{\color{DodgerBlue3}‚स्याद‚निश्चितः ।} य‚था द्र‚व्य‚मेक‚मुपाधिमुप‚क‚रोति । त‚था प‚रान‚पि (।) त‚त एको‚{\tiny $_{lb}$}‚पाध्युप‚कार‚क‚त्वे विधिरूपेण गृह्य‚माणे\edtext{}{\edlabel{pvv.307-1}\label{pvv.307-1}\lemma{माणे}\Bfootnote{स‚ति स‚म्ब‚न्धाद् द्व‚य‚ग्र‚ह उपाध्युपाधिम‚तोः}} स‚र्व्वोपाध्युप‚का‚{\tiny $_{1}$}‚र‚क‚त्वं गृह्येत । त‚था च ‚{\tiny $_{lb}$}‚स‚र्व्वोपाधिग्र‚ह‚ण‚प्र‚स‚ङ्गः । न हि य‚त्सापेक्षं य‚द्रूपं त‚द\edtext{}{\edlabel{pvv.307-2}\label{pvv.307-2}\lemma{द}\Bfootnote{स्वाग्र‚ह‚णे स्वाम्य‚ग्र‚ह‚व‚त् ।}}ग्र‚ह‚णे त‚द् गृहीतुं श‚क्यं ॥
	\pend% ending standard par
      

	  \pstart \leavevmode% starting standard par
	न‚नु धूमापेक्षं व‚ह्नेः कार‚ण‚त्वं न च त‚द्ग्र‚हे धूम‚ग्र‚हः । नैत‚द् युक्तं (।) ‚{\tiny $_{lb}$}‚त‚थाहि (।)
	\pend% ending standard par
      

	  \pstart \leavevmode% starting standard par
	किं पुनः कार‚ण‚त्व‚मिष्टं व‚ह्नेर्य‚द्य‚विद्य‚मान‚त्वाद् धूमात् पूर्व्व‚काल‚भाविता । ‚{\tiny $_{lb}$}‚धूमासाहित्यं व‚र्त्त‚मान‚काल‚स‚त्ता । न त‚द् धूमापेक्षं स्व‚कार‚णात् त‚थोत्प‚त्तेः । धूम‚{\tiny $_{lb}$}‚म‚न्त‚रेणैव च भावात् । न च त‚द्ग्र‚हे धूम‚ग्र‚हो विरुद्ध‚{\tiny $_{2}$}‚त्वात् । न हि धूम‚ग्र‚ह‚णे त‚स्मात् ‚{\tiny $_{lb}$}‚प्राग्भावित्व‚स्य त‚द‚साहित्य‚स्य च ग्र‚ह‚णं । अथ धूम‚स्यायं हेतुरिति निश्च‚य‚विष‚य‚त्वं ‚{\tiny $_{lb}$}‚कार‚ण‚त्व‚मिष्ट‚न्त‚दाऽस्त्येवेदृश‚कार‚ण‚त्व‚स्य ग्र‚हे धूम‚प्र‚तीतिरुभ‚य‚प्र‚तीत्याकार‚त्वाद‚स्य ‚{\tiny $_{lb}$}‚निश्च‚य‚स्य द्व‚यान्व‚य‚व्य‚तिरेक‚ग्र‚ह‚ण‚सापेक्ष‚त्वाच्च ॥
	\pend% ending standard par
      

	  \pstart \leavevmode% starting standard par
	\hphantom{.}न‚नु च गृहीत‚त‚दुत्प‚त्तेर‚ग्निमात्र‚द‚र्श‚नात् कार‚ण‚त्व‚प्र‚तीतिर्न च त‚दा ‚{\color{DodgerBlue3}‚धूम‚{\tiny $_{lb}$}‚प्र‚तीतिः} ॥
	\pend% ending standard par
      

	  \pstart \leavevmode% starting standard par
	(सिद्धान्ती ।) न च त‚त्र प्र‚त्य‚क्षात् कार‚ण‚त्व‚स्य प्र‚तीतिः किन्त्व‚ग्निरू[8]प‚{\tiny $_{lb}$}‚मात्र‚स्य (।) अन्य‚था स‚र्व्व‚स्य त‚द्द‚र्शिनः कार‚ण‚त्व‚प्र‚तीतिप्र‚स‚ङ्गः ।\edtext{\textsuperscript{*}}{\edlabel{pvv.307-3}\label{pvv.307-3}\lemma{*}\Bfootnote{क‚थ‚न्त‚र्हि त‚त्प्र‚तीतिरित्याह ।}} किन्तु व्याप्ति‚{\tiny $_{lb}$}‚ग्र‚ह‚सापेक्षा आनुमानिकी त‚त्प्र‚तीतिः । त‚था हि गृहीत‚त‚दुत्प‚त्तेरेवेयं भ‚व‚ति नान्य‚स्य । ‚{\tiny $_{lb}$}‚त‚दुत्प‚त्तिग्र‚हे च य‚देवं रूपं त‚देत‚त्कार‚ण‚मिति गृहीता व्याप्तिर‚नुमान‚प्र‚तीतौ च धूम‚{\tiny $_{lb}$}‚कार‚णं व‚ह्निरिति व्य‚व‚च्छेद‚द्व‚य‚म‚न्योन्य‚सापेक्षं प्र‚तीय‚त एव इत्य‚लं त‚त्प्र‚स‚ङ्गेन । ‚{\tiny $_{lb}$}‚न ह्य‚नुत्खातित‚कार्यः कार‚ण‚त्व‚निश्च‚{\tiny $_{4}$}‚यः । अत्य‚न्त‚म‚भ्यासाच्च व्याप्तिग्र‚ह‚संस्कार‚स्य ‚{\tiny $_{lb}$}‚प्र‚बुद्ध‚त्वाद‚पेक्षा नास्तीत्य‚त‚त्व‚विवेकिनां नानुमान‚त्व‚प्र‚बोधः । न ह्य‚नुमान‚म‚ह‚{\tiny $_{lb}$}‚मित्युत्प‚न्न‚म‚नुमान‚मुच्य‚ते । किन्तु त्रिरूप‚लिङ्ग‚ज‚म‚नुमेय‚विष‚यं कार‚ण‚त्व‚ज्ञान‚ञ्च ‚{\tiny $_{lb}$}‚तादृश‚मेव । द्र‚व्य‚न्तु येनैव रूपेण एक‚मुपाधिमुप‚क‚रोति तेनैवाप‚रान‚पीत्येकोपोध्युप‚{\tiny $_{lb}$}‚कार‚क‚त्वे गृह्य‚माणे नानेकोपाध्युप‚कार‚क‚त्व‚स्य ग्र‚{\tiny $_{5}$}‚ह‚णात् स‚र्व्वोपाधिग्र‚ह‚ण‚प्र‚स‚ङ्गः ‚{\tiny $_{lb}$}‚प‚र‚स्प‚र‚सापेक्ष‚त्वात् । त‚त‚श्च ‚{\color{DodgerBlue3}‚त‚यो}‚रुपाधिक‚लाप‚त‚दुप‚कार‚क‚त्व‚यो‚{\color{DodgerBlue3}‚रात्म‚नि स‚म्ब‚न्धात्} । ‚{\tiny $_{lb}$}‚\leavevmode\ledsidenote{\textenglish{308/s}} द्र‚व्य‚स्य स‚म्ब‚न्धा‚{\color{DodgerBlue3}‚देक}‚स्योपाध्युप‚कार‚क‚त्व‚स्य ‚{\color{DodgerBlue3}‚ज्ञाने द्व‚य‚स्य ग्र‚हः} स्यात् । एक‚निय‚ता ‚{\tiny $_{lb}$}‚प्र‚तीतिर्न भ‚वेदित्य‚र्थः ॥ (५२)
	\pend% ending standard par
      \label{div_pvv.3.53}
	  
	% new div opening: depth here is 2
	

	  \pstart \leavevmode% starting standard par
	अथ द्र‚व्यादुपाध्युप‚कारिकाः श‚क्त‚यो भिन्ना एव द्र‚व्य‚ग्र‚ह‚णे तासाम‚ग्र‚ह‚णात् ‚{\tiny $_{lb}$}‚न स‚र्व्वोपाधिग्र‚ह‚ण‚प्र‚स‚ङ्ग इत्याह ।
	\pend% ending standard par
      
	  \bigskip
	  \begingroup
	
	    \large
	  
	    \begin{quote}
	  
	    
	    \stanza[\smallbreak]
	\label{pv.3.53}\flagstanza{\tiny\textenglish{...v.3.53}}ध‚र्मोप‚कार‚श‚क्तीनां भेदे ताः त‚स्य किं य‚दि ।&नोप‚कार‚स्त‚तः तासां त‚था स्याद‚न‚व‚स्थितिः ॥ ५३ ॥\&[\smallbreak]


	
	    \end{quote}
	  
	  \endgroup
	

	  \pstart \leavevmode% starting standard par
	\hphantom{.}‚{\color{DodgerBlue3}‚ध‚र्मा}‚णामुपाधीना‚{\tiny $_{6}$}‚ ‚{\color{DodgerBlue3}‚उप‚कार}‚स्य निमित्त‚भूतानां ‚{\color{DodgerBlue3}‚श‚क्ती}‚नां द्र‚व्याद् ‚{\color{DodgerBlue3}‚भेदे} स्वीक्रिय‚{\tiny $_{lb}$}‚माणे ‚{\color{DodgerBlue3}‚ताः} श‚क्त‚य‚{\color{DodgerBlue3}‚स्त‚स्य} द्र‚व्य‚स्य ‚{\color{DodgerBlue3}‚किं} क‚स्मात् (।) य‚दि ‚{\color{DodgerBlue3}‚नोप‚कार‚स्त‚तो} द्र‚व्यात् ‚{\tiny $_{lb}$}‚‚{\color{DodgerBlue3}‚तासां} श‚क्तीनामुप‚कार‚म‚न्त‚रेण स‚म्ब‚न्धेऽतिप्र‚स‚ङ्गात् (।) अथ तासां श‚क्तीनां ‚{\tiny $_{lb}$}‚द्र‚व्येणोप‚कारः क्रिय‚ते स्व‚रूपेण त‚दा श‚क्त्युप‚कार‚त्व‚स्य ग्र‚ह‚णाच्छ‚क्तीनां ग्र‚हः । ‚{\tiny $_{lb}$}‚त‚द्ग्र‚हाच्च स‚र्व्वोपाधिग्र‚ह‚प्र‚स‚ङ्गः त‚द‚व‚स्थः ।
	\pend% ending standard par
      \textsuperscript{\textenglish{61b/MA}}

	  \pstart \leavevmode% starting standard par
	स्यादेत‚त् (।) श‚क्तीर‚पि भि‚{\tiny $_{7}$}‚न्नाभिःश‚क्तिभिरुप‚क‚रोति ‚{\color{DodgerBlue3}‚त‚था स्याद‚न‚व‚स्थितिः} । ‚{\tiny $_{lb}$}‚त‚था हि यास्ताः श‚क्त्युप‚कारिकाः श‚क्त्य‚प‚स्ता द्र‚व्य‚स्योप‚कार‚द्वारा\edtext{}{\edlabel{pvv.308-1}\label{pvv.308-1}\lemma{द्वारा}\Bfootnote{य‚दोपाधिष्वेव श‚ब्दादिवृत्तिस्त‚दा नोक्तो दोषः किन्तु श‚ब्दाद्यैर‚नाक्षे‚{\tiny $_{lb}$}‚पादुपाधिम‚ति प्र‚वृत्तिर्न स्यादिति व्य‚र्थः श‚ब्द‚प्र‚योगः । अर्थ‚क्रियाश्र‚यो हि व्य‚व‚हार ‚{\tiny $_{lb}$}‚उपाध (य) श्चात्र व्य‚व‚हारेऽस‚म‚र्थाः (।) स‚म‚र्थ‚श्चोपाधिमान्नोच्य‚ते (।) किञ्च (।) ‚{\tiny $_{lb}$}‚निश्च‚य‚गृहीतेप्य‚र्थे भ्रान्तिनिवृत्य‚र्थं प्र‚माणान्त‚र‚मिच्छ‚ता निश्च‚य‚विष‚य‚श्च न च ‚{\tiny $_{lb}$}‚निश्चित इत्य‚भ्युपेतं स्यात् अन्य‚था भ्रान्त्य‚योगात् (।) त‚च्चायुक्त‚मित्याह ।}}द् (?) य‚दीष्टा‚{\tiny $_{lb}$}‚स्त‚दास्व‚रूपेणोप‚कार‚क‚त्वे स‚र्व्वोपाधिग्र‚ह‚प्र‚स‚ङ्ग‚भ‚याद‚न्याः श‚क्त‚य एष्ट‚व्याः । त‚था ‚{\tiny $_{lb}$}‚चान‚व‚स्था व्य‚क्ता । (५३)
	\pend% ending standard par
      \label{div_pvv.3.54}
	  
	% new div opening: depth here is 2
	

	  \pstart \leavevmode% starting standard par
	उक्त‚म‚र्थं संगृह्ण‚न्नाह ।
	\pend% ending standard par
      
	  \bigskip
	  \begingroup
	
	    \large
	  
	    \begin{quote}
	  
	    
	    \stanza[\smallbreak]
	\label{pv.3.54}\flagstanza{\tiny\textenglish{...v.3.54}}एकोप‚कार‚के ग्राह्येऽदृष्टाः त‚स्मिन्न स‚न्ति ते ।&स‚र्वोप‚कार‚कं ह्येकं त‚द्ग्र‚हे स‚क‚ल‚ग्र‚हः ॥ ५४ ॥\&[\smallbreak]


	
	    \end{quote}
	  
	  \endgroup
	

	  \pstart \leavevmode% starting standard par
	\hphantom{.}‚{\color{DodgerBlue3}‚एक}‚स्योपाधे‚{\color{DodgerBlue3}‚रुप‚कार‚के} द्र‚व्ये ‚{\color{DodgerBlue3}‚ग्राह्ये}‚ऽभिम‚ते त‚त एकोपाध्युप‚कार‚क‚द्र‚व्य‚स्व‚{\tiny $_{lb}$}‚भावाद‚प‚रे उपाध्य‚न्त‚राणामुप‚कार‚का उप‚{\tiny $_{1}$}‚कार‚श‚क्तिभेदा दृष्टे ‚{\color{DodgerBlue3}‚त‚स्मिन्ने}‚कोपाध्युप‚{\tiny $_{lb}$}‚कार‚के द्र‚व्येऽदृष्टा ये ‚{\color{DodgerBlue3}‚ते न स‚न्ति} (।) एक‚मेव हि रूपं स‚र्व्वोपाध्युप‚कार‚क‚म‚त‚स्त‚{\tiny $_{lb}$}‚‚{\color{DodgerBlue3}‚स्यैको}‚पाधिम‚तो ‚{\color{DodgerBlue3}‚ग्र‚हे स‚क‚लोपाधिग्र‚हः} स्यात् ॥ (५४)
	\pend% ending standard par
      \textsuperscript{\textenglish{309/s}}\label{div_pvv.3.55}
	  
	% new div opening: depth here is 2
	

	  \begin{center}%% label @type='head'
	\textbf{क. न्याय‚मीमांसाम‚त‚निरासः}
	\end{center}
	

	  \begin{center}%% label @type='head'
	\textbf{(क) व्यावृत्त‚स्व‚भावा भावाः}
	\end{center}
	
	  \bigskip
	  \begingroup
	
	    \large
	  
	    \begin{quote}
	  
	    
	    \stanza[\smallbreak]
	\label{pv.3.55}\flagstanza{\tiny\textenglish{...v.3.55}}य‚दि भ्रान्तिनिवृत्त्य‚र्थं गृहीतेप्य‚न्य‚दिष्य‚ते ।&त‚द्व्य‚व‚च्छेद‚विष‚यं सिद्ध‚न्त‚द्व‚त् त‚तोऽप‚र‚म् ॥ ५५ ॥\&[\smallbreak]


	
	    \end{quote}
	  
	  \endgroup
	

	  \pstart \leavevmode% starting standard par
	\hphantom{.}स‚र्व्वात्म‚ना विक‚ल्पेन ‚{\color{DodgerBlue3}‚गृहीतेपि} व‚स्तुनि भ्रान्त्या त‚था न निश्च‚य इति ‚{\color{DodgerBlue3}‚य‚दि ‚{\tiny $_{lb}$}‚भ्रान्तिनिवृत्त्य‚र्थ‚म‚न्य‚त्} प्र‚माणान्त‚र‚{\color{DodgerBlue3}‚मिष्य‚ते} (।) त‚द्भ्रान्तिनिव‚र्त्त‚कं प्र‚माणान्त‚रं ‚{\tiny $_{lb}$}‚‚{\color{DodgerBlue3}‚व्य‚व‚च्छेद‚विष‚यं सिद्धं}\edtext{}{\edlabel{pvv.309-1}\label{pvv.309-1}\lemma{रं}\Bfootnote{उत्पित्सुस‚मारोप‚निषेध‚द्वारेण ।}} भ्र‚मारोपित‚त्वापो‚{\tiny $_{2}$}‚ह‚विष‚य‚त्वाद् । ‚{\color{DodgerBlue3}‚त‚तो} भ्रान्तिनिव‚र्त्त‚काद‚{\color{DodgerBlue3}‚प‚रं} य‚त् पूर्व्व‚मुत्प‚न्न‚म्व‚स्तुविष‚य‚मिष्टं ‚{\color{DodgerBlue3}‚त‚द्व}‚द‚पोह‚विष‚यं सिद्धं ॥ (५५)
	\pend% ending standard par
      \label{div_pvv.3.56}
	  
	% new div opening: depth here is 2
	

	  \pstart \leavevmode% starting standard par
	क‚स्मादित्याह (।)
	\pend% ending standard par
      
	  \bigskip
	  \begingroup
	
	    \large
	  
	    \begin{quote}
	  
	    
	    \stanza[\smallbreak]
	\label{pv.3.56}\flagstanza{\tiny\textenglish{...v.3.56}}त‚द्विप‚क्ष‚स‚मारोप‚विष‚ये य‚दि निश्च‚यैः ।&निश्चीय‚ते न य‚द् रूपं त‚त्तेषां विष‚यः क‚थ‚म् ॥ ५६ ॥\&[\smallbreak]


	
	    \end{quote}
	  
	  \endgroup
	

	  \pstart \leavevmode% starting standard par
	\hphantom{.}अविद्य‚मानान्य‚स‚मारोपे ‚{\color{DodgerBlue3}‚विष‚ये वृत्तेः} । विक‚ल्पो हि व्य‚व‚च्छेद‚विष‚यं निश्चिन्व‚न् ‚{\tiny $_{lb}$}‚त‚द्विप‚क्ष‚स‚मारोप‚विष‚ये न भ‚व‚ति । किन्तु त‚द्व्य‚व‚च्छेद‚निश्च‚यारोप‚म‚न‚सोर्ब्बाध्य‚{\tiny $_{lb}$}‚बाध‚क‚भाव‚त इत्युक्तं ।
	\pend% ending standard par
      

	  \pstart \leavevmode% starting standard par
	अपि\edtext{}{\edlabel{pvv.309-2}\label{pvv.309-2}\lemma{अपि}\Bfootnote{किञ्च ।}} च निश्च‚यैरेकाकार‚प्र‚वृत्तै\edtext{}{\edlabel{pvv.309-3}\label{pvv.309-3}\lemma{वृत्तै}\Bfootnote{य‚देषां स्व‚विष‚य‚निश्च‚य‚न‚मेव स्वार्थे वृत्तिः ।}}‚{\color{DodgerBlue3}‚र्य‚द् रूपं न निश्चीय‚ते त‚त्तेषाम्विष‚यः क}‚{\tiny $_{3}$}‚थ‚{\tiny $_{lb}$}‚मुच्य‚ते । निश्चित‚ञ्चाप्र‚तिप‚न्नं चेति विप्र‚ति‚{\color{DodgerBlue3}‚सि} (? षि) द्धं । य‚दि कृत‚क‚त्व‚निश्च‚ये‚{\tiny $_{lb}$}‚ऽनित्य‚त्वाद्य‚पि निश्चितं क‚थं त‚स्याप्र‚तिप‚त्तिः । कृत (क) त्व‚स्यापि वा माभूत् ॥ ‚{\tiny $_{lb}$}‚(५६)
	\pend% ending standard par
      \label{div_pvv.3.57}
	  
	% new div opening: depth here is 2
	

	  \pstart \leavevmode% starting standard par
	एव‚न्त‚र्हि प्र‚त्य‚क्ष‚गृहीते व‚स्तुनि निश्च‚यानिश्च‚यौ न स्यातामित्याह ।
	\pend% ending standard par
      
	  \bigskip
	  \begingroup
	
	    \large
	  
	    \begin{quote}
	  
	    
	    \stanza[\smallbreak]
	\label{pv.3.57}\flagstanza{\tiny\textenglish{...v.3.57}}प्र‚त्य‚क्षेण गृहीतेपि विशेषेंश‚विव‚र्जिते ।&य‚द्विशेषाव‚सायेस्ति प्र‚त्य‚यः स प्र‚तीय‚ते ॥ ५७ ॥\&[\smallbreak]


	
	    \end{quote}
	  
	  \endgroup
	

	  \pstart \leavevmode% starting standard par
	\hphantom{.}‚{\color{DodgerBlue3}‚प्र‚त्य‚क्षेण गृहीतेपि विशेषे} स्व‚ल‚क्ष‚णे स‚र्व्व‚तो व्यावृत्तें‚{\color{DodgerBlue3}‚ऽशै}‚र्भागै‚{\color{DodgerBlue3}‚र्व्विव‚र्जिते} य‚स्य ‚{\tiny $_{lb}$}‚‚{\color{DodgerBlue3}‚विशेष}‚स्य व्य‚व‚च्छेद‚स्या‚{\color{DodgerBlue3}‚व‚साये}‚स्ति ‚{\color{DodgerBlue3}‚प्र‚त्य‚यः} स‚ह‚कारी प्र‚क‚र‚णाभ्यास‚पाट‚वा‚{\tiny $_{4}$}‚दिः ‚{\color{DodgerBlue3}‚स} \leavevmode\ledsidenote{\textenglish{310/s}} ‚{\color{DodgerBlue3}‚प्र‚तीय‚ते} नेत‚रः । न ख‚ल्व‚स्म‚न्म‚ते प्र‚त्य‚क्षं निश्च\edtext{}{\edlabel{pvv.310-1}\label{pvv.310-1}\lemma{निश्च}\Bfootnote{क‚थ‚मिदानीम‚निश्चीय‚मानं प्र‚त्य‚क्षेणापि ग्र‚हीत‚मिति चेन्न प्र‚त्य‚क्षं निश्च‚{\tiny $_{lb}$}‚येन गृह्णाति किन्तु त‚त्प्र‚तिभासेन (।) श‚ब्द‚प्र‚तिप‚त्योर्भेद‚स्तु संकेत‚भेदान्न त‚त्त्व‚तो ‚{\tiny $_{lb}$}‚वाच्य‚भेदः । संकेत‚भेद‚श्चान‚न्त‚रं भेदान्त‚रेत्याद्युक्तः ।}}यात्म‚कं नाप्य‚नुभ‚व‚मात्राधीनो ‚{\tiny $_{lb}$}‚निश्च‚यो\edtext{}{\edlabel{pvv.310-2}\label{pvv.310-2}\lemma{यो}\Bfootnote{अनुभ‚वो हि प‚टीयान् स्मृतिबीज‚माध‚त्ते निश्च‚यः स्मृतिरूपः ।}} येन स‚र्व्व‚था निश्च‚य‚प्र‚स‚ङ्गः । किन्तु य‚त्र व्य‚व‚च्छेदे\edtext{}{\edlabel{pvv.310-3}\label{pvv.310-3}\lemma{च्छेदे}\Bfootnote{य‚थोपाध्याये पित‚र्याग‚च्छ‚ति पिता मे आग‚च्छ‚तीति त‚त्रापि तार‚त‚म्यात् ‚{\tiny $_{lb}$}‚पूर्व्व‚प‚र‚विक‚ल्प‚ज‚न‚न‚श‚क्तिः ।}}ऽभ्यासाद‚यः\edtext{}{\edlabel{pvv.310-4}\label{pvv.310-4}\lemma{यः}\Bfootnote{अथ गोर‚श्वाद‚व्यावृत्तिर्व्यावृत्तो गौरिति वा य‚दा क्रिय‚ते त‚दा व्यावृत्ति‚{\tiny $_{lb}$}‚त‚द्व‚तोर्व्व‚स्तुतो भेदे सामान्यं नामान्त‚रेणोक्तं स्यात् व्यावृत्तिरिति । अभेदेपि ज्ञान‚{\tiny $_{lb}$}‚श‚ब्द‚योर्भेदो (स्व‚ल‚क्ष‚णात्) न स्यादित्याह (।) एक‚मानेन स‚र्व‚सिद्धौ मानान्त‚रादि‚{\tiny $_{lb}$}‚वैय‚र्थ्यं त‚द‚व‚स्थं ।}}स‚ह‚{\tiny $_{lb}$}‚कारिणः प्र‚त्य‚क्ष‚स्य स‚न्ति स निश्चीय‚ते नान्य इति युक्तो विभागः । (५७)
	\pend% ending standard par
      \label{div_pvv.3.58}
	  
	% new div opening: depth here is 2
	
	  \bigskip
	  \begingroup
	
	    \large
	  
	    \begin{quote}
	  
	    
	    \stanza[\smallbreak]
	\label{pv.3.58}\flagstanza{\tiny\textenglish{...v.3.58}}त‚त्रापि चान्य‚व्यावृत्तिर‚न्य‚व्यावृत्त इत्य‚पि ।&श‚ब्दाश्च निश्च‚याश्चैव निमित्त‚म‚नुरुन्ध‚ते ॥ ५८ ॥\&[\smallbreak]


	
	    \end{quote}
	  
	  \endgroup
	

	  \pstart \leavevmode% starting standard par
	\hphantom{.}‚{\color{DodgerBlue3}‚त‚त्रापि} चान्यापोहेपि श‚ब्द‚विक‚ल्प‚विष‚ये‚{\color{DodgerBlue3}‚ऽन्य‚व्यावृत्तिर‚न्य‚व्यावृत्त इत्य‚पि (।) ‚{\tiny $_{lb}$}‚ये श‚ब्दाश्च निश्च‚याश्चैव} भिन्न‚विष‚या इव प्र‚व‚र्त‚न्ते न ते‚{\tiny $_{5}$}‚ व्यावृत्तिव्यावृत्त‚यो‚{\tiny $_{lb}$}‚र्व्वास्त‚व‚भेद‚निब‚न्ध‚नाः किन्त‚र्हि संकेत‚मेव भेद‚व्य‚व‚हार‚व्य‚स्थाप‚कं व‚क्ष्य‚माण‚{\tiny $_{lb}$}‚‚{\color{DodgerBlue3}‚निमित्त‚म‚नुरुन्ध‚ते}‚ऽनुव‚र्त्त‚न्ते । य‚दि गोर‚न्याऽगोव्यावृत्तिस्त‚दा य‚था गौर‚गोर‚श्वादे‚{\tiny $_{lb}$}‚र्व्यावृत्त‚स्त‚थाऽगोव्यावृत्तेर‚पि व्यावृत्तः । त‚त‚श्चाश्वादिव‚द् गोर‚गौरेव स्यात् । यो ‚{\tiny $_{lb}$}‚ह्य‚गोव्यावृत्तेर्व्यावृत्तः सोऽगौर्य‚थाश्वादिर्गौश्च त‚थेति स्यात् । त‚थाऽगोव्यावृत्ति[6]‚{\tiny $_{lb}$}‚र‚पि न प्राप्नोति । स‚र्व्व‚स्यैवागोत्वात् । गोर‚गोर‚न्य‚त्वे हि त‚स्माद् व्यावृत्तिः ‚{\tiny $_{lb}$}‚स्यात् । य‚दा तु गौरेव नास्ति त‚दा क‚स्य क‚स्माद् व्यावृत्तिः । त‚स्मान्न व्यावृत्ति‚{\tiny $_{lb}$}‚व्यावृत्त‚योर्व्व‚स्तुतो भेदः । (५८)
	\pend% ending standard par
      \label{div_pvv.3.59}
	  
	% new div opening: depth here is 2
	

	  \pstart \leavevmode% starting standard par
	त‚त‚श्च ॥
	\pend% ending standard par
      
	  \bigskip
	  \begingroup
	
	    \large
	  
	    \begin{quote}
	  
	    
	    \stanza[\smallbreak]
	\label{pv.3.59}\flagstanza{\tiny\textenglish{...v.3.59}}द्व‚योरेकाभिधानेपि विभ‚क्तिर्व्य‚तिरेकिणी ।&भिन्न‚म‚र्थ‚मिवान्वेति वाच्य‚ते स विशेष‚तः ॥ ५९ ॥\&[\smallbreak]


	
	    \end{quote}
	  
	  \endgroup
	

	  \pstart \leavevmode% starting standard par
	\hphantom{.}‚{\color{DodgerBlue3}‚द्व‚यो}‚र्व्यावृत्तिव्यावृत्त‚योरेक‚स्यार्थ‚स्या‚{\color{DodgerBlue3}‚भिधानेपि} व‚स्तुतो ‚{\color{DodgerBlue3}‚विभ‚क्तिः} ष‚ष्ठ्यादि‚{\tiny $_{lb}$}‚‚{\color{DodgerBlue3}‚र्व्य‚तिरेको} वाच्य‚भेदः त‚द्व‚ती । व्यावृत्त‚स्य व्यावृत्तिरिति ‚{\color{DodgerBlue3}‚भिन्न‚मिवार्थ‚म‚न्वे}‚त्य‚नु‚{\tiny $_{lb}$}‚\leavevmode\ledsidenote{\textenglish{311/s}} ग‚च्छ‚ति वाच‚क‚त्वेन । वाच्य‚स्य ले‚{\tiny $_{7}$}‚श‚विशेष‚तो\edtext{}{\edlabel{pvv.311-1}\label{pvv.311-1}\lemma{तो}\Bfootnote{श‚ब्दा हीच्छाधीना न व‚स्त्व‚धीनाः ते य‚था प्र‚योक्तुमिष्य‚न्ते भेदेऽभेदे वा ‚{\tiny $_{lb}$}‚त‚था वाच‚काः (।) य‚था राज्ञः पुरुष आत्मात्म‚नो द्र‚ष्ट्रिति ।}}ऽल्प‚भेदात् साङ्केतिकाद‚भिन्ने-\leavevmode\ledsidenote{\textenglish{62a/MA}} ‚{\tiny $_{lb}$}‚प्य‚र्थे ॥ (५९)
	\pend% ending standard par
      \label{div_pvv.3.60}
	  
	% new div opening: depth here is 2
	

	  \pstart \leavevmode% starting standard par
	किम‚र्थं संकेत‚भेदः क‚श्च वाच्य‚विशेष इत्याह ।
	\pend% ending standard par
      
	  \bigskip
	  \begingroup
	
	    \large
	  
	    \begin{quote}
	  
	    
	    \stanza[\smallbreak]
	\label{pv.3.60}\flagstanza{\tiny\textenglish{...v.3.60}}भेदान्त‚र‚प्र‚तिक्षेपाप्र‚तिक्षेपौ त‚योर्द्व‚योः ।&प‚दं स‚ङ्केत‚भेद‚स्य ज्ञातृवाञ्छानुरोधिनः ॥ ६० ॥\&[\smallbreak]


	
	    \end{quote}
	  
	  \endgroup
	

	  \pstart \leavevmode% starting standard par
	\hphantom{.}‚{\color{DodgerBlue3}‚भेदान्त}‚र‚स्य\edtext{}{\edlabel{pvv.311-2}\label{pvv.311-2}\lemma{स्य}\Bfootnote{गोत्वापेक्ष‚या द्र‚व्य‚त्व‚पार्थिव‚त्वादेः ।}} प्र‚तिपाद्य‚मानाद् व्य‚व‚च्छेदाद‚न्य‚स्य व्य‚व‚च्छेद‚स्य ‚{\color{DodgerBlue3}‚प्र‚तिक्षेपः\edtext{}{\edlabel{pvv.311-3}\label{pvv.311-3}\lemma{तिक्षेपः}\Bfootnote{अस्वीकारः ।}}} । ‚{\tiny $_{lb}$}‚सामानाधिक‚र‚ण्य‚स्य स्व‚भावोऽ‚{\color{DodgerBlue3}‚प्र‚तिक्षेप}‚स्त‚त्स‚म्भ‚वः । ‚{\color{DodgerBlue3}‚तौ\edtext{}{\edlabel{pvv.311-4}\label{pvv.311-4}\lemma{तौ}\Bfootnote{य‚थाक्र‚मं ।}} त‚योर्द्व‚यो}‚र्व्यावृत्ति‚{\tiny $_{lb}$}‚व्यावृत्त\edtext{}{\edlabel{pvv.311-5}\label{pvv.311-5}\lemma{व्यावृत्त}\Bfootnote{ध‚र्म‚ध‚र्मिवाचिनोः ।}}श‚ब्द‚योः\edtext{}{\edlabel{pvv.311-6}\label{pvv.311-6}\lemma{योः}\Bfootnote{प्र‚योज‚नं ।}} ‚{\color{DodgerBlue3}‚संकेत‚भेद‚स्य ज्ञातृवाञ्छानुरोधिनः} प‚दं कार‚णं (।) ज्ञाता\edtext{}{\edlabel{pvv.311-7}\label{pvv.311-7}\lemma{ज्ञाता}\Bfootnote{श्रोता ।}} हि ‚{\tiny $_{lb}$}‚क‚दाचित् गौर‚न‚श्व‚त्वं निष्कृष्ट‚ध‚र्मान्त‚र\edtext{}{\edlabel{pvv.311-8}\label{pvv.311-8}\lemma{र}\Bfootnote{(अम‚हिष‚त्वादि) किम‚स्याश्वाद् व्यावृतं रूप‚म‚स्तीति ।}} स‚म्ब‚न्ध‚योग्य‚त्वं जिज्ञास‚ते\edtext{}{\edlabel{pvv.311-9}\label{pvv.311-9}\lemma{ते}\Bfootnote{न सामानाधिक‚र‚ण्य‚म्बिशेष‚ण‚विशेष्य‚भावो वा त्य‚क्त्वा भेदान्त‚र‚त्वा‚{\tiny $_{lb}$}‚देव गोत्व‚शुक्ल‚त्वाभ्यां युक्त‚मेकं ध‚र्मिणं गृहीत्वा बुद्धेर‚प्र‚तिभास‚नात् ।}} (।) त‚दा ‚{\tiny $_{lb}$}‚गो‚{\tiny $_{1}$}‚त्व‚म‚स्येत्युच्य‚ते न तु गोत्व‚म‚स्य शुक्ल‚मिति । य‚दा तु व्यावृत्ति\edtext{}{\edlabel{pvv.311-10}\label{pvv.311-10}\lemma{व्यावृत्ति}\Bfootnote{कुतोस्य व्यावृत्तिरिति त‚दा भेदान्त‚राक्षेपात् त‚त्साकांक्ष‚त्वाच्च ।}}रेवानिष्कृष्ट‚{\tiny $_{lb}$}‚ध‚र्मान्त‚र‚स‚म्ब‚न्ध‚योग्या जिज्ञासिता भ‚व‚ति (।) त‚दा व्यावृत्त‚श‚ब्द‚संकेतो य‚था गौर‚य‚{\tiny $_{lb}$}‚मिति ध‚र्म्मान्त‚र‚सामानाधिक‚र‚ण्य‚ञ्च गौः\edtext{}{\edlabel{pvv.311-11}\label{pvv.311-11}\lemma{गौः}\Bfootnote{अनेक‚ध‚र्म‚व‚न्तं ध‚र्मिण‚मेक‚मिव द‚र्श‚य‚न्ती बुद्धिर‚त्र य‚स्मात् ।}}शुक्ल इत्यादि\edtext{}{\edlabel{pvv.311-12}\label{pvv.311-12}\lemma{इत्यादि}\Bfootnote{अयं नीलादि ।}}॥ (६०)
	\pend% ending standard par
      \label{div_pvv.3.61}
	  
	% new div opening: depth here is 2
	
	  \bigskip
	  \begingroup
	
	    \large
	  
	    \begin{quote}
	  
	    
	    \stanza[\smallbreak]
	\label{pv.3.61}\flagstanza{\tiny\textenglish{...v.3.61}}भेदोय‚मेव स‚र्व‚त्र द्र‚व्य‚भावाभिधायिनोः ।&श‚ब्द‚योर्न त‚योर्वाच्ये विशेष‚स्तेन क‚श्च‚न ॥ ६१ ॥\&[\smallbreak]


	
	    \end{quote}
	  
	  \endgroup
	

	  \pstart \leavevmode% starting standard par
	\hphantom{.}‚{\color{DodgerBlue3}‚भेदोय‚मेव} संकेत‚कृतो ध‚र्मान्त‚र\edtext{}{\edlabel{pvv.311-13}\label{pvv.311-13}\lemma{र}\Bfootnote{नाप‚रं सामान्य‚गुणादिकं बाधित‚त्वात् ।}}प्र‚तिक्षेपाप्र‚तिक्षेप‚प्र‚तिप‚त्तिफ‚लो ‚{\color{DodgerBlue3}‚द्र‚व्य\edtext{}{\edlabel{pvv.311-14}\label{pvv.311-14}\lemma{व्य}\Bfootnote{स‚र्व्व‚त्र सामान्य‚त‚द्व‚ति गुण‚त‚द्व‚ति क्रियात‚द्व‚ति ।}}भावा‚{\tiny $_{lb}$}‚भिधायिनो}‚र्द्ध‚र्मिध‚र्म‚वाचिनोः ‚{\color{DodgerBlue3}‚श‚ब्द‚योर्न} वास्त‚वः । ‚{\color{DodgerBlue3}‚तेन त‚योर्व्वाच्ये} निश्च‚य‚विष‚ये ‚{\tiny $_{lb}$}‚‚{\color{DodgerBlue3}‚न क}‚श्चिद् ‚{\color{DodgerBlue3}‚विशे‚{\tiny $_{2}$}‚षः} । भेदान्त‚र‚प्र‚तिक्षेपाप्र‚तिक्षेपाभ्यामेक‚स्यैव प्र‚त्याय‚नात् ॥ (६१)
	\pend% ending standard par
      \textsuperscript{\textenglish{312/s}}\label{div_pvv.3.62}
	  
	% new div opening: depth here is 2
	
	  \bigskip
	  \begingroup
	
	    \large
	  
	    \begin{quote}
	  
	    
	    \stanza[\smallbreak]
	\label{pv.3.62}\flagstanza{\tiny\textenglish{...v.3.62}}जिज्ञाप‚यिषुर‚र्थं तं त‚द्धितेन कृतापि वा ।&अन्येन वा य‚दि ब्रूयात् भेदो नास्ति त‚तः प‚रः ॥ ६२ ॥\&[\smallbreak]


	
	    \end{quote}
	  
	  \endgroup
	

	  \pstart \leavevmode% starting standard par
	\hphantom{.}त‚था व्य‚व‚ह‚र्त्ता ‚{\color{DodgerBlue3}‚जिज्ञाप‚यिषुर‚र्थं तं} साङ्केतिकं भेदं ‚{\color{DodgerBlue3}‚त‚द्धितेन कृतापि वा\edtext{}{\edlabel{pvv.312-1}\label{pvv.312-1}\lemma{वा}\Bfootnote{जातिगुण‚क्रियास‚म्ब‚न्धैश्च‚तुष्ट‚यी वृत्तिरुक्तानेनैव ।}}} पाच‚क‚त्व‚म‚स्य पाच‚कोय‚म्पाकः पाक्यो वेत्यादि । ‚{\color{DodgerBlue3}‚अन्येन वा} स्व‚यंकृतेन स‚म‚येन ‚{\tiny $_{lb}$}‚‚{\color{DodgerBlue3}‚य‚दि ब्रूयात्} (।) त‚थापि ‚{\color{DodgerBlue3}‚त‚तो} भेद‚प्र‚तिपाद‚कात् त‚द्धितादे‚{\color{DodgerBlue3}‚र्भेदः} प‚रो वास्त‚वो ‚{\tiny $_{lb}$}‚‚{\color{DodgerBlue3}‚नास्ति}‚॥ (६२)
	\pend% ending standard par
      \label{div_pvv.3.63}
	  
	% new div opening: depth here is 2
	
	  \bigskip
	  \begingroup
	
	    \large
	  
	    \begin{quote}
	  
	    
	    \stanza[\smallbreak]
	\label{pv.3.63}\flagstanza{\tiny\textenglish{...६३}}तेनान्यापोह‚विष‚ये त‚द्दोषोप‚व‚र्ण्ण‚न‚म् ।\edtext{\textsuperscript{*}}{\edlabel{pvv.312-asterisk}\label{pvv.312-asterisk}\lemma{*}\Bfootnote{त‚द्व‚त्प‚क्षोप‚व‚र्ण्ण‚न‚म् ।}}&प्र‚त्याख्यातं पृथ‚क्त्वे हि स्याद् दोषो जातित‚द्व‚तोः ॥  ॥\&[\smallbreak]


	
	    \end{quote}
	  
	  \endgroup
	

	  \pstart \leavevmode% starting standard par
	\hphantom{.}‚{\color{DodgerBlue3}‚तेन} व्यावृत्तिव्यावृत्त‚योर‚भेदे‚{\color{DodgerBlue3}‚नान्यापोह‚विष‚ये} जातिमान् श‚ब्द‚वाच्य इति‚{\tiny $_{3}$}‚ ‚{\tiny $_{lb}$}‚प‚क्षः त‚द्व‚त् प‚क्ष‚भूर्दोषोऽन्यापोहेपि स्यादिति । ‚{\color{DodgerBlue3}‚त‚द्दोषोप‚व‚र्ण्ण‚नं प्र‚त्याख्यातं} । त‚द्व‚त् ‚{\tiny $_{lb}$}‚प‚क्षो हि जातिम‚भिधाय श‚ब्द‚स्त‚द्व‚ति व‚र्त्त‚त इति त‚द्व‚च‚ने\edtext{}{\edlabel{pvv.312-2}\label{pvv.312-2}\lemma{ने}\Bfootnote{जाति ।}} स्वात‚न्त्र्य‚म\edtext{}{\edlabel{pvv.312-3}\label{pvv.312-3}\lemma{म}\Bfootnote{श‚ब्द‚स्य ।}}स्य ‚{\tiny $_{lb}$}‚न स्यात् सामानाधिक‚र‚ण्य‚ञ्च न भ‚वेत् (।) गौः शुक्ल इति जातेर‚शुक्ल‚त्वात् (।) ‚{\tiny $_{lb}$}‚न चोप‚चाराश्र‚येण स्वात‚न्त्र्यं सामान्याधिक‚र‚ण्य‚ञ्चास्ख‚ल‚द्ग‚तियुक्त‚मित्यादि ‚{\tiny $_{lb}$}‚‚{\color{DodgerBlue3}‚जातित‚द्व‚तोः पृथ‚क्त्वे हि स्याद् दोष एषः} । न तु व्यावृत्तिव्यावृत्तिम‚तोर्भेद इति ‚{\tiny $_{lb}$}‚नात्र त‚त्प‚क्षोक्त‚दोषः ॥ (६३)
	\pend% ending standard par
      \label{div_pvv.3.64}
	  
	% new div opening: depth here is 2
	

	  \pstart \leavevmode% starting standard par
	य‚दि\edtext{}{\edlabel{pvv.312-4}\label{pvv.312-4}\lemma{दि}\Bfootnote{क‚थ‚मिदानीमेक‚स्यान‚नुग‚माद‚न्य‚व्यावृत्तिः सामान्यं । स्व‚ल‚क्ष‚णानुभ‚वोत्त‚र‚{\tiny $_{lb}$}‚मेक‚कार्येष्वेक‚माकार‚माद‚र्श (य) न्विक‚ल्पः सामान्यं अत‚त्कार्येभ्यो भिद्य‚माना अर्थाः ‚{\tiny $_{lb}$}‚सामान्यादिव्य‚व‚हार‚विष‚य इत्य‚न्यापोह‚विष‚य‚त्वं ।}} व्यावृत्तित‚{\tiny $_{4}$}‚द्व‚तोर्न भ‚वेस्त‚दा गोर्गोत्वं शुक्ल‚त्व‚सास्नादिम‚त्वाद‚य‚श्चेति ‚{\tiny $_{lb}$}‚ष‚ष्ठीव‚च‚न‚भेदादि न प्राप्नोतीत्याह ।
	\pend% ending standard par
      
	  \bigskip
	  \begingroup
	
	    \large
	  
	    \begin{quote}
	  
	    
	    \stanza[\smallbreak]
	\label{pv.3.64}\flagstanza{\tiny\textenglish{...v.3.64}}येषां व‚स्तुव‚शा वाचो न विव‚क्षाप‚राश्र‚याः ।&ष‚ष्ठीव‚च‚न‚भेदादि चोद्यं तान् प्र‚ति युक्तिम‚त् ॥ ६४ ॥\&[\smallbreak]


	
	    \end{quote}
	  
	  \endgroup
	

	  \pstart \leavevmode% starting standard par
	\hphantom{.}‚{\color{DodgerBlue3}‚येषां} बाह्यानां म‚ते ‚{\color{DodgerBlue3}‚व‚स्तुव‚शा} व‚स्त्वाय‚त्ता ‚{\color{DodgerBlue3}‚वाचो न विव‚क्षाप‚र आश्र‚यः} कार‚णं ‚{\tiny $_{lb}$}‚यासां तास्त‚था । ‚{\color{DodgerBlue3}‚ष‚ष्ठीव‚च‚न‚भेदादि चोद्यं तान् प्र‚ति युक्तिम‚त्} ॥ (६४)
	\pend% ending standard par
      \label{div_pvv.3.65}
	  
	% new div opening: depth here is 2
	
	  \bigskip
	  \begingroup
	
	    \large
	  
	    \begin{quote}
	  
	    
	    \stanza[\smallbreak]
	\label{pv.3.65}\flagstanza{\tiny\textenglish{...v.3.65}}य‚द् य‚था वाच‚क‚त्वेन व‚क्‏तृभिर्विनिय‚म्य‚ते ।&अन‚पेक्षित‚बाह्यार्थं त‚त् त‚था वाच‚कं व‚चः ॥ ६५ ॥\&[\smallbreak]


	
	    \end{quote}
	  
	  \endgroup
	\textsuperscript{\textenglish{313/s}}

	  \pstart \leavevmode% starting standard par
	\hphantom{.}अस्माकं य‚द्व‚चो ‚{\color{DodgerBlue3}‚य‚था} ध‚र्मान्त‚र‚स्य प्र‚तिक्षेपेणाप्र‚ति‚{\color{DodgerBlue3}‚वाच‚क‚त्वेन व‚क्तृभिर्निय‚म्य‚ते} विशेषेण व्य‚व‚स्थाप्य‚ते‚{\color{DodgerBlue3}‚ऽन‚पेक्षित‚वाह्या‚{\tiny $_{5}$}‚र्थ} संकेत‚मात्रानुरोधित्वात् ‚{\color{DodgerBlue3}‚त‚द्व‚च‚स्त‚था वाच‚{\tiny $_{lb}$}‚क‚मि}‚ष्ट‚मिति न चोद्याव‚काशः ॥ (६५)
	\pend% ending standard par
      \label{div_pvv.3.66}
	  
	% new div opening: depth here is 2
	
	  \bigskip
	  \begingroup
	
	    \large
	  
	    \begin{quote}
	  
	    
	    \stanza[\smallbreak]
	\label{pv.3.66}\flagstanza{\tiny\textenglish{...v.3.66}}दाराः ष‚ण्ण‚ग‚रीत्यादौ भेदाभेद‚व्य‚व‚स्थितेः ।&ख‚स्य स्व‚भावः ख‚त्वं वेत्य‚त्र वा किं निब‚न्ध‚न‚म् ॥ ६६ ॥\&[\smallbreak]


	
	    \end{quote}
	  
	  \endgroup
	

	  \pstart \leavevmode% starting standard par
	\hphantom{.}य‚स्य तु वास्त‚व एव शाब्दो व्य‚व‚हार‚स्त‚स्य ‚{\color{DodgerBlue3}‚दाराः ष‚ण्ण‚ग‚री\edtext{}{\edlabel{pvv.313-1}\label{pvv.313-1}\lemma{री}\Bfootnote{क्रियातो गुण‚तो वा स‚माहारो द्र‚व्याश्रितः । न‚ग‚र‚न्तु विजातीयानार‚म्भाद‚द्र‚व्यं ।}}त्यादा}‚वादिश‚ब्दात् ‚{\tiny $_{lb}$}‚गृहा विंश‚तिरित्यादौ च य‚थाक्र‚म‚म‚भिन्ने भिन्ने च व‚स्तुतो ‚{\color{DodgerBlue3}‚भेदाभेद‚यो}‚र्ब‚हुव‚च‚नैक‚{\tiny $_{lb}$}‚व‚च‚न‚निमित्त‚यो‚{\color{DodgerBlue3}‚र्व्य‚व‚स्थितेः । ख‚स्य स्व‚भावः ख‚त्वं वेत्य‚त्र} ध‚र्मिध‚र्म‚भेद‚स्य ‚{\color{DodgerBlue3}‚किम्वा} निमित्तं न किञ्च‚न । न ह्येक‚स्या (:) स्त्रिया ब‚हुत्वं ष‚ण्णां न‚ग‚रा‚{\tiny $_{6}$}‚णां वा एक‚त्व‚{\tiny $_{lb}$}‚माकाश‚स्य स्व‚भावो भि\edtext{}{\edlabel{pvv.313-2}\label{pvv.313-2}\lemma{भि}\Bfootnote{भेदान्त‚रादिना एत‚दुक्त‚म्भ‚व‚ति । अत‚श्व‚त्वाम‚हिष‚त्वादिषु भेदान्त‚रेष्वेक‚{\tiny $_{lb}$}‚पिण्ड‚निष्ठेषु स‚त्स्व‚प्य‚न‚पेक्षित‚भेदान्त‚र‚म‚श्व‚व्य‚व‚च्छेद‚ल‚क्ष‚ण‚म‚न‚श्व‚त्व‚मात्र‚म‚न‚श्व‚त्व‚{\tiny $_{lb}$}‚श‚ब्द‚स्य ध‚र्म‚व‚च‚न‚स्य संकेत‚कार‚णं । अत्य‚क्त‚भेदान्त‚र‚न्तु त‚देवान‚श्व‚त्वं अन‚श्व इति ‚{\tiny $_{lb}$}‚ध‚र्मिव‚च‚न‚स्य संकेत‚भेदे कार‚णं (।) एत‚च्च पूर्व्व‚श्लोकाव‚तार‚णेन ज्ञेयं ।}}न्नः सामान्यं वास्ति । अथ चास्ति श‚ब्द‚वृत्तिस्त‚तः क‚ल्पित ‚{\tiny $_{lb}$}‚एव त‚द्विष‚यो व‚क्त‚व्यः ॥ (६६)
	\pend% ending standard par
      \label{div_pvv.3.67}
	  
	% new div opening: depth here is 2
	

	  \pstart \leavevmode% starting standard par
	य‚दि स‚र्व‚तो व्यावृत्त‚स्व‚भावा भावा न तेषु सामान्य‚म‚स्ति क‚थं गोत्व‚मित्यादि‚{\tiny $_{lb}$}‚सामान्य‚प्र‚तीतिरित्याह ।
	\pend% ending standard par
      
	  \bigskip
	  \begingroup
	
	    \large
	  
	    \begin{quote}
	  
	    
	    \stanza[\smallbreak]
	\label{pv.3.67}\flagstanza{\tiny\textenglish{...v.3.67}}एकार्थ‚प्र‚तिभासिन्या भावानाश्रित्य भेदिनः ।&रूपं प‚रेषां व्यावृत्तं सा धीः संवृतिरुच्य‚ते ॥ ६७ ॥\&[\smallbreak]


	
	    \end{quote}
	  
	  \endgroup
	

	  \pstart \leavevmode% starting standard par
	\hphantom{.}‚{\color{DodgerBlue3}‚भेदिनः} स‚र्व्व‚तो व्यावृत्तान् ‚{\color{DodgerBlue3}‚भावानाश्रित्य} प‚र‚म्प‚रापातेभ्य\edtext{}{\edlabel{pvv.313-3}\label{pvv.313-3}\lemma{रापातेभ्य}\Bfootnote{त‚द‚न्य‚व्य‚तिरेकिणः प‚दार्थानाश्रित्य ।}} उत्प‚द्य ‚{\color{DodgerBlue3}‚य‚या} धिया ‚{\color{DodgerBlue3}‚एकार्थ‚प्र‚तिभासिन्या} एकार्थाध्य‚व‚साय‚स्वाकार‚या स्व‚रूपेण स्व‚प्र‚तिभासेन ‚{\tiny $_{lb}$}‚‚{\color{DodgerBlue3}‚प‚रेषां} स्व‚ल‚क्ष‚णानां ‚{\color{DodgerBlue3}‚रूपं} स‚र्व्व‚तो‚{\tiny $_{7}$}‚ व्यावृत्तं\edtext{}{\edlabel{pvv.313-4}\label{pvv.313-4}\lemma{व्यावृत्तं}\Bfootnote{त‚देवं स‚मारोप‚प‚क्षे प‚रोक्तं दूष‚णं प‚रिहृत्यान्य‚व्यावृत्तिप‚क्षे तान‚भ्युप‚ग‚मादेवं ‚{\tiny $_{lb}$}‚व्यावृत्ताभावा एक‚त्वेनाध्य‚व‚सिताः सामान्य‚मित्युक्त्वा बुद्ध्याकारे सामान्ये प‚र‚दूष‚ण‚{\tiny $_{lb}$}‚म‚प‚न‚य‚ति । त‚त्तु बुद्ध्याकार‚श्च बुद्धिस्थो नार्थ‚बुद्ध्य‚न्त‚रानुगः । नाभिप्रेतार्थ‚कारी च ‚{\tiny $_{lb}$}‚सोपि वाच्यो न त‚त्व‚तः ॥ अनुप्र‚वेशे सामान्यं न स्यादित्य‚व्य‚तिरिक्त‚दूष‚णं ।}} संव्रिय‚ते प्र‚च्छाद्य‚ते ‚{\color{DodgerBlue3}‚सा\edtext{}{\edlabel{pvv.313-5}\label{pvv.313-5}\lemma{सा}\Bfootnote{प्र‚कृत्या एकाकार‚प‚राम‚र्श‚हेतून् भावानाश्रित्य विक‚ल्प‚बुद्धिरेकाकारो‚{\tiny $_{lb}$}‚त्प‚द्य‚माना यं एक‚माकारं भावेष्व‚र्प्य‚य‚ति स एव बुद्ध्याकारः श‚ब्द‚प्र‚वृत्य‚ङ‚गः ‚{\tiny $_{lb}$}‚सामान्यं सिद्धान्तिनापि बीज‚म‚स्य वाच्य‚म‚नाश्र‚य‚स्यानुत्प‚त्तेरित्याह ।}} बुद्धिः}\leavevmode\ledsidenote{\textenglish{62b/MA}} ‚{\tiny $_{lb}$}‚‚{\color{DodgerBlue3}‚संवृतिरुच्य‚ते} ॥ (६७)
	\pend% ending standard par
      \textsuperscript{\textenglish{314/s}}\label{div_pvv.3.68}
	  
	% new div opening: depth here is 2
	
	  \bigskip
	  \begingroup
	
	    \large
	  
	    \begin{quote}
	  
	    
	    \stanza[\smallbreak]
	\label{pv.3.68}\flagstanza{\tiny\textenglish{...v.3.68}}त‚या संवृत‚नानात्वाः संवृत्या भेदिनः स्व‚य‚म् ।&अभेदिन इवाभान्ति भेदा\edtext{}{\edlabel{pvv.314-1}\label{pvv.314-1}\lemma{भेदा}\Bfootnote{PH. भावाः}}रूपेण केन‚चित् ॥ ६८ ॥\&[\smallbreak]


	
	    \end{quote}
	  
	  \endgroup
	

	  \pstart \leavevmode% starting standard par
	\hphantom{.}‚{\color{DodgerBlue3}‚त‚या} संवृत्त्या ‚{\color{DodgerBlue3}‚स्व‚य}‚मात्म‚ना ‚{\color{DodgerBlue3}‚भेदिनः} स्व‚स्व‚भाव‚व्य‚व‚स्थिता भावाः ‚{\color{DodgerBlue3}‚संवृत‚नानात्वाः} स्थ‚गित‚भेदाः ‚{\color{DodgerBlue3}‚केन‚चिद् रूपेण} विजातीय‚व्यावृत्त्युप‚क‚ल्पितेन गोत्वादिनाऽ‚{\color{DodgerBlue3}‚भेदिन ‚{\tiny $_{lb}$}‚इवाभान्ति} ॥ (६८)
	\pend% ending standard par
      \label{div_pvv.3.69}
	  
	% new div opening: depth here is 2
	
	  \bigskip
	  \begingroup
	
	    \large
	  
	    \begin{quote}
	  
	    
	    \stanza[\smallbreak]
	\label{pv.3.69}\flagstanza{\tiny\textenglish{...v.3.69}}त‚स्या अभिप्राय‚व‚शात् सामान्यं स‚त्प्र‚कीर्तित‚म् ।&य‚था त‚योप‚क‚ल्पितं त‚द‚स‚त प‚र‚मार्थ‚तः ॥ ६९ ॥\&[\smallbreak]


	
	    \end{quote}
	  
	  \endgroup
	

	  \pstart \leavevmode% starting standard par
	\hphantom{.}‚{\color{DodgerBlue3}‚त‚स्या} बुद्धेः सामान्य‚रूप‚त‚याऽध्य‚व‚सिताकारायाऽ (?‚{\color{DodgerBlue3}‚अ}‚) ‚{\color{DodgerBlue3}‚भिप्राय‚व‚शात्} सामान्यं ‚{\tiny $_{lb}$}‚‚{\color{DodgerBlue3}‚स‚त् प्र‚कीर्त्तितं} । विजातीय‚व्यावृत्तेर्व्व‚स्तुष्व‚भावात् । त‚दुप‚क‚ल्पितं गोत्वादि सामान्य ‚{\tiny $_{lb}$}‚रूपेण‚{\tiny $_{1}$}‚ बुद्ध्याकार‚म‚ध्य‚व‚{\color{DodgerBlue3}‚श्य} (? स्य) न्ति व्य‚व‚ह‚र्त्तारः । अध्य‚व‚सायानुरोधेन च ‚{\tiny $_{lb}$}‚सामान्यं स‚दित्युच्य‚ते (।) ‚{\color{DodgerBlue3}‚य‚था} व‚स्तुत्वेन ‚{\color{DodgerBlue3}‚त‚त्} सामान्यं ‚{\color{DodgerBlue3}‚त‚या} संवृत्तिबुद्ध्या क‚ल्पितं ‚{\tiny $_{lb}$}‚‚{\color{DodgerBlue3}‚त‚थाऽस‚त् प‚र‚मार्थ‚तः} ॥ (६९)
	\pend% ending standard par
      \label{div_pvv.3.70}
	  
	% new div opening: depth here is 2
	

	  \pstart \leavevmode% starting standard par
	त‚देवास‚त्त्व‚माह ।
	\pend% ending standard par
      
	  \bigskip
	  \begingroup
	
	    \large
	  
	    \begin{quote}
	  
	    
	    \stanza[\smallbreak]
	\label{pv.3.70}\flagstanza{\tiny\textenglish{...v.3.70}}व्य‚क्त‚यो नानुय‚न्त्य‚न्य‚द‚नुयायि न भास‚ते ।&ज्ञानाद‚व्य‚तिरिक्तं वा क‚थ‚म‚र्थान्त‚रं व्र‚जेत् ॥ ७० ॥\&[\smallbreak]


	
	    \end{quote}
	  
	  \endgroup
	

	  \pstart \leavevmode% starting standard par
	\hphantom{.}‚{\color{DodgerBlue3}‚व्य‚क्त‚य}‚स्ताव‚{\color{DodgerBlue3}‚न्न} प‚र‚स्प‚र‚{\color{DodgerBlue3}‚म‚नुय‚न्ति} भेदात्\edtext{}{\edlabel{pvv.314-2}\label{pvv.314-2}\lemma{भेदात्}\Bfootnote{व्य‚तिरिक्ते ।}} । तास्व‚{\color{DodgerBlue3}‚नु\edtext{}{\edlabel{pvv.314-3}\label{pvv.314-3}\lemma{नु}\Bfootnote{विक‚ल्प‚प्र‚तिभासेपि दोष‚श्चेत् स‚मानाकारो भात्येव य‚था प्र‚तिभास‚{\tiny $_{lb}$}‚ञ्चास्त्येवाव‚स्तुत्वात् ।}}यायि} च किञ्चिन्न ‚{\tiny $_{lb}$}‚‚{\color{DodgerBlue3}‚भास‚ते} व्य‚क्तिमात्र‚वेद‚नात् । य‚च्च ‚{\color{DodgerBlue3}‚ज्ञानाद‚व्य‚तिरिक्त}‚माकार\edtext{}{\edlabel{pvv.314-4}\label{pvv.314-4}\lemma{माकार}\Bfootnote{विक‚ल्प‚बुद्ध्याकारोस्तु सामान्यं स च ज्ञान‚व‚द् व‚स्तुस‚न्नित्य‚त्राह ।}}स्व‚रूपं त‚त्क‚{\color{DodgerBlue3}‚थ‚म‚र्थान्त‚रं} ज्ञानान्त‚रं\edtext{}{\edlabel{pvv.314-5}\label{pvv.314-5}\lemma{रं}\Bfootnote{बाह्य‚व्य‚क्तेर्ज्ञान‚स्य चाव्याप‚ने स्यात् सामान्यं ।}} ‚{\color{DodgerBlue3}‚वा व्र‚जेत्} स्व‚ल‚क्ष‚ण‚रूप‚त्वाद‚स्य ॥ (७०)
	\pend% ending standard par
      \label{div_pvv.3.71}
	  
	% new div opening: depth here is 2
	
	  \bigskip
	  \begingroup
	
	    \large
	  
	    \begin{quote}
	  
	    
	    \stanza[\smallbreak]
	\label{pv.3.71}\flagstanza{\tiny\textenglish{...v.3.71}}त‚स्मान्मिथ्याविक‚ल्पोय‚म‚र्थेष्वेकात्म‚ताग्र‚हः ।&इत‚रेत‚र‚भेदोस्य बीजं संज्ञा य‚द‚र्थिका ॥ ७१ ॥\&[\smallbreak]


	
	    \end{quote}
	  
	  \endgroup
	

	  \pstart \leavevmode% starting standard par
	\hphantom{.}‚{\color{DodgerBlue3}‚त‚स्माद‚र्थ‚रूप}‚{\tiny $_{2}$}‚स्य ज्ञान‚रूप‚स्य च सामान्य‚स्य योगा‚{\color{DodgerBlue3}‚न्मिथ्याविक‚ल्पोय‚म}‚र्थ‚शून्य एष ‚{\tiny $_{lb}$}‚विक‚ल्पोय‚म‚{\color{DodgerBlue3}‚र्थेष्वेकात्म‚तायाः} सामान्य‚रूप‚ताया ‚{\color{DodgerBlue3}‚ग्र‚हः । अस्य}\edtext{}{\edlabel{pvv.314-6}\label{pvv.314-6}\lemma{ताया}\Bfootnote{एक‚प्र‚त्य‚व‚म‚र्ष‚ज्ञान‚साध‚ने निय‚ता इति साध्य‚मुक्त‚म् ।}} चैकात्म‚त‚या प्र‚ति‚{\tiny $_{lb}$}‚\leavevmode\ledsidenote{\textenglish{315/s}} भासिनो मिथ्याविक‚ल्प‚स्य ‚{\color{DodgerBlue3}‚बीजं} हेतुरित‚र‚स्माद‚त‚त्कार्य\edtext{}{\edlabel{pvv.315-1}\label{pvv.315-1}\lemma{त्कार्य}\Bfootnote{वाह‚दोहाद्य‚कारिणः ।}} कारिणो ‚{\color{DodgerBlue3}‚भेदो} व्यावृत्तिः । ‚{\tiny $_{lb}$}‚‚{\color{DodgerBlue3}‚य‚द‚र्थिका संज्ञा} श‚ब्दोपि विजातीय‚व्यावृत्तौ संकेत्य‚ते त‚द्विष‚य‚श्च ॥ (७१)
	\pend% ending standard par
      \label{div_pvv.3.72}
	  
	% new div opening: depth here is 2
	

	  \begin{center}%% label @type='head'
	\textbf{(ख) भिन्नानाम‚भिन्नं कार्य‚म्}
	\end{center}
	

	  \pstart \leavevmode% starting standard par
	क‚थं पुन‚र्भिन्नानाम‚भिन्नं कार्य‚मित्याह ।
	\pend% ending standard par
      
	  \bigskip
	  \begingroup
	
	    \large
	  
	    \begin{quote}
	  
	    
	    \stanza[\smallbreak]
	\label{pv.3.72}\flagstanza{\tiny\textenglish{...v.3.72}}एक‚प्र‚त्य‚व‚म‚र्शार्थ‚ज्ञानाद्येकार्थ‚साध‚ने ।&केचिद् भेदेपि निय‚ताः स्व‚भावेनेन्द्रियादिव‚त् ॥ ७२ ॥\&[\smallbreak]


	
	    \end{quote}
	  
	  \endgroup
	

	  \pstart \leavevmode% starting standard par
	\hphantom{.}‚{\color{DodgerBlue3}‚एक‚प्र‚त्य‚व‚म‚र्श} एकाकाराध्य‚व‚सा‚{\tiny $_{3}$}‚योऽ‚{\color{DodgerBlue3}‚र्थ‚ज्ञानं} रूपादि‚{\color{DodgerBlue3}‚ज्ञानं}\edtext{}{\edlabel{pvv.315-2}\label{pvv.315-2}\lemma{रूपादि}\Bfootnote{पूर्व्वेण स‚ह द्व‚न्द्वः ।}} च त‚{\color{DodgerBlue3}‚दादि}‚र्य‚स्य ‚{\tiny $_{lb}$}‚ज्व‚र‚ह‚र‚णादेः । त‚स्यैकार्थ‚स्य ‚{\color{DodgerBlue3}‚साध‚ने} क‚र‚णे ‚{\color{DodgerBlue3}‚स्व‚भावेन} प्र‚कृत्या स्व‚हेतुद‚त्त‚या ‚{\color{DodgerBlue3}‚भेदेपि} भेदाविशेषेपि ‚{\color{DodgerBlue3}‚केचिद्} बाहुलेयाद‚यो ‚{\color{DodgerBlue3}‚निय‚ता} न क‚र्काद‚य ‚{\color{DodgerBlue3}‚इन्द्रियादिव‚त्\edtext{}{\edlabel{pvv.315-3}\label{pvv.315-3}\lemma{त्}\Bfootnote{अर्थ‚ज्ञानादिदृष्टान्तः ।}}} । य‚था ‚{\tiny $_{lb}$}‚च‚क्षूरूपालोक‚म‚न‚स्काराद‚य\edtext{}{\edlabel{pvv.315-4}\label{pvv.315-4}\lemma{य}\Bfootnote{अस‚त्य‚प्येक‚कार्य‚निय‚ते सामान्ये ।}}एव भेदाविशेषेपि रूप‚विज्ञानं ज‚न‚य‚न्ति न श्रोत्र‚{\tiny $_{lb}$}‚श‚ब्दाद‚यः ॥ (७२)
	\pend% ending standard par
      \label{div_pvv.3.73}
	  
	% new div opening: depth here is 2
	

	  \pstart \leavevmode% starting standard par
	एत‚देव दृष्टान्त‚रेण दृढ‚य‚न्नाह ।
	\pend% ending standard par
      
	  \bigskip
	  \begingroup
	
	    \large
	  
	    \begin{quote}
	  
	    
	    \stanza[\smallbreak]
	\label{pv.3.73}\flagstanza{\tiny\textenglish{...v.3.73}}ज्व‚रादिश‚म‚ने काश्चित् स‚ह प्र‚त्येक‚मेव वा ।&दृष्टा य‚था वौष‚ध‚यो नानात्वेपि न चाप‚राः ॥ ७३ ॥\&[\smallbreak]


	
	    \end{quote}
	  
	  \endgroup
	

	  \pstart \leavevmode% starting standard par
	\hphantom{.}‚{\color{DodgerBlue3}‚ज्व‚रादिश‚म‚ने} क‚र्त्त‚व्ये‚{\tiny $_{4}$}‚ ‚{\color{DodgerBlue3}‚काश्चित्} गूडूच्याद‚यः ‚{\color{DodgerBlue3}‚स‚ह} प‚र‚स्प‚रं ‚{\color{DodgerBlue3}‚प्र‚त्येक‚म्वा दृष्टा । ‚{\tiny $_{lb}$}‚नानात्वेपि न चाप‚राः । य‚था} गूडूचीमुस्ताद‚यो भिन्नास्त‚था त्र‚पुषाद‚योपि । त‚थापि ‚{\tiny $_{lb}$}‚काश्चिज्ज्व‚रं श‚म‚य‚न्ति न स‚र्व्वाः । एवं शाब‚लेयाद‚य एक‚प्र‚त्य‚व‚म‚र्शं कुर्व्व‚न्ति । ‚{\tiny $_{lb}$}‚‚{\color{DodgerBlue3}‚न} क‚र्काद‚यः ॥ (७३)
	\pend% ending standard par
      \label{div_pvv.3.74}
	  
	% new div opening: depth here is 2
	

	  \pstart \leavevmode% starting standard par
	स्यादेत‚त् । तास्वोष‚धीषु सामान्यं किञ्चिद‚स्ति त‚त् ज्व‚रादिश‚म‚नं क‚रोति । ‚{\tiny $_{lb}$}‚त‚न्नैष दृष्टान्त (: ।) इत्याह ।
	\pend% ending standard par
      
	  \bigskip
	  \begingroup
	
	    \large
	  
	    \begin{quote}
	  
	    
	    \stanza[\smallbreak]
	\label{pv.3.74}\flagstanza{\tiny\textenglish{...v.3.74}}अविशेषान्न सामान्य‚म‚विशेष‚प्र‚स‚ङ्ग‚तः ।&तासां क्षेत्रादिभेदेपि ध्रौव्याच्चानुप‚कार‚तः ॥ ७४ ॥\&[\smallbreak]


	
	    \end{quote}
	  
	  \endgroup
	

	  \pstart \leavevmode% starting standard par
	\hphantom{.}‚{\color{DodgerBlue3}‚अविशेषात्} सामान्य‚स्याक्रियाव‚{\tiny $_{5}$}‚स्थातः क्रियाव‚स्थायां ‚{\color{DodgerBlue3}‚न सामान्यं} काञ्चिद‚र्थ‚{\tiny $_{lb}$}‚क्रियामुप‚क‚ल्प‚य‚ति । ‚{\color{DodgerBlue3}‚तासां} गूडूच्यादिव्य‚क्तीनां ‚{\color{DodgerBlue3}‚क्षेत्रादिभेदेपि} सामान्य‚स्य कार्यिणो‚{\tiny $_{lb}$}‚ऽविशिष्ट‚त्वात् ज्व‚र‚श‚म‚नादेः कार्य‚स्या‚{\color{DodgerBlue3}‚विशेष‚प्र‚स‚ङ्ग‚तः} चिर‚क्षिप्र‚प्र‚श‚म‚नाद्य‚भावा‚{\tiny $_{lb}$}‚स‚क्तेः ।\edtext{\textsuperscript{*}}{\edlabel{pvv.315-5}\label{pvv.315-5}\lemma{*}\Bfootnote{प्र‚स‚ङ्गात् ।}} ‚{\color{DodgerBlue3}‚ध्रौव्याच्चानुप‚कार‚तः} । सामान्य‚स्य नित्य‚त्वात् । अन्येभ्यः स‚ह‚कारिभ्य ‚{\tiny $_{lb}$}‚उप‚काराभावात् स‚कृत् त‚त्कार्याणि स्युः ॥ (७४)
	\pend% ending standard par
      \textsuperscript{\textenglish{316/s}}\label{div_pvv.3.75}
	  
	% new div opening: depth here is 2
	

	  \begin{center}%% label @type='head'
	\textbf{(ग) अपोह‚स्य विजातीय‚व्याव‚र्त्त‚क‚त्व‚म्}
	\end{center}
	

	  \pstart \leavevmode% starting standard par
	अपोह‚विष‚य‚त्वे श‚ब्द‚वि‚{\tiny $_{6}$}‚क‚ल्प‚योः सामान्यं विशेष‚ण‚विशेष्य‚भावं ध‚र्मिध‚र्म‚{\tiny $_{lb}$}‚भाव‚ञ्च व्य‚व‚स्थाप‚यितुमाह ।
	\pend% ending standard par
      
	  \bigskip
	  \begingroup
	
	    \large
	  
	    \begin{quote}
	  
	    
	    \stanza[\smallbreak]
	\label{pv.3.75}\flagstanza{\tiny\textenglish{...v.3.75}}त‚त्स्व‚भाव‚विक‚ल्पा धीस्त‚द‚र्थे वाप्य‚न‚र्थिका ।&विक‚ल्पिकाऽत‚त्कार्यार्थ‚भेद‚निष्ठा प्र‚जाय‚ते ॥ ७५ ॥\&[\smallbreak]


	
	    \end{quote}
	  
	  \endgroup
	

	  \pstart \leavevmode% starting standard par
	\hphantom{.}‚{\color{DodgerBlue3}‚त‚स्यो}‚त्प‚लादेः श‚ब्दादेश्च स्व‚ल‚क्ष‚ण‚स्य ‚{\color{DodgerBlue3}‚स्व‚भाव}‚ग्र‚हादूर्ध्वं या ‚{\color{DodgerBlue3}‚विक‚ल्पिका धीः ‚{\tiny $_{lb}$}‚प्र‚जाय‚ते} व‚स्तुतो‚{\color{DodgerBlue3}‚ऽन‚र्थि}‚कापि ‚{\color{DodgerBlue3}‚त‚द‚र्थेव} स्व‚ल‚क्ष‚ण‚विष‚येवाऽध्य‚व‚सायानुरोधात् प‚र‚मा‚{\tiny $_{lb}$}‚र्थ‚ता\edtext{}{\edlabel{pvv.316-1}\label{pvv.316-1}\lemma{ता}\Bfootnote{तेन य‚द् [कुमारिल] भ ट्टः (।) अन्य निवृत्तिमात्रापोहे अनीलादि‚{\tiny $_{lb}$}‚व्यावृत्ताव‚नुत्प‚लादिव्यावृत्त्य‚भावः । एव‚म‚नुत्प‚ले त‚तो न विशेष‚ण‚विशेष्य‚ता । नापि ‚{\tiny $_{lb}$}‚सामानाधिक‚र‚ण्य‚म‚पोह‚योर्भेदात् (।) न च स्व‚ल‚क्ष‚णं श‚ब्द‚विष‚यः । अपोह‚योर्व्वाऽ‚{\tiny $_{lb}$}‚प्र‚तीतेः । त‚न्निर‚स्तं । बाह्याभिन्न‚स्य स्वाकार‚स्य श‚ब्दादिविष‚य‚त्वेनेष्ट‚त्वात् (।) ‚{\tiny $_{lb}$}‚तेन नीलोत्प‚लादिश‚ब्दे श‚ब‚लार्थाभिधान‚मेव ।}} ‚{\color{DodgerBlue3}‚ऽत‚त्कार्येभ्यो अर्थेभ्यो भेदो} व्यावृत्तिस्त‚त्र ‚{\color{DodgerBlue3}‚निष्ठा}‚ऽव‚स्थानं य‚स्या सा त‚था ‚{\tiny $_{lb}$}‚विजातीय‚व्यावृत्तिविष‚येत्य‚र्थः ॥ (७५)
	\pend% ending standard par
      \label{div_pvv.3.76_3.77}
	  
	% new div opening: depth here is 2
	
	  \bigskip
	  \begingroup
	
	    \large
	  
	    \begin{quote}
	  
	    
	    \stanza[\smallbreak]
	\label{pv.3.76}\flagstanza{\tiny\textenglish{...v.3.76}}त‚स्यां य‚द्रूप‚माभाति बाह्य‚मेक‚मिवान्य‚तः ।&व्यावृत्त‚मिव निस्त‚त्वं प‚रीक्षान‚ङ्ग‚भाव‚तः ॥ ७६ ॥\&[\smallbreak]


	
	    \end{quote}
	  
	  \endgroup
	
	  \bigskip
	  \begingroup
	
	    \large
	  
	    \begin{quote}
	  
	    
	    \stanza[\smallbreak]
	\label{pv.3.77}\flagstanza{\tiny\textenglish{...v.3.77}}अर्था ज्ञान‚निविष्टास्त एवं व्यावृत्त‚रूप‚काः ।&अभिन्ना इव चाभान्ति व्यावृत्ताः पुन‚र‚न्य‚तः ॥ ७७ ॥\&[\smallbreak]


	
	    \end{quote}
	  
	  \endgroup
	\textsuperscript{\textenglish{63a/MA}}

	  \pstart \leavevmode% starting standard par
	\hphantom{.}‚{\color{DodgerBlue3}‚त‚स्यां} विक‚ल्प‚बु‚{\tiny $_{7}$}‚द्धौ ‚{\color{DodgerBlue3}‚य‚द् रूपं} य आकारो दृश्य‚विक‚ल्प‚योरेक‚त्वाध्य‚व‚साया‚{\tiny $_{lb}$}‚भ्यास‚दाढ्‏र्याद‚बाह्य‚म‚पि बाह्य‚मिवासाधार‚ण‚म‚प्येक‚मिव स‚र्व्व‚व्य‚क्तिषु स‚दृश‚वृत्तेः । ‚{\tiny $_{lb}$}‚य‚था य‚था व्य‚क्त‚यो दृश्य‚न्ते त‚था त‚थैवाध्य‚व‚सायात् व्यावृत्त‚मिव विजातीय‚व्यावृत्त‚{\tiny $_{lb}$}‚व‚स्त्व‚भेदेन निश्च‚यात् । न च त‚द्व्यावृत्तं गोरूप‚त्व‚प्र‚स‚ङ्गात् । अत एव निस्त‚त्वं ‚{\tiny $_{lb}$}‚निःस्व‚रूपं य‚थाभूत‚रूप‚तिरोधानेनान्य‚थाध्य‚व‚सायात् । त‚था चाप‚री‚{\tiny $_{1}$}‚क्षाया विचार‚{\tiny $_{lb}$}‚स्या‚{\color{DodgerBlue3}‚न‚ङ्ग‚भावा}‚द‚नाश्र‚यान्निस्त‚त्वं त‚त्\edtext{}{\edlabel{pvv.316-2}\label{pvv.316-2}\lemma{त्}\Bfootnote{प‚रीक्षाङ्गं ।}} द्विविधं त्व‚र्थो ज्ञानं वा । न चैत‚त् त‚था । य‚तः\edtext{}{\edlabel{pvv.316-3}\label{pvv.316-3}\lemma{तः}\Bfootnote{य‚तो बुद्धिप्र‚तिभासि रूपं निस्त‚त्व‚म‚त‚स्त‚द्विष‚यो ‚{\tiny $_{lb}$}‚व्य‚व‚हारोपि मिथ्यार्थ इत्याह ।}} ‚{\tiny $_{lb}$}‚कार‚णात् तेन क‚ल्पितेन सामान्य‚रूपेण ‚{\color{DodgerBlue3}‚तेऽर्था} एकार्थ‚क्रियाकारिणोऽत‚त्कार्ये\edtext{}{\edlabel{pvv.316-4}\label{pvv.316-4}\lemma{त्कार्ये}\Bfootnote{य‚तो विजातीयाद् व्यावृत्तिरूप‚व‚न्तः । य‚थाऽनुत्प‚लाद् व्यावृत्तिरुपिण उत्प‚{\tiny $_{lb}$}‚लार्थाः ।}}भ्यो ‚{\tiny $_{lb}$}‚\leavevmode\ledsidenote{\textenglish{317/s}} ‚{\color{DodgerBlue3}‚व्यावृ}‚त्त‚स्व‚भावा ‚{\color{DodgerBlue3}‚ज्ञान‚निविष्टा} विक‚ल्प‚बुद्ध्यारूढा ‚{\color{DodgerBlue3}‚अभिन्ना इवा\edtext{}{\edlabel{pvv.317-1}\label{pvv.317-1}\lemma{इवा}\Bfootnote{न व‚स्तुतो बुद्धिरूप‚स्यालीक‚त्वात् ।}} भान्त्युत्प‚ल‚त्वादिना} श‚ब्द‚त्वादिना ‚{\color{DodgerBlue3}‚च} ।
	\pend% ending standard par
      

	  \pstart \leavevmode% starting standard par
	एतेन सामान्य‚व्य‚व‚स्थो\edtext{}{\edlabel{pvv.317-2}\label{pvv.317-2}\lemma{स्थो}\Bfootnote{व्य‚व‚हार‚निमित्तं ।}}क्ता (।) त\edtext{}{\edlabel{pvv.317-3}\label{pvv.317-3}\lemma{त}\Bfootnote{त एवेति प‚रिकारिकाया आकृष्टं ।}} एवैक‚जा\edtext{}{\edlabel{pvv.317-4}\label{pvv.317-4}\lemma{जा}\Bfootnote{त एव ज्ञान (नि) विष्टा व्यावृत्ताः स‚न्तः पुन‚र‚न्य‚तः स‚जातीयाद‚पि व्या‚{\tiny $_{lb}$}‚वृत्ता भान्ति य‚था नीला अनीलाद‚त‚श्च व्यावृत्तिद्व‚य‚स्य भानात् सामानाधिक‚र‚ण्य‚{\tiny $_{lb}$}‚बीज‚मुक्तं ।}}त्य‚व‚सिताऽन्य‚तोऽनीलात् ‚{\tiny $_{lb}$}‚नित्याच्च नील‚मित्यादिविक‚{\tiny $_{2}$}‚ल्पाकारेण एकेन त‚त्कारिता\edtext{}{\edlabel{pvv.317-5}\label{pvv.317-5}\lemma{त्कारिता}\Bfootnote{व्यावृत्त‚भाव ।}} व्यावृत्त‚रूप‚ताऽध्य‚व‚{\tiny $_{lb}$}‚साय‚विष‚येण विशेषिता व्यावृत्ता आभान्ति नीलोत्प‚ल‚मिति श‚ब्द\edtext{}{\edlabel{pvv.317-6}\label{pvv.317-6}\lemma{ब्द}\Bfootnote{उत्प‚लानित्य‚मात्र‚जाती बुद्ध्या नील‚श‚ब्देन विशेषिते ।}}स्यानित्य‚त्व‚{\tiny $_{lb}$}‚मिति ॥ (७६, ७७)
	\pend% ending standard par
      \label{div_pvv.3.78}
	  
	% new div opening: depth here is 2
	
	  \bigskip
	  \begingroup
	
	    \large
	  
	    \begin{quote}
	  
	    
	    \stanza[\smallbreak]
	\label{pv.3.78}\flagstanza{\tiny\textenglish{...v.3.78}}त एव तेषां सामान्य‚स‚मानाधार‚गोच‚रैः ।&ज्ञानाभिधानैर्व्य‚व‚हारो मिथ्यार्थः प्र‚त‚न्य‚ते ॥ ७८ ॥\&[\smallbreak]


	
	    \end{quote}
	  
	  \endgroup
	

	  \pstart \leavevmode% starting standard par
	\hphantom{.}‚{\color{DodgerBlue3}‚तेषा}‚मुभ‚य‚व्यावृत्तिविशेषितानां विक‚ल्पारूढानाम‚र्थानां ‚{\color{DodgerBlue3}‚सामान्य‚यो}‚र्व्विशेष‚ण‚{\tiny $_{lb}$}‚विशेष्य‚भूत‚योर्द्ध‚र्म‚ध‚र्मिरूप‚योश्च सामानाधिक‚र‚ण्यं ‚{\color{DodgerBlue3}‚स‚मानाधारो} भाव‚प्र‚धान‚त्वान्नि‚{\tiny $_{lb}$}‚र्देश‚स्य । स ‚{\color{DodgerBlue3}‚गोच‚रो} येषान्तै‚{\color{DodgerBlue3}‚र्ज्ञानाभिधानै‚{\tiny $_{3}$}‚}‚र्व्विक‚ल्प‚श‚ब्दैर्व्विशेष‚ण‚विशेष्य‚भाव‚स्य ‚{\tiny $_{lb}$}‚ध‚र्मिध‚र्म‚भाव‚स्य ‚{\color{DodgerBlue3}‚व्य‚व‚हारो मिथ्यार्थः प्र‚त‚न्य‚ते} विस्तार्य‚ते ॥ (७८)
	\pend% ending standard par
      \label{div_pvv.3.79_3.80}
	  
	% new div opening: depth here is 2
	
	  \bigskip
	  \begingroup
	
	    \large
	  
	    \begin{quote}
	  
	    
	    \stanza[\smallbreak]
	\label{pv.3.79}\flagstanza{\tiny\textenglish{...v.3.79}}स च स‚र्वः प‚दार्थानाम‚न्योन्याभाव‚संश्र‚यः ।&तेनान्यापोह‚विष‚यः त‚द‚त‚त्कार्य‚कारिणाम् ॥ ७९ ॥\&[\smallbreak]


	
	    \end{quote}
	  
	  \endgroup
	
	  \bigskip
	  \begingroup
	
	    \large
	  
	    \begin{quote}
	  
	    
	    \stanza[\smallbreak]
	\label{pv.3.80a}\flagstanza{\tiny\textenglish{....3.80a}}व‚स्तुलाभाश्र‚यो;\&[\smallbreak]


	
	    \end{quote}
	  
	  \endgroup
	

	  \pstart \leavevmode% starting standard par
	\hphantom{.}‚{\color{DodgerBlue3}‚स च} ज्ञानाभिधान‚ल‚क्ष‚णो विशेष‚ण‚विशेष्य‚भावादिव्य‚व‚हारः ‚{\color{DodgerBlue3}‚स‚र्व्वः प‚दार्थानां} त‚द‚त‚त्कार्य‚कारिणा‚{\color{DodgerBlue3}‚म‚न्योत्य‚स्य} इत‚रेत‚र‚स्या‚{\color{DodgerBlue3}‚भावो} व्य‚व‚च्छेदः\edtext{}{\edlabel{pvv.317-7}\label{pvv.317-7}\lemma{च्छेदः}\Bfootnote{व्यावृत्त‚प‚दार्थानुभ‚व‚द्वारेणोत्प‚त्तेः ।}} ‚{\color{DodgerBlue3}‚संश्र‚यो} विष‚यो य‚स्य ‚{\tiny $_{lb}$}‚स त‚था । ‚{\color{DodgerBlue3}‚तेन} व्य‚व‚च्छेद‚विष‚य‚त्वेना‚{\color{DodgerBlue3}‚पोह‚विष‚यो} व्य‚व‚स्थाप्य‚ते ।‚{\tiny $_{4}$}‚ ‚{\color{DodgerBlue3}‚व‚स्तुनो\edtext{}{\edlabel{pvv.317-8}\label{pvv.317-8}\lemma{स्तुनो}\Bfootnote{विधिविष‚य‚त्वाद‚स्य ।}} लाभ‚स्य} प्राप्तेश्चा‚{\color{DodgerBlue3}‚श्र‚यो} निमित्तं स व्य‚व‚हारो भ‚व‚ति ॥\edtext{\textsuperscript{*}}{\edlabel{pvv.317-9}\label{pvv.317-9}\lemma{*}\Bfootnote{न स‚र्व्वः किन्तु ।}}(७९)
	\pend% ending standard par
      
	  \bigskip
	  \begingroup
	
	    \large
	  
	    \begin{quote}
	  
	    
	    \stanza[\smallbreak]
	\label{pv.3.80b}\flagstanza{\tiny\textenglish{....3.80b}}य‚त्र य‚थोक्तानुमित्तौ य‚था ।&नान्य‚त्र भ्रान्तिसाम्येपि दीप‚तेजो म‚णौ य‚था ॥ ८० ॥\&[\smallbreak]


	
	    \end{quote}
	  
	  \endgroup
	

	  \pstart \leavevmode% starting standard par
	य‚त्र व्य‚व‚हारे व‚स्तुनः स‚म्ब‚न्धः प‚रंप‚र‚या त‚दुत्प‚त्तेर‚स्ति\edtext{}{\edlabel{pvv.317-10}\label{pvv.317-10}\lemma{स्ति}\Bfootnote{उदाह‚र‚ण‚माह ।}} ‚{\color{DodgerBlue3}‚य‚थोक्तानुमितौ} \leavevmode\ledsidenote{\textenglish{318/s}} यादृशी साध्य‚प्र‚तिब‚द्ध‚कार्य‚स्व‚भावानुप‚ल‚म्भ‚लिङ्ग‚जानुमितिरुक्ता । त‚त्र ‚{\color{DodgerBlue3}‚य‚था} प‚रंप‚र‚या व‚स्तुस‚म्ब‚न्धाद् व‚स्तुप्रा‚{\color{DodgerBlue3}‚प्तिर्नान्य‚त्र\edtext{}{\edlabel{pvv.318-1}\label{pvv.318-1}\lemma{त्र}\Bfootnote{स्थिरादिविक‚ल्पे त‚त्र प (।) रंप‚र्येणापि व‚स्त्व‚स‚म्ब‚न्धात् व‚स्तुनोऽस्थिरादि‚{\tiny $_{lb}$}‚त्वाद् ।}} भ्रान्तिसाम्येपि दीप‚तेजो म‚णौ ‚{\tiny $_{lb}$}‚य‚था} । कुञ्चिकाविव‚र‚देश‚स्थे दीप‚तेज‚सि भ्रा‚{\tiny $_{5}$}‚न्त्या स‚मारोपिते म‚णौ\edtext{}{\edlabel{pvv.318-2}\label{pvv.318-2}\lemma{णौ}\Bfootnote{अधिग‚न्त‚व्ये ।}} ‚{\tiny $_{lb}$}‚प‚र‚म्प‚र‚यापि व‚स्तुस‚म्ब‚न्धाभावान्न प्राप्तिः । एवं य‚द्य‚पि स‚र्व्व‚स्य विक‚ल्प‚स्य ‚{\tiny $_{lb}$}‚स्व‚प्र‚तिभासेऽन‚र्थेर्थाध्य‚व‚सायेन वृत्तेर्भ्रान्त‚त्वं त‚थापि यो व‚स्तुस‚म्ब‚न्ध‚वान् स ‚{\tiny $_{lb}$}‚त‚त्प्राप‚को नेत‚र इति युक्तो विभागः ॥ (८०)
	\pend% ending standard par
      \label{div_pvv.3.81}
	  
	% new div opening: depth here is 2
	

	  \pstart \leavevmode% starting standard par
	न‚नु य‚दि ज्ञान‚निविष्टानाम‚र्थानां साम‚र्थ्यादिव्य‚व‚हार‚स्त‚दा बाह्येषु स न ‚{\tiny $_{lb}$}‚स्यादित्याह ।
	\pend% ending standard par
      
	  \bigskip
	  \begingroup
	
	    \large
	  
	    \begin{quote}
	  
	    
	    \stanza[\smallbreak]
	\label{pv.3.81}\flagstanza{\tiny\textenglish{...v.3.81}}त‚त्रानेकोपि कार्यैका न त‚त्कार्य‚प‚राश्र‚यैः ।&ज्ञानाभिधानैरेक‚त्वात् व्य‚व‚हारः प्र‚तार्य‚ते ॥ ८१ ॥\&[\smallbreak]


	
	    \end{quote}
	  
	  \endgroup
	

	  \pstart \leavevmode% starting standard par
	\hphantom{.}‚{\color{DodgerBlue3}‚त‚त्र} ग‚वाश्वादिव्य‚क्तिषु म‚ध्ये‚{\color{DodgerBlue3}‚ऽनेकोपि} सा (?शा) व‚लेय‚बाहुलेयादिरे‚{\tiny $_{6}$}‚‚{\tiny $_{lb}$}‚क‚म‚भेदाव‚सायादिकार्यं य‚स्य ‚{\color{DodgerBlue3}‚ज्ञानाभिधानैः} कीदृशैस्त‚द्वाहाद्य‚कार्यं ‚{\color{DodgerBlue3}‚कार्यं} न भ‚व‚ति ‚{\tiny $_{lb}$}‚येषां क‚र्कादीनां तेभ्योऽन्य‚ता त‚द्व्य‚व‚च्छेदः स ‚{\color{DodgerBlue3}‚आश्र‚यो} विष‚यो येषां ‚{\color{DodgerBlue3}‚तैरेक‚त्वेन ‚{\tiny $_{lb}$}‚व्य‚व‚हारं प्र‚तार्य‚ते} प्राप्य‚ते । शाव‚लेयादिर्विजातीय‚व्यावृतौ व्यावृत्त्याश्र‚यैः श‚ब्द‚{\tiny $_{lb}$}‚ज्ञानैरेक‚त्वेन व्य‚व‚ह्निय‚ते इत्य‚र्थः ॥ (८१)
	\pend% ending standard par
      \label{div_pvv.3.82}
	  
	% new div opening: depth here is 2
	
	  \bigskip
	  \begingroup
	
	    \large
	  
	    \begin{quote}
	  
	    
	    \stanza[\smallbreak]
	\label{pv.3.82}\flagstanza{\tiny\textenglish{...v.3.82}}त‚त‚श्चैकोप्य‚नेक‚कृत् त‚द्भाव‚प‚रिदीप‚नात् ।&अत‚त्कार्यार्थ‚भेदेन नानाध‚र्मः प्र‚तीय‚ते ॥ ८२ ॥\&[\smallbreak]


	
	    \end{quote}
	  
	  \endgroup
	

	  \pstart \leavevmode% starting standard par
	\hphantom{.}‚{\color{DodgerBlue3}‚त‚त‚श्\edtext{}{\edlabel{pvv.318-3}\label{pvv.318-3}\lemma{श्}\Bfootnote{त‚थेत्य‚न्त‚रं सामान्य‚व्य‚पेक्ष‚या सामानाधिक‚र‚ण्य‚विशेष‚ण‚विशेष्य‚भाव‚व्य‚{\tiny $_{lb}$}‚व‚हार‚श्च बाह्येष्वेवेति द‚र्श‚य‚न्नाह ।}}चैकोपि दीपादिरा}‚लोकान्ध‚काराप‚न‚य‚न‚व‚र्तिदाहा‚{\color{DodgerBlue3}‚द्य‚नेक‚कार्य‚कृत्‚{\tiny $_{7}$}‚} (।) ‚{\tiny $_{lb}$}‚\leavevmode\ledsidenote{\textenglish{63b/MA}} ‚{\color{DodgerBlue3}‚त‚द्भाव‚स्या}‚नेक‚कार्य‚कारित्व‚स्य ‚{\color{DodgerBlue3}‚प‚रिदीप‚न}‚निमित्त‚{\color{DodgerBlue3}‚म‚त‚त्कार्यार्थे}‚भ्य एकैक‚कार्य‚{\tiny $_{lb}$}‚स‚म‚र्थेभ्यो ‚{\color{DodgerBlue3}‚भेदेन} व्य‚व‚च्छेदेन ‚{\color{DodgerBlue3}‚नानाध‚र्मः प्र‚तीय‚ते} श‚ब्द‚विक‚ल्पैः । य‚थाऽनालोक‚{\tiny $_{lb}$}‚कारिभ्यो भेदादालोक‚कृद‚न‚न्ध‚कार‚ह‚न्तृभ्यो भेदाद‚न्ध‚कार‚ह‚न्ता दीप उच्य‚त ‚{\tiny $_{lb}$}‚इत्यादि ॥ (८२)
	\pend% ending standard par
      \label{div_pvv.3.83}
	  
	% new div opening: depth here is 2
	
	  \bigskip
	  \begingroup
	
	    \large
	  
	    \begin{quote}
	  
	    
	    \stanza[\smallbreak]
	\label{pv.3.83}\flagstanza{\tiny\textenglish{...v.3.83}}य‚थाप्र‚तीति क‚थितः श‚ब्दार्थोसाव‚स‚न्न‚पि ।&सामानाधिक‚र‚ण्यं च व‚स्तुन्य‚स्य न स‚म्भ‚वः ॥ ८३ ॥\&[\smallbreak]


	
	    \end{quote}
	  
	  \endgroup
	\textsuperscript{\textenglish{319/s}}

	  \pstart \leavevmode% starting standard par
	\hphantom{.}त‚देवं ‚{\color{DodgerBlue3}‚य‚था\edtext{}{\edlabel{pvv.319-1}\label{pvv.319-1}\lemma{था}\Bfootnote{य‚दि बाह्ये सामान्यादिव्य‚व‚हार‚स्त‚र्हि वास्त‚वः प्राप्त इत्याह ।}}प्र‚तीति} संव्य‚व‚हारान‚तिक्र‚मेण\edtext{}{\edlabel{pvv.319-2}\label{pvv.319-2}\lemma{मेण}\Bfootnote{विक‚ल्प‚बुद्ध्य‚नुरोधेन ।}} ‚{\color{DodgerBlue3}‚श‚ब्दार्थः} सामान्य‚ल‚क्ष‚णः ‚{\tiny $_{lb}$}‚सामानाधिक‚र‚ण्यं ‚{\color{DodgerBlue3}‚विशेष‚ण‚विशेष्य‚भाव‚श्च च श‚ब्दात् क‚थितोऽस‚न्न‚पि प‚र‚मार्थ‚तः । ‚{\tiny $_{lb}$}‚य‚तो व‚स्तुन्य‚स्य श‚ब्दार्थ‚स्य सामान्यादेः पार‚मार्थिक‚स्यासंभ‚वः । स‚र्व्व‚तो व्यावृत्त‚स्य ‚{\tiny $_{lb}$}‚व‚स्तुमात्र‚स्याध्य‚क्षेणोप‚ल‚म्भात् । त‚द्व्यावुत्त्याश्र‚येण क‚ल्प्य‚मानं सामान्यं त‚त्सामा‚{\tiny $_{lb}$}‚नाधिक‚र‚ण्यं} चाव‚स्त्वेव ॥ (८३)
	\pend% ending standard par
      \label{div_pvv.3.84}
	  
	% new div opening: depth here is 2
	
	  \bigskip
	  \begingroup
	
	    \large
	  
	    \begin{quote}
	  
	    
	    \stanza[\smallbreak]
	\label{pv.3.84}\flagstanza{\tiny\textenglish{...v.3.84}}ध‚र्म्म‚ध‚र्म्मिव्य‚व‚स्थानं भेदोऽभेद‚श्च यादृशः ।&अस‚मीक्षित‚त‚त्वोर्थो य‚था लोके प्र‚तीय‚ते ॥ ८४ ॥\&[\smallbreak]


	
	    \end{quote}
	  
	  \endgroup
	

	  \pstart \leavevmode% starting standard par
	\hphantom{.}त‚था ‚{\color{DodgerBlue3}‚ध‚र्म\edtext{}{\edlabel{pvv.319-3}\label{pvv.319-3}\lemma{र्म}\Bfootnote{अय‚म‚पि ज्ञान‚प्र‚तिभासिन्य‚र्थ इत्याह ।}} ध‚र्म्मिणोर्व्य‚व‚स्थानिय‚मः‚{\tiny $_{1}$}‚} श‚ब्दो ध‚र्म्येव कृत‚क‚त्वं ‚{\color{DodgerBlue3}‚ध‚र्म एव । त‚यो‚{\tiny $_{lb}$}‚र्भेदः\edtext{}{\edlabel{pvv.319-4}\label{pvv.319-4}\lemma{र्भेदः}\Bfootnote{श‚ब्द‚स्य कृत‚क‚मिति ।}}} श‚ब्दो ध‚र्मी कृत‚क‚त्वं ध‚र्म इत्य‚{\color{DodgerBlue3}‚भेद}‚श्च कृत‚कोऽनित्य‚श्च श‚ब्द ‚{\color{DodgerBlue3}‚इत्यादि (।) ‚{\tiny $_{lb}$}‚यादृशो} ध‚र्मान्त‚र‚प्र‚तिक्षेपाप्र‚तिक्षेपाभ्यामुक्तोऽ‚{\color{DodgerBlue3}‚स‚मीक्षित\edtext{}{\edlabel{pvv.319-5}\label{pvv.319-5}\lemma{मीक्षित}\Bfootnote{विक‚ल्पारोपित‚त्वात् ।}}त‚त्वोर्थो}‚ऽल‚क्षित‚त‚त्वो ‚{\tiny $_{lb}$}‚य‚था\edtext{}{\edlabel{pvv.319-6}\label{pvv.319-6}\lemma{था}\Bfootnote{बुद्ध्यारूढोप्य‚ध्य‚व‚सित‚त‚द्भाव‚त‚या ।}} लोके ‚{\color{DodgerBlue3}‚प्र‚तीय‚ते} ॥ (८४)
	\pend% ending standard par
      \label{div_pvv.3.85}
	  
	% new div opening: depth here is 2
	
	  \bigskip
	  \begingroup
	
	    \large
	  
	    \begin{quote}
	  
	    
	    \stanza[\smallbreak]
	\label{pv.3.85}\flagstanza{\tiny\textenglish{...v.3.85}}तं त‚थैव स‚माश्रित्य साध्य‚साध‚न‚संस्थितिः ।&प‚र‚मार्थाव‚ताराय विद्व‚द्भिर‚व‚क‚ल्प्य‚ते ॥ ८५ ॥\&[\smallbreak]


	
	    \end{quote}
	  
	  \endgroup
	

	  \pstart \leavevmode% starting standard par
	\hphantom{.}‚{\color{DodgerBlue3}‚तं} ध‚र्म्मिध‚र्म्मादिविभागं ‚{\color{DodgerBlue3}‚त‚थैव स‚माश्रित्य साध्य‚साध‚न‚संस्थितिर्व्विद्व‚द्‏भि‚{\tiny $_{lb}$}‚र‚व‚क‚ल्प्य‚ते} (।) ‚{\color{DodgerBlue3}‚प‚र‚मार्थे} व‚स्तुस्व‚भाव‚भूते क्ष‚णिक‚त्वादा‚{\color{DodgerBlue3}‚व‚व‚ताराय}‚{\tiny $_{2}$}‚ लोक‚स्य ‚{\tiny $_{lb}$}‚व‚स्तुत्व‚क्ष‚णिक‚त्व‚योर्भेदः प‚रः क‚ल्पितः व‚स्तु तु क्ष‚णिक‚मेव । त‚च्च साध्य‚साध‚न‚{\tiny $_{lb}$}‚क‚ल्प‚न‚या श‚क्यं प्र‚त्येतुं ॥(८५)
	\pend% ending standard par
      \label{div_pvv.3.86}
	  
	% new div opening: depth here is 2
	

	  \pstart \leavevmode% starting standard par
	क‚स्मात् पुन‚र्व्व‚स्तुनि सामान्य‚ध‚र्मिध‚र्मादि नास्तीत्याह (।)
	\pend% ending standard par
      
	  \bigskip
	  \begingroup
	
	    \large
	  
	    \begin{quote}
	  
	    
	    \stanza[\smallbreak]
	\label{pv.3.86}\flagstanza{\tiny\textenglish{...v.3.86}}संसृज्य‚न्ते न भिद्य‚न्ते स्व‚तोर्थाः पार‚मार्थिकाः ।&रूप‚मेक‚म‚नेक‚ञ्च तेषु बुद्धेरुप‚प्ल‚वः ॥ ८६ ॥\&[\smallbreak]


	
	    \end{quote}
	  
	  \endgroup
	

	  \pstart \leavevmode% starting standard par
	\hphantom{.}‚{\color{DodgerBlue3}‚पार‚मार्थिका अर्थाः स्व‚तः} स्व‚रूपेण ‚{\color{DodgerBlue3}‚न संसृज्य‚न्ते} (।) य‚तः सामान्यं व‚स्तु ‚{\tiny $_{lb}$}‚स्यात् । ‚{\color{DodgerBlue3}‚नापि भिद्य‚न्ते} कृत‚क‚त्व‚श‚ब्द‚त्वादिना य‚तो ध‚र्मिध‚र्म‚भेदो भ‚वेत् । य‚त्तु ‚{\color{DodgerBlue3}‚तेषु}\edtext{}{\edlabel{pvv.319-7}\label{pvv.319-7}\lemma{त्तु}\Bfootnote{ब‚हुष्व‚र्थेषु ।}} ‚{\tiny $_{lb}$}‚‚{\color{DodgerBlue3}‚रूप‚मेकं} गोत्वाद्य‚नुयायि । ‚{\color{DodgerBlue3}‚अनेक}‚ञ्च \edtext{}{\edlabel{pvv.319-8}\label{pvv.319-8}\lemma{ञ्च}\Bfootnote{एक‚त्रार्थ ।}}श‚ब्द‚कृत‚क‚त्वादि व्य‚व‚ह्रिय‚तेऽसौ ‚{\tiny $_{lb}$}‚बुद्धेर‚नादिवास‚नोप‚ह‚ताया ‚{\color{DodgerBlue3}‚उप‚ल्प‚वो} मिथ्योप‚द‚र्श‚नं ॥ (८६)
	\pend% ending standard par
      \textsuperscript{\textenglish{320/s}}\label{div_pvv.3.87}
	  
	% new div opening: depth here is 2
	
	  \bigskip
	  \begingroup
	
	    \large
	  
	    \begin{quote}
	  
	    
	    \stanza[\smallbreak]
	\label{pv.3.87}\flagstanza{\tiny\textenglish{...v.3.87}}भेद‚स्त‚तोपि बौद्धेऽर्थे सामान्यं भेद इत्य‚पि ।&त‚स्यैव चान्य‚व्यावृत्त्या ध‚र्म‚भेदः प्र‚क‚ल्प्य‚ते ॥ ८७ ॥\&[\smallbreak]


	
	    \end{quote}
	  
	  \endgroup
	

	  \pstart \leavevmode% starting standard par
	\hphantom{.}‚{\color{DodgerBlue3}‚त‚तो}‚यं विशेष इदं ‚{\color{DodgerBlue3}‚सामान्य}‚मिति ‚{\color{DodgerBlue3}‚भेद} इदं साध्य‚मिदं साध‚न‚मित्य‚पि ‚{\tiny $_{lb}$}‚भेदो ‚{\color{DodgerBlue3}‚बौद्धे} बुद्धिप‚रिक‚ल्पितेऽ‚{\color{DodgerBlue3}‚र्थे} न व‚स्तुनि । क‚थ‚न्त‚र्हि स्व‚ल‚क्ष‚णे कृत‚क‚त्वादि‚{\color{DodgerBlue3}‚भेद} इत्याह । ‚{\color{DodgerBlue3}‚त‚स्यैव} स्व‚ल‚क्ष‚ण‚स्यान्य‚स्माद‚कृत‚कादे‚{\color{DodgerBlue3}‚र्व्यावृत्या} त‚दाश्र‚येण ‚{\color{DodgerBlue3}‚ध‚र्म‚भेदः ‚{\tiny $_{lb}$}‚प्र‚क‚ल्प्य‚ते ॥}‚(८७)
	\pend% ending standard par
      \label{div_pvv.3.88}
	  
	% new div opening: depth here is 2
	

	  \pstart \leavevmode% starting standard par
	क‚स्मा‚{\tiny $_{4}$}‚त् क‚ल्पित‚भेद‚द्वारेण साध्य‚साध‚न‚भाव इष्टो न व‚स्तुभेदेनेत्याह ।
	\pend% ending standard par
      
	  \bigskip
	  \begingroup
	
	    \large
	  
	    \begin{quote}
	  
	    
	    \stanza[\smallbreak]
	\label{pv.3.88}\flagstanza{\tiny\textenglish{...v.3.88}}साध्य‚साध‚न‚संक‚ल्पे व‚स्तुद‚र्श‚न‚हानितः ।&भेदः सामान्य‚संसृष्टो ग्राह्यो नात्र स्व‚ल‚क्ष‚ण‚म् ॥ ८८ ॥\&[\smallbreak]


	
	    \end{quote}
	  
	  \endgroup
	

	  \pstart \leavevmode% starting standard par
	\hphantom{.}‚{\color{DodgerBlue3}‚साध्य‚साध‚न‚संक‚ल्पे} इदं साध्य‚मिदं साध‚न‚मिति विक‚ल्पे क्रिय‚माणे ‚{\tiny $_{lb}$}‚‚{\color{DodgerBlue3}‚व‚स्तुद‚र्श‚न‚स्य हानितः} क‚ल्पित एव भेदः । न ख‚लु विक‚ल्पे व‚स्तुद‚र्श‚न‚म‚स्ति ‚{\tiny $_{lb}$}‚क‚ल्पित‚गोच‚र\edtext{}{\edlabel{pvv.320-1}\label{pvv.320-1}\lemma{र}\Bfootnote{कुतः स्व‚ल‚क्ष‚ण‚स्य सामान्य‚स‚हित‚स्य ग्र‚ह इति वा ।}}त्वात् त‚स्य । आलोच‚नाज्ञानं व‚स्तुविष‚यं न किञ्चित् त‚द् ‚{\tiny $_{lb}$}‚विभ‚ज‚ति ।
	\pend% ending standard par
      

	  \pstart \leavevmode% starting standard par
	\hphantom{.}न‚न्वाचार्यादि ग्ना स्य ‚{\color{DodgerBlue3}‚भेदः सामान्य‚संसृष्टो ग्राह्य} इष्टः । ‚{\color{DodgerBlue3}‚अत्र}‚{\tiny $_{5}$}‚ भेदः सामा‚{\tiny $_{lb}$}‚न्य‚संसृष्टः प्र‚तीय‚ते इत्य‚स्मिन् व‚च‚ने ‚{\color{DodgerBlue3}‚न स्व‚ल‚क्ष‚णं} ग्राह्य‚त‚या निर्दिष्टं किन्त्व‚ध्य‚{\tiny $_{lb}$}‚व‚सेय‚त‚या\edtext{}{\edlabel{pvv.320-2}\label{pvv.320-2}\lemma{या}\Bfootnote{बाह्य एव भेदास्तेनापोह‚ल‚क्ष‚णेन सामान्येन संसृष्टा अध्य‚व‚सीय‚न्ते ।}}॥ (८८)
	\pend% ending standard par
      \label{div_pvv.3.89}
	  
	% new div opening: depth here is 2
	

	  \pstart \leavevmode% starting standard par
	क‚स्मादेव\edtext{}{\edlabel{pvv.320-3}\label{pvv.320-3}\lemma{स्मादेव}\Bfootnote{त‚त्र बोद्ध‚व्यं ।}}मित्याह ।
	\pend% ending standard par
      
	  \bigskip
	  \begingroup
	
	    \large
	  
	    \begin{quote}
	  
	    
	    \stanza[\smallbreak]
	\label{pv.3.89}\flagstanza{\tiny\textenglish{...v.3.89}}स‚मान‚भिन्नाद्याकारैर्न त‚द् ग्राह्यं क‚थ‚ञ्च‚न ।&भेदानां ब‚हुभेदानां त‚त्रैक‚स्मिन्न‚योग‚तः ॥ ८९ ॥\&[\smallbreak]


	
	    \end{quote}
	  
	  \endgroup
	

	  \pstart \leavevmode% starting standard par
	\hphantom{.}‚{\color{DodgerBlue3}‚स‚मान‚भिन्नाद्याकारैः} सामान्याकार‚ध‚र्मिध‚र्म‚भेदाकार‚सामानाधिक‚र‚ण्याद्याका‚{\tiny $_{lb}$}‚रैश्च ‚{\color{DodgerBlue3}‚त‚त्} स्व‚ल‚क्ष‚णं ‚{\color{DodgerBlue3}‚क‚थ‚ञ्च‚न ग्राह्यं} न भ‚व‚ति । किं कार‚ण‚मित्याह । ‚{\color{DodgerBlue3}‚भेदानां}\edtext{\textsuperscript{*}}{\edlabel{pvv.320-4}\label{pvv.320-4}\lemma{*}\Bfootnote{व‚स्तुरूपाणां ।}} ‚{\tiny $_{lb}$}‚ध‚र्माणां कृत‚क‚त्वादीनां ‚{\color{DodgerBlue3}‚ब‚हुभेदा}‚नाम‚नेक‚प्र‚काराणां ‚{\color{DodgerBlue3}‚त‚त्र} स्व‚ल\edtext{}{\edlabel{pvv.320-5}\label{pvv.320-5}\lemma{ल}\Bfootnote{निरंश‚त्वात् ।}}क्ष‚ण ‚{\color{DodgerBlue3}‚एक‚स्मिन्न‚{\tiny $_{lb}$}‚योग‚तः} ॥ (८९)
	\pend% ending standard par
      \textsuperscript{\textenglish{321/s}}\label{div_pvv.3.90}
	  
	% new div opening: depth here is 2
	
	  \bigskip
	  \begingroup
	
	    \large
	  
	    \begin{quote}
	  
	    
	    \stanza[\smallbreak]
	\label{pv.3.90}\flagstanza{\tiny\textenglish{...v.3.90}}त‚द्रूपं स‚र्व‚तो भिन्नं त‚था त‚त्प्र‚तिपादिका ।&न श्रुतिः क‚ल्प‚ना वास्ति सामान्येनैव वृत्तितः ॥ ९० ॥\&[\smallbreak]


	
	    \end{quote}
	  
	  \endgroup
	

	  \pstart \leavevmode% starting standard par
	\hphantom{.}त‚स्मात् त‚स्य स्व‚ल‚क्ष‚ण‚स्य ‚{\color{DodgerBlue3}‚रूपं स‚र्व्व‚तः} स‚जातीय‚विजातीयाद् ‚{\color{DodgerBlue3}‚भिन्नं त‚था} तेन स‚र्व्व‚तो भिन्नेन रूपेण ‚{\color{DodgerBlue3}‚त‚त्प्र‚तिपादिका श्रुतिः क‚ल्प‚ना वा नास्ति । सामान्येनैव} क‚ल्पितेन रूपेण श‚ब्द‚विक‚ल्प‚यो‚{\color{DodgerBlue3}‚र्वृत्तितः} ॥ (९०)
	\pend% ending standard par
      \label{div_pvv.3.91}
	  
	% new div opening: depth here is 2
	

	  \pstart \leavevmode% starting standard par
	किं पुनः स्व‚ल‚क्ष‚ण‚मेव श‚ब्दैर्नोच्य‚ते इत्याह\edtext{}{\edlabel{pvv.321-1}\label{pvv.321-1}\lemma{इत्याह}\Bfootnote{येन त‚त्प्र‚तिपादिका न श्रुतिः ।}} ।
	\pend% ending standard par
      
	  \bigskip
	  \begingroup
	
	    \large
	  
	    \begin{quote}
	  
	    
	    \stanza[\smallbreak]
	\label{pv.3.91}\flagstanza{\tiny\textenglish{...v.3.91}}श‚ब्दाः संकेतितं प्राहुर्व्य‚व‚हाराय स स्मृतः ।&त‚दा स्व‚ल‚क्ष‚णं नास्ति स‚ङ्केत‚स्तेन त‚त्र न ॥ ९१ ॥\&[\smallbreak]


	
	    \end{quote}
	  
	  \endgroup
	\textsuperscript{\textenglish{64a/MA}}

	  \pstart \leavevmode% starting standard par
	\hphantom{.}‚{\color{DodgerBlue3}‚श‚ब्दाः संकेतित}‚म‚र्थ‚मा‚{\color{DodgerBlue3}‚हुर्न} यं क‚ञ्चित् ।\edtext{\textsuperscript{*}}{\edlabel{pvv.321-2}\label{pvv.321-2}\lemma{*}\Bfootnote{सामान्येनैवेत्युक्तेप्य‚धिक‚प‚रिहारायोप‚न्यासः ।}} ‚{\color{DodgerBlue3}‚स} संकेतो ‚{\color{DodgerBlue3}‚व्य‚व‚हाराय ‚{\tiny $_{lb}$}‚स्मृतः} (।) संकेतित\edtext{}{\edlabel{pvv.321-3}\label{pvv.321-3}\lemma{संकेतित}\Bfootnote{संकेत‚विष‚यीकृतं ।}}म‚र्थं श‚ब्दा‚{\tiny $_{7}$}‚दुच्च‚रितात् प्र‚तिप‚द्येय‚मिति संकेत‚ग्र‚ह‚णं । ‚{\tiny $_{lb}$}‚‚{\color{DodgerBlue3}‚त‚दा} व्य‚व‚हार‚काले च ‚{\color{DodgerBlue3}‚स्व‚ल‚क्ष‚णं}\edtext{}{\edlabel{pvv.321-4}\label{pvv.321-4}\lemma{च}\Bfootnote{एक‚स्व‚ल‚क्ष‚ण‚स्यापि क्ष‚णिक‚त्वान्नानुग‚मोऽक्ष‚णिक‚स्यापि संकेत‚ज्ञानाज‚न‚{\tiny $_{lb}$}‚क‚त्वात् । किमुत देश‚काल‚भिन्नेषु ।}} संकेत‚विष‚यो ‚{\color{DodgerBlue3}‚नास्ति तेन त‚त्र} स्व‚ल‚क्ष‚णे ‚{\color{DodgerBlue3}‚संकेतो ‚{\tiny $_{lb}$}‚न} युक्तः ॥ (९१)
	\pend% ending standard par
      \label{div_pvv.3.92}
	  
	% new div opening: depth here is 2
	

	  \pstart \leavevmode% starting standard par
	एव‚न्त‚र्हि सामान्ये\edtext{}{\edlabel{pvv.321-5}\label{pvv.321-5}\lemma{सामान्ये}\Bfootnote{वै शे षि क स्य व्य‚तिरिक्ते सां ख्य स्याव्य‚तिरिक्ते ।}} व्य‚व‚हार‚कालानुयायिनि संकेतः \edtext{}{\edlabel{pvv.321-6}\label{pvv.321-6}\lemma{संकेतः}\Bfootnote{भ‚व‚तु ।}} स्यादित्याह ।
	\pend% ending standard par
      
	  \bigskip
	  \begingroup
	
	    \large
	  
	    \begin{quote}
	  
	    
	    \stanza[\smallbreak]
	\label{pv.3.92}\flagstanza{\tiny\textenglish{...v.3.92}}अपि प्र‚व‚र्त्तेत पुमान् विज्ञायार्थ‚क्रियाक्ष‚मान् ।&त‚त्साध‚नायेत्य‚र्थेषु संयोज्य‚न्तेऽभिधाक्रियाः ॥ ९२ ॥\&[\smallbreak]


	
	    \end{quote}
	  
	  \endgroup
	

	  \pstart \leavevmode% starting standard par
	\hphantom{.}‚{\color{DodgerBlue3}‚अपि\edtext{}{\edlabel{pvv.321-7}\label{pvv.321-7}\lemma{अपि}\Bfootnote{क‚थं नाम ।}}प्र‚व‚र्त्तेत पुमान्} श‚ब्दाद् ‚{\color{DodgerBlue3}‚विज्ञायार्थ‚क्रियाक्ष‚मान्} । त‚स्या अर्थ‚क्रियायाः ‚{\tiny $_{lb}$}‚‚{\color{DodgerBlue3}‚साध‚नायेति} । एत‚द‚र्थ‚म‚{\color{DodgerBlue3}‚र्थेषु संयोज्य‚न्ते} संकेत्य‚न्ते श‚ब्दा न व्य‚स‚नित‚या ॥ (९२)
	\pend% ending standard par
      \label{div_pvv.3.93}
	  
	% new div opening: depth here is 2
	
	  \bigskip
	  \begingroup
	
	    \large
	  
	    \begin{quote}
	  
	    
	    \stanza[\smallbreak]
	\label{pv.3.93}\flagstanza{\tiny\textenglish{...v.3.93}}अत्रान‚र्थ‚क्रियायोग्यो नास्ति त‚द्वान‚लं स च ।&साक्षान्न योज्य‚ते क‚स्मादान‚न्त्याच्चेदिदं स‚म‚म् ॥ ९३ ॥\&[\smallbreak]


	
	    \end{quote}
	  
	  \endgroup
	

	  \pstart \leavevmode% starting standard par
	त‚त्रैवं स‚त्य‚न‚र्थ‚क्रियायां योग्योऽर्थ‚क्रियायां‚{\tiny $_{1}$}‚ श‚क्ता जातिरिति न त‚त्र स‚ङ्केतो ‚{\tiny $_{lb}$}‚युक्तः । ‚{\color{DodgerBlue3}‚त‚द्वान्} जातिमान् विशेषो‚{\color{DodgerBlue3}‚ऽलं} श‚क्तो\edtext{}{\edlabel{pvv.321-8}\label{pvv.321-8}\lemma{क्तो}\Bfootnote{सामान्येन श‚ब्द‚ल‚क्षितेन स‚म्ब‚न्धाद् व्य‚क्तिर‚पि ल‚क्ष्य‚ते}}र्थ‚क्रियायामिति त‚त्र संकेत ‚{\tiny $_{lb}$}‚इति चेत् । ‚{\color{DodgerBlue3}‚स} विशेष‚श्च ‚{\color{DodgerBlue3}‚साक्षात्} संङ्केते ‚{\color{DodgerBlue3}‚क‚स्मान्न योज्य‚ते} (।) किं जातिव्य‚व‚धि‚{\tiny $_{lb}$}‚\leavevmode\ledsidenote{\textenglish{322/s}} स्वीकारेण अन‚र्थ‚क्रियाकारित्वाद‚स्याः स्व‚ल‚क्ष‚ण‚स्य च विप‚र्य‚यात् । व्य‚क्तीना‚{\color{DodgerBlue3}‚मा‚{\tiny $_{lb}$}‚न‚न्त्यान्न} त‚त्र श‚क्य इति ‚{\color{DodgerBlue3}‚चेत् । इद}‚मान‚न्त्यं ‚{\color{DodgerBlue3}‚स‚मं}\edtext{}{\edlabel{pvv.322-1}\label{pvv.322-1}\lemma{न्त्यं}\Bfootnote{न हि गोश‚ब्दाद् गोत्व‚बुद्ध्या व्य‚क्तिबोधोस्ति । सामान्ये वातोदिते क‚थं ‚{\tiny $_{lb}$}‚व्य‚क्तौ वृतिः । न हि स‚म्ब‚न्धेपि द‚ण्डं (ि) च्छ‚न्धीति द‚ण्डिनं छिन‚त्ति क‚श्चित् ।}} जातिम‚द्व्य‚क्तिष्व‚पि\edtext{}{\edlabel{pvv.322-2}\label{pvv.322-2}\lemma{पि}\Bfootnote{जातौ कृते संकेते व्य‚क्ताव‚प्र‚तीतिर्न च जातित‚द्व‚तोः स‚म्ब‚न्धोऽत‚दुत्प‚त्तेः ।}}॥ (९३)
	\pend% ending standard par
      \label{div_pvv.3.94}
	  
	% new div opening: depth here is 2
	
	  \bigskip
	  \begingroup
	
	    \large
	  
	    \begin{quote}
	  
	    
	    \stanza[\smallbreak]
	\label{pv.3.94}\flagstanza{\tiny\textenglish{...v.3.94}}त‚त्कारिणाम‚त‚त्कारिभेद‚साम्ये न किं कृतः ।&त‚द्व‚द्दोष‚स्य साम्याच्चेद‚स्तु जातिर‚लं प‚रा ॥ ९४ ॥\&[\smallbreak]


	
	    \end{quote}
	  
	  \endgroup
	

	  \pstart \leavevmode% starting standard par
	\hphantom{.}किञ्च याम‚र्थ‚क्रियामुद्दिश्य श‚ब्द‚नियोग‚{\color{DodgerBlue3}‚स्त‚त्कारि‚{\tiny $_{2}$}‚णा}‚म‚र्थाना‚{\color{DodgerBlue3}‚म‚त‚त्कारि}‚भ्यो ‚{\tiny $_{lb}$}‚यो ‚{\color{DodgerBlue3}‚भेदो} व्य‚व‚च्छेद‚स्त‚देव ‚{\color{DodgerBlue3}‚साम्यं} सामान्य‚म‚न्यापोहः साधार‚ण‚त्वात् । त‚त्र ‚{\color{DodgerBlue3}‚किं} संकेतो न कृतः सामान्य‚व‚द‚पोहोपि साधार‚णोऽन‚र्थ‚क्रियाकारी स्व‚ल‚क्ष‚ण‚स‚म्ब‚न्धात् ‚{\tiny $_{lb}$}‚त‚दुप‚ल‚भ्य‚रूप‚श्चः । ‚{\color{DodgerBlue3}‚त‚द्व‚द्दोष‚स्य} जाति\edtext{}{\edlabel{pvv.322-3}\label{pvv.322-3}\lemma{जाति}\Bfootnote{जातिम‚त्प‚क्षें यो [दिग्] नागोक्त‚दोषः । त‚द्व‚तो नास्व‚त‚न्त्र‚त्वादित्यादिना ‚{\tiny $_{lb}$}‚त‚द्दोषाव‚ताराद् भेदेन्य‚व्यावृत्तिल‚क्ष‚णे न श‚ब्द‚नियोगः जातिक‚ल्प‚न‚म‚न‚र्थ‚निर्व्व‚न्ध ‚{\tiny $_{lb}$}‚एव नित्य‚व्यापिताद्य‚योगाद‚य‚माश‚यो व्यावृत्त्य‚मिधाने नाग‚स्य जातिर्न प्र‚वृत्तियोग्या ‚{\tiny $_{lb}$}‚त‚द्द्वातोदितेपि न वृत्तिरित्युक्तेः अर्थाश‚क्तेः ।}}म‚ति संकेत‚विष‚ये आन‚न्त्यात् संकेताक‚र‚ण‚स्य ‚{\tiny $_{lb}$}‚‚{\color{DodgerBlue3}‚साम्याच्चेद‚स्तु} दोषः स‚मान‚त्वात् (।) द्व‚योर‚स्त ‚{\color{DodgerBlue3}‚जातिः प‚राऽल‚म‚नु}‚प‚युक्ता ‚{\tiny $_{lb}$}‚जातिम‚भ्युपेत्यापि विजातीय‚व्य‚व‚च्छेदो‚{\tiny $_{3}$}‚ऽ\edtext{}{\edlabel{pvv.322-4}\label{pvv.322-4}\lemma{ऽ}\Bfootnote{भावानाम‚न्य‚व्यावृत्त्य‚भावे व‚स्तुभूताऽनेकास‚म‚वेता जातिर‚पि न स्यात् ।}} व‚श्याश्र‚य‚णीयः । य‚दि गौर‚श्वादिभ्यो न ‚{\tiny $_{lb}$}‚व्य‚व‚च्छिन्न‚स्त‚दाऽश्व एव स्यात् । य‚श्चैक‚स्य व्य‚व‚च्छेदः\edtext{}{\edlabel{pvv.322-5}\label{pvv.322-5}\lemma{च्छेदः}\Bfootnote{विजातीयाद् भेद एव स‚जातीयाभेदः इति आन‚न्त्यादिदोषः ख‚ण्डितः ‚{\tiny $_{lb}$}‚स्यात् जातिध‚र्म‚श्च द‚र्शित इति व्य‚व‚च्छेदं त्य‚क्त्वा ।}} स स‚र्व्व‚स्य (।) स एव ‚{\tiny $_{lb}$}‚स‚ङ्केत‚विष‚यो\edtext{}{\edlabel{pvv.322-6}\label{pvv.322-6}\lemma{यो}\Bfootnote{व्य‚व‚च्छेद‚विशिष्टोर्थः ।}}ऽस्तु किं प्र‚माण‚बाधित‚जातिस्वीकारेण । अत एव त‚द्व‚द्दोषोपि न ‚{\tiny $_{lb}$}‚स‚म्भ‚व‚ति । अभ्युप‚ग‚म्य तु सांप्र‚त‚मापादितं ॥ (९४) किञ्च (।)
	\pend% ending standard par
      \label{div_pvv.3.95}
	  
	% new div opening: depth here is 2
	
	  \bigskip
	  \begingroup
	
	    \large
	  
	    \begin{quote}
	  
	    
	    \stanza[\smallbreak]
	\label{pv.3.95}\flagstanza{\tiny\textenglish{...v.3.95}}त‚द‚न्य‚प‚रिहारेण प्र‚व‚र्त्तेतेति च ध्व‚निः ।&तेन तेभ्योऽव्य‚व‚च्छेदे प्र‚व‚र्त्तेत क‚थ‚ञ्च सः ॥ ९५ ॥\&[\smallbreak]


	
	    \end{quote}
	  
	  \endgroup
	

	  \pstart \leavevmode% starting standard par
	त‚स्मा\edtext{}{\edlabel{pvv.322-7}\label{pvv.322-7}\lemma{स्मा}\Bfootnote{अधुना श‚ब्देनाव‚श्यं व्यावृत्तिश्चोद‚नीयेत्याह ।}}देकार्थ‚क्रियाकारिणो\edtext{}{\edlabel{pvv.322-8}\label{pvv.322-8}\lemma{क्रियाकारिणो}\Bfootnote{श्रोत्रा प्र‚तिपाद‚केन ।}}‚{\color{DodgerBlue3}‚ऽन्य}‚स्य ‚{\color{DodgerBlue3}‚प‚रिहा}‚रेण श‚ब्दात् ‚{\color{DodgerBlue3}‚प्र‚व‚र्त्त‚तेति ‚{\tiny $_{lb}$}‚ध्व}‚निरुच्य‚ते । ‚{\color{DodgerBlue3}‚तेन} ध्व‚निना ‚{\color{DodgerBlue3}‚तेभ्यो}‚ऽत‚त्कारिभ्योऽ‚{\color{DodgerBlue3}‚स्य} त‚त्कारिणोऽव्य‚व‚च्छे‚{\tiny $_{4}$}‚दे ‚{\tiny $_{lb}$}‚\leavevmode\ledsidenote{\textenglish{323/s}} व्य‚व‚च्छेदेऽक्रिय‚माणे ‚{\color{DodgerBlue3}‚क‚थं} स श्रोता प्र‚तिनिय‚त‚प‚दार्थार्थी\edtext{}{\edlabel{pvv.323-1}\label{pvv.323-1}\lemma{दार्थार्थी}\Bfootnote{आन‚येति स‚र्व्व‚स्य अग्निमिति व्य‚व‚च्छेद‚वैय‚र्थ्यं अव्य‚व‚च्छेदेनाभिधाने ।}} प्र‚व‚र्त्तेत ‚{\color{DodgerBlue3}‚विष‚यान‚भि‚{\tiny $_{lb}$}‚धानात्} ॥ (९५)
	\pend% ending standard par
      \label{div_pvv.3.96}
	  
	% new div opening: depth here is 2
	

	  \pstart \leavevmode% starting standard par
	अथ श‚ब्दैर्व्य‚व‚च्छेदः क्रिय‚त एव प्र‚वृत्तिविष‚य‚स्तु जातिरुच्य‚त इत्याह ।
	\pend% ending standard par
      
	  \bigskip
	  \begingroup
	
	    \large
	  
	    \begin{quote}
	  
	    
	    \stanza[\smallbreak]
	\label{pv.3.96}\flagstanza{\tiny\textenglish{...v.3.96}}व्य‚व‚च्छेदोस्ति चेद‚स्य न‚न्वेताव‚त् प्र‚योज‚न‚म् ।&श‚ब्दानामिति किं त‚त्र सामान्येनाप‚रेण वः ॥ ९६ ॥\&[\smallbreak]


	
	    \end{quote}
	  
	  \endgroup
	

	  \pstart \leavevmode% starting standard par
	\hphantom{.}‚{\color{DodgerBlue3}‚अस्य} जातिम‚तो ‚{\color{DodgerBlue3}‚व्य‚व‚च्छेदोऽस्ति चेत् । न‚न्वेताव‚द‚न्य‚व्य‚व‚च्छेदे\edtext{}{\edlabel{pvv.323-2}\label{pvv.323-2}\lemma{च्छेदे}\Bfootnote{त‚द‚स्वीकारे न जातिरित्युक्तं ।}}न प्र‚व‚र्त्त‚नं ‚{\tiny $_{lb}$}‚श‚ब्दानां प्र‚योज‚न}‚मिष्ट‚{\color{DodgerBlue3}‚मिति सामान्येनाप‚रेण} किं कार्यं वः । त‚द‚न्त‚रेण च श‚ब्दा‚{\tiny $_{lb}$}‚द‚न्य‚व्य‚व‚च्छेदाभिधानेपि प्र‚वृत्तिसंभ‚वात् ॥ (९६)
	\pend% ending standard par
      \label{div_pvv.3.97}
	  
	% new div opening: depth here is 2
	

	  \begin{center}%% label @type='head'
	\textbf{(६) सामान्याभावे प्र‚त्य‚भिज्ञासंग‚तिः}
	\end{center}
	

	  \pstart \leavevmode% starting standard par
	य‚दि नास्ति जातिस्त‚दा भिन्न‚स्व‚भावेषु भा‚{\tiny $_{5}$}‚वेषु स एवायं गौरित्यादि प्र‚त्य‚{\tiny $_{lb}$}‚भिज्ञानं न स्यादित्याह ।
	\pend% ending standard par
      
	  \bigskip
	  \begingroup
	
	    \large
	  
	    \begin{quote}
	  
	    
	    \stanza[\smallbreak]
	\label{pv.3.97}\flagstanza{\tiny\textenglish{...v.3.97}}ज्ञानाद्य‚र्थ‚क्रियां तांस्तां दृष्ट्वा भेदेन कुर्व‚तः ।&अर्थान्त‚द‚न्य‚विश्लेष‚विष‚यैर्ध्व‚निभिः स‚ह ॥ ९७ ॥\&[\smallbreak]


	
	    \end{quote}
	  
	  \endgroup
	

	  \pstart \leavevmode% starting standard par
	\hphantom{.}‚{\color{DodgerBlue3}‚ज्ञान‚मादि}‚र्य‚स्या बाहा‚{\color{DodgerBlue3}‚द्य‚र्थ‚क्रिया}‚यास्तामेकाकार‚प‚राम‚र्श‚विष‚यां ‚{\color{DodgerBlue3}‚भेदेन} नानात्वेपि ‚{\tiny $_{lb}$}‚‚{\color{DodgerBlue3}‚कुर्व्व‚तो}‚र्थान् ‚{\color{DodgerBlue3}‚दृष्ट्वा त‚द‚न्य}‚स्माद् यो ‚{\color{DodgerBlue3}‚विश्लेषः} स ‚{\color{DodgerBlue3}‚विष‚यो} येषां ‚{\color{DodgerBlue3}‚तैर्द्ध‏्् व‚निभिः ‚{\tiny $_{lb}$}‚स‚ह} ॥ (९७)
	\pend% ending standard par
      \label{div_pvv.3.98}
	  
	% new div opening: depth here is 2
	
	  \bigskip
	  \begingroup
	
	    \large
	  
	    \begin{quote}
	  
	    
	    \stanza[\smallbreak]
	\label{pv.3.98}\flagstanza{\tiny\textenglish{...v.3.98}}संयोज्य प्र‚त्य‚भिज्ञानं पूर्व‚दृष्टान्य‚द‚र्श‚ने ।&प‚र‚स्यापि न सा बुद्धिः सामान्यादेव केव‚लात् ॥ ९८ ॥\&[\smallbreak]


	
	    \end{quote}
	  
	  \endgroup
	

	  \pstart \leavevmode% starting standard par
	\hphantom{.}‚{\color{DodgerBlue3}‚संयोज्य} स एवायं गौरित्यादि‚{\color{DodgerBlue3}‚प्र‚त्य‚भिज्ञानं पूर्व‚दृष्टा}‚द‚र्थाद् विल‚क्ष‚ण‚स्य ‚{\color{DodgerBlue3}‚द‚र्श‚नेपि} कुर्यात् । त‚द‚न्य‚विश्लेष‚स्य स‚र्व्व‚त्र साम्यात् ॥
	\pend% ending standard par
      

	  \pstart \leavevmode% starting standard par
	\hphantom{.}किञ्च (।) य‚दि नानात्वात् प्र‚त्य‚भिज्ञान‚म‚युक्तं त‚दा ‚{\color{DodgerBlue3}‚प‚र‚स्यापि न सा} प्र‚त्य‚{\tiny $_{lb}$}‚भिज्ञा ‚{\tiny $_{6}$}‚ ‚{\color{DodgerBlue3}‚बुद्धिः सामान्यादेव केव‚लादिष्टा} किन्तु सामान्य‚विशेषाभ्यां\edtext{}{\edlabel{pvv.323-3}\label{pvv.323-3}\lemma{विशेषाभ्यां}\Bfootnote{मी मां स क स्य म‚तेन ।}}। (९८)
	\pend% ending standard par
      \label{div_pvv.3.99_3.100_3.101}
	  
	% new div opening: depth here is 2
	

	  \pstart \leavevmode% starting standard par
	क‚स्मादित्याह (।)
	\pend% ending standard par
      
	  \bigskip
	  \begingroup
	
	    \large
	  
	    \begin{quote}
	  
	    
	    \stanza[\smallbreak]
	\label{pv.3.99a}\flagstanza{\tiny\textenglish{....3.99a}}नित्यं त‚न्मात्र‚विज्ञाने व्य‚क्त‏्य‚ज्ञान‚प्र‚स‚ङ्ग‚तः ।\&[\smallbreak]


	
	    \end{quote}
	  
	  \endgroup
	

	  \pstart \leavevmode% starting standard par
	\hphantom{.}‚{\color{DodgerBlue3}‚नित्य}‚त‚या ‚{\color{DodgerBlue3}‚त‚स्य} सामान्य‚{\color{DodgerBlue3}‚मात्र‚स्य विज्ञाने व्य‚क्त्य‚ज्ञान‚प्र‚स‚ङ्ग‚तः} ॥
	\pend% ending standard par
      \textsuperscript{\textenglish{324/s}}

	  \begin{center}%% label @type='head'
	\textbf{(ङ) त‚द्व‚त्तानिश्च‚यः}
	\end{center}
	
	  \bigskip
	  \begingroup
	
	    \large
	  
	    \begin{quote}
	  
	    
	    \stanza[\smallbreak]
	\label{pv.3.99b}\flagstanza{\tiny\textenglish{....3.99b}}त‚दा क‚दाचित् स‚म्ब‚द्ध‚स्यागृहीत‚स्य त‚द्व‚तः ॥ ९९ ॥\&[\smallbreak]


	
	    \end{quote}
	  
	  \endgroup
	
	  \bigskip
	  \begingroup
	
	    \large
	  
	    \begin{quote}
	  
	    
	    \stanza[\smallbreak]
	\label{pv.3.100a}\flagstanza{\tiny\textenglish{...3.100a}}त‚द्व‚त्तानिश्च‚यो न स्याद् व्य‚व‚हार‚स्त‚तः क‚थ‚म् ।\&[\smallbreak]


	
	    \end{quote}
	  
	  \endgroup
	

	  \pstart \leavevmode% starting standard par
	\hphantom{.}य‚दा सामान्य‚ज्ञान‚स्य न विशेषो विष‚य‚{\color{DodgerBlue3}‚स्त‚दा क‚दाचिद‚गृहीत‚स्य स‚म्ब‚द्ध‚स्य} सामान्येन ‚{\color{DodgerBlue3}‚त‚द्व‚तो} विशेष‚स्य (९९) ‚{\color{DodgerBlue3}‚त‚द्व‚त्तानिश्च‚य} इद‚म‚स्य सामान्य‚मिति ज्ञानं ‚{\color{DodgerBlue3}‚न ‚{\tiny $_{lb}$}‚स्यात्} । त‚त‚स्त‚द्व‚त्ताव्य‚व‚हारः क‚थं त्व‚न्म‚ते ॥
	\pend% ending standard par
      
	  \bigskip
	  \begingroup
	
	    \large
	  
	    \begin{quote}
	  
	    
	    \stanza[\smallbreak]
	\label{pv.3.100b}\flagstanza{\tiny\textenglish{...3.100b}}एक‚व‚स्तुस‚हायाश्चेद् व्य‚क्त‚यो ज्ञान‚कार‚ण‚म् ॥ १०० ॥\&[\smallbreak]


	
	    \end{quote}
	  
	  \endgroup
	
	  \bigskip
	  \begingroup
	
	    \large
	  
	    \begin{quote}
	  
	    
	    \stanza[\smallbreak]
	\label{pv.3.101}\flagstanza{\tiny\textenglish{....3.101}}त‚देकं व‚स्तु किं तासां नानात्वं स‚म‚पोह‚ति ।&नानात्वाच्चैक‚विज्ञान‚हेतुता तासु नेष्य‚ते ॥ १०१ ॥\&[\smallbreak]


	
	    \end{quote}
	  
	  \endgroup
	\textsuperscript{\textenglish{64b/MA}}

	  \pstart \leavevmode% starting standard par
	एकं व‚स्तु सामान्यं त‚त्स‚हायाश्चेद् व्य‚क्त‚यो ज्ञान‚स्य प्र‚त्य‚भि‚{\tiny $_{7}$}‚ज्ञान‚स्य कार‚ण‚{\tiny $_{lb}$}‚मिष्य‚न्ते (।) ‚{\color{DodgerBlue3}‚त‚देकं} सामान्यं ‚{\color{DodgerBlue3}‚व‚स्तु किन्तासां} व्य‚क्तीनाम‚मिश्र‚स्व‚भावानां ‚{\color{DodgerBlue3}‚नानात्वं ‚{\tiny $_{lb}$}‚स‚म‚पोह‚ति} (।) येन प्र‚त्य‚भिज्ञान‚हेतुत्वे ।\edtext{\textsuperscript{*}}{\edlabel{pvv.324-1}\label{pvv.324-1}\lemma{*}\Bfootnote{किं पुन‚स्तासां नानात्वापोहः भाष्य‚त इत्याह ।}} ‚{\color{DodgerBlue3}‚नानात्वाच्चैक‚स्य\edtext{}{\edlabel{pvv.324-2}\label{pvv.324-2}\lemma{स्य}\Bfootnote{ह्य‚र्थे चः ।}}} प्र‚त्य‚भिज्ञा‚{\tiny $_{lb}$}‚‚{\color{DodgerBlue3}‚ज्ञान‚स्य हेतुता तासु} त्व‚या ‚{\color{DodgerBlue3}‚नेष्य‚ते} । त‚च्चेत् त‚थैव\edtext{}{\edlabel{pvv.324-3}\label{pvv.324-3}\lemma{थैव}\Bfootnote{व्य‚क्तिषु ।}} न\edtext{}{\edlabel{pvv.324-4}\label{pvv.324-4}\lemma{न}\Bfootnote{पूर्व‚व‚द् व्य‚क्तिग्र‚हः ।}} स्यात्\edtext{}{\edlabel{pvv.324-5}\label{pvv.324-5}\lemma{स्यात्}\Bfootnote{च एवार्थे ।}}प्र‚त्य‚भिज्ञा‚{\tiny $_{lb}$}‚न‚म् ॥ (१००, १०१)
	\pend% ending standard par
      \label{div_pvv.3.102}
	  
	% new div opening: depth here is 2
	
	  \bigskip
	  \begingroup
	
	    \large
	  
	    \begin{quote}
	  
	    
	    \stanza[\smallbreak]
	\label{pv.3.102}\flagstanza{\tiny\textenglish{....3.102}}अनेक‚म‚पि य‚द्येक‚म‚पेक्ष्याभिन्न‚बुद्धिकृत् ।&ताभिर्विनापि प्र‚त्येकं क्रिय‚माणान्धियं प्र‚ति ॥ १०२ ॥\&[\smallbreak]


	
	    \end{quote}
	  
	  \endgroup
	
	  \bigskip
	  \begingroup
	
	    \large
	  
	    \begin{quote}
	  
	    
	    \stanza[\smallbreak]
	\label{pv.3.103a}\flagstanza{\tiny\textenglish{...3.103a}}तेनैकेनापि सामान्यात् तासां नेत्य‚ग्र‚हे धिया ।\&[\smallbreak]


	
	    \end{quote}
	  
	  \endgroup
	

	  \pstart \leavevmode% starting standard par
	\hphantom{.}‚{\color{DodgerBlue3}‚अनेक‚म‚पि} व्य‚क्तिरूपं ‚{\color{DodgerBlue3}‚य‚द्येकं} सामान्य‚{\color{DodgerBlue3}‚म‚पेक्ष्याभिन्न‚बुद्धिकृत्} प्र‚त्य‚भिज्ञान‚कारी‚{\tiny $_{lb}$}‚ष्य‚ते । ‚{\color{DodgerBlue3}‚ताभि}‚र्व्य‚क्तिभिः प्र‚त्येकं\edtext{}{\edlabel{pvv.324-6}\label{pvv.324-6}\lemma{त्येकं}\Bfootnote{न स‚म‚स्ताभिः किन्तु प्र‚त्येकं शाव‚लेयाभावे बाहुलेये गोबुद्धिस्त‚द‚भावे‚{\tiny $_{lb}$}‚न्य‚त्र (।) एवं स‚र्व्वासां प्र‚त्येक‚म‚भावेपि ।}} ‚{\color{DodgerBlue3}‚विनैकै}‚क‚या प्रेरितेन सामान्येनैकेन ‚{\color{DodgerBlue3}‚क्रिय‚माणां ‚{\tiny $_{lb}$}‚धियं‚{\tiny $_{1}$}‚} प्र‚त्य‚भिज्ञां प्र‚तिभासां व्य‚क्तीनां\edtext{}{\edlabel{pvv.324-7}\label{pvv.324-7}\lemma{क्तीनां}\Bfootnote{प्र‚त्येकं व्य‚क्त्य‚भावेपि ज्ञान‚भावात् ।}}साम‚र्थ्यं नेति त‚या धिया तासाम‚ग्र‚हः । ‚{\tiny $_{lb}$}‚केव‚लं सामान्य‚ग्र‚ह‚णे च त‚द्व‚त्ता निश्च‚यो न स्यादिति दुःप‚रिह‚रं ॥ (१०२)
	\pend% ending standard par
      \label{div_pvv.3.103}
	  
	% new div opening: depth here is 2
	

	  \pstart \leavevmode% starting standard par
	न‚नु य‚था नीलादीनामेकैकापायेपि च‚क्षुर्व्विज्ञानं दृष्टं त‚त्स‚मुदायेपि दृश्य‚ते । ‚{\tiny $_{lb}$}‚एवं य‚द्य‚पि प्र‚त्येकं व्य‚क्तीनां व्य‚भिचार‚स्त‚थापि य‚दा स‚त्वं\edtext{}{\edlabel{pvv.324-8}\label{pvv.324-8}\lemma{त्वं}\Bfootnote{बाह्य‚म‚ध्य‚व‚स्य‚तीत्य‚लीकेति न स‚म्वाद इत्याह ।}} त‚दा विष‚य‚त्व‚{\tiny $_{lb}$}‚मित्याह ॥
	\pend% ending standard par
      \textsuperscript{\textenglish{325/s}}
	  \bigskip
	  \begingroup
	
	    \large
	  
	    \begin{quote}
	  
	    
	    \stanza[\smallbreak]
	\label{pv.3.103b}\flagstanza{\tiny\textenglish{...3.103b}}नीलादेर्नेत्र‚विज्ञाने पृथ‚क् साम‚र्थ्य‚द‚र्श‚नात् ॥ १०३ ॥\&[\smallbreak]


	
	    \end{quote}
	  
	  \endgroup
	

	  \pstart \leavevmode% starting standard par
	\hphantom{.}‚{\color{DodgerBlue3}‚नीलादेर्नेत्र‚विज्ञाने} कार्ये ‚{\color{DodgerBlue3}‚पृथ‚क्} प्र‚त्येकं ‚{\color{DodgerBlue3}‚साम‚र्थ्य‚द‚र्श‚नात्} । (१०३)
	\pend% ending standard par
      \label{div_pvv.3.104}
	  
	% new div opening: depth here is 2
	
	  \bigskip
	  \begingroup
	
	    \large
	  
	    \begin{quote}
	  
	    
	    \stanza[\smallbreak]
	\label{pv.3.104a}\flagstanza{\tiny\textenglish{...3.104a}}श‚क्तिसिद्धिः स‚मूहेपि नैवं व्य‚क्तेः क‚थ‚ञ्च‚न ।\&[\smallbreak]


	
	    \end{quote}
	  
	  \endgroup
	

	  \pstart \leavevmode% starting standard par
	\hphantom{.}‚{\color{DodgerBlue3}‚श‚क्तिसिद्धिः स‚मूहेपि युक्ता । एवं व्य‚क्तेर्न्न} प्र‚त्य‚भिज्ञा‚{\tiny $_{2}$}‚न‚ज‚न‚ने साम‚र्थ्यं ‚{\tiny $_{lb}$}‚क‚थ‚ञ्च‚न दृष्टं येन स‚मुदायेपि साम‚र्थ्य‚क‚ल्प‚ना स्यात्\edtext{}{\edlabel{pvv.325-1}\label{pvv.325-1}\lemma{स्यात्}\Bfootnote{नीलादिस्व‚स्व‚रूप‚भेद‚व‚त् ज्ञानान्य‚पि भिन्नान्याकार‚भेदात् । नैवं शुक्ला‚{\tiny $_{lb}$}‚दिस‚हित‚सामान्य‚ज‚निते ज्ञाने भेदः ।}} ॥
	\pend% ending standard par
      
	  \bigskip
	  \begingroup
	
	    \large
	  
	    \begin{quote}
	  
	    
	    \stanza[\smallbreak]
	\label{pv.3.104b}\flagstanza{\tiny\textenglish{...3.104b}}तासाम‚न्य‚त‚मापेक्ष्यं त‚च्चेच्छ‚क्तं न केव‚ल‚म् ॥ १०४ ॥\&[\smallbreak]


	
	    \end{quote}
	  
	  \endgroup
	

	  \pstart \leavevmode% starting standard par
	\hphantom{.}‚{\color{DodgerBlue3}‚तासां} व्य‚क्ती‚{\color{DodgerBlue3}‚नाम‚न्य‚त‚मापेक्ष्यं} त‚त् सामान्यं प्र‚त्य‚भिज्ञाने ‚{\color{DodgerBlue3}‚श‚क्तं} न ‚{\color{DodgerBlue3}‚केव‚ल‚मिति ‚{\tiny $_{lb}$}‚चेत्}\edtext{}{\edlabel{pvv.325-2}\label{pvv.325-2}\lemma{न}\Bfootnote{एवं स‚ति ।}}॥ (१०४)
	\pend% ending standard par
      \label{div_pvv.3.105}
	  
	% new div opening: depth here is 2
	
	  \bigskip
	  \begingroup
	
	    \large
	  
	    \begin{quote}
	  
	    
	    \stanza[\smallbreak]
	\label{pv.3.105}\flagstanza{\tiny\textenglish{....3.105}}त‚देक‚मुप‚कुर्युस्ताः क‚थ‚मेकान्धिय‚ञ्च न ।&कार्य‚श्च तासां प्राप्तोसौ ज‚न‚नं य‚दुप‚क्रिया ॥ १०५ ॥\&[\smallbreak]


	
	    \end{quote}
	  
	  \endgroup
	

	  \pstart \leavevmode% starting standard par
	\hphantom{.}त‚त्सामान्य‚{\color{DodgerBlue3}‚मेकं} क‚थ‚न्ता व्य‚क्त‚य ‚{\color{DodgerBlue3}‚उप‚कुर्युः} (।) ‚{\color{DodgerBlue3}‚न त्वेका}‚न्धिय‚म‚नुप‚क्रिय‚{\tiny $_{lb}$}‚माण‚स्य स‚म‚वेत‚त्वेऽतिप्र‚स‚ङ्गाद् व्य‚क्तिभिः सामान्य‚मुप‚क‚र्त्त‚व्यं । त‚था धीर‚प्ये‚{\tiny $_{lb}$}‚कोप‚क‚र्त्त‚व्या । य‚च्च सामान्यादिरुप‚क्रिय‚ते व्य‚क्तिभिः ‚{\color{DodgerBlue3}‚कार्य‚श्च तासां प्राप्तोसौ ‚{\tiny $_{lb}$}‚य‚द्य}‚स्मा‚{\color{DodgerBlue3}‚दुप‚क्रिया} ज‚न‚न‚मेव । न ह्युप‚क्रिय‚माणाद‚न्य‚स्मिन्नुप‚काराख्ये व‚स्तुनि कृतेपि ‚{\tiny $_{lb}$}‚त‚स्य किञ्चित् । त‚त्स‚म्ब‚न्धाच्चेत् । स‚म्ब‚न्धो भिन्न एवेति किन्त‚स्य जातं । ‚{\color{DodgerBlue3}‚अभिन्ने} तूप‚कारे स एव कृतः स्यात् । त‚था ज‚न‚न‚मेवोप‚कारः ॥ (१०५)
	\pend% ending standard par
      \label{div_pvv.3.106}
	  
	% new div opening: depth here is 2
	
	  \bigskip
	  \begingroup
	
	    \large
	  
	    \begin{quote}
	  
	    
	    \stanza[\smallbreak]
	\label{pv.3.106}\flagstanza{\tiny\textenglish{....3.106}}अभिन्न‚प्र‚तिभासा धीर्न भिन्नेष्विति चेन्म‚त‚म् ।&प्र‚तिभासो धिया भिन्नः स‚माना इति त‚द्ग्र‚हात् ॥ १०६ ॥\&[\smallbreak]


	
	    \end{quote}
	  
	  \endgroup
	

	  \pstart \leavevmode% starting standard par
	\hphantom{.}एक‚सामान्याभावे‚{\color{DodgerBlue3}‚ऽभिन्न‚प्र‚तिभासा} एकाकारा ‚{\color{DodgerBlue3}‚धीर्न भिन्नेष्विति चेन्म‚तं} । ‚{\tiny $_{lb}$}‚न‚नु \edtext{}{\edlabel{pvv.325-3}\label{pvv.325-3}\lemma{नु}\Bfootnote{व्य‚तिरिक्ताव्य‚तिरिक्त‚सामान्य‚योगाद् भ्रान्तिरेव व्य‚क्तिष्वेकाकारः प्र‚तिभास ‚{\tiny $_{lb}$}‚इत्युक्त्वा नैवास्त्येक‚प्र‚तिभासो व्य‚क्तिष्वित्याहाधुना ।}} ‚{\color{DodgerBlue3}‚प्र‚तिभासो धिया भिन्नो} नैकाकारः ‚{\color{DodgerBlue3}‚स‚माना} इमा ‚{\color{DodgerBlue3}‚इति ता}‚{\tiny $_{4}$}‚सां व्य‚क्तीनां ‚{\color{DodgerBlue3}‚ग्र‚हात्} ।\edtext{\textsuperscript{*}}{\edlabel{pvv.325-4}\label{pvv.325-4}\lemma{*}\Bfootnote{न ह्येक‚स्मिन् प्र‚तिभासे स‚माना इति युक्तं किन्तु त‚देवेति ।}} ‚{\tiny $_{lb}$}‚न हि भूत\edtext{}{\edlabel{pvv.325-5}\label{pvv.325-5}\lemma{भूत}\Bfootnote{न‚व‚ग्र‚हं ।}} क‚ण्ठ‚गुण‚व‚देकं सामान्यं\edtext{}{\edlabel{pvv.325-6}\label{pvv.325-6}\lemma{सामान्यं}\Bfootnote{व्य‚क्तिविशिष्ट‚सामान्य‚ग्र‚होपि न युक्त इत्याह ।}}स‚र्व्वानुपा (?या) यि व्य‚क्तिव्य‚तिरिक्तं प्र‚ति‚{\tiny $_{lb}$}‚भाति । किन्तु गौर्गौरिति सामान्य‚म‚व‚सीय‚ते त‚च्च भेदाधिष्ठान‚मेव ॥ (१०६)
	\pend% ending standard par
      \label{div_pvv.3.107}
	  
	% new div opening: depth here is 2
	\textsuperscript{\textenglish{326/s}}
	  \bigskip
	  \begingroup
	
	    \large
	  
	    \begin{quote}
	  
	    
	    \stanza[\smallbreak]
	\label{pv.3.107}\flagstanza{\tiny\textenglish{....3.107}}क‚थ‚न्ता भिन्न‚धीग्राह्याः स‚माश्चेदेक‚कार्य‚ता ।&सादृश्यं न‚नु धीः कार्यं तासां सा च विभिद्य‚ते ॥ १०७ ॥\&[\smallbreak]


	
	    \end{quote}
	  
	  \endgroup
	

	  \pstart \leavevmode% starting standard par
	\edtext{\textsuperscript{*}}{\edlabel{pvv.326-1}\label{pvv.326-1}\lemma{*}\Bfootnote{प‚रोत्र विरोध‚माह स‚माश्चेत् (।) क‚थ‚म्भिन्न‚धीग्राह्या एकाभिन्न‚धीग्राह्या ‚{\tiny $_{lb}$}‚अपि स्युर्व्य‚क्तिसामान्ये स‚माना इति अन्य‚था घ‚ट‚प‚टादिव‚द‚भेद एव स्यात् ।}}न‚नु स‚माश्चेद् व्य‚क्त‚योध्य‚व‚सीय‚न्ते । ‚{\color{DodgerBlue3}‚क‚थं ता भिन्न‚धीग्राह्याः} । \edtext{\textsuperscript{*}}{\edlabel{pvv.326-2}\label{pvv.326-2}\lemma{*}\Bfootnote{न स‚मान‚तान्य‚थानुप‚प‚त्त्या स्व‚हेतुभ्य‚स्त‚थोत्प‚त्तेः केषाञ्चित् ।}}न ख‚लु ‚{\tiny $_{lb}$}‚‚{\color{DodgerBlue3}‚स‚म}‚त्व‚मेक‚त्वं कि‚{\color{DodgerBlue3}‚न्त्वेक‚कार्य‚तासादृश्यं} (।) न हि स एवायं त‚द‚त्र वेति नि‚{\color{DodgerBlue3}‚श्च‚यः} । ‚{\tiny $_{lb}$}‚अपि त्व‚य‚म‚पि गौरित्य‚ध्य‚व‚सायः । त‚था चैकार्थ‚क्रियाकारित्व‚मे‚{\tiny $_{5}$}‚व सादृश्यं । ‚{\tiny $_{lb}$}‚‚{\color{DodgerBlue3}‚न‚नु धीः कार्यं तासां} व्य‚क्तीनां ‚{\color{DodgerBlue3}‚सा च} प्र‚तिव्य‚क्ति ‚{\color{DodgerBlue3}‚भिद्य‚ते} त‚त्क‚थ‚मेकार्थ‚क्रिया‚{\tiny $_{lb}$}‚कारित्वं सादृश्यं\edtext{}{\edlabel{pvv.326-3}\label{pvv.326-3}\lemma{सादृश्यं}\Bfootnote{उद‚काह‚र‚णादि प्र‚तिव्य‚क्ति भिन्नं अनुभ‚व‚ज्ञान‚मेव व्य‚क्तिकार्यं त‚द‚पि ‚{\tiny $_{lb}$}‚भिन्न‚मेव न विक‚ल्प‚कं व्य‚क्त्य‚भावेपि भावात् धीर‚विक‚ल्पापि ।}}। (१०७)
	\pend% ending standard par
      \label{div_pvv.3.108}
	  
	% new div opening: depth here is 2
	

	  \pstart \leavevmode% starting standard par
	अत आह ।
	\pend% ending standard par
      
	  \bigskip
	  \begingroup
	
	    \large
	  
	    \begin{quote}
	  
	    
	    \stanza[\smallbreak]
	\label{pv.3.108}\flagstanza{\tiny\textenglish{....3.108}}एक‚प्र‚त्य‚व‚म‚र्श‚स्य हेतुत्वाद् धीर‚भेदिनी ।&एक‚धीहेतुभावेन व्य‚क्तीनाम‚प्य‚भिन्न‚ता ॥ १०८ ॥\&[\smallbreak]


	
	    \end{quote}
	  
	  \endgroup
	

	  \pstart \leavevmode% starting standard par
	\hphantom{.}य‚द्य‚पि प्र‚तिव्य‚क्ति भिन्ना त‚थाप्येक‚{\color{DodgerBlue3}‚प्र‚त्य‚व‚म‚र्श‚स्य}\edtext{}{\edlabel{pvv.326-4}\label{pvv.326-4}\lemma{थाप्येक}\Bfootnote{स्व‚विष‚य‚स्यैकाकार‚प्र‚त्य‚य‚स्य ।}} हेतुत्वाद् धीर‚भेदिन्ये‚{\tiny $_{lb}$}‚काऽभिधीय‚ते (।) त‚थाविधायाश्चैक‚स्या धियो\edtext{}{\edlabel{pvv.326-5}\label{pvv.326-5}\lemma{धियो}\Bfootnote{अध्य‚व‚सितैक‚रूपायाः प्र‚त्येकं व्य‚क्तिग्राहिधियां भेदेपि प्र‚त्य‚भिज्ञ‚या ‚{\tiny $_{lb}$}‚तासामेक‚त्व‚म‚ध्य‚व‚सीय‚त इत्य‚र्थः एत‚देवैक‚कार्यंतासादृश्यं ।}} हेतुभावेन ‚{\color{DodgerBlue3}‚व्य‚क्तीनाम‚भिन्न‚तो}‚{\tiny $_{lb}$}‚च्य‚ते ॥ (१०८)
	\pend% ending standard par
      \label{div_pvv.3.109}
	  
	% new div opening: depth here is 2
	
	  \bigskip
	  \begingroup
	
	    \large
	  
	    \begin{quote}
	  
	    
	    \stanza[\smallbreak]
	\label{pv.3.109}\flagstanza{\tiny\textenglish{....3.109}}सा चात‚त्कार्य‚विश्लेष‚स्त‚द‚न्य‚स्यानुव‚र्तिनः ।&अदृष्टेः प्र‚तिषेधाच्च संकेत‚स्त‚द्विद‚र्थिकः ॥ १०९ ॥\&[\smallbreak]


	
	    \end{quote}
	  
	  \endgroup
	

	  \pstart \leavevmode% starting standard par
	\hphantom{.}‚{\color{DodgerBlue3}‚सा चा}‚भिन्न‚ता‚{\color{DodgerBlue3}‚ऽत‚त्का}‚र्येभ्यः प‚दार्थेभ्यो ‚{\color{DodgerBlue3}‚विश्लेषो} व्य‚व‚च्छेद‚स्त‚त्कारिणां व‚स्तुभूत‚{\tiny $_{lb}$}‚जातिरेव किन्ने‚{\tiny $_{6}$}‚ष्य‚ते इत्याह । ‚{\color{DodgerBlue3}‚त‚तो} व्य‚व‚च्छेदा‚{\color{DodgerBlue3}‚द‚न्य‚स्य} (अनुव‚र्त्तिनः) सामान्य‚स्य ‚{\tiny $_{lb}$}‚‚{\color{DodgerBlue3}‚व‚स्तुस‚तोऽदृष्टेः} स्व‚भावानुप‚ल‚ब्ध्या ‚{\color{DodgerBlue3}‚प्र‚तिषेधाच्च\edtext{}{\edlabel{pvv.326-6}\label{pvv.326-6}\lemma{तिषेधाच्च}\Bfootnote{पूर्व्वोक्तात् ।}}} । त‚त‚श्च ‚{\color{DodgerBlue3}‚संकेत‚स्त‚द्विद‚र्थिक}‚{\tiny $_{lb}$}‚\edtext{}{\edlabel{pvv.326-7}\label{pvv.326-7}\lemma{श्च}\Bfootnote{विक‚ल्पाध्य‚स्त‚बाह्य‚प्र‚तिप‚त्त्य‚र्थः ।}}स्त‚स्य व्य‚व‚च्छेद‚स्य विद‚र्थो य‚स्यास्ति स त‚था ॥ (१०९)
	\pend% ending standard par
      \label{div_pvv.3.110}
	  
	% new div opening: depth here is 2
	
	  \bigskip
	  \begingroup
	
	    \large
	  
	    \begin{quote}
	  
	    
	    \stanza[\smallbreak]
	\label{pv.3.110}\flagstanza{\tiny\textenglish{....3.110}}अत‚त्कारिविवेकेन प्र‚वृत्य‚र्थ‚त‚या श्रुतिः ।&अकार्य‚कृतित‚त्कारितुल्य‚रूपाव‚भासिनीम् ॥ ११० ॥\&[\smallbreak]


	
	    \end{quote}
	  
	  \endgroup
	\textsuperscript{\textenglish{327/s}}

	  \pstart \leavevmode% starting standard par
	\hphantom{.}संकेत‚{\color{DodgerBlue3}‚श्चात‚त्कारिविवेकेन प्र‚वृत्त्य‚र्य‚त‚या} प्र‚वृत्तिप्र‚योज‚न‚त्वेन कृतः ।\edtext{\textsuperscript{*}}{\edlabel{pvv.327-1}\label{pvv.327-1}\lemma{*}\Bfootnote{श‚ब्द‚ज‚निता धीः स्वाकारं बाह्य‚म‚ध्य‚व‚स्य‚तीत्य‚लीकेति न स‚म्वादः इत्याह ।}} त‚त‚श्च ‚{\tiny $_{lb}$}‚श्रुतिर‚पि धियं ज‚न‚य‚न्त्य‚र्थेन विसंवादिकेति स‚म्ब‚न्ध‚नीयं । कीदृशीं ‚{\color{DodgerBlue3}‚धियं} (।) स्वाकारे‚{\tiny $_{lb}$}‚‚{\color{DodgerBlue3}‚ऽकार्य‚कृति} । कार्य‚कार‚णास‚म‚र्थे ‚{\color{DodgerBlue3}‚त‚त्कारि} तु‚{\tiny $_{7}$}‚ल्य‚रूपाव‚भासिनीमेकार्थ‚क्रियाकारि-\leavevmode\ledsidenote{\textenglish{65a/MA}} ‚{\tiny $_{lb}$}‚स‚र्व्व‚व‚स्तुसाधार‚णैक‚रूपाध्य‚व‚सायिनीं ॥ (११०)
	\pend% ending standard par
      \label{div_pvv.3.111}
	  
	% new div opening: depth here is 2
	
	  \bigskip
	  \begingroup
	
	    \large
	  
	    \begin{quote}
	  
	    
	    \stanza[\smallbreak]
	\label{pv.3.111}\flagstanza{\tiny\textenglish{....3.111}}धियं व‚स्तुपृथ‚ग्भाव‚मात्र‚बीजाम‚न‚र्थिकाम् ।&ज‚न‚य‚न्त्य‚प्य‚त‚त्कारिप‚रिहाराङ्ग‚भावातः ॥ १११ ॥\&[\smallbreak]


	
	    \end{quote}
	  
	  \endgroup
	

	  \pstart \leavevmode% starting standard par
	\hphantom{.}‚{\color{DodgerBlue3}‚व‚स्तु}\edtext{\textsuperscript{*}}{\edlabel{pvv.327-2}\label{pvv.327-2}\lemma{*}\Bfootnote{एतेन ब‚हिः प्र‚वृत्य‚ङ्ग‚त्वं श्रुतेः ।}} न‚स्त‚त्कारिणोऽत‚त्कारिभ्य‚श्च ‚{\color{DodgerBlue3}‚पृथ‚ग्भाव} एव केव‚लः ‚{\color{DodgerBlue3}‚स बीजं कार‚णं य‚स्या}‚{\tiny $_{lb}$}‚स्ता‚{\color{DodgerBlue3}‚म‚न‚र्थिका}‚म‚र्थ‚शून्यां\edtext{}{\edlabel{pvv.327-3}\label{pvv.327-3}\lemma{शून्यां}\Bfootnote{विधित्वेन व‚स्त्व‚ध्य‚व‚सा (या)त् क‚थ‚म‚न्यापोह‚विष‚य‚तेत्याह त‚तोऽत‚त्कारि‚{\tiny $_{lb}$}‚प‚रिहारे हेतुत्वात् ।}} प‚र‚मार्थ‚तः ।
	\pend% ending standard par
      

	  \pstart \leavevmode% starting standard par
	क‚थ‚म‚विस‚म्वादिका त‚र्हि सेत्याह ।
	\pend% ending standard par
      

	  \pstart \leavevmode% starting standard par
	\hphantom{.}‚{\color{DodgerBlue3}‚अत‚त्का}‚रिणां विजातीयानां ‚{\color{DodgerBlue3}‚प‚रिहार}‚स्या‚{\color{DodgerBlue3}‚ङ्ग‚भाव‚तः} कार‚ण‚त्वात् ‚{\color{DodgerBlue3}‚त‚द्व्य‚व‚च्छे}‚{\tiny $_{lb}$}‚द‚स्य विष‚य‚त्वात्\edtext{}{\edlabel{pvv.327-4}\label{pvv.327-4}\lemma{त्वात्}\Bfootnote{अन्य‚व्यावृत्त‚व‚स्त्व‚ध्य‚व‚सायिबुद्धिज‚न‚नात् स्व‚ल‚क्ष‚णे प्र‚व‚र्त्त‚य‚ति ।}}त‚त्र प्र‚वृत्तिः ॥ (१११)
	\pend% ending standard par
      \label{div_pvv.3.112}
	  
	% new div opening: depth here is 2
	
	  \bigskip
	  \begingroup
	
	    \large
	  
	    \begin{quote}
	  
	    
	    \stanza[\smallbreak]
	\label{pv.3.112}\flagstanza{\tiny\textenglish{....3.112}}व‚स्तुभेदाश्र‚याच्चार्थे न विसंवादिका म‚ता ।&त‚तोन्यापोह‚विष‚या त‚त्क‚र्त्राश्रित‚भाव‚तः ॥ ११२ ॥\&[\smallbreak]


	
	    \end{quote}
	  
	  \endgroup
	

	  \pstart \leavevmode% starting standard par
	व‚स्तुभेदाश्र‚याच्च व‚स्तुविष‚य‚स्य प‚रंप‚र‚या हेतुत्वात् त‚त्र प्र‚व‚र्त्त‚य‚{\tiny $_{1}$}‚न्ती श्रुतिर‚र्थे ‚{\tiny $_{lb}$}‚न विसंवादिका म‚ता ।\edtext{\textsuperscript{*}}{\edlabel{pvv.327-5}\label{pvv.327-5}\lemma{*}\Bfootnote{विधित्वेन व‚स्त्व‚ध्य‚व‚सा (या) त् क‚थ‚म‚न्यापोह‚विष‚य‚तेत्याह (।) त‚तोऽ‚{\tiny $_{lb}$}‚त‚त्कारिप‚रिहार‚हेतुः ।}} त‚तोन्यायोह‚विष‚या त‚त्क‚र्त्राश्रित‚भाव‚तः । ‚{\tiny $_{lb}$}‚त‚त्कार्य‚क‚र्तृ व‚स्त्वाश्रित‚त्वात्\edtext{}{\edlabel{pvv.327-6}\label{pvv.327-6}\lemma{त्वात्}\Bfootnote{स्वार्थाभिधानाच्छु तेर‚पोहे क‚र्त्तृत्वं व्य‚व‚हारे संकेते व‚स्तुभेदाश्र‚य‚त्वाद‚{\tiny $_{lb}$}‚पोहाश्रित‚त्वं ।}}॥ (११२)
	\pend% ending standard par
      \label{div_pvv.3.113}
	  
	% new div opening: depth here is 2
	
	  \bigskip
	  \begingroup
	
	    \large
	  
	    \begin{quote}
	  
	    
	    \stanza[\smallbreak]
	\label{pv.3.113}\flagstanza{\tiny\textenglish{....3.113}}अवृक्ष‚व्य‚तिरेकेण वृक्षार्थ‚ग्र‚ह‚णे द्व‚य‚म् ।&अन्योन्याश्र‚य‚मित्येक‚ग्र‚हाभावे द्व‚याग्र‚हः ॥ ११३ ॥\&[\smallbreak]


	
	    \end{quote}
	  
	  \endgroup
	

	  \pstart \leavevmode% starting standard par
	\hphantom{.}न‚नु संकेत‚काले‚{\color{DodgerBlue3}‚ऽवृक्ष‚व्य‚तिरेकेण वृक्षार्थ}‚स्य ‚{\color{DodgerBlue3}‚ग्र‚ह‚णे द्व‚य‚म‚न्योन्याश्र‚यं} वृक्षो\edtext{}{\edlabel{pvv.327-7}\label{pvv.327-7}\lemma{वृक्षो}\Bfootnote{याव‚द् गां नाबुद्ध ताव‚द् गां न बुध्य‚ते इति प‚रः त‚त्साम‚न्य‚मेष्ट‚व्यं ।}}‚{\tiny $_{lb}$}‚ऽवृक्ष‚श्चान्योन्यापेक्ष इत्येक‚स्य वृक्ष‚स्यावृक्ष‚स्य चाव्य‚व‚स्थित‚त्वात् ‚{\color{DodgerBlue3}‚ग्र‚हाभावे द्व‚याग्र‚हः} \leavevmode\ledsidenote{\textenglish{328/s}} प्राप्तः । निश्चिते हि वृक्षे त‚द‚भावः श‚क्यो निश्चेतुं । अवृक्ष‚त्वे च निश्चिते वृक्षो ‚{\tiny $_{lb}$}‚निश्चेय इत्येकानि‚{\tiny $_{2}$}‚श्च‚याद् द्व‚यानिश्च‚यः ॥ (११३)
	\pend% ending standard par
      \label{div_pvv.3.114_3.115}
	  
	% new div opening: depth here is 2
	
	  \bigskip
	  \begingroup
	
	    \large
	  
	    \begin{quote}
	  
	    
	    \stanza[\smallbreak]
	\label{pv.3.114a}\flagstanza{\tiny\textenglish{...3.114a}}स‚ङ्केतास‚म्भ‚व‚स्त‚स्मादिति केचित् प्र‚च‚क्ष‚ते ।&तेषाम‚वृक्षास्स‚ङ्केते व्य‚व‚च्छिन्ना न वा;\&[\smallbreak]


	
	    \end{quote}
	  
	  \endgroup
	

	  \pstart \leavevmode% starting standard par
	\hphantom{.}‚{\color{DodgerBlue3}‚सेङ्केतास‚म्भ‚व‚स्त‚स्मादिति केचि} ज्जै मि नी याः ‚{\color{DodgerBlue3}‚प्र‚च‚क्ष‚ते । तेषामेव}\edtext{}{\edlabel{pvv.328-1}\label{pvv.328-1}\lemma{याः}\Bfootnote{तुल्य‚दोष‚तामाह ।}} वादिनां ‚{\tiny $_{lb}$}‚जाताव‚पि ‚{\color{DodgerBlue3}‚स‚ङ्केते} क्रिय‚माणेऽ‚{\color{DodgerBlue3}‚वृक्षा व्य‚व‚च्छिन्ना न वा} ।
	\pend% ending standard par
      
	  \bigskip
	  \begingroup
	
	    \large
	  
	    \begin{quote}
	  
	    
	    \stanza[\smallbreak]
	\label{pv.3.114b}\flagstanza{\tiny\textenglish{...3.114b}}य‚दि ॥ ११४ ॥\&[\smallbreak]


	
	    \end{quote}
	  
	  \endgroup
	
	  \bigskip
	  \begingroup
	
	    \large
	  
	    \begin{quote}
	  
	    
	    \stanza[\smallbreak]
	\label{pv.3.115}\flagstanza{\tiny\textenglish{....3.115}}व्य‚व‚च्छिन्नाः क‚थं ज्ञाताः प्राग्वृक्ष‚ग्र‚ह‚णादृते ।&अनिराक‚र‚णे तेषां स‚ङ्केते व्य‚व‚हारिणाम् ॥ ११५ ॥\&[\smallbreak]


	
	    \end{quote}
	  
	  \endgroup
	

	  \pstart \leavevmode% starting standard par
	\hphantom{.}‚{\color{DodgerBlue3}‚य‚दि व्य‚व‚च्छिन्ना} इष्य‚न्ते ‚{\color{DodgerBlue3}‚क‚थं} संकेतात् प्राक् ‚{\color{DodgerBlue3}‚ज्ञाता वृक्ष‚ग्र‚ह‚णा}‚दृते संकेताद् ‚{\tiny $_{lb}$}‚वृक्ष‚स्याज्ञात‚त्वाद् अवृक्षाः क‚थं ज्ञाताः । अथ न व्य‚व‚च्छिन्नाः त‚दाऽनिराक‚र‚णे ‚{\tiny $_{lb}$}‚‚{\color{DodgerBlue3}‚तेषां} संकेते ‚{\color{DodgerBlue3}‚व्य‚व‚हारिणां} व्य‚व‚हार‚काले\edtext{}{\edlabel{pvv.328-2}\label{pvv.328-2}\lemma{काले}\Bfootnote{न हि संकेते प‚राव्य‚व‚च्छेदेन निवेशिच्छ‚ब्दाद् व्य‚व‚हारे त‚त्प‚रिहारेण प्र‚वृत्तिः ।}}॥ (११५)
	\pend% ending standard par
      \label{div_pvv.3.116_3.117_3.118_3.119_3.120}
	  
	% new div opening: depth here is 2
	
	  \bigskip
	  \begingroup
	
	    \large
	  
	    \begin{quote}
	  
	    
	    \stanza[\smallbreak]
	\label{pv.3.116}\flagstanza{\tiny\textenglish{....3.116}}न स्यात् त‚त्प‚रिहारेण प्र‚वृत्तिर्वृक्ष‚भेद‚व‚त् ।&अविधाय निषिध्यान्य‚त् प्र‚द‚र्श्यैकं पुरः स्थित‚म् ॥ ११६ ॥\&[\smallbreak]


	
	    \end{quote}
	  
	  \endgroup
	
	  \bigskip
	  \begingroup
	
	    \large
	  
	    \begin{quote}
	  
	    
	    \stanza[\smallbreak]
	\label{pv.3.117a}\flagstanza{\tiny\textenglish{...3.117a}}वृक्षोय‚मिति संकेतः क्रिय‚ते त‚त् प्र‚प‚द्य‚ते ।&ब्य‚व‚हारेपि तेनाय‚म‚दोष इति चेत्;\&[\smallbreak]


	
	    \end{quote}
	  
	  \endgroup
	

	  \pstart \leavevmode% starting standard par
	\hphantom{.}तेषाम‚वृक्षाणां ‚{\color{DodgerBlue3}‚प‚रिहारेण न स्यात् प्र‚वृत्तिः वृक्ष‚भेद‚व‚द्} (।) य‚था वृक्ष‚विशेषाणां ‚{\tiny $_{lb}$}‚वृक्ष‚संकेतेऽव्य‚व‚च्छिन्न‚त्वात् प्र‚वृत्तिविष‚य‚त्व‚मेव‚म‚वृक्षाणाम‚पि स्यात् । न‚न्वे\edtext{}{\edlabel{pvv.328-3}\label{pvv.328-3}\lemma{न्वे}\Bfootnote{व‚स्तुसामान्य‚वादी प्राह ।}} ‚{\color{DodgerBlue3}‚कं} शाखादिम‚न्तं ‚{\color{DodgerBlue3}‚पुर (:) स्थितं प्र‚द‚र्श्य} त‚स्मा‚{\color{DodgerBlue3}‚द‚न्य‚द‚विधाय निषिध्य} च । (११६) ‚{\tiny $_{lb}$}‚‚{\color{DodgerBlue3}‚वृक्षोय‚मिति च संकेतः क्रिय‚ते} व‚स्तुसामान्यादिभिः ‚{\color{DodgerBlue3}‚त‚त्} संकेत‚विष‚यं सामान्यं\edtext{}{\edlabel{pvv.328-4}\label{pvv.328-4}\lemma{सामान्यं}\Bfootnote{सामान्ये कृतो व्य‚क्तिषु व्याप‚कः स्यात् ।}} ‚{\tiny $_{lb}$}‚त‚त्स‚म्ब‚न्धिबाह्य‚ल‚क्ष‚णं ‚{\color{DodgerBlue3}‚व्य‚व‚हारे प्र‚प‚द्य‚ते तेनाय‚म}‚न‚न्त‚रोक्त‚{\color{DodgerBlue3}‚दोषो न} भ‚व‚तीति ‚{\color{DodgerBlue3}‚चेत्} ॥
	\pend% ending standard par
      
	  \bigskip
	  \begingroup
	
	    \large
	  
	    \begin{quote}
	  
	    
	    \stanza[\smallbreak]
	\label{pv.3.117b}\flagstanza{\tiny\textenglish{...3.117b}}त‚रुः ॥ ११७ ॥\&[\smallbreak]


	
	    \end{quote}
	  
	  \endgroup
	
	  \bigskip
	  \begingroup
	
	    \large
	  
	    \begin{quote}
	  
	    
	    \stanza[\smallbreak]
	\label{pv.3.118a}\flagstanza{\tiny\textenglish{...3.118a}}अय‚म‚प्य‚य‚मेवेति प्र‚स‚ङ्गो न निव‚र्त्त‚ते ।&एक‚प्र‚त्य‚व‚म‚र्शाख्ये ज्ञाने;\&[\smallbreak]


	
	    \end{quote}
	  
	  \endgroup
	\textsuperscript{\textenglish{329/s}}

	  \pstart \leavevmode% starting standard par
	अत्र चा‚{\tiny $_{4}$}‚‚{\color{DodgerBlue3}‚य\edtext{}{\edlabel{pvv.329-1}\label{pvv.329-1}\lemma{य}\Bfootnote{त‚रुर‚य‚म‚पीति प‚क्षेऽन्य‚स्यापि त‚रुत्व‚म‚निषिद्ध‚मिति व्य‚व‚हारो न निय‚तः । ‚{\tiny $_{lb}$}‚अय‚मेवेति प‚क्षेऽत‚रुव्य‚व‚च्छेदः स्यात् । त‚त्र चोक्तं प्र‚तिपाद्येन संकेते वृक्षावृक्षौ ‚{\tiny $_{lb}$}‚क‚थं ज्ञाताविति त‚द‚व‚स्थो दोषः ।}}म‚पि} शाखादिमांस्त‚रुर‚{\color{DodgerBlue3}‚य‚मेव} त‚रु‚{\color{DodgerBlue3}‚रिति} पूर्व्व‚कोऽवृक्षाव्य‚व‚च्छेद‚{\tiny $_{lb}$}‚‚{\color{DodgerBlue3}‚प्र‚स‚ङ्गो न} निव‚र्त्त‚तेऽत्र चोक्त एव दोष इति द्व‚योर‚प्येंकाग्र‚हात् द्व‚यास‚म्भ‚वात् ‚{\tiny $_{lb}$}‚स‚ङ्केतास‚म्भ‚वः स‚मानः । इदानीं प‚रिह‚र्तुमाह ।\edtext{\textsuperscript{*}}{\edlabel{pvv.329-2}\label{pvv.329-2}\lemma{*}\Bfootnote{न दृष्ट‚विप‚रीत‚स्य सुज्ञानात् । व्य‚व‚च्छेदे तु नैव‚म‚न‚न्व‚यात् एक‚त्र दृष्ट‚स्य ‚{\tiny $_{lb}$}‚क्व (चि) द‚पि । एत‚त्तुल्यं य‚स्मादित्याह ।}} ‚{\color{DodgerBlue3}‚एक‚प्र‚त्य‚व‚म‚र्शा}‚ख्ये ‚{\color{DodgerBlue3}‚ज्ञाने} भेदा‚{\tiny $_{lb}$}‚विशेषेपि केचिदेव भावा एकाकाराध्य‚व‚साय‚हेत‚वो नेत‚रे ॥ (११७)
	\pend% ending standard par
      
	  \bigskip
	  \begingroup
	
	    \large
	  
	    \begin{quote}
	  
	    
	    \stanza[\smallbreak]
	\label{pv.3.118b}\flagstanza{\tiny\textenglish{...3.118b}}एक‚त्र हि स्थितः ॥ ११८ ॥\&[\smallbreak]


	
	    \end{quote}
	  
	  \endgroup
	
	  \bigskip
	  \begingroup
	
	    \large
	  
	    \begin{quote}
	  
	    
	    \stanza[\smallbreak]
	\label{pv.3.119}\flagstanza{\tiny\textenglish{....3.119}}प्र‚प‚त्ता त‚द‚त‚द्धेतून‚र्थान् विभ‚ज‚ते स्व‚य‚म् ।&त‚द्बुद्धिव‚र्तिनो भावान् भातो हेतुत‚या धियः ॥ ११९ ॥\&[\smallbreak]


	
	    \end{quote}
	  
	  \endgroup
	
	  \bigskip
	  \begingroup
	
	    \large
	  
	    \begin{quote}
	  
	    
	    \stanza[\smallbreak]
	\label{pv.3.120}\flagstanza{\tiny\textenglish{....3.120}}अहेतुरूप‚विक‚लानेक‚रूपानिव स्व‚य‚म् ।&भेदेन प्र‚तिप‚द्येतेत्युक्तिर्भेदे नियुज्य‚ते ॥ १२० ॥\&[\smallbreak]


	
	    \end{quote}
	  
	  \endgroup
	

	  \pstart \leavevmode% starting standard par
	\hphantom{.}त‚{\color{DodgerBlue3}‚त्रैक}‚स्मिन्नेकाकार‚ज्ञानाध्य‚व‚साये ‚{\color{DodgerBlue3}‚स्थितः प्र‚तिप‚त्ता} पुरुष‚{\color{DodgerBlue3}‚स्त‚द‚{\tiny $_{5}$}‚त‚द्धेतून‚र्थान् ‚{\tiny $_{lb}$}‚विभ‚ज‚ते स्व‚यं} । यानेकाकार‚प‚राम‚र्श‚विष‚य‚बुद्ध्यादिहेतून‚ध्य‚व‚स्य‚ति । याँश्चा‚{\tiny $_{lb}$}‚न्य‚था तान् य‚थाक्र‚मं त‚द्धेतून‚त‚द्धेतूश्चं भेदेन व्य‚व‚स्थाप‚य‚ति संकेतात् ‚{\color{DodgerBlue3}‚प्रागेव\edtext{}{\edlabel{pvv.329-3}\label{pvv.329-3}\lemma{प्रागेव}\Bfootnote{अत‚द्धेतुभ्य‚स्त‚द्धेतून् विभ‚ज्य स्थाप‚य‚ति या बुद्धिः ।}} ‚{\tiny $_{lb}$}‚त‚द्बुद्धिव‚र्त्तिन} एकाकार‚प‚राम‚र्श‚विष‚यान् ‚{\color{DodgerBlue3}‚भ‚वान्} शाखादिम‚तो ‚{\color{DodgerBlue3}‚धिय} एकाकाराया ‚{\tiny $_{lb}$}‚‚{\color{DodgerBlue3}‚हेतुत‚या भातः} प्र‚तिभास‚माना ‚{\color{DodgerBlue3}‚न हेतो}‚रेकाकार‚बुद्ध्य‚कार‚ण‚स्य शाखादिम‚त्व‚र‚हित‚स्य ‚{\tiny $_{lb}$}‚‚{\color{DodgerBlue3}‚रूपेण विक‚लान्} प‚र‚मार्थ‚भिन्ना‚{\tiny $_{6}$}‚‚{\color{DodgerBlue3}‚न‚प्येक‚रूपानिव\edtext{}{\edlabel{pvv.329-4}\label{pvv.329-4}\lemma{रूपानिव}\Bfootnote{नियुंक्ते}} स्व‚य‚मा}‚त्म‚ना संकेत‚यिता म‚न्य‚मानः ‚{\tiny $_{lb}$}‚प्र‚तिपाद्योपि तान‚त‚त्कारिभ्यः शाखादिम‚त्व‚र‚हितेभ्यो ‚{\color{DodgerBlue3}‚भेदे}\edtext{}{\edlabel{pvv.329-5}\label{pvv.329-5}\lemma{हितेभ्यो}\Bfootnote{एक‚व्य‚क्तौ गोश‚ब्द‚स‚ङ्केते स‚त्य‚पि संकेत‚विष‚य‚स्य व्य‚क्त्य‚न्त‚रेऽनुग‚मात् स ‚{\tiny $_{lb}$}‚एवायं गौरिति स्यात् प्र‚तीतिः ।}}नासंक‚रेण ‚{\color{DodgerBlue3}‚प्र‚तिप‚द्येतेति । ‚{\tiny $_{lb}$}‚उक्तिः} श‚ब्दो ‚{\color{DodgerBlue3}‚भेदेऽ}‚न्यापोहे ‚{\color{DodgerBlue3}‚नियुज्य‚ते} ।\edtext{\textsuperscript{*}}{\edlabel{pvv.329-6}\label{pvv.329-6}\lemma{*}\Bfootnote{स‚ङ्केत्य‚ते ।}}(११८-२०)
	\pend% ending standard par
      \label{div_pvv.3.121_3.122abc}
	  
	% new div opening: depth here is 2
	
	  \bigskip
	  \begingroup
	
	    \large
	  
	    \begin{quote}
	  
	    
	    \stanza[\smallbreak]
	\label{pv.3.121}\flagstanza{\tiny\textenglish{....3.121}}तं त‚स्या धीर्विक‚ल्पिका भ्रान्त्यैकं व‚स्त्विवेक्ष‚ते ।&क्व‚चिन्निवेश‚नायार्थे विनिव‚र्त्य कुत‚श्च‚न ॥ १२१ ॥\&[\smallbreak]


	
	    \end{quote}
	  
	  \endgroup
	
	  \bigskip
	  \begingroup
	
	    \large
	  
	    \begin{quote}
	  
	    
	    \stanza[\smallbreak]
	\label{pv.3.122a}\flagstanza{\tiny\textenglish{...3.122a}}बुद्धेः प्र‚युज्य‚ते श‚ब्द‚स्त‚द‚र्थ‚स्याव‚धार‚णात् ।&व्य‚र्थोन्य‚था प्र‚योगः स्यात्;\&[\smallbreak]


	
	    \end{quote}
	  
	  \endgroup
	\textsuperscript{\textenglish{330/s}}

	  \pstart \leavevmode% starting standard par
	\hphantom{.}‚{\color{DodgerBlue3}‚तं} भेदं ‚{\color{DodgerBlue3}‚त‚स्या} उक्तेरुच्चारिताया वाच्य‚त‚या प्र‚तिय‚ती ‚{\color{DodgerBlue3}‚धीर्व्विक‚ल्पिका} प्र‚कृति‚{\tiny $_{lb}$}‚‚{\color{DodgerBlue3}‚भ्रान्त्या एक‚मिव व‚स्त्वीक्ष‚ते} । त‚स्मात् कुत‚श्च‚नाकार्य‚कारिणो\edtext{}{\edlabel{pvv.330-1}\label{pvv.330-1}\lemma{कारिणो}\Bfootnote{ज्ञेयादिदोषे व्य‚तिरेक उक्तो राद्धान्तः ।}}ऽर्थाद् विनिव‚र्त्त्य ‚{\tiny $_{lb}$}‚\leavevmode\ledsidenote{\textenglish{65b/MA}} ‚{\color{DodgerBlue3}‚क्व‚चिदे}‚कार्थ‚क्रियाकारिण्य‚र्थे ‚{\color{DodgerBlue3}‚बुद्धे‚{\tiny $_{7}$}‚र्निवेश‚नाय श‚ब्दः प्र‚युज्य‚ते} (।) ‚{\color{DodgerBlue3}‚त‚द‚र्थ‚स्य} श‚ब्दार्थ‚स्या‚{\color{DodgerBlue3}‚व‚धार‚णात्} । घ‚टेनोद‚क‚मान‚येत्यादौ प्र‚तिप‚द‚म‚व‚धार‚ण‚मिष्टं । ‚{\color{DodgerBlue3}‚व्य‚र्थोऽन्य‚था ‚{\tiny $_{lb}$}‚प्र‚योगः स्यात्} । य‚दि येन केन‚चिदान‚य‚न‚मिष्ट‚मुद‚क‚मान‚येत्युच्येत । य‚दि\edtext{}{\edlabel{pvv.330-2}\label{pvv.330-2}\lemma{दि}\Bfootnote{अन्यापोहे श‚ब्दार्थे प‚रोऽव्यापित्व‚माह । अज्ञेयाद् विज्ञेय‚स्य भेदेन विष‚यी‚{\tiny $_{lb}$}‚क‚र‚णं वाच्यं त‚तोऽज्ञेयोपि ज्ञेयः स्याद‚विष‚यीकृताद् व्य‚व‚च्छेदाश‚क्तेः । एकाद्य‚स‚र्व‚{\tiny $_{lb}$}‚ञ्चेन्न व्य‚तिरिक्त‚स‚मुदायास्वीकृतेर‚न‚र्थ‚क‚ता स्यात् । य‚त्प‚रः श‚ब्द स श‚ब्दार्थ इति ‚{\tiny $_{lb}$}‚विधाय‚क‚स्य व्य‚व‚च्छेदोप्य‚र्थ इत्य‚र्थः ।}} श‚ब्दानां ‚{\tiny $_{lb}$}‚व्य‚व‚च्छेदो वाच्य‚स्त‚दा ‚{\color{DodgerBlue3}‚ज्ञेयादिप‚दानां} स‚र्व्व‚स्य ज्ञेय‚त्वेन व्य‚व‚च्छेद्या‚{\tiny $_{1}$}‚भावात् । ‚{\tiny $_{lb}$}‚(१२१, १२२)
	\pend% ending standard par
      \label{div_pvv.3.122d_3.123_3.124}
	  
	% new div opening: depth here is 2
	

	  \pstart \leavevmode% starting standard par
	अर्थो न स्यादित्याह\edtext{}{\edlabel{pvv.330-3}\label{pvv.330-3}\lemma{स्यादित्याह}\Bfootnote{अन्य‚था य‚दि श‚ब्देन क‚श्चिद‚र्थो न व्य‚व‚च्छिद्य‚ते व्य‚र्थः श‚ब्द‚प्र‚योगः ‚{\tiny $_{lb}$}‚स्यादुक्त‚युक्त्या ।}}
	\pend% ending standard par
      
	  \bigskip
	  \begingroup
	
	    \large
	  
	    \begin{quote}
	  
	    
	    \stanza[\smallbreak]
	\label{pv.3.122b}\flagstanza{\tiny\textenglish{...3.122b}}त‚ज्ज्ञेयादिप‚देष्व‚पि ॥ १२२ ॥\&[\smallbreak]


	
	    \end{quote}
	  
	  \endgroup
	
	  \bigskip
	  \begingroup
	
	    \large
	  
	    \begin{quote}
	  
	    
	    \stanza[\smallbreak]
	\label{pv.3.123a}\flagstanza{\tiny\textenglish{...3.123a}}व्य‚व‚हारोप‚नीतेषु व्य‚व‚च्छेद्योस्ति क‚श्च‚न ।\&[\smallbreak]


	
	    \end{quote}
	  
	  \endgroup
	

	  \pstart \leavevmode% starting standard par
	य‚स्माद् व्य‚व‚च्छेद‚म‚न्त‚रेण न श‚ब्द‚प्र‚योगः त‚त् त‚स्माज्ज्ञेया\edtext{}{\edlabel{pvv.330-4}\label{pvv.330-4}\lemma{स्माज्ज्ञेया}\Bfootnote{ज्ञेयाः स‚र्व्व‚प‚दार्थाः स‚र्व‚ज्ञ‚ज्ञान‚स्येत्य‚त्रापि य‚द‚ज्ञेय‚त्व‚माश‚ङ्कित‚न्त‚द्व्य‚व‚च्छेद्यं ।}}दिप‚देष्व‚पि ‚{\tiny $_{lb}$}‚कुत‚श्चित् प्र‚क‚र‚णात् ‚{\color{DodgerBlue3}‚व्य‚व‚हारोप‚नीतेषु}\edtext{}{\edlabel{pvv.330-5}\label{pvv.330-5}\lemma{णात्}\Bfootnote{विधिप्र‚तिषेध‚प्र‚योग‚स्थेषु ।}} विशेष‚विष‚येषु ‚{\color{DodgerBlue3}‚क‚श्च}‚न त‚दित‚रोस्ति ‚{\tiny $_{lb}$}‚क‚ल्पितो वा ।\edtext{\textsuperscript{*}}{\edlabel{pvv.330-6}\label{pvv.330-6}\lemma{*}\Bfootnote{य‚दि विधिश‚ब्दार्थोर्थाद‚न्य‚निषेध‚स्त‚र्हि नागोक्त‚म्विरुद्ध‚मित्य‚त आह वृक्ष‚स्यायं ‚{\tiny $_{lb}$}‚भेदो न स्याद् घ‚ट‚व‚त् ।}}
	\pend% ending standard par
      
	  \bigskip
	  \begingroup
	
	    \large
	  
	    \begin{quote}
	  
	    
	    \stanza[\smallbreak]
	\label{pv.3.123b}\flagstanza{\tiny\textenglish{...3.123b}}निवेश‚नं च यो य‚स्माद् भिद्य‚ते त‚न्निव‚र्त‚नात् ॥ १२३ ॥\&[\smallbreak]


	
	    \end{quote}
	  
	  \endgroup
	
	  \bigskip
	  \begingroup
	
	    \large
	  
	    \begin{quote}
	  
	    
	    \stanza[\smallbreak]
	\label{pv.3.124a}\flagstanza{\tiny\textenglish{...3.124a}}त‚द्भेदे भिद्य‚मानानां स‚मानाकार‚भासिनि ।\&[\smallbreak]


	
	    \end{quote}
	  
	  \endgroup
	

	  \pstart \leavevmode% starting standard par
	\hphantom{.}‚{\color{DodgerBlue3}‚यो य‚स्माद}‚त‚त्कारिणो भिद्य‚ते (।) त‚म‚त‚त्कारिण‚म्विनिव‚र्त्य ‚{\color{DodgerBlue3}‚भिद्य‚मानानां ‚{\tiny $_{lb}$}‚त‚द्भेदे}‚ऽत‚त्कारिभेदे ‚{\color{DodgerBlue3}‚स‚मानाकार\edtext{}{\edlabel{pvv.330-7}\label{pvv.330-7}\lemma{मानाकार}\Bfootnote{विक‚ल्पेनैक‚त्वेनारोप्य स‚र्व्व‚त्र ।}} भासिनि निवेश‚न‚ञ्च श‚ब्दानां\edtext{}{\edlabel{pvv.330-8}\label{pvv.330-8}\lemma{ब्दानां}\Bfootnote{स‚ङ्केतेपि विधिरुक्तोऽनेन ।}}} ॥
	\pend% ending standard par
      
	  \bigskip
	  \begingroup
	
	    \large
	  
	    \begin{quote}
	  
	    
	    \stanza[\smallbreak]
	\label{pv.3.124b}\flagstanza{\tiny\textenglish{...3.124b}}स चाय‚म‚न्य‚व्यावृत्या ग‚म्य‚ते त‚स्य व‚स्तुनः ॥ १२४ ॥\&[\smallbreak]


	
	    \end{quote}
	  
	  \endgroup
	\textsuperscript{\textenglish{331/s}}

	  \pstart \leavevmode% starting standard par
	\hphantom{.}‚{\color{DodgerBlue3}‚स चाय‚म‚न्य}‚व्य‚व‚च्छेदः प्रोक्त आ चा र्ये ण ‚{\color{DodgerBlue3}‚अन्य‚व्यावृ‚{\tiny $_{2}$}‚त्त्या}‚ऽन्य‚व्यावृत्त‚त्वेन ‚{\tiny $_{lb}$}‚‚{\color{DodgerBlue3}‚ग‚म्य‚ते त‚स्य व‚स्तुनः} (॥ १२४)
	\pend% ending standard par
      \label{div_pvv.3.125}
	  
	% new div opening: depth here is 2
	
	  \bigskip
	  \begingroup
	
	    \large
	  
	    \begin{quote}
	  
	    
	    \stanza[\smallbreak]
	\label{pv.3.125a}\flagstanza{\tiny\textenglish{...3.125a}}क‚श्चिद् भाग इति प्रोक्तो रूपं नास्यापि किञ्च‚न ।\&[\smallbreak]


	
	    \end{quote}
	  
	  \endgroup
	

	  \pstart \leavevmode% starting standard par
	\hphantom{.}‚{\color{DodgerBlue3}‚क‚श्चिद् भागो} ध‚र्म ‚{\color{DodgerBlue3}‚इत्य}‚र्थ‚वाच‚केन श‚ब्दोऽर्थान्त‚र‚निवृत्तिविशिष्टानेव ‚{\tiny $_{lb}$}‚भावानाहेत्यादिना ग्र‚न्थेन\edtext{}{\edlabel{pvv.331-1}\label{pvv.331-1}\lemma{न्थेन}\Bfootnote{प्रोक्तो निर्द्दिष्टो ग‚म्य‚त इति स‚म्ब‚न्धः ।}} । न त्व‚न्य‚व्यावृत्तिर्नाम काचिद‚न्या । त‚द्विशिष्ट‚ञ्च ‚{\tiny $_{lb}$}‚व‚स्तु वाच्य‚मिति किन्त्वाक्षिप्त‚व्यावृत्तिको ध‚र्म एव क‚श्चित् क‚ल्पित‚भेदो ब‚हिर‚ध्य‚व‚{\tiny $_{lb}$}‚साय‚विष‚यः श‚ब्द‚वाच्यः । व‚स्तुतो ‚{\color{DodgerBlue3}‚रूपं नास्यापि किञ्च‚नास्ति} ।
	\pend% ending standard par
      
	  \bigskip
	  \begingroup
	
	    \large
	  
	    \begin{quote}
	  
	    
	    \stanza[\smallbreak]
	\label{pv.3.125b}\flagstanza{\tiny\textenglish{...3.125b}}त‚द्ग‚तावेव श‚ब्देभ्यो ग‚म्य‚तेन्य‚निव‚र्त्त‚न‚म् ॥ १२५ ॥\&[\smallbreak]


	
	    \end{quote}
	  
	  \endgroup
	

	  \pstart \leavevmode% starting standard par
	\hphantom{.}‚{\color{DodgerBlue3}‚श‚ब्देभ्य‚स्त‚स्य} ध‚र्म‚स्य नीला‚{\color{DodgerBlue3}‚देर्ग‚ता‚{\tiny $_{3}$}‚वेव\edtext{}{\edlabel{pvv.331-2}\label{pvv.331-2}\lemma{वेव}\Bfootnote{य‚दि भेद‚स्य नीरूप‚त्वं क‚थं त‚र्हि निवृत्तिविशिष्टं वाच्य‚मित्याह त‚द्ग‚तेति ।}} ग‚म्य‚ते अन्य}‚स्य ‚{\color{DodgerBlue3}‚निव‚र्त्त}‚न‚म‚संक‚र‚प्र‚तीति‚{\tiny $_{lb}$}‚साम‚र्थ्यात् । (१२५)
	\pend% ending standard par
      \label{div_pvv.3.126}
	  
	% new div opening: depth here is 2
	
	  \bigskip
	  \begingroup
	
	    \large
	  
	    \begin{quote}
	  
	    
	    \stanza[\smallbreak]
	\label{pv.3.126}\flagstanza{\tiny\textenglish{....3.126}}न त‚त्र ग‚म्य‚ते क‚श्चित् केन‚चिद् भेद‚वान् प‚रः ।&न चापि श‚ब्दो द्व‚य‚कृद‚न्योन्याभाव इत्य‚सौ ॥ १२६ ॥\&[\smallbreak]


	
	    \end{quote}
	  
	  \endgroup
	

	  \pstart \leavevmode% starting standard par
	\hphantom{.}‚{\color{DodgerBlue3}‚न त‚त्र} शाब्द्यां बुद्धौ ‚{\color{DodgerBlue3}‚क‚श्चि}‚न्नीला\edtext{}{\edlabel{pvv.331-3}\label{pvv.331-3}\lemma{न्नीला}\Bfootnote{अन्य‚तो भेद‚स्यानुयायिनः श‚ब्द‚वाच्य‚त्वे सामान्यं त‚देव स्यादित्याह असा‚{\tiny $_{lb}$}‚विति श‚ब्द‚विशेष‚योर्भेदः ।}}दिः ‚{\color{DodgerBlue3}‚केन‚चिद्} व्यावृत्त्यादिना विशिष्टः ‚{\color{DodgerBlue3}‚प‚रो ‚{\tiny $_{lb}$}‚ग‚म्य‚ते । न चापि\edtext{}{\edlabel{pvv.331-4}\label{pvv.331-4}\lemma{चापि}\Bfootnote{न‚न्वेकः श‚ब्दो विधिप्र‚तिषेध‚कृत् क‚थ‚मित्याह ।}} श‚ब्दो} द्व‚य‚स्य व्यावृत्तिव‚च‚न‚स्य त‚द्विशिष्ट‚व‚च‚न‚स्य मुख्य‚तः ‚{\tiny $_{lb}$}‚कृत् क‚र्त्ताऽसंकीर्ण्ण‚ध‚र्म व‚द‚न् साम‚र्थ्याद् व्यावृत्तिञ्चाह । त‚देवाह (।) ‚{\color{DodgerBlue3}‚अन्योन्याभाव ‚{\tiny $_{lb}$}‚इति} (।) य‚स्मात् प‚र‚स्प‚राभाव‚रूपः स‚र्व्वो ध‚र्म‚स्त‚स्मात् त‚द्व‚च‚ने व्यावृत्तिर‚पि ‚{\tiny $_{lb}$}‚साम‚{\tiny $_{4}$}‚र्थ्यादुक्ता । (१२६)
	\pend% ending standard par
      \label{div_pvv.3.127}
	  
	% new div opening: depth here is 2
	
	  \bigskip
	  \begingroup
	
	    \large
	  
	    \begin{quote}
	  
	    
	    \stanza[\smallbreak]
	\label{pv.3.127}\flagstanza{\tiny\textenglish{....3.127}}अरूपो रूप‚व‚त्त्वेन द‚र्श‚नं बुद्धिविप्ल‚वः ।&तेनैवाप‚र‚मार्थोसाव‚न्य‚था न हि व‚स्तुनः ॥ १२७ ॥\&[\smallbreak]


	
	    \end{quote}
	  
	  \endgroup
	

	  \pstart \leavevmode% starting standard par
	\hphantom{.}य‚श्चायं भेदोसाव‚रूपो ‚{\color{DodgerBlue3}‚रूप‚व‚त्वेन} य‚त् ‚{\color{DodgerBlue3}‚त‚द्द‚र्श‚न}\edtext{}{\edlabel{pvv.331-5}\label{pvv.331-5}\lemma{त्}\Bfootnote{एताव‚न्मात्रेण व्यावृत्तिविशिष्ट‚त्वं न द‚ण्डिव‚त्}} म‚स्य स ‚{\color{DodgerBlue3}‚बुद्धिविप्ल‚वः ।\edtext{\textsuperscript{*}}{\edlabel{pvv.331-6}\label{pvv.331-6}\lemma{*}\Bfootnote{दृश्य‚विक‚ल्पैक्येन व‚क्तूश्रोत्रोः ।}} तेनैव} बुद्धिविप्ल‚वेनाप‚र‚मा‚{\color{DodgerBlue3}‚प‚र‚मार्थो}‚ऽस‚त्योसौ भेदः । ‚{\color{DodgerBlue3}‚अन्य‚था} प‚र‚मार्थ‚त्वे व्यावृत्तिर्व्व‚{\color{DodgerBlue3}‚स्तुनो} न व‚स्तु स्यात् । (१२७)
	\pend% ending standard par
      \label{div_pvv.3.128_3.129_3.130_3.131}
	  
	% new div opening: depth here is 2
	

	  \pstart \leavevmode% starting standard par
	त‚च्चायुक्तं ।
	\pend% ending standard par
      \textsuperscript{\textenglish{332/s}}
	  \bigskip
	  \begingroup
	
	    \large
	  
	    \begin{quote}
	  
	    
	    \stanza[\smallbreak]
	\label{pv.3.128a}\flagstanza{\tiny\textenglish{...3.128a}}व्यावृत्तिव‚स्तु भ‚व‚ति भेदोस्यास्मादितीर‚णात् ।\&[\smallbreak]


	
	    \end{quote}
	  
	  \endgroup
	

	  \pstart \leavevmode% starting standard par
	\hphantom{.}न हि व‚स्तुनो ‚{\color{DodgerBlue3}‚व्यावृ\edtext{}{\edlabel{pvv.332-1}\label{pvv.332-1}\lemma{व्यावृ}\Bfootnote{अभिन्ना वाऽवृक्षाद् वृक्ष‚स्य भिन्ना वा उभ‚य‚थापि दूष‚ण‚माह निवृत्तिर‚रूप‚त्वेना‚{\tiny $_{lb}$}‚बुद्ध‚त्वात् स‚ङ्केतावृत्तेश्च न श‚ब्द‚विष‚य‚त्व‚म‚नुभ‚वात्तु वृक्षोयं नावृक्ष इति निश्च‚य (:।) ‚{\tiny $_{lb}$}‚तेनान्य‚निवृत्तिः प्र‚तिषेध‚विक‚ल्पेन क‚ल्पितावृक्षादौ च वृत्तेः श‚ब्दोन्य‚निवृत्त‚माक्षि‚{\tiny $_{lb}$}‚प‚ति । तेनान्य‚निवृत्तिविशिष्ट‚त्व‚मुक्त‚माचार्येण ।}}त्तिर्व्व‚स्तु} भ‚वितुम‚र्ह‚ति । ‚{\color{DodgerBlue3}‚अस्मा}‚देव वृक्षा‚{\color{DodgerBlue3}‚द‚स्य} वृक्ष‚स्य ‚{\color{DodgerBlue3}‚भेद इतीर‚णाद्} विक‚ल्प‚नात् भेद‚स्य व‚स्तुत्वे भिद्य‚माना शिंश‚पैव वा भेदः ‚{\tiny $_{lb}$}‚स्याद् व‚स्त्व‚न्त‚रं‚{\tiny $_{5}$}‚ वा । न ताव‚च्छिंश‚पा ध‚वादेर‚वृक्षाद् भेदाभाव‚प्र‚स‚ङ्गात् । न हि ‚{\tiny $_{lb}$}‚शिंश‚पास्व‚भाव‚ल‚क्ष‚णो भेदो ध‚वादेर‚स्ति येन तेप्य‚वृक्ष‚स्य भेदाः स्युः ॥
	\pend% ending standard par
      

	  \pstart \leavevmode% starting standard par
	स‚र्व्व‚त्र एव हि भिद्य‚मानाभावास्त‚दात्म‚न इति चेत् । न त‚र्ह्येको भेद‚स्त‚त्का‚{\tiny $_{lb}$}‚रिणाम‚त‚त्कारिभ्यो यः श‚ब्द‚वाच्यः । व्य‚क्तिस्व‚भाव‚स्य तु भेद‚स्य संकेताविष‚य‚त्वा‚{\tiny $_{lb}$}‚द‚वाच्य‚तैव । अथ व‚स्त्व‚न्त‚रं भेद‚स्त‚दा त‚स्माद् भेदाख्या (ना) द् व‚स्त्व‚न्त‚राद् ‚{\tiny $_{lb}$}‚भिद्य‚{\tiny $_{6}$}‚मान‚स्य शिंश‚पादेर्भेदो\edtext{}{\edlabel{pvv.332-2}\label{pvv.332-2}\lemma{पादेर्भेदो}\Bfootnote{भेद‚शिंश‚प‚योर्म‚ध्येऽप‚रः ।}} व‚क्त‚व्यः अन्य‚था व‚स्त्व‚न्त‚र‚त्वायोगात् । एव‚ञ्चा‚{\tiny $_{lb}$}‚वृक्ष‚व्यावृत्तेद्व‚र्यावृत्त‚त्वात् शिंश‚पादिर‚वृक्षः क‚र्कादिव‚त् । भेद‚स्य च वृक्षाद् भिन्न‚त्वं ‚{\tiny $_{lb}$}‚भेदान्त‚रोपाधिक‚मेवेति द्र‚व्यान्त‚र‚व‚र्ण्ण‚भेदः स्यात् । न ह्य‚न्योन्य‚स्य भेदः स‚म्ब‚न्धा‚{\tiny $_{lb}$}‚भावेनातिप्र‚स‚ङ्गात् । स‚ति स‚म्ब‚न्धे कार्य‚कार‚ण‚भाव एवासौ । त‚तः स‚र्व्वं कार्यं ‚{\tiny $_{lb}$}‚\leavevmode\ledsidenote{\textenglish{66a/MA}} स्व‚कार‚ण‚स्य भेदः (व्यावृत्तिः) स्यात् ।\edtext{\textsuperscript{*}}{\edlabel{pvv.332-3}\label{pvv.332-3}\lemma{*}\Bfootnote{अत‚दो व्यावृत्तिस्तेनैवं न चैवं त‚द्भेदाभिम‚तेपि मा भूत् ।}} त‚स्माद् योऽयं भे‚{\tiny $_{7}$}‚दः श‚ब्द‚व्य‚व‚हार‚विष‚यो‚{\tiny $_{lb}$}‚ऽनुयायी स क‚ल्पितोऽप‚र‚मार्थः । स्व‚स्व‚भाव‚व्य‚व‚स्थितास्तु भावाः पार‚मार्थिको भेदः ॥
	\pend% ending standard par
      

	  \pstart \leavevmode% starting standard par
	न‚नु य‚दि व‚स्त्वेक\edtext{}{\edlabel{pvv.332-4}\label{pvv.332-4}\lemma{स्त्वेक}\Bfootnote{पूर्व्व‚युक्त्या न व्य‚तिरिक्ता व्यावृर्त्तिर्नापि सामान्यं ।}}रूपं त‚दैकेन श‚ब्देन लिङ्गेन वा व‚स्तुनि प्र‚तिपादिते\edtext{}{\edlabel{pvv.332-5}\label{pvv.332-5}\lemma{तिपादिते}\Bfootnote{अख‚ण्डे स्वीक्रिय‚माणे त‚स्माद् यो येन ध‚र्मेणेत्युक्तेपि प्राग‚धिक‚विधानायाह ‚{\tiny $_{lb}$}‚(।) अ (? आ) कारान्त‚र‚स‚मारोपोत्र (?) श्लेषः (।) स च प्र‚तिप‚त्तिभेदेनानेकः ।}} ‚{\tiny $_{lb}$}‚श‚ब्द‚प्र‚माणान्त‚र‚वृत्तिर्न स्यात् ॥
	\pend% ending standard par
      
	  \bigskip
	  \begingroup
	
	    \large
	  
	    \begin{quote}
	  
	    
	    \stanza[\smallbreak]
	\label{pv.3.128b}\flagstanza{\tiny\textenglish{...3.128b}}एकार्थ‚श्लेष‚विच्छेद एको व्याप्रिय‚ते ध्व‚निः ॥ १२८ ॥\&[\smallbreak]


	
	    \end{quote}
	  
	  \endgroup
	
	  \bigskip
	  \begingroup
	
	    \large
	  
	    \begin{quote}
	  
	    
	    \stanza[\smallbreak]
	\label{pv.3.129a}\flagstanza{\tiny\textenglish{...3.129a}}लिङ्गं वा त‚त्र विच्छिन्नं वाच्यं व‚स्तु न किञ्च‚न ।\&[\smallbreak]


	
	    \end{quote}
	  
	  \endgroup
	

	  \pstart \leavevmode% starting standard par
	\hphantom{.}‚{\color{DodgerBlue3}‚एक‚स्य} नित्य‚त्वादेर‚र्थ‚स्य ‚{\color{DodgerBlue3}‚श्लेषः} स‚म्ब‚न्ध‚स्त‚स्य ‚{\color{DodgerBlue3}‚विच्छेदे} व्यावृत्ता\edtext{}{\edlabel{pvv.332-6}\label{pvv.332-6}\lemma{व्यावृत्ता}\Bfootnote{बुद्धिप्र‚तिभासिनि ध‚र्मिणि बाह्य‚भिन्न‚त‚याध्य‚स्ते श‚ब्द‚स्य (? मा) त्रेण वृत्तेरिति ‚{\tiny $_{lb}$}‚त‚त्राभिप्रायः ।}}वेको ‚{\color{DodgerBlue3}‚ध्व‚नि-} \leavevmode\ledsidenote{\textenglish{333/s}} ‚{\color{DodgerBlue3}‚र्लिङ्गं} वा व्याप्रिय‚ते\edtext{}{\edlabel{pvv.333-1}\label{pvv.333-1}\lemma{ते}\Bfootnote{स्वार्थाभिधाद्वाराऽरोप‚ख‚ण्डेन श‚ब्द‚प्र‚मान्त‚र‚साफ‚ल्य‚मित्य‚र्थः ।}} । त‚त्र ध्व‚नौ ‚{\color{DodgerBlue3}‚लिङ्गो} वा ‚{\color{DodgerBlue3}‚विच्छिन्नं} स‚र्व्व‚तो व्यावृत्तं ‚{\color{DodgerBlue3}‚न किञ्च‚न ‚{\tiny $_{lb}$}‚वाच्य‚म‚{\tiny $_{1}$}‚स्ति} एका व्यावृत्तिः श‚ब्देन ‚{\color{DodgerBlue3}‚लिङ्गेन} वा प्र‚तिपाद्य‚ते न व‚स्त्वित्य‚र्थः ।
	\pend% ending standard par
      
	  \bigskip
	  \begingroup
	
	    \large
	  
	    \begin{quote}
	  
	    
	    \stanza[\smallbreak]
	\label{pv.3.129b}\flagstanza{\tiny\textenglish{...3.129b}}य‚स्याभिधान‚तो व‚स्तुसाम‚र्थ्याद‚खिले ग‚तिः ॥ १२९ ॥\&[\smallbreak]


	
	    \end{quote}
	  
	  \endgroup
	
	  \bigskip
	  \begingroup
	
	    \large
	  
	    \begin{quote}
	  
	    
	    \stanza[\smallbreak]
	\label{pv.3.130a}\flagstanza{\tiny\textenglish{...3.130a}}भ‚वेन्नानाफ‚लः श‚ब्द एकाधारो भ‚व‚त्य‚तः ॥\&[\smallbreak]


	
	    \end{quote}
	  
	  \endgroup
	

	  \pstart \leavevmode% starting standard par
	\hphantom{.}‚{\color{DodgerBlue3}‚य‚स्य} व‚स्तुनोऽ‚{\color{DodgerBlue3}‚भिधान‚तो व‚स्तुसाम‚र्थ्याद‚खिले} कृत‚कानित्यादिव‚स्तुरूपे ‚{\color{DodgerBlue3}‚ग‚ति‚{\tiny $_{lb}$}‚र्भ‚वेत्} । येन श‚ब्द‚प्र‚माणान्त‚र‚वैय‚र्थ्यं स्यात् । ‚{\color{DodgerBlue3}‚अतः श‚ब्दो} भूयान्नानाफ‚लोऽनेक‚ध‚र्म‚{\tiny $_{lb}$}‚प्र‚तीतिफ‚ल ‚{\color{DodgerBlue3}‚एकाधार} एक‚ध‚र्मिनिष्टो ‚{\color{DodgerBlue3}‚भ‚व‚ति} ।
	\pend% ending standard par
      

	  \pstart \leavevmode% starting standard par
	साध्य‚साध‚न‚भावादिमाख्याय सामानाधिक‚र‚ण्य‚न्द‚र्श‚यितुमाह ।\edtext{\textsuperscript{*}}{\edlabel{pvv.333-2}\label{pvv.333-2}\lemma{*}\Bfootnote{य‚दि व‚स्त्वेव श‚ब्दादेर्विष‚य‚स्त‚दा स‚र्वाकार‚प्र‚तीतिप्र‚संगोऽस‚मानाधिक‚र‚ण्या‚{\tiny $_{lb}$}‚द‚य‚श्चेत्य‚न्यापोहे विष‚य‚त्व‚माह ।}}
	\pend% ending standard par
      
	  \bigskip
	  \begingroup
	
	    \large
	  
	    \begin{quote}
	  
	    
	    \stanza[\smallbreak]
	\label{pv.3.130b}\flagstanza{\tiny\textenglish{...3.130b}}विच्छेदं सूच‚य‚न्नेक‚म‚प्र‚तिक्षिप्य व‚र्त्त‚ते ॥ १३० ॥\&[\smallbreak]


	
	    \end{quote}
	  
	  \endgroup
	
	  \bigskip
	  \begingroup
	
	    \large
	  
	    \begin{quote}
	  
	    
	    \stanza[\smallbreak]
	\label{pv.3.131a}\flagstanza{\tiny\textenglish{...3.131a}}य‚दान्य‚त्; तेन स व्याप्त एक‚त्वेन न भास‚ते ।\&[\smallbreak]


	
	    \end{quote}
	  
	  \endgroup
	

	  \pstart \leavevmode% starting standard par
	\hphantom{.}‚{\color{DodgerBlue3}‚विच्छेद}‚म‚नीलादिव्य‚व‚च्छेद‚{\color{DodgerBlue3}‚मेकै‚{\tiny $_{2}$}‚कं} नीलादिश‚ब्दः ‚{\color{DodgerBlue3}‚सूच‚य‚न् य‚दाऽप्र‚तिक्षिप्यान्य‚द‚{\tiny $_{lb}$}‚नुत्प‚ल‚व्य‚व‚च्छेदादौ व‚र्त्त‚ते (।)} त‚द्वाच‚क\edtext{}{\edlabel{pvv.333-3}\label{pvv.333-3}\lemma{क}\Bfootnote{अनुत्प‚ल‚प्र‚योगेऽनेन व्याप्तो नील‚श‚ब्दः ।}}श‚ब्द‚प्र‚योगे स‚ति ‚{\color{DodgerBlue3}‚तेन व्याप्तः} शिष्ट ‚{\tiny $_{lb}$}‚एक‚त्वेन च भास‚ते ।
	\pend% ending standard par
      
	  \bigskip
	  \begingroup
	
	    \large
	  
	    \begin{quote}
	  
	    
	    \stanza[\smallbreak]
	\label{pv.3.131b}\flagstanza{\tiny\textenglish{...3.131b}}सामानाधिक‚र‚ण्यं स्यात् त‚दा बुद्ध्य‚नुरोध‚तः ॥ १३१ ॥\&[\smallbreak]


	
	    \end{quote}
	  
	  \endgroup
	

	  \pstart \leavevmode% starting standard par
	\hphantom{.}‚{\color{DodgerBlue3}‚त‚दा सामानाधिक‚र‚ण्यं स्या}‚न्नीलोत्प‚ल‚मित्युभ‚य‚व्यावृत्तिविशिष्टैक‚व‚स्तुव्य‚व‚{\tiny $_{lb}$}‚सायिकाया ‚{\color{DodgerBlue3}‚बुद्धेर‚नुरोध‚तः} ॥ (१३१)
	\pend% ending standard par
      \label{div_pvv.3.132}
	  
	% new div opening: depth here is 2
	

	  \pstart \leavevmode% starting standard par
	किञ्च ।
	\pend% ending standard par
      
	  \bigskip
	  \begingroup
	
	    \large
	  
	    \begin{quote}
	  
	    
	    \stanza[\smallbreak]
	\label{pv.3.132}\flagstanza{\tiny\textenglish{....3.132}}व‚स्तुध‚र्म‚स्य संस्प‚र्शो विच्छेद‚क‚र‚णे ध्व‚नेः ।&स्यात् स‚त्यं स‚ति त‚त्त्वे हि नैक‚व‚स्त्व‚भिधायिनि ॥ १३२ ॥\&[\smallbreak]


	
	    \end{quote}
	  
	  \endgroup
	

	  \pstart \leavevmode% starting standard par
	\hphantom{.}‚{\color{DodgerBlue3}‚ध्व‚नेर्विच्छेद‚क‚र‚णे} व्यावृत्ति\edtext{}{\edlabel{pvv.333-4}\label{pvv.333-4}\lemma{व्यावृत्ति}\Bfootnote{स्वार्थांभिधान‚द्वारेण ।}}प्र‚तिपाद‚क‚त्वेऽव‚स्थिते ‚{\color{DodgerBlue3}‚व‚स्तुध‚र्म‚स्य} नीलादेः ‚{\tiny $_{lb}$}‚‚{\color{DodgerBlue3}‚संस्प‚र्शः} प्र‚वृत्तिविष‚य‚त्वं‚{\tiny $_{3}$}‚ ‚{\color{DodgerBlue3}‚स्यात्} । स विच्छेदो हि य‚स्मात् त‚त्र व‚स्तुनि ‚{\color{DodgerBlue3}‚स‚त्यं} स‚न् ‚{\tiny $_{lb}$}‚त‚स्मात् सूच‚ने त‚द्व‚ती प्र‚वृत्तिर्युक्ता । ‚{\color{DodgerBlue3}‚एकं} सामान्यं ‚{\color{DodgerBlue3}‚व‚स्तु त‚द‚भिधायिनि} तु ध्व‚नौ ‚{\tiny $_{lb}$}‚न व‚स्तुसंस्प‚र्शः ॥ (१३२)
	\pend% ending standard par
      \label{div_pvv.3.133_3.134_3.135}
	  
	% new div opening: depth here is 2
	

	  \pstart \leavevmode% starting standard par
	हेतुमाह (।)
	\pend% ending standard par
      \textsuperscript{\textenglish{334/s}}
	  \bigskip
	  \begingroup
	
	    \large
	  
	    \begin{quote}
	  
	    
	    \stanza[\smallbreak]
	\label{pv.3.133}\flagstanza{\tiny\textenglish{....3.133}}बुद्धाव‚भास‚मान‚स्य दृश्य‚स्याभाव‚निश्च‚यात् ।&तेनान्यापोह‚विष‚याः प्रोक्ताः सामान्य‚गोच‚राः ॥ १३३ ॥\&[\smallbreak]


	
	    \end{quote}
	  
	  \endgroup
	
	  \bigskip
	  \begingroup
	
	    \large
	  
	    \begin{quote}
	  
	    
	    \stanza[\smallbreak]
	\label{pv.3.134a}\flagstanza{\tiny\textenglish{...3.134a}}श‚ब्दाश्च बुद्ध‚य‚श्चैव व‚स्तुन्येषाम‚संभ‚वात् ।\&[\smallbreak]


	
	    \end{quote}
	  
	  \endgroup
	

	  \pstart \leavevmode% starting standard par
	\hphantom{.}‚{\color{DodgerBlue3}‚बुद्धाव‚भास‚मान्य‚स्य दृश्य‚स्य} स‚म्म‚त‚स्य त‚स्य स्व‚भावानुप‚ल‚ब्धेर‚{\color{DodgerBlue3}‚भाव‚निश्च‚यात्} । ‚{\tiny $_{lb}$}‚न व‚स्तुनि स‚त्त्व‚मिति क‚थं त‚त्प्र‚तिपादिकायाः श्रुतेर्व्व‚स्तुनि वृत्तिः । ‚{\color{DodgerBlue3}‚तेन}\edtext{\textsuperscript{*}}{\edlabel{pvv.334-1}\label{pvv.334-1}\lemma{*}\Bfootnote{य‚तो व‚स्तुनि श‚ब्दार्थे दोषः ।}} व्य‚व‚च्छेद‚स्य ‚{\tiny $_{lb}$}‚व‚स्तुनि स‚त्त्वेन ‚{\color{DodgerBlue3}‚सामान्य \edtext{}{\edlabel{pvv.334-2}\label{pvv.334-2}\lemma{सामान्य}\Bfootnote{विक‚ल्पिका इत्य‚र्थः ।}}गोच‚राश्श‚ब्दाः‚{\tiny $_{4}$}‚ बुद्ध‚य‚श्च} क‚ल्पिका ‚{\color{DodgerBlue3}‚अन्या\edtext{}{\edlabel{pvv.334-3}\label{pvv.334-3}\lemma{अन्या}\Bfootnote{अन्योऽपोह्य‚तेऽनेंनेति विक‚ल्पाकारोपोहः ।}}पोह‚विष‚या} आ चा र्ये ण प्रोक्ता (:) । अपोहः श‚ब्द‚लिङ्गाभ्यां प्र‚तिपाद्य‚त इति ब्रुव‚ता । न तु ‚{\tiny $_{lb}$}‚भूत‚सामान्य‚विष‚या । ‚{\color{DodgerBlue3}‚व‚स्तु}‚न्येषां नील‚त्वादीनाम‚नुप‚ल‚ब्धिबाधित‚त्वेना‚{\color{DodgerBlue3}‚स‚म्भ‚वात्} ।
	\pend% ending standard par
      
	  \bigskip
	  \begingroup
	
	    \large
	  
	    \begin{quote}
	  
	    
	    \stanza[\smallbreak]
	\label{pv.3.134b}\flagstanza{\tiny\textenglish{...3.134b}}एक‚त्वाद् व‚स्तुरूप‚स्य भिन्न‚रूपा म‚तिः कुतः ॥ १३४ ॥\&[\smallbreak]


	
	    \end{quote}
	  
	  \endgroup
	
	  \bigskip
	  \begingroup
	
	    \large
	  
	    \begin{quote}
	  
	    
	    \stanza[\smallbreak]
	\label{pv.3.135a}\flagstanza{\tiny\textenglish{...3.135a}}अन्व‚य‚व्य‚तिरेकौ वा नैक‚स्यैकार्थ‚गोच‚रौ ।\&[\smallbreak]


	
	    \end{quote}
	  
	  \endgroup
	

	  \pstart \leavevmode% starting standard par
	\hphantom{.}य‚दि वाच्यं ‚{\color{DodgerBlue3}‚त‚देक\edtext{}{\edlabel{pvv.334-4}\label{pvv.334-4}\lemma{देक}\Bfootnote{त‚च्छ‚ब्द‚वाच्यं सामान्यं स्व‚ल‚क्ष‚णाद् भिन्न‚म‚भिन्नं वाऽतः प्राह । वैशेषिक‚{\tiny $_{lb}$}‚साङ्ख्यादेः ।}} त्वाद् व‚स्तुरूप‚स्य} । त‚स्मिन् ‚{\color{DodgerBlue3}‚भिन्न‚रूपा}\edtext{}{\edlabel{pvv.334-5}\label{pvv.334-5}\lemma{स्मिन्}\Bfootnote{स्व‚सामान्याकारा ।}} अनित्य‚कृत‚क‚त्वाद्या ‚{\tiny $_{lb}$}‚‚{\color{DodgerBlue3}‚म‚तिः कुतः} ।\edtext{\textsuperscript{*}}{\edlabel{pvv.334-6}\label{pvv.334-6}\lemma{*}\Bfootnote{अख‚ण्डं य‚दि श‚ब्द‚वाच्यं ।}} एक‚त्वाद् विष‚य‚स्यैक‚रूप‚व‚द् बुद्धिर्युक्ता विशेष‚श्च स‚र्व‚तो व्यावृत्त ‚{\tiny $_{lb}$}‚इति त‚दात्म‚भूतं सामान्य‚म‚{\tiny $_{5}$}‚पि त‚था स्यात् । त‚त‚श्चैक‚स्य सामान्य‚स्यान्व‚य‚व्य‚ति‚{\tiny $_{lb}$}‚रेकानुवृत्त्य‚न‚नुवृत्ती ‚{\color{DodgerBlue3}‚एकार्थ‚गोच‚रौ} व्य‚क्त्य‚न्त‚र‚विष‚यौ न स‚म्भ‚व‚तः । त‚था हि सामान्यं ‚{\tiny $_{lb}$}‚व्य‚क्त्य‚न्त‚रानुयायि न व्य‚क्तिरिति व‚द‚ता व्य‚क्त्य‚भिन्नात्म‚नः सामान्य‚स्यान्व‚य‚{\tiny $_{lb}$}‚‚{\color{DodgerBlue3}‚व्य‚तिरेकौ} विरुद्धाव‚भ्युग‚तौ स्यातां । त‚च्चायुक्तं ।
	\pend% ending standard par
      
	  \bigskip
	  \begingroup
	
	    \large
	  
	    \begin{quote}
	  
	    
	    \stanza[\smallbreak]
	\label{pv.3.135b}\flagstanza{\tiny\textenglish{...3.135b}}अभेद‚व्य‚व‚हाराश्च भेदे स्युर‚निब‚न्ध‚नाः ॥ १३५ ॥\&[\smallbreak]


	
	    \end{quote}
	  
	  \endgroup
	

	  \pstart \leavevmode% starting standard par
	अथ‚वा\edtext{}{\edlabel{pvv.334-7}\label{pvv.334-7}\lemma{वा}\Bfootnote{भेद‚प‚क्षे दोष‚माह ॥}} व्य‚क्त्यात्म‚नो व्य‚क्तिरूप‚व‚द् ‚{\color{DodgerBlue3}‚भेदे} च सामान्य‚स्या‚{\color{DodgerBlue3}‚भेद\edtext{}{\edlabel{pvv.334-8}\label{pvv.334-8}\lemma{भेद}\Bfootnote{सामान्याधिक‚र‚ण्याद‚यः ।}}व्य‚व‚हारा ‚{\tiny $_{lb}$}‚अनिब‚न्ध‚नाः स्युरे}‚क‚स्यानुयायिनोऽभा‚{\tiny $_{6}$}‚वात् ।
	\pend% ending standard par
      

	  \pstart \leavevmode% starting standard par
	अस्म‚न्म‚ते तु (।)
	\pend% ending standard par
      
	  \bigskip
	  \begingroup
	
	    \large
	  
	    \begin{quote}
	  
	    
	    \stanza[\smallbreak]
	\label{pv.3.136a}\flagstanza{\tiny\textenglish{...3.136a}}स‚र्व‚त्र भावाद् व्यावृत्तेर्नैते दोषाः प्र‚स‚ङ्गिनः ।\&[\smallbreak]


	
	    \end{quote}
	  
	  \endgroup
	\textsuperscript{\textenglish{335/s}}

	  \pstart \leavevmode% starting standard par
	\hphantom{.}‚{\color{DodgerBlue3}‚स‚र्व्व‚त्र} व्य‚क्तिषु ‚{\color{DodgerBlue3}‚भावात् व्यावृत्तेः} स‚जातीयाद्\edtext{}{\edlabel{pvv.335-1}\label{pvv.335-1}\lemma{जातीयाद्}\Bfootnote{स‚जातिभावो विजातिव्यावृत्तिरिति स‚म्ब‚न्धः ।}} विजातीयाच्चैते ‚{\color{DodgerBlue3}‚दोषा} भिन्नाभास\edtext{}{\edlabel{pvv.335-2}\label{pvv.335-2}\lemma{भिन्नाभास}\Bfootnote{अनित्य‚कृत‚त्वादि ।}}बुद्धिविष‚त्वाभावान्व‚य\edtext{}{\edlabel{pvv.335-3}\label{pvv.335-3}\lemma{य}\Bfootnote{व्य‚क्तिरूप‚त्वे ।}}व्य‚तिरेकादि\edtext{}{\edlabel{pvv.335-4}\label{pvv.335-4}\lemma{तिरेकादि}\Bfootnote{अभेद्य‚व्य‚व‚हाराः ।}} विरुद्ध‚ध‚र्माध्यासा ‚{\color{DodgerBlue3}‚अप्र‚स‚ङ्गिनो}\edtext{}{\edlabel{pvv.335-5}\label{pvv.335-5}\lemma{र्माध्यासा}\Bfootnote{य‚थैको र्गौर‚गोर्भिन्न‚स्त‚थान्येपीति नासामान्य‚तादोषः ।}} ‚{\tiny $_{lb}$}‚भ‚व‚न्ति । श‚ब्द‚विक‚ल्पानां भिन्न‚भिन्न‚व्यावृत्तिविष‚य‚त्वात् विजातीय‚व्यावृत्त्याश्र‚येणा‚{\tiny $_{lb}$}‚न्व‚य‚बुद्धिविष‚य‚त्वात् । स‚जातीय‚व्यावृत्त्याश्र‚येण व्य‚तिरेक‚बुद्धिविष‚य‚त्वाच्चेति ॥
	\pend% ending standard par
      

	  \pstart \leavevmode% starting standard par
	क‚स्मात् पुन‚र‚न्य‚व्यावृत्तौ श‚ब्द‚स‚ङ्केतो न स्व‚ल‚{\tiny $_{7}$}‚क्ष‚ण इत्याह ।\leavevmode\ledsidenote{\textenglish{66b/MA}}
	\pend% ending standard par
      \label{div_pvv.3.136_3.137}
	  
	% new div opening: depth here is 2
	
	  \bigskip
	  \begingroup
	
	    \large
	  
	    \begin{quote}
	  
	    
	    \stanza[\smallbreak]
	\label{pv.3.136b}\flagstanza{\tiny\textenglish{...3.136b}}एक‚कार्येषु भावेषु त‚त्कार्य‚प‚रिचोद‚ने ॥ १३६ ॥\&[\smallbreak]


	
	    \end{quote}
	  
	  \endgroup
	
	  \bigskip
	  \begingroup
	
	    \large
	  
	    \begin{quote}
	  
	    
	    \stanza[\smallbreak]
	\label{pv.3.137}\flagstanza{\tiny\textenglish{....3.137}}गौर‚वाश‚क्तिवैफ‚ल्याद् भेदाख्यायाः स‚मा श्रुतिः ।&कृता वृद्धैर‚त‚त्कार्य‚व्यावृत्तिविनिब‚न्ध‚ना ॥ १३७ ॥\&[\smallbreak]


	
	    \end{quote}
	  
	  \endgroup
	

	  \pstart \leavevmode% starting standard par
	\hphantom{.}‚{\color{DodgerBlue3}‚एक\edtext{}{\edlabel{pvv.335-6}\label{pvv.335-6}\lemma{एक}\Bfootnote{विजातीय‚व्यावृत्तं भावं स‚र्व्व‚त्र बुद्ध्या स्वाकाराभेदेनाध्य‚स्त‚मेकं श‚ब्दाभिधेय‚मुक्त्वाधुनाऽभिन्नाकार‚म्विनाप्येक‚कार्येषु च भावेषु कः श‚ब्दो नियोज्य‚त इत्याह ।}}कार्येष्वे}‚क‚त्वाध्य‚व‚साय‚विष‚य‚कार्य‚कारिषु भावेषु ‚{\color{DodgerBlue3}‚त‚त्कार्य‚प‚रि\edtext{}{\edlabel{pvv.335-7}\label{pvv.335-7}\lemma{रि}\Bfootnote{एक‚कार्याणां चोद‚नार्थं त्रिकाल‚स्थानां ।}}चोद}‚न‚निमित्तं ‚{\tiny $_{lb}$}‚प्र‚तिव्य‚क्ति ‚{\color{DodgerBlue3}‚भेदाख्याया} भिन्न‚स्य श‚ब्द‚स्य योज‚नेन संकेत‚क्रियाया व्य‚क्त्यान‚न्त्याद् ‚{\tiny $_{lb}$}‚‚{\color{DodgerBlue3}‚गौर‚वाद}‚श‚क्ते\edtext{}{\edlabel{pvv.335-8}\label{pvv.335-8}\lemma{क्ते}\Bfootnote{स्व‚ल‚क्ष‚ण‚स्याबाध्य‚त्वात्}}वैफ‚ल्याच्च । ‚{\color{DodgerBlue3}‚स‚मा} एका ‚{\color{DodgerBlue3}‚श्रुति}‚र‚त‚त्कार्येभ्यो वा व्यावृत्तिस्त‚न्नि‚{\tiny $_{lb}$}‚व‚न्ध‚ना त‚दाश्र‚या ‚{\color{DodgerBlue3}‚वृद्धैः\edtext{}{\edlabel{pvv.335-9}\label{pvv.335-9}\lemma{वृद्धैः}\Bfootnote{व्य‚व‚हार‚ज्ञैः ।}} कृता} संकेतिता । (१३६, १३७)
	\pend% ending standard par
      \label{div_pvv.3.138}
	  
	% new div opening: depth here is 2
	
	  \bigskip
	  \begingroup
	
	    \large
	  
	    \begin{quote}
	  
	    
	    \stanza[\smallbreak]
	\label{pv.3.138}\flagstanza{\tiny\textenglish{....3.138}}न भावे स‚र्व‚भावानां स्व‚स्व‚भाव‚व्य‚व‚स्थितेः ।&य‚द् रूपं शाव‚लेय‚स्य बाहुलेय‚स्य नास्ति त‚त् ॥ १३८ ॥\&[\smallbreak]


	
	    \end{quote}
	  
	  \endgroup
	

	  \pstart \leavevmode% starting standard par
	\hphantom{.}‚{\color{DodgerBlue3}‚न भावे} स्व‚ल‚क्ष‚णे क‚स्मादित्याह । ‚{\color{DodgerBlue3}‚स‚र्व्व‚भावानां स्व‚स्व‚भाव‚व्य‚व‚स्थितेः\edtext{}{\edlabel{pvv.335-10}\label{pvv.335-10}\lemma{स्थितेः}\Bfootnote{असांक‚र्यात् ।}}} कार‚ण‚तः । ‚{\color{DodgerBlue3}‚य‚द्रूपं शाव‚लेय‚स्य बाहुले‚{\tiny $_{1}$}‚य‚स्य नास्ति त‚द्} रूपं । (१३८)
	\pend% ending standard par
      \label{div_pvv.3.139}
	  
	% new div opening: depth here is 2
	
	  \bigskip
	  \begingroup
	
	    \large
	  
	    \begin{quote}
	  
	    
	    \stanza[\smallbreak]
	\label{pv.3.139a}\flagstanza{\tiny\textenglish{...3.139a}}अत‚त्कार्य‚प‚रावृत्तिर्द्व‚योर‚पि च विद्य‚ते ।\&[\smallbreak]


	
	    \end{quote}
	  
	  \endgroup
	

	  \pstart \leavevmode% starting standard par
	\hphantom{.}‚{\color{DodgerBlue3}‚अत‚त्कार्येभ्यः प‚रावृत्तिश्च द्व‚योः} शाव‚लेय‚बाहुलेय‚यो‚{\color{DodgerBlue3}‚र‚पि विद्य‚ते} ।\edtext{\textsuperscript{*}}{\edlabel{pvv.335-11}\label{pvv.335-11}\lemma{*}\Bfootnote{स‚वार्थाभेदः श‚ब्दाभेद‚स्य कार‚ण‚मेष्ट‚व्यं ।}}त‚त‚स्त‚त्रैव ‚{\tiny $_{lb}$}‚संकेतो न स्व‚ल‚क्ष‚णे ।
	\pend% ending standard par
      

	  \pstart \leavevmode% starting standard par
	स्व‚ल‚क्ष‚णान्येव त‚र्ह्य‚भिन्न‚स्य\edtext{}{\edlabel{pvv.335-12}\label{pvv.335-12}\lemma{स्य}\Bfootnote{एक‚स्य स‚र्व्व‚स‚जातीये}}श‚ब्द‚स्य वाच्यानि स्युरित्याह ।
	\pend% ending standard par
      \textsuperscript{\textenglish{336/s}}
	  \bigskip
	  \begingroup
	
	    \large
	  
	    \begin{quote}
	  
	    
	    \stanza[\smallbreak]
	\label{pv.3.139b}\flagstanza{\tiny\textenglish{...3.139b}}अर्थाभेदेन च विना श‚ब्दाभेदो न युज्य‚ते ॥ १३९ ॥\&[\smallbreak]


	
	    \end{quote}
	  
	  \endgroup
	

	  \pstart \leavevmode% starting standard par
	\hphantom{.}‚{\color{DodgerBlue3}‚अर्थ}‚स्या‚{\color{DodgerBlue3}‚भेदेन} च ‚{\color{DodgerBlue3}‚विना श‚ब्द}‚स्या‚{\color{DodgerBlue3}‚भेदो न युज्य‚ते}‚ऽतिप्र‚स‚ङ्गात् । (१३९)
	\pend% ending standard par
      \label{div_pvv.3.140_3.141_3.142_3.143_3.144}
	  
	% new div opening: depth here is 2
	

	  \pstart \leavevmode% starting standard par
	एव‚न्त‚र्ह्येक‚कार्य‚तैव भिन्न‚श‚ब्द‚वाच्या स्यादित्याह ।
	\pend% ending standard par
      
	  \bigskip
	  \begingroup
	
	    \large
	  
	    \begin{quote}
	  
	    
	    \stanza[\smallbreak]
	\label{pv.3.140a}\flagstanza{\tiny\textenglish{...3.140a}}त‚स्मात् त‚त्कार्य‚तापीष्टाऽत‚त्कार्यादेव भिन्न‚ता ।\&[\smallbreak]


	
	    \end{quote}
	  
	  \endgroup
	

	  \pstart \leavevmode% starting standard par
	य‚स्माद‚र्थाभेदाच्छ‚ब्दाभेदः कार्य‚ञ्च प्र‚तिव्य‚क्ति भिन्न‚मिति त‚त्कार्य‚ताप्येक‚{\tiny $_{lb}$}‚कार्योच्य‚ते । साप्य‚त‚त्कार्याद् भिन्न\edtext{}{\edlabel{pvv.336-1}\label{pvv.336-1}\lemma{भिन्न}\Bfootnote{न तु त‚त्कार्य‚ता नाम सामान्य‚म‚स्ति ।}}तैव‚{\tiny $_{2}$}‚ स चापोहः ।
	\pend% ending standard par
      

	  \pstart \leavevmode% starting standard par
	अत‚त्कार्य‚व्यावृत्तौ संकेत‚क्रियां द्र‚ढ‚यितुं दृष्टान्त‚माह ।
	\pend% ending standard par
      
	  \bigskip
	  \begingroup
	
	    \large
	  
	    \begin{quote}
	  
	    
	    \stanza[\smallbreak]
	\label{pv.3.140b}\flagstanza{\tiny\textenglish{...3.140b}}च‚क्षुरादाव‚नेक‚त्र रूप‚विज्ञान‚के क्व‚चित् ॥ १४० ॥\&[\smallbreak]


	
	    \end{quote}
	  
	  \endgroup
	
	  \bigskip
	  \begingroup
	
	    \large
	  
	    \begin{quote}
	  
	    
	    \stanza[\smallbreak]
	\label{pv.3.141}\flagstanza{\tiny\textenglish{....3.141}}अविशेषेण त‚त्कार्य‚चोद‚नासंभ‚वे स‚ति ।&स‚कृत् स‚र्व्व‚प्र‚तीत्य‚र्थं क‚श्चित् साङ्केतिकीं श्रुतिम् ॥ १४१ ॥\&[\smallbreak]


	
	    \end{quote}
	  
	  \endgroup
	
	  \bigskip
	  \begingroup
	
	    \large
	  
	    \begin{quote}
	  
	    
	    \stanza[\smallbreak]
	\label{pv.3.142a}\flagstanza{\tiny\textenglish{...3.142a}}कुर्यादृतेपि त‚द्रूप‚सामान्याद् व्य‚तिरेकिणः ।\&[\smallbreak]


	
	    \end{quote}
	  
	  \endgroup
	

	  \pstart \leavevmode% starting standard par
	\hphantom{.}‚{\color{DodgerBlue3}‚य‚था च‚क्षुरादा}‚व‚नेक‚त्र ‚{\color{DodgerBlue3}‚क्व‚चिद् रूप‚विज्ञान}‚फ‚ले विष‚य‚भूते‚{\color{DodgerBlue3}‚ऽविशेषेण} सामान्येन\edtext{}{\edlabel{pvv.336-2}\label{pvv.336-2}\lemma{सामान्येन}\Bfootnote{असाधार‚ण‚कार‚ण‚चोद‚ने तु च‚क्षुरादीनां नैकाश्रुतिः ।}} ‚{\tiny $_{lb}$}‚‚{\color{DodgerBlue3}‚त‚स्य} च‚क्षुरादि‚{\color{DodgerBlue3}‚कार्य}‚स्य कार‚ण‚वाच‚कैक‚श‚ब्द‚द्वारेण ‚{\color{DodgerBlue3}‚चोद‚ना}‚याः प‚रेभ्यः प्र‚काश‚नाया‚{\tiny $_{lb}$}‚श्च‚क्षुरालोकादिष्वेक‚सामान्य‚भावेपि ‚{\color{DodgerBlue3}‚संभ‚वे स‚ति क‚श्चित्} स‚न्धेय‚व्य‚व‚हार‚रुचिः ‚{\tiny $_{lb}$}‚‚{\color{DodgerBlue3}‚स‚कृत् स‚र्व्व‚स्य} च‚क्षुरादेः ‚{\color{DodgerBlue3}‚प्र‚तीत्य‚र्थं\edtext{}{\edlabel{pvv.336-3}\label{pvv.336-3}\lemma{र्थं}\Bfootnote{व्य‚व‚हार‚लाघ‚वार्थं । अविशेषेण साम‚ग्रीप्र‚श्ने ।}} सांकेतिकीं} श्रुतिं ‚{\color{DodgerBlue3}‚कुर्यात्} (।) कुतो रूपं\edtext{}{\edlabel{pvv.336-4}\label{pvv.336-4}\lemma{रूपं}\Bfootnote{प‚रो रूप‚विज्ञान‚हेतुः स‚रो वेति ।}} ‚{\tiny $_{lb}$}‚विज्ञान‚मिति ‚{\color{DodgerBlue3}‚ऋते}‚{\tiny $_{3}$}‚विना‚{\color{DodgerBlue3}‚पि त‚द्रूपाद्} रूप‚विज्ञान‚ज‚न‚कात् ‚{\color{DodgerBlue3}‚सामान्या}‚च्च‚क्षुरादि‚{\color{DodgerBlue3}‚व्य‚ति‚{\tiny $_{lb}$}‚रेकिणः} ।
	\pend% ending standard par
      
	  \bigskip
	  \begingroup
	
	    \large
	  
	    \begin{quote}
	  
	    
	    \stanza[\smallbreak]
	\label{pv.3.142b}\flagstanza{\tiny\textenglish{...3.142b}}एक‚वृत्तेर‚नेकोपि य‚द्येक‚श्रुतिमान् भ‚वेत् ॥ १४२ ॥\&[\smallbreak]


	
	    \end{quote}
	  
	  \endgroup
	
	  \bigskip
	  \begingroup
	
	    \large
	  
	    \begin{quote}
	  
	    
	    \stanza[\smallbreak]
	\label{pv.3.143a}\flagstanza{\tiny\textenglish{...3.143a}}वृत्तिराधेय‚ता व्य‚क्तिरिति त‚स्मिन्न युज्य‚ते ।&नित्य‚स्यानुप‚कार्य‚त्वात् नाधारः;\&[\smallbreak]


	
	    \end{quote}
	  
	  \endgroup
	

	  \pstart \leavevmode% starting standard par
	\hphantom{.}अथैक‚स्य सामान्य‚स्य ‚{\color{DodgerBlue3}‚वृत्तेर‚नेकोपि}\edtext{}{\edlabel{pvv.336-5}\label{pvv.336-5}\lemma{स्य}\Bfootnote{एक‚कार्य‚त्वेनैकः श‚ब्दो ब‚हुषु सामान्येन वेत्य‚विशेषं म‚न्य‚ते प‚रः ।}} विष‚यो ‚{\color{DodgerBlue3}‚य‚द्येक‚श्रुतिमान् भ‚वेत्} त‚दा को ‚{\tiny $_{lb}$}‚दोषः ।\edtext{\textsuperscript{*}}{\edlabel{pvv.336-6}\label{pvv.336-6}\lemma{*}\Bfootnote{उप‚ल‚भ्य‚त्वेऽनुप‚ल‚ब्धिर‚नुप‚ल‚भ्य‚त्वे च ब‚हुष्वेक‚श‚ब्दः तुल्य‚ज्ञान‚ञ्च नेत्युक्तं ।}} न‚नु केय‚म्वृत्तिरिष्टा (।) किमाधेय‚ता उत व्य‚क्तिर‚भिव्य‚क्तिः । ‚{\tiny $_{lb}$}‚इत्येत‚द् द्व‚य‚म‚पि त‚स्मिन् सामान्ये न ‚{\color{DodgerBlue3}‚युज्य‚ते} । क‚थ‚मित्याह । ‚{\color{DodgerBlue3}‚नित्य}‚स्यानाधेयाति‚{\tiny $_{lb}$}‚\leavevmode\ledsidenote{\textenglish{337/s}} श‚य‚स्या\edtext{}{\edlabel{pvv.337-1}\label{pvv.337-1}\lemma{स्या}\Bfootnote{इह गोत्व‚मिति स‚म‚वाय‚श्चेन्न स स‚म‚वेत‚स्य क्व‚चित् स्यात् त‚च्च त‚दाय‚त्त‚त्वे ‚{\tiny $_{lb}$}‚त‚दुत्प‚त्त्यान्य‚स्य ।}}विशिष्ट‚रूप‚स्य केन‚चिद‚{\color{DodgerBlue3}‚नुप‚कार्य‚त्वात् नाधारः} क‚श्चित्‚{\tiny $_{4}$}‚ न ह्य‚नुप‚कार‚क ‚{\tiny $_{lb}$}‚आधारोऽतिप्र‚स‚ङ्गात् ।\edtext{\textsuperscript{*}}{\edlabel{pvv.337-2}\label{pvv.337-2}\lemma{*}\Bfootnote{पूर्व्व‚त्रानेकान्त‚माह प‚रः ।}}
	\pend% ending standard par
      

	  \pstart \leavevmode% starting standard par
	न‚नु व‚द‚र‚स्यानुप‚कार‚क‚म‚पि कुण्ड‚माधार इत्याह ।
	\pend% ending standard par
      
	  \bigskip
	  \begingroup
	
	    \large
	  
	    \begin{quote}
	  
	    
	    \stanza[\smallbreak]
	\label{pv.3.143b}\flagstanza{\tiny\textenglish{...3.143b}}प्र‚विस‚र्प्प‚तः ॥ १४३ ॥\&[\smallbreak]


	
	    \end{quote}
	  
	  \endgroup
	
	  \bigskip
	  \begingroup
	
	    \large
	  
	    \begin{quote}
	  
	    
	    \stanza[\smallbreak]
	\label{pv.3.144}\flagstanza{\tiny\textenglish{....3.144}}श‚क्तिस्त‚द्देश‚ज‚न‚नं कुण्डादेर्ब्ब‚द‚रादिषु ।&न संभ‚व‚ति साप्य‚त्र त‚द‚भावेप्य‚व‚स्थितेः ॥ १४४ ॥\&[\smallbreak]


	
	    \end{quote}
	  
	  \endgroup
	

	  \pstart \leavevmode% starting standard par
	\hphantom{.}‚{\color{DodgerBlue3}‚ब‚द‚रादिषु} गुरुत‚या ‚{\color{DodgerBlue3}‚प्र‚विस‚र्प्प‚तो} विस‚र्प्प‚णात् । प्र‚तिक्ष‚ण‚म‚स‚मान\edtext{}{\edlabel{pvv.337-3}\label{pvv.337-3}\lemma{मान}\Bfootnote{कुण्डात् तुल्य‚देशे ।}}देशोत्प‚त्तेः ‚{\tiny $_{lb}$}‚‚{\color{DodgerBlue3}‚कुण्डादेः} स‚ह‚कारिणः ‚{\color{DodgerBlue3}‚त‚द्देश‚ज‚न‚नं}\edtext{}{\edlabel{pvv.337-4}\label{pvv.337-4}\lemma{कारिणः}\Bfootnote{कुण्ड‚व‚द‚र‚क्ष‚णाद् विशिष्ट‚देश‚ज‚त्वेनैक‚साम‚ग्र‚य‚धीन‚त्व‚मुक्तं ।}} ब‚द‚रोपादान‚देश‚ज‚न‚नं ‚{\color{DodgerBlue3}‚श‚क्तिः} कार‚ण‚त्व‚मा‚{\tiny $_{lb}$}‚धार‚ता (।) सापि श‚क्तिल‚क्ष‚णाधार‚तात्र सामान्ये ‚{\color{DodgerBlue3}‚न} संभ‚व‚ति ज‚न्य‚त्वाभावात् । ‚{\tiny $_{lb}$}‚सामा‚{\tiny $_{5}$}‚न्य‚स्य स्थितिं कुर्व्व‚न् विशेष आधारः स्यादित्याह । ‚{\color{DodgerBlue3}‚त}‚स्य विशेष‚स्या‚{\color{DodgerBlue3}‚भावेप्य‚{\tiny $_{lb}$}‚व‚स्थितेः} (। १४३,१४४)
	\pend% ending standard par
      \label{div_pvv.3.145_3.146}
	  
	% new div opening: depth here is 2
	
	  \bigskip
	  \begingroup
	
	    \large
	  
	    \begin{quote}
	  
	    
	    \stanza[\smallbreak]
	\label{pv.3.145a}\flagstanza{\tiny\textenglish{...3.145a}}न स्थितिः साप्य‚युक्तैव भेदाभेद‚विवेच‚ने ।\&[\smallbreak]


	
	    \end{quote}
	  
	  \endgroup
	

	  \pstart \leavevmode% starting standard par
	न स्थितिर्व्विशेषात् सामान्य‚स्य स्व‚रूपातिरिक्ताया स्थितेर्निष्क्रिय‚त्वेनाभावात् ‚{\tiny $_{lb}$}‚स्व‚रूप‚स्थितिः । त‚च्च प्र‚त्येकं व्य‚क्त्य‚पायेप्य‚स्तीति नासौ त‚तः । अस्तु वा स्थितिः ‚{\tiny $_{lb}$}‚‚{\color{DodgerBlue3}‚साप्य‚युक्तैव भेदाभेद‚विवेच‚ने} क्रिय‚माणे । त‚था ह्याश्र‚य‚हेतुका स्थितिः सामान्या‚{\tiny $_{lb}$}‚भिन्नाऽभिन्ना वा स्यात् । भिन्ना चेत् त‚स्याः कार‚ण‚त्वा‚{\tiny $_{6}$}‚दाश्र‚य स्यात् । न तु ‚{\tiny $_{lb}$}‚सामान्य‚स्य स्थितिर‚स‚म्ब‚न्धात् । सामान्य‚स्यापि चेद् भेदे स‚ति कोऽन‚योः स‚म्ब‚न्धः । ‚{\tiny $_{lb}$}‚न कार्य‚कार‚ण‚भावः स्थितेराश्र‚यादुत्प‚त्तेः । सामान्यादुत्प‚त्तौ वा न व्य‚क्तिराधारः ‚{\tiny $_{lb}$}‚स्यात्(।) स्थितिहेतुत्वाभावात् । सामान्य‚ञ्च\edtext{}{\edlabel{pvv.337-5}\label{pvv.337-5}\lemma{ञ्च}\Bfootnote{अभेद‚प‚क्षे स्थितिक्रिया सामान्य‚क्रियैव सा चायुक्ता ।}} नित्य‚त्वाद‚कार्य‚मेव । न चान्यः\edtext{}{\edlabel{pvv.337-6}\label{pvv.337-6}\lemma{चान्यः}\Bfootnote{व्य‚क्तिसामान्य‚यो (:) स्थितेर‚न्यः ।}} ‚{\tiny $_{lb}$}‚स‚म्ब‚न्धोस्ति निराक‚र‚णात् ।
	\pend% ending standard par
      

	  \pstart \leavevmode% starting standard par
	एवं ग‚म‚न‚प्र‚तिब‚न्धादिष्व‚पि वाच्यं । त‚देवं न ताव‚दाधेय‚ता वृत्तिः ।
	\pend% ending standard par
      

	  \pstart \leavevmode% starting standard par
	व्य‚क्तिर‚पि न यु‚{\tiny $_{7}$}‚क्ता (।)
	\pend% ending standard par
      \textsuperscript{\textenglish{338/s}}
	  \bigskip
	  \begingroup
	
	    \large
	  
	    \begin{quote}
	  
	    
	    \stanza[\smallbreak]
	\label{pv.3.145b}\flagstanza{\tiny\textenglish{...3.145b}}विज्ञानात्प‚त्तियोग्य‚त्वायात्म‚न्य‚न्यानुरोधि य‚त् ॥ १४५ ॥\&[\smallbreak]


	
	    \end{quote}
	  
	  \endgroup
	
	  \bigskip
	  \begingroup
	
	    \large
	  
	    \begin{quote}
	  
	    
	    \stanza[\smallbreak]
	\label{pv.3.146}\flagstanza{\tiny\textenglish{....3.146}}त‚द् व्य‚ङ्ग्यं योग्य‚तायाश्च कार‚णं कार‚कं म‚त‚म् ।&प्रागेवास्य च योग्य‚त्वे त‚द‚पेक्षा न युज्य‚ते ॥ १४६ ॥\&[\smallbreak]


	
	    \end{quote}
	  
	  \endgroup
	\textsuperscript{\textenglish{67a/MA}}

	  \pstart \leavevmode% starting standard par
	\hphantom{.}‚{\color{DodgerBlue3}‚य‚त्} स्व‚रूपेण स्थित‚मेव (सामान्यं) स्व‚विष‚य‚{\color{DodgerBlue3}‚विज्ञानोत्प‚त्तियोग्य‚त्वायात्म‚न्य‚{\tiny $_{lb}$}‚न्यानुरोधि} प‚रापेक्षं ‚{\color{DodgerBlue3}‚त‚द् व्य‚ङ्ग्य‚मु}‚च्य‚ते । य‚च्च योग्य‚तायाः स्व‚रूप‚भूतायाः ‚{\tiny $_{lb}$}‚कार‚कं\edtext{}{\edlabel{pvv.338-1}\label{pvv.338-1}\lemma{कं}\Bfootnote{दीपादि । पूर्व‚म‚योग्य‚स्य योग्य‚त्त्वेनोत्पाद‚नात् ।}} त‚त् कार‚णं म‚तं । व्य‚तिरिक्त‚योग्य‚ताक‚र‚णे तु न स्यात् सामान्य\edtext{}{\edlabel{pvv.338-2}\label{pvv.338-2}\lemma{सामान्य}\Bfootnote{व्य‚ञ्ज‚कात्तु कार‚क‚स्य विशेषो ज‚न‚न‚मात्रं न ज्ञान‚ज‚न‚न‚योग्य‚ता धूमेनाज‚न‚{\tiny $_{lb}$}‚केनापि न व्य‚भिचारोत्र व‚ह्निर्ज‚न‚को धूम‚ज‚त्वाज्ज्ञान‚स्यान्य‚थाग्निस्व‚ल‚क्ष‚णाकार‚त्वं ‚{\tiny $_{lb}$}‚स्यात् ।}} स्योप‚{\tiny $_{lb}$}‚ल‚ब्धिः । उप‚ल‚म्भायोग्य‚स्व‚भाव‚त्वात् । योग्य‚तास‚म्ब‚न्धेप्य‚नुप‚ल‚भ्य‚स्व‚भावान‚पायात् ‚{\tiny $_{lb}$}‚त‚द‚व‚स्थोऽनुप‚ल‚म्भः । अथ व्य‚ञ्ज‚काभिम‚तैर्व्विशेषैर्न योग्य‚ता क्रिय‚ते‚{\tiny $_{1}$}‚ त‚दा सा ‚{\tiny $_{lb}$}‚प्रागेवास्तीति स्यात् । त‚था ‚{\color{DodgerBlue3}‚च प्रागेवास्य योग्य‚त्वे त‚द‚पेक्षा} व्य‚ञ्ज‚कापेक्षा ‚{\color{DodgerBlue3}‚न ‚{\tiny $_{lb}$}‚युज्य‚ते} (। १४५, १४६)
	\pend% ending standard par
      \label{div_pvv.3.147}
	  
	% new div opening: depth here is 2
	
	  \bigskip
	  \begingroup
	
	    \large
	  
	    \begin{quote}
	  
	    
	    \stanza[\smallbreak]
	\label{pv.3.147a}\flagstanza{\tiny\textenglish{...3.147a}}सामान्य‚स्याविकार्य‚स्य त‚त्सामान्य‚व‚तः कुतः ।\&[\smallbreak]


	
	    \end{quote}
	  
	  \endgroup
	

	  \pstart \leavevmode% starting standard par
	\hphantom{.}‚{\color{DodgerBlue3}‚त‚त्} त‚स्मात् ‚{\color{DodgerBlue3}‚सामान्य‚व‚तो} विशेष‚स्य स‚काशात् ‚{\color{DodgerBlue3}‚सामान्य‚स्याविकार्य‚स्य} व्य‚क्ति‚{\tiny $_{lb}$}‚र‚पि कुतः स‚म्भ‚व‚ति ।
	\pend% ending standard par
      

	  \pstart \leavevmode% starting standard par
	अथ न सामान्य‚स्य संस्काराद् व्य‚क्तिर्व्य‚ञ्जिका किन्त्विन्द्रिय‚स्याञ्ज‚न‚व‚दि‚{\tiny $_{lb}$}‚त्याह ।
	\pend% ending standard par
      
	  \bigskip
	  \begingroup
	
	    \large
	  
	    \begin{quote}
	  
	    
	    \stanza[\smallbreak]
	\label{pv.3.147b}\flagstanza{\tiny\textenglish{...3.147b}}अञ्ज‚नादेरिव व्य‚क्तेः संस्कारोऽक्ष‚स्य न भ‚वेत् ॥ १४७ ॥\&[\smallbreak]


	
	    \end{quote}
	  
	  \endgroup
	
	  \bigskip
	  \begingroup
	
	    \large
	  
	    \begin{quote}
	  
	    
	    \stanza[\smallbreak]
	\label{pv.3.148a}\flagstanza{\tiny\textenglish{...3.148a}}त‚त्प्र‚तिप‚त्तेर‚भिन्न‚त्वात् त‚द्भावाभाव‚काल‚योः ।\&[\smallbreak]


	
	    \end{quote}
	  
	  \endgroup
	

	  \pstart \leavevmode% starting standard par
	\hphantom{.}‚{\color{DodgerBlue3}‚अञ्ज‚नादेरिवे}‚न्द्रिय‚स्य ‚{\color{DodgerBlue3}‚न} संस्कारो ‚{\color{DodgerBlue3}‚व्य‚क्तेः} स‚काशात् । ‚{\color{DodgerBlue3}‚त\edtext{}{\edlabel{pvv.338-3}\label{pvv.338-3}\lemma{त}\Bfootnote{ज्ञान‚स्य व्य‚क्त्य‚स‚त्वेप्य‚भेदात् स्प‚ष्ट‚त्वेन ।}}त्प्र‚तिप‚त्ते}‚स्त‚स्या ‚{\tiny $_{lb}$}‚व्य‚क्ते\edtext{}{\edlabel{pvv.338-4}\label{pvv.338-4}\lemma{क्ते}\Bfootnote{व्य‚ञ्जिकायाः ।}} ‚{\color{DodgerBlue3}‚र्भावाभाव‚काल‚योर‚भिन्न‚त्वात्} । न हि य‚थे‚{\tiny $_{2}$}‚न्द्रिय‚स्याञ्ज‚न‚संस्कार‚भावाभाव‚{\tiny $_{lb}$}‚योर‚र्थ‚प्र‚तिप‚त्तेः स्प‚ष्टास्प‚ष्ट‚ल‚क्ष‚णो विशेषः त‚था व्य‚क्तिभावाभाव‚काल‚योः सामान्य‚{\tiny $_{lb}$}‚प्र‚तीतेः स‚र्व्व‚दा तुल्याकार‚त्वात् त‚स्याः ॥
	\pend% ending standard par
      \label{div_pvv.3.148_3.149_3.150_3.151_3.152_3.153_3.154_3.155_3.156}
	  
	% new div opening: depth here is 2
	

	  \pstart \leavevmode% starting standard par
	किञ्च ।
	\pend% ending standard par
      
	  \bigskip
	  \begingroup
	
	    \large
	  
	    \begin{quote}
	  
	    
	    \stanza[\smallbreak]
	\label{pv.3.148b}\flagstanza{\tiny\textenglish{...3.148b}}व्य‚ञ्ज‚क‚स्य च जातीनां जातिम‚त्ता य‚दीष्य‚ते ॥ १४८ ॥\&[\smallbreak]


	
	    \end{quote}
	  
	  \endgroup
	
	  \bigskip
	  \begingroup
	
	    \large
	  
	    \begin{quote}
	  
	    
	    \stanza[\smallbreak]
	\label{pv.3.149a}\flagstanza{\tiny\textenglish{...3.149a}}प्राप्तो गोत्वादिना त‚द्वान् प्र‚दीपादिप्र‚काश‚कः ।\&[\smallbreak]


	
	    \end{quote}
	  
	  \endgroup
	\textsuperscript{\textenglish{339/s}}

	  \pstart \leavevmode% starting standard par
	\hphantom{.}‚{\color{DodgerBlue3}‚व्य‚ञ्ज‚क‚स्य}\edtext{\textsuperscript{*}}{\edlabel{pvv.339-1}\label{pvv.339-1}\lemma{*}\Bfootnote{व्य‚ञ्जिकात्वेपि व्य‚क्तेर्जातिम‚त्व‚म‚युक्त‚म‚तिप्र‚स‚ङ्गादित्याह ।}} विशेष‚स्य ‚{\color{DodgerBlue3}‚जातीनां जातिम‚त्ता य‚दीष्य‚ते} (।) त‚दा ‚{\color{DodgerBlue3}‚गोत्वादिना} सामान्येन ‚{\color{DodgerBlue3}‚प्र‚दीपादि}\edtext{}{\edlabel{pvv.339-2}\label{pvv.339-2}\lemma{सामान्येन}\Bfootnote{आदिनेन्द्रिय‚म‚न‚स्कारादि ।}}स्त‚स्य ‚{\color{DodgerBlue3}‚प्र‚काश‚कः त‚द्वान्} सामान्य‚वान् प्राप्तः\edtext{}{\edlabel{pvv.339-3}\label{pvv.339-3}\lemma{प्राप्तः}\Bfootnote{शाव‚लेयादिव‚त् ।}} व्य‚ञ्ज‚क‚{\tiny $_{lb}$}‚ल‚क्ष‚ण‚त्वात् त‚द्व‚तः । त‚स्मान्न वृत्तिर‚भिव्य‚क्तिर‚{\tiny $_{3}$}‚पि त‚त्क‚थ‚मेक‚वृत्तेर‚नेकोप्येक‚श‚ब्द‚{\tiny $_{lb}$}‚वान् स्यात् ॥
	\pend% ending standard par
      

	  \pstart \leavevmode% starting standard par
	किञ्च (।) \edtext{\textsuperscript{*}}{\edlabel{pvv.339-4}\label{pvv.339-4}\lemma{*}\Bfootnote{य‚द्वा स‚र्व्व‚ग‚त‚त्वेपि व्य‚क्तिः स‚त्तानुरोध‚तः । व्य‚क्तिशून्येऽद‚र्श‚नं व्य‚ञ्ज‚{\tiny $_{lb}$}‚काभावात् अत एव ।}}
	\pend% ending standard par
      
	  \bigskip
	  \begingroup
	
	    \large
	  
	    \begin{quote}
	  
	    
	    \stanza[\smallbreak]
	\label{pv.3.149b}\flagstanza{\tiny\textenglish{...3.149b}}व्य‚क्तेर‚न्याथ‚वान‚न्या येषां जातिस्तु विद्य‚ते ॥ १४९ ॥\&[\smallbreak]


	
	    \end{quote}
	  
	  \endgroup
	
	  \bigskip
	  \begingroup
	
	    \large
	  
	    \begin{quote}
	  
	    
	    \stanza[\smallbreak]
	\label{pv.3.150a}\flagstanza{\tiny\textenglish{...3.150a}}तेषां व्य‚क्तिष्व‚पूर्वासु क‚थं सामान्य‚बुद्ध‚यः ।\&[\smallbreak]


	
	    \end{quote}
	  
	  \endgroup
	

	  \pstart \leavevmode% starting standard par
	व्य‚क्तेर‚न्याऽथ‚वाऽन‚न्या अव्य‚तिरिक्ता येषां वै शे षि क सां ख्या दीनां जाति‚{\tiny $_{lb}$}‚र्व्विद्य‚ते\edtext{}{\edlabel{pvv.339-5}\label{pvv.339-5}\lemma{ते}\Bfootnote{व‚स्तुस‚तीत्य‚र्थः ।}} एवेति म‚तं (।) तेषां म‚ते व्य‚क्तिष्व‚पूर्व्वासु क‚थं सामान्य‚बुद्ध‚यः स्युः ।
	\pend% ending standard par
      

	  \pstart \leavevmode% starting standard par
	त‚था हि (।)
	\pend% ending standard par
      
	  \bigskip
	  \begingroup
	
	    \large
	  
	    \begin{quote}
	  
	    
	    \stanza[\smallbreak]
	\label{pv.3.150b}\flagstanza{\tiny\textenglish{...3.150b}}एक‚त्र त‚त्स‚तोन्य‚त्र द‚र्श‚नास‚म्भ‚वात् स‚तः ॥ १५० ॥\&[\smallbreak]


	
	    \end{quote}
	  
	  \endgroup
	
	  \bigskip
	  \begingroup
	
	    \large
	  
	    \begin{quote}
	  
	    
	    \stanza[\smallbreak]
	\label{pv.3.151a}\flagstanza{\tiny\textenglish{...3.151a}}अन‚न्य‚त्वेऽन्व‚याभावाद‚न्य‚त्वेप्य‚न‚पाश्र‚यात् ।\&[\smallbreak]


	
	    \end{quote}
	  
	  \endgroup
	

	  \pstart \leavevmode% starting standard par
	\hphantom{.}अन‚न्य‚त्वे एक‚त्र व्य‚क्तौ त‚दात्म‚नः स‚तः सामान्य‚स्यान्य‚त्र व्य‚क्त्य‚न्त‚रेऽ‚{\color{DodgerBlue3}‚न्व‚या‚{\tiny $_{lb}$}‚भावात्} द‚र्श‚नास‚म्भ‚वात् । क‚थं सामान्य‚बुद्धिः । अन्य‚त्वेपि जातेर‚{\color{DodgerBlue3}‚न‚पाश्र‚या}‚दुक्त‚{\tiny $_{4}$}‚‚{\tiny $_{lb}$}‚युक्त्या\edtext{}{\edlabel{pvv.339-6}\label{pvv.339-6}\lemma{युक्त्या}\Bfootnote{आधाराध‚येत्व‚निषेधात् ।}}ऽश्र‚य‚त्वाभावात् न व्य‚क्तिष्व‚पूर्व्वासु त‚न्निब‚न्ध‚ना धीः स्यात् ।
	\pend% ending standard par
      

	  \pstart \leavevmode% starting standard par
	अपि च\edtext{}{\edlabel{pvv.339-7}\label{pvv.339-7}\lemma{च}\Bfootnote{केन‚चिच्चात्म‚नैक‚त्वं नानात्व‚ञ्चास्य केन‚चिदिति \href{http://sarit.indology.info/?cref=śv.560}{(श्लोक‚वा॰ ५६०)} ‚{\tiny $_{lb}$}‚भिन्नाभिन्न‚प‚क्षोप्य‚युक्त इत्याह । य‚द्व‚स्तुत्वे स‚त्य‚त‚द्रूपं त‚स्य त‚तोन्य‚त्व‚मेव । य‚था ‚{\tiny $_{lb}$}‚सुखाद् दुःख‚स्य । व‚स्तुत्वे स‚त्याश‚क्तिरूप‚ञ्चेष्य‚ते सामान्य‚मित्येत‚द्रूप‚त्वेनान्य‚त्वे ‚{\tiny $_{lb}$}‚व्य‚व‚हार‚स्य साध्य‚त्वात् स्व‚भाव‚हेतुः ॥ त‚च्चेत् सामान्य‚स्य रूप‚म‚न‚न्य‚त्त‚देव त‚द् ‚{\tiny $_{lb}$}‚भ‚व‚ति । अत‚त्वेऽन्य‚त्वात् । अन‚न्य‚त्वेन्व‚याभाव‚स्यैव दोषः ।}} सामान्यं स्वाश्र‚य‚ग‚तं वा\edtext{}{\edlabel{pvv.339-8}\label{pvv.339-8}\lemma{वा}\Bfootnote{पिण्डेष्वेव च सामान्यं नान्त‚रा गृह्य‚ते य‚त इति \href{http://sarit.indology.info/?cref=śv.551}{(कुमारिल) भ‚ट्टः (श्लोक‚वार्तिके ५५१)} ।}} क‚ल्प्य‚ते स‚र्व्व‚ग‚तं वा आकाशादिव‚त् । ‚{\tiny $_{lb}$}‚त‚त्र प्र‚थ‚म‚प‚क्षं दूष‚यितुमाह (।)
	\pend% ending standard par
      
	  \bigskip
	  \begingroup
	
	    \large
	  
	    \begin{quote}
	  
	    
	    \stanza[\smallbreak]
	\label{pv.3.151b}\flagstanza{\tiny\textenglish{...3.151b}}न याति न च त‚त्रासीद‚स्ति प‚श्चान्न चांश‚व‚त् ॥ १५१ ॥\&[\smallbreak]


	
	    \end{quote}
	  
	  \endgroup
	
	  \bigskip
	  \begingroup
	
	    \large
	  
	    \begin{quote}
	  
	    
	    \stanza[\smallbreak]
	\label{pv.3.152a}\flagstanza{\tiny\textenglish{...3.152a}}ज‚हाति पूर्व्वं नाधार‚म‚हो व्य‚स‚न‚स‚न्त‚तिः ।\&[\smallbreak]


	
	    \end{quote}
	  
	  \endgroup
	

	  \pstart \leavevmode% starting standard par
	\leavevmode\ledsidenote{\textenglish{340/s}}एक‚व्य‚क्तिस्थितं सामान्यं व्य‚क्त्य‚न्त‚र‚मुत्प‚द्य‚मानं निष्क्रिय‚त्वा‚{\color{DodgerBlue3}‚न्न याति ।\edtext{\textsuperscript{*}}{\edlabel{pvv.340-1}\label{pvv.340-1}\lemma{*}\Bfootnote{न प्र‚तिबिम्ब‚व‚द‚व‚स्तुत्वात् विरुद्ध‚प‚रिमाणेषु व‚ज्राद‚र्श‚त‚लादिषु । प‚र्व‚तादिस्व‚भावानां भावानां नास्ति स‚म्भ‚वो . य‚तः ।}} न ‚{\tiny $_{lb}$}‚त‚त्र} व्य‚क्त्युत्प‚त्तिदेशे प्रागा‚{\color{DodgerBlue3}‚सीत्} । व्य‚क्तिमात्र‚निष्ठ‚त्वात् । ‚{\color{DodgerBlue3}‚अस्ति} च ‚{\color{DodgerBlue3}‚प‚श्चादु}‚त्प‚त्तेः ‚{\tiny $_{lb}$}‚सामान्यं\edtext{}{\edlabel{pvv.340-2}\label{pvv.340-2}\lemma{सामान्यं}\Bfootnote{यैः प्र‚कारैर्व्य‚क्त्य‚न्त‚रे स‚म्भ‚व‚स्तेनेष्य‚न्ते सामान्य‚ञ्चात्रेति व्याघातः (।) ‚{\tiny $_{lb}$}‚व्य‚क्तिस‚म‚वेत‚भान‚म‚नुत्पादेपीति किमुत्पादेनेति चेन्न । प्र‚तिभासोऽलीकः सामान्या‚{\tiny $_{lb}$}‚भावेपीति न त‚न्मात्रात् स‚त्त्य‚त्व‚मिति नेष्ट‚सिद्धिः ।}} सामान्य\edtext{}{\edlabel{pvv.340-3}\label{pvv.340-3}\lemma{सामान्य}\Bfootnote{न व्य‚क्तिस‚होत्प‚न्नं नित्य‚त्वात् । न व्य‚क्त्युत्पाद एव त‚दुत्पादाभेदात् ।}}शून्याया व्य‚क्तेः स्थित्य‚नुप‚ग‚मा‚{\tiny $_{5}$}‚‚{\color{DodgerBlue3}‚न्न चांश‚व‚त्} (।) न हि सामान्यं ‚{\tiny $_{lb}$}‚साव‚य‚वं येन क्व‚चिदेकेनाव‚य‚वेन स‚म‚वेतं स‚द् अव‚य‚वान्त‚रैरुत्प‚द्य‚मान‚व्य‚क्तिभिः ‚{\tiny $_{lb}$}‚स‚म्ब‚ध्य‚ते । ‚{\color{DodgerBlue3}‚न च पूर्व्वो}‚त्प‚न्न‚{\color{DodgerBlue3}‚माधारं ज‚हाति}\edtext{}{\edlabel{pvv.340-4}\label{pvv.340-4}\lemma{न्न}\Bfootnote{अनंश‚म्वा}} (।) उत्पित्सुव्य‚क्त्य‚न्त‚रेण ‚{\tiny $_{lb}$}‚स‚म्ब‚न्धार्थ‚म‚न्य‚त्र स्थित‚स्याप‚र‚स‚म्ब‚न्ध‚स्त‚द्देशाग‚म‚नं किल न युज्य‚त ‚{\color{DodgerBlue3}‚इत्य‚हो ‚{\tiny $_{lb}$}‚व्य‚स‚न‚स‚न्त‚तिः} स्वाश्र‚य‚ग‚त‚सामान्य‚वादिनां ।
	\pend% ending standard par
      

	  \pstart \leavevmode% starting standard par
	किञ्च (।)
	\pend% ending standard par
      
	  \bigskip
	  \begingroup
	
	    \large
	  
	    \begin{quote}
	  
	    
	    \stanza[\smallbreak]
	\label{pv.3.152b}\flagstanza{\tiny\textenglish{...3.152b}}अन्य‚त्र व‚र्त्त‚मान‚स्य त‚तोन्य‚स्थान‚ज‚न्म‚नि ॥ १५२ ॥\&[\smallbreak]


	
	    \end{quote}
	  
	  \endgroup
	
	  \bigskip
	  \begingroup
	
	    \large
	  
	    \begin{quote}
	  
	    
	    \stanza[\smallbreak]
	\label{pv.3.153a}\flagstanza{\tiny\textenglish{...3.153a}}स्व‚स्थानाद‚च‚ल‚तोन्य‚त्र वृत्तिर‚युक्तिम‚त् ।\&[\smallbreak]


	
	    \end{quote}
	  
	  \endgroup
	

	  \pstart \leavevmode% starting standard par
	\hphantom{.}‚{\color{DodgerBlue3}‚अन्य‚त्र} पूर्व्व‚स्थितायां व्य‚क्तौ ‚{\color{DodgerBlue3}‚व‚र्त्त‚मान‚स्य} सामान्य‚स्य स्व‚स्मात्‚{\tiny $_{6}$}‚ स्था‚{\color{DodgerBlue3}‚नादा-} श्र‚याद‚{\color{DodgerBlue3}‚च‚ल‚तः त‚तः} पूर्व्व‚व्य‚क्ते‚{\color{DodgerBlue3}‚र‚न्य‚त्र} स्थाने ‚{\color{DodgerBlue3}‚ज‚न्म} य‚स्य त‚त्र द्र‚व्ये ‚{\color{DodgerBlue3}‚वृत्तिरित्य‚{\tiny $_{lb}$}‚युक्तिम‚त्} । न हि व्य‚क्त्य‚न्त‚र‚स्थित‚स्य व्य‚क्त्य‚न्त‚र‚म‚नाग‚च्छ‚त‚स्तेन स‚हानुत्प‚द्य‚मान‚स्य ‚{\tiny $_{lb}$}‚च त‚त्स‚म्ब‚न्धो युक्तः ।
	\pend% ending standard par
      

	  \pstart \leavevmode% starting standard par
	त‚था\edtext{}{\edlabel{pvv.340-5}\label{pvv.340-5}\lemma{था}\Bfootnote{पूर्व्व‚व्य‚क्तिदेशाद‚विच‚ल‚द‚पि त‚तोन्य‚देश‚द्र‚व्यं व्याप्नोतीत्य‚त्राह ।}} (।)
	\pend% ending standard par
      
	  \bigskip
	  \begingroup
	
	    \large
	  
	    \begin{quote}
	  
	    
	    \stanza[\smallbreak]
	\label{pv.3.153b}\flagstanza{\tiny\textenglish{...3.153b}}य‚त्रासौ व‚र्त्त‚ते भाव‚स्ते न संब‚ध्य‚तेपि न ॥ १५३ ॥\&[\smallbreak]


	
	    \end{quote}
	  
	  \endgroup
	
	  \bigskip
	  \begingroup
	
	    \large
	  
	    \begin{quote}
	  
	    
	    \stanza[\smallbreak]
	\label{pv.3.154a}\flagstanza{\tiny\textenglish{...3.154a}}त‚द्देशिन‚ञ्च व्याप्नोति किम‚प्येत‚न्म‚हाद्भ्ुत‚म् ।\&[\smallbreak]


	
	    \end{quote}
	  
	  \endgroup
	

	  \pstart \leavevmode% starting standard par
	\hphantom{.}‚{\color{DodgerBlue3}‚य‚त्र} देशेऽसौ\edtext{}{\edlabel{pvv.340-6}\label{pvv.340-6}\lemma{देशेऽसौ}\Bfootnote{प‚श्चाद्भावी ।}} शाव‚लेयादि‚{\color{DodgerBlue3}‚र्भावो व‚र्त्त‚ते} तेन देशेन सामान्यं न स‚म्ब‚ध्य‚ते स्व‚व्य‚{\tiny $_{lb}$}‚क्तिनिष्ठ‚त्वात् त‚स्य । ‚{\color{DodgerBlue3}‚त‚द्देशिनं} सामान्य‚स‚म्ब‚न्ध‚र‚हितो देशो य‚स्य तं विशेष‚ञ्च ‚{\tiny $_{lb}$}‚\leavevmode\ledsidenote{\textenglish{67b/MA}} ‚{\color{DodgerBlue3}‚व्याप्नो‚{\tiny $_{7}$}‚ती}‚ति ‚{\color{DodgerBlue3}‚किम‚पि म‚हाद्भुत‚मेत‚त्} दुर्बुद्धिविल‚पितेषु ॥
	\pend% ending standard par
      \textsuperscript{\textenglish{341/s}}
	  \bigskip
	  \begingroup
	
	    \large
	  
	    \begin{quote}
	  
	    
	    \stanza[\smallbreak]
	\label{pv.3.154b}\flagstanza{\tiny\textenglish{...3.154b}}व्य‚क्त्य‚वैक‚त्र व्य‚क्ताऽथ स‚र्व‚गा जातिरिष्य‚ते ॥ १५४ ॥\&[\smallbreak]


	
	    \end{quote}
	  
	  \endgroup
	
	  \bigskip
	  \begingroup
	
	    \large
	  
	    \begin{quote}
	  
	    
	    \stanza[\smallbreak]
	\label{pv.3.155a}\flagstanza{\tiny\textenglish{...3.155a}}स‚र्व‚त्र दृश्येताभेदात् सापि न व्य‚क्त्य‚पेक्षिणी ।\&[\smallbreak]


	
	    \end{quote}
	  
	  \endgroup
	

	  \pstart \leavevmode% starting standard par
	\hphantom{.}‚{\color{DodgerBlue3}‚अथ\edtext{}{\edlabel{pvv.341-1}\label{pvv.341-1}\lemma{अथ}\Bfootnote{स्वाश्र‚येन्द्रिय‚संयोगात् प‚क्ष‚त्वात्त‚च्छून्येन दृश्य‚ते जातिरिति स‚माधातुर‚पि ।}} स‚र्व्व‚त्र‚गा} जातिरिष्य‚ते ‚{\color{DodgerBlue3}‚त‚दैक‚त्रैव} देशे ‚{\color{DodgerBlue3}‚व्य‚क्त्या व्य‚क्ता} सा जातिः ‚{\tiny $_{lb}$}‚‚{\color{DodgerBlue3}‚स‚र्व्व‚त्र}\edtext{}{\edlabel{pvv.341-2}\label{pvv.341-2}\lemma{जातिः}\Bfootnote{व्य‚क्तिशून्येपि नो चेत् स्व‚भाव‚नानात्वाप्र‚स‚ङ्गः ।}} देशे ‚{\color{DodgerBlue3}‚दृश्येताभेदात्} । एक‚व्य‚क्तिव्य‚क्तं रूपं ‚{\color{DodgerBlue3}‚स‚र्व्व‚त्र} विद्य‚मान‚म‚भिन्न‚मित्युप‚{\tiny $_{lb}$}‚ल‚ब्धिप्र‚स‚ङ्गः । व्य‚ङ्ग्य‚व्य‚ञ्ज‚क‚भाव‚श्चायुक्त इत्युक्तं\edtext{}{\edlabel{pvv.341-3}\label{pvv.341-3}\lemma{इत्युक्तं}\Bfootnote{वृत्तिराधेय‚तेत्यादिना ।}} (।) भ‚व‚तु वा त‚थापि न ‚{\tiny $_{lb}$}‚‚{\color{DodgerBlue3}‚सा} जाति‚{\color{DodgerBlue3}‚र्व्य‚क्त्य‚पेक्षिणी} ।
	\pend% ending standard par
      

	  \pstart \leavevmode% starting standard par
	त‚था हि ।
	\pend% ending standard par
      
	  \bigskip
	  \begingroup
	
	    \large
	  
	    \begin{quote}
	  
	    
	    \stanza[\smallbreak]
	\label{pv.3.155b}\flagstanza{\tiny\textenglish{...3.155b}}व्य‚ञ्ज‚क‚स्याप्र‚तीतौ न व्य‚ङ्ग्यं स‚म्य‚क् प्र‚तीय‚ते ॥ १५५ ॥\&[\smallbreak]


	
	    \end{quote}
	  
	  \endgroup
	
	  \bigskip
	  \begingroup
	
	    \large
	  
	    \begin{quote}
	  
	    
	    \stanza[\smallbreak]
	\label{pv.3.156a}\flagstanza{\tiny\textenglish{...3.156a}}विप‚र्य‚यः पुनः क‚स्मादिष्टः सामान्य‚त‚द्व‚तोः ।\&[\smallbreak]


	
	    \end{quote}
	  
	  \endgroup
	

	  \pstart \leavevmode% starting standard par
	व्य‚ञ्ज‚क‚स्य\edtext{}{\edlabel{pvv.341-4}\label{pvv.341-4}\lemma{स्य}\Bfootnote{नागृहीत‚विशेष‚णा विशेष्ये बुद्धिरिति सामान्य‚ग्र‚ह‚द्वारा व्य‚क्तिग्र‚ह इष्टः (।) स च न्याय‚विप‚रीतः ।}} प्र‚दीपादेर‚{\color{DodgerBlue3}‚प्र‚तीतौ न व्य‚ङ्ग्यं} घ‚टादिः ‚{\color{DodgerBlue3}‚संप्र‚तीय‚त} इति‚{\tiny $_{1}$}‚ ताव‚त् ‚{\tiny $_{lb}$}‚स्थितं । ‚{\color{DodgerBlue3}‚सामान्य‚त‚द्व‚तोः पुनः क‚स्माद् विप‚र्य‚य इष्टः} । य‚दि व्य‚क्तिः सामान्य‚स्य ‚{\tiny $_{lb}$}‚व्य‚ञ्जिका त‚दा व्य‚क्तेर्ग्र‚हे सामान्यं न गृह्येत । इह तु सामान्य‚स्य व्य‚ङ्ग्य‚स्य प्राग् ‚{\tiny $_{lb}$}‚ग्र‚ह‚णं (।) त‚त‚स्त‚द्विशिष्ट‚त्वेन व्य‚क्तिर्ग‚ह्य‚त इति किमिष्य‚ते । व्य‚ञ्ज‚काग्र‚हे ‚{\tiny $_{lb}$}‚व्य‚ङ्ग्याग्र‚ह्णात् ।
	\pend% ending standard par
      

	  \pstart \leavevmode% starting standard par
	किञ्च (।)
	\pend% ending standard par
      
	  \bigskip
	  \begingroup
	
	    \large
	  
	    \begin{quote}
	  
	    
	    \stanza[\smallbreak]
	\label{pv.3.156b}\flagstanza{\tiny\textenglish{...3.156b}}पाच‚कादिष्व‚भिन्नेन विनाप्य‚र्थेन वाच‚कः ॥ १५६ ॥\&[\smallbreak]


	
	    \end{quote}
	  
	  \endgroup
	

	  \pstart \leavevmode% starting standard par
	\hphantom{.}य‚दि सामान्य‚मेवाभिन्नाभिधान‚हेतुर्न व्य‚क्त‚यः त‚दा ‚{\color{DodgerBlue3}‚पाच‚कादिष्व‚भिन्नेनार्थेन} पाच‚क‚त्वादि\edtext{}{\edlabel{pvv.341-5}\label{pvv.341-5}\lemma{त्वादि}\Bfootnote{पाच‚कादिर‚न्य‚त्रापीत्य‚नुवृत्त‚प्र‚त्य‚यः ।}}सामान्येन ‚{\color{DodgerBlue3}‚विनैव‚{\tiny $_{2}$}‚} त्व‚न्म‚तेपि पाच‚कादिश‚ब्दः क‚थं ‚{\color{DodgerBlue3}‚वाच‚कः} ।
	\pend% ending standard par
      

	  \pstart \leavevmode% starting standard par
	प्र‚त्य‚य‚श्चानुयायी । क‚र्म प‚च‚नादि\edtext{}{\edlabel{pvv.341-6}\label{pvv.341-6}\lemma{नादि}\Bfootnote{अध्य‚य‚नादि ।}} त‚द्धेतुरिति चेत्\edtext{}{\edlabel{pvv.341-7}\label{pvv.341-7}\lemma{चेत्}\Bfootnote{व्य‚क्तिभ्य एव प्र‚त्य‚योस्तु किं क‚र्मादिना ।}} । आह ।
	\pend% ending standard par
      \label{div_pvv.3.157_3.158_3.159_3.160_3.161_3.162_3.163_3.164_3.165_3.166_3.167_3.168_3.169_3.170_3.171_3.172_3.173}
	  
	% new div opening: depth here is 2
	
	  \bigskip
	  \begingroup
	
	    \large
	  
	    \begin{quote}
	  
	    
	    \stanza[\smallbreak]
	\label{pv.3.157a}\flagstanza{\tiny\textenglish{...3.157a}}भेदान्न हेतुः क‚र्मास्य न जातिः क‚र्म‚संश्र‚यात् ।\&[\smallbreak]


	
	    \end{quote}
	  
	  \endgroup
	\textsuperscript{\textenglish{342/s}}

	  \pstart \leavevmode% starting standard par
	भेदान्न हेतुः क‚र्मादि । क‚र्मापि प्र‚तिव्य‚क्त्येक‚स्याम‚पि पौर्व्वाप‚र्येण भिन्न\edtext{}{\edlabel{pvv.342-1}\label{pvv.342-1}\lemma{भिन्न}\Bfootnote{भिन्न‚म‚भिन्न‚प्र‚त्य‚य‚हेतुर्न भ‚व‚तीत्येकं सामान्य‚मिष्टं त‚द् य‚दि भिन्न‚म‚पि क‚र्माभिन्नं प्र‚त्य‚यं ज‚न‚येत् व्य‚क्तिभिः कोप‚राधः कृतः ।}}मेव ‚{\tiny $_{lb}$}‚त‚त्क‚थं व्य‚क्तिव‚देकाभिधान‚हेतुः । क‚र्म‚सामान्यं\edtext{}{\edlabel{pvv.342-2}\label{pvv.342-2}\lemma{सामान्यं}\Bfootnote{क‚र्म‚णः पाक‚क्रियास‚म‚वेतं ।}} हेतुश्चेदाह । अस्य क‚र्म‚णो ‚{\tiny $_{lb}$}‚जातिर्नाभिन्नाऽभिधान‚हेतुः। त‚स्याः क‚र्म‚संश्र‚यात् । क‚र्म‚णि\edtext{}{\edlabel{pvv.342-3}\label{pvv.342-3}\lemma{णि}\Bfootnote{द्र‚व्याद‚र्थान्त‚र‚भूते ।}} स‚म‚वेत‚त्वात् त‚त्रैवा‚{\tiny $_{lb}$}‚भिधान‚हेतुता स्यान्न‚{\tiny $_{3}$}‚ द्र‚व्ये\edtext{}{\edlabel{pvv.342-4}\label{pvv.342-4}\lemma{व्ये}\Bfootnote{पाच‚क‚श‚ब्दादिना द्र‚व्य‚स्यैव प‚रिच्छेदः ।}} ।
	\pend% ending standard par
      

	  \pstart \leavevmode% starting standard par
	अथ क‚र्म‚सामान्यं क‚र्म‚णि स‚म्ब‚द्धं त‚च्च द्र‚व्य इति प‚र‚म्प‚र‚या त‚त्स‚म्ब‚द्धं त‚न्नि‚{\tiny $_{lb}$}‚मित्तः श‚ब्दो द्र‚व्ये व‚र्त्त‚त इत्याह ।
	\pend% ending standard par
      
	  \bigskip
	  \begingroup
	
	    \large
	  
	    \begin{quote}
	  
	    
	    \stanza[\smallbreak]
	\label{pv.3.157b}\flagstanza{\tiny\textenglish{...3.157b}}श्रुत्य‚न्त‚र‚निमित्त‚त्वात् क‚र्म‚णो न निमित्त‚ता ॥ १५७ ॥\&[\smallbreak]


	
	    \end{quote}
	  
	  \endgroup
	
	  \bigskip
	  \begingroup
	
	    \large
	  
	    \begin{quote}
	  
	    
	    \stanza[\smallbreak]
	\label{pv.3.158a}\flagstanza{\tiny\textenglish{...3.158a}}असंबंधान्न सामान्यं नायुक्तं श‚ब्द‚कार‚णात् ।&अतिप्र‚संगात् ;\&[\smallbreak]


	
	    \end{quote}
	  
	  \endgroup
	

	  \pstart \leavevmode% starting standard par
	\hphantom{.}‚{\color{DodgerBlue3}‚श्रुत्य‚न्त‚र‚स्य} पाच‚कादिश‚ब्दाद‚न्य‚स्य पाक\edtext{}{\edlabel{pvv.342-5}\label{pvv.342-5}\lemma{पाक}\Bfootnote{पाकः पाक इति हि त‚तः क‚र्म‚जातेः स्यान्न पाच‚क इति ।}} इत्यादिश‚ब्द‚स्य ‚{\color{DodgerBlue3}‚निमित‚त्वा}‚त् ‚{\color{DodgerBlue3}‚क‚र्म‚णो} द्र‚व्ये पाच‚क‚श‚ब्द‚वृत्तौ नास्ति निमित्त‚ता प‚रंप‚र‚यापि । त‚थाऽनित्य‚त्वात् स्थित्य‚{\tiny $_{lb}$}‚भावाच्च\edtext{}{\edlabel{pvv.342-6}\label{pvv.342-6}\lemma{भावाच्च}\Bfootnote{प‚च‚त एव क‚र्मास्ति न स‚र्व‚दा क‚र्म ।}} क‚र्म‚ण‚स्त‚स्मि\edtext{}{\edlabel{pvv.342-7}\label{pvv.342-7}\lemma{स्मि}\Bfootnote{क‚र्म‚णि}}न्निवृत्ते पाच‚कादिप्र‚तिप‚त्ति\edtext{}{\edlabel{pvv.342-8}\label{pvv.342-8}\lemma{त्ति}\Bfootnote{क‚र्म‚निमित्त‚त्वे स‚ति पाच‚क इति नोच्येत उच्य‚ते च त‚न्न व‚स्तुभूत‚क्रियानिमित्त‚को व्य‚प‚देशः ।}}र्न्न स्यात् । अप‚च‚त्य‚पि‚{\tiny $_{4}$}‚ ‚{\tiny $_{lb}$}‚त‚द्व्य‚प‚देशो दृश्य‚ते । नित्य‚त्वात् क‚र्म‚सामान्यं पाच‚क‚व्य‚प‚देशे हेतुश्चेदाह (।) ‚{\tiny $_{lb}$}‚अस‚म्ब‚न्धात्\edtext{}{\edlabel{pvv.342-9}\label{pvv.342-9}\lemma{न्धात्}\Bfootnote{न‚ष्टे क‚र्म‚णि सामान्यं न क‚र्म‚णि न क‚र्त्त‚रीति स‚म्ब‚द्ध‚स‚म्ब‚न्धोपि नेत्य‚स‚म्ब‚न्धान्न श‚ब्द‚ज्ञान‚हेतुः ।}} । द्र‚व्येण क‚र्म‚सामान्य‚स्य न स‚म्ब‚न्धः साक्षात् त‚त्रास‚म‚वायात् । ‚{\tiny $_{lb}$}‚नापि प‚रंप‚र‚या क‚र्म‚णा त‚त्स‚म‚वेत‚स्य न‚ष्ट‚त्वात्\edtext{}{\edlabel{pvv.342-10}\label{pvv.342-10}\lemma{त्वात्}\Bfootnote{क‚र्म‚णः ।}}स‚म‚वायिना सामान्येन स‚म्ब‚न्धा‚{\tiny $_{lb}$}‚भावात् । न\edtext{}{\edlabel{pvv.342-11}\label{pvv.342-11}\lemma{न}\Bfootnote{अस‚म्ब‚द्ध‚म‚पि हेतुरित्याह ।}}सामान्यं ‚{\color{DodgerBlue3}‚श‚ब्द}‚स्य पाच‚कादेः ‚{\color{DodgerBlue3}‚कार‚णान्नायुक्तं} किंत्व‚युक्त‚मेवाति‚{\tiny $_{lb}$}‚‚{\color{DodgerBlue3}‚प्र‚स‚ङ्गात्} । क‚र्म‚{\tiny $_{5}$}‚त्व‚स्य स‚म्ब‚न्ध‚व्य‚तिरेकेण व्य‚प‚देश‚हेतुत्वे स‚र्व्व‚त्र त‚थात्व‚प्र\edtext{}{\edlabel{pvv.342-12}\label{pvv.342-12}\lemma{प्र}\Bfootnote{गोत्व‚म‚प्य‚श्व‚ज्ञान‚हेतुः स्यात् ।}}स‚ङ्गः ‚{\tiny $_{lb}$}‚(१५७) ॥
	\pend% ending standard par
      

	  \pstart \leavevmode% starting standard par
	अप‚च‚त्य‚पि पुरुषेऽतीतानाग‚तं क‚र्म त‚द्व्य‚प‚देश‚निमित्त‚मिति चेत् । आह (।)
	\pend% ending standard par
      \textsuperscript{\textenglish{343/s}}
	  \bigskip
	  \begingroup
	
	    \large
	  
	    \begin{quote}
	  
	    
	    \stanza[\smallbreak]
	\label{pv.3.158b}\flagstanza{\tiny\textenglish{...3.158b}}क‚र्मापि नास‚ज्ज्ञानाभिधान‚योः ॥ १५८ ॥\&[\smallbreak]


	
	    \end{quote}
	  
	  \endgroup
	
	  \bigskip
	  \begingroup
	
	    \large
	  
	    \begin{quote}
	  
	    
	    \stanza[\smallbreak]
	\label{pv.3.159a}\flagstanza{\tiny\textenglish{...3.159a}}अनैमित्तिक‚ताप‚त्तेः ;\&[\smallbreak]


	
	    \end{quote}
	  
	  \endgroup
	

	  \pstart \leavevmode% starting standard par
	न क‚र्मास‚द‚पि ज्ञानाभिधान‚योर्हेतुर‚नैमित्तिक‚ताया\edtext{}{\edlabel{pvv.343-1}\label{pvv.343-1}\lemma{ताया}\Bfootnote{अहेतुकौ ज्ञान‚श‚ब्दौ स्यातां ।}} आप‚त्तेः । हेतुं विना ‚{\tiny $_{lb}$}‚भ‚व‚तोऽहेतुक‚त्व‚प्र‚स‚ङ्गात् (।) अतीतानाग‚त‚ञ्च क‚र्माविद्य‚मान‚त्वाद‚स‚देव ।
	\pend% ending standard par
      

	  \pstart \leavevmode% starting standard par
	पाकादिनिर्व‚र्त्तिका श‚क्तिः पाच‚कादिव्य‚प‚देश‚{\tiny $_{6}$}‚हे\edtext{}{\edlabel{pvv.343-2}\label{pvv.343-2}\lemma{हे}\Bfootnote{पाच‚क‚स्था ।}}तुरिति चेत् । आह (।)
	\pend% ending standard par
      
	  \bigskip
	  \begingroup
	
	    \large
	  
	    \begin{quote}
	  
	    
	    \stanza[\smallbreak]
	\label{pv.3.159b}\flagstanza{\tiny\textenglish{...3.159b}}न च श‚क्तिर‚न‚न्व‚यात् ।\&[\smallbreak]


	
	    \end{quote}
	  
	  \endgroup
	

	  \pstart \leavevmode% starting standard par
	\hphantom{.}‚{\color{DodgerBlue3}‚श‚क्ति}‚र्हि द्र‚व्याद‚भिन्ना भिन्ना वा प्र‚तिप‚न्ना । अभिन्ना चेत् त‚दा द्र‚व्य‚व‚{\color{DodgerBlue3}‚द‚न‚न्व‚यान्न} चान्व‚यिनोऽन्व‚यिश‚ब्द‚हेतुता\edtext{}{\edlabel{pvv.343-3}\label{pvv.343-3}\lemma{हेतुता}\Bfootnote{अन्व‚यित्वात् श‚ब्द‚ज्ञानादेर‚पि ।}} । अथ भिन्ना त‚दा\edtext{}{\edlabel{pvv.343-4}\label{pvv.343-4}\lemma{दा}\Bfootnote{श‚क्तेरेवोप‚योगाद् द्र‚व्यानुप‚योगास‚ङ्गः ।}}ऽनुप‚कार‚क‚योः स‚म्ब‚न्धानुप‚प‚त्ते‚{\tiny $_{lb}$}‚र्द्र‚व्येणोप‚क्रिय‚माणा श‚क्तिस्त‚त्स‚म्ब‚न्धिनीति व‚क्त‚व्यं (।) अत्रापि द्र‚व्यं श‚क्त्य‚न्त‚रेण ‚{\tiny $_{lb}$}‚स्व‚य‚मेवास‚म‚र्थ‚स्व‚भाव‚त‚या श‚क्तिमुप‚क‚रोतीति वाच्यं । आद्येऽन‚व‚स्था द्वितीये तु ‚{\tiny $_{lb}$}‚कार्य‚मेव किन्न क‚रोतीत्य‚लं श‚क्तिस्वी‚{\tiny $_{7}$}‚कारेण\edtext{}{\edlabel{pvv.343-5}\label{pvv.343-5}\lemma{कारेण}\Bfootnote{द्र‚व्य‚ञ्च नान्वेतीत्य‚न्व‚यी श‚ब्दो न स्यात् ।}} ॥
	\pend% ending standard par
      \textsuperscript{\textenglish{68a/MA}}‚{\tiny $_{lb}$}‚
	  \bigskip
	  \begingroup
	
	    \large
	  
	    \begin{quote}
	  
	    
	    \stanza[\smallbreak]
	\label{pv.3.159c}\flagstanza{\tiny\textenglish{...3.159c}}सामान्यं पाच‚क‚त्वादि य‚दि प्रागेव त‚द्भ‚वेत् ॥ १५९ ॥\&[\smallbreak]


	
	    \end{quote}
	  
	  \endgroup
	
	  \bigskip
	  \begingroup
	
	    \large
	  
	    \begin{quote}
	  
	    
	    \stanza[\smallbreak]
	\label{pv.3.160}\flagstanza{\tiny\textenglish{....3.160}}व्य‚क्तं स‚त्तादिव‚न्नो चेन्न प‚श्चाद‚विशेष‚तः ।&क्रियोप‚कारापेक्ष‚स्य व्य‚ञ्ज‚क‚त्वेऽविकारिणः ॥ १६० ॥\&[\smallbreak]


	
	    \end{quote}
	  
	  \endgroup
	
	  \bigskip
	  \begingroup
	
	    \large
	  
	    \begin{quote}
	  
	    
	    \stanza[\smallbreak]
	\label{pv.3.161a}\flagstanza{\tiny\textenglish{...3.161a}}अतिश‚ये वा ह्य‚प्य‚स्य क्ष‚णिक‚त्वात् क्रिया कुतः ।\&[\smallbreak]


	
	    \end{quote}
	  
	  \endgroup
	

	  \pstart \leavevmode% starting standard par
	\hphantom{.}अभिन्नाभिधाने निमित्तं ‚{\color{DodgerBlue3}‚सामान्य}‚मेव\edtext{}{\edlabel{pvv.343-6}\label{pvv.343-6}\lemma{मेव}\Bfootnote{पाच‚कादिद्र‚व्येषु ।}} ‚{\color{DodgerBlue3}‚पाच‚क‚त्वादि य‚दी}‚ष्य‚ते त‚दा पाकादि‚{\tiny $_{lb}$}‚क्रियातः ‚{\color{DodgerBlue3}‚प्रागेव त‚त्} पाच‚क‚त्वादि सामान्यं द्र‚व्योत्प‚त्तावेव ‚{\color{DodgerBlue3}‚व्य‚क्तं भ‚वेत्}\edtext{}{\edlabel{pvv.343-7}\label{pvv.343-7}\lemma{त्तावेव}\Bfootnote{य‚दैव व‚स्तु त‚दैव गोत्वादिस‚त्तादियोगात् य‚था स‚त्ता द्र‚व्य‚त्वादि याव‚द्द्र‚व्य‚भावि ।}} (।) ‚{\tiny $_{lb}$}‚नो चेत् प्राग‚व्य‚क्तं न ‚{\color{DodgerBlue3}‚प‚श्चा}‚द‚पि व्य‚क्तं ‚{\color{DodgerBlue3}‚स्याद\edtext{}{\edlabel{pvv.343-8}\label{pvv.343-8}\lemma{स्याद}\Bfootnote{सामान्याश्र‚य‚द्र‚व्य‚स्य ।}}विशेष}‚तो विशेषाभावात् (।) ‚{\tiny $_{lb}$}‚नित्यैक‚रूप‚स्य सामान्य‚स्य क्रियायाः क‚र्म‚णः पाकादि न ‚{\color{DodgerBlue3}‚उप‚कार}‚स्त‚द‚पेक्ष‚स्य द्र‚व्य‚स्य ‚{\tiny $_{lb}$}‚पाच‚क‚त्वादिसामान्य‚{\color{DodgerBlue3}‚व्य‚ञ्ज‚क‚त्वे} चेष्य‚माणे‚{\color{DodgerBlue3}‚ऽविकारिणः} स्थिर‚स्य\edtext{}{\edlabel{pvv.343-9}\label{pvv.343-9}\lemma{स्य}\Bfootnote{अक्ष‚णिक‚त्वान्न स‚ह‚कार्य‚पेक्षा ।}} द्र‚{\tiny $_{1}$}‚व्य‚स्यापेक्षा ‚{\tiny $_{lb}$}‚नास्ति । ‚{\color{DodgerBlue3}‚अतिश‚ये वा} प्राग‚व‚स्थातः स्वीक्रिय‚माणे क्ष‚णिक‚त्वं स्यात् । ‚{\color{DodgerBlue3}‚क्ष‚णिक‚त्वा}‚{\tiny $_{lb}$}‚\leavevmode\ledsidenote{\textenglish{344/s}} दुत्प‚त्त्य‚न‚न्त‚र‚विनाशित्वात् ‚{\color{DodgerBlue3}‚क्रिया} क‚र्म कुतः स‚म्भ‚व‚ति । उत्प‚त्त्य‚न‚न्त‚रं स्थितौ हि ‚{\tiny $_{lb}$}‚क‚र्म कुर्यात् । सा च क्ष‚णिक‚स्य नास्तीति न तेनोप‚कारोपि ।
	\pend% ending standard par
      
	  \bigskip
	  \begingroup
	
	    \large
	  
	    \begin{quote}
	  
	    
	    \stanza[\smallbreak]
	\label{pv.3.161b}\flagstanza{\tiny\textenglish{...3.161b}}तुल्ये भेदे य‚या जातिः प्र‚त्यास‚त्त्या प्र‚स‚र्प‚ति ॥ १६१ ॥\&[\smallbreak]


	
	    \end{quote}
	  
	  \endgroup
	
	  \bigskip
	  \begingroup
	
	    \large
	  
	    \begin{quote}
	  
	    
	    \stanza[\smallbreak]
	\label{pv.3.162a}\flagstanza{\tiny\textenglish{...3.162a}}क्व‚चिन्नान्य‚त्र सैवास्तु श‚ब्द‚ज्ञान‚निब‚न्ध‚न‚म् ।\&[\smallbreak]


	
	    \end{quote}
	  
	  \endgroup
	

	  \pstart \leavevmode% starting standard par
	\hphantom{.}व‚स्तुभूत‚सामान्य‚वादिनोपि व्य‚क्तीनां ‚{\color{DodgerBlue3}‚तुल्ये भेदे} प‚र‚स्प‚र‚मेकार्थ‚क‚र‚णादिक‚या ‚{\tiny $_{lb}$}‚‚{\color{DodgerBlue3}‚य‚या प्र‚त्यास‚त्त्या क्व‚चिद्} व्य‚क्तौ शाव‚लेय‚बाहुलेय‚यो‚{\color{DodgerBlue3}‚र्जातिः‚{\tiny $_{2}$}‚ प्र‚स‚र्प्प‚ति\edtext{}{\edlabel{pvv.344-1}\label{pvv.344-1}\lemma{ति}\Bfootnote{व्याप्य व‚र्त्त‚ते विना सामान्यं ।}} । ‚{\tiny $_{lb}$}‚नान्य‚त्र क‚र्कादौ\edtext{}{\edlabel{pvv.344-2}\label{pvv.344-2}\lemma{र्कादौ}\Bfootnote{नाश्व‚विशेषादौ ।}}} । सैव प्र‚त्यास‚त्तिर‚भिन्न‚स्य ‚{\color{DodgerBlue3}‚श‚ब्द‚ज्ञान‚स्य निब‚न्ध‚म}‚स्तु किं ‚{\tiny $_{lb}$}‚प्र‚माणान्त‚र‚बाधित‚जातिस्वीकारेण ॥
	\pend% ending standard par
      

	  \begin{center}%% label @type='head'
	\textbf{ख. सांख्य‚म‚त‚निरासः}
	\end{center}
	

	  \pstart \leavevmode% starting standard par
	सां ख्या\edtext{}{\edlabel{pvv.344-3}\label{pvv.344-3}\lemma{ख्या}\Bfootnote{त‚स्माद्व्यावृत्तेरेवैक‚त्वाध्य‚व‚सायाद्भावेष्य‚न्व‚य इति स्थित‚म् ।}} प्राहुः ।
	\pend% ending standard par
      
	  \bigskip
	  \begingroup
	
	    \large
	  
	    \begin{quote}
	  
	    
	    \stanza[\smallbreak]
	\label{pv.3.162b}\flagstanza{\tiny\textenglish{...3.162b}}न निवृत्तिं विहायास्ति य‚दि भावान्व‚योप‚रः ॥ १६२ ॥\&[\smallbreak]


	
	    \end{quote}
	  
	  \endgroup
	
	  \bigskip
	  \begingroup
	
	    \large
	  
	    \begin{quote}
	  
	    
	    \stanza[\smallbreak]
	\label{pv.3.163a}\flagstanza{\tiny\textenglish{...3.163a}}एक‚स्य कार्य‚म‚न्य‚स्य न स्याद‚त्य‚न्त‚भेद‚तः ।\&[\smallbreak]


	
	    \end{quote}
	  
	  \endgroup
	

	  \pstart \leavevmode% starting standard par
	\hphantom{.}‚{\color{DodgerBlue3}‚य‚दि निवृत्तिं} विजातीय\edtext{}{\edlabel{pvv.344-4}\label{pvv.344-4}\lemma{विजातीय}\Bfootnote{य‚द् बीज‚स्याङ्कुर‚ज‚न‚कं रूपं न त‚त्पृथिव्यादाव‚स्ति य‚द‚स्ति बुद्ध्यारोपित‚व्यावृत्त‚त्वं न त‚ज्ज‚न‚कं निःस्व‚भाव‚त्वात् । सां ख्य स्तु य‚ज्ज‚न‚कं रूपं बीजे त‚च्चेद‚न्य‚स्यापि स्यात् म‚तेन स्व‚भावेन त‚तोऽभिन्न इत्य‚न्व‚य‚माह । भेदेपि केचित् निय‚ता इत्य‚त्र व्य‚भिचारः ।}}व्य‚व‚च्छेदं ‚{\color{DodgerBlue3}‚विहाय भावानां} स‚त्व‚र‚ज‚स्त‚मःसाम्या‚{\tiny $_{lb}$}‚व‚स्थास्व‚भाव‚प्र‚कृत्यात्म‚नाऽद्व‚य एक‚रूपोऽ‚{\color{DodgerBlue3}‚न्व‚योऽप‚रो नास्ति} (।) ‚{\color{DodgerBlue3}‚त‚देक‚स्य कार्य}‚म‚ङ्कुरो‚{\tiny $_{lb}$}‚‚{\color{DodgerBlue3}‚ऽन्य‚स्य} क्षित्यादेर्न ‚{\color{DodgerBlue3}‚स्यात्} । बीज‚क्षित्यादीनाम‚{\color{DodgerBlue3}‚त्य‚न्त‚भे‚{\tiny $_{3}$}‚द‚तः} । अङ्कुर‚कार‚कं य‚द् रूपं ‚{\tiny $_{lb}$}‚बीज‚स्य त‚च्चेत् क्षित्यादेर्न्नास्ति न स्याद‚सौ त‚त्कार‚कः । अस्तित्वे चैक‚रूपान्व‚यः (।)
	\pend% ending standard par
      

	  \pstart \leavevmode% starting standard par
	सिद्धान्त‚वाद्याह ।
	\pend% ending standard par
      
	  \bigskip
	  \begingroup
	
	    \large
	  
	    \begin{quote}
	  
	    
	    \stanza[\smallbreak]
	\label{pv.3.163b}\flagstanza{\tiny\textenglish{...3.163b}}य‚द्येकात्म‚त‚यानेकः कार्य‚स्यैक‚स्य कार‚कः ॥ १६३ ॥\&[\smallbreak]


	
	    \end{quote}
	  
	  \endgroup
	
	  \bigskip
	  \begingroup
	
	    \large
	  
	    \begin{quote}
	  
	    
	    \stanza[\smallbreak]
	\label{pv.3.164a}\flagstanza{\tiny\textenglish{...3.164a}}आत्मैक‚त्रापि वास्तीति व्य‚र्थाः स्युः स‚ह‚कारिणः ।\&[\smallbreak]


	
	    \end{quote}
	  
	  \endgroup
	

	  \pstart \leavevmode% starting standard par
	\hphantom{.}‚{\color{DodgerBlue3}‚य‚द्येकात्म‚त‚याऽनेको} बीज‚श(?स)लिलादिः ‚{\color{DodgerBlue3}‚कार्य‚स्या}‚ङ्कुर‚{\color{DodgerBlue3}‚स्यैक‚स्य कार‚क‚स्त-} दाङ्कुर‚कार‚क ‚{\color{DodgerBlue3}‚आत्मा}‚ऽनुयायी । ‚{\color{DodgerBlue3}‚एक‚त्रापि} बीजेऽन्य‚त्र वा‚{\color{DodgerBlue3}‚स्तीति व्य‚र्थाः स‚ह‚कारिणः ‚{\tiny $_{lb}$}‚स्युः} एक‚त्र विद्य‚मानेनैव तेनात्म‚ना कार्योत्प‚त्तेः ।
	\pend% ending standard par
      \textsuperscript{\textenglish{345/s}}
	  \bigskip
	  \begingroup
	
	    \large
	  
	    \begin{quote}
	  
	    
	    \stanza[\smallbreak]
	\label{pv.3.164b}\flagstanza{\tiny\textenglish{...3.164b}}नापैत्य‚भिन्नं त‚द् रूपं विशेषाः ख‚ल्व‚पायिनः ॥ १६४ ॥\&[\smallbreak]


	
	    \end{quote}
	  
	  \endgroup
	
	  \bigskip
	  \begingroup
	
	    \large
	  
	    \begin{quote}
	  
	    
	    \stanza[\smallbreak]
	\label{pv.3.165a}\flagstanza{\tiny\textenglish{...3.165a}}एकापाये फ‚लाभावाद् विशेषेभ्य‚स्त‚दुद्भ‚वः ।\&[\smallbreak]


	
	    \end{quote}
	  
	  \endgroup
	

	  \pstart \leavevmode% starting standard par
	स‚ह\edtext{}{\edlabel{pvv.345-1}\label{pvv.345-1}\lemma{ह}\Bfootnote{एत‚देव द्र‚ढ‚य‚न्नाह ।}}कारिणा‚{\tiny $_{4}$}‚म‚भावेपि बीज‚स‚त्वे‚{\color{DodgerBlue3}‚ऽभिन्नं त‚त्} प्र‚धानं\edtext{}{\edlabel{pvv.345-2}\label{pvv.345-2}\lemma{धानं}\Bfootnote{सामान्यं ।}} क‚र्त्तृ‚{\color{DodgerBlue3}‚रूपं नापैति}\edtext{}{\edlabel{pvv.345-3}\label{pvv.345-3}\lemma{र्त्तृ}\Bfootnote{अस्य न विशेषो विशेषेऽभेद‚हानेः ।}} ‚{\tiny $_{lb}$}‚नित्य‚त्वात्\edtext{}{\edlabel{pvv.345-4}\label{pvv.345-4}\lemma{त्वात्}\Bfootnote{एक‚स्य अप्य‚न्ते ।}} व्य‚क्ति\edtext{}{\edlabel{pvv.345-5}\label{pvv.345-5}\lemma{क्ति}\Bfootnote{त्रैगुण्य‚स्य स‚र्व्वात्म‚ना स‚र्व्व‚त्र स‚र्व‚दा स‚त्त्वात् ।}}स‚त्वाच्च । ‚{\color{DodgerBlue3}‚विशे\edtext{}{\edlabel{pvv.345-6}\label{pvv.345-6}\lemma{विशे}\Bfootnote{व्य‚क्तिभेदाः ।}}षाः ख‚लु नून‚म‚पायिनो} न प्र‚धानं ।\edtext{\textsuperscript{*}}{\edlabel{pvv.345-7}\label{pvv.345-7}\lemma{*}\Bfootnote{अत एक‚व्य‚क्तिस‚त्त्वेपि कार्यं स्यान्न चास्त्य‚तः ।}} ‚{\color{DodgerBlue3}‚एक‚स्य} स‚ह‚कारिणो विशेष‚स्या‚{\color{DodgerBlue3}‚पाये}‚पि फ‚ल‚स्या‚{\color{DodgerBlue3}‚भावात् । विशेषेभ्य‚स्त}‚स्य कार्य‚स्य ‚{\color{DodgerBlue3}‚उद्भ‚वो} निश्चीय‚ते न प्र‚धानात् (।) विशेष‚स्यान्व‚य\edtext{}{\edlabel{pvv.345-8}\label{pvv.345-8}\lemma{य}\Bfootnote{पाथ‚पृथिव्यादेर‚ङ्कुरेण ।}}व्य‚तिरेकानुविधानात् । सामान्य‚स्य\edtext{}{\edlabel{pvv.345-9}\label{pvv.345-9}\lemma{स्य}\Bfootnote{एक‚विशेष‚स्थिताव‚प्य‚विक‚ल‚स्य ।}} ‚{\tiny $_{lb}$}‚विप‚र्य‚यात् ।
	\pend% ending standard par
      
	  \bigskip
	  \begingroup
	
	    \large
	  
	    \begin{quote}
	  
	    
	    \stanza[\smallbreak]
	\label{pv.3.165b}\flagstanza{\tiny\textenglish{...3.165b}}स पार‚मार्थिको भावो य एवार्थ‚क्रियाक्ष‚मः ॥ १६५ ॥\&[\smallbreak]


	
	    \end{quote}
	  
	  \endgroup
	
	  \bigskip
	  \begingroup
	
	    \large
	  
	    \begin{quote}
	  
	    
	    \stanza[\smallbreak]
	\label{pv.3.166a}\flagstanza{\tiny\textenglish{...3.166a}}स च नान्वेति योन्वेति न त‚स्मात् कार्य‚स‚म्भ‚वः ।\&[\smallbreak]


	
	    \end{quote}
	  
	  \endgroup
	

	  \pstart \leavevmode% starting standard par
	\hphantom{.}स एव ‚{\color{DodgerBlue3}‚पार‚मार्थिको भावो योर्थ‚क्रियाक्ष‚मः ।‚{\tiny $_{5}$}‚ स च} विशेषोर्थ‚क्रियाक्ष‚मो ‚{\tiny $_{lb}$}‚‚{\color{DodgerBlue3}‚नान्वेति} प‚र‚स्प‚रं भेदात् (।) ‚{\color{DodgerBlue3}‚योऽन्वेति} प्र‚धानाख्यो भावो ‚{\color{DodgerBlue3}‚न त‚स्मात् कार्य‚स्य ‚{\tiny $_{lb}$}‚स‚म्भ‚वः} ।
	\pend% ending standard par
      

	  \pstart \leavevmode% starting standard par
	य‚द्येक‚रूपा\edtext{}{\edlabel{pvv.345-10}\label{pvv.345-10}\lemma{रूपा}\Bfootnote{प्र‚धान‚स्य ।}}न‚नुग‚म‚स्त‚दा विशेषेष्व‚पि केचिज्ज‚न‚य‚न्ति नाप‚र इति कुतोयं\edtext{}{\edlabel{pvv.345-11}\label{pvv.345-11}\lemma{कुतोयं}\Bfootnote{स‚र्व्व सामान्यं निर‚स्य पृथिव्यादेर्ज‚न‚क‚त्वे न कोप्य‚ज‚न‚क इत्याश‚यः ।}} ‚{\tiny $_{lb}$}‚विभाग इत्याह ।
	\pend% ending standard par
      
	  \bigskip
	  \begingroup
	
	    \large
	  
	    \begin{quote}
	  
	    
	    \stanza[\smallbreak]
	\label{pv.3.166b}\flagstanza{\tiny\textenglish{...3.166b}}तेनात्म‚नापि भेदे हि हेतुः क‚श्चिन्न चाप‚रः ॥ १६६ ॥\&[\smallbreak]


	
	    \end{quote}
	  
	  \endgroup
	
	  \bigskip
	  \begingroup
	
	    \large
	  
	    \begin{quote}
	  
	    
	    \stanza[\smallbreak]
	\label{pv.3.167a}\flagstanza{\tiny\textenglish{...3.167a}}स्व‚भावोय‚म‚भेदे तु स्यातां नाशोद्भ‚वौ स‚कृत् ।\&[\smallbreak]


	
	    \end{quote}
	  
	  \endgroup
	

	  \pstart \leavevmode% starting standard par
	येनैक\edtext{}{\edlabel{pvv.345-12}\label{pvv.345-12}\lemma{येनैक}\Bfootnote{एक‚स्य कार्य‚म‚न्य‚स्य न स्याद‚त्य‚न्त‚भेव‚त \href{http://sarit.indology.info/?cref=pv.3.163}{(३।१६३)} इत्य‚त्राह । ज्व‚रादिश‚म‚न \cref{pv.3.73} इत्याद्युक्तेप्य‚धिकार्थं ।}}रूपान्व‚ये स‚ह‚कारिवैक‚ल्य‚प्र‚स‚ङ्ग‚{\color{DodgerBlue3}‚स्तेन} कार‚णेना‚{\color{DodgerBlue3}‚त्म‚ना} स्व‚रूपेण बीज‚ज‚ल‚{\tiny $_{lb}$}‚शिलान‚लादीनां ‚{\color{DodgerBlue3}‚भेदेपि क‚श्चित्} क्षितिश(? स)लिल‚बीजादिर‚ङ्कुर‚स्य ‚{\color{DodgerBlue3}‚हेतुर्न ‚{\tiny $_{lb}$}‚चाप‚रः} शिलान‚लादिः । ‚{\color{DodgerBlue3}‚हि} य‚स्मात् ‚{\color{DodgerBlue3}‚स्व‚भावोय‚म}‚ङ्कुर‚ज‚न‚न‚श‚क्तः । ‚{\color{DodgerBlue3}‚त‚दित‚र‚श्च} प्र‚माण‚दृष्टो बीजादीनां शिलादीनाञ्च नात्र प‚र्य‚नुयोगाव‚तारः । कुत उत्प‚न्न इति ‚{\tiny $_{lb}$}‚चेत् । स्व‚स्व‚हेतोः । सोपि क‚स्मात् त‚ज्ज‚न‚क इति चेत् । स्व‚हेतोः (।) ‚{\color{DodgerBlue3}‚त‚त्रापि} \leavevmode\ledsidenote{\textenglish{346/s}} ‚{\color{DodgerBlue3}‚प्र‚श्ने त‚देवोत्त‚रं} । अनादिश्च हेतुफ‚ल‚प‚रंप‚रैक‚रूपानुग‚माद्(।) विशेषाणा‚{\color{DodgerBlue3}‚म‚भेदे\edtext{}{\edlabel{pvv.346-1}\label{pvv.346-1}\lemma{भेदे}\Bfootnote{भिन्नानां क‚श्चिद्धेतुर्नान्यः स्व‚भावादित्य‚त्र न किञ्चिद् बाध‚कं भ‚व‚तु बाधेत्याह ।}}तु} \leavevmode\ledsidenote{\textenglish{68b/MA}} स्वीक्रिय‚माणे ‚{\color{DodgerBlue3}‚स‚कृदे}‚क‚स्य ‚{\color{DodgerBlue3}‚नाशोद्भ‚{\tiny $_{7}$}‚वौ\edtext{}{\edlabel{pvv.346-2}\label{pvv.346-2}\lemma{वौ}\Bfootnote{उप‚ल‚क्ष‚ण‚मेत‚त् सूत्रे स‚र्व्व‚त्र स‚र्व्वोत्पादिर‚पि ज्ञेयः ।}} स्यातां} (।) विशेषान्त‚र‚योर्य‚थासंभ‚वं नाशो ‚{\tiny $_{lb}$}‚ज‚न्म‚नि च त‚द‚प‚र‚स्य विशेष‚स्य त‚द‚भिन्न‚रूप‚त‚या नाशोद्भ‚वौ स्याता\edtext{}{\edlabel{pvv.346-3}\label{pvv.346-3}\lemma{स्याता}\Bfootnote{व्य‚प‚देशोपि स‚र्व्वं हि व‚स्तु रूपेण भिद्य‚ते  --- न प‚र‚स्प‚रं । त‚न्म‚तं सामान्यं स‚र्व्व‚त्र स्थितं विशेषा न‚श्य‚न्ति ।}} (।)
	\pend% ending standard par
      
	  \bigskip
	  \begingroup
	
	    \large
	  
	    \begin{quote}
	  
	    
	    \stanza[\smallbreak]
	\label{pv.3.167b}\flagstanza{\tiny\textenglish{...3.167b}}भेदोपि तेन नैवं चेद् य एक‚स्मिन् विन‚श्य‚ति ॥ १६७ ॥\&[\smallbreak]


	
	    \end{quote}
	  
	  \endgroup
	
	  \bigskip
	  \begingroup
	
	    \large
	  
	    \begin{quote}
	  
	    
	    \stanza[\smallbreak]
	\label{pv.3.168a}\flagstanza{\tiny\textenglish{...3.168a}}तिष्ठ‚त्यात्मा न त‚स्यातो न स्यात् सामान्य‚भेद‚धीः ।\&[\smallbreak]


	
	    \end{quote}
	  
	  \endgroup
	

	  \pstart \leavevmode% starting standard par
	\hphantom{.}अथ प‚र‚स्प‚रं विशेषाणामेकान्तेन नाभेदः (।) किन्त‚र्हि ‚{\color{DodgerBlue3}‚भेदो} विशेष‚रूप‚त‚या । ‚{\tiny $_{lb}$}‚‚{\color{DodgerBlue3}‚तेन} भेदे‚{\color{DodgerBlue3}‚नैवं} स‚कृन्नाशोद्भ‚व‚प्र‚स‚ङ्गो ‚{\color{DodgerBlue3}‚न चेत्} । य‚द्येक‚{\color{DodgerBlue3}‚स्मिन्} विशेषे ‚{\color{DodgerBlue3}‚विन‚श्य‚ति} (।) ‚{\tiny $_{lb}$}‚‚{\color{DodgerBlue3}‚तिष्ठ‚त्यात्मा}‚नुयायी प्र‚धानाख्य‚स्त‚दा स ‚{\color{DodgerBlue3}‚त‚स्य} विशेष‚स्यात्मा न भ‚व‚ति (।) न हि ‚{\tiny $_{lb}$}‚य‚स्मिन् विन‚श्य‚ति यो ‚{\color{DodgerBlue3}‚न} न‚ष्टः स त‚स्य‚{\tiny $_{1}$}‚ स्व‚भावो विरुद्ध‚ध‚र्माध्यास‚ल‚क्ष‚ण‚त्वाद‚{\tiny $_{lb}$}‚भेद‚स्य । ‚{\color{DodgerBlue3}‚अतः} प‚र‚स्प‚रं स‚र्व्व‚था भेदा‚{\color{DodgerBlue3}‚न्न स्यात् सामान्य‚भेद‚धीः}\edtext{}{\edlabel{pvv.346-4}\label{pvv.346-4}\lemma{भेदा}\Bfootnote{प‚र‚स्प‚र‚स‚म्ब‚द्धा}} । अन्व‚यि सामान्य‚{\tiny $_{lb}$}‚म‚न‚न्व‚यी भेद उच्य‚ते (।) य‚दा च सामान्यं भेदाद् भिन्नं त‚दाऽनुगामिव्य‚क्तिस्व‚रूपं ‚{\tiny $_{lb}$}‚न सामान्यं (।) न च त‚दात्म‚को भेदः । अत‚स्त‚द्बुद्धिर‚पि न भ‚वेत् ।
	\pend% ending standard par
      

	  \pstart \leavevmode% starting standard par
	न‚न्व‚न्य‚निवृत्ताव‚पि स‚मान‚मेत‚त् । भावे न‚श्य‚ति अन्य‚निवृत्तिर्न‚श्य‚ति न वा । ‚{\tiny $_{lb}$}‚य‚दि न‚श्य‚ति न स्याद् व्य‚क्त्य‚न्त‚रे त‚द्बुद्धिः (।)‚{\tiny $_{2}$}‚ अथ न न‚श्य‚ति त‚दाऽत्य‚न्त‚भेदात ‚{\tiny $_{lb}$}‚सामान्य‚भेद‚बुद्धिर्न स्यादित्याह ।
	\pend% ending standard par
      
	  \bigskip
	  \begingroup
	
	    \large
	  
	    \begin{quote}
	  
	    
	    \stanza[\smallbreak]
	\label{pv.3.168b}\flagstanza{\tiny\textenglish{...3.168b}}निवृत्तेर्निःस्व‚भाव‚त्वात् न स्थानास्थान‚क‚ल्प‚ना ॥ १६८ ॥\&[\smallbreak]


	
	    \end{quote}
	  
	  \endgroup
	
	  \bigskip
	  \begingroup
	
	    \large
	  
	    \begin{quote}
	  
	    
	    \stanza[\smallbreak]
	\label{pv.3.169a}\flagstanza{\tiny\textenglish{...3.169a}}उप‚प्ल‚व‚श्च सामान्य‚धिय‚स्तेनाप्य‚दूष‚णा ।\&[\smallbreak]


	
	    \end{quote}
	  
	  \endgroup
	

	  \pstart \leavevmode% starting standard par
	\hphantom{.}‚{\color{DodgerBlue3}‚निवृत्ते\edtext{}{\edlabel{pvv.346-5}\label{pvv.346-5}\lemma{निवृत्ते}\Bfootnote{विजातिव्यावृत्त‚दाह्यं स्वाकाराभेदेनाध्य‚स्तं श‚ब्द‚विक‚ल्प‚विष‚य‚म‚पोहं मुक्त्वाऽन्य‚निवृत्तिमात्रे प्राह (दिग्)नागाभिम‚ते वा ।}}र्निःस्व‚भाव‚त्वात् न स्थानास्थान‚योः} स्थितिनिवृत्त्योः ‚{\color{DodgerBlue3}‚क‚ल्प‚ना} युक्ता । ‚{\tiny $_{lb}$}‚क‚ल्पिता ह्य‚न्य‚निवृत्तिः निःस्व‚भावा सा किं भावाद् भिन्नाऽभिन्ना वेति न युक्ता ‚{\tiny $_{lb}$}‚क‚ल्प‚ना । न हि श‚श‚विषाणं भिन्न‚म‚भिन्न‚म्वेति युक्तं क‚ल्पियितुं\edtext{}{\edlabel{pvv.346-6}\label{pvv.346-6}\lemma{ल्पियितुं}\Bfootnote{विक‚ल्प‚बुद्ध्यारोपित‚सामान्येपि विशेषे न‚श्य‚तीत्यादिनेत्याह यः सामान्याकारोऽनेक‚प‚दार्थाभिन्नः स भ्रान्तो ब‚हिर्वास्ति ।}} । उप‚प्ल‚वो ‚{\tiny $_{lb}$}‚\leavevmode\ledsidenote{\textenglish{347/s}} मिथ्यात्व‚ञ्च सामान्य‚धियो विष‚याभावात् (।) तेन\edtext{}{\edlabel{pvv.347-1}\label{pvv.347-1}\lemma{तेन}\Bfootnote{न प्ल‚व‚त्वेन हेतुना नास्या युक्त‚दूष‚णं सामान्य‚बुद्धौ ।}} नाप्य‚दूष‚णा ॥
	\pend% ending standard par
      
	  \bigskip
	  \begingroup
	
	    \large
	  
	    \begin{quote}
	  
	    
	    \stanza[\smallbreak]
	\label{pv.3.169b}\flagstanza{\tiny\textenglish{...3.169b}}य‚त्त‚स्य ज‚न‚कं रूपं त‚तोन्यो ज‚न‚कः क‚थ‚म् ॥ १६९ ॥\&[\smallbreak]


	
	    \end{quote}
	  
	  \endgroup
	
	  \bigskip
	  \begingroup
	
	    \large
	  
	    \begin{quote}
	  
	    
	    \stanza[\smallbreak]
	\label{pv.3.170a}\flagstanza{\tiny\textenglish{...3.170a}}भिन्ना विशेषा ज‚न‚काः ;\&[\smallbreak]


	
	    \end{quote}
	  
	  \endgroup
	

	  \pstart \leavevmode% starting standard par
	\edtext{\textsuperscript{*}}{\edlabel{pvv.347-2}\label{pvv.347-2}\lemma{*}\Bfootnote{तुल्य‚दोष‚त्व‚म‚प‚नीय प्र‚कारान्त‚रेण प्र‚क्रान्त‚चोद्याप‚न‚या(या)ह ।}}न‚नूक्तं य‚दि भावाः‚{\tiny $_{3}$}‚ स‚र्व्व‚था भिन्नास्त‚दा ‚{\color{DodgerBlue3}‚य‚त्त‚स्य} बीज‚स्य ‚{\color{DodgerBlue3}‚ज‚न‚कं रूपं न} त‚द‚न्य‚स्य क्षित्यादेरिति ‚{\color{DodgerBlue3}‚त‚तोन्यो ज‚न‚कः क‚थ}‚मिति । अत्रोत्त‚र‚म‚प्यु\edtext{}{\edlabel{pvv.347-3}\label{pvv.347-3}\lemma{प्यु}\Bfootnote{एकापाये फ‚लाभावाद् विशेषेभ्य‚स्त‚दुद्भ‚व \href{http://sarit.indology.info/?cref=pv.3.165}{(३।१६५)} इति प्र‚साध‚नं प्रागुक्तं स्मार‚य‚ति ।}}क्त‚म‚भिन्न‚रूपा‚{\tiny $_{lb}$}‚न्व‚य‚व्य‚तिरेकानुविधानाभावाद् भिन्ना\edtext{}{\edlabel{pvv.347-4}\label{pvv.347-4}\lemma{भिन्ना}\Bfootnote{नात्र विरोधोस्ति ।}} ‚{\color{DodgerBlue3}‚विशेषा} एव ‚{\color{DodgerBlue3}‚ज‚न‚का}‚स्त‚द‚न्व‚य‚व्य‚तिरेकानु‚{\tiny $_{lb}$}‚विधानात् कार्य‚स्य स‚र्व्व एव ते त‚त्कार्य‚स्योत्पाद‚क‚त‚योत्प‚न्नास्त‚दुत्पाद‚य‚न्ति ।
	\pend% ending standard par
      
	  \bigskip
	  \begingroup
	
	    \large
	  
	    \begin{quote}
	  
	    
	    \stanza[\smallbreak]
	\label{pv.3.170b}\flagstanza{\tiny\textenglish{...3.170b}}अप्य‚भेदोपि तेषु चेत् ।&तेन तेऽज‚न‚काः प्रोक्ताः ;\&[\smallbreak]


	
	    \end{quote}
	  
	  \endgroup
	

	  \pstart \leavevmode% starting standard par
	तेषु विशेषेष्व‚न्व‚यिना रूपेणाभेदोप्य‚स्ति तेनैव ते ज‚न‚का इति चेत्‚{\tiny $_{4}$}‚(।) ‚{\tiny $_{lb}$}‚तेनान्व‚यिरूपेण ते विशेषा अज‚न‚काः प्रोक्ताः\edtext{}{\edlabel{pvv.347-5}\label{pvv.347-5}\lemma{प्रोक्ताः}\Bfootnote{सामान्यान्व‚य‚व्य‚तिरेकान‚नुविधानादिति नापैत्य‚भिन्न‚मित्यादिना ।}} एक‚त्रापि विशेषे त‚द्भावात् स‚ह‚कारि‚{\tiny $_{lb}$}‚वैफ‚ल्य‚प्र‚स‚क्तेः । त‚द‚न्व‚य‚व्य‚तिरेकान‚र्थ‚विधानाद् विशेष‚स्य विप‚र्य‚याच्चेति\edtext{}{\edlabel{pvv.347-6}\label{pvv.347-6}\lemma{याच्चेति}\Bfootnote{स्यातां नाशोद्भ‚वौ स‚कृदित्यादिना \href{http://sarit.indology.info/?cref=pv.3.167}{(३।१६७)} भेदं प्र‚साध्य प्र‚तिभास‚भेदेनापि भेद‚माह ।}} ।
	\pend% ending standard par
      

	  \pstart \leavevmode% starting standard par
	किञ्च (।)
	\pend% ending standard par
      
	  \bigskip
	  \begingroup
	
	    \large
	  
	    \begin{quote}
	  
	    
	    \stanza[\smallbreak]
	\label{pv.3.170c}\flagstanza{\tiny\textenglish{...3.170c}}प्र‚तिभासोपि भेद‚कः ॥ १७० ॥\&[\smallbreak]


	
	    \end{quote}
	  
	  \endgroup
	
	  \bigskip
	  \begingroup
	
	    \large
	  
	    \begin{quote}
	  
	    
	    \stanza[\smallbreak]
	\label{pv.3.171a}\flagstanza{\tiny\textenglish{...3.171a}}अन‚न्य‚भाक् स एवार्थ‚स्त‚स्य व्यावृत्त‚योप‚रे ।\&[\smallbreak]


	
	    \end{quote}
	  
	  \endgroup
	

	  \pstart \leavevmode% starting standard par
	\hphantom{.}‚{\color{DodgerBlue3}‚प्र‚तिभासोप्य‚न‚न्य‚भाग्\edtext{}{\edlabel{pvv.347-7}\label{pvv.347-7}\lemma{भाग्}\Bfootnote{प्र‚तिव्य‚क्तिभिन्नः ।}} भेद‚को} विशेषाणाम‚संकीर्ण्ण‚रूप‚व्य‚व‚स्थाप‚नात् । अपि ‚{\tiny $_{lb}$}‚श‚ब्दादुत्प‚त्तिस्थितिविनाशाद‚य‚श्च स‚मानाः । त‚त‚श्च विशेष एवार्थः पार‚मार्थिकः । ‚{\tiny $_{lb}$}‚ये ‚{\color{DodgerBlue3}‚त्व‚प‚रे} सामान्याद‚यो‚{\tiny $_{5}$}‚ ध‚र्मास्ते ‚{\color{DodgerBlue3}‚त‚स्य} शेष‚स्य विजातीयाद् ‚{\color{DodgerBlue3}‚व्यावृत्त‚यः} क‚ल्पिता ‚{\tiny $_{lb}$}‚अन‚र्थ‚क्रियाकारिणः ।
	\pend% ending standard par
      \textsuperscript{\textenglish{348/s}}
	  \bigskip
	  \begingroup
	
	    \large
	  
	    \begin{quote}
	  
	    
	    \stanza[\smallbreak]
	\label{pv.3.171b}\flagstanza{\tiny\textenglish{...3.171b}}त‚त् कार्यं कार‚ण‚ञ्चोक्तं त‚त्स्व‚ल‚क्ष‚ण‚मिष्य‚ते ॥ १७१ ॥\&[\smallbreak]


	
	    \end{quote}
	  
	  \endgroup
	
	  \bigskip
	  \begingroup
	
	    \large
	  
	    \begin{quote}
	  
	    
	    \stanza[\smallbreak]
	\label{pv.3.172a}\flagstanza{\tiny\textenglish{...3.172a}}त‚त्त्यागाप्तिफ‚लाः स‚र्वाः पुरुषाणां प्र‚वृत्त‚यः ।\&[\smallbreak]


	
	    \end{quote}
	  
	  \endgroup
	

	  \pstart \leavevmode% starting standard par
	अर्थ‚क्रियाकारि तु\edtext{}{\edlabel{pvv.348-1}\label{pvv.348-1}\lemma{तु}\Bfootnote{विशेष‚रूपं ।}} य‚द् रूपं त‚त्कार्य ‚{\color{DodgerBlue3}‚कार‚ण‚ञ्चोक्तं त‚त्स्व‚ल‚क्ष‚ण‚मिष्य‚ते (।) ‚{\tiny $_{lb}$}‚त‚स्य त्यागाप्तिस्त‚त्फ‚लाः पुरुषाणाम}‚र्थान‚र्थ‚प्राप्तिप‚रिहारैषिणां ‚{\color{DodgerBlue3}‚प्र‚वृत्त‚यः स‚र्व्वाः} ।
	\pend% ending standard par
      

	  \pstart \leavevmode% starting standard par
	किञ्च\edtext{}{\edlabel{pvv.348-2}\label{pvv.348-2}\lemma{किञ्च}\Bfootnote{एवं मी मां स का दि म‚तेन प्रातिभासिकं सामान्यं निर‚स्यानुमानिकं पूर्व्वोक्तं प्र‚त्याह ।}} (।)
	\pend% ending standard par
      
	  \bigskip
	  \begingroup
	
	    \large
	  
	    \begin{quote}
	  
	    
	    \stanza[\smallbreak]
	\label{pv.3.172b}\flagstanza{\tiny\textenglish{...3.172b}}य‚थाऽभेदाविशेषेपि न स‚र्वं स‚र्व्व‚साध‚न‚म् ॥ १७२ ॥\&[\smallbreak]


	
	    \end{quote}
	  
	  \endgroup
	
	  \bigskip
	  \begingroup
	
	    \large
	  
	    \begin{quote}
	  
	    
	    \stanza[\smallbreak]
	\label{pv.3.173a}\flagstanza{\tiny\textenglish{...3.173a}}त‚थाऽभेदाविशेषेपि न स‚र्व्वं स‚र्व्व‚साध‚न‚म् ।\&[\smallbreak]


	
	    \end{quote}
	  
	  \endgroup
	

	  \pstart \leavevmode% starting standard par
	\hphantom{.}सां ख्य स्यापि म‚ते ‚{\color{DodgerBlue3}‚य‚थाऽभेद}‚स्य प्र‚धानात्म‚त‚या स‚र्व्व‚भावेष्व‚{\color{DodgerBlue3}‚विशेषेपि न स‚र्व्व} व्य‚क्तं ‚{\color{DodgerBlue3}‚स‚र्व्व}‚स्य कार्य‚स्य ‚{\color{DodgerBlue3}‚साध‚नं} हेतुः । ‚{\color{DodgerBlue3}‚त‚था भेदाविशे‚{\tiny $_{6}$}‚षेपि\edtext{}{\edlabel{pvv.348-3}\label{pvv.348-3}\lemma{षेपि}\Bfootnote{बौद्ध‚स्य ।}} न स‚र्व्व} क्षितिबीजा‚{\tiny $_{lb}$}‚न‚ल‚शिलादिकं ‚{\color{DodgerBlue3}‚स‚र्व्व}‚स्य तापाङ्कुरादेः ‚{\color{DodgerBlue3}‚साध‚नं} ।
	\pend% ending standard par
      
	  \bigskip
	  \begingroup
	
	    \large
	  
	    \begin{quote}
	  
	    
	    \stanza[\smallbreak]
	\label{pv.3.173b}\flagstanza{\tiny\textenglish{...3.173b}}भेदे हि कार‚कं किञ्चिद् व‚स्तुध‚र्म‚त‚या भ‚वेत् ॥ १७३ ॥\&[\smallbreak]


	
	    \end{quote}
	  
	  \endgroup
	

	  \pstart \leavevmode% starting standard par
	\hphantom{.}विशेषान्त‚राद् ‚{\color{DodgerBlue3}‚भेदे हि} स‚ति ‚{\color{DodgerBlue3}‚व‚स्तुध‚र्म‚त‚या}\edtext{}{\edlabel{pvv.348-4}\label{pvv.348-4}\lemma{ति}\Bfootnote{अभेदे तु त‚स्य स‚र्व्व‚त्राभिन्न‚त्वात् क्रियाक्रिये विरुद्धे ।}} व‚स्तुस्व‚भाव‚त्वात् ‚{\color{DodgerBlue3}‚किञ्चिद्} व‚स्तु ‚{\tiny $_{lb}$}‚‚{\color{DodgerBlue3}‚कार‚कं भ‚वेत्} । न स‚र्व्व‚मिति युक्तं स्व‚हेतुब‚लायात‚त्वाद् भिन्न‚श‚क्तिक‚त्व‚स्य ।(१७३)
	\pend% ending standard par
      \label{div_pvv.3.174}
	  
	% new div opening: depth here is 2
	
	  \bigskip
	  \begingroup
	
	    \large
	  
	    \begin{quote}
	  
	    
	    \stanza[\smallbreak]
	\label{pv.3.174a}\flagstanza{\tiny\textenglish{...3.174a}}अभेदे तु विरुध्येते त‚स्यैक‚स्य क्रियाक्रिये ।\&[\smallbreak]


	
	    \end{quote}
	  
	  \endgroup
	

	  \pstart \leavevmode% starting standard par
	\hphantom{.}‚{\color{DodgerBlue3}‚अभेदे तु} स‚र्व्व‚भ‚वानां ‚{\color{DodgerBlue3}‚विरुध्येते त‚स्यै\edtext{}{\edlabel{pvv.348-5}\label{pvv.348-5}\lemma{स्यै}\Bfootnote{त्रैगुण्य‚स्य बीज‚द‚ह‚नाद्यात्म‚क‚स्य ।}}क‚स्या}‚ङ्कुरादेः ‚{\color{DodgerBlue3}‚क्रियाक्रिये} बीज‚रू‚{\tiny $_{lb}$}‚प‚त‚या प्र‚धान‚म‚ङ्कुर‚कार‚क‚म‚कार‚क‚ञ्च द‚ह‚न‚रूप‚त‚या । विप्र‚तिषिद्ध‚ञ्चैत‚त्‚{\tiny $_{7}$}‚ ‚{\tiny $_{lb}$}‚\leavevmode\ledsidenote{\textenglish{69a/MA}} क‚थ‚मेक‚स्य युक्तं ॥
	\pend% ending standard par
      
	  \bigskip
	  \begingroup
	
	    \large
	  
	    \begin{quote}
	  
	    
	    \stanza[\smallbreak]
	\label{pv.3.174b}\flagstanza{\tiny\textenglish{...3.174b}}भेदोप्य‚स्त्य‚क्रियात‚श्चेत् न कुर्युः स‚ह‚कारिणः ॥ १७४ ॥\&[\smallbreak]


	
	    \end{quote}
	  
	  \endgroup
	

	  \pstart \leavevmode% starting standard par
	\hphantom{.}व्य‚क्तीनां मिथो व्य‚क्तिरूप‚त‚या ‚{\color{DodgerBlue3}‚भेदोप्य‚स्ति} । व्य‚क्त्य‚न्त‚र‚साध्य‚स्य कार्य‚स्या‚{\tiny $_{lb}$}‚‚{\color{DodgerBlue3}‚क्रियात‚श्चेत्} । त‚दा भेदात् ‚{\color{DodgerBlue3}‚स‚ह‚कारिणः} कार्यं ‚{\color{DodgerBlue3}‚न कुर्युः} । एक‚स्य यः कार‚कः स्व‚भाव‚{\tiny $_{lb}$}‚स्त‚स्यान्य‚त्राभावात् । (१७४)
	\pend% ending standard par
      \label{div_pvv.3.175}
	  
	% new div opening: depth here is 2
	
	  \bigskip
	  \begingroup
	
	    \large
	  
	    \begin{quote}
	  
	    
	    \stanza[\smallbreak]
	\label{pv.3.175a}\flagstanza{\tiny\textenglish{...3.175a}}प‚र्यायेणाथ क‚र्त्तृत्वं स किं त‚स्यैव व‚स्तुनः ।\&[\smallbreak]


	
	    \end{quote}
	  
	  \endgroup
	\textsuperscript{\textenglish{349/s}}

	  \pstart \leavevmode% starting standard par
	अथ\edtext{}{\edlabel{pvv.349-1}\label{pvv.349-1}\lemma{अथ}\Bfootnote{य‚व‚बीज‚ञ्चे (त्) शाल्य‚ङ्कुर‚कार‚णं प‚र्यायेण य‚दा त‚च्छालिबीज‚त्वेन प‚रिण‚मेत् । प्र‚धान‚श‚क्त्य‚धिष्ठित‚भेद‚प‚रिणामे प्र‚धान‚श‚क्तेर्भेद‚त्वेन प‚रिणामे वा स‚र्व्व‚त्र स‚र्व्वोप‚योगात् ।}} कार‚कैक‚स्व‚भावानुग‚मात् स‚ह‚कारिणां ‚{\color{DodgerBlue3}‚प‚र्यायेण क‚र्त्तृत्वं} । स प‚र्याय‚{\tiny $_{lb}$}‚स्त‚स्यान्व‚यिन एक‚स्य \edtext{}{\edlabel{pvv.349-2}\label{pvv.349-2}\lemma{स्य}\Bfootnote{शालिबीज‚स्य य‚व‚बीज‚त्वेन प‚रिणामो न युक्त इत्य‚र्थः भेदाधिष्ठान‚त्वात् प‚र्याय‚स्य ।}} ‚{\color{DodgerBlue3}‚व‚स्तुनः किं} (क‚स्मात्) युक्तः । न ह्येकं व‚स्तु कार्य कुर्व्व‚त ‚{\tiny $_{lb}$}‚प‚र्यायेण न क‚रोतीति युक्तं व्य‚प‚देष्टुं तेनै‚{\tiny $_{1}$}‚व क्रिय‚माण‚त्वात् । स‚ह‚कारिणां ब‚हूनां ‚{\tiny $_{lb}$}‚प‚र्यायेण क्रिया स‚म्भाव्य‚ते । तेषु चैक‚स्य कार‚कं रूप‚म‚न्य‚त्र नास्तीति न ते कुर्युः । ‚{\tiny $_{lb}$}‚स‚र्व्वेषाञ्च स्व‚हेतुव‚लायात एक‚कार्य‚कार‚कः स्व‚भावोऽस्माभिरिव सां ख्यै र्नेष्य‚ते॥
	\pend% ending standard par
      

	  \pstart \leavevmode% starting standard par
	किञ्च\edtext{}{\edlabel{pvv.349-3}\label{pvv.349-3}\lemma{किञ्च}\Bfootnote{प‚रिणामं निर‚स्याभिन्नं भिन्नं भिन्नाभिन्नं स‚र्व्व‚सु चोत्त‚रोत्त‚राव‚स्यास्व‚नुयायित्वादूर्ध्व‚वृत्तिर्वा स‚मं स‚र्व्वासु व्य‚क्तिष्व‚नुयायित्वात् तिर्य‚ग्वृत्ति वा सामान्यं सांख्य‚मीमांस‚क‚नैयायिकादेर्दू ष‚यितुमाह किञ्चेति ।}} ।
	\pend% ending standard par
      
	  \bigskip
	  \begingroup
	
	    \large
	  
	    \begin{quote}
	  
	    
	    \stanza[\smallbreak]
	\label{pv.3.175b}\flagstanza{\tiny\textenglish{...3.175b}}अत्य‚न्त‚भेदाभेदौ तु स्यातां त‚द्व‚ति व‚स्तुनि ॥ १७५ ॥\&[\smallbreak]


	
	    \end{quote}
	  
	  \endgroup
	

	  \pstart \leavevmode% starting standard par
	\hphantom{.}‚{\color{DodgerBlue3}‚त‚द्व‚ति} सामान्य‚विशेष‚व‚ति ‚{\color{DodgerBlue3}‚व‚स्तुनि} स्वीक्रिय‚माणे‚{\color{DodgerBlue3}‚ऽत्य‚न्त‚मे}‚कान्तेन सामान्य‚{\tiny $_{lb}$}‚विशेष‚योर्भेदाभेदो ‚{\color{DodgerBlue3}‚स्यातां} । सामान्य‚स्व‚रूप‚त्वाद् भेद‚स्य सामान्य‚मेव‚{\tiny $_{2}$}‚ भ‚वेन्न ‚{\tiny $_{lb}$}‚भेदो भेदात्म‚त्वात् सामान्य‚स्य भेद एव भ‚वेत् सामान्यं ।\edtext{\textsuperscript{*}}{\edlabel{pvv.349-4}\label{pvv.349-4}\lemma{*}\Bfootnote{निर‚स्त‚म‚भिन्नं ।}}(१७५)
	\pend% ending standard par
      \label{div_pvv.3.176_3.177_3.178_3.179_3.180_3.181_3.182_3.183_3.184ab}
	  
	% new div opening: depth here is 2
	

	  \pstart \leavevmode% starting standard par
	अथ त‚योः क‚थ‚ञ्च‚न\edtext{}{\edlabel{pvv.349-5}\label{pvv.349-5}\lemma{न}\Bfootnote{स्व‚भावाभेदेपि ल‚क्ष‚ण‚भेदात् अव‚स्थायाः ।}} भेदोप्य‚स्तीति चेदाह (।)
	\pend% ending standard par
      
	  \bigskip
	  \begingroup
	
	    \large
	  
	    \begin{quote}
	  
	    
	    \stanza[\smallbreak]
	\label{pv.3.176a}\flagstanza{\tiny\textenglish{...3.176a}}अन्योन्यं वा त‚योर्भेदः स‚दृशास‚दृशात्म‚नोः ।\&[\smallbreak]


	
	    \end{quote}
	  
	  \endgroup
	

	  \pstart \leavevmode% starting standard par
	\hphantom{.}‚{\color{DodgerBlue3}‚त‚योः} सामान्य‚विशेष‚योः ‚{\color{DodgerBlue3}‚स‚दृशास‚दृशात्म‚नोः \edtext{}{\edlabel{pvv.349-6}\label{pvv.349-6}\lemma{नोः}\Bfootnote{अनुग‚त‚व्यावृत्त‚योर‚श्लेषात् ।}} साधार}‚णासाधार‚ण‚स्व‚रूप‚यो‚{\tiny $_{lb}$}‚र‚न्योन्यं ‚{\color{DodgerBlue3}‚भेद} एव ‚{\color{DodgerBlue3}‚वा} भ‚वेत् । न क‚थ‚ञ्चिदेक‚त्वं ।
	\pend% ending standard par
      

	  \pstart \leavevmode% starting standard par
	एत‚देव स्फुट‚यितुं पूर्व्व‚प‚क्ष‚य‚ति (।)
	\pend% ending standard par
      
	  \bigskip
	  \begingroup
	
	    \large
	  
	    \begin{quote}
	  
	    
	    \stanza[\smallbreak]
	\label{pv.3.176b}\flagstanza{\tiny\textenglish{...3.176b}}त‚योर‚पि भ‚वेद् भेदो य‚दि येनात्म‚ना त‚योः ॥ १७६ ॥\&[\smallbreak]


	
	    \end{quote}
	  
	  \endgroup
	
	  \bigskip
	  \begingroup
	
	    \large
	  
	    \begin{quote}
	  
	    
	    \stanza[\smallbreak]
	\label{pv.3.177a}\flagstanza{\tiny\textenglish{...3.177a}}भेदः सामान्य‚मित्येत‚द् य‚दि भेद‚स्त‚दात्म‚ना ।&भेद एव;\&[\smallbreak]


	
	    \end{quote}
	  
	  \endgroup
	\textsuperscript{\textenglish{350/s}}

	  \pstart \leavevmode% starting standard par
	न स‚र्व्वात्म‚नाऽभेद\edtext{}{\edlabel{pvv.350-1}\label{pvv.350-1}\lemma{नाऽभेद}\Bfootnote{सामान्यं विशेष इति भेद‚स्थाप‚नात् ।}} ‚{\color{DodgerBlue3}‚स्त‚योः} सामान्य‚विशेषो‚{\color{DodgerBlue3}‚र‚पि} क‚थ‚ञ्चिद् ‚{\color{DodgerBlue3}‚भेदो ‚{\tiny $_{3}$}‚ भ‚वेद् ‚{\tiny $_{lb}$}‚य‚दि} (।) अन्य‚था सामान्य‚विशेष‚भावानुप‚प‚त्तिः । अत्राह । ‚{\color{DodgerBlue3}‚येनात्म‚ना} स्व‚रूपेण ‚{\tiny $_{lb}$}‚साधार‚णेन चासाधार‚णेन च ‚{\color{DodgerBlue3}‚त‚योः} सामान्य‚विशेष‚यो‚{\color{DodgerBlue3}‚र्भेदः सामान्यं} विशेष ‚{\color{DodgerBlue3}‚इत्येत‚द्} व्य‚व‚स्थाप्य‚ते । तेनात्म‚ना साधार‚णासाधार‚णेन ‚{\color{DodgerBlue3}‚य‚दि भेद‚स्त‚दा} भेद एव ‚{\color{DodgerBlue3}‚त‚योः} स्यात् ‚{\tiny $_{lb}$}‚भिन्न‚ल‚क्ष‚ण‚त्वात् ।
	\pend% ending standard par
      
	  \bigskip
	  \begingroup
	
	    \large
	  
	    \begin{quote}
	  
	    
	    \stanza[\smallbreak]
	\label{pv.3.177b}\flagstanza{\tiny\textenglish{...3.177b}}त‚था च स्यान्निःसामान्य‚विशेष‚ता ॥ १७७ ॥\&[\smallbreak]


	
	    \end{quote}
	  
	  \endgroup
	
	  \bigskip
	  \begingroup
	
	    \large
	  
	    \begin{quote}
	  
	    
	    \stanza[\smallbreak]
	\label{pv.3.178a}\flagstanza{\tiny\textenglish{...3.178a}}भेद‚सामान्य‚योर्य‚द्व‚द् घ‚टादीनां प‚र‚स्प‚र‚म् ।\&[\smallbreak]


	
	    \end{quote}
	  
	  \endgroup
	

	  \pstart \leavevmode% starting standard par
	\hphantom{.}‚{\color{DodgerBlue3}‚त‚था च निःसामान्य‚विशेष‚ता} स‚मान्य‚विशेष‚रूप‚ताऽभावः सामान्य‚वि‚{\tiny $_{4}$}‚‚{\tiny $_{lb}$}‚शेष‚यो\edtext{}{\edlabel{pvv.350-2}\label{pvv.350-2}\lemma{यो}\Bfootnote{भेदो निःसामान्यः सामान्यं निर्व्विशेषः ।}}र‚भित‚योः ‚{\color{DodgerBlue3}‚स्यात् । य‚द्व‚द् घ‚टादीनां} विशेषाणां ‚{\color{DodgerBlue3}‚प‚र‚स्प‚रं} न सामान्य‚{\tiny $_{lb}$}‚विशेष‚ता ।\edtext{\textsuperscript{*}}{\edlabel{pvv.350-3}\label{pvv.350-3}\lemma{*}\Bfootnote{अज‚न्य‚ज‚न‚क‚त्वेन स‚म्ब‚न्धाभावात् ।}} विशेषः स्व(ल‚क्ष‚ण)रूप‚म‚नुगामि सामान्यं त‚देवानुगामि विशेष ‚{\tiny $_{lb}$}‚उच्य‚ते (।) य‚दि तु त‚योर्भेद एव न सामान्य‚विशेष‚भावः स्यात् ।\edtext{\textsuperscript{*}}{\edlabel{pvv.350-4}\label{pvv.350-4}\lemma{*}\Bfootnote{दिग‚म्ब‚र‚स्योर्ध्व‚तासामान्यं सांख्य‚स्य तिर्य‚क् एक‚दोषेण (देशिनं) निर‚स्य तिर्य‚क्सामान्यं पुन‚राह ।}}
	\pend% ending standard par
      
	  \bigskip
	  \begingroup
	
	    \large
	  
	    \begin{quote}
	  
	    
	    \stanza[\smallbreak]
	\label{pv.3.178b}\flagstanza{\tiny\textenglish{...3.178b}}य‚मात्मानं पुर‚स्कृत्य पुरुषोयं प्र‚व‚र्त‚ते ॥ १७८ ॥\&[\smallbreak]


	
	    \end{quote}
	  
	  \endgroup
	
	  \bigskip
	  \begingroup
	
	    \large
	  
	    \begin{quote}
	  
	    
	    \stanza[\smallbreak]
	\label{pv.3.179a}\flagstanza{\tiny\textenglish{...3.179a}}त‚त्साध्य‚फ‚ल‚वाञ्छावान् भेदाभेदौ त‚दाश्र‚यौ ।&चिन्त्य‚ते स्वात्म‚ना भेदः ;\&[\smallbreak]


	
	    \end{quote}
	  
	  \endgroup
	

	  \pstart \leavevmode% starting standard par
	\hphantom{.}अपि चायं व्य‚व‚हारी ‚{\color{DodgerBlue3}‚पुरुषो य‚म}‚र्थ‚स्या‚{\color{DodgerBlue3}‚त्मानं} स्व‚भावं ‚{\color{DodgerBlue3}‚पुर‚स्कृत्य} प्र‚वृत्तिविष‚य‚{\tiny $_{lb}$}‚त्वेनाग्र‚हं कृत्वा ‚{\color{DodgerBlue3}‚त‚त्साध्य‚फ‚ल‚वाञ्छावान्} अर्थ‚साध्य‚फ‚ल‚स‚मीहायु‚{\tiny $_{5}$}‚क्तः स‚न ‚{\tiny $_{lb}$}‚\edtext{}{\edlabel{pvv.350-5}\label{pvv.350-5}\lemma{न}\Bfootnote{अर्थार्थिनः सामान्य‚स्य भेदाभेद‚चिन्त‚या न कि(ञ्चि)द‚न‚र्थ‚त्वात् ।}} ‚{\color{DodgerBlue3}‚प्र‚व‚र्त्त‚ते (।) त‚दाश्र‚यौ} त‚द‚र्थ‚विष‚यौ ‚{\color{DodgerBlue3}‚भेदाभेदौ} शास्त्र‚कारैश्चिन्त्येते । न त्व‚र्थ‚क्रिया‚{\tiny $_{lb}$}‚नुप‚युक्त‚सामान्य‚विष‚यौ (।) तेषाञ्चार्थ‚क्रियाकारिणां पुरुष‚प्र‚वृत्तिविष‚याणाम‚र्थानां ‚{\tiny $_{lb}$}‚‚{\color{DodgerBlue3}‚स्वात्म‚ना} स्व‚रूपेण ‚{\color{DodgerBlue3}‚भेदः} ।\edtext{\textsuperscript{*}}{\edlabel{pvv.350-6}\label{pvv.350-6}\lemma{*}\Bfootnote{आत्य‚न्तिकोस्त्येव ।}}
	\pend% ending standard par
      

	  \pstart \leavevmode% starting standard par
	क‚थं त‚र्ह्येक‚बुद्धिश‚ब्द‚विष‚य‚तेत्याह ।
	\pend% ending standard par
      
	  \bigskip
	  \begingroup
	
	    \large
	  
	    \begin{quote}
	  
	    
	    \stanza[\smallbreak]
	\label{pv.3.179b}\flagstanza{\tiny\textenglish{...3.179b}}व्यावृत्त्या च स‚मान‚ता ॥ १७९ ॥\&[\smallbreak]


	
	    \end{quote}
	  
	  \endgroup
	
	  \bigskip
	  \begingroup
	
	    \large
	  
	    \begin{quote}
	  
	    
	    \stanza[\smallbreak]
	\label{pv.3.180a}\flagstanza{\tiny\textenglish{...3.180a}}अस्त्येव व‚स्तु नान्वेति प्र‚वृत्त्यादिप्र‚स‚ङ्ग‚तः ।\&[\smallbreak]


	
	    \end{quote}
	  
	  \endgroup
	\textsuperscript{\textenglish{351/s}}

	  \pstart \leavevmode% starting standard par
	\hphantom{.}‚{\color{DodgerBlue3}‚विजातीयाद् व्यावृत्या च सा स‚मान‚ताऽस्त्येव} । स्व‚स्व‚भाव‚निय‚त‚न्तु स्व‚ल‚क्ष‚णं ‚{\tiny $_{lb}$}‚‚{\color{DodgerBlue3}‚व‚स्तु नान्वेति} स‚र्व्व‚त्र ‚{\color{DodgerBlue3}‚प्र‚वृत्त्यादिप्र‚स‚ङ्ग‚तः} । अग्निर‚पि‚{\tiny $_{6}$}‚ ‚{\color{DodgerBlue3}‚श} (=स) लिल\edtext{}{\edlabel{pvv.351-1}\label{pvv.351-1}\lemma{लिल}\Bfootnote{अथ स्व‚ल‚क्ष‚ण‚म‚न्वेष्य‚तीति किं क‚ल्पित‚व्यावृत्त्यापीत्य‚पि न प‚र‚स्प‚रं भेदात् । ‚{\tiny $_{lb}$}‚य‚दि घ‚टः प‚टे स्यादुद‚कार्थी प‚टेपि प्र‚व‚र्त्तेत न चास्तीति न प्र‚वृत्त्यादिना तुल्योत्प‚त्ति‚{\tiny $_{lb}$}‚निरोधादि ।}}स्व‚{\tiny $_{lb}$}‚भाव एवेति ‚{\color{DodgerBlue3}‚श} (=स) लिलार्थी त‚त्रापि प्र‚व‚र्त्तेत । त‚स्मात् स्थित‚मेत‚त् (।) न ‚{\tiny $_{lb}$}‚किञ्चित् किम‚प्य‚न्वेतीति\edtext{}{\edlabel{pvv.351-2}\label{pvv.351-2}\lemma{न्वेतीति}\Bfootnote{सामान्य‚विशेष‚योर्व‚स्तुत एक‚त्वात् कृत‚कानित्य‚त्वानि वा ‚{\tiny $_{lb}$}‚व्य‚भिचारो व्यावृत्त‚स्व‚ल‚क्ष‚ण‚स्यैव सामान्य‚त्वात् बौद्धे ।}} ।
	\pend% ending standard par
      

	  \begin{center}%% label @type='head'
	\textbf{ग. जैन‚म‚त‚निरासः}
	\end{center}
	
	  \bigskip
	  \begingroup
	
	    \large
	  
	    \begin{quote}
	  
	    
	    \stanza[\smallbreak]
	\label{pv.3.180b}\flagstanza{\tiny\textenglish{...3.180b}}एतेनैव य‚दाह्रीकाः किम‚प्य‚युक्त‚माकुल‚म् ॥ १८० ॥\&[\smallbreak]


	
	    \end{quote}
	  
	  \endgroup
	
	  \bigskip
	  \begingroup
	
	    \large
	  
	    \begin{quote}
	  
	    
	    \stanza[\smallbreak]
	\label{pv.3.181a}\flagstanza{\tiny\textenglish{...3.181a}}प्र‚ल‚प‚न्ति प्र‚तिक्षिप्तं त‚द‚प्येकान्त‚स‚म्भ‚वात् ।\&[\smallbreak]


	
	    \end{quote}
	  
	  \endgroup
	

	  \pstart \leavevmode% starting standard par
	\hphantom{.}‚{\color{DodgerBlue3}‚एतेन सां ख्य} म‚त‚निराक‚र‚णे‚{\color{DodgerBlue3}‚नैवाह्रीका} दि ग म्ब रा\edtext{}{\edlabel{pvv.351-3}\label{pvv.351-3}\lemma{रा}\Bfootnote{स‚र्वः स‚र्व‚स्व‚भावो न च स‚र्व‚स्व‚भावः ।}} ‚{\color{DodgerBlue3}‚य‚त्} स्यादुष्ट्रो द‚धि ‚{\tiny $_{lb}$}‚व‚स्तुत्वात् । न वा स्यादुष्ट्रो\edtext{}{\edlabel{pvv.351-4}\label{pvv.351-4}\lemma{स्यादुष्ट्रो}\Bfootnote{द‚ध्य‚व‚स्थायां ।}} विशेष‚रूप‚त‚ये ति । ‚{\color{DodgerBlue3}‚किम‚प्य‚युक्त}‚त‚या हेयोपादेय‚विष‚{\tiny $_{lb}$}‚याप‚रिनिष्ठाना‚{\color{DodgerBlue3}‚दाकुलं प्र‚ल‚प‚न्ति त‚द‚पि प्र‚तिक्षिप्त‚मेकान्त‚स्य भेद‚स्य स‚म्भ‚वात्} ।
	\pend% ending standard par
      

	  \pstart \leavevmode% starting standard par
	आकुल‚त्व‚मेवाख्यातुमाह ।
	\pend% ending standard par
      
	  \bigskip
	  \begingroup
	
	    \large
	  
	    \begin{quote}
	  
	    
	    \stanza[\smallbreak]
	\label{pv.3.181b}\flagstanza{\tiny\textenglish{...3.181b}}स‚र्वंस्योभ‚य‚रूप‚त्वे त‚द्विशेष‚निराकृतेः ॥ १८१ ॥\&[\smallbreak]


	
	    \end{quote}
	  
	  \endgroup
	
	  \bigskip
	  \begingroup
	
	    \large
	  
	    \begin{quote}
	  
	    
	    \stanza[\smallbreak]
	\label{pv.3.182a}\flagstanza{\tiny\textenglish{...3.182a}}चोदितो 
	  \bigskip
	  \begingroup
	द‚धि खादे
	  \endgroup
	 ति किमुष्ट्रं नाभिधाव‚ति ।\&[\smallbreak]


	
	    \end{quote}
	  
	  \endgroup
	

	  \pstart \leavevmode% starting standard par
	\hphantom{.}‚{\color{DodgerBlue3}‚स‚र्व्व‚स्य} व‚स्तुन ‚{\color{DodgerBlue3}‚उभ‚य‚रुप‚त्वे}\edtext{}{\edlabel{pvv.351-5}\label{pvv.351-5}\lemma{स्तुन}\Bfootnote{द‚ध्यादेरुष्ट्रादिषु तादात्म्यानुग‚मात् ।}} स्व‚प‚र‚रूप‚त्वे स‚ति ‚{\color{DodgerBlue3}‚त‚द्विशेष}‚स्य द‚ध्येव द‚धिनोष्ट्र\leavevmode\ledsidenote{\textenglish{69b/MA}} ‚{\tiny $_{lb}$}‚उष्ट्र एवोष्ट्रो न द‚धीत्य‚स्य भेद‚स्य ‚{\color{DodgerBlue3}‚निराकृतेः । द‚धि खादेति चोदितो} नियोज्यः ‚{\tiny $_{lb}$}‚‚{\color{DodgerBlue3}‚किमुष्ट्रं} प्र‚ति ‚{\color{DodgerBlue3}‚नाभिधाव‚ति} ।
	\pend% ending standard par
      
	  \bigskip
	  \begingroup
	
	    \large
	  
	    \begin{quote}
	  
	    
	    \stanza[\smallbreak]
	\label{pv.3.182b}\flagstanza{\tiny\textenglish{...3.182b}}अथार‚स्त्य‚तिश‚यः क‚शिच‚द् येन भेदेन व‚र्त्त‚ते ॥ १८२ ॥\&[\smallbreak]


	
	    \end{quote}
	  
	  \endgroup
	
	  \bigskip
	  \begingroup
	
	    \large
	  
	    \begin{quote}
	  
	    
	    \stanza[\smallbreak]
	\label{pv.3.183a}\flagstanza{\tiny\textenglish{...3.183a}}स एव विशेषोऽन्य‚त्र नास्तीत्य‚नुभ‚यं प‚र‚म् ।\&[\smallbreak]


	
	    \end{quote}
	  
	  \endgroup
	

	  \pstart \leavevmode% starting standard par
	\hphantom{.}‚{\color{DodgerBlue3}‚अथास्ति} द‚ध्नः स‚काशाद् उष्ट्र‚स्या‚{\color{DodgerBlue3}‚तिश‚यो} विशेषः ‚{\color{DodgerBlue3}‚क‚श्चिद् येन} विशेषेण ‚{\tiny $_{lb}$}‚चोदितेन ‚{\color{DodgerBlue3}‚भेदेन} प्र‚तिनिय‚मेन द‚धिश‚ब्दाद् द‚ध्न्येव उष्ट्र‚श‚ब्दादुष्ट्र एव ‚{\color{DodgerBlue3}‚प्र‚व‚र्त्त‚ते} (।) ‚{\tiny $_{lb}$}‚एव‚न्त‚र्हि ‚{\color{DodgerBlue3}‚स}‚{\tiny $_{1}$}‚ विशेष एवान्य‚त्रास‚म्भ‚वी उष्ट्रो विशेषो द‚धिल‚क्ष‚{\color{DodgerBlue3}‚णोऽन्य‚त्र} व‚स्तुनि ‚{\tiny $_{lb}$}‚‚{\color{DodgerBlue3}‚नास्तीति} स‚र्व्वं व‚{\color{DodgerBlue3}‚स्त्व‚नुभ‚यं}\edtext{}{\edlabel{pvv.351-6}\label{pvv.351-6}\lemma{व}\Bfootnote{स्व‚रूप‚भेद‚व‚ल्ल‚क्ष‚ण‚भेद‚श्चेत् न च द्र‚व्य‚त्वं द्र‚व्यादिव्य‚तिरिक्तं भावि ।}} न स्व‚प‚र‚रूपं किन्तु ‚{\color{DodgerBlue3}‚प‚र}‚मेव प‚र‚स्मात् ।
	\pend% ending standard par
      \textsuperscript{\textenglish{352/s}}

	  \pstart \leavevmode% starting standard par
	किञ्च (।)
	\pend% ending standard par
      
	  \bigskip
	  \begingroup
	
	    \large
	  
	    \begin{quote}
	  
	    
	    \stanza[\smallbreak]
	\label{pv.3.183b}\flagstanza{\tiny\textenglish{...3.183b}}स‚र्व्वात्म‚त्वे च स‚र्व्वेषां भिन्नौ स्यातां न धीध्व‚नी ॥ १८३ ॥\&[\smallbreak]


	
	    \end{quote}
	  
	  \endgroup
	
	  \bigskip
	  \begingroup
	
	    \large
	  
	    \begin{quote}
	  
	    
	    \stanza[\smallbreak]
	\label{pv.3.184a}\flagstanza{\tiny\textenglish{...3.184a}}भेद‚संहार‚वाद‚स्य त‚द‚भेदाद‚स‚म्भ‚वः ।\&[\smallbreak]


	
	    \end{quote}
	  
	  \endgroup
	

	  \pstart \leavevmode% starting standard par
	\hphantom{.}‚{\color{DodgerBlue3}‚स‚र्व्वेषां} भावानां ‚{\color{DodgerBlue3}‚स‚र्व्वात्म‚त्वे च भिन्नौ धीध्व‚नी} न स्यातामेक‚विष‚य‚त्वात् । ‚{\tiny $_{lb}$}‚त‚योर्धीध्व‚न्यो‚{\color{DodgerBlue3}‚र‚भेदात् भेद‚संहार‚वाद‚स्यास‚म्भ‚वः} स्यात् । उष्ट्राद् भिन्नं द‚धीति ‚{\tiny $_{lb}$}‚भेद‚व्य‚व‚हारो द‚ध्येवोष्ट्र इति च त‚दात्म‚तोप‚संहार‚व्य‚व‚हार‚श्च‚{\tiny $_{2}$}‚ बुद्धिश‚ब्द‚योर‚भे‚{\tiny $_{lb}$}‚दान्न स्यात् । न हि बुद्धिश‚ब्द‚योर्भेद‚व्य‚व‚हारो युक्तः । त‚न्निब‚न्ध‚न‚त्वात् त‚स्य । ‚{\tiny $_{lb}$}‚त‚द‚भावेपि भावे चातिप्र‚स‚ङ्गात् । भेद‚प्र‚तीत्योर्भावात् तादात्म्योप‚संहार‚श्च क‚थं ‚{\tiny $_{lb}$}‚त‚द‚धीन‚त्वात् त‚स्य ।
	\pend% ending standard par
      
	  
	% new div opening: depth here is 1
	
\chapter*[{३. श‚ब्द‚चिन्ता}]{३. श‚ब्द‚चिन्ता}\label{div_pvv.3.184cd_3.185_3.186_3.187_3.188_3.189_3.190_3.191_3.192_3.193_3.194_3.195_3.196_3.197_3.198_3.199_3.200}
	  
	% new div opening: depth here is 2
	
	  \bigskip
	  \begingroup
	
	    \large
	  
	    \begin{quote}
	  
	    
	    \stanza[\smallbreak]
	\label{pv.3.184b}\flagstanza{\tiny\textenglish{...3.184b}}द्र‚व्याभावाद‚भाव‚स्य श‚ब्दा रूपाभिधायिनः ॥ १८४ ॥\&[\smallbreak]


	
	    \end{quote}
	  
	  \endgroup
	
	  \bigskip
	  \begingroup
	
	    \large
	  
	    \begin{quote}
	  
	    
	    \stanza[\smallbreak]
	\label{pv.3.185a}\flagstanza{\tiny\textenglish{...3.185a}}नाश‚ङ्क्या एव सिद्धास्तेऽतोव्य‚व‚च्छेद‚वाच‚काः ।\&[\smallbreak]


	
	    \end{quote}
	  
	  \endgroup
	

	  \pstart \leavevmode% starting standard par
	उक्तं ताव‚द् भावादिश‚ब्दानां व्य‚व‚च्छेद‚विष‚य‚त्वं । येप्य‚भावादिश‚ब्दा ‚{\tiny $_{lb}$}‚‚{\color{DodgerBlue3}‚अभाव‚स्य} स्व‚रूपा‚{\color{DodgerBlue3}‚भावात्} तेपि ‚{\color{DodgerBlue3}‚रूपाभिधायिनो} व‚स्तुवाच‚का ‚{\color{DodgerBlue3}‚नाश‚ङ‚क्या एव} । ‚{\tiny $_{lb}$}‚स‚म्भ‚व‚द् व‚स्तु वाच्यं‚{\tiny $_{3}$}‚ स्यान्न वेति चिन्त्येतापि ।\edtext{\textsuperscript{*}}{\edlabel{pvv.352-1}\label{pvv.352-1}\lemma{*}\Bfootnote{य‚थाऽविष‚य‚श‚ब्दाभावाग्राह‚क‚त्वाद‚पोह‚विष‚याः सिद्धा इति दृष्टान्तेन प‚रः ‚{\tiny $_{lb}$}‚साध्य‚स्त‚थाभाव‚विष‚या अप्य‚पोह‚विष‚याभाव‚स्व‚रूपाग्राह‚क‚त्वात् (।) केव‚ल‚म‚ध्य‚व‚सा‚{\tiny $_{lb}$}‚यात्तृ त‚द्विष‚य‚त्वं ।}} य‚त्र तु व‚स्त्वेव नास्ति त‚त्र का ‚{\tiny $_{lb}$}‚चिन्ता । ‚{\color{DodgerBlue3}‚अत‚स्ते}‚ऽभावादिश‚ब्दा ‚{\color{DodgerBlue3}‚व्य‚व‚च्छेद‚स्य} भाव‚व्यावृत्ते‚{\color{DodgerBlue3}‚र्व्वाच‚काः सिद्धाः} । ‚{\tiny $_{lb}$}‚त‚स्मात् स्थित‚मेत‚त् स्व‚भाव‚हेतोर्व्व‚स्तुतः साध्यात्म‚क‚त्वेपि न प्र‚तिज्ञार्थैक‚देशो ‚{\tiny $_{lb}$}‚हेतुः । साध्य‚साध‚न‚ध‚र्मिध्व‚निविक‚ल्पानां भिन्न\edtext{}{\edlabel{pvv.352-2}\label{pvv.352-2}\lemma{भिन्न}\Bfootnote{कार‚णैर्न कृत इति द्वितीयादिक्ष‚ण‚स्थायीति श‚ब्दे भिन्न‚भिन्नारोप‚व्य‚{\tiny $_{lb}$}‚व‚च्छेद‚क‚त्वेन ।}}व्य‚व‚च्छेद‚विष‚य\edtext{}{\edlabel{pvv.352-3}\label{pvv.352-3}\lemma{य}\Bfootnote{व‚स्तुविष‚य‚त्वे स्यात् प्र‚तिज्ञार्थैक‚देश‚ता ।}}त्वात् । न च ‚{\tiny $_{lb}$}‚साध्यादीनां क‚ल्पित‚त्वं भेद‚स्य क‚ल्प‚नात् । श‚ब्द‚त्वेन निश्चितो ध‚{\tiny $_{4}$}‚र्मी स‚त्त्वेन च ‚{\tiny $_{lb}$}‚हेतुः (।) क्ष‚णिक‚त्वेन निश्चितः साध्यः । इत्य‚क‚ल्पिता एव ध‚र्म्याद‚य इत्युक्तं ।
	\pend% ending standard par
      

	  \pstart \leavevmode% starting standard par
	स चायं स्व‚भावो हेतुः ॥
	\pend% ending standard par
      
	  \bigskip
	  \begingroup
	
	    \large
	  
	    \begin{quote}
	  
	    
	    \stanza[\smallbreak]
	\label{pv.3.185b}\flagstanza{\tiny\textenglish{...3.185b}}उपाधिभेदापेक्षो वा स्व‚भावः केव‚लोऽथ‚वा ॥ १८५ ॥\&[\smallbreak]


	
	    \end{quote}
	  
	  \endgroup
	
	  \bigskip
	  \begingroup
	
	    \large
	  
	    \begin{quote}
	  
	    
	    \stanza[\smallbreak]
	\label{pv.3.186a}\flagstanza{\tiny\textenglish{...3.186a}}उच्य‚ते साध्य‚सिध्य‚र्थं नाशे कार्य‚त्व‚स‚त्त्व‚व‚त् ।\&[\smallbreak]


	
	    \end{quote}
	  
	  \endgroup
	\textsuperscript{\textenglish{353/s}}

	  \pstart \leavevmode% starting standard par
	क्व‚चिदुपाधिभेदो विशेष‚ण‚विशेषो भिन्नो\edtext{}{\edlabel{pvv.353-1}\label{pvv.353-1}\lemma{भिन्नो}\Bfootnote{य‚द्य‚पि कृत‚के स‚त्त्व‚म‚स्ति त‚थापि हेतुकृतोयं कृत‚क एत‚न्मात्र‚म्विव‚क्षितं ‚{\tiny $_{lb}$}‚न साम‚र्थ्यं ।}}ऽभिन्नो\edtext{}{\edlabel{pvv.353-2}\label{pvv.353-2}\lemma{ऽभिन्नो}\Bfootnote{स्थान‚कार‚णादिप्र‚त्य‚य‚भेदित्व‚पुरूष‚प्र‚य‚त्न‚ज‚त्वं स्व‚भाव‚भूत‚ध‚र्म‚भेदेनोत्प‚त्ति‚{\tiny $_{lb}$}‚म‚त्त्वं ।}} वा त‚द‚पेक्षः । ‚{\color{DodgerBlue3}‚केव‚लो} विशेष‚ण‚र‚हितः शुद्धो‚{\color{DodgerBlue3}‚थ‚वा साध्य‚सिध्य‚र्थ‚मुच्य‚ते\edtext{}{\edlabel{pvv.353-3}\label{pvv.353-3}\lemma{ते}\Bfootnote{अनित्ये कृत‚क‚मुपाधिभेदापेक्षं स‚त्त्व‚म‚न‚पेक्षं ।}} नाशे कार्य‚त्व‚स‚त्त्व‚व‚त्} । नाशे ‚{\tiny $_{lb}$}‚साध्ये कार्य‚त्वं भिन्न‚विशेष‚णा‚{\color{DodgerBlue3}‚पेक्षः स्व‚भावः} त‚ज्ज‚न्म‚न्य‚पेक्षित‚प‚{\tiny $_{5}$}‚र‚व्यापार‚स्य ‚{\tiny $_{lb}$}‚कार्य‚त्वात् । एवं प्र‚त्य‚य‚भेद‚भेदित्वाद‚यो द्र‚ष्ट‚व्याः । उत्प‚त्तिम‚त्वं पुन‚र‚भिन्न‚{\tiny $_{lb}$}‚विशेष‚ण‚मुत्प‚त्त्या स्व‚भाव‚भूत‚या क‚ल्पित‚भेद‚या विशेष‚णात् । स‚त्त्व‚न्तु केव‚लं नाश ‚{\tiny $_{lb}$}‚एव साध्ये स्व‚भावो विशेष‚णानुपादानात् ।
	\pend% ending standard par
      
	  \bigskip
	  \begingroup
	
	    \large
	  
	    \begin{quote}
	  
	    
	    \stanza[\smallbreak]
	\label{pv.3.186b}\flagstanza{\tiny\textenglish{...3.186b}}स‚त्तास्व‚भावो हेतुश्चेत् सा स‚त्ता साध्य‚ते क‚थ‚म् ॥ १८६ ॥\&[\smallbreak]


	
	    \end{quote}
	  
	  \endgroup
	
	  \bigskip
	  \begingroup
	
	    \large
	  
	    \begin{quote}
	  
	    
	    \stanza[\smallbreak]
	\label{pv.3.187a}\flagstanza{\tiny\textenglish{...3.187a}}भेदेनान‚न्व‚यात् सोयं व्याह‚तो हेतुसाध्य‚योः ।\&[\smallbreak]


	
	    \end{quote}
	  
	  \endgroup
	

	  \pstart \leavevmode% starting standard par
	\hphantom{.}न‚नु ‚{\color{DodgerBlue3}‚स\edtext{}{\edlabel{pvv.353-4}\label{pvv.353-4}\lemma{स}\Bfootnote{स‚त्त्वे}}त्तास्व‚भावो हेतुश्चेद}‚भिम‚तः सामान्य‚रूपो विशेष‚स्यान‚न्व‚यात् । ‚{\color{DodgerBlue3}‚त‚दा ‚{\tiny $_{lb}$}‚सा स‚त्ता} प्र‚धाना\edtext{}{\edlabel{pvv.353-5}\label{pvv.353-5}\lemma{धाना}\Bfootnote{सांख्य‚स्यास्ति प्र‚धान‚मिति साध्यं ।}}ख्या स‚र्व्व‚व्य‚क्तिव्यापिनी ‚{\color{DodgerBlue3}‚क‚थं साध्य‚ते}‚ऽचे‚{\tiny $_{2}$}‚\edtext{}{\edlabel{pvv.353-6}\label{pvv.353-6}\lemma{ऽचे}\Bfootnote{अथ स‚त्तायाः सामान्ये साध्ये सिद्ध‚साध्य‚ताऽतो विशेषः साध्यः त‚त्रान्व‚यात् ‚{\tiny $_{lb}$}‚साध्य‚शून्यो दृष्टान्तोतो न स‚त्ता ।}}त‚न‚त्वा‚{\tiny $_{lb}$}‚दिहेतोः । अथ भेदानां प‚र‚स्प‚र‚म‚त्य‚न्तं ‚{\color{DodgerBlue3}‚भेदेनान‚न्व‚यान्न} साध्य‚ते त‚दा ‚{\color{DodgerBlue3}‚सोयं} भेदाना‚{\tiny $_{lb}$}‚म‚न‚न्व‚यो हि ‚{\color{DodgerBlue3}‚हे\edtext{}{\edlabel{pvv.353-7}\label{pvv.353-7}\lemma{हे}\Bfootnote{स‚त्त्व‚हेताव‚साध्य‚शून्यो दृष्टान्त‚विशेषाणां दुष्टो हेतौ साध्यं च ।}}तुसाध्य‚योर्व्याहितः} ।
	\pend% ending standard par
      

	  \pstart \leavevmode% starting standard par
	अर्थान‚न्व‚यिनः साध्य‚ता न युक्ता त‚था साध‚न‚तापि । त‚त् क‚थं स‚त्त्वं साध‚नं ‚{\tiny $_{lb}$}‚(।) त‚स्मात् स‚त्ता सामान्यं साध्य‚ञ्च \edtext{}{\edlabel{pvv.353-8}\label{pvv.353-8}\lemma{ञ्च}\Bfootnote{असिद्ध‚त्वात् ।}} साध‚न‚ञ्चान‚न्व‚यात् स्यात् \edtext{}{\edlabel{pvv.353-9}\label{pvv.353-9}\lemma{स्यात्}\Bfootnote{इति सांख्योक्तौ बौद्धो दोष‚माह ।}} ।
	\pend% ending standard par
      

	  \pstart \leavevmode% starting standard par
	अत्राह ।
	\pend% ending standard par
      
	  \bigskip
	  \begingroup
	
	    \large
	  
	    \begin{quote}
	  
	    
	    \stanza[\smallbreak]
	\label{pv.3.187b}\flagstanza{\tiny\textenglish{...3.187b}}भावोपादान‚मात्रे तु साध्ये सामान्य‚ध‚र्मिणि ॥ १८७ ॥\&[\smallbreak]


	
	    \end{quote}
	  
	  \endgroup
	
	  \bigskip
	  \begingroup
	
	    \large
	  
	    \begin{quote}
	  
	    
	    \stanza[\smallbreak]
	\label{pv.3.188a}\flagstanza{\tiny\textenglish{...3.188a}}न क‚श्चिद‚र्थः सिद्धः स्याद‚निषिद्ध‚ञ्च तादृश‚म् ।\&[\smallbreak]


	
	    \end{quote}
	  
	  \endgroup
	

	  \pstart \leavevmode% starting standard par
	\hphantom{.}‚{\color{DodgerBlue3}‚भावः} स‚त्ता सा ‚{\color{DodgerBlue3}‚उपादानं} विशेष‚णं य‚स्य स भावोपादानः । स एव केव‚ल‚स्त‚{\tiny $_{lb}$}‚‚{\color{DodgerBlue3}‚न्मात्रं} त‚स्मिन् ‚{\tiny $_{7}$}‚ ‚{\color{DodgerBlue3}‚साध्ये सामान्य‚ध‚र्मिणि} सामान्य‚ध‚र्म‚व‚ति ‚{\color{DodgerBlue3}‚न क‚श्चिद‚र्थः} प्र‚धान-\leavevmode\ledsidenote{\textenglish{70a/MA}} ‚{\tiny $_{lb}$}‚\leavevmode\ledsidenote{\textenglish{354/s}} सिद्धिल‚क्ष‚णः सां ख्य स्य \edtext{}{\edlabel{pvv.354-1}\label{pvv.354-1}\lemma{स्य}\Bfootnote{त्रैगुण्यादिर्य‚तो नित्यः}} ‚{\color{DodgerBlue3}‚सिद्धः स्यात्} (।) स‚त्व‚मात्र‚विशेष‚ण‚स्य ध‚र्मिणः साध‚{\tiny $_{lb}$}‚नात् । ‚{\color{DodgerBlue3}‚अनिषिद्ध‚ञ्च तादृशं} साध्यं भेदानां स‚त्व‚स्येष्ट‚त्वात्\edtext{}{\edlabel{pvv.354-2}\label{pvv.354-2}\lemma{त्वात्}\Bfootnote{सिद्ध‚साध्य‚तोक्ता}} ॥
	\pend% ending standard par
      
	  \bigskip
	  \begingroup
	
	    \large
	  
	    \begin{quote}
	  
	    
	    \stanza[\smallbreak]
	\label{pv.3.188b}\flagstanza{\tiny\textenglish{...3.188b}}उपात्त‚भेदे साध्येस्मिन् भ‚वेद्धेतुर‚न‚न्व‚यः ॥ १८८ ॥\&[\smallbreak]


	
	    \end{quote}
	  
	  \endgroup
	
	  \bigskip
	  \begingroup
	
	    \large
	  
	    \begin{quote}
	  
	    
	    \stanza[\smallbreak]
	\label{pv.3.189a}\flagstanza{\tiny\textenglish{...3.189a}}स‚त्तायां तेन साध्यायां विशेषः साधितो भ‚वेत् ।\&[\smallbreak]


	
	    \end{quote}
	  
	  \endgroup
	

	  \pstart \leavevmode% starting standard par
	\hphantom{.}अथैक‚सुखाद्यात्म‚क‚मित्य‚प्र‚धान‚विशेष (ण) विशेषितं स‚त्त्वं साध्यं । त‚दो‚{\color{DodgerBlue3}‚पात्त‚भेदे ‚{\tiny $_{lb}$}‚साध्येऽस्मिन्}\edtext{}{\edlabel{pvv.354-3}\label{pvv.354-3}\lemma{दो}\Bfootnote{निर‚न्व‚य‚विनाशाभावान्नित्ये त्रिगुण‚त्वात् सुख‚दुःख‚मोहात्म‚के क‚र्तृत्वादि‚{\tiny $_{lb}$}‚युक्ते धूमाग्न्योस्त्व‚योग‚व्य‚व‚च्छेदेन व्याप्तिः प्र‚देश‚निष्ठ‚ताऽसिद्धा साध्येति न दोषः ।}} स‚त्त्वे भ‚वेद्धेतुर‚न‚न्व‚योऽन्व‚य‚र‚हितः प्र‚धान‚स्य क्व‚चिद‚न्व‚यासिद्धेः ‚{\tiny $_{lb}$}‚(।) सामान्य‚मेव ‚{\tiny $_{1}$}‚ पुनः किन्न साध्य‚ते । सामान्ये साध्ये सिद्ध‚साध‚न‚दोषात् । ‚{\tiny $_{lb}$}‚‚{\color{DodgerBlue3}‚स‚त्तायां साध्याया}‚मिष्टायां ‚{\color{DodgerBlue3}‚तेन} वादिना ‚{\color{DodgerBlue3}‚विशेष} एवाभिम‚तः ‚{\color{DodgerBlue3}‚साधितो भ‚वेत्} ।
	\pend% ending standard par
      

	  \pstart \leavevmode% starting standard par
	अत्र चान‚न्व‚य‚दोष उक्तः ।
	\pend% ending standard par
      

	  \pstart \leavevmode% starting standard par
	अस्माक‚न्तु\edtext{}{\edlabel{pvv.354-4}\label{pvv.354-4}\lemma{न्तु}\Bfootnote{दोष‚माह ।}}(।)
	\pend% ending standard par
      
	  \bigskip
	  \begingroup
	
	    \large
	  
	    \begin{quote}
	  
	    
	    \stanza[\smallbreak]
	\label{pv.3.189b}\flagstanza{\tiny\textenglish{...3.189b}}अप‚रामृष्ट‚त‚द्भेदे व‚स्तुमात्रे तु साध‚ने ॥ १८९ ॥\&[\smallbreak]


	
	    \end{quote}
	  
	  \endgroup
	
	  \bigskip
	  \begingroup
	
	    \large
	  
	    \begin{quote}
	  
	    
	    \stanza[\smallbreak]
	\label{pv.3.190a}\flagstanza{\tiny\textenglish{...3.190a}}त‚न्मात्र‚व्यापिनः साध्य‚स्यान्व‚यो न विह‚न्य‚ते ।\&[\smallbreak]


	
	    \end{quote}
	  
	  \endgroup
	

	  \pstart \leavevmode% starting standard par
	\hphantom{.}‚{\color{DodgerBlue3}‚अप‚रामुष्टो}‚ऽन‚ध्य‚व‚सितः त‚स्य व‚स्तुनो\edtext{}{\edlabel{pvv.354-5}\label{pvv.354-5}\lemma{स्तुनो}\Bfootnote{स‚त्तायाः ।}} ‚{\color{DodgerBlue3}‚भेदो} य‚स्मिन् त‚स्मिन् ‚{\color{DodgerBlue3}‚व‚स्तुमात्रे ‚{\tiny $_{lb}$}‚तु साध‚ने त‚न्मात्र‚व्यापिनः साध्य‚स्या}‚नित्य‚त्व‚स्या‚{\color{DodgerBlue3}‚न्व‚यो न विह‚न्य‚ते}‚ऽतः साध‚न‚त्वं ‚{\tiny $_{lb}$}‚स‚त्त्व‚स्य युक्तं ।
	\pend% ending standard par
      

	  \pstart \leavevmode% starting standard par
	न‚नु धूमाद‚ग्नि‚{\tiny $_{2}$}‚साध‚नेपि स‚मान‚मेत‚त् । त‚था हि य‚दि धूमाद‚ग्निस‚त्तामात्रं ‚{\tiny $_{lb}$}‚साध्य‚ते त‚दा सिद्ध‚साध्य‚ता । अथ प‚र्व्व‚तेऽस्तीति साध्य‚ते त‚दाऽन्व‚यासिद्धिः । ‚{\tiny $_{lb}$}‚नैत‚द‚स्ति । \edtext{\textsuperscript{*}}{\edlabel{pvv.354-6}\label{pvv.354-6}\lemma{*}\Bfootnote{य‚त्र धूम‚स्त‚त्राग्निरित्य‚ग्निमात्रेण व्याप्तोग्निनान्त‚रीयो धूमः सिद्धो य‚त्रै‚{\tiny $_{lb}$}‚व दृश्य‚ते त‚त्रैवाग्निर्बुद्धिं ज‚न‚य‚ति त‚त्र च साध्य‚निर्देशेन न किञ्चित् ।}} न हि प‚क्षायोग‚व्य‚व‚च्छेद‚स्त‚म‚ग्निं विशेषी\edtext{}{\edlabel{pvv.354-7}\label{pvv.354-7}\lemma{विशेषी}\Bfootnote{प‚क्षोग्नेर्न विशेष‚णं ।}}क‚रोति । अन‚ग्निव्यावृत्त‚स्य ‚{\tiny $_{lb}$}‚साध‚नात् । स‚त्तासाध‚ने तु त‚द् (स‚त्त्व) विशेष एव प्र‚धानाख्यः साध्यः ।
	\pend% ending standard par
      

	  \pstart \leavevmode% starting standard par
	किञ्च (।)
	\pend% ending standard par
      
	  \bigskip
	  \begingroup
	
	    \large
	  
	    \begin{quote}
	  
	    
	    \stanza[\smallbreak]
	\label{pv.3.190b}\flagstanza{\tiny\textenglish{...3.190b}}नासिद्धे भाव‚ध‚र्मोस्ति व्य‚भिचार्युभ‚याश्र‚यः ॥ १९० ॥\&[\smallbreak]


	
	    \end{quote}
	  
	  \endgroup
	
	  \bigskip
	  \begingroup
	
	    \large
	  
	    \begin{quote}
	  
	    
	    \stanza[\smallbreak]
	\label{pv.3.191a}\flagstanza{\tiny\textenglish{...3.191a}}ध‚र्मो विरुद्धोऽभाव‚स्य सा स‚त्ता साध्य‚ते क‚थ‚म् ।\&[\smallbreak]


	
	    \end{quote}
	  
	  \endgroup
	\textsuperscript{\textenglish{355/s}}

	  \pstart \leavevmode% starting standard par
	स‚त्तायां साध‚न‚म‚चेत‚न‚त्वादीष्टं स‚द्भाव‚ध‚र्मोऽभाव‚ध‚र्म उभ‚य‚{\tiny $_{3}$}‚ ध‚र्मो वा भ‚वेत् । ‚{\tiny $_{lb}$}‚त‚त्र साध‚नात् प्राग‚{\color{DodgerBlue3}‚सिद्धे} भावे ‚{\color{DodgerBlue3}‚भाव‚ध‚र्मो} नास्तीत्य‚सिद्धासौ । ‚{\color{DodgerBlue3}‚उभ‚याश्र‚यो} भावाभा‚{\tiny $_{lb}$}‚व‚ध‚र्म‚श्च ‚{\color{DodgerBlue3}‚व्य‚भिचार्य}‚नौकान्तिकः । न ह्युभ‚य‚ध‚र्म एका\edtext{}{\edlabel{pvv.355-1}\label{pvv.355-1}\lemma{एका}\Bfootnote{यो भाव‚स्य ध‚र्मः स्व‚भावः स क‚थं न भाव‚स्य ।}}न्तेनैक‚स‚त्तां ग‚म‚य‚ति । ‚{\tiny $_{lb}$}‚अमूर्त‚त्व‚मिव व‚स्तुतां\edtext{}{\edlabel{pvv.355-2}\label{pvv.355-2}\lemma{स्तुतां}\Bfootnote{प्र‚स‚ङ्ग‚स्याश्रित‚त्वाद‚त्र मूर्त्त‚निवृत्तिमात्रं भावे विज्ञानेपि निरूपाख्येपीति ‚{\tiny $_{lb}$}‚विप‚क्षाद‚व्यावृत्तेर‚ग‚म‚कं ।}} । ‚{\color{DodgerBlue3}‚अभाव‚स्य} तु ‚{\color{DodgerBlue3}‚ध‚र्मो विरुद्धो}‚ऽस‚त्त्व‚साध‚नात्\edtext{}{\edlabel{pvv.355-3}\label{pvv.355-3}\lemma{नात्}\Bfootnote{हेतुसिद्धिं स्वीकृत्यानेकान्त‚विरुद्ध‚तोक्तिरिहासिद्धाव‚भावात् ।}} । त‚त‚स्त्रि‚{\tiny $_{lb}$}‚विध‚दोष‚दुष्ट‚त्वात् साध‚न‚स्य\edtext{}{\edlabel{pvv.355-4}\label{pvv.355-4}\lemma{स्य}\Bfootnote{य‚त् किंञ्चित् साध‚न‚मुपादीय‚ते त‚स्य ।}} ‚{\color{DodgerBlue3}‚सा स‚त्ता क‚थं साध्य‚ते} ।
	\pend% ending standard par
      

	  \pstart \leavevmode% starting standard par
	साध‚न‚प‚क्षेतु स‚त्ता ध‚र्मिणि सिद्ध‚त्वान्नासिद्धा‚{\tiny $_{4}$}‚ । अनित्य‚ताव्याप्तिप्राप्तेः\edtext{}{\edlabel{pvv.355-5}\label{pvv.355-5}\lemma{ताव्याप्तिप्राप्तेः}\Bfootnote{निश्च‚यात् ।}}विरोध‚{\tiny $_{lb}$}‚व्य‚भिचारौ चापास्तौ । त‚स्माद् ।
	\pend% ending standard par
      
	  \bigskip
	  \begingroup
	
	    \large
	  
	    \begin{quote}
	  
	    
	    \stanza[\smallbreak]
	\label{pv.3.191b}\flagstanza{\tiny\textenglish{...3.191b}}सिद्धः स्व‚भावो ग‚म‚कोऽतो ग‚म्य‚स्त‚स्य व्याप‚कः ॥ १९१ ॥\&[\smallbreak]


	
	    \end{quote}
	  
	  \endgroup
	
	  \bigskip
	  \begingroup
	
	    \large
	  
	    \begin{quote}
	  
	    
	    \stanza[\smallbreak]
	\label{pv.3.192a}\flagstanza{\tiny\textenglish{...3.192a}}सिद्धः स्व‚भाव‚निय‚तः स्व‚निवृत्तौ निव‚र्त‚कः ।\&[\smallbreak]


	
	    \end{quote}
	  
	  \endgroup
	

	  \pstart \leavevmode% starting standard par
	\hphantom{.}‚{\color{DodgerBlue3}‚व्याप्य‚स्व‚भावः} साध्यात्म‚त‚या ‚{\color{DodgerBlue3}‚सिद्धो} (निश्चितो) ग‚म‚कः । त‚स्य व्याप्य‚स्य ‚{\tiny $_{lb}$}‚व्याप‚कः स्व‚भावः सिद्धो ग‚म्यः । व्याप‚क‚स्य त‚त्र भाव एव । व्याप्य‚स्य च त‚त्रैव ‚{\tiny $_{lb}$}‚भाव इत्युभ‚य‚ध‚र्म‚रूपाया व्याप्तेः सिद्ध‚त्वात् ।
	\pend% ending standard par
      

	  \pstart \leavevmode% starting standard par
	\hphantom{.}‚{\color{DodgerBlue3}‚अत‚श्चायं} \edtext{\textsuperscript{*}}{\edlabel{pvv.355-6}\label{pvv.355-6}\lemma{*}\Bfootnote{ग‚म्य‚ग‚म‚क‚त्व‚मुक्त्वा निव‚र्त्त्य‚निव‚र्त्त‚क‚माह ।}}व्याप‚कः ‚{\color{DodgerBlue3}‚स्व‚निवृत्तौ} त‚स्य व्याप्य‚स्य ‚{\color{DodgerBlue3}‚निव‚र्त्त‚कः} । अनेन साध‚र्म्य‚{\tiny $_{lb}$}‚वैध‚र्म्य‚प्र‚योगावु‚{\tiny $_{5}$}‚द्दिष्टौ ।
	\pend% ending standard par
      

	  \begin{center}%% label @type='head'
	\textbf{(१) निर्हेतुक‚विनाशः}
	\end{center}
	

	  \pstart \leavevmode% starting standard par
	उदाह‚र‚ण‚माह ।
	\pend% ending standard par
      
	  \bigskip
	  \begingroup
	
	    \large
	  
	    \begin{quote}
	  
	    
	    \stanza[\smallbreak]
	\label{pv.3.192b}\flagstanza{\tiny\textenglish{...3.192b}}अनित्य‚त्वे य‚था कार्य‚म‚कार्यं वाऽविनाशिनि ॥ १९२ ॥\&[\smallbreak]


	
	    \end{quote}
	  
	  \endgroup
	
	  \bigskip
	  \begingroup
	
	    \large
	  
	    \begin{quote}
	  
	    
	    \stanza[\smallbreak]
	\label{pv.3.193a}\flagstanza{\tiny\textenglish{...3.193a}}अहेतुत्वाद् विनाश‚स्य स्व‚भावाद‚नुब‚न्धिता ।\&[\smallbreak]


	
	    \end{quote}
	  
	  \endgroup
	

	  \pstart \leavevmode% starting standard par
	\hphantom{.}‚{\color{DodgerBlue3}‚अनित्य‚त्वे} साध्ये ‚{\color{DodgerBlue3}‚कार्य}\edtext{}{\edlabel{pvv.355-7}\label{pvv.355-7}\lemma{साध्ये}\Bfootnote{कृत‚क‚त्वं ।}} हेतु‚{\color{DodgerBlue3}‚र्य‚था} य‚त् कृत‚कं त‚द‚नित्यं त‚था घ‚टः कृत‚क‚श्च श‚ब्द ‚{\tiny $_{lb}$}‚इति साध‚र्म्य‚प्र‚योगः । अकार्य‚म‚कार्य‚स्व‚भावो वा‚{\color{DodgerBlue3}‚ऽविनाशिनि} नाशाभाव इति । ‚{\tiny $_{lb}$}‚अनित्य‚त्व‚निवृत्तौ कृत‚क‚त्व‚निवृत्तिर्य‚थाकाशे कृत‚क‚श्च श‚ब्द इति वैध‚र्म्य‚प्र‚योगः । ‚{\tiny $_{lb}$}‚क‚थं पुन‚र्ग‚म्य‚ते स‚त्त्व‚मात्रानुब‚न्धिनी न‚श्व‚र‚तेत्याह (।) ‚{\color{DodgerBlue3}‚अहेतुत्वात्}‚{\tiny $_{6}$}‚ अ\edtext{}{\edlabel{pvv.355-8}\label{pvv.355-8}\lemma{अ}\Bfootnote{व्याप्तिविष‚यं प्र‚माणं विप‚क्षे आध‚क‚माह पुष्टः प‚रेण ।}}हेतुकृत‚त्वाद् ‚{\tiny $_{lb}$}‚\leavevmode\ledsidenote{\textenglish{356/s}} ‚{\color{DodgerBlue3}‚विनाश‚स्य स्व‚भावाद्} व‚स्तुस‚त्तामात्रेणा‚{\color{DodgerBlue3}‚नुब‚न्धिता}\edtext{}{\edlabel{pvv.356-1}\label{pvv.356-1}\lemma{त्तामात्रेणा}\Bfootnote{भाव‚स्ताव‚दित्यादिना निर्हेतुक‚त्वे सिद्धे स्व‚र‚स‚ते (ा) निव‚र्त्त‚माने घ‚टे ‚{\tiny $_{lb}$}‚मुद्‏ग‚रादिस‚हाये क‚पाल‚ज‚न‚के स‚दृश‚क्ष‚णानार‚म्भान्म‚न्द‚म‚तीनां स‚हेतुत्वाव‚सायो मुद्‚{\tiny $_{lb}$}‚ग‚राच्छेदात् संतान‚स्य न प्र‚तीतिबाधापि ।}} ।
	\pend% ending standard par
      

	  \pstart \leavevmode% starting standard par
	क‚स्मादेव‚मित्याह ।
	\pend% ending standard par
      
	  \bigskip
	  \begingroup
	
	    \large
	  
	    \begin{quote}
	  
	    
	    \stanza[\smallbreak]
	\label{pv.3.193b}\flagstanza{\tiny\textenglish{...3.193b}}सापेक्षाणां हि भावानां नाव‚श्य‚म्भावितेंक्ष्य‚ते ॥ १९३ ॥\&[\smallbreak]


	
	    \end{quote}
	  
	  \endgroup
	
	  \bigskip
	  \begingroup
	
	    \large
	  
	    \begin{quote}
	  
	    
	    \stanza[\smallbreak]
	\label{pv.3.194a}\flagstanza{\tiny\textenglish{...3.194a}}बाहुल्येपीति चेत् त‚स्य हेतोः क्व‚चिद‚स‚म्भ‚वः ।\&[\smallbreak]


	
	    \end{quote}
	  
	  \endgroup
	

	  \pstart \leavevmode% starting standard par
	\hphantom{.}‚{\color{DodgerBlue3}‚सापेक्षाणां भावानां हि} य‚स्माद‚{\color{DodgerBlue3}‚व‚श्यंभाविता\edtext{}{\edlabel{pvv.356-2}\label{pvv.356-2}\lemma{श्यंभाविता}\Bfootnote{नित्योपि स्यात् क‚श्चित् ।}} नेक्ष्य‚ते} राग‚स्येव वास‚सि । ‚{\tiny $_{lb}$}‚त‚त‚श्च क‚श्चिद् घ‚टो न विन‚श्येद‚पि । विनाश‚कानां हेतूनां बाहुल्याद‚व‚श्यं इति चेत् । ‚{\tiny $_{lb}$}‚‚{\color{DodgerBlue3}‚बाहुल्येपि} त‚स्य नाश‚क‚स्य हेतोः\edtext{}{\edlabel{pvv.356-3}\label{pvv.356-3}\lemma{हेतोः}\Bfootnote{स‚र्व्व‚ज्ञ‚स‚त्वे तु न दोषो य‚तः य‚दुप‚दिश्य‚ते त‚ज्ज्ञान‚पूर्व्वं य‚थान्य‚त् उप‚दिश्य‚ते ‚{\tiny $_{lb}$}‚च च‚तुरार्य‚स‚त्त्यं स ज्ञानी स‚र्व्व‚वित्}} क्व‚चिद् घ‚टादाव‚स‚म्भ‚वः स्यात् । त‚द्व्या‚{\tiny $_{lb}$}‚\leavevmode\ledsidenote{\textenglish{70b/MA}} घात‚कानाम‚पि‚{\tiny $_{7}$}‚ बाहुल्यात् ।
	\pend% ending standard par
      
	  \bigskip
	  \begingroup
	
	    \large
	  
	    \begin{quote}
	  
	    
	    \stanza[\smallbreak]
	\label{pv.3.194b}\flagstanza{\tiny\textenglish{...3.194b}}एतेन व्य‚भिचारित्व‚मुक्तं कार्याव्य‚व‚स्थितेः ॥ १९४ ॥\&[\smallbreak]


	
	    \end{quote}
	  
	  \endgroup
	
	  \bigskip
	  \begingroup
	
	    \large
	  
	    \begin{quote}
	  
	    
	    \stanza[\smallbreak]
	\label{pv.3.195a}\flagstanza{\tiny\textenglish{...3.195a}}स‚र्वेषां नाश‚हेतूनां हेतुम‚न्नाश‚वादिनाम् ।\&[\smallbreak]


	
	    \end{quote}
	  
	  \endgroup
	

	  \pstart \leavevmode% starting standard par
	\hphantom{.}‚{\color{DodgerBlue3}‚एतेन} नाश\edtext{}{\edlabel{pvv.356-4}\label{pvv.356-4}\lemma{नाश}\Bfootnote{लिङ्ग‚त्वेनोपात्तानां ।}}हेतूनां प्र‚तिरोध‚स‚म्भ‚वेन हेतु\edtext{}{\edlabel{pvv.356-5}\label{pvv.356-5}\lemma{हेतु}\Bfootnote{ये य‚द्भावं प्र‚त्य‚न‚पेक्षास्ते त‚द्भाव‚निय‚ता य‚थाऽस‚म्भ‚व‚त्‏प्र‚तिब‚न्धान्त्या ‚{\tiny $_{lb}$}‚साम‚ग्री कार्योत्पाद‚नेऽन्यान‚पेक्ष‚श्च कृत‚को विनाशे इति स्व‚भाव‚हेतुः ।}}म‚न्नाश‚वादिनां म‚तेन ‚{\color{DodgerBlue3}‚नाश‚हेतूनां} मुद्‏ग‚रादीनां ‚{\color{DodgerBlue3}‚स‚र्व्वेषां} नाशे कार्येऽनुमाप‚यित‚व्ये ‚{\color{DodgerBlue3}‚व्य‚भिचारित्व‚मुक्तं} बोद्ध‚व्यं । ‚{\color{DodgerBlue3}‚कार्या‚{\tiny $_{lb}$}‚व्य‚व‚स्थितेः} । विनाश‚हेतोर्विनाश‚स्योत्प‚त्तिनिय‚माभावात् ।
	\pend% ending standard par
      

	  \pstart \leavevmode% starting standard par
	क‚थं पुन‚र्व्विनाश‚स्याहेतुतेत्याह ।
	\pend% ending standard par
      
	  \bigskip
	  \begingroup
	
	    \large
	  
	    \begin{quote}
	  
	    
	    \stanza[\smallbreak]
	\label{pv.3.195b}\flagstanza{\tiny\textenglish{...3.195b}}असाम‚र्थ्याच्च त‚द्धेतोर्भ‚व‚त्येव स्व‚भाव‚तः ॥ १९५ ॥\&[\smallbreak]


	
	    \end{quote}
	  
	  \endgroup
	
	  \bigskip
	  \begingroup
	
	    \large
	  
	    \begin{quote}
	  
	    
	    \stanza[\smallbreak]
	\label{pv.3.196a}\flagstanza{\tiny\textenglish{...3.196a}}य‚त्र नाम भ‚व‚त्य‚स्माद‚न्य‚त्रापि स्व‚भाव‚तः ।\&[\smallbreak]


	
	    \end{quote}
	  
	  \endgroup
	

	  \pstart \leavevmode% starting standard par
	\hphantom{.}‚{\color{DodgerBlue3}‚असाम‚र्थ्याच्च त‚द्धेतो}‚र्व्विन‚श्व‚र‚स्य वा भाव‚स्य विनाशः क्रि‚{\tiny $_{1}$}‚य‚ते नाश‚हेतुना । ‚{\tiny $_{lb}$}‚त‚त्र विन‚श्व‚र‚स्य स्व‚य‚मेव विनाशाद‚लं नाश‚हेतुना । अविन‚श्व‚र‚स्य नाशं क‚र्त्तुं न ‚{\tiny $_{lb}$}‚क‚श्चित् स‚म‚र्थः । च श‚ब्दात् क्रिया\edtext{}{\edlabel{pvv.356-6}\label{pvv.356-6}\lemma{क्रिया}\Bfootnote{कार‚क‚त्वं ।}}प्र‚तिषेधः स्यात् । त‚था ह्य‚भावो य‚दि प‚र्युदासो ‚{\tiny $_{lb}$}‚\leavevmode\ledsidenote{\textenglish{357/s}} भावान्त‚रं क‚पालादिकं त‚दा त‚स्य मुद्ग‚रादिहेतुतेष्य‚त एव । त‚दुत्पादेपि घ‚ट‚स्य ‚{\tiny $_{lb}$}‚न किञ्चिदिति प्राग्व‚दुप‚ल‚ब्ध्यादिप्र‚स‚ङ्गः । त‚स्माद‚भावं क‚रोति. भावं न क‚रोतीति ‚{\tiny $_{lb}$}‚स्यात् । त‚था चाक‚र्त्तुर‚हेतुतैव । त‚स्माद् विनाश‚हेतोर‚{\tiny $_{2}$}‚योगात् एव विनाशः स्व‚भाव‚तः ‚{\tiny $_{lb}$}‚स्व‚हेतोर्भ‚व‚ति ‚{\color{DodgerBlue3}‚अस्मादि}‚ति\edtext{}{\edlabel{pvv.357-1}\label{pvv.357-1}\lemma{ति}\Bfootnote{स्व‚भाव‚मात्र‚भावात् ।}} भावो हेतुः । ‚{\color{DodgerBlue3}‚य‚त्र नाम} विनाशो ‚{\color{DodgerBlue3}‚भ‚व}‚तीति मुद्‏ग‚रात् ‚{\tiny $_{lb}$}‚क‚पालोत्प‚त्तौ लोकाभिमान‚स्त‚त्र विनाश‚कायोगात् स्व‚हेतोरेव विन‚श्व‚र‚स्व‚भाव‚{\tiny $_{lb}$}‚त‚योत्प‚त्तेर्द्वितीये क्ष‚णे न भ‚व‚तीति व‚क्त‚व्यं । त‚स्माद‚हेतु\edtext{}{\edlabel{pvv.357-2}\label{pvv.357-2}\lemma{हेतु}\Bfootnote{हेतुनिर‚पेक्ष‚त्वात् ।}}त्व‚व्य‚व‚स्थानाद‚न्य‚त्रापि ‚{\tiny $_{lb}$}‚य‚त्र विस‚दृशानुत्प‚त्त्या विनाशोत्प‚त्त्य‚भिमानो नास्ति त‚त्रापि ‚{\color{DodgerBlue3}‚स्व‚भाव‚तः} स्व‚हेतोरेव ‚{\tiny $_{lb}$}‚विन‚श्य‚{\tiny $_{3}$}‚तीति विनाश एक‚क्ष‚ण‚स्थायी भावो जाय‚ते ।
	\pend% ending standard par
      

	  \pstart \leavevmode% starting standard par
	त‚थाविधे च भाव‚मात्रानुरोधिनि विनाशे स‚त्वं हेतुर‚व्य‚भिचारः कार्य‚स्य कार‚णे ‚{\tiny $_{lb}$}‚त‚द‚धीन‚त्वादित्युक्तं । त‚देवाह\edtext{}{\edlabel{pvv.357-3}\label{pvv.357-3}\lemma{देवाह}\Bfootnote{अत्र द्वौ व‚स्तुसाध‚नाविति प्रागुक्त‚मुप‚संह‚र‚ति ।}} ।
	\pend% ending standard par
      
	  \bigskip
	  \begingroup
	
	    \large
	  
	    \begin{quote}
	  
	    
	    \stanza[\smallbreak]
	\label{pv.3.196b}\flagstanza{\tiny\textenglish{...3.196b}}या काचिद् भाव‚विष‚याऽनुमितिर्द्विविधैव सा ॥ १९६ ॥\&[\smallbreak]


	
	    \end{quote}
	  
	  \endgroup
	
	  \bigskip
	  \begingroup
	
	    \large
	  
	    \begin{quote}
	  
	    
	    \stanza[\smallbreak]
	\label{pv.3.197a}\flagstanza{\tiny\textenglish{...3.197a}}स्व‚साध्ये कार्य‚भावाभ्यां स‚म्ब‚न्ध‚निय‚मात् त‚योः ।\&[\smallbreak]


	
	    \end{quote}
	  
	  \endgroup
	

	  \pstart \leavevmode% starting standard par
	\hphantom{.}‚{\color{DodgerBlue3}‚या काचि\edtext{}{\edlabel{pvv.357-4}\label{pvv.357-4}\lemma{काचि}\Bfootnote{विधि ।}}द् भाव‚विष‚यानुमितिः} सा ‚{\color{DodgerBlue3}‚द्विविधैव कार्य‚भावाभ्यां} हेतुभ्यां कार‚ण‚{\tiny $_{lb}$}‚व्याप‚क‚विष‚या भ‚व‚न्ती ‚{\color{DodgerBlue3}‚त‚योः} कार्य‚स्व‚भाव‚योः एव ‚{\color{DodgerBlue3}‚स्व‚साध्ये स‚म्ब‚न्ध‚स्य निय‚मात् ।} अन्य‚स्य तु साध्येऽनाप‚त्तेर‚त‚दात्म‚त्वाच्च नाव्य‚भिचार‚निय‚म‚{\tiny $_{4}$}‚: ।
	\pend% ending standard par
      

	  \begin{center}%% label @type='head'
	\textbf{(२) अनुप‚ल‚ब्धिचिन्ता}
	\end{center}
	

	  \begin{center}%% label @type='head'
	\textbf{क. अनुप‚ल‚ब्धेः प्रामाण्य‚म्}
	\end{center}
	

	  \pstart \leavevmode% starting standard par
	अनुप‚ल‚ब्धेर‚प्युप‚ल‚ब्धिनिवृत्तिमात्र‚ल‚क्ष‚णायाः प्रामाण्य‚माख्यातुमाह ।
	\pend% ending standard par
      
	  \bigskip
	  \begingroup
	
	    \large
	  
	    \begin{quote}
	  
	    
	    \stanza[\smallbreak]
	\label{pv.3.197b}\flagstanza{\tiny\textenglish{...3.197b}}प्र‚वृत्तेर्बुद्धिपूर्व‚त्वात् त‚द्भावानुप‚ल‚म्भ‚ने ॥ १९७ ॥\&[\smallbreak]


	
	    \end{quote}
	  
	  \endgroup
	
	  \bigskip
	  \begingroup
	
	    \large
	  
	    \begin{quote}
	  
	    
	    \stanza[\smallbreak]
	\label{pv.3.198a}\flagstanza{\tiny\textenglish{...3.198a}}प्र‚व‚र्त्तित‚व्यं नेत्युक्ताऽनुप‚ल‚ब्धेः प्र‚माण‚ता ।\&[\smallbreak]


	
	    \end{quote}
	  
	  \endgroup
	

	  \pstart \leavevmode% starting standard par
	\hphantom{.}स‚ज्ज्ञान‚श‚ब्द‚व्य‚व‚हाराणां‚{\color{DodgerBlue3}‚प्र‚वृत्तेर्बुद्धिपूर्व्व‚क‚त्वात् त}‚स्या बुद्धे\edtext{}{\edlabel{pvv.357-5}\label{pvv.357-5}\lemma{बुद्धे}\Bfootnote{त‚स्य प्र‚वृत्तिविष‚य‚स्य भाव‚स्यानुप‚ल‚म्भ‚ने प्र‚त्य‚क्षानुमानाभ्यां प्रेक्षाव‚ता ।}} ‚{\color{DodgerBlue3}‚र्भावानुप‚ल‚म्भ‚ने} कार‚णाभावात् प्र‚व‚र्तित‚व्यं नेति साम‚र्थ्यात् सिध्य‚ति (।) न हि कार‚णाभावं ‚{\tiny $_{lb}$}‚कार्यं युक्तं । अतोऽनुप‚ल‚ब्धेरुप‚ल‚ब्धिनिवृत्तिरुपायाः प्र‚वृत्तिनिषेधे साध्ये प्र‚माण‚{\tiny $_{lb}$}‚तोक्ता चा र्ये ण । न पिशाचादिकं घ‚टादिकं स‚दिति‚{\tiny $_{5}$}‚ व‚क्त‚व्य‚म‚नुप‚ल‚ब्धेरिति ‚{\tiny $_{lb}$}‚स‚द्व्य‚व‚हार‚प्र‚तिषेध‚मात्रं साध्य‚ते न त्व‚स‚त्त्व‚व्य‚व‚हारः ।
	\pend% ending standard par
      \textsuperscript{\textenglish{358/s}}

	  \pstart \leavevmode% starting standard par
	य‚त्र त‚र्हि प्र‚त्य‚क्षानुमान‚योः शास्त्र‚स्य निवृत्तिस्त‚स्याभाव एव साध‚यितुं युक्त ‚{\tiny $_{lb}$}‚इत्याह ।
	\pend% ending standard par
      
	  \bigskip
	  \begingroup
	
	    \large
	  
	    \begin{quote}
	  
	    
	    \stanza[\smallbreak]
	\label{pv.3.198b}\flagstanza{\tiny\textenglish{...3.198b}}शास्त्राधिकारास‚म्ब‚द्धा ब‚ह‚वोर्था अतीन्द्रियाः ॥ १९८ ॥\&[\smallbreak]


	
	    \end{quote}
	  
	  \endgroup
	
	  \bigskip
	  \begingroup
	
	    \large
	  
	    \begin{quote}
	  
	    
	    \stanza[\smallbreak]
	\label{pv.3.199a}\flagstanza{\tiny\textenglish{...3.199a}}अलिङ्गाश्च क‚थ‚न्तेषाम‚भावोनुप‚ल‚ब्धितः ।\&[\smallbreak]


	
	    \end{quote}
	  
	  \endgroup
	

	  \pstart \leavevmode% starting standard par
	\hphantom{.}‚{\color{DodgerBlue3}‚शास्त्राधिकारे\edtext{}{\edlabel{pvv.358-1}\label{pvv.358-1}\lemma{शास्त्राधिकारे}\Bfootnote{शास्त्राधिकारो य‚त्र प्र‚क‚र‚णे त‚त्रान्त‚रीय‚काः ।}}ऽस‚म्ब‚द्धा} अनेन शास्त्र‚विष‚य‚त्व‚माह (।) ‚{\color{DodgerBlue3}‚ब‚ह‚वोऽर्था} अनिय‚त‚{\tiny $_{lb}$}‚कार‚णोप‚निपात‚ज‚न्याः सूक्ष्मा दुर्ल्ल‚क्ष‚भेदा म‚नोवृत्त‚यो ज‚न्मिनां । देश‚फ‚ल‚व्य‚व‚हिता ‚{\tiny $_{lb}$}‚वाऽनुत्प‚न्ना द्र‚व्य‚विशेषा ‚{\color{DodgerBlue3}‚अतीन्द्रियाः} । अने‚{\tiny $_{6}$}‚न प्र‚त्य‚क्षाविष‚य‚तामाह । लिङ्गाच्चा‚{\tiny $_{lb}$}‚न‚नुमेय‚तामाह । ‚{\color{DodgerBlue3}‚तेषाम}‚र्थानां प्र‚त्य‚क्षानुमान‚शास्त्र‚निवृत्तिल‚क्ष‚णाया ‚{\color{DodgerBlue3}‚अनुप‚ल‚ब्धितः ‚{\tiny $_{lb}$}‚क‚थ‚म‚भावः} साध‚यितुं युक्तः । स‚त्य‚प्य‚नुप‚ल‚म्भे तेषां स‚त्त्व‚स‚म्भ‚वात् ।
	\pend% ending standard par
      

	  \pstart \leavevmode% starting standard par
	अत ईदृश्य‚नुप‚ल‚ब्धिः ।
	\pend% ending standard par
      
	  \bigskip
	  \begingroup
	
	    \large
	  
	    \begin{quote}
	  
	    
	    \stanza[\smallbreak]
	\label{pv.3.199b}\flagstanza{\tiny\textenglish{...3.199b}}स‚द‚स‚न्निश्च‚य‚फ‚ला नेति स्याद् वाऽप्र‚माण‚ता ॥ १९९ ॥\&[\smallbreak]


	
	    \end{quote}
	  
	  \endgroup
	
	  \bigskip
	  \begingroup
	
	    \large
	  
	    \begin{quote}
	  
	    
	    \stanza[\smallbreak]
	\label{pv.3.200a}\flagstanza{\tiny\textenglish{...3.200a}}प्र‚माण‚म‚पि काचित् स्याद् लिङ्गातिश‚य‚भाविनी ।\&[\smallbreak]


	
	    \end{quote}
	  
	  \endgroup
	

	  \pstart \leavevmode% starting standard par
	\hphantom{.}‚{\color{DodgerBlue3}‚स‚तोऽस‚न्निश्च‚य‚फ‚ला नेति अप्र‚माण‚ता\edtext{}{\edlabel{pvv.358-2}\label{pvv.358-2}\lemma{ता}\Bfootnote{स‚न्निश्च‚य‚फ‚ला न स‚द्व्य‚व‚हार‚निमित्ताभावात् । नाप्य‚स‚न्निश्च‚य‚फ‚ला स‚न्देहात् । ‚{\tiny $_{lb}$}‚इति हेतोः स्याद्वाऽस्या अप्र‚माण‚ता । निश्च‚य‚फ‚लं हि प्र‚माणं ।}}} वाऽस्याः स्यात् । स‚त्त्व‚प्र‚तिषेधे साध्ये ‚{\tiny $_{lb}$}‚‚{\color{DodgerBlue3}‚काचित्} त्व‚नुप‚ल‚ब्धिर्ल्लिङ्ग‚जा प्र‚तीतिर‚स्मिन्नेव\edtext{}{\edlabel{pvv.358-3}\label{pvv.358-3}\lemma{स्मिन्नेव}\Bfootnote{शास्त्रं हि पुरुषार्थ‚म‚धिवृत्तं त‚त्र च न स‚र्व्व‚म‚धिकृतं पुरुष‚चेतोवृत्तीनां ‚{\tiny $_{lb}$}‚प्र‚त्येक‚मान‚न्त्येनाश‚क्य‚व‚च‚न‚त्वाद‚निय‚तान्निमित्ताद् भ‚व‚न‚शील‚त्वाच्च ब‚हुत्वं । ‚{\tiny $_{lb}$}‚देशादिविप्र‚कृष्टाः पुरुषानुप‚योगिनो न निर्देश्याः । लिङ्ग‚स्यानुप‚ल‚ब्धेर‚तिश‚ये ‚{\tiny $_{lb}$}‚उप‚ल‚ब्धिल‚क्ष‚ण‚प्राप्त‚त्वं त‚द्भावो य‚त्रास्ति ।}} साध्ये ‚{\color{DodgerBlue3}‚प्र‚माण‚म‚पि‚{\tiny $_{7}$}‚ स्याल्लिङ्गाति‚{\tiny $_{lb}$}‚\leavevmode\ledsidenote{\textenglish{71a/MA}} श‚य‚भाविनी} लिङ्ग‚विशेष‚प्र‚भ‚वा । य‚थोक्तं प्राग्घेतुभेद‚व्य‚पेक्ष‚येति (।) उप‚ल‚ब्धि‚{\tiny $_{lb}$}‚ल‚क्ष‚ण‚प्राप्तानुप‚ल‚ब्धिलिङ्ग‚जेत्य‚र्थः ।
	\pend% ending standard par
      
	  \bigskip
	  \begingroup
	
	    \large
	  
	    \begin{quote}
	  
	    
	    \stanza[\smallbreak]
	\label{pv.3.200b}\flagstanza{\tiny\textenglish{...3.200b}}स्व‚भाव‚ज्ञाप‚काज्ञान‚स्यायं न्याय उदाहृतः ॥ २०० ॥\&[\smallbreak]


	
	    \end{quote}
	  
	  \endgroup
	

	  \pstart \leavevmode% starting standard par
	य‚त् पुन‚रुक्त‚म‚प्र‚माण‚म‚नुप‚ल‚ब्धिरिति ।\edtext{\textsuperscript{*}}{\edlabel{pvv.358-4}\label{pvv.358-4}\lemma{*}\Bfootnote{भ‚स्म‚विशेषेण किन्तु पिशाचादेः ।}} य‚स्य क‚स्य‚चित् ‚{\color{DodgerBlue3}‚स्व‚भा}‚व‚स्य ज्ञाप‚क‚स्य\edtext{}{\edlabel{pvv.358-5}\label{pvv.358-5}\lemma{स्य}\Bfootnote{स्व‚भाव‚ज्ञाप‚क‚योर‚ज्ञानं त‚स्य ।}} ‚{\tiny $_{lb}$}‚लिङ्ग‚स्य चा‚{\color{DodgerBlue3}‚ज्ञान‚स्या}‚नुप‚ल‚ब्धे\edtext{}{\edlabel{pvv.358-6}\label{pvv.358-6}\lemma{ब्धे}\Bfootnote{अदृश्य‚विष‚यायाः ।}} ‚{\color{DodgerBlue3}‚र‚यं न्याय उदाहृतः} । न हि स्व‚भावो नोप‚ल‚भ्य‚त\edtext{}{\edlabel{pvv.358-7}\label{pvv.358-7}\lemma{त}\Bfootnote{स्व‚भावाज्ञानेन प्र‚त्य‚क्ष‚निवृत्तिरुक्ता ।}} ‚{\tiny $_{lb}$}‚इत्येव नास्ति । देश‚काल‚स्व‚भाव‚विप्र‚कृष्टानाम‚द‚र्श‚नेपि स‚त्त्वाविरोधात् । त‚तो ‚{\tiny $_{lb}$}‚\leavevmode\ledsidenote{\textenglish{359/s}} नास्ति‚{\tiny $_{1}$}‚ विर‚क्त\edtext{}{\edlabel{pvv.359-1}\label{pvv.359-1}\lemma{क्त}\Bfootnote{अनुमानिवृत्तिरुक्ता लिङ्गाज्ञानेन ।}}चेत इत्याद्य\edtext{}{\edlabel{pvv.359-2}\label{pvv.359-2}\lemma{इत्याद्य}\Bfootnote{हिंसाविर‚तिचेत‚नादेर्नाभ्युद‚य‚हेतुतादि ।}}युक्तं (।) न च कार‚ण‚मित्येव कार्याण्य‚व्य‚व‚धान‚तो ‚{\tiny $_{lb}$}‚भ‚व‚न्ति (।) त‚तो नास्ति दान‚हिंसाविर‚तिचेत‚नानाम‚भ्युद‚य‚हेतुता ‚{\color{DodgerBlue3}‚फ‚लान‚न्त‚र्या}‚{\tiny $_{lb}$}‚भावादित्य‚युक्तं\edtext{}{\edlabel{pvv.359-3}\label{pvv.359-3}\lemma{युक्तं}\Bfootnote{मूषिकादिविष‚विकार‚व‚द् व्य‚व‚हितं फ‚लं स्यात् ।}}। (२००)
	\pend% ending standard par
      \label{div_pvv.3.201_3.202_3.203_3.204_3.205_3.206_3.207_3.208_3.209_3.210_3.211_3.212ab}
	  
	% new div opening: depth here is 2
	
	  \bigskip
	  \begingroup
	
	    \large
	  
	    \begin{quote}
	  
	    
	    \stanza[\smallbreak]
	\label{pv.3.201a}\flagstanza{\tiny\textenglish{...3.201a}}कार्ये तु कार‚काज्ञान‚म‚भाव‚स्यैव साध‚क‚म् ।\&[\smallbreak]


	
	    \end{quote}
	  
	  \endgroup
	

	  \pstart \leavevmode% starting standard par
	कार‚काज्ञान\edtext{}{\edlabel{pvv.359-4}\label{pvv.359-4}\lemma{काज्ञान}\Bfootnote{कार‚णानुप‚ल‚म्भः साध्येऽनुप‚ल‚ब्धिर्या दृश्योपि ।}}न्तु कार्येऽभाव‚स्यैव साध‚कं (।) न ह्य‚स‚ति कार‚णे कार्य‚संभ‚वः\edtext{}{\edlabel{pvv.359-5}\label{pvv.359-5}\lemma{वः}\Bfootnote{य‚दि क‚थ‚ञ्चित् कार‚णाभावः सिध्येत् त‚दा कार्याभावः साध्यः य‚था नात्र ‚{\tiny $_{lb}$}‚धूमोन‚ग्नेः ।}} ॥
	\pend% ending standard par
      
	  \bigskip
	  \begingroup
	
	    \large
	  
	    \begin{quote}
	  
	    
	    \stanza[\smallbreak]
	\label{pv.3.201b}\flagstanza{\tiny\textenglish{...3.201b}}स्व‚भावानुप‚ल‚म्भ‚श्च स्व‚भावेर्थ‚स्य लिङ्गिनि ॥ २०१ ॥\&[\smallbreak]


	
	    \end{quote}
	  
	  \endgroup
	
	  \bigskip
	  \begingroup
	
	    \large
	  
	    \begin{quote}
	  
	    
	    \stanza[\smallbreak]
	\label{pv.3.202a}\flagstanza{\tiny\textenglish{...3.202a}}त‚द‚भावः प्र‚तीयेत हेतुना य‚दि केन‚चित् ॥\&[\smallbreak]


	
	    \end{quote}
	  
	  \endgroup
	

	  \pstart \leavevmode% starting standard par
	\hphantom{.}त‚था‚{\color{DodgerBlue3}‚र्थ‚स्य} व्याप‚क‚त‚या निश्चित‚स्य ‚{\color{DodgerBlue3}‚स्व‚भाव‚स्यानुप‚ल‚म्भ‚श्च स्व‚भावे} व्याप्ये ‚{\tiny $_{lb}$}‚लिङ्गिन्य‚स‚त्त‚या साध्ये साध‚नं‚{\tiny $_{2}$}‚त‚दा च कार‚ण‚व्याप‚कानुप‚ल‚ब्धी ग‚मिके ‚{\color{DodgerBlue3}‚य‚दि केन} चिद्धेतुनोप‚ल‚ब्धिल‚क्ष‚ण‚प्राप्तानुप‚ल‚म्भेनान्येन वा त‚योः कार‚ण‚व्याप‚क‚योर‚भावः प्र‚ती‚{\tiny $_{lb}$}‚य‚ते (।) न तूप‚ल‚म्भाभाव‚मात्रेण स‚न्दिग्धासिद्ध‚त्वात् ।
	\pend% ending standard par
      

	  \begin{center}%% label @type='head'
	\textbf{ख. स्व‚भावानुप‚ल‚ब्धिः}
	\end{center}
	

	  \pstart \leavevmode% starting standard par
	स्व‚भावानुप‚ल‚म्भ‚माह ।
	\pend% ending standard par
      
	  \bigskip
	  \begingroup
	
	    \large
	  
	    \begin{quote}
	  
	    
	    \stanza[\smallbreak]
	\label{pv.3.202b}\flagstanza{\tiny\textenglish{...3.202b}}दृश्य‚स्य द‚र्श‚नाभाव‚कार‚णाऽस‚म्भ‚वे स‚ति ॥ २०२ ॥\&[\smallbreak]


	
	    \end{quote}
	  
	  \endgroup
	
	  \bigskip
	  \begingroup
	
	    \large
	  
	    \begin{quote}
	  
	    
	    \stanza[\smallbreak]
	\label{pv.3.203a}\flagstanza{\tiny\textenglish{...3.203a}}भाव‚स्यानुप‚ल‚म्भ‚स्य भावाभावः प्र‚तीय‚ते ।\&[\smallbreak]


	
	    \end{quote}
	  
	  \endgroup
	

	  \pstart \leavevmode% starting standard par
	\hphantom{.}‚{\color{DodgerBlue3}‚दृश्य‚स्य} व‚स्तुनो ‚{\color{DodgerBlue3}‚द‚र्श‚नाभाव}‚स्य य‚त् ‚{\color{DodgerBlue3}‚कार‚णं} व्य‚व‚धानेन्द्रिय‚वैक‚ल्यादि त‚स्या‚{\color{DodgerBlue3}‚स‚म्भ‚वे} स‚ति ‚{\color{DodgerBlue3}‚भाव‚स्या}‚नुप‚ल‚ब्ध‚स्य ‚{\color{DodgerBlue3}‚भाव} (=‚{\color{DodgerBlue3}‚स‚त्ता}‚)‚{\color{DodgerBlue3}‚ाभावः} स्व‚भा‚{\tiny $_{3}$}‚वानुप‚ल‚ब्धेः ‚{\color{DodgerBlue3}‚प्र‚तीय‚ते} ॥
	\pend% ending standard par
      

	  \begin{center}%% label @type='head'
	\textbf{ग. अनुप‚ल‚ब्धिरेवाभावः}
	\end{center}
	
	  \bigskip
	  \begingroup
	
	    \large
	  
	    \begin{quote}
	  
	    
	    \stanza[\smallbreak]
	\label{pv.3.203b}\flagstanza{\tiny\textenglish{...3.203b}}विरुद्ध‚स्य च भाव‚स्य भावे त‚द्भाव‚बाध‚नात् ॥ २०३ ॥\&[\smallbreak]


	
	    \end{quote}
	  
	  \endgroup
	
	  \bigskip
	  \begingroup
	
	    \large
	  
	    \begin{quote}
	  
	    
	    \stanza[\smallbreak]
	\label{pv.3.204a}\flagstanza{\tiny\textenglish{...3.204a}}त‚द्विरुद्धोप‚ल‚ब्धौ स्याद‚स‚त्ताया विनिश्च‚यः ।\&[\smallbreak]


	
	    \end{quote}
	  
	  \endgroup
	\textsuperscript{\textenglish{360/s}}

	  \pstart \leavevmode% starting standard par
	\hphantom{.}‚{\color{DodgerBlue3}‚विरुद्ध‚स्य}\edtext{\textsuperscript{*}}{\edlabel{pvv.360-1}\label{pvv.360-1}\lemma{*}\Bfootnote{य‚द‚भावः साध्य‚स्त‚द्विरुद्ध‚स्यान‚योरुपादान‚योर‚न्योन्य‚वैगुण्य‚स्याश्र‚य‚त्वेनार‚म्भ‚{\tiny $_{lb}$}‚विरोधात् ।}} निव‚र्त्त‚क‚स्य व‚ह्न‚यादे‚{\color{DodgerBlue3}‚र्भाव‚स्य भावे त}‚स्य ‚{\color{DodgerBlue3}‚श}‚(?स)लिलादेर्निव‚र्त्त्य‚स्य ‚{\tiny $_{lb}$}‚‚{\color{DodgerBlue3}‚भाव‚बाध‚नात् । त‚द्विरुद्धोप‚ल‚ब्धौ} स‚त्याम‚{\color{DodgerBlue3}‚स‚त्ता\edtext{}{\edlabel{pvv.360-2}\label{pvv.360-2}\lemma{त्ता}\Bfootnote{प्र‚तिषेध्य‚स्य ।}}या विनिश्च‚यः} स्यात् ।
	\pend% ending standard par
      

	  \pstart \leavevmode% starting standard par
	न‚न्व‚नुप‚ल‚ब्धेर‚भाव‚साध‚ने को दृष्टान्तः । व्योम‚कुसुमादिरिति चेत् । अत्रापि ‚{\tiny $_{lb}$}‚य‚द्य‚नुप‚ल‚ब्धेर‚भाव‚सिद्धिस्त‚दा दृष्टान्तान्त‚रापेक्षायाम‚न‚व‚स्थाप्र‚स‚ङ्गः । त‚द‚न‚पेक्षायां ‚{\tiny $_{lb}$}‚त‚द‚नुप‚ल‚ब्धिर‚भावाख्यं प्र‚माण‚म‚{\tiny $_{4}$}‚स्तु ।
	\pend% ending standard par
      

	  \pstart \leavevmode% starting standard par
	अस‚म्ब‚द्ध‚मेत‚त् । न ह्य‚भावोऽनुप‚ल‚ब्ध्या साध्य‚ते । अनुप‚ल‚ब्धि\edtext{}{\edlabel{pvv.360-3}\label{pvv.360-3}\lemma{ब्धि}\Bfootnote{उप‚ल‚ब्धिल‚क्ष‚ण‚प्राप्त‚स्य ।}}रेव ह्य‚भावः ‚{\tiny $_{lb}$}‚स च सिद्ध एव । त‚थापि तु मूढः\edtext{}{\edlabel{pvv.360-4}\label{pvv.360-4}\lemma{मूढः}\Bfootnote{अभाव‚त‚त्स्व‚भावानुप‚ल‚ब्धाव‚भाव‚व्य‚व‚हार एव साध्य‚ते ।}} त‚म‚व्य‚व‚ह‚र‚न् निमित्तो\edtext{}{\edlabel{pvv.360-5}\label{pvv.360-5}\lemma{निमित्तो}\Bfootnote{अनुप‚ल‚म्भेन ।}}प‚द‚र्श‚नेन व्य‚व‚हार्य‚ते\edtext{}{\edlabel{pvv.360-6}\label{pvv.360-6}\lemma{ते}\Bfootnote{व्य‚व‚हारं ।}} । ‚{\tiny $_{lb}$}‚त‚था च भावो\edtext{}{\edlabel{pvv.360-7}\label{pvv.360-7}\lemma{भावो}\Bfootnote{निरुपाख्योपि ।}}ऽभाव‚स्य द‚ष्टान्तः\edtext{}{\edlabel{pvv.360-8}\label{pvv.360-8}\lemma{ष्टान्तः}\Bfootnote{य‚था दृश्यः स‚न् दृश्य‚ते त‚था दृश्योऽस‚न्न‚नुप‚ल‚ब्धेः ।}} । त‚योः स्व‚नैमित्तिक\edtext{}{\edlabel{pvv.360-9}\label{pvv.360-9}\lemma{नैमित्तिक}\Bfootnote{उप‚ल‚ब्ध्य‚नुप‚ल‚ब्ध्योर्भावाभाव‚व्य‚व‚हार‚प्र‚व‚र्त्त‚न‚स्य । य‚स्य य‚त्र निमित्तं स‚क‚{\tiny $_{lb}$}‚ल‚म‚प्र‚तिब‚द्ध‚म‚स्ति त‚त्र तेन भ‚वित‚व्यं य‚थाऽङ्कुरादि । अस्ति चोप‚ल‚ब्धिल‚क्ष‚ण‚प्राप्त‚{\tiny $_{lb}$}‚स्यानुप‚ल‚ब्धाव‚स‚द्व्य‚व‚हार‚निमित्त‚मिति स्व‚भाव‚हेतुः अनुप‚ल‚ब्धिः ।}}प्र‚व‚र्त्त‚न‚स्य सिद्ध‚त्वात् ।
	\pend% ending standard par
      

	  \pstart \leavevmode% starting standard par
	न‚नु य‚दिदं\edtext{}{\edlabel{pvv.360-10}\label{pvv.360-10}\lemma{दिदं}\Bfootnote{(दिग्) नागादिनोक्तं ।}}न स‚न्ति प्र‚धानाद‚योऽनुप‚ल‚ब्धेरिति त‚त्र क‚थ‚म‚स‚द्व‚य‚व‚हार‚विधिः ‚{\tiny $_{lb}$}‚स‚द्व्य‚व‚हार‚प्र‚तिषेधो वा । प्र‚धानादिश‚ब्द‚वाच्य‚स्य‚{\tiny $_{5}$}‚ प्र‚तिषेधे त\edtext{}{\edlabel{pvv.360-11}\label{pvv.360-11}\lemma{त}\Bfootnote{वाच्य‚म्विना वाच‚काप्र‚योगात् प्र‚धानं ।}}च्छ‚ब्दाप्र‚योगात्\edtext{}{\edlabel{pvv.360-12}\label{pvv.360-12}\lemma{योगात्}\Bfootnote{प्र‚धान‚श‚ब्दाप्र‚योगे प्र‚तिषेधोपि प्र‚तिषेध्याकीर्त्त‚नान्निर्व्विष‚य‚स्याप्र‚योगाद‚{\tiny $_{lb}$}‚युक्त इति चोद‚काश‚यः ।}}।
	\pend% ending standard par
      

	  \pstart \leavevmode% starting standard par
	अत्राह ।
	\pend% ending standard par
      
	  \bigskip
	  \begingroup
	
	    \large
	  
	    \begin{quote}
	  
	    
	    \stanza[\smallbreak]
	\label{pv.3.204b}\flagstanza{\tiny\textenglish{...3.204b}}अनादिवास‚नोद्भूत‚विक‚ल्प‚प‚रिनिष्ठितः ॥ २०४ ॥\&[\smallbreak]


	
	    \end{quote}
	  
	  \endgroup
	
	  \bigskip
	  \begingroup
	
	    \large
	  
	    \begin{quote}
	  
	    
	    \stanza[\smallbreak]
	\label{pv.3.205a}\flagstanza{\tiny\textenglish{...3.205a}}श‚ब्दार्थ‚स्त्रिविधो ध‚र्मो भावाभावोभ‚याश्र‚यः ।\&[\smallbreak]


	
	    \end{quote}
	  
	  \endgroup
	

	  \pstart \leavevmode% starting standard par
	\hphantom{.}‚{\color{DodgerBlue3}‚अनादिवि}‚क‚ल्पाभ्यास‚{\color{DodgerBlue3}‚वास‚नाया उद्भूत‚विक‚ल्पे प‚रिनिष्ठितः} प्र‚तिभास‚मानः ‚{\tiny $_{lb}$}‚श‚ब्दार्थो ध‚र्मो त्रिविधः । क‚थ\edtext{}{\edlabel{pvv.360-13}\label{pvv.360-13}\lemma{थ}\Bfootnote{क‚थं भावाश्र‚योर्थ‚ज‚न्य‚त्वेनाविक‚ल्प‚त्व‚प्र‚स‚ङ्गात् क‚थ‚म‚भावाश्र‚य‚स्त‚स्याकार‚{\tiny $_{lb}$}‚ण‚त्वात् । उभ‚येऽहेतुक‚त्वात् त्य‚क्तः ।}}मित्याह (।) ‚{\color{DodgerBlue3}‚भावाभावोभ‚याश्र‚यः} । स‚द‚स‚दुभ‚य‚{\tiny $_{lb}$}‚\leavevmode\ledsidenote{\textenglish{361/s}} विक‚ल्प‚वास‚नाप्र‚भ‚व‚त्वात् । त‚द‚ध्य‚व‚सायेन त‚द्विष‚य‚त्वात् । ‚{\color{DodgerBlue3}‚त‚त्र भावोपादानो} विक‚ल्पः प‚टादिर‚भावोपादानः श‚श‚वि‚{\color{DodgerBlue3}‚शा} (?षा) णादिः । ‚{\color{DodgerBlue3}‚उभ‚योपादानः प्र‚धाने}‚{\tiny $_{lb}$}‚श्व‚रादिः ।
	\pend% ending standard par
      
	  \bigskip
	  \begingroup
	
	    \large
	  
	    \begin{quote}
	  
	    
	    \stanza[\smallbreak]
	\label{pv.3.205b}\flagstanza{\tiny\textenglish{...3.205b}}त‚स्मिन् भावानुपादाने साध्येऽस्यानुप‚ल‚म्भ‚न‚म् ॥ २०५ ॥\&[\smallbreak]


	
	    \end{quote}
	  
	  \endgroup
	
	  \bigskip
	  \begingroup
	
	    \large
	  
	    \begin{quote}
	  
	    
	    \stanza[\smallbreak]
	\label{pv.3.206a}\flagstanza{\tiny\textenglish{...3.206a}}त‚था हेतुर्न त‚स्यैवाभावः श‚ब्द‚प्र‚योग‚तः ।\&[\smallbreak]


	
	    \end{quote}
	  
	  \endgroup
	

	  \pstart \leavevmode% starting standard par
	\hphantom{.}‚{\color{DodgerBlue3}‚त‚स्मि‚{\tiny $_{6}$}‚न्} श‚ब्दार्थे प्र‚धानादौ ‚{\color{DodgerBlue3}‚भावानुपादाने} भाव‚भूत‚प्र‚धानाश्र‚ये ‚{\color{DodgerBlue3}‚साध्ये}‚ऽस्य\leavevmode\ledsidenote{\textenglish{71b/MA}} ‚{\tiny $_{lb}$}‚प्र‚धाना\edtext{}{\edlabel{pvv.361-1}\label{pvv.361-1}\lemma{धाना}\Bfootnote{बुद्धिव‚र्त्तिनो ध‚र्मिणः ।}}दे‚{\color{DodgerBlue3}‚स्त‚था} (बाह्य) भावाश्र‚य‚त्वेना‚{\color{DodgerBlue3}‚नुप‚ल‚म्भ‚नं हेतु\edtext{}{\edlabel{pvv.361-2}\label{pvv.361-2}\lemma{हेतु}\Bfootnote{प्र‚धानादिविक‚ल्प‚प्र‚तिभास‚स्य बाह्योपादान‚त्वानुप‚ल‚म्भोस्तीति नाप‚क्ष‚ध‚र्मः ।}}}‚र्व्य‚व‚हार‚साध‚नः । न तु ‚{\tiny $_{lb}$}‚‚{\color{DodgerBlue3}‚त‚स्य}\edtext{}{\edlabel{pvv.361-3}\label{pvv.361-3}\lemma{तु}\Bfootnote{विक‚ल्प‚स्थ‚स्य ॥}} श‚ब्दार्थ‚स्यैवाभावः प्र‚धानादि‚{\color{DodgerBlue3}‚श‚ब्द}‚स्य त‚त्प्र‚तिपाद‚क‚स्य ‚{\color{DodgerBlue3}‚प्र‚योग‚तः}\edtext{}{\edlabel{pvv.361-4}\label{pvv.361-4}\lemma{स्य}\Bfootnote{य‚दि स्व‚ल‚क्ष‚ण‚म‚भिधेयं स्यात् त‚दा त‚त्प्र‚तिषेधेऽन‚र्थ‚क‚स्य श‚ब्द‚स्याप्र‚योगः ‚{\tiny $_{lb}$}‚स्यात् न चैत‚त् ।}} ॥
	\pend% ending standard par
      

	  \pstart \leavevmode% starting standard par
	य‚दि तु व‚स्त्वेव श‚ब्द‚विष‚य‚स्त‚दा (।)
	\pend% ending standard par
      
	  \bigskip
	  \begingroup
	
	    \large
	  
	    \begin{quote}
	  
	    
	    \stanza[\smallbreak]
	\label{pv.3.206b}\flagstanza{\tiny\textenglish{...3.206b}}प‚र‚मार्थैक‚तान‚त्वे श‚ब्दानाम‚निब‚न्ध‚ना ॥ २०६ ॥\&[\smallbreak]


	
	    \end{quote}
	  
	  \endgroup
	
	  \bigskip
	  \begingroup
	
	    \large
	  
	    \begin{quote}
	  
	    
	    \stanza[\smallbreak]
	\label{pv.3.207a}\flagstanza{\tiny\textenglish{...3.207a}}न स्यात् प्र‚वृत्तिर‚र्थेषु द‚र्श‚नान्त‚र‚भेदिषु ।\&[\smallbreak]


	
	    \end{quote}
	  
	  \endgroup
	

	  \pstart \leavevmode% starting standard par
	\hphantom{.}‚{\color{DodgerBlue3}‚प‚र‚मार्थै\edtext{}{\edlabel{pvv.361-5}\label{pvv.361-5}\lemma{मार्थै}\Bfootnote{स्व‚ल‚क्ष (त्वे)}}क‚तान‚त्वे} प‚र‚मार्थैक‚प‚र(वृत्ति)त्वे ‚{\color{DodgerBlue3}‚श‚ब्दानाम‚र्थेषु द‚र्श‚नान्त‚र‚भेदिषु} प्र‚तिद‚र्श‚नं भिन्नाभ्युप\edtext{}{\edlabel{pvv.361-6}\label{pvv.361-6}\lemma{भिन्नाभ्युप}\Bfootnote{सिद्धान्त (त्वे) ।}}ग‚{\tiny $_{1}$}‚मेन नित्य‚त्वानित्य‚त्व‚त्रिगुणीम‚य‚त्वादिक‚ल्पि‚{\color{DodgerBlue3}‚त‚भेदेषु ‚{\tiny $_{lb}$}‚अनिब‚न्ध‚ना} प‚र‚मार्थ‚निब‚न्ध‚न‚र‚हिता ‚{\color{DodgerBlue3}‚प्र‚वृत्तिर्न स्यात्} । न हि प‚र‚स्प‚र‚विरुद्धा ब‚ह‚वो ‚{\tiny $_{lb}$}‚ध‚र्मा एक‚त्र स‚न्ति ।
	\pend% ending standard par
      
	  \bigskip
	  \begingroup
	
	    \large
	  
	    \begin{quote}
	  
	    
	    \stanza[\smallbreak]
	\label{pv.3.207b}\flagstanza{\tiny\textenglish{...3.207b}}अतीताजात‚योर्वापि न च स्याद‚नृतार्थ‚ता ॥ २०७ ॥\&[\smallbreak]


	
	    \end{quote}
	  
	  \endgroup
	
	  \bigskip
	  \begingroup
	
	    \large
	  
	    \begin{quote}
	  
	    
	    \stanza[\smallbreak]
	\label{pv.3.208a}\flagstanza{\tiny\textenglish{...3.208a}}वाचः क‚स्याश्चिदित्येषा बौद्धार्थ‚विष‚या म‚ता ।\&[\smallbreak]


	
	    \end{quote}
	  
	  \endgroup
	

	  \pstart \leavevmode% starting standard par
	\hphantom{.}‚{\color{DodgerBlue3}‚अतीताजात‚योर्व्व‚प्य‚स‚तोर्न} स्याच्छ‚ब्द‚वृत्तिः । ‚{\color{DodgerBlue3}‚न च क‚स्याश्चिद् वाचोऽनृ‚{\tiny $_{lb}$}‚तार्थ‚ता स्यात्} । अर्थ‚म‚न्त‚रेण श‚ब्दाभावात् । य‚स्मादेते दोषा व‚स्तुविष‚य‚त्वे वाच ‚{\tiny $_{lb}$}‚इति त‚स्मादेषा ‚{\color{DodgerBlue3}‚बौद्धा\edtext{}{\edlabel{pvv.361-7}\label{pvv.361-7}\lemma{बौद्धा}\Bfootnote{वाक् ।}}र्थ‚विष‚या} क‚ल्पि\edtext{}{\edlabel{pvv.361-8}\label{pvv.361-8}\lemma{ल्पि}\Bfootnote{विक‚ल्प‚भासी ।}}तार्थ‚गोच‚रा म‚ता ।
	\pend% ending standard par
      

	  \pstart \leavevmode% starting standard par
	य‚श्च श‚ब्दार्थः‚{\tiny $_{2}$}‚ त‚स्य भावानुपादान‚त्वं साध्य‚ते न तु स एव निषिध्य‚तेऽन्य‚था (।)
	\pend% ending standard par
      
	  \bigskip
	  \begingroup
	
	    \large
	  
	    \begin{quote}
	  
	    
	    \stanza[\smallbreak]
	\label{pv.3.208b}\flagstanza{\tiny\textenglish{...3.208b}}श‚ब्दार्थाप‚ह्न‚वे साध्ये ध‚र्माधार‚निराकृतेः ॥ २०८ ॥\&[\smallbreak]


	
	    \end{quote}
	  
	  \endgroup
	
	  \bigskip
	  \begingroup
	
	    \large
	  
	    \begin{quote}
	  
	    
	    \stanza[\smallbreak]
	\label{pv.3.209a}\flagstanza{\tiny\textenglish{...3.209a}}न साध्यः स‚मुदायः स्यात् सिद्धो ध‚र्म‚श्च केव‚लः ।\&[\smallbreak]


	
	    \end{quote}
	  
	  \endgroup
	\textsuperscript{\textenglish{362/s}}

	  \pstart \leavevmode% starting standard par
	\hphantom{.}‚{\color{DodgerBlue3}‚श‚ब्दार्थ‚स्याप‚ह्न‚वे साध्ये ध‚र्माधार}\edtext{\textsuperscript{*}}{\edlabel{pvv.362-1}\label{pvv.362-1}\lemma{*}\Bfootnote{नास्तित्वं साध्यो ध‚र्मः सिद्धान्ती ।}}स्य ध‚र्मिणो\edtext{}{\edlabel{pvv.362-2}\label{pvv.362-2}\lemma{र्मिणो}\Bfootnote{प्र‚धानादिश‚ब्दार्थ‚स्य ।}} ‚{\color{DodgerBlue3}‚निराकृतेः स‚मुदायः साध्यो ‚{\tiny $_{lb}$}‚न स्यात्} । ध‚र्मिध‚र्म‚स‚मुदाय‚श्चानुमेयः । ध‚र्म एव केव‚लः साध्य‚ते इति चेत् ‚{\color{DodgerBlue3}‚सिद्धो\edtext{}{\edlabel{pvv.362-3}\label{pvv.362-3}\lemma{सिद्धो}\Bfootnote{नास्तित्व‚मात्र‚स्य क्व‚चित् सिद्ध‚त्वात् ।}} ‚{\tiny $_{lb}$}‚ध‚र्म‚श्चा}‚भावादिः ‚{\color{DodgerBlue3}‚केव‚लः} (।) किम‚र्थं साध‚नीयः प्र‚धानादिविक‚ल्प‚स्य भावानुपा‚{\tiny $_{lb}$}‚दान‚त्वं तु न सिद्धं त‚देव साध्यं‚{\tiny $_{3}$}‚ युक्तं ।
	\pend% ending standard par
      

	  \pstart \leavevmode% starting standard par
	किञ्च\edtext{}{\edlabel{pvv.362-4}\label{pvv.362-4}\lemma{किञ्च}\Bfootnote{प्र‚तिज्ञा ।}} (।)
	\pend% ending standard par
      
	  \bigskip
	  \begingroup
	
	    \large
	  
	    \begin{quote}
	  
	    
	    \stanza[\smallbreak]
	\label{pv.3.209b}\flagstanza{\tiny\textenglish{...3.209b}}स‚द‚स‚त्प‚क्ष‚भेदेन श‚ब्दार्थान‚प‚वादिभिः ॥ २०९ ॥\&[\smallbreak]


	
	    \end{quote}
	  
	  \endgroup
	
	  \bigskip
	  \begingroup
	
	    \large
	  
	    \begin{quote}
	  
	    
	    \stanza[\smallbreak]
	\label{pv.3.210a}\flagstanza{\tiny\textenglish{...3.210a}}व‚स्त्वेव चिन्त्य‚ते ह्य‚त्र प्र‚तिब‚द्धः फ‚लोद‚यः ।\&[\smallbreak]


	
	    \end{quote}
	  
	  \endgroup
	

	  \pstart \leavevmode% starting standard par
	\hphantom{.}‚{\color{DodgerBlue3}‚स‚द‚स‚त्प‚क्ष‚भेदेन व‚स्त्वेव} व्य‚व‚हारिभिः श‚ब्दार्थान‚प‚वादिभिः चिन्त्य‚ते नाव‚स्तु । ‚{\tiny $_{lb}$}‚हि य‚स्माद् ‚{\color{DodgerBlue3}‚अत्र} व‚स्तुनि ‚{\color{DodgerBlue3}‚फ‚ल}‚स्यार्थ‚क्रियाया ‚{\color{DodgerBlue3}‚उद‚यः प्र‚तिब‚द्धः} ।
	\pend% ending standard par
      

	  \pstart \leavevmode% starting standard par
	त‚त‚श्च (।)
	\pend% ending standard par
      
	  \bigskip
	  \begingroup
	
	    \large
	  
	    \begin{quote}
	  
	    
	    \stanza[\smallbreak]
	\label{pv.3.210b}\flagstanza{\tiny\textenglish{...3.210b}}अर्थ‚क्रियाऽस‚म‚र्थ‚स्य विचारैः किं त‚द््र्थिनाम् ॥ २१० ॥\&[\smallbreak]


	
	    \end{quote}
	  
	  \endgroup
	
	  \bigskip
	  \begingroup
	
	    \large
	  
	    \begin{quote}
	  
	    
	    \stanza[\smallbreak]
	\label{pv.3.211a}\flagstanza{\tiny\textenglish{...3.211a}}ष‚ण्ढ‚स्य रूपे वैरूप्ये कामिन्या किं प‚रीक्ष‚या ।\&[\smallbreak]


	
	    \end{quote}
	  
	  \endgroup
	

	  \pstart \leavevmode% starting standard par
	\hphantom{.}‚{\color{DodgerBlue3}‚अर्थ‚क्रियायाम‚स‚म‚र्थ‚स्य} श‚ब्दार्थादे‚{\color{DodgerBlue3}‚र्व्विचारैः} स‚द‚स‚त्प‚क्ष‚चिन्ताभि‚{\color{DodgerBlue3}‚स्त‚द‚र्थिना}‚म‚र्थ‚{\tiny $_{lb}$}‚क्रियार्थिनां ‚{\color{DodgerBlue3}‚किं} न किञ्चित् प्र‚योज‚नं\edtext{}{\edlabel{pvv.362-5}\label{pvv.362-5}\lemma{नं}\Bfootnote{दृष्टान्त‚माह ।}} (।) ‚{\color{DodgerBlue3}‚ष‚ण्ढ‚स्य} न‚पुन्स‚क‚स्य ‚{\color{DodgerBlue3}‚रूपे वैरूप्ये} वा ‚{\tiny $_{lb}$}‚कामिन्या वृष‚स्य‚न्त्या‚{\tiny $_{4}$}‚ योषितः ‚{\color{DodgerBlue3}‚किं प‚रीक्ष‚या} ।\edtext{\textsuperscript{*}}{\edlabel{pvv.362-6}\label{pvv.362-6}\lemma{*}\Bfootnote{उ द्यो त क रा द्युक्त‚दोष‚निरासाय पृच्छ‚ति नागेन (? दिग्नागः ) ।}}
	\pend% ending standard par
      

	  \begin{center}%% label @type='head'
	\textbf{घ. क‚ल्पित‚स्यानुप‚ल‚ब्धिर्ध‚र्मः}
	\end{center}
	

	  \pstart \leavevmode% starting standard par
	य‚त् पुन‚राचार्येणोक्तं क‚ल्पित‚स्यानुप‚ल‚ब्धिर्द्ध‚र्म्म इति त‚स्य कोर्थः ।\edtext{\textsuperscript{*}}{\edlabel{pvv.362-7}\label{pvv.362-7}\lemma{*}\Bfootnote{उत्त‚र‚माह ।}}
	\pend% ending standard par
      
	  \bigskip
	  \begingroup
	
	    \large
	  
	    \begin{quote}
	  
	    
	    \stanza[\smallbreak]
	\label{pv.3.211b}\flagstanza{\tiny\textenglish{...3.211b}}श‚ब्दार्थः क‚ल्प‚नाज्ञान‚विष‚य‚त्वेन क‚ल्पितः ॥ २११ ॥\&[\smallbreak]


	
	    \end{quote}
	  
	  \endgroup
	
	  \bigskip
	  \begingroup
	
	    \large
	  
	    \begin{quote}
	  
	    
	    \stanza[\smallbreak]
	\label{pv.3.212a}\flagstanza{\tiny\textenglish{...3.212a}}ध‚र्मो व‚स्त्वाश्र‚याऽसिद्धिर‚स्योक्ता न्याय‚वादिना ।\&[\smallbreak]


	
	    \end{quote}
	  
	  \endgroup
	

	  \pstart \leavevmode% starting standard par
	\hphantom{.}‚{\color{DodgerBlue3}‚क‚ल्प‚नाज्ञान}‚स्य ‚{\color{DodgerBlue3}‚विष‚य‚त्वेन क‚ल्पित} इष्टः प्र‚धानादि‚{\color{DodgerBlue3}‚श‚ब्दार्थः । अस्य\edtext{}{\edlabel{pvv.362-8}\label{pvv.362-8}\lemma{अस्य}\Bfootnote{प्र‚धानादिश‚ब्दार्थ‚स्य ।}}} क‚ल्पित‚स्य ‚{\tiny $_{lb}$}‚‚{\color{DodgerBlue3}‚व‚स्त्वा\edtext{}{\edlabel{pvv.362-9}\label{pvv.362-9}\lemma{स्त्वा}\Bfootnote{व‚स्तुनो बाह्य‚स्य प्र‚धान‚स्याश्र‚य‚णं तेनासिद्धिर‚नुप‚ल‚ब्धिर्द्ध‚र्म उक्तो लिङ्ग‚भूतो ‚{\tiny $_{lb}$}‚भावानुपादान‚त्वे साध्ये प्र‚माणेन चेत् स‚त्त्वं प्र‚धान‚स्य । नानुप‚ल‚ब्धिः ध‚र्मः । ‚{\tiny $_{lb}$}‚अस‚त्त्वेप्य‚स‚त्त्वादिति प‚रोक्त‚म‚पास्त‚म‚नेन श‚ब्दार्थ‚स्यैव क‚ल्पित‚त्वात् ।}}श्र‚याऽसिद्धि}‚र्व्व‚स्त्वाधिष्ठान‚त्वानुप‚ल‚ब्धि‚{\color{DodgerBlue3}‚र्द्ध‚र्मो न्याय‚वादिना}‚चार्येणोक्ता ।
	\pend% ending standard par
      
	  
	% new div opening: depth here is 1
	
\chapter*[{४. आग‚म‚चिन्ता}]{४. आग‚म‚चिन्ता}\label{div_pvv.3.212cd_3.213_3.214_3.215_3.216}
	  
	% new div opening: depth here is 2
	

	  \pstart \leavevmode% starting standard par
	\leavevmode\ledsidenote{\textenglish{363/s}}न‚नु य‚दुक्तं (।) प्र‚माण‚त्र‚य\edtext{}{\edlabel{pvv.363-1}\label{pvv.363-1}\lemma{य}\Bfootnote{आग‚मेन स‚ह ।}}निवृत्ताव‚पि नार्थाभाव‚निश्च‚य इति त‚न्मा ‚{\color{DodgerBlue3}‚भूत्} प्र‚त्य‚क्षानुमान‚योर‚स‚र्व्व‚विष‚य‚{\tiny $_{5}$}‚त्वात् त‚न्निवृत्त्या (ऽ) भाव‚निश्च‚यः । आग‚म‚स्तु स‚र्व्व‚{\tiny $_{lb}$}‚विष‚य इति त‚न्निवृत्तौ युक्तोऽर्थास‚त्त्व‚निश्च‚य इत्याह\edtext{}{\edlabel{pvv.363-2}\label{pvv.363-2}\lemma{इत्याह}\Bfootnote{सिद्धान्ती ।}} ।
	\pend% ending standard par
      
	  \bigskip
	  \begingroup
	
	    \large
	  
	    \begin{quote}
	  
	    
	    \stanza[\smallbreak]
	\label{pv.3.212b}\flagstanza{\tiny\textenglish{...3.212b}}नान्त‚रीय‚क‚ताऽभावाच्छ‚ब्दानां व‚स्तुभिस्स‚ह ॥ २१२ ॥\&[\smallbreak]


	
	    \end{quote}
	  
	  \endgroup
	
	  \bigskip
	  \begingroup
	
	    \large
	  
	    \begin{quote}
	  
	    
	    \stanza[\smallbreak]
	\label{pv.3.213a}\flagstanza{\tiny\textenglish{...3.213a}}नार्थ‚सिद्धिस्त‚त‚स्ते हि व‚क्त्त्र‚भिप्राय‚सूच‚काः ॥\&[\smallbreak]


	
	    \end{quote}
	  
	  \endgroup
	

	  \pstart \leavevmode% starting standard par
	\hphantom{.}‚{\color{DodgerBlue3}‚नान्त‚रीय‚क‚ता}‚या अविनाभाव‚स्या‚{\color{DodgerBlue3}‚भावाद् व‚स्तुभिः स‚ह श‚ब्दानां । त‚तः} श‚ब्देभ्यो ‚{\tiny $_{lb}$}‚‚{\color{DodgerBlue3}‚नार्थ‚स्य सिद्धि}‚र्निश्च‚यः । किन्त‚र्हि तेभ्यो ग‚म्य‚त इत्याह\edtext{}{\edlabel{pvv.363-3}\label{pvv.363-3}\lemma{इत्याह}\Bfootnote{...शास्त्राधिकार \href{http://sarit.indology.info/?cref=pv.3.198}{(३।१९८)} इत्य‚त्र नाग‚मे स‚र्व्वार्थ‚निब‚न्ध‚न‚मुक्तं य‚त् त‚द् ‚{\tiny $_{lb}$}‚बाह्यार्थे य‚द्य‚पि घ‚ट‚विव‚क्षातः प‚ट‚श‚ब्द‚स्योत्प‚त्तिस्त‚थापि स्थान‚क‚र‚णाभिधातादेरेव ‚{\tiny $_{lb}$}‚साक्षात्कार‚णादुत्प‚त्तेर्व्य‚भिचाराभावान्नाहेतुक‚त्वं ।}} । ‚{\color{DodgerBlue3}‚व‚क्तुर‚भिप्राय‚स्य} विव‚क्षाया‚{\color{DodgerBlue3}‚स्ते} श‚ब्दाः ‚{\color{DodgerBlue3}‚सूच‚का}‚स्त‚द‚न्व‚य‚व्य‚तिरेकानुविधायित्वात् । न च विव‚क्षा ‚{\tiny $_{lb}$}‚य‚थार्थं भ‚व‚ति\edtext{}{\edlabel{pvv.363-4}\label{pvv.363-4}\lemma{ति}\Bfootnote{आग‚म‚प्रामाण्य‚म‚भ्युप‚ग‚म्यात्र तु नैव बाह्यार्थेऽस्य प्रामाण्य‚मित्युक्त‚म् ।}} ‚{\tiny $_{6}$}‚येन प‚रंप‚र‚या त‚त्स‚म्वादः स्यात् । विस‚म्वादाभिप्रायाद‚ज्ञानाद् ‚{\tiny $_{lb}$}‚वाऽन्य‚थापि विव‚क्षास‚म्भ‚वात् ।
	\pend% ending standard par
      \textsuperscript{\textenglish{72a/MA}}‚{\tiny $_{lb}$}‚
	  \bigskip
	  \begingroup
	
	    \large
	  
	    \begin{quote}
	  
	    
	    \stanza[\smallbreak]
	\label{pv.3.213b}\flagstanza{\tiny\textenglish{...3.213b}}आप्त‚वादाविसंवाद‚सामान्याद‚नुमान‚ता ॥ २१३ ॥\&[\smallbreak]


	
	    \end{quote}
	  
	  \endgroup
	

	  \pstart \leavevmode% starting standard par
	य‚द्येवं स‚र्व्व‚मेव व‚च‚नं प्र‚वृत्तिकामानां प‚रीक्षार्हं स्यात् । क‚श्च स‚म्वादार्थः ‚{\tiny $_{lb}$}‚क‚थ‚ञ्चाप्त‚वाद‚सामान्याद‚नुमान‚तास्या चा र्ये णोक्तेत्याह ।\edtext{\textsuperscript{*}}{\edlabel{pvv.363-5}\label{pvv.363-5}\lemma{*}\Bfootnote{यो य आप्त‚वादः सोऽविसंवादी य‚था क्ष‚णिकाः स‚र्व्व‚संस्कार इत्यादिकः । ‚{\tiny $_{lb}$}‚आप्त‚वाद‚श्चाय‚म‚त्य‚न्त‚प‚रोक्षेप्य‚र्थे इत्य‚विस‚म्वाद‚सामान्यादाग‚म‚स्य बाह्येर्थे (दिग्)‚{\tiny $_{lb}$}‚नागेनानुमान‚मुक्त‚मित्य‚भ्युपेत‚बाधामाह । उच्य‚ते न पुरुषोऽनाश्रित्याग‚म‚प्रामाण्य‚{\tiny $_{lb}$}‚म‚सितुं स‚म‚र्थः । प्र‚त्य‚क्ष‚फ‚लाया हिंसादिविर‚तेः स्व‚र्गादिश्रुतेर‚विर‚तेर्न‚कादिश्रुतेः । ‚{\tiny $_{lb}$}‚त‚द्भावे विरोधाभाव‚च्च । त‚त् स‚ति प्र‚व‚र्त्तित‚व्ये व‚र‚मेवं प्र‚वृत्त इति प‚रीक्ष‚या ‚{\tiny $_{lb}$}‚प्रामाण्य‚माह (।) त‚च्च शास्त्रं न स‚र्व्व‚म‚धिकृतं किन्तु ।}}
	\pend% ending standard par
      
	  \bigskip
	  \begingroup
	
	    \large
	  
	    \begin{quote}
	  
	    
	    \stanza[\smallbreak]
	\label{pv.3.214}\flagstanza{\tiny\textenglish{....3.214}}स‚म्ब‚द्धानुगुणोपायं पुरुषार्थाभिधाय‚क‚म् ।&प‚रीक्षाधिकृतं वाक्य‚म‚तोन‚धिकृत‚म्प‚र‚म् ॥ २१४ ॥\&[\smallbreak]


	
	    \end{quote}
	  
	  \endgroup
	

	  \pstart \leavevmode% starting standard par
	\hphantom{.}‚{\color{DodgerBlue3}‚स‚म्ब‚द्ध}‚वाक्यानां प‚र‚स्प‚राभिस‚म्ब‚द्धानामेकार्थोप‚संहारात् (।) न च द‚श‚दाडि‚{\tiny $_{lb}$}‚मादिवाक्य‚मिवैकार्थान‚भिधायि । ‚{\color{DodgerBlue3}‚अनुगुणोणायं} श‚{\tiny $_{1}$}‚क्यानुष्ठानोपेय‚साध‚नं न तु ‚{\tiny $_{lb}$}‚\leavevmode\ledsidenote{\textenglish{364/s}} विष‚श‚म‚न‚त‚क्ष‚क‚फ‚णार‚त्नाल‚ङ्कारोप‚देश‚क‚मिव ‚{\color{DodgerBlue3}‚पुरुषार्थ‚स्य} स्व‚र्गाप‚व‚र्ग‚स्या‚{\color{DodgerBlue3}‚भिधाय‚कं} न तु काक‚द‚न्त‚{\color{DodgerBlue3}‚प‚रीक्षो}‚प‚देश‚क‚मिव प‚रीक्षायां प्र‚वृत्त्य‚र्ह‚विष‚य‚म‚{\color{DodgerBlue3}‚धिकृतं वाक्यं । ‚{\tiny $_{lb}$}‚अतोऽप‚र‚म‚न‚धिकृत‚म}‚न‚व‚धानार्ह‚त्वात् ।
	\pend% ending standard par
      
	  \bigskip
	  \begingroup
	
	    \large
	  
	    \begin{quote}
	  
	    
	    \stanza[\smallbreak]
	\label{pv.3.215}\flagstanza{\tiny\textenglish{....3.215}}प्र‚त्य‚क्षेणानुमानेन द्विविधेनाप्य‚बाध‚न‚म् ।&दृष्टादृष्टार्थ‚योर‚स्याविसंवादः त‚द‚र्थ‚योः ॥ २१५ ॥\&[\smallbreak]


	
	    \end{quote}
	  
	  \endgroup
	

	  \pstart \leavevmode% starting standard par
	\hphantom{.}अस्य च प‚रीक्षार्ह‚स्य वाक्य‚स्याविस‚म्वादः ‚{\color{DodgerBlue3}‚त‚द‚र्थ‚यो}\edtext{}{\edlabel{pvv.364-1}\label{pvv.364-1}\lemma{म्वादः}\Bfootnote{गुण‚त्र‚य‚युक्त‚ञ्च य‚द्य‚विसंवादि त‚दा प्र‚व‚र्तित्त‚व्य‚मित्य‚विस‚म्वाद‚माह ।}} राग‚माभिधेय‚यो‚{\color{DodgerBlue3}‚र्दृष्टा‚{\tiny $_{lb}$}‚दृष्ट‚योः}‚प्र‚त्य‚क्षा‚{\tiny $_{2}$}‚प्र‚त्य‚क्ष\edtext{}{\edlabel{pvv.364-2}\label{pvv.364-2}\lemma{क्ष}\Bfootnote{प्र‚त्य‚क्षानुमान‚विष‚य‚योः ।}} योर‚र्थ‚योः ‚{\color{DodgerBlue3}‚प्र‚त्य‚क्षेणानुमानेन} च ‚{\color{DodgerBlue3}‚द्विविधेने}‚ति व‚स्तुब‚ल‚भाविना‚{\tiny $_{lb}$}‚ऽग‚माश्र‚येण चा‚{\color{DodgerBlue3}‚बाध‚न}‚म‚न्येषाञ्च बाध‚नं नाम । य‚था प्र‚त्य‚क्ष‚त्वेन स‚म्म‚तानां प‚ञ्चानां ‚{\tiny $_{lb}$}‚स्क‚न्धानां प्र‚त्य‚क्षेणाबाध‚नं सिद्धिरेव अप्र‚त्य‚क्ष‚त्वेनेष्टानां श‚ब्दादि\edtext{}{\edlabel{pvv.364-3}\label{pvv.364-3}\lemma{ब्दादि}\Bfootnote{सां ख्ये श‚ब्दादिस्व‚भावानां सुख‚दुःख‚मोहानां प्र‚त्य‚क्षेणाप्र‚तीतेः । वैशेषिका‚{\tiny $_{lb}$}‚देर्द्र‚व्य‚माकाशादि । अनेक‚द्र‚व्य‚ञ्च द्र‚व्य‚म‚व‚य‚वी । क‚र्मोत्क्षेप‚णादि । सामान्यं स‚त्ता ‚{\tiny $_{lb}$}‚गोत्वादि । आदिना विभागादि ।}}त्रिगुण‚म‚य‚त्व‚{\tiny $_{lb}$}‚द्र‚व्य‚क‚र्म‚सामान्य‚संयोगादीनाञ्च तेन बाध‚नं\edtext{}{\edlabel{pvv.364-4}\label{pvv.364-4}\lemma{नं}\Bfootnote{नीलादिव्य‚तिरेकेणानुप‚ल‚ब्धेः ।}} । अनुमेय‚त्वेनेष्टानां च‚तुरार्य‚स‚त्यानां ‚{\tiny $_{lb}$}‚व‚स्तुब‚ल‚प्र‚वृत्ते\edtext{}{\edlabel{pvv.364-5}\label{pvv.364-5}\lemma{वृत्ते}\Bfootnote{आग‚मापेक्षेण ।}}नानुमा‚{\tiny $_{3}$}‚नेनाबाध‚नं सिद्धिरेव । अन‚नुमेय‚त्वेनेष्टानाञ्चात्मेश्व‚रा‚{\tiny $_{lb}$}‚दीनाम‚नुमानेन\edtext{}{\edlabel{pvv.364-6}\label{pvv.364-6}\lemma{नुमानेन}\Bfootnote{लिङ्गाभावान्नानुमेय‚ता ।}} बाध एव । अत्य‚न्त‚प‚रोक्षाणां रागादिहेतुकाऽध‚र्म‚प्र‚हाणादीनामा‚{\tiny $_{lb}$}‚ग‚माश्र‚यानुमानेनाबाध‚नं सिद्धिरेवं\edtext{}{\edlabel{pvv.364-7}\label{pvv.364-7}\lemma{सिद्धिरेवं}\Bfootnote{राग‚द्वेषादिस्व‚भाव‚म‚ध‚र्म‚न्त‚दुत्थं काय‚क‚र्मादि चाध‚र्म‚म‚भ्युपेत्य स्नानाद्य‚{\tiny $_{lb}$}‚नुक्तेः ।}} रागादिहेतुत्वेनेष्ट‚स्य हेतुत्वानुप‚रोधिनः ‚{\tiny $_{lb}$}‚स्नोनोवासाग्निहोत्रादेः प्र‚हाणोपाय‚त‚याऽनुप‚देशात् हेतुव्याघात‚स्योपाय‚त्वेनोप‚देशा‚{\tiny $_{lb}$}‚च्चैवं‚{\tiny $_{4}$}‚ विध‚म‚बाध‚न‚म‚{\color{DodgerBlue3}‚विस‚म्वाद} इष्टः ।
	\pend% ending standard par
      
	  \bigskip
	  \begingroup
	
	    \large
	  
	    \begin{quote}
	  
	    
	    \stanza[\smallbreak]
	\label{pv.3.216}\flagstanza{\tiny\textenglish{....3.216}}आप्त‚वादाविसंवाद‚सामान्याद‚नुमान‚ता ।&बुद्धेर‚ग‚त्याऽभिहिता निषिद्धाप्य‚स्य गोच‚रे ॥ २१६ ॥\&[\smallbreak]


	
	    \end{quote}
	  
	  \endgroup
	

	  \pstart \leavevmode% starting standard par
	\hphantom{.}त‚स्यैवं भूत‚स्या‚{\color{DodgerBlue3}‚प्त‚वाद}‚स्यादृष्ट‚व्य‚भिचार‚स्या‚{\color{DodgerBlue3}‚विस‚म्वाद‚सामान्यात्}\edtext{}{\edlabel{pvv.364-8}\label{pvv.364-8}\lemma{स्या}\Bfootnote{श‚क्य‚प‚रिच्छेदेऽविस‚म्वाद‚व‚त्प‚रोक्षेप्याप्त‚वादाल्लिङ्गादुत्प‚न्नायाः स‚म्वाद‚बुद्धेर‚{\tiny $_{lb}$}‚नुमान‚ता ।}} प्र‚त्य‚क्षानु‚{\tiny $_{lb}$}‚मानाग‚म्येप्य‚र्थे उत्प‚न्नाया ‚{\color{DodgerBlue3}‚बुद्धे}‚र‚विस‚म्वादा‚{\color{DodgerBlue3}‚द‚नुमान‚ता} चा चार्य दि ग्ना गे ना‚{\color{DodgerBlue3}‚भिहिता}‚{\tiny $_{lb}$}‚\leavevmode\ledsidenote{\textenglish{365/s}} ‚{\color{DodgerBlue3}‚ग‚त्या} । अत्य‚न्त‚प‚रोक्षेष्व‚र्थेषु दान‚हिंसाचेत‚नादिष्व‚र्थान‚र्थ‚श्र‚व‚णादाग‚म‚प्रामाण्य‚{\tiny $_{lb}$}‚म‚नाश्रित्य स्थातुम‚साम‚र्थ्यादेत‚द्भावे विरोधाभावाच्च स‚त्यां प्र‚वृत्तौ\edtext{}{\edlabel{pvv.365-1}\label{pvv.365-1}\lemma{वृत्तौ}\Bfootnote{न तु प्र‚माण‚ग‚म्य एवार्थे विसंवाद‚कात् ।}} व‚र‚{\tiny $_{5}$}‚मेवं ‚{\tiny $_{lb}$}‚प्र‚वृत्तिरित्य‚ग‚त्या\edtext{}{\edlabel{pvv.365-2}\label{pvv.365-2}\lemma{त्या}\Bfootnote{अतोन्य‚थाप‚रोक्षे प्र‚वृत्त्य‚संभ‚वात् ।}}ऽनुमान‚तोक्ता । न तु व‚स्तुतो व‚च‚नानाम‚र्थेषु नान्त‚रीय‚क‚त्वा‚{\tiny $_{lb}$}‚भावात् । (२१६)
	\pend% ending standard par
      \label{div_pvv.3.217}
	  
	% new div opening: depth here is 2
	

	  \pstart \leavevmode% starting standard par
	किम्वा\edtext{}{\edlabel{pvv.365-3}\label{pvv.365-3}\lemma{किम्वा}\Bfootnote{अथ‚वा ।}} (।)
	\pend% ending standard par
      
	  \bigskip
	  \begingroup
	
	    \large
	  
	    \begin{quote}
	  
	    
	    \stanza[\smallbreak]
	\label{pv.3.217}\flagstanza{\tiny\textenglish{....3.217}}हेयोपादेय‚त‚त्त्व‚स्य सोपाय‚स्य प्र‚सिद्धितः ।&प्र‚धानार्थाविस‚म्वादाद‚नुमान‚म्प‚र‚त्र वा ॥ २१७ ॥\&[\smallbreak]


	
	    \end{quote}
	  
	  \endgroup
	

	  \pstart \leavevmode% starting standard par
	\hphantom{.}‚{\color{DodgerBlue3}‚हेय}‚स्य दुःख‚स‚त्य‚स्य ‚{\color{DodgerBlue3}‚उपादे}‚य‚स्य निरोध‚स‚त्य‚स्य\edtext{}{\edlabel{pvv.365-4}\label{pvv.365-4}\lemma{स्य}\Bfootnote{त‚त्त्व‚म‚विप‚रीतं रूपं ।}} ‚{\color{DodgerBlue3}‚सोपाय‚स्य} य‚थाक्र‚मं स‚मुद‚य‚{\tiny $_{lb}$}‚स‚त्य‚स्य स‚मार्ग‚स‚त्य‚स्य चाग‚मोक्त‚व‚स्तुब‚ल‚प्र‚वृत्तेनानुमाने ‚{\color{DodgerBlue3}‚प्र‚सिद्धितो} निश्च‚यात् । ‚{\tiny $_{lb}$}‚स‚त्य‚च‚तुष्ट‚याधिग‚म‚स्य निर्व्वाण‚हेतुत्वेन ‚{\color{DodgerBlue3}‚प्र‚धानार्थ}‚स्या‚{\color{DodgerBlue3}‚विस‚म्वादात्} । ‚{\color{DodgerBlue3}‚प‚र‚त्रा}‚त्य‚न्त‚{\tiny $_{lb}$}‚प‚रोक्षेप्य‚र्थे भ‚ग‚{\tiny $_{6}$}‚ व‚द्व‚च‚नादुत्प‚न्नं ज्ञान‚म‚{\color{DodgerBlue3}‚नुमानं} युक्त‚मिति वा प‚क्षान्त‚रं । (२१७)
	\pend% ending standard par
      \label{div_pvv.3.218}
	  
	% new div opening: depth here is 2
	
	  \bigskip
	  \begingroup
	
	    \large
	  
	    \begin{quote}
	  
	    
	    \stanza[\smallbreak]
	\label{pv.3.218}\flagstanza{\tiny\textenglish{....3.218}}पुरुषातिश‚यापेक्षं य‚थार्थ‚म‚प‚रे विदुः ।&इष्टोय‚म‚र्थः प्र‚त्येतुं श‚क्यः सोतिश‚यो य‚दि ॥ २१८ ॥\&[\smallbreak]


	
	    \end{quote}
	  
	  \endgroup
	

	  \pstart \leavevmode% starting standard par
	\hphantom{.}‚{\color{DodgerBlue3}‚अप‚रे} नै या यि का द‚यः ‚{\color{DodgerBlue3}‚पुरुष}‚स्या‚{\color{DodgerBlue3}‚तिश‚यापेक्षं} य‚थाभूतार्थ‚द‚र्शि त‚दाख्यात् पुरुष‚{\tiny $_{lb}$}‚प्र‚णीतं व‚च‚नं ‚{\color{DodgerBlue3}‚य‚थार्थं} स‚त्यार्थं प्र‚तिजानीयुः (।)
	\pend% ending standard par
      

	  \pstart \leavevmode% starting standard par
	\hphantom{.}सिद्धान्त‚माह (।) पुरुषातिश‚य‚प्र‚णीतं व‚च‚न‚म प्र‚माण‚मिती‚{\color{DodgerBlue3}‚ष्टोऽय‚म‚र्थो य‚दि} पुरुषाणां ‚{\color{DodgerBlue3}‚सोतिश}‚यो ज्ञातुं श‚क्यः स्यात् ।
	\pend% ending standard par
      \label{div_pvv.3.219_3.220_3.221_3.222_3.223}
	  
	% new div opening: depth here is 2
	

	  \begin{center}%% label @type='head'
	\textbf{(१) पौरुषेय‚त्वे}
	\end{center}
	

	  \begin{center}%% label @type='head'
	\textbf{क. पुरुषातिश‚य‚प्र‚णीतं व‚च‚नं प्र‚माण‚म्}
	\end{center}
	

	  \pstart \leavevmode% starting standard par
	किन्तु (।)
	\pend% ending standard par
      
	  \bigskip
	  \begingroup
	
	    \large
	  
	    \begin{quote}
	  
	    
	    \stanza[\smallbreak]
	\label{pv.3.219}\flagstanza{\tiny\textenglish{....3.219}}अय‚मेवं न वेत्य‚न्य‚दोषानिर्दोष‚तापि वा ।&दुर्ल‚भ‚त्वात् प्र‚माणानां दुर्बोधेत्य‚प‚रे विदुः ॥ २१९ ॥\&[\smallbreak]


	
	    \end{quote}
	  
	  \endgroup
	

	  \pstart \leavevmode% starting standard par
	\hphantom{.}‚{\color{DodgerBlue3}‚अयं} पुमा‚{\color{DodgerBlue3}‚नेवं}‚दोष‚वान् ‚{\color{DodgerBlue3}‚न वा} निर्दोष इत्य‚न्य‚स्य‚{\tiny $_{7}$}‚ ‚{\color{DodgerBlue3}‚दोषा निर्दोष‚तापि वा प्र‚मा‚{\tiny $_{lb}$}‚णानां\edtext{}{\edlabel{pvv.365-5}\label{pvv.365-5}\lemma{णानां}\Bfootnote{अन्य‚गुण‚दोष‚निश्चाय‚कानां ।}} दुर्ल‚भ‚त्वाद् दुर्बोधे\edtext{}{\edlabel{pvv.365-6}\label{pvv.365-6}\lemma{दुर्बोधे}\Bfootnote{चैत‚सानाम‚तीन्द्रिय‚त्वात् नापि रागाद्य‚नुमेयाः काय‚व‚च‚सां प्र‚तिसंख्य‚यान्य‚थात्व‚स्य श‚क्य‚त्वात् ।}}त्य‚प‚रे} सौ ग ता ‚{\color{DodgerBlue3}‚विदुः} । दुर्ब्बोधेत्य‚न्य‚दोषा इत्य‚नेन\leavevmode\ledsidenote{\textenglish{72b/MA}} ‚{\tiny $_{lb}$}‚लिङ्ग\edtext{}{\edlabel{pvv.365-7}\label{pvv.365-7}\lemma{लिङ्ग}\Bfootnote{पुल्लिङ्ग‚ब‚हुव‚च‚नं कृत्वा ।}} व‚च‚न‚विप‚रिणामेन स‚म्ब‚न्ध‚नीयं ।
	\pend% ending standard par
      

	  \pstart \leavevmode% starting standard par
	\leavevmode\ledsidenote{\textenglish{366/s}}त‚त्किम‚श‚क्योच्छेदा दोषा नेत्याह ।
	\pend% ending standard par
      
	  \bigskip
	  \begingroup
	
	    \large
	  
	    \begin{quote}
	  
	    
	    \stanza[\smallbreak]
	\label{pv.3.220}\flagstanza{\tiny\textenglish{....3.220}}स‚र्व्वेषां स‚विप‚क्ष‚त्वान्निर्ह्नासातिश‚यं श्रितः ।&सात्मीभावात् त‚द‚भ्यासाद् हीयेर‚न्नास्र‚वाः क्व‚चित् ॥ २२० ॥\&[\smallbreak]


	
	    \end{quote}
	  
	  \endgroup
	

	  \pstart \leavevmode% starting standard par
	\hphantom{.}‚{\color{DodgerBlue3}‚स‚र्व्वेषा}‚मास्र‚वाणां रागादीनां‚{\tiny $_{1}$}‚प्र‚तिप‚क्ष‚संमुखीभावाभाव‚योर्न्निर्ह्रासातिश‚यावुप‚च‚{\tiny $_{lb}$}‚याप‚च‚यौ श्र‚य‚न्त इति ‚{\color{DodgerBlue3}‚निर्ह्रासातिश‚यं श्रित}‚स्तेषां ‚{\color{DodgerBlue3}‚स‚विप‚क्ष‚त्वात्} प्र‚तिप‚क्ष\edtext{}{\edlabel{pvv.366-1}\label{pvv.366-1}\lemma{क्ष}\Bfootnote{नैरात्म्य‚स्य ।}}स‚म्भ‚वात् ‚{\tiny $_{lb}$}‚त‚स्य प्र‚तिप‚क्ष‚स्या‚{\color{DodgerBlue3}‚भ्यासात् सात्मीभावादास्र‚वा} क्व1चिच्चित्त‚स‚न्ताने ‚{\color{DodgerBlue3}‚हीयेर‚न्नि}‚ति ‚{\tiny $_{lb}$}‚ख‚लु निर्दोष‚पुरुषाप‚लापः क्रिय‚ते किन्तु त‚द‚व‚धार‚णोपायो नास्तीत्युच्य‚ते । इच्छा‚{\tiny $_{lb}$}‚धीन‚स्य व्याहार‚स्यान्य‚थापि क‚र्त्तुं श‚क्य‚त्वात् । न त‚त‚स्त‚थार्थ‚निश्च‚यः ।
	\pend% ending standard par
      

	  \pstart \leavevmode% starting standard par
	अथ सात्मीय‚भूत‚प्र‚तिप‚क्ष‚स्य मार्गाभ्यासान्निर्दोष‚तायाम‚पि स‚त्यां विप‚क्षाभ्या‚{\tiny $_{lb}$}‚सात् पुन‚र्दोषोत्प‚त्तिरित्याह ।
	\pend% ending standard par
      
	  \bigskip
	  \begingroup
	
	    \large
	  
	    \begin{quote}
	  
	    
	    \stanza[\smallbreak]
	\label{pv.3.221}\flagstanza{\tiny\textenglish{....3.221}}निरुप‚द्र‚व‚भूतार्थ‚स्व‚भाव‚स्य विप‚र्य‚यैः ।&न बाधा य‚त्न‚व‚त्वेऽपि बुद्धेस्त‚त्प‚क्ष‚पात‚तः ॥ २२१ ॥\&[\smallbreak]


	
	    \end{quote}
	  
	  \endgroup
	

	  \pstart \leavevmode% starting standard par
	\hphantom{.}‚{\color{DodgerBlue3}‚निरुप‚द्र‚व}‚स्य दोष‚राशेरुद्वेज‚क‚स्य प्र‚हाणात् । ‚{\color{DodgerBlue3}‚भूतार्थ}‚स्य प्र‚माण‚प‚रिदृष्टा\edtext{}{\edlabel{pvv.366-2}\label{pvv.366-2}\lemma{रिदृष्टा}\Bfootnote{भूतार्थ‚त्वादेव मार्ग‚श्चित्त‚स्य स्व‚भाव उक्तः ।}}र्थ‚{\tiny $_{lb}$}‚विष‚य‚त्वात् । ‚{\color{DodgerBlue3}‚स्व\edtext{}{\edlabel{pvv.366-3}\label{pvv.366-3}\lemma{स्व}\Bfootnote{दोष‚प्र‚तिप‚क्ष‚स्य विप‚र्य‚यैः सो य‚त्राभूतार्था स्व‚भावैर्दोषैः न बाधार्थः श्रोत्रियः ‚{\tiny $_{lb}$}‚स‚न् कापालिको भ‚व‚ति त‚स्य पूर्व्वा घृणा य‚थाऽय‚त्न‚म‚श‚क्य‚निव‚र्त्त्या त‚द्व‚त् ।}}भाव}‚स्यानारोपित‚त्वात् (।) मार्ग‚सात्म्य‚स्य विप‚क्षेण ‚{\color{DodgerBlue3}‚न बाधा ‚{\tiny $_{lb}$}‚य‚त्न‚व‚त्वेपि} । य‚त्न एव ताव‚न्न स‚म्भ‚व‚ति विप‚क्षाभ्यासे दोष‚द‚र्श‚नात् । ‚{\color{DodgerBlue3}‚य‚त्न‚व‚त्वेपि} तु ‚{\color{DodgerBlue3}‚बुद्धेस्त‚त्र} मार्ग‚सात्म्येऽभिरुचिविष‚य‚त्वेन ‚{\color{DodgerBlue3}‚प‚क्ष‚पात‚तो\edtext{}{\edlabel{pvv.366-4}\label{pvv.366-4}\lemma{तो}\Bfootnote{ब‚हुमान‚तः ।}} न बाधा} । न हि र‚ज्वां ‚{\tiny $_{lb}$}‚निवृत्त‚स‚र्प्प‚भ्र‚मः स‚र्प्पं भाव‚यितुं य‚त‚ते क‚श्चित् (।) भूतार्थ‚स्य द‚र्श‚नात् ।
	\pend% ending standard par
      

	  \begin{center}%% label @type='head'
	\textbf{ख. स‚त्काय‚द‚र्श‚नं दोष‚कार‚ण‚म्}
	\end{center}
	

	  \pstart \leavevmode% starting standard par
	कः पुन‚र्दोषाणां हेतुर्य‚त्प्र‚हाणाद‚मी प्र‚हीय‚न्त इत्या3ह ।
	\pend% ending standard par
      
	  \bigskip
	  \begingroup
	
	    \large
	  
	    \begin{quote}
	  
	    
	    \stanza[\smallbreak]
	\label{pv.3.222a}\flagstanza{\tiny\textenglish{...3.222a}}स‚र्व्वासां दोष‚जातीनां जातिः स‚त्काय‚द‚र्श‚नात् ।\&[\smallbreak]


	
	    \end{quote}
	  
	  \endgroup
	

	  \pstart \leavevmode% starting standard par
	\hphantom{.}‚{\color{DodgerBlue3}‚स‚र्व्वासां दोष‚जातीनां} दोष‚प्र‚काराणां ‚{\color{DodgerBlue3}‚जाति}‚र्ज‚न्म ‚{\color{DodgerBlue3}‚स‚त्काय‚द‚र्श‚नादा}‚त्माभिनिवे‚{\tiny $_{lb}$}‚शात् ।
	\pend% ending standard par
      

	  \pstart \leavevmode% starting standard par
	न‚न्व‚विद्याहेतुकाः क्लेशा भ ग व तो क्तो इत्याह ।
	\pend% ending standard par
      
	  \bigskip
	  \begingroup
	
	    \large
	  
	    \begin{quote}
	  
	    
	    \stanza[\smallbreak]
	\label{pv.3.222b}\flagstanza{\tiny\textenglish{...3.222b}}साऽविद्या त‚त्र त‚त्स्नेह‚स्त‚स्माद् द्वेषादिस‚म्भ‚वः ॥ २२२ ॥\&[\smallbreak]


	
	    \end{quote}
	  
	  \endgroup
	\textsuperscript{\textenglish{367/s}}

	  \pstart \leavevmode% starting standard par
	\hphantom{.}‚{\color{DodgerBlue3}‚साऽविद्या} स‚त्काय‚द‚र्श‚न‚मेवाविद्याऽन्य‚त्रोच्य‚त इति नास्ति विरोधः । ‚{\color{DodgerBlue3}‚त‚त्र} स‚त्काय‚द‚र्श‚ने स‚ति तेष्वात्मीयेषु ‚{\color{DodgerBlue3}‚स्नेह‚स्त‚स्मा}‚दात्मीयेषु स्नेहात् त‚द‚प‚कारिषु ‚{\color{DodgerBlue3}‚द्वेषा‚{\tiny $_{lb}$}‚दी}‚नां ‚{\color{DodgerBlue3}‚स‚म्भ‚व} इति दोषोत्प‚त्तिक्र‚मः । अतो नैरात्म्य‚द‚र्श‚नं मा‚{\tiny $_{4}$}‚र्गो युक्तः स‚त्काय‚{\tiny $_{lb}$}‚दृष्टिप्र‚तिप‚क्ष‚त्वात् ।
	\pend% ending standard par
      

	  \pstart \leavevmode% starting standard par
	य‚त‚श्च स‚त्त्व‚दृष्टिर‚विद्या ।
	\pend% ending standard par
      
	  \bigskip
	  \begingroup
	
	    \large
	  
	    \begin{quote}
	  
	    
	    \stanza[\smallbreak]
	\label{pv.3.223a}\flagstanza{\tiny\textenglish{...3.223a}}मोहो निदानं दोषाणां अत्र एवाभिधीय‚ते ।&स‚त्काय‚दृष्टिर‚न्य‚त्र;\&[\smallbreak]


	
	    \end{quote}
	  
	  \endgroup
	

	  \pstart \leavevmode% starting standard par
	\hphantom{.}‚{\color{DodgerBlue3}‚अत एव मोहो}‚ऽविद्या ‚{\color{DodgerBlue3}‚दोषाणां निदान‚म‚भिधीय‚ते} । भ‚ग‚व‚ताऽविद्या हेतुकाः ‚{\tiny $_{lb}$}‚स‚र्वै क्लेशा इति पुन‚र‚न्य‚त्र प्र‚देशे स‚त्काय‚दृष्टिर्दोष‚निदान‚म ‚{\color{DodgerBlue3}‚भिधीय‚ते} ।
	\pend% ending standard par
      

	  \pstart \leavevmode% starting standard par
	न‚न्व‚न्येपीन्द्रिय‚विष‚या योनिशोम‚न‚स्काराद‚यो दोष‚हेत‚व‚स्त‚त्किम‚विद्या‚{\tiny $_{lb}$}‚स‚त्काय‚दृष्टी एवाभिहिते इत्याह ।
	\pend% ending standard par
      
	  \bigskip
	  \begingroup
	
	    \large
	  
	    \begin{quote}
	  
	    
	    \stanza[\smallbreak]
	\label{pv.3.223b}\flagstanza{\tiny\textenglish{...3.223b}}त‚त्प्र‚हाणे प्र‚हाण‚तः ॥ २२३ ॥\&[\smallbreak]


	
	    \end{quote}
	  
	  \endgroup
	

	  \pstart \leavevmode% starting standard par
	\hphantom{.}त‚स्य मोह‚स्य ‚{\color{DodgerBlue3}‚स‚त्काय‚दृष्टि‚{\tiny $_{5}$}‚}‚ल‚क्ष‚ण‚स्य ‚{\color{DodgerBlue3}‚प्र‚हाणे} दोषाणां ‚{\color{DodgerBlue3}‚प्र‚हाण‚तः} प्राधान्यात् ‚{\tiny $_{lb}$}‚स एवोक्तो नेत‚र इत्य‚र्थः ॥ (२२३)
	\pend% ending standard par
      
	  
	% new div opening: depth here is 1
	
\chapter*[{५. अपौरुषेय‚त्वे}]{५. अपौरुषेय‚त्वे}\label{div_pvv.3.224}
	  
	% new div opening: depth here is 2
	

	  \begin{center}%% label @type='head'
	\textbf{(१) वेद‚प्रामाण्य‚निरासः}
	\end{center}
	

	  \pstart \leavevmode% starting standard par
	वेद‚प्र‚माण्यं निराचिकीर्ष‚न् प‚र‚म‚त‚मुत्थाप‚य‚ति ।
	\pend% ending standard par
      
	  \bigskip
	  \begingroup
	
	    \large
	  
	    \begin{quote}
	  
	    
	    \stanza[\smallbreak]
	\label{pv.3.224}\flagstanza{\tiny\textenglish{....3.224}}गिराम्मिथ्यात्व‚हेतूनां दोषाणां पुरुषाश्र‚यात् ।&अपौरुषेयं स‚त्यार्थ‚मिति केचित् प्र‚च‚क्ष‚ते ॥ २२४ ॥\&[\smallbreak]


	
	    \end{quote}
	  
	  \endgroup
	

	  \pstart \leavevmode% starting standard par
	\hphantom{.}‚{\color{DodgerBlue3}‚गिरां} वाचां ‚{\color{DodgerBlue3}‚मिथ्यात्वं}\edtext{}{\edlabel{pvv.367-1}\label{pvv.367-1}\lemma{वाचां}\Bfootnote{मृष (ा)र्थ‚त्व‚स्य ये हेत‚वो दोषा रागाद‚य‚स्तेषां वाच‚श्च पुरुष आश्र‚य‚स्तैः ‚{\tiny $_{lb}$}‚पुरुष‚स्य प‚रिगृहीत‚त्वाद‚प्र‚माण‚त्वं (।) द्विधा श‚ब्दार्थो निस‚र्ग‚सिद्धो वेदादौ औपाधिकः ‚{\tiny $_{lb}$}‚पुरुषाधीनोन्य‚त्र । न मिथ्यात्वं वेदे पुरुष‚निवृत्तेः । न संश‚योऽप्र‚तिभासात् । नाज्ञानं ‚{\tiny $_{lb}$}‚वेदाद‚र्थ‚ग‚तेः (।) पुरुष‚कार‚णाभावान्मिथ्यात्व‚कार्याभाव‚सिद्धिः ।}}स्य ‚{\color{DodgerBlue3}‚हेतूनां दोषाणाम}‚ज्ञान‚विस‚म्वादाभिप्रायादीनां वा ‚{\tiny $_{lb}$}‚‚{\color{DodgerBlue3}‚पुरुष}‚स्या‚{\color{DodgerBlue3}‚श्र‚या}‚दाश्र‚य‚ण‚त्वात् ‚{\color{DodgerBlue3}‚अपौरुषेयं} वाक्यं मिथ्यात्व‚हेतोः पुरुष‚दोष‚स्याभावात् ‚{\tiny $_{lb}$}‚‚{\color{DodgerBlue3}‚स‚त्यार्थ‚मिति केचित्} जै मि नी याः ‚{\color{DodgerBlue3}‚प्र‚च‚क्ष‚ते} ॥‚{\tiny $_{6}$}‚ (२२४)
	\pend% ending standard par
      \label{div_pvv.3.225}
	  
	% new div opening: depth here is 2
	\textsuperscript{\textenglish{368/s}}
	  \bigskip
	  \begingroup
	
	    \large
	  
	    \begin{quote}
	  
	    
	    \stanza[\smallbreak]
	\label{pv.3.225}\flagstanza{\tiny\textenglish{....3.225}}गिरां स‚त्य‚त्व‚हेतूनां गुणानां पुरुषाश्र‚यात् ।&अपौरुषेयं मिथ्यार्थं किं नेत्य‚न्ये प्र‚च‚क्ष‚ते ॥ २२५ ॥\&[\smallbreak]


	
	    \end{quote}
	  
	  \endgroup
	

	  \pstart \leavevmode% starting standard par
	तानेव\edtext{}{\edlabel{pvv.368-1}\label{pvv.368-1}\lemma{तानेव}\Bfootnote{मी मां स कानेव शास्त्र‚कारः प‚र‚मुखेनाह ।}} प्र‚ति गिरां ‚{\color{DodgerBlue3}‚स‚त्य‚त्व‚स्य हेतूनां\edtext{}{\edlabel{pvv.368-2}\label{pvv.368-2}\lemma{हेतूनां}\Bfootnote{द‚यादीनां ।}}} द‚याध‚र्म‚प‚र‚त्वादीनां ‚{\color{DodgerBlue3}‚गुणानां पुरुष}‚स्या‚{\tiny $_{lb}$}‚‚{\color{DodgerBlue3}‚श्र‚याद‚पौरुषेयं} वाक्यं स‚त्य‚ता\edtext{}{\edlabel{pvv.368-3}\label{pvv.368-3}\lemma{ता}\Bfootnote{पुरुष‚निवृत्त्या स‚त्त्व‚कार‚ण‚निवृत्तेः कार्य‚स्यापि स‚त्त्व‚स्य निवृत्तितः श‚ब्दे ‚{\tiny $_{lb}$}‚स‚त्य‚त्व‚मिथ्यात्व‚योः पुरुषाय‚त्त‚त्वात् पुरुष‚निवृत्तौ स‚त्त्य‚व‚न्मिथ्यात्वं च स्यात् ।}}हेतोः पुरुष‚गुण‚स्याभावात् ‚{\color{DodgerBlue3}‚मिथ्यार्थं किं न} भ‚व‚तीत्य‚न्ये ‚{\tiny $_{lb}$}‚सौ ग ताः प्र‚च‚क्ष‚ते ॥ (२२५)
	\pend% ending standard par
      \label{div_pvv.3.226}
	  
	% new div opening: depth here is 2
	

	  \pstart \leavevmode% starting standard par
	किञ्च (।)
	\pend% ending standard par
      
	  \bigskip
	  \begingroup
	
	    \large
	  
	    \begin{quote}
	  
	    
	    \stanza[\smallbreak]
	\label{pv.3.226}\flagstanza{\tiny\textenglish{....3.226}}अर्थ‚ज्ञाप‚न‚हेतुर्हि स‚ङ्केतः पुरुषाश्र‚यः ।&गिराम‚पौरुषेय‚त्वेप्य‚तो मिथ्यात्व‚स‚म्भ‚वः ॥ २२६ ॥\&[\smallbreak]


	
	    \end{quote}
	  
	  \endgroup
	

	  \pstart \leavevmode% starting standard par
	स‚ङ्केत‚म‚न्त‚रेणापौरुषे\edtext{}{\edlabel{pvv.368-4}\label{pvv.368-4}\lemma{रेणापौरुषे}\Bfootnote{व‚स्तुतः पुंनिवृत्त्या स‚त्त्य‚मिथ्यार्थ‚त्व‚निवृत्तेरान‚र्थ‚क्याद‚नुत्प‚त्तिल‚क्ष‚ण‚मान‚र्थं‚{\tiny $_{lb}$}‚क्यं स‚ङ्केत‚काभावात् स्व‚भाव‚तोर्थ‚बोधानुप‚प‚त्तिः ।}}याद‚पि वाक्याद‚र्थ‚प्र‚तीतेर‚भावात् । ‚{\color{DodgerBlue3}‚अर्थ‚ज्ञाप‚न‚हेतु}‚रिह‚{\tiny $_{lb}$}‚‚{\color{DodgerBlue3}‚स‚ङ्केतः} स्वीक‚र्त‚व्यः । स च पुरुष‚कृत‚त्वात् ‚{\color{DodgerBlue3}‚पुरुषाश्र‚यः । अतः} संकेत‚स्य‚{\tiny $_{7}$}‚\leavevmode\ledsidenote{\textenglish{73a/MA}} पुरुषा‚{\tiny $_{lb}$}‚श्र‚य‚त्वात् ‚{\color{DodgerBlue3}‚गिराम‚पौरुषेय‚त्वेपि मिथ्यात्व‚स्य स‚म्भ‚वः} । संकेत‚व‚शेन वाचोऽर्थं ब्रुव‚ते । ‚{\tiny $_{lb}$}‚स च दोषाश्र‚येण पुरुषेण क्रिय‚त इति तासां न विसंवाद‚श‚ङ्कानिरासः । पौरुषेय‚{\tiny $_{lb}$}‚वाक्य‚व‚दिति व्य‚र्थ‚म‚पौरुषेय‚त्व‚क‚ल्प‚नं । (२२६)
	\pend% ending standard par
      \label{div_pvv.3.227}
	  
	% new div opening: depth here is 2
	

	  \begin{center}%% label @type='head'
	\textbf{क. अपौरुषेय‚त्वेप्य‚प्रामाण्य‚म्}
	\end{center}
	

	  \pstart \leavevmode% starting standard par
	अथ श‚ब्दार्थ\edtext{}{\edlabel{pvv.368-5}\label{pvv.368-5}\lemma{ब्दार्थ}\Bfootnote{श‚ब्दार्थानादितेव स‚म्ब‚न्धोनादिः (।) स च त्रिप्र‚माण‚कः श्रोतुर्ब्बाधायै ‚{\tiny $_{lb}$}‚केन श‚ब्दे प्र‚युक्ते पार्श्व‚स्थः प्र‚योक्तारं वाच्यं वाच‚क‚ञ्च बुध्य‚तेऽध्य‚क्षेण । श्रोतुश्च ‚{\tiny $_{lb}$}‚प्र‚तिप‚न्न‚त्वं प्र‚वृत्तिलिङ्गानुमानेन । अर्थ‚प्र‚तिप‚त्त्य‚न्य‚थानुप‚प‚त्त्या च श‚ब्दार्थाश्रितां ‚{\tiny $_{lb}$}‚वाच्य‚वाच‚क‚श‚क्तिम‚व‚ग‚च्छ‚त्य‚र्थाप‚त्त्येति त्रीणि प्र‚माणानि स‚म्ब‚न्ध‚स्य बोधे ।}}योः स‚म्ब‚न्धो न पौरुषेयः किन्तु स्वाभाविकः त‚तो न मिथ्यात्व‚{\tiny $_{lb}$}‚स‚म्भ‚वः । त‚दा (।)
	\pend% ending standard par
      
	  \bigskip
	  \begingroup
	
	    \large
	  
	    \begin{quote}
	  
	    
	    \stanza[\smallbreak]
	\label{pv.3.227a}\flagstanza{\tiny\textenglish{...3.227a}}स‚म्ब‚न्धापौरुषेय‚त्वे स्यात् प्र‚तीतिर‚संविदः ।\&[\smallbreak]


	
	    \end{quote}
	  
	  \endgroup
	

	  \pstart \leavevmode% starting standard par
	\hphantom{.}‚{\color{DodgerBlue3}‚स‚म्ब‚न्धापौरुषेय‚त्वे}‚पीष्य‚माणे1 ‚{\color{DodgerBlue3}‚स्याद}‚र्थानां ‚{\color{DodgerBlue3}‚प्र‚तीतिर‚संविदो}‚ऽविद्य‚मान‚स‚ङ्केत‚{\tiny $_{lb}$}‚प्र‚तीतेः पुंसः । न चेच्छ‚ब्दार्थ‚योः सांकेतिको वाच्य‚वाच‚क‚तास‚म्ब‚न्धः किन्तु स्वाभा‚{\tiny $_{lb}$}‚विकः । त‚दाऽगृहीत‚श (?स) ङ्केतोपि श्रुताच्छ‚ब्दाद‚र्थं प्र‚तिप‚द्येतेति ।
	\pend% ending standard par
      \textsuperscript{\textenglish{p369/s}}

	  \pstart \leavevmode% starting standard par
	अथ संकेतात् स‚तोपि त‚स्य स‚म्ब‚न्ध‚स्याभिव्य‚क्ति (ः) प्र‚दीपादिव‚द् घ‚टादेर‚तो ‚{\tiny $_{lb}$}‚नागृहीत‚व्य‚ञ्ज‚क‚स्य व्य‚ङ्ग्य‚प्र‚तीतिः । त‚दा (।)
	\pend% ending standard par
      
	  \bigskip
	  \begingroup
	
	    \large
	  
	    \begin{quote}
	  
	    
	    \stanza[\smallbreak]
	\label{pv.3.227b}\flagstanza{\tiny\textenglish{...3.227b}}संकेतात् त‚द‚भिव्य‚क्ताव‚स‚म‚र्थान्य‚क‚ल्प‚ना ॥ २२७ ॥\&[\smallbreak]


	
	    \end{quote}
	  
	  \endgroup
	

	  \pstart \leavevmode% starting standard par
	\hphantom{.}‚{\color{DodgerBlue3}‚संङ्केतात् त}‚स्य स‚म्ब‚न्ध‚स्या‚{\color{DodgerBlue3}‚भिव्य‚क्ता}‚विष्य‚मा\edtext{}{\edlabel{pvv.369-1}\label{pvv.369-1}\lemma{मा}\Bfootnote{प्र‚तीत्य‚न्य‚थानुप‚प‚त्त्या स‚म्ब‚न्ध‚क‚ल्प‚न‚म‚त्य‚र्थेपि चेन्न प्र‚तीतिः किन्त‚त्क‚ल्प‚न‚या संकेत एवान्व‚य‚व्य‚तिरेकात् ।}}णायां संकेता‚{\color{DodgerBlue3}‚द‚न्य}‚स्य स‚म्ब‚न्ध‚स्य ‚{\tiny $_{lb}$}‚‚{\color{DodgerBlue3}‚क‚ल्प‚नाऽस‚म‚र्था} स‚म्ब‚न्ध‚व्य‚व‚स्थाप‚नाय । संकेतादेव वाच्य‚वाच‚क‚भाव‚स्य क‚ल्पित‚स्य ‚{\tiny $_{lb}$}‚घ‚ट‚मान‚त्वात् ह‚स्त‚संज्ञादेरिवार्थ‚प्र‚तिपाद‚न‚स्य । (२२७)
	\pend% ending standard par
      \label{div_pvv.3.228}
	  
	% new div opening: depth here is 2
	

	  \pstart \leavevmode% starting standard par
	किञ्च (।) वाचा किमेकेनार्थेन स‚ह वाच्य‚वाच‚क‚स‚म्ब‚न्धः । अथानेकैः । ‚{\tiny $_{lb}$}‚त‚त्र\edtext{}{\edlabel{pvv.369-2}\label{pvv.369-2}\lemma{त्र}\Bfootnote{नित्ये स‚म्ब‚न्ध‚दोष‚माह ।}} (।)
	\pend% ending standard par
      
	  \bigskip
	  \begingroup
	
	    \large
	  
	    \begin{quote}
	  
	    
	    \stanza[\smallbreak]
	\label{pv.3.228a}\flagstanza{\tiny\textenglish{...3.228a}}गिरामेकार्थ‚निय‚मे न स्याद‚र्थान्त‚रे ग‚तिः ।\&[\smallbreak]


	
	    \end{quote}
	  
	  \endgroup
	

	  \pstart \leavevmode% starting standard par
	\hphantom{.}‚{\color{DodgerBlue3}‚गिरामेक}‚स्मिन्न‚र्थे वाच‚क‚त‚या ‚{\color{DodgerBlue3}‚निय‚मे} स‚ति संकेत‚व‚शाद‚न्य‚{\color{DodgerBlue3}‚त्रार्थे न स्याद् ग‚तिः} । ‚{\tiny $_{lb}$}‚दृश्य‚ते च वि‚{\tiny $_{3}$}‚व‚क्षातोऽनेकार्थाभिधान‚म्\edtext{}{\edlabel{pvv.369-3}\label{pvv.369-3}\lemma{म्}\Bfootnote{अनेकार्थ‚प्र‚तिपाद‚न‚स्य द‚र्श‚नात् स‚र्व्वे स‚र्व्वार्थ‚वाच‚काश्चेत् ।}} ॥
	\pend% ending standard par
      
	  \bigskip
	  \begingroup
	
	    \large
	  
	    \begin{quote}
	  
	    
	    \stanza[\smallbreak]
	\label{pv.3.228b}\flagstanza{\tiny\textenglish{...3.228b}}अनेकार्थाभिस‚म्ब‚न्धे विरुद्ध‚व्य‚क्तिस‚म्भ‚वः ॥ २२८ ॥\&[\smallbreak]


	
	    \end{quote}
	  
	  \endgroup
	

	  \pstart \leavevmode% starting standard par
	\hphantom{.}‚{\color{DodgerBlue3}‚अने}‚कैर‚{\color{DodgerBlue3}‚र्थै}‚र्व्वाच‚क‚त्वा‚{\color{DodgerBlue3}‚भिस‚म्ब‚न्धे विरुद्ध}‚स्यार्थ‚स्य ‚{\color{DodgerBlue3}‚व्य‚क्तेः\edtext{}{\edlabel{pvv.369-4}\label{pvv.369-4}\lemma{क्तेः}\Bfootnote{अभिम‚त एव स‚म‚य इत्य‚निय‚मात् स‚र्व्व‚वाच‚क‚त्वे किं स्व‚र्ग‚साध‚न एवाग्नि‚{\tiny $_{lb}$}‚होत्रादिसंकेतः किम्वा त‚द्विरुद्धे बुद्धिमान्द्यादिति संश‚यात् ।}}} प्र‚तीतेः ‚{\color{DodgerBlue3}‚स‚म्भ‚वः} स्यात् । अग्निष्टोमः स्व‚र्ग‚स्य साध‚न‚मिति विप‚र्य‚योप्य‚व‚सीयेत\edtext{}{\edlabel{pvv.369-5}\label{pvv.369-5}\lemma{सीयेत}\Bfootnote{द्विधा श‚ब्द‚विष‚यः साक्षाज्जातिस्त‚ल्ल‚क्षिता च व्य‚क्तिरिति व्य‚क्त्या स‚म्ब‚न्धे स‚म्ब‚न्धीत्यादि ।}}(।) त‚त‚श्चा‚{\tiny $_{lb}$}‚प्र‚वृत्तिरेव स्यात् स्व‚र्गीर्थिनः । (२२८)
	\pend% ending standard par
      \label{div_pvv.3.229}
	  
	% new div opening: depth here is 2
	

	  \pstart \leavevmode% starting standard par
	अथानेकार्थाभिधाय्य‚पि श‚ब्दः पुरुषेण स‚ङ्केताद‚भिम‚तार्थाभिधायित्वेन निय‚{\tiny $_{lb}$}‚म्य‚ते त‚दा (।)
	\pend% ending standard par
      
	  \bigskip
	  \begingroup
	
	    \large
	  
	    \begin{quote}
	  
	    
	    \stanza[\smallbreak]
	\label{pv.3.229}\flagstanza{\tiny\textenglish{....3.229}}अपौरुषेय‚तायाञ्च व्य‚र्था स्यात् प‚रिक‚ल्प‚ना ।&वाच्य‚श्च हेतुर्भिन्नानां स‚म्ब‚न्ध‚स्य व्य‚व‚स्थितेः ॥ २२९ ॥\&[\smallbreak]


	
	    \end{quote}
	  
	  \endgroup
	

	  \pstart \leavevmode% starting standard par
	\hphantom{.}‚{\color{DodgerBlue3}‚अपौरुषेय‚तायाञ्च व्य‚र्था प‚रिक‚ल्प‚ना स्यात्} । त‚द‚भ्युप‚ग‚{\tiny $_{4}$}‚मेपि पुरुष‚स्वात‚न्त्र्या‚{\tiny $_{lb}$}‚भ्युप‚ग‚मात् । त‚दार्थेभ्यो ‚{\color{DodgerBlue3}‚भिन्नानां} श‚ब्दानां तैः स‚ह ‚{\color{DodgerBlue3}‚स‚म्ब‚न्ध‚स्य व्य‚व}‚स्थितेः\edtext{}{\edlabel{pvv.369-6}\label{pvv.369-6}\lemma{स्थितेः}\Bfootnote{व्य‚व‚स्थाय (ा)ः ष‚ष्ठी ।}} । हेतुश्च ‚{\tiny $_{lb}$}‚\leavevmode\ledsidenote{\textenglish{370/s}} वाच्यो\edtext{}{\edlabel{pvv.370-1}\label{pvv.370-1}\lemma{वाच्यो}\Bfootnote{श‚ब्दार्थ‚स‚म्ब‚न्ध‚वादिना ।}}येनाव्य‚भिचारः । न हि श‚ब्दार्थ‚योस्तादात्म्यं भेदात् । नापि त‚दुत्प‚त्तिर‚र्थ‚{\tiny $_{lb}$}‚म‚न्त‚रेणापि विव‚क्षातः श‚ब्दोत्प‚त्तेः । अन्य‚था चाव्य‚भिचाराभावात् । (२२९)
	\pend% ending standard par
      \label{div_pvv.3.230}
	  
	% new div opening: depth here is 2
	

	  \pstart \leavevmode% starting standard par
	उक्त‚म‚र्थं संगृह्णान्नाह ।
	\pend% ending standard par
      
	  \bigskip
	  \begingroup
	
	    \large
	  
	    \begin{quote}
	  
	    
	    \stanza[\smallbreak]
	\label{pv.3.230}\flagstanza{\tiny\textenglish{....3.230}}असंस्कार्य‚त‚या पुंभिः स‚र्व्व‚था स्यान्निर‚र्थ‚ता ।&संस्कारोप‚ग‚मे मुख्यं ग‚ज‚स्नान‚निभं भ‚वेत् ॥ २३० ॥\&[\smallbreak]


	
	    \end{quote}
	  
	  \endgroup
	

	  \pstart \leavevmode% starting standard par
	य‚दि स‚ङ्केत‚निर‚पेक्षाणां स्व‚त एव वाच‚क‚त्वं श‚ब्दानां त‚दा\edtext{}{\edlabel{pvv.370-2}\label{pvv.370-2}\lemma{दा}\Bfootnote{स‚त्य‚भिथ्यार्थ‚त्व‚योः पुरुष‚संस्कार‚प्र‚तिब‚द्ध‚त्वात् ।}} ‚{\color{DodgerBlue3}‚पुंभि‚{\tiny $_{5}$}‚र‚संस्कार्य‚{\tiny $_{lb}$}‚त‚या} स‚ङ्केत‚द्वारेण निय‚म्य‚त‚या ‚{\color{DodgerBlue3}‚स‚र्व्व‚था निर‚र्थ‚ता स्यात्} । पुरुष‚स‚ङ्केत‚निर‚पेक्षाच्छ‚{\tiny $_{lb}$}‚ब्दाद‚र्थंप्र‚तीतेर‚भावात् । एत‚द्दोष‚भ‚यात् ‚{\color{DodgerBlue3}‚संस्कार‚स्योप‚ग‚मे} स्वीकारे इद‚म‚पौरुषेय‚त्वं ‚{\tiny $_{lb}$}‚मुख्य‚म‚नुप‚च‚रितं ‚{\color{DodgerBlue3}‚ग‚ज‚स्नान\edtext{}{\edlabel{pvv.370-3}\label{pvv.370-3}\lemma{स्नान}\Bfootnote{मिथ्यार्थ‚तापि स्यात् । जातिचोद‚नेपि न प्र‚योज‚नं निर्ल्लोठित‚मेत‚द‚न्यापोह ‚{\tiny $_{lb}$}‚अपि प्र‚व‚र्त्तेत पुमान् विज्ञायार्थ‚क्रियाक्ष‚मान् । \cref{pv.3.92} ।}}म्भ‚वेत्} । ग‚जो हि स्नाने प‚ङ्क‚म‚प‚नीय पुन‚स्तेनात्मानं ‚{\tiny $_{lb}$}‚लिम्प‚ति । त‚थाऽपौरुषेय‚त्वं स‚म्ब‚न्ध‚स्य स्वीकृत्यापि पुनः स‚ङ्केते पुरुषा‚{\tiny $_{6}$}‚पेक्षेति ‚{\tiny $_{lb}$}‚व्य‚क्तं साम्यं । (२३०)
	\pend% ending standard par
      \label{div_pvv.3.231_3.232_3.233_3.234_3.235_3.236ab}
	  
	% new div opening: depth here is 2
	

	  \begin{center}%% label @type='head'
	\textbf{ख. स‚म्ब‚न्ध‚चिन्ता}
	\end{center}
	

	  \begin{center}%% label @type='head'
	\textbf{(क) स‚म्ब‚न्ध्य‚नित्य‚त्वे स‚म्ब‚न्ध‚नित्य‚ता}
	\end{center}
	
	  \bigskip
	  \begingroup
	
	    \large
	  
	    \begin{quote}
	  
	    
	    \stanza[\smallbreak]
	\label{pv.3.231a}\flagstanza{\tiny\textenglish{...3.231a}}स‚म्ब‚न्धिनाम‚नित्य‚त्वान्न संब‚न्धेस्ति नित्य‚ता ।\&[\smallbreak]


	
	    \end{quote}
	  
	  \endgroup
	

	  \pstart \leavevmode% starting standard par
	\hphantom{.}त‚था ‚{\color{DodgerBlue3}‚स‚म्ब‚न्धिनाम}‚र्थाना‚{\color{DodgerBlue3}‚म‚नित्य‚त्वात् संब‚न्धे नित्य‚ता नास्ति} न ह्याश्र‚या‚{\tiny $_{lb}$}‚पाये भ‚व‚त्याश्रितं ।
	\pend% ending standard par
      

	  \pstart \leavevmode% starting standard par
	किञ्च (।)
	\pend% ending standard par
      
	  \bigskip
	  \begingroup
	
	    \large
	  
	    \begin{quote}
	  
	    
	    \stanza[\smallbreak]
	\label{pv.3.231b}\flagstanza{\tiny\textenglish{...3.231b}}नित्य‚स्यानुप‚कार्य‚त्वाद‚कुर्वाण‚श्च नाश्र‚यः ॥ २३१ ॥\&[\smallbreak]


	
	    \end{quote}
	  
	  \endgroup
	
	  \bigskip
	  \begingroup
	
	    \large
	  
	    \begin{quote}
	  
	    
	    \stanza[\smallbreak]
	\label{pv.3.232a}\flagstanza{\tiny\textenglish{...3.232a}}अर्थैर‚तः स श‚ब्दानां संस्कार्यः पुरुषैर्धिया ।\&[\smallbreak]


	
	    \end{quote}
	  
	  \endgroup
	

	  \pstart \leavevmode% starting standard par
	\hphantom{.}‚{\color{DodgerBlue3}‚नित्य‚स्य} स‚म्ब‚न्ध‚स्या‚{\color{DodgerBlue3}‚नुप‚कार्य‚त्वात् । अकुर्व्वाणो}‚ऽनुप‚कुर्व्वाणः श‚ब्दोऽर्थ‚श्चा‚{\color{DodgerBlue3}‚श्र‚यो ‚{\tiny $_{lb}$}‚न} युक्तः । य‚तः स्वाभाविक‚स‚म्ब‚न्धानुप‚प‚त्ति‚{\color{DodgerBlue3}‚र‚तः} स‚म्ब‚न्धो‚{\color{DodgerBlue3}‚ऽर्थैः} स‚ह ‚{\color{DodgerBlue3}‚श‚ब्दानां}\leavevmode\ledsidenote{\textenglish{73b/MA}} पुरुषै‚{\tiny $_{lb}$}‚र्व्य‚व‚ह‚र्तृभिर‚र्थ‚प्र‚तिपाद‚नाभिप्रायान्व‚य‚व्य‚तिरेकानुविधान‚मा‚{\tiny $_{7}$}‚श्रित्य ‚{\color{DodgerBlue3}‚धिया} क‚ल्पिक‚या\edtext{}{\edlabel{pvv.370-4}\label{pvv.370-4}\lemma{या}\Bfootnote{श‚ब्दार्थाव‚संब‚न्धिनाव‚पि स‚म्ब‚द्धौ पुरुष‚स्य प्र‚तिभासेते विक‚ल्प‚बुद्धौ ‚{\tiny $_{lb}$}‚अनादिव्य‚व‚हाराभ्यासाद‚र्थ‚कार्यः श‚ब्दास्त‚द्भानुविधायित्वादित्य‚ध्य‚व‚साय‚व‚शात् ‚{\tiny $_{lb}$}‚स‚म्ब‚न्ध‚व्य‚व‚स्था ।}} ‚{\tiny $_{lb}$}‚‚{\color{DodgerBlue3}‚संस्कार्यो} व्य‚व‚स्थाप्यः ।
	\pend% ending standard par
      \textsuperscript{\textenglish{371/s}}

	  \pstart \leavevmode% starting standard par
	अथानित्य एव स‚म्ब‚न्ध‚स्त‚दा स‚म्ब‚न्धिनां नाशे स‚म्ब‚न्ध‚स्य न‚ष्ट‚त्वान्निर‚र्थ‚कः ‚{\tiny $_{lb}$}‚श‚ब्दः स्यात् ।
	\pend% ending standard par
      

	  \pstart \leavevmode% starting standard par
	अथ‚वा (।)
	\pend% ending standard par
      
	  \bigskip
	  \begingroup
	
	    \large
	  
	    \begin{quote}
	  
	    
	    \stanza[\smallbreak]
	\label{pv.3.232b}\flagstanza{\tiny\textenglish{...3.232b}}अर्थैरेव स‚होत्पादे न स्व‚भाव‚विप‚र्य‚यः ॥ २३२ ॥\&[\smallbreak]


	
	    \end{quote}
	  
	  \endgroup
	
	  \bigskip
	  \begingroup
	
	    \large
	  
	    \begin{quote}
	  
	    
	    \stanza[\smallbreak]
	\label{pv.3.233a}\flagstanza{\tiny\textenglish{...3.233a}}श‚ब्देषु युक्तः ;\&[\smallbreak]


	
	    \end{quote}
	  
	  \endgroup
	

	  \pstart \leavevmode% starting standard par
	\hphantom{.}वाच्यै‚{\color{DodgerBlue3}‚र‚र्थै}‚रेव ‚{\color{DodgerBlue3}‚स‚ह} स‚म्ब‚न्ध‚स्यो‚{\color{DodgerBlue3}‚त्पाद} इष्य‚ते (।) त‚दोत्पाद इष्य‚माणेपि ‚{\tiny $_{lb}$}‚पूर्व्व‚म‚र्थेन स‚ह स‚म्ब‚न्ध‚स्य विन‚ष्ट‚त्वात् । अर्थ‚स‚म्ब‚न्ध‚र‚हितात्म‚सु ‚{\color{DodgerBlue3}‚श‚ब्दे}‚षु स्व‚भाव‚स्य ‚{\tiny $_{lb}$}‚स‚म्ब‚न्ध‚विक‚ल‚स्य ‚{\color{DodgerBlue3}‚विप‚र्य‚यः} स‚म्ब‚न्ध‚योगी ‚{\color{DodgerBlue3}‚न युक्तः}\edtext{}{\edlabel{pvv.371-1}\label{pvv.371-1}\lemma{योगी}\Bfootnote{अर्थेन स‚होत्प‚न्न‚स्यानुप‚कारिणि श‚ब्देनाश्र‚य‚णाच्च ।}} । न हि नित्य‚स्य पूर्व्वाप‚रैक‚{\tiny $_{lb}$}‚स्व‚भाव‚{\tiny $_{1}$}‚स्यान्य‚थात्वं युक्तं । अन्य‚था नित्य‚ताहानिप्र‚स‚ङ्गात् ।
	\pend% ending standard par
      

	  \begin{center}%% label @type='head'
	\textbf{(ख) स‚म्ब‚न्धः क‚ल्पितः}
	\end{center}
	

	  \pstart \leavevmode% starting standard par
	अस्म‚न्म‚ते तु (।)
	\pend% ending standard par
      
	  \bigskip
	  \begingroup
	
	    \large
	  
	    \begin{quote}
	  
	    
	    \stanza[\smallbreak]
	\label{pv.3.233b}\flagstanza{\tiny\textenglish{...3.233b}}स‚म्ब‚न्धे नायं दोषो विक‚ल्पिते ।\&[\smallbreak]


	
	    \end{quote}
	  
	  \endgroup
	

	  \pstart \leavevmode% starting standard par
	विक‚ल्पिते क‚ल्प‚नानिर्मिते स‚म्ब‚न्धेऽयं स्व‚भावान्य‚त्व‚प्र‚स‚ङ्ग‚दोषो न भ‚व‚ति । ‚{\tiny $_{lb}$}‚न हि क‚ल्प‚नाक्लृप्तो ध‚र्मः स्व‚भावं व‚स्तुतः स्पृश‚ति ।
	\pend% ending standard par
      
	  \bigskip
	  \begingroup
	
	    \large
	  
	    \begin{quote}
	  
	    
	    \stanza[\smallbreak]
	\label{pv.3.233c}\flagstanza{\tiny\textenglish{...3.233c}}नित्य‚त्वादाश्र‚यापायेप्य‚नाशो य‚दि स‚म्म‚तः ॥ २३३ ॥\&[\smallbreak]


	
	    \end{quote}
	  
	  \endgroup
	
	  \bigskip
	  \begingroup
	
	    \large
	  
	    \begin{quote}
	  
	    
	    \stanza[\smallbreak]
	\label{pv.3.234a}\flagstanza{\tiny\textenglish{...3.234a}}नित्येष्वाश्र‚य‚साम‚र्थ्यं किं येनेष्टः स चाश्र‚यः ।\&[\smallbreak]


	
	    \end{quote}
	  
	  \endgroup
	

	  \pstart \leavevmode% starting standard par
	स‚म्ब‚न्ध‚स्य\edtext{}{\edlabel{pvv.371-2}\label{pvv.371-2}\lemma{स्य}\Bfootnote{स‚म्ब‚न्धिनाम‚नित्य‚त्वादित्यादौ प‚रः ।}} ‚{\color{DodgerBlue3}‚नित्य‚त्वात् आश्र‚य}‚स्य वाच्य‚स्या‚{\color{DodgerBlue3}‚पाये}‚प्य‚{\color{DodgerBlue3}‚नाशो} य‚दि जाते\edtext{}{\edlabel{pvv.371-3}\label{pvv.371-3}\lemma{जाते}\Bfootnote{नित्य‚त्वादाश्र‚य‚नाशेप्य‚नाश‚व‚त् ।}}रिव ‚{\tiny $_{lb}$}‚‚{\color{DodgerBlue3}‚स‚म्म‚तः} त‚दा ‚{\color{DodgerBlue3}‚नित्येषु} जाति\edtext{}{\edlabel{pvv.371-4}\label{pvv.371-4}\lemma{जाति}\Bfootnote{प्र‚सिद्धिमात्रं त‚न्निर्व्व‚स्तुकं ।}}स‚म्ब‚न्धादिष्वा‚{\color{DodgerBlue3}‚श्र‚य}‚स्य वाच्य‚स्या वाच‚क‚स्य ‚{\color{DodgerBlue3}‚च किं साम‚र्थ्य-} मुप‚कार‚विशेषाधाय‚कं‚{\tiny $_{2}$}‚ ‚{\color{DodgerBlue3}‚येन} साम‚र्थ्येन स वाच्य‚दिरा‚{\color{DodgerBlue3}‚श्र‚य इष्टः} । न ह्य‚नुप‚कार्य‚{\tiny $_{lb}$}‚माश्रित‚म‚तिप्र‚स‚ङ्गात् । नित्य‚स्य चोप‚कारास‚म्भ‚वः । भेदाभेद‚क‚ल्प‚नायाम‚युक्त‚त्वात्
	\pend% ending standard par
      

	  \pstart \leavevmode% starting standard par
	अथ नित्य‚स्यापि जातिस‚म्ब‚न्धादेराश्र‚येणाभिव्य‚क्तिल‚क्ष‚ण उप‚कारः क्रिय‚ते । ‚{\tiny $_{lb}$}‚न चाव्य‚क्तिहेतुः कार‚को दीपादिव‚त् घ‚टादेरित्याह ।
	\pend% ending standard par
      
	  \bigskip
	  \begingroup
	
	    \large
	  
	    \begin{quote}
	  
	    
	    \stanza[\smallbreak]
	\label{pv.3.234b}\flagstanza{\tiny\textenglish{...3.234b}}ज्ञानोत्पादेन हेतूनां स‚म्ब‚न्धात् स‚ह‚कारिणाम् ॥ २३४ ॥\&[\smallbreak]


	
	    \end{quote}
	  
	  \endgroup
	
	  \bigskip
	  \begingroup
	
	    \large
	  
	    \begin{quote}
	  
	    
	    \stanza[\smallbreak]
	\label{pv.3.235a}\flagstanza{\tiny\textenglish{...3.235a}}त‚दुत्पाद‚न‚योग्य‚त्वेनोत्प‚त्तिर्व्य‚क्तिरिष्य‚ते ।&घ‚टादिष्व‚पि युक्तिज्ञैः ;\&[\smallbreak]


	
	    \end{quote}
	  
	  \endgroup
	\textsuperscript{\textenglish{372/s}}

	  \pstart \leavevmode% starting standard par
	\hphantom{.}‚{\color{DodgerBlue3}‚ज्ञानोत्पादेन हेतूनां} दीपादीनां ‚{\color{DodgerBlue3}‚स‚ह‚कारिणां\edtext{}{\edlabel{pvv.372-1}\label{pvv.372-1}\lemma{कारिणां}\Bfootnote{योग्य‚देशाव‚स्थानात् ।}}} स‚म्ब‚न्धात् ‚{\color{DodgerBlue3}‚त‚दुत्पाद‚न‚योग्य‚त्वेन}‚{\tiny $_{3}$}‚ ‚{\tiny $_{lb}$}‚ज्ञानोत्पाद‚न‚स‚म‚र्थ‚त्वेनोत्प‚त्ति‚{\color{DodgerBlue3}‚र्घ‚टादिष्व‚पि} भावेषु ‚{\color{DodgerBlue3}‚युक्तिज्ञै}‚र्न्याय‚विद्भि‚{\color{DodgerBlue3}‚र्व्य‚क्तिरिष्य‚ते-} ऽन्य‚था ज्ञानोत्पाद‚न‚योग्य‚स्य स्व‚भाव‚स्यानुत्प‚त्तौ ज्ञानोत्पाद‚नं न स्यात् (।)
	\pend% ending standard par
      
	  \bigskip
	  \begingroup
	
	    \large
	  
	    \begin{quote}
	  
	    
	    \stanza[\smallbreak]
	\label{pv.3.235b}\flagstanza{\tiny\textenglish{...3.235b}}अविशेषेऽविकारिणाम् ॥ २३५ ॥\&[\smallbreak]


	
	    \end{quote}
	  
	  \endgroup
	
	  \bigskip
	  \begingroup
	
	    \large
	  
	    \begin{quote}
	  
	    
	    \stanza[\smallbreak]
	\label{pv.3.236a}\flagstanza{\tiny\textenglish{...3.236a}}व्य‚ञ्ज‚कैः स्वैः कुतः कोर्थो व्य‚क्तास्तैस्ते य‚तो म‚ताः ।\&[\smallbreak]


	
	    \end{quote}
	  
	  \endgroup
	

	  \pstart \leavevmode% starting standard par
	\hphantom{.}नित्यानां जातिस‚म्ब‚न्धादीनाम‚{\color{DodgerBlue3}‚विकारिणां} कुत‚श्चिद‚विशेषे विशेषास‚म्भ‚वे ‚{\tiny $_{lb}$}‚स्वैर्व्य‚ञ्ज‚कैराश्र‚याभिम‚तैः ‚{\color{DodgerBlue3}‚कोर्थः} स्व‚भावान्य‚थात्वादिः ‚{\color{DodgerBlue3}‚कुतो} न क‚श्चित् । ‚{\color{DodgerBlue3}‚य‚तो}‚{\tiny $_{lb}$}‚ऽर्थात् कृतात् तै‚{\color{DodgerBlue3}‚र्व्य‚ञ्ज}‚कैस्ते ‚{\tiny $_{4}$}‚ जात्याद‚यो ‚{\color{DodgerBlue3}‚व्य‚क्ता म‚ताः} ।
	\pend% ending standard par
      \label{div_pvv.3.236bc_3.237_3.238_3.239_3.240_3.241_3.242_3.243_3.244_3.245_3.246}
	  
	% new div opening: depth here is 2
	

	  \begin{center}%% label @type='head'
	\textbf{(ग) भेदाभेद‚व्य‚व‚स्थातोऽपि स‚म्ब‚न्ध‚स्याव‚स्तुत्व‚म्}
	\end{center}
	

	  \pstart \leavevmode% starting standard par
	किञ्च \edtext{}{\edlabel{pvv.372-2}\label{pvv.372-2}\lemma{किञ्च}\Bfootnote{न च स‚म्ब‚न्ध‚स्त्रिप्र‚माण‚क इति द‚र्श‚य‚न्नाह व‚र्ण्णा न वाच‚कास्तेन वाच्य‚वाच‚क‚स‚म्ब‚न्ध‚स्यावृत्तिर्व‚र्ण्णेषु निर‚र्थ‚क‚त्वात् त‚द्वृतौ स‚म्ब‚न्ध‚स्य वाच‚काङ्ग‚त्वं स्यात् ।}} (।)
	\pend% ending standard par
      
	  \bigskip
	  \begingroup
	
	    \large
	  
	    \begin{quote}
	  
	    
	    \stanza[\smallbreak]
	\label{pv.3.236b}\flagstanza{\tiny\textenglish{...3.236b}}स‚म्ब‚न्ध‚स्य च व‚स्तुत्वे स्याद् भेदाद् बुद्धिचित्र‚ता ॥ २३६ ॥\&[\smallbreak]


	
	    \end{quote}
	  
	  \endgroup
	
	  \bigskip
	  \begingroup
	
	    \large
	  
	    \begin{quote}
	  
	    
	    \stanza[\smallbreak]
	\label{pv.3.237a}\flagstanza{\tiny\textenglish{...3.237a}}ताभ्याम‚भेदे तावेव नातोन्या व‚स्तुनो ग‚तिः ।\&[\smallbreak]


	
	    \end{quote}
	  
	  \endgroup
	

	  \pstart \leavevmode% starting standard par
	\hphantom{.}य‚दि ‚{\color{DodgerBlue3}‚स‚म्ब‚न्ध‚स्य व‚स्तुत्व}‚न्त‚दा व‚स्तुत्वे स‚ति भेदोऽभेदो वाऽभ्युप‚ग‚न्त‚व्यः । ‚{\tiny $_{lb}$}‚त‚त्र ‚{\color{DodgerBlue3}‚भेदाद्\edtext{}{\edlabel{pvv.372-3}\label{pvv.372-3}\lemma{भेदाद्}\Bfootnote{स‚म्ब‚न्धिभ्यां स‚म्ब‚न्ध‚स्य भेदात् ।}} बुद्धेश्चित्र‚ता} स्यात् (।) वाच्य‚वाच‚कौ स‚म्ब‚न्ध‚श्चेति त्रित‚यं दृश्येत\edtext{}{\edlabel{pvv.372-4}\label{pvv.372-4}\lemma{दृश्येत}\Bfootnote{य‚त्स‚म्ब‚न्धाभ्यां(?)भेदेन नोप‚ल‚भ्य‚ते त‚त्त‚तो नान्य‚त् य‚द् दृश्यं नोप‚ल‚भ्य‚ते त‚न्नास्ति ।}} । ‚{\tiny $_{lb}$}‚न चेक्ष्य‚ते । अथ द्वितीयः प‚क्षः त‚दा ‚{\color{DodgerBlue3}‚ताभ्याम‚भेदे} स‚म्ब‚न्ध‚स्य ‚{\color{DodgerBlue3}‚तौ} वाच्य‚वाच‚का‚{\color{DodgerBlue3}‚वेव} स्यातां न तु स‚म्ब‚न्धो नाम क‚श्चित् । अथ स‚म्ब‚न्धो न भिन्नो नाप्य‚भिन्नः\edtext{}{\edlabel{pvv.372-5}\label{pvv.372-5}\lemma{भिन्नः}\Bfootnote{त‚त्त्वान्य‚त्त्व‚र‚हितः स‚म्ब‚न्ध इत्य‚त्राह ।}} । ‚{\tiny $_{lb}$}‚‚{\color{DodgerBlue3}‚अतो} भेदाभे‚{\tiny $_{5}$}‚दाभ्या‚{\color{DodgerBlue3}‚म‚न्या} व‚स्तुनो ‚{\color{DodgerBlue3}‚ग‚ति}‚र्नास्ति । अन्योन्य‚व‚च्छेदात्म‚क‚त्वाद‚न‚{\tiny $_{lb}$}‚यो राश्य‚न्त‚रास‚म्भ‚वान्नान्यः प्र‚कारोस्ति व‚स्तुनः ।
	\pend% ending standard par
      

	  \pstart \leavevmode% starting standard par
	त‚स्माद् (।)
	\pend% ending standard par
      
	  \bigskip
	  \begingroup
	
	    \large
	  
	    \begin{quote}
	  
	    
	    \stanza[\smallbreak]
	\label{pv.3.237b}\flagstanza{\tiny\textenglish{...3.237b}}भिन्न‚त्वाद् व‚स्तुरूप‚स्य स‚म्ब‚न्धः क‚ल्प‚नाकृतः ॥ २३७ ॥\&[\smallbreak]


	
	    \end{quote}
	  
	  \endgroup
	
	  \bigskip
	  \begingroup
	
	    \large
	  
	    \begin{quote}
	  
	    
	    \stanza[\smallbreak]
	\label{pv.3.238a}\flagstanza{\tiny\textenglish{...3.238a}}स‚द्द्र‚व्यं स्यात् प‚राधीनं स‚म्ब‚न्धोन्य‚स्य वा क‚थ‚म् ।\&[\smallbreak]


	
	    \end{quote}
	  
	  \endgroup
	\textsuperscript{\textenglish{373/s}}

	  \pstart \leavevmode% starting standard par
	\hphantom{.}भिन्न‚त्वाद् ‚{\color{DodgerBlue3}‚व‚स्तुनोः} स‚म्ब‚न्धिनोः ‚{\color{DodgerBlue3}‚रूप‚स्य स‚म्ब‚न्धः} श्लेष‚ल‚क्ष‚णः ‚{\color{DodgerBlue3}‚क‚ल्प‚न‚या कृतो} न वास्त‚वः । अन्य‚था ‚{\color{DodgerBlue3}‚स‚द्द्र‚व्यं} स‚म्ब‚न्धाख्यं ‚{\color{DodgerBlue3}‚प‚राधीनं} स‚म्ब‚न्धाय‚त्तं ‚{\color{DodgerBlue3}‚क‚थं स्यात् ।\edtext{\textsuperscript{*}}{\edlabel{pvv.373-1}\label{pvv.373-1}\lemma{*}\Bfootnote{एतेनार्थान्त‚र‚त्वे स‚म्ब‚न्ध‚स्याश्रित‚त्वं श्लेष‚ञ्च ग‚तः ।}} ‚{\tiny $_{lb}$}‚क‚थ‚म्वा} भिन्न‚योः स‚म्ब‚न्धिनोः श्लेष‚ल‚क्ष‚णः ‚{\color{DodgerBlue3}‚स‚म्ब‚न्धः} प‚र‚स्प‚र‚म‚मिश्र‚स्व‚भाव‚त्वात् ‚{\tiny $_{lb}$}‚स‚र्व्व‚स्य त‚था‚{\tiny $_{6}$}‚पि स‚म्ब‚न्धेतिप्र‚स‚ङ्गात् ।
	\pend% ending standard par
      

	  \begin{center}%% label @type='head'
	\textbf{(६) व‚र्ण‚प‚दादिषु स‚म्ब‚न्ध‚स्यास‚द्भावः}
	\end{center}
	

	  \pstart \leavevmode% starting standard par
	किञ्च (।)
	\pend% ending standard par
      
	  \bigskip
	  \begingroup
	
	    \large
	  
	    \begin{quote}
	  
	    
	    \stanza[\smallbreak]
	\label{pv.3.238b}\flagstanza{\tiny\textenglish{...3.238b}}व‚र्ण्णा निर‚र्थ‚काः स‚न्तः;\&[\smallbreak]


	
	    \end{quote}
	  
	  \endgroup
	

	  \pstart \leavevmode% starting standard par
	\hphantom{.}अयं स‚म्ब‚न्धो व‚र्त्त‚मानो व‚र्ण्णेषु प‚दादिषु वा व‚र्त्त‚त । त‚त्र ‚{\color{DodgerBlue3}‚व‚र्ण्णाः स‚न्तो\edtext{}{\edlabel{pvv.373-2}\label{pvv.373-2}\lemma{न्तो}\Bfootnote{व‚स्तुस‚न्तो विद्य‚माना अपि ।}} ‚{\tiny $_{lb}$}‚निर‚र्थ‚काः} प्र‚त्येकं तेषाम‚र्थ‚प्र‚तिपाद‚क‚त्वाभावात् । नानाप्र‚योक्तृप्र‚युक्तेभ्योऽर्था‚{\tiny $_{lb}$}‚प्र‚तिप‚त्तेश्च व्य‚तिक्र‚म‚प्र‚युक्तेभ्य‚श्च स\edtext{}{\edlabel{pvv.373-3}\label{pvv.373-3}\lemma{स}\Bfootnote{साहित्याभावेन । नानुमानाल्लिङ्गाभावान्न हि केचिद् दृष्टान्ते स‚म्ब‚न्धं कार्याऽर्थ‚प्र‚तीतिः स‚म्ब‚न्ध‚स्यातीन्द्रिय‚त्वेनेन्द्रियादिव‚त् साध‚नापेक्ष‚णात् ।}}रो र‚स इत्यादिभ्य\edtext{}{\edlabel{pvv.373-4}\label{pvv.373-4}\lemma{इत्यादिभ्य}\Bfootnote{क्र‚म‚विशेषेणैक‚प्र‚योक्तृप्र‚युक्ता व‚र्ण्णा एव वाच‚का इति न दोष‚श्चेत् । न क्र‚म‚स्य नार्थान्त‚त्वेन य‚द् रूपं स‚रे त‚द्रूपं र‚सेपीति तुल्या प्र‚तीतिः स्यात् ।}}स्तुल्या स्यात् ‚{\tiny $_{lb}$}‚प्र‚तिप‚त्तिस्त‚त्स‚मुदाय‚स्य चास‚म्भ‚वः क्र‚मेणोप‚ल‚म्भात् । न च स‚मुदायो नाम ‚{\tiny $_{lb}$}‚स‚मुदायिभ्यो भिन्नोऽनुप‚ल‚म्भ‚बाधित‚{\tiny $_{7}$}‚त्वात् । तेषाञ्च वाच‚क‚त्वादेक‚स्माद‚पि\leavevmode\ledsidenote{\textenglish{74a/MA}} ‚{\tiny $_{lb}$}‚प्र‚तीतिः स्यात् । प्र‚त्येकं न चेद् वाच‚काः स‚मुदितेभ्योपि तेभ्यो न स्यात् प्र‚ती‚{\tiny $_{lb}$}‚तिस्त‚दाप्य‚न्य‚स्याभावात् ।
	\pend% ending standard par
      

	  \pstart \leavevmode% starting standard par
	अथ क्र‚मेण व‚र्ण्णेषु गृहीतेषु त‚त्संस्कार‚स‚हायेनाध्य‚क्षेण गृहीताद‚न्त्य‚व‚र्ण्णा‚{\tiny $_{lb}$}‚द‚र्थ‚प्र‚तीतिः\edtext{}{\edlabel{pvv.373-5}\label{pvv.373-5}\lemma{तीतिः}\Bfootnote{न व‚र्ण्णानुभ‚वाहित‚संस्कार‚स्य व‚र्ण्णेष्वेव स्मृतिहेतुत्वान्नार्थे न हि ग‚वानुभ‚वाहित‚संस्कारोऽश्व‚स्म‚र‚ण‚माद‚ध‚ति । दृष्ट‚त्वादित्य‚पि न संकेतं विनाऽस‚त्यात् । संकेत‚श्च सामान्य‚विष‚यो न व‚र्ण्ण‚स्व‚ल‚क्ष‚णे ।}} । त‚त्किम‚न्त्य एव व‚र्ण्णो वाच‚को नान्ये । त‚था तेद् व्य‚र्थं ‚{\tiny $_{lb}$}‚तेषामुच्चार‚णं । स‚व्वेषु प्र‚तीतेष्व‚र्थ‚प्र‚तीतिरिति चेत् । किम‚न्त्य‚व‚र्ण्ण‚ग्राहिक‚या ‚{\tiny $_{lb}$}‚बुद्ध्या1 स‚र्व्व‚त्र ग्र‚ह‚णं । अन्यान्य‚बुद्ध्यैव चेत् । ताः किं बुद्ध‚योऽन्त्य‚व‚र्ण्ण‚बुद्धिकाले ‚{\tiny $_{lb}$}‚भ‚व‚न्ति । येन त‚दार्थ‚प्र‚तीतिरुच्य‚ते । अन्यान्य‚काल एवेति चेत् । य‚दि ताभिर्व्वाच‚का ‚{\tiny $_{lb}$}‚व‚र्ण्णा गृह्य‚न्ते (।) एकैक‚व‚र्ण्ण‚ग्र‚ह‚णेप्य‚र्थ‚प्र‚तीतिः स्यात् । वाच‚केषु स‚र्व्व‚षु गृहीतेषु ‚{\tiny $_{lb}$}‚प्र‚तीतिरिति चेत् । त‚दा तु न प्र‚त्येकं वाच‚क‚स्त‚द‚तिरिक्त‚श्च स‚मुदायो नास्ति ॥
	\pend% ending standard par
      \textsuperscript{\textenglish{374/s}}

	  \pstart \leavevmode% starting standard par
	प्र‚त्येकं स‚म‚र्थाः स्थितिबीजाद‚योऽङ्कुर‚ज‚न‚ने न च केव‚ला‚{\tiny $_{2}$}‚ज‚न‚य‚न्तीति चेत् । ‚{\tiny $_{lb}$}‚ये स‚म‚र्था न तेषां क्ष‚णिक‚त्वात् पृथ‚ग्भाव इत्य‚स‚मानं । स‚मुदिता एव तु स‚म‚र्थाः । ‚{\tiny $_{lb}$}‚न त्वेवं व‚र्ण्णानां क्वापि स‚मुदायः क्र‚मोप‚ल‚भ्य‚त्वात् । त‚दा चेद् प्र‚तिपाद‚का अवाच‚का ‚{\tiny $_{lb}$}‚एव । पूर्व्व‚व‚र्ण्ण‚ग्र‚ह‚ण‚संस्कारेपि किम‚न्त्य‚व‚र्ण्ण‚बुद्धौ स‚र्व्वे प्र‚तिभान्ति न वा । न ‚{\tiny $_{lb}$}‚ताव‚दुप‚ल‚भ्य‚न्ते त‚त‚स्त‚दुप‚द‚र्श‚म‚पि व्य‚र्थं स‚र्व्वेषु क्र‚मात् प्र‚तीतेषु स्मृतिः स‚मुदा‚{\tiny $_{3}$}‚य‚{\tiny $_{lb}$}‚विष‚या भ‚व‚तीति चेत् । किम्व‚र्ण्णानां स‚मुदायोस्ति प्र‚तीतो वा यः स्म‚र्य‚ते केव‚लं ‚{\tiny $_{lb}$}‚क‚ल्प्य‚ते । क‚ल्पित‚स्य वाच‚क‚त्वाभ्युप‚ग‚मे न विवादः । त‚स्मान्न व‚र्ण्णे स‚म्ब‚न्ध‚{\tiny $_{lb}$}‚वृत्तिः ।
	\pend% ending standard par
      

	  \pstart \leavevmode% starting standard par
	प‚द‚वाक्यादिषु त‚र्हि स्यादिति चेत् ।
	\pend% ending standard par
      
	  \bigskip
	  \begingroup
	
	    \large
	  
	    \begin{quote}
	  
	    
	    \stanza[\smallbreak]
	\label{pv.3.238c}\flagstanza{\tiny\textenglish{...3.238c}}प‚दादिप‚रिक‚ल्पित‚म् ॥ २३८ ॥\&[\smallbreak]


	
	    \end{quote}
	  
	  \endgroup
	
	  \bigskip
	  \begingroup
	
	    \large
	  
	    \begin{quote}
	  
	    
	    \stanza[\smallbreak]
	\label{pv.3.239a}\flagstanza{\tiny\textenglish{...3.239a}}अव‚स्तुनि क‚थं वृत्तिः स‚म्ब‚न्ध‚स्यास्य व‚स्तुनः ।\&[\smallbreak]


	
	    \end{quote}
	  
	  \endgroup
	

	  \pstart \leavevmode% starting standard par
	\edtext{\textsuperscript{*}}{\edlabel{pvv.374-1}\label{pvv.374-1}\lemma{*}\Bfootnote{वैयाक‚र‚णानां व‚र्ण्णादिव्य‚तिरिक्तं प‚दादि निर‚स्य‚ते ।}} न हि क्र‚मोच्चारितेभ्यो व‚र्ण्णेभ्यो व्य‚तिरिक्तं प‚दादिक‚मुप‚ल‚भ्य‚ते (।) ‚{\tiny $_{lb}$}‚केव‚लं क‚ल्प‚नाबुद्ध्या\edtext{}{\edlabel{pvv.374-2}\label{pvv.374-2}\lemma{नाबुद्ध्या}\Bfootnote{भिन्न‚व‚र्ण्णानुभ‚वात् क‚थ‚मेक‚प‚दाद्य‚व‚भासौ विक‚ल्पः । अस्ति चेत्य‚नुभ‚वोस्तीत्याह क्र‚म‚व‚र्ण्णानुभ‚व‚दृष्ट‚भाविम‚नोविज्ञानं व‚र्ण्णान्य‚दादित्वेनैक‚स्व‚भावान‚ध्य‚व‚स्य‚ति मिथ्याविभ्र‚मोऽनादिः ।}} क्र‚मोच्चारितानां व‚र्ण्णानां स‚मुदायः क‚ल्पितः प‚दं । ‚{\tiny $_{lb}$}‚प‚दानाञ्च स‚मुदायः क‚{\tiny $_{4}$}‚ल्पितो वाक्य‚मुच्य‚ते । त‚च्च क‚ल्पित‚त्वाद‚व‚स्तु । ‚{\color{DodgerBlue3}‚अव‚स्तुनि ‚{\tiny $_{lb}$}‚स‚म्ब‚न्ध‚स्य व‚स्तुनः क‚थ‚म्वृत्तिः} । न हि श‚श‚वि‚{\color{DodgerBlue3}‚शा} (?षा)ण‚स्य नीलादिर्द्ध‚र्मो ‚{\tiny $_{lb}$}‚युक्तः । त‚देवं न स‚म्ब‚न्धो नित्योऽनित्यो वा युक्त इति स्थितं ॥
	\pend% ending standard par
      

	  \begin{center}%% label @type='head'
	\textbf{ग. नापौरुषेय‚ता}
	\end{center}
	
	  \bigskip
	  \begingroup
	
	    \large
	  
	    \begin{quote}
	  
	    
	    \stanza[\smallbreak]
	\label{pv.3.239b}\flagstanza{\tiny\textenglish{...3.239b}}अपौरुषेय‚तापीष्टा क‚र्त्तृणाम‚स्मृतेः किल ॥ २३९ ॥\&[\smallbreak]


	
	    \end{quote}
	  
	  \endgroup
	
	  \bigskip
	  \begingroup
	
	    \large
	  
	    \begin{quote}
	  
	    
	    \stanza[\smallbreak]
	\label{pv.3.240a}\flagstanza{\tiny\textenglish{...3.240a}}स‚न्त्य‚स्याप्य‚नुव‚क्तार इति धिग्व्याप‚कं त‚मः ।\&[\smallbreak]


	
	    \end{quote}
	  
	  \endgroup
	

	  \pstart \leavevmode% starting standard par
	\hphantom{.}वेद‚वाक्यानाम‚{\color{DodgerBlue3}‚पौरुषेय‚तापि} केन‚चि न्मी मां स क\edtext{}{\edlabel{pvv.374-3}\label{pvv.374-3}\lemma{क}\Bfootnote{ब‚हूनां जीर्ण्ण‚कूपादीनां क‚र्त्ता न स्म‚र्य‚ते न च ताव‚ताऽक‚र्त्तृता । इति व्य‚भिचाराद‚युक्तं लिङ्गं जैमि(नि)नोक्तं ।}} प्र‚व‚रेणेष्टा ‚{\color{DodgerBlue3}‚क‚र्त्तृणाम‚स्मृतेः} । ‚{\tiny $_{lb}$}‚लिङ्गात् ‚{\color{DodgerBlue3}‚किल} । अक्ष‚मायां किल श‚ब्दः । अस्याप्य‚र्थ‚स्य न्यायाद् दूर‚मायात‚स्या‚{\tiny $_{lb}$}‚\leavevmode\ledsidenote{\textenglish{375/s}} नुव‚क्तारः प‚ण्डितंम‚न्याः\edtext{}{\edlabel{pvv.375-1}\label{pvv.375-1}\lemma{न्याः}\Bfootnote{कुमारिलाद‚यः ।}} स‚न्ति । त‚स्यै‚{\tiny $_{5}$}‚व\edtext{}{\edlabel{pvv.375-2}\label{pvv.375-2}\lemma{व}\Bfootnote{यः क‚र्त्तुर‚स्म‚र‚णाद‚पौरुषेय‚तामाह जै मि नि (ः) एव‚मेत‚दिति निःकृप‚माक्रान्तं ज‚ग‚त् येन त‚म‚सा । अज्ञान‚स्यैव धिग्वादो न प्राणिनः ।}} ताव‚दीदृशं\edtext{}{\edlabel{pvv.375-3}\label{pvv.375-3}\lemma{दीदृशं}\Bfootnote{य‚त इदं साध‚न‚म‚सिद्ध‚म‚नेक‚ञ्चेत्याह ।}} प्र‚ज्ञास्ख‚लितं क‚थ‚म्वृ‚{\tiny $_{lb}$}‚त्त‚मिति स‚विस्म‚यानुक‚म्प‚न्न‚श्चेतः । त‚द‚प्य‚प‚रेऽनुव‚द‚न्ती\edtext{}{\edlabel{pvv.375-4}\label{pvv.375-4}\lemma{न्ती}\Bfootnote{अतिस्थूलं स‚ह विस्म‚येनानुक‚म्प‚या च व‚र्त्त‚ते श्रुत‚व‚तोपीदृश‚म‚विद्याविल‚सित‚मिति स‚विस्म‚यं । गाढेनाविद्याब‚न्धेन स‚त्वाः पीड्य‚न्त इति सानुक‚म्पं अप‚रे त‚न्म‚तानुगाः ।}}ति निर्द‚याक्रान्त‚भुव‚नं ‚{\tiny $_{lb}$}‚धिग‚व्याप‚क‚न्त‚मः ।‚{\tiny $_{6}$}‚ त‚था हि क‚र्त्तुः स्म‚र‚ण‚म‚सिद्धं स्म‚र‚न्ति सौग‚ता म‚न्त्राणां क‚र्तॄन् ‚{\tiny $_{lb}$}‚अष्ट‚कादीन् । का णा\edtext{}{\edlabel{pvv.375-5}\label{pvv.375-5}\lemma{णा}\Bfootnote{वैशेषिकाः ।}}दा श्च वि धा ता रं । मिथ्या त‚त्स्म‚र‚ण‚ञ्चेत् । कु मा र‚{\tiny $_{lb}$}‚स म्भ वा देर‚पि का लि दा सादिक‚र्त्तृस्म‚र‚णं मिथ्येति त‚द‚प्य‚पौरुषेयं । त‚त्रैकं स्म‚र‚ण‚{\tiny $_{lb}$}‚म‚प्र‚माण‚म‚न्य‚च्चान्य‚थेति नात्र विभाग‚{\tiny $_{6}$}‚कार‚णं । ब‚हूनां संप्र‚तिप‚त्तिविप्र‚तिप‚त्त‚य‚श्च ‚{\tiny $_{lb}$}‚न प्र‚माणेत‚र‚ल‚क्ष‚णे स‚म्वाद‚स‚त्वेन सिध्य‚तः । त‚त‚श्चास्मृत‚क‚र्त्तृक‚मित्य‚श‚क्य‚निश्च‚यं ‚{\tiny $_{lb}$}‚स‚न्दिग्ध‚विप‚क्ष‚व्यावृत्तिक‚त्वात्\edtext{}{\edlabel{pvv.375-6}\label{pvv.375-6}\lemma{त्वात्}\Bfootnote{साध‚नान्त‚रं निर‚स्य‚ति । वेद‚स्याध्य‚य‚नं स‚र्व्वं गुर्व्व‚ध्य‚य‚न‚पूर्व्व‚क‚म् । विद्या (?वेदा) ध्य‚य‚न‚वाच्य‚त्वाद‚धुनाध्य‚य‚नं य‚थे ति ‚{\tiny $_{lb}$}‚\href{http://sarit.indology.info/?cref=śv.949}{(श्लोक‚वार्तिके ९४९)} दूष‚य‚न्नाह ।}} ।
	\pend% ending standard par
      

	  \begin{center}%% label @type='head'
	\textbf{घ. न नित्य‚ता}
	\end{center}
	

	  \begin{center}%% label @type='head'
	\textbf{(क) a गुर्व‚ध्य‚य‚न‚पूर्व‚क‚त्वाद‚पि न}
	\end{center}
	
	  \bigskip
	  \begingroup
	
	    \large
	  
	    \begin{quote}
	  
	    
	    \stanza[\smallbreak]
	\label{pv.3.240b}\flagstanza{\tiny\textenglish{...3.240b}}य‚थाय‚म‚न्य‚तोऽश्रुत्वा नेमं व‚र्ण‚प‚द‚क्र‚म‚म् ॥ २४० ॥\&[\smallbreak]


	
	    \end{quote}
	  
	  \endgroup
	
	  \bigskip
	  \begingroup
	
	    \large
	  
	    \begin{quote}
	  
	    
	    \stanza[\smallbreak]
	\label{pv.3.241a}\flagstanza{\tiny\textenglish{...3.241a}}व‚क्तुं स‚म‚र्थः पुरुष‚स्त‚थान्योपीति क‚श्च‚न ।\&[\smallbreak]


	
	    \end{quote}
	  
	  \endgroup
	

	  \pstart \leavevmode% starting standard par
	\hphantom{.}य‚दापि वेदान‚धीयानो य‚थाय‚मिदानीन्त‚नो माण‚व‚कोऽ‚{\color{DodgerBlue3}‚न्य‚त} उपाध्याया‚{\color{DodgerBlue3}‚द‚श्रुत्वा ‚{\tiny $_{lb}$}‚इम}‚मुच्य‚मानं ‚{\color{DodgerBlue3}‚व‚र्ण्ण‚प‚द‚योः क्र‚म}‚मानुपूर्व्वीविशिष्टां वेदाख्यां ‚{\color{DodgerBlue3}‚व‚क्तुम‚स‚म‚र्थो} वेद‚क‚पाठ‚क‚त्वात् । ‚{\color{DodgerBlue3}‚त‚थाऽन्य} उपाध्याय‚स्त‚दुपाध्या‚{\tiny $_{7}$}‚योपीत्य‚नादिरेष क्र‚म इति\leavevmode\ledsidenote{\textenglish{74b/MA}} ‚{\tiny $_{lb}$}‚क‚श्चि ‚{\color{DodgerBlue3}‚न्मी मां स कः} । स‚र्व्व‚स्यैव वेद‚पाठः प‚रोप‚देशादिति नित्य‚तैव वेदानां ।
	\pend% ending standard par
      

	  \pstart \leavevmode% starting standard par
	त‚त्राप्याह ।
	\pend% ending standard par
      
	  \bigskip
	  \begingroup
	
	    \large
	  
	    \begin{quote}
	  
	    
	    \stanza[\smallbreak]
	\label{pv.3.241b}\flagstanza{\tiny\textenglish{...3.241b}}\leavevmode\ledsidenote{\textenglish{376/s}}अन्यो वा र‚चितो ग्र‚न्थः स‚म्प्र‚दायाद् ऋते प‚रैः ॥ २४१ ॥\&[\smallbreak]


	
	    \end{quote}
	  
	  \endgroup
	
	  \bigskip
	  \begingroup
	
	    \large
	  
	    \begin{quote}
	  
	    
	    \stanza[\smallbreak]
	\label{pv.3.242a}\flagstanza{\tiny\textenglish{...3.242a}}दृष्टः कोऽभिहितो येन सोप्येवं नानुमीय‚ते ।\&[\smallbreak]


	
	    \end{quote}
	  
	  \endgroup
	

	  \pstart \leavevmode% starting standard par
	\hphantom{.}‚{\color{DodgerBlue3}‚अन्यो वा} वेदादित‚रः काव्यादि‚{\color{DodgerBlue3}‚र्ग्र‚न्थो र‚चितः} क‚विप्र‚भृतिभिः ‚{\color{DodgerBlue3}‚संप्र‚दायाद् ऋते} उप‚देशाद्विना प‚रैर‚ध्येतृभिर‚{\color{DodgerBlue3}‚भिहितः को दृष्टो} न क‚श्चित् (।) ‚{\color{DodgerBlue3}‚येन} प‚रोप‚देशेऽ‚{\tiny $_{lb}$}‚स‚त्य‚श‚क्याध्य‚य‚न‚त्वेन ‚{\color{DodgerBlue3}‚सोपि} काव्यादि‚{\color{DodgerBlue3}‚रेवं} नित्यं ‚{\color{DodgerBlue3}‚नानुमी}‚य‚ते\edtext{}{\edlabel{pvv.376-1}\label{pvv.376-1}\lemma{ते}\Bfootnote{विनोप‚देशं पाठाश‚क्तेः काव्येपि स‚त्त्वात् ।}} । त‚त्रापि हेतुर‚यं सिद्ध ‚{\tiny $_{lb}$}‚एव । अथ पुरु‚{\tiny $_{1}$}‚षेण त‚त्क‚र‚णाविरोधात् स‚न्दिग्ध‚व्य‚तिरेक‚ताऽस्य हेतोस्त‚दा वेदेपि ‚{\tiny $_{lb}$}‚दुष्ट‚त्व‚म‚स्या (ः) क‚थं निवार्यं ।
	\pend% ending standard par
      

	  \pstart \leavevmode% starting standard par
	अपि च (।)
	\pend% ending standard par
      
	  \bigskip
	  \begingroup
	
	    \large
	  
	    \begin{quote}
	  
	    
	    \stanza[\smallbreak]
	\label{pv.3.242b}\flagstanza{\tiny\textenglish{...3.242b}}य‚ज्जातीयो य‚तः सिद्धः सोऽविशिष्टोग्निकाष्ठ‚व‚त् ॥ २४२ ॥\&[\smallbreak]


	
	    \end{quote}
	  
	  \endgroup
	
	  \bigskip
	  \begingroup
	
	    \large
	  
	    \begin{quote}
	  
	    
	    \stanza[\smallbreak]
	\label{pv.3.243a}\flagstanza{\tiny\textenglish{...3.243a}}अदृष्ट‚हेतुर‚प्य‚न्य‚स्त‚द्भ‚वः संप्र‚तीय‚ते ।\&[\smallbreak]


	
	    \end{quote}
	  
	  \endgroup
	

	  \pstart \leavevmode% starting standard par
	\hphantom{.}‚{\color{DodgerBlue3}‚य‚ज्जातीयो} य‚द्द्र‚व्य‚स‚मान‚जातीयो ‚{\color{DodgerBlue3}‚यः} प‚दार्थो ‚{\color{DodgerBlue3}‚य‚तो} हेतोः ‚{\color{DodgerBlue3}‚सिद्धो}‚ऽन्व‚य‚व्य‚ति‚{\tiny $_{lb}$}‚रेकाभ्यां निश्चितः ‚{\color{DodgerBlue3}‚स} त‚ज्जातीय‚त्वेना‚{\color{DodgerBlue3}‚विशिष्टो अन्योऽदृष्ट‚हेतुर‚पि} त‚द्धेतुक‚त्वेन ‚{\tiny $_{lb}$}‚संप्र‚तीय‚ते (।) किमिव । ‚{\color{DodgerBlue3}‚अग्निकाष्ठ‚व‚त्} । काष्ठ‚कार्य‚त्वेन व‚ह्नेर्निश्चित‚त्वात् (।) ‚{\tiny $_{lb}$}‚(।) व‚ह्निद‚र्श‚नाद‚दृष्ट‚म‚पि काष्ठ‚म‚नुमीय‚ते\edtext{}{\edlabel{pvv.376-2}\label{pvv.376-2}\lemma{ते}\Bfootnote{अपौरुषेय‚त्व‚प्र‚तिज्ञाया अनुमान‚बाधोक्तानेन ।}} । न च\edtext{}{\edlabel{pvv.376-3}\label{pvv.376-3}\lemma{च}\Bfootnote{काष्ठ‚हेतुकोय‚म‚ग्निरिति ।}} वैदिक‚{\tiny $_{2}$}‚पौरुषेय‚वाक्यानां ‚{\tiny $_{lb}$}‚क‚श्चिद्भेदः (।) स‚र्व्वेषां दुर्भ‚ण‚त्वादीनां म‚न्त्रादिसामार्थ्यानाञ्च साधार‚ण‚त्वात् ।
	\pend% ending standard par
      

	  \pstart \leavevmode% starting standard par
	इदानीम्वेद‚क‚र‚ण‚स‚म‚र्थ‚पुरुषाद‚र्श‚नं भा र तादिष्व‚पि स‚मानं\edtext{}{\edlabel{pvv.376-4}\label{pvv.376-4}\lemma{मानं}\Bfootnote{इदानीन्त‚द‚कार‚णात् ।}} । त‚तोऽस‚त्य‚वान्त‚र‚{\tiny $_{lb}$}‚भेदे भेदाद्धेतूप‚न्यासो न युक्तः । स‚ति तु व‚स्तुतः (।)
	\pend% ending standard par
      
	  \bigskip
	  \begingroup
	
	    \large
	  
	    \begin{quote}
	  
	    
	    \stanza[\smallbreak]
	\label{pv.3.243b}\flagstanza{\tiny\textenglish{...3.243b}}त‚त्राप्र‚द‚र्श्य ये भेदं कार्य‚सामान्य‚द‚र्श‚नात् ॥ २४३ ॥\&[\smallbreak]


	
	    \end{quote}
	  
	  \endgroup
	
	  \bigskip
	  \begingroup
	
	    \large
	  
	    \begin{quote}
	  
	    
	    \stanza[\smallbreak]
	\label{pv.3.244a}\flagstanza{\tiny\textenglish{...3.244a}}हेत‚वः प्र‚वित‚न्य‚न्ते स‚र्वे ते व्य‚भिचारिणः ।\&[\smallbreak]


	
	    \end{quote}
	  
	  \endgroup
	

	  \pstart \leavevmode% starting standard par
	त‚त्र साध‚नीकृते\edtext{}{\edlabel{pvv.376-5}\label{pvv.376-5}\lemma{नीकृते}\Bfootnote{एत‚स्मिन्न्याये स्थिते ।}} व‚स्तुन्य‚वान्त‚र‚भेदे भेद\edtext{}{\edlabel{pvv.376-6}\label{pvv.376-6}\lemma{भेद}\Bfootnote{लौकिक‚वैदिक‚योः ।}} ‚{\color{DodgerBlue3}‚म‚प्र‚द‚र्श्य कार्य‚सा\edtext{}{\edlabel{pvv.376-7}\label{pvv.376-7}\lemma{सा}\Bfootnote{पुरुष‚कार्यैः श‚ब्दैः सामान्य‚स्य तुल्य‚स्य वैदिक‚श‚ब्देषु द‚र्श‚नात् ।}}मान्य}‚स्य ‚{\tiny $_{lb}$}‚विजातीय‚व्यावृत्तिमात्र‚स्य ‚{\color{DodgerBlue3}‚द‚र्श‚नात्} । ये ‚{\color{DodgerBlue3}‚हेत‚वो} य‚द्वेदाध्य‚य‚{\tiny $_{3}$}‚नं त‚द्वेदाध्य‚य‚न‚पूर्व्व‚कं ‚{\tiny $_{lb}$}‚न क‚र‚ण‚पूर्व्व‚कं\edtext{}{\edlabel{pvv.376-8}\label{pvv.376-8}\lemma{कं}\Bfootnote{स्व‚य‚म‚कृत्वा वेद‚स्य अध्य‚य‚नं दार्ष्टान्तिक‚म‚ग्नेर्दृष्टान्तो नैकान्तिक‚त्व‚साध‚न‚स्य ।}} य‚था यः प‚थिकाग्निः\edtext{}{\edlabel{pvv.376-9}\label{pvv.376-9}\lemma{थिकाग्निः}\Bfootnote{आद्योपि प‚थिक‚कृतोग्निर‚दृष्ट‚हेतुत्वात् कालान्त‚र‚हेतुकः प‚थिकाग्नित्वात् ज्वालान्त‚र‚संभूत‚दृश्य‚मानाग्निव‚त् ज्वालार‚णिज‚न्म‚नोर‚बाध्य‚बाध‚क‚त्वाद‚नैकान्तिको व‚ह्निसामान्ये ।}} स ज्वालापूर्व्व‚को न वार‚णिनिर्म‚थ‚न‚पूर्व्व‚क ‚{\tiny $_{lb}$}‚\leavevmode\ledsidenote{\textenglish{377/s}} इत्याद‚यः ‚{\color{DodgerBlue3}‚प्र‚वित‚न्य‚न्ते} विस्तार्य‚न्ते ‚{\color{DodgerBlue3}‚स‚र्व्व ते व्य‚भिचारिणोऽनैकान्तिकाः न हि\edtext{}{\edlabel{pvv.377-1}\label{pvv.377-1}\lemma{हि}\Bfootnote{वेद‚क्रियाश‚क्तिर‚हित‚स्योप‚देष्ट्ट‚पूर्व्व‚काध्याय‚दृष्टिर्न्न ब्र‚ह्मादेर‚पि त‚थात्वं स्व‚यं र‚च‚यित्वाध्य‚य‚न‚संभ‚वाविरोधात् ।}}} वेदाध्य‚य‚न‚मित्येवाध्य‚य‚न‚पूर्व्व‚कं ‚{\color{DodgerBlue3}‚कृत्वा} क‚र‚ण\edtext{}{\edlabel{pvv.377-2}\label{pvv.377-2}\lemma{ण}\Bfootnote{य‚द्वेदाध्य‚य‚न‚मित्यादौ वा । ७ स्व‚यं कृत्वा ।}}पूर्व्व‚क‚स्याप्य‚ध्य‚य‚न‚स्योप‚प‚त्तेः । ‚{\tiny $_{lb}$}‚अर‚णिनिर्म‚थ‚न‚स्य च व‚ह्नेर्भावाविरोधात् ।
	\pend% ending standard par
      

	  \pstart \leavevmode% starting standard par
	भ‚व‚तु वा
	\pend% ending standard par
      
	  \bigskip
	  \begingroup
	
	    \large
	  
	    \begin{quote}
	  
	    
	    \stanza[\smallbreak]
	\label{pv.3.244b}\flagstanza{\tiny\textenglish{...3.244b}}स‚र्व्व‚थाऽनादिता सिद्ध्येदेवं नापुरुषाश्र‚यः ॥ २४४ ॥\&[\smallbreak]


	
	    \end{quote}
	  
	  \endgroup
	
	  \bigskip
	  \begingroup
	
	    \large
	  
	    \begin{quote}
	  
	    
	    \stanza[\smallbreak]
	\label{pv.3.245a}\flagstanza{\tiny\textenglish{...3.245a}}त‚स्माद‚पौरुषेय‚त्वे स्याद‚न्योप्य‚न‚राश्र‚यः ।\&[\smallbreak]


	
	    \end{quote}
	  
	  \endgroup
	

	  \pstart \leavevmode% starting standard par
	एवं वेदाध्य‚य‚न‚म‚ध्य‚य‚न‚पूर्व्व‚तासा‚{\tiny $_{4}$}‚ ध‚नं\edtext{}{\edlabel{pvv.377-3}\label{pvv.377-3}\lemma{नं}\Bfootnote{अभ्युप‚ग‚म्याह ।}} त‚थापि ‚{\color{DodgerBlue3}‚स‚र्व्व‚थाऽनादिता} वेदाध्य‚य‚न‚स्य ‚{\tiny $_{lb}$}‚‚{\color{DodgerBlue3}‚सिध्येत्} डिम्भ‚क‚पांशुक्रीडादीनामिव ‚{\color{DodgerBlue3}‚नापुरुषाश्र‚यः} पुरुषाश्र‚य‚णाभाव‚स्तु ‚{\color{DodgerBlue3}‚न} सिध्येत् । ‚{\tiny $_{lb}$}‚डिम्भ‚क‚पांशुक्रीडा\edtext{}{\edlabel{pvv.377-4}\label{pvv.377-4}\lemma{पांशुक्रीडा}\Bfootnote{आदिना भोज‚नादि बाल‚स्य ।}}द‚यो हि द‚र्श‚न‚पूर्व्व‚का अनाद‚य‚श्च न चापौरुषेयाः डिम्भ‚कैरेव ‚{\tiny $_{lb}$}‚क्रिय‚माण‚त्वात् । एवं श‚ब्दा अप्य‚ध्येतृभिरेव क्रिय‚न्ते न तु स्व‚य‚मात्मानं ध्व‚न‚य‚न्ति ‚{\tiny $_{lb}$}‚येनापौरुषेयाः स्युः । अनादिस्तादृक् क‚र‚{\tiny $_{5}$}‚ण‚क्र‚मः पुरुष‚प‚रंप‚राया अनादेराग‚त‚त्वात् । ‚{\tiny $_{lb}$}‚अथानादित्वादेवापौरुषेय‚तेत्याह । त‚स्यानादित्वा‚{\color{DodgerBlue3}‚द‚पौरुषेय‚त्वे साध्ये स्याद‚न्यो\edtext{}{\edlabel{pvv.377-5}\label{pvv.377-5}\lemma{न्यो}\Bfootnote{लोक‚व्य‚व‚हारः ।}}} प्य‚पौरुषेयो‚{\color{DodgerBlue3}‚ऽन‚राश्र‚य} इति प्र‚स‚ङ्गः ।
	\pend% ending standard par
      

	  \pstart \leavevmode% starting standard par
	b. त‚मेवाह ।
	\pend% ending standard par
      
	  \bigskip
	  \begingroup
	
	    \large
	  
	    \begin{quote}
	  
	    
	    \stanza[\smallbreak]
	\label{pv.3.245b}\flagstanza{\tiny\textenglish{...3.245b}}म्लेच्छादिव्य‚व‚हाराणां नास्तिक्य‚व‚च‚साम‚पि ॥ २४५ ॥\&[\smallbreak]


	
	    \end{quote}
	  
	  \endgroup
	
	  \bigskip
	  \begingroup
	
	    \large
	  
	    \begin{quote}
	  
	    
	    \stanza[\smallbreak]
	\label{pv.3.246a}\flagstanza{\tiny\textenglish{...3.246a}}अनादित्वाद् त‚थाभावः ;\&[\smallbreak]


	
	    \end{quote}
	  
	  \endgroup
	

	  \pstart \leavevmode% starting standard par
	\hphantom{.}‚{\color{DodgerBlue3}‚म्लेच्छादेर्व्य‚व‚हाराणां} मा\edtext{}{\edlabel{pvv.377-6}\label{pvv.377-6}\lemma{मा}\Bfootnote{मृते भ‚र्त्त‚रि पुत्रेण मातृविवाहः कार्यः । वृद्धादीनां मार‚णं संसार‚मोच‚नार्थं आदिना म‚द‚न‚त्र‚योद‚श्यां म‚द‚नोत्स‚वः पुत्रादिज‚न्मोत्स‚वः ।}}तृविवाह‚मुक्तिप्राप‚ण‚मार‚णादीनां ‚{\color{DodgerBlue3}‚नास्तिक्य‚व‚च‚{\tiny $_{lb}$}‚सा\edtext{}{\edlabel{pvv.377-7}\label{pvv.377-7}\lemma{सा}\Bfootnote{लोकाय‚तिकानां ।}}म‚पि} प‚र‚लोक‚क‚र्म‚फ‚लाद्य‚प‚वादिनाम‚नादित्वात् त‚थाभावोऽपौरुषेय‚त्वं स्यात् ।
	\pend% ending standard par
      

	  \pstart \leavevmode% starting standard par
	य‚दि पौ‚{\tiny $_{6}$}‚रुषेयाः क‚थ‚म‚नाद‚य इत्याह ।
	\pend% ending standard par
      
	  \bigskip
	  \begingroup
	
	    \large
	  
	    \begin{quote}
	  
	    
	    \stanza[\smallbreak]
	\label{pv.3.246b}\flagstanza{\tiny\textenglish{...3.246b}}पूर्व‚संस्कार‚स‚न्त‚तेः ।\&[\smallbreak]


	
	    \end{quote}
	  
	  \endgroup
	\textsuperscript{\textenglish{378/s}}

	  \pstart \leavevmode% starting standard par
	\hphantom{.}‚{\color{DodgerBlue3}‚पूर्व्व‚संस्काराद}‚नादेः ‚{\color{DodgerBlue3}‚स‚न्त‚तेः} स‚न्तानेन प्र‚वृत्तेः ।
	\pend% ending standard par
      

	  \pstart \leavevmode% starting standard par
	अथानादित्वात् म्लेच्छादिव्य‚व‚हाराणाञ्चापौरुषेय‚तास्तित्वाह ।
	\pend% ending standard par
      
	  \bigskip
	  \begingroup
	
	    \large
	  
	    \begin{quote}
	  
	    
	    \stanza[\smallbreak]
	\label{pv.3.246c}\flagstanza{\tiny\textenglish{...3.246c}}तादृशेऽपौरुषेय‚त्वे कः सिद्धेपि गुणो भ‚वेत् ॥ २४६ ॥\&[\smallbreak]


	
	    \end{quote}
	  
	  \endgroup
	

	  \pstart \leavevmode% starting standard par
	\hphantom{.}‚{\color{DodgerBlue3}‚तादृशे} म्लेच्छादिव्य‚व‚हार‚साधार‚णेऽ‚{\color{DodgerBlue3}‚पौरुषे}‚य‚त्वे ‚{\color{DodgerBlue3}‚सिद्धेपि को गुणः} अविस‚म्वा‚{\tiny $_{lb}$}‚\leavevmode\ledsidenote{\textenglish{75a/MA}}द‚क‚ल‚क्ष‚णो भ‚वेत् । पौरुषेय‚वाक्यानां विस‚म्वादाद‚र्श‚नाद‚पौरुषेय‚त्व‚मिष्टं । स च ‚{\tiny $_{lb}$}‚त‚स्मिन्न‚पि स‚ति म्लेच्छादिव्य‚व‚हाराणामिव‚{\tiny $_{7}$}‚ दुष्प‚रिह‚रः ।
	\pend% ending standard par
      \label{div_pvv.3.247}
	  
	% new div opening: depth here is 2
	

	  \begin{center}%% label @type='head'
	\textbf{(ख) a. अनादित्वेऽर्थ‚संस्कार‚भेदेन संश‚यः}
	\end{center}
	

	  \pstart \leavevmode% starting standard par
	अथ वेद‚वाक्यानामेवानादित्वाद‚पौरुषेय‚त्वं त‚दा (।)
	\pend% ending standard par
      
	  \bigskip
	  \begingroup
	
	    \large
	  
	    \begin{quote}
	  
	    
	    \stanza[\smallbreak]
	\label{pv.3.247a}\flagstanza{\tiny\textenglish{...3.247a}}अर्थ‚संस्कार‚भेदानां द‚र्श‚नात् संश‚यः पुनः ।\&[\smallbreak]


	
	    \end{quote}
	  
	  \endgroup
	

	  \pstart \leavevmode% starting standard par
	\hphantom{.}अर्थ‚स्य ‚{\color{DodgerBlue3}‚संस्कारो} व्याख्यानं त‚स्य ‚{\color{DodgerBlue3}‚भेदानां} विक‚ल्पानां प्र‚कृतिप्र‚त्य‚यानेकार्थ\edtext{}{\edlabel{pvv.378-1}\label{pvv.378-1}\lemma{यानेकार्थ}\Bfootnote{अपौरुषेय‚त्वं क‚ल्प‚यित्वापि पुनः संश‚य एव य‚तो वेदार्थ‚व्याख्याविक‚ल्पानामाचार्य‚भेदेन भेदः ।}}त्वात् ‚{\tiny $_{lb}$}‚रूढेर्निरुक्तादिभ्य‚श्च य‚थाप्र‚तिभं पुंसान्द‚{\color{DodgerBlue3}‚र्श‚नात् संश‚यो}‚ऽर्थ‚निश्च‚याभावः ।
	\pend% ending standard par
      

	  \pstart \leavevmode% starting standard par
	b. क‚स्य चापौरुषेय‚त्व‚मिष्टं किम्व‚र्ण्णानामुत प‚द‚वाक्यानामित्याह\edtext{}{\edlabel{pvv.378-2}\label{pvv.378-2}\lemma{वाक्यानामित्याह}\Bfootnote{व‚र्ण्ण‚विक‚ल्प‚म‚धिकृत्य ।}} ।
	\pend% ending standard par
      
	  \bigskip
	  \begingroup
	
	    \large
	  
	    \begin{quote}
	  
	    
	    \stanza[\smallbreak]
	\label{pv.3.247b}\flagstanza{\tiny\textenglish{...3.247b}}अन्याविशेषाद् व‚र्ण्णानां साध‚ने किं फ‚लं भ‚वेत् ॥ २४७ ॥\&[\smallbreak]


	
	    \end{quote}
	  
	  \endgroup
	

	  \pstart \leavevmode% starting standard par
	\hphantom{.}‚{\color{DodgerBlue3}‚अन्यै}‚र्लोकिकैर्व‚र्ण्णैर्वैदिकानाम‚{\color{DodgerBlue3}‚विशेषात्}‚प्र‚त्य‚{\color{DodgerBlue3}‚भिज्ञा}‚य‚मान‚त्वेनैक‚त्वाद‚पौरुषेय‚त्व‚स्य‚{\tiny $_{1}$}‚ ‚{\tiny $_{lb}$}‚‚{\color{DodgerBlue3}‚साध‚ने किं फ‚लं भ‚वेत्} (।) त‚थात्वे लौकिकानाम‚र्थ‚व्य‚भिचारात् ।
	\pend% ending standard par
      \label{div_pvv.3.248_3.249_3.250}
	  
	% new div opening: depth here is 2
	
	  \bigskip
	  \begingroup
	
	    \large
	  
	    \begin{quote}
	  
	    
	    \stanza[\smallbreak]
	\label{pv.3.348}\flagstanza{\tiny\textenglish{....3.348}}c. वाक्यं भिन्नं न व‚र्णेभ्यो विद्य‚तेऽनुप‚ल‚म्भ‚तः ।&अनेकाव‚य‚वात्म‚त्वे पृथ‚क् तेषां निर‚र्थ‚ता ॥ २४८ ॥\&[\smallbreak]


	
	    \end{quote}
	  
	  \endgroup
	
	  \bigskip
	  \begingroup
	
	    \large
	  
	    \begin{quote}
	  
	    
	    \stanza[\smallbreak]
	\label{pv.3.249a}\flagstanza{\tiny\textenglish{...3.249a}}अत‚द्रूपे च ताद्रूप्यं क‚ल्पितं सिंह‚तादिव‚त् ।\&[\smallbreak]


	
	    \end{quote}
	  
	  \endgroup
	

	  \pstart \leavevmode% starting standard par
	\hphantom{.}‚{\color{DodgerBlue3}‚वाक्यं} प‚द‚ञ्च ‚{\color{DodgerBlue3}‚व‚र्ण्णेभ्यो भिन्नं न विद्य‚तेऽनुप‚ल‚म्भात्} (।) त‚त्क‚थ‚म‚स्यापौ‚{\tiny $_{lb}$}‚रुषेय‚त्वं\edtext{}{\edlabel{pvv.378-3}\label{pvv.378-3}\lemma{त्वं}\Bfootnote{प्र‚त्येकं व‚र्ण्णान‚र्थ‚क्याद‚र्थ‚प्र‚तीतिकार्यान्य‚थानुप‚प‚त्त्यास्ति प‚दादीति चेन्न यावान्व‚र्ण्ण‚स‚मुदायोर्थ‚प्र‚तीत‚ये संकेत‚स्ताव‚तोर्थ‚प्र‚तीत्य‚भाव एत‚स्मान्न चैवं दृश्यानुप‚ल‚म्भाच्च प‚दादेः ।}} साध्यं । अभिन्नं चेत् त‚दा‚{\color{DodgerBlue3}‚नेकाव‚य‚वात्म‚त्वे} वाक्य‚स्य तेषाम‚व‚य‚वानां ‚{\tiny $_{lb}$}‚पृथ‚क् प्र‚त्येकं निर‚र्थ‚क‚तेति प‚दात्म‚क‚म‚न‚र्थ‚क‚मेव स्यात् । त‚त‚श्चात‚द्रूपेऽवाच‚क‚{\tiny $_{lb}$}‚\leavevmode\ledsidenote{\textenglish{379/s}} रूपे ताद्रूप्यं वाच‚क‚त्वं क‚ल्पितं क‚ल्प‚नाबुद्धिनिर्मितं माण‚व‚कादाविव सिं‚{\tiny $_{2}$}‚ह‚तादि । ‚{\tiny $_{lb}$}‚त‚तः\edtext{}{\edlabel{pvv.379-1}\label{pvv.379-1}\lemma{तः}\Bfootnote{वाच‚क‚त्व‚स्य पुरुषेण क‚ल्पित‚त्वात् ।}} पौरुषेय‚मेव वाच‚क‚त्वं स्यादिति प्र‚स्तुत‚क्ष‚तिः ।
	\pend% ending standard par
      

	  \pstart \leavevmode% starting standard par
	d. अथ प्र‚त्येक‚म‚व‚य‚वानां सार्थ‚क‚त्वं त‚दा (।)
	\pend% ending standard par
      
	  \bigskip
	  \begingroup
	
	    \large
	  
	    \begin{quote}
	  
	    
	    \stanza[\smallbreak]
	\label{pv.3.249b}\flagstanza{\tiny\textenglish{...3.249b}}प्र‚त्येकं सार्थ‚क‚त्वेपि मिथ्यानेक‚त्व‚क‚ल्प‚ना ॥ २४९ ॥\&[\smallbreak]


	
	    \end{quote}
	  
	  \endgroup
	
	  \bigskip
	  \begingroup
	
	    \large
	  
	    \begin{quote}
	  
	    
	    \stanza[\smallbreak]
	\label{pv.3.250a}\flagstanza{\tiny\textenglish{...3.250a}}एकाव‚य‚व‚ग‚त्या च वाक्यार्थ‚प्र‚तिप‚द् भ‚वेत् ।\&[\smallbreak]


	
	    \end{quote}
	  
	  \endgroup
	

	  \pstart \leavevmode% starting standard par
	प्र‚त्येकं सार्थ‚क‚त्वे\edtext{}{\edlabel{pvv.379-2}\label{pvv.379-2}\lemma{त्वे}\Bfootnote{वाक्यार्थेन ।}}पि मिथ्यानेक‚त्व‚स्यानेकाव‚य‚वात्म‚क‚स्य क‚ल्प‚ना\edtext{}{\edlabel{pvv.379-3}\label{pvv.379-3}\lemma{ना}\Bfootnote{एक‚स्याप्य‚व‚य‚व‚स्य वाक्यार्थ‚व‚त्वाद‚व‚य‚वान्त‚रापेक्षा न स्यात् ।}} एक‚{\tiny $_{lb}$}‚स्माद‚र्थ‚प्र‚तीतेः ।\edtext{\textsuperscript{*}}{\edlabel{pvv.379-4}\label{pvv.379-4}\lemma{*}\Bfootnote{य‚दा चैकोप्य‚व‚य‚वोर्थ‚वान् त‚दा ।}} त‚थैक‚स्याव‚य‚व‚स्य ग‚त्या वाक्यार्थ‚स्य प्र‚तिप‚त् प्र‚तीतिर्भ‚वेत् (।) ‚{\tiny $_{lb}$}‚एक‚स्यापि वाच‚क‚त्वात् । त‚त्स‚मुदायो वाच‚क इति चेत् । स किन्तेभ्यो भिन्नः । ‚{\tiny $_{lb}$}‚स च प्र‚त्युक्तः (।) अव‚य‚वा‚{\tiny $_{3}$}‚ मिलिताः स‚मुदाय इति चेत् । स च\edtext{}{\edlabel{pvv.379-5}\label{pvv.379-5}\lemma{च}\Bfootnote{प‚र‚व‚र्ण्णोच्चार‚णे पूर्व्व‚ध्वंसात् ।}} न स‚म्भ‚व‚त्येव । ‚{\tiny $_{lb}$}‚त‚द् य‚दि प्र‚त्येकं वाच‚क‚त्व‚मेव त‚दैषां प्र‚त्येकं वाच‚क‚त्वे चैक‚स्माद‚पि स्याद‚र्थ‚प्र‚तीतिः ।
	\pend% ending standard par
      

	  \pstart \leavevmode% starting standard par
	अथ । (।)
	\pend% ending standard par
      
	  \bigskip
	  \begingroup
	
	    \large
	  
	    \begin{quote}
	  
	    
	    \stanza[\smallbreak]
	\label{pv.3.250b}\flagstanza{\tiny\textenglish{...3.250b}}स‚कृच्छ्रुतौ च स‚र्वेषां काल‚भेदो न युज्य‚ते ॥ २५० ॥\&[\smallbreak]


	
	    \end{quote}
	  
	  \endgroup
	

	  \pstart \leavevmode% starting standard par
	\hphantom{.}‚{\color{DodgerBlue3}‚एकाव‚य‚व‚ग‚त्याऽर्थ‚प्र‚तिप‚त्ते}‚र‚व‚य‚वान्त‚र‚वैफ‚ल्य‚दोषात् स‚र्व्वाव‚य‚वानां स‚कृच्छ्रुति‚{\tiny $_{lb}$}‚रिष्य‚ते । त‚दा स‚र्व्वेषाम‚व‚य‚वानां ‚{\color{DodgerBlue3}‚स‚कृच्छ्रुतौ} चाभिम‚तायाम‚व‚य‚व‚श्र‚व‚ण‚स्य ‚{\color{DodgerBlue3}‚काल‚भेदो\edtext{}{\edlabel{pvv.379-6}\label{pvv.379-6}\lemma{भेदो}\Bfootnote{एकाव‚य‚व‚बोध‚काले स‚र्व्व‚त्र श्र‚व‚णात् ।}} ‚{\tiny $_{lb}$}‚न युज्य‚ते} । दृश्येत च ।
	\pend% ending standard par
      \label{div_pvv.3.251_3.252_3.253_3.254_3.255_3.256_3.257_3.258_3.259_3.260}
	  
	% new div opening: depth here is 2
	

	  \begin{center}%% label @type='head'
	\textbf{(ग) स्फोट‚निरासः}
	\end{center}
	

	  \pstart \leavevmode% starting standard par
	अथान‚व‚{\tiny $_{4}$}‚य‚व‚मेकं व‚र्ण्णेभ्यो व्य‚तिरिक्तं स्फोट‚रूपं वाक्यं त‚च्च क्र‚म‚व‚द्भिर्न्निय‚{\tiny $_{lb}$}‚तानुपूर्व्वीकैर्ध्व‚निभिः क्र‚मेण व्य‚ज्य‚ते । व्य‚क्त्य‚नुक्र‚मेणैव च क्र‚म‚व‚त् प्र‚तीय‚ते ‚{\tiny $_{lb}$}‚त‚द्रूपाविभागेन नाद\edtext{}{\edlabel{pvv.379-7}\label{pvv.379-7}\lemma{नाद}\Bfootnote{नादानां भेदात् ।}}रूपाणाम्व‚र्ण्णानां ग्र‚ह‚णात् व‚र्ण्ण‚विभाग‚व‚च्च ल‚क्ष्य‚ते । ‚{\tiny $_{lb}$}‚व‚स्तुत‚स्त‚थार‚हित‚म‚पीति केचित् ।
	\pend% ending standard par
      
	  \bigskip
	  \begingroup
	
	    \large
	  
	    \begin{quote}
	  
	    
	    \stanza[\smallbreak]
	\label{pv.3.251a}\flagstanza{\tiny\textenglish{...3.251a}}एक‚त्वेपि ह्य‚भिन्न‚स्य क्र‚म‚शो ग‚त्य‚स‚म्भ‚वात् ।\&[\smallbreak]


	
	    \end{quote}
	  
	  \endgroup
	

	  \pstart \leavevmode% starting standard par
	\hphantom{.}‚{\color{DodgerBlue3}‚त‚देक‚त्वेपि ह्य‚भिन्न‚स्या}‚न‚व‚य‚व‚स्य स्फोट‚रूप‚स्य वाक्य‚स्य प्र‚थ‚म‚ध्व‚निनापि ‚{\tiny $_{lb}$}‚व्य‚क्त‚त्वात् ‚{\color{DodgerBlue3}‚क्र‚म‚{\tiny $_{5}$}‚शो ग‚तेः\edtext{}{\edlabel{pvv.379-8}\label{pvv.379-8}\lemma{तेः}\Bfootnote{गृहीतागृहीत‚योर‚भेदात् ।}}} प्र‚तीतेर‚{\color{DodgerBlue3}‚स‚म्भ‚वात्} स‚कृत् प्र‚तीतिप्र‚स‚ङ्गः\edtext{}{\edlabel{pvv.379-9}\label{pvv.379-9}\lemma{ङ्गः}\Bfootnote{जातिस्फोट‚स्तु जात्य‚भावादेव निर‚स्तः ।}} । य‚दि ‚{\tiny $_{lb}$}‚प्र‚थ‚म‚व्य‚क्तौ न प्र‚तीतिर‚प‚रास्व‚पि न स्यात् । त‚देक‚व्य‚ञ्च‚क‚त्वाद् व्य‚क्तीनां ।
	\pend% ending standard par
      \textsuperscript{\textenglish{380/s}}

	  \pstart \leavevmode% starting standard par
	किञ्च (।) वाक्य‚न्त‚न्नित्य‚म‚नित्य‚म्वा स्यात् ।
	\pend% ending standard par
      
	  \bigskip
	  \begingroup
	
	    \large
	  
	    \begin{quote}
	  
	    
	    \stanza[\smallbreak]
	\label{pv.3.251b}\flagstanza{\tiny\textenglish{...3.251b}}अनित्यं य‚त्न‚स‚म्भूतं पौरुषेयं क‚थ‚न्न त‚त् ॥ २५१ ॥\&[\smallbreak]


	
	    \end{quote}
	  
	  \endgroup
	
	  \bigskip
	  \begingroup
	
	    \large
	  
	    \begin{quote}
	  
	    
	    \stanza[\smallbreak]
	\label{pv.3.252a}\flagstanza{\tiny\textenglish{...3.252a}}नित्योप‚ल‚ब्धिनित्य‚त्वेऽप्य‚नाव‚र‚ण‚स‚म्भ‚वात् ।\&[\smallbreak]


	
	    \end{quote}
	  
	  \endgroup
	

	  \pstart \leavevmode% starting standard par
	\hphantom{.}‚{\color{DodgerBlue3}‚य‚द्य‚नित्यं पौरुषेयं क‚थं न त‚त् य‚त्न‚स‚म्भ}‚वात्\edtext{}{\edlabel{pvv.380-1}\label{pvv.380-1}\lemma{वात्}\Bfootnote{अव‚श्यं ह्युत्प‚त्तिम‚द‚नित्यं कुत‚श्चित् स्यान्निर्हेतुक‚त्वे देशादिनिय‚माभावात् त‚च्च पुरुष‚प्र‚य‚त्नान्व‚य‚व्य‚तिरेकानुविधायिपौरुषेय‚मेव स्यात् । किञ्चिदेव प्र‚तिनिय‚तं व‚स्तुस्थित्यास्ति त‚त्क‚दाचित् क‚स्य‚चिद् भ‚व‚ति तेन क्विचित् क‚दाचिच्छ्र‚व‚णं ।}} घ‚टादिव‚त् । ‚{\color{DodgerBlue3}‚नित्य‚त्वेप्यु‚{\tiny $_{lb}$}‚प‚ल‚ब्धिर्नित्या} भ‚वेत् । वाक्यानाम‚{\color{DodgerBlue3}‚नाव‚र‚ण}‚स‚म्भ‚वात् । आव‚र‚णायोगात् । य‚दि नित्यं ‚{\tiny $_{lb}$}‚क‚दाचिदुप‚ल‚भ्य‚ते त‚दा त‚स्यानावृत‚स्व‚भाव‚ता‚{\tiny $_{6}$}‚ऽभ्युप‚ग‚न्त‚व्या । तादृशं नित्य‚मुप‚{\tiny $_{lb}$}‚ल‚भ्येताभिम‚त‚काल इव । आवृत‚स्व‚भावे तु स‚र्व्व‚दाऽनुप‚ल‚म्भः स्यात् । त‚स्मात् ‚{\tiny $_{lb}$}‚काल‚भेदेनोप‚ल‚भ्यानुप‚ल‚भ्य‚स्व‚भाव‚त्वादेक‚स्य नाशाद‚न्य‚स्योद‚य इति स्यात् । त‚था ‚{\tiny $_{lb}$}‚च नित्य‚त्व‚क्ष‚तिः ।
	\pend% ending standard par
      

	  \pstart \leavevmode% starting standard par
	एक‚स्व‚भाव एव श‚ब्दः प‚रं (।)
	\pend% ending standard par
      
	  \bigskip
	  \begingroup
	
	    \large
	  
	    \begin{quote}
	  
	    
	    \stanza[\smallbreak]
	\label{pv.3.252b}\flagstanza{\tiny\textenglish{...3.252b}}अश्रुतिर्व्विक‚ल‚त्वाच्चेत् क‚स्य‚चित् स‚ह‚कारिणः ॥ २५२ ॥\&[\smallbreak]


	
	    \end{quote}
	  
	  \endgroup
	
	  \bigskip
	  \begingroup
	
	    \large
	  
	    \begin{quote}
	  
	    
	    \stanza[\smallbreak]
	\label{pv.3.253a}\flagstanza{\tiny\textenglish{...3.253a}}काम‚म‚न्य‚प्र‚तीक्षोक्तिर्निय‚म‚स्तु विरुध्य‚ते ।\leavevmode\ledsidenote{\textenglish{75b/MA}}\&[\smallbreak]


	
	    \end{quote}
	  
	  \endgroup
	

	  \pstart \leavevmode% starting standard par
	\hphantom{.}‚{\color{DodgerBlue3}‚क‚स्य‚चित् स‚ह‚कारिणो} ज्ञान‚ज‚न‚क‚स्य ‚{\color{DodgerBlue3}‚वैक‚ल्याद‚श्रुति}‚रिति ‚{\color{DodgerBlue3}‚चेत् । काम‚म‚न्य}‚स्य ‚{\tiny $_{lb}$}‚स‚ह‚कारिण उप‚कार‚क‚स्य प्र‚ती‚{\tiny $_{7}$}‚क्षापेक्ष‚ण‚म‚स्तु निय‚मः पूर्व्वाप‚रैक‚स्व‚भाव‚ताव‚र‚ण‚न्तु ‚{\tiny $_{lb}$}‚श‚ब्दानां ‚{\color{DodgerBlue3}‚विरुध्य‚ते} । न ह्येक‚स्व‚भावः प‚र‚म‚पेक्ष‚ते चेति क्ष‚मं । उप‚कार‚क‚स्यापेक्ष‚णी‚{\tiny $_{lb}$}‚य‚त्वात् । उप‚कारान्त‚र‚स्य च स्व‚भावान्त‚र‚ल‚क्ष‚ण‚त्वात् ।
	\pend% ending standard par
      

	  \pstart \leavevmode% starting standard par
	अपि च श‚ब्दा अव्यापिनो व्यापिनो वा स्युः । त‚त्र (।)
	\pend% ending standard par
      
	  \bigskip
	  \begingroup
	
	    \large
	  
	    \begin{quote}
	  
	    
	    \stanza[\smallbreak]
	\label{pv.3.253b}\flagstanza{\tiny\textenglish{...3.253b}}स‚र्व‚त्रानुप‚ल‚म्भः स्यात् तेषाम‚व्यापिता य‚दि ॥ २५३ ॥\&[\smallbreak]


	
	    \end{quote}
	  
	  \endgroup
	
	  \bigskip
	  \begingroup
	
	    \large
	  
	    \begin{quote}
	  
	    
	    \stanza[\smallbreak]
	\label{pv.3.254a}\flagstanza{\tiny\textenglish{...3.254a}}स‚र्वेषामुप‚ल‚म्भः स्यात् युग‚प‚द् व्यापिता य‚दि ।\&[\smallbreak]


	
	    \end{quote}
	  
	  \endgroup
	

	  \pstart \leavevmode% starting standard par
	\hphantom{.}‚{\color{DodgerBlue3}‚य‚द्य‚व्यापिता तेषां स‚र्व्व‚त्र देशेऽऽनुप‚ल‚म्भः स्यात्} । न ह्येक‚देश‚स्थितः शैलः ‚{\tiny $_{lb}$}‚स‚र्व्व‚त्रोप‚{\tiny $_{1}$}‚ल‚भ्य‚ते । योग्य‚तातिश‚य‚लाभाद् व्य‚ञ्च‚केभ्यो दृश्य‚ते एवेति चेत् । ‚{\tiny $_{lb}$}‚एव‚न्त‚र्हि स‚र्व्वैः स‚र्व्व‚देश‚स्थैरुप‚ल‚भ्येत । त‚था ‚{\color{DodgerBlue3}‚व्यापिता य‚दि\edtext{}{\edlabel{pvv.380-2}\label{pvv.380-2}\lemma{दि}\Bfootnote{श्रोत्र‚ज‚प्र‚तिज्ञ‚या नित्य‚त्वं व्यापित्व‚ञ्च श‚ब्दानामित्याह ।}} स‚र्व्वेषां} श‚ब्दानां ‚{\tiny $_{lb}$}‚‚{\color{DodgerBlue3}‚युग‚प‚दुप‚ल‚म्भः स्यात्} । व्यापित्वात् स‚र्व्वेषां स‚र्व्व‚त्र भावात् ।
	\pend% ending standard par
      

	  \pstart \leavevmode% starting standard par
	अथ व्यापित्वेपि य एवाभिव्य‚क्त्या संस्कृतः\edtext{}{\edlabel{pvv.380-3}\label{pvv.380-3}\lemma{संस्कृतः}\Bfootnote{प्र‚य‚त्नाभिह‚त‚वायुना संस्कृत‚स्य श‚ब्द‚स्य संस्कृतेनैवेन्द्रियेणोप‚ल‚ब्धेर्न प्र‚स‚ङ्गः ।}} स एवोप‚ल‚भ्य‚ते नेत‚रः ।
	\pend% ending standard par
      \textsuperscript{\textenglish{381/s}}
	  \bigskip
	  \begingroup
	
	    \large
	  
	    \begin{quote}
	  
	    
	    \stanza[\smallbreak]
	\label{pv.3.254b}\flagstanza{\tiny\textenglish{...3.254b}}संस्कृत‚स्योप‚ल‚म्भे च कः संस्क‚र्त्ताऽविकारिणः ॥ २५४ ॥\&[\smallbreak]


	
	    \end{quote}
	  
	  \endgroup
	
	  \bigskip
	  \begingroup
	
	    \large
	  
	    \begin{quote}
	  
	    
	    \stanza[\smallbreak]
	\label{pv.3.255a}\flagstanza{\tiny\textenglish{...3.255a}}इन्द्रिय‚स्य स्यात् संस्कारः शृणुयान्निखिल‚ञ्च त‚त् ।\&[\smallbreak]


	
	    \end{quote}
	  
	  \endgroup
	

	  \pstart \leavevmode% starting standard par
	\hphantom{.}‚{\color{DodgerBlue3}‚संस्कृत‚स्य चोप‚ल‚म्भे च} स्वीक्रिय‚माणे नित्य‚त्वाद‚{\color{DodgerBlue3}‚विकारिणः कः संस्क‚र्त्ता} नाम । न ह्य‚नुप‚कुर्व्व‚न् संस्क‚र्त्ता‚{\tiny $_{2}$}‚ अनुप‚कार्यो वा संस्कार्यः । अथेन्द्रिय‚म‚नित्य‚त्वात् ‚{\tiny $_{lb}$}‚संस्कार्य त‚तः संस्कृत एव प‚श्य‚ति नान्य इति विभागः । स‚त्य‚मिन्द्रिय‚स्य स्यात् ‚{\tiny $_{lb}$}‚संस्कारः । किन्तु त‚दिन्द्रियं ‚{\color{DodgerBlue3}‚निखिलं च} श‚ब्द‚ग्रामं ‚{\color{DodgerBlue3}‚शृणुयात्} न त्वेक‚श‚ब्दं ।\edtext{\textsuperscript{*}}{\edlabel{pvv.381-1}\label{pvv.381-1}\lemma{*}\Bfootnote{अथा य‚था श‚ब्द‚प्र‚तिप‚त्त्य‚न्य‚थानुप‚प‚त्त्येन्द्रिय‚स्य संस्कार‚क‚ल्प‚ना त‚था संस्कार‚भेदो य‚तः प्र‚तिविष‚यं भिन्न‚त्वादिन्द्रिय‚स्यैक‚श‚ब्द‚स्य ग्र‚ह‚णं । प्र‚तिनिय‚ता हि संस्काराः श‚ब्दानां त‚त्र केन‚चित् संस्कृत‚मिन्द्रियं क‚स्य(चि)देव ग्राह‚कं ।}}
	\pend% ending standard par
      
	  \bigskip
	  \begingroup
	
	    \large
	  
	    \begin{quote}
	  
	    
	    \stanza[\smallbreak]
	\label{pv.3.255b}\flagstanza{\tiny\textenglish{...3.255b}}संस्कार‚भेद‚भिन्न‚त्वादेकार्थ‚निय‚मो य‚दि ॥ २५५ ॥\&[\smallbreak]


	
	    \end{quote}
	  
	  \endgroup
	
	  \bigskip
	  \begingroup
	
	    \large
	  
	    \begin{quote}
	  
	    
	    \stanza[\smallbreak]
	\label{pv.3.256a}\flagstanza{\tiny\textenglish{...3.256a}}अनेक‚श‚ब्द‚संघाते श्रुतिः क‚ल‚क‚ले क‚थ‚म् ।\&[\smallbreak]


	
	    \end{quote}
	  
	  \endgroup
	

	  \pstart \leavevmode% starting standard par
	\hphantom{.}‚{\color{DodgerBlue3}‚संस्कार}‚स्य श‚ब्द‚विष‚य‚स्य ‚{\color{DodgerBlue3}‚भेदा}‚त् प्र‚तिनिय‚माद् ‚{\color{DodgerBlue3}‚भिन्न‚त्वा}‚दिन्द्रिय‚संस्काराणा‚{\tiny $_{lb}$}‚‚{\color{DodgerBlue3}‚मेक}‚स्मि‚{\color{DodgerBlue3}‚न्न‚र्थे} श‚ब्दे ‚{\color{DodgerBlue3}‚निय‚मः} श्रुति‚{\color{DodgerBlue3}‚र्य‚दी}‚ष्य‚ते त‚दा‚{\color{DodgerBlue3}‚नेक‚श‚ब्द‚संघाते क‚ल‚क‚ले श्रुति}‚र‚ने‚{\color{DodgerBlue3}‚केषां} श‚ब्दानां ‚{\color{DodgerBlue3}‚क‚थं‚{\tiny $_{3}$}‚संस्कार‚प्र‚तिनिय‚मादिन्द्रियं} नानेक‚श‚ब्द‚ग्राहि स्यात् \edtext{}{\edlabel{pvv.381-2}\label{pvv.381-2}\lemma{स्यात्}\Bfootnote{त‚स्मात् ताल्वादिना श‚ब्द‚क‚र‚ण‚मेव तेन याव‚न्तः श्रूय‚न्ते ।}} ।
	\pend% ending standard par
      

	  \pstart \leavevmode% starting standard par
	अथ (।)
	\pend% ending standard par
      
	  \bigskip
	  \begingroup
	
	    \large
	  
	    \begin{quote}
	  
	    
	    \stanza[\smallbreak]
	\label{pv.3.256b}\flagstanza{\tiny\textenglish{...3.256b}}ध्व‚न‚यः केव‚लं त‚त्र श्रूय‚न्ते चेन्न वाच‚काः ॥ २५६ ॥\&[\smallbreak]


	
	    \end{quote}
	  
	  \endgroup
	
	  \bigskip
	  \begingroup
	
	    \large
	  
	    \begin{quote}
	  
	    
	    \stanza[\smallbreak]
	\label{pv.3.257a}\flagstanza{\tiny\textenglish{...3.257a}}ध्व‚निभ्यो भिन्न‚म‚स्तीति श्र‚द्धेय‚म‚विव‚क्षित‚म् ।\&[\smallbreak]


	
	    \end{quote}
	  
	  \endgroup
	

	  \pstart \leavevmode% starting standard par
	\hphantom{.}‚{\color{DodgerBlue3}‚त‚त्र} क‚ल‚क‚ले ‚{\color{DodgerBlue3}‚ध्व‚न‚यः} श‚ब्द‚व्य‚ञ्ज‚काः ‚{\color{DodgerBlue3}‚केव‚लं श्रूय‚न्ते न वाच‚काः} श‚ब्दाः\edtext{}{\edlabel{pvv.381-3}\label{pvv.381-3}\lemma{ब्दाः}\Bfootnote{वाच‚क एव प्र‚तिनिय‚त‚श‚क्तीन्द्रियं न ध्व‚निषु ।}} ‚{\tiny $_{lb}$}‚(।) ‚{\color{DodgerBlue3}‚ध्व‚निभ्यः} श्रूय‚माणेभ्यो ‚{\color{DodgerBlue3}‚भिन्नं\edtext{}{\edlabel{pvv.381-4}\label{pvv.381-4}\lemma{भिन्नं}\Bfootnote{व‚र्ण्ण‚प‚दादिश‚ब्द‚स्य ध्व‚निविशेष‚त्वात् ।}}} श‚ब्द‚रूप‚{\color{DodgerBlue3}‚म‚स्तीति} य‚त्किञ्जिदिद‚म‚ति‚{\color{DodgerBlue3}‚श्र‚द्धेयं} । ‚{\tiny $_{lb}$}‚श्र‚द्धाव‚शाद् य‚द्येत‚द‚ङ्गीक्रिय‚ते न तु प्र‚माण‚ब‚लात् ।
	\pend% ending standard par
      

	  \pstart \leavevmode% starting standard par
	किञ्ज (।)
	\pend% ending standard par
      
	  \bigskip
	  \begingroup
	
	    \large
	  
	    \begin{quote}
	  
	    
	    \stanza[\smallbreak]
	\label{pv.3.257b}\flagstanza{\tiny\textenglish{...3.257b}}स्थितेष्व‚न्येषु श‚ब्देषु श्रूय‚ते वाच‚कः क‚थ‚म् ॥ २५७ ॥\&[\smallbreak]


	
	    \end{quote}
	  
	  \endgroup
	

	  \pstart \leavevmode% starting standard par
	क‚ल‚क‚ले ध्व‚निमात्रं य‚दि श्रूय‚ते न श‚ब्दः त‚दा ब‚हूनां व्याह‚र्त्तॄणां तूष्णीम्भा‚{\tiny $_{lb}$}‚‚{\color{DodgerBlue3}‚वाद‚न्येष्व‚प्येषु श‚ब्देषु स्थितेषु} एक‚स्मि‚{\tiny $_{4}$}‚न् पुरुषे व्याह‚र‚ति ‚{\color{DodgerBlue3}‚क‚थं वाच‚कः श्रूय‚ते} \leavevmode\ledsidenote{\textenglish{382/s}} \edtext{\textsuperscript{*}}{\edlabel{pvv.382-1}\label{pvv.382-1}\lemma{*}\Bfootnote{अवाच‚क‚श्र‚व‚णादेव । वाच‚केन स‚ह पृथ‚ग्वा ध्व‚नेर‚पि श्र‚व‚णं स्यादित्य‚र्थः ।}}क‚ल‚क‚लइ व ध्व‚निमा\edtext{}{\edlabel{pvv.382-2}\label{pvv.382-2}\lemma{निमा}\Bfootnote{क‚ल‚क‚लेपि वाच‚क‚श्रुतिः स्यात् प‚द‚वाक्य‚च्छेद‚बाधात् ।}}त्र‚श्रुतिस्त‚दापि स्यात् । अथ वाच‚क‚स्योल‚ब्धिप्र‚त्य‚य‚भावा‚{\tiny $_{lb}$}‚\edtext{}{\edlabel{pvv.382-3}\label{pvv.382-3}\lemma{भावा}\Bfootnote{पूर्व‚पूर्व‚ध्व‚निभाग‚स्योत्त‚रोत्त‚रेण ध्व‚निभागेनास‚न्धानाद् वैयाक‚र‚णाद्यैर‚वाच‚का इष्टा ते दोषाः । त‚स्माद‚नुप‚ल‚ब्धिवादितो ध्व‚निः}}दुप‚ल‚ब्धिस्त‚दा क‚ल‚क‚लेपि स्याद् विशेषाभावात् (।)
	\pend% ending standard par
      

	  \pstart \leavevmode% starting standard par
	य‚दि चेन्द्रियाणां संस्कार‚विशेषाच्छ‚ब्द‚विशेषोप‚ल‚ब्धिप्र‚तिनिय‚म‚स्त‚दा (।)
	\pend% ending standard par
      
	  \bigskip
	  \begingroup
	
	    \large
	  
	    \begin{quote}
	  
	    
	    \stanza[\smallbreak]
	\label{pv.3.258a}\flagstanza{\tiny\textenglish{...3.258a}}क‚थं वा श‚क्तिनिय‚माद् भिन्न‚ध्व‚निग‚तिर्भ‚वेत् ।\&[\smallbreak]


	
	    \end{quote}
	  
	  \endgroup
	

	  \pstart \leavevmode% starting standard par
	\hphantom{.}श‚क्तिनिय‚म‚स्त‚दा ‚{\color{DodgerBlue3}‚श‚क्तिनिय‚मादि}‚न्द्रियाणां ‚{\color{DodgerBlue3}‚क‚थं भिन्न‚ध्व‚निग‚तिर्भ‚वेत् ।\edtext{\textsuperscript{*}}{\edlabel{pvv.382-4}\label{pvv.382-4}\lemma{*}\Bfootnote{ग‚तौ वा युग‚प‚न्नानारूप‚ध्व‚निश्रुतिव‚च्छ‚ब्दानाम‚पि स्यात् न हि तैः किञ्चिद‚प‚राद्धं ।}}} श‚ब्द‚विशेष‚व‚त् ध्व‚निविशेष‚स्यैव ग्राह‚क‚मिन्द्रियं स्यात् त‚त्क‚{\tiny $_{5}$}‚थं क‚ल‚क‚ल‚ध्व‚नि‚{\tiny $_{lb}$}‚प्र‚तीतिः ।
	\pend% ending standard par
      

	  \pstart \leavevmode% starting standard par
	किञ्च (।)
	\pend% ending standard par
      
	  \bigskip
	  \begingroup
	
	    \large
	  
	    \begin{quote}
	  
	    
	    \stanza[\smallbreak]
	\label{pv.3.258b}\flagstanza{\tiny\textenglish{...3.258b}}ध्व‚न‚यः संम‚ता यैस्ते दोषैः कैर‚प्य‚वाच‚काः ॥ २५८ ॥\&[\smallbreak]


	
	    \end{quote}
	  
	  \endgroup
	
	  \bigskip
	  \begingroup
	
	    \large
	  
	    \begin{quote}
	  
	    
	    \stanza[\smallbreak]
	\label{pv.3.259a}\flagstanza{\tiny\textenglish{...3.259a}}ध्व‚निभिर्ब्य‚ज्य‚मानेस्मिन् वाच‚केपि क‚थं न ते ।\&[\smallbreak]


	
	    \end{quote}
	  
	  \endgroup
	

	  \pstart \leavevmode% starting standard par
	\hphantom{.}‚{\color{DodgerBlue3}‚यैः कैर‚पि दोषैर्ध्व‚न‚योऽवाच‚काः स‚म्म‚तास्ते} दोषा ‚{\color{DodgerBlue3}‚व्य‚ज्य‚मानेपि} वाच‚के‚{\color{DodgerBlue3}‚स्मिन् ‚{\tiny $_{lb}$}‚क‚थ‚न्न} भ‚व‚न्ति । त‚थाहि । य‚था प्र‚त्येक‚म‚वाच‚क‚त्वाद्वाच(क)त्वे ‚{\color{DodgerBlue3}‚वा} ध्व‚न्य‚न्त‚र‚वैफ‚ल्यात् ‚{\tiny $_{lb}$}‚साहित्याभावाच्च ध्व‚न‚योऽवाच‚कास्त‚था ध्व‚निभिः प्र‚त्येकं वाच‚कान‚भिव्य‚क्तेर‚भि‚{\tiny $_{lb}$}‚व्य‚क्तौ वा ध्व‚न्य‚न्त‚रे वैफ‚ल्यात् साहित्याभावाच्च नाभिव्य‚ज्येत\edtext{}{\edlabel{pvv.382-5}\label{pvv.382-5}\lemma{ज्येत}\Bfootnote{त‚न्न व‚र्ण्णाद्य‚पौरुषेय‚ता ।}} श‚ब्दः । अन‚भिव्य‚{\tiny $_{lb}$}‚क्त‚श्च क‚म‚र्थ प्र‚{\tiny $_{6}$}‚तिपाद‚येत् ।
	\pend% ending standard par
      

	  \begin{center}%% label @type='head'
	\textbf{(घ) a. व‚र्णानुपूर्विचिन्ता}
	\end{center}
	
	  \bigskip
	  \begingroup
	
	    \large
	  
	    \begin{quote}
	  
	    
	    \stanza[\smallbreak]
	\label{pv.3.259b}\flagstanza{\tiny\textenglish{...3.259b}}व‚र्ण्णानुपूर्वी वाक्यं चेन्न व‚र्ण्णानाम‚भेद‚तः ॥ २५९ ॥\&[\smallbreak]


	
	    \end{quote}
	  
	  \endgroup
	
	  \bigskip
	  \begingroup
	
	    \large
	  
	    \begin{quote}
	  
	    
	    \stanza[\smallbreak]
	\label{pv.3.260a}\flagstanza{\tiny\textenglish{...3.260a}}तेषाञ्च न व्य‚व‚स्थानं क्र‚मान्त‚र‚विरोध‚तः ।\&[\smallbreak]


	
	    \end{quote}
	  
	  \endgroup
	

	  \pstart \leavevmode% starting standard par
	\hphantom{.}‚{\color{DodgerBlue3}‚व‚र्ण्णानामानुपूर्व्वी} प‚रिपाटिविशेषो ‚{\color{DodgerBlue3}‚वाक्यं\edtext{}{\edlabel{pvv.382-6}\label{pvv.382-6}\lemma{वाक्यं}\Bfootnote{न व‚र्ण‚व्य‚तिरिक्तं त‚च्चापौरुषेयं ।}}} । त‚च्चोप‚ल‚भ्य‚त एवेति ‚{\color{DodgerBlue3}‚चेत् । ‚{\tiny $_{lb}$}‚न व‚र्ण्णाना}‚मानुपूर्व्याया ‚{\color{DodgerBlue3}‚अभेद‚तः} । न हि व‚र्ण्णेभ्यो भिन्ना आनुपूर्व्वी प्र‚तीति‚{\tiny $_{lb}$}‚\leavevmode\ledsidenote{\textenglish{383/s}} विष‚यः । त‚त‚श्च व‚र्ण्णा एव वाक्य‚मिति स्यात् । तेषाञ्च लोक‚वेद‚योर्न विशेष ‚{\tiny $_{lb}$}‚इति स‚र्व्व‚त्र प्रामाण्यं । अथाप्रामाण्य‚म्वा स्यात् । अथ विशेषानुपूर्व्वीका व‚र्ण्णा ‚{\tiny $_{lb}$}‚एव वेद‚वाक्यं नेत‚र इत्याह \edtext{}{\edlabel{pvv.383-1}\label{pvv.383-1}\lemma{इत्याह}\Bfootnote{अथ क्र‚मो व‚र्ण्णानां ध‚र्म‚मात्रं न व‚स्त्व‚न्त‚रं त‚तोय‚म‚दोष इत्याह ।}} ‚{\color{DodgerBlue3}‚तेषा‚{\tiny $_{7}$}‚ञ्च} व‚र्ण्णानां ‚{\color{DodgerBlue3}‚न व्य‚व‚स्थान}\edtext{}{\edlabel{pvv.383-2}\label{pvv.383-2}\lemma{र्ण्णानां}\Bfootnote{न व्य‚व‚स्थित‚क्र‚म‚त्वं}}क्र‚म‚निय‚मः ।\leavevmode\ledsidenote{\textenglish{76a/MA}} ‚{\tiny $_{lb}$}‚‚{\color{DodgerBlue3}‚क्र‚मान्त‚र‚स्य} लौकिक‚स्य ‚{\color{DodgerBlue3}‚विरोध‚तः} । त‚था हि व‚र्ण्ण‚व‚दानुपूर्व्वी नित्या न च व‚र्ण्णा ‚{\tiny $_{lb}$}‚ब‚ह‚वः \edtext{}{\edlabel{pvv.383-3}\label{pvv.383-3}\lemma{वः}\Bfootnote{येन केन‚चिद् व्य‚व‚स्थि(त)क्र‚मा वैदिकाः स्युर‚न्ये लौकिका य‚थेष्ट‚प‚रावृत्त‚यः ।}} स‚न्ति । स‚मान‚जातीया वा एक‚त्वाद् व‚र्ण्ण‚स्य । त‚त‚श्चाग्निरित्येवा\edtext{}{\edlabel{pvv.383-4}\label{pvv.383-4}\lemma{श्चाग्निरित्येवा}\Bfootnote{क‚र्ण्ण एक‚त्र एवाकार‚ग‚कारो ग‚कार‚श्च क‚र‚णं ।}}‚{\tiny $_{lb}$}‚कार‚ग‚कार‚न‚काराणामानुपूर्व्वीविशिष्टः स्यात् न‚ग‚मित्य‚न्य‚था न भ‚वेत् । कृत‚{\tiny $_{lb}$}‚कानाम‚पि बीजाङ्त्कुर‚प‚त्रादीना\edtext{}{\edlabel{pvv.383-5}\label{pvv.383-5}\lemma{त्रादीना}\Bfootnote{हेम‚न्तादि ।}}मृतुसं\edtext{}{\edlabel{pvv.383-6}\label{pvv.383-6}\lemma{मृतुसं}\Bfootnote{शौक्र‚वार्ह‚स्प‚त्यादि आदिना ग्र‚हादि ।}}व‚त्स‚रादीनां विशिष्टानुपूर्वी नान्य‚था ‚{\tiny $_{lb}$}‚भ‚व‚ति किम्पुन‚र्नित्यानां ।
	\pend% ending standard par
      

	  \pstart \leavevmode% starting standard par
	सा चेय‚मानुपू‚{\tiny $_{1}$}‚र्व्वी व‚र्ण्णानां (।)
	\pend% ending standard par
      
	  \bigskip
	  \begingroup
	
	    \large
	  
	    \begin{quote}
	  
	    
	    \stanza[\smallbreak]
	\label{pv.3.260b}\flagstanza{\tiny\textenglish{...3.260b}}देश‚काल‚क्र‚माभावो व्याप्तिनित्य‚त्व‚व‚र्ण्ण‚नात् ॥ २६० ॥\&[\smallbreak]


	
	    \end{quote}
	  
	  \endgroup
	

	  \pstart \leavevmode% starting standard par
	\hphantom{.}‚{\color{DodgerBlue3}‚देश}‚कृता वा \edtext{}{\edlabel{pvv.383-7}\label{pvv.383-7}\lemma{वा}\Bfootnote{देश‚कालाभ्यां कृतो यः क्र‚म‚स्त‚स्य व‚र्ण्णेष्व‚भावः ।}} पिपीलिकानामिव पंक्ती स्यात् । ‚{\color{DodgerBlue3}‚काल}‚कृता वा \edtext{}{\edlabel{pvv.383-8}\label{pvv.383-8}\lemma{वा}\Bfootnote{बीज‚काले नाङ्कुर‚स्त‚त्काले न प‚त्रादि ।}} बीजाङ्कुरा‚{\tiny $_{lb}$}‚दीनामिव । द्व‚योर‚पि ‚{\color{DodgerBlue3}‚देश‚काल‚क्र‚म‚योर‚भावो} व‚र्ण्णानां ‚{\color{DodgerBlue3}‚व्याप्तिनित्य‚त्व‚योर्व्व‚र्ण्ण‚नात्} । ‚{\tiny $_{lb}$}‚अन्योन्य‚देश‚प‚रिहारेण । वृत्तिर्हि देश‚पौर्वाप‚र्य । त‚च्च स‚र्व्व‚गानाम‚स‚म्भ‚वि । त‚था‚{\tiny $_{lb}$}‚न्योन्य‚काल‚प‚रिहारेण वृत्तिः काल‚पौर्वाप‚र्य‚ञ्च नित्यानाम‚स‚म्भ‚वि । (२६०)
	\pend% ending standard par
      \label{div_pvv.3.261_3.262_3.263_3.264_3.265_3.266_3.267_3.268_3.269_3.270_3.271_3.272}
	  
	% new div opening: depth here is 2
	

	  \pstart \leavevmode% starting standard par
	अथानुपूर्व्वी स‚म‚र्थ‚नार्थ‚म‚नित्य‚ताऽव्या‚{\tiny $_{2}$}‚पितेष्य‚ते ।
	\pend% ending standard par
      
	  \bigskip
	  \begingroup
	
	    \large
	  
	    \begin{quote}
	  
	    
	    \stanza[\smallbreak]
	\label{pv.3.261a}\flagstanza{\tiny\textenglish{...3.261a}}अनित्याव्यापितायाञ्च दोषः प्रागेव कीर्त्तितः ।\&[\smallbreak]


	
	    \end{quote}
	  
	  \endgroup
	

	  \pstart \leavevmode% starting standard par
	\hphantom{.}‚{\color{DodgerBlue3}‚अनित्या\edtext{}{\edlabel{pvv.383-9}\label{pvv.383-9}\lemma{अनित्या}\Bfootnote{अनित्यं य‚त्न‚संभूतं पौरुषेयं क‚थ‚न्न त‚दि\href{http://sarit.indology.info/?cref=pv.3.251}{(३।२५१)}त्यादिना ।}}व्यापितायाञ्च दोषः प्रागेवोक्तः} । अनित्य‚त्वे पौरुषेय‚ता । ‚{\tiny $_{lb}$}‚अव्यापित्वे च स‚र्व्व‚त्रोप‚ल‚ब्धिश्च न स्यात्\edtext{}{\edlabel{pvv.383-10}\label{pvv.383-10}\lemma{स्यात्}\Bfootnote{न व‚र्ण्णानुपूर्व्वी वाक्यं येनायं दोषः स्यात् किन्तु व‚र्ण्ण‚व्य‚क्तेरित्याह ।}}।
	\pend% ending standard par
      
	  \bigskip
	  \begingroup
	
	    \large
	  
	    \begin{quote}
	  
	    
	    \stanza[\smallbreak]
	\label{pv.3.261b}\flagstanza{\tiny\textenglish{...3.261b}}व्य‚क्तिक्र‚मोपि वाक्यं न नित्य‚व्य‚क्तिनिराकृतेः ॥ २६१ ॥\&[\smallbreak]


	
	    \end{quote}
	  
	  \endgroup
	
	  \bigskip
	  \begingroup
	
	    \large
	  
	    \begin{quote}
	  
	    
	    \stanza[\smallbreak]
	\label{pv.3.262a}\flagstanza{\tiny\textenglish{...3.262a}}व्यापारादेव त‚त्सिद्धेः क‚र‚णानां च कार्य‚ता ।\&[\smallbreak]


	
	    \end{quote}
	  
	  \endgroup
	\textsuperscript{\textenglish{384/s}}

	  \pstart \leavevmode% starting standard par
	\hphantom{.}इति नित्य‚व्यापिनाम‚पि श‚ब्दानां ‚{\color{DodgerBlue3}‚व्य‚क्ते}‚र‚भिव्य‚क्ते\edtext{}{\edlabel{pvv.384-1}\label{pvv.384-1}\lemma{क्ते}\Bfootnote{य‚दा कार्य‚स्याक्रिया व्य‚क्तिः ।}} प्र‚तिनिय‚त‚देश‚कालायाः ‚{\tiny $_{lb}$}‚‚{\color{DodgerBlue3}‚क्र‚मो वाक्यं न} युक्तः । ‚{\color{DodgerBlue3}‚नित्य‚स्य व्य‚क्तिनि\edtext{}{\edlabel{pvv.384-2}\label{pvv.384-2}\lemma{क्तिनि}\Bfootnote{व्य‚ञ्ज‚क‚कृतेन साक्षाज्ज‚न‚न‚क्त्युप‚धानेन ज्ञान‚ज‚न‚नास‚म‚र्थानां कार्य‚विशेष एव व्य‚क्तिः ।}}राकृते}‚र्ज्ञानोत्पाद‚न‚हेतूनामित्या‚{\tiny $_{lb}$}‚दिना । त‚स्मात् ‚{\color{DodgerBlue3}‚क‚र‚णानां व्याप‚रादेव} ‚{\color{DodgerBlue3}‚ते}‚षाम्व‚र्ण्णानां सिद्धेः ‚{\color{DodgerBlue3}‚कार्य}‚तैषां युक्ता न ‚{\tiny $_{lb}$}‚व्य‚ङ्ग्य‚ता ।
	\pend% ending standard par
      

	  \pstart \leavevmode% starting standard par
	b. किञ्च (।)
	\pend% ending standard par
      
	  \bigskip
	  \begingroup
	
	    \large
	  
	    \begin{quote}
	  
	    
	    \stanza[\smallbreak]
	\label{pv.3.262b}\flagstanza{\tiny\textenglish{...3.262b}}स्व‚ज्ञानेनान्य‚धीहेतुः सिद्धेर्थे व्य‚ञ्च‚को म‚तः ॥ २६२ ॥\&[\smallbreak]


	
	    \end{quote}
	  
	  \endgroup
	
	  \bigskip
	  \begingroup
	
	    \large
	  
	    \begin{quote}
	  
	    
	    \stanza[\smallbreak]
	\label{pv.3.263a}\flagstanza{\tiny\textenglish{...3.263a}}य‚था दीपोन्य‚था वापि को विशेषोस्य कार‚कात् ।\&[\smallbreak]


	
	    \end{quote}
	  
	  \endgroup
	

	  \pstart \leavevmode% starting standard par
	\hphantom{.}‚{\color{DodgerBlue3}‚सिद्धे} विद्य‚मानेऽर्थे \edtext{}{\edlabel{pvv.384-3}\label{pvv.384-3}\lemma{मानेऽर्थे}\Bfootnote{स्व‚कार‚णादुत्प‚न्ने व्य‚ङ्ग्ये ।}} ‚{\color{DodgerBlue3}‚स्व‚ज्ञानेन} कार‚णेना\edtext{}{\edlabel{pvv.384-4}\label{pvv.384-4}\lemma{णेना}\Bfootnote{दीपः स्व‚ज्ञान‚द्वारा ध‚टं बोध‚य‚ति ।}} ‚{\color{DodgerBlue3}‚न्य}‚स्य ज्ञान‚हेतु‚{\color{DodgerBlue3}‚र्व्य‚ञ्ज‚को म‚तः । ‚{\tiny $_{lb}$}‚प्र‚दीपो} घ‚ट‚स्य ‚{\color{DodgerBlue3}‚य‚था । अन्य‚था वापीति} य‚दि व्य‚ङ्ग्यः प्राक् सिद्धो न भ‚वेत् त‚दास्य ‚{\tiny $_{lb}$}‚\edtext{}{\edlabel{pvv.384-5}\label{pvv.384-5}\lemma{दास्य}\Bfootnote{श‚ब्दो वा घ‚टादिव‚त् कार्य‚व‚त् कार्य एव ? यः प‚रार्थं प्र‚युज्य‚ते स प्र‚योगात् प्राग् विद्य‚मानो य‚था वाश्या(?वास्या)दि च्छिदायां । प्र‚युज्य‚ते च श‚ब्दः प‚र‚प्र‚त्याय‚नाय । प्र‚त्य‚भिज्ञ‚यापि सिद्धः ॥ ग्र‚हे क्ष‚णिकेपि क‚र्म‚णि प्र‚योगो दीपादौ च प्र‚त्य‚भिज्ञेत्य‚नेकान्ता एते ।}}व्य‚ञ्ज‚क‚स्य ‚{\color{DodgerBlue3}‚कार‚का}‚द्धेतोः ‚{\color{DodgerBlue3}‚को विशेषो} न क‚श्चित् । अपूर्व्व‚प्र‚तिप‚त्तिहेतुत्वा‚{\tiny $_{lb}$}‚विशेषात् ।
	\pend% ending standard par
      

	  \pstart \leavevmode% starting standard par
	c. त‚था (।)
	\pend% ending standard par
      
	  \bigskip
	  \begingroup
	
	    \large
	  
	    \begin{quote}
	  
	    
	    \stanza[\smallbreak]
	\label{pv.3.263b}\flagstanza{\tiny\textenglish{...3.263b}}क‚र‚णानां स‚म‚ग्राणां व्यापारादुप‚ल‚ब्धितः ॥ २६३ ॥\&[\smallbreak]


	
	    \end{quote}
	  
	  \endgroup
	
	  \bigskip
	  \begingroup
	
	    \large
	  
	    \begin{quote}
	  
	    
	    \stanza[\smallbreak]
	\label{pv.3.264a}\flagstanza{\tiny\textenglish{...3.264a}}निय‚मेन च कार्य‚त्वं व्य‚ञ्च‚के त‚द‚स‚म्भ‚वात् ।\&[\smallbreak]


	
	    \end{quote}
	  
	  \endgroup
	

	  \pstart \leavevmode% starting standard par
	क‚र‚णानां स‚म‚ग्राणां व्यापारात् निय‚मेनोप‚ल‚ब्धित‚श्च कार्य‚त्व‚मे‚{\tiny $_{4}$}‚व व‚र्ण्णानां ‚{\tiny $_{lb}$}‚व्य‚ञ्च‚के दीपादौ त‚स्य व्य‚ङ्ग्योप‚ल‚ब्धिनिय‚म‚स्यास‚म्भ‚वात् । न हि दीप इत्येव ‚{\tiny $_{lb}$}‚घ‚ट‚प्र‚तीतिः \edtext{}{\edlabel{pvv.384-6}\label{pvv.384-6}\lemma{तीतिः}\Bfootnote{घ‚ट‚शून्ये देशेऽनुप‚ल‚ब्धे ।}} क‚र‚ण‚साम‚ग्र्य‚न्तु कार्य‚म‚व‚श्य‚म्भाव‚य‚तीति क‚र‚ण‚साम‚ग्र्ये निय‚तो‚{\tiny $_{lb}$}‚प‚ल‚म्भ‚स्य कार्य‚तैव ।
	\pend% ending standard par
      
	  \bigskip
	  \begingroup
	
	    \large
	  
	    \begin{quote}
	  
	    
	    \stanza[\smallbreak]
	\label{pv.3.264b}\flagstanza{\tiny\textenglish{...3.264b}}d. त‚द्रूपाव‚र‚ण‚नां च व्य‚क्तिस्ते विग‚मो य‚दि ॥ २६४ ॥\&[\smallbreak]


	
	    \end{quote}
	  
	  \endgroup
	
	  \bigskip
	  \begingroup
	
	    \large
	  
	    \begin{quote}
	  
	    
	    \stanza[\smallbreak]
	\label{pv.3.265a}\flagstanza{\tiny\textenglish{...3.265a}}अभावे क‚र‚ण‚ग्राम‚साम‚र्थ्यं किं नु त‚द्भ‚वेत् ।\&[\smallbreak]


	
	    \end{quote}
	  
	  \endgroup
	\textsuperscript{\textenglish{385/s}}

	  \pstart \leavevmode% starting standard par
	\hphantom{.}‚{\color{DodgerBlue3}‚तेषा}‚म्व‚र्ण्णानां ‚{\color{DodgerBlue3}‚रूप}‚स्य स्तिमित‚वाय‚वीयाव‚य‚व‚संयोग‚रूपाणामा‚{\color{DodgerBlue3}‚व‚र‚णानां} प्र‚य‚त्न‚{\tiny $_{lb}$}‚प्रेरितेन वायुना ‚{\color{DodgerBlue3}‚विग‚मो} य‚दि ‚{\color{DodgerBlue3}‚व्य‚क्तिस्ते मी मां स क} स्येष्टा त‚दा‚{\tiny $_{5}$}‚ पूर्व्वाव‚स्थात्यागे ‚{\tiny $_{lb}$}‚नातिश‚यो न व्य‚क्तिर‚नित्य‚त्वास‚क्तेः । उप‚ल‚म्भाव‚र‚ण‚विग‚मो वा श‚ब्दाल‚म्ब‚नं ‚{\tiny $_{lb}$}‚ज्ञान‚म्वा व्य‚क्तिः स्यादित्य‚त्राह कार्यं व्य‚क्तिर‚श‚क्या य‚स्मादाव‚र‚ण‚विग‚मेऽभावे ‚{\tiny $_{lb}$}‚नीरूपे ‚{\color{DodgerBlue3}‚क‚र‚ण‚ग्राम‚स्य किं नु} त‚त् ‚{\color{DodgerBlue3}‚साम‚र्थ्यं भ‚वेत्} । क्व‚चित् क‚र्त्त‚व्ये साम‚र्थ्यं स्यात् । ‚{\tiny $_{lb}$}‚न तु क‚र्त्त‚व्याभावे \edtext{}{\edlabel{pvv.385-1}\label{pvv.385-1}\lemma{व्याभावे}\Bfootnote{नित्य‚स्यानाधेयातिश‚य‚त्वान्नाव‚र‚ण‚मित्युक्तेश्च ।}} । य‚दि स‚म‚स्त‚का‚{\color{DodgerBlue3}‚त्व‚र्य} (? र्य‚त्व)स‚म्भ‚वेपि श‚ब्दानां न ‚{\tiny $_{lb}$}‚कार्य‚ता त‚दा (।)
	\pend% ending standard par
      
	  \bigskip
	  \begingroup
	
	    \large
	  
	    \begin{quote}
	  
	    
	    \stanza[\smallbreak]
	\label{pv.3.265b}\flagstanza{\tiny\textenglish{...3.265b}}श‚ब्दाविशेषाद‚न्येषाम‚पि व्य‚क्तिः प्र‚स‚ज्य‚ते ॥ २६५ ॥\&[\smallbreak]


	
	    \end{quote}
	  
	  \endgroup
	
	  \bigskip
	  \begingroup
	
	    \large
	  
	    \begin{quote}
	  
	    
	    \stanza[\smallbreak]
	\label{pv.3.266a}\flagstanza{\tiny\textenglish{...3.266a}}त‚थाभ्युप‚ग‚मे स‚र्व्व‚कार‚काणां निर‚र्थ‚ता ।\&[\smallbreak]


	
	    \end{quote}
	  
	  \endgroup
	

	  \pstart \leavevmode% starting standard par
	\hphantom{.}‚{\color{DodgerBlue3}‚श‚ब्दाद‚विशेषा\edtext{}{\edlabel{pvv.385-2}\label{pvv.385-2}\lemma{विशेषा}\Bfootnote{न किञ्चिद‚पि कार्यं स्यात् ।}}द‚न्य}‚व्यापारान्त‚रादुप‚ल‚भ्य‚मान‚त्व‚निय‚मेनान्येषां घ‚टादीना‚{\color{DodgerBlue3}‚म‚पि} ‚{\color{DodgerBlue3}‚व्य‚क्तिः प्र‚स‚ज्येत} कार्य‚ता न स्यात् । \edtext{\textsuperscript{*}}{\edlabel{pvv.385-3}\label{pvv.385-3}\lemma{*}\Bfootnote{स‚र्व्व‚स्य व्य‚ङ्ग्य‚त्व‚मिष्ट‚ञ्चेत् ।}} ‚{\color{DodgerBlue3}‚त‚थाभ्युप‚ग‚मे} च \edtext{}{\edlabel{pvv.385-4}\label{pvv.385-4}\lemma{च}\Bfootnote{व्य‚क्तेः प‚क्ष‚त्र‚यान‚तिक्र‚मात् त‚स्यात्रासंभ‚वाज्ञान‚स्य सिद्ध‚त्वात् ।}} ‚{\color{DodgerBlue3}‚स‚र्व्वेषां कार‚काणां} व्य‚ञ्च\edtext{}{\edlabel{pvv.385-5}\label{pvv.385-5}\lemma{ञ्च}\Bfootnote{व्य‚ञ्ज‚क‚विक‚ल्प‚त्र‚ये द्व‚यं निर‚स्तं । न ज्ञान‚क‚र‚णात् स‚त्कार्य‚का (?वा) दे व‚स्तुव‚त् । अस‚त्कार्ये तु स‚र्व्व‚स्य कार्य‚तास‚क्तिर्ज्ञान‚व‚त् स्यात् ।}}काभिम‚तानां ‚{\color{DodgerBlue3}‚निर‚र्थ‚क‚ता} । उत्पा‚{\tiny $_{6}$}‚द‚कं हि कार‚ण‚भिष्य‚ते न तु व्य‚ञ्ज‚कं ‚{\tiny $_{lb}$}‚श‚ब्दानामिव ।
	\pend% ending standard par
      
	  \bigskip
	  \begingroup
	
	    \large
	  
	    \begin{quote}
	  
	    
	    \stanza[\smallbreak]
	\label{pv.3.266b}\flagstanza{\tiny\textenglish{...3.266b}}e. साध‚नं प्र‚त्य‚भिज्ञानं स‚त्प्र‚योगादि य‚न्म‚त‚म् ॥ २६६ ॥\&[\smallbreak]


	
	    \end{quote}
	  
	  \endgroup
	
	  \bigskip
	  \begingroup
	
	    \large
	  
	    \begin{quote}
	  
	    
	    \stanza[\smallbreak]
	\label{pv.3.267a}\flagstanza{\tiny\textenglish{...3.267a}}अनुदाह‚र‚णं स‚र्व्व‚भावानां क्ष‚ण‚भ‚ङ्ग‚तः ।\&[\smallbreak]


	
	    \end{quote}
	  
	  \endgroup
	

	  \pstart \leavevmode% starting standard par
	य‚च्च\edtext{}{\edlabel{pvv.385-6}\label{pvv.385-6}\lemma{च्च}\Bfootnote{किञ्च ।}} नित्य‚त्व‚सिद्ध‚ये व‚र्ण्णानां ‚{\color{DodgerBlue3}‚प्र‚त्य‚भिज्ञानं\edtext{}{\edlabel{pvv.385-7}\label{pvv.385-7}\lemma{भिज्ञानं}\Bfootnote{अष्ट‚कृत्वो गोश‚ब्द उच्चारितो नाष्टौ गोश‚ब्दाः । य‚त्प्र‚युज्य‚ते त‚त्प्राक् स‚द् य‚था वास्यादि । य‚त्प‚रार्थ प्र‚त्याय‚त्त‚मुच्चार्य‚ते । उच्च‚रित‚प्र‚ध्वंसित्वे प‚र‚प्र‚त्याय‚नं न स्यात् ।}} स‚त्प्र‚योग} आदिश‚ब्दादुच्चा‚{\tiny $_{lb}$}‚र्य‚माण‚त्वादि ‚{\color{DodgerBlue3}‚साध‚नं म‚तं} स‚न्नेव हि प्र‚युज्य‚ते । य‚था वास्यादि च्छिदायां (।) त‚तः ‚{\tiny $_{lb}$}‚प्र‚योगात् प्र‚योगेपि विद्य‚मान‚त्वाद् व‚र्ण्णा नित्या एव । त‚{\color{DodgerBlue3}‚द‚नुदाह‚र‚णं} दृष्टान्त‚विक‚लं\edtext{}{\edlabel{pvv.385-8}\label{pvv.385-8}\lemma{लं}\Bfootnote{विनाश‚स्याकार‚ण‚त्वादिना ।}} ‚{\tiny $_{lb}$}‚‚{\color{DodgerBlue3}‚स‚र्व्वे}‚षा‚{\color{DodgerBlue3}‚म्भावानां क्ष‚ण‚भ‚ङ्ग‚तः} ।
	\pend% ending standard par
      
	  \bigskip
	  \begingroup
	
	    \large
	  
	    \begin{quote}
	  
	    
	    \stanza[\smallbreak]
	\label{pv.3.267b}\flagstanza{\tiny\textenglish{...3.267b}}दूष्यः कुहेतुर‚न्योपि ;\&[\smallbreak]


	
	    \end{quote}
	  
	  \endgroup
	

	  \pstart \leavevmode% starting standard par
	\hphantom{.}अन‚यैव दिशा प‚रैरुच्य‚मानः ‚{\color{DodgerBlue3}‚कुहेतुर‚{\tiny $_{7}$}‚}‚न्यो\edtext{}{\edlabel{pvv.385-9}\label{pvv.385-9}\lemma{न्यो}\Bfootnote{........... साध‚नार्थः ।}}पि दूष्यः ।\leavevmode\ledsidenote{\textenglish{76b/MA}}
	\pend% ending standard par
      \textsuperscript{\textenglish{386/s}}

	  \pstart \leavevmode% starting standard par
	f. अथ बुद्धिर‚भिव्य‚क्तिर्व्व‚र्ण्णानां सा च क्र‚म‚व‚ती वाक्य‚मिष्टं त‚दा\edtext{}{\edlabel{pvv.386-1}\label{pvv.386-1}\lemma{दा}\Bfootnote{......ज्ञानं व्य‚क्तिरित्य‚त्र बुद्धीनां अनुक्र‚मो वाक्याव्य‚क्तीनां क्र‚म‚व‚त्वात् । त‚च्चायुक्त‚न्न बुद्धिरूप‚त्वाद् वाक्य‚स्य अभ्युप‚ग‚म्य दोष‚माह ।}} वाक्य‚{\tiny $_{lb}$}‚स्यापौरुषेय‚त्वेनेष्ट‚त्वाद् बुद्धिर‚पौरुषेयी स्यात् । त‚त्र (।)
	\pend% ending standard par
      
	  \bigskip
	  \begingroup
	
	    \large
	  
	    \begin{quote}
	  
	    
	    \stanza[\smallbreak]
	\label{pv.3.267c}\flagstanza{\tiny\textenglish{...3.267c}}बुद्धेर‚पुरुषाश्र‚ये ॥ २६७ ॥\&[\smallbreak]


	
	    \end{quote}
	  
	  \endgroup
	
	  \bigskip
	  \begingroup
	
	    \large
	  
	    \begin{quote}
	  
	    
	    \stanza[\smallbreak]
	\label{pv.3.268a}\flagstanza{\tiny\textenglish{...3.268a}}बाधाभ्युपेत‚प्र‚त्य‚क्ष‚प्र‚तीतानुमितैः स‚म‚म् ।\&[\smallbreak]


	
	    \end{quote}
	  
	  \endgroup
	

	  \pstart \leavevmode% starting standard par
	\hphantom{.}‚{\color{DodgerBlue3}‚बुद्धेर‚पुरुषाश्र‚ये} पुरुषानाश्र‚य‚णे‚{\color{DodgerBlue3}‚\edtext{\textsuperscript{*}}{\edlabel{pvv.386-2}\label{pvv.386-2}\lemma{*}\Bfootnote{साध्ये प्र‚तिज्ञायाः ।}}बाधा} । कैरित्याह । ‚{\color{DodgerBlue3}‚अभ्युपेत‚प्र‚त्य‚क्ष‚प्र‚तीतानुमितैः ‚{\tiny $_{lb}$}‚स‚मं} एक‚काल‚मेव बुद्धेर‚पुरुषाश्र‚य‚त्व‚स्याभ्युपेतेन पुरुष\edtext{}{\edlabel{pvv.386-3}\label{pvv.386-3}\lemma{पुरुष}\Bfootnote{तीर्थ‚स्य ।}}गुण‚त्वाभ्युप‚ग‚मेन ‚{\tiny $_{lb}$}‚प्र‚त्य‚क्ष‚प्र‚तीतेन च पुरुष‚कार्य\edtext{}{\edlabel{pvv.386-4}\label{pvv.386-4}\lemma{कार्य}\Bfootnote{भाव‚स्यैव कार्य‚त्वात् ।}}त्वेन । त‚त्प्र‚य‚त्न‚का\edtext{}{\edlabel{pvv.386-5}\label{pvv.386-5}\lemma{का}\Bfootnote{त‚द्भाव‚भावित्वात् ।}} ‚{\tiny $_{1}$}‚ र्य‚त्वेन वा कादाचित्क‚त्वा‚{\tiny $_{lb}$}‚नुमितेन कार्य‚त्वेन च बाधा ।
	\pend% ending standard par
      

	  \pstart \leavevmode% starting standard par
	g. व‚र्ण्णानामानुपूर्व्वी वाक्यं स्यादित्याह ।
	\pend% ending standard par
      
	  \bigskip
	  \begingroup
	
	    \large
	  
	    \begin{quote}
	  
	    
	    \stanza[\smallbreak]
	\label{pv.3.268b}\flagstanza{\tiny\textenglish{...3.268b}}आनुपूर्व्याश्च व‚र्ण्णेभ्यो भेदः स्फोटेन चिन्तितः ॥ २६८ ॥\&[\smallbreak]


	
	    \end{quote}
	  
	  \endgroup
	
	  \bigskip
	  \begingroup
	
	    \large
	  
	    \begin{quote}
	  
	    
	    \stanza[\smallbreak]
	\label{pv.3.269a}\flagstanza{\tiny\textenglish{...3.269a}}क‚ल्प‚नारोपिता सा स्यात् क‚थं वाऽपुरुषाश्र‚या ।\&[\smallbreak]


	
	    \end{quote}
	  
	  \endgroup
	

	  \pstart \leavevmode% starting standard par
	\hphantom{.}‚{\color{DodgerBlue3}‚व‚र्ण्णेभ्य आनुपूर्व्या भेद‚श्च स्फोटेन} चिन्तितेन एक‚त्वेन ह्य‚भिन्न‚स्य क्र‚म‚श ‚{\tiny $_{lb}$}‚इत्यादिना ‚{\color{DodgerBlue3}‚चिन्तितः} । न हि व‚र्ण्णेभ्यो व्य‚तिरिक्ता आनुपूर्व्वी काचिदुप‚ल‚भ्य‚ते । ‚{\tiny $_{lb}$}‚अभेद‚प‚क्षे च स‚रो र‚स इत्यादौ प्र‚तिप‚त्तेर्भेदाभाव‚प्र‚स‚ङ्गः । प्र‚कारान्त‚र‚स्य चाभावः । ‚{\tiny $_{lb}$}‚त‚स्माद् व‚स्तुभूतानुपूर्व्याऽयोगात् ‚{\color{DodgerBlue3}‚क‚ल्प‚ना}‚{\tiny $_{2}$}‚‚{\color{DodgerBlue3}‚रोपिता} सा ‚{\color{DodgerBlue3}‚स्यात्} । त‚था ‚{\color{DodgerBlue3}‚चापुरुषाश्र‚या ‚{\tiny $_{lb}$}‚क‚थ‚मुच्य‚ते} ।
	\pend% ending standard par
      

	  \begin{center}%% label @type='head'
	\textbf{(ङ) निर्हेतुको विनाशः}
	\end{center}
	

	  \pstart \leavevmode% starting standard par
	क‚थं पुन‚र‚व‚ग‚म्य‚ते ध्व‚निर‚व‚श्य‚म‚नित्य इत्याह ।
	\pend% ending standard par
      
	  \bigskip
	  \begingroup
	
	    \large
	  
	    \begin{quote}
	  
	    
	    \stanza[\smallbreak]
	\label{pv.3.269b}\flagstanza{\tiny\textenglish{...3.269b}}स‚त्तामात्रानुब‚न्धित्वात् नाश‚स्यानित्य‚ता ध्व‚नेः ॥ २६९ ॥\&[\smallbreak]


	
	    \end{quote}
	  
	  \endgroup
	
	  \bigskip
	  \begingroup
	
	    \large
	  
	    \begin{quote}
	  
	    
	    \stanza[\smallbreak]
	\label{pv.3.270a}\flagstanza{\tiny\textenglish{...3.270a}}अविनाशात् स एवास्य विनाश इति चेत् क‚थ‚म् ।&अन्योर्थोन्य‚स्य नाशोस्तु काष्ठं क‚स्मान्न दृश्य‚ते ॥ २७० ॥\&[\smallbreak]


	
	    \end{quote}
	  
	  \endgroup
	

	  \pstart \leavevmode% starting standard par
	\hphantom{.}‚{\color{DodgerBlue3}‚स‚त्तामात्रानुब‚न्धित्वान्नाश‚स्य ध्व‚नेर‚नित्य‚ता} । न ख‚ल्व‚स‚ताम‚न्य‚स्मान्ना‚{\tiny $_{lb}$}‚‚{\color{DodgerBlue3}‚शोत्प‚त्तिः} स्व‚हेतोरेव तु विन‚श्व‚र‚स्व‚भाव‚त‚योत्प‚न्ना भावा विन‚श्य‚न्ति । \edtext{\textsuperscript{*}}{\edlabel{pvv.386-6}\label{pvv.386-6}\lemma{*}\Bfootnote{अभ्युप‚ग‚म्याह ।}} विनाशो ‚{\tiny $_{lb}$}‚हि क्रिय‚माणो भावाद् व्य‚तिरिक्तो वा भ‚वेत् । अव्य‚तिरेक‚प‚क्षे भाव एव क्रिय‚त ‚{\tiny $_{lb}$}‚\leavevmode\ledsidenote{\textenglish{387/s}} इति स्यात् । त‚च्चाश‚{\tiny $_{3}$}‚क्य‚क्रिय‚मुत्प‚न्न‚त्वात् । व्य‚तिरेकेप्य‚ग्नेः स‚काशात् अर्था‚{\tiny $_{lb}$}‚न्त‚र‚स्य विनाशाख्य‚स्योत्प‚त्तौ काष्ठ‚स्य द‚र्श‚न‚म्भ‚{\color{DodgerBlue3}‚वेद‚विनाशात्} ।
	\pend% ending standard par
      

	  \pstart \leavevmode% starting standard par
	स एवाग्निज‚न्माऽर्थोऽस्य\edtext{}{\edlabel{pvv.387-1}\label{pvv.387-1}\lemma{न्माऽर्थोऽस्य}\Bfootnote{काष्ठ‚स्याभावोऽभूत‚त्वान्न दृश्य‚ते ।}} ‚{\color{DodgerBlue3}‚विनाश‚स्तेनाद‚र्श‚न‚मिति चेत्\edtext{}{\edlabel{pvv.387-2}\label{pvv.387-2}\lemma{चेत्}\Bfootnote{भ‚व‚त्व‚ग्निजार्थ‚स्याभाव इति नाम त‚थापि ।}} क‚थ‚म‚न्योर्थोन्य‚स्य} विनाशो युक्तः । एवं ह्य‚ति\edtext{}{\edlabel{pvv.387-3}\label{pvv.387-3}\lemma{ति}\Bfootnote{स‚र्व्वे प‚दार्थाः काष्ठ‚स्य विनाशः स्युः ।}}प्र‚स‚ङ्गः स्यात् । काष्ठेऽग्निकृतः स्व‚भावो\edtext{}{\edlabel{pvv.387-4}\label{pvv.387-4}\lemma{भावो}\Bfootnote{त‚तो नातिप्र‚स‚ङ्गः ।}} विनाशो ‚{\tiny $_{lb}$}‚न स‚र्व्व इति चेत् । काष्ठ‚विनाश‚योः कः स‚म्ब‚न्धः । नाश्र‚याश्र‚यिभावो निषेत्स्य‚{\tiny $_{lb}$}‚मान‚त्वा‚{\tiny $_{4}$}‚त् । कार्य‚कार‚ण‚भाव‚श्चेत् अग्नेर‚पि स विनाशः स्यात् । त‚स्यापि ‚{\tiny $_{lb}$}‚कार्य‚त्वात् । त‚स्मान्न भावान्त‚र‚म‚र्थ‚स्य नाशः (।) ‚{\color{DodgerBlue3}‚अस्तु वा नाशः काष्ठ‚ञ्चेद-} प्र‚च्युत‚प्राचीन‚स्व‚भावं ‚{\color{DodgerBlue3}‚क‚स्मान्न दृश्य‚ते} ।
	\pend% ending standard par
      

	  \pstart \leavevmode% starting standard par
	न‚नु योसाव‚र्थान्त‚र‚स्व‚भावो व‚ह्निकृतः\edtext{}{\edlabel{pvv.387-5}\label{pvv.387-5}\lemma{ह्निकृतः}\Bfootnote{तेन प‚रिग्र‚हात् स्वीकारात् काष्ठं न दृश्य‚ते ।}} स काष्ठ‚स्य नाशो नाश‚रूप‚त‚या ‚{\tiny $_{lb}$}‚प्र‚तीतेः । विनाश‚श्चाभावो य‚श्चाभावः स काष्ठ‚विरोधिरूप एव क्रिय‚ते । ‚{\tiny $_{lb}$}‚न चाय‚म‚र्थान्त‚र‚त्वाद् घ‚ट‚व‚द् वि‚{\tiny $_{5}$}‚रोधिरूप‚ता कुर्त्तुम‚श‚क्यः । न हि घ‚ट‚व‚द‚{\tiny $_{lb}$}‚र्थान्त‚र‚त्वाद् धूमोऽग्निकार्यो न भ‚व‚ति । त‚स्माद् य‚थान्त‚र‚रूपोपि धूमोऽग्निना ‚{\tiny $_{lb}$}‚क्रिय‚ते त‚था विरोधिरूपोपि विनाशीक्रियेत । य‚योश्च प‚र‚स्प‚र‚प‚रिहारेण विरोध‚{\tiny $_{lb}$}‚स्त‚योरेक‚भाव एवाप‚र‚स्प‚र‚स्याद‚र्श‚न‚मिति क‚थ‚म‚ग्निकृत‚स्यार्थान्त‚र‚स्य विनाश‚{\tiny $_{lb}$}‚संज्ञित‚स्य विरोधिनो भावे काष्ठ‚स्य द‚र्श‚न‚मुच्य‚ते ।
	\pend% ending standard par
      

	  \pstart \leavevmode% starting standard par
	अत्रोच्य‚ते । योसा‚{\tiny $_{6}$}‚व‚र्थान्त‚र‚स्व‚भावो नाश‚स्तेन स‚ह काष्ठ‚स्य को विरोधः ।\leavevmode\ledsidenote{\textenglish{77a/MA}} ‚{\tiny $_{lb}$}‚य‚दि स‚हान‚व‚स्थान‚ल‚क्ष‚णः स‚म्भाव्य‚त एव । त‚द्भाव‚योर्निव‚र्त्त्य‚निव‚र्त्त‚क‚भाव‚द‚र्श‚ना‚{\tiny $_{lb}$}‚द‚ग्निशीत‚योरिव । किन्तु भावान्त‚र‚स्य य‚दि काष्ठ‚विनाश‚क‚त्वं । स च नाशः ‚{\tiny $_{lb}$}‚किम‚र्थान्त‚र‚निवृत्तिः । अर्थान्त‚र‚त्वे तुल्यः प्र‚स‚ङ्गः ।\edtext{\textsuperscript{*}}{\edlabel{pvv.387-6}\label{pvv.387-6}\lemma{*}\Bfootnote{त‚दुत्पादेपि काष्ठं त‚थैव दृश्य‚ते ।}} निवृत्तिश्च निःस्व‚भावा ‚{\tiny $_{lb}$}‚न त‚त्र हेतुव्यापार इति व‚क्ष्य‚ते । प‚र‚स्प‚र‚प‚रिहार‚स्थितिल‚{\tiny $_{7}$}‚क्ष‚ण‚श्चेद् विरोधः । ‚{\tiny $_{lb}$}‚एव‚म‚प्य‚तिप्र‚स‚ङ्गः । य‚था काष्ठाद् भिन्न‚म‚भिम‚त‚प‚दार्थान्त‚रं त‚थान्य‚द‚पि घ‚टा‚{\tiny $_{lb}$}‚दिक‚मिति त‚द‚पि काष्ठ‚स्य विनाशः स्यात् ।
	\pend% ending standard par
      

	  \pstart \leavevmode% starting standard par
	न‚नु य एव काष्ठ‚स्य नाश‚रूप‚त‚या विरोधिरूप‚त‚या च स्व‚हेतुभिः क्रिय‚ते ‚{\tiny $_{lb}$}‚स एव नाशो न तु यः क‚श्चिद‚र्थ इति क‚थ‚म‚तिप्र‚स‚क्तिः ।
	\pend% ending standard par
      \textsuperscript{\textenglish{388/s}}

	  \pstart \leavevmode% starting standard par
	त‚द‚प्य‚युक्तं । प‚र‚स्प‚र‚प‚रिहारेण हि विरोधी नाश इष्टं । स च प‚र‚स्प‚र‚{\tiny $_{lb}$}‚प‚रिहारोऽभिम‚त‚प‚दार्थ‚व‚द् घ‚टा‚{\tiny $_{1}$}‚दीनाम‚पीति न विशेषः । अथ य एव काष्ठ‚निवृत्ति‚{\tiny $_{lb}$}‚रूपः स एव त‚न्नाशो नान्यः । किमिदं निवृत्तिरूप‚त्वं काष्ठाद‚न्य‚त्वं निवृत्तिमात्रा‚{\tiny $_{lb}$}‚त्म‚क‚त्व‚म्वा । अन्य‚त्व‚ञ्चेत् त‚दित‚र‚स्यापि स‚मानं । निवृत्तिमात्रात्म‚क‚त्व‚ञ्च ‚{\tiny $_{lb}$}‚भावान्त‚र‚स्यायुक्तं स्व‚भावाविशेष‚व‚त्वात् । य‚दुत्प‚त्तौ य‚न्निवृत्तिः स विरोधी नाश‚{\tiny $_{lb}$}‚श्चेति चेत् । अन्या त‚र्हि विनाशान्निवृत्तिः । त‚त्र च स‚मानः स‚र्व्व एव प्र‚स‚ङ्गः । ‚{\tiny $_{lb}$}‚त‚स्मात् काष्ठं स्व‚{\tiny $_{2}$}‚र‚स‚निरोधित‚या निव‚र्त्त‚तेऽग्निकाष्ठादिसाम‚ग्र‚यास्त्व‚ङ्गारादिकं ‚{\tiny $_{lb}$}‚जाय‚त इति युक्तं । अर्थान्त‚र‚स्व‚भावे तु स‚ति नाशे काष्ठं क‚स्मान्न दृश्य‚त ‚{\tiny $_{lb}$}‚इत्य‚निवार्यः प्र‚स‚ङ्गः ।
	\pend% ending standard par
      
	  \bigskip
	  \begingroup
	
	    \large
	  
	    \begin{quote}
	  
	    
	    \stanza[\smallbreak]
	\label{pv.3.271a}\flagstanza{\tiny\textenglish{...3.271a}}त‚त्प‚रिग्र‚ह‚त‚श्चेन्न तेनानाव‚र‚णं य‚तः ।&विनाश‚स्य विनाशित्वं ;\&[\smallbreak]


	
	    \end{quote}
	  
	  \endgroup
	

	  \pstart \leavevmode% starting standard par
	\hphantom{.}‚{\color{DodgerBlue3}‚तेन} विनाशाख्येन भावान्त‚रेण ‚{\color{DodgerBlue3}‚प‚रिग्र‚ह‚तो}‚व‚ष्ट‚ब्ध‚त्वात् काष्ठाद‚र्श‚न‚मिति ‚{\color{DodgerBlue3}‚चेत् । ‚{\tiny $_{lb}$}‚न तेन} भावान्त‚रेण काष्ठ‚स्या‚{\color{DodgerBlue3}‚नाव‚र‚णं य‚त}‚स्त‚तः काष्ठ‚स्याप‚रिग्र‚हः । न हि नाशो ‚{\tiny $_{lb}$}‚व‚स्त्वाव‚र‚ण‚म‚{\color{DodgerBlue3}‚विनाशित्व‚प्र}‚स‚ङ्गा‚{\tiny $_{3}$}‚त् । न च पूर्व्वाप‚रैक‚स्व‚भाव‚स्य‚व‚र‚णं युक्त‚{\tiny $_{lb}$}‚मित्युक्तं । किञ्च (।)
	\pend% ending standard par
      
	  \bigskip
	  \begingroup
	
	    \large
	  
	    \begin{quote}
	  
	    
	    \stanza[\smallbreak]
	\label{pv.3.271b}\flagstanza{\tiny\textenglish{...3.271b}}स्यादुत्प‚त्तेस्त‚तः पुनः ॥ २७१ ॥\&[\smallbreak]


	
	    \end{quote}
	  
	  \endgroup
	
	  \bigskip
	  \begingroup
	
	    \large
	  
	    \begin{quote}
	  
	    
	    \stanza[\smallbreak]
	\label{pv.3.272a}\flagstanza{\tiny\textenglish{...3.272a}}काष्ठ‚स्य द‚र्श‚नं ;\&[\smallbreak]


	
	    \end{quote}
	  
	  \endgroup
	

	  \pstart \leavevmode% starting standard par
	भावान्त‚र‚भूत‚स्य नाश‚स्य य‚द्युत्प‚त्तिरिष्य‚ते त‚दोत्प‚त्तिलिङ्गाद् विनाशित्वं ‚{\tiny $_{lb}$}‚नाश\edtext{}{\edlabel{pvv.388-1}\label{pvv.388-1}\lemma{नाश}\Bfootnote{य‚दुत्प‚त्तिम‚त् त‚द्विन‚श्व‚रं ।}}स्य स्यात् घ‚टादिव‚त् । त‚तः काष्ठ‚नाश‚स्य नाशात् पुनः काष्ठ‚द‚र्श‚नं ‚{\tiny $_{lb}$}‚भ‚वेत् ।
	\pend% ending standard par
      
	  \bigskip
	  \begingroup
	
	    \large
	  
	    \begin{quote}
	  
	    
	    \stanza[\smallbreak]
	\label{pv.3.272b}\flagstanza{\tiny\textenglish{...3.272b}}ह‚न्तृघाते चैत्रापुन‚र्भ‚वः ।&य‚थात्राप्येव‚मिति चेत् ह‚न्तुर्न्नाम‚र‚ण‚त्व‚तः ॥ २७२ ॥\&[\smallbreak]


	
	    \end{quote}
	  
	  \endgroup
	

	  \pstart \leavevmode% starting standard par
	\hphantom{.}न‚नु ‚{\color{DodgerBlue3}‚चैत्र‚स्य} ह‚न्तुर्व्याघाते कृते चैत्र‚स्य ‚{\color{DodgerBlue3}‚पुन‚र्भावो} नास्ति ‚{\color{DodgerBlue3}‚य‚था}‚, त‚था‚{\color{DodgerBlue3}‚त्रापि} काष्ठ‚{\tiny $_{lb}$}‚नाश‚निवृत्त्या न काष्ठ‚पुन‚र्भाव ‚{\color{DodgerBlue3}‚इति चेत् । न ह‚न्तुर‚म‚र‚ण‚त्व‚तः ।‚{\tiny $_{4}$}‚} न हि ह‚न्ता ‚{\tiny $_{lb}$}‚म‚र‚णं चैत्र‚स्य । किन्त्व‚न्य एवेन्द्रियायुर्निरोधः । येन ह‚न्तृम‚र‚णे चैत्र‚पुन‚रुज्जीव‚न‚{\tiny $_{lb}$}‚प्र‚स‚ङ्गः । य‚दि त्विन्द्रियादिनिरोध‚निवृत्तिः स्यात् स्यादेवोज्जीव‚नं त‚च्च त्व‚न्म‚ते ‚{\tiny $_{lb}$}‚प्राप्तं । निरोध‚स्योत्प‚त्तिभाव‚योरिष्टः । (२७२)
	\pend% ending standard par
      \label{div_pvv.3.273}
	  
	% new div opening: depth here is 2
	\textsuperscript{\textenglish{389/s}}
	  \bigskip
	  \begingroup
	
	    \large
	  
	    \begin{quote}
	  
	    
	    \stanza[\smallbreak]
	\label{pv.3.273}\flagstanza{\tiny\textenglish{....3.273}}अन‚न्य‚त्वे विनाश‚स्य स्यान्नाशः काष्ठ‚मेव तु ।&त‚स्यास‚त्वाद‚हेतुत्वं नातोन्या विद्येते ग‚तिः ॥ २७३ ॥\&[\smallbreak]


	
	    \end{quote}
	  
	  \endgroup
	

	  \pstart \leavevmode% starting standard par
	अथ न भावाद्भिन्नो नाशः किन्त‚र्ह्य‚भिन्नः । अन‚न्य‚त्वे विनाश‚स्य नाशः ‚{\tiny $_{lb}$}‚काष्ठ‚मेव तु स्यात् (।) त‚स्य काष्ठ\edtext{}{\edlabel{pvv.389-1}\label{pvv.389-1}\lemma{काष्ठ}\Bfootnote{अहेतोरुत्प‚न्न‚स्य न व‚ह‚न्यादिभिः किञ्चित्क‚र्त्त‚व्य‚मिति नाशोऽस्तु ।}}स्याग्निस‚न्निधानात् (।) प्रागेव स‚त्वाद‚{\tiny $_{lb}$}‚हेतुत्वं ‚{\tiny $_{5}$}‚ नाश‚काभिम‚त‚स्य । नातो भेदाभेद‚प्र‚काराद‚न्या ग‚तिरुत्प‚त्तिम‚तोस्ति । ‚{\tiny $_{lb}$}‚द्विधापि च नाश‚हेत्व‚योगः \edtext{}{\edlabel{pvv.389-2}\label{pvv.389-2}\lemma{योगः}\Bfootnote{प्र‚ध्वंसाभावं नाशं गृहीत्वा प‚र‚तोद्य‚माश‚ङ्क‚तेऽहेतुहिते क्ष‚णिक‚वादिनः नित्यं भ‚वेत् भाव‚कालेपीति स‚ह‚भावः स्यात् ।}}। (२७३)
	\pend% ending standard par
      \label{div_pvv.3.274}
	  
	% new div opening: depth here is 2
	
	  \bigskip
	  \begingroup
	
	    \large
	  
	    \begin{quote}
	  
	    
	    \stanza[\smallbreak]
	\label{pv.3.274}\flagstanza{\tiny\textenglish{....3.274}}अहेतुत्वेपि नाश‚स्य नित्य‚त्वाद् भाव‚नाश‚योः ।&स‚ह‚भाव‚प्र‚स‚ङ्ग‚श्चेद‚स‚तो नित्य‚ता कुतः ॥ २७४ ॥\&[\smallbreak]


	
	    \end{quote}
	  
	  \endgroup
	

	  \pstart \leavevmode% starting standard par
	\hphantom{.}‚{\color{DodgerBlue3}‚अहेतुत्वेपि नाश‚स्य नित्य‚त्वादा}‚काशादिव‚त् ‚{\color{DodgerBlue3}‚भाव‚नाश‚यो}‚र‚न्योन्याभाव‚स्व‚भा‚{\tiny $_{lb}$}‚व‚योः ‚{\color{DodgerBlue3}‚स‚ह‚भाव‚प्र‚स‚ङ्ग‚श्चेत्} (।) न‚नु नाश‚स्यास‚तो नीरूप‚त्वान्नित्य‚ता कुतः । ‚{\tiny $_{lb}$}‚व‚स्तु हि नित्य‚म‚नित्य‚म्वा स्यात् । य‚त्तु न किञ्चित् त‚त्क‚थ‚मु\edtext{}{\edlabel{pvv.389-3}\label{pvv.389-3}\lemma{मु}\Bfootnote{विनाश‚स्य भाव‚निवृत्तिल‚क्ष‚ण‚त्वात् ।}}च्य‚तां (। २७४)
	\pend% ending standard par
      \label{div_pvv.3.275}
	  
	% new div opening: depth here is 2
	
	  \bigskip
	  \begingroup
	
	    \large
	  
	    \begin{quote}
	  
	    
	    \stanza[\smallbreak]
	\label{pv.3.275}\flagstanza{\tiny\textenglish{....3.275}}अस‚त्वेऽभाव‚नाशित्व‚प्र‚स‚ङ्गोपि न युज्य‚ते ।&नाशेन य‚स्माद् भाव‚स्य न विनाश‚न‚मिष्य‚ते ॥ २७५ ॥\&[\smallbreak]


	
	    \end{quote}
	  
	  \endgroup
	

	  \pstart \leavevmode% starting standard par
	\hphantom{.}अत‚श्चानित्य‚त्वा‚{\color{DodgerBlue3}‚द‚स‚त्वे} नाश‚स्याभाव‚{\color{DodgerBlue3}‚नाशित्व}‚स्य ‚{\tiny $_{6}$}‚ ‚{\color{DodgerBlue3}‚प्र‚स‚ङ्गो\edtext{}{\edlabel{pvv.389-4}\label{pvv.389-4}\lemma{ङ्गो}\Bfootnote{बौद्ध‚स्य य‚दि नाशोऽस‚न्निष्ट‚स्त‚दा भाव‚स्य नाशित्वं न स्यात् क‚थ‚म‚स‚न् विनाशो भावं नाश‚येदिति अस‚तो व्यापारायोगात् ।}}पि न ‚{\tiny $_{lb}$}‚युज्य‚ते य‚स्माद् भाव‚स्य} काष्ठादे‚{\color{DodgerBlue3}‚र्नाशेन} हेतुना ‚{\color{DodgerBlue3}‚नाश‚नं नेष्य‚ते} । य‚दि हि नाशेन ‚{\tiny $_{lb}$}‚नाशः क्रिय‚ते इतीष्य‚ते त‚दा नाशाभावे व‚स्तुनाशो न स्यात् । किन्तु \edtext{}{\edlabel{pvv.389-5}\label{pvv.389-5}\lemma{किन्तु}\Bfootnote{क‚थ‚न्त‚र्हि भावो न‚ष्ट इति व्य‚प‚देश इत्याह ।}}स्व‚हेतुत ‚{\tiny $_{lb}$}‚एव भावा एक‚क्ष‚ण‚स्थितिध‚र्माण उत्प‚न्ना द्वितीये क्ष‚णे न भ‚व‚न्ति न तु नाम क‚श्चिद् ‚{\tiny $_{lb}$}‚भ‚व‚ति । य‚स्य नित्य‚त्वानित्य(त्व)योर्दोषाव‚काशः । (२७५)
	\pend% ending standard par
      \label{div_pvv.3.276}
	  
	% new div opening: depth here is 2
	

	  \pstart \leavevmode% starting standard par
	a. क‚थ‚न्त‚र्हीदानीम‚हेतुको नाशो भ‚व‚तीत्युच्य‚त \edtext{}{\edlabel{pvv.389-6}\label{pvv.389-6}\lemma{त}\Bfootnote{भाव‚स्य नाश इति व्य‚तिरेकः क‚थं य‚स्य स्व‚भाव एव नास्ति त‚स्य किम‚हेतुकः स‚हेतुको वेति चिन्त‚या ।}} इत्याह ‚{\tiny $_{7}$}‚(।)
	\pend% ending standard par
      
	  \bigskip
	  \begingroup
	
	    \large
	  
	    \begin{quote}
	  
	    
	    \stanza[\smallbreak]
	\label{pv.3.276}\flagstanza{\tiny\textenglish{....3.276}}न‚श्य‚न् भावः प‚रापेक्षो न त‚स्य ज्ञाप‚नाय सा ।&अव‚स्थाऽहेतुरुक्तास्या भेद‚मारोप्य चेत‚सा ॥ २७६ ॥\&[\smallbreak]


	
	    \end{quote}
	  
	  \endgroup
	\textsuperscript{\textenglish{390/s}}

	  \pstart \leavevmode% starting standard par
	\leavevmode\ledsidenote{\textenglish{77b/MA}}‚{\color{DodgerBlue3}‚न‚श्य‚न् भाव} एव क्ष‚ण‚स्थितिध‚र्म‚त‚या स्व‚हेतोरुत्प‚त्तेर्द्वितीये क्ष‚णेऽभ‚व‚न्न ‚{\color{DodgerBlue3}‚प‚रा‚{\tiny $_{lb}$}‚पेक्षः} कार‚णान्त‚र‚निर‚पेक्ष इति । ‚{\color{DodgerBlue3}‚त‚स्य} कार‚णान्त‚रान‚पेक्ष‚नाशित्व‚स्य ‚{\color{DodgerBlue3}‚ज्ञाप‚नाय ‚{\tiny $_{lb}$}‚सा} भावानाम‚{\color{DodgerBlue3}‚व‚स्थाऽहेतुरुक्ता} । अहेतुको विनाशो भ‚व‚तीत्यादि व‚च‚सैवा‚{\color{DodgerBlue3}‚स्या} अव‚स्थाया ध‚र्मिणः स‚काशात्\edtext{}{\edlabel{pvv.390-1}\label{pvv.390-1}\lemma{काशात्}\Bfootnote{अर्थान्त‚र‚मिव नाशं ।}} भेदान्त‚र‚प्र‚तिक्षेपेण ‚{\color{DodgerBlue3}‚चेत‚सा} विक‚ल्प‚केन\edtext{}{\edlabel{pvv.390-2}\label{pvv.390-2}\lemma{केन}\Bfootnote{भाव‚स्य नाशः किम‚न्य‚स्मान्न वेति जिज्ञासायां ।}} ‚{\color{DodgerBlue3}‚भेद‚मा‚{\tiny $_{lb}$}‚रोप्य} न तु व‚स्तुतो नाशो नाम क‚श्चित् भावाद् भि‚{\tiny $_{1}$}‚न्न‚स्व‚भावो भ‚व‚ति । (२७६)
	\pend% ending standard par
      \label{div_pvv.3.277}
	  
	% new div opening: depth here is 2
	
	  \bigskip
	  \begingroup
	
	    \large
	  
	    \begin{quote}
	  
	    
	    \stanza[\smallbreak]
	\label{pv.3.277}\flagstanza{\tiny\textenglish{....3.277}}स्व‚तोपि भावेऽभाव‚स्य विक‚ल्प‚श्चेद‚यं स‚मः ।&न त‚स्य किञ्चिद् भ‚व‚ति न भ‚व‚त्येव केव‚ल‚म् ॥ २७७ ॥\&[\smallbreak]


	
	    \end{quote}
	  
	  \endgroup
	

	  \pstart \leavevmode% starting standard par
	\edtext{\textsuperscript{*}}{\edlabel{pvv.390-3}\label{pvv.390-3}\lemma{*}\Bfootnote{न‚नु य‚स्याहेतुको नाशो बौद्ध‚स्य त‚न्म‚ते ।}}स्व‚तोप्य‚{\color{DodgerBlue3}‚भाव\edtext{}{\edlabel{pvv.390-4}\label{pvv.390-4}\lemma{भाव}\Bfootnote{नाश‚स्य ।}}स्य भावेऽय}‚न्त‚त्वान्य‚त्व‚{\color{DodgerBlue3}‚विक‚ल्पः स‚म‚श्चेत्} । त‚था हि य‚दि ‚{\tiny $_{lb}$}‚भावाद् भिन्नोप्य‚भावो ज्ञातो भावः किमिति न दृश्य‚तेऽभेद तु भाव एव नाश इति ‚{\tiny $_{lb}$}‚क‚थं न‚ष्टः । अत्राह (।) ‚{\color{DodgerBlue3}‚न त‚स्य} भाव‚स्य ‚{\color{DodgerBlue3}‚किञ्चिद्\edtext{}{\edlabel{pvv.390-5}\label{pvv.390-5}\lemma{किञ्चिद्}\Bfootnote{व्य‚तिरिक्त‚म‚व्य‚तिरिक्त‚म्वा ।}}} विनाशोऽन्यो\edtext{}{\edlabel{pvv.390-6}\label{pvv.390-6}\lemma{विनाशोऽन्यो}\Bfootnote{स्थित्य‚न्य‚थात्वादिर्द्ध‚र्मः ।}}वा ‚{\color{DodgerBlue3}‚भ‚व‚ति} । ‚{\tiny $_{lb}$}‚किन्त‚र्हि स ‚{\color{DodgerBlue3}‚एव केव‚लं} न भ‚व‚ति । व्य‚व‚ह‚र्त्त‚व्यैक‚रूप‚त्वात् त‚स्य । त‚त्र च ‚{\tiny $_{lb}$}‚भेदाभेद‚विक‚ल्पान‚व‚तारः । (२७७)
	\pend% ending standard par
      \label{div_pvv.3.278}
	  
	% new div opening: depth here is 2
	
	  \bigskip
	  \begingroup
	
	    \large
	  
	    \begin{quote}
	  
	    
	    \stanza[\smallbreak]
	\label{pv.3.278a}\flagstanza{\tiny\textenglish{...3.278a}}भावे ह्येष विक‚ल्पः स्याद् विधेर्व‚स्त्व‚नुरोध‚तः ।\&[\smallbreak]


	
	    \end{quote}
	  
	  \endgroup
	

	  \pstart \leavevmode% starting standard par
	हि य‚स्मात् भावे विक‚ल्प ए‚{\tiny $_{2}$}‚ष भेदाभेदात्म‚कः स्यात् विधेर्व्व‚स्त्व‚नुरोध‚तः । ‚{\tiny $_{lb}$}‚‚{\color{DodgerBlue3}‚नाश‚स्तु} प्र‚स‚ज्य‚प्र‚तिषेध‚रूपो निःस्व‚भाव‚त्वात् भेदाभेद‚विक‚ल्पाक्ष‚मः । य‚दि च ‚{\tiny $_{lb}$}‚प्र‚स‚ज्य‚प्र‚तिषेधेपि व‚स्त्व‚न्त‚र‚विधिस्त‚दा प‚र्युदासान्न भिद्येतोभ‚य‚त्रापि विधेः प्राधा‚{\tiny $_{lb}$}‚न्यात् । प‚र्युदासो वा न सिध्येत् एक‚निवृत्ताव‚प‚र‚विधाने स स्यात् निवृत्त्य‚सिद्धौ ‚{\tiny $_{lb}$}‚तु क‚थं युक्तः\edtext{}{\edlabel{pvv.390-7}\label{pvv.390-7}\lemma{युक्तः}\Bfootnote{इति प‚र्युदासोपि न स्यात् ।}} ।
	\pend% ending standard par
      
	  \bigskip
	  \begingroup
	
	    \large
	  
	    \begin{quote}
	  
	    
	    \stanza[\smallbreak]
	\label{pv.3.278b}\flagstanza{\tiny\textenglish{...3.278b}}न भावो भ‚व‚तीत्युक्त‚म‚भावो भ‚व‚तीति न ॥ २७८ ॥\&[\smallbreak]


	
	    \end{quote}
	  
	  \endgroup
	

	  \pstart \leavevmode% starting standard par
	य‚दा च भाव‚निवृत्तिर्विनाशार्थः त‚{\tiny $_{3}$}‚दा‚{\color{DodgerBlue3}‚ऽभावो भ‚व‚तीत्यादि} वाक्येन ‚{\color{DodgerBlue3}‚भावो न ‚{\tiny $_{lb}$}‚भ‚व‚ती}‚त्युक्तं\edtext{}{\edlabel{pvv.390-8}\label{pvv.390-8}\lemma{त्युक्तं}\Bfootnote{एवं हि भावो निव‚र्त्तितो भ‚व‚ति य‚दि किञ्चिन्न विधीय‚ते ।}}। अभाव‚स्य भावायोगात् । अत‚श्च हेतुर‚पि नाश‚स्य न क‚श्चित् । (२७८)
	\pend% ending standard par
      \label{div_pvv.3.279}
	  
	% new div opening: depth here is 2
	
	  \bigskip
	  \begingroup
	
	    \large
	  
	    \begin{quote}
	  
	    
	    \stanza[\smallbreak]
	\label{pv.3.279}\flagstanza{\tiny\textenglish{....3.279}}b. अपेक्षेत प‚रः कार्यं य‚दि विद्येत किञ्च‚न ।&य‚द‚किञ्चित्क‚रं व‚स्तु किं केन‚चिद‚पेक्ष्य‚ते ॥ २७९ ॥\&[\smallbreak]


	
	    \end{quote}
	  
	  \endgroup
	\textsuperscript{\textenglish{391/s}}

	  \pstart \leavevmode% starting standard par
	य‚स्मात् प‚रः कार‚णाभिम‚तोऽपेक्ष्य‚त\edtext{}{\edlabel{pvv.391-1}\label{pvv.391-1}\lemma{त}\Bfootnote{विन‚श्य‚ता भावेनापेक्षेत य‚दि भाव‚स्य क‚र्त‚व्यं स्यात् ।}}य‚दि किञ्च‚न कार्यं विद्येत । अन्य‚था ‚{\tiny $_{lb}$}‚य‚द‚किञ्चित्क‚रं व‚स्तु त‚त् केन‚चित् किम‚पेक्ष्य‚ते । (२७९)
	\pend% ending standard par
      \label{div_pvv.3.280}
	  
	% new div opening: depth here is 2
	
	  \bigskip
	  \begingroup
	
	    \large
	  
	    \begin{quote}
	  
	    
	    \stanza[\smallbreak]
	\label{pv.3.280}\flagstanza{\tiny\textenglish{....3.280}}एतेनाहेतुक‚त्वेपि ह्य‚भूत्वा नाश‚भाव‚तः ।&स‚त्तानाशित्व‚दोष‚स्य प्र‚त्याख्यातं प्र‚स‚ञ्च‚न‚म् ॥ २८० ॥\&[\smallbreak]


	
	    \end{quote}
	  
	  \endgroup
	

	  \pstart \leavevmode% starting standard par
	\hphantom{.}‚{\color{DodgerBlue3}‚\edtext{\textsuperscript{*}}{\edlabel{pvv.391-2}\label{pvv.391-2}\lemma{*}\Bfootnote{प‚र‚ता हेतुकेपि नाशेऽभूत्वा भावात् स‚त्ताऽनित्य‚त्व‚ञ्च दुर्न्निवारं अभूत्वा ‚{\tiny $_{lb}$}‚भ‚व‚न्न‚हेतुकं इत्य‚पि विरुद्धं कादाचित्क‚स्य स‚हेतुत्वात् ।}}एतेन} नाश‚स्य निःस्व‚भाव‚त्व‚क‚थ‚नेना‚{\color{DodgerBlue3}‚हेतुक‚त्वेपि\edtext{}{\edlabel{pvv.391-3}\label{pvv.391-3}\lemma{त्वेपि}\Bfootnote{नाश‚स्य स्वीकृते ।}}} स‚त्य‚{\color{DodgerBlue3}‚भूत्वा नाश‚स्य ‚{\tiny $_{lb}$}‚भाव‚तः\edtext{}{\edlabel{pvv.391-4}\label{pvv.391-4}\lemma{तः}\Bfootnote{स नाशो अभूत्वा भ‚व‚तीतु हेतोः स‚त्ता च नाशित्व‚ञ्चेति ।}} । स‚त्तानाशित्व‚{\tiny $_{4}$}‚दोष‚स्य प्र‚स‚ञ्ज‚नं\edtext{}{\edlabel{pvv.391-5}\label{pvv.391-5}\lemma{नं}\Bfootnote{य‚त्प‚रेण ।}}} घ‚टादाविव\edtext{}{\edlabel{pvv.391-6}\label{pvv.391-6}\lemma{टादाविव}\Bfootnote{त‚देतेनैव ।}} प्र‚त्याख्यातं\edtext{}{\edlabel{pvv.391-7}\label{pvv.391-7}\lemma{त्याख्यातं}\Bfootnote{नाशे क‚स्यापि भावान‚भ्युप (ग) मात् ।}}। न ‚{\tiny $_{lb}$}‚ह्य‚भावो नाम क‚श्चिद् भ‚व‚ति य‚स्याभूत्वा भावात् स‚त्त्वं नाशित्व‚म्वा स्यात् ॥ (२८०)
	\pend% ending standard par
      \label{div_pvv.3.281}
	  
	% new div opening: depth here is 2
	
	  \bigskip
	  \begingroup
	
	    \large
	  
	    \begin{quote}
	  
	    
	    \stanza[\smallbreak]
	\label{pv.3.281}\flagstanza{\tiny\textenglish{....3.281}}c. य‚था केषाञ्चिदेवेष्टः प्र‚तिघो ज‚न्मिनां त‚था ।&नाश(ः)स्व‚भावो भावानां नानुत्प‚त्तिम‚तां य‚दि ॥ २८१ ॥\&[\smallbreak]


	
	    \end{quote}
	  
	  \endgroup
	

	  \pstart \leavevmode% starting standard par
	\hphantom{.}न‚नु ‚{\color{DodgerBlue3}‚य‚था ज‚न्मिनां} बुद्ध्यादीनां म‚ध्ये ‚{\color{DodgerBlue3}‚केषाञ्चिदेव} घ‚टादीनां प्र‚तिघः स्व‚देशे ‚{\tiny $_{lb}$}‚व‚स्त्व‚न्त‚रोत्प‚त्तिव्याघात ‚{\color{DodgerBlue3}‚इष्टः} । न बुद्ध्यादीनां । त‚था ‚{\color{DodgerBlue3}‚य‚दि} स‚तामुत्प‚त्तिम‚तामेव ‚{\tiny $_{lb}$}‚‚{\color{DodgerBlue3}‚नाशः स्व‚भावः} स्यात् ‚{\color{DodgerBlue3}‚नानुत्प‚त्तिम‚तां} श‚ब्दाकाशादीनां त‚दा\edtext{}{\edlabel{pvv.391-8}\label{pvv.391-8}\lemma{दा}\Bfootnote{एतेन स‚त्व‚ञ्चान‚श्व‚र‚ञ्चेति व्य‚भिचारः ।}} क‚थ‚मुक्तं‚{\tiny $_{5}$}‚ स‚त्ता\edtext{}{\edlabel{pvv.391-9}\label{pvv.391-9}\lemma{त्ता}\Bfootnote{न‚नु नाश‚स्व‚भावो भावानां नानुत्प‚त्तिम‚तां य‚दीति प्र‚कृतं चोद्यं न चात्र ‚{\tiny $_{lb}$}‚नानित्येत्यादिः प‚रिहारः । नाकाशादेः स्व‚हेतुतो न‚श्व‚र‚ताऽनुत्प‚त्तिम‚त्वात् ॥ अत्र ‚{\tiny $_{lb}$}‚सिद्धान्त्येव‚म्म‚न्य‚ते । य‚था स‚त्वं व्य‚भिचार्युक्त(ः)त‚था कृत‚कोपि क‚श्चिन्न‚श्व‚रः क‚श्चि‚{\tiny $_{lb}$}‚न्नेत्याश‚ङ्क्य‚ते । तेनादौ कृत‚क‚व्य‚भिचारं प‚रिह‚र‚ति ॥ स‚त्वेप्युच्य‚ते । ये क्व‚चित् ‚{\tiny $_{lb}$}‚क‚दाचित् केन‚चिद‚ज्ञाता ज्ञाय‚न्ते पुन‚र्न ज्ञाय‚न्ते तेषां स‚त्तानुब‚न्धी नाश इति व्याप्तिः । ‚{\tiny $_{lb}$}‚अन्य‚था नित्त्य‚त्वात् स‚दा ज्ञान‚ज‚न‚न‚प्र‚स‚ङ्गः । स‚ह‚कार्य‚पेक्षापि न । नित्त्य‚त्वात् पूर्व्व‚मेव ‚{\tiny $_{lb}$}‚निष्प‚त्तेर्ज्ञान‚ज (न) क‚स्य ।}}‚{\tiny $_{lb}$}‚मात्रानुब‚न्धित्वात् नाश‚स्यानित्य‚ता ध्व‚नेरिति । (२८१)
	\pend% ending standard par
      \label{div_pvv.3.282}
	  
	% new div opening: depth here is 2
	

	  \pstart \leavevmode% starting standard par
	अत्राह (।)
	\pend% ending standard par
      
	  \bigskip
	  \begingroup
	
	    \large
	  
	    \begin{quote}
	  
	    
	    \stanza[\smallbreak]
	\label{pv.3.282}\flagstanza{\tiny\textenglish{....3.282}}स्व‚भाव‚निय‚माद्धेतोः स्व‚भाव‚निय‚मः फ‚ले ।&नानित्ये रूप‚भेदोस्ति भेद‚कानाम‚भाव‚तः ॥ २८२ ॥\&[\smallbreak]


	
	    \end{quote}
	  
	  \endgroup
	\textsuperscript{\textenglish{392/s}}

	  \pstart \leavevmode% starting standard par
	\hphantom{.}‚{\color{DodgerBlue3}‚हेतोः स्व‚भाव}‚स्य विशिष्ट\edtext{}{\edlabel{pvv.392-1}\label{pvv.392-1}\lemma{विशिष्ट}\Bfootnote{अयं स‚प्र‚तिघ‚स्य ज‚न‚कोऽयं नेति ।}} कार्योत्पाद‚न‚योग्य‚ताया ‚{\color{DodgerBlue3}‚निय‚मात् । फ‚ले} कार्ये ‚{\tiny $_{lb}$}‚‚{\color{DodgerBlue3}‚स्व‚भाव‚स्य} प्र‚तिघाप्र‚तिघादे‚{\color{DodgerBlue3}‚र्निय‚मः} । त‚तो नित्य‚त्वाभिम‚तानाञ्च हेतुम‚न्त‚रेण ‚{\tiny $_{lb}$}‚स्व‚भाव‚निय‚मायोगात् हेतुरेष्ट‚व्यः । कृत‚के ‚{\color{DodgerBlue3}‚पुन‚र‚नित्ये} वा भावे ‚{\color{DodgerBlue3}‚रूप}‚स्य न‚श्व‚रान‚{\tiny $_{lb}$}‚श्व‚र‚स्य ‚{\color{DodgerBlue3}‚भेदो नास्ति} । क‚स्मादित्याह । नित्यानि‚{\tiny $_{6}$}‚त्य‚स्व‚भाव‚त‚या कृत‚क‚स्य ‚{\color{DodgerBlue3}‚भेद‚कानां} हेतूनाम‚{\color{DodgerBlue3}‚भाव‚तः} । न हि क‚श्चिदेव हेतुर‚नित्यं ज‚न‚य‚ति\edtext{}{\edlabel{pvv.392-2}\label{pvv.392-2}\lemma{ति}\Bfootnote{स‚र्व्वेषां विन‚श्व‚र‚स्व‚भाव‚स्यैव ज‚न‚नात् ।}}नाप‚रः स‚र्व्वेषां कृत‚काना‚{\tiny $_{lb}$}‚म‚र्थ‚क्रियाकारित्वात् । त‚स्य चानित्य‚ताव्याप्तेः । (२८२)
	\pend% ending standard par
      \label{div_pvv.3.283}
	  
	% new div opening: depth here is 2
	
	  \bigskip
	  \begingroup
	
	    \large
	  
	    \begin{quote}
	  
	    
	    \stanza[\smallbreak]
	\label{pv.3.283}\flagstanza{\tiny\textenglish{....3.283}}d. प्र‚त्याख्येयाऽत एवैषां स‚म्ब‚न्ध‚स्यापि नित्य‚ता ।&स‚म्ब‚न्ध‚दोषैः प्रागुक्तैः श‚ब्द‚श‚क्तिश्च दूषिता ॥ २८३ ॥\&[\smallbreak]


	
	    \end{quote}
	  
	  \endgroup
	

	  \pstart \leavevmode% starting standard par
	\hphantom{.}‚{\color{DodgerBlue3}‚अतः\edtext{}{\edlabel{pvv.392-3}\label{pvv.392-3}\lemma{अतः}\Bfootnote{अन‚न्त‚रोक्तात् व‚स्तुमात्रानुब‚न्धात् ।}}} स‚र्व्व‚भाव‚क्ष‚णिक‚त्व‚साध‚कात् प्र‚माणादे‚{\color{DodgerBlue3}‚वैषां} श‚ब्दानाम‚र्थ‚{\color{DodgerBlue3}‚स‚म्ब‚न्ध‚स्यापि ‚{\tiny $_{lb}$}‚नित्य‚ता प्र‚त्याख्येया} । स‚म्ब‚न्ध‚स्य व‚स्तुत्वे क्ष‚णिक‚त्वात् । या च श‚ब्द‚श‚क्ति\leavevmode\ledsidenote{\textenglish{78a/MA}}‚{\tiny $_{lb}$}‚र्योग्य‚ताख्याऽर्थ‚प्र‚तिप‚त्त्याश्र‚यो‚{\tiny $_{7}$}‚ व‚र्ण्ण्य‚ते\edtext{}{\edlabel{pvv.392-4}\label{pvv.392-4}\lemma{ते}\Bfootnote{जैमिनीयैः ।}} सा\edtext{}{\edlabel{pvv.392-5}\label{pvv.392-5}\lemma{सा}\Bfootnote{व्य‚तिरिक्त एव नास्तीत्य‚तोधुना ।}} च श‚ब्दाद् व्य‚तिरिक्तैवेति त‚द्व\edtext{}{\edlabel{pvv.392-6}\label{pvv.392-6}\lemma{द्व}\Bfootnote{श‚ब्द‚व‚त्}}‚{\tiny $_{lb}$}‚द‚नित्या । अथ भिन्ना\edtext{}{\edlabel{pvv.392-7}\label{pvv.392-7}\lemma{भिन्ना}\Bfootnote{श‚ब्द‚श‚क्तिः स‚म्ब‚न्ध ।}} तादृशी च ‚{\color{DodgerBlue3}‚स‚म्ब‚न्ध‚दोषैः} स‚म्ब‚न्धिनाम‚नित्य‚त्वान्न स‚म्ब‚न्धे\edtext{}{\edlabel{pvv.392-8}\label{pvv.392-8}\lemma{न्धे}\Bfootnote{अव्य‚तिरेकात् ।}} ‚{\tiny $_{lb}$}‚अस्ति नित्य‚ते \href{http://sarit.indology.info/?cref=pv.3.231}{(३।२३१)}त्यादिना ‚{\color{DodgerBlue3}‚प्रागुक्तैर्दूषिते}‚ति न पुन‚रुच्य‚ते ।\edtext{\textsuperscript{*}}{\edlabel{pvv.392-9}\label{pvv.392-9}\lemma{*}\Bfootnote{त‚देवं ।}} (२८३)
	\pend% ending standard par
      \label{div_pvv.3.284}
	  
	% new div opening: depth here is 2
	

	  \begin{center}%% label @type='head'
	\textbf{(२) a न‚व्य‚मीमांस‚क(भादृ)म‚त‚निरासः}
	\end{center}
	

	  \begin{center}%% label @type='head'
	\textbf{क. अपौरुषेय‚त्वेऽपि दोषाः}
	\end{center}
	

	  \begin{center}%% label @type='head'
	\textbf{(क) अपौरुषेय‚त्वान्न याथार्थ्य‚सिद्धिः}
	\end{center}
	

	  \pstart \leavevmode% starting standard par
	\hphantom{.}‚{\color{DodgerBlue3}‚न} ताव‚द‚पौरुषेयं व‚च‚न‚म‚स्तीत्युक्तं । भ‚व‚तु वा । त‚थापि (।)
	\pend% ending standard par
      
	  \bigskip
	  \begingroup
	
	    \large
	  
	    \begin{quote}
	  
	    
	    \stanza[\smallbreak]
	\label{pv.3.284}\flagstanza{\tiny\textenglish{....3.284}}नाऽपौरुषेय‚मित्येव य‚थार्थ‚ज्ञान‚साध‚न‚म् ।&दृष्टोऽन्य‚थापि व‚ह्न्यादिर‚दुष्टः पुरुषाग‚सा ॥ २८४ ॥\&[\smallbreak]


	
	    \end{quote}
	  
	  \endgroup
	\textsuperscript{\textenglish{392/s}}

	  \pstart \leavevmode% starting standard par
	\hphantom{.}‚{\color{DodgerBlue3}‚नापौरुषेय‚मित्येव व‚च‚नं य‚थाऽर्थ}‚स्याविस‚म्वादिनो ‚{\color{DodgerBlue3}‚ज्ञान}‚स्य ‚{\color{DodgerBlue3}‚साध‚नं}\edtext{}{\edlabel{pvv.393-1}\label{pvv.393-1}\lemma{स्य}\Bfootnote{श‚क्य‚निश्च‚यः ।}} य‚स्मात् ‚{\tiny $_{lb}$}‚पुरुष‚स्या‚{\color{DodgerBlue3}‚ग‚सा} दोषेणा‚{\color{DodgerBlue3}‚दुष्टोपि व‚ह्न्यादि\edtext{}{\edlabel{pvv.393-2}\label{pvv.393-2}\lemma{ह्न्यादि}\Bfootnote{ज्योत्स्नादि ।}}}‚र्नीलोत्प‚ला\edtext{}{\edlabel{pvv.393-3}\label{pvv.393-3}\lemma{ला}\Bfootnote{न हि रागादि पुंदोष‚संस्कारादेवार्थेषु ज्ञाय्येषु ज्ञाप‚कानां श‚ब्दानां ज्ञान‚भ्र‚मः ‚{\tiny $_{lb}$}‚प्र‚कृत्यापि मिथ्याज्ञान‚ज‚न‚न‚स्य संभाव्य‚त्वाद् व‚ने द‚वो रात्रौ नीलोत्प‚लो र‚क्त‚प्र‚ति‚{\tiny $_{lb}$}‚भासि ज्ञान‚हेतु ज्योत्स्ना पीते शुक्ल‚ज्ञान‚हेतुरिति विना पुरुष‚ञ्च दृष्टेर्व्व‚न‚द‚वादेः;}} ‚{\color{DodgerBlue3}‚३दाव‚न्य‚था अप‚{\tiny $_{1}$}‚रार्थ‚ज्ञान‚{\tiny $_{lb}$}‚हेतु}‚र्दृष्टः ।\edtext{\textsuperscript{*}}{\edlabel{pvv.393-4}\label{pvv.393-4}\lemma{*}\Bfootnote{व‚ह्न्यादेः स‚ह‚कारि ब‚लाद‚स्त्व‚न्य‚थात्वं नित्ये तु नैव‚मिति चेन्न श‚ब्देपि ‚{\tiny $_{lb}$}‚स‚ङ्केत‚स‚ह‚कार्य‚पेक्ष‚त्वाद‚न‚पेक्ष्य ज्ञान‚ज‚न‚न‚स्व‚भाव‚त्वे स‚ङ्केत‚क‚र‚ण‚व्यापार‚म्विनापि वेदा‚{\tiny $_{lb}$}‚द‚र्थ‚ज्ञानं स‚र्व्व‚स्य स‚दा स्यान्न चैवं त‚न्न स्थित‚स्व‚भाव‚तेति मिथ्याज्ञान‚हेतुत्वं त‚द‚व‚स्थं ।}}(२८४)
	\pend% ending standard par
      \label{div_pvv.3.285}
	  
	% new div opening: depth here is 2
	

	  \begin{center}%% label @type='head'
	\textbf{(ख) अकृत‚क‚त्वे ज्ञान‚हेतुत्वाभावः}
	\end{center}
	

	  \pstart \leavevmode% starting standard par
	किञ्च (।)
	\pend% ending standard par
      
	  \bigskip
	  \begingroup
	
	    \large
	  
	    \begin{quote}
	  
	    
	    \stanza[\smallbreak]
	\label{pv.3.285}\flagstanza{\tiny\textenglish{....3.285}}न ज्ञान‚हेतुतैव स्यात् त‚स्मिन्न‚कृत‚के म‚ते ।&नित्येभ्यो व‚स्तुसाम‚र्थ्यात् न च ज‚न्मास्ति क‚स्य‚चित् ॥ २८५ ॥\&[\smallbreak]


	
	    \end{quote}
	  
	  \endgroup
	

	  \pstart \leavevmode% starting standard par
	\hphantom{.}‚{\color{DodgerBlue3}‚त‚स्मिन्} श‚ब्देऽ‚{\color{DodgerBlue3}‚कृत‚के म‚ते\edtext{}{\edlabel{pvv.393-5}\label{pvv.393-5}\lemma{ते}\Bfootnote{इष्टे स‚ति ।}} ज्ञान‚हेतुतैव न स्यात् न} हि ‚{\color{DodgerBlue3}‚नित्येभ्यो व‚स्तु\edtext{}{\edlabel{pvv.393-6}\label{pvv.393-6}\lemma{स्तु}\Bfootnote{प्र‚तीत्य ज‚न्म‚काले य‚त् त‚द‚ज‚न‚कं रूपं त‚त्स्थ‚स्याज‚न‚क‚त्वात् ।}}‚{\tiny $_{lb}$}‚साम‚र्थ्यात्} कार्य‚ज‚न्माश‚क्तेः ‚{\color{DodgerBlue3}‚क‚स्य‚चित्} ज्ञान‚स्यान्य‚स्य वा\edtext{}{\edlabel{pvv.393-7}\label{pvv.393-7}\lemma{वा}\Bfootnote{एत‚त्प‚रिहाराय नित्यं स्व‚कार्य‚ज‚न‚न‚ञ्च मिथ्याऽद‚र्श‚नात्}} ‚{\color{DodgerBlue3}‚ज‚न्मास्ति} (।) ‚{\tiny $_{lb}$}‚क्र‚माक्र‚म‚योर्व्याप‚क‚योर्नित्यान्निवृत्तेः । त‚द्व्याप्य‚स्य साम‚र्थ्य‚स्याभावात्\edtext{}{\edlabel{pvv.393-8}\label{pvv.393-8}\lemma{स्याभावात्}\Bfootnote{नित्याकाशादिभ्यो बुद्ध‚यो भ‚व‚न्तीत्य‚पि मृषा न ता (ः) त‚द्भाव‚भाविन्यः ‚{\tiny $_{lb}$}‚क्र‚म‚यौग‚प‚द्यार्थ‚क्रियाविरोधात् ।}}। (२८५)
	\pend% ending standard par
      \label{div_pvv.3.286}
	  
	% new div opening: depth here is 2
	

	  \begin{center}%% label @type='head'
	\textbf{(ग) श‚ब्दे स‚मारोपित‚गोच‚रा बुद्ध‚यः}
	\end{center}
	

	  \pstart \leavevmode% starting standard par
	त‚स्मात् (।)
	\pend% ending standard par
      
	  \bigskip
	  \begingroup
	
	    \large
	  
	    \begin{quote}
	  
	    
	    \stanza[\smallbreak]
	\label{pv.3.286}\flagstanza{\tiny\textenglish{....3.286}}विक‚ल्प‚वास‚नोद्भूताः स‚मारोपित‚गोच‚राः ।&जाय‚न्ते बुद्ध‚य‚स्त‚त्र केव‚लं नार्थ‚गोच‚राः ॥ २८६ ॥\&[\smallbreak]


	
	    \end{quote}
	  
	  \endgroup
	

	  \pstart \leavevmode% starting standard par
	\hphantom{.}त‚त्रानित्यात्म‚न्युच्च‚रिते श‚ब्देऽनाद्य‚न्तेन ‚{\color{DodgerBlue3}‚विक‚ल्प‚ने} चित्त‚स‚न्त‚तावारोपिताया ‚{\tiny $_{lb}$}‚‚{\color{DodgerBlue3}‚वास‚नाया} विक‚ल्पिका ‚{\color{DodgerBlue3}‚बुद्ध‚यो जाय‚न्ते\edtext{}{\edlabel{pvv.393-9}\label{pvv.393-9}\lemma{न्ते}\Bfootnote{स्वाग‚म‚संस्कार‚क‚ल्प‚नाया बाह्य‚त्वेन त‚तः ।}} स‚मारोपित\edtext{}{\edlabel{pvv.393-10}\label{pvv.393-10}\lemma{मारोपित}\Bfootnote{आकाशाद्याकारः ।}}गोच‚राः‚{\tiny $_{2}$}‚} क‚ल्पितार्थ‚{\tiny $_{lb}$}‚\leavevmode\ledsidenote{\textenglish{394/s}} ‚{\color{DodgerBlue3}‚विष‚या नार्थ‚गोच‚रा} न स्व\edtext{}{\edlabel{pvv.394-1}\label{pvv.394-1}\lemma{स्व}\Bfootnote{आकाशादि ।}}ल‚क्ष‚ण‚विष‚याः । त‚थार्थ‚त्वे श‚ब्द‚प्र‚माणान्त‚र‚{\tiny $_{lb}$}‚वैफ‚ल्य‚स्योक्तेः ।\edtext{\textsuperscript{*}}{\edlabel{pvv.394-2}\label{pvv.394-2}\lemma{*}\Bfootnote{स‚त्यार्थं वेद‚वाक्य‚म‚कृत‚क‚त्वादित्य‚त्रान्व‚याभावात् व्य‚तिरेकिप्र‚योग‚माह ।}}(२८६)
	\pend% ending standard par
      \label{div_pvv.3.287_3.288}
	  
	% new div opening: depth here is 2
	

	  \begin{center}%% label @type='head'
	\textbf{ख. कृत‚क‚त्वेऽपि न दोषः}
	\end{center}
	

	  \begin{center}%% label @type='head'
	\textbf{(क) कृत‚क‚त्वान्न मिथ्यात्व‚निय‚मः}
	\end{center}
	

	  \pstart \leavevmode% starting standard par
	न‚नु (।)
	\pend% ending standard par
      
	  \bigskip
	  \begingroup
	
	    \large
	  
	    \begin{quote}
	  
	    
	    \stanza[\smallbreak]
	\label{pv.3.287}\flagstanza{\tiny\textenglish{....3.287}}मिथ्यात्वं कृत‚केष्वेव दृष्ट‚मित्य‚कृतं व‚चः ।&स‚त्यार्थं व्य‚तिरेक‚स्य विरोधिव्याप‚नाद् य‚दि ॥ २८७ ॥\&[\smallbreak]


	
	    \end{quote}
	  
	  \endgroup
	
	  \bigskip
	  \begingroup
	
	    \large
	  
	    \begin{quote}
	  
	    
	    \stanza[\smallbreak]
	\label{pv.3.288}\flagstanza{\tiny\textenglish{....3.288}}हेताव‚स‚म्भ‚वे भाव‚स्त‚त् त‚स्यापि श‚ङ्क्य‚ते ।&विरुद्धानाम्प‚दार्थानाम‚पि व्याप‚क‚द‚र्श‚नात् ॥ २८८ ॥\&[\smallbreak]


	
	    \end{quote}
	  
	  \endgroup
	

	  \pstart \leavevmode% starting standard par
	\hphantom{.}‚{\color{DodgerBlue3}‚मिथ्यात्व}‚म‚र्थ‚शून्य‚त्वं ‚{\color{DodgerBlue3}‚कृत‚केष्वेव} वाक्येषु ‚{\color{DodgerBlue3}‚दृष्ट‚मि\edtext{}{\edlabel{pvv.394-3}\label{pvv.394-3}\lemma{मि}\Bfootnote{हेतोः ।}}त्य‚कृतं व‚चः} (।) ‚{\color{DodgerBlue3}‚स‚त्यार्थं ‚{\tiny $_{lb}$}‚य‚दि} स्यात् त‚दा को दोषः । क‚स्मादेव‚मित्याह प‚रः । अकृत‚क‚त्व‚स्य साध‚न‚स्य ‚{\tiny $_{lb}$}‚विरोधिना कृत‚क‚त्वेन साध्य\edtext{}{\edlabel{pvv.394-4}\label{pvv.394-4}\lemma{साध्य}\Bfootnote{स‚त्यार्थ‚त्व‚स्य ।}}विप‚र्य‚य‚स्य मिथ्यात्व‚स्य व्याप‚नात्\edtext{}{\edlabel{pvv.394-5}\label{pvv.394-5}\lemma{नात्}\Bfootnote{वेदे कृत‚क‚निवृत्तौ मिथ्यानिवृत्तेः स‚त्य‚ता सेष्टैव ।}} । अत्राह । ‚{\tiny $_{lb}$}‚विप‚क्षान्मिथ्यात्वाद‚कृत‚क‚स्य हेतोर‚संभ‚वेऽसंभ‚व‚निमित्तं हेतौ बाध‚क‚प्र‚माणेऽनुक्ते ‚{\tiny $_{lb}$}‚त‚स्याकृत‚क‚त्व‚स्यापि मिथ्यात्वे विप‚क्षे भावः शंक्य‚ते बाध‚क‚प्र‚माणाद‚र्श‚नात् ।
	\pend% ending standard par
      

	  \pstart \leavevmode% starting standard par
	\hphantom{.}न‚नु मिथ्यात्वं कृत‚केषु दृष्टं त‚द‚कृत‚केषु विरोधिषु क‚थं स्यादित्याह । ‚{\color{DodgerBlue3}‚विरुद्धाना‚{\tiny $_{lb}$}‚म‚पि} हि व्याप्यानां ‚{\color{DodgerBlue3}‚प‚दार्थानामे}‚क‚स्य ‚{\color{DodgerBlue3}‚व्याप‚क}‚स्य ‚{\color{DodgerBlue3}‚द‚र्श‚नात्} । य‚था प्र‚य‚त्नान‚न्त‚रीय‚{\tiny $_{lb}$}‚‚{\color{DodgerBlue3}‚क‚त्व‚योरेकेन} कृत‚क‚त्वेनानित्य‚त्वेन वा‚{\tiny $_{5}$}‚ व्याप्तिः । (२८७,२८८)
	\pend% ending standard par
      \label{div_pvv.3.289}
	  
	% new div opening: depth here is 2
	

	  \pstart \leavevmode% starting standard par
	किञ्च (।)
	\pend% ending standard par
      
	  \bigskip
	  \begingroup
	
	    \large
	  
	    \begin{quote}
	  
	    
	    \stanza[\smallbreak]
	\label{pv.3.289}\flagstanza{\tiny\textenglish{....3.289}}नास‚त्ता सिद्धिरित्युक्तं स‚र्व‚तोनुप‚ल‚म्भ‚नात् ।&असिद्धायाम‚स‚त्तायां स‚न्दिग्धा व्य‚तिरेकिता ॥ २८९ ॥\&[\smallbreak]


	
	    \end{quote}
	  
	  \endgroup
	

	  \pstart \leavevmode% starting standard par
	\hphantom{.}‚{\color{DodgerBlue3}‚स‚र्व्व‚तो} विप‚क्षाद्धेतोर‚स‚त्ताया ‚{\color{DodgerBlue3}‚अनुप‚ल‚म्भात्} प्र‚तिब्ध‚म‚न्त‚रेण ‚{\color{DodgerBlue3}‚न सिद्धिरित्युक्तं ‚{\tiny $_{lb}$}‚प्राक्} । न चाद‚र्श‚न‚मात्रेण विप‚क्षेऽव्य‚भिचारिते \cref{pv.3.12}त्यादिना । ‚{\color{DodgerBlue3}‚असिद्धा‚{\tiny $_{lb}$}‚याम्वास‚त्यायां} विप‚क्षाद् ‚{\color{DodgerBlue3}‚व्य‚तिरेकिता स‚न्दिग्धा} (।)
	\pend% ending standard par
      \textsuperscript{\textenglish{395/s}}

	  \pstart \leavevmode% starting standard par
	\hphantom{.}क‚स्मात् पुन‚र्व्य‚तिरेक‚निश्च‚यापेक्षा । न हि साध‚न‚त‚योपात्तो ‚{\color{DodgerBlue3}‚ध‚र्मं} इत्येव ‚{\tiny $_{lb}$}‚साध‚नं । (२८९)
	\pend% ending standard par
      \label{div_pvv.3.290}
	  
	% new div opening: depth here is 2
	

	  \pstart \leavevmode% starting standard par
	किन्त‚र्हि (।)
	\pend% ending standard par
      
	  \bigskip
	  \begingroup
	
	    \large
	  
	    \begin{quote}
	  
	    
	    \stanza[\smallbreak]
	\label{pv.3.290}\flagstanza{\tiny\textenglish{....3.290}}अन्व‚यो व्य‚तिरेको वा स‚त्त्वं वा साध्य‚ध‚र्मिणि ।&त‚न्निश्च‚य‚फ‚लैर्ज्ञानैः सिद्ध्य‚न्ति य‚दि साध‚न‚म् ॥ २९० ॥\&[\smallbreak]


	
	    \end{quote}
	  
	  \endgroup
	

	  \pstart \leavevmode% starting standard par
	\hphantom{.}‚{\color{DodgerBlue3}‚अन्व‚यः} स‚प‚क्षे\edtext{}{\edlabel{pvv.395-1}\label{pvv.395-1}\lemma{क्षे}\Bfootnote{स्व‚साध्येन हेतोर्व्याप्तिः ।}} ‚{\color{DodgerBlue3}‚व्य‚तिरेको} विप‚क्षात्\edtext{}{\edlabel{pvv.395-2}\label{pvv.395-2}\lemma{क्षात्}\Bfootnote{व्यावृत्तिर्व्वा ।}} । ‚{\color{DodgerBlue3}‚साध्य‚ध‚र्मिणि स‚त्व\edtext{}{\edlabel{pvv.395-3}\label{pvv.395-3}\lemma{त्व}\Bfootnote{हेतोः प‚क्ष‚ध‚र्म‚त्वं ।}}म्वा} त‚{\tiny $_{5}$}‚‚{\color{DodgerBlue3}‚न्निश्च‚य‚फ‚लै}‚र‚न्व‚यादिनिश्च‚य‚प्र‚योज‚नै‚{\color{DodgerBlue3}‚र्ज्ञानैः} प्र‚माणात्म‚{\color{DodgerBlue3}‚भिर्य‚दि सिध्य‚न्ति} त‚दा ‚{\tiny $_{lb}$}‚‚{\color{DodgerBlue3}‚साध‚नं भ‚व‚ति} । निश्चित‚त्रैरूप्य‚स्यैव हेतुत्वात् । (२९०)
	\pend% ending standard par
      \label{div_pvv.3.291}
	  
	% new div opening: depth here is 2
	

	  \pstart \leavevmode% starting standard par
	य‚द‚पि व्य‚तिरेकी हेतुरित्युक्तं त‚च्चायुक्त‚मित्याह \edtext{}{\edlabel{pvv.395-4}\label{pvv.395-4}\lemma{मित्याह}\Bfootnote{य एव मिथ्यात्व‚व्य‚व‚च्छेद‚स्य विष‚यः ।}} ।
	\pend% ending standard par
      
	  \bigskip
	  \begingroup
	
	    \large
	  
	    \begin{quote}
	  
	    
	    \stanza[\smallbreak]
	\label{pv.3.291}\flagstanza{\tiny\textenglish{....3.291}}य‚त्र साध्य‚विप‚क्ष‚स्य व‚र्ण्ण‚ते व्य‚तिरेकिता ।&स एवास्य स‚प‚क्षः स्यात् स‚र्वो हेतुर‚न‚न्व‚यी ॥ २९१ ॥\&[\smallbreak]


	
	    \end{quote}
	  
	  \endgroup
	

	  \pstart \leavevmode% starting standard par
	\hphantom{.}‚{\color{DodgerBlue3}‚य‚त्र} ध‚र्मिणि ‚{\color{DodgerBlue3}‚साध्य‚विप‚क्ष‚स्य} मिथ्यात्व‚स्य साध‚नाविप‚क्ष‚व्य‚तिरेकाद् ‚{\color{DodgerBlue3}‚व्य‚ति‚{\tiny $_{lb}$}‚रेकिता व‚र्ण्ण्य‚ते} । य‚त् कृत‚कं न भ‚व‚ति त‚न्मिथ्यार्थ‚ञ्च न भ‚व‚तीति ‚{\color{DodgerBlue3}‚स एव} साध्य‚साध‚न‚विप‚र्य‚य‚निवृत्तिद‚र्श‚न‚विष‚यो ध‚र्मी ‚{\color{DodgerBlue3}‚स‚प‚{\tiny $_{6}$}‚क्षः स्याद‚स्या}‚कृत‚क‚त्व‚स्य हेतोः । ‚{\tiny $_{lb}$}‚विप‚र्य‚य‚निषेधेन विधेरेव प्र‚तिपाद‚नात् । य‚त्र च साध्य‚साध‚न‚स‚त्त‚व‚निश्च‚यः स एव ‚{\tiny $_{lb}$}‚प‚क्षः । ‚{\color{DodgerBlue3}‚अतः स‚र्व्वो हेतुर‚न‚न्व‚यी} न केव‚ल‚व्य‚तिरेकी । (२९१)
	\pend% ending standard par
      \label{div_pvv.3.292}
	  
	% new div opening: depth here is 2
	

	  \begin{center}%% label @type='head'
	\textbf{(ख) स‚म‚य‚त्वान्म‚न्त्राणां कृत‚कारिता}
	\end{center}
	

	  \pstart \leavevmode% starting standard par
	न‚नु \edtext{}{\edlabel{pvv.395-5}\label{pvv.395-5}\lemma{नु}\Bfootnote{साध्य‚ध‚र्म‚सामान्येन स‚मानोर्थः स‚प‚क्षः ।}} प‚क्ष‚स्य वेद‚स्य क‚थं स‚प‚क्ष‚ता स‚प‚क्ष‚ल‚क्ष‚ण\edtext{}{\edlabel{pvv.395-6}\label{pvv.395-6}\lemma{ण}\Bfootnote{साध‚र्म्य‚दृष्टान्त उच्य‚ते न चाय‚मिहास्ति साध्य‚त्वेनान्व‚य एव स‚प‚क्ष उच्य‚ते ।}} योगात् । एवं त‚र्हि स‚र्व्वः ‚{\tiny $_{lb}$}‚प‚क्षः स‚प‚क्षः स्यात् । भ‚व‚त्येव किम‚न्येनेति चेत् । अन्य‚स्यापि लाक्ष‚णिकं त‚त्क‚थं ‚{\tiny $_{lb}$}‚त्य‚ज्य‚तां । किञ्च (।)
	\pend% ending standard par
      
	  \bigskip
	  \begingroup
	
	    \large
	  
	    \begin{quote}
	  
	    
	    \stanza[\smallbreak]
	\label{pv.3.292a}\flagstanza{\tiny\textenglish{...3.292a}}स‚म‚य‚त्वे हि म‚न्त्राणां क‚स्य‚चित् कार्य‚द‚र्श‚न‚म् ।\&[\smallbreak]


	
	    \end{quote}
	  
	  \endgroup
	

	  \pstart \leavevmode% starting standard par
	किञ्चित्का‚{\tiny $_{7}$}‚र्य\edtext{}{\edlabel{pvv.395-7}\label{pvv.395-7}\lemma{र्य}\Bfootnote{स‚ङ्केत‚त्वादेव पौरुषेय‚त्व‚माह ।}}कारिणो म‚न्त्रा अपौरुषेयाश्चेति व्याह‚तं (।) त‚था हि\leavevmode\ledsidenote{\textenglish{78b/MA}} ‚{\tiny $_{lb}$}‚क‚स्य‚चित् स‚त्य‚त‚पःप्र‚भाव‚तः पुंसः स‚म‚य‚त्वे संकेत‚त्वे म‚न्त्राणाम‚भ्युप‚ग‚म्य‚माने ‚{\tiny $_{lb}$}‚\leavevmode\ledsidenote{\textenglish{396/s}} कार्य‚स्य म‚न्त्र‚प्र‚योग‚निस्प (? ष्प)ाद्य‚त्वेनेष्ट‚स्य साध‚नं सिद्धिः स्यात् । केन‚चि‚{\tiny $_{lb}$}‚च्छ‚क्तिविशेष‚व‚ता म‚त्प्र‚णीतं म‚न्त्र‚मेवंप्र‚युञ्जान‚स्याय‚म‚र्थः सेत्स्य‚तीति स‚म‚य‚स्य ‚{\tiny $_{lb}$}‚कृत‚त्वात् सिद्धिः स्यात् । क‚विस‚म‚यादिव‚त् काव्यापाठ‚क‚स्या\edtext{}{\edlabel{pvv.396-1}\label{pvv.396-1}\lemma{स्या}\Bfootnote{म‚त्काव्यं यः प‚ठ‚ति त‚स्मै म‚येदं देय‚मिति स‚म‚यात् क‚र्त्तुः ।}}र्थ‚सिद्धिः ।
	\pend% ending standard par
      
	  \bigskip
	  \begingroup
	
	    \large
	  
	    \begin{quote}
	  
	    
	    \stanza[\smallbreak]
	\label{pv.3.292b}\flagstanza{\tiny\textenglish{...3.292b}}अथापि भाव‚श‚क्तिः स्याद‚न्य‚थाप्य‚विशेष‚तः ॥ २९२ ॥\&[\smallbreak]


	
	    \end{quote}
	  
	  \endgroup
	

	  \pstart \leavevmode% starting standard par
	\hphantom{.}‚{\color{DodgerBlue3}‚अथापि भाव‚श‚क्तिः‚{\tiny $_{1}$}‚ सादृश्य‚पौरुषेयाणां स्यात्} । य‚या-\edtext{\textsuperscript{*}}{\edlabel{pvv.396-2}\label{pvv.396-2}\lemma{*}\Bfootnote{विधिना नियोगं ।}} ‚{\tiny $_{lb}$}‚भिम‚त‚सिद्धिः । एव‚न्त‚र्ह्य‚{\color{DodgerBlue3}‚न्य‚था} प्र‚युक्ते‚{\color{DodgerBlue3}‚पि} म‚न्त्रे \edtext{}{\edlabel{pvv.396-3}\label{pvv.396-3}\lemma{न्त्रे}\Bfootnote{त‚स्य त‚द्रूप‚विशिष्टं विधिशून्ये विप‚रीत‚पाठादौ च ।}} स्याद‚भिम‚त‚म‚{\color{DodgerBlue3}‚विशेष‚तो} व‚र्ण्णा‚{\tiny $_{lb}$}‚त्म‚क‚स्य म‚न्त्र‚स्य । (२९२)
	\pend% ending standard par
      \label{div_pvv.3.293}
	  
	% new div opening: depth here is 2
	

	  \begin{center}%% label @type='head'
	\textbf{(ग) व‚र्ण‚क्र‚मो म‚न्त्रेष्व‚किञ्चित्क‚रः}
	\end{center}
	

	  \pstart \leavevmode% starting standard par
	व‚र्ण्णानां क्र‚म‚विशेषो म‚न्त्रः त‚त‚स्त‚द्ग्र‚ह‚णेन फ‚ल‚मिति चेत् ।
	\pend% ending standard par
      
	  \bigskip
	  \begingroup
	
	    \large
	  
	    \begin{quote}
	  
	    
	    \stanza[\smallbreak]
	\label{pv.3.293}\flagstanza{\tiny\textenglish{....3.293}}क्र‚म‚स्यार्थान्त‚र‚त्व‚ञ्च पूर्व्व‚मेव निराकृत‚म् ।&नित्त्यं त‚द‚र्थ‚सिद्धिः स्याद‚साम‚र्थ्य‚म‚पेक्ष्य‚ते ॥ २९३ ॥\&[\smallbreak]


	
	    \end{quote}
	  
	  \endgroup
	

	  \pstart \leavevmode% starting standard par
	\hphantom{.}‚{\color{DodgerBlue3}‚क्र‚म‚स्य} व‚र्ण्णेभ्यो‚{\color{DodgerBlue3}‚ऽर्थान्त‚र‚त्व}‚ञ्च ‚{\color{DodgerBlue3}‚पूर्व्व‚मेव निराकृतं} । व‚र्ण्णानुपूर्व्वी वाक्य‚ञ्चे‚{\tiny $_{lb}$}‚दित्य \href{http://sarit.indology.info/?cref=pv.3.259}{(३।२५९)}त्रान्त‚रे । अस्तु वा क्र‚मः त‚स्य नित्य‚त्वात् त‚द‚र्थ‚स्य ‚{\tiny $_{lb}$}‚त‚न्नि‚{\color{DodgerBlue3}‚स्पा} (? ष्पा) द्य‚स्या‚{\color{DodgerBlue3}‚र्थ‚स्य नित्यं सिद्धिः} स्यात् । न ह्य‚विक‚ले \edtext{}{\edlabel{pvv.396-4}\label{pvv.396-4}\lemma{ले}\Bfootnote{आधेयातिश‚य‚स्य ।}}कार‚णे ‚{\tiny $_{lb}$}‚कार्य‚क्षेपो युक्तः । अथ प्र‚योग‚विधानाद्य‚पेक्षास्ति साप्य‚युक्ता । अपेक्ष‚णे क‚स्य‚चित् ‚{\tiny $_{lb}$}‚स‚ह‚कारिणो‚{\color{DodgerBlue3}‚ऽसाम‚र्थ्यं} स्व‚भाव‚तः स्यात् । स‚म‚र्थ‚स्यान्यापेक्षाऽयोगात् । अपेक्ष‚{\tiny $_{lb}$}‚णीयात् स‚म‚र्थ‚स्व‚भावोत्प‚त्तौ तु स्याद‚पेक्षा । (२९३)
	\pend% ending standard par
      \label{div_pvv.3.294}
	  
	% new div opening: depth here is 2
	

	  \pstart \leavevmode% starting standard par
	किञ्च (।)
	\pend% ending standard par
      
	  \bigskip
	  \begingroup
	
	    \large
	  
	    \begin{quote}
	  
	    
	    \stanza[\smallbreak]
	\label{pv.3.294a}\flagstanza{\tiny\textenglish{...3.294a}}स‚र्व‚स्य साध‚नं ते स्युर्भाव‚श‚क्तिर्य‚दीदृशी ।\&[\smallbreak]


	
	    \end{quote}
	  
	  \endgroup
	

	  \pstart \leavevmode% starting standard par
	म‚न्त्राणां भाव‚श‚क्तिर्य‚दीदृशी कार्य‚विशेष‚साधिका त‚दा स‚र्व्व‚स्य य‚ज‚मान‚स्य ‚{\tiny $_{lb}$}‚इत‚र‚स्य\edtext{}{\edlabel{pvv.396-5}\label{pvv.396-5}\lemma{स्य}\Bfootnote{पात‚क्यादेश्च ।}} च ते म‚न्त्रा अभिम‚तार्थ‚सिद्धेः साध‚नं स्युः । ‚{\tiny $_{3}$}‚न हि कार्य‚कारिता तेषां ‚{\tiny $_{lb}$}‚क‚ञ्चिदेव प्र‚ति नेत‚रान् साधार‚ण‚त्वाद् भाव‚स्व‚भाव‚स्य ।
	\pend% ending standard par
      

	  \begin{center}%% label @type='head'
	\textbf{ग. नित्य‚त्वे दोषः}
	\end{center}
	

	  \begin{center}%% label @type='head'
	\textbf{(क) असंस्कार्य‚स्य न प्र‚योक्तृभेदापेक्षा}
	\end{center}
	

	  \pstart \leavevmode% starting standard par
	अथ प्र‚योक्तुर्विशेष‚म‚पेक्ष्य फ‚ल‚प्रायाः । त‚च्चास‚त् ।
	\pend% ending standard par
      \textsuperscript{\textenglish{397/s}}
	  \bigskip
	  \begingroup
	
	    \large
	  
	    \begin{quote}
	  
	    
	    \stanza[\smallbreak]
	\label{pv.3.294b}\flagstanza{\tiny\textenglish{...3.294b}}प्र‚योक्तृभेदापेक्षा च नासंस्कार्य‚स्य युज्य‚ते ॥ २९४ ॥\&[\smallbreak]


	
	    \end{quote}
	  
	  \endgroup
	

	  \pstart \leavevmode% starting standard par
	प्र‚योक्तुर्भेदो य‚ज‚मान‚त्वं त‚द‚पेक्षा च नित्य‚स्य प‚रैर‚संस्कार्य‚स्य न युज्य‚ते । (२९४)
	\pend% ending standard par
      \label{div_pvv.3.295}
	  
	% new div opening: depth here is 2
	
	  \bigskip
	  \begingroup
	
	    \large
	  
	    \begin{quote}
	  
	    
	    \stanza[\smallbreak]
	\label{pv.3.295}\flagstanza{\tiny\textenglish{....3.295}}संस्कार्य‚स्यापि भाव‚स्य व‚स्तुभेदो हि भेद‚कः ।&प्र‚योक्तृभेदान्निय‚मः श‚क्तौ न स‚म‚ये भ‚वेत् ॥ २९५ ॥\&[\smallbreak]


	
	    \end{quote}
	  
	  \endgroup
	

	  \pstart \leavevmode% starting standard par
	संस्कार्य‚स्यापि भाव‚स्य\edtext{}{\edlabel{pvv.397-1}\label{pvv.397-1}\lemma{स्य}\Bfootnote{आधेयातिश‚य‚स्य ।}} व‚स्तुनः\edtext{}{\edlabel{pvv.397-2}\label{pvv.397-2}\lemma{स्तुनः}\Bfootnote{कार‚ण‚स्य ।}} संस्क‚र्त्तुर्भेदो हि भेद‚को भ‚वितुम‚र्ह‚ति । ‚{\tiny $_{lb}$}‚न तु ब्राह्म‚ण‚शूद्रादीनां व‚स्तुतो जातिभेदः क‚श्चिद‚स्ति व्य‚व‚हा/?/र‚मात्र‚त्वात् त‚स्य । ‚{\tiny $_{lb}$}‚त‚त‚श्च प्र‚योक्तुर्भेदाद‚पि म‚न्त्राणां श‚क्तौ निय‚मो न स‚म्भ‚व‚ति । न‚नूप‚ल‚भ्य‚ते ‚{\tiny $_{lb}$}‚निय‚ता म‚न्त्राणां श‚क्तिः । स‚त्य‚मुप‚ल‚भ्य‚ते किन्तु सा पुरुष‚कृते स‚म‚येऽभ्युप‚ग‚म्य‚माने ‚{\tiny $_{lb}$}‚भ‚वेत् न त्व‚कृत‚त्वे । \edtext{\textsuperscript{*}}{\edlabel{pvv.397-3}\label{pvv.397-3}\lemma{*}\Bfootnote{य‚दा तु स‚म‚यो म‚न्त्र‚स्त‚दा स‚म‚य‚क‚र्त्ता व‚स्त्व‚न‚पेक्षः स‚म‚यं क‚रोतीत्याह ।}}यो हि ब्राह्म‚ण इति प्र‚सिद्धः त‚स्यैवायं विधिप्र‚युक्तो म‚न्त्रः ‚{\tiny $_{lb}$}‚फ‚ल‚प्र‚दो नेत‚र‚स्येति क‚र्त्रा पुरुषेण श‚क्तिविशेष‚व‚ता स‚मित‚त्वान्म‚न्त्र‚स्य । (२९५)
	\pend% ending standard par
      \label{div_pvv.3.296}
	  
	% new div opening: depth here is 2
	

	  \begin{center}%% label @type='head'
	\textbf{(ख) नित्त्यानां म‚न्त्राणां प्र‚योज‚को निर‚र्थ‚कः}
	\end{center}
	

	  \pstart \leavevmode% starting standard par
	किञ्च (।)
	\pend% ending standard par
      
	  \bigskip
	  \begingroup
	
	    \large
	  
	    \begin{quote}
	  
	    
	    \stanza[\smallbreak]
	\label{pv.3.296}\flagstanza{\tiny\textenglish{....3.296}}अनाधेय‚विशेषाणां किंकुर्वाणः प्र‚योज‚कः ।&प्र‚योगो य‚द्य‚भिव्य‚क्तिः सा प्रागेव निराकृता ॥ २९६ ॥\&[\smallbreak]


	
	    \end{quote}
	  
	  \endgroup
	

	  \pstart \leavevmode% starting standard par
	नित्या‚{\tiny $_{5}$}‚नाम‚नाधेय‚विशेषाणां प्र‚योक्ता\edtext{}{\edlabel{pvv.397-4}\label{pvv.397-4}\lemma{योक्ता}\Bfootnote{अन्य‚थापि न क‚श्चित् फ‚ल‚म‚श्नुतेऽन्यो न शूद्रादिरिति कुतोयं विभागः ।}} पुरुषः किंकुर्व्वाणः प्र‚योज‚क इष्टः । ‚{\tiny $_{lb}$}‚अनुप‚कार‚क‚स्य प्र‚योज‚क‚त्वे स‚र्व्व‚स्य त‚थात्व‚प्र‚स‚ङ्गात् । य‚द्य‚भिव्य‚क्तिः प्र‚योग ‚{\tiny $_{lb}$}‚उच्य‚ते\edtext{}{\edlabel{pvv.397-5}\label{pvv.397-5}\lemma{ते}\Bfootnote{नोत्पाद‚नं ।}} साऽभिव्य‚क्तिर्नित्यानां प्रागेव सामान्य‚प्र‚स्तावे निराकृता । न हि स्व‚रूप‚{\tiny $_{lb}$}‚प‚रिणाम\edtext{}{\edlabel{pvv.397-6}\label{pvv.397-6}\lemma{रिणाम}\Bfootnote{नित्त्य‚स्य त‚द‚नुप‚प‚त्तेः नित्त्य‚स्य तु युक्तापेक्षा ।}} आव‚र‚ण‚विग‚मो वा सा\edtext{}{\edlabel{pvv.397-7}\label{pvv.397-7}\lemma{सा}\Bfootnote{अभिव्य‚क्तिर्न घ‚ट‚ते ।}}नित्यातां घ‚ट‚ते । (२९६)
	\pend% ending standard par
      \label{div_pvv.3.297}
	  
	% new div opening: depth here is 2
	

	  \begin{center}%% label @type='head'
	\textbf{(ग) नित्य‚स्य व्य‚क्तिर‚सिद्धा}
	\end{center}
	

	  \pstart \leavevmode% starting standard par
	बुद्धिः क‚दाचित् स‚म्भाव्य‚ते\edtext{}{\edlabel{pvv.397-8}\label{pvv.397-8}\lemma{ते}\Bfootnote{नित्त्य‚त्वान्न योग्योत्प‚त्तिः किन्तु श‚ब्द‚विष‚या बुद्धिः ।}} । त‚दा (।)
	\pend% ending standard par
      
	  \bigskip
	  \begingroup
	
	    \large
	  
	    \begin{quote}
	  
	    
	    \stanza[\smallbreak]
	\label{pv.3.297}\flagstanza{\tiny\textenglish{....3.297}}व्य‚क्तिश्च बुद्धिः सा य‚स्मात् स फ‚लैर्य‚दि युज्य‚ते ।&स्याच्छ्रोतुः फ‚ल‚संब‚न्धो व‚क्ता हि व्य‚क्तिकार‚ण‚म् ॥ २९७ ॥\&[\smallbreak]


	
	    \end{quote}
	  
	  \endgroup
	\textsuperscript{\textenglish{398/s}}

	  \pstart \leavevmode% starting standard par
	\hphantom{.}त‚द्‚{\color{DodgerBlue3}‚व्य‚क्तिश्च बुद्धि}‚रुच्येत । ‚{\color{DodgerBlue3}‚सा\edtext{}{\edlabel{pvv.398-1}\label{pvv.398-1}\lemma{सा}\Bfootnote{व्य‚क्तिः ।}}य‚स्माद्} व‚क्तु \edtext{}{\edlabel{pvv.398-2}\label{pvv.398-2}\lemma{क्तु}\Bfootnote{पुरुषात् ।}} र्भ‚व‚ति ‚{\color{DodgerBlue3}‚स फ‚{\tiny $_{6}$}‚ लेन युज्य‚ते य‚दि} त‚दा नातिप्र‚स‚ङ्गः । न‚न्वेम‚पि ‚{\color{DodgerBlue3}‚श्रोतुः फ‚ल‚स‚म्ब‚न्धः \edtext{}{\edlabel{pvv.398-3}\label{pvv.398-3}\lemma{न्धः}\Bfootnote{यो व‚क्त्रा प‚ठ्य‚मानं मंत्रं शृणोति ।}}} स्यान्न व‚क्तुरेव । \edtext{\textsuperscript{*}}{\edlabel{pvv.398-4}\label{pvv.398-4}\lemma{*}\Bfootnote{य‚स्मात् ।}} ‚{\color{DodgerBlue3}‚व‚क्ता ‚{\tiny $_{lb}$}‚हि व्य‚क्ते}‚र्ज्ञान‚स्य ‚{\color{DodgerBlue3}‚कार‚ण\edtext{}{\edlabel{pvv.398-5}\label{pvv.398-5}\lemma{ण}\Bfootnote{नित्ये श‚ब्दे बुद्धिज‚न्मार्थं पुंसो व्यापाराभा (वा) द‚नुप‚कार्योप‚कार‚क‚त्वान्न ‚{\tiny $_{lb}$}‚व‚क्ता श्रोतुरुप‚कार इति विशेषोन‚योर्न्नास्ति ।}}मिति} फ‚ले युज्य‚ते । त‚च्च \edtext{}{\edlabel{pvv.398-6}\label{pvv.398-6}\lemma{च्च}\Bfootnote{म‚न्त्र‚विष‚यं ।}}ज्ञान‚हेतुत्वं श्रोतुर‚प्य‚स्त्येव ।\edtext{\textsuperscript{*}}{\edlabel{pvv.398-7}\label{pvv.398-7}\lemma{*}\Bfootnote{किं न फ‚लेन युज्य‚ते ।}} ‚{\tiny $_{lb}$}‚(२९७)
	\pend% ending standard par
      \label{div_pvv.3.298}
	  
	% new div opening: depth here is 2
	

	  \pstart \leavevmode% starting standard par
	किञ्च (।)
	\pend% ending standard par
      
	  \bigskip
	  \begingroup
	
	    \large
	  
	    \begin{quote}
	  
	    
	    \stanza[\smallbreak]
	\label{pv.3.298}\flagstanza{\tiny\textenglish{....3.298}}अन‚भिव्य‚क्त‚श‚ब्दानां क‚र‚णानां प्र‚योज‚न‚म् ।&म‚नोज‚पो वा व्य‚र्थः स्याच्छ‚ब्दो हि श्रोत्र‚गोच‚रः ॥ २९८ ॥\&[\smallbreak]


	
	    \end{quote}
	  
	  \endgroup
	

	  \pstart \leavevmode% starting standard par
	\hphantom{.}‚{\color{DodgerBlue3}‚अन‚भिव्य\edtext{}{\edlabel{pvv.398-8}\label{pvv.398-8}\lemma{भिव्य}\Bfootnote{श्रोत्र‚विष‚यं न नीतः यैस्ताल्वादिभिः ।}}क्तो}‚ऽनिवेदितः \edtext{}{\edlabel{pvv.398-9}\label{pvv.398-9}\lemma{ऽनिवेदितः}\Bfootnote{प्र‚योगो व्य‚र्थः स्यादिति संब‚न्धः । य‚त्रौष्ठ‚प्र‚स्प‚न्द‚मात्रेण उपांशुज‚पः क्रिय‚ते ‚{\tiny $_{lb}$}‚सोपि व्य‚र्थः स्यादित्यादावुत्प‚त्य व्य‚क्तिं नित्येषु अनाधेयातिश‚येषु ।}} ‚{\color{DodgerBlue3}‚श‚ब्दो} यैस्तेषां ‚{\color{DodgerBlue3}‚क‚र‚णा}‚दीनां ‚{\color{DodgerBlue3}‚प्र‚योज‚नं} प्र‚योगः ‚{\tiny $_{lb}$}‚\leavevmode\ledsidenote{\textenglish{79a/MA}} य‚दा । त‚दा ताल्वादिक‚र‚ण‚प्र‚स्प‚न्द‚मात्र‚यो\edtext{}{\edlabel{pvv.398-10}\label{pvv.398-10}\lemma{यो}\Bfootnote{य‚द‚पि विशिष्टः प्र‚योक्ता म‚न्त्र‚फ‚ल‚म‚श्नुत इति त‚त्रापि स‚मीरितार्थोत्पाद‚{\tiny $_{lb}$}‚योग्योत्पाद‚न‚मुत्प‚न्न‚स्य उत्त‚रोत्त‚र‚विशेषोत्पाद‚न‚म्वार्थः स‚विशेष‚ज‚न्म‚नि स्यात् ।}}पांशुज‚पः क्रिय‚ते न तु व्य‚क्त‚मुच्य‚ते । ‚{\tiny $_{lb}$}‚य‚दा च ताम‚पि ‚{\tiny $_{7}$}‚ विना \edtext{}{\edlabel{pvv.398-11}\label{pvv.398-11}\lemma{विना}\Bfootnote{ताल्वादिक‚र‚ण‚प्र‚स्य‚न्द‚मात्रां ।}}म‚नोमात्रेण ‚{\color{DodgerBlue3}‚म‚नोज‚पो वा} क्रिय‚ते त‚दा द्वाव‚पि ज‚पाविमौ ‚{\tiny $_{lb}$}‚‚{\color{DodgerBlue3}‚व्य‚र्थो} स्यातां । \edtext{\textsuperscript{*}}{\edlabel{pvv.398-12}\label{pvv.398-12}\lemma{*}\Bfootnote{य‚स्माच्छ्रोत्र‚ग्राह्य एव श‚ब्दः त‚त्स्व‚भाव‚श्च म‚न्त्रः । उपांशुम‚नोज‚प‚{\tiny $_{lb}$}‚योश्च श्रोत्र‚ग्र‚ह‚णाभावाद‚श‚ब्द‚त्वं त‚त्त्वाद‚म‚न्त्रः क‚थं फ‚ल‚वान् । अय‚माश‚यः श‚ब्दा‚{\tiny $_{lb}$}‚त्म‚नां म‚न्त्राणां व्य‚क्तिहेतुः श‚ब्द‚ग्राहिज्ञान‚हेतुपुरुषः प्र‚योक्ता त‚स्य फ‚लेन स‚म्ब‚न्धः ‚{\tiny $_{lb}$}‚चेत् । त‚दोपांशुम‚नोजापी न फ‚ल‚योगी स्याद‚भिव्य‚क्त्य‚हेतुत्वात् । म‚न्त्र‚स्य श्रोत्र‚{\tiny $_{lb}$}‚ग्राह्यापौरुषेयं नित्य‚त्वाभ्युप‚ग‚मात् ।}} ‚{\color{DodgerBlue3}‚श‚ब्दो हि श्रोत्र‚गोच‚र} उच्य‚ते न च ज‚प‚योर‚न‚योः श्रोत्र‚गोच‚रः ‚{\tiny $_{lb}$}‚क‚श्चिद‚स्ति । (२९८)
	\pend% ending standard par
      \label{div_pvv.3.299}
	  
	% new div opening: depth here is 2
	
	  \bigskip
	  \begingroup
	
	    \large
	  
	    \begin{quote}
	  
	    
	    \stanza[\smallbreak]
	\label{pv.3.299}\flagstanza{\tiny\textenglish{....3.299}}पार‚म्प‚र्येण त‚ज्ज‚त्वात् त‚द्व्य‚क्तिः सापि चेन्म‚तिः ।&ते त‚था स्युस्त‚द‚र्था चेद‚सिद्धं क‚ल्प‚नान्व‚यात् ॥ २९९ ॥\&[\smallbreak]


	
	    \end{quote}
	  
	  \endgroup
	\textsuperscript{\textenglish{399/s}}

	  \pstart \leavevmode% starting standard par
	न‚नु याप्युपांशुम‚नोज‚प‚काले श‚ब्दाभासा धीः सापि\edtext{}{\edlabel{pvv.399-1}\label{pvv.399-1}\lemma{सापि}\Bfootnote{त‚द्व्य‚क्तिस्त‚स्य श‚ब्द‚स्य व्य‚क्तिर्ज्ञानं पूर्व्व‚श‚ब्द‚ज‚ज्ञाना हि संस्कार‚स्य म‚नो‚{\tiny $_{lb}$}‚ज‚पे श‚ब्द‚प्र‚तिभासात् म‚न्त्र‚त्वं ।}} म‚तिः ‚{\color{DodgerBlue3}‚पार‚म्प‚र्येण ‚{\tiny $_{lb}$}‚त‚ज्ज‚त्वात्} त‚स्य ‚{\color{DodgerBlue3}‚व्य‚क्ति}‚रिति चेत् । य‚द्येवं श‚ब्दाविक‚ल्प‚व‚द‚र्थ‚विक‚ल्पा \edtext{}{\edlabel{pvv.399-2}\label{pvv.399-2}\lemma{ल्पा}\Bfootnote{संकेतात् किल वादिनो म‚नोज‚पादिसाफ‚ल्य‚माह दृश्य‚विक‚ल्प्यैक्यात् ।}} अपि ‚{\tiny $_{lb}$}‚त‚त्प्र‚भ‚वा इति ‚{\color{DodgerBlue3}‚ते}‚पि \edtext{}{\edlabel{pvv.399-3}\label{pvv.399-3}\lemma{पि}\Bfootnote{विक‚ल्पा अपि म‚न्त्रा इति त‚त्प्र‚योक्तापि फ‚ल‚भाग् स्यात् । एतेनातिप्र‚स‚ङ्ग उक्तः ।}} ‚{\color{DodgerBlue3}‚त‚था} श‚ब्द‚व्य‚क्त‚यः ‚{\color{DodgerBlue3}‚स्युः} । प‚रः न केव‚लाच्छ‚ब्द‚प्र‚भ‚{\tiny $_{1}$}‚‚{\tiny $_{lb}$}‚व‚त्वात् त‚द्व्य‚क्तिः\edtext{}{\edlabel{pvv.399-4}\label{pvv.399-4}\lemma{क्तिः}\Bfootnote{य‚त‚स्त‚द्वान् प्र‚योक्ता स्यात् ।}} किन्तु ‚{\color{DodgerBlue3}‚त‚द‚र्था} श‚ब्द‚विष‚या स‚तीति ‚{\color{DodgerBlue3}‚चेत्} न चार्थ‚विक‚ल्पः श‚ब्द‚{\tiny $_{lb}$}‚विष‚यः । श‚ब्द‚विक‚ल्प‚स्य \edtext{}{\edlabel{pvv.399-5}\label{pvv.399-5}\lemma{स्य}\Bfootnote{स्व‚ल‚क्ष‚णाविष‚य‚त्वाद् विक‚ल्प‚स्य स‚म‚य‚काराभिप्राय‚संवाद‚नात् फ‚ल‚{\tiny $_{lb}$}‚प्राप्तेः । न तु व‚स्तुन्य‚विरोधः श‚ब्दात् फ‚ले म‚न‚न‚म‚श‚ब्दो य‚तः ।}} श‚ब्द‚विष‚य‚त्व‚म‚सिद्धं ‚{\color{DodgerBlue3}‚क‚ल्प‚नाया} वाच्य‚वाच‚क‚योज‚{\tiny $_{lb}$}‚नाया ‚{\color{DodgerBlue3}‚अन्व‚यात्} स‚म्ब‚न्धात् न च क‚ल्प‚ना व‚स्तुविष‚येति \edtext{}{\edlabel{pvv.399-6}\label{pvv.399-6}\lemma{येति}\Bfootnote{वास‚नाप्र‚बाधादुत्प‚त्तेर्ब्बाह्यास‚त्त्वेपि ।}}क‚थ्य‚ते । (२९९)
	\pend% ending standard par
      \label{div_pvv.3.300}
	  
	% new div opening: depth here is 2
	

	  \begin{center}%% label @type='head'
	\textbf{घ. स‚म‚य‚काण्णामुक्त्या फ‚ल‚विशेषः}
	\end{center}
	

	  \pstart \leavevmode% starting standard par
	अस्माक‚न्तु म‚ते (।)
	\pend% ending standard par
      
	  \bigskip
	  \begingroup
	
	    \large
	  
	    \begin{quote}
	  
	    
	    \stanza[\smallbreak]
	\label{pv.3.300}\flagstanza{\tiny\textenglish{....3.300}}स्व‚सामान्य‚स्व‚भावानामेक‚भाव‚विव‚क्ष‚या ।&उक्तेः स‚म‚य‚काण्णाम‚विरोधो न व‚स्तुनि ॥ ३०० ॥\&[\smallbreak]


	
	    \end{quote}
	  
	  \endgroup
	

	  \pstart \leavevmode% starting standard par
	स्व‚सामान्य‚स्व‚भावानां श‚ब्द‚स्व‚ल‚क्ष‚ण‚सामान्य‚ल‚क्ष‚णानां प्र‚त्य‚क्ष‚विक‚ल्प‚बुद्धि‚{\tiny $_{lb}$}‚विष‚याणामेक\edtext{}{\edlabel{pvv.399-7}\label{pvv.399-7}\lemma{याणामेक}\Bfootnote{दृश्य‚विक‚ल्पैक्यात् ।}}त्वाध्य‚व‚साय‚व‚शादेक‚भाव‚स्यैक‚त्व‚स्य ।‚{\tiny $_{2}$}‚ विव‚क्ष‚या स‚म‚य‚काराणां स‚म‚य‚{\tiny $_{lb}$}‚क‚र्त्तृभिः श‚क्तिम‚त्पुंभिर्म‚न्त्राणामुक्तेर‚विरोधः । उपांशुज‚प‚म‚नोज‚प‚योर्व्वैफ‚ल्य\edtext{}{\edlabel{pvv.399-8}\label{pvv.399-8}\lemma{ल्य}\Bfootnote{स‚म‚य‚काराभिप्राय‚स‚म्पाद‚नात् फ‚ल‚प्राप्तेः न तु व‚स्तुन्य‚विरोधः श‚ब्दात् फ‚ले ‚{\tiny $_{lb}$}‚म‚न‚न‚म‚श‚ब्दो य‚तः ।}}‚{\tiny $_{lb}$}‚विरोधाभावो दृश्य‚विक‚ल्प्यावेकाध्य‚व‚सायादेक‚कार्य‚कारित्वेनाधिष्ठित‚त्वात् ‚{\tiny $_{lb}$}‚त‚त्कुरुतः । ये तु व‚स्तुभूतं म‚न्त्र‚मिच्छ‚न्ति तेषाम्विरोध एव त‚दाह । न व‚स्तुनि ‚{\tiny $_{lb}$}‚विरोधाभावः । न ह्युपांशुज‚पादिविष‚यो व‚स्तुक‚ल्पित‚त्वात्\edtext{}{\edlabel{pvv.399-9}\label{pvv.399-9}\lemma{त्वात्}\Bfootnote{य‚दुक्तं व‚र्ण्णा एव म‚न्त्र‚स्त‚त्र ।}}। (३००)
	\pend% ending standard par
      \label{div_pvv.3.301}
	  
	% new div opening: depth here is 2
	\textsuperscript{\textenglish{400/s}}

	  \begin{center}%% label @type='head'
	\textbf{ङ व‚र्ण्णानुपूर्विचिन्ता}
	\end{center}
	

	  \begin{center}%% label @type='head'
	\textbf{(क) आनुपूर्व्य‚भावे नार्थ‚भेदः}
	\end{center}
	

	  \pstart \leavevmode% starting standard par
	अथ त्व‚न्म‚तेपि (।)
	\pend% ending standard par
      
	  \bigskip
	  \begingroup
	
	    \large
	  
	    \begin{quote}
	  
	    
	    \stanza[\smallbreak]
	\label{pv.3.301}\flagstanza{\tiny\textenglish{....3.301}}आनुपूर्व्याम‚स‚त्यां स्यात् स‚रो र‚स इति श्रुतौ ।&न कार्य‚भेद इति चेद् अस्ति सा पुरुषाश्र‚या ॥ ३०१ ॥\&[\smallbreak]


	
	    \end{quote}
	  
	  \endgroup
	

	  \pstart \leavevmode% starting standard par
	आनुपू‚{\tiny $_{3}$}‚र्व्या व‚र्ण्ण‚व्य‚तिरिक्तायाम‚स‚त्यां स‚रो र‚स इति प्र‚सिद्धानुलोम‚विलोम ‚{\tiny $_{lb}$}‚क्र‚मायां श्रुतौ कार्य‚स्य स्व‚ज्ञान‚स्य भेदो न स्यादिति चेत् । अस्त्य‚स्म‚न्म‚ते सानु‚{\tiny $_{lb}$}‚पूर्व्वी पुरुषाश्र‚या\edtext{}{\edlabel{pvv.400-1}\label{pvv.400-1}\lemma{या}\Bfootnote{व‚र्ण्णाव्य‚तिरिक्ता ।}} पुरुष‚कृता प्र‚तिप‚दं\edtext{}{\edlabel{pvv.400-2}\label{pvv.400-2}\lemma{दं}\Bfootnote{त‚त्रैक‚त्वाध्य‚व‚सायः प‚रं म‚न्दानां ।}}भिन्ना त‚तो न ज्ञान‚भेद‚प्र‚स‚ङ्गः । (३०१)
	\pend% ending standard par
      \label{div_pvv.3.302}
	  
	% new div opening: depth here is 2
	

	  \pstart \leavevmode% starting standard par
	त‚था हि (।)
	\pend% ending standard par
      
	  \bigskip
	  \begingroup
	
	    \large
	  
	    \begin{quote}
	  
	    
	    \stanza[\smallbreak]
	\label{pv.3.302}\flagstanza{\tiny\textenglish{....3.302}}यो य‚द्व‚र्ण‚स‚मुत्थान‚ज्ञान‚जाज्ज्ञान‚तो ध्व‚निः ।&जाय‚ते त‚दुपाधिः स श्रुत्या स‚म‚व‚सीय‚ते ॥ ३०२ ॥\&[\smallbreak]


	
	    \end{quote}
	  
	  \endgroup
	

	  \pstart \leavevmode% starting standard par
	यो ध्व‚निर्जाय‚ते त‚द्व‚र्ण्ण‚स‚मुत्थो ज्ञान‚तः । य‚श्चासौ व‚र्ण्ण‚श्च य‚द्व‚र्ण्ण‚स्त‚स्य ‚{\tiny $_{lb}$}‚स‚मुत्थानं कार‚णं त‚च्च त‚त् ज्ञान‚ञ्च य‚द्व‚र्ण्ण‚स‚मुत्थान‚{\tiny $_{4}$}‚ज्ञानं त‚स्माज्जातं य‚द्व‚र्ण्ण‚{\tiny $_{lb}$}‚स‚मुत्थान‚ज्ञान‚जं त‚स्मात् ज्ञान‚म‚तः । अय‚म‚र्थ आद्य‚स्य व‚र्ण्ण\edtext{}{\edlabel{pvv.400-3}\label{pvv.400-3}\lemma{र्ण्ण}\Bfootnote{व‚क्तृस्थं पूर्व‚पूर्व‚व‚र्ण्ण‚स‚मुत्थाप‚क‚हेतु ।}}स्य य‚त्स‚मुत्थाप‚कं ‚{\tiny $_{lb}$}‚विव‚क्षात्म‚कं ज्ञानं तेन स‚म‚न‚न्त‚र‚प्र‚त्य‚येन स‚ता द्वितीय‚व‚र्ण्ण‚स‚मुत्थाप‚कं ज्ञानं ज‚न्य‚ते ‚{\tiny $_{lb}$}‚तेन च द्वितीयो व‚र्ण्ण एवं द्वितीय‚व‚र्ण्ण‚स‚मुत्थाप‚कात् ज्ञानात् तृतीय‚व‚र्ण्णेत्थाप‚क‚{\tiny $_{lb}$}‚ज्ञानोत्प‚त्तौ तृतीय‚व‚र्णोत्प‚त्तिरिति कार‚ण‚क्र‚माद्व‚र्णोत्प‚त्तिक्र‚म उक्तः\edtext{}{\edlabel{pvv.400-4}\label{pvv.400-4}\lemma{उक्तः}\Bfootnote{कार‚ण‚भेदात्कार्य‚भेद‚मुक्त्वा ।}} । पुनः ‚{\tiny $_{lb}$}‚कार्य‚{\tiny $_{5}$}‚क्र‚मेण क्र‚मं द‚र्श‚यितु माह\edtext{}{\edlabel{pvv.400-5}\label{pvv.400-5}\lemma{माह}\Bfootnote{व‚र्णाश्च क्र‚मेणोत्प‚न्नाः श्रोतृसंतान‚स्थानां स्व‚विष‚य‚ज्ञानानां क्र‚मेण हेत‚वो ‚{\tiny $_{lb}$}‚भ‚व‚न्तो जाय‚न्त इति द‚र्श‚य‚न्नाह ।}}। स उत्त‚रो व‚र्ण‚स्त‚दुपाधिः पूर्ण‚व‚र्ण‚विशेष‚ण‚स्त‚{\tiny $_{lb}$}‚द‚न‚न्त‚र इत्य‚र्थः । श्रुत्या श्र‚व‚ण‚ज्ञानेन ग्राह्य‚व‚र्ण‚कार्येण श्रोतृस‚न्तान‚व‚र्त्तिना ‚{\tiny $_{lb}$}‚स‚म‚व‚सीय‚ते । क्र‚मोत्प‚न्ना व‚र्णाः स्व‚स्व‚ज‚नित‚ज्ञानैर‚स‚ह‚भाविन एव गृह्य‚न्ते । (३०२)
	\pend% ending standard par
      \label{div_pvv.3.303}
	  
	% new div opening: depth here is 2
	

	  \pstart \leavevmode% starting standard par
	न‚नु क्र‚म‚भाविनां स‚ह‚द‚र्श‚नाभावात्क‚थं पूर्व‚व‚र्ण्णोपाधि\edtext{}{\edlabel{pvv.400-6}\label{pvv.400-6}\lemma{र्ण्णोपाधि}\Bfootnote{प‚र‚काले पूर्व‚व‚र्ण‚ध्वंसात् ।}} ग्र‚ह‚ण‚मित्याह (।)
	\pend% ending standard par
      \textsuperscript{\textenglish{401/s}}
	  \bigskip
	  \begingroup
	
	    \large
	  
	    \begin{quote}
	  
	    
	    \stanza[\smallbreak]
	\label{pv.3.303}\flagstanza{\tiny\textenglish{....3.303}}त‚ज्ज्ञान‚ज‚नित‚ज्ञानः स श्रुताव‚प‚टुश्रुतिः ।&अपेक्ष्य त‚त्स्मृतिं प‚श्चात् स्मृतिमाध‚त्त आत्म‚नि ॥ ३०३ ॥\&[\smallbreak]


	
	    \end{quote}
	  
	  \endgroup
	

	  \pstart \leavevmode% starting standard par
	\hphantom{.}त‚स्य पूर्व्व‚व‚र्ण‚स्य ‚{\color{DodgerBlue3}‚ज्ञानेन} ग्राह‚के‚{\tiny $_{6}$}‚णोत्त‚र‚व‚र्ण्ण‚स‚ह‚कारिणा ‚{\color{DodgerBlue3}‚ज‚नितं} ग्राह‚कं\edtext{}{\edlabel{pvv.401-1}\label{pvv.401-1}\lemma{कं}\Bfootnote{साकाराल‚म्ब‚नं ज्ञान‚काल एवाकार‚स‚मुत्थाप‚क‚चित्तेनाकारो ज‚नित इति स‚म‚काल‚ता ।}} ‚{\color{DodgerBlue3}‚ज्ञानं} य‚स्मिन् स त‚त् ज्ञान‚ज‚नित उत्त‚रो व‚र्ण्णः म‚न्द‚मुच्चार्य‚माण‚त्वात्\edtext{}{\edlabel{pvv.401-2}\label{pvv.401-2}\lemma{त्वात्}\Bfootnote{श‚नैरुच्चारितो य‚दा व‚र्ण्णः ।}} श्रुतौ ‚{\tiny $_{lb}$}‚श्र‚व‚ण‚ज्ञानेऽ‚{\color{DodgerBlue3}‚प‚ट‚श्रुति}‚र्म‚न्द‚चारि\edtext{}{\edlabel{pvv.401-3}\label{pvv.401-3}\lemma{चारि}\Bfootnote{यो म‚न‚सापि ज‚पेत् त‚द‚र्थ‚क‚रोह‚मिति य‚त्र विभ‚क्ता व‚र्ण्णा अव‚धार्य‚न्तेव‚स्थायां ।}}श्र‚व‚ण‚ज्ञानः । त‚स्य पूर्व्व‚व‚र्ण्ण‚स्य ‚{\color{DodgerBlue3}‚स्मृतिम‚पेक्ष्या\edtext{}{\edlabel{pvv.401-4}\label{pvv.401-4}\lemma{पेक्ष्या}\Bfootnote{त्व‚रिते क्र‚म‚श्रुतिः कुतः ।}}‚{\tiny $_{lb}$}‚त्म‚नि} स्मृतिं पूर्व्व‚व‚ण‚निन्त‚र‚त्वेनाध‚त्ते (।) य‚स्मात् पूर्व्व‚व‚र्णान‚न्त‚र‚त्वेनोत्त‚रः ‚{\tiny $_{lb}$}‚स्म‚र्य‚ते (।) त‚स्मात्त‚द‚न‚न्त‚र एवासौ गृहीत इत्य‚र्थः । (३०३)
	\pend% ending standard par
      \label{div_pvv.3.304}
	  
	% new div opening: depth here is 2
	

	  \begin{center}%% label @type='head'
	\textbf{(ख) आनुपूर्वी पौरुषेयी}
	\end{center}
	
	  \bigskip
	  \begingroup
	
	    \large
	  
	    \begin{quote}
	  
	    
	    \stanza[\smallbreak]
	\label{pv.3.304}\flagstanza{\tiny\textenglish{....3.304}}इत्येषा पौरुषेय्येव त‚द्धेतुग्रीहिचेत‚साम् ।&कार्य‚कार‚ण‚ता व‚र्णे ह्यानुपूर्वीति क‚थ्य‚ते ॥ ३०४ ॥\&[\smallbreak]


	
	    \end{quote}
	  
	  \endgroup
	

	  \pstart \leavevmode% starting standard par
	इत्येव‚मुक्तेन क्र‚मे‚{\tiny $_{7}$}‚ण \edtext{}{\edlabel{pvv.401-5}\label{pvv.401-5}\lemma{ण}\Bfootnote{एवं रेफाकार‚विस‚र्ज‚नीयोत्थाप‚कानि पूर्व‚पूर्व‚स‚म‚न‚न्त‚र‚प्र‚त्य‚यानि ।}} व‚र्ण्णेषु क्र‚म‚भाविषु त‚द्धेतुचेत‚सां व‚र्ण्णो\edtext{}{\edlabel{pvv.401-6}\label{pvv.401-6}\lemma{र्ण्णो}\Bfootnote{कार्य‚ज‚न्य‚त्वात् ।}}त्थाप‚क‚चेत‚सां\leavevmode\ledsidenote{\textenglish{79b/MA}} ‚{\tiny $_{lb}$}‚व‚क्तृस‚न्तान‚व‚र्तिनां । ‚{\color{DodgerBlue3}‚त‚द्ग्राहिचेत‚सां} व‚र्ण्ण‚ग्राह‚क‚चेत‚सां श्रोतृस‚न्तान‚व‚र्तिनां ‚{\tiny $_{lb}$}‚‚{\color{DodgerBlue3}‚कार्य‚कार‚ण‚तैषा} य‚थायोगं व‚र्ण्णापेक्षा क्र‚म‚व‚ती कार‚ण‚ता कार्य‚ता चा‚{\color{DodgerBlue3}‚नुपूर्वी} क‚थ्य‚ते (।) सा च पुरुष‚निर्व‚र्त्त्य‚त्वात् ‚{\color{DodgerBlue3}‚पौरुषेय्येव} । (३०४)
	\pend% ending standard par
      \label{div_pvv.3.305}
	  
	% new div opening: depth here is 2
	
	  \bigskip
	  \begingroup
	
	    \large
	  
	    \begin{quote}
	  
	    
	    \stanza[\smallbreak]
	\label{pv.3.305}\flagstanza{\tiny\textenglish{....3.305}}अन्य‚देव त‚तो रूपं त‚द्व‚र्ण्णानां प‚दे प‚दे ।&क‚र्त्तृसंस्कार‚तो भिन्नं स‚हितं कार्य‚भेद‚कृत् ॥ ३०५ ॥\&[\smallbreak]


	
	    \end{quote}
	  
	  \endgroup
	

	  \pstart \leavevmode% starting standard par
	\hphantom{.}य‚तो न नित्य‚त्वं ‚{\color{DodgerBlue3}‚त‚तोन्य‚देव व‚र्ण्णानां त‚द्रूपं प‚दे प‚दे} प्र‚तिप‚द‚मेकाध्य‚व‚साय‚विष‚{\tiny $_{lb}$}‚य‚त्वेपि\edtext{}{\edlabel{pvv.401-7}\label{pvv.401-7}\lemma{त्वेपि}\Bfootnote{अन्य‚था देशाद्य‚निय‚मान्निय‚मे च देशादेरेवेन्ध‚न‚त्व ।}} ‚{\color{DodgerBlue3}‚क‚र्त्तृ\edtext{}{\edlabel{pvv.401-8}\label{pvv.401-8}\lemma{र्त्तृ}\Bfootnote{क्र‚म‚भेद एव व‚र्ण्ण‚भेदः केव‚लं र‚स‚प‚दात्र स प‚दान्त‚र‚स्याभेदो नाव‚धार्य‚ते ।}}सं‚{\tiny $_{1}$}‚स्कार‚तो} व‚र्ण्ण‚स‚मुत्थाप‚क‚चित्त‚श‚क्तिभेदाज्जातं ‚{\color{DodgerBlue3}‚भिन्नं\edtext{}{\edlabel{pvv.401-9}\label{pvv.401-9}\lemma{भिन्नं}\Bfootnote{य‚तः ।}}} । क्र‚मेण ‚{\tiny $_{lb}$}‚चानुभूय स्मृत्या ‚{\color{DodgerBlue3}‚स‚हितं} स्मृतं स‚त् ‚{\color{DodgerBlue3}‚कार्य‚भेद‚कृद‚र्थ}‚प्र‚तिप‚त्तिविशेष‚कारि । (३०५)
	\pend% ending standard par
      \label{div_pvv.3.306_3.307}
	  
	% new div opening: depth here is 2
	\textsuperscript{\textenglish{402/s}}
	  \bigskip
	  \begingroup
	
	    \large
	  
	    \begin{quote}
	  
	    
	    \stanza[\smallbreak]
	\label{pv.3.306}\flagstanza{\tiny\textenglish{....3.306}}सा चानुपूर्वी व‚र्ण्णानां त‚द्धेतुग्राहिचेत‚साम् ।&इच्छाऽविरुद्ध‚सिद्धीनां स्थितिक्र‚म‚विरोध‚तः ॥ ३०६ ॥\&[\smallbreak]


	
	    \end{quote}
	  
	  \endgroup
	
	  \bigskip
	  \begingroup
	
	    \large
	  
	    \begin{quote}
	  
	    
	    \stanza[\smallbreak]
	\label{pv.3.307}\flagstanza{\tiny\textenglish{....3.307}}कार्य‚कार‚ण‚तासिद्धेः पुंभ्यो व‚र्ण‚क्र‚म‚स्य च ।&स‚र्वो व‚र्ण‚क्र‚मः पुंभ्यो द‚ह‚नेन्ध‚न‚युक्तिव‚त् ॥ ३०७ ॥\&[\smallbreak]


	
	    \end{quote}
	  
	  \endgroup
	

	  \pstart \leavevmode% starting standard par
	\hphantom{.}‚{\color{DodgerBlue3}‚सा च व‚र्णानामानुपूर्व्वी\edtext{}{\edlabel{pvv.402-1}\label{pvv.402-1}\lemma{र्णानामानुपूर्व्वी}\Bfootnote{(व‚र्ण्ण‚निर‚र्थ‚क‚त्वेपि एक‚विक‚ल्पेन विष‚यीकृताः क्र‚मिणो व‚र्ण्णाः स‚हिता उक्ताः) ।}}} त‚द्धेतुग्राहिचेत‚सां र‚च‚नाकृतः पुरुषात्प्र‚वृत्तेति न ‚{\tiny $_{lb}$}‚स्थित‚क्र‚मा व‚र्णाः पुरुषेच्छ‚याऽविरुद्ध‚सिद्धीनां व‚र्ण्णानां स्थित‚स्य क्र‚म‚स्य विरोधात् ‚{\tiny $_{lb}$}‚(।) न हि स्थित‚क्र‚माणां हिम‚व‚द्विन्ध्य‚म‚ल‚यानामिच्छ‚या विप‚रीत‚क्र‚मः श‚क्यः (।) ‚{\tiny $_{lb}$}‚व‚र्ण्णास्त्विच्छ‚या विप‚र्य्यास्य‚न्तें विक‚ल्प‚क्र‚मानुविधायित्वाच्च (।) त‚द्व‚त् क‚ल्पितो‚{\tiny $_{lb}$}‚प्य‚र्थो न प्र‚माणं ‚{\color{DodgerBlue3}‚कार्य‚कार‚ण‚तायाः सिद्धेः (।) स‚र्व्वो} वैदिकोऽन्य‚श्च ‚{\color{DodgerBlue3}‚व‚र्ण्ण‚क्र‚मः\edtext{}{\edlabel{pvv.402-2}\label{pvv.402-2}\lemma{मः}\Bfootnote{(पुरुषः कार‚णं व‚र्ण्ण‚क्र‚म‚स्येति) ।}} ‚{\tiny $_{lb}$}‚पुंभ्यो\edtext{}{\edlabel{pvv.402-3}\label{pvv.402-3}\lemma{पुंभ्यो}\Bfootnote{पुंविक‚ल्पानुक्र‚मे स‚ति भावाद‚स‚ति वा भावात् ।}}} भ‚व‚तीत्य‚व‚धार‚णं । ‚{\color{DodgerBlue3}‚द‚ह‚नेन्ध‚न‚युक्तिव‚त्} (।) \edtext{\textsuperscript{*}}{\edlabel{pvv.402-4}\label{pvv.402-4}\lemma{*}\Bfootnote{त‚था लोक‚व‚र्ण्ण‚क्र‚म‚व‚द् वैदिकः साध्य‚ते ।}}य‚थैक‚स्य द‚ह‚न‚स्य इन्ध‚न‚पूर्व्व‚{\tiny $_{lb}$}‚क‚त्व‚दृष्ट्या स‚र्व्वोऽग्निरिन्ध‚न‚पूर्व्व इति न्यायः । (३०६, ३०७)
	\pend% ending standard par
      \label{div_pvv.3.308}
	  
	% new div opening: depth here is 2
	
	  \bigskip
	  \begingroup
	
	    \large
	  
	    \begin{quote}
	  
	    
	    \stanza[\smallbreak]
	\label{pv.3.308}\flagstanza{\tiny\textenglish{....3.308}}असाधार‚ण‚ता सिद्धा पुंसां च क्र‚म‚कारिणां ।&अतो ज्ञान‚प्र‚भावाभ्याम‚न्येषां त‚द‚भाव‚तः ॥ ३०८ ॥\&[\smallbreak]


	
	    \end{quote}
	  
	  \endgroup
	

	  \pstart \leavevmode% starting standard par
	अतएव च म‚न्त्राख्य‚{\tiny $_{2}$}‚ व‚र्ण्ण‚क्र‚म‚कारिणां पुंसां ज्ञान\edtext{}{\edlabel{pvv.402-5}\label{pvv.402-5}\lemma{ज्ञान}\Bfootnote{स‚मीहित‚फ‚ल‚साध‚न‚व‚र्ण्ण‚क्र‚म‚ज्ञानं स‚मीहित‚संपाद‚न‚श‚क्तिप्र‚भावः ।}}प्र‚भावाभ्यां पुरुषान्त‚रैः ‚{\tiny $_{lb}$}‚स‚हासा‚{\color{DodgerBlue3}‚धार‚ण‚ता}‚ऽस‚मान‚ता\edtext{}{\edlabel{pvv.402-6}\label{pvv.402-6}\lemma{ता}\Bfootnote{त‚द‚यं क्र‚म‚व‚त्त्वेन ज्ञातुः स प‚रोक्ष‚दृश्य‚स्ति ।}} ‚{\color{DodgerBlue3}‚सिद्धा । अन्येषा\edtext{}{\edlabel{pvv.402-7}\label{pvv.402-7}\lemma{अन्येषा}\Bfootnote{प्राक्त‚नानाम् ।}}न्त‚योर्ज्ञान‚प्र‚भा}‚व‚यो‚{\color{DodgerBlue3}‚र‚भाव‚तः} । (३०८)
	\pend% ending standard par
      \label{div_pvv.3.309}
	  
	% new div opening: depth here is 2
	

	  \begin{center}%% label @type='head'
	\textbf{च. आप्त‚चिन्ता}
	\end{center}
	

	  \begin{center}%% label @type='head'
	\textbf{(क) आप्त‚सिद्धिः}
	\end{center}
	

	  \pstart \leavevmode% starting standard par
	a. न‚नु त‚न्त्र‚ज्ञा र‚त्थापुरुषा अपि म‚न्त्रं किञ्चित् कार्य‚स‚म‚र्थ प्र‚ण‚य‚न्तो दृश्य‚न्ते । ‚{\tiny $_{lb}$}‚त‚तो म‚न्त्र‚क‚र‚णा\edtext{}{\edlabel{pvv.402-8}\label{pvv.402-8}\lemma{णा}\Bfootnote{एव‚म‚न्योपि स्यात् क‚थं पुरुषातिश‚य‚सिद्धिः ।}}न्नातिश‚य‚सिद्धिरित्याह ।
	\pend% ending standard par
      
	  \bigskip
	  \begingroup
	
	    \large
	  
	    \begin{quote}
	  
	    
	    \stanza[\smallbreak]
	\label{pv.3.309}\flagstanza{\tiny\textenglish{....3.309}}येपि म‚न्त्र‚विदः केचिन् म‚न्त्रान् कांश्च‚न कुर्व्व‚ते ।&प्र‚भोः प्र‚भाव‚स्तेषां स त‚दुक्त‚न्याय‚वृत्तितः ॥ ३०९ ॥\&[\smallbreak]


	
	    \end{quote}
	  
	  \endgroup
	\textsuperscript{\textenglish{403/s}}

	  \pstart \leavevmode% starting standard par
	\hphantom{.}‚{\color{DodgerBlue3}‚य‚पि} केचित् साधार‚णा ‚{\color{DodgerBlue3}‚म‚न्त्र‚विदो} म‚न्त्र‚शास्त्र‚ज्ञाः ‚{\color{DodgerBlue3}‚कांश्च‚न म‚न्त्रान्} विषादिश‚{\tiny $_{lb}$}‚म‚नान् ‚{\color{DodgerBlue3}‚कुर्व्व‚ते तेषां} प्र‚भोर्म‚न्त्र‚प्र‚णेतुर‚तिश‚यित‚श‚क्तेः ‚{\color{DodgerBlue3}‚स प्र‚भावः} साम‚र्थ्यं\edtext{}{\edlabel{pvv.403-1}\label{pvv.403-1}\lemma{र्थ्यं}\Bfootnote{प्र‚भुस्तुष्ट‚स्त‚त्प्र‚णीतान‚प्य‚धितिष्ठ‚तीति भावः ।}} ‚{\color{DodgerBlue3}‚त‚दुक्त}‚स्य ‚{\tiny $_{lb}$}‚न्याय‚स्य स‚म‚यानुष्ठानादे‚{\color{DodgerBlue3}‚र्वृत्तितः} । आराधित‚स्य प्र‚भोः प्र‚भावादीदृग‚क् श‚क्तिलाभ ‚{\tiny $_{lb}$}‚इत्य‚र्थः\edtext{}{\edlabel{pvv.403-2}\label{pvv.403-2}\lemma{र्थः}\Bfootnote{अपि च केचित् म‚न्त्र(ान्) कुर्व्व‚ते न स‚र्व्व इति व‚द‚ता पुरुषातिश‚य एव स‚म‚र्थितः स्यात् ।}}। (३०९)
	\pend% ending standard par
      \label{div_pvv.3.310}
	  
	% new div opening: depth here is 2
	

	  \pstart \leavevmode% starting standard par
	त‚स्मात् (।)
	\pend% ending standard par
      
	  \bigskip
	  \begingroup
	
	    \large
	  
	    \begin{quote}
	  
	    
	    \stanza[\smallbreak]
	\label{pv.3.310}\flagstanza{\tiny\textenglish{....3.310}}कृत‚कः पौरुषेयाश्च म‚न्त्रा वाच्याः फ‚लेप्सुना ।&अश‚क्तिसाध‚नं पुंसाम‚नेनैव निराकृत‚म् ॥ ३१० ॥\&[\smallbreak]


	
	    \end{quote}
	  
	  \endgroup
	

	  \pstart \leavevmode% starting standard par
	\hphantom{.}कृत‚काः पौरुषेयाश्च म‚न्त्रा वाच्याः । ‚{\color{DodgerBlue3}‚फ‚लेप्सुना} स‚र्व्वेण येन पुरुषाः श‚क्ति‚{\tiny $_{lb}$}‚विशेष‚व‚न्तो म‚न्त्रान् प्र‚णेतुमीश‚ते (।) \edtext{\textsuperscript{*}}{\edlabel{pvv.403-3}\label{pvv.403-3}\lemma{*}\Bfootnote{म‚न्त्र‚क‚र्त्तुर्ज्ञानातिश‚य‚साध‚नेन व‚स्तुब‚लायात‚स्य निवार‚ण‚साध‚नाभावात् ।}} ‚{\color{DodgerBlue3}‚अनेनैव} म‚न्त्रादिप्र‚ण‚य‚नं प्र‚ति ‚{\color{DodgerBlue3}‚पुंसाम‚{\tiny $_{lb}$}‚श‚क्तिसाध‚नं} य‚त् किम‚पि‚{\tiny $_{4}$}‚ मी मां स कैरिष्टं त‚{\color{DodgerBlue3}‚न्निराकृतं} बोद्ध‚व्यं । (३१०)
	\pend% ending standard par
      \label{div_pvv.3.311}
	  
	% new div opening: depth here is 2
	

	  \pstart \leavevmode% starting standard par
	एत‚देव स्फुट‚यितुमाह (।)
	\pend% ending standard par
      
	  \bigskip
	  \begingroup
	
	    \large
	  
	    \begin{quote}
	  
	    
	    \stanza[\smallbreak]
	\label{pv.3.311}\flagstanza{\tiny\textenglish{....3.311}}बुद्धीन्द्रियोक्तिपुंस्त्वादिसाध‚नं य‚त्तु व‚र्ण्य‚ते ।&प्र‚माणाभं य‚थार्थास्ति न हि शेष‚व‚तो ग‚तिः ॥ ३११ ॥\&[\smallbreak]


	
	    \end{quote}
	  
	  \endgroup
	

	  \pstart \leavevmode% starting standard par
	\hphantom{.}‚{\color{DodgerBlue3}‚\edtext{\textsuperscript{*}}{\edlabel{pvv.403-4}\label{pvv.403-4}\lemma{*}\Bfootnote{स‚त्त्वादिन्द्रिय‚त्वाद् व‚च‚नात् पुंस्त्व‚न्त‚स्या(?)पुरुष‚व‚दित्य‚त आह । आदिना प्राण्यादिम‚त्वात् ।}}बुद्धीन्द्रियोक्तिपुंस्त्वादिसाध‚नं} श‚क्तिविशेष‚निराक‚र‚णं ‚{\color{DodgerBlue3}‚य‚त्तु व‚र्ण्य‚ते} त‚त् ‚{\tiny $_{lb}$}‚‚{\color{DodgerBlue3}‚प्र‚माणाभ‚म‚नै}‚कान्तिकं (।) न हि बुद्धिम‚त एक‚स्य श‚क्तिविशेषो न दृष्ट इत्य‚न्य‚स्यापि ‚{\tiny $_{lb}$}‚त‚था । अभ्यासाधीनो हि प्र‚क‚र्षो गुणेषु त‚द‚भिलाषाद‚भ्यासोपि स‚म्भ‚वी (।) दृश्य‚ते ‚{\tiny $_{lb}$}‚च प्र‚ज्ञादिगुणानाम‚तिश‚यितः प्र‚क‚र्ष इति बुद्धि‚{\tiny $_{5}$}‚म‚त्वाद्य‚साध‚नं ।\edtext{\textsuperscript{*}}{\edlabel{pvv.403-5}\label{pvv.403-5}\lemma{*}\Bfootnote{विप‚क्ष‚वृत्तेः स‚न्दे(हे)न स‚र्व‚स्य शेष‚व‚त्वात् ।}} ‚{\color{DodgerBlue3}‚न हि शेष‚व‚तो} लिङ्गात् (।) ‚{\color{DodgerBlue3}‚य‚थार्था}‚नुमेय‚स्य ‚{\color{DodgerBlue3}‚ग‚ति}‚र्भ‚वितुम‚र्ह‚ति । (३११)
	\pend% ending standard par
      \label{div_pvv.3.312}
	  
	% new div opening: depth here is 2
	

	  \pstart \leavevmode% starting standard par
	b. अपि च (।) पुरुषातिश‚य‚म‚निच्छ‚तां जैमिनीयानां म‚ह‚त् दुःश्लिष्टं । ‚{\tiny $_{lb}$}‚त‚था हि (।)
	\pend% ending standard par
      
	  \bigskip
	  \begingroup
	
	    \large
	  
	    \begin{quote}
	  
	    
	    \stanza[\smallbreak]
	\label{pv.3.312}\flagstanza{\tiny\textenglish{....3.312}}अर्थोयं नाय‚म‚र्थो न इति श‚ब्दा व‚द‚न्ति न ।&क‚ल्प्योय‚म‚र्थः पुरुषैस्ते च रागादिसंयुताः ॥ ३१२ ॥\&[\smallbreak]


	
	    \end{quote}
	  
	  \endgroup
	\textsuperscript{\textenglish{404/s}}

	  \pstart \leavevmode% starting standard par
	\hphantom{.}‚{\color{DodgerBlue3}‚अय‚म‚र्थो}‚ऽस्य श‚ब्द‚स्यायं ‚{\color{DodgerBlue3}‚नेति} न ताव‚{\color{DodgerBlue3}‚च्छ‚ब्दाः} स्व‚यं ‚{\color{DodgerBlue3}‚व‚द‚न्ति} । त‚स्मात् ‚{\color{DodgerBlue3}‚पुरुषै‚{\tiny $_{lb}$}‚र‚य‚म‚र्थः क‚ल्पः । ते च रागादियुक्ता} नाव‚धार\edtext{}{\edlabel{pvv.404-1}\label{pvv.404-1}\lemma{धार}\Bfootnote{त‚त्क‚ल्पितो न प्र‚माणं पुरुषातिश‚याभाव‚प्र‚तिज्ञापि विरोधिता त्व‚यं ।}}ण‚प‚ट‚वः । (३१२)
	\pend% ending standard par
      \label{div_pvv.3.313}
	  
	% new div opening: depth here is 2
	
	  \bigskip
	  \begingroup
	
	    \large
	  
	    \begin{quote}
	  
	    
	    \stanza[\smallbreak]
	\label{pv.3.313}\flagstanza{\tiny\textenglish{....3.313}}स एक‚स्त‚त्त्व‚विप्न्नान्य इति भेद‚श्च किंकृतः ।&त‚द्व‚त् पुंस्त्वे क‚थ‚म‚पि ज्ञानी क‚श्चित् क‚थं न वः ॥ ३१३ ॥\&[\smallbreak]


	
	    \end{quote}
	  
	  \endgroup
	

	  \pstart \leavevmode% starting standard par
	\hphantom{.}अथ रागिष्व‚पि क‚श्चि ज्जै मि न्या दिर्जानात्येव । त‚त्रै‚{\color{DodgerBlue3}‚क‚स्त\edtext{}{\edlabel{pvv.404-2}\label{pvv.404-2}\lemma{स्त}\Bfootnote{ज्ञान‚वान‚न्यो प्र‚स‚क्तः ।}}त्त्व‚विद‚न्यो नेति ‚{\tiny $_{lb}$}‚भेद‚श्च किंकृत} एषः । रागि‚{\tiny $_{6}$}‚त्वाविशेषात् स‚र्व्व एवाज्ञः प्राप्तो न वा क‚श्चित् । अथ ‚{\tiny $_{lb}$}‚पुरुष‚त्व‚साम्येपि जैमिन्यादिर‚तीन्द्रियार्थ‚द‚र्शित्वाद् वैदिक‚श‚ब्दानाम‚तीन्द्रियार्थ‚स‚म्ब‚न्ध‚{\tiny $_{lb}$}‚वेत्ता क‚ल्प्य‚ते (।) त‚दा ‚{\color{DodgerBlue3}‚पुंस्त्वे} स‚त्य‚पि क‚श्चिद‚न्योपि ‚{\color{DodgerBlue3}‚ज्ञानी} ज्ञानातिश‚य‚वान् ‚{\color{DodgerBlue3}‚क‚थ‚म‚प्य-} भ्यासादिना ‚{\color{DodgerBlue3}‚त‚द्व‚ज्जैमिन्या}‚दिव‚त् । ‚{\color{DodgerBlue3}‚क‚थ‚म्वो\edtext{}{\edlabel{pvv.404-3}\label{pvv.404-3}\lemma{म्वो}\Bfootnote{म‚न्त्र‚क‚र्त्तुर्ज्ञानातिश‚य‚साध‚नेन व‚स्तुब‚लाग‚त‚स्य निवार‚णे को दोषः (?) ।}}} मीमांस‚कानां नाभिम‚तः । (३१३)
	\pend% ending standard par
      \label{div_pvv.3.314}
	  
	% new div opening: depth here is 2
	

	  \begin{center}%% label @type='head'
	\textbf{(ख) आप्त‚ल‚क्ष‚ण‚म्}
	\end{center}
	

	  \pstart \leavevmode% starting standard par
	स्यादेत‚न्न व‚यं पुरुष‚प्र‚माण्याद् व्याख्यान‚म‚नुम‚न्याम‚हे । किन्तु (।)
	\pend% ending standard par
      
	  \bigskip
	  \begingroup
	
	    \large
	  
	    \begin{quote}
	  
	    
	    \stanza[\smallbreak]
	\label{pv.3.314}\flagstanza{\tiny\textenglish{....3.314}}प्र‚माण‚म‚विसंवादि व‚च‚नं सोर्थ‚विद् य‚दि ।&न ह्य‚त्य‚न्त‚प‚रोक्षेषु प्र‚माण‚स्यास्ति स‚म्भ‚वः ॥ ३१४ ॥\&[\smallbreak]


	
	    \end{quote}
	  
	  \endgroup
	

	  \pstart \leavevmode% starting standard par
	\hphantom{.}‚{\color{DodgerBlue3}‚य‚स्य प्र‚मा‚{\tiny $_{7}$}‚ण‚म‚विस‚म्वादि व‚च‚नं सोर्थ‚वित्} वेदार्थ‚ज्ञाता ‚{\color{DodgerBlue3}‚य‚दी}‚ष्य‚ते । न‚न्व‚{\color{DodgerBlue3}‚त्य‚न्त}‚{\tiny $_{lb}$}‚\leavevmode\ledsidenote{\textenglish{80a/MA}} ‚{\color{DodgerBlue3}‚प‚रोक्षेषु} वेदार्थेषु ‚{\color{DodgerBlue3}‚प्र‚माण‚स्य स‚म्भ‚वो न हि} क‚स्य‚चि‚{\color{DodgerBlue3}‚द‚स्ति} त‚त्क‚थं स‚म्वादाद‚र्थ‚विद् ‚{\tiny $_{lb}$}‚व्य‚व‚स्थाप्य‚ते । (३१४)
	\pend% ending standard par
      \label{div_pvv.3.315}
	  
	% new div opening: depth here is 2
	

	  \pstart \leavevmode% starting standard par
	अपि च\edtext{}{\edlabel{pvv.404-4}\label{pvv.404-4}\lemma{च}\Bfootnote{ब‚हुषु व्याख्यातृषु यः प्र‚माणं प्र‚त्य‚क्षादि संस्य‚न्द‚य‚ति त‚द्भाषितं गृह्य‚त इति ब्रुव‚ताऽपौरुषेय‚त्वादाग‚म‚ल‚क्ष‚णाद‚न्य‚देव‚माग‚म‚ल‚क्ष‚णं स्यादित्याह ।}} (।)
	\pend% ending standard par
      
	  \bigskip
	  \begingroup
	
	    \large
	  
	    \begin{quote}
	  
	    
	    \stanza[\smallbreak]
	\label{pv.3.315}\flagstanza{\tiny\textenglish{....3.315}}य‚स्य प्र‚माण‚संवादि व‚च‚नं त‚त्कृतं व‚चः ।&स आग‚म इति प्राप्तं निर‚र्थाऽपौरुषेय‚ता ॥ ३१५ ॥\&[\smallbreak]


	
	    \end{quote}
	  
	  \endgroup
	

	  \pstart \leavevmode% starting standard par
	\hphantom{.}‚{\color{DodgerBlue3}‚य‚स्य प्र‚माण‚स‚म्वादि व‚च‚नं} स य‚दि व्याख्याता त‚दा स‚म्वाद आग‚म‚ल‚क्ष‚ण‚मिति ‚{\tiny $_{lb}$}‚‚{\color{DodgerBlue3}‚तेन संस्कृतं\edtext{}{\edlabel{pvv.404-5}\label{pvv.404-5}\lemma{संस्कृतं}\Bfootnote{प्र‚माणानुगृहीत‚त्वे ख्याप‚नं संस्कृत‚त्वं ।}} व‚च आग‚म इति\edtext{}{\edlabel{pvv.404-6}\label{pvv.404-6}\lemma{इति}\Bfootnote{न प्र‚काश‚य‚ति जैमिनिश‚व‚र‚स्वाम्यादिवैय‚र्थ्यास‚ङ्गात् ।}} प्राप्त}‚मिति ‚{\color{DodgerBlue3}‚निर‚र्थाऽपौरुषेय‚ता} वेदानां क‚ल्पिता । ‚{\tiny $_{lb}$}‚(३१५)
	\pend% ending standard par
      \label{div_pvv.3.316}
	  
	% new div opening: depth here is 2
	\textsuperscript{\textenglish{405/s}}

	  \begin{center}%% label @type='head'
	\textbf{(ग) प‚र‚प‚क्षे दोषाः}
	\end{center}
	

	  \pstart \leavevmode% starting standard par
	a. त‚था (।)
	\pend% ending standard par
      
	  \bigskip
	  \begingroup
	
	    \large
	  
	    \begin{quote}
	  
	    
	    \stanza[\smallbreak]
	\label{pv.3.316}\flagstanza{\tiny\textenglish{....3.316}}य‚द्य‚त्य‚न्त‚प‚रोक्षेर्थेऽनाग‚म‚ज्ञान‚स‚म्भ‚वः ।&अतीन्द्रियार्थ‚वित् क‚श्चिद‚स्तीत्य‚भिम‚तं भ‚वेत् ॥ ३१६ ॥\&[\smallbreak]


	
	    \end{quote}
	  
	  \endgroup
	

	  \pstart \leavevmode% starting standard par
	\hphantom{.}‚{\color{DodgerBlue3}‚य‚द्य‚त्य‚न्त‚प‚रोक्षेर्थे} स्व‚र्ग‚स‚म्ब‚न्धादौ ‚{\color{DodgerBlue3}‚जै मि न्या देर‚नाग‚म}‚स्याग‚म‚निर‚पे‚{\tiny $_{1}$}‚क्ष\edtext{}{\edlabel{pvv.405-1}\label{pvv.405-1}\lemma{क्ष}\Bfootnote{प‚रोक्षे ।}}स्य ‚{\tiny $_{lb}$}‚‚{\color{DodgerBlue3}‚ज्ञान‚स्य स‚म्भ‚वः} । त‚दाऽ‚{\color{DodgerBlue3}‚तीन्द्रियार्थ}‚द‚र्शी\edtext{}{\edlabel{pvv.405-2}\label{pvv.405-2}\lemma{र्शी}\Bfootnote{अनुमान‚स्याध्य‚क्ष‚पूर्व‚त्वात् ।}} ‚{\color{DodgerBlue3}‚क‚श्चिद‚स्तीत्य‚भिम‚तं भ‚वेत्} । त‚त‚स्त‚त्प्र‚ति‚{\tiny $_{lb}$}‚क्षेपो न युक्तः । (३१६)
	\pend% ending standard par
      \label{div_pvv.3.317}
	  
	% new div opening: depth here is 2
	

	  \pstart \leavevmode% starting standard par
	य‚दि तु न क‚श्चिद‚तीन्द्रियार्थ‚द‚र्शी त‚दा (।)
	\pend% ending standard par
      
	  \bigskip
	  \begingroup
	
	    \large
	  
	    \begin{quote}
	  
	    
	    \stanza[\smallbreak]
	\label{pv.3.317}\flagstanza{\tiny\textenglish{....3.317}}स्व‚यं रागादिमान्नार्थं वेत्ति वेद‚स्य नान्य‚तः ।&न वेद‚य‚ति वेदोपि वेदार्थ‚स्य कुतो ग‚तिः ॥ ३१७ ॥\&[\smallbreak]


	
	    \end{quote}
	  
	  \endgroup
	

	  \pstart \leavevmode% starting standard par
	\hphantom{.}‚{\color{DodgerBlue3}‚स्व‚यं रागादिमान्} पुमान् ‚{\color{DodgerBlue3}‚वेद‚स्यार्थं न वेत्ति न चान्य‚तः\edtext{}{\edlabel{pvv.405-3}\label{pvv.405-3}\lemma{तः}\Bfootnote{न ह्य‚न्धेनाकृष्य‚माणोन्धोव‚ग‚च्छ‚ति व‚र्त्म ।}}} अन्य‚स्यापि रागा‚{\tiny $_{lb}$}‚दिम‚त्वेऽज्ञान‚त्वात् । ‚{\color{DodgerBlue3}‚वेदोपि} स्वार्थ ‚{\color{DodgerBlue3}‚न \edtext{}{\edlabel{pvv.405-4}\label{pvv.405-4}\lemma{न}\Bfootnote{अज्ञातार्थ‚त्वेन वेद‚स्य ।}} वेद‚य‚ति} त‚त‚श्च ‚{\color{DodgerBlue3}‚कुतो वेदार्थ‚स्य ग‚तिः} । न ‚{\tiny $_{lb}$}‚कुत‚श्चिदिति पुरुषा एव य‚थाप्र‚तिभं क‚ल्प‚येयुः । (३१७)
	\pend% ending standard par
      \label{div_pvv.3.318}
	  
	% new div opening: depth here is 2
	
	  \bigskip
	  \begingroup
	
	    \large
	  
	    \begin{quote}
	  
	    
	    \stanza[\smallbreak]
	\label{pv.3.318}\flagstanza{\tiny\textenglish{....3.318}}तेनाग्निहोत्रं जुहुयात् स्व‚र्ग‚काम इति श्रुतौ ।&खादेच्छ‚व‚मांस‚मित्येष नार्थ इत्य‚त्र का प्र‚मा ॥ ३१८ ॥\&[\smallbreak]


	
	    \end{quote}
	  
	  \endgroup
	

	  \pstart \leavevmode% starting standard par
	\hphantom{.}‚{\color{DodgerBlue3}‚तेनाग्निहोत्रं जुहु‚{\tiny $_{2}$}‚यात् स्व‚र्ग‚काम इति श्रुतौ\edtext{}{\edlabel{pvv.405-5}\label{pvv.405-5}\lemma{श्रुतौ}\Bfootnote{क्व‚चिद‚प्य‚र्थे प्र‚त्यास‚त्तिविप्र‚क‚र्ष‚र‚हितायां ।}}} वेद‚वाक्ये ‚{\color{DodgerBlue3}‚श्व‚मांसं स्वादेदि‚{\tiny $_{lb}$}‚त्येष नार्थ इत्य‚त्र का प्र‚मा} । इत्य‚पि क‚ल्प‚यितुं श‚क्य‚त्वात्\edtext{}{\edlabel{pvv.405-6}\label{pvv.405-6}\lemma{त्वात्}\Bfootnote{अनुमान‚स्याप्य‚ध्य‚क्ष‚पूर्व्व‚त्वात् ।}}। (३१८)
	\pend% ending standard par
      \label{div_pvv.3.319}
	  
	% new div opening: depth here is 2
	 {\ b.}
	  \bigskip
	  \begingroup
	
	    \large
	  
	    \begin{quote}
	  
	    
	    \stanza[\smallbreak]
	\label{pv.3.319}\flagstanza{\tiny\textenglish{....3.319}}प्र‚सिद्धो लोक‚वाद‚श्चेत्त‚त्र कोतीन्द्रियार्थ‚दृक् ।&अनेकार्थेषु श‚ब्देषु येनार्थोयं विवेचितः ॥ ३१९ ॥\&[\smallbreak]


	
	    \end{quote}
	  
	  \endgroup
	

	  \pstart \leavevmode% starting standard par
	\hphantom{.}अग्निर्दाहादिस‚म‚र्थः त‚स्मिन् घृतादिप्र‚क्षेप‚श्च ह‚व‚न‚मिति ‚{\color{DodgerBlue3}‚प्र‚सिद्धो लोक‚वादः} त‚तो नान्यार्थ‚क‚ल्प‚नेति चेत् । ‚{\color{DodgerBlue3}‚त‚त्र} लोके रागादिम‚ति ‚{\color{DodgerBlue3}‚कोऽतीन्द्रियार्थ‚दृग‚स्ति ‚{\tiny $_{lb}$}‚येनानेकार्थे}‚ष्व‚नेकार्थाभिधान‚योग्येषु ‚{\color{DodgerBlue3}‚श‚ब्देष्व}‚य‚म‚भिम‚तो‚{\color{DodgerBlue3}‚\edtext{\textsuperscript{*}}{\edlabel{pvv.405-7}\label{pvv.405-7}\lemma{*}\Bfootnote{अय‚म‚र्थो वेद‚स्य नान्य इति विभ‚क्तो ।}}र्थो विवेचितो} वाच्य‚त‚या ‚{\tiny $_{3}$}‚ ‚{\tiny $_{lb}$}‚विशेष‚ण‚व्य‚व‚स्थापितः । य‚दि दाह‚क‚द्र‚व्ये घृत‚प्र‚क्षेप‚स्य स्व‚र्गेण स‚म्ब‚न्धः श्व‚मांस‚भ‚क्ष‚णेन ‚{\tiny $_{lb}$}‚\leavevmode\ledsidenote{\textenglish{406/s}} च नास्तीति केनापि दृष्टं स्यात् स्याद‚य‚म‚र्थ‚विवेकः । त‚द्द‚र्शी तु रागादिम‚त्वात् ‚{\tiny $_{lb}$}‚क‚श्चिन्नेष्ट इति नास्ति विप‚रीत‚क‚ल्प‚नानिरासः । (३१९)
	\pend% ending standard par
      \label{div_pvv.3.320}
	  
	% new div opening: depth here is 2
	

	  \pstart \leavevmode% starting standard par
	न च लोक‚प्र‚सिद्धानुसाराद् वेदार्थ‚व्य‚व‚स्था । त‚था हि (।)
	\pend% ending standard par
      
	  \bigskip
	  \begingroup
	
	    \large
	  
	    \begin{quote}
	  
	    
	    \stanza[\smallbreak]
	\label{pv.3.320}\flagstanza{\tiny\textenglish{....3.320}}स्व‚र्गोर्व‚श्यादिश‚ब्द‚श्च दृष्टोरूढार्थ‚वाच‚कः ।&श‚ब्दान्त‚रेषु तादृक्षु तादृश्येवास्तु क‚ल्प‚ना ॥ ३२० ॥\&[\smallbreak]


	
	    \end{quote}
	  
	  \endgroup
	

	  \pstart \leavevmode% starting standard par
	\hphantom{.}‚{\color{DodgerBlue3}‚स्व‚र्गोर्व‚श्यादिश‚ब्द‚श्चारूढार्थ‚वाच‚को\edtext{}{\edlabel{pvv.406-1}\label{pvv.406-1}\lemma{को}\Bfootnote{वेद‚वादिना कृतो ।}} दृष्टः} । म‚नुष्यातिशायिपुरुष‚विशेष‚{\tiny $_{lb}$}‚निकेतो‚{\tiny $_{4}$}‚ऽतिमानुष‚सुखाधिष्ठानो नानोप‚कार‚णः स्व‚र्गः । त‚त्स्थाऽप्स‚रा उ र्व्व शी ति ‚{\tiny $_{lb}$}‚प्र‚सिद्धो लोक‚वादः\edtext{}{\edlabel{pvv.406-2}\label{pvv.406-2}\lemma{वादः}\Bfootnote{पौरुषेये नायं दोषः संप्र‚दाय‚स‚म्भ‚वात् । लोक‚संकेत‚प्र‚सिद्ध‚श‚ब्दानुविधानात् । श‚ब्दानाम‚नेकार्थ‚त्वेपि । पुरुषोप‚योगिन‚मेवाग‚मार्थं च‚तुःस‚त्यं निश्चिन्व‚न्ति सौग‚ता न प्र‚वाद‚मात्रं वृद्ध‚स्येत्य‚तोप्य‚दोषः ॥ एत‚त् त्र‚यं वेदेनेति त्याज्य‚न्त‚दित्याह ।}} । त‚मुल्लंध्य दुःखेनास‚म्भिन््ना निर‚तिश‚या प्रीतिः स्व‚र्गः । ‚{\tiny $_{lb}$}‚उर्व्व‚शी चार‚णिः पात्री वेत्य‚प्र‚सिद्धार्थ‚क‚ल्प‚नेति । ‚{\color{DodgerBlue3}‚श‚ब्दान्त‚रे\edtext{}{\edlabel{pvv.406-3}\label{pvv.406-3}\lemma{रे}\Bfootnote{प्र‚देशान्त‚रे व‚न्हौ घृत‚ह‚व‚न‚मुक्तं त‚तो निश्च‚य इत्य‚पि न त‚स्याप्य‚सिद्धार्थ‚त्वात् । ‚{\tiny $_{lb}$}‚श्व‚मांस‚क‚ल्प‚नैवास्तु ।}}}‚ष्व‚ग्निहोत्रादिषु ‚{\color{DodgerBlue3}‚तादृक्षु} स‚र्व्वार्थ‚योग्येषु ‚{\color{DodgerBlue3}‚तादृश्येवा}‚प्र‚सिद्धार्थैव ‚{\color{DodgerBlue3}‚क‚ल्प‚नास्तु} न्याय‚स्य तुल्य‚त्वात् । (३२०)
	\pend% ending standard par
      \label{div_pvv.3.321}
	  
	% new div opening: depth here is 2
	

	  \pstart \leavevmode% starting standard par
	किञ्च (।)
	\pend% ending standard par
      
	  \bigskip
	  \begingroup
	
	    \large
	  
	    \begin{quote}
	  
	    
	    \stanza[\smallbreak]
	\label{pv.3.321}\flagstanza{\tiny\textenglish{....3.321}}प्र‚सिद्ध‚श्च नृणां वादः प्र‚माणं स च नेष्य‚ते ।&त‚त‚श्च भूयोर्थ‚ग‚तिः किमेत‚द् द्विष्ठ‚कामित‚म् ॥ ३२१ ॥\&[\smallbreak]


	
	    \end{quote}
	  
	  \endgroup
	

	  \pstart \leavevmode% starting standard par
	\hphantom{.}‚{\color{DodgerBlue3}‚प्र‚सिद्ध‚श्च नृणाम्वाद} एव न त्व‚न्य‚था काचित् । ‚{\color{DodgerBlue3}‚स ‚{\tiny $_{5}$}‚ च} पुरुष‚प्र‚व‚र्त्तित‚त्वात् ‚{\tiny $_{lb}$}‚‚{\color{DodgerBlue3}‚प्र‚माणं नेष्य‚ते} भ‚व‚ता । ‚{\color{DodgerBlue3}‚भूयः} पुन‚{\color{DodgerBlue3}‚स्त‚तो} ज‚न‚वादा‚{\color{DodgerBlue3}‚द‚र्थ}‚स्य ‚{\color{DodgerBlue3}‚ग‚ति}‚रिति ‚{\color{DodgerBlue3}‚किमेत‚त्} द्विष्ठ‚{\tiny $_{lb}$}‚कामितं । लोक‚वादोऽप्र‚माण‚त्वात् द्विष्ठः संप्र‚ति काम्य‚ते इति व्य‚क्तो विरोधः । (३२१)
	\pend% ending standard par
      \label{div_pvv.3.322}
	  
	% new div opening: depth here is 2
	
	  \bigskip
	  \begingroup
	
	    \large
	  
	    \begin{quote}
	  
	    
	    \stanza[\smallbreak]
	\label{pv.3.322}\flagstanza{\tiny\textenglish{....3.322}}अथ प्र‚सिद्धिमुल्लंध्य क‚ल्प‚ने न निब‚न्ध‚न‚म् ।&प्र‚सिद्धेर‚प्र‚माण‚त्वात् त‚द्ग्र‚हे किं निब‚न्ध‚न‚म् ॥ ३२२ ॥\&[\smallbreak]


	
	    \end{quote}
	  
	  \endgroup
	

	  \pstart \leavevmode% starting standard par
	\hphantom{.}‚{\color{DodgerBlue3}‚अथ} लोक‚स्य ‚{\color{DodgerBlue3}‚प्र‚सिद्धिमुल्लंध्य} श्व‚मांस‚भ‚क्ष‚णाद्य‚र्थ‚{\color{DodgerBlue3}‚क‚ल्प‚ने न निब‚न्ध‚न‚म‚स्ति । ‚{\tiny $_{lb}$}‚न‚नु प्र‚सिद्धेर‚प्र‚माण‚त्वात्} त‚स्या ‚{\color{DodgerBlue3}‚ग्र‚हे\edtext{}{\edlabel{pvv.406-4}\label{pvv.406-4}\lemma{हे}\Bfootnote{प्र‚सिद्धिग्र‚होपि मा भूत् ।}} किं निब‚न्ध‚नं} विचार‚क‚स्य न किञ्चित् । (३२२)
	\pend% ending standard par
      \label{div_pvv.3.323}
	  
	% new div opening: depth here is 2
	

	  \pstart \leavevmode% starting standard par
	किञ्च (।)
	\pend% ending standard par
      
	  \bigskip
	  \begingroup
	
	    \large
	  
	    \begin{quote}
	  
	    
	    \stanza[\smallbreak]
	\label{pv.3.323}\flagstanza{\tiny\textenglish{....3.323}}उत्पादिता प्र‚सिद्ध्यैव श‚ङ्का श‚ब्दार्थ‚निश्च‚ये ।&य‚स्मान्नानार्थ‚वृत्तित्वं श‚ब्दानां त‚त्र दृश्य‚ते ॥ ३२३ ॥\&[\smallbreak]


	
	    \end{quote}
	  
	  \endgroup
	\textsuperscript{\textenglish{407/s}}

	  \pstart \leavevmode% starting standard par
	प्र‚सिद्ध्यैव श‚ब्दार्थ‚निश्च‚ये विप‚रीतार्थ‚क‚ल्प‚नाया‚{\tiny $_{6}$}‚ श‚ङ्कोत्पादिता । ‚{\tiny $_{lb}$}‚य‚स्मान्नानार्थ‚वृत्तित्वं श‚ब्दानां ग‚वाक्ष‚प्र‚भृतीनां त‚त्र प्र‚सिद्धौ दृश्य‚ते (। ३२३)
	\pend% ending standard par
      \label{div_pvv.3.324}
	  
	% new div opening: depth here is 2
	

	  \pstart \leavevmode% starting standard par
	c. त‚था (।)
	\pend% ending standard par
      
	  \bigskip
	  \begingroup
	
	    \large
	  
	    \begin{quote}
	  
	    
	    \stanza[\smallbreak]
	\label{pv.3.324}\flagstanza{\tiny\textenglish{....3.324}}अन्य‚थास‚म्भ‚वाभावात् नानाश‚क्तेः स्व‚यं ध्व‚नेः ।&अव‚श्यं श‚ङ्क‚या भाव्यं नियाम‚क‚म‚प‚श्य‚ताम् ॥ ३२४ ॥\&[\smallbreak]


	
	    \end{quote}
	  
	  \endgroup
	

	  \pstart \leavevmode% starting standard par
	\hphantom{.}‚{\color{DodgerBlue3}‚नानाश‚क्ते}‚र‚नेकार्थ‚प्र‚तिपाद‚न‚योग्य‚स्य ‚{\color{DodgerBlue3}‚ध्व‚नेः स्व‚य}‚मात्म‚नाऽन्य‚थास‚म्भ‚व‚स्य ‚{\tiny $_{lb}$}‚एकार्थ‚प्र‚तिपाद‚न\edtext{}{\edlabel{pvv.407-1}\label{pvv.407-1}\lemma{न}\Bfootnote{साध्यार्थाद‚न्य‚त्र वृत्तिर‚न्य‚था । त‚द‚स‚म्भ‚वो य‚स्त‚स्याभावादिति द्वौ निषेधौ स‚म्भ‚वोन्य‚थापि ।}} योग्य‚तास‚म्भ‚व‚स्यास‚म्भ‚वात् । ‚{\color{DodgerBlue3}‚अव‚श्य}‚म‚र्थान्त‚र‚{\color{DodgerBlue3}‚श‚ङ्क‚या भाव्यं} । ‚{\tiny $_{lb}$}‚क‚थ‚मित्याह । ‚{\color{DodgerBlue3}‚नियाम‚क‚म‚प‚श्य‚ताम}‚नेकार्थ‚स्य श‚ब्द‚स्य एक‚वृत्तिनिय‚म‚कार‚ण‚{\tiny $_{lb}$}‚म‚प‚श्य‚तां पुंसां ।\edtext{\textsuperscript{*}}{\edlabel{pvv.407-2}\label{pvv.407-2}\lemma{*}\Bfootnote{य‚त एव त‚स्माद‚ज्ञातार्थ‚श‚ब्देष्व‚प्र‚माण‚कार्यारोप‚न्निश्चित्य व्याच‚क्षाणो जैमिनिस्त‚द्व्याजेन स्व‚म‚त‚मेवाहेत्य‚र्थं द‚र्श‚य‚न्नाह ।}}(३२४)
	\pend% ending standard par
      \label{div_pvv.3.325}
	  
	% new div opening: depth here is 2
	

	  \pstart \leavevmode% starting standard par
	d. किञ्च (।)
	\pend% ending standard par
      
	  \bigskip
	  \begingroup
	
	    \large
	  
	    \begin{quote}
	  
	    
	    \stanza[\smallbreak]
	\label{pv.3.325}\flagstanza{\tiny\textenglish{....3.325}}एष स्थाणुर‚यं मार्ग इति व‚क्तीति क‚श्च‚न ।&अन्यः स्व‚यं ब्र‚वीमीति त‚योर्भेदः प‚रीक्ष्य‚ताम् ॥ ३२५ ॥\&[\smallbreak]


	
	    \end{quote}
	  
	  \endgroup
	

	  \pstart \leavevmode% starting standard par
	स्व‚य‚म‚र्थ‚प्र‚तिपाद‚{\tiny $_{7}$}‚न‚शून्यो नेता\edtext{}{\edlabel{pvv.407-3}\label{pvv.407-3}\lemma{नेता}\Bfootnote{वैदिक‚श‚ब्दान्...वैक‚त्वे निय‚मः ।}}न‚भिम‚तार्थ‚द्योत‚क‚त्वेन प्र‚तिपाद‚य‚न् क‚र्त्तुर्न्न\leavevmode\ledsidenote{\textenglish{80b/MA}} ‚{\tiny $_{lb}$}‚भिद्य‚ते । त‚स्याप्येवं व्यापार‚त्वात् । य‚था केन‚चित् ‚{\color{DodgerBlue3}‚क‚श्च‚न\edtext{}{\edlabel{pvv.407-4}\label{pvv.407-4}\lemma{न}\Bfootnote{पाट‚लिपुत्र‚स्य ।}}} प‚न्थानं पृष्ठ आह । ‚{\tiny $_{lb}$}‚नाहं जाने किन्तु ‚{\color{DodgerBlue3}‚स्थाणुरेष\edtext{}{\edlabel{pvv.407-5}\label{pvv.407-5}\lemma{स्थाणुरेष}\Bfootnote{मार्गोप‚देश‚साम‚र्थ्य‚शून्य‚स्थाणुव्याजेन मार्ग‚माह ।}} व‚क्ति (।) अयं मार्ग इति । अन्यः} पुनः प्र‚त्याह (।) ‚{\tiny $_{lb}$}‚अहं ‚{\color{DodgerBlue3}‚स्व‚यं ब्र‚वीमि} (।) ‚{\color{DodgerBlue3}‚अयं} मार्ग ‚{\color{DodgerBlue3}‚इति} (।) ‚{\color{DodgerBlue3}‚त‚योरेवं} वादिनोः प्र‚तिपाद‚क‚त्व‚स्य ‚{\color{DodgerBlue3}‚भेदः ‚{\tiny $_{lb}$}‚प‚रीक्ष्य‚तां} (।) एको निर‚भिप्राय\edtext{}{\edlabel{pvv.407-6}\label{pvv.407-6}\lemma{भिप्राय}\Bfootnote{स्थाणोः ।}}व‚च‚नं व‚क्तृत्वेन व्य‚प‚दिश‚ति ।‚{\tiny $_{1}$}‚ अन्य‚स्तु नेति ‚{\tiny $_{lb}$}‚व‚च‚न‚वैद‚ग्ध्य\edtext{}{\edlabel{pvv.407-7}\label{pvv.407-7}\lemma{ग्ध्य}\Bfootnote{ज‚ड‚स्य प्र‚तिप‚त्तिमान्द्याद‚न्य‚त्र विशेषो नान‚योः ।}}मेवान‚योर्भिद्य‚ते नोप‚देश‚प्र‚वृत्तिनिवृत्ती । त‚था स्व‚य‚म‚र्थ प्र‚तिपाद‚य‚तोपि ‚{\tiny $_{lb}$}‚वेदार्थ‚विशेषाभिधायिनोऽभिद‚ध‚त् क‚र्त्तंव वाच‚क‚तायाः । (३२५)
	\pend% ending standard par
      \label{div_pvv.3.326}
	  
	% new div opening: depth here is 2
	

	  \pstart \leavevmode% starting standard par
	e. अथ त‚द‚र्थ‚प्र‚तिपाद‚ने योग्या एव श‚ब्दाः । न‚न्वेवं (।)
	\pend% ending standard par
      
	  \bigskip
	  \begingroup
	
	    \large
	  
	    \begin{quote}
	  
	    
	    \stanza[\smallbreak]
	\label{pv.3.326}\flagstanza{\tiny\textenglish{....3.326}}स‚र्व्व‚त्र योग्य‚स्यैकार्थ‚द्योत‚ने निय‚मः कुतः ।&ज्ञाता वातीन्द्रियाः केन विव‚क्षाव‚च‚नादृते ॥ ३२६ ॥\&[\smallbreak]


	
	    \end{quote}
	  
	  \endgroup
	\textsuperscript{\textenglish{408/s}}

	  \pstart \leavevmode% starting standard par
	\hphantom{.}‚{\color{DodgerBlue3}‚स‚र्व्व‚त्रार्थे योग्य‚स्य} श‚ब्द‚स्य ‚{\color{DodgerBlue3}‚एकार्थ‚द्योतेने} निय‚मः ‚{\color{DodgerBlue3}‚कुतो} जातः (।) पुरुष‚{\tiny $_{lb}$}‚श्चेन्नियाम‚को न भ‚व‚ति । स्यादेत‚त् (।) प्र‚तिनिय‚ता एवार्थास्तेषां न ते पुरुषेण ‚{\tiny $_{lb}$}‚निय‚म्य‚न्ते ।‚{\tiny $_{2}$}‚ एव‚म‚पि \edtext{}{\edlabel{pvv.408-1}\label{pvv.408-1}\lemma{पि}\Bfootnote{त‚थापि तं ज्ञातुम‚श‚क्तः पुरुषः ।}} ‚{\color{DodgerBlue3}‚ज्ञाता वा अतीन्द्रियाः} प्र‚तिनिय‚ता अर्थाः\edtext{}{\edlabel{pvv.408-2}\label{pvv.408-2}\lemma{अर्थाः}\Bfootnote{पुरुषातिश‚यानिष्टेः ।}} ‚{\color{DodgerBlue3}‚केन} पुरुषेण ‚{\tiny $_{lb}$}‚रागादिम‚ता ‚{\color{DodgerBlue3}‚विव‚क्षा}‚या(ः) प्र‚काश‚कात् ‚{\color{DodgerBlue3}‚व‚च‚नात्} प्र‚तिपाद‚ना‚{\color{DodgerBlue3}‚दृते (।) न} हि वेदेषु ‚{\tiny $_{lb}$}‚क‚स्य‚चिद् विव‚क्षास्ति (।) त‚द‚व‚ग‚तिम‚न्त‚रेण च प्र‚तिनिय‚तार्थ‚ता न श‚क्या बोद्धुं । ‚{\tiny $_{lb}$}‚न हि श‚ब्दाः स्व‚यं स्वार्थ‚निय‚मं क‚थ‚य‚न्ति संकेत‚म‚न्त‚रेण(।) स च पौरुषेयः । न ‚{\tiny $_{lb}$}‚चार्थ‚निय‚माव‚ग‚मं विना संकेत‚क‚र‚णं अर्थ‚निय‚म‚प्र‚तीतिश्च ‚{\tiny $_{3}$}‚ विना संकेतं नास्तीति ‚{\tiny $_{lb}$}‚‚{\color{DodgerBlue3}‚व्य‚क्त}‚मित‚रेत‚राश्र‚य‚त्वं । (३२६)
	\pend% ending standard par
      \label{div_pvv.3.327}
	  
	% new div opening: depth here is 2
	

	  \pstart \leavevmode% starting standard par
	f. अपि च (।)
	\pend% ending standard par
      
	  \bigskip
	  \begingroup
	
	    \large
	  
	    \begin{quote}
	  
	    
	    \stanza[\smallbreak]
	\label{pv.3.327}\flagstanza{\tiny\textenglish{....3.327}}विव‚क्षानिय‚मे हेतुः स‚ङ्केत‚स्त‚त्प्र‚काश‚नः ।&अपौरुषेये सा नास्ति त‚स्य सैकार्थ‚ता कुतः ॥ ३२७ ॥\&[\smallbreak]


	
	    \end{quote}
	  
	  \endgroup
	

	  \pstart \leavevmode% starting standard par
	\hphantom{.}‚{\color{DodgerBlue3}‚विव‚क्षार्थ}‚स्य ‚{\color{DodgerBlue3}‚निय‚मे हेतुः संकेतः त‚त्प्र‚काश‚नो}‚ऽर्थ‚निय‚म‚बोध‚नः । ‚{\color{DodgerBlue3}‚अपौरुषेये च} वेदे ‚{\color{DodgerBlue3}‚सा} विव‚क्षा नास्ति पुरुष‚प्र‚योज्य‚त्वात् त‚स्याः । त‚त‚स्त‚त्प्र‚साध्या सा ‚{\color{DodgerBlue3}‚एकार्थ‚ता} वेद‚स्य ‚{\color{DodgerBlue3}‚कुतः} । (२२७)
	\pend% ending standard par
      \label{div_pvv.3.328}
	  
	% new div opening: depth here is 2
	

	  \pstart \leavevmode% starting standard par
	अथ (।)
	\pend% ending standard par
      
	  \bigskip
	  \begingroup
	
	    \large
	  
	    \begin{quote}
	  
	    
	    \stanza[\smallbreak]
	\label{pv.3.328}\flagstanza{\tiny\textenglish{....3.328}}स्व‚भाव‚निय‚मेन्य‚त्र न योज्येत त‚या पुनः ।&स‚ङ्केत‚श्च निर‚र्थः स्याद् व्य‚क्तौ च निय‚मः कुतः ॥ ३२८ ॥\&[\smallbreak]


	
	    \end{quote}
	  
	  \endgroup
	

	  \pstart \leavevmode% starting standard par
	\hphantom{.}‚{\color{DodgerBlue3}‚स्व‚भाव‚तः} श‚ब्दाः क्व‚चिद‚र्थे निय‚ता\edtext{}{\edlabel{pvv.408-3}\label{pvv.408-3}\lemma{ता}\Bfootnote{न विव‚क्षातः ।}}स्त‚दा स्व‚भाव‚निय‚मे\edtext{}{\edlabel{pvv.408-4}\label{pvv.408-4}\lemma{मे}\Bfootnote{स श‚ब्दो य‚थेष्टं स‚म्वादितं तुल्य‚र‚स‚मिति ।}} ‚{\color{DodgerBlue3}‚ऽन्य‚त्रार्थे} वाच‚क‚त्वेन ‚{\tiny $_{lb}$}‚‚{\color{DodgerBlue3}‚त‚या} विव‚क्ष‚या ‚{\color{DodgerBlue3}‚न योज्येत‚{\tiny $_{4}$}‚ पुनः} । न हि स्व‚भावेन प्र‚तिनिय‚ताविष‚योन्य‚था क‚र्तुं ‚{\tiny $_{lb}$}‚श‚क्य‚ते\edtext{}{\edlabel{pvv.408-5}\label{pvv.408-5}\lemma{ते}\Bfootnote{च‚क्षुरिव रूपे ।}} । ‚{\color{DodgerBlue3}‚संकेत‚श्च निर‚र्थः स्यात्} स्व‚भाव‚प्र‚तिनिय‚मे स‚ति न हि च‚क्षुरिन्द्रियं ‚{\tiny $_{lb}$}‚रूप‚ग्र‚ह‚ण‚प्र‚तिनिय‚तं त‚त्र स‚ङ्के\edtext{}{\edlabel{pvv.408-6}\label{pvv.408-6}\lemma{ङ्के}\Bfootnote{संकेतापेक्ष‚प्र‚तीत‚य‚स्तु स्व‚प्र‚तीत्य‚र्थ‚संकेतित‚राज‚चिह्न‚व‚त् ।}}त‚म‚पेक्ष‚ते ।\edtext{\textsuperscript{*}}{\edlabel{pvv.408-7}\label{pvv.408-7}\lemma{*}\Bfootnote{वैदिकार्थो निस‚र्ग‚सिद्धः संकेतेन व्य‚ज्य‚त इत्याह ।}} स‚ङ्केताद् वाच‚क‚ताया\edtext{}{\edlabel{pvv.408-8}\label{pvv.408-8}\lemma{ताया}\Bfootnote{स्व‚भाव‚विशेष‚स्य ।}} ‚{\color{DodgerBlue3}‚व्य‚क्ता}‚विष्य‚{\tiny $_{lb}$}‚माणायां ‚{\color{DodgerBlue3}‚निय‚मः कुतः} (।) अय‚मेवास्य श‚ब्द‚स्यार्थ इति निय‚मो न युज्य‚ते ‚{\tiny $_{lb}$}‚पुरुषाधीन‚स‚ङ्केतापेक्षायां । (३२८)
	\pend% ending standard par
      \label{div_pvv.3.329}
	  
	% new div opening: depth here is 2
	

	  \pstart \leavevmode% starting standard par
	त‚थाहि (।)
	\pend% ending standard par
      \textsuperscript{\textenglish{409/s}}
	  \bigskip
	  \begingroup
	
	    \large
	  
	    \begin{quote}
	  
	    
	    \stanza[\smallbreak]
	\label{pv.3.329}\flagstanza{\tiny\textenglish{....3.329}}य‚त्र स्वात‚न्त्र्य‚मिच्छाया निय‚मो नाम त‚त्र कः ।&द्योत‚येन् तेन स‚ङ्केतो नेष्टामेवास्य योग्य‚ताम् ॥ ३२९ ॥\&[\smallbreak]


	
	    \end{quote}
	  
	  \endgroup
	

	  \pstart \leavevmode% starting standard par
	य‚त्र\edtext{}{\edlabel{pvv.409-1}\label{pvv.409-1}\lemma{त्र}\Bfootnote{संकेते ।}} पुरुष‚स्य स्वात ‚{\tiny $_{5}$}‚न्त्र्य‚मिच्छायास्त‚त्र निय‚मो नाम कः स‚ङ्ग‚तः । \edtext{\textsuperscript{*}}{\edlabel{pvv.409-2}\label{pvv.409-2}\lemma{*}\Bfootnote{अनिय‚त‚त्वेन ।}}तेनेच्छा‚{\tiny $_{lb}$}‚धीन‚त्वेन संकेतो नेष्टामेव\edtext{}{\edlabel{pvv.409-3}\label{pvv.409-3}\lemma{नेष्टामेव}\Bfootnote{वैदिक‚श‚ब्द‚स्य ।}}योग्य‚तामुद्द्योत‚येत् । अन‚भिम‚ताम‚पि व्य‚ञ्ज‚येदित्य‚र्थः । ‚{\tiny $_{lb}$}‚त‚देव‚म‚पौरुषेय‚त्व‚माग‚म‚ल‚क्ष‚णं ब्रुवाणा मी मां स काः प्र‚तिक्षिप्ताः । (३२९)
	\pend% ending standard par
      \label{div_pvv.3.330}
	  
	% new div opening: depth here is 2
	

	  \begin{center}%% label @type='head'
	\textbf{(२ ) b. बुद्ध‚मीमांस‚क(जैमिनि)म‚त‚निरासः}
	\end{center}
	

	  \begin{center}%% label @type='head'
	\textbf{क. वेदैंक‚देश‚संवादित्वे न स‚र्व‚स्य प्रामाण्य‚म्}
	\end{center}
	

	  \pstart \leavevmode% starting standard par
	\edtext{\textsuperscript{*}}{\edlabel{pvv.409-4}\label{pvv.409-4}\lemma{*}\Bfootnote{त‚देव‚म‚पौरुषेय‚त्वं नाग‚म‚ल‚क्ष‚ण‚मित्युक्त्वा एक‚देशाविस‚म्वाद‚न‚माग‚म‚ल‚क्ष‚णं दूष‚यितुमाह ।}}संप्र‚ति बृ द्ध मी मां स कानां म‚तं दूष‚यितुमुत्थाप‚य‚ति ।
	\pend% ending standard par
      
	  \bigskip
	  \begingroup
	
	    \large
	  
	    \begin{quote}
	  
	    
	    \stanza[\smallbreak]
	\label{pv.3.330}\flagstanza{\tiny\textenglish{....3.330}}य‚स्मात् किलेदृशं स‚त्यं य‚थाग्निः शीत‚नोद‚नः ।&वाक्यं वेदैक‚देश‚त्वाद‚न्य‚द‚प्य‚प‚रोऽव्र‚वीत् ॥ ३३० ॥\&[\smallbreak]


	
	    \end{quote}
	  
	  \endgroup
	

	  \pstart \leavevmode% starting standard par
	\edtext{\textsuperscript{*}}{\edlabel{pvv.409-5}\label{pvv.409-5}\lemma{*}\Bfootnote{य‚स्मात् स‚त्यं त‚न्न दूष्य‚मिति प‚रः}} य‚था वैदिकं वाक्यम‚ग्निर्हिम‚स्य भेष‚ज‚मिति ‚{\color{DodgerBlue3}‚स‚त्यं शीत‚नोद‚न‚त्वेन व‚ह्नेः} प्र‚माण‚सिद्ध‚त्वा‚{\tiny $_{6}$}‚त् त‚थाऽन्न्य‚द‚पि\edtext{}{\edlabel{pvv.409-6}\label{pvv.409-6}\lemma{पि}\Bfootnote{विशेष‚स्य प‚क्षीक‚र‚णात् साध्य‚माह । य‚त्र बौद्ध‚स्य विप्र‚तिप‚त्तिरेक‚देशे त‚द‚वित‚थ‚मिति ।}} ‚{\color{DodgerBlue3}‚वाक्य}म‚ग्निष्टोमेन जुहुयात् स्व‚र्ग‚काम इत्यादि ‚{\tiny $_{lb}$}‚(।) ‚{\color{DodgerBlue3}‚वेदैक‚देश‚त्वात्\edtext{}{\edlabel{pvv.409-7}\label{pvv.409-7}\lemma{त्वात्}\Bfootnote{हेतुः सामान्यं प्र‚तिज्ञार्थ‚तानिरासाय ।}}} ‚{\color{DodgerBlue3}‚स‚त्य}‚मित्य‚{\color{DodgerBlue3}‚प‚रो} मी मां स को\edtext{}{\edlabel{pvv.409-8}\label{pvv.409-8}\lemma{को}\Bfootnote{वृद्धः च‚क्षुर्दोषोप‚ह‚त‚त्वात् ।}}ऽब्र‚वीदुक्त‚वान् (।) ‚{\color{DodgerBlue3}‚किल} श‚ब्दोऽक्ष‚मायां । (३३०)
	\pend% ending standard par
      \label{div_pvv.3.331}
	  
	% new div opening: depth here is 2
	
	  \bigskip
	  \begingroup
	
	    \large
	  
	    \begin{quote}
	  
	    
	    \stanza[\smallbreak]
	\label{pv.3.331}\flagstanza{\tiny\textenglish{....3.331}}र‚स‚व‚त् तुल्य‚रूप‚त्वादेक‚भाण्डे च पाक‚व‚त् ।&शेष‚व‚द् व्य‚भिचारित्वात् क्षिप्तं न्याय‚विदेदृश‚म् ॥ ३३१ ॥\&[\smallbreak]


	
	    \end{quote}
	  
	  \endgroup
	

	  \pstart \leavevmode% starting standard par
	\hphantom{.}ईदृश‚म‚नुमानं ‚{\color{DodgerBlue3}‚शेष‚व‚द}‚नैकान्तिकं ‚{\color{DodgerBlue3}‚व्य‚भिचारित्वान्न्याय‚विदाचार्य}\edtext{}{\edlabel{pvv.409-9}\label{pvv.409-9}\lemma{नैकान्तिकं}\Bfootnote{नैयायिक‚शेष‚व‚द‚नुमान‚व्य‚भिचार‚मुद्भाव‚य‚ता (प्र‚माण) स‚मुच्च‚ये ।}}दि ग्ना‚{\tiny $_{lb}$}‚गे न प्र‚ति‚{\color{DodgerBlue3}‚क्षिप्तं र‚स‚व‚त् तुल्य‚रूप‚त्वात्} । स्वादित‚फ‚लेन तुल्य‚रूप‚त्वात् फ‚लान्त‚र‚स्य ‚{\tiny $_{lb}$}‚तादृग्\edtext{}{\edlabel{pvv.409-10}\label{pvv.409-10}\lemma{तादृग्}\Bfootnote{अस्वादितं तुल्य‚र‚स‚मिति ।}}र‚सानुमान‚व‚त् । ‚{\color{DodgerBlue3}‚एक‚भाण्डा}‚न्त‚र्ग‚त्वा‚{\tiny $_{7}$}‚त् दृष्ट‚{\color{DodgerBlue3}‚पाक}‚त‚ण्डुला\edtext{}{\edlabel{pvv.409-11}\label{pvv.409-11}\lemma{ण्डुला}\Bfootnote{एक‚भाण्डे प‚च‚नात् । अदृष्टा अपि त‚ण्डुलाः प‚क्वा इति साध्यं ।}}नुमान‚व‚त् ।\leavevmode\ledsidenote{\textenglish{81a/MA}} ‚{\tiny $_{lb}$}‚\leavevmode\ledsidenote{\textenglish{410/s}} न ह्येतादृश‚स्य हेतोः साध्य‚प्र‚तिब‚न्धोस्ति विप‚र्य‚य‚बाध‚क‚प्र‚माणादिति विप‚ञ्चितं ‚{\tiny $_{lb}$}‚प्राक् \edtext{}{\edlabel{pvv.410-1}\label{pvv.410-1}\lemma{प्राक्}\Bfootnote{य‚स्याद‚र्श‚न‚मात्रेण व्य‚तिरेकः प्र‚द‚र्श्य‚त \cref{pv.3.13} इत्यादिना ।}}। (३३१)
	\pend% ending standard par
      \label{div_pvv.3.332_3.333_3.334}
	  
	% new div opening: depth here is 2
	

	  \begin{center}%% label @type='head'
	\textbf{ख. वेद‚प्रामाण्य‚घोष‚णा जैमिनेर्धार्ष्ट्य‚म्}
	\end{center}
	

	  \pstart \leavevmode% starting standard par
	दृश्य‚ते च ब‚हुत‚र‚विष‚ये विस‚म्वादो वेद‚स्य क‚थ‚मेक‚स‚त्य‚त‚या स‚र्व्व‚त्र त‚थात्वं । ‚{\tiny $_{lb}$}‚त‚था \edtext{}{\edlabel{pvv.410-2}\label{pvv.410-2}\lemma{था}\Bfootnote{श‚क्य‚प‚रिच्छेदेपि विष‚ये प्र‚माण‚विरोधाद‚युक्त‚त्वेमेवाह ।}} च (।)
	\pend% ending standard par
      
	  \bigskip
	  \begingroup
	
	    \large
	  
	    \begin{quote}
	  
	    
	    \stanza[\smallbreak]
	\label{pv.3.332}\flagstanza{\tiny\textenglish{....3.332}}नित्य‚स्य पुंसः क‚र्तृत्वं नित्यान् भावान‚तीन्द्रियान् ।&ऐन्द्रियान्विष‚मं हेतुं भावानां विष‚मां स्थितिम् ॥ ३३२ ॥\&[\smallbreak]


	
	    \end{quote}
	  
	  \endgroup
	
	  \bigskip
	  \begingroup
	
	    \large
	  
	    \begin{quote}
	  
	    
	    \stanza[\smallbreak]
	\label{pv.3.333}\flagstanza{\tiny\textenglish{....3.333}}निवृत्तिं च प्र‚माणाभ्याम‚न्य‚द् वा व्य‚स्त‚गोच‚र‚म् ।&विरुद्ध‚माग‚मापेक्षेणानुमानेन वा व‚द‚त् ॥ ३३३ ॥\&[\smallbreak]


	
	    \end{quote}
	  
	  \endgroup
	
	  \bigskip
	  \begingroup
	
	    \large
	  
	    \begin{quote}
	  
	    
	    \stanza[\smallbreak]
	\label{pv.3.334}\flagstanza{\tiny\textenglish{....3.334}}विरोध‚म‚स‚माधाय शास्त्रार्थं चाप्र‚द‚र्श्य सः ।&स‚त्यार्थं प्र‚तिज्ञानानो ज‚येद् धाष्ट‚र्येन ब‚न्ध‚कीम् ॥ ३३४ ॥\&[\smallbreak]


	
	    \end{quote}
	  
	  \endgroup
	

	  \pstart \leavevmode% starting standard par
	\hphantom{.}प‚रैर‚नाधेय‚विशेष‚स्य ‚{\color{DodgerBlue3}‚पुंस} आत्म‚नः क्र‚म‚ज‚न्म‚सु क‚र्मादिषु \edtext{}{\edlabel{pvv.410-3}\label{pvv.410-3}\lemma{र्मादिषु}\Bfootnote{त‚त्फ‚ल‚भोक्तृत्वं ।}} ‚{\color{DodgerBlue3}‚क‚र्तृत्वं} व‚द\edtext{}{\edlabel{pvv.410-4}\label{pvv.410-4}\lemma{द}\Bfootnote{व‚द‚दिति स‚र्व‚त्र योज्यं त्रिभिः श्लोकैर‚त्र स‚म्ब‚न्धः । सुख‚दुःखादिस‚म्वित्तिस‚म‚वाय‚स्तु भोक्तृता । त‚न्न नित्यानां कार्य‚क‚र‚ण‚त्व‚स्य निर‚स्त‚त्वात् ।}}च्छा‚{\tiny $_{lb}$}‚स्त्रं । त‚था ‚{\color{DodgerBlue3}‚नित्यान् भावान्} दिक्कालाकाशादीन‚र्थ‚क्रियार‚हितान् स‚तो \edtext{}{\edlabel{pvv.410-5}\label{pvv.410-5}\lemma{तो}\Bfootnote{अक्ष‚णिक‚स्य क्र‚म‚यौग‚प‚द्यार्थ‚क्रियाविरोध‚ने व‚स्तुध‚र्मातिक्र‚माद‚युक्तं त‚त् ।}} व‚द‚त् । ‚{\tiny $_{lb}$}‚व‚स्तुतो‚{\color{DodgerBlue3}‚ऽतीन्द्रियान्}‚{\tiny $_{1}$}‚ गुण‚क‚र्म‚सामान्यादीन् ‚{\color{DodgerBlue3}‚ऐन्द्रियान्} \edtext{\textsuperscript{*}}{\edlabel{pvv.410-6}\label{pvv.410-6}\lemma{*}\Bfootnote{त‚द‚युक्त‚म‚न‚ध्य‚क्ष‚स्याध्य‚क्ष‚त्व‚विरोधात् ।}} प्र‚त्य‚क्षान् व‚द‚त् । त‚था ‚{\tiny $_{lb}$}‚‚{\color{DodgerBlue3}‚भावानां विष‚मं हेतुं} प्राग\edtext{}{\edlabel{pvv.410-7}\label{pvv.410-7}\lemma{प्राग}\Bfootnote{अर्थ‚क्रियाः प्राग‚ज‚न‚कं प‚श्चात् स‚ह‚कार्य‚पेक्ष‚या ज‚न‚कं त‚द‚युक्तं नित्य‚स्याविशेषात् ।}}ज‚न‚कं व‚द‚त् । त‚था ‚{\color{DodgerBlue3}‚विष‚मां स्थितिं निस्प} (? ष्प) ‚{\tiny $_{lb}$}‚न्नानां भावानामाश्र‚य‚व‚शेन \edtext{}{\edlabel{pvv.410-8}\label{pvv.410-8}\lemma{शेन}\Bfootnote{संश्च स‚र्व्व‚निराशंसो भावः क‚थ‚म‚पेक्ष‚ते प‚रं ।}} ‚{\color{DodgerBlue3}‚स्थितिं} व‚द‚त् । ‚{\color{DodgerBlue3}‚निवृत्तिञ्च विष‚मां} स्व‚तोऽन‚श्व‚र‚{\tiny $_{lb}$}‚स्व‚भावा\edtext{}{\edlabel{pvv.410-9}\label{pvv.410-9}\lemma{भावा}\Bfootnote{विनाश‚स्याहेतुत्वोक्तेः ।}}नाम‚न्य‚कृतां निवृत्तिं व‚द‚त् । ‚{\color{DodgerBlue3}‚अन्य‚द्वा} \edtext{\textsuperscript{*}}{\edlabel{pvv.410-10}\label{pvv.410-10}\lemma{*}\Bfootnote{अन्य‚द‚प्य‚भावादिकं स‚दित्याह ।}}व‚स्तु ‚{\color{DodgerBlue3}‚व्य‚स्त‚गोच‚रं प्र‚माणा‚{\tiny $_{lb}$}‚भ्यां} प्र‚त्य‚क्षानुमानाभ्यां निर‚स्ताव‚काशं व‚द‚त् । ‚{\color{DodgerBlue3}‚आग‚मापेक्षेणा}‚ग‚म‚सिद्ध‚लिङ्ग‚{\tiny $_{lb}$}‚\leavevmode\ledsidenote{\textenglish{411/s}} त्रैरूप्येणा‚{\color{DodgerBlue3}‚नुमानेन} ‚{\tiny $_{2}$}‚ च विरुद्ध‚म‚ग्निहोत्र‚स्नानादेः पाप‚श‚म‚नं ‚{\color{DodgerBlue3}‚व‚द‚त्} । \edtext{\textsuperscript{*}}{\edlabel{pvv.411-1}\label{pvv.411-1}\lemma{*}\Bfootnote{आग‚माश्र‚यानुमान‚बाधित‚मेत‚दित्याह ।}} न हि ‚{\tiny $_{lb}$}‚रागादिप्र‚भ‚वो ध‚र्म‚स्त‚द‚प‚न‚य‚न‚म‚न्त‚रेण स्नानादेर्निवृत्तिम‚र्ह‚ति । एव‚म्विध‚विस‚{\tiny $_{lb}$}‚म्वाद‚भाज‚म‚र्थं व‚द‚त् शास्त्रं ‚{\color{DodgerBlue3}‚विरोधं} \edtext{\textsuperscript{*}}{\edlabel{pvv.411-2}\label{pvv.411-2}\lemma{*}\Bfootnote{श‚क्य‚विचारे व‚स्तुनि ।}} प्रामाणिक‚{\color{DodgerBlue3}‚म‚स‚माधाय} अप‚रिह्य‚त्य प्र‚वृत्ति‚{\tiny $_{lb}$}‚कामोचितं ‚{\color{DodgerBlue3}‚शास्त्रार्थ‚म‚नु}\edtext{}{\edlabel{pvv.411-3}\label{pvv.411-3}\lemma{कामोचितं}\Bfootnote{पुरुष‚प्र‚वृत्तिनिमित्तं ।}}गुणोपायं पुरुषार्थ‚ल‚क्ष‚ण‚{\color{DodgerBlue3}‚ञ्चाप्र‚द‚र्श्य स} ज‚र न्मी मां स क ‚{\tiny $_{lb}$}‚एक‚देश‚स‚म्वाद‚द‚र्श‚नात् ‚{\color{DodgerBlue3}‚स‚त्यार्थं प्र‚तिजानानो धा‚{\tiny $_{3}$}‚र्ष्ट्येन ब‚न्ध‚कीं} साक्षाद्दृष्ट‚व्य‚लीकां ‚{\tiny $_{lb}$}‚स्व‚प‚तिं स्व‚शील‚प्रामाण्योद्भाव‚नेन भ्रान्त‚मावेद‚य‚न्तीं ‚{\color{DodgerBlue3}‚ज‚येत्} । (३३२-३३४)
	\pend% ending standard par
      \label{div_pvv.3.335}
	  
	% new div opening: depth here is 2
	

	  \pstart \leavevmode% starting standard par
	किञ्च (।)
	\pend% ending standard par
      
	  \bigskip
	  \begingroup
	
	    \large
	  
	    \begin{quote}
	  
	    
	    \stanza[\smallbreak]
	\label{pv.3.335}\flagstanza{\tiny\textenglish{....3.335}}सिध्येत् प्र‚माणं य‚द्येव‚म‚प्र‚माण‚म‚थेह किम् ।&न ह्येकं नास्ति स‚त्यार्थं पुरुषे ब‚हुभाषिणि ॥ ३३५ ॥\&[\smallbreak]


	
	    \end{quote}
	  
	  \endgroup
	

	  \pstart \leavevmode% starting standard par
	\hphantom{.}‚{\color{DodgerBlue3}‚य‚द्येवं\edtext{}{\edlabel{pvv.411-4}\label{pvv.411-4}\lemma{द्येवं}\Bfootnote{स‚र्व्वः पुरुषः स‚र्व्व‚त्रार्थे प्र‚माणं स्यादित्याह ।}}} दृष्टैक‚देश‚स‚म्वाद‚स्याव‚य‚व‚त्वाद‚न्य‚द‚पि ‚{\color{DodgerBlue3}‚प्र‚माणं सिध्येत् । अथेह} न्याये ‚{\tiny $_{lb}$}‚‚{\color{DodgerBlue3}‚किम‚प्र‚माणं} शास्त्रं भ‚विष्य‚ति । ‚{\color{DodgerBlue3}‚न हि ब‚हुभाषिणि पुरुषे} स‚त्यार्थ‚{\color{DodgerBlue3}‚मेकं} व‚चे ‚{\color{DodgerBlue3}‚नास्ति} किन्त्व‚स्त्येव । तेनैव दृष्टान्तेन त‚द्व‚च‚नं प्र‚माणं स्यात् । (३३५)
	\pend% ending standard par
      \label{div_pvv.3.336}
	  
	% new div opening: depth here is 2
	
	  \bigskip
	  \begingroup
	
	    \large
	  
	    \begin{quote}
	  
	    
	    \stanza[\smallbreak]
	\label{pv.3.336}\flagstanza{\tiny\textenglish{....3.336}}नायं स्व‚भावः कार्यं वा व‚स्तूनां व‚क्त‚रि ध्व‚निः ।&न च त‚द्व्य‚तिरिक्त‚स्य विद्य‚तेव्य‚भिचारिता ॥ ३३६ ॥\&[\smallbreak]


	
	    \end{quote}
	  
	  \endgroup
	

	  \pstart \leavevmode% starting standard par
	\hphantom{.}‚{\color{DodgerBlue3}‚नायं} ध्व‚निर्व्व‚{\color{DodgerBlue3}‚स्तूनां स्व‚भावः । कार्य‚म्वा‚{\tiny $_{4}$}‚ \edtext{\textsuperscript{*}}{\edlabel{pvv.411-5}\label{pvv.411-5}\lemma{*}\Bfootnote{विना वाच्यं न श‚ब्द‚वृत्तिश्चेत् ।}}} य‚स्माद् ‚{\color{DodgerBlue3}‚व‚क्त‚रि ध्व‚निर्भ‚व‚ति} । न हि ‚{\tiny $_{lb}$}‚व‚स्तुनः स्व‚भावोन्य‚त्र ध‚र्मिणि व‚र्त्त‚ते । कार्य‚म्वाऽन्य‚तो भ‚वितुम‚र्ह‚ति । ‚{\color{DodgerBlue3}‚न च त‚द्व्य‚ति ‚{\tiny $_{lb}$}‚रिक्त‚स्य} कार्य‚स्व‚भावाभ्याम‚प‚र‚स्या‚{\color{DodgerBlue3}‚व्य‚भिचारि}‚ता ‚{\color{DodgerBlue3}‚विद्य‚त} इति निवेदितं । (३३६)
	\pend% ending standard par
      \label{div_pvv.3.337}
	  
	% new div opening: depth here is 2
	

	  \pstart \leavevmode% starting standard par
	स्यादेत‚द् (।)
	\pend% ending standard par
      
	  \bigskip
	  \begingroup
	
	    \large
	  
	    \begin{quote}
	  
	    
	    \stanza[\smallbreak]
	\label{pv.3.337}\flagstanza{\tiny\textenglish{....3.337}}प्र‚वृत्तिर्वाच‚कानाञ्च वाच्य‚दृष्टिकृतेति चेत् ।&प‚र‚स्प‚र‚विरुद्धार्था क‚थ‚मेक‚त्र सा भ‚वेत् ॥ ३३७ ॥\&[\smallbreak]


	
	    \end{quote}
	  
	  \endgroup
	

	  \pstart \leavevmode% starting standard par
	\hphantom{.}‚{\color{DodgerBlue3}‚वाच‚कानां} श‚ब्दानां ‚{\color{DodgerBlue3}‚प्र‚वृत्तिर्व्वाच्य‚दृष्टिकृता} अभिधेय‚द‚र्श‚नाग‚ता त‚तः प‚रंप‚र‚या\edtext{}{\edlabel{pvv.411-6}\label{pvv.411-6}\lemma{या}\Bfootnote{प‚दानां स‚ङ्ग‚तिः स‚म्ब‚न्धः । श‚क्य‚साध‚न उपायोनुगुणः । अभ्युद‚यादिः पुरुषार्थ इति शास्त्र‚ध‚र्मा प्र‚द‚र्श्य विरोध‚म‚स‚माधाय चात्य‚न्त‚प्र‚सिद्ध‚स‚त्यार्थ‚तामात्रेण प्र‚ज्ञाप्र‚क‚र्षेणापि दुर‚व‚गाहेपि स‚त्यार्थ‚तां साध‚य‚न् दुश्चारिणीं ज‚येत् । सा स्वामिना प‚रेण स‚ङ्ग‚ता त्व‚मित्युपाल‚ब्धाऽह । प‚श्य‚त पुंसो वैप‚रीत्यं ध‚र्म‚प‚त्न्या प्र‚त्य‚य‚म‚कृत्वा स्व‚नेत्र‚बुद्बुद‚योः प्र‚त्येति । ज‚र‚त्काण‚ग्राम्य‚काष्ट‚हारेण प्रार्थिताऽस‚ङ्ग‚ता । रूप‚गुणानुरागेण किल म‚न्त्रिमुख्य‚दार‚कं काम‚येह‚मिति तुल्यं दृष्ट‚विरोध‚स्यातिप‚रोक्षेऽविस‚म्वादानुमानेन \href{http://sarit.indology.info/?cref=pvsv}{(स्व‚वृत्तौ)} ।}} ‚{\tiny $_{lb}$}‚\leavevmode\ledsidenote{\textenglish{412/s}} त‚त्कार्य‚तैवैषामि‚{\color{DodgerBlue3}‚ति चेत्} । एवं त‚र्हि सा श‚ब्द‚प्र‚वृत्ति‚{\color{DodgerBlue3}‚रेक‚त्र} व‚स्तुनि ‚{\color{DodgerBlue3}‚प‚र‚स्प‚र‚विरुद्धार्था ‚{\tiny $_{lb}$}‚क‚थ‚म्भ‚वेत्} ।‚{\tiny $_{5}$}‚ (३३७)
	\pend% ending standard par
      \label{div_pvv.3.338}
	  
	% new div opening: depth here is 2
	

	  \pstart \leavevmode% starting standard par
	य‚दि य‚था\edtext{}{\edlabel{pvv.412-1}\label{pvv.412-1}\lemma{था}\Bfootnote{वाच्यार्थ‚स्य ग‚म‚को हि त‚ज्ज‚न्य‚स्त‚त्स्व‚भावो वा स्यादित्याह श‚ब्द‚स्त्विच्छाय‚त्तो न ब‚हिर‚धीनः (।) स‚ति वाच्ये त‚द्द‚र्श‚नं दृष्टेर्विव‚क्षा । त‚तो व‚च‚नं प‚र‚म्प‚रा(?)}} व‚स्त्वेव श‚ब्द‚स्त‚दा व‚स्तुत एक‚रूपेण\edtext{}{\edlabel{pvv.412-2}\label{pvv.412-2}\lemma{रूपेण}\Bfootnote{आग‚म्य‚ते तेन ।}} श‚ब्दे नित्यः किम‚य‚म‚नित्यो ‚{\tiny $_{lb}$}‚वेत्यादि श‚ब्द‚स‚न्द‚र्भो न भ‚वेत् ।
	\pend% ending standard par
      
	  \bigskip
	  \begingroup
	
	    \large
	  
	    \begin{quote}
	  
	    
	    \stanza[\smallbreak]
	\label{pv.3.338}\flagstanza{\tiny\textenglish{....3.338}}व‚स्तुभिर्नाग‚मास्तेन क‚थ‚ञ्चिन्नान्त‚रीय‚काः ।&प्र‚तिप‚त्तुर्न सिध्य‚न्ति कुत‚स्तेभ्योऽर्थ‚निश्च‚यः ॥ ३३८ ॥\&[\smallbreak]


	
	    \end{quote}
	  
	  \endgroup
	

	  \pstart \leavevmode% starting standard par
	\hphantom{.}‚{\color{DodgerBlue3}‚तेन}\edtext{\textsuperscript{*}}{\edlabel{pvv.412-3}\label{pvv.412-3}\lemma{*}\Bfootnote{श‚ब्दानां व‚स्त्व‚स‚म्ब‚न्धेन ।}} प्र‚तिब‚न्धाभावेन ‚{\color{DodgerBlue3}‚व‚स्तुभिः} स‚ह ‚{\color{DodgerBlue3}‚नान्त‚रीय‚का आग‚मा प्र‚तिप‚त्तु}‚र‚र्थं श‚ब्दात् ‚{\tiny $_{lb}$}‚प्र‚तिप‚द्य‚मान‚स्य ‚{\color{DodgerBlue3}‚क‚थ‚ञ्चिन्न} सिध्य‚न्ति । त‚त् कुत‚स्तेभ्य आग‚मार्थ‚निश्च‚यः । (३३८)
	\pend% ending standard par
      \label{div_pvv.3.339}
	  
	% new div opening: depth here is 2
	
	  \bigskip
	  \begingroup
	
	    \large
	  
	    \begin{quote}
	  
	    
	    \stanza[\smallbreak]
	\label{pv.3.339}\flagstanza{\tiny\textenglish{....3.339}}त‚स्मान्न त‚न्निवृत्त्यापि भावाभावः प्र‚सिध्य‚ति ।&तेनास‚न्निश्च‚य‚फ‚लाऽनुप‚ल‚ब्धिर्न सिध्य‚ति ॥ ३३९ ॥\&[\smallbreak]


	
	    \end{quote}
	  
	  \endgroup
	

	  \pstart \leavevmode% starting standard par
	\hphantom{.}य‚स्मात् प्र‚व‚र्त्त‚मानादाग‚मान्नास्त्य‚र्थ‚सिद्धिस्त्‚{\color{DodgerBlue3}‚स्मात् त}‚स्याग‚म‚स्य ‚{\color{DodgerBlue3}‚निवृत्त्यापि ‚{\tiny $_{lb}$}‚भाव‚स्याभावो न सिध्य‚ति । तेन} प्र‚मा‚{\tiny $_{6}$}‚ण‚त्र‚य‚निवृत्तिल‚क्ष‚णाप्य‚{\color{DodgerBlue3}‚नुप‚ल‚ब्धि}‚र‚र्थाना‚{\tiny $_{lb}$}‚‚{\color{DodgerBlue3}‚म‚स‚न्निश्च‚य‚फ‚ला न सिध्य‚ति} । त‚तो युक्त‚मुक्तं स‚द‚स‚न्निश्च‚य\edtext{}{\edlabel{pvv.412-4}\label{pvv.412-4}\lemma{य}\Bfootnote{स‚र्व्व‚विष‚य‚त्वादाग‚म‚स्य स‚ति व‚स्तुन्य‚विस‚म्वादेनावृत्तेस्त‚न्निवृत्तिल‚क्ष‚णानुप‚ल‚ब्धिर‚भाव‚साध‚न‚मित्य‚युक्त‚मेव प‚र‚स्य । विनापि व‚स्त्वाग‚म‚प्र‚वृत्तेः स‚र्व्व‚विष‚य‚त्व‚ञ्च निर‚स्त‚म‚प्र‚स्तुताव‚च‚नात् ।}}फ‚ला नेति स्याद् वा‚{\tiny $_{lb}$}‚ऽप्र‚माण‚तेति । स‚द्व्य‚व‚हार‚साध‚ने चाधिकृते व्य‚व‚स्थितेत्य‚व‚स्थितं ॥ (३३९)
	\pend% ending standard par
      
	    
	    \pstart
	    \begin{center}
	  आचार्य म नो र थ न न्दि कृतायां वार्तिक‚वृत्तौ तृतीयः प‚रिच्छेदः ॥
	    \end{center}
	    \pend
	  
	  
	    
	    \endnumbering% ending numbering from div
	    \endgroup
	    
	  
	  
	% new div opening: depth here is 0
	
	    
	    \begingroup
	    \beginnumbering% beginning numbering from div depth=0
	    
	  
\chapter*[{च‚तुर्थः प‚रिच्छेदःच‚तुर्थो व्याख्याय‚ते । त‚त्र स‚म्ब‚न्ध‚म‚न्त‚रेण वाक्यार्थावृत्तेरिति तृतीय‚प‚रिच्छेदं व्याख्यायानेनोत्त‚र‚स‚म्ब‚न्धं क‚थ‚य‚न् प्र‚तिप‚र्तृसुखार्थ‚म‚न्त्य‚प‚रिच्छेदे स‚र्व‚मेव शास्त्र‚श‚रीर‚म‚नुक्र‚मेण वृत्तिकृत् क‚थ‚य‚ति अनुमान‚मित्यादि ।: प‚रार्थानुमानं}]{च‚तुर्थः प‚रिच्छेदःच‚तुर्थो व्याख्याय‚ते । त‚त्र स‚म्ब‚न्ध‚म‚न्त‚रेण वाक्यार्थावृत्तेरिति तृतीय‚प‚रिच्छेदं व्याख्यायानेनोत्त‚र‚स‚म्ब‚न्धं क‚थ‚य‚न् प्र‚तिप‚र्तृसुखार्थ‚म‚न्त्य‚प‚रिच्छेदे स‚र्व‚मेव शास्त्र‚श‚रीर‚म‚नुक्र‚मेण वृत्तिकृत् क‚थ‚य‚ति अनुमान‚मित्यादि ।
	: प‚रार्थानुमानं}\label{div_pvv.iv.0}
	  
	% new div opening: depth here is 1
	\textsuperscript{\textenglish{413/s}}

	  \pstart \leavevmode% starting standard par
	स्वार्थानुमानं‚{\tiny $_{7}$}‚ व्याख्याय प‚रार्थानुमानं व्याख्यातुमाह (।)
	\pend% ending standard par
      \textsuperscript{\textenglish{81b/MA}}‚{\tiny $_{lb}$}‚

	  \pstart \leavevmode% starting standard par
	त‚त्र प‚रार्थानुमानं\edtext{}{\edlabel{pvv.413-2}\label{pvv.413-2}\lemma{रार्थानुमानं}\Bfootnote{[त‚त्रैव]---प‚रार्थ‚म‚नुमानं तु ।}} स्व‚दृष्टार्थ‚प्र‚काश‚न‚मित्याचार्यीय‚ल‚क्ष‚णं । स्वेन\edtext{}{\edlabel{pvv.413-3}\label{pvv.413-3}\lemma{स्वेन}\Bfootnote{वादिप्र‚तिवादिभ्यामिति प्र‚क‚र‚णात् ।}} ‚{\tiny $_{lb}$}‚दृष्टः स्व‚दृष्टः । स्व‚दृष्ट‚श्चासाव‚र्थ‚श्चेति त्रिरूपो हेतुः । त‚स्य प्र‚काश‚न‚म्व‚च‚नं ‚{\tiny $_{lb}$}‚अनुमान‚हेतुत्वादित्य‚र्थः ।
	\pend% ending standard par
      
	  
	% new div opening: depth here is 1
	
\chapter*[{१. प‚रार्थानुमान‚ल‚क्ष‚ण‚म् (दिग्नाग‚स्य)}]{१. प‚रार्थानुमान‚ल‚क्ष‚ण‚म् (दिग्नाग‚स्य)}\label{div_pvv.4.1}
	  
	% new div opening: depth here is 2
	

	  \begin{center}%% label @type='head'
	\textbf{(१) त‚त्र स्व‚दृष्ट‚ग्र‚ह‚ण‚फ‚ल‚म्}
	\end{center}
	

	  \begin{center}%% label @type='head'
	\textbf{क. प‚र‚म‚त‚निरासाय}
	\end{center}
	

	  \pstart \leavevmode% starting standard par
	न‚नु त्रिरूप‚लिङ्गाख्यान‚मित्येवास्तु किं स्व‚दृष्ट‚ग्र‚ह‚णेनेति श‚ङ्कायां प‚र‚म‚{\tiny $_{lb}$}‚त‚निरासार्थ‚ताम‚स्य द‚र्श‚यितुमाह ।
	\pend% ending standard par
      
	  \bigskip
	  \begingroup
	
	    \large
	  
	    \begin{quote}
	  
	    
	    \stanza[\smallbreak]
	\label{pv.4.1}\flagstanza{\tiny\textenglish{...pv.4.1}}प‚र‚स्य प्र‚तिपाद्य‚त्वात् अदृष्टोपि स्व‚यं प‚रैः ।&दृष्ट‚साध‚न‚मित्येके त‚त्क्षेपायात्म‚दृग्व‚चः ॥ १ ॥\&[\smallbreak]


	
	    \end{quote}
	  
	  \endgroup
	

	  \pstart \leavevmode% starting standard par
	\hphantom{.}‚{\color{DodgerBlue3}‚प‚र‚स्य} प‚रार्थानुमानेन ‚{\color{DodgerBlue3}‚प्र‚तिपाद्य‚त्वात्} साध‚न‚{\tiny $_{1}$}‚वादिना स्व‚य‚म‚{\color{DodgerBlue3}‚दृष्टोपि प्र‚माणेन ‚{\tiny $_{lb}$}‚प‚रैः} प्र‚तिवादिभिराग‚माद् ‚{\color{DodgerBlue3}‚दृष्ट‚साध‚नं} लिङ्ग‚{\color{DodgerBlue3}‚मित्येके} सां ख्याः । ते हि सुखादी‚{\tiny $_{lb}$}‚नामुत्प‚त्तिम‚त्वाद‚नित्याद‚चेत‚न‚त्वं रूपादीनामिव बौद्धं प्र‚त्याहुः । ‚{\color{DodgerBlue3}‚न ह्य‚स‚त} उत्प‚त्ति‚{\tiny $_{lb}$}‚म‚त्वं स‚त‚श्च निर‚न्व‚य‚विनाशोऽनित्य‚त्वं हेतुः सांख्यासिद्धः । बौद्ध‚स्य पुन‚राग‚मात् ‚{\tiny $_{lb}$}‚\leavevmode\ledsidenote{\textenglish{414/s}} सिद्धः । ताव‚तैव हेतुरिति म‚न्य‚न्ते । त‚स्य प‚र‚म‚त‚स्य ‚{\color{DodgerBlue3}‚क्षेपाय} प्र‚तिषेधाया‚{\color{DodgerBlue3}‚त्म‚{\tiny $_{lb}$}‚दृश्र/?/व‚चः} अदृष्ट‚व‚च‚नं सूत्रे । न प्र‚तिवादिमात्र‚सिद्ध‚स्य हेतुत्वं किन्तूभ‚य‚सिद्ध‚स्यैव‚{\tiny $_{lb}$}‚त्य‚र्थः ॥ (१)
	\pend% ending standard par
      \label{div_pvv.4.2}
	  
	% new div opening: depth here is 2
	

	  \begin{center}%% label @type='head'
	\textbf{ख. अनुमान‚विष‚ये नाग‚म‚स्य प्रामाण्य‚म्}
	\end{center}
	

	  \pstart \leavevmode% starting standard par
	न चाग‚मात् प्र‚तिवादिनोपि साध‚न‚सिद्धिर्युक्तेति व‚क्तुमाह ।
	\pend% ending standard par
      
	  \bigskip
	  \begingroup
	
	    \large
	  
	    \begin{quote}
	  
	    
	    \stanza[\smallbreak]
	\label{pv.4.2}\flagstanza{\tiny\textenglish{...pv.4.2}}अनुमाविष‚ये नेष्टं प‚रीक्षित‚प‚रिग्र‚हात् ।&वाचः प्रामाण्य‚म‚स्मिन् हि नानुमानं प्र‚व‚र्त‚ते ॥ २ ॥\&[\smallbreak]


	
	    \end{quote}
	  
	  \endgroup
	

	  \pstart \leavevmode% starting standard par
	\hphantom{.}‚{\color{DodgerBlue3}‚अनुमान‚स्य} व‚स्तुब‚ल‚प्र‚वृत्त‚स्य ‚{\color{DodgerBlue3}‚विष‚ये} य‚स्माद‚नुमान‚ज्ञान‚मुत्प‚द्य‚ते (।) स च ‚{\tiny $_{lb}$}‚त्रिरूपो हेतुः (।) त‚त्र प्र‚तिपाद्ये ‚{\color{DodgerBlue3}‚वाच} आग‚मात्मिकायाः ‚{\color{DodgerBlue3}‚प्रामाण्यं} नेष्ट‚म‚र्थ‚प्र‚ति‚{\tiny $_{lb}$}‚ब‚न्धाभावादित्युक्तं ।
	\pend% ending standard par
      

	  \pstart \leavevmode% starting standard par
	किञ्च (।) प्र‚माणान्त‚र‚स‚म्वादा‚{\tiny $_{3}$}‚त् ‚{\color{DodgerBlue3}‚प‚रीक्षित‚स्य} प्र‚माणोप‚प‚न्न‚स्याग‚मा‚{\tiny $_{lb}$}‚र्थ‚स्य ‚{\color{DodgerBlue3}‚प‚रिग्र‚हात्} स्वीकारान्नेष्टं व‚च‚न‚प्रामाण्यं । य‚दि व‚च‚न‚मित्येव प्र‚माण‚न्त‚दा ‚{\tiny $_{lb}$}‚प्र‚तिज्ञाप‚दादेव साध्य‚स्य सिद्धेर्निष्फ‚लं हेतुदृष्ट‚न्तादिव‚च‚नं स्यात् । प्र‚माणान्त‚र‚{\tiny $_{lb}$}‚स‚म्वादापेक्षा च न भ‚वेत् । स‚म्वाद‚ज्ञान‚स्यैवार्थ‚भावानुविधायित्वात् त‚स्मि‚{\tiny $_{lb}$}‚न्नाग‚मार्थे प्रामाण्यं । व‚च‚न‚स्य तु विप‚र्य‚यात् । य‚दि त्वाग‚म इत्ये‚{\tiny $_{4}$}‚व प्र‚माणं ‚{\tiny $_{lb}$}‚(।) त‚दा प्र‚माणान्त‚र‚स‚म्वादापेक्षा न स्यात् । हि य‚स्मात् । ‚{\color{DodgerBlue3}‚अस्मिन्ना}‚ग‚मार्थे ‚{\tiny $_{lb}$}‚प्र‚माण‚प्र‚तिपादित‚त्वान्निश्चितेऽ‚{\color{DodgerBlue3}‚नुमानं न प्र‚वृर्त्त‚ते} (।) य‚दि ह्याग‚मार्थः स‚न्दिह्येत ‚{\tiny $_{lb}$}‚त‚दा त‚न्निश्च‚यार्था प्र‚माणान्त‚र‚वृत्तिर‚पेक्ष्येत ॥ (२)
	\pend% ending standard par
      \label{div_pvv.4.3}
	  
	% new div opening: depth here is 2
	
	  \bigskip
	  \begingroup
	
	    \large
	  
	    \begin{quote}
	  
	    
	    \stanza[\smallbreak]
	\label{pv.4.3}\flagstanza{\tiny\textenglish{...pv.4.3}}बाध‚नायाग‚म‚स्योक्तेः साध‚न‚स्य प‚रं प्र‚ति ।&सोप्र‚माणं त‚दाऽसिद्धं त‚त्सिद्ध‚म‚खिल‚न्त‚तः ॥ ३ ॥\&[\smallbreak]


	
	    \end{quote}
	  
	  \endgroup
	

	  \pstart \leavevmode% starting standard par
	\hphantom{.}य‚दि चाग‚{\color{DodgerBlue3}‚म‚स्य} सुखादिचैत‚न्य‚प्र‚तिपाद‚क‚स्य ‚{\color{DodgerBlue3}‚बाध‚नाय साध‚न}‚स्योत्प‚त्ति‚{\tiny $_{lb}$}‚म‚त्वादेः सां ख्ये न ‚{\color{DodgerBlue3}‚प‚रं बौद्धं प्र‚त्युक्तिः} त‚तः कार‚णात् ‚{\color{DodgerBlue3}‚स} आग‚मोऽ‚{\tiny $_{5}$}‚‚{\color{DodgerBlue3}‚प्र‚माणं । त‚दा} । ‚{\tiny $_{lb}$}‚न हि प्र‚माण‚स्य बाधो युक्तः । त‚त आग‚माप्रामाण्यात् तेनाग‚मेन साक्षात् ‚{\tiny $_{lb}$}‚पारंप‚र्याभ्यां ‚{\color{DodgerBlue3}‚सिद्ध}‚मुत्प‚त्तिम‚त्वादि साध्यं चाचैत‚न्य‚{\color{DodgerBlue3}‚म‚खिल‚मिद‚म‚सिद्धं} ॥ (३)
	\pend% ending standard par
      \label{div_pvv.4.4}
	  
	% new div opening: depth here is 2
	

	  \begin{center}%% label @type='head'
	\textbf{ग. प्र‚माणेन बाध्य‚मान‚स्याग‚म‚स्य न सिद्धिः}
	\end{center}
	
	  \bigskip
	  \begingroup
	
	    \large
	  
	    \begin{quote}
	  
	    
	    \stanza[\smallbreak]
	\label{pv.4.4}\flagstanza{\tiny\textenglish{...pv.4.4}}त‚दाग‚म‚व‚तः सिद्धं य‚दि क‚स्य क आग‚मः ।&बाध्य‚मानः प्र‚माणेन स सिद्धः क‚थ‚माग‚मः ॥ ४ ॥\&[\smallbreak]


	
	    \end{quote}
	  
	  \endgroup
	

	  \pstart \leavevmode% starting standard par
	\hphantom{.}स च सुखाद्युत्प‚त्तिम‚त्व‚प्र‚तिपाद‚क आग‚मः (च) ‚{\color{DodgerBlue3}‚त‚दाग‚मः} त‚द्व‚त‚स्त‚त्स‚म्ब‚न्धिनो ‚{\tiny $_{lb}$}‚‚{\color{DodgerBlue3}‚बौद्ध‚स्य सिद्ध}‚मुत्प‚त्तिम‚त्त्वादिलिङ्ग‚मिति चेत् । ‚{\color{DodgerBlue3}‚क‚स्य} पुरुष‚स्य क आग‚मः स‚म्ब‚न्धी । न ‚{\tiny $_{lb}$}‚\leavevmode\ledsidenote{\textenglish{415/s}} ताव‚द‚व‚य‚व‚त् आग‚मोपि पुरुष‚स्य स‚ह‚ज‚स‚म्ब‚न्धेन‚{\tiny $_{6}$}‚ स‚म्ब‚द्धः\edtext{}{\edlabel{pvv.415-1}\label{pvv.415-1}\lemma{द्धः}\Bfootnote{तादात्मा(?त्म्या)भाव उक्तः ।}} । नापि युक्त्युप‚प‚न्न‚य‚तो\edtext{}{\edlabel{pvv.415-2}\label{pvv.415-2}\lemma{तो}\Bfootnote{त‚दुत्प‚त्तेः ।}}‚{\tiny $_{lb}$}‚पाधिना\edtext{}{\edlabel{pvv.415-3}\label{pvv.415-3}\lemma{पाधिना}\Bfootnote{अस्येद‚मिति ।}} प्र‚त्य‚क्षानुमान‚व‚त् स‚म्ब‚न्धः । त‚था हि ‚{\color{DodgerBlue3}‚प्र‚माणे}‚नोत्प‚त्तिम‚त्वादिलिङ्ग‚जेना‚{\tiny $_{lb}$}‚नुमानेनाग‚म‚प्र‚तिपादित‚स्य सुखादिचैत‚न्य‚स्य बाध‚नात् ‚{\color{DodgerBlue3}‚बाध्य‚मान} आग‚मः हेतुः । ‚{\color{DodgerBlue3}‚स ‚{\tiny $_{lb}$}‚क‚थ‚माग‚म‚सिद्धो} येन युक्तिस‚म्वादोपाधिनापि पुरुषेण स‚म्ब‚न्ध‚म‚नुभ‚वेत् । (४)
	\pend% ending standard par
      \label{div_pvv.4.5}
	  
	% new div opening: depth here is 2
	

	  \pstart \leavevmode% starting standard par
	स्यादेत‚त् । सुखादीनाम‚चैत‚न्यं तेनैवाग‚मेन प्र‚देशान्त‚रे द‚र्शित‚म‚तोऽबाध्य‚त्वात् ‚{\tiny $_{lb}$}‚प्र7माणाद‚स्मात् साध‚न‚विधिर्युक्त इत्याह ।\leavevmode\ledsidenote{\textenglish{82a/MA}}
	\pend% ending standard par
      
	  \bigskip
	  \begingroup
	
	    \large
	  
	    \begin{quote}
	  
	    
	    \stanza[\smallbreak]
	\label{pv.4.5}\flagstanza{\tiny\textenglish{...pv.4.5}}त‚द्विरुद्धाभ्युप‚ग‚म‚स्तेनैव च क‚थं भ‚वेत् ।&त‚द‚न्योप‚ग‚मे त‚स्य त्यागांग‚स्याप्र‚माण‚ता ॥ ५ ॥\&[\smallbreak]


	
	    \end{quote}
	  
	  \endgroup
	

	  \pstart \leavevmode% starting standard par
	\hphantom{.}त‚स्माच्चैत‚न्यात् प्र‚तिपादिताद् ‚{\color{DodgerBlue3}‚विरुद्ध}‚स्याचैत‚न्य‚स्या‚{\color{DodgerBlue3}‚भ्युप‚ग‚म‚स्तेन} चैत‚न्य‚प्र‚ति‚{\tiny $_{lb}$}‚पाद‚केनैव चाग‚मेन ‚{\color{DodgerBlue3}‚क‚थ‚म्भ‚वेत्} । स‚म्भ‚वे वा विरुद्धार्थाभिधायित्वेनाप्र‚माणादाग‚मा‚{\tiny $_{lb}$}‚द्धेतुसिद्धिर‚युक्तैव (।) आग‚म‚प्र‚तीतेनोत्प‚त्तिम‚त्त्वादिना । त‚स्मादाग‚मोक्त ‚{\color{DodgerBlue3}‚चैत‚न्या‚{\tiny $_{lb}$}‚द‚न्य}‚स्याचैत‚न्य‚स्यो‚{\color{DodgerBlue3}‚प‚ग‚मे} स्वीकारे वा प्र‚तिवादीना क्रिय‚माणे ‚{\color{DodgerBlue3}‚त‚स्या}‚ग‚म‚स्य‚{\tiny $_{1}$}‚ ‚{\color{DodgerBlue3}‚त्यागां‚{\tiny $_{lb}$}‚ग‚स्याप्र‚माण‚ता}‚ऽभ्युप‚ग‚ता स्यात् (।) य‚दैवाग‚म‚प्रामाण्य‚स्य बाध‚के हेतावाद्रिय‚ते त‚दैव ‚{\tiny $_{lb}$}‚त‚त्र संश‚यितः । न चाप्र‚माणाद्धेतुसिद्धिः ॥ (५)
	\pend% ending standard par
      \label{div_pvv.4.6}
	  
	% new div opening: depth here is 2
	

	  \pstart \leavevmode% starting standard par
	किञ्च (।)
	\pend% ending standard par
      
	  \bigskip
	  \begingroup
	
	    \large
	  
	    \begin{quote}
	  
	    
	    \stanza[\smallbreak]
	\label{pv.4.6}\flagstanza{\tiny\textenglish{...pv.4.6}}त‚त् क‚स्मात् साध‚नं नोक्तं स्व‚प्र‚तोतिर्य‚दुद्भ‚वा ।&युक्त्या य‚याग‚मो ग्राह्यः प‚र‚स्यापि च सा न किम् ॥ ६ ॥\&[\smallbreak]


	
	    \end{quote}
	  
	  \endgroup
	

	  \pstart \leavevmode% starting standard par
	\hphantom{.}अय‚म्वादी ताव‚त् स्व‚य‚म्प्र‚तीत‚मेवाचैत‚न्यं ‚{\color{DodgerBlue3}‚प‚र}\edtext{}{\edlabel{pvv.415-4}\label{pvv.415-4}\lemma{न्यं}\Bfootnote{अज्ञातं म‚मेति व‚च‚ने उप‚ह‚स‚नीयः स्यात् ।}}स्मै प्र‚तिपाद‚य‚ति । ‚{\color{DodgerBlue3}‚व च साध‚न}‚{\tiny $_{lb}$}‚म‚न्त‚रेणैव प्र‚तीतिः । त‚तः प्र‚तिपाद‚यितुः\edtext{}{\edlabel{pvv.415-5}\label{pvv.415-5}\lemma{यितुः}\Bfootnote{सांख्य‚स्य ।}} स्व‚स्य ‚{\color{DodgerBlue3}‚प्र‚तीतिश्चै}‚त‚न्य‚विष‚या य‚स्मात् ‚{\tiny $_{lb}$}‚साध‚नादुद्भ‚वो य‚स्याःसा य‚दुद्भ‚वा । ‚{\color{DodgerBlue3}‚त‚त्साध‚नं} स्वाग‚मादिकं क‚स्मात्\edtext{}{\edlabel{pvv.415-6}\label{pvv.415-6}\lemma{स्मात्}\Bfootnote{त‚दुत्प‚त्तेः ।}} प्र‚तिपाद्यं प्र‚ति ‚{\tiny $_{lb}$}‚‚{\color{DodgerBlue3}‚नोक्तं} । स्व‚यं प्र‚तिप‚न्न‚साम‚र्थ्य‚मेव साध‚नं व‚क्तुमुचितं । त‚था निर्युक्तिक‚स्याग‚म‚स्या‚{\tiny $_{lb}$}‚प्रामाण्यात् ‚{\color{DodgerBlue3}‚य‚या युक्त्यो}‚प‚प‚न्न ‚{\color{DodgerBlue3}‚आग‚मः} साध‚न‚त्वेन ‚{\color{DodgerBlue3}‚गाह्यो} वादिना सा युक्तिः ‚{\color{DodgerBlue3}‚प‚र‚स्य} प्र‚तिपाद्य‚स्यापि किन्न साध‚न‚येन त‚त्प‚रित्य‚ज्यान्य‚द‚प्र‚माण‚सिद्ध‚मुत्प‚त्त्याद्युच्य‚ते । (६)
	\pend% ending standard par
      \label{div_pvv.4.7}
	  
	% new div opening: depth here is 2
	

	  \pstart \leavevmode% starting standard par
	अथ योग‚ब‚ल‚जेन प्र‚त्य‚क्षेण सुखादीनाम‚चैत‚न्यं प्र‚तीतं\edtext{}{\edlabel{pvv.415-7}\label{pvv.415-7}\lemma{तीतं}\Bfootnote{क‚पिलऋषिणा ।}} न त‚द‚न्य‚था प्र‚{\tiny $_{3}$}‚तिपाद‚यितुं ‚{\tiny $_{lb}$}‚प‚र‚स्य श‚क्य‚त इति प‚रोप‚ग‚तं साध‚न‚मुच्य‚ते ॥
	\pend% ending standard par
      \textsuperscript{\textenglish{416/s}}

	  \pstart \leavevmode% starting standard par
	न‚नु त‚थापि नाक‚स्मिको बुद्धिसुखाद्य‚चैत‚न्य‚विष‚य‚स्य निश्च‚यः । त‚त्साक्षा‚{\tiny $_{lb}$}‚त्कारिप्र‚त्य‚क्ष‚त‚त्साध‚न‚योगानुष्ठान‚योः \edtext{}{\edlabel{pvv.416-1}\label{pvv.416-1}\lemma{योः}\Bfootnote{ज्ञात्वा हि लिङ्गैस्त‚दुपायोनुष्ठीय‚तेऽन्य‚था भाव‚नानुप‚प‚त्तेः ।}} साध्य‚साध‚न‚भावाव‚धार‚ण‚ञ्चाव‚श्य‚क‚{\tiny $_{lb}$}‚र्त्त‚व्यं । त‚न्निश्च‚य‚म‚न्त‚रेण योगानुष्ठानायोगात् । त‚था च (।)
	\pend% ending standard par
      
	  \bigskip
	  \begingroup
	
	    \large
	  
	    \begin{quote}
	  
	    
	    \stanza[\smallbreak]
	\label{pv.4.7}\flagstanza{\tiny\textenglish{...pv.4.7}}प्राकृत‚स्य स‚तः प्राग् यैः प्र‚तिप‚त्त्य‚क्ष‚संभ‚वौ ।&साध‚नैः साध‚नान्य‚र्थ‚श‚क्तिज्ञानेस्य तान्य‚ल‚म् ॥ ७ ॥\&[\smallbreak]


	
	    \end{quote}
	  
	  \endgroup
	

	  \pstart \leavevmode% starting standard par
	\hphantom{.}‚{\color{DodgerBlue3}‚प्राकृत}‚स्यार्व्वाग्द‚र्श‚न‚स्य स‚तो यैः ‚{\color{DodgerBlue3}‚साध‚नैः} सुखाचैत‚न्यादिविष‚यं त‚द्ग्राहि ‚{\tiny $_{lb}$}‚‚{\color{DodgerBlue3}‚प्र‚त्य‚क्ष‚{\tiny $_{4}$}‚}‚त‚त्साध‚न‚योः स‚म्ब‚न्ध‚ञ्च निश्चित्य प्र‚वृत्त‚स्योपाये प्र‚तिप‚त्तुर‚नुष्ठान‚स्य । ‚{\tiny $_{lb}$}‚त‚त्फ‚ल‚स्याक्ष‚स्य प्र‚त्य‚क्ष‚स्याचैत‚न्यादिविष‚य‚ग्राहिणः स‚म्भ‚वौ भ‚व‚तः (।) ‚{\color{DodgerBlue3}‚तानि} ‚{\color{DodgerBlue3}‚साध‚नान्य‚र्थ}‚स्योपायाभ्यास‚स्य ‚{\color{DodgerBlue3}‚श‚क्ते}‚र‚चैत‚न्य‚ग्राहिप्र‚त्य‚क्ष‚ज‚निकाया ‚{\color{DodgerBlue3}‚ज्ञाने} क‚र्त्त‚व्ये‚{\tiny $_{lb}$}‚‚{\color{DodgerBlue3}‚ऽस्य} प्र‚तिपाद्य‚स्यालं स‚म‚र्थानि । त‚त‚स्यान्येव साध‚नान्युच्य‚न्तां \edtext{}{\edlabel{pvv.416-2}\label{pvv.416-2}\lemma{न्तां}\Bfootnote{य‚थास्माभिर्द्वितीय‚प‚रिच्छेदे उक्तं प‚रो मोक्षेऽविद्या(स‚त्काय)तृष्णाभ्यांक्षेपि ज‚न्मेति स‚म्ब‚न्धः शून्य‚तादृष्टिस्त‚योः प्र‚तिप‚क्ष इति य‚थेह तृष्ण‚या प्र‚वृत्तिर‚विद्याप्रेर‚ण‚या ।}} किं स्व‚य‚म‚दृष्ट‚{\tiny $_{lb}$}‚साध्य\edtext{}{\edlabel{pvv.416-3}\label{pvv.416-3}\lemma{साध्य}\Bfootnote{य‚त्रैव स्क‚न्धेऽविद्याद‚य‚स्त‚त्रानारोपितेऽविद्याशून्य‚मिति स‚म्ब‚न्धः ।}}प्र‚तिपाद‚न‚{\color{DodgerBlue3}‚साम‚र्थ्येनो‚{\tiny $_{5}$}‚त्प}‚त्तिम‚त्वादिनोक्तेन ॥ (७)
	\pend% ending standard par
      \label{div_pvv.4.8}
	  
	% new div opening: depth here is 2
	

	  \pstart \leavevmode% starting standard par
	स्यादेत‚त् । प्र‚त्य‚क्ष‚स्याचैत‚न्यादिविष‚य‚स्य योगाद्य‚भ्यासेन स‚ह साध्य‚साध‚न‚{\tiny $_{lb}$}‚स‚म्ब‚न्धो विप्र‚कृष्टात्वात् सामान्याकारेणापि न प्र‚तीय‚ते त‚तोस्यानुप‚द‚र्श‚न‚मित्याह ।
	\pend% ending standard par
      
	  \bigskip
	  \begingroup
	
	    \large
	  
	    \begin{quote}
	  
	    
	    \stanza[\smallbreak]
	\label{pv.4.8}\flagstanza{\tiny\textenglish{...pv.4.8}}विच्छिन्नानुग‚मा येपि सामान्येनाप्य‚गोच‚राः ।&साध्य‚साध‚न‚चिन्तास्ति न तेष्व‚र्थेषु काच‚न ॥ ८ ॥\&[\smallbreak]


	
	    \end{quote}
	  
	  \endgroup
	

	  \pstart \leavevmode% starting standard par
	\hphantom{.}‚{\color{DodgerBlue3}‚येपि विच्छिन्नो} विप्र‚कृष्टोऽ‚{\color{DodgerBlue3}‚नुग‚मः} स‚म्ब‚न्धो येषां ते प्र‚त्य‚क्ष‚त‚दुपायाद‚यः ‚{\tiny $_{lb}$}‚‚{\color{DodgerBlue3}‚सामान्ये}‚नाविशेषाकारेणा‚{\color{DodgerBlue3}‚प्य‚गोच‚रास्तेष्व‚र्थेष्विदं} साध‚न‚मिदं साध्य‚मिति‚{\tiny $_{6}$}‚ च ‚{\tiny $_{lb}$}‚‚{\color{DodgerBlue3}‚साध्य‚साध‚न‚चिन्ता काच‚न नास्ति} । त‚तोऽर्व्वाग्द‚र्श‚न‚स्याबुद्धिविष‚यीकृत‚यो रूप‚यो‚{\tiny $_{lb}$}‚रुप‚योपेय‚योरेक‚त्रानुष्ठानाव‚प‚र‚त्र निष्पाद‚न‚मिति कुतः ॥ (८)
	\pend% ending standard par
      \label{div_pvv.4.9}
	  
	% new div opening: depth here is 2
	

	  \pstart \leavevmode% starting standard par
	\hphantom{.}‚{\color{DodgerBlue3}‚किञ्च} (।) अप्र‚माण‚कादाग‚मादिच्छामात्रेण प्र‚तिपाद्यो हेतुं कुर्व्व‚न् पुन‚{\tiny $_{lb}$}‚रिच्छ‚या त‚मेव प‚रिह‚र‚न् हेतुत‚दाभास‚योरिच्छाधीन‚तां स्वीकुर्यादित्याख्यातुमाह ।
	\pend% ending standard par
      
	  \bigskip
	  \begingroup
	
	    \large
	  
	    \begin{quote}
	  
	    
	    \stanza[\smallbreak]
	\label{pv.4.9}\flagstanza{\tiny\textenglish{...pv.4.9}}पुंसाम‚भिप्राय‚व‚शात् त‚त्त्वात‚त्त्व‚व्य‚व‚स्थितौ ।&लुप्तौ हेतुत‚दाभासौ त‚स्य व‚स्त्व‚स‚माश्र‚यात् ॥ ९ ॥\&[\smallbreak]


	
	    \end{quote}
	  
	  \endgroup
	\textsuperscript{\textenglish{82b/MA}}

	  \pstart \leavevmode% starting standard par
	\hphantom{.}‚{\color{DodgerBlue3}‚पुंसाम‚भिप्राय‚व‚शादि}‚च्छानुरोधात् ‚{\color{DodgerBlue3}‚त‚त्त्वा‚{\tiny $_{7}$}‚त‚त्त्व}‚योर्हेतुत‚दाभास‚त्व‚यो‚{\color{DodgerBlue3}‚र्व्य‚व‚स्थि‚{\tiny $_{lb}$}‚ता}‚विष्य‚माणायां ‚{\color{DodgerBlue3}‚हेतुत‚दाभासौ लुप्तौ} स्यातां अव्य‚व‚स्थित‚त्वात् ‚{\color{DodgerBlue3}‚त‚स्य} पुरुषाभिप्रा‚{\tiny $_{lb}$}‚\leavevmode\ledsidenote{\textenglish{417/s}} य‚स्य ‚{\color{DodgerBlue3}‚व‚स्त्व}‚संश्र‚याद् य‚थाव‚स्तुप्र‚वृत्तिनिय‚माभावात् ॥ (९)
	\pend% ending standard par
      \label{div_pvv.4.10}
	  
	% new div opening: depth here is 2
	

	  \pstart \leavevmode% starting standard par
	अपि च (।)
	\pend% ending standard par
      
	  \bigskip
	  \begingroup
	
	    \large
	  
	    \begin{quote}
	  
	    
	    \stanza[\smallbreak]
	\label{pv.4.10}\flagstanza{\tiny\textenglish{...v.4.10}}स‚न्न‚र्थो ज्ञान‚सापेक्षो नास‚न् ज्ञानेन साध‚कः ।&स‚तोपि व‚स्त्व‚संश्लिष्टाऽसंग‚त्या स‚दृशी ग‚तिः ॥ १० ॥\&[\smallbreak]


	
	    \end{quote}
	  
	  \endgroup
	

	  \pstart \leavevmode% starting standard par
	\hphantom{.}‚{\color{DodgerBlue3}‚स‚न्न‚र्थो ज्ञान‚सापेक्षः} साध्य‚{\color{DodgerBlue3}‚साध‚कः} प्र‚तीतो य‚था धूमादिः । ‚{\color{DodgerBlue3}‚नास‚न्न}‚र्थो ‚{\color{DodgerBlue3}‚ज्ञानेन} प्र‚तीतिमात्रेण ‚{\color{DodgerBlue3}‚साध‚कः} साध्य‚स्य य‚था क‚ल्पितो धूमः । अथाचैत‚न्य‚म्व‚स्तुतोस्त्येव ‚{\tiny $_{lb}$}‚त‚द् य‚था क‚थ‚ञ्चित् प‚र‚स्मै‚{\tiny $_{1}$}‚ प्र‚तिपाद‚नीयं । अतः प‚राभ्युप‚ग‚तो हेतुः क्रिय‚त इत्याह । ‚{\tiny $_{lb}$}‚‚{\color{DodgerBlue3}‚स‚तोप्य}‚चैत‚न्य‚स्य ‚{\color{DodgerBlue3}‚व‚स्त्व‚संश्लिष्टा} व‚स्तुभूत‚लिङ्गाप्र‚तिब‚द्धा ग‚तिर‚स‚{\color{DodgerBlue3}‚ङ्ग‚त्या}‚ऽस‚तः ‚{\tiny $_{lb}$}‚प्र‚तीत्या ‚{\color{DodgerBlue3}‚स‚दृशी} स‚म्य‚क् प्र‚तीत‚त्वाभाव‚त् । अन्येत्व‚स‚द्ग‚त्या दोष‚व‚त्प्र‚तीत्या ‚{\tiny $_{lb}$}‚स‚दृशीति व्याच‚क्ष‚ते (।) तेषां स‚तोप्य‚व‚स्तुकृता प्र‚तिप‚त्तिर‚स‚त्प्र‚तिप‚त्तिं नाति‚{\tiny $_{lb}$}‚शेते । अप्र‚त्य‚य‚त्वादिति विनिश्च‚य‚ग्र‚न्थेन स‚ह एक‚वाक्य‚ता न स्यात् ॥ (१०)
	\pend% ending standard par
      \label{div_pvv.4.11}
	  
	% new div opening: depth here is 2
	

	  \pstart \leavevmode% starting standard par
	किञ्च (।)
	\pend% ending standard par
      
	  \bigskip
	  \begingroup
	
	    \large
	  
	    \begin{quote}
	  
	    
	    \stanza[\smallbreak]
	\label{pv.4.11}\flagstanza{\tiny\textenglish{...v.4.11}}लिङ्गं स्व‚भावः कार्यं वा दृश्याद‚र्श‚न‚मेव वा ।&स‚म्ब‚द्धं व‚स्तुत‚स्सिद्धं त‚द‚सिद्धं किमात्म‚नः ॥ ११ ॥\&[\smallbreak]


	
	    \end{quote}
	  
	  \endgroup
	

	  \pstart \leavevmode% starting standard par
	\hphantom{.}लिङ्गं साध्यार्थ‚{\color{DodgerBlue3}‚स‚{\tiny $_{2}$}‚म्ब‚द्धं} नान्य‚त् व्य‚भिचारात् । त‚च्च ‚{\color{DodgerBlue3}‚स्व‚भावः कार्यं ‚{\tiny $_{lb}$}‚दृश्याद‚र्श‚न}‚म‚नुप‚ल‚ब्धिरेवान्य‚स्य प्र‚तिब‚न्धाभावादित्युक्तं । त‚दुत्प‚त्त्यादिकं य‚दि ‚{\tiny $_{lb}$}‚त्रिषु हेतुष्व‚न्त‚र्भूतं त‚दा ‚{\color{DodgerBlue3}‚व‚स्तुतः} प‚र‚मार्थ‚तः ‚{\color{DodgerBlue3}‚सिद्धं} प्र‚तिपाद्य‚स्य ‚{\color{DodgerBlue3}‚किं} क‚स्मा‚{\color{DodgerBlue3}‚दात्म‚नः} सां ख्य स्य वादिनो\edtext{}{\edlabel{pvv.417-1}\label{pvv.417-1}\lemma{वादिनो}\Bfootnote{उभ‚य‚सिद्ध‚म‚स्तु ।}} ‚{\color{DodgerBlue3}‚ऽसिद्धं} । उत्प‚त्तिम‚त्वादिस्व‚भाव‚हेतुः स्यात् । स च ध‚र्मि‚{\tiny $_{lb}$}‚ग्राह‚कात् प्र‚माणाद‚न्य‚तो वा शिंश‚पा\edtext{}{\edlabel{pvv.417-2}\label{pvv.417-2}\lemma{पा}\Bfootnote{वृक्षं ग्राह‚य‚ति ।}}त्व‚व‚त् कृत‚क‚त्वादिव‚द‚र्थान् प्र‚तिवादिन ‚{\tiny $_{lb}$}‚इव वादिनोपि सि३ध्येत् । (११)
	\pend% ending standard par
      \label{div_pvv.4.12}
	  
	% new div opening: depth here is 2
	

	  \pstart \leavevmode% starting standard par
	अथ\edtext{}{\edlabel{pvv.417-3}\label{pvv.417-3}\lemma{अथ}\Bfootnote{प‚राभ्युप‚ग‚तेन दूष‚यित्वा प्र‚माऽन‚न्त‚र्भाव‚माह ।}} त्रिविधे हेतौ नान्त‚र्भ‚व‚ति उत्प‚त्त्यादि त‚दा (।)
	\pend% ending standard par
      
	  \bigskip
	  \begingroup
	
	    \large
	  
	    \begin{quote}
	  
	    
	    \stanza[\smallbreak]
	\label{pv.4.12a}\flagstanza{\tiny\textenglish{....4.12a}}प‚रेणाप्य‚न्य‚तो ग‚न्तुम‚युक्तं;\&[\smallbreak]


	
	    \end{quote}
	  
	  \endgroup
	

	  \pstart \leavevmode% starting standard par
	\hphantom{.}‚{\color{DodgerBlue3}‚प‚रेण} प्र‚तिवादिनापि त्रिविधाद्धेतो‚{\color{DodgerBlue3}‚र‚न्य‚तो} हेतोर‚चैत‚न्यं ‚{\color{DodgerBlue3}‚ग‚न्तुं} प्र‚त्येतु‚{\color{DodgerBlue3}‚म‚युक्तं} (।) ‚{\tiny $_{lb}$}‚य‚दि प‚राभ्युप‚ग‚म‚सिद्ध‚म‚साध‚नं त‚दा प्र‚स‚ङ्ग‚हेतुर‚हेतुः स्यादित्याह ।
	\pend% ending standard par
      
	  \bigskip
	  \begingroup
	
	    \large
	  
	    \begin{quote}
	  
	    
	    \stanza[\smallbreak]
	\label{pv.4.12b}\flagstanza{\tiny\textenglish{....4.12b}}प‚र‚क‚ल्पितैः&प्र‚स‚ङ्गो द्व‚य‚स‚म्ब‚न्धादेकापायेन्य‚हान‚ये ॥ १२ ॥\&[\smallbreak]


	
	    \end{quote}
	  
	  \endgroup
	\textsuperscript{\textenglish{418/s}}

	  \pstart \leavevmode% starting standard par
	\hphantom{.}‚{\color{DodgerBlue3}‚प‚र‚क‚ल्पितैः} साध‚नैः ‚{\color{DodgerBlue3}‚प्र‚स‚ङ्गः} क्रिय‚ते य‚था सामान्य‚स्य प‚रोप‚ग‚तानेक‚वृत्तित्वाद् ‚{\tiny $_{lb}$}‚अनेक‚त्व‚मापाद्य‚ते न त्व‚यं पार‚मार्थिको हेतुस्त्रैरूप्याभावात् । य‚द्य‚यं न हेतुः ‚{\tiny $_{4}$}‚ त‚दा ‚{\tiny $_{lb}$}‚किम‚र्थ‚मुच्य‚त इत्याह । ‚{\color{DodgerBlue3}‚द्व‚योः} साध्य‚साध‚न‚योः ‚{\color{DodgerBlue3}‚स‚म्ब‚न्धाद्} व्याप्य‚व्याप‚क‚भावात् ‚{\tiny $_{lb}$}‚स्फारिता‚{\color{DodgerBlue3}‚देक‚स्य} साध्य‚स्यापायेऽन्य‚स्य साध‚न‚स्य ‚{\color{DodgerBlue3}‚हान}‚ये । य‚था चानेकं\edtext{}{\edlabel{pvv.418-1}\label{pvv.418-1}\lemma{चानेकं}\Bfootnote{प‚क्ष‚ध‚र्मोप‚संहार एषः ।}}सामान्यं ‚{\tiny $_{lb}$}‚त‚स्मान्नानेक‚वृत्तीति विप‚र्य‚य‚प्र‚योगे साध्याभावे साध‚नाभावः क‚थ्य‚ते । प्र‚स‚ङ्ग‚{\tiny $_{lb}$}‚विप‚र्य‚योत्र मौलो हेतुः साध्य‚साध‚न्य‚व्याप्तिग्राह‚क‚प्र‚माण‚स्मार‚क‚स्तु प्र‚स‚ङ्गे प्र‚योग ‚{\tiny $_{lb}$}‚इत्य‚र्थः । भिन्न‚देश‚{\tiny $_{5}$}‚कालादिष्व‚नेकासु व्य‚क्तिषु वृत्त‚स्य त‚द‚त‚द्देश‚त्वादिविरुद्ध‚ध‚र्मा‚{\tiny $_{lb}$}‚ध्यासाद‚नेक‚त्व‚सिद्धेर‚नेक‚वृत्त‚त्वानेक‚त्व‚योर्व्याप्तिसिद्धिर्बोद्ध‚व्या । (१)
	\pend% ending standard par
      \label{div_pvv.4.13}
	  
	% new div opening: depth here is 2
	

	  \pstart \leavevmode% starting standard par
	उक्तं स्व‚दृष्ट‚ग्र‚ह‚ण‚स्य साफ‚ल्यं \edtext{}{\edlabel{pvv.418-2}\label{pvv.418-2}\lemma{ल्यं}\Bfootnote{उभ‚य‚सिद्धो हेतुः सूचितः प्र‚मोप‚प‚त्तिः ।}} ॥
	\pend% ending standard par
      

	  \begin{center}%% label @type='head'
	\textbf{(२) अर्थ‚ग्र‚ह‚ण‚फ‚ल‚म्}
	\end{center}
	

	  \pstart \leavevmode% starting standard par
	अर्थ‚ग्र‚ह‚ण‚स्येदानीं व‚क्त‚व्यं । अदृष्टार्थ‚प्र‚काश‚न‚मित्य‚त्र सूत्रे य‚दुपात्तं (।)
	\pend% ending standard par
      
	  \bigskip
	  \begingroup
	
	    \large
	  
	    \begin{quote}
	  
	    
	    \stanza[\smallbreak]
	\label{pv.4.13}\flagstanza{\tiny\textenglish{...v.4.13}}त‚द‚र्थ‚ग्र‚ह‚णं श‚ब्द‚क‚ल्प‚नारोपितात्म‚नाम् ।&अलिङ्ग‚त्व‚प्र‚सिद्ध्य‚र्थ‚म‚र्थाद‚र्थ‚स्य सिद्धितः ॥ १३ ॥\&[\smallbreak]


	
	    \end{quote}
	  
	  \endgroup
	

	  \pstart \leavevmode% starting standard par
	\hphantom{.}‚{\color{DodgerBlue3}‚त‚द‚र्थ‚ग्र‚ह‚णं श‚ब्देन \edtext{}{\edlabel{pvv.418-3}\label{pvv.418-3}\lemma{ब्देन}\Bfootnote{य‚था स‚र्व्व‚ग‚त आत्मा स‚र्व्व‚त्रोप‚ल‚भ्य‚मान‚गुण‚त्वादाकाश‚व‚त् ॥ नित्योऽनित्यो वा श‚ब्दः प‚क्ष‚स‚प‚क्षान्य‚त‚र‚त्वात्प‚र‚माणुव‚द् घ‚ट‚व‚द्वा ।}} क‚ल्प‚न‚या चारोपित आत्मा} येषां प‚क्ष‚स‚प‚क्षान्य‚त‚र‚त्वा‚{\tiny $_{lb}$}‚दीनां ‚{\color{DodgerBlue3}‚तेषाम‚लिङ्ग‚त्व‚प्र‚सिद्ध्य‚र्थं} बोद्ध‚व्यं । क‚स्मात् पुनः क‚ल्पित‚स्या‚{\tiny $_{6}$}‚लिङ्ग‚त्व‚{\tiny $_{lb}$}‚मित्याह । ‚{\color{DodgerBlue3}‚अर्थाद्} व‚स्तुभूताल्लिङ्गाद‚र्थ‚स्य साध्य‚स्य ‚{\color{DodgerBlue3}‚सिद्धितः} ॥ (१)
	\pend% ending standard par
      \label{div_pvv.4.14}
	  
	% new div opening: depth here is 2
	
	  \bigskip
	  \begingroup
	
	    \large
	  
	    \begin{quote}
	  
	    
	    \stanza[\smallbreak]
	\label{pv.4.14}\flagstanza{\tiny\textenglish{...v.4.14}}क‚ल्प‚नाग‚म‚योः क‚र्त्तुरिच्छामात्रानुवृत्तितः ।&व‚स्तुन‚श्चान्य‚थाभावात् त‚त्कृता व्य‚भिचारिणः ॥ १४ ॥\&[\smallbreak]


	
	    \end{quote}
	  
	  \endgroup
	

	  \pstart \leavevmode% starting standard par
	\hphantom{.}‚{\color{DodgerBlue3}‚क‚ल्प‚नाया आग‚म‚स्य} श‚ब्द‚स्य च ‚{\color{DodgerBlue3}‚क‚र्त्तुः} पुरुष‚{\color{DodgerBlue3}‚स्येच्छामात्र}‚स्या‚{\color{DodgerBlue3}‚नुवृत्तितो}‚नुरोधा‚{\tiny $_{lb}$}‚त् । ‚{\color{DodgerBlue3}‚व‚स्तुन‚श्चान्य‚थाभावात्} । क‚र्त्तुरिच्छानुवृत्ते‚{\color{DodgerBlue3}‚स्ताभ्यां} श‚ब्द‚क‚ल्प‚नाभ्यां ‚{\color{DodgerBlue3}‚कृता} हेत‚वो ‚{\color{DodgerBlue3}‚व्य‚भिचारिणो} नैकान्तिकाः । (१४)
	\pend% ending standard par
      

	  \pstart \leavevmode% starting standard par
	उक्त‚म‚र्थ‚ग्र‚ह‚ण‚प्र‚योज‚नं ॥
	\pend% ending standard par
      

	  \pstart \leavevmode% starting standard par
	स्व‚दृष्टार्थ‚प्र‚काश‚न‚श‚ब्देन त्रिरूप‚लिङ्ग‚व‚च‚न‚मिष्टं । न प‚क्ष‚व‚च‚न‚म‚पीति व‚क्त‚व्यं ॥
	\pend% ending standard par
      
	  
	% new div opening: depth here is 1
	
\chapter*[{२. प‚क्ष‚चिन्ता}]{२. प‚क्ष‚चिन्ता}\label{div_pvv.4.15}
	  
	% new div opening: depth here is 2
	\textsuperscript{\textenglish{419/s}}

	  \begin{center}%% label @type='head'
	\textbf{(२) प‚क्ष‚हेतुव‚च‚न‚म‚साध‚न‚म्}
	\end{center}
	

	  \begin{center}%% label @type='head'
	\textbf{क. हेतुव‚च‚नं न साध‚न‚म्}
	\end{center}
	

	  \pstart \leavevmode% starting standard par
	य‚दि सा‚{\tiny $_{7}$}‚क्षात् पार‚म्प‚र्येण वा प‚क्ष‚व‚च‚नं साध्य‚प्र‚त्तिप‚त्तावुप‚युज्य‚ते । त‚दो-\leavevmode\ledsidenote{\textenglish{83a/MA}} ‚{\tiny $_{lb}$}‚च्येत किन्त्वेत‚न्नास्तीत्याह ।
	\pend% ending standard par
      
	  \bigskip
	  \begingroup
	
	    \large
	  
	    \begin{quote}
	  
	    
	    \stanza[\smallbreak]
	\label{pv.4.15}\flagstanza{\tiny\textenglish{...v.4.15}}अर्थाद‚र्थ‚ग‚तेः श‚क्तिः प‚क्ष‚हेत्व‚भिधान‚योः ।&नार्थे तेन त‚योर्न्नास्ति स्व‚तः साध‚न‚संस्थिति ॥ १५ ॥\&[\smallbreak]


	
	    \end{quote}
	  
	  \endgroup
	

	  \pstart \leavevmode% starting standard par
	\edtext{\textsuperscript{*}}{\edlabel{pvv.419-1}\label{pvv.419-1}\lemma{*}\Bfootnote{प‚ञ्चाव‚य‚व‚त्वात् प‚रे प‚क्ष‚हेतुव‚च‚न‚योः साध‚न‚त्व‚माहुः । त‚न्निषेधायाह । साध‚नं भ‚व‚त् साक्षात् पारंप‚र्येण वा स्यात् त‚त्र प्र‚तिज्ञाहेतुदृष्टान्तउप‚न‚य‚निग‚म‚नाख्यं ।}} साक्षात् ताव‚त प‚क्षाभिधान‚स्य हेत्व‚भिधान‚स्य च ‚{\color{DodgerBlue3}‚प्र‚तिपाद्येऽर्थे श‚क्तिर्न} विद्य‚ते (।) किं कार‚ण‚मित्याह ।
	\pend% ending standard par
      

	  \pstart \leavevmode% starting standard par
	\hphantom{.}‚{\color{DodgerBlue3}‚अर्थाद्} व‚च‚न‚प्र‚तिपाद्या‚{\color{DodgerBlue3}‚द‚र्थ} साध्य‚स्य ‚{\color{DodgerBlue3}‚ग‚तेर्न} व‚च‚नात् । ‚{\color{DodgerBlue3}‚तेन} साक्षाद‚र्थ‚प्र‚तिपाद‚{\tiny $_{lb}$}‚क‚त्वाभावेन त‚योः ‚{\color{DodgerBlue3}‚प‚क्ष‚हेत्व‚भिधान‚योः स्व‚तः} स्व‚रूपेण ‚{\color{DodgerBlue3}‚साध‚न‚संस्थितिः} । साध‚न‚त्व‚{\tiny $_{lb}$}‚व्य‚व‚स्था ‚{\color{DodgerBlue3}‚नास्ति} \edtext{\textsuperscript{*}}{\edlabel{pvv.419-2}\label{pvv.419-2}\lemma{*}\Bfootnote{न‚नु आचार्येण शाब्दं प्र‚माण‚मिष्टं क‚थ‚न्त‚तो नार्थ इत्याह ।}}य‚त‚{\tiny $_{1}$}‚श्च प‚क्ष‚व‚च‚नं साक्षाद‚र्थे न प्र‚माणं ॥ (१५)
	\pend% ending standard par
      \label{div_pvv.4.16}
	  
	% new div opening: depth here is 2
	

	  \begin{center}%% label @type='head'
	\textbf{ख. प‚क्ष‚व‚च‚न‚म‚साध‚न‚म्}
	\end{center}
	
	  \bigskip
	  \begingroup
	
	    \large
	  
	    \begin{quote}
	  
	    
	    \stanza[\smallbreak]
	\label{pv.4.16}\flagstanza{\tiny\textenglish{...v.4.16}}त‚त् प‚क्ष‚व‚च‚नं व‚क्तुर‚भिप्राय‚निवेद‚ने ।&प्र‚माणं संश‚योत्प‚त्तेस्त‚तः साक्षान्न साध‚न‚म् ॥ १६ ॥\&[\smallbreak]


	
	    \end{quote}
	  
	  \endgroup
	

	  \pstart \leavevmode% starting standard par
	\hphantom{.}त‚त्त‚स्मात् ‚{\color{DodgerBlue3}‚प‚क्ष‚व‚च‚नं} ‚{\color{DodgerBlue3}‚व‚क्तुर‚भिप्राय‚निवेद‚ने प्र‚माणं} श‚ब्द‚प्रामाण्य‚मा चा र्य स्य ‚{\tiny $_{lb}$}‚व‚द‚तोऽभिम‚त‚मिति बोद्ध‚व्यं । त‚त्प‚क्ष‚व‚च‚नात् साध्येर्थ ‚{\color{DodgerBlue3}‚संश‚योत्प‚त्तेर‚निश्च‚यान्न साक्षात्} साध‚न‚म‚र्थ‚स्य त‚त् ॥ (१६)
	\pend% ending standard par
      \label{div_pvv.4.17}
	  
	% new div opening: depth here is 2
	

	  \pstart \leavevmode% starting standard par
	प‚रंप‚र‚या साध्य‚साध‚नात् प्र‚माणं प‚क्ष‚व‚च‚नं व‚क्तुर‚भिप्राय‚निवेद‚ने प्र‚माण‚न्त‚र्हि ‚{\tiny $_{lb}$}‚स्यादित्याह ।
	\pend% ending standard par
      
	  \bigskip
	  \begingroup
	
	    \large
	  
	    \begin{quote}
	  
	    
	    \stanza[\smallbreak]
	\label{pv.4.17}\flagstanza{\tiny\textenglish{...v.4.17}}साध्य‚स्यैवाभिधानेन पारंप‚र्येण नाप्य‚ल‚म् ।&श‚क्त‚स्य सूच‚कं हेतुव‚चोऽश‚क्त‚म‚पि स्व‚य‚म् ॥ १७ ॥\&[\smallbreak]


	
	    \end{quote}
	  
	  \endgroup
	\textsuperscript{\textenglish{420/s}}

	  \pstart \leavevmode% starting standard par
	\hphantom{.}‚{\color{DodgerBlue3}‚पार‚म्प‚र्येणापि} प‚क्ष‚{\tiny $_{2}$}‚व‚च‚न‚{\color{DodgerBlue3}‚म‚लं} स‚म‚र्थं ‚{\color{DodgerBlue3}‚न} साध्य‚सिद्धौ । ‚{\color{DodgerBlue3}‚साध्य‚स्यैव} केव‚ल‚{\tiny $_{lb}$}‚‚{\color{DodgerBlue3}‚स्याभिधानात्} । न हि प‚क्ष‚व‚च‚सा साध‚कं किञ्चिदुच्य‚ते साध्य‚मात्र‚स्यैवाभिधा‚{\tiny $_{lb}$}‚नात् । \edtext{\textsuperscript{*}}{\edlabel{pvv.420-1}\label{pvv.420-1}\lemma{*}\Bfootnote{हेतुव‚चोव‚त् पारंप‚र्येण स्यादौप‚चारिकं साध‚नं त‚त्तु साध्य‚स्यासिद्ध‚स्याभिधाय‚क‚मिति नोप‚च‚रितोपि साध‚नं उक्त‚ञ्च ।  --- त‚त्रानुमेय‚निर्देशो हेत्व‚र्थ‚विष‚यो म‚त इति ।}}त्रिरूप‚स्य ‚{\color{DodgerBlue3}‚हेतोर्व्व‚चो} व‚च‚न‚न्तु ‚{\color{DodgerBlue3}‚स्व‚यं} साक्षात् सिद्धाव‚{\color{DodgerBlue3}‚श‚क्त‚म‚पि} (।) ‚{\tiny $_{lb}$}‚‚{\color{DodgerBlue3}‚श‚क्त‚स्य} त्रिरूप‚लिङ्ग‚स्य ‚{\color{DodgerBlue3}‚सूच‚कं} प्र‚तिपाद‚क‚मिति साध‚न‚मुचितं ॥ (१७)
	\pend% ending standard par
      \label{div_pvv.4.18}
	  
	% new div opening: depth here is 2
	

	  \begin{center}%% label @type='head'
	\textbf{ग. (दिग्नाग‚स्य) प‚क्ष‚व‚च‚न‚म‚साध‚न‚त्वेनेष्ट‚म्}
	\end{center}
	

	  \pstart \leavevmode% starting standard par
	न‚न्वाचार्य‚स्य प‚क्ष‚व‚च‚न‚म‚साध‚न‚त्वेनेष्ट‚मिति क‚थं ज्ञाय‚त इत्याह ।
	\pend% ending standard par
      
	  \bigskip
	  \begingroup
	
	    \large
	  
	    \begin{quote}
	  
	    
	    \stanza[\smallbreak]
	\label{pv.4.18}\flagstanza{\tiny\textenglish{...v.4.18}}हेत्व‚र्थ‚विष‚य‚त्वेन त‚द‚श‚क्तोक्तिरीरिता ।&श‚क्तिस्त‚स्यापि चेद्धेतुव‚च‚न‚स्य प्र‚वृर्त्त‚नात् ॥ १८ ॥\&[\smallbreak]


	
	    \end{quote}
	  
	  \endgroup
	

	  \pstart \leavevmode% starting standard par
	हेतोर‚र्थः सा‚{\tiny $_{3}$}‚ध्यः स विष‚योस्येति ‚{\color{DodgerBlue3}‚हेत्व‚र्थ‚विष‚यः} त‚त्वेन साध्यार्थोप‚द‚र्श‚क‚त्वेन ‚{\tiny $_{lb}$}‚त‚स्य प‚क्ष‚व‚च‚न‚स्य साध्य‚साध‚नं प्र‚त्य‚{\color{DodgerBlue3}‚श‚क्त}‚स्यो‚{\color{DodgerBlue3}‚क्तिरीरिता} निर्दिष्टाचार्येण (।) ‚{\tiny $_{lb}$}‚त‚त्रानुमेय‚निर्देशो हेत्व‚र्थ‚विष‚यो म‚त इत्य‚नेन ग्र‚न्थेन । त‚तो ज्ञाय‚ते प‚क्ष‚व‚च‚न‚{\tiny $_{lb}$}‚म‚साध‚न‚मिष्ट‚माचार्य‚स्येति ।
	\pend% ending standard par
      

	  \pstart \leavevmode% starting standard par
	\hphantom{.}न‚नु ‚{\color{DodgerBlue3}‚त‚स्य} प‚क्ष‚व‚च‚न‚स्यापि साध्य‚सिद्धौ ‚{\color{DodgerBlue3}‚श‚क्तिर‚स्ति} । त‚त्साध‚क‚स्य ‚{\color{DodgerBlue3}‚हेतुव‚च‚न‚स्य} ‚{\color{DodgerBlue3}‚प्र‚व‚{\tiny $_{4}$}‚र्त्त‚नात्} । न‚ह्य‚नुद्दिष्टेर्थे साध‚न‚प्र‚स्तावः । त‚तः साध‚न‚प्र‚स्ताव‚नाहेतुत्वेन ‚{\tiny $_{lb}$}‚प‚क्ष‚व‚च‚न‚स्य साध‚क‚त्व‚म‚स्तीति चेत् । एवं (। १८)
	\pend% ending standard par
      \label{div_pvv.4.19}
	  
	% new div opening: depth here is 2
	
	  \bigskip
	  \begingroup
	
	    \large
	  
	    \begin{quote}
	  
	    
	    \stanza[\smallbreak]
	\label{pv.4.19a}\flagstanza{\tiny\textenglish{....4.19a}}त‚त्संश‚येन जिज्ञासोर्भ‚वेत् प्र‚क‚र‚णाश्र‚यः ।\&[\smallbreak]


	
	    \end{quote}
	  
	  \endgroup
	

	  \pstart \leavevmode% starting standard par
	\hphantom{.}त‚स्य साध्य‚{\color{DodgerBlue3}‚संश‚येन} जिज्ञासा त‚स्याञ्च स‚त्यां साध‚न‚मुच्य‚त इति\edtext{}{\edlabel{pvv.420-2}\label{pvv.420-2}\lemma{इति}\Bfootnote{प‚र‚स्य द‚शाव‚य‚वं वाक्यं । त‚त्र जिज्ञासासंश‚य‚प्र‚योज‚न‚श‚क्य‚प्राप्तिसंश‚य‚ष्युदासाः प‚ञ्च प्र‚वृत्य‚ङ्गानि । प्र‚तिज्ञादिप‚ञ्चाव‚य‚वाः । प्र‚वृत्त्य‚ङ्गैर‚नेकान्त‚माह साध‚न‚व‚च‚नाश्र‚व‚त्वेन संश‚यादीनाम‚पि साध‚न‚त्वं स्यात् ।}} ‚{\color{DodgerBlue3}‚जिज्ञासोः} ‚{\color{DodgerBlue3}‚पुंसः संश‚यो} जिज्ञासा च ‚{\color{DodgerBlue3}‚प्र‚क‚र‚ण‚स्य} साध‚न‚प्र‚स्ताव‚स्या‚{\color{DodgerBlue3}‚श्र‚यो} निमित्त‚मिति ‚{\color{DodgerBlue3}‚भ‚वेत्} साध‚नं प‚क्ष‚व‚च‚न‚व‚त् ॥
	\pend% ending standard par
      

	  \pstart \leavevmode% starting standard par
	न‚नु संश‚य‚जिज्ञासे प्र‚तिपाद्य‚प्र‚व‚र्तिते‚{\tiny $_{5}$}‚ प‚क्ष‚व‚च‚न‚न्तु वादिप्र‚व‚र्तितं त‚त्क‚थं ‚{\tiny $_{lb}$}‚त‚त्स‚मुदाय‚स्य साध‚न‚स्य वादिना निर्देश‚संभ‚व इत्याह ।
	\pend% ending standard par
      \textsuperscript{\textenglish{421/s}}
	  \bigskip
	  \begingroup
	
	    \large
	  
	    \begin{quote}
	  
	    
	    \stanza[\smallbreak]
	\label{pv.4.19b}\flagstanza{\tiny\textenglish{....4.19b}}विप‚क्षोप‚ग‚मेप्येत‚त् तुल्य‚मित्य‚न‚व‚स्थितिः ॥ १९ ॥\&[\smallbreak]


	
	    \end{quote}
	  
	  \endgroup
	

	  \pstart \leavevmode% starting standard par
	\hphantom{.}एव‚न्त‚र्हि ‚{\color{DodgerBlue3}‚विप‚क्ष}‚स्य साध्य‚विरुद्ध‚स्य ध‚र्म‚{\color{DodgerBlue3}‚स्योप‚ग‚मे\edtext{}{\edlabel{pvv.421-1}\label{pvv.421-1}\lemma{मे}\Bfootnote{वादीना ।}}} प‚राभिप्रायेण ‚{\color{DodgerBlue3}‚नित्य‚श‚ब्द} इति साध्य‚निर्देश‚कृते पार्श्व‚स्थानां त‚था संश‚य‚निरासार्थं य‚त्कृत‚कं त‚द‚नित्यं य‚था ‚{\tiny $_{lb}$}‚घ‚टः कृत‚क‚श्च श‚ब्द इति पुन‚र्वादिनैवोच्य‚ते । त‚दा नित्य‚त्व‚प्र‚तिज्ञायास्त‚द‚नित्य‚त्वा‚{\tiny $_{lb}$}‚व्य‚भिचारिकृत‚क‚{\tiny $_{6}$}‚त्व‚हेतुप्र‚व‚र्त्त‚क‚त्वं ‚{\color{DodgerBlue3}‚तुल्य‚मिति} नित्य‚त्व‚प्र‚तिज्ञाप्य‚नित्य‚प्र‚तिज्ञाव‚त्\leavevmode\ledsidenote{\textenglish{83b/MA}} ‚{\tiny $_{lb}$}‚साध‚नं स्या‚{\color{DodgerBlue3}‚दित्य‚न‚व‚स्थितिः} साध‚नाव‚य‚वानां\edtext{}{\edlabel{pvv.421-2}\label{pvv.421-2}\lemma{वानां}\Bfootnote{न चैवं त‚स्मान्न हेतुप्र‚व‚र्त्त‚क‚त्वेन प्र‚तिज्ञायाः साध‚न‚त्वं ।}}॥ (१९)
	\pend% ending standard par
      \label{div_pvv.4.20}
	  
	% new div opening: depth here is 2
	

	  \pstart \leavevmode% starting standard par
	\hphantom{.}य‚त‚श्च प‚क्ष‚व‚च‚न‚स्य साध्य‚सिद्धौ साक्षात् पार‚म्प‚र्येण वा ‚{\color{DodgerBlue3}‚साम‚र्थ्यं नास्ति त‚तः(।)}
	\pend% ending standard par
      
	  \bigskip
	  \begingroup
	
	    \large
	  
	    \begin{quote}
	  
	    
	    \stanza[\smallbreak]
	\label{pv.4.20}\flagstanza{\tiny\textenglish{...v.4.20}}अन्त‚र‚ङ्गं तु साम‚र्थ्यं त्रिषु रूपेषु संस्थित‚म् ।&त‚त्र स्मृतिस‚माधानं त‚द्व‚च‚स्येव संस्थित‚म् ॥ २० ॥\&[\smallbreak]


	
	    \end{quote}
	  
	  \endgroup
	

	  \pstart \leavevmode% starting standard par
	\hphantom{.}अन्त‚र‚ङ्गं ‚{\color{DodgerBlue3}‚साम‚र्थ्यं तु} श‚ब्दोऽव‚धार‚णे भिन्न‚क्र‚म‚श्चेति । ‚{\color{DodgerBlue3}‚त्रिष्वेव} प‚क्ष‚ध‚र्म‚तादिषु ‚{\tiny $_{lb}$}‚रूपेषु संस्थितं । ‚{\color{DodgerBlue3}‚त‚त्र} त्रिरूप‚लिङ्गे साध्य‚साध‚न‚श‚क्ति‚{\color{DodgerBlue3}‚स्मृतेः स‚मा\edtext{}{\edlabel{pvv.421-3}\label{pvv.421-3}\lemma{मा}\Bfootnote{उत्पाद‚नं ।}}धान‚मारोप‚णं‚{\tiny $_{1}$}‚ ‚{\tiny $_{lb}$}‚त‚द्व‚च‚सि} त्रिरूप‚लिङ्ग‚प्र‚तिपाद‚क‚व‚च‚न ‚{\color{DodgerBlue3}‚एव संस्थितं} । अत‚स्त‚देव पार‚म्प‚र्येण साध्य‚{\tiny $_{lb}$}‚सिद्धेर‚ङ्ग‚त्वात् प्र‚माण‚न्न प‚क्ष‚व‚च‚नं ॥ (२०)
	\pend% ending standard par
      \label{div_pvv.4.21_4.22}
	  
	% new div opening: depth here is 2
	

	  \pstart \leavevmode% starting standard par
	न‚नु (।)
	\pend% ending standard par
      
	  \bigskip
	  \begingroup
	
	    \large
	  
	    \begin{quote}
	  
	    
	    \stanza[\smallbreak]
	\label{pv.4.21}\flagstanza{\tiny\textenglish{...v.4.21}}अख्यापिते हि विष‚ये हेतुवृत्तेर‚संभ‚वात् ।&विष‚य‚ख्याप‚नादेव सिद्धौ चेत्त‚स्य श‚क्त‚ता ॥ २१ ॥\&[\smallbreak]


	
	    \end{quote}
	  
	  \endgroup
	
	  \bigskip
	  \begingroup
	
	    \large
	  
	    \begin{quote}
	  
	    
	    \stanza[\smallbreak]
	\label{pv.4.22a}\flagstanza{\tiny\textenglish{....4.22a}}उक्त‚म‚त्र;\&[\smallbreak]


	
	    \end{quote}
	  
	  \endgroup
	

	  \pstart \leavevmode% starting standard par
	\hphantom{.}‚{\color{DodgerBlue3}‚अख्यापिते}‚ऽप्र‚तिपादिते साध‚न‚स्य ‚{\color{DodgerBlue3}‚विष‚ये} साध्ये ‚{\color{DodgerBlue3}‚हेतोर्वृत्ते}‚रेव ‚{\color{DodgerBlue3}‚ह्य‚स‚म्भ‚वात्} त‚स्य ‚{\tiny $_{lb}$}‚प‚क्ष‚व‚च‚न‚स्य ‚{\color{DodgerBlue3}‚विष‚य‚ख्याप‚नादेव} साध्य‚स्य सिद्धौ पार‚म्प‚र्येण श‚क्त‚तेति ‚{\color{DodgerBlue3}‚चेत्} । (२१) ‚{\tiny $_{lb}$}‚उक्त‚म‚त्र संश‚य‚जिज्ञास‚योर‚पि साध‚न‚प्र‚वृर्त्त‚क‚त्वात्\edtext{}{\edlabel{pvv.421-2-bis}\label{pvv.421-2-bis}\lemma{त्वात्}\Bfootnote{न चैवं त‚स्मान्न हेतुप्र‚व‚र्त्त‚क‚त्वेन प्र‚तिज्ञायाः साध‚न‚त्वं ।}} साध‚न‚त्व‚प्र‚स‚ङ्ग इति ।
	\pend% ending standard par
      

	  \pstart \leavevmode% starting standard par
	किञ्च (।)
	\pend% ending standard par
      
	  \bigskip
	  \begingroup
	
	    \large
	  
	    \begin{quote}
	  
	    
	    \stanza[\smallbreak]
	\label{pv.4.22b}\flagstanza{\tiny\textenglish{....4.22b}}विनाप्य‚स्मात् कृत‚कः श‚ब्द ईदृशाः ।&स‚र्वेऽनित्या इति प्रोक्तेप्य‚र्थात् त‚न्नाश‚धोर्भ‚वेत् ॥ २२ ॥\&[\smallbreak]


	
	    \end{quote}
	  
	  \endgroup
	

	  \pstart \leavevmode% starting standard par
	\hphantom{.}‚{\color{DodgerBlue3}‚अस्मात्} प‚क्ष‚व‚च‚नाद् ‚{\color{DodgerBlue3}‚विनापि कृत‚कः श‚ब्द ईद‚शा} ये कृत‚कास्ते ‚{\color{DodgerBlue3}‚स‚र्व्वेऽनित्या ‚{\tiny $_{lb}$}‚इति} प‚क्ष‚ध‚र्म‚ताव्याप्तिव‚च‚ने ‚{\color{DodgerBlue3}‚प्रोक्तेप्य‚र्थात्} त‚स्य श‚ब्द‚स्य ‚{\color{DodgerBlue3}‚नाश‚धीर्भ‚वेत्} । अनित्य‚{\tiny $_{lb}$}‚\leavevmode\ledsidenote{\textenglish{422/s}} त्वाव्य‚भिचारि कृत‚क‚त्वं श‚ब्दे व‚र्त्त‚मान‚म‚नित्य‚तां त‚त्र ग‚म‚य‚त्येवेति निष्फ‚लं \edtext{}{\edlabel{pvv.422-1}\label{pvv.422-1}\lemma{लं}\Bfootnote{द्वितीय‚कृत‚क‚श‚ब्द‚व‚दुच्य‚माने निग्र‚ह‚स्थानं ।}} ‚{\tiny $_{lb}$}‚‚{\color{DodgerBlue3}‚प‚क्ष‚व‚च‚नं} । (२२)
	\pend% ending standard par
      \label{div_pvv.4.23}
	  
	% new div opening: depth here is 2
	

	  \begin{center}%% label @type='head'
	\textbf{(२) प्र‚तिज्ञा न साध‚नाव‚य‚वः}
	\end{center}
	

	  \pstart \leavevmode% starting standard par
	त‚स्मात् (।)
	\pend% ending standard par
      
	  \bigskip
	  \begingroup
	
	    \large
	  
	    \begin{quote}
	  
	    
	    \stanza[\smallbreak]
	\label{pv.4.23}\flagstanza{\tiny\textenglish{...v.4.23}}अनुक्ताव‚पि प‚क्ष‚स्य सिद्धेर‚प्र‚तिब‚न्ध‚तः ।&त्रिष्व‚न्य‚त‚म‚रूप‚स्यैवानुक्तिर्न्यून‚तोदिता ॥ २३ ॥\&[\smallbreak]


	
	    \end{quote}
	  
	  \endgroup
	

	  \pstart \leavevmode% starting standard par
	\hphantom{.}‚{\color{DodgerBlue3}‚प‚क्ष‚स्यानुक्ताव‚पि साध्य‚सिद्धेर‚प्र‚तिब‚न्ध‚तः} अविरोधात् ‚{\color{DodgerBlue3}‚त्रि‚{\tiny $_{3}$}‚} प‚क्ष‚ध‚र्म‚तादिषु ‚{\tiny $_{lb}$}‚‚{\color{DodgerBlue3}‚रूपेष्व‚न्य‚त‚म‚स्य} एक‚स्या‚{\color{DodgerBlue3}‚नुक्तिर्न्यून‚तोदिता} साध‚न‚दोषो न तु प‚क्षाद्य‚व‚च‚नं । (२३)
	\pend% ending standard par
      \label{div_pvv.4.24}
	  
	% new div opening: depth here is 2
	

	  \pstart \leavevmode% starting standard par
	य‚दि \edtext{}{\edlabel{pvv.422-2}\label{pvv.422-2}\lemma{दि}\Bfootnote{प‚क्ष‚स्यासाध‚न‚त्व गुण‚माह ।}} च प्र‚तिज्ञा साध‚न‚मिष्य‚ते त‚दा साध्य‚निर्देशः प्र‚तिज्ञेति प्र‚तिज्ञाल‚क्ष‚ण‚{\tiny $_{lb}$}‚म‚तिव्यापि स्यात् । असिद्ध‚स्य हेतोर्दृष्टान्त‚स्य चासिद्ध‚स्य साध्य‚त्वं साध‚न‚त्व‚ञ्चा‚{\tiny $_{lb}$}‚स्तीति प्र‚तिज्ञार्थं स्यात् । य‚स्य तु म‚ते प्र‚तिज्ञा न साध‚नं (।)
	\pend% ending standard par
      
	  \bigskip
	  \begingroup
	
	    \large
	  
	    \begin{quote}
	  
	    
	    \stanza[\smallbreak]
	\label{pv.4.24}\flagstanza{\tiny\textenglish{...v.4.24}}साध्योक्तिं वा प्र‚तिज्ञां स व‚द‚न् दोषैर्न युज्य‚ते ।&साध‚नाधिकृतेरेव हेत्वाभासाप्र‚स‚ङ्ग‚तः ॥ २४ ॥\&[\smallbreak]


	
	    \end{quote}
	  
	  \endgroup
	

	  \pstart \leavevmode% starting standard par
	\hphantom{.}‚{\color{DodgerBlue3}‚स साध्योक्तिं} साध्य‚निर्देशं ‚{\color{DodgerBlue3}‚प्र‚तिज्ञां व‚द‚न्न}‚न‚{\tiny $_{4}$}‚न्त‚रोक्तै‚{\color{DodgerBlue3}‚र्दोषैर्न युज्य‚ते । साध‚ना‚{\tiny $_{lb}$}‚धिकृतेः} साध‚न‚त्वेनाधिकार‚{\color{DodgerBlue3}‚देव} प्र‚तिज्ञार्थ‚स्य \edtext{}{\edlabel{pvv.422-3}\label{pvv.422-3}\lemma{स्य}\Bfootnote{साध‚न‚स्य विजातीय‚त्वात् ।}} ‚{\color{DodgerBlue3}‚हेत्वाभासेष्व‚प्र‚स‚ङ्ग‚तः} (२४)
	\pend% ending standard par
      \label{div_pvv.4.25}
	  
	% new div opening: depth here is 2
	

	  \pstart \leavevmode% starting standard par
	य‚स्माद् (।)
	\pend% ending standard par
      
	  \bigskip
	  \begingroup
	
	    \large
	  
	    \begin{quote}
	  
	    
	    \stanza[\smallbreak]
	\label{pv.4.25}\flagstanza{\tiny\textenglish{...v.4.25}}अविशेषोक्तिर‚प्येक‚जातीये संश‚याव‚हा ।&अन्य‚था स‚र्व्व‚साध्योक्तेः प्र‚तिज्ञात्वं प्र‚स‚ज्य‚ते ॥ २५ ॥\&[\smallbreak]


	
	    \end{quote}
	  
	  \endgroup
	

	  \pstart \leavevmode% starting standard par
	\hphantom{.}‚{\color{DodgerBlue3}‚अविशेषोक्तिः} \edtext{\textsuperscript{*}}{\edlabel{pvv.422-4}\label{pvv.422-4}\lemma{*}\Bfootnote{साध‚नासाध‚न‚विभागं विनोक्तिः ।}}सामान्याभिधान‚म‚{\color{DodgerBlue3}‚प्येक‚जातीये \edtext{}{\edlabel{pvv.422-5}\label{pvv.422-5}\lemma{जातीये}\Bfootnote{साध्य एव न साध‚ने ।}} संश‚याव‚हा} त‚त्त्वार्थ‚श‚ङ्को‚{\tiny $_{lb}$}‚पाधिका न स‚र्व्व‚त्रेति न्यायः । त‚तोऽसाध‚न‚मेव साध्यं प्र‚तिज्ञा । न‚त्व‚सिद्ध‚{\tiny $_{lb}$}‚हेतुदृष्टान्तादिकं त‚स्य साध‚न‚त्वेनेष्ट‚त्वात् । ‚{\color{DodgerBlue3}‚अन्य‚था} य‚द्येवं नाभ्यु‚{\tiny $_{5}$}‚प‚ग‚म्य‚ते त‚दा ‚{\tiny $_{lb}$}‚‚{\color{DodgerBlue3}‚स‚र्व्व‚स्याः साध्योक्ते}‚र्घ‚टं \edtext{}{\edlabel{pvv.422-6}\label{pvv.422-6}\lemma{टं}\Bfootnote{असिद्ध‚स्य क‚र‚णात् ।}}क‚रोतीत्यादेः ‚{\color{DodgerBlue3}‚प्र‚तिज्ञात्वं प्र‚स‚ज्य‚ते} । ज्ञाप‚क‚हेत्व‚धिका‚{\tiny $_{lb}$}‚रात् त‚त्साध्य‚स्यैव प्र‚तिज्ञात्वं न कार‚क‚हेतुसाध्य‚स्येति चेत् । य‚द्येव‚म‚साध‚न‚भूत‚{\tiny $_{lb}$}‚साध्य‚निर्देशः प्र‚तिज्ञा न साध‚न‚निर्देश इति सिद्धं । (२५)
	\pend% ending standard par
      \label{div_pvv.4.26}
	  
	% new div opening: depth here is 2
	

	  \pstart \leavevmode% starting standard par
	\leavevmode\ledsidenote{\textenglish{423/s}}स्यादेत‚त् \edtext{}{\edlabel{pvv.423-1}\label{pvv.423-1}\lemma{त्}\Bfootnote{सिद्धो हि श‚ब्दादिकः साध्य‚ध‚र्मी । अन्य‚थाऽश्र‚यासिद्धो हेतुर‚प‚क्ष‚ध‚र्म‚त्वात् स्यात् क‚थं साध‚कः ।}} (।)
	\pend% ending standard par
      
	  \bigskip
	  \begingroup
	
	    \large
	  
	    \begin{quote}
	  
	    
	    \stanza[\smallbreak]
	\label{pv.4.26}\flagstanza{\tiny\textenglish{...v.4.26}}सिद्धोक्तेः साध‚न‚त्वाच्च प‚र‚स्यापि न दुष्य‚ति ।&इदानीं साध्य‚निर्देशः साध‚नाव‚य‚वः क‚थ‚म् ॥ २६ ॥\&[\smallbreak]


	
	    \end{quote}
	  
	  \endgroup
	

	  \pstart \leavevmode% starting standard par
	\hphantom{.}‚{\color{DodgerBlue3}‚प‚र‚स्यापि \edtext{}{\edlabel{pvv.423-2}\label{pvv.423-2}\lemma{स्यापि}\Bfootnote{योपि प्र‚तिज्ञासाध‚न‚माह । त‚स्यापि हेत्वाभास‚व‚च‚ने न प्र‚तिज्ञात्वं ।}} सिद्धोक्तेः} सिद्धार्थ‚प्र‚तिपाद‚क‚त्वात् । ‚{\color{DodgerBlue3}‚साध‚न‚त्वाच्च} प्र‚तिज्ञात्व‚{\tiny $_{lb}$}‚मिष्टं । अतो हेत्वाभासादि प्र‚तिज्ञात्व‚{\tiny $_{6}$}‚प्र‚स‚ङ्गेन ‚{\color{DodgerBlue3}‚न दुष्य‚ति । हेत्वाभासाद्य‚सि}‚-\leavevmode\ledsidenote{\textenglish{84a/MA}} ‚{\tiny $_{lb}$}‚द्ध‚त्वाद‚साध‚न‚मंसाध‚न‚त्वाच्च न प्र‚तिज्ञा । ‚{\color{DodgerBlue3}‚इदानी\edtext{}{\edlabel{pvv.423-3}\label{pvv.423-3}\lemma{इदानी}\Bfootnote{य‚द्य‚सिद्धाभिधानाद्धेत्वाभासा न साध‚नं त‚दा ।}}}‚म‚स्मिन्न‚भ्युप‚ग‚मे ‚{\color{DodgerBlue3}‚साध्य‚निर्देशो}‚{\tiny $_{lb}$}‚ऽसिद्ध‚त्वात् ‚{\color{DodgerBlue3}‚साध‚नाव‚य‚वः क‚थं} (।) न हि प्र‚तिज्ञार्थः सिद्धः । त‚द‚र्थ‚मेव साध‚नोप‚न्या‚{\tiny $_{lb}$}‚सात् । असिद्ध‚श्च न साध‚नं हेत्वाभास‚व‚त् ॥ (२६)
	\pend% ending standard par
      \label{div_pvv.4.27}
	  
	% new div opening: depth here is 2
	

	  \pstart \leavevmode% starting standard par
	या च स्व\edtext{}{\edlabel{pvv.423-4}\label{pvv.423-4}\lemma{स्व}\Bfootnote{न्याय‚मुख‚टीकाकारादेः ।}}यूथ्यानां पूर्व्व‚प‚क्ष‚प‚रिहारोक्तिः (।)
	\pend% ending standard par
      
	  \bigskip
	  \begingroup
	
	    \large
	  
	    \begin{quote}
	  
	    
	    \stanza[\smallbreak]
	\label{pv.4.27}\flagstanza{\tiny\textenglish{...v.4.27}}साभासोक्याद्युप‚क्षेप‚प‚रिहार‚विड‚म्ब‚ना ।&अस‚म्ब‚द्धा त‚था ह्येष न न्याय्य इति सूचित‚म् ॥ २७ ॥\&[\smallbreak]


	
	    \end{quote}
	  
	  \endgroup
	

	  \pstart \leavevmode% starting standard par
	प‚क्ष\edtext{}{\edlabel{pvv.423-5}\label{pvv.423-5}\lemma{क्ष}\Bfootnote{प्र‚योग‚स्तु प‚क्ष‚व‚च‚नं साध‚नं साध‚न‚काले उपादानात् हेतुदृष्टान्त‚व‚त् । पूर्व्व‚प‚क्षः ।}}व‚च‚नं साध‚नं साभास(साभासार्थ)त्वादिति चेत् । न प्र‚त्य‚क्षेणा-\edtext{\textsuperscript{*}}{\edlabel{pvv.423-6}\label{pvv.423-6}\lemma{*}\Bfootnote{प‚रिहारः ।}} ‚{\tiny $_{lb}$}‚नेकान्तात् प्र‚त्य‚क्षं साभास‚म‚पि न क‚स्य‚{\tiny $_{1}$}‚चित् प्र‚माण‚स्य साध‚नं/?/ व‚च‚ना\edtext{}{\edlabel{pvv.423-7}\label{pvv.423-7}\lemma{ना}\Bfootnote{तेनैव प‚र आश‚ङ्क्य‚ते ।}}त्म‚त्वे स‚ति ‚{\tiny $_{lb}$}‚साभास\edtext{}{\edlabel{pvv.423-8}\label{pvv.423-8}\lemma{साभास}\Bfootnote{प्र‚त्य‚क्ष‚म‚व‚च‚नात्म‚कं ।}}त्वात् साध‚न‚त्व‚मिति चेत् । न दूष‚णेना\edtext{}{\edlabel{pvv.423-9}\label{pvv.423-9}\lemma{णेना}\Bfootnote{प‚रिहारः । साध‚न‚काले दूष‚ण‚म‚प्युपादीय‚ते न च त‚त्साध‚नं ।}}नेकान्तात् । दूष‚णं साभासं‚{\tiny $_{lb}$}‚व‚च‚नात्म‚त्वेपि न साध‚न‚मितिसापि साभासोक्तिरा\edtext{}{\edlabel{pvv.423-10}\label{pvv.423-10}\lemma{साभासोक्तिरा}\Bfootnote{अत्र य‚दि प‚रः प्र‚दूष‚य‚ति अदूष‚ण‚त्वे स‚ति साभासोक्तिर्हेतुस्त‚दा नैवानेकान्तः । एवं य‚त्र य‚त्र न्या य मुख‚टीकाकृता व्य‚भिचारा उच्य‚ते । त‚त्र त‚त्र प‚रेण विशेष‚ण‚मुच्य‚त इति प‚रंप‚रा । विरुद्धाव्य‚भिचार्युप‚क्षेपे च प‚क्ष‚व‚च‚नं न साध‚न‚म‚सिद्धोक्तेर‚सिद्ध‚दृष्टान्त‚व‚च‚न‚व‚त् । इत्युक्त एव बाधितः स्यात् ।}}दिर्य‚स्य तौ‚{\color{DodgerBlue3}‚साभासोक्त्यादी} \leavevmode\ledsidenote{\textenglish{424/s}} ‚{\color{DodgerBlue3}‚उप‚क्षेप‚प‚रिहारौ} । तावेव ‚{\color{DodgerBlue3}‚विड‚म्व‚ना}‚ऽयुक्त‚त‚या । अत एवाह (।) ‚{\color{DodgerBlue3}‚अस‚म्ब‚द्धा । त‚था ‚{\tiny $_{lb}$}‚हि} साभास‚त्व‚स्य विप‚र्य‚ये बाध‚क\edtext{}{\edlabel{pvv.424-1}\label{pvv.424-1}\lemma{क}\Bfootnote{व‚च‚ने साभास‚त्वासाध‚न‚त्व‚योर‚विरोधात् ।}}प्र‚माण‚भावादेवाहेतुत्वाद‚स‚म्ब‚न्धः पूर्व्व‚प‚क्षः । ‚{\tiny $_{lb}$}‚त‚त‚स्त‚म‚नुम\edtext{}{\edlabel{pvv.424-2}\label{pvv.424-2}\lemma{नुम}\Bfootnote{विरुद्धाव्य‚भिचार्युप‚क्षेपादिशोभ‚नं स्यात् स‚न्दिग्धं व्य‚तिरेकाद् भाव‚नं न्याय्यं ।}}त्य प्र‚त्य‚क्षेणानेकान्त‚तापाद‚न‚म‚शो\edtext{}{\edlabel{pvv.424-2-bis}\label{pvv.424-2-bis}\lemma{शो}\Bfootnote{विरुद्धाव्य‚भिचार्युप‚क्षेपादिशोभ‚नं स्यात् स‚न्दिग्धं व्य‚तिरेकाद् भाव‚नं न्याय्यं ।}}भ‚नं । पुन‚र्व‚च‚नात्म‚त्वं विशेष‚णं ‚{\tiny $_{lb}$}‚प‚रोक्त‚म‚प्र‚तिक्षिप्य दूष‚णाभासेनानेकान्त‚तापाद‚नं चायुक्तं । त‚देव\edtext{}{\edlabel{pvv.424-3}\label{pvv.424-3}\lemma{देव}\Bfootnote{त‚त्प्र‚तिक्षेपोपाय‚माह ।}} हि हेतोर्व्विशेष‚{\tiny $_{lb}$}‚ण‚मुप‚युक्तं(।)य‚द्विप‚क्षाद्धेतुं व्याव‚र्त्त‚य‚ति न च व‚च‚नात्म‚त्वाऽसाध‚न‚त्व‚योः क‚श्चिद्वि‚{\tiny $_{lb}$}‚रोधो येनासाध‚नाद् व‚च‚नात्म‚त्व‚निवृत्तेर्व्विशेष‚ण‚साफ‚ल्यं स्यात् । त‚था ह्येष विप‚क्षाद‚{\tiny $_{lb}$}‚व्याव‚र्त्त‚क‚{\tiny $_{3}$}‚हेतुविशेष‚णोप‚न्यासो ‚{\color{DodgerBlue3}‚न न्याय्य} इति व‚र्ण्णितं\edtext{}{\edlabel{pvv.424-4}\label{pvv.424-4}\lemma{र्ण्णितं}\Bfootnote{युक्तं तु विशेष‚णं वेद‚चिन्तायां अपौरुषेयं ।}}प्राक् वेद‚नित्य‚तासिध्य‚र्थं‚{\tiny $_{lb}$}‚म‚ध्य‚य‚न‚पूर्व्व‚क‚मित्युक्ते भार‚ताध्य‚य‚नेनानेकान्त‚तामापादितां प्र‚तिषेद्धुं वेदाध्य‚य‚न‚त्वे ‚{\tiny $_{lb}$}‚स‚तीति विशेष‚णं मी मां स केनोप‚न्य‚स्तं त‚द‚पि क‚र‚ण(कृति)पूर्व्व‚कं भार‚ताध्य‚य‚न‚व‚द् ‚{\tiny $_{lb}$}‚स्यात् न क‚श्चिद् विरोधः । त‚तो विप‚क्षाद‚व्याव‚र्त्त‚कं विशेष‚ण‚म‚युक्त‚मित्युक्तं ‚{\tiny $_{lb}$}‚प्रा‚{\tiny $_{4}$}‚क् ॥ (२७)
	\pend% ending standard par
      \label{div_pvv.4.28}
	  
	% new div opening: depth here is 2
	

	  \begin{center}%% label @type='head'
	\textbf{(३) प‚क्ष‚ल‚क्ष‚ण‚क‚र‚णे प्र‚योज‚न‚म्}
	\end{center}
	

	  \pstart \leavevmode% starting standard par
	न‚नु य‚दि प‚क्ष‚व‚च‚न‚म‚साध‚नं साम‚र्थ्य‚ग‚म्याभिधेय‚ञ्च त‚दाचा र्ये ण प‚क्ष‚ल‚क्ष‚णं ‚{\tiny $_{lb}$}‚कृतं किम‚र्थ‚मित्याह (।)
	\pend% ending standard par
      
	  \bigskip
	  \begingroup
	
	    \large
	  
	    \begin{quote}
	  
	    
	    \stanza[\smallbreak]
	\label{pv.4.28}\flagstanza{\tiny\textenglish{...v.4.28}}ग‚म्यार्थ‚त्वेपि साध्योक्तेर‚संमोहाय ल‚क्ष‚ण‚म् ।&त‚च्च‚तुर्ल‚क्ष‚णं रूप‚निपातेषु स्व‚यं प‚दैः ॥ २८ ॥\&[\smallbreak]


	
	    \end{quote}
	  
	  \endgroup
	

	  \pstart \leavevmode% starting standard par
	\hphantom{.}‚{\color{DodgerBlue3}‚साध्य‚व्या}‚प्त‚प‚क्ष‚व‚च‚न‚साम‚र्थ्याद् ‚{\color{DodgerBlue3}‚ग‚म्यार्थ‚त्वेपि साध्योक्तेः} प‚क्ष‚व‚च‚न‚स्य ‚{\color{DodgerBlue3}‚ल‚क्ष‚ण}‚{\tiny $_{lb}$}‚मुक्त‚म‚{\color{DodgerBlue3}‚संमोहाय} विप्र‚तिप‚त्तिनिराक‚र‚णेन साध्य‚प्र‚तिप‚त्त्य‚र्थं । त‚था ह्यात्मार्थ‚त्वं ‚{\tiny $_{lb}$}‚‚{\color{DodgerBlue3}‚साध्य‚म‚पि सां ख्या} असाध्य‚माच‚क्ष‚ते प‚रार्थ‚त्व‚म‚साध्य‚म‚पि साध्य‚मिति स‚न्ति वि‚{\tiny $_{5}$}‚प्र‚ति‚{\tiny $_{lb}$}‚प‚त्त‚यः । त‚च्च साध्यं ‚{\color{DodgerBlue3}‚च‚तुर्ल‚क्ष‚ण}‚मुक्तं । स्व‚रूपेणैव निर्देश्यः स्व‚य‚मिष्टोऽनिराकृतः ‚{\tiny $_{lb}$}‚प‚क्ष इत्य‚त्र ल‚क्ष‚ण‚च‚तुष्ट‚य‚प्र‚तिपाद‚कै रुप‚निपातेषु ‚{\color{DodgerBlue3}‚स्व‚यं प‚दै}‚र्य‚थाक्र‚म‚म् (। २८)
	\pend% ending standard par
      \label{div_pvv.4.29}
	  
	% new div opening: depth here is 2
	
	  \bigskip
	  \begingroup
	
	    \large
	  
	    \begin{quote}
	  
	    
	    \stanza[\smallbreak]
	\label{pv.4.29}\flagstanza{\tiny\textenglish{...v.4.29}}असिद्धासाध‚नार्थोक्त‚वाद्य‚भ्युप‚ग‚त‚ग्र‚हः ।&अनुक्तोपीच्छ‚या व्याप्तः साध्य आत्मार्थ‚व‚न्म‚तः ॥ २९ ॥\&[\smallbreak]


	
	    \end{quote}
	  
	  \endgroup
	

	  \pstart \leavevmode% starting standard par
	\hphantom{.}‚{\color{DodgerBlue3}‚असिद्ध}‚स्या‚{\color{DodgerBlue3}‚साध‚न}‚स्यार्थो‚{\color{DodgerBlue3}‚क्त}‚स्य ‚{\color{DodgerBlue3}‚वाद्य‚भ्युप‚ग‚त‚स्य ग्र‚हः} । असिद्ध‚स्व‚भाव‚त्वात् ‚{\tiny $_{lb}$}‚साध्य‚स्य न सिद्ध‚स्य ग्र‚ह‚णं । त‚तः सिद्धं चाक्षुष‚त्वादि रूपादेर्न साध्यं । निपातैव ‚{\tiny $_{lb}$}‚\leavevmode\ledsidenote{\textenglish{425/s}} कार‚क‚र‚णेनासाध‚न‚स्य ग्र‚ह‚{\tiny $_{6}$}‚णं त‚त‚श्चाक्षुष‚त्वाद्य‚सिद्ध‚म‚पि साध‚न‚त्वेन श‚ब्देभिधी‚{\tiny $_{lb}$}‚य‚मानं न साध्यं । दृष्ट‚श‚ब्देनार्थोक्त‚स्यापि ग्र‚ह‚णं । त‚तोऽनुक्त‚म‚प्यात्मार्थ‚त्वं ‚{\tiny $_{lb}$}‚संघात‚त्वाच्च‚क्षुरादेः सांख्य‚स्य साध्यं । स्व‚यं श‚ब्देन वाद्य‚भ्युप‚ग‚त‚स्य ग्र‚ह‚णं । त‚तः ‚{\tiny $_{lb}$}‚शास्त्राभ्युप‚ग‚त‚स्याकाश‚गुण‚त्वादेः श‚ब्दे ध‚र्मिणि वादिनाऽनित्य‚त्वे साध‚यितुमार‚ब्धे-\leavevmode\ledsidenote{\textenglish{84b/MA}} ‚{\tiny $_{lb}$}‚ऽसाध्य‚ता । अल्प‚व‚क्त‚व्य‚त‚याऽर्थोक्त‚स्य ता‚{\tiny $_{7}$}‚व‚त् साध्य‚तां स‚म‚र्थ‚यितुमाह । वादिनो‚{\tiny $_{lb}$}‚‚{\color{DodgerBlue3}‚ऽनुक्तोपीच्छ‚या व्याप्तः साध्यो म‚तः । आत्मार्थ‚व‚त्} । य‚था आत्मास्ति न वेति विवादे ‚{\tiny $_{lb}$}‚त‚त्साध‚नार्थं सां ख्ये न प‚रार्थाश्च‚क्षुराद‚यः संघात‚त्वात् श‚य‚नाश(? स)नाद्य‚ङ्ग‚व‚त् । ‚{\tiny $_{lb}$}‚इत्युक्त‚स्य साध‚न‚स्यात्मार्थ‚त्व‚म‚नुक्त‚म‚पि साध्य‚मिच्छाविष‚य‚त्वात् ॥ (२९)
	\pend% ending standard par
      \label{div_pvv.4.30}
	  
	% new div opening: depth here is 2
	

	  \pstart \leavevmode% starting standard par
	न‚न्विष्ट श‚ब्देनानिष्ट‚स्य स‚र्व्व‚स्य निरासात् । शास्त्रोप‚ग‚त‚स्यापि वाद्य‚निष्ठ‚{\tiny $_{lb}$}‚स्यासाध्य‚त्वं सिद्धं । त‚न्निष्फ‚लं स्व‚यंप‚द‚मित्याह । व्य‚व‚च्छेद‚फ‚ल‚त्वाच्छ‚ब्दा-‚{\tiny $_{1}$}‚ ‚{\tiny $_{lb}$}‚नामिष्ट‚श‚ब्दात् (।)
	\pend% ending standard par
      
	  \bigskip
	  \begingroup
	
	    \large
	  
	    \begin{quote}
	  
	    
	    \stanza[\smallbreak]
	\label{pv.4.30}\flagstanza{\tiny\textenglish{...v.4.30}}स‚र्वान्येष्ट‚निवृत्ताव‚प्याश‚ङ्कास्थान‚वार‚ण‚म् ।&वृत्तौ स्व‚यंश्रुतेनाह कृता चैषा त‚द‚र्थिका ॥ ३० ॥\&[\smallbreak]


	
	    \end{quote}
	  
	  \endgroup
	

	  \pstart \leavevmode% starting standard par
	\hphantom{.}‚{\color{DodgerBlue3}‚स‚र्व्व}‚स्य वादिनोऽ‚{\color{DodgerBlue3}‚न्येन} शास्त्रोदिना इ‚{\color{DodgerBlue3}‚ष्ट}‚स्य ‚{\color{DodgerBlue3}‚निवृत्तौ} सिद्धाया‚{\color{DodgerBlue3}‚म‚पि} शास्त्रेणेष्टं ‚{\tiny $_{lb}$}‚वादिनोपीष्ट‚मेवेति ‚{\color{DodgerBlue3}‚श‚ङ्कास्थान}‚स्य विप्र‚तिप‚त्तिविष‚य‚स्य ‚{\color{DodgerBlue3}‚वार‚णं} फ‚लं ‚{\color{DodgerBlue3}‚स्व‚यंश्रुते}‚नाचार्यो ‚{\tiny $_{lb}$}‚‚{\color{DodgerBlue3}‚वृत्ता}‚वाह । स्व‚य‚मिति शास्त्रान‚पेक्ष‚म‚भ्युप‚ग‚म‚न्द‚र्श‚य‚ति । ‚{\color{DodgerBlue3}‚एवा} स्व‚यंश्रुति‚{\color{DodgerBlue3}‚स्त‚द‚र्थिका} विप्र‚तिप‚त्तिनिराक‚र‚णार्था ‚{\color{DodgerBlue3}‚कृता} ॥ (३०)
	\pend% ending standard par
      \label{div_pvv.4.31}
	  
	% new div opening: depth here is 2
	

	  \begin{center}%% label @type='head'
	\textbf{(४) आत्मार्थ‚त्व‚विवादे दोषः ।}
	\end{center}
	

	  \pstart \leavevmode% starting standard par
	\edtext{\textsuperscript{*}}{\edlabel{pvv.425-1}\label{pvv.425-1}\lemma{*}\Bfootnote{य‚दि वादिनेष्ट एव साध्य‚स्त‚दा ध‚र्म‚विशेष‚विप‚रीत‚साध‚नादीनां विरुद्धानाम‚स‚म्भ‚व एवेत्याह ।}} य एवेच्छ‚या विष‚यीकृतः स (।)
	\pend% ending standard par
      
	  \bigskip
	  \begingroup
	
	    \large
	  
	    \begin{quote}
	  
	    
	    \stanza[\smallbreak]
	\label{pv.4.31}\flagstanza{\tiny\textenglish{...v.4.31}}विशेष‚स्त‚द्व्य‚पेक्ष‚त्वात् क‚थितो ध‚र्म‚ध‚र्मिणोः ।&अनुक्ताव‚पि वाञ्छाया भ‚वेत् प्र‚क‚र‚णाद् ग‚तिः ॥ ३१ ॥\&[\smallbreak]


	
	    \end{quote}
	  
	  \endgroup
	

	  \pstart \leavevmode% starting standard par
	\hphantom{.}विशेषो ‚{\color{DodgerBlue3}‚ध‚र्म‚ध‚र्मिणोः} स‚म्ब‚न्धी 1/?/ ‚{\color{DodgerBlue3}‚व्य‚पेक्ष}‚त्वात् । इच्छार‚चितात् स‚म्ब‚न्धात् ‚{\tiny $_{lb}$}‚साध्य‚त्वे‚{\tiny $_{2}$}‚न ‚{\color{DodgerBlue3}‚क‚थितः} । च‚क्षुरादीनां संह‚त‚विष‚यं पारार्थ्य‚मिति ध‚र्म‚स्य ‚{\color{DodgerBlue3}‚विशेषः} साध्यः । ‚{\tiny $_{lb}$}‚प‚रार्थ‚स्य साध्य‚त्वात् । प‚रार्थाः स‚न्त‚श्च‚क्षुराद‚योऽसंह‚तार्था इति ध‚र्मिणो विशेषः ‚{\tiny $_{lb}$}‚साध्यः । च‚क्षुराद‚योऽनेकाणुस‚ञ्च‚यात्मिकाः क्र‚मेणैक‚काल‚ञ्च संह‚ताः । ज्ञानादि तु ‚{\tiny $_{lb}$}‚\leavevmode\ledsidenote{\textenglish{426/s}} काल‚भेदेनानेक‚त्वात् संह‚तानि । तेषां प‚रार्थानां स‚ताम‚संह‚त‚विष‚य‚त्व‚मेवेच्छाविष‚य‚{\tiny $_{lb}$}‚त्वात् साध्यं । आत्म‚नः स‚र्व्व‚काल‚मेक‚त्वे‚{\tiny $_{3}$}‚ नासंह‚त‚त्वात् । क‚थं पुन‚रात्मार्थ‚त्व‚स्यानुक्तौ ‚{\tiny $_{lb}$}‚त‚द्विष‚याया वाञ्छायाः प्र‚तीतिरित्याह । ‚{\color{DodgerBlue3}‚वाञ्छाया अनुक्ताव‚पि} मुख्यं श‚ब्देन ‚{\tiny $_{lb}$}‚‚{\color{DodgerBlue3}‚प्र‚क‚र‚णा}‚दात्मास्ति नास्तीति संश‚ये स‚ति त‚त्साध‚नोप‚न्यास‚प्र‚स्तावाद् ‚{\color{DodgerBlue3}‚ग‚तिः} प्र‚तीति‚{\color{DodgerBlue3}‚र्भ‚व‚ति} ॥ (३१)
	\pend% ending standard par
      \label{div_pvv.4.32}
	  
	% new div opening: depth here is 2
	

	  \pstart \leavevmode% starting standard par
	आत्मार्थ‚त्व‚स्य विवादे को दोष इत्याह (।)
	\pend% ending standard par
      
	  \bigskip
	  \begingroup
	
	    \large
	  
	    \begin{quote}
	  
	    
	    \stanza[\smallbreak]
	\label{pv.4.32}\flagstanza{\tiny\textenglish{...v.4.32}}अन‚न्व‚योपि दृष्टान्ते दोष‚स्त‚स्य य‚थोदित‚म् ।&आत्मा प‚र‚श्चेत् सोऽसिद्ध इति त‚त्रेष्ट‚घात‚व‚त् ॥ ३२ ॥\&[\smallbreak]


	
	    \end{quote}
	  
	  \endgroup
	

	  \pstart \leavevmode% starting standard par
	\hphantom{.}‚{\color{DodgerBlue3}‚त‚स्ये}‚च्छाविष‚य‚स्यात्मार्थ‚त्व‚स्य साध्या‚{\color{DodgerBlue3}‚न‚न्व‚यो दृष्टान्ते दोषः\edtext{}{\edlabel{pvv.426-1}\label{pvv.426-1}\lemma{दोषः}\Bfootnote{उक्त विरुद्ध‚त्वं ।}}} । अपिश‚ब्दा‚{\tiny $_{lb}$}‚द्व‚क्ष्य‚माण इष्ट‚विधात‚श्च । ‚{\color{DodgerBlue3}‚य‚थोदित}‚माचार्य‚{\tiny $_{4}$}‚व सु ब न्धु ना । प‚रार्थाश्च‚क्षुराद‚य ‚{\tiny $_{lb}$}‚इत्य‚त्र ‚{\color{DodgerBlue3}‚प‚र‚श्चेदात्मा} विव‚क्षितः ‚{\color{DodgerBlue3}‚सोऽसिद्धो} दृष्टान्त ‚{\color{DodgerBlue3}‚इति । त‚त्रा}‚न्व‚ये स‚ती‚{\color{DodgerBlue3}‚ष्ट‚विधा‚{\tiny $_{lb}$}‚त‚व‚त्} । साध‚नं इष्टात्मार्थ‚त्व‚विप‚र्य‚येणान्व‚यात् त‚त्साध‚क‚त्वात् ॥ (३२)
	\pend% ending standard par
      \label{div_pvv.4.33}
	  
	% new div opening: depth here is 2
	

	  \pstart \leavevmode% starting standard par
	अथात्मार्थ‚त्व न साध्य‚मित्याह (।)
	\pend% ending standard par
      
	  \bigskip
	  \begingroup
	
	    \large
	  
	    \begin{quote}
	  
	    
	    \stanza[\smallbreak]
	\label{pv.4.33}\flagstanza{\tiny\textenglish{...v.4.33}}साध‚नं य‚द्विवादे न न्य‚स्तं त‚च्चेन्न साध्य‚ते ।&किं साध्य‚म‚न्य‚थानिष्टं भ‚वेद् वैफ‚ल्य‚मेव वा ॥ ३३ ॥\&[\smallbreak]


	
	    \end{quote}
	  
	  \endgroup
	

	  \pstart \leavevmode% starting standard par
	\hphantom{.}‚{\color{DodgerBlue3}‚य}‚स्यात्म‚नोर्थ‚स्य ‚{\color{DodgerBlue3}‚विवादे}‚ऽस्ति नास्तीति स‚न्देहे न साध‚नं ‚{\color{DodgerBlue3}‚न्य‚स्त}‚मुप‚न्य‚स्तं ‚{\tiny $_{lb}$}‚‚{\color{DodgerBlue3}‚त‚च्चेन्न साध्य‚ते कि}‚मिदानीं ‚{\color{DodgerBlue3}‚साध्यं} स्यात् । ‚{\color{DodgerBlue3}‚अन्य‚था} विवाद‚विष‚यो य‚दि न साध्यं ‚{\tiny $_{lb}$}‚त‚दा‚{\tiny $_{5}$}‚‚{\color{DodgerBlue3}‚निष्टं} विप‚र्य‚य‚सिद्धिः स्यात् । य‚था व्युत्प‚न्न‚स‚र्व्व‚श‚ब्द‚वादिनं\edtext{}{\edlabel{pvv.426-2}\label{pvv.426-2}\lemma{वादिनं}\Bfootnote{वैयाक‚र‚णं ।}} प्र‚त्य‚व्युत्प‚न्न‚{\tiny $_{lb}$}‚संज्ञाश‚ब्द‚वादिना त‚द‚र्थ‚व‚त्व‚सिद्ध्य‚र्थं साध‚न‚मुच्य‚ते ॥
	\pend% ending standard par
      

	  \pstart \leavevmode% starting standard par
	संज्ञिस‚म्ब‚न्धात् प्राग‚र्थ‚व\edtext{}{\edlabel{pvv.426-3}\label{pvv.426-3}\lemma{व}\Bfootnote{संज्ञी नास्तीति न संज्ञाश‚ब्द‚स्त‚द‚र्थ‚वानिति म‚त्वायं ।}}च्छ‚ब्द‚रूपं विभ‚क्तिद‚र्श‚नात् त‚द‚न्य‚श‚ब्द‚व‚दिति । ‚{\tiny $_{lb}$}‚अत्र \edtext{}{\edlabel{pvv.426-4}\label{pvv.426-4}\lemma{अत्र}\Bfootnote{नित्ये श‚ब्दार्थ‚स‚म्ब‚न्धे ।}} ग‚च्छ‚तीति गौरित्येकार्थ‚स‚म‚वायात् क्रियोप‚ल‚क्षितेन बाह्य‚सामान्येनार्थ‚{\tiny $_{lb}$}‚वान् गोश‚ब्दः सिद्धोऽव्युत्प‚न्न‚वादिनः । न‚तु स्व‚रूप‚मात्रेणा\edtext{}{\edlabel{pvv.426-5}\label{pvv.426-5}\lemma{मात्रेणा}\Bfootnote{त‚स्य साध्य‚त्वेनान‚भ्युप‚ग‚मात् ।}}र्थेनार्थ‚वान् । अर्थ‚{\tiny $_{lb}$}‚मात्र‚ज‚ञ्च साध्य‚त्वे‚{\tiny $_{6}$}‚नोद्दिष्टं न तु स्व‚रूपेणार्थेनार्थ‚व‚त्वं । त‚तो दृष्टान्ते विभ‚क्त्य‚{\tiny $_{lb}$}‚न्त‚स्य वाक्यार्थ‚व‚त्वेन व्याप्तिसिद्धेर्देव‚द‚त्तादाव‚पि प‚क्षीकृते संज्ञाश‚ब्दे देवैर्द‚त्तो देव‚द‚त्त ‚{\tiny $_{lb}$}‚इति वाक्यार्थ‚व‚त्वं स्व‚रूपार्थ‚व‚त्वे विरुद्धं सिध्य‚ति ॥
	\pend% ending standard par
      \textsuperscript{\textenglish{427/s}}

	  \pstart \leavevmode% starting standard par
	अथ‚वेष्ट‚स्य साध्य‚त्वाभावे प‚रार्थाश्च‚क्षुराद‚यः संह‚त‚त्वादित्य‚त्रात्मार्थ‚त्व‚स्या‚{\tiny $_{lb}$}‚संह‚त‚पारार्थ्य‚स्यासाध्य‚त्वात् । ज्ञान‚हेतुत्वेन संह‚त‚पारार्थ‚स्य बौद्धेनापीष्टे । साध‚{\tiny $_{7}$}‚-\leavevmode\ledsidenote{\textenglish{85a/MA}} ‚{\tiny $_{lb}$}‚न‚वैफ‚ल्य‚मेव वा स्यात् । (३३)
	\pend% ending standard par
      \label{div_pvv.4.34_4.35}
	  
	% new div opening: depth here is 2
	
	  \bigskip
	  \begingroup
	
	    \large
	  
	    \begin{quote}
	  
	    
	    \stanza[\smallbreak]
	\label{pv.4.34a}\flagstanza{\tiny\textenglish{....4.34a}}स‚द्वितीय‚प्र‚योगेषु निर‚न्व‚य‚विरुद्ध‚ते ।&एतेन क‚थिते साध्यं;\&[\smallbreak]


	
	    \end{quote}
	  
	  \endgroup
	

	  \pstart \leavevmode% starting standard par
	\hphantom{.}‚{\color{DodgerBlue3}‚एतेन} साध्य‚त्वेनेष्ट‚स्यान‚न्व‚य‚दोष‚द‚र्श‚नेन ‚{\color{DodgerBlue3}‚स‚द्वितीय‚प्र‚योगेषु} चा र्व्वा क‚कृतेषु ‚{\tiny $_{lb}$}‚य‚थाभिव्य‚क्त‚चैत‚न्य‚श‚रीर‚ल‚क्ष‚ण‚पुरुष\edtext{}{\edlabel{pvv.427-1}\label{pvv.427-1}\lemma{पुरुष}\Bfootnote{घ‚ट‚योर‚न्य‚त‚रेण ।}}स‚द्वितीयो घ‚टः । अनुत्प‚न्न‚त्वात् । कुड्य‚व‚दिति ‚{\tiny $_{lb}$}‚श‚रीर‚मेवाभिव्य‚क्त‚चैत‚न्यं पुरुषो नात्मा क‚श्चित् प‚र‚लोकी तेन स‚द्वितीय‚त्वं (स‚स‚हा‚{\tiny $_{lb}$}‚य‚त्त्वं) घ‚ट‚स्य साध्य‚त इति प्र‚योग‚फ‚लं । त‚त्र च ‚{\color{DodgerBlue3}‚निर‚न्व‚य‚विरुद्ध‚ते क‚थिते} । त‚था ‚{\tiny $_{lb}$}‚ह्य‚भिव्य‚{\tiny $_{1}$}‚क्त‚चैत‚न्य‚देह‚ल‚क्ष‚ण‚पुरुषेण स‚द्वितीय‚त्वं ‚{\color{DodgerBlue3}‚साध्यं} ।\edtext{\textsuperscript{*}}{\edlabel{pvv.427-2}\label{pvv.427-2}\lemma{*}\Bfootnote{प्र‚त्य‚क्ष‚विष‚य‚त्वाद‚नुमान‚मुक्तं ।}} तेन च कुड्येन्व‚यो न ‚{\tiny $_{lb}$}‚दृष्ट इति निर‚न्व‚य‚ता (।) घ‚ट‚स्य तु कुड्येऽन्व‚यो दृष्ट इति तेन स‚द्वितीय‚त्व‚{\tiny $_{lb}$}‚साध‚नात् विरुद्ध‚ता स्यात् ।
	\pend% ending standard par
      

	  \begin{center}%% label @type='head'
	\textbf{(व्य‚क्त्य‚सिद्धौ न सामान्य‚म्)}
	\end{center}
	
	  \bigskip
	  \begingroup
	
	    \large
	  
	    \begin{quote}
	  
	    
	    \stanza[\smallbreak]
	\label{pv.4.34b}\flagstanza{\tiny\textenglish{....4.34b}}सामान्येनाथ स‚म्म‚त‚म् ॥ ३४ ॥\&[\smallbreak]


	
	    \end{quote}
	  
	  \endgroup
	
	  \bigskip
	  \begingroup
	
	    \large
	  
	    \begin{quote}
	  
	    
	    \stanza[\smallbreak]
	\label{pv.4.35}\flagstanza{\tiny\textenglish{...v.4.35}}त‚देवार्थान्त‚राभावाद् देहानाप्तौ न सिध्य‚ति ।&वाच्य‚शून्य प्र‚ल‚प‚तां त‚देत‚ज्जाड्य‚व‚र्ण्णित‚म् ॥ ३५ ॥\&[\smallbreak]


	
	    \end{quote}
	  
	  \endgroup
	

	  \pstart \leavevmode% starting standard par
	\hphantom{.}‚{\color{DodgerBlue3}‚अथ सामान्येन}\edtext{\textsuperscript{*}}{\edlabel{pvv.427-3}\label{pvv.427-3}\lemma{*}\Bfootnote{अन्य‚त‚रार्थान्त‚र‚भावः सामान्यं साध्यं ।}} विशेष‚म‚नुल्लिख्य स‚द्वितीय‚त्वं साध्यं कुड्ये स‚द्वितीय‚त्व‚{\tiny $_{lb}$}‚मात्रेणान्व‚यात् । एव‚म‚पि ‚{\color{DodgerBlue3}‚त‚त्सा}‚मान्य‚{\color{DodgerBlue3}‚मेव} न सिध्य‚ति प्र‚तिवादिनः । घ‚टाद‚भि‚{\tiny $_{lb}$}‚व्य‚क्त‚चैत‚न्य‚स्व‚भाव‚त‚या‚{\color{DodgerBlue3}‚ऽर्थान्त‚{\tiny $_{2}$}‚राभावात्} । अर्थान्त‚र‚त्वास‚म्भ‚वात् । ‚{\color{DodgerBlue3}‚देह}‚स्या‚{\color{DodgerBlue3}‚नाप्ता}‚{\tiny $_{lb}$}‚व‚र्थान्त‚र‚त्वेनासिद्धौ द्व‚योर्भिन्न‚योर‚न्य‚त‚र‚स‚द्वितीय‚त्वं सामान्यं स्यात् ।\edtext{\textsuperscript{*}}{\edlabel{pvv.427-4}\label{pvv.427-4}\lemma{*}\Bfootnote{न हि गोव्य‚क्त्य‚भावे सामान्यं ।}} न हि घ‚टः ‚{\tiny $_{lb}$}‚स्व‚रुपेणैवान्य‚त‚र‚स‚द्वितीयः । देह‚न्तु नाभिव्य‚क्त‚चैत‚न्य‚ल‚क्ष‚ण‚पुरुष‚मिच्छ‚ति प्र‚तिवादिति ‚{\tiny $_{lb}$}‚भेदाभावात् । तेनापि नान्य‚त‚र‚स‚द्वितीय‚त्व‚सिद्धिः । य‚त‚श्च न घ‚ट‚स्य स्व‚रूपेणैव ‚{\tiny $_{lb}$}‚स‚द्वितीय‚त्व‚स‚म्भ‚वः । भेदाधिष्ठान‚त्वात् त‚स्य, ना‚{\tiny $_{3}$}‚पि देहेनान्व‚याभावात् । त‚तो ‚{\tiny $_{lb}$}‚वाच्य‚शून्य‚त्व‚म‚र्थ‚शून्य‚त्व‚म‚न्य‚त‚र‚स‚द्वितीय‚त्वं ‚{\color{DodgerBlue3}‚प्र‚ल‚प‚तां} प‚र‚लोकाप‚वादीनां ‚{\color{DodgerBlue3}‚त‚देत‚द}‚न्य‚{\tiny $_{lb}$}‚त‚र‚स‚द्वितीय‚त्व‚साध्य‚व‚च‚नं ‚{\color{DodgerBlue3}‚जाड्य‚स्य व‚र्ण्णितं} चेष्टितं ॥ (३४,३५)
	\pend% ending standard par
      \label{div_pvv.4.36}
	  
	% new div opening: depth here is 2
	\textsuperscript{\textenglish{428/s}}
	  \bigskip
	  \begingroup
	
	    \large
	  
	    \begin{quote}
	  
	    
	    \stanza[\smallbreak]
	\label{pv.4.36}\flagstanza{\tiny\textenglish{...v.4.36}}तुल्यं नाशेपि चेच्छ‚ब्द‚घ‚ट‚भेदेन क‚ल्प‚ने ।&न सिद्धेन विनाशेन त‚द्व‚तः साध‚नाद् ध्व‚नेः ॥ ३६ ॥\&[\smallbreak]


	
	    \end{quote}
	  
	  \endgroup
	

	  \pstart \leavevmode% starting standard par
	\hphantom{.}‚{\color{DodgerBlue3}‚नाशेपि} साध्ये ‚{\color{DodgerBlue3}‚श‚ब्द‚घ‚ट‚योः} साध्य‚दृष्टान्त‚ध‚र्मिणोः स‚म्ब‚न्धित‚या ‚{\color{DodgerBlue3}‚भेदेन\edtext{}{\edlabel{pvv.428-1}\label{pvv.428-1}\lemma{भेदेन}\Bfootnote{श‚ब्दानित्य‚ता घ‚ट‚नित्य‚तेति ।}} ‚{\tiny $_{lb}$}‚क‚ल्प‚ने} श‚ब्द‚स‚म्ब‚न्धिनो नाश‚स्य घ‚टेन्व‚याभावाद‚साध्य‚त्वं । घ‚ट‚स‚म्ब‚न्धिन‚श्च श‚ब्दे‚{\tiny $_{lb}$}‚ऽस‚म्भ‚वा‚{\tiny $_{4}$}‚द‚साध्य‚तेति तुल्य‚मिद‚मिति ‚{\color{DodgerBlue3}‚चेत् । न तुल्यं वानाशेन} प्र‚ध्वंस‚ल‚क्ष‚णेन ‚{\tiny $_{lb}$}‚‚{\color{DodgerBlue3}‚सिद्धेन} निश्चितेन ‚{\color{DodgerBlue3}‚ध्व‚नेस्त‚द्व‚तो} विनाश‚व‚तः ‚{\color{DodgerBlue3}‚साध‚ना}‚द्विनाश‚सामान्यं साध्यं सिद्धं केव‚लं ‚{\tiny $_{lb}$}‚त‚द्व‚त्ता श‚ब्द‚स्य न सिद्धेति साध्य‚ते । य‚था विनाशे साम‚न्येन सिद्धे स‚त्य‚सिद्ध‚स्त‚द्वान् ‚{\tiny $_{lb}$}‚श‚ब्दः साध्य‚ते\edtext{}{\edlabel{pvv.428-2}\label{pvv.428-2}\lemma{ते}\Bfootnote{एव‚म‚र्थान्त‚भावः सिद्धो नास्ति ।}}। (३६)
	\pend% ending standard par
      \label{div_pvv.4.37}
	  
	% new div opening: depth here is 2
	
	  \bigskip
	  \begingroup
	
	    \large
	  
	    \begin{quote}
	  
	    
	    \stanza[\smallbreak]
	\label{pv.4.37a}\flagstanza{\tiny\textenglish{....4.37a}}त‚थार्थान्त‚र‚भावे स्यात् त‚द्वान् कुम्भोपि ।\&[\smallbreak]


	
	    \end{quote}
	  
	  \endgroup
	

	  \pstart \leavevmode% starting standard par
	\hphantom{.}‚{\color{DodgerBlue3}‚त‚थार्थान्त‚र‚भावे}‚ऽभि\edtext{}{\edlabel{pvv.428-3}\label{pvv.428-3}\lemma{ऽभि}\Bfootnote{घ‚ट‚व्य‚क्त्या ।}}व्य‚क्त‚चैत‚न्य‚स्व‚भाव‚त‚या देह‚स्य घ‚टात् वैजात्ये सिद्धे स‚ति ‚{\tiny $_{lb}$}‚‚{\color{DodgerBlue3}‚त‚द्वान् कु‚{\tiny $_{5}$}‚म्भोपि} सिध्येत् । न चैत‚त् प्र‚तिवादी बोधियितुं श‚क्य‚ते । तेनाचैत‚न्य‚स्य ‚{\tiny $_{lb}$}‚भूत‚व्य‚तिरिक्त‚स्यैव स्वीकारात् । य‚दि पुन‚र‚चेत‚न‚स्व‚भाव‚त‚या घ‚ट‚जातीयेनैव देहेन\edtext{}{\edlabel{pvv.428-3-bis}\label{pvv.428-3-bis}\lemma{देहेन}\Bfootnote{घ‚ट‚व्य‚क्त्या ।}} ‚{\tiny $_{lb}$}‚स‚द्वितीय‚त्वं घ‚ट‚स्य साध्य‚ते त‚दा सिध्य‚त्येव । त‚थाविध‚स्य स‚द्वितीय‚त्व‚स्य सिद्ध‚त्वाद् ‚{\tiny $_{lb}$}‚विनाश‚व‚त् । किन्तु वादिनो नेष्टिसिद्धिः । देह‚स्य चेत‚न‚स्व‚भाव‚त‚याऽसिद्धेः ॥
	\pend% ending standard par
      
	  \bigskip
	  \begingroup
	
	    \large
	  
	    \begin{quote}
	  
	    
	    \stanza[\smallbreak]
	\label{pv.4.37b}\flagstanza{\tiny\textenglish{....4.37b}}अनित्य‚ता ।&विशिष्टा ध्व‚निनान्वेति नो चेन्नायोग‚वाण्णात् ॥ ३७ ॥\&[\smallbreak]


	
	    \end{quote}
	  
	  \endgroup
	\textsuperscript{\textenglish{85b/MA}}

	  \pstart \leavevmode% starting standard par
	अथ ध्व‚निना स्व‚स‚म्ब‚न्धित‚या विशिष्टाऽनित्य‚{\tiny $_{6}$}‚ता दृष्टान्तं ‚{\color{DodgerBlue3}‚ना}‚न्वेतीति ‚{\tiny $_{lb}$}‚‚{\color{DodgerBlue3}‚चेत्} । ‚{\color{DodgerBlue3}‚ना}‚न‚न्व‚य‚दोषो विशेष‚णेना‚{\color{DodgerBlue3}‚योग}‚स्यास‚म्ब‚न्ध‚स्य ‚{\color{DodgerBlue3}‚वार‚णात्} ॥ (३७)
	\pend% ending standard par
      \label{div_pvv.4.38}
	  
	% new div opening: depth here is 2
	
	  \bigskip
	  \begingroup
	
	    \large
	  
	    \begin{quote}
	  
	    
	    \stanza[\smallbreak]
	\label{pv.4.38}\flagstanza{\tiny\textenglish{...v.4.38}}द्विविधो हि व्य‚व‚च्छेदो वियोगाप‚र‚योग‚योः ।&व्य‚व‚च्छेदाद‚योगे तु वार्ये नान‚न्व‚याग‚मः ॥ ३८ ॥\&[\smallbreak]


	
	    \end{quote}
	  
	  \endgroup
	

	  \pstart \leavevmode% starting standard par
	\hphantom{.}‚{\color{DodgerBlue3}‚द्विविधो हि व्य‚व‚च्छेदो} विशेष‚णेन दृष्टो ‚{\color{DodgerBlue3}‚वियोगाप‚र‚योग‚योर}‚योगान्य‚योग‚यो‚{\tiny $_{lb}$}‚‚{\color{DodgerBlue3}‚र्व्य‚व‚च्छेदात्} । य‚था चैत्रो ध‚नुर्द्ध‚रः पा र्थो ध‚नुर्द्ध‚र इति । त‚त्र ध‚र्मिणा विशेष‚{\tiny $_{lb}$}‚णेनायोगे । वार्येऽनित्य‚ताया ‚{\color{DodgerBlue3}‚नान‚न्व‚याग‚मो}‚ऽन‚न्व‚याप‚त्तिर्न भ‚व‚ति । श‚ब्दोऽनित्यो न ‚{\tiny $_{lb}$}‚वेत्य‚योगः श‚ङ्कितो विशेष‚णेन‚{\tiny $_{1}$}‚व्याव‚र्त्त्य‚ते श‚ब्दोऽनित्य इति । एव‚म्विधा चानित्य‚ता ‚{\tiny $_{lb}$}‚नान्य‚स‚म्ब‚न्धेन विरुध्य‚त इति नान‚न्व‚यो दृष्टान्ते ॥ (३८)
	\pend% ending standard par
      \label{div_pvv.4.39}
	  
	% new div opening: depth here is 2
	

	  \pstart \leavevmode% starting standard par
	उक्तार्थ‚संग्र‚ह‚माह ।
	\pend% ending standard par
      \textsuperscript{\textenglish{429/s}}
	  \bigskip
	  \begingroup
	
	    \large
	  
	    \begin{quote}
	  
	    
	    \stanza[\smallbreak]
	\label{pv.4.39}\flagstanza{\tiny\textenglish{...v.4.39}}सामान्य‚मेव त‚त्साध्यं न च सिद्ध‚प्र‚साध‚म् ।&विशिष्टं ध‚र्मिणा त‚च्च न निर‚न्व‚य‚दोष‚व‚त् ॥ ३९ ॥\&[\smallbreak]


	
	    \end{quote}
	  
	  \endgroup
	

	  \pstart \leavevmode% starting standard par
	\hphantom{.}‚{\color{DodgerBlue3}‚त‚द}‚नित्य‚तादि ‚{\color{DodgerBlue3}‚सामान्य‚मेव साध्यं} न विशेषो येनान‚न्व‚य‚दोषः स्यात् । न‚न्व‚{\tiny $_{lb}$}‚नित्य‚तादि सामान्यं ‚{\color{DodgerBlue3}‚सिद्ध‚मेव क्व\edtext{}{\edlabel{pvv.429-1}\label{pvv.429-1}\lemma{क्व}\Bfootnote{विद्युदादौ ।}}चित्} साध‚ने वैय‚र्थ्य‚मित्याह । ‚{\color{DodgerBlue3}‚न च सिद्ध‚स्य} क्व‚चित् स‚त्तामात्रेणानित्य‚त्व‚स्य ‚{\color{DodgerBlue3}‚प्र‚साध‚नं} । ध‚र्मिण्य‚योग‚व्य‚व‚च्छेद‚स्यासिद्ध‚स्य ‚{\tiny $_{lb}$}‚प्र‚साध‚नात् । न च ‚{\color{DodgerBlue3}‚ध‚र्मिणा‚{\tiny $_{2}$}‚}‚ऽयोग‚व्य‚व‚च्छेद‚तो ‚{\color{DodgerBlue3}‚विशिष्टं त‚च्चा}‚नित्य‚तादि दृष्टान्ते ‚{\tiny $_{lb}$}‚‚{\color{DodgerBlue3}‚निर‚न्व‚य‚दोष‚व‚त्} (।) (न च)अयोग‚व्य‚व‚च्छेदेन ध‚र्मिविशेषित‚स्य ध‚र्म्य‚न्त‚र‚स‚म्ब‚न्धा‚{\tiny $_{lb}$}‚विरोधात् ॥ (३९)
	\pend% ending standard par
      \label{div_pvv.4.40}
	  
	% new div opening: depth here is 2
	
	  \bigskip
	  \begingroup
	
	    \large
	  
	    \begin{quote}
	  
	    
	    \stanza[\smallbreak]
	\label{pv.4.40}\flagstanza{\tiny\textenglish{...v.4.40}}एतेन ध‚र्मिध‚र्माभ्यां विशिष्टौ ध‚र्म‚ध‚र्मिणौ ।&प्र‚त्याख्यातौ निराकुर्व‚न् ध‚र्मिण्येव‚म‚साध‚नात् ॥ ४० ॥\&[\smallbreak]


	
	    \end{quote}
	  
	  \endgroup
	

	  \pstart \leavevmode% starting standard par
	\hphantom{.}‚{\color{DodgerBlue3}‚एतेने}‚ष्ट‚स्य साध्य‚त्व‚व‚च‚नेन ‚{\color{DodgerBlue3}‚ध‚र्मिध‚र्माभ्यां विशिष्टौ ध‚र्म‚ध‚र्मिणा}‚व‚न‚न्व‚या‚{\tiny $_{lb}$}‚‚{\color{DodgerBlue3}‚न्निराकुर्व‚न्} चार्व्वाको य‚था न श‚ब्दानित्य‚त्व‚वान् श‚ब्दो नानित्य‚श‚ब्द‚वान् वा श‚ब्द ‚{\tiny $_{lb}$}‚इति । न हि श‚ब्दानित्य‚त्वेनानित्य‚श‚ब्देन वा क्व‚चिद् घ‚टादौ दृष्टान्ते कृ‚{\tiny $_{3}$}‚त‚क‚{\tiny $_{lb}$}‚त्व‚स्यान्व‚योस्ति त‚त इष्ट‚विप‚र्यासिनाद् विरुद्धं कृत‚क‚त्व‚मिति स एवं व‚द‚न् प्र‚त्या‚{\tiny $_{lb}$}‚ख्यातः क‚थ‚मित्याह । ‚{\color{DodgerBlue3}‚ध‚र्मिणि} श‚ब्दे ‚{\color{DodgerBlue3}‚एवं} ध‚र्म्मिविशिष्ट‚स्य ध‚र्म‚स्य ध‚र्माविशिष्ट‚स्य ‚{\tiny $_{lb}$}‚वा ध‚र्मिणोऽ‚{\color{DodgerBlue3}‚साध‚नाद}‚नित्य‚त्व‚मात्र‚स्य श‚ब्दे साध्य‚त्वेनेष्ट‚त्वात् । अन्य‚थाऽनित्य‚{\tiny $_{lb}$}‚श‚ब्द‚व‚ति श‚ब्दे सिद्धेपि श‚ब्दो नानित्यः स्यात् । य‚दि ध‚र्म‚मात्रं साध्यं त‚दा स‚मुदायः ‚{\tiny $_{lb}$}‚साध्यो न स्यात्\edtext{}{\edlabel{pvv.429-2}\label{pvv.429-2}\lemma{स्यात्}\Bfootnote{न ध‚र्मिणा स‚ह स‚मुदाय‚साध‚नात् । न हि श‚ब्दे प‚रः श‚ब्दानित्य‚त्व‚स‚मुदायः साध्यः त‚तः त‚स्य निराक‚र‚णेपि न दोषः ।}}। हेतोस्त‚द‚प‚वादो‚{\tiny $_{4}$}‚विरुद्ध‚स्य न स्यादित्याह । ध‚र्म‚मात्र‚स्य ‚{\tiny $_{lb}$}‚ध‚र्मि\edtext{}{\edlabel{pvv.429-3}\label{pvv.429-3}\lemma{र्मि}\Bfootnote{आश्र‚यासिद्धिर‚न्य‚था ।}}साध्य‚त्वात् । (४०)
	\pend% ending standard par
      \label{div_pvv.4.41}
	  
	% new div opening: depth here is 2
	
	  \bigskip
	  \begingroup
	
	    \large
	  
	    \begin{quote}
	  
	    
	    \stanza[\smallbreak]
	\label{pv.4.41}\flagstanza{\tiny\textenglish{...v.4.41}}स‚मुदायाप‚वादो हि न ध‚र्मिणि विरुध्य‚ते ।&साध्यं य‚त‚स्त‚था नेष्टं साध्यो ध‚र्मोत्र केव‚लः ॥ ४१ ॥\&[\smallbreak]


	
	    \end{quote}
	  
	  \endgroup
	

	  \pstart \leavevmode% starting standard par
	\hphantom{.}स‚मुदाय एव साध्यो ‚{\color{DodgerBlue3}‚हि} य‚स्मात् त‚स्माद् विरुद्ध‚स्य हेतोरिष्ट‚{\color{DodgerBlue3}‚स‚मुदाया\edtext{}{\edlabel{pvv.429-4}\label{pvv.429-4}\lemma{मुदाया}\Bfootnote{ ॥ ॰ ॥ स्व‚यं श‚ब्दात् । स च द्व‚योरेकाभावे स‚मुदाय एव निराकृतः स्यात् ।}}प‚{\tiny $_{lb}$}‚वादो न विरुध्य‚ते} । न‚नु स‚मुदाय‚स्य साध्य‚त्वेऽन‚न्व‚य‚दोष इष्ट‚विघातो वा स्यादि‚{\tiny $_{lb}$}‚त्याह (।) त‚था ध‚र्मिध‚र्म‚स‚मुदायोऽन्य‚ध‚र्मिस‚म्ब‚न्धित‚या ‚{\color{DodgerBlue3}‚साध्यं य‚तो नेष्टं} (।) ‚{\tiny $_{lb}$}‚त‚स्मान्नान‚न्व‚यो विरुद्ध‚ता वा । ‚{\color{DodgerBlue3}‚त‚था ह्य‚त्र} ‚{\tiny $_{5}$}‚ श‚ब्दादौ ध‚र्मिणि ‚{\color{DodgerBlue3}‚ध‚र्मो}‚ऽनित्य‚तादिः ‚{\tiny $_{lb}$}‚‚{\color{DodgerBlue3}‚केव‚लः साध्य} इष्टः । त‚स्य चान्व‚योस्तीति न दोषः । (४१)
	\pend% ending standard par
      \label{div_pvv.4.42}
	  
	% new div opening: depth here is 2
	

	  \pstart \leavevmode% starting standard par
	\leavevmode\ledsidenote{\textenglish{430/s}}उक्त‚मिष्ट‚ग्र‚ह‚ण‚स्य प्र‚योज‚नं ॥
	\pend% ending standard par
      

	  \begin{center}%% label @type='head'
	\textbf{(५) स्व‚यंश‚ब्द‚ग्र‚ह‚ण‚फ‚ल‚म् ॥}
	\end{center}
	

	  \pstart \leavevmode% starting standard par
	स्व‚यं श‚ब्द‚स्येदानीं व‚क्तुमाह ।
	\pend% ending standard par
      
	  \bigskip
	  \begingroup
	
	    \large
	  
	    \begin{quote}
	  
	    
	    \stanza[\smallbreak]
	\label{pv.4.42}\flagstanza{\tiny\textenglish{...v.4.42}}एक‚स्य ध‚र्मिणः शास्त्रें नानाध‚र्म‚स्थिताव‚पि ।&साध्यः स्यादात्म‚नैवेष्ट इत्युपात्ता स्व‚यं श्रुतिः ॥ ४२ ॥\&[\smallbreak]


	
	    \end{quote}
	  
	  \endgroup
	

	  \pstart \leavevmode% starting standard par
	\hphantom{.}‚{\color{DodgerBlue3}‚एक‚स्य} श‚ब्दादे‚{\color{DodgerBlue3}‚र्द्ध‚मिणः शास्त्रे नानाध‚र्मा}‚णाम‚मूर्त्त‚त्वानित्य‚ताकाश‚गुण‚त्वादीनां\leavevmode\ledsidenote{\textenglish{86a/MA}} ‚{\tiny $_{lb}$}‚‚{\color{DodgerBlue3}‚स्थिताव}‚भ्युप‚ग‚मेपि वादिना ‚{\color{DodgerBlue3}‚आत्म‚नैव} साध‚नोप‚न्यास‚काले साध‚यितुमिष्टो ध‚र्मः ‚{\tiny $_{lb}$}‚‚{\color{DodgerBlue3}‚साध्यः स्यात्} । नान्य ‚{\color{DodgerBlue3}‚इति स्व‚यं} श्रुति‚{\tiny $_{6}$}‚तिराचार्येणो‚{\color{DodgerBlue3}‚पात्ता} । (४२)
	\pend% ending standard par
      \label{div_pvv.4.43}
	  
	% new div opening: depth here is 2
	

	  \pstart \leavevmode% starting standard par
	य‚दि पुनः (।)
	\pend% ending standard par
      
	  \bigskip
	  \begingroup
	
	    \large
	  
	    \begin{quote}
	  
	    
	    \stanza[\smallbreak]
	\label{pv.4.43}\flagstanza{\tiny\textenglish{...v.4.43}}शास्त्राभ्युप‚ग‚मादेव स‚र्वादानात् प्र‚बाध‚ने ।&त‚त्रैक‚स्यापि दोषः स्याद् य‚दि हेतुप्र‚तिज्ञ‚योः ॥ ४३ ॥\&[\smallbreak]


	
	    \end{quote}
	  
	  \endgroup
	

	  \pstart \leavevmode% starting standard par
	\hphantom{.}‚{\color{DodgerBlue3}‚शास्त्रे}‚णा‚{\color{DodgerBlue3}‚भ्युप‚ग‚मादेव} स‚र्व्वेषां ध‚र्माणा‚{\color{DodgerBlue3}‚मादानात्} प‚रिग्र‚हात् वादिना ‚{\color{DodgerBlue3}‚त‚त्र} तेषु म‚ध्ये ‚{\tiny $_{lb}$}‚‚{\color{DodgerBlue3}‚एक‚स्यापि} ध‚र्म‚स्योप‚न्य‚स्त‚हेतुना बाध‚ने ‚{\color{DodgerBlue3}‚हेतुप्र‚तिज्ञ‚यो}‚र्व्विरुद्ध‚ता ‚{\color{DodgerBlue3}‚दोष} उच्य‚ते\edtext{}{\edlabel{pvv.430-1}\label{pvv.430-1}\lemma{ते}\Bfootnote{य‚दि त‚दाप‚रः श्लोकः ।}}॥ (४३)
	\pend% ending standard par
      \label{div_pvv.4.44}
	  
	% new div opening: depth here is 2
	

	  \pstart \leavevmode% starting standard par
	य‚था श‚ब्दे शास्त्रेष्ट‚माकाशाश्र‚य‚त्वं बाध‚मान‚स्य\edtext{}{\edlabel{pvv.430-2}\label{pvv.430-2}\lemma{स्य}\Bfootnote{आकाश‚स्य नित्य‚त्वात् त‚दाश्रित‚ञ्च नित्यं स्यात् । त‚द‚नित्य‚त्वेन बाध्य‚ते ।}} वादीष्ट‚म‚नित्य‚त्वं साध‚{\tiny $_{lb}$}‚य‚तोपि कृत‚क‚त्व‚स्य विरुद्ध‚त्वं प्र‚तिज्ञाविरोधो वाभिधीय‚ते । कृत‚कं हि क्ष‚णिकं । ‚{\tiny $_{lb}$}‚न च क्ष‚णिक‚मुत्पादान‚न्न‚रं क्ष‚ण‚म‚प्य‚स्ति । त‚तः ‚{\tiny $_{1}$}‚ कृत‚क‚त्व‚माकाशाश्र‚य‚त्व‚बाध‚नं ।
	\pend% ending standard par
      
	  \bigskip
	  \begingroup
	
	    \large
	  
	    \begin{quote}
	  
	    
	    \stanza[\smallbreak]
	\label{pv.4.44}\flagstanza{\tiny\textenglish{...v.4.44}}श‚ब्द‚नाशे प्र‚साध्ये स्याद् ग‚न्ध‚भूगुण‚ताक्ष‚तेः ।&हेतुर्व्विरुद्धोप्र‚कृतेर्न्नोचेद‚न्य‚त्र सा स‚मा ॥ ४४ ॥\&[\smallbreak]


	
	    \end{quote}
	  
	  \endgroup
	

	  \pstart \leavevmode% starting standard par
	\hphantom{.}त‚था कृत‚क‚त्वात् ‚{\color{DodgerBlue3}‚श‚ब्द}‚स्य ‚{\color{DodgerBlue3}‚नाशे सा}‚ध्य\edtext{}{\edlabel{pvv.430-3}\label{pvv.430-3}\lemma{ध्य}\Bfootnote{वैशेषिक‚कृता य‚था श‚ब्दे आकाश‚गुण‚त्व‚मिष्टं एवं ग‚न्धे पृथिवीगुण‚त्व‚म‚पीति कोत्र विशेषो येनैक‚त्र हेतुप्र‚तिज्ञादोषो नान्य‚त्रेति भावः अप्र‚कृत‚त्वाच्चेद‚त्रापि स‚मानं । अनेन वेदापौरुषेय‚वादीनं प्र‚त्युक्त‚नित्यः श‚ब्दः कृत‚क‚त्वात् ज‚लादिव‚त् । ग‚न्धेपि कृत‚क‚त्वाद‚नित्य‚त्वे भूगुण‚त्व‚क्ष‚तेः ।}}माने ‚{\color{DodgerBlue3}‚ग‚न्ध‚स्य}\edtext{}{\edlabel{pvv.430-4}\label{pvv.430-4}\lemma{माने}\Bfootnote{ध‚र्मिणः}} कृत‚क‚त्वान्न‚श्व‚र‚स्य ‚{\tiny $_{lb}$}‚‚{\color{DodgerBlue3}‚भूगुण‚ता}‚या ः पृथिव्याश्रित‚तायाः शास्त्रेष्टायाश्च ‚{\color{DodgerBlue3}‚क्ष‚तेः} । विप‚र्यास‚नात् ‚{\color{DodgerBlue3}‚हेतुः} प्र‚य‚{\tiny $_{lb}$}‚\leavevmode\ledsidenote{\textenglish{431/s}} त्नान‚न्त‚रीय‚क‚त्वादि\edtext{}{\edlabel{pvv.431-1}\label{pvv.431-1}\lemma{त्वादि}\Bfootnote{कृत‚क‚त्वादिरादिना ।}} ‚{\color{DodgerBlue3}‚विरुद्धः} स्यात् । प्र‚तिज्ञा --- विरुद्धा स्यात् । उपात्तो हेतुः ‚{\tiny $_{lb}$}‚वादिनानिष्टं शास्त्रेष्टं बाध‚त इत्येव य‚दि विरुद्धः त‚दा ग‚न्ध‚भूगुण‚तां बाध‚मान‚स्य ‚{\tiny $_{lb}$}‚कृत‚क‚त्व‚स्यं श‚ब्दे विरुद्ध‚ता स्यात् । शास्त्रेष्ट‚बाध‚क‚ताया अ‚{\tiny $_{2}$}‚विशेषात् । (४४)
	\pend% ending standard par
      \label{div_pvv.4.45}
	  
	% new div opening: depth here is 2
	

	  \pstart \leavevmode% starting standard par
	अथ ग‚न्ध‚भूगुण‚ताया साध्य‚त्वेनाप्र‚कृतेर‚प्र‚स्तुते त‚द्वाध‚नेपि श‚ब्दे कृत‚क‚त्वं विरुद्धं ‚{\tiny $_{lb}$}‚नो चेत् सा प्र‚कृतिर‚न्य‚त्राकाश‚गुण‚त्वेपि स‚मा । न हि वादिना आकाश‚गुण‚त्वं ‚{\tiny $_{lb}$}‚साध‚यितुमिष्टं किन्त्व‚नित्य‚त्वं । अतोऽप्र‚कृत‚स्यास्य बाध‚ने न विरुद्धः स्याद्धेतुः ।
	\pend% ending standard par
      
	  \bigskip
	  \begingroup
	
	    \large
	  
	    \begin{quote}
	  
	    
	    \stanza[\smallbreak]
	\label{pv.4.45}\flagstanza{\tiny\textenglish{...v.4.45}}अथात्र ध‚र्मी प्र‚कृत‚स्त‚त्र शास्त्रार्थ‚बाध‚न‚म् ।&अथ वादीष्ट‚तां ब्रूयाद् ध‚र्मिध‚र्मादिसाध‚नैः ॥ ४५ ॥\&[\smallbreak]


	
	    \end{quote}
	  
	  \endgroup
	

	  \pstart \leavevmode% starting standard par
	\hphantom{.}‚{\color{DodgerBlue3}‚अथात्रा}‚काश‚गुण‚त्वादौ ‚{\color{DodgerBlue3}‚ध‚र्मी} श‚ब्दः ‚{\color{DodgerBlue3}‚प्र‚कृतः} । ‚{\color{DodgerBlue3}‚त‚त्र शास्त्रार्थ}‚स्याकाश‚गुण‚{\tiny $_{lb}$}‚त्वादेः साध‚नं त‚द्बाध‚ने च विरुद्ध‚ता‚{\tiny $_{3}$}‚ हेतोः । भूगुण‚त्वे तु ग‚न्धो ध‚र्म्य‚प्र‚कृत इति ‚{\tiny $_{lb}$}‚त‚द्बाध‚नेपि न विरोधः । ‚{\color{DodgerBlue3}‚नैष प‚रिहारः । त‚थाहि} न वादीष्ट‚विप‚र्यास‚नेन दोष ‚{\tiny $_{lb}$}‚उक्तः । किन्तु शास्त्रार्थ‚विरोधेन त‚था च प्र‚कृत‚त्व‚म‚नुप‚युक्तं । ‚{\color{DodgerBlue3}‚अथ} वाद्य‚निष्ट‚त‚याऽ‚{\tiny $_{lb}$}‚प्र‚कृत‚त्वं त‚च्चाकाश‚गुण‚त्व‚योः स‚मानं । अथाकाश‚गुण‚त्व‚स्य ‚{\color{DodgerBlue3}‚वादीष्ट‚तां} प‚रो ‚{\color{DodgerBlue3}‚ब्रूयात् ‚{\tiny $_{lb}$}‚ध‚र्म्मिध‚र्म्मादिसाध‚नैः} \edtext{\textsuperscript{*}}{\edlabel{pvv.431-2}\label{pvv.431-2}\lemma{*}\Bfootnote{यो ध‚र्मिणो विशेषः साध्य‚स‚मुदायैक‚देश‚विशेषो वा स साध्यः ।}}। साध्य‚ध‚र्मिध‚र्म‚त्वात् त‚देक‚देश‚त्वाद्वाऽका‚{\tiny $_{4}$}‚श‚गुण‚त्व‚{\tiny $_{lb}$}‚मिष्टं वादिनोऽनित्य‚त्व‚व‚दिति । (४५)
	\pend% ending standard par
      \label{div_pvv.4.46}
	  
	% new div opening: depth here is 2
	

	  \pstart \leavevmode% starting standard par
	न‚नु (।)
	\pend% ending standard par
      
	  \bigskip
	  \begingroup
	
	    \large
	  
	    \begin{quote}
	  
	    
	    \stanza[\smallbreak]
	\label{pv.4.46}\flagstanza{\tiny\textenglish{...v.4.46}}कैश्चित् प्र‚क‚र‚णैरिच्छा भ‚वेत् सा ग‚म्य‚ते च तैः ।&ब‚लात् त‚वेच्छेय‚मिति व्य‚क्त‚मीश्व‚र‚चेष्टित‚म् ॥ ४६ ॥\&[\smallbreak]


	
	    \end{quote}
	  
	  \endgroup
	

	  \pstart \leavevmode% starting standard par
	\hphantom{.}‚{\color{DodgerBlue3}‚कैश्चित् प्र‚क‚र‚णै}‚र्व्विवादादि‚{\color{DodgerBlue3}‚भिरिच्छा}‚वादिनः क‚स्मिश्चिद् ध‚र्मे ‚{\color{DodgerBlue3}‚भ‚वेत्} । ‚{\tiny $_{lb}$}‚तैरेव च प्र‚क‚र‚णैः ‚{\color{DodgerBlue3}‚सा} इच्छा ‚{\color{DodgerBlue3}‚ग‚म्य‚ते} प‚रेणापि । न तु त‚स्य ध‚र्मिणो ध‚र्म इत्येव वादि‚{\tiny $_{lb}$}‚नेष्य‚ते । त‚तो ध‚मिध‚र्म‚त्वादिस‚न्दिग्ध‚विप‚क्ष‚व्य‚तिरेकित्वात् शेष‚व‚त् । स्व‚य‚म‚नि‚{\tiny $_{lb}$}‚च्छ‚त‚श्च वादिनः साध‚न‚स्य ‚{\color{DodgerBlue3}‚ब‚लात् त‚वेच्छेय‚मिति} य‚दुच्य‚ते ‚{\color{DodgerBlue3}‚व्य‚क्त}‚मिद‚{\color{DodgerBlue3}‚मीश्व‚र‚{\tiny $_{5}$}‚‚{\tiny $_{lb}$}‚चेश्टित}‚मित्युप‚ह‚स‚ति । (४६)
	\pend% ending standard par
      \label{div_pvv.4.47}
	  
	% new div opening: depth here is 2
	

	  \pstart \leavevmode% starting standard par
	किञ्च (।)
	\pend% ending standard par
      
	  \bigskip
	  \begingroup
	
	    \large
	  
	    \begin{quote}
	  
	    
	    \stanza[\smallbreak]
	\label{pv.4.47}\flagstanza{\tiny\textenglish{...v.4.47}}व‚द‚न्न‚कार्य‚लिङ्गां तां व्य‚भिचारेण बाध्य‚ते ।&अनान्त‚रीय‚के चार्थे बाधितेन्य‚स्य का क्ष‚तिः ॥ ४७ ॥\&[\smallbreak]


	
	    \end{quote}
	  
	  \endgroup
	\textsuperscript{\textenglish{432/s}}

	  \pstart \leavevmode% starting standard par
	\hphantom{.}‚{\color{DodgerBlue3}‚ता}‚मिच्छाम‚{\color{DodgerBlue3}‚कार्य‚लिङ्ग‚जां} कार्येत‚र‚लिङ्गाम‚नुमेय‚त्वेन व‚द‚न् प‚रो ‚{\color{DodgerBlue3}‚व्य‚भिचा‚{\tiny $_{lb}$}‚रेण बाध्य‚ते} । न ह्य‚न्योऽकार्योऽन्यं न व्य‚भिच‚र‚तीति निय‚मोस्ति । अपि च साध्य‚{\tiny $_{lb}$}‚सानित्य‚त्व‚स्या‚{\color{DodgerBlue3}‚न‚न्त‚रीय‚केऽर्थे} आकाश‚गुण‚त्वादौ ‚{\color{DodgerBlue3}‚बाधित}‚त्वेप्य‚{\color{DodgerBlue3}‚न्य‚स्य} साध्य‚स्य ‚{\color{DodgerBlue3}‚का ‚{\tiny $_{lb}$}‚क्ष‚तिः} । न ह्य‚नित्य‚त्व‚माकाश‚गुण‚त्व‚नान्त‚रीय‚कं येन त‚द‚भावे त‚द‚पि न स्यात् । ‚{\tiny $_{lb}$}‚कृत‚क‚त्व‚न्त्व‚नि‚{\tiny $_{6}$}‚त्व‚त्य‚ताऽव्य‚भिचारीति त‚स्मान्नानुमान‚म‚नैकान्तिकं । (४७)
	\pend% ending standard par
      
	  
	% new div opening: depth here is 1
	
\chapter*[{३. श‚ब्दाप्रामाण्य‚चिन्ता}]{३. श‚ब्दाप्रामाण्य‚चिन्ता}\label{div_pvv.4.48}
	  
	% new div opening: depth here is 2
	

	  \begin{center}%% label @type='head'
	\textbf{(१) शास्त्र‚विरोधोऽकिञ्चित्क‚रः}
	\end{center}
	
	  \bigskip
	  \begingroup
	
	    \large
	  
	    \begin{quote}
	  
	    
	    \stanza[\smallbreak]
	\label{pv.4.48}\flagstanza{\tiny\textenglish{...v.4.48}}उक्त‚ञ्च नाग‚मापेक्ष‚म‚नुमानं स्व‚गोच‚रे ।&सिद्धं तेन सुसिद्ध त‚न्न त‚दा शास्त्र‚मीक्ष्य‚ते ॥ ४८ ॥\&[\smallbreak]


	
	    \end{quote}
	  
	  \endgroup
	

	  \pstart \leavevmode% starting standard par
	\hphantom{.}अनुमाविष‚ये नेष्टं वाचः प्रामाण्य (३।३१०) मित्यादि‚{\color{DodgerBlue3}‚नोक्तं} प्राक् । ‚{\tiny $_{lb}$}‚व‚स्तुब‚ल‚प्र‚वृत्त‚{\color{DodgerBlue3}‚म‚नुमानं नाग‚मापेक्षं स्व}‚स्य ‚{\color{DodgerBlue3}‚गोच‚रे} साध्य इति । त‚स्मात् ‚{\color{DodgerBlue3}‚तेन} व‚स्तु‚{\tiny $_{lb}$}‚ब‚ल‚प्र‚वृत्तेनाग‚मान‚पेक्षिणाऽनुमाने य‚त् ‚{\color{DodgerBlue3}‚सिद्धं सुसिद्ध}‚न्त‚{\color{DodgerBlue3}‚त्त‚दा} च ‚{\color{DodgerBlue3}‚न शास्त्र‚मीक्ष्य‚ते} बाधितं न वेति व‚स्तुब‚ल‚प्र‚वृत्तानुमानेन त‚द‚पेक्षाभावात् । त‚दानुकाले शास्त्र‚स्या‚{\tiny $_{lb}$}‚नाश्र‚य‚णात् ।‚{\tiny $_{7}$}‚ (४८)
	\pend% ending standard par
      \label{div_pvv.4.49}
	  
	% new div opening: depth here is 2
	
	  \bigskip
	  \begingroup
	
	    \large
	  
	    \begin{quote}
	  
	    
	    \stanza[\smallbreak]
	\label{pv.4.49}\flagstanza{\tiny\textenglish{...v.4.49}}वाद‚त्याग‚स्त‚दा स्याच्चेन्न त‚दान‚भ्युपाय‚तः ।&उपायो ह्य‚भ्युपायेऽय‚म‚न‚ङ्गं स त‚दापि स‚न् ॥ ४९ ॥\&[\smallbreak]


	
	    \end{quote}
	  
	  \endgroup
	\textsuperscript{\textenglish{86b/MA}}

	  \pstart \leavevmode% starting standard par
	\hphantom{.}‚{\color{DodgerBlue3}‚वाद‚त्यागः स्याच्चेत्} । न वाद‚त्यागः सिसाध‚यिषितैक‚ध‚र्माद‚प‚र‚स्य शास्त्रा‚{\tiny $_{lb}$}‚भ्युपेत‚स्य ‚{\color{DodgerBlue3}‚त‚दा} साध‚नोप‚न्यास‚का\edtext{}{\edlabel{pvv.432-1}\label{pvv.432-1}\lemma{का}\Bfootnote{व‚स्तुब‚ल‚प्र‚वृत्ते ।}}ले साध्य‚त‚याऽ‚{\color{DodgerBlue3}‚न‚भ्युपाय‚तो}‚ऽस्वीकारात् । न‚नु ‚{\tiny $_{lb}$}‚शास्त्राभ्युप‚ग‚माद् य‚दा वादः क्रिय‚ते त‚दा शास्त्रार्थ‚बाध‚नात् वाद‚त्यागः स्यादेवे‚{\tiny $_{lb}$}‚त्याह । शास्त्र‚स्या‚{\color{DodgerBlue3}‚भ्युपाये\edtext{}{\edlabel{pvv.432-2}\label{pvv.432-2}\lemma{भ्युपाये}\Bfootnote{स्वीकारे}}ऽय}‚म्विचार\edtext{}{\edlabel{pvv.432-3}\label{pvv.432-3}\lemma{म्विचार}\Bfootnote{विचार्य‚ग्र‚हात् ।}} ‚{\color{DodgerBlue3}‚उपायः} । त‚त‚{\color{DodgerBlue3}‚स्त‚दा} विचार‚काले ‚{\color{DodgerBlue3}‚स‚न्न‚प्य}‚{\tiny $_{lb}$}‚भ्युप‚ग‚मोऽ‚{\color{DodgerBlue3}‚न‚ङ्गं} शास्त्रार्थ‚ग्र‚ह‚णे गृहीत‚स्य त्यागः स्यात् । न वि‚{\tiny $_{1}$}‚चारात् प्राग् ग्र‚ह‚ण‚{\tiny $_{lb}$}‚म‚भ्युप‚ग‚मान्न्याय्यं । (४८)
	\pend% ending standard par
      \label{div_pvv.4.50}
	  
	% new div opening: depth here is 2
	

	  \pstart \leavevmode% starting standard par
	क‚दा त‚र्हि शास्त्रेण बाधेष्य‚ते इत्याह ।
	\pend% ending standard par
      
	  \bigskip
	  \begingroup
	
	    \large
	  
	    \begin{quote}
	  
	    
	    \stanza[\smallbreak]
	\label{pv.4.50}\flagstanza{\tiny\textenglish{...v.4.50}}त‚दा विशुद्धे विष‚य‚द्व‚ये शास्त्र‚प‚रिग्र‚ह‚म् ।&चिकीर्षोः स हि कालः स्यात् त‚दा शास्त्रेण बाध‚न‚म् ॥ ५० ॥\&[\smallbreak]


	
	    \end{quote}
	  
	  \endgroup
	\textsuperscript{\textenglish{433/s}}

	  \pstart \leavevmode% starting standard par
	\hphantom{.}‚{\color{DodgerBlue3}‚शास्त्रो}‚प‚द‚र्शिते ‚{\color{DodgerBlue3}‚विष‚य\edtext{}{\edlabel{pvv.433-1}\label{pvv.433-1}\lemma{य}\Bfootnote{विष‚यास्त्र‚य‚स्त‚त्र प्र‚त्य‚क्षानुमान‚विष‚यो प्राग्विचारेण विशोध‚नीयौ । त‚था विशुद्धे प‚श्चाच्छास्त्र‚ग्र‚हं चिकीर्षा ।}}द्व‚ये} प्र‚त्य‚क्ष‚प‚रोक्षे रूप\edtext{}{\edlabel{pvv.433-2}\label{pvv.433-2}\lemma{रूप}\Bfootnote{रूपाद्य‚क्ष‚स्यान्य‚द‚नुमायाः ।}}नैरात्म्यादौ त‚दा प्र‚माण‚प्र‚वृत्त्या ‚{\tiny $_{lb}$}‚विशुद्धे निर्ण्णीते स‚ति प‚श्चाद‚त्य‚न्त‚प‚रोक्षे स्व‚र्गादौ शास्त्रेण शास्त्राश्र‚य‚णेनानुमानं ‚{\tiny $_{lb}$}‚चिकीर्षोः स‚तः स हि कालोऽभ्युग‚म्य य‚दि शास्त्र‚बाधो न भ‚वेत् । अत‚{\color{DodgerBlue3}‚स्त‚दा शास्त्रेण} ‚{\color{DodgerBlue3}‚बाध‚नं} साध्य‚साध‚नादेरिष्य‚{\tiny $_{2}$}‚ते । (५०)
	\pend% ending standard par
      \label{div_pvv.4.51}
	  
	% new div opening: depth here is 2
	

	  \pstart \leavevmode% starting standard par
	त‚द‚पि क‚र‚माच्छास्त्र‚बाधेष्य‚त इत्याह । (।)
	\pend% ending standard par
      
	  \bigskip
	  \begingroup
	
	    \large
	  
	    \begin{quote}
	  
	    
	    \stanza[\smallbreak]
	\label{pv.4.51}\flagstanza{\tiny\textenglish{...v.4.51}}त‚द्विरोधेन चिन्तायास्त‚त्सिद्धार्थेष्व‚योग‚तः ।&तृतीय‚स्थान‚संक्रान्तौ न्याय्यः शास्त्र‚प‚रिग्र‚हः ॥ ५१ ॥\&[\smallbreak]


	
	    \end{quote}
	  
	  \endgroup
	

	  \pstart \leavevmode% starting standard par
	\hphantom{.}‚{\color{DodgerBlue3}‚त}‚स्य शास्त्र‚स्य ‚{\color{DodgerBlue3}‚विरोधेन त‚त्सिद्धेष्व‚र्थेषु} लिङ्गादिष्व‚सिद्ध‚क‚ल्पेषु ग‚म‚क‚{\tiny $_{lb}$}‚‚{\color{DodgerBlue3}‚चिन्ताया अयोग‚तः} । य‚स्मात् प्र‚त्य‚क्ष‚प‚रोक्षार्थ‚योर्नाग‚माधिकारः त‚स्मात् ‚{\color{DodgerBlue3}‚तृती‚{\tiny $_{lb}$}‚य‚स्थाने}‚ऽतीन्द्रिये विष‚ये विचार‚{\color{DodgerBlue3}‚संक्रान्तो शास्त्र‚प‚रिग्र‚हो न्याय्यः} । प्र‚कारान्त‚राभा‚{\tiny $_{lb}$}‚वात् । (५१)
	\pend% ending standard par
      \label{div_pvv.4.52}
	  
	% new div opening: depth here is 2
	
	  \bigskip
	  \begingroup
	
	    \large
	  
	    \begin{quote}
	  
	    
	    \stanza[\smallbreak]
	\label{pv.4.52}\flagstanza{\tiny\textenglish{...v.4.52}}त‚त्रापि साध्य‚ध‚र्म‚स्य नान्त‚रीय‚क‚बाध‚न‚म् ।&प‚रिहार्यं न चान्येषाम‚न‚व‚स्थाप्र‚स‚ङ्ग‚तः ॥ ५२ ॥\&[\smallbreak]


	
	    \end{quote}
	  
	  \endgroup
	

	  \pstart \leavevmode% starting standard par
	\hphantom{.}‚{\color{DodgerBlue3}‚त‚त्र} शास्त्रे प‚रि‚{\color{DodgerBlue3}‚ग्र‚हेपि} त‚दा साध‚यितुमार‚ब्ध‚स्य ‚{\color{DodgerBlue3}‚साध्य‚ध‚र्म‚स्य य‚न्नान्त‚रीय‚{\tiny $_{3}$}‚कं} स‚म्ब‚द्धं य‚था क्ष‚णिक‚त्व‚स्य नैरात्म्यं त‚स्य ‚{\color{DodgerBlue3}‚बाध‚नं} प‚रिहार्यं । व‚स्तुत‚स्तादात्म्याद‚न‚यो‚{\tiny $_{lb}$}‚र्नैरात्म्य‚बाध‚ने क्ष‚णिक‚त्व‚बाध‚न‚प्र‚स‚ङ्गात् । ‚{\color{DodgerBlue3}‚न त्व‚न्येषां} साध्य‚स्याकार‚णाव्याप‚क‚{\tiny $_{lb}$}‚भूतानां ध‚र्माणां बाध‚नं प‚रिहार्य‚म‚न‚व‚स्थाप्र‚स‚ङ्ग‚तः । न हि शास्त्र‚द‚र्शित‚स‚म्भ‚व‚ध‚र्म‚{\tiny $_{lb}$}‚व्याप्तिलिङ्ग‚स्य दृष्टान्ते द‚र्श‚यितुं श‚क्य‚ते येन क‚श्चिदाग‚माश्र‚यो हेतुर‚व‚{\tiny $_{lb}$}‚तिष्ठेत । (५२)
	\pend% ending standard par
      \label{div_pvv.4.53}
	  
	% new div opening: depth here is 2
	

	  \pstart \leavevmode% starting standard par
	न‚नु शास्त्र‚{\tiny $_{4}$}‚म‚न‚पेक्ष्य न वादः क‚र्त्त‚व्य इति व‚स्तुब‚ल‚प्र‚वृत्तानुमानेपि शास्त्रा‚{\tiny $_{lb}$}‚पेक्षेत्याह ।
	\pend% ending standard par
      
	  \bigskip
	  \begingroup
	
	    \large
	  
	    \begin{quote}
	  
	    
	    \stanza[\smallbreak]
	\label{pv.4.53}\flagstanza{\tiny\textenglish{...v.4.53}}केनेयं स‚र्व‚चिन्तासु शास्त्रं ग्राह्य‚मिति स्थितिः ।&कृतेदानीम‚सिद्धान्तैर्ग्राह्यो धूमेन नान‚लः ॥ ५३ ॥\&[\smallbreak]


	
	    \end{quote}
	  
	  \endgroup
	

	  \pstart \leavevmode% starting standard par
	\hphantom{.}‚{\color{DodgerBlue3}‚स‚र्व्वा}‚सु प‚रोक्षात्य‚न्त‚प‚रोक्षार्थ‚{\color{DodgerBlue3}‚चिन्तासु शास्त्रं ग्राह्य‚मिति केनेयं स्थितिः} कृता ‚{\tiny $_{lb}$}‚(।) नैत‚द‚नुम‚न्य‚न्ते विद्वान्सः । ‚{\color{DodgerBlue3}‚इदानीम}‚विदुषाम‚स्मिन्न‚भ्युप‚ग‚मेऽ‚{\color{DodgerBlue3}‚सिद्धान्तैः} सिद्धान्त‚{\tiny $_{lb}$}‚\leavevmode\ledsidenote{\textenglish{434/s}} विशेषाश्र‚य‚र‚हितैर्गोपाल‚कादि‚{\color{DodgerBlue3}‚भिर्द्धू मेन} लिङ्गेन ‚{\color{DodgerBlue3}‚नान‚लो ग्राह्यः\edtext{}{\edlabel{pvv.434-1}\label{pvv.434-1}\lemma{ग्राह्यः}\Bfootnote{शास्त्राप‚रिग्र‚हेण सिद्धिप्र‚तिब‚न्धात् ।}}} । (५३)
	\pend% ending standard par
      \label{div_pvv.4.54}
	  
	% new div opening: depth here is 2
	

	  \pstart \leavevmode% starting standard par
	न च क‚स्य‚चित् सिद्धान्त‚स‚म्ब‚न्धो युक्तः ।‚{\tiny $_{5}$}‚ त‚था हि स‚म्ब‚न्धो भ‚व‚न् स‚ह‚जो ‚{\tiny $_{lb}$}‚वा भ‚वेदौपाधिको वा (।) द्व‚य‚म‚पि निषेद्ध्ुमाह ।
	\pend% ending standard par
      
	  \bigskip
	  \begingroup
	
	    \large
	  
	    \begin{quote}
	  
	    
	    \stanza[\smallbreak]
	\label{pv.4.54}\flagstanza{\tiny\textenglish{...v.4.54}}रिक्त‚स्य ज‚न्तोर्ज्जात‚स्य गुण‚दोष‚म‚प‚श्य‚तः ।&विल‚ब्धा व‚त केनामी सिद्धान्त‚विष‚म‚ग्र‚हाः ॥ ५४ ॥\&[\smallbreak]


	
	    \end{quote}
	  
	  \endgroup
	

	  \pstart \leavevmode% starting standard par
	\hphantom{.}‚{\color{DodgerBlue3}‚रिक्त‚स्य} तुच्छ‚स्य सिद्धान्त‚र‚हित‚स्य ‚{\color{DodgerBlue3}‚ज‚न्तोर्जात‚स्या}‚नेन स‚ह‚ज\edtext{}{\edlabel{pvv.434-2}\label{pvv.434-2}\lemma{ज}\Bfootnote{स‚ह‚जः क‚र्णादिव‚त् । औपाधिकः स्व‚यं गुण‚दोष‚प‚रीक्ष‚याभ्युप‚ग‚च्छ‚तः ।}}स‚म्ब‚न्धाभाव‚{\tiny $_{lb}$}‚निमित्त‚मुक्तं । ‚{\color{DodgerBlue3}‚गुण‚दोषं} प्रामाण्याप्रामाण्य‚निब‚न्ध‚न‚{\color{DodgerBlue3}‚म‚प‚श्य‚तः\edtext{}{\edlabel{pvv.434-3}\label{pvv.434-3}\lemma{तः}\Bfootnote{आग‚म‚भात्रेण ।}}} । अनेनौपाधिक‚{\tiny $_{lb}$}‚स‚म्ब‚न्ध‚निमित्ताभाव उक्तः । \edtext{\textsuperscript{*}}{\edlabel{pvv.434-4}\label{pvv.434-4}\lemma{*}\Bfootnote{उप‚ह‚स‚ति ।}} ‚{\color{DodgerBlue3}‚केनान}‚र्थ‚प‚टीय‚सा ‚{\color{DodgerBlue3}‚सिद्धान्ता}\edtext{}{\edlabel{pvv.434-5}\label{pvv.434-5}\lemma{सा}\Bfootnote{त्याज‚यितुम‚श‚क्य‚त्वात् ।}}एव ‚{\color{DodgerBlue3}‚विष‚म‚ग्र‚हा} दुष्प‚रिहार‚त्वा‚{\color{DodgerBlue3}‚दिमे विल‚ब्धा} \edtext{\textsuperscript{*}}{\edlabel{pvv.434-6}\label{pvv.434-6}\lemma{*}\Bfootnote{अस्याय‚माग‚मो नास्येति ।}} ‚{\color{DodgerBlue3}‚व‚त} येषु स्वामि‚{\tiny $_{6}$}‚त्वेन ज‚न्त‚वो व्य‚व‚ह‚र‚न्ति । ‚{\tiny $_{lb}$}‚न हि क‚र्ण्ण‚नाश(? स)मिव सिद्धान्तः प्राण‚स‚ह‚जः । नापि दोषोर्ज्जित‚गुणो‚{\tiny $_{lb}$}‚प‚प‚न्नः प्र‚माण‚मिव विचारात् प्राक् सिद्धान्तः सिद्धो येन स‚म‚र्थ‚विष‚यः ‚{\tiny $_{lb}$}‚स्यात् । (५४)
	\pend% ending standard par
      \label{div_pvv.4.55}
	  
	% new div opening: depth here is 2
	

	  \pstart \leavevmode% starting standard par
	किञ्च (।)
	\pend% ending standard par
      
	  \bigskip
	  \begingroup
	
	    \large
	  
	    \begin{quote}
	  
	    
	    \stanza[\smallbreak]
	\label{pv.4.55}\flagstanza{\tiny\textenglish{...v.4.55}}य‚दि साध‚न एक‚त्र स‚र्वं शास्त्रं निद‚र्श‚ने ।&द‚र्श‚येत् साध‚नं स्यादित्येषा लोकोत्त‚रा स्थितिः ॥ ५५ ॥\&[\smallbreak]


	
	    \end{quote}
	  
	  \endgroup
	

	  \pstart \leavevmode% starting standard par
	\hphantom{.}‚{\color{DodgerBlue3}‚य‚दि साध‚न एक‚त्र स‚र्व्वं शास्त्रं} शास्त्रार्थं ‚{\color{DodgerBlue3}‚निद‚र्श‚ने} दृष्टान्ते वादी ‚{\color{DodgerBlue3}‚द‚र्श‚येत्} त‚दा ‚{\tiny $_{lb}$}‚त‚त् ‚{\color{DodgerBlue3}‚साध‚नं स्यात्} (।) न त्वेक‚स्य शास्त्र‚द‚र्शित‚ध‚र्म‚स्यान्व‚ये । एत‚च्च न क्व‚चित् ‚{\tiny $_{lb}$}‚\leavevmode\ledsidenote{\textenglish{87a/MA}} साध‚ने क‚र्त्तुं श‚क्य‚{\color{DodgerBlue3}‚मिति लोका}‚ति‚{\tiny $_{7}$}‚क्रान्ता ‚{\color{DodgerBlue3}‚स्थितिरेषा} । (५५)
	\pend% ending standard par
      \label{div_pvv.4.56}
	  
	% new div opening: depth here is 2
	

	  \pstart \leavevmode% starting standard par
	अपि च (।)
	\pend% ending standard par
      
	  \bigskip
	  \begingroup
	
	    \large
	  
	    \begin{quote}
	  
	    
	    \stanza[\smallbreak]
	\label{pv.4.56}\flagstanza{\tiny\textenglish{...v.4.56}}अस‚म्ब‚द्ध‚स्य ध‚र्म‚स्य किम‚सिद्धौ न सिद्ध‚ति ।&हेतुस्त‚त्साध‚नायोक्तः किं दुष्ट‚स्त‚त्र सिध्य‚ति ॥ ५६ ॥\&[\smallbreak]


	
	    \end{quote}
	  
	  \endgroup
	

	  \pstart \leavevmode% starting standard par
	\hphantom{.}‚{\color{DodgerBlue3}‚अस‚म्ब‚द्ध‚स्य} साध‚यितुम‚प्र‚वृत्त‚स्य ‚{\color{DodgerBlue3}‚ध‚र्म}‚स्याकाश‚गुण‚त्वादेः कृत‚क‚त्वाद्धेतोर्व्वि‚{\tiny $_{lb}$}‚प‚र्यास‚नाद‚{\color{DodgerBlue3}‚सिद्धौ} स‚त्यां सिसाध‚यिषितं हेतुव्याप‚क‚म‚नित्य‚त्वं ‚{\color{DodgerBlue3}‚किं न सिध्य‚ति} । न हि ‚{\tiny $_{lb}$}‚व्याप‚क‚म‚न्त‚रेण व्याप्यं भ‚व‚ति । ‚{\color{DodgerBlue3}‚हेतुस्त}‚स्य व्याप‚क‚स्य ‚{\color{DodgerBlue3}‚साध‚नायोक्तः किन्दुष्टः} \leavevmode\ledsidenote{\textenglish{435/s}} ‚{\color{DodgerBlue3}‚त‚त्र} स्व‚साध्ये ‚{\color{DodgerBlue3}‚सिध्य‚ति} साध्य‚प्र‚तिपाद‚नं साध्य‚व्यापारः‚{\color{DodgerBlue3}‚त‚च्चेद‚स्ति क‚थं दुष्टः} ।(५६)
	\pend% ending standard par
      \label{div_pvv.4.57}
	  
	% new div opening: depth here is 2
	

	  \pstart \leavevmode% starting standard par
	\hphantom{.}शास्त्रार्थ‚बाध‚नेऽभिम‚त‚स्यापि न सिद्धिरिति चेत् । ‚{\color{DodgerBlue3}‚आह} ॥
	\pend% ending standard par
      
	  \bigskip
	  \begingroup
	
	    \large
	  
	    \begin{quote}
	  
	    
	    \stanza[\smallbreak]
	\label{pv.4.57}\flagstanza{\tiny\textenglish{...v.4.57}}ध‚र्मान‚नुप‚नीयैव दृष्टान्ते ध‚र्मिणोऽखिलान् ।&वाग्धूमादेर्ज‚नोन्वेति चैत‚न्य‚द‚ह‚नादिक‚म् ॥ ५७ ॥\&[\smallbreak]


	
	    \end{quote}
	  
	  \endgroup
	

	  \pstart \leavevmode% starting standard par
	ध‚र्मिणो‚{\tiny $_{1}$}‚‚{\color{DodgerBlue3}‚ध‚र्मान्} शास्त्र‚द‚र्शिता‚{\color{DodgerBlue3}‚न‚खिलान्} हेतुव्याप‚क‚त्वे‚{\color{DodgerBlue3}‚नानुप‚नीया}‚प्र‚द‚र्श्यं ‚{\tiny $_{lb}$}‚‚{\color{DodgerBlue3}‚वाग्धूमा}‚देर्हेतो‚{\color{DodgerBlue3}‚श्चैत‚न्य‚द‚ह‚नादिकं} य‚थाक्र‚मं स्व‚स‚न्तान\edtext{}{\edlabel{pvv.435-1}\label{pvv.435-1}\lemma{न्तान}\Bfootnote{चेत‚नोयं व‚च‚नाद‚ह‚मिव ।}}व‚न्म‚हान‚स‚व‚च्च ‚{\color{DodgerBlue3}‚ज‚नोऽन्वेति} प्र‚तिप‚द्य‚ते ॥ (५७)
	\pend% ending standard par
      \label{div_pvv.4.58}
	  
	% new div opening: depth here is 2
	

	  \pstart \leavevmode% starting standard par
	किञ्च (।)
	\pend% ending standard par
      
	  \bigskip
	  \begingroup
	
	    \large
	  
	    \begin{quote}
	  
	    
	    \stanza[\smallbreak]
	\label{pv.4.58}\flagstanza{\tiny\textenglish{...v.4.58}}स्व‚भावं कार‚णं चार्थोऽव्य‚भिचारेण साध‚य‚न् ।&क‚स्य‚चिद् वाद‚बाधायां स्व‚भावान्न निव‚र्त‚ते ॥ ५८ ॥\&[\smallbreak]


	
	    \end{quote}
	  
	  \endgroup
	

	  \pstart \leavevmode% starting standard par
	\hphantom{.}‚{\color{DodgerBlue3}‚स्व‚भावं} व्याप‚कं वृक्ष‚त्वादि ‚{\color{DodgerBlue3}‚कार‚णं} व‚ह्न्यादि ‚{\color{DodgerBlue3}‚चार्थो} व्याप्यः शिंश‚पादिः ‚{\color{DodgerBlue3}‚कार्यं} धूमादिर‚{\color{DodgerBlue3}‚व्य‚भिचारेणा}‚विनाभावित्वात् ‚{\color{DodgerBlue3}‚साध‚य‚न् क‚स्य‚चित्} शास्त्र‚प‚राधीन‚स्य ‚{\color{DodgerBlue3}‚वा}‚दिनो ‚{\tiny $_{lb}$}‚वाद‚स्य ‚{\color{DodgerBlue3}‚बाधायां स्व‚भावात्} व्या‚{\tiny $_{2}$}‚प‚क‚कार‚ण‚ग‚म‚कान्न ‚{\color{DodgerBlue3}‚निव‚र्त्त‚ते} त‚तः ‚{\color{DodgerBlue3}‚शास्त्रेषु} ध‚र्मान्त‚र‚व्याह‚ताव‚पि हेतुः साध्यीकृतं स्व‚स‚म्ब‚द्ध‚म‚र्थं प्र‚तिपाद‚य‚ति । (५८)
	\pend% ending standard par
      \label{div_pvv.4.59}
	  
	% new div opening: depth here is 2
	

	  \pstart \leavevmode% starting standard par
	त‚त‚श्च ।
	\pend% ending standard par
      
	  \bigskip
	  \begingroup
	
	    \large
	  
	    \begin{quote}
	  
	    
	    \stanza[\smallbreak]
	\label{pv.4.59}\flagstanza{\tiny\textenglish{...v.4.59}}प्र‚प‚द्य‚मान‚श्चान्य‚स्तं नान्त‚रीय‚क‚मीप्सितैः ।&साध्यार्थैहेतुना तेन क‚थ‚म‚प्र‚तिपादितः ॥ ५९ ॥\&[\smallbreak]


	
	    \end{quote}
	  
	  \endgroup
	

	  \pstart \leavevmode% starting standard par
	\hphantom{.}‚{\color{DodgerBlue3}‚साध्यैर‚र्थैरीप्सितै}‚राप्तं प्र‚त्येतुम‚ष्टै‚{\color{DodgerBlue3}‚र्न्नान्तिरीय}‚क‚म‚विनाभाविनं ‚{\color{DodgerBlue3}‚तं} हेतुं ‚{\color{DodgerBlue3}‚प्र‚प‚द्य‚मानः} प्र‚तिप‚द्य‚मानो‚{\color{DodgerBlue3}‚न्यः} प्र‚तिवादी ‚{\color{DodgerBlue3}‚क‚थ‚न्तेन हेतुनाऽप्र‚तिपादितः} साध्य‚नान्त‚रीय‚क‚त‚या ‚{\tiny $_{lb}$}‚क्व‚चिद्ध‚र्मिणि साध‚न‚प्र‚तीतिरेव हि साध्य‚प्र‚तीतिः सा चास्ति प्र‚तिवा‚{\tiny $_{3}$}‚दिनः ॥ (५९)
	\pend% ending standard par
      \label{div_pvv.4.60}
	  
	% new div opening: depth here is 2
	

	  \pstart \leavevmode% starting standard par
	किञ्च (।)
	\pend% ending standard par
      

	  \pstart \leavevmode% starting standard par
	हेतुना\edtext{}{\edlabel{pvv.435-2}\label{pvv.435-2}\lemma{हेतुना}\Bfootnote{प्र‚कृत‚साध‚केन य‚था कृत‚क‚त्वेनाकाश‚गुण‚त्वादि ।}} यः शास्त्रार्थो बाध्य‚ते किन्त‚स्मिन् साध्ये वादीना हेतुर‚साधूक्तः । ‚{\tiny $_{lb}$}‚आहोश्वि(? स्वि)त् त‚त्र साध्ये हेतुरुच्य‚तां मा वा व‚स्तुत‚स्त‚द्वाध‚कोसौ हेतुरिति ‚{\tiny $_{lb}$}‚दुष्ट‚ता । त‚त्राद्य‚प‚क्षे भ‚व‚त्येव दोषो य‚द्येव‚मिष्य‚ते (।)
	\pend% ending standard par
      

	  \pstart \leavevmode% starting standard par
	अथ (।)
	\pend% ending standard par
      
	  \bigskip
	  \begingroup
	
	    \large
	  
	    \begin{quote}
	  
	    
	    \stanza[\smallbreak]
	\label{pv.4.60}\flagstanza{\tiny\textenglish{...v.4.60}}उक्तोऽनुक्तोपि वा हेतुर्विरोद्धा वादिनोत्र किम् ।&न हि त‚स्योक्तिदोषेण स जातः शास्त्र‚बाध‚नः ॥ ६० ॥\&[\smallbreak]


	
	    \end{quote}
	  
	  \endgroup
	\textsuperscript{\textenglish{436/s}}

	  \pstart \leavevmode% starting standard par
	\hphantom{.}‚{\color{DodgerBlue3}‚उक्तोऽनुक्तोपि} वा ‚{\color{DodgerBlue3}‚हेतुः} व‚स्तुत एव त‚स्य विरोद्धा प्र‚तिघात‚क‚स्त‚दा ‚{\tiny $_{lb}$}‚त‚त्र शास्त्रार्थ‚बाध‚ने ‚{\color{DodgerBlue3}‚वादिनः किन्दू}‚ष‚णं न किञ्चित् । ‚{\color{DodgerBlue3}‚हि} य‚स्मात् ‚{\color{DodgerBlue3}‚त‚स्य}‚{\tiny $_{4}$}‚ वादिन ‚{\tiny $_{lb}$}‚‚{\color{DodgerBlue3}‚उक्तिदोषेण} स कृत‚क‚त्वादिहेतुः ‚{\color{DodgerBlue3}‚शास्त्र}‚स्य शास्त्रार्थ‚स्याकाश‚गुण‚त्वादेः ‚{\color{DodgerBlue3}‚बाध‚नो ‚{\tiny $_{lb}$}‚बाध‚को न जातः} ॥ (६०)
	\pend% ending standard par
      \label{div_pvv.4.61}
	  
	% new div opening: depth here is 2
	
	  \bigskip
	  \begingroup
	
	    \large
	  
	    \begin{quote}
	  
	    
	    \stanza[\smallbreak]
	\label{pv.4.61}\flagstanza{\tiny\textenglish{...v.4.61}}बाध‚क‚स्याभिधानाच्चेद् दोषो य‚दि व‚देन्न सः ।&किन्न बाधेत सोऽकुर्व‚न्न‚युक्तं केन दुष्य‚ति ॥ ६१ ॥\&[\smallbreak]


	
	    \end{quote}
	  
	  \endgroup
	

	  \pstart \leavevmode% starting standard par
	\hphantom{.}शास्त्रार्थ‚{\color{DodgerBlue3}‚बाध‚क‚स्य} हेतोर‚{\color{DodgerBlue3}‚भिधानात्} वादिनोपि ‚{\color{DodgerBlue3}‚दोष‚श्चेत् य‚दि} तं हेतुं ‚{\color{DodgerBlue3}‚न व‚देत् ‚{\tiny $_{lb}$}‚स} वादी । त‚दा किम‚सौ हेतुः शास्त्रार्थ न ‚{\color{DodgerBlue3}‚बाधेत} । व‚स्तुत‚स्त‚द्विरोधित्वाद‚व‚श्यं ‚{\tiny $_{lb}$}‚बाध‚ते । त‚त‚श्च स वाद्य‚{\color{DodgerBlue3}‚कुर्व‚न्न‚युक्तं केन} कार‚णेन ‚{\color{DodgerBlue3}‚दुष्य‚ति} प‚राजितः स्यात् ॥ (६९)
	\pend% ending standard par
      \label{div_pvv.4.62}
	  
	% new div opening: depth here is 2
	

	  \pstart \leavevmode% starting standard par
	न‚नु य‚दि दुष्ट‚हेतु‚{\tiny $_{5}$}‚व‚च‚नेपि न वादिनो दुष्ट‚ता । त‚दाऽसिद्धादिव‚च‚नेपि न ‚{\tiny $_{lb}$}‚दोषः स्यादित्याह ।
	\pend% ending standard par
      
	  \bigskip
	  \begingroup
	
	    \large
	  
	    \begin{quote}
	  
	    
	    \stanza[\smallbreak]
	\label{pv.4.62}\flagstanza{\tiny\textenglish{...v.4.62}}अन्येषु हेत्वाभासेषु स्वेष्ट‚स्येवाप्र‚साध‚नात् ।&दुष्येद् व्य‚र्थाभिधानेन नात्र त‚स्य प्र‚साध‚नात् ॥ ६२ ॥\&[\smallbreak]


	
	    \end{quote}
	  
	  \endgroup
	

	  \pstart \leavevmode% starting standard par
	\hphantom{.}‚{\color{DodgerBlue3}‚अन्येष्व}‚सिद्धादिषु ‚{\color{DodgerBlue3}‚हेत्वाभासेषु} वाद्युक्तेषु ‚{\color{DodgerBlue3}‚स्वेष्ट‚स्य} वादीष्ट‚स्यै‚{\color{DodgerBlue3}‚वाप्र‚साध‚नाद्} वादी ‚{\color{DodgerBlue3}‚दुष्य‚ति । व्य‚र्थ}‚स्य साध्य‚साध‚नानुप‚युक्त‚स्य साध‚न‚स्या‚{\color{DodgerBlue3}‚भिधानात्}\edtext{}{\edlabel{pvv.436-1}\label{pvv.436-1}\lemma{स्या}\Bfootnote{प्र‚योग‚वैफ‚ल्येन ।}} । ‚{\color{DodgerBlue3}‚अत्र} कृत‚क‚त्वे तु वाद्युक्ते वाञ्छित‚स्यानित्य‚त्व‚स्य ‚{\color{DodgerBlue3}‚प्र‚साध‚नान्न} वादी दुष्य‚ति । शास्त्रा‚{\tiny $_{lb}$}‚र्थे तु वाद्य‚निष्टे बाध्य‚माने शास्त्र‚मेव दुष्टं भ‚विष्य‚ति ॥ (६२)
	\pend% ending standard par
      \label{div_pvv.4.63}
	  
	% new div opening: depth here is 2
	

	  \pstart \leavevmode% starting standard par
	य‚द्य‚पि स्वेष्ट‚{\tiny $_{6}$}‚स्य तेन साध‚नं त‚थापि शास्त्रार्थ‚स्य बाध‚न‚मिति दुष्ट एवेत्याह ।
	\pend% ending standard par
      
	  \bigskip
	  \begingroup
	
	    \large
	  
	    \begin{quote}
	  
	    
	    \stanza[\smallbreak]
	\label{pv.4.63}\flagstanza{\tiny\textenglish{...v.4.63}}य‚दि किञ्चित् क्व‚चित् शास्त्रे न युक्तं प्र‚तिषिध्य‚ते ।&ब्रुवाणो युक्त‚म‚प्य‚न्य‚दिति राज‚कुल‚स्थितिः ॥ ६३ ॥\&[\smallbreak]


	
	    \end{quote}
	  
	  \endgroup
	

	  \pstart \leavevmode% starting standard par
	\hphantom{.}‚{\color{DodgerBlue3}‚क्व‚चिद्} वै शे षि का दि‚{\color{DodgerBlue3}‚शास्त्रे} निर्दिष्टं ‚{\color{DodgerBlue3}‚य‚दि किञ्चि}‚दाकाश‚गुण‚त्वादि बाध्य‚{\tiny $_{lb}$}‚मान‚त्वा‚{\color{DodgerBlue3}‚द‚युक्तं} । ताव‚ताऽ‚{\color{DodgerBlue3}‚न्य‚द}‚नित्य‚त्वादि‚{\color{DodgerBlue3}‚युक्त‚म‚पि} कृत‚क‚त्व‚हेतुना ‚{\color{DodgerBlue3}‚ब्रुवाणः} प्र‚ति‚{\tiny $_{lb}$}‚पाद‚य‚न् ‚{\color{DodgerBlue3}‚प्र‚तिषिध्य‚ते} शास्त्रार्थ‚बाध‚नात् विरोधो\edtext{}{\edlabel{pvv.436-2}\label{pvv.436-2}\lemma{विरोधो}\Bfootnote{अयुक्तं त्व‚योक्त‚मिति ।}}प‚न्यासेनेति व्य‚क्त‚मियं ‚{\color{DodgerBlue3}‚राज‚कुल}‚{\tiny $_{lb}$}‚\leavevmode\ledsidenote{\textenglish{87b/MA}} ‚{\color{DodgerBlue3}‚स्थितिः} । राज‚शास‚न‚स्यैव ब‚ल‚प्र‚वृत्त‚स्य युक्तायुक्त‚विचार‚{\tiny $_{7}$}‚णाब‚हिर्भावात् । (६३)
	\pend% ending standard par
      \label{div_pvv.4.64}
	  
	% new div opening: depth here is 2
	

	  \pstart \leavevmode% starting standard par
	किञ्च (।)
	\pend% ending standard par
      
	  \bigskip
	  \begingroup
	
	    \large
	  
	    \begin{quote}
	  
	    
	    \stanza[\smallbreak]
	\label{pv.4.64}\flagstanza{\tiny\textenglish{...v.4.64}}स‚र्वान‚र्थान् स‚मीकृत्य व‚क्तुं श‚क्यं न साध‚न‚म् ।&स‚र्व‚त्र तेन सुच्छ‚न्नेयं साध्य‚साध‚न‚संस्थितिः ॥ ६४ ॥\&[\smallbreak]


	
	    \end{quote}
	  
	  \endgroup
	\textsuperscript{\textenglish{437/s}}

	  \pstart \leavevmode% starting standard par
	\hphantom{.}‚{\color{DodgerBlue3}‚स‚र्व्वान्} शास्त्र‚दृष्टा‚{\color{DodgerBlue3}‚न‚र्थान्} साध्य‚त्वेन ‚{\color{DodgerBlue3}‚स‚मीकृत्य} किञ्चित् ‚{\color{DodgerBlue3}‚साध‚नं व‚क्तु‚{\tiny $_{lb}$}‚म‚श‚क्यं} दृष्टान्ते शास्त्र‚दृष्टाखिल‚ध‚र्म‚व्याप्त्य‚नुप‚ल‚म्भात् । ‚{\color{DodgerBlue3}‚तेन} कार‚णेन ‚{\color{DodgerBlue3}‚स‚र्व्व‚त्र} ध‚र्मिणि ‚{\color{DodgerBlue3}‚साध्य‚साध‚न‚योः संस्थिति}‚र्व्य‚व‚{\color{DodgerBlue3}‚स्थेयं सुच्छ‚न्ना} स्यात् ॥ (६४)
	\pend% ending standard par
      \label{div_pvv.4.65}
	  
	% new div opening: depth here is 2
	

	  \begin{center}%% label @type='head'
	\textbf{(श‚ब्द‚स्य नाकाश‚गुण‚त्व‚म्)}
	\end{center}
	

	  \pstart \leavevmode% starting standard par
	य‚दि त‚र्ह्याकाश‚गुण‚त्वाभावेप्य‚नित्य‚त्वं सिध्य‚द‚बाध्यं श‚ब्दे त‚दा श्राव‚ण‚त्वादि‚{\tiny $_{lb}$}‚हेतुना नित्य‚त्व‚म‚पि साध्य‚मान‚म‚बाध्यं स्यादित्याह ।
	\pend% ending standard par
      
	  \bigskip
	  \begingroup
	
	    \large
	  
	    \begin{quote}
	  
	    
	    \stanza[\smallbreak]
	\label{pv.4.65}\flagstanza{\tiny\textenglish{...v.4.65}}विरुद्ध‚योरेक‚ध‚र्मिण्य‚योगाद‚स्तु बाध‚न‚म् ।&विरुद्धैकान्तिके नात्र त‚द्व‚द‚स्ति विरोधिता ॥ ६५ ॥\&[\smallbreak]


	
	    \end{quote}
	  
	  \endgroup
	

	  \pstart \leavevmode% starting standard par
	\hphantom{.}‚{\color{DodgerBlue3}‚विरुद्ध‚यो}‚र्न्नित्य‚त्वानित्य‚त्व‚यो‚{\tiny $_{1}$}‚‚{\color{DodgerBlue3}‚रेक‚त्र} श‚ब्दे ‚{\color{DodgerBlue3}‚ध‚र्मिण्य‚योगाद् विरुद्धैकान्तिके} विरुद्धाव्य‚भिचारिणि श्राव‚ण‚त्वादौ नित्य‚त्व‚साध‚के ‚{\color{DodgerBlue3}‚बाध‚न‚म‚स्तु} । न‚ह्येक‚त्र ध‚र्मिणि ‚{\tiny $_{lb}$}‚विरुद्धौ ध‚र्मो भ‚वितुम‚र्ह‚तः । ‚{\color{DodgerBlue3}‚त‚द्व‚न्नि}‚त्य‚त्व‚योरिवात्रान‚योः प्र‚कृताप्र‚कृत‚योर‚नित्य‚त्वा‚{\tiny $_{lb}$}‚काश‚गुण‚त्वाभाव‚यो‚{\color{DodgerBlue3}‚र्व्विरोधिता} नास्ति । त‚तः कृत‚क‚त्वाच्छ‚ब्देऽनित्य‚त्व‚सिद्धावाकाश‚{\tiny $_{lb}$}‚गुण‚त्वाभावो न बाध्य‚ते ॥ (६५)
	\pend% ending standard par
      \label{div_pvv.4.66}
	  
	% new div opening: depth here is 2
	

	  \pstart \leavevmode% starting standard par
	स्यादेत‚त् । प्र‚कृताप्र‚कृत‚योर‚नित्य‚त्वाकाश‚{\tiny $_{2}$}‚गुण‚त्वाभाव‚योः प‚र‚स्प‚रं (।)
	\pend% ending standard par
      
	  \bigskip
	  \begingroup
	
	    \large
	  
	    \begin{quote}
	  
	    
	    \stanza[\smallbreak]
	\label{pv.4.66}\flagstanza{\tiny\textenglish{...v.4.66}}अबाध्य‚बाध‚क‚त्वेपि त‚योः शास्त्रार्थ‚विल्प‚वात् ।&अस‚म्ब‚न्धेपि बाधा चेत् स्यात् स‚र्वं स‚र्व‚बाध‚न‚म् ॥ ६६ ॥\&[\smallbreak]


	
	    \end{quote}
	  
	  \endgroup
	

	  \pstart \leavevmode% starting standard par
	\hphantom{.}‚{\color{DodgerBlue3}‚बाध्य‚बाध‚क}‚त्वाभा‚{\color{DodgerBlue3}‚वेप्ये}‚क‚स्मिन्न‚नित्य‚त्वे कृत‚क‚त्वात् सिध्य‚ति ‚{\color{DodgerBlue3}‚श‚ब्दे ध‚र्मिणि ‚{\tiny $_{lb}$}‚शास्त्रार्थ‚स्य शास्त्रा}‚भ्युप‚ग‚त‚स्याकाश‚गुण‚त्व‚स्य ‚{\color{DodgerBlue3}‚विप्ल‚वात्} कार‚णाद् अस‚म्ब‚द्धे (।) ‚{\tiny $_{lb}$}‚अप्र‚कृताकाश‚गुण‚त्व‚{\color{DodgerBlue3}‚स‚म्ब‚न्ध‚र‚हि}‚तेऽनित्य‚त्वे‚{\color{DodgerBlue3}‚पि बाधा} भ‚व‚तीति ‚{\color{DodgerBlue3}‚चेत्} । एव‚न्त‚र्हि ‚{\tiny $_{lb}$}‚प्र‚य‚न्तान‚न्त‚रीय‚क‚त्वाद् ग‚न्धे पृथिवीगुण‚त्व‚बाध‚ने ‚{\color{DodgerBlue3}‚स‚र्व्वं} कृत‚क‚त्वादि ‚{\color{DodgerBlue3}‚स‚र्व्व}‚{\tiny $_{lb}$}‚स्यानित्य‚त्वादेः साध्य‚स्य ‚{\color{DodgerBlue3}‚बाध‚नं} ‚{\tiny $_{3}$}‚ ‚{\color{DodgerBlue3}‚स्यात्} । श‚ब्दादौ ध‚र्मिण्य‚प्र‚कृत‚शास्त्रार्थ‚{\tiny $_{lb}$}‚बाध‚न‚स्य तुल्य‚त्वात् ॥ (६६)
	\pend% ending standard par
      \label{div_pvv.4.67}
	  
	% new div opening: depth here is 2
	
	  \bigskip
	  \begingroup
	
	    \large
	  
	    \begin{quote}
	  
	    
	    \stanza[\smallbreak]
	\label{pv.4.67}\flagstanza{\tiny\textenglish{...v.4.67}}स‚म्ब‚न्ध‚स्तेन त‚स्यैव बाध‚नाद‚स्ति चेद‚स‚त् ।&हेतोः स‚र्व्व‚स्य चिन्त्य‚त्वात् स्व‚साध्ये गुण‚दोष‚योः ॥ ६७ ॥\&[\smallbreak]


	
	    \end{quote}
	  
	  \endgroup
	

	  \pstart \leavevmode% starting standard par
	\hphantom{.}अथ त‚त्र श‚ब्द एव ध‚र्मिणि आकाश‚गुण‚त्व‚स्य स‚त्त्वात् ‚{\color{DodgerBlue3}‚स‚म्ब‚न्धोस्ति तेन} कृत‚क‚त्वात् ‚{\color{DodgerBlue3}‚त‚स्यैव बाध‚नाद्} विरोधः । पृथिवीगुण‚त्व‚न्तु श‚ब्दे ध‚र्मिण्य‚स‚म्ब‚द्धं । ‚{\tiny $_{lb}$}‚त‚त‚स्त‚द्बाध‚नेपि श‚ब्दे कृत‚क‚त्व‚म‚विरुद्ध‚मिति ‚{\color{DodgerBlue3}‚चेत् । अस‚दे}‚त‚त् । ‚{\color{DodgerBlue3}‚स‚र्व्व‚स्य हेतोः ‚{\tiny $_{lb}$}‚स्व‚साध्ये} प्र‚कृते ‚{\color{DodgerBlue3}‚गुण‚दोष‚योश्चिन्त्य‚त्वात्} । य‚त्पुन‚र‚प्र‚कृतं ध‚र्मिस‚म्ब‚द्ध‚म‚पि ‚{\tiny $_{4}$}‚ न ‚{\tiny $_{lb}$}‚त‚त् साध्यं । त‚द्वाध‚नेपि न काचित् । क्ष‚तिः ॥ (६७)
	\pend% ending standard par
      \label{div_pvv.4.68}
	  
	% new div opening: depth here is 2
	

	  \pstart \leavevmode% starting standard par
	\leavevmode\ledsidenote{\textenglish{438/s}}किञ्च (।) ध‚र्मिणि स‚त्तामात्रं न स‚म्ब‚न्धः । किन्तु (।)
	\pend% ending standard par
      
	  \bigskip
	  \begingroup
	
	    \large
	  
	    \begin{quote}
	  
	    
	    \stanza[\smallbreak]
	\label{pv.4.68}\flagstanza{\tiny\textenglish{...v.4.68}}नान्त‚रीय‚क‚ता साध्ये स‚म्ब‚न्धः सेह नेक्ष्य‚ते ।&केव‚लं शास्त्र‚पीडेति दोषः सान्य‚कृते स‚मा ॥ ६८ ॥\&[\smallbreak]


	
	    \end{quote}
	  
	  \endgroup
	

	  \pstart \leavevmode% starting standard par
	\hphantom{.}‚{\color{DodgerBlue3}‚साध्ये नान्त‚रीय‚क‚ता\edtext{}{\edlabel{pvv.438-1}\label{pvv.438-1}\lemma{ता}\Bfootnote{साध्य‚स्य त‚द‚न्य‚नान्त‚रीय‚क‚ता य‚थाऽनित्य‚त्व‚स्य दुःखादिनान्त‚रीय‚क‚ता ।}}} साध्याविनाभावित्वं ‚{\color{DodgerBlue3}‚स‚म्ब‚न्ध उच्य‚ते (।) सा} साध्य‚{\tiny $_{lb}$}‚नान्त‚रीय‚ता ‚{\color{DodgerBlue3}‚इह} प्र‚कृताकाश‚गुण‚त्व‚बाध‚ने स‚ति नेक्ष्य‚ते \edtext{}{\edlabel{pvv.438-2}\label{pvv.438-2}\lemma{ते}\Bfootnote{केव‚लं संयोगादिविप‚र्यास‚नाच्छास्त्र‚वाचा ।}} (।) य‚द्य‚पि स‚त्व‚नान्त‚{\tiny $_{lb}$}‚रीय‚क‚माकाश‚गुण‚त्वं श‚ब्दे स्यात् न बाध्येत । ‚{\color{DodgerBlue3}‚केव‚लं} शास्त्राभ्युप‚ग‚त‚ध‚र्म‚बाध‚ना‚{\tiny $_{lb}$}‚‚{\color{DodgerBlue3}‚च्छास्त्र‚पीडे}‚ति दोषः । ‚{\color{DodgerBlue3}‚सा} च कृत‚क‚त्वाद‚नित्य‚त्व‚{\tiny $_{5}$}‚सिद्धौ दृश्य‚ते शास्त्र‚पीडा‚{\tiny $_{lb}$}‚‚{\color{DodgerBlue3}‚ऽन्ये}‚न प्र‚य‚त्नान‚न्त‚रीय‚क‚त्वादिना ग‚न्धे पृथिवीगुण‚त्व‚बाध‚नेपि ‚{\color{DodgerBlue3}‚स‚मेति} कृत‚क‚त्वं ‚{\tiny $_{lb}$}‚श‚ब्दे विरुद्धं स्यात् ॥ (६८)
	\pend% ending standard par
      \label{div_pvv.4.69}
	  
	% new div opening: depth here is 2
	

	  \pstart \leavevmode% starting standard par
	\hphantom{.}य‚द‚प्याहुरा‚{\color{DodgerBlue3}‚चार्यीयाः} शास्त्र‚म‚भ्युप‚ग‚म्य य‚दा वादः क्रिय‚ते त‚दा शास्त्र‚दृष्ट‚स्य ‚{\tiny $_{lb}$}‚स‚क‚ल‚स्य ध‚र्म‚स्य साध्य‚तेत्य‚त्राह ।
	\pend% ending standard par
      
	  \bigskip
	  \begingroup
	
	    \large
	  
	    \begin{quote}
	  
	    
	    \stanza[\smallbreak]
	\label{pv.4.69}\flagstanza{\tiny\textenglish{...v.4.69}}शास्त्राभ्युप‚ग‚मात् साध्यः शास्त्र‚दृष्टोऽखिलो य‚दि ।&प्र‚तिज्ञाऽसिद्ध‚दृष्टान्त‚हेतुवादः प्र‚स‚ज्य‚ते ॥ ६९ ॥\&[\smallbreak]


	
	    \end{quote}
	  
	  \endgroup
	

	  \pstart \leavevmode% starting standard par
	\hphantom{.}‚{\color{DodgerBlue3}‚शास्त्राभ्युप‚ग‚माच्छास्त्र‚दृष्टोऽखिलो} ध‚र्मो ‚{\color{DodgerBlue3}‚य‚दि साध्य} इष्य‚ते त‚दाऽ‚{\color{DodgerBlue3}‚सिद्ध‚योः} शास्त्रोद्दिष्ट‚यो‚{\color{DodgerBlue3}‚र्हेतुदृष्टान्त‚योर्व्वादः} ‚{\tiny $_{6}$}‚ ‚{\color{DodgerBlue3}‚प्र‚तिज्ञा} साध्यं ‚{\color{DodgerBlue3}‚प्र‚स‚ज्य‚ते} । शास्त्रे दृष्ट‚स्या‚{\tiny $_{lb}$}‚सिद्ध‚स्य साध्य‚त्वात् ॥ (६९)
	\pend% ending standard par
      \label{div_pvv.4.70}
	  
	% new div opening: depth here is 2
	

	  \pstart \leavevmode% starting standard par
	स्यादेत‚त् । किन्तु (।)
	\pend% ending standard par
      
	  \bigskip
	  \begingroup
	
	    \large
	  
	    \begin{quote}
	  
	    
	    \stanza[\smallbreak]
	\label{pv.4.70}\flagstanza{\tiny\textenglish{...v.4.70}}उक्त‚योः साध‚न‚त्वेन नो चेदीप्सित‚वाद‚तः ।&न्याय‚प्राप्तं न साध्य‚त्वं व‚च‚नाद् विनिव‚र्त्त‚ते ॥ ७० ॥\&[\smallbreak]


	
	    \end{quote}
	  
	  \endgroup
	

	  \pstart \leavevmode% starting standard par
	\hphantom{.}‚{\color{DodgerBlue3}‚वादिनेप्सित}‚स्य साध्य‚त्वेन वाद‚तः स्व‚यं साध्य‚त्वेनेप्सितः प‚क्षो विरुद्ध‚त्वा‚{\tiny $_{lb}$}‚न्निराकृत इत्यादिकात् \edtext{}{\edlabel{pvv.438-3}\label{pvv.438-3}\lemma{इत्यादिकात्}\Bfootnote{साध्य‚त्वेनेप्सित इति कृतं अन्य‚त्र स्व‚रूपेणैवेत्य‚व‚धार‚ण‚म‚तः ।}} ‚{\color{DodgerBlue3}‚साध‚न‚त्वेनोक्त‚यो}‚र‚सिद्ध‚हेतुदृष्टान्त‚योः साध्य‚ता ‚{\color{DodgerBlue3}‚नो ‚{\tiny $_{lb}$}‚चेत्} । न‚न्व‚सिद्ध‚स्य शास्त्राभ्युप‚ग‚त‚स्य ‚{\color{DodgerBlue3}‚साध्य‚त्वं न्याय‚प्राप्तं व‚च‚न}‚मात्रादीप्सित‚{\tiny $_{lb}$}‚साध्य‚त्वं प्र‚तिपाद‚क‚{\color{DodgerBlue3}‚त्वान्न विनिव‚र्त्त‚ते} \edtext{\textsuperscript{*}}{\edlabel{pvv.438-4}\label{pvv.438-4}\lemma{*}\Bfootnote{और्ण्य‚मिवाग्नेः ।}}॥ (७०)
	\pend% ending standard par
      \label{div_pvv.4.71}
	  
	% new div opening: depth here is 2
	\textsuperscript{\textenglish{439/s}}
	  \bigskip
	  \begingroup
	
	    \large
	  
	    \begin{quote}
	  
	    
	    \stanza[\smallbreak]
	\label{pv.4.71}\flagstanza{\tiny\textenglish{...v.4.71}}अनीप्सित‚म‚साध्य‚ञ्चेद् वादिनान्योप्य‚नीप्सितः ।&ध‚र्मोऽसाध्य‚स्त‚दाऽसाध्यं बाध‚मानं विरोधि किम् ॥ ७१ ॥\&[\smallbreak]


	
	    \end{quote}
	  
	  \endgroup
	

	  \pstart \leavevmode% starting standard par
	\hphantom{.}शास्त्राभ्युप‚ग‚मेपि वादिनाऽ‚{\color{DodgerBlue3}‚नीप्सि‚{\tiny $_{7}$}‚त‚म‚साध्यं चेत्} । एव‚न्त‚र्ह्याकाश‚गुण-\leavevmode\ledsidenote{\textenglish{88a/MA}} ‚{\tiny $_{lb}$}‚त्वादिर‚पि ‚{\color{DodgerBlue3}‚ध‚र्मो वादिना} साध्य‚त्वेना‚{\color{DodgerBlue3}‚नीप्सितोऽसाध्यः} स्यात् । ‚{\color{DodgerBlue3}‚त‚दा त‚द‚साध्य}‚{\tiny $_{lb}$}‚माकाश‚गुण‚त्वं ‚{\color{DodgerBlue3}‚बाध‚मानं} कृत‚क‚त्वं ‚{\color{DodgerBlue3}‚किं} क‚स्माद् ‚{\color{DodgerBlue3}‚विरोधि} । (७१)
	\pend% ending standard par
      \label{div_pvv.4.72}
	  
	% new div opening: depth here is 2
	

	  \begin{center}%% label @type='head'
	\textbf{(२) अन्य‚था स्व‚यंश‚ब्दोऽन‚र्थ‚कः}
	\end{center}
	

	  \pstart \leavevmode% starting standard par
	किञ्च (।)
	\pend% ending standard par
      
	  \bigskip
	  \begingroup
	
	    \large
	  
	    \begin{quote}
	  
	    
	    \stanza[\smallbreak]
	\label{pv.4.72}\flagstanza{\tiny\textenglish{...v.4.72}}प‚क्ष‚ल‚क्ष‚ण‚बाह्यार्थः स्व‚यंश‚ब्दोप्य‚न‚र्थ‚कः ।&शास्त्रेष्विच्छाप्र‚वृत्त्य‚र्थो य‚दि श‚ङ्का कुतोन्विय‚म् ॥ ७२ ॥\&[\smallbreak]


	
	    \end{quote}
	  
	  \endgroup
	

	  \pstart \leavevmode% starting standard par
	य‚दि शास्त्राभ्युप‚ग‚त‚त्वं प‚क्ष\edtext{}{\edlabel{pvv.439-1}\label{pvv.439-1}\lemma{क्ष}\Bfootnote{शास्त्रार्थः स‚र्व्वः साध्यः ।}}ल‚क्ष‚णं त‚दा ‚{\color{DodgerBlue3}‚स्व‚यंश‚ब्दोपि प‚क्ष‚ल‚क्ष‚ण‚बाह्यार्थो} भिन्नाभिधेयोऽ‚{\color{DodgerBlue3}‚न‚र्थ‚कः} स्यात् । शास्त्राभ्युप‚ग‚मे शास्त्रेष्ट‚स्य साध्य‚ताप्राप्तौ वादीष्ट‚{\tiny $_{lb}$}‚मात्रं साध्यं नान्य‚दिति हि स्व‚यंश‚ब्द‚स्य प्र‚योज‚नं । शास्त्रेष्ट‚मात्र‚स्य तु ‚{\color{DodgerBlue3}‚साध्य‚त्वे} निष्फ‚ल‚मे‚{\tiny $_{1}$}‚व त‚त् । ‚{\color{DodgerBlue3}‚शास्त्रेष्विच्छ‚या प्र‚वृत्त्य‚र्थः} \edtext{\textsuperscript{*}}{\edlabel{pvv.439-2}\label{pvv.439-2}\lemma{*}\Bfootnote{स्वीकृत‚शास्त्रं मुक्त्वा वाद‚काले शास्त्रान्त‚र‚मिच्छ‚या ल‚भ्य‚तेङ्गीक‚र्त्तुं ।}} स्व‚यंश‚ब्दो ‚{\color{DodgerBlue3}‚य‚दि} क‚थ्य‚ते स्व‚यं‚{\tiny $_{lb}$}‚श‚ब्द‚म‚न्त‚रेण शास्त्र‚मिच्छ‚या न ग्र‚हीत‚व्य‚मिति ‚{\color{DodgerBlue3}‚श‚ङ्केयं कुतो} नु हेतोर्जाता । ‚{\color{DodgerBlue3}‚येन} त‚न्निवृत्त्य‚र्था स्व‚यंश्रुतिर्व्व‚र्ण्य‚ते ॥ (७२)
	\pend% ending standard par
      \label{div_pvv.4.73}
	  
	% new div opening: depth here is 2
	
	  \bigskip
	  \begingroup
	
	    \large
	  
	    \begin{quote}
	  
	    
	    \stanza[\smallbreak]
	\label{pv.4.73}\flagstanza{\tiny\textenglish{...v.4.73}}सोनिषिद्धः प्र‚माणेन गृह्ण‚न् केन निवार्य‚ते ।&निषिद्ध‚श्चेत् प्र‚माणेन वाचा केन प्र‚वृर्त्त्य‚ते ॥ ७३ ॥\&[\smallbreak]


	
	    \end{quote}
	  
	  \endgroup
	

	  \pstart \leavevmode% starting standard par
	\hphantom{.}‚{\color{DodgerBlue3}‚स} वादी ‚{\color{DodgerBlue3}‚प्र‚माणेन} शास्त्रार्थ‚बाध‚केना‚{\color{DodgerBlue3}‚निषिद्धः} शास्त्रं ‚{\color{DodgerBlue3}‚गृह्ण‚न् केन निवार्य‚ते} न केन‚चित् । य‚तः स्व‚यंग्र‚ह‚णं शास्त्रं ग्राह‚य‚त् स‚फ‚लं स्यात् । ‚{\color{DodgerBlue3}‚प्र‚माणेन} चे‚{\tiny $_{lb}$}‚च्छास्त्रार्थ‚बाध‚केन ‚{\color{DodgerBlue3}‚निषिद्धो} वादी ‚{\color{DodgerBlue3}‚वाचा} स्व‚यंश‚ब्देन शा‚{\tiny $_{2}$}‚स्त्राभ्युप‚ग‚मे ‚{\color{DodgerBlue3}‚केन} ल‚क्ष‚ण‚{\tiny $_{lb}$}‚क‚र्त्रा ‚{\color{DodgerBlue3}‚प्र‚व‚र्त्त्य‚ते} न केन‚चित् ॥ (७३)
	\pend% ending standard par
      \label{div_pvv.4.74}
	  
	% new div opening: depth here is 2
	

	  \pstart \leavevmode% starting standard par
	किञ्च (।)
	\pend% ending standard par
      
	  \bigskip
	  \begingroup
	
	    \large
	  
	    \begin{quote}
	  
	    
	    \stanza[\smallbreak]
	\label{pv.4.74}\flagstanza{\tiny\textenglish{...v.4.74}}पूर्व‚म‚प्येष सिद्धान्तं स्वेच्छ‚यैव गृहीत‚वान् ।&किञ्चिद‚न्यं स (तु) पुन‚र्ग्र‚हीतुं ल‚भ‚ते न किंम् ॥ ७४ ॥\&[\smallbreak]


	
	    \end{quote}
	  
	  \endgroup
	

	  \pstart \leavevmode% starting standard par
	\hphantom{.}‚{\color{DodgerBlue3}‚एष} वादी ‚{\color{DodgerBlue3}‚पूर्व्व‚म‚पि स्वेच्छेयैव सिद्धान्तं} क णा दा दिप्र‚णीतं ‚{\color{DodgerBlue3}‚गृहीत‚वान्} । न ‚{\tiny $_{lb}$}‚त्व‚न्य‚पुराणादिब‚लात् । स क‚थ‚मिच्छ‚या शास्त्रोद्दिष्ट(ङ्) ‚{\color{DodgerBlue3}‚किञ्चिद्} ध‚र्म‚व्य‚भि‚{\tiny $_{lb}$}‚\leavevmode\ledsidenote{\textenglish{440/s}} चार‚द‚र्श‚नादिना शास्त्रेषु स‚फ‚ल‚ध‚र्म‚क‚लाप‚साध्य‚त्वाद‚न्यं सिद्धान्त‚माकाश‚गुण‚त्व‚{\tiny $_{lb}$}‚र‚हितानित्य‚तादिकं ‚{\color{DodgerBlue3}‚ग्र‚हीतुं किन्न ल‚भ‚ते} (।) इच्छाधीन‚त्वे निय‚मायोगात् । त‚स्मात् ‚{\tiny $_{lb}$}‚स्व‚{\tiny $_{3}$}‚यंग्र‚ह‚णं शास्त्रेच्छाप्र‚वृत्त्य‚र्थ‚मित्य‚युक्तं ॥ (७४)
	\pend% ending standard par
      \label{div_pvv.4.75}
	  
	% new div opening: depth here is 2
	

	  \pstart \leavevmode% starting standard par
	न‚न्विष्ट‚स्यापि स्वेच्छ‚यैव साध्य‚ताप‚रिग्र‚हः सिद्ध इति व्य‚र्थं स्व‚यंग्र‚ह‚ण‚मित्याह ।
	\pend% ending standard par
      
	  \bigskip
	  \begingroup
	
	    \large
	  
	    \begin{quote}
	  
	    
	    \stanza[\smallbreak]
	\label{pv.4.75}\flagstanza{\tiny\textenglish{...v.4.75}}दृष्टेर्विप्र‚तिप‚त्तीनाम‚त्राकार्षीत् स्व‚यंश्रुतिम् ।&इष्टाक्ष‚तिम‚साध्य‚त्व‚म‚न‚व‚स्थाञ्च द‚र्श‚य‚न् ॥ ७५ ॥\&[\smallbreak]


	
	    \end{quote}
	  
	  \endgroup
	

	  \pstart \leavevmode% starting standard par
	\hphantom{.}शास्त्र‚कार‚स्येष्टं साध्य‚मिति ‚{\color{DodgerBlue3}‚विप्र‚तिप‚त्तीनां दृष्टे}‚स्त‚न्निराक‚णार्थ‚{\color{DodgerBlue3}‚म‚त्र} प‚क्ष‚{\tiny $_{lb}$}‚ल‚क्ष‚णे ‚{\color{DodgerBlue3}‚स्व‚यंश्रुति}‚माचार्योऽ‚{\color{DodgerBlue3}‚कार्षीत्} । शास्त्रेष्वाकाश‚गुण‚त्वासिद्धाव‚पि वादी‚{\tiny $_{lb}$}‚‚{\color{DodgerBlue3}‚ष्ट‚स्याक्ष‚तिं} शास्त्रेष्ट‚स्या‚{\color{DodgerBlue3}‚साध्य‚त्वं} शास्त्रेष्ट‚ध‚र्मान्त‚रासिद्धौ हेतुब‚ल‚प्र‚सिद्ध‚साध्य‚{\tiny $_{lb}$}‚बाध‚ने \edtext{}{\edlabel{pvv.440-1}\label{pvv.440-1}\lemma{ने}\Bfootnote{त‚द‚स‚म्ब‚द्धानित्य‚त्वे बाधाभ्युप‚ग‚मे ।}}ग‚न्धे भूत‚{\tiny $_{4}$}‚गुण‚ताबाधा\edtext{}{\edlabel{pvv.440-2}\label{pvv.440-2}\lemma{ताबाधा}\Bfootnote{त‚त्प‚रिहृताव‚प्येवं विरोधः स्यादित्य‚न‚व‚स्था ।}}यां श‚ब्दे कृत‚क‚त्व‚म‚नित्य‚त्व‚साध‚नं विरुद्धं स्या‚{\tiny $_{lb}$}‚दित्यादि काम‚न‚{\color{DodgerBlue3}‚व‚स्थाञ्च} प‚र‚स्य ‚{\color{DodgerBlue3}‚द‚र्श‚य‚न्नाचा} र्यः स्व‚यंश्रुतिम‚कार्षीदिति पूर्व्वेण ‚{\tiny $_{lb}$}‚स‚म्ब‚न्धः ॥ (७५)
	\pend% ending standard par
      \label{div_pvv.4.76}
	  
	% new div opening: depth here is 2
	
	  \bigskip
	  \begingroup
	
	    \large
	  
	    \begin{quote}
	  
	    
	    \stanza[\smallbreak]
	\label{pv.4.76}\flagstanza{\tiny\textenglish{...v.4.76}}स‚म‚याहित‚भेद‚स्य प‚रिहारेण ध‚र्मिणः ।&प्र‚सिद्ध‚स्य गृहीत्य‚र्था ज‚गादान्यः स्व‚यंश्रुतिम् ॥ ७६ ॥\&[\smallbreak]


	
	    \end{quote}
	  
	  \endgroup
	

	  \pstart \leavevmode% starting standard par
	\hphantom{.}‚{\color{DodgerBlue3}‚स‚म‚येन} \edtext{\textsuperscript{*}}{\edlabel{pvv.440-3}\label{pvv.440-3}\lemma{*}\Bfootnote{टीकाकार‚क‚ल्पितार्थ‚दूष‚णायाह ।}} सिद्धान्ते‚{\color{DodgerBlue3}‚नाहित} आरोपितो ‚{\color{DodgerBlue3}‚भेदो} विशेष आकाश‚गुण‚त्वादिर्य‚स्य ‚{\tiny $_{lb}$}‚‚{\color{DodgerBlue3}‚त‚स्य ध‚र्मिणः} प‚रिहारेणाग‚म‚निर‚पेक्ष‚प्र‚माण‚ब‚लात् ‚{\color{DodgerBlue3}‚प्र‚सिद्ध‚स्य} ध‚र्मिणः श‚ब्द‚मात्रा‚{\tiny $_{lb}$}‚‚{\color{DodgerBlue3}‚देर्गृहीतिरित्य‚र्थः} प्र‚योज‚नं य‚स्तास्तां ‚{\color{DodgerBlue3}‚स्व‚यंश्रुतिम‚न्यो ‚{\tiny $_{5}$}‚ ज‚गाद} । स्वं प्र‚सिद्धो ‚{\tiny $_{lb}$}‚ध‚र्मी कार्यो नाग‚म‚सिद्ध इत्य‚र्थः । (७६)
	\pend% ending standard par
      \label{div_pvv.4.77}
	  
	% new div opening: depth here is 2
	

	  \pstart \leavevmode% starting standard par
	अत्राह ।
	\pend% ending standard par
      
	  \bigskip
	  \begingroup
	
	    \large
	  
	    \begin{quote}
	  
	    
	    \stanza[\smallbreak]
	\label{pv.4.77}\flagstanza{\tiny\textenglish{...v.4.77}}विचार‚प्र‚स्तुतेरेव प्र‚सिद्धःसिद्ध आश्र‚यः ।&स्वेच्छाक‚ल्पित‚भेदेषु प‚दार्थेष्व‚विवाद‚तः ॥ ७७ ॥\&[\smallbreak]


	
	    \end{quote}
	  
	  \endgroup
	

	  \pstart \leavevmode% starting standard par
	\hphantom{.}ध‚र्मिणि साध्य‚ध‚र्म‚स्य भावाभाव‚{\color{DodgerBlue3}‚विचार‚प्र‚स्तुतेरे}‚वाग‚म‚म‚न‚पेक्ष्य ‚{\color{DodgerBlue3}‚प्र‚सिद्ध आश्र‚यो} ध‚र्मी सिद्धः स्वेच्छ‚या ‚{\color{DodgerBlue3}‚क‚ल्पितो भेदो} विशेषो येषां तेषु ‚{\color{DodgerBlue3}‚प‚दार्थेष्व‚विवाद‚तो} विवा‚{\tiny $_{lb}$}‚दाभावात् । न हि क‚ल्पित‚भेदे ध‚र्मिणि क‚श्चित् प्रेक्षावान् क‚स्य‚चिद् ध‚र्म‚स्य साध‚नं ‚{\tiny $_{lb}$}‚बाध‚नं वेह‚ते (।) किन्तु प्र‚माण‚प्र‚तीते व‚स्तुनि । अत‚स्त‚द‚र्थ‚म‚{\tiny $_{6}$}‚पि स्व‚यंग्र‚ह‚ण‚{\tiny $_{lb}$}‚म‚नुप‚युक्तं ॥ (७७)
	\pend% ending standard par
      \textsuperscript{\textenglish{441/s}}\label{div_pvv.4.78}
	  
	% new div opening: depth here is 2
	
	  \bigskip
	  \begingroup
	
	    \large
	  
	    \begin{quote}
	  
	    
	    \stanza[\smallbreak]
	\label{pv.4.78}\flagstanza{\tiny\textenglish{...v.4.78}}असाध्य‚ताम‚थ प्राह सिद्धादेशेन ध‚र्मिणः ।&स्व‚रूपेणैव निर्देश्य इत्य‚नेनैव त‚द्ग‚तं ॥ ७८ ॥\&[\smallbreak]


	
	    \end{quote}
	  
	  \endgroup
	

	  \pstart \leavevmode% starting standard par
	\hphantom{.}‚{\color{DodgerBlue3}‚अथ सिद्धादेशेन} प्र‚सिद्धार्थ‚वाच‚केन स्व‚यंश‚ब्देन ‚{\color{DodgerBlue3}‚ध‚र्मिणो\edtext{}{\edlabel{pvv.441-1}\label{pvv.441-1}\lemma{र्मिणो}\Bfootnote{असिद्ध‚ध‚र्मिप‚रिहारेण सिद्ध‚ध‚र्मिग्र‚ह‚माह ।}}ऽसाध्य‚तां प्राह} । ‚{\tiny $_{lb}$}‚य‚थाऽस्ति प्र‚धानं भेदाना\edtext{}{\edlabel{pvv.441-2}\label{pvv.441-2}\lemma{भेदाना}\Bfootnote{ध‚र्म्यैव प्र‚धान‚म‚सिद्धः क्वास्तित्वं साध्यं ।}}म‚नुप‚द‚र्श‚नादिति । इद‚म‚प्य‚युक्तं । य‚स्मात् ‚{\color{DodgerBlue3}‚स्व‚रूपेण} साध्य‚त्वेनैव ‚{\color{DodgerBlue3}‚निर्देश्य इत्य‚नेन} प‚क्ष‚ल‚क्ष‚णाव‚य‚वे‚{\color{DodgerBlue3}‚नैव} च ‚{\color{DodgerBlue3}‚त‚द्ध}‚र्मिणः सिद्ध‚स्यासाध्य‚त्वं ‚{\tiny $_{lb}$}‚‚{\color{DodgerBlue3}‚ग‚तं} प्र‚तीतं ॥ (७८)
	\pend% ending standard par
      \label{div_pvv.4.79}
	  
	% new div opening: depth here is 2
	

	  \pstart \leavevmode% starting standard par
	त‚था ह्य‚य‚मिष्टोऽनिराकृतः प‚क्ष इत्य‚नेन (।)
	\pend% ending standard par
      
	  \bigskip
	  \begingroup
	
	    \large
	  
	    \begin{quote}
	  
	    
	    \stanza[\smallbreak]
	\label{pv.4.79}\flagstanza{\tiny\textenglish{...v.4.79}}सिद्ध‚साध‚न‚रूपेण निर्देश‚स्य हि स‚म्भ‚वे ।&साध्य‚त्वैनेव निर्देश्य इतीदं फ‚ल‚व‚द् भ‚वेत् ॥ ७९ ॥\&[\smallbreak]


	
	    \end{quote}
	  
	  \endgroup
	

	  \pstart \leavevmode% starting standard par
	सिद्ध‚स्य सिद्ध\edtext{}{\edlabel{pvv.441-3}\label{pvv.441-3}\lemma{सिद्ध}\Bfootnote{त‚द्ध्याकार एव ।}}रूपेण निर्देश‚स्य ध‚र्म‚व‚च‚न‚{\tiny $_{7}$}‚स्यासिद्ध‚स्यापि ‚{\color{DodgerBlue3}‚साध‚न‚रूपेण}\leavevmode\ledsidenote{\textenglish{88b/MA}} ‚{\tiny $_{lb}$}‚‚{\color{DodgerBlue3}‚निर्देश}‚स्यासिद्ध‚व‚च‚न‚स्य प‚क्ष‚त्व‚{\color{DodgerBlue3}‚स‚म्भ‚वे} हि त‚त्प्र‚तिषेधं विद‚ध‚त् ‚{\color{DodgerBlue3}‚साध्य‚त्वेनैव निर्देश्य} इतीदं ‚{\color{DodgerBlue3}‚फ‚ल‚व‚द् भ‚वेत्} । साध्य‚स्यैव निर्देशः प‚क्ष इति सिद्ध‚स्य ध‚र्मिणोऽसिद्ध‚स्य ‚{\tiny $_{lb}$}‚च साध‚न‚त्वेनोक्त‚स्य निरासः ॥ (७९)
	\pend% ending standard par
      \label{div_pvv.4.80}
	  
	% new div opening: depth here is 2
	

	  \pstart \leavevmode% starting standard par
	किञ्च (।)
	\pend% ending standard par
      
	  \bigskip
	  \begingroup
	
	    \large
	  
	    \begin{quote}
	  
	    
	    \stanza[\smallbreak]
	\label{pv.4.80}\flagstanza{\tiny\textenglish{...v.4.80}}अनुमान‚स्य सामान्य‚विष‚य‚त्व‚ञ्च व‚र्ण्णित‚म् ।&इहैवं न ह्य‚नुक्तोपि किञ्चित् प‚क्षे विरुध्य‚ते ॥ ८० ॥\&[\smallbreak]


	
	    \end{quote}
	  
	  \endgroup
	

	  \pstart \leavevmode% starting standard par
	\hphantom{.}‚{\color{DodgerBlue3}‚अनुमान‚स्य सामान्य‚विष‚य‚त्वं} स्व‚य‚माचार्येण व‚र्ण्णितं (।) य‚दि च ‚{\color{DodgerBlue3}‚ध‚र्मी} प‚क्षः त‚दा त‚स्य स्व‚ल‚क्ष‚ण‚त्वात् साम‚न्य‚विष‚य‚ता व्याह‚न्येत । कि‚{\color{DodgerBlue3}‚ञ्चेह} प‚क्ष‚ल‚क्ष‚ण ‚{\tiny $_{lb}$}‚‚{\color{DodgerBlue3}‚एव‚{\tiny $_{1}$}‚मुक्त}‚क्र(मे)णा‚{\color{DodgerBlue3}‚नुक्तेपि} स्व‚यंश‚ब्दे सिद्ध‚ध‚र्म्य‚सिद्ध‚साध‚न‚व्य‚व‚च्छेदार्थे ‚{\color{DodgerBlue3}‚किञ्चित्} ‚{\color{DodgerBlue3}‚प‚क्षे} प्र‚तिपाद्ये ‚{\color{DodgerBlue3}‚न विरुध्य‚ते} ॥ (८०)
	\pend% ending standard par
      \label{div_pvv.4.81}
	  
	% new div opening: depth here is 2
	

	  \pstart \leavevmode% starting standard par
	न‚न्व‚नुक्ते स्व‚यंश‚ब्दे प‚क्ष‚ल‚क्ष‚णं (।)
	\pend% ending standard par
      
	  \bigskip
	  \begingroup
	
	    \large
	  
	    \begin{quote}
	  
	    
	    \stanza[\smallbreak]
	\label{pv.4.81}\flagstanza{\tiny\textenglish{...v.4.81}}कुर्याच्चेद् ध‚र्मिणं साध्यं त‚तः किन्त‚न्न श‚क्य‚ते ।&क‚स्माद्धेत्व‚न्व‚याभावान्न च दोष‚स्त‚योर‚पि ॥ ८१ ॥\&[\smallbreak]


	
	    \end{quote}
	  
	  \endgroup
	

	  \pstart \leavevmode% starting standard par
	\hphantom{.}‚{\color{DodgerBlue3}‚ध‚र्मिणं साध्यं कुर्या}‚दिति दोषः । ‚{\color{DodgerBlue3}‚त‚तो} ध‚र्मिणः प‚क्ष‚त्व‚प्र‚स‚ङ्गात् ‚{\color{DodgerBlue3}‚किन्दूष‚णं} ॥ ‚{\tiny $_{lb}$}‚त‚द्ध‚र्म्मिप‚क्ष‚त्वं क‚र्त्तुं ‚{\color{DodgerBlue3}‚न श‚क्य‚त} इति अश‚क्य‚तादूष‚णं । ‚{\color{DodgerBlue3}‚क‚स्मात्} कार‚णाद् ‚{\color{DodgerBlue3}‚ध‚र्मी} \leavevmode\ledsidenote{\textenglish{442/s}} प‚क्षो भ‚वितु नार्ह‚ति ध‚र्मिणः साध्य‚त्वेनासिद्ध‚तायां \edtext{}{\edlabel{pvv.442-1}\label{pvv.442-1}\lemma{तायां}\Bfootnote{आश्र‚यासिद्ध्या ।}} ‚{\color{DodgerBlue3}‚हेतो}‚र‚भावात् । विशेष‚स्य ‚{\tiny $_{lb}$}‚ध‚र्मिणो दृष्टान्ते‚{\tiny $_{2}$}‚ऽस‚म्भ‚वात् (।) ‚{\color{DodgerBlue3}‚अन्व‚याभावा}‚च्च ध‚र्मी प‚क्षः क‚र्त्तुं ‚{\color{DodgerBlue3}‚न} श‚क्य‚ते । (८१)
	\pend% ending standard par
      \label{div_pvv.4.82}
	  
	% new div opening: depth here is 2
	

	  \pstart \leavevmode% starting standard par
	न‚न्व‚यं दोष‚स्त‚योर्हेतुदृष्टान्त‚योर्न तु प‚क्ष‚स्य ॥ त‚था हि (।)
	\pend% ending standard par
      
	  \bigskip
	  \begingroup
	
	    \large
	  
	    \begin{quote}
	  
	    
	    \stanza[\smallbreak]
	\label{pv.4.82}\flagstanza{\tiny\textenglish{...v.4.82}}उत्त‚राव‚य‚वापेक्षो न दोषः प‚क्ष इष्य‚ते ।&त‚था हेत्वादिदोषोपि प‚क्ष‚दोषः प्र‚स‚ज्य‚ते ॥ ८२ ॥\&[\smallbreak]


	
	    \end{quote}
	  
	  \endgroup
	

	  \pstart \leavevmode% starting standard par
	\hphantom{.}साध‚न‚वाक्य‚स्य प‚क्षादु‚{\color{DodgerBlue3}‚त्त‚रेऽव‚य‚वे} हेतुदृष्टान्तादिकेऽ‚{\color{DodgerBlue3}‚पेक्षा} य‚स्यासौ ‚{\color{DodgerBlue3}‚दोषः} ‚{\color{DodgerBlue3}‚प‚क्षे नेष्य‚ते} हेतुदृष्टान्त‚स‚म्ब‚न्धित्वात् त‚स्य । य‚दि तूत्त‚राव‚य‚वापेक्षोपि प‚क्ष‚स्य ‚{\tiny $_{lb}$}‚बाध‚नात् प‚क्ष‚दोष उच्य‚ते ‚{\color{DodgerBlue3}‚त‚था} स‚ति ‚{\color{DodgerBlue3}‚हेत्वादिदोषोपि प‚क्ष‚दोषः प्र‚स‚ज्य‚ते} (। ८२)
	\pend% ending standard par
      \label{div_pvv.4.83}
	  
	% new div opening: depth here is 2
	
	  \bigskip
	  \begingroup
	
	    \large
	  
	    \begin{quote}
	  
	    
	    \stanza[\smallbreak]
	\label{pv.4.83}\flagstanza{\tiny\textenglish{...v.4.83}}स‚र्वैः प‚क्ष‚स्य बाधात‚स्त‚स्मात् त‚न्मात्र‚स‚ङ्गिनः ।&प‚क्ष‚दोषा म‚ता नान्ये प्र‚त्य‚क्षादिविरोध‚व‚त् ॥ ८३ ॥\&[\smallbreak]


	
	    \end{quote}
	  
	  \endgroup
	

	  \pstart \leavevmode% starting standard par
	\hphantom{.}‚{\color{DodgerBlue3}‚स‚र्व्वै}‚र्हेत्वादिदोषैः ‚{\tiny $_{3}$}‚ ‚{\color{DodgerBlue3}‚प‚क्ष‚स्य बाधात् त‚स्मात् त‚न्मात्र‚ङ्गिनः} प‚क्ष‚मात्र‚स‚म्ब‚द्धा ‚{\tiny $_{lb}$}‚दोषाः ‚{\color{DodgerBlue3}‚प‚क्ष‚दोषा म‚ताः} । ‚{\color{DodgerBlue3}‚नान्ये}‚ऽव‚य‚वान्त‚रापेक्षाः ‚{\color{DodgerBlue3}‚प्र‚त्य‚क्षादिविरोध‚व‚त्} । य‚था ‚{\tiny $_{lb}$}‚प्र‚त्य‚क्षादिबाधित‚त्व‚म‚व‚य‚वान्त‚रान‚पेक्षं प‚क्ष‚दोषः । (८३)
	\pend% ending standard par
      \label{div_pvv.4.84}
	  
	% new div opening: depth here is 2
	

	  \pstart \leavevmode% starting standard par
	त‚स्माद् (।)
	\pend% ending standard par
      
	  \bigskip
	  \begingroup
	
	    \large
	  
	    \begin{quote}
	  
	    
	    \stanza[\smallbreak]
	\label{pv.4.84}\flagstanza{\tiny\textenglish{...v.4.84}}हेत्वादिल‚क्ष‚णैर्ब्बाध्यं मुक्त्वा प‚क्ष‚स्य ल‚क्ष‚ण‚म् ।&उच्य‚ते प‚रिहारार्थ‚म‚व्याप्तिव्य‚तिरेक‚योः ॥ ८४ ॥\&[\smallbreak]


	
	    \end{quote}
	  
	  \endgroup
	

	  \pstart \leavevmode% starting standard par
	\hphantom{.}‚{\color{DodgerBlue3}‚हेत्वादीनां ल‚क्ष‚णैर्ब्बाध्यं} प‚रिह‚र्त्त‚व्यं दोष‚म‚न्व‚य‚विरोधादिकं ‚{\color{DodgerBlue3}‚मुक्त्वा} प‚क्ष‚मा‚{\tiny $_{lb}$}‚त्रानुष‚ङ्गिणो‚{\color{DodgerBlue3}‚र‚व्याप्तिव्य‚तिरेक‚योः} प‚रिहारार्थं ‚{\color{DodgerBlue3}‚प‚क्ष‚ल‚क्ष‚ण‚मुच्य‚ते}\edtext{}{\edlabel{pvv.442-2}\label{pvv.442-2}\lemma{रिहारार्थं}\Bfootnote{अन्य‚था हेत्वादिल‚क्ष‚णं निर्विष‚यं स्यात् ।}}व्य‚तिरेके आधि‚{\tiny $_{lb}$}‚‚{\tiny $_{4}$}‚ क्य‚म‚भिव्याप्तिरित्य‚र्थः ॥ (८४)
	\pend% ending standard par
      \label{div_pvv.4.85}
	  
	% new div opening: depth here is 2
	

	  \pstart \leavevmode% starting standard par
	त‚त्र येन प‚देन य‚द् दूष‚णं प‚रिह्रिय‚ते त‚दाह ।
	\pend% ending standard par
      
	  \bigskip
	  \begingroup
	
	    \large
	  
	    \begin{quote}
	  
	    
	    \stanza[\smallbreak]
	\label{pv.4.85}\flagstanza{\tiny\textenglish{...v.4.85}}स्व‚य‚न्निपात‚रूपाख्या व्य‚तिरेक‚स्य बाधिकाः ।&स‚हानिराकृतेनेष्ट‚श्रुतिर‚व्याप्तिबाध‚नी ॥ ८५ ॥\&[\smallbreak]


	
	    \end{quote}
	  
	  \endgroup
	

	  \pstart \leavevmode% starting standard par
	\hphantom{.}‚{\color{DodgerBlue3}‚स्व‚य‚ञ्च निपात‚ञ्च} एवं ‚{\color{DodgerBlue3}‚रूपं} स्व‚रूप‚ञ्चा‚{\color{DodgerBlue3}‚ख्या} श्रुत‚यः स‚हानिराकृतेन प‚देन ‚{\tiny $_{lb}$}‚‚{\color{DodgerBlue3}‚व्य‚तिरेक}‚स्यातिव्याप्ते‚{\color{DodgerBlue3}‚र्ब्बाधिकाः} । स्व‚यंश‚ब्देन शास्त्रेष्ट‚स्य निपातेनासिद्ध‚{\tiny $_{lb}$}‚स्यापि साध‚न‚त्वेनोक्त‚स्य स्व‚रूप‚श‚ब्देन सिद्ध‚स्य । ‚{\color{DodgerBlue3}‚निराकृत}‚श‚ब्देन प्र‚त्य‚क्षादि‚{\tiny $_{lb}$}‚\leavevmode\ledsidenote{\textenglish{443/s}} निराकृत‚स्य प‚क्ष‚त्वं स‚क्तं निषिध्य‚ते । ‚{\color{DodgerBlue3}‚इष्ट‚श्रुतिर‚व्याप्तेर्ब्बाध‚नी । इ‚{\tiny $_{5}$}‚ष्ट‚श‚ब्दे ह्य‚क्रि}‚{\tiny $_{lb}$}‚य‚माणे निर्द्दिष्ट‚मेव साध्यं साध्यं स्यात् न प्र‚क‚र‚णाप‚न्न‚मिष्टं ॥ (८५)
	\pend% ending standard par
      \label{div_pvv.4.86}
	  
	% new div opening: depth here is 2
	

	  \pstart \leavevmode% starting standard par
	\hphantom{.}य‚दि स्व‚यंनिपात‚रूपाख्या व्य‚तिरेक‚स्य वाधिकाः प्र मा ण स मु च्च य‚{\tiny $_{lb}$}‚ल क्ष णे निर्द्दिष्टास्त‚दा न्याय‚मुखे साध्य‚त्वेनेप्सितः प‚क्षो विरुद्धार्थानिराकृत ‚{\tiny $_{lb}$}‚इति प‚क्ष‚ल‚क्ष‚णे ता न स‚न्तीति क‚थ‚न्तेनाव्याप्तिव्य‚तिरेक‚योः प‚रिहार ‚{\color{DodgerBlue3}‚इत्याह} साध्य‚त्वेनेप्सितः प‚क्ष इति (।)
	\pend% ending standard par
      
	  \bigskip
	  \begingroup
	
	    \large
	  
	    \begin{quote}
	  
	    
	    \stanza[\smallbreak]
	\label{pv.4.86}\flagstanza{\tiny\textenglish{...v.4.86}}साध्याभ्युप‚ग‚मः प‚क्ष‚ल‚क्ष‚णं तेष्व‚प‚क्ष‚ता ।&निराकृते बाध‚न‚तः शेषेऽल‚क्ष‚ण‚वृत्तितः ॥ ८६ ॥\&[\smallbreak]


	
	    \end{quote}
	  
	  \endgroup
	

	  \pstart \leavevmode% starting standard par
	\hphantom{.}‚{\color{DodgerBlue3}‚साध्याभ्युप‚ग‚मः} प‚क्ष इति ‚{\color{DodgerBlue3}‚प‚क्ष‚ल‚क्ष‚ण‚म}‚{\tiny $_{6}$}‚व‚तिष्ठ‚ते । त‚था च ‚{\color{DodgerBlue3}‚तेषु} शास्त्रेष्टा‚{\color{DodgerBlue3}‚दिषु} प‚ञ्च‚सु व्याव‚र्त्त्येषु म‚ध्ये ‚{\color{DodgerBlue3}‚निराकृते} प्र‚त्य‚क्षादिबाधिते ‚{\color{DodgerBlue3}‚बाध‚न‚तोऽप‚क्ष‚ता} विरुद्धार्था ‚{\tiny $_{lb}$}‚निराकृत‚स्य प‚क्ष‚विधानात् । ‚{\color{DodgerBlue3}‚शेषे} शास्त्रेष्टे वादिनाऽनिष्टे साध‚ने च सिद्धे साध‚{\tiny $_{lb}$}‚यितुमिष्ट एष्य‚माणे सिद्धे च साध्य‚विप‚रीतेऽप्र‚स्तुते चोक्त‚मात्रे ‚{\color{DodgerBlue3}‚ल‚क्ष‚ण}‚स्य साध्य‚{\tiny $_{lb}$}‚त्वेनेप्सित‚त्व‚स्या‚{\color{DodgerBlue3}‚वृत्तितो}‚ऽप‚क्ष‚ता सिद्धेति प‚रिपूर्ण्ण‚मिद‚म‚पि ल‚क्ष‚णं ॥ (८६)
	\pend% ending standard par
      \label{div_pvv.4.87}
	  
	% new div opening: depth here is 2
	

	  \pstart \leavevmode% starting standard par
	न‚नु य‚था स‚त्य‚र्थे‚{\tiny $_{7}$}‚भ्यो \edtext{}{\edlabel{pvv.443-1}\label{pvv.443-1}\lemma{भ्यो}\Bfootnote{म‚तिबुद्धिपूजार्थेभ्यः ।}}व‚र्त्त‚माने कुविधानादीप्सित‚श‚ब्दो व‚र्त्त‚मान‚मिच्छा-\leavevmode\ledsidenote{\textenglish{89a/MA}} ‚{\tiny $_{lb}$}‚माह । त‚थेष्ट‚श‚ब्दोपि त‚त्र एषिष्य‚माणे प‚क्ष‚त्व‚म‚प्र‚स‚क्त‚मेव त‚त्किं प्र मा ण‚{\tiny $_{lb}$}‚स मु च्च य ल‚क्ष‚णेऽव‚धार‚णं कृत‚मित्याह ।
	\pend% ending standard par
      
	  \bigskip
	  \begingroup
	
	    \large
	  
	    \begin{quote}
	  
	    
	    \stanza[\smallbreak]
	\label{pv.4.87}\flagstanza{\tiny\textenglish{...v.4.87}}स्व‚य‚मिष्टाभिधानेन ग‚तार्थेप्य‚व‚धार‚णे ।&कृत्यान्तेनाभिस‚म्ब‚न्धादुक्तं कालान्त‚र‚च्छिदे ॥ ८७ ॥\&[\smallbreak]


	
	    \end{quote}
	  
	  \endgroup
	

	  \pstart \leavevmode% starting standard par
	\hphantom{.}‚{\color{DodgerBlue3}‚स्व‚य‚मिष्ट} इत्य‚न‚योः प‚द‚योर‚{\color{DodgerBlue3}‚भिधानेनाव‚धार‚णे} निपातार्थे ग‚ते प्र‚तीतेपि ‚{\tiny $_{lb}$}‚‚{\color{DodgerBlue3}‚कृत्यान्तेन} निर्देश्य श‚ब्देन स‚र्व्व‚काल‚स‚म्ब‚न्ध‚योग्याभिधायिनाऽ‚{\color{DodgerBlue3}‚भिस‚म्ब‚न्धा}‚दिष्ट‚{\tiny $_{lb}$}‚श‚ब्द‚स्याव‚र्त्त‚मान‚कालेच्छाविष‚य‚स्यापि प‚क्ष‚त्वं स्यात् । य‚था ‚{\tiny $_{1}$}‚ आग‚तो देव‚द‚त्तो ‚{\tiny $_{lb}$}‚द्र‚ष्ट‚व्य इति (।) य‚दा ग‚मिष्य‚ति त‚दा द्र‚क्ष्य‚त इत्य‚र्थः । अतो व‚र्त‚मान‚कालात् ‚{\tiny $_{lb}$}‚‚{\color{DodgerBlue3}‚कालान्त‚र}‚स्य भ‚विष्य‚दादेः साध्ये साध्येच्छाविष‚य‚स्य ‚{\color{DodgerBlue3}‚च्छिदे} प्र‚तिषेधार्थ‚{\color{DodgerBlue3}‚मुक्त}‚म‚व‚धा‚{\tiny $_{lb}$}‚र‚णं स्व‚रूपेणैवेति । (८७)
	\pend% ending standard par
      \label{div_pvv.4.88}
	  
	% new div opening: depth here is 2
	

	  \pstart \leavevmode% starting standard par
	य‚स्मात् कृत्यान्तेनाभिस‚म्ब‚न्धात् काल‚सामान्य‚वृत्तिः (।)
	\pend% ending standard par
      
	  \bigskip
	  \begingroup
	
	    \large
	  
	    \begin{quote}
	  
	    
	    \stanza[\smallbreak]
	\label{pv.4.88}\flagstanza{\tiny\textenglish{...v.4.88}}इहान‚ङ्ग‚मिषेर्न्निष्ठा तेनेप्सित‚प‚दे पुनः ।&अङ्ग‚मेव त‚याऽसिद्ध‚हेत्वादि प्र‚तिषिध्य‚ते ॥ ८८ ॥\&[\smallbreak]


	
	    \end{quote}
	  
	  \endgroup
	\textsuperscript{\textenglish{444/s}}

	  \pstart \leavevmode% starting standard par
	\hphantom{.}‚{\color{DodgerBlue3}‚इह} प्र मा ण स मु च्च (य)ल‚क्ष‚णे निष्ठा व‚र्त्त‚मान‚साध्य‚त्वेष्टिप्र‚तिपाद‚न‚{\tiny $_{lb}$}‚‚{\color{DodgerBlue3}‚म‚त्य‚न‚ङ्ग}‚हेतुः । न्या य मु खे ‚{\color{DodgerBlue3}‚तेन} कृत्यान्तेन स‚म्ब‚न्धाभावे‚{\tiny $_{2}$}‚नेप्सित‚प‚दे पुन‚र‚ङ्ग‚{\tiny $_{lb}$}‚मेव निष्ठा व‚र्त्त‚मान‚साध्येच्छाबोध‚ने ‚{\color{DodgerBlue3}‚त‚या}\edtext{}{\edlabel{pvv.444-1}\label{pvv.444-1}\lemma{ने}\Bfootnote{निष्ठ‚या ।}}व‚र्त्त‚मान‚साध्येच्छाबोधिक‚याऽ‚{\color{DodgerBlue3}‚सिद्ध‚{\tiny $_{lb}$}‚हेत्वाद्य}‚पीष्य‚माणं साध्य‚त्वेन ‚{\color{DodgerBlue3}‚प्र‚तिषिध्य‚ते} । त‚स्मान्न ध‚र्मिणः साध्य‚ता\edtext{}{\edlabel{pvv.444-2}\label{pvv.444-2}\lemma{ता}\Bfootnote{स‚म‚याहिता ह्युक्ता ।}}प्र‚ति‚{\tiny $_{lb}$}‚क्षेपार्थं स्व‚यंग्र‚ह‚णं सिद्ध‚त्वे\edtext{}{\edlabel{pvv.444-3}\label{pvv.444-3}\lemma{त्वे}\Bfootnote{विचार‚प्र‚स्तुतेरेव प्र‚सिद्धः सिद्ध आश्र‚यः ।}}नैव त‚त्प‚रिहार‚स्य ल‚ब्ध‚त्वात् ॥ (८८)
	\pend% ending standard par
      \label{div_pvv.4.89}
	  
	% new div opening: depth here is 2
	

	  \pstart \leavevmode% starting standard par
	नापि शास्त्रेष्विच्छाप्र‚वृत्त्य‚र्थ\edtext{}{\edlabel{pvv.444-4}\label{pvv.444-4}\lemma{र्थ}\Bfootnote{असाध्य‚तामित्याद्युक्त ।}}मिच्छामात्रेणैव त‚द्ग्र‚ह‚ण‚स्य सिद्ध‚त्वादि‚{\tiny $_{lb}$}‚त्युक्तं ।
	\pend% ending standard par
      
	  \bigskip
	  \begingroup
	
	    \large
	  
	    \begin{quote}
	  
	    
	    \stanza[\smallbreak]
	\label{pv.4.89}\flagstanza{\tiny\textenglish{...v.4.89}}अवाच‚क‚त्व‚च्चायुक्तं तेनेष्टं स्व‚य‚मात्म‚ना ।&अन‚पेक्ष्याखिलं शास्त्रं त‚द्वादीष्ट‚स्य साध्य‚ता ॥ ८९ ॥\&[\smallbreak]


	
	    \end{quote}
	  
	  \endgroup
	

	  \pstart \leavevmode% starting standard par
	\hphantom{.}‚{\color{DodgerBlue3}‚अवाच‚क‚त्वाच्चायुक्तं} । त‚द‚र्थं ‚{\color{DodgerBlue3}‚स्व‚य}‚ग्र‚ह‚णं । न हि‚{\tiny $_{3}$}‚ स्व‚यंश‚ब्दः स्वेच्छ‚या ‚{\tiny $_{lb}$}‚शास्त्रं ग्राह्यामित्येत‚द‚र्थ‚वाच‚कं किन्तु वादिन एव वाच‚कः । ‚{\color{DodgerBlue3}‚तेन} त‚द्वाच‚क‚त्वेन स्व‚यं ‚{\tiny $_{lb}$}‚वादिनाऽ‚{\color{DodgerBlue3}‚त्म‚ना शास्त्र‚म‚खिल‚म‚न‚पेक्ष्य य‚दिष्टं त}‚स्य वादी‚{\color{DodgerBlue3}‚ष्ट‚स्य साध्य‚ते}‚ष्य‚ते न ‚{\tiny $_{lb}$}‚शास्त्रेष्ट‚स्येत्युप‚संहारः । (८९)
	\pend% ending standard par
      \label{div_pvv.4.90}
	  
	% new div opening: depth here is 2
	
	  \bigskip
	  \begingroup
	
	    \large
	  
	    \begin{quote}
	  
	    
	    \stanza[\smallbreak]
	\label{pv.4.90}\flagstanza{\tiny\textenglish{...v.4.90}}तेनान‚भीष्ट‚संसृष्ट‚स्येष्ट‚स्यापि हि बाध‚ने ।&य‚था साध्य‚म‚बाध‚तः प‚क्ष‚हेतून दुष्य‚तः ॥ ९० ॥\&[\smallbreak]


	
	    \end{quote}
	  
	  \endgroup
	

	  \pstart \leavevmode% starting standard par
	\hphantom{.}‚{\color{DodgerBlue3}‚तेन} \edtext{\textsuperscript{*}}{\edlabel{pvv.444-5}\label{pvv.444-5}\lemma{*}\Bfootnote{दृष्टानुमेय‚व‚च‚नेन ।}}कार‚णेन वादिनो‚{\color{DodgerBlue3}‚न‚भीष्टे}‚नाकाश‚गुण‚त्वेन ‚{\color{DodgerBlue3}‚संसृष्ट‚स्येष्ट‚स्या}‚नित्य‚त्व‚स्या‚{\color{DodgerBlue3}‚पि ‚{\tiny $_{lb}$}‚हि बाध‚ने}‚ऽभिधीय‚माने ‚{\color{DodgerBlue3}‚प‚क्ष‚हेतू न दुष्य‚तः} । किङ्गाकार‚ण‚{\tiny $_{4}$}‚मित्याह । ‚{\color{DodgerBlue3}‚य‚था साध्य‚म‚{\tiny $_{lb}$}‚बाधातः} । न हि वादिनाऽकाश‚गुण‚त्वैकार्थ‚स‚म‚वाय्य‚नित्य‚त्वं साध‚यितुमिष्टं येनास्य ‚{\tiny $_{lb}$}‚बाधः स्यात् किन्त्व‚नित्य‚त्व‚मात्रं । न चास्य प्र‚त्य‚क्षादिबाधास्ति । हेतोर्व्वा त‚द‚पेक्ष‚या ‚{\tiny $_{lb}$}‚विरुद्ध‚{\color{DodgerBlue3}‚तादिकं} । त‚देवं स्व‚यं-निपात‚रूपाख्या व्य‚तिरेक‚स्य बाधिकाः स‚हानिराकृतेनेति ‚{\tiny $_{lb}$}‚व्याख्यातं ॥ (९०)
	\pend% ending standard par
      \label{div_pvv.4.91}
	  
	% new div opening: depth here is 2
	

	  \begin{center}%% label @type='head'
	\textbf{(३) स‚हानिराकृत‚ग्र‚ह‚ण‚फ‚ल‚म्}
	\end{center}
	

	  \pstart \leavevmode% starting standard par
	अनिराकृत‚प‚दं व्याख्यातुमाह ।
	\pend% ending standard par
      
	  \bigskip
	  \begingroup
	
	    \large
	  
	    \begin{quote}
	  
	    
	    \stanza[\smallbreak]
	\label{pv.4.91}\flagstanza{\tiny\textenglish{...v.4.91}}अनिषिद्धः प्र‚माणाभ्यां स चोप‚ग‚म इष्य‚ते ।&स‚न्दिग्धे हेतुव‚च‚नाद् व्य‚स्तो हेतोर‚नाश्र‚यः ॥ ९१ ॥\&[\smallbreak]


	
	    \end{quote}
	  
	  \endgroup
	\textsuperscript{\textenglish{445/s}}

	  \pstart \leavevmode% starting standard par
	\hphantom{.}‚{\color{DodgerBlue3}‚स चो}‚क्त‚ल‚क्ष‚णः साध्य‚{\color{DodgerBlue3}‚स्योप‚ग‚मः} प‚क्षः । ‚{\color{DodgerBlue3}‚प्र‚मा‚{\tiny $_{5}$}‚णाभ्यां} प्र‚त्य‚क्षानुमानाभ्याम‚{\tiny $_{lb}$}‚निषिद्ध इष्य‚ते । क‚स्मादित्याह । संदिग्धेऽर्थे साध‚क‚बाध‚क‚प्र‚माण‚विष‚ये ‚{\color{DodgerBlue3}‚हेतोर्व्व‚च‚नाद् ‚{\tiny $_{lb}$}‚व्य‚स्तः} प्र‚माण‚प्र‚तिक्षिप्तो ‚{\color{DodgerBlue3}‚हेतोर‚नाश्र‚यो}‚ऽविष‚यः । (९१)
	\pend% ending standard par
      \label{div_pvv.4.92}
	  
	% new div opening: depth here is 2
	

	  \begin{center}%% label @type='head'
	\textbf{(४) बाधा च‚तुर्विधा}
	\end{center}
	

	  \pstart \leavevmode% starting standard par
	य‚दि द्विविधौ प‚क्ष‚बाधौ त‚दा प्र‚त्य‚क्षानुमानाप्तैः प्र‚सिद्धेनेति क‚थ मा चा र्ये ण ‚{\tiny $_{lb}$}‚च‚तुर्व्विधा सा द‚र्शितेत्याह ।
	\pend% ending standard par
      
	  \bigskip
	  \begingroup
	
	    \large
	  
	    \begin{quote}
	  
	    
	    \stanza[\smallbreak]
	\label{pv.4.92}\flagstanza{\tiny\textenglish{...v.4.92}}अनुमान‚स्य भेदेन सा बाधोक्ता च‚तुर्विधा ।&त‚त्राभ्युपायः कार्याङ्गं स्व‚भावाङ्गं ज‚ग‚स्त्थितिः ॥ ९२ ॥\&[\smallbreak]


	
	    \end{quote}
	  
	  \endgroup
	

	  \pstart \leavevmode% starting standard par
	\hphantom{.}‚{\color{DodgerBlue3}‚अनुमान‚स्य भेदेन} त्रैविध्येन प्र‚त्य‚क्षेण चैकेन ‚{\color{DodgerBlue3}‚स‚ह सा बाधा च‚तुर्व्विधोक्ता (।) ‚{\tiny $_{lb}$}‚त‚त्र} ते‚{\tiny $_{6}$}‚षु बाध‚केष्व‚{\color{DodgerBlue3}‚भ्युपाय} आप्त‚स्व‚व‚च‚ने ‚{\color{DodgerBlue3}‚कार्य‚म‚ङ्गं ज‚ग‚तः स्थिति}‚र्व्य‚व‚हृतिः ‚{\tiny $_{lb}$}‚प्र‚सिद्धिः ‚{\color{DodgerBlue3}‚स्व‚भावोऽङ्गं} हेतुः ॥ (९२)
	\pend% ending standard par
      \label{div_pvv.4.93}
	  
	% new div opening: depth here is 2
	

	  \begin{center}%% label @type='head'
	\textbf{(५) आग‚म‚स्व‚व‚च‚न‚योस्तुल्य‚ब‚ल‚ता}
	\end{center}
	

	  \pstart \leavevmode% starting standard par
	क‚स्मात् पुन‚राप्त‚व‚च‚नं स्व‚व‚च‚न‚ञ्चाभ्युपाय उच्य‚त इत्याह (।)
	\pend% ending standard par
      
	  \bigskip
	  \begingroup
	
	    \large
	  
	    \begin{quote}
	  
	    
	    \stanza[\smallbreak]
	\label{pv.4.93}\flagstanza{\tiny\textenglish{...v.4.93}}आत्माप‚रोधाभिम‚तो भूत‚निश्च‚य‚युक्त‚वाक् ।&आप्तः स्व‚व‚च‚नं शास्त्रं चैक‚मुक्तं स‚म‚त्व‚तः ॥ ९३ ॥\&[\smallbreak]


	
	    \end{quote}
	  
	  \endgroup
	

	  \pstart \leavevmode% starting standard par
	\hphantom{.}‚{\color{DodgerBlue3}‚आत्माप‚रोधाभिम‚तो भूत}‚स्यार्थ‚स्य ‚{\color{DodgerBlue3}‚निश्च‚येन युक्ता} प्र‚युक्ता ‚{\color{DodgerBlue3}‚वाग्} य‚स्य ‚{\color{DodgerBlue3}‚स आप्त} उच्य‚ते (।) एवं प‚रंप‚र‚याऽर्थ‚कार्य‚त्वेन ‚{\color{DodgerBlue3}‚स्व‚व‚च‚नं शास्त्र‚ञ्च स‚म‚त्व‚तो}‚ऽभ्युपाय इति ‚{\tiny $_{lb}$}‚‚{\color{DodgerBlue3}‚स‚म}‚स्य कार्य‚लिङ्ग‚{\color{DodgerBlue3}‚मेक\edtext{}{\edlabel{pvv.445-1}\label{pvv.445-1}\lemma{मेक}\Bfootnote{आप्त‚वाच्यं ।}}मुक्तं} । (९३)
	\pend% ending standard par
      \label{div_pvv.4.94}
	  
	% new div opening: depth here is 2
	

	  \pstart \leavevmode% starting standard par
	किञ्च‚{\tiny $_{7}$}‚ (।)
	\pend% ending standard par
      
	  \bigskip
	  \begingroup
	
	    \large
	  
	    \begin{quote}
	  
	    
	    \stanza[\smallbreak]
	\label{pv.4.94}\flagstanza{\tiny\textenglish{...v.4.94}}य‚थात्म‚नोऽप्र‚माण‚त्वे व‚च‚नं न प्र‚व‚र्त्त‚ते ।&शास्त्र‚सिद्धे त‚था नार्थे विचार‚स्त‚द‚नाश्र‚ये ॥ ९४ ॥\&[\smallbreak]


	
	    \end{quote}
	  
	  \endgroup
	

	  \pstart \leavevmode% starting standard par
	\hphantom{.}व‚क्तु‚{\color{DodgerBlue3}‚रात्म‚नोऽप्र‚माण‚त्वे} प्रामाण्य‚निमित्ताभावात् ‚{\color{DodgerBlue3}‚व‚च‚नं}‚प्रामाण्यं ‚{\color{DodgerBlue3}‚न प्र‚व‚र्त्त‚ते} । न\leavevmode\ledsidenote{\textenglish{89b/MA}} ‚{\tiny $_{lb}$}‚ह्य‚स‚त्यार्थेन व‚च‚नेन प‚रः प्र‚तिपाद‚यितुं श‚क्य‚ते । विस‚म्वाद‚नाश्र‚य‚स्य व‚च‚नाद‚र्था‚{\tiny $_{lb}$}‚प‚त्तेः । त‚तो य‚था प्र‚तिपाद‚यितुः प्रामाण्य एव स‚ति व‚च‚नं ‚{\color{DodgerBlue3}‚प्र‚व‚र्त्त‚ते} (।) ‚{\color{DodgerBlue3}‚त‚था शास्त्र‚{\tiny $_{lb}$}‚सिद्धेऽर्थे त}‚स्य शास्त्र‚प्रामाण्य‚स्या‚{\color{DodgerBlue3}‚नाश्र‚येण न विचारः} प्र‚व‚र्त्त‚ते (।) प्र‚माण‚विष‚यो ‚{\tiny $_{lb}$}‚\leavevmode\ledsidenote{\textenglish{446/s}} ल‚ब्धः साक्षाद‚स्यैव प्र‚माण‚{\tiny $_{1}$}‚स्य शास्त्र‚स्यात्य‚न्त‚प‚रोक्षार्थे विचारः प्र‚स्तूय‚ते प्रेक्षै‚{\tiny $_{lb}$}‚र्नान‚स्य । त‚तः शास्त्र‚स्व‚व‚च‚न‚योः प्रामाण्येऽभ्युप‚ग‚तेपि साम्य‚मुक्तं ॥ (९४)
	\pend% ending standard par
      \label{div_pvv.4.95}
	  
	% new div opening: depth here is 2
	

	  \pstart \leavevmode% starting standard par
	साम्य‚मेव पुनः किम‚र्थ‚मुप‚द‚र्शितामित्याह (।)
	\pend% ending standard par
      
	  \bigskip
	  \begingroup
	
	    \large
	  
	    \begin{quote}
	  
	    
	    \stanza[\smallbreak]
	\label{pv.4.95}\flagstanza{\tiny\textenglish{...v.4.95}}त‚त्प्र‚स्तावाश्र‚य‚त्वे हि शास्त्रं बाध‚क‚मित्य‚मुम् ।&व‚क्तुम‚र्थं स्व‚वाचास्य स‚होक्तिः साम्य‚दृष्ट‚ये ॥ ९५ ॥\&[\smallbreak]


	
	    \end{quote}
	  
	  \endgroup
	

	  \pstart \leavevmode% starting standard par
	\hphantom{.}‚{\color{DodgerBlue3}‚हि} य‚स्मात् ‚{\color{DodgerBlue3}‚त‚त्प्र‚स्ताव}‚स्य विचार‚प्र‚क्र‚म‚स्या‚{\color{DodgerBlue3}‚श्र‚य‚त्वेऽधि}‚क‚र‚ण‚त्वे स‚ति ‚{\color{DodgerBlue3}‚शास्त्रं ‚{\tiny $_{lb}$}‚सिद्धे} ध‚र्मिणि शास्त्रं प्र‚तिज्ञार्थ‚विरुद्धं‚{\color{DodgerBlue3}‚बाध‚कं} न व‚स्तुब‚ल‚प्र‚वृत्तानुमान‚{\color{DodgerBlue3}‚मित्य‚मुम‚र्थं ‚{\tiny $_{lb}$}‚व‚क्तु}‚म‚स्य शास्त्र‚स्य ‚{\color{DodgerBlue3}‚स्व‚{\tiny $_{2}$}‚वाचा साम्य‚स्य दृष्ट‚ये} द‚र्श‚नार्थं ‚{\color{DodgerBlue3}‚स‚होक्ति}‚र‚न‚योर‚भ्युपाय‚{\tiny $_{lb}$}‚तानिर्देशे । स्व‚व‚च‚न‚म‚पि ह्युच्चार‚ण‚साम‚र्थ्यादु\edtext{}{\edlabel{pvv.446-1}\label{pvv.446-1}\lemma{र्थ्यादु}\Bfootnote{स्वाप्रामाण्य‚बोधेनुच्चार‚णात् ।}}प‚ग‚त‚प्रामाण्यं सिद्धे ध‚र्मिणि ‚{\tiny $_{lb}$}‚विचार‚प्र‚क्र‚मे प्र‚तिज्ञार्थ‚विष‚य‚बाध‚कं ‚{\color{DodgerBlue3}‚बाध‚कं} दृष्टं (।) न व‚स्तुब‚ल‚प्र‚वृत्तानुमानेन ‚{\tiny $_{lb}$}‚इद‚म‚न‚योः साम्य‚द‚र्श‚न‚प्र‚योज‚नं । (९५)
	\pend% ending standard par
      \label{div_pvv.4.96_4.97}
	  
	% new div opening: depth here is 2
	
	  \bigskip
	  \begingroup
	
	    \large
	  
	    \begin{quote}
	  
	    
	    \stanza[\smallbreak]
	\label{pv.4.96}\flagstanza{\tiny\textenglish{...v.4.96}}उदाहार‚ण‚म‚प्य‚त्र स‚दृशं तेन व‚र्णित‚म् ।&प्र‚माणानाम‚भावे हि शास्त्र‚वाचोर‚योगातः ॥ ९६ ॥\&[\smallbreak]


	
	    \end{quote}
	  
	  \endgroup
	

	  \pstart \leavevmode% starting standard par
	\hphantom{.}अत्र शास्त्र‚व‚च‚न‚योः व्याघातेना चा र्ये ‚{\color{DodgerBlue3}‚णोदाह‚र‚ण‚म‚पि} स‚दृश‚म‚भिन्नं ‚{\color{DodgerBlue3}‚व‚र्ण्णितं} ।‚{\tiny $_{3}$}‚ ‚{\tiny $_{lb}$}‚‚{\color{DodgerBlue3}‚य‚था न स‚न्ति प्र‚माणानि} प्र‚मेयार्थानीति । क‚थं पुन‚र‚त्र शास्त्रार्थ‚स्व‚व‚च‚नाभ्यां ‚{\tiny $_{lb}$}‚व्याघात इत्याह । ‚{\color{DodgerBlue3}‚प्र‚माणानाम‚भावे हि शास्त्र‚वाचोर‚योगातः} । अनुप‚प‚त्तेः ‚{\tiny $_{lb}$}‚प्र‚माण‚स‚म्भ‚वे हि शास्त्र‚व‚च‚न‚योः प‚र‚प्र‚तिपाद‚नार्थ‚मुक्तिर्युक्ता । त‚स्मादुच्चार‚ण‚{\tiny $_{lb}$}‚‚{\color{DodgerBlue3}‚साम‚र्थ्याद‚भ्यु}‚प‚ग‚त‚प्रामाण्यात् प्र‚योग‚व‚च‚नादेव प्र‚माणाभाव‚प्र‚तिज्ञा बाध्य‚ते ।
	\pend% ending standard par
      

	  \pstart \leavevmode% starting standard par
	एत‚च्चोदाह‚र‚ण‚म् (।)
	\pend% ending standard par
      
	  \bigskip
	  \begingroup
	
	    \large
	  
	    \begin{quote}
	  
	    
	    \stanza[\smallbreak]
	\label{pv.4.97}\flagstanza{\tiny\textenglish{...v.4.97}}स्व‚वाग्विरोधे विस्प‚ष्ट‚मुदाह‚र‚ण‚माग‚मे ।&दिङ्मात्र‚द‚र्श‚नं त‚त्र प्र‚त्येध‚र्मोसुख‚प्र‚दः ॥ ९७ ॥\&[\smallbreak]


	
	    \end{quote}
	  
	  \endgroup
	

	  \pstart \leavevmode% starting standard par
	\hphantom{.}‚{\color{DodgerBlue3}‚स्व‚वाग्विरोधे} स्व‚व‚च‚न‚व्याह‚तौ ‚{\tiny $_{4}$}‚ ‚{\color{DodgerBlue3}‚विस्प‚ष्टं} त‚था हि प्र‚माणाभाव‚प्र‚तिज्ञा‚{\tiny $_{lb}$}‚व‚च‚नोच्चार‚ण‚साम‚र्थ्याभ्युप‚ग‚त‚प्रामाण्येन व‚च‚नेनैव बाध्य‚ते । आग‚मे शास्त्रे ‚{\tiny $_{lb}$}‚पुन‚रुदाह‚र‚ण‚स्य ‚{\color{DodgerBlue3}‚दिङ्मात्र‚द‚र्श‚न}‚मुप‚ल‚क्ष‚ण‚मात्र‚मेत‚त् । न तु मुख्य‚मुदाह‚र‚णं । ‚{\tiny $_{lb}$}‚प्र‚माण‚स्य ध‚र्मिण आग‚म‚सिद्ध‚त्वाभावात् । इदं पुन‚र्मुख्य‚मुदाह‚र‚णं आग‚मे ‚{\color{DodgerBlue3}‚प्रेत्य} प‚र‚{\tiny $_{lb}$}‚लोके ‚{\color{DodgerBlue3}‚ध‚र्मोऽसुख‚प्र‚दः} (।) आग‚म‚सिद्धे ध‚र्मिणि ध‚र्मे सामान्येऽसुख‚प्र‚द‚त्व‚{\tiny $_{5}$}‚स्य ‚{\tiny $_{lb}$}‚विशेष‚स्य सुख‚प्र‚द‚त्वेन विरुद्धेनाग‚म‚सिद्धेन बाध‚नात् । (९७)
	\pend% ending standard par
      \label{div_pvv.4.98}
	  
	% new div opening: depth here is 2
	

	  \pstart \leavevmode% starting standard par
	\hphantom{.}‚{\color{DodgerBlue3}‚किन्तु (।)}
	\pend% ending standard par
      \textsuperscript{\textenglish{447/s}}
	  \bigskip
	  \begingroup
	
	    \large
	  
	    \begin{quote}
	  
	    
	    \stanza[\smallbreak]
	\label{pv.4.98}\flagstanza{\tiny\textenglish{...v.4.98}}शास्त्रिणोप्य‚त‚दाल‚म्बे विरुद्धोक्तौ तु व‚स्तुनि ।&न बाधा प्र‚तिब‚न्धः स्यात् तुल्य‚क‚क्ष‚त‚या त‚योः ॥ ९८ ॥\&[\smallbreak]


	
	    \end{quote}
	  
	  \endgroup
	

	  \pstart \leavevmode% starting standard par
	\hphantom{.}‚{\color{DodgerBlue3}‚शास्त्रिणो}‚ऽभ्युप‚ग‚त‚शास्त्र‚स्या‚{\color{DodgerBlue3}‚त‚दाल‚म्बे} शास्त्रासिद्धे\edtext{}{\edlabel{pvv.447-1}\label{pvv.447-1}\lemma{शास्त्रासिद्धे}\Bfootnote{यो न शास्त्रेण द‚र्शितः ।}} प्र‚माण‚सिद्धे ‚{\color{DodgerBlue3}‚व‚स्तुनि} ध‚र्मिणि शास्त्र‚प्र‚तिज्ञा‚{\color{DodgerBlue3}‚विरुद्ध‚स्य ध‚र्म‚स्योक्तौ न सा बाधा} । य‚था मी मां स क स्य ‚{\tiny $_{lb}$}‚गृहीत‚शास्त्र‚स्य प्र‚त्य‚क्ष‚सिद्धे श‚ब्दे ध‚र्मिणि कृत‚क‚त्वाद‚नित्य‚त्वोक्ताव‚पि ‚{\color{DodgerBlue3}‚शास्त्र}‚{\tiny $_{lb}$}‚प्र‚तिज्ञातेन नित्य‚त्वेन न बाधा । य‚दि बाधा किन्त‚र्हि भ‚व‚तीत्याह । ‚{\color{DodgerBlue3}‚प्र‚तिब‚न्धः‚{\tiny $_{6}$}‚ ‚{\tiny $_{lb}$}‚स्यात्} । क‚स्मादित्याह । ‚{\color{DodgerBlue3}‚तुल्य‚क‚क्ष}‚त्वात् । स‚म‚ब‚ल‚त्वात् । य‚था माता मे ब‚न्ध्या ‚{\tiny $_{lb}$}‚चेति स्व‚वाचि तुल्य‚क‚क्ष‚त्वात् प‚द‚योः प‚र‚स्प‚रं प्र‚तिब‚न्धः । (९८)
	\pend% ending standard par
      \label{div_pvv.4.99}
	  
	% new div opening: depth here is 2
	

	  \pstart \leavevmode% starting standard par
	न‚नु स्व‚व‚च‚न‚योस्तुल्य‚क‚क्ष‚त्वाद् युक्तः प्र‚तिब‚न्धः । आग‚म‚स्व‚व‚च‚न‚योस्तुल्य‚{\tiny $_{lb}$}‚ब‚ल‚तैव क‚थ‚मित्याह ।
	\pend% ending standard par
      
	  \bigskip
	  \begingroup
	
	    \large
	  
	    \begin{quote}
	  
	    
	    \stanza[\smallbreak]
	\label{pv.4.99}\flagstanza{\tiny\textenglish{...v.4.99}}य‚था स्व‚वाचि त‚च्चास्य त‚दा स्व‚व‚च‚नात्म‚क‚म् ।&त‚योः प्र‚माणं य‚स्यास्ति त‚त् स्याद‚न्य‚स्य बाध‚क‚म् ॥ ९९ ॥\&[\smallbreak]


	
	    \end{quote}
	  
	  \endgroup
	

	  \pstart \leavevmode% starting standard par
	\hphantom{.}‚{\color{DodgerBlue3}‚त‚च्च} शास्त्रं नित्य‚त्व‚प्र‚तिज्ञात‚स्य वादिनः ‚{\color{DodgerBlue3}‚त‚दा} प्र‚सिद्धे ध‚र्मिणि शास्त्र‚विरुद्ध-\leavevmode\ledsidenote{\textenglish{90a/MA}} ‚{\tiny $_{lb}$}‚प्र‚तिज्ञास‚म‚ये स्वोप‚ग‚म‚स्वीकृत‚प्र‚माण‚त्वात् ‚{\color{DodgerBlue3}‚स्व‚व‚च‚नात्म‚कं} \edtext{\textsuperscript{*}}{\edlabel{pvv.447-2}\label{pvv.447-2}\lemma{*}\Bfootnote{इति व‚च‚नात्म‚त्वाविशेषः ।}} जा‚{\tiny $_{7}$}‚तं व‚च‚नं शास्त्र‚ञ्च ‚{\tiny $_{lb}$}‚स्व‚य‚म‚भ्युप‚ग‚त‚प्रामाण्य‚युक्तं तुल्य‚क‚क्षं ‚{\color{DodgerBlue3}‚य‚था स्व‚वाचि} माता मे ब‚न्ध्येति व‚च‚न‚{\tiny $_{lb}$}‚मात्र‚योः प्र‚तिब‚न्धोन्योन्यं । ‚{\color{DodgerBlue3}‚त‚योः} शास्त्र‚व‚च‚न‚योर्व्विरूद्धार्थाभिधायिनोर्म‚ध्ये ‚{\color{DodgerBlue3}‚य‚स्य ‚{\tiny $_{lb}$}‚प्र‚माण}‚म‚नुव‚र्त्त‚क‚{\color{DodgerBlue3}‚म‚स्ति त‚त्} प्र‚माण‚व‚त् । ‚{\color{DodgerBlue3}‚अन्य‚स्या}‚प्र‚माण‚क‚स्य ‚{\color{DodgerBlue3}‚बाध‚कं} भ‚व‚ति । य‚था‚{\tiny $_{lb}$}‚ऽनित्य‚त्वं नित्य‚स्य श‚ब्दे । (९९)
	\pend% ending standard par
      \label{div_pvv.4.100}
	  
	% new div opening: depth here is 2
	
	  \bigskip
	  \begingroup
	
	    \large
	  
	    \begin{quote}
	  
	    
	    \stanza[\smallbreak]
	\label{pv.4.100}\flagstanza{\tiny\textenglish{....4.100}}प्र‚तिज्ञाम‚नुमानं वा प्र‚तिज्ञाऽपेत‚युक्तिका ।&तुल्य‚क‚क्षा य‚थार्थ‚म्वा बाधेत क‚थ‚म‚न्य‚था ॥ १०० ॥\&[\smallbreak]


	
	    \end{quote}
	  
	  \endgroup
	

	  \pstart \leavevmode% starting standard par
	अन्य‚था य‚दि प्र‚माण‚सिद्धे ध‚र्मिणि प्र‚माणान‚नुगृहीत‚योः प‚र‚स्प‚रं बाधा । प्र‚मा‚{\tiny $_{lb}$}‚णाभावे केनेत‚र‚स्य नाबाधेतीष्य‚ते त‚दा ‚{\color{DodgerBlue3}‚प्र‚तिज्ञा}‚विप‚री‚{\tiny $_{1}$}‚ता शास्त्र‚व‚च‚नाख्या‚{\color{DodgerBlue3}‚ऽपेत‚{\tiny $_{lb}$}‚युक्तिका} प्र‚तिज्ञा प्र‚माण्याभ्युप‚ग‚मात् ‚{\color{DodgerBlue3}‚तुल्य‚क‚क्षा} स‚ती क‚थं बाधेत । ‚{\color{DodgerBlue3}‚य‚था} श‚ब्दे ‚{\tiny $_{lb}$}‚नित्य‚त्व‚प्र‚तिज्ञा शास्त्रोक्ता अनित्य‚त्व‚{\color{DodgerBlue3}‚प्र‚तिज्ञां} विद्य‚मान‚{\color{DodgerBlue3}‚म‚नुमान‚म्वा} य‚थार्थं । व‚स्तुभूत‚{\tiny $_{lb}$}‚कृत‚क‚त्व‚लि‚{\color{DodgerBlue3}‚ङ्ग}‚स‚मुत्थं नित्य‚त्व‚प्र‚तिज्ञ‚या ‚{\color{DodgerBlue3}‚क‚थं} बाधितं \edtext{}{\edlabel{pvv.447-3}\label{pvv.447-3}\lemma{बाधितं}\Bfootnote{अपि च आग‚मानां प्रामाण्यं प्रागेव निवारित‚मिति कुतो बाधा । त‚स्माच्छास्त्र‚म‚भ्युप‚ग‚म्य ध‚र्म‚विचारेष्व‚य‚न्दोषः शास्त्र‚बाधाख्य इष्य‚ते ।}}स्याद‚नित्य‚त्वात् । (१००)
	\pend% ending standard par
      \label{div_pvv.4.101_4.102}
	  
	% new div opening: depth here is 2
	\textsuperscript{\textenglish{448/s}}
	  \bigskip
	  \begingroup
	
	    \large
	  
	    \begin{quote}
	  
	    
	    \stanza[\smallbreak]
	\label{pv.4.101}\flagstanza{\tiny\textenglish{...१०१}}प्रामाण्य‚माग‚मानाञ्च प्रागेव विनिवारित‚म् ।&अभ्युपाय‚विचारेषु त‚स्माद् दोषोय‚मिष्य‚ते\edtext{}{\edlabel{pvv.448-asterisk}\label{pvv.448-asterisk}\lemma{ते}\Bfootnote{नास्ति वृत्तौ ॥}} ॥  ॥\&[\smallbreak]


	
	    \end{quote}
	  
	  \endgroup
	
	  \bigskip
	  \begingroup
	
	    \large
	  
	    \begin{quote}
	  
	    
	    \stanza[\smallbreak]
	\label{pv.4.102}\flagstanza{\tiny\textenglish{....4.102}}त‚स्माद् विष‚य‚भेद‚स्य द‚र्श‚नार्थं पृथ‚क्कृतः ।&अनुमानाब‚र्हिर्भूतोप्य‚भ्युपायः प्र‚बाध‚नात् ॥ १०२ ॥\&[\smallbreak]


	
	    \end{quote}
	  
	  \endgroup
	

	  \pstart \leavevmode% starting standard par
	\hphantom{.}‚{\color{DodgerBlue3}‚त‚स्मात्} कार्य‚लिङ्ग‚त्वाद‚नुमानाद‚ब‚हिर्भूतोप्य‚भ्युपायो ‚{\color{DodgerBlue3}‚विष‚य‚स्य भेदो} नानात्वं ‚{\tiny $_{lb}$}‚‚{\color{DodgerBlue3}‚त‚द्द‚र्श‚नार्थ}‚म‚नुमानात् ‚{\color{DodgerBlue3}‚पृथ‚क्कृतः} । अनु‚{\tiny $_{2}$}‚मानं स‚र्व्व‚त्र बाध‚कं शास्त्र‚न्तु शास्त्रा‚{\tiny $_{lb}$}‚श्र‚ये ध‚र्मिणीति बाध‚क‚त्व‚विष‚य‚भेदोप‚द‚र्श‚नं पृथ‚क्क‚र‚ण‚फ‚लं । क‚थं ज्ञाय‚तेऽ‚{\color{DodgerBlue3}‚नुमाना‚{\tiny $_{lb}$}‚ब‚हिर्भूतं} शास्त्र‚मित्याह । स्व‚सिद्धे ध‚र्मिणि स्वोप‚ग‚म‚विरुद्ध‚स्य ध‚र्म‚स्य ‚{\color{DodgerBlue3}‚प्र‚बाध‚नात्} न ह्य‚प्र‚माण‚म्बाध‚कं । प्र‚माण‚ञ्चाप्र‚त्य‚क्ष‚त्वात् अनुमान‚मेव ॥ (१०१,१०२)
	\pend% ending standard par
      \label{div_pvv.4.103}
	  
	% new div opening: depth here is 2
	
	  \bigskip
	  \begingroup
	
	    \large
	  
	    \begin{quote}
	  
	    
	    \stanza[\smallbreak]
	\label{pv.4.103a}\flagstanza{\tiny\textenglish{...4.103a}}अन्य‚थाऽतिप्र‚स‚ङ्गः स्याद् व्य‚र्थ‚ता वा पृथ‚क्कृतेः ।\&[\smallbreak]


	
	    \end{quote}
	  
	  \endgroup
	

	  \pstart \leavevmode% starting standard par
	\hphantom{.}‚{\color{DodgerBlue3}‚अन्य‚था} य‚दि विष‚य‚भेदोप‚द‚र्श‚न‚फ‚लं पृथ‚क्क‚र‚णं नेष्य‚ते त‚दा प्र‚भेद‚द‚र्श‚नार्थं ‚{\tiny $_{lb}$}‚व‚क्त‚व्यं । त‚था ‚{\tiny $_{3}$}‚ च कार्य‚स्व‚भावानुप‚ल‚म्भानां प्र‚भेदो याव‚त् \edtext{}{\edlabel{pvv.448-1}\label{pvv.448-1}\lemma{त्}\Bfootnote{तैर‚पि बाध‚नात् ।}} स‚म्भ‚वं वाच्य ‚{\tiny $_{lb}$}‚इत्य‚ति‚{\color{DodgerBlue3}‚प्र‚स‚ङ्गः} स्यात् ।
	\pend% ending standard par
      

	  \pstart \leavevmode% starting standard par
	\hphantom{.}अथ स‚द‚पि भेदान्त‚रं नोच्य‚ते त‚दाऽनुमानाच्छास्त्र‚स्य ‚{\color{DodgerBlue3}‚पृथ‚क् कृतेर्व्य‚र्थ‚ता वा} स्यात् । प्र‚भेद‚व‚च‚न‚स्याविव‚क्षित‚त्वात् । सामान्य‚व‚च‚न‚स्यानुमानेनैव सिद्ध‚त्वात् । ‚{\tiny $_{lb}$}‚य‚दि विष‚य‚भेदोप‚द‚र्श‚नार्थ‚म‚नुमानात् पृथ‚ग्व‚च‚नं शास्त्र‚स्य त‚दाभ्युप‚ग‚मात् स्व‚व‚{\tiny $_{lb}$}‚च‚न‚माचार्येण किम‚र्थं पृथ‚क्कृतं (न्याय‚मु खे) । य‚था स‚र्व्व‚मु‚{\tiny $_{4}$}‚क्तं मृषेति (।) ‚{\tiny $_{lb}$}‚त‚था औलूक्य‚स्य नित्यः श‚ब्द इति ।
	\pend% ending standard par
      

	  \pstart \leavevmode% starting standard par
	अत्राह (।)
	\pend% ending standard par
      
	  \bigskip
	  \begingroup
	
	    \large
	  
	    \begin{quote}
	  
	    
	    \stanza[\smallbreak]
	\label{pv.4.103b}\flagstanza{\tiny\textenglish{...4.103b}}भेदो वाङ्मात्र‚व‚च‚ने प्र‚तिब‚न्धः स्व‚वाच्य‚पि ॥ १०३ ॥\&[\smallbreak]


	
	    \end{quote}
	  
	  \endgroup
	

	  \pstart \leavevmode% starting standard par
	\hphantom{.}‚{\color{DodgerBlue3}‚स्व‚वाच्य‚पि भेदः} स्व‚व‚च‚न‚स्यापि शास्त्रात् पृथ‚क्क‚र‚णं अप्र‚माण‚कं व‚च‚नं ‚{\tiny $_{lb}$}‚‚{\color{DodgerBlue3}‚वाङ्मात्रं} त‚स्मिन् ‚{\color{DodgerBlue3}‚प्र‚तिब‚न्धः} । उच्चार‚ण‚साम‚र्थ्याद‚भ्युप‚ग‚त‚प्रामाण्यं स्व‚व‚च‚नं ।
	\pend% ending standard par
      

	  \pstart \leavevmode% starting standard par
	अन्य‚थोच्चार‚ण‚मेव न स्यात् । स‚त्यार्थ‚ता च प्रामाण्यं त‚न्मृषार्थ‚त‚या वाच्य‚या ‚{\tiny $_{lb}$}‚निषेध्य‚त इति तुल्य‚क‚क्ष‚त‚या प्र‚तिब‚न्ध एवान‚योर्न बाधा । त‚देनं वाक्यं स्वार्थं ‚{\tiny $_{lb}$}‚प्र‚{\tiny $_{5}$}‚तिब‚ध्नाति । वाक्यान्त‚र‚निर्दिष्ट‚म्व‚स्तु शास्त्र‚मित्य‚न‚योर्भेद इत्युक्तं । (१०३)
	\pend% ending standard par
      \label{div_pvv.4.104}
	  
	% new div opening: depth here is 2
	

	  \pstart \leavevmode% starting standard par
	स्यादेत‚त् (।)
	\pend% ending standard par
      \textsuperscript{\textenglish{449/s}}
	  \bigskip
	  \begingroup
	
	    \large
	  
	    \begin{quote}
	  
	    
	    \stanza[\smallbreak]
	\label{pv.4.104}\flagstanza{\tiny\textenglish{....4.104}}तेनाभ्युपाग‚माच्छास्त्रं प्र‚माणं स‚र्व‚व‚स्तुषु ।&बाध‚कं, य‚दि नेच्छेत् स बाध‚कं किम्पुन‚र्भ‚वेत् ॥ १०४ ॥\&[\smallbreak]


	
	    \end{quote}
	  
	  \endgroup
	

	  \pstart \leavevmode% starting standard par
	\hphantom{.}शास्त्र‚स्य ‚{\color{DodgerBlue3}‚तेन} वादिना‚{\color{DodgerBlue3}‚भ्युप‚ग‚मात् स‚र्व्व‚त्र व‚स्तुषु} ध‚र्मिषु शास्त्र‚सिद्धे व‚स्तुब‚ल‚{\tiny $_{lb}$}‚प्र‚वृत्ते प्र‚माण‚निश्चितेषु च ‚{\color{DodgerBlue3}‚शास्त्रं प्र‚माणं} स‚त् ‚{\color{DodgerBlue3}‚बाध‚क‚मेव} स्यात् (।) विप‚रीत‚{\tiny $_{lb}$}‚प्र‚तिज्ञाया न प्र‚तिब‚न्ध‚कं । त‚त्क‚थं प्र‚माणीकृत‚शास्त्र‚स्य मीमान्स‚क‚स्य श‚ब्दे प्र‚त्य‚{\tiny $_{lb}$}‚क्ष‚सिद्धे कृत‚क‚त्वाद‚नित्य‚त्व‚प्र‚तिज्ञायाः साध्य‚मानायाः शास्त्रेण प्र‚तिब‚{\tiny $_{6}$}‚न्धो न ‚{\tiny $_{lb}$}‚बाधेत्युक्तं । \edtext{\textsuperscript{*}}{\edlabel{pvv.449-1}\label{pvv.449-1}\lemma{*}\Bfootnote{सिद्धान्ती ब‚न्धाति ।}} अथ ‚{\color{DodgerBlue3}‚स} वादी ‚{\color{DodgerBlue3}‚य‚दि} शास्त्रं प्र‚माणं ‚{\color{DodgerBlue3}‚नेच्छेत्} त‚दा ‚{\color{DodgerBlue3}‚किं पुन‚र्द्ध‚र्म‚स्यासुख}‚{\tiny $_{lb}$}‚प्र‚द‚त्व‚स्य बाधायां प्र‚माणं ‚{\color{DodgerBlue3}‚भ‚वेत्} । न ह्य‚प्र‚माणं क्व‚चित् प्र‚माणीकृत‚शास्त्र‚स्य श‚ब्दे ‚{\tiny $_{lb}$}‚प्र‚त्य‚क्ष‚सिद्धे भ‚वितुम‚र्ह‚ति । (१०४)
	\pend% ending standard par
      \label{div_pvv.4.105}
	  
	% new div opening: depth here is 2
	

	  \pstart \leavevmode% starting standard par
	ध‚र्म‚स्य सुख‚प्र‚द‚त्वेन निर्देशाद‚सुख‚प्र‚द‚त्वे साध्ये स्व‚व‚च‚न‚विरोध एवेति चेत् । ‚{\tiny $_{lb}$}‚एवं त‚र्हि (।)
	\pend% ending standard par
      
	  \bigskip
	  \begingroup
	
	    \large
	  
	    \begin{quote}
	  
	    
	    \stanza[\smallbreak]
	\label{pv.4.105}\flagstanza{\tiny\textenglish{....4.105}}स्व‚वाग्विरोधेऽभेदः स्यात् स्व‚वाक्शास्त्र‚विरोध‚योः ।&पुरुषेच्छाकृता चास्य प‚रिपूर्ण्णा प्र‚माण‚ता ॥ १०५ ॥\&[\smallbreak]


	
	    \end{quote}
	  
	  \endgroup
	

	  \pstart \leavevmode% starting standard par
	\hphantom{.}‚{\color{DodgerBlue3}‚स्व‚वाग्विरोधे}‚ऽभ्युप‚ग‚म्य‚माने ‚{\color{DodgerBlue3}‚स्व‚वाक्शास्त्र‚विरोध‚योर‚भेद} एव ‚{\color{DodgerBlue3}‚स्यात्} । ‚{\tiny $_{lb}$}‚द्व‚यो‚{\tiny $_{7}$}‚र‚भ्युप‚ग‚म‚सिद्ध‚प्र‚माण‚त्वात् । \leavevmode\ledsidenote{\textenglish{90b/MA}}‚{\color{DodgerBlue3}‚पुरुषेच्छाकृता चास्य} शास्त्र‚स्य ‚{\color{DodgerBlue3}‚प‚रिपूर्ण्णा ‚{\tiny $_{lb}$}‚प्र‚माण‚ते}‚त्युप‚ह‚स‚ति । स‚त्य‚प्याग‚म‚स्व‚व‚च‚न‚योर‚भ्युप‚ग‚माहित‚प्रामाण्याद‚भेदे ‚{\color{DodgerBlue3}‚भेद}‚{\tiny $_{lb}$}‚द‚र्श‚न‚निमित्त‚ञ्चोक्तं । (१०५)
	\pend% ending standard par
      \label{div_pvv.4.106}
	  
	% new div opening: depth here is 2
	

	  \pstart \leavevmode% starting standard par
	य‚स्माच्छास्त्रं त‚त्सिद्ध एव ध‚र्मिणि लिङ्गे च बाध‚कं न तु प्र‚माण‚सिद्धेपि ।
	\pend% ending standard par
      
	  \bigskip
	  \begingroup
	
	    \large
	  
	    \begin{quote}
	  
	    
	    \stanza[\smallbreak]
	\label{pv.4.106}\flagstanza{\tiny\textenglish{....4.106}}त‚स्मात् प्र‚सिद्ध‚ष्व‚र्थेषु शास्त्र‚त्यागेपि न क्ष‚तिः ।&प‚रोक्षेष्वाग‚माऽनिष्टौ न चिन्तैव प्र‚व‚र्त्त‚ते ॥ १०६ ॥\&[\smallbreak]


	
	    \end{quote}
	  
	  \endgroup
	

	  \pstart \leavevmode% starting standard par
	\hphantom{.}‚{\color{DodgerBlue3}‚त‚स्मात्} प्र‚त्य‚क्षानुमानाभ्यां ‚{\color{DodgerBlue3}‚प्र‚सिद्धेषु} ध‚र्मिलिङ्ग‚साध्य‚स‚म्ब‚न्धादिषु स‚त्सु ‚{\tiny $_{lb}$}‚‚{\color{DodgerBlue3}‚शास्त्र‚त्यागेपि न क्ष‚तिः\edtext{}{\edlabel{pvv.449-2}\label{pvv.449-2}\lemma{तिः}\Bfootnote{इष्टाप्र‚तिब‚न्धात् ।}}} । य‚था श‚ब्द‚कृत‚त्वानित्य‚त्व‚स‚म्ब‚न्धादिषु प्र‚माण‚सि-‚{\tiny $_{1}$}‚ ‚{\tiny $_{lb}$}‚द्धेषु शास्त्र‚स्याकाश‚गुण‚त्व‚प्र‚तिपाद‚क‚स्य त्यागेपि नानिष्टं । ‚{\color{DodgerBlue3}‚प‚रोक्षेषु} ध‚र्मा‚{\tiny $_{lb}$}‚ध‚र्मादिषु \edtext{}{\edlabel{pvv.449-3}\label{pvv.449-3}\lemma{र्मादिषु}\Bfootnote{आग‚मो ग्राह्य एव ।}} पुन‚राग‚म‚स्य प्र‚माण‚त्वेनानिष्टौ ‚{\color{DodgerBlue3}‚चिन्तैव न प्र‚व‚र्त्त‚ते(।) न‚ह्य‚सिद्धे ध‚र्मिणि} विचारः । (१०६)
	\pend% ending standard par
      \label{div_pvv.4.107}
	  
	% new div opening: depth here is 2
	

	  \pstart \leavevmode% starting standard par
	न‚नु शास्त्र‚ञ्चेन्न प्र‚माणं क‚थ‚न्त‚त्सिद्धे ध‚र्मिणि लिङ्गादौ वा विचार इत्याह ।
	\pend% ending standard par
      \textsuperscript{\textenglish{450/s}}
	  \bigskip
	  \begingroup
	
	    \large
	  
	    \begin{quote}
	  
	    
	    \stanza[\smallbreak]
	\label{pv.4.107}\flagstanza{\tiny\textenglish{....4.107}}विरोधोद्भाव‚न‚प्राया प‚रीक्षाप्य‚त्र त‚द्य‚था ।&अध‚र्म‚मूलं रागादि स्नान‚ञ्चाध‚र्म‚नाश‚न‚म् ॥ १०७ ॥\&[\smallbreak]


	
	    \end{quote}
	  
	  \endgroup
	

	  \pstart \leavevmode% starting standard par
	\hphantom{.}‚{\color{DodgerBlue3}‚अत्र} शास्त्रे ‚{\color{DodgerBlue3}‚प‚रीक्षापि} या क्रिय‚ते सा पूर्व्वाप‚राभ्यां ‚{\color{DodgerBlue3}‚विरोधोद्भाव‚न‚प्राया न} वास्त‚वी ‚{\color{DodgerBlue3}‚त‚द्य‚थाऽध‚र्म‚स्य मूलं रागादीति} क्व‚चिदुक्तं ‚{\color{DodgerBlue3}‚स्नान‚ञ्चा\edtext{}{\edlabel{pvv.450-1}\label{pvv.450-1}\lemma{ञ्चा}\Bfootnote{ज‚प‚होमादि ।}}ध‚र्म‚{\tiny $_{2}$}‚नाश‚न-\edtext{\textsuperscript{*}}{\edlabel{pvv.450-2}\label{pvv.450-2}\lemma{*}\Bfootnote{चित्त‚म‚न्त‚र्ग‚तं दुष्टं ती(र्थ)स्नानैर्न शुध्य‚ति ।  --- श‚त‚शोपि त‚द्धौतं क्षुराभाण्ड‚मिवाशुचि ॥  --- पुनः  --- ग‚ङ्गाद्वारे कुशाव‚र्ते विल्व‚कीनील‚प‚र्व‚ते ।  --- स्नात्वा क‚न‚ख‚ले तीर्थे स‚म्भ‚वेन्न पुन‚र्भ‚वे ॥}}} मित्य‚त्रोच्य‚मान‚म्विरुण‚द्धि । रागाद‚यो हि पाप‚निदानं न च निदानाविरोधे निदा‚{\tiny $_{lb}$}‚निनो बाधा । त‚त्क‚थं रागाद्य‚विरोधि स्नानं पाप‚विरोधि स्यात् ।\edtext{\textsuperscript{*}}{\edlabel{pvv.450-3}\label{pvv.450-3}\lemma{*}\Bfootnote{य‚द्य‚न्निदानं न बाध‚ते त‚न्न त‚द्बाध‚कं य‚था श्लेष्म‚णो म‚धुरादि ॥ येन य‚न्निदान बाध‚नास्तेभ्यः त‚न्निवृत्तिर्य‚था शीलं दुश्च‚रित‚स्य स‚माधिः प‚र्य‚व‚स्थान‚स्य । प्र‚ज्ञाऽनुश‚य‚स्य बाधिका ।}} अप्र‚माणे शास्त्रे ‚{\tiny $_{lb}$}‚विरोधोद्भाव‚न‚प्रायापि चिन्ता क‚स्मात् प्र‚व‚र्त्त्य‚ते इति चेत् । दानादिचेत‚नानां ‚{\tiny $_{lb}$}‚प्र‚वृत्तेर्म‚हानु‚{\color{DodgerBlue3}‚संशा}‚(? शंसा)श्र‚व‚णात् (।) हिंसादिचेत‚नानां म‚हापाप‚श्र‚व‚णाच्च ‚{\tiny $_{lb}$}‚(।) अपेक्षित‚फ‚लेषु दानादिष्व‚यं पुरुषः प्र‚वृत्तिका‚{\tiny $_{3}$}‚मो नाग‚म‚प्रामाण्य‚म‚नाश्रित्या‚{\tiny $_{lb}$}‚सितुं स‚म‚र्थः । (१०७)
	\pend% ending standard par
      \label{div_pvv.4.108}
	  
	% new div opening: depth here is 2
	

	  \begin{center}%% label @type='head'
	\textbf{(६) प्र‚तीतिबाधा ।}
	\end{center}
	‚{\tiny $_{lb}$}‚

	  \pstart \leavevmode% starting standard par
	त‚तः (।)
	\pend% ending standard par
      
	  \bigskip
	  \begingroup
	
	    \large
	  
	    \begin{quote}
	  
	    
	    \stanza[\smallbreak]
	\label{pv.4.108}\flagstanza{\tiny\textenglish{....4.108}}शास्त्रं य‚त्सिद्ध‚या युक्त्या स्व‚वाचा च न बाध्य‚ते ।&दृष्टेऽदृष्टेपि त‚द् ग्राह्य‚मिति चिन्ता प्र‚व‚र्त्त्य‚ते ॥ १०८ ॥\&[\smallbreak]


	
	    \end{quote}
	  
	  \endgroup
	

	  \pstart \leavevmode% starting standard par
	\hphantom{.}‚{\color{DodgerBlue3}‚य‚च्छास्त्रं दृष्टे} प्र‚माणे-विष‚ये ‚{\color{DodgerBlue3}‚युक्त्या} प्र‚त्य‚क्षाद्याख्य‚या ‚{\color{DodgerBlue3}‚न बाध्य‚ते । अदृष्टे} प्र‚माण‚विष‚ये च स्व‚वाचाऽग‚माश्र‚ये\edtext{}{\edlabel{pvv.450-4}\label{pvv.450-4}\lemma{ये}\Bfootnote{अत्य‚न्त‚प‚रोक्षे ।}}णानुमानेन न बाध्य‚ते (।) त‚त्प्र‚माण‚त्वेनादृष्टे ‚{\tiny $_{lb}$}‚विष‚ये प्र‚वृत्तिकाम‚स्य ‚{\color{DodgerBlue3}‚ग्राह्यं} न तु य‚त्किञ्चि‚{\color{DodgerBlue3}‚दित्य‚नेन} प्र‚योज‚नेन शास्त्रे विरोधो‚{\tiny $_{lb}$}‚द्भाव‚न‚प्राया ‚{\color{DodgerBlue3}‚चिन्ता प्र‚व‚र्त्त्य‚ते} । (१०८)
	\pend% ending standard par
      \label{div_pvv.4.109}
	  
	% new div opening: depth here is 2
	

	  \pstart \leavevmode% starting standard par
	शास्त्र‚स्व‚व‚च‚न‚विरोधौ व्याख्यातौ ।
	\pend% ending standard par
      \textsuperscript{\textenglish{451/s}}

	  \begin{center}%% label @type='head'
	\textbf{(क. आप्त‚ल‚क्ष‚ण‚म्)}
	\end{center}
	

	  \pstart \leavevmode% starting standard par
	प्र‚तीतिबा‚{\tiny $_{4}$}‚धां व्याख्यातुमाह ।
	\pend% ending standard par
      
	  \bigskip
	  \begingroup
	
	    \large
	  
	    \begin{quote}
	  
	    
	    \stanza[\smallbreak]
	\label{pv.4.109}\flagstanza{\tiny\textenglish{....4.109}}अर्थेष्व‚प्र‚तिषिद्ध‚त्वात् पुरुषेच्छानुरोधिनः ।&इष्ट‚श‚ब्दाभिधेय‚स्याप्तो वाक्ष‚त‚वाग्ज‚नः ॥ १०९ ॥\&[\smallbreak]


	
	    \end{quote}
	  
	  \endgroup
	

	  \pstart \leavevmode% starting standard par
	\hphantom{.}इष्ट‚श‚ब्दाभिधेय‚त्व‚स्याभिम‚त‚वाच्य‚त्व‚स्य ‚{\color{DodgerBlue3}‚पुरुषेच्छानुरोधिनः} पुरुषेच्छाधीन‚स्या‚{\tiny $_{lb}$}‚‚{\color{DodgerBlue3}‚र्थेष्व‚प्र‚तिषिद्ध‚त्वात्} (।) न हि पुरुषेच्छायाम‚पि श‚शी च‚न्द्र‚श‚ब्दं वाच‚क‚त‚या न ‚{\tiny $_{lb}$}‚स्वीक‚रोति । त‚त‚श्चात्रेष्ट‚{\color{DodgerBlue3}‚श‚ब्दाभिधेय}‚त्वे विष‚ये ‚{\color{DodgerBlue3}‚आप्तो} व्य‚व‚ह‚र्त्ता ‚{\color{DodgerBlue3}‚ज‚नोऽक्ष‚त‚वाग}‚{\tiny $_{lb}$}‚प्र‚तिषिद्धेष्ट‚व‚च‚नः । अनेन श‚ब्देनाय‚म‚र्थो म‚याऽ‚{\color{DodgerBlue3}‚भिधात‚व्य इति क‚ल्प‚नाविष‚य‚त्व}‚{\tiny $_{lb}$}‚मिष्ट‚श‚ब्दाभि‚{\tiny $_{5}$}‚धेय‚त्वं । त‚त्र च पुरुष‚स्यारोपेण स्वेच्छाधीना व‚च‚न‚प्र‚वृत्तिः । त‚स्मा‚{\tiny $_{lb}$}‚दिष्ट‚श‚ब्दाभिधेय‚त्व‚बाधः प‚क्षीक्रिय‚मास्तेनैव स्व‚स‚म्वेद‚न‚सिद्धेन बाध्य‚ते । (१०९)
	\pend% ending standard par
      \label{div_pvv.4.110}
	  
	% new div opening: depth here is 2
	
	  \bigskip
	  \begingroup
	
	    \large
	  
	    \begin{quote}
	  
	    
	    \stanza[\smallbreak]
	\label{pv.4.110}\flagstanza{\tiny\textenglish{....4.110}}उक्तः प्र‚सिद्ध‚श‚ब्देन ध‚र्म‚स्त‚द्व्य‚व‚हार‚जः ।&प्र‚त्य‚क्षादिमिता मान‚श्रुत्यारोपेण सूचिताः ॥ ११० ॥\&[\smallbreak]


	
	    \end{quote}
	  
	  \endgroup
	

	  \pstart \leavevmode% starting standard par
	\hphantom{.}च‚न्द्र‚श्च‚न्द्र इत्यादिश‚ब्द‚{\color{DodgerBlue3}‚व्य‚व‚हाराज्जातो ध‚र्मः} क‚ल्प‚नाविष‚यो योग्य‚ताख्य ‚{\tiny $_{lb}$}‚आचार्येण ‚{\color{DodgerBlue3}‚प्र‚सिद्ध‚श‚ब्देन} त‚द्य‚था शाब्द‚प्र‚सिद्धेनेत्यादिनोक्तः । शाब्दी प्र‚सिद्धिर्व्य‚व‚हारः ‚{\tiny $_{lb}$}‚शाब्द‚प्र‚सिद्धिः त‚द्भ‚वो विष‚यः शाब्द‚प्र‚सिद्धः ते‚{\tiny $_{6}$}‚न बाधेत्य‚र्थः । न केव‚ल‚मिहैव ‚{\tiny $_{lb}$}‚प्र‚त्य‚क्षादिबाधास्व‚पि ‚{\color{DodgerBlue3}‚मान‚श्र‚तौ\edtext{}{\edlabel{pvv.451-1}\label{pvv.451-1}\lemma{तौ}\Bfootnote{नेह प्र‚त्य‚क्षाबाध‚कं किं त्वेतैर्मिताः प्र‚त्य‚क्षानुमान‚शास्त्रं प‚रिच्छिन्नाः श्राव‚ण‚त्वाद‚यः प्र‚तिज्ञार्थ‚स्य । प्र‚त्य‚क्षातिश्रुतौ श्राव‚ण‚त्वाद्यारोपेण ।}}} मेय‚स्या‚{\color{DodgerBlue3}‚रोपेण} प्र‚त्य‚क्षादिभ्याम‚नुमानाग‚माभ्यां ‚{\tiny $_{lb}$}‚मितार्था एव विरोधिनो वाच‚क‚त्वे प‚क्ष‚स्य ‚{\color{DodgerBlue3}‚सूचिताः} । (११०)
	\pend% ending standard par
      \label{div_pvv.4.111}
	  
	% new div opening: depth here is 2
	

	  \pstart \leavevmode% starting standard par
	न हि प्र‚त्य‚क्षानुमाने पुरुष‚चित्त‚व‚र्तिनी आग‚म‚श्च बाध्य‚माने ध‚र्मिणि स‚म्भ‚{\tiny $_{lb}$}‚व‚न्ति\edtext{}{\edlabel{pvv.451-2}\label{pvv.451-2}\lemma{न्ति}\Bfootnote{एषः (?।) त‚त्रोप‚ल‚ब्धात् त‚स्य च ध‚र्मा (ः) ।}} । त‚दुप‚ल‚ब्ध‚त‚द्ध‚र्म्माः स‚न्तो युज्य‚न्ते बाध‚कास्त‚स्मात्(।)
	\pend% ending standard par
      
	  \bigskip
	  \begingroup
	
	    \large
	  
	    \begin{quote}
	  
	    
	    \stanza[\smallbreak]
	\label{pv.4.111}\flagstanza{\tiny\textenglish{....4.111}}त‚दाश्र‚य‚भुवामिच्छानुरोधाद‚निषेधिनाम् ।&कृतानाम‚कृतानां च योग्यं विश्वं स्व‚भाव‚तः ॥ १११ ॥\&[\smallbreak]


	
	    \end{quote}
	  
	  \endgroup
	

	  \pstart \leavevmode% starting standard par
	त‚स्य लोक‚स्याश्र‚येण भुवां भ‚व‚तां श‚ब्दानां व्याव‚हारिकाणां‚{\tiny $_{7}$}‚इच्छानुरोधिनां(अ)\leavevmode\ledsidenote{\textenglish{91a/MA}} ‚{\tiny $_{lb}$}‚ऽर्थेषु वाच‚क‚त्व‚प्र‚वृत्तेः कार‚णात् क्व‚चिद‚पि विष‚येऽनिषेधिनां निषेध‚र‚हितानां कृतानां ‚{\tiny $_{lb}$}‚संकेतितानाम‚संकेतितानाञ्च विश्व‚मिदं वाच्य‚त्वेन स्व‚भावादेव योग्यं ॥ (१११)
	\pend% ending standard par
      \label{div_pvv.4.112}
	  
	% new div opening: depth here is 2
	

	  \begin{center}%% label @type='head'
	\textbf{ख. योग्य‚ता प्र‚सिद्धिश‚ब्दार्थ;}
	\end{center}
	

	  \pstart \leavevmode% starting standard par
	य‚त‚श्चेष्ट‚श‚ब्दाभिधेय‚त्व‚योग्य‚ता स‚र्व्व‚त्र स‚म्भ‚व‚ति । अतो विशेषान‚पेक्ष‚णाद् (।)
	\pend% ending standard par
      \textsuperscript{\textenglish{452/s}}
	  \bigskip
	  \begingroup
	
	    \large
	  
	    \begin{quote}
	  
	    
	    \stanza[\smallbreak]
	\label{pv.4.112}\flagstanza{\tiny\textenglish{....4.112}}अर्थ‚मात्रानुरोधिन्या भाविन्या भूत‚यापि वा ।&बाध्य‚ते प्र‚तिरुन्धानः श‚ब्द‚योग्य‚त‚या त‚या ॥ ११२ ॥\&[\smallbreak]


	
	    \end{quote}
	  
	  \endgroup
	

	  \pstart \leavevmode% starting standard par
	\hphantom{.}‚{\color{DodgerBlue3}‚अर्थ‚मात्रानुरोधिन्या त‚या श‚ब्द‚योग्य‚त‚या भा}‚विसंकेतापेक्ष‚या ‚{\color{DodgerBlue3}‚भाविन्या} । ‚{\tiny $_{lb}$}‚अतीत‚संकेतापेक्ष‚या ‚{\color{DodgerBlue3}‚भूत‚यापि वा} व्य‚व‚स्थि‚{\tiny $_{1}$}‚त‚या तामेव (‚{\color{DodgerBlue3}‚योग्य‚तां प्र‚तिरुन्धानः‚{\tiny $_{1}$}‚} । ‚{\tiny $_{lb}$}‚‚{\color{DodgerBlue3}‚य‚थाऽच‚न्द्रः} श‚शी स‚त्वादिति ‚{\color{DodgerBlue3}‚बाध्य‚ते}‚) । (११२)
	\pend% ending standard par
      \label{div_pvv.4.113}
	  
	% new div opening: depth here is 2
	
	  \bigskip
	  \begingroup
	
	    \large
	  
	    \begin{quote}
	  
	    
	    \stanza[\smallbreak]
	\label{pv.4.113}\flagstanza{\tiny\textenglish{....4.113}}त‚द्योग्य‚ताब‚लादेव व‚स्तुतो घ‚टितो ध्व‚निः ।&स‚र्व्वोस्याम‚प्र‚तीतेपि त‚स्मिंस्त‚त्सिद्ध‚ता त‚तः ॥ ११३ ॥\&[\smallbreak]


	
	    \end{quote}
	  
	  \endgroup
	

	  \pstart \leavevmode% starting standard par
	योग्य‚ताब‚लादेव व‚स्तुतः साम‚र्थ्यात् स‚र्व्वः संकेतितोऽसंकित‚त‚श्च ध्व‚निर‚स्यां ‚{\tiny $_{lb}$}‚योग्य‚तायां घ‚टितः स‚म्ब‚द्धः साक्षात् वाच‚क‚त्वेन त‚स्मिन् श‚ब्देऽप्र‚तीतेपि व‚स्तुनि ‚{\tiny $_{lb}$}‚अप्रातिकूल्य‚ल‚क्ष‚ण‚स्य योग्य‚त्व‚स्य स‚र्व्व‚दा स्थितेः । य‚त एवं त‚त‚स्त‚स्या योग्य‚{\tiny $_{lb}$}‚तायः श‚ब्द‚प्र‚सिद्ध‚ताचार्येणोक्ता । य‚द् य‚त्र स‚म‚र्थ त‚द‚संमुखीभावेपि त‚त् तेन ‚{\tiny $_{lb}$}‚व्य‚प‚दिष्य (? श्य)ते य‚था पाच‚क इति । स‚म‚र्थ‚ञ्च व‚स्त्विष्ट‚श‚ब्दाभिधे‚{\tiny $_{lb}$}‚‚{\tiny $_{2}$}‚य‚त्व इति कृत्वा शाब्द‚प्र‚सिद्ध इति । (११३)
	\pend% ending standard par
      \label{div_pvv.4.114}
	  
	% new div opening: depth here is 2
	
	  \bigskip
	  \begingroup
	
	    \large
	  
	    \begin{quote}
	  
	    
	    \stanza[\smallbreak]
	\label{pv.4.114}\flagstanza{\tiny\textenglish{....4.114}}असाधार‚ण‚ता न स्यात् बाधाहेतोरिहान्य‚था ।&त‚न्निषेधोऽनुमानात् स्याच्छ‚ब्दार्थेऽन‚क्ष‚वृत्तितः ॥ ११४ ॥\&[\smallbreak]


	
	    \end{quote}
	  
	  \endgroup
	

	  \pstart \leavevmode% starting standard par
	\hphantom{.}‚{\color{DodgerBlue3}‚अन्य‚था} य‚दि श‚ब्दोऽसंमुखीभ‚व‚न्न‚पि योग्य‚तायां न स‚म्ब‚द्धः । त‚देह श‚ब्द‚यो‚{\tiny $_{lb}$}‚ग्य‚ताप्र‚तिषेधे क‚र्त्त‚व्ये ‚{\color{DodgerBlue3}‚बाधाहेतोर‚च‚न्द्रः} श‚शी स‚त्वादेरित्यादेर‚{\color{DodgerBlue3}‚साधार‚ण‚तोक्ता न ‚{\tiny $_{lb}$}‚स्यात्} । स‚र्व्व‚स्य च‚न्द्र‚श‚ब्द‚योग्य‚त्वे स‚प‚क्षाभावात्म‚त्व‚साधार‚णं स्यान्नान्य‚था । ‚{\tiny $_{lb}$}‚‚{\color{DodgerBlue3}‚श‚ब्दार्थे} योग्य‚ताल‚क्ष‚णे क‚ल्पितेऽ‚{\color{DodgerBlue3}‚न‚क्ष‚वृत्तितः} । अक्ष‚वृत्त्य‚भावात् । प्र‚त्य‚क्ष‚बाध‚क‚मिति‚{\tiny $_{3}$}‚ ‚{\tiny $_{lb}$}‚त‚स्या योग्य‚ताया ‚{\color{DodgerBlue3}‚निषेधो}‚नुमानात् ‚{\color{DodgerBlue3}‚स्यात्} । (११४)
	\pend% ending standard par
      \label{div_pvv.4.115}
	  
	% new div opening: depth here is 2
	
	  \bigskip
	  \begingroup
	
	    \large
	  
	    \begin{quote}
	  
	    
	    \stanza[\smallbreak]
	\label{pv.4.115}\flagstanza{\tiny\textenglish{....4.115}}असाधार‚ण‚ता त‚त्र हेतूनां य‚त्र नान्व‚यि ।&स‚त्त्व‚मित्य‚स्योदाहारो हेतोरेवं कुतो म‚तः ॥ ११५ ॥\&[\smallbreak]


	
	    \end{quote}
	  
	  \endgroup
	

	  \pstart \leavevmode% starting standard par
	\hphantom{.}‚{\color{DodgerBlue3}‚त‚त्र} योग्य‚ताप्र‚तिषेधे क‚र्त्त‚व्ये स‚र्व्वेषां ‚{\color{DodgerBlue3}‚हेतूनां} स‚प‚क्षाभावात् ‚{\color{DodgerBlue3}‚असाधार‚ण‚ता\edtext{}{\edlabel{pvv.452-1}\label{pvv.452-1}\lemma{ता}\Bfootnote{स‚र्वः श‚ब्दो योग्य‚तायां स्व‚भाव‚तो घ‚ट‚त इति न क‚श्चिद‚च‚न्द्रोस्ति स‚प‚क्षो य‚त्र वृत्तं लिङ्ग‚म‚न्व‚यि स्यात् ।}}} क‚थ‚मेत‚दित्याह (।) ‚{\color{DodgerBlue3}‚य‚त्र} साध्ये ‚{\color{DodgerBlue3}‚स‚त्त्व}‚म‚पि लिङ्गं स‚र्व्व‚व‚स्तुव्यापि ‚{\color{DodgerBlue3}‚नाऽन्व‚यि} साधा‚{\tiny $_{lb}$}‚र‚णं भ‚व‚ति त‚त्रान्य‚स्य का क‚थेति ‚{\color{DodgerBlue3}‚हेतोः} कृत‚क\edtext{}{\edlabel{pvv.452-2}\label{pvv.452-2}\lemma{क}\Bfootnote{अथ‚वाऽसाधार‚ण‚त्वाद‚नुमानाभाव इत्य‚स्याम‚न्योर्थः । योऽच‚न्द्र‚त्वं प्र‚तिजानीते तं प्र‚ति ब्रुव‚तो लोक‚स्यानुमानेत्युच्य‚ते ताव‚न्ताव‚निष्टौ क‚थ‚म‚न्य‚त्रेष्टिः स्यात् ॥}}त्व‚स्यो‚{\color{DodgerBlue3}‚दाहार} आचार्येस्यैवं फ‚लः ‚{\tiny $_{lb}$}‚‚{\color{DodgerBlue3}‚स‚र्व्व‚हेत्व‚साधार‚ण‚त्व‚प्र‚तिपाद‚न‚प्र‚योद‚नो म‚तः} । (११५)
	\pend% ending standard par
      \label{div_pvv.4.116}
	  
	% new div opening: depth here is 2
	

	  \pstart \leavevmode% starting standard par
	\leavevmode\ledsidenote{\textenglish{453/s}}क‚थं ग‚म्य‚ते ‚{\tiny $_{4}$}‚ स‚र्व्वेषां श‚ब्दानां स‚र्व्व‚त्रार्थे सिद्धिरित्याह ।
	\pend% ending standard par
      
	  \bigskip
	  \begingroup
	
	    \large
	  
	    \begin{quote}
	  
	    
	    \stanza[\smallbreak]
	\label{pv.4.116}\flagstanza{\tiny\textenglish{....4.116}}संकेत‚संश्र‚याः श‚ब्दाः स चेच्छामात्र‚संश्र‚यः ।&नासिद्धिः श‚ब्द‚सिद्धानामिति शाब्द‚प्र‚सिद्धिवाक् ॥ ११६ ॥\&[\smallbreak]


	
	    \end{quote}
	  
	  \endgroup
	

	  \pstart \leavevmode% starting standard par
	\hphantom{.}स‚ङ्केत‚म‚न्त‚रेण वाच‚कादृष्टेः ‚{\color{DodgerBlue3}‚संकेत‚संश्र‚याः श‚ब्दाः स च} संकेतः पुरुषे‚{\color{DodgerBlue3}‚च्छामात्र‚{\tiny $_{lb}$}‚संश्र‚यः} (।) त‚द‚तिरिक्त‚स्यापेक्ष‚णीय‚स्याभावात् । त‚स्मा‚{\color{DodgerBlue3}‚च्छ‚ब्द‚सिद्धानाम}‚भिधे‚{\tiny $_{lb}$}‚य‚त्वादीनां क्व‚चिद‚प्य‚र्थे ‚{\color{DodgerBlue3}‚नासिद्धिः} । इति हेतोः ‚{\color{DodgerBlue3}‚शाब्द‚प्र‚सिद्धिराचार्य‚स्य} ॥ (११६)
	\pend% ending standard par
      \label{div_pvv.4.117}
	  
	% new div opening: depth here is 2
	

	  \pstart \leavevmode% starting standard par
	एत‚च्च शाब्द‚प्र‚सिद्धिव‚च‚न‚म् (।)
	\pend% ending standard par
      
	  \bigskip
	  \begingroup
	
	    \large
	  
	    \begin{quote}
	  
	    
	    \stanza[\smallbreak]
	\label{pv.4.117}\flagstanza{\tiny\textenglish{....4.117}}अनुमान‚प्र‚सिद्धेषु विरुद्धाव्य‚भिचारिणः ।&अभावः द‚र्श‚य‚त्येवं प्र‚तीतेर‚नुमात्व‚तः ॥ ११७ ॥\&[\smallbreak]


	
	    \end{quote}
	  
	  \endgroup
	

	  \pstart \leavevmode% starting standard par
	\hphantom{.}व‚स्तुब‚ल‚प्र‚वृत्तेना‚{\color{DodgerBlue3}‚नुमानेन प्र‚सिद्धे}‚ष्व‚र्थेषु विप‚रीत‚ध‚र्मो‚{\tiny $_{5}$}‚प‚स्थाप‚क‚स्य ‚{\color{DodgerBlue3}‚विरुद्धाव्य‚{\tiny $_{lb}$}‚भिचारिणः} साध‚नान्त‚र‚स्या‚{\color{DodgerBlue3}‚भाव‚न्द‚र्श‚य‚ति} । क‚स्मादित्याह । ‚{\color{DodgerBlue3}‚एव‚मी\edtext{}{\edlabel{pvv.453-1}\label{pvv.453-1}\lemma{मी}\Bfootnote{ल‚क्ष‚ण‚युक्ते बाधासंभ‚वे ल‚क्ष‚ण‚मेन दूषितं स्यात् ।}}}‚दृश्याः श‚ब्द‚{\tiny $_{lb}$}‚सिद्धाया योग्य‚तायाः ‚{\color{DodgerBlue3}‚प्र‚तीते \edtext{}{\edlabel{pvv.453-2}\label{pvv.453-2}\lemma{तीते}\Bfootnote{अनुमीय‚तेऽन‚येति प्र‚सिद्धेर‚नुमात्वात् ।}}} स्व‚भाव‚लिङ्ग‚स‚मुत्थ‚त्वात् ‚{\color{DodgerBlue3}‚अनुमात्व‚तः} । य‚था ‚{\tiny $_{lb}$}‚श‚ब्द‚सिद्धा योग्य‚ताऽनुमान‚सिद्धेति बाध्या स‚त्वादिहेतुना (।) त‚थाऽन्योपि ‚{\tiny $_{lb}$}‚व‚स्तुब‚ल‚प्र‚वृत्तानुमान‚विष‚यः स‚मान‚त्वात् न्याय‚स्येत्य‚र्थः । (११७)
	\pend% ending standard par
      \label{div_pvv.4.118}
	  
	% new div opening: depth here is 2
	
	  \bigskip
	  \begingroup
	
	    \large
	  
	    \begin{quote}
	  
	    
	    \stanza[\smallbreak]
	\label{pv.4.118}\flagstanza{\tiny\textenglish{....4.118}}अथ‚वा ब्रुव‚तो लोक‚स्यानुमाऽभाव उच्य‚ते ।&किन्तेन भिन्न‚विष‚या प्र‚तीतिर‚नुमान‚तः ॥ ११८ ॥\&[\smallbreak]


	
	    \end{quote}
	  
	  \endgroup
	

	  \pstart \leavevmode% starting standard par
	अथ‚वाऽच‚न्द्रःश‚शी स‚त्वादिति विप्र‚ति\edtext{}{\edlabel{pvv.453-3}\label{pvv.453-3}\lemma{ति}\Bfootnote{लोक‚श्च‚न्द्रः श‚शी ह्लाद‚नाद्भास‚नाद्वा क‚र्पूरादिव‚त् न त‚त्त‚स्यासिद्धं ।}}प‚द्य‚मानं प्र‚तिप‚त् प्र‚ति‚{\tiny $_{6}$}‚पाद‚नार्थं ‚{\color{DodgerBlue3}‚लोक‚स्य ‚{\tiny $_{lb}$}‚ब्रुव‚तः} शाब्द‚प्र‚सिद्धेनासाधार‚ण‚त्वाद‚{\color{DodgerBlue3}‚नुमानाभाव} आचार्येणो‚{\color{DodgerBlue3}‚च्य‚ते} । पार‚मार्थिक‚स्य ‚{\tiny $_{lb}$}‚बाध्य‚त्व‚स्याभावात् \edtext{}{\edlabel{pvv.453-4}\label{pvv.453-4}\lemma{स्याभावात्}\Bfootnote{किं फ‚ल‚मित्याह ? व‚स्तुतः श‚शिनि च‚न्द्र‚त्व‚म‚पि श‚ब्द‚ब‚लात् ।}} । क‚ल्पितं निषेध्यं त‚च्च पुरुषेच्छामात्राधीन‚त्वात् स‚र्व्व‚त्र ‚{\tiny $_{lb}$}‚स‚म्भ‚व‚तीति स‚र्व्व‚स्य च‚न्द्र‚श‚ब्द‚योग्य‚तायोगान्न क‚श्चिद‚च‚न्द्रः प‚क्षोस्ति य‚त्र व‚र्त्त‚मानं‚{\tiny $_{lb}$}‚स‚त्त्व‚म‚साधार‚ण‚तां ज‚ह्या/?/ । एव‚न्द‚र्शिते ‚{\color{DodgerBlue3}‚किम्भ}‚व‚तीति चेत् । ‚{\color{DodgerBlue3}‚तेना}‚नुमानाभावाभिधा-‚{\tiny $_{7}$}‚ ‚{\tiny $_{lb}$}‚यिना श‚ब्द‚प्र‚सिद्धाभिधानेन श‚ब्द‚सिद्धा ‚{\color{DodgerBlue3}‚प्र‚तीति}‚र्व्व‚स्तुब‚ल‚प्र‚वृत्ता‚{\color{DodgerBlue3}‚नुमान}‚तो ‚{\color{DodgerBlue3}‚भिन्न‚विष}‚-\leavevmode\ledsidenote{\textenglish{91b/MA}} ‚{\color{DodgerBlue3}‚यो}‚क्ता भ‚व‚ति । व‚स्तुविष‚यं ह्य‚नुमानं क‚ल्पित‚गोच‚रान्त‚रा शाब्दी प्र‚तीति‚{\tiny $_{lb}$}‚रित्य‚र्थः ॥ (११८)
	\pend% ending standard par
      \label{div_pvv.4.119}
	  
	% new div opening: depth here is 2
	\textsuperscript{\textenglish{454/s}}

	  \begin{center}%% label @type='head'
	\textbf{(ग. व‚स्तुब‚ल‚प्र‚वृत्त‚म‚नुमान‚म्)}
	\end{center}
	
	  \bigskip
	  \begingroup
	
	    \large
	  
	    \begin{quote}
	  
	    
	    \stanza[\smallbreak]
	\label{pv.4.119}\flagstanza{\tiny\textenglish{....4.119}}तेनानुमानाद् व‚स्तूनां स‚द‚स‚त्तानुरोधिनः ।&भिन्न‚स्यात‚द्व‚शा वृत्तिस्त‚दिच्छाजेति सूचित‚म् ॥ ११९ ॥\&[\smallbreak]


	
	    \end{quote}
	  
	  \endgroup
	

	  \pstart \leavevmode% starting standard par
	\hphantom{.}‚{\color{DodgerBlue3}‚तेन\edtext{}{\edlabel{pvv.454-1}\label{pvv.454-1}\lemma{तेन}\Bfootnote{भेद‚द‚र्श‚नेपि किं फ‚ल‚मित्याह ।}}} विष‚य‚भेदेन ‚{\color{DodgerBlue3}‚व‚स्तूनां स‚द‚स‚त्तानुरोधिनोऽनुमानात् भिन्न‚स्य} क‚ल्पित‚{\tiny $_{lb}$}‚स्यार्थ‚स्य श‚ब्द‚योग्य‚त्व‚स्य ‚{\color{DodgerBlue3}‚वृत्तिर‚त‚द्व‚शा} व‚स्त्व‚नाय‚त्ता ‚{\color{DodgerBlue3}‚त‚स्य} पुरुष‚{\color{DodgerBlue3}‚स्येच्छाजेति सूचितं} भ‚व‚ति । (११९)
	\pend% ending standard par
      \label{div_pvv.4.120}
	  
	% new div opening: depth here is 2
	

	  \pstart \leavevmode% starting standard par
	किञ्च (।)
	\pend% ending standard par
      
	  \bigskip
	  \begingroup
	
	    \large
	  
	    \begin{quote}
	  
	    
	    \stanza[\smallbreak]
	\label{pv.4.120}\flagstanza{\tiny\textenglish{....4.120}}च‚न्द्र‚तां श‚शिनोऽनिच्छान् कां प्र‚तीति स वाञ्छ‚ति ।&इति तं प्र‚त्य‚दृष्टान्तं त‚द‚साधार‚णं म‚त‚म् ॥ १२० ॥\&[\smallbreak]


	
	    \end{quote}
	  
	  \endgroup
	

	  \pstart \leavevmode% starting standard par
	\hphantom{.}‚{\color{DodgerBlue3}‚श‚शिनः} स‚र्व्व‚ज‚न‚सिद्धां व्याव‚हारि‚{\tiny $_{1}$}‚की ‚{\color{DodgerBlue3}‚च‚न्द्र‚ता} च‚न्द्र‚श‚ब्द‚वाच्य‚तां ‚{\color{DodgerBlue3}‚अनिच्छ‚न्} प‚रः‚{\color{DodgerBlue3}‚काम‚न्यां प्र‚तीतिं स वाञ्छ‚ति} (।) य‚या वाच्य‚तासिद्धिः क्व‚चित् स्यात् । म‚ता ‚{\tiny $_{lb}$}‚च पार‚मार्थिकी वाच्य‚ताप्र‚तीतिर्न क्व‚चिद‚स्ति । व‚स्तुतः स‚र्व्व‚स्यावाच्य‚ता‚{\tiny $_{lb}$}‚प्र‚द‚र्श‚नात् । क‚ल्पिवाच्य‚ताप्र‚तीतिस्तु पुरुषेच्छामात्र‚प्र‚भ‚व‚त्वात् स‚र्व्व‚त्राव्याह‚तैव । ‚{\tiny $_{lb}$}‚अतः स‚र्व्व‚स्य च‚न्द्र‚श‚ब्द‚वाच्य‚तायोगात् स‚प‚क्षो नास्तीति तं च‚न्द्र‚ताप‚लापिनं वादिनं ‚{\tiny $_{lb}$}‚‚{\color{DodgerBlue3}‚प्र‚ति} स‚त्वं ‚{\color{DodgerBlue3}‚लिङ्ग‚{\tiny $_{2}$}‚म‚दृष्टान्त‚म‚साधार‚ण}‚मुक्त‚माचार्येण न तु च‚न्द्र‚स्यैक‚स्यान्य‚त्रास‚म्भ‚वात् ‚{\tiny $_{lb}$}‚स‚प‚क्ष‚विप‚क्ष‚योर‚भावाद‚साधार‚ण‚त्व‚म‚भिप्रेत‚मा चा र्य स्य । अच‚न्द्र‚त्वे साध्ये घ‚टादेः ‚{\tiny $_{lb}$}‚स‚प‚क्ष‚स्य स‚त्वात् । च‚न्द्र‚स्तु विप‚क्षो मा भूत् (।) त‚थापि हेतुनिवृत्तिर‚स्माद‚व्याह‚तैव । ‚{\tiny $_{lb}$}‚अस‚तोपि हेतुनिवृत्तेः साध‚नात् । (१२०)
	\pend% ending standard par
      \label{div_pvv.4.121}
	  
	% new div opening: depth here is 2
	

	  \pstart \leavevmode% starting standard par
	अपिच (।)
	\pend% ending standard par
      
	  \bigskip
	  \begingroup
	
	    \large
	  
	    \begin{quote}
	  
	    
	    \stanza[\smallbreak]
	\label{pv.4.121}\flagstanza{\tiny\textenglish{....4.121}}नोदाह‚र‚ण‚मेवेद‚म‚धिकृत्येद‚मुच्य‚ते ।&ल‚क्ष‚ण‚त्वात् त‚था वृक्षोऽधात्रीत्युक्तौ च बाध‚नात् ॥ १२१ ॥\&[\smallbreak]


	
	    \end{quote}
	  
	  \endgroup
	

	  \pstart \leavevmode% starting standard par
	\hphantom{.}अच‚न्द्रः श‚शी स‚त्त्वादित्येत‚देवोदा‚{\color{DodgerBlue3}‚ह‚र‚ण‚म‚धिकृत्येदं} श‚ब्द‚वा‚{\tiny $_{8}$}‚च्य‚त्व‚प्र‚तिक्षेप‚{\tiny $_{lb}$}‚हेतोर‚साधार‚ण‚त्वं ‚{\color{DodgerBlue3}‚नोच्य‚ते । ल‚क्ष‚ण‚त्वात् \edtext{}{\edlabel{pvv.454-2}\label{pvv.454-2}\lemma{त्वात्}\Bfootnote{स‚र्व्व‚प्र‚तीतिविरोधानां सामान्येन ल‚क्ष‚ण‚त्वात् ।}}} । ल‚क्ष‚णेन हि ल‚क्ष्यं व्याप्तं द‚र्श‚नीयं ‚{\tiny $_{lb}$}‚न च द्वितीय‚च‚न्द्राभावेनोसाध‚र‚ण‚तोक्तिरुदाह‚र‚णान्त‚रं व्याप्नोति । य‚था वा ‚{\tiny $_{lb}$}‚च‚न्द्र‚तायाः प्र‚तीत्या बाधेष्य‚ते ‚{\color{DodgerBlue3}‚त‚थाऽधात्री वृक्षः} स‚त्त्वात् \edtext{}{\edlabel{pvv.454-3}\label{pvv.454-3}\lemma{त्त्वात्}\Bfootnote{पार्थिव‚त्वात् ।}} घ‚टादिव‚दित्युदाह‚र‚णो‚{\color{DodgerBlue3}‚क्तौ} \leavevmode\ledsidenote{\textenglish{455/s}} स‚र्व्व‚लोक‚सिद्ध‚या पुरुषेच्छाधीन‚या घ‚टादाव‚पि वृक्ष‚श‚ब्द‚योग्य‚ताप्र‚तीत्या ‚{\color{DodgerBlue3}‚स‚प‚क्षाभावे-} नासाधार‚{\tiny $_{4}$}‚ण‚त्वाद् वृक्ष‚श‚ब्द‚वाच्य‚त्वाभाव‚स्य साध‚नात् (।) ‚{\color{DodgerBlue3}‚य‚थोक्त‚मेवा}‚{\tiny $_{lb}$}‚साधार‚ण‚त्व‚माचार्य‚स्येष्टं । (१२१)
	\pend% ending standard par
      \label{div_pvv.4.122}
	  
	% new div opening: depth here is 2
	

	  \pstart \leavevmode% starting standard par
	य‚द‚प्यु\edtext{}{\edlabel{pvv.455-1}\label{pvv.455-1}\lemma{प्यु}\Bfootnote{न्या य मु ख टीकाकार‚मुप‚क्षिप‚ति ।}}च्य‚ते द्वितीय‚स्य च‚न्द्र‚स्याभावाद‚साधार‚ण‚तेति त‚त्राह (।)
	\pend% ending standard par
      
	  \bigskip
	  \begingroup
	
	    \large
	  
	    \begin{quote}
	  
	    
	    \stanza[\smallbreak]
	\label{pv.4.122}\flagstanza{\tiny\textenglish{....4.122}}अत्रापि लोके दृष्ट‚त्वात् क‚र्पूर‚र‚ज‚तादिषु ।&स‚म‚याद्व‚र्त‚मान‚स्य काऽसाधार‚ण‚तापि वा ॥ १२२ ॥\&[\smallbreak]


	
	    \end{quote}
	  
	  \endgroup
	

	  \pstart \leavevmode% starting standard par
	\hphantom{.}‚{\color{DodgerBlue3}‚लोके क‚र्प्पूर‚र‚ज‚तादिषु} गान्धिक‚वाचिकादीनां ‚{\color{DodgerBlue3}‚स‚म‚याद्व‚र्त्त‚मान‚स्य दृष्ट‚त्वा-} द‚त्राच‚न्द्रः श‚शी स‚त्त्वादित्युदाह‚र‚णे हेतोर‚{\color{DodgerBlue3}‚साधार‚ण‚तापि का वा} । य‚दि द्वितीय‚{\tiny $_{lb}$}‚च‚न्द्रो न भ‚वेत् । एव‚म‚साधार‚ण‚ता ‚{\tiny $_{5}$}‚ स्याद् व‚स्तुत्व‚स्य । (१२२)
	\pend% ending standard par
      \label{div_pvv.4.123}
	  
	% new div opening: depth here is 2
	

	  \pstart \leavevmode% starting standard par
	स्यादेत‚त् (।) त‚त्स‚म‚याद‚पि व‚र्त्त‚मान‚स्य च‚न्द्र‚त्वादेर्न श‚ब्द‚वाच्य‚ता । त‚तो ‚{\tiny $_{lb}$}‚य‚था न व‚ह्निश‚ब्द‚वाच्य‚ता क‚र्प्पूर‚स्य त‚था च‚न्द्र‚श‚ब्द‚वाच्य‚ता च न स्यादित्य‚{\tiny $_{lb}$}‚साधार‚ण‚तैवेत्याह ।
	\pend% ending standard par
      
	  \bigskip
	  \begingroup
	
	    \large
	  
	    \begin{quote}
	  
	    
	    \stanza[\smallbreak]
	\label{pv.4.123a}\flagstanza{\tiny\textenglish{...4.123a}}य‚दि त‚स्य क्व‚चित् सिध्येत् सिद्धं व‚स्तुब‚लेन त‚त् ।\&[\smallbreak]


	
	    \end{quote}
	  
	  \endgroup
	

	  \pstart \leavevmode% starting standard par
	\hphantom{.}‚{\color{DodgerBlue3}‚त‚स्यैवं}‚वादिनः स‚त्य‚पि साम‚यिके च‚न्द्रे ‚{\color{DodgerBlue3}‚क्व‚चिद‚र्थ}‚विशेषे च‚न्द्र‚श‚ब्द‚वाच्य‚त्वं ‚{\tiny $_{lb}$}‚य‚दि सिध्येत् न तु स‚र्व्व‚त्र क‚र्पूर‚र‚ज‚तादौ त‚दा ‚{\color{DodgerBlue3}‚त‚च्च}‚न्द्र‚श‚ब्द‚वाच्य‚त्व ‚{\color{DodgerBlue3}‚व‚स्तुब‚लेन सिद्धं} स्यात् । न त्वेवं दृश्य‚{\tiny $_{6}$}‚ते स‚र्व्व‚स्य क‚र्पूरादेश्च श‚ब्दाभिधेय‚त्व‚द‚र्श‚नात् ।
	\pend% ending standard par
      

	  \pstart \leavevmode% starting standard par
	अथ य‚त्रैव क‚ल्प्य‚ते च‚न्द्र‚त्वं त‚देव त‚च्छ‚ब्द‚वाच्यं प्र‚तीय‚त इति त‚थाभ्युप‚ग‚म्य‚ते ‚{\tiny $_{lb}$}‚त‚दा ।
	\pend% ending standard par
      
	  \bigskip
	  \begingroup
	
	    \large
	  
	    \begin{quote}
	  
	    
	    \stanza[\smallbreak]
	\label{pv.4.123b}\flagstanza{\tiny\textenglish{...4.123b}}प्र‚तीतिसिद्ध‚प‚ग‚मेऽश‚शिन्य‚प्य‚निवार‚ण‚म् ॥ १२३ ॥\&[\smallbreak]


	
	    \end{quote}
	  
	  \endgroup
	

	  \pstart \leavevmode% starting standard par
	साम‚यिके च‚न्द्र‚त्वे क्वापि प्र‚तीत्या सिद्ध‚स्य च‚न्द्र‚श‚ब्द‚वाच्य‚त्व‚स्योप‚ग‚मे । ‚{\tiny $_{lb}$}‚श‚{\color{DodgerBlue3}‚शिन्य}‚पि त‚च्छ‚ब्द‚वाच्य‚स्या‚{\color{DodgerBlue3}‚निवार‚णं} । (१२३)
	\pend% ending standard par
      \label{div_pvv.4.124}
	  
	% new div opening: depth here is 2
	
	  \bigskip
	  \begingroup
	
	    \large
	  
	    \begin{quote}
	  
	    
	    \stanza[\smallbreak]
	\label{pv.4.124}\flagstanza{\tiny\textenglish{....4.124}}त‚स्य व‚स्तुनि सिद्ध‚स्य श‚शिन्य‚प्य‚निवार‚ण‚म् ।&त‚द्व‚स्त्त्व‚भावे श‚शिनि वार‚णेपि न दुष्य‚ति ॥ १२४ ॥\&[\smallbreak]


	
	    \end{quote}
	  
	  \endgroup
	

	  \pstart \leavevmode% starting standard par
	\hphantom{.}शुक्ल‚तादिके निमित्त‚भूते व‚स्तुनि । ‚{\color{DodgerBlue3}‚त‚स्य} च‚न्द्र‚श‚ब्दाभिधेय‚त्व‚स्य ‚{\color{DodgerBlue3}‚सिद्ध‚स्य} ‚{\color{DodgerBlue3}‚श‚शिन्य‚पि} निमित्त‚स‚द्भावाद‚{\color{DodgerBlue3}‚निवार‚णं‚{\tiny $_{7}$}‚} त‚स्य निमित्त‚भूत‚स्य व‚स्तुनः ‚{\color{DodgerBlue3}‚श‚शिन्य‚भावे} तु\leavevmode\ledsidenote{\textenglish{92a/MA}} च‚न्द्र‚श‚ब्द‚वाच्य‚त्व‚स्य व‚स्तुत्वाद्धेतोर्व्वार‚णेपि न किञ्चिद् ‚{\color{DodgerBlue3}‚दुष्य‚ति} । निमित्ताभावे ‚{\tiny $_{lb}$}‚\leavevmode\ledsidenote{\textenglish{456/s}} नैमित्तिकाभावास्येष्ट‚त्वात् । साम‚यिक‚न्तु स‚र्व्व‚त्राश‚क्य‚वार‚ण‚मिति ‚{\color{DodgerBlue3}‚त‚च्छ‚ब्द‚योग्य‚{\tiny $_{lb}$}‚ताप्य‚बाध्या} । (१२४)
	\pend% ending standard par
      \label{div_pvv.4.125}
	  
	% new div opening: depth here is 2
	
	  \bigskip
	  \begingroup
	
	    \large
	  
	    \begin{quote}
	  
	    
	    \stanza[\smallbreak]
	\label{pv.4.125}\flagstanza{\tiny\textenglish{....4.125}}त‚स्माद‚व‚स्तुनिय‚त‚संकेत‚ब‚ल‚भाविनाम् ।&योग्याः प‚दार्था ध‚र्माणामिच्छाया अनिरोधानात् ॥ १२५ ॥\&[\smallbreak]


	
	    \end{quote}
	  
	  \endgroup
	

	  \pstart \leavevmode% starting standard par
	\hphantom{.}‚{\color{DodgerBlue3}‚त‚स्माद‚व‚स्तुनिय‚तो} व‚स्तुन्य‚निय‚त इच्छाधीन‚त्वात् ‚{\color{DodgerBlue3}‚संकेत}‚स्त‚स्य ब‚लाद् ‚{\color{DodgerBlue3}‚भाविनां ‚{\tiny $_{lb}$}‚ध‚र्म्माणां} वाच्य‚त्वादीनां ‚{\color{DodgerBlue3}‚प‚दार्थाः} स‚र्व्वे ध‚र्मित्वेन ‚{\color{DodgerBlue3}‚योग्याः} । इच्छातः ‚{\color{DodgerBlue3}‚पुंसः केन‚चिद् ‚{\tiny $_{lb}$}‚वाच्य‚त्वाद्युत्था}‚{\tiny $_{1}$}‚पिकाया ‚{\color{DodgerBlue3}‚इच्छाया अनिरोधात्} । (१२५)
	\pend% ending standard par
      \label{div_pvv.4.126}
	  
	% new div opening: depth here is 2
	
	  \bigskip
	  \begingroup
	
	    \large
	  
	    \begin{quote}
	  
	    
	    \stanza[\smallbreak]
	\label{pv.4.126}\flagstanza{\tiny\textenglish{....4.126}}तां योग्य‚तां विरुन्धानं संकेताप्र‚तिषेध‚जा ।&प्र‚तिह‚न्ति प्र‚तीत्याख्या योग्य‚ताविष‚येऽनुमा ॥ १२६ ॥\&[\smallbreak]


	
	    \end{quote}
	  
	  \endgroup
	

	  \pstart \leavevmode% starting standard par
	\hphantom{.}‚{\color{DodgerBlue3}‚तामि}‚ष्ट‚श‚ब्दाभिधेय‚त्व‚{\color{DodgerBlue3}‚योग्य‚तां} प‚दार्थानां स‚त्त्वादिकाद्धेतोर्व्वि‚{\color{DodgerBlue3}‚रुन्धानं प्र‚ति‚{\tiny $_{lb}$}‚क्षिप‚न्तं} वादिनं ‚{\color{DodgerBlue3}‚प्र‚तीत्या}‚ख्या प्र‚तीतिसंज्ञिताऽ‚{\color{DodgerBlue3}‚नुमा प्र‚तिह‚न्ति । संकेताप्र‚तिषे‚{\tiny $_{lb}$}‚ध‚जेति} स्व‚भाव‚लिङ्ग‚ज‚त्व‚माह । इच्छाधीन‚त्वात् योग्य‚ताविष‚ये विप‚रीत‚ध‚र्मो‚{\tiny $_{lb}$}‚प‚स्थान‚माह । प्र‚योगः पुनः (।) यः पुरुषेच्छानुभिधायी स स‚र्व्व‚त्र स‚म्भ‚वी त‚द्य‚था ‚{\tiny $_{lb}$}‚विक‚ल्पः पुरुषेच्छानुविधा‚{\tiny $_{2}$}‚यि चार्थेष्विष्ट‚श‚ब्दाभिधेय‚त्व‚मिति । (१२६)
	\pend% ending standard par
      \label{div_pvv.4.127}
	  
	% new div opening: depth here is 2
	
	  \bigskip
	  \begingroup
	
	    \large
	  
	    \begin{quote}
	  
	    
	    \stanza[\smallbreak]
	\label{pv.4.127}\flagstanza{\tiny\textenglish{....4.127}}श‚ब्दानाम‚र्थ‚निय‚मः संकेतानुविधायिनाम् ।&नेत्य‚नेनोक्त‚म‚त्रैषां प्र‚तिषेधो विरुध्य‚ते ॥ १२७ ॥\&[\smallbreak]


	
	    \end{quote}
	  
	  \endgroup
	

	  \pstart \leavevmode% starting standard par
	\hphantom{.}‚{\color{DodgerBlue3}‚अनेन} चेष्ट‚श‚ब्दाभिधेय‚त्व‚योग्य‚ताप्र‚तिषेध‚बाध‚नेन द‚र्शितेन ‚{\color{DodgerBlue3}‚श‚ब्दानां संके‚{\tiny $_{lb}$}‚तानुविधायि}‚नाम‚{\color{DodgerBlue3}‚र्थ‚निय‚मः} प्र‚तिनिय‚त‚वाच‚क‚त्वं ‚{\color{DodgerBlue3}‚नेत्युक्त}‚म्भ‚व‚ति । त‚तोस्य स्वेच्छा‚{\tiny $_{lb}$}‚क‚ल्पितोऽर्थे ‚{\color{DodgerBlue3}‚एषां} श‚ब्दानां वाच‚क‚त्व‚स्य ‚{\color{DodgerBlue3}‚प्र‚तिषेधः} स‚त्त्वादिहेतोः ‚{\color{DodgerBlue3}‚क्रिय‚माणो ‚{\tiny $_{lb}$}‚विरुध्य\edtext{}{\edlabel{pvv.456-1}\label{pvv.456-1}\lemma{विरुध्य}\Bfootnote{प्र‚तिज्ञादोषो भ‚व‚ति ।}}ते} । (१२७)
	\pend% ending standard par
      \label{div_pvv.4.128}
	  
	% new div opening: depth here is 2
	

	  \pstart \leavevmode% starting standard par
	य‚द्येव‚न्त‚दा क्व‚चिद‚र्थे निषेधं कुर्व्व‚ता श‚ब्दो बाध्यः स्यात् । त‚त‚श्च गुणा‚{\tiny $_{lb}$}‚‚{\tiny $_{3}$}‚ दिकं निमित्त‚भूतं गुणादि\edtext{}{\edlabel{pvv.456-2}\label{pvv.456-2}\lemma{गुणादि}\Bfootnote{जात्यादि ।}}श‚ब्दानां गुण‚गुणिस‚म्ब‚न्धादिपार‚मार्थिक‚म‚र्थं गुणि‚{\tiny $_{lb}$}‚श‚ब्दादीनां निषेध‚न् बाध्यः स्यादित्याह । येन क‚ल्पित‚म‚र्थं श‚ब्दानां बाध‚मानः ‚{\tiny $_{lb}$}‚प्र‚तिक्षिप्य‚ते । न तु गुणादिकं तेन (।)
	\pend% ending standard par
      
	  \bigskip
	  \begingroup
	
	    \large
	  
	    \begin{quote}
	  
	    
	    \stanza[\smallbreak]
	\label{pv.4.128}\flagstanza{\tiny\textenglish{....4.128}}नैमित्तिक्याः श्रुतेर‚र्थ‚म‚र्थ‚म्वा पार‚मार्थिक‚म् ।&श‚ब्दानां प्र‚तिरुन्धानोऽबाध‚नार्हो हि व‚र्ण्णितः ॥ १२८ ॥\&[\smallbreak]


	
	    \end{quote}
	  
	  \endgroup
	\textsuperscript{\textenglish{457/s}}

	  \pstart \leavevmode% starting standard par
	\hphantom{.}‚{\color{DodgerBlue3}‚नैमित्तिक्या}\edtext{\textsuperscript{*}}{\edlabel{pvv.457-1}\label{pvv.457-1}\lemma{*}\Bfootnote{निमित्ताभावे ।}} व‚स्तुभूत‚गुणादिनिमित्त‚व‚त्याः ‚{\color{DodgerBlue3}‚श्रुतेर‚र्थं गुणादिकं पार‚मार्थि‚{\tiny $_{lb}$}‚क‚म‚र्थं} \edtext{\textsuperscript{*}}{\edlabel{pvv.457-2}\label{pvv.457-2}\lemma{*}\Bfootnote{न च‚न्द्र‚त्वं प‚र‚मार्थ‚तोस्तीति ।}} गुणिगुणादिस‚म्ब‚न्धं ‚{\color{DodgerBlue3}‚श‚ब्दानां} गुण्यादिवाचिनां ‚{\color{DodgerBlue3}‚प्र‚तिरुन्धानो}‚ऽबाध‚नार्हो ‚{\tiny $_{lb}$}‚बाधां नार्ह‚ती‚{\tiny $_{4}$}‚त्युक्तो \edtext{}{\edlabel{pvv.457-3}\label{pvv.457-3}\lemma{त्युक्तो}\Bfootnote{अबाध‚नार्हः ।}}भ‚व‚ति ॥ (१२८)
	\pend% ending standard par
      \label{div_pvv.4.129}
	  
	% new div opening: depth here is 2
	

	  \pstart \leavevmode% starting standard par
	य‚स्माच्च सांकेतिकार्थ‚निराक‚र‚णे प्र‚तीतिबाधा (।)
	\pend% ending standard par
      
	  \bigskip
	  \begingroup
	
	    \large
	  
	    \begin{quote}
	  
	    
	    \stanza[\smallbreak]
	\label{pv.4.129}\flagstanza{\tiny\textenglish{....4.129}}त‚स्माद् विष‚य‚भेद‚स्य द‚र्श‚नाय पृथ‚क्कृता ।&अनुमानाब‚हिर्भूता प्र‚तीतिर‚पि पूर्व्व‚व‚त् ॥ १२९ ॥\&[\smallbreak]


	
	    \end{quote}
	  
	  \endgroup
	

	  \pstart \leavevmode% starting standard par
	\hphantom{.}‚{\color{DodgerBlue3}‚त‚स्मात्} स्व‚भाव‚लिङ्ग‚ज‚त्वेना‚{\color{DodgerBlue3}‚नुमानाद‚ब‚हिर्भूता प्र‚तीतिर‚पि} त‚स्मात्\edtext{}{\edlabel{pvv.457-4}\label{pvv.457-4}\lemma{स्मात्}\Bfootnote{अनुमानात् ।}} ‚{\color{DodgerBlue3}‚पृथ‚क्‚{\tiny $_{lb}$}‚कृता} (।) किम‚र्थ‚मित्याह । ‚{\color{DodgerBlue3}‚विष‚य‚स्य भेदः} क‚ल्पिताक‚ल्पित‚त्वं त‚स्य ‚{\color{DodgerBlue3}‚द‚र्श‚नाय} । ‚{\tiny $_{lb}$}‚क‚ल्पितार्थ‚विष‚या प्र‚तीतिः । व‚स्तुविष‚य‚न्त्व‚नुमान‚मित्य‚र्थः । ‚{\color{DodgerBlue3}‚पूर्व्व‚व‚दिति} । ‚{\tiny $_{lb}$}‚य‚था आग‚म‚स्व‚व‚च‚नेऽभ्युप‚ग‚त‚प्रामाण्येऽप्र‚त्य‚क्ष‚त्वाद‚नुमानान्त‚र्ग‚ते‚{\tiny $_{5}$}‚पि विष‚य‚भेद‚द‚र्श‚नाय ‚{\tiny $_{lb}$}‚पृथ‚ग्द‚र्शिते व‚स्तुब‚ल‚प्र‚वृत्तेनुमानं स‚र्व्व‚विष‚य‚विचारे त्वाग‚म‚स्व‚व‚च‚ने अधिकृते । ‚{\tiny $_{lb}$}‚त‚था प्र‚तीत्य‚नुमाने अपि भिन्न‚विष‚ये इत्य‚र्थः । (१२९)
	\pend% ending standard par
      \label{div_pvv.4.130}
	  
	% new div opening: depth here is 2
	

	  \pstart \leavevmode% starting standard par
	अनुमान‚बाधायाम‚न्त‚र्भावाद‚न‚योर‚भ्यु\edtext{}{\edlabel{pvv.457-5}\label{pvv.457-5}\lemma{भ्यु}\Bfootnote{स्व‚व‚च‚नाप्त‚व‚च‚न‚योर्ग्र‚होनेन ।}}प‚ग‚म‚प्र‚तीतिबाध‚योः (।)
	\pend% ending standard par
      
	  \bigskip
	  \begingroup
	
	    \large
	  
	    \begin{quote}
	  
	    
	    \stanza[\smallbreak]
	\label{pv.4.130}\flagstanza{\tiny\textenglish{....4.130}}सिद्ध‚योः पृथ‚गाख्याने द‚र्श‚यंश्च प्र‚योज‚न‚म् ।&एते स‚हेतुके प्राह नानुमाध्य‚क्ष‚बाध‚ने ॥ १३० ॥\&[\smallbreak]


	
	    \end{quote}
	  
	  \endgroup
	

	  \pstart \leavevmode% starting standard par
	\hphantom{.}‚{\color{DodgerBlue3}‚सिद्ध‚यो\edtext{}{\edlabel{pvv.457-6}\label{pvv.457-6}\lemma{यो}\Bfootnote{अनुमानापृथ(?)क्त्वेन निश्चित‚योः ।}}}‚र‚पि ‚{\color{DodgerBlue3}‚पृथ‚गाख्याने} विष‚य‚भेद‚ल‚क्ष‚णं ‚{\color{DodgerBlue3}‚प्र‚योज‚न‚न्द‚र्श‚य‚न्ना} चा र्य ‚{\color{DodgerBlue3}‚एते} अभ्यु‚{\tiny $_{lb}$}‚प‚ग‚म‚प्र‚तीतिबाधे ‚{\color{DodgerBlue3}‚स‚हेतुके प्राह\edtext{}{\edlabel{pvv.457-7}\label{pvv.457-7}\lemma{प्राह}\Bfootnote{नागः ।}}} । न स‚न्ति \edtext{}{\edlabel{pvv.457-8}\label{pvv.457-8}\lemma{न्ति}\Bfootnote{स‚मुच्च‚य‚माह प्राक् प्रामाण्य‚माज्ञाय विरोध‚म‚नेन ।}} प्र‚माणानि प्र‚मेयार्थानीति\edtext{}{\edlabel{pvv.457-9}\label{pvv.457-9}\lemma{मेयार्थानीति}\Bfootnote{हेतुनानेन स‚हेतुक‚माह ।}} ‚{\tiny $_{lb}$}‚प्र‚तिज्ञामात्रे\edtext{}{\edlabel{pvv.457-10}\label{pvv.457-10}\lemma{तिज्ञामात्रे}\Bfootnote{शास्त्र‚स्व‚व‚च‚न‚प्रामाण्याख्येन ।}}णेति (।) अत्र प्र‚तिज्ञा‚{\tiny $_{6}$}‚मात्रं शास्त्र‚स्व‚व‚च‚न‚योः सिद्ध‚योर‚प्रामाण्य‚प्र‚ति‚{\tiny $_{lb}$}‚ज्ञाबाध‚क‚मुक्त‚म् (।) अतोप्य\edtext{}{\edlabel{pvv.457-11}\label{pvv.457-11}\lemma{अतोप्य}\Bfootnote{दृष्टान्ताभावात् ।}}साधार‚ण‚त्वाद‚नुमानाभावे शाब्द‚प्र‚सिद्धेनापोद्य‚ते न ‚{\tiny $_{lb}$}‚स‚प‚क्ष इति । अत्र शाब्द‚प्र‚सिद्धेन श‚शिन‚श्च‚न्द्र‚त्वेनाच‚न्द्र‚त्व‚प्र‚तिज्ञाया बाध‚न‚मुक्तं । ‚{\tiny $_{lb}$}‚‚{\color{DodgerBlue3}‚अनुमाध्य‚क्ष‚बाध‚ने} तु न स‚हेतुके प्राह ।\edtext{\textsuperscript{*}}{\edlabel{pvv.457-12}\label{pvv.457-12}\lemma{*}\Bfootnote{उद्योत‚क‚रादिनोक्तः । स‚म्ब‚न्धो नाध्य‚क्ष इत्युक्त्वा ।}}अश्राव‚णः श‚ब्दो नित्यो घ‚ट इति त‚स्माद् ‚{\tiny $_{lb}$}‚विष‚य‚भेदोप‚ल‚क्ष‚णार्थ स‚हेतुत्वाहेतुत्व‚द‚र्श‚नं । प्र‚त्य‚क्षानुमा‚{\tiny $_{7}$}‚न बाधे स‚र्व्वाविष‚ये ।\leavevmode\ledsidenote{\textenglish{92b/MA}} अभ्युप‚ग‚म‚प्र‚तीतिबाधे तु निय‚त‚विष‚ये इत्य‚र्थः ॥ (१३०)
	\pend% ending standard par
      \label{div_pvv.4.131_4.132}
	  
	% new div opening: depth here is 2
	

	  \pstart \leavevmode% starting standard par
	उक्ता प्र‚तीतिबाधा ॥
	\pend% ending standard par
      \textsuperscript{\textenglish{458/s}}

	  \begin{center}%% label @type='head'
	\textbf{(७) प्र‚त्य‚क्ष‚बाधा}
	\end{center}
	

	  \pstart \leavevmode% starting standard par
	प्र‚त्य‚क्ष‚बाधा व‚क्त‚व्या । न केव‚लं शाब्द‚प्र‚सिद्धे व्य‚व‚हार‚ध‚र्म‚प्र‚सिद्धौ त‚त्प्र‚तिरोद्धा ‚{\tiny $_{lb}$}‚बाध्य‚ते । किन्तु (।)
	\pend% ending standard par
      
	  \bigskip
	  \begingroup
	
	    \large
	  
	    \begin{quote}
	  
	    
	    \stanza[\smallbreak]
	\label{pv.4.131a}\flagstanza{\tiny\textenglish{...4.131a}}अत्राप्य‚ध्य‚क्ष‚बाधायां नानारूप‚त‚या ध्व‚नौ ।&प्र‚सिद्ध‚स्य श्रुतौ;\&[\smallbreak]


	
	    \end{quote}
	  
	  \endgroup
	

	  \pstart \leavevmode% starting standard par
	\hphantom{.}‚{\color{DodgerBlue3}‚अत्राप्य‚ध्य‚क्ष‚बाधायां} व्याव‚हारिक‚क‚ल्प‚नाव‚शात् ‚{\color{DodgerBlue3}‚नानारूप‚त‚या} लोके ‚{\color{DodgerBlue3}‚प्र‚सिद्ध‚स्य} ख्यात‚स्य ‚{\color{DodgerBlue3}‚ध्व‚नौ श्रुतौ} श्र‚व‚ण‚ज्ञाने ॥
	\pend% ending standard par
      
	  \bigskip
	  \begingroup
	
	    \large
	  
	    \begin{quote}
	  
	    
	    \stanza[\smallbreak]
	\label{pv.4.131b}\flagstanza{\tiny\textenglish{...4.131b}}रूपं य‚देव प्र‚तिभास‚ते ॥ १३१ ॥\&[\smallbreak]


	
	    \end{quote}
	  
	  \endgroup
	
	  \bigskip
	  \begingroup
	
	    \large
	  
	    \begin{quote}
	  
	    
	    \stanza[\smallbreak]
	\label{pv.4.132}\flagstanza{\tiny\textenglish{....4.132}}अद्व‚यं श‚ब‚लाभास‚स्यादृष्टेर्बुद्धिज‚न्म‚नः ।&त‚द‚र्थार्थोक्तिर‚स्यैव क्षेपेऽध्य‚क्षेण बाध‚न‚म् ॥ १३२ ॥\&[\smallbreak]


	
	    \end{quote}
	  
	  \endgroup
	

	  \pstart \leavevmode% starting standard par
	\hphantom{.}‚{\color{DodgerBlue3}‚य‚देव रूप‚म‚द्व‚यं} ध‚र्मादिद्व‚य‚शून्यं ‚{\color{DodgerBlue3}‚प्र‚तिभास‚ते श‚ब‚लाभास‚स्य} नाना‚{\tiny $_{1}$}‚कार‚स्य ‚{\tiny $_{lb}$}‚‚{\color{DodgerBlue3}‚बुद्धिज‚न्म‚नोऽदृष्टेः} । य‚दि नानाक‚र‚ता श‚ब्द‚स्य वास्त‚वी स्यात् (।) त‚थैव श्रुतिज्ञाने ‚{\tiny $_{lb}$}‚प्र‚तिभासेत । ‚{\color{DodgerBlue3}‚त‚द‚र्था} त‚त्प्र‚तिपाद‚न‚फ‚लाऽचार्य‚स्य प्र‚त्य‚क्षानुमानार्थ‚प्र‚सिद्धेन निराकृत‚{\tiny $_{lb}$}‚इत्य‚त्रा‚{\color{DodgerBlue3}‚र्थोक्तिर‚र्थ\edtext{}{\edlabel{pvv.458-1}\label{pvv.458-1}\lemma{र्थ}\Bfootnote{श्रोत्र‚श‚ब्द‚योर्यः स‚म्ब‚न्धो ग्राह्य‚गाह‚क‚ल‚क्ष‚ण‚स्त‚द्धित‚वाच्यः स श्राव‚ण‚श‚ब्द‚स्यार्थः ।}}}‚ग्र‚ह‚ण‚म‚स्याध्य‚क्ष\edtext{}{\edlabel{pvv.458-2}\label{pvv.458-2}\lemma{क्ष}\Bfootnote{स्व‚ल‚क्ष‚ण‚स्यैव ।}}सिद्ध‚स्यैव रूप‚स्य \edtext{}{\edlabel{pvv.458-3}\label{pvv.458-3}\lemma{स्य}\Bfootnote{न ग्राह्य‚ग्राह‚क‚त्व‚प्र‚तिषेधे सामान्य‚स्य वा ।}} ‚{\color{DodgerBlue3}‚क्षेपेऽध्य‚क्षेण बाध‚न}‚मिष्टं । ‚{\tiny $_{lb}$}‚(१३१,१३२)
	\pend% ending standard par
      
	  
	% new div opening: depth here is 1
	
\chapter*[{४. सामान्य‚चिन्ता}]{४. सामान्य‚चिन्ता}

	  \begin{center}%% label @type='head'
	\textbf{(१) सामान्यं व्यावृत्तिल‚क्ष‚ण‚म्}
	\end{center}
	\label{div_pvv.4.133}
	  
	% new div opening: depth here is 2
	
	  \bigskip
	  \begingroup
	
	    \large
	  
	    \begin{quote}
	  
	    
	    \stanza[\smallbreak]
	\label{pv.4.133}\flagstanza{\tiny\textenglish{....4.133}}त‚देव रूपं त‚त्रार्थः शेषं व्यावृत्तिल‚क्ष‚ण‚म् ।&अव‚स्तुभूतं सामान्य‚म‚त‚स्त‚न्नाक्ष‚गोच‚रः ॥ १३३ ॥\&[\smallbreak]


	
	    \end{quote}
	  
	  \endgroup
	

	  \pstart \leavevmode% starting standard par
	\hphantom{.}‚{\color{DodgerBlue3}‚त‚त्र} श्रुतिज्ञाने ‚{\color{DodgerBlue3}‚भास}‚मानं ‚{\color{DodgerBlue3}‚त‚द्रूप}‚म‚र्थः स्व‚ल‚क्ष‚णं । त‚द‚तिरिक्तं ‚{\color{DodgerBlue3}‚शेषं} । ध‚र्मिध‚र्मादि\edtext{}{\edlabel{pvv.458-4}\label{pvv.458-4}\lemma{र्मादि}\Bfootnote{जात्यादि ।}} ‚{\tiny $_{lb}$}‚‚{\color{DodgerBlue3}‚व्यावृत्तिल‚क्ष‚ण}‚म‚न्य‚व्य‚व‚च्छेद‚स्व‚भाव‚म‚{\tiny $_{2}$}‚‚{\color{DodgerBlue3}‚व‚स्तु}‚भूतं स‚र्व्व‚त्र स‚म्भ‚वात् ‚{\color{DodgerBlue3}‚सामा}‚न्य‚ञ्च । ‚{\tiny $_{lb}$}‚‚{\color{DodgerBlue3}‚अतोऽ}‚व‚स्तुत्वादेस्त‚द्गुण‚जात्या‚{\color{DodgerBlue3}‚दि नाक्ष‚गोच‚रः} । (१३३)
	\pend% ending standard par
      \label{div_pvv.4.134}
	  
	% new div opening: depth here is 2
	\textsuperscript{\textenglish{459/s}}
	  \bigskip
	  \begingroup
	
	    \large
	  
	    \begin{quote}
	  
	    
	    \stanza[\smallbreak]
	\label{pv.4.134}\flagstanza{\tiny\textenglish{....4.134}}तेन स‚मान्य‚ध‚र्माणाम‚प्र‚त्य‚क्ष‚त्व‚सिद्धितः ।&प्र‚तिक्षेपेप्य‚बाधेति श्राव‚णोक्त्या प्र‚काशित‚म् ॥ १३४ ॥\&[\smallbreak]


	
	    \end{quote}
	  
	  \endgroup
	

	  \pstart \leavevmode% starting standard par
	\hphantom{.}‚{\color{DodgerBlue3}‚तेना}‚व‚स्तुत्वेन कार‚णेन ‚{\color{DodgerBlue3}‚सामान्य‚ध‚र्माणां} प्र‚मेय‚त्वादीनाम‚{\color{DodgerBlue3}‚प्र‚त्य‚क्ष‚त्व‚स्य ‚{\tiny $_{lb}$}‚सिद्धितः} । केन‚चिद्वादिना प्र‚त्य‚क्ष‚सिद्धितः ‚{\color{DodgerBlue3}‚प्र‚तिक्षेपेपि} क्रिय‚माणे न बाध्य‚बाध‚क‚{\tiny $_{lb}$}‚भाव इत्य‚श्राव‚ण इत्य‚त्र ‚{\color{DodgerBlue3}‚श्राव‚णोक्त्या} निषेध्य‚द‚र्शिक‚या प्र‚काशितं स्व‚ल‚क्ष‚ण‚बाध‚ने ‚{\tiny $_{lb}$}‚प्र‚त्य‚क्ष‚{\color{DodgerBlue3}‚बाधेत्य‚र्थः} ॥ (१३४)
	\pend% ending standard par
      \label{div_pvv.4.135}
	  
	% new div opening: depth here is 2
	

	  \pstart \leavevmode% starting standard par
	\edtext{\textsuperscript{*}}{\edlabel{pvv.459-1}\label{pvv.459-1}\lemma{*}\Bfootnote{इन्द्रिय‚विष‚य‚स्व‚भावःस्व‚ल‚क्ष‚णं स य‚दि श्राव‚ण‚श‚ब्देनाभिम‚तः ।}} एवं त‚र्हि श‚ब्द‚स्व‚{\tiny $_{3}$}‚ल‚क्ष‚णं नास्तीत्येव क‚स्मान्नोच्य‚ते । किं श्राव‚ण‚त्व\edtext{}{\edlabel{pvv.459-2}\label{pvv.459-2}\lemma{त्व}\Bfootnote{किं क्रियानिमित्तेनोच्य‚ते ।}}‚{\tiny $_{lb}$}‚मुख्यं निषेध्य युक्तिरित्याह ।
	\pend% ending standard par
      
	  \bigskip
	  \begingroup
	
	    \large
	  
	    \begin{quote}
	  
	    
	    \stanza[\smallbreak]
	\label{pv.4.135}\flagstanza{\tiny\textenglish{....4.135}}स‚र्व्व‚थाऽवाच्य‚रूप‚त्वात् सिद्ध्या त‚स्य स‚माश्र‚यात् ।&बाध‚नात् त‚द्ब‚लेनोक्तः श्राव‚णेनाक्ष‚गोच‚रः ॥ १३५ ॥\&[\smallbreak]


	
	    \end{quote}
	  
	  \endgroup
	

	  \pstart \leavevmode% starting standard par
	\hphantom{.}स्व‚ल‚क्ष‚ण‚स्य ‚{\color{DodgerBlue3}‚स‚र्व‚था} केन‚चिच्छ‚ब्देना‚{\color{DodgerBlue3}‚वाच्य}‚त्वात्\edtext{}{\edlabel{pvv.459-3}\label{pvv.459-3}\lemma{त्वात्}\Bfootnote{स्व‚ल‚क्ष‚ण‚स्य ।}} मुख्य‚म‚भिधानं नास्त्येव । ‚{\tiny $_{lb}$}‚अथ सामान्य‚वृत्तिर‚पि स्व‚ल‚क्ष‚ण‚श‚ब्दोऽध्य‚व‚सायानुरोधात् स्व‚ल‚क्ष‚ण‚मुप‚ल‚क्ष‚य‚ति । एवं ‚{\tiny $_{lb}$}‚श्राव‚ण‚श‚ब्दोपीन्द्रिय‚ग्राह्य‚तोप‚ल‚क्ष‚णं श‚ब्द‚स्व‚ल‚क्ष‚ण‚मुप‚ल‚क्ष‚यिष्य‚तीति न क‚श्चि‚{\tiny $_{lb}$}‚द्विशेषः । अथ‚वास्त्येव श्राव‚ण‚श‚ब्देना‚{\tiny $_{4}$}‚भिधाने प्र‚योज‚न‚मित्याह । श्रोत्रेन्द्रिय‚विष‚स्य ‚{\tiny $_{lb}$}‚या सिद्धिस्त‚था भावः । त‚या सिद्ध्या ‚{\color{DodgerBlue3}‚त‚स्ये}‚न्द्रिय‚ज्ञान‚स्य ‚{\color{DodgerBlue3}‚स‚माश्र‚या}‚च्छ‚ब्द‚स्य ‚{\tiny $_{lb}$}‚स्व‚रूप‚व्य‚व‚स्थित्या हेतुना तेन व्य‚प‚देश इन्द्रिय‚ज्ञान‚ब‚लेन श‚ब्द‚स्व‚रूप‚पाव‚स्थितः (।) ‚{\tiny $_{lb}$}‚त‚थाभिधान‚मित्य‚र्थः । किञ्च त‚स्येन्द्रिय‚ज्ञान‚स्य ‚{\color{DodgerBlue3}‚ब‚लेन} श‚ब्द‚स्व‚ल‚क्ष‚ण‚{\color{DodgerBlue3}‚विप‚र्य‚य‚भाव‚स्य\edtext{}{\edlabel{pvv.459-4}\label{pvv.459-4}\lemma{स्य}\Bfootnote{त‚थाभूतार्थ‚प्र‚तिषेध‚क‚स्य पुरुष‚स्य ।}}} ‚{\color{DodgerBlue3}‚बाधानात्} कार‚णात् ‚{\color{DodgerBlue3}‚श्राव‚णेनाक्ष‚गोच‚रः} स्व‚ल‚क्ष‚ण‚{\color{DodgerBlue3}‚मुक्तः} । नित्यो घ‚ट इत्य‚नुमाने ‚{\tiny $_{lb}$}‚नित्य‚त्व‚विष‚येण कृत‚क‚त्व‚लिङ्ग‚भुवा‚{\tiny $_{5}$}‚ बाधितः प‚क्षः । स चास‚कृद्द‚र्शित एवेति ‚{\tiny $_{lb}$}‚नेह विप‚ञ्चितः ॥ (१३५)
	\pend% ending standard par
      \label{div_pvv.4.136}
	  
	% new div opening: depth here is 2
	

	  \begin{center}%% label @type='head'
	\textbf{(२) स्व‚ध‚र्मिग्र‚ह‚ण‚प्र‚योज‚न‚म्}
	\end{center}
	

	  \pstart \leavevmode% starting standard par
	इदानीं प्र‚त्य‚क्षानुमानाप्त‚प्र‚सिद्धेन स्व\edtext{}{\edlabel{pvv.459-5}\label{pvv.459-5}\lemma{स्व}\Bfootnote{वादिनेष्ट‚स्य स्व‚स्य ध‚र्मी स्व‚ध‚र्मी त‚त्र ।}}ध‚र्मिणीति स्व‚ध‚र्मिग्र‚ह‚ण‚स्य ‚{\color{DodgerBlue3}‚साफ‚ल्य}‚{\tiny $_{lb}$}‚माख्यातुमाह ।
	\pend% ending standard par
      
	  \bigskip
	  \begingroup
	
	    \large
	  
	    \begin{quote}
	  
	    
	    \stanza[\smallbreak]
	\label{pv.4.136}\flagstanza{\tiny\textenglish{....4.136}}स‚र्व्व‚त्र वादिनो ध‚र्मो यः स्व‚साध्य‚त‚योप्सितः ।&त‚द्ध‚र्म‚व‚ति बाधा स्यान्नान्य‚ध‚र्मेण ध‚र्मिणि ॥ १३६ ॥\&[\smallbreak]


	
	    \end{quote}
	  
	  \endgroup
	\textsuperscript{\textenglish{460/s}}

	  \pstart \leavevmode% starting standard par
	\hphantom{.}‚{\color{DodgerBlue3}‚स‚र्व्व‚त्र} वाद‚काले साध्य‚काले वा ‚{\color{DodgerBlue3}‚वादिनः स्व‚साध्य‚त‚या यो ध‚र्म ईप्सितः ‚{\tiny $_{lb}$}‚त‚द्ध‚र्भ‚व‚ति} ध‚र्मिणि ‚{\color{DodgerBlue3}‚बाधा स्यात्} । य‚था श्राव‚ण‚त्व‚व‚ति श‚ब्दे बाधिते बाधा ‚{\tiny $_{lb}$}‚प‚क्ष‚स्य । ‚{\color{DodgerBlue3}‚न} तु वादीष्टाद् ध‚र्माद‚{\color{DodgerBlue3}‚न्येन ध‚र्मेण} ध‚र्म‚व‚ति ‚{\color{DodgerBlue3}‚ध‚र्मिणि} बाधिते बाधा ‚{\tiny $_{lb}$}‚प‚क्ष‚स्य स्यादिति ध‚र्मिग्र‚ह‚प्र‚{\tiny $_{6}$}‚योज‚नं य‚थाकाश‚गुण‚त्व‚व‚ति श‚ब्दे बाधितेनं ‚{\tiny $_{lb}$}‚प‚क्ष‚बाधा । (१३६)
	\pend% ending standard par
      \label{div_pvv.4.137}
	  
	% new div opening: depth here is 2
	
	  \bigskip
	  \begingroup
	
	    \large
	  
	    \begin{quote}
	  
	    
	    \stanza[\smallbreak]
	\label{pv.4.137}\flagstanza{\tiny\textenglish{....4.137}}अन्य‚थास्योप‚रोधः को बाधितेन्य‚त्र ध‚र्मिणि ।&ग‚तार्थे ल‚क्ष‚णेनास्मिन् स्व‚ध‚र्मिव‚च‚नं पुनः ॥ १३७ ॥\&[\smallbreak]


	
	    \end{quote}
	  
	  \endgroup
	

	  \pstart \leavevmode% starting standard par
	\hphantom{.}‚{\color{DodgerBlue3}‚अन्य‚था} य‚द्येवं नेष्य‚ते त‚{\color{DodgerBlue3}‚दान्य‚त्र} ध‚र्मे ‚{\color{DodgerBlue3}‚ध‚र्मिणि बाधितेऽस्य} प्र‚कृत‚ध‚र्म‚विशिष्ट‚स्य ‚{\tiny $_{lb}$}‚ध‚र्मिणः क ‚{\color{DodgerBlue3}‚उप‚रोधो} बाध\edtext{}{\edlabel{pvv.460-1}\label{pvv.460-1}\lemma{बाध}\Bfootnote{येन त‚दाश‚ङ्कानिवृत्त्य‚र्थः स्यात् ।}} ॥ न‚नु स्व‚रूपेणैव निर्देश्यः स्व‚य‚मिष्टोऽनिराकृतः ‚{\tiny $_{lb}$}‚प‚क्ष इति प‚क्ष‚ल‚क्ष‚णेनैवानिष्ट‚ध‚र्म‚व‚तो ध‚र्मिणो बाधा न प‚क्ष‚बाधेति ल‚भ्य‚त ‚{\tiny $_{lb}$}‚\leavevmode\ledsidenote{\textenglish{93a/MA}} एवेत्याह । प‚क्ष‚स्य ‚{\color{DodgerBlue3}‚ल‚क्ष‚णेनास्मिन्} वादीष्ट‚ध‚र्म‚व‚ति ध‚र्मिणि बाध्य‚{\tiny $_{7}$}‚त्वेन ‚{\color{DodgerBlue3}‚ग‚तार्थे पुनः} ‚{\color{DodgerBlue3}‚स्व‚ध‚र्मिव‚च‚नं} य‚त्कृतं । (१३७)
	\pend% ending standard par
      \label{div_pvv.4.138}
	  
	% new div opening: depth here is 2
	
	  \bigskip
	  \begingroup
	
	    \large
	  
	    \begin{quote}
	  
	    
	    \stanza[\smallbreak]
	\label{pv.4.138}\flagstanza{\tiny\textenglish{....4.138}}बाधायां ध‚र्मिणोपि स्यात् बाधेत्य‚स्य प्र‚सिद्ध‚ये ।&आश्र‚य‚स्य विरोधेन त‚दाश्रित‚विरोध‚नात् ॥ १३८ ॥\&[\smallbreak]


	
	    \end{quote}
	  
	  \endgroup
	

	  \pstart \leavevmode% starting standard par
	\hphantom{.}‚{\color{DodgerBlue3}‚ध‚र्मिणोपि\edtext{}{\edlabel{pvv.460-2}\label{pvv.460-2}\lemma{र्मिणोपि}\Bfootnote{त‚न्न साध्य‚स्यैव बाधायां बाधा किन्तु ध‚र्मिणोपि ।}} बाधायां} ध‚र्म‚स्य बाध‚ने प‚क्ष‚{\color{DodgerBlue3}‚बाधा} य‚था ‚{\color{DodgerBlue3}‚स्यादित्य‚स्या}‚र्थ‚स्य ‚{\color{DodgerBlue3}‚प्र‚सि‚{\tiny $_{lb}$}‚द्ध‚ये} क्व‚चिदा‚{\color{DodgerBlue3}‚श्र‚य‚स्य} ध‚र्मिणो ‚{\color{DodgerBlue3}‚विरोधेन} प्र‚तिक्षेपेण त‚दा‚{\color{DodgerBlue3}‚श्रित‚स्य} ध‚र्म‚स्य ‚{\color{DodgerBlue3}‚विरोध‚नात्} ध‚र्म‚द्वारेण ध‚र्मिद्वारेण वा स‚मुदाय‚बाधायां प‚क्ष‚बाधेत्य‚र्थः । (१३८)
	\pend% ending standard par
      \label{div_pvv.4.139}
	  
	% new div opening: depth here is 2
	
	  \bigskip
	  \begingroup
	
	    \large
	  
	    \begin{quote}
	  
	    
	    \stanza[\smallbreak]
	\label{pv.4.139}\flagstanza{\tiny\textenglish{....4.139}}अन्य‚थैवंविधो ध‚र्मः साध्य इत्य‚भिधान‚तः ।&त‚द्बाधामेव म‚न्येत स्व‚ध‚र्मिग्र‚ह‚ण‚न्त‚तः ॥ १३९ ॥\&[\smallbreak]


	
	    \end{quote}
	  
	  \endgroup
	

	  \pstart \leavevmode% starting standard par
	\hphantom{.}‚{\color{DodgerBlue3}‚अन्य‚था} ध‚र्मिद्वारेण स‚मुदाय‚बाधायाः संग्र‚हार्थं स्व‚ध‚र्मिग्र‚ह‚णं य‚दि न क्रिय‚ते ‚{\tiny $_{lb}$}‚‚{\color{DodgerBlue3}‚त‚दैव‚म्विधः} साध्य‚त्वेनैवेष्टो ‚{\color{DodgerBlue3}‚ध‚र्मः} साध्य ‚{\color{DodgerBlue3}‚इत्य‚भिधान‚तः} । ‚{\color{DodgerBlue3}‚त}‚स्यं ध‚{\tiny $_{1}$}‚र्म‚मात्र‚स्यैव ‚{\tiny $_{lb}$}‚‚{\color{DodgerBlue3}‚बाधां म‚न्येत} प्र‚तिप‚त्ता न तु ध‚र्मिबाधाम‚पि । त‚तः (।) उभ‚य‚संग्र‚हार्थं ‚{\color{DodgerBlue3}‚स्व‚ध‚र्मि}‚{\tiny $_{lb}$}‚ग्र‚ह‚ण‚मा चा र्य स्य ॥ (१३९)
	\pend% ending standard par
      \label{div_pvv.4.140}
	  
	% new div opening: depth here is 2
	
	  \bigskip
	  \begingroup
	
	    \large
	  
	    \begin{quote}
	  
	    
	    \stanza[\smallbreak]
	\label{pv.4.140}\flagstanza{\tiny\textenglish{....4.140}}न‚न्वेत‚द‚प्य‚र्थ‚सिद्धं स‚त्यं केचित्तु ध‚र्मिणः ।&केव‚ल‚स्योप‚रोधेपि दोष‚व‚त्तामुपाग‚ताः ॥ १४० ॥\&[\smallbreak]


	
	    \end{quote}
	  
	  \endgroup
	\textsuperscript{\textenglish{461/s}}

	  \pstart \leavevmode% starting standard par
	\hphantom{.}‚{\color{DodgerBlue3}‚न‚न्वे}‚त‚दुभ‚य‚संग्र‚ह‚ण‚{\color{DodgerBlue3}‚म‚प्य‚र्थ‚तः} \edtext{\textsuperscript{*}}{\edlabel{pvv.461-1}\label{pvv.461-1}\lemma{*}\Bfootnote{साक्षाद् ध‚र्मिद्वारेण चेत्य‚विशेषात् ।}} साम‚र्थ्यंतः ‚{\color{DodgerBlue3}‚सिद्धं} । न हि केव‚लो ध‚र्मोस्ति ‚{\tiny $_{lb}$}‚क्व‚चित् । त‚द्वाध‚ने त‚द्विशिष्ट‚स्य ‚{\color{DodgerBlue3}‚ध‚र्मिणो}‚पि बाध‚नात् स‚मुदाय‚बाधैवैषित‚व्या । सा च ‚{\tiny $_{lb}$}‚ध‚र्मिद्वारेण वा भ‚व‚तु ध‚र्म‚द्वारेण वा न क‚श्चिद् विशेष इत्य‚र्थः ।
	\pend% ending standard par
      

	  \pstart \leavevmode% starting standard par
	\hphantom{.}अत्राह स‚त्य‚मेत‚त् ‚{\color{DodgerBlue3}‚केचित्तु} वादिनो \edtext{}{\edlabel{pvv.461-2}\label{pvv.461-2}\lemma{वादिनो}\Bfootnote{य‚त्र ध‚र्मिबाधेपि साध्य‚स्य न क्ष‚तिः त‚त्रापि दोषः ।}} ‚{\color{DodgerBlue3}‚ध‚र्मिणोः केव‚ल‚स्योप‚रोधे साध्य-‚{\tiny $_{2}$}‚} ध‚र्म‚स्याबाधायाम‚पि प‚क्ष‚स्य ‚{\color{DodgerBlue3}‚दोष‚व‚त्तामुपाग‚ताः} प्र‚तिप‚न्नाः । (१४०)
	\pend% ending standard par
      \label{div_pvv.4.141}
	  
	% new div opening: depth here is 2
	
	  \bigskip
	  \begingroup
	
	    \large
	  
	    \begin{quote}
	  
	    
	    \stanza[\smallbreak]
	\label{pv.1.141}\flagstanza{\tiny\textenglish{....1.141}}य‚था प‚रैर‚नुत्पाद्यापूर्व्व‚रूप‚न्न खादिक‚म् ।&स‚कृच्छ‚ब्दाद्य‚हेतुत्वादित्युक्ते प्राह दूष‚कः ॥ १४१ ॥\&[\smallbreak]


	
	    \end{quote}
	  
	  \endgroup
	

	  \pstart \leavevmode% starting standard par
	\hphantom{.}‚{\color{DodgerBlue3}‚य‚था प‚रैः} स‚ह‚कारि\edtext{}{\edlabel{pvv.461-3}\label{pvv.461-3}\lemma{कारि}\Bfootnote{वै शे षि (कं) प्र‚ति सौ त्रा न्ति केन ।}}भिर‚{\color{DodgerBlue3}‚नुत्पाद्यापूर्व्व‚रूपं\edtext{}{\edlabel{pvv.461-4}\label{pvv.461-4}\lemma{रूपं}\Bfootnote{न क्ष‚त‚स्व‚भाव‚मिति साध्यं ध‚र्मी ।}} न खादिक}‚माकाश‚दिक्कालादि ‚{\tiny $_{lb}$}‚न भ‚व‚ति । अपि तूत्पाद्य‚पूर्व्व‚रूप‚मेव भ‚व‚ति । स‚कृदेक‚काल‚न्त‚दुत्पाद्य‚स्य\edtext{}{\edlabel{pvv.461-5}\label{pvv.461-5}\lemma{स्य}\Bfootnote{श‚ब्द‚हेतुराकाशं त‚द् य‚दि नित्य एक‚दोत्प‚त्तिप्र‚स‚ङ्गः स‚दा स‚न्निधानात् हेतोः ।}} कार्य‚{\tiny $_{lb}$}‚काल‚प‚स्य ‚{\color{DodgerBlue3}‚श‚ब्दादेर‚हेतुत्वाद}‚विशिष्टैक‚रूपात् कार‚णात् स‚कृत् स‚र्व्व‚कार्योत्प‚त्ति‚{\tiny $_{lb}$}‚प्र‚स‚ङ्गा‚{\color{DodgerBlue3}‚दिति} वादि‚{\color{DodgerBlue3}‚नोक्ते} ‚{\color{DodgerBlue3}‚दूष‚कः} प्र‚तिवाद्याह\edtext{}{\edlabel{pvv.461-6}\label{pvv.461-6}\lemma{तिवाद्याह}\Bfootnote{विरुद्ध‚तां ।}}। (१४१)
	\pend% ending standard par
      \label{div_pvv.4.142}
	  
	% new div opening: depth here is 2
	
	  \bigskip
	  \begingroup
	
	    \large
	  
	    \begin{quote}
	  
	    
	    \stanza[\smallbreak]
	\label{pv.4.142}\flagstanza{\tiny\textenglish{....4.142}}त‚द्व‚द् व‚स्तुस्व‚भावोऽस‚न् ध‚र्मी व्योमादिरित्य‚पि ।&नैव‚मिष्ट‚स्य साध्य‚स्य बाधा काच‚न विद्य‚ते ॥ १४२ ॥\&[\smallbreak]


	
	    \end{quote}
	  
	  \endgroup
	

	  \pstart \leavevmode% starting standard par
	\hphantom{.}‚{\color{DodgerBlue3}‚त‚द्व‚द्य}‚थानुत्पाद्यापूर्व्व‚रूप आकाशादिर्न‚{\tiny $_{5}$}‚ भ‚व‚ति ।\edtext{\textsuperscript{*}}{\edlabel{pvv.461-7}\label{pvv.461-7}\lemma{*}\Bfootnote{व‚स्तुस्व‚भाव आकाशादिर्द्ध‚र्मो न वेति विव‚क्षाम‚न‚ङ्गीकृत्याकाश‚स‚त्त्व‚वादिना सौ त्रा न्ति के न सामान्येन प्र‚कृते आकाशादौ स्थिर‚रूप‚त्व‚ध‚र्म‚व्य‚व‚च्छेद‚मात्रे साध्ये युग‚प‚द्धेतुत्व‚व्य‚व‚च्छेद‚रूपे हेतौ य‚द्याह प‚रः व‚स्तुभूताकाशाभावं । तादृशे निराकृतेपि नैवेष्ट‚स्य व्य‚व‚च्छेद‚मात्र‚स्य बाधा प्र‚ज्ञ‚प्तिम‚ति ध‚र्मिणि व्य‚व‚च्छेद मात्र‚स्याव्याघातात् ।}} त‚था व‚स्तु‚{\color{DodgerBlue3}‚स्व‚भावो ध‚र्मी ‚{\tiny $_{lb}$}‚व्योमादिर‚स‚न्नित्य‚पि} स्यात् । अर्थ‚क्रियाऽस‚म‚र्थ‚स्य व‚स्तुत्वाभावादिति ध‚र्मिणः ‚{\tiny $_{lb}$}‚केव‚ल‚स्य बाध‚नं न ध‚र्म‚स्य । ‚{\color{DodgerBlue3}‚एवं} ध‚र्मिबाध‚नेपी‚{\color{DodgerBlue3}‚ष्ट‚स्य साध्य‚स्य काच‚न बाधा न ‚{\tiny $_{lb}$}‚विद्य‚ते} । हेतोर्व्वाऽसिद्धिः । अस‚त्य‚पि कार्यानुत्पाद‚स्य व्य‚व‚च्छेद‚स्य सिद्धेः । (१४२)
	\pend% ending standard par
      \label{div_pvv.4.143}
	  
	% new div opening: depth here is 2
	

	  \pstart \leavevmode% starting standard par
	त‚तः (।)
	\pend% ending standard par
      \textsuperscript{\textenglish{462/s}}
	  \bigskip
	  \begingroup
	
	    \large
	  
	    \begin{quote}
	  
	    
	    \stanza[\smallbreak]
	\label{pv.4.143}\flagstanza{\tiny\textenglish{....4.143}}द्व‚य‚स्यापि हि साध्य‚त्वे साध्य‚ध‚र्मोप‚रोधि य‚त् ।&बाध‚नं ध‚र्मिण‚स्त‚त्र बाधेत्येतेन व‚र्ण्णित‚म् ॥ १४३ ॥\&[\smallbreak]


	
	    \end{quote}
	  
	  \endgroup
	

	  \pstart \leavevmode% starting standard par
	\hphantom{.}‚{\color{DodgerBlue3}‚द्व‚य‚स्य} ध‚र्मिध‚र्म‚स‚मुदाय‚स्यापि हि ‚{\color{DodgerBlue3}‚साध्य‚त्वे} स‚ति य‚त्र ‚{\color{DodgerBlue3}‚ध‚र्मिणो बाध‚नं साध्य‚{\tiny $_{lb}$}‚ध‚र्मोप‚रोधि} त‚त्र प‚क्ष‚{\color{DodgerBlue3}‚बाधा} युक्तेत्ये‚{\color{DodgerBlue3}‚तेन} स्व‚ध‚{\tiny $_{4}$}‚र्मिग्र‚ह‚णेन ‚{\color{DodgerBlue3}‚व‚र्ण्णितं} । न हि साध्य‚{\tiny $_{lb}$}‚ध‚र्म‚मात्र‚साध‚नार्थं क‚श्चित् साध‚न‚म‚न्वेष‚ते । त‚स्य ज‚ग‚ति क्व‚चित् स‚त्तायां विवा‚{\tiny $_{lb}$}‚दाभावात् । त‚था निश्च‚ये प्र‚वृत्त्य‚योगाच्च । किन्तु ध‚र्मिविशेष‚निष्ठं साध्य‚{\tiny $_{lb}$}‚मिष्टं ॥ (१४३)
	\pend% ending standard par
      \label{div_pvv.4.144_4.145}
	  
	% new div opening: depth here is 2
	
	  \bigskip
	  \begingroup
	
	    \large
	  
	    \begin{quote}
	  
	    
	    \stanza[\smallbreak]
	\label{pv.4.144a}\flagstanza{\tiny\textenglish{...4.144a}}त‚थैव ध‚र्मिणोप्य‚त्र साध्य‚त्वात् केव‚ल‚स्य न ।&य‚द्येव‚म‚त्र बाधा स्यात्;\&[\smallbreak]


	
	    \end{quote}
	  
	  \endgroup
	

	  \pstart \leavevmode% starting standard par
	\hphantom{.}त‚त्र य‚था ध‚र्मिविशिष्ट‚स्य साध्य‚त्वं ध‚र्म‚स्य ‚{\color{DodgerBlue3}‚त‚थैव} साध्य‚त्व‚विशिष्ट‚त्वेन ‚{\color{DodgerBlue3}‚ध‚र्मि‚{\tiny $_{lb}$}‚णोपि साध्य‚त्वात् केव‚ल‚स्य} ध‚र्म‚स्य न क्व‚चित् साध्य‚ता । त‚तो न केव‚ल‚स्य ध‚र्मिणो ‚{\tiny $_{lb}$}‚बाध‚ने प‚क्ष‚बाधा ॥ ‚{\color{DodgerBlue3}‚य‚दि} साध्य‚{\tiny $_{5}$}‚ध‚र्मोप‚रोधिनि ध‚र्मिणि बाधिते प‚क्ष‚बाधेष्य‚ते ‚{\tiny $_{lb}$}‚(।) ‚{\color{DodgerBlue3}‚एवं} स‚त्य‚त्र हेतौ सां ख्यं प्र‚ति बौद्धेनोक्ते ध‚र्मिबाधाद्वारेण ध‚र्म‚{\color{DodgerBlue3}‚बाधा स्यात्} ।
	\pend% ending standard par
      
	  \bigskip
	  \begingroup
	
	    \large
	  
	    \begin{quote}
	  
	    
	    \stanza[\smallbreak]
	\label{pv.4.144b}\flagstanza{\tiny\textenglish{...4.144b}}नान्यानुत्पाद्य‚श‚क्तिकः ॥ १४४ ॥\&[\smallbreak]


	
	    \end{quote}
	  
	  \endgroup
	
	  \bigskip
	  \begingroup
	
	    \large
	  
	    \begin{quote}
	  
	    
	    \stanza[\smallbreak]
	\label{pv.4.145}\flagstanza{\tiny\textenglish{....4.145}}स‚कृच्छ‚ब्दाद्य‚हेतुत्वात् सुखादिरिति पूर्व्व‚व‚त् ।&विरोधिता भ‚वेद‚त्र हेतुरैकान्तिको य‚दि ॥ १४५ ॥\&[\smallbreak]


	
	    \end{quote}
	  
	  \endgroup
	

	  \pstart \leavevmode% starting standard par
	\hphantom{.}त‚द्य‚था ‚{\color{DodgerBlue3}‚सुखादिः} सुख‚दुःख‚मोहात्म‚कं ‚{\color{DodgerBlue3}‚प्र‚धानं नान्येन} स‚हाकारिणाऽ‚{\color{DodgerBlue3}‚नुत्पाद्य‚{\tiny $_{lb}$}‚श‚क्तिकोऽ}‚नाधेय‚साम‚र्थ्यः ‚{\color{DodgerBlue3}‚स‚कृच्छ‚ब्दा}‚दीनां कार्याणाम‚हे‚{\color{DodgerBlue3}‚हेतुत्वा}‚दिति । ‚{\color{DodgerBlue3}‚अत्र पूर्व्व‚व‚त्} प‚रैर‚नुत्पाद्येत्यादिप्र‚योग‚त्वाच्च । सुखाद्यात्म‚क‚स्य नित्य‚त्व‚स्य बाध‚नात् सुखादौ ‚{\tiny $_{lb}$}‚ध‚{\tiny $_{6}$}‚र्म्मिणि बाधिते त‚द्ध‚र्म्म‚स्य नित्य‚त्व‚स्य विरोध‚ने विप‚र्य‚य‚साध‚ने बाधा स्यात् ।\edtext{\textsuperscript{*}}{\edlabel{pvv.462-1}\label{pvv.462-1}\lemma{*}\Bfootnote{साम‚र्थ्याद‚न्याभावात् ।}} ‚{\tiny $_{lb}$}‚अनित्य‚स्व‚भावो हि सुखादिः साध‚यितुमिष्टः । सुखादिस्व‚भाव‚भूत‚नित्य‚त्व‚{\tiny $_{lb}$}‚बाध‚ने च सुखादिरेव बाधित इति ध‚र्मोप‚रोधिनि ध‚र्मिणि बाधिते प‚क्ष‚बाधा ‚{\tiny $_{lb}$}‚स्यात् । अत्राह । ‚{\color{DodgerBlue3}‚भ‚वेद‚त्र} हेतौ प‚क्ष‚बाधा । ‚{\color{DodgerBlue3}‚य‚दि} न स‚कृच्छ‚ब्दाद्य\edtext{}{\edlabel{pvv.462-2}\label{pvv.462-2}\lemma{ब्दाद्य}\Bfootnote{अादिना स्प‚र्श‚रूप‚र‚स‚ग‚न्ध‚ग्र‚हः ।}}नुत्पादादिति\leavevmode\ledsidenote{\textenglish{93b/MA}} ‚{\tiny $_{lb}$}‚‚{\color{DodgerBlue3}‚हेतुः} साध्य‚स्य । व‚स्तुभूत‚सुखाद्य‚नित्य‚त्व‚स्य । विप‚र्य‚ये सुखादिध‚र्म्य‚भावाद‚नि-‚{\tiny $_{7}$}‚ ‚{\tiny $_{lb}$}‚त्य‚त्वाभावे‚{\color{DodgerBlue3}‚नैकान्तिको}‚ऽव्य‚भिचारी भ‚वेत् । (१४४,१४५)
	\pend% ending standard par
      \label{div_pvv.4.146}
	  
	% new div opening: depth here is 2
	

	  \pstart \leavevmode% starting standard par
	याव‚ता या च (।)
	\pend% ending standard par
      \textsuperscript{\textenglish{463/s}}
	  \bigskip
	  \begingroup
	
	    \large
	  
	    \begin{quote}
	  
	    
	    \stanza[\smallbreak]
	\label{pv.4.146}\flagstanza{\tiny\textenglish{....4.146}}क्र‚म‚क्रियाऽनित्य‚त‚योर‚विरोधाद् विप‚क्ष‚त्तः ।&व्यावृत्तेः संश‚यान्नायं शेष‚व‚द् भेद इष्य‚ते ॥ १४६ ॥\&[\smallbreak]


	
	    \end{quote}
	  
	  \endgroup
	

	  \pstart \leavevmode% starting standard par
	\hphantom{.}‚{\color{DodgerBlue3}‚क्र‚म‚क्रिया} हेतुः । या च साध्य‚{\color{DodgerBlue3}‚ऽनित्य‚ता त‚योर‚विरोधात्} । अव‚{\color{DodgerBlue3}‚स्तुभूत} (‚{\color{DodgerBlue3}‚ध‚र्मि}‚) ‚{\tiny $_{lb}$}‚सुखादिध‚र्मानित्य‚त्वे विप‚रीते साध्ये व‚स्तुभूत‚सुखादिध‚र्मानित्य‚त्वं ‚{\color{DodgerBlue3}‚विप‚क्ष‚स्त‚तो} हेतोः क्र‚म‚क‚र‚ण‚{\color{DodgerBlue3}‚व्यावृत्तेः संश‚यान्नायं} विरुद्धो \edtext{}{\edlabel{pvv.463-1}\label{pvv.463-1}\lemma{विरुद्धो}\Bfootnote{ध‚र्मोप‚रोधाद् विरुद्धः सांख्य‚स्याचार्य‚विरुद्ध‚त्वं नेह ।}} हेतुः । ‚{\color{DodgerBlue3}‚शेष‚व‚द् भेदोऽ}‚नैकान्तिक‚{\tiny $_{lb}$}‚विशेष‚{\color{DodgerBlue3}‚स्त्विष्य‚ते} । स‚ति च विरुद्ध‚त्वे ध‚र्मिबाधाद्वारेण ध‚र्म‚बाधा स्यात् । (१४६)
	\pend% ending standard par
      \label{div_pvv.4.147}
	  
	% new div opening: depth here is 2
	

	  \begin{center}%% label @type='head'
	\textbf{(३) ध‚र्मिस्व‚रूप‚निरासः}
	\end{center}
	
	  \bigskip
	  \begingroup
	
	    \large
	  
	    \begin{quote}
	  
	    
	    \stanza[\smallbreak]
	\label{pv.4.147}\flagstanza{\tiny\textenglish{....4.147}}स्व‚य‚मिष्टो य‚तो ध‚र्मः साध्य‚स्त‚स्मात् त‚दाश्र‚यः ।&बाध्यो न केव‚लो नान्य‚संश्र‚यो वेति सूचित‚म् ॥ १४७ ॥\&[\smallbreak]


	
	    \end{quote}
	  
	  \endgroup
	

	  \pstart \leavevmode% starting standard par
	\hphantom{.}‚{\color{DodgerBlue3}‚य‚तः} कार‚णात् ‚{\color{DodgerBlue3}‚स्व‚य}‚म्वादि‚{\color{DodgerBlue3}‚नेष्टो‚{\tiny $_{1}$}‚ ध‚र्मः साध्य‚स्त‚स्मात्} साध्य‚ध‚र्म‚सा‚{\color{DodgerBlue3}‚श्र‚यः} यः स एव बाध्यः ‚{\color{DodgerBlue3}‚केव‚लो न बाध्यः} । य‚था व‚स्तुभूताकाश‚बाधायाम‚पि नित्यैक‚{\tiny $_{lb}$}‚रूप‚त्वाभाव‚स्य साध्य‚ध‚र्म‚स्य न क्ष‚तिः । साध्य‚ध‚र्मा‚{\color{DodgerBlue3}‚द‚न्य}‚स्य च ध‚र्म‚स्या‚{\color{DodgerBlue3}‚श्र‚यो न बाध्य ‚{\tiny $_{lb}$}‚इति} स्व‚यंश‚ब्देन ‚{\color{DodgerBlue3}‚सूचितं} । य‚थाऽनित्य‚त्वे साध्ये श‚ब्दे आकाश‚गुण‚त्वा‚{\color{DodgerBlue3}‚श्र‚य‚त्वेन} बाधायाम‚पि न दोषः । (१४७)
	\pend% ending standard par
      \label{div_pvv.4.148}
	  
	% new div opening: depth here is 2
	

	  \pstart \leavevmode% starting standard par
	त‚स्मात् (।)
	\pend% ending standard par
      
	  \bigskip
	  \begingroup
	
	    \large
	  
	    \begin{quote}
	  
	    
	    \stanza[\smallbreak]
	\label{pv.4.148}\flagstanza{\tiny\textenglish{....4.148}}स्व‚यंश्रुत्यान्य‚ध‚र्माणां बाधाऽबाधेति क‚थ्य‚ते ।&त‚था स्व‚ध‚र्मिणान्य‚स्य ध‚र्मिणोपीति क‚थ्य‚ते ॥ १४८ ॥\&[\smallbreak]


	
	    \end{quote}
	  
	  \endgroup
	

	  \pstart \leavevmode% starting standard par
	\hphantom{.}‚{\color{DodgerBlue3}‚स्व‚यंश्रुत्या} साध्याद् ध‚र्माद‚न्येषां ‚{\color{DodgerBlue3}‚ध‚र्माणां} य‚था ‚{\color{DodgerBlue3}‚बाधा} या सा ऽ‚{\color{DodgerBlue3}‚बाधेति क‚थ्य‚ते । ‚{\tiny $_{lb}$}‚त‚था स्व‚ध‚र्मिणा}‚{\tiny $_{2}$}‚ स्व‚ध‚र्मिव‚च‚नेन साध्या‚{\color{DodgerBlue3}‚द‚न्य‚स्य} ध‚र्म‚स्य ‚{\color{DodgerBlue3}‚ध‚र्मिणो} बाधाऽ‚{\color{DodgerBlue3}‚बाधेति} क‚थ्य‚ते ॥ (१४८)
	\pend% ending standard par
      
	  
	% new div opening: depth here is 1
	
\chapter*[{५---प‚क्ष‚दोषाः}]{५---प‚क्ष‚दोषाः}

	  \begin{center}%% label @type='head'
	\textbf{(१) हेतुनिर‚पेक्षः प‚क्ष‚दोषः}
	\end{center}
	\label{div_pvv.4.149}
	  
	% new div opening: depth here is 2
	

	  \pstart \leavevmode% starting standard par
	त‚थाऽप‚रेपि प‚क्षाभासाः स‚न्ति ते क‚स्मान्नोच्य‚न्ते । त‚था ह्य‚प्र‚सिद्ध‚विशेष्यः ‚{\tiny $_{lb}$}‚क‚श्चित् प‚क्षो य‚था\edtext{}{\edlabel{pvv.463-2}\label{pvv.463-2}\lemma{था}\Bfootnote{वैशेषिक‚स्य ।}} विभुरात्मा । \edtext{\textsuperscript{*}}{\edlabel{pvv.463-3}\label{pvv.463-3}\lemma{*}\Bfootnote{सिद्धान्ती स‚ति ध‚र्मिणि ध‚र्माश्चिन्त्य‚न्ते ।}} आत्म‚न एव बौद्ध स्यासिद्ध‚त्वात् । क‚श्चिद्‚{\tiny $_{lb}$}‚\leavevmode\ledsidenote{\textenglish{464/s}} -प्र‚सिद्ध‚विशेष‚णः । य‚था विनाशी श‚ब्दः ॥ सां ख्यं प्र‚माणं\edtext{}{\edlabel{pvv.464-1}\label{pvv.464-1}\lemma{माणं}\Bfootnote{बौद्ध‚स्य ।}} प्र‚ति त‚स्य विनाशासिद्धेः । ‚{\tiny $_{lb}$}‚क‚श्चिद‚प्र‚सिद्धोभ‚यः (।) य‚था स‚म‚वायिकार‚ण‚मात्मा । बौद्ध‚स्योभ‚यासिद्धेरिति‚{\tiny $_{3}$}‚ ॥
	\pend% ending standard par
      

	  \pstart \leavevmode% starting standard par
	न‚न्व‚सिद्धोप्यात्मा प‚क्षी भ‚विष्य‚ति त‚ल्ल‚क्ष‚ण‚योगात् (।) न‚ह्य‚सिद्ध‚स्य प‚क्ष‚ता ‚{\tiny $_{lb}$}‚निर‚स्ता । आश्र‚यासिद्ध‚त्वाद्धेतोर्न प‚क्ष इति चेत् । न ।\edtext{\textsuperscript{*}}{\edlabel{pvv.464-2}\label{pvv.464-2}\lemma{*}\Bfootnote{अत्राह सिद्धान्ती ।}}
	\pend% ending standard par
      

	  \pstart \leavevmode% starting standard par
	साध‚न‚दोषोयं न प‚क्ष‚दोषः । त‚था श‚ब्देऽनित्य‚त्वं विषेश‚ण‚म‚सिद्ध‚मिति गुण ‚{\tiny $_{lb}$}‚एवायं न प‚क्ष‚दोषः । असिद्ध‚स्यैव साध्य‚त्वात् । अथ विप‚र्य‚य‚सिद्ध्याऽसिद्ध‚मुच्य‚ते । ‚{\tiny $_{lb}$}‚त‚थापि मूढ‚स्य विप‚र्य‚य‚सिद्धाव‚पि नायं प‚क्ष‚दोषः । विरोधो नाम हेतुदोष ए-‚{\tiny $_{4}$}‚ ‚{\tiny $_{lb}$}‚वायं । प्र‚माणेन च विप‚र्य‚य‚सिद्धौ प्र‚माण‚बाधित‚त्व‚मेव प‚क्ष‚दोषोस्तु । अल‚म‚प्र‚सिद्ध‚{\tiny $_{lb}$}‚विशेष‚ण‚त्वाभिधानेन । अप्र‚सिद्धोभ‚य‚स्य तूभ‚य‚दोषाच्च स‚र्व्वेऽमी हेतुदोषा एवेति ‚{\tiny $_{lb}$}‚किं प‚क्ष‚दोषा व‚क्त‚व्याः । अथ (।)
	\pend% ending standard par
      
	  \bigskip
	  \begingroup
	
	    \large
	  
	    \begin{quote}
	  
	    
	    \stanza[\smallbreak]
	\label{pv.4.149}\flagstanza{\tiny\textenglish{....4.149}}स‚र्व‚साध‚न‚दोषेण प‚क्ष एवोप‚रुध्य‚ते ।&त‚थापि प‚क्ष‚दोष‚त्वं प्र‚तिज्ञामात्र‚भाविनः ॥ १४९ ॥\&[\smallbreak]


	
	    \end{quote}
	  
	  \endgroup
	

	  \pstart \leavevmode% starting standard par
	\hphantom{.}‚{\color{DodgerBlue3}‚स‚र्व्वेण साध‚न}‚स्य ‚{\color{DodgerBlue3}‚दोषे}‚णासिद्ध‚त्वादिना ‚{\color{DodgerBlue3}‚प‚क्ष एवोप‚रुध्य‚ते} तेन प‚क्ष‚दोषा ‚{\tiny $_{lb}$}‚असिद्ध‚विशेष्याद‚योऽभिधीय‚न्ते । य‚द्य‚पि प‚क्षोप‚रोध‚फ‚लाः स‚र्व्वे दोषास्त‚{\color{DodgerBlue3}‚थापि ‚{\tiny $_{lb}$}‚प्र‚{\tiny $_{5}$}‚तिज्ञामा}‚त्रेण \edtext{}{\edlabel{pvv.464-3}\label{pvv.464-3}\lemma{त्रेण}\Bfootnote{नोत्त‚र‚त्वेत्यादिदोषेण ।}} भ‚व‚न‚शील‚स्य दोष‚स्य ‚{\color{DodgerBlue3}‚प‚क्ष‚वोष‚त्व‚मु}‚क्त‚मिष्टं । (१४९)
	\pend% ending standard par
      \label{div_pvv.4.150}
	  
	% new div opening: depth here is 2
	

	  \pstart \leavevmode% starting standard par
	य‚स्मात् साक्षात् (।)
	\pend% ending standard par
      
	  \bigskip
	  \begingroup
	
	    \large
	  
	    \begin{quote}
	  
	    
	    \stanza[\smallbreak]
	\label{pv.4.150}\flagstanza{\tiny\textenglish{....4.150}}उत्त‚राव‚य‚वापेक्षो यो दोषः सोनुब‚ध्य‚ते ।&तेनेयुक्त‚म‚तोऽप‚क्ष‚दोषोऽसिद्धाश्र‚यादिकः ॥ १५० ॥\&[\smallbreak]


	
	    \end{quote}
	  
	  \endgroup
	

	  \pstart \leavevmode% starting standard par
	\hphantom{.}‚{\color{DodgerBlue3}‚उत्त‚रोऽव‚य‚वो} हेतुदृष्टान्तादिस्त‚द‚{\color{DodgerBlue3}‚पेक्षो यो दोषः स तेन} हेत्वादिनानु‚{\color{DodgerBlue3}‚ब‚ध्य‚ते} । ‚{\tiny $_{lb}$}‚आत्म‚नि स‚म्ब‚ध्य‚ते ‚{\color{DodgerBlue3}‚इत्युक्तं} प्राक् । उत्त‚राव‚य‚वापेक्षो न दोषः प‚क्ष इष्य‚त इत्या‚{\tiny $_{lb}$}‚‚{\color{DodgerBlue3}‚दिना । अतोऽसिद्धाश्र‚यादिक} आश्र‚यासिद्ध‚त्वादिरुत्त‚राव‚य‚वापेक्षो ‚{\color{DodgerBlue3}‚न प‚क्ष‚दोषो ‚{\tiny $_{lb}$}‚म‚तः} ॥ (१५०)
	\pend% ending standard par
      \label{div_pvv.4.151}
	  
	% new div opening: depth here is 2
	

	  \pstart \leavevmode% starting standard par
	न‚न्व‚श्राव‚णः श‚ब्दो नित्यो घ‚टः । नानुमानं प्र‚माणं ‚{\tiny $_{6}$}‚ (।) अच‚न्द्रः ‚{\tiny $_{lb}$}‚श‚शीत्युदाह‚र‚णैरेभिर्द्ध‚र्म‚स्व‚रूप\edtext{}{\edlabel{pvv.464-4}\label{pvv.464-4}\lemma{रूप}\Bfootnote{नित्यो घ‚ट इति प्र‚तिज्ञायां नित्य‚त्व‚स्य बाधित‚त्वात् ।}}निराक‚र‚णेन बाधा द‚र्शिता य‚था-प्र‚तिज्ञात‚ध‚र्म‚मात्र‚स्य ‚{\tiny $_{lb}$}‚विप‚रीत‚ध‚र्मोप‚स्थाप‚नेन निराक‚र‚णात् । ध‚र्मिविशेष‚स्य ध‚र्म‚विशेष‚स्य ध‚र्मिस्व‚रू‚{\tiny $_{lb}$}‚प‚स्य च बाध‚नेन प‚क्ष‚बाधास्ति \edtext{}{\edlabel{pvv.464-5}\label{pvv.464-5}\lemma{बाधास्ति}\Bfootnote{अव्याख्याता स मु च्च ये ।}} सा क‚थ‚म‚व‚ग‚न्त‚व्येत्याह ।
	\pend% ending standard par
      \textsuperscript{\textenglish{465/s}}
	  \bigskip
	  \begingroup
	
	    \large
	  
	    \begin{quote}
	  
	    
	    \stanza[\smallbreak]
	\label{pv.4.151}\flagstanza{\tiny\textenglish{....4.151}}ध‚र्मिध‚र्म‚विशेषाणां स्व‚रूप‚स्य च ध‚र्म्मिणः ।&बाधा साध्याङ्ग‚भूतानाम‚नेनैवोप‚द‚र्शिता ॥ १५१ ॥\&[\smallbreak]


	
	    \end{quote}
	  
	  \endgroup
	

	  \pstart \leavevmode% starting standard par
	\hphantom{.}‚{\color{DodgerBlue3}‚ध‚र्मिध‚र्म}‚यो‚{\color{DodgerBlue3}‚र्व्विशेषाणां} व्य‚क्तिभेदापेक्ष‚या ब‚हुव‚च‚नं । ‚{\color{DodgerBlue3}‚ध‚र्मिणः स्व‚रूप‚स्य च} स‚र्व्वेषामेषां ‚{\color{DodgerBlue3}‚साध्यं} प्र‚त्य‚{\color{DodgerBlue3}‚ङ्ग‚भूतानां बाधा । अनेनै‚{\tiny $_{7}$}‚व} ध‚र्म‚स्व‚रूप‚निराक‚र‚ण‚प‚रेणो-\leavevmode\ledsidenote{\textenglish{94a/MA}} दाह‚र‚णेन साध्य‚ते । प‚क्ष‚ल‚क्ष‚ण‚त्वाद्वाधो‚{\color{DodgerBlue3}‚प‚द‚र्शिता} (।) (१५१)
	\pend% ending standard par
      \label{div_pvv.4.152}
	  
	% new div opening: depth here is 2
	

	  \begin{center}%% label @type='head'
	\textbf{(२) वा/?/क्य‚विनिरास}
	\end{center}
	
	  \bigskip
	  \begingroup
	
	    \large
	  
	    \begin{quote}
	  
	    
	    \stanza[\smallbreak]
	\label{pv.4.152}\flagstanza{\tiny\textenglish{....4.152}}त‚त्रोदाहृतिदिङ् मात्र‚मुच्य‚तेऽर्थ‚स्य दृष्ट‚ये ।&द्र‚व्य‚ल‚क्ष‚ण‚युक्तोन्यः संयोगेर्थोस्ति दृष्टिभाक् ॥ १५२ ॥\&[\smallbreak]


	
	    \end{quote}
	  
	  \endgroup
	

	  \pstart \leavevmode% starting standard par
	\edtext{\textsuperscript{*}}{\edlabel{pvv.465-1}\label{pvv.465-1}\lemma{*}\Bfootnote{य‚द्येव‚मिद‚मेवोदाह‚र‚ण‚म‚स्तु किम‚र्थं नान्याऽव(य)व्य‚व‚य‚वेभ्य इत्यादिक‚मा ‚{\tiny $_{lb}$}‚चा र्ये णो क्त‚मित्याह ।}} ‚{\color{DodgerBlue3}‚त‚त्रा}‚श्राव‚णः श‚ब्द इत्यादिषू‚{\color{DodgerBlue3}‚दाह‚र‚ण‚दिङ‚मात्र‚मुच्य‚ते} । साध्याङ‚ग‚भू\edtext{}{\edlabel{pvv.465-2}\label{pvv.465-2}\lemma{भू}\Bfootnote{नान्त‚रीय‚क‚स्य ।}}त‚स्य ‚{\tiny $_{lb}$}‚स‚र्व्व‚स्यैव बाधा भ‚व‚तीत्य‚{\color{DodgerBlue3}‚र्थ‚स्य दृष्ट‚ये} द‚र्श‚नार्थ । त‚त्र प‚र‚स्याव‚य‚वेभ्योऽव‚य‚विनो ‚{\tiny $_{lb}$}‚गुरुत्वादिगुण‚योगिनोऽन्य‚त्वेऽभिम‚ते य‚दोच्य‚ते नान्योऽव‚य‚व्य‚व‚य‚वेभ्य‚स्तुलान‚तिविशे‚{\tiny $_{lb}$}‚षाग्र‚ह‚णादिति (।) एत‚द्ध‚र्म‚विशेष‚निराक‚र‚णेनोदा‚{\tiny $_{1}$}‚ह‚र‚णं बोद्ध‚व्यं । त‚थाहि नात्रान्य‚{\tiny $_{lb}$}‚त्व‚मात्रं निषेद्ध्ुमिष्टं । त‚थात्वे ध‚र्म‚स्व‚रूप‚निराक‚र‚णोदाह‚र‚ण‚मेत‚त् स्यात् । ‚{\tiny $_{lb}$}‚त‚स्माद‚न्य‚त्व‚स्य साध्य‚ध‚र्म‚स्य नान्त‚रीय‚का गौर‚वाद‚यो विशेषा निराक‚र्त्तुमिष्टाः । ‚{\tiny $_{lb}$}‚त‚था च ध‚र्म‚विशेषोदाह‚र‚ण‚मेव त‚त् । त‚थाहि प‚रैरेक‚स्याव‚य‚व‚स्यान्यान्याव‚य‚व‚{\color{DodgerBlue3}‚संयोगे} स‚ति त‚द‚न्योऽर्थोऽव‚य‚विसंज्ञितो ‚{\color{DodgerBlue3}‚द्र‚व्य‚स्य ल‚क्ष}‚णेन क्रियाव‚द् ‚{\color{DodgerBlue3}‚गुण‚व‚त्स‚म‚वायिकार‚ण‚ञ्चे‚{\tiny $_{2}$}‚} त्य‚नेन युक्तो ‚{\color{DodgerBlue3}‚दृष्टिभाक्} दृश्योस्तीतीष्ट‚मेव । (१५२)
	\pend% ending standard par
      \label{div_pvv.4.153}
	  
	% new div opening: depth here is 2
	

	  \pstart \leavevmode% starting standard par
	अन्य‚था (।)
	\pend% ending standard par
      
	  \bigskip
	  \begingroup
	
	    \large
	  
	    \begin{quote}
	  
	    
	    \stanza[\smallbreak]
	\label{pv.4.153}\flagstanza{\tiny\textenglish{....4.153}}अदृश्य‚स्या विशिष्ट‚स्य प्र‚तिज्ञा निष्प्र‚योज‚ना ।&इष्टो ह्य‚व‚य‚वी कार्यं दृष्ट्वाऽदृश्येष्व‚स‚म्भ‚वि ॥ १५३ ॥\&[\smallbreak]


	
	    \end{quote}
	  
	  \endgroup
	

	  \pstart \leavevmode% starting standard par
	\hphantom{.}‚{\color{DodgerBlue3}‚अदृश्य‚स्य} संव्य‚व‚हाराविष‚य‚स्य द्र‚व्य‚ल‚क्ष‚णेन ‚{\color{DodgerBlue3}‚विशिष्ट}‚सामान्य‚मात्र‚स्यास्तित्व‚{\tiny $_{lb}$}‚‚{\color{DodgerBlue3}‚प्र‚तिज्ञा निष्प्र‚योज‚ना} भ‚वेत् । त‚था हि प‚र‚माणुष्व‚दृश्येष्ब‚द‚र्श‚नाव‚र‚ण‚प्र‚तिघातादि‚{\tiny $_{lb}$}‚‚{\color{DodgerBlue3}‚कार्य‚म‚संभ‚वि} दृष्ट्वाऽ‚{\color{DodgerBlue3}‚व‚य‚वी चेष्टः\edtext{}{\edlabel{pvv.465-3}\label{pvv.465-3}\lemma{चेष्टः}\Bfootnote{वैशेषिकेण ।}}} । त‚स्मिन् द‚र्श‚नादिकार्य‚योगात् । (१५३)
	\pend% ending standard par
      \label{div_pvv.4.154}
	  
	% new div opening: depth here is 2
	
	  \bigskip
	  \begingroup
	
	    \large
	  
	    \begin{quote}
	  
	    
	    \stanza[\smallbreak]
	\label{pv.4.154}\flagstanza{\tiny\textenglish{....4.154}}अविशिष्ट‚स्य चान्य‚स्य साध‚ने सिद्ध‚साध‚न‚म् ।&गुरुत्वाधोग‚ती स्यातां य‚द्य‚स्य स्यात् तुलान‚तिः ॥ १५४ ॥\&[\smallbreak]


	
	    \end{quote}
	  
	  \endgroup
	\textsuperscript{\textenglish{466/s}}

	  \pstart \leavevmode% starting standard par
	य‚दि पुन‚र‚निन्द्रिय‚ग्राह्य‚त्वं द्र‚व्य‚ल‚क्ष‚णेनाविशिष्ट‚न्त‚द‚व‚य‚वि द्र‚व्यं साध्य‚ते त‚दा‚{\tiny $_{3}$}‚‚{\tiny $_{lb}$}‚‚{\color{DodgerBlue3}‚ऽविशिष्ट}‚स्यान्य‚स्य च ‚{\color{DodgerBlue3}‚साध‚ने} बौद्ध‚स्य न काचित् क्ष‚तिरिति ‚{\color{DodgerBlue3}‚सिद्ध‚साध‚नं} स्यात् । ‚{\tiny $_{lb}$}‚त‚स्माद् दृश्यो द्र‚व्य‚गुण‚वान् भावोऽव‚य‚वीति साध्यं । त‚त‚श्चास्याव‚य‚विनो ‚{\color{DodgerBlue3}‚गुरुत्वं ‚{\tiny $_{lb}$}‚गुणोऽधोग‚तिश्च क‚र्म य‚दि स्यातां} त‚दा मृदादिख‚ण्ड‚योः स‚ह‚तोलित‚योर्याव‚ती ‚{\color{DodgerBlue3}‚तुला‚{\tiny $_{lb}$}‚न‚ति}‚गौर‚व‚व‚शाद् दृष्टा त‚तोधिका तुलान‚तिः स्यात् । य‚दा त‚योर्मृदादिख‚ण्ड‚योः ‚{\tiny $_{lb}$}‚संयोगे स‚ति द्र‚व्यान्त‚र‚मुत्प‚द्य‚{\tiny $_{4}$}‚ ते त‚दा त‚योः पूर्व्वाव‚स्थित‚योः पूर्व्वाव‚स्थितं गौर‚वं ‚{\tiny $_{lb}$}‚त‚दोत्प‚न्न‚स्य च द्र‚व्य‚स्याधिक‚गौर‚व‚विशेषात् तुलान‚तिविश‚षो दृश्येत\edtext{}{\edlabel{pvv.466-1}\label{pvv.466-1}\lemma{दृश्येत}\Bfootnote{प‚र‚माण‚वोऽदृश्या नोद‚काह‚र‚ण‚क्ष‚मा अतोव‚य‚वी वै शे षि क‚स्वीकृतः । स ध‚र्मी ‚{\tiny $_{lb}$}‚अन्य‚त्वं साध्यं त‚त्र नान्योव‚य‚वीत्युक्तं मृत्पिण्ड‚योस्तोल्य‚योस्तृतीय‚क्षेपे गौर‚वा‚{\tiny $_{lb}$}‚न्य‚त्व‚व‚त् तृतीयाव‚य‚विनि नेति ।}}। न चैवं ‚{\tiny $_{lb}$}‚त‚स्मान्न त‚त्र कार्य‚द्र‚व्य‚स‚म्भ‚व इत्य‚व‚य‚वी निर्गुणो निष्क्रिय‚श्च स्यात् (। १५४)
	\pend% ending standard par
      \label{div_pvv.4.155}
	  
	% new div opening: depth here is 2
	
	  \bigskip
	  \begingroup
	
	    \large
	  
	    \begin{quote}
	  
	    
	    \stanza[\smallbreak]
	\label{pv.4.155}\flagstanza{\tiny\textenglish{....4.155}}त‚न्निर्गुण‚क्रिय‚स्त‚स्मात् स‚म‚वायि न कार‚ण‚म् ।&त‚त एव न दृश्योसाव‚दृष्टेः कार्य‚रूप‚योः ॥ १५५ ॥\&[\smallbreak]


	
	    \end{quote}
	  
	  \endgroup
	

	  \pstart \leavevmode% starting standard par
	\hphantom{.}‚{\color{DodgerBlue3}‚त‚स्मान्निर्गुण}‚क्रिय‚त्वात् गुण‚क‚र्म‚णोर्न ‚{\color{DodgerBlue3}‚स‚म‚वायि कार‚ण}‚म‚व‚य‚वी । ‚{\color{DodgerBlue3}‚त‚त एव} द्र‚व्य‚ल‚क्ष‚ण‚योगान्न त‚स्याव‚र‚णादि ‚{\color{DodgerBlue3}‚कार्यं स्व‚रूप}‚म्वा किञ्चित् दृश्य‚ते (।) अद‚र्श‚{\tiny $_{lb}$}‚‚{\color{DodgerBlue3}‚नाच्च न} दृश्योऽव‚य‚वी (। १५५)
	\pend% ending standard par
      \label{div_pvv.4.156_4.157_4.158}
	  
	% new div opening: depth here is 2
	
	  \bigskip
	  \begingroup
	
	    \large
	  
	    \begin{quote}
	  
	    
	    \stanza[\smallbreak]
	\label{pv.4.156a}\flagstanza{\tiny\textenglish{...4.156a}}[त‚द्बाधान्य‚विशेष‚स्य नान्त‚रीय‚क‚भाविनः ।\&[\smallbreak]


	
	    \end{quote}
	  
	  \endgroup
	

	  \pstart \leavevmode% starting standard par
	\hphantom{.}‚{\color{DodgerBlue3}‚त‚त्} त‚स्माद‚साव‚व‚य‚विनो ‚{\color{DodgerBlue3}‚बाधाऽन्य}‚स्य \edtext{}{\edlabel{pvv.466-2}\label{pvv.466-2}\lemma{स्य}\Bfootnote{कीदृश‚स्येत्याह ।}} ‚{\color{DodgerBlue3}‚विशेष}‚स्य गौर‚वाधोग‚त्यादे‚{\color{DodgerBlue3}‚र्ना‚{\tiny $_{lb}$}‚न्त‚रीय‚क‚भाविनः} साध्या\edtext{}{\edlabel{pvv.466-3}\label{pvv.466-3}\lemma{साध्या}\Bfootnote{य‚त्साद्य‚म‚न्य‚त्व‚न्त‚द‚विना ।}}न्य‚त्वाविनाभाविनः । बाधेति ध‚र्म‚विशेष‚निराक‚र‚णे ‚{\tiny $_{lb}$}‚निर्देशो युक्तः ।
	\pend% ending standard par
      
	  \bigskip
	  \begingroup
	
	    \large
	  
	    \begin{quote}
	  
	    
	    \stanza[\smallbreak]
	\label{pv.4.156b}\flagstanza{\tiny\textenglish{...4.156b}}आसूक्ष्माद् द्र‚व्य‚मालायास्तोल्य‚त्वादंशुपात‚व‚त् ॥ १५६ ॥\&[\smallbreak]


	
	    \end{quote}
	  
	  \endgroup
	
	  \bigskip
	  \begingroup
	
	    \large
	  
	    \begin{quote}
	  
	    
	    \stanza[\smallbreak]
	\label{pv.4.157a}\flagstanza{\tiny\textenglish{...4.157a}}द्र‚व्यान्त‚र‚गुरुत्व‚स्य ग‚तिर्नेत्य‚प‚रोऽब्र‚वीत् ।\&[\smallbreak]


	
	    \end{quote}
	  
	  \endgroup
	

	  \pstart \leavevmode% starting standard par
	\hphantom{.}‚{\color{DodgerBlue3}‚आसूक्ष्माद्} द्व्य‚णुका\edtext{}{\edlabel{pvv.466-4}\label{pvv.466-4}\lemma{णुका}\Bfootnote{त्र्य‚णुकादौ म‚ह‚त्वोक्तेः ।}}दार‚भ्य ‚{\color{DodgerBlue3}‚द्र‚व्य‚मालायाः} ‚{\tiny $_{5}$}‚ स्थूलाव‚य‚विप‚र्य‚न्तायास्तो‚{\tiny $_{lb}$}‚‚{\color{DodgerBlue3}‚ल्य‚त्वा}‚त् त‚द्द‚द्र‚व्य‚मालाव‚र्त्तिनः स्थूल‚स्य ‚{\color{DodgerBlue3}‚द्र‚व्यान्त‚र}‚स्य ‚{\color{DodgerBlue3}‚न ग‚ति}‚र्भ‚व‚ति । तुलाया‚{\tiny $_{lb}$}‚‚{\color{DodgerBlue3}‚मंशुपात‚व‚त्} । क‚र्पास‚भार‚प‚तित‚स्यांशोरेक‚स्य य‚था गुरुत्वं स‚द‚पि न प्र‚तीय‚ते (।) ‚{\tiny $_{lb}$}‚त‚था द्र‚व्य‚मालाव‚र्त्तिनः स्थूल‚द्र‚व्य‚स्य ‚{\color{DodgerBlue3}‚इत्य‚प‚रो\edtext{}{\edlabel{pvv.466-5}\label{pvv.466-5}\lemma{रो}\Bfootnote{उ द्यो त क रा दिः ।}}ऽब्र‚वीत्} ।
	\pend% ending standard par
      \textsuperscript{\textenglish{467/s}}
	  \bigskip
	  \begingroup
	
	    \large
	  
	    \begin{quote}
	  
	    
	    \stanza[\smallbreak]
	\label{pv.4.157b}\flagstanza{\tiny\textenglish{...4.157b}}त‚स्य क्र‚मेण संयुक्ते पांशुराशौ स‚कृद् युते ॥ १५७ ॥\&[\smallbreak]


	
	    \end{quote}
	  
	  \endgroup
	
	  \bigskip
	  \begingroup
	
	    \large
	  
	    \begin{quote}
	  
	    
	    \stanza[\smallbreak]
	\label{pv.4.158}\flagstanza{\tiny\textenglish{....4.158}}भेदः स्याद् गौर‚वे त‚स्मात् पृथ‚क् स‚ह च तोलिते ।&सुव‚र्ण‚माष‚कादीनां संख्यासाम्यं न युज्य‚ते ॥ १५८ ॥\&[\smallbreak]


	
	    \end{quote}
	  
	  \endgroup
	

	  \pstart \leavevmode% starting standard par
	\hphantom{.}‚{\color{DodgerBlue3}‚त‚स्यै}‚वंवादिनो म‚ते ‚{\color{DodgerBlue3}‚क्र‚मेण} सूक्ष्माव‚य‚व‚संयोगाभिवृद्धिप‚रिपाट्या ‚{\color{DodgerBlue3}‚संयुक्ते} स्थूलाव‚य‚वितां ग‚ते ‚{\color{DodgerBlue3}‚पांशुराशा}‚व‚व‚य‚व‚संयोगाभिवृद्धिक्र‚म‚म‚न‚पेक्ष्य ‚{\tiny $_{6}$}‚ ‚{\color{DodgerBlue3}‚स‚कृदे}‚क‚कालं ‚{\tiny $_{lb}$}‚‚{\color{DodgerBlue3}‚युक्ते} स्थूलाव‚य‚वितां ‚{\color{DodgerBlue3}‚ग‚ते भेदो गौर‚वे स्यात्} । त‚था हि द्व्य‚णुकादिक्र‚मेण कार्य‚{\tiny $_{lb}$}‚द्र‚व्य‚संयोग‚प‚रंप‚र‚या च द्र‚व्य‚मुत्प‚द्य‚ते । त‚त्रानेक‚द्र‚व्य‚भार‚स‚द्भावात् म‚ह‚द्गौर‚वं ‚{\tiny $_{lb}$}‚भ‚वेत् । य‚त्र त्वेक एव स‚कृत् पांशुराशिः संयोगाज्जाय‚ते त‚त्रैक‚स्य द्र‚व्य‚स्याल्पीयो ‚{\tiny $_{lb}$}‚गौर‚वं भ‚व‚ति न चास्त्येत‚त् । किञ्च (।) ‚{\color{DodgerBlue3}‚त‚स्माद्} गौर‚व‚भेदात् ‚{\color{DodgerBlue3}‚पृथ‚क्} प्र‚त्येकं माष‚{\tiny $_{lb}$}‚कादौ ‚{\color{DodgerBlue3}‚तोलिते} पिण्डा‚{\tiny $_{7}$}‚व‚स्थ‚योः ‚{\color{DodgerBlue3}‚स‚ह चा}‚प‚रैर्मास (? ष) कादिभिस्तोलिते सुव‚र्ण्ण‚मास-\leavevmode\ledsidenote{\textenglish{94b/MA}} कादिभिन्नायाः संख्यायाः ‚{\color{DodgerBlue3}‚साम्यं \edtext{}{\edlabel{pvv.467-1}\label{pvv.467-1}\lemma{साम्यं}\Bfootnote{र‚क्तिक‚श‚तेऽव‚य‚विराशेः क‚थ‚मेकाव‚य‚विना साम्यं । एकानेक‚प‚ल‚पिण्ड‚व‚त् ।}} न युज्य‚ते} (।) मास‚काव‚य‚विनां विनाशे ‚{\tiny $_{lb}$}‚त‚त्संख्यानां गौर‚वाणां च नाशादेकं सुव‚र्ण्ण‚मित्येव स्यात् । दृश्य‚ते च प्र‚त्येकं ‚{\tiny $_{lb}$}‚माष‚कादीनां तुलायां याव‚ती संख्या ताव‚त्येव स‚ह‚तोलितानाम‚पि ॥ (१५७, १५८)
	\pend% ending standard par
      \label{div_pvv.4.159}
	  
	% new div opening: depth here is 2
	
	  \bigskip
	  \begingroup
	
	    \large
	  
	    \begin{quote}
	  
	    
	    \stanza[\smallbreak]
	\label{pv.4.159}\flagstanza{\tiny\textenglish{....4.159}}स‚र्ष‚पादाम‚हाराशेरुत्त‚रोत्त‚र‚वृद्धिम‚त् ।&गौर‚वं कार्य‚मालाया य‚दि नैवोप‚ल‚भ्य‚ते ॥ १५९ ॥\&[\smallbreak]


	
	    \end{quote}
	  
	  \endgroup
	

	  \pstart \leavevmode% starting standard par
	अथ\edtext{}{\edlabel{pvv.467-2}\label{pvv.467-2}\lemma{अथ}\Bfootnote{मा भूत् संख्यावैष‚म्य‚मिति ।}} र‚क्तिकायाश्च‚तुर्थो भागः ‚{\color{DodgerBlue3}‚स‚र्ष‚प}‚स्त‚स्मादार‚भ्य ‚{\color{DodgerBlue3}‚आम‚हाराशेः} स्थूलाव‚{\tiny $_{lb}$}‚य‚विनं याव‚त् क्र‚म‚वृद्धिम‚तां ‚{\color{DodgerBlue3}‚का1र्याणां मालाया उत्त‚रोत्त‚र‚वृद्धिम‚द् गौर‚वं स‚द‚पि} नोप‚ल‚भ्य‚त इति य‚दुच्य‚ते (। १५९) त‚दा (।)
	\pend% ending standard par
      \label{div_pvv.4.160}
	  
	% new div opening: depth here is 2
	
	  \bigskip
	  \begingroup
	
	    \large
	  
	    \begin{quote}
	  
	    
	    \stanza[\smallbreak]
	\label{pv.4.160}\flagstanza{\tiny\textenglish{....4.160}}आस‚र्ष‚पाद् गौर‚व‚न्तु दुर्ल‚क्ष‚त‚म‚न‚ल्प‚क‚म् ।&तोल्यं त‚त्कार‚णं कार्य‚गौर‚वानुप‚ल‚क्ष‚णात् ॥ १६० ॥\&[\smallbreak]


	
	    \end{quote}
	  
	  \endgroup
	

	  \pstart \leavevmode% starting standard par
	\hphantom{.}‚{\color{DodgerBlue3}‚आस‚र्ष‚पात्} स‚र्ष‚पादार‚भ्य द्ब्य‚णुकं याव‚त् पूर्व्वं पूर्व्वं ‚{\color{DodgerBlue3}‚गौर‚वं} त‚त् ‚{\color{DodgerBlue3}‚दुर्ल‚क्ष‚त‚म‚{\tiny $_{lb}$}‚न‚ल्प‚क‚मिति} सुत‚राम‚नुप‚ल‚भ्यं स्यात् । स‚र्ष‚पादुत्त‚र‚न्तु गौर‚वं स्व‚य‚मेव दुर्ल्ल‚क्ष‚{\tiny $_{lb}$}‚मिष्ट‚मिति कार्य‚द्र‚व्य‚गौर‚वानुप‚ल‚क्ष‚णात् त‚स्य कार्य‚द्र‚व्य‚स्य ‚{\color{DodgerBlue3}‚कार‚णं} प‚र‚माण‚वः ‚{\tiny $_{lb}$}‚पारिशेष्यात् ‚{\color{DodgerBlue3}‚तोल्यं} स्यात् । न च प‚र‚माणूनां\edtext{}{\edlabel{pvv.467-3}\label{pvv.467-3}\lemma{माणूनां}\Bfootnote{त‚द‚नुप‚ल‚म्भादेवाव‚य‚विक‚ल्प‚नात् ।}} गु‚{\tiny $_{2}$}‚रुत्वादिगुणोप‚ल‚म्भोस्ति ‚{\color{DodgerBlue3}‚त‚द‚{\tiny $_{lb}$}‚नुप‚ल‚क्ष‚णेन} च नाव‚य‚विनः प‚र‚माणूनाम्वोप‚ल‚म्भोस्तीति स‚र्व्व‚थार्थानाम‚प्र‚तिप‚तिः ‚{\tiny $_{lb}$}‚स्यात् । (१६०)
	\pend% ending standard par
      \label{div_pvv.4.161}
	  
	% new div opening: depth here is 2
	
	  \bigskip
	  \begingroup
	
	    \large
	  
	    \begin{quote}
	  
	    
	    \stanza[\smallbreak]
	\label{pv.4.161a}\flagstanza{\tiny\textenglish{...4.161a}}न‚न्व‚दृष्टोंऽशुव‚त् सोर्थो न च त‚त्कार्य‚मीक्ष्य‚ते ।\&[\smallbreak]


	
	    \end{quote}
	  
	  \endgroup
	\textsuperscript{\textenglish{468/s}}

	  \pstart \leavevmode% starting standard par
	\hphantom{.}‚{\color{DodgerBlue3}‚न‚नु} गुरुत्वाप्र‚तीताव‚पि अंशुः प्र‚तीय‚ते त‚द्व‚द‚व‚य‚व्यादिक‚म‚पि प्र‚त्येष्य‚त इत्याह । ‚{\tiny $_{lb}$}‚न च गुरुत्वानुप‚ल‚क्ष‚णे‚{\color{DodgerBlue3}‚प्यंशु}‚रिव ‚{\color{DodgerBlue3}‚सोऽर्थो}‚ऽव‚य‚व्यादिरीक्ष्य‚ते । ‚{\color{DodgerBlue3}‚न च त‚त्कार्यं} गौर‚वा‚{\tiny $_{lb}$}‚‚{\color{DodgerBlue3}‚व‚र‚णादिरीक्ष्य‚ते} । त‚स्माद‚प्र‚त्य‚क्ष‚तैव स‚र्व्व‚था स्यात् ।
	\pend% ending standard par
      

	  \pstart \leavevmode% starting standard par
	किञ्च (।)
	\pend% ending standard par
      
	  \bigskip
	  \begingroup
	
	    \large
	  
	    \begin{quote}
	  
	    
	    \stanza[\smallbreak]
	\label{pv.4.161b}\flagstanza{\tiny\textenglish{...4.161b}}गुरुत्वाग‚तिव‚त् स‚र्व‚त‚द्गुणानुप‚ल‚क्षाणात् ॥ १६१ ॥\&[\smallbreak]


	
	    \end{quote}
	  
	  \endgroup
	
	  \bigskip
	  \begingroup
	
	    \large
	  
	    \begin{quote}
	  
	    
	    \stanza[\smallbreak]
	\label{pv.4.162a}\flagstanza{\tiny\textenglish{...4.162a}}माष‚कादेर‚नाधिक्य‚म्;\&[\smallbreak]


	
	    \end{quote}
	  
	  \endgroup
	

	  \pstart \leavevmode% starting standard par
	\hphantom{.}‚{\color{DodgerBlue3}‚गुरु‚{\tiny $_{3}$}‚त्वा}‚दिगुणा‚{\color{DodgerBlue3}‚ग‚तिव‚त् स‚र्व्वे}‚षां रूपादीनाम‚धिकानां त‚स्य द्र‚व्य‚स्य ‚{\color{DodgerBlue3}‚गुणाना}‚म‚{\color{DodgerBlue3}‚नु‚{\tiny $_{lb}$}‚प‚ल‚क्ष‚णाद}‚व‚य‚वेभ्यो ‚{\color{DodgerBlue3}‚भाष‚कादे}‚र‚व‚य‚विनोर‚{\color{DodgerBlue3}‚नाधिक्यं} । य‚दि ह्य‚व‚य‚वेभ्योधिकं त‚दा ‚{\tiny $_{lb}$}‚गुरुत्वादिव‚त् रूपादिव‚त् अव‚य‚वेषु व‚र्द्ध‚मानेषु व‚र्द्ध‚मानं दृश्येत । (१६१)
	\pend% ending standard par
      \label{div_pvv.4.162}
	  
	% new div opening: depth here is 2
	

	  \pstart \leavevmode% starting standard par
	न‚नु तुलान‚तिविशेषाग्र‚ह‚णादित्युक्त‚मा चा र्येण त‚त्किं रूपादिग्र‚ह‚णाभाव ‚{\tiny $_{lb}$}‚उच्य‚त इत्याह ।
	\pend% ending standard par
      
	  \bigskip
	  \begingroup
	
	    \large
	  
	    \begin{quote}
	  
	    
	    \stanza[\smallbreak]
	\label{pv.4.162b}\flagstanza{\tiny\textenglish{...4.162b}}अन‚तिः सोप‚ल‚क्ष‚ण‚म् ।&य‚थास्व‚म‚क्षेणादृष्टे रूपादाव‚धिकाधिके ॥ १६२ ॥\&[\smallbreak]


	
	    \end{quote}
	  
	  \endgroup
	

	  \pstart \leavevmode% starting standard par
	\hphantom{.}‚{\color{DodgerBlue3}‚अन‚ति}‚राचार्येण या निर्द्दिष्टा ‚{\color{DodgerBlue3}‚सोप‚ल‚क्ष‚ण}‚मात्रं न तु निय‚मः । अन‚तिः क्वोप‚{\tiny $_{lb}$}‚ल‚क्ष‚ण‚मित्याह । ‚{\color{DodgerBlue3}‚रूपादौ} द्र‚व्याभिवृद्ध्या‚{\color{DodgerBlue3}‚ऽधिकाधिके य‚थास्वं} य‚स्य य‚दात्मीयं ग्राह‚कं ‚{\tiny $_{lb}$}‚‚{\color{DodgerBlue3}‚तेनाक्षेणे}‚न्द्रिय‚ज्ञानेना‚{\color{DodgerBlue3}‚दृष्टेः} \edtext{\textsuperscript{*}}{\edlabel{pvv.468-1}\label{pvv.468-1}\lemma{*}\Bfootnote{अनुप‚ल‚ब्धिसाध‚न‚मुक्तं ।}}। य‚द्रूपादिविष‚यं य‚दिन्द्रिय‚प्र‚त्य‚क्षं त‚देव त‚स्य ‚{\tiny $_{lb}$}‚ध‚र्मिविशेष‚स्य प‚क्ष‚बाध‚क‚मित्य‚र्थः । त‚स्माद् गुण‚क्रियाव‚त् दृश्याव‚य‚व्य‚भाव एवेति ‚{\tiny $_{lb}$}‚ध‚र्म‚विशेष‚वाग्द्वारेणोदाह‚र‚ण‚मुक्तं ।‚{\tiny $_{5}$}‚ (१६२)
	\pend% ending standard par
      \label{div_pvv.4.163}
	  
	% new div opening: depth here is 2
	
	  \bigskip
	  \begingroup
	
	    \large
	  
	    \begin{quote}
	  
	    
	    \stanza[\smallbreak]
	\label{pv.4.163}\flagstanza{\tiny\textenglish{....4.163}}अभ्युपाय‚स्व‚वागाद्य‚बाधायाः संभ‚वेन तु ।&उदाह‚र‚ण‚म‚प्य‚न्य‚द्दिशा ग‚म्यं य‚थोक्त‚या ॥ १६३ ॥\&[\smallbreak]


	
	    \end{quote}
	  
	  \endgroup
	

	  \pstart \leavevmode% starting standard par
	\hphantom{.}‚{\color{DodgerBlue3}‚अभ्युपायो}‚ऽभ्युप‚ग‚मः ‚{\color{DodgerBlue3}‚स्व‚वागादि\edtext{}{\edlabel{pvv.468-2}\label{pvv.468-2}\lemma{वागादि}\Bfootnote{आदिना प्र‚त्य‚क्षानुमान‚प्र‚तीतिसंभ‚व‚ग्र‚हः ।}}}‚र्य‚स्य तेना‚{\color{DodgerBlue3}‚बाधायाः संभ‚वेन त्व}‚न्य‚द‚प्यु‚{\tiny $_{lb}$}‚दाह‚र‚णं । ध‚र्मिविशेष‚निराक‚र‚णेन ध‚र्मिस्व‚रूप‚निराक‚र‚णेन ‚{\color{DodgerBlue3}‚य‚थोक्त‚या}‚ऽन‚योदा‚{\tiny $_{lb}$}‚‚{\color{DodgerBlue3}‚ह‚र‚ण‚दिशा ग‚म्यं} ॥ (१६३)
	\pend% ending standard par
      \label{div_pvv.4.164}
	  
	% new div opening: depth here is 2
	

	  \pstart \leavevmode% starting standard par
	त‚त्र \edtext{}{\edlabel{pvv.468-3}\label{pvv.468-3}\lemma{त्र}\Bfootnote{त‚दाचार्योक्त‚माह ।}} प‚रेणाव‚य‚विनः स‚काशाद‚व‚य‚वानाम‚न्य‚त्वे प्र‚तिज्ञाते य‚दुच्य‚ते (।) ‚{\tiny $_{lb}$}‚नान्येऽव‚य‚वा अव‚य‚विनः अप्र‚त्य‚क्ष‚त्व‚प्र‚स‚ङ्गा\edtext{}{\edlabel{pvv.468-4}\label{pvv.468-4}\lemma{ङ्गा}\Bfootnote{अव‚य‚वा ध‚र्मिणः अन्य‚त्वं साध्यं प‚रेण त‚न्म‚ध्ये केषाञ्चित् प्र‚त्य‚क्ष‚त्व‚विशेषो‚{\tiny $_{lb}$}‚भिम‚त‚स्तेन । स चान्य‚त्वेनुप‚प‚न्नं त‚स्य भेदेनाभास‚नात् । य‚द्भेदेन य‚तो न भाति ‚{\tiny $_{lb}$}‚न स्व‚प्र‚त्य‚क्षो । य‚था प्र‚थ‚म‚स्याम‚ह‚तोव‚य‚विनोऽव‚य‚वाः । न भान्ति चान्य‚स्याप्य‚{\tiny $_{lb}$}‚व‚य‚विनोऽव‚य‚वाभेदेनेति व्याप‚काभावः ।}}दिति । त‚द्ध‚र्मिविशेष‚निराक‚र‚णो‚{\tiny $_{lb}$}‚\leavevmode\ledsidenote{\textenglish{469/s}} दाह‚र‚णं । त‚था ह्य‚व‚य‚वा‚{\tiny $_{6}$}‚नां भेद‚मिच्छ‚न् प्र‚त्य‚क्ष‚ताम‚पीच्छ‚ति । अन्य‚त्वे च ‚{\tiny $_{lb}$}‚निराकृते प्र‚त्य‚क्ष‚तायाश्च निरासात् । अव‚य‚वानां ध‚र्मिविशेष‚निराक‚र‚णोदाह‚र‚{\tiny $_{lb}$}‚ण‚त्वं व्य‚क्तं । अभ्युप‚ग‚म एव चात्र बाध‚कः । अव‚य‚वाद‚र्श‚ने द्र‚व्याद‚र्श‚न‚स्वीकारात् ॥ ‚{\tiny $_{lb}$}‚गुण‚व्य‚तिरिक्तं द्र‚व्य‚म‚स्तीति प‚रेणोक्ते य‚दोच्य‚ते ‚{\color{DodgerBlue3}‚नास्ति द्र‚व्यं गुण‚द्र‚व्याणां द्र‚व्या-} द्र‚व्य‚त्व‚प्र‚स\edtext{}{\edlabel{pvv.469-1}\label{pvv.469-1}\lemma{स}\Bfootnote{द्र‚व्यं पृथिव्यादि गुणो गुरुत्वादि ।}}ङ्गात् । त‚द्ध‚र्मिस्व‚रूप‚निराक‚र‚णोदाह‚र‚णं ‚{\tiny $_{7}$}‚ (।)
	\pend% ending standard par
      

	  \pstart \leavevmode% starting standard par
	\edtext{\textsuperscript{*}}{\edlabel{pvv.469-2}\label{pvv.469-2}\lemma{*}\Bfootnote{स‚त्तायोगात् क ण भु जो द्र‚व्यादित्र‚यं स‚त् । प‚दार्थ‚स‚त्क‚री स‚त्तेति व‚च‚{\tiny $_{lb}$}‚नात् । अतो गुणानाम‚पि द्र‚व्य‚ताप‚त्तिः । दृष्ट‚स्थित‚स‚त्वेन स‚म्ब‚न्धात् । य‚द् द्र‚व्य‚{\tiny $_{lb}$}‚स‚म‚वायिस्व‚भावं स‚त्त्वं त‚न्न गुण इति एक‚त्व‚हानिः सा च नेष्टेति गुणेपि द्र‚व्य‚स्थित‚{\tiny $_{lb}$}‚रूप‚स्यैव वृत्तिरिति द्र‚व्य‚त्व‚प्र‚स‚ङ्गः इति दृष्टान्त‚सूच‚नं । य‚त् स‚त्ताव‚न्न त‚द् द्र‚व्यं ‚{\tiny $_{lb}$}‚य‚था गुणः । स‚त्ताव‚च्च द्र‚व्य‚मिति विरुद्ध‚व्याप्तेन ध‚र्मिस्व‚रूप‚निकार‚णात् प्र‚तिज्ञा‚{\tiny $_{lb}$}‚दोष एवं गुणेपि योज्यं ।}} त‚थाहि ध‚र्मिण एव द्र‚व्य‚स्य स्व‚रूप‚मात्रं निराक्रिय‚ते (।) गुण‚द्र‚व्याणाम\leavevmode\ledsidenote{\textenglish{95a/MA}}‚{\tiny $_{lb}$}‚न्योन्यं भेदः (स्वीकृतः) गुणोपि द्र‚व्यं स्यात् । द्र‚व्य‚ञ्च गुणः । भेदाविशेषा‚{\tiny $_{lb}$}‚दित्य‚भ्युपाय‚स्य बाध‚क‚त्वं ।
	\pend% ending standard par
      

	  \begin{center}%% label @type='head'
	\textbf{(३) नेयायिक‚प‚क्ष‚ल‚क्ष‚णे दोषः}
	\end{center}
	

	  \pstart \leavevmode% starting standard par
	न‚नु \edtext{}{\edlabel{pvv.469-3}\label{pvv.469-3}\lemma{नु}\Bfootnote{व्याख्येयान्त‚र्ग‚त‚त्वेन प‚र‚म‚त‚माह ।}} साध्य\edtext{}{\edlabel{pvv.469-4}\label{pvv.469-4}\lemma{साध्य}\Bfootnote{सिद्ध‚हेत्वादिनिवृत्त्य‚र्थः ।}}निर्देशः प्र‚तिज्ञेति प‚क्ष‚ल‚क्ष‚णं नै या यि का नां त‚त्र को दोषः । ‚{\tiny $_{lb}$}‚असिद्ध‚हेतुदृष्टान्त‚स्यापि प‚क्ष‚त्व‚प्र‚स‚ङ्ग इत्युक्तं । न‚नु साध्य‚त इति साध्यं । हेतु‚{\tiny $_{lb}$}‚दृष्टान्तौ तु साध‚यिष्य‚ते त‚तो नान‚योः प‚क्ष‚त्व‚प्र‚स‚ङ्ग इत्याह ।
	\pend% ending standard par
      
	  \bigskip
	  \begingroup
	
	    \large
	  
	    \begin{quote}
	  
	    
	    \stanza[\smallbreak]
	\label{pv.4.164}\flagstanza{\tiny\textenglish{....4.164}}त्रिकाल‚विष‚य‚त्वात्तु कृत्यानाम‚त‚थात्म‚क‚म् ।&त‚था प‚रं प्र‚ति न्य‚स्तं साध्यं नेष्टं त‚दापि त‚त् ॥ १६४ ॥\&[\smallbreak]


	
	    \end{quote}
	  
	  \endgroup
	

	  \pstart \leavevmode% starting standard par
	कृत्यानां प्र‚त्य‚यानां काल‚सामान्य‚वि‚{\tiny $_{1}$}‚हित‚त्वेन ‚{\color{DodgerBlue3}‚त्रिकाल‚विष‚य‚त्वात् साध्य}‚{\tiny $_{lb}$}‚श‚ब्देन ‚{\color{DodgerBlue3}‚कृत्या}‚न्तेन न साध्य‚मात्र‚स्य ग्र‚ह‚णं साध‚यिष्य‚माण‚स्यापि ग्र‚ह‚णात् । त‚था च ‚{\tiny $_{lb}$}‚नित्यः श‚ब्दो मूर्त‚त्वात् बुद्धिव‚दित्या\edtext{}{\edlabel{pvv.469-5}\label{pvv.469-5}\lemma{दित्या}\Bfootnote{सिद्ध‚हेत्वादिनिवृत्त्य‚र्थं ।}}दि प्र‚योगे त‚दा वाद\edtext{}{\edlabel{pvv.469-6}\label{pvv.469-6}\lemma{वाद}\Bfootnote{नित्य‚त्व‚साध‚क ।}}काले त‚द्धेतुदृष्टान्ता‚{\tiny $_{lb}$}‚दिक‚म‚{\color{DodgerBlue3}‚त‚थात्म‚कं} व‚स्तु\edtext{}{\edlabel{pvv.469-7}\label{pvv.469-7}\lemma{स्तु}\Bfootnote{नित्ये साध्येऽनित्यं ।}}तोऽत‚त्स्व‚भावात्म‚कं (।) ‚{\color{DodgerBlue3}‚प‚रं प्र‚ति त‚था}‚ऽत‚द्रूप‚त्वेन ‚{\color{DodgerBlue3}‚न्य‚स्त}‚{\tiny $_{lb}$}‚\leavevmode\ledsidenote{\textenglish{470/s}} ‚{\color{DodgerBlue3}‚मुप‚न्य‚स्तं} वादिना य‚त्न‚तः ‚{\color{DodgerBlue3}‚साध्यं} । य‚द्य‚पि श‚ब्दे मूर्त्त‚त्वं नास्ति (‚{\color{DodgerBlue3}‚अनिष्ट}‚ञ्च साध्य‚{\tiny $_{lb}$}‚त्वेन) त‚थापि हेतुदृष्टान्त‚योरुप‚न्यासाद‚व‚श्यं साध्यं । साध्य‚नि‚{\tiny $_{2}$}‚र्देश‚श्च प्र‚तिज्ञेति ‚{\tiny $_{lb}$}‚प्र‚तिज्ञात्व‚ञ्च दुर्व्वारं । (१६४)
	\pend% ending standard par
      \label{div_pvv.4.165}
	  
	% new div opening: depth here is 2
	

	  \begin{center}%% label @type='head'
	\textbf{(साध्य‚श‚ब्द‚चिन्ता)}
	\end{center}
	
	  \bigskip
	  \begingroup
	
	    \large
	  
	    \begin{quote}
	  
	    
	    \stanza[\smallbreak]
	\label{pv.4.165}\flagstanza{\tiny\textenglish{....4.165}}प्र‚त्याय‚नाधिकारे तु स‚र्वासिद्धाव‚रोधिनी ।&य‚स्मात् साध्य‚श्रुतिर्नेष्टं विशेष‚म‚व‚ल‚म्ब‚ते ॥ १६५ ॥\&[\smallbreak]


	
	    \end{quote}
	  
	  \endgroup
	

	  \pstart \leavevmode% starting standard par
	\hphantom{.}‚{\color{DodgerBlue3}‚प्र‚त्याय‚न}‚स्य ज्ञाप‚क‚स्य हेतो‚{\color{DodgerBlue3}‚र‚धिकारे तु} साध‚न‚स्य ‚{\color{DodgerBlue3}‚स‚र्व्व}‚स्याभ्युप‚ग‚म‚हेतुदृष्टा‚{\tiny $_{lb}$}‚न्तादे‚{\color{DodgerBlue3}‚र‚सिद्धाव‚रोधः} संग्र‚हः (।) त‚द्व‚ती य‚स्मात् ‚{\color{DodgerBlue3}‚साध्य‚श्रुतिरिष्टं विशेषं न} साध्य‚{\tiny $_{lb}$}‚‚{\color{DodgerBlue3}‚त्वेनाव‚ल‚म्ब‚ते} प‚रिगृह्णाति ॥ (१६५)
	\pend% ending standard par
      \label{div_pvv.4.166}
	  
	% new div opening: depth here is 2
	

	  \pstart \leavevmode% starting standard par
	येन प्र‚त्याय‚नाधिकारे\edtext{}{\edlabel{pvv.470-1}\label{pvv.470-1}\lemma{नाधिकारे}\Bfootnote{य‚दि साध्य इति सामान्य‚श‚ब्दः किमिति हेतुदृष्टान्त एव स‚र्व्व‚म‚सिद्ध‚म‚स्ती‚{\tiny $_{lb}$}‚त्याह ।}} साध‚न‚स्यासिद्ध‚स्य प‚क्ष‚त्व‚प्र‚स‚ङ्गः (।)
	\pend% ending standard par
      
	  \bigskip
	  \begingroup
	
	    \large
	  
	    \begin{quote}
	  
	    
	    \stanza[\smallbreak]
	\label{pv.4.166}\flagstanza{\tiny\textenglish{....4.166}}तेनाप्र‚सिद्ध‚दृष्टान्त‚हेतूदाह‚र‚णं कृत‚म् ।&अन्य‚था श‚श‚शृङ्गादौ स‚र्वासिद्धेऽपि साध्य‚ता ॥ १६६ ॥\&[\smallbreak]


	
	    \end{quote}
	  
	  \endgroup
	

	  \pstart \leavevmode% starting standard par
	\hphantom{.}‚{\color{DodgerBlue3}‚तेनाप्र‚सिद्धाभ्यां दृष्टान्त‚हेतु}‚भ्यां प्र‚तिज्ञाप्र‚स‚ङ्गादुदाह‚र‚णं कृतं प्राक् । त‚था ‚{\tiny $_{lb}$}‚चासिद्ध‚{\tiny $_{3}$}‚दृष्टान्त‚हेतुवादः प्र‚स‚ज्य‚त इत्य‚नेन । ‚{\color{DodgerBlue3}‚अन्य‚था} य‚दि साध‚न‚म‚सिद्धं विव‚क्षितं ‚{\tiny $_{lb}$}‚न स्यात् त‚दा ‚{\color{DodgerBlue3}‚श‚श‚शृङ्गादा}‚व‚पि ‚{\color{DodgerBlue3}‚स‚र्व‚स्यासिद्ध}‚स्य प‚क्ष‚त्व‚प्र‚स‚ङ्गः ॥ (१६६)
	\pend% ending standard par
      \label{div_pvv.4.167}
	  
	% new div opening: depth here is 2
	

	  \pstart \leavevmode% starting standard par
	न‚नु साध्यं क‚र्म (।) क‚र्म‚णि कृत्य‚विधानात् । क‚र्म चेप्सित‚त‚मं (।) त‚च्च ‚{\tiny $_{lb}$}‚क्रियाप्य‚मात्रं प‚य‚सोद‚नं भुक्तं इत्य‚त्र क्रियाप्य‚त्वेप्य‚नीप्सित‚त‚म‚त्वात् प‚यः क‚र‚णं । ‚{\tiny $_{lb}$}‚त‚तः साध‚नीय‚त्वेपि वादिनोऽनीप्सित‚त‚म‚त्वात् हेतुदृष्टान्तादिकं न प्र‚तिज्ञा भ‚विष्य‚{\tiny $_{lb}$}‚ति ।‚{\tiny $_{4}$}‚ त‚था \edtext{}{\edlabel{pvv.470-2}\label{pvv.470-2}\lemma{था}\Bfootnote{द्वितीयः स‚न्ति प्र‚माणानि प्र‚मेयार्थानि स‚र्व्व‚संम‚त्या ।}} च‚तुर्व्विधः सिद्धान्तः \edtext{}{\edlabel{pvv.470-3}\label{pvv.470-3}\lemma{सिद्धान्तः}\Bfootnote{य‚था स्वं नित्य‚म‚नित्य‚म्वा ।}} । स‚र्व्व‚त‚न्त्र‚सिद्धान्तः । \edtext{\textsuperscript{*}}{\edlabel{pvv.470-4}\label{pvv.470-4}\lemma{*}\Bfootnote{य‚त् प्र‚सिद्धाव‚न्य‚सिद्धिः सांख्य‚स्य वान्त‚राभ‚व‚निषेधे आत्मैव स‚ञ्च‚र‚त्य‚{\tiny $_{lb}$}‚श‚रीराऽप‚रीक्षिताभ्युप‚ग‚मात् ।}}प्र‚तित‚न्त्र‚{\tiny $_{lb}$}‚सिद्धान्तः । अधिक‚र‚ण‚सिद्धान्तः । अभ्युप‚ग‚म‚सिद्धान्त‚श्च । त‚त्राभ्युप‚ग‚म‚सिद्धान्त‚त्वं\edtext{}{\edlabel{pvv.470-5}\label{pvv.470-5}\lemma{त्वं}\Bfootnote{प‚रीक्षात्मास्तित्व‚व‚त् न च त‚दा हेत्वाद‚योभ्युप‚ग‚ताः ।}} ‚{\tiny $_{lb}$}‚शास्त्र‚दृष्ट‚स्येति । त‚दाश्र‚येण साध्य‚स्य निर्देशः प्र‚तिज्ञा न हेतुदृष्टान्तादेः प‚क्ष‚त्व‚{\tiny $_{lb}$}‚प्र‚स‚ङ्ग इत्याह ।
	\pend% ending standard par
      \textsuperscript{\textenglish{471/s}}
	  \bigskip
	  \begingroup
	
	    \large
	  
	    \begin{quote}
	  
	    
	    \stanza[\smallbreak]
	\label{pv.4.167}\flagstanza{\tiny\textenglish{....4.167}}स‚र्व‚स्य चाप्र‚सिद्ध‚त्वात् क‚थ‚ञ्चित् तेन न क्ष‚माः ।&क‚र्मादिभेदोप‚क्षेप‚प‚रिहाराविवेच‚ने ॥ १६७ ॥\&[\smallbreak]


	
	    \end{quote}
	  
	  \endgroup
	

	  \pstart \leavevmode% starting standard par
	\hphantom{.}‚{\color{DodgerBlue3}‚स‚र्व्व‚स्य} प्र‚तिज्ञाहेतुदृष्टान्तादे‚{\color{DodgerBlue3}‚श्चाप्र‚सिद्ध‚त्वात्} । साध‚यितुमीप्सित‚त‚म‚त्वेना‚{\tiny $_{lb}$}‚भ्युप‚ग‚मात् क‚र्म‚त्वं (।) क‚र्म च साध्यं प्र‚तिज्ञेति प्र‚तिज्ञात्व‚प्र‚स‚ङ्गो दुर्व्वारः । ‚{\tiny $_{lb}$}‚अभ्युप‚ग‚{\tiny $_{5}$}‚म‚श्च य‚था नित्य‚त्वे त‚था मूर्त्त‚त्त्वादाव‚पि शास्त्राभ्युप‚ग‚म‚योश्च भेदः ‚{\tiny $_{lb}$}‚प्रागुक्त‚स्तेन क‚र्म‚त्वाविशेषेण साध्यं क‚र्म साध‚नं क‚र‚ण‚मिति ‚{\color{DodgerBlue3}‚क‚र्मादिभेद‚स्योप‚क्षे‚{\tiny $_{lb}$}‚पेणोप‚न्यासेन प‚रिहाराविवेच‚ने} साध्य‚विशेषाव‚ग‚माप‚ने क‚थ‚{\color{DodgerBlue3}‚ञ्चिन्न क्ष‚माः} । (१६७)
	\pend% ending standard par
      \label{div_pvv.4.168}
	  
	% new div opening: depth here is 2
	

	  \pstart \leavevmode% starting standard par
	य‚द्य‚प्युच्य‚ते यः साध्योऽव‚य‚वः \edtext{}{\edlabel{pvv.471-1}\label{pvv.471-1}\lemma{वः}\Bfootnote{प्र‚तिज्ञादिषु प्र‚क्रान्तेषु साध‚न‚विष‚य एव यः साध्याव‚य‚वः ।}} त‚स्य निर्देशः प्र‚तिज्ञा । त‚तो हेत्वादेः सिद्ध‚{\tiny $_{lb}$}‚स्याव‚य‚व‚स्य न प्र‚तिज्ञात्व‚मिति । त‚द‚प्य‚स‚त् \edtext{}{\edlabel{pvv.471-2}\label{pvv.471-2}\lemma{त्}\Bfootnote{त‚था य‚दि साध्यं व‚स्तुत एव साध‚नाद्भेदेन स्यात् न चैवं ।}} ।
	\pend% ending standard par
      

	  \pstart \leavevmode% starting standard par
	किञ्च
	\pend% ending standard par
      
	  \bigskip
	  \begingroup
	
	    \large
	  
	    \begin{quote}
	  
	    
	    \stanza[\smallbreak]
	\label{pv.4.168}\flagstanza{\tiny\textenglish{....4.168}}प्राग‚सिद्ध‚स्व‚भाव‚त्वात् साध्योव‚य‚व इत्य‚स‚त् ।&तुल्या सिद्धान्त‚ता ते हि येनोप‚ग‚म‚ल‚क्ष‚णाः ॥ १६८ ॥\&[\smallbreak]


	
	    \end{quote}
	  
	  \endgroup
	

	  \pstart \leavevmode% starting standard par
	\hphantom{.}ल‚क्ष‚ण‚व‚च‚नात् ‚{\color{DodgerBlue3}‚प्राक्} ‚{\tiny $_{6}$}‚ साध्यासाध्य‚यो‚{\color{DodgerBlue3}‚र‚सिद्ध‚स्व‚भाव‚त्वात्} । ल‚क्ष‚णेन हि ‚{\tiny $_{lb}$}‚साध्य‚ता प्र‚तिप‚त्त‚व्या त्रिकाल‚विष‚य‚त्वात् (।) कृत्य‚प्र‚त्य‚य‚स्य साध‚यिष्य‚माणेपि ‚{\tiny $_{lb}$}‚साध्य इति (।) साध‚नं च साध्यं स्यात् । ‚{\color{DodgerBlue3}‚सिद्धान्त‚ता} हेतुदृष्टान्तादेर‚पि ‚{\color{DodgerBlue3}‚तुल्या । ‚{\tiny $_{lb}$}‚ते हि} प्र‚तित‚न्त्रादि‚{\color{DodgerBlue3}‚सिद्धान्ता येन} कार‚णेन ‚{\color{DodgerBlue3}‚उप‚ग‚म‚ल‚क्ष‚णा} अभ्युप‚ग‚म‚स्व‚भावाः । ‚{\tiny $_{lb}$}‚य‚था हि नित्य‚त्व‚म‚भ्युप‚ग‚म्य‚ते\edtext{}{\edlabel{pvv.471-3}\label{pvv.471-3}\lemma{ते}\Bfootnote{य‚द्य‚पि त‚दा नाभ्युप‚ग‚ता सिद्धान्त‚ता त‚थापि कृत्य‚प्र‚त्य‚निर्देशादेवाभ्युप‚ग‚न्त‚{\tiny $_{lb}$}‚व्येन्त‚र्भावः ।}}त‚था मूर्त्त‚त्वादिक‚म‚पीति न विशेषः । (१६८)
	\pend% ending standard par
      \label{div_pvv.4.169}
	  
	% new div opening: depth here is 2
	

	  \pstart \leavevmode% starting standard par
	किञ्च साध्यो‚{\tiny $_{7}$}‚ ध‚र्मो ध‚र्मी द्व‚य‚म्वा स्यात् \edtext{}{\edlabel{pvv.471-4}\label{pvv.471-4}\lemma{स्यात्}\Bfootnote{अतिव्याप्तिमुक्त्वान्य‚द‚प्याह ।}} । य‚दि ध‚र्म‚स्त‚दा साध्य‚साध‚र्म्यात् \leavevmode\ledsidenote{\textenglish{95b/MA}} ‚{\tiny $_{lb}$}‚त‚द्ध‚र्म‚भ‚वीदृष्टान्त उदाह‚र‚ण‚मिति \edtext{}{\edlabel{pvv.471-5}\label{pvv.471-5}\lemma{मिति}\Bfootnote{नैयायिक‚स्य ।}} दृष्टान्त‚ल‚क्ष‚णं विरुध्य‚ते । न हि साध्य‚ध‚र्म‚स्या‚{\tiny $_{lb}$}‚नित्य‚त्वादेर्द्ध‚र्म उत्प‚त्तिम‚त्वादिः (।) किन्तु \edtext{}{\edlabel{pvv.471-6}\label{pvv.471-6}\lemma{किन्तु}\Bfootnote{उत्प‚त्तिमान् श‚ब्दो घ‚ट‚श्चेति साध्य‚साध‚र्म्यं त‚तः साध‚यितुमिष्टो ध‚र्म‚स्त‚द्ध‚र्मः ‚{\tiny $_{lb}$}‚त‚स्य भावः स‚त्ता । सोस्यास्तीति ध‚र्म‚स्य ध‚र्मान्त‚रास‚म्ब‚न्धात् स्व‚रूप‚हानिप्र‚स‚{\tiny $_{lb}$}‚ङ्गाच्च । दृष्टान्त‚स्यापि त‚द्ध‚र्म‚भावित्वं नास्ति ।}} श‚ब्द‚स्य (।) त‚तः साध्य‚ध‚र्मेण ‚{\tiny $_{lb}$}‚साध‚र्म्यात् दृष्टान्त‚स्य त‚द्ध‚र्म‚भावित्वं साध्य‚ध‚र्मोत्प‚त्तिम‚त्वादिभाव‚व‚त्वं नास्ति ।
	\pend% ending standard par
      

	  \pstart \leavevmode% starting standard par
	\leavevmode\ledsidenote{\textenglish{472/s}}अथ ध‚र्मी साध्यः त‚दोदाह‚र‚ण‚साध‚र्म्यात् साध्य‚साध‚नं हेतुरिति हेतुल‚क्ष‚णं ‚{\tiny $_{lb}$}‚न युज्येत । न हि श‚ब्दो ध‚र्म्म‚सिद्धो ये‚{\tiny $_{1}$}‚न त‚त्साध‚नात् साध्य‚साध‚नो हेतुः स्यात् ।
	\pend% ending standard par
      

	  \pstart \leavevmode% starting standard par
	अथो\edtext{}{\edlabel{pvv.472-1}\label{pvv.472-1}\lemma{अथो}\Bfootnote{पृथ‚क् पृथ‚क् च‚तुर्थः स‚मुदाय‚प‚क्षो व‚क्ष्य‚माणः ।}}भ‚यं साध्यं त‚दोभ‚य‚प‚क्ष‚भाविदोष‚प्र‚स‚ङ्गः । स‚मुदाय‚स्यासिद्ध‚त्वात् ‚{\tiny $_{lb}$}‚स एव साध्य इत्याह ।
	\pend% ending standard par
      
	  \bigskip
	  \begingroup
	
	    \large
	  
	    \begin{quote}
	  
	    
	    \stanza[\smallbreak]
	\label{pv.4.169}\flagstanza{\tiny\textenglish{....4.169}}स‚मुदाय‚स्य साध्य‚त्वेप्य‚न्योन्य‚स्य विशेष‚ण‚म् ।&साध्यं द्व‚यं त‚दाऽसिद्धं हेतुदृष्टान्त‚ल‚क्ष‚ण‚म् ॥ १६९ ॥\&[\smallbreak]


	
	    \end{quote}
	  
	  \endgroup
	

	  \pstart \leavevmode% starting standard par
	\hphantom{.}‚{\color{DodgerBlue3}‚स‚मुदाय‚स्य साध्य‚त्वे}‚पीष्य‚माणे‚{\color{DodgerBlue3}‚ऽन्योन्य‚स्य} प‚र‚स्प‚रं ‚{\color{DodgerBlue3}‚विशेष‚ण}‚म्व‚क्त‚व्यं । ध‚र्म‚वि‚{\tiny $_{lb}$}‚\leavevmode\ledsidenote{\textenglish{96b/MA}} शिष्टो\edtext{}{\edlabel{pvv.472-asterisk}\label{pvv.472-asterisk}\lemma{शिष्टो}\Bfootnote{अत्र[[[आद‚र्श‚पुस्त‚क‚स्थ‚कुण्ड‚लीकृतो ध‚र्मिविशिष्टो वा साध्य इत्यादि य‚दि ‚{\tiny $_{lb}$}‚नित्यः श‚ब्दः स‚र्व‚स्यानित्य‚त्वादिति वैध‚र्म्य‚दृष्टान्तोप‚द‚र्श‚न‚मेत‚त् इत्य‚न्त‚ग्र‚न्थांशोऽत्र ‚{\tiny $_{lb}$}‚य‚थानुक्र‚म‚मेवोद्ध्ृत्य योजितः ।]]]प‚तित‚म‚धःप‚त्र‚पृष्टेस्ति कुण्ड‚लीकृतं ।  --- \marginnote{\textenglish{95a-1/MA}}त‚दा[[१७०-१७७ कारिकाणां व्याख्यान्त‚र‚माद‚र्श‚पुस्त‚क‚स्थ‚म‚धो विन्य‚स्त‚म् ।]]चासिद्धं हेतुल‚क्ष‚णं ध‚र्मिध‚र्म‚स‚मुदाय‚ध‚र्मेणोत्प‚त्तिम‚त्वादिनोदाह‚र‚ण‚साध ‚{\tiny $_{lb}$}‚र्म्याभावात् । \textbf{दृष्टान्त‚ल‚क्ष‚ण}म‚प्य\textbf{सिद्धं} दृष्टान्ते ध‚र्म‚ध‚र्मिस‚मुदायेन साध‚र्म्याभावात् ॥ असंभ‚वात् साध्य‚श‚ब्दो ध‚र्मिवृत्तिर्य‚दीष्य‚ते । शास्त्रेणालं य‚थायोगं लोक एव प्र‚व‚र्त्त‚ताम् ॥ १७० ॥   --- अथोत्प‚त्तिम‚त्वादेः स‚मुदाय‚ध‚र्म‚त्वाभावात् \textbf{साध्य‚श‚ब्दो ध‚र्मिवृत्तिरिष्य‚ते} ‚{\tiny $_{lb}$}‚य‚था बौ द्धे नोक्त उप‚चारात् । त‚दा \textbf{शास्त्रेण} ल‚क्ष‚णे\textbf{नालं} किं प्र‚योज‚नं प्र‚सिद्ध‚{\tiny $_{lb}$}‚त्वात् \textbf{य‚थायोगं} बुद्धा \textbf{लोक एव प्र‚व‚र्त्त‚तां} (।) स‚विक‚ल्प‚कं हि ज्ञानं त‚स्य प्र‚सिद्धं । ‚{\tiny $_{lb}$}‚बौद्ध‚स्तेन लोक‚सिद्ध‚म‚नुव‚द‚ति किन्तु न्या‚{\tiny $_{1}$}‚य‚मिति त‚स्य प‚क्षैक‚देश‚योग्य‚ध‚र्मिणो ‚{\tiny $_{lb}$}‚ ध‚र्म‚सामान्य‚स्य दृष्टान्तेपि क‚थ‚नार्थं जिज्ञासित‚विशेषो ध‚र्मीति ल‚क्ष‚णं युक्तं । न च ‚{\tiny $_{lb}$}‚साध्य‚श‚ब्द‚स्यासिद्ध‚दृष्टान्तादौ प्र‚स‚ङ्गात् प‚क्ष‚त्व‚न्तूप‚चारात् ॥  --- किञ्च (।) अत्र साध्य‚निर्देश एव प्र‚तिज्ञेति पूर्व्वाव‚धार‚ण‚न्ताव‚न्न भ‚व‚ति । साध‚नाख्यान‚साम‚र्थ्यात् त‚द‚र्थे साध्य‚ता ग‚ता । हेत्वादिव‚च‚नैर्व्याप्तेर‚नाश‚ङ्क्यं च साध‚न‚म् ॥ १७१ ॥   --- य‚स्मात् \textbf{त‚द‚र्थे} सिद्ध‚निवृत्त्याऽसिद्ध‚ग्र‚हार्थे \textbf{साध्य‚ता} ग‚म्य‚त एव (।) \textbf{सिद्ध‚साध‚न}‚{\tiny $_{lb}$}‚वैय‚र्थ्येन साध‚न‚ञ्चोक्त‚म‚त्रेति \textbf{त‚त्साम‚र्थ्यात्} ।  --- अपि च साध्य‚निर्देश एव प्र‚तिज्ञेति पूर्व्वाव‚{\tiny $_{2}$}‚धार‚णे \textbf{साध‚नं} न प्र‚तिज्ञेति साध्यं ‚{\tiny $_{lb}$}‚(।) त‚च्चा\textbf{नाश‚ङ्क्यं} त‚स्य हेतूदाह‚र‚णोप‚न‚याद्यैरेव व्याप्त‚त्वात् (।) पूर्व्वाव‚धार‚णे तेन प्र‚तिज्ञाल‚क्ष‚णाभिधा । व्य‚र्था व्याप्तिफ‚ला सोक्तिः साम‚र्थ्याद् ग‚म्य‚ते त‚तः ॥ १७२ ॥   --- तेन पूर्व्वाव‚धार‚णे प्र‚तिज्ञाल‚क्ष‚णाभिधा व्य‚र्था । त‚तः कार‚णात् \textbf{साम‚र्थ्यात्} ‚{\tiny $_{lb}$}‚साध्य‚निर्देशः प्र‚तिज्ञैवेत्युत्त‚राव‚धार\textbf{णोक्तिर्व्याप्तिफ‚ला}ऽयोग‚व्य‚व‚च्छेद‚फ‚ला । ‚{\tiny $_{lb}$}‚इत्य‚सिद्ध‚हेत्वा(दे)र‚पि प्र‚तिज्ञात्व‚प्र‚स‚ङ्गः ॥  --- प्र‚तिज्ञाहेत्वोर्व्विरोधो नोक्तः प‚क्षाभाषे (? से)ष्वित्य‚व्याप्तिल‚क्ष‚ण‚मिति ‚{\tiny $_{lb}$}‚नै या यिकाव‚काशं भ‚त्वा चा र्यो3 क्तं व्याख्यातुमाह । प्र‚तिज्ञाहेत्वोर्व्विरोधः प्र‚तिज्ञा‚{\tiny $_{lb}$}‚दोष इति य‚था नित्यः श‚ब्दः स‚र्व्व‚स्यानित्य‚त्वात् । अत्र य‚दि श‚ब्दो नित्यः स‚र्व्व‚{\tiny $_{lb}$}‚स्यानित्य‚त्वं विरुद्ध‚मिति प्र‚तिज्ञ‚या हेतोर्व्विरोधः । स‚र्व्वानित्य‚त्वे त‚द‚न्त‚र्ग‚त‚त्वाच्छ‚{\tiny $_{lb}$}‚ब्द‚स्येति हेतुर्न प्र‚तिज्ञाविरोधः । उत्त‚र‚माह (।) विरुद्ध‚तेष्टास‚म्ब‚न्धोऽनुप‚कार‚स‚हास्थिती । एवं स‚र्व्वाङ्ग‚दोष‚णां प्र‚तिज्ञादोष‚ता भ‚वेत् ॥ १७३ ॥   --- \textbf{इष्टेनार्थेना[[साध्य‚ध‚र्मेण ।]]स‚म्ब‚न्धो विरुद्ध‚तात्र} स्यात् । सा चा\textbf{नुप‚कार‚स‚हास्थिती} । ‚{\tiny $_{lb}$}‚हेतुस्त‚त्राभावात् साध्यं नोप‚क‚रोतीति वा स्थित‚हेतुना इष्ट‚ध‚र्म‚विरोधात् स‚हा‚{\tiny $_{4}$}‚‚{\tiny $_{lb}$}‚न‚व‚स्थान‚म्वा स्यात् । त‚त्र नानुप‚कारः । य‚दि हेतौ दुष्टे साध्योप‚रोधात् प्र‚तिज्ञादोष ‚{\tiny $_{lb}$}‚एवं \textbf{स‚र्व्व}लिङ्ग\textbf{दोषाणाम}सिद्धादीनां \textbf{प्र‚तिज्ञादोष‚ता} स्यात् । प‚क्ष‚दोषः प‚रापेक्षो नेति च प्र‚तिपादित‚म् । इष्टास‚म्भ‚व्य‚सिद्ध‚श्च स एव स्यात् निराकृतः ॥ १७४ ॥   --- उत्त‚राव‚य‚वापेक्षः प‚क्ष‚दोषो न भ‚व‚तीति प्रागेवोक्तं ।  --- अथेष्टेन साध्य‚ध‚र्मेणास‚म्भ‚वी स‚हान‚व‚स्थायी हेतुर्व्विरुद्धः । त‚न्न (।) ‚{\tiny $_{lb}$}‚\textbf{इष्टास‚म्भ‚व्य‚सिद्ध‚श्च स्याद्} हेतुदोषो न प‚क्षे । \textbf{निराकृतो} हेत्वाभासेष्वेवो‚{\tiny $_{lb}$}‚क्त‚त्वात् । अधुना हेतुप्र‚योग एवायं नित्या‚{\tiny $_{5}$}‚त् । न स‚र्व्वानित्य‚त्वेन श‚ब्दे नित्य‚{\tiny $_{lb}$}‚त्व‚निराक्रिया य‚तो विरोधः स्यात् (।) य‚स्मादेवं स साध्य‚ध‚र्मो निराकृतः स्यात् । अनित्य‚त‚व‚स‚हेतुत्वे श‚ब्द एवं प्र‚कीर्त्त‚येत् ॥ दृष्टान्ताख्यान‚तोऽन्य‚त् किम‚स्त्य‚त्रार्थानुद‚र्श‚न‚म् ॥ १७५ ॥   --- य‚द्य‚नित्य‚त्वं श‚ब्दे स्यात् । त‚तोऽ\textbf{नित्य‚त्व‚स‚हेतुत्वे} स‚ति \textbf{श‚ब्दे एवं} नित्यः श‚ब्दः  --- स‚र्व्व‚स्यानित्य‚त्वादिति \textbf{प्र‚कीर्त‚येत्} । न चैवं (।)हेतुल‚क्ष‚णाभावात् । \textbf{वैध‚र्म्य}‚{\tiny $_{lb}$}‚दृष्टान्तः प‚र‚म‚त्र । \textbf{अतोन्य‚त्र किञ्चिद‚र्थ‚स्य प्र‚द‚र्श‚नं} ।  --- त‚थाहि (।) विशेषे भिन्न‚माख्याय सामान्य‚स्यानुव‚र्त्त‚ने । न त‚द् व्याप्तिः फ‚लं वा किं सामान्येनानुव‚र्त्त‚ने ॥ १७६ ॥   --- \textbf{विशे}षे श‚ब्द एव \textbf{भिन्नं} विरुद्धं नित्य‚त्व\textbf{माख्याय} स‚र्व्व‚स्यानित्य‚त्वादिति \textbf{सामा‚{\tiny $_{lb}$}‚न्य‚स्यानु‚{\tiny $_{6}$}‚व‚र्त्त‚ने । न त‚स्य} श‚ब्द‚स्य विशेष‚स्य व्याप्तिर‚नित्य‚ताप्राप्तिर्व्विशेष‚{\tiny $_{lb}$}‚प‚रिहार‚णैव वृत्तेः त‚क्रं कौण्डिन्याय ब्राह्म‚णेभ्यो द‚धि दीय‚तामितिव‚त् । श‚ब्दे ‚{\tiny $_{lb}$}‚नित्य‚त्वं विधाय स‚र्व्वानित्य‚त्व‚मुच्य‚मानं न त‚मास्क‚न्द‚ति । अतः श‚ब्देऽनित्य‚त्व‚म‚{\tiny $_{lb}$}‚भ‚व‚त् क‚थं स्व‚विरुद्ध‚म‚पाकुर्यात् ।  --- य‚दि श‚ब्द‚व्य‚तिरिक्त‚स्यानित्य‚त्वं नेष्टं स्या(त्) त‚दा \textbf{फ‚लं वा किं सामा‚{\tiny $_{lb}$}‚न्यानुव‚र्त्त‚ने} स‚र्व्व‚स्यानित्य‚त्वादिति । श‚ब्द‚स्यानित्य‚त्वादित्येव वाच्यं एवं हि ‚{\tiny $_{lb}$}‚स्फुटो विरोधः स्यात् (।) त‚स्मान्न श‚ब्देऽनि‚{\tiny $_{7}$}‚त्य‚त्वं (।) । स्यान्निराक‚र‚णं श‚ब्दे स्थिते नैवेत्य‚तोब्र‚वीत् ।   --- अत एवा चार्योऽब्र‚वीत् स‚मुच्च‚ये स्यान्निराक‚र‚णं श‚ब्दानित्य‚त्वेनेति । xx}} ध‚र्मिविशिष्टो वा\edtext{}{\edlabel{pvv.472-2}\label{pvv.472-2}\lemma{वा}\Bfootnote{य‚था स‚र्व्व‚म‚नित्यं न च स‚र्व्वं श‚ब्दः इत्य‚स‚र्व‚त्वान्नित्यः ।}}साध्यः । य‚था श‚ब्द‚विशिष्ट‚म‚नि‚{\tiny $_{7}$}‚त्य‚मिति त‚थाच ‚{\tiny $_{lb}$}‚द्व‚यं साध्यं स्यात् । त‚दा द्व‚य‚साध्य‚त्वाभ्युप‚ग‚मे तु हेतुदृष्टान्त‚योर्ल‚क्ष‚ण‚म‚सिद्धं ‚{\tiny $_{lb}$}‚स्यात् । न हि ध‚र्मिध‚र्म‚विशिष्टेन ध‚र्मेणानित्य‚श‚ब्द‚स‚म्ब‚न्धिना उत्प‚त्तिम‚त्वादिना ‚{\tiny $_{lb}$}‚घ‚टादेः साध‚र्म्य‚म‚स्ति येन दृष्टान्त‚ता भ‚वेत् । त‚था ध‚र्म‚विशिष्टे ध‚र्मिणि प्राग‚नु‚{\tiny $_{lb}$}‚मानाद्धेतुर‚पि न क‚स्य‚चित् सिद्ध इति हेतुल‚क्ष‚णं च न स्यात् । (१६९)
	\pend% ending standard par
      \label{div_pvv.4.170}
	  
	% new div opening: depth here is 2
	\textsuperscript{\textenglish{473/s}}
	  \bigskip
	  \begingroup
	
	    \large
	  
	    \begin{quote}
	  
	    
	    \stanza[\smallbreak]
	\label{pv.4.170}\flagstanza{\tiny\textenglish{....4.170}}असंभ‚वात् साध्य‚श‚ब्दो ध‚र्मिवृत्तिर्य‚दीष्य‚ते ।&शास्त्रेणालं य‚थायोगं लोक एव प्र‚व‚र्त्त‚ताम् ॥ १७० ॥\&[\smallbreak]


	
	    \end{quote}
	  
	  \endgroup
	

	  \pstart \leavevmode% starting standard par
	य‚थोत्प‚त्तिम‚त्वादेः स‚मुदाय‚ध‚र्म‚त्वासंभ‚वाद्ध‚र्मिध‚र्म‚त्व‚संभ‚वाच्च साध्य‚श‚ब्दो ‚{\tiny $_{lb}$}‚य‚दि ध‚र्मिवृत्तिरुप‚चारादिष्य‚ते‚{\tiny $_{1}$}‚ त‚दा शास्त्रेण दुर्विहितेनालं लोक एव य‚थाक्र‚मं ‚{\tiny $_{lb}$}‚य‚थायोगं य‚स्य यादृशं ल‚क्ष‚णं युक्तं त‚द‚व‚धार्य प्र‚तिज्ञाहेत्वादिषु प्र‚व‚र्त्त‚तां । य‚त्तु ‚{\tiny $_{lb}$}‚प‚क्षो ध‚र्मी अव‚य‚व‚स‚मुदायोप‚चारादित्य‚स्माभिरुच्य‚ते । त‚त्स‚र्वं ध‚र्मिध‚र्म‚प्र‚तिषेधार्थं ‚{\tiny $_{lb}$}‚उप‚चार‚योग्य‚प‚रिग्र‚हार्थं । त‚स्माद‚युक्तं प‚र‚स्य प्र‚तिज्ञाल‚क्ष‚णं । (१७०)
	\pend% ending standard par
      \label{div_pvv.4.171}
	  
	% new div opening: depth here is 2
	

	  \pstart \leavevmode% starting standard par
	किञ्च ।
	\pend% ending standard par
      

	  \pstart \leavevmode% starting standard par
	स‚र्व‚वाक्यानाम‚व‚धार‚ण‚फ‚ल‚त्वात्साध्य‚निर्देश एव प्र‚तिज्ञेति पूर्व‚प‚दाव‚धार‚णं वा ‚{\tiny $_{lb}$}‚स्यात् । साध्य‚निर्देशः प्र‚तिज्ञैवेत्युत्त‚र‚प‚दाव‚धार‚णं वा स्यात् । त‚त्र प्र‚थ‚म‚प‚क्षे सिद्ध‚{\tiny $_{lb}$}‚निवृत्याऽसिद्ध‚प‚रिग्र‚हः प्र‚योज‚नं । त‚च्चान्य‚थापि ल‚भ्य‚ते । त‚था हि ।
	\pend% ending standard par
      
	  \bigskip
	  \begingroup
	
	    \large
	  
	    \begin{quote}
	  
	    
	    \stanza[\smallbreak]
	\label{pv.4.171}\flagstanza{\tiny\textenglish{....4.171}}साध‚नाख्यान‚साम‚र्थ्यात्त‚द‚र्थे साध्य‚ता ग‚ता ।&हेत्वादिव‚च‚नैर्व्याप्तेर‚नाश‚ङ्क्यं च साध‚न‚म् ॥ १७१ ॥\&[\smallbreak]


	
	    \end{quote}
	  
	  \endgroup
	

	  \pstart \leavevmode% starting standard par
	सिद्धेऽपि साध‚नोप‚न्यासोऽनुप‚युक्त इति साध‚नाख्यान‚साम‚र्थ्यात् त‚स्याः ‚{\tiny $_{lb}$}‚प्र‚तिज्ञाया अर्थेऽसिद्धे साध्य‚ता ‚{\color{DodgerBlue3}‚प्र‚तीता} प्र‚तिज्ञाल‚क्ष‚णं विनापि
	\pend% ending standard par
      \textsuperscript{\textenglish{474/s}}

	  \begin{center}%% label @type='head'
	\textbf{(४) प्र‚तिज्ञाल‚क्ष‚णे दोषः}
	\end{center}
	

	  \pstart \leavevmode% starting standard par
	न‚चासिद्धे हेतुदृष्टान्तादिके प्र‚तिज्ञार्थ‚प्र‚संगः । त‚था हि । हेत्वादिव‚च‚नैः पृथ‚क् ‚{\tiny $_{lb}$}‚ल‚क्ष‚ण‚प्र‚तिपाद‚कैर्व्याप्तेर्विष‚यि‚{\tiny $_{3}$}‚कृत‚त्वात् साध‚नं हेतुदृष्टान्तं असिद्धं च प्र‚तिज्ञार्थे ‚{\tiny $_{lb}$}‚वाऽनाश‚ङ्क्यं । (१७१)
	\pend% ending standard par
      \label{div_pvv.4.172}
	  
	% new div opening: depth here is 2
	\textsuperscript{\textenglish{475/s}}
	  \bigskip
	  \begingroup
	
	    \large
	  
	    \begin{quote}
	  
	    
	    \stanza[\smallbreak]
	\label{pv.4.172}\flagstanza{\tiny\textenglish{....4.172}}पूर्वाव‚धार‚णे तेन प्र‚तिज्ञाल‚क्ष‚णाभिधा ।&व्य‚र्था व्याप्तिफ‚ला सोक्तिः साम‚र्थ्याद्ग‚म्य‚ते त‚तः ॥ १७२ ॥\&[\smallbreak]


	
	    \end{quote}
	  
	  \endgroup
	

	  \pstart \leavevmode% starting standard par
	तेनातिप्र‚संगाभावेन पूर्व‚स्य प‚द‚स्याव‚धार‚णे प्र‚तिज्ञाल‚क्ष‚णाभिधा व्य‚र्था । त‚तः ‚{\tiny $_{lb}$}‚पारिशेष्यात् साम‚र्थ्यात्साध्य‚निर्देशः प्र‚तिज्ञैवेत्य‚योग‚व्य‚व‚च्छेद‚फ‚ल‚मुत्त‚र‚प‚दाव‚धार‚णं ‚{\tiny $_{lb}$}‚स्यात् । त‚थाचासिद्ध‚हेतुदृष्टान्तादाव‚पि प्र‚तिज्ञात्वं दुर्वारं । असिद्ध‚स्य साध‚नाङ्ग‚स्य ‚{\tiny $_{lb}$}‚क‚थ‚म‚पि प्र‚तिज्ञात्वायोग‚ताविर‚हात् निर‚स्तं प्र‚ति‚{\tiny $_{4}$}‚ज्ञाल‚क्ष‚णं । (१७२)
	\pend% ending standard par
      \label{div_pvv.4.173_4.174_4.175}
	  
	% new div opening: depth here is 2
	

	  \pstart \leavevmode% starting standard par
	प‚र‚स्य प्र‚तिज्ञाभास‚ल‚क्ष‚णं संप्र‚ति निराक‚र‚णीयं (।) हेतुप्र‚तिज्ञ‚योर्व्याघातः ‚{\tiny $_{lb}$}‚प्र‚तिज्ञादोषो म‚तः । य‚था नित्यः श‚ब्दः स‚र्व‚स्य नित्य‚त्वात् । य‚दि स‚र्व‚म‚नित्यं त‚दा ‚{\tiny $_{lb}$}‚श‚ब्द‚स्यापि स‚र्व‚त्रान्त‚र्भावात् कुतो नित्य‚ता । अथ श‚ब्दो नित्यः क‚थं स‚र्व‚स्या‚{\tiny $_{lb}$}‚नित्य‚तेति प्र‚तिज्ञाहेत्वोर्विरोधात् । प्र‚तिज्ञाविरुद्ध‚तादोषः ।
	\pend% ending standard par
      
	  \bigskip
	  \begingroup
	
	    \large
	  
	    \begin{quote}
	  
	    
	    \stanza[\smallbreak]
	\label{pv.4.173}\flagstanza{\tiny\textenglish{....4.173}}विरुद्ध‚तेष्टास‚म्ब‚न्धोऽनुप‚कार‚स‚हास्थिती ।&एवं स‚र्वाङ्ग‚दोषाणां प्र‚तिज्ञादोष‚ता भ‚वेत् ॥ १७३ ॥\&[\smallbreak]


	
	    \end{quote}
	  
	  \endgroup
	

	  \pstart \leavevmode% starting standard par
	विरुद्ध‚ता चेष्ट‚स्या साध्य‚ध‚र्म‚स्य ध‚र्मिण्य‚स‚म्ब‚न्धो नाम । स च विचार्य‚माणो ‚{\tiny $_{lb}$}‚\leavevmode\ledsidenote{\textenglish{476/s}} हेतुना‚{\tiny $_{5}$}‚ साध्य‚ध‚र्म‚स्यानुप‚कारोऽनिश्चाय‚नं वा स्यात् । ध‚र्मिणि साध्येन स‚हास्थितिर्वा ‚{\tiny $_{lb}$}‚स्यात् । त‚त्र य‚दि हेतोः साध्ये प्र‚तिपाद‚क‚त्वाभावात् प्र‚तिज्ञादोष उच्य‚ते (।) एवं ‚{\tiny $_{lb}$}‚स‚ति स‚र्वेषाम‚ङ्ग‚स्य हेतोर्दोषाणां प्र‚तिज्ञादोष‚ता भ‚वेत् स‚र्वैर्हेंतुदोषैः प्र‚तिज्ञाया एव ‚{\tiny $_{lb}$}‚व्याह‚न‚नात् ।
	\pend% ending standard par
      
	  \bigskip
	  \begingroup
	
	    \large
	  
	    \begin{quote}
	  
	    
	    \stanza[\smallbreak]
	\label{pv.4.174}\flagstanza{\tiny\textenglish{....4.174}}प‚क्ष‚दोषः प‚राप‚क्षो नेति च प्र‚तिपादित‚म् ।&इष्टास‚म्भ‚व्य‚सिद्ध‚श्च स एवं स्यान्निराकृतः ॥ १७४ ॥\&[\smallbreak]


	
	    \end{quote}
	  
	  \endgroup
	

	  \pstart \leavevmode% starting standard par
	प्र‚तिज्ञामात्र‚भागी च प‚क्ष‚दोषः । प‚रापेक्षः साध‚नादिसापेक्षः न दोष इति च ‚{\tiny $_{lb}$}‚प्र‚तिपादितं प्रागुत्त‚रा‚{\tiny $_{6}$}‚व‚य‚वापेक्षो न दोषः प‚क्ष इष्य‚त इत्यादिना । अथ स‚हा‚{\tiny $_{lb}$}‚स्थितिस्व‚भावो विरोधः । त‚देष्टे प‚क्षेऽस‚म्भ‚वी हेतुदोष एवायं न प‚क्ष‚दोषः । अथ ‚{\tiny $_{lb}$}‚तेन हेतुना प्र‚तिज्ञार्थ‚निराक‚र‚णात् प्र‚तिज्ञाविरोधः प‚क्ष-दोष एव (।) त‚च्चायुक्तं ‚{\tiny $_{lb}$}‚त‚था हि (।) साध्य‚ध‚र्मो ध‚र्मिण्येवं निराकृतः स्यात् ।
	\pend% ending standard par
      
	  \bigskip
	  \begingroup
	
	    \large
	  
	    \begin{quote}
	  
	    
	    \stanza[\smallbreak]
	\label{pv.4.175}\flagstanza{\tiny\textenglish{....4.175}}अनित्य‚त्व‚स‚हेतुत्वे श‚ब्द एवं प्र‚कीर्त्त‚येत् ।&दृष्टान्ताख्यान‚तोऽन्य‚त् किम‚स्त्य‚त्रार्थानुद‚र्श‚म् ॥ १७५ ॥\&[\smallbreak]


	
	    \end{quote}
	  
	  \endgroup
	

	  \pstart \leavevmode% starting standard par
	य‚दि श‚ब्दे ध‚र्मिण्य‚नित्य‚त्वेन ध‚र्मेण स‚हेतुत्वे प्र‚तिपाद्ये एवं स‚र्व‚स्यानित्य‚त्वादि ‚{\tiny $_{lb}$}‚\leavevmode\ledsidenote{\textenglish{97a/MA}} प्र‚कीर्त्त‚येत्‚{\tiny $_{7}$}‚ । न चेदृशं वादिनो विव‚क्षितं स‚र्व्व‚स्य प‚र‚स्यानित्य‚त्वात् । श‚ब्दो नित्य ‚{\tiny $_{lb}$}‚इति विव‚क्षित‚त्वात् । त‚था च सामान्य‚विशेष‚भावाद् विरोध‚भावः । भ‚व‚तु वा ‚{\tiny $_{lb}$}‚श‚ब्देऽनित्य‚त्व‚निराक‚र‚णं विव‚क्षितं (।) त‚थापि प्र‚माण‚बाधैवाप‚क्ष‚तेति न प्र‚तिज्ञा‚{\tiny $_{lb}$}‚विरोधो नाम प‚क्ष‚दोषः । त‚स्माद्धेत्व‚र्थ‚तानुप‚प‚त्तेः स‚र्व्व‚स्यानित्य‚त्वादित्य‚त्र वैध‚र्म्येण ‚{\tiny $_{lb}$}‚दृष्टान्ताख्यान‚तोऽन्य‚द‚र्थानुद‚र्श‚नं किम‚स्ति । वैध‚र्म्य‚दृष्टान्त एव ‚{\tiny $_{1}$}‚ सुशिक्षितै‚{\tiny $_{lb}$}‚रित्थ‚माख्यातः । स‚र्व्व‚स्य नित्य‚त्वे व्याप्तिद‚र्श‚नार्थं य‚दि पुन‚र्व्वैध‚र्म्य‚दृष्टान्तोप‚द‚{\tiny $_{lb}$}‚र्श‚न‚मेत‚न्न भ‚व‚ति (।) त‚दा ।
	\pend% ending standard par
      \label{div_pvv.4.176_4.177_4.178_4.179_4.180_4.181_4.182_4.183_4.184_4.185_4.186_4.187_4.188}
	  
	% new div opening: depth here is 2
	

	  \begin{center}%% label @type='head'
	\textbf{(५) सामान्य‚चिन्ता}
	\end{center}
	

	  \begin{center}%% label @type='head'
	\textbf{क. सामान्यानुव‚र्त्त‚ने निष्फ‚ल‚म्}
	\end{center}
	
	  \bigskip
	  \begingroup
	
	    \large
	  
	    \begin{quote}
	  
	    
	    \stanza[\smallbreak]
	\label{pv.4.176}\flagstanza{\tiny\textenglish{....4.176}}विशेषेभिन्न‚माख्याय सामान्य‚स्यानुव‚र्त्त‚ने ।&न त‚द्व्याप्तिः फ‚लं वा किं सामान्येनानुव‚र्त्त‚ने ॥ १७६ ॥\&[\smallbreak]


	
	    \end{quote}
	  
	  \endgroup
	

	  \pstart \leavevmode% starting standard par
	विशेषेण श‚ब्दे भिन्नं नित्य‚त्व‚माख्याय स‚र्व्व‚स्यानित्य‚त्वादिति व्यापित्वात् ‚{\tiny $_{lb}$}‚सामान्य‚स्यानित्य‚त्व‚स्यानुव‚र्त्त‚ने क्रिय‚माणे त‚स्यानित्य‚त्व‚स्य व्याप्तिर‚शेष‚प‚दार्थ‚ग्र‚हो ‚{\tiny $_{lb}$}‚न भ‚व‚ति । य‚था कौण्डिन्य‚स्य त‚क्र‚दानं विहितं ब्राह्म‚णेभ्यः सामान्येन विहित‚द‚धि‚{\tiny $_{lb}$}‚दानेन न बाध्य‚ते । प्र‚क‚ल्प्याप‚{\tiny $_{2}$}‚वाद‚विष‚य‚मुत्स‚र्ग‚स्य प्र‚वृत्तेः । त‚था श‚ब्दे नित्य‚त्व‚स्य ‚{\tiny $_{lb}$}‚\leavevmode\ledsidenote{\textenglish{477/s}} विशेष‚विहित‚स्य स‚र्व्वानित्य‚त्वेन सामान्य‚विहितेन बाधाश‚ङ्का नास्तीति क‚थं प्र‚तिज्ञा‚{\tiny $_{lb}$}‚हेत्वोर्व्विरोधः । य‚दि श‚ब्द‚व्य‚तिरिक्त‚स्य स‚र्व्व‚स्यानित्य‚त्व‚मिष्टं न स्यात् (।) त‚दा ‚{\tiny $_{lb}$}‚स‚र्व्व‚स्यानित्य‚त्वादिति सामान्येनानुव‚र्त्त‚ने किं फ‚लं स्यात् । नित्यः श‚ब्दः श‚ब्द‚स्या‚{\tiny $_{lb}$}‚नित्य‚त्वादित्येव वाच्यं । एवं विरोध‚स्य व‚क्त‚व्य‚त्वात् श‚ब्द एवोदाह‚र‚ण‚म्भ‚विष्य‚{\tiny $_{3}$}‚ ‚{\tiny $_{lb}$}‚तीति चेत् । नानुन्म‚त्त एवं ब्रूयात् । य‚द्यात्म‚नोऽनित्य‚त्वं हेतुः सिद्धः । क‚थं त‚द्विरुद्धं ‚{\tiny $_{lb}$}‚साध्यं (।) त‚त्रैव प्र‚तिजानीयात् । अथासिद्ध‚स्त‚दा हेतुदोष एवासौ न प्र‚तिज्ञा‚{\tiny $_{lb}$}‚दोषः (।) त‚स्मान्नास्ति श‚ब्दे नित्य‚त्वं । अतः स्व‚विरोधिन‚म‚पि निराक‚र्त्तुम‚श‚क्तं ।
	\pend% ending standard par
      
	  \bigskip
	  \begingroup
	
	    \large
	  
	    \begin{quote}
	  
	    
	    \stanza[\smallbreak]
	\label{pv.4.177a}\flagstanza{\tiny\textenglish{...4.177a}}स्यान्निराक‚र‚णं श‚ब्दे स्थितेनैवेत्य‚तोब्र‚वीत् ।\&[\smallbreak]


	
	    \end{quote}
	  
	  \endgroup
	

	  \pstart \leavevmode% starting standard par
	\hphantom{.}अत एवाचार्य ‚{\color{DodgerBlue3}‚श‚ब्देस्थितेनैवा}‚नित्य‚त्वेन ‚{\color{DodgerBlue3}‚निराक‚र‚णं} नित्य‚त्व‚स्य ‚{\color{DodgerBlue3}‚स्यादित्य‚ब्र‚वीत्} । ‚{\tiny $_{lb}$}‚त‚दा च स्याद‚त्र प्र‚तिज्ञार्थ‚स्य निराक‚र‚णं ॥
	\pend% ending standard par
      

	  \pstart \leavevmode% starting standard par
	य‚दि नित्यः ‚{\tiny $_{4}$}‚ श‚ब्दः स‚र्व्व‚स्यानित्य‚त्वादिति वैध‚र्म्य‚दृष्टान्तोप‚द‚र्श‚न‚मेत‚त् । ‚{\tiny $_{lb}$}‚य‚था नित्य‚त्व‚विशिष्ट श‚ब्द इति । त‚दा हेतु-निर्देशोन स्यादित्याह (।) अ\edtext{}{\edlabel{pvv.477-1}\label{pvv.477-1}\lemma{अ}\Bfootnote{साध‚र्म्य‚वैध‚र्म्योदाह‚र‚णापेक्षः त‚था न त‚थेति प‚क्ष‚ध‚र्मोप‚संहार उप‚न‚य इह तु ‚{\tiny $_{lb}$}‚वैध‚र्म्योप‚न‚यः । य‚दाह त‚स्मात् स‚र्व‚त्वान्नानित्य इति ।}}स‚र्व‚श्च ‚{\tiny $_{lb}$}‚श‚ब्द इत्युप‚न‚याद्धेतुर्व‚क्त‚व्यः ।
	\pend% ending standard par
      
	  \bigskip
	  \begingroup
	
	    \large
	  
	    \begin{quote}
	  
	    
	    \stanza[\smallbreak]
	\label{pv.4.177b}\flagstanza{\tiny\textenglish{...4.177b}}विरुद्ध‚विष‚येन्य‚स्मिन् व‚द‚न्नाहान्य‚तां श्रुतेः ॥ १७७ ॥\&[\smallbreak]


	
	    \end{quote}
	  
	  \endgroup
	

	  \pstart \leavevmode% starting standard par
	त‚थाहि नित्य‚त्व‚स्य वि‚{\tiny $_{2}$}‚रुद्धं श‚ब्दाद‚न्य‚स्मिन् व‚द‚न् स‚र्व्व‚स्माद‚न्य‚तां श्रुतेः 95b ‚{\tiny $_{lb}$}‚श‚ब्द\edtext{}{\edlabel{pvv.477-2}\label{pvv.477-2}\lemma{ब्द}\Bfootnote{स‚र्व‚म‚नित्य‚त्वेन व्याप्तं श‚ब्द‚म‚नित्य‚त्वात्त‚तोऽन्य इत्य‚स‚र्व‚त्वं हेतुः स्यात् ।}}स्याह ।
	\pend% ending standard par
      
	  \bigskip
	  \begingroup
	
	    \large
	  
	    \begin{quote}
	  
	    
	    \stanza[\smallbreak]
	\label{pv.4.178a}\flagstanza{\tiny\textenglish{...4.178a}}स च भेदोप्र‚तिक्षेपात् सामान्यानान्न विद्य‚ते ।\&[\smallbreak]


	
	    \end{quote}
	  
	  \endgroup
	

	  \pstart \leavevmode% starting standard par
	स च स‚र्व्व‚स्माद् भेदोऽस‚र्व्व‚ल‚क्ष‚णः श‚ब्द‚स्य न विद्य‚ते (।) सामान्यानां ‚{\tiny $_{lb}$}‚व्यापिनां भेद‚स्याप्र‚तिक्षेपात् स्वीकारात् (।) निः शेषार्थ‚संगृहीत‚त्वात् स‚र्व्व‚श‚ब्दो ‚{\tiny $_{lb}$}‚न किञ्चित् प‚रिह‚र‚ति ।
	\pend% ending standard par
      

	  \pstart \leavevmode% starting standard par
	न‚नु सामान्यानां विशेष‚प्र‚तिक्षेपो दृश्य‚ते एव । य‚था किं शिंश‚पैव वृ‚{\tiny $_{3}$}‚क्षो ‚{\tiny $_{lb}$}‚न वेति प्र‚श्ने क‚थ्य‚ते न शिंश‚पैव वृक्षः । त‚दा शिंश‚पावृक्ष‚त्व‚प्र‚तिक्षेपो भ‚व‚त्येव ।
	\pend% ending standard par
      

	  \pstart \leavevmode% starting standard par
	अस‚देत‚त् । न हि त‚त्र शिंश‚पावृक्ष‚त्वं निषिध्य‚ते किन्तु शिंश‚पैव वृक्ष इति ‚{\tiny $_{lb}$}‚निय‚मः प्र‚तिक्षिप्य‚ते (।) त‚दित‚र‚स्यापि वृक्ष‚त्वात् ।
	\pend% ending standard par
      
	  \bigskip
	  \begingroup
	
	    \large
	  
	    \begin{quote}
	  
	    
	    \stanza[\smallbreak]
	\label{pv.4.178b}\flagstanza{\tiny\textenglish{...4.178b}}वृक्षो न शिंश‚पैवेति य‚था प्र‚क‚र‚णे क्व‚चित् ॥ १७८ ॥\&[\smallbreak]


	
	    \end{quote}
	  
	  \endgroup
	\textsuperscript{\textenglish{478/s}}
	  \bigskip
	  \begingroup
	
	    \large
	  
	    \begin{quote}
	  
	    
	    \stanza[\smallbreak]
	\label{pv.4.179}\flagstanza{\tiny\textenglish{....4.179}}स‚र्व‚श्रुतेरेक‚वृत्तिनिषेधः स्यान्न चेय‚ता ।&सोस‚र्वः स‚र्व‚भेदानाम‚त‚त्त्वे त‚द‚स‚म्भ‚वात् ॥ १७९ ॥\&[\smallbreak]


	
	    \end{quote}
	  
	  \endgroup
	

	  \pstart \leavevmode% starting standard par
	य‚था च क्व‚चित् प्र‚क‚र‚णे शिंश‚पामात्र‚वृक्ष‚त्व‚प्र‚श्न‚हेतौ य‚था वृक्षो न शिंश‚पैवेति ‚{\tiny $_{lb}$}‚शिंश‚पामात्र‚वृक्ष‚त्व‚निषेध इष्टः । त‚था निःशेष‚व‚स्तुसंग्राहिकाया‚{\tiny $_{4}$}‚ः स‚र्व्व‚श्रुतेरेक‚त्र ‚{\tiny $_{lb}$}‚श‚ब्द‚मात्र‚वृत्तिनिषेधः स्यात् (।) न चेय‚ता स श‚ब्दोऽस‚र्व्वः स‚र्व्वान्त‚र्ग‚मात् त‚स्य । ‚{\tiny $_{lb}$}‚प्र‚त्येकं स‚र्व्वेषां भेदानां विशेषाणाम‚त‚त्वेऽस‚र्व्व‚त्वे त‚स्य स‚र्व्व‚स्यास‚म्भ‚वात् ।
	\pend% ending standard par
      

	  \pstart \leavevmode% starting standard par
	अथ पारिभाषिकं स‚र्व्व‚त्वं श‚ब्दादित‚र‚त्वं त‚थाऽस‚र्व्व‚त्वं श‚ब्दे सिद्ध‚मेवेति चेत् । ‚{\tiny $_{lb}$}‚त‚द‚युक्तं\edtext{}{\edlabel{pvv.478-1}\label{pvv.478-1}\lemma{युक्तं}\Bfootnote{श‚ब्दोऽस‚र्व्व इति प‚र्यायाः । त‚था च य‚था नित्यः श‚ब्दः श‚ब्द‚त्वादिति ‚{\tiny $_{lb}$}‚प्र‚तिज्ञार्थैक‚देशो हेतुर‚सिद्ध‚स्त‚थेहाप्य‚सिद्धिरिति स‚मुदायार्थः ।}} (।) एवं हि स‚र्व्व‚त्व‚मेव स‚र्व्व‚श‚ब्देनोक्तं स्यात् । त‚थाऽप्र‚सिद्ध‚तैव । त‚था हि ।
	\pend% ending standard par
      
	  \bigskip
	  \begingroup
	
	    \large
	  
	    \begin{quote}
	  
	    
	    \stanza[\smallbreak]
	\label{pv.4.180}\flagstanza{\tiny\textenglish{....4.180}}ज्ञाप्य‚ज्ञाप‚क‚योर्भेदात् ध‚र्मिणो हेतुभाविनः ।&असिद्धेर्ज्ञापिक‚त्व‚स्य ध‚र्म्य‚सिद्धः स्व‚साध‚ने ॥ १८० ॥\&[\smallbreak]


	
	    \end{quote}
	  
	  \endgroup
	

	  \pstart \leavevmode% starting standard par
	ज्ञाप्य‚ज्ञाप‚क‚योर्भेदात् कार‚णा ‚{\tiny $_{5}$}‚ त् साध्याभिन्नं साध‚नं व‚क्त‚व्यं । त‚तो ‚{\tiny $_{lb}$}‚ध‚र्मिणो हेतुत्वेन भाविनो हेतुर्भ‚विष्य‚तो नित्य‚त्वेन साध्य‚त्वात् । ज्ञाप‚क‚स्यासिद्धेः ‚{\tiny $_{lb}$}‚कार‚णात् । ध‚र्मी साध्य‚त्वात् स्व‚स्य साध‚नेऽसिद्धः । असिद्धं हि साध्यं । सिद्ध‚ञ्च ‚{\tiny $_{lb}$}‚साध‚नं । अन‚योः क‚थ‚मैकात्म्यं ॥
	\pend% ending standard par
      

	  \pstart \leavevmode% starting standard par
	न‚नु श‚ब्द‚त्वं सिद्ध‚मेव । नित्य‚व‚त्त‚या तु त‚द‚सिद्धं साध्य‚ते । न च त‚थैव ‚{\tiny $_{lb}$}‚त‚त्साध‚नं । श‚ब्द‚त्व‚मात्रेण साध‚न‚त्वात् । त‚त्क‚थ‚न्ध‚र्मी स्व‚साध‚नेऽसिद्ध उच्य‚ते ॥ ‚{\tiny $_{lb}$}‚अय‚म‚भिप्रायः । न श‚ब्द‚त्व‚मित्येव ग‚म‚क‚त्वं किन्तु साध्य‚प्याप्तं त‚स्य चासाधार‚ण‚त्वेन ‚{\tiny $_{lb}$}‚नान्य‚त्र व्याप्त्युप‚ल‚म्भः । त‚त‚श्चान्य‚त्रानित्य‚त्व‚निय‚मात् श‚ब्दे श‚ब्द‚त्वं नित्य‚ताव्याप्तं ‚{\tiny $_{lb}$}‚स‚द्धेतुर्व्व‚क्त‚व्यः । त‚था च य एव साध्यः स एव हेतुरिति साध्य‚सिद्धेर्हेत्व‚{\tiny $_{lb}$}‚सिद्धिश्चेति युक्त‚मुक्तं ध‚र्म्य‚सिद्धः प्र‚साध‚न इति ॥
	\pend% ending standard par
      

	  \begin{center}%% label @type='head'
	\textbf{ख. सामान्य‚निरासः}
	\end{center}
	
	  \bigskip
	  \begingroup
	
	    \large
	  
	    \begin{quote}
	  
	    
	    \stanza[\smallbreak]
	\label{pv.4.181}\flagstanza{\tiny\textenglish{....4.181}}ध‚र्म‚ध‚र्मिविवेक‚स्य स‚र्व‚भावेष्व‚सिद्धितः ।&स‚र्व‚त्र दोष‚स्तुल्य‚श्चेन्न संवृत्या विशेष‚तः ॥ १८१ ॥\&[\smallbreak]


	
	    \end{quote}
	  
	  \endgroup
	

	  \pstart \leavevmode% starting standard par
	न‚न्वेवं स‚ति स‚र्व्व‚त्र भावेषु ध‚र्म‚याः साध्य‚साध‚न‚योर्द्ध‚र्मिण‚श्च विवेक‚स्य भेद‚{\tiny $_{lb}$}‚स्यासिद्धितः \edtext{}{\edlabel{pvv.478-2}\label{pvv.478-2}\lemma{स्यासिद्धितः}\Bfootnote{त‚त्रैव---उप‚रिम‚स्य प‚तित‚मेत‚त् ।}} ‚{\tiny $_{7}$}‚ ।
	\pend% ending standard par
      \textsuperscript{\textenglish{479/s}}

	  \pstart \leavevmode% starting standard par
	\edtext{\textsuperscript{*}}{\edlabel{pvv.479-1}\label{pvv.479-1}\lemma{*}\Bfootnote{स्व‚भाव‚हेतौ ।}} ‚{\color{DodgerBlue3}‚स‚र्व्व‚त्र तुल्यो दोषः} । न‚हि श‚ब्दाद‚न्य‚त् नित्य‚त्वं कृत‚क‚त्व‚म्वा । अनित्य-\leavevmode\ledsidenote{\textenglish{96a/MA}} त्वासिद्धौ च य‚था नित्यः साध्य‚स्त‚था त‚दात्म‚वान् कृत‚कोपि श‚ब्दोऽसिद्ध ऐव‚ति चेत् । ‚{\tiny $_{lb}$}‚नैष दोषः (।) ‚{\color{DodgerBlue3}‚संवृत्त्या} भिन्न‚व्यावृत्तिविष‚य‚या साध्य‚साध‚न‚ध‚र्मिणो ‚{\color{DodgerBlue3}‚विशेष‚तो} भेदात् । य‚था स्वं विक‚ल्पैः संकेत‚वास‚नानुग‚म‚निय‚मितैकैक‚वृत्तिमात्र‚विष‚यार्थैः ‚{\tiny $_{lb}$}‚श‚ब्द‚त्व‚कृत‚क‚त्व‚नित्य‚त्वाद्य‚संकीर्ण्णान्येव ध‚र्मिसाध‚न‚साध्य‚त‚या व्य‚व‚स्थाप्य‚न्त इति ‚{\tiny $_{lb}$}‚‚{\tiny $_{1}$}‚न दोषः ।
	\pend% ending standard par
      

	  \pstart \leavevmode% starting standard par
	य‚दि ध‚र्म‚ध‚र्मिविवेकोऽस्त्येव त‚दा त‚त्त्वान्य‚त्व‚प्र‚तिषेधः क‚थं कृत इत्याह ।
	\pend% ending standard par
      
	  \bigskip
	  \begingroup
	
	    \large
	  
	    \begin{quote}
	  
	    
	    \stanza[\smallbreak]
	\label{pv.4.182}\flagstanza{\tiny\textenglish{....4.182}}प‚र‚मार्थ‚विचारेषु त‚थाभूताप्र‚सिद्धितः ॥&त‚त्त्वान्य‚त्वं प‚दार्थेषु सांबृतेषु निषिध्य‚ते ॥ १८२ ॥\&[\smallbreak]


	
	    \end{quote}
	  
	  \endgroup
	

	  \pstart \leavevmode% starting standard par
	\hphantom{.}सांवृतेषु क‚ल्प‚नाविष‚येषु ध‚र्मिध‚र्मादिषु त‚त्त्वान्य‚त्वं ‚{\color{DodgerBlue3}‚प‚र‚मार्थ‚स्य} त‚त्त्व‚स्य ‚{\tiny $_{lb}$}‚‚{\color{DodgerBlue3}‚विचारेषु} निषिध्य‚ते ‚{\color{DodgerBlue3}‚त‚थाभूत‚स्य} य‚थाक‚ल्प‚नं प‚र‚स्प‚र‚तो भिन्न‚स्य ध‚र्मिध‚र्मादेः ‚{\tiny $_{lb}$}‚प्र‚माणेना‚{\color{DodgerBlue3}‚प्र‚सिद्धितः} । न तु सांवृतोपि तेषां भेदाभावः । न चेय‚ता क‚ल्पिते ध‚र्मिणि ‚{\tiny $_{lb}$}‚क‚ल्पितात् साध‚नात् क‚ल्पित‚स्य साध्य‚स्य सिद्धिरित्य‚नुमा‚{\tiny $_{2}$}‚नाद‚व‚स्तुप्र‚तीत्य‚भाव‚{\tiny $_{lb}$}‚प्र‚स‚ङ्ग । अश‚ब्द‚व्यावृत्त्या निश्चित‚स्य व‚स्तुन एव ध‚र्मित्वात् । एवं कृत‚क‚त्वा‚{\tiny $_{lb}$}‚नित्य‚त्वाभ्यां निश्चित‚स्य त‚स्यैव साध‚न‚त्वात् साध्य‚त्वाच्चेत्युक्तेः । अन्योन्य‚स्य ‚{\tiny $_{lb}$}‚तेषां भेदः पुनः क‚ल्पित एव ।
	\pend% ending standard par
      
	  \bigskip
	  \begingroup
	
	    \large
	  
	    \begin{quote}
	  
	    
	    \stanza[\smallbreak]
	\label{pv.4.183}\flagstanza{\tiny\textenglish{....4.183}}अनुमानानुमेयार्थ‚व्य‚व‚हार‚स्थितिस्त्विय‚म् ।&भेदं प्र‚त्य‚य‚संसिद्ध‚म‚व‚ल‚म्ब्य च क‚ल्प्य‚ते ॥ १८३ ॥\&[\smallbreak]


	
	    \end{quote}
	  
	  \endgroup
	

	  \pstart \leavevmode% starting standard par
	\hphantom{.}अतो‚{\color{DodgerBlue3}‚नुमान}‚हेतुत्वाद‚नुमान‚स्य लिङ्ग‚स्या‚{\color{DodgerBlue3}‚नुमेयार्थ}‚स्यान‚योरुप‚ल‚क्ष‚ण‚त्वात् (।)
	\pend% ending standard par
      

	  \pstart \leavevmode% starting standard par
	\hphantom{.}ध‚र्मिण‚श्च ‚{\color{DodgerBlue3}‚व्य‚व‚हार‚स्थितिस्त्विय}‚म‚नित्यः कृत‚क‚त्वादित्यादौ क्रिय‚माणा तेषां ‚{\tiny $_{lb}$}‚प‚र‚स्प‚र‚तो ‚{\color{DodgerBlue3}‚भेद‚{\tiny $_{3}$}‚ प्र‚त्य‚तेन} विक‚ल्प‚केनैक‚व्यावृत्तिमात्र‚विष‚येण ‚{\color{DodgerBlue3}‚संसिद्धं} निश्चितं । ‚{\tiny $_{lb}$}‚अव‚ल‚म्व्याश्रित्य च क‚ल्प्य‚ते ॥
	\pend% ending standard par
      
	  \bigskip
	  \begingroup
	
	    \large
	  
	    \begin{quote}
	  
	    
	    \stanza[\smallbreak]
	\label{pv.4.184}\flagstanza{\tiny\textenglish{....4.184}}य‚था स्वं भेद‚निष्ठेषु प्र‚त्य‚येषु विवेकिनः ।&ध‚र्मी ध‚र्माश्च भास‚न्ते व्य‚व‚हार‚स्त‚दाश्र‚यः ॥ १८४ ॥\&[\smallbreak]


	
	    \end{quote}
	  
	  \endgroup
	

	  \pstart \leavevmode% starting standard par
	\hphantom{.}त‚थाहि ‚{\color{DodgerBlue3}‚य‚था स्वं} य‚स्य य आत्मीयो ग्राह्यो ‚{\color{DodgerBlue3}‚भेदो} व्यावृत्तिस्त‚{\color{DodgerBlue3}‚न्निष्ठेषु} विक‚ल्पेषु ‚{\tiny $_{lb}$}‚‚{\color{DodgerBlue3}‚विवेकिनो ध‚र्मी ध‚र्माश्च} साध्य‚साध‚नाद‚यो ‚{\color{DodgerBlue3}‚भास‚न्ते । त‚दाश्र‚यो} विक‚ल्प‚गोच‚रा‚{\tiny $_{lb}$}‚श्र‚यो ध‚र्मिध‚र्मादिभेद‚स्य ‚{\color{DodgerBlue3}‚व्य‚व‚हारः} प्र‚व‚र्त्त‚ते ॥
	\pend% ending standard par
      \textsuperscript{\textenglish{480/s}}

	  \pstart \leavevmode% starting standard par
	एव‚न्त‚र्हि नित्यः श‚ब्दोऽस‚र्व्व‚त्वादिति । अत्रापि व्यावृत्तिभेदादे‚{\tiny $_{4}$}‚व साध्य‚{\tiny $_{lb}$}‚साध‚न‚भावो भ‚विष्य‚तीत्याह ।
	\pend% ending standard par
      
	  \bigskip
	  \begingroup
	
	    \large
	  
	    \begin{quote}
	  
	    
	    \stanza[\smallbreak]
	\label{pv.4.185}\flagstanza{\tiny\textenglish{....4.185}}व्य‚व‚हारोप‚नीतोत्र स एवाश्लिष्ट‚भेद‚धीः ।&साध्यः साध‚न‚तां नीत‚स्तेनासिद्धः प्र‚काशितः ॥ १८५ ॥\&[\smallbreak]


	
	    \end{quote}
	  
	  \endgroup
	

	  \pstart \leavevmode% starting standard par
	\hphantom{.}‚{\color{DodgerBlue3}‚अत्र} प्र‚योगे ‚{\color{DodgerBlue3}‚स} श‚ब्द ‚{\color{DodgerBlue3}‚एव साध्योऽश्लिष्ट‚भेद‚धी}‚र‚संस्पृष्टान्यादृश‚बुद्धिः । ‚{\tiny $_{lb}$}‚द्वाभ्याम‚पि श‚ब्दाभ्यामेक‚स्या व्यावृत्तेः प्र‚तिपाद‚नात् ‚{\color{DodgerBlue3}‚व्य‚व‚हारेण} व्यावृत्तिस‚माश्र‚{\tiny $_{lb}$}‚येणो‚{\color{DodgerBlue3}‚प‚नीतः} प्र‚त्युप‚स्थापितः येन कार‚णेन ‚{\color{DodgerBlue3}‚साध‚न‚तां नीतः तेनासिद्धः प्र‚काशितः} ।
	\pend% ending standard par
      

	  \pstart \leavevmode% starting standard par
	न‚नु संस्कृत‚श‚ब्दोऽनित्यः संस्कृत‚त्वादिति प्र‚तिज्ञार्थैक‚देश‚स्य य‚था‚{\tiny $_{5}$}‚ हेतुत्वं त‚था ‚{\tiny $_{lb}$}‚नित्यः श‚ब्दः श‚ब्द‚त्वादित्य‚स्यापि स्यादित्याह ।
	\pend% ending standard par
      
	  \bigskip
	  \begingroup
	
	    \large
	  
	    \begin{quote}
	  
	    
	    \stanza[\smallbreak]
	\label{pv.4.186}\flagstanza{\tiny\textenglish{....4.186}}भेद‚सामान्य‚योर्द्ध‚र्म‚भेदादंगांगिता त‚तः ।&य‚थाऽनित्यः प्र‚य‚त्नोत्थः प्र‚य‚त्नोत्थ‚त‚या ध्व‚निः ॥ १८६ ॥\&[\smallbreak]


	
	    \end{quote}
	  
	  \endgroup
	

	  \pstart \leavevmode% starting standard par
	\hphantom{.}साध्य‚ध‚र्मिमात्र‚निष्ठ‚त्वात् स‚र्व्व‚ध‚र्म‚गोच‚र‚त्वाच्च ‚{\color{DodgerBlue3}‚भेद‚सामान्य‚योर्द्ध‚र्म‚भेदाद्} व्यावृत्तिभेदात् । साध्य\edtext{}{\edlabel{pvv.480-1}\label{pvv.480-1}\lemma{साध्य}\Bfootnote{अनित्यं श‚ब्दे ।}}ध‚र्मो हि साध्य‚ध‚र्मिनिष्ठ‚त्वेन स‚जातीयाद् विजाती‚{\tiny $_{lb}$}‚याच्च व्यावृत्त‚त्वाद् विशेषः । साध‚न‚ध‚र्म‚स्तु विजातीय‚मात्र‚व्यावृत्त‚त्वेन सामान्यं । ‚{\tiny $_{lb}$}‚त‚तो भेद‚सामान्य‚भावेन ‚{\color{DodgerBlue3}‚भेदाद‚ङ्गाङ्गिता} हेतुसाध्य‚ता युक्ता । विशेषः सा‚{\tiny $_{6}$}‚ध्यः ‚{\tiny $_{lb}$}‚सामान्यं हेतुरिति कुतः प्र‚तिज्ञार्थैक‚देश‚ता । ‚{\color{DodgerBlue3}‚य‚थाऽनित्यः प्र‚य‚त्नोत्थो ध्व‚निरिति} प्र‚तिज्ञा ‚{\color{DodgerBlue3}‚प्र‚य‚त्नोत्थ‚त‚येति} हेतुः । श‚ब्दः पुन‚र‚भिन्न\edtext{}{\edlabel{pvv.480-2}\label{pvv.480-2}\lemma{भिन्न}\Bfootnote{नित्यः श‚ब्दः श‚ब्द‚त्त्वादित्य‚त्र ।}}विष‚यो हेतुः साध्य‚श्चेति ‚{\tiny $_{lb}$}‚प्र‚तिज्ञार्थैक‚देश‚त्वं ॥
	\pend% ending standard par
      

	  \pstart \leavevmode% starting standard par
	प्र‚य‚त्नान‚न्त‚रीय‚क‚त्व‚स्य ध‚र्मिविशेष‚ण‚त्वात् प्र‚तिज्ञार्थैक‚देश‚त्व‚म‚स्त्येवेत्याह ।
	\pend% ending standard par
      
	  \bigskip
	  \begingroup
	
	    \large
	  
	    \begin{quote}
	  
	    
	    \stanza[\smallbreak]
	\label{pv.4.187}\flagstanza{\tiny\textenglish{....4.187}}प‚क्षाङ्ग‚त्वेप्य‚बाध‚त्वान्नासिद्धिर्भिन्न‚ध‚र्मिणि ।&य‚थाश्वो न विषाणित्वादेष पिंडो विषाण‚वान् ॥ १८७ ॥\&[\smallbreak]


	
	    \end{quote}
	  
	  \endgroup
	\textsuperscript{\textenglish{97a/MA}}

	  \pstart \leavevmode% starting standard par
	प‚क्षाङ्ग‚त्वे विशेष‚ण‚त्वेपि नास्त्येव ताव‚त् साध‚न‚स्य प‚क्षाङ्ग‚त्वं विशेष‚साध्य‚{\tiny $_{lb}$}‚त्वात् । भ‚व‚तु वा त‚थापि नासिद्ध‚स्य । तेन विशेष‚णेन ‚{\color{DodgerBlue3}‚भिन्ने} विशेषिते ‚{\color{DodgerBlue3}‚ध‚र्मिणि} ध‚र्म्म्य‚न्त‚र‚व्या\edtext{}{\edlabel{pvv.480-3}\label{pvv.480-3}\lemma{व्या}\Bfootnote{साध्य‚स्य ध‚र्मिणो ध‚र्म‚स्य वा साध‚न‚त‚या विरोध एक‚स्य ज्ञाप्य‚ज्ञाप‚क‚त्व‚{\tiny $_{lb}$}‚विरोधात् (।) अत्र तु विशेषः साध्यः सामान्यं साध‚नं त‚त्त्वे प्र‚तिज्ञ‚या सिद्धि‚{\tiny $_{lb}$}‚प्र‚स‚क्तेः ।}}वृत्ते प्र‚य‚त्नान‚न्त‚रीय‚क‚त्व‚स्याविरोधाद‚बाध‚त्वात् ध‚र्मिणं विशेष‚य‚द‚पि ‚{\tiny $_{lb}$}‚\leavevmode\ledsidenote{\textenglish{481/s}} प्र‚य‚त्नोत्थ‚त्वं श‚ब्दे प्र‚सिद्ध‚मेव । ‚{\color{DodgerBlue3}‚य‚था} ब‚हुषु पिण्डेषु दृश्य‚मानेषु ‚{\color{DodgerBlue3}‚किम‚य‚म‚श्वो} न‚वे‚{\tiny $_{5}$}‚ति ‚{\tiny $_{lb}$}‚संश‚योऽभिधीय‚ते ‚{\color{DodgerBlue3}‚विशा( ? षा)ण‚वानेष पिण्डो} ध‚र्मी ‚{\color{DodgerBlue3}‚नाश्वो} विषाणित्वादिति । ‚{\tiny $_{lb}$}‚विषाणित्वं ध‚र्मिणं विशेष‚य‚दिप प्र‚माण‚प्र‚तीत‚त्वान्नासिद्धो हेतुः ।
	\pend% ending standard par
      
	  \bigskip
	  \begingroup
	
	    \large
	  
	    \begin{quote}
	  
	    
	    \stanza[\smallbreak]
	\label{pv.4.188}\flagstanza{\tiny\textenglish{....4.188}}साध्य‚कालाङ्ग‚ता वा न निवृत्तेरुप‚ल‚क्ष्य त‚त् ।&विशेषोपि प्र‚तिज्ञार्थो ध‚र्म‚भेदान्न युज्य‚ते ॥ १८८ ॥\&[\smallbreak]


	
	    \end{quote}
	  
	  \endgroup
	

	  \pstart \leavevmode% starting standard par
	अथ‚वा विशेष‚ण‚स्य प्र‚य‚त्नोत्थ‚त्वादेस्त‚च्छ‚ब्द‚रूपं विशेष्य‚त्वेनोप‚ल‚क्ष्यानुमानात् ‚{\tiny $_{lb}$}‚प्रागेव निवृत्तेः (।) ‚{\color{DodgerBlue3}‚साध्य‚का}‚लेऽनुमेय‚प्र‚तीते कालेऽ‚{\color{DodgerBlue3}‚ङ्ग‚ता} विशेष‚ण‚ता नास्त्येव ‚{\tiny $_{lb}$}‚अप्र‚तीतं ह्य‚नुमानात् प्र‚त्येत‚व्यं । प्र‚य‚त्नोत्थ‚त्वादि‚{\tiny $_{6}$}‚ति च प्र‚तीतं । न च त‚त्र ‚{\tiny $_{lb}$}‚विवादः । त‚तो ध‚र्मिमात्रोप‚ल‚क्ष‚ण‚न्त‚त् । य‚था काको देव‚द‚त्त‚गृहोप‚ल‚क्ष‚ण‚त्वान्न ‚{\tiny $_{lb}$}‚कार्योप‚योगी (।) य‚द्य‚पि\edtext{}{\edlabel{pvv.481-1}\label{pvv.481-1}\lemma{पि}\Bfootnote{य‚दि च ।}} विशिष्टे ध‚र्मिणि प्र‚तिज्ञार्थैक‚देश‚त्वं हेतोः । त‚था नित्यः ‚{\tiny $_{lb}$}‚श‚ब्दः श्राव‚ण‚त्वादिति श‚ब्द‚स्व‚भाव‚भूत‚स्य श्राव‚ण‚त्व‚स्य श‚ब्द‚त्व‚व‚त् प्र‚तिज्ञार्थैक‚{\tiny $_{lb}$}‚देश‚ता स्यात् नासाधार‚ण‚तेत्याह । न केव‚लं विषाणित्वादि‚{\color{DodgerBlue3}‚र्व्विशेषः} श्राव‚ण‚त्वादि\leavevmode\ledsidenote{\textenglish{97b/MA}}‚{\tiny $_{lb}$}‚र‚पि ‚{\color{DodgerBlue3}‚प्र‚तिज्ञार्थै}‚{\tiny $_{7}$}‚क‚देशो ‚{\color{DodgerBlue3}‚न युज्य‚ते । ध‚र्म}‚स्य व्यावृत्ते‚{\color{DodgerBlue3}‚र्भेदात्} । श्राव‚ण‚त्वं श्र‚व‚ण‚ग्रा‚{\tiny $_{lb}$}‚ह्य‚ताऽश्राव‚ण‚व्यावृत्तिः (।) त‚च्च श‚ब्देपि क्व‚चित् ‚{\color{DodgerBlue3}‚क‚ञ्चित् पुरुष‚म‚पेक्ष्य भ‚व‚ति} । ‚{\tiny $_{lb}$}‚अश‚ब्द‚व्यावृत्तिस्तु श‚ब्द‚त्वं त‚च्च स‚र्व्व‚त्रास्ति त‚तो व्यावृत्तिभेदात् श‚ब्दे सिद्ध‚स्य ‚{\tiny $_{lb}$}‚श्राव‚ण‚त्व‚स्यान्य‚त्रान‚नुवृत्तेर‚साधार‚ण‚तैव युक्तेत्युक्तं स‚प‚रिक‚रं प‚क्ष‚ल‚क्ष‚ण‚म् ॥
	\pend% ending standard par
      

	  \pstart \leavevmode% starting standard par
	इति प‚क्ष‚ल‚क्ष‚ण(म् । )
	\pend% ending standard par
      
	  
	% new div opening: depth here is 1
	
\chapter*[{६. हेतुचिन्ता}]{६. हेतुचिन्ता}

	  \begin{center}%% label @type='head'
	\textbf{(२) हेतुल‚क्ष‚ण‚म्}
	\end{center}
	

	  \begin{center}%% label @type='head'
	\textbf{म. प‚क्ष‚ध‚र्म‚प्र‚भेद‚क‚थ‚ने कार‚ण‚म्}
	\end{center}
	\label{div_pvv.4.189_4.190_4.191_4.192_4.193_4.194}
	  
	% new div opening: depth here is 2
	

	  \pstart \leavevmode% starting standard par
	\hphantom{.}हेतुल‚क्ष‚ण‚मिदानीम्व‚क्त‚व्यं । त‚त्र हेतुल‚क्ष‚ण‚मेव  त‚त्र यः स‚न् स‚जातीयेइत्या‚{\tiny $_{lb}$}‚दिकं युक्तं ‚{\tiny $_{1}$}‚ व‚क्तुं । स‚प‚क्षे स‚न्न‚स‚न् द्वेधा प‚क्ष‚ध‚र्मः पुन‚स्त्रिधेत्यादिना न‚व‚धा ‚{\tiny $_{lb}$}‚प‚क्ष‚ध‚र्म‚प्र‚भेद‚स्तु क‚स्मादुक्त \edtext{}{\edlabel{pvv.481-2}\label{pvv.481-2}\lemma{स्मादुक्त}\Bfootnote{स मु च्च ये ।}} इत्याह ।
	\pend% ending standard par
      
	  \bigskip
	  \begingroup
	
	    \large
	  
	    \begin{quote}
	  
	    
	    \stanza[\smallbreak]
	\label{pv.4.189}\flagstanza{\tiny\textenglish{....4.189}}प‚क्ष‚ध‚र्म‚प्र‚भेदेन सुख‚ग्र‚ह‚ण‚सिद्ध‚ये ।&हेतुप्र‚क‚र‚णार्थ‚स्य सूत्र‚संक्षेप उच्य‚ते ॥ १८९ ॥\&[\smallbreak]


	
	    \end{quote}
	  
	  \endgroup
	\textsuperscript{\textenglish{482/s}}

	  \pstart \leavevmode% starting standard par
	\hphantom{.}‚{\color{DodgerBlue3}‚प‚क्ष‚ध‚र्म}‚स्य न‚व‚धा ‚{\color{DodgerBlue3}‚प्र‚भेदेन हेतुप्र‚क‚र‚णार्थ‚स्य} हेतुहेत्वाभास‚ल‚क्ष‚णात्म‚क‚स्य ‚{\tiny $_{lb}$}‚‚{\color{DodgerBlue3}‚सुखेन ग्र‚ह‚ण}‚स्य ‚{\color{DodgerBlue3}‚सिद्ध‚ये} त‚द‚र्थ‚वाच‚कानां ‚{\color{DodgerBlue3}‚सूत्रा}‚णां ‚{\color{DodgerBlue3}‚संक्षेप‚तः} संग्र‚ह ‚{\color{DodgerBlue3}‚उच्य‚ते} ॥ स‚प‚क्षे ‚{\tiny $_{lb}$}‚स‚न्नित्यादिना (।)
	\pend% ending standard par
      

	  \begin{center}%% label @type='head'
	\textbf{ख. सूत्रे निपात‚ग्र‚ह‚ण‚फ‚ल‚म्}
	\end{center}
	

	  \pstart \leavevmode% starting standard par
	य‚दि प‚क्ष‚स्य ध‚र्मो हेतुस्त‚दा त‚द्विशेष‚णापेक्ष‚स्य ध‚र्म‚स्यान्य‚त्र ध‚र्मिण्य‚न‚नुवृत्ते‚{\tiny $_{lb}$}‚र‚साधार‚ण‚ता स्या‚{\tiny $_{2}$}‚त् । अथ प‚क्षेण न विशिष्य‚ते त‚दा न प‚क्ष‚ध‚र्मो हेतुः स्यात् । ‚{\tiny $_{lb}$}‚अस‚देत‚त् । न ह्य‚न्य‚योग‚व्य‚व‚च्छेदेनैव विशेष‚ण‚म‚न्य‚थापि स‚म्भ‚वादिति द‚र्श‚यितुमाह ।
	\pend% ending standard par
      
	  \bigskip
	  \begingroup
	
	    \large
	  
	    \begin{quote}
	  
	    
	    \stanza[\smallbreak]
	\label{pv.4.190}\flagstanza{\tiny\textenglish{....4.190}}अयोगं योग‚म‚प‚रैर‚त्य‚न्तायोग‚मेव च ।&व्य‚व‚च्छिन‚त्ति ध‚र्म‚स्य निपातो व्य‚तिरेच‚कः ॥ १९० ॥\&[\smallbreak]


	
	    \end{quote}
	  
	  \endgroup
	

	  \pstart \leavevmode% starting standard par
	\hphantom{.}‚{\color{DodgerBlue3}‚निपात} एव‚कारो ‚{\color{DodgerBlue3}‚व्य‚तिरेच‚कः} नियाम‚कः क्व‚चिद् ‚{\color{DodgerBlue3}‚ध‚र्म‚स्य} विशेष‚ण‚स्या‚{\tiny $_{lb}$}‚‚{\color{DodgerBlue3}‚योगं व्य‚व‚च्छिन‚त्ति} क्व‚चिद‚प‚रैर्व्विशेष्या‚{\color{DodgerBlue3}‚द‚न्यैर्योगं} व्य‚व‚च्छिन‚त्ति । क्व‚चि‚{\color{DodgerBlue3}‚द‚त्य‚न्तायोगं} व्य‚व‚च्छिन‚त्ति ॥
	\pend% ending standard par
      

	  \pstart \leavevmode% starting standard par
	न‚नु निपातो न स्व‚यं वाच‚कः किन्तु द्योत‚कः । त‚द‚स्य क‚थ‚{\tiny $_{3}$}‚म‚य‚म‚र्थ‚प्र‚भेद ‚{\tiny $_{lb}$}‚इत्याह ।
	\pend% ending standard par
      
	  \bigskip
	  \begingroup
	
	    \large
	  
	    \begin{quote}
	  
	    
	    \stanza[\smallbreak]
	\label{pv.4.191a}\flagstanza{\tiny\textenglish{...4.191a}}विशेष‚ण‚विशेषाभ्यां क्रिय‚या च स‚होदितः ।\&[\smallbreak]


	
	    \end{quote}
	  
	  \endgroup
	

	  \pstart \leavevmode% starting standard par
	\hphantom{.}द्योत‚क‚त्वादेव निपातो ‚{\color{DodgerBlue3}‚विशेष‚णेन} स‚होदितोऽयोग‚स्य व्य‚व‚च्छेद‚कः । ‚{\color{DodgerBlue3}‚विशे‚{\tiny $_{lb}$}‚ष्येण} स‚होक्तोन्य‚योग‚स्य । ‚{\color{DodgerBlue3}‚क्रिय‚या च स‚होक्तो}‚ऽत्य‚न्तायोग‚स्येति विशेष‚णादि‚{\tiny $_{lb}$}‚प‚द‚वाच्य एवायोग‚व्य‚व‚च्छेदादिस्त‚त्स‚होक्त‚निपात‚द्योत्य इत्य‚र्थः ।
	\pend% ending standard par
      

	  \pstart \leavevmode% starting standard par
	भ‚व‚तु ताव‚न्निपात‚प्र‚योगे व्य‚व‚च्छेद‚विशेष‚स्य प्र‚तीतिः प‚क्ष\edtext{}{\edlabel{pvv.482-1}\label{pvv.482-1}\lemma{क्ष}\Bfootnote{...प‚क्ष‚ध‚र्मः ।}} इत्यादौ तु ‚{\tiny $_{lb}$}‚क‚थ‚मित्याह ।
	\pend% ending standard par
      
	  \bigskip
	  \begingroup
	
	    \large
	  
	    \begin{quote}
	  
	    
	    \stanza[\smallbreak]
	\label{pv.4.191b}\flagstanza{\tiny\textenglish{...4.191b}}विव‚क्षातोऽप्र‚योगेपि स‚र्वोऽर्थोयं प्र‚तीय‚ते ॥ १९१ ॥\&[\smallbreak]


	
	    \end{quote}
	  
	  \endgroup
	
	  \bigskip
	  \begingroup
	
	    \large
	  
	    \begin{quote}
	  
	    
	    \stanza[\smallbreak]
	\label{pv.4.192}\flagstanza{\tiny\textenglish{....4.192}}व्य‚व‚च्छेद‚फ‚लं वाक्यं य‚त‚श्चैत्रो ध‚नुर्ध‚रः ।&पार्थो ध‚नुर्द्ध‚रो नीलं स‚रोज‚मिति वा य‚था ॥ १९२ ॥\&[\smallbreak]


	
	    \end{quote}
	  
	  \endgroup
	

	  \pstart \leavevmode% starting standard par
	\hphantom{.}‚{\color{DodgerBlue3}‚अप्र‚योगेपि} निपात‚{\tiny $_{4}$}‚स्य व‚क्तु‚{\color{DodgerBlue3}‚र्व्विव‚क्षातः} स‚र्व्वोंय‚म‚योग‚व्य‚व‚च्छेदा‚{\color{DodgerBlue3}‚दिर‚र्थः ‚{\tiny $_{lb}$}‚प्र‚तीय‚ते । य‚तो व्य‚व‚च्छेद‚फ‚लं वाक्य}‚मित्युक्तं प्राक् । वाक्य‚ञ्चोप‚ल‚क्ष‚णं प‚द‚म‚पि ‚{\tiny $_{lb}$}‚व्य‚व‚च्छेद‚फ‚लं । न हि घ‚टेनोद‚क‚मान‚येति प्र‚तिपाद‚म‚न‚व‚धार‚णेऽघ‚टेनान‚य‚न‚प्र‚तिषेधः । ‚{\tiny $_{lb}$}‚प‚क्ष इत्यादौ तु क‚थ‚मित्याह । अप्र‚योगेपि निपात‚स्य व‚क्तुर्व्विव‚क्षातः स‚र्व्वोय‚म‚{\tiny $_{lb}$}‚योग‚व्य‚व‚च्छेदोऽनुद‚कान‚य‚न‚प्र‚तिषेधः ‚{\tiny $_{5}$}‚ अनान‚य‚न‚निवृत्तिर्व्वा श‚क्योप‚द‚र्श‚ना । अयोग‚{\tiny $_{lb}$}‚\leavevmode\ledsidenote{\textenglish{483/s}} व्य‚व‚च्छेदादीनामुदाह‚र‚ण‚माह । ‚{\color{DodgerBlue3}‚य‚था चैत्रो ध‚नुर्द्ध‚रः । पार्थो ध‚नुर्द्ध‚रः । नीलं ‚{\tiny $_{lb}$}‚स‚रोज‚मिति} (।) चैत्रे ध‚नुर्द्ध‚र‚त्व‚स‚न्देहाद् विशेष‚णेनायोग‚मात्रं व्य‚व‚च्छिद्य‚ते । ‚{\color{DodgerBlue3}‚पार्थे} ध‚नुर्द्ध‚र‚त्वं प्र‚सिद्ध‚मेव (।) किन्तु तादृश‚म‚न्य‚स्यापि किम‚स्तीति स‚न्देहेऽन्य‚{\tiny $_{lb}$}‚योग‚व्य‚व‚च्छेद‚फ‚लं विशेष‚णं । न ख‚लु स‚र्व्व‚मेव नीलं स‚रोजं येना‚{\tiny $_{6}$}‚योग‚व्य‚व‚च्छेदः ‚{\tiny $_{lb}$}‚स्यात् । नापि स‚रोज‚मेव नीलं येनान्य‚योग‚व्य‚व‚च्छेदो भ‚वेत् । किन्तु नीलं स‚रोजं ‚{\tiny $_{lb}$}‚संभ‚व‚ति न वेत्य‚न्तायोग‚संदेहे विशेष‚णेन स एव व्य‚व‚च्छिद्य‚ते ।
	\pend% ending standard par
      

	  \pstart \leavevmode% starting standard par
	न‚नु भ‚व‚तु ताव‚च्चैत्रो ध‚नुर्द्ध‚र एव । पार्थ एव ध‚नुर्द्ध‚रः । स‚रोजं नीलं संभ‚व‚{\tiny $_{lb}$}‚त्येवेति निपात‚प्र‚योगे विव‚क्षाव‚शात् विशेष‚णादिप‚दानामेव वा व्य‚व‚च्छेद‚प्र‚तिपा‚{\tiny $_{lb}$}‚द‚क‚त्वाद‚प्र‚योगेऽपि नि‚{\tiny $_{7}$}‚पात‚स्य चैत्रो ध‚नुर्द्ध‚र इत्यादिप्र‚योगेषु योग‚व्य‚व‚च्छेदादीनां \leavevmode\ledsidenote{\textenglish{98a/MA}} ‚{\tiny $_{lb}$}‚प्र‚तीतिः । प‚क्ष‚ध‚र्म इत्य‚त्र पुनःप‚क्षो विशेष‚णं । ध‚र्मो विशेष्यः । त‚द् य‚दि प‚क्ष‚स्यैव ‚{\tiny $_{lb}$}‚ध‚र्म इति विशेष‚णेन स‚ह निपात उच्य‚ते अप्र‚योगेपि वा विशेष‚ण‚स्य त‚द‚र्थ‚वृत्तितेष्य‚ते । ‚{\tiny $_{lb}$}‚उप(?) भ‚य‚थाप्य‚न्य‚योग‚व्य‚व‚च्छेद‚प्र‚तिपाद‚क‚त्व‚मेव स्यात् ।
	\pend% ending standard par
      

	  \pstart \leavevmode% starting standard par
	अथ ध‚र्म एव विशेष्यो निपात‚स‚ह‚च‚रः । त‚द‚र्थ वृत्तिर्व्वेष्टः त‚दाऽयोग‚व्य‚व‚{\tiny $_{lb}$}‚च्छेदो ल‚भ्य‚त एव प‚रं ‚{\tiny $_{1}$}‚ विशेष‚ण‚स‚हितोऽन्य‚योगं व्य‚व‚च्छिन‚त्तीत्युक्तेर्व्विरुध्य‚ते । ‚{\tiny $_{lb}$}‚अत्रोच्य‚ते । बाह्यो निपातः श्रूय‚माणोपि विशेष्यादिभिः स‚ह व‚क्तृविव‚क्षाव‚शाद् ‚{\tiny $_{lb}$}‚ग‚म्य‚मानो वाऽयोग‚व्य‚व‚च्छेद‚को भ‚व‚तीति दृष्टान्त‚प्र‚द‚र्श‚नार्थ‚मुक्तं । त‚थैव चैत्रो ध‚नु‚{\tiny $_{lb}$}‚र्द्ध‚र इत्याद्युदाह‚र‚ण‚प्र‚द‚र्श‚नात् । व्य‚व‚च्छेद‚फ‚लं वाक्य‚मित्युक्तेश्च । स‚मासे तु ‚{\tiny $_{lb}$}‚विशेष‚ण‚मेवायोगादिव्य‚व‚च्छेद‚कं विव‚क्षाव‚शादिष्टं । अयोग‚व्य‚व‚च्छे‚{\tiny $_{2}$}‚देन विशेष‚{\tiny $_{lb}$}‚णादित्युक्तेः । प‚क्ष‚ध‚र्म इति प‚क्ष‚श‚ब्दोऽयोग‚व्य‚व‚च्छेद‚कः । प‚क्षास‚म्ब‚द्धो न भ‚व‚{\tiny $_{lb}$}‚तीत्य‚र्थः । चाक्षुषं रूप‚मिति चाक्षुष‚त्व‚स्य रूपे विवादाभावात् (।) श‚ब्दादीनां ‚{\tiny $_{lb}$}‚विशेष‚णेन व्य‚व‚च्छेदः क्रिय‚ते । नीलोत्प‚ल‚मिति । उत्प‚ले नील‚त्व‚निय‚माभावात् (।) ‚{\tiny $_{lb}$}‚प‚रेष्व‚भावाद‚योगान्य‚योग‚योर्व्य‚व‚च्छेदाभावात् नील‚श‚ब्देनात्य‚न्तास‚म्भ‚व‚मात्रं व्य‚व‚{\tiny $_{lb}$}‚च्छिद्य‚ते । इति न क‚श्चिद्विरोधः ।
	\pend% ending standard par
      

	  \pstart \leavevmode% starting standard par
	न‚नु य‚{\tiny $_{3}$}‚था पार्थ एव ध‚नुर्द्ध‚र इति विशेष‚ण‚स्य स‚न्निधानात् निपात‚स्य विशे‚{\tiny $_{lb}$}‚ष्यान्त‚र‚व्य‚व‚च्छेदः । त‚था विशेष‚ण‚स‚न्निधानाद‚व‚धार‚ण‚स्य चैत्रो ध‚नुर्द्ध‚र एवेति ‚{\tiny $_{lb}$}‚गुणान्त‚र‚व्य‚व‚च्छेदः स्यादित्याह ।
	\pend% ending standard par
      
	  \bigskip
	  \begingroup
	
	    \large
	  
	    \begin{quote}
	  
	    
	    \stanza[\smallbreak]
	\label{pv.4.193}\flagstanza{\tiny\textenglish{....4.193}}प्र‚तियोगिव्य‚व‚च्छेद‚स्त‚त्राप्य‚र्थेषु ग‚म्य‚ते ।&त‚था प्र‚सिद्धेः साम‚र्थ्याद् विव‚क्षानुग‚माद् ध्व‚नेः ॥ १९३ ॥\&[\smallbreak]


	
	    \end{quote}
	  
	  \endgroup
	

	  \pstart \leavevmode% starting standard par
	\hphantom{.}‚{\color{DodgerBlue3}‚त‚त्र} विशेष‚णादिष्व‚{\color{DodgerBlue3}‚र्थेषु} व्य‚व‚च्छेदेपि क्रिय‚माणे प्र‚क‚र‚णाद् बुद्धिविष‚यीकृत‚स्य ‚{\tiny $_{lb}$}‚‚{\color{DodgerBlue3}‚प्र‚तियोगि}‚न ‚{\color{DodgerBlue3}‚व्य‚व‚च्छेदो} विशेष‚णेन ‚{\color{DodgerBlue3}‚ग‚म्य‚ते} नान्य‚स्य । ‚{\color{DodgerBlue3}‚त‚था प्र‚सिद्धेः} प्र‚तियोगिन ‚{\tiny $_{lb}$}‚\leavevmode\ledsidenote{\textenglish{484/s}} एव बुद्धिस्थीकृत्य विशेष‚णे‚{\tiny $_{4}$}‚न व्य‚व‚च्छेदो नेत‚र‚स्येति लोक‚प्र‚सिद्धेः । ‚{\color{DodgerBlue3}‚विव‚क्षाया ‚{\tiny $_{lb}$}‚अनुग‚मात् ध्व‚ने}‚र्व्य‚व‚च्छेदादौ ‚{\color{DodgerBlue3}‚साम‚र्थ्या}‚न्नाविव‚क्षित‚व्य‚व‚च्छेदः । य‚त एवायोग‚व्य‚व‚{\tiny $_{lb}$}‚च्छेदोप्य‚स्ति (।)
	\pend% ending standard par
      
	  \bigskip
	  \begingroup
	
	    \large
	  
	    \begin{quote}
	  
	    
	    \stanza[\smallbreak]
	\label{pv.4.194}\flagstanza{\tiny\textenglish{....4.194}}त‚द‚योग‚व्य‚व‚च्छेदाद् ध‚र्मी-ध‚र्म‚विशेष‚ण‚म् ।&त‚द्विशिष्ट‚त‚या ध‚र्मो न निर‚न्व‚य‚दोष‚भाक् ॥ १९४ ॥\&[\smallbreak]


	
	    \end{quote}
	  
	  \endgroup
	

	  \pstart \leavevmode% starting standard par
	\hphantom{.}‚{\color{DodgerBlue3}‚त‚त्} त‚स्माद् ध‚र्मी प‚क्षो ध‚र्म‚स्या‚{\color{DodgerBlue3}‚योग‚व्य‚व‚च्छेदाद् विशेष‚णं} ध‚र्मिणो नाध‚र्मो\edtext{}{\edlabel{pvv.484-1}\label{pvv.484-1}\lemma{र्मो}\Bfootnote{अपि तु ध‚र्म एव ।}} ‚{\tiny $_{lb}$}‚हेतुरित्य‚र्थः । अयोग‚व्य‚व‚च्छेदात् तेन ध‚र्मिणा ‚{\color{DodgerBlue3}‚विशिष्ट‚त‚या ध‚र्मो निर‚न्व‚य‚दोष‚भाग् ‚{\tiny $_{lb}$}‚न} भ‚व‚ति ।
	\pend% ending standard par
      

	  \pstart \leavevmode% starting standard par
	स‚प‚क्षे घ‚टादौ स‚न् श‚ब्दानित्य‚त्वे साध्ये‚{\tiny $_{5}$}‚ कृत‚क‚त्वं हेतुः । अस‚न् स‚प‚क्षे व्योमादौ ‚{\tiny $_{lb}$}‚श‚ब्दानित्य‚त्वे साध्ये कृत‚क‚त्वं हेतुः । स‚प‚क्षे द्वेधा । स‚न्न‚सँश्च (।) श‚ब्दानित्य‚त्वे ‚{\tiny $_{lb}$}‚साध्ये य‚त्न‚ज‚त्वं हेतुः । घ‚टादौ स‚प‚क्षे स‚न् विद्युदादौ चास‚न् । पुन‚स्त्रिधा (।) ‚{\tiny $_{lb}$}‚स‚प‚क्षे स‚न् अस‚न् स‚द‚सँश्च (।) श‚ब्द‚स्य य‚त्न‚ज‚त्वे साध्ये स‚प‚क्षे घ‚टादाव‚नित्य‚त्वं ‚{\tiny $_{lb}$}‚हेतुः स‚न् । श‚ब्दानित्य‚त्वे साध्ये य‚त्न‚ज‚त्वं हेतुर‚स‚न् विद्युदादौ स‚प‚क्षे ॥ श‚ब्दे‚{\tiny $_{lb}$}‚ऽ‚{\tiny $_{6}$}‚य‚त्न‚ज‚त्वे साध्येऽनित्य‚त्वं हेतुः स‚प‚क्षे विद्युदादौ स‚न् । व्योमादौ चास‚न् । एवं ‚{\tiny $_{lb}$}‚प्र‚त्येक‚म‚स‚प‚क्षेपि स‚न् अस‚न् द्वेधा चेति योज्यं । श‚ब्द‚नित्य‚त्वे साध्ये प्र‚मेय‚त्वं हेतुः (।) ‚{\tiny $_{lb}$}‚अस‚प‚क्षे घ‚टादौ स‚न् (।) श‚ब्द‚नित्य‚त्वे साध्ये श्राव‚ण‚त्वं हेतुर‚स‚प‚क्षेऽस‚न् श‚ब्द‚{\tiny $_{lb}$}‚नित्य‚त्वे साध्येऽस्प‚र्श‚व‚त्वं हेतुर‚स‚प‚क्षे घ‚टादाव‚स‚न् (।) बुद्ध्यादौ स‚न्निति द्विधा ‚{\tiny $_{lb}$}‚\leavevmode\ledsidenote{\textenglish{98b/MA}} प‚क्ष‚ध‚र्म‚निर्देशः । किम‚र्थ हे‚{\tiny $_{7}$}‚तुप्र‚क‚र‚णे न‚व‚धा प‚क्ष‚ध‚र्म‚निर्देशः ।---
	\pend% ending standard par
      \label{div_pvv.4.195_4.196_4.197_4.198}
	  
	% new div opening: depth here is 2
	

	  \begin{center}%% label @type='head'
	\textbf{(२) हेतुभेदा}
	\end{center}
	

	  \pstart \leavevmode% starting standard par
	---स‚म्य‚ग्धेतुर‚सिद्ध‚विरुद्धानैकान्तिक‚हेत्वाभासा एव युक्त‚निर्देशा इत्याह ।
	\pend% ending standard par
      
	  \bigskip
	  \begingroup
	
	    \large
	  
	    \begin{quote}
	  
	    
	    \stanza[\smallbreak]
	\label{pv.4.195}\flagstanza{\tiny\textenglish{....4.195}}स्व‚भाव‚कार्य‚सिध्य‚र्थं द्वौ द्वौ हेतुविप‚र्य‚यौ ।&विवादाद् भेद‚सामान्ये शेषो व्यावृत्तिसाध‚नः ॥ १९५ ॥\&[\smallbreak]


	
	    \end{quote}
	  
	  \endgroup
	

	  \pstart \leavevmode% starting standard par
	\hphantom{.}‚{\color{DodgerBlue3}‚स्व‚भाव‚कार्य}‚योरेव हेतुत्वेन ‚{\color{DodgerBlue3}‚सिद्ध्य‚र्थं} त‚त्र ‚{\color{DodgerBlue3}‚द्वौ} श‚ब्देऽनित्य‚त्व‚सिद्ध्य‚र्थं कृत‚क‚त्व‚{\tiny $_{lb}$}‚प्र‚य‚त्नान‚न्त‚रीय‚क‚त्वाख्यौ हेतू निर्दिष्टौ । त‚था श‚ब्द एव नित्य‚त्व‚साध‚ने द्वौ ‚{\color{DodgerBlue3}‚हेतु‚{\tiny $_{lb}$}‚विप‚र्य‚यौ} विरुद्धौ हेतुभावे चोक्तौ । य‚था हि स‚म्य‚ग्धेतोः स्व‚साध्ये व्याप्य‚कार्य‚{\tiny $_{lb}$}‚त‚या प्र‚तिब‚द्ध‚स्य ग‚म‚क‚{\tiny $_{1}$}‚त्वं । त‚था साध्य‚विप‚र्य‚ये व्याप्य‚कार्य‚त‚या प्र‚तिब‚द्ध‚स्यैव ‚{\tiny $_{lb}$}‚\leavevmode\ledsidenote{\textenglish{485/s}} त‚द्ग‚म‚क‚त्वेन विरुद्ध‚ता नान्य‚स्येत्य‚र्थः । व्य‚तिरेकी\edtext{}{\edlabel{pvv.485-1}\label{pvv.485-1}\lemma{तिरेकी}\Bfootnote{नैयायिक‚स्य ।}}अन्व‚यी च हेतुरिति प‚रेषां ‚{\color{DodgerBlue3}‚विवा‚{\tiny $_{lb}$}‚दात्} त‚त्प्र‚तिषेधार्थं ‚{\color{DodgerBlue3}‚भेद‚सामान्ये}‚ऽसाधार‚ण‚साधार‚णे श्राव‚ण‚त्व‚प्र‚मेय‚त्वे निर्द्दिष्टे । य‚दि ‚{\tiny $_{lb}$}‚विप‚क्षे नास्तीति सात्म‚क‚त्वे साध्ये प्राणादिम‚त्वं हेतुस्त‚दा श्राव‚ण‚त्व‚म‚पि स्यात् । ‚{\tiny $_{lb}$}‚न चैवं (।) त‚स्मान्न व्य‚तिरेकी हेतुः । य‚दि च केव‚लान्व‚यिनो द‚र्श‚न‚मात्राध्य‚व‚सिता‚{\tiny $_{lb}$}‚व्य‚भिचा‚{\tiny $_{2}$}‚र‚स्य हेतुत्वं त‚दा प्र‚मेय‚त्व‚स्याकाशादौ द‚र्श‚नाद‚व्य‚भिचार‚निश्च‚ये स‚ति ‚{\tiny $_{lb}$}‚श‚ब्दे नित्य‚त्व‚ग‚म‚क‚त्वं स्यान्न चैवं । त‚तो न केव‚लान्व‚यी हेतुः । ‚{\color{DodgerBlue3}‚शेषो}‚ऽप्र‚य‚त्नोत्थः ‚{\tiny $_{lb}$}‚श‚ब्दोऽनित्य‚त्वात् । नित्यः श‚ब्दोऽस्प‚र्श‚व‚त्वात् । प्र‚य‚त्नान‚न्त‚रीयः श‚ब्दोऽनित्य‚त्वा‚{\tiny $_{lb}$}‚दिति हेतुत्र‚यं विप‚क्षाद्धेतो‚{\color{DodgerBlue3}‚र्व्यावृत्तिसाध‚नः} ।---
	\pend% ending standard par
      

	  \pstart \leavevmode% starting standard par
	\hphantom{.}---य‚दि हि स‚प‚क्षे द‚र्श‚न‚मात्रेण ग‚म‚क‚त्वं त‚दैते हेत‚वः प्राप्ताः (।) ‚{\color{DodgerBlue3}‚स‚र्व्वेषां स‚प‚क्षे} स‚त्वात्‚{\tiny $_{3}$}‚ (।) त‚स्मान्नान्व‚य‚स‚म्ब‚न्ध‚मात्रेण ग‚म‚क‚त्वं (।) किन्तु विप‚क्षाद् व्य‚तिरेक‚{\tiny $_{lb}$}‚निश्च‚ये न चास्ति व्य‚तिरेक‚निश्च‚यः (।) प्र‚धानं हेतुत्व‚निब‚न्ध‚न‚मित्य‚र्थः । क‚थं ‚{\tiny $_{lb}$}‚पुन‚र्ज्ञाय‚ते स्व‚भाव‚हेतुः कृत‚क‚त्वं । कार्य‚हेतुः प्र‚य‚त्नान‚न्त‚रीय‚क‚मित्याह ।
	\pend% ending standard par
      
	  \bigskip
	  \begingroup
	
	    \large
	  
	    \begin{quote}
	  
	    
	    \stanza[\smallbreak]
	\label{pv.4.196}\flagstanza{\tiny\textenglish{....4.196}}न हि स्व‚भावाद‚न्येन व्याप्तिर्ग‚म्य‚स्य कार‚णे ।&स‚म्भ‚वाद् व्य‚भिचार‚स्य द्विधावृत्तिफ‚लं त‚तः ॥ १९६ ॥\&[\smallbreak]


	
	    \end{quote}
	  
	  \endgroup
	

	  \pstart \leavevmode% starting standard par
	\hphantom{.}‚{\color{DodgerBlue3}‚न हि स्व‚भावा}‚द्धेतो‚{\color{DodgerBlue3}‚र‚न्येन} हेतुना ‚{\color{DodgerBlue3}‚ग‚म्य‚स्य} साध्य‚स्य ‚{\color{DodgerBlue3}‚व्याप्तिः} । य‚त्र य‚त्रा‚{\tiny $_{lb}$}‚नित्य‚त्वं त‚त्र कृत‚क‚त्व‚मिति दृश्य‚ते च व्याप्तिः (।) त‚स्मात् स्व‚भाव‚हेतुरेव\edtext{}{\edlabel{pvv.485-2}\label{pvv.485-2}\lemma{हेतुरेव}\Bfootnote{कार‚णेपि व्याप्तिः स्यादित्याह ।}} । ‚{\tiny $_{lb}$}‚नाव‚श्यं कार‚णानि कार्य‚व‚न्ति भ‚व‚न्तीति कार‚णे कार्य‚{\tiny $_{4}$}‚स्य ‚{\color{DodgerBlue3}‚व्य‚भिचार‚स‚म्भ‚वात्} । ‚{\tiny $_{lb}$}‚कार्यं कार‚ण‚व्याप‚कं न भ‚व‚ति । \edtext{\textsuperscript{*}}{\edlabel{pvv.485-3}\label{pvv.485-3}\lemma{*}\Bfootnote{स्व‚भाव‚हेतौ ।}} त‚त्र स‚प‚क्षे ‚{\color{DodgerBlue3}‚द्विधावृत्ति}‚भावाभावात् प्र‚त्य‚त्ना\edtext{}{\edlabel{pvv.485-4}\label{pvv.485-4}\lemma{त्ना}\Bfootnote{श‚ब्दोच्चार‚ण‚प्र‚य‚त्नान‚न्त‚र‚जं श‚ब्दाल‚म्ब‚नं ज्ञानं ।}}न‚न्त‚री‚{\tiny $_{lb}$}‚य‚त्वं ‚{\color{DodgerBlue3}‚फ‚लं} कार्य‚हेतुः\edtext{}{\edlabel{pvv.485-5}\label{pvv.485-5}\lemma{हेतुः}\Bfootnote{भ‚वार्थे ग‚हादित्वाच्छः प्र‚य‚त्नान‚न्त‚रीय‚क‚त्वं घ‚टादौ स‚त् विद्युदादौ चास‚दिति स‚प‚क्षे द्विधावृत्तिर्य‚तः ।}} प्र‚य‚त्न‚कार्य‚स्य त‚थाभिधानात् ।
	\pend% ending standard par
      

	  \pstart \leavevmode% starting standard par
	न‚नु प्र‚य‚त्न‚जेन ज्ञानेनानित्यः श‚ब्दोऽनुमेय इष्टः (।) श‚ब्दाश्च नित्या एव ‚{\tiny $_{lb}$}‚त‚द्विष‚य‚ज्ञानोत्पादादिना य‚त्नेन ते व्य‚ज्य‚न्ते(।)त‚त्क‚थं कार्य‚हेतूदाह‚र‚ण‚मिद‚मित्याह ।
	\pend% ending standard par
      
	  \bigskip
	  \begingroup
	
	    \large
	  
	    \begin{quote}
	  
	    
	    \stanza[\smallbreak]
	\label{pv.4.197}\flagstanza{\tiny\textenglish{....4.197}}प्र‚य‚त्नान‚न्त‚रं ज्ञानं प्राक्स‚तो निय‚मेन न ।&त‚स्यावृत्य‚क्ष‚श‚ब्देषु स‚र्व‚थाऽनुप‚योग‚तः ॥ १९७ ॥\&[\smallbreak]


	
	    \end{quote}
	  
	  \endgroup
	

	  \pstart \leavevmode% starting standard par
	\hphantom{.}प्र‚य‚त्नात् ‚{\color{DodgerBlue3}‚प्राक्} स‚तः श‚ब्द‚स्य ‚{\color{DodgerBlue3}‚निय‚मेन प्र‚य‚त्नान‚न्त‚रं ज्ञानं‚{\tiny $_{5}$}‚ न} य‚ज्य‚ते । प्र‚य‚त्नं ‚{\tiny $_{lb}$}‚विनापि क‚दाचिदुप‚ल‚भ्येत । न च प्र‚य‚त्न‚व्य‚ज्य‚ता श‚ब्दानां युक्ता । ‚{\color{DodgerBlue3}‚त‚स्य} प्र‚य‚त्न‚स्य ‚{\tiny $_{lb}$}‚\leavevmode\ledsidenote{\textenglish{486/s}} ‚{\color{DodgerBlue3}‚आवृता}‚वुप‚ल‚म्भाव‚र‚णे श‚ब्द‚विष‚य‚ज्ञान‚ज‚न‚के श्रोत्रे ‚{\color{DodgerBlue3}‚श‚ब्देषु} च विष‚येषु ‚{\color{DodgerBlue3}‚स‚र्व्व‚था}‚{\tiny $_{lb}$}‚ऽकिञ्चित्क‚र‚त्वे‚{\color{DodgerBlue3}‚नानुप‚योग‚त} इत्युक्तं प्राक्\edtext{}{\edlabel{pvv.486-1}\label{pvv.486-1}\lemma{प्राक्}\Bfootnote{श्रुतिप‚रीक्षायां ।}} ।
	\pend% ending standard par
      

	  \pstart \leavevmode% starting standard par
	किञ्च (।)
	\pend% ending standard par
      
	  \bigskip
	  \begingroup
	
	    \large
	  
	    \begin{quote}
	  
	    
	    \stanza[\smallbreak]
	\label{pv.4.198}\flagstanza{\tiny\textenglish{....4.198}}क‚दाचिन्निर‚पेक्ष‚स्य कार्याऽकृतिविरोध‚तः ।&कादाचित्क‚फ‚लं सिद्धं त‚ल्लिङ्गं ज्ञान‚मीदृश‚म् ॥ १९८ ॥\&[\smallbreak]


	
	    \end{quote}
	  
	  \endgroup
	

	  \pstart \leavevmode% starting standard par
	\hphantom{.}स‚ह‚कारिभिर‚नाधेयातिश‚य‚त्वेन्नित्य‚स्य ‚{\color{DodgerBlue3}‚क‚दाचित्} प्र‚य‚त्न‚काले कार्य‚स्य ज्ञान‚स्य‚{\tiny $_{lb}$}‚(ा)कृतिविरोध‚तः कार‚णात् त‚च्छ्रुतिविष‚यं ज्ञानं ‚{\color{DodgerBlue3}‚कादाचित्क}‚स्यानित्य‚स्य श‚ब्द‚स्य‚{\tiny $_{6}$}‚ ‚{\tiny $_{lb}$}‚‚{\color{DodgerBlue3}‚फ‚लं लिङ्गं} कार्य‚{\color{DodgerBlue3}‚मीदृशं} निय‚मेन प्र‚य‚त्नान‚न्त‚र‚भावि ‚{\color{DodgerBlue3}‚सिद्ध‚म्} ॥ (१९८)
	\pend% ending standard par
      \label{div_pvv.4.199}
	  
	% new div opening: depth here is 2
	

	  \begin{center}%% label @type='head'
	\textbf{(३) कार्य‚स्व‚भाव‚हेत्वोर्निर्देश‚स्य फ‚ल‚म्}
	\end{center}
	
	  \bigskip
	  \begingroup
	
	    \large
	  
	    \begin{quote}
	  
	    
	    \stanza[\smallbreak]
	\label{pv.4.199}\flagstanza{\tiny\textenglish{....4.199}}एताव‚तैव सिद्धेपि स्व‚भाव‚स्य पृथ‚क् कृतिः ।&कार्येण स‚ह निर्द्देशे मा ज्ञासीत् स‚र्व‚मीदृश‚म् ॥ १९९ ॥\&[\smallbreak]


	
	    \end{quote}
	  
	  \endgroup
	

	  \pstart \leavevmode% starting standard par
	एताव‚ता प्र‚य‚त्नान‚न्त‚रीय‚क‚त्वेनैव स्व‚भाव‚हेतुनिर्देशेपि सिद्धे प्र‚य‚त्नान‚न्त‚र‚{\tiny $_{lb}$}‚मुत्पाद‚स्याभिव्य‚क्तेश्च त‚थाभिधानात् । त‚थापि स्व‚भाव‚स्य कृत‚क‚त्व‚स्य हेतोर्या ‚{\tiny $_{lb}$}‚पृथ‚क्कृतिः सा कार्येण स‚ह श्लेषेण निर्देशे मा ज्ञासीत् प्र‚तिप‚त्ता स‚र्व्वं स्व‚भाव‚हेतुमी‚{\tiny $_{lb}$}‚\leavevmode\ledsidenote{\textenglish{99a/MA}}दृशं स‚प‚क्षे द्विधावृत्तिं कृत‚क‚त्वादेः‚{\tiny $_{7}$}‚ विप‚र्य‚य‚व्याप्तिस‚म्भ‚वात् । (१९९)
	\pend% ending standard par
      \label{div_pvv.4.200}
	  
	% new div opening: depth here is 2
	

	  \pstart \leavevmode% starting standard par
	किञ्च (।)
	\pend% ending standard par
      
	  \bigskip
	  \begingroup
	
	    \large
	  
	    \begin{quote}
	  
	    
	    \stanza[\smallbreak]
	\label{pv.4.200}\flagstanza{\tiny\textenglish{....4.200}}व्युत्प‚त्त्य‚र्था च हेतूक्तिरुक्तार्थानुमितौ कृता ।&अत्र प्र‚भेद आख्यातः ल‚क्ष‚ण‚न्तु न भिद्य‚ते ॥ २०० ॥\&[\smallbreak]


	
	    \end{quote}
	  
	  \endgroup
	

	  \pstart \leavevmode% starting standard par
	\hphantom{.}अनुमितौ स्वार्थानुमाने ‚{\color{DodgerBlue3}‚हेतूक्तिरुक्तार्था}‚ऽभिहित‚ल‚क्ष‚णापि प्र‚तिप‚त्तॄणां व्युत्प‚{\tiny $_{lb}$}‚‚{\color{DodgerBlue3}‚त्त्य‚र्था च} प‚रार्थानुमाने ‚{\color{DodgerBlue3}‚कृता} । अत्र च ‚{\color{DodgerBlue3}‚प्र‚भेद आख्यातः । ल‚क्ष‚णं} पुन‚र्हेतोर्न ‚{\color{DodgerBlue3}‚भिद्य‚ते} ।\edtext{\textsuperscript{*}}{\edlabel{pvv.486-2}\label{pvv.486-2}\lemma{*}\Bfootnote{स्वार्थानुमानोक्तात् ।}} ‚{\tiny $_{lb}$}‚त‚थाविध‚ल‚क्ष‚ण‚हेतुव‚च‚न‚स्य प‚रार्थानुमान‚त्वात् । (२००)
	\pend% ending standard par
      \label{div_pvv.4.201}
	  
	% new div opening: depth here is 2
	
	  \bigskip
	  \begingroup
	
	    \large
	  
	    \begin{quote}
	  
	    
	    \stanza[\smallbreak]
	\label{pv.4.201}\flagstanza{\tiny\textenglish{....4.201}}तेनात्र कार्य‚लिङ्गेन स्व‚भावोप्येक‚देश‚भाक् ।&स‚दृशोदाहृतिश्चातः प्र‚य‚त्नाद् व्य‚क्तिज‚न्म‚नः ॥ २०१ ॥\&[\smallbreak]


	
	    \end{quote}
	  
	  \endgroup
	

	  \pstart \leavevmode% starting standard par
	\hphantom{.}‚{\color{DodgerBlue3}‚तेन} प्र‚य‚त्नान्त‚रीय‚क‚त्वेन ‚{\color{DodgerBlue3}‚कार्य‚लिङ्गेन} श्लेष‚निर्देशात् । स्व‚भाव‚हेतुध‚र्म‚{\tiny $_{lb}$}‚भाजा ‚{\color{DodgerBlue3}‚स्व‚भावोपि} स‚प‚{\color{DodgerBlue3}‚क्षैक‚देश‚भाक्} भ‚व‚ति इत्य‚{\tiny $_{1}$}‚क्तो भ‚व‚ति । ‚{\color{DodgerBlue3}‚अत} एव कार्य‚{\tiny $_{lb}$}‚\leavevmode\ledsidenote{\textenglish{487/s}} स्व‚भाव‚त‚या ‚{\color{DodgerBlue3}‚स‚दृश}‚स्य प्र‚य‚त्नान‚न्त‚रीय‚क‚त्व‚{\color{DodgerBlue3}‚स्योदाहृति}‚रा चा र्ये ण कृता । ‚{\color{DodgerBlue3}‚प्र‚य‚त्नाद्\edtext{}{\edlabel{pvv.487-1}\label{pvv.487-1}\lemma{त्नाद्}\Bfootnote{उप‚ल‚ब्धेः कार्य‚त्वं ज्ञान‚त्वात् । उत्प‚त्तेः स्व‚भाव‚त्वं श‚ब्द‚रूप‚त्वात् ज्ञान‚स्य ।}} ‚{\tiny $_{lb}$}‚व्य‚क्तेर्ज‚न्म}‚न‚श्च भावात् । य‚दा प्र‚य‚त्नाद् व्य‚क्तिस्त‚दा कार्य‚हेतुः । य‚दा ज‚न्म\edtext{}{\edlabel{pvv.487-2}\label{pvv.487-2}\lemma{न्म}\Bfootnote{श‚ब्द‚स्य ।}}त‚दा ‚{\tiny $_{lb}$}‚स्व‚भाव‚हेतुः । (२०१)
	\pend% ending standard par
      \label{div_pvv.4.202}
	  
	% new div opening: depth here is 2
	

	  \pstart \leavevmode% starting standard par
	कार्य‚स्व‚भाव‚योः प्र‚भेद‚निर्देश‚स्य किं फ‚ल‚मित्याह ।
	\pend% ending standard par
      
	  \bigskip
	  \begingroup
	
	    \large
	  
	    \begin{quote}
	  
	    
	    \stanza[\smallbreak]
	\label{pv.4.202}\flagstanza{\tiny\textenglish{....4.202}}य‚न्नान्त‚रीय‚का स‚त्ता यो वात्म‚न्य‚विभाग‚वान् ।&स तेनाव्य‚भिचारी स्यादित्य‚र्थं त‚त्प्र‚भेद‚न‚म् ॥ २०२ ॥\&[\smallbreak]


	
	    \end{quote}
	  
	  \endgroup
	

	  \pstart \leavevmode% starting standard par
	\hphantom{.}‚{\color{DodgerBlue3}‚य‚न्नान्त‚रीय‚का स‚त्ता} भेदे स‚ति य‚म‚न्त‚रेण हेतु\edtext{}{\edlabel{pvv.487-3}\label{pvv.487-3}\lemma{हेतु}\Bfootnote{कार्य‚हेतुः ।}}र्न भ‚व‚ति । ‚{\color{DodgerBlue3}‚यो वा} स्व आत्मीयः ‚{\tiny $_{lb}$}‚\edtext{}{\edlabel{pvv.487-4}\label{pvv.487-4}\lemma{आत्मीयः}\Bfootnote{स्व‚भावः ।}}साध्या‚{\color{DodgerBlue3}‚द‚विभाग‚वान्} (।)\edtext{\textsuperscript{*}}{\edlabel{pvv.487-5}\label{pvv.487-5}\lemma{*}\Bfootnote{हेत्व‚न्त‚रान‚पेक्ष‚त्वार्थं ।}} आत्मा स्व‚भावः ‚{\color{DodgerBlue3}‚स‚{\tiny $_{2}$}‚ तेन} कार‚णेन ‚{\color{DodgerBlue3}‚व्याप‚केन} वाऽ‚{\color{DodgerBlue3}‚व्य‚भिचारी} व्य‚भिचार‚र‚हितः ‚{\color{DodgerBlue3}‚स्यात्} नान्य ‚{\color{DodgerBlue3}‚इत्य‚र्थ}‚मेत‚त् प्र‚योज‚नं (।) त‚योः ‚{\tiny $_{lb}$}‚कार्य‚स्व‚भाव‚योः प्र‚भेद‚नं । (२०२)
	\pend% ending standard par
      \label{div_pvv.4.203}
	  
	% new div opening: depth here is 2
	

	  \pstart \leavevmode% starting standard par
	एव‚ञ्च स‚ति (।)
	\pend% ending standard par
      
	  \bigskip
	  \begingroup
	
	    \large
	  
	    \begin{quote}
	  
	    
	    \stanza[\smallbreak]
	\label{pv.4.203}\flagstanza{\tiny\textenglish{....4.203}}संयोग्यादिषु येष्व‚स्ति प्र‚तिब‚न्धो न तादृशः ।&न ते हेत‚व इत्युक्तं व्य‚भिचार‚स्य स‚म्भ‚वात् ॥ २०३ ॥\&[\smallbreak]


	
	    \end{quote}
	  
	  \endgroup
	

	  \pstart \leavevmode% starting standard par
	\hphantom{.}‚{\color{DodgerBlue3}‚संयोगि}\edtext{\textsuperscript{*}}{\edlabel{pvv.487-6}\label{pvv.487-6}\lemma{*}\Bfootnote{वैशेषिकादिक‚ल्पिताः ।}} स‚म‚वायि एकार्थ‚स‚म‚वायि आकाशा‚{\color{DodgerBlue3}‚दिषु} प‚राभिम‚तेषु हेतुषु ‚{\color{DodgerBlue3}‚येषु प्र‚ति‚{\tiny $_{lb}$}‚ब‚न्धः तादृश}‚स्तादात्म्य‚त‚दुत्प‚त्तिल‚क्ष‚णो ‚{\color{DodgerBlue3}‚नास्ति (।) न ते हेत‚व इत्युक्तं भ‚व‚ति} । ‚{\tiny $_{lb}$}‚अत‚दात्म‚नोऽत‚दुत्प‚त्तेश्च साध्य‚{\color{DodgerBlue3}‚व्य‚भिचार‚स्य स‚म्भ‚वात्} । (२०३)
	\pend% ending standard par
      \label{div_pvv.4.204}
	  
	% new div opening: depth here is 2
	

	  \pstart \leavevmode% starting standard par
	अथ संयोग्यादिषु त‚दुत्प‚त्तिप्र‚तिब‚न्धोस्ति त‚दा (।)
	\pend% ending standard par
      
	  \bigskip
	  \begingroup
	
	    \large
	  
	    \begin{quote}
	  
	    
	    \stanza[\smallbreak]
	\label{pv.4.204}\flagstanza{\tiny\textenglish{....4.204}}स‚ति वा प्र‚तिब‚न्धेस्तु स एव ग‚तिसाध‚नः ।&निय‚मो ह्य‚विनाभावोऽनिय‚त‚श्च न साध‚न‚म् ॥ २०४ ॥\&[\smallbreak]


	
	    \end{quote}
	  
	  \endgroup
	

	  \pstart \leavevmode% starting standard par
	\hphantom{.}‚{\color{DodgerBlue3}‚स‚ति प्र‚तिब‚न्धे‚{\tiny $_{4}$}‚ स एव ग‚तिसाध‚नः} साध्य‚प्र‚तिप‚त्तिहेतुर‚स्तु निष्फ‚ला संयो‚{\tiny $_{lb}$}‚ग्यादिक‚ल्प‚ना (।) ‚{\color{DodgerBlue3}‚निय‚मो} निय‚त‚त्वं ‚{\color{DodgerBlue3}‚हि} साध‚न‚स्य साक्षा‚{\color{DodgerBlue3}‚द‚विनाभाव} उच्य‚ते (।) ‚{\tiny $_{lb}$}‚स च तादात्म्य‚त‚दुत्प‚त्तिभ्यां नान्य‚था (।) ‚{\color{DodgerBlue3}‚स च} त‚द्विक‚ल‚त्वात् साध्ये‚{\color{DodgerBlue3}‚ऽनिय‚तः} । ‚{\tiny $_{lb}$}‚न स साध‚नं । य‚था संयोगित्वेपि स व‚ह्निर्द्धूम‚स्येतिस्व‚भाव‚कार्य‚सिद्ध्य‚र्थं द्वौ द्वौ ‚{\tiny $_{lb}$}‚हेतुविप‚र्य‚यावि ति व्याख्यातं । (२०४)
	\pend% ending standard par
      \label{div_pvv.4.205}
	  
	% new div opening: depth here is 2
	\textsuperscript{\textenglish{488/s}}

	  \begin{center}%% label @type='head'
	\textbf{(४) विवादाद् भेद‚सामान्य इत्य‚स्य व्याख्यान‚म्}
	\end{center}
	

	  \pstart \leavevmode% starting standard par
	विवादाद् भेद‚सामान्य इति व्याख्यात‚{\tiny $_{5}$}‚व्यं ।
	\pend% ending standard par
      

	  \pstart \leavevmode% starting standard par
	स्यात् प्राणादिम‚त्वं हेतुर्य‚दि विप‚क्षाद्धेतुव्य‚तिरेकः स्यात् (।) स एव तु न ‚{\tiny $_{lb}$}‚सिध्य‚तीत्याह ।
	\pend% ending standard par
      
	  \bigskip
	  \begingroup
	
	    \large
	  
	    \begin{quote}
	  
	    
	    \stanza[\smallbreak]
	\label{pv.4.205}\flagstanza{\tiny\textenglish{....4.205}}ऐकान्तिक‚त्वं व्यावृत्तेर‚विनाभाव उच्य‚ते ।&त‚च्च नाप्र‚तिब‚द्धेषु त‚त एवान्व‚य‚स्थितिः ॥ २०५ ॥\&[\smallbreak]


	
	    \end{quote}
	  
	  \endgroup
	

	  \pstart \leavevmode% starting standard par
	\hphantom{.}विप‚क्षाद्भेदो ‚{\color{DodgerBlue3}‚व्यावृत्ते}‚र्ब्बाध‚क‚प्र‚माण‚निश्चित‚त्वात् । ‚{\color{DodgerBlue3}‚ऐकान्तिक‚त्व‚म‚विनाभाव ‚{\tiny $_{lb}$}‚उच्य‚ते (।) त‚च्च} व्यावृत्तेरैकान्तिक‚त्वं स्व‚साध्या‚{\color{DodgerBlue3}‚प्र‚तिब‚द्धेषु} प्राणादिषु नास्ति । ‚{\tiny $_{lb}$}‚य‚दि ह्यात्म‚नि प्र‚तिब‚द्धाः प्राणाद‚यः त‚दात्म‚निवृत्तौ निवृत्ता विप‚क्षाद् ग‚म्येर‚न् (।) ‚{\tiny $_{lb}$}‚अन्य‚था तु प‚क्ष एव स‚न्देहः (।) किम‚मी आत्माभावेपि व‚र्त्त‚न्ते‚{\tiny $_{6}$}‚ उत नेति (।) ‚{\tiny $_{lb}$}‚य‚था च प्र‚तिब‚न्धाद् व्य‚तिरेक‚निश्च‚यः । त‚था त‚तः प्र‚तिब‚न्धादेवा‚{\color{DodgerBlue3}‚न्व‚य‚स्य स्थितिः} । ‚{\tiny $_{lb}$}‚न तु स‚ह‚द‚र्श‚न‚मात्रेण । (२०५)
	\pend% ending standard par
      \label{div_pvv.4.206}
	  
	% new div opening: depth here is 2
	

	  \pstart \leavevmode% starting standard par
	त‚स्मात् साध्य‚स्य (।)
	\pend% ending standard par
      
	  \bigskip
	  \begingroup
	
	    \large
	  
	    \begin{quote}
	  
	    
	    \stanza[\smallbreak]
	\label{pv.4.206}\flagstanza{\tiny\textenglish{....4.206}}स्वात्म‚त्वे हेतुभावे वा सिद्धे हि व्य‚तिरेकिता ।&सिध्येद‚तो विशेषे न व्य‚तिरेको न चान्व‚यः ॥ २०६ ॥\&[\smallbreak]


	
	    \end{quote}
	  
	  \endgroup
	

	  \pstart \leavevmode% starting standard par
	\hphantom{.}‚{\color{DodgerBlue3}‚स्वात्म‚त्वे} हेतुस्व‚भावात्म‚क‚त्वे ‚{\color{DodgerBlue3}‚हेतुभावे वा सिद्धे} स‚ति (त‚द‚भावे निय‚मेन ‚{\tiny $_{lb}$}‚व्य‚तिरेकात्) विप‚क्षाद् व्याप‚क‚कार‚ण‚व्य‚तिरेके स्व‚भाव‚कार्य‚हेत्वो‚{\color{DodgerBlue3}‚र्व्य‚तिरेकिता} सिध्येत् । नान्य‚थेति न्याय एषः । य‚तः प्र‚तिब‚न्धेनैवान्व‚य‚व्य‚तिरेक‚सिद्धिः । अतो ‚{\tiny $_{lb}$}‚\leavevmode\ledsidenote{\textenglish{99b/MA}}विशेषेऽसाधार‚णे हेतौ नान्व‚यो न वा‚{\tiny $_{7}$}‚ व्य‚तिरेकः सिध्य‚ति ॥ (२०६)
	\pend% ending standard par
      \label{div_pvv.4.207}
	  
	% new div opening: depth here is 2
	

	  \pstart \leavevmode% starting standard par
	एवं त‚र्हि स‚प‚क्ष‚विप‚क्ष‚व्य‚तिरेकाभ्यां व्यावृत्तेर्विशेषः क‚थ मा चा र्ये णोक्तः ‚{\tiny $_{lb}$}‚इत्याह ।
	\pend% ending standard par
      
	  \bigskip
	  \begingroup
	
	    \large
	  
	    \begin{quote}
	  
	    
	    \stanza[\smallbreak]
	\label{pv.4.207}\flagstanza{\tiny\textenglish{....4.207}}अदृष्टिमात्र‚मादाय केव‚लं व्य‚तिरेकिता ।&उक्ताऽनैकान्तिक‚स्त‚स्माद‚न्य‚था ग‚म‚को भ‚वेत् ॥ २०७ ॥\&[\smallbreak]


	
	    \end{quote}
	  
	  \endgroup
	

	  \pstart \leavevmode% starting standard par
	\hphantom{.}‚{\color{DodgerBlue3}‚अदृष्टिमात्रं} बाद‚क‚प्र‚माण‚र‚हित‚{\color{DodgerBlue3}‚मादाय} प‚राभिप्रायेण स‚प‚क्षाद् ‚{\color{DodgerBlue3}‚व्य‚तिरेकि‚{\tiny $_{lb}$}‚तोक्ता} । त‚स्याद‚र्श‚न‚मात्रेण व्य‚तिरेकानिश्च‚या‚{\color{DodgerBlue3}‚द‚नैकान्तिक} आचार्येणोक्तः । ‚{\tiny $_{lb}$}‚‚{\color{DodgerBlue3}‚अन्य‚था} विप‚क्षाद् व्य‚तिरेक‚निश्च‚ये ‚{\color{DodgerBlue3}‚ग‚म‚को} हेतु‚{\color{DodgerBlue3}‚र्भ‚वेत्} । त‚त‚श्चानैकान्तिक‚व‚र्गे न ‚{\tiny $_{lb}$}‚प्र‚क्षिप्येत । (२०७)
	\pend% ending standard par
      \label{div_pvv.4.208}
	  
	% new div opening: depth here is 2
	\textsuperscript{\textenglish{489/s}}

	  \begin{center}%% label @type='head'
	\textbf{(५) साध्याभाव‚स्य साध‚नाभावेन न व्याप्त‚ता}
	\end{center}
	

	  \begin{center}%% label @type='head'
	\textbf{क. जीव‚च्छ‚रीरं प्राणादिम‚त्त्वादित्य‚त्र दोषः}
	\end{center}
	

	  \pstart \leavevmode% starting standard par
	न‚नु साध्य‚निवृत्तौ\edtext{}{\edlabel{pvv.489-1}\label{pvv.489-1}\lemma{निवृत्तौ}\Bfootnote{तादात्म्य‚निवृत्ति ।}} निय‚मेन साध‚नं‚{\tiny $_{1}$}‚ निव‚र्त्त‚त इति ‚{\color{DodgerBlue3}‚साध्याभावः साध‚नाभावेन} व्याप्तः । त‚त‚श्च (।)
	\pend% ending standard par
      
	  \bigskip
	  \begingroup
	
	    \large
	  
	    \begin{quote}
	  
	    
	    \stanza[\smallbreak]
	\label{pv.4.208}\flagstanza{\tiny\textenglish{....4.208}}प्राणाद्य‚भावो नैरात्म्य‚व्यापीति विनिव‚र्त्त‚ने ।&आत्म‚नो विनिव‚र्त्तेत प्राणादिर्य‚दि त‚च्च न ॥ २०८ ॥\&[\smallbreak]


	
	    \end{quote}
	  
	  \endgroup
	

	  \pstart \leavevmode% starting standard par
	\hphantom{.}‚{\color{DodgerBlue3}‚प्राणादेः} साध‚न‚स्या‚{\color{DodgerBlue3}‚भावो नैरात्म्य}‚स्य सात्म‚क‚त्व‚साध्याभाव‚स्य ‚{\color{DodgerBlue3}‚व्यापीति} हेतो‚{\color{DodgerBlue3}‚रात्म‚नो} घ‚टादेर्व्विप‚क्षाद् ‚{\color{DodgerBlue3}‚विनिव‚र्त्त‚ने} स‚ति ‚{\color{DodgerBlue3}‚प्राणादिर्व्विनिव‚र्त्त}‚ते । न ह्य‚न्य‚था ‚{\tiny $_{lb}$}‚साध‚नाभावः साध्याभाव‚स्य च व्याप‚को भ‚व‚ति । त‚तो य‚त्र प्राणादिभाव‚नि‚{\tiny $_{lb}$}‚वृत्तिस्त‚त्र सात्म‚क‚त्व‚भाव‚निवृत्तिर‚पीति य‚द्युच्य‚ते ‚{\color{DodgerBlue3}‚त‚च्च न} युक्तं । (२०८)
	\pend% ending standard par
      \label{div_pvv.4.209_4.210}
	  
	% new div opening: depth here is 2
	
	  \bigskip
	  \begingroup
	
	    \large
	  
	    \begin{quote}
	  
	    
	    \stanza[\smallbreak]
	\label{pv.4.209a}\flagstanza{\tiny\textenglish{...4.209a}}अन्य‚स्य विनिवृत्यान्य‚विनिवृत्त्येर‚योग‚तः ।&त‚दात्मा त‚त्प्र‚सूतिश्चेन्नैत‚त्;\&[\smallbreak]


	
	    \end{quote}
	  
	  \endgroup
	

	  \pstart \leavevmode% starting standard par
	\hphantom{.}‚{\color{DodgerBlue3}‚अन्य‚स्य} कार‚ण‚स्यात्म‚नो ‚{\color{DodgerBlue3}‚विनिवृत्त्या} प्राणादे‚{\color{DodgerBlue3}‚र्व्विनिवृत्तेर‚योग‚तः} । तादात्म्य‚{\tiny $_{lb}$}‚त‚दुत्प‚त्तावेव हि साध्य‚निवृत्तौ साध‚न‚निवृत्तिर्युक्ता त‚त्स्व‚भाव‚त्वात् त‚दाय‚त्त‚त्वाच्च । ‚{\tiny $_{lb}$}‚स‚ति च साध्य‚साध‚न‚योर्निय‚मेन निव‚र्त्त्य‚निव‚र्त्त‚क‚भावे साध्याभावः साध‚नाभावेन ‚{\tiny $_{lb}$}‚व्याप्य‚ते । प्राणादि‚{\color{DodgerBlue3}‚स्त}‚स्यात्म‚न ‚{\color{DodgerBlue3}‚आत्मा} स्व‚भावः । ‚{\color{DodgerBlue3}‚त}‚स्मात् ‚{\color{DodgerBlue3}‚प्र}‚सूतिर्व्वास्येति चेत् । ‚{\tiny $_{lb}$}‚‚{\color{DodgerBlue3}‚नैत‚द}‚स्ति युक्तं\edtext{}{\edlabel{pvv.489-2}\label{pvv.489-2}\lemma{युक्तं}\Bfootnote{न चेष्टं प‚र‚स्य केव‚ल‚मेतावानेव प्र‚तिब‚न्धे इत्युप‚न्यासः ।}} । त‚था हि (।)
	\pend% ending standard par
      
	  \bigskip
	  \begingroup
	
	    \large
	  
	    \begin{quote}
	  
	    
	    \stanza[\smallbreak]
	\label{pv.4.209b}\flagstanza{\tiny\textenglish{...4.209b}}आत्मोप‚ल‚म्भ‚ने ॥ २०९ ॥\&[\smallbreak]


	
	    \end{quote}
	  
	  \endgroup
	
	  \bigskip
	  \begingroup
	
	    \large
	  
	    \begin{quote}
	  
	    
	    \stanza[\smallbreak]
	\label{pv.4.210}\flagstanza{\tiny\textenglish{....4.210}}त‚स्योप‚ल‚ब्धाव‚ग‚ताव‚ग‚तौ च प्र‚सिध्य‚ति ।&ते चात्य‚न्त‚प‚रोक्ष‚स्य दृष्ट्य‚दृष्टी न सिध्य‚तः ॥ २१० ॥\&[\smallbreak]


	
	    \end{quote}
	  
	  \endgroup
	

	  \pstart \leavevmode% starting standard par
	\hphantom{.}‚{\color{DodgerBlue3}‚आत्म‚न उप‚ल‚म्भ‚ने} स‚ति ‚{\color{DodgerBlue3}‚त‚स्य} प्राणादे‚{\color{DodgerBlue3}‚रुप‚ल‚ब्धौ}‚{\tiny $_{3}$}‚ स‚त्यामात्म‚नोऽ‚{\color{DodgerBlue3}‚ग‚तौ प्राणा}‚{\tiny $_{lb}$}‚देर‚{\color{DodgerBlue3}‚ग‚तौ च} स‚त्यां कार्य‚कार‚ण‚भावः ‚{\color{DodgerBlue3}‚प्र‚सिध्य‚ति । ते च दृष्ट्य‚दृष्टी} कार्य‚कार‚ण‚{\tiny $_{lb}$}‚भाव‚साध‚नेऽ‚{\color{DodgerBlue3}‚त्य‚न्त‚प}‚रो‚{\color{DodgerBlue3}‚क्ष}‚स्यात्म‚नो ‚{\color{DodgerBlue3}‚न सिध्य‚तः} । अत्य‚न्त‚प‚रोक्ष‚स्य क‚थ‚म‚दृष्टिर‚पि न ‚{\tiny $_{lb}$}‚सिध्य‚तीति चेत् । अभाव‚साधिका दृश्यानुप‚ल‚ब्धिर्न सिध्य‚त्येव । अदृष्टिमात्र‚न्तु ‚{\tiny $_{lb}$}‚नाभाव‚साध‚कं । त‚तः प्राणादेरात्म‚ना स‚ह प्र‚तिब‚न्धासिद्धेर्नात्म‚निवृत्त्या प्राणा‚{\tiny $_{lb}$}‚दिनिवृत्तिसिद्धिः । (२०९, २१०)
	\pend% ending standard par
      \label{div_pvv.4.211}
	  
	% new div opening: depth here is 2
	

	  \pstart \leavevmode% starting standard par
	किञ्च‚{\tiny $_{4}$}‚ (।)
	\pend% ending standard par
      \textsuperscript{\textenglish{490/s}}
	  \bigskip
	  \begingroup
	
	    \large
	  
	    \begin{quote}
	  
	    
	    \stanza[\smallbreak]
	\label{pv.4.211}\flagstanza{\tiny\textenglish{....4.211}}अन्य‚त्रादृष्ट‚रूप‚स्य घ‚टादौ नेति वा कुतः ।&अज्ञात‚व्य‚तिरेक‚स्य व्यावृत्तेर्व्यापिता कुतः ॥ २११ ॥\&[\smallbreak]


	
	    \end{quote}
	  
	  \endgroup
	

	  \pstart \leavevmode% starting standard par
	\hphantom{.}आत्म‚नो‚{\color{DodgerBlue3}‚न्य‚त्र} जीव‚च्छ‚रीरे‚{\color{DodgerBlue3}‚ऽदृष्ट‚रूप}‚स्य ‚{\color{DodgerBlue3}‚घ‚टादौ न} स‚त्त्व‚{\color{DodgerBlue3}‚मिति} य‚दुच्य‚ते त‚{\color{DodgerBlue3}‚त्कुतः} सिद्धं । दृश्यानुप‚ल‚ब्ध्या ह्य‚भाव‚सिद्धिः । न चादृष्ट‚च‚रे त‚त्स‚म्भ‚वः । ‚{\color{DodgerBlue3}‚अज्ञातो ‚{\tiny $_{lb}$}‚व्य‚तिरेको}‚ऽभावो य‚स्य त‚स्यात्म‚नो या ‚{\color{DodgerBlue3}‚व्यावृत्ति}‚र्घ‚टादौ त‚स्याः प्राणादिनिवृत्त्या ‚{\tiny $_{lb}$}‚‚{\color{DodgerBlue3}‚व्यापिता} व्याप‚नं ‚{\color{DodgerBlue3}‚कुतः} स‚म्भ‚वि । येन जीव‚च्छ‚रीरे प्राणादिनिवृत्त्य‚भावात् सात्म‚{\tiny $_{lb}$}‚क‚त्व‚निवृत्त्य‚भावे स‚त्यात्म‚सिद्धिः स्यात् ॥ (२११)
	\pend% ending standard par
      \label{div_pvv.4.212}
	  
	% new div opening: depth here is 2
	

	  \pstart \leavevmode% starting standard par
	न‚नु य‚था घ‚टादौ प्राणा‚{\tiny $_{5}$}‚द्य‚भाव‚निश्च‚य‚स्त‚थात्माभाव‚निश्च‚योपि किं न भ‚व‚{\tiny $_{lb}$}‚तीत्याह ।
	\pend% ending standard par
      
	  \bigskip
	  \begingroup
	
	    \large
	  
	    \begin{quote}
	  
	    
	    \stanza[\smallbreak]
	\label{pv.4.212}\flagstanza{\tiny\textenglish{....4.212}}प्राणादेश्च क्व‚चिद् दृष्ट्या स‚त्त्वास‚त्त्वं प्र‚तीय‚ते ।&त‚थात्मा य‚दि दृश्येत स‚त्त्वास‚त्त्वं प्र‚तीय‚ते ॥ २१२ ॥\&[\smallbreak]


	
	    \end{quote}
	  
	  \endgroup
	

	  \pstart \leavevmode% starting standard par
	\hphantom{.}‚{\color{DodgerBlue3}‚प्राणादेः क्व‚चि}‚ज्जीव‚च्छ‚रीरे ‚{\color{DodgerBlue3}‚दृष्ट्या स‚त्त्वं} प्र‚तीय‚ते । मृत‚देहे चोप‚ल‚ब्धिल‚क्ष‚ण‚{\tiny $_{lb}$}‚प्राप्त‚स्यास्यादृष्ट्या‚{\color{DodgerBlue3}‚ऽस‚त्त्वं} प्र‚तीय‚ते । ‚{\color{DodgerBlue3}‚त‚था य‚दि} क्व‚चिद् देहे ‚{\color{DodgerBlue3}‚आत्मा दृश्येत} त‚दा ‚{\tiny $_{lb}$}‚‚{\color{DodgerBlue3}‚त‚त्र स‚त्त्व}‚म‚स्य प्र‚तीयेत । अन्य‚त्रोप‚ल‚भ्य‚स्व‚रूप‚स्यास्यानुप‚ल‚ब्धेर‚{\color{DodgerBlue3}‚स‚त्त्वं प्र‚तीयेत । न} चात्मोप‚ल‚ब्धिर‚स्ति क्व‚चिदिति न त‚द‚भाव‚निश्च‚यः । (२१२)
	\pend% ending standard par
      \label{div_pvv.4.213}
	  
	% new div opening: depth here is 2
	

	  \pstart \leavevmode% starting standard par
	य‚द‚प्युच्य‚ते (।) य‚दि‚{\tiny $_{6}$}‚ न सात्म‚कं जीव‚च्छ‚रीरं त‚दास्य प्राणादिविर‚ह‚प्र‚स‚ङ्गो ‚{\tiny $_{lb}$}‚न‚रात्म्याद् घ‚ट‚व‚दिति प्र‚स‚ङ्ग‚साध‚नं । त‚च्चायुक्तं द‚र्श‚यितुमाह ।
	\pend% ending standard par
      
	  \bigskip
	  \begingroup
	
	    \large
	  
	    \begin{quote}
	  
	    
	    \stanza[\smallbreak]
	\label{pv.4.213}\flagstanza{\tiny\textenglish{....4.213}}य‚स्य हेतोर‚भावेन घ‚टे प्राणो न दृश्य‚ते ।&देहेपि य‚द्य‚सौ न स्याद् युक्तो देहे न स‚म्भ‚वः ॥ २१३ ॥\&[\smallbreak]


	
	    \end{quote}
	  
	  \endgroup
	

	  \pstart \leavevmode% starting standard par
	\hphantom{.}‚{\color{DodgerBlue3}‚य‚स्य} बुद्धिदेहातिश‚य‚प्र‚य‚त्ना‚{\color{DodgerBlue3}‚देर्हेतोर‚भावेन घ‚टे प्राणो न दृश्य‚ते (।) देहेपि ‚{\tiny $_{lb}$}‚य‚द्य‚सौ} बुद्धिदेहातिश‚य‚प्र‚य‚त्नादि‚{\color{DodgerBlue3}‚र्न स्यात्} (।) त‚दा देहेपि हेतुविर‚हात् प्राणादेर्न\edtext{}{\edlabel{pvv.490-1}\label{pvv.490-1}\lemma{प्राणादेर्न}\Bfootnote{शाश्व‚त आत्मा त‚द्विप‚रीताः प्राणाद‚य इति न तादात्म्यं ।}} ‚{\tiny $_{lb}$}‚\leavevmode\ledsidenote{\textenglish{100a/MA}}‚{\color{DodgerBlue3}‚संभ‚वो युक्तो} य‚था मृत‚श‚रीरे । जीव‚द्देहे तु बुद्ध्यादिस‚द्भावादेव‚{\tiny $_{7}$}‚ प्राणादिरुचित‚{\tiny $_{lb}$}‚संभ‚व इति नात्माभावे त‚द‚भाव‚प्र‚स‚ङ्गः स‚ङ्ग‚तः । (२१३)
	\pend% ending standard par
      \label{div_pvv.4.214}
	  
	% new div opening: depth here is 2
	

	  \pstart \leavevmode% starting standard par
	अथ घ‚ट‚जीव‚च्छ‚रीरादीनां नैरात्म्येन य‚था साध‚र्म्यं त‚था प्राणादिर‚हित‚त‚यापि ‚{\tiny $_{lb}$}‚स्यात् । न च भ‚व‚ति (।) त‚स्मान्निरात्म‚के घ‚टादौ प्राणादिर‚दृष्टो य‚त्र व‚र्त्त‚ते त‚त्र ‚{\tiny $_{lb}$}‚सात्म‚क‚त्वं साध‚य‚तीत्याह ।
	\pend% ending standard par
      
	  \bigskip
	  \begingroup
	
	    \large
	  
	    \begin{quote}
	  
	    
	    \stanza[\smallbreak]
	\label{pv.4.214}\flagstanza{\tiny\textenglish{....4.214}}भिन्नेपि किञ्चित् साध‚र्म्याद् य‚दि त‚त्त्वं प्र‚तीय‚ते ।&प्र‚मेय‚त्वाद् घ‚टादीनां सात्म‚त्वं किन्न मीय‚ते ॥ २१४ ॥\&[\smallbreak]


	
	    \end{quote}
	  
	  \endgroup
	\textsuperscript{\textenglish{491/s}}

	  \pstart \leavevmode% starting standard par
	\hphantom{.}घ‚ट‚भिन्नेपि जीव‚द्देहे ‚{\color{DodgerBlue3}‚किञ्चि}‚न्मात्रेण निरात्म‚क‚त्वेन ‚{\color{DodgerBlue3}‚साध‚र्म्यात् य‚दि त‚त्त्वं} प्राणादिविर‚हित‚त‚या घ‚ट‚स‚दृश‚त्वं ‚{\color{DodgerBlue3}‚प्र‚तीय‚ते} त‚दा‚{\tiny $_{1}$}‚ ‚{\color{DodgerBlue3}‚प्र‚मेय‚त्वा}‚द्धेतो‚{\color{DodgerBlue3}‚र्घ‚टादीनां} जीव‚च्छ‚रीर‚{\tiny $_{lb}$}‚‚{\color{DodgerBlue3}‚सात्म‚क‚त्वं किन्न मीय‚ते} (। २१४)
	\pend% ending standard par
      \label{div_pvv.4.215}
	  
	% new div opening: depth here is 2
	
	  \bigskip
	  \begingroup
	
	    \large
	  
	    \begin{quote}
	  
	    
	    \stanza[\smallbreak]
	\label{pv.4.215}\flagstanza{\tiny\textenglish{....4.215}}अनिष्टेश्चेत् प्र‚माणं हि स‚र्वेष्टीनां निब‚न्ध‚न‚म् ।&भावाभाव‚व्य‚व‚क्थां कः ? क‚र्त्तुं तेन विना प्र‚भुः ॥ २१५ ॥\&[\smallbreak]


	
	    \end{quote}
	  
	  \endgroup
	

	  \pstart \leavevmode% starting standard par
	\hphantom{.}‚{\color{DodgerBlue3}‚अनिष्टेश्चेत्} (।) न‚नु ‚{\color{DodgerBlue3}‚प्र‚माणं हि स‚र्व्वेष्टीनां निब‚न्ध‚नं} । त‚द्व‚शेनार्थान्न ‚{\tiny $_{lb}$}‚स्थितेः । ‚{\color{DodgerBlue3}‚तेन} प्र‚माणेन ‚{\color{DodgerBlue3}‚भावाभाव}‚यो‚{\color{DodgerBlue3}‚र्व्य‚व‚स्थां क‚र्त्तुं कः प्र‚भुः} । य‚दि च किञ्चित् ‚{\tiny $_{lb}$}‚साध‚र्म्य‚मात्रात् साध‚नं साध्य‚साध‚कं । त‚दा प्र‚मेय‚त्वाद् घ‚ट‚स्यापि सात्म‚क‚त्व‚साध‚{\tiny $_{lb}$}‚नाद‚निष्टिर‚नुप‚युक्ता । (२१५)
	\pend% ending standard par
      \label{div_pvv.4.216_4.217}
	  
	% new div opening: depth here is 2
	

	  \begin{center}%% label @type='head'
	\textbf{ख. स्मृतीच्छाद‚यः प्राणादिहेतुः}
	\end{center}
	

	  \pstart \leavevmode% starting standard par
	य‚दि त‚र्हि नात्मा प्राणादेर्हेतुः (।) क‚स्त‚र्हि भ‚विष्य‚तीत्याह । य‚था‚{\tiny $_{2}$}‚ योगं (।)
	\pend% ending standard par
      
	  \bigskip
	  \begingroup
	
	    \large
	  
	    \begin{quote}
	  
	    
	    \stanza[\smallbreak]
	\label{pv.4.216a}\flagstanza{\tiny\textenglish{...4.216a}}स्मृतीच्छाय‚त्न‚जः प्राण‚निमेषादिस्त‚दुद्भ‚वः ।&विष‚येन्द्रिय‚चित्तेभ्यः ;\&[\smallbreak]


	
	    \end{quote}
	  
	  \endgroup
	

	  \pstart \leavevmode% starting standard par
	\hphantom{.}‚{\color{DodgerBlue3}‚स्मृतीच्छा}‚प्र‚{\color{DodgerBlue3}‚य‚त्ने}\edtext{}{\edlabel{pvv.491-1}\label{pvv.491-1}\lemma{प्र}\Bfootnote{प्राण‚वायोः प्रेर‚कोऽन्त‚र्व्यापारः प्र‚य‚त्नः ।}}भ्यो जातः ‚{\color{DodgerBlue3}‚प्राण‚निमेषादिः} स‚माहित‚स्य निरुद्ध‚वायोः स‚माधि‚{\tiny $_{lb}$}‚व्युत्थित‚स्य स्म‚र‚णात् प्राण‚वृत्तिर्गाढ‚प्र‚हारादिभिर्व्याह‚त‚प्राण‚स्य इच्छाप्र‚य‚त्नाभ्यां ‚{\tiny $_{lb}$}‚प्राण‚प्र‚वृत्तिः । स्व‚स्थ‚स्य साद्गुण्यात्\edtext{}{\edlabel{pvv.491-2}\label{pvv.491-2}\lemma{साद्गुण्यात्}\Bfootnote{किञ्चित् प‚श्य‚तोऽक्षान्त‚र‚प्र‚स्फुर‚णे श‚रीर‚साद्गुण्य‚बुद्धिप्र‚य‚त्नाद‚यः प्राणादिषु ।}} निमेषादेरिच्छाप्र‚य‚त्न‚ज‚त्वं व्य‚क्तं । तेषां च ‚{\tiny $_{lb}$}‚स्मृतीच्छाय‚त्नाना‚{\color{DodgerBlue3}‚मुद्भ‚वो विष‚येन्द्रिय‚चित्तेभ्यो} य‚था\edtext{}{\edlabel{pvv.491-3}\label{pvv.491-3}\lemma{था}\Bfootnote{स्मृतिर‚नुभ‚वात् त‚त इन्द्रिय‚विकार‚स्फुर‚णे । अनुभूत‚स्मृत्या स्फुर‚णं नानास्र‚वः । मातुलुङ्ग‚द‚र्श‚ने ।}}योगं (।) क्व‚चिद् विष‚याद् ‚{\tiny $_{lb}$}‚व‚द‚रादे र‚स‚नास्र‚वादिहेतो र‚सादि‚{\tiny $_{3}$}‚स्मृतिर्भ‚व‚ति । इन्द्रियाद्वा विलादेर‚प‚टुजान‚हेतोः ‚{\tiny $_{lb}$}‚प्र‚दीप‚म‚ण्ड‚लादिस्म‚र‚णं भ‚व‚ति बुद्धेरेवात‚द्विष‚य‚स्यातीतानुकूल‚तादिस्मृतिरुद्भ‚व‚ति ।
	\pend% ending standard par
      

	  \pstart \leavevmode% starting standard par
	न‚नु ष‚ट् प्र‚वृत्तिबुद्ध‚य एवात्म‚जाः प्र‚वृत्तिबुद्धिजाश्च स्मृत्याद‚यः । त‚द्भ‚वाश्च ‚{\tiny $_{lb}$}‚प्राणाद‚यः इति प‚रंप‚राऽत्म‚हेतुका एवेत्याह । ताः ष‚ड्बुद्ध‚यः (।)
	\pend% ending standard par
      
	  \bigskip
	  \begingroup
	
	    \large
	  
	    \begin{quote}
	  
	    
	    \stanza[\smallbreak]
	\label{pv.4.216b}\flagstanza{\tiny\textenglish{...4.216b}}ताः स्व‚जातिस‚मुद्भ‚वाः ॥ २१६ ॥\&[\smallbreak]


	
	    \end{quote}
	  
	  \endgroup
	
	  \bigskip
	  \begingroup
	
	    \large
	  
	    \begin{quote}
	  
	    
	    \stanza[\smallbreak]
	\label{pv.4.217}\flagstanza{\tiny\textenglish{....4.217}}अन्योन्य‚प्र‚त्य‚यापेक्षा अन्व‚य‚व्य‚तिरेक‚भाक् ।&एताव‚त्यात्म‚भावोय‚म‚न‚व‚स्थान्य‚क‚ल्प‚ने ॥ २१७ ॥\&[\smallbreak]


	
	    \end{quote}
	  
	  \endgroup
	\textsuperscript{\textenglish{492/s}}

	  \pstart \leavevmode% starting standard par
	\hphantom{.}‚{\color{DodgerBlue3}‚अन्योन्य‚प्र‚त्य‚यापेक्षा} य‚स्या य आत्मीयः प्र‚त्य‚यः स‚ह‚कारिकार‚णं‚{\tiny $_{4}$}‚ इन्द्रियादि ‚{\tiny $_{lb}$}‚त‚स्मिन्न‚पेक्षाऽय‚त्तिर्यासां तास्त‚था स‚त्यः स्व‚जातिस‚मुद्भ‚वाः स‚म‚न‚न्त‚र‚प्र‚भ‚वा न ‚{\tiny $_{lb}$}‚तु प‚रंप‚र‚याप्यात्मापेक्षिण्य इत्य‚र्थः । ‚{\color{DodgerBlue3}‚एताव}‚ति कार‚ण‚क‚लापे‚{\color{DodgerBlue3}‚ऽय‚मात्म‚भावः} ष‚डाय‚त‚नं ‚{\tiny $_{lb}$}‚गृहीतो‚{\color{DodgerBlue3}‚ऽन्व‚य‚व्य‚तिरेक‚भाक्} प्र‚तिब‚द्धः कार‚ण‚त्वेन त‚तोन्य‚स्य\edtext{}{\edlabel{pvv.492-1}\label{pvv.492-1}\lemma{स्य}\Bfootnote{आत्मादेः ।}} ‚{\color{DodgerBlue3}‚क‚ल्प‚नेऽन‚व‚स्था} कार‚णा‚{\tiny $_{lb}$}‚नां । त‚स्मान्नादृष्ट‚साम‚र्थ्य‚स्यात्म‚नो निवृत्तौ प्राणा‚{\tiny $_{5}$}‚दिनिवृत्तिर्युक्ता ॥ (२१६, २१७)
	\pend% ending standard par
      \label{div_pvv.4.218}
	  
	% new div opening: depth here is 2
	

	  \pstart \leavevmode% starting standard par
	किञ्च (।)
	\pend% ending standard par
      
	  \bigskip
	  \begingroup
	
	    \large
	  
	    \begin{quote}
	  
	    
	    \stanza[\smallbreak]
	\label{pv.4.218}\flagstanza{\tiny\textenglish{....4.218}}श्राव‚ण‚त्वेन त‚त् तुल्यं प्राणादि व्य‚भिचार‚तः ।&न त‚स्य व्य‚भिचारित्वाद् व्य‚तिरेकेपि चेत् क‚थ‚म् ॥ २१८ ॥\&[\smallbreak]


	
	    \end{quote}
	  
	  \endgroup
	

	  \pstart \leavevmode% starting standard par
	\hphantom{.}‚{\color{DodgerBlue3}‚त‚त् प्राणादि}‚साध‚नं ‚{\color{DodgerBlue3}‚श्राव‚ण‚त्वेन} हेतुना ‚{\color{DodgerBlue3}‚व्य‚भिचार‚तो}‚ऽनैकान्तिक‚त्वात्\edtext{}{\edlabel{pvv.492-2}\label{pvv.492-2}\lemma{त्वात्}\Bfootnote{उभ‚य‚त्र व्य‚तिरेक‚व्य‚भिचार‚स्य तुल्य‚त्वात् ।}} ‚{\color{DodgerBlue3}‚तुल्यं} । ‚{\tiny $_{lb}$}‚नैत\edtext{}{\edlabel{pvv.492-3}\label{pvv.492-3}\lemma{नैत}\Bfootnote{प‚रः ।}}द‚स्ति । त‚स्य श्राव‚ण‚त्व‚स्य विप‚क्षाद् ब‚हुलं व्य‚तिरेकेपि व्य‚तिरेक‚स्य ‚{\color{DodgerBlue3}‚व्य‚भि‚{\tiny $_{lb}$}‚चार‚तः} । न हि श्राव(ण)त्व‚म‚नित्येभ्यः प्रायो व्यावृत्त‚मित्य‚नित्य‚त्व‚व्य‚तिरेकाव्य‚भि‚{\tiny $_{lb}$}‚चारि श‚क्य‚म‚व‚सातुं ।\edtext{\textsuperscript{*}}{\edlabel{pvv.492-4}\label{pvv.492-4}\lemma{*}\Bfootnote{य‚द्य‚पि कुड्यादेर‚नित्याद् व्यावृत्तः स‚प‚क्षाकाशात् त‚थापि व्य‚भिचारोऽच्छ‚टादिश‚ब्देऽनित्य‚त्वादिति ।}}प‚क्ष एवानित्येन स‚हाकार्य‚विरोधात् । प्राणादि तु निय‚मेन ‚{\tiny $_{lb}$}‚विप‚क्षाद् व्य‚तिरेकाव्य‚भिचारीति न युक्तं‚{\tiny $_{6}$}‚ श्राव‚ण‚त्वेन तुल्य‚मिति चेत् । पृच्छ‚{\tiny $_{lb}$}‚त्या चा र्यः । प्राणादिश्राव‚ण‚त्वेनातुल्यं क‚थ‚मिति । (२१८)
	\pend% ending standard par
      \label{div_pvv.4.219}
	  
	% new div opening: depth here is 2
	

	  \pstart \leavevmode% starting standard par
	इत‚र आह ।
	\pend% ending standard par
      
	  \bigskip
	  \begingroup
	
	    \large
	  
	    \begin{quote}
	  
	    
	    \stanza[\smallbreak]
	\label{pv.4.219}\flagstanza{\tiny\textenglish{....4.219}}नासाध्यादेव विश्लेष‚स्त‚स्य न‚न्वेव‚मुच्य‚ते ।&साध्येनुवृत्य‚भावोर्थात् त‚स्यान्य‚त्राप्य‚सौ स‚मः ॥ २१९ ॥\&[\smallbreak]


	
	    \end{quote}
	  
	  \endgroup
	

	  \pstart \leavevmode% starting standard par
	त‚स्य श्राव‚ण‚त्व‚स्यासाध्याद् विप‚क्षादेव न विश्लेषो व्यावृत्तिः किन्तु स‚प‚क्षाद‚पि । ‚{\tiny $_{lb}$}‚प्राणादेस्तु स‚र्व्व‚स्य जीव‚च्छ‚रीर‚स्य सात्म‚त्वेन साध्य‚त्वात् (।) स‚प‚क्ष एव नास्तीति ‚{\tiny $_{lb}$}‚क‚थ‚न्त‚तो व्यावृत्तिरिति (।) अत्राह । ‚{\color{DodgerBlue3}‚न‚न्वेव‚म‚साध्यादेव} श्राव‚ण‚त्व‚स्य ‚{\color{DodgerBlue3}‚न विश्लेष} \leavevmode\ledsidenote{\textenglish{100b/MA}}इत्याख्याने‚{\tiny $_{7}$}‚ ‚{\color{DodgerBlue3}‚साध्य} साध्य‚व‚ति स‚प‚क्षे‚{\color{DodgerBlue3}‚ऽनुवृत्ते}‚र‚न्व‚य‚स्या‚{\color{DodgerBlue3}‚भावोऽर्था}‚दुक्तः स्यात् । यो हि ‚{\tiny $_{lb}$}‚विप‚क्ष‚मात्राद‚व्यावृत्तः स स‚प‚क्षाद‚पि व्यावृत्तेर‚न्व‚य‚र‚हित उक्तो भ‚व‚ति । त‚था च ‚{\tiny $_{lb}$}‚प्राणादेर‚सौ स‚प‚क्षानुवृत्त्य‚भावः ‚{\color{DodgerBlue3}‚स‚मः} । न हि प्राणादिस‚प‚क्षे क्व‚चित् सिद्धं । स‚र्व्वंस्य ‚{\tiny $_{lb}$}‚जीव‚च्छ‚रीर‚स्य प‚क्ष‚त्वात् । (२१९)
	\pend% ending standard par
      \label{div_pvv.4.220}
	  
	% new div opening: depth here is 2
	
	  \bigskip
	  \begingroup
	
	    \large
	  
	    \begin{quote}
	  
	    
	    \stanza[\smallbreak]
	\label{pv.4.220}\flagstanza{\tiny\textenglish{....4.220}}असाध्यादेव विच्छेद इति साध्येस्तितोच्य‚ते ।&अर्थाप‚त्याऽत एवोक्त‚मेकेनोभ‚य‚द‚र्श‚न‚म् ॥ २२० ॥\&[\smallbreak]


	
	    \end{quote}
	  
	  \endgroup
	\textsuperscript{\textenglish{493/s}}

	  \pstart \leavevmode% starting standard par
	\hphantom{.}‚{\color{DodgerBlue3}‚असाध्याद्} विप‚क्षा‚{\color{DodgerBlue3}‚देव} साध‚न‚स्य ‚{\color{DodgerBlue3}‚विच्छेदो} व्यावृत्ति‚{\color{DodgerBlue3}‚रित्य}‚भिधानेनार्थाप‚त्त्य‚{\tiny $_{lb}$}‚साम‚र्थ्येन ‚{\color{DodgerBlue3}‚साध्ये} स‚प‚क्षे‚{\color{DodgerBlue3}‚ऽस्तितोच्य‚ते} । य‚दि तु ‚{\tiny $_{1}$}‚ स‚प‚क्षाद‚पि व्यावृत्तिस्त‚दा विप‚क्षा‚{\tiny $_{lb}$}‚देव व्यावृत्तिरित्य‚स‚ङ्ग‚तं । य‚तो विप‚क्षादेव व्यावृत्तिरिति व्य‚तिरेक आक्षिप्तान्व‚य ‚{\tiny $_{lb}$}‚एव भ‚व‚ति । अत एवा चा र्ये ण ‚{\color{DodgerBlue3}‚एकेन} व्य‚तिरेकेणान्व‚येन वा निय‚म‚व‚ता ‚{\color{DodgerBlue3}‚दृष्टेनोभ‚य‚{\tiny $_{lb}$}‚द‚र्श‚न}‚मुक्तं । अर्थाप‚त्त्याऽन्य‚त‚रेणोभ‚य‚द‚र्श‚नादिति । (२२०)
	\pend% ending standard par
      \label{div_pvv.4.221}
	  
	% new div opening: depth here is 2
	

	  \pstart \leavevmode% starting standard par
	य‚स्मादैकान्तिको व्य‚तिरेक आक्षिप्तान्व‚य एव भ‚व‚ति । अतः (।)
	\pend% ending standard par
      
	  \bigskip
	  \begingroup
	
	    \large
	  
	    \begin{quote}
	  
	    
	    \stanza[\smallbreak]
	\label{pv.4.221}\flagstanza{\tiny\textenglish{....4.221}}ईदृग‚व्य‚भिचारोतोऽन‚न्व‚यिषु न सिध्य‚ति ।&प्र‚तिषेध‚निषेध‚श्च विधानात् कीदृशोऽप‚रः ॥ २२१ ॥\&[\smallbreak]


	
	    \end{quote}
	  
	  \endgroup
	

	  \pstart \leavevmode% starting standard par
	\hphantom{.}‚{\color{DodgerBlue3}‚ईदृग्} विप‚क्षे व्य‚तिरेको (ऽ) ‚{\color{DodgerBlue3}‚व्य‚भिचारः अन‚न्व‚यिषु} हेतुषु ‚{\color{DodgerBlue3}‚न सिध्य‚ति} । य ‚{\tiny $_{lb}$}‚एव‚{\tiny $_{2}$}‚ साध्येनान्वितो हेतुस्त‚स्यैव विप‚क्षादेव व्य‚तिरेकः । य‚दि तु स‚त्य‚पि साध्ये ‚{\tiny $_{lb}$}‚हेत्व‚भाव‚स्त‚दा स‚प‚क्षाद‚पि व्य‚तिरेकात् क‚थं विप‚क्षादेव निवृत्तिः । किञ्च (।) ‚{\tiny $_{lb}$}‚प्राणादेः स‚प‚क्षे ‚{\color{DodgerBlue3}‚प्र‚तिषेध}‚स्य निवृत्ते\edtext{}{\edlabel{pvv.493-1}\label{pvv.493-1}\lemma{निवृत्ते}\Bfootnote{प्र‚तिषेध‚व्याख्या (।) प्राणादेः स‚प‚क्षेऽभावो नास्तीति ।}} ‚{\color{DodgerBlue3}‚निषेधो विधानाद‚प‚रः कीदृशः} । प्र‚तिषेध‚निषेधो ‚{\tiny $_{lb}$}‚हि विधिरेव प‚र‚स्प‚र‚प‚रिहारेणाव‚स्थितेः । त‚तः प्राणादेः ‚{\color{DodgerBlue3}‚स‚प‚क्षान्निवृत्तिर्नास्तीत्य}‚{\tiny $_{lb}$}‚र्थाद् वृत्तिरेवोक्ता स्यादिति न व्य‚तिरेकित्वं । (२२१)
	\pend% ending standard par
      \label{div_pvv.4.222}
	  
	% new div opening: depth here is 2
	

	  \pstart \leavevmode% starting standard par
	स्यादेवं य‚दि स‚प‚क्षो‚{\tiny $_{3}$}‚ भ‚व‚ति । किन्तु (।)
	\pend% ending standard par
      
	  \bigskip
	  \begingroup
	
	    \large
	  
	    \begin{quote}
	  
	    
	    \stanza[\smallbreak]
	\label{pv.4.222}\flagstanza{\tiny\textenglish{....4.222}}निवृत्तिर्नास‚तः साध्याद‚साध्येष्वेव नो त‚तः ।&नेति सैव निवृत्तिः किं निवृत्तेर‚स‚तो म‚ता ॥ २२२ ॥\&[\smallbreak]


	
	    \end{quote}
	  
	  \endgroup
	

	  \pstart \leavevmode% starting standard par
	\hphantom{.}सात्म‚क‚त्वे साध्ये स‚प‚क्षो नास्तीत्य‚{\color{DodgerBlue3}‚स‚तः साध्यात्} स‚प‚क्षात् प्राणादे‚{\color{DodgerBlue3}‚र्निवृत्तिर्नास्ति ‚{\tiny $_{lb}$}‚त‚तोऽसाध्येषु} विप‚क्षेष्वेव ‚{\color{DodgerBlue3}‚नो} वृत्तिरिति व्य‚तिरेकित्व‚मिष्टं । एव‚न्त‚र्हि स‚प‚क्षाद‚{\color{DodgerBlue3}‚स‚तो} हेतो‚{\color{DodgerBlue3}‚र्निवृत्तेर्निवृत्ति}‚र‚स्माकं अस्तित्वेन येष्टा सैव ‚{\color{DodgerBlue3}‚किन्ने\edtext{}{\edlabel{pvv.493-2}\label{pvv.493-2}\lemma{किन्ने}\Bfootnote{क‚स्मात् ।}}ति} भ‚व‚तो ‚{\color{DodgerBlue3}‚म‚ता} । य‚दि ‚{\tiny $_{lb}$}‚ह्य‚स‚न् निवृत्तेर्नाधिक‚र‚णं त‚दा निवृत्त‚निवृत्तेः क‚थं भ‚विष्य‚ति । (२२२)
	\pend% ending standard par
      \label{div_pvv.4.223}
	  
	% new div opening: depth here is 2
	

	  \pstart \leavevmode% starting standard par
	किञ्च (।)
	\pend% ending standard par
      
	  \bigskip
	  \begingroup
	
	    \large
	  
	    \begin{quote}
	  
	    
	    \stanza[\smallbreak]
	\label{pv.4.223}\flagstanza{\tiny\textenglish{....4.223}}निवृत्त्य‚भाव‚स्तु विधिर्व्व‚स्तुभावोऽस‚तोपि स‚न् ।&व‚स्त्व‚भाव‚स्तु नास्तीति प‚श्य बान्ध्य‚विजृम्भित‚म् ॥ २२३ ॥\&[\smallbreak]


	
	    \end{quote}
	  
	  \endgroup
	

	  \pstart \leavevmode% starting standard par
	\hphantom{.}‚{\color{DodgerBlue3}‚अस‚तोपि} स‚प‚क्षाद्धेतु‚{\color{DodgerBlue3}‚निवृत्ते}‚र्नीरूपाया अ‚{\tiny $_{4}$}‚‚{\color{DodgerBlue3}‚भाव‚स्तु विधिर्व‚स्तुभावो} हेतुस‚म्भ‚वः ‚{\tiny $_{lb}$}‚स नेष्य‚ते (।) य‚द्य‚स‚ति स‚प‚क्षे हेतुनिवृत्तिर्नास्ति त‚दा हेतुरेवास्तीत्युक्तं स्यात् । ‚{\tiny $_{lb}$}‚\leavevmode\ledsidenote{\textenglish{494/s}} ‚{\color{DodgerBlue3}‚व‚स्तुनो} हेतो‚{\color{DodgerBlue3}‚र‚भाव‚स्तु} निवृत्तिश‚ब्द‚वाच्यो ‚{\color{DodgerBlue3}‚नास्तीति प‚श्य बान्ध्य‚विजृम्भितं} प‚रेषां । ‚{\tiny $_{lb}$}‚अस‚त्य‚स‚त्त‚व‚म‚विरुद्धं । विरुद्ध‚न्तु स‚त्त्व‚मित्य‚र्थः । (२२३)
	\pend% ending standard par
      \label{div_pvv.4.224}
	  
	% new div opening: depth here is 2
	

	  \pstart \leavevmode% starting standard par
	अपि च (।)
	\pend% ending standard par
      
	  \bigskip
	  \begingroup
	
	    \large
	  
	    \begin{quote}
	  
	    
	    \stanza[\smallbreak]
	\label{pv.4.224}\flagstanza{\tiny\textenglish{....4.224}}निवृत्तिर्य‚दि त‚स्मिन्न हेतोर्वृत्तिः किमिष्य‚ते ।&सापि न प्र‚तिषेधोयं निवृत्तिः किं निषिध्य‚ते ॥ २२४ ॥\&[\smallbreak]


	
	    \end{quote}
	  
	  \endgroup
	

	  \pstart \leavevmode% starting standard par
	\hphantom{.}‚{\color{DodgerBlue3}‚हेतोस्त‚स्मिन्न‚स}‚ति स‚प‚क्षे ‚{\color{DodgerBlue3}‚निवृत्तिर्य‚दि} नास्ति त‚दा ‚{\color{DodgerBlue3}‚किं वृत्तिरिष्य‚ते} विधि‚{\tiny $_{lb}$}‚निषेध‚योर‚न्योन्य‚व्य‚व‚च्छेदा‚{\tiny $_{5}$}‚त्म‚त्वादेक‚प्र‚तिषेध‚स्य त‚द‚प‚र‚विधिनान्त‚रीय‚क‚त्वात् । त‚त‚{\tiny $_{lb}$}‚श्चान्व‚य‚व्य‚तिरेकित्वात् प्राणादिर्न व्य‚तिरेकी । ‚{\color{DodgerBlue3}‚सा} वृत्तिर‚{\color{DodgerBlue3}‚पि ना}‚स‚ति स‚प‚क्ष इति ‚{\tiny $_{lb}$}‚चेत् । न‚नु वृत्तिनिषेधः ‚{\color{DodgerBlue3}‚प्र‚तिषेधोयं} निवृत्त्यात्म‚कः । त‚त‚श्चास‚तो वृत्तिनिवृत्तेर‚धि‚{\tiny $_{lb}$}‚क‚र‚ण‚त्वात् । त‚स्मि‚{\color{DodgerBlue3}‚न्निवृत्तिः किं निषिध्य‚ते} । स‚त्याञ्च निवृत्तौ श्राव‚ण‚त्व‚व‚द‚साधा‚{\tiny $_{lb}$}‚र‚णं प्राणादि स्यात् । (२२४)
	\pend% ending standard par
      \label{div_pvv.4.225}
	  
	% new div opening: depth here is 2
	

	  \begin{center}%% label @type='head'
	\textbf{ग. शाब्दो व्य‚व‚हारो विधिप्र‚तिषेध‚प्र‚योज‚नः}
	\end{center}
	

	  \pstart \leavevmode% starting standard par
	अथाधिक‚र‚ण‚त‚याऽपादान‚त‚या वा स‚तः‚{\tiny $_{6}$}‚ प्र‚तीतिर्नास्तीति न त‚स्मान्निवृत्ति‚{\tiny $_{lb}$}‚रित्युच्य‚ते । एत‚द‚र्थं हि (।)
	\pend% ending standard par
      
	  \bigskip
	  \begingroup
	
	    \large
	  
	    \begin{quote}
	  
	    
	    \stanza[\smallbreak]
	\label{pv.4.225}\flagstanza{\tiny\textenglish{....4.225}}विधानं प्र‚तिषेध‚ञ्च मुक्त्वा शाब्दोस्ति नाप‚रः ।&व्य‚व‚हारः स चास‚त्सु नेति प्राप्तात्र मूक‚ता ॥ २२५ ॥\&[\smallbreak]


	
	    \end{quote}
	  
	  \endgroup
	

	  \pstart \leavevmode% starting standard par
	\hphantom{.}‚{\color{DodgerBlue3}‚विधानं प्र‚तिषेध‚ञ्च मुक्त्वाऽप‚रः श(ा)ब्दो} व्य‚व‚हारो ‚{\color{DodgerBlue3}‚नास्ति} त‚योरेव श‚ब्द‚{\tiny $_{lb}$}‚प्र‚तिपाद्य‚त्वात् । ‚{\color{DodgerBlue3}‚स च} विधिप्र‚तिषेध‚{\color{DodgerBlue3}‚व्य‚व‚हारोऽस‚त्सु नास्तीति । अत्रा}‚स‚त्सु ‚{\color{DodgerBlue3}‚मूक‚तैव ‚{\tiny $_{lb}$}‚प्राप्ता} । त‚था ह्य‚स‚ति विधिव्य‚व‚हार‚स्ताव‚न्नास्त्येव । प्र‚तिषेधोपि त्व‚न्म‚ते नास्तीति ‚{\tiny $_{lb}$}‚युक्तं मूक‚त्वं । (२२५)
	\pend% ending standard par
      \label{div_pvv.4.226}
	  
	% new div opening: depth here is 2
	

	  \pstart \leavevmode% starting standard par
	किञ्च (।)
	\pend% ending standard par
      
	  \bigskip
	  \begingroup
	
	    \large
	  
	    \begin{quote}
	  
	    
	    \stanza[\smallbreak]
	\label{pv.4.226}\flagstanza{\tiny\textenglish{....4.226}}स‚ताञ्च न निषेधोऽस्ति सोऽस‚त्सु च न विद्य‚ते ।&ज‚ग‚त्य‚नेन न्यायेन न‚ञ‚र्थः प्र‚ल‚यं ग‚तः ॥ २२६ ॥\&[\smallbreak]


	
	    \end{quote}
	  
	  \endgroup
	

	  \pstart \leavevmode% starting standard par
	\hphantom{.}‚{\color{DodgerBlue3}‚स‚ताञ्च} प‚दार्थानां ‚{\color{DodgerBlue3}‚न निषेधोस्ति} विद्य‚मान‚त्वात् । स प्र‚तिषेधोऽ‚{\color{DodgerBlue3}‚स‚त्सु‚{\tiny $_{7}$}‚ न} त्व‚द‚भिप्रायाद् ‚{\color{DodgerBlue3}‚विद्य‚ते । अनेन न्यायेन न‚ञोर्थः} प्र‚तिषेधो ‚{\color{DodgerBlue3}‚ज‚ग‚ति} विष‚ये ‚{\color{DodgerBlue3}‚प्र‚ल‚यं} \leavevmode\ledsidenote{\textenglish{101a/MA}}‚{\color{DodgerBlue3}‚ग‚तः} । (२२६)
	\pend% ending standard par
      \label{div_pvv.4.227}
	  
	% new div opening: depth here is 2
	
	  \bigskip
	  \begingroup
	
	    \large
	  
	    \begin{quote}
	  
	    
	    \stanza[\smallbreak]
	\label{pv.4.227}\flagstanza{\tiny\textenglish{....4.227}}देश‚काल‚निषेध‚श्चेद् य‚थास्ति स निषिध्य‚ते ।&न त‚था न य‚था सोस्ति त‚थापि न निषिध्य‚ते ॥ २२७ ॥\&[\smallbreak]


	
	    \end{quote}
	  
	  \endgroup
	

	  \pstart \leavevmode% starting standard par
	\hphantom{.}क्व‚चित् स‚तामेवार्थानाम‚न्य‚यो‚{\color{DodgerBlue3}‚र्देश‚काल}‚योर्न‚ञादिना ‚{\color{DodgerBlue3}‚निषेध} इष्ट‚{\color{DodgerBlue3}‚श्चेत् । य‚था} ‚{\tiny $_{lb}$}‚\leavevmode\ledsidenote{\textenglish{495/s}} य‚द्देश‚काल‚स‚म्ब‚द्ध‚त्वेन ‚{\color{DodgerBlue3}‚सोऽर्थो} नास्ति त‚था ‚{\color{DodgerBlue3}‚निषिध्य‚ते} त्व‚द‚भिप्रायाद‚तो ‚{\color{DodgerBlue3}‚निषेधायो}‚{\tiny $_{lb}$}‚गात् । ‚{\color{DodgerBlue3}‚य‚था} चास्ति स ‚{\color{DodgerBlue3}‚त‚था}‚पि न निषिध्य‚ते स‚त्वात् । (२२७)
	\pend% ending standard par
      \label{div_pvv.4.228}
	  
	% new div opening: depth here is 2
	
	  \bigskip
	  \begingroup
	
	    \large
	  
	    \begin{quote}
	  
	    
	    \stanza[\smallbreak]
	\label{pv.4.228}\flagstanza{\tiny\textenglish{....4.228}}त‚स्मादाश्रित्य श‚ब्दार्थं भ‚वाभाव‚स‚माश्र‚य‚म् ।&अबाह्याश्र‚य‚म‚त्रेष्टं स‚र्वं विधिनिषेध‚न‚म् ॥ २२८ ॥\&[\smallbreak]


	
	    \end{quote}
	  
	  \endgroup
	

	  \pstart \leavevmode% starting standard par
	\hphantom{.}‚{\color{DodgerBlue3}‚त‚स्माच्छ‚ब्द}‚स्यार्थ‚मारोपित‚ब‚हीरूप‚म‚न्य‚व्य‚व‚च्छेद‚म‚{\color{DodgerBlue3}‚बा‚{\tiny $_{1}$}‚ह्याश्र}‚यं बाह्य‚विष‚य‚र‚हितं ‚{\tiny $_{lb}$}‚य एव ‚{\color{DodgerBlue3}‚भावाभाव}‚योर्व्विधिप्र‚तिषेध‚विक‚ल्प‚प्र‚तिपाद्य‚योः ‚{\color{DodgerBlue3}‚स‚माश्र‚य}‚स्त‚मा‚{\color{DodgerBlue3}‚श्रित्य} व्य‚व‚हारे ‚{\tiny $_{lb}$}‚‚{\color{DodgerBlue3}‚स‚र्व्वं विधिनिषेध\edtext{}{\edlabel{pvv.495-1}\label{pvv.495-1}\lemma{विधिनिषेध}\Bfootnote{बौद्धार्थे ।}}न‚मिष्टं} । (२२८)
	\pend% ending standard par
      \label{div_pvv.4.229}
	  
	% new div opening: depth here is 2
	

	  \pstart \leavevmode% starting standard par
	य‚दि व‚स्तु श‚ब्द‚विक‚ल्पाभ्यां विष‚यीक्रिय‚ते स‚र्व्व‚था प्र‚तीतेः श‚ब्द‚प्र‚माणान्त‚र‚{\tiny $_{lb}$}‚वैफ‚ल्य‚प्र‚स‚ङ्गः । त‚तोन्यापोह एव विधिनिषेध‚स‚म्ब‚न्ध‚योर्व्विष‚यः । त‚था श‚ब्दात् ‚{\tiny $_{lb}$}‚स‚त्त्वास‚त्त्वाभ्यां व‚स्तुनः प्र‚तीतौ विधिनिषेध‚योर‚वैफ‚ल्य‚प्र‚स‚ङ्गः । त‚स्मात्‚{\tiny $_{2}$}‚ भावाभाव‚{\tiny $_{lb}$}‚साधार‚णोऽन्व‚य‚व्य‚व‚च्छेदो विधिप्र‚तिषेधाभ्यां स‚म्ब‚ध्य‚त इति ।
	\pend% ending standard par
      
	  \bigskip
	  \begingroup
	
	    \large
	  
	    \begin{quote}
	  
	    
	    \stanza[\smallbreak]
	\label{pv.4.229}\flagstanza{\tiny\textenglish{....4.229}}ताभ्यां स ध‚र्मी स‚म्ब‚द्धः ख्यात्य‚भावेपि तादृशः ।&श‚ब्द‚प्र‚वृत्तेर‚स्तीति सोपीष्टो व्य‚व‚हार‚भाक् ॥ २२९ ॥\&[\smallbreak]


	
	    \end{quote}
	  
	  \endgroup
	

	  \pstart \leavevmode% starting standard par
	\hphantom{.}‚{\color{DodgerBlue3}‚ताभ्यां} विधिप्र‚तिषेधाभ्यां ‚{\color{DodgerBlue3}‚स} श‚ब्दार्थो ‚{\color{DodgerBlue3}‚ध‚र्मी स‚म्ब‚द्धः ख्याति}‚व‚स्तुतो‚{\color{DodgerBlue3}‚ऽभावेपि ‚{\tiny $_{lb}$}‚तादृशो} विधिप्र‚तिषेध‚स‚म्ब‚द्ध\edtext{}{\edlabel{pvv.495-2}\label{pvv.495-2}\lemma{द्ध}\Bfootnote{ल‚क्ष‚ण‚स्य ।}}स्यार्थ‚स्य (।) न हि श‚ब्दार्थ एव क‚श्चित् बाह्या‚{\tiny $_{lb}$}‚बाह्यार्थ‚यो\edtext{}{\edlabel{pvv.495-3}\label{pvv.495-3}\lemma{यो}\Bfootnote{द्व‚यं नास्ति ।}}र‚योगात् । त‚दित‚र\edtext{}{\edlabel{pvv.495-4}\label{pvv.495-4}\lemma{र}\Bfootnote{त‚योर‚न्य‚स्य ।}}स्य चाभावादित्युक्तेः । कुत एव विधिनिषेधाभ्यां ‚{\tiny $_{lb}$}‚त‚स्या स‚म्ब‚न्धः स्यात् । त‚था‚{\color{DodgerBlue3}‚पि स} विधिः प्र‚तिष‚म‚ध‚श्च‚{\tiny $_{3}$}‚ ‚{\color{DodgerBlue3}‚श‚ब्द‚प्र‚वृत्तेर‚स्तीति व्य‚व‚हार‚{\tiny $_{lb}$}‚भागिष्टः} । श‚ब्दो हि प्र‚व‚र्त्त‚मानो विधिप्र‚तिषेध‚व्य‚व‚हारं व‚स्तुतोऽस‚न्त‚म‚प्य‚विद्या‚{\tiny $_{lb}$}‚भ्यास‚तो वास‚नाव‚शादुप‚द‚र्श‚य‚तीति त‚द‚नुरोधात् स‚न्नुच्य‚ते\edtext{}{\edlabel{pvv.495-5}\label{pvv.495-5}\lemma{ते}\Bfootnote{य‚दि न बौद्धो विधिप्र‚तिषेधः ध‚र्मिणा ध‚र्माणाम‚भेदो भेदो वेति दूष‚य‚ति ।}}। (२२९)
	\pend% ending standard par
      \label{div_pvv.4.230_4.231}
	  
	% new div opening: depth here is 2
	
	  \bigskip
	  \begingroup
	
	    \large
	  
	    \begin{quote}
	  
	    
	    \stanza[\smallbreak]
	\label{pv.4.230}\flagstanza{\tiny\textenglish{....4.230}}अन्य‚था स्यात् प‚दार्थानां विधान‚प्र‚तिषेध‚ने ।&एक‚ध‚र्म‚स्य स‚र्वात्म‚विधान‚प्र‚तिषेध‚न‚म् ॥ २३० ॥\&[\smallbreak]


	
	    \end{quote}
	  
	  \endgroup
	
	  \bigskip
	  \begingroup
	
	    \large
	  
	    \begin{quote}
	  
	    
	    \stanza[\smallbreak]
	\label{pv.4.231a}\flagstanza{\tiny\textenglish{...4.231a}}अनानात्म‚त‚या;\&[\smallbreak]


	
	    \end{quote}
	  
	  \endgroup
	

	  \pstart \leavevmode% starting standard par
	\hphantom{.}‚{\color{DodgerBlue3}‚अन्य‚था} य‚द्येवं नेष्य‚ते त‚दा ‚{\color{DodgerBlue3}‚प‚दार्थानां विधान‚प्र‚तिषेध‚ने}‚ऽभ्युप‚ग‚म्य‚माने य‚दि ‚{\tiny $_{lb}$}‚ध‚र्मा ध‚र्मिणोऽभिन्नास्त‚दैक‚स्य ‚{\color{DodgerBlue3}‚ध‚र्म‚स्य} विधाने प्र‚तिषेध‚ने वा कृते ‚{\color{DodgerBlue3}‚स‚र्व्वेषां} ध‚र्मा‚{\tiny $_{4}$}‚णां ‚{\tiny $_{lb}$}‚ध‚{\color{DodgerBlue3}‚र्म्यात्म}‚भूतानां ‚{\color{DodgerBlue3}‚विधान‚प्र‚तिषेध‚नं स्यात्} । (२३०) \edtext{\textsuperscript{*}}{\edlabel{pvv.495-6}\label{pvv.495-6}\lemma{*}\Bfootnote{किं कार‚णं ।}}तेषां ध‚र्माणामेक‚ध‚र्मिस्व‚भाव‚{\tiny $_{lb}$}‚त्वेनानानात्म‚त‚यैक‚स्व‚भाव‚त‚यैक‚स्य ध‚र्म‚स्य विधिः प्र‚तिषेधो वा स‚र्व्व‚स्य भ‚वेत् ॥
	\pend% ending standard par
      \textsuperscript{\textenglish{496/s}}
	  \bigskip
	  \begingroup
	
	    \large
	  
	    \begin{quote}
	  
	    
	    \stanza[\smallbreak]
	\label{pv.4.231b}\flagstanza{\tiny\textenglish{...4.231b}}भेदे नानाविधिनिषेध‚व‚त् ।&एक‚ध‚र्मिण्य‚संहारो विधान‚प्र‚तिषेध‚योः ॥ २३१ ॥\&[\smallbreak]


	
	    \end{quote}
	  
	  \endgroup
	

	  \pstart \leavevmode% starting standard par
	\hphantom{.}ध‚र्मिणः स‚काशाद् ध‚र्माणां ‚{\color{DodgerBlue3}‚भेदे}‚ऽभ्युप‚ग‚म्य‚माने ‚{\color{DodgerBlue3}‚एक}‚स्मिन् ‚{\color{DodgerBlue3}‚ध‚र्मिणि} ध‚र्माणां ‚{\color{DodgerBlue3}‚विधान‚प्र‚तिषेध‚योर‚संहारः} सामानाधिक‚र‚ण्यं न स्यात् । ‚{\color{DodgerBlue3}‚नानाविधि‚{\tiny $_{lb}$}‚निषेध‚व‚त्} । स्व‚त‚न्त्रा‚{\tiny $_{5}$}‚नेक‚प‚दार्थ‚विधिनिषेधाविव नैक‚त्र व‚स्तुनि सामान्याधि‚{\tiny $_{lb}$}‚क‚र‚ण्य‚भाजौ । (२३१)
	\pend% ending standard par
      \label{div_pvv.4.232_4.233}
	  
	% new div opening: depth here is 2
	

	  \pstart \leavevmode% starting standard par
	अस्म‚न्म‚ते तु (।)
	\pend% ending standard par
      
	  \bigskip
	  \begingroup
	
	    \large
	  
	    \begin{quote}
	  
	    
	    \stanza[\smallbreak]
	\label{pv.4.232}\flagstanza{\tiny\textenglish{....4.232}}एक‚ध‚र्मिण‚मुद्दिश्य नानाध‚र्म‚स‚माश्र‚य‚म् ।&विधावेक‚स्य त‚द्भाज‚मिवान्येषामुपेक्ष‚क‚म् ॥ २३२ ॥\&[\smallbreak]


	
	    \end{quote}
	  
	  \endgroup
	
	  \bigskip
	  \begingroup
	
	    \large
	  
	    \begin{quote}
	  
	    
	    \stanza[\smallbreak]
	\label{pv.4.233}\flagstanza{\tiny\textenglish{....4.233}}निषेधे त‚द्विविक्त‚ञ्ज त‚द‚न्येषाम‚पेक्ष‚क‚म् ।&व्य‚व‚हार‚म‚स‚त्यार्थं प्र‚क‚ल्प‚य‚ति धीर्य‚था ॥ २३३ ॥\&[\smallbreak]


	
	    \end{quote}
	  
	  \endgroup
	

	  \pstart \leavevmode% starting standard par
	\hphantom{.}श‚ब्दार्थ ध‚र्मिण‚मेकं ‚{\color{DodgerBlue3}‚नानाध‚र्म‚स‚माश्र‚य‚मेक‚स्य} ध‚र्म‚स्य ‚{\color{DodgerBlue3}‚विधौ} क्रिय‚माणे ‚{\color{DodgerBlue3}‚त‚द्भाज‚{\tiny $_{lb}$}‚मेक}‚ध‚र्म‚स‚म्ब‚द्ध‚मि‚{\color{DodgerBlue3}‚वान्येषां} विधिप्र‚तिषेधाभ्याम‚प‚रामृष्टानां ध‚र्माणामु‚{\color{DodgerBlue3}‚पेक्ष‚कं} (।) ‚{\tiny $_{lb}$}‚त‚थैक‚ध‚र्म‚स्य‚{\tiny $_{6}$}‚ ‚{\color{DodgerBlue3}‚निषेधे} क्रिय‚माणे एकं ध‚र्मिणं नानाध‚र्म‚स‚माश्र‚य‚णं ‚{\color{DodgerBlue3}‚तेन} निषिध्य‚{\tiny $_{lb}$}‚मानेन ध‚र्मेण ‚{\color{DodgerBlue3}‚विविक्त}‚मेव ‚{\color{DodgerBlue3}‚त‚द‚न्येषां} निषिध्य‚मान‚ध‚र्मेत‚रेषां ध‚र्माणा‚{\color{DodgerBlue3}‚म‚पेक्ष‚क‚मु}‚द्दिश्य ‚{\tiny $_{lb}$}‚‚{\color{DodgerBlue3}‚य‚था} क‚ल्पिका ‚{\color{DodgerBlue3}‚धीर्व्य‚व‚हारं} प‚र‚मार्थ‚तो‚{\color{DodgerBlue3}‚ऽस‚त्यार्थं क‚ल्प‚य‚ति} । (२३२, २३३)
	\pend% ending standard par
      \label{div_pvv.4.234}
	  
	% new div opening: depth here is 2
	
	  \bigskip
	  \begingroup
	
	    \large
	  
	    \begin{quote}
	  
	    
	    \stanza[\smallbreak]
	\label{pv.4.234}\flagstanza{\tiny\textenglish{....4.234}}तं त‚थैवाविक‚ल्पार्थं-भेदाश्र‚य‚मुपाग‚ताः ।&अनादिवास‚नोद्भूतं बाध‚न्तेऽर्थं न लौकिक‚म् ॥ २३४ ॥\&[\smallbreak]


	
	    \end{quote}
	  
	  \endgroup
	

	  \pstart \leavevmode% starting standard par
	\hphantom{.}‚{\color{DodgerBlue3}‚त‚थैव} क‚ल्प‚नान‚तिक्र‚मेण ‚{\color{DodgerBlue3}‚त}‚म्व्य‚व‚हार‚म‚र्थ‚{\color{DodgerBlue3}‚भेदाश्र‚य‚म}‚न्य‚व्यावृत्तिविष‚य‚{\color{DodgerBlue3}‚म‚नादि‚{\tiny $_{lb}$}‚\leavevmode\ledsidenote{\textenglish{101b/MA}}वास‚नाया उद्भूतं} । य‚था त‚{\tiny $_{7}$}‚त्त्वेन विक‚ल्पोप‚ग‚ताः प्र‚तिप‚न्ना व्य‚व‚ह‚र्त्तारो लौकिकं ‚{\tiny $_{lb}$}‚बौद्ध‚{\color{DodgerBlue3}‚म‚र्थं} श‚ब्द‚प्र‚तिपादित‚{\color{DodgerBlue3}‚न्न बाध‚न्ते} व्य‚व‚हारोच्छेद‚प्र‚स‚ङ्गात् । (२३४)
	\pend% ending standard par
      \label{div_pvv.4.235}
	  
	% new div opening: depth here is 2
	

	  \begin{center}%% label @type='head'
	\textbf{घ. ध‚र्म‚भेद‚व्य‚व‚हार‚विचारः}
	\end{center}
	

	  \pstart \leavevmode% starting standard par
	क‚थं पुन‚र्द्ध‚र्म‚भेदो व्य‚व‚ह्रिय‚ते इत्याह ।
	\pend% ending standard par
      
	  \bigskip
	  \begingroup
	
	    \large
	  
	    \begin{quote}
	  
	    
	    \stanza[\smallbreak]
	\label{pv.4.235}\flagstanza{\tiny\textenglish{....4.235}}त‚त्फ‚लोऽत‚त्फ‚ल‚श्चार्थो भिन्न एक‚स्त‚त‚स्त‚तः ।&तैस्तैरुप‚प्ल‚वैर्नीत‚स‚ञ्च‚याप‚च‚यैरिव ॥ २३५ ॥\&[\smallbreak]


	
	    \end{quote}
	  
	  \endgroup
	

	  \pstart \leavevmode% starting standard par
	\hphantom{.}अर्थः श‚ब्दादिस्त‚{\color{DodgerBlue3}‚त्फ‚लः} श्रोत्र‚विज्ञानादिहेतु‚{\color{DodgerBlue3}‚र‚त‚त्फ‚ल‚श्च}‚क्षुर्व्विज्ञानाद्य‚हेतुस्त‚तो ‚{\tiny $_{lb}$}‚ग‚न्धादेः श्रोत्र‚ज्ञानाहेतोः (।) ‚{\color{DodgerBlue3}‚त‚तो} रूपादिकाच्च‚क्षुर्व्विज्ञान‚हेतो‚{\color{DodgerBlue3}‚र्भिन्नो} व्यावृत्त‚स्त‚{\tiny $_{lb}$}‚त्त‚द्व्यावृत्तिविशिष्ट‚त‚या क‚ल्पित‚ध‚र्मिध‚र्म‚नानात्वः‚{\tiny $_{1}$}‚ प‚र‚मार्थ‚त ‚{\color{DodgerBlue3}‚एक‚स्तैस्तैरुप‚प्ल‚वैः} ‚{\tiny $_{lb}$}‚\leavevmode\ledsidenote{\textenglish{497/s}} क‚ल्पित‚ध‚र्म‚भेद‚विधिनिषेध‚योर्गोच‚रैर्व्विक‚ल्पैर्य‚थाक्र‚मं ध‚र्मिध‚र्माणां ‚{\color{DodgerBlue3}‚नीचः प्रापितः ‚{\tiny $_{lb}$}‚स‚ञ्च‚योऽप‚च}‚य‚श्च यैस्तै‚{\color{DodgerBlue3}‚रिव} य‚थाध्य‚व‚सायं । (२३५)
	\pend% ending standard par
      \label{div_pvv.4.236}
	  
	% new div opening: depth here is 2
	

	  \pstart \leavevmode% starting standard par
	व‚स्तुतः (।)
	\pend% ending standard par
      
	  \bigskip
	  \begingroup
	
	    \large
	  
	    \begin{quote}
	  
	    
	    \stanza[\smallbreak]
	\label{pv.4.236}\flagstanza{\tiny\textenglish{....4.236}}अत‚द्वान‚पि स‚म्ब‚न्धात् कुत‚श्चिदुप‚नीय‚ते ।&दृष्टिं भेदाश्र‚यैस्तेपि त‚स्माद‚ज्ञात‚विप्ल‚वाः ॥ २३६ ॥\&[\smallbreak]


	
	    \end{quote}
	  
	  \endgroup
	

	  \pstart \leavevmode% starting standard par
	\hphantom{.}‚{\color{DodgerBlue3}‚भेदाश्र‚यै}‚र्व्यावृत्तिविष‚यै‚{\color{DodgerBlue3}‚र‚त‚द्वान्} ध‚र्म‚भेदेन त‚त्स‚ञ्च‚याप‚च‚याभ्यां र‚हितोपि श‚ब्दा‚{\tiny $_{lb}$}‚दि‚{\color{DodgerBlue3}‚दृष्टिं} प्र‚तीत‚व्य‚व‚हार‚मु‚{\color{DodgerBlue3}‚प‚नी}‚य‚ते । विक‚ल्पास्ताव‚द् वास‚नाव‚शात् त‚थोप‚द‚र्श‚य‚न्ति । ‚{\tiny $_{lb}$}‚व्य‚व‚ह‚र्त्तार‚स्तु क‚{\tiny $_{2}$}‚स्माद् विमृश्य न निव‚र्त्त‚न्त इत्याह । अर्थे ‚{\color{DodgerBlue3}‚तेषां व्यावृत्त्याश्र‚येण} क‚ल्पित‚ध‚र्माणां ‚{\color{DodgerBlue3}‚कुत‚श्चित् स‚म्ब‚न्धात्} त‚द‚र्थ‚क्रियाप्राप्तेः प‚रितोषाद‚र्थ‚क्रियार्थिनां ‚{\tiny $_{lb}$}‚व्य‚व‚हारिणां ध‚र्म‚भेदाद् य‚थार्थ‚त्वेन विम‚र्शादाद‚र इति ‚{\color{DodgerBlue3}‚ते} व्य‚व‚ह‚र्त्तारो‚{\color{DodgerBlue3}‚प्य‚ज्ञात‚विप्ल‚वा} ध‚र्म‚भेद‚विधिप्र‚तिषेध‚व्य‚व‚हार‚स्येति य‚थार्थ‚मेव त‚त् म‚न्य‚न्त इति । त‚स्मात् प‚र‚मा‚{\tiny $_{lb}$}‚र्थ‚तोऽस‚त्येव ध‚र्मिणि विधिप्र‚तिषेध‚व्य‚व‚हारात्‚{\tiny $_{3}$}‚ अस‚तोपि स‚प‚क्षाद् विप‚क्षाद् वा ‚{\tiny $_{lb}$}‚हेतुनिवृत्तिः स‚न्दिग्धा । (२३६)
	\pend% ending standard par
      \label{div_pvv.4.237}
	  
	% new div opening: depth here is 2
	

	  \pstart \leavevmode% starting standard par
	किञ्च (।) प्राणादेः स‚प‚क्षो नास्तीति त‚तो न व्य‚तिरेक इति य‚द्युच्य‚ते त‚द‚स‚ङ्ग‚त‚{\tiny $_{lb}$}‚मित्याह ।
	\pend% ending standard par
      
	  \bigskip
	  \begingroup
	
	    \large
	  
	    \begin{quote}
	  
	    
	    \stanza[\smallbreak]
	\label{pv.4.237}\flagstanza{\tiny\textenglish{....4.237}}स‚त्तासाध‚न‚वृत्तेश्च स‚न्दिग्धः स्याद‚स‚न्न सः ।&अस‚त्व‚ञ्चाभ्युप‚ग‚माद‚प्र‚माण‚न्न य‚ज्य‚ते ॥ २३७ ॥\&[\smallbreak]


	
	    \end{quote}
	  
	  \endgroup
	

	  \pstart \leavevmode% starting standard par
	\hphantom{.}आत्म‚नः ‚{\color{DodgerBlue3}‚स‚त्तायां साध‚न}‚स्य प्राणादे‚{\color{DodgerBlue3}‚र्वृत्तेः} कार‚णात् वादिप्र‚तिवादिनोरात्मा ‚{\tiny $_{lb}$}‚‚{\color{DodgerBlue3}‚स‚न्दिग्ध स्यात्} । न ह्य‚स‚त्त‚या निश्चितेऽर्थे क‚श्चित् साध‚न‚माह । प्र‚तिवादी च ‚{\tiny $_{lb}$}‚त‚त्साध‚नं शृण्व‚न् क‚थ‚म‚स‚न्दिग्धो नाम । आत्म‚स‚न्देहाच्च स‚प‚क्षो ‚{\color{DodgerBlue3}‚नास‚न्}‚{\tiny $_{4}$}‚ त‚त् क‚थ‚{\tiny $_{lb}$}‚मुच्य‚ते स‚प‚क्षास‚त्वात् न त‚तो व्यावृत्त इति व्य‚तिरेकी प्राणादिः । स्यादेत‚त् (।) ‚{\tiny $_{lb}$}‚प्र‚तिवादिनः सात्म‚त्वेन क‚स्याश्चिद‚निष्टेः स‚प‚क्षाभाव उच्य‚ते ।
	\pend% ending standard par
      

	  \pstart \leavevmode% starting standard par
	न‚नु प‚राभ्युप‚ग‚म‚प्र‚माण‚म‚प्र‚माण‚म्वा । प्र‚माण‚ञ्चेन्नैरात्म्य‚मेव त‚र्हि सिद्धं ।) ‚{\tiny $_{lb}$}‚अल‚मात्म‚साध‚नोप‚न्यास‚प्र‚यासेन (।) अथा‚{\color{DodgerBlue3}‚प्र‚माणं} (।) प‚र‚स्याभ्युप‚ग‚मात् अप्र‚माण‚{\tiny $_{lb}$}‚कात् स‚प‚क्ष‚स‚त्वं च ‚{\color{DodgerBlue3}‚न युज्य‚ते} । (२३७) त‚स्माद् (।)
	\pend% ending standard par
      \label{div_pvv.4.238}
	  
	% new div opening: depth here is 2
	
	  \bigskip
	  \begingroup
	
	    \large
	  
	    \begin{quote}
	  
	    
	    \stanza[\smallbreak]
	\label{pv.4.238}\flagstanza{\tiny\textenglish{....4.238}}अस‚तो व्य‚तिरेकेपि स‚प‚क्षाद् विनिव‚र्त्त‚न‚म् ।&स‚न्दिग्धं त‚स्य स‚न्देहाद् विप‚क्षाद् विनिव‚र्त्त‚न‚म् ॥ २३८ ॥\&[\smallbreak]


	
	    \end{quote}
	  
	  \endgroup
	

	  \pstart \leavevmode% starting standard par
	\hphantom{.}अ‚{\color{DodgerBlue3}‚स‚तः स‚प‚क्षा}‚द्धेतो‚{\color{DodgerBlue3}‚र्व्य}‚ति‚{\color{DodgerBlue3}‚रेकेप्यु}‚प‚ग‚म्य‚माने प्राणादेः स‚प‚क्षाद् ‚{\color{DodgerBlue3}‚विनिव‚र्त्त‚नं स‚न्दिग्धं ‚{\tiny $_{lb}$}‚त‚स्य} स‚प‚क्ष‚{\color{DodgerBlue3}‚स‚न्देहात्} । य‚दि स‚प‚क्षाभावेाऽस‚न्दिग्ध एवं हेतुव्यावृत्तिभावो\edtext{}{\edlabel{pvv.497-1}\label{pvv.497-1}\lemma{हेतुव्यावृत्तिभावो}\Bfootnote{विप‚क्षात् ।}}प्य‚स‚न्दिग्धः ‚{\tiny $_{lb}$}‚\leavevmode\ledsidenote{\textenglish{498/s}} स्यात् । त‚त्स‚न्देहे त‚स्यापि स‚न्देहः । स‚प‚क्षाद् व्यावृत्तिस‚न्देहे ‚{\color{DodgerBlue3}‚विप‚क्षाद् विनिव‚र्त्त‚नं} प्राणादेः स‚न्दिग्धं । (२३८)
	\pend% ending standard par
      \label{div_pvv.4.239}
	  
	% new div opening: depth here is 2
	

	  \pstart \leavevmode% starting standard par
	क‚थ‚मित्याह (।)
	\pend% ending standard par
      
	  \bigskip
	  \begingroup
	
	    \large
	  
	    \begin{quote}
	  
	    
	    \stanza[\smallbreak]
	\label{pv.4.239}\flagstanza{\tiny\textenglish{....4.239}}एक‚त्र निय‚मे सिद्धे सिध्य‚त्य‚न्य‚निव‚र्त्त‚न‚म् ।&द्वैराश्ये स‚ति दृष्टेषु स्याद‚दृष्टेपि संश‚यः ॥ २३९ ॥\&[\smallbreak]


	
	    \end{quote}
	  
	  \endgroup
	

	  \pstart \leavevmode% starting standard par
	\hphantom{.}‚{\color{DodgerBlue3}‚एक‚त्र} स‚प‚क्षे स‚त्त्व‚स्य ‚{\color{DodgerBlue3}‚निय‚मे सिद्धे} स‚ति ‚{\color{DodgerBlue3}‚अन्य}‚तो विप‚क्षा‚{\color{DodgerBlue3}‚न्निव‚र्त्त‚नं सिध्य‚ति} । यो ‚{\tiny $_{lb}$}‚हि प‚{\tiny $_{6}$}‚क्ष एव भ‚व‚ति स निय‚माद् विप‚क्षे न भ‚व‚ति । य‚दि तु विप‚क्षेपि स्यात् स‚प‚क्षे ‚{\tiny $_{lb}$}‚स‚त्ता निय‚मो व्याह‚न्येत ।
	\pend% ending standard par
      

	  \pstart \leavevmode% starting standard par
	न‚नु सात्म‚क‚त्वानात्म‚क‚त्वाभ्यां द्वैराश्यं भावानां । त‚त्र निरात्म‚केषु घ‚टादिषु ‚{\tiny $_{lb}$}‚दृष्टः प्राणादिर‚र्थात् सात्म‚केषु व्य‚व‚तिष्ठ‚त इति प्राणादेर‚तो युक्तं सात्म‚क‚त्वानुमान‚{\tiny $_{lb}$}‚मित्याह । द्वैराश्ये स‚ति भावानां घ‚टादिषु निरात्म‚केषु दृष्टेषु ब‚हुलं प्राणादाव ‚{\tiny $_{lb}$}‚\leavevmode\ledsidenote{\textenglish{102a/MA}}‚{\color{DodgerBlue3}‚दृष्टेपि}‚{\tiny $_{7}$}‚ देशादिविप्र‚कृष्टेषु प्राणादिस‚त्ता‚{\color{DodgerBlue3}‚संश‚यः} क‚दाचित् क्व‚चिन्निरात्म‚का अपि ‚{\tiny $_{lb}$}‚प्राणादियुक्ताः स्युः बाध‚काभावात् । (२३९) त‚थाहि ।
	\pend% ending standard par
      \label{div_pvv.4.240}
	  
	% new div opening: depth here is 2
	
	  \bigskip
	  \begingroup
	
	    \large
	  
	    \begin{quote}
	  
	    
	    \stanza[\smallbreak]
	\label{pv.4.240}\flagstanza{\tiny\textenglish{....4.240}}अव्य‚क्तिव्यापिनोप्य‚र्थाः स‚न्ति त‚ज्जातिभाविनः ।&क्व‚चिन्न निय‚मोदृष्ट्या पार्थिवालोह‚लेख्य‚व‚त् ॥ २४० ॥\&[\smallbreak]


	
	    \end{quote}
	  
	  \endgroup
	

	  \pstart \leavevmode% starting standard par
	\hphantom{.}‚{\color{DodgerBlue3}‚त‚स्या}‚मेक‚स्यां ‚{\color{DodgerBlue3}‚जातौ} स‚म्भ‚{\color{DodgerBlue3}‚विनोप्य‚र्था} ध‚र्मा ‚{\color{DodgerBlue3}‚अव्य‚क्तिव्यापिनो} निःशेष‚त‚द्व्य‚क्त्य‚{\tiny $_{lb}$}‚‚{\color{DodgerBlue3}‚स‚म्भ‚विनः स‚न्तिः} त‚तः क्व‚चिद् घ‚टादौ निरात्म‚के प्राणादिर्नांस्तीत्य‚{\color{DodgerBlue3}‚दृष्ट्या}‚द‚र्श‚न‚{\tiny $_{lb}$}‚मात्रेण स‚म‚स्तेषु निरात्म‚केषु न प्राणाद्य‚भाव‚स्य ‚{\color{DodgerBlue3}‚निय‚मः} । ब‚हुषु ‚{\color{DodgerBlue3}‚पार्थि}‚वेषु काष्ठ‚पाषाणा‚{\tiny $_{lb}$}‚दिषु लो‚{\tiny $_{1}$}‚ह‚लेख्य‚त्व‚द‚र्श‚नेपि पार्थिव एव\edtext{}{\edlabel{pvv.498-1-bis}\label{pvv.498-1-bis}\lemma{एव}\Bfootnote{लोह‚लेख्यं व‚ज्रं पार्थिव‚त्वात् काष्ठ‚व‚त् ।}} व‚ज्रेऽलोह‚लेख्य‚त्व‚द‚र्श‚नेपि पार्थिव एव\edtext{}{\edlabel{pvv.498-1}\label{pvv.498-1}\lemma{एव}\Bfootnote{लोह‚लेख्यं व‚ज्रं पार्थिव‚त्वात् काष्ठ‚व‚त् ।}} ‚{\tiny $_{lb}$}‚व‚ज्रे‚{\color{DodgerBlue3}‚ऽलोह‚लेख्य‚त्व‚व‚त्} । अलोह‚लेख्य‚स्येव नास‚म्भ‚व‚स्य निय‚मः । (२४०)
	\pend% ending standard par
      \label{div_pvv.4.241}
	  
	% new div opening: depth here is 2
	
	  \bigskip
	  \begingroup
	
	    \large
	  
	    \begin{quote}
	  
	    
	    \stanza[\smallbreak]
	\label{pv.4.241a}\flagstanza{\tiny\textenglish{...4.241a}}भावे विरोध‚स्यादृष्टेः कः स‚न्तेहं निवार‚येत् ।\&[\smallbreak]


	
	    \end{quote}
	  
	  \endgroup
	

	  \pstart \leavevmode% starting standard par
	\hphantom{.}निरात्म‚क‚त्वेन स‚ह प्राणादे‚{\color{DodgerBlue3}‚र्व्विरोध‚स्यादृष्टे}‚र्निरात्म‚केष्व‚पि भावेषु प्राणादेर्भावे ‚{\tiny $_{lb}$}‚स‚त्तायां ‚{\color{DodgerBlue3}‚कः स‚न्देहं निवार‚येत्} । न हि प्राणादेर्नैरात्म्येन स‚हान‚व‚स्थान‚ल‚क्ष‚णो ‚{\tiny $_{lb}$}‚विरोधः (।) क्व‚चिन्निव‚र्त्त्य‚निव‚र्त्त‚क‚भावानुप‚ल‚ब्धेः नाप्य‚न्योन्य‚प‚रिहार‚स्थिति‚{\tiny $_{lb}$}‚ल‚क्ष‚णः प‚र‚स्प‚र‚व्य‚व‚{\tiny $_{2}$}‚च्छेदात्म‚क‚त्वाभावात् ।
	\pend% ending standard par
      

	  \pstart \leavevmode% starting standard par
	स्यादेत‚त् (।) नैरात्म्य‚विरुद्धेनात्म‚ना व्याप्त‚त्वात् प्राणादेः प‚र‚म्प‚र‚या नैरा‚{\tiny $_{lb}$}‚त्म्येन विरोध इत्याह ।
	\pend% ending standard par
      
	  \bigskip
	  \begingroup
	
	    \large
	  
	    \begin{quote}
	  
	    
	    \stanza[\smallbreak]
	\label{pv.4.241b}\flagstanza{\tiny\textenglish{...4.241b}}क्व‚चिद् विनिय‚मात् कोन्य‚स्त‚त्कार्यात्म‚त‚या स च ॥ २४१ ॥\&[\smallbreak]


	
	    \end{quote}
	  
	  \endgroup
	\textsuperscript{\textenglish{499/s}}

	  \pstart \leavevmode% starting standard par
	\hphantom{.}त‚स्यात्म‚नः ‚{\color{DodgerBlue3}‚कार्य‚त‚या} आत्म‚त‚या प्राणादेः ‚{\color{DodgerBlue3}‚क्व‚चिदा}‚त्म‚नि ‚{\color{DodgerBlue3}‚विनिय‚मा}‚न्निय‚त‚त्वात् ‚{\tiny $_{lb}$}‚‚{\color{DodgerBlue3}‚कोन्यः} प‚र‚म्प‚र‚या विरोध‚श्चोक्तः स्यात् । तादात्म्य‚त‚दुत्प‚त्तिभ्यामेव केन‚चित् ‚{\tiny $_{lb}$}‚किञ्चिद् व्याप्य‚ते नान्य‚था । (२४१)
	\pend% ending standard par
      \label{div_pvv.4.242}
	  
	% new div opening: depth here is 2
	

	  \pstart \leavevmode% starting standard par
	न चात्म‚नोऽत्य‚न्त‚प‚रोक्ष‚त‚या ते सिध्य‚तः । त‚त्क‚थ‚न्त‚न्निमित्त‚क‚{\tiny $_{3}$}‚विरोध‚स्थितिः ।
	\pend% ending standard par
      
	  \bigskip
	  \begingroup
	
	    \large
	  
	    \begin{quote}
	  
	    
	    \stanza[\smallbreak]
	\label{pv.4.242a}\flagstanza{\tiny\textenglish{...4.242a}}नैरात्म्याद‚पि तेनास्य स‚न्दिग्धं विनिव‚र्त‚न‚म् ।\&[\smallbreak]


	
	    \end{quote}
	  
	  \endgroup
	

	  \pstart \leavevmode% starting standard par
	\hphantom{.}‚{\color{DodgerBlue3}‚तेन} साक्षात्प‚र‚म्प‚र‚या च विरोधाभावेन ‚{\color{DodgerBlue3}‚नैरात्म्याद्} विप‚क्षा‚{\color{DodgerBlue3}‚द‚पि} श‚ब्दान्न केव‚लं ‚{\tiny $_{lb}$}‚स‚प‚क्षा‚{\color{DodgerBlue3}‚द‚स्य} प्राणादे‚{\color{DodgerBlue3}‚र्निव‚र्त्त‚नं स‚न्दिग्धं} । स‚न्दिग्धान्व‚य‚व्य‚तिरेकः प्राणादिरित्य‚र्थः ।
	\pend% ending standard par
      

	  \pstart \leavevmode% starting standard par
	नास्ति ताव‚दुक्त‚क्र‚मेण नैरात्म्यात् प्राणादिनिवृत्तिः (।)
	\pend% ending standard par
      
	  \bigskip
	  \begingroup
	
	    \large
	  
	    \begin{quote}
	  
	    
	    \stanza[\smallbreak]
	\label{pv.4.242b}\flagstanza{\tiny\textenglish{...4.242b}}अस्तु नाम त‚थाप्यात्मा नानैरात्म्यात् प्र‚सिध्य‚ति ॥ २४२ ॥\&[\smallbreak]


	
	    \end{quote}
	  
	  \endgroup
	

	  \pstart \leavevmode% starting standard par
	\hphantom{.}‚{\color{DodgerBlue3}‚अस्तु नामा}‚ङ्गीकारात् ‚{\color{DodgerBlue3}‚त‚थापि} जीव‚च्छ‚रीरे‚{\color{DodgerBlue3}‚ऽनैरात्म्यान्नै}‚रात्म्याभावात् ‚{\tiny $_{lb}$}‚प्राणादिम‚त्व‚साधिता‚{\color{DodgerBlue3}‚दात्मा न सिध्य‚ति} । (२४२)
	\pend% ending standard par
      \label{div_pvv.4.243}
	  
	% new div opening: depth here is 2
	

	  \pstart \leavevmode% starting standard par
	य‚न्न‚{\tiny $_{4}$}‚ विरुद्ध‚योरेकाभावाद‚न्य‚त‚र‚स्याव‚श्यं सिद्धिर्भ‚त्येवेत्याह ।
	\pend% ending standard par
      
	  \bigskip
	  \begingroup
	
	    \large
	  
	    \begin{quote}
	  
	    
	    \stanza[\smallbreak]
	\label{pv.4.243}\flagstanza{\tiny\textenglish{....4.243}}येनासौ व्य‚तिरेक‚स्य नाभावं भाव‚मिच्छ‚ति ।&य‚था नाव्य‚तिरेकेपि प्राणादिर्न्न स‚प‚क्ष‚तः ॥ २४३ ॥\&[\smallbreak]


	
	    \end{quote}
	  
	  \endgroup
	

	  \pstart \leavevmode% starting standard par
	\hphantom{.}‚{\color{DodgerBlue3}‚येन} कार‚णे‚{\color{DodgerBlue3}‚नासौ} वादी व्य‚तिरेक‚स्या‚{\color{DodgerBlue3}‚भावं भाव}‚म‚न्व‚यं ‚{\color{DodgerBlue3}‚नेच्छ‚ति । य‚था} प्राणादि‚{\tiny $_{lb}$}‚र‚स‚तः स‚प‚क्ष‚तः । ‚{\color{DodgerBlue3}‚अव्य‚तिरेके} व्य‚तिरेकाभावे‚{\color{DodgerBlue3}‚पि न स‚प‚क्षे} स‚त्वेनेष्टः । त‚था नैरा‚{\tiny $_{lb}$}‚त्म्य‚निवृत्ताव‚पि आत्म‚भावो जीव‚च्छ‚रीरे न स्यात् । (२४३)
	\pend% ending standard par
      \label{div_pvv.4.244}
	  
	% new div opening: depth here is 2
	

	  \pstart \leavevmode% starting standard par
	त‚स्मात् (।)
	\pend% ending standard par
      
	  \bigskip
	  \begingroup
	
	    \large
	  
	    \begin{quote}
	  
	    
	    \stanza[\smallbreak]
	\label{pv.4.244}\flagstanza{\tiny\textenglish{....4.244}}स‚प‚क्षाव्य‚तिरेकी चेद्धेतुर्हेतुर‚तोन्व‚यी ।&नान्व‚य्य‚व्य‚तिरेकी चेद‚नैराम्त्यं न सात्म‚क‚म् ॥ २४४ ॥\&[\smallbreak]


	
	    \end{quote}
	  
	  \endgroup
	

	  \pstart \leavevmode% starting standard par
	\hphantom{.}‚{\color{DodgerBlue3}‚स‚प‚क्षाव्य‚तिरेकी} प्राणादिर्नैरात्म्ये निव‚र्त्त‚मा‚{\tiny $_{5}$}‚नः स‚त्तामात्म‚नो ग‚म‚य‚न् ‚{\color{DodgerBlue3}‚हेतुश्चे‚{\tiny $_{lb}$}‚दिष्ट}‚स्त‚दाऽत एव न्यायात् प्राणादि‚{\color{DodgerBlue3}‚हेतुर‚न्व‚यी} स्यात् । स‚प‚क्षात् प्राणादेर्निवृत्त्य‚भाव ‚{\tiny $_{lb}$}‚एव भावः । स चान्व‚यः (।) स‚प‚क्षा‚{\color{DodgerBlue3}‚द‚व्य‚तिरेकी} व्य‚तिरेक‚र‚हितः प्राणादि‚{\tiny $_{6}$}‚‚{\color{DodgerBlue3}‚र्नान्व‚यी ‚{\tiny $_{lb}$}‚चेदि}‚ष्य‚ते त‚दा त‚द्व‚द‚{\color{DodgerBlue3}‚नैरात्म्यं} नैरात्म्य‚र‚हितं श‚रीरं\edtext{}{\edlabel{pvv.499-1}\label{pvv.499-1}\lemma{रीरं}\Bfootnote{नात्र नैरात्म्य‚म‚स्तीति ।}} ‚{\color{DodgerBlue3}‚न सात्म‚कं} स्यात् । (२४४)
	\pend% ending standard par
      \label{div_pvv.4.245}
	  
	% new div opening: depth here is 2
	

	  \pstart \leavevmode% starting standard par
	किञ्च (।)
	\pend% ending standard par
      
	  \bigskip
	  \begingroup
	
	    \large
	  
	    \begin{quote}
	  
	    
	    \stanza[\smallbreak]
	\label{pv.4.245}\flagstanza{\tiny\textenglish{....4.245}}य‚न्नान्त‚रीय‚कः स्वात्मा य‚स्य सिद्धः प्र‚वृत्तिषु ।&निव‚र्त्त‚कः स एवातः प्र‚वृत्तौ च प्र‚व‚र्त्त‚कः ॥ २४५ ॥\&[\smallbreak]


	
	    \end{quote}
	  
	  \endgroup
	

	  \pstart \leavevmode% starting standard par
	\hphantom{.}‚{\color{DodgerBlue3}‚य‚स्य} स्वात्म(ा) स्व‚भावो ‚{\color{DodgerBlue3}‚य‚न्नान्त‚रीय‚कः\edtext{}{\edlabel{pvv.499-2}\label{pvv.499-2}\lemma{कः}\Bfootnote{लिङ्गं धूमादिः ।}}} य‚द‚न्व‚य‚व्य‚तिरेकानुविधायी प्र‚तिब‚न्ध‚{\tiny $_{lb}$}‚\leavevmode\ledsidenote{\textenglish{500/s}} क‚स्य व‚ह्न्यादेः ‚{\color{DodgerBlue3}‚प्र‚वृत्तिषु} वि‚{\tiny $_{6}$}‚धिषु सिद्धः । ‚{\color{DodgerBlue3}‚स एव} निव‚र्त्त‚मानः त‚स्यानुविहितान्व‚य‚{\tiny $_{lb}$}‚व्य‚तिरेक‚स्य ‚{\color{DodgerBlue3}‚निव‚र्त्त‚कः} । ‚{\color{DodgerBlue3}‚अतः} प्र‚वृत्तिविष‚य‚त्वात् ‚{\color{DodgerBlue3}‚प्र‚वृत्तौ} स्व‚स‚त्तायाञ्च त‚स्याव‚श्यं ‚{\tiny $_{lb}$}‚प्र‚वृत्तेः क‚र्त्ता हेतुर्भ‚व‚ति । य‚था धूमो द‚ह‚न‚नान्त‚रीय‚क‚त‚या सिद्धो द‚ह‚न‚निवृत्तिप्र‚{\tiny $_{lb}$}‚वृत्तिभ्यां निव‚र्त्त‚ते प्र‚व‚र्त्त‚ते च (। २४५)
	\pend% ending standard par
      \label{div_pvv.4.246}
	  
	% new div opening: depth here is 2
	
	  \bigskip
	  \begingroup
	
	    \large
	  
	    \begin{quote}
	  
	    
	    \stanza[\smallbreak]
	\label{pv.4.246}\flagstanza{\tiny\textenglish{....4.246}}नान्त‚रीय‚क‚ता सा च साध‚नं स‚म‚पेक्ष‚ते ।&कार्ये दृष्टिर‚दृष्टिश्च कार्य‚कार‚ण‚ता हिता ॥ २४६ ॥\&[\smallbreak]


	
	    \end{quote}
	  
	  \endgroup
	

	  \pstart \leavevmode% starting standard par
	\hphantom{.}‚{\color{DodgerBlue3}‚सा च नान्त‚रीय‚क‚ता साध‚नं} निश्चाय‚कं मान‚{\color{DodgerBlue3}‚म‚पेक्ष‚ते} ग‚म‚क‚त्व‚हेत्व‚धिकारे‚{\tiny $_{lb}$}‚ऽनिश्चित‚ग‚म‚{\tiny $_{7}$}‚क‚त्व‚निब‚न्ध‚स्याहेतुत्वात् । साध‚न‚ञ्च ‚{\color{DodgerBlue3}‚कार्ये} कार‚णान्व‚य‚व‚ति ‚{\color{DodgerBlue3}‚दृष्टि}‚{\tiny $_{lb}$}‚\leavevmode\ledsidenote{\textenglish{102b/MA}}स्त‚द्व्य‚तिरेक‚व‚ति ‚{\color{DodgerBlue3}‚चादृष्टिः} । न तु स‚प‚क्ष‚विप‚क्ष‚योर्द‚र्श‚नाद‚र्श‚न‚मात्र‚कं । हि य‚स्मात् ‚{\tiny $_{lb}$}‚त‚त्कार्य‚दृष्ट्य‚दृष्टी कार्य‚कार‚ण‚ता कार‚ण‚भावाभाव‚प्र‚युक्ते कार्य‚भावाभाव‚द‚र्श‚ने कार्य‚{\tiny $_{lb}$}‚कार‚ण‚तोक्ता (।) अन्य‚निश्च‚योपाय‚ताद‚र्श‚नार्थं । (२४६)
	\pend% ending standard par
      \label{div_pvv.4.247}
	  
	% new div opening: depth here is 2
	

	  \pstart \leavevmode% starting standard par
	त‚त‚श्च (।)
	\pend% ending standard par
      
	  \bigskip
	  \begingroup
	
	    \large
	  
	    \begin{quote}
	  
	    
	    \stanza[\smallbreak]
	\label{pv.4.247a}\flagstanza{\tiny\textenglish{...4.247a}}अर्थान्त‚र‚स्य त‚द्भावेऽभावानिय‚म‚तोऽग‚तिः ।\&[\smallbreak]


	
	    \end{quote}
	  
	  \endgroup
	

	  \pstart \leavevmode% starting standard par
	\hphantom{.}अकार‚ण‚स्या‚{\color{DodgerBlue3}‚र्थान्त‚र‚स्या}‚त्म‚न‚{\color{DodgerBlue3}‚स्त}‚स्य प्राणादे‚{\color{DodgerBlue3}‚र्भावे}‚ऽभावा‚{\color{DodgerBlue3}‚निय‚म}‚तोऽव‚श्य‚म्भावा‚{\tiny $_{lb}$}‚‚{\color{DodgerBlue3}‚भावात् अग}‚तिर‚प्र‚तीतिः ।
	\pend% ending standard par
      

	  \pstart \leavevmode% starting standard par
	न‚न्वात्मा‚{\tiny $_{1}$}‚न्व‚य‚व्य‚तिरेकानुविधानात् प्राणाद‚य आत्मानं त‚त्कार्य‚त‚यानुमाप‚{\tiny $_{lb}$}‚यिष्य‚न्तीत्याह ।
	\pend% ending standard par
      
	  \bigskip
	  \begingroup
	
	    \large
	  
	    \begin{quote}
	  
	    
	    \stanza[\smallbreak]
	\label{pv.4.247b}\flagstanza{\tiny\textenglish{...4.247b}}अभावास‚म्भ‚वात् तेषाम‚भावे नित्य‚भाविनः ॥ २४७ ॥\&[\smallbreak]


	
	    \end{quote}
	  
	  \endgroup
	

	  \pstart \leavevmode% starting standard par
	\hphantom{.}त‚स्यात्म‚नो ‚{\color{DodgerBlue3}‚नित्य‚भाविनः} स‚र्व्व‚काल‚स्थायिनः क‚दाचिद‚भावे स‚ति ‚{\color{DodgerBlue3}‚तेषां} प्राणा‚{\tiny $_{lb}$}‚दीना‚{\color{DodgerBlue3}‚म‚भाव}‚स्या‚{\color{DodgerBlue3}‚स‚म्भ‚वात्} (।) य‚दि व्य‚तिरेक‚म‚न्त‚रेणान्व‚य‚मात्रादात्म‚कार्य‚ता ‚{\tiny $_{lb}$}‚प्राणादीनां त‚दा कालाकालादिकार्य‚तापि स्यात् (।) या य‚स्य स‚मान‚त्वात् (।) ‚{\tiny $_{lb}$}‚त‚त‚श्च तान‚प्य‚नुमाप‚येयुः कालादिनिवृत्तिश्च प्राणादिनिवृत्त्या व्या‚{\tiny $_{2}$}‚प्य‚ते । त‚तो ‚{\tiny $_{lb}$}‚य‚त्र प्राणादिनिवृत्तिर‚स्ति त‚त्र कालादिनिवृत्तिर‚पि स्यात् । एवं व्याप्य‚स‚द्भावात् । ‚{\tiny $_{lb}$}‚न च काल‚र‚हितः क‚श्चिद‚स्ति । (२४७)
	\pend% ending standard par
      \label{div_pvv.4.248}
	  
	% new div opening: depth here is 2
	

	  \begin{center}%% label @type='head'
	\textbf{(६) क. साम‚ग्रीश‚क्तिभेदाद् विश्व‚रूप‚ता}
	\end{center}
	

	  \pstart \leavevmode% starting standard par
	किञ्च (।)
	\pend% ending standard par
      
	  \bigskip
	  \begingroup
	
	    \large
	  
	    \begin{quote}
	  
	    
	    \stanza[\smallbreak]
	\label{pv.4.248}\flagstanza{\tiny\textenglish{....4.248}}कार्य‚स्व‚भाव‚भेदानां कार‚णेभ्यः स‚मुद्भ‚वात् ।&तैर्विना भ‚व‚तोन्य‚स्मात् त‚ज्जं रूपं क‚थं भ‚वेत् ॥ २४८ ॥\&[\smallbreak]


	
	    \end{quote}
	  
	  \endgroup
	

	  \pstart \leavevmode% starting standard par
	\leavevmode\ledsidenote{\textenglish{501/s}}‚{\color{DodgerBlue3}‚कार्य‚स्व‚भाव‚भेदानां\edtext{}{\edlabel{pvv.501-1}\label{pvv.501-1}\lemma{भेदानां}\Bfootnote{अन्व‚य‚मात्रात् प‚रो म‚न्य‚ते त‚न्निषेध‚ति ।}}} य‚था स्वं कार‚णेभ्यः\edtext{}{\edlabel{pvv.501-2}\label{pvv.501-2}\lemma{णेभ्यः}\Bfootnote{आक‚स्मिकानुप‚प‚त्तेः ।}} ‚{\color{DodgerBlue3}‚स‚मुद्भ‚वात्} कार‚णात् ‚{\tiny $_{lb}$}‚केव‚लाद‚न्व‚यात् कार्य‚कार‚ण‚भावेपि कार्य‚भावाङ्गीकारात् तैः कार‚णै‚{\color{DodgerBlue3}‚र्व्विना ‚{\tiny $_{lb}$}‚भ‚व‚तः} कार्य‚स्य ‚{\color{DodgerBlue3}‚रूपं त‚ज्जं क‚थं भ‚वेत्} । न हि व‚ह्निजं युक्तं प्र‚त्येकं व्य‚भिचारा‚{\tiny $_{lb}$}‚दे‚{\tiny $_{3}$}‚र्हेतुत्व‚प्र‚स‚ङ्गात् ॥ (२४८)
	\pend% ending standard par
      \label{div_pvv.4.249}
	  
	% new div opening: depth here is 2
	
	  \bigskip
	  \begingroup
	
	    \large
	  
	    \begin{quote}
	  
	    
	    \stanza[\smallbreak]
	\label{pv.4.249}\flagstanza{\tiny\textenglish{....4.249}}साम‚ग्रीश‚क्तिभेदाद्धि व‚स्तूनां विश्व‚रूप‚ता ।&सा चेन्न भेदिका प्राप्त‚मेक‚रूप‚मिदं ज‚ग‚त् ॥ २४९ ॥\&[\smallbreak]


	
	    \end{quote}
	  
	  \endgroup
	

	  \pstart \leavevmode% starting standard par
	\hphantom{.}‚{\color{DodgerBlue3}‚साम‚ग्रीणां श‚क्तिभेदाद्धि व‚स्तूनां} कार्याणां ‚{\color{DodgerBlue3}‚विश्व‚रूप‚ता} नानात्म‚ता । ‚{\color{DodgerBlue3}‚साम‚ग्री ‚{\tiny $_{lb}$}‚चेत्} स्व‚भेदेन कार्याणां ‚{\color{DodgerBlue3}‚न} भेदिका । ‚{\color{DodgerBlue3}‚ज‚ग‚दिद‚मेक‚रूपं प्राप्तं} । (२४९)
	\pend% ending standard par
      \label{div_pvv.4.250}
	  
	% new div opening: depth here is 2
	

	  \pstart \leavevmode% starting standard par
	न‚नु\edtext{}{\edlabel{pvv.501-3}\label{pvv.501-3}\lemma{नु}\Bfootnote{अग्नितोप्य‚ग्निधूमादि । अग्निस्व‚भाव‚काश‚क्र‚मूर्ध्नोपि धूमादिति व्य‚भिचार इत्य‚न्य‚त उत्पादेऽहेतुत्वं नास्ति । अग्न्यादिविल‚क्ष‚ण‚साम‚ग्र‚याश्च भेद‚क‚म‚भेद‚क‚ञ्च रूप‚म‚स्ति येन धूम‚ञ्ज‚न‚य‚ति त‚द‚भेद‚कं ।}} कार‚णानि कार्य‚णानि कार्य‚मात्राणि ज‚न‚य‚न्ति । न तेषां प‚र‚स्प‚र‚तो भेद‚म‚पि\edtext{}{\edlabel{pvv.501-4}\label{pvv.501-4}\lemma{पि}\Bfootnote{ज‚न‚य‚न्ति ।}} त‚तो ‚{\tiny $_{lb}$}‚धूमः पाव‚कादिव श‚क्र‚मूर्ध्नोपि जायेत । इत्याह ।
	\pend% ending standard par
      
	  \bigskip
	  \begingroup
	
	    \large
	  
	    \begin{quote}
	  
	    
	    \stanza[\smallbreak]
	\label{pv.4.250}\flagstanza{\tiny\textenglish{....4.250}}भेद‚काभेद‚क‚त्वे स्याद् व्याह‚ता भिन्न‚रूप‚ता ।&एक‚स्य नानारूप‚त्वे द्वे रूपे पाव‚केत‚रौ ॥ २५० ॥\&[\smallbreak]


	
	    \end{quote}
	  
	  \endgroup
	

	  \pstart \leavevmode% starting standard par
	\hphantom{.}अस्याः साम‚ग्र‚या ‚{\color{DodgerBlue3}‚भेद‚क‚त्वे} भिन्न‚त्वे भिन्न‚साम‚ग्रीकार्य‚विल‚{\tiny $_{4}$}‚क्ष‚ण‚कार्य‚ज‚न‚क‚त्वे‚{\tiny $_{lb}$}‚ऽभेद‚क‚त्वे च साम‚ग्र‚य‚न्त‚र‚कार्य‚ज‚न‚क‚त्वेऽभ्युप‚ग‚म्य‚माने एक‚स्याः साम‚ग्र‚या ‚{\color{DodgerBlue3}‚भिन्न‚{\tiny $_{lb}$}‚रूप‚ता} नानात्माऽभ्युप‚ग‚ता ‚{\color{DodgerBlue3}‚स्यात्} । सा च ‚{\color{DodgerBlue3}‚व्याह‚ता} । त‚था हि ‚{\color{DodgerBlue3}‚धूमाग्न्यादिसाम‚ग्री} साम‚ग्र‚य‚न्त‚र‚कार्य‚विल‚क्ष‚णं ज‚न‚य‚न्ती त‚स्य भेदिका प्र‚तीता । य‚दि च साम‚ग्र‚य‚न्त‚र‚{\tiny $_{lb}$}‚कार्यं च सा ज‚न‚येत् ‚{\color{DodgerBlue3}‚त‚दाऽभेदिका} च स्यात् । त‚था ‚{\color{DodgerBlue3}‚चैक‚स्य} श‚क्र‚मूर्ध्नो ‚{\color{DodgerBlue3}‚नाना‚{\tiny $_{5}$}‚रूप‚त्वे ‚{\tiny $_{lb}$}‚द्वे रूपे पाव‚केत‚रौ} स्यातां । धूम‚ज‚न‚क‚त्वाद् व‚ह्नित्वं विल‚क्ष‚ण‚रूप‚त्वाच्चाव‚ह्नि‚{\tiny $_{lb}$}‚रूप‚त्वं । न चैते एक‚स्य युक्ते रूप‚भेद‚ल‚क्ष‚ण‚त्वाद् व‚स्तुभेद‚स्य । (२५०)
	\pend% ending standard par
      \label{div_pvv.4.251}
	  
	% new div opening: depth here is 2
	
	  \bigskip
	  \begingroup
	
	    \large
	  
	    \begin{quote}
	  
	    
	    \stanza[\smallbreak]
	\label{pv.4.251a}\flagstanza{\tiny\textenglish{...4.251a}}त‚त् त‚स्या ज‚न‚नं रूप‚म‚न्य‚स्य य‚दि सैव सा ।\&[\smallbreak]


	
	    \end{quote}
	  
	  \endgroup
	

	  \pstart \leavevmode% starting standard par
	\hphantom{.}‚{\color{DodgerBlue3}‚त‚त्} त‚स्मात् ‚{\color{DodgerBlue3}‚त‚स्याः} श‚क्र‚मूर्द्धादिसाम‚ग्र‚या ‚{\color{DodgerBlue3}‚ज‚न‚नं} धूमोत्पाद‚कं ‚{\color{DodgerBlue3}‚रूपं य‚दि} विद्य‚ते ‚{\tiny $_{lb}$}‚त‚दा ‚{\color{DodgerBlue3}‚सैव} द‚ह‚नात्मिकैव ‚{\color{DodgerBlue3}‚सा} श‚क्र‚मूर्द्धादिसाम‚ग्री । अग्निधूम‚योर‚न्व‚य‚व्य‚तिरेक‚द‚र्श‚नात् ‚{\tiny $_{lb}$}‚स एवाग्निर्यो‚{\tiny $_{6}$}‚ धूम‚ज‚न‚कः\edtext{}{\edlabel{pvv.501-5}\label{pvv.501-5}\lemma{कः}\Bfootnote{इति नास्त्येव व्य‚भिचार इत्याश‚यः अव‚ह्नेर‚नुत्प‚त्तेः ।}} । स एव धूमो योऽग्निज‚न्य इति निश्चित‚त्वात् ।
	\pend% ending standard par
      
	  \bigskip
	  \begingroup
	
	    \large
	  
	    \begin{quote}
	  
	    
	    \stanza[\smallbreak]
	\label{pv.4.251b}\flagstanza{\tiny\textenglish{...4.251b}}न त‚स्या ज‚न‚नं रूपं त‚त् त‚स्याः संभ‚वेत् क‚थ‚म् ॥ २५१ ॥\&[\smallbreak]


	
	    \end{quote}
	  
	  \endgroup
	\textsuperscript{\textenglish{502/s}}

	  \pstart \leavevmode% starting standard par
	\hphantom{.}अथ ‚{\color{DodgerBlue3}‚त‚स्याः} श‚क्र‚मूर्द्धादिसाम‚ग्र‚या ‚{\color{DodgerBlue3}‚ज‚न‚नं रूपं} नास्तीति त‚दा ‚{\color{DodgerBlue3}‚त‚द्} धूमाख्यं रूपं ‚{\tiny $_{lb}$}‚व‚ह्निज‚न्य‚{\color{DodgerBlue3}‚म‚स्याः क‚थं संभ‚वेत्} (।) न ह्य‚धूम‚हेतोर्द्धूम‚ज‚न्म युक्तं (।) य‚तो धूम‚{\tiny $_{lb}$}‚ज‚न‚क एव व‚ह्निर्व‚ह्निज‚न्य‚श्च धूमः । (२५१)
	\pend% ending standard par
      \label{div_pvv.4.252}
	  
	% new div opening: depth here is 2
	
	  \bigskip
	  \begingroup
	
	    \large
	  
	    \begin{quote}
	  
	    
	    \stanza[\smallbreak]
	\label{pv.4.252}\flagstanza{\tiny\textenglish{....4.252}}त‚तः स्व‚भावौ निय‚ताव‚न्योन्यं हेतुकार्य‚योः ।&त‚स्मात् स्व‚दृष्टाविव त‚द् दृष्टे कार्येपि ग‚म्य‚ते ॥ २५२ ॥\&[\smallbreak]


	
	    \end{quote}
	  
	  \endgroup
	

	  \pstart \leavevmode% starting standard par
	\hphantom{.}‚{\color{DodgerBlue3}‚त‚तो हेतुकार्य‚यो}‚र‚ग्निधूमाद्योः ‚{\color{DodgerBlue3}‚स्व‚भावौ निय‚तौ अन्योन्यं} त‚ज्ज‚न‚क‚त्व‚त‚द्भा‚{\tiny $_{lb}$}‚\leavevmode\ledsidenote{\textenglish{103a/MA}} वित्वाभ्यां (।) ‚{\color{DodgerBlue3}‚त‚स्मात्} कार‚{\tiny $_{7}$}‚ण‚कार्य‚योर्निय‚माद‚व्य‚भिचार‚हेतोः य‚था ‚{\color{DodgerBlue3}‚स्व‚स्य} कार‚{\tiny $_{lb}$}‚णात्म‚ने(ा)व‚ह्नेः ‚{\color{DodgerBlue3}‚दृष्टौ} त‚त्कार‚ण‚त्वं ग‚म्य‚ते त‚द्व‚त् साक्षाद् द‚र्श‚नाभावे ‚{\color{DodgerBlue3}‚कार्येपि} धूमे ‚{\tiny $_{lb}$}‚‚{\color{DodgerBlue3}‚दृष्टे} त‚त्कार‚णे प‚रोक्ष‚म‚नुमान‚बुद्ध्या ‚{\color{DodgerBlue3}‚ग‚म्य‚ते\edtext{}{\edlabel{pvv.502-1}\label{pvv.502-1}\lemma{ते}\Bfootnote{इत्य‚व्य‚भिचारिकार्य‚लिङ्ग‚म् ।}}} साक्षात् प‚रंप‚र‚या वा त‚दुत्प‚न्नेन ज्ञानेन ‚{\tiny $_{lb}$}‚प्र‚तीय‚मान‚स्य व्य‚भिचाराभावात् ॥ (२५२)
	\pend% ending standard par
      \label{div_pvv.4.253}
	  
	% new div opening: depth here is 2
	
	  \bigskip
	  \begingroup
	
	    \large
	  
	    \begin{quote}
	  
	    
	    \stanza[\smallbreak]
	\label{pv.4.253a}\flagstanza{\tiny\textenglish{...4.253a}}एकं क‚थ‚म‚नेक‚स्मात् क्लेद‚व‚द्दुग्ध‚वारिणः ।\&[\smallbreak]


	
	    \end{quote}
	  
	  \endgroup
	

	  \pstart \leavevmode% starting standard par
	\hphantom{.}(प‚र आह ।) न य‚द्येक‚मेक‚स्मादेव भ‚व‚ति त‚{\color{DodgerBlue3}‚दैकं} कार्यं ‚{\color{DodgerBlue3}‚क‚थ‚म‚नेक‚स्माद्} भ‚व‚ति (।) ‚{\tiny $_{lb}$}‚उदाह‚र‚ण‚माह । ‚{\color{DodgerBlue3}‚क्लेद‚व‚त्} त‚ण्डुलाद्य‚व‚य‚व‚शैथिल्य‚मिव ‚{\color{DodgerBlue3}‚दुग्धाद्वारिण‚श्च} । अग्निस‚ह‚{\tiny $_{lb}$}‚कारि दुग्धं वारि च‚{\tiny $_{1}$}‚ स्व‚यं पृथ‚क् त‚ण्डुलादेर्व्विक्लित्तिं ज‚न‚य‚द् दृश्य‚ते ॥
	\pend% ending standard par
      
	  \bigskip
	  \begingroup
	
	    \large
	  
	    \begin{quote}
	  
	    
	    \stanza[\smallbreak]
	\label{pv.4.253b}\flagstanza{\tiny\textenglish{...4.253b}}द्र‚व‚श‚क्तेः य‚तः क्लेदः सा त्व‚कैव द्व‚योर‚पि ॥ २५३ ॥\&[\smallbreak]


	
	    \end{quote}
	  
	  \endgroup
	

	  \pstart \leavevmode% starting standard par
	(सिद्धान्ती) अयुक्त‚मेत‚त् । अन्यो हि दुग्धेन ज‚नितः क्लेदोऽन्य‚श्च वारिणा । ‚{\tiny $_{lb}$}‚र‚स‚वीर्य‚प‚रिणामाकार‚भेदात् । त‚त् क‚थ‚मेक‚स्माद‚नेकोत्पादः ।
	\pend% ending standard par
      

	  \pstart \leavevmode% starting standard par
	\hphantom{.}अथ स‚त्य‚प्य‚वान्त‚र‚भेदे विजातीय‚व्यावृत्त्या ‚{\color{DodgerBlue3}‚क्लेद} एक उच्य‚ते त‚दा दुग्ध‚वारि‚{\tiny $_{lb}$}‚णोर्य‚तो य‚स्या ‚{\color{DodgerBlue3}‚द्र‚व्य}‚ज‚निकायाः ‚{\color{DodgerBlue3}‚श‚क्तेः} क्लेदो जाय‚ते ‚{\color{DodgerBlue3}‚सा} श‚क्तिस्तु ‚{\color{DodgerBlue3}‚द्व‚योर‚पि} विजा‚{\tiny $_{lb}$}‚तीय‚व्यावृत्ते‚{\color{DodgerBlue3}‚रेकैव} । त‚त् क‚{\tiny $_{2}$}‚थ‚म‚नेक‚स्मादेकोत्प‚त्तिः ॥ (२५३)
	\pend% ending standard par
      \label{div_pvv.4.254}
	  
	% new div opening: depth here is 2
	
	  \bigskip
	  \begingroup
	
	    \large
	  
	    \begin{quote}
	  
	    
	    \stanza[\smallbreak]
	\label{pv.4.254}\flagstanza{\tiny\textenglish{....4.254}}भिन्नाभिन्नः किम‚स्यात्मा भिन्नोथ द्र‚व‚ता क‚थ‚म् ।&अभिन्नेत्युच्य‚ते बुद्धेस्त‚द्रूपाया अभेद‚तः ॥ २५४ ॥\&[\smallbreak]


	
	    \end{quote}
	  
	  \endgroup
	

	  \pstart \leavevmode% starting standard par
	(प‚रः) य‚दि दुग्ध‚स्य क्लेद‚ज‚न‚न‚श‚क्त्यात्म‚त‚या वारिणा स‚हाभेद इष्य‚ते त‚दा ‚{\tiny $_{lb}$}‚‚{\color{DodgerBlue3}‚भिन्नाभिन्नोऽस्यात्मा किम}‚भ्युप‚ग‚न्त‚व्यः । श‚क्त्यात्म‚त‚याऽभेदात् । आकार‚भेदाच्च ‚{\tiny $_{lb}$}‚भेदात् । उत्त‚र‚माह (।)
	\pend% ending standard par
      

	  \pstart \leavevmode% starting standard par
	\hphantom{.}न भिन्नाभिन्न आत्मा किन्तु ‚{\color{DodgerBlue3}‚भिन्न} एव । ‚{\color{DodgerBlue3}‚क‚थ‚न्त‚र्हि} द्व‚योर‚पि ‚{\color{DodgerBlue3}‚द्र‚व‚ता}‚ऽभिन्ना ‚{\tiny $_{lb}$}‚क्लेद‚हेतुरित्युच्य‚ते द्र‚व‚त्व‚स्याभेदात् । अभेदोक्तिर्भेदोक्त्या विरुध्य‚ते ।
	\pend% ending standard par
      \textsuperscript{\textenglish{503/s}}

	  \pstart \leavevmode% starting standard par
	य‚दि नाम स्व‚स्व‚भाव‚स्थित‚त्वा‚{\tiny $_{3}$}‚द् भावानां भेद एव पार‚मार्थिकः त‚थापि ‚{\tiny $_{lb}$}‚केचिद् भावा भिन्ना अपि स्व‚हेतुब‚लायात‚स्व‚रूप‚विशेषात् एक‚कार्य‚कृत इति ‚{\color{DodgerBlue3}‚बुद्धे‚{\tiny $_{lb}$}‚र्विक}‚ल्पिकाया‚{\color{DodgerBlue3}‚स्त‚द्रूपाया} एक‚प्र‚त्य‚व‚म‚र्शाकाराया ‚{\color{DodgerBlue3}‚अभेद‚तः} कार‚णात् द्र‚व‚ताक्लेद‚{\tiny $_{lb}$}‚हेतु‚{\color{DodgerBlue3}‚र‚भिन्नेत्युच्य‚ते} । त‚देक‚व्यावृत्तिविष‚य‚स्याव‚साय‚स्यानुरोधात् । एक‚त्व‚व्य‚व‚हारो न ‚{\tiny $_{lb}$}‚पार‚मार्थिक इत्य‚र्थः । (२५४)
	\pend% ending standard par
      \label{div_pvv.4.255}
	  
	% new div opening: depth here is 2
	
	  \bigskip
	  \begingroup
	
	    \large
	  
	    \begin{quote}
	  
	    
	    \stanza[\smallbreak]
	\label{pv.4.255}\flagstanza{\tiny\textenglish{....4.255}}त‚द्व‚द् भेदेपि द‚ह‚नो द‚ह‚न‚प्र‚त्य‚याश्र‚यः ।&येनांशेनाद‚ध‚द् धूमं तेनांशेन त‚था ग‚तिः ॥ २५५ ॥\&[\smallbreak]


	
	    \end{quote}
	  
	  \endgroup
	

	  \pstart \leavevmode% starting standard par
	\hphantom{.}‚{\color{DodgerBlue3}‚त‚द्व‚त्} क्षीरोद‚क‚व‚त् । अवान्त‚र‚{\color{DodgerBlue3}‚भेदे} स‚त्य‚पि ‚{\color{DodgerBlue3}‚द‚ह‚नो‚{\tiny $_{4}$}‚ येनांशेन} स्व‚भावेन स्व‚रूपोष्ण‚{\tiny $_{lb}$}‚स्प‚र्शाद्यात्म‚केनान‚ग्निव्यावृत्तेन ‚{\color{DodgerBlue3}‚द‚ह‚न} इति ‚{\color{DodgerBlue3}‚प्र‚त्य‚य}‚स्य व्य‚व‚हार‚स्य‚{\color{DodgerBlue3}‚श्र‚यो} विष‚य‚स्तेन ‚{\tiny $_{lb}$}‚स्व‚भावेन धूम‚माद‚ध‚त् ज‚न‚य‚त् धूमा‚{\color{DodgerBlue3}‚श्र}‚यो भ‚व‚ति (।) ‚{\color{DodgerBlue3}‚तेन धूमाश्र‚य‚त्वेन} कार‚णेन ‚{\tiny $_{lb}$}‚धूमालिङ्गाद् द‚ह‚नो हेतुः । ‚{\color{DodgerBlue3}‚त‚था}‚ऽन‚ग्निव्यावृत्तित्वेन ग‚म्य‚ते । (२५५)
	\pend% ending standard par
      \label{div_pvv.4.256}
	  
	% new div opening: depth here is 2
	

	  \pstart \leavevmode% starting standard par
	त‚था हि (।)
	\pend% ending standard par
      
	  \bigskip
	  \begingroup
	
	    \large
	  
	    \begin{quote}
	  
	    
	    \stanza[\smallbreak]
	\label{pv.4.256}\flagstanza{\tiny\textenglish{....4.256}}द‚ह‚न‚प्र‚त्य‚याङ्गादेवान्यापेक्षात् स‚मुद्भ‚वात् ।&धूमोत‚द्व्य‚भिचारीति त‚द्व‚त् कार्यं त‚थाप‚र‚म् ॥ २५६ ॥\&[\smallbreak]


	
	    \end{quote}
	  
	  \endgroup
	

	  \pstart \leavevmode% starting standard par
	\hphantom{.}‚{\color{DodgerBlue3}‚द‚ह‚न‚प्र‚त्य‚य}‚स्या‚{\color{DodgerBlue3}‚ङ्गा}‚न्निमित्ता‚{\color{DodgerBlue3}‚देवा}‚ग्निल‚क्ष‚णाद‚{\color{DodgerBlue3}‚न्यापेक्षादा}‚र्द्रेन्ध‚नादिस‚हायात् ‚{\color{DodgerBlue3}‚स‚मु‚{\tiny $_{lb}$}‚द‚भ‚{\tiny $_{5}$}‚वा}‚दुत्प‚त्तेः कार‚णाद् ‚{\color{DodgerBlue3}‚धूमोऽत‚द्व्य‚भिचारीति} त‚द्व‚त् धूम‚व‚त् । ‚{\color{DodgerBlue3}‚अप‚र‚म‚पि\edtext{}{\edlabel{pvv.503-1}\label{pvv.503-1}\lemma{पि}\Bfootnote{देशादिभिन्नं विजातीय‚व्यावृत्त्या त्व‚भिन्नं ।}} कार्यं} त‚था स्व‚कार‚णाव्य‚भिचारि बोद्ध‚व्यं । (२५६)
	\pend% ending standard par
      \label{div_pvv.4.257}
	  
	% new div opening: depth here is 2
	

	  \pstart \leavevmode% starting standard par
	किं पुन‚र्द्धूमो व‚ह्रिं न व्य‚भिच‚र‚तीत्याह\edtext{}{\edlabel{pvv.503-2}\label{pvv.503-2}\lemma{तीत्याह}\Bfootnote{एत‚देव स्थिर‚यितुमाह ।}} ।
	\pend% ending standard par
      
	  \bigskip
	  \begingroup
	
	    \large
	  
	    \begin{quote}
	  
	    
	    \stanza[\smallbreak]
	\label{pv.4.257}\flagstanza{\tiny\textenglish{....4.257}}धूमेन्ध‚न‚विकाराङ्ग‚ताप‚दे द‚ह‚न‚स्थितेः ।&अन‚ग्निश्चेद‚धूमोऽसौ स‚धूम‚श्चेत् स पाव‚कः ॥ २५७ ॥\&[\smallbreak]


	
	    \end{quote}
	  
	  \endgroup
	

	  \pstart \leavevmode% starting standard par
	\hphantom{.}‚{\color{DodgerBlue3}‚धूम}‚स्येन्ध‚न‚{\color{DodgerBlue3}‚विकारा}‚देर‚{\color{DodgerBlue3}‚ङ्गं} हेतुस्त‚स्य भावो धूमेन्ध‚न‚विकाराङ्ग‚ता त‚स्याः प‚दे ‚{\tiny $_{lb}$}‚आश्र‚य‚व‚स्तुनि ‚{\color{DodgerBlue3}‚द‚ह‚न‚स्थिते}‚र‚ग्निताव्य‚व‚हाराच्छ‚क्र‚मूर्धा‚{\color{DodgerBlue3}‚ऽन‚ग्नि}‚र्द्धूमेन्ध‚न‚विकार‚कारी न ‚{\tiny $_{lb}$}‚‚{\color{DodgerBlue3}‚चेदि}‚ष्य‚ते (।) त‚तो जाय‚मानः प‚दार्थो‚{\color{DodgerBlue3}‚सौ} दृश्य‚मानो‚{\tiny $_{6}$}‚ऽ‚{\color{DodgerBlue3}‚धू}‚मो वाष्पादिरेव । ‚{\color{DodgerBlue3}‚न} ह्य‚ग्निज‚न्योऽन‚ग्निर्भ‚व‚ति । स‚र्व्व‚था साम्यात् ‚{\color{DodgerBlue3}‚स‚धूम‚श्चेदि}‚ष्य‚ते त‚र्हि ‚{\color{DodgerBlue3}‚स} व‚ल्मीकः ‚{\tiny $_{lb}$}‚‚{\color{DodgerBlue3}‚पाव‚क} एव धूम‚ज‚न‚क‚स्यैव पाव‚क‚त्वात् । देश‚काल‚स्व‚भाव‚प्र‚तीतिनिय‚मो हि भावानां ‚{\tiny $_{lb}$}‚नाक‚स्मिकः । त‚थात्वेऽतिप्र‚स‚ङ्गात् । न च नानाहेतुकः प्र‚त्येकः व्य‚भिचारेऽहेतुत्व‚{\tiny $_{lb}$}‚प्र‚स‚ङ्गात् त‚स्मान्निय‚त‚हेतुकः । य‚दि चान्व‚य‚व्य‚तिरेकाभ्यामुप‚ल‚ब्धाग्निकार‚ण‚{\tiny $_{lb}$}‚\leavevmode\ledsidenote{\textenglish{504/s}} \leavevmode\ledsidenote{\textenglish{103b/MA}}भावो धूमः श‚क्र‚{\tiny $_{7}$}‚मूर्ध्नो जाय‚ते । त‚दा व‚ह्नेरेव श‚क्र‚मूर्द्धा । अथान‚ग्निस्त‚थाऽदृश्य‚{\tiny $_{lb}$}‚मान‚त्वात् त‚था सोपि न धूमो वास्यादिरेव आकार‚साम्यात्तु धूमाभः । त‚स्मात्\edtext{}{\edlabel{pvv.504-1}\label{pvv.504-1}\lemma{स्मात्}\Bfootnote{एवं ताव‚त् कार्य‚स्य निय‚मः ।}} ‚{\tiny $_{lb}$}‚कार्य‚कार‚ण‚भाव‚स्य नान्त‚रीय‚क\edtext{}{\edlabel{pvv.504-2}\label{pvv.504-2}\lemma{क}\Bfootnote{निय‚मः ।}}तेति स्थितं ॥ (२५७)
	\pend% ending standard par
      \label{div_pvv.4.258}
	  
	% new div opening: depth here is 2
	
	  \bigskip
	  \begingroup
	
	    \large
	  
	    \begin{quote}
	  
	    
	    \stanza[\smallbreak]
	\label{pv.4.258}\flagstanza{\tiny\textenglish{....4.258}}नान्त‚रीय‚क‚ता ज्ञेया य‚थास्वं हेत्व‚पेक्ष‚या ।&स्व‚भाव‚स्य य‚थोक्तं प्राक् विनाश‚कृत‚क‚त्व‚योः ॥ २५८ ॥\&[\smallbreak]


	
	    \end{quote}
	  
	  \endgroup
	

	  \pstart \leavevmode% starting standard par
	\hphantom{.}‚{\color{DodgerBlue3}‚स्व‚भाव‚स्य} च हेतो‚{\color{DodgerBlue3}‚र्नान्त‚रीय‚क‚ता} साध्या अविनाभाविता ‚{\color{DodgerBlue3}‚ज्ञेया । य‚थास्वं} य‚स्य स्व‚भाव‚{\color{DodgerBlue3}‚हेतोर्य} आत्मीय‚स्तादात्म्य‚साध‚को हेतुः साध‚नं त‚स्या‚{\color{DodgerBlue3}‚पेक्ष‚या । य‚था ‚{\tiny $_{lb}$}‚प्रागुक्तं विनाश‚कृत‚क‚त्व‚यो}‚{\tiny $_{1}$}‚स्तादात्म्य‚साध‚नं अव‚श्यं हि कृत‚कानाम्विनाशः । न ‚{\tiny $_{lb}$}‚चार्थ‚सापेक्षाणाम‚व‚श्यं भावः । त‚तः स्व‚भाव‚त एव कृत‚कान‚न्न‚श्व‚राणां विनाशं ‚{\tiny $_{lb}$}‚प्र‚त्य‚न‚पेक्षित्वादिति द‚र्शितं । त‚देव स्व‚भाव‚कार्य‚योर‚व्य‚भिचारित्वाद्धेतुत्वं । न च ‚{\tiny $_{lb}$}‚प्राणादेरात्म‚कार्य‚त्वं सिद्ध‚मिति क‚थं हेतुता । (२५८)
	\pend% ending standard par
      \label{div_pvv.4.259}
	  
	% new div opening: depth here is 2
	

	  \begin{center}%% label @type='head'
	\textbf{ख. प्राणादेरुक्तो दोष आचार्येण}
	\end{center}
	

	  \pstart \leavevmode% starting standard par
	न‚न्व‚यं प्राणादेरुक्तो दोष‚क‚लापि आ चा र्ये णे ष्ट इति क‚थं ग‚म्य‚त इत्याह ।
	\pend% ending standard par
      
	  \bigskip
	  \begingroup
	
	    \large
	  
	    \begin{quote}
	  
	    
	    \stanza[\smallbreak]
	\label{pv.4.259}\flagstanza{\tiny\textenglish{....4.259}}अहेतुत्व‚ग‚तिन्यायः स‚र्वोयं व्य‚तिरेकिणः ।&अभ्यूह्यः श्राव‚ण‚त्वोक्तेः कृतायाः साम्य‚दृष्ट‚ये ॥ २५९ ॥\&[\smallbreak]


	
	    \end{quote}
	  
	  \endgroup
	

	  \pstart \leavevmode% starting standard par
	स‚र्व्व‚स्यासादार‚{\tiny $_{2}$}‚ण‚स्य दोष‚दुष्ट‚त्वेन ‚{\color{DodgerBlue3}‚साम्य‚दृष्ट‚ये} तुल्य‚तोप‚द‚र्श‚नाया चा र्ये ण ‚{\color{DodgerBlue3}‚श्राव‚{\tiny $_{lb}$}‚ण‚त्व}‚स्यासाधार‚ण‚स्य ‚{\color{DodgerBlue3}‚योक्तिः कृता} त‚स्या ‚{\color{DodgerBlue3}‚एवायं व्य‚तिरेकिणः} प्राणादेर‚न्य‚स्य च ‚{\tiny $_{lb}$}‚हेतोर‚{\color{DodgerBlue3}‚हेतुत्व‚ग‚तिन्यायः स‚र्व्वोऽभ्यूह्यः} । (२५९)
	\pend% ending standard par
      
	  
	% new div opening: depth here is 1
	
\chapter*[{(७) अनुप‚ल‚ब्धिचिन्ता}]{(७) अनुप‚ल‚ब्धिचिन्ता}

	  \begin{center}%% label @type='head'
	\textbf{(१) अनुप‚ल‚ब्धेः पृथ‚ग‚ग्र‚ह‚णे कार‚ण‚म्}
	\end{center}
	\label{div_pvv.4.260}
	  
	% new div opening: depth here is 2
	

	  \pstart \leavevmode% starting standard par
	न‚नु य‚था स्व‚भाव‚कार्य‚सिद्ध्य‚र्थं द्वौ हेतू उक्तौ त‚थाऽनुप‚ल‚ब्धिर‚पि व‚क्तुं युक्ता ‚{\tiny $_{lb}$}‚हेतुत्वात् । स्व‚भावानुप‚ल‚ब्धिस्ताव‚त् तादात्म्य‚प्र‚तिब‚न्धात् स्व‚भाव‚हेतोर्न भिद्य‚{\tiny $_{3}$}‚त ‚{\tiny $_{lb}$}‚इति स्व‚भाव‚हेतुर्निर्देशादेव निर्द्दिष्टा । कार‚ण‚व्याप‚कानुप‚ल‚ब्धिभ्याञ्च निषेध्या‚{\tiny $_{lb}$}‚नुप‚ल‚ब्धिरेव प्र‚तिपाद्य‚त इति न तेपि स्व‚भावानुप‚ल‚ब्धेर्भिद्य‚न्ते । उदाह‚र‚ण‚न्त‚र्हि ‚{\tiny $_{lb}$}‚क‚स्मान्नोक्त‚मित्याह ।
	\pend% ending standard par
      \textsuperscript{\textenglish{505/s}}
	  \bigskip
	  \begingroup
	
	    \large
	  
	    \begin{quote}
	  
	    
	    \stanza[\smallbreak]
	\label{pv.4.260}\flagstanza{\tiny\textenglish{....4.260}}हेतुस्व‚भाव‚निवृत्त्यैवार्थ‚निवृत्तिव‚र्ण‚नात् ।&सिद्धोदाह‚र‚णेत्युक्तानुप‚ल‚ब्धिः पृथ‚ग् न तु ॥ २६० ॥\&[\smallbreak]


	
	    \end{quote}
	  
	  \endgroup
	

	  \pstart \leavevmode% starting standard par
	\hphantom{.}त्रिविधाप्य‚नुप‚ल‚ब्धिर्व्विप‚क्षा‚{\color{DodgerBlue3}‚द्धेतोः} कार‚ण‚स्य ‚{\color{DodgerBlue3}‚स्व‚भाव}‚स्य व्याप‚क‚स्य निवृत्यै‚{\tiny $_{lb}$}‚वार्थ‚स्य कार्य‚स्य व्याप्य‚स्य च ‚{\color{DodgerBlue3}‚निवृत्तेर्व्व‚र्ण‚नात् सिद्धोदाह‚र‚णेति न पृथ‚गुक्ता} । ‚{\tiny $_{lb}$}‚स्व‚भा‚{\tiny $_{4}$}‚‚{\color{DodgerBlue3}‚वानुप(ल)ब्धिः} कार‚ण‚व्याप‚काभाव‚साधिका । त‚योर‚नुप‚ल‚ब्धिस्तु कार्य‚{\tiny $_{lb}$}‚व्याप्याभाव‚साध‚नीति त्रिविधानुप‚ल‚म्भोदाह‚र‚णं सिद्धं । (२६०)
	\pend% ending standard par
      \label{div_pvv.4.261}
	  
	% new div opening: depth here is 2
	

	  \pstart \leavevmode% starting standard par
	\hphantom{.}उप‚ल‚ब्धिल‚क्ष‚ण‚प्राप्त‚विष‚य‚त्व‚म‚नुप‚ल‚ब्धेर्व्विशेष‚णं क‚थं ‚{\color{DodgerBlue3}‚सिद्ध‚मिति चेदाह} ।
	\pend% ending standard par
      
	  \bigskip
	  \begingroup
	
	    \large
	  
	    \begin{quote}
	  
	    
	    \stanza[\smallbreak]
	\label{pv.4.261}\flagstanza{\tiny\textenglish{....4.261}}त‚त्राप्य‚दृश्यात् पुरुषात् प्राणादेर‚निव‚र्त‚नात् ।&स‚न्देह‚हेतुताख्यात्या दृश्येर्थे सेति सूचित‚म् ॥ २६१ ॥\&[\smallbreak]


	
	    \end{quote}
	  
	  \endgroup
	

	  \pstart \leavevmode% starting standard par
	\hphantom{.}‚{\color{DodgerBlue3}‚प्राणादे}‚र्हेतोर‚{\color{DodgerBlue3}‚दृश्यात् पुरुषादा}‚त्म\edtext{}{\edlabel{pvv.505-1}\label{pvv.505-1}\lemma{त्म}\Bfootnote{अन्य‚त्रादृष्ट‚रूप‚स्येत्यादिनात्म‚नोऽदृश्य‚स्याव्य‚तिरेकेण ।}}नोऽ‚{\color{DodgerBlue3}‚निवृत्ते}‚र्निवृत्त्य‚सिद्धे जीव‚च्छ‚रीरे ‚{\color{DodgerBlue3}‚स‚न्देह‚{\tiny $_{lb}$}‚हेतु}‚त‚या ‚{\color{DodgerBlue3}‚आख्यात्या} क‚थ‚नेन त‚त्रानुप‚ल‚ब्धाव‚{\tiny $_{5}$}‚पि ‚{\color{DodgerBlue3}‚दृश्येऽर्थे} याऽनुप‚ल‚ब्धिः ‚{\color{DodgerBlue3}‚सा} भाव‚{\tiny $_{lb}$}‚साधिका नान्ये‚{\color{DodgerBlue3}‚ति सूचित‚मा} चा र्ये ण । य‚दि त्व‚नुप‚ल‚ब्धिरित्येव प्र‚माणं त‚दाऽत्म‚नि ‚{\tiny $_{lb}$}‚प्राणादेर्निवृत्तिः सिध्येत् । त‚त‚श्च जीव‚च्छ‚रीरे व‚र्त्त‚मान‚मात्मानं ग‚म‚येदिति न ‚{\tiny $_{lb}$}‚स‚न्देह‚हेतुः स्यात् । (२६१)
	\pend% ending standard par
      \label{div_pvv.4.262}
	  
	% new div opening: depth here is 2
	

	  \pstart \leavevmode% starting standard par
	न‚नु य‚था व्य‚व‚च्छेद‚विष‚याऽनुप‚ल‚ब्धिः त‚था कार्य‚स्व‚भावाव‚पीति अनुप‚{\tiny $_{lb}$}‚ल‚ब्धिरेवैको हेतुः स्यादित्याह ।
	\pend% ending standard par
      
	  \bigskip
	  \begingroup
	
	    \large
	  
	    \begin{quote}
	  
	    
	    \stanza[\smallbreak]
	\label{pv.4.262}\flagstanza{\tiny\textenglish{....4.262}}अन‚ङ्गीकृत‚व‚स्त्वंशो निषेधः साध्य‚तेन‚या ।&व‚स्तुन्य‚पि तु पूर्वाभ्यां प‚र्युदासो विधान‚तः ॥ २६२ ॥\&[\smallbreak]


	
	    \end{quote}
	  
	  \endgroup
	

	  \pstart \leavevmode% starting standard par
	\hphantom{.}‚{\color{DodgerBlue3}‚अन‚या}‚नुप‚ल‚ब्ध्या‚{\tiny $_{6}$}‚ न केव‚ल‚{\color{DodgerBlue3}‚म्व‚स्तुनि} अव‚स्तुन्य‚पि ‚{\color{DodgerBlue3}‚निषेधः} केव‚लो ‚{\color{DodgerBlue3}‚नाङ्गीकृत‚{\tiny $_{lb}$}‚व‚स्त्वंशः} कार‚ण‚व्याप‚कानुप‚ल‚ब्धिभ्यां व‚स्तुव्य‚व‚च्छेद‚मात्रं अभाव‚व्य‚व‚हार‚श्च ‚{\tiny $_{lb}$}‚साध्य‚ते (।) य‚था प्र‚देशे ध‚र्मिणि घ‚टाभावः । ‚{\color{DodgerBlue3}‚पूर्व्वाभ्यां} कार्य‚स्व‚भावाभ्यां त्व‚न्य‚{\tiny $_{lb}$}‚व्य‚व‚च्छेदे नैक‚स्य ‚{\color{DodgerBlue3}‚विधान‚तः प‚र्युदासः} साध्य‚ते (।) य‚थाऽन‚ग्निव्य‚व‚च्छेदेनाग्निः । ‚{\tiny $_{lb}$}‚नित्य‚व्य‚व‚च्छेदेनानित्य‚त्वं । (२६२)
	\pend% ending standard par
      \label{div_pvv.4.263}
	  
	% new div opening: depth here is 2
	

	  \pstart \leavevmode% starting standard par
	न‚नु स्व‚भावानुप‚ल‚ब्धौ क‚थं तादात्म्यं प्र‚ति‚{\tiny $_{7}$}‚ब‚न्ध इत्याह । \leavevmode\ledsidenote{\textenglish{104a/MA}}
	\pend% ending standard par
      
	  \bigskip
	  \begingroup
	
	    \large
	  
	    \begin{quote}
	  
	    
	    \stanza[\smallbreak]
	\label{pv.4.263a}\flagstanza{\tiny\textenglish{...4.263a}}त‚त्रोप‚ल‚भ्येष्व‚स्तित्व‚मुप‚ल‚ब्धेर्न चाप‚र‚म् ।\&[\smallbreak]


	
	    \end{quote}
	  
	  \endgroup
	

	  \pstart \leavevmode% starting standard par
	\hphantom{.}‚{\color{DodgerBlue3}‚त‚त्रोप‚ल‚भ्येषु} भावेषु विचार्य‚माण‚म‚{\color{DodgerBlue3}‚स्तित्व‚मुप‚ल‚ब्धेर‚प‚रं न} भ‚व‚ति । य‚दि ‚{\tiny $_{lb}$}‚ह्युप‚ल‚ब्धिः क‚र्म‚ध‚र्म‚स्त‚दोप‚ल‚भ्य‚मान‚तास्तित्वं । अथ क‚र्तृध‚र्मो ज्ञानं । त‚दा ‚{\tiny $_{lb}$}‚\leavevmode\ledsidenote{\textenglish{506/s}} त‚द‚न्व‚य‚व्य‚तिरेकानुविधानाद् भाव‚स‚त्ताव्य‚व‚स्थान‚स्योप‚चारात् सैव स‚त्ता । य‚था ‚{\tiny $_{lb}$}‚चोप‚ल‚ब्धिरेव स‚त्ता त‚थाऽनुप‚ल‚ब्धिरेवास‚त्तेति तादात्म्य‚म‚न‚योः स‚म्ब‚न्धः ।
	\pend% ending standard par
      

	  \begin{center}%% label @type='head'
	\textbf{अनुप‚ल‚ब्धेर‚भाव‚सिद्धौ लिङ्ग‚लिङ्गिग्र‚ह‚ण‚प्र‚योज‚न‚म्}
	\end{center}
	

	  \pstart \leavevmode% starting standard par
	एवं त‚र्ह्य‚नुप‚ल‚ब्धिरेवाभाव‚सिद्धिरित्य‚लं लिङ्गि‚{\tiny $_{1}$}‚लिङ्ग‚भावेनेत्याह ।
	\pend% ending standard par
      
	  \bigskip
	  \begingroup
	
	    \large
	  
	    \begin{quote}
	  
	    
	    \stanza[\smallbreak]
	\label{pv.4.263b}\flagstanza{\tiny\textenglish{...4.263b}}इत्य‚ज्ञ‚ज्ञाप‚नायैकानुपाख्योदाहृतिर्म‚ता ॥ २६३ ॥\&[\smallbreak]


	
	    \end{quote}
	  
	  \endgroup
	

	  \pstart \leavevmode% starting standard par
	\hphantom{.}अभावः सिद्ध‚त्वान्न साध‚नार्हः । त‚स्माद‚भावं प‚श्य‚तोप्य‚व्य‚व‚ह‚र‚तो‚{\color{DodgerBlue3}‚ऽज्ञ}‚स्य मूढ‚स्य ‚{\tiny $_{lb}$}‚‚{\color{DodgerBlue3}‚ज्ञाप‚ना}‚याभाव‚व्य‚व‚हाराय ‚{\color{DodgerBlue3}‚एका} स्व‚भावानुप‚ल‚म्भो‚{\color{DodgerBlue3}‚दाहृतिर्म‚ता} (।) सा चानुप‚{\tiny $_{lb}$}‚ल‚ब्धिर‚{\color{DodgerBlue3}‚नुपाख्या} व‚स्तुतोऽभावात्मिका । (२६३)
	\pend% ending standard par
      \label{div_pvv.4.264_4.265}
	  
	% new div opening: depth here is 2
	

	  \begin{center}%% label @type='head'
	\textbf{(३) स्व‚भावानुप‚ल‚ब्ध्या विष‚यिणः प्र‚तिषेधः}
	\end{center}
	

	  \pstart \leavevmode% starting standard par
	य‚द्य‚भावः स्व‚भावानुप‚ल‚ब्ध्या न साध्य‚ते किन्त‚र्हि साध्य‚त इत्याह ।
	\pend% ending standard par
      
	  \bigskip
	  \begingroup
	
	    \large
	  
	    \begin{quote}
	  
	    
	    \stanza[\smallbreak]
	\label{pv.4.264}\flagstanza{\tiny\textenglish{....4.264}}विष‚यास‚त्त्व‚त‚स्त‚त्र विष‚यि प्र‚तिषिध्य‚ते ।&ज्ञानाभिधान‚स‚न्देहं य‚थाऽदाहाद‚पाव‚कः ॥ २६४ ॥\&[\smallbreak]


	
	    \end{quote}
	  
	  \endgroup
	
	  \bigskip
	  \begingroup
	
	    \large
	  
	    \begin{quote}
	  
	    
	    \stanza[\smallbreak]
	\label{pv.4.265}\flagstanza{\tiny\textenglish{....4.265}}त‚थान्या नोप‚ल‚भ्येषु नास्तितानुप‚ल‚म्भ‚नात् ।&त‚ज्ज्ञान‚श‚ब्दाः साध्य‚न्ते त‚द्भावात् त‚न्निब‚न्ध‚नाः ॥ २६५ ॥\&[\smallbreak]


	
	    \end{quote}
	  
	  \endgroup
	

	  \pstart \leavevmode% starting standard par
	\hphantom{.}‚{\color{DodgerBlue3}‚त‚त्र} स्व‚भावानु(प)ल‚म्भे ‚{\color{DodgerBlue3}‚विष‚यास‚त्व‚तो} ज्ञानाभिधान\edtext{}{\edlabel{pvv.506-1}\label{pvv.506-1}\lemma{ज्ञानाभिधान}\Bfootnote{स‚त्ताज्ञानं । त‚द‚भिधानं । अस्ति न वेति स‚न्देहः ।}}स‚न्देहानां विष‚य‚स्या‚{\tiny $_{lb}$}‚स‚त्त्व‚तः ‚{\tiny $_{2}$}‚ कार‚णाद् ‚{\color{DodgerBlue3}‚विष‚यि प्र‚तिषिध्य‚ते} । किन्त‚द्विष‚यीत्याह । ज्ञान‚म‚भिधान‚ञ्च ‚{\tiny $_{lb}$}‚स‚न्देह‚श्च ‚{\color{DodgerBlue3}‚ज्ञानाभिधान‚संदेहं} । स‚दिति ज्ञानं स‚दित्य‚भिधानं । अस्ति न वेति ‚{\tiny $_{lb}$}‚स‚न्देह‚श्चानुप‚ल‚ब्धिल‚क्ष‚णाद‚भावान्निषिध्य‚ते (।) स‚त्त्व‚विष‚या हि स‚त्-ज्ञानाद‚यः । ‚{\tiny $_{lb}$}‚त‚द‚भावे निषिध्य‚न्त इति न्याय्यं । दृष्टान्त‚माह । ‚{\color{DodgerBlue3}‚य‚था} गुञ्जादौ व‚ह्नित्व‚स‚न्देहे ‚{\tiny $_{lb}$}‚क‚श्चिदाह पाव‚कोय‚मिति (।) दाहादिनिमित्तो हि व‚ह्नि‚{\tiny $_{3}$}‚त्व‚व्य‚व‚हारः ‚{\tiny $_{lb}$}‚त‚द‚भावात् प्र‚तिषिद्धः । य‚थोप‚ल‚ब्धेर‚न्या नास्तिता त‚थोप‚ल‚भ्येषूप‚ल‚ब्धिल‚क्ष‚ण‚{\tiny $_{lb}$}‚प्राप्तेष्व‚नुप‚ल‚म्भाद‚न्या नास्तिता न भ‚व‚ति । किन्त्व‚नुप‚ल‚ब्धिरेव नास्तित्वं (।) ‚{\tiny $_{lb}$}‚त‚स्मात् त‚न्निब‚न्ध‚ना अनुप‚ल‚ब्धिनिमित्ताः त‚त् ज्ञानाभिधान‚व्य‚व‚हारास्त‚स्यानुप‚{\tiny $_{lb}$}‚ल‚म्भ‚न‚स्य भावात् साध्य‚न्ते । (२६४,२६५)
	\pend% ending standard par
      \label{div_pvv.4.266}
	  
	% new div opening: depth here is 2
	

	  \pstart \leavevmode% starting standard par
	य‚द्य‚नुप‚ल‚म्भेन निमित्तेन नैमित्तिकोऽस‚त् ज्ञानादिः साध्य‚ते त‚दा निमित्ते ‚{\tiny $_{lb}$}‚\leavevmode\ledsidenote{\textenglish{507/s}} ‚{\tiny $_{4}$}‚ स‚ति नैमित्तिक‚भाव‚निय‚माभावात् स‚त्य‚नुप‚ल‚म्भेऽस‚त्‌ज्ञानादिव्य‚व‚हार ऐका‚{\tiny $_{lb}$}‚न्तिको न स्यादित्याह (।)
	\pend% ending standard par
      
	  \bigskip
	  \begingroup
	
	    \large
	  
	    \begin{quote}
	  
	    
	    \stanza[\smallbreak]
	\label{pv.4.266}\flagstanza{\tiny\textenglish{....4.266}}सिद्धो हि व्य‚व‚हारोयं दृश्यादृष्टाव‚स‚न्निति ।&त‚स्याः सिद्धाव‚स‚न्दिग्धौ त‚त्कार्य‚त्वेपि धीध्व‚नी ॥ २६६ ॥\&[\smallbreak]


	
	    \end{quote}
	  
	  \endgroup
	

	  \pstart \leavevmode% starting standard par
	\hphantom{.}स‚र्व्व‚स्यैव ‚{\color{DodgerBlue3}‚हि दृश्यादृष्टाव‚स‚न्निति व्य‚व‚हारोयं} निमित्तान्त‚र‚निर‚पेक्षः ‚{\color{DodgerBlue3}‚सिद्धः । ‚{\tiny $_{lb}$}‚त‚स्या} दृश्यादृष्टेः ‚{\color{DodgerBlue3}‚सिद्धौ} स‚त्यां अस‚दिति ‚{\color{DodgerBlue3}‚धीध्व‚नी कार्य‚त्वे}‚प्य‚{\color{DodgerBlue3}‚स‚दिग्धौ} निय‚त‚{\color{DodgerBlue3}‚प्र‚व}‚{\tiny $_{lb}$}‚र्त्त‚नौ सिद्धौ । य‚द्य‚पि कार्यं न कार‚णेन निय‚त‚भावं त‚थापि य‚दि क्व‚चिद् भाव‚व्य‚व‚हारः ‚{\tiny $_{lb}$}‚प्र‚{\tiny $_{5}$}‚व‚र्त्त्य‚ते त‚दानुप‚ल‚ब्धिमान‚निमित्त‚त्वाद‚न्य‚त्राप्य‚नुप‚ल‚ब्धौ स‚त्यां स प्र‚व‚र्त्त‚नीय ‚{\tiny $_{lb}$}‚इत्य‚र्थः । (२६६)
	\pend% ending standard par
      \label{div_pvv.4.267}
	  
	% new div opening: depth here is 2
	
	  \bigskip
	  \begingroup
	
	    \large
	  
	    \begin{quote}
	  
	    
	    \stanza[\smallbreak]
	\label{pv.4.267}\flagstanza{\tiny\textenglish{....4.267}}विद्य‚मानेपि विष‚ये मोहाद‚त्रान‚नुब्रुव‚न् ।&केव‚लं सिद्ध‚साध‚र्म्यात् स्मार्य‚ते स‚म‚यं प‚रः ॥ २६७ ॥\&[\smallbreak]


	
	    \end{quote}
	  
	  \endgroup
	

	  \pstart \leavevmode% starting standard par
	\hphantom{.}अभाव‚व्य‚व‚हार‚स्य ‚{\color{DodgerBlue3}‚विष‚ये} दृश्याद‚र्श‚ने ‚{\color{DodgerBlue3}‚विद्य‚माने}‚पि ‚{\color{DodgerBlue3}‚केव‚लं मोहाद‚स‚त् ज्ञान}‚{\tiny $_{lb}$}‚श‚ब्द‚व्य‚व‚हारान‚{\color{DodgerBlue3}‚न‚नुब्रुव‚न्} नानुव‚द‚न् प‚रोऽस‚द्व्य‚व‚हार‚विष‚य‚त‚या ‚{\color{DodgerBlue3}‚सिद्धे}‚न घ‚टेन ‚{\tiny $_{lb}$}‚दृष्टान्तेन दृश्याद‚र्श‚न‚व‚त्तायाः ‚{\color{DodgerBlue3}‚साध‚र्म्यात् स‚म‚यं} व्य‚व‚हारं ‚{\color{DodgerBlue3}‚स्मार्य‚ते} । पूर्व्व‚म‚पि ‚{\tiny $_{lb}$}‚त्व‚या दृश्याद‚र्श‚न‚मात्र‚कोऽस‚द्व्य‚व‚हा‚{\tiny $_{6}$}‚रः प्र‚व‚र्त्तितः । त‚त्स‚{\color{DodgerBlue3}‚द्भावा}‚दिहापि प्र‚व‚र्त्त‚{\tiny $_{lb}$}‚येति प‚रः प्र‚तिपाद्य‚ते । (२६७)
	\pend% ending standard par
      \label{div_pvv.4.268}
	  
	% new div opening: depth here is 2
	

	  \pstart \leavevmode% starting standard par
	दृष्टान्त‚माह ।
	\pend% ending standard par
      
	  \bigskip
	  \begingroup
	
	    \large
	  
	    \begin{quote}
	  
	    
	    \stanza[\smallbreak]
	\label{pv.4.268}\flagstanza{\tiny\textenglish{....4.268}}कार्य‚कार‚ण‚ता य‚द्व‚त् साध्य‚ते दृष्ट्य‚दृष्टितः ।&कार्यादिश‚ब्दा हि त‚योर्व्य‚व‚हाराय क‚ल्पिताः ॥ २६८ ॥\&[\smallbreak]


	
	    \end{quote}
	  
	  \endgroup
	

	  \pstart \leavevmode% starting standard par
	\hphantom{.}‚{\color{DodgerBlue3}‚कार्य‚कार‚ण‚ता दृष्ट्य‚दृष्टितो} द‚र्श‚नाद‚र्श‚नाभ्याम‚न्व‚य‚व्य‚तिरेक‚ग्राह‚काभ्यां ‚{\color{DodgerBlue3}‚य‚द्व‚त् साध्य}‚ते । न हि व‚स्तुतो द‚र्श‚नाद‚र्श‚नाभ्याम‚न्या कार्य‚कार‚ण‚ता । किन्तु ‚{\color{DodgerBlue3}‚त‚यो}‚र्द‚र्श‚नाद‚र्श‚न‚यो‚{\color{DodgerBlue3}‚र्हि} संक्षेपेण ‚{\color{DodgerBlue3}‚व्य‚व‚हाराय कार्यादिश‚ब्दाः क‚ल्पिताः} । (२६८) त‚त‚श्च (।)
	\pend% ending standard par
      \label{div_pvv.4.269}
	  
	% new div opening: depth here is 2
	
	  \bigskip
	  \begingroup
	
	    \large
	  
	    \begin{quote}
	  
	    
	    \stanza[\smallbreak]
	\label{pv.4.269}\flagstanza{\tiny\textenglish{....4.269}}कार‚णात् कार्य‚संसिद्धिः स्व‚भावान्त‚र्ग‚मादिय‚म् ।&हेतुप्र‚भेदाख्याने न द‚र्शितोदाहृतिः पृथ‚क् ॥ २६९ ॥\&[\smallbreak]


	
	    \end{quote}
	  
	  \endgroup
	

	  \pstart \leavevmode% starting standard par
	\hphantom{.}‚{\color{DodgerBlue3}‚कार‚णात्} दृश्याद‚र्श‚नात् अस‚त्ज्ञान‚श‚ब्द‚व्य‚व‚हार‚{\tiny $_{7}$}‚‚{\color{DodgerBlue3}‚कार्य}‚योग्य‚ता‚{\color{DodgerBlue3}‚संसिद्धिः} 104b ‚{\color{DodgerBlue3}‚स्व‚भाव}‚हेताव‚{\color{DodgerBlue3}‚न्त‚र्ग‚मात्र} हेत्व‚न्त‚रं । तेनेयं स्व‚भावानुप‚ल‚ब्धि‚{\color{DodgerBlue3}‚र्हेतु}‚ना ‚{\color{DodgerBlue3}‚प्र‚भेद‚स्याख्याने} क्रिय‚माणे स्व‚भाव‚हेतुनैव ‚{\color{DodgerBlue3}‚द‚र्शितोदाहृतिर्न पृथ‚क्} निर्द्दिष्टा । (२६९)
	\pend% ending standard par
      \label{div_pvv.4.270}
	  
	% new div opening: depth here is 2
	

	  \pstart \leavevmode% starting standard par
	भ‚व‚तु ताव‚द् दृश्यानुप‚ल‚ब्धेर‚स‚द्व्य‚व‚हार‚योग्य‚तासिद्धिः । सैव तु क‚थं सिध्य- तीत्याह ।
	\pend% ending standard par
      \textsuperscript{\textenglish{508/s}}
	  \bigskip
	  \begingroup
	
	    \large
	  
	    \begin{quote}
	  
	    
	    \stanza[\smallbreak]
	\label{pv.4.270}\flagstanza{\tiny\textenglish{....4.270}}एकोप‚ल‚म्भानुभ‚वादिदं नोप‚ल‚भे इति ।&बुद्धेरुप‚ल‚भे वेति क‚ल्पिकायाः स‚मुद्भ‚वः ॥ २७० ॥\&[\smallbreak]


	
	    \end{quote}
	  
	  \endgroup
	

	  \pstart \leavevmode% starting standard par
	\hphantom{.}‚{\color{DodgerBlue3}‚एक}‚ज्ञान‚संस‚र्गिण एक‚स्य प्र‚देश‚स\edtext{}{\edlabel{pvv.508-1}\label{pvv.508-1}\lemma{स}\Bfootnote{अन‚न्य‚संस‚र्गिणः ।}}यो‚{\color{DodgerBlue3}‚प‚ल‚म्भानुभ‚वात्} स्व‚स‚म्वेद‚नाद‚न‚न्त‚रं घ‚टादि नोप‚ल‚भ्य‚ते । ‚{\color{DodgerBlue3}‚इदं} प्र‚देशोभ्युप‚ल ‚{\tiny $_{1}$}‚ भ्य‚त इति ‚{\color{DodgerBlue3}‚क‚ल्पिकाया बुद्धेः} स्व‚वेद‚न- विष‚यीकृत‚विधिप्र‚तिषेधानुकारिण्याः ‚{\color{DodgerBlue3}‚स‚मुद्भ‚वो} भ‚व‚तीति स्व‚स‚म्वेद‚नादेवानुप‚ल‚म्भ- सिद्धिः । (२७०)
	\pend% ending standard par
      \label{div_pvv.4.271}
	  
	% new div opening: depth here is 2
	

	  \begin{center}%% label @type='head'
	\textbf{(४) विशिष्ट‚वेद‚नाद‚र्थाकां विशेषाव‚ग‚मः}
	\end{center}
	

	  \pstart \leavevmode% starting standard par
	भ‚व‚तु ज्ञानं स्व‚स‚म्वेद‚न‚विष‚य‚योर्ज्ञान‚योर‚न्योन्य‚भेद‚स्तु केन ज्ञाय‚ते इत्याह\edtext{}{\edlabel{pvv.508-2}\label{pvv.508-2}\lemma{इत्याह}\Bfootnote{एकोप‚ल‚म्भादिति वाच्येऽनुभ‚व‚ग्र‚ह‚ण‚प्र‚योज‚न‚माह ।}} ।
	\pend% ending standard par
      
	  \bigskip
	  \begingroup
	
	    \large
	  
	    \begin{quote}
	  
	    
	    \stanza[\smallbreak]
	\label{pv.4.271}\flagstanza{\tiny\textenglish{....4.271}}विशेषो ग‚म्य‚तेऽर्थानां विशिष्टादेव वेद‚नात् ।&त‚थाभूतात्म‚संप‚त्तिर्भेद‚धीहेतुर‚स्य च ॥ २७१ ॥\&[\smallbreak]


	
	    \end{quote}
	  
	  \endgroup
	

	  \pstart \leavevmode% starting standard par
	\hphantom{.}‚{\color{DodgerBlue3}‚अर्थानां विषेशो} (? शेषो)ऽन्योन्यं भेदो ग‚म्य‚ते (।) ‚{\color{DodgerBlue3}‚विशिष्ट}‚प्र‚तिनिय‚ता- कारा‚{\color{DodgerBlue3}‚देव \edtext{}{\edlabel{pvv.508-3}\label{pvv.508-3}\lemma{देव}\Bfootnote{नार्थ‚स‚त्तामात्रेण ।}} वेद‚नात्} । अस्य स‚म्वेद‚न‚स्य त‚यो‚{\color{DodgerBlue3}‚र्भेदा}‚निय‚ताकार‚त्वं त‚स्य धियः‚{\tiny $_{2}$}‚ स‚म्वेद‚न‚स्य तु साध‚नं\edtext{}{\edlabel{pvv.508-4}\label{pvv.508-4}\lemma{नं}\Bfootnote{किमित्याह ।}} ‚{\color{DodgerBlue3}‚त‚थाभूता} प्र‚तिनिय‚ताकारा प‚र‚निर‚पेक्ष‚प्र‚काशा‚{\color{DodgerBlue3}‚त्म‚संप‚त्ति}‚- र‚प‚रोक्ष‚ता । (२७१)
	\pend% ending standard par
      \label{div_pvv.4.272}
	  
	% new div opening: depth here is 2
	
	  \bigskip
	  \begingroup
	
	    \large
	  
	    \begin{quote}
	  
	    
	    \stanza[\smallbreak]
	\label{pv.4.272}\flagstanza{\tiny\textenglish{....4.272}}त‚स्मात् स्व‚तो धियोर्भेद‚सिद्धिस्ताभ्यां त‚द‚र्थ‚योः ।&अन्य‚था ह्य‚न‚व‚स्थातो भेदः सिध्येन्न क‚स्य‚चित् ॥ २७२ ॥\&[\smallbreak]


	
	    \end{quote}
	  
	  \endgroup
	

	  \pstart \leavevmode% starting standard par
	\hphantom{.}‚{\color{DodgerBlue3}‚त‚स्माद् धियो भेद‚स्य स्व‚तः} प्र‚तिनिय‚तात् अप‚रोक्ष‚प्र‚काशात् स्व‚रूप‚त\edtext{}{\edlabel{pvv.508-5}\label{pvv.508-5}\lemma{त}\Bfootnote{नार्थ‚स्यैवान्य‚तो ज्ञानात् ।}}एव ‚{\color{DodgerBlue3}‚सिद्धिः । ताभ्यां} भिन्न‚त‚या सिद्धाभ्यां त‚यो‚{\color{DodgerBlue3}‚र‚र्थ‚योः} सारूप्य‚ज‚न‚क‚योर्भेद‚सिद्धिः । ‚{\color{DodgerBlue3}‚अन्य‚था} य‚द्येवं नेष्य‚ते त‚दाऽप‚राभ्यां भेद‚सिद्धिर्व्व‚क्त‚व्या । त‚योश्च भेद‚सिद्धौ‚{\tiny $_{3}$}‚ भेद‚ग्राह‚क‚ता युक्तेति त‚द्भेद‚ग्राह‚क‚म‚प‚रं द्व‚य‚मेष्ट‚व्यं । एव‚म‚प‚राप‚रापेक्षाया- ‚{\color{DodgerBlue3}‚म‚न‚व‚स्थातः न क‚स्य‚चिद् भेदः सिध्येत्} । (२७२)
	\pend% ending standard par
      \label{div_pvv.4.273}
	  
	% new div opening: depth here is 2
	
	  \bigskip
	  \begingroup
	
	    \large
	  
	    \begin{quote}
	  
	    
	    \stanza[\smallbreak]
	\label{pv.4.273}\flagstanza{\tiny\textenglish{....4.273}}विशिष्ट‚रूपानुभ‚वाद‚न्य‚थान्य‚निराक्रिया ।&त‚द्विशिष्टोप‚ल‚म्भोतः त‚स्याप्य‚नुप‚ल‚म्भ‚न‚म् ॥ २७३ ॥\&[\smallbreak]


	
	    \end{quote}
	  
	  \endgroup
	

	  \pstart \leavevmode% starting standard par
	\hphantom{.}अतो ‚{\color{DodgerBlue3}‚विशिष्ट}‚स्य निय‚त‚स्य ‚{\color{DodgerBlue3}‚रूप}‚स्या‚{\color{DodgerBlue3}‚नुभ‚वाद}‚न्य‚{\color{DodgerBlue3}‚थान्य}‚स्य ‚{\color{DodgerBlue3}‚निराक्रिया} न भ‚व‚ति । ‚{\color{DodgerBlue3}‚अत‚स्त}‚स्मात् प्र‚तिषेध्याद् ‚{\color{DodgerBlue3}‚विशिष्ट}‚स्य भिन्न‚स्य प्र‚देशादे‚{\color{DodgerBlue3}‚रुप‚ल‚म्भ‚स्त‚स्य} प्र‚तिषे- ध्य‚स्या‚{\color{DodgerBlue3}‚प्य‚नुप‚ल‚म्भ‚नं} । (२७३)
	\pend% ending standard par
      \label{div_pvv.4.274}
	  
	% new div opening: depth here is 2
	\textsuperscript{\textenglish{509/s}}
	  \bigskip
	  \begingroup
	
	    \large
	  
	    \begin{quote}
	  
	    
	    \stanza[\smallbreak]
	\label{pv.4.274}\flagstanza{\tiny\textenglish{....4.274}}त‚स्माद‚नुप‚ल‚म्भोयं स्व‚यं प्र‚त्य‚क्ष‚तो ग‚तः ।&स्व‚मात्र‚वृत्तेर्ग‚म‚क‚स्त‚द‚भाव‚व्य‚व‚स्थितेः ॥ २७४ ॥\&[\smallbreak]


	
	    \end{quote}
	  
	  \endgroup
	

	  \pstart \leavevmode% starting standard par
	\hphantom{.}‚{\color{DodgerBlue3}‚त‚स्माद‚यं} ज्ञानात्म‚काऽ‚{\color{DodgerBlue3}‚नुप‚ल‚म्भः स्व‚य}‚मात्म‚{\tiny $_{4}$}‚ना ‚{\color{DodgerBlue3}‚प्र‚त्य‚क्ष‚तः} स्व‚स‚म्वेद‚नाद् ‚{\color{DodgerBlue3}‚ग‚तः} प्र‚तीतः स‚न् ‚{\color{DodgerBlue3}‚स्व‚मात्र‚वृत्ते}‚रात्म‚मात्र‚प्र‚तीताया‚{\color{DodgerBlue3}‚स्त}‚स्यानुप‚ल‚भ्य‚मान‚स्या‚{\color{DodgerBlue3}‚भाव‚व्य‚व‚स्थिते- र्ग‚म‚कः} । अनुप‚ल‚म्भो हि निमित्तं \edtext{}{\edlabel{pvv.509-1}\label{pvv.509-1}\lemma{निमित्तं}\Bfootnote{ज्ञानेन निमित्तेन नैमित्तिको व्य‚व‚हारः साध्यः । अर्थो विष‚योनुप‚ल‚म्भ‚श्चेद् विष‚यी व्य‚व‚हारः ।}}विष‚यो वाऽभाव‚व्य‚व‚हार‚स्येत्युभ‚य‚थापि स्व‚स‚त्तामात्रेणाभाव‚व्य‚व‚हार‚हेतुः । (२७४)
	\pend% ending standard par
      \label{div_pvv.4.275}
	  
	% new div opening: depth here is 2
	
	  \bigskip
	  \begingroup
	
	    \large
	  
	    \begin{quote}
	  
	    
	    \stanza[\smallbreak]
	\label{pv.4.275}\flagstanza{\tiny\textenglish{....4.275}}[अन्य‚थार्थ‚स्य नास्तित्वं ग‚म्य‚तेनुप‚ल‚म्भ‚तः ।&उप‚ल‚म्भ‚स्य नास्तित्व‚म‚न्येनेत्य‚न‚व‚स्थितिः ॥ २७५ ॥\&[\smallbreak]


	
	    \end{quote}
	  
	  \endgroup
	

	  \pstart \leavevmode% starting standard par
	\hphantom{.}‚{\color{DodgerBlue3}‚अन्य‚था} य‚दि स्व‚तो नानुप‚ल‚म्भ‚सिद्धिस्त‚दाऽ‚{\color{DodgerBlue3}‚र्थ‚स्य नास्तित्व‚म‚नुप‚ल‚म्भ‚स्त‚तो ग‚म्य‚ते} (।) ‚{\color{DodgerBlue3}‚उप‚ल‚म्भ‚स्य नास्तित्व‚म}‚नुप‚{\tiny $_{5}$}‚ल‚ब्धिर‚{\color{DodgerBlue3}‚न्येना}‚नुप‚ल‚म्भेन ग‚म्य‚ते । सोप्य- न‚प‚ल‚म्भान्त‚रेणे‚{\color{DodgerBlue3}‚ति} अपेक्षाया‚{\color{DodgerBlue3}‚म‚न‚व‚स्थितिः} स्यात् । (२७५)
	\pend% ending standard par
      \label{div_pvv.4.276}
	  
	% new div opening: depth here is 2
	

	  \begin{center}%% label @type='head'
	\textbf{(५) दृश्यानुप‚ल‚ब्धिः स‚द्व्य‚व‚हार‚बाधिका}
	\end{center}
	

	  \pstart \leavevmode% starting standard par
	भ‚व‚तु ताव‚द् दृश्येष्व‚नुप‚ल‚ब्धाव‚भाव‚प्र‚तीतिर‚दृश्ये पुनः क‚थ‚मित्याह ।
	\pend% ending standard par
      
	  \bigskip
	  \begingroup
	
	    \large
	  
	    \begin{quote}
	  
	    
	    \stanza[\smallbreak]
	\label{pv.4.276}\flagstanza{\tiny\textenglish{....4.276}}अदृश्ये निश्च‚यायोगात् स्थितिर‚न्य‚त्र बाध्य‚ते ।&य‚थाऽलिङ्गोऽन्य‚स‚त्त्वेषु विक‚ल्पादिर्न सिध्य‚ति ॥ २७६ ॥\&[\smallbreak]


	
	    \end{quote}
	  
	  \endgroup
	

	  \pstart \leavevmode% starting standard par
	\hphantom{.}‚{\color{DodgerBlue3}‚अन्य‚त्र} दृश्यानुप‚ल‚ब्धौ स‚त्याम‚{\color{DodgerBlue3}‚दृश्ये} विष‚येऽभाव‚{\color{DodgerBlue3}‚निश्च‚यायोगात्} । स‚द्व्य‚व- हार‚स्य ‚{\color{DodgerBlue3}‚स्थितिर्ब्बाध्य‚ते} उप‚ल‚म्भ‚पूर्व्व‚क‚त्वात् त‚स्याः । ‚{\color{DodgerBlue3}‚य‚थाऽन्येषु स‚त्त्वेषु प्राणिषु} रागादिविष‚यो ‚{\color{DodgerBlue3}‚विक‚ल्पो}‚{\tiny $_{6}$}‚ प‚र‚चित्त‚ज्ञानादिरादिनाऽलिङ्गो लिङ्ग‚र‚हितः स‚द्व्य‚व- हार‚विष‚य‚त्वेन न सिध्य‚ति । (२७६)
	\pend% ending standard par
      \label{div_pvv.4.277}
	  
	% new div opening: depth here is 2
	

	  \pstart \leavevmode% starting standard par
	किं पुन‚र‚दृश्यानुप‚ल‚ब्धौ स‚त्याम‚पि न सिध्य‚तीत्याह ।
	\pend% ending standard par
      
	  \bigskip
	  \begingroup
	
	    \large
	  
	    \begin{quote}
	  
	    
	    \stanza[\smallbreak]
	\label{pv.4.277a}\flagstanza{\tiny\textenglish{...4.277a}}अनिश्च‚य‚फ‚ला ह्येषा नालं व्यावृत्तिसाध‚ने ।\&[\smallbreak]


	
	    \end{quote}
	  
	  \endgroup
	

	  \pstart \leavevmode% starting standard par
	\hphantom{.}‚{\color{DodgerBlue3}‚एषा हि} स‚त्य‚प्य‚र्थे स‚म्भ‚व‚न्ती अभाव‚स्या‚{\color{DodgerBlue3}‚निश्च‚य‚फ‚ला} (।) त‚स्माद् ‚{\color{DodgerBlue3}‚व्यावृत्ते}‚- र‚भाव‚स्य ‚{\color{DodgerBlue3}‚साध‚ने नालं} श‚क्ता ।
	\pend% ending standard par
      
	  \bigskip
	  \begingroup
	
	    \large
	  
	    \begin{quote}
	  
	    
	    \stanza[\smallbreak]
	\label{pv.4.277b}\flagstanza{\tiny\textenglish{...4.277b}}आद्याधिक्रिय‚ते हेतोर्न्निश्चितेनैव साध‚ने ॥ २७७ ॥\&[\smallbreak]


	
	    \end{quote}
	  
	  \endgroup
	

	  \pstart \leavevmode% starting standard par
	\hphantom{.}‚{\color{DodgerBlue3}‚आद्या} दृश्यानुप‚ल‚ब्धिः पुन‚र्व्यावृत्तिसाध‚ने‚{\color{DodgerBlue3}‚ऽधिक्रिय‚ते} । क‚स्मादित्याह । \leavevmode\ledsidenote{\textenglish{510/s}} \leavevmode\ledsidenote{\textenglish{105a/MA}}‚{\color{DodgerBlue3}‚हेतो}‚र्व्विप‚क्षात् कार‚ण‚व्याप‚कानुप‚ल‚ब्धिभ्यां व्यावृत्ते‚{\color{DodgerBlue3}‚र्निश्चिते\edtext{}{\edlabel{pvv.510-1}\label{pvv.510-1}\lemma{र्निश्चिते}\Bfootnote{तृतीयेन रूपेण ।}}नै‚{\tiny $_{7}$}‚व} साध्यार्थ‚{\color{DodgerBlue3}‚साध‚ने} ऽधिकारात् । न हि विप‚क्षाद‚निश्चित‚व्यावृत्तिको हेतुर्ग‚म‚कः । न हि कार‚ण- व्याप‚कानुप‚ल‚ब्धिभ्याम‚न्यो व्यावृत्तिसाध‚नः कार‚ण‚विरुद्धोप‚ल‚म्भादिष्व‚पि कार‚णा- भावाद‚भावः प्र‚तिपाद्यः । (२७७ )
	\pend% ending standard par
      \label{div_pvv.4.278}
	  
	% new div opening: depth here is 2
	
	  \bigskip
	  \begingroup
	
	    \large
	  
	    \begin{quote}
	  
	    
	    \stanza[\smallbreak]
	\label{pv.4.278}\flagstanza{\tiny\textenglish{....4.278}}त‚स्याः स्व‚यं प्र‚योगेषु स्व‚रूपं वा प्र‚युज्य‚ते ।&अर्थ‚बाध‚न‚रूप‚म्वा भावे भावाद‚भाव‚तः ॥ २७८ ॥\&[\smallbreak]


	
	    \end{quote}
	  
	  \endgroup
	

	  \pstart \leavevmode% starting standard par
	\hphantom{.}‚{\color{DodgerBlue3}‚त‚स्या} अनुप‚ल‚ब्धेः ‚{\color{DodgerBlue3}‚प्र‚योगेषु स्व‚यं} श‚ब्द‚प्र‚तिपादितं वा ‚{\color{DodgerBlue3}‚स्व‚रूपं प्र‚युज्य‚ते} । य‚था स्व‚भाव‚कार‚ण‚व्याप‚कानुप‚ल‚ब्ध्यादिषु । निषेध्य‚स्यार्थ‚स्य ‚{\color{DodgerBlue3}‚बाध‚नं} विरुद्धं ‚{\color{DodgerBlue3}‚वा} प्र‚युज्य‚ते य‚था स्व‚भाव‚कार‚ण‚व्याप‚क‚विरुद्धोप‚ल‚ब्ध्यादिषु प्र‚{\tiny $_{1}$}‚युज्य‚त इत्याह । अविक‚ल‚कार‚ण‚त‚या शीत‚स्य ‚{\color{DodgerBlue3}‚भावे} स‚त्तायां स‚त्यां द‚ह‚न‚स्य ‚{\color{DodgerBlue3}‚भावात् । अभाव‚तो} निवृत्तेः । य‚था शीताभाव‚साध‚ने स‚हान‚व‚स्थान‚विरुद्धो व‚ह्नेः प्र‚युज्य‚ते नात्र शीत‚स्प‚र्शो व‚ह्रेरिति । (२७८)
	\pend% ending standard par
      \label{div_pvv.4.279}
	  
	% new div opening: depth here is 2
	
	  \bigskip
	  \begingroup
	
	    \large
	  
	    \begin{quote}
	  
	    
	    \stanza[\smallbreak]
	\label{pv.4.279}\flagstanza{\tiny\textenglish{....4.279}}अन्योन्य‚भेद‚सिद्धेर्वा ध्रुव‚भाव‚विनाश‚व‚त् ।&प्र‚माणान्त‚र‚बाधाद् वा सापेक्ष‚ध्रुव‚भाव‚व‚त् ॥ २७९ ॥\&[\smallbreak]


	
	    \end{quote}
	  
	  \endgroup
	

	  \pstart \leavevmode% starting standard par
	\hphantom{.}निषेध्य‚विधीय‚मान‚योर‚{\color{DodgerBlue3}‚न्योन्य}‚स्याभावात्म‚त‚या ‚{\color{DodgerBlue3}‚भेद‚सिद्धेर्व्वा}‚ऽर्थ‚बाध‚न‚रूपं प्र‚युज्य‚ते । ‚{\color{DodgerBlue3}‚ध्रुव‚भाव‚विनाश‚व‚त्} । नित्य‚त्वानित्य‚त्व‚योर‚न्योन्याभावात्म‚क‚त्वेन भेद‚सिद्धौ प‚र‚स्प‚र‚प‚रिहार‚स्थितिल‚क्ष‚{\tiny $_{2}$}‚ण‚त‚या विरुद्धं विनाशित्वं नित्य‚त्व‚बाध‚ने प्र‚युज्य‚ते । य‚था नित्यः श‚ब्दो विनाशित्वात् । स चायं प‚र‚स्प‚र‚प‚रिहार‚स्थितिविरुद्धः क्व‚चित् साक्षाद् वा प्र‚युज्य‚ते य‚था नित्य‚त्व‚स्य विनाशित्वं । क्व‚चिद् ध‚र्मान्त‚र- विरोध‚ग्राहिणा ‚{\color{DodgerBlue3}‚प्र‚माणान्त}‚रेण प‚र‚म्प‚र‚या ‚{\color{DodgerBlue3}‚बा}‚ध‚नात् । अबाधित‚विरुद्ध‚भावो ‚{\color{DodgerBlue3}‚वा} प्र‚युज्य‚ते । ‚{\color{DodgerBlue3}‚सापेक्ष‚ध्रुव‚भावित्व‚व‚त्} । य‚था सापेक्ष‚ध्रुव‚भाव‚योः साक्षाद् विरो- धाभावेपि ध्रुव‚भा‚{\tiny $_{3}$}‚वित्व‚स्य य‚द् व्याप‚कं तेन विरुद्धं सापेक्ष‚त्वं व्याप्य‚व्याप‚क- योश्च व‚स्तुत‚स्तादात्म्याद‚यो य‚द्व्याप‚के विरुध्य‚ते स त‚द्व्याप्येनापीति प्र‚माणान्त‚र- बाध‚नादेवाव‚धृत‚विरुद्ध‚भावं सापेक्ष‚त्वं ध्रुव‚भावित्व‚बाध‚ने प्र‚युज्य‚ते (।) य‚था न ध्रुव‚भावी कृत‚क‚स्य विनाशो हेत्व‚न्त‚र‚सापेक्ष‚त्वादिति । (२७९)
	\pend% ending standard par
      
	  
	% new div opening: depth here is 2
	

	  \pstart \leavevmode% starting standard par
	क‚स्मात् पुन‚र्हेत्व‚न्त‚र‚सापेक्षो न ध्रुव‚भावीत्याह ।
	\pend% ending standard par
      \textsuperscript{\textenglish{511/s}}
	  
	% new div opening: depth here is 1
	
\chapter*[{८. भाव‚स्व‚भाव‚चिन्ता}]{८. भाव‚स्व‚भाव‚चिन्ता}

	  \begin{center}%% label @type='head'
	\textbf{(१) हेत्व‚न्त‚र‚सापेक्षो न ध्रुव‚भावः}
	\end{center}
	\label{div_pvv.4.280}
	  
	% new div opening: depth here is 2
	
	  \bigskip
	  \begingroup
	
	    \large
	  
	    \begin{quote}
	  
	    
	    \stanza[\smallbreak]
	\label{pv.4.280a}\flagstanza{\tiny\textenglish{...4.280a}}हेत्व‚न्त‚र‚स‚मुत्थ‚स्य स‚न्निधौ निय‚तः कुतः ।\&[\smallbreak]


	
	    \end{quote}
	  
	  \endgroup
	

	  \pstart \leavevmode% starting standard par
	\hphantom{.}उत्पाद‚काद्धेतो‚{\color{DodgerBlue3}‚र्हेत्व‚न्त‚र‚{\tiny $_{4}$}‚स‚मुत्थ‚स्य} ध‚र्म‚स्य ‚{\color{DodgerBlue3}‚स‚न्निधौ} स‚न्निधाने ‚{\color{DodgerBlue3}‚निय‚तः कुतः} । य‚था कार‚णान्त‚र‚सापेक्ष‚स्य वास‚सि नाव‚श्यंभाव‚निय‚मो राग‚स्य ।
	\pend% ending standard par
      

	  \begin{center}%% label @type='head'
	\textbf{(२) न भाव‚न‚श्व‚र‚स्व‚भाव‚निय‚तो भावः}
	\end{center}
	

	  \pstart \leavevmode% starting standard par
	स्यादेत‚त् । भाव‚हेतुरेवानित्य‚त्वाख्यं ध‚र्मं भाव‚नाश‚कं ज‚न‚य‚ति तेन न‚श्व‚र- स्व‚भाव‚निय‚तो भाव इत्याह ।
	\pend% ending standard par
      
	  \bigskip
	  \begingroup
	
	    \large
	  
	    \begin{quote}
	  
	    
	    \stanza[\smallbreak]
	\label{pv.4.280b}\flagstanza{\tiny\textenglish{...4.280b}}भाव‚हेतुभ‚व‚त्वे किं पार‚म्प‚र्य‚प‚रिश्र‚मैः ॥ २८० ॥\&[\smallbreak]


	
	    \end{quote}
	  
	  \endgroup
	

	  \pstart \leavevmode% starting standard par
	\hphantom{.}‚{\color{DodgerBlue3}‚भाव‚हेतुभ‚व‚त्वे}‚ऽनित्य‚त्वाख्य‚स्य ध‚र्म‚स्य भाव‚नाश‚क‚स्येष्य‚माणे ‚{\color{DodgerBlue3}‚पार‚म्प‚र्ये प‚रिश्र‚मै}‚रेभिः स्वीकृतैः ‚{\color{DodgerBlue3}‚किं} प्र‚योज‚नं । (२८०)
	\pend% ending standard par
      \label{div_pvv.4.281}
	  
	% new div opening: depth here is 2
	

	  \pstart \leavevmode% starting standard par
	त‚था ‚{\tiny $_{5}$}‚ हि (।)
	\pend% ending standard par
      
	  \bigskip
	  \begingroup
	
	    \large
	  
	    \begin{quote}
	  
	    
	    \stanza[\smallbreak]
	\label{pv.4.281}\flagstanza{\tiny\textenglish{....4.281}}नाश‚नं ज‚न‚यित्वान्यं स हेतुस्त‚स्य नाश‚कः&त‚मेव न‚श्व‚रं भावं ज‚न‚येद् य‚दि किम्भ‚वेत् ॥ २८१ ॥\&[\smallbreak]


	
	    \end{quote}
	  
	  \endgroup
	

	  \pstart \leavevmode% starting standard par
	\hphantom{.}‚{\color{DodgerBlue3}‚स} भाव‚{\color{DodgerBlue3}‚हेतुर‚न्य‚म}‚नित्य‚त्वाख्यं ध‚र्मं ‚{\color{DodgerBlue3}‚भाव‚नाश‚नं ज‚न‚यित्वा त‚स्या}‚भाव‚स्य ‚{\color{DodgerBlue3}‚नाश‚क} इष्य‚ते प‚रंप‚र‚या । ‚{\color{DodgerBlue3}‚य‚दि} तु ‚{\color{DodgerBlue3}‚त‚मेव} न‚{\color{DodgerBlue3}‚श्व‚रं भावं} साक्षाद् भाव‚हेतु‚{\color{DodgerBlue3}‚र्ज‚न‚येत् । त‚दा- किं}‚न्दूष‚णं ‚{\color{DodgerBlue3}‚भ‚वेत्} । न किञ्चित् । (२८१)
	\pend% ending standard par
      \label{div_pvv.4.282}
	  
	% new div opening: depth here is 2
	

	  \pstart \leavevmode% starting standard par
	अपि च (।)
	\pend% ending standard par
      
	  \bigskip
	  \begingroup
	
	    \large
	  
	    \begin{quote}
	  
	    
	    \stanza[\smallbreak]
	\label{pv.4.282}\flagstanza{\tiny\textenglish{....4.282}}आत्मोप‚कार‚कः कः स्यात् त‚स्य सिद्धात्म‚नः स‚तः ।&नात्मोप‚कार‚कः कः स्यात् तेन यः स‚म‚पेक्ष्य‚ते ॥ २८२ ॥\&[\smallbreak]


	
	    \end{quote}
	  
	  \endgroup
	

	  \pstart \leavevmode% starting standard par
	\hphantom{.}यो यावान‚नित्य‚त्वाख्यो ध‚र्म ‚{\color{DodgerBlue3}‚आत्मा} भाव‚स्य स किमु‚{\color{DodgerBlue3}‚प‚कार‚कः} स‚न् ‚{\color{DodgerBlue3}‚विनाश‚को} भाव‚स्य उतानुप‚कार‚क एव । त‚त्राद्ये प‚क्षे ‚{\color{DodgerBlue3}‚सिद्धात्म‚नो} भाव‚स्य ‚{\color{DodgerBlue3}‚स‚तः} स‚र्व्व‚तो निरा- शंस्य‚{\tiny $_{6}$}‚स्य ‚{\color{DodgerBlue3}‚क आत्मा}‚ऽनित्याख्यो ध‚र्मोऽन्यो वा ‚{\color{DodgerBlue3}‚उप‚कार‚कः स्यात्} । अथ द्वितीयः प‚क्षः (।) त‚दाऽ‚{\color{DodgerBlue3}‚नात्मा उप‚कार‚कः} स्यात् । ‚{\color{DodgerBlue3}‚तेन} भावेन ‚{\color{DodgerBlue3}‚यो} नाश‚क‚त्वेन ‚{\color{DodgerBlue3}‚स‚म‚पेक्ष्येत} । उप‚कार‚क एव ह्य‚पेक्ष्य‚ते अनुप‚कार‚क‚त्वे का त‚स्यापेक्षा नाश‚क‚ता वा ।\edtext{\textsuperscript{*}}{\edlabel{pvv.511-1}\label{pvv.511-1}\lemma{*}\Bfootnote{व्य‚र्था नाश‚हेतुक‚ल्प‚ना}}(२८२)
	\pend% ending standard par
      \label{div_pvv.4.283}
	  
	% new div opening: depth here is 2
	\textsuperscript{\textenglish{512/s}}

	  \begin{center}%% label @type='head'
	\textbf{(३) अन‚पेक्ष्य एव भावो न‚श्व‚र‚त्वे}
	\end{center}
	

	  \pstart \leavevmode% starting standard par
	अथ नाश‚हेत्व‚योगात् । अन‚पेक्ष्य एव भावो न‚श्व‚र‚तायां, त‚दा (।)
	\pend% ending standard par
      
	  \bigskip
	  \begingroup
	
	    \large
	  
	    \begin{quote}
	  
	    
	    \stanza[\smallbreak]
	\label{pv.4.283}\flagstanza{\tiny\textenglish{....4.283}}अन‚पेक्ष‚श्च किम्भावोऽत‚थाभूतः क‚दाच‚न ।&य‚था न क्षेप‚भागिष्टः स एवोद्भूत‚नाश‚कः ॥ २८३ ॥\&[\smallbreak]


	
	    \end{quote}
	  
	  \endgroup
	

	  \pstart \leavevmode% starting standard par
	\leavevmode\ledsidenote{\textenglish{105b/MA}}‚{\color{DodgerBlue3}‚अन‚पेक्ष‚श्च} नाशे ‚{\color{DodgerBlue3}‚भावः किं} क‚स्मात् ‚{\color{DodgerBlue3}‚क‚दाच‚नात‚थाभूतो} ‚{\tiny $_{7}$}‚ अन‚श्व‚र‚स्व‚भावः । स‚र्व‚दैव न‚श्व‚र‚स्व‚भाव‚ताऽस्य युक्ता । ‚{\color{DodgerBlue3}‚य‚था} त्व‚न्म‚ते ‚{\color{DodgerBlue3}‚स एव} कृत‚को भाव ‚{\color{DodgerBlue3}‚उद्भूत- नाश‚क} उन्मुख‚नाश‚कानित्य‚त्व‚स्व‚भावो नाश‚कालेऽ‚{\color{DodgerBlue3}‚क्षेप}‚भाग‚चिर‚विनाशी इष्टः । त‚थोत्पादान‚न्त‚रं न‚श्व‚र‚स्व‚भाव‚त‚या विन‚श्येदिति । (२८३)
	\pend% ending standard par
      \label{div_pvv.4.284}
	  
	% new div opening: depth here is 2
	
	  \bigskip
	  \begingroup
	
	    \large
	  
	    \begin{quote}
	  
	    
	    \stanza[\smallbreak]
	\label{pv.4.284}\flagstanza{\tiny\textenglish{....4.284}}क्ष‚ण‚म‚प्य‚न‚पेक्ष‚त्वे भावो भाव‚स्य नेति चेत् ।&भावो हि स त‚था भूतोऽभावे भाव‚स्त‚था क‚थ‚म् ॥ २८४ ॥\&[\smallbreak]


	
	    \end{quote}
	  
	  \endgroup
	

	  \pstart \leavevmode% starting standard par
	\hphantom{.}क्ष‚ण‚क्ष‚यिस्व‚भावा भावाः स्व‚हेतोरेव जाय‚न्ते । विनाशं प्र‚त्य‚न‚{\color{DodgerBlue3}‚पेक्ष‚त्वे भाव‚स्य} य‚था द्वितीये क्ष‚णे ‚{\color{DodgerBlue3}‚भावो ना}‚स्ति त‚था प्र‚थ‚मे‚{\tiny $_{1}$}‚पि क्ष‚णे न स्यादिति । क्ष‚ण‚म‚पि भाव‚स्य भावो न स्यादिति चेत् । अयुक्त‚मेत‚त् हि य‚स्मात् ‚{\color{DodgerBlue3}‚स भाव}‚स्त‚{\color{DodgerBlue3}‚था} न‚श्व‚र‚स्व‚भाव इष्य‚ते । य‚दा तु भाव एव नास्ति त‚दाऽ‚{\color{DodgerBlue3}‚भावे} भाव‚स्य ‚{\color{DodgerBlue3}‚भाव‚स्त‚था} न‚श्व‚रः ‚{\color{DodgerBlue3}‚क‚थ}‚मुच्य‚ते (।) त‚तो ल‚ब्ध‚ज‚न्म‚तो भाव‚स्य क्ष‚णान्त‚रा- न‚नुवृत्तेर्न‚श्व‚र‚ता ॥ (२८३) ।
	\pend% ending standard par
      

	  \pstart \leavevmode% starting standard par
	त‚स्मात् ।
	\pend% ending standard par
      
	  \bigskip
	  \begingroup
	
	    \large
	  
	    \begin{quote}
	  
	    
	    \stanza[\smallbreak]
	\label{pv.4.285}\flagstanza{\tiny\textenglish{....4.285}}येऽप‚रापेक्ष‚त‚द्भावास्त‚द्भाव‚निय‚ता हि ते ।&अस‚म्भ‚व‚द्विब‚न्धा च साम‚ग्री कार्य‚क‚र्म‚णि ॥ २८५ ॥\&[\smallbreak]


	
	    \end{quote}
	  
	  \endgroup
	

	  \pstart \leavevmode% starting standard par
	\hphantom{.}‚{\color{DodgerBlue3}‚ये} भावा अन‚पेक्ष‚{\color{DodgerBlue3}‚त‚द्भावाः प‚रापेक्षां} विना स‚म्भ‚व‚द्ध‚र्म‚विशेष‚संभ‚वाः । ‚{\color{DodgerBlue3}‚ते त‚स्य भावे} ध‚र्म‚स्य ‚{\color{DodgerBlue3}‚निय‚ताः}‚{\tiny $_{2}$}‚ । कार‚ण‚साम‚ग्रीव ‚{\color{DodgerBlue3}‚अस‚म्भ‚व‚द्विब‚न्धा} विब‚न्ध‚कार‚ण- र‚हिता\edtext{}{\edlabel{pvv.512-1}\label{pvv.512-1}\lemma{हिता}\Bfootnote{अन्त्या साम‚ग्री}} ‚{\color{DodgerBlue3}‚कार्य‚स्य क‚र्म‚णि} क्रियायां निय‚ता\edtext{}{\edlabel{pvv.512-2}\label{pvv.512-2}\lemma{ता}\Bfootnote{दृष्टान्तः}} । प‚र‚निर‚पेक्षाश्च भावाः स्व‚नाश इति । क्ष‚णिकाः स‚र्व्व‚संस्कारा इत्य‚क‚म्प्यः सौ ग तः सिंह‚नादः ॥
	\pend% ending standard par
      

	  \pstart \leavevmode% starting standard par
	न य‚दिह त‚न्न न्याय्यं तेनोदितेन च किं फ‚लं । य‚दिं ब‚हुश‚स्त‚स्या वृत्तौ गुणः क‚थं क‚स्य कः ।
	\pend% ending standard par
      \textsuperscript{\textenglish{513/s}}

	  \pstart \leavevmode% starting standard par
	य‚दि प‚र‚म‚सौ व्याख्येयार्थ‚ग्र‚ह‚स्य विरोधिनी (।) विवृत्तिर‚च‚नामात्रे‚{\tiny $_{3}$}‚ त‚स्मात् कृतोत्र भ‚याद‚रः ॥
	\pend% ending standard par
      
	    
	    \stanza[\smallbreak]
	आचार्य‚श्रीम‚नोर‚थ‚न‚न्दिकृतायां वार्त्तिक‚वृत्तौ च‚तुर्थः प‚रिच्छेदः स‚माप्तः ॥\&[\smallbreak]


	

	  \pstart \leavevmode% starting standard par
	लिखितेयं पंडि (त) वि भू ति च न्द्रे ण य‚द‚त्र पुण्यं त‚द्भ‚व‚त्वाचार्योपाध्याय- पूर्व्व‚ङ्ग‚म‚स‚क‚ल‚स‚त्व‚राशेर‚नुत्त‚र‚ज्ञान‚फ‚लावाप्त‚य इति ।
	\pend% ending standard par
      \footnote{\label{pvv.513-1a}  1a साध्य‚हेतुदृष्टान्तोप‚न‚य‚निग‚म‚नानि प‚ञ्च । \begin{english} --- Placement of note uncertain; marked with a question mark in the edition (see encoding description for details)\end{english}}\footnote{\label{pvv.513-1b}  1b पा ... क‚स्य इत्यादौ अव‚य‚व्येक‚त्वं हेतुः । तेन स‚र्व‚क‚स्य प्र‚संगः साध्यः । विप‚र्य‚ये स‚र्व‚क‚स्य साध्य‚स्याभावः साध‚नं । तेनाव‚य‚व्येक‚त्व‚स्य हेतोर‚भावः साध्यः । (एवं व‚क्ष‚माणेत्यादि) । अव‚य‚व्येक‚रूप(हेतु)त्वे एक‚स्यावृत्तौ स‚र्व‚स्यावृत्तिः (साध्य‚प्र‚संगः) स्यात् । विप‚र्य‚ये स‚र्वावृत्ते साध्य‚स्याभावः साध‚नं । तेनाव‚य‚व्येक‚त्व‚स्य हेतोर‚भावः साध्यः । अव‚य‚व्येक‚रूप‚हेतुत्वे एक‚स्य र‚क्त‚त्वे स‚र्व‚स्य (प्र‚संगं साध्यं) र‚क्त‚त्वं स्यात् । विप‚र्य‚ये स‚र्व‚र‚क्त‚त्व‚स्य साध्य‚स्याभावः साध‚नं । तेनाव‚य‚व्येक‚त्व‚स्य हेतोर‚भावः साध्यः । \begin{english} --- Placement of note uncertain; marked with a question mark in the edition (see encoding description for details)\end{english}}
	    
	    \endnumbering% ending numbering from div
	    \endgroup
	    
	  % running endDocumentHook
     \backmatter 
	 \chapter{The TEI Header}
	 \begin{minted}[fontfamily=rmfamily,fontsize=\footnotesize,breaklines=true]{xml}
       <teiHeader xmlns="http://www.tei-c.org/ns/1.0" xml:lang="en">
   <fileDesc>
      <titleStmt>
         <title type="main" subtype="base-text">Pramāṇavārttika</title>
         <title type="sub" subtype="commentary">Pramāṇavārttikavṛtti</title>
         <author role="base-author">Dharmakīrti</author>
         <author role="commentator">Manorathanandin</author>
         <funder>Deutsche Forschungsgemeinschaft</funder>
         <funder>The National Endowment for the Humanities</funder>
         <principal>
	           <persName>Birgit Kellner</persName>
	        </principal>
         <respStmt>
            <resp>data entry by</resp>
            <name key="name_swift">SWIFT Information Technologies, Mumbai</name>
         </respStmt>
         <respStmt>
            <resp>prepared for SARIT by</resp>
            <persName key="name_person_lo">Liudmila Olalde</persName>
         </respStmt>
      </titleStmt>
      <editionStmt>
         <p> </p>
      </editionStmt>
      <publicationStmt>
         <publisher>SARIT: Search and Retrieval of Indic Texts. DFG/NEH Project (NEH-No. HG5004113), 2013-2016 </publisher>
         <idno>2014-04-15</idno>
         <availability status="restricted">
            <p>Copyright Notice:</p>
            <p>Copyright 2014-2016 SARIT</p>
            <licence> 
	              <p>Distributed under a <ref target="https://creativecommons.org/licenses/by-sa/4.0/">Creative Commons Attribution-ShareAlike 4.0 International licence.</ref> Under this licence, you are free to:</p>
	              <list>
                  <item>Share — copy and redistribute the material in any medium or format.</item>
                  <item>Adapt — remix, transform, and build upon the material for any purpose, even commercially.</item>
               </list>
	              <p>The licensor cannot revoke these freedoms as long as you follow the license terms.</p>
	              <p>Under the following terms:</p>
	              <list>
                  <item>Attribution — You must give appropriate credit, provide a link to the license, and indicate if changes were made. You may do so in any reasonable manner, but not in any way that suggests the licensor endorses you or your use.</item>
                  <item>ShareAlike — If you remix, transform, or build upon the material, you must distribute your contributions under the same license as the original.</item>
               </list>
	              <p>More information and fuller details of this license are given on the Creative Commons website.</p>
	           </licence>
            <p>SARIT assumes no responsibility for unauthorised use that infringes the rights of any copyright owners, known or unknown.</p>
         </availability>
         <date>2014</date>
      </publicationStmt>
      <sourceDesc>
         <bibl xml:id="pvv-sankrtyayana-book">
	           <title type="main">
               <persName>Dharmakīrti</persName>'s Pramāṇavārttika</title>
	           <title type="sub">with a commentary by <persName>Manorathanandin</persName>
            </title>
	           <author>Dharmakīrti</author>
	           <author>Manorathanandin</author>
	           <editor key="name_person_rs">Rāhula Sāṅkṛtyāyana</editor>
	           <publisher>Bihar and Orissa Research Society</publisher>
	           <pubPlace>Patna</pubPlace>
	           <date>1938-1940</date>
	           <note>Appendix to the Journal of the Bihar and Orissa Research Society</note>
	           <note>The manuscript consulted by Sāṅkṛtyāyana is described below.</note>
	           <note>The copy that was used for digitizing this text is the one that belonged to <persName xml:id="frw">Erich Frauwallner</persName> and is now in the ``Collection Frauwallner" in the <ref target="http://bibliothek.univie.ac.at/fb-suedasien_tibet_buddhismuskunde/south_asian_tibetan_and_buddhist_studies_library.html">
South Asian, Tibetan and Buddhist Studies Library</ref> with signature S-28/V/21b (or S-0132-B).</note>
	        </bibl>
         <msDesc>
            <msIdentifier>
               <idno/>
               <altIdentifier>
                  <idno>Manuscript nr. 237 (henceforth MA).</idno>
                  <note>In: Sāṅkṛtyāyana, "Second Search of Sanskrit Palm-leaf Mss. in Tibet". JBORS 23,1 (1937) 1-57.</note>
               </altIdentifier>
            </msIdentifier>
            <msContents>
               <msItem>
                  <author>Manorathanandin</author>
                  <title>Pramāṇavārttikavṛtti</title>
               </msItem>
            </msContents>
            <physDesc>
               <objectDesc>
                  <p>Written by Vibhūticandra in early old Bengali script (Sāṅkṛtyāyana refers to the script as Kuṭilā), the manuscript comprises 105 leaves of seven lines each that according to Sāṅkṛtyāyana measure 67.31 x 5.80 cm.</p>
               </objectDesc>
            </physDesc>
            <history>
               <p>On July 28, 1936, Sāṅkṛtyāyana found this paper manuscript of Manorathanandin’s Pramāṇavārttikavṛtti in the hermitage Zha lu ri phug.</p>
            </history>
         </msDesc>
      </sourceDesc>
   </fileDesc>
   <encodingDesc>
      <p>In the source file, there were two types of line breaks: returns (and possible surrounding space) and hyphens+returns. These were replaced with lb-elements. I didn't check whether the source was consequent in this respect. The ed-attribute "s" refers to Sāṅkṛtyāyana's edition<ref sameAs="#pvv-sankrtyayana-book"/>.</p>
      <p>The folio numbers on the margins were encoded as pb-elements. The ed-attribute "MA" refers to the manuscript used by Sāṅkṛtyāyana. The line numbers in the manuscript were encoded as lb-elements with the ed-attribute "MA".</p>
      <p>The text is structured in three div-levels:<list>
            <item>Four chapters encoded as: div n="..." type="chapter" subtype="pariccheda"</item>
            <item>Subchapters encoded as: div n="[roman numbers]" type="subchapter". The subchapter number is not reflected in the verse numbers.</item>
            <item>The lowest div-level encloses a verse (or a group of verses) and its corresponding commentary, e.g.: div n="3.121 3.122abc" for a div enclosing verses 121 and the first three padas of 122 of the first pariccheda as well their commentary.</item>
         </list>
      </p>
      <p>The notes represent marginal notes on the manuscript MA, written by Vibhūticandra. The editor Rāhula Sāṅkṛtyāyana is responsible for linking them with particular passages in the text. His linkages await further study. In some places his notes are prefixed with question marks, which we interpret as indicating Sāṅkṛtyāyana's uncertainty regarding where they belong. We placed them right after the immediately preceding note in the text, as a convention.</p>
      <p>The verse numbers are those of Sāṅkṛtyāyana's edition<ref sameAs="#pvv-sankrtyayana-book"/>.</p>
      <p>Abbreviations used in the attributes ed, cRef and xml:id's in this file: <!-- this is a provisory list and has to be replaced by a refsDecl -->
      <list ana="abbreviations">
            <item>Divy = Divyāvadāna; the page number refers to P.L. Vaidya, Divyāvadāna, The Mithila Institute of Post-Graduate Studies and Research in Sanskrit Learning, Darbhanga 1959 (Buddhist Sanskrit Texts, 20)</item>
            <item>MA = Sanskrit manuscript of Manorathanandin's Pramāṇavārttikavṛtti used by Sāṅkṛtyāyana</item>
            <item>ps = Pramāṇasamuccaya; the verse numbers correspond to <ref target="http://www.ikga.oeaw.ac.at/Mat/dignaga_PS_1.pdf">Steinkellner, Dignāga's Pramāṇasamuccaya, Chapter 1</ref>
            </item>
            <item>psv = Pramāṇasamuccayavṛtti; the verse numbers correspond to to <ref target="http://www.ikga.oeaw.ac.at/Mat/dignaga_PS_1.pdf">Steinkellner, Dignāga's Pramāṇasamuccaya, Chapter 1</ref>
            </item>
            <item>pv = Dharmakīrti's Pramāṇavārttika</item>
            <item>pvv = Manorathanandin's Pramāṇavārttikavṛtti</item>
            <item>nsū = Gautama's Nyāyasūtra</item>
            <item>VāPa = Bhartṛhari's Vākyapadīya</item>
            <item>śv = Kumārila's Ślokavārttika</item>
            <item>pvsv = Dharmakīrti's Pramāṇavārttikasvavṛtti</item>
            <item>ts = Tattvasaṅgraha; the verse numbers correspond to Krishnamacharya's edition of Tattvasaṅgrahapañjikā</item>
         </list>
      </p>
   </encodingDesc>
   <profileDesc><!-- ... --></profileDesc>
   <revisionDesc>
      <change who="#lo" when="2014-05-18">I corrected the verse number of verse 2.450.</change>
      <change who="#lo" when="2014-06-04">I corrected the verse number of verse 4.116.</change>
      <change who="#lo" when="2014-06-04">I corrected the folio number 65b, which was wrong in the printed edition.</change>
      <change who="#lo" when="2014-06-30">I changed फ&#130;ल&#130;भि to फ&#130;ल&#130;मि on p. 214, line 15 from below.</change>
      <change who="#lo" when="2014-07-23">I corrected the verse number (३।१६३) in footnote 345-12. In the printed edition it reads (३।१६५).</change>
      <change who="#lo" when="2014-07-23">In footnote 370-3 I changed प्र&#130;व&#130;र्त्त&#130;त to प्र&#130;व&#130;र्त्तेत.</change>
      <change who="#lo" when="2015-08-18">Added manually an index to the Pramāṇavārttika.</change>
      <change who="#lo" when="2015-12-30">Added @xml:lang to the front-element.</change>
      <change who="#lo" when="2016-03-08">Deleted the Index to the Pramāṇavārttika.</change>
   </revisionDesc>
</teiHeader>
	 \end{minted}
       
      \clearpage
      \begin{english}
      \printshorthands
      \printbibliography
      \end{english}
    
\end{document}
