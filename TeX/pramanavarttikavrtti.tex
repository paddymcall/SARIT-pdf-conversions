\documentclass[article,12pt,a4paper]{memoir}
  \usepackage{euler}
  \usepackage{xltxtra}
  \usepackage{polyglossia}
  \PolyglossiaSetup{sanskrit}{
  hyphenmins={2,3},% default is {1,3}
  }
  \setdefaultlanguage{sanskrit}
  % english should be available, notes and stuff
  \setotherlanguage{english}
  \setotherlanguage[numerals=arabic]{tibetan}
  \usepackage{fontspec}
  \usepackage{xunicode}
  \catcode`⃥=\active \def⃥{\textbackslash}
  \catcode`❴=\active \def❴{\{}
  \catcode`〔=\active \def〔{{[}}% translate 〔OPENING TORTOISE SHELL BRACKET
  \catcode`〕=\active \def〕{{]}}% translate 〕CLOSING TORTOISE SHELL BRACKET
  \catcode`❴=\active \def❴{\{}
  \catcode`❵=\active \def❵{\}}
  \catcode` =\active \def {\,}
  %% show a lot of tolerance
  \tolerance=9000
  \def\textJapanese{\fontspec{Kochi Mincho}}
  \def\textChinese{\fontspec{HAN NOM A}}
  \def\textKorean{\fontspec{Baekmuk Gulim} }
  % make sure English font is there
  \newfontfamily\englishfont[Mapping=tex-text]{TeX Gyre Schola}
    % set up a devanagari font
  \newfontfamily\devanagarifont[Script=Devanagari,Mapping=devanagarinumerals]{Chandas}
	\newfontfamily\rmlatinfont[Mapping=tex-text]{TeX Gyre Pagella}
	\newfontfamily\tibetanfont[Script=Tibetan,Scale=1.2]{Tibetan Machine Uni}
  \newcommand\bo\tibetanfont
  
    \defaultfontfeatures{Scale=MatchLowercase,Mapping=tex-text}
	\setmainfont{Chandas}
    \setsansfont{TeX Gyre Bonum}
  
  \setmonofont{DejaVu Sans Mono}
	    % numbering depth
	    \maxtocdepth{section}
	    \setsecnumdepth{all}
	    \newenvironment{docImprint}{\vskip 6pt}{\ifvmode\par\fi }
	    \newenvironment{docDate}{}{\ifvmode\par\fi }
	    \newenvironment{docAuthor}{\ifvmode\vskip4pt\fontsize{16pt}{18pt}\selectfont\fi\itshape}{\ifvmode\par\fi }
	    % \newenvironment{docTitle}{\vskip6pt\bfseries\fontsize{18pt}{22pt}\selectfont}{\par }
	    \newcommand{\docTitle}[1]{#1}
	    \newenvironment{titlePart}{ }{ }
	    \newenvironment{byline}{\vskip6pt\itshape\fontsize{16pt}{18pt}\selectfont}{\par }
	    % setup title page; see CTAN /info/latex-samples/TitlePages/, and memoir
	  \newcommand*{\plogo}{\fbox{$\mathcal{SARIT}$}}
	  \newcommand*{\makeCustomTitle}{\begin{english}\begingroup% from example titleTH, T&H Typography
	  \thispagestyle{empty}
	  \raggedleft
	  \vspace*{\baselineskip}
	  
	      % author(s)
	    {\Large Dharmakīrti and Manorathanandin}\\[0.167\textheight]
	    % maintitle
	    {\Huge Pramāṇavārttika}\\[\baselineskip]
	    % titlesubtitle
	    {\small  — Pramāṇavārttikavṛtti}\\[\baselineskip]
	    {\Large SARIT}\\\vspace*{\baselineskip}\plogo\par
	  \vspace*{3\baselineskip}
	  \endgroup
	  \end{english}}
	  \newcommand{\gap}[1]{}
	  \newcommand{\corr}[1]{($^{x}$#1)}
	  \newcommand{\sic}[1]{($^{!}$#1)}
	  \newcommand{\reg}[1]{#1}
	  \newcommand{\orig}[1]{#1}
	  \newcommand{\abbr}[1]{#1}
	  \newcommand{\expan}[1]{#1}
	  \newcommand{\unclear}[1]{($^{?}$#1)}
	  \newcommand{\add}[1]{($^{+}$#1)}
	  \newcommand{\deletion}[1]{($^{-}$#1)}
	  \newcommand{\pratIka}[1]{\textcolor{cyan}{#1}}
	  \newcommand{\name}[1]{#1}
	  \newcommand{\persName}[1]{#1}
	  \newcommand{\placeName}[1]{#1}
	  % running latexPackages template
     \usepackage[x11names]{xcolor}
     \definecolor{shadecolor}{gray}{0.95}
     \usepackage{longtable}
     \usepackage{ctable}
     \usepackage{rotating}
     \usepackage{lscape}
     \usepackage{ragged2e}
     
	 \usepackage{titling}
	 \usepackage{marginnote}
	 \renewcommand*{\marginfont}{\color{black}\rmlatinfont\scriptsize}
	 \setlength\marginparwidth{.75in}
	 \usepackage{graphicx}
	 \usepackage{csquotes}
       
	 \def\Gin@extensions{.pdf,.png,.jpg,.mps,.tif}
       %% biblatex stuff start
	 \usepackage[backend=biber,citestyle=authoryear,bibstyle=authoryear]{biblatex}
	 
		 \addbibresource[location=remote]{https://raw.githubusercontent.com/paddymcall/Stylesheets/HEAD/profiles/sarit/latex/bib/sarit.bib}
	 \renewcommand*{\citesetup}{%
	 \rmlatinfont
	 \biburlsetup
	 \frenchspacing}
	 \renewcommand{\bibfont}{\rmlatinfont}
	 \DeclareFieldFormat{postnote}{:#1}
	 \renewcommand{\postnotedelim}{}
	 %% biblatex stuff end
	 
	 \setcounter{errorcontextlines}{400}
       
	 \usepackage{lscape}
	 \usepackage{minted}
       
	   % pagestyles
	   \pagestyle{ruled}
	   
	   \makeoddfoot{ruled}{{\tiny\rmlatinfont \textit{Compiled: \today}}}{}{\rmlatinfont\thepage}
	   \makeevenfoot{ruled}{\rmlatinfont\thepage}{}{{\tiny\rmlatinfont \textit{Compiled: \today}}}
	   
	 
	 \usepackage[noend,series={A}]{reledmac}
	 
       % simplify what ledmac does with fonts, because it breaks. From the documentation of ledmac:
       % The notes are actually given seven parameters: the page, line, and sub-line num-
       % ber for the start of the lemma; the same three numbers for the end of the lemma;
       % and the font specifier for the lemma. 
       \makeatletter
       \def\select@lemmafont#1|#2|#3|#4|#5|#6|#7|%
       {}
       \makeatother
       \setlength{\stanzaindentbase}{20pt}
     \setstanzaindents{8,2,2,2,2,2,2,2,2,2,2,}
     % \setstanzapenalties{1,5000,10500}
     \lineation{page}
     % \linenummargin{inner}
     \linenumberstyle{arabic}
     \firstlinenum{5}
    \linenumincrement{5}
    \addtolength{\skip\Afootins}{1.5mm}
    \Xnotenumfont{\bfseries\footnotesize}
    \sidenotemargin{outer}
    \linenummargin{inner}
       
       \usepackage[destlabel=true,% use labels as destination names; ; see dvipdfmx.cfg, option 0x0010, if using xelatex
       pdftitle={Dharmakīrti's Pramāṇavārttika with a commentary by Manorathanandin},
       pdfauthor={SARIT: Search and Retrieval of Indic Texts. DFG/NEH Project (NEH-No. HG5004113), 2013-2016 }]{hyperref}
       \hyperbaseurl{}
       \usepackage[english]{cleveref}% clashes with eledmac < 1.10.1!
       % \newcommand{\cref}{\href}
     
\begin{document}
    
     \makeCustomTitle
     \let\tabcellsep&
	\frontmatter
	\tableofcontents
	% \listoffigures
	% \listoftables
	\cleardoublepage
         \begin{english}
      \chapter[Title Page]{Title Page}
    \begin{docTitle}  \begin{titlePart} Dharmakīrti's Pramāṇavārttika with a commentary by Manorathanandin\end{titlePart} \end{docTitle} \begin{docAuthor} Dharmakīrti\end{docAuthor} \begin{docAuthor} Manorathanandin\end{docAuthor}
      \cleardoublepage
    \begin{sanskrit}\par
नमो मञ्जुश्रिये ॥\end{sanskrit}\end{english}\mainmatter 
	  
	% new div opening: depth here is 0
	
	    
	    \begingroup
	    \beginnumbering% beginning numbering from div depth=0
	    
	  
\chapter[{प्रथमः परिच्छेदः ॥: प्रमाणसिद्धिः}]{प्रथमः परिच्छेदः ॥: प्रमाणसिद्धिः}\leavevmode\marginnote{\textenglish{1b/MA}}
	  
	% new div opening: depth here is 1
	
	  
	% new div opening: depth here is 2
	
	    
	    \stanza[\smallbreak]
	विमुक्तावरणक्लेशं दीप्ताखिलगुणश्रियं ।&स्वैकवेद्यात्मसम्पत्तिं नमस्यामि महामुनिम् ॥\&[\smallbreak]


	
	    
	    \stanza[\smallbreak]
	स्वयमपि कृतिनां महद्भिरन्यैरपि गमितो बहुविस्तरैर्न्न योयम् ।&तदपि च सुगमो न मद्विधानामिति विवृतिच्छलतः करोमि चिन्ताम् ॥\&[\smallbreak]


	
	    
	    \stanza[\smallbreak]
	अहमपि न निजैकलाभलुब्धो न च परकृत्यरसाभिलाषमुक्तः ।&फलति पुनरियं परार्थवाञ्छाव्रततिरभीष्टफलानि पुण्यभाजाम् ॥\&[\smallbreak]


	\label{div_pvv.1.1}\edlabel{div_pvv.1.1}
	  
	% new div opening: depth here is 2
	

	  \begin{center}%% label @type='head'
	\textbf{क. नमस्कारश्लोकः}
	\end{center}
	

	  \pstart शास्त्रा\edlabel{pvv.1-1}\footnote{\label{pvv.1-1}  १ स्तुत्या पुण्योपचयात् ।}दावविध्नेन तत्समाप्त्यर्थं भगवति प्रसादजनने श्रोतृ\edlabel{pvv.1-2}\footnote{\label{pvv.1-2}  २ व्याख्यातुश्चेति परार्थदर्शनात् ।}जनानुग्रहार्थञ्च स्तुतिपूर्वकमाचार्यो नमस्कारश्लोकमाह ।
	\pend
      
	  \bigskip
	  \begingroup
	  \large
	
	    
	    \stanza[\smallbreak]
	विधूतकल्पनाजालगम्भीरोदारमूर्तये ।&नमः समन्तभद्राय समन्तस्फरणत्विषे ॥ १ ॥\&[\smallbreak]


	
	  \endgroup
	

	  \pstart विधूतं\edlabel{pvv.1-3}\footnote{\label{pvv.1-3}  ३ सर्व्वावरणविगमात् ।} विध्वस्तं अनुत्पत्तिकधर्मतामापादितं {\color{DodgerBlue3}“कल्पना”} ग्राह्यग्राहकाध्यारोपः सैव {\color{DodgerBlue3}“जालं”} बन्धनहेतुत्वात् यासां ता विधूतकल्पनाजालाः । एतेन धर्मकाय उक्तः\edlabel{pvv.1-4}\footnote{\label{pvv.1-4}  ४ पूजेयं नमः शब्दात्प्रणामतः शिष्टैश्च तेश्च । तच्च स्वपरार्थतदुभयसम्पत्तिस्ततः अवृत्तिहानिगाम्भीर्य्यौदार्यविशेषैस्त्रिभिः स्वार्थ उक्तः ।} । द्वयशून्यताया धर्मधातुत्वात् (।) तदधिगमस्य धर्मकायत्वात् । गम्भीराश्च खड्ग\leavevmode\marginnote{\textenglish{002/s}} श्रावकाद्यविषयत्वात् । उदाराश्च सकलज्ञेयसत्वार्थव्यापनादिति {\color{DodgerBlue3}“गम्भीरोदाराः”} । आभ्यां साम्भोगिकनैर्म्माणिककायावुक्तौ तयोरेव स्वरूपत्वात् । विधूतकल्पनाजाला गम्भीरोदारा {\color{DodgerBlue3}“मूर्त्तयो”} यस्य स {\color{DodgerBlue3}“विधूतकल्पनाजालगम्भोरोदारमर्त्तिः”} (।) एतेन स्वार्थसम्पदुक्ता त्रिकायलक्षणत्वात्तस्याः ।
	\pend
      

	  \pstart {\color{DodgerBlue3}“समन्तं”} निरवशेषं भद्रं कल्याणं परार्थसम्पत्सम्भारलक्षणं यस्मादसौ {\color{DodgerBlue3}“समन्तभद्रः”} (।) अनया भगवन्नामव्युत्पत्त्या परार्थसम्पदभिहिता\edlabel{pvv.2-1}\footnote{\label{pvv.2-1}  १ तदर्थिनां यथा ।} । समन्ततः स्फरन्तीति समन्तस्फरण्यः त्विषः ताव (त्) त्विषो देशना यस्य स {\color{DodgerBlue3}“समन्तस्फरणत्विट्”} वस्तुतत्त्वावभासनोपायता च त्विड्देशनयोः साधर्म्यं (।) अनेन परार्थसम्पदुपायो दर्शितः । देशनाद्वारेण भगवता जगदर्थकरणात् । एतेन स्तुतिरुक्ता असाधारणानां स्वपरार्थसम्पत्तितदुपायानामुपदर्शनात् । सर्व्वत्र नमःश\edlabel{pvv.2-2}\footnote{\label{pvv.2-2}  २ अकारान्तः सकारान्तस्तृतीयार्थ इत्यन्ये । जनकायः ।}ब्दयोगाच्चतुर्थी । अनेन नमस्कारोभिहितः । यदा तु समन्तभद्रशब्दो रूढ्या बोधिसत्त्वविशेषे वर्तते तदापि पदव्याख्यानं पूर्ववदेव । अयन्तु विशेषः । विधूतकल्पनाजालत्वं बोधिसत्त्वभूम्यावरणप्रहाणतो वेदितव्यं । गाम्भीर्यन्तु खड्गश्रावकाविषयत्वात् । औदार्यन्तु दर्श भूमीश्वरबोधिसत्त्वमाहात्म्यातिशयतः । कायत्रयन्तु बोधिसत्त्वानामप्यस्ति प्रकर्षनिष्ठागमनात्तु भगवतां व्यवस्थाप्यते । देशना च प्रसिद्धैव तेषां ॥ (१) ॥
	\pend
      \label{div_pvv.1.2}\edlabel{div_pvv.1.2}
	  
	% new div opening: depth here is 2
	

	  \begin{center}%% label @type='head'
	\textbf{ख. शास्त्रारम्भप्रयोजनम्}
	\end{center}
	

	  \pstart श्रोतृदोषबाहुल्याच्छास्त्रेण परोपकारमपश्यन् सूक्ताभ्यासभावितचित्ततामेवात्मनः शास्त्रारम्भकारणन्दर्शयन् वक्रोक्त्या दोषतापनयनेन शास्त्रे श्रोतृन् प्रवर्तयितुम(ा) ह ।
	\pend
      
	  \bigskip
	  \begingroup
	  \large
	
	    
	    \stanza[\smallbreak]
	\edlabel{pvv.2-asterisk}\footnote{\label{pvv.2-asterisk}  * द्रष्टव्यं परिशिष्टं १।१-४} प्रायः प्राकृतसक्तिरप्रतिबलप्रज्ञो जनः केवलं,नानर्थ्येव सुभाषितैः परिगता विद्वेष्ट्यपीर्ष्यामलैः ।&तेनायं न परोपकार इति नश्चिन्तापि चेतस्ततः,सूक्ताभ्यासविबर्द्धितव्यसनमित्यत्रानुबद्धस्पृहम् ॥ २ ॥\&[\smallbreak]


	
	  \endgroup
	

	  \pstart {\color{DodgerBlue3}“प्रायो”} भूयान् बाहुल्येन वा {\color{DodgerBlue3}“जनः प्राकृतेषु”} बहिःशास्त्रेषु {\color{DodgerBlue3}“सक्ति”}रभिष्वङ्गो यस्य स प्राकृतसक्तिरनेन कुप्रज्ञत्वं श्रोतृदोष उक्तः । {\color{DodgerBlue3}“अप्रतिबला”} शास्त्रार्थग्रहणं प्रत्य{\color{DodgerBlue3}“शक्ता प्रज्ञा”} यस्यासावप्रतिबलप्रज्ञः अनेनाज्ञत्वमुक्तं । {\color{DodgerBlue3}“केवलं, नानर्थ्येव सूभाषितैः”} । किन्तु सुभाषिताभिधायिनं ईर्ष्या परसंपत्तौ चेतसो व्यारोषः सैव मलश्चित्त\edlabel{pvv.2-3}\footnote{\label{pvv.2-3}  ३ आत्मात्मीयास्त्रैधातुकाश्च चैत्ताः सवासनाः ।}मलिनी\leavevmode\marginnote{\textenglish{003/s}} करणात् । तैः {\color{DodgerBlue3}“परिगतो”} युक्तः सन् {\color{DodgerBlue3}“विद्वेष्ट्यपि । ईर्ष्यामलैरिति”} व्यक्त्यपेक्षया बहुवचनं । अनेन यथाक्रममनर्थित्वममाध्यस्थ्यञ्चोक्तं (।) {\color{DodgerBlue3}“तेन श्रोतृदोषकलार्पेन अयमा”}रिप्सितो वार्त्तिकाख्यो ग्रन्थः (।) परमुपकरोतीति {\color{DodgerBlue3}“परोपकार”} इति {\color{DodgerBlue3}“नोऽस्माकञ्चिन्तापि”} नास्ति । कथन्तर्हि शास्त्रकरणे प्रवृत्तिरित्याह चेतश्चिरं {\color{DodgerBlue3}“दीर्घकालं \leavevmode\marginnote{\textenglish{2a/MA}} सूक्त”}स्या{\color{DodgerBlue3}“भ्यासेन विवर्द्धितव्यसनं”} विस्तारिताभिष्वङ्ग{\color{DodgerBlue3}“मिति”}हेतो{\color{DodgerBlue3}“रत्र वार्तिककरणेऽनुबद्धस्पृहं”} जाताभिलाषं । एतेन कुप्रज्ञतादिदोषजातमात्मनो बोधिताः श्रोतारस्तत्परिहारेण शास्त्रे प्रवर्तिता एव भवन्ति ॥ (२)
	\pend
      
	  
	% new div opening: depth here is 2
	

	  \begin{center}%% label @type='head'
	\textbf{ग. प्रमाणसिद्धिः}
	\end{center}
	

	  \pstart अयमाचा र्यो बृहदाचार्यीय प्र मा ण स मु च्च य शा स्त्रे वार्त्तिकं चिकीर्षुः स्वतः कृतभगवन्नमस्कार(:) तच्छास्त्रारम्भसमये तदाचार्यकृतभगवन्नमस्कारश्लोकं व्याख्यातुकामः प्रथमं\edlabel{pvv.3-1}\footnote{\label{pvv.3-1}  १ द्वितीयां सम्वित्तिसिद्धिम् । प्रमाणं भूतो जातो भगवान् मानमिव किन्तदित्याह ।} प्रमाणसामान्यलक्षणमाह (।)
	\pend
      
	  
	% new div opening: depth here is 1
	
\section[{१. प्रमाणलक्षणम्}]{१. प्रमाणलक्षणम्}

	  \begin{center}%% label @type='head'
	\textbf{(१) अविसंवादि ज्ञानम्}
	\end{center}
	\label{div_pvv.1.3}\edlabel{div_pvv.1.3}
	  
	% new div opening: depth here is 2
	
	  \bigskip
	  \begingroup
	  \large
	
	    
	    \stanza[\smallbreak]
	\label{pv.1.3a}\edlabel{pv.1.3a}\flagstanza{\tiny\textenglish{...v.1.3a}}प्रमाणमविसंवादि ज्ञानं;\&[\smallbreak]


	
	  \endgroup
	

	  \pstart ज्ञानं प्रमाणं\edlabel{pvv.3-2}\footnote{\label{pvv.3-2}  २ प्रमाणं सम्यग्ज्ञानमपूर्व्वगोचरमिति लक्षणं ।}नाज्ञानमिन्द्रियार्थसन्निकर्षादि । कीदृशमविसंवादि । विसंवादनं विसंवादो वञ्जनं तद्योगाद्विसंवादि । न तथाऽसावविसंवादि । अविसम्वादनमुक्तमित्यर्थः । किं पुनरित्याह (।)
	\pend
      
	  \bigskip
	  \begingroup
	  \large
	
	    
	    \stanza[\smallbreak]
	\label{pv.1.3b}\edlabel{pv.1.3b}\flagstanza{\tiny\textenglish{...v.1.3b}}अर्थक्रियास्थितिः ।&अविसंवादनं;\&[\smallbreak]


	
	  \endgroup
	

	  \pstart यथोपदर्शितार्थस्य क्रियायाः स्थितिः प्रमाणयोग्यताऽविसंवादनं(।) अतश्च यतो ज्ञानादर्थं परिच्छिद्यापि\edlabel{pvv.3-3}\footnote{\label{pvv.3-3}  ३ मरुमरीच्यादौ ।} न प्रवर्तते प्रवृत्तो वा कुतश्चित्प्रतिबन्धादेरर्थक्रियान्नाधिगच्छति । तदपि प्रमाणमेव प्रमाणयोग्यतालक्षणस्याविसंवादस्य सत्त्वात् । सैव प्रमाणयोग्यता कथमसत्यामर्थक्रियाप्राप्तौ निश्चीयत इति चेत् (।) यत्तावदसकृद्व्यवहाराभ्यासाद्दर्शनमात्रेणोपलक्षितभ्रमविविक्तस्वरूपविशेषं साधनाध्यक्षं तस्य\edlabel{pvv.3-4}\footnote{\label{pvv.3-4}  ४ प्रमेयस्य ।} \leavevmode\marginnote{\textenglish{004/s}} स्वत एव प्रमाणयोग्यतानिश्चयः कृत्रिमाकृत्रिममणिरुप्यादितत्वनिश्चयवत् । अंनुमानस्य च साध्यप्रतिबद्धजन्मनो व्यभिचाराशङ्काविरहात् । अर्थक्रियानिर्भासन्तु प्रत्यक्षं स्वत एवार्थक्रियानुभवात्मकं न तत्र परार्थक्रियाऽपेक्ष्यत इति तदपि स्वतो निश्चितप्रामाण्यं । अत एवार्थक्रियापरंपरानुसरणादनवस्थादोषोपि दुस्थ एव । यत्त्वनभ्यस्तदशायां संदिग्धप्रामाण्यमुत्पत्तौ\edlabel{pvv.4-1}\footnote{\label{pvv.4-1}  १ सत्यां ।} तस्यार्थक्रियाज्ञानादनुमानाद्वा प्रामाण्यं निश्चीयते । एतच्चाविसंवादनं बाह्यार्थेतरवादयोः समानं प्रमाणलक्षणं (।) वि ज्ञा न नयेपि\edlabel{pvv.4-2}\footnote{\label{pvv.4-2}  २ अव्यापकत्वं निरस्यन्नाह ।} साधननिर्भासज्ञानानन्तर\edlabel{pvv.4-3}\footnote{\label{pvv.4-3}  ३ वह्निज्ञानान्तरं दाहादिज्ञानं ।} मर्थंक्रिया\edlabel{pvv.4-4}\footnote{\label{pvv.4-4}  ४ रविचन्द्राम्बुदचित्रादीनां दर्शनमेवार्थक्रियास्थितिः ।}निर्भासज्ञानमेव\edlabel{pvv.4-5}\footnote{\label{pvv.4-5}  ५ यदर्थाकारं ज्ञानं तद् बाह्यार्थाविनाभावि यथा अर्थक्रियानिर्भासं ।} संवादः । अतो विज्ञप्तिमात्रत्वे प्रमाणेतरविभागव्यवहारोऽसंकीर्ण्णः।
	\pend
      

	  \pstart ननु शब्दगन्धरसस्पर्शान् चित्ररूपञ्च पश्यतो ज्ञानस्य परमर्थक्रियाज्ञानं नास्तीति तत्प्रमाणन्न\edlabel{pvv.4-6}\footnote{\label{pvv.4-6}  ६ विना भ्रान्तिं प्रयुक्ते ।} स्यादित्याह ।
	\pend
      
	  \bigskip
	  \begingroup
	  \large
	
	    
	    \stanza[\smallbreak]
	\label{pv.1.3c}\edlabel{pv.1.3c}\flagstanza{\tiny\textenglish{...v.1.3c}}शाब्देप्यभिप्रायनिवेदनात् ॥ ३ ॥\&[\smallbreak]


	
	  \endgroup
	

	  \pstart {\color{DodgerBlue3}“शाब्दे”} शब्दजनिते ज्ञानेऽपि शब्दाद् गन्धादिविषयेऽपि {\color{DodgerBlue3}“अभिप्राय”}स्याभिप्रेतार्थक्रिया (या) {\color{DodgerBlue3}“निवेदनात्”} प्रतिपादनात्प्रामाण्यं (।) अर्थक्रिया हि क्वचित्स्वरूपप्रतिपत्तिरेव । क्वचित्ततोऽन्या यथासम्भवं व्यवहारविषयः । तत्प्रापणञ्च प्रामाण्यमिति नाव्यापकं प्रमाणलक्षणम् । (३)
	\pend
      \label{div_pvv.1.4}\edlabel{div_pvv.1.4}
	  
	% new div opening: depth here is 2
	

	  \pstart ननु शब्दस्यार्थप्रतिबन्धाभावान्न प्रामाण्यं स्यादिष्यते चानुमानत्वादित्याह(।)
	\pend
      
	  \bigskip
	  \begingroup
	  \large
	
	    
	    \stanza[\smallbreak]
	वक्तृव्यापारविषयो योर्थो बुद्धौ प्रकाशते ।&प्रामाण्यं तत्र शब्दस्य नार्थतत्त्वनिबन्धनम् ॥ ४ ॥\&[\smallbreak]


	
	  \endgroup
	

	  \pstart {\color{DodgerBlue3}“वक्तृर्व्यापारो”} विवक्षा {\color{DodgerBlue3}“तस्य विषयो योऽर्थः”} समारोपितबही रूपो ज्ञानाकारः प्रकाशते {\color{DodgerBlue3}“बुद्धौ”} विवक्षा\edlabel{pvv.4-7}\footnote{\label{pvv.4-7}  ७ संकेतबलात् ।}त्मिकायां (।) {\color{DodgerBlue3}“तत्र शब्दस्य प्रामाण्यं”} लिङ्गत्वं । शब्दादुच्चरिताद्विवक्षितार्थप्रतिभासी विकल्पो\edlabel{pvv.4-8}\footnote{\label{pvv.4-8}  ८ विकल्पशब्द . . . न नदी . . . . . . । . . .}नुमीयत इत्यर्थः । तत्कार्यत्वात्तच्छब्दस्य । {\color{DodgerBlue3}“न पुनरर्थतत्त्वनिबन्धनं”} तत्प्रतिबन्धाभावात् ॥ (४)
	\pend
      \label{div_pvv.1.5}\edlabel{div_pvv.1.5}
	  
	% new div opening: depth here is 2
	

	  \pstart ननु घ\edlabel{pvv.4-9}\footnote{\label{pvv.4-9}  ९ येन ज्ञात्वा प्रवृत्तस्यार्थसंवादस्तच्चेत्प्रमाणं घटविकल्पोपि स्यात्प्रमा ॥}टोयमित्यादिज्ञानात्प्रवर्तमानस्य सम्बन्धोस्त्येवेति तत् प्रमाणं स्यात् (।) इत्याह ।
	\pend
      \leavevmode\marginnote{\textenglish{005/s}}
	  \bigskip
	  \begingroup
	  \large
	
	    
	    \stanza[\smallbreak]
	गृहीतग्रहणान्नेष्टं सांवृतं, धोप्रमाणता ।&प्रवृत्तेस्तत्प्रधानत्वात् हेयोपादेयवस्तुनि ॥ ५ ॥\&[\smallbreak]


	
	  \endgroup
	

	  \pstart {\color{DodgerBlue3}“गृहीतग्रहणान्नेष्टं सांवृतं”} दर्शनोत्तरकालं सांवृतं विकल्पज्ञानं प्रमाणं नेष्टं \leavevmode\marginnote{\textenglish{2b/MA}} दर्शनगृहीतस्यैव ग्रहणात् तेनैव च प्रापयितुं शक्यत्वात् सांवृतम\edlabel{pvv.5-1}\footnote{\label{pvv.5-1}  १ . . . . . . . घटः । तद्गतसत्ता महासामान्यं । तत्संख्यान्तर्ग्गतः । उत्क्षेपणं कर्म तस्यैवैते व्यपदेशा इति सांवृताः ।}किञ्चित्करमेव । कस्मात्पुनर्द्धियः {\color{DodgerBlue3}“प्रमाण”}तेष्यते नेन्द्रियादेः(।) {\color{DodgerBlue3}“हेयोपादेयवस्तु”}विषयायाः {\color{DodgerBlue3}“प्रवृत्ते\edlabel{pvv.5-2}\footnote{\label{pvv.5-2}  २ ज्ञात्वैव पुंसः प्रवृत्तेः ।}स्तत्प्रधानत्वात्”} ज्ञानप्रधानत्वात् धिय एव प्रामाण्यं (।) न हीन्द्रियमस्तीत्येव प्रवृत्तिः किन्तर्हि ज्ञानसद्भवात् साधकतमञ्च प्रमाणं तस्याव्यवहितव्यापारत्वात् । (५)
	\pend
      \label{div_pvv.1.6}\edlabel{div_pvv.1.6}
	  
	% new div opening: depth here is 2
	

	  \pstart एवं फलार्थिनां प्रवृत्तिव्यवहारकारित्वेन धियः प्रामाण्यं प्रतिपादितं । साम्प्रतमधिगमफलविभागकारित्वमाह ।
	\pend
      
	  \bigskip
	  \begingroup
	  \large
	
	    
	    \stanza[\smallbreak]
	\label{pv.1.6a}\edlabel{pv.1.6a}\flagstanza{\tiny\textenglish{...v.1.6a}}विषयाकारभेदाच्च धियोधिगमभेदतः ।\&[\smallbreak]


	
	  \endgroup
	

	  \pstart {\color{DodgerBlue3}“धियो”} विषयस्येवाकारो {\color{DodgerBlue3}“विषयाकारः\edlabel{pvv.5-3}\footnote{\label{pvv.5-3}  ३ रूपं}”} नीलादिस्तस्य {\color{DodgerBlue3}“भेदात् विशेषादधिगम\edlabel{pvv.5-4}\footnote{\label{pvv.5-4}  ४ धियः ।}”} स्यार्थप्रतीते{\color{DodgerBlue3}“र्भेदा”}द्विशेषाद्धिय एव प्रामाण्यं नीलस्वरूपं हि ज्ञानं नीलप्रतीतिरन्यादृशमन्यथेति धीरेव प्रमाणं ।
	\pend
      

	  \pstart ननु यथाधिगमसाधन\edlabel{pvv.5-5}\footnote{\label{pvv.5-5}  ५ अधिगमस्य साधनमिन्द्रियादुत्पत्तेः ॥}माकारस्तथेन्द्रियमपि तदुत्पत्तेरत आह (।)
	\pend
      
	  \bigskip
	  \begingroup
	  \large
	
	    
	    \stanza[\smallbreak]
	\label{pv.1.6b}\edlabel{pv.1.6b}\flagstanza{\tiny\textenglish{...v.1.6b}}भावादेवास्य तद्भावे;\&[\smallbreak]


	
	  \endgroup
	

	  \pstart तस्याकारस्य {\color{DodgerBlue3}“भावेऽस्या”}धिगमस्य {\color{DodgerBlue3}“भावादेव”} साधनमव्यवहितत्वात्‌न त्विन्द्रियादि तद्भावेपि ज्ञानानुत्पत्तावधिगमाभावात् । कश्चि\edlabel{pvv.5-6}\footnote{\label{pvv.5-6}  ६ मीमांसकः ।}दाह (।) सर्वज्ञानानामबाधितत्वलक्षणं प्रामाण्यं \edlabel{pvv.5-7}\footnote{\label{pvv.5-7}  ७ उत्तरकालभाविनानभ्यासजेन चेत् ।}स्वत एव सिध्यते\edlabel{pvv.5-8}\footnote{\label{pvv.5-8}  ८ विषयाकारस्य स्वसम्वेदनात् ज्ञानसत्तासिद्धिः ।} बाधकारणदोषज्ञानाभ्यां क्वचित्तदपोह्यते यथा शुक्तिकायां रजतज्ञाने\edlabel{pvv.5-9}\footnote{\label{pvv.5-9}  ९ स्वतः प्रामाण्यस्य ज्ञाते ज्ञाने तदात्मभूतस्य प्रामाण्यस्यापि ज्ञातत्वात् ।} चन्द्रद्वयदर्शने वा । तच्चेदमयुक्तं यतः (।)
	\pend
      
	  \bigskip
	  \begingroup
	  \large
	
	    
	    \stanza[\smallbreak]
	\label{pv.1.6c}\edlabel{pv.1.6c}\flagstanza{\tiny\textenglish{...v.1.6c}}स्वरूपस्य स्वतो गतिः ॥ ६ ॥\&[\smallbreak]


	
	  \endgroup
	\leavevmode\marginnote{\textenglish{006/s}}

	  \pstart {\color{DodgerBlue3}“स्वरूपस्य स्वतो गति\edlabel{pvv.6-1}\footnote{\label{pvv.6-1}  १ ज्ञानं । साधनं ।}”} र्न प्रामाण्यस्य । स्वतो हि प्रामाण्यस्याभिव्यक्ति\edlabel{pvv.6-1-bis}\footnote{\label{pvv.6-1-bis}  १ ज्ञानं । साधनं ।}र्व्वक्तव्या । न तूत्पत्तिः । ज्ञानात्मभूतस्य स्वस्मादुत्पत्तिविरोधात् । यदि च स्वतोऽबाधितत्वं प्रामाण्यमभिव्यक्त्या व्यवस्थापितं न तस्य बाधकानां सहस्रेणापि बाधोयुक्तः । अथ सम्भवति बाधके बाधकादर्शनं यत्र तत्राबाधितत्वमपोद्यते बाधकदर्शनेन । एवन्तर्हि बाधकादर्शनान्नाबाधितत्वं किन्तर्हि बाधकाभावात् । तस्य चादर्शनादन्यन्न साधनं । तच्चेदसाधनं नाबाधितत्वं नाम प्रामाण्यं नाप्यस्य स्वतः सिद्धिः । अतः प्रथमं बाधकादर्शनात्प्रसक्तमबाधितत्वं बाधदर्शनादपोद्यत इति किमत्रायुक्तं । ईदृश एव बाध्य\edlabel{pvv.6-2}\footnote{\label{pvv.6-2}  २ सतो न वाचा सम्वादे वाऽसतः स्वयमेवासत्त्वात् । ---सिद्धान्ती}बाधकभावः सर्व्वतः ॥
	\pend
      

	  \pstart कीदशोऽत्र प्रस\edlabel{pvv.6-3}\footnote{\label{pvv.6-3}  ३ अबाधितत्वं प्रसक्तमिति ।}ङ्गार्थः (।) किं सत्त्वमुत सत्त्वनिश्चयौ । सम्भावनामात्रम्वा । तत्र न तावदनयोः पक्षयोरपवादो युक्तः । सतः केनचिदपि बाधितुमशक्यत्वात् अन्त्येपि न स्व\edlabel{pvv.6-4}\footnote{\label{pvv.6-4}  ४ प्रमाणमिदमिति निश्चयरूपा न च व्यक्तिः । प्रमाणतदाभासयोरुत्पत्तिकाले संशयादभ्यासम्विना ॥}तोऽबाधितत्वनिश्चयः तद्विरुद्धत्वात् सम्भावनायाः । अनिश्चितमेव तत्राबाधितत्वं कथ\edlabel{pvv.6-5}\footnote{\label{pvv.6-5}  ५ निश्चितत्वादित्वेन ।}मन्यथोत्पद्यत इति चेत् । न तर्हि स्वतःप्रामाण्यनिश्चय इति यत्रापि बाधदर्शनं नास्ति तत्राप्यनाश्वास एव (।) एवञ्च बाध्यबाधकभावः {\color{DodgerBlue3}“सदसत्तापक्षयोरसङ्ग\edlabel{pvv.6-6}\footnote{\label{pvv.6-6}  ६ स्वतःप्रमाण्यसिद्धिरित्युत्पत्तिर्व्यक्तिर्व्वा स्यात् सिद्धः शब्दः सिद्ध ओदनवत् । नोत्पत्तिः । जाताजातज्ञानयोरकारकत्वात् । असत्त्वे नाजातस्य जातस्य प्रामाण्यात्सत्वात् ।}तो”} वेदितव्यः ॥(६)
	\pend
      \label{div_pvv.1.7}\edlabel{div_pvv.1.7}
	  
	% new div opening: depth here is 2
	
	  \bigskip
	  \begingroup
	  \large
	
	    
	    \stanza[\smallbreak]
	\label{pv.1.7a}\edlabel{pv.1.7a}\flagstanza{\tiny\textenglish{...v.1.7a}}\edlabel{pvv.6-asterisk}\footnote{\label{pvv.6-asterisk}  * द्रष्टव्यं परिशिष्टं १।६} प्रामाण्यं व्यवहारेण;\&[\smallbreak]


	
	  \endgroup
	

	  \pstart यदि स्वरूपमात्रं स्वतो गम्यते न प्रामाण्यं कथन्तर्हि तदवगम्यमित्याह । {\color{DodgerBlue3}“प्रामाण्यम्व्यवहारेणार्थक्रियाज्ञानेन”} (।) यस्य साधनज्ञानस्य तादात्म्यादनुभूतेपि प्रामाण्ये साशङ्का व्यवहर्त्तारोऽनभ्यासवशादनुत्पन्नानुरूपनिश्चयाः तत्रार्थक्रियाज्ञानेन प्रामाण्यनिश्चयः । अन्यत्र तु विभ्र\edlabel{pvv.6-7}\footnote{\label{pvv.6-7}  ७ इति बहिरर्थे प्रामाण्याभावोस्य ।}मशङ्कासङ्कोचादुत्पत्तावेव स्वरूप\edlabel{pvv.6-8}\footnote{\label{pvv.6-8}  ८ स्वसम्वेदनेन ।}स्य प्रामाण्यस्य स्वतो गतिरित्युक्तं ॥ अथवा\edlabel{pvv.6-9}\footnote{\label{pvv.6-9}  ९ स्वरूपस्य स्वतो गतिः । प्रामाण्यं व्यवहारेणेत्यस्य व्याख्यान्तरमाह ।} चक्षुर्विज्ञानेन रूपक्षण एको दृश्यते न भावी प्राप्यो नापि स्पर्शः तत्कयमन्यदर्शनमन्यप्राप्त्या प्रमाणं । एवं ह्यतिप्रसङ्गः स्यात् । \leavevmode\marginnote{\textenglish{007/s}} अनुमानञ्च व्याप्तिग्रहणसापेक्षं व्याप्तिश्च प्रत्यक्षेण पुरोवर्त्तिरूपमात्रग्राहिणा कथं शक्यग्रहा । देशकालव्यक्तिव्याप्त्या च व्याप्तिरुच्यते यत्र यत्र धूमस्तत्र तत्राग्नि\leavevmode\marginnote{\textenglish{3a/MA}}रिति प्रत्यक्षपृष्ठजश्च विकल्पो न प्रमाणं प्रमाणव्यापारानुकारी त्वसाविष्यते । यत्रायमध्यक्षव्यापारमतिक्रम्याधिकमारोपयति {\color{DodgerBlue3}“तत्र न प्रमाणं अमूलकत्वात्तस्य”} प्रमाणप्रमेयस्य । विजातीयव्यावृत्तेरध्यक्षेण दृष्टत्वादस्त्येव मूलमिति चेत् न सजातीयव्यावृत्त्या विशेषितत्वात्तस्याः । अन्यथा शाबलेयनाशम्प्रतियता प्रत्यक्षेण गोमात्रनाशो व्यवस्थाप्येत । अनुमानाच्च व्याप्तिग्रहणेऽनवस्थापत्तिः । अत आह । {\color{DodgerBlue3}“स्वरूपस्य स्वतो गतिः”} ॥
	\pend
      

	  \pstart स्वरूपमात्रं स्वतो गम्यते न प्राप्यरूपसापेक्षं प्रामाण्यन्नाम किञ्चिदस्ति । कथन्तर्हि तद्व्यवस्थेत्याह । {\color{DodgerBlue3}“प्रामाण्यम्व्यवहारेण ।”}
	\pend
      

	  \pstart सांव्यवहारिकस्येदं प्रमाणस्य लक्षणं संव्यवहारश्च भाविभूतरूपादिक्षणानामेकत्वेन संवादविषयोनवगीतः सर्व्वस्य । साध्यसाधनयोरेकव्यक्ति\edlabel{pvv.7-1}\footnote{\label{pvv.7-1}  १ यदि धूमो वह्नेरन्यतोपि जायेतेह वह्नेर्न जायेत द्विकारणमकारणं यतः ।}दर्शने {\color{DodgerBlue3}“समस्त”}तज्जातीयतथात्वव्यवस्थानं सम्वा\edlabel{pvv.7-2}\footnote{\label{pvv.7-2}  २ साध्याधिगतिः साधनं सारूप्ये तयोः ।} दमवधारयन्ति व्यवह\edlabel{pvv.7-3}\footnote{\label{pvv.7-3}  ३ प्रागदृष्टो धूमाधीः ।}र्त्तारः । तदनुरोधात् प्रामाण्यम्व्यवस्थाप्यते । तत्त्वतस्तु स्वसम्वेदनमात्रमप्रवृत्तिनिवृत्तिकं ॥
	\pend
      

	  \pstart ननु यत्र तावदभ्यस्तसाधनज्ञानादिषु निरस्तभ्रमा व्यवहारिणस्तेषां स्वत एव प्रामाण्यनिश्चयः यत्त्वनभ्यस्तसाधनं ज्ञानं तस्यापि व्यवहारेणेति निष्फलं शास्त्रप्रणयनमित्याह ।
	\pend
      
	  \bigskip
	  \begingroup
	  \large
	
	    
	    \stanza[\smallbreak]
	\label{pv.1.7b}\edlabel{pv.1.7b}\flagstanza{\tiny\textenglish{...v.1.7b}}शास्त्रं मोहनिर्तनं ।\&[\smallbreak]


	
	  \endgroup
	

	  \pstart यदि व्यवहारतः प्रमाणस्वरूपसिद्धिः परस्परविरो\edlabel{pvv.7-4}\footnote{\label{pvv.7-4}  ४ सम्यग्ज्ञानाद् धर्मास्तित्वपरलोकनिश्चयः ततो मोक्षाधिगमात् प्रत्यक्षपृष्ठविकल्पाख्यं (।)}धीनि लक्षणशास्त्राणि न स्युः । तस्माच्छास्त्रेण लक्षणोपदर्शनात् तद्विषयः संमोहो निवर्तनीयः । येन\edlabel{pvv.7-5}\footnote{\label{pvv.7-5}  ५ अस्याज्ञातस्य ग्राह्यत्वेपि नायमर्थः ।} परलोकनिःश्रेयसादेर्व्यवहाराप्रसिद्धस्य सिद्धिर्भवति ।
	\pend
      

	  \begin{center}%% label @type='head'
	\textbf{(२) अज्ञातार्थप्रकाशकम्}
	\end{center}
	

	  \pstart तदेवमविसम्वादनं प्रमाणलक्षणमुक्तभिदानीमन्यदाह ।
	\pend
      \leavevmode\marginnote{\textenglish{008/s}}
	  \bigskip
	  \begingroup
	  \large
	
	    
	    \stanza[\smallbreak]
	\label{pv.1.7c}\edlabel{pv.1.7c}\flagstanza{\tiny\textenglish{...v.1.7c}}अज्ञातार्थप्रकाशो वा;\&[\smallbreak]


	
	  \endgroup
	

	  \pstart प्रकाशनं प्रकाशोऽज्ञा\edlabel{pvv.8-1}\footnote{\label{pvv.8-1}  १ बुद्धोप्यन्याज्ज्ञातज्ञानात् अत्राप्यविसम्वादादेव ।} तस्यार्थस्य प्रकाशो ज्ञानं (।) तत्प्रमाणं । अर्थग्रहणेन द्विचन्द्रादिज्ञानस्य निरासः । अज्ञातग्रहणेन साम्वृतस्यावयव्यादिविषयस्य । पृथग् गृहीतानामेव रूपादीनामेकत्वेन विकल्पनात् (।) स्मरणञ्च पूर्वगृहीतार्थविकल्परूपत्वान्नाधिकग्राहि (।) गृहीते च प्राक्तनमेव प्रमाणं । इदानीन्तु स्मरणमप्र\edlabel{pvv.8-2}\footnote{\label{pvv.8-2}  २ पूर्व्वदृष्टस्य यदस्तित्वं ।} वर्तकं तस्यैव सन्देहात् ॥
	\pend
      

	  \pstart नन्वविसम्वादादेवाज्ञातार्थप्रकाशो ज्ञातव्यः । अन्यथा पीतशंखज्ञानमपि प्रमाणं स्यात् । तथा चाविसम्वादित्वमेव प्रमाणमस्तु किमनेनाभि\edlabel{pvv.8-3}\footnote{\label{pvv.8-3}  ३ ज्ञानञ्चासच्च स्यान्न बाह्ये प्रकाशकं यथा विकल्पकं ।} हितेन (।) स्यादेतद्यदि सम्भवित्वमात्रे लक्षणं स्यात् । किं नूद्दिष्ट\edlabel{pvv.8-4}\footnote{\label{pvv.8-4}  ४ यद्यनधि (गम) विषयं प्रमाणं प्रत्यक्षाग्र(?गृ) हीतं सामान्यं स्वलक्ष णविषयत्वात् परन्तु प्रत्यक्षग्राह्यं रूपिसमवायात्तच्चाक्षुषमाह मीमांसकादेः ।} त्वेनान्यथा ज्ञानत्वसत्त्वादिकमपि लक्षणं स्यात् ।
	\pend
      

	  \pstart नन्वविसम्वादिभ्योऽज्ञातार्थप्रकाशकं ज्ञायते\edlabel{pvv.8-5}\footnote{\label{pvv.8-5}  ५ विसम्वादिनः प्रकाशकत्वायोगात् ।}न तु ज्ञानत्वादिभ्य इति पूर्व्वस्यापेक्षणीयता लक्षणेन । न तु परेषामिति विशेषः । यद्येवं तदाऽविसम्वादित्वेप्यज्ञातार्थप्रकाशनमपेक्षत एव (।) नान्यथा सांवृतस्य निरासः शक्यः कर्त्तुं (।) तस्मादुभयमपि परस्परसापेक्षमेव लक्षणम्बोद्धव्यं ॥ (७)
	\pend
      \label{div_pvv.1.8}\edlabel{div_pvv.1.8}
	  
	% new div opening: depth here is 2
	

	  \pstart ननु स्वलक्षणप्रतीतेरूर्द्ध्वं सामा\edlabel{pvv.8-6}\footnote{\label{pvv.8-6}  ६ प्रमाणसमुच्चये आगमे च ।} न्यविषयं ज्ञानमज्ञातार्थप्रकाशकत्वात्प्रमाणं प्राप्तं तदेवाह (।)
	\pend
      
	  \bigskip
	  \begingroup
	  \large
	
	    
	    \stanza[\smallbreak]
	\label{pv.1.7d}\edlabel{pv.1.7d}\flagstanza{\tiny\textenglish{...v.1.7d}}स्वरूपाधिगतेः परम् ॥ ७ ॥\&[\smallbreak]


	
	  \endgroup
	
	  \bigskip
	  \begingroup
	  \large
	
	    
	    \stanza[\smallbreak]
	\label{pv.1.8a}\edlabel{pv.1.8a}\flagstanza{\tiny\textenglish{...v.1.8a}}प्राप्तं सामान्यविज्ञानं ;\&[\smallbreak]


	
	  \endgroup
	

	  \pstart प्रमाणमिति शेषः\edlabel{pvv.8-7}\footnote{\label{pvv.8-7}  ७ अस्तु वाऽनधिगमस्तथापि ।} ।
	\pend
      

	  \pstart अत्राह ।
	\pend
      
	  \bigskip
	  \begingroup
	  \large
	
	    
	    \stanza[\smallbreak]
	\label{pv.1.8b}\edlabel{pv.1.8b}\flagstanza{\tiny\textenglish{...v.1.8b}}अविज्ञाते स्वलक्षणे (।)&यज्ज्ञानमित्यभिप्रायात्;\&[\smallbreak]


	
	  \endgroup
	\leavevmode\marginnote{\textenglish{009/s}}

	  \pstart अज्ञातस्वलक्षणविष\edlabel{pvv.9-1}\footnote{\label{pvv.9-1}  १ ज्ञानत्वादीनां ।} यं {\color{DodgerBlue3}“यज्ज्ञानं”}तत्प्रमाणं न ज्ञातविषय{\color{DodgerBlue3}“मित्यभिप्रायान्नातिप्र”}स\edlabel{pvv.9-2}\footnote{\label{pvv.9-2}  २ अनधिगते स्वलक्षणे यदनधि (गम) विषयमिति सविशेषणं लक्षणं वाच्यं ।} ङ्गः ॥ एवन्तर्हि अनुमानमपि सामा\edlabel{pvv.9-3}\footnote{\label{pvv.9-3}  ३ शब्दः प्रत्यक्षोऽन्यथाऽश्रयासिद्धिः स्यात् । दृष्टस्य शब्दानित्यत्वस्य प्रत्यायनात् ज्ञातस्य ग्राह्यत्वे . . . . . .।}न्यविषयत्वात् प्रमाणं न स्यात् । नैतदस्ति तदपि च लक्ष\edlabel{pvv.9-4}\footnote{\label{pvv.9-4}  ४ अध्यक्षेण तु स एवागृहीतो यत्र निश्चयो जनितः । व्यावहारिकाधिकारात् अन्यापोहविषयाच्च ॥} णमेवानित्यादिरूपतया विषयीकरोति ॥
	\pend
      \leavevmode\marginnote{\textenglish{3b/MA}}

	  \pstart किं पुनरधिगतस्वलक्षणविषयमेव प्रमाणमिष्टं ।
	\pend
      
	  \bigskip
	  \begingroup
	  \large
	
	    
	    \stanza[\smallbreak]
	\label{pv.1.8d}\edlabel{pv.1.8d}\flagstanza{\tiny\textenglish{...v.1.8d}}स्वलक्षणविचारतः ॥ ८ ॥\&[\smallbreak]


	
	  \endgroup
	

	  \pstart {\color{DodgerBlue3}“स्वलक्षणविचारतो”}ऽर्थक्रियार्थिभिः स्वलक्षण\edlabel{pvv.9-5}\footnote{\label{pvv.9-5}  ५ अनुपलब्धिश्च प्रदेशः ज्ञानम्वेति वस्तुतो वस्त्वधिष्ठानैव ।} मेव प्रमाणेनान्विष्यते तस्यैवार्थक्रियासाधनत्वा\edlabel{pvv.9-6}\footnote{\label{pvv.9-6}  ६ आत्माकाशादौ तद्वस्तुभूतशून्या तद्बुद्धिरेवाश्रयः ।}त् । यदेव च तैरन्विष्यते तदेव शास्त्रे विचार्यते सांव्यवहारिकप्रमाणाधिकारात्\edlabel{pvv.9-7}\footnote{\label{pvv.9-7}  ७ आकाशादिविभुत्ववदीश्वरप्रामाण्यं कटाक्षयति । द्विविधेन यथोक्तेन लक्षणेन निर्दिष्टं यदेतत् प्रमाणं । प्रमाणसाधर्म्यन्तु साधयिष्यमाणं सिद्धं कृत्वोदाहृतं ।}॥ (८)
	\pend
      \label{div_pvv.1.9}\edlabel{div_pvv.1.9}
	  
	% new div opening: depth here is 2
	

	  \begin{center}%% label @type='head'
	\textbf{(३) भगवतः प्रामाण्यम्}
	\end{center}
	

	  \pstart यथोक्तद्विविधलक्षणमुक्तं यत्प्रमाणं (।)
	\pend
      
	  \bigskip
	  \begingroup
	  \large
	
	    
	    \stanza[\smallbreak]
	\label{pv.1.9a}\edlabel{pv.1.9a}\flagstanza{\tiny\textenglish{...v.1.9a}}तद्वत् प्रमाणं भगवान्;\&[\smallbreak]


	
	  \endgroup
	

	  \pstart {\color{DodgerBlue3}“तद्वद् भगवान् प्रमाणं”} । यथाभिहितस्य सत्यचतुष्टयस्याविसम्वादनात्तस्यैव परैरज्ञातस्य प्र\edlabel{pvv.9-8}\footnote{\label{pvv.9-8}  ८ लक्षणं ।}काशनाच्च ॥
	\pend
      

	  \pstart यद्येवं नमस्कारश्लोके प्रमाणायेत्येवास्तु “प्रगाणभूताये”ति किमर्थमित्याह ।
	\pend
      
	  \bigskip
	  \begingroup
	  \large
	
	    
	    \stanza[\smallbreak]
	\label{pv.1.9b}\edlabel{pv.1.9b}\flagstanza{\tiny\textenglish{...v.1.9b}}अभू\edlabel{pvv.9-9}\footnote{\label{pvv.9-9}  ९ अजातत्वनिवृत्त्यर्थ जातत्वोक्तं ।}तविनिवृत्तये (।)&भूतोन्क्तिः;\&[\smallbreak]


	
	  \endgroup
	\leavevmode\marginnote{\textenglish{010/s}}

	  \pstart {\color{DodgerBlue3}“भूतशब्द”}निर्देशोऽभूतस्य नित्यस्य निवृत्त्यर्थ {\color{DodgerBlue3}“नित्यं प्रमाणं ना”}स्तीत्यर्थः ।
	\pend
      
	  \bigskip
	  \begingroup
	  \large
	
	    
	    \stanza[\smallbreak]
	\label{pv.1.9c}\edlabel{pv.1.9c}\flagstanza{\tiny\textenglish{...v.1.9c}}साधनापेक्षा ततो युक्तां प्रमाणता ॥ ९ ॥\&[\smallbreak]


	
	  \endgroup
	

	  \pstart ततः साध\edlabel{pvv.10-1}\footnote{\label{pvv.10-1}  १ यदा भगवज्ज्ञानमुत्पन्नं तदा नाकस्मिकमिति स्वकारणं सूचयतीति अनुष्ठितप्रामाण्याविपरीतसाधनश्च भगवानिति स्वभावहेतुः ।}नापेक्षा प्रमाणता युक्ता भगवतश्च प्रामाण्यसाधनं वक्ष्यमाणं (९) ।
	\pend
      \label{div_pvv.1.10_1.11}\edlabel{div_pvv.1.10_1.11}
	  
	% new div opening: depth here is 2
	

	  \begin{center}%% label @type='head'
	\textbf{(४) ईश्वरादेरप्रामाण्यम्}
	\end{center}
	

	  \begin{center}%% label @type='head'
	\textbf{क. नित्यानिंत्ययोरप्रमाणता}
	\end{center}
	

	  \begin{center}%% label @type='head'
	\textbf{(क) नित्यस्याप्रमाणता}
	\end{center}
	
	  \bigskip
	  \begingroup
	  \large
	
	    
	    \stanza[\smallbreak]
	\label{pv.1.10a}\edlabel{pv.1.10a}\flagstanza{\tiny\textenglish{....1.10a}}नित्यं प्रमाणं नैवास्ति प्रामाण्यात्;\&[\smallbreak]


	
	  \endgroup
	

	  \pstart कस्मात् पुनर्नित्यं प्रमाणं नैवास्ति(।)आ\edlabel{pvv.10-2}\footnote{\label{pvv.10-2}  २ नित्यमीश्वरं नैयायिकाः प्रमाणमाहुः । आसंसारमेकं प्रतिसत्त्वं बुद्धिं प्रमाणमाहुः सांख्याः ।}ह वस्तुनोर्थक्रियाकारिणः सतो गतेर्ज्ञानस्य {\color{DodgerBlue3}“प्रामाण्यान्नास्ति नित्यं प्रमाणं”} । अत्रैव कारणमाह ।
	\pend
      
	  \bigskip
	  \begingroup
	  \large
	
	    
	    \stanza[\smallbreak]
	\label{pv.1.10b}\edlabel{pv.1.10b}\flagstanza{\tiny\textenglish{....1.10b}}वस्तुसंगतेः ।&ज्ञेयानित्यतया तस्या अध्रौव्यात्;\&[\smallbreak]


	
	  \endgroup
	

	  \pstart {\color{DodgerBlue3}“ज्ञेय”}स्य वस्तुनोऽर्थक्रियाकारित्वेना{\color{DodgerBlue3}“नित्य”}त्वात् {\color{DodgerBlue3}“तस्या वस्तुसङ्गते”}रपि तज्जन्यायाऽ(?अ) {\color{DodgerBlue3}“ध्रौव्याद”}नित्यत्वात् ।
	\pend
      

	  \pstart स्यादेतद् (।) अनित्यविषयमनित्यमेव ज्ञानं केवलं यस्य तत् ज्ञानं स ज्ञाता {\color{DodgerBlue3}“नित्यो”} भविष्यतीत्याह (।)
	\pend
      
	  \bigskip
	  \begingroup
	  \large
	
	    
	    \stanza[\smallbreak]
	\label{pv.1.10c}\edlabel{pv.1.10c}\flagstanza{\tiny\textenglish{....1.10c}}क्रमजन्मनां ॥ १० ॥\&[\smallbreak]


	
	  \endgroup
	
	  \bigskip
	  \begingroup
	  \large
	
	    
	    \stanza[\smallbreak]
	\label{pv.1.11a}\edlabel{pv.1.11a}\flagstanza{\tiny\textenglish{....1.11a}}नित्यादुत्पत्तिविश्लेषादपेक्षाया अयोगतः ।\&[\smallbreak]


	
	  \endgroup
	

	  \pstart ज्ञानस्य नित्यात् ज्ञातुरु{\color{DodgerBlue3}“त्पत्तेर्व्वि\edlabel{pvv.10-3}\footnote{\label{pvv.10-3}  ३ अयोगात्} श्लेषात्”} । नित्यं हि सदैकरूपं यदि {\color{DodgerBlue3}“क्रमजन्मनां”} ज्ञानानामर्जनसमर्थं । सकृदेव तानि कुर्य्यात् । अथ समर्थमपि नित्यं क्रमिसहकार्य्यपेक्षया क्रमेण करोति तदयुक्त{\color{DodgerBlue3}“मपेक्षाया अयोगात्”} ।
	\pend
      

	  \pstart कस्मात् सहकार्यपेक्षा न युक्तेत्याह (।),
	\pend
      
	  \bigskip
	  \begingroup
	  \large
	
	    
	    \stanza[\smallbreak]
	\label{pv.1.11b}\edlabel{pv.1.11b}\flagstanza{\tiny\textenglish{....1.11b}}कथञ्चिन्नोपकार्यत्वात्;\&[\smallbreak]


	
	  \endgroup
	\leavevmode\marginnote{\textenglish{011/s}}

	  \pstart नित्यस्य सर्व्वदाऽविशिष्टस्वभावस्य परैः सहकारिभिः {\color{DodgerBlue3}“कथञ्चिन्नोपकार्यत्वात्”} क्व तदपेक्षा । ततः सर्व्वज्ञानानि सकृदेव कुर्य्यादित्यवार्य्यः प्रसङ्गः ॥
	\pend
      

	  \begin{center}%% label @type='head'
	\textbf{(ख) अनित्यस्याप्यप्रमाणता}
	\end{center}
	

	  \pstart स्या\edlabel{pvv.11-1}\footnote{\label{pvv.11-1}  १ यो यत्साधनमविपरीतमनुतिष्ठति तस्य तत्प्राप्तिर्भवति । यथातुरस्यारोग्यसाधनमविपरीतमनुतिष्ठतः ।} देतत् (।) सन्तानविशेषान्निसर्गसिद्धांपरापरक्षणात्मकं साधनापेक्षा\edlabel{pvv.11-2}\footnote{\label{pvv.11-2}  २ ईश्वरज्ञानं ।} शून्यमनित्यमेव प्रमाणं भविष्यतीत्याह ।
	\pend
      
	  \bigskip
	  \begingroup
	  \large
	
	    
	    \stanza[\smallbreak]
	\label{pv.1.11c}\edlabel{pv.1.11c}\flagstanza{\tiny\textenglish{....1.11c}}अनित्येप्यप्रमाणता ॥११ ॥\&[\smallbreak]


	
	  \endgroup
	

	  \pstart {\color{DodgerBlue3}“अनित्येपि अपि”}शब्दान्नित्येप्य{\color{DodgerBlue3}“प्रमाणता”} साधनाभाव इत्यर्थः ॥ (११)
	\pend
      
	  
	% new div opening: depth here is 2
	

	  \pstart ननु सन्त्येव साधनानि यथा स्थित्वा प्रवृत्तेः संस्थानविशेषादर्थक्रियासाधनत्वात् कार्यत्वादेश्च विमत्यधिकरणानि तनुभुवनकर\edlabel{pvv.11-3}\footnote{\label{pvv.11-3}  ३ इन्द्रियशरीरादीनां कालान्तरं स्थित्वा प्रवृत्तेः कार्यहेतुषु ।}णादीन्युपादानाद्यभिज्ञबुद्धिमत्पूर्व्वकाणि तुर्य्यादिवत् प्रासादादिवत् वास्यादिवत् घटादिवच्चे\edlabel{pvv.11-4}\footnote{\label{pvv.11-4}  ४ वैधर्म्येणात्यन्ताभावः ।} त्येवमादीनि । स्थित्वा प्रवृत्त्यादयः । तुरीतन्त्वादिषूपादानाद्यभिज्ञबुद्धिमत्पूर्व्वकत्वमात्रेणोपलब्धव्याप्तयोधिकरणसिद्धा\edlabel{pvv.11-5}\footnote{\label{pvv.11-5}  ५ यत्प्रसिद्धावन्यप्रकरणसिद्धिः सोऽधिकरणसिद्धान्तः । (न्या॰ सू॰ १।१।३०) यथा सांख्यस्यान्तराभवनिषेधे आत्मैव सञ्चरत्यशरीर इति अपरीक्षिताभ्युपगमात् ॥}न्तन्यायेन नित्यव्यापिसर्व्वज्ञनित्यबुद्ध्याश्रयात्मविशेषविशिष्टमेव पक्षे साध्यमुपनयन्ति धूम इव पर्वतवर्त्तिनं दहनं । न ह्यनित्येनाव्यापिना वा नानादेशकालकार्यजातं शक्यक्रियं । नापि सर्व्वस्य कार्यस्योपादानकारणान्यसर्व्वविद् वेदितुं समर्थः । नाप्यनित्यया बुद्ध्याऽतीतानागतकालवर्त्ति वस्तुजातं ज्ञातुं शक्यं(।) तद्वेदनाच्च सर्व्ववित्(।) न चैते हेतवोऽसिद्धा धर्मिणि सत्त्वनिश्चयात् । न च विरुद्धाः सपक्षे सत्त्वात्(।) नानैकान्तिकाः साध्यसाधनयोर्व्याप्तिनिश्चयात् । न च कालात्ययापदिष्टाः साध्यसाधनबाधनाभावात्(।) नापि प्रकरणसमा विपर्ययसाधकहेत्वभावादिति । अत्राह ।
	\pend
      \label{div_pvv.1.12}\edlabel{div_pvv.1.12}
	  
	% new div opening: depth here is 2
	

	  \begin{center}%% label @type='head'
	\textbf{(ख. ईश्वरदूषणम्)}
	\end{center}
	

	  \begin{center}%% label @type='head'
	\textbf{(क) सन्निवेशमात्रान्नेश्वरानुमानम्}
	\end{center}
	
	  \bigskip
	  \begingroup
	  \large
	
	    
	    \stanza[\smallbreak]
	\label{pv.1.12a}\edlabel{pv.1.12a}\flagstanza{\tiny\textenglish{....1.12a}}स्थित्वा प्रवृत्तिः संस्थानविशेषार्थक्रियादिषु ।&इष्टसिद्धिः;\&[\smallbreak]


	
	  \endgroup
	\leavevmode\marginnote{\textenglish{012/s}}

	  \pstart {\color{DodgerBlue3}“स्थित्वा प्रवृ\edlabel{pvv.12-1}\footnote{\label{pvv.12-1}  १ उपदर्शितमर्थ प्राप्तुकामा कायवाग्‏व्यापारसहाया बुद्धिः प्रवृत्तिः ।}त्तिः-संस्थानविशेषार्थक्रियार्थक्रियादिषु”} । साधनेषूपादानाद्यभिज्ञबुद्धिमत्पूर्व्वकत्वे साध्ये {\color{DodgerBlue3}“इष्टसिद्धि”}रस्माकं । वयमपि साधारणासाधारणचेतनालक्षणकर्मनिर्मितं जगद्विचित्रमिच्छामः(।) तत्साधयता च परेण साहाय्यकमनुष्ठितं । येन साध्यगतेन विशेषेण विना धर्मिणि लिङ्गमनुपपन्नं तस्यैवाधिकरणसिद्धान्तेन \leavevmode\marginnote{\textenglish{4a/MA}} प्रतीतिर्यथा पर्व्वतवर्त्तिनो धूमाद्वह्नेः पर्व्वतवर्त्तित्वस्य(।) न ह्यन्यदेशस्थेनाग्निना जन्यमानस्य धूमस्य पर्वतवर्त्तित्वमुपपद्यते(।) नत्वेवं नित्यत्वसर्व्वज्ञत्वव्यापित्वादि विना चेतनस्य तनुभुवनादिगतं स्थित्वा प्रवृत्त्यादिकमनुपपन्नं(।) नित्यत्वादिविपर्यययोगिनापि चेतनेन क्रियमाणं तद् घटत एव ॥
	\pend
      

	  \pstart ननूक्तमेवानित्याव्यापिनः सर्व्वदेशकालवर्त्ति कार्यमकार्यं असर्व्वविदश्च सर्व्वोपादानाद्यभिज्ञता नास्तीत्यादिना । सत्यमुक्तमयुक्तन्तूक्तं कर्त्तुरेकत्वासिद्धेः । एकस्य कर्त्तुरनेकदेशकालं कार्यं कुर्व्वाणस्य तदुपादानादिकं जानानस्य नित्यत्वादिकमन्तरेण स्थित्वा प्रवृत्त्यादि नोपपद्यत इति स्यादपि तादृशस्याधिकरणसिद्धान्तेन प्रतिपत्तिः । अनेकेनापि तु नानादेशकालवर्त्तिना स्वस्वकार्यस्योपादानादिजानता क्रियमाणं स्थित्वा प्रवृत्त्यादिसङ्गतमेवात एवैकत्वस्यापि तत एव सिद्धिरयुक्ताऽनेकस्यापि हेतुत्वयोगात् । पक्षायोगव्यवच्छेद एव तु साध्यस्याधिकरणसिद्धान्तेन सिध्यतु न तु तदधिको वह्नेरिव चान्दनत्वादिः ॥
	\pend
      

	  \pstart नन्वनेक एव ते चेतनावन्तोभिमताः\edlabel{pvv.12-2}\footnote{\label{pvv.12-2}  २ बौद्धेन ।}(।) न चोपादानादिकं तन्वादीनां ते जानते । नापि तत्कर्तृ त्वमात्मनो मन्यन्ते तत्कथममी कर्त्तारः । \edlabel{pvv.12-3}\footnote{\label{pvv.12-3}  ३ ईश्वरादिकं विशेषं त्यक्त्वा सामान्येन चेतनामात्रपूर्व्वत्वं बौद्धेनेष्टसिद्धिरुक्ता न युक्ता ।}अथाधिपत्यमात्रेणैषां कर्त्तृत्वं न तूपकरणाद्यायोजनेन व्यापारेण यथा चन्द्रस्य चन्द्रकान्तद्रुतौ । तेनोपादानाद्यभिज्ञता कर्तृत्वाभिमानश्च नैषां । एवन्तर्ह्युपादानाद्यभिज्ञचेतनपूर्व्वकत्वे साध्ये नेष्ट\edlabel{pvv.12-4}\footnote{\label{pvv.12-4}  ४ बौ[द्धः] प्राह ।} सिद्धिर्व्वक्तव्या । यद्येवं भवत एव सूक्ष्मेक्षिका, तदा चेतनापूर्व्वकत्वमात्रं साधय, दृष्टान्ते च कुम्भकारत्ववत् सदप्युपादानाद्यभिज्ञत्वमप्रयोजकं मन्यस्व ॥
	\pend
      

	  \pstart सत्कथं परिहर्त्तव्यमिति चेत्(।) तत्किं\edlabel{pvv.12-5}\footnote{\label{pvv.12-5}  ५ साध्ये ।} कुम्भकारत्वमपरिहार्य । \edlabel{pvv.12-6}\footnote{\label{pvv.12-6}  ६ ईश्वरं साधयतः}अप्रयोजकत्वाद् त्यज्यत इति चेत् उपादानाद्यभि\edlabel{pvv.12-7}\footnote{\label{pvv.12-7}  ७ कुम्भकृच्चेत् युज्यते । ईश्वरोप्यत्र न प्रयोजको दृष्टः इति त्यज्यतां ।}ज्ञत्वेपि समानमेतत्(।) उपादानादिकमजानतोपि बुद्धिमतोऽनेकस्याधिष्ठा\edlabel{pvv.12-8}\footnote{\label{pvv.12-8}  ८ मयूरचन्द्रिकोपादानं मयूरो न वेत्यर्थे च तदधिष्ठानाच्चन्द्रिकोत्पद्यते ।}नमात्रेणापि कार्य्याणामुत्पत्तियोगात् ।
	\pend
      \leavevmode\marginnote{\textenglish{013/s}}
	  \bigskip
	  \begingroup
	  \large
	
	    
	    \stanza[\smallbreak]
	\label{pv.1.12b}\edlabel{pv.1.12b}\flagstanza{\tiny\textenglish{....1.12b}}असिद्धिर्वा दृष्टान्ते संशयोऽथवा ॥ १२ ॥\&[\smallbreak]


	
	  \endgroup
	

	  \pstart अथ नित्यत्वादिविशिष्टपुरुषपूर्व्वकत्वमेव साध्यं तदाऽ{\color{DodgerBlue3}“सिद्धिर्दृष्टान्ते”} । साध्यशून्यो दृष्टान्त इत्यर्थः । न हि क्वचिद् दृष्टान्ते तादृशं साध्यमुपलब्धं येन व्याप्तिः प्रतीयेत(।) असिद्धव्याप्तिकश्च हेतुरनैकान्तिक एव । स्थित्वा प्रवृत्तेरर्थक्रियासमर्थत्वादिति हेत्वोः {\color{DodgerBlue3}“संश\edlabel{pvv.13-1}\footnote{\label{pvv.13-1}  १ सन्दिग्धासिद्धोयं यथाग्निजन्यत्वेनानिश्चितो धूमः ।}योऽथवाऽ”}नैकान्तिकत्वं तेनैव त्वभि\edlabel{pvv.13-2}\footnote{\label{pvv.13-2}  २ स यथाऽन्यानपेक्षः करोत्येवं दृष्टकारणेपि स्यात् स्थित्वा स्थित्वा कार्यं प्रवर्तयति ।}मतपुरुषेण । न ह्यसौ स्थि\edlabel{pvv.13-3}\footnote{\label{pvv.13-3}  ३ संशयश्चात्र दृष्टकारणस्य तदपेक्षत्वे सोपि तथा स्यात् । न च तथा । व...नामेकः प्रधानं मन्दाश्चर(?)भात्यत्र । एकाभावेप्यभावा कार्यस्य गुणप्रधानभाव इति कृत्वा स्थित्वेत्यादि उपयोगिनां ज्ञानस्य प्रधानत्वानिश्चयात् संशयः ।}त्वा प्रवर्तमानः कार्य्येष्वर्थ क्रिया\edlabel{pvv.13-4}\footnote{\label{pvv.13-4}  ४ क्रमकरणमपि न नित्यत्वादेव ।}कारी वा तथा पुरुषाधिष्ठितः । अनवस्थाप्रसङ्गात् ।(१२)
	\pend
      \label{div_pvv.1.13}\edlabel{div_pvv.1.13}
	  
	% new div opening: depth here is 2
	
	  \bigskip
	  \begingroup
	  \large
	
	    
	    \stanza[\smallbreak]
	\label{pv.1.13a}\edlabel{pv.1.13a}\flagstanza{\tiny\textenglish{....1.13a}}सिद्धं यादृगधिष्ठातृभावाभावानुवृत्तिमत् ।&सन्निवेशादि;\&[\smallbreak]


	
	  \endgroup
	

	  \pstart अथवा {\color{DodgerBlue3}“या\edlabel{pvv.13-5}\footnote{\label{pvv.13-5}  ५ अर्थक्रियाऽकरणेपि न वस्तुत्वमनव स्थातः । स्थित्वा प्रवृत्तिरपि न नित्यत्वादिति नित्यपक्षे ।}दृ”}शबुद्धिमत्पूर्व्वकं {\color{DodgerBlue3}“सन्निवेशादि”} दृष्टान्ते दृष्टं तस्य धर्मिण्यसिद्धिरित्याह । घटे दृष्टान्तधर्मिणि {\color{DodgerBlue3}“सन्निवेशादि यादृशं”} पृथुबुध्नोदरादि दृष्टैकव्याक्तिजात्या विशेषितमधि{\color{DodgerBlue3}“ष्ठातुः”} पुंसोऽन्वयव्यतिरेकानुविधानवत् व्यवहारप्रगल्भपुरुषाणां तत्सिद्धान्तानुरोधरहितानां प्रत्यक्षबलेनानुरूपनिश्चयोत्पा{\color{DodgerBlue3}“दात्सिद्धं”} निश्चितं ॥
	\pend
      
	  \bigskip
	  \begingroup
	  \large
	
	    
	    \stanza[\smallbreak]
	\label{pv.1.13c}\edlabel{pv.1.13c}\flagstanza{\tiny\textenglish{....1.13c}}तद्युक्तं तस्माद् यदनुमीयते ॥१३ ॥\&[\smallbreak]


	
	  \endgroup
	

	  \pstart तस्मात्सन्निवेशादपरत्रानुपलब्धपुरुषजन्मनि घटे यद् बुद्धिमदधिष्ठानम{\color{DodgerBlue3}“नुमीयते तद्युक्तं”} तस्यैव पुरुषकार्यत्वेन निश्चयात् ॥ \edlabel{pvv.13-6}\footnote{\label{pvv.13-6}  ६ न पुनर्वस्तुभेदे ।}असन्निवेशव्यावृत्तं सन्निवेशमात्रं तु सदपि न तत्कार्यतया प्रत्यक्षमुपस्थापयति । प्रत्यक्षव्यापारविवादे च पटुप्रचारा व्यवहारिणः शरणं (।) न हि कश्चिद् व्यवहारी घटं पुरुषकृतं पश्यन् शरावादि पर्व्वतादिकम्वा तत्कृतमवधारयति (।) यदा तु शरावादीनपि तत उदयमासादयतः पश्यति तदा तानपि तत्कृतानवैति (।) अतः सन्निवेशविशेषं\leavevmode\marginnote{\textenglish{4b/MA}} पुरुषकार्यं दृष्टवतः सन्निवेशमात्रात्तदनुमानमयुक्तं (। १३)
	\pend
      \label{div_pvv.1.14}\edlabel{div_pvv.1.14}
	  
	% new div opening: depth here is 2
	\leavevmode\marginnote{\textenglish{014/s}}

	  \pstart एतदेवाह (।)
	\pend
      
	  \bigskip
	  \begingroup
	  \large
	
	    
	    \stanza[\smallbreak]
	\label{pv.1.14}\edlabel{pv.1.14}\flagstanza{\tiny\textenglish{...v.1.14}}वस्तुभेदे प्रसिद्धस्य शब्दसाम्यादभेदिनः ।&न युक्तानुमितिः पाण्डुद्रव्यादिवद् हुताशने ॥ १४ ॥\&[\smallbreak]


	
	  \endgroup
	

	  \pstart {\color{DodgerBlue3}“वस्तुभेदे”} घटे {\color{DodgerBlue3}“प्रसिद्धस्य”} पुरुषपूर्व्वकत्वस्य सन्निवेश इति {\color{DodgerBlue3}“शब्दसाम्यादभेदिनः”} सन्निवेशमात्रात् पर्व्वतादौ {\color{DodgerBlue3}“न युक्तानुमितिः पाण्डुद्रव्यादिवद् हुताशने”} । यथा पाण्डुविशेषस्य धूमस्य कारणत्वेन दृष्टे वह्नौ पाण्डुशब्दसाम्यादभेदिनो यतः कुतश्चिद् पाण्डुद्रव्याद् धूमादेरनुमानमनुचितमतो यत्तद् बुद्धिमद्‏व्याप्तं सन्निवेशा\edlabel{pvv.14-1}\footnote{\label{pvv.14-1}  १ धटादेः ।} दि त\edlabel{pvv.14-2}\footnote{\label{pvv.14-2}  २ पर्व्वतादौ ।}द्धर्भिणि नास्तीत्यसिद्धिर्हेतूनां ॥ (१४)
	\pend
      \label{div_pvv.1.15}\edlabel{div_pvv.1.15}
	  
	% new div opening: depth here is 2
	

	  \pstart ननु सन्निवेशादिसामान्यादसन्निवेशादिव्यावृत्त्या हेतवो भविष्यन्तीत्यत आह (।)
	\pend
      
	  \bigskip
	  \begingroup
	  \large
	
	    
	    \stanza[\smallbreak]
	\label{pv.1.15}\edlabel{pv.1.15}\flagstanza{\tiny\textenglish{...v.1.15}}अन्यथा कुम्भकारेण मृद्विकारस्य कस्यचित् ।&घटादेः करणात् सिध्येद् वल्मीकस्यापि तत्कृतिः ॥ १५ ॥\&[\smallbreak]


	
	  \endgroup
	

	  \pstart {\color{DodgerBlue3}“अन्यथा”} यदि साध्यव्याप्तं विशेषं त्यक्त्वा सामान्यं लिङ्गं क्रियते तदा {\color{DodgerBlue3}“कुम्भकारेण कस्यचिद् घटादेर्मृद्विकारस्य करणाद्वल्मीकस्यापि”} मृद्वि\edlabel{pvv.14-3}\footnote{\label{pvv.14-3}  ३ घटवत् ।}कारत्वा{\color{DodgerBlue3}“त्तेन”} कुम्भ\edlabel{pvv.14-4}\footnote{\label{pvv.14-4}  ४ घटस्य कृतकृत्वं शब्दे नास्तीति यथाकार्यें जात्युत्तरं तथेदमिति चेत् ।}कारेण {\color{DodgerBlue3}“कृतिः”} करणं {\color{DodgerBlue3}“सिध्येत।”} । (१५)
	\pend
      \label{div_pvv.1.16}\edlabel{div_pvv.1.16}
	  
	% new div opening: depth here is 2
	

	  \pstart स्यादेतत् (।) न मृद्विकारत्वं कुम्भकारव्याप्तं तमन्तरेणापि वल्मी\edlabel{pvv.14-2-bis}\footnote{\label{pvv.14-2-bis}  २ पर्व्वतादौ ।}कस्योत्पत्तेः । यद्ये\edlabel{pvv.14-5}\footnote{\label{pvv.14-5}  ५ वह्नेः ।}वं बुद्धिमन्तमन्तरेणैतत्तन्वादि जायत इति तदपि पक्षो न स्यात् । अथ पक्षेण न व्यभिचारः वल्मीकेप्ययं न्यायः समानः (।) तस्मात्सन्निवेशादिसामान्यं व्याप्त्यसिद्धेरनैकान्तिकमेव ॥\edlabel{pvv.14-6}\footnote{\label{pvv.14-6}  ६ नन्वेतत् कार्यसमं सामान्ये हेतौ विशेषपरिकल्पनादिति ॥} न चैतत्कार्यसमं दूषणं यतः (।)
	\pend
      
	  \bigskip
	  \begingroup
	  \large
	
	    
	    \stanza[\smallbreak]
	\label{pv.1.16}\edlabel{pv.1.16}\flagstanza{\tiny\textenglish{...v.1.16}}साध्येनानुगमात् कार्ये सामान्येनापि साधने ।&सम्बन्धिभेदाद् भेदोक्तिदोषः कार्यसमो मतः ॥ १६ ॥\&[\smallbreak]


	
	  \endgroup
	

	  \pstart {\color{DodgerBlue3}“साध्येनानि”}त्यत्वेना{\color{DodgerBlue3}“नुगमाद्”} व्यापनात् {\color{DodgerBlue3}“कार्ये”} कृतकत्वे {\color{DodgerBlue3}“सामान्येनापि साधने”} कृते {\color{DodgerBlue3}“सम्बधिनोः”} साध्यधर्मिदृष्टान्तधर्मिणो{\color{DodgerBlue3}“र्भेदात्”} सम्बद्धस्य साधनस्य {\color{DodgerBlue3}“भेदोक्त्या”} यदि साध्यधर्मिगतं कार्यत्वं हेतुस्तदा नान्वयसिद्धिरथ दृष्टान्तगतं तदाऽसिद्धो हेतुरिति यो दोषः {\color{DodgerBlue3}“स कार्यसमो मतः”} ॥ (१६)
	\pend
      \label{div_pvv.1.17}\edlabel{div_pvv.1.17}
	  
	% new div opening: depth here is 2
	

	  \pstart अत्र तु सन्निवेशादिसामान्यं न साध्यव्या\edlabel{pvv.14-7}\footnote{\label{pvv.14-7}  ७ अविशिष्टसंस्थानसामान्यस्यैवाभावात् ।}प्तं सिद्धमिति विशेषेऽसिद्धत्वं सामान्ये \leavevmode\marginnote{\textenglish{015/s}} चानैकान्तिकदू\edlabel{pvv.15-1}\footnote{\label{pvv.15-1}  १ कुम्भकारादिनैव कृतत्वसम्भावनया ।}षणं न जात्युत्तरं ॥
	\pend
      

	  \pstart ननु सन्निवेशादिशब्दवाच्योऽर्थः {\color{DodgerBlue3}“कुम्भे बुद्धिमद्‏व्याप्तः प्रतीतः स च तन्वा-”} दिष्वपि दृश्यते ततो विशेषे विकल्पो न युक्त इत्याह (।)
	\pend
      
	  \bigskip
	  \begingroup
	  \large
	
	    
	    \stanza[\smallbreak]
	\label{pv.1.17}\edlabel{pv.1.17}\flagstanza{\tiny\textenglish{...v.1.17}}जात्यन्तरे प्रसिद्धस्य शब्दसामान्यदर्शनात् ।&न युक्तं साधनं गोत्वाद् वागादीनां विषाणिवत् ॥ १७ ॥\&[\smallbreak]


	
	  \endgroup
	

	  \pstart {\color{DodgerBlue3}“जात्यन्तरे”} जातिविशेषे घटसन्निवेशादौ {\color{DodgerBlue3}“प्रसिद्धस्य”} साध्यस्य बुद्धिमत्पूर्व्वकत्वस्य तं विशेषं परित्यज्य {\color{DodgerBlue3}“शब्दसामान्यदर्शनाद्”} सन्निवेशादिमात्रेण {\color{DodgerBlue3}“न युक्तं साधनं”} वागादीनां {\color{DodgerBlue3}“गोत्वात्”} गोशब्दवाच्यत्वाद्विषाणिवत् विषाणित्वस्येव न युक्तं साधनं विशिष्टजातेरेव विषाणव्याप्त्युपलब्धेः । (१७)
	\pend
      \label{div_pvv.1.18}\edlabel{div_pvv.1.18}
	  
	% new div opening: depth here is 2
	

	  \pstart किञ्च (।)
	\pend
      
	  \bigskip
	  \begingroup
	  \large
	
	    
	    \stanza[\smallbreak]
	\label{pv.1.18}\edlabel{pv.1.18}\flagstanza{\tiny\textenglish{...v.1.18}}विवक्षापरतन्त्रत्वान्न शब्दाः सन्ति कुत्र वा ।&तद्‏भावाद्; अर्थसिद्धौ तु सर्वं सर्वस्य सिध्यति ॥ १८ ॥\&[\smallbreak]


	
	  \endgroup
	

	  \pstart {\color{DodgerBlue3}“विवक्षापरतन्त्रत्वान्न श\edlabel{pvv.15-2}\footnote{\label{pvv.15-2}  २ यद्यपि न प्रवर्त्तितः शब्दः । तथापि यदैव प्रवर्त्त्यते तदैव भवतीति ।}ब्दाः सन्ति कुत्र वा”} सर्व्वत्रैव सन्ति । तस्य शब्दस्य {\color{DodgerBlue3}“भावात् । अर्थसिद्धौ”} सत्यां {\color{DodgerBlue3}“सर्व्व”} यथासमीहितं साध्यं {\color{DodgerBlue3}“सर्व्वस्य”} पुंसः {\color{DodgerBlue3}“सिध्यति”} उक्तमीश्वरसाधनस्य दूषणं ॥ (१८)
	\pend
      \label{div_pvv.1.19}\edlabel{div_pvv.1.19}
	  
	% new div opening: depth here is 2
	

	  \pstart अमुमेव न्यायमन्यत्राप्यतिदिशन्नाह ।
	\pend
      
	  \bigskip
	  \begingroup
	  \large
	
	    
	    \stanza[\smallbreak]
	\label{pv.1.19}\edlabel{pv.1.19}\flagstanza{\tiny\textenglish{...v.1.19}}एतेन का पि ला दीनामचैतन्यादि चिन्तितम् ।&अनित्यादेश्च चैतन्य मरणात् त्वगपोहतः ॥ १९ ॥\&[\smallbreak]


	
	  \endgroup
	

	  \pstart {\color{DodgerBlue3}“एतेन”} शब्दसामान्यमात्रस्यार्थशून्यस्याहेतुत्वकथनेन यत्का{\color{DodgerBlue3}“पिलादे”}र्ब्बुद्धिसुखा{\color{DodgerBlue3}“दीना\edlabel{pvv.15-3}\footnote{\label{pvv.15-3}  ३ रूपादिवत् ।}म”}नित्यत्वोत्पत्तिमत्त्वादिहेतुतोऽचेतनत्वमिष्टं तथा दि ग म्ब रा णां चैतन्यं तरूणां सर्व्वस्याः {\color{DodgerBlue3}“त्वचोऽपोह”}तोऽपगते{\color{DodgerBlue3}“र्मर”}णादभिमतं {\color{DodgerBlue3}“तच्चिन्तितं”} वेदितव्यं (।) यथा ह्यप्रच्युतप्राच्यरूपस्य तिरोधानमनित्यत्वं सां\edlabel{pvv.15-4}\footnote{\label{pvv.15-4}  ४ आसर्गप्रलयान्नित्यैका बुद्धिर्न वेदना, प्रकृतिर्भोग्या भोक्ता पुरुषः सांख्यस्य (।) सांख्यः स्वस्वभावाच्युतस्य तिरोधानमनित्यतामाहातिरोधानं बौद्धस्यासिद्धं । निरन्वयनाशः सांख्यस्य । आत्मनः सञ्चरन्तो वृक्षाद्यवस्था भवन्तीति क्षपणः । अनित्यता सामान्या सिद्धिर्व्विनिश्चयेस्ति ।}ख्य स्येष्टं बौ द्ध स्य तु निरन्वयविनाशित्वं । तस्य यथाक्रममुपादाने प्रतिवाद्यसिद्धता वाद्यसिद्धता च ।
	\pend
      \leavevmode\marginnote{\textenglish{016/s}}

	  \pstart ननूभयासिद्धमनित्यत्वमस्ति किञ्चिदृते शब्दसाम्याद् तथा विज्ञानेन्द्रियायुर्निरोधलक्षणं मरणमिष्टं बौद्धसिद्धान्ते तस्य च तरुष्व\edlabel{pvv.16-1}\footnote{\label{pvv.16-1}  १ तत्सिद्धौ चैतन्यं सिद्धमिति साध्यं ।} सिद्धिः । चेतनत्वस्यै\edlabel{pvv.16-2}\footnote{\label{pvv.16-2}  २ न शोषमात्रस्य ।} व साध्यत्वात् । न ह्यसिद्धेषु त\edlabel{pvv.16-3}\footnote{\label{pvv.16-3}  ३ विज्ञानादिषु तत्र ।}न्निरोधो युक्तः शोष\edlabel{pvv.16-4}\footnote{\label{pvv.16-4}  ४ शोषोमरणमित्यपि न वर्ण्णवादिनानेकान्तात् । अपि तु यच्छोषवत्तच्चेतनावन्न सिद्धं विपर्ययेऽबाधात् ॥}मात्रन्तु तरुषु मरणमुपचारा\leavevmode\marginnote{\textenglish{5a/MA}} दुच्यते । यदि च मरणवाच्यत्वमात्रं हेतुः तदा तैला\edlabel{pvv.16-5}\footnote{\label{pvv.16-5}  ५ घृतहिङ्ग्वादिषु ।}दिष्वपि तत्सत्त्वात् साधारणानैकान्तिकता ॥(१९)
	\pend
      \label{div_pvv.1.20}\edlabel{div_pvv.1.20}
	  
	% new div opening: depth here is 2
	

	  \pstart नन्वेवं कृतकत्वादिकमपि हेतु\edlabel{pvv.16-6}\footnote{\label{pvv.16-6}  ६ वैशेषिकेणापौरुषेयशब्दनिषेधाय मीमांसककृतः ।} र्न्न स्यादाकाशगुणश्शब्दस्य धर्मो बौद्धस्यासिद्धः । अन्यथा चान्यस्येत्यत आह ।
	\pend
      
	  \bigskip
	  \begingroup
	  \large
	
	    
	    \stanza[\smallbreak]
	\label{pv.1.20}\edlabel{pv.1.20}\flagstanza{\tiny\textenglish{...v.1.20}}वस्तुस्वरूपेऽसिद्धेऽयं न्यायः सिद्धे विशेषणम् ।&अबाधकमसिद्धावप्याकाशाश्रयवद् ध्वनेः ॥ २० ॥\&[\smallbreak]


	
	  \endgroup
	

	  \pstart {\color{DodgerBlue3}“वस्तुस्वरूपेऽसिद्धेऽयन्न्यायः”} यथोक्तासिद्धिचोदनालक्षणः वस्तुस्वरूपे तु धर्मिणि हेतौ {\color{DodgerBlue3}“सिद्धे विशे\edlabel{pvv.16-7}\footnote{\label{pvv.16-7}  ७ व्याप्यो हेतुर्न चायं विशेषणव्याप्तः । हेतुवादीष्टञ्च साध्यं विशेषणञ्च तदनिष्टं ।} षणमसिद्धावप्यबाधकं”} किमिवाका{\color{DodgerBlue3}“शाश्रयवद् ध्वनेः”} । यथा शब्दस्याकाशगुणत्वं विशेषणमपि न बाधकं धर्मितायाः कृतकत्वादिहेतोर्व्वा विशेषणासिद्धावपि हि शब्दो धर्मी प्रत्यक्षसिद्धः । कृतकत्वादि चानुमानसि\edlabel{pvv.16-8}\footnote{\label{pvv.16-8}  ८ यः प्रत्ययभेदभेदी स कृतकः ।}द्धं तावतैव च साध्यसाधनभावो निर्व्विरोधः । (२०)
	\pend
      \label{div_pvv.1.21_1.22}\edlabel{div_pvv.1.21_1.22}
	  
	% new div opening: depth here is 2
	

	  \pstart यत्र शब्दोप्यसिद्धो वस्तु तु सिद्धं तत्र कथमित्याह ।
	\pend
      
	  \bigskip
	  \begingroup
	  \large
	
	    
	    \stanza[\smallbreak]
	\label{pv.1.21}\edlabel{pv.1.21}\flagstanza{\tiny\textenglish{...v.1.21}}असिद्धावपि शब्दस्य सिद्धे वस्तुनि सिध्यति ।&औ लू क्य स्य यथा बौद्धेनोक्तं मूर्त्यादिसाधनम् ॥ २१ ॥\&[\smallbreak]


	
	  \endgroup
	

	  \pstart {\color{DodgerBlue3}“असिद्धावपि शब्दस्य सिद्धे वस्तुनि”} साधनाभिमते {\color{DodgerBlue3}“सिध्यति”}\edlabel{pvv.16-9}\footnote{\label{pvv.16-9}  ९ वस्त्वेव वस्तुनः प्रतिबन्धाद् गमकं न शब्दः ।} साध्यो\edlabel{pvv.16-10}\footnote{\label{pvv.16-10}  १० सप्रतिभासादिवादिनां ।}र्थो {\color{DodgerBlue3}“यथा औलूक्यस्य”} वै शे षि क स्य परमाणूनामनित्यत्वसाधनार्थं {\color{DodgerBlue3}“बौद्धेन मूर्त्त्यादिसाधन”}\edlabel{pvv.16-11}\footnote{\label{pvv.16-11}  ११ अनित्याः परमाणवो मूर्त्तत्वाद् घटादिवत् ।} \leavevmode\marginnote{\textenglish{017/s}} {\color{DodgerBlue3}“मुक्तं”} शब्दासिद्धावपि सिध्यति (।) तथा हि वैशेषिकस्यासर्व\edlabel{pvv.17-1}\footnote{\label{pvv.17-1}  १ द्विविधं द्रव्यं सर्व्वगतं पृथिव्यादि असर्व्वगतं घटादि ।} गतं द्रव्यपरिमाणं मूर्त्तिरिष्टा बौद्धस्य स्पर्शवति\edlabel{pvv.17-2}\footnote{\label{pvv.17-2}  २ न हि स परिमाणं नाम किञ्चिदिच्छति ।} सा प्रसिद्धा । ततो नोभयसम्प्रतिपन्ना मूर्त्तिशब्दवृत्तिर्व्वाच्यभेदात् ।---
	\pend
      

	  \pstart ---शब्दस्यासिद्धावपि च स्पर्शवत्त्वलक्षणोर्थो द्वयोरपि सिद्धः स एव हेतुत्वेनाभिप्रेत इति भवति साधनमतश्च (।)
	\pend
      
	  \bigskip
	  \begingroup
	  \large
	
	    
	    \stanza[\smallbreak]
	\label{pv.1.22}\edlabel{pv.1.22}\flagstanza{\tiny\textenglish{...v.1.22}}तस्यैव व्यभिचारादौ शब्देप्यव्यभिचारिणि ।&दोषवत् साधनं ज्ञेयं वस्तुनो वस्तुसिद्धितः ॥ २२ ॥\&[\smallbreak]


	
	  \endgroup
	

	  \pstart {\color{DodgerBlue3}“तस्यैवा”}र्थस्य {\color{DodgerBlue3}“व्यभिचारा”}दावादिशब्दादसिद्धत्वे विपर्य्ययव्याप्तौ च {\color{DodgerBlue3}“शब्देप्य\edlabel{pvv.17-3}\footnote{\label{pvv.17-3}  ३ यथा विषाणी शाबलेयः कलभो वा गोत्वात् हस्तित्वाद्वा गमनात् हस्तयोगाद्वेत्यर्थः ।}व्यभिचारिणि दोषवत्साधनं ज्ञेयं”} कस्मा{\color{DodgerBlue3}“द्वस्तुनो”} हेतो{\color{DodgerBlue3}“र्व्वस्तुनः”} साध्यस्य {\color{DodgerBlue3}“सिद्धितः”} । एवं साध्यव्याप्तार्थशून्यशब्दमात्रकाणीश्वरसाधनानि दोषवन्ति बोद्धव्यानि । (२२)
	\pend
      \label{div_pvv.1.23}\edlabel{div_pvv.1.23}
	  
	% new div opening: depth here is 2
	

	  \begin{center}%% label @type='head'
	\textbf{(ख) ईश्वरबाधकं प्रमाणम्}
	\end{center}
	

	  \pstart ननु किं पुनरीश्वरस्य बाधकं\edlabel{pvv.17-4}\footnote{\label{pvv.17-4}  ४ विश्वकारणमीश्वर इत्यत्र ।} प्रमाणमित्याह (।)
	\pend
      
	  \bigskip
	  \begingroup
	  \large
	
	    
	    \stanza[\smallbreak]
	\label{pv.1.23}\edlabel{pv.1.23}\flagstanza{\tiny\textenglish{...v.1.23}}यथा तत् कारणं वस्तु तथैव तदकारणम् ।&यदा तत् कारणं केन मतं नेष्टमकारणम् ॥ २३ ॥\&[\smallbreak]


	
	  \endgroup
	

	  \pstart {\color{DodgerBlue3}“यथा”} सदृशेन स्वभावेन {\color{DodgerBlue3}“तदी”}\edlabel{pvv.17-5}\footnote{\label{pvv.17-5}  ५ योऽकारकावस्थाऽविशिष्टौ न स करोति स इव अविशिष्टश्चायमिति व्यापकानुपलब्धिः ।} श्वराख्यं {\color{DodgerBlue3}“वस्तु कारणमिष्टं”} सर्ग्गावस्थायां {\color{DodgerBlue3}“तथैव”} तेनैव स्वभावेन सर्ग्गात् प्राक् तन्न {\color{DodgerBlue3}“कारणं”} केन विशेषेण मतं । न त्व\edlabel{pvv.17-6}\footnote{\label{pvv.17-6}  ६ अकारणमेवेष्यतां कारणवस्थायामप्यकारकावस्थाऽविशेषात् ।}कारणमिष्टं । कारणत्वं ह्यकारकावस्थाविशिष्टत्वेन व्याप्तं तदभावात्कारणत्वाभावः । (२३)
	\pend
      \label{div_pvv.1.24}\edlabel{div_pvv.1.24}
	  
	% new div opening: depth here is 2
	

	  \begin{center}%% label @type='head'
	\textbf{a. अकारकं न कारणम्}
	\end{center}
	

	  \pstart यदि पुनरकारकावस्थाऽविशिष्टोपीश्वरः कारणमुच्यते तदा (।)
	\pend
      
	  \bigskip
	  \begingroup
	  \large
	
	    
	    \stanza[\smallbreak]
	\label{pv.1.24}\edlabel{pv.1.24}\flagstanza{\tiny\textenglish{...v.1.24}}शस्त्रौषधाभिसम्बन्धाच्चैत्रस्य व्रणरोहणे ।&असम्बद्धस्य किं स्थाणोः कारणत्वं न कल्प्यते ॥ २४ ॥\&[\smallbreak]


	
	  \endgroup
	\leavevmode\marginnote{\textenglish{018/s}}

	  \pstart {\color{DodgerBlue3}“चैत्रस्य शस्त्रौषधयोः सम्बन्धाद्”} व्रणे {\color{DodgerBlue3}“व्रणरोहणे”} च वृत्तेऽ{\color{DodgerBlue3}“सम्बद्धस्य”} व्यापारद्वारेणाप्रत्यासन्नस्य {\color{DodgerBlue3}“स्थाणोः किं कारणत्वं न कल्प्यते”} निमित्तस्य समानत्वात् । (२४)
	\pend
      \label{div_pvv.1.25}\edlabel{div_pvv.1.25}
	  
	% new div opening: depth here is 2
	

	  \pstart नन्वकारणावस्थातो व्यापारसमावेशादस्ति विशेषः कारणवस्थायामित्याह\edlabel{pvv.18-1}\footnote{\label{pvv.18-1}  १ पूर्व्वस्यासिद्धिं परिहरति शशविषाणवत् ।} ।
	\pend
      
	  \bigskip
	  \begingroup
	  \large
	
	    
	    \stanza[\smallbreak]
	\label{pv.1.25a}\edlabel{pv.1.25a}\flagstanza{\tiny\textenglish{....1.25a}}स्वभावभेदेन विना व्यापारोपि न युज्यते ।\&[\smallbreak]


	
	  \endgroup
	

	  \pstart {\color{DodgerBlue3}“स्वभावभेदेन विना”} न केवलं कारकत्वं (।) {\color{DodgerBlue3}“व्यापारोपि”} निर्व्यापारस्य नित्यस्य {\color{DodgerBlue3}“न युज्यते”} (।)
	\pend
      

	  \pstart किञ्च (।)
	\pend
      
	  \bigskip
	  \begingroup
	  \large
	
	    
	    \stanza[\smallbreak]
	\label{pv.1.25b}\edlabel{pv.1.25b}\flagstanza{\tiny\textenglish{....1.25b}}नित्यस्याव्यतिरेकित्वात् सामर्थ्यञ्च दुरन्वयम् ॥ २५ ॥\&[\smallbreak]


	
	  \endgroup
	

	  \pstart {\color{DodgerBlue3}“नित्यस्याव्यतिरेकित्वात् सामर्थ्यञ्च दुरन्वयं”} दुरवगमं न ह्यस्तीति कारणम् (।)अपि तु यदभावात्कार्याभावः स तत्कारणमन्यथाऽकाशादीनामपि हेतुत्वप्रसङ्गः । (२५)
	\pend
      \label{div_pvv.1.26}\edlabel{div_pvv.1.26}
	  
	% new div opening: depth here is 2
	

	  \pstart अपि च (।)
	\pend
      
	  \bigskip
	  \begingroup
	  \large
	
	    
	    \stanza[\smallbreak]
	\label{pv.1.26}\edlabel{pv.1.26}\flagstanza{\tiny\textenglish{...v.1.26}}येषु सत्सु भवत्येव यत् तेभ्योऽन्यस्य कल्पने ।&तद्धेतुत्वेन सर्वत्र हेतूनामनवस्थितिः ॥ २६ ॥\&[\smallbreak]


	
	  \endgroup
	

	  \pstart {\color{DodgerBlue3}“येषु”} कारणेषु {\color{DodgerBlue3}“सत्सु\edlabel{pvv.18-2}\footnote{\label{pvv.18-2}  २ निमित्तकारणमीशस्तन्तुवायवदित्याह ।}”} यत्कार्य{\color{DodgerBlue3}“म्भवत्येव तेभ्यः”} कारणेभ्यो{\color{DodgerBlue3}“न्यस्य”} पदार्थस्य तत्कार्य{\color{DodgerBlue3}“हेतुत्वेन”} क\edlabel{pvv.18-3}\footnote{\label{pvv.18-3}  ३ यदा तदा तत्कारणं ।}ल्पने {\color{DodgerBlue3}“सर्व्वत्र”} कार्य{\color{DodgerBlue3}“हेतूनामनवस्थितिः”} प्राप्नोत्यपरापरकल्पनया (।) तस्माद् दृष्टसामर्थ्या एव क्षितिबीजादयः कारणमङ्कुरस्य नेश्वरादिरदृष्टसामर्थ्यः । (२६)
	\pend
      \label{div_pvv.1.27}\edlabel{div_pvv.1.27}
	  
	% new div opening: depth here is 2
	

	  \pstart ननु क्षित्यादिरप्यकारकावस्थातो न विशिष्टस्व\edlabel{pvv.18-4}\footnote{\label{pvv.18-4}  ४ एतेन पूर्व्वस्यानेकान्तमाह । असंस्कृतक्षेत्रादयः (।)} भावः कारणावस्थायामङ्कुरस्येत्याह (।)
	\pend
      
	  \bigskip
	  \begingroup
	  \large
	
	    
	    \stanza[\smallbreak]
	\label{pv.1.27}\edlabel{pv.1.27}\flagstanza{\tiny\textenglish{...v.1.27}}स्वभावपरिणामे-न हेतुरङ् कुरजन्मनि ।&भूम्यादिस्तस्य संस्कारे तद्विशेषस्य दर्शनात् ॥ २७ ॥\&[\smallbreak]


	
	  \endgroup
	

	  \pstart उपसर्पणप्रत्यया\edlabel{pvv.18-5}\footnote{\label{pvv.18-5}  ५ आद्युत्पत्तौ जगतो निमित्तमिति चेन्न तेनैवानेकान्त ईश्वरान्तरप्रसङ्गात् तन्मात्रहेतुत्वे कर्मनैफल्यं ॥}ददृष्टसह\edlabel{pvv.18-6}\footnote{\label{pvv.18-6}  ६ वृष्ट्यादि ।} कारिणः प्राप्तकार्योत्पादानुगुणातिशया {\color{DodgerBlue3}“भूम्यादिः स्वभावपरिणामे-न”} कार्यानुगुणातिशयतारतम्यमुक्तापरापरक्षणलक्षणेनान्त्यावस्था\leavevmode\marginnote{\textenglish{019/s}} प्राप्ताऽ{\color{DodgerBlue3}“ङ्कुरजन्मनि हेतु”}र्भवति (।) स तु पूर्व्वपरैकरूपः क्रमाक्रमयोरर्थक्रियाविरोधात् । कुत एतदिति चेत् तस्य भूम्यादेः कर्षणपांसुप्रक्षेपादिना {\color{DodgerBlue3}“संस्कारे”} तस्याङकुरस्य पुष्टतरादि{\color{DodgerBlue3}“विशेषस्य दर्शनात्”} । (२७)
	\pend
      \label{div_pvv.1.28}\edlabel{div_pvv.1.28}
	  
	% new div opening: depth here is 2
	
	  \bigskip
	  \begingroup
	  \large
	
	    
	    \stanza[\smallbreak]
	\label{pv.1.28}\edlabel{pv.1.28}\flagstanza{\tiny\textenglish{...v.1.28}}यथा विशेषेण विना विषयेन्द्रियसंहतिः ।&बुद्धेर्हेतुस्तंथेदं चेन्न तत्रापि विशेषतः ॥ २८ ॥\&[\smallbreak]


	
	  \endgroup
	

	  \pstart ननु {\color{DodgerBlue3}“यथा विशेषेण विना विषयेन्द्रियसंहति”}रक्षेपक्रियाधर्मिणी {\color{DodgerBlue3}“बुद्धेर्हेतु”}र्भवति1\leavevmode\marginnote{\textenglish{5b/MA}} {\color{DodgerBlue3}“तथेदमी”}श्वरादि वस्तुविशेष\edlabel{pvv.19-1}\footnote{\label{pvv.19-1}  १ पुनरनेकान्तत्वमिह ।} म्विना सहकारिसन्निधानेन कार्यं करोतीति {\color{DodgerBlue3}“चेत्”} । न (।) {\color{DodgerBlue3}“तत्रापि”} विषयेन्द्रियादिसंहतौ प्रागवस्थावदुपसर्प्पणप्रत्ययजनिताद्विज्ञानजननशक्तक्षणप्रज्ञालक्षणा{\color{DodgerBlue3}“द्विशेषात्”} । (२८)
	\pend
      \label{div_pvv.1.29}\edlabel{div_pvv.1.29}
	  
	% new div opening: depth here is 2
	

	  \pstart अन्य\edlabel{pvv.19-2}\footnote{\label{pvv.19-2}  २ नातिशयोत्पत्त्याऽपि तु संयोगं जनयन्ति सहिताः संयोगात्कार्यमित्याह यथाकार्यजननाय नालं तथा संयोगेपि ।}था (।)
	\pend
      
	  \bigskip
	  \begingroup
	  \large
	
	    
	    \stanza[\smallbreak]
	\label{pv.1.29}\edlabel{pv.1.29}\flagstanza{\tiny\textenglish{...v.1.29}}पृथक् पृथगशक्तानां स्वभावातिशयेऽसति ।&संहतावप्यसामर्थ्यं स्यात् सिद्धोऽतिशयस्ततः ॥ २९ ॥\&[\smallbreak]


	
	  \endgroup
	

	  \pstart {\color{DodgerBlue3}“पृथक् पृथगशक्तानां”} विषयेन्द्रियाणां {\color{DodgerBlue3}“स्वभावातिशयेऽसति संहतावप्य\edlabel{pvv.19-3}\footnote{\label{pvv.19-3}  ३ कार्यासामर्थ्यवत् ।} सामर्थ्यं स्यात्”} । ज्ञानार्जनम्प्रति स्वरूपाभेदात् । उत्पद्यते च ज्ञानं {\color{DodgerBlue3}“सिद्धोतिशयस्ततो”} ज्ञानोत्पादात् । (२९)
	\pend
      \label{div_pvv.1.30}\edlabel{div_pvv.1.30}
	  
	% new div opening: depth here is 2
	

	  \begin{center}%% label @type='head'
	\textbf{b. संहतौ हेतुता नेश्वरादौ}
	\end{center}
	
	  \bigskip
	  \begingroup
	  \large
	
	    
	    \stanza[\smallbreak]
	\label{pv.1.30}\edlabel{pv.1.30}\flagstanza{\tiny\textenglish{...v.1.30}}तस्मात् पृथगशक्तेषु येषु संभाव्यते गुणः ।&संहतौ हेतुता तेषां नेश्वरादेरभेदतः ॥ ३० ॥\&[\smallbreak]


	
	  \endgroup
	

	  \pstart {\color{DodgerBlue3}“तस्मात् पृथगशक्तेषु येषु सम्भाव्यते गुणः”} स्वरूपान्तरोत्पादलक्षणं {\color{DodgerBlue3}“संहतौ हेतुता तेषां”} क्षणिकानां {\color{DodgerBlue3}“नेश्वरादेर\edlabel{pvv.19-4}\footnote{\label{pvv.19-4}  ४ आदिना स्थिरात्मनां ग्रहः ।} भेदतः”} । ईश्वरप्रधानपुरुषादेरकारकाभिन्नस्वरूपत्वान्न हेतुत्वमित्यु\edlabel{pvv.19-5}\footnote{\label{pvv.19-5}  ५ तस्मात् स्थितमेव तत्र नित्यं प्रमाणमिति ।}पसंहारः । उक्तमीश्वरादिदूषणं ॥ (३०) ॥
	\pend
      
	  
	% new div opening: depth here is 1
	
\section[{२. भगवान् प्रमाणम्}]{२. भगवान् प्रमाणम्}

	  \begin{center}%% label @type='head'
	\textbf{(१) ज्ञानवत्वात्}
	\end{center}
	\label{div_pvv.1.31}\edlabel{div_pvv.1.31}
	  
	% new div opening: depth here is 2
	

	  \pstart भगवतोपि साधनाभावादप्रामाण्यं परमतेनाशङ्कते ।
	\pend
      \leavevmode\marginnote{\textenglish{020/s}}
	  \bigskip
	  \begingroup
	  \large
	
	    
	    \stanza[\smallbreak]
	\label{pv.1.31}\edlabel{pv.1.31}\flagstanza{\tiny\textenglish{...v.1.31}}प्रामाण्यञ्च परोक्षार्थज्ञानं तत्साधनस्य च ।&अभावात्; नास्त्यनुष्ठानमिति केचित् प्रचक्षते ॥ ३१ ॥\&[\smallbreak]


	
	  \endgroup
	

	  \pstart {\color{DodgerBlue3}“प्रामाण्यञ्च परोक्षार्थज्ञानं”}, न सर्व्वस्येष्यते । {\color{DodgerBlue3}“तत्साधनस्याभावात् । अनुष्ठा\edlabel{pvv.20-1}\footnote{\label{pvv.20-1}  १ यत्साधनानुष्ठानात्प्रामाण्यं ।}नं”} कस्यचिन्ना{\color{DodgerBlue3}“स्तीति”} कथन्तथाविधप्रमाणोपपत्तिरि{\color{DodgerBlue3}“ति केचित्”} जै मि नी याः {\color{DodgerBlue3}“प्रचक्षते”} । (३१)
	\pend
      \label{div_pvv.1.32}\edlabel{div_pvv.1.32}
	  
	% new div opening: depth here is 2
	

	  \pstart अत्राह ।
	\pend
      
	  \bigskip
	  \begingroup
	  \large
	
	    
	    \stanza[\smallbreak]
	\label{pv.1.32}\edlabel{pv.1.32}\flagstanza{\tiny\textenglish{...v.1.32}}ज्ञानवान् मृग्यते कश्चित् तदुक्तप्रतिपत्तये ।&अज्ञोपदेशकरणे विप्रलम्भनशङ्‏किभिः ॥ ३२ ॥\&[\smallbreak]


	
	  \endgroup
	

	  \pstart न खलु व्यसनितया प्रमाणमन्विष्यते प्रेक्षावद्‏भिरपि तु स्वर्ग्गापवर्गप्रधानपुरुषार्थं प्रेप्सुभिः तद्विषय{\color{DodgerBlue3}“ज्ञानवान् कश्चिदन्विष्यते तदुक्त”}स्योपायस्य {\color{DodgerBlue3}“प्रतिपत्तये”}ऽनुष्ठानार्थं {\color{DodgerBlue3}“अज्ञोपदेशकरणे विप्रलम्भनशङ्किभिः”} विसम्वादं सम्भावयद्‏भिः ॥ (३२)
	\pend
      \label{div_pvv.1.33}\edlabel{div_pvv.1.33}
	  
	% new div opening: depth here is 2
	
	  \bigskip
	  \begingroup
	  \large
	
	    
	    \stanza[\smallbreak]
	\label{pv.1.33}\edlabel{pv.1.33}\flagstanza{\tiny\textenglish{...v.1.33}}तस्मादनुष्ठेयगतं ज्ञानमस्य विचार्यताम् ।&कीटसंख्यापरिज्ञानं तस्य नः क्वोपयुज्यते ॥ ३३ ॥\&[\smallbreak]


	
	  \endgroup
	

	  \pstart {\color{DodgerBlue3}“दुःखोपशमोपायोपदेष्टुर्ज्ञानं मृग्यते यतस्तस्मादनुष्ठेयगतं संसारदुःखप्रशमोपायं ज्ञानमस्य प्रमाणपुरुषस्य विचार्यतां । अनुपयोगि कीटसंख्यापरिज्ञानन्तस्योपदेष्टुर्नोऽस्माकं न क्वचित्पुरुषार्थे उपयुज्यते इति न तद्विचार्यमिति प्रतिज्ञा प्रत्युपकाराद्यपेक्षासमं सर्वसत्वेषु। तस्माद्य”}देवं प्रेक्षावता{\color{DodgerBlue3}“मनुष्ठेयन्त”}द्विषयमुपदेष्टुं {\color{DodgerBlue3}“ज्ञानमु”}पयुक्तं नान्यविषयं । (३३)
	\pend
      \label{div_pvv.1.34}\edlabel{div_pvv.1.34}
	  
	% new div opening: depth here is 2
	

	  \begin{center}%% label @type='head'
	\textbf{(२) हेयोपादेयवेदकत्वात् न तु सर्ववेदकत्वात्}
	\end{center}
	
	  \bigskip
	  \begingroup
	  \large
	
	    
	    \stanza[\smallbreak]
	\label{pv.1.34}\edlabel{pv.1.34}\flagstanza{\tiny\textenglish{...v.1.34}}हेयोपादेयतत्त्वस्य साभ्युपायस्य वेदकः ।&यः प्रमाणमसाविष्टो न तु सर्वस्य वेदकः ॥ ३४ ॥\&[\smallbreak]


	
	  \endgroup
	

	  \pstart तस्माद्धेयतत्त्वस्य दुःखसत्यस्य {\color{DodgerBlue3}“साभ्युपायस्य”} समुदयसत्यान्वित{\color{DodgerBlue3}“स्योपादेयतत्त्वस्य”} निरोधसत्यस्य {\color{DodgerBlue3}“साभ्युपायस्य”} मार्ग्गसत्यसहितस्य प्रमाणपरिशुद्धस्य यो वेदकः {\color{DodgerBlue3}“स प्रमाणमिष्टो न तु सर्व्वस्य”} यस्य कस्यचिद्वि{\color{DodgerBlue3}“वेदकः”} । न खलु सकलज्ञानादार्यसत्यचतुष्टयदेशनाऽपि तु तज्ज्ञानत्वात् तदुपदेष्टृतयैव च प्रामाण्यमिष्यते (। ३४)
	\pend
      \label{div_pvv.1.35}\edlabel{div_pvv.1.35}
	  
	% new div opening: depth here is 2
	

	  \pstart तदेवाह ॥
	\pend
      \leavevmode\marginnote{\textenglish{021/s}}
	  \bigskip
	  \begingroup
	  \large
	
	    
	    \stanza[\smallbreak]
	\label{pv.1.35}\edlabel{pv.1.35}\flagstanza{\tiny\textenglish{...v.1.35}}दूरं पश्यतु वा मा वा तत्त्वमिष्टन्तु पश्यतु ।&प्रमाणं दूरदर्शी चेदेत गृध्रानुपास्महे ॥ ३५ ॥\&[\smallbreak]


	
	  \endgroup
	

	  \pstart {\color{DodgerBlue3}“दू\edlabel{pvv.21-1}\footnote{\label{pvv.21-1}  १ अतीन्द्रियं सर्व्वातीन्द्रियं ।} रं पश्यतु वा मा वा द्राक्षीत्तत्वमिष्ट”}न्त्वार्यसत्यचतुष्टयं {\color{DodgerBlue3}“पश्यति”} तावतैव भगवान् {\color{DodgerBlue3}“प्रमाण”}मन्यथाऽतत्त्वदर्श्यपि प्रमाणं {\color{DodgerBlue3}“दूरदर्शी चेदिष्यते । एतागच्छत मुमुक्षवो गृध्रानुपास्महे”} दूरदर्शिनो दीर्घश्रुती\edlabel{pvv.21-2}\footnote{\label{pvv.21-2}  २ आकाशगान् तिरश्चः परचित्तानुसारिणः क्षणिकादीन् ।}श्च (? न् च) वराहानित्युपहसति । उक्तमभिमतं प्रामा\edlabel{pvv.21-3}\footnote{\label{pvv.21-3}  ३ पुरुषार्थज्ञत्वेन । यतः सत्यावबोधाद्धर्मचक्रादौ भगवान् सार्वज्ञं प्रतिज्ञातवात् ।}ण्यं (३५) ॥
	\pend
      \label{div_pvv.1.36}\edlabel{div_pvv.1.36}
	  
	% new div opening: depth here is 2
	

	  \begin{center}%% label @type='head'
	\textbf{(३) कारुणिकत्वात् प्रमाणम्}
	\end{center}
	

	  \begin{center}%% label @type='head'
	\textbf{क. जन्मान्तरसिद्धिः}
	\end{center}
	

	  \begin{center}%% label @type='head'
	\textbf{(करुणा जन्मान्तराभ्यासात्)}
	\end{center}
	

	  \pstart नन्वीदृशस्य प्रमाणस्य किं साधनमित्याह (।)
	\pend
      
	  \bigskip
	  \begingroup
	  \large
	
	    
	    \stanza[\smallbreak]
	\label{pv.1.36}\edlabel{pv.1.36}\flagstanza{\tiny\textenglish{...v.1.36}}साधनं करुणाभ्यासात् सा बुद्धेर्देहसंश्रयात् ।&असिद्धाऽभ्यास इति चेन्नाश्रयप्रतिषेधतः ॥ ३६ ॥\&[\smallbreak]


	
	  \endgroup
	

	  \pstart {\color{DodgerBlue3}“साधनं करुणा”} दुःखादुःखहेतोश्च समुद्धरणकामता करुणा सा भगवतः प्रामाण्यस्य साधनं । सैव करुणा इत्यत आह । {\color{DodgerBlue3}“अभ्यासात्सा”} गौत्रविशेषात्कल्याणमित्रसंसर्ग्गादनुशयदर्शनाच्च कश्चिन्महासत्त्वः कृपायामुपजातस्पृहः सा\edlabel{pvv.21-4}\footnote{\label{pvv.21-4}  ४ वृत्तिस्तु “निरुपद्रवभूतार्थस्वभावस्ये”\href{http://http://sarit.indology.info/?cref=pv.1.212}{(प्र॰१।२१२)}त्यत्रोक्तो ।} दरनिरन्तरानेकजन्मपरम्पराप्रभवाभ्यासेन सात्मीभूतकृपया प्रेर्यमाणः सर्व्वसत्त्वानां समुदयहान्या दुःखहानाय मार्गभावनया निरोधप्रापणाय च देशनां कर्तुकामः स्वयमसाक्षात्कृतस्य देशनायां विप्रलम्भसम्भावनाच्चतुरार्यसत्यानि साक्षात् करोतीति भगवति साधनं कृपा प्रामाण्यस्य । {\color{DodgerBlue3}“बुद्धेर्देहसंश्रयादसिद्धोऽभ्यास इति चेत्”} बुद्धिर्देहमाश्रिता कार्यत्वात् प्रदीपमिव प्रभा । शक्तिरूपत्वाद्वा मद्यमिव मदशक्तिः । गुणत्वाद्वा पटमिव शुक्लता । त्रेधाप्याश्रयविनाशे तस्य नाशात् । कुतो जन्मान्तराणि कथम्वा तेष्वभ्यासः कृपादेरिति चा र्व्वा काः ।
	\pend
      \leavevmode\marginnote{\textenglish{022/s}}

	  \pstart तदेतन्न युक्तमा{\color{DodgerBlue3}“श्रयप्रतिषेधतो”} बुद्धेः । बुद्धिरहितः कायो ना\edlabel{pvv.22-1}\footnote{\label{pvv.22-1}  १ कारणत्वेनाश्रयो न ।}श्रयः कारणत्वात् गुणित्वात् शक्तिमत्त्वाद्वा । (३६)
	\pend
      \label{div_pvv.1.37}\edlabel{div_pvv.1.37}
	  
	% new div opening: depth here is 2
	

	  \begin{center}%% label @type='head'
	\textbf{(क) भूतचैतन्यमतनिरासः\edlabel{pvv.22-asterisk}[[* द्रष्टव्यं परिशिष्टं १।९,३७]]}
	\end{center}
	

	  \begin{center}%% label @type='head'
	\textbf{बीजपक्षनिरासः}
	\end{center}
	

	  \begin{center}%% label @type='head'
	\textbf{a. तत्र कारणत्वं प्रतिषेद्धुमाह ।}
	\end{center}
	
	  \bigskip
	  \begingroup
	  \large
	
	    
	    \stanza[\smallbreak]
	\label{pv.1.37}\edlabel{pv.1.37}\flagstanza{\tiny\textenglish{...v.1.37}}प्राणापानेन्द्रियधियां देहादेव न केवलात् ।&सजातिनिरपेक्षाणां जन्म जन्मपरिग्रहे ॥ ३७ ॥\&[\smallbreak]


	
	  \endgroup
	
	  \bigskip
	  \begingroup
	  \large
	
	    
	    \stanza[\smallbreak]
	\label{pv.1.38a}\edlabel{pv.1.38a}\flagstanza{\tiny\textenglish{....1.38a}}अतिप्रसङ्गात्;\&[\smallbreak]


	
	  \endgroup
	

	  \pstart {\color{DodgerBlue3}“प्राणो”} वायुरूद्‏र्ध्वग तद्विपरीतो{\color{DodgerBlue3}“ऽपानः इन्द्रि”}याणि चक्षुरादीनि । {\color{DodgerBlue3}“धी”}र्बुद्धिः तासां सजातिनिरपेक्षाणां कारणभूतपूर्व्व{\color{DodgerBlue3}“सजाति”}प्राणादिपुञ्ज{\color{DodgerBlue3}“निरपेक्षाणां देहादेव \leavevmode\marginnote{\textenglish{6a/MA}} केवलान्न जन्म”} भवति (।) कुत इत्याह (।) {\color{DodgerBlue3}“जन्मपरिग्रहे”}ऽति (प्र) सङ्गात् (।) यदि महाभूतेभ्य एव केवलेभ्यः प्राणादीनां जन्मग्रहस्तदा सर्व्वस्माद् भवेयुरिति सर्वं प्राणिमयं जगत् स्यात् (।) न चास्त्येतत्तस्मात्पूर्व्वसजातिसापेक्षाणामेवाक्षादीनां देहाज्जन्मेति पूर्व्वजन्मप्रतिबन्धसिद्धिः । (३७)
	\pend
      \label{div_pvv.1.38}\edlabel{div_pvv.1.38}
	  
	% new div opening: depth here is 2
	
	  \bigskip
	  \begingroup
	  \large
	
	    
	    \stanza[\smallbreak]
	\label{pv.1.38b}\edlabel{pv.1.38b}\flagstanza{\tiny\textenglish{....1.38b}}भाविजन्मपरम्परासिद्ध्यर्थमप्याह (।)\&[\smallbreak]


	
	  \endgroup
	
	  \bigskip
	  \begingroup
	  \large
	
	    
	    \stanza[\smallbreak]
	\label{pv.1.38c}\edlabel{pv.1.38c}\flagstanza{\tiny\textenglish{....1.38c}}यद् दृष्टं प्रतिसन्धानशक्तिमत् ।&किमासीत् तस्य यन्नास्ति पश्चाद् येन न सन्धिमत् ॥ ३८ ॥\&[\smallbreak]


	
	  \endgroup
	

	  \pstart यत्प्राणापानादिमध्यावस्थायां\edlabel{pvv.22-2}\footnote{\label{pvv.22-2}  २ वर्तमानायां । तदपि प्राणादीनां सत्त्वात् पूर्व्वावस्थावत् नाविकलकारणो न सन्धत्ते । प्राणापानेन्द्रियधियां ।} प्राणादीनामुत्पादनशक्तियुक्तं दृष्टं {\color{DodgerBlue3}“तस्य किमासीत् प्रतिसन्धान”}कालेऽधिकं {\color{DodgerBlue3}“यत् पश्चा”}न्मरणका\edlabel{pvv.22-3}\footnote{\label{pvv.22-3}  ३ कुतस्तद्देहनाशे बुद्धेरभावः ।}ले {\color{DodgerBlue3}“नास्ति\edlabel{pvv.22-4}\footnote{\label{pvv.22-4}  ४ तदापि प्राणादीनां सत्त्वात् पूर्व्वावस्थावत् नाविकलकारणो न संधत्ते ।} येन”} तद्वैकल्यात्तदा {\color{DodgerBlue3}“न सन्धि”}मत्समग्राप्रतिबद्धकारणत्वात् प्रतिसन्धानं प्राप्तमित्यर्थः । (३८)
	\pend
      \label{div_pvv.1.39}\edlabel{div_pvv.1.39}
	  
	% new div opening: depth here is 2
	

	  \pstart b. स्यादेतत् (।) पूर्व्वसजातिहेतुकमिन्द्रियादि न भवति किन्तु देहहेतुकमेवेदं न चातिप्रसङ्गः केषाञ्चिदेव भूतपरिणामानां देहात्मकानां तद्धेतुत्वात् । अन्येषाञ्च तद्विरुद्धस्वभावानामहेतुत्वात् सुवर्ण्णबीजाबीजपाषाणवत् (।) अत्राह ।
	\pend
      \leavevmode\marginnote{\textenglish{023/s}}
	  \bigskip
	  \begingroup
	  \large
	
	    
	    \stanza[\smallbreak]
	\label{pv.1.39}\edlabel{pv.1.39}\flagstanza{\tiny\textenglish{...v.1.39}}न स कश्चित् पृथिव्यादेरंशो यत्र न जन्तवः ।&संस्वेदजाद्या जायन्ते सर्वं बीजात्मकं ततः ॥ ३९ ॥\&[\smallbreak]


	
	  \endgroup
	

	  \pstart {\color{DodgerBlue3}“न स कश्चित्पृथिव्यादेरंशः”} प्रदेशो {\color{DodgerBlue3}“यत्र जन्तवः संस्वेदजाद्या”} आद्यशब्दाज्जरायुजाण्डजप्रभृतयो {\color{DodgerBlue3}“न जायन्ते (।) ततः सर्वं”} भूतपरिणतिजातं प्राणादिजनने {\color{DodgerBlue3}“बीजात्मक”}मिति नास्ति बीजविरुद्धस्वभावता कस्यचित् । सुवर्ण्णासुवर्ण्णबीजत्वन्तु पाषाणादीनां सुवर्ण्णपरमाणूनामेव सदसत्त्वाभ्यामिति विषमो दृष्टान्तः । (३९)
	\pend
      \label{div_pvv.1.40}\edlabel{div_pvv.1.40}
	  
	% new div opening: depth here is 2
	

	  \pstart ननु भूतमात्रहेतुकत्वाविशेषेपि परिणामस्य यथा विशेषः सुवर्ण्णपरमाणुमयत्वेतराभ्यान्तथा प्राणादिहेतुत्वाहेतुत्वाभ्यामपि स्यात् ।
	\pend
      

	  \pstart उक्तमेवात्र सर्व्वत्र प्राणिनां दृष्टेः । सर्व्व तद्बीजात्मकमिति नास्त्येवाबीजात्मकता कस्यचित्परिणामस्य । अत्रापि वा तुल्यः प्रसङ्गः । यदि भूतमात्रहेतुकोऽयं बीजपरिणामः सर्व्वस्तथा स्यात् हेत्वविशेषे कार्यविशेषायोगात् ।
	\pend
      

	  \begin{center}%% label @type='head'
	\textbf{ii. शक्तिमत्त्वनिरासः}
	\end{center}
	

	  \pstart a. ननु भूतान्यप्यवान्तरानेकविविधविशेषभाञ्जि विचित्राः परिणतीर्जनयन्तीति न समानताप्रसङ्गः । न तावत्सविशेषो भूतमात्रात् सर्व्वत्र प्रसङ्गात् । न चान्यतोऽन्यस्याभावात् । अस्माकन्तु कर्मापि सहकारि सम्मतं तद्वैचित्र्यात् विचित्रं कार्यजातमुचितं । शक्तिपक्षं निषेद्धुमाह ।
	\pend
      
	  \bigskip
	  \begingroup
	  \large
	
	    
	    \stanza[\smallbreak]
	\label{pv.1.40}\edlabel{pv.1.40}\flagstanza{\tiny\textenglish{...v.1.40}}तत् स्वजात्यनपेक्षाणामक्षादीनां समुद्भवे ।&परिणामो यथैकस्य स्यात् सर्वस्याविशेषतः ॥ ४० ॥\&[\smallbreak]


	
	  \endgroup
	

	  \pstart तत्तस्माद् भूतमात्रहेतुत्वात् {\color{DodgerBlue3}“स्वजात्यनपेक्षाणामक्षादीनां समुद्भवे”} स्वीक्रियमाणे {\color{DodgerBlue3}“परिणामो यथैकस्य”} देहस्यातद्भूतप्राणापानेन्द्रियचैतन्यशक्तितया {\color{DodgerBlue3}“तथा सर्व्वस्य”} लोष्टादेरपि स्याद्धेतोरविशेषतः । अत एव हि प्रसङ्गाद् गुणपक्षोपि प्रतिक्षिप्तो बोद्धव्यः । तस्मादाश्रयत्वमपि नास्त्यनिन्द्रियस्य कायस्य कारणत्वशक्तिमत्वे गुण\edlabel{pvv.23-1}\footnote{\label{pvv.23-1}  १ चार्वाको नित्यानि भूतान्येव परिणत्याद्भुतचैतन्यानि सत्वव्यपदेशभाञ्जि । तान्येव विरोधिप्रत्ययादनभिव्यक्तचैतन्यानीत्युक्तेः ।}त्वानामयोगतः । (४०)
	\pend
      \label{div_pvv.1.41}\edlabel{div_pvv.1.41}
	  
	% new div opening: depth here is 2
	
	  \bigskip
	  \begingroup
	  \large
	
	    
	    \stanza[\smallbreak]
	\label{pv.1.41a}\edlabel{pv.1.41a}\flagstanza{\tiny\textenglish{....1.41a}}b. सेन्द्रियोपि कायो बुद्धेराश्रयो न युक्त इति वक्तुमाह ।&प्रत्येकमुपघातेऽपि नेन्द्रियाणां मनोमतेः ।&उपघातोस्ति;\&[\smallbreak]


	
	  \endgroup
	\leavevmode\marginnote{\textenglish{024/s}}

	  \pstart {\color{DodgerBlue3}“इन्द्रियाणां प्रत्येकं”} यथास्वं प्रत्ययैरु\edlabel{pvv.24-1}\footnote{\label{pvv.24-1}  १ प्रसुप्तिर्नाम व्याधिनाऽन्तर्ब्बहिः कायेन्द्रियेपि नाशिते बुद्धेर्भावात् ?} {\color{DodgerBlue3}“पघातेपि”}\edlabel{pvv.24-2}\footnote{\label{pvv.24-2}  २ नानिन्द्रियो भूतदेहे केशनखाग्रादौ चैतन्यप्रसङ्गात् ।} सति {\color{DodgerBlue3}“मनो”}मतेर्व्विकल्पबुद्धेरुपघात (ो) पटुमन्दतादिलक्षणो नास्ति (।) यद्धि यदाश्रितं तत्तद्विकारे विक्रियते यथा घटस्य दाहादौ तच्छुक्लत्वादि(।) तस्मान्न तदाश्रिता बुद्धिः ।
	\pend
      

	  \pstart विपर्ययमेवाख्यातुमाह ।
	\pend
      
	  \bigskip
	  \begingroup
	  \large
	
	    
	    \stanza[\smallbreak]
	\label{pv.1.41b}\edlabel{pv.1.41b}\flagstanza{\tiny\textenglish{....1.41b}}भङ्गेऽस्यास्तेषां भङ्गश्च दृश्यते ॥ ४१ ॥\&[\smallbreak]


	
	  \endgroup
	

	  \pstart भयशोकादिभिरस्या मनोबुद्धे{\color{DodgerBlue3}“र्भङ्गे”}\edlabel{pvv.24-3}\footnote{\label{pvv.24-3}  ३ यस्मिँस्थिते यन्निवर्तते तत्ततो भिन्नं यथोदकेऽग्निः (।) स्थिते मृतशरीरे निवर्त(?र्त्य) ते प्राणादिभिरिति स्वभावहेतुः । यत्र पूर्व्वस्थिते प्रागविद्यमानं तत्र पश्चात् भवति तत्ततो भिन्नं तद्यथा पूर्व्वस्थिते घटे प्रागसन् पश्चात्क्रियमाणो दीपः (।) पूर्व्वस्थितेषु भूतेषु च तत्राविद्यमानाः स्युः प्राणादय इति स्वभावः पूर्व्वापरजन्मसाधनद्वयं ।} (नाशे) विकारे सति {\color{DodgerBlue3}“तेषा”}मिन्द्रियाणां {\color{DodgerBlue3}“भङ्गो”} विकार{\color{DodgerBlue3}“श्च दृश्यते”} । (४१)
	\pend
      \label{div_pvv.1.42}\edlabel{div_pvv.1.42}
	  
	% new div opening: depth here is 2
	
	  \bigskip
	  \begingroup
	  \large
	
	    
	    \stanza[\smallbreak]
	\label{pv.1.42}\edlabel{pv.1.42}\flagstanza{\tiny\textenglish{...v.1.42}}तस्मात् स्थित्याश्रयो बुद्धेर्बुद्धिमेव समाश्रितः ।&कश्चिन्निमित्तमक्षाणां तस्मादक्षाणि बुद्धितः ॥ ४२ ॥\&[\smallbreak]


	
	  \endgroup
	

	  \pstart हर्षादिना च परिपुष्टिरिति विकल्पबुद्धिविकारविकारित्वादिन्द्रियाण्येव तदाश्रितानि (।) तस्माद् {\color{DodgerBlue3}“बुद्धेः स्थि”}तेराश्रयः समानजातीयः {\color{DodgerBlue3}“कश्चिन्न”} सेन्द्रियः\edlabel{pvv.24-4}\footnote{\label{pvv.24-4}  ४ पूर्व्वक्षणो बुद्धेः ।} कायः । स तर्हि तदाश्रितो भविष्यति न स च {\color{DodgerBlue3}“बुद्धिमेव समाश्रितः । निमि\edlabel{pvv.24-5}\footnote{\label{pvv.24-5}  ५ जनकश्च ।}त्तञ्चाक्षाणां तस्मादक्षाणि बुद्धितो”} न बुद्धिस्तेभ्यः । (४२)
	\pend
      \label{div_pvv.1.43}\edlabel{div_pvv.1.43}
	  
	% new div opening: depth here is 2
	

	  \begin{center}%% label @type='head'
	\textbf{(ख) विज्ञानसिद्धिः}
	\end{center}
	

	  \begin{center}%% label @type='head'
	\textbf{I. अन्वयतः}
	\end{center}
	

	  \begin{center}%% label @type='head'
	\textbf{A. प्रतिसन्धिः}
	\end{center}
	

	  \begin{center}%% label @type='head'
	\textbf{a. कायाश्रयो बुद्धिः}
	\end{center}
	

	  \begin{center}%% label @type='head'
	\textbf{b. प्रतिसन्ध्याक्षेपिका बुद्धिः}
	\end{center}
	
	  \bigskip
	  \begingroup
	  \large
	
	    
	    \stanza[\smallbreak]
	\label{pv.1.43a}\edlabel{pv.1.43a}\flagstanza{\tiny\textenglish{....1.43a}}यादृश्याक्षेपिका साऽसीत् पश्चादप्यस्तु तादृशी ।\&[\smallbreak]


	
	  \endgroup
	

	  \pstart ततो जन्मादौ {\color{DodgerBlue3}“यादृशी”} दृष्टाऽत्मग्रहयोगिनी बुद्धिरा{\color{DodgerBlue3}“क्षेपिका”} बुद्धीन्द्रियादीनामासीत् {\color{DodgerBlue3}“पश्चादपि”} मरणावस्थायामपि {\color{DodgerBlue3}“तादृ\edlabel{pvv.24-6}\footnote{\label{pvv.24-6}  ६ अविकलकारणत्वात् ।} श्या”}क्षेपिका भवतु । जन्मान्तरस्य \leavevmode\marginnote{\textenglish{025/s}} शरीरा\edlabel{pvv.25-1}\footnote{\label{pvv.25-1}  १ सिद्धोऽभ्यासो जन्मान्तरसाधनात् । नापि बुद्धेर्देह आश्रयो येन देहनाशे बुद्धिनाशादभ्यासो न स्यात् ।}न्तरसम्बद्धबुद्धीन्द्रियाद्युत्पादनलक्षणस्येति पूर्व्वोक्तनिगमनं ॥
	\pend
      

	  \begin{center}%% label @type='head'
	\textbf{C. कायाश्रितं मनोविज्ञानम्}
	\end{center}
	

	  \pstart न\edlabel{pvv.25-2}\footnote{\label{pvv.25-2}  २ आगमविरोधमाह चार्व्वाकः ।}नु कायाश्रितत्वं मनसोप्युक्तं भगवताऽन्योन्यानु\edlabel{pvv.25-3}\footnote{\label{pvv.25-3}  ३ अन्योन्यबीजकं ह्येतद्द्वयं यदुत सेन्द्रियत्वं कत्यो विज्ञानञ्चेत्यभिधर्मो ।}विधायित्वं कायमनसो-\leavevmode\marginnote{\textenglish{6b/MA}} रिति वदता । तत्कथमित्याह (।)
	\pend
      
	  \bigskip
	  \begingroup
	  \large
	
	    
	    \stanza[\smallbreak]
	\label{pv.1.43b}\edlabel{pv.1.43b}\flagstanza{\tiny\textenglish{....1.43b}}तज्ज्ञानैरुपकार्यत्वादुक्तं कायाश्रितं मनः ॥ ४३ ॥\&[\smallbreak]


	
	  \endgroup
	

	  \pstart {\color{DodgerBlue3}“तज्ज्ञानैः”} कायविषयैर्ज्ञानैरूपादिग्राहि{\color{DodgerBlue3}“भिरुपकार्यत्वात्”} मनसः सुखोत्साहादिरूपे{\color{DodgerBlue3}“णोक्तं”} भ ग व ता {\color{DodgerBlue3}“कायाश्रितं मनो”}विज्ञानं न साक्षात्तदुत्पत्तेः । (४३)
	\pend
      \label{div_pvv.1.44}\edlabel{div_pvv.1.44}
	  
	% new div opening: depth here is 2
	

	  \begin{center}%% label @type='head'
	\textbf{d. अन्योन्यहेतुके कायमनसी}
	\end{center}
	

	  \pstart न\edlabel{pvv.25-4}\footnote{\label{pvv.25-4}  ४ यतः समस्तव्यस्तेन्द्रियघातेपि न मनोघातः ।} न्विन्द्रियाणि विना न बुद्धि\edlabel{pvv.25-5}\footnote{\label{pvv.25-5}  ५ चक्षुरादिज्ञानपञ्चकं बुद्धिः । यथेन्द्रियधीरकल्पं रूपादि गृह्णाति तथा मनोपि गृह्णीयात् कायेन जनितत्वात्सेन्द्रियेण ।} रिति युक्तमाश्रयत्वमेषामित्याह ।
	\pend
      
	  \bigskip
	  \begingroup
	  \large
	
	    
	    \stanza[\smallbreak]
	\label{pv.1.44}\edlabel{pv.1.44}\flagstanza{\tiny\textenglish{...v.1.44}}यद्यप्यक्षैर्विना बुद्धिर्न्न तान्यपि तया विना ।&तथाप्यन्योन्यहेतुत्वं ततोऽप्यन्योन्यहेतुके ॥ ४४ ॥\&[\smallbreak]


	
	  \endgroup
	

	  \pstart {\color{DodgerBlue3}“यद्यप्यक्षै”}र्विना बुद्धिर्न भवति । {\color{DodgerBlue3}“तान्यप्य”}क्षाणि {\color{DodgerBlue3}“तया विना न”} भवन्ति । भूतमात्रादुत्पत्तेः सर्व्वस्मादुत्पत्तिप्रसङ्गादित्युक्तेः । {\color{DodgerBlue3}“तथाप्यन्योन्यहेतुत्वमुक्तम्भवति । ततोप्यन्योन्यहेतुके”} कायमनसी मध्यावस्थावत् अनादितादृक्‏प्रवाहवती इति सिद्धः परलोकः ।(४४)
	\pend
      \label{div_pvv.1.45}\edlabel{div_pvv.1.45}
	  
	% new div opening: depth here is 2
	

	  \pstart कि\edlabel{pvv.25-6}\footnote{\label{pvv.25-6}  ६ यदा कातरस्यान्यं शोणितादिविकृतं दृष्ट्वैन्द्रियज्ञानं विकृतं मनो विकारयति । यदि नित्यात्कायादेव प्राणादयस्तदा ।}ञ्च ।
	\pend
      
	  \bigskip
	  \begingroup
	  \large
	
	    
	    \stanza[\smallbreak]
	\label{pv.1.45}\edlabel{pv.1.45}\flagstanza{\tiny\textenglish{...v.1.45}}नाक्रमात् क्रमिणो भावो नाप्यपेक्षाऽविशेषिणः ।&क्रमाद् भवन्ती धीः कायात् क्रमं तस्यापि शंसति ॥ ४५ ॥\&[\smallbreak]


	
	  \endgroup
	\leavevmode\marginnote{\textenglish{026/s}}

	  \pstart {\color{DodgerBlue3}“नाक्रमात् क्रमिणः”}\edlabel{pvv.26-1}\footnote{\label{pvv.26-1}  १ क्षणिकस्य ।} कार्यस्य {\color{DodgerBlue3}“भावः”} । क्रमरहितत्वात् कारणस्य । त\edlabel{pvv.26-2}\footnote{\label{pvv.26-2}  २ सर्दथा विशेषेण हेतुरिति चेत् इत्यस्य न नियमः ।} न्निष्पा\edlabel{pvv.26-3}\footnote{\label{pvv.26-3}  ३ इन्द्रियविज्ञानादीनि ।} द्यानि कार्याणि सकृज्जायेरन् । क्रमवतः सहकारिणोऽ{\color{DodgerBlue3}“पेक्ष्य”} क्रमाज्जनयिष्यतीति चेत् {\color{DodgerBlue3}“नाप्यविशेषिणः”} स्थिरैकरूपस्य परैरनाधेयविशेषस्य परेषां सहकारिणामपेक्षाऽस्ति । तस्मा{\color{DodgerBlue3}“त्क्रमाद् भवन्ती धीः कायात्क्रमन्तस्यापि”} कायस्य शंसति\edlabel{pvv.26-4}\footnote{\label{pvv.26-4}  ४ अनुमापयति ।}। (४५)
	\pend
      \label{div_pvv.1.46}\edlabel{div_pvv.1.46}
	  
	% new div opening: depth here is 2
	

	  \pstart ततश्च (।)
	\pend
      
	  \bigskip
	  \begingroup
	  \large
	
	    
	    \stanza[\smallbreak]
	\label{pv.1.46}\edlabel{pv.1.46}\flagstanza{\tiny\textenglish{...v.1.46}}प्रतिक्षणमपूर्वस्य पूर्वः पूर्वः क्षणो भवेत् ।&तस्य हेतुरतो हेतुर्दृष्ट एवास्तु सर्वदा ॥ ४६ ॥\&[\smallbreak]


	
	  \endgroup
	

	  \pstart {\color{DodgerBlue3}“प्रतिक्षणमपूर्व्वस्य”} बुद्धीन्द्रियकायसमुदायस्य कार्यस्य पूर्व्वः {\color{DodgerBlue3}“पूर्व्वः क्षण”}स्तादृशो {\color{DodgerBlue3}“हेतुर्भवेदतः”} कारणादनन्तरस्य बुद्धीन्द्रियादेर्हेतुर्मध्यावस्थावद् {\color{DodgerBlue3}“दृष्ट एव”} बुद्धीन्द्रियादिकलापः {\color{DodgerBlue3}“सर्व्वदा”} ऐहिकजन्मादौ वामुत्रिकजन्मादौ चा{\color{DodgerBlue3}“स्तु”} । (४६)
	\pend
      \label{div_pvv.1.47}\edlabel{div_pvv.1.47}
	  
	% new div opening: depth here is 2
	

	  \begin{center}%% label @type='head'
	\textbf{e. चित्तान्तरेण सन्धानम्}
	\end{center}
	

	  \pstart ननु यदि सविज्ञानकायत्वात् तथाभूतजननानुमानेन परलोकसिद्धिः तदा मरणचित्तत्वाच्चित्तान्तराप्रतिसन्धानमर्हच्चरमचित्तवदिति कस्मान्नानुमीयत इत्याह (।)
	\pend
      
	  \bigskip
	  \begingroup
	  \large
	
	    
	    \stanza[\smallbreak]
	\label{pv.1.47}\edlabel{pv.1.47}\flagstanza{\tiny\textenglish{...v.1.47}}चित्तान्तरस्य सन्धाने को विरोधोन्त्यचेतसः ।&तद्वदप्यर्हतश्चित्तं असन्धानं कुतो मतम् ॥ ४७ ॥\&[\smallbreak]


	
	  \endgroup
	

	  \pstart {\color{DodgerBlue3}“चित्तान्तरस्य सन्धाने को विरोधोऽन्त्यचेतसो”} मरणचित्तस्य न कश्चित् । तथा हि(।)न तावन्मरणचित्तेन चित्तान्तरसन्धानस्य सहा\edlabel{pvv.26-5}\footnote{\label{pvv.26-5}  ५ प्रीतिदौर्मनस्यादित्वेन भिन्ना प्रतिक्षणं वेद्यमाना बुद्धिर्यदि नित्येष्यते तदातत्त्वान्तरं पञ्चमं स्यात् आलोकान्धकारवत् ।}नवस्थानलक्षणो विरोधः निवर्त्त्यनिवर्तकत्वाभावात् । नापि परस्परपरिहारस्थितिलक्षणः\edlabel{pvv.26-6}\footnote{\label{pvv.26-6}  ६ भावाभाववत् ! मनोज्ञानं ।}  अम\edlabel{pvv.26-7}\footnote{\label{pvv.26-7}  ७ न तु संतानचित्त (?) व्यवच्छेदेन ।} रणचित्तव्यवच्छेदेन मरणचित्तस्यावस्थानात् । {\color{DodgerBlue3}“तद्वदप्यर्हतश्चित्तमसन्धानं कुतः”} प्रमाणात् {\color{DodgerBlue3}“मतं”} येन दृष्टान्तः स्यात् । न ह्यर्हन् भवतां\edlabel{pvv.26-8}\footnote{\label{pvv.26-8}  ८ चार्व्वाकाणाम् ।}सिद्धः । तद्‏बाधनाय यत्नात् । (४७)
	\pend
      \label{div_pvv.1.48}\edlabel{div_pvv.1.48}
	  
	% new div opening: depth here is 2
	

	  \pstart अथाभ्युपगम्यते तदा तस्य क्लेशविसंयोगकृतमसन्धानं नान्यथा स च न पृथग्‏जनानामिति कथन्तेषां मरणचित्तमसन्धानं । क्लेशविसंयोगो हि प्रतिसन्धानविरोधी लक्ष्यते न मरणचित्तं ॥\edlabel{pvv.26-9}\footnote{\label{pvv.26-9}  ९ अनन्यसत्त्वनेयस्येत्यादिना प्रतिसन्धानं स्नेहात् वक्ष्यते न सोऽर्हतः । भाविकारणानुपलब्धिः । एवं मन्यते न परेण ।}
	\pend
      \leavevmode\marginnote{\textenglish{027/s}}

	  \pstart स्यादेतद्(।) भवत्सिद्धान्तसिद्धमर्हच्चित्तमसन्धानन्तत एव दृष्टान्तसिद्धिः । अत्राह (।)
	\pend
      
	  \bigskip
	  \begingroup
	  \large
	
	    
	    \stanza[\smallbreak]
	\label{pv.1.48}\edlabel{pv.1.48}\flagstanza{\tiny\textenglish{...v.1.48}}असिद्धार्थः प्रमाणेन किं सिद्धान्तोऽनुगम्यते ॥&हेतोर्वैकल्यतस्तच्चेत् किन्तदेवाऽत्र नोदितम् ॥ ४८ ॥\&[\smallbreak]


	
	  \endgroup
	

	  \pstart {\color{DodgerBlue3}“असिद्धार्थो”}ऽनिश्चितार्थः {\color{DodgerBlue3}“प्रमाणेन किं सिद्धान्तोऽनुगम्यते”} तदनुरोधेन परलोकस्याभ्युपगमप्रसङ्गात् । अथ {\color{DodgerBlue3}“हेतो”}राश्वासप्रश्वासेन्द्रियपाट\edlabel{pvv.27-1}\footnote{\label{pvv.27-1}  १ अर्हतश्चित्तस्य हेतोः ।} वादे{\color{DodgerBlue3}“र्वैकल्यतो भरणचित्तस्य”} तदप्रतिसन्धानमिति चेत् । {\color{DodgerBlue3}“किन्तदेव”} हेतुवैक\edlabel{pvv.27-2}\footnote{\label{pvv.27-2}  २ मरणचित्तत्वादिति हित्वा ।} ल्य\edlabel{pvv.27-3}\footnote{\label{pvv.27-3}  ३ अत्र रथ्यापुरुषे ।} मत्राप्रतिसन्धाने साधनत्वे {\color{DodgerBlue3}“नोदितं”} । दृष्टान्तविकलन्तु मरणचित्तमुक्तं । आश्वासादिहेतुत्वनिषेधञ्च वक्ष्यति । (४८)
	\pend
      \label{div_pvv.1.49}\edlabel{div_pvv.1.49}
	  
	% new div opening: depth here is 2
	

	  \begin{center}%% label @type='head'
	\textbf{f. कायस्याहेतुत्वम्}
	\end{center}
	
	  \bigskip
	  \begingroup
	  \large
	
	    
	    \stanza[\smallbreak]
	\label{pv.1.49}\edlabel{pv.1.49}\flagstanza{\tiny\textenglish{...v.1.49}}तद्धीवद् ग्रहणप्राप्तेर्मनोज्ञानं न सेन्द्रियात् ।&ज्ञानोत्पादनसामर्थ्यभेदान्न सकलादपि ॥ ४९ ॥\&[\smallbreak]


	
	  \endgroup
	

	  \pstart कायश्च हेतुर्भवन् सेन्द्रियो वा स्यादनिन्द्रियो वा । सेन्द्रियोपि प्रत्येकमिन्द्रियैः सहितः समस्तैर्व्वा (।) तत्र प्रथमपक्षे तस्या इन्द्रियधिय\edlabel{pvv.27-4}\footnote{\label{pvv.27-4}  ४ अव्यवधानेन च । तस्मादुत्तरबुद्धिजननसमर्था वासनारूपा कर्म्मसंज्ञिता पूर्व्वबुद्धिरेवाश्रय इति स्थितं ।} इव विकल्प्यमानेषु रूपादिषु स्पष्टतरस्य {\color{DodgerBlue3}“ग्रहणस्य प्राप्तेः । मनोज्ञानं न सेन्द्रियात कायात्”} । इन्द्रियजन्यस्य स्पष्टतयाऽभ्रान्तज्ञानादिषु व्याप्तिदृष्टेः । द्वितीयपक्षेप्याह (।) प्रत्येकमिन्द्रियाणां रूपादिग्रहणप्रतिनियत{\color{DodgerBlue3}“ज्ञानोत्पादनसामर्थ्यभेदात्”} दृष्टात् । {\color{DodgerBlue3}“न सकला-”} दपीन्द्रियकलापात् प्रतिनियतविषयाग्राहिणो मनोविज्ञानस्य सम्भवः\edlabel{pvv.27-5}\footnote{\label{pvv.27-5}  ५ सामान्यविषयत्वादस्य ।} । एकेन्द्रि-\leavevmode\marginnote{\textenglish{7a/MA}} यवैकल्येप्यनुत्पादप्रस\edlabel{pvv.27-6}\footnote{\label{pvv.27-6}  ६ तृतीयः तदा सम्बन्धाद् गवाश्ववत् पृथग्‏भावः स्यात् ।}ङ्गाच्च । (४९)
	\pend
      \label{div_pvv.1.50}\edlabel{div_pvv.1.50}
	  
	% new div opening: depth here is 2
	

	  \pstart नाप्यनिन्द्रियो हेतुरित्याह
	\pend
      
	  \bigskip
	  \begingroup
	  \large
	
	    
	    \stanza[\smallbreak]
	\label{pv.1.50a}\edlabel{pv.1.50a}\flagstanza{\tiny\textenglish{....1.50a}}अचेतनत्वान्नान्यस्माद्;\&[\smallbreak]


	
	  \endgroup
	

	  \pstart {\color{DodgerBlue3}“अचेतनत्वान्नान्यस्माद”}निन्द्रियात् केशनखादेरिव मनोज्ञानं ।
	\pend
      

	  \pstart नन्वचेतनत्वं किमिन्द्रियज्ञानरहितत्वमुत मनोज्ञानविमुक्तत्वं(।) तत्राद्यमिष्टमेव \leavevmode\marginnote{\textenglish{028/s}} इन्द्रियस्याभावे तज्ज्ञानाभावात् । अन्त्ये साध्याविशिष्टत्वं हेतोः । मनोज्ञानस्यैव साध्यत्वात् ।
	\pend
      

	  \pstart उच्यते (।) यथा स्पर्शा\edlabel{pvv.28-1}\footnote{\label{pvv.28-1}  १ स्पर्शमात्रस्य केवलासम्भवादादिशब्दः । तज्ज्ञानैरुपकार्यत्वादिनोक्तमपि ।}दयः स्पर्शज्ञानेन चेतयन्ते न तथा नखकेशादय इत्यचेतनाः । चेतनाप्रतिबद्धञ्च मनोविज्ञानं तदभावे न स्यात् ।
	\pend
      

	  \pstart (B.कर्म)
	\pend
      

	  \pstart (a. कर्मभेदात् कायमनसोः सहस्थितिः)
	\pend
      

	  \pstart ननु यदि न काय आश्रयः कथं {\color{DodgerBlue3}“सहस्थिति”}रित्यत आह ।
	\pend
      
	  \bigskip
	  \begingroup
	  \large
	
	    
	    \stanza[\smallbreak]
	\label{pv.1.50b}\edlabel{pv.1.50b}\flagstanza{\tiny\textenglish{....1.50b}}हेत्वभेंदात् सहस्थितिः ।&अक्षवद् रूपरसवदर्थद्वारेण विक्रिया ॥ ५० ॥\&[\smallbreak]


	
	  \endgroup
	

	  \pstart हेतोः कर्म्मसंज्ञितस्याभेदात् । एकसामग्रीप्रतिबद्धत्वात् सहस्थितिः । न त्वाश्रयाश्रयिभावात् ।\edlabel{pvv.28-2}\footnote{\label{pvv.28-2}  २ किमिव ।} {\color{DodgerBlue3}“अक्षवद्रूपरसवत्”} । यथाऽक्षाणि रूपरसादयश्च परस्परमनाश्रयाश्रयिभूता अप्येकसामग्र्‏यधीनत्वात् सहस्थितिमन्तः ।\edlabel{pvv.28-3}\footnote{\label{pvv.28-3}  ३ नियतसहभावो नेति च तुल्यं देहमनसोर्वा कल्प्यमनोमात्रत्वात् ।} ननु यो यद्विकारेण विक्रियते स तदाश्रितो यथा चक्षुरादिविकारेण विक्रियमाणन्तज्ज्ञानं चक्षुराद्याश्रितं । विषश्लेष्मादिना कायविकारे च विक्रियते मनोविज्ञानं अतस्तदाश्रितमित्याह (।) तज्ज्ञानैरुपकार्यत्वादिनोक्तमपि । {\color{DodgerBlue3}“अर्थद्वारेण विक्रिया”} आलम्व्यमाना हि शस्त्रप्रहारादयो विकारयन्ति मानसं न त्वाश्रय\edlabel{pvv.28-4}\footnote{\label{pvv.28-4}  ४ एतेन कारणात्प्राङ् निष्पन्नं तत्सहभावि वा न कार्यमिति भिन्नाभिन्नसन्तानयोः सामान्यः कार्यकारणभावः ।}त्वेन ।(५०)
	\pend
      \label{div_pvv.1.51}\edlabel{div_pvv.1.51}
	  
	% new div opening: depth here is 2
	

	  \begin{center}%% label @type='head'
	\textbf{i. उपकारको निवर्त्तकश्च हेतुः}
	\end{center}
	
	  \bigskip
	  \begingroup
	  \large
	
	    
	    \stanza[\smallbreak]
	\label{pv.1.51}\edlabel{pv.1.51}\flagstanza{\tiny\textenglish{...v.1.51}}सत्तोपकारिणो यस्य नित्यं तदनुबन्धतः ।&स हेतुः सप्तमी तस्मादुत्पादादिति चोच्यते ॥ ५१ ॥\&[\smallbreak]


	
	  \endgroup
	

	  \pstart न चोपकारक इत्येवाश्रयः किन्तु निर्व्वर्तकः । तदाह । {\color{DodgerBlue3}“यस्य”} सत्ता निर्वर्त्यस्य {\color{DodgerBlue3}“उपकारिणी”} स हेतुः स एवाश्रयः । कथमुपकारिणी(।) {\color{DodgerBlue3}“नित्यं तस्य”} निर्वर्त्त्यस्या{\color{DodgerBlue3}“नुबन्धतो”}ऽनुवर्तनात् । यस्य तु कदा\edlabel{pvv.28-5}\footnote{\label{pvv.28-5}  ५ स्वसन्निधिभावेन. . . . त्या कार्यकारीत्यर्थः मृत्पिण्डं घटकारीव । यथोपाघ्यायः तदसत्त्वेऽप्यस्य सत्त्वं ।} चिदुपकारकत्वमसौ विशेषस्यैव हेतुर्न धर्मिणः \leavevmode\marginnote{\textenglish{029/s}} तदभावेऽपि तद्भावात् । चित्तमात्रप्रतिबद्धञ्च चित्तं तद्विशेषस्तु कायादिहेतुकः । अतो नायं हेतुः न च तन्निवृ\edlabel{pvv.29-1}\footnote{\label{pvv.29-1}  १ आरूप्ये भावात् ।} त्त्या निवृत्तिश्चित्तस्य । नित्यानुबन्धितया हेतुत्वमभिप्रेत्य चास्मिन् सतीदं भवतीति सप्तम्युच्यते ।\edlabel{pvv.29-2}\footnote{\label{pvv.29-2}  २ कार्योत्पादनसमर्थस्य पूर्व भावः कारणत्वं सप्तम्या पञ्चम्या च निर्दिष्टं तदुत्तरत्वं कार्यत्वं प्रथममाह ।} तस्मादिति च पञ्चमी तद्धेत्वनुवद्धां कार्यताञ्चाभिप्रेत्यो{\color{DodgerBlue3}“त्पादादिति चो\edlabel{pvv.29-3}\footnote{\label{pvv.29-3}  ३ न चेदृशोन्वयः कायचित्तयोरस्ति दृश्यते वा ।}च्यते”} । (५१)
	\pend
      \label{div_pvv.1.52}\edlabel{div_pvv.1.52}
	  
	% new div opening: depth here is 2
	

	  \pstart ननु हेतुरुपकारकः कथ\edlabel{pvv.29-4}\footnote{\label{pvv.29-4}  ४ परं पृच्छति । यदि वह्न्यादिवत्तथा प्रदीपहेतुः तथा देहश्चैतन्यादेश्च ।}ञ्चिदुपकुर्व्वन् दृश्यते च कदाचिद्देहश्चित्तमत आह (।)
	\pend
      
	  \bigskip
	  \begingroup
	  \large
	
	    
	    \stanza[\smallbreak]
	\label{pv.1.52}\edlabel{pv.1.52}\flagstanza{\tiny\textenglish{...v.1.52}}अस्तूपकारको वापि कदाचिच्चित्तसन्ततेः ।&वह्न्यादिवद् घटादीनां विनिवृत्तिर्न तावता ॥ ५२ ॥\&[\smallbreak]


	
	  \endgroup
	

	  \pstart {\color{DodgerBlue3}“अस्तूपकार\edlabel{pvv.29-5}\footnote{\label{pvv.29-5}  ५ कायाश्रितज्ञानद्वारेण ।}को वापि कदाचिच्चित्तसन्तते”}र्देहो {\color{DodgerBlue3}“वह्न्यादिवद् घटादीनां न च तावतो”}पकारकनिवृत्त्या उपकार्यस्य {\color{DodgerBlue3}“निवृत्तिः”} । न हि वह्निर्घटस्यापाकमुपकारं कुर्व्वन्नपि स्वनिवृत्त्या निवर्तकः । (५२)
	\pend
      \label{div_pvv.1.53}\edlabel{div_pvv.1.53}
	  
	% new div opening: depth here is 2
	

	  \pstart किञ्च\edlabel{pvv.29-6}\footnote{\label{pvv.29-6}  ६ यदि देहः कारणं बुद्धेः ।} (।)
	\pend
      
	  \bigskip
	  \begingroup
	  \large
	
	    
	    \stanza[\smallbreak]
	\label{pv.1.53a}\edlabel{pv.1.53a}\flagstanza{\tiny\textenglish{....1.53a}}अनिवृत्तिप्रसङ्गश्च देहे तिष्ठति चेतसः ।\&[\smallbreak]


	
	  \endgroup
	

	  \pstart {\color{DodgerBlue3}“अनेवृत्तिप्रसङ्गश्च देहे तिष्ठति चेतसः”} । स्वजातिनिरपेक्षस्य देहमात्रहेतुकस्याविकलहेतोरनुत्पत्त्ययोगात् ततो यावद्धेतुस्तावन्न मरणं भवेत् ।
	\pend
      

	  \begin{center}%% label @type='head'
	\textbf{(ii. न प्राणापानहेतुकं चित्तम्)}
	\end{center}
	

	  \pstart प्राणापानावपि चित्तकारणं तयोर्व्वैकल्यान्मरणावस्थायान्न चित्तोत्पाद इति चेत् । आह (।)
	\pend
      
	  \bigskip
	  \begingroup
	  \large
	
	    
	    \stanza[\smallbreak]
	\label{pv.1.53b}\edlabel{pv.1.53b}\flagstanza{\tiny\textenglish{....1.53b}}तद्भावभावाद् वश्यत्वात् प्राणापनौ ततो न तत् ॥ ५३ ॥\&[\smallbreak]


	
	  \endgroup
	

	  \pstart तस्य चित्तस्य {\color{DodgerBlue3}“भावे भावात् प्राणापा\edlabel{pvv.29-7}\footnote{\label{pvv.29-7}  ७ आकुञ्चनप्रसारणवत् ।} नौ ततो”} भवतः । {\color{DodgerBlue3}“चित्तवश्यत्वाच्च”} तत एव तौ न तु ताभ्यां तच्चित्तं विपर्ययात् ।(५३)
	\pend
      \leavevmode\marginnote{\textenglish{030/s}}\label{div_pvv.1.54}\edlabel{div_pvv.1.54}
	  
	% new div opening: depth here is 2
	

	  \pstart एतद्धेतुद्वयसिद्ध्यर्थमाह (।)
	\pend
      
	  \bigskip
	  \begingroup
	  \large
	
	    
	    \stanza[\smallbreak]
	\label{pv.1.54a}\edlabel{pv.1.54a}\flagstanza{\tiny\textenglish{....1.54a}}प्रेरणाकर्षणो वायोः प्रयत्नेन विना कुतः ।\&[\smallbreak]


	
	  \endgroup
	

	  \pstart बहिरन्तश्च {\color{DodgerBlue3}“प्रेरणाकर्षणे वायो”}र्यथाक्रमं प्राणा\edlabel{pvv.30-1}\footnote{\label{pvv.30-1}  १ वृत्तिकृता (?) एतत्पदमनङ्गीकृत्य व्याख्यातम् । पुनः परमतमाशङ्कते न कार्यं चैतन्यस्य कारणं किन्तु ।}पानौ पुरुषस्य बुद्धिलक्षणं प्रयत्न{\color{DodgerBlue3}“म्वि\edlabel{pvv.30-2}\footnote{\label{pvv.30-2}  २ तद्‏भावात् प्रेरणाकर्षणवत् ।}ना”} कुतः सम्भवतस्तस्मात् तदधीनौ ।
	\pend
      

	  \pstart किञ्च (।)
	\pend
      
	  \bigskip
	  \begingroup
	  \large
	
	    
	    \stanza[\smallbreak]
	\label{pv.1.54b}\edlabel{pv.1.54b}\flagstanza{\tiny\textenglish{....1.54b}}निर्ह्रासातिशयापत्तिनिर्ह्रासातिशयात् तयोः ॥ ५४ ॥\&[\smallbreak]


	
	  \endgroup
	

	  \pstart यदि प्राणापानहेतुकञ्चितं तदा चित्तस्य {\color{DodgerBlue3}“निर्ह्रासातिशयापत्ति”}रपकर्षोत्कर्षापत्तिः । {\color{DodgerBlue3}“निर्ह्रासातिशयात्तयोः”} प्राणापानयोः कारणविशेषानुकारित्वात् कार्य\edlabel{pvv.30-3}\footnote{\label{pvv.30-3}  ३ यथा मृतावस्थायां शरीरे तिष्ठति किन्तच्चैतन्यं न भवतीति चोद्यं बौद्धेन तथा प्राणापानावपि कस्मान्निवर्त्त्येते येन तदभावान्न चैतन्यम् ।}विशेषस्य । (५४)
	\pend
      \label{div_pvv.1.55}\edlabel{div_pvv.1.55}
	  
	% new div opening: depth here is 2
	
	  \bigskip
	  \begingroup
	  \large
	
	    
	    \stanza[\smallbreak]
	\label{pv.1.55a}\edlabel{pv.1.55a}\flagstanza{\tiny\textenglish{....1.55a}}तुल्यः प्रसङ्गोपि तयोः;\&[\smallbreak]


	
	  \endgroup
	

	  \pstart \edlabel{pvv.30-4}\footnote{\label{pvv.30-4}  ४ परः मृतदेहे तिष्ठिति चित्तं किन्न स्यादिति बौद्धस्य प्रसङ्गेन सह ।}अपि च {\color{DodgerBlue3}“तयोरपि”} प्राणापानयोर्देहे तिष्ठत्यनिवृ\edlabel{pvv.30-5}\footnote{\label{pvv.30-5}  ५ अनेकान्तत्वमुक्तं परस्य ।}त्ति{\color{DodgerBlue3}“प्रसङ्गस्तुल्य”}स्तद्धेतुत्वा{\color{DodgerBlue3}“त्तयोः”} । ततो मरणकाले प्राणापानवैकल्याभावाच्चित्तानिवृत्तिप्रसङ्गस्तदवस्थः ।
	\pend
      

	  \pstart (b. स्यित्यावेधकत्वात् कारणं कर्म)
	\pend
      

	  \pstart स्यादेतत् (।) {\color{DodgerBlue3}“चित्तकारणे”}ऽपि चित्ते स्वीक्रियमाणे मरणसमये चित्तानपायात् चित्तोत्पत्तौ यावद्देहेऽनिवृत्तिप्रसङ्ग इत्याह (सिद्धान्ती)।
	\pend
      
	  \bigskip
	  \begingroup
	  \large
	
	    
	    \stanza[\smallbreak]
	\label{pv.1.55b}\edlabel{pv.1.55b}\flagstanza{\tiny\textenglish{....1.55b}}न तुल्यं चित्तकारणे ।&स्थित्यावेधकमन्यच्च यतः कारणमिष्यते ॥ ५५ ॥\&[\smallbreak]


	
	  \endgroup
	

	  \pstart {\color{DodgerBlue3}“न तुल्यं चित्तकारणे”} चैतन्ये प्रसञ्जनं । न खल्वनुवर्तकमेव कारणमिष्टं येन नित्यं तस्मिन्नाश्रये स्यात् । किन्तर्हि (।) {\color{DodgerBlue3}“स्थित्यावेधकमन्यच्च”} कर्म्माख्यं {\color{DodgerBlue3}“कारणं यत इष्यते”} तस्मात्तेन कर्मणा यावत्कालं तद्देहे स्थितिराक्षिप्ता तत ऊर्ध्व हेत्वपायान्न प्रवर्तते देहान्तरे तु वर्तत इत्यसमानत्वं ।(५५)
	\pend
      \label{div_pvv.1.56}\edlabel{div_pvv.1.56}
	  
	% new div opening: depth here is 2
	\leavevmode\marginnote{\textenglish{031/s}}

	  \begin{center}%% label @type='head'
	\textbf{(i. अन्यथा दोषाभावे पुनरुज्जीवनम्)}
	\end{center}
	
	  \bigskip
	  \begingroup
	  \large
	
	    
	    \stanza[\smallbreak]
	\label{pv.1.56}\edlabel{pv.1.56}\flagstanza{\tiny\textenglish{...v.1.56}}न दोषैर्विगुणो दोहो हेतुर्वर्त्यादिवद् यदि ।&मृते शमीकृते दोषे पुनरुज्जीवनं भवेत् ॥ ५६ ॥\&[\smallbreak]


	
	  \endgroup
	

	  \pstart दौषैर्व्वातादिभिर्विगुणीकृतो {\color{DodgerBlue3}“देहो”} न हेतुश्चैतन्यस्य प्राणापानादेश्च {\color{DodgerBlue3}“वर्त्त्यादिवद् यदि”} । यथा विगुणीकृतो वर्त्त्यादिर्न दीपस्य हेतुः । तदा {\color{DodgerBlue3}“मृते”}\edlabel{pvv.31-1}\footnote{\label{pvv.31-1}  १ कस्यचित्स्वभावस्याकाराभिमतेनाकार्यत्वादकिञ्चित्करो नाश्रयः । असद्धि शशविषाणकल्पं किंमाश्रयेत ।} शरीरे चैतन्यादि-\leavevmode\marginnote{\textenglish{7b/MA}} नाशान्मृतस्य शमीभवन्ति दोषा इति {\color{DodgerBlue3}“शमीकृते दोषे”} विकारविकारिण्यारोग्यलाभाद्धेतोर्देहस्य {\color{DodgerBlue3}“पुनरुज्जीवनं”} प्राणा\edlabel{pvv.31-2}\footnote{\label{pvv.31-2}  २ तस्माद्दोषाभावेऽप्यभावान्न दोषसन्निधिमात्रं चैतन्यवृत्तिं प्रतिबध्नातीति स्थितं ।} पानं चैतन्योत्पत्तिलक्ष{\color{DodgerBlue3}“णम्भ\edlabel{pvv.31-3}\footnote{\label{pvv.31-3}  ३ परः प्राह ।} वेत्”} । (५६)
	\pend
      \label{div_pvv.1.57}\edlabel{div_pvv.1.57}
	  
	% new div opening: depth here is 2
	

	  \begin{center}%% label @type='head'
	\textbf{ii. देहश्चित्तस्य नोपादानम्}
	\end{center}
	
	  \bigskip
	  \begingroup
	  \large
	
	    
	    \stanza[\smallbreak]
	\label{pv.1.57}\edlabel{pv.1.57}\flagstanza{\tiny\textenglish{...v.1.57}}निवृत्तेप्यनले काष्ठविकाराविनिवृत्तिवत् ।&तस्याऽनिवृत्तिरिति चेन्न चिकित्साप्रयोगतः ॥ ५७ ॥\&[\smallbreak]


	
	  \endgroup
	

	  \pstart {\color{DodgerBlue3}“निवृत्तेप्यनले”} हेतौ {\color{DodgerBlue3}“काष्ठविकार”}स्याङ्गारादेर{\color{DodgerBlue3}“निवृत्तिवत् । तस्य”} चैतन्यादिनिरोधस्या{\color{DodgerBlue3}“निवृत्तिर्निवृत्तिर्न”} भवति तद्दोषे मृतशरीरान्निवृत्ते{\color{DodgerBlue3}“पीति चेत्\edlabel{pvv.31-4}\footnote{\label{pvv.31-4}  ४ परिहरति} । न चिकित्साप्रयोगतः”} । (५७)
	\pend
      \label{div_pvv.1.58}\edlabel{div_pvv.1.58}
	  
	% new div opening: depth here is 2
	

	  \pstart स्यादेतद् (।) यद्यनिवर्त्त्या दोषाणां विकार(ा)ः स्यात्(?स्युः) दृश्यन्ते च चिकित्स्यमाना दोषविकारा ज्वरादयः ।
	\pend
      

	  \pstart एत\edlabel{pvv.31-5}\footnote{\label{pvv.31-5}  ५ पूर्व्वोक्तं स्पष्टयति ।}देवाह ।
	\pend
      
	  \bigskip
	  \begingroup
	  \large
	
	    
	    \stanza[\smallbreak]
	\label{pv.1.58}\edlabel{pv.1.58}\flagstanza{\tiny\textenglish{...v.1.58}}अपुनर्भावतः किञ्चिद् विकारजननं क्वचित् ।&किञ्चिद् विपर्ययादग्निर्यथा काष्ठसुवर्णयोः ॥ ५८ ॥\&[\smallbreak]


	
	  \endgroup
	

	  \pstart {\color{DodgerBlue3}“किञ्चि”}द्वस्तु {\color{DodgerBlue3}“क्वचि”}द्विकार्ये {\color{DodgerBlue3}“विका”}रस्य {\color{DodgerBlue3}“जननं”} जनक{\color{DodgerBlue3}“मपुनर्भावतो”} यथा विकार्यस्यापुनर्भावो निर्व्विकारत्वं पुनर्न भवति {\color{DodgerBlue3}“यथाग्नि काष्ठे (।) किञ्चिदेव तद्विपर्य्य-”} याद्यथा{\color{DodgerBlue3}“ग्नि सुवर्ण्णे”} । (५८)
	\pend
      \label{div_pvv.1.59}\edlabel{div_pvv.1.59}
	  
	% new div opening: depth here is 2
	
	  \bigskip
	  \begingroup
	  \large
	
	    
	    \stanza[\smallbreak]
	\label{pv.1.59}\edlabel{pv.1.59}\flagstanza{\tiny\textenglish{...v.1.59}}आद्यस्याल्पोप्यसंहार्यः प्रत्यानेयस्तु यत्कृतः ।&विकारः स्यात् पुनर्भावः तस्य हेम्नि खरत्ववत् ॥ ५९ ॥\&[\smallbreak]


	
	  \endgroup
	\leavevmode\marginnote{\textenglish{032/s}}

	  \pstart {\color{DodgerBlue3}“तत्राद्यस्य”} विकारस्या{\color{DodgerBlue3}“ल्पोपि”} विकारः श्यामतादिकोऽसंहार्योऽनिवर्तनीयः । {\color{DodgerBlue3}“प्रत्यानेयो”} निवर्तनीयस्तु {\color{DodgerBlue3}“यत्कृतो विकारस्तस्य”} विकार्यस्य {\color{DodgerBlue3}“पुनर्भावः स्या”}त् विकारनिवृत्त्या पूर्व्वावस्थापत्तिः । वह्निकृते द्रवत्वे निवृत्ते हेम्नि खरत्ववत् । दृढत्वमिव ॥ (५९)
	\pend
      \label{div_pvv.1.60}\edlabel{div_pvv.1.60}
	  
	% new div opening: depth here is 2
	

	  \pstart ननु\edlabel{pvv.32-1}\footnote{\label{pvv.32-1}  १ चिकित्स्यते च ज्ञानादिकृतो विकारः तस्मात्पूर्व्वप्रसङ्गस्तदवस्थः ।} यदि निवर्तनीयो दोषविकारस्तदा न कश्चिदसाध्यो वि\edlabel{pvv.32-2}\footnote{\label{pvv.32-2}  २ काष्ठविकारवदेनमाह परः ।}धिः स्यादित्याह ॥
	\pend
      
	  \bigskip
	  \begingroup
	  \large
	
	    
	    \stanza[\smallbreak]
	\label{pv.1.60}\edlabel{pv.1.60}\flagstanza{\tiny\textenglish{...v.1.60}}दुर्लभत्वात् समाधातुरसाध्यं किञ्चिदीरितम् ।&आयुःक्षयाद् वा; दोषे तु केवले नास्त्यसाध्यता ॥ ६० ॥\&[\smallbreak]


	
	  \endgroup
	

	  \pstart {\color{DodgerBlue3}“यत्किञ्चिदसाध्यं”} व्या\edlabel{pvv.32-3}\footnote{\label{pvv.32-3}  ३ सिद्धान्ती ।}धिजातमुक्तं त{\color{DodgerBlue3}“त्समाधातुः”} सद्यो दोषशमनौषधरसायनादेर्दुर्ल\edlabel{pvv.32-4}\footnote{\label{pvv.32-4}  ४ असाध्यव्याधिश्च तत्तत्‏कर्मवशात् । अकालमरणे ॥} {\color{DodgerBlue3}“भत्वात्”} कर्म्माक्षिप्तस्यायुषस्तच्छरीरसहचरचैतन्यादिस्थितिहेतोर्व्वा क्ष\edlabel{pvv.32-5}\footnote{\label{pvv.32-5}  ५ कालमरणे ।}यात् । भूतमात्रवादिनः {\color{DodgerBlue3}“केवले दोषे तु”} स्वीक्रियमाणे {\color{DodgerBlue3}“नास्त्यसाध्यता”} कस्यचिद् व्याधेः । भूतमात्रारब्धस्य दोषस्य चिकित्सादृष्टेः । विशे\edlabel{pvv.32-6}\footnote{\label{pvv.32-6}  ६ तुल्यव्याध्योस्तुल्यौषधयोश्च चिरक्षिप्रोपशमश्च दृश्यत एव कर्मकृतः ।}षकारिणश्च हेतोरभावात् । (६०)
	\pend
      \label{div_pvv.1.61}\edlabel{div_pvv.1.61}
	  
	% new div opening: depth here is 2
	

	  \pstart तथा (।)
	\pend
      
	  \bigskip
	  \begingroup
	  \large
	
	    
	    \stanza[\smallbreak]
	\label{pv.1.61}\edlabel{pv.1.61}\flagstanza{\tiny\textenglish{...v.1.61}}मृते विषादिसंहारात् तद्दंशच्छेदतोऽपि वा ।&विकारहेतोर्विगमे स नोच्छ्‏वसिति किं पुनः ॥ ६१ ॥\&[\smallbreak]


	
	  \endgroup
	

	  \pstart विषादिना {\color{DodgerBlue3}“मृते”} प्राणिनि सति दंश\edlabel{pvv.32-7}\footnote{\label{pvv.32-7}  ७ यावज्जीवति तावद् व्याप्नोति विषं मूते तु दंशस्थानं यातीति नियमः ।} स्थाने {\color{DodgerBlue3}“विषादेः संहारात्”} संगणनात्तस्य {\color{DodgerBlue3}“दंश”}स्थानस्य {\color{DodgerBlue3}“च्छेदतोपि वा विकारहेतो”}र्मरणहेतोर्व्विषादे{\color{DodgerBlue3}“र्व्विगमात् स”} मृतः {\color{DodgerBlue3}“पुनर्नोच्छ्‏वसिति”} । कस्मात् (।) चैतन्यहेतोर्देहस्य विकारत्वात् युक्तमुच्छ्‏वसितुं । (६१)
	\pend
      \label{div_pvv.1.62}\edlabel{div_pvv.1.62}
	  
	% new div opening: depth here is 2
	

	  \pstart किञ्च (।)
	\pend
      
	  \bigskip
	  \begingroup
	  \large
	
	    
	    \stanza[\smallbreak]
	\label{pv.1.62}\edlabel{pv.1.62}\flagstanza{\tiny\textenglish{...v.1.62}}उपादानाविकारेण नोपादेयस्य विक्रिया ।&कर्त्तुं शक्याऽविकारेण मृदः कुण्डादिनो यथा ॥ ६२ ॥\&[\smallbreak]


	
	  \endgroup
	

	  \pstart यद्यपादानं देहश्चित्तस्य तदो{\color{DodgerBlue3}“पादान”}स्य देहस्या{\color{DodgerBlue3}“विकारे”}ण विकारं विनो{\color{DodgerBlue3}“पादेयस्य”} चित्तस्य {\color{DodgerBlue3}“विक्रिया न कर्त्तुं शक्या”} स्यात् । {\color{DodgerBlue3}“यथा”} कुण्डाद्युपादानभूताया {\color{DodgerBlue3}“मृदो”} विकारेण \leavevmode\marginnote{\textenglish{033/s}} विना कु{\color{DodgerBlue3}“ण्डादिनो”} विकारः कर्त्तुभशक्यः । (६२)
	\pend
      \label{div_pvv.1.63}\edlabel{div_pvv.1.63}
	  
	% new div opening: depth here is 2
	

	  \pstart कस्मादेवमित्याह (।)
	\pend
      
	  \bigskip
	  \begingroup
	  \large
	
	    
	    \stanza[\smallbreak]
	\label{pv.1.63}\edlabel{pv.1.63}\flagstanza{\tiny\textenglish{...v.1.63}}अविकृत्य हि यद् वस्तु यः पदार्थों विकार्यते ।&उपादानं न तत् तस्य युक्तं गोगवयादिवत् ॥ ६३ ॥\&[\smallbreak]


	
	  \endgroup
	

	  \pstart हिर्यस्मात् {\color{DodgerBlue3}“अविकृत्य यद्वस्तु”} किञ्चिद्यः {\color{DodgerBlue3}“पदार्थो विकार्यते उपादानन्न तत्तस्य”} पदार्थस्य यु{\color{DodgerBlue3}“क्तं गोगवयादिवत्”} गोगवययोरिव नोपादानोपादेयभावः । एकस्याविकारेणापरस्य विकारात् । (६३)
	\pend
      \label{div_pvv.1.64}\edlabel{div_pvv.1.64}
	  
	% new div opening: depth here is 2
	

	  \begin{center}%% label @type='head'
	\textbf{(II. व्यतिरेकतः)}
	\end{center}
	

	  \begin{center}%% label @type='head'
	\textbf{A. आश्रयाश्रयिभावनिरासः}
	\end{center}
	

	  \begin{center}%% label @type='head'
	\textbf{a. सामान्येन आश्रयाश्रयिभावनिरासः}
	\end{center}
	

	  \begin{center}%% label @type='head'
	\textbf{i. न कायचेतसोः सहस्थितिराश्रयाश्रयितया}
	\end{center}
	
	  \bigskip
	  \begingroup
	  \large
	
	    
	    \stanza[\smallbreak]
	\label{pv.1.64a}\edlabel{pv.1.64a}\flagstanza{\tiny\textenglish{....1.64a}}चेतःशरीरयोरेवं तद्धेतोः कार्यजन्मनः ।\&[\smallbreak]


	
	  \endgroup
	

	  \pstart {\color{DodgerBlue3}“चेतःशरीरयोरेवं”} शरीरमविकृत्यैव भयशोकादिना समनन्तरप्रत्ययविकारमात्रेण चेतसो विकारोत्पत्तेः नोपादानोपादेयभावः । यदि नाम देहोऽपि कथञ्चिदुपकारकश्चित्तस्यावस्थाविशेषहेतुत्वात् तथापि सन्तानहेतुत्वाभावात् नोपादानमतस्तन्निवृत्त्या न चैतन्यनिवृत्तिः । अवस्थाविशेष एव निवर्तेत ।
	\pend
      

	  \pstart कथन्तर्हि सहाव (स्था) नं {\color{DodgerBlue3}“चेतःशरीरयो”}रित्याह ॥
	\pend
      
	  \bigskip
	  \begingroup
	  \large
	
	    
	    \stanza[\smallbreak]
	\label{pv.1.64b}\edlabel{pv.1.64b}\flagstanza{\tiny\textenglish{....1.64b}}सहकारात् सहस्थानमग्निताम्रद्रवत्ववत् ॥ ६४ ॥\&[\smallbreak]


	
	  \endgroup
	

	  \pstart तस्य चेतसः शरीरस्य च {\color{DodgerBlue3}“हेतोः”} पूर्व्वचित्तक्षणस्य कललादेश्च । {\color{DodgerBlue3}“सहकारात्सह-”} कारणात् {\color{DodgerBlue3}“कार्य”}योश्चित्तदेहयो{\color{DodgerBlue3}“र्जन्मन”} उत्पादात् तयोः सहस्थानं भवति । यथा{\color{DodgerBlue3}“ग्निताम्र”}जन्मनोर्व्वह्निताम्रद्रवत्वयोः सहावस्थितिः ॥ (६४)
	\pend
      \label{div_pvv.1.65}\edlabel{div_pvv.1.65}
	  
	% new div opening: depth here is 2
	

	  \begin{center}%% label @type='head'
	\textbf{(ii र्स्थितिस्थात्रोरव्यतिरेकात्)}
	\end{center}
	

	  \pstart ननु देहश्चित्तस्याश्रयस्ततः सहस्थितिः स्यादित्याह (।)
	\pend
      
	  \bigskip
	  \begingroup
	  \large
	
	    
	    \stanza[\smallbreak]
	\label{pv.1.65}\edlabel{pv.1.65}\flagstanza{\tiny\textenglish{...v.1.65}}अनाश्रयात्सदसतोर्नाश्रयः स्थितिकारणम् ।&सतश्चेदाश्रयो नास्याः स्थातुरव्यतिरेकतः ॥ ६५ ॥\&[\smallbreak]


	
	  \endgroup
	

	  \pstart {\color{DodgerBlue3}“अना\edlabel{pvv.33-1}\footnote{\label{pvv.33-1}  १ आश्रयत्वायोगात् ।} श्रयात् सदसतोर्नाश्रयः न”} ह्यकारणमाश्रयः । अतिप्रसङ्गात् । {\color{DodgerBlue3}“सतश्च”} \leavevmode\marginnote{\textenglish{034/s}} निष्पन्नत्वात् ना\edlabel{pvv.34-1}\footnote{\label{pvv.34-1}  १ यस्य पुनरदृष्टं किञ्चिदपरं सहकारि नास्ति यच्छक्तिक्षयादस्थिभिश्चैतन्ये स्यात् ।}श्रयः कश्चित् । असतोऽपि वा कारणं किञ्चित् स्याच्\edlabel{pvv.34-2}\footnote{\label{pvv.34-2}  २ एकाहिदष्टयोरपि कालदष्टो म्रियतेऽन्यो जीवतीत्यस्ति ।} न त्वाश्रयः स्वरूपस्यैवाभावात् ॥ सतः {\color{DodgerBlue3}“स्थितिकारण”}माश्रयश्चेदिष्यते (।) नैतदपि युक्त{\color{DodgerBlue3}“मस्याः स्थातुरव्यतिरेकतः”} । न हि स्थितिर्नाम स्थातुः पदार्थात् भिन्ना यां कुर्वत आश्रयत्वं । (६५)
	\pend
      \label{div_pvv.1.66_1.67}\edlabel{div_pvv.1.66_1.67}
	  
	% new div opening: depth here is 2
	

	  \begin{center}%% label @type='head'
	\textbf{(iii न्यतिरेकेऽपि नाश्रयः)}
	\end{center}
	
	  \bigskip
	  \begingroup
	  \large
	
	    
	    \stanza[\smallbreak]
	\label{pv.1.66a}\edlabel{pv.1.66a}\flagstanza{\tiny\textenglish{....1.66a}}व्यतिरेकेऽपि तद्धेतुस्तेन भावस्य किं कृतम् ।\&[\smallbreak]


	
	  \endgroup
	

	  \pstart {\color{DodgerBlue3}“व्यतिरेके”}ऽपि वा स्वीक्रियमाणे {\color{DodgerBlue3}“त\edlabel{pvv.34-3}\footnote{\label{pvv.34-3}  ३ स्थितेः ।}द्धे”}तुराश्रयः स्यात् । {\color{DodgerBlue3}“तेन”} चाश्रयाभिमतेन \leavevmode\marginnote{\textenglish{8a/MA}} {\color{DodgerBlue3}“भावस्य”} स्थितिमतः {\color{DodgerBlue3}“किं कृतं”} येनासावाश्रयः ।
	\pend
      

	  \begin{center}%% label @type='head'
	\textbf{(iv. नोत्यन्नया स्थित्या भावस्थापना)}
	\end{center}
	

	  \pstart स्यादेतद् (।) भावसम्बन्धिनी स्थितिर्भावं स्थापयति तेन तत्कर्त्तुराश्रयत्वमिर्त्याह (।)
	\pend
      
	  \bigskip
	  \begingroup
	  \large
	
	    
	    \stanza[\smallbreak]
	\label{pv.1.66b}\edlabel{pv.1.66b}\flagstanza{\tiny\textenglish{....1.66b}}अविनाशप्रसङ्गः स नाशहेतोर्मतो यदि ॥ ६६ ॥\&[\smallbreak]


	
	  \endgroup
	
	  \bigskip
	  \begingroup
	  \large
	
	    
	    \stanza[\smallbreak]
	\label{pv.1.67a}\edlabel{pv.1.67a}\flagstanza{\tiny\textenglish{....1.67a}}तुल्यः प्रसङ्गस्तत्रापि;\&[\smallbreak]


	
	  \endgroup
	

	  \pstart यदि स्थित्योत्पन्नया भावः स्थाप्यते तदान कदाचिदस्य भा\edlabel{pvv.34-4}\footnote{\label{pvv.34-4}  ४ न स्थितिकारणात्कुण्डादिवाश्रयः किन्तु ।}वस्य विनाशः स्यात् {\color{DodgerBlue3}“स नाशहेतो”}र्मुद्‏गरा{\color{DodgerBlue3}“देर्मतो यदि”} (। ६६) {\color{DodgerBlue3}“तुल्यः प्रसङ्गस्तत्रापि”} । नाशोऽपि न तावद् भावादव्यतिरिक्तः क्रियते तस्योत्पन्नत्वात् । व्यतिरिक्तेऽपि नाशे कृते भावस्तदवस्थ इति प्राग्‏वदुपलम्भादिप्रसङ्गः ।
	\pend
      

	  \pstart किञ्च (।)
	\pend
      

	  \begin{center}%% label @type='head'
	\textbf{(v. नाशस्य सहेतुत्वे स्थितिहेतुर्निष्फलः)}
	\end{center}
	
	  \bigskip
	  \begingroup
	  \large
	
	    
	    \stanza[\smallbreak]
	\label{pv.1.67b}\edlabel{pv.1.67b}\flagstanza{\tiny\textenglish{....1.67b}}किं पुनः स्थितिहेतुना ॥&आ नाशकागमात् स्थानं ततश्चेद् वस्तुधर्मता ॥ ६७ ॥\&[\smallbreak]


	
	  \endgroup
	
	  \bigskip
	  \begingroup
	  \large
	
	    
	    \stanza[\smallbreak]
	\label{pv.1.68a}\edlabel{pv.1.68a}\flagstanza{\tiny\textenglish{....1.68a}}नाशास्य;\&[\smallbreak]


	
	  \endgroup
	

	  \pstart यदि नाशहेतुना नाशः क्रियते तदा {\color{DodgerBlue3}“किं पुनः स्थितिहेतुना”}श्रयेण यावन्नाशहेतुर्नापतति तावत् स्वयमेव स्थास्यति । आपतिताच्च तस्मान्नैष रक्षणक्षम इति \leavevmode\marginnote{\textenglish{035/s}} किमनेन स्वीकृतेन । {\color{DodgerBlue3}“आ नाशकागमाद्धि नाशकागमनं यावत्तत आश्रयात् स्थान”}ञ्चेदिष्यते । एवन्तर्हि वस्तु{\color{DodgerBlue3}“धर्मता ना\edlabel{pvv.35-1}\footnote{\label{pvv.35-1}  १ यदि स्थापकं विना न तिष्ठति ।} शस्य”} । (६७)
	\pend
      \label{div_pvv.1.68}\edlabel{div_pvv.1.68}
	  
	% new div opening: depth here is 2
	

	  \pstart यदि नश्वरो भावः\edlabel{pvv.35-2}\footnote{\label{pvv.35-2}  २ नाशशीलः स्वयञ्चेद् स्यात ।} तदाश्रयेण नाशकोपनिपातं {\color{DodgerBlue3}“यावत्स्थाप्येतान्य\edlabel{pvv.35-3}\footnote{\label{pvv.35-3}  ३ अनश्वरत्वे ।}था”} तु स्वयमेव स्थितिमान् किमाश्रयेण (।)
	\pend
      
	  \bigskip
	  \begingroup
	  \large
	
	    
	    \stanza[\smallbreak]
	\label{pv.1.68b}\edlabel{pv.1.68b}\flagstanza{\tiny\textenglish{....1.68b}}सत्यबाधोसाविति किं स्थितिहेतुना ॥&यथा जलादेराधार इति चेत् तुल्यमत्र च ॥ ६८ ॥\&[\smallbreak]


	
	  \endgroup
	

	  \pstart एवं {\color{DodgerBlue3}“सत्य\edlabel{pvv.35-4}\footnote{\label{pvv.35-4}  ४ विनाशित्वेऽस्य नाशहेतुनापि न किञ्चिन्नखवत्त्वादेव ।}बाधोऽसाविति किं स्थितिहेतुना\edlabel{pvv.35-5}\footnote{\label{pvv.35-5}  ५ भावस्य नाशित्वात् ।}”} विद्यमाने वस्तुनि स्वभा\edlabel{pvv.35-6}\footnote{\label{pvv.35-6}  ६ कुण्डादयः किं कुर्व्वाणा आश्रय इति ।}वत्वात् अबाधो बाधरहितोसौ नाश इति किं स्थितिहेतुना स्वीकृतेनापि । {\color{DodgerBlue3}“यथा जलादेः”} सत एवाधारो घटादिस्तथा चित्तस्य देह {\color{DodgerBlue3}“इति चेत् । तुल्यमत्र च”} प्रागुक्तं सकलं ॥ (६८)
	\pend
      \label{div_pvv.1.69}\edlabel{div_pvv.1.69}
	  
	% new div opening: depth here is 2
	

	  \begin{center}%% label @type='head'
	\textbf{(vi. भावसन्ततेर्हेतुत्वादाश्रयत्वम्)}
	\end{center}
	

	  \pstart कथन्तर्ह्याधारव्यवहारः समर्थनीय इत्या\edlabel{pvv.35-3-bis}\footnote{\label{pvv.35-3-bis}  ३ अनश्वरत्वे ।} ह ।
	\pend
      
	  \bigskip
	  \begingroup
	  \large
	
	    
	    \stanza[\smallbreak]
	\label{pv.1.69}\edlabel{pv.1.69}\flagstanza{\tiny\textenglish{...v.1.69}}प्रतिक्षणविनाशे हि भावानां भावसन्ततेः ।&तथोत्पत्तेः सहेतुत्वादाश्रयोऽयुक्तमन्यथा ॥ ६९ ॥\&[\smallbreak]


	
	  \endgroup
	

	  \pstart भावानां हि विनश्वरस्वभावतया {\color{DodgerBlue3}“प्रतिक्षणविनाशे”} यो भावः सहकारी भवासन्ततेः तथा तादृश्याः स्वोपादानदे\edlabel{pvv.35-7}\footnote{\label{pvv.35-7}  ७ आधारस्य ।}शाया {\color{DodgerBlue3}“उत्पत्ते”}र्न्निमित्तं {\color{DodgerBlue3}“सहेतुत्वादाश्रयो”} न {\color{DodgerBlue3}“त्वन्यथा”}ऽयुक्तत्वात् । (६९)
	\pend
      \label{div_pvv.1.70}\edlabel{div_pvv.1.70}
	  
	% new div opening: depth here is 2
	

	  \begin{center}%% label @type='head'
	\textbf{(b. i विशेषणाश्रयाश्रयिभावदूषणम्)}
	\end{center}
	

	  \begin{center}%% label @type='head'
	\textbf{ii गुणसामान्यकर्मणां दूषणम्}
	\end{center}
	

	  \pstart एवं सामान्येनाश्रयाश्रयिभावदूषणमभिधाय द्रव्यदूषणादौ विशेषे दूषणं वक्तुमाह ।
	\pend
      
	  \bigskip
	  \begingroup
	  \large
	
	    
	    \stanza[\smallbreak]
	\label{pv.1.70}\edlabel{pv.1.70}\flagstanza{\tiny\textenglish{...v.1.70}}स्यादाधारो जलादीनां गमनप्रतिबन्धतः ।&अगतीनां किमाधारैर्गुणसामान्यकर्मणाम् ॥ ७० ॥\&[\smallbreak]


	
	  \endgroup
	

	  \pstart स्यादाधारो {\color{DodgerBlue3}“जलादीनां”} प्रसर्पणधर्मणां {\color{DodgerBlue3}“गमनप्रतिबन्धतः”} कुण्डादिः {\color{DodgerBlue3}“अगती”}नां \leavevmode\marginnote{\textenglish{036/s}} {\color{DodgerBlue3}“निष्क्रियत्वात्किमाधारैः”} गुणिव्यक्त्यादि{\color{DodgerBlue3}“भिर्गुणसामान्यकर्मणां”} पदार्थानां । (७०)
	\pend
      \label{div_pvv.1.71}\edlabel{div_pvv.1.71}
	  
	% new div opening: depth here is 2
	

	  \begin{center}%% label @type='head'
	\textbf{(B. समवायसमवायिभावनिरासः)}
	\end{center}
	
	  \bigskip
	  \begingroup
	  \large
	
	    
	    \stanza[\smallbreak]
	\label{pv.1.71}\edlabel{pv.1.71}\flagstanza{\tiny\textenglish{...v.1.71}}एतेन समवायश्च समवायि च कारणम् ।&व्यवस्थितत्वं जात्यादेर्निरस्तमनपाश्रयात् ॥ ७१ ॥\&[\smallbreak]


	
	  \endgroup
	

	  \pstart {\color{DodgerBlue3}“एतेनाश्र”}याश्रयिभावप्रतिषेधेन {\color{DodgerBlue3}“समवा”}यो\edlabel{pvv.36-1}\footnote{\label{pvv.36-1}  १ अपृथक् सिद्धानां द्रव्यगुणादीनां ।} ऽयुतसिद्धानामाधार्याधारभूतानामिहेति प्रत्ययहेतुर्यथा व्यक्तिसामान्ययोः {\color{DodgerBlue3}“समवायिकारणञ्च”} स्वसमवेतकार्यजनकं यथात्मादि बुद्ध्यादीनां {\color{DodgerBlue3}“व्यवस्थितत्वं जात्यादेः”} कासुचिदेव व्यक्तिषु\edlabel{pvv.36-2}\footnote{\label{pvv.36-2}  २ गव्येव ।} गोत्वं वर्तते केषु\edlabel{pvv.36-3}\footnote{\label{pvv.36-3}  ३ आदिना ।} चिदेव च देहाकारपरिणतेषु चैतन्यमित्या\edlabel{pvv.36-4}\footnote{\label{pvv.36-4}  ४ आश्रयनिषेधे समवायादेव सिद्धेः । अंकुरादीनां वा ।} दि {\color{DodgerBlue3}“निरस्तमनपाश्रयादा-”} श्रयप्रतिषेधात् । तन्मूलत्वाच्चासां व्यवस्थानां । (७१)
	\pend
      \label{div_pvv.1.72}\edlabel{div_pvv.1.72}
	  
	% new div opening: depth here is 2
	

	  \begin{center}%% label @type='head'
	\textbf{(स्थितिस्थापकतानिरासे संग्रहश्लोकः)}
	\end{center}
	

	  \pstart उक्तमर्थं श्लोकत्रयेण संगृह्णन्नाह ।
	\pend
      
	  \bigskip
	  \begingroup
	  \large
	
	    
	    \stanza[\smallbreak]
	\label{pv.1.72}\edlabel{pv.1.72}\flagstanza{\tiny\textenglish{...v.1.72}}परतो भावनाशश्चेत् तस्य किं स्थितिहेतुना ।&स विनश्येद् विनाऽप्यन्यैरशक्ताः स्थितिहेतवः ॥ ७२ ॥\&[\smallbreak]


	
	  \endgroup
	

	  \pstart {\color{DodgerBlue3}“परतो”} मुद्गरादे{\color{DodgerBlue3}“र्भाव”}स्यानश्वरस्य {\color{DodgerBlue3}“नाशश्चेत् तस्य किं स्थितिहेतुना”}श्रयेण स्वयमनश्वरत्वादेव न नक्ष्यति । अथ नश्वरस्वभावोसौ तदा {\color{DodgerBlue3}“स विनश्येत विनाप्यन्यै”}र्नाशहेतुभिरं{\color{DodgerBlue3}“शक्ताः स्थितिहेतवो”} भावं नश्वरं स्थापयितुं नश्व\edlabel{pvv.36-5}\footnote{\label{pvv.36-5}  ५ सर्व्वः । नाशयति}वरस्वभावस्यावश्यं नाशात् । (७२)
	\pend
      \label{div_pvv.1.73}\edlabel{div_pvv.1.73}
	  
	% new div opening: depth here is 2
	

	  \pstart किञ्च (।)
	\pend
      
	  \bigskip
	  \begingroup
	  \large
	
	    
	    \stanza[\smallbreak]
	\label{pv.1.73a}\edlabel{pv.1.73a}\flagstanza{\tiny\textenglish{....1.73a}}स्थितिमान् साश्रयः सर्वः सर्वोत्पत्तौ च साश्रयः ।\&[\smallbreak]


	
	  \endgroup
	

	  \pstart {\color{DodgerBlue3}“स्थितिमान् साश्रयः सर्व्वो”} भावः । तत्र यो नाम कश्चि\edlabel{pvv.36-6}\footnote{\label{pvv.36-6}  ६ यो नामानित्याश्रयस्तस्याप्यन्यो यावन्नित्याः परमाणव इत्याह ।}न्नित्याश्रयो यथा सुखादिरात्माश्रितः । सर्व्वः स नित्यं स्थितिमान् स्यात् स्थापकस्य सदा स्थितेः । कश्चिदनित्याश्रयो यथा शुक्लत्वादिः कार्यद्रव्याश्रितः । {\color{DodgerBlue3}“सर्व्वोत्पत्ताव”}सर्व्वश्चोत्पद्यमानः । {\color{DodgerBlue3}“साश्रय”} इति द्रव्यादिरपि साश्रयः । तदाश्रयोपि कपालादिः खण्डोवयविषु समवेतः ।
	\pend
      \leavevmode\marginnote{\textenglish{037/s}}

	  \pstart ते चान्येष्विति यावत् परमाणव आश्रयावध\edlabel{pvv.37-1}\footnote{\label{pvv.37-1}  १ इति नित्यमेव स्थितिः ।}यः तेषां नित्यत्वात्तदाश्रितं द्व्यणुकं नित्यमित्यनया परम्परया गुणोपि नित्यः सादित्याह (।)
	\pend
      
	  \bigskip
	  \begingroup
	  \large
	
	    
	    \stanza[\smallbreak]
	\label{pv.1.73b}\edlabel{pv.1.73b}\flagstanza{\tiny\textenglish{....1.73b}}तस्मात् सर्वस्य भावस्य न विनाशः कदाचन ॥ ७३ ॥\&[\smallbreak]


	
	  \endgroup
	

	  \pstart {\color{DodgerBlue3}“तस्मात्सर्व्वस्य भावस्य”} बुद्ध्या\edlabel{pvv.37-2}\footnote{\label{pvv.37-2}  २ नित्यभूताश्रितत्वात् चैतन्यस्यापि ।}देः शुक्लादेश्च {\color{DodgerBlue3}“न विना\edlabel{pvv.37-3}\footnote{\label{pvv.37-3}  ३ अस्ति च नाश इति किमाश्रयस्वीकारेण ।}शः कदाचन”} प्राप्नोति । (७३)
	\pend
      \label{div_pvv.1.74}\edlabel{div_pvv.1.74}
	  
	% new div opening: depth here is 2
	

	  \pstart किञ्च (।)
	\pend
      
	  \bigskip
	  \begingroup
	  \large
	
	    
	    \stanza[\smallbreak]
	\label{pv.1.74}\edlabel{pv.1.74}\flagstanza{\tiny\textenglish{...v.1.74}}स्वयं विनश्वरात्मा चेत् तस्य कः स्थापकः परः ।&स्वयं न नश्वरात्मा चेत् तस्य कः स्थापकः परः ॥ ७४ ॥\&[\smallbreak]


	
	  \endgroup
	

	  \pstart {\color{DodgerBlue3}“स्वयम्विनश्व”}रात्मा चेत् भावस्तस्य {\color{DodgerBlue3}“कः स्थापकः पर”} आश्रया{\color{DodgerBlue3}“भिमतो न कश्चि”}दसामार्थ्यात् । {\color{DodgerBlue3}“स्वयं न नश्वरात्मा चेत् तस्य कः स्थापकः परः”} स्वयमविनाशितयैव स्थितेर्व्वैयर्थ्यात्स्थापकस्य । (७४)
	\pend
      \label{div_pvv.1.75}\edlabel{div_pvv.1.75}
	  
	% new div opening: depth here is 2
	

	  \begin{center}%% label @type='head'
	\textbf{(c. उपादानोपादेयभावनिरासः)}
	\end{center}
	

	  \pstart पुनश्चित्तशरीरयोरुपादानोपादेयतां निषेद्ध्ुमाह(।)
	\pend
      
	  \bigskip
	  \begingroup
	  \large
	
	    
	    \stanza[\smallbreak]
	\label{pv.1.75}\edlabel{pv.1.75}\flagstanza{\tiny\textenglish{...v.1.75}}बुद्धिव्यापारभेदेन निर्ह्रासातिशयावपि ।&प्रज्ञादेर्भवतो देहनिर्ह्रासातिशयौ विना ॥ ७५ ॥\&[\smallbreak]


	
	  \endgroup
	

	  \pstart {\color{DodgerBlue3}“बुद्धिव्यापारभेदेन”} न मनोज्ञानस्याभ्यासविशेषेण {\color{DodgerBlue3}“निर्ह्रासाति”}शयावु\edlabel{pvv.37-4}\footnote{\label{pvv.37-4}  ४ सन्मित्रासन्मित्रसंयोगादिना ।}पचयापचया{\color{DodgerBlue3}“वपि प्रज्ञादे”}रादिशब्दान्मैत्रीकरुणावैराग्या\edlabel{pvv.37-5}\footnote{\label{pvv.37-5}  ५ नाशः सहेतुको निर्हेतुको वा उभयथा स्थितिहेतुवैफल्यार्थ ।}दीनां {\color{DodgerBlue3}“भवतो देहस्य निर्ह्रासातिशयौ विना”} । तस्माद् बुद्धिरेवोपादानकारणं तद्विकारविकारित्वात् । न देहो विपर्ययात् । (७५)
	\pend
      \label{div_pvv.1.76}\edlabel{div_pvv.1.76}
	  
	% new div opening: depth here is 2
	
	  \bigskip
	  \begingroup
	  \large
	
	    
	    \stanza[\smallbreak]
	\label{pv.1.76a}\edlabel{pv.1.76a}\flagstanza{\tiny\textenglish{....1.76a}}इदं दीपप्रभादीनामाश्रितानां न विद्यते ॥\&[\smallbreak]


	
	  \endgroup
	

	  \pstart इदमाश्रयविकारं विनापि विकारित्वं {\color{DodgerBlue3}“दीपप्रभादी\edlabel{pvv.37-6}\footnote{\label{pvv.37-6}  ६ यत् प्रज्ञादीनां देहोत्कर्षापकर्षनिरपेक्षत्वं बुद्ध्यधीनत्वञ्च ।}नामाश्रितानां न विद्यते”} त\edlabel{pvv.37-7}\footnote{\label{pvv.37-7}  ७ तन्न दीपप्रभादीनां तन्न देहमाश्रय इत्यभिप्रायः ॥}द्विकारविकारित्वात् ।
	\pend
      \leavevmode\marginnote{\textenglish{038/s}}\leavevmode\marginnote{\textenglish{8b/MA}}

	  \pstart अथ देहादपि स्वस्थात्प्रज्ञादेरुत्कर्षो दृश्यत इति तद्विकारविकारित्वमस्त्येवेत्याह ।
	\pend
      
	  \bigskip
	  \begingroup
	  \large
	
	    
	    \stanza[\smallbreak]
	\label{pv.1.76b}\edlabel{pv.1.76b}\flagstanza{\tiny\textenglish{....1.76b}}स्यात् ततोऽपि विशेषोऽस्य न चित्तेऽनुपकारिणि ॥ ७६ ॥\&[\smallbreak]


	
	  \endgroup
	

	  \pstart {\color{DodgerBlue3}“स्यात् ततो”} देहा{\color{DodgerBlue3}“दपि विशेषो”}स्य प्रज्ञादे\edlabel{pvv.38-1}\footnote{\label{pvv.38-1}  १ न मुख्यः किन्तु परम्परया ।} {\color{DodgerBlue3}“र्न चित्तेऽनुपकारिणि”} । चित्तं हि स्वस्थदेहोपकृतसौमनस्यमभ्यासविशेषवत् प्रज्ञादिकमुत्कर्षयति । न गुणेपि चित्ते देह ए\edlabel{pvv.38-2}\footnote{\label{pvv.38-2}  २ उत्कर्षति ।}व । (७६)
	\pend
      \label{div_pvv.1.77}\edlabel{div_pvv.1.77}
	  
	% new div opening: depth here is 2
	

	  \pstart अमुमेव न्यायं रागादावाह ।
	\pend
      
	  \bigskip
	  \begingroup
	  \large
	
	    
	    \stanza[\smallbreak]
	\label{pv.1.77}\edlabel{pv.1.77}\flagstanza{\tiny\textenglish{...v.1.77}}रागादिवृद्धिः पुष्ट्यादेः कदाचित् सुखदुःखजा ।&तयोश्च धातुसाम्यादेरन्तरर्थस्य सन्निधेः ॥ ७७ ॥\&[\smallbreak]


	
	  \endgroup
	

	  \pstart या\edlabel{pvv.38-3}\footnote{\label{pvv.38-3}  ३ बुद्ध्युत्कर्षादिनिरपेक्षः केवलो देहो रागादिनिमित्तमित्याह ।}पि {\color{DodgerBlue3}“रागादिवृद्धिः पुष्ट\edlabel{pvv.38-4}\footnote{\label{pvv.38-4}  ४ धातुसाम्याद् वैषम्याच्चान्तःस्प्रष्टव्यविशेषेण कायविज्ञानमनुगृह्यते विक्रियते च तद्विकल्पं जनयति ततो रागद्वेषौ परम्परया कायात् ।}यादेः”} सापि न सर्व्वदापि तु {\color{DodgerBlue3}“कदाचित्”} प्रकृत्या मन्दरागस्य प्रतिसंख्यानबलिनश्च पुष्टस्यापि रागावृद्धेः । यदापि भवति तदापि न केवलात्पुष्ट्यादेः किन्तु {\color{DodgerBlue3}“सुखदुःखजा”} । सुखाद्रागः दुःखाद् द्वेषः इति न चित्तनिरपेक्षो रागादिहेतुर्देहः । {\color{DodgerBlue3}“तयोश्च”} सुखदुःखयो{\color{DodgerBlue3}“र्द्धातुसाम्यादेरन्तरर्थस्या”}नुग्राहकस्यान्तः स्प्रष्टव्यविशेषस्यान्तरस्य स्पर्शज्ञानविषयीकृतस्य {\color{DodgerBlue3}“सन्निधेर्जन्म”} । सुखदुःखज्ञाने अपि विशिष्टविषयपूर्व्वज्ञानसापेक्षे एव न देहमात्रजन्ये इत्यर्थः । (७७)
	\pend
      \label{div_pvv.1.78}\edlabel{div_pvv.1.78}
	  
	% new div opening: depth here is 2
	
	  \bigskip
	  \begingroup
	  \large
	
	    
	    \stanza[\smallbreak]
	\label{pv.1.78}\edlabel{pv.1.78}\flagstanza{\tiny\textenglish{...v.1.78}}एतेन सन्निपातादेः स्मृतिभ्रंशादयो गताः ।&विकारयति धीरेव ह्यन्तरर्थविशेषजा ॥ ७८ ॥\&[\smallbreak]


	
	  \endgroup
	

	  \pstart {\color{DodgerBlue3}“एते”}नान्तरोक्तन्यायेन {\color{DodgerBlue3}“सन्निपातादे”}रादिग्रहणाज्ज्वरादेः {\color{DodgerBlue3}“स्मृतिभ्रंशादयो गता”} व्याख्याताः । धीरेव हि पूर्व्विकाऽ{\color{DodgerBlue3}“न्तरर्थविशेषा”}द्धातुवैषम्याज्जाता तद्‏ग्राहि\edlabel{pvv.38-5}\footnote{\label{pvv.38-5}  ५ वैषम्योपप्लुता}णीं चित्तसन्ततिं {\color{DodgerBlue3}“विकारयति”} स्मृतिप्रमोषाद्युपहतां करोति । (७८)
	\pend
      \label{div_pvv.1.79}\edlabel{div_pvv.1.79}
	  
	% new div opening: depth here is 2
	
	  \bigskip
	  \begingroup
	  \large
	
	    
	    \stanza[\smallbreak]
	\label{pv.1.79}\edlabel{pv.1.79}\flagstanza{\tiny\textenglish{...v.1.79}}शार्दूलशोणितादीनां सन्तानातिशये क्वचित् ।&मोहादयः सम्भवन्ति श्रवणेक्षणतो यथा ॥ ७९ ॥\&[\smallbreak]


	
	  \endgroup
	\leavevmode\marginnote{\textenglish{039/s}}

	  \pstart यथा {\color{DodgerBlue3}“शार्दू\edlabel{pvv.39-1}\footnote{\label{pvv.39-1}  १ देहश्चित्तमुपकरोति चित्तं प्रज्ञादीन् । चित्तेऽनुपकारिणि सति तु न विशेषः । व्याघ्र इहेति श्रुत्वा बिभेति ।} लशोणितादीनां”} यथाक्रमं {\color{DodgerBlue3}“श्रवणेक्षणतः सन्तानातिशये क्वचिद्”} भीरुतमे {\color{DodgerBlue3}“मोहादय”} आदिशब्दाद् भयरोमहर्षादयो विषयविकृतबुद्धिद्वारेणैव {\color{DodgerBlue3}“सम्भवन्ति”} । न हि मोहादीनां शार्दूलशोणितादय उपादानकारणानि किन्तु विषयाः संन्तः परम्परयोपकारकाः तथा रागस्मृतिभ्रंशादयोपि बोद्धव्याः । (७९)
	\pend
      \label{div_pvv.1.80}\edlabel{div_pvv.1.80}
	  
	% new div opening: depth here is 2
	
	  \bigskip
	  \begingroup
	  \large
	
	    
	    \stanza[\smallbreak]
	\label{pv.1.80}\edlabel{pv.1.80}\flagstanza{\tiny\textenglish{...v.1.80}}तस्मात् स्वस्यैव संस्कारं नियमेनानुवर्तते ।&तन्नान्तरीयकं चित्तमतश्चित्तसमाश्रितम् ॥ ८० ॥\&[\smallbreak]


	
	  \endgroup
	

	  \pstart {\color{DodgerBlue3}“तस्मा”}त् स्व{\color{DodgerBlue3}“स्यैव संस्कारं नियमेनानुव\edlabel{pvv.39-2}\footnote{\label{pvv.39-2}  २ समानजातीयविकल्पविज्ञानस्य पूर्वकस्य ।}र्तते । तन्नान्तरीयकं चित्त”}मेषितव्य{\color{DodgerBlue3}“मतश्चित्तमाश्रितं”} चित्तं चित्तसँस्कारस्यैवानुवर्तनात् । देहसंस्कारन्तु व्यभिचरति प्रतिसंख्यानबलिनामित्युक्तं । (८०)
	\pend
      \label{div_pvv.1.81}\edlabel{div_pvv.1.81}
	  
	% new div opening: depth here is 2
	

	  \pstart चेतःशरीरयोर्भेदपक्षेणाश्रयाश्रयिभावो ऽयुक्त इति प्रतिपाद्य शक्तिपक्षेप्यभेदात्मके दोषमाह (।)
	\pend
      
	  \bigskip
	  \begingroup
	  \large
	
	    
	    \stanza[\smallbreak]
	\label{pv.1.81}\edlabel{pv.1.81}\flagstanza{\tiny\textenglish{...v.1.81}}यथा श्रुतादिसंस्कारः कृतश्चेतसि चेतसि ।&कालेन व्यज्यतेऽभेदात् स्याद् देहेपि ततो गुणः ॥ ८१ ॥\&[\smallbreak]


	
	  \endgroup
	

	  \pstart {\color{DodgerBlue3}“यथा श्रुतादिसंस्कारः कृतश्चेतसि”} पुनर्यथा-प्रबोधप्रत्ययं चेतसि {\color{DodgerBlue3}“कालेन क्रम-”} भाविना {\color{DodgerBlue3}“व्यज्यते”} तथा {\color{DodgerBlue3}“अभेदाच्चित्त”}शरीरयोः {\color{DodgerBlue3}“स्याद् देहेपि ततः”} संस्कारप्रबोधकात् प्रत्ययात् गुणोऽभिव्यक्तः (।) ततश्च देहं पश्यता {\color{DodgerBlue3}“श्रुतादिसंस्कारोपि”} तदात्मभूतो दृश्यो दृश्येत ।\edlabel{pvv.39-3}\footnote{\label{pvv.39-3}  ३ न च दृश्यते । श्रुतादिसंस्कारेण संस्क्रियमाणेपि मनोविज्ञाने मनो न देहसंस्कारः ।}तस्माद् देहस्याश्रयत्वप्रतिषेधात् तद्विनाशे चित्तविनाशो नेति जन्मपरम्परासु युक्तः कृपाभ्यासः । (८१)
	\pend
      \label{div_pvv.1.82}\edlabel{div_pvv.1.82}
	  
	% new div opening: depth here is 2
	

	  \begin{center}%% label @type='head'
	\textbf{(ग) पुनर्जन्मपरिग्रहः}
	\end{center}
	

	  \pstart कथं पुनर्जन्मपरिग्रह इत्याह (।)
	\pend
      
	  \bigskip
	  \begingroup
	  \large
	
	    
	    \stanza[\smallbreak]
	\label{pv.1.82}\edlabel{pv.1.82}\flagstanza{\tiny\textenglish{...v.1.82}}अनन्यसत्वनेयस्य हीनस्थानपरिग्रहः ।&आत्मस्नेहवतो दुःखसुखत्यागाप्तिवाञ्छया ॥ ८२ ॥\&[\smallbreak]


	
	  \endgroup
	\leavevmode\marginnote{\textenglish{040/s}}

	  \pstart {\color{DodgerBlue3}“अनन्यसत्वनेयस्य”} ईश्वरप्रतिषेधात् । {\color{DodgerBlue3}“हीनस्थानप\edlabel{pvv.40-1}\footnote{\label{pvv.40-1}  १ अपराधीनस्य मक्षिकाणामशुचिस्थानग्रहकामिनां स्त्रीकुणपशरीरादिपरिग्रहवत् ।}रिग्रहो”} गर्भस्याश्रयत्वेन स्वीकारः {\color{DodgerBlue3}“आत्मस्नेहवतः”} सतृष्णस्य दुःखे सुखमिति विपर्यासः । तस्य {\color{DodgerBlue3}“दुःखसुखयोर्यथाक्रमं त्यागाप्तिवाञ्छया”} । सतृष्णो हि दुःखे सुखमिति विपर्यस्तः आत्मनि स्निग्धो जन्माक्षेपककर्मवशात् सुखहेतुं गर्भस्थानं मन्यमानः परिगृह्णाति । (८२)
	\pend
      \label{div_pvv.1.83}\edlabel{div_pvv.1.83}
	  
	% new div opening: depth here is 2
	

	  \begin{center}%% label @type='head'
	\textbf{I. अविद्या-तृष्णे बन्धकारणम्}
	\end{center}
	

	  \pstart ततश्च (।)
	\pend
      
	  \bigskip
	  \begingroup
	  \large
	
	    
	    \stanza[\smallbreak]
	\label{pv.1.83}\edlabel{pv.1.83}\flagstanza{\tiny\textenglish{...v.1.83}}दुःखे विपर्यासमतिः तृष्णा चाबन्धकारणम् ।&जन्मिनो यस्य ते न स्तो न स जन्माधिगच्छति ॥ ८३ ॥\&[\smallbreak]


	
	  \endgroup
	

	  \pstart {\color{DodgerBlue3}“दुः\edlabel{pvv.40-2}\footnote{\label{pvv.40-2}  २ दुःखे गर्भादिस्थानेभिरतिः सुखमेतदिति ।}खे विपर्यासमतिस्तृष्णा चाबन्धकारणं”} आश्लेषहेतु{\color{DodgerBlue3}“र्जन्मिनः”} । तृष्णया आत्मस्नेहोप्याक्षिप्तो हेतुवद्वेदितव्यः । यस्य तून्मू\edlabel{pvv.40-3}\footnote{\label{pvv.40-3}  ३ परिणतोऽभिरतिपुरःसरः प्रपातपातादिविलक्षणः ।}लितात्मग्रहस्य {\color{DodgerBlue3}“ते”} विपर्यासस्तृष्णा {\color{DodgerBlue3}“च न स्तो”} न विद्येते {\color{DodgerBlue3}“न स जन्माधिगच्छति”} ॥ (८३)
	\pend
      \label{div_pvv.1.84}\edlabel{div_pvv.1.84}
	  
	% new div opening: depth here is 2
	

	  \begin{center}%% label @type='head'
	\textbf{II. गत्यागत्योरदर्शनं इन्द्रियापाटवात्}
	\end{center}
	
	  \bigskip
	  \begingroup
	  \large
	
	    
	    \stanza[\smallbreak]
	\label{pv.1.84}\edlabel{pv.1.84}\flagstanza{\tiny\textenglish{...v.1.84}}गत्यागती न दृष्टे चेदिन्द्रियाणामपाटवात् ।&अदृष्टिर्मन्दनेत्रस्य तनुधूमागतिर्यथा ॥ ८४ ॥\&[\smallbreak]


	
	  \endgroup
	

	  \pstart भाविजन्मन्यतीताच्च जन्मनो यथाक्रमं {\color{DodgerBlue3}“गत्यागती न दृष्टे चेत् । इन्द्रियाणामपाटवात्सा अदृष्टि”}रदर्शनं ते सूक्ष्मस्यान्तराभवशरीरस्य । {\color{DodgerBlue3}“मन्दनेत्र\edlabel{pvv.40-4}\footnote{\label{pvv.40-4}  ४ किमिवेत्याह उपहतचक्षुषः ।}स्य”} पुंसस्त\edlabel{pvv.40-5}\footnote{\label{pvv.40-5}  ५ विरलः ।} {\color{DodgerBlue3}“नधू”}मस्यागतिरदर्शनं {\color{DodgerBlue3}“यथा”} । न ह्यदृश्यस्यादर्शनादभावः । (८४)
	\pend
      \label{div_pvv.1.85}\edlabel{div_pvv.1.85}
	  
	% new div opening: depth here is 2
	

	  \begin{center}%% label @type='head'
	\textbf{III. मूर्त्तस्याऽपि मूर्त्ते प्रवेशः}
	\end{center}
	\leavevmode\marginnote{\textenglish{9a/MA}}

	  \pstart ननु मूर्त्तं न मूर्त्तान्तरमनुप्रविशति प्रतिघातात्(।) मूर्त्तञ्चान्तराभवशरीरमित्य\edlabel{pvv.40-6}\footnote{\label{pvv.40-6}  ६ कथं गर्भ प्रविशति ।}त आह (।)
	\pend
      
	  \bigskip
	  \begingroup
	  \large
	
	    
	    \stanza[\smallbreak]
	\label{pv.1.85}\edlabel{pv.1.85}\flagstanza{\tiny\textenglish{...v.1.85}}तनुत्वान्मूर्तमपि तु किञ्चित् क्वचिदशक्तिमत् ।&जलवत् सूतवद्धेम्नि नादृष्टेनासदेव वा ॥ ८५ ॥\&[\smallbreak]


	
	  \endgroup
	\leavevmode\marginnote{\textenglish{041/s}}

	  \pstart {\color{DodgerBlue3}“तनुत्वान्मूर्त्तमपि तु किञ्चित्क्वचि”}\edlabel{pvv.41-1}\footnote{\label{pvv.41-1}  १ प्रविशति ।}न्मूर्त्ते{\color{DodgerBlue3}“ऽशक्तिमद”}प्रतिघातवत् {\color{DodgerBlue3}“जलवत्”} घटे {\color{DodgerBlue3}“सूतवद्”} हेम्नि । जलसूतौ हि मूर्त्तावपि घटहेम्नी भिन्दन्तौ दृश्ये\edlabel{pvv.41-2}\footnote{\label{pvv.41-2}  २ सूर्यरश्मयश्च स्फटिकं भित्वेन्धनं विशन्तीत्यनेकान्तः ।}तेऽ(?अ)न्तराभवशरीरमशक्तिमत् न दृष्टमिति चेत् {\color{DodgerBlue3}“नाद”}ष्टेना{\color{DodgerBlue3}“सदेव”} वा भवति तादृशमन्तराभवशरीरं । (८५)
	\pend
      \label{div_pvv.1.86}\edlabel{div_pvv.1.86}
	  
	% new div opening: depth here is 2
	

	  \begin{center}%% label @type='head'
	\textbf{IV. नोपादानभूतं शरीरं बुद्धेः}
	\end{center}
	

	  \begin{center}%% label @type='head'
	\textbf{A. अवयविनिरासः}
	\end{center}
	

	  \pstart किञ्च (।) शरीरमुपादानं बुद्धेर्भवदे\edlabel{pvv.41-3}\footnote{\label{pvv.41-3}  ३ निरवयवं ।}कमवयविरूपं वा स्यात् । अनेकं परमाणुसञ्चयस्वभावं वा । तत्रावयविरूपं किमवयवा एव हस्तादय उत तेभ्योऽन्यत् (।) द्वयमपि प्रतिषेद्ध्ुमाह\edlabel{pvv.41-4}\footnote{\label{pvv.41-4}  ४ एकञ्चेत्} ।
	\pend
      
	  \bigskip
	  \begingroup
	  \large
	
	    
	    \stanza[\smallbreak]
	\label{pv.1.86}\edlabel{pv.1.86}\flagstanza{\tiny\textenglish{...v.1.86}}पाण्यादिकम्पे सर्वस्य कम्पप्राप्तेर्विरोधिनः ।&एकस्मिन् कर्मणोऽयोगात् स्यात् पृथक् सिद्धिरन्यथा ॥ ८६ ॥\&[\smallbreak]


	
	  \endgroup
	

	  \pstart {\color{DodgerBlue3}“पाण्यादिकम्पे”} सर्व्वस्य कम्पप्राप्तेः । यदि पाण्यादयोऽवयवा एव अवयव्येकरूपः तदा पाण्यादेः कम्पे सति सर्व्वस्य पादादेरपि कम्पः प्राप्नोति । {\color{DodgerBlue3}“एक”}स्मिँस्तस्मिन् {\color{DodgerBlue3}“कर्मणः”} कम्पस्य विरोधिनोऽकम्पस्या{\color{DodgerBlue3}“योगात्”} । एकन्तु द्रव्यं तत्समवेतश्च कम्प इति सर्व्वं कम्पेत । अवयवानामेकावयविरूपत्वं हेतुः पराभ्युपगमसिद्धः । सर्व्वस्य कम्पप्रसङ्गः । न च कम्पोस्ति इति साध्याभावेनैकावयविरूपत्वाभावप्रसङ्गविपर्ययः । एवम्वक्ष्यमाणावपि प्रसङ्गतद्विपर्ययौ वेदितव्यौ ॥ {\color{DodgerBlue3}“अथावयवेभ्यो”} भिन्नोऽवयवी । अत एवैकस्मिन्नवयवे कम्पमाने नावयवान्तरस्य कम्पः तदापि {\color{DodgerBlue3}“स्यात् पृथ”}क्{\color{DodgerBlue3}“‏सिद्धिरन्य”}थाऽवयवावयविनोर्भेदे पृथक्‏कम्पमानादवयवादकम्पमानस्यावयविनः समवेतस्य भेदेन तत्रै\edlabel{pvv.41-5}\footnote{\label{pvv.41-5}  ५ यथा एकदेशलग्नमुदकं वस्त्रैकदेश एव दृश्यते तद्वत् ।}वावयवे सिद्धिः स्यात् वस्त्रो\edlabel{pvv.41-6}\footnote{\label{pvv.41-6}  ६ पृथग्‏भावः स्यादवयवावयविनोः अवयवी भेदेन दृश्येत ।}दकवत् । अत्राप्यवयवावयविनोर्भेदः पराभ्युपगमसिद्धो हेतुः । पृथक्‏सिद्धिः प्रसज्यते । साध्याभावे साध\edlabel{pvv.41-7}\footnote{\label{pvv.41-7}  ७ सर्वाकम्पार्थं पृथक् स्वीकृतिः पृभगवयव्यसिद्धौ अवयवावयविभेदस्य साधनस्याभावः । ततः सर्व्वकम्पप्रसङ्गस्यापरिहारः ।}नाभावो विपर्ययः । एवम्वक्ष्यमाणौ च प्रसङ्गविपर्ययौ । (८६)
	\pend
      \label{div_pvv.1.87}\edlabel{div_pvv.1.87}
	  
	% new div opening: depth here is 2
	
	  \bigskip
	  \begingroup
	  \large
	
	    
	    \stanza[\smallbreak]
	\label{pv.1.87}\edlabel{pv.1.87}\flagstanza{\tiny\textenglish{...v.1.87}}एकस्य चावृतौ सर्वस्यावृतिः स्यादनावृतौ ।&दृश्येत रक्ते चैकस्मिन् रागोऽरक्तस्य वाऽगतिः ॥ ८७ ॥\&[\smallbreak]


	
	  \endgroup
	\leavevmode\marginnote{\textenglish{042/s}}

	  \pstart अथाभेदपक्षे {\color{DodgerBlue3}“एकस्या”}वयवस्या{\color{DodgerBlue3}“वृतौ सर्व्वस्यावृतिश्च स्यादि”}ति प्रसङ्गः । भेदपक्षमाश्रित्या{\color{DodgerBlue3}“नावृ”}तौ चावयविनः स्वीक्रियमाणायामावृत एवावयवेऽनावृतोसौ दृश्येतेति प्रसङ्गः । अथाभेदपक्षे {\color{DodgerBlue3}“रक्ते चैकस्मिन्न”}वयवे सर्व्वत्रावयवे {\color{DodgerBlue3}“रागो”} दृश्येतेति प्रसङ्गः । भेदपक्षे तु रक्त एवावयवेऽ{\color{DodgerBlue3}“रक्तस्य”} चावयविनो {\color{DodgerBlue3}“वाऽगतिः”} स्यादिति {\color{DodgerBlue3}“प्रसङ्गः”} । (८७)
	\pend
      \label{div_pvv.1.88_1.89}\edlabel{div_pvv.1.88_1.89}
	  
	% new div opening: depth here is 2
	

	  \pstart सर्व्वत्र साध्याभावेन साधनाभावः प्रसङ्गविपर्ययः । तमाह (।)
	\pend
      
	  \bigskip
	  \begingroup
	  \large
	
	    
	    \stanza[\smallbreak]
	\label{pv.1.88a}\edlabel{pv.1.88a}\flagstanza{\tiny\textenglish{....1.88a}}नास्त्येकसमुदायोऽस्मादनेकत्वेऽपि पूर्ववत् ।\&[\smallbreak]


	
	  \endgroup
	

	  \pstart {\color{DodgerBlue3}“नास्त्येक”}स्मिन् नास्त्येकोऽवयवी {\color{DodgerBlue3}“समुदा”}योऽवयवाना{\color{DodgerBlue3}“मस्मा”}त्कम्पादिसाध्याभावात् । अ\edlabel{pvv.42-1}\footnote{\label{pvv.42-1}  १ आवृतावयवादवयविभेदपक्षेपि । अवयविदर्शनप्रसङ्गस्य साध्यस्याभावेनावृताद् भेदस्य साधनस्याभावो विपर्ययः । ततश्चाभेदपक्ष इव सर्व्वावरणादि स्यात् ।}वयवावयविनोर{\color{DodgerBlue3}“नेकत्वेपि पूर्व्ववत्”} । अभेदपक्ष इव साध्याभावेन साधनाभावो विपर्ययः ।
	\pend
      
	  \bigskip
	  \begingroup
	  \large
	
	    
	    \stanza[\smallbreak]
	\label{pv.1.88b}\edlabel{pv.1.88b}\flagstanza{\tiny\textenglish{....1.88b}}अविशेषादणुत्वाच्च न गतिश्चेन्न सिध्यति ॥ ८८ ॥\&[\smallbreak]


	
	  \endgroup
	
	  \bigskip
	  \begingroup
	  \large
	
	    
	    \stanza[\smallbreak]
	\label{pv.1.89a}\edlabel{pv.1.89a}\flagstanza{\tiny\textenglish{....1.89a}}अविशेषः;\&[\smallbreak]


	
	  \endgroup
	

	  \pstart अथ शरीरादौ प्रत्यक्षदृष्टे धर्मिणि कम्पाकम्पादिविरुद्धधर्म्माध्या\edlabel{pvv.42-2}\footnote{\label{pvv.42-2}  २ अत्रोद्योतकरादय आहुर्यदि}सात् । स्वतन्त्र\edlabel{pvv.42-3}\footnote{\label{pvv.42-3}  ३ सिद्धान्त ।}हेतोरेकत्वप्रतिषेधः साधनीय इति । अवयविनोऽभावात् परमाणुपुंजरूपं शरीरादि तदपि परस्परसङ्गमावस्थातः\edlabel{pvv.42-4}\footnote{\label{pvv.42-4}  ४ सकाशात् ।} पुञ्जावस्थाया{\color{DodgerBlue3}“मविशेषाद्”} विशेषाभावात् {\color{DodgerBlue3}“अणुत्वाच्च”} । दर्शनानर्हसूक्ष्मतयापि शरीरादेर्न {\color{DodgerBlue3}“गतिश्चेत्\edlabel{pvv.42-5}\footnote{\label{pvv.42-5}  ५ अस्माकं मते}। न सिध्यति (८८) अविशेषः”} ।
	\pend
      

	  \begin{center}%% label @type='head'
	\textbf{(a. आवरणाद्यभावनिरासः)}
	\end{center}
	
	  \bigskip
	  \begingroup
	  \large
	
	    
	    \stanza[\smallbreak]
	\label{pv.1.89b}\edlabel{pv.1.89b}\flagstanza{\tiny\textenglish{....1.89b}}विशिष्टानामैन्द्रियत्वमतोऽनणुः ।\&[\smallbreak]


	
	  \endgroup
	

	  \pstart परस्परासङ्गतेभ्यः प\edlabel{pvv.42-6}\footnote{\label{pvv.42-6}  ६ क्षणिकत्वात् ।}रमाणुभ्योऽदृष्टसहकारिभ्यो दृश्यानामेवान्योन्यसंहतानामुत्पत्तेः । तेषां {\color{DodgerBlue3}“विशिष्टानामैन्द्रिय”}त्वमिन्द्रियग्राह्यत्वं अत\edlabel{pvv.42-7}\footnote{\label{pvv.42-7}  ७ येऽतीन्द्रिया अणवो न त एव पश्चात् किन्तु तानाश्रित्यान्य एव विशिष्टा उत्पद्यन्ते ।} ऐन्द्रियत्वा{\color{DodgerBlue3}“दनणु”}रिष्टः (।) इन्द्रियागोचरेष्वणुत्वं प्रसिद्धं तदेव तु हेतुकृतं । पुञ्जीभूतास्तु दृश्यमाना नाणव \leavevmode\marginnote{\textenglish{043/s}} उच्यन्ते किन्तु शरीरादिव्यपदेश्याः । यथा तन्तवः पटावस्था न तन्तवः उच्यन्तेऽपि तु पट इत्यसिद्धौ हेतू\edlabel{pvv.43-1}\footnote{\label{pvv.43-1}  १ अविशेषादणुत्वाच्चेति परोक्तौ ।} ।
	\pend
      
	  \bigskip
	  \begingroup
	  \large
	
	    
	    \stanza[\smallbreak]
	\label{pv.1.89c}\edlabel{pv.1.89c}\flagstanza{\tiny\textenglish{....1.89c}}एतेनावरणादीनामभावश्च निराकृतः ॥ ८९ ॥\&[\smallbreak]


	
	  \endgroup
	

	  \pstart {\color{DodgerBlue3}“एतेन”} परमाणूनां पूर्व्वावस्थातो विशिष्टत्वकथनेनावर\edlabel{pvv.43-2}\footnote{\label{pvv.43-2}  २ अवयवित्वायोगात् ।} {\color{DodgerBlue3}“णाधारादीनामभावः”}\leavevmode\marginnote{\textenglish{9b/MA}} परैरसंहतावस्थायामिव यः प्रतिपादिततः {\color{DodgerBlue3}“स च निराकृतो”} बोद्धव्यः । {\color{DodgerBlue3}“केचित् परमाणवः”} संहता जाता आवरणधारणादिक्षमा भवन्ति नान्ये । (८९)
	\pend
      \label{div_pvv.1.90}\edlabel{div_pvv.1.90}
	  
	% new div opening: depth here is 2
	
	  \bigskip
	  \begingroup
	  \large
	
	    
	    \stanza[\smallbreak]
	\label{pv.1.90}\edlabel{pv.1.90}\flagstanza{\tiny\textenglish{...v.1.90}}कथं वा सूतहेमादिमिश्रं तप्तोपलादि वा ।&दृश्यं पृथगशक्तानामक्षादीनां गतिः कथम् ॥ ९० ॥\&[\smallbreak]


	
	  \endgroup
	

	  \pstart {\color{DodgerBlue3}“कथम्वा सूतहेमादिमिश्रं”} पिष्टिकावस्थायां तेजःपरमा\edlabel{pvv.43-3}\footnote{\label{pvv.43-3}  ३ यथा यद्येकं द्रव्यं नेष्यते तदा यथा तेषां पूर्व्वावस्थायां परमाणुरूपं नावृनो/?/ति न दधाति च तथा पश्चादप्यविशेषान्नावृणुयान्न दध्याच्चेति निराकृतं । तैजसाः परमाणवः काष्ठोपलादिपरमाणुभिः परस्परसंभेदेनाविशन्तीति परः ।}णुसञ्चयरूपे {\color{DodgerBlue3}“तप्तो”}पलादि वा मिश्रं कथं दृश्यं विजातीयानां द्रव्यानारम्भात् न तदवयवि द्रव्यं । परमाणवश्च त्वन्मते न दृश्या इति न तेषां दर्शनं स्यात् । यदि च परमाणवः पृथगवस्थायां ज्ञानजननेऽशक्ता इति पुञ्जावस्था अपि तथा । तदा {\color{DodgerBlue3}“पृथगशक्तानामक्षादीनां”}\edlabel{pvv.43-4}\footnote{\label{pvv.43-4}  ४ चक्षूरूपालोकमनसिकाराणां ।} ज्ञानजनने संहतौ {\color{DodgerBlue3}“गतिर्ज्ञानं कथं”} । (९०)
	\pend
      \label{div_pvv.1.91}\edlabel{div_pvv.1.91}
	  
	% new div opening: depth here is 2
	
	  \bigskip
	  \begingroup
	  \large
	
	    
	    \stanza[\smallbreak]
	\label{pv.1.91}\edlabel{pv.1.91}\flagstanza{\tiny\textenglish{...v.1.91}}संयोगाच्चेत् समानोऽत्र प्रसङ्गो हेमसूतयोः ।&दृश्यः संयोग इति चेत् कुतोऽदृश्याश्रये गतिः ॥ ९१ ॥\&[\smallbreak]


	
	  \endgroup
	

	  \pstart तस्मादनैकान्तिकमपि हेतुद्वयं नेन्द्रियादेर्ज्ञानजन्म किन्तु संयो\edlabel{pvv.43-5}\footnote{\label{pvv.43-5}  ५ आत्मा मनसा युज्यते मनः इन्द्रियेण इन्द्रियमर्थेन ततो ज्ञानोत्पत्तिः ।}गात्तदीयादिति चेत् । {\color{DodgerBlue3}“समानोऽत्र प्रसङ्गः”} । यथा पृथगिन्द्रियादयः {\color{DodgerBlue3}“संयोगन्न जनयन्ति । तथा”}मिलिता अपि न जनयेयुः । {\color{DodgerBlue3}“हेमसूतयोर्दृश्यः संयोगो”} दृश्यत {\color{DodgerBlue3}“इति चेत् । कुतोऽदृश्याश्रये गतिः”} । हेमसूतपरमाणवो हि संयोगस्याश्रयास्तेषामदृश्यत्वे कथन्तदाश्रितस्य संयोगस्य दर्शनं (।) न हि कश्चित् पिशाचयोः संयोगमुपलभते (।) किञ्च (।) नानाद्रव्यारब्धपानकादिरपि संयोगो गुण एषितव्यः । तस्मिन् {\color{DodgerBlue3}“मृष्टं पानकं स्वरूपं”} पानकमिति निर्गुणत्वाद् गुणानां विरुद्धः ॥ (९१)
	\pend
      \label{div_pvv.1.92}\edlabel{div_pvv.1.92}
	  
	% new div opening: depth here is 2
	
	  \bigskip
	  \begingroup
	  \large
	
	    
	    \stanza[\smallbreak]
	\label{pv.1.92}\edlabel{pv.1.92}\flagstanza{\tiny\textenglish{...v.1.92}}रसरूपादियोगश्च संयोग उपचारतः ।&इष्टश्चेद् बुद्धिभेदोऽस्तु पंक्तिर्दीर्घेति वा कथम् ॥ ९२ ॥\&[\smallbreak]


	
	  \endgroup
	\leavevmode\marginnote{\textenglish{044/s}}

	  \pstart नानागुणिद्रव्येषु संयोगो ॥ रसरूपादयश्च समवेता इत्येकार्थसमवायात्तद्धर्मस्य पानकादिषूप\edlabel{pvv.44-1}\footnote{\label{pvv.44-1}  १ द्रव्यधर्मे गुणे ।} {\color{DodgerBlue3}“चारतो रसरूपादियोग इष्टश्चेत्”} । तर्हि क्षीरा\edlabel{pvv.44-2}\footnote{\label{pvv.44-2}  २ आदिना क्षीरोदकादि ।}दौ पानकादौ च मुख्यामुख्यत्वेन स्पष्टास्पष्टतया मृष्टादि{\color{DodgerBlue3}“बुद्धेर्भेदः”} स्यात्\edlabel{pvv.44-3}\footnote{\label{pvv.44-3}  ३ यदि यत्र परमाणुषु रसादयः समवेतास्तत्र संयोगोपीति संयोगो रसोपचारस्तर्हि क्षारादिवत्पानके रसबुद्धिर्माभूदस्ति च ।} । न हि माणवकेऽग्निबुद्धिरुपचारादग्नाविव भवति । {\color{DodgerBlue3}“पंक्तिर्दीर्घेति वा कथं”}\edlabel{pvv.44-4}\footnote{\label{pvv.44-4}  ४ यद्येकार्थसमवायात्तद्धर्मोपचारतो उपदेशः पंक्तिः संयोगत्वाद् गुणः । न तत्र दैर्ध्य गुणोस्ति नापि पक्षिषु कथमुपचारतोपि निर्द्देशः । न हि प्रत्येकं पङिक्तषु दैर्ध्यमस्ति तथाभूतं । येनैकार्थसमवाय उपचारवीजं स्यात् ।} । पंक्तिषु समवेता पङ‏क्तिः सं\edlabel{pvv.44-5}\footnote{\label{pvv.44-5}  ५ यद्यवयविनं विनाऽदर्शनं तदा कथं सूतहेममिश्रं दृश्यते । न हि तद्‏द्रव्यारम्भकं विजातीयानारम्भात् अन्यथा सर्व सर्व्वैरारभ्येतेति सिद्धान्तः ।}ख्या गुणो वा स्यात्(।) न च तत्र दैर्ध्यमस्ति निर्गुणत्वाद् गुणानां (।) नापि पङ‏क्तिषु दैर्ध्यमस्ति येनैकार्थसमवायादुपचारः स्यात् (। ९२)
	\pend
      \label{div_pvv.1.93}\edlabel{div_pvv.1.93}
	  
	% new div opening: depth here is 2
	

	  \begin{center}%% label @type='head'
	\textbf{(b. संख्यादिनिरासः)}
	\end{center}
	

	  \pstart एतच्चाभ्युपगम्योक्तं न तु संख्यादयः सन्ति तदाह (।)
	\pend
      
	  \bigskip
	  \begingroup
	  \large
	
	    
	    \stanza[\smallbreak]
	\label{pv.1.93}\edlabel{pv.1.93}\flagstanza{\tiny\textenglish{...v.1.93}}संख्यासंयोगकर्मादेरपि तद्वत्-स्वरूपतः ।&अभिलापाच्च भेदेन रूपं बुद्धौ न भासते ॥ ९३ ॥\&[\smallbreak]


	
	  \endgroup
	

	  \pstart {\color{DodgerBlue3}“संख्यासंयोगकर्म्मादे”}(ः।) आदिशब्दाद्विभागपरत्वापरत्वसामान्यादेरपि {\color{DodgerBlue3}“तद्वत्स्वरूपतो”} द्रव्यस्वभावाद् {\color{DodgerBlue3}“भेदेनाभिलापाच्च”} । संख्यासंयोग इत्यादिकात् । {\color{DodgerBlue3}“बुद्धौ”} द्रव्यग्राहिण्यां {\color{DodgerBlue3}“रूपं न भासते”}ऽभासमानञ्च दृश्याभिमतं दृश्यानुपलब्धया बाधितं ॥ (९३)
	\pend
      \label{div_pvv.1.94}\edlabel{div_pvv.1.94}
	  
	% new div opening: depth here is 2
	

	  \pstart यदि संख्यादयो न सन्ति कथमेको घटः संयुक्तो महान् पततीत्यादि व्यपदिश्यते ऽर्थभेदाभावे पर्य्यायता प्राप्नोतीत्यत आह ।
	\pend
      
	  \bigskip
	  \begingroup
	  \large
	
	    
	    \stanza[\smallbreak]
	\label{pv.1.94}\edlabel{pv.1.94}\flagstanza{\tiny\textenglish{...v.1.94}}शब्दज्ञाने विकल्पेन वस्तुभेदानुसारिणा ।&गुणादिष्विव कल्प्यार्थे नष्टाजातेषु वा यथा ॥ ९४ ॥\&[\smallbreak]


	
	  \endgroup
	

	  \pstart {\color{DodgerBlue3}“शब्दज्ञाने”} एको घट इत्यादिके कल्प्यार्थे कल्पितार्थे {\color{DodgerBlue3}“विकल्पेन”} (।) कीदृशेन {\color{DodgerBlue3}“वस्तुभेदानुसारिणा”} वस्तुनो भेदो विजातीयाद् व्यावृत्तिस्ताम्विषयत्वेनानुसरता {\color{DodgerBlue3}“गुणादि”}\leavevmode\marginnote{\textenglish{045/s}} {\color{DodgerBlue3}“ष्विव”} यथा पङ्क्त्यादौ {\color{DodgerBlue3}“एका महती गच्छतीत्यादिशब्दज्ञाने । न हि तत्र संख्यादयः”} सन्ति निर्गुणत्वाद् गुणानां । {\color{DodgerBlue3}“नष्टाजातेषु वा यथा । एको द्वौ बहवो नष्टा भवि-”} ष्यन्ति वेति शब्दज्ञानवत् । {\color{DodgerBlue3}“नष्टमजातञ्च स्वयमेव नास्ति किं पुनस्तत्र संख्यादयो”} भविष्यन्ति । (९४)
	\pend
      \label{div_pvv.1.95}\edlabel{div_pvv.1.95}
	  
	% new div opening: depth here is 2
	
	  \bigskip
	  \begingroup
	  \large
	
	    
	    \stanza[\smallbreak]
	\label{pv.1.95}\edlabel{pv.1.95}\flagstanza{\tiny\textenglish{...v.1.95}}मतो यद्युपचारोऽत्र स इष्टो यन्निबन्धनः ।&स एव सर्वभावेषु हेतुः किं नेष्यते तयोः ॥ ९५ ॥\&[\smallbreak]


	
	  \endgroup
	

	  \pstart {\color{DodgerBlue3}“मतो यद्युपचारोत्र”} गुणादिषु संख्यादिव्यपदेशस्य {\color{DodgerBlue3}“स”} उपचार {\color{DodgerBlue3}“इष्टो यन्निबन्धनो”}\edlabel{pvv.45-1}\footnote{\label{pvv.45-1}  १ लोकव्यवहारनिबन्धनः ।} गुणादिषु संख्याद्युपचारस्य हेतुर्यः {\color{DodgerBlue3}“स एव सर्व्वभावेषु हेतुः किन्नेष्यते तयोः”} शब्दज्ञानयोः येन संख्यादयः कल्प्यन्ते प्रमाणबाधिताः ॥ (९५)
	\pend
      \label{div_pvv.1.96}\edlabel{div_pvv.1.96}
	  
	% new div opening: depth here is 2
	
	  \bigskip
	  \begingroup
	  \large
	
	    
	    \stanza[\smallbreak]
	\label{pv.1.96}\edlabel{pv.1.96}\flagstanza{\tiny\textenglish{...v.1.96}}उपचारो न सर्वत्र यदि भिन्नविशेषणम् ।&मुख्यमित्येव च कुतोऽभिन्नेऽभिन्नार्थतेति चेत् ॥ ९६ ॥\&[\smallbreak]


	
	  \endgroup
	

	  \pstart {\color{DodgerBlue3}“उपचारो न सर्व्वत्र यदि”} मुख्ये सत्युपचारो भवति\edlabel{pvv.45-2}\footnote{\label{pvv.45-2}  २ पृष्टः सिद्धान्तिनैकं मुख्यमिति आह ।}न तु सर्व्वत्रैवा\edlabel{pvv.45-3}\footnote{\label{pvv.45-3}  ३ तादृशं ते मुख्यलक्षणं । अविशिष्टं स्वतो द्रव्यं विशिष्यते संख्यादिना तत् विशेषणं तस्य भिन्नं ।}सौ ॥ {\color{DodgerBlue3}“भिन्नविशेषणं”} मुख्यं यत्र भिन्नं विशेषणमस्ति {\color{DodgerBlue3}“तन्मुख्यं अन्यत्रोपचारः । भिन्नविशेषणं”}\leavevmode\marginnote{\textenglish{10a/MA}} {\color{DodgerBlue3}“मुख्यमित्ये”}तदेव {\color{DodgerBlue3}“कुतो”} निश्चितं {\color{DodgerBlue3}“अभिन्नेऽभिन्नार्थतेति चेत्”} । (९६)
	\pend
      \label{div_pvv.1.97}\edlabel{div_pvv.1.97}
	  
	% new div opening: depth here is 2
	

	  \pstart यदि संख्यादि भिन्नं नास्ति तदा घट एकः {\color{DodgerBlue3}“संयुक्तो महान् गच्छतीति शब्दानां”} पर्यायता स्यात् । न चास्ति ॥
	\pend
      
	  \bigskip
	  \begingroup
	  \large
	
	    
	    \stanza[\smallbreak]
	\label{pv.1.97}\edlabel{pv.1.97}\flagstanza{\tiny\textenglish{...v.1.97}}अनर्थान्तरहेतुत्वेप्यपर्यायः सितादिषु ।&संख्यादियोगिनः शब्दास्तत्राप्यर्थान्तरं यदि ॥ ९७ ॥\&[\smallbreak]


	
	  \endgroup
	

	  \pstart {\color{DodgerBlue3}“नन्व\edlabel{pvv.45-4}\footnote{\label{pvv.45-4}  ४ तस्मान्नार्थविषयाः सिद्धा । अत्र बौद्ध आहानेकान्ततां ।}नर्थान्तरहेतुत्वेप्यपर्यायः सितादि”}गुणे\edlabel{pvv.45-5}\footnote{\label{pvv.45-5}  ५ गुणेन गुणान्तरमित्यर्थान्तराभावात् यथैकं शुक्लं संयुक्तं विभक्तमित्यादि ।}षु {\color{DodgerBlue3}“संख्या\edlabel{pvv.45-6}\footnote{\label{pvv.45-6}  ६ अर्थान्तरनिमित्तं ।}दियोगिनः शब्दा”} दृश्यन्ते । {\color{DodgerBlue3}“अत्राप्यर्थान्तरं यदि”} । सितादौ संख्यादि तिष्ठति । {\color{DodgerBlue3}“ततस्तच्छब्दानाम”}पर्यायतापत्तिः । (९७)
	\pend
      \label{div_pvv.1.98}\edlabel{div_pvv.1.98}
	  
	% new div opening: depth here is 2
	\leavevmode\marginnote{\textenglish{046/s}}
	  \bigskip
	  \begingroup
	  \large
	
	    
	    \stanza[\smallbreak]
	\label{pv.1.98}\edlabel{pv.1.98}\flagstanza{\tiny\textenglish{...v.1.98}}गुणद्रव्याविशेषः स्याद् भिन्नो व्यावृत्तिभेदतः ।&स्यादनर्थान्तरार्थत्वेप्यकर्माद्रव्यशब्दवत् ॥ ९८ ॥\&[\smallbreak]


	
	  \endgroup
	

	  \pstart एवञ्च सति {\color{DodgerBlue3}“गुणानां द्रव्याणाञ्चाविशेषः स्यात्”} संकीर्ण्णलक्षणत्वात् । क्रियावत् गुणवत्समवायिकारणं द्रव्यमिति लक्षण\edlabel{pvv.46-1}\footnote{\label{pvv.46-1}  १ गुणद्रव्ययोः गुणकर्मणोश्च समवायात् ।}स्य द्वयोरपि भावात् । कथं पुनरभिन्नार्थत्वेप्यपर्यायत्व\edlabel{pvv.46-2}\footnote{\label{pvv.46-2}  २ बौद्धस्यापि ।}मित्याह । विजातीयेभ्योऽनेकासंयुक्तादिभ्यो {\color{DodgerBlue3}“व्यावृत्तेर्भेदात् भिन्नः”} प्रत्ययश्चैको घटः संयुक्तो घट इत्यादि । अकर्म द्रव्यमद्रव्यं कर्मेति\edlabel{pvv.46-3}\footnote{\label{pvv.46-3}  ३ अकर्मकत्वादौ परेण व्यतिरेको नेष्टोऽभिमतादतो व्यावृत्तिमुखेनैव ते स्थाप्याः अकर्मत्वमद्रव्यत्वञ्चास्येति ।} शब्दाविव भिन्नो व्यतिरिक्तार्थाभावेपि । न हि त्वन्मतेप्यकर्मद्रव्यशब्दयोरर्थो द्रव्य\edlabel{pvv.46-4}\footnote{\label{pvv.46-4}  ४ अत्र व्यावृत्तिरेव शरणं ।}कर्मभ्यां भिन्नो वस्तुभूतोस्ति ॥ (९८)
	\pend
      \label{div_pvv.1.99}\edlabel{div_pvv.1.99}
	  
	% new div opening: depth here is 2
	

	  \begin{center}%% label @type='head'
	\textbf{(B. संख्याद्यभावेऽप्येकत्वसंयोगयोर्व्यपदेशः)}
	\end{center}
	

	  \pstart यदि न सन्ति संख्यादयः कथं घटस्यै\edlabel{pvv.46-5}\footnote{\label{pvv.46-5}  ५ व्यतिरेकषष्ठी ।}कत्वं संयोगो वा इत्यादि व्यपदेश इत्याह ।
	\pend
      
	  \bigskip
	  \begingroup
	  \large
	
	    
	    \stanza[\smallbreak]
	\label{pv.1.99}\edlabel{pv.1.99}\flagstanza{\tiny\textenglish{...v.1.99}}व्यतिरेकीव यच्चापि सूच्यते भाववाचिभिः ।&संख्यादितद्वतः शब्दैस्तद्धर्मान्तरभेदकम् ॥ ९९ ॥\&[\smallbreak]


	
	  \endgroup
	

	  \pstart {\color{DodgerBlue3}“व्यतिरे”}कीव भेदवदिव {\color{DodgerBlue3}“यच्चापि संख्यादि तद्वतो”} द्रव्याद् {\color{DodgerBlue3}“भाववाचिभिः”} द्रव्याभिधायिभिः {\color{DodgerBlue3}“शब्दैः सूच्यते”} घटस्यैकत्वमित्यादि । तत्सूच्यमानमेकत्वादि{\color{DodgerBlue3}“धर्म्मान्तर”}स्य शुक्लत्वादेः {\color{DodgerBlue3}“भेदकं”} । यद्यपि घटानैकत्वादयो भिन्ना वस्तुतस्तथाप्येकत्वशुक्लत्वादयो घटात्परस्परं चानुवृत्त्यननुवृत्तिभ्यां कल्पितभेदाः । (९९)
	\pend
      \label{div_pvv.1.100}\edlabel{div_pvv.1.100}
	  
	% new div opening: depth here is 2
	
	  \bigskip
	  \begingroup
	  \large
	
	    
	    \stanza[\smallbreak]
	\label{pv.1.100}\edlabel{pv.1.100}\flagstanza{\tiny\textenglish{....1.100}}श्रुतिस्तन्मात्रजिज्ञासोरनाक्षिप्ताखिलापरा ।&भिन्नं धर्ममिवाचष्टे योगोऽङ्गुल्या इति क्वचित् ॥ १०० ॥\&[\smallbreak]


	
	  \endgroup
	

	  \pstart तेषु यदैको धर्मः प्रतिपि\edlabel{pvv.46-6}\footnote{\label{pvv.46-6}  ६ प्रतिपत्तुमिच्छुः ।}त्सितः तदा भेदेन निर्दिष्टः स धर्मान्तरप्रतिक्षेपको भवति घटस्यैकत्वमित्यादि । तदा {\color{DodgerBlue3}“तन्मात्र”}स्यैकधर्ममात्रप्रतिपित्सोः प्रतिपाद्यस्यान्{\color{DodgerBlue3}“रोधेन च तथा”} संकेतवशात् प्रयुक्ता {\color{DodgerBlue3}“श्रुतिरनाक्षिप्तो”}ऽविषयीकृतोऽ{\color{DodgerBlue3}“परोखिलो”} धर्मो (व्यवच्छेदकं ।) यथासम्भवी यया सा तादृशी धर्मिणो धर्म्मान्तरेभ्यश्च {\color{DodgerBlue3}“भिन्नं”} निष्कृष्टभिव {\color{DodgerBlue3}“धर्ममाचष्टे”} यथाऽ{\color{DodgerBlue3}“ङ्गुल्या योग”} इति । (१००)
	\pend
      \label{div_pvv.1.101}\edlabel{div_pvv.1.101}
	  
	% new div opening: depth here is 2
	
	  \bigskip
	  \begingroup
	  \large
	
	    
	    \stanza[\smallbreak]
	\label{pv.1.101}\edlabel{pv.1.101}\flagstanza{\tiny\textenglish{....1.101}}युक्ताङ्गुलीति सर्वेषां आक्षेपाद् धर्मिवाचिनी ।&ख्यातैकार्थाभिधानेऽपि तथा विहितसंस्थितिः ॥ १०१ ॥\&[\smallbreak]


	
	  \endgroup
	\leavevmode\marginnote{\textenglish{047/s}}

	  \begin{center}%% label @type='head'
	\textbf{(a. धर्मवाचिन्येव श्रुतिर्धर्मिवाचिनी)}
	\end{center}
	

	  \pstart यदा तु स एव धर्मो धर्मान्तरसम्बन्धयोगो जिज्ञासितस्तदा तथा संकेताद् {\color{DodgerBlue3}“युक्ताङ्गुली”}ति श्रुतिः {\color{DodgerBlue3}“सर्व्वेषां”} धर्म्मान्तराणा\edlabel{pvv.47-1}\footnote{\label{pvv.47-1}  १ अङ्गुलीगतानां ।} {\color{DodgerBlue3}“माक्षेपाद् धर्मिवाचिनी ख्याता”} । एकस्य विजातीयव्यावृत्तिलक्षणस्यार्थस्या{\color{DodgerBlue3}“भिधानेप्य”}यम्विभागो युक्तः । {\color{DodgerBlue3}“यस्मात्तेन प्रकारेण विहितसंस्थितिः”} कृतव्यवस्था सा श्रुतिः संकेतेन । (१०१)
	\pend
      \label{div_pvv.1.102}\edlabel{div_pvv.1.102}
	  
	% new div opening: depth here is 2
	

	  \begin{center}%% label @type='head'
	\textbf{(b. समुदायवाचिनी श्रुति)}
	\end{center}
	

	  \pstart तदेवं धर्मवाचिन्यपि श्रुतिर्द्धर्मान्तरप्रतिक्षेपाप्रतिक्षेपाभ्यां {\color{DodgerBlue3}“धर्मवाचिनी धर्मि”}वाचिनी वेति निर्दिष्टं ।
	\pend
      

	  \pstart इदानीं समु\edlabel{pvv.47-2}\footnote{\label{pvv.47-2}  २ रूपरसादेः ।} दायवाचिनीं दर्शयति ।
	\pend
      
	  \bigskip
	  \begingroup
	  \large
	
	    
	    \stanza[\smallbreak]
	\label{pv.1.102}\edlabel{pv.1.102}\flagstanza{\tiny\textenglish{....1.102}}रूपादिशक्तिभेदानामनाक्षेपेण वर्त्तते ।&तत्समानफलाऽहेतुव्यवच्छेदे घटश्रुतिः ॥ १०२ ॥\&[\smallbreak]


	
	  \endgroup
	

	  \pstart घटव्यपदेशभाजां {\color{DodgerBlue3}“रूपादी”}नामवान्तररंजना\edlabel{pvv.47-3}\footnote{\label{pvv.47-3}  ३ रक्तं चित्तं स्याद्विशिष्टरूपादौ चक्षुर्व्विज्ञानादि ।} {\color{DodgerBlue3}“दिशक्तिभेदानामनाक्षेपेण तेषां”} रूपादीनां {\color{DodgerBlue3}“यत्समानं”} फलमुदकाहरणादि तस्याहेतोरश्वादे{\color{DodgerBlue3}“र्व्यवच्छेदे”} प्रतिपाद्ये {\color{DodgerBlue3}“घट”}\edlabel{pvv.47-4}\footnote{\label{pvv.47-4}  ४ रूपरसाद्ये नैकत्र द्रव्ये एककार्यकारणशक्तिख्यापनाय ।}श्रुतिर्व्वर्तते उदकाहरणाद्यहेतुव्यावृत्तिं समुदायवाची घटशब्द आहेत्यर्थः । (१०२)
	\pend
      \label{div_pvv.1.103}\edlabel{div_pvv.1.103}
	  
	% new div opening: depth here is 2
	
	  \bigskip
	  \begingroup
	  \large
	
	    
	    \stanza[\smallbreak]
	\label{pv.1.103}\edlabel{pv.1.103}\flagstanza{\tiny\textenglish{....1.103}}अतो न रूपं घट इत्येकाधिकरण श्रुतिः ।&भेदोयमीदृशो जातिसमुदायाभिधायिनोः ॥ १०३ ॥\&[\smallbreak]


	
	  \endgroup
	

	  \pstart अतः समुदायाऽभिधायित्वात् न रूपं घट इत्येकाधिकरणा श्रुतिः । {\color{DodgerBlue3}“रूपशब्दो”} हि धर्मवाची घटशब्दस्तु समुदायाभिधायी । कथमनयोरेकार्थता । {\color{DodgerBlue3}“भेदोऽयमी”}दृ\edlabel{pvv.47-5}\footnote{\label{pvv.47-5}  ५ स्वविकल्पोपरचितं न मुख्यं गौर्व्वाहीकवदुपचारश्च, न गुणेषु संख्यादिवृत्तिः । न ह्येकं रूपं एको घट इति प्रत्यये लोकस्यावसायभेदः ।}शो {\color{DodgerBlue3}“जातिसमुदायाभिधायिनो”}र्घटत्वघटशब्दयोश्च ज्ञेयः । (१०३)
	\pend
      \label{div_pvv.1.104}\edlabel{div_pvv.1.104}
	  
	% new div opening: depth here is 2
	

	  \pstart समुदायाभिधायी चाप्रतिक्षिप्त\edlabel{pvv.47-6}\footnote{\label{pvv.47-6}  ६ जातिशब्दत्वात् ।}धर्मान्तर एव भवतीति सौव\edlabel{pvv.47-7}\footnote{\label{pvv.47-7}  ७ समुदायभेदापेक्षयाऽन्यानाक्षेपाय ।}र्ण्णो घट इत्यादि सामानाधिकरण्यं । स च\edlabel{pvv.47-8}\footnote{\label{pvv.47-8}  ८ समुदायाभिधायी ।} द्विविधोऽनेकवृत्तिरन्यथा\edlabel{pvv.47-9}\footnote{\label{pvv.47-9}  ९ अनेकघटवृक्षादौ ।} च । {\color{DodgerBlue3}“तथा घटादि”}शब्दो\edlabel{pvv.47-10}\footnote{\label{pvv.47-10}  १० एकगिरौ स्वसमुदायापेक्षः ।}वि न्ध्या द्रि शब्दश्च । यदि रूपादयः केवला नास्त्यवयवी तदा कथं घटस्य \leavevmode\marginnote{\textenglish{048/s}} रूपादय इति सम्बन्ध (।) इत्या\edlabel{pvv.48-1}\footnote{\label{pvv.48-1}  १ द्विधा रूपादीनां शक्तिः सामान्या यथा घटादेरुदकाहरणादि । प्रतिनियता च चक्षुर्व्विज्ञानादिजनिका ।}ह (।)
	\pend
      
	  \bigskip
	  \begingroup
	  \large
	
	    
	    \stanza[\smallbreak]
	\label{pv.1.104}\edlabel{pv.1.104}\flagstanza{\tiny\textenglish{....1.104}}रूपादयो घटस्येति तत्सामान्योपसर्जनाः ।&तच्छक्तिभेदाः ख्याप्यन्ते वाच्योऽन्योपि दिशाऽनया ॥ १०४ ॥\&[\smallbreak]


	
	  \endgroup
	

	  \pstart {\color{DodgerBlue3}“रूपादयो घटस्य”} इति सम्बन्धवाचिन्या श्रुत्या {\color{DodgerBlue3}“तत्सामान्योपसर्जना”} घटत्व\leavevmode\marginnote{\textenglish{10b/MA}} सामान्यविशेषितास्तेषां रूपादीनां शक्तिभेदा रञ्जनादयः ख्याप्यन्ते घटव्यपदेशविषयसमुदायान्तर्ग्गतं रञ्जनक्षमरूपं निष्कृष्योच्यत इत्यर्थः । अन्योपि चन्दनस्य गन्ध इत्यादिव्यपदेशोऽनया दिशा वाच्यः । विस्तरस्तृतीयपरि\edlabel{pvv.48-2}\footnote{\label{pvv.48-2}  २ सामान्यादिचिन्तायां (३।५५) ।}च्छेद एवास्य । तदेवमवयव्यादीनां प्रतिषेधात् पानकादिरिव परमाणुपुञ्जरूप एव देहः प्रत्यक्षेणेक्ष्यत इति स्थितं ॥ (१०४)
	\pend
      \label{div_pvv.1.105}\edlabel{div_pvv.1.105}
	  
	% new div opening: depth here is 2
	

	  \begin{center}%% label @type='head'
	\textbf{(घ) विज्ञानं कारणम्}
	\end{center}
	

	  \begin{center}%% label @type='head'
	\textbf{I. परमाणूनां समुदितानां न बुद्धिहेतुत्वम्}
	\end{center}
	

	  \pstart न चास्य बुद्धिहेतुत्वं युक्तमिति प्रकृतमनुबन्धं परमाणूनां समुदितानां प्रत्येकं बुद्धिहेतुत्वाभावं वक्तुमाह ।
	\pend
      
	  \bigskip
	  \begingroup
	  \large
	
	    
	    \stanza[\smallbreak]
	\label{pv.1.105}\edlabel{pv.1.105}\flagstanza{\tiny\textenglish{....1.105}}हेतुत्वे च समस्तानामेकाङ्गविकलेपि न ।&प्रत्येकमपि सामर्थ्ये युगपद् बहुसम्भवः ॥ १०५ ॥\&[\smallbreak]


	
	  \endgroup
	

	  \pstart यदि {\color{DodgerBlue3}“समस्तानां”} देहपरमाणूनां {\color{DodgerBlue3}“हेतुत्वं”} तदाऽव\edlabel{pvv.48-3}\footnote{\label{pvv.48-3}  ३ कर्ण्णादि ।}यवच्छेदादिना {\color{DodgerBlue3}“एकाङ्गविकलेपि”} देहे न स्यात् बुद्धिः (।) न चैतदस्ति । अथ प्रत्येकं ते समर्थास्तदा {\color{DodgerBlue3}“प्रत्येकमपि सामर्थ्ये”} स्वीक्रियमाणे {\color{DodgerBlue3}“युगपद् बहुसम्भवः”} । परमाणुसंख्यानि ज्ञानानि युगपज्जायरेन् सर्व्वेषां प्रत्येकं सामर्थ्यात् । (१०५)
	\pend
      \label{div_pvv.1.106}\edlabel{div_pvv.1.106}
	  
	% new div opening: depth here is 2
	

	  \begin{center}%% label @type='head'
	\textbf{(II. प्राणापानयोरनियामकता)}
	\end{center}
	

	  \pstart अथ प्राणापानाभ्यां नियामकाभ्यामेकमेव ज्ञानमभिव्यज्यते ततो न युगपदनेका व्यक्तिरित्यत आह (।)
	\pend
      
	  \bigskip
	  \begingroup
	  \large
	
	    
	    \stanza[\smallbreak]
	\label{pv.1.106}\edlabel{pv.1.106}\flagstanza{\tiny\textenglish{....1.106}}नानेकत्वस्य तुल्यत्वात् प्राणापानौ नियामकौ ॥&एकत्वेऽपि बहुव्यक्तिस्तद्धेतोर्नित्यसन्निधेः ॥ १०६ ॥\&[\smallbreak]


	
	  \endgroup
	

	  \pstart तन्न । परमाणुसञ्चयात्मकत्वेन देहवत् प्राणापानयोर{\color{DodgerBlue3}“प्यनेकत्वस्य तुल्यत्वात्”} । {\color{DodgerBlue3}“तेषां प्रत्येकमभिव्यञ्जकत्वे युगपदनेका”}भिव्यक्तिप्रसङ्गादेकज्ञानाभिव्यक्तौ \leavevmode\marginnote{\textenglish{049/s}} प्राणापानौ नियामकौ न युक्तौ । {\color{DodgerBlue3}“एकत्वेपि”} प्राणस्या\edlabel{pvv.49-1}\footnote{\label{pvv.49-1}  १ अथ माभूदेव दोष इति प्राणादिरेकं द्रव्यमिष्यते ।}पानस्य च {\color{DodgerBlue3}“बहूनां नाना”}कालभाविनां ज्ञानानां युगपद{\color{DodgerBlue3}“भिव्यक्तिः”} स्यात् । तस्याभिव्य\edlabel{pvv.49-2}\footnote{\label{pvv.49-2}  २ नित्यत्वादस्य । यत् सन्निहिताविकलकारणं तद् भवत्येवेति न्यायादाह ।}क्ते{\color{DodgerBlue3}“र्हेतोः”} प्राणा\edlabel{pvv.49-3}\footnote{\label{pvv.49-3}  ३ देहस्य च ।}देर्नित्यं सन्निधेः ॥ (१०६)
	\pend
      \label{div_pvv.1.107}\edlabel{div_pvv.1.107}
	  
	% new div opening: depth here is 2
	
	  \bigskip
	  \begingroup
	  \large
	
	    
	    \stanza[\smallbreak]
	\label{pv.1.107}\edlabel{pv.1.107}\flagstanza{\tiny\textenglish{....1.107}}नानेकहेतुरिति चेन्नाविशेषात् क्रमादपि ॥&नैकप्राणेप्यनेकार्थग्रहणान्नियमस्ततः ॥ १०७ ॥\&[\smallbreak]


	
	  \endgroup
	

	  \pstart युगप{\color{DodgerBlue3}“न्नानेक”}स्या{\color{DodgerBlue3}“भिव्यक्तिहेतुः”} प्राणा{\color{DodgerBlue3}“दिरिति चेत्”}\edlabel{pvv.49-4}\footnote{\label{pvv.49-4}  ४ प्राणापानदेहस्य य आत्मा विज्ञानहेतुः पश्चात्स पूर्व्वमपीति ।} (।) {\color{DodgerBlue3}“न”} स्याद्धेतुर{\color{DodgerBlue3}“विशेषात्”} ।\edlabel{pvv.49-5}\footnote{\label{pvv.49-5}  ५ अनेकत्वेपि प्राणापानयोर्देहस्य च एकत्रैव ज्ञाने सामर्थ्यं न बहुषु चेत् क्रमेणानेकज्ञानहेतुत्वमनेनाभ्युपेयमेव ।} {\color{DodgerBlue3}“क्रमादपि”} ॥
	\pend
      

	  \pstart किञ्चै\edlabel{pvv.49-6}\footnote{\label{pvv.49-6}  ६ यद्यप्युक्तमेकः प्राणादिरेकां धियं व्यनक्तीति तन्न ।}कस्मिन्नपि {\color{DodgerBlue3}“प्राणेऽनेकेषा”}म{\color{DodgerBlue3}“र्थानां ग्रहणाद”}नेका{\color{DodgerBlue3}“भिर्ब्बु”}द्धि\edlabel{pvv.49-7}\footnote{\label{pvv.49-7}  ७ अनेकक्षणभावित्वादेकप्राणस्य एकप्राणकालेऽनेका बुद्धिः प्रवर्तते निवर्तते च ।}भिरेकः प्राणः एकां बुद्धिमभिव्यनक्तीति नायन्तत एकप्राणा{\color{DodgerBlue3}“न्नियमः”} । (१०७)
	\pend
      \label{div_pvv.1.108}\edlabel{div_pvv.1.108}
	  
	% new div opening: depth here is 2
	
	  \bigskip
	  \begingroup
	  \large
	
	    
	    \stanza[\smallbreak]
	\label{pv.1.108}\edlabel{pv.1.108}\flagstanza{\tiny\textenglish{....1.108}}एकयाऽनेकविज्ञाने बुद्ध्याऽस्तु सकृदेव तत् ।&अविरोधात्; क्रमेणापि मा भूत् तदविशेषतः ॥ १०८ ॥\&[\smallbreak]


	
	  \endgroup
	

	  \pstart {\color{DodgerBlue3}“एक\edlabel{pvv.49-8}\footnote{\label{pvv.49-8}  ८ आसर्गप्रलयादेरेकैव बुद्धिरिति सिद्धान्तात् ।}या बुद्ध्याऽनेकवि\edlabel{pvv.49-9}\footnote{\label{pvv.49-9}  ९ अथ क्रमवत्येकैव तदा बुद्धिरिष्यते तदा ।}ज्ञाने”} च स्वीक्रियमाणेऽ{\color{DodgerBlue3}“स्तु सकृदेव तत्”} यावद् ग्रहीतव्यग्रहण{\color{DodgerBlue3}“मविरोधात्”} यद्येकस्या अनेकग्रहणं विरुध्यते कतिचित्पदार्थग्रहणमप्येकदा न स्यात् । न विरुध्यते चेत् यावद्ग्रहीतव्यं गृह्णीया\edlabel{pvv.49-10}\footnote{\label{pvv.49-10}  १० दीर्घश्च सितादौ यावन्तस्ते तत्रैव ।}त् । {\color{DodgerBlue3}“अन्यथा क्रमेणापि मा भूद”}नेकग्रहणं । {\color{DodgerBlue3}“तदविशेषतो”} बुद्धेर्व्विशेषाभावात् । (१०८)
	\pend
      \label{div_pvv.1.109}\edlabel{div_pvv.1.109}
	  
	% new div opening: depth here is 2
	
	  \bigskip
	  \begingroup
	  \large
	
	    
	    \stanza[\smallbreak]
	\label{pv.1.109}\edlabel{pv.1.109}\flagstanza{\tiny\textenglish{....1.109}}बहवः क्षणिकाः प्राणा अस्वजातीयकालिकाः ।&तादृशामेव चित्तानां कल्प्यन्ते यदि कारणम् ॥ १०९ ॥\&[\smallbreak]


	
	  \endgroup
	

	  \pstart {\color{DodgerBlue3}“बह\edlabel{pvv.49-11}\footnote{\label{pvv.49-11}  ११ नैकः प्राणोऽनेकविज्ञानहेतुः किन्तु ।}वःक्षणिकाः प्राणा अस्वजातीयकालिका”} असह\edlabel{pvv.49-12}\footnote{\label{pvv.49-12}  १२ यदेकः प्राणो न तदाऽपरः ।}भाविनः । {\color{DodgerBlue3}“तादृ”}\edlabel{pvv.49-13}\footnote{\label{pvv.49-13}  १३ क्रमिणां ।} शामेकक्षणिकानां बहूनामसहभाविनां {\color{DodgerBlue3}“चित्तानां कल्प्यन्ते यदि कारणं”} । (। १०९)
	\pend
      \label{div_pvv.1.110}\edlabel{div_pvv.1.110}
	  
	% new div opening: depth here is 2
	\leavevmode\marginnote{\textenglish{050/s}}
	  \bigskip
	  \begingroup
	  \large
	
	    
	    \stanza[\smallbreak]
	\label{pv.1.110}\edlabel{pv.1.110}\flagstanza{\tiny\textenglish{....1.110}}क्रमवन्तः कथं ते स्युः क्रमवद्धेतुना विना ।&पूर्वस्वजातिहेतुत्वे न स्यादाद्यस्य सम्भवः ॥ ११० ॥\&[\smallbreak]


	
	  \endgroup
	

	  \pstart तदां {\color{DodgerBlue3}“क्रमवन्तः कथन्ते”} प्राणाः {\color{DodgerBlue3}“स्युः क्रमवद्धेतुना विना”} । शरी\edlabel{pvv.50-1}\footnote{\label{pvv.50-1}  १ अथैतद्दोषतया देहं त्यक्त्वा पूर्व्वपूर्व्वप्राणहेतुकाः प्राणा इष्टास्तदा ।}रं तेषां हेतुः तच्चाक्रमं । न चाक्रमात्क्रमिकार्यं युक्तं {\color{DodgerBlue3}“पूर्व्वस्वजा”}तिहेतुत्वे {\color{DodgerBlue3}“न स्यादाद्यस्य”} प्राणस्य {\color{DodgerBlue3}“सम्भवः”} । (११०)
	\pend
      \label{div_pvv.1.111}\edlabel{div_pvv.1.111}
	  
	% new div opening: depth here is 2
	

	  \begin{center}%% label @type='head'
	\textbf{(III. परलोकानागतस्य प्राणे निर्हेतुकता)}
	\end{center}
	

	  \pstart न हि परलोकागतः प्राणोस्ति य आद्यस्य शरीरसम्बन्धिनो हेतुः स्यात् । एतदेवाह ।
	\pend
      
	  \bigskip
	  \begingroup
	  \large
	
	    
	    \stanza[\smallbreak]
	\label{pv.1.111}\edlabel{pv.1.111}\flagstanza{\tiny\textenglish{....1.111}}तद्धेतुस्तादृशो नास्ति सति वाऽकनेता ध्रुवम् ।&प्राणानां भिन्नदेशत्वात् सकृज्जन्म धियामतः ॥ १११ ॥\&[\smallbreak]


	
	  \endgroup
	

	  \pstart \edlabel{pvv.50-2}\footnote{\label{pvv.50-2}  २ तदा ज्ञानेपि पटुत्वादि ।}तद्धेतुस्तादृशो नास्ति । {\color{DodgerBlue3}“सति”} च तद्धे\edlabel{pvv.50-3}\footnote{\label{pvv.50-3}  ३ अभ्युपगम्याह ।}ताव{\color{DodgerBlue3}“नेकता ध्रुवं प्राणानां भिन्नदेशत्वात्”} । दे\edlabel{pvv.50-4}\footnote{\label{pvv.50-4}  ४ बहुत्वं यूथवन्मन्यते ।}शभेदात् प्राणाः प्रतिदेशं भिन्ना {\color{DodgerBlue3}“अतो”}ऽनेकत्वात् प्राणानां प्रत्येकं समर्थानां तेभ्यो व्यञ्चकोऽनेकत्वादतो {\color{DodgerBlue3}“धियां जन्म सकृत्”} स्यादिति प्रसङ्गः । (१११)
	\pend
      \label{div_pvv.1.112}\edlabel{div_pvv.1.112}
	  
	% new div opening: depth here is 2
	
	  \bigskip
	  \begingroup
	  \large
	
	    
	    \stanza[\smallbreak]
	\label{pv.1.112}\edlabel{pv.1.112}\flagstanza{\tiny\textenglish{....1.112}}यद्येककालिकोऽनेकोऽप्येकचैतन्यकारणम् ।&एकस्यापि न वैकल्ये स्यान्मन्दश्वसितादिषु ॥ ११२ ॥\&[\smallbreak]


	
	  \endgroup
	

	  \pstart {\color{DodgerBlue3}“यद्येककालिकोऽनेकः”} प्राण\edlabel{pvv.50-5}\footnote{\label{pvv.50-5}  ५ युगपद्‏बहुसम्भवनिरासार्थं ।} {\color{DodgerBlue3}“एकचैतन्यकारण”}मिष्यते तदै{\color{DodgerBlue3}“कस्यापि”} प्राणस्य {\color{DodgerBlue3}“मन्दश्वसितादिषु वैक ल्ये”} सति {\color{DodgerBlue3}“न स्याच्चै”}तन्यं कारणानामसमग्रत्वात् । (११२)
	\pend
      \label{div_pvv.1.113}\edlabel{div_pvv.1.113}
	  
	% new div opening: depth here is 2
	
	  \bigskip
	  \begingroup
	  \large
	
	    
	    \stanza[\smallbreak]
	\label{pv.1.113}\edlabel{pv.1.113}\flagstanza{\tiny\textenglish{....1.113}}अथ हेतुर्यथाभावं ज्ञानेऽपि स्याद् विशिष्टता ।&न हि तत् तस्य कार्यं यद् यस्य भेदान्न भिद्यते ॥ ११३ ॥\&[\smallbreak]


	
	  \endgroup
	

	  \pstart {\color{DodgerBlue3}“अथ हेतुर्यथाभावं”} यथासम्भवं प्राणानां हेतुत्वं स्यात् । तदा {\color{DodgerBlue3}“ज्ञानेपि स्याद् विशिष्टता”} (।) प्राणानामुपचयापचयाभ्यां ज्ञानमपि तादृशं स्यात् । {\color{DodgerBlue3}“न हि तत्तस्य कार्यं”}\edlabel{pvv.50-6}\footnote{\label{pvv.50-6}  ६ न च ज्ञानं तदर्थकारि ।}युक्तं {\color{DodgerBlue3}“यद्यस्य भेदान्न भिद्यते”} । (११३)
	\pend
      \label{div_pvv.1.114}\edlabel{div_pvv.1.114}
	  
	% new div opening: depth here is 2
	

	  \begin{center}%% label @type='head'
	\textbf{(IV. शक्तिनियमाद् न धियां सकृज्जन्त)}
	\end{center}
	

	  \pstart त्वन्मतेपि धियां सकृज्जन्म कस्मान्न भवतीत्याह ।
	\pend
      \leavevmode\marginnote{\textenglish{051/s}}
	  \bigskip
	  \begingroup
	  \large
	
	    
	    \stanza[\smallbreak]
	\label{pv.1.114a}\edlabel{pv.1.114a}\flagstanza{\tiny\textenglish{...1.114a}}विज्ञानं शक्तिनियमादेकमेकस्य कारणम् ।\&[\smallbreak]


	
	  \endgroup
	

	  \pstart {\color{DodgerBlue3}“विज्ञानं शक्तिनियमादेकंविज्ञानमे\edlabel{pvv.51-1}\footnote{\label{pvv.51-1}  १ सजातीयं ।} कं”} । {\color{DodgerBlue3}“शक्तेः”} स्वका\edlabel{pvv.51-2}\footnote{\label{pvv.51-2}  २ न युगपत्समानजातीयानि विज्ञानानि विजातीयानि तु स्युः ।}रणकृताया {\color{DodgerBlue3}“नियमादेकस्य”} विज्ञानस्य {\color{DodgerBlue3}“कारण”}मिति न सकृद् धियां जन्म ।
	\pend
      

	  \pstart कुत एवदिति चेदाह ।
	\pend
      
	  \bigskip
	  \begingroup
	  \large
	
	    
	    \stanza[\smallbreak]
	\label{pv.1.114b}\edlabel{pv.1.114b}\flagstanza{\tiny\textenglish{...1.114b}}अन्यार्थासक्तिविगुणे ज्ञाने नार्थान्तराग्रहात् ॥ ११४ ॥\&[\smallbreak]


	
	  \endgroup
	

	  \pstart अन्यस्मिन्नर्थे आसक्त्या पुनः पुनः प्रवृत्यात्मिकया {\color{DodgerBlue3}“विगुणे”} विषयान्तरसञ्चारि-\leavevmode\marginnote{\textenglish{11a/MA}} ज्ञानोत्पादविरोधानि {\color{DodgerBlue3}“ज्ञाने”} पूर्व्वके सति विषयान्तरस्याग्रहणात् । अविगुणे तु ग्रहणात्\edlabel{pvv.51-3}\footnote{\label{pvv.51-3}  ३ कालुष्यप्रसादादिमत्स्वपूर्व्वपूर्व्वविज्ञानेषूत्तराण्यपि तथा स्युः ।}। तस्मात् ज्ञानकार्यं ज्ञानं । तदन्वयव्यतिरेकानुविधानात् । (११४)
	\pend
      \label{div_pvv.1.115}\edlabel{div_pvv.1.115}
	  
	% new div opening: depth here is 2
	

	  \begin{center}%% label @type='head'
	\textbf{(ङ) कर्मसिद्धिः}
	\end{center}
	

	  \begin{center}%% label @type='head'
	\textbf{I. सहस्थितिकारणं कर्म,}
	\end{center}
	
	  \bigskip
	  \begingroup
	  \large
	
	    
	    \stanza[\smallbreak]
	\label{pv.1.115}\edlabel{pv.1.115}\flagstanza{\tiny\textenglish{....1.115}}शरीरात् सकृदुत्पन्ना धीः स्वजात्या नियम्यते ।&परतश्चेत् समर्थस्य देहस्य विरतिः कुतः ॥ ११५ ॥\&[\smallbreak]


	
	  \endgroup
	

	  \pstart अथ {\color{DodgerBlue3}“शरीरा\edlabel{pvv.51-4}\footnote{\label{pvv.51-4}  ४ गर्भादौ कायपरमाणवो जनयन्त्येकं चैतन्यं ।} त्सकृ”}त्प्रथममुत्पन्ना धीः {\color{DodgerBlue3}“स्वजात्या नियम्यते परत”} एकस्या बुद्धेरेका बुद्धिर्भवतीति न सकृज्जन्मप्रसङ्गः । ननु बुद्धिजनने प्रथमं {\color{DodgerBlue3}“समर्थस्य देहस्य”} पश्चा{\color{DodgerBlue3}“द्विरति”}स्तज्जननात् {\color{DodgerBlue3}“कुतः”} । येन बुद्धिर्न्नियामिका स्यात् परतः । (११५)
	\pend
      \label{div_pvv.1.116}\edlabel{div_pvv.1.116}
	  
	% new div opening: depth here is 2
	
	  \bigskip
	  \begingroup
	  \large
	
	    
	    \stanza[\smallbreak]
	\label{pv.1.116}\edlabel{pv.1.116}\flagstanza{\tiny\textenglish{....1.116}}अनाश्रयान्निवृत्ते स्याच्छरीरे चेतसः स्थितिः ।&केवलस्येति चेच्चित्तसन्तानस्थितिकारणम् ॥ ११६ ॥\&[\smallbreak]


	
	  \endgroup
	

	  \pstart ननु यदि बुद्धेर्न देह आश्रयस्तदाऽ{\color{DodgerBlue3}“नाश्रयादाश्र”}याभावात् {\color{DodgerBlue3}“निवृत्ते शरीरे केवलस्य चेतसः स्थितिः स्यादिति चेत्”} । स्यादेतत्\edlabel{pvv.51-5}\footnote{\label{pvv.51-5}  ५ आरूप्यभाव ।} शरीरेण सह {\color{DodgerBlue3}“चित्तसन्तानस्य स्थितिकारणं”} दृष्टं\edlabel{pvv.51-6}\footnote{\label{pvv.51-6}  ६ पञ्चायतनं}सहायं यत्कर्म (११६ ।)
	\pend
      \label{div_pvv.1.117}\edlabel{div_pvv.1.117}
	  
	% new div opening: depth here is 2
	

	  \begin{center}%% label @type='head'
	\textbf{(II. आमुत्रिकदेहहेतुः पञ्चायतनमैहिकम्)}
	\end{center}
	
	  \bigskip
	  \begingroup
	  \large
	
	    
	    \stanza[\smallbreak]
	\label{pv.1.117a}\edlabel{pv.1.117a}\flagstanza{\tiny\textenglish{...1.117a}}तद्धेतुवृत्तिलाभाय नाङ्गतां यदि गच्छति ।\&[\smallbreak]


	
	  \endgroup
	

	  \pstart a. तद्यदि तस्य पारलौकिकस्य {\color{DodgerBlue3}“देहस्य हेतु”}रैहिकमन्त्यं पञ्चा\edlabel{pvv.51-7}\footnote{\label{pvv.51-7}  ७ इन्द्रियं ।}यतनं तस्य {\color{DodgerBlue3}“वृत्तिः”} \leavevmode\marginnote{\textenglish{052/s}} पारलौकिकदेहजननाद्याभिमुख्यं । तस्य {\color{DodgerBlue3}“लाभाय”} प्राप्तयेऽ{\color{DodgerBlue3}“ङ्गतां”} सहकारितां {\color{DodgerBlue3}“न गच्छति”} तदा केवलं चित्तं तिष्ठति । यथा विरूपे धातौ ।
	\pend
      

	  \pstart कः पुनरामुत्रिकदेहहेतुः यस्य चित्तं सहकारीत्याह ।
	\pend
      
	  \bigskip
	  \begingroup
	  \large
	
	    
	    \stanza[\smallbreak]
	\label{pv.1.117b}\edlabel{pv.1.117b}\flagstanza{\tiny\textenglish{...1.117b}}हेतुर्देहान्तरोत्पत्तौ पञ्चायतनमैहिकम् ॥ ११७ ॥\&[\smallbreak]


	
	  \endgroup
	

	  \pstart {\color{DodgerBlue3}“हेतुर्देहान्तरोत्पत्तौ पञ्चायतनं”} पञ्चेन्द्रियाणि । ऐहिकमिदं जन्मभवं (। ११७)
	\pend
      \label{div_pvv.1.118}\edlabel{div_pvv.1.118}
	  
	% new div opening: depth here is 2
	
	  \bigskip
	  \begingroup
	  \large
	
	    
	    \stanza[\smallbreak]
	\label{pv.1.118}\edlabel{pv.1.118}\flagstanza{\tiny\textenglish{....1.118}}तदङ्गभावहेतुत्वनिषेधेऽनुपलम्भनम् ।&अनिश्चयकरं प्रोक्तं । इन्द्रियाद्यपि शेषवत् ॥ ११८ ॥\&[\smallbreak]


	
	  \endgroup
	

	  \pstart यच्च तयोः कर्म-ऐहिकपञ्चायतनयोर्यथाक्रम{\color{DodgerBlue3}“मङ्गभावहेतुत्वयोः”} सहकारितोपादानत्वयो{\color{DodgerBlue3}“र्निषेधे”} कर्त्तव्येऽ\edlabel{pvv.52-1}\footnote{\label{pvv.52-1}  १ न दृष्टमेतत्सहकारित्वादीति ।} {\color{DodgerBlue3}“नुपलम्भनं”} परैरुच्यते । तद{\color{DodgerBlue3}“निश्चयक”}रमनैकान्तिकमदृश्यविषयत्वात् {\color{DodgerBlue3}“प्रोक्तं इन्द्रियाद्यपि शेषवत्”} । यजपीन्द्रियादि शरीरान्तरसम्बन्धीन्द्रियादि प्रतिसन्धातृ न भ\edlabel{pvv.52-2}\footnote{\label{pvv.52-2}  २ मध्यावस्थेन्द्रियवत् । अव्याप्तेः ।} \edlabel{pvv.52-3a}\footnote{\label{pvv.52-3a}  ३a अन्त्यं शरीरं ।\begin{english}\par
Placement of note uncertain; marked with a question mark in the edition (see encoding description for details).\end{english}} वतीन्द्रियत्वादेरिति तदपि शे\edlabel{pvv.52-3}\footnote{\label{pvv.52-3}  ३ यथा चैत्रेन्द्रियं मैत्रेन्द्रियस्यातिपरस्य विजातीयासंन्धानं दृष्टान्तः । स्वस्य तु सजातीयं सन्धायकं पूर्व्वेन्द्रियेण परसन्धानदृष्टेः ।} \edlabel{pvv.52-4a}\footnote{\label{pvv.52-4a}  ४a अन्त्यचित्तं न देहसहकारि देहान्तरोत्पादने चित्तत्वात् पूर्व्वचित्तवता\begin{english}\par
Placement of note uncertain; marked with a question mark in the edition (see encoding description for details)\end{english}} षवदनैकान्तिकं बोद्धव्यं । आदिशब्दात्प्राणापानत्वादि ज्ञानत्वादि\edlabel{pvv.52-4}\footnote{\label{pvv.52-4}  ४ ज्ञानासहकारित्वे पूतीभावः स्यात् ।}च । (११८)
	\pend
      \label{div_pvv.1.119}\edlabel{div_pvv.1.119}
	  
	% new div opening: depth here is 2
	

	  \pstart b विरुद्धत्वमपि दर्शयति ।
	\pend
      
	  \bigskip
	  \begingroup
	  \large
	
	    
	    \stanza[\smallbreak]
	\label{pv.1.119}\edlabel{pv.1.119}\flagstanza{\tiny\textenglish{....1.119}}दृष्टा च शक्तिः पूर्वेषामिन्द्रियाणां स्वजातिषु ।&विकारदर्शनात् सिद्धं अपरापरजन्म च ॥ ११९ ॥\&[\smallbreak]


	
	  \endgroup
	

	  \pstart {\color{DodgerBlue3}“दृष्टा च शक्तिः पूर्व्वेषामिन्द्रियाणां स्वजातिषु”} कर्त्तव्येषु मध्यावस्थायां तत इन्द्रियादित्वात् । स्वजातिप्रतिसन्धातृत्वमेवैषां युक्तं । मध्यावस्थायाञ्च पूर्व्वावस्थातः पाट\edlabel{pvv.52-5}\footnote{\label{pvv.52-5}  ५ प्रसन्नाविलत्वादिना ।}वादि{\color{DodgerBlue3}“विकारदर्शनात् सिद्धं”} प्रतिक्षण{\color{DodgerBlue3}“मपरापरजन्म चे”}न्द्रियादीनामिति नादृष्टान्तो हेतुः । (११९)
	\pend
      \label{div_pvv.1.120}\edlabel{div_pvv.1.120}
	  
	% new div opening: depth here is 2
	

	  \pstart c ननु शरीरादेवेन्द्रियादीनां जन्मेत्याह ।
	\pend
      
	  \bigskip
	  \begingroup
	  \large
	
	    
	    \stanza[\smallbreak]
	\label{pv.1.120}\edlabel{pv.1.120}\flagstanza{\tiny\textenglish{....1.120}}शरीराद् यदि तज्जन्म प्रसङ्गः पूर्ववद् भवेत् ।&चित्ताच्चेत् तत एवास्तु जन्म देहान्तरस्य च ॥ १२० ॥\&[\smallbreak]


	
	  \endgroup
	\leavevmode\marginnote{\textenglish{053/s}}

	  \pstart {\color{DodgerBlue3}“शरीराद्यदि”} तेषामिन्द्रियादीनां {\color{DodgerBlue3}“जन्म”}तदा {\color{DodgerBlue3}“पूर्व्वव”}द्धे\edlabel{pvv.53-1}\footnote{\label{pvv.53-1}  १ युगपद् बहुत्वं मृतेपि ।} तुत्वे च समस्तानामित्यादिनोक्तः {\color{DodgerBlue3}“प्रसङ्गो भवेत्”} । {\color{DodgerBlue3}“चित्ताच्चे”}दिन्द्रियचित्तादीनां जन्म शरीराज्जन्मनि दोषदर्शनादिष्यते तदान्त्यावस्थायामप्यवि{\color{DodgerBlue3}“कलत्वा”}च्चित्तस्य {\color{DodgerBlue3}“तत”} एवास्तु {\color{DodgerBlue3}“जन्म देहान्तरस्य”} पञ्चायतनरूपस्यानगतस्य । (१२०)
	\pend
      \label{div_pvv.1.121}\edlabel{div_pvv.1.121}
	  
	% new div opening: depth here is 2
	
	  \bigskip
	  \begingroup
	  \large
	
	    
	    \stanza[\smallbreak]
	\label{pv.1.121}\edlabel{pv.1.121}\flagstanza{\tiny\textenglish{....1.121}}तस्मान्न हेतुवैकल्यात् सर्वेषामन्त्यचेतसाम् ।&असन्धिरीदृशं तेन शेषवत् साधनं मतम् ॥ १२१ ॥\&[\smallbreak]


	
	  \endgroup
	

	  \pstart d यतश्चित्तमेव चित्तस्य हेतुः । तृष्णाकर्मसहायञ्च पञ्चायतनस्य । {\color{DodgerBlue3}“तस्मान्न हेतुवैकल्यात् । सर्व्वेषामन्त्यचेतसामसन्धिः”} चित्तस्य पञ्चायतनस्य {\color{DodgerBlue3}“च”} हेत्ववैकल्यात् कार्योत्पादस्यावश्यम्भावित्वात् (।) तेनेदृशमन्त्यचित्तत्वादि {\color{DodgerBlue3}“शेष”}वदनैकान्तिकं {\color{DodgerBlue3}“साधनं मतं”} ॥ (१२१)
	\pend
      \label{div_pvv.1.122_1.123}\edlabel{div_pvv.1.122_1.123}
	  
	% new div opening: depth here is 2
	

	  \begin{center}%% label @type='head'
	\textbf{(ख. युक्तः करुणाभ्यासः)}
	\end{center}
	

	  \begin{center}%% label @type='head'
	\textbf{(क) चित्तमात्रप्रतिबद्धत्वात्}
	\end{center}
	

	  \pstart तदेवं चित्तमात्रप्रतिबद्धत्वाच्चित्तजन्मनो देहनिवृत्तावपि जन्मपरम्परासम्भवे युक्तः कृपाभ्यास इत्यभ्यासात्सेति समर्थितं ।
	\pend
      
	  \bigskip
	  \begingroup
	  \large
	
	    
	    \stanza[\smallbreak]
	\label{pv.1.122a}\edlabel{pv.1.122a}\flagstanza{\tiny\textenglish{...1.122a}}अभ्यासेन विशेषेऽपि लङ्घनोदकतापवत् ।&स्वभावातिक्रमो मा भूदिति चेद्;\&[\smallbreak]


	
	  \endgroup
	

	  \pstart {\color{DodgerBlue3}“नन्वभ्यासेन विशेषेपि”} सत्यल्पीयसि {\color{DodgerBlue3}“स्वभावस्य”} कृपादेस्तद्विपक्षसंकीर्णत्वस्यातिक्रमो विपक्षाव्यवकीर्ण्णस्वरसप्रवृत्तकृपादिमयता सात्मीभावो {\color{DodgerBlue3}“मा भूत् लंघनोदकतापवत्”} ।
	\pend
      

	  \begin{center}%% label @type='head'
	\textbf{(ख) पुनर्यत्नापेक्षा}
	\end{center}
	

	  \pstart न हि पुरुषोत्यर्थं लङ्घने कृताभ्यासो योजनमर्द्धयोजनं वा लङ्घयति । नाप्युदकमेकान्तं ताप्यमानं दहनीभवति । किन्तु प्रकृतिसिद्धात् लङ्घनात् स्पर्शाच्च विशेषमात्रं भवति यथा (।) तथोत्कर्षमात्रं स्यात् कृपया न तु सात्मीभाव {\color{DodgerBlue3}“इति चेत्”} । अत्राह (।)
	\pend
      
	  \bigskip
	  \begingroup
	  \large
	
	    
	    \stanza[\smallbreak]
	\label{pv.1.122b}\edlabel{pv.1.122b}\flagstanza{\tiny\textenglish{...1.122b}}आहितः स चेत् ॥ १२२ ॥\&[\smallbreak]


	
	  \endgroup
	
	  \bigskip
	  \begingroup
	  \large
	
	    
	    \stanza[\smallbreak]
	\label{pv.1.123}\edlabel{pv.1.123}\flagstanza{\tiny\textenglish{....1.123}}पुनर्यत्नमपेक्षेत यदि स्याच्चास्थिराश्रयः ।&विशेषो नैव वर्द्धेत स्वभावश्च न तादृशः ॥ १२३ ॥\&[\smallbreak]


	
	  \endgroup
	\leavevmode\marginnote{\textenglish{054/s}}

	  \pstart {\color{DodgerBlue3}“आहितः स चेत्”} विशेष (:। १२२) आधायकनिवृत्तावात्मलाभाय {\color{DodgerBlue3}“पुनर्यत्नमपेक्षेत”} न स्वरसवाही स्यात् । लङ्घनं यथाभ्यस्तमपि पुनर्यत्नापेक्षयैव प्रवर्तते न {\color{DodgerBlue3}“स्वरसवाहि”} । {\color{DodgerBlue3}“यदि स्याच्चास्थिराश्रय”} उदकतापवत् । क्वाथ्यमानं ह्यदकं क्षीय\edlabel{pvv.54-1}\footnote{\label{pvv.54-1}  १ न ज्वलति ।} त एव इत्यस्थिराश्रय उदकतापः पुनर्यत्नापेक्षी च स्वरसवाहित्वाभावात् । तदा \leavevmode\marginnote{\textenglish{11b/MA}} {\color{DodgerBlue3}“विशेषो नैव वर्द्धेत”} प्रकर्षनिष्ठां न गच्छेत् (।) {\color{DodgerBlue3}“तादृशश्च”} विशेषो {\color{DodgerBlue3}“नैव स्वभावः”} ।\edlabel{pvv.54-2}\footnote{\label{pvv.54-2}  २ लंघनादिविशेषवन्न प्रकृतिः । अन्यानपेक्षत्वात्स्वभावस्याकाशवत् ।} प्रकृतिर्व्विशेष\edlabel{pvv.54-3}\footnote{\label{pvv.54-3}  ३ नैवं प्रज्ञादयोऽभ्यासात् काष्ठानिष्टां प्रतिष्ठिताः ।}वत् । हे\edlabel{pvv.54-4}\footnote{\label{pvv.54-4}  ४ विशेषस्य ।}तुसन्निधानव्यवधानसापेक्षत्वात् प्रवृत्तिनिवृत्त्योः । (१२३)
	\pend
      \label{div_pvv.1.124}\edlabel{div_pvv.1.124}
	  
	% new div opening: depth here is 2
	

	  \pstart तथा च (।)
	\pend
      
	  \bigskip
	  \begingroup
	  \large
	
	    
	    \stanza[\smallbreak]
	\label{pv.1.124}\edlabel{pv.1.124}\flagstanza{\tiny\textenglish{....1.124}}तत्रोपयुक्तशक्तीनां विशेषानुत्तरान् प्रति ।&साधनानामसामर्थ्यान्नित्यञ्चानाश्रयस्थितेः ॥ १२४ ॥\&[\smallbreak]


	
	  \endgroup
	

	  \pstart {\color{DodgerBlue3}“तत्र”} पूर्व्वदृष्ट एव विशेषे {\color{DodgerBlue3}“उपयुक्तशक्तीनां साधनानां”} यत्नादीनां पुनरपि {\color{DodgerBlue3}“विशेषानुत्तरान् प्रत्यसामर्थ्यात् नित्यञ्चानाश्रयस्थितेः”} । (१२४)
	\pend
      \label{div_pvv.1.125}\edlabel{div_pvv.1.125}
	  
	% new div opening: depth here is 2
	
	  \bigskip
	  \begingroup
	  \large
	
	    
	    \stanza[\smallbreak]
	\label{pv.1.125}\edlabel{pv.1.125}\flagstanza{\tiny\textenglish{....1.125}}विशेषस्यास्वभावत्वाद् वृद्धावप्याहितो यदा ।&नापेक्षेत पूनर्यत्नं यत्नोन्यः स्याद् विशेषकृत् ॥ १२५ ॥\&[\smallbreak]


	
	  \endgroup
	

	  \pstart आश्रयस्थित्यभावात् तादृशस्य {\color{DodgerBlue3}“विशेषस्यास्वभावत्वात् वृद्धावपि”} व्यवस्थितोत्कर्षतैव । पुनर्यत्नापेक्षित्वेनास्थिराश्रयत्वेन व्यवस्थितोत्कर्षता व्याप्तेत्यर्थः । {\color{DodgerBlue3}“यदा”} तु विशेष {\color{DodgerBlue3}“आहितो नापेक्षेत पुनर्यत्नं”} प्रागुत्पन्नस्यात्मनो लाभायापि तु स्वरसवाही {\color{DodgerBlue3}“भवति तदा यत्नोऽन्यः”} क्रियमाणो {\color{DodgerBlue3}“विशेषकृत्”} यथाभ्यासमुत्तरोत्तरविशेषाधायी भवति । (१२५)
	\pend
      \label{div_pvv.1.126}\edlabel{div_pvv.1.126}
	  
	% new div opening: depth here is 2
	

	  \begin{center}%% label @type='head'
	\textbf{(ग) स्वरसेनाभ्यासजः करुणादिप्रवाहः}
	\end{center}
	
	  \bigskip
	  \begingroup
	  \large
	
	    
	    \stanza[\smallbreak]
	\label{pv.1.126}\edlabel{pv.1.126}\flagstanza{\tiny\textenglish{....1.126}}काष्ठपारदहेमादेरग्न्यादेरिव चेतसि ।&अभ्यासजाः प्रवर्त्तन्ते स्वरसेन कृपादयः ॥ १२६ ॥\&[\smallbreak]


	
	  \endgroup
	

	  \pstart {\color{DodgerBlue3}“काष्ठपारदेहमादेरग्न्यादेरिव”} । यथाग्निना हेमचारणजार\edlabel{pvv.54-5}\footnote{\label{pvv.54-5}  ५ तुषदाह्ये प्रक्षिप्तानां तत्रैव क्षयनयनं ।}णादिना पुटपाकादिना यथाक्रमं का\edlabel{pvv.54-6}\footnote{\label{pvv.54-6}  ६ अग्निना ।}ष्ठे पा\edlabel{pvv.54-7}\footnote{\label{pvv.54-7}  ७ हेमादिना ।}रदे हेम्नी\edlabel{pvv.54-8}\footnote{\label{pvv.54-8}  ८ पुटादिना कल्याणसुवर्ण्णता ।} व दरदाहः । रुप्यरञ्जनसामर्थ्यवर्ण्णिका\edlabel{pvv.54-9}\footnote{\label{pvv.54-9}  ९ वानी (?) ।}\leavevmode\marginnote{\textenglish{055/s}} वृद्धय आहिताः स्वरसवाहिन्यो न पुनर्यत्नसापेक्षाः । तेषु यदा पुनर्वह्न्यादयो व्याप्रियन्ते तदा समधिकमङ्गारा\edlabel{pvv.55-1}\footnote{\label{pvv.55-1}  १ यावद् भस्मशात् स्यात् ।} दिविशेषमादधति (।) {\color{DodgerBlue3}“तथाभ्यासजाः कृपादयः”} पुनर्यत्नानपेक्षित्वात् स्थिराश्रयत्वाच्च {\color{DodgerBlue3}“स्वरसेन प्रवर्तन्ते”} । (१२६)
	\pend
      \label{div_pvv.1.127}\edlabel{div_pvv.1.127}
	  
	% new div opening: depth here is 2
	
	  \bigskip
	  \begingroup
	  \large
	
	    
	    \stanza[\smallbreak]
	\label{pv.1.127}\edlabel{pv.1.127}\flagstanza{\tiny\textenglish{....1.127}}तस्मात् स तेषामुत्पन्नः स्वभावो जायते गुणः ।&तदुत्तरोत्तरो यत्नो विशेषस्य विधायकः ॥ १२७ ॥\&[\smallbreak]


	
	  \endgroup
	

	  \pstart {\color{DodgerBlue3}“तस्मात्”} स्वरसवाहित्वात्स {\color{DodgerBlue3}“तेषा”}मभ्यासवतां पुंसा{\color{DodgerBlue3}“मुत्पन्नो”} गुणः {\color{DodgerBlue3}“कृपादिः स्वभावो”} जायते मनसः प्रकृतिर्भवति । {\color{DodgerBlue3}“तदुत्तरोत्तरो यत्नः”} पूर्व्वपूर्व्वाभ्यासादपरापरः प्रयत्नो विशेषकृद् भवति पूर्व्वप्रयत्नकृतस्य {\color{DodgerBlue3}“विशेषस्य”} सुस्थितत्वात् । आधिक्याधानमेवापरयत्नात् । (१२७)
	\pend
      \label{div_pvv.1.128}\edlabel{div_pvv.1.128}
	  
	% new div opening: depth here is 2
	
	  \bigskip
	  \begingroup
	  \large
	
	    
	    \stanza[\smallbreak]
	\label{pv.1.128}\edlabel{pv.1.128}\flagstanza{\tiny\textenglish{....1.128}}यस्माच्च तुल्यजातीयपूर्वबीजप्रवृद्धयः ।&कृपादिबुद्धयस्तासां सत्यभ्यासे कुतः स्थितिः ॥ १२८ ॥\&[\smallbreak]


	
	  \endgroup
	

	  \pstart {\color{DodgerBlue3}“यस्माच्च”} कारणा{\color{DodgerBlue3}“त्तुल्यजातीया”}त् {\color{DodgerBlue3}“पूर्व्व”}स्मात् बीजाद्वासनागर्भसमनन्तरप्रत्ययात् {\color{DodgerBlue3}“प्रवृद्धि”}रुत्कर्षो यासान्तास्तथा {\color{DodgerBlue3}“कृपादिबुद्धयः”} (।) {\color{DodgerBlue3}“तासां सत्यभ्यासे कुतः”} स्थितिर्व्यवस्थितोत्कर्षता । (१२८)
	\pend
      \label{div_pvv.1.129}\edlabel{div_pvv.1.129}
	  
	% new div opening: depth here is 2
	
	  \bigskip
	  \begingroup
	  \large
	
	    
	    \stanza[\smallbreak]
	\label{pv.1.129}\edlabel{pv.1.129}\flagstanza{\tiny\textenglish{....1.129}}न चैवं लंघनादेव लंघनं बलयत्नयोः ।&तद्धेत्वोः स्थितशक्तित्वाल्लंघनस्य स्थितात्मता ॥ १२९ ॥\&[\smallbreak]


	
	  \endgroup
	

	  \pstart {\color{DodgerBlue3}“न चैवं लंघनादेव लंघनं”} यथा कृपादिभ्य एव कृपादयः । तथा न लंघनादेव लंघनमपि तु बलयत्नाभ्यां {\color{DodgerBlue3}“बलयत्नयोस्तद्धेत्वोः स्थितशक्तित्वात्”} सामर्थ्यनियमात् लंघनस्य {\color{DodgerBlue3}“स्थितात्मता”} व्यवस्थितोत्कर्षता भवति । (१२९)
	\pend
      \label{div_pvv.1.130}\edlabel{div_pvv.1.130}
	  
	% new div opening: depth here is 2
	

	  \pstart यदि बलयत्नाभ्यामेव लंघनं न स्वभावजातीयात्तदाभ्यासात्प्रागपि तावत्परिमाणं स्यादित्याह (।)
	\pend
      
	  \bigskip
	  \begingroup
	  \large
	
	    
	    \stanza[\smallbreak]
	\label{pv.1.130}\edlabel{pv.1.130}\flagstanza{\tiny\textenglish{....1.130}}तस्यादौ देहवैगुण्यात् पश्चाद्वदविलंघनम् ।&शनैर्यत्नेन वैगुण्ये निरस्ते स्वबले स्थितिः ॥ १३० ॥\&[\smallbreak]


	
	  \endgroup
	

	  \pstart {\color{DodgerBlue3}“तस्य”} लंघयितुरादावभ्यासात् पूर्व्वं {\color{DodgerBlue3}“देहवैगुण्यात्”} श्लेष्मादिकृतगौरवात् {\color{DodgerBlue3}“पश्चाद्व”}दभ्यासानन्तरमिवा{\color{DodgerBlue3}“विलंघनं शनैर्यत्नेन”} व्यायामादिना {\color{DodgerBlue3}“वैगुण्ये निरस्ते स्वबले स्थितिः”} शरीरस्य भवति । तेन पूर्व्वस्माल्लंघनं विशिष्यते बलानुरूपस्थितिकञ्च । (१३०)
	\pend
      \label{div_pvv.1.131}\edlabel{div_pvv.1.131}
	  
	% new div opening: depth here is 2
	

	  \begin{center}%% label @type='head'
	\textbf{(घ) करुणा स्वबीजप्रभवा}
	\end{center}
	

	  \pstart मनोगुणास्तु सत्यभ्यासे विपक्षानभ्यासे च प्रकर्षंनिष्टां गच्छन्ति । तदाहे (।)
	\pend
      \leavevmode\marginnote{\textenglish{056/s}}
	  \bigskip
	  \begingroup
	  \large
	
	    
	    \stanza[\smallbreak]
	\label{pv.1.131}\edlabel{pv.1.131}\flagstanza{\tiny\textenglish{....1.131}}कृपा स्वबीजप्रभवा स्वबीजप्रभवैर्न चेत् ।&विपक्षैर्बाध्यते चित्ते प्रयात्यत्यन्तसात्मताम् ॥ १३१ ॥\&[\smallbreak]


	
	  \endgroup
	

	  \pstart {\color{DodgerBlue3}“कृपा स्वबीजप्रभवा”} भूयोऽभ्यस्तस्वजा\edlabel{pvv.56-1}\footnote{\label{pvv.56-1}  १ न सर्व्वः किन्तु यस्य बीजं कृपायामस्ति ।} तीयसँस्कारवत् समनन्तरप्रत्ययप्रसूतार्थ{\color{DodgerBlue3}“बीजप्रभवैर्न चेत् विपक्षै”}र्द्वेषादिभि{\color{DodgerBlue3}“र्ब्बाध्यते”} स्वोत्पत्त्या व्याहन्यते {\color{DodgerBlue3}“चित्ते”} चित्तसन्ताने {\color{DodgerBlue3}“प्रयात्यत्यन्तसात्मतां”} विपक्षासंकीर्ण्णसात्मतां प्रकृतितां । (१३१)
	\pend
      \label{div_pvv.1.132}\edlabel{div_pvv.1.132}
	  
	% new div opening: depth here is 2
	
	  \bigskip
	  \begingroup
	  \large
	
	    
	    \stanza[\smallbreak]
	\label{pv.1.132}\edlabel{pv.1.132}\flagstanza{\tiny\textenglish{....1.132}}तथापि मूलमभ्यासः पूर्वः पूर्वः परस्य तु ।&कृपावैराग्यबोधादेश्चित्तधर्मस्य पाटवे ॥ १३२ ॥\&[\smallbreak]


	
	  \endgroup
	

	  \pstart {\color{DodgerBlue3}“तथा हि मूलं”} कारणमभ्यासः {\color{DodgerBlue3}“पूर्व्वः पूर्व्वः परस्यो”}त्तरस्य {\color{DodgerBlue3}“कृपावैराग्यबो\edlabel{pvv.56-2}\footnote{\label{pvv.56-2}  २ बुद्धयादेः ।} धादेश्चित्तधर्मस्य”} मनोगुणस्य {\color{DodgerBlue3}“पाटवे”} प्र\edlabel{pvv.56-3}\footnote{\label{pvv.56-3}  ३ कारणमिति सम्बन्धः ।}कर्षे न तूत्पत्तौ । तस्याः सम्भवात् । (। १३२)
	\pend
      \label{div_pvv.1.133}\edlabel{div_pvv.1.133}
	  
	% new div opening: depth here is 2
	

	  \begin{center}%% label @type='head'
	\textbf{(ङ) अभ्यसात् करुणात्मकत्वम्}
	\end{center}
	

	  \pstart ततः (।)
	\pend
      
	  \bigskip
	  \begingroup
	  \large
	
	    
	    \stanza[\smallbreak]
	\label{pv.1.133}\edlabel{pv.1.133}\flagstanza{\tiny\textenglish{....1.133}}कृपात्मकत्वमभ्यासाद् घृणावैराग्यरागवत् ।&निष्पन्नकरुणोत्कर्षपरदुःखक्षमेस्तिः ॥ १३३ ॥\&[\smallbreak]


	
	  \endgroup
	

	  \pstart {\color{DodgerBlue3}“कृपात्मकत्वमभ्यासाद्”} भवति {\color{DodgerBlue3}“घृणाबैराग्यरागवत्”} । यथाभ्यासात् घृणा क्वचिद्विषये वैराग्यं रागश्च सात्मीभवति (।) तदेवमभ्यासात् {\color{DodgerBlue3}“कृपा”} प्र{\color{DodgerBlue3}“कर्ष”}विशे\edlabel{pvv.56-4}\footnote{\label{pvv.56-4}  ४ नैरात्म्यदर्शनं ।}षवती {\color{DodgerBlue3}“निष्पद्यत”} इति समर्थितर्मिय\edlabel{pvv.56-5}\footnote{\label{pvv.56-5}  ५ साधनं करणेभिप्रस्तुत्य कृपावैराग्यरागवदित्यन्तेन ॥}ता च जगद्धितैषित्वञ्च (।) व्याख्यातं कृपायाः प्रामाण्यसाधनत्वं (। १३३)
	\pend
      \label{div_pvv.1.134}\edlabel{div_pvv.1.134}
	  
	% new div opening: depth here is 2
	

	  \begin{center}%% label @type='head'
	\textbf{(४) शास्तृत्वात् भगवान् प्रमाणम्}
	\end{center}
	

	  \begin{center}%% label @type='head'
	\textbf{क. शास्तृत्वव्याख्यानम्}
	\end{center}
	

	  \pstart शास्तृत्वव्याख्यानाय (दयां) दर्शयितुमाह (।)
	\pend
      
	  \bigskip
	  \begingroup
	  \large
	
	    
	    \stanza[\smallbreak]
	\label{pv.1.134}\edlabel{pv.1.134}\flagstanza{\tiny\textenglish{....1.134}}दयावान् दुःखहानार्थमुपायेष्वभियुज्यते ।&परोक्षोपेयतद्धेतोस्तदाख्यानं हि दुष्करम् ॥ १३४ ॥\&[\smallbreak]


	
	  \endgroup
	

	  \pstart {\color{DodgerBlue3}“दयावान्”} बोधिसत्त्वः परदुःखं शमयितुकामः दुःखहानार्थमात्मनः {\color{DodgerBlue3}“उपायेषु”} दुःखशमनोपायेष्व{\color{DodgerBlue3}“भियुज्यते”} । कस्मात्पुनः परदुःखशमनोपायेषु युज्यत \leavevmode\marginnote{\textenglish{057/s}} इत्याह (।) {\color{DodgerBlue3}“परोक्ष उपेयो”} दुःखप्रशमः {\color{DodgerBlue3}“तद्धेतुश्च”} मा\edlabel{pvv.57-1}\footnote{\label{pvv.57-1}  १ अशुचिविज्ञगुप्त्या (?) सुखदुःखयोरुद्वेगानुद्वेगौ च घृणाद्यर्थः ।}र्गो यस्य तस्य {\color{DodgerBlue3}“तदाख्यां\edlabel{pvv.57-2}\footnote{\label{pvv.57-2}  २ न ह्यमार्गज्ञोऽविपरीतमार्ग्गोपदेशोऽधिक्रियते ।} नं”} यस्माद् {\color{DodgerBlue3}“दुष्करं”} (। १३४)
	\pend
      \label{div_pvv.1.135}\edlabel{div_pvv.1.135}
	  
	% new div opening: depth here is 2
	

	  \begin{center}%% label @type='head'
	\textbf{(ख. दुःखहेतुपरीक्षणम्)}
	\end{center}
	

	  \pstart तत्र साक्षात्करणे परीक्षणमङ्गं तदाह (।)
	\pend
      
	  \bigskip
	  \begingroup
	  \large
	
	    
	    \stanza[\smallbreak]
	\label{pv.1.135}\edlabel{pv.1.135}\flagstanza{\tiny\textenglish{....1.135}}युक्त्यागमाभ्यां विमृशन् दुःखहेतुं परीक्षते ।&तस्यानित्यादि रूपं च दुःखस्यैव विशेषणैः ॥ १३५ ॥\&[\smallbreak]


	
	  \endgroup
	

	  \pstart {\color{DodgerBlue3}“यु\edlabel{pvv.57-3}\footnote{\label{pvv.57-3}  ३ एतेन युक्त्यायुक्तिशून्यागमाग्रहः । तर्कमात्रत्याग आगमेन तत्र निग्रहस्थानानान्तत्वज्ञानान्मोक्ष इति नैयायिकः । प्रकृतिपुरुषान्तज्ञानादिति सांख्याः कर्मक्षयादिति दिगम्बराः ।} क्त्यागमाभ्या”}मनुमानप्रवचनाभ्यां परस्परमविरुद्धाभ्यां {\color{DodgerBlue3}“विमृशन्”} विचारयन् दुःखस्य जन्मनो {\color{DodgerBlue3}“हेतुं परीक्षते”} मुमुक्षुः {\color{DodgerBlue3}“तस्य”} दुःखहेतोर{\color{DodgerBlue3}“नित्यादिरूपं”} आदिशब्दान्निवर्तनयोग्यतादिकञ्च परीक्षते । {\color{DodgerBlue3}“कथमित्याह । दुःखस्यैव विशेषणैः । कादाचित्कत्वादिभिः”} । (१३५)
	\pend
      \label{div_pvv.1.136_1.137}\edlabel{div_pvv.1.136_1.137}
	  
	% new div opening: depth here is 2
	

	  \pstart कस्मात् पुनर्दुःखस्य हेतोरनित्यत्वादि परीक्षणीयमित्याह ।
	\pend
      
	  \bigskip
	  \begingroup
	  \large
	
	    
	    \stanza[\smallbreak]
	\label{pv.1.136}\edlabel{pv.1.136}\flagstanza{\tiny\textenglish{....1.136}}यतस्तथा स्थिते हेतौ निवृत्तिर्नेति पश्यति ।&फलस्य हेतोर्हानार्थं तद्विपक्षं परीक्षते ॥ १३६ ॥\&[\smallbreak]


	
	  \endgroup
	

	  \pstart {\color{DodgerBlue3}“यतस्तथा स्थिते हेतौ नित्यत्वात्सदास्थिते हेतौ फलस्य दुःखस्य निवृत्तिर्नेति पश्यति जानाति(।) तस्मात् दुःखस्य हेतोर्हानार्थञ्च तस्या विपक्षं परीक्षते दुःखहेतुबिरुद्धं, यस्याभ्यासाद् दुःखहेतुरपैति”} ।
	\pend
      
	  \bigskip
	  \begingroup
	  \large
	
	    
	    \stanza[\smallbreak]
	\label{pv.1.137}\edlabel{pv.1.137}\flagstanza{\tiny\textenglish{....1.137}}साध्यते तद्विपक्षोपि हेतो रूपावबोधतः ॥&आत्मात्मीयग्रहकृतः स्नेहः संस्कारगोचरः ॥ १३७ ॥\&[\smallbreak]


	
	  \endgroup
	

	  \pstart {\color{DodgerBlue3}“साध्यते”} निश्चीयते {\color{DodgerBlue3}“तद्विपक्षोपि हेतो रूपावबोधतः”} । ज्ञाते हि हे\edlabel{pvv.57-4}\footnote{\label{pvv.57-4}  ४ दुःखहेतौ नित्यशुच्याद्याकारे ।} तौ तद्विरोधी बोद्धुं शक्यः । {\color{DodgerBlue3}“आत्मा\edlabel{pvv.57-5}\footnote{\label{pvv.57-5}  ५ तच्छून्ये तदभिनिवेशः । अस्य चाविद्या सहजा ।}त्मीयग्रहा”}भ्यां {\color{DodgerBlue3}“कृतः स्नेहः सँस्कारगोचरः”} (। १३७)
	\pend
      \label{div_pvv.1.138}\edlabel{div_pvv.1.138}
	  
	% new div opening: depth here is 2
	\leavevmode\marginnote{\textenglish{058/s}}

	  \begin{center}%% label @type='head'
	\textbf{(ग. नैरात्म्यदर्शनतो वासनाहानिः)}
	\end{center}
	
	  \bigskip
	  \begingroup
	  \large
	
	    
	    \stanza[\smallbreak]
	\label{pv.1.138}\edlabel{pv.1.138}\flagstanza{\tiny\textenglish{....1.138}}हेतुविरोधि नैरात्म्यदर्शनं तस्य बाधकम् ॥&बहुशो बहुधोपायं कालेन बहुनास्य च ॥ १३८ ॥\&[\smallbreak]


	
	  \endgroup
	\leavevmode\marginnote{\textenglish{12a/MA}}

	  \pstart अध्यात्मस्क\edlabel{pvv.58-1}\footnote{\label{pvv.58-1}  १ संस्काराख्येषु ।}न्धेषु तदुपकारकेषु च बाह्येषु स्नेहोस्य {\color{DodgerBlue3}“हेतु”}र्दुःखस्या\edlabel{pvv.58-2}\footnote{\label{pvv.58-2}  २ कः पुनरसौ जन्मलक्षणस्य दुःखस्य हेतुस्तद्विरुद्धो वा धर्म्म इत्याह (।) दुःखे विपर्य्यसेत्यादिनोक्तः स्नेहः कीदृश इत्यत्रात्मेत्यादिदुःखभूता आत्मीयरहिताः स्कन्धादय एवास्य गोचरो विषयः ।}त्मात्मीयग्रहवतः स्निग्धस्य तृष्णया जन्मपरिग्रहात् । तस्य {\color{DodgerBlue3}“नैरात्म्यदर्शनं विरोधि”} विपरीतालम्बनाकारत्वात् {\color{DodgerBlue3}“बाधकं”} विपक्षः । एतं दुःखहेतुं तद्विपक्षञ्चागमादुपश्रुत्यानुमानान्निश्चित्य {\color{DodgerBlue3}“बहुशो”} अनेकशो {\color{DodgerBlue3}“बहुधोपायम”}नेकप्रकारं {\color{DodgerBlue3}“कालेन च बहुनास्य”} बोधिसत्त्वस्य (। १३८)
	\pend
      \label{div_pvv.1.139}\edlabel{div_pvv.1.139}
	  
	% new div opening: depth here is 2
	
	  \bigskip
	  \begingroup
	  \large
	
	    
	    \stanza[\smallbreak]
	\label{pv.1.139}\edlabel{pv.1.139}\flagstanza{\tiny\textenglish{....1.139}}गच्छन्त्यभ्यस्यतस्तत्र गुणदोषाः प्रकाशताम् ।&बुद्धेश्च पाटवाद्धेतोर्वासनातः प्रहीयते ॥ १३९ ॥\&[\smallbreak]


	
	  \endgroup
	

	  \pstart {\color{DodgerBlue3}“अभ्यस्यतो”} भावयतः {\color{DodgerBlue3}“तत्र”} दुःखहेतौ तद्विपक्षे च {\color{DodgerBlue3}“गुणदोषा”} यथायोगं {\color{DodgerBlue3}“प्रकाशतां गच्छन्ति”} । अभ्यासाधीनो हि भाव्यमानबुद्ध्याकारविशदीभावः ।
	\pend
      

	  \pstart {\color{DodgerBlue3}“अतो”}ऽभ्यासाद् {\color{DodgerBlue3}“बुद्धेश्च पाटवाद्धेतो”}रात्मग्रहस्य तृष्णायाश्च {\color{DodgerBlue3}“वासना”} । कायवाग्बुद्धिवैगुण्यहेतुतः शक्तिलेशः {\color{DodgerBlue3}“प्रही\edlabel{pvv.58-3}\footnote{\label{pvv.58-3}  ३ एकमात्मानन्दमयिष्यामीति नैषामात्मग्रहोस्ति ।} यते”} निःशेषमपैति । (१३९)
	\pend
      \label{div_pvv.1.140}\edlabel{div_pvv.1.140}
	  
	% new div opening: depth here is 2
	

	  \begin{center}%% label @type='head'
	\textbf{(घ. प्रत्येकबुद्धादिभ्यो बुद्धे विशेषः}
	\end{center}
	
	  \bigskip
	  \begingroup
	  \large
	
	    
	    \stanza[\smallbreak]
	\label{pv.1.140a}\edlabel{pv.1.140a}\flagstanza{\tiny\textenglish{...1.140a}}परार्थवृत्तेः खङ्गादेर्विशेषोऽयं महामुनेः ।\&[\smallbreak]


	
	  \endgroup
	

	  \pstart {\color{DodgerBlue3}“अयमेव”} वासनाहानिलक्षणः खङ्गः प्रत्येकबुद्ध {\color{DodgerBlue3}“आदि”}र्यस्य श्रा\edlabel{pvv.58-4}\footnote{\label{pvv.58-4}  ४ वासनास्ति ।}वकस्य तस्मात् सकाशात् {\color{DodgerBlue3}“महामु\edlabel{pvv.58-5}\footnote{\label{pvv.58-5}  ५ क्लेशविसंयोगे}नेः”} सम्यक्सम्बुद्धस्य {\color{DodgerBlue3}“विशेषः”}\edlabel{pvv.58-6}\footnote{\label{pvv.58-6}  ६ स च परार्थवृत्तित्वात् ।} स्वार्थसम्पत्तेः ।
	\pend
      

	  \pstart नन्वेवमुपायाभ्यासो दर्शितो न शास्तृत्वं तस्योपदेष्ट्ठत्वादित्याह ।
	\pend
      
	  \bigskip
	  \begingroup
	  \large
	
	    
	    \stanza[\smallbreak]
	\label{pv.1.140b}\edlabel{pv.1.140b}\flagstanza{\tiny\textenglish{...1.140b}}उपायाभ्यास एवायं तादर्थ्याच्छासनं मतम् ॥ १४० ॥\&[\smallbreak]


	
	  \endgroup
	

	  \pstart {\color{DodgerBlue3}“उपायाभ्यास एवायं शासनं मतं तादर्थ्यात्”} शासनार्थत्वात् । कारणे कार्योपचारात् । (१४०)
	\pend
      \label{div_pvv.1.141}\edlabel{div_pvv.1.141}
	  
	% new div opening: depth here is 2
	\leavevmode\marginnote{\textenglish{059/s}}
	  \bigskip
	  \begingroup
	  \large
	
	    
	    \stanza[\smallbreak]
	\label{pv.1.141a}\edlabel{pv.1.141a}\flagstanza{\tiny\textenglish{...1.141a}}निष्पत्तेः प्रथमं भावाद्धेतुरुक्तमिदं द्वयम् ॥\&[\smallbreak]


	
	  \endgroup
	

	  \pstart सुगतत्त्वस्य फलस्य {\color{DodgerBlue3}“निष्पत्तेः प्रथमं भावा”}त्ताव{\color{DodgerBlue3}“देतत् द्वयं”} हि हितैषित्वं शास्तृत्वं {\color{DodgerBlue3}“हेतुरुक्तं”} हेत्ववस्थाया अभिधानात् ॥
	\pend
      

	  \begin{center}%% label @type='head'
	\textbf{(५) सुगतत्वात् भगवान् प्रमाणम्}
	\end{center}
	

	  \pstart सुगतत्वं व्याचिख्यासुराह (।)
	\pend
      
	  \bigskip
	  \begingroup
	  \large
	
	    
	    \stanza[\smallbreak]
	\label{pv.1.141b}\edlabel{pv.1.141b}\flagstanza{\tiny\textenglish{...1.141b}}हेतोः प्रहाणं त्रिगुणं सुगतत्वमनिश्रयात् ॥ १४१ ॥\&[\smallbreak]


	
	  \endgroup
	

	  \pstart {\color{DodgerBlue3}“हेतोः”} समुदायस्य {\color{DodgerBlue3}“प्रहाणं”} निरोधः {\color{DodgerBlue3}“सुगतत्वं”} तच्च {\color{DodgerBlue3}“त्रिगुणं”} गुणत्रययुक्तं (।) सुशब्दस्य त्रिविधोऽर्थः प्रशस्तता सुरूपवत् । अपुनरावृत्तिः अनष्टज्वरवत् । निःशेषता च अपूर्ण्णघटवत् ।
	\pend
      

	  \pstart तत्र प्रशस्तं भगवान् ज्ञातवान् सुगत इति प्रशस्तार्थमाह । {\color{DodgerBlue3}“अनिःश्रयाद”}नाश्रय\edlabel{pvv.59-1}\footnote{\label{pvv.59-1}  १ लोके हि सुखं तदनुबन्धि च प्रशस्तं तद्विपरीतं सास्रवं ।}णाद् । (१४१)
	\pend
      \label{div_pvv.1.142}\edlabel{div_pvv.1.142}
	  
	% new div opening: depth here is 2
	
	  \bigskip
	  \begingroup
	  \large
	
	    
	    \stanza[\smallbreak]
	\label{pv.1.142a}\edlabel{pv.1.142a}\flagstanza{\tiny\textenglish{...1.142a}}दुःखस्य शस्तं नैरात्म्यदृष्टेश्च युक्तितोऽपि वा ।\&[\smallbreak]


	
	  \endgroup
	

	  \pstart {\color{DodgerBlue3}“दुःखस्य”} शस्तं सुगतत्वं । तत् पुनर्दुःखानाश्रयणं {\color{DodgerBlue3}“नैरात्म्यदृष्टेः”} । आत्मदर्शी ह्यात्मनि स्निह्यन् तद्दुःखसुखपरिहारप्राप्तीच्छया जन्म दुःखरूपमादत्ते । प्रहीणात्मदर्शनस्तु नैतादृश इति प्रहीणदुःखोपायः {\color{DodgerBlue3}“युक्तितोपि वा”} युक्तिपरिदृष्टेनोपा\edlabel{pvv.59-2}\footnote{\label{pvv.59-2}  २ अनन्तरोक्तेन ।}येन वा गमनात्तत्सुगतत्त्वं प्रशस्तं ।
	\pend
      

	  \pstart अथवाऽपुनरावृत्त्या गमनं सुगतत्वं (।) तदाख्यातुमाह ।
	\pend
      
	  \bigskip
	  \begingroup
	  \large
	
	    
	    \stanza[\smallbreak]
	\label{pv.1.142b}\edlabel{pv.1.142b}\flagstanza{\tiny\textenglish{...1.142b}}पुनरावृत्तिरित्युक्तौ जन्मदोषसमुद्भवौ ॥ १४२ ॥\&[\smallbreak]


	
	  \endgroup
	

	  \pstart {\color{DodgerBlue3}“जन्मनो”} रागादेश्च {\color{DodgerBlue3}“दोषस्य समुद्भवौ”} पुनः पुनरावर्तनात् {\color{DodgerBlue3}“पुनरावृत्तिरित्युक्तौ”} (१४२)
	\pend
      \label{div_pvv.1.143_1.144_1.145}\edlabel{div_pvv.1.143_1.144_1.145}
	  
	% new div opening: depth here is 2
	

	  \begin{center}%% label @type='head'
	\textbf{(क. आत्मदर्शनबीजहानात् मुक्तिः)}
	\end{center}
	
	  \bigskip
	  \begingroup
	  \large
	
	    
	    \stanza[\smallbreak]
	\label{pv.1.143a}\edlabel{pv.1.143a}\flagstanza{\tiny\textenglish{...1.143a}}आत्मदर्शनबीजस्य हानादपुनरागमः ।&तद्भूतभिन्नात्मतया;\&[\smallbreak]


	
	  \endgroup
	

	  \pstart नैरात्म्यभावनासात्म्ये तु । {\color{DodgerBlue3}“आत्मदर्शनस्य”} जन्मप्रबन्ध{\color{DodgerBlue3}“बीजस्य हानादपुनरागमो”}ऽपुनरावृत्तिः । तत्पुनरात्मदर्शनबीजस्य हानं {\color{DodgerBlue3}“भूता”}\edlabel{pvv.59-3}\footnote{\label{pvv.59-3}  ३ भूतत्वेनात्मदर्शनसमारोपान्याकारताभिज्ञत्वेन तद्विरुद्धता । आत्मत्वेन स्वभावता कथिता (।) भूतो भिन्न आत्मा यस्य तद्भावस्तया ।} त्सत्यात् नैरात्म्याद् {\color{DodgerBlue3}“भिन्ना”} \leavevmode\marginnote{\textenglish{060/s}} {\color{DodgerBlue3}“त्मतया”}ऽन्यत्वात् । न हि सात्मीभूतप्रतिपक्षस्य विपक्षबीजसम्भवः । {\color{DodgerBlue3}“निःशेष”}म्वागमनात् सुगतत्वं (।)
	\pend
      

	  \pstart तदाह (।)
	\pend
      
	  \bigskip
	  \begingroup
	  \large
	
	    
	    \stanza[\smallbreak]
	\label{pv.1.143b}\edlabel{pv.1.143b}\flagstanza{\tiny\textenglish{...1.143b}}शेषमक्लेशनिर्ज्वरम् ॥ १४३ ॥\&[\smallbreak]


	
	  \endgroup
	
	  \bigskip
	  \begingroup
	  \large
	
	    
	    \stanza[\smallbreak]
	\label{pv.1.144a}\edlabel{pv.1.144a}\flagstanza{\tiny\textenglish{...1.144a}}कायवाग्बुद्धिवैगुण्यं मार्गोंक्त्यपटुताऽपि वा ।&अशेषहानमभ्यासाद्;\&[\smallbreak]


	
	  \endgroup
	

	  \pstart {\color{DodgerBlue3}“काय”}वैगुण्यमचापलेप्युत्प्लुत्य गमनादि । वाग्वैगुण्यं मानाभावेपि वृषलीवादादि । {\color{DodgerBlue3}“बुद्धिवैगुण्यं”} नित्या{\color{DodgerBlue3}“समाधाना”}दव्याकृतचित्तावस्थानं एतत् त्रयं शेषं सकलक्लेशोपक्लेशप्रशमादक्लेशं । निर्ज्वरञ्चादोषमूलत्वात् । {\color{DodgerBlue3}“मार्ग्ग”}स्य क्षणिकनैरात्म्यभावनादे{\color{DodgerBlue3}“रुक्तावपटुतापि वा”} शेषं । तत्परित्यागाद{\color{DodgerBlue3}“शेषहानमभ्यासा”}दिति निःशेषगमनात्सुगतत्वं ।
	\pend
      

	  \pstart दर्शितं त्रिगुणं सुगतत्वं ॥ ॰ ॥
	\pend
      

	  \begin{center}%% label @type='head'
	\textbf{(ख. बुद्धत्वबाधकयुक्तिनिरासः}
	\end{center}
	

	  \pstart तदेवं सर्व्वज्ञस्य सम्भवानुमानं प्रतिपाद्य तद्बाधकं दूषयितुमाह (।)
	\pend
      
	  \bigskip
	  \begingroup
	  \large
	
	    
	    \stanza[\smallbreak]
	\label{pv.1.144b}\edlabel{pv.1.144b}\flagstanza{\tiny\textenglish{...1.144b}}उक्त्यादेर्दोषसंक्षयः ॥ १४४ ॥\&[\smallbreak]


	
	  \endgroup
	
	  \bigskip
	  \begingroup
	  \large
	
	    
	    \stanza[\smallbreak]
	\label{pv.1.145}\edlabel{pv.1.145}\flagstanza{\tiny\textenglish{....1.145}}नेत्येके व्यतिरेकोस्य संदिग्धो व्यभितचार्यतः ।&अक्षयित्वं च दोषाणां नित्यत्वादनुपायतः ॥ १४५ ॥\&[\smallbreak]


	
	  \endgroup
	

	  \pstart {\color{DodgerBlue3}“एके”} जै मि नी या उक्त्यादेर्हेतो रथ्यापुरुषवत् रागादिदो\edlabel{pvv.60-1}\footnote{\label{pvv.60-1}  १ रागादिमान् विवक्षितः पुरुषो वक्तृत्वात् ।}षसंक्षयः कस्यचिन्नास्तीत्याहुः । {\color{DodgerBlue3}“व्यति\edlabel{pvv.60-2}\footnote{\label{pvv.60-2}  २ असति रागादिमत्वे न भवति वक्तृत्वमिति ।}रेको”} विपक्षाद् व्यावृत्तिः अस्य वक्तृत्वादिहेतोः {\color{DodgerBlue3}“संदिग्धोऽतो व्यभिचार्य”}नैकान्तिकोयमिति विस्तरतो विपञ्चयिष्यते ।
	\pend
      

	  \pstart किञ्च (।) {\color{DodgerBlue3}“अक्षयित्वं दोषाणां”} यो मन्यते स {\color{DodgerBlue3}“नित्यत्वाद्वा”}ऽकाशवत् {\color{DodgerBlue3}“अनुपायत”} उपायाभावात् वा । नष्टस्य पुनरुन्मज्जनवत् । (१४४) (१४५)
	\pend
      \label{div_pvv.1.146}\edlabel{div_pvv.1.146}
	  
	% new div opening: depth here is 2
	

	  \pstart एषां त्रयाणामपि हेतूनामसिद्धत्वं दर्शयति (।)
	\pend
      
	  \bigskip
	  \begingroup
	  \large
	
	    
	    \stanza[\smallbreak]
	\label{pv.1.146}\edlabel{pv.1.146}\flagstanza{\tiny\textenglish{....1.146}}उपायस्यापरिज्ञानादपि वा परिकल्पयेत् ।&हेतुमत्त्वात् विरुद्धस्य हेतोरभ्यासतः क्षयात्त् ॥ १४६ ॥\&[\smallbreak]


	
	  \endgroup
	\leavevmode\marginnote{\textenglish{061/s}}

	  \pstart {\color{DodgerBlue3}“उपायस्यापरिज्ञानाद् वा”} मूर्खस्य शब्दज्ञानवत् । {\color{DodgerBlue3}“परि\edlabel{pvv.61-1}\footnote{\label{pvv.61-1}  १ कश्चित्सवेता (?) ।} कल्पयेत्”} ।
	\pend
      

	  \pstart {\color{DodgerBlue3}“हेतुमत्त्वाद् विरुद्धस्य”} हेतुमत्त्वाद् दोषाणामनित्यत्वं ततोऽसिद्धं नित्यत्वं दोषाणां । हेतोरात्मदर्शनस्य विरुद्धस्य नैरात्म्यस्या{\color{DodgerBlue3}“भ्यासतः क्षयादुपायाभावोप्यसिद्धः”} । (१४६)
	\pend
      \label{div_pvv.1.147}\edlabel{div_pvv.1.147}
	  
	% new div opening: depth here is 2
	
	  \bigskip
	  \begingroup
	  \large
	
	    
	    \stanza[\smallbreak]
	\label{pv.1.147a}\edlabel{pv.1.147a}\flagstanza{\tiny\textenglish{...1.147a}}हेतुस्वभावज्ञानेन तज्ज्ञानमपि साध्यते ॥ ॰ ॥\&[\smallbreak]


	
	  \endgroup
	

	  \pstart {\color{DodgerBlue3}“हेतो”}रात्मदर्शनस्य {\color{DodgerBlue3}“स्वभावज्ञानेन तत्-ज्ञानमपि”} तन्निवृत्त्युपायस्य तद्वि-\leavevmode\marginnote{\textenglish{12b/MA}} पर्ययरूपस्य ज्ञानञ्च {\color{DodgerBlue3}“साध्यते”} । यथा दाने ज्ञाते तद्विपर्ययरूपत्वान्मात्सर्यस्य तद्विपक्षताऽवसीयत इति तृतीयोप्यसिद्धः । प्रत्युक्तं सर्व्वज्ञस्य बाधनं ॥
	\pend
      

	  \begin{center}%% label @type='head'
	\textbf{(६) तायित्वाद् भगवान् प्रमाणम्}
	\end{center}
	

	  \pstart तायित्वं व्याख्यातुमाह ।
	\pend
      
	  \bigskip
	  \begingroup
	  \large
	
	    
	    \stanza[\smallbreak]
	\label{pv.1.147b}\edlabel{pv.1.147b}\flagstanza{\tiny\textenglish{...1.147b}}तायः स्वदृष्टमार्गोक्तिः, वैफल्याद् वक्ति नानृतम् ॥ १४७ ॥\&[\smallbreak]


	
	  \endgroup
	

	  \pstart दुःखहेतुनिवर्तकत्वेन {\color{DodgerBlue3}“स्वयं दृष्टस्य \edlabel{pvv.61-2}\footnote{\label{pvv.61-2}  २ अनुलोमतः पूर्व्वपूर्व्वाज्जगद्धितैषित्वादेरुत्तरोत्तरस्य सम्भावनानुमानेनात्यन्ताभावनिरासः । नावश्यं कारणं कार्यवदिति न नियमहेतुः ।}मार्ग्ग”}स्यो{\color{DodgerBlue3}“क्ते”}र्देशना {\color{DodgerBlue3}“तायः”} । कारणे\edlabel{pvv.61-3}\footnote{\label{pvv.61-3}  ३ आस्रवक्षयाऽनुशासनीप्रातिहार्यनामके कारणे ।} कार्योपचारात् । तया हि सत्त्वान् तायते तद्योगात् तायित्वं (।) स च {\color{DodgerBlue3}“वैफल्याद्वक्ति नानृतं”} । आत्मसुखाद्यभिलाषादिना कश्चिदसत्यं वदति अज्ञानाद्वा । प्रहीणात्मदर्शनस्य साक्षात्कृततत्त्वस्य तदुभयं नास्ति । (१४७)
	\pend
      \label{div_pvv.1.148}\edlabel{div_pvv.1.148}
	  
	% new div opening: depth here is 2
	

	  \begin{center}%% label @type='head'
	\textbf{(करुणाहेतुकं सत्याभिधानम्)}
	\end{center}
	

	  \pstart विशेषतः सत्याभिधानहेतुरेव कृपाऽस्तीत्याह (।)
	\pend
      
	  \bigskip
	  \begingroup
	  \large
	
	    
	    \stanza[\smallbreak]
	\label{pv.1.148a}\edlabel{pv.1.148a}\flagstanza{\tiny\textenglish{...1.148a}}दयालुत्वात् परार्थञ्च सर्वारम्भाभियोगतः ।&तस्मात् प्रमाणं;\&[\smallbreak]


	
	  \endgroup
	

	  \pstart {\color{DodgerBlue3}“दयालुत्वा”}च्च {\color{DodgerBlue3}“परार्थञ्च सर्व्वस्य”} मार्ग्गाभ्यासादेरा{\color{DodgerBlue3}“रम्भेऽभियोगतः”} परार्थमेवोद्दिश्य भगवानभिसंबुद्धः कथन्तस्य मिथ्याभिधानेन सत्त्ववञ्चनाशङ्कापि । {\color{DodgerBlue3}“तस्मा”}त्तायित्वात् {\color{DodgerBlue3}“प्रमाणं”} भगवान् । यथादृष्टार्थप्रवक्तृत्वं हि सम्वादित्वमेवेति प्रथमप्रमाणलक्षणयोगात् प्रामाण्यमनेनोक्तं ।
	\pend
      \leavevmode\marginnote{\textenglish{062/s}}

	  \begin{center}%% label @type='head'
	\textbf{क. तायः चतुःसत्त्यप्रकाशनम्}
	\end{center}
	

	  \pstart द्वितीयलक्षणयोगमप्याह\edlabel{pvv.62-1}\footnote{\label{pvv.62-1}  १ अज्ञातार्थप्रकाशो वा द्वितीयलक्षणं ।} (।)
	\pend
      
	  \bigskip
	  \begingroup
	  \large
	
	    
	    \stanza[\smallbreak]
	\label{pv.1.148b}\edlabel{pv.1.148b}\flagstanza{\tiny\textenglish{...1.148b}}तायो वा चतुःसत्यप्रकाशनम् ॥ १४८ ॥\&[\smallbreak]


	
	  \endgroup
	

	  \pstart {\color{DodgerBlue3}“तायो वा चतुःसत्यप्रकाशनं”} । परै\edlabel{pvv.62-2}\footnote{\label{pvv.62-2}  २ अनुलोमतो व्याख्यायेदानीं प्रतिलोमतः कार्यप्रतिपत्त्या कारणसिद्धिदर्शनेन प्रामाण्यमाह ।}रज्ञातस्य सत्यचतुष्टयस्य प्रकाशनम्वा तायः । तद्योगात् तायी प्रमाणं भगवानुक्तः । (१४८)
	\pend
      \label{div_pvv.1.149}\edlabel{div_pvv.1.149}
	  
	% new div opening: depth here is 2
	

	  \begin{center}%% label @type='head'
	\textbf{ख. चत्वारि आर्य-सत्यानि}
	\end{center}
	

	  \begin{center}%% label @type='head'
	\textbf{(क) दुःखसत्त्यम्}
	\end{center}
	

	  \begin{center}%% label @type='head'
	\textbf{I. संसारिणः स्कन्धा दुःखम्}
	\end{center}
	

	  \pstart आर्यसत्येषु दुःखमाह ।
	\pend
      
	  \bigskip
	  \begingroup
	  \large
	
	    
	    \stanza[\smallbreak]
	\label{pv.1.149a}\edlabel{pv.1.149a}\flagstanza{\tiny\textenglish{...1.149a}}दुःखं संसारिणः स्कन्धाः;\&[\smallbreak]


	
	  \endgroup
	

	  \pstart रूपवेदनासंज्ञासंस्कारविज्ञानाख्याः पञ्च {\color{DodgerBlue3}“स्कन्धाः”} । जन्ममरणप्रबन्धः {\color{DodgerBlue3}“संसारः”} । तद्व\edlabel{pvv.62-3}\footnote{\label{pvv.62-3}  ३ संसारिणः स्कन्धाः ।}न्तो {\color{DodgerBlue3}“दुःखं”} तिसृभिर्दुः खताभिः ।
	\pend
      

	  \pstart ननु यदि स्कन्धा एव प्रतीत्यसमुत्पन्ना न तु कश्चित्सत्त्वो यः संसरति तदा {\color{DodgerBlue3}“रागादयो”} यादृच्छिका अहेतवः स्युरित्याह (।)
	\pend
      
	  \bigskip
	  \begingroup
	  \large
	
	    
	    \stanza[\smallbreak]
	\label{pv.1.149b}\edlabel{pv.1.149b}\flagstanza{\tiny\textenglish{...1.149b}}रागादेः पाटवेक्षणात् ।&अभ्यासान्न यदृच्छातोऽहेतोर्जन्मविरोधतः ॥ १४९ ॥\&[\smallbreak]


	
	  \endgroup
	

	  \pstart अभ्यासा{\color{DodgerBlue3}“द्रागादेः पाटवस्येक्षणात् । अभ्यासादेव”} ते भवन्ति {\color{DodgerBlue3}“न”} तु {\color{DodgerBlue3}“यदृच्छातः”} । न ह्यकारणाद्विशेषसम्भवः । नाप्यहेतुका रागादयोऽ{\color{DodgerBlue3}“हेतो”}र्हेतुरहितस्य {\color{DodgerBlue3}“जन्मविरोधतः”} । न ह्याकाशं कदाचिज्जायते । (१४९)
	\pend
      \label{div_pvv.1.150}\edlabel{div_pvv.1.150}
	  
	% new div opening: depth here is 2
	

	  \begin{center}%% label @type='head'
	\textbf{II. रागादीनां वातादिदोषजत्वनिरासः}
	\end{center}
	

	  \pstart स्यादेतत् (।) वातप्रकृतिर्मोहवान् । पित्तप्रकृतिर्द्वेषवान् । श्लेष्मप्रकृती रागवानिति वातादिधर्मो दोषगण इत्याह ।
	\pend
      \leavevmode\marginnote{\textenglish{063/s}}
	  \bigskip
	  \begingroup
	  \large
	
	    
	    \stanza[\smallbreak]
	\label{pv.1.150}\edlabel{pv.1.150}\flagstanza{\tiny\textenglish{....1.150}}व्यभिचारान्न वातादिधर्मः प्रकृतिसंकरात् ।&अदोषश्च तदन्योऽपि धर्मः किं तस्य नेक्ष्यते ॥ १५० ॥\&[\smallbreak]


	
	  \endgroup
	

	  \pstart a. {\color{DodgerBlue3}“न व्यभिचाराद् वातादिधर्मो”} रागादिः । वातप्रकृतिरपि {\color{DodgerBlue3}“न मोहबहुलः”} । पित्तप्रकृतिरपि न पटुद्वेषः । श्लेष्मप्रकृतिश्च नोद्भूतरागविशेषः कश्चिद् दृश्यत इति वातादिव्यभिचारिणो मोहादयो न तद्धेतवः ॥ {\color{DodgerBlue3}“प्रकृतिसंकाराददोषश्चेत्”} संकीर्ण्णप्रकृतयो हि पुरुषाः प्रत्येकं वातपित्तश्लेष्मणां सत्त्वात् । अतो {\color{DodgerBlue3}“दोषाणां न”} कारणव्यभिचारः । यद्येवं तस्याद्वेषादेरन्यो\edlabel{pvv.63-1}\footnote{\label{pvv.63-1}  १ एकः कोपनः प्राज्ञः प्रस्वेदादिमांश्च न स्यादस्ति च (।) पित्तगुणसमुदायः कोपपरिणामकाले प्रज्ञापरिणामोपि मा भूत् (।) न हि लाभे प्रवेशच्छेदेपि पित्तप्रकृतेः निःशरणं युक्तं ।} धर्मः खरत्वादिः\edlabel{pvv.63-2}\footnote{\label{pvv.63-2}  २ गन्धादिर्न दृश्यते कुतः ।\begin{english}\par
Placement of note uncertain; marked with a question mark in the edition (see encoding description for details).\end{english}} {\color{DodgerBlue3}“तस्य वातादेः किन्नेक्ष्यते\edlabel{pvv.63-3}\footnote{\label{pvv.63-3}  ३ न दृश्यते च तदसदेतत् ।\begin{english}\par
Placement of note uncertain; marked with a question mark in the edition (see encoding description for details).\end{english}}”} । (१५०)
	\pend
      \label{div_pvv.1.151}\edlabel{div_pvv.1.151}
	  
	% new div opening: depth here is 2
	

	  \pstart b. अथ प्रत्येकं स\edlabel{pvv.63-4}\footnote{\label{pvv.63-4}  ४ वातादीनां ।}र्व्वेषां रागादिर्धर्मस्ततो न व्यभिचार इत्याह (।)
	\pend
      
	  \bigskip
	  \begingroup
	  \large
	
	    
	    \stanza[\smallbreak]
	\label{pv.1.151}\edlabel{pv.1.151}\flagstanza{\tiny\textenglish{....1.151}}न सर्वधर्मः सर्वेंषां समरागप्रसङ्गतः ।&रूपादिवददोषश्चेत तुल्यं तत्रापि चोदनम् ॥ १५१ ॥\&[\smallbreak]


	
	  \endgroup
	

	  \pstart {\color{DodgerBlue3}“न सर्व्वधर्मः सर्व्वेषां स\edlabel{pvv.63-5}\footnote{\label{pvv.63-5}  ५ एवं तुल्यद्वेषादयः ।}मस्य रागस्य प्रसङ्गात्”} । नानाप्रकृतिकत्वेपि रागादिहेतोः समानत्वात् । {\color{DodgerBlue3}“रूपादिवददोषश्चेत्”} । यथा भूतमात्रहेतुकत्वेपि रू\edlabel{pvv.63-6}\footnote{\label{pvv.63-6}  ६ क्वचिद्रसः क्वचित् स्पर्शः ।}पादय उत्कृष्यन्तेऽयकृष्यन्ते च क्वचित् तथा रागादयोपीति चेत् । {\color{DodgerBlue3}“तुल्यं तत्र”} रूपादा{\color{DodgerBlue3}“वपि चोदनं”} समत्वस्य । (१५१)
	\pend
      \label{div_pvv.1.152}\edlabel{div_pvv.1.152}
	  
	% new div opening: depth here is 2
	
	  \bigskip
	  \begingroup
	  \large
	
	    
	    \stanza[\smallbreak]
	\label{pv.1.152a}\edlabel{pv.1.152a}\flagstanza{\tiny\textenglish{...1.152a}}आधिपत्यं विशिष्टानां यदि तत्र न कर्मणाम् ।\&[\smallbreak]


	
	  \endgroup
	

	  \pstart {\color{DodgerBlue3}“आधिपत्यं विशिष्टानां कर्मणां”} श्रुताश्रुतलक्षणानां । {\color{DodgerBlue3}“यदि तत्र”} रूपादौ कार्ये नेष्यते भूतसहकारिणां कर्मणां वैशिष्ट्यात् रूपादिविशेष इत्यर्थः ।
	\pend
      

	  \pstart स्यादेतन्न दोषमात्राद्रागादयोऽपि तु तेषां परिणाम\edlabel{pvv.63-7}\footnote{\label{pvv.63-7}  ७ रागादेर्यः स्वस्वकीयः परिणामविशेषः तदभावात् ।} विशेषात् यथा व्याधयः ।\edlabel{pvv.63-8}\footnote{\label{pvv.63-8}  ८ द्वेषपरिणामभावेपि ।} ततो न समरागतादिप्रसङ्ग इत्याह (।)
	\pend
      
	  \bigskip
	  \begingroup
	  \large
	
	    
	    \stanza[\smallbreak]
	\label{pv.1.152b}\edlabel{pv.1.152b}\flagstanza{\tiny\textenglish{...1.152b}}विशेषेपि च दोषाणामविशेषाद्;\&[\smallbreak]


	
	  \endgroup
	\leavevmode\marginnote{\textenglish{064/s}}

	  \pstart {\color{DodgerBlue3}“विशेषेपि चे दोषाणां”} प्रकोपादिनाऽ{\color{DodgerBlue3}“विशेषात्”} रागादीनां न दोषपरिणामहेतुता । (१५२)
	\pend
      \label{div_pvv.1.153_1.154_1.155}\edlabel{div_pvv.1.153_1.154_1.155}
	  
	% new div opening: depth here is 2
	

	  \pstart नन्वविशेषादित्यसिद्धो हेतुः । कफाद्युत्कर्षे रागाद्युत्कर्षदृष्टेरित्या\edlabel{pvv.64-1}\footnote{\label{pvv.64-1}  १ निजो योस्ति कर्कशत्वादौ ।} ह ।
	\pend
      
	  \bigskip
	  \begingroup
	  \large
	
	    
	    \stanza[\smallbreak]
	\label{pv.1.152c}\edlabel{pv.1.152c}\flagstanza{\tiny\textenglish{...1.152c}}असिद्धता ॥ १५२ ॥\&[\smallbreak]


	
	  \endgroup
	
	  \bigskip
	  \begingroup
	  \large
	
	    
	    \stanza[\smallbreak]
	\label{pv.1.153a}\edlabel{pv.1.153a}\flagstanza{\tiny\textenglish{...1.153a}}न विकाराद् विकारेण सर्वेषां न च सर्वजाः ।\&[\smallbreak]


	
	  \endgroup
	

	  \pstart C. {\color{DodgerBlue3}“नासिद्धता”}ऽविशेषस्य । तथा हि {\color{DodgerBlue3}“सर्व्वेषां”} कफादीनां विकारेणोत्कर्षेण पीडया विकारात् द्वेषो भवति न रागादयः । सर्व्वजत्वाददोष इति चेत् । न च {\color{DodgerBlue3}“सर्व्वजाः”} । समरागतादिप्रसङ्गादित्युक्तेः ॥
	\pend
      

	  \pstart किञ्च (।)
	\pend
      
	  \bigskip
	  \begingroup
	  \large
	
	    
	    \stanza[\smallbreak]
	\label{pv.1.153b}\edlabel{pv.1.153b}\flagstanza{\tiny\textenglish{...1.153b}}कारणे वर्द्धमाने च कार्यहानिर्न युज्यते ॥ १५३ ॥\&[\smallbreak]


	
	  \endgroup
	
	  \bigskip
	  \begingroup
	  \large
	
	    
	    \stanza[\smallbreak]
	\label{pv.1.154a}\edlabel{pv.1.154a}\flagstanza{\tiny\textenglish{...1.154a}}तापादिष्विव;\&[\smallbreak]


	
	  \endgroup
	

	  \pstart सन्निपातावस्थायां कारणे कफादिके {\color{DodgerBlue3}“वर्द्धमाने कार्यस्य”} रागादे{\color{DodgerBlue3}“र्हानिर्न युज्यते”} (१५३) {\color{DodgerBlue3}“तापादिष्विव”} तापादीनामिव पित्तादिवृद्धौ ।
	\pend
      

	  \pstart d. स्यादेतद् (।) दोषाणां साम्ये रागादयो भवन्ति वैषम्ये तु द्वेषादयः । ततो \leavevmode\marginnote{\textenglish{13a/MA}} विशेषेपि दोषाणां न विशिष्यन्ते रागादय इत्यसिद्धो हेतुरित्याह (।)
	\pend
      
	  \bigskip
	  \begingroup
	  \large
	
	    
	    \stanza[\smallbreak]
	\label{pv.1.154b}\edlabel{pv.1.154b}\flagstanza{\tiny\textenglish{...1.154b}}रागादेर्विकारोऽपि सुखादिजः ।\&[\smallbreak]


	
	  \endgroup
	

	  \pstart {\color{DodgerBlue3}“रागादेर्दो”}षसाम्यावस्थायां {\color{DodgerBlue3}“विकारोपि”} वृद्धिलक्षणो यः स {\color{DodgerBlue3}“सुखादिजः”} आन्तरधातुसाम्यस्पर्शप्रभवेन सुखादिना वर्द्धन्ते रागाद\edlabel{pvv.64-2}\footnote{\label{pvv.64-2}  २ परा श्लेष्मादिवृद्धावपि यन्न रागादिवृद्धिः सा ।}यः ।
	\pend
      
	  \bigskip
	  \begingroup
	  \large
	
	    
	    \stanza[\smallbreak]
	\label{pv.1.154c}\edlabel{pv.1.154c}\flagstanza{\tiny\textenglish{...1.154c}}वैषम्यजेन दुःखेन रागास्यानुद्भवो यदि ॥ १५४ ॥\&[\smallbreak]


	
	  \endgroup
	
	  \bigskip
	  \begingroup
	  \large
	
	    
	    \stanza[\smallbreak]
	\label{pv.1.155a}\edlabel{pv.1.155a}\flagstanza{\tiny\textenglish{...1.155a}}वाच्यं केनोद्भवः साम्यान्मदवृद्धिः स्मरस्ततः ।\&[\smallbreak]


	
	  \endgroup
	

	  \pstart {\color{DodgerBlue3}“वैषम्यजेन”} दुःखेन द्वेषस्योत्पादनात् तद्विरुद्धस्य {\color{DodgerBlue3}“रागास्यानुद्भवो यदि”} मतः (१५४) तदा {\color{DodgerBlue3}“वाच्यं केन”} हेतूना राग{\color{DodgerBlue3}“स्योद्भवः”} । {\color{DodgerBlue3}“साम्याद्”} दोषाणां {\color{DodgerBlue3}“मद”}स्य शुक्रस्य {\color{DodgerBlue3}“वृद्धिः”} (।) {\color{DodgerBlue3}“स्मरो”} रागः । शुक्रवृद्धेर्यदीष्टमेतदप्ययुक्तं यतः (।)
	\pend
      
	  \bigskip
	  \begingroup
	  \large
	
	    
	    \stanza[\smallbreak]
	\label{pv.1.155b}\edlabel{pv.1.155b}\flagstanza{\tiny\textenglish{...1.155b}}रागी विषमदोषोऽपि दृष्टः साम्येऽपि नापरः ॥ १५५ ॥\&[\smallbreak]


	
	  \endgroup
	

	  \pstart {\color{DodgerBlue3}“रागी विषमदो\edlabel{pvv.64-3}\footnote{\label{pvv.64-3}  ३ धातुवैषम्येपि ।}षोऽपि”} रागचरितः कश्चिद् {\color{DodgerBlue3}“दृष्टः साम्येपि नापरः”} प्रतिसंख्यानबली मन्दरागप्रकृतिर्व्वा दृष्टः । (१५५)
	\pend
      \label{div_pvv.1.156}\edlabel{div_pvv.1.156}
	  
	% new div opening: depth here is 2
	\leavevmode\marginnote{\textenglish{065/s}}
	  \bigskip
	  \begingroup
	  \large
	
	    
	    \stanza[\smallbreak]
	\label{pv.1.156a}\edlabel{pv.1.156a}\flagstanza{\tiny\textenglish{...1.156a}}क्षयादसृकस्रुतोऽप्यन्ये;\&[\smallbreak]


	
	  \endgroup
	

	  \pstart {\color{DodgerBlue3}“क्षया”}च्छुक्रस्यासुकश्रु (? स्रु) तो रक्तं {\color{DodgerBlue3}“क्षर”}\edlabel{pvv.65-1}\footnote{\label{pvv.65-1}  १ इन्द्रियेण ।} {\color{DodgerBlue3}“न्तोप्यन्ये रागबहुला दृष्टा इति”} शुक्रमपि न रागहेतुः ।
	\pend
      

	  \pstart किञ्च (।)
	\pend
      
	  \bigskip
	  \begingroup
	  \large
	
	    
	    \stanza[\smallbreak]
	\label{pv.1.156b}\edlabel{pv.1.156b}\flagstanza{\tiny\textenglish{...1.156b}}नैकस्त्रीनियतो मदः ।\&[\smallbreak]


	
	  \endgroup
	

	  \pstart नैकस्त्रीनियतो मदः । न ह्येकां स्त्रियमपेक्ष्य शुक्रं शुक्रीभवति । अपि तु सर्व्वाः ।
	\pend
      
	  \bigskip
	  \begingroup
	  \large
	
	    
	    \stanza[\smallbreak]
	\label{pv.1.156c}\edlabel{pv.1.156c}\flagstanza{\tiny\textenglish{...1.156c}}तेनैकस्यां न तीव्रः स्याद्, अङ्गरूपाद्यपीति चेत् ॥ १५६ ॥\&[\smallbreak]


	
	  \endgroup
	

	  \pstart {\color{DodgerBlue3}“तेन”} साधारणत्वेन शुक्रस्य तज्जन्यो राग {\color{DodgerBlue3}“एकस्यां”} स्त्रियां {\color{DodgerBlue3}“न तीव्रः स्यात्\edlabel{pvv.65-2}\footnote{\label{pvv.65-2}  २ कस्यचिद् भवति च ततो न मदः कारण रागस्य ।}”} किन्तु साधारणः । {\color{DodgerBlue3}“अङ्गरूपाद्य\edlabel{pvv.65-3}\footnote{\label{pvv.65-3}  ३ रूप उपचारगुणरागी चेति त्रिविध आदिना रागी ।}पीति चेत्”} । (१५६)
	\pend
      \label{div_pvv.1.157}\edlabel{div_pvv.1.157}
	  
	% new div opening: depth here is 2
	

	  \pstart e. रूपयौवनोपचारादि च सहकारि रागवृद्धेरिति चेत् ।
	\pend
      
	  \bigskip
	  \begingroup
	  \large
	
	    
	    \stanza[\smallbreak]
	\label{pv.1.157}\edlabel{pv.1.157}\flagstanza{\tiny\textenglish{....1.157}}न सर्वेषामनेकान्तान्न चाप्यनियतो भवेत् ।&अगुणग्राहिणोऽपि स्यात् अङ्गं सो ऽपि गुणग्रहः ॥ १५७ ॥\&[\smallbreak]


	
	  \endgroup
	

	  \pstart न युक्तमेतत् । {\color{DodgerBlue3}“सर्व्वेषां”} रूपादीना{\color{DodgerBlue3}“मनेकान्त”}त्वात् । रूपादिरहितेष्वपि रागोत्कर्षदृष्टेः । किञ्च (।) यदि शुक्रं रूपादि च स्त्रियाः कारणं {\color{DodgerBlue3}“रागस्य”} तदा {\color{DodgerBlue3}“न चाप्यनियतोऽ”}विषयीकृतस्त्रीविशेषः साधारणो रागो न {\color{DodgerBlue3}“भवेत्”} । रूपादिसहकारिणोऽव्यापारात् । तथा रूपवत्या गुणमशुभाभावनाभावितं {\color{DodgerBlue3}“गृह्णतोपि”} रागः स्यात् । शुक्ररूपयोस्तद्धेत्वोः सद्भावात् । अङ्गन्निमित्तं {\color{DodgerBlue3}“सोपि गुणग्रह”}स्ततोऽशुभां भावयतो न स्याद्रागः (। १५७)
	\pend
      \label{div_pvv.1.158}\edlabel{div_pvv.1.158}
	  
	% new div opening: depth here is 2
	
	  \bigskip
	  \begingroup
	  \large
	
	    
	    \stanza[\smallbreak]
	\label{pv.1.158a}\edlabel{pv.1.158a}\flagstanza{\tiny\textenglish{...1.158a}}यदि सर्वो गुणग्राही स्याद्, हेतोरविशेषतः ।\&[\smallbreak]


	
	  \endgroup
	

	  \pstart {\color{DodgerBlue3}“यदि सर्व्वः”} शुभाशुभभावको {\color{DodgerBlue3}“गुणग्राही”} स्यात् । गुणग्रहस्य {\color{DodgerBlue3}“हेतो रूपादेः”} सर्व्वान् {\color{DodgerBlue3}“प्रत्यविशेषतः”} ।
	\pend
      

	  \pstart किञ्च (।)
	\pend
      
	  \bigskip
	  \begingroup
	  \large
	
	    
	    \stanza[\smallbreak]
	\label{pv.1.158b}\edlabel{pv.1.158b}\flagstanza{\tiny\textenglish{...1.158b}}यदवस्थो मतो रागी न द्वेषी स्याच्च तादृशः ॥ १५८ ॥\&[\smallbreak]


	
	  \endgroup
	

	  \pstart {\color{DodgerBlue3}“यदवस्थः”} श्लेष्मप्रकृतिस्थः पुरुषो {\color{DodgerBlue3}“रागी मतस्त”}दपस्थो {\color{DodgerBlue3}“न द्वेषी स्यात्”} । (१५८)
	\pend
      \label{div_pvv.1.159_1.160_1.161_1.162_1.163_1.164_1.165_1.166}\edlabel{div_pvv.1.159_1.160_1.161_1.162_1.163_1.164_1.165_1.166}
	  
	% new div opening: depth here is 2
	

	  \pstart कुत इत्याह (।)
	\pend
      \leavevmode\marginnote{\textenglish{066/s}}
	  \bigskip
	  \begingroup
	  \large
	
	    
	    \stanza[\smallbreak]
	\label{pv.1.159a}\edlabel{pv.1.159a}\flagstanza{\tiny\textenglish{...1.159a}}तयोरसमरूपत्वान्नियमश्चात्र नेक्ष्यते ।\&[\smallbreak]


	
	  \endgroup
	

	  \pstart {\color{DodgerBlue3}“तयो”} रागद्वेषयोरसमरूपत्वाद्विरुद्धत्वात् तदुत्पादिकयोरवस्थयोरपि विरोधः ॥ स्यादेतद्रागोत्पादिकायामवस्थायां द्वेषो न भवत्येवेत्याह (।) {\color{DodgerBlue3}“नियमश्चात्र नेक्ष्यते”} । श्लेष्मावस्थो रागी न द्वेषीति नात्र नियमः ।
	\pend
      

	  \pstart तवापि कस्मादमी समरागप्रसङ्गादयो दोषा न भवन्तीत्याह (।)
	\pend
      
	  \bigskip
	  \begingroup
	  \large
	
	    
	    \stanza[\smallbreak]
	\label{pv.1.159b}\edlabel{pv.1.159b}\flagstanza{\tiny\textenglish{...1.159b}}सजातिवासनाभेदप्रतिबद्धप्रवृत्तयः ॥ १५९ ॥\&[\smallbreak]


	
	  \endgroup
	
	  \bigskip
	  \begingroup
	  \large
	
	    
	    \stanza[\smallbreak]
	\label{pv.1.160a}\edlabel{pv.1.160a}\flagstanza{\tiny\textenglish{...1.160a}}यस्य रागादयस्तस्य नैते दोषाः प्रसङ्गिनः ।\&[\smallbreak]


	
	  \endgroup
	

	  \pstart {\color{DodgerBlue3}“सजातिवासना”} आत्मात्मीयग्रहमूलस्य स\edlabel{pvv.66-1}\footnote{\label{pvv.66-1}  १ सत्कायदर्शनस्य ।} जातेः पूर्व्वपूर्व्वाभ्यस्तस्य रागादेर्व्वासनाऽपरापररागादिजनिकाः शक्तयः तासां {\color{DodgerBlue3}“भेदः”} परस्परतः तत्र {\color{DodgerBlue3}“प्रतिबद्धा प्रवृत्ति”}र्जन्म येषां ते तथा (१५९) {\color{DodgerBlue3}“रागादयो यस्य”} बौद्धस्य मते {\color{DodgerBlue3}“तस्य नैते”}ऽनन्तरमुक्ता {\color{DodgerBlue3}“दोषाः प्रसङ्गिनः”} ।
	\pend
      

	  \begin{center}%% label @type='head'
	\textbf{III. रागादीनां भूतधर्मत्वनिरासः}
	\end{center}
	
	  \bigskip
	  \begingroup
	  \large
	
	    
	    \stanza[\smallbreak]
	\label{pv.1.160b}\edlabel{pv.1.160b}\flagstanza{\tiny\textenglish{...1.160b}}एतेन भूतधर्मत्वं निषिद्धं;\&[\smallbreak]


	
	  \endgroup
	

	  \pstart {\color{DodgerBlue3}“एतेन”} वातादिधर्मत्वनिषेधेन {\color{DodgerBlue3}“भूतधर्मत्वं निषिद्धं”} रागादेर्ब्बोंद्धव्यं {\color{DodgerBlue3}“दोषा\edlabel{pvv.66-2}\footnote{\label{pvv.66-2}  २ वातपित्तश्लेष्मणां क्रमात् महदादित्वेन रागादेर्भूतधर्मत्वं न भूत/?/त्वे हेतुरयं ।}णां”} मरुत्तेजोऽम्भः स्वभावत्वात् ॥
	\pend
      

	  \pstart भवन्तु सभा\edlabel{pvv.66-3}\footnote{\label{pvv.66-3}  ३ स्वसदृशहेतुका रागादयः सुराशक्त्योरिव ।}गहेतुकाः पृथिव्याद्याश्रितास्तु स्युर्द्धवलादिवदित्याह ।
	\pend
      
	  \bigskip
	  \begingroup
	  \large
	
	    
	    \stanza[\smallbreak]
	\label{pv.1.160c}\edlabel{pv.1.160c}\flagstanza{\tiny\textenglish{...1.160c}}निश्चयस्य च ॥ १६० ॥\&[\smallbreak]


	
	  \endgroup
	
	  \bigskip
	  \begingroup
	  \large
	
	    
	    \stanza[\smallbreak]
	\label{pv.1.161a}\edlabel{pv.1.161a}\flagstanza{\tiny\textenglish{...1.161a}}निषेधान्न पृथिव्यादिनिःश्रिता धवलादयः ।\&[\smallbreak]


	
	  \endgroup
	

	  \pstart {\color{DodgerBlue3}“निश्रयस्या”}श्रयस्य {\color{DodgerBlue3}“निषेधाद”}नाश्रयात्सदसतोरित्यादिना {\color{DodgerBlue3}“न पृथिव्यादिनिःश्रिता धवलादयोपि”} । कुत एव रागादयः ।
	\pend
      

	  \pstart कथन्तर्हि भूतान्याश्रित्योपादाय रूपमुत्पद्यत इतीष्टमित्याह (।)
	\pend
      
	  \bigskip
	  \begingroup
	  \large
	
	    
	    \stanza[\smallbreak]
	\label{pv.1.161b}\edlabel{pv.1.161b}\flagstanza{\tiny\textenglish{...1.161b}}तदुपादाय-शब्दश्च हेत्वर्थः स्वाश्रयेण च ॥ १६१ ॥\&[\smallbreak]


	
	  \endgroup
	
	  \bigskip
	  \begingroup
	  \large
	
	    
	    \stanza[\smallbreak]
	\label{pv.1.162a}\edlabel{pv.1.162a}\flagstanza{\tiny\textenglish{...1.162a}}अविनिर्भागवर्तित्वाद् रूपादेराश्रयोऽपि वा\&[\smallbreak]


	
	  \endgroup
	\leavevmode\marginnote{\textenglish{067/s}}

	  \pstart {\color{DodgerBlue3}“तदुपादाय शब्दश्च”} तानि भूतानि उपादाय शब्दश्च {\color{DodgerBlue3}“हेत्वर्थः”} । भूतानि हेतूकृत्योपादाय रूपमुत्पद्यत इत्यर्थः । {\color{DodgerBlue3}“स्वा\edlabel{pvv.67-1}\footnote{\label{pvv.67-1}  १ अभ्युपगम्याह ।}श्रयेण”} भूतचतुष्केण {\color{DodgerBlue3}“रूपादे”}रेकसामग्र्‏यधीनत्वेन (१६१) {\color{DodgerBlue3}“अविनिर्भागवर्त्तित्वा”}द्विभागेनानवस्थिते{\color{DodgerBlue3}“राश्रयोपि वा”} भूतचतुष्कं ॥
	\pend
      
	  \bigskip
	  \begingroup
	  \large
	
	    
	    \stanza[\smallbreak]
	\label{pv.1.162b}\edlabel{pv.1.162b}\flagstanza{\tiny\textenglish{...1.162b}}मदादिशक्तेरिव चेद् विनिर्भागो;\&[\smallbreak]


	
	  \endgroup
	

	  \pstart स्यादेतत् (।) सुराया {\color{DodgerBlue3}“मद”}शक्तेरा{\color{DodgerBlue3}“दि”}शब्दात् क\edlabel{pvv.67-2}\footnote{\label{pvv.67-2}  २ धत्तूर}नकादेरुन्माद{\color{DodgerBlue3}“शक्ते\edlabel{pvv.67-3}\footnote{\label{pvv.67-3}  ३ शुक्तिभावे, मद्यनिवृत्तेर्व्वस्त्वनिवृत्तौ ।}रिवाश्रित”}त्वेपि {\color{DodgerBlue3}“विनिर्भागो”} भूतैः सह चैतन्यस्य स्यात् ।---
	\pend
      

	  \pstart a. भूतचैन्ययोर्भेदात्
	\pend
      

	  \pstart यथा मदशक्तिरहिता व्यापन्ना सुरा दृश्यते । एवं चैतन्यरहितानि भूतान्यपि स्युरिति चेत् (।)
	\pend
      
	  \bigskip
	  \begingroup
	  \large
	
	    
	    \stanza[\smallbreak]
	\label{pv.1.162c}\edlabel{pv.1.162c}\flagstanza{\tiny\textenglish{...1.162c}}न वस्तुनः ॥ १६२ ॥\&[\smallbreak]


	
	  \endgroup
	
	  \bigskip
	  \begingroup
	  \large
	
	    
	    \stanza[\smallbreak]
	\label{pv.1.163}\edlabel{pv.1.163}\flagstanza{\tiny\textenglish{....1.163}}शक्तिरर्थान्तरं वस्तु नश्येन्नाश्रितमाश्रये ।&तिष्ठत्यविकले याति, तत्तुल्यं चेन्न भेदतः ॥ १६३ ॥\&[\smallbreak]


	
	  \endgroup
	
	  \bigskip
	  \begingroup
	  \large
	
	    
	    \stanza[\smallbreak]
	\label{pv.1.164a}\edlabel{pv.1.164a}\flagstanza{\tiny\textenglish{...1.164a}}भूतचैतनयोः;\&[\smallbreak]


	
	  \endgroup
	

	  \pstart {\color{DodgerBlue3}“न वस्तुनः”} सुरादेः {\color{DodgerBlue3}“शक्तिरर्थान्तरं”} किन्तु {\color{DodgerBlue3}“वस्त्वे”}वार्थक्रियाशक्तं शुक्त्याद्यवस्थायां {\color{DodgerBlue3}“न\edlabel{pvv.67-4}\footnote{\label{pvv.67-4}  ४ क्षणिकत्वात् ।}श्येत्”} । असमर्थस्य चोत्पत्ति \edlabel{pvv.67-5}\footnote{\label{pvv.67-5}  ५ अथ सुरास्थितैव तदाह ।} {\color{DodgerBlue3}“र्नाश्रितं”} शक्त्याद्य{\color{DodgerBlue3}“विकल आश्रये तिष्ठति यात्य-”} पति । {\color{DodgerBlue3}“तत्तुल्यञ्चेत्”} भूतचेतनयोर\edlabel{pvv.67-6}\footnote{\label{pvv.67-6}  ६ शुक्लत्वादिवद् भूताश्रितत्वं ।}प्यैकात्म्यं । तद्भूतचैतन्ये भूते नष्टे निश्चेतनं भूतान्तरमुत्पद्यत इति चेत् । {\color{DodgerBlue3}“न”} युक्तमेतत् । {\color{DodgerBlue3}“भेदतो”} (१६३) भूतचैतन्ययोः ।---
	\pend
      

	  \pstart कथमित्याह (।)
	\pend
      
	  \bigskip
	  \begingroup
	  \large
	
	    
	    \stanza[\smallbreak]
	\label{pv.1.164b}\edlabel{pv.1.164b}\flagstanza{\tiny\textenglish{...1.164b}}भिन्नप्रतिभासावबोधतः ।&अविकारञ्च कायस्य तुल्यरूपं भवेन्मनः ॥ १६४ ॥\&[\smallbreak]


	
	  \endgroup
	
	  \bigskip
	  \begingroup
	  \large
	
	    
	    \stanza[\smallbreak]
	\label{pv.1.165a}\edlabel{pv.1.165a}\flagstanza{\tiny\textenglish{...1.165a}}रूपादिवत्;\&[\smallbreak]


	
	  \endgroup
	

	  \pstart {\color{DodgerBlue3}“भिन्नप्रतिभासावबोधतः”} । भिन्नाकारज्ञानविषयतया भिन्ने\edlabel{pvv.67-7}\footnote{\label{pvv.67-7}  ७ भिन्नाकारज्ञानाभ्यामनयोर्ग्रहादन्यथातिप्रसङ्गः विश्वमेकं स्यात् कायमनइन्द्रियाभ्यां च ग्राह्यत्वात् । यदा शरीरमनालम्व्य विषयान्तरमालम्बते तदाऽशरीराकारोदयात् ।} भूतचैतन्ये {\color{DodgerBlue3}“नान्यथा”} \leavevmode\marginnote{\textenglish{068/s}} {\color{DodgerBlue3}“क्वचिदपि भेदसिद्धिः”} । यदि च देहचितयोरैक्यं तदा {\color{DodgerBlue3}“अविकारञ्च कायस्य”} यावन्न विक्रियते देहन्तावत्तुल्यमेकाकार{\color{DodgerBlue3}“म्भवेन्मनो”}\edlabel{pvv.68-1}\footnote{\label{pvv.68-1}  १ भिन्नभिन्नार्थाकारं मनोज्ञानं न स्यात् ।}(१६४) {\color{DodgerBlue3}“रूपादिवत्”} ।
	\pend
      

	  \begin{center}%% label @type='head'
	\textbf{(b. वासनाभेदतो भेदात्)}
	\end{center}
	

	  \pstart स्यादेतत् (।) एकरूपत्वेपि देहस्यार्थानां नानारूपत्वात् ज्ञानमपि तथे\leavevmode\marginnote{\textenglish{13b/MA}} त्याह (।)
	\pend
      
	  \bigskip
	  \begingroup
	  \large
	
	    
	    \stanza[\smallbreak]
	\label{pv.1.165b}\edlabel{pv.1.165b}\flagstanza{\tiny\textenglish{...1.165b}}विकल्पस्य कैवार्थपरतन्त्रता ।\&[\smallbreak]


	
	  \endgroup
	

	  \pstart {\color{DodgerBlue3}“विकल्पस्य कैवार्थपरतन्त्रता”} । येनार्थनानात्वात् कल्पनापि तथा स्यात् । अनपेक्षितसन्निधयः उपादानवशेन विकल्पाः प्रवर्तन्ते इति वासनाभेदादेषां भेदः ।
	\pend
      
	  \bigskip
	  \begingroup
	  \large
	
	    
	    \stanza[\smallbreak]
	\label{pv.1.165c}\edlabel{pv.1.165c}\flagstanza{\tiny\textenglish{...1.165c}}अनपेक्ष्य यदा कायं वासनाबोधकारणम् ॥ १६५ ॥\&[\smallbreak]


	
	  \endgroup
	
	  \bigskip
	  \begingroup
	  \large
	
	    
	    \stanza[\smallbreak]
	\label{pv.1.166}\edlabel{pv.1.166}\flagstanza{\tiny\textenglish{....1.166}}ज्ञानं स्यात् कस्यचित् किञ्चित् कुतश्चित् तेन किञ्चन ।&अविज्ञानस्य विज्ञानानुपादानाच्च सिध्यति ॥ १६६ ॥\&[\smallbreak]


	
	  \endgroup
	

	  \pstart एवञ्चा \edlabel{pvv.68-2}\footnote{\label{pvv.68-2}  २ स्वमतं समर्थयते ।} {\color{DodgerBlue3}“नपेक्ष्य यदा कायं”} किञ्चिज्ज्ञानं {\color{DodgerBlue3}“वासनाबोधस्य”} प्रबोधस्य {\color{DodgerBlue3}“कारणं”} स्यात् । (१६५) {\color{DodgerBlue3}“कस्य\edlabel{pvv.68-3}\footnote{\label{pvv.68-3}  ३ कार्यस्य ।}चि”}दुत्पद्यमानस्य ज्ञानस्य तदा {\color{DodgerBlue3}“तेन”} प्रबोधकानुरोधेन {\color{DodgerBlue3}“कुतश्चि”}ज्ज्ञानादनन्तरं {\color{DodgerBlue3}“किञ्चन ज्ञानं स्यादिति”} सुस्थमस्य पक्षे । {\color{DodgerBlue3}“अविज्ञा”}\edlabel{pvv.68-4}\footnote{\label{pvv.68-4}  ४ एतच्च तयोर्व्विरोधात् ।} {\color{DodgerBlue3}“नस्य”} विज्ञानशून्यस्य लोष्टादे{\color{DodgerBlue3}“र्विज्ञानानुपादानाच्च\edlabel{pvv.68-5}\footnote{\label{pvv.68-5}  ५ न भूतधर्मत्वप्रतिक्षेपादेरेव ।} सिध्यति”} विज्ञानादेव विज्ञानं ।\edlabel{pvv.68-6}\footnote{\label{pvv.68-6}  ६ अविज्ञानान्न विज्ञानमिति यदुक्तं तदिष्टमेव सांख्यस्येति सिद्धसाधनं ।} (१६६)
	\pend
      \label{div_pvv.1.167_1.168_1.169}\edlabel{div_pvv.1.167_1.168_1.169}
	  
	% new div opening: depth here is 2
	

	  \begin{center}%% label @type='head'
	\textbf{(c. अविज्ञानतो विज्ञानानुत्पादात् ।)}
	\end{center}
	
	  \bigskip
	  \begingroup
	  \large
	
	    
	    \stanza[\smallbreak]
	\label{pv.1.167}\edlabel{pv.1.167}\flagstanza{\tiny\textenglish{....1.167}}विज्ञानशक्तिसम्बन्धादिष्टञ्चेत् सर्ववस्तुनः ।&एतच्छांख्यपशोः कोऽन्यः सलज्जो वक्तुमीहते ॥ १६७ ॥\&[\smallbreak]


	
	  \endgroup
	
	  \bigskip
	  \begingroup
	  \large
	
	    
	    \stanza[\smallbreak]
	\label{pv.1.168a}\edlabel{pv.1.168a}\flagstanza{\tiny\textenglish{...1.168a}}अदृष्टपूर्वमस्तीति तृणाग्रे करिणां शतम् ।\&[\smallbreak]


	
	  \endgroup
	

	  \pstart {\color{DodgerBlue3}“सर्व्वस्य वस्तुनो विज्ञानशक्तिसम्बन्धाद”}विज्ञानाद्विज्ञानानुत्पादन{\color{DodgerBlue3}“मिष्टं\edlabel{pvv.68-7}\footnote{\label{pvv.68-7}  ७ सर्व्वत्र शक्तिव्यक्ती तन्मतेन ।}”} चेत् । ए\edlabel{pvv.68-8}\footnote{\label{pvv.68-8}  ८ सिद्धसाधनत्वं परिहरति ।} तद्विज्ञानस्य शक्तिरूपतयाऽवस्थानं सां ख्य{\color{DodgerBlue3}“पशोः”} {\color{DodgerBlue3}“कोऽन्यः सलज्जो वक्तुमीहते”} । (१६७) यो ब्रूयाद{\color{DodgerBlue3}“दृष्ट”}मपि {\color{DodgerBlue3}“तृणाग्रे करिणां शतमस्तीति”}।
	\pend
      \leavevmode\marginnote{\textenglish{069/s}}

	  \pstart तथाहि यदि शक्तिरभिव्यक्तिरूपाद् {\color{DodgerBlue3}“विज्ञानादन्या तदाऽविज्ञानत्वं सिद्धं”} । अथान्यथा तदा (।)
	\pend
      
	  \bigskip
	  \begingroup
	  \large
	
	    
	    \stanza[\smallbreak]
	\label{pv.1.168b}\edlabel{pv.1.168b}\flagstanza{\tiny\textenglish{...1.168b}}यद् रूपं दृश्यतां यातं तद् रूपं प्राङ् न दृश्यते ॥ १६८ ॥\&[\smallbreak]


	
	  \endgroup
	
	  \bigskip
	  \begingroup
	  \large
	
	    
	    \stanza[\smallbreak]
	\label{pv.1.169a}\edlabel{pv.1.169a}\flagstanza{\tiny\textenglish{...1.169a}}शतधा विप्रकीर्णेऽपि हेतौ तद् विद्यते कथम् ।\&[\smallbreak]


	
	  \endgroup
	

	  \pstart यद्रूपं ख्यानाख्यं {\color{DodgerBlue3}“दृश्यतां यात”}मभिव्यक्तावस्थायां {\color{DodgerBlue3}“तद्रूपं प्राक्”} शक्त्यवस्थायां {\color{DodgerBlue3}“शतधापि विप्रकीर्ण्णे हेतौ न दृश्यते”} (१६८) । {\color{DodgerBlue3}“कथन्तद्विद्यत”} इति नाविज्ञानत्वमसिद्धं (।)
	\pend
      

	  \pstart किञ्च (।)
	\pend
      
	  \bigskip
	  \begingroup
	  \large
	
	    
	    \stanza[\smallbreak]
	\label{pv.1.169b}\edlabel{pv.1.169b}\flagstanza{\tiny\textenglish{...1.169b}}रागाद्यनियमोऽपूर्वप्रादुर्भावे प्रसज्यते ॥ १६९ ॥\&[\smallbreak]


	
	  \endgroup
	

	  \pstart न चेत् परलोकादागच्छति चित्तसन्तानः तदाऽ{\color{DodgerBlue3}“पूर्व्व”}सत्त्व{\color{DodgerBlue3}“प्रादुर्भावे रागाद्य”}नियमश्चक्षुराद्यनियमवत् {\color{DodgerBlue3}“प्रसज्यते”} । यथा चक्षुःकरचरणश्यामतानियमः पुरुषे नेक्ष्यते {\color{DodgerBlue3}“कदाचित् कस्यचिदभावदर्शनात्”} । तथा रागी रागितरो नीरागश्च कश्चित् स्यात् । (१६९)
	\pend
      \label{div_pvv.1.170_1.171}\edlabel{div_pvv.1.170_1.171}
	  
	% new div opening: depth here is 2
	

	  \pstart (d. भूतात्मतया समानरागताप्रसङ्गात् ।
	\pend
      
	  \bigskip
	  \begingroup
	  \large
	
	    
	    \stanza[\smallbreak]
	\label{pv.1.170}\edlabel{pv.1.170}\flagstanza{\tiny\textenglish{....1.170}}भूतात्मताऽनतिक्रान्तः सर्वो रागादिमान् यदि ।&सर्वः समानरागः स्याद् भूतातिशयतो न चेत् ॥ १७० ॥\&[\smallbreak]


	
	  \endgroup
	
	  \bigskip
	  \begingroup
	  \large
	
	    
	    \stanza[\smallbreak]
	\label{pv.1.171}\edlabel{pv.1.171}\flagstanza{\tiny\textenglish{....1.171}}भूतानां प्राणिताऽभेदेऽप्ययं भेदो यदाश्रयः ।&तन्निर्ह्रासातिशयवत् तद्भावात् तानि हापयेत् ॥ १७१ ॥\&[\smallbreak]


	
	  \endgroup
	

	  \pstart {\color{DodgerBlue3}“भूतात्मता”}या {\color{DodgerBlue3}“अनतिक्रान्तेः सर्व्वः”} पुमान् {\color{DodgerBlue3}“रागादि\edlabel{pvv.69-1}\footnote{\label{pvv.69-1}  १ आदिना सुखादयः ।}मान् यदि”} तदा {\color{DodgerBlue3}“सर्व्वंः समानरागः स्यात्”} । भूतात्मताया अविशेषात् । {\color{DodgerBlue3}“भूता”}नामवान्तरा{\color{DodgerBlue3}“दतिशयतः समानरागता”}प्रसङ्गो {\color{DodgerBlue3}“न चेत्”} (१७०) {\color{DodgerBlue3}“भूतानां प्राणि”}ताया अ{\color{DodgerBlue3}“भेदेप्य\edlabel{pvv.69-2}\footnote{\label{pvv.69-2}  २ अयं प्राणी प्राणितरः प्राणितम इत्यभेदे ।}यं भेद”} उत्कटमन्दरागत्वादिको {\color{DodgerBlue3}“यदाश्रयो”} यदवान्तरविशेषवत्\edlabel{pvv.69-3}\footnote{\label{pvv.69-3}  ३ भूतातिशयं ।} कारणमाश्रित्य । तत्कारणं {\color{DodgerBlue3}“निर्ह्रासातिशयवत्”} अपच\edlabel{pvv.69-4}\footnote{\label{pvv.69-4}  ४ यो यत्रोत्कर्षवान् स तत्र सम्भवदुच्छेदधर्म्माद्यनिबन्धनं रागादि निवर्तयेत् तत्र शुक्लत्वादिहेतुवत् ।}यतारतम्यवत् {\color{DodgerBlue3}“तद्भावाद्रागादिमत्त्वात् तानि”} भूतानि {\color{DodgerBlue3}“हापयेत्”} भ्रंशयेदिति नीरागोपि कश्चित् सत्त्वः स्यात् । (१७१)
	\pend
      \label{div_pvv.1.172}\edlabel{div_pvv.1.172}
	  
	% new div opening: depth here is 2
	\leavevmode\marginnote{\textenglish{070/s}}
	  \bigskip
	  \begingroup
	  \large
	
	    
	    \stanza[\smallbreak]
	\label{pv.1.172}\edlabel{pv.1.172}\flagstanza{\tiny\textenglish{....1.172}}न चेद् भेदेऽपि रागादिहेतुतुल्यात्मताक्षयः ।&सर्वत्र रागः सदृशः स्याद्धेतोस्सदृशात्मनः ॥ १७२ ॥\&[\smallbreak]


	
	  \endgroup
	

	  \pstart {\color{DodgerBlue3}“भेदे मन्दो”}त्कटरागादिजनकेऽवान्तरभूतविशेषेपि {\color{DodgerBlue3}“रागादि”}हेतुत्वेन या {\color{DodgerBlue3}“तुल्यात्म\edlabel{pvv.70-1}\footnote{\label{pvv.70-1}  १ रागादिहेतुः सामान्यं, तस्य यदि क्षयः स्यात्तदा विरागः स्यात्, न चास्ति क्षयः (।) यतोऽक्षयस्तेन वीतरागो न चेत् तदा ।}ता”} सदृशता तस्याः {\color{DodgerBlue3}“क्षयो न चेत् सर्व्वत्र”} पुंसि {\color{DodgerBlue3}“सदृशो रागादिः स्यात्”} । {\color{DodgerBlue3}“सदृशात्मनो हेतो”}र्भावात् । (१७२)
	\pend
      \label{div_pvv.1.173}\edlabel{div_pvv.1.173}
	  
	% new div opening: depth here is 2
	
	  \bigskip
	  \begingroup
	  \large
	
	    
	    \stanza[\smallbreak]
	\label{pv.1.173}\edlabel{pv.1.173}\flagstanza{\tiny\textenglish{....1.173}}न हि गोप्रत्ययस्यास्ति समानात्मभुवः क्वचित् ।&तारतम्यं पृथिव्यादौ प्राणितादेरिहापि वा ॥ १७३ ॥\&[\smallbreak]


	
	  \endgroup
	

	  \pstart {\color{DodgerBlue3}“न\edlabel{pvv.70-2}\footnote{\label{pvv.70-2}  २ तुल्यहेतुकं विशिष्यत इति दृष्टान्तमाह} हि गोप्रत्ययस्यास्ति समानात्मभुवः”} सदृशकारणोत्पत्तेः । {\color{DodgerBlue3}“क्वचित्”} शाबलेयादौ {\color{DodgerBlue3}“तार\edlabel{pvv.70-3}\footnote{\label{pvv.70-3}  ३ गोतरो गोतम इति ।} तम्य”}मपि तु सर्व्वत्र समानतैव । {\color{DodgerBlue3}“पृथिव्यादौ”} सदृशे हेतौ {\color{DodgerBlue3}“प्राणितादेरिह”} चा र्व्वा क मते{\color{DodgerBlue3}“पि वा”} विशेषो न विद्यते\edlabel{pvv.70-4}\footnote{\label{pvv.70-4}  ४ यथा न प्राणी प्राणितर इति तथा रागी रागितर इत्यपि न युक्तं तुल्यहेतुसम्भवात् ।}। (१७३)
	\pend
      \label{div_pvv.1.174}\edlabel{div_pvv.1.174}
	  
	% new div opening: depth here is 2
	
	  \bigskip
	  \begingroup
	  \large
	
	    
	    \stanza[\smallbreak]
	\label{pv.1.174}\edlabel{pv.1.174}\flagstanza{\tiny\textenglish{....1.174}}औष्णयस्य तारतम्येऽपि नानुष्णोऽग्निः कदाचन ।&तथेहापीति चेन्नाग्नेरौष्ण्याद् भेदनिषेधतः ॥ १७४ ॥\&[\smallbreak]


	
	  \endgroup
	

	  \pstart {\color{DodgerBlue3}“औष्ण्यस्य तारतम्येपि”} खादिराग्न्यादौ\edlabel{pvv.70-5}\footnote{\label{pvv.70-5}  ५ सांख्यादिराह ।} {\color{DodgerBlue3}“नानुष्णोऽग्निः कदाचन”} । {\color{DodgerBlue3}“तथेह”} रागादि{\color{DodgerBlue3}“तारतम्येपि”} न वीतरागः कश्चि{\color{DodgerBlue3}“दिति”} {\color{DodgerBlue3}“चेत्”} कदाचन नैतद्युक्तं । {\color{DodgerBlue3}“अग्नेरौष्ण्याद् भेदस्य निषेधतः”} । नैतद्युक्तं । भास्वररूपोष्णस्पर्शादिरग्निरुच्यते । तेनौष्ण्याभावेऽग्निरेव न स्यात् । रागादिस्तु भूतेभ्योऽन्यस्तदभावेपि तेषां भावात् । (१७४)
	\pend
      \label{div_pvv.1.175}\edlabel{div_pvv.1.175}
	  
	% new div opening: depth here is 2
	

	  \pstart अतः सविशे\edlabel{pvv.70-6}\footnote{\label{pvv.70-6}  ६ धर्मिणो गुणा इति विशेषणं न चौष्ण्यस्याग्निर्धर्मी ।}षणं हेतुन्दर्शयति ।
	\pend
      
	  \bigskip
	  \begingroup
	  \large
	
	    
	    \stanza[\smallbreak]
	\label{pv.1.175}\edlabel{pv.1.175}\flagstanza{\tiny\textenglish{....1.175}}तारतम्यानुभविनो यस्यान्यस्य सतो गुणाः ।&ते क्वचित् प्रतिहन्यन्ते तद्भेदे धवलादिवत् ॥ १७५ ॥\&[\smallbreak]


	
	  \endgroup
	

	  \pstart {\color{DodgerBlue3}“गुणेभ्योऽन्यस्य”} यस्य {\color{DodgerBlue3}“सतो”} धर्मिणो ये {\color{DodgerBlue3}“तारतम्यानुभाविनो गुणास्ते क्वचिद्”} धर्मिणि {\color{DodgerBlue3}“प्रतिह\edlabel{pvv.70-7}\footnote{\label{pvv.70-7}  ७ उच्छिद्यन्ते ।}न्यन्ते”} । {\color{DodgerBlue3}“तद्भेदे”} भूतभेदे {\color{DodgerBlue3}“धवलादिवत्”} ।\edlabel{pvv.70-8}\footnote{\label{pvv.70-8}  ८ कृष्टस्य जन्तोर्दृ ष्टेः ।}न हि सर्व्वो भूतपरिणामः शुक्लः । (१७५)
	\pend
      \label{div_pvv.1.176}\edlabel{div_pvv.1.176}
	  
	% new div opening: depth here is 2
	\leavevmode\marginnote{\textenglish{071/s}}

	  \begin{center}%% label @type='head'
	\textbf{(f. न रूपवद् रागोऽपि भूतधर्मः)}
	\end{center}
	

	  \pstart स्यादेतद् (।) भूतधर्मो रूपादिर्यथावश्यं अभूत्वा भवति तथा रागोपि देहिनः स्यादित्याह (।)
	\pend
      
	  \bigskip
	  \begingroup
	  \large
	
	    
	    \stanza[\smallbreak]
	\label{pv.1.176}\edlabel{pv.1.176}\flagstanza{\tiny\textenglish{....1.176}}रूपादिवन्न नियमस्तेषां भूताविभागतः ॥&तत् तुल्यञ्चेन्न रागादेः सहोत्पत्तिप्रसङ्गतः ॥ १७६ ॥\&[\smallbreak]


	
	  \endgroup
	

	  \pstart {\color{DodgerBlue3}“रूपा\edlabel{pvv.71-1}\footnote{\label{pvv.71-1}  १ सामान्यमत्र विशेषे व्यभिचारात् ।}\edlabel{pvv.71-1a}\footnote{\label{pvv.71-1a}  1a सहवृत्तिनियमेनेत्यर्थः ।\begin{english}\par
Placement of note uncertain; marked with a question mark in the edition (see encoding description for details).\end{english}} दिवन्न नियमो”} रागादे{\color{DodgerBlue3}“स्तेषां”} रूपादीनां {\color{DodgerBlue3}“भूताविभागतः”} । न हि रूपादिसामान्यं विना भूता\edlabel{pvv.71-2}\footnote{\label{pvv.71-2}  २ सामान्यं कर्तृ ।} नि वर्तते । रागादेरपि तद्भूताविनिर्भागवर्त्तित्वं {\color{DodgerBlue3}“तुल्यञ्चेत्”} । नैतदस्ति । {\color{DodgerBlue3}“रागा\edlabel{pvv.71-3}\footnote{\label{pvv.71-3}  ३ द्वेषादेर्युगपत् रागादिहेतुः ।}”} देर्भूतैः {\color{DodgerBlue3}“सहोत्पत्तिप्रसङ्गतो”} ॥ विषयैः कादाचित्कसन्निधानैर्नियमितत्वात् । (१७६)
	\pend
      \label{div_pvv.1.177}\edlabel{div_pvv.1.177}
	  
	% new div opening: depth here is 2
	

	  \pstart न सहोत्पादप्रसङ्ग इति चेदाह (।)
	\pend
      
	  \bigskip
	  \begingroup
	  \large
	
	    
	    \stanza[\smallbreak]
	\label{pv.1.177}\edlabel{pv.1.177}\flagstanza{\tiny\textenglish{....1.177}}विकल्प्यविषयत्वाच्च विषया न नियामकाः ।&सभागहेतुविरहाद् रागादेर्नियमो न वा ॥ १७७ ॥\&[\smallbreak]


	
	  \endgroup
	

	  \pstart विकल्प्यः कल्पितस्त{\color{DodgerBlue3}“द्विषयत्वाच्च”} रागादे{\color{DodgerBlue3}“र्व्विषया”} रूपादयोऽग्राह्यात्वान्न {\color{DodgerBlue3}“नियामकाः”} । किञ्चाविज्ञानस्य विज्ञानहेतुत्वात् नेत्यु\edlabel{pvv.71-4}\footnote{\label{pvv.71-4}  ४ अविज्ञानं न विज्ञानहेतुरित्युक्तं परं प्रति । इदानीमपि परं प्रत्याह ।}क्तं । त्वन्मते {\color{DodgerBlue3}“सभागस्य”} च\edlabel{pvv.71-5}\footnote{\label{pvv.71-5}  ५ वा समुच्चयार्थः ।} {\color{DodgerBlue3}“हेतोर्विरहा\edlabel{pvv.71-6}\footnote{\label{pvv.71-6}  ६ यद्यपि तृणादयो हस्त्यात्मना परिणतास्तत्रापि तेषां न तच्छक्तिरस्ति ।}त् रागादेर्नियमो”} देशकालस्वभावविषयो {\color{DodgerBlue3}“न वा”} स्यादहेतुत्वात् । (१७७)
	\pend
      \label{div_pvv.1.178_1.179}\edlabel{div_pvv.1.178_1.179}
	  
	% new div opening: depth here is 2
	

	  \begin{center}%% label @type='head'
	\textbf{(f. न भूतान्येद हेतुः)}
	\end{center}
	

	  \pstart भूतान्येव हि हेतुरिति चेत् ।
	\pend
      
	  \bigskip
	  \begingroup
	  \large
	
	    
	    \stanza[\smallbreak]
	\label{pv.1.178a}\edlabel{pv.1.178a}\flagstanza{\tiny\textenglish{...1.178a}}सर्वदा सर्वबुद्धीनां जन्म वा हेतुसन्निधेः ।\leavevmode\marginnote{\textenglish{14a/MA}}\&[\smallbreak]


	
	  \endgroup
	

	  \pstart तर्हि {\color{DodgerBlue3}“सर्व्वदा सर्व्वबुद्धीनां”} सुखदुःखेच्छाद्वेषरागकृपादीनां {\color{DodgerBlue3}“जन्म वा स्यात् । हेतो”}र्भूतसंघातस्यं {\color{DodgerBlue3}“सन्निधेः”} । तदेवं रा\edlabel{pvv.71-7}\footnote{\label{pvv.71-7}  ७ दुःखसत्यव्याख्यारम्भे ।}गादेः पाटवेक्षणादित्यादिना चतुर्ण्णामरूपिणां स्कन्धानां सभागहेतुकत्वे साधिते साधितं संसारित्वमुपादानस्कन्धानां । त एव दुःखमित्युक्ताः ।
	\pend
      \leavevmode\marginnote{\textenglish{072/s}}

	  \begin{center}%% label @type='head'
	\textbf{IV. चतुराकरं दुःखसत्यम्}
	\end{center}
	

	  \pstart दुःखसत्यञ्चानित्यतो दुःखतः शून्यतोऽनात्मतश्चेति चतुराकारमाख्यातुमाह (।)
	\pend
      
	  \bigskip
	  \begingroup
	  \large
	
	    
	    \stanza[\smallbreak]
	\label{pv.1.178b}\edlabel{pv.1.178b}\flagstanza{\tiny\textenglish{...1.178b}}कदाचिदुपलम्भात् तदध्रुवं दोषनिश्रयात् ॥ १७८ ॥\&[\smallbreak]


	
	  \endgroup
	
	  \bigskip
	  \begingroup
	  \large
	
	    
	    \stanza[\smallbreak]
	\label{pv.1.179a}\edlabel{pv.1.179a}\flagstanza{\tiny\textenglish{...1.179a}}दुःखं हेतुवशत्वाच्च न चात्मा नाप्यधिष्ठितम् ।\&[\smallbreak]


	
	  \endgroup
	

	  \pstart {\color{DodgerBlue3}“कदाचिदुपलम्भाद्”} दुःखम{\color{DodgerBlue3}“ध्रुवम”}नित्यं । {\color{DodgerBlue3}“दोषनिश्रया\edlabel{pvv.72-1}\footnote{\label{pvv.72-1}  १ सास्रवत्वात् उत्पत्त्या क्लिश्यन्ति विपाकेन च ।}त्”} रागादिदोषाश्रयेणोत्पत्तेः । (१७८) {\color{DodgerBlue3}“हेतुवशत्वाच्च”} । सर्व्वं परवशं {\color{DodgerBlue3}“दुःख”}मिति न्यायात् दुःखं तत् । {\color{DodgerBlue3}“न चात्मा”}श्रयं । अना\edlabel{pvv.72-2}\footnote{\label{pvv.72-2}  २ हेतुवशत्वाच्च ।}त्मनः आत्मविलक्षणत्वात् । {\color{DodgerBlue3}“ना\edlabel{pvv.72-3}\footnote{\label{pvv.72-3}  ३ नात्मीयं}प्यधिष्ठितं”} । अधिष्ठातुरात्म\edlabel{pvv.72-4}\footnote{\label{pvv.72-4}  ४ सभागतो जातत्वात् ।}नोऽभावात् । अनेन शून्यत इत्याख्यातं ।
	\pend
      

	  \pstart कस्मात्पुनरात्मा नाधिष्ठातेत्याह---
	\pend
      
	  \bigskip
	  \begingroup
	  \large
	
	    
	    \stanza[\smallbreak]
	\label{pv.1.179b}\edlabel{pv.1.179b}\flagstanza{\tiny\textenglish{...1.179b}}नाकारणमधिष्ठाता नित्यं वा कारणं कथम् ॥ १७९॥\&[\smallbreak]


	
	  \endgroup
	

	  \pstart {\color{DodgerBlue3}“नाकारणमधिष्ठाता”}ऽतिप्रसङ्गात् । {\color{DodgerBlue3}“नित्यं वा”} द्रव्यं {\color{DodgerBlue3}“कारणं कथं”} (।) तस्य क्रमयौगपद्याभ्यामर्थक्रियाविरहात् । (१७९)
	\pend
      \label{div_pvv.1.180_1.181_1.182_1.183_1.184}\edlabel{div_pvv.1.180_1.181_1.182_1.183_1.184}
	  
	% new div opening: depth here is 2
	
	  \bigskip
	  \begingroup
	  \large
	
	    
	    \stanza[\smallbreak]
	\label{pv.1.180a}\edlabel{pv.1.180a}\flagstanza{\tiny\textenglish{...1.180a}}तस्मादनेकमेकस्माद् भिन्नकालं न जायते ।\&[\smallbreak]


	
	  \endgroup
	

	  \pstart {\color{DodgerBlue3}“तस्मादनेकं भिन्नकाल”}दृश्यमानं सुखदुःखादिकार्यं {\color{DodgerBlue3}“नैकस्माज्जायते”} एकस्यानेककरणे समर्थस्य सकृदेव तत्क्रियाप्रसङ्गात् ।
	\pend
      
	  \bigskip
	  \begingroup
	  \large
	
	    
	    \stanza[\smallbreak]
	\label{pv.1.180b}\edlabel{pv.1.180b}\flagstanza{\tiny\textenglish{...1.180b}}कार्यानुत्पादतोऽन्येषु सङ्गतेष्वपि हेतुषु ॥ १८० ॥\&[\smallbreak]


	
	  \endgroup
	
	  \bigskip
	  \begingroup
	  \large
	
	    
	    \stanza[\smallbreak]
	\label{pv.1.181a}\edlabel{pv.1.181a}\flagstanza{\tiny\textenglish{...1.181a}}हेत्वन्तरानुमानं स्यान्नैतन् नित्येषु विद्यते ।\&[\smallbreak]


	
	  \endgroup
	

	  \pstart किञ्चानेकेषु {\color{DodgerBlue3}“हेतुषु”} सङ्ग{\color{DodgerBlue3}“तेषु”} मिथो मिलितेष्वपि {\color{DodgerBlue3}“कार्यानुत्पाद”}तो (१८०) {\color{DodgerBlue3}“हेत्वन्तरानुमानं स्यात्”} (।) यथा रूपालोकमनस्कारेषु सत्स्वपि चक्षुर्व्विज्ञानमनुत्पद्यमानं चक्षुरनुमापयति । {\color{DodgerBlue3}“नैतत्का”}र्यानुत्पत्त्याऽनुमानं {\color{DodgerBlue3}“नित्येषु विद्यते”} तेषामव्यति\edlabel{pvv.72-5}\footnote{\label{pvv.72-5}  ५ तद्व्यतिरेकेपि कार्योत्पत्तेः ।}रेकित्वात् (।)
	\pend
      

	  \begin{center}%% label @type='head'
	\textbf{(ख) समुदयसत्यम्}
	\end{center}
	

	  \begin{center}%% label @type='head'
	\textbf{I. चतुराकारः समुदयः}
	\end{center}
	

	  \pstart चतुराकारं दुःखं व्याख्याय समुदयतो हेतुतः प्रत्ययतः प्रभवतश्चेति चतुराकारं {\color{DodgerBlue3}“समुदायं व्याख्यातुमाह”} (।)
	\pend
      \leavevmode\marginnote{\textenglish{073/s}}
	  \bigskip
	  \begingroup
	  \large
	
	    
	    \stanza[\smallbreak]
	\label{pv.1.181b}\edlabel{pv.1.181b}\flagstanza{\tiny\textenglish{...1.181b}}कादाचित्कतया सिद्धा दुःखस्यास्य सहेतुता ॥ १८१ ॥\&[\smallbreak]


	
	  \endgroup
	
	  \bigskip
	  \begingroup
	  \large
	
	    
	    \stanza[\smallbreak]
	\label{pv.1.182a}\edlabel{pv.1.182a}\flagstanza{\tiny\textenglish{...1.182a}}नित्यं सत्वमसत्वं वाऽहेतोर्बाह्यानपेक्षणात् ॥\&[\smallbreak]


	
	  \endgroup
	

	  \pstart a. {\color{DodgerBlue3}“कादाचित्कतया दुःखस्य सहेतुता सिद्धा”} (१८१)। {\color{DodgerBlue3}“नित्यं सत्त्वमसत्त्वंम्वा हेतो”}र्भवति यथाऽऽकाशस्य शशविषाणस्य बाह्या{\color{DodgerBlue3}“नपेक्षणात्”} । य एव च दुःखहेतुः स एव समुदायः ।
	\pend
      
	  \bigskip
	  \begingroup
	  \large
	
	    
	    \stanza[\smallbreak]
	\label{pv.1.182b}\edlabel{pv.1.182b}\flagstanza{\tiny\textenglish{...1.182b}}तैक्ष्ण्यादीनां यथा नास्ति कारणं कण्टकादिषु ॥ १८२ ॥\&[\smallbreak]


	
	  \endgroup
	
	  \bigskip
	  \begingroup
	  \large
	
	    
	    \stanza[\smallbreak]
	\label{pv.1.183a}\edlabel{pv.1.183a}\flagstanza{\tiny\textenglish{...1.183a}}तथाऽकारणमेतत् स्यादिति केचित् प्रचक्षते ।\&[\smallbreak]


	
	  \endgroup
	

	  \pstart न\edlabel{pvv.73-1}\footnote{\label{pvv.73-1}  १ कः पद्मनालदलकेशरकर्ण्णिकानां संस्थानवर्ण्णरचनामृदुतादिहेतुः (।) पत्राणि चित्रयति कोत्र पतत्रिणाम्वा स्वाभाविकं जगदिदं नियतं तथैव ॥ दुःखमहेतुकं चार्व्वाकः ।}नु {\color{DodgerBlue3}“यथा कण्टकादिषु तैक्ष्ण्यादीनां कारणं नास्ति”} (१८२) {\color{DodgerBlue3}“तथाऽकारणमेतत्”} दुःखं स्यात् । तत् कुतः समुदय {\color{DodgerBlue3}“इति केचित्”} स्वभाववादिनः {\color{DodgerBlue3}“प्रचक्षते”} (।)
	\pend
      

	  \pstart b. ते एवं वक्तव्या (:।)
	\pend
      
	  \bigskip
	  \begingroup
	  \large
	
	    
	    \stanza[\smallbreak]
	\label{pv.1.183b}\edlabel{pv.1.183b}\flagstanza{\tiny\textenglish{...1.183b}}सत्येव यस्मिन् यज्जन्म विकारे वाऽपि विक्रिया ॥ १८३ ॥\&[\smallbreak]


	
	  \endgroup
	
	  \bigskip
	  \begingroup
	  \large
	
	    
	    \stanza[\smallbreak]
	\label{pv.1.184a}\edlabel{pv.1.184a}\flagstanza{\tiny\textenglish{...1.184a}}तत् तस्य कारणं प्राहुस्तत् तेषामपि विद्यते ।\&[\smallbreak]


	
	  \endgroup
	

	  \pstart {\color{DodgerBlue3}“सत्येव यस्मिन्”} वस्तुनि {\color{DodgerBlue3}“यस्य जन्म”} । यस्य {\color{DodgerBlue3}“विकारे”} सत्येव वा यस्य {\color{DodgerBlue3}“विक्रिया”} (१८३) {\color{DodgerBlue3}“तत्तस्य”} जन्मिनो विकारिणश्च {\color{DodgerBlue3}“कारणं प्राहु”}र्व्विद्वांसः । तज्जन्म सत्येव बीजोदकपृथिव्यादिषु तदुत्कर्षापकर्षादिविकारे च विकृतत्वं {\color{DodgerBlue3}“तेषां”} कण्टकादीनामप्यस्तीति तेपि सहेतुका एव । एवं स्कन्धा अपि ॥
	\pend
      

	  \pstart न१नु स्पर्शे सति भवति चक्षुर्व्विज्ञानं । असति च न भवति\edlabel{pvv.73-2}\footnote{\label{pvv.73-2}  २ स्पर्शवत्येवं संहते चक्षुर्व्विज्ञानं मन्यते ।} (।) न च तत् कारणमतोऽतिव्याप्तिरित्याह (।)
	\pend
      
	  \bigskip
	  \begingroup
	  \large
	
	    
	    \stanza[\smallbreak]
	\label{pv.1.184b}\edlabel{pv.1.184b}\flagstanza{\tiny\textenglish{...1.184b}}स्पर्शस्य रूपहेतुत्वाद् दर्शनेऽस्ति निमित्तता ॥ १८४ ॥\&[\smallbreak]


	
	  \endgroup
	

	  \pstart {\color{DodgerBlue3}“स्पर्शस्य”} रूपाद्यविनिर्भा\edlabel{pvv.73-3}\footnote{\label{pvv.73-3}  ३ स्पर्शीभूतचतुष्कात्मा उपादाय रूपस्य हेतुः ।}गिनः सहकारिभावेन {\color{DodgerBlue3}“रूपहेतुत्वात् । दर्शने”} चक्षुर्व्विज्ञानेऽ{\color{DodgerBlue3}“स्ति निमित्तता”} पारम्पर्येणेति नातिव्याप्तिः । (१८४)
	\pend
      \label{div_pvv.1.185_1.186}\edlabel{div_pvv.1.185_1.186}
	  
	% new div opening: depth here is 2
	

	  \pstart एतच्च व्यतिरे\edlabel{pvv.73-4}\footnote{\label{pvv.73-4}  ४ अतिव्याप्ति ।}कमभ्युपगम्योक्तं न तु रूपमुपदर्श्य स्पर्शाभावे नेत्रबुद्धेरभावः शक्यदर्शनः । रूपस्पर्शयोरविनिर्भागवर्त्तित्वात् । तस्य च दुःखस्य (।)
	\pend
      \leavevmode\marginnote{\textenglish{074/s}}
	  \bigskip
	  \begingroup
	  \large
	
	    
	    \stanza[\smallbreak]
	\label{pv.1.185a}\edlabel{pv.1.185a}\flagstanza{\tiny\textenglish{...1.185a}}नित्यानां प्रतिषेधेन नेश्वरादेश्च सम्भवः ।&असामर्थ्यादतो हेतुर्भववाञ्छा;\&[\smallbreak]


	
	  \endgroup
	

	  \pstart {\color{DodgerBlue3}“नित्यानां”} क्रमाक्रमाभ्यामर्थक्रियायामसामर्थ्यात् । {\color{DodgerBlue3}“प्रतिषेधेन”}\edlabel{pvv.74-1}\footnote{\label{pvv.74-1}  १ “कारणम्विकृतिङ्गच्छज्जायतेन्यस्य कारणमि”त्यादिना ।} च {\color{DodgerBlue3}“नेश्वरादे”}रादिग्रहणात् प्रधानपुरुषादेः कारणात्सम्भव उत्पादः । अतो नित्यादनुत्पत्तेर्दुःखस्य हेतु{\color{DodgerBlue3}“र्भववाञ्छा”} जन्मतृष्णा, जन्मस्थानावस्था\edlabel{pvv.74-2}\footnote{\label{pvv.74-2}  २ अवस्था मनुष्यदेवादि ।} सत्त्वाद्यभिला\edlabel{pvv.74-3}\footnote{\label{pvv.74-3}  ३ सहायाः सत्त्वाः उपकरणञ्चन्दनादि । गर्भादि ।}षात्मिका । (१८५)
	\pend
      
	  \bigskip
	  \begingroup
	  \large
	
	    
	    \stanza[\smallbreak]
	\label{pv.1.185b}\edlabel{pv.1.185b}\flagstanza{\tiny\textenglish{...1.185b}}परिग्रहः ॥ १८५ ॥\&[\smallbreak]


	
	  \endgroup
	
	  \bigskip
	  \begingroup
	  \large
	
	    
	    \stanza[\smallbreak]
	\label{pv.1.186a}\edlabel{pv.1.186a}\flagstanza{\tiny\textenglish{...1.186a}}यस्माद् देशविशेषस्य तत्प्राप्त्याशाकृतो तृणाम् ।\&[\smallbreak]


	
	  \endgroup
	

	  \pstart {\color{DodgerBlue3}“यस्माद् देशविशेषस्य”} परिग्रहः {\color{DodgerBlue3}“तत्प्राप्ति”}तृष्णा{\color{DodgerBlue3}“कृतो नृणां”} । नृशब्दः प्राण्युपलक्षणः । ततो गर्भस्थानादानमपि तत्तृष्णाकृतमेव । (१८६)
	\pend
      \label{div_pvv.1.187_1.188_1.189}\edlabel{div_pvv.1.187_1.188_1.189}
	  
	% new div opening: depth here is 2
	

	  \pstart c. ननूक्तं भ ग व ता “तत्र कतमस्समुदय आर्यसत्यं पौनर्भाविकी न\edlabel{pvv.74-4}\footnote{\label{pvv.74-4}  ४ द्वयो रागयोः समवधानाभावादाह । वर्तमानार्थालम्बनाकृष्टा प्रीतिर्नन्दी ।}न्दी रागसहगता तत्र तत्राभिनान्दिनी यदुत कामतृष्णा भवतृष्णा विभवतृष्णा चे”\edlabel{pvv.74-asterisk}\footnote{\label{pvv.74-asterisk}  * दीघनिकाय २।२२} ति तत्कथमेका भवतृष्णोच्यते समुदयसत्यमिति । अत्राह (।)
	\pend
      
	  \bigskip
	  \begingroup
	  \large
	
	    
	    \stanza[\smallbreak]
	\label{pv.1.186b}\edlabel{pv.1.186b}\flagstanza{\tiny\textenglish{...1.186b}}सा भवेच्छाऽप्त्यनाप्तीच्छोः प्रवृत्तिः सुखदुःखयोः ॥ १८६ ॥\&[\smallbreak]


	
	  \endgroup
	
	  \bigskip
	  \begingroup
	  \large
	
	    
	    \stanza[\smallbreak]
	\label{pv.1.187a}\edlabel{pv.1.187a}\flagstanza{\tiny\textenglish{...1.187a}}यतोऽपि प्राणिनः कामविभवेच्छे च ते मते ।\&[\smallbreak]


	
	  \endgroup
	

	  \pstart {\color{DodgerBlue3}“यतः”} कारणात् प्राणिनः {\color{DodgerBlue3}“सुखदुःखयोः”} क्रमेणा{\color{DodgerBlue3}“प्त्यनाप्तीच्छोः”} प्रवृत्तिर्गर्भस्थानपरिग्रहायातः {\color{DodgerBlue3}“सा भवेच्छापि”} (१८६) {\color{DodgerBlue3}“कामविभवे”}च्छे {\color{DodgerBlue3}“च ते मते”} । सुखप्राप्तीच्छा कामतृष्णा । दुःखवियोगेच्छा विभवतृष्णा । भवतृष्णायां सुखदुःखप्राप्तिपरिहारेच्छापूर्व्विकायां गर्भस्थानोपादानेच्छात्मिकायां द्वयोरपि संग्रहादविरोधः । (१८७)
	\pend
      
	  \bigskip
	  \begingroup
	  \large
	
	    
	    \stanza[\smallbreak]
	\label{pv.1.187b}\edlabel{pv.1.187b}\flagstanza{\tiny\textenglish{...1.187b}}सर्वत्र चात्मस्नेहस्य हेतुत्वात् संप्रवर्तते ॥ १८७ ॥\&[\smallbreak]


	
	  \endgroup
	
	  \bigskip
	  \begingroup
	  \large
	
	    
	    \stanza[\smallbreak]
	\label{pv.1.188a}\edlabel{pv.1.188a}\flagstanza{\tiny\textenglish{...1.188a}}असुखे सुखसंज्ञस्य;\&[\smallbreak]


	
	  \endgroup
	

	  \pstart समुदयस्य च समुदयात्मकत्वात् {\color{DodgerBlue3}“सर्व्वत्र”} विषयेऽ{\color{DodgerBlue3}“सुखे”} सुखरहिते {\color{DodgerBlue3}“सुखसंज्ञ”}स्य दुःखविपर्ययस्तस्यानेनाशुचौ शुचिविपर्यासोपि कथितः । न ह्यशुचौ तथा मन्यमान कस्यचित् सुखसंज्ञाऽऽत्मस्नेहेनेति {\color{DodgerBlue3}“आत्मस्नेहस्य हेतुत्वादिति”} । अहङ्कारममकारो\leavevmode\marginnote{\textenglish{075/s}} त्थापितस्य ।\edlabel{pvv.75-1}\footnote{\label{pvv.75-1}  १ अहंकारममकारो यस्य ।} अनेनात्मनि आत्मस्नेहविपर्यास उक्तः । संप्रवर्तत इत्यनेनानित्यविपर्यासः सूचितः । न हि नित्यविपर्यासम्विना फलार्थी प्रवर्तते ।
	\pend
      \leavevmode\marginnote{\textenglish{14b/MA}}
	  \bigskip
	  \begingroup
	  \large
	
	    
	    \stanza[\smallbreak]
	\label{pv.1.188b}\edlabel{pv.1.188b}\flagstanza{\tiny\textenglish{...1.188b}}तस्मात् तृष्णा भवाश्रयः ॥ ॥\&[\smallbreak]


	
	  \endgroup
	

	  \pstart तदेवं चतुर्व्विपर्यासवासितमानस एवा{\color{DodgerBlue3}“त्मस्नेहात्”} सुखदुःखप्राप्तिपरिजिहीर्षया {\color{DodgerBlue3}“प्रवर्तते”} । तथा गर्भस्थानेपि सुखदुःखप्राप्तिपरिहारेच्छैव तृष्णा । {\color{DodgerBlue3}“तस्मात्तृष्णा भवस्याश्रयो”} हेतुः । अनेन हेतुत आख्यातं ।
	\pend
      
	  \bigskip
	  \begingroup
	  \large
	
	    
	    \stanza[\smallbreak]
	\label{pv.1.188c}\edlabel{pv.1.188c}\flagstanza{\tiny\textenglish{...1.188c}}विरक्तजन्मादृष्टेरित्याचार्याः संप्रचक्षते ॥ १८८ ॥\&[\smallbreak]


	
	  \endgroup
	
	  \bigskip
	  \begingroup
	  \large
	
	    
	    \stanza[\smallbreak]
	\label{pv.1.189a}\edlabel{pv.1.189a}\flagstanza{\tiny\textenglish{...1.189a}}अदेहरागादृष्टेश्च देहाद् रागसमुद्भवः ।\&[\smallbreak]


	
	  \endgroup
	

	  \pstart ननु “विरक्तस्य जन्मादृष्टेरि” \href{http://http://sarit.indology.info/?cref=ns.3.1.25}{(न्यायसू॰ ३।१।२५)} {\color{DodgerBlue3}“त्याचार्या”} गौ त मा द\edlabel{pvv.75-2}\footnote{\label{pvv.75-2}  २ वसुवन्धुनैवमुक्ते चार्व्वाक आह ।}योपि {\color{DodgerBlue3}“संप्रचक्षते”} (। १८८) {\color{DodgerBlue3}“ऽदेहस्य रागादृष्टेश्च देहाद्रागसमुद्भवः”} स्थितस्तंतो न देही वीतरागः ।
	\pend
      

	  \pstart अत्राह (।)
	\pend
      
	  \bigskip
	  \begingroup
	  \large
	
	    
	    \stanza[\smallbreak]
	\label{pv.1.189b}\edlabel{pv.1.189b}\flagstanza{\tiny\textenglish{...1.189b}}निमित्तोपगमादिष्टमुपादानं तु वार्यते ॥ १८९ ॥\&[\smallbreak]


	
	  \endgroup
	

	  \pstart निमित्तस्य सहकारिकारणस्यो{\color{DodgerBlue3}“पगमात्”} देहोस्य रागस्य सहाकारिकारण{\color{DodgerBlue3}“मिष्टं”} ततो नानिष्टमापद्यते । {\color{DodgerBlue3}“उपादानन्तु”} देहो रागस्य {\color{DodgerBlue3}“वार्यते”} । राग एव तू\edlabel{pvv.75-3}\footnote{\label{pvv.75-3}  ३ देहस्य}पादानकारणं । न च रागो जन्महेतुः । विरक्तस्य करुणया जन्मसम्भवात् । र\edlabel{pvv.75-4}\footnote{\label{pvv.75-4}  ४ न रागमात्रात्}क्तस्यापि तृष्णयैव जन्मग्रहः । अनेन प्रत्ययत इति व्याख्या\edlabel{pvv.75-5}\footnote{\label{pvv.75-5}  ५ प्रभवाकारमाह ।}तं । (१८९)
	\pend
      \label{div_pvv.1.190}\edlabel{div_pvv.1.190}
	  
	% new div opening: depth here is 2
	
	  \bigskip
	  \begingroup
	  \large
	
	    
	    \stanza[\smallbreak]
	\label{pv.1.190}\edlabel{pv.1.190}\flagstanza{\tiny\textenglish{....1.190}}इमां तु युक्तिमन्विच्छन् बाधते स्वमतं स्वयम् ।&जन्मना सहभावश्चेत् जातानां रागदर्शनात् ॥ १९० ॥\&[\smallbreak]


	
	  \endgroup
	

	  \pstart d. विरक्तजन्मादृष्टेरिती{\color{DodgerBlue3}“मां युक्तिमन्विच्छन्”} चा र्व्वा कः {\color{DodgerBlue3}“स्वमतं स्वयं बाधते”} । रागहेतुको देहः तद्धेतुकश्च राग इति । अन्योन्यहेतुत्वात् जन्मप्रबन्धसिद्धेः । वीतरागाभ्युपगमाच्च स्वमतबाधास्य । {\color{DodgerBlue3}“जन्मना सहभावो”} रागादीनां । न पूर्व्वं रागोस्ति जातानां {\color{DodgerBlue3}“रागदर्शनात्”} । अतो न रागो देहहेतुरिति चेत् । (१९०)
	\pend
      \label{div_pvv.1.191_1.192_1.193_1.194}\edlabel{div_pvv.1.191_1.192_1.193_1.194}
	  
	% new div opening: depth here is 2
	
	  \bigskip
	  \begingroup
	  \large
	
	    
	    \stanza[\smallbreak]
	\label{pv.1.191a}\edlabel{pv.1.191a}\flagstanza{\tiny\textenglish{...1.191a}}सभागजातेः प्राक् सिद्धिः;\&[\smallbreak]


	
	  \endgroup
	\leavevmode\marginnote{\textenglish{076/s}}

	  \begin{center}%% label @type='head'
	\textbf{(II. तृष्णा जन्मसमुदयः)}
	\end{center}
	

	  \pstart नन्वेवं देहोपि न स्यात् रागहेतुः सहभावात् । न चाहेतुकता । ततः सभागात् सजातीयाद्राग{\color{DodgerBlue3}“ज्जाते”}रुत्पादात् । {\color{DodgerBlue3}“प्राग्रा”}गस्य {\color{DodgerBlue3}“सिद्धि”}रित्यायातं ।
	\pend
      

	  \begin{center}%% label @type='head'
	\textbf{(III. कर्माऽपि)}
	\end{center}
	

	  \pstart नन्वविद्या तृष्णा कर्म च जन्मकारणं तत्कथं तृष्णैव केवला समुदय उक्त इत्याह (।)
	\pend
      
	  \bigskip
	  \begingroup
	  \large
	
	    
	    \stanza[\smallbreak]
	\label{pv.1.191b}\edlabel{pv.1.191b}\flagstanza{\tiny\textenglish{...1.191b}}कारणत्वेऽपि नोदितम् ।&अज्ञानं; उक्ता तृष्णैव सन्तानप्रेरणाद् भवे ॥ १९१ ॥\&[\smallbreak]


	
	  \endgroup
	
	  \bigskip
	  \begingroup
	  \large
	
	    
	    \stanza[\smallbreak]
	\label{pv.1.192a}\edlabel{pv.1.192a}\flagstanza{\tiny\textenglish{...1.192a}}आनन्तर्याच्च कर्मापि सति तस्मिन्नसंभवात् ।\&[\smallbreak]


	
	  \endgroup
	

	  \pstart {\color{DodgerBlue3}“कारणत्वेपि नोदितमज्ञान”}मविद्या मोहापरसंज्ञकं । {\color{DodgerBlue3}“उक्ता तृष्णैव”} तयैव {\color{DodgerBlue3}“सन्तान”}स्य पञ्चस्कन्धसन्ततेः {\color{DodgerBlue3}“प्रेरणात्\edlabel{pvv.76-1}\footnote{\label{pvv.76-1}  १ इयमेव प्रभविष्णुत्वात् प्रभवः ।}”} किमर्थं {\color{DodgerBlue3}“भवे”}(१९१) जन्मनिमित्तं {\color{DodgerBlue3}“कर्मापि”} हेतुत्वेऽपि नोक्तं समुदयत्वेन {\color{DodgerBlue3}“सति तस्मिन्न”}ज्ञाने कर्मणि च । तृष्णाऽसंमुखीभावेऽ{\color{DodgerBlue3}“भावा”}ज्जन्मनः । आनन्तर्याच्च तृष्णायाः सतोरपि मोहकर्मणोस्तृष्णाया असंमुखीभावे जन्मन आक्षेपकत्वासंभवात्\edlabel{pvv.76-2}\footnote{\label{pvv.76-2}  २ आनन्तर्य्याच्च तृष्णायाः ।} यथा विमुक्ति (:) चित्तस्येत्युक्तः समुदयः ।
	\pend
      
	  \bigskip
	  \begingroup
	  \large
	
	    
	    \stanza[\smallbreak]
	\label{pv.1.192b}\edlabel{pv.1.192b}\flagstanza{\tiny\textenglish{...1.192b}}तदनान्यन्तिकं हेतोः प्रतिबन्धादिसम्भवात् ॥ १९२ ॥\&[\smallbreak]


	
	  \endgroup
	

	  \pstart तदेतद्यथोक्तकारण (समुदय) स्वभावं दुःखमनात्यन्तिकं संभवदुच्छेदं दुःखहेतोस्तृष्णायाः प्रतिबन्धस्य संभाव्यमानत्वात् । आदिशब्दादविद्यादेः सहकारिणो वैकल्य{\color{DodgerBlue3}“सम्भवात्”} ।
	\pend
      

	  \pstart अनेन निरोध एव नास्तीति वादि\edlabel{pvv.76-3}\footnote{\label{pvv.76-3}  ३ मीमांसकादयः ।}नः प्रतिनिरोधत इति कथितं ।
	\pend
      

	  \begin{center}%% label @type='head'
	\textbf{(ग) निरोधसत्यम्}
	\end{center}
	
	  \bigskip
	  \begingroup
	  \large
	
	    
	    \stanza[\smallbreak]
	\label{pv.1.193a}\edlabel{pv.1.193a}\flagstanza{\tiny\textenglish{...1.193a}}संसारित्वादनिर्मोक्षो नेष्टत्वादप्रसिद्धितः ।\&[\smallbreak]


	
	  \endgroup
	

	  \pstart ननु {\color{DodgerBlue3}“संसारित्वादनिर्म्मोक्षो”} मुक्तिर्नास्ति कस्यचित् । तत्कथं निरोधसम्भावना । {\color{DodgerBlue3}“नैष दोष इष्टत्वात्”} । को नाम मुक्तिं संसारिण इच्छति । {\color{DodgerBlue3}“अप्रसि”}द्धितस्तस्य । न हि संसारीं कश्चिदस्ति । किन्तु दुःखं केवलं हेतुबलात्प्रवर्तते तदभावाच्च न भवतीति ब्रूमः ।
	\pend
      \leavevmode\marginnote{\textenglish{077/s}}

	  \begin{center}%% label @type='head'
	\textbf{(I. संसार्यभावे मुक्तिव्यवस्था)}
	\end{center}
	

	  \pstart यदि संसारी कश्चिन्नास्ति को मुक्त्यर्थी किमर्थं प्रवर्तते इत्याह (।)
	\pend
      
	  \bigskip
	  \begingroup
	  \large
	
	    
	    \stanza[\smallbreak]
	\label{pv.1.193b}\edlabel{pv.1.193b}\flagstanza{\tiny\textenglish{...1.193b}}यावच्चात्मनि न प्रेम्णो हानिः स परितस्यति ॥ १९३ ॥\&[\smallbreak]


	
	  \endgroup
	
	  \bigskip
	  \begingroup
	  \large
	
	    
	    \stanza[\smallbreak]
	\label{pv.1.194a}\edlabel{pv.1.194a}\flagstanza{\tiny\textenglish{...1.194a}}तावद् दुःखितमारोप्य न च स्वस्थोऽवतिष्ठते ॥\&[\smallbreak]


	
	  \endgroup
	

	  \pstart {\color{DodgerBlue3}“यावच्चात्मनि”} एकत्वाहंकारविषयेषु स्कन्धेषु {\color{DodgerBlue3}“प्रेम्णः”} स्नेहस्य {\color{DodgerBlue3}“न हानि”}स्तावद् दुःखितमात्मानमारोप्य {\color{DodgerBlue3}“स”} प्राण्यभिमतो दुःखसन्तानः {\color{DodgerBlue3}“परितस्यति”} (१९३) दुःखमास्ते । {\color{DodgerBlue3}“न च”} दुःखहेत्वपगमोपायाभ्यां स क्लेशम्विना {\color{DodgerBlue3}“स्वस्थोऽवतिष्ठते”} ।
	\pend
      \leavevmode\marginnote{\textenglish{15a/MA}}

	  \pstart किन्तु (।)
	\pend
      
	  \bigskip
	  \begingroup
	  \large
	
	    
	    \stanza[\smallbreak]
	\label{pv.1.194b}\edlabel{pv.1.194b}\flagstanza{\tiny\textenglish{...1.194b}}मिथ्याध्यारोपहानार्थं यत्नोऽसत्यपि मोक्तरि ॥ १९४ ॥\&[\smallbreak]


	
	  \endgroup
	

	  \pstart {\color{DodgerBlue3}“मिथ्याध्यारो”}पस्य संसारित्वाध्यवसायस्य {\color{DodgerBlue3}“हानार्थं यत्नोऽसत्यपि”} कस्मिँश्चिदात्मादौ {\color{DodgerBlue3}“मोक्तरि”} । न हि यथावस्त्त्वेव व्यवहारः । किन्तु यथावसायञ्च । तथाहि रज्जुरपि सर्पाध्यवसायविषयत्वात् परिहारविषयः । एवमहमेव बद्धोऽहमेव मोक्ष्यामीत्यध्यारोपान्मुक्त्यर्थं व्यायामः । (१९४)
	\pend
      \label{div_pvv.1.195_1.196_1.197}\edlabel{div_pvv.1.195_1.196_1.197}
	  
	% new div opening: depth here is 2
	

	  \begin{center}%% label @type='head'
	\textbf{(II. मुक्तानां संसारे स्थितिः)}
	\end{center}
	

	  \pstart भवत्वात्मग्रहविपर्यस्तानां सुखाद्यभिलाषात्प्रवृत्तिलक्षणा संसारे स्थितिरुन्मीलितात्मग्रहयोनिसकलदोषराशयस्तु कस्मादासत इत्याह (।)
	\pend
      
	  \bigskip
	  \begingroup
	  \large
	
	    
	    \stanza[\smallbreak]
	\label{pv.1.195a}\edlabel{pv.1.195a}\flagstanza{\tiny\textenglish{...1.195a}}अवस्था वीतरागाणां दयया कर्मणाऽपि वा ।\&[\smallbreak]


	
	  \endgroup
	

	  \pstart a. {\color{DodgerBlue3}“अवस्था वीतरागाणां दययापि कर्मणापि वा”} वीतमोहानामपि दुःखाद् दुःखहेतोश्च लोकमुद्धर्तुकामतया स्थितिः । तावत्कालानुबन्धिशरीराक्षेपकेन कर्मणा वा स्थितिः ।
	\pend
      

	  \pstart तदेवाह (।)
	\pend
      
	  \bigskip
	  \begingroup
	  \large
	
	    
	    \stanza[\smallbreak]
	\label{pv.1.195b}\edlabel{pv.1.195b}\flagstanza{\tiny\textenglish{...1.195b}}आक्षिप्तेऽविनिवृत्तीष्टेः ।\&[\smallbreak]


	
	  \endgroup
	

	  \pstart {\color{DodgerBlue3}“आक्षिप्ते”} कर्मणा कायेऽविनिवृत्ते{\color{DodgerBlue3}“र्निवृत्य”}भाव{\color{DodgerBlue3}“स्येष्टेः”} । यद्येवं जन्मान्तरा{\color{DodgerBlue3}“क्षेपकस्य”} कर्मणः सद्भावात् भवा\edlabel{pvv.77-1}\footnote{\label{pvv.77-1}  १ तत्र च ते प्रेक्षकाः ।} न्तरञ्च स्यादित्याह---
	\pend
      
	  \bigskip
	  \begingroup
	  \large
	
	    
	    \stanza[\smallbreak]
	\label{pv.1.195c}\edlabel{pv.1.195c}\flagstanza{\tiny\textenglish{...1.195c}}सहकारिक्षयादलम् ॥ १९५ ॥\&[\smallbreak]


	
	  \endgroup
	
	  \bigskip
	  \begingroup
	  \large
	
	    
	    \stanza[\smallbreak]
	\label{pv.1.196a}\edlabel{pv.1.196a}\flagstanza{\tiny\textenglish{...1.196a}}नाक्षेप्तुमपरं कर्म भवतृष्णाविलंघिनाम् ।\&[\smallbreak]


	
	  \endgroup
	\leavevmode\marginnote{\textenglish{078/s}}

	  \pstart {\color{DodgerBlue3}“सहकारिण”} आत्मात्मीयादिविपर्यासिज्ञानस्य तृष्णायाश्च {\color{DodgerBlue3}“क्षयान्नालं”} (१९५) न शक्तं {\color{DodgerBlue3}“कर्माक्षेप्तुम”}परं भवं {\color{DodgerBlue3}“भवतृष्णाविलंधिनां”} । तृष्णालंघना नैरात्म्यदृष्टिरित्यात्मात्मीयादिमोहनिवृत्तिश्चोक्ता । अनेन सहाकारिवैकल्यमुक्तं ।
	\pend
      

	  \pstart ननु दया सत्त्वदर्शनात् तच्च क्षीणं मुक्तानां तत्कथन्दयया स्थितिरित्याह (।)
	\pend
      
	  \bigskip
	  \begingroup
	  \large
	
	    
	    \stanza[\smallbreak]
	\label{pv.1.196b}\edlabel{pv.1.196b}\flagstanza{\tiny\textenglish{...1.196b}}दुःखज्ञानेऽविरुद्धस्य पूर्वसंस्कारवाहिनी ॥ १९६ ॥\&[\smallbreak]


	
	  \endgroup
	
	  \bigskip
	  \begingroup
	  \large
	
	    
	    \stanza[\smallbreak]
	\label{pv.1.197a}\edlabel{pv.1.197a}\flagstanza{\tiny\textenglish{...1.197a}}वस्तुधर्मो दयोत्पत्तिर्न सा सत्वानुरोधिनी ।\&[\smallbreak]


	
	  \endgroup
	

	  \pstart {\color{DodgerBlue3}“दुःख”}स्यानित्यदुःखशून्यानात्मकाकारस्य {\color{DodgerBlue3}“ज्ञाने”} सत्य{\color{DodgerBlue3}“विरुद्धस्य”} द्वेषाभावात् सर्व्वत्राप्रतिह\edlabel{pvv.78-1}\footnote{\label{pvv.78-1}  १ द्वेषो हि कृपाविरोधी स प्रहीणद्वेषः ।} तस्य {\color{DodgerBlue3}“पूर्व्वसंस्कारवाहिनी”} (१९६) पूर्व्वाभ्या\edlabel{pvv.78-2}\footnote{\label{pvv.78-2}  २ करुणाया । शास्तुः ।}सप्रवृत्ता या {\color{DodgerBlue3}“दयोत्पत्तिः सा न सत्त्वानुरोधिनी”} सत्त्वदृष्टिवशा किन्तु वस्तुनो दुःखस्य कृपाविषयतयाऽभ्यस्तस्य ध\edlabel{pvv.78-3}\footnote{\label{pvv.78-3}  ३ एतेन धर्मालम्बनी कृपोक्ता}र्मः । उन्मूलितात्मदृष्टीनामपि दुःखस्य कृपाविषयतयाऽभ्यस्तस्य संमुखीभावमात्रेण दयोत्पद्यत इत्यर्थः ।
	\pend
      

	  \pstart b. एवन्तर्हि रागोपि मुक्तानां स्यादित्याह (।)
	\pend
      
	  \bigskip
	  \begingroup
	  \large
	
	    
	    \stanza[\smallbreak]
	\label{pv.1.197b}\edlabel{pv.1.197b}\flagstanza{\tiny\textenglish{...1.197b}}आत्मान्तरसमारोपाद् रागो धर्मेऽतदात्मके ॥ १९७ ॥\&[\smallbreak]


	
	  \endgroup
	

	  \pstart {\color{DodgerBlue3}“आ\edlabel{pvv.78-4}\footnote{\label{pvv.78-4}  ४ उक्तस्कन्धेभ्योऽन्यत्वादन्तशब्दः ।}त्मान्तरस्य”} स्थिरसुखात्मात्मीयरूपस्या{\color{DodgerBlue3}“रोपात् धर्मे”} स्कन्धमात्ररूपेऽ{\color{DodgerBlue3}“तदात्मके”} वस्तुनोऽस्थिरादिस्वभावे {\color{DodgerBlue3}“रागो”}ऽभिष्वङ्गरूपो भवति ॥ (१९७) ।
	\pend
      \label{div_pvv.1.198_1.199_1.200}\edlabel{div_pvv.1.198_1.199_1.200}
	  
	% new div opening: depth here is 2
	
	  \bigskip
	  \begingroup
	  \large
	
	    
	    \stanza[\smallbreak]
	\label{pv.1.198a}\edlabel{pv.1.198a}\flagstanza{\tiny\textenglish{...1.198a}}दुःखसन्तानसंस्पर्शमात्रेणैव दयोदयः ।\&[\smallbreak]


	
	  \endgroup
	

	  \pstart {\color{DodgerBlue3}“दयोदयस्तु दुःखसन्तानस्य संस्पर्शो”} दर्शनं त{\color{DodgerBlue3}“न्मात्रेणैवं”} भ\edlabel{pvv.78-5}\footnote{\label{pvv.78-5}  ५ इति रागादन्या कृपा ।}वति । न तत्र सत्त्वदर्शनापेक्षा ।
	\pend
      

	  \pstart c. यथा दुःखदर्शनात् दयोत्पत्तिस्तथापकारिणि द्वेषोपि स्यादित्याह (।)
	\pend
      
	  \bigskip
	  \begingroup
	  \large
	
	    
	    \stanza[\smallbreak]
	\label{pv.1.198b}\edlabel{pv.1.198b}\flagstanza{\tiny\textenglish{...1.198b}}मोहश्च मूलं दोषाणां स च सत्वग्रहः;\&[\smallbreak]


	
	  \endgroup
	

	  \pstart {\color{DodgerBlue3}“मोहश्च\edlabel{pvv.78-6}\footnote{\label{pvv.78-6}  ६ अमूढस्य पापावृत्तेः ।}”} मूलमादिकारणं {\color{DodgerBlue3}“दोषाणां स च”} मोहः {\color{DodgerBlue3}“सत्त्वग्रहः”} । उन्मूलितसत्त्वदृष्टेश्च । (१९८)
	\pend
      
	  \bigskip
	  \begingroup
	  \large
	
	    
	    \stanza[\smallbreak]
	\label{pv.1.198c}\edlabel{pv.1.198c}\flagstanza{\tiny\textenglish{...1.198c}}विना ॥ १९८ ॥\&[\smallbreak]


	
	  \endgroup
	
	  \bigskip
	  \begingroup
	  \large
	
	    
	    \stanza[\smallbreak]
	\label{pv.1.199a}\edlabel{pv.1.199a}\flagstanza{\tiny\textenglish{...1.199a}}तेनोद्यहेतौ न द्वेषो, न दोषोऽतः कृपा मता ।\&[\smallbreak]


	
	  \endgroup
	\leavevmode\marginnote{\textenglish{079/s}}

	  \pstart {\color{DodgerBlue3}“तेन”} सत्त्वग्रहेण विनाऽ{\color{DodgerBlue3}“द्यहेता”}वपकारिणि {\color{DodgerBlue3}“न द्वेषो”}स्ति आत्मनोऽदर्शनात्तदपकारभ्रान्त्यभावात् । {\color{DodgerBlue3}“अतो”} दोषमूलस्यात्मग्रहस्याभावादुत्पद्यमाना {\color{DodgerBlue3}“कृपा न दोषो मता”} (।)
	\pend
      

	  \pstart b. यदि पूर्व्वकर्म्मावेधतस्तदैव न निर्व्वाणं\edlabel{pvv.79-1}\footnote{\label{pvv.79-1}  १ किन्तु स्थितिस्तर्हि कृपापरवशस्य ।} तदा सांसारिकतैव स्यादित्याह (।)
	\pend
      
	  \bigskip
	  \begingroup
	  \large
	
	    
	    \stanza[\smallbreak]
	\label{pv.1.199b}\edlabel{pv.1.199b}\flagstanza{\tiny\textenglish{...1.199b}}नामुक्तिः पूर्वसंस्कारक्षयेऽन्याप्रतिसन्धितः ॥ १९९ ॥\&[\smallbreak]


	
	  \endgroup
	
	  \bigskip
	  \begingroup
	  \large
	
	    
	    \stanza[\smallbreak]
	\label{pv.1.200}\edlabel{pv.1.200}\flagstanza{\tiny\textenglish{....1.200}}अक्षीणशक्तिः संस्कारो येषां तिष्ठन्ति तेऽनधाः ॥&मन्दत्वात् करुणायाश्च न यत्नः स्थापने महान् ॥ २०० ॥\&[\smallbreak]


	
	  \endgroup
	

	  \pstart {\color{DodgerBlue3}“नामुक्तिः”} किन्तु मुक्तिरेव {\color{DodgerBlue3}“पूर्व्वसंस्कारक्षये”} पूर्व्वकर्मावेधक्षये {\color{DodgerBlue3}“सत्यन्यस्य”} दुःखस्य हेतुवैकल्याद{\color{DodgerBlue3}“प्रतिसन्धितः”} । (१९९) येषां पुनर्महाकृपाणां प्राणिधानपरिपुष्टस्य जन्माक्षेपककर्मणः {\color{DodgerBlue3}“सँस्कारोऽक्षीणशक्तिस्तेऽनधाः”} सम्यक्‏सम्बुद्धाः यावदाकाश{\color{DodgerBlue3}“न्तिष्ठन्त्येव”} । श्रावकाणान्तु कर्मणो नियतकालस्थितिकदेहाक्षेपकत्वा{\color{DodgerBlue3}“न्मन्दत्वात्”} करुणाया {\color{DodgerBlue3}“यत्नश्च महान् स्था”}पने नास्तीति {\color{DodgerBlue3}“न”} सदा स्थितिः (। २००)
	\pend
      \label{div_pvv.1.201_1.202_1.203_1.204_1.205}\edlabel{div_pvv.1.201_1.202_1.203_1.204_1.205}
	  
	% new div opening: depth here is 2
	
	  \bigskip
	  \begingroup
	  \large
	
	    
	    \stanza[\smallbreak]
	\label{pv.1.201a}\edlabel{pv.1.201a}\flagstanza{\tiny\textenglish{...1.201a}}तिष्ठन्त्येव पराधीना येषां तु महती कृपा ।\&[\smallbreak]


	
	  \endgroup
	

	  \pstart तिष्ठन्त्येव सर्व्वदा ते {\color{DodgerBlue3}“पराधीनाः”} परेषामुपकरणीकृतात्मानो महामुनयः । येषामकारणवत्सलानां {\color{DodgerBlue3}“मह”}ती {\color{DodgerBlue3}“कृपा”} ।
	\pend
      

	  \begin{center}%% label @type='head'
	\textbf{III. सत्कार्यदृष्टेर्विगमः}
	\end{center}
	
	  \bigskip
	  \begingroup
	  \large
	
	    
	    \stanza[\smallbreak]
	\label{pv.1.201b}\edlabel{pv.1.201b}\flagstanza{\tiny\textenglish{...1.201b}}सत्कायदृष्टेर्विगमादाद्य एवाभवो भवेत् ॥ २०१ ॥\&[\smallbreak]


	
	  \endgroup
	
	  \bigskip
	  \begingroup
	  \large
	
	    
	    \stanza[\smallbreak]
	\label{pv.1.202a}\edlabel{pv.1.202a}\flagstanza{\tiny\textenglish{...1.202a}}मार्गे चेत् सहजाहानेर्न हानौ वा भवः कुतः ।\&[\smallbreak]


	
	  \endgroup
	

	  \pstart a. नन्वाद्य एव मार्ग्गे दर्शनमार्ग्गे {\color{DodgerBlue3}“सत्कायदृष्टेर्व्वि”}गमात् स्रोत आपन्नस्य {\color{DodgerBlue3}“भवो”} जन्मान्तरबन्धो {\color{DodgerBlue3}“न भवे”}दिति {\color{DodgerBlue3}“चेत्”} (२०१)। {\color{DodgerBlue3}“सहजाहानेर्न”}। द्विधा हि सत्कायदृष्टिराभिसाँस्कारिकी या स्कन्धव्यतिरिक्तात्माध्यवसायिनी, सहजा च । तत्र प्रथमा दर्शनमार्ग्गे हीयते {\color{DodgerBlue3}“न”} द्वितीया भावनामार्गहेया । सा च मोहः तृष्णायाश्च हेतुरिति भवति जन्मप्रबन्धः । यदि तु पटुतरप्रज्ञस्याद्य एव मार्ग्गो मार्ग्गान्तरस्वभावः तदा {\color{DodgerBlue3}“हानौ वा”} सहजाया आत्मदृष्टेः पुन{\color{DodgerBlue3}“र्भवः कुत”}ः ।
	\pend
      

	  \pstart b. कीदृशं पुनस्तत्सहजसत्त्वदर्शनमित्याह (।)
	\pend
      
	  \bigskip
	  \begingroup
	  \large
	
	    
	    \stanza[\smallbreak]
	\label{pv.1.202b}\edlabel{pv.1.202b}\flagstanza{\tiny\textenglish{...1.202b}}सुखी भवेयं दुःखी वा मा भूवमिति तृष्यतः ॥ २०२ ॥\&[\smallbreak]


	
	  \endgroup
	
	  \bigskip
	  \begingroup
	  \large
	
	    
	    \stanza[\smallbreak]
	\label{pv.1.203a}\edlabel{pv.1.203a}\flagstanza{\tiny\textenglish{...1.203a}}यैवाऽहमिति धीः सैव सहजं सत्त्वदर्शनम् ।\&[\smallbreak]


	
	  \endgroup
	\leavevmode\marginnote{\textenglish{080/s}}

	  \pstart {\color{DodgerBlue3}“सुखी भवेयं दुःखी वा माभूवमिति तृष्यतः”} । कांक्षमाणस्य {\color{DodgerBlue3}“यैवाहमिति धीः सैव सहजं सत्त्वदर्शन”}मुच्यते ।
	\pend
      

	  \pstart c. तदप्रहीणं स्रोत-आपन्नस्येति कथं ज्ञायत इत्याह (।)
	\pend
      
	  \bigskip
	  \begingroup
	  \large
	
	    
	    \stanza[\smallbreak]
	\label{pv.1.203b}\edlabel{pv.1.203b}\flagstanza{\tiny\textenglish{...1.203b}}न ह्यपश्यन्नहमिति कश्चिदात्मनि स्निह्यति ॥ २०३ ॥\&[\smallbreak]


	
	  \endgroup
	
	  \bigskip
	  \begingroup
	  \large
	
	    
	    \stanza[\smallbreak]
	\label{pv.1.204a}\edlabel{pv.1.204a}\flagstanza{\tiny\textenglish{...1.204a}}न चात्मनि विना प्रेम्णा सुखकामोऽभिधावति ।\&[\smallbreak]


	
	  \endgroup
	

	  \pstart {\color{DodgerBlue3}“नह्यपश्यन्नहमिति कश्चिदात्मनि स्निह्यति”} किन्तु पश्यन्नेव (२०३) ।
	\pend
      

	  \pstart {\color{DodgerBlue3}“न चात्मनि विना प्रेम्णा सुखकामः”} कश्चिद् गर्भस्थानादि{\color{DodgerBlue3}“मभिधावति”} । अभिधावति \leavevmode\marginnote{\textenglish{15b/MA}} च गर्भस्थानं (।) प्रहीणाभिसंस्कारिक सत्त्वदृष्टिरपि स्रोत आपन्नं तस्याप्रहीणं सहजं सत्त्वदर्शनं ।
	\pend
      

	  \begin{center}%% label @type='head'
	\textbf{(IV. बंधमोक्षव्यवस्था)}
	\end{center}
	

	  \pstart ननु सत्यात्मनि बन्धमोक्षावेकाधिकरणौ युक्तौ । नेत्याह ।\edlabel{pvv.80-1}\footnote{\label{pvv.80-1}  १ सत्यमप्यात्मनि बन्धाद्यभावमाह ।}
	\pend
      
	  \bigskip
	  \begingroup
	  \large
	
	    
	    \stanza[\smallbreak]
	\label{pv.1.204b}\edlabel{pv.1.204b}\flagstanza{\tiny\textenglish{...1.204b}}दुःखस्योत्पादहेतुत्वं बन्धः, नित्यस्य तत् कुतः ॥ २०४ ॥\&[\smallbreak]


	
	  \endgroup
	
	  \bigskip
	  \begingroup
	  \large
	
	    
	    \stanza[\smallbreak]
	\label{pv.1.205a}\edlabel{pv.1.205a}\flagstanza{\tiny\textenglish{...1.205a}}अदुःखोत्पादहेतुत्वं मोक्षः, नित्यस्य तत् कुतः ।\&[\smallbreak]


	
	  \endgroup
	

	  \pstart a. {\color{DodgerBlue3}“दुःखस्यो”}पादानस्कन्धाना{\color{DodgerBlue3}“मुत्पादहेतुत्वं”} बन्धः । {\color{DodgerBlue3}“तत्कुतो नित्यस्य”} क्रमयौगपद्याभ्यामर्थक्रिया विरहान् (२०४) । {\color{DodgerBlue3}“अदुःखोत्पादहेतुत्वं”} दुःखं प्रत्ययहेतुता {\color{DodgerBlue3}“मोक्षः । तच्च नित्य\edlabel{pvv.80-2}\footnote{\label{pvv.80-2}  २ बन्धासिद्धो ।} स्य कुतः”} । न हि पूर्व्वापरैकस्य भावस्य दुःखहेतोः पश्चादहेतुत्वं युक्तं ।
	\pend
      

	  \pstart b. स्यादेतत् (।) न नित्यस्य हेतुत्वं बन्धमोक्षौ च युक्ताविति । नित्यत्वानित्य {\color{DodgerBlue3}“त्वाभ्यामवाच्यस्य पुद्गलस्य”} तौ भविष्यत इति मन्वानं वै भा षि क म्प्रत्याह (।)
	\pend
      
	  \bigskip
	  \begingroup
	  \large
	
	    
	    \stanza[\smallbreak]
	\label{pv.1.205b}\edlabel{pv.1.205b}\flagstanza{\tiny\textenglish{...1.205b}}अनित्यत्वेन योऽवाच्यः स हेतुर्न हि कस्यचित् ॥ २०५ ॥\&[\smallbreak]


	
	  \endgroup
	

	  \pstart {\color{DodgerBlue3}“अनित्यत्वेन योऽवाच्यः”} । अनित्यत्वमुपलक्षणं । नित्यत्वेनाप्यवाच्यः {\color{DodgerBlue3}“सहेतुर्न हि कस्यचित्”} । तस्मैवाभावात् । तथाहि यद्यसावनित्यो न भवति स्यान्नित्यः । (२०५)
	\pend
      \label{div_pvv.1.206}\edlabel{div_pvv.1.206}
	  
	% new div opening: depth here is 2
	

	  \pstart अथ नित्यो न भवति स्यादनित्यः । अन्योन्याभावलक्षणत्वादनयोरेकविधि प्रतिषेधस्यापरप्रतिषधविधिनान्तरीयकत्वात् न क्वचिद्वस्तुनि द्वयप्रतिषेधसम्भव इति पुद्गलस्याभावादहेतु\edlabel{pvv.80-2-bis}\footnote{\label{pvv.80-2-bis}  २ बन्धासिद्धो ।}त्वं अतो {\color{DodgerBlue3}“बन्धमोक्षावप्यवाच्ये”} पुद्गले {\color{DodgerBlue3}“न युज्येते कथञ्चन”} ।
	\pend
      \leavevmode\marginnote{\textenglish{081/s}}

	  \pstart C. अथ नित्यत्वेनावाच्यत्वान्न दोष इति चेत् । नन्वेवमनित्य एवोक्तः स्यात् । तथाहि\edlabel{pvv.81-1}\footnote{\label{pvv.81-1}  १ न नित्यमन्यदेव किञ्चिद्यावता ।} (।)
	\pend
      
	  \bigskip
	  \begingroup
	  \large
	
	    
	    \stanza[\smallbreak]
	\label{pv.1.206}\edlabel{pv.1.206}\flagstanza{\tiny\textenglish{....1.206}}नित्यं तमाहुर्विद्वांसो यः स्वभावो न नश्यति ॥ २०६ ॥\&[\smallbreak]


	
	  \endgroup
	

	  \pstart {\color{DodgerBlue3}“नित्यं तमाहुर्व्विद्वांसो यः स्वभावो:”} सर्वदा {\color{DodgerBlue3}“न नश्यति”} स चेदीदृशो न भवत्यनित्य एव स्यात् । (२०६)
	\pend
      \label{div_pvv.1.207}\edlabel{div_pvv.1.207}
	  
	% new div opening: depth here is 2
	
	  \bigskip
	  \begingroup
	  \large
	
	    
	    \stanza[\smallbreak]
	\label{pv.1.207a}\edlabel{pv.1.207a}\flagstanza{\tiny\textenglish{...1.207a}}त्यक्त्वेमां ह्रेपणीं दृष्टिमतोऽनित्यः स उच्यताम् ॥\&[\smallbreak]


	
	  \endgroup
	

	  \pstart अक्षे{\color{DodgerBlue3}“मां”} पुद्गल{\color{DodgerBlue3}“दृष्टिं”} पुद्‏गलनैरात्म्यवादिनः संस्कारानित्यतावा{\color{DodgerBlue3}“दिनश्च शास्तुः”} शिष्याणां {\color{DodgerBlue3}“ह्रेपणीं”} लज्जावर्द्धनीं {\color{DodgerBlue3}“त्यक्त्वाऽनित्यः स”} पुद्गल {\color{DodgerBlue3}“उच्यतां”} येन {\color{DodgerBlue3}“बन्धमोक्षौ”} युज्येते । एकाकारो निरोधो व्याख्यातः । X ॥ X ।
	\pend
      

	  \begin{center}%% label @type='head'
	\textbf{(घ) मार्गसत्यम्}
	\end{center}
	

	  \begin{center}%% label @type='head'
	\textbf{(चतुराकारं मार्गसत्यम्)}
	\end{center}
	

	  \pstart त्रय आकारा वक्ष्यन्ते ।
	\pend
      
	  \bigskip
	  \begingroup
	  \large
	
	    
	    \stanza[\smallbreak]
	\label{pv.1.207b}\edlabel{pv.1.207b}\flagstanza{\tiny\textenglish{...1.207b}}उक्तो मार्गः, तदभ्यासादाश्रयः परिवर्तते ॥ २०७ ॥\&[\smallbreak]


	
	  \endgroup
	

	  \pstart मार्गसत्यं चतुराकारं वक्तुमाह (।)
	\pend
      

	  \pstart a. {\color{DodgerBlue3}“उक्तो मार्गः”} शास्तृपदव्याख्यावसरे\edlabel{pvv.81-2}\footnote{\label{pvv.81-2}  २ “उपायाभ्यास एवायन्तादात्म्याच्छासनं मतमि” \href{http://http://sarit.indology.info/?cref=pv.1.140}{(१।१४०)} त्यनेन भ्रान्तिनिवर्तकागमनकालमात्रमस्यावस्थानात् ।} नैरात्म्यदर्शनलक्षणः । {\color{DodgerBlue3}“तस्याभ्यासादाश्रयः”} क्लेशवासनाभूतया ल य वि ज्ञा नं {\color{DodgerBlue3}“परिवर्तते”} क्लिष्टदशानिरोधात् क्लेशविसंयुक्तचित्तप्रबन्धात्मना परिणमति । (२०७)
	\pend
      \label{div_pvv.1.208_1.209_1.210}\edlabel{div_pvv.1.208_1.209_1.210}
	  
	% new div opening: depth here is 2
	

	  \pstart अनेन मार्गत इति मार्गाकारो दर्शितः क्लेशविसंयोगहेतुत्वात् ।
	\pend
      
	  \bigskip
	  \begingroup
	  \large
	
	    
	    \stanza[\smallbreak]
	\label{pv.1.208a}\edlabel{pv.1.208a}\flagstanza{\tiny\textenglish{...1.208a}}सात्म्येऽपि दोषभावश्चेन्मार्गवत् ।\&[\smallbreak]


	
	  \endgroup
	

	  \pstart मार्गस्याभ्यासप्रकर्षात् {\color{DodgerBlue3}“सात्म्येपि”} प्रकृतित्वे च प्राप्ते पुन{\color{DodgerBlue3}“र्दोषा”}णां मोहादीनां {\color{DodgerBlue3}“भावः”} प्राप्नोति {\color{DodgerBlue3}“चेत् । मार्गवत्”} यथा बन्धावस्थायां दोषसात्म्येपि मार्गोऽभ्यासवशादाविर्भवति ।
	\pend
      

	  \pstart अत्राह (।)
	\pend
      
	  \bigskip
	  \begingroup
	  \large
	
	    
	    \stanza[\smallbreak]
	\label{pv.1.208b}\edlabel{pv.1.208b}\flagstanza{\tiny\textenglish{...1.208b}}नाविभुत्वतः ।\&[\smallbreak]


	
	  \endgroup
	\leavevmode\marginnote{\textenglish{082/s}}

	  \pstart {\color{DodgerBlue3}“नाविभुत्वतः”} असामर्थ्यात् । मार्गसात्म्येऽपि स्थितस्य चेतसि न दोषाणामुत्पत्तुं सामर्थ्य\edlabel{pvv.82-1}\footnote{\label{pvv.82-1}  १ प्रकृतिशुद्धहेमवत् ।}मस्ति । तन्निदानभूतस्य सत्वदर्शनस्योन्मूलितत्वात् ।
	\pend
      

	  \pstart {\color{DodgerBlue3}“अनेन”} शान्तत इति निरोधाकार उक्तो दोषाणां सर्व्वथा शान्तत्वात् ।
	\pend
      

	  \begin{center}%% label @type='head'
	\textbf{(ग. सत्कायदृष्टिः)}
	\end{center}
	

	  \begin{center}%% label @type='head'
	\textbf{(क) सत्वदर्शनाभावे कारणम्}
	\end{center}
	

	  \pstart सत्वदर्शनमेव पुनः कस्मान्न भवतीत्याह (।)
	\pend
      
	  \bigskip
	  \begingroup
	  \large
	
	    
	    \stanza[\smallbreak]
	\label{pv.1.208c}\edlabel{pv.1.208c}\flagstanza{\tiny\textenglish{...1.208c}}विषयग्रहणं धर्मों विज्ञानस्य यथास्ति सः ॥ २०८ ॥\&[\smallbreak]


	
	  \endgroup
	
	  \bigskip
	  \begingroup
	  \large
	
	    
	    \stanza[\smallbreak]
	\label{pv.1.209a}\edlabel{pv.1.209a}\flagstanza{\tiny\textenglish{...1.209a}}गृह्यते सोऽस्य जनको विद्यमानात्मनेति च ।\&[\smallbreak]


	
	  \endgroup
	

	  \pstart {\color{DodgerBlue3}“विषयग्रहणम्विज्ञानस्य”} तावत्सति ग्राह्यग्राहकभावो {\color{DodgerBlue3}“धर्म\edlabel{pvv.82-2}\footnote{\label{pvv.82-2}  २ अर्थग्राहिज्ञानमिच्छतो वक्तव्यः ।}: यथा”} चास्ति {\color{DodgerBlue3}“स”} स विषयस्तथा {\color{DodgerBlue3}“गृह्यते”} । विज्ञानेन विषयिणा (२०८) । {\color{DodgerBlue3}“स”} च विषयोस्य विज्ञानस्य {\color{DodgerBlue3}“जनको विद्यमानेनात्मना”} यथावस्थितेन रूपेण ।
	\pend
      
	  \bigskip
	  \begingroup
	  \large
	
	    
	    \stanza[\smallbreak]
	\label{pv.1.209b}\edlabel{pv.1.209b}\flagstanza{\tiny\textenglish{...1.209b}}एषा प्रकृतिरस्यास्तन्निमित्तान्तरतः स्खलत् ॥ २०९ ॥\&[\smallbreak]


	
	  \endgroup
	
	  \bigskip
	  \begingroup
	  \large
	
	    
	    \stanza[\smallbreak]
	\label{pv.1.210a}\edlabel{pv.1.210a}\flagstanza{\tiny\textenglish{...1.210a}}व्यावृत्तौ प्रत्ययापेक्षमदृढं सर्पबुद्धिवत् ।\&[\smallbreak]


	
	  \endgroup
	

	  \pstart यथावस्थितवस्तुग्रहणञ्च {\color{DodgerBlue3}“नात्मैषा प्रकृतिः”} स्वभावो ज्ञानस्य विषयिणः । यथास्वभावं स्वग्राहिज्ञानजननञ्च विषयस्य प्रकृतिः । अस्याः प्रकृतेस्तज्ज्ञानं {\color{DodgerBlue3}“निमित्तान्तरतः”} आन्तरादविद्यारूपादागन्तुकाच्च विषयदोषादेः {\color{DodgerBlue3}“स्खलद्वि”}षयग्रहण{\color{DodgerBlue3}“विपरीताकारं\edlabel{pvv.82-3}\footnote{\label{pvv.82-3}  ३ सत्वाध्यवसायि ।}”} भवति (। २०९) {\color{DodgerBlue3}“तच्च व्यावृत्तौ”} विषयविपरीतग्रहणाकारताया निवृत्त्यर्थ {\color{DodgerBlue3}“प्रत्ययापेक्षं”} भ्रान्तिनिवर्तककारणमपेक्ष्यमाण{\color{DodgerBlue3}“मदृढम”}स्था\edlabel{pvv.82-4}\footnote{\label{pvv.82-4}  ४ मन्दमन्दप्रकाशे सर्प्पोचिते प्रदेशे ।}ष्णु {\color{DodgerBlue3}“सर्पबुद्धिवत्”} । यथा सर्पबुद्धी रज्वा भ्रान्तिनिमित्ता\edlabel{pvv.82-5}\footnote{\label{pvv.82-5}  ५ असहजागन्तुर्दोषः । नैरात्म्यं प्रकृतिः प्रमासिद्धिः ।} जाता रज्जुस्वरूपग्राहिणः प्रत्ययान्निवृत्ता न पुनरुद्भवतीति । तथा भ्रान्तिनिमित्तनिरासात् दृष्टे नैरात्म्ये वस्तुनि सति नास्ति \leavevmode\marginnote{\textenglish{16a/MA}} सत्त्वे दृष्टिसम्भवः ज्ञानस्य विषयस्वरूपग्रहणप्रवणत्वात् । विषयस्य च स्वाकारार्पणप्रवृत्तत्वात् ।
	\pend
      

	  \pstart a. किञ्च (।)
	\pend
      
	  \bigskip
	  \begingroup
	  \large
	
	    
	    \stanza[\smallbreak]
	\label{pv.1.210b}\edlabel{pv.1.210b}\flagstanza{\tiny\textenglish{...1.210b}}प्रभास्वरमिदञ्चित्तं प्रकृत्यागन्तवो मलाः ॥ २१० ॥\&[\smallbreak]


	
	  \endgroup
	

	  \pstart {\color{DodgerBlue3}“प्रभास्वरमना”}त्मभूतदोषसञ्चय{\color{DodgerBlue3}“मिदं चित्तं प्रकृत्या”} स्वभावेन । ये तु मनोदोषा \leavevmode\marginnote{\textenglish{083/s}} दृश्यन्ते ते भ्रान्तिनिमित्तोपनीतत्वादा{\color{DodgerBlue3}“गन्तवो”}ऽस्वभावभूताश्चेतसः । तमः {\color{DodgerBlue3}“तुहिनादय”} इव नभसः (२१०) ।
	\pend
      \label{div_pvv.1.211_1.212_1.213_1.214_1.215_1.216_1.217_1.218_1.219_1.220_1.221_1.222}\edlabel{div_pvv.1.211_1.212_1.213_1.214_1.215_1.216_1.217_1.218_1.219_1.220_1.221_1.222}
	  
	% new div opening: depth here is 2
	
	  \bigskip
	  \begingroup
	  \large
	
	    
	    \stanza[\smallbreak]
	\label{pv.1.211a}\edlabel{pv.1.211a}\flagstanza{\tiny\textenglish{...1.211a}}तत्प्रागप्यसमर्थानां पश्चाच्छक्तिः क तन्मये ॥\&[\smallbreak]


	
	  \endgroup
	

	  \pstart {\color{DodgerBlue3}“तत्प्रागपि”} तस्मान्नैरात्म्यदर्शनात्पूर्व्वमपि ततः श्रुतचिन्ताध्यवसानेप्यापातविष्कम्भनादुत्पत्तुम{\color{DodgerBlue3}“समर्थानां”} मलानां {\color{DodgerBlue3}“पश्चा”}न्मार्गनिष्पत्तौ {\color{DodgerBlue3}“शक्ति”}रुत्पत्तुं {\color{DodgerBlue3}“क्व तन्मये”} मार्गसात्म्ये स्वरूपे (।)
	\pend
      

	  \pstart एतदेवाह (।)
	\pend
      
	  \bigskip
	  \begingroup
	  \large
	
	    
	    \stanza[\smallbreak]
	\label{pv.1.211b}\edlabel{pv.1.211b}\flagstanza{\tiny\textenglish{...1.211b}}नालं प्ररोढुमत्यन्तं स्यन्दिन्यामग्निवद् भुवि ॥ २११ ॥\&[\smallbreak]


	
	  \endgroup
	
	  \bigskip
	  \begingroup
	  \large
	
	    
	    \stanza[\smallbreak]
	\label{pv.1.212a}\edlabel{pv.1.212a}\flagstanza{\tiny\textenglish{...1.212a}}बाधकोत्पत्तिसामर्थ्यगर्भे शक्तोऽपि वस्तुनि ।\&[\smallbreak]


	
	  \endgroup
	

	  \pstart शक्तोऽप्यद्धतोपि मनोमलो {\color{DodgerBlue3}“नालमत्यन्तं प्ररोढुं वस्तुनि”} चित्तसन्ताने कीदृशे बाधकोत्पत्तिसामर्थ्यगर्भे सर्वदृष्टि{\color{DodgerBlue3}“बाधक”}स्य नैरात्म्यदर्शनस्य मार्गसत्यो{\color{DodgerBlue3}“त्पत्तिस्तस्यां सामर्थ्यं”} प्रभविष्णुत्वं {\color{DodgerBlue3}“तद्गर्भे”} आनन्तर्यस्य गर्भे तस्मिन् {\color{DodgerBlue3}“स्यन्दिन्यां\edlabel{pvv.83-1}\footnote{\label{pvv.83-1}  १ नित्यमुदकस्राविण्यां ।} भुवि अग्निवत्”} (२११) । यथा हि वह्नेर्हेतुबलादुत्पन्नोपि स्यन्दिन्यां भुवि बाधकवत्यां नात्यन्तं प्ररोहति तथा मार्गोत्पत्तिसामर्थ्यगर्भे चेतस्युत्पन्ना अपि मला नात्यन्तं विरोहन्ति । सात्मीभूतमार्ग्गें तु हेतुवैकल्यान्नोत्पद्यन्त एव ।
	\pend
      

	  \begin{center}%% label @type='head'
	\textbf{(ख) नैरात्म्यदर्शने दोषोत्पत्तिव्यवस्था}
	\end{center}
	

	  \pstart कथञ्च नैरात्म्यदर्शिनो मलोत्पत्तिराशंक्यते । न तावत् हेतुसाकल्यात्सत्वदृष्टेर्हेतुभूताया अभावात् नैरात्म्यदर्शनवत् {\color{DodgerBlue3}“सत्त्वदर्शनञ्च भावनयोत्पन्नं हेतुरिति”} चेत् ॥
	\pend
      

	  \pstart कुतस्तस्य {\color{DodgerBlue3}“भावना”} । किन्नैरात्म्यस्य सो\edlabel{pvv.83-2}\footnote{\label{pvv.83-2}  २ नैरात्म्ये यदि दोषः स्यात्तदा सत्त्वदर्शनं भावयेत् । तच्च नास्ति दोषत्रयस्याप्यभावात् ।}पद्रवत्वात् । किञ्चाभूतत्वेन भ्रमहेतुत्वात् । अस्वभावेनोपहन्तुं शक्यत्वाद्वा । एतत्त्रयमसङ्गतमित्याह (।)
	\pend
      
	  \bigskip
	  \begingroup
	  \large
	
	    
	    \stanza[\smallbreak]
	\label{pv.1.212b}\edlabel{pv.1.212b}\flagstanza{\tiny\textenglish{...1.212b}}निरुपद्रवभूतार्थस्वभावस्य विपर्ययैः ॥ २१२ ॥\&[\smallbreak]


	
	  \endgroup
	
	  \bigskip
	  \begingroup
	  \large
	
	    
	    \stanza[\smallbreak]
	\label{pv.1.213a}\edlabel{pv.1.213a}\flagstanza{\tiny\textenglish{...1.213a}}न बाधा यत्नवत्वेऽपि बुद्धेस्तत्पक्षपाततः ॥\&[\smallbreak]


	
	  \endgroup
	

	  \pstart दोषराशेरुद्वेजकस्य प्रहाणेन {\color{DodgerBlue3}“निरुपद्रवस्य”} प्रमाणसम्वादित्वेन {\color{DodgerBlue3}“भूतार्थस्य”} सत्यार्थस्यानारोपितत्वेन {\color{DodgerBlue3}“स्वभावस्य”} प्रकृतेर्नैरात्म्यस्याभिरुचितविषयस्य विपर्ययेष्वात्माद्याकारेष्वभ्यासे सोपद्रवत्वादिना {\color{DodgerBlue3}“प्रयत्न एव”} तावन्न {\color{DodgerBlue3}“सम्भवति प्रेक्षस्य”} । सम्भवेपि वा {\color{DodgerBlue3}“विपर्ययैः”} (। २१२)
	\pend
      \leavevmode\marginnote{\textenglish{084/s}}

	  \pstart a. {\color{DodgerBlue3}“न बाधा”} नैरात्म्यस्य सात्मीभूतस्य स्वभावस्यास्ति । \edlabel{pvv.84-1}\footnote{\label{pvv.84-1}  १ चित्तसंप्रयुक्तत्वात् चैत्तानां संप्रयुक्ताऽविद्येत्याह ।} {\color{DodgerBlue3}“बुद्धेस्तत्र दोषप्रतिपक्षे गुणवति मार्ग्गे पक्षपातात्”} ।
	\pend
      

	  \pstart न हि स्वभावः साक्षात्कृतोऽन्यथा कर्तु शक्यः । अनेन प्रणीत इति निरोधाकारो दर्शितः ॥
	\pend
      

	  \pstart b. ननु यदि नैरात्म्यसात्मत्वदर्शनयोः परस्परभेदाद् बा\edlabel{pvv.84-2}\footnote{\label{pvv.84-2}  २ विरोधात् ।}धा तदा रागप्रतिघयोरपि स्या\edlabel{pvv.84-3}\footnote{\label{pvv.84-3}  ३ न चास्त्येकभावनयाऽपरस्यात्यन्तनिरोधो लोक इत्यनेकान्तमाह ।}दित्याह (।)
	\pend
      
	  \bigskip
	  \begingroup
	  \large
	
	    
	    \stanza[\smallbreak]
	\label{pv.1.213b}\edlabel{pv.1.213b}\flagstanza{\tiny\textenglish{...1.213b}}आत्मग्रहैकयोनित्वात्;&कार्यकारणभावतः ॥ २१३ ॥\&[\smallbreak]


	
	  \endgroup
	
	  \bigskip
	  \begingroup
	  \large
	
	    
	    \stanza[\smallbreak]
	\label{pv.1.214a}\edlabel{pv.1.214a}\flagstanza{\tiny\textenglish{...1.214a}}रागप्रतिधयोर्बाधा भेदेऽपि न परस्परम् ।\&[\smallbreak]


	
	  \endgroup
	

	  \pstart {\color{DodgerBlue3}“आत्मग्रह एको योनि”}ष्कारणं ययोस्तौ तयोर्भावस्तस्मादेककारण{\color{DodgerBlue3}“त्वात्”} रागप्रतिधयोर्भेद\edlabel{pvv.84-4}\footnote{\label{pvv.84-4}  ४ लक्षणभेदेपि नात्यन्तविरोधस्तयोः ।}पि न परस्परं बाधा रूपरसयोरिव । तथा {\color{DodgerBlue3}“कार्यकारणभावतोपि”} (२१३) नान्योन्यं {\color{DodgerBlue3}“बाधा”} । तथाहि यदैकस्मिन् रागस्तदा तदपकारिणि द्वेषः । यदा च क्वचिद् द्वेषस्तदा तदपकारिणि रागोपीति परस्परकार्यकारणभावान्नास्ति विरोधः । चक्षु\edlabel{pvv.84-5}\footnote{\label{pvv.84-5}  ५ भेदेप्येकमन्यहेतुः ।}रादिबुद्धीनामिव ।
	\pend
      

	  \pstart c. ननु द्वेषादिप्रतिपक्षा मैत्र्यादयो न च तानुच्छिन्दन्तीत्याह (।)
	\pend
      
	  \bigskip
	  \begingroup
	  \large
	
	    
	    \stanza[\smallbreak]
	\label{pv.1.214b}\edlabel{pv.1.214b}\flagstanza{\tiny\textenglish{...1.214b}}मोहाविरोधान्मैत्र्यादेर्न्नात्यन्तं दोषनिग्रहः ॥ २१४ ॥\&[\smallbreak]


	
	  \endgroup
	
	  \bigskip
	  \begingroup
	  \large
	
	    
	    \stanza[\smallbreak]
	\label{pv.1.215a}\edlabel{pv.1.215a}\flagstanza{\tiny\textenglish{...1.215a}}तन्मूलाश्च मलाः सर्वे;\&[\smallbreak]


	
	  \endgroup
	

	  \pstart {\color{DodgerBlue3}“मोहाविरोधान्मैत्र्यादेः”} । मैत्रीकरुणादयो मोहाविरोधिनोऽतश्च दोषकारणावैकल्या{\color{DodgerBlue3}“न्नात्यन्तं”} द्वेषादि{\color{DodgerBlue3}“दोषा”}णां {\color{DodgerBlue3}“निग्रहो”} मैत्र्यादेः । आपातविष्कम्भनमात्रन्तु भवति मोहस्यानपायात् (२१४) । {\color{DodgerBlue3}“तन्मूलाश्च मलाः सर्व्वे”} प्रसूयन्ते ।
	\pend
      

	  \begin{center}%% label @type='head'
	\textbf{(ग) मोहः सत्कायदृष्टिः}
	\end{center}
	

	  \pstart ननु सत्त्वदर्शनं दोषमूलं न मोह इत्याह (।)
	\pend
      
	  \bigskip
	  \begingroup
	  \large
	
	    
	    \stanza[\smallbreak]
	\label{pv.1.215b}\edlabel{pv.1.215b}\flagstanza{\tiny\textenglish{...1.215b}}स च सत्कायदर्शनम् ।\&[\smallbreak]


	
	  \endgroup
	

	  \pstart {\color{DodgerBlue3}“स च”} मोहः {\color{DodgerBlue3}“सत्कायदर्शनं”} ।
	\pend
      \leavevmode\marginnote{\textenglish{085/s}}

	  \pstart ननु मोहोऽसंप्रख्यानरूपः । {\color{DodgerBlue3}“सत्त्वदृष्टिः तु विपरीतार्थप्रतिपत्तिरूपा । तत्कथं”} मोह एव सत्त्वदर्शनमित्याह (।)
	\pend
      
	  \bigskip
	  \begingroup
	  \large
	
	    
	    \stanza[\smallbreak]
	\label{pv.1.215c}\edlabel{pv.1.215c}\flagstanza{\tiny\textenglish{...1.215c}}विद्यायाः प्रतिपक्षत्वाच्चैत्तत्त्वेनोपलब्धितः ॥ २१५ ॥\&[\smallbreak]


	
	  \endgroup
	
	  \bigskip
	  \begingroup
	  \large
	
	    
	    \stanza[\smallbreak]
	\label{pv.1.216a}\edlabel{pv.1.216a}\flagstanza{\tiny\textenglish{...1.216a}}मिथ्योपलब्धिरज्ञानं युक्तेश्चान्यद्युक्तिमत् ॥\&[\smallbreak]


	
	  \endgroup
	

	  \pstart {\color{DodgerBlue3}“विद्यायाः प्रतिपक्षत्वात्”} । विद्याया नैरात्म्यदृष्टेर्व्विपक्षोऽविद्या । {\color{DodgerBlue3}“स चा”}प्रख्यान\edlabel{pvv.85-1}\footnote{\label{pvv.85-1}  १ अभावस्तस्य ।}मात्रम्वा\edlabel{pvv.85-2}\footnote{\label{pvv.85-2}  २ तदर्थः ।}रूपादि वा न भवति निर्व्वाणेपि {\color{DodgerBlue3}“तयोर्भावात् ।\edlabel{pvv.85-3}\footnote{\label{pvv.85-3}  ३ विद्याविरुद्धाऽविद्या ।} किन्तु मिथ्योप”}लब्धिरज्ञानमविद्याऽधर्मानृतवत् । विद्यायाः सदर्थत्वात्\edlabel{pvv.85-4}\footnote{\label{pvv.85-4}  ४ युगपदनुत्पत्तिस्तु रागानिमित्ताद् ग्रहाघातवस्तुबहुलीकरणाद्यसाधारणं कारणवत्वात् ।} । {\color{DodgerBlue3}“चैत्तत्वेनोपलब्धितश्च”} । चैत्तत्वेन करणेनोपलब्धिरूप\edlabel{pvv.85-5}\footnote{\label{pvv.85-5}  ५ अविद्यायाः ।} {\color{DodgerBlue3}“त्वाच्च”} । (२१५) {\color{DodgerBlue3}“आश्रयालम्बनाकारकालद्रव्य”}समतादिभिः समं प्रयुक्ताः संप्रयुक्ता इति संप्रयुक्तलक्षणं । {\color{DodgerBlue3}“न चासंप्रख्यानस्य”} नीरूपस्यालम्बनाकारयोग इति {\color{DodgerBlue3}“मिथ्याज्ञान”}मविद्या । {\color{DodgerBlue3}“उक्ते”}\edlabel{pvv.85-6}\footnote{\label{pvv.85-6}  ६ युक्तिं प्रतिपाद्यागममाह ।}ऽर्थे विरोधात् । {\color{DodgerBlue3}“भ ग”}वता प्युक्तं “याः काश्चन लोकव्यवहारोपपत्तयः सर्व्वास्ता आत्माभिनिवेशतो\leavevmode\marginnote{\textenglish{16b/MA}} भवन्ति आत्माभिनिवेशविगमतो न भवन्ति” इत्यनेन सत्त्वदृष्टिरेव जन्महेतुरुक्ता । आत्माभिनिवेशलक्षणत्वात्त\edlabel{pvv.85-7}\footnote{\label{pvv.85-7}  ७ अविद्यायाः ।}स्याः । {\color{DodgerBlue3}“अतोऽन्यदसंप्रख्यानलक्षणमज्ञानमयुक्तं निर्व्वा”}णेपि तत्स\edlabel{pvv.85-8}\footnote{\label{pvv.85-8}  ८ असंप्रख्यानस्य सत्त्वात् ।}त्त्वात् ।
	\pend
      

	  \pstart a. ननु यद्यविद्या दृष्टिरेव तथा च दृष्टिसंप्रयुक्ताऽविद्येति संप्रयुक्तार्थो न स्यात् । न सा तेनैव सम्प्रयुक्ता किन्तु सकलक्लेशानुगताऽविद्या\edlabel{pvv.85-9}\footnote{\label{pvv.85-9}  ९ तस्मात्पूर्व्वोक्तैवाविद्या ।} (।) सत्कायदृष्टिस्तु तदेकदे\edlabel{pvv.85-10}\footnote{\label{pvv.85-10}  १० यथा पलाशादियुक्तं वनं पाण्याददियुक्तं शरीरं ॥}शः ततश्चगमाविरोध इत्या\edlabel{pvv.85-11}\footnote{\label{pvv.85-11}  ११ इत्यभिप्राय आह ।}ह । (।)
	\pend
      
	  \bigskip
	  \begingroup
	  \large
	
	    
	    \stanza[\smallbreak]
	\label{pv.1.216b}\edlabel{pv.1.216b}\flagstanza{\tiny\textenglish{...1.216b}}व्याख्येयोऽत्र विरोधो यः;\&[\smallbreak]


	
	  \endgroup
	

	  \pstart {\color{DodgerBlue3}“अत्र”} विद्यानिर्देशे आगम{\color{DodgerBlue3}“विरोध यः”} प्रसजति {\color{DodgerBlue3}“स सा\edlabel{pvv.85-12}\footnote{\label{pvv.85-12}  १२ अविद्या ।}मान्यविशे\edlabel{pvv.85-13}\footnote{\label{pvv.85-13}  १३ दृष्टि ।}षभावेन”} भेदकल्प\edlabel{pvv.85-14}\footnote{\label{pvv.85-14}  १४ पाणेरिव शरीरात् ।}नया {\color{DodgerBlue3}“व्याख्येयः”} समर्थनीयः । यथा पलाशयुक्तम्वनमिति । दृष्टिस्वभावाऽविद्याप्राधान्येन क्लेशहेतुरित्युपदर्शनञ्च प्रयोजनं ।
	\pend
      

	  \pstart b. प्रकृतमाह (।) यतश्चैवमात्मदर्शनप्रभूताः सर्व्वक्लेशाः ।
	\pend
      \leavevmode\marginnote{\textenglish{086/s}}
	  \bigskip
	  \begingroup
	  \large
	
	    
	    \stanza[\smallbreak]
	\label{pv.1.216c}\edlabel{pv.1.216c}\flagstanza{\tiny\textenglish{...1.216c}}तद्विरोधाच्च तन्मयैः ॥ २१६ ॥\&[\smallbreak]


	
	  \endgroup
	
	  \bigskip
	  \begingroup
	  \large
	
	    
	    \stanza[\smallbreak]
	\label{pv.1.217a}\edlabel{pv.1.217a}\flagstanza{\tiny\textenglish{...1.217a}}विरोधः शून्यतादृष्टेः सर्वदोषैः प्रसिध्यति ।\&[\smallbreak]


	
	  \endgroup
	

	  \pstart तया {\color{DodgerBlue3}“सत्त्व”}दृष्ट्या विरोधाच्च {\color{DodgerBlue3}“शून्यतादृष्टे”}र्नैरात्म्यदृष्टेः । {\color{DodgerBlue3}“तन्मयैः”} (२१६) सत्त्वदृष्टिहेतुकैः {\color{DodgerBlue3}“सर्व्वदोषै”}र्व्विरोधः {\color{DodgerBlue3}“सिध्यति”} । शीतविरुद्धस्याग्नेरिव तत्कार्ये रोमहर्षादिभिरतः सात्मीभूतनैरात्म्यानां न पुनर्दोषलेशोत्पत्तिरनेन निःशरणत इति निरोधाकारो निर्द्दिष्टः दोषेभ्यः सर्व्वथा निःशरणात् ।
	\pend
      

	  \begin{center}%% label @type='head'
	\textbf{(घ) दोषाः प्रतीत्यसमुत्पन्नाः}
	\end{center}
	
	  \bigskip
	  \begingroup
	  \large
	
	    
	    \stanza[\smallbreak]
	\label{pv.1.217b}\edlabel{pv.1.217b}\flagstanza{\tiny\textenglish{...1.217b}}नाक्षयः प्राणिधर्मत्वाद् रूपादिवदसिद्धितः ॥ २१७ ॥\&[\smallbreak]


	
	  \endgroup
	
	  \bigskip
	  \begingroup
	  \large
	
	    
	    \stanza[\smallbreak]
	\label{pv.1.218a}\edlabel{pv.1.218a}\flagstanza{\tiny\textenglish{...1.218a}}सम्बन्धे प्रतिपक्षस्य त्यागस्यादर्शनादपि ।\&[\smallbreak]


	
	  \endgroup
	

	  \pstart a. स्यादेत{\color{DodgerBlue3}“दक्षयो”} रागादिः {\color{DodgerBlue3}“प्राणिधर्मत्वाद्रूपादिवत् न”} युक्तमेतद{\color{DodgerBlue3}“सिद्धितः”} । न हि {\color{DodgerBlue3}“प्राणी”} कश्चिदस्ति यद्धर्मा रागादयः सिद्ध्यन्ति । सत्त्वे दृष्टौ तु सत्यां केवल\edlabel{pvv.86-1}\footnote{\label{pvv.86-1}  १ रज्ज्वां सर्पबुद्धिवदपैति च ।}मुपलभ्यन्त इति प्रतीत्यसमुत्पादमात्रमेतत् (२१७) । तेषां {\color{DodgerBlue3}“प्रतिपक्षस्य”} नैरात्म्यदर्शनस्य सम्बन्धे संमुखीभावे त्यागस्यापातविष्कम्भणस्या{\color{DodgerBlue3}“दर्शनादपि”} सम्भवत्प्रतिपक्षत्वेनोच्छेदसम्भवात् नाक्षयित्वं ।
	\pend
      

	  \pstart b. स्यादे\edlabel{pvv.86-2}\footnote{\label{pvv.86-2}  २ वैधर्म्यदृष्टान्तमाह ।}तत् (।)ताम्रादीनामग्नियोगाद् द्रवावस्थायां नष्टमपि काठिन्यं पुनः शीतसम्पर्कादुत्पद्यते । तद्वन्नष्टानामपि दोषाणां मार्गसात्म्ये पुनः कुतश्चिद्धेतोरुत्पत्तिः स्यात् । अत्राह (।)
	\pend
      
	  \bigskip
	  \begingroup
	  \large
	
	    
	    \stanza[\smallbreak]
	\label{pv.1.218b}\edlabel{pv.1.218b}\flagstanza{\tiny\textenglish{...1.218b}}न काठिन्यवदुत्पत्तिः पुनर्दोषविरोधिनः ॥ २१८ ॥\&[\smallbreak]


	
	  \endgroup
	
	  \bigskip
	  \begingroup
	  \large
	
	    
	    \stanza[\smallbreak]
	\label{pv.1.219a}\edlabel{pv.1.219a}\flagstanza{\tiny\textenglish{...1.219a}}सात्मत्वेनानपायत्वात् अनेकान्ताच्च भस्मवत् ।\&[\smallbreak]


	
	  \endgroup
	

	  \pstart {\color{DodgerBlue3}“न काठिन्यवदुत्पत्तिः पुनर्दो”}षाणां । दोष{\color{DodgerBlue3}“विरोधिनो”} (२१८ ) नैरात्म्यस्य {\color{DodgerBlue3}“सात्मत्वेन”} प्रकृतित्वेना{\color{DodgerBlue3}“नपायात्”} । न ह्यव्याहते विरोधिनि तद्विरुद्धस्योत्पत्तिरग्नाविव शीतकाठिन्यं कादाचित्कत्वादग्निनिवृत्तौ तद्विरोधिन्या द्रवतायाः स्वरसनिरोधादुत्पद्यतेऽ{\color{DodgerBlue3}“नैकान्ताच्च भस्मवत्”} । यथा भस्मनि भूते पुनर्न काष्ठोत्पत्तिस्तथा नैरात्म्यसात्मतायां न पुनर्नष्टानां दोषाणामुत्पत्तिरित्यनैकान्तिकता नष्टोत्पत्तेः ।
	\pend
      

	  \pstart C. नन्वात्मभावनया{\color{DodgerBlue3}“पि मोक्षोस्ति”} तत्किं नैरात्म्यभावनया यदाहु“रात्मा मन्तव्यो निदिध्यासितव्य” इत्यादि । अत्राह (।)
	\pend
      
	  \bigskip
	  \begingroup
	  \large
	
	    
	    \stanza[\smallbreak]
	\label{pv.1.219b}\edlabel{pv.1.219b}\flagstanza{\tiny\textenglish{...1.219b}}यः पश्यत्यात्मानं तत्रास्याहमिति शाश्वतः स्नेहः ॥ २१९ ॥\&[\smallbreak]


	
	  \endgroup
	
	  \bigskip
	  \begingroup
	  \large
	
	    
	    \stanza[\smallbreak]
	\label{pv.1.220a}\edlabel{pv.1.220a}\flagstanza{\tiny\textenglish{...1.220a}}स्नेहात् सुखेषु तृष्यति तृष्णा दोषांस्तिरस्कुरुते ।\&[\smallbreak]


	
	  \endgroup
	\leavevmode\marginnote{\textenglish{087/s}}

	  \pstart {\color{DodgerBlue3}“यः पश्यत्यात्मानं तत्रा”}त्मन्य{\color{DodgerBlue3}“स्य”} द्रष्टु{\color{DodgerBlue3}“रहमिति शाश्वतोऽनुपा”}यि{\color{DodgerBlue3}“स्नेहो”} भवति (२१९) ।
	\pend
      

	  \pstart {\color{DodgerBlue3}“स्नेहा”}दात्मस्नेहात्सुखेषु {\color{DodgerBlue3}“तृष्यति । तृष्णा”}वान् भवतीति । {\color{DodgerBlue3}“तृष्णा च सुखसा-”} धनत्वेनाध्यवसितानां वस्तूनां {\color{DodgerBlue3}“दोषान”}शुचित्वादीन् {\color{DodgerBlue3}“तिरस्कुरुते”} प्रच्छादयति दोषतिरस्करणात् ।
	\pend
      
	  \bigskip
	  \begingroup
	  \large
	
	    
	    \stanza[\smallbreak]
	\label{pv.1.220b}\edlabel{pv.1.220b}\flagstanza{\tiny\textenglish{...1.220b}}गुणदर्शी परितृष्यन् ममेति तत्साधनान्युपादत्ते ॥ २२० ॥\&[\smallbreak]


	
	  \endgroup
	
	  \bigskip
	  \begingroup
	  \large
	
	    
	    \stanza[\smallbreak]
	\label{pv.1.221a}\edlabel{pv.1.221a}\flagstanza{\tiny\textenglish{...1.221a}}तेनात्माभिनिवेशो यावत् तावत् स संसारे ॥\&[\smallbreak]


	
	  \endgroup
	

	  \pstart {\color{DodgerBlue3}“गुणदर्शी”} शुचित्वेष्टत्वगुणान् पश्यन् {\color{DodgerBlue3}“परितृष्यन्”} ममेति {\color{DodgerBlue3}“ममेदं सुखमिति”} गर्द्धमानस्तस्य सुखस्य {\color{DodgerBlue3}“साधनानि”} गर्भगमनादी{\color{DodgerBlue3}“न्युपादत्ते”} । (२२० )
	\pend
      

	  \pstart तेनात्मदर्शनमूल\edlabel{pvv.87-1}\footnote{\label{pvv.87-1}  १ आत्मदर्शनं मूलं यस्य जन्मादेः ।}त्वेन जन्मादेरा{\color{DodgerBlue3}“त्माभिनिवेशो यावत्तावत् स”} आत्मदर्शी {\color{DodgerBlue3}“संसार”} एव (।)
	\pend
      

	  \pstart d. न केवलं जन्मप्रबन्धस्तस्य दोषा अपि समस्ताः सन्तीत्याह (।)
	\pend
      
	  \bigskip
	  \begingroup
	  \large
	
	    
	    \stanza[\smallbreak]
	\label{pv.1.221b}\edlabel{pv.1.221b}\flagstanza{\tiny\textenglish{...1.221b}}अत्मनि सति परसंज्ञा स्वपरविभागात् परिग्रहद्वेषौ ॥ २२१ ॥\&[\smallbreak]


	
	  \endgroup
	
	  \bigskip
	  \begingroup
	  \large
	
	    
	    \stanza[\smallbreak]
	\label{pv.1.222a}\edlabel{pv.1.222a}\flagstanza{\tiny\textenglish{...1.222a}}अनयोः संप्रतिबद्धाः सर्वे दोषाः प्रजायन्ते ॥\&[\smallbreak]


	
	  \endgroup
	

	  \pstart {\color{DodgerBlue3}“आत्मनि सति”} ततोऽन्यस्मिन् {\color{DodgerBlue3}“परसंज्ञा”} परबुद्धिर्भवति । {\color{DodgerBlue3}“स्वपरविभागाच्च”} कारणात् स्वपरयोर्यथाक्रमं {\color{DodgerBlue3}“परिग्र”}होऽभिष्व{\color{DodgerBlue3}“ङ्गो द्वेषः”} परित्यागस्तौ भवतः (२२१) । {\color{DodgerBlue3}“अनयो”}रनुनयप्रतिषेधयोः {\color{DodgerBlue3}“संप्रतिबद्धाः सर्व्वे दोषा”} रागमात्सर्येर्ष्यादयः {\color{DodgerBlue3}“प्रजायन्ते”} ।
	\pend
      

	  \pstart e. यद्यप्यात्मनि स्नेहवान् तथाप्यत्मीये {\color{DodgerBlue3}“सुखसाधने वैराग्यान्न संसरतीति चेत्”} । नैतद्युक्तं यत (ः।)
	\pend
      
	  \bigskip
	  \begingroup
	  \large
	
	    
	    \stanza[\smallbreak]
	\label{pv.1.222b}\edlabel{pv.1.222b}\flagstanza{\tiny\textenglish{...1.222b}}नियमेनात्मनि स्निह्यंस्तदीये न विरज्यते ॥ २२२ ॥\&[\smallbreak]


	
	  \endgroup
	

	  \pstart {\color{DodgerBlue3}“आत्मनि स्निह्यन्”} प्रीयमाण{\color{DodgerBlue3}“स्तदीय आत्मी”}ये {\color{DodgerBlue3}“सुखसा”}धने {\color{DodgerBlue3}“नियमेन”} ({\color{DodgerBlue3}“न”}) {\color{DodgerBlue3}“विरज्यते”}ऽभिष्वजत्येव तत्कथमात्मीयविरागान्मुक्तिः ॥ आत्मस्नेहस्यात्मीयवैराग्यविरोधित्वात् । (२२२)
	\pend
      \label{div_pvv.1.223}\edlabel{div_pvv.1.223}
	  
	% new div opening: depth here is 2
	

	  \pstart f. तमे\edlabel{pvv.87-2}\footnote{\label{pvv.87-2}  २ आत्मस्नेहं ।}व त्यजतीति चेत् आह (।)
	\pend
      
	  \bigskip
	  \begingroup
	  \large
	
	    
	    \stanza[\smallbreak]
	\label{pv.1.223}\edlabel{pv.1.223}\flagstanza{\tiny\textenglish{....1.223}}ने चास्त्यात्मनि निर्दोषे स्नेहापगमकारणम्\edlabel{pvv.87-3}\footnote{\label{pvv.87-3}  ३ नैतच्छ्लोकार्द्ध विवृतं वृत्तिकृता ।} ॥&स्नेहः सदोष इति चेत् ततः किं तस्य वर्जनम् ॥ २२३ ॥\&[\smallbreak]


	
	  \endgroup
	\leavevmode\marginnote{\textenglish{088/s}}

	  \pstart य\edlabel{pvv.88-1}\footnote{\label{pvv.88-1}  १ “नचात्मनि निर्दोषे स्नेंहापगमकारणमस्ति” इतिवक्तुमुचितं व्याख्यानम् ।}द्यप्यात्मा निर्दोषस्तथापि {\color{DodgerBlue3}“स्नेहः सदोष इति चेत् । ततः”} सदोषत्वात् {\color{DodgerBlue3}“किं कर्तव्यं”} तस्य स्नेहस्य {\color{DodgerBlue3}“वर्जनम्”} (। २२३)
	\pend
      \label{div_pvv.1.224_1.225_1.226}\edlabel{div_pvv.1.224_1.225_1.226}
	  
	% new div opening: depth here is 2
	
	  \bigskip
	  \begingroup
	  \large
	
	    
	    \stanza[\smallbreak]
	\label{pv.1.224a}\edlabel{pv.1.224a}\flagstanza{\tiny\textenglish{...1.224a}}अदुषितेऽस्य विषये न शक्यं तस्य वर्जनम् ।\&[\smallbreak]


	
	  \endgroup
	

	  \pstart {\color{DodgerBlue3}“अदूषितेस्य विषय”} आत्मनि {\color{DodgerBlue3}“न शक्यं तस्य वर्जनं”} । न हि स्नेहः स्वगुणदोषा\leavevmode\marginnote{\textenglish{17a/MA}} भ्यामुपादीयते त्यज्यते वा किन्तु विषयस्य विषयश्च निर्दोष इति कथमस्य व\edlabel{pvv.88-2}\footnote{\label{pvv.88-2}  २ इदमेव समर्थयते ।}र्जनं ।
	\pend
      

	  \pstart {\color{DodgerBlue3}“किञ्च (।)”}
	\pend
      
	  \bigskip
	  \begingroup
	  \large
	
	    
	    \stanza[\smallbreak]
	\label{pv.1.224b}\edlabel{pv.1.224b}\flagstanza{\tiny\textenglish{...1.224b}}प्रहाणिरिच्छाद्वेषादेर्गुणदोषानुबन्धिनः ॥ २२४ ॥\&[\smallbreak]


	
	  \endgroup
	
	  \bigskip
	  \begingroup
	  \large
	
	    
	    \stanza[\smallbreak]
	\label{pv.1.225a}\edlabel{pv.1.225a}\flagstanza{\tiny\textenglish{...1.225a}}तयोरदृष्टिर्विषये;\&[\smallbreak]


	
	  \endgroup
	

	  \pstart {\color{DodgerBlue3}“इच्छाद्वेषादेर्गुणदोषानुबन्धिनो”} यथाक्रमं विषयस्य गुणदोषानुवर्तिनः प्रहाणिः प्रहाण्युपायः (२२४) । तयोर्गुणदोषयोर{\color{DodgerBlue3}“दृष्टिर्विष\edlabel{pvv.88-3}\footnote{\label{pvv.88-3}  ३ इच्छाविषये गुणदृष्टेः प्रवर्ततो द्वेषो दोषदृष्टेः तेनात्मदृष्ट्याऽगतो भिष्वङ्गस्त(द्)दृष्टेरपैति ।}ये”}ऽनन्योपायतादर्शनार्थमुपचारः । विषयगुणदोषादर्शने एवेच्छा द्वेषादि{\color{DodgerBlue3}“प्रहाण्यु”}पाय इत्यर्थः ॥
	\pend
      

	  \pstart नन्वदृष्टोपि इत्यादिरनिच्छामात्रात् त्यज्यमानो दृश्यत\edlabel{pvv.88-4}\footnote{\label{pvv.88-4}  ४ तत्किमात्मनि तद्विषयेऽवश्यन्तयाऽदर्शनमपेक्षते ।} इत्याह (।)
	\pend
      
	  \bigskip
	  \begingroup
	  \large
	
	    
	    \stanza[\smallbreak]
	\label{pv.1.225b}\edlabel{pv.1.225b}\flagstanza{\tiny\textenglish{...1.225b}}न तु बाह्येषु यः क्रमः ।\&[\smallbreak]


	
	  \endgroup
	

	  \pstart न तु बाह्येषु वस्तुषु यः क्रमोऽनिच्छामात्रकृतत्यागरूपः स आन्तरेष्वपि स्नेहादिषु युक्तः । बाह्याधीनं बाह्यमनिच्छया त्यक्तुं शक्यं आत्मदर्शनाधीनन्तु न शक्यपरिहारं । अविकलहेतुत्वेनोत्पत्तेः ।
	\pend
      

	  \pstart किञ्च (।)
	\pend
      
	  \bigskip
	  \begingroup
	  \large
	
	    
	    \stanza[\smallbreak]
	\label{pv.1.225c}\edlabel{pv.1.225c}\flagstanza{\tiny\textenglish{...1.225c}}न हि स्नेहगुणात् स्नेहः किन्त्वर्थगुणदर्शनात् ॥ २२५ ॥\&[\smallbreak]


	
	  \endgroup
	
	  \bigskip
	  \begingroup
	  \large
	
	    
	    \stanza[\smallbreak]
	\label{pv.1.226a}\edlabel{pv.1.226a}\flagstanza{\tiny\textenglish{...1.226a}}कारणेऽविकले तस्मिन् कार्यं केन निवार्यते ।\&[\smallbreak]


	
	  \endgroup
	

	  \pstart न हि {\color{DodgerBlue3}“स्नेहगुणात्स्नेहः क्रियते किन्त्वर्थ”}स्य विषयस्य {\color{DodgerBlue3}“गुणदोषदर्शनात्”} जायते (२२५) {\color{DodgerBlue3}“कारणेऽविकले तस्मिन्”} विषयगुणे {\color{DodgerBlue3}“कार्यं”} स्नेहः {\color{DodgerBlue3}“केन निवार्यते”} । न केनचित् ।
	\pend
      
	  \bigskip
	  \begingroup
	  \large
	
	    
	    \stanza[\smallbreak]
	\label{pv.1.226b}\edlabel{pv.1.226b}\flagstanza{\tiny\textenglish{...1.226b}}का वा सदोषता दृष्टा स्नेहे दुःखसमाश्रयः ॥ २२६ ॥\&[\smallbreak]


	
	  \endgroup
	\leavevmode\marginnote{\textenglish{089/s}}

	  \pstart {\color{DodgerBlue3}“का वा सदोषता दृष्टा स्नेहे”} येनायं वर्जयितव्यः ॥ {\color{DodgerBlue3}“दुःखस्य समाश्रयश्चेद्दोषः”} । तथा ह्यात्मनि स्निह्यन् तत्सुखसाधनेषु तृष्णावान् दुःखभूतं संसारमुपादत्ते ॥ (२२६)
	\pend
      \label{div_pvv.1.227}\edlabel{div_pvv.1.227}
	  
	% new div opening: depth here is 2
	
	  \bigskip
	  \begingroup
	  \large
	
	    
	    \stanza[\smallbreak]
	\label{pv.1.227a}\edlabel{pv.1.227a}\flagstanza{\tiny\textenglish{...1.227a}}तथापि न विरागोऽत्र स्वत्वदृष्टेर्यथात्मनि ।\&[\smallbreak]


	
	  \endgroup
	

	  \pstart {\color{DodgerBlue3}“तथापि”} दुःखहेतुत्वेपि {\color{DodgerBlue3}“न विरोगोऽत्र”} स्नेहे {\color{DodgerBlue3}“यथा”}त्मनि स्वत्व{\color{DodgerBlue3}“दृष्टेः”} दुःखनिदानभूताया हेता{\color{DodgerBlue3}“वात्मनि”} न विरागः ।
	\pend
      

	  \pstart यदि दुःखहेतौ विरागस्तथा स्वत्वे दृष्टिद्वारेण सर्व्वं दुःखमिति तस्य हेतावात्मन्येव स युक्तः । न चास्त्येतत् ।
	\pend
      
	  \bigskip
	  \begingroup
	  \large
	
	    
	    \stanza[\smallbreak]
	\label{pv.1.227b}\edlabel{pv.1.227b}\flagstanza{\tiny\textenglish{...1.227b}}न तैर्विना दुःखहेतुरात्मा चेत् तेऽपि तादृशाः ॥ २२७ ॥\&[\smallbreak]


	
	  \endgroup
	

	  \pstart {\color{DodgerBlue3}“न तैः स्नेह”}बुद्धीन्द्रियादिभिरात्मीयै{\color{DodgerBlue3}“र्व्विना दुःखहेतु”}रात्मा चेत् । {\color{DodgerBlue3}“तेपि”} स्नेहादयः {\color{DodgerBlue3}“तादृशा”} आत्मानमन्तरेण न दुःखहेतवः । (२२७)
	\pend
      \label{div_pvv.1.228}\edlabel{div_pvv.1.228}
	  
	% new div opening: depth here is 2
	
	  \bigskip
	  \begingroup
	  \large
	
	    
	    \stanza[\smallbreak]
	\label{pv.1.228}\edlabel{pv.1.228}\flagstanza{\tiny\textenglish{....1.228}}निर्दोषं द्वयमप्येवं वैराग्यान्न द्वयोस्ततः ॥&दुःखभावनया स्याच्चेदहिदष्टाङ्गहानिवत् ॥ २२८ ॥\&[\smallbreak]


	
	  \endgroup
	

	  \pstart {\color{DodgerBlue3}“एवं”} परस्परसापेक्षत्वे {\color{DodgerBlue3}“निर्दोषं द्वयमपि”} स्नेहादिरात्मा च । {\color{DodgerBlue3}“वैराग्यन्न द्वयो”}रपि {\color{DodgerBlue3}“ततः”} का\edlabel{pvv.89-1}\footnote{\label{pvv.89-1}  १ करणीयं ।}र्यं । तथा च संसारो दोषाश्च दुर्व्वारा हेतुसाकल्यात् ॥ प्रवृत्तिः यदि स्नेहादिषु {\color{DodgerBlue3}“दुःखभावनया”} हानिः {\color{DodgerBlue3}“स्यात् । अहिदष्टस्याङ्गस्य हानिवत्”} । यथात्मीयमप्यहिदष्टमङ्गं दुःखवशाद् विरज्य त्यज्यते । अनुपभोगाश्रयत्वात् । (२२८)
	\pend
      \label{div_pvv.1.229_1.230_1.231}\edlabel{div_pvv.1.229_1.230_1.231}
	  
	% new div opening: depth here is 2
	
	  \bigskip
	  \begingroup
	  \large
	
	    
	    \stanza[\smallbreak]
	\label{pv.1.229}\edlabel{pv.1.229}\flagstanza{\tiny\textenglish{....1.229}}आत्मीयबुद्धिहान्याऽत्र त्यागो न तु वियर्यये ॥&उपभोगाश्रयत्वेन गृहीतेष्विन्द्रियादिषु ॥ २२९ ॥\&[\smallbreak]


	
	  \endgroup
	
	  \bigskip
	  \begingroup
	  \large
	
	    
	    \stanza[\smallbreak]
	\label{pv.1.230a}\edlabel{pv.1.230a}\flagstanza{\tiny\textenglish{...1.230a}}स्वत्वधीः केन वार्येत वैराग्यं तत्र तत् कुतः ॥\&[\smallbreak]


	
	  \endgroup
	

	  \pstart {\color{DodgerBlue3}“आत्मीयबुद्धिहान्या”} । तत्राहिदष्टाङ्गे {\color{DodgerBlue3}“त्यागो न तु विपर्यये”} आत्मीयबुद्धिसत्तायां । यस्मादु{\color{DodgerBlue3}“पभोग”}स्या{\color{DodgerBlue3}“श्रयत्वेन”} कारणत्वेन {\color{DodgerBlue3}“गृहीतेष्विन्द्रियादिषु”} (२२९) {\color{DodgerBlue3}“स्वत्वे धीरा”}त्मीयत्वबुद्धिः {\color{DodgerBlue3}“के\edlabel{pvv.89-2}\footnote{\label{pvv.89-2}  २ तस्मात्सा ।}न”} हेतुना {\color{DodgerBlue3}“वार्येत न”} केनचित् । तत्कुत{\color{DodgerBlue3}“स्तत्रो”}पभोगसाधने स्वीयावयवे {\color{DodgerBlue3}“वैराग्यं”} येन त्यज्यते । {\color{DodgerBlue3}“ततो”} यत्त्यज्यते {\color{DodgerBlue3}“आत्मीयबुद्धिहान्या”} एव । न चैवं स्नेहादिष्वात्मीयबुद्धिहानिरस्ति येनैषां त्यागः स्यात् ।
	\pend
      \leavevmode\marginnote{\textenglish{090/s}}
	  \bigskip
	  \begingroup
	  \large
	
	    
	    \stanza[\smallbreak]
	\label{pv.1.230b}\edlabel{pv.1.230b}\flagstanza{\tiny\textenglish{...1.230b}}प्रत्यक्षमेव सर्वस्य केशादिषु कलेवरात् ॥ २३० ॥\&[\smallbreak]


	
	  \endgroup
	
	  \bigskip
	  \begingroup
	  \large
	
	    
	    \stanza[\smallbreak]
	\label{pv.1.231a}\edlabel{pv.1.231a}\flagstanza{\tiny\textenglish{...1.231a}}च्युतेषु सधृणा बुद्धिर्जायतेऽन्येषु सस्पृहा ।*\&[\smallbreak]


	
	  \endgroup
	

	  \pstart एतच्च {\color{DodgerBlue3}“प्रत्यक्षमेव सर्व्वस्य केशादिषु कलेवरात्”} । (२३० ) {\color{DodgerBlue3}“च्युतेष्वा”}त्मीयबुद्धिविषयेषु {\color{DodgerBlue3}“सधृणा”} बुद्धिर्जायते । जनस्यान्यत्राच्युतेष्वात्मीयबुद्धिविषयेषु सस्पृहा ।
	\pend
      

	  \begin{center}%% label @type='head'
	\textbf{(ङ) आत्मात्मीयबुद्ध्योर्हानिः}
	\end{center}
	

	  \pstart दुःखभावनया आत्मीयबुद्धिरेव हीयत इति चेत् । न युक्तमिदं (।)
	\pend
      
	  \bigskip
	  \begingroup
	  \large
	
	    
	    \stanza[\smallbreak]
	\label{pv.1.231b}\edlabel{pv.1.231b}\flagstanza{\tiny\textenglish{...1.231b}}समवायादिसम्बन्धजनिता तत्र हि स्वधीः ॥ २३१ ॥\&[\smallbreak]


	
	  \endgroup
	

	  \pstart {\color{DodgerBlue3}“समवायादिसम्बन्धजनिता”} हि यस्मात्तत्र बुद्ध्यादौ {\color{DodgerBlue3}“स्वधीः”} । तथा चात्मनः सुखादीना समवायः सम्बन्धः । शरीरेण संयोगः । शरीराश्रितैः रूपादिभिः संयुक्तसमवायः । श्रोत्रेन्द्रियेण संयोगश्चक्षुरादिभिः संयोगिसंयोग आत्मसम्बन्धः । (२३१)
	\pend
      \label{div_pvv.1.232}\edlabel{div_pvv.1.232}
	  
	% new div opening: depth here is 2
	
	  \bigskip
	  \begingroup
	  \large
	
	    
	    \stanza[\smallbreak]
	\label{pv.1.232}\edlabel{pv.1.232}\flagstanza{\tiny\textenglish{....1.232}}स तथैवेति सा दोषदृष्टावपि न हीयते ।&समवायाद्यभावेऽपि सर्वत्रास्त्युपकारिता ॥ २३२ ॥\&[\smallbreak]


	
	  \endgroup
	

	  \pstart {\color{DodgerBlue3}“स”} दुःखभावनायामपि {\color{DodgerBlue3}“तथैवेति सा”} स्वधी{\color{DodgerBlue3}“र्दोषदृष्टावपि न हीयते”} निमित्तस्यावैकल्यात् । अथ स\edlabel{pvv.90-1}\footnote{\label{pvv.90-1}  १ सांख्यं प्रत्याह (।) प्रकृतिपुरुषान्तरज्ञानान्मुक्तेः ।}मवायादिर्नास्त्येव तदा । {\color{DodgerBlue3}“समवायाद्यभावेऽपि सर्व्वत्र”} बुद्ध्यादाव{\color{DodgerBlue3}“स्त्युपकारिता”} तत्कृता । (२३२)
	\pend
      \label{div_pvv.1.233}\edlabel{div_pvv.1.233}
	  
	% new div opening: depth here is 2
	
	  \bigskip
	  \begingroup
	  \large
	
	    
	    \stanza[\smallbreak]
	\label{pv.1.233}\edlabel{pv.1.233}\flagstanza{\tiny\textenglish{....1.233}}दुःखोपकारान्न भवेदङ्‏गुल्यामिव चेत् स्वधीः ।&न ह्येकान्तेन तद् दुखं भूयसा सविषान्नवत् ॥ २३३ ॥\&[\smallbreak]


	
	  \endgroup
	

	  \pstart तत्र स्वधीरश\edlabel{pvv.90-2}\footnote{\label{pvv.90-2}  २ उपभोगाङ्गत्वेन वृत्तेः ।}क्यधारणा {\color{DodgerBlue3}“दुःखोपकारात्”} दुःखोपनिधाना\edlabel{pvv.90-3}\footnote{\label{pvv.90-3}  ३ उपसमीपे निधानमर्पणन्ततः}त् {\color{DodgerBlue3}“न भवे”}दहिदष्टाया{\color{DodgerBlue3}“मङ्गुल्यामिव”} स्नेहबुद्ध्यादौ {\color{DodgerBlue3}“स्वधी”}रिति {\color{DodgerBlue3}“चेत् । न ह्येकान्तेन”} तत्स्नेहादि{\color{DodgerBlue3}“दुःखं”} दुःखहेतुः पर्यायेण सुखहेतुत्वादपि । किन्तु {\color{DodgerBlue3}“भूयसा”} तद्दुःखं {\color{DodgerBlue3}“सवि\edlabel{pvv.90-4}\footnote{\label{pvv.90-4}  ४ यथात्मा न त्यज्यते तथायमपि ।}षान्नवत्”} । परिणतिदुःखहेतुरपि विषान्नमापातसुखं च । (२३३)
	\pend
      \label{div_pvv.1.234}\edlabel{div_pvv.1.234}
	  
	% new div opening: depth here is 2
	

	  \pstart a. ननु सविषमन्नं सुखमिश्रञ्च वैराग्यविषयः स्वहितकामानामेवं स्नेहादिरपि स्यादित्याह (।)
	\pend
      
	  \bigskip
	  \begingroup
	  \large
	
	    
	    \stanza[\smallbreak]
	\label{pv.1.234a}\edlabel{pv.1.234a}\flagstanza{\tiny\textenglish{...1.234a}}विशिष्टसुखसङ्गात् स्यात् तद्विरुद्धे विरागिता ।\&[\smallbreak]


	
	  \endgroup
	

	  \pstart {\color{DodgerBlue3}“विशिष्टे”} सुखे सुखसाधने तदात्वपरिणाभयोरनुग्रहीतरि विषादिदोषरहितभोजनादौ {\color{DodgerBlue3}“सङ्गाद”}भिष्वङ्गात् {\color{DodgerBlue3}“स्यात्तद्विरुद्धे”} सविषान्नादौ {\color{DodgerBlue3}“विरागिता”} ।
	\pend
      \leavevmode\marginnote{\textenglish{091/s}}

	  \pstart वैराग्याच्च (।)
	\pend
      
	  \bigskip
	  \begingroup
	  \large
	
	    
	    \stanza[\smallbreak]
	\label{pv.1.234b}\edlabel{pv.1.234b}\flagstanza{\tiny\textenglish{...1.234b}}किञ्चित् परित्यजेत् सौख्यं विशिष्टसुखतृष्णया ॥ २३४ ॥\&[\smallbreak]


	
	  \endgroup
	

	  \pstart {\color{DodgerBlue3}“किञ्चित्सौख्यं”} परिणतिदुःखबहुलं {\color{DodgerBlue3}“परित्यजे”}दात्मकामो {\color{DodgerBlue3}“विशिष्टस्य सुखस्य”} परिणामाविरुद्धस्य {\color{DodgerBlue3}“तृष्णया”} अभिलाषेण । यदा त्वात्मनि सति न किञ्चित्सुखैकरूपं सर्व्वं सुखं दुःखसंभिश्रं तदा क्व परिहारः कस्मिन्ननुरागः स्वीकारो न चेच्छया\leavevmode\marginnote{\textenglish{17b/MA}} शक्यपरिहाराः स्नेहादयः तत्कारणस्यात्मनोऽवैकल्यादित्युक्तं ॥ (२३४)
	\pend
      \label{div_pvv.1.235}\edlabel{div_pvv.1.235}
	  
	% new div opening: depth here is 2
	

	  \pstart b. नन्व\edlabel{pvv.91-1}\footnote{\label{pvv.91-1}  १ यदि नैरात्म्यमेव सर्व्वधर्म्माणां वस्तुतस्तदाऽप्रवृत्तिस्तत्कार्य स्यान्न चास्तीत्याह । नैराश्ये नैरात्म्ये तत्त्वे मूलव्याख्या, स्यादेतद्यदि नैरात्म्यज्ञानं स्यात् किन्त्वज्ञानादारोपात् प्रवर्तते ।}नैरात्म्यपक्षेपि दुःखं सकलं । {\color{DodgerBlue3}“तन्निरोधः परसुखं । तत्क्वचित्तृष्णया”} प्रवृत्तिर्न स्यादित्याह (।)
	\pend
      
	  \bigskip
	  \begingroup
	  \large
	
	    
	    \stanza[\smallbreak]
	\label{pv.1.235}\edlabel{pv.1.235}\flagstanza{\tiny\textenglish{....1.235}}नैरात्म्ये तु यथालाभमात्मस्नेहात् प्रवर्तते ।&अलाभे मत्तकासिन्या दृष्टा तिर्यक्षु कामिता ॥ २३५ ॥\&[\smallbreak]


	
	  \endgroup
	

	  \pstart {\color{DodgerBlue3}“नैरात्म्यत”}त्त्वेऽधायोपादा\edlabel{pvv.91-2}\footnote{\label{pvv.91-2}  २ आत्मारोपात् ।} {\color{DodgerBlue3}“त्मस्नेहा”}त्सुखविपर्यासात् दुःखेष्वपि सुखतयाऽध्यवसितेषु विषयेषु यथालाभं प्राप्त्यनुक्रमेण प्रवृत्तिर्भवति निर्व्वाणसुखव्युत्पत्त्यभावात् । व्युत्पत्तावपि तन्मार्गसात्मत्वाभावात् । सात्मीकृतमार्ग्गास्तु क्वचिन्न प्रवर्तन्त एव । तथा {\color{DodgerBlue3}“चालाभे मत्तकासिन्या”} मत्तगजगामिन्या कामुकस्य {\color{DodgerBlue3}“तिर्यक्षु कामिता दृष्टा”} । बलवानात्मस्नेहो विशिष्टसुखसाधनस्यालाभे सुखाभासहेतौ च प्रवर्तयति । (२३५)
	\pend
      \label{div_pvv.1.236_1.237}\edlabel{div_pvv.1.236_1.237}
	  
	% new div opening: depth here is 2
	

	  \pstart c. किञ्च (।) यदि बुद्धीन्द्रियशरीरादीनां दुःखहेतुत्वात् तेषु वैराग्यात्तत्त्यागात् कैवल्यमा{\color{DodgerBlue3}“त्मन इष्टं”} । मुक्तिदशायान्तदात्मभोगादिसकलपरिच्छेदाभावात् नाशाविशेषान्नाश एवेष्टः स्यात् (।) तच्चेदमयुक्तमित्याह (।)
	\pend
      
	  \bigskip
	  \begingroup
	  \large
	
	    
	    \stanza[\smallbreak]
	\label{pv.1.236a}\edlabel{pv.1.236a}\flagstanza{\tiny\textenglish{...1.236a}}यस्यात्मा वल्लभस्तस्य स नाशं कथमिच्छति ।\&[\smallbreak]


	
	  \endgroup
	

	  \pstart {\color{DodgerBlue3}“य\edlabel{pvv.91-3}\footnote{\label{pvv.91-3}  ३ सांख्यस्य ।}स्यात्मा वल्लभ\edlabel{pvv.91-4}\footnote{\label{pvv.91-4}  ४ देही एव मुक्तः ।}स्तस्य स कथं नाशमिच्छति”} न नाशं {\color{DodgerBlue3}“कैवल्यमप्रतीति”}विषयत्वादिच्छति ।
	\pend
      

	  \pstart कथं पुनः केवलमात्मानमिच्छन् नाशमिच्छतीत्याह (।)
	\pend
      
	  \bigskip
	  \begingroup
	  \large
	
	    
	    \stanza[\smallbreak]
	\label{pv.1.236b}\edlabel{pv.1.236b}\flagstanza{\tiny\textenglish{...1.236b}}निवृत्तसर्वानुभवव्यवहारगुणश्रयम् ॥ २३६ ॥\&[\smallbreak]


	
	  \endgroup
	
	  \bigskip
	  \begingroup
	  \large
	
	    
	    \stanza[\smallbreak]
	\label{pv.1.237a}\edlabel{pv.1.237a}\flagstanza{\tiny\textenglish{...1.237a}}इच्छेत् प्रेम कथं;\&[\smallbreak]


	
	  \endgroup
	\leavevmode\marginnote{\textenglish{092/s}}

	  \pstart {\color{DodgerBlue3}“निवृत्तः सर्व्व”}स्याव{\color{DodgerBlue3}“नुभवव्यव\edlabel{pvv.92-1}\footnote{\label{pvv.92-1}  १ सुखादिभोक्तेत्यादि ।}हार”}स्य समाश्रय आश्रयणं यस्मात्तं नाशलक्षणाविशिष्टमित्यर्थः । नाशा\edlabel{pvv.92-2}\footnote{\label{pvv.92-2}  २ नैरात्म्ये विशेषात् ।}विशिष्टञ्चात्मानं {\color{DodgerBlue3}“कथं प्रेम”} स्नेहातिशय {\color{DodgerBlue3}“इच्छेत्”} । {\color{DodgerBlue3}“यस्मात्”} (।)
	\pend
      
	  \bigskip
	  \begingroup
	  \large
	
	    
	    \stanza[\smallbreak]
	\label{pv.1.237b}\edlabel{pv.1.237b}\flagstanza{\tiny\textenglish{...1.237b}}प्रेम्णः प्रकृतिर्न हि तादृशी ।\&[\smallbreak]


	
	  \endgroup
	

	  \pstart {\color{DodgerBlue3}“प्रेम्णः प्रकृतिस्तादृशी”} स्वविषयनाशैषणस्वभावा न भवति (।) तस्मात् (।)
	\pend
      
	  \bigskip
	  \begingroup
	  \large
	
	    
	    \stanza[\smallbreak]
	\label{pv.1.237c}\edlabel{pv.1.237c}\flagstanza{\tiny\textenglish{...1.237c}}सर्वथात्मग्रहः स्नेहमात्मनि द्रढयत्यलम् ॥ २३७ ॥\&[\smallbreak]


	
	  \endgroup
	

	  \pstart {\color{DodgerBlue3}“सर्व्वथा आत्मग्रह आत्मनि स्नेहमलम”}त्यर्थ {\color{DodgerBlue3}“द्रढयति”} (। २३७)
	\pend
      \label{div_pvv.1.238_1.239}\edlabel{div_pvv.1.238_1.239}
	  
	% new div opening: depth here is 2
	

	  \pstart स आत्मग्रहश्चात्मोपकारिषु ।
	\pend
      
	  \bigskip
	  \begingroup
	  \large
	
	    
	    \stanza[\smallbreak]
	\label{pv.1.238a}\edlabel{pv.1.238a}\flagstanza{\tiny\textenglish{...1.238a}}आत्मीयस्नेहबीजन्तु तदवस्थं व्यवस्थितम् ।\&[\smallbreak]


	
	  \endgroup
	

	  \pstart {\color{DodgerBlue3}“आत्मीयस्नेहबीजं तदवस्थं व्यवस्थित”}मिति तत्प्रतिबद्धानां दोषाणाञ्चानिवृत्तिः ॥
	\pend
      

	  \pstart d. स्यादेतद् (।) आत्मीये दोषदर्शनाद्वैराग्यमुत्पद्यते इत्याह (।)
	\pend
      
	  \bigskip
	  \begingroup
	  \large
	
	    
	    \stanza[\smallbreak]
	\label{pv.1.238b}\edlabel{pv.1.238b}\flagstanza{\tiny\textenglish{...1.238b}}यत्नेऽप्यात्मीयवैराग्यं गुणलेशसमाश्रयात् ॥ २३८ ॥\&[\smallbreak]


	
	  \endgroup
	
	  \bigskip
	  \begingroup
	  \large
	
	    
	    \stanza[\smallbreak]
	\label{pv.1.239a}\edlabel{pv.1.239a}\flagstanza{\tiny\textenglish{...1.239a}}वृत्तिमान् प्रतिबध्नाति, तद्दोषान् संवृणोति च ।\&[\smallbreak]


	
	  \endgroup
	

	  \pstart दोषदर्शनात् {\color{DodgerBlue3}“यत्नेपि”} सति तावत्कालमा{\color{DodgerBlue3}“त्मीये”}षु {\color{DodgerBlue3}“वैराग्यं”} यदुत्पन्नं तदात्मस्नेहो {\color{DodgerBlue3}“वृत्तिमाना”}त्मीयेषु {\color{DodgerBlue3}“गुणले”}शस्य सुखसाधनत्वस्य {\color{DodgerBlue3}“समाश्रय”}णात् (२३८) {\color{DodgerBlue3}“प्रतिबध्नाति । तद्दोषाँश्च”} दुःखसाधनादीन् {\color{DodgerBlue3}“संवृणोति”} । तत् कुत आत्मस्नेहवत आत्मीये वैराग्ययोगः । आत्मस्नेहस्य सर्व्वदोषमूलत्वात् (।)
	\pend
      
	  \bigskip
	  \begingroup
	  \large
	
	    
	    \stanza[\smallbreak]
	\label{pv.1.239b}\edlabel{pv.1.239b}\flagstanza{\tiny\textenglish{...1.239b}}आत्मन्यपि विरागश्चेदिदानीं यो विरज्यते ॥ २३९ ॥\&[\smallbreak]


	
	  \endgroup
	

	  \pstart {\color{DodgerBlue3}“आत्मन्यपि विरागश्चे”}द्वाध्यते । ननूक्तमत्र । न चात्मनि निर्दोषे स्नेहापगमकारणमस्ति । भवतु तावत्तथापीदानीमत्रापि पक्षे यो तत्र {\color{DodgerBlue3}“विरज्यते”} (। २३९)
	\pend
      \label{div_pvv.1.240}\edlabel{div_pvv.1.240}
	  
	% new div opening: depth here is 2
	
	  \bigskip
	  \begingroup
	  \large
	
	    
	    \stanza[\smallbreak]
	\label{pv.1.240}\edlabel{pv.1.240}\flagstanza{\tiny\textenglish{....1.240}}त्यजत्यसौ यथात्मानं व्यर्थाऽतो दुःखभावना ।&दुःखभावनयाऽप्येष दुःखमेव विभावयेत् ॥ २४० ॥\&[\smallbreak]


	
	  \endgroup
	

	  \pstart तेन स तं {\color{DodgerBlue3}“त्यजति यथात्मानं”} । न ह्यात्मनि विरक्तोपि तं त्यजति, तथात्मीयेपि विरक्तस्तं न त्यक्ष्यतीति {\color{DodgerBlue3}“व्यर्थ[ा]तो दुःखभाव”}नाऽत्मनोऽत्यागात् । {\color{DodgerBlue3}“दुःख-”} \leavevmode\marginnote{\textenglish{093/s}} {\color{DodgerBlue3}“भावनयापि”} एष भावको {\color{DodgerBlue3}“दुःखमेव”} भाव्यमानं {\color{DodgerBlue3}“विभावयेत्”} प्रका\edlabel{pvv.93-1}\footnote{\label{pvv.93-1}  १ प्रत्यक्षावसानत्वाद् भावनायाः ।}शयेत् । (२४०)
	\pend
      \label{div_pvv.1.241}\edlabel{div_pvv.1.241}
	  
	% new div opening: depth here is 2
	
	  \bigskip
	  \begingroup
	  \large
	
	    
	    \stanza[\smallbreak]
	\label{pv.1.241}\edlabel{pv.1.241}\flagstanza{\tiny\textenglish{....1.241}}प्रत्यक्षं पूर्वमपि तत् तथापि न विरागवान् ॥&यद्यप्येकत्र दोषेण तत्क्षणं चलिता मतिः ॥ २४१ ॥\&[\smallbreak]


	
	  \endgroup
	

	  \pstart तच्च भावनातः {\color{DodgerBlue3}“पूर्व्वमपि प्रत्यक्ष”}मेव दुःखमात्म\edlabel{pvv.93-2}\footnote{\label{pvv.93-2}  २ इति व्यर्था भावना ।}स्नेहादिशस्त्रप्रहाराद्यनुभवकाले {\color{DodgerBlue3}“तथापि”} प्रत्यक्षीकृतात्मस्नेहादिदुःखत्वे{\color{DodgerBlue3}“पि”} न {\color{DodgerBlue3}“विरागवान्”} तेषु कश्र्चित्तदा । ततो भावनाप्रकर्षेप्येवं स्यात् साक्षात्करणत्वात्तस्य । {\color{DodgerBlue3}“तच्च न”} विरागहेतुः प्रागिव । यद्यप्येकत्रापराधकारिणि दोषदर्शनात् {\color{DodgerBlue3}“तत्क्षणं”} नियतकालमनुरागा{\color{DodgerBlue3}“च्चलिता मति-”} र्व्विरागभजनात् (। २४१)
	\pend
      \label{div_pvv.1.242}\edlabel{div_pvv.1.242}
	  
	% new div opening: depth here is 2
	
	  \bigskip
	  \begingroup
	  \large
	
	    
	    \stanza[\smallbreak]
	\label{pv.1.242}\edlabel{pv.1.242}\flagstanza{\tiny\textenglish{....1.242}}विरक्तो नैव तत्रापि कामीव वनितान्तरे ॥&त्याज्योपादेयभेदे हि सक्तिर्यैवैकभाविनी ॥ २४२ ॥\&[\smallbreak]


	
	  \endgroup
	

	  \pstart तथाप्यसौ {\color{DodgerBlue3}“तत्र विरक्तः”} सर्व्वथा पर्यायेण रागोत्पत्तेः । किं पुनरन्यत्र {\color{DodgerBlue3}“कामीव”} क्वचित् कामिन्यां रागकारिण्यां विरक्तोपि {\color{DodgerBlue3}“न वनितान्तरे”} विरक्तः । तस्यामपि वा क्रमेण ।
	\pend
      

	  \pstart किञ्च (।) प्रतिधानुनयविषयत्वात् {\color{DodgerBlue3}“त्याज्योपादेयभे\edlabel{pvv.93-3}\footnote{\label{pvv.93-3}  ३ आत्मनीत्यादिवदयमुपदेशः ।}दे हि सति सक्तिरासक्तिर्यैवे”}कस्मिन् {\color{DodgerBlue3}“भाविनी”} द्वेषविषयतयाऽनुरागविषयतया वा (। २४२)
	\pend
      \label{div_pvv.1.243}\edlabel{div_pvv.1.243}
	  
	% new div opening: depth here is 2
	
	  \bigskip
	  \begingroup
	  \large
	
	    
	    \stanza[\smallbreak]
	\label{pv.1.243}\edlabel{pv.1.243}\flagstanza{\tiny\textenglish{....1.243}}सा बीजं सर्वसक्तीनां पर्यायेण समुद्भवे ॥&निर्दोषविषयः स्नेहो निर्दोषः साधनानि च ॥ २४३ ॥\&[\smallbreak]


	
	  \endgroup
	

	  \pstart e. {\color{DodgerBlue3}“सा”} सक्ति{\color{DodgerBlue3}“र्बीजं”} कारणं {\color{DodgerBlue3}“सर्व्वसक्तीनां पर्यायेण”} परिपाट्‏या {\color{DodgerBlue3}“समुद्भव”}निमित्तं । तथा हि क्वचिद् द्वेषासक्त्या तदनुकूलप्रतिकूलयोः प्रतिधानुनयौ भवतः । तथानुरागासक्त्यापि क्वचित्तयोरेवानु{\color{DodgerBlue3}“नयद्वेषौ भवतः”} । तदेवममात्नो निर्दोषत्वा{\color{DodgerBlue3}“न्निर्दोषविषयः”} स्नेहो {\color{DodgerBlue3}“निर्दोषः”} स्वयं {\color{DodgerBlue3}“उपभोगसाधनानि”} चेन्द्रियशरीरादिनि शब्दरसरूपादीनि निर्दोषाणि । आत्मादीनां सर्व्वेषां प्रत्येकं दुःखहेतुत्वाभावात् ।\edlabel{pvv.93-4}\footnote{\label{pvv.93-4}  ४ निर्द्दोषस्यात्मनः सतः ।} (२४३)
	\pend
      \label{div_pvv.1.244}\edlabel{div_pvv.1.244}
	  
	% new div opening: depth here is 2
	
	  \bigskip
	  \begingroup
	  \large
	
	    
	    \stanza[\smallbreak]
	\label{pv.1.244}\edlabel{pv.1.244}\flagstanza{\tiny\textenglish{....1.244}}एतावदेव च जगत् क्वेदानीं स विरज्यते ॥&सदोषताऽपि चेत् तस्य तत्रात्मन्यपि सा समा ॥ २४४ ॥\&[\smallbreak]


	
	  \endgroup
	\leavevmode\marginnote{\textenglish{094/s}}

	  \pstart {\color{DodgerBlue3}“एता\edlabel{pvv.94-1}\footnote{\label{pvv.94-1}  १ आत्मस्नेहः साधनञ्च ।}वदेव च जगत्”} । त्रिभिर्जगतः संग्रहात् । {\color{DodgerBlue3}“क्वेदानीं स मो”}क्तुकामो {\color{DodgerBlue3}“विरज्यते”} ॥ समुदायाद्दोषदर्शनात् तत एव विरज्यते चेत् । नन्वेवमात्मन्यपि वैराग्यं प्राप्तं ॥ न केवलं गुणवत्ता {\color{DodgerBlue3}“सदोषतापि चे\edlabel{pvv.94-2}\footnote{\label{pvv.94-2}  २ दुःखाश्रयत्वात् ।}त्तस्य”} स्नेहेन्द्रियादेः । {\color{DodgerBlue3}“तत्रात्मन्यपि”} कैवल्येनेष्टे {\color{DodgerBlue3}“सा”} सदोषता {\color{DodgerBlue3}“समा”} तस्या अपि स्नेहादिदोषवत्त्वात् । (२४४)
	\pend
      \label{div_pvv.1.245}\edlabel{div_pvv.1.245}
	  
	% new div opening: depth here is 2
	\leavevmode\marginnote{\textenglish{18a/MA}}

	  \pstart f. एवन्तर्ह्यात्मदोषमेव वैराग्यभावनाया जह्यादिति चेत् ।
	\pend
      
	  \bigskip
	  \begingroup
	  \large
	
	    
	    \stanza[\smallbreak]
	\label{pv.1.245a}\edlabel{pv.1.245a}\flagstanza{\tiny\textenglish{...1.245a}}तत्राविरक्तस्तद्दोषे क्वेदानीं स विरज्यते ।\&[\smallbreak]


	
	  \endgroup
	

	  \pstart {\color{DodgerBlue3}“तत्रात्म”}न्य{\color{DodgerBlue3}“विरक्त”}स्त{\color{DodgerBlue3}“द्दोषे क्वेदा”}नीमात्मदर्शनकाले {\color{DodgerBlue3}“स”} मुमुक्षु{\color{DodgerBlue3}“र्विर”}ज्यते । यथा सदोषेप्यात्मन्यात्मदर्शनादविरक्तस्तथा तद्दोषेपि स्नेहादात्मीयत्वदर्शनान्न विरज्येत (।) अपि च (।)
	\pend
      
	  \bigskip
	  \begingroup
	  \large
	
	    
	    \stanza[\smallbreak]
	\label{pv.1.245b}\edlabel{pv.1.245b}\flagstanza{\tiny\textenglish{...1.245b}}गुणदर्शनसम्भूतं स्नेहं बाधितदोषदृक् ॥ २४५ ॥\&[\smallbreak]


	
	  \endgroup
	

	  \pstart {\color{DodgerBlue3}“गुणदर्शनसम्भूतं स्नेहं बार्धितदोष”}स्य {\color{DodgerBlue3}“दृक्”} दृष्टिः (२४५)
	\pend
      \label{div_pvv.1.246_1.247}\edlabel{div_pvv.1.246_1.247}
	  
	% new div opening: depth here is 2
	
	  \bigskip
	  \begingroup
	  \large
	
	    
	    \stanza[\smallbreak]
	\label{pv.1.246}\edlabel{pv.1.246}\flagstanza{\tiny\textenglish{....1.246}}स चेन्द्रियादौ न त्वेवं बालादेरपि सम्भवात् ।&दोषवत्यपि सद्भावात् स्वभावाद् गुणवत्यपि ॥ २४६ ॥\&[\smallbreak]


	
	  \endgroup
	
	  \bigskip
	  \begingroup
	  \large
	
	    
	    \stanza[\smallbreak]
	\label{pv.1.247a}\edlabel{pv.1.247a}\flagstanza{\tiny\textenglish{...1.247a}}अन्यत्र;\&[\smallbreak]


	
	  \endgroup
	

	  \pstart {\color{DodgerBlue3}“स च”} स्नेहजे{\color{DodgerBlue3}“न्द्रियादावे”}वं गुण\edlabel{pvv.94-3}\footnote{\label{pvv.94-3}  ३ आत्मीयबुद्धेरबाधकं दुःखदर्शनं कथमात्मीयस्नेहमपनुदेत् ।}दर्शनान्न दृष्टः । {\color{DodgerBlue3}“बालादेरपि”} गुणपरीक्षाऽभ्यु(पग)म्य चक्षुरादावात्मीयत्वमात्रेण स्नेहस्य {\color{DodgerBlue3}“सम्भवात्”} । गुणदोषदर्शनात् न स्नेहभावाभावी । किन्त्वात्मीयत्वदर्शनादर्शनात्तदन्वयव्यतिरेकानुविधानादि\edlabel{pvv.94-4}\footnote{\label{pvv.94-4}  ४ तदेवाह ।}ति । दर्शयति च\edlabel{pvv.94-5}\footnote{\label{pvv.94-5}  ५ शिष्यान् परतः पूर्व्वार्थसमर्थनमाचार्यः ।} स्वकीये चक्षुरादौ गुणविकले काणत्वादि{\color{DodgerBlue3}“दोषवत्यप्या”}त्मीयत्वपरामर्शात् स्नेहस्य {\color{DodgerBlue3}“सद्भावाद”}न्यत्र परकीये नेत्रादौ दोषरहिते {\color{DodgerBlue3}“गुणवत्य”}\edlabel{pvv.94-6}\footnote{\label{pvv.94-6}  ६ स्नेहाभावात् ।}पि (। २४६)
	\pend
      
	  \bigskip
	  \begingroup
	  \large
	
	    
	    \stanza[\smallbreak]
	\label{pv.1.247b}\edlabel{pv.1.247b}\flagstanza{\tiny\textenglish{...1.247b}}आत्मीयतायां वा व्यतीतादौ विहानितः ।&तत एव च नात्मीयबुद्धेरपि गुणेक्षणम् ॥ २४७ ॥\&[\smallbreak]


	
	  \endgroup
	
	  \bigskip
	  \begingroup
	  \large
	
	    
	    \stanza[\smallbreak]
	\label{pv.1.248a}\edlabel{pv.1.248a}\flagstanza{\tiny\textenglish{...1.248a}}कारणम्;\&[\smallbreak]


	
	  \endgroup
	

	  \pstart {\color{DodgerBlue3}“आत्मीयतायामपि वाऽतीतादौ”} केशनखादौ आदिशब्दाल्लूनाङ्गुल्यादौ वर्तमानेन स्नेह आत्मीयत्वेनादृष्टे{\color{DodgerBlue3}“र्व्विहानितः”} परित्यागात् स्वत्वस्य {\color{DodgerBlue3}“तत एव च”} बालादेरपि भावात् । {\color{DodgerBlue3}“आत्मीयबुद्धेरपि न गुणेक्षणं कारणं”} किन्त्वात्मदर्शनमेव (। २४७)
	\pend
      \label{div_pvv.1.248_1.249}\edlabel{div_pvv.1.248_1.249}
	  
	% new div opening: depth here is 2
	\leavevmode\marginnote{\textenglish{095/s}}
	  \bigskip
	  \begingroup
	  \large
	
	    
	    \stanza[\smallbreak]
	\label{pv.1.248b}\edlabel{pv.1.248b}\flagstanza{\tiny\textenglish{...1.248b}}हीयते साऽपि तस्मान्नागुणदर्शनात् ।\&[\smallbreak]


	
	  \endgroup
	

	  \pstart {\color{DodgerBlue3}“तस्माद् गुण”}दर्शनहेतुकत्वाभावात् सा आत्मीयबुद्धिरपि अगुणस्य {\color{DodgerBlue3}“दोषस्य दर्शना”}न्न हीयते । कारणविरुद्धो हि धर्मी निवर्तकः {\color{DodgerBlue3}“कस्यचिद्यथाग्नी रोमाञ्च”}विशेषस्य । आत्मदर्शनं हि स्नेहात्मीयदृशादेः कारणं न च तद्विरोधिनी {\color{DodgerBlue3}“दोषदृक्”} ।
	\pend
      
	  \bigskip
	  \begingroup
	  \large
	
	    
	    \stanza[\smallbreak]
	\label{pv.1.248c}\edlabel{pv.1.248c}\flagstanza{\tiny\textenglish{...1.248c}}अपि चासद्गुणारोपः स्नेहात् तत्र हि दृश्यते ॥ २४८ ॥\&[\smallbreak]


	
	  \endgroup
	
	  \bigskip
	  \begingroup
	  \large
	
	    
	    \stanza[\smallbreak]
	\label{pv.1.249a}\edlabel{pv.1.249a}\flagstanza{\tiny\textenglish{...1.249a}}तस्मात् तत्कारणबाधी विधिस्तं बाधते कथम् ।\&[\smallbreak]


	
	  \endgroup
	

	  \pstart {\color{DodgerBlue3}“अपि चासतां गुणानामारोपस्तत्रा”}त्मीये {\color{DodgerBlue3}“स्नेहाद्धि”} यस्माद् {\color{DodgerBlue3}“दृश्यते”} (२४८) {\color{DodgerBlue3}“तस्मात्त”}स्य स्नेहादेः {\color{DodgerBlue3}“कारण”}स्यात्मदर्शनस्या{\color{DodgerBlue3}“बाधी”} अबाधको {\color{DodgerBlue3}“विधि”}र्दीक्षा दुःखभावनादिरूपः {\color{DodgerBlue3}“तं”} स्नेहादिं {\color{DodgerBlue3}“बाधते कथं”} । कारणानिवृत्त्या कार्यनिषेधस्य कर्तुमशक्यत्वात् ।
	\pend
      

	  \begin{center}%% label @type='head'
	\textbf{(च) प्रकृतिपुरुषयोर्भेंदप्रतीतावपि न मोक्षः}
	\end{center}
	

	  \pstart सां ख्या स्तु मन्यंते । चेतनाचेतनयोः पुरुषप्रधा\edlabel{pvv.95-1}\footnote{\label{pvv.95-1}  १ प्रकृतिः}नयोर्यावदैक्यं मन्यते पुरुषः । तावत्स स्नेहवान् अयुक्तश्च भेदप्रतीतौ न स्नेहो वियुक्तश्चेति । अत्राह\edlabel{pvv.95-2}\footnote{\label{pvv.95-2}  २ एकबुद्धिरेवेन्द्रियादिष्वात्मनो नास्त्यतः पृथगात्मनो वेदनादित्याह ।} (।)
	\pend
      
	  \bigskip
	  \begingroup
	  \large
	
	    
	    \stanza[\smallbreak]
	\label{pv.1.249b}\edlabel{pv.1.249b}\flagstanza{\tiny\textenglish{...1.249b}}परापरप्रार्थनातो विनाशोत्पादबुद्धितः ॥ २४९ ॥\&[\smallbreak]


	
	  \endgroup
	

	  \pstart काणत्वादिदोषयुक्ता {\color{DodgerBlue3}“परापर”}स्य विशिष्टविशिष्टस्य चक्षुःशरीरादिकस्य {\color{DodgerBlue3}“प्रार्थनातः”} । आत्मनश्चान्यस्यानभिलाषतः । तस्मात्पृथग्भूतमात्मानमयममुक्तोपि जनो वेत्ति । तथा {\color{DodgerBlue3}“विनाशोत्पादबुद्धितः”} (। २४९)
	\pend
      \label{div_pvv.1.250}\edlabel{div_pvv.1.250}
	  
	% new div opening: depth here is 2
	
	  \bigskip
	  \begingroup
	  \large
	
	    
	    \stanza[\smallbreak]
	\label{pv.1.250}\edlabel{pv.1.250}\flagstanza{\tiny\textenglish{....1.250}}इन्द्रियादौ पृथग्भूतमात्मानं वेत्त्ययं जनः ।&तस्मान्नैकत्वदृष्ट्यापि स्नेहः । स्निह्यन् स आत्मनि ॥ २५० ॥\&[\smallbreak]


	
	  \endgroup
	

	  \pstart शरीरे{\color{DodgerBlue3}“न्द्रियादौ”} विपर्ययाच्चात्मनि {\color{DodgerBlue3}“भिन्नमात्मानं”} तेभ्यो {\color{DodgerBlue3}“वेत्ति । तस्मान्नैकत्वदृष्ट्यापि स्नेहः”} किन्त्वात्मदर्शनात् ॥ {\color{DodgerBlue3}“स”} आत्मदर्शी {\color{DodgerBlue3}“स्निह्यन्नात्मनि”} (। २५०)
	\pend
      \label{div_pvv.1.251}\edlabel{div_pvv.1.251}
	  
	% new div opening: depth here is 2
	
	  \bigskip
	  \begingroup
	  \large
	
	    
	    \stanza[\smallbreak]
	\label{pv.1.251}\edlabel{pv.1.251}\flagstanza{\tiny\textenglish{....1.251}}उपलम्भान्तरङ्गेषु प्रकृत्यैवानुरज्यते ॥&प्रत्युत्पन्नात् तु यो दुःखान्निर्वेदो द्वेष ईदृशः ॥ २५१ ॥\&[\smallbreak]


	
	  \endgroup
	
	  \bigskip
	  \begingroup
	  \large
	
	    
	    \stanza[\smallbreak]
	\label{pv.1.252a}\edlabel{pv.1.252a}\flagstanza{\tiny\textenglish{...1.252a}}न वैराग्यं;\&[\smallbreak]


	
	  \endgroup
	

	  \pstart {\color{DodgerBlue3}“उपलम्भान्तरङ्गेषू”}पभोगसाधनेष्विन्द्रियादिषु {\color{DodgerBlue3}“प्रकृत्या”} स्वभावेनै{\color{DodgerBlue3}“वानुरज्यते”} । {\color{DodgerBlue3}“प्रत्युत्पन्नात्तु”} वर्तमानात्पुन{\color{DodgerBlue3}“र्दुःखान्निर्व्वेदो यः”} स न {\color{DodgerBlue3}“वैराग्यं”} किन्तु {\color{DodgerBlue3}“द्वेष ईदृशः\edlabel{pvv.95-3}\footnote{\label{pvv.95-3}  ३ स्यादेतदात्मीयस्नेहस्यात्मीयबुद्धिरेव हेतुः सा तु गुणदर्शनादित्याह ।}”} । (२५१)
	\pend
      \label{div_pvv.1.252}\edlabel{div_pvv.1.252}
	  
	% new div opening: depth here is 2
	\leavevmode\marginnote{\textenglish{096/s}}
	  \bigskip
	  \begingroup
	  \large
	
	    
	    \stanza[\smallbreak]
	\label{pv.1.252b}\edlabel{pv.1.252b}\flagstanza{\tiny\textenglish{...1.252b}}तदप्यस्य स्नेहोऽवस्थान्तरेषणात् ।&द्वेषस्य दुःखयोनित्वात् स तावन्मात्रसंस्थितिः ॥ २५२ ॥\&[\smallbreak]


	
	  \endgroup
	

	  \pstart यस्मा{\color{DodgerBlue3}“त्तदापि”} निर्व्वेदावस्थायां {\color{DodgerBlue3}“स्नेहो”}स्यास्ति न च विरक्तस्य स्नेहसम्भवः । तदस्तित्वमेव कुत इति चेत् । {\color{DodgerBlue3}“अवस्थान्त”}रस्य दुःखहेतोर्निर्व्वेदकारिण्यां अवस्थायां विलक्षणस्यै{\color{DodgerBlue3}“षणात्”} । न हि स्नेहमन्तरेणैकत्यागादपरवाञ्छा । {\color{DodgerBlue3}“द्वेषस्य दुःखस्य योनित्वात् । स”} निर्व्वेदाख्यो द्वेषो यावद् दुःखमनुवर्तते {\color{DodgerBlue3}“तावन्मात्रं”} तावत्कालपरिमाणं {\color{DodgerBlue3}“संस्थिति”}रस्येति (। २५२)
	\pend
      \label{div_pvv.1.253_1.254}\edlabel{div_pvv.1.253_1.254}
	  
	% new div opening: depth here is 2
	

	  \begin{center}%% label @type='head'
	\textbf{(छ) हानोपादानहानित औदासीन्यम्}
	\end{center}
	
	  \bigskip
	  \begingroup
	  \large
	
	    
	    \stanza[\smallbreak]
	\label{pv.1.253a}\edlabel{pv.1.253a}\flagstanza{\tiny\textenglish{...1.253a}}तस्मिन् निवृत्तेप्रकृतिं स्वामेव भजते पुनः ।\&[\smallbreak]


	
	  \endgroup
	

	  \pstart तथा {\color{DodgerBlue3}“तस्मिन्”} दुःखे कारणनिरोधा{\color{DodgerBlue3}“न्निवृत्ते पुनः स्वामेव प्रकृतिं”} विषयेष्वविरागलक्षणां {\color{DodgerBlue3}“भजते”} सत्त्वदर्शी ।
	\pend
      

	  \pstart कीदृशं तर्हि वैराग्यं युक्तं\edlabel{pvv.96-1}\footnote{\label{pvv.96-1}  १ आह ।} ।
	\pend
      
	  \bigskip
	  \begingroup
	  \large
	
	    
	    \stanza[\smallbreak]
	\label{pv.1.253b}\edlabel{pv.1.253b}\flagstanza{\tiny\textenglish{...1.253b}}औदासीन्यं तु सर्वत्र त्यागोपादानहानितः ॥ २५३ ॥\&[\smallbreak]


	
	  \endgroup
	
	  \bigskip
	  \begingroup
	  \large
	
	    
	    \stanza[\smallbreak]
	\label{pv.1.254a}\edlabel{pv.1.254a}\flagstanza{\tiny\textenglish{...1.254a}}वासीचन्दनकल्पानां वैराग्यं नाम कथ्यते ।\&[\smallbreak]


	
	  \endgroup
	

	  \pstart सत्त्वदृष्ट्यभावात् सर्व्वत्र विष\edlabel{pvv.96-2}\footnote{\label{pvv.96-2}  २ आत्मभावे तदुपकरणे च ।}ये प्रतिकूलत्वानुकूलत्वाभ्यां {\color{DodgerBlue3}“अनध्यवसिते त्यागोप\edlabel{pvv.96-3}\footnote{\label{pvv.96-3}  ३ उद्वेगोहं ममेति ग्रहश्च ।}दानयोर्हानितो (२५३) वासीचन्द्रनकल्पनां”} वासीचन्दनयोः कल्पाः सदृशा ये वासीचन्दनकल्पा वा ये तेषां साक्षात्कृतनैरात्म्यतत्त्वानामौदासीन्यमनुनयप्रतिघरहितत्वं {\color{DodgerBlue3}“पुनर्वैराग्यं नाम”} आगमप्रसिद्धं {\color{DodgerBlue3}“कथ्यते”} ।
	\pend
      

	  \begin{center}%% label @type='head'
	\textbf{I. संस्कारदुःखभावात् दुःखभावना}
	\end{center}
	

	  \pstart ननु यदि दुःखभावनया स्नेहादिहान्या न मुक्तिः तत्कथं भगवतोक्ता दुःखभावनेत्याह (।)
	\pend
      
	  \bigskip
	  \begingroup
	  \large
	
	    
	    \stanza[\smallbreak]
	\label{pv.1.254b}\edlabel{pv.1.254b}\flagstanza{\tiny\textenglish{...1.254b}}संस्कारदुःखतां मत्वा कथिता दुःखभावना ॥ २५४ ॥\&[\smallbreak]


	
	  \endgroup
	

	  \pstart {\color{DodgerBlue3}“संस्कारदुःखतां मत्त्वा कथिता दुःखभावना”} । न हि दुःखदुःखतामभिसन्धाय तद्भावनोक्ता किन्तर्हि संस्कारदुःखतां । (२५४)
	\pend
      \label{div_pvv.1.155}\edlabel{div_pvv.1.155}
	  
	% new div opening: depth here is 2
	

	  \pstart a. सैव किमुच्यत इत्याह (।)
	\pend
      
	  \bigskip
	  \begingroup
	  \large
	
	    
	    \stanza[\smallbreak]
	\label{pv.1.255a}\edlabel{pv.1.255a}\flagstanza{\tiny\textenglish{...1.255a}}सा च नः प्रत्ययोत्पत्तिः सा नैरात्म्यदृगाश्रयः ।\&[\smallbreak]


	
	  \endgroup
	\leavevmode\marginnote{\textenglish{097/s}}

	  \pstart {\color{DodgerBlue3}“सा च संस्कारदुःखता नो”}ऽस्माकं सौ ग ता नां {\color{DodgerBlue3}“प्रत्ययोत्पत्ति”}र्हेतुपारतन्त्र्यं । {\color{DodgerBlue3}“सा”} प्रत्ययोत्पत्ति{\color{DodgerBlue3}“र्नैरात्म्यस्य\edlabel{pvv.97-1}\footnote{\label{pvv.97-1}  १ सापि न साक्षाद्भाविनामुक्तमिति ।} दृशो”} दर्शनस्या{\color{DodgerBlue3}“श्रयः”} कारणं ।
	\pend
      \leavevmode\marginnote{\textenglish{18b/MA}}

	  \pstart b. तथा हि हेतुफलभूताः {\color{DodgerBlue3}“क्षणक्षयिणो भावाः प्रवृत्तयो नात्मरूपा नाप्यात्माधि”}ष्ठिता इति संस्कारदुःखताभावना नैरात्म्यदर्शनानुकूला सैव च मुक्तिहेतुरित्याह(।)
	\pend
      
	  \bigskip
	  \begingroup
	  \large
	
	    
	    \stanza[\smallbreak]
	\label{pv.1.255b}\edlabel{pv.1.255b}\flagstanza{\tiny\textenglish{...1.255b}}मुक्तिस्तु शून्यतादृष्टेस्तदर्थाः शेषभावनाः ॥ २५५ ॥\&[\smallbreak]


	
	  \endgroup
	

	  \pstart {\color{DodgerBlue3}“मुक्तिस्तु शून्यताया”} निरात्मताया {\color{DodgerBlue3}“दृष्टेः । शेषस्य”} नित्यदुःखादे{\color{DodgerBlue3}“र्भावनास्तदर्था”} निरात्मदर्शनार्थाः । (२५५)
	\pend
      \label{div_pvv.1.256}\edlabel{div_pvv.1.256}
	  
	% new div opening: depth here is 2
	

	  \begin{center}%% label @type='head'
	\textbf{II. अनित्यदुःखानात्मता}
	\end{center}
	
	  \bigskip
	  \begingroup
	  \large
	
	    
	    \stanza[\smallbreak]
	\label{pv.1.256a}\edlabel{pv.1.256a}\flagstanza{\tiny\textenglish{...1.256a}}अनित्यात् प्राह तेनैव दुःखं; दुःखान्निरात्मताम् ॥\&[\smallbreak]


	
	  \endgroup
	

	  \pstart {\color{DodgerBlue3}“तेनैवा”}नित्यदुःखभावनायाः शून्यताभावनानुकूलत्वेन भगवाननित्यादनित्यत्वाद् {\color{DodgerBlue3}“दुःखं”} संसारिस्कन्धानां हा (नाद्) {\color{DodgerBlue3}“दुःखाद् दुःखत्वान्निरात्मतामाह”} (।) तद्यथा “रूपं भिक्षवो नित्यमनित्यं वा । अनित्यं भदन्त । यदनित्यं तद् दुःखं सुखम्वा । दुःखम्भदन्त । यदनित्यं दुःखं विपरिणाम धर्मकं कल्प्यन्नु तदेवं द्रष्टुः एतन्मम एषोहमस्मि एष मे आत्मेति । नो हीदं भदन्त ।”\edlabel{pvv.97-2}\footnote{\label{pvv.97-2}  २ मज्झिमनिकाये महापुण्णमसुत्तन्ते (१०९) ।} इत्येवं हेतुफलभाव\edlabel{pvv.97-3}\footnote{\label{pvv.97-3}  ३ हेतुफलभावकथनेनात्मदर्शनमेव ।}नेनात्मदर्शनमेव मुक्तेरुपाय इति कथितं । तदेवाचार्येणोक्तं (।) उक्तो मार्गः तदभ्यासादाश्रयः परिवर्तत इति नास्ति विरोधः ।
	\pend
      

	  \pstart यः पुनरात्मदर्शी सोऽविरक्त एव ।
	\pend
      
	  \bigskip
	  \begingroup
	  \large
	
	    
	    \stanza[\smallbreak]
	\label{pv.1.256b}\edlabel{pv.1.256b}\flagstanza{\tiny\textenglish{...1.256b}}अविरक्तश्च तृष्णावान् सर्वारम्भसमाश्रितः ॥ २५६ ॥\&[\smallbreak]


	
	  \endgroup
	

	  \pstart {\color{DodgerBlue3}“अविरक्तश्च तृष्णावान्”} । हान्युपादानलक्षणान् कर्मप्रसवहेतून् {\color{DodgerBlue3}“समाश्रितः”} । (२५६)
	\pend
      \label{div_pvv.1.257}\edlabel{div_pvv.1.257}
	  
	% new div opening: depth here is 2
	
	  \bigskip
	  \begingroup
	  \large
	
	    
	    \stanza[\smallbreak]
	\label{pv.1.257}\edlabel{pv.1.257}\flagstanza{\tiny\textenglish{....1.257}}सोऽमुक्तः क्लेशकर्मभ्यां संसारी नाम तादृशः ॥&आत्मीयमेव यो नेच्छेद् भोक्ताप्यस्य न विद्यते ॥ २५७ ॥\&[\smallbreak]


	
	  \endgroup
	

	  \pstart {\color{DodgerBlue3}“सोऽमुक्तः क्लेशकर्मभ्यां”} आत्मदर्शनप्रवृत्तिकारणकाभ्यां {\color{DodgerBlue3}“संसारी नाम प्रसिद्धः तादृशः”} । तदेवमात्मनि सति नात्मीयत्यागः । तथाऽमुक्तिरित्युक्तं । भवतु वाऽत्मीयत्यागस्तथाप्या{\color{DodgerBlue3}“त्मीयमेव यो नेच्छे”}त्तन्मतेऽस्यात्मीयस्य {\color{DodgerBlue3}“भोक्ता न विद्यते”} । भोग्यापेक्षत्वात् भोक्तृत्वस्य (२५७) ।
	\pend
      \label{div_pvv.1.258_1.159_1.160_1.161_1.162_1.163}\edlabel{div_pvv.1.258_1.159_1.160_1.161_1.162_1.163}
	  
	% new div opening: depth here is 2
	\leavevmode\marginnote{\textenglish{098/s}}

	  \begin{center}%% label @type='head'
	\textbf{III. संसारी क्लेशकर्मभ्याममुक्तः}
	\end{center}
	

	  \begin{center}%% label @type='head'
	\textbf{(ज) सत्कायदृष्टिर्मूलम्}
	\end{center}
	
	  \bigskip
	  \begingroup
	  \large
	
	    
	    \stanza[\smallbreak]
	\label{pv.1.258}\edlabel{pv.1.258}\flagstanza{\tiny\textenglish{....1.258}}आत्माऽपि न तदा तस्य क्रियाभोगौ हि लक्षणम् ।&तस्मादनादिसन्तानतुल्यजातीयबीजिकाम् ॥ २५८ ॥\&[\smallbreak]


	
	  \endgroup
	
	  \bigskip
	  \begingroup
	  \large
	
	    
	    \stanza[\smallbreak]
	\label{pv.1.259a}\edlabel{pv.1.259a}\flagstanza{\tiny\textenglish{...1.259a}}उत्खातमूलां कुरुत सत्वदृष्टिं मुमुक्षवः ।\&[\smallbreak]


	
	  \endgroup
	

	  \pstart भोक्त्रभावे {\color{DodgerBlue3}“आत्मापि ना”}स्तीति प्रसङ्गात् । कुत इति चेत् । हिर्यस्मादस्यात्मनः {\color{DodgerBlue3}“क्रियाभोगौ लक्षणं”} कर्त्ता भोक्ता चात्मोच्यते (।) यदा चात्मीयमेव नास्ति । किमर्थं कर्म कर्तव्यं किम्वा भोक्तव्यं । कर्तृत्वभोक्तृत्वाभावादात्माभाव एव स्वीकृतः स्यात् । तस्मात्सत्यात्मनि आत्मीयं तत्स्नेहादिसत्त्वेऽनुच्छेद एवं संसारस्य । {\color{DodgerBlue3}“तस्मात्”} संसारादुद्विजमाना मुमुक्षव उ\edlabel{pvv.98-1}\footnote{\label{pvv.98-1}  १ यथोक्तमिह “यावच्चतुर्थः श्रमणः शून्याः परप्रवादाः श्रमणैर्ब्राह्मणैर्व्वा” ।}त्खातमूलामुद्ध्ृतकारणां सत्त्वदृष्टिं कुरुत ।
	\pend
      

	  \pstart ननु किमस्या मूलमित्याह (।)
	\pend
      

	  \pstart {\color{DodgerBlue3}“अनादिसन्तानस्तुल्यजातीयः”} पूर्व्वपूर्व्वसत्त्वदर्शनस्वभावोऽविद्यारूपो {\color{DodgerBlue3}“बीजं”} कारणं यस्यास्तां (२५८) {\color{DodgerBlue3}“सत्त्वद”}र्शनमविद्यास्वभावं पूर्व्वपूर्व्वमुत्तरोत्तरस्य {\color{DodgerBlue3}“सत्त्व”}दर्शनस्य हेतुरित्यर्थः ।
	\pend
      

	  \begin{center}%% label @type='head'
	\textbf{(आगममात्रेण न मुक्तिः)}
	\end{center}
	

	  \pstart ननूक्तमीश्वरेणागमेऽस्त्यात्मा मोक्षश्चास्य दीक्षाविधिनेति । तत्किमत्र चिन्त्यते तत्कारणा वा धीविधिस्तं बाधते कथमित्यादिनोक्तत्वात् (।) अत्राह (।)
	\pend
      
	  \bigskip
	  \begingroup
	  \large
	
	    
	    \stanza[\smallbreak]
	\label{pv.1.259b}\edlabel{pv.1.259b}\flagstanza{\tiny\textenglish{...1.259b}}आगमस्य तथाभावनिबन्धनमपश्यताम् ॥ २५९ ॥\&[\smallbreak]


	
	  \endgroup
	
	  \bigskip
	  \begingroup
	  \large
	
	    
	    \stanza[\smallbreak]
	\label{pv.1.260a}\edlabel{pv.1.260a}\flagstanza{\tiny\textenglish{...1.260a}}मुक्तिमागममात्रेण वदन्न परितोषकृद् ।\&[\smallbreak]


	
	  \endgroup
	

	  \pstart {\color{DodgerBlue3}“आगमस्य तथाभावस्य”} प्रतिपादितार्थसम्वादित्वस्य {\color{DodgerBlue3}“निबन्धनं”} हेतुम{\color{DodgerBlue3}“पश्यतां”} । (२५९) मुमुक्षूणा{\color{DodgerBlue3}“मागममात्रेण मुक्तिं वदन्”} न {\color{DodgerBlue3}“परितोष”}कृद् भवति ।
	\pend
      

	  \pstart {\color{DodgerBlue3}“दीक्षाऽकिञ्चित्करी---”}
	\pend
      

	  \pstart नन्वस्ति प्रामाण्यनिबन्धनमागमस्य दीक्षाविधिस्पृष्टस्यानारोहधर्मकत्वदर्शनं । यथा हि बीजं दीक्षाविधिस्पृष्टं न प्ररोहति तथा पुमानपि दीक्षितो न पुनर्भवतीत्याह (।)
	\pend
      \leavevmode\marginnote{\textenglish{099/s}}
	  \bigskip
	  \begingroup
	  \large
	
	    
	    \stanza[\smallbreak]
	\label{pv.1.260b}\edlabel{pv.1.260b}\flagstanza{\tiny\textenglish{...1.260b}}नालं बीजादिसंसिद्धो विधिः पुंसामजन्मने ॥ २६० ॥\&[\smallbreak]


	
	  \endgroup
	
	  \bigskip
	  \begingroup
	  \large
	
	    
	    \stanza[\smallbreak]
	\label{pv.1.261a}\edlabel{pv.1.261a}\flagstanza{\tiny\textenglish{...1.261a}}तैलाभ्यङ्गाग्निहादेरपि मुक्ति प्रसङ्गतः ॥\&[\smallbreak]


	
	  \endgroup
	

	  \pstart नालं शक्तो {\color{DodgerBlue3}“बीजादिषु संसिद्धो विधिर्दी”}क्षायाः {\color{DodgerBlue3}“पुंसामजन्मने”} । (२६०) {\color{DodgerBlue3}“तैलाभ्यङ्गाग्निदाहदेरपि”} संसार{\color{DodgerBlue3}“न्मुक्तिप्रसङ्गतः”} । तैलेनाभ्यक्तं बीजमग्निना च स्पृष्टं न प्ररोहति\edlabel{pvv.99-1}\footnote{\label{pvv.99-1}  १ क्षोदश्वेदविदलनमादिना ।} तथा पुरुषोपि तैलाभ्यङ्गाग्निदाहाभ्यां न पुनर्भवेत् ।
	\pend
      

	  \begin{center}%% label @type='head'
	\textbf{I. आत्मनोऽमूर्त्तत्वे न पापगौरवलाघवम्}
	\end{center}
	

	  \pstart a. स्यादेतत् । दीक्षायाः {\color{DodgerBlue3}“प्राक्”} पापगुरोरुत्तरं तुलया लाघवात् । पापाभावोपलम्भ आगमप्रामाण्यनिबन्धनमित्याह (।)
	\pend
      
	  \bigskip
	  \begingroup
	  \large
	
	    
	    \stanza[\smallbreak]
	\label{pv.1.261b}\edlabel{pv.1.261b}\flagstanza{\tiny\textenglish{...1.261b}}प्राग् गुरोर्लाघवात् पश्चान्न पापहरणं कृतम् ॥ २६१ ॥\&[\smallbreak]


	
	  \endgroup
	
	  \bigskip
	  \begingroup
	  \large
	
	    
	    \stanza[\smallbreak]
	\label{pv.1.262a}\edlabel{pv.1.262a}\flagstanza{\tiny\textenglish{...1.262a}}मा भूद् गौरवमेवास्य न पापं गुर्वमूर्त्तितः ॥\&[\smallbreak]


	
	  \endgroup
	

	  \pstart प्राग् गुरोर्लाघवात् पश्चान्न पापहरणं कृतं । दीक्षायाः {\color{DodgerBlue3}“प्राग् गुरोः पश्चाल्लाघवात्”} दीक्षया {\color{DodgerBlue3}“न पापहरण”}स्य दीक्षितस्य किन्तु {\color{DodgerBlue3}“गौरवमेवास्य कृतं”} सत् (२६१) {\color{DodgerBlue3}“मा भूदि”}ति कस्मान्न कल्प्यते । लाघवं हि गौरवविरोधि दृश्यमानं तदभावमेव गमयेत् । न पापभावं । पापमेव गुर्वितिचेत् । {\color{DodgerBlue3}“न पापं गुरु । अमूर्त्तितो”} मूर्त्तत्वाभावात् । भूर्त्तधर्मो हि गौरवं कथं पापस्यामूर्त्तस्य स्यात् ।
	\pend
      

	  \pstart ननु त्वत्पक्षेऽपि नैरात्म्यदर्शने भूतेपि कस्मान्न जन्मेत्याह (।)
	\pend
      
	  \bigskip
	  \begingroup
	  \large
	
	    
	    \stanza[\smallbreak]
	\label{pv.1.262b}\edlabel{pv.1.262b}\flagstanza{\tiny\textenglish{...1.262b}}मिथ्याज्ञानतदुद्भूततर्षसञ्चेतनावशात् ॥ २६२ ॥\&[\smallbreak]


	
	  \endgroup
	
	  \bigskip
	  \begingroup
	  \large
	
	    
	    \stanza[\smallbreak]
	\label{pv.1.263a}\edlabel{pv.1.263a}\flagstanza{\tiny\textenglish{...1.263a}}हीनस्थानगतिर्जन्म ततस्तच्छिन्न जायते ।\&[\smallbreak]


	
	  \endgroup
	

	  \pstart {\color{DodgerBlue3}“मिथ्याज्ञानं”} दुःखे विपर्यासमतिः । {\color{DodgerBlue3}“तदुद्भूतस्तर्षो”} मिथ्याज्ञानप्रभवा तृष्णा । ताभ्यां संप्रयुक्ते {\color{DodgerBlue3}“चेतने”} । तद्वशाद्या {\color{DodgerBlue3}“हीनस्थानगति”}स्त{\color{DodgerBlue3}“ज्जन्मे”}त्युक्तं । {\color{DodgerBlue3}“अतस्तच्छित्”} । अज्ञानतृष्णाच्छेदको नैरात्म्यदर्शी {\color{DodgerBlue3}“न जातये”} कारणाभावात् ।
	\pend
      

	  \pstart तदेवाह (।)
	\pend
      
	  \bigskip
	  \begingroup
	  \large
	
	    
	    \stanza[\smallbreak]
	\label{pv.1.263b}\edlabel{pv.1.263b}\flagstanza{\tiny\textenglish{...1.263b}}तयोरेव हि सामर्थ्यं जातौ तन्मात्रभावतः ॥ २६३ ॥\&[\smallbreak]


	
	  \endgroup
	

	  \pstart {\color{DodgerBlue3}“तयोरेवा”}ज्ञानतृष्णयो{\color{DodgerBlue3}“र्हि सामर्थ्यं जातौ”} जन्मनिमित्तं {\color{DodgerBlue3}“तन्मात्रेण\edlabel{pvv.99-2}\footnote{\label{pvv.99-2}  २ इह च तृष्णया देशान्तरं कुर्य्यात् । तृष्णाज्ञानभावाभावानुकारात् ।} भावतः”} ॥
	\pend
      \leavevmode\marginnote{\textenglish{19a/MA}}

	  \pstart ते च दीक्षितस्यापि स्त इति स जायते । (२६३)
	\pend
      \leavevmode\marginnote{\textenglish{100/s}}\label{div_pvv.1.264_1.265_1.266_1.267_1.268_1.269_1.270}\edlabel{div_pvv.1.264_1.265_1.266_1.267_1.268_1.269_1.270}
	  
	% new div opening: depth here is 2
	

	  \pstart c. ननु कर्मापि जन्मकारणमिष्टं तत्कथमज्ञानतृष्णे एवोक्ते इत्याह (।)
	\pend
      
	  \bigskip
	  \begingroup
	  \large
	
	    
	    \stanza[\smallbreak]
	\label{pv.1.264}\edlabel{pv.1.264}\flagstanza{\tiny\textenglish{....1.264}}ते चेतने स्वयं कर्मेत्यखण्डं जन्मकारणम् ॥&गतिप्रतीत्योःकरणान्याश्रयस्तान्यदृष्टतः ॥ २६४ ॥\&[\smallbreak]


	
	  \endgroup
	
	  \bigskip
	  \begingroup
	  \large
	
	    
	    \stanza[\smallbreak]
	\label{pv.1.265a}\edlabel{pv.1.265a}\flagstanza{\tiny\textenglish{...1.265a}}अदृष्टनाशादगतिः तत्संस्कारो न चेतना ॥\&[\smallbreak]


	
	  \endgroup
	

	  \pstart {\color{DodgerBlue3}“ते चेतने”} मिथ्याज्ञानतदुद्भूतर्षसंचेतने {\color{DodgerBlue3}“स्वय”}मात्मना पूर्व्वशुभाशुभकर्मसँस्कारसहाये {\color{DodgerBlue3}“कर्म”} कर्मस्वभावे {\color{DodgerBlue3}“इत्यखण्डम”}न्यूनं {\color{DodgerBlue3}“जन्मकारण”}मुक्तमिति न विरोधः । स्यादेतद्(।) {\color{DodgerBlue3}“गतिप्रतीत्यो”}रभिमतदेशगमनस्य ज्ञानस्य च {\color{DodgerBlue3}“करणानी”}न्द्रियाणि । {\color{DodgerBlue3}“आश्रयः”} कारणं इन्द्रियेभ्य उत्पन्नेन ज्ञानेन विषयं परिच्छिद्य प्रवृत्तेः । तानीन्द्रियाणि {\color{DodgerBlue3}“चादृष्टतः”} शुभादिलक्षणाद् (२६४) । दीक्षया चा{\color{DodgerBlue3}“दृष्टनाशात्त”}त्कार्याणां कारणानामनुत्पत्तेर्व्विषयस्य परिच्छेदतृष्णयोरभावात् {\color{DodgerBlue3}“अगति”}र्जन्मस्थान इति नास्ति पुनर्भवो दीक्षितस्य (।) {\color{DodgerBlue3}“तददृ”}\edlabel{pvv.100-1}\footnote{\label{pvv.100-1}  १ विंशति कुशलाकुशलाः ।}ष्टञ्चात्म{\color{DodgerBlue3}“संस्कारो न चे\edlabel{pvv.100-2}\footnote{\label{pvv.100-2}  २ बौद्धोक्तः ।}तना”} स्यात् दीक्षितस्यापि सास्तीति जन्मापि\edlabel{pvv.100-3}\footnote{\label{pvv.100-3}  ३ इति पराभिप्रायः ।} स्यात ।
	\pend
      

	  \pstart b. अत्रा\edlabel{pvv.100-4}\footnote{\label{pvv.100-4}  ४ समाधानं ।}ह (।)
	\pend
      
	  \bigskip
	  \begingroup
	  \large
	
	    
	    \stanza[\smallbreak]
	\label{pv.1.265b}\edlabel{pv.1.265b}\flagstanza{\tiny\textenglish{...1.265b}}सामर्थ्यं करणोत्पत्तेर्भावाभावानुवृत्तितः ॥ २६५ ॥\&[\smallbreak]


	
	  \endgroup
	
	  \bigskip
	  \begingroup
	  \large
	
	    
	    \stanza[\smallbreak]
	\label{pv.1.266a}\edlabel{pv.1.266a}\flagstanza{\tiny\textenglish{...1.266a}}दृष्टं बुद्धेर्न चान्यस्या सन्ति तानि नयन्ति किम् ।\&[\smallbreak]


	
	  \endgroup
	

	  \pstart बुद्धे\edlabel{pvv.100-5}\footnote{\label{pvv.100-5}  ५ मिथ्याज्ञानतर्षयुतायाः ।} {\color{DodgerBlue3}“र्भावाभावानुवृत्तितो”}ऽन्वयव्यतिरेकानुवृत्त्या {\color{DodgerBlue3}“करणा”}नामेकस्माद्देशादप\edlabel{pvv.100-6}\footnote{\label{pvv.100-6}  ६ यत्रोत्पत्स्यते ।}रदेशसम्बद्धाना{\color{DodgerBlue3}“मुत्पत्तेः”} कारणा{\color{DodgerBlue3}“त्सामार्थ्यं बुद्धे”}रिन्द्रियजननं प्रति(२६५){\color{DodgerBlue3}“दृष्टं”} नान्यस्य संस्काररूपस्यादृष्टस्य तदन्वयव्यतिरेकानुविधानानुपलम्भात् । सा बुद्धिश्चास्ति देशान्तरसम्बद्धकरणजनिका दीक्षितस्या {\color{DodgerBlue3}“तत्कि”}न्तानि करणानि गर्भस्थानन्न {\color{DodgerBlue3}“यन्ति”} गच्छन्ति हेत्ववैकल्याद् भवितव्यं (२६६) ।
	\pend
      

	  \pstart c. गमनेनावश्यं दीक्षयोपहता बुद्धिर्देशान्तरं नेतुमिन्द्रियाण्यशक्ता चेत् ।
	\pend
      
	  \bigskip
	  \begingroup
	  \large
	
	    
	    \stanza[\smallbreak]
	\label{pv.1.266b}\edlabel{pv.1.266b}\flagstanza{\tiny\textenglish{...1.266b}}धारणप्रेरणक्षोभनिरोधाश्चेतनावशाः ॥ २६६ ॥\&[\smallbreak]


	
	  \endgroup
	
	  \bigskip
	  \begingroup
	  \large
	
	    
	    \stanza[\smallbreak]
	\label{pv.1.267}\edlabel{pv.1.267}\flagstanza{\tiny\textenglish{....1.267}}न स्युस्तेषामसामर्थ्ये तस्य दीक्षाद्यनन्तरम् ।&अथ बुद्धेस्तदाभावान्न स्युः सन्धीयते मलैः ॥ २६७ ॥\&[\smallbreak]


	
	  \endgroup
	

	  \pstart नन्वेवन्तस्य मुमुक्षोर्दी{\color{DodgerBlue3}“क्षानन्तरं”} बुद्धेर{\color{DodgerBlue3}“सामर्थ्ये तेषा”}मिन्द्रियाणां स्वविषये व्यवस्थापनं धारणं । तत्रा\edlabel{pvv.100-7}\footnote{\label{pvv.100-7}  ७ आकृष्य नयनं ।}योजनं {\color{DodgerBlue3}“प्रेरणं । क्षोभो”}\edlabel{pvv.100-8}\footnote{\label{pvv.100-8}  ८ भ्रूभङ्गादि ।} विकारः । स्वविषयान्निवर्तनं \leavevmode\marginnote{\textenglish{101/s}} {\color{DodgerBlue3}“निरोधः”} । ते बुद्धिनिबन्धनवृत्तयो {\color{DodgerBlue3}“न स्युः”} ॥ {\color{DodgerBlue3}“अथ बुद्धेस्तदा”} मरणकाले{\color{DodgerBlue3}“ऽभावान्न स्युः”} धारणप्रेरणादयः ।
	\pend
      

	  \pstart अत्राह । सन्धीयते जन्यते {\color{DodgerBlue3}“मलै”}र्मिथ्याज्ञानैर्मिथ्याज्ञानात्मस्नेहादिभिर्बुद्धिर्मरणसमयेपीति कुतो बुद्ध्यभावः । (२६७)
	\pend
      
	  \bigskip
	  \begingroup
	  \large
	
	    
	    \stanza[\smallbreak]
	\label{pv.1.268}\edlabel{pv.1.268}\flagstanza{\tiny\textenglish{....1.268}}बुद्धेस्तेषामसामर्थ्ये जीवतोऽपि स्युरक्षमाः ॥&निर्ह्रासातिशयात् पुष्टौ प्रतिपक्षस्वपक्षयोः ॥ २६८ ॥\&[\smallbreak]


	
	  \endgroup
	
	  \bigskip
	  \begingroup
	  \large
	
	    
	    \stanza[\smallbreak]
	\label{pv.1.269a}\edlabel{pv.1.269a}\flagstanza{\tiny\textenglish{...1.269a}}दोषाः स्वबीजसन्ताना दीक्षितेऽप्यनिवारिताः ॥\&[\smallbreak]


	
	  \endgroup
	

	  \pstart {\color{DodgerBlue3}“अथ”} मला अपि दीक्षयोपहतसामर्थ्या न बुद्धि सन्दधति । तदा {\color{DodgerBlue3}“तेषां”} मलानाम{\color{DodgerBlue3}“सामर्थ्ये”} स्वीक्रियमाणे {\color{DodgerBlue3}“जीवतोपि”} दीक्षितस्य मला बुद्धिसन्धानं {\color{DodgerBlue3}“प्रत्यक्षमाः स्युः”} । दीक्षा मलानां बाधिकेत्यपि मिथ्या । तथा प्रतिपक्षस्य नैरात्म्यभावनायाः स्वपक्षस्यायोनिशोमनस्कारस्याभ्यासात् {\color{DodgerBlue3}“पुष्टौ”} सत्यां दोषाणां यथाक्रमं {\color{DodgerBlue3}“नर्ह्रा”}सादपचयात् । {\color{DodgerBlue3}“अतिशया”}दभिवृद्धे (२६८) {\color{DodgerBlue3}“र्दोषाः”} स्वकीयाद् {\color{DodgerBlue3}“बीजा”}त्तुल्यजातीयकारणात् {\color{DodgerBlue3}“सन्तानो”} येषां ते {\color{DodgerBlue3}“दीक्षितेपि”} दोषबीजे परिपोषवत्य{\color{DodgerBlue3}“निवारिताः”} ।
	\pend
      

	  \begin{center}%% label @type='head'
	\textbf{II. आत्मनो नित्यत्वे न पुनर्जन्म}
	\end{center}
	

	  \pstart स्यादेतद् (।) आत्मनोपि गर्भगतकरणादिजनने व्यापारः स एव दीक्षया निरुद्ध इति न पुनर्जन्मेत्याह (।)
	\pend
      
	  \bigskip
	  \begingroup
	  \large
	
	    
	    \stanza[\smallbreak]
	\label{pv.1.269b}\edlabel{pv.1.269b}\flagstanza{\tiny\textenglish{...1.269b}}नित्यस्य निरपेक्षत्वात् क्रमोत्पत्तिर्विरुध्यते ॥ २६९ ॥\&[\smallbreak]


	
	  \endgroup
	
	  \bigskip
	  \begingroup
	  \large
	
	    
	    \stanza[\smallbreak]
	\label{pv.1.270}\edlabel{pv.1.270}\flagstanza{\tiny\textenglish{....1.270}}क्रियायामक्रियायाञ्च क्रिययोः सदृशात्मनः ॥&ऐक्यञ्च हेतुफलयोर्व्यतिरेकस्ततस्तयोः ॥ २७० ॥\&[\smallbreak]


	
	  \endgroup
	

	  \pstart नित्यस्यानुपकार्यतया {\color{DodgerBlue3}“निरपेक्षत्वात्”} करणादीनां {\color{DodgerBlue3}“क्रमे\edlabel{pvv.101-1}\footnote{\label{pvv.101-1}  १पूर्व्व कृतवान् पश्चादपि गर्भे व्याप्रियत इति क्रमः ।}णोत्पत्तिर्व्वि”}रुध्यते समत्वहेतुसद्भावात् सकृदुत्पादप्रसक्तेः । (२६९) इन्द्रियादेः क्रियायामक्रियायाञ्च सदृशात्मनस्तुल्यरूपस्यात्मनस्तयोः कालयोस्ते ते {\color{DodgerBlue3}“विरुध्येते”} ।
	\pend
      

	  \pstart यद्यसौ कार्यकरणस्वभावः तदा कार्यादेव {\color{DodgerBlue3}“क्रियावि”}रामोस्य विरुध्यते । एव{\color{DodgerBlue3}“मक्रियायामपि”} वाच्यं । किञ्च (।) आत्मनः कर्मकर्तृता हेतुत्वं भोक्तृत्वञ्च {\color{DodgerBlue3}“फलं”} ते चात्मन एकरूपस्य रूपे इति हेतुफलाभेदप्रसङ्गः । {\color{DodgerBlue3}“व्यतिरेको भेदस्तत आत्मनस्तयोः”} कर्तृ त्वभोक्तृत्वयोर्धर्मयोस्ततो नैक्यप्रसङ्ग इति चेत् । (२७० ।)
	\pend
      \leavevmode\marginnote{\textenglish{102/s}}\label{div_pvv.1.271_1.272_1.273_1.274_1.275_1.276}\edlabel{div_pvv.1.271_1.272_1.273_1.274_1.275_1.276}
	  
	% new div opening: depth here is 2
	

	  \begin{center}%% label @type='head'
	\textbf{III. नैरात्म्ये स्मृतिसंगतिः}
	\end{center}
	
	  \bigskip
	  \begingroup
	  \large
	
	    
	    \stanza[\smallbreak]
	\label{pv.1.271a}\edlabel{pv.1.271a}\flagstanza{\tiny\textenglish{...1.271a}}कर्तृभोक्तृत्वहानिः स्यात् सामर्थ्यञ्च न सिध्यति ।\&[\smallbreak]


	
	  \endgroup
	

	  \pstart एवं {\color{DodgerBlue3}“कर्तृ त्वभोक्तृत्वाहानिः स्यात्”} । आत्मनोऽतत्स्वभावत्वात् । तत्सम्बन्धात् कर्ता भोक्ता च । अनुपकृतस्य सम्बन्धित्वेऽतिप्रसङ्गात् उपकृतत्वं वेदितव्यं तच्चाशक्यसाधनं यस्मान्नित्यस्याव्यतिरेकित्वात् {\color{DodgerBlue3}“सामर्थ्यञ्च न सिध्यति”} ।
	\pend
      

	  \pstart ननु यद्यात्मा नास्ति । तदाऽन्येनानुभूतं कर्म च कृतमन्यः स्मरति भुडत्क्ते फलमिति स्यात्तथा चातिप्रसङ्ग इत्याह (।)
	\pend
      
	  \bigskip
	  \begingroup
	  \large
	
	    
	    \stanza[\smallbreak]
	\label{pv.1.271b}\edlabel{pv.1.271b}\flagstanza{\tiny\textenglish{...1.271b}}अन्यस्मरणभोगादिप्रसङ्गश्च न बाधकाः ॥ २७१ ॥\&[\smallbreak]


	
	  \endgroup
	
	  \bigskip
	  \begingroup
	  \large
	
	    
	    \stanza[\smallbreak]
	\label{pv.1.272a}\edlabel{pv.1.272a}\flagstanza{\tiny\textenglish{...1.272a}}अस्मृतेः; कस्य चित् । तेन ह्यनुभूतेः स्मृतोद्भवः ।\&[\smallbreak]


	
	  \endgroup
	

	  \pstart {\color{DodgerBlue3}“अन्यस्य स्मरणभोगादिप्रसङ्गाश्च न बाधका”} भवन्ति (२७१) अस्मृतेः । {\color{DodgerBlue3}“कस्यचित्”} स्मर्त्तुरभावात् । एवं भोगोपि नास्ति भोक्त्रभावात् । {\color{DodgerBlue3}“तेन”} हि तस्मात् \leavevmode\marginnote{\textenglish{19b/MA}} स्मर्त्रभावात् कारणाद्विषयाणाम{\color{DodgerBlue3}“नुभूतेः”} सकाशात् {\color{DodgerBlue3}“स्मृति”}रेव स्मृतं तस्यो{\color{DodgerBlue3}“द्भवः”} । वस्तुधर्मो ह्येष यदनुभवः पटीयान् स्मरणबीजाधानद्वारेण स्मरणं जनयति । शुभाशुभचेतनाश्च संस्कारा भोगाकारं सम्विदं प्रवर्तयन्ति (।) तत्किं स्मर्तृभोक्तृदुर्ग्रहेण । तत्तदाकारः प्रतीत्यसमुत्पन्नो बुद्धिप्रबन्ध एव केवलो न तु संसारी नाम कश्चित् ॥
	\pend
      

	  \pstart h. यदि नास्त्यात्मा कथं स कोपदृष्टिः संसारप्रवृत्तिर्व्वेत्याह (।)
	\pend
      
	  \bigskip
	  \begingroup
	  \large
	
	    
	    \stanza[\smallbreak]
	\label{pv.1.272b}\edlabel{pv.1.272b}\flagstanza{\tiny\textenglish{...1.272b}}स्थिरं सुखं ममाहञ्चेत्यादि सत्यचतुष्टये ॥ २७२ ॥\&[\smallbreak]


	
	  \endgroup
	
	  \bigskip
	  \begingroup
	  \large
	
	    
	    \stanza[\smallbreak]
	\label{pv.1.273a}\edlabel{pv.1.273a}\flagstanza{\tiny\textenglish{...1.273a}}अभूतान् षोडशाकारान् आरोप्य परितृष्यति ॥\&[\smallbreak]


	
	  \endgroup
	

	  \pstart {\color{DodgerBlue3}“स्थिर\edlabel{pvv.102-1}\footnote{\label{pvv.102-1}  १ नित्यमिति वाच्ये क्षणात् परं स्थायी सर्व्वो नित्य इत्यर्थः ।}म”}क्षणिकं सुखं त्रिदुःखताविपरीतं ममेत्यात्मीयमहमित्यात्माहंकारश्चेति दुःखसत्यस्य विपरीता आकारा इत्याद्यान् प्रति सत्यं\edlabel{pvv.102-2}\footnote{\label{pvv.102-2}  २ असमुदया हेत्वप्रभवाद्याः ।} चतुरः कृत्वा {\color{DodgerBlue3}“षोडशाकारान्-भूतान्”} प्राक् यथोक्तभूताकारविपरीतान् {\color{DodgerBlue3}“सत्यचतुष्टये”} (२७२)ऽरोपयतीति भ्रान्तिरेव सत्कायदृष्टिः स्वबीजप्रभवा । {\color{DodgerBlue3}“आरो”}\edlabel{pvv.102-3}\footnote{\label{pvv.102-3}  ३ स सत्कायदृष्टिमान् तामारोप्य च ।}प्य च किञ्चित् स्वसूखसाधनञ्च मन्यमानस्तदर्थं {\color{DodgerBlue3}“परितृष्यति”} । तृष्णया च जन्मस्थानोपांदानमिति संसारप्रवृत्तिः ।
	\pend
      \leavevmode\marginnote{\textenglish{103/s}}

	  \begin{center}%% label @type='head'
	\textbf{घ. सम्यग्दृष्टिनैरात्म्यदृष्टिः}
	\end{center}
	

	  \pstart मार्गविपर्ययं संसारहेतुमुक्त्वा मार्गमाह (।)
	\pend
      
	  \bigskip
	  \begingroup
	  \large
	
	    
	    \stanza[\smallbreak]
	\label{pv.1.273b}\edlabel{pv.1.273b}\flagstanza{\tiny\textenglish{...1.273b}}तत्रैव तद्विरुद्धार्थतत्त्वाकारानुरोधिनी ॥ २७३ ॥\&[\smallbreak]


	
	  \endgroup
	
	  \bigskip
	  \begingroup
	  \large
	
	    
	    \stanza[\smallbreak]
	\label{pv.1.274a}\edlabel{pv.1.274a}\flagstanza{\tiny\textenglish{...1.274a}}हन्ति सानुचरां तृष्णां सम्यग्दृष्टिः सुभाविता ॥\&[\smallbreak]


	
	  \endgroup
	

	  \pstart {\color{DodgerBlue3}“तत्र”} सत्यचतुष्टय {\color{DodgerBlue3}“एव म्यग्दृष्टिर्ने”}रात्म्यदृष्टिः {\color{DodgerBlue3}“तद्विरुद्धार्थतत्त्वाकारानुरोधिनी”} (२७३) तेषां स्थिरसुखाद्याकाराणामविद्यारोपितानां विरुद्धोऽर्थस्तस्य तत्त्वानि भूता आकारा अनित्या सुखादयः षोडशाकारास्ताननुरोद्धुं {\color{DodgerBlue3}“शीलं यस्याः”} सा तथा सुभाविता सादरनिरन्तरदीर्घकालाभ्यासप्राप्तवैशद्या {\color{DodgerBlue3}“हन्ति तृष्णां जन्म-”} हेतुं {\color{DodgerBlue3}“सानुचरां”} मात्सर्यादिपरिवारां ॥
	\pend
      

	  \begin{center}%% label @type='head'
	\textbf{(क) तृष्णाक्षयात् मोक्षः}
	\end{center}
	

	  \pstart ननु तृष्णाक्षयेपि कर्मदेहयोर्जन्महेत्वोर्भावात् जन्म किं न भवतीत्याह (।)
	\pend
      
	  \bigskip
	  \begingroup
	  \large
	
	    
	    \stanza[\smallbreak]
	\label{pv.1.274b}\edlabel{pv.1.274b}\flagstanza{\tiny\textenglish{...1.274b}}त्रिहेतोर्नोद्भवः कर्मदेहयोः स्थितयोरपि ॥ २७४ ॥\&[\smallbreak]


	
	  \endgroup
	
	  \bigskip
	  \begingroup
	  \large
	
	    
	    \stanza[\smallbreak]
	\label{pv.1.275a}\edlabel{pv.1.275a}\flagstanza{\tiny\textenglish{...1.275a}}एकाभावाद् विना बीजं नाङ्कुरस्येव सम्भवः ॥\&[\smallbreak]


	
	  \endgroup
	

	  \pstart त्रिहेतोस्तृष्णाकर्मदेहहेतोर्जन्मनस्तृष्णायाः क्षये कर्म{\color{DodgerBlue3}“देहयोः स्थितयोरपि नोद्भवः”} उत्पत्तिर्न भवति (२७४) {\color{DodgerBlue3}“एकाभावात्”} हेतुसामग्र्‏यवैकल्यात् । {\color{DodgerBlue3}“विना”} {\color{DodgerBlue3}“बीजं”} क्षित्युदकादिभावेपि {\color{DodgerBlue3}“नाङ्कुरस्योद्भवः”} ।
	\pend
      

	  \pstart a. ननु त्रिहेतोर्जन्मन एकाभावेपि चेदनुत्पत्तिः तत्किं क\edlabel{pvv.103-1}\footnote{\label{pvv.103-1}  १ तृष्णामुपेक्ष्याप्रहीणामेव ।}र्म्मणो देहस्य वा क्षयो नाभ्यस्यते । परे चाहुः कर्मक्षयान्मुक्तिरिति । अत्राह (।)
	\pend
      
	  \bigskip
	  \begingroup
	  \large
	
	    
	    \stanza[\smallbreak]
	\label{pv.1.275b}\edlabel{pv.1.275b}\flagstanza{\tiny\textenglish{...1.275b}}असम्भवाद् विपक्षस्य न हानिः कर्मदेहयोः ॥ २७५ ॥\&[\smallbreak]


	
	  \endgroup
	
	  \bigskip
	  \begingroup
	  \large
	
	    
	    \stanza[\smallbreak]
	\label{pv.1.276a}\edlabel{pv.1.276a}\flagstanza{\tiny\textenglish{...1.276a}}अशक्यत्वाच्च तृष्णायां स्थितायां पुनरुद्भवात् ॥\&[\smallbreak]


	
	  \endgroup
	

	  \pstart {\color{DodgerBlue3}“असम्भवाद्विपक्षस्य न हानिः कर्मदेहयोरस्ति”} । (२७५) {\color{DodgerBlue3}“अशक्यत्वाच्च”}। सत्यपि वा विपक्षे\edlabel{pvv.103-2}\footnote{\label{pvv.103-2}  २ बोद्धमशक्ये ।} तदभ्यासाद् देहकर्मनिवृत्तिरशक्यक्रिया । {\color{DodgerBlue3}“तृष्णायां स्थितायां”} तत्प्रचितस्यात्मग्रहवतो गर्भस्थानपरिग्रहे सति {\color{DodgerBlue3}“पुनरुद्भवात्”} देहस्य शरीरि{\color{DodgerBlue3}“णश्च”} तृष्णयैव प्रवृत्त्या सर्व्वत्र शुभाशुभप्रवृत्तेश्च (।)
	\pend
      

	  \pstart b. एवन्तर्हि तृष्णा कर्म च क्षपयितव्यं मुमुक्षुणेत्याह (।)
	\pend
      
	  \bigskip
	  \begingroup
	  \large
	
	    
	    \stanza[\smallbreak]
	\label{pv.1.276b}\edlabel{pv.1.276b}\flagstanza{\tiny\textenglish{...1.276b}}द्वयक्षयार्थं यत्ने च व्यर्थः कर्मक्षर्थ श्रमः ॥ २७६ ॥\&[\smallbreak]


	
	  \endgroup
	\leavevmode\marginnote{\textenglish{104/s}}

	  \pstart {\color{DodgerBlue3}“द्वयक्षयार्थं यत्ने”} क्रियमाणे {\color{DodgerBlue3}“कर्मक्षये व्यर्थः श्रमः”} । तृष्णाक्षयमात्राज्जन्माभावसिद्धेः सदपि कर्मानुपयुक्तमित्यलं तत्क्षयप्रयासेन । न च कर्मक्षयः प्रति\edlabel{pvv.104-1}\footnote{\label{pvv.104-1}  १ सत्यां तृष्णायां ।}पक्षाभावात् शक्य इत्युक्तं ।
	\pend
      

	  \begin{center}%% label @type='head'
	\textbf{(ख) अक्षीणकर्मणो न मोक्षः}
	\end{center}
	

	  \pstart सन्तापक्लेशोपभोगात् पूर्व्वार्जितकर्मक्षयोऽपरस्य चाकारणं ततो मुक्तिरित्यपि मो\edlabel{pvv.104-2}\footnote{\label{pvv.104-2}  २ दिगम्बरस्य ।}हः । (२७६)
	\pend
      \label{div_pvv.1.277}\edlabel{div_pvv.1.277}
	  
	% new div opening: depth here is 2
	

	  \pstart तथाहि (।)
	\pend
      
	  \bigskip
	  \begingroup
	  \large
	
	    
	    \stanza[\smallbreak]
	\label{pv.1.277}\edlabel{pv.1.277}\flagstanza{\tiny\textenglish{....1.277}}फलवैचित्र्यदृष्टेश्च शक्तिभेदोऽनुमीयते ।&कर्मणां तापसंक्लेशात् नैकरूपात् ततः क्षयः ॥ २७७ ॥\&[\smallbreak]


	
	  \endgroup
	

	  \pstart कर्मणां {\color{DodgerBlue3}“फलवैचित्र्यस्य\edlabel{pvv.104-3}\footnote{\label{pvv.104-3}  ३ भोगाभोगरोगारोग्यादिः ।}”} नानागत्युपभोग्याने कविधोपकरणसाध्यविविधसुखदुःखोपभोगप्रकारस्य {\color{DodgerBlue3}“दृष्टेश्च शक्तिभेदः”} सामर्थ्यनानात्वम{\color{DodgerBlue3}“नुमीयतेऽतो”} नानाप्रकारफलजननसामर्थ्यात् का\edlabel{pvv.104-4}\footnote{\label{pvv.104-4}  ४ उत्पित्सुश्लोकावताराय परमतं लिख्यते । स्यादेतत् (।) तृष्णा हि कर्मनिदानं तस्याश्च मार्ग्गात् क्षयो निदानिनोपि कर्मणः क्षयात् त्वयापि कर्मणां क्षय एवेष्टः । येपि दोषविरोधिन उपाया नैरात्म्यदर्शनलक्षणा इष्यन्ते । दोषनिदानं तृष्णान्ते उपघ्नन्ति । असत्यां तृष्णायां अनागतदोषानुत्पत्तिः । उत्पित्सुदोषनिर्घातात्ते शक्ता । अनेन चातीतकर्मनिदानतृष्णादिबाधको न भवत्येव मार्ग्गस्तदुत्पत्तिकाले तस्याभावादेवेत्यसिद्धिरुक्ता परे (रैः?) । यवगोधूमाद्यनेकबीजेष्वनेकाङकुरवत् ।}रणा{\color{DodgerBlue3}“देकरूपा”}त् फला{\color{DodgerBlue3}“त्तापसंक्लेशान्न कर्मणां क्षयः”} । (२७७)
	\pend
      \label{div_pvv.1.278}\edlabel{div_pvv.1.278}
	  
	% new div opening: depth here is 2
	

	  \pstart सर्व्वेषां कर्मणां ताप एव फलमिति चेत् ।
	\pend
      
	  \bigskip
	  \begingroup
	  \large
	
	    
	    \stanza[\smallbreak]
	\label{pv.1.278}\edlabel{pv.1.278}\flagstanza{\tiny\textenglish{....1.278}}फलं कथञ्चित् तज्जन्यमल्पं स्यान्न विजातिमत् ॥&अथाऽपि तपसः शक्त्या शक्तिसङ्करसंक्षयैः ॥ २७८ ॥\&[\smallbreak]


	
	  \endgroup
	

	  \pstart {\color{DodgerBlue3}“तस्य”} कर्मणो {\color{DodgerBlue3}“जन्यं फलं\edlabel{pvv.104-5}\footnote{\label{pvv.104-5}  ५ क्षेत्रासंस्कारेल्पफलवत् ।} कथञ्चित्”} जुगुप्सादिभिरल्पं {\color{DodgerBlue3}“स्यात् न”} तु {\color{DodgerBlue3}“विजाति\leavevmode\marginnote{\textenglish{20a/MA}} मत्”} । दानं दत्त्वा हिंसित्वा वा जुगुप्सुस्तयोः फलमल्पीयोऽनुभवति । कारणता न वापादानात् । न तु शुभस्यदुःखमशुभस्य वा सुखं फलं भवितुमर्हति ॥ {\color{DodgerBlue3}“अथापि तपसः शक्त्या शक्तिसंकरेण”} तापक्लेशमात्रफलेन तानि हीयन्ते । तपःशक्त्या कर्मणां संक्षयेण वा जन्माभावः । (२७८)
	\pend
      \label{div_pvv.1.279_1.280_1.281}\edlabel{div_pvv.1.279_1.280_1.281}
	  
	% new div opening: depth here is 2
	\leavevmode\marginnote{\textenglish{105/s}}
	  \bigskip
	  \begingroup
	  \large
	
	    
	    \stanza[\smallbreak]
	\label{pv.1.279a}\edlabel{pv.1.279a}\flagstanza{\tiny\textenglish{...1.279a}}क्लेशात् कुतश्चिद्धीयेताशेषमक्लेशलेशतः ॥\&[\smallbreak]


	
	  \endgroup
	

	  \pstart b. यच्च किञ्चिदवशिष्टं तत् {\color{DodgerBlue3}“क्लेशात् कुतश्चित्”} केशोल्लु{\color{DodgerBlue3}“ञ्चनादेः क्षीयते”} । कर्मक्षयाच्च मुक्तिः । अत्राह\edlabel{pvv.105-1}\footnote{\label{pvv.105-1}  १ अत्र शक्तेः क्षयः साङ्कर्यम्वेति पक्षौ द्वौ ।} (।)
	\pend
      

	  \pstart {\color{DodgerBlue3}“हीयेताशेषमक्लेशलेशतः”} । यदि तपसा कर्मक्षयोऽशेषं कर्म हीयेताक्लेशतो विनैव केशोल्लुञ्चनादिदुःखात्कर्मणः क्षीणत्वाद्यथा नारकादि दुःखं न भवति तथाऽल्पीयोपि न स्यात् । शक्तिसांकर्येपि लेशतः सन्तापक्लेशात् केवलात् कर्म हीयेत । न दुःखान्तरानुबन्धी संसारप्रबन्धः तपस्विनः स्यात् ।
	\pend
      
	  \bigskip
	  \begingroup
	  \large
	
	    
	    \stanza[\smallbreak]
	\label{pv.1.279b}\edlabel{pv.1.279b}\flagstanza{\tiny\textenglish{...1.279b}}यदीष्टमपरं क्लेशात् तत् तपः क्लेश एव चेत् ॥ २७९ ॥\&[\smallbreak]


	
	  \endgroup
	
	  \bigskip
	  \begingroup
	  \large
	
	    
	    \stanza[\smallbreak]
	\label{pv.1.280a}\edlabel{pv.1.280a}\flagstanza{\tiny\textenglish{...1.280a}}तत् कर्मफलमित्यस्मान्न शक्तेः सङ्करादिकम् ॥\&[\smallbreak]


	
	  \endgroup
	

	  \pstart तपसः शक्त्या शक्तिसङ्करसंक्षयश्च तदा वक्तुं शक्यो {\color{DodgerBlue3}“यदि स्यादिष्टं क्लेशादपर\edlabel{pvv.105-2}\footnote{\label{pvv.105-2}  २ केशोल्लुञ्चनादेरन्यत् कुशलरूपं तपोपीष्टं स्यात् ।}”}मन्यत्तपो नान्यथा । {\color{DodgerBlue3}“क्लेश एव चेत्तत्तपः”} । (२७९) {\color{DodgerBlue3}“तत्क्लेशरूपं”} तपः {\color{DodgerBlue3}“कर्मफलं”} (तपोऽवशेषितस्य कर्मणः फलमिष्ट) {\color{DodgerBlue3}“मित्यस्मात्कर्मफलभूतात्तप\edlabel{pvv.105-3}\footnote{\label{pvv.105-3}  ३ फलस्य कर्तृत्वायोगात् संकरक्षये ।}सः शक्तिसंकरादिकं न”} युक्तं । आदिशब्दात्संक्षयश्च ।
	\pend
      

	  \pstart C. तस्य\edlabel{pvv.105-4}\footnote{\label{pvv.105-4}  ४ यन्मते मार्ग्गात्तृष्णाक्षये कर्मणोपि क्षय इष्टः ।} मते तु (।)
	\pend
      
	  \bigskip
	  \begingroup
	  \large
	
	    
	    \stanza[\smallbreak]
	\label{pv.1.280b}\edlabel{pv.1.280b}\flagstanza{\tiny\textenglish{...1.280b}}उत्पित्सुदोषनिर्घाताद् येऽपि दोषविरोधिनः ॥ २८० ॥\&[\smallbreak]


	
	  \endgroup
	
	  \bigskip
	  \begingroup
	  \large
	
	    
	    \stanza[\smallbreak]
	\label{pv.1.281a}\edlabel{pv.1.281a}\flagstanza{\tiny\textenglish{...1.281a}}तज्जे कर्माणि शक्ताः स्युः कृतहानिः कथं भवेत् ॥\&[\smallbreak]


	
	  \endgroup
	

	  \pstart उत्पित्सोस्तृष्णादे{\color{DodgerBlue3}“र्दोष”}स्य {\color{DodgerBlue3}“निर्घातात् येपि दोषविरोधिनो”} नैरात्म्याभ्यासादय उपायाः (२८०) {\color{DodgerBlue3}“तज्जे”} तृष्णादिदोषप्रभवे {\color{DodgerBlue3}“कर्मणि”} कारणवारणद्वारेण व्याहन्तुं शक्ता न प्राग्जनिते तस्योत्प\edlabel{pvv.105-5}\footnote{\label{pvv.105-5}  ५ पूर्व्वकृतकमेक्षयो नेति न कृतस्य हानिः ।}न्नत्वात् । अतः {\color{DodgerBlue3}“कृतस्य”} कर्मणो {\color{DodgerBlue3}“हानिः कथम्भवेत्”} ।
	\pend
      

	  \pstart d. ननु यथा दोषेभ्यः कर्म तथा कर्मणो दोषाश्च भवन्तीत्यक्षीणकर्मणो न स्यात् मुक्तिरित्याह (।)
	\pend
      
	  \bigskip
	  \begingroup
	  \large
	
	    
	    \stanza[\smallbreak]
	\label{pv.1.281b}\edlabel{pv.1.281b}\flagstanza{\tiny\textenglish{...1.281b}}दोषा न कर्मणो दुष्टः करोति न विपर्ययात् ॥ २८१ ॥\&[\smallbreak]


	
	  \endgroup
	\leavevmode\marginnote{\textenglish{106/s}}

	  \pstart {\color{DodgerBlue3}“दोषा न कर्मणो”} भवन्ति किन्तु दोषै{\color{DodgerBlue3}“र्दुष्टः”} प्राणी कर्मकरो भवति {\color{DodgerBlue3}“न विपर्ययात्”} । नादुष्टः कर्म करोति नैरात्म्यदर्शिनस्तृष्णाभावात् । क्वचित् प्रवृत्तिनिवृत्त्यसम्भवात् । (२८१)
	\pend
      \label{div_pvv.1.282}\edlabel{div_pvv.1.282}
	  
	% new div opening: depth here is 2
	

	  \pstart ननु कर्मणः शुभात् सुखं सुखादभिलाषोऽभिलाषाच्च राग इति कर्मणो दोषजन्मेत्याह (।)
	\pend
      
	  \bigskip
	  \begingroup
	  \large
	
	    
	    \stanza[\smallbreak]
	\label{pv.1.282a}\edlabel{pv.1.282a}\flagstanza{\tiny\textenglish{...1.282a}}मिथ्याविकल्पेन विना नाभिलाषः सुखादपि ।\&[\smallbreak]


	
	  \endgroup
	

	  \pstart यश्च {\color{DodgerBlue3}“सुखादप्यभिलाषो”} रागहेतुर्दृश्यते स च {\color{DodgerBlue3}“मिथ्याविकल्पेन विना”} स्थिरसुखमदीयाहंकारविकल्पनमन्तरेण न भवतीत्यायोनिशोमनस्कार एव दोषहेतुः न कर्म । ततः सत्यपि कर्मण्युन्मूलितात्मदृष्टयो निर्दोषा निर्व्वान्ति । निर्व्वाणञ्च दुःखनिरोधलक्षणं । दुःखं परिज्ञाय तत्समुदयं मार्गभावनया प्रहाय प्राप्यते नान्यथेति ।
	\pend
      

	  \pstart चतुःसत्यप्रकाशक एव मुमुक्षूणामुपास्य इति तायित्वमुक्तं ।
	\pend
      

	  \begin{center}%% label @type='head'
	\textbf{ङ. तायात् सुगतत्वसिद्धिः}
	\end{center}
	

	  \pstart तदेवमनुलोमतः “>प्रमाणभूताये” त्यादि पञ्चपदानि व्याख्याय प्रतिलोमतो लिङ्गलैङ्गिकत्वं दर्शयन्नाह (।)
	\pend
      
	  \bigskip
	  \begingroup
	  \large
	
	    
	    \stanza[\smallbreak]
	\label{pv.1.282b}\edlabel{pv.1.282b}\flagstanza{\tiny\textenglish{...1.282b}}तायात् तत्त्वस्थिराशेषविशेषज्ञानसाधनम् ॥ २८२ ॥\&[\smallbreak]


	
	  \endgroup
	

	  \pstart {\color{DodgerBlue3}“तायाच्चतुः”}सत्यप्रकाशलक्षणात् लिङ्गात् {\color{DodgerBlue3}“तत्त्वस्थिराशेष”}विशेष{\color{DodgerBlue3}“ज्ञान”}स्य\edlabel{pvv.106-1}\footnote{\label{pvv.106-1}  १ तत्त्वञ्च स्थिरञ्चाशेषञ्च तैर्व्विशिष्यत इति विशेषज्ञानं ।} त्रिगुणस्य सुगतत्वस्य {\color{DodgerBlue3}“साध\edlabel{pvv.106-2}\footnote{\label{pvv.106-2}  २ चतुःसत्यविषयमशेषत्वं ।}नं”} सिद्धिः । तत्त्वस्य क्षणिकनैरात्म्यस्य ज्ञानात् प्रशस्तं । (२८२)
	\pend
      \label{div_pvv.1.283_1.284}\edlabel{div_pvv.1.283_1.284}
	  
	% new div opening: depth here is 2
	
	  \bigskip
	  \begingroup
	  \large
	
	    
	    \stanza[\smallbreak]
	\label{pv.1.283a}\edlabel{pv.1.283a}\flagstanza{\tiny\textenglish{...1.283a}}बोधार्थत्वाद् गमेः;\&[\smallbreak]


	
	  \endgroup
	

	  \pstart अपुनरावृत्त्या च स्थिरं निःशेषं विशेषज्ञानं त्रिगुणं सुगतत्वं {\color{DodgerBlue3}“बोधार्थत्वाद्”} गमेर्गतशब्दः प्रकृतिः । न हि सम्वादिनोऽसाक्षात्कृतस्यार्थस्य उपदेशः शक्यक्रियः (।) न चानुमानेन ज्ञातस्योपदेश इति युक्तमुक्तं । क्षणिकत्वनैरात्म्यादिविषयस्यानुमानस्य भगवदुपदेशमन्तरेणोत्पत्तिबीजाभावात् अगृहीतोपदेशानामभावात् । न च नित्यपरोक्षस्यार्थस्य स्थैर्यादिविपर्ययाध्यवसायिनः कश्चिन्निश्चयोऽस्ति । \leavevmode\marginnote{\textenglish{107/s}} तस्मात्प्रमाणसम्वादिनः परोक्षार्थस्योपदेशस्तत्साक्षात्कारपूर्व्वक एवेति युक्तं तायित्वात् सुगतत्वानुमानं भगवतः । स च भगवान् तायः सुगतत्वात् त्रिगुणात् गुणानुक्रमेण ।
	\pend
      
	  \bigskip
	  \begingroup
	  \large
	
	    
	    \stanza[\smallbreak]
	\label{pv.1.283b}\edlabel{pv.1.283b}\flagstanza{\tiny\textenglish{...1.283b}}बाह्यशैक्षाशैक्षाधिकस्ततः ॥\&[\smallbreak]


	
	  \endgroup
	

	  \pstart {\color{DodgerBlue3}“बाह्यशैक्षाशैक्षे”}भ्योऽ{\color{DodgerBlue3}“धिकः”} ये लौकिकभावनामार्गेण वीतरागा बाह्या अतत्वदर्शिनस्तेभ्यः तत्त्वदर्शित्वादधिकः । ये शैक्षा अबाह्याः परिहाणिधर्माणस्तेभ्योऽपुनरावृत्त्या । ये चाशैक्षाः श्रावका अप्रहीणक्लेशवासना असाक्षात्कृतसर्व्वाकारवस्तवस्तेभ्यो निःशेषप्रतीत्या । तस्मात् सुगतत्वात् शासनं शास्तृत्वमनुमीयते ।
	\pend
      

	  \pstart कि पुनः शासनमित्याह (।)
	\pend
      
	  \bigskip
	  \begingroup
	  \large
	
	    
	    \stanza[\smallbreak]
	\label{pv.1.283c}\edlabel{pv.1.283c}\flagstanza{\tiny\textenglish{...1.283c}}परार्थज्ञानघटनं तस्मात् तच्छासनं ततः ॥ २८३ ॥\&[\smallbreak]


	
	  \endgroup
	
	  \bigskip
	  \begingroup
	  \large
	
	    
	    \stanza[\smallbreak]
	\label{pv.1.284a}\edlabel{pv.1.284a}\flagstanza{\tiny\textenglish{...1.284a}}दयापरार्थतंत्रत्वं ;\&[\smallbreak]


	
	  \endgroup
	

	  \pstart {\color{DodgerBlue3}“तच्छासनं”} कारणे कार्योपचारात् {\color{DodgerBlue3}“परार्थं यज्ज्ञानं”} सुगतत्वं तदर्थं {\color{DodgerBlue3}“घटनं”} व्यायामः । बुद्धत्वसाधनमार्गाभ्यास इत्यर्थः । न ह्युपायमन्तरेणोपेयसम्भवः । {\color{DodgerBlue3}“ततः”} शासनाद् (२८३) {\color{DodgerBlue3}“दयापरार्थतन्त्रत्वं”} परार्थप्रधानत्वं जगद्धितैषित्वमनुमीयते इत्यर्थः ।
	\pend
      

	  \pstart ननु नावश्यं कारुणिकस्य मोक्षमार्गाभ्यासः । {\color{DodgerBlue3}“स्वार्थबुद्ध्यापि बाह्यानामिव”} सम्भवात् तत्कथमुपायाभ्यासाद्दयानुमानमित्याह (।)
	\pend
      
	  \bigskip
	  \begingroup
	  \large
	
	    
	    \stanza[\smallbreak]
	\label{pv.1.284b}\edlabel{pv.1.284b}\flagstanza{\tiny\textenglish{...1.284b}}सिद्धार्थस्याऽविरामतः ॥\&[\smallbreak]


	
	  \endgroup
	

	  \pstart {\color{DodgerBlue3}“सिद्धार्थस्य”} निष्पन्नमोक्षलक्षणात्मसम्वादस्यापि सुगतस्य खड्गादिवत् परार्थक्रियातो{\color{DodgerBlue3}“ऽविरामतो”}ऽनिवृत्तेः फलावस्थायान्दयासद्भावाद्धेत्ववस्थायामपि तस्यास्तित्वमनुमानञ्च ।
	\pend
      

	  \pstart जगद्धितैषित्वस्य सुगतत्वशास्तृत्वतायित्वसहितस्य प्रामाण्यसाधनत्वमाह (।)
	\pend
      
	  \bigskip
	  \begingroup
	  \large
	
	    
	    \stanza[\smallbreak]
	\label{pv.1.284c}\edlabel{pv.1.284c}\flagstanza{\tiny\textenglish{...1.284c}}दयया श्रेय आचष्टे;\&[\smallbreak]


	
	  \endgroup
	

	  \pstart यतो {\color{DodgerBlue3}“दयया”} जगद्धितैषित्वेन {\color{DodgerBlue3}“श्रेय आचष्टे”} । निर्दयस्तु विसम्वादनाभिप्रायोपि ब्रूयात् ।
	\pend
      

	  \pstart सदयोप्यभूतमज्ञो वक्तीत्याह (।)
	\pend
      
	  \bigskip
	  \begingroup
	  \large
	
	    
	    \stanza[\smallbreak]
	\label{pv.1.284d}\edlabel{pv.1.284d}\flagstanza{\tiny\textenglish{...1.284d}}ज्ञानात् सत्यं ससाधनम् ॥ २८४ ॥\&[\smallbreak]


	
	  \endgroup
	

	  \pstart {\color{DodgerBlue3}“ज्ञाना”}त् सुगतत्वात् भूतमाचष्टे । तच्च ज्ञानं {\color{DodgerBlue3}“ससाधनं”} विद्यमानोपायाभ्यासं विद्यमानशास्तृत्वमित्यर्थः । (२८४)
	\pend
      \label{div_pvv.1.285_1.286_1.287}\edlabel{div_pvv.1.285_1.286_1.287}
	  
	% new div opening: depth here is 2
	
	  \bigskip
	  \begingroup
	  \large
	
	    
	    \stanza[\smallbreak]
	\label{pv.1.285a}\edlabel{pv.1.285a}\flagstanza{\tiny\textenglish{...1.285a}}तच्चाभियोगवान् वक्तुं यतस्तस्मात् प्रमाणता ॥\&[\smallbreak]


	
	  \endgroup
	\leavevmode\marginnote{\textenglish{108/s}}

	  \pstart {\color{DodgerBlue3}“तच्च”} सत्यचतुष्टयं विनेयानां {\color{DodgerBlue3}“वक्तुमभियोगः”} सादरसततप्रवृत्तिस्तद्वान् तायी चेत्यर्थः । {\color{DodgerBlue3}“तस्मात्”} कारुणिकत्वात् सुगतत्वात् शास्तृत्वात् तायित्वाच्च भगवतः {\color{DodgerBlue3}“प्रमाणता”} यथोपदर्शितार्थसम्वादकताऽन्यैरज्ञातचतुःसत्यार्थप्रकाशकता वाऽनुमीयत इत्यर्थः ।
	\pend
      

	  \pstart एवञ्चानुमानानुमेयव्यवहारे स्थिते प्रामाण्यात्तायित्वं हितैषित्वादुपाया\leavevmode\marginnote{\textenglish{20b/MA}} भ्यासाच्च सुगतत्वम्भवतीत्युक्तं ।
	\pend
      

	  \begin{center}%% label @type='head'
	\textbf{(७) संवादकत्वात् भगवान् प्रमाणम्}
	\end{center}
	

	  \pstart कस्मा\edlabel{pvv.108-1}\footnote{\label{pvv.108-1}  १ “प्रमाणभूताये”त्येव स्तुतिपदमाह \cref{ps.1.1} । [“
	    \begin{verse}
	\label{ps.1.1}प्रमाणभूताय जगद्धितैषिणे प्रणम्य शास्त्रे सुगताय तायिने ।\\
	    प्रमाणसिद्धयै स्वमतात् समुच्चयः करिष्यते विप्रसृतादिहैकतः ॥ (१।१) ॥\\
	    
	    \end{verse}
	  ”]}त्पुनरनेकगुणसम्भारसम्भवेपि प्रामाण्येनैव भगवतः स्तुतिरित्याह (।)
	\pend
      
	  \bigskip
	  \begingroup
	  \large
	
	    
	    \stanza[\smallbreak]
	\label{pv.1.285b}\edlabel{pv.1.285b}\flagstanza{\tiny\textenglish{...1.285b}}उपदेशतथाभावस्तुतिस्तदुपदेशतः ॥ २८५ ॥\&[\smallbreak]


	
	  \endgroup
	
	  \bigskip
	  \begingroup
	  \large
	
	    
	    \stanza[\smallbreak]
	\label{pv.1.286a}\edlabel{pv.1.286a}\flagstanza{\tiny\textenglish{...1.286a}}प्रमाणतत्वसिध्यर्थं;\&[\smallbreak]


	
	  \endgroup
	

	  \pstart {\color{DodgerBlue3}“उपदेशस्य तथाभावः”} सम्वादकत्वं प्रामाण्यं । तेन {\color{DodgerBlue3}“स्तुति”}राचार्येण कृता । तस्य भगवत {\color{DodgerBlue3}“उपदेशतः”} (२८५) {\color{DodgerBlue3}“प्रमाण”}स्य {\color{DodgerBlue3}“तत्त्वं”} लक्षणं त{\color{DodgerBlue3}“त्सिध्य”}र्थं भ ग व द्दे शनायाः\edlabel{pvv.108-2}\footnote{\label{pvv.108-2}  २ सकाशात्प्रमाणयोः ।} {\color{DodgerBlue3}“प्रमाणविनिश्चयो”} नोत्प्रेक्षामात्रेणेत्याख्यातुमित्यर्थः ।
	\pend
      

	  \pstart a. ननु “नीलसमङ्गी पुरुषो नीलं जानाति नो तु नीलमि”ति ब्रुवता भगवता प्रत्यक्षं दर्शितं । अनुमानं नोक्तं । कथमागमात् प्रत्येतव्यमित्याह (।)
	\pend
      
	  \bigskip
	  \begingroup
	  \large
	
	    
	    \stanza[\smallbreak]
	\label{pv.1.286b}\edlabel{pv.1.286b}\flagstanza{\tiny\textenglish{...1.286b}}अनुमानेऽप्यवारणात् ।\&[\smallbreak]


	
	  \endgroup
	
	  \bigskip
	  \begingroup
	  \large
	
	    
	    \stanza[\smallbreak]
	\label{pv.1.286c}\edlabel{pv.1.286c}\flagstanza{\tiny\textenglish{...1.286c}}प्रयोगदर्शनाद् वाऽस्य;\&[\smallbreak]


	
	  \endgroup
	

	  \pstart {\color{DodgerBlue3}“अनुमानेप्यवारणादि”}ष्टिर्दर्शिता । “शून्याः परप्रवादा” इत्यादिना शाब्दादेरेव {\color{DodgerBlue3}“निषेधात् प्रयोग”}\edlabel{pvv.108-3}\footnote{\label{pvv.108-3}  ३ अन्यान्यप्याक्षिप्तानीति ग्राह्याणि स्युरित्याह ।}स्य परार्थानुमानरूपस्य {\color{DodgerBlue3}“दर्शनाद्वागमेऽस्या”}नुमानस्य निर्देशः कृत एव {\color{DodgerBlue3}“भगवता”} (।)
	\pend
      

	  \pstart तमेव प्रयोगमाह (।)
	\pend
      
	  \bigskip
	  \begingroup
	  \large
	
	    
	    \stanza[\smallbreak]
	\label{pv.1.286d}\edlabel{pv.1.286d}\flagstanza{\tiny\textenglish{...1.286d}}यत् किञ्चिदुदयात्मकम् ॥ २८६ ॥\&[\smallbreak]


	
	  \endgroup
	
	  \bigskip
	  \begingroup
	  \large
	
	    
	    \stanza[\smallbreak]
	\label{pv.1.287a}\edlabel{pv.1.287a}\flagstanza{\tiny\textenglish{...1.287a}}निरोधधर्मकं सर्वं तदित्यादावनेकधा ॥\&[\smallbreak]


	
	  \endgroup
	\leavevmode\marginnote{\textenglish{109/s}}

	  \pstart {\color{DodgerBlue3}“यत्किञ्चिदुदयात्मकं”} (२८६) {\color{DodgerBlue3}“तत्सर्वं निरोधधर्मकमित्यादावागमवाक्येऽनेकधा”} स्वभावादिलिङ्गजमनेकप्रकारमनुमानं दुश्यते ।
	\pend
      

	  \pstart b. कथं पुनरनेनानुमानमुक्तमित्याह (।)
	\pend
      
	  \bigskip
	  \begingroup
	  \large
	
	    
	    \stanza[\smallbreak]
	\label{pv.1.287b}\edlabel{pv.1.287b}\flagstanza{\tiny\textenglish{...1.287b}}अनुमानाश्रयो लिङ्गमविनाभावलक्षणम् ॥ २८७ ॥\&[\smallbreak]


	
	  \endgroup
	
	  \bigskip
	  \begingroup
	  \large
	
	    
	    \stanza[\smallbreak]
	\label{pv.1.288}\edlabel{pv.1.288}\flagstanza{\tiny\textenglish{....1.288}}व्याप्तिप्रदर्शनाद्धेतोः साध्येनोक्तञ्च तत् स्फुटम् ॥\&[\smallbreak]


	
	  \endgroup
	

	  \pstart {\color{DodgerBlue3}“अनुमानस्याश्रयः”} कारणं {\color{DodgerBlue3}“लिङ्ग”}म्वक्तव्यमनुमाननिर्देशार्थमन्यथाऽशक्यत्वात् ।
	\pend
      

	  \pstart किं लक्षणमित्याह (।) {\color{DodgerBlue3}“अविनाभावः”} साध्याव्यभिचारित्वं {\color{DodgerBlue3}“तल्लक्षणं”} यस्य तत्तथा । (२८७) स चाविनाभावो लिङ्गं लक्षणं {\color{DodgerBlue3}“हेतो”}रुदयधर्मकत्वस्य {\color{DodgerBlue3}“साध्येन”} निरोधधर्मकत्वेन {\color{DodgerBlue3}“व्याप्तेः प्रदर्शनात्”} “यत्किञ्चिदुदयधर्मकं {\color{DodgerBlue3}“तत्सर्व्वं निरोधधर्म-”} कमि” \edlabel{pvv.109-1}\footnote{\label{pvv.109-1}  १ महावग्गे १।१।८}त्यादिना {\color{DodgerBlue3}“स्फुटः”} प्रव्यक्तो दर्शित (:।)
	\pend
      

	  \pstart इत्यनुमानप्रामाण्यनिर्देशोपि भगवदुपज्ञमेव (।) तदेवं भगवानेव प्रमाणभूतस्तायी मुमुक्षुभिरुपास्यो नेतर इति दर्शनार्थमा चा र्ये ण {\color{DodgerBlue3}“तस्य स्तुतिरुक्तेति युक्तं”} ॥
	\pend
      
	    
	    \pstart
	    \begin{center}
	  आचार्य म नो र थ न न्दि कृतायां प्रमाणावर्त्तिकवृत्तौ प्रथमः परिच्छेदः ॥
	    \end{center}
	    \pend
	  
	  
	    
	    \endnumbering% ending numbering from div
	    \endgroup
	    
	  
	  
	% new div opening: depth here is 0
	
	    
	    \begingroup
	    \beginnumbering% beginning numbering from div depth=0
	    
	  
\chapter[{द्वतीयः परिच्छेदः: प्रत्यक्षम्}]{द्वतीयः परिच्छेदः\footnote{* द्रष्टव्यं परिशिष्टं १।५}: प्रत्यक्षम्}
	  
	% new div opening: depth here is 1
	

	  \pstart \leavevmode\marginnote{\textenglish{110/s}}प्रथमपरिच्छेदेन प्रमाणसामान्यलक्षणं व्यवस्थाप्य विशेषलक्षणमाख्यातुं द्वितीयपरिच्छेदारम्भः ।
	\pend
      
	  
	% new div opening: depth here is 1
	
\section[{(१. प्रमाणसंख्याविप्रतिपत्तिनिरासः)}]{(१. प्रमाणसंख्याविप्रतिपत्तिनिरासः)}\label{div_pvv.2.1_2.2}\edlabel{div_pvv.2.1_2.2}
	  
	% new div opening: depth here is 2
	

	  \pstart विप्रतिपत्तयश्चात्र संख्यालक्षणगोचरफलविषयाः सन्ति । तत्र संख्याविप्रतिपत्तिनिराकरणार्थमाह (।)
	\pend
      
	  \bigskip
	  \begingroup
	  \large
	
	    
	    \stanza[\smallbreak]
	\label{pv.2.1a}\edlabel{pv.2.1a}\flagstanza{\tiny\textenglish{...v.2.1a}}मानं द्विविधं विषयद्वैविध्यात्;\&[\smallbreak]


	
	  \endgroup
	

	  \pstart {\color{DodgerBlue3}“मानं द्विविधं”} । यत्तत्प्रमाणमविसम्वादित्वादज्ञातार्थप्रकाशकत्वात्सामान्य- लक्षणमुक्तं तद् द्विविधं । प्रत्यक्षानुमानभेदेन । कस्माद् (।) {\color{DodgerBlue3}“विषय”}स्य स्वलक्षणसामान्यलक्षणरूपतया {\color{DodgerBlue3}“द्वैविध्यात्”} । शाब्दादिकमपि हि प्रमाणम्भवत्सविषयं वक्तव्यं (।) विषयश्च स्वसामान्यलक्षणादतिरिक्तो नास्ति । ततस्तद्विषयत्वे प्रत्यक्षानुमानतैव । नापि सामान्यविशेषात्मक एकोस्ति विषयः\edlabel{pvv.110-1}\footnote{\label{pvv.110-1}  १ नीलोत्पलादि परस्य अर्थसामर्थ्ये स्वलक्षणत्वमेव ।} । परस्परविरुद्धयोरैकात्म्यायोगात् ।
	\pend
      

	  \pstart ननु कथं विषयद्वैविध्यसिद्धिः (।) न प्रत्यक्षान्नाप्यनुमानतो यथाक्रमं स्वलक्षणसामान्यलक्षणत्वादनयोः । द्वाभ्यां द्वयसिद्धिरिति चेत् ।
	\pend
      

	  \pstart ननु प्रत्यक्षस्य सामान्याविषयत्वे साध्यसाधनसम्बन्धाग्रहणादनुमानमेव न स्यात् । सामान्यविषयत्वे च प्रत्यक्षत्वादेव तत्सिद्धेर्व्विफलमनुमानस्य प्रामाण्यकल्पनं ।
	\pend
      

	  \pstart अत्रोच्यते । प्रत्यक्षमपि स्वलक्षणं विषयीकुर्व्वत् तत्सम्भवि विजातीयव्यावृत्त्युपकल्पितं सामान्यं पृष्ठविकल्पेन निश्चिन्वत्तद्विषयमपि निश्चयविषयेण \leavevmode\marginnote{\textenglish{111/s}} च प्रत्यक्षविषयव्यवस्था । एवन्तर्हि स्वलक्षणविषयता न स्यादिति चेत्\edlabel{pvv.111-1}\footnote{\label{pvv.111-1}  १ ननु यदतीन्द्रियं केशादिव्यवहितं स्वलक्षणं, यच्च सामान्यं न गोचरोनुमानस्य, तस्य कथं व्यवस्था । \par
“
	    \begin{verse}
	यज्जातीयैः प्रमाणैश्च यज्जातीयार्थदर्शनं ।\\
	    भवेदिदानीं लोकस्य तथा कालान्तरेष्वपी\\
	    
	    \end{verse}
	  ”ति \href{http://http://sarit.indology.info/?cref=śv}{[कुमारिल]वचनात्} ।} । न(।) सजातीयव्यावृत्तत्वेनापि ततो निश्चयात्। द्वे च व्यावृत्ती स्वलक्षणे स्तो निश्चिते च प्रत्यक्षबलात्। न चैवमप्यनुमानस्य वैयर्थ्य (।) न हि सामान्यमित्येव प्रत्यक्षविषयः । परोक्षे तस्याप्रवृत्तेः । न च यदेकदाऽपरोक्षं तत्सर्वदा तथा । स्वलक्षणं कदाचिदपरोक्षमप्यन्यदा परोक्षं एवं सामान्यमपि । ततोऽपरोक्षे सामान्ये गृहीतायां व्याप्तौ परोक्षे तस्मिन्ननुमानवृत्तिरिति न कश्चिद्विरोधः । तस्मात् प्रत्यक्षत्वाद्वा विषयद्वैविध्यसिद्धिः प्रत्यक्षानुमानाभ्यां {\color{DodgerBlue3}“वेत्युभय”}थाप्युपपन्नं ।
	\pend
      

	  \pstart विषयद्वैविध्यमेव कस्मादित्याह (।)
	\pend
      
	  \bigskip
	  \begingroup
	  \large
	
	    
	    \stanza[\smallbreak]
	\label{pv.2.1b}\edlabel{pv.2.1b}\flagstanza{\tiny\textenglish{...v.2.1b}}शक्त्यशक्तितः ।&अर्थक्रियायां;\&[\smallbreak]


	
	  \endgroup
	

	  \pstart {\color{DodgerBlue3}“शक्त्यशक्तितोऽर्थक्रियायां”} । स्वलक्षणस्यार्थक्रियाशक्तत्वात् । {\color{DodgerBlue3}“विजातीयव्या”}वृत्त्युपकल्पितस्य च सामान्यस्याशक्तत्वात् विषयद्वैविध्यं । न ह्येकस्य {\color{DodgerBlue3}“विरुद्धाविमौ”} धर्मौ युज्येते । यद्यनर्थक्रियाकारि सामान्यं {\color{DodgerBlue3}“केशोण्डुकज्ञानप्रतिभासि केशाद्यपि”} सामान्यं स्यात् ।
	\pend
      
	  \bigskip
	  \begingroup
	  \large
	
	    
	    \stanza[\smallbreak]
	\label{pv.2.1c}\edlabel{pv.2.1c}\flagstanza{\tiny\textenglish{...v.2.1c}}केशादिर्न्नार्थोनर्थाधिमोक्षतः ॥ १ ॥\&[\smallbreak]


	
	  \endgroup
	

	  \pstart न {\color{DodgerBlue3}“के\edlabel{pvv.111-2}\footnote{\label{pvv.111-2}  २ अनर्थक्रियातो न स्वलक्षणं, स्पष्टप्रतिभास्यनन्वयित्वाभ्यां न च सामान्यमिति विषयान्तरत्वमस्य ॥}शादिर”}र्थः सामान्यरूपोऽ{\color{DodgerBlue3}“नर्थाधिमोक्षतः”} । (१) यत्र हि व्यवहर्तॄ\edlabel{pvv.111-3}\footnote{\label{pvv.111-3}  ३ अथारोप्यैतत् अन्यथोत्पत्तिसारूप्याभ्यां विषयत्वे आकारो बहिर्भासिकेशादेर्नोत्पादको न सरूपको नापि सम्वित्तस्येति चोद्यानवकाश एव सांव्यवहारि प्रमाणमेतत् । देशादिविप्रकृष्टन्तु (।) “
	    \begin{verse}
	यज्जातीयैः प्रमाणैश्च यज्जातीयार्थदर्शनं ।\\
	    भवेदिदानीं लोकस्य तथा कालान्तरेष्वपि\\
	    
	    \end{verse}
	  ” \href{http://http://sarit.indology.info/?cref=śv}{[कुमारिलस्य]} ॥}णा-\leavevmode\marginnote{\textenglish{21a/MA}} मर्थाध्यवसायः सोऽर्थः स्वलक्षणं सामान्यम्वा स्यात् । यत्र पुनरर्थबुद्धिरेव नास्ति स कथं सामान्यमुच्यतां ।
	\pend
      \leavevmode\marginnote{\textenglish{112/s}}
	  \bigskip
	  \begingroup
	  \large
	
	    
	    \stanza[\smallbreak]
	\label{pv.2.2}\edlabel{pv.2.2}\flagstanza{\tiny\textenglish{...pv.2.2}}सदृशासदृशत्वाच्च विषयाविषयत्वतः ।&शब्दस्यान्यनिमित्तानां भावे धीः सदसत्वतः ॥ २ ॥\&[\smallbreak]


	
	  \endgroup
	

	  \pstart {\color{DodgerBlue3}“तथा\edlabel{pvv.112-1}\footnote{\label{pvv.112-1}  १ सांख्यमतेनाह ।} तदृशासदृशत्वाच्च”} विषयद्वैविध्यं । सदृशं सामान्यं सर्वव्यक्तिसाधारणत्वात् । असदृशं स्वलक्षणं सर्व्वतो व्यावृत्तत्वात् । अनयोश्चान्योन्यव्यवच्छेदरूपत्वात् न राश्यन्तरं । ततो यदि कल्प्यमानं सदृशं तदा सामान्यमेव तत् । अथासदृशं स्वलक्षणमेवेति द्वैविध्यमेव विषयस्य । तथा\edlabel{pvv.112-2}\footnote{\label{pvv.112-2}  २ ज्ञानद्वारेण निराकृत्य शब्दमुखेनाह ।} शब्दस्य विषयाविषयत्वतश्च द्वैविध्यं । {\color{DodgerBlue3}“शब्दस्य”} विषयः {\color{DodgerBlue3}“सामान्यं”} । अविषयः स्वलक्षणं । न च शब्दविषयाविषयाभ्या{\color{DodgerBlue3}“मन्यो”}स्ति सर्व्वस्य संग्रहात् द्वैविध्यमेव । तथा विषयाद{\color{DodgerBlue3}“न्यषान्निमित्तानां”} मनस्कारवत् साद्गुण्यसंकेतग्रहणानां {\color{DodgerBlue3}“भावे”} ग्राहिकाया धियः सा\edlabel{pvv.112-3}\footnote{\label{pvv.112-3}  ३ व्यवहर्तृ व्यवसायान्न वस्तुतः ।}मान्ये {\color{DodgerBlue3}“सत्त्वात्”} स्वल\edlabel{pvv.112-4}\footnote{\label{pvv.112-4}  ४ मनस्कारादीनां भावेपि यदभावे धियोऽभावस्तत्स्वलक्षणं ।}क्षणे चाभावात् विषयद्वैविध्यं । यत्र विषयव्यतिरिक्तनिमित्तसद्भावे भवति बुद्धिस्तत्सामान्यं (।) यत्र\edlabel{pvv.112-5}\footnote{\label{pvv.112-5}  ५ विषये सत्येव बुद्धिर्भवति ।}तु न भवति तत् स्वलक्षणं । प्रकारान्तरञ्च न सम्भवतीति बुद्धिविषयाविषयत्वे सामान्यस्वलक्षणतैवेति द्वैविध्यमेव विषयस्य । (२)
	\pend
      
	  
	% new div opening: depth here is 1
	
\section[{(२. सत्यद्वयचिन्ता)}]{(२. सत्यद्वयचिन्ता)}\label{div_pvv.2.3}\edlabel{div_pvv.2.3}
	  
	% new div opening: depth here is 2
	

	  \pstart तदेवार्थक्रियासामर्थ्यादिकं स्वलक्षणादौ योजयन्नाह ।
	\pend
      
	  \bigskip
	  \begingroup
	  \large
	
	    
	    \stanza[\smallbreak]
	\label{pv.2.3}\edlabel{pv.2.3}\flagstanza{\tiny\textenglish{...pv.2.3}}अर्थक्रियासमर्थ यत् तदत्र परमार्थसत् ।&अन्यत् संवृतिसत् प्रोक्तं; ते स्वसामान्यलक्षणो ॥ ३ ॥\&[\smallbreak]


	
	  \endgroup
	

	  \pstart {\color{DodgerBlue3}“अर्थक्रियायां”} ज्ञानादिकायां स्वरूपोपधानेन {\color{DodgerBlue3}“समर्थं यत्तदत्र”} वस्तुविचारो {\color{DodgerBlue3}“परमार्थसत्”} । एवं यदसदृशं शब्दाविषयोऽन्यनिमित्तभावे ज्ञानाभावश्च तत्परमार्थसत् । {\color{DodgerBlue3}“अतोऽन्यद”}शक्तं सदृशं शब्दविषयः । अन्यनिमित्तभावे बुद्धेर्व्विषयश्च तत् {\color{DodgerBlue3}“संवृतिसत् प्रोक्तं”} कल्पनामात्रव्यवहार्यत्वात् । ते परमार्थसंवृती {\color{DodgerBlue3}“स्वसामान्यलक्षणे”} । (३)
	\pend
      
	  
	% new div opening: depth here is 1
	
\section[{(३. सामान्यतत्कल्यनानिरासः)}]{(३. सामान्यतत्कल्यनानिरासः)}\label{div_pvv.2.4}\edlabel{div_pvv.2.4}
	  
	% new div opening: depth here is 2
	
	  \bigskip
	  \begingroup
	  \large
	
	    
	    \stanza[\smallbreak]
	\label{pv.2.4a}\edlabel{pv.2.4a}\flagstanza{\tiny\textenglish{...v.2.4a}}अशक्तं सर्वमिति चेद् बीजादेरङ्कुरादिषु ।&दृष्टा शक्तिः;\&[\smallbreak]


	
	  \endgroup
	

	  \pstart स्वलक्षणसामान्यलक्षणे इष्टे ॥ {\color{DodgerBlue3}“सर्व्व”}मर्थकारित्वेने\edlabel{pvv.112-6}\footnote{\label{pvv.112-6}  ६ माध्यमिको सिद्धतामाह ।}ष्ट{\color{DodgerBlue3}“मशक्तं”} । न किञ्चित् कर्त्तुं सम{\color{DodgerBlue3}“र्थमिति चेत् । बीजादेः”} कारणाभिमतस्या{\color{DodgerBlue3}“ङ्कुरादौ”} कार्यसंमते {\color{DodgerBlue3}“दृष्टा शक्ति-”} \leavevmode\marginnote{\textenglish{113/s}} र्जननलक्षणा । बीजान्वयव्यतिरेकानुविधाय्यङ्कुरो दृश्यते (।) इदमेव कारणस्य शक्तत्वं यत्प्रागदृष्टस्य तद्भाव एव भावः ॥
	\pend
      
	  \bigskip
	  \begingroup
	  \large
	
	    
	    \stanza[\smallbreak]
	\label{pv.2.4b}\edlabel{pv.2.4b}\flagstanza{\tiny\textenglish{...v.2.4b}}मता सा चेत् संवृत्या;\&[\smallbreak]


	
	  \endgroup
	

	  \pstart सा शक्तिः {\color{DodgerBlue3}“संवृत्या मता चेत्”} कार्यकारणभावो हि व्यवहारमात्रतः सिद्धः न परमार्थतः । न तावत्प्रत्यक्षं तद्ग्रहणसमर्थं बीजाङ्कुरग्राहिणोः प्रत्यक्षयोः स्वविषयमात्रव्यवस्थापनात् केनान्वयव्यतिरेकग्रहणं । क्रमेण द्वयोर्गृहीतयोस्तद्बलभाविना स्मरणेन ग्रहणमिति चेत् ।
	\pend
      

	  \pstart ननु केनान्वयव्यतिरेकौ गृहीतौ । न प्रत्येकं बीजाङ्कुरज्ञानाभ्यां स्वस्वविषयग्रहणात् । नापि द्वाभ्यां ज्ञानयोर्ज्ञेययोश्च साहित्याभावात् । असाहित्ये बीजाङ्कुरमात्रस्य ग्रहणं नान्वयव्यतिरेकयोः । क्रमग्रहणमेव कार्यकारणभावग्रहणम् । तत्तु घटकुलालयोरप्यस्ति इति चेत् (।) न च क्रमोपि केनचिच्छक्यग्रहणः\edlabel{pvv.113-1}\footnote{\label{pvv.113-1}  १ स्वरूपमात्रवेदनात् ।}प्रतियोग्यवेदना\edlabel{pvv.113-2}\footnote{\label{pvv.113-2}  २ अक्रमवेद(ना)पेक्षत्वात् ।}त् । पूर्वापरग्रहणमत एव नास्ति स्वज्ञानेन वर्तमानताग्रहणाच्च । कार्यकाले च कारणं पूर्व्वमुच्यते तदा च तदेव नास्ति । तदेतन्मृतस्यारोग्यं । अथ यदैव बीजं तदैवाङ्कुरात् पूर्व्वं न तु पश्चात् अस्य पूर्व्वत्वं सम्भवति ॥
	\pend
      

	  \pstart नन्वेवं पूर्व्वतया प्रतिभासोस्य प्राप्तः । न चैतदस्ति । अङ्कुरसाहित्यं पूर्व्वत्वं तच्च गृह्यत एवेति चेत् । तादृशं पूर्व्वत्वमन्येषामप्यस्तीति तेपि कारणानि स्युः । किञ्च (।) पूर्व्वत्वं वर्तमानकालात्प्राग्भावित्वमुच्यते (।) तद्यदि वस्तुनो रूपं तदा वर्तमानं कदापि न स्यात् । वर्तमानतातत्प्राग्भावित्वयोर्व्विरोधात् । स्यादेतदङ्कुरवर्तमानतायाः प्राग्भावित्वं पूर्व्वत्वं । तच्च बीजस्य वर्तमानत्वेनाविरुद्धमिति न तस्याभावः । एवन्तर्हि बीजग्रहणे पूर्व्वताग्रहणं प्राप्तं । न च बीजस्वरूपमिव पूर्व्वतामपि तद्ग्राहिणि ज्ञाने कश्चिदुपलभते ।
	\pend
      

	  \pstart न\edlabel{pvv.113-3}\footnote{\label{pvv.113-3}  ३ परः ।}न्वापेक्षिकमिदं पूर्व्वत्वं प्रतियोगिनोऽप्रतीतौ कथं प्रतीयतां । य\edlabel{pvv.113-4}\footnote{\label{pvv.113-4}  ४ माध्यमिकः ।}द्येवं वस्तुनो नेदं केवलं भावान्तरापेक्षया व्यवह्रियत इत्या(या)तं(।) न हि वस्तुरूपं सति वस्तुनि\leavevmode\marginnote{\textenglish{21b/MA}} नास्ति (।) यच्चास्ति तेन वस्तुरूपावभासिनि ज्ञाने प्रतिभासितव्यमेव । अन्यथा ज्ञानज्ञेययोर्व्विषयविषयितैव न स्यात् (।) तस्मान्नाध्यक्षात् कार्यकारणताग्रहः । अतश्च नानुमानादपि । न हि सर्व्वदा परोक्षेऽन्येऽनुमानवृत्तिः व्याप्तिग्रहणपूर्व्वकत्वात्तस्य । नापि प्रत्यक्षबलभाविस्मरणं तद्ग्रहणप्रवणं(।)न हि तत्स्वतन्त्रं प्रमाणं किन्तु प्रमाण\leavevmode\marginnote{\textenglish{114/s}} व्यापारव्यवस्थापकं । यदि यथानुभवं प्रवर्तते नान्यथा । न च कार्यकारणभावानुभवो भूत इत्युक्तं । अतः स्मरणत्वमप्यस्य नास्ति । अनूभूतविकल्पनस्य स्मरणत्वात् (।) ततो विकल्पमात्रमेतत् न ततो वस्तुव्यवस्थेति संवृत्त्यैवाविचारितरमणीयया कार्यकारणभावव्यवहारो न परमार्थतः ।
	\pend
      
	  \bigskip
	  \begingroup
	  \large
	
	    
	    \stanza[\smallbreak]
	\label{pv.2.4c}\edlabel{pv.2.4c}\flagstanza{\tiny\textenglish{...v.2.4c}}अस्तु यथा तथा ॥ ४ ॥\&[\smallbreak]


	
	  \endgroup
	

	  \pstart {\color{DodgerBlue3}“अस्तु\edlabel{pvv.114-1}\footnote{\label{pvv.114-1}  १ सिद्धान्ती(---)अर्थक्रिया तावदस्ति । अन्यथा जगद्व्यालोपः ।} यथा तथा”} । सांवृतमपि कार्यकारणभावमाश्रित्य साध्यसाधनादिव्यवहारसम्वादसम्प्रत्ययात् समाप्तो लोकव्यवहारः । सांव्यवहारिकञ्च प्रमाणं तावतैव सुस्थं । (४)
	\pend
      \label{div_pvv.2.5}\edlabel{div_pvv.2.5}
	  
	% new div opening: depth here is 2
	

	  \begin{center}%% label @type='head'
	\textbf{(१) सामान्यचिन्ता}
	\end{center}
	

	  \begin{center}%% label @type='head'
	\textbf{(क) सामान्ये नार्थक्रियासामर्थ्यम्}
	\end{center}
	
	  \bigskip
	  \begingroup
	  \large
	
	    
	    \stanza[\smallbreak]
	\label{pv.2.5}\edlabel{pv.2.5}\flagstanza{\tiny\textenglish{...pv.2.5}}सास्ति सर्वत्र चेद् बुद्धेर्नान्वयव्यतिरेकयोः ।&सामान्यलक्षणेऽदृष्टेः चक्षूरूपादिबुद्धिवत् ॥ ५ ॥\&[\smallbreak]


	
	  \endgroup
	

	  \pstart {\color{DodgerBlue3}“सार्थ”}क्रिया {\color{DodgerBlue3}“सर्व्वत्र”} स्वलक्षणे सामा\edlabel{pvv.114-2}\footnote{\label{pvv.114-2}  २ चक्षुःश्रोत्रादिरूपशब्दादि ।}न्येप्य{\color{DodgerBlue3}“स्ति चेत् न”} सामान्यलक्षणे । अस्तु तावदन्यस्य कार्यस्य {\color{DodgerBlue3}“बुद्धेरप्यन्वयव्यतिरेक”}योरदृष्टेः । अन्त्यं हि भावानां कार्यं बुद्धिः । सापि यदन्वयव्यतिरेकौ नानुविधत्ते (।) तस्य कुतोऽर्थक्रिया । प्रत्यक्षं तावन्न {\color{DodgerBlue3}“सामान्य”}विषयं {\color{DodgerBlue3}“स्वलक्षण”}मात्रस्य प्रतिभासनात् । घट इत्यादिविकल्पबुद्धिस्तु सामान्यावसायात् त\edlabel{pvv.114-3}\footnote{\label{pvv.114-3}  ३ सामान्यविषया ।}द्विषया । सापि समयाभो\edlabel{pvv.114-4}\footnote{\label{pvv.114-4}  ४ चित्ताभोगो मनस्कारः ।}गादिमात्रादुत्पत्तेर्न सामान्यान्वयव्यतिरेकानुविधायिनी । {\color{DodgerBlue3}“चक्षूरूपादिबुद्धिवत्”} चक्षूरूपादिजबुद्धेः । चक्षुराद्यन्वयव्यतिरेकानुविधानमिव सामर्थ्ये सति भावादेकापायेप्यभावाच्च । नैवं सामान्यबुद्धिराभोगमात्रादुत्पत्तेः । (५)
	\pend
      \label{div_pvv.2.6}\edlabel{div_pvv.2.6}
	  
	% new div opening: depth here is 2
	
	  \bigskip
	  \begingroup
	  \large
	
	    
	    \stanza[\smallbreak]
	\label{pv.2.6}\edlabel{pv.2.6}\flagstanza{\tiny\textenglish{...pv.2.6}}एतेन समयाभोगाद्यन्तरङ्गानुरोधतः ।&घटोत्क्षेपणसामान्यसंख्यादिषु धियो गताः ॥ ६ ॥\&[\smallbreak]


	
	  \endgroup
	

	  \pstart {\color{DodgerBlue3}“एतेन”} सामान्यस्य ज्ञानमात्रकार्येप्यसामर्थ्यकथनेन {\color{DodgerBlue3}“समयाभोगादे”}रन्तरङ्गस्या{\color{DodgerBlue3}“नुरोधतः । घटो”}ऽवयवि द्रव्यं । {\color{DodgerBlue3}“उत्क्षेपणं”} कर्म । {\color{DodgerBlue3}“सामान्यं संख्या”} गुणः समवायः {\color{DodgerBlue3}“आदि”} शब्दात्संयोगविभागादयः । तेषु या {\color{DodgerBlue3}“धियो”} विकल्पिका भवन्ति ता {\color{DodgerBlue3}“गता”} व्याख्याताः । \leavevmode\marginnote{\textenglish{115/s}} न हि रूपा\edlabel{pvv.115-1}\footnote{\label{pvv.115-1}  १ ननु प्रत्यक्षगम्यमेवोत्क्षेपणादीत्याह (।)}दिव्यतिरिक्तं द्रव्यं । अपरापरदेशिजन्यहस्तादिक्षणाद् भिन्नं कर्म, व्यक्तिव्यतिरेकि सामान्यं प्रत्यक्षबुद्धावभासते । विकल्पबुद्धिरस्तु कल्पिका । सा च\edlabel{pvv.115-2}\footnote{\label{pvv.115-2}  २ समयाभोगं व्याचष्टे ।} संकेतसँस्कारप्रबोधमात्रभाविनी नार्थाधीनेति न ते\edlabel{pvv.115-3}\footnote{\label{pvv.115-3}  ३ अर्थानामनुमाजनने ।}षां सामर्थ्यं समर्थयति ॥ (६)
	\pend
      \label{div_pvv.2.7}\edlabel{div_pvv.2.7}
	  
	% new div opening: depth here is 2
	

	  \pstart ननु यद्यर्थनिरपेक्षाणामन्यनिमित्तानां भावे यत्र बुद्धिस्तत्सामान्यं तदा तैमिरिकबुद्धिप्रतिभासिनः केशादयोपि सामान्यं स्तुः । तेष्वपि धियोऽर्थनिरपेक्षचक्षुरादिमात्रनिमित्तत्वादित्याह ।
	\pend
      
	  \bigskip
	  \begingroup
	  \large
	
	    
	    \stanza[\smallbreak]
	\label{pv.2.7a}\edlabel{pv.2.7a}\flagstanza{\tiny\textenglish{...v.2.7a}}केशादयो न सामान्यमनर्थाभिनिवेशतः ॥\&[\smallbreak]


	
	  \endgroup
	

	  \pstart {\color{DodgerBlue3}“केशादय”}स्तैमिरिकप्रतिभासिनो {\color{DodgerBlue3}“न सामान्यमनर्थाभिनिवेश\edlabel{pvv.115-4}\footnote{\label{pvv.115-4}  ४ प्रत्यक्षविषयेप्यर्थाभिनिवेशमात्रं नार्थ इति सूचनया मध्याप्रवेशमाह ।}तः”} । न हि तेषु विषयबुद्धिर्व्यवहारिणां विषये च सामान्यचिन्तेयं । यदि ज्ञेयत्वेनाध्यवसायाभावान्न सामान्यं तदाऽभावोप्यनुपलब्धिविषयः {\color{DodgerBlue3}“सामान्यं न स्यात्”} ।---
	\pend
      

	  \pstart ततश्चानुपलब्धिरनुमानं न भवेदित्याह (।)
	\pend
      
	  \bigskip
	  \begingroup
	  \large
	
	    
	    \stanza[\smallbreak]
	\label{pv.2.7b}\edlabel{pv.2.7b}\flagstanza{\tiny\textenglish{...v.2.7b}}ज्ञेयत्वेन ग्रहाद् दोषो नाभावेषु प्रसज्यते ॥ ७ ॥\&[\smallbreak]


	
	  \endgroup
	

	  \pstart {\color{DodgerBlue3}“नाभावेषु”} सामान्यरूपत्वाभाव{\color{DodgerBlue3}“दोषः प्रसज्यते ज्ञेयत्वेन ग्रहात्”} । यद्यप्यभावेऽर्थरूपताया अध्यवसायो नास्ति ज्ञे\edlabel{pvv.115-5}\footnote{\label{pvv.115-5}  ५ भावव्यावृत्तिः ।}यतया त्वस्ति लोकस्य (।) {\color{DodgerBlue3}“तच्च ज्ञेयं”} यदर्थकारि तत्सामान्यमेव । (७)
	\pend
      \label{div_pvv.2.8_2.9}\edlabel{div_pvv.2.8_2.9}
	  
	% new div opening: depth here is 2
	

	  \pstart यदि तु (।)
	\pend
      
	  \bigskip
	  \begingroup
	  \large
	
	    
	    \stanza[\smallbreak]
	\label{pv.2.8a}\edlabel{pv.2.8a}\flagstanza{\tiny\textenglish{...v.2.8a}}तेषामपि तथाभावेऽप्रतिषेधात्;\&[\smallbreak]


	
	  \endgroup
	

	  \pstart {\color{DodgerBlue3}“तेषां”} तैमिरिकगम्यानां केशादीना{\color{DodgerBlue3}“मपि तथाभावे”} ज्ञानान्तरेण ज्ञेयत्वे समानरूपतेष्यते तदाऽ{\color{DodgerBlue3}“प्रतिषेधादि”}ष्टमेवास्माकं । तैमिरिकज्ञानगम्याः केशादय इति यदा बुद्ध्यन्तरेण परामृश्यन्ते तदा केशादयो ज्ञेयत्वेन सामान्यमिष्टा एव ॥
	\pend
      

	  \pstart यदि तैमिरिकदृष्टाः केशादयो न वस्तु तदा {\color{DodgerBlue3}“स्फुटाभता”} न स्यादित्याह ।
	\pend
      
	  \bigskip
	  \begingroup
	  \large
	
	    
	    \stanza[\smallbreak]
	\label{pv.2.8b}\edlabel{pv.2.8b}\flagstanza{\tiny\textenglish{...v.2.8b}}स्फुटाभता ।&ज्ञानरूपतयार्थत्वात्;\&[\smallbreak]


	
	  \endgroup
	

	  \pstart {\color{DodgerBlue3}“ज्ञानरूपतया”} ज्ञानाकारतया केशादीनामर्थत्वात्। स्वलक्षणत्वात्स्फुटाभता ॥
	\pend
      \leavevmode\marginnote{\textenglish{22a/MA}}

	  \pstart यदि ज्ञानाकारतया केशादयः स्वलक्षणन्तदा सामान्यं कथमित्याह (।)
	\pend
      \leavevmode\marginnote{\textenglish{116/s}}
	  \bigskip
	  \begingroup
	  \large
	
	    
	    \stanza[\smallbreak]
	\label{pv.2.8c}\edlabel{pv.2.8c}\flagstanza{\tiny\textenglish{...v.2.8c}}केशादीति मतिः पुनः ॥ ८ ॥\&[\smallbreak]


	
	  \endgroup
	
	  \bigskip
	  \begingroup
	  \large
	
	    
	    \stanza[\smallbreak]
	\label{pv.2.9a}\edlabel{pv.2.9a}\flagstanza{\tiny\textenglish{...v.2.9a}}सामान्यविषया;\&[\smallbreak]


	
	  \endgroup
	

	  \pstart {\color{DodgerBlue3}“केशादीति पुन”}र्व्विकल्पिका {\color{DodgerBlue3}“मतिः”} तैमिरिकोपलब्धकेशाद्यध्यवसायिनी {\color{DodgerBlue3}“सामान्यविषया”} । अयमर्थः (।) तैमिरिकदृष्टं केशादि सामान्यरूपेण व्यवस्यन्ती विकल्पिका बुद्धिरध्यवसायेन विषयेण सामान्यविषया । तैमिरिकधीस्तु स्वलक्षणविषया बुद्ध्याकारस्य स्वलक्षणत्वात् ॥ (८)
	\pend
      

	  \pstart किन्तर्हि निर्व्विषयमित्याह (।)
	\pend
      
	  \bigskip
	  \begingroup
	  \large
	
	    
	    \stanza[\smallbreak]
	\label{pv.2.9b}\edlabel{pv.2.9b}\flagstanza{\tiny\textenglish{...v.2.9b}}केशप्रतिभासमनर्थकम् ।\&[\smallbreak]


	
	  \endgroup
	

	  \pstart तैमिरिक{\color{DodgerBlue3}“केशप्रतिभासं”} ज्ञान{\color{DodgerBlue3}“मनर्थ”}कम्बाह्यकेशाभावात् ।
	\pend
      
	  \bigskip
	  \begingroup
	  \large
	
	    
	    \stanza[\smallbreak]
	\label{pv.2.9c}\edlabel{pv.2.9c}\flagstanza{\tiny\textenglish{...v.2.9c}}ज्ञानरूपतयार्थत्वे सामान्ये चेत् प्रसज्यते ॥ ९ ॥\&[\smallbreak]


	
	  \endgroup
	

	  \pstart ननु {\color{DodgerBlue3}“ज्ञानरूपतयार्थत्वे”} केशादीति विकल्पबुद्धिप्रतिभासिनि {\color{DodgerBlue3}“सामान्ये”} स्वलक्षणता {\color{DodgerBlue3}“प्रसज्यते चेत्”} (। ९)
	\pend
      \label{div_pvv.2.10}\edlabel{div_pvv.2.10}
	  
	% new div opening: depth here is 2
	
	  \bigskip
	  \begingroup
	  \large
	
	    
	    \stanza[\smallbreak]
	\label{pv.2.10a}\edlabel{pv.2.10a}\flagstanza{\tiny\textenglish{....2.10a}}तथेष्टत्वाददोषः;\&[\smallbreak]


	
	  \endgroup
	

	  \pstart तथा ज्ञानाकारतया सामान्यस्यापि स्वलक्षणताया {\color{DodgerBlue3}“इष्टत्वाददोषः”} ॥
	\pend
      

	  \begin{center}%% label @type='head'
	\textbf{(ख) सामान्यस्य ज्ञानाकारता}
	\end{center}
	

	  \pstart कथन्तर्हि सामान्यरूपतेत्याह (।)
	\pend
      
	  \bigskip
	  \begingroup
	  \large
	
	    
	    \stanza[\smallbreak]
	\label{pv.2.10b}\edlabel{pv.2.10b}\flagstanza{\tiny\textenglish{....2.10b}}अर्थरूपत्वेन समानता ।\&[\smallbreak]


	
	  \endgroup
	

	  \pstart अर्थरूपत्वेनाध्यवसीयमानज्ञेयरूपत्वेन समानता न तु बुद्ध्याकारत्वेन ।
	\pend
      

	  \begin{center}%% label @type='head'
	\textbf{(ग) विजातीयव्यावर्तकं सामान्यम्}
	\end{center}
	

	  \pstart {\color{DodgerBlue3}“कथं पुनरर्थरूपत्वे”}नापि {\color{DodgerBlue3}“समानते”}त्याह (।)
	\pend
      
	  \bigskip
	  \begingroup
	  \large
	
	    
	    \stanza[\smallbreak]
	\label{pv.2.10c}\edlabel{pv.2.10c}\flagstanza{\tiny\textenglish{....2.10c}}सर्वत्र समरूपत्वात् तद्व्यावृत्तिसमाश्रयात् ॥ १० ॥\&[\smallbreak]


	
	  \endgroup
	

	  \pstart {\color{DodgerBlue3}“सर्व्वत्र”} व्यक्तिषु {\color{DodgerBlue3}“तद्व्यावृत्तिसमाश्रयात्”} विजातीयव्यावृत्त्याश्रयेण दृश्यविकल्प्यैकत्वाध्यवसायादुपकल्पितस्य सामान्यस्य {\color{DodgerBlue3}“समरूपत्वात्”} साधारणरूपत्वादर्थत्वेन सामान्यमुच्यते ज्ञानाकारस्य सामान्यस्य संप्रति व्यतिरिक्तस्य\edlabel{pvv.116-1}\footnote{\label{pvv.116-1}  १ ज्ञानाकारात् ।}च पूर्व्वं प्रतिषेधात् ॥ (१०)
	\pend
      \leavevmode\marginnote{\textenglish{117/s}}\label{div_pvv.2.11}\edlabel{div_pvv.2.11}
	  
	% new div opening: depth here is 2
	

	  \pstart रूपादिस्वभावमेवाविशेषेण सामान्यं किन्नेष्यत इत्याह (।)
	\pend
      
	  \bigskip
	  \begingroup
	  \large
	
	    
	    \stanza[\smallbreak]
	\label{pv.2.11a}\edlabel{pv.2.11a}\flagstanza{\tiny\textenglish{....2.11a}}न तद्वस्त्वभिधेयत्वात् साफल्यादक्षसंहतेः ॥\&[\smallbreak]


	
	  \endgroup
	

	  \pstart तत्सामान्यं न {\color{DodgerBlue3}“वस्तुरू”}पादिस्वभाव{\color{DodgerBlue3}“मभिधेयत्वात्”} । शब्द\edlabel{pvv.117-1}\footnote{\label{pvv.117-1}  १ शब्दस्यायन्धर्मो यदुत शब्दज्ञानगोचरत्वं ।}धर्मवत् {\color{DodgerBlue3}“सामान्यं”} शब्दज्ञानगोचरः । न च शब्दविषयो वस्तु (।) कस्मादित्याह (।) {\color{DodgerBlue3}“साफल्यादक्षसंहतेः”} । यदि शब्दविषयो वस्तु भवेत् तदा रूपादिशब्दादेवार्थादेरपि रूपादिप्रतीतौ न किञ्चिदक्षैः । न चैवं । ततो वस्तुविषयेणेन्द्रियज्ञानेन शब्दस्य न तुल्यविषयता । तथा चावस्तुतैव शाब्दज्ञानावभासिनः सामान्यस्य न रूपादिता ॥
	\pend
      

	  \pstart न\edlabel{pvv.117-2}\footnote{\label{pvv.117-2}  २ वैभाषिकमतमाशङ्कते ।}नु रूपादयो न शब्दस्य विषयः किन्तर्हि ना\edlabel{pvv.117-3}\footnote{\label{pvv.117-3}  ३ विज्ञानवाद्यतिरिक्ते ।}मनिमित्ते विप्रयुक्तसंस्कारसंज्ञे ते विज्ञानाद् व्यतिरिक्ते वै भा षि क स्येष्टे । तदापि (।)
	\pend
      
	  \bigskip
	  \begingroup
	  \large
	
	    
	    \stanza[\smallbreak]
	\label{pv.2.11b}\edlabel{pv.2.11b}\flagstanza{\tiny\textenglish{....2.11b}}नामादिवचने वक्तृश्रोतृवाच्यानुबन्धिनि ॥ ११ ॥\&[\smallbreak]


	
	  \endgroup
	

	  \pstart {\color{DodgerBlue3}“नामादिवचन”} इष्यमाणे किन्तन्नामादिकं वक्तरि श्रोतर्यर्थे वा {\color{DodgerBlue3}“सम्बद्धम-”} सम्बद्धमेव वा क्वचित्\edlabel{pvv.117-4}\footnote{\label{pvv.117-4}  ४ इति चत्वारो विकल्पाः ।} । सर्व्वथा {\color{DodgerBlue3}“वक्तृश्रोतृवाच्यानुबन्धिनि”}\edlabel{pvv.117-5}\footnote{\label{pvv.117-5}  ५ वक्तृसम्बन्धिनि । श्रोतृसम्बन्धिनि । अर्थसम्बन्धिनि ।}। (११)
	\pend
      \label{div_pvv.2.12}\edlabel{div_pvv.2.12}
	  
	% new div opening: depth here is 2
	
	  \bigskip
	  \begingroup
	  \large
	
	    
	    \stanza[\smallbreak]
	\label{pv.2.12a}\edlabel{pv.2.12a}\flagstanza{\tiny\textenglish{....2.12a}}असम्बन्धिनि नामादावर्थे स्यादप्रवर्त्तनम् ॥\&[\smallbreak]


	
	  \endgroup
	

	  \pstart {\color{DodgerBlue3}“असम्बन्धिनि वा नामादौ”} शब्देन चोदिते{\color{DodgerBlue3}“ऽर्थेऽप्रवर्तनं स्याद”}चोदितत्वात् ॥\edlabel{pvv.117-6}\footnote{\label{pvv.117-6}  ६ वै॰ ।}
	\pend
      

	  \pstart भवतु तावदन्यसम्बन्धिनि असम्बन्धिनि वा नामादावर्थेऽप्रवर्तनं । अर्थसम्बन्धिनि तु कथमप्रवृत्तिः ।\edlabel{pvv.117-7}\footnote{\label{pvv.117-7}  ७ सिद्धान्ती, अर्थस्य शब्देनाचोदितत्वात् ।} अचोदितत्वात् ।
	\pend
      
	  \bigskip
	  \begingroup
	  \large
	
	    
	    \stanza[\smallbreak]
	\label{pv.2.12b}\edlabel{pv.2.12b}\flagstanza{\tiny\textenglish{....2.12b}}सारूप्याद् भ्रान्तितो वृत्तिरर्थे चेत् स्यान्न सर्वदा ॥ १२ ॥\&[\smallbreak]


	
	  \endgroup
	
	  \bigskip
	  \begingroup
	  \large
	
	    
	    \stanza[\smallbreak]
	\label{pv.2.13a}\edlabel{pv.2.13a}\flagstanza{\tiny\textenglish{....2.13a}}देशभ्रान्तिश्च,\&[\smallbreak]


	
	  \endgroup
	

	  \pstart न हि देवदत्ते प्रतिपादिते तत्पितरि प्रवृत्तिः ॥ निमित्तस्यार्थ{\color{DodgerBlue3}“सारूप्यात् । तद्भ्रान्तितोऽर्थे”} प्रवृत्ति{\color{DodgerBlue3}“श्चेत्”} सम्भाव्यत एतत् किन्तु {\color{DodgerBlue3}“न स्यात् सर्व्वदा”} । न हि यमलकयोर्नियमेन भ्रान्त्याऽ\edlabel{pvv.117-8}\footnote{\label{pvv.117-8}  ८ यावताऽयं पुरुषः सर्व्वदार्थ एव दृष्टः शब्दात्प्रवर्तमानः । न कदाचिन्नास्ति वक्त्रादिसम्बद्धे । तन्न भ्रान्त्या वृत्तिः ।}न्यत्र प्रवृत्तिः । कदाचित्तत्रापि दर्शनात् । तथा\edlabel{pvv.117-9}\footnote{\label{pvv.117-9}  ९ देशभ्रान्त्यार्थे वृत्तिरिति चेत् ।}\leavevmode\marginnote{\textenglish{118/s}} {\color{DodgerBlue3}“देशभ्रान्तिश्च न”} स्यात् । वक्तृश्रोत्रादिसम्बन्धिनि नियतदेशे नामादौ प्रतिपादिते तदन्यदेशे घटादौ सारूप्यादपि न यु\edlabel{pvv.118-1}\footnote{\label{pvv.118-1}  १ तथाहि त्वयेदं करणीयमिति नियुक्तः पुमान् सारूप्यादन्यत्र न प्रवर्तत एवेति दृष्टेः ।}क्ता प्रवृत्तिः ॥ (१२)
	\pend
      \label{div_pvv.2.13}\edlabel{div_pvv.2.13}
	  
	% new div opening: depth here is 2
	

	  \pstart ननु त्वन्मतेऽपि ज्ञानाकारस्याभिधेयत्वात्कथं बाह्ये प्रवृत्तिरित्याह (।)
	\pend
      
	  \bigskip
	  \begingroup
	  \large
	
	    
	    \stanza[\smallbreak]
	\label{pv.2.13b}\edlabel{pv.2.13b}\flagstanza{\tiny\textenglish{....2.13b}}न ज्ञाने तुल्यमुत्पत्तितो धियः ।&तथाविधायाः;\&[\smallbreak]


	
	  \endgroup
	

	  \pstart {\color{DodgerBlue3}“न ज्ञाने”} ज्ञानाकारे वाच्येऽर्थेऽप्रवर्तनं {\color{DodgerBlue3}“तुल्यं”} । धियस्त{\color{DodgerBlue3}“थाविधाया”} बहिस्त्वेनाध्यवसिताकाराया {\color{DodgerBlue3}“उत्पत्तितः”} । शब्दजनिता हि बुद्धिर्व्वस्तुतः स्वांशालम्बनाप्यनाद्यविद्यावशाद् बहिर्व्विषया व्य\edlabel{pvv.118-2}\footnote{\label{pvv.118-2}  २ दृश्यविकल्पयोरैक्येन वृत्तेः ।}वसीयत इतियुक्तमर्थे प्रवर्तनं ॥
	\pend
      
	  \bigskip
	  \begingroup
	  \large
	
	    
	    \stanza[\smallbreak]
	\label{pv.2.13c}\edlabel{pv.2.13c}\flagstanza{\tiny\textenglish{....2.13c}}अन्यत्र तत्रानुपगमाद् धियः ॥ १३ ॥\&[\smallbreak]


	
	  \endgroup
	

	  \pstart {\color{DodgerBlue3}“अन्यत्र”} नामादिविष\edlabel{pvv.118-3}\footnote{\label{pvv.118-3}  ३ यस्यापि न नामादिर्व्विप्रयुक्तोऽभिधेयः ।}यिणि ज्ञाने तदर्थाध्यवसायात् । अर्थे प्रवर्तनं न युक्तं । {\color{DodgerBlue3}“निराकार”}बुद्धिवादि {\color{DodgerBlue3}“वै भा षि क”} मते बाह्यार्थप्रतिभासाया ज्ञानाकाराया धियोऽनुपगमतः । (१३)
	\pend
      \label{div_pvv.2.14}\edlabel{div_pvv.2.14}
	  
	% new div opening: depth here is 2
	
	  \bigskip
	  \begingroup
	  \large
	
	    
	    \stanza[\smallbreak]
	\label{pv.2.14a}\edlabel{pv.2.14a}\flagstanza{\tiny\textenglish{....2.14a}}बाह्यार्थप्रतिभासाया उपाये वाऽप्रमाणाता ।&विज्ञानव्यतिरिक्तस्य;\&[\smallbreak]


	
	  \endgroup
	

	  \pstart यदि ज्ञेयाकारा बुद्धिः स्यात् स्यात्तत्प्रतीत्याऽभिमानात्प्रवृत्तिरपि । अप्रवृत्तिदोषदर्शनाद् {\color{DodgerBlue3}“बाह्यार्थप्रतिभासाया”} बुद्धेरु\edlabel{pvv.118-4}\footnote{\label{pvv.118-4}  ४ अभ्युपगमे ।}पाये स्वीकारे {\color{DodgerBlue3}“वाऽप्रमाणता”} नामादेः विप्रयुक्तसंस्कारस्य {\color{DodgerBlue3}“विज्ञानव्यतिरिक्तस्य”} (।)
	\pend
      

	  \pstart कथमित्याह (।)
	\pend
      
	  \bigskip
	  \begingroup
	  \large
	
	    
	    \stanza[\smallbreak]
	\label{pv.2.14b}\edlabel{pv.2.14b}\flagstanza{\tiny\textenglish{....2.14b}}व्यतिरेकाप्रसिद्धितः ॥ १४ ॥\&[\smallbreak]


	
	  \endgroup
	\leavevmode\marginnote{\textenglish{22b/MA}}

	  \pstart {\color{DodgerBlue3}“व्यतिरेकाप्रसिद्धितः”} । प्रतिभासमानस्याकारस्य ज्ञानत्वात् । न चेन्द्रियादिषु सत्स्वपि विज्ञानकार्यानुत्पत्तितोऽर्थ इव नामादिरपि शक्यव्यवस्थानः । न हि गृहीतसंकेतस्य शब्दश्रुतौ क्वचिदर्थाप्रति\edlabel{pvv.118-5}\footnote{\label{pvv.118-5}  ५ परार्थानुमानेन ।}पत्तिः ॥ (१४)
	\pend
      \label{div_pvv.2.15}\edlabel{div_pvv.2.15}
	  
	% new div opening: depth here is 2
	
	  \bigskip
	  \begingroup
	  \large
	
	    
	    \stanza[\smallbreak]
	\label{pv.2.15a}\edlabel{pv.2.15a}\flagstanza{\tiny\textenglish{....2.15a}}सर्वज्ञानार्थवत्वाच्चेत् स्वप्नादावन्यथेक्षणात् ।&अयुक्तं;\&[\smallbreak]


	
	  \endgroup
	\leavevmode\marginnote{\textenglish{119/s}}

	  \pstart नन्वस्ति प्रमाणं {\color{DodgerBlue3}“सर्व्व”}स्य {\color{DodgerBlue3}“ज्ञान”}स्या{\color{DodgerBlue3}“र्थवत्त्वात्”} । शाब्दमपि ज्ञानमर्थवदेव । न च रूपादयो विषय इति पारिशेष्यान्नामादिकमेवेति चेत् । {\color{DodgerBlue3}“स्वप्नादावा”}दिशब्दात् तैमिरिकज्ञानादिष्व{\color{DodgerBlue3}“न्यथा”} अर्थशून्य{\color{DodgerBlue3}“स्येक्षणात्”} स\edlabel{pvv.119-1}\footnote{\label{pvv.119-1}  १ अनैकान्तिक ।}र्व्वज्ञानार्थवच्छाब्दस्य नामादिविषयत्वानुमानमयुक्तं । स्वप्नविज्ञानमपि निमित्तविषयमेवेति चेदाह (।)
	\pend
      
	  \bigskip
	  \begingroup
	  \large
	
	    
	    \stanza[\smallbreak]
	\label{pv.2.15b}\edlabel{pv.2.15b}\flagstanza{\tiny\textenglish{....2.15b}}न च संस्कारान्नीलादिप्रतिभासतः ॥ १५ ॥\&[\smallbreak]


	
	  \endgroup
	

	  \pstart न च {\color{DodgerBlue3}“संस्कारान्नि”}मित्ताख्यात् स्वप्नज्ञानं {\color{DodgerBlue3}“नीलादे”}र्व्वर्णसंस्थानविशेषतः {\color{DodgerBlue3}“प्रतिभासतः”} । न च विप्रयुक्तसंस\edlabel{pvv.119-2}\footnote{\label{pvv.119-2}  २ सुखादिवत् ।}का/?/रो वर्ण्णसंस्थानविशेषवान् ॥ (१५)
	\pend
      \label{div_pvv.2.16}\edlabel{div_pvv.2.16}
	  
	% new div opening: depth here is 2
	

	  \pstart नीलादिरेवासावर्थ इति चेत् (।)
	\pend
      
	  \bigskip
	  \begingroup
	  \large
	
	    
	    \stanza[\smallbreak]
	\label{pv.2.16a}\edlabel{pv.2.16a}\flagstanza{\tiny\textenglish{....2.16a}}नीलाद्यप्रतिघातान्न;\&[\smallbreak]


	
	  \endgroup
	

	  \pstart स्वप्नप्रतिभासि न {\color{DodgerBlue3}“नीलादि”} वस्तु {\color{DodgerBlue3}“अप्रतिघातात्”} । नीलादयो {\color{DodgerBlue3}“ह्यर्थाः”} स्वदेशे पदार्थान्तरस्य व्याघातकाः स्वप्नोपलब्धास्तु नैवं । पिहितद्वाराववरकोदरसुप्तस्थानान्तरगमना(त्) हस्तियूथादिदर्शनात् ।
	\pend
      

	  \pstart किन्तर्हि तदित्याह (।)
	\pend
      
	  \bigskip
	  \begingroup
	  \large
	
	    
	    \stanza[\smallbreak]
	\label{pv.2.16b}\edlabel{pv.2.16b}\flagstanza{\tiny\textenglish{....2.16b}}ज्ञानं तद्योग्यदेशकैः ।&अज्ञातस्य स्वयं ज्ञानात्;\&[\smallbreak]


	
	  \endgroup
	

	  \pstart {\color{DodgerBlue3}“ज्ञानं तत्”} नीलादि {\color{DodgerBlue3}“योग्यदेशकैः”} स्वप्नायमा\edlabel{pvv.119-3}\footnote{\label{pvv.119-3}  ३ यदुत्पाद्यरूपं योगी पश्यति न तेन व्यभिचारस्तत्रान्यस्यायोग्यत्वात् ।}नेन समानदेशैः पुरुषान्तरै{\color{DodgerBlue3}“रज्ञा”}तस्य स्वयं {\color{DodgerBlue3}“ज्ञानात्”} । स्वयम्वेद्यमसामान्यमिति ज्ञानलक्षणं । तच्च स्वप्नदृष्टनीलादिषु व्यक्तं ।
	\pend
      
	  \bigskip
	  \begingroup
	  \large
	
	    
	    \stanza[\smallbreak]
	\label{pv.2.16c}\edlabel{pv.2.16c}\flagstanza{\tiny\textenglish{....2.16c}}नामाद्येतेन वर्णितम् ॥ १६ ॥\&[\smallbreak]


	
	  \endgroup
	

	  \pstart {\color{DodgerBlue3}“एतेन”} स्वप्नदृष्टस्य ज्ञानत्वसाधनेन {\color{DodgerBlue3}“नामादि वर्ण्णितं”} । नामादिकमपि तन्न भवति योग्यदेशकैरज्ञातस्य स्वयं ज्ञानात् ज्ञानं तत्\edlabel{pvv.119-4}\footnote{\label{pvv.119-4}  ४ नामादि ।}। तस्मान्नास्त्यर्थः सामान्यबुद्धिरिति । (१६)
	\pend
      \label{div_pvv.2.17}\edlabel{div_pvv.2.17}
	  
	% new div opening: depth here is 2
	

	  \pstart अपि च (।)
	\pend
      
	  \bigskip
	  \begingroup
	  \large
	
	    
	    \stanza[\smallbreak]
	\label{pv.2.17}\edlabel{pv.2.17}\flagstanza{\tiny\textenglish{...v.2.17}}सैवेष्टार्थवती केन चक्षुरादिमतिः स्मृता ।&अर्थसामर्थ्यदृष्टेश्चेदन्यत् प्राप्तमनर्थकम् ॥ १७ ॥\&[\smallbreak]


	
	  \endgroup
	

	  \pstart यैव रूपादिविषयत्वे{\color{DodgerBlue3}“नेष्टा सैव”} चक्षुरादिमति{\color{DodgerBlue3}“रर्थवती केन”} हेतुना {\color{DodgerBlue3}“मता । अर्थ”}स्य रूपा{\color{DodgerBlue3}“देश्चक्षुरादिमति”}जनने {\color{DodgerBlue3}“सामर्थ्यदृष्टेश्चेत् । अन्य”}सामान्यविषयं \leavevmode\marginnote{\textenglish{120/s}} विकल्पज्ञान{\color{DodgerBlue3}“मनर्थकं प्राप्तं”}(।) न हि यथा चक्षुरादिबुद्धेरर्थव्यतिरेकाद् व्यतिरेकः तथा सामान्यबुद्धेर्व्यतिरेकः । आभोगमात्रेण भावात् ॥ (१७)
	\pend
      \label{div_pvv.2.18}\edlabel{div_pvv.2.18}
	  
	% new div opening: depth here is 2
	
	  \bigskip
	  \begingroup
	  \large
	
	    
	    \stanza[\smallbreak]
	\label{pv.2.18a}\edlabel{pv.2.18a}\flagstanza{\tiny\textenglish{....2.18a}}अप्रवृत्तिरसम्बन्धेप्यर्थसम्बन्धवद् यदि ।\&[\smallbreak]


	
	  \endgroup
	

	  \pstart न\edlabel{pvv.120-1}\footnote{\label{pvv.120-1}  १ तदेवं नामादावप्रवृत्तिः ।} केवलं वक्तृश्रोतृसम्बन्धिनि\edlabel{pvv.120-2}\footnote{\label{pvv.120-2}  २ सम्बन्धेपि ।} {\color{DodgerBlue3}“असम्बन्धेपि”} नामादावर्थेऽ{\color{DodgerBlue3}“प्रवृत्तिः”} स्यात् । {\color{DodgerBlue3}“अर्थसम्ब\edlabel{pvv.120-3}\footnote{\label{pvv.120-3}  ३ अर्थेन सम्बद्धं ।}न्धवत्”} नामाद्यर्थे प्रवृत्त्यर्थं {\color{DodgerBlue3}“यदी”}ष्यते तदा अतीतानागतं नामादि तदभिधायिनां शब्दानां वाच्यं न स्यात् । अभूत् मा न्धा ता भविष्यति शं ख श्चक्रवर्त्तीति (।)
	\pend
      

	  \pstart कस्मादित्याह (।)
	\pend
      
	  \bigskip
	  \begingroup
	  \large
	
	    
	    \stanza[\smallbreak]
	\label{pv.2.18b}\edlabel{pv.2.18b}\flagstanza{\tiny\textenglish{....2.18b}}अतीतानागतं वाच्यं न स्यादर्थेन तत्क्षयात् ॥ १८ ॥\&[\smallbreak]


	
	  \endgroup
	

	  \pstart {\color{DodgerBlue3}“अर्थेनातीतानागतेन”} सह तस्य नामादेः सम्बन्धिनः {\color{DodgerBlue3}“स्यात्”} स्वरूपा\edlabel{pvv.120-4}\footnote{\label{pvv.120-4}  ४ अनागतस्य चानुत्पत्तेः ।} भावात् ॥ (१८)
	\pend
      \label{div_pvv.2.19}\edlabel{div_pvv.2.19}
	  
	% new div opening: depth here is 2
	

	  \begin{center}%% label @type='head'
	\textbf{(२) परमते दोषः}
	\end{center}
	

	  \begin{center}%% label @type='head'
	\textbf{क. व्यक्तिसामान्ययोर्भेदे दोषः}
	\end{center}
	

	  \pstart अथ (।)
	\pend
      
	  \bigskip
	  \begingroup
	  \large
	
	    
	    \stanza[\smallbreak]
	\label{pv.2.19}\edlabel{pv.2.19}\flagstanza{\tiny\textenglish{...v.2.19}}सामान्यग्रहणाच्छब्दादप्रसंगो मतो यदि ।&तन्न केवलसामान्याग्रहणाद् ग्रहणेपि वा ॥ १९ ॥\&[\smallbreak]


	
	  \endgroup
	

	  \pstart {\color{DodgerBlue3}“शब्दात् सामान्यग्र\edlabel{pvv.120-5}\footnote{\label{pvv.120-5}  ५ आनुषङ्गिकमपनीय प्रकृतं सामान्यमाह (।) शब्देन सामान्यस्यैव ग्रहाद्रूपादौ सफलमक्षं ॥}हणात्”} तत्प्रतीत्या फलवन्त्यक्षाणीति तेषां वैफल्यस्या{\color{DodgerBlue3}“प्रसङ्गो मतो यदि”}(।) तदेतन्न युक्तं (।) {\color{DodgerBlue3}“केवल”}स्य व्यक्तिशून्यस्य {\color{DodgerBlue3}“सामान्यस्याग्रहणात्”} । व्यक्तिव्यंग्यं हि सामान्यं व्यञ्जकाग्रहणे कथं गृह्यते । केवलस्य {\color{DodgerBlue3}“ग्रहणेपि वा”} । (१९)
	\pend
      \label{div_pvv.2.20}\edlabel{div_pvv.2.20}
	  
	% new div opening: depth here is 2
	
	  \bigskip
	  \begingroup
	  \large
	
	    
	    \stanza[\smallbreak]
	\label{pv.2.20}\edlabel{pv.2.20}\flagstanza{\tiny\textenglish{...v.2.20}}अतत्समानता-व्यक्ती तेन नित्योपलम्भनम् ।&नित्यत्वाच्च यदि व्यक्तिर्व्यक्तेः प्रत्यक्षतां प्रति ॥ २० ॥\&[\smallbreak]


	
	  \endgroup
	

	  \pstart {\color{DodgerBlue3}“अतत्स\edlabel{pvv.120-6}\footnote{\label{pvv.120-6}  ६ समानानां भावः समानता व्यक्तेः सामान्यं न स्यादित्यर्थः । व्यक्तिरभिव्यक्तिरपि न स्यात् ।}मानता”}-व्य क्ती । यत्प्रतीत्या यत्प्रतीयते तत्तस्य सामान्यं व्यंग्यञ्च । स्वतन्त्रप्रतीतिस्तु न सामान्यं व्यङ्ग्यं वा । {\color{DodgerBlue3}“तेना”}पराधीनप्रतीतित्वेन {\color{DodgerBlue3}“नित्योपलम्भनं”} सामान्यं प्राप्तं । व्यक्तिसदसत्त्वयोरप्युपलब्धिः स्यात् । स्वप्रतीतौ व्यक्तिनिरपेक्षत्वात् । तथोपलभ्यस्वभाव\edlabel{pvv.120-7}\footnote{\label{pvv.120-7}  ७ व्यक्त्यनुत्पत्तिविनाशेपि ।}स्य सामान्यस्य {\color{DodgerBlue3}“नित्यत्वाच्च नित्यमुपलम्भनं”} भवेत् । \leavevmode\marginnote{\textenglish{121/s}} यदि व्यक्तेर्व्विशेषात् व्यक्तिरभि{\color{DodgerBlue3}“व्यक्तिः”} सामान्यस्य {\color{DodgerBlue3}“प्रत्यक्षतां प्रति”} प्रत्यक्षभावनिमित्तमिष्टा । प्रत्यक्षभूतसामान्यविषया प्रतीतिर्व्यञ्जिकाया व्यक्तेर्भवति । अन्या तु सामान्यबुद्धिर्व्यञ्जकमन्तरेणापि ॥ (२०)
	\pend
      \label{div_pvv.2.21}\edlabel{div_pvv.2.21}
	  
	% new div opening: depth here is 2
	

	  \pstart ननु (।)
	\pend
      
	  \bigskip
	  \begingroup
	  \large
	
	    
	    \stanza[\smallbreak]
	\label{pv.2.21}\edlabel{pv.2.21}\flagstanza{\tiny\textenglish{...v.2.21}}आत्मनि ज्ञानजनने यच्छक्तं शक्तमेव तत् ।&अथाशक्तं कदाचिच्चेदशक्तं सर्वदैव तत् ॥ २१ ॥\&[\smallbreak]


	
	  \endgroup
	

	  \pstart {\color{DodgerBlue3}“य\edlabel{pvv.121-1}\footnote{\label{pvv.121-1}  १ सिद्धान्त्याह (।) सामान्यं तत्स्वभावतः प्रत्यक्षजनकं न वेति ।}त्सा”}मान्य{\color{DodgerBlue3}“मात्मनि”} प्रत्यक्षात्मक{\color{DodgerBlue3}“ज्ञानजनने”} व्यञ्जकव्यापारकाले {\color{DodgerBlue3}“शक्त-”} मन्यदापि {\color{DodgerBlue3}“शक्तमे”}व तन्नित्यैकस्वभावतयाऽनुपकार्यत्वात् । ततश्च केवलस्य सामान्यस्य प्रत्यक्षजननादक्षवैकल्यप्रसङ्गः । {\color{DodgerBlue3}“अथाशक्तं तत्कदाचित्”} केवलाद्यवस्था-\leavevmode\marginnote{\textenglish{23a/MA}} याश्चेत् तदा {\color{DodgerBlue3}“सर्व्वदै”}वाशक्तं {\color{DodgerBlue3}“तदु”}पेयं एकस्य स्वभावद्वयायोगात् स्वभावभेदलक्षणत्वाद्वस्तुभेदस्य । (२१)
	\pend
      \label{div_pvv.2.22}\edlabel{div_pvv.2.22}
	  
	% new div opening: depth here is 2
	

	  \pstart एतदेव स्फुटयति (।)
	\pend
      
	  \bigskip
	  \begingroup
	  \large
	
	    
	    \stanza[\smallbreak]
	\label{pv.2.22}\edlabel{pv.2.22}\flagstanza{\tiny\textenglish{...v.2.22}}तस्य शक्तिरशक्तिर्वा या स्वभावेन संस्थिता ।&नित्यत्वादपि किं तस्य कस्तां क्षपयितुं क्षमः ॥ २२ ॥\&[\smallbreak]


	
	  \endgroup
	

	  \pstart तस्य सामान्यस्य {\color{DodgerBlue3}“शक्तिरशक्तिर्व्वा”} स्वविषयज्ञानजननादौ {\color{DodgerBlue3}“या स्वभावेन”} संस्थिता । तां शक्तिमशक्तिम्वा {\color{DodgerBlue3}“नित्यत्वादचिकि”}त्स्यस्यानपनेयप्राचीनस्वभावस्य कोऽन्यः {\color{DodgerBlue3}“क्षपयितुं क्षमः”} । (२२)
	\pend
      \label{div_pvv.2.23}\edlabel{div_pvv.2.23}
	  
	% new div opening: depth here is 2
	
	  \bigskip
	  \begingroup
	  \large
	
	    
	    \stanza[\smallbreak]
	\label{pv.2.23}\edlabel{pv.2.23}\flagstanza{\tiny\textenglish{...v.2.23}}तच्च सामान्यविज्ञानमनुरुन्धन् विभाव्यते ।&नीलाद्याकारलेशो यः स तस्मिन् केन निर्मितः ॥ २३ ॥\&[\smallbreak]


	
	  \endgroup
	

	  \pstart किन्तु यत्तद्व्यक्तिनिरपेक्षमि\edlabel{pvv.121-2}\footnote{\label{pvv.121-2}  २ सामान्यविषयं ।}ष्टं{\color{DodgerBlue3}“तच्च सामान्यविज्ञानमनुरुन्धन्”} विषयभावेनानुवर्तमानो {\color{DodgerBlue3}“यो विभाव्यते नीलाद्याकारलेशोऽसं”}पूर्ण्णस्फुटीभावः {\color{DodgerBlue3}“स \edlabel{pvv.121-3}\footnote{\label{pvv.121-3}  ३ विकल्पाविकल्पा तन्मतेन ।}तस्मिन्”} सामान्यज्ञाने {\color{DodgerBlue3}“केना”}र्थेन {\color{DodgerBlue3}“निर्मितः”} । न तावज्जात्या तस्यास्त\edlabel{pvv.121-4}\footnote{\label{pvv.121-4}  ४ व्यक्त्याकारासम्भवात् ।}दसम्भवात् । नापि व्यक्त्यातत्र तस्या\edlabel{pvv.121-5}\footnote{\label{pvv.121-5}  ५ ? व्यक्तिरिति ।} अप्रतिभासनात् । ततश्च केवलसामान्यग्राहि तत् ज्ञानमिति न युक्तं । विशेषग्रहणे चाक्षवैफल्यं तदवस्थं\edlabel{pvv.121-6}\footnote{\label{pvv.121-6}  ६ ? अभ्युपगन्तव्यं ।}॥ (२३)
	\pend
      \label{div_pvv.2.24}\edlabel{div_pvv.2.24}
	  
	% new div opening: depth here is 2
	

	  \pstart स्यादेतत् (।) द्विविधो भावानां प्रत्ययः । प्रत्यक्षोऽप्रत्यक्षश्च । तत्र (।)
	\pend
      \leavevmode\marginnote{\textenglish{122/s}}
	  \bigskip
	  \begingroup
	  \large
	
	    
	    \stanza[\smallbreak]
	\label{pv.2.24a}\edlabel{pv.2.24a}\flagstanza{\tiny\textenglish{....2.24a}}प्रत्यक्षप्रत्ययार्थत्वान्नाक्षाणां व्यर्थतेति चेत् ।\&[\smallbreak]


	
	  \endgroup
	

	  \pstart शब्दादिभावेषु प्रत्ययोऽप्रत्यक्षः । प्रत्यक्षस्तु अक्षेभ्य इति न तद्वैफल्यं । अत्राह (।)
	\pend
      
	  \bigskip
	  \begingroup
	  \large
	
	    
	    \stanza[\smallbreak]
	\label{pv.2.24b}\edlabel{pv.2.24b}\flagstanza{\tiny\textenglish{....2.24b}}सैवैकरूपाच्छब्दादेर्भिन्नाभासा मतिः कुतः ॥ २४ ॥\&[\smallbreak]


	
	  \endgroup
	

	  \pstart {\color{DodgerBlue3}“सैव”}\edlabel{pvv.122-1}\footnote{\label{pvv.122-1}  १ स्पष्टास्पष्टा ।} प्रत्यक्षाप्रत्यक्षाभासा {\color{DodgerBlue3}“भिन्नाभासा”} म\edlabel{pvv.122-2}\footnote{\label{pvv.122-2}  २ तत्त्वान्यत्वावाच्यस्य सामान्यत्वे शशशृङ्गादेः प्रसङ्गश्चेत् ।}तिरेकरूपात् शब्दादेरादिशब्दाद् गन्धरसादेः कुतः । एकरूपविषया च भिन्नप्रतिभासा चेति विरुद्धं । (२४)
	\pend
      \label{div_pvv.2.25}\edlabel{div_pvv.2.25}
	  
	% new div opening: depth here is 2
	

	  \pstart किञ्च (।) जातिर्जातिमतो रूपाद् भिन्नाऽभिन्ना वा ।
	\pend
      
	  \bigskip
	  \begingroup
	  \large
	
	    
	    \stanza[\smallbreak]
	\label{pv.2.25}\edlabel{pv.2.25}\flagstanza{\tiny\textenglish{...v.2.25}}न जातिर्जातिमद् व्यक्तिरूपं येनापराश्रयम् ।&सिद्धं; पृथक् चेत् कार्यत्वं ह्यपेक्षेत्यभिधीयते ॥ २५ ॥\&[\smallbreak]


	
	  \endgroup
	

	  \pstart तत्र न ताव{\color{DodgerBlue3}“ज्‏जातिर्जातिम\edlabel{pvv.122-3}\footnote{\label{pvv.122-3}  ३ अभेदः ।}देव व्यक्तिरूपं येन”} कारणेना{\color{DodgerBlue3}“पराश्रयमन”}न्यानुयायि {\color{DodgerBlue3}“सिद्धं”} । न ह्येकस्या व्यक्ते रूपमन्यत्रास्ति । सर्व्वानुयायि च सामान्यमिष्टं व्यक्तेः सकाशात् {\color{DodgerBlue3}“पृथक् चेत्सा”}मान्यं । अस्येदं सामान्यमिति भाविकसम्बन्धानुपपत्तिः । अपेक्षालक्षणः सम्बन्धश्चेत् । {\color{DodgerBlue3}“नन्वपेक्षेति कार्यत्व”}मुच्यते तत्सामान्यस्य नित्यस्यासम्भवि ॥ (२५)
	\pend
      \label{div_pvv.2.26}\edlabel{div_pvv.2.26}
	  
	% new div opening: depth here is 2
	
	  \bigskip
	  \begingroup
	  \large
	
	    
	    \stanza[\smallbreak]
	\label{pv.2.26}\edlabel{pv.2.26}\flagstanza{\tiny\textenglish{...v.2.26}}निष्पत्तेरपराधीनमपि कार्यं स्वहेतुतः ।&सम्बध्यते कल्पनया किमकार्यं कथञ्चन ॥ २६ ॥\&[\smallbreak]


	
	  \endgroup
	

	  \pstart {\color{DodgerBlue3}“कार्यमपि निष्यत्ते”}र्नित्यत्वाद{\color{DodgerBlue3}“पर\edlabel{pvv.122-4}\footnote{\label{pvv.122-4}  ४ यदप्यङ्कुरादि बीजादेः तदपि निष्पन्नमपराधीनं ।}ाधीनं”} सर्व्वत्र निराशंसं वस्तुतो न क्वचि{\color{DodgerBlue3}“त्सम्बध्यते”} । केवलं {\color{DodgerBlue3}“कल्प\edlabel{pvv.122-5}\footnote{\label{pvv.122-5}  ५ बीजादेरङ्कुरादीति ।}नया”} कारणात्मनि सम्बध्यते । यत्तु {\color{DodgerBlue3}“कथञ्चन”} सामान्य{\color{DodgerBlue3}“मकार्य तत्किं”} क्वचित् सम्भ\edlabel{pvv.122-6}\footnote{\label{pvv.122-6}  ६ सम्बन्धमनुभविष्यति ।}त्स्यते । (२६)
	\pend
      \label{div_pvv.2.27}\edlabel{div_pvv.2.27}
	  
	% new div opening: depth here is 2
	
	  \bigskip
	  \begingroup
	  \large
	
	    
	    \stanza[\smallbreak]
	\label{pv.2.27a}\edlabel{pv.2.27a}\flagstanza{\tiny\textenglish{....2.27a}}अन्यत्वे तदसम्बद्धं । सिद्धाऽतो निःस्वभावता ।&जातिप्रसंगोऽभावस्य न;\&[\smallbreak]


	
	  \endgroup
	

	  \pstart तस्माद् (।)
	\pend
      

	  \pstart व्यक्तेः सकाशा{\color{DodgerBlue3}“दन्यत्वे तत्”} सामान्य{\color{DodgerBlue3}“मसम्बद्धमतो”}ऽस्य {\color{DodgerBlue3}“निःस्वभावता सिद्धा”} व्यक्तिभ्यस्तत्त्वान्यत्वाभ्यां व्यवस्थापयितुमशक्यत्वात् ॥ यद्यपि निः स्वभावा जातिस्तथापि ना{\color{DodgerBlue3}“भाव”}स्य शशविषाणादेरपि जातेर्जातिरूपतायाः प्रसङ्गः । न हि यो योऽभावः स जातिरुच्यते । किन्तु सामान्यं यत्तन्निःस्वभावं ।
	\pend
      \leavevmode\marginnote{\textenglish{123/s}}

	  \pstart ननु निःस्वभावत्वे कथं सामान्यमित्याह (।)
	\pend
      
	  \bigskip
	  \begingroup
	  \large
	
	    
	    \stanza[\smallbreak]
	\label{pv.2.27b}\edlabel{pv.2.27b}\flagstanza{\tiny\textenglish{....2.27b}}अपेक्षाभावतस्तयोः ॥ २७ ॥\&[\smallbreak]


	
	  \endgroup
	

	  \pstart {\color{DodgerBlue3}“अपेक्षाभावतस्तयोः”} । शाबलेयागोव्यावृत्त्योः परस्परमपेक्षाभावतस्तद्व्यावृत्तिरेव सामान्यं न शशविषाणादिः । न हि तदपेक्षा क्वचिदस्ति । अतद्‏व्यावृत्तिस्तु निःस्वभावाप्यनुगामिप्रत्ययहेतुः सामान्यं । अभावत्वमिव प्रागभावादिषु पदार्थत्वमिव द्रव्यादिषु ॥ (२७)
	\pend
      \label{div_pvv.2.28}\edlabel{div_pvv.2.28}
	  
	% new div opening: depth here is 2
	
	  \bigskip
	  \begingroup
	  \large
	
	    
	    \stanza[\smallbreak]
	\label{pv.2.28}\edlabel{pv.2.28}\flagstanza{\tiny\textenglish{...v.2.28}}तस्मादरूपा रूपाणां माश्रयेणोपकल्पिता ।&तद्विशेषावगाहार्थैर्जातिः शब्दैः प्रकाश्यते ॥ २८ ॥\&[\smallbreak]


	
	  \endgroup
	

	  \pstart {\color{DodgerBlue3}“तस्माद्व”}स्तुतो जा{\color{DodgerBlue3}“तिररूपा”} निःस्वभावा {\color{DodgerBlue3}“रूपाणां”} शाबलेयादीनां {\color{DodgerBlue3}“नाश्रयेणोपकल्पिता”} जातिः । शब्दैस्तद्विशेषावगाहार्थैः । ते च ते विशेषाश्च तद्वि{\color{DodgerBlue3}“शेषा”}स्तेषाम{\color{DodgerBlue3}“वगाहः”} प्रवृत्तिविषयत्वेन व्यापनं {\color{DodgerBlue3}“सोऽर्थः”} प्रयोजनं येषां तैः सकृदेककार्यानेकप्रतिपत्त्यर्थं तदाश्रयेणोपकल्पिता {\color{DodgerBlue3}“जाति”}रतद्व्यावृत्तिलक्षणा {\color{DodgerBlue3}“शब्दै”}रभिधेया । (२८)
	\pend
      \label{div_pvv.2.29_2.30}\edlabel{div_pvv.2.29_2.30}
	  
	% new div opening: depth here is 2
	

	  \pstart यदि जातिर्निःस्वभावा कथं स्वभावविशिष्टा व्यवसीयत इत्याह (।)
	\pend
      
	  \bigskip
	  \begingroup
	  \large
	
	    
	    \stanza[\smallbreak]
	\label{pv.2.29a}\edlabel{pv.2.29a}\flagstanza{\tiny\textenglish{....2.29a}}तस्यां रूपावभासोयं तत्वेनार्थस्य वा ग्रहः ।&भ्रान्तिः सा;\&[\smallbreak]


	
	  \endgroup
	

	  \pstart {\color{DodgerBlue3}“तस्यां”} जातौ रूपस्य स्वभावस्या{\color{DodgerBlue3}“वभासो यस्तत्वेन”} जातिस्वभावेन {\color{DodgerBlue3}“वार्थस्य यो ग्रग्रो भ्रान्तिः सा”} जातेनिःस्वभावत्वात् विशेषात्मकत्वाभावात् ।
	\pend
      

	  \pstart किन्तर्हि भ्रान्तेर्ब्बीजमित्याह (।)
	\pend
      
	  \bigskip
	  \begingroup
	  \large
	
	    
	    \stanza[\smallbreak]
	\label{pv.2.29b}\edlabel{pv.2.29b}\flagstanza{\tiny\textenglish{....2.29b}}अनादिकालीनदर्शनाभ्यासनिर्मिता ॥ २९ ॥\&[\smallbreak]


	
	  \endgroup
	

	  \pstart {\color{DodgerBlue3}“अनादिकालीना”}नामनादिकालिकानां तथाभूताध्यवसायज्ञानानाम{\color{DodgerBlue3}“भ्यासेन निर्मिता”} । (२९)
	\pend
      \leavevmode\marginnote{\textenglish{23b/MA}}

	  \pstart यद्यन्यव्यावृत्तिः सामान्यं किन्तस्य रूपमित्याह (।)
	\pend
      
	  \bigskip
	  \begingroup
	  \large
	
	    
	    \stanza[\smallbreak]
	\label{pv.2.30}\edlabel{pv.2.30}\flagstanza{\tiny\textenglish{...v.2.30}}अर्थानां यच्च सामान्यमन्यव्यावृत्तिलक्षणम् ।&यन्निष्ठास्त इमे शब्दा न रूपं तस्य किञ्चन ॥ ३० ॥\&[\smallbreak]


	
	  \endgroup
	

	  \pstart {\color{DodgerBlue3}“अर्थानां”} विशेषाणां {\color{DodgerBlue3}“यच्च सामान्यमन्यव्यावृत्तिलक्षणं । यन्निष्ठा”} यद्विषया{\color{DodgerBlue3}“स्ते इमे”} सांकेतिकाः {\color{DodgerBlue3}“शब्दास्तस्य रूपं”} स्वभावो {\color{DodgerBlue3}“न किञ्चन”} । वस्तुतः कल्पितत्वात् । (३०)
	\pend
      \label{div_pvv.2.31_2.32}\edlabel{div_pvv.2.31_2.32}
	  
	% new div opening: depth here is 2
	

	  \pstart ननु बुद्ध्याकारः स स्वभावः स एव तर्हि सामान्यम्भविष्यतीत्याह(।)
	\pend
      \leavevmode\marginnote{\textenglish{124/s}}
	  \bigskip
	  \begingroup
	  \large
	
	    
	    \stanza[\smallbreak]
	\label{pv.2.31a}\edlabel{pv.2.31a}\flagstanza{\tiny\textenglish{....2.31a}}सामान्यबुद्धौ सामान्येनारूपायामवीक्षणात् ।&अर्थभ्रान्तिरपीष्येत सामान्यं साऽपिः\&[\smallbreak]


	
	  \endgroup
	

	  \pstart {\color{DodgerBlue3}“सामान्यबुद्धावरूपायां”} ग्रा\edlabel{pvv.124-1}\footnote{\label{pvv.124-1}  १ बुद्धिप्रतिभासः सामान्यमित्यत्राह ।}ह्यरूपरहितायामपि {\color{DodgerBlue3}“सामान्येन”} विजातीयव्यावृत्त्युपकल्पिता-भेदेनाकारेण भेदेष्वर्थेष्वी{\color{DodgerBlue3}“क्षणात्”} सामान्यमपि यद्व्यवस्थाप्यते सा{\color{DodgerBlue3}“प्यर्थभ्रान्तिरिष्येत”} विवेचकैः । न हि बुद्ध्याकारः {\color{DodgerBlue3}“सामान्य”}मुक्तं स्वलक्षणत्वात् ।
	\pend
      
	  \bigskip
	  \begingroup
	  \large
	
	    
	    \stanza[\smallbreak]
	\label{pv.2.31b}\edlabel{pv.2.31b}\flagstanza{\tiny\textenglish{....2.31b}}अभिप्लवात् ॥ ३१ ॥\&[\smallbreak]


	
	  \endgroup
	
	  \bigskip
	  \begingroup
	  \large
	
	    
	    \stanza[\smallbreak]
	\label{pv.2.32}\edlabel{pv.2.32}\flagstanza{\tiny\textenglish{...v.2.32}}अर्थरूपतया तत्वेनाभावाच्च न रूपिणी ।&निःस्वभावतयाऽवाच्यं कुतश्चिद् वचनान्मतम् ॥ ३२ ॥\&[\smallbreak]


	
	  \endgroup
	

	  \pstart अर्थनिष्ठ{\color{DodgerBlue3}“तया”} तु सामान्ये गृह्यमाणेऽर्थभ्रमः\edlabel{pvv.124-2}\footnote{\label{pvv.124-2}  २ अतस्मिँस्तद्ग्रहात् ।} । त\edlabel{pvv.124-3}\footnote{\label{pvv.124-3}  ३ तत्रैवाह हेतुं ।}था बुद्ध्याकारस्यार्थरूपत्वेन व्यक्तिष्व\edlabel{pvv.124-4}\footnote{\label{pvv.124-4}  ४ भाति स्वाकारमर्थेर्प्पयन्ती ।}भिप्लवात् अभिसम्बन्धात् । {\color{DodgerBlue3}“त\edlabel{pvv.124-5}\footnote{\label{pvv.124-5}  ५ वस्तुतोर्थरूपतया ।}त्त्वे”}नार्थरूपत्वे{\color{DodgerBlue3}“नाभावा\edlabel{pvv.124-6}\footnote{\label{pvv.124-6}  ६ नाप्यर्थधर्मोन्यापोहोत्र युक्तः ।}च्च”} । {\color{DodgerBlue3}“न रूपि\edlabel{pvv.124-7}\footnote{\label{pvv.124-7}  ७ निःस्वभावापूर्व्वहेतोश्च}णी”} तथा-भूतबाह्यविषयवती सामान्यबुद्धिः अतश्च । ततश्च {\color{DodgerBlue3}“निःस्वभावतया”} सामान्यं भेदाभेदाभ्याम{\color{DodgerBlue3}“वाच्यं”} व्यक्तिभ्यः । स्वभावं हि भिन्नमभिन्नं वा स्यात् । {\color{DodgerBlue3}“कुतश्चिद्”}सामान्यात् भेदेन {\color{DodgerBlue3}“वचनात्”} यदि वस्तुसामान्यं {\color{DodgerBlue3}“मतं”} । (३१, ३२)
	\pend
      \label{div_pvv.2.33}\edlabel{div_pvv.2.33}
	  
	% new div opening: depth here is 2
	
	  \bigskip
	  \begingroup
	  \large
	
	    
	    \stanza[\smallbreak]
	\label{pv.2.33}\edlabel{pv.2.33}\flagstanza{\tiny\textenglish{...v.2.33}}यदि वस्तुनि वस्तूनामवाच्यत्वं कथञ्चन ।&नैव वाच्यमुपादानभेदाद् भेदोपचारतः ॥ ३३ ॥\&[\smallbreak]


	
	  \endgroup
	

	  \pstart यद्येवं व्यक्तेर\edlabel{pvv.124-8}\footnote{\label{pvv.124-8}  ८ वस्तुत्वात् । ? अश्व ।}पि भेदाभेदाभ्यां वाच्यं स्यात् । यस्मान्न {\color{DodgerBlue3}“वस्तूनामवाच्यत्वं कथञ्चन”} । यथा हि सामान्यं स्वरूपवत्त्वात् सामान्यान्तराद् भेदेनोच्यते तथा व्यक्तेरपि तत्त्वान्यत्वाभ्यां यथासम्भवमुच्येत\edlabel{pvv.124-9}\footnote{\label{pvv.124-9}  ९ वस्तुभवत्सत्तत्वान्यत्वम्वा नातिक्रामति ।} । तस्माद्यत् कुतश्चिदपि वस्तुनस्तत्वान्यत्वाभ्यामवाच्यं तदवस्त्विति सविशेषणो हेतुः । अथवा निर्व्विशेषणे हेतौ नासिद्धिः । यत\edlabel{pvv.124-10}\footnote{\label{pvv.124-10}  १० पटादपि घटत्वमन्यत्वेन नैव वाच्यं । न चोपचरितभेदेन वस्तुत्वं । न हि कलायविदलं स्वर्ण्णमुपचर्य हेमवान् (?वत्) स्यात् ।}सामान्यान्तरादपि सामान्यं भेदेन {\color{DodgerBlue3}“नैव वाच्यमुपा”}दानस्य कर्कशाबलेयादे{\color{DodgerBlue3}“र्भेदा”}त्तदाश्रयोत्पन्नबुद्ध्यालम्बना अश्वत्वगोत्वादीनां {\color{DodgerBlue3}“भेद\edlabel{pvv.124-11}\footnote{\label{pvv.124-11}  ११ न तु भिन्नसामान्ये तद्रूपादय एव केवला भिन्नाः । उपचरितभेदान्न वस्तुत्वसिद्धिः ।}स्योपचारात्”} । (३३)
	\pend
      \label{div_pvv.2.34}\edlabel{div_pvv.2.34}
	  
	% new div opening: depth here is 2
	\leavevmode\marginnote{\textenglish{125/s}}

	  \pstart किञ्च (।)
	\pend
      
	  \bigskip
	  \begingroup
	  \large
	
	    
	    \stanza[\smallbreak]
	\label{pv.2.34}\edlabel{pv.2.34}\flagstanza{\tiny\textenglish{...v.2.34}}अतीतानागतेप्यर्थे सामान्यविनिबन्धनाः ।&श्रुतयो निविशन्ते सदसद्धर्मः कथम्भवेत् ॥ ३४ ॥\&[\smallbreak]


	
	  \endgroup
	

	  \pstart {\color{DodgerBlue3}“अतीतानागतेप्यर्थे सामान्यनिबन्धना”} सामान्याश्रयाः {\color{DodgerBlue3}“श्रुतयो निविशन्ते”} व्यवतिष्ठन्ते । आसीत् घटो भविष्यतीत्यादयः । तथा चासतो घटस्य सामान्यं धर्म इत्युक्तं स्यात् । तच्च सामान्यं {\color{DodgerBlue3}“सदसतो”}ऽतीतादे{\color{DodgerBlue3}“र्धर्मः कथम्भवेत्”} । न हि तैक्ष्ण्यं शशविषाणस्य भवति\edlabel{pvv.125-1}\footnote{\label{pvv.125-1}  १ अतीतादौ न सामान्यनिबन्धना शब्दवृत्तिः किन्तूपचारादित्याह ।}। (३४)
	\pend
      \label{div_pvv.2.35}\edlabel{div_pvv.2.35}
	  
	% new div opening: depth here is 2
	
	  \bigskip
	  \begingroup
	  \large
	
	    
	    \stanza[\smallbreak]
	\label{pv.2.35}\edlabel{pv.2.35}\flagstanza{\tiny\textenglish{...v.2.35}}उपचारात् तदिष्टं चेद् वर्त्तमानघटस्य का ।&प्रत्यासत्तिरभावेन या पटादौ न विद्यते ॥ ३५ ॥\&[\smallbreak]


	
	  \endgroup
	

	  \pstart {\color{DodgerBlue3}“अथोपचारात्त”}दसद्धर्मत्व{\color{DodgerBlue3}“मिष्टं”} सामान्यस्य न वस्तुत इति {\color{DodgerBlue3}“चेत्”} । स\edlabel{pvv.125-2}\footnote{\label{pvv.125-2}  २ घटस्य रूपादिभ्योऽवाच्यत्वेपि पटाद् घटे भिन्नता......वाच्य एव । वर्तमाने घटत्वदर्शनात् ।}ति घटे तद्धर्मतादृष्टेरसत्यपि तस्मिन् सा कल्प्यत इत्यर्थः ।
	\pend
      

	  \pstart ननु सामान्यसम्बन्धिनो\edlabel{pvv.125-3}\footnote{\label{pvv.125-3}  ३ वर्त्तमानस्य ।}ऽपि घटस्या{\color{DodgerBlue3}“भावे”}नातीतादिघटलक्षणेन का {\color{DodgerBlue3}“प्रत्यास”}त्तिरुपचारनिबन्धनमस्ति {\color{DodgerBlue3}“या पटादौ न विद्यते”} यदभावा\edlabel{pvv.125-4}\footnote{\label{pvv.125-4}  ४ यदि वस्तुभूतसामान्याश्रयेण शब्दवृत्तिस्तदातीतादौ न स्यात् ।}त् घटसम्बन्धिता सामान्यस्य प\edlabel{pvv.125-5}\footnote{\label{pvv.125-5}  ५ पटे यस्याः प्रत्यासत्तेरभावान्नोपचार इति ।}टे नोपचर्यते (।) सादृश्यं प्रत्यासत्तिरित चेत् । तन्न(।) सदसतो सादृश्याभावात् ॥ प्राक् तादृगासीदिति चेत् । यदासीन्न (तदा) तदुपचारः सम्बन्धस्य सत्त्वात् । यदा यनास्ति तदापि न सादृश्यं । सति च किञ्चिदुपचर्येत नासति । (३५)
	\pend
      \label{div_pvv.2.36}\edlabel{div_pvv.2.36}
	  
	% new div opening: depth here is 2
	

	  \pstart किञ्च (।)
	\pend
      
	  \bigskip
	  \begingroup
	  \large
	
	    
	    \stanza[\smallbreak]
	\label{pv.2.36}\edlabel{pv.2.36}\flagstanza{\tiny\textenglish{...v.2.36}}बुद्धेरस्खलिता वृत्तिर्मुख्यारोपितयोः सदा ।&सिंहे माणवके तद्वद् घोषणाप्यस्ति लौकिकी ॥ ३६ ॥\&[\smallbreak]


	
	  \endgroup
	

	  \pstart {\color{DodgerBlue3}“मुख्यारोपितयोरर्थयोर्बुद्धे”}र्ग्रांहिकाया {\color{DodgerBlue3}“अस्खलि\edlabel{pvv.125-6}\footnote{\label{pvv.125-6}  ६ अस्खलिता वृत्तिरित्यावर्त्त्य यत्नात् ।}ता”} दृढा {\color{DodgerBlue3}“वृत्तिः”} । {\color{DodgerBlue3}“तन्त्राद”}वृत्तिश्च {\color{DodgerBlue3}“सदा”} । यथा {\color{DodgerBlue3}“सिंहे माणवके”} च सिंहबुद्धेंरस्खलिता । स्खलिता च वृत्तिरिति {\color{DodgerBlue3}“लौकिक्यपि घोषणास्ति”} न केवलं प्रामाणिकी । न चातीतानागतघटादिबुद्धिः स्खलन्ती जायते येनातीतानागतव्यक्तिधर्मता सामान्यस्यारोपिता स्यात् । (३६)
	\pend
      \label{div_pvv.2.37}\edlabel{div_pvv.2.37}
	  
	% new div opening: depth here is 2
	\leavevmode\marginnote{\textenglish{126/s}}
	  \bigskip
	  \begingroup
	  \large
	
	    
	    \stanza[\smallbreak]
	\label{pv.2.37}\edlabel{pv.2.37}\flagstanza{\tiny\textenglish{...v.2.37}}यत्र रूढ्याऽसदर्थोपि जनैः शब्दो निवेशितः ।&स मुख्यस्तत्र तत्साम्याद् गौणोन्यत्र स्खलद्गतिः ॥ ३७ ॥\&[\smallbreak]


	
	  \endgroup
	

	  \pstart तस्मा{\color{DodgerBlue3}“द्यत्र”}\edlabel{pvv.126-1}\footnote{\label{pvv.126-1}  १ सिंहादौ ।} विषये{\color{DodgerBlue3}“ऽसदर्थो”} वाच्यरहितोपि {\color{DodgerBlue3}“शब्दो”} रूढ्या वाचकत्वेन {\color{DodgerBlue3}“जनैर्निवेशितः”} संकेतितः {\color{DodgerBlue3}“स मृख्यः तत्रार्थे । तत्साम्यात्”} । तद्विषयसादृश्या{\color{DodgerBlue3}“दन्यत्र”} स शब्दः {\color{DodgerBlue3}“स्खलद्वत्तिर”}दृढतया प्रत्ययहैतुः । {\color{DodgerBlue3}“गौणः”} । ततश्च सदर्थविषयत्वं मुख्यत्वं असदर्थविषयत्वञ्चामुख्यत्वमिति मुख्यागौणलक्षणमपास्तं । संकेतवशेन नियमाभावात् । (३७)
	\pend
      \label{div_pvv.2.38}\edlabel{div_pvv.2.38}
	  
	% new div opening: depth here is 2
	
	  \bigskip
	  \begingroup
	  \large
	
	    
	    \stanza[\smallbreak]
	\label{pv.2.38}\edlabel{pv.2.38}\flagstanza{\tiny\textenglish{...v.2.38}}यथा भावेप्यभावाख्यां यथाकल्पनमेव वा ।&कुर्यादशक्ते शक्ते वा प्रधानादिश्रुतिं जनः ॥ ३८ ॥\&[\smallbreak]


	
	  \endgroup
	

	  \pstart {\color{DodgerBlue3}“यथा भावेपि”} पुत्रादौ तत्कार्यासमर्थत्वादसत्कल्पे{\color{DodgerBlue3}“ऽभावाख्यां”} शशविषाणं बन्ध्यासुत इत्यादिकां जनः कुर्य्यात् । तत्र भावेप्यमुख्योऽभावशब्दः । अभावे तु मुख्यः त\edlabel{pvv.126-2}\footnote{\label{pvv.126-2}  २ पुनर्व्यभिचारमाह ।}था {\color{DodgerBlue3}“यथा कल्पनमेव”} सां ख्या द्यभिमते वस्तुतो{\color{DodgerBlue3}“ऽशक्ते”} प्रधानादौ {\color{DodgerBlue3}“शक्ते”} वा पुरुषादा\leavevmode\marginnote{\textenglish{24a/MA}}वनेककार्यसम\edlabel{pvv.126-3}\footnote{\label{pvv.126-3}  ३ अवस्तुनः कुतो भेदः ।} र्थे तत्साम्यात् । {\color{DodgerBlue3}“प्रधानादिश्रुतिं जनः कुर्य्यात्”} । स प्रधानशब्दोऽभां\edlabel{pvv.126-4}\footnote{\label{pvv.126-4}  ४ कर्तृत्वादिमारोप्य त्रिगुणीमये प्रधानकल्पना तस्य च लोकेनानिष्टेरुपरतव्यापारेऽनेककार्यसमर्थेऽयं सांख्यपुरुष इति ।}व एव मुख्यो भावे चामुख्यः । तस्मात्सदसदर्थविषयता मुख्यगौणविषयतेति व्यभिचारिलक्षणं\edlabel{pvv.126-5}\footnote{\label{pvv.126-5}  ५ सांख्यमते प्राह ।}। (३८)
	\pend
      \label{div_pvv.2.39}\edlabel{div_pvv.2.39}
	  
	% new div opening: depth here is 2
	

	  \pstart यथासंकेतमेव तु शब्दवृत्तिरिति युक्त्तं न चैतद्वस्तुविष(य)त्वे न्याय्यं । तथा\edlabel{pvv.126-6}\footnote{\label{pvv.126-6}  ६ अभिधया वाच्यत्वाभ्यामवस्तुत्वात्तत्रैवोपपत्त्यन्तरमाह ।}हि (।)
	\pend
      
	  \bigskip
	  \begingroup
	  \large
	
	    
	    \stanza[\smallbreak]
	\label{pv.2.39a}\edlabel{pv.2.39a}\flagstanza{\tiny\textenglish{....2.39a}}शब्देभ्यो यादृशी बुद्धिर्नष्टेऽनष्टेपि दृश्यते ।&तादृश्येव;\&[\smallbreak]


	
	  \endgroup
	

	  \pstart {\color{DodgerBlue3}“शब्देभ्यो यादृशी”} यादृशाकारा {\color{DodgerBlue3}“बुद्धिः नष्टे”} विषये । तादृश्येवानष्टेपि विषयेऽव्याप्ततेन्द्रियस्य {\color{DodgerBlue3}“दृश्यते”} । न हि शब्दजनिता घटबुद्धिर्निमीलितनयनस्य नष्टेऽनष्टे वा घटे विशिष्यते । तुल्याकारत्वात् । व्याप्तेन्द्रियस्य तु शब्दं श्रृण्वतो या स्पष्टा बुद्धिः सा प्रत्यक्षैव न शब्दकृता । तस्मादवस्तुविषयैव शाब्दी बुद्धिः ।
	\pend
      
	  \bigskip
	  \begingroup
	  \large
	
	    
	    \stanza[\smallbreak]
	\label{pv.2.39b}\edlabel{pv.2.39b}\flagstanza{\tiny\textenglish{....2.39b}}सदर्थानां नैतच्छोत्रादिचेतसाम् ॥ ३९ ॥\&[\smallbreak]


	
	  \endgroup
	\leavevmode\marginnote{\textenglish{127/s}}

	  \pstart {\color{DodgerBlue3}“सदर्थानां”} वस्तुविषयाणां तु {\color{DodgerBlue3}“श्रोत्रा”}दीन्द्रियजातानां {\color{DodgerBlue3}“चेतसां नैत”}द्विषयसदसत्ताकालयोः साम्यं शब्दबुद्धेरिवार्थाभावे इन्द्रियबुद्धेरनुत्पत्तेः । (३९)
	\pend
      \label{div_pvv.2.40}\edlabel{div_pvv.2.40}
	  
	% new div opening: depth here is 2
	

	  \pstart ननु नष्टेऽनष्टेपि वस्तुनि ये शाब्दो चेतसी जायेते ।
	\pend
      
	  \bigskip
	  \begingroup
	  \large
	
	    
	    \stanza[\smallbreak]
	\label{pv.2.40a}\edlabel{pv.2.40a}\flagstanza{\tiny\textenglish{....2.40a}}सामान्यमात्रग्रहणात् सामान्यं चेतसोर्द्वयोः ॥\&[\smallbreak]


	
	  \endgroup
	

	  \pstart ताभ्यां {\color{DodgerBlue3}“समान्यमात्रस्य ग्रहणात्”} । तयो{\color{DodgerBlue3}“र्द्वयोश्चेतसोः सामान्यं”} साम्यं {\color{DodgerBlue3}“ग्राह्य”}प्रतिभासकृतमस्ति । एतच्चायुक्तं यस्मात् (।)
	\pend
      
	  \bigskip
	  \begingroup
	  \large
	
	    
	    \stanza[\smallbreak]
	\label{pv.2.40b}\edlabel{pv.2.40b}\flagstanza{\tiny\textenglish{....2.40b}}तस्यापि केवलस्य प्राग् ग्रहणं विनिवारितम् ॥ ४० ॥\&[\smallbreak]


	
	  \endgroup
	

	  \pstart {\color{DodgerBlue3}“तस्यापि”} सामान्यस्य {\color{DodgerBlue3}“केवलस्य”} व्यक्तिशून्यस्य {\color{DodgerBlue3}“ग्रहणं”} (२।१४) {\color{DodgerBlue3}“प्राग्निवारितम्”} अतत्समानताऽव्यक्ती (२।२०) इत्यादिना (४०)
	\pend
      \label{div_pvv.2.41}\edlabel{div_pvv.2.41}
	  
	% new div opening: depth here is 2
	

	  \begin{center}%% label @type='head'
	\textbf{ख. व्यक्तिसामान्ययोरभेदे दोषः}
	\end{center}
	

	  \pstart भिन्नं सामान्यं निषिध्याभिन्नमपि निषेद्ध्ुमाह (।)
	\pend
      
	  \bigskip
	  \begingroup
	  \large
	
	    
	    \stanza[\smallbreak]
	\label{pv.2.41}\edlabel{pv.2.41}\flagstanza{\tiny\textenglish{...v.2.41}}परस्परविशिष्टानामविशिष्टं कथं भवेत् ।&तथा द्विरूपतायां वा तद् वस्त्वेकं कथं भवेत् ॥ ४१ ॥\&[\smallbreak]


	
	  \endgroup
	

	  \pstart {\color{DodgerBlue3}“प\edlabel{pvv.127-1}\footnote{\label{pvv.127-1}  १ पूर्व्वं न जातिर्जातिमदेव केवलमित्युक्तमधुना तत्र युक्तिरुच्यते ।}रस्परविशिष्टानां”} विशेषाणा{\color{DodgerBlue3}“मविशिष्टमभिन्नं रूपं कथम्भवेत्”} । अवश्यं हि व्यक्तीनां भेदःकथञ्चिदङ्गीकर्त्तव्यः । सत्त्वर\edlabel{pvv.127-2}\footnote{\label{pvv.127-2}  २ योपि जगत एकत्वमिच्छति तेनापि ।}जस्तमसामिव प्रकृतिचैतन्ययोरिव वाऽन्यथा सामान्यमेव न स्यात् । भिन्नानामभिन्नस्य रूपस्य तत्त्वा\edlabel{pvv.127-3}\footnote{\label{pvv.127-3}  ३ सामान्यत्वात् ।}त् । ये च भिन्नस्वभावास्ते नाभिन्ना भवितुमर्हन्ति विरुद्धत्वात् । अथैकस्यापि समानमसमानञ्च द्वे रूपे । तथा {\color{DodgerBlue3}“द्विरूपतायां तद्वस्त्वेकं कथम्भवेत्”} । अन्यो हि समानादसमानः स्वभावः । अतश्च द्वे वस्तुनी स्यातां नत्वेकं द्विरूपं । (४१)
	\pend
      \label{div_pvv.2.42}\edlabel{div_pvv.2.42}
	  
	% new div opening: depth here is 2
	

	  \pstart अथ द्वयोरेकं तद्रूपं सामान्यं ।
	\pend
      
	  \bigskip
	  \begingroup
	  \large
	
	    
	    \stanza[\smallbreak]
	\label{pv.2.42a}\edlabel{pv.2.42a}\flagstanza{\tiny\textenglish{....2.42a}}ताभ्यां तदन्यदेव स्याद् यदि रूपं समं तयोः ।\&[\smallbreak]


	
	  \endgroup
	

	  \pstart {\color{DodgerBlue3}“यदि तयो”}र्द्वयोः {\color{DodgerBlue3}“समं”} समानं {\color{DodgerBlue3}“रूपं”} तदा (।) {\color{DodgerBlue3}“ताभ्या”}मेवा{\color{DodgerBlue3}“न्यदेव तत्स्यात्”} ।
	\pend
      

	  \pstart पृथग्भूतमेव सामान्यं भविष्यति को दोष इति चेदाह (।)
	\pend
      
	  \bigskip
	  \begingroup
	  \large
	
	    
	    \stanza[\smallbreak]
	\label{pv.2.42b}\edlabel{pv.2.42b}\flagstanza{\tiny\textenglish{....2.42b}}तयोरिति न सम्बन्धो व्यावृत्तिस्तु न दुष्यति ॥ ४२ ॥\&[\smallbreak]


	
	  \endgroup
	\leavevmode\marginnote{\textenglish{128/s}}

	  \pstart {\color{DodgerBlue3}“तयो”}स्तत्सामान्य{\color{DodgerBlue3}“मिति न सम्बन्धः”} । उपकार्योपकारकत्वाभावात् । तथासम्बन्धेऽतिप्रसङ्गात् । अस्मन्मते {\color{DodgerBlue3}“तु व्यावृत्तिः”} सामान्यं {\color{DodgerBlue3}“न दुष्यति”} । अतत्कार्यव्यावृ\edlabel{pvv.128-1}\footnote{\label{pvv.128-1}  १ तर्ज्जन्या यथाङगुष्ठाद् भेदस्तथा परिशिष्टानामेव । वस्तुत्वे ।}त्तेरवस्तुत्वात् । अन्यथाऽन्यानन्यत्वपक्षोक्तो दोषः । (४२)
	\pend
      \label{div_pvv.2.43}\edlabel{div_pvv.2.43}
	  
	% new div opening: depth here is 2
	
	  \bigskip
	  \begingroup
	  \large
	
	    
	    \stanza[\smallbreak]
	\label{pv.2.43a}\edlabel{pv.2.43a}\flagstanza{\tiny\textenglish{....2.43a}}तस्मात् समानतैवास्मिन् सामान्येऽवस्तुलक्षणम् ।\&[\smallbreak]


	
	  \endgroup
	

	  \pstart यतः समानत्वेनावस्तुता {\color{DodgerBlue3}“तस्मात्समानतैवास्मिन् सामा”}न्येऽ{\color{DodgerBlue3}“वस्तुलक्षणं”} यत्सामान्यं तदवस्थितिव्याप्तिसिद्धेः । भेदेऽभेदे च वस्तुत्वायोगात् प्रकारान्तरस्य चाभावात् ।
	\pend
      

	  \pstart किञ्च (।)
	\pend
      
	  \bigskip
	  \begingroup
	  \large
	
	    
	    \stanza[\smallbreak]
	\label{pv.2.43b}\edlabel{pv.2.43b}\flagstanza{\tiny\textenglish{....2.43b}}कार्यञ्चेत् तदनेकं स्यान्नश्वरञ्च न तन्मतम् ॥ ४३ ॥\&[\smallbreak]


	
	  \endgroup
	

	  \pstart सामान्यं कार्यमकार्यम्वा स्यात् । सम्बन्धिनीनां व्यक्तीनां {\color{DodgerBlue3}“कार्यञ्चेत्”} प्रतिव्यक्ति सामान्योक्ता{\color{DodgerBlue3}“वनेकं स्यात्”} । एकञ्च सामा न्यमिष्टं । अनेकत्वे सामान्यरूपतानाशः । {\color{DodgerBlue3}“नश्वरञ्च न तत्सा”}मान्यं {\color{DodgerBlue3}“मतं”} । कार्यत्वान्नश्वरञ्च तत्प्राप्नोति कार्यत्वस्य नाशित्वेन स्वभावस्य व्याप्तेः । (४३)
	\pend
      \label{div_pvv.2.44}\edlabel{div_pvv.2.44}
	  
	% new div opening: depth here is 2
	

	  \pstart किञ्च (।)
	\pend
      
	  \bigskip
	  \begingroup
	  \large
	
	    
	    \stanza[\smallbreak]
	\label{pv.2.44a}\edlabel{pv.2.44a}\flagstanza{\tiny\textenglish{....2.44a}}वस्तुमात्रानुबन्धित्वाद् विनाशस्य न नित्यता ।\&[\smallbreak]


	
	  \endgroup
	

	  \pstart {\color{DodgerBlue3}“वस्तु”}मात्रा{\color{DodgerBlue3}“नुबन्धित्वात् नाशस्य”} वस्तुनो\edlabel{pvv.128-2}\footnote{\label{pvv.128-2}  २ किन्तु क्षणिकता स्यात् ।} {\color{DodgerBlue3}“न नित्यता”} स्यात् । अ\edlabel{pvv.128-3}\footnote{\label{pvv.128-3}  ३ वस्तुत्वे च सामान्यस्यानित्यतैव । अनित्यत्वेऽनेकत्वादसामान्यत्वं ।}नित्यताविरहे तु वस्तुविरहोपि व्याप्याभावस्य व्यापकाभावनियतत्वात् ।
	\pend
      

	  \pstart अथ द्वितीयः पक्षः । तदा (।)
	\pend
      
	  \bigskip
	  \begingroup
	  \large
	
	    
	    \stanza[\smallbreak]
	\label{pv.2.44b}\edlabel{pv.2.44b}\flagstanza{\tiny\textenglish{....2.44b}}असम्बन्धश्च जातीनामकार्यत्वादरूपता ॥ ४४ ॥\&[\smallbreak]


	
	  \endgroup
	

	  \pstart {\color{DodgerBlue3}“अकार्यत्वात्”} जातीनां स्वव्यक्तिभिः सह सम्बन्धश्च न स्यात् । कार्यकारणभावाभावे तस्येदमिति सम्बन्धस्यानुत्पत्तिरित्युक्तं कार्यत्वं ह्यपेक्षेत्यभिधीयते (२।२५) इत्यत्रान्तरेऽ{\color{DodgerBlue3}“कार्यत्वाज्जातीनामरूप”}ता निःस्वभावता । उत्पद्यमानं हि सस्वभावं भवेन्नेतरत् । नि\edlabel{pvv.128-4}\footnote{\label{pvv.128-4}  ४ लोकेनासम्मतत्वादसन्मुख्यः प्रधानं कुतः । देवदत्तादिः सन्नमुख्यः}यामकाभावेन स्वभावाभावप्रसङ्गात् । (४४)
	\pend
      \label{div_pvv.2.45}\edlabel{div_pvv.2.45}
	  
	% new div opening: depth here is 2
	

	  \pstart शाब्दप्रत्ययस्यावस्तुविषयतायामुपपत्त्यन्तरमाह (।)
	\pend
      \leavevmode\marginnote{\textenglish{129/s}}
	  \bigskip
	  \begingroup
	  \large
	
	    
	    \stanza[\smallbreak]
	\label{pv.2.45}\edlabel{pv.2.45}\flagstanza{\tiny\textenglish{...v.2.45}}यच्च वस्तुबलाज्ज्ञानं जायते तदपेक्षते ।&न सङ्केतं न सामान्यबुद्धिष्वेतद् विभाव्यते ॥ ४५ ॥\&[\smallbreak]


	
	  \endgroup
	

	  \pstart {\color{DodgerBlue3}“यच्च”} ज्ञानमिन्द्रियजं {\color{DodgerBlue3}“वस्तुनो रूपादेर्ब्बलाज्जायते तन्न संकेतमपेक्षते,”} बालबधिरादेरपि भावात् । {\color{DodgerBlue3}“सामान्यबुद्धिषु त्वेतत् संकेतानपेक्षत्वं न विभाव्यते”} संकेतग्रहणस्मरणापेक्षत्वात् । तस्मान्न वस्तुबलभाविन्यस्ता\edlabel{pvv.129-1}\footnote{\label{pvv.129-1}  १ सामान्यबुद्धयः ।}ः । तथात्वे सतीन्द्रियार्थसन्निपाते क्षेपायोगात्\edlabel{pvv.129-2}\footnote{\label{pvv.129-2}  २ ? संकेतापेक्षा न स्यात्}। (४५)
	\pend
      \label{div_pvv.2.46}\edlabel{div_pvv.2.46}
	  
	% new div opening: depth here is 2
	
	  \bigskip
	  \begingroup
	  \large
	
	    
	    \stanza[\smallbreak]
	\label{pv.2.46}\edlabel{pv.2.46}\flagstanza{\tiny\textenglish{...v.2.46}}याप्यभेदानुगा बुद्धिः काचिद् वस्तुद्वयेक्षणे ।&सङ्केतेन विना सार्थप्रत्यासत्तिनिबन्धना ॥ ४६ ॥\&[\smallbreak]


	
	  \endgroup
	

	  \pstart या\edlabel{pvv.129-3}\footnote{\label{pvv.129-3}  ३ प्रत्यक्षं सामान्यमिति वादिनं शङ्कते ।}पि {\color{DodgerBlue3}“संकेतेन विना वस्तुद्वयेक्षणे”} शावलेयं दृष्ट्वा बाहुलेयं पश्यतः अभेदानुगमात् स एवायमित्यभेदमध्यवस्यन्ती बुद्धिरुत्पद्यते शावलेयकर्कदर्शने तु नोत्पद्यते । {\color{DodgerBlue3}“साऽर्था\edlabel{pvv.129-4}\footnote{\label{pvv.129-4}  ४ सिद्धान्तयति}नां”} शावलेयादीनामसत्यपि सामान्ये या {\color{DodgerBlue3}“प्रत्यासत्तिरे-”} कबुद्ध्यादिकार्यत्वं त{\color{DodgerBlue3}“न्निबन्धना”} । (४६)
	\pend
      \label{div_pvv.2.47}\edlabel{div_pvv.2.47}
	  
	% new div opening: depth here is 2
	

	  \pstart सामान्यं विनैककार्यतैव न स्यादित्याह (।)
	\pend
      
	  \bigskip
	  \begingroup
	  \large
	
	    
	    \stanza[\smallbreak]
	\label{pv.2.47}\edlabel{pv.2.47}\flagstanza{\tiny\textenglish{...v.2.47}}प्रत्यासत्तिर्विना जात्या यथेष्टा चक्षुरादिषु ।&ज्ञानकार्येषु जातिर्वा ययान्वेति विभागतः ॥ ४७ ॥\&[\smallbreak]


	
	  \endgroup
	

	  \pstart {\color{DodgerBlue3}“प्रत्यासत्तिर्विना जान्या । चक्षुरादिर्ये”}षां विषयालोकमनस्काराणां तेषु\leavevmode\marginnote{\textenglish{24b/MA}} ज्ञानकार्येषु ज्ञानहेतुषु जात्यादि विना रूपज्ञानैककार्यजनकत्वं प्रत्यासत्ति{\color{DodgerBlue3}“र्यथेष्टा”} । यथा वा शावलेयाद्बाहुलेयादीनां कर्कादीनाञ्च तुल्ये भेदे {\color{DodgerBlue3}“यथा”} प्रत्यासत्या जात्यन्तरं विनैव {\color{DodgerBlue3}“विभागतो जातिरन्वेति”} । शावलेयबाहुलेयादिष्वेव गोत्वं समवेतं न कर्कादिषु । सैवानुगामिप्रत्ययनिबन्धनमास्तामलं जात्या । (४७)
	\pend
      \label{div_pvv.2.48}\edlabel{div_pvv.2.48}
	  
	% new div opening: depth here is 2
	

	  \begin{center}%% label @type='head'
	\textbf{>ग. न चक्षुरादिभिः प्रत्येयं सामान्यम्}
	\end{center}
	

	  \pstart इतश्च न वस्तु सामान्यं (।)
	\pend
      
	  \bigskip
	  \begingroup
	  \large
	
	    
	    \stanza[\smallbreak]
	\label{pv.2.48a}\edlabel{pv.2.48a}\flagstanza{\tiny\textenglish{....2.48a}}कथञ्चिदपि विज्ञाने तद्रूपानवभासतः ।\&[\smallbreak]


	
	  \endgroup
	

	  \pstart दृश्यत्वेनाभिमतस्य {\color{DodgerBlue3}“तद्रूपस्य”} स्वग्राहिणि {\color{DodgerBlue3}“विज्ञाने कथञ्चिदप्यनवभासतः”} ।
	\pend
      \leavevmode\marginnote{\textenglish{130/s}}

	  \pstart ननु सन्त्यपीन्द्रियाणि नोपलभ्यन्ते । ततोऽत्रानुपलम्भादसत्वं न । न हीन्द्रियाणि स्वग्राहिणि ज्ञाने प्रत्यवभासमानत्वात् सन्तीष्यन्ते (।) किन्तर्हि (।) सत्स्वपि विषयमनस्कारादिषु कदाचित्प्रवर्तते ज्ञानं कदाचिन्नेति व्यभिचारबलादतीन्द्रियाणि कानिचिदिन्द्रियव्यपदेश्यानि व्यवस्थाप्यन्ते ।
	\pend
      
	  \bigskip
	  \begingroup
	  \large
	
	    
	    \stanza[\smallbreak]
	\label{pv.2.48b}\edlabel{pv.2.48b}\flagstanza{\tiny\textenglish{....2.48b}}यदि नामेन्द्रियाणां स्याद् द्रष्टा भासेत तद्वपुः ॥ ४८ ॥\&[\smallbreak]


	
	  \endgroup
	
	  \bigskip
	  \begingroup
	  \large
	
	    
	    \stanza[\smallbreak]
	\label{pv.2.49a}\edlabel{pv.2.49a}\flagstanza{\tiny\textenglish{....2.49a}}रूपवत्वात्;\&[\smallbreak]


	
	  \endgroup
	

	  \pstart यदि त्वतीन्द्रियाणामतीन्द्रियदर्शी द्रष्टा नाम स्यात् । {\color{DodgerBlue3}“भासेत”} रूपवत्वात्तेषामिन्द्रियाणां {\color{DodgerBlue3}“वपुः”} । (४८)
	\pend
      \label{div_pvv.2.49}\edlabel{div_pvv.2.49}
	  
	% new div opening: depth here is 2
	
	  \bigskip
	  \begingroup
	  \large
	
	    
	    \stanza[\smallbreak]
	\label{pv.2.49b}\edlabel{pv.2.49b}\flagstanza{\tiny\textenglish{....2.49b}}न जातीनां केवलानामदर्शनात् ।&व्यक्तिग्रहे च तच्छ्रब्दरूपादन्यन्न दृश्यते ॥ ४९ ॥\&[\smallbreak]


	
	  \endgroup
	

	  \pstart {\color{DodgerBlue3}“जाती”}नान्तु दुश्याभिमतानां {\color{DodgerBlue3}“केवलानां”} व्यक्तिस्वरूपव्यतिरिक्तानाम{\color{DodgerBlue3}“दर्शनादभाव”} एव । {\color{DodgerBlue3}“व्यक्तिग्रहे”} तस्या व्यक्तेः । {\color{DodgerBlue3}“शब्दस्य”} गौरित्यस्य {\color{DodgerBlue3}“रूपादन्यत्”} सामान्यं {\color{DodgerBlue3}“न दृश्यते”} । (४९)
	\pend
      \label{div_pvv.2.50}\edlabel{div_pvv.2.50}
	  
	% new div opening: depth here is 2
	
	  \bigskip
	  \begingroup
	  \large
	
	    
	    \stanza[\smallbreak]
	\label{pv.2.50}\edlabel{pv.2.50}\flagstanza{\tiny\textenglish{...v.2.50}}ज्ञानमात्रार्थकरणेप्ययोग्यमत् एव तत् ।&तदयोग्यतयाऽरूपं तद्ध्यवस्तुषु लक्षणम् ॥ ५० ॥\&[\smallbreak]


	
	  \endgroup
	

	  \pstart {\color{DodgerBlue3}“अत एवा”}दृश्यमानत्वात् {\color{DodgerBlue3}“ज्ञानमात्र”}स्यार्थस्यार्थक्रियायाः {\color{DodgerBlue3}“करणेप्ययोग्यमेव तत्”} । अन्त्या हीयं भावानामर्थक्रिया यदुत स्वज्ञानजननं । तत्राप्य{\color{DodgerBlue3}“योग्यतया”} तत्सामान्यम{\color{DodgerBlue3}“रूपं”} निःस्वभावं । हिर्यस्मात् सर्व्वार्थक्रियायामशक्तत्व{\color{DodgerBlue3}“मवस्तुषु लक्षणं”} सर्वसामर्थ्यरहितं ह्यवस्त्विष्यते । (५०)
	\pend
      \label{div_pvv.2.51}\edlabel{div_pvv.2.51}
	  
	% new div opening: depth here is 2
	

	  \pstart तथा च सामान्यमिति न वस्तु\edlabel{pvv.130-1}\footnote{\label{pvv.130-1}  १ रूपाहितवासनामाश्रित्योत्पत्तेः ।} ।
	\pend
      
	  \bigskip
	  \begingroup
	  \large
	
	    
	    \stanza[\smallbreak]
	\label{pv.2.51a}\edlabel{pv.2.51a}\flagstanza{\tiny\textenglish{....2.51a}}यथोक्तविपरीतं यत् तत् स्वलक्षणमिष्यते ।\&[\smallbreak]


	
	  \endgroup
	

	  \pstart {\color{DodgerBlue3}“यथो”}क्तात्सामान्या{\color{DodgerBlue3}“द्विपरीतं”} यत्त{\color{DodgerBlue3}“त्स्वलक्षण”}मुच्यते । इदञ्च वैपरीत्यं । अनभिधेयत्वं । तत्वान्यत्वाभ्यां वाच्यत्वं । असदर्थप्रत्ययाविशिष्टप्रतिभासविषयत्वं (।) असाधारणत्वं । संकेतस्मरणानपेक्षप्रतिपत्तिकत्वं । अन्यरूपविविक्तस्वरूपप्रतिभासवत्त्वं (।) अर्थक्रियाक्षमत्वञ्च । एतद्युक्तं स्वलक्षणमिष्यते । एतद्विपर्ययस्यावस्तुत्वसाधनस्य सामान्यलक्षणत्वात् ।
	\pend
      

	  \pstart यच्चातत्कार्यव्यवच्छेदलक्षणमुक्तं (।)
	\pend
      \leavevmode\marginnote{\textenglish{131/s}}
	  \bigskip
	  \begingroup
	  \large
	
	    
	    \stanza[\smallbreak]
	\label{pv.2.51b}\edlabel{pv.2.51b}\flagstanza{\tiny\textenglish{....2.51b}}सामान्यं त्रिविधं, तच्च भावाभावोभयाश्रयात् ॥ ५१ ॥\&[\smallbreak]


	
	  \endgroup
	

	  \pstart {\color{DodgerBlue3}“सामान्यं तच्च त्रिविधं”} बोद्धव्यं {\color{DodgerBlue3}“भावाभावोभयाश्रयात्”} । किञ्चिद् भावोपादानं सामान्यं यथा रूपादीन् भावानाश्रित्य कृ\edlabel{pvv.131-1}\footnote{\label{pvv.131-1}  १ पंक्तिसेनादिवत् ।}तकत्वादिशब्दवाच्यं लिङ्गं । अभावोपादानं यथोपलब्धिलक्षणप्राप्तस्यासतोऽनुपलब्धिरनुत्पत्तिमत्त्वादि च । तद्ध्यभावाश्रयं सामान्यं लिङ्गं । उभयाश्रयमनुपलब्धिमात्रं\edlabel{pvv.131-2}\footnote{\label{pvv.131-2}  २ किं पिशाचादयोऽत्यन्तमसन्तोऽथ सन्तोपि नेक्ष्यन्त इत्युभयोपादानत्वं प्रत्येतुं ज्ञानधर्ममुक्त्वा विषयधर्ममाह ज्ञेयत्वादि ।}ज्ञेयत्वादि च भावाभावसाधरणत्वात् । (५१)
	\pend
      \label{div_pvv.2.52}\edlabel{div_pvv.2.52}
	  
	% new div opening: depth here is 2
	
	  \bigskip
	  \begingroup
	  \large
	
	    
	    \stanza[\smallbreak]
	\label{pv.2.52}\edlabel{pv.2.52}\flagstanza{\tiny\textenglish{...v.2.52}}यदि भावाश्रयं ज्ञानं भावे भावानुबन्धतः ॥&नोक्तोत्तरत्वाद् दृष्टत्वाद्; अतीतादिषु चान्यथा ॥ ५२ ॥\&[\smallbreak]


	
	  \endgroup
	

	  \pstart {\color{DodgerBlue3}“यदि”} किञ्चित्सामान्यं {\color{DodgerBlue3}“भावाश्रयं”} तत् {\color{DodgerBlue3}“ज्ञानं तद्भावे”} भावविषयं प्राप्नोति । भावानुबन्धतो भावान्वयव्यतिरेकानुविधानात् । नैतद्युक्तं न तद्वस्तु । अभि\edlabel{pvv.131-3}\footnote{\label{pvv.131-3}  ३ यद्यपि भावानुविधानं पारम्पर्येण तथापि न साक्षाद्वस्तुविषयत्वं सिध्यति ।}धेयत्वादिनोक्तोत्तरत्वात् । निषिद्धं हि सामान्यस्य वस्तुत्वमनन्तरमेव प्रपञ्चेनेत्यनुमानबाधितत्वं प्रतिज्ञा\edlabel{pvv.131-4}\footnote{\label{pvv.131-4}  ४ भावविषयत्वमिति परप्रतिज्ञाऽभिधेयत्वादिनानुमानेन बाध्यते । परंपरया भावानुविधानेपि न साक्षाद्वस्तुविषयता सिद्ध्यति ।}याः । दृष्टत्वात् । {\color{DodgerBlue3}“अतीतादिषु चान्यथा”} वस्तुव्यतिरेकेणैवासीद् घटो भविष्यति चेति सामान्यबुद्धिरुत्पद्यते इत्यसिद्धतापि भावानुबन्धत इति हेतोः । (५२)
	\pend
      \label{div_pvv.2.53}\edlabel{div_pvv.2.53}
	  
	% new div opening: depth here is 2
	
	  \bigskip
	  \begingroup
	  \large
	
	    
	    \stanza[\smallbreak]
	\label{pv.2.53a}\edlabel{pv.2.53a}\flagstanza{\tiny\textenglish{....2.53a}}भावधर्मत्वहानिश्चेद् भावग्रहणपूर्वकम् ।&तज्ज्ञानमित्यदोषोयं;\&[\smallbreak]


	
	  \endgroup
	

	  \pstart वस्तु विना सामान्यबुद्ध्युत्पादे {\color{DodgerBlue3}“भावधर्मत्वहानिः”} सामान्यस्य प्रसज्यते चेत् । {\color{DodgerBlue3}“भाव”}स्य रूपादे{\color{DodgerBlue3}“र्ग्रहणपूर्व्वकं”}\edlabel{pvv.131-5}\footnote{\label{pvv.131-5}  ५ रूपाहितवासनामाश्रित्योत्पत्तेः ।} तस्य सामान्यस्य साधारणबाह्यरूपतयाऽध्यवसितबुद्ध्याकारलक्षणस्याध्यवसायेन {\color{DodgerBlue3}“ज्ञानमित्य”}यमवस्तु\edlabel{pvv.131-6}\footnote{\label{pvv.131-6}  ६ वस्तुधर्मत्वहान्या ।}धर्मत्वलक्षणोऽदोषो {\color{DodgerBlue3}“दोषो”} न भवति । न हि सामान्यं\edlabel{pvv.131-7}\footnote{\label{pvv.131-7}  ७ ? भावस्य सत्वे भवत् । भावाश्रयत्वात् ।} रूपादिरिव भावरूपतया ज्ञानविषय इति भावधर्म इष्टं\edlabel{pvv.131-8}\footnote{\label{pvv.131-8}  ८ ? रूपादिरिव ।} ।
	\pend
      \leavevmode\marginnote{\textenglish{132/s}}
	  \bigskip
	  \begingroup
	  \large
	
	    
	    \stanza[\smallbreak]
	\label{pv.2.53b}\edlabel{pv.2.53b}\flagstanza{\tiny\textenglish{....2.53b}}मेयं त्वेकं स्वलक्षणाम् ॥ ५३ ॥\&[\smallbreak]


	
	  \endgroup
	

	  \pstart किन्तु भाववासनाप्रबोधप्रसूतविकल्पकल्पितत्वात्\edlabel{pvv.132-1}\footnote{\label{pvv.132-1}  १ यत एवं नान्यदर्थक्रियाक्षमं ।} । परमार्थतो {\color{DodgerBlue3}“मेयं त्वेकं स्वलक्षणं”} । तस्यैव रूपबुद्ध्युत्पादकत्वात् । (५३)
	\pend
      \label{div_pvv.2.54}\edlabel{div_pvv.2.54}
	  
	% new div opening: depth here is 2
	
	  \bigskip
	  \begingroup
	  \large
	
	    
	    \stanza[\smallbreak]
	\label{pv.2.54a}\edlabel{pv.2.54a}\flagstanza{\tiny\textenglish{....2.54a}}तस्मादर्थक्रियासिद्धेः सदसत्ताविचारणात् ॥\&[\smallbreak]


	
	  \endgroup
	\leavevmode\marginnote{\textenglish{25a/MA}}

	  \pstart {\color{DodgerBlue3}“सामान्यस्य”} तु कल्पितत्वात् । सामर्थ्याभावात्स्वलक्षणमेकं प्रमेयं {\color{DodgerBlue3}“तस्मादर्थक्रियासिद्धेः”} । अर्थक्रियार्थिभिः {\color{DodgerBlue3}“सदसत्ताभ्यां”} तस्यैव {\color{DodgerBlue3}“विचारणात्”} । असत्त्वमपि स्वलक्षण\edlabel{pvv.132-2}\footnote{\label{pvv.132-2}  २ व्याप्तिः पक्षधर्मता चेत्यत्र नेदं वाच्यं । अन्वयव्यतिरेकपक्षधर्मताकालेपि न वाच्यं । अन्वयव्यतिरेकमात्रे तु वाच्यं । अस्ति न वेति कृत्वा । दृष्टत्वात}स्यैव विचिन्त्यते ।
	\pend
      

	  \pstart यद्येकमेव प्रमेयं तदा\edlabel{pvv.132-3}\footnote{\label{pvv.132-3}  ३ दिग्नागेन ।}चार्येण प्रमेयद्वैविध्यं यदुक्तं न स्वसामान्यलक्षणाभ्यामन्यत्प्रमेयमस्तीति तद् विरुध्यते इत्याह ।
	\pend
      
	  \bigskip
	  \begingroup
	  \large
	
	    
	    \stanza[\smallbreak]
	\label{pv.2.54b}\edlabel{pv.2.54b}\flagstanza{\tiny\textenglish{....2.54b}}तस्य स्वपररूपाभ्यां गतेर्मेयद्वयं मतम् ॥ ५४ ॥\&[\smallbreak]


	
	  \endgroup
	

	  \pstart {\color{DodgerBlue3}“तस्य”} स्वलक्षणस्य प्रत्यक्षतः {\color{DodgerBlue3}“स्व”}रूपेणानुमानतः {\color{DodgerBlue3}“पररूपेण”} सामान्याकारेण {\color{DodgerBlue3}“गतेर्मेयद्वयं मतं”} न तु भूतसामान्यस्य सत्त्वात् (५४) ।
	\pend
      
	  
	% new div opening: depth here is 1
	
\section[{४. अनुमानचिन्ता}]{४. अनुमानचिन्ता}

	  \begin{center}%% label @type='head'
	\textbf{(१) अनुमानसिद्धिः}
	\end{center}
	

	  \begin{center}%% label @type='head'
	\textbf{क. भ्रान्तमनुमानं प्रमाणाम्}
	\end{center}
	\label{div_pvv.2.55}\edlabel{div_pvv.2.55}
	  
	% new div opening: depth here is 2
	
	  \bigskip
	  \begingroup
	  \large
	
	    
	    \stanza[\smallbreak]
	\label{pv.2.55a}\edlabel{pv.2.55a}\flagstanza{\tiny\textenglish{....2.55a}}अयथाभिनिवेशेन द्वितीया भ्रान्तिरिष्यते ।\&[\smallbreak]


	
	  \endgroup
	

	  \pstart या च {\color{DodgerBlue3}“द्वितीया”} पररूपेण गति{\color{DodgerBlue3}“रयथाभिनिवेशेन भ्रान्तिरिष्यते”} साऽनुमानं यथाऽर्थोस्ति यथा वा स्वाकारस्तथा नाभिनिविशते किन्तु स्वाकारं बाह्यं साधारणतया मन्यते ।
	\pend
      
	  \bigskip
	  \begingroup
	  \large
	
	    
	    \stanza[\smallbreak]
	\label{pv.2.55b}\edlabel{pv.2.55b}\flagstanza{\tiny\textenglish{....2.55b}}गतिश्चेत् पररूपेण न च भ्रान्तेः प्रमाणता ॥ ५५ ॥\&[\smallbreak]


	
	  \endgroup
	

	  \pstart नन्वनुमानं {\color{DodgerBlue3}“पररूपेण गतिश्चेत्”} तदा भ्रान्तिरेव । {\color{DodgerBlue3}“न च भ्रान्तेः प्रमाणते”}ष्यते मृगतृष्णादेरिव । (५५)
	\pend
      \leavevmode\marginnote{\textenglish{133/s}}\label{div_pvv.2.56}\edlabel{div_pvv.2.56}
	  
	% new div opening: depth here is 2
	

	  \pstart अत्रोच्यते ।
	\pend
      
	  \bigskip
	  \begingroup
	  \large
	
	    
	    \stanza[\smallbreak]
	\label{pv.2.56a}\edlabel{pv.2.56a}\flagstanza{\tiny\textenglish{....2.56a}}अभिप्रायाविसंवादादपि भ्रान्तेः प्रमाणता ॥\edlabel{pvv.133-asterisk}\footnote{\label{pvv.133-asterisk}  *द्रष्टव्यं परिशिष्टं १।११}\&[\smallbreak]


	
	  \endgroup
	
	  \bigskip
	  \begingroup
	  \large
	
	    
	    \stanza[\smallbreak]
	\label{pv.2.56b}\edlabel{pv.2.56b}\flagstanza{\tiny\textenglish{....2.56b}}गतिरप्यन्यथा दृष्टा;\&[\smallbreak]


	
	  \endgroup
	

	  \pstart {\color{DodgerBlue3}“भ्रान्तेरपि प्रमाणता । अभिप्रायस्या”}र्थक्रियार्थिभिः ज्ञानगोचरतयाऽभिप्रायविष\edlabel{pvv.133-1}\footnote{\label{pvv.133-1}  १ प्रवृत्तिविषयस्य ।}यीकृतस्यार्थक्रियासमर्थस्यार्थस्य {\color{DodgerBlue3}“संवादात्”} । अर्थक्रियार्थिनो हि तत्साधनसमर्थार्थप्रापकं प्रमाणमिच्छन्ति । {\color{DodgerBlue3}“अन्यथा”} पररूपेण {\color{DodgerBlue3}“गतिरपि”} काचिदभिप्रेतार्थसम्वादिका {\color{DodgerBlue3}“दृष्टे”}ति प्रमाणमेव ।
	\pend
      

	  \pstart नन्वनुमानं वस्त्वेव गृह्णत् प्रमाणमस्तु किं भ्रान्तिरिष्यते इत्याह ।
	\pend
      
	  \bigskip
	  \begingroup
	  \large
	
	    
	    \stanza[\smallbreak]
	\label{pv.2.56c}\edlabel{pv.2.56c}\flagstanza{\tiny\textenglish{....2.56c}}पक्षश्चायं कृतोत्तरः ॥ ५६ ॥\&[\smallbreak]


	
	  \endgroup
	

	  \pstart {\color{DodgerBlue3}“पक्षश्चायं”} प्रागेव न तद्वस्तु अभिधेयत्वादित्यादिना (२।११) {\color{DodgerBlue3}“कृतोत्तरः”} । (५६)
	\pend
      \label{div_pvv.2.57}\edlabel{div_pvv.2.57}
	  
	% new div opening: depth here is 2
	

	  \pstart ननु भ्रान्तमपि यद्यनुमानं प्रमाणं तदा सर्व्वैव भ्रान्तिः प्रमाणं स्यादित्याह ।
	\pend
      
	  \bigskip
	  \begingroup
	  \large
	
	    
	    \stanza[\smallbreak]
	\label{pv.2.57}\edlabel{pv.2.57}\flagstanza{\tiny\textenglish{...v.2.57}}\edlabel{pvv.133-asterisk-bis}\footnote{\label{pvv.133-asterisk-bis}  *द्रष्टव्यं परिशिष्टं १।११}मणिप्रदीपप्रभयोर्मणिबुद्ध्याभिधावतोः ।&मिथ्याज्ञानाविशेषेपि विशेषोर्थक्रियां प्राति ॥ ५७ ॥\&[\smallbreak]


	
	  \endgroup
	

	  \pstart {\color{DodgerBlue3}“मणिप्रदीपयो”}र्ये {\color{DodgerBlue3}“प्रभे”} तयो{\color{DodgerBlue3}“र्मणिबुद्ध्या”} मणिरेवायमित्यध्यवसायेन तद्‏ग्रहणार्थं {\color{DodgerBlue3}“धावतो मिथ्याज्ञानस्य”} भ्रान्तत्वस्या{\color{DodgerBlue3}“विशेषेपि विशेषोऽर्थक्रियाम्प्रति”} । मणिप्रभायांमप्यध्यवसायी मणिसाध्यामर्थक्रियां प्राप्नोति । (५७)
	\pend
      \label{div_pvv.2.58}\edlabel{div_pvv.2.58}
	  
	% new div opening: depth here is 2
	
	  \bigskip
	  \begingroup
	  \large
	
	    
	    \stanza[\smallbreak]
	\label{pv.2.58}\edlabel{pv.2.58}\flagstanza{\tiny\textenglish{...v.2.58}}यथा तथाऽयथार्थत्वेप्यनुमानतदाभयोः ।&अर्थक्रियानुरोधेन प्रमाणत्वं व्यवस्थितम् ॥ ५८ ॥\&[\smallbreak]


	
	  \endgroup
	

	  \pstart दीपप्रभायान्तु मणिव्यवसायी तन्न प्राप्नोतीति {\color{DodgerBlue3}“यथा तथा”} त्रिरूपलिङ्गज{\color{DodgerBlue3}“मनुमानं । तदा”}भञ्च तन्न न तथा । तयोः स्वाकारे बाह्यध्यवसायप्रवृत्तत्वात् अयथार्थत्वेपि {\color{DodgerBlue3}“प्रमाण”}त्वं {\color{DodgerBlue3}“व्यवस्थितं”} । विशेषेणावस्थित{\color{DodgerBlue3}“मर्थक्रियानुरोधेना”}नुमानमेव प्रमाणं । परम्परयाऽर्थादुत्पत्तेः तत्प्रापकत्वात् । नेतरद्विपर्ययात् । तस्मात्प्रमेयद्वित्वं गतिभेदात् । (५८)
	\pend
      \label{div_pvv.2.59_2.60}\edlabel{div_pvv.2.59_2.60}
	  
	% new div opening: depth here is 2
	

	  \pstart तयोर्लक्षणं ग्रहणञ्चाख्यातुमाह ।
	\pend
      
	  \bigskip
	  \begingroup
	  \large
	
	    
	    \stanza[\smallbreak]
	\label{pv.2.59}\edlabel{pv.2.59}\flagstanza{\tiny\textenglish{...v.2.59}}बुद्धिर्यत्रार्थसामर्थ्यादन्वयव्यतिरेकिणी ।&तस्य स्वतत्रं ग्रहणमतोऽन्यद् वस्त्वतीन्द्रियम् ॥ ५९ ॥\&[\smallbreak]


	
	  \endgroup
	\leavevmode\marginnote{\textenglish{134/s}}

	  \pstart अर्थसामर्थ्याज्जायमाना बुद्धिर्यत्रान्वयव्यतिरेकिणी तत्स्वलक्षणन्तस्य ग्रहणमपराश्रयं स्वरूपग्रहणं स्वजन्ययाऽर्थस्वरूपया बुद्ध्या साक्षात्तस्य ग्रहणात् । अतः स्वलक्षणत्वादन्यद्वस्तु यदन्वयव्यतिरेकौ नानुकरोति बुद्धिः साक्षादप्रतिभासमानं केवलमध्यवसायविषयः तत्सामान्यलक्षण{\color{DodgerBlue3}“मतीन्द्रियं”} बुद्धिष्वप्रतिभासनात् (५९) ।
	\pend
      

	  \pstart कथन्तत्प्रत्येतव्यमित्याह (।)
	\pend
      
	  \bigskip
	  \begingroup
	  \large
	
	    
	    \stanza[\smallbreak]
	\label{pv.2.60}\edlabel{pv.2.60}\flagstanza{\tiny\textenglish{...v.2.60}}तस्यादृष्टात्मरूपस्य गतेरन्योर्थ आश्रयः ।&तदाश्रयेण सम्बन्धी यदि स्याद् गमकस्तदा ॥ ६० ॥\&[\smallbreak]


	
	  \endgroup
	

	  \pstart {\color{DodgerBlue3}“तस्य”} सामान्यस्य स्वलक्षणविशेषेण स्वलक्षणविवेकेना{\color{DodgerBlue3}“दृष्टात्मरूपस्य\edlabel{pvv.134-1}\footnote{\label{pvv.134-1}  १ साक्षान्न गतिः । तत्सत्तामात्रस्य । ? तस्यैव क्षणज्ञानस्य ।}”} गतेः प्रतीतेः तस्माद{\color{DodgerBlue3}“न्योऽर्थो”} लिङ्गभूत {\color{DodgerBlue3}“आश्रयः”} सिद्धिनिमित्तं {\color{DodgerBlue3}“यदि”} ततः प्रत्येतव्येनानाग्निव्यवच्छेदादिना {\color{DodgerBlue3}“तदाश्रयेण”} च धर्मिणा {\color{DodgerBlue3}“सम्बन्धी”} सम्बन्धवान् {\color{DodgerBlue3}“स्यात्तदा गमको”} नान्यथा । अनने साध्यप्रतिबन्धोऽन्वयव्यतिरेकरूपः पक्षधर्मता च लिङ्गस्योक्ता (६०) ।
	\pend
      \label{div_pvv.2.61}\edlabel{div_pvv.2.61}
	  
	% new div opening: depth here is 2
	

	  \pstart नन्वन्याश्रयेण च स्वरूपप्रतीतिर्भविष्यति तथापि परोक्षता कथमित्याह (।)
	\pend
      
	  \bigskip
	  \begingroup
	  \large
	
	    
	    \stanza[\smallbreak]
	\label{pv.2.61}\edlabel{pv.2.61}\flagstanza{\tiny\textenglish{...v.2.61}}गमकानुगसामान्यरूपेणैव तदा गतिः ।&तस्मात् सर्वः परोक्षोर्थो विशेषेण न गम्यते ॥ ६१ ॥\&[\smallbreak]


	
	  \endgroup
	

	  \pstart {\color{DodgerBlue3}“तदाऽर्था”}न्तरात्प्रतीतिकाले {\color{DodgerBlue3}“गमकं”} पक्षसपक्षानुयायि लिङ्गं ।\edlabel{pvv.134-2}\footnote{\label{pvv.134-2}  २ हेतुसत्तयैव साध्यसत्त्वात् ।} तदनुयायिना तद्‏व्यापकेन {\color{DodgerBlue3}“सामान्यरूपेणैव”} परोक्षोर्थो गम्यते । न तु सर्व्वतो व्यावृत्तेन विशिष्टेन रूपेण । {\color{DodgerBlue3}“तस्मात्सर्वः परोक्षोऽर्थः”} प्रतीयमानो {\color{DodgerBlue3}“न विशेषेण”} कथञ्चिद् {\color{DodgerBlue3}“गम्यते”} येन परोक्षताहानिः स्यात् । (६१)
	\pend
      \label{div_pvv.2.62}\edlabel{div_pvv.2.62}
	  
	% new div opening: depth here is 2
	
	  \bigskip
	  \begingroup
	  \large
	
	    
	    \stanza[\smallbreak]
	\label{pv.2.62}\edlabel{pv.2.62}\flagstanza{\tiny\textenglish{...v.2.62}}या च सम्बन्धिनो धर्माद् भूतिर्धर्मिणि ज्ञायते ।&सानुमानं परोक्षाणामेकान्तेनैव साधनम् ॥ ६२ ॥\&[\smallbreak]


	
	  \endgroup
	

	  \pstart {\color{DodgerBlue3}“या च सम्बन्धिनो धर्माद”}न्वयव्यतिरेकतो लिङ्गात् तदाश्रये {\color{DodgerBlue3}“धर्मिणि ज्ञायते”} परोक्षार्थप्रतीतिः । {\color{DodgerBlue3}“सानुमानं”} त्रिरूपलिङ्गप्रभवत्त्वात् । तदेवानुमानं {\color{DodgerBlue3}“परोक्षाणामेकान्तेनैव साधनं”} । प्रत्यक्षस्य तत्रावृत्तेः । (६२)
	\pend
      \label{div_pvv.2.63}\edlabel{div_pvv.2.63}
	  
	% new div opening: depth here is 2
	
	  \bigskip
	  \begingroup
	  \large
	
	    
	    \stanza[\smallbreak]
	\label{pv.2.63}\edlabel{pv.2.63}\flagstanza{\tiny\textenglish{...v.2.63}}न प्रत्यक्षपरोक्षाभ्यां मेयस्यान्यस्य सम्भवः ।&तस्मात् प्रमेयद्वित्वेन प्रमाणद्वित्वमिष्यते ॥ ६३ ॥\&[\smallbreak]


	
	  \endgroup
	

	  \pstart {\color{DodgerBlue3}“न”} च {\color{DodgerBlue3}“प्रत्यक्षपरोक्षाभ्यामन्यस्य प्रमेयस्य सम्भव”} इति दर्शितं प्राक् । {\color{DodgerBlue3}“तस्मात्प्रमेयस्य द्वित्वेन प्रमाणद्वित्वमिष्यते”} । प्रमेयमधिगच्छत् प्रमाणमुच्यते । (६३)
	\pend
      \leavevmode\marginnote{\textenglish{135/s}}\label{div_pvv.2.64}\edlabel{div_pvv.2.64}
	  
	% new div opening: depth here is 2
	

	  \begin{center}%% label @type='head'
	\textbf{ख. द्वितीयं प्रमाणमनुमानम्}
	\end{center}
	
	  \bigskip
	  \begingroup
	  \large
	
	    
	    \stanza[\smallbreak]
	\label{pv.2.64a}\edlabel{pv.2.64a}\flagstanza{\tiny\textenglish{....2.64a}}त्र्येकसंख्यानिरासो वा प्रमेयद्वयदर्शनात् ॥\&[\smallbreak]


	
	  \endgroup
	

	  \pstart तच्च द्विधमिति तद्‏ग्राहकं द्वयमपि प्रमाणमेव ॥ {\color{DodgerBlue3}“प्रमेयद्वयस्य दर्शनात्”} । {\color{DodgerBlue3}“त्र्येकसंख्यानिरासो वा”} बोद्धव्यः (।) तृतीयादिकं न प्रमाणं तृतीयादिप्रमेयाभावात् । नाप्येकं द्वितीयस्य प्रमेयस्य तेनानधिगतेः । न हि प्रत्यक्षं स्वलक्षणसामर्थ्यात् तदा-\leavevmode\marginnote{\textenglish{25b/MA}} कारग्राहि जातं सामान्यं प्रत्येति कल्पनागम्यत्वात्तस्य ॥
	\pend
      
	  \bigskip
	  \begingroup
	  \large
	
	    
	    \stanza[\smallbreak]
	\label{pv.2.64b}\edlabel{pv.2.64b}\flagstanza{\tiny\textenglish{....2.64b}}एकमेवाप्रमेयत्वादसतश्चेन्मतं च नः ॥ ६४ ॥\&[\smallbreak]


	
	  \endgroup
	

	  \pstart ननु प्रत्यक्ष{\color{DodgerBlue3}“मेकमेव”} प्रमाणमसतो{\color{DodgerBlue3}“ऽप्रमेयत्वात्”} । असच्च सामान्यमत्राह (।) {\color{DodgerBlue3}“असतोऽप्रमेयत्वं म\edlabel{pvv.135-1}\footnote{\label{pvv.135-1}  १ चार्व्वाकं {\color{DodgerBlue3}“प्रति सिद्धसाधनमाह ? तत्सत्तामात्रस्य”} ।}तञ्च नः”} । (६४)
	\pend
      \label{div_pvv.2.65_2.66}\edlabel{div_pvv.2.65_2.66}
	  
	% new div opening: depth here is 2
	
	  \bigskip
	  \begingroup
	  \large
	
	    
	    \stanza[\smallbreak]
	\label{pv.2.65a}\edlabel{pv.2.65a}\flagstanza{\tiny\textenglish{....2.65a}}अनेकान्तोऽप्रमेयत्वेऽसद्भावस्य विनिश्चयः ।&तन्निश्चयप्रमाणां वा द्वितीयं;\&[\smallbreak]


	
	  \endgroup
	

	  \pstart किमनिष्टमापद्यते । स्वलक्षणमेव तु पररूपेण गतेः सामान्यलक्षणमिष्टं । तच्च सदेवेति कथमप्रमेयं ततस्तत्साधनमपि प्रमाणमेव ॥ तथाऽसतोऽ{\color{DodgerBlue3}“प्रमेयत्वे”} साध्येऽसत्त्वादिति हेतुर{\color{DodgerBlue3}“नेकान्तोपि”} । तथा हि परलोकादेरसत्तया चा र्व्वा के णापीष्यत एव केनापि प्रमाणेन निश्चयः । अथवा प्रमेयत्वाभावस्याप्यसत्त्वहेतुनैव निश्चय इति व्यक्तमनैकान्तिकत्वं । अतश्च यत एव प्रमाणात्तस्या{\color{DodgerBlue3}“भावस्य निश्चय”}स्तदेवद्वितीयम्प्रमाणमनुमानं नाध्यक्षं ।
	\pend
      

	  \pstart कस्मादेवमित्याह (।)
	\pend
      
	  \bigskip
	  \begingroup
	  \large
	
	    
	    \stanza[\smallbreak]
	\label{pv.2.65b}\edlabel{pv.2.65b}\flagstanza{\tiny\textenglish{....2.65b}}नाक्षजा मतिः ॥ ३५ ॥\&[\smallbreak]


	
	  \endgroup
	
	  \bigskip
	  \begingroup
	  \large
	
	    
	    \stanza[\smallbreak]
	\label{pv.2.66}\edlabel{pv.2.66}\flagstanza{\tiny\textenglish{...v.2.66}}अभावेऽर्थबलाज्जातेरर्थशक्त्यनपेक्षणो ।&व्यवधानादिभावेपि जायेतेन्द्रियजा मतिः ॥ ६६ ॥\&[\smallbreak]


	
	  \endgroup
	

	  \pstart {\color{DodgerBlue3}“ना\edlabel{pvv.135-2}\footnote{\label{pvv.135-2}  २ परलोकादेः प्रत्यक्षादेवाभावनिश्चयोऽस्तीत्याह ।}क्षजा मति”}रभावे विषये प्रवर्ततेऽर्थस्य ग्राह्यस्य बलाज्जातेः । यद्बलेन प्रत्यक्षं प्रवर्तते तदेव\edlabel{pvv.135-3}\footnote{\label{pvv.135-3}  ३ प्रत्यक्षं निश्चिनोति ।}प्रतिपद्यते न चाभावस्य सामर्थ्य नाम । यदि पुनरर्थसामर्थ्यानपेक्षणमस्य तदा ग्राह्यस्यार्थस्य शक्त्यनपेक्षणे तद्‏व्यवधानादिभावेपीन्द्रियजा मतिर्जायेत । न चैतदस्ति । ततोऽर्थसामर्थ्यापेक्षि नाभावविषयं भवितुमर्हति । (६५,६६)
	\pend
      \label{div_pvv.2.67}\edlabel{div_pvv.2.67}
	  
	% new div opening: depth here is 2
	

	  \pstart स्यादेतत् (।)
	\pend
      \leavevmode\marginnote{\textenglish{136/s}}
	  \bigskip
	  \begingroup
	  \large
	
	    
	    \stanza[\smallbreak]
	\label{pv.2.67}\edlabel{pv.2.67}\flagstanza{\tiny\textenglish{...v.2.67}}अभावे विनिवृत्तिश्चेत् प्रत्यक्षस्यैव निश्चयः ।&विरुद्धं सैव वा लिङ्गमन्वयव्यतिरेकिणी ॥ ६७ ॥\&[\smallbreak]


	
	  \endgroup
	

	  \pstart {\color{DodgerBlue3}“प्रत्यक्षस्यैव”} प्रवृत्तिर्भावे सत्त्वनिश्चयो {\color{DodgerBlue3}“विनिवृत्तिश्चाभावे”} नि\edlabel{pvv.136-1}\footnote{\label{pvv.136-1}  १ प्रत्यक्षनिवृत्त्याभावस्य निश्चयश्च स्यात् ।}श्चयो न तु प्रमाणान्तरवृत्तिरिति चेत् ।\edlabel{pvv.136-2}\footnote{\label{pvv.136-2}  २ स्ववचनविरुद्धं ।} विरुद्धमिदं यः प्रत्यक्षाभावान्निश्चयः स प्रत्यक्षादिति । न हि प्रत्यक्षतनिवृत्त्योरैका\edlabel{pvv.136-3}\footnote{\label{pvv.136-3}  ३ ऐकात्म्यभाव एव स्यात् । ? अवैति । स्वपुत्रादौ ।}त्म्यं तथा ह्यभावो भाव एव स्यात् । भावोपि चाभावः । न च प्रत्यक्षतनिवृत्तावप्यवश्यमभावः । व्यवधानादिष्वर्थसत्त्वेपि तस्याभावात् । अथार्थान्वयव्यतिरेकानुविधायिनी प्रत्यक्षनिवृत्तिरेकज्ञानसंसर्गिपदार्थान्तरोपलब्धिरूपाऽभावनिश्चयहेतुस्तदा {\color{DodgerBlue3}“सैव”} प्रत्यक्षनिवृत्तिरनुपलब्ध्याख्या{\color{DodgerBlue3}“न्वयव्यतिरेकिणी लिङ्गमिति”} तज्जा प्रतीतिरनुमानमेवेति कथं नाप्रत्यक्षं प्रमाणं । (६७)
	\pend
      \label{div_pvv.2.68}\edlabel{div_pvv.2.68}
	  
	% new div opening: depth here is 2
	

	  \begin{center}%% label @type='head'
	\textbf{(ग. प्रमाणाद्वयसिद्धिः)}
	\end{center}
	

	  \pstart किञ्च (।)
	\pend
      
	  \bigskip
	  \begingroup
	  \large
	
	    
	    \stanza[\smallbreak]
	\label{pv.2.68a}\edlabel{pv.2.68a}\flagstanza{\tiny\textenglish{....2.68a}}सिद्धं च परचैतन्यप्रतिपत्तेः प्रमाद्वयम् ।\&[\smallbreak]


	
	  \endgroup
	

	  \pstart {\color{DodgerBlue3}“परचैतन्यप्रतिपत्तेः प्रमाणद्वयं सिद्धं”} । न हि प्रत्यक्षादर्व्वाग्दर्शनः परचैतन्यमवैति । किन्तु स्वसन्ताने बुद्धिपूर्वकत्वेनोपलब्धचेष्टादिदर्शनात्तदनुमानं परचित्तनिश्चयः किमस्तीत्याह (।)
	\pend
      
	  \bigskip
	  \begingroup
	  \large
	
	    
	    \stanza[\smallbreak]
	\label{pv.2.68b}\edlabel{pv.2.68b}\flagstanza{\tiny\textenglish{....2.68b}}व्यवहारादौ प्रवृत्तेश्च सिद्धस्तद्भावनिश्चयः ॥ ६८ ॥\&[\smallbreak]


	
	  \endgroup
	

	  \pstart परस्परप्रेषणाध्येषणव्यवहारादौ {\color{DodgerBlue3}“प्रवृत्तेश्च सिद्धश्च”} तस्याः परचैतन्यप्रतिपत्ते{\color{DodgerBlue3}“र्भावनिश्चयः”} । (६८)
	\pend
      \label{div_pvv.2.69}\edlabel{div_pvv.2.69}
	  
	% new div opening: depth here is 2
	

	  \begin{center}%% label @type='head'
	\textbf{(घ. अविसंवादादनुमानं प्रमाणम्)}
	\end{center}
	

	  \pstart नन्वनुमानाभिमता प्रतीतिर्नास्त्येवेति न ब्रूमः । किन्तु प्रामाण्ये तस्याः विप्रतिपद्यामह इत्याह (।)
	\pend
      
	  \bigskip
	  \begingroup
	  \large
	
	    
	    \stanza[\smallbreak]
	\label{pv.2.69}\edlabel{pv.2.69}\flagstanza{\tiny\textenglish{...v.2.69}}प्रमाणमविसंवादात् तत् क्वचिद् व्यभिचारतः ।&नाश्वास इति चेल्लिङ्गं-दुर्दृष्टिरतदीदृशम् ॥ ६९ ॥\&[\smallbreak]


	
	  \endgroup
	

	  \pstart {\color{DodgerBlue3}“प्रमाणन्त”}दनुमानम{\color{DodgerBlue3}“विसम्वादात्”}। प्रत्यक्षमपि हि सम्वादकत्वात् प्रमाणं तच्चानुमानस्यापि समानं । {\color{DodgerBlue3}“क्वचि”}च्छ्यामतादिसाधनार्थमुपात्ते तत्पुत्रादौ लिङ्गे \leavevmode\marginnote{\textenglish{137/s}} {\color{DodgerBlue3}“व्यभिचारतः”} सम्वादेऽ{\color{DodgerBlue3}“नाश्वास”} इति चेत् । लिङ्गस्य {\color{DodgerBlue3}“दुर्वृष्टि”}र्भ्रान्तिरलिङ्ग एव लिङ्गबुद्धिस्तस्या एतल्लि{\color{DodgerBlue3}“ङ्गमीदृशं”} व्यभिचारि भवतः प्रतिभाति । न खलु त्रिविधं लिङ्गं साध्यव्यभिचारि । यच्च व्यभिचारि तत्त्रिविधमेव न भवति । (६९)
	\pend
      \label{div_pvv.2.70}\edlabel{div_pvv.2.70}
	  
	% new div opening: depth here is 2
	

	  \pstart एतदेवाह (।)
	\pend
      
	  \bigskip
	  \begingroup
	  \large
	
	    
	    \stanza[\smallbreak]
	\label{pv.2.70}\edlabel{pv.2.70}\flagstanza{\tiny\textenglish{...v.2.70}}यतः कदाचित्सिद्धाऽस्य प्रतीतिर्वस्तुनः क्वचित् ।&तदवश्यं ततो जातं तत्स्वभावोपि वा भवेत् ॥ ७० ॥\&[\smallbreak]


	
	  \endgroup
	

	  \pstart {\color{DodgerBlue3}“यतो वस्तुनो”} धूमशिंशपादेः लिङ्गा{\color{DodgerBlue3}“दस्य”} साध्यस्य वह्रिवृक्षादेः {\color{DodgerBlue3}“क्वचिद्धर्मिणि कदाचिद”}नुमानकाले {\color{DodgerBlue3}“प्रतीतिः सिद्धा तद्”} धूमादिकं {\color{DodgerBlue3}“ततो”} वह्न्यादेर{\color{DodgerBlue3}“वश्यं जातं”} । तच्छिंशपादिकं तस्य वृक्षस्य {\color{DodgerBlue3}“स्वभावोपि वाऽवश्यम्भवेत्”} । (७०)
	\pend
      \label{div_pvv.2.71}\edlabel{div_pvv.2.71}
	  
	% new div opening: depth here is 2
	

	  \pstart न च कार्यस्वभावयोः कारणव्यापकव्यभिचारो यस्मात् (।)
	\pend
      
	  \bigskip
	  \begingroup
	  \large
	
	    
	    \stanza[\smallbreak]
	\label{pv.2.71}\edlabel{pv.2.71}\flagstanza{\tiny\textenglish{...v.2.71}}स्वनिमित्तात् स्वभावाद् वा विना नार्थस्य सम्भवः ॥&यच्च रूपं तयोर्दृष्टं तेदवान्यत्र लक्षणम् ॥ ७१ ॥\&[\smallbreak]


	
	  \endgroup
	

	  \pstart {\color{DodgerBlue3}“स्वस्य निमित्ता”}त्कारणाद्विना, {\color{DodgerBlue3}“स्वभावाद्”} व्यापकाद्वा {\color{DodgerBlue3}“विनाऽर्थस्य”} कार्यस्य\leavevmode\marginnote{\textenglish{26a/MA}} व्याप्यस्य च न {\color{DodgerBlue3}“सम्भवः”} । तदुत्पाद्यत्वात्तद्रूपत्वाच्च । ततस्तदाश्रयेणोत्पन्नाऽनुमानप्रतीतिरव्यभिचारिण्येव {\color{DodgerBlue3}“यच्च तयोर्धू”}मशिंशपयो {\color{DodgerBlue3}“रूपं”} साध्यकार्यस्वभाव\edlabel{pvv.137-1}\footnote{\label{pvv.137-1}  १ साध्येन सह कार्यस्य स्वभावस्य चाव्यभिचारनिमित्तं ।}त्वस्याव्यभिचारनिमित्तं {\color{DodgerBlue3}“दृष्टं तदेवान्य”}त्रापि हेतौ {\color{DodgerBlue3}“लक्षणं”} बोद्धव्यं । न च तत्पुत्रत्वं श्यामत्वस्य कार्यं स्वभावो वा । ततो यद् व्यभिचारि तदलिङ्गमेव । (७१)
	\pend
      \label{div_pvv.2.72}\edlabel{div_pvv.2.72}
	  
	% new div opening: depth here is 2
	

	  \begin{center}%% label @type='head'
	\textbf{(ङ. अनुपलब्धेः प्रतिबन्धः)}
	\end{center}
	

	  \pstart अनुपलब्धेरव्यभिचारं दर्शयितुमाह (।)
	\pend
      
	  \bigskip
	  \begingroup
	  \large
	
	    
	    \stanza[\smallbreak]
	\label{pv.2.72}\edlabel{pv.2.72}\flagstanza{\tiny\textenglish{...v.2.72}}स्वभावे स्वनिमित्ते वा दृश्ये दर्शनहेतुषु ।&अन्येषु सत्स्वदृश्ये च सत्ता वा तद्वतः कथम् ॥ ७२ ॥\&[\smallbreak]


	
	  \endgroup
	

	  \pstart प्रतिषेधस्य {\color{DodgerBlue3}“स्वभावे”} स्वस्य निषेध्यात्मनो {\color{DodgerBlue3}“निमित्ते”} का\edlabel{pvv.137-2}\footnote{\label{pvv.137-2}  २ सत्स्वप्यन्येषु ज्ञानानुत्पादात्}रणे । {\color{DodgerBlue3}“दृश्ये”} दर्शनयोग्ये {\color{DodgerBlue3}“दर्शन”}स्य हेतुष्विन्द्रियमनस्कारादिष्वन्येषु सत्सु विद्यमानेषुदृश्येऽनुपलभ्यमाने च {\color{DodgerBlue3}“तद्वतो”} दृश्यानुपलम्भवतो भावस्य {\color{DodgerBlue3}“सत्ता वा कथं\edlabel{pvv.137-3}\footnote{\label{pvv.137-3}  ३ सर्वथा नैवेत्यर्थः ।}”} युज्यते । न हि {\color{DodgerBlue3}“सत्स्वन्येषू”}पलम्भप्रत्ययेषु दृश्यस्य सतः कदाचिदनुपलम्भसम्भवः । \leavevmode\marginnote{\textenglish{138/s}} तस्माद् दृश्यानुपलब्धिरर्थाभाव एव भवतीति तत्प्रभवाऽभावप्रतीतिरविसम्वादिनी । तस्मात्साध्यप्रतिबद्धलिङ्ग-प्रसूतत्वात् त्रिविधलिङ्गजेप्यनुमाने नास्त्यनाश्वासः । (७२)
	\pend
      \label{div_pvv.2.73}\edlabel{div_pvv.2.73}
	  
	% new div opening: depth here is 2
	

	  \begin{center}%% label @type='head'
	\textbf{(नः अनुमानं त्रिविधम्)}
	\end{center}
	

	  \begin{center}%% label @type='head'
	\textbf{(छः तत्र वार्वाकमतनिरासः)}
	\end{center}
	
	  \bigskip
	  \begingroup
	  \large
	
	    
	    \stanza[\smallbreak]
	\label{pv.2.73a}\edlabel{pv.2.73a}\flagstanza{\tiny\textenglish{....2.73a}}अप्रामाण्ये च सामान्यबुद्धेस्तल्लोप आगतः ।&प्रेत्यभाववद्;\&[\smallbreak]


	
	  \endgroup
	

	  \pstart एवमपि {\color{DodgerBlue3}“त्वप्रामाण्ये”} सामान्यबुद्धेरिष्यमाणे तस्य परोक्षस्यार्थस्यानुमानव्यवस्थाप्यमानस्य {\color{DodgerBlue3}“लोपो”}ऽभाव {\color{DodgerBlue3}“आगतः”} । {\color{DodgerBlue3}“प्रेत्यभाववत्”} परलोकस्येव । यदि हि प्रत्यक्षमेकं प्रमाणं तदा यत्र तन्न प्रवर्तते तस्याप्य\edlabel{pvv.138-1}\footnote{\label{pvv.138-1}  १ एवञ्च कारकं तत् ज्ञापकञ्चेष्यते ।}भाव एव स्यात् । न चैतदस्ति । न हि चार्व्वाको देशान्तरस्थं स्वपितरमनुपलभमानस्तभावं व्यवस्थापयितुमर्हति ।
	\pend
      

	  \pstart स्यादेतत् (।)
	\pend
      
	  \bigskip
	  \begingroup
	  \large
	
	    
	    \stanza[\smallbreak]
	\label{pv.2.73b}\edlabel{pv.2.73b}\flagstanza{\tiny\textenglish{....2.73b}}अक्षैस्तत् पर्यायेण प्रतीयते ॥ ७३ ॥\&[\smallbreak]


	
	  \endgroup
	

	  \pstart नोपलभ्यमानमेवास्ति किन्तु {\color{DodgerBlue3}“पर्य्यायेण”} परिपाट्‏या‏ऽक्षैर्यत् {\color{DodgerBlue3}“प्रतीयते”} तदप्यस्ति चेत् । पित्रादयश्चोपलब्धा उपलप्स्यन्ते चेति सन्त्येव । (७३)
	\pend
      \label{div_pvv.2.74}\edlabel{div_pvv.2.74}
	  
	% new div opening: depth here is 2
	
	  \bigskip
	  \begingroup
	  \large
	
	    
	    \stanza[\smallbreak]
	\label{pv.2.74}\edlabel{pv.2.74}\flagstanza{\tiny\textenglish{...v.2.74}}तच्च नेन्द्रियशक्त्यादावक्षबुद्धेरसम्भवात् ।&अभावप्रतिपत्तौ स्याद् बुद्धेर्जन्मानिमित्तकम् ॥ ७४ ॥\&[\smallbreak]


	
	  \endgroup
	

	  \pstart {\color{DodgerBlue3}“तच्च न युक्तं”} इन्द्रियाख्यायां {\color{DodgerBlue3}“शक्ता”}वादिग्रहणादाहारादेः क्षुदुपघातादिसामर्थ्ये च पर्य्यायेणापीन्द्रिय{\color{DodgerBlue3}“बुद्धेरसम्भवादभावप्रतीतौ”} सत्यां प्रत्यक्षाया {\color{DodgerBlue3}“बुद्धेर्जन्मानिमित्तकं स्यात्”} । न हि दृष्टमात्रेभ्यो विषयालोकमनस्कारचक्षुर्गोलकेभ्योऽध्यक्षजन्म सत्स्वपि तेष्वभावात् व्यतिरे\edlabel{pvv.138-2}\footnote{\label{pvv.138-2}  २ अनित्यः शब्द इत्येकां बुद्धिं मन्यते ।}कादतिरिक्तं किञ्चिददृश्यं कारणमिष्यते यस्येन्द्रियमिति व्यपदेशः । तस्य पर्य्यायेणापि नाध्यक्षं ग्राहकमस्तीत्यभावः स्यात् । ततश्चाकारण\edlabel{pvv.138-3}\footnote{\label{pvv.138-3}  ३ असकलकारणं ।}कं प्रत्यक्षजन्म प्राप्तं । अहेतोश्च नित्यं सत्वमसत्त्वम्वा स्यादि (१।८२)त्युक्तं ॥ (७४)
	\pend
      \label{div_pvv.2.75}\edlabel{div_pvv.2.75}
	  
	% new div opening: depth here is 2
	

	  \begin{center}%% label @type='head'
	\textbf{(जः प्रत्यक्षान्न सामान्यप्रतीतिः)}
	\end{center}
	

	  \pstart स्यादेतत् (।) प्रत्यक्षादेव सामान्यप्रतीतिर्भविष्यति किमनुमानेनेत्याह (।)
	\pend
      \leavevmode\marginnote{\textenglish{139/s}}
	  \bigskip
	  \begingroup
	  \large
	
	    
	    \stanza[\smallbreak]
	\label{pv.2.75}\edlabel{pv.2.75}\flagstanza{\tiny\textenglish{...v.2.75}}स्वलक्षणे च प्रत्यक्षमविकल्पतया विना ।&विकल्पेन न सामान्यग्रहस्तस्मिस्ततोऽनुमा ॥ ७५ ॥\&[\smallbreak]


	
	  \endgroup
	

	  \pstart {\color{DodgerBlue3}“स्वलक्षणे च प्रत्यक्षमविकल्पतया प्रवर्तते सामान्यस्य तु ग्रहो विकल्पेन विना न”} भवति । {\color{DodgerBlue3}“ततः”} कारणा{\color{DodgerBlue3}“त्तस्मिन्”} सामान्येऽनुमैव विकल्पिका । नाध्यक्षमविकल्पकं ॥ (७५)
	\pend
      \label{div_pvv.2.76}\edlabel{div_pvv.2.76}
	  
	% new div opening: depth here is 2
	

	  \pstart ननु (।)
	\pend
      
	  \bigskip
	  \begingroup
	  \large
	
	    
	    \stanza[\smallbreak]
	\label{pv.2.76}\edlabel{pv.2.76}\flagstanza{\tiny\textenglish{...v.2.76}}प्रमेयनियमे वर्णानित्यता न प्रतीयते ।&प्रमाणमन्यत् तद्-बुद्धिर्विना लिङ्गेन संभवात् ॥ ७६ ॥\&[\smallbreak]


	
	  \endgroup
	

	  \pstart प्रत्यक्षं स्वलक्षणविषयमनुमानं सामान्यविषयमिति प्रमेयस्य {\color{DodgerBlue3}“नियमे”} स्वीक्रियमाणे {\color{DodgerBlue3}“वर्ण्ण”}स्य नीलादे\edlabel{pvv.139-1}\footnote{\label{pvv.139-1}  १ अन्यक्षणे नाशं दृष्ट्वा । आदिना शब्दस्यान्त्यक्षणस्यानित्यता । तथा हि शब्दादि स्वलक्षणं । अनित्यतादि सामान्यं । अनयोः संकरेण ग्रहणात् प्रमेयान्तरमेतत् ।}र्व्विषयस्या{\color{DodgerBlue3}“नित्यता”}ऽनित्यतासामान्यात्मता {\color{DodgerBlue3}“न प्रतीयत”} इति प्राप्तं । न हि सामान्यविशेषात्मकं प्रमेयं विशेषमात्रविषयेणाध्यक्षेण सामान्यमात्रविषयेणानुमानेन वा प्रत्येतुं शक्यं ।\edlabel{pvv.139-2}\footnote{\label{pvv.139-2}  २ अनित्यः शब्द इत्येकां बुद्धिं मन्यते ।} प्रतीयते चात{\color{DodgerBlue3}“स्तद्‏बुद्धिर्प्रमाणमन्यत्”} स्यात् । न प्रत्यक्षं सामान्यस्य ग्रहणात् । नाप्यनुमानं {\color{DodgerBlue3}“विना लिङ्गेन सम्भवात्”} । विशेषस्यापि ग्रहणाच्च । (७६)
	\pend
      \label{div_pvv.2.77}\edlabel{div_pvv.2.77}
	  
	% new div opening: depth here is 2
	

	  \pstart तथा (।)
	\pend
      
	  \bigskip
	  \begingroup
	  \large
	
	    
	    \stanza[\smallbreak]
	\label{pv.2.77}\edlabel{pv.2.77}\flagstanza{\tiny\textenglish{...v.2.77}}विशेषदृष्टे लिङ्गस्य सम्बन्धस्याप्रसिद्धितः ।&तत्प्रमाणान्तरं मेयबहुत्वाद् बहुतापि वा ॥ ७७ ॥\&[\smallbreak]


	
	  \endgroup
	

	  \pstart {\color{DodgerBlue3}“वि\edlabel{pvv.139-3}\footnote{\label{pvv.139-3}  ३ असकृद्वेति [प्रमाण]समुच्चयं व्याचष्टे । विशेषदष्टे च यज्ज्ञानं तदपि प्रमान्तरमित्याह ।}शेषदृष्टे”} प्रत्यक्षेणाग्निं दृष्ट्वा क्रमात्तमेव धूमाल्लिङ्गात् स एवायं व\edlabel{pvv.139-4}\footnote{\label{pvv.139-4}  ४ न सकृद्वृत्तेन समाप्तिः किन्तु तत्र पुनर्मानवृत्तिः समयान्तरे ।}ह्निरिति निश्चिनोत्यनुमानेन । {\color{DodgerBlue3}“लिङ्गस्य सम्बन्धासिद्धितः तत्प्रमाणान्तरं”}\leavevmode\marginnote{\textenglish{26b/MA}} स्यात् । न हि तत्प्रत्यक्षं लिङ्गं-बलादुत्पत्तेः । नाप्यनुमानं दहनधूम\edlabel{pvv.139-5}\footnote{\label{pvv.139-5}  ५ पर्व्वतवर्त्तिधूमेन वह्नेः सम्बन्धासिद्धेः ।}विशेषयोः सम्बन्धाग्रहणात् । तस्मात् {\color{DodgerBlue3}“मेया”}नां प्रत्यक्षं लक्षणं सामान्यं सामान्यविशेषाणां {\color{DodgerBlue3}“बहुत्वात् बहुतापि वा”} (७७)
	\pend
      \label{div_pvv.2.78}\edlabel{div_pvv.2.78}
	  
	% new div opening: depth here is 2
	
	  \bigskip
	  \begingroup
	  \large
	
	    
	    \stanza[\smallbreak]
	\label{pv.2.78}\edlabel{pv.2.78}\flagstanza{\tiny\textenglish{...v.2.78}}प्रमाणानामनेकस्य वृत्तेरेकत्र वा यथा ।&विशेषदृष्टेरेकत्रिसंख्यापोहो न वा भवेत् ॥ ७८ ॥\&[\smallbreak]


	
	  \endgroup
	

	  \pstart {\color{DodgerBlue3}“प्रमाणानां स्यात् । अनेकस्य”} वा प्रमाणस्यैकत्र विषये वृत्तेः प्रमाणबहुता स्यात् । \leavevmode\marginnote{\textenglish{140/s}} यथैकमेव स्वलक्षणं प्रत्यक्षेण विशेषदृष्टेन चानुमानेन प्रतीयते । प्रमेयस्य द्वित्वग्रहणयोर्नियमे प्रमाद्वित्वं स्यात् नान्यथा । त्र्येकसंख्याया अपोहो य उक्तः प्रमेयद्वित्वात् स {\color{DodgerBlue3}“न वा भवेत्”} ॥ (७८)
	\pend
      \label{div_pvv.2.79}\edlabel{div_pvv.2.79}
	  
	% new div opening: depth here is 2
	
	  \bigskip
	  \begingroup
	  \large
	
	    
	    \stanza[\smallbreak]
	\label{pv.2.79a}\edlabel{pv.2.79a}\flagstanza{\tiny\textenglish{....2.79a}}विषयानियमादन्यप्रमेयस्य च सम्भवात् ।\&[\smallbreak]


	
	  \endgroup
	

	  \pstart {\color{DodgerBlue3}“विषयस्य”} ग्रहणा{\color{DodgerBlue3}“नियमात्”} यदा ह्येकेनापि प्रमाणेनानेकं गृह्यते तदा प्रमेयद्वित्वेप्येकमेव प्रमाणं स्यात् । प्रमेयद्वया{\color{DodgerBlue3}“दन्यस्य च”} सामान्यविशेषस्य {\color{DodgerBlue3}“मेयस्य सम्भवात्”} । त्र्यादिकम्वा तद्‏ग्राहकं मानं भवेत् ।\edlabel{pvv.140-1}\footnote{\label{pvv.140-1}  १ अन्त्यमध्यक्षं प्रवाहविच्छेदे तद्विविक्तोपलम्भश्चेत्यध्यक्षद्वयजनितेन ।}
	\pend
      

	  \pstart अत्रोच्यते (।)
	\pend
      
	  \bigskip
	  \begingroup
	  \large
	
	    
	    \stanza[\smallbreak]
	\label{pv.2.79b}\edlabel{pv.2.79b}\flagstanza{\tiny\textenglish{....2.79b}}योजनाद् वर्णसामान्ये नायं दोषः प्रसज्यते ॥ ७९ ॥\&[\smallbreak]


	
	  \endgroup
	

	  \pstart विक\edlabel{pvv.140-2}\footnote{\label{pvv.140-2}  २ तदेवं प्रत्यक्षमनुमानञ्च प्रमाणे लक्षणद्वयं प्रमेयमित्याख्याय तस्य सन्धानेन प्रमाणान्तरं न च पुनः पुनरभिज्ञानेऽनिष्टासक्तेः स्मृतादिवत्यस्य वृत्तिर्यत्तर्हीदमनित्यादिभिराकारैर्वर्ण्णादि गृह्यतेऽसकृद्वेति व्याख्याता ।}ल्पकेन ज्ञानेनानित्यताया {\color{DodgerBlue3}“वर्ण्णसा\edlabel{pvv.140-3}\footnote{\label{pvv.140-3}  ३ न तु विशेषे विशेषस्य विकल्पकेनाग्रहात् ।}मान्ये योजनादयं”} सामान्यविशेषात्मकप्रमेयग्राहकप्रमाणान्तराभ्युपगमलक्षणो {\color{DodgerBlue3}“दोषो न प्रसंज्यते”} । न हि विशेषोऽनित्यतया योज्यते । विकल्पानामतद्‏विषयत्वस्योक्तेर्व्वक्ष्यमाणत्वाच्च । (७९)
	\pend
      \label{div_pvv.2.80}\edlabel{div_pvv.2.80}
	  
	% new div opening: depth here is 2
	

	  \begin{center}%% label @type='head'
	\textbf{(फ. अनित्यादयो नावस्तुधर्माः)}
	\end{center}
	

	  \pstart ननु वर्ण्णसामान्यस्यावस्तुत्वात्तद्योजिताऽनित्यतादयोऽवस्तु\edlabel{pvv.140-4}\footnote{\label{pvv.140-4}  ४ सामान्यधर्माः । ? न स्वलक्षणस्य}धर्माः स्युरित्याह (।)
	\pend
      
	  \bigskip
	  \begingroup
	  \large
	
	    
	    \stanza[\smallbreak]
	\label{pv.2.80}\edlabel{pv.2.80}\flagstanza{\tiny\textenglish{...v.2.80}}नावस्तुरूपं तस्यैव तथा सिद्धे प्रसाधनात् ।&अन्यत्र नान्यसिद्धिश्चेन्न तस्यैव प्रसिद्धितः ॥ ८० ॥\&[\smallbreak]


	
	  \endgroup
	

	  \pstart {\color{DodgerBlue3}“नावस्तुनो”} रूपमनित्यत्वादि । {\color{DodgerBlue3}“तस्यैव”} वस्तुनस्तथाऽनित्यत्वादिभिराकारैः {\color{DodgerBlue3}“सिद्धे”} र्निश्चयस्य प्राक् {\color{DodgerBlue3}“प्रसाधनात्”} । अध्यवसायानुरोधेन हि विकल्पानां विषयव्यवस्था (।) यद्यपि चैते स्वाकारग्राहिणस्तथापि बाह्यमेव विषयतया व्यवस्थाप्यन्ते अनाद्यभ्यासविशेषात् । अनुमानन्तु वस्तुप्रतिबद्धलिङ्गप्रभवत्वाद्यथावस्थितमेव वस्तु व्यवस्यतीति वस्त्वेव क्षणस्थितिधर्मकमस्मात्सिद्धं । ततो नावस्तुरूपमनित्यत्वादि ॥
	\pend
      \leavevmode\marginnote{\textenglish{141/s}}

	  \pstart स्यादेतद् (।) अ (न्य) त्रावस्तुनि सामान्येऽनित्यतादिसम्बन्धिनि सिध्यत्यन्यस्य वस्तुनोऽनित्यरूपस्य न सिद्धिरिति चेत् । नैतद्युक्तं {\color{DodgerBlue3}“त\edlabel{pvv.141-1}\footnote{\label{pvv.141-1}  १ स्वयं विप्लुतोपि वस्तुप्रतिबद्धजन्मतया वस्तु साधयन् प्रमाणं ।}स्यैव”} वस्तुनः तस्यैवानित्यरूपस्याध्यवसायवशेनानुमानात् {\color{DodgerBlue3}“प्रसिद्धितः”} । (८०)
	\pend
      \label{div_pvv.2.81}\edlabel{div_pvv.2.81}
	  
	% new div opening: depth here is 2
	

	  \begin{center}%% label @type='head'
	\textbf{(ञ. लिङ्गधीसंवादकता)}
	\end{center}
	

	  \pstart कथं पुनर्व्वस्त्वध्यवसायेऽपि तत्सम्वाद\edlabel{pvv.141-2}\footnote{\label{pvv.141-2}  २ दर्प्पणप्रतिबिम्बे तिलकादिसिद्ध्या मुखे तिलकादिसिद्धिवत् ।}कतेत्याह (।)
	\pend
      
	  \bigskip
	  \begingroup
	  \large
	
	    
	    \stanza[\smallbreak]
	\label{pv.2.81}\edlabel{pv.2.81}\flagstanza{\tiny\textenglish{...v.2.81}}यो हि भावो यथाभूतो स तादृग्लिङ्गचेतसः ।&हेतुस्तज्जा तथाभूते तस्माद् वस्तुनि लिङ्गिधीः ॥ ८१ ॥\&[\smallbreak]


	
	  \endgroup
	

	  \pstart {\color{DodgerBlue3}“यो हि”} साध्यधर्मो {\color{DodgerBlue3}“यथाभूतः”} का\edlabel{pvv.141-3}\footnote{\label{pvv.141-3}  ३ वह्निधूमादिजनकः । अनित्याकारवान् शब्दः । तल्लिङ्गयतीति तादृग्‏लिङ्गम् ।}रणव्यापकस्वभावः{\color{DodgerBlue3}“स तादृशः”} कारणकार्यतया व्यापकव्याप्यतया गृहीतव्याप्तिकस्य {\color{DodgerBlue3}“लिङ्गस्य चेतसः”} परंपराहेतुः ।\edlabel{pvv.141-4}\footnote{\label{pvv.141-4}  ४ इति कृत्वा} {\color{DodgerBlue3}“तज्जा तस्मा”}ल्लिङ्गचेतसो जाता । {\color{DodgerBlue3}“तथाभूते वस्तुनि”} कारणव्यापकरूपे साध्यधर्मे {\color{DodgerBlue3}“लिङ्गिधीः”} । तस्मात्परंपरया साध्यप्रतिबन्धात् लिङ्गिधीः सत्येव वस्तुनि भवन्ती तत्सम्वादात्प्रमाणमेव । (८१)
	\pend
      \label{div_pvv.2.82}\edlabel{div_pvv.2.82}
	  
	% new div opening: depth here is 2
	

	  \pstart ननु लिङ्गमपि लिङ्गिवत्सामान्यमेव । तथा धूमः कृतकं वेत्येव न लिङ्गं किन्तर्हि वह्निकार्यतयाऽनित्यत्वव्याप्यतया च गृहीतं । न च विशेषे व्याप्तिग्रहः । सामान्यञ्च नाध्यक्षगम्यं । विकल्पमात्रेण तत्प्रतीतावनाश्वासः । नैष दोषः । प्रत्यक्षेण कारणकार्ययोर्व्यावृत्तिद्वयविशिष्टयोर्गृ हीतयोर्व्विजातीयव्यावृत्त्याश्रयेणोत्पन्नविकल्पेन क्वचिदनुमानेन व्याप्तिं गृहीतवतः पश्चाद् धूमकृतकत्वादिदर्शनात्ताद्रूप्ये कार्यव्याप्यबुद्धिर्लिङ्गबुद्धिः (।) सा च तत्प्रतिबन्धादनुमानमेवेति नास्त्यनाश्वास (:।) एतदेवाह (।)
	\pend
      
	  \bigskip
	  \begingroup
	  \large
	
	    
	    \stanza[\smallbreak]
	\label{pv.2.82}\edlabel{pv.2.82}\flagstanza{\tiny\textenglish{...v.2.82}}लिङ्गलिङ्गिधियोरेवं पारम्पर्येण वस्तुनि ।&प्रतिबन्धात् तदाभासशून्ययोरप्यवंचनम् ॥ ८२ ॥\&[\smallbreak]


	
	  \endgroup
	

	  \pstart लिङ्गलिङ्गि{\color{DodgerBlue3}“धियोरेवमुक्त”}क्रमात् {\color{DodgerBlue3}“पारम्पर्येण वस्तुनि प्रतिबन्धात्त”}योर्लिङ्ग\leavevmode\marginnote{\textenglish{142/s}} लिङ्गिनो{\color{DodgerBlue3}“राभासः”} साक्षात् स्वरूपप्रतिभासः । {\color{DodgerBlue3}“तच्छून्य\edlabel{pvv.142-1a}\footnote{\label{pvv.142-1a}  1a प्रत्यक्षेण धूमादिः प्रतीयते न तस्य लिङ्गतापि निर्व्विकल्पेन ।\begin{english}\par
Placement of note uncertain; marked with a question mark in the edition (see encoding description for details).\end{english}}\edlabel{pvv.142-1}\footnote{\label{pvv.142-1}  १ वासनाप्रबोधात्तयोः कल्पनं वह्निर्न भात्येव लिङ्गत्वेनापि न भानं धूमस्यैव भानात् ।}योरपि”} लिङ्गलिङ्गिवस्तुनि \leavevmode\marginnote{\textenglish{27a/MA}} {\color{DodgerBlue3}“अवञ्चनं”} सम्वादनं । (८२)
	\pend
      \label{div_pvv.2.83}\edlabel{div_pvv.2.83}
	  
	% new div opening: depth here is 2
	

	  \pstart ननु लिङ्गबुद्धेरपि लिङ्गिबुद्धित्वात् पृथक् उपादानमनर्थकं (।) नानर्थकं । प्रत्यक्षमुद्भूतविकल्पो वा तद्बुद्धिरिति विप्रतिपत्तिनिरासार्थत्वात् ।\edlabel{pvv.142-2}\footnote{\label{pvv.142-2}  २ परम्परया वस्त्वविसंवादात् । ? वस्तुनि ।} यदि प्रमाणे लिङ्गलिङ्गिधियौ तदा प्रत्यक्षवदभ्रान्ते स्यातामित्याह (।)
	\pend
      
	  \bigskip
	  \begingroup
	  \large
	
	    
	    \stanza[\smallbreak]
	\label{pv.2.83}\edlabel{pv.2.83}\flagstanza{\tiny\textenglish{...v.2.83}}तद्रूपाध्यवसायाच्च तयोस्तद्रूपशून्ययोः ।&तद्रूपावञ्चकत्वेपि कृता भ्रान्तिव्यवस्थितिः ॥ ८३ ॥\&[\smallbreak]


	
	  \endgroup
	

	  \pstart {\color{DodgerBlue3}“तयो”}र्द्वयोः {\color{DodgerBlue3}“तद्रूप”}स्य लिङ्गलिङ्गि{\color{DodgerBlue3}“रूप”}स्या{\color{DodgerBlue3}“ध्यवसायात्”} प्रवर्तने सति वस्तुनि परम्परया तत्प्रतिबद्धतया च {\color{DodgerBlue3}“तद्रूप”}स्या{\color{DodgerBlue3}“वञ्चकत्वे”} सम्वादकत्वे{\color{DodgerBlue3}“पि भ्रान्तिव्यवस्थितिः”} कृता । कस्मादित्याह {\color{DodgerBlue3}“तद्रूपशून्ययोः”} । न हि लिङ्गलिङ्गिस्वरूपप्रतिभासिन्यौ धियाविमे स्वप्रतिभासेऽनर्थेऽर्थाध्यावसायेन प्रवृत्तत्वात् । (८३)
	\pend
      \label{div_pvv.2.84}\edlabel{div_pvv.2.84}
	  
	% new div opening: depth here is 2
	
	  \bigskip
	  \begingroup
	  \large
	
	    
	    \stanza[\smallbreak]
	\label{pv.2.84}\edlabel{pv.2.84}\flagstanza{\tiny\textenglish{...v.2.84}}तस्माद् वस्तुनि बोद्धव्ये व्यापकं व्याप्यचेतसः&निमित्तं तत्स्वभावो वा कारणं, तच्च तद्धियः ॥ ८४ ॥\&[\smallbreak]


	
	  \endgroup
	

	  \pstart {\color{DodgerBlue3}“तस्माद्वस्तुनि”} विधि\edlabel{pvv.142-3}\footnote{\label{pvv.142-3}  ३ विधिद्वारेण, न प्रतिषेधमुखेन ।}ना {\color{DodgerBlue3}“बोद्धव्ये व्यापकं”} साध्यं । {\color{DodgerBlue3}“व्याप्यचेतसो”} लिङ्गबुद्धे{\color{DodgerBlue3}“र्निमित्ते\edlabel{pvv.142-4}\footnote{\label{pvv.142-4}  ४ सद्‏व्यापकबलेन लिङ्गबुद्ध्युत्पत्तेः । ? निमित्तं ।}”} परम्परया । यस्मा{\color{DodgerBlue3}“त्स्वभावो”} वा तद् व्यापकं । व्याप्यस्य यथानित्यत्वं कृतकत्वस्य । {\color{DodgerBlue3}“कारण”}म्वा दहनो {\color{DodgerBlue3}“धूमस्य । तच्च”} व्याप्यचेतस{\color{DodgerBlue3}“स्तद्धियः”} स्वभावकारणधियो निमित्तमिति तयाध्यवसितस्य सम्वादनियमः सम्वादित्वमेवं प्र\edlabel{pvv.142-5}\footnote{\label{pvv.142-5}  ५ स्वलक्षणेन जनितं । स्वलक्षणव्यवस्थापकञ्च ।}त्यक्षस्यापि प्रामाण्यलक्षणं । (८४)
	\pend
      \label{div_pvv.2.85}\edlabel{div_pvv.2.85}
	  
	% new div opening: depth here is 2
	

	  \begin{center}%% label @type='head'
	\textbf{(२) अनुपलव्धिचिन्ता\edlabel{pvv.142-asterisk}[[* द्रष्टव्यं परिशिष्टं १।८]]}
	\end{center}
	
	  \bigskip
	  \begingroup
	  \large
	
	    
	    \stanza[\smallbreak]
	\label{pv.2.85}\edlabel{pv.2.85}\flagstanza{\tiny\textenglish{...v.2.85}}प्रतिषेधस्तु सर्वत्र साध्यतेऽनुपलम्भतः ।&सिद्धिं प्रमाणैर्वदतामर्थादेव विपर्ययात् ॥ ८५ ॥\&[\smallbreak]


	
	  \endgroup
	\leavevmode\marginnote{\textenglish{143/s}}

	  \pstart {\color{DodgerBlue3}“प्रतिषेध”}स्तु यत्र साक्षान्निषेध्यानुपलब्धिर्व्विरुद्धा ह्युपलब्धिर्व्वा दर्श्यते {\color{DodgerBlue3}“तत्र सर्व्वत्रानुपलम्भतः साध्यते”} । यस्मा{\color{DodgerBlue3}“त्प्रमाणै”}रर्थस्य {\color{DodgerBlue3}“सिद्धिं वदतामर्था”}त्सामार्थ्या{\color{DodgerBlue3}“देव विपर्यया”}त्प्रमाणाभावादर्थाभावः सिध्यति । (८५)
	\pend
      \label{div_pvv.2.86}\edlabel{div_pvv.2.86}
	  
	% new div opening: depth here is 2
	

	  \pstart ननु विरुद्धाद्युपलब्धावनुपलम्भः कथं निषेधसाधक इत्याह (।)
	\pend
      
	  \bigskip
	  \begingroup
	  \large
	
	    
	    \stanza[\smallbreak]
	\label{pv.2.86}\edlabel{pv.2.86}\flagstanza{\tiny\textenglish{...v.2.86}}दृष्टा विरुद्धधर्मोक्तिस्तस्य तत्कारणस्य वा ।&निषेधे यापि तस्यैव साऽप्रमाणत्वसूचना ॥ ८६ ॥\&[\smallbreak]


	
	  \endgroup
	

	  \pstart दृष्टा विरुद्धधर्मस्योक्तिर्या {\color{DodgerBlue3}“तस्य”} निषेध्यस्य {\color{DodgerBlue3}“तत्कारणस्य वा निषेधे”} कर्तव्ये यथा नात्र शीतस्पर्शो न वा रोमहर्षयुक्तपुरुषवानयं प्रदेशो वह्नेरिति (।) {\color{DodgerBlue3}“सापि तस्यैव”} निषेध्यस्यैवा{\color{DodgerBlue3}“प्रमाणत्वस्य”} प्रमाणरहितत्वस्य {\color{DodgerBlue3}“सूचना”} । विरुद्धस्य कारणविरुद्धत्वस्य चोपलम्भेन निषेध्यानुपलम्भ एव ख्याप्यते । (८६)
	\pend
      \label{div_pvv.2.87_2.88}\edlabel{div_pvv.2.87_2.88}
	  
	% new div opening: depth here is 2
	
	  \bigskip
	  \begingroup
	  \large
	
	    
	    \stanza[\smallbreak]
	\label{pv.2.87a}\edlabel{pv.2.87a}\flagstanza{\tiny\textenglish{....2.87a}}अन्यथैकस्य धर्मस्य स्वभावोक्त्या परस्य तत् ।&नास्तित्वं केन गम्येत;\&[\smallbreak]


	
	  \endgroup
	

	  \pstart {\color{DodgerBlue3}“अन्यथैकस्य”} विरुद्धादे{\color{DodgerBlue3}“र्धर्मस्य स्वभावोक्त्या”} सत्ताप्रतिपादनेन {\color{DodgerBlue3}“परस्य”} निषेध्यस्य तन्नास्तित्वं सिसाधयिषितं {\color{DodgerBlue3}“केन”} कारणेन {\color{DodgerBlue3}“गम्येत”} । न ह्यन्यसत्त्वेऽपरस्य सत्वाभावोतिप्रसङ्गात् । अनिषिद्धोपलब्धेरभावासिद्धेः ॥
	\pend
      
	  \bigskip
	  \begingroup
	  \large
	
	    
	    \stanza[\smallbreak]
	\label{pv.2.87b}\edlabel{pv.2.87b}\flagstanza{\tiny\textenglish{....2.87b}}विरोधाच्चेदसावपि ॥ ८७ ॥\&[\smallbreak]


	
	  \endgroup
	
	  \bigskip
	  \begingroup
	  \large
	
	    
	    \stanza[\smallbreak]
	\label{pv.2.88a}\edlabel{pv.2.88a}\flagstanza{\tiny\textenglish{....2.88a}}सिद्धः केन;\&[\smallbreak]


	
	  \endgroup
	

	  \pstart अथ यन्निषिध्यते यच्चोपदर्श्यते तयो{\color{DodgerBlue3}“र्व्विरोधात्”} सहानवस्थानलक्षणादेकभावेऽन्याभाव इति {\color{DodgerBlue3}“चेत्”} । नन्व{\color{DodgerBlue3}“सावपि”} (८७) विरोधः {\color{DodgerBlue3}“केन”} प्रका\edlabel{pvv.143-1}\footnote{\label{pvv.143-1}  १ प्रमाणेन ।}रेण {\color{DodgerBlue3}“सिद्धः”} ॥
	\pend
      
	  \bigskip
	  \begingroup
	  \large
	
	    
	    \stanza[\smallbreak]
	\label{pv.2.88b}\edlabel{pv.2.88b}\flagstanza{\tiny\textenglish{....2.88b}}असहस्थानादिति चेत् तत् कुतो मतम् ॥&दृश्यस्य दर्शनाभावादिति चेत् साऽप्रमाणता ॥ ८८ ॥\&[\smallbreak]


	
	  \endgroup
	

	  \pstart द्वयोः सहानवस्थानाद्विरोधः सिद्ध इति चेत् । तत्सहानवस्थानं कुतो हेतोर्मतं येन विरोधव्यवस्था ॥ अविकलकारणस्य प्रवर्तमानस्य {\color{DodgerBlue3}“दृश्यस्या”}न्यभावेऽ{\color{DodgerBlue3}“दर्शनात्”} सहानवस्थानगति{\color{DodgerBlue3}“रिति चेत्”} । ननु यदेवादर्शनं {\color{DodgerBlue3}“सा(ऽ)प्रमाणता”} प्रमाणरहितताऽनुपलब्धिरित्यर्थः । (८८)
	\pend
      \label{div_pvv.2.89}\edlabel{div_pvv.2.89}
	  
	% new div opening: depth here is 2
	
	  \bigskip
	  \begingroup
	  \large
	
	    
	    \stanza[\smallbreak]
	\label{pv.2.89}\edlabel{pv.2.89}\flagstanza{\tiny\textenglish{...v.2.89}}तस्मात् स्वशब्देनोक्ताऽपि साऽभावस्य प्रसाधिका ।&यस्याप्रमाणं साऽवाच्यो निषेधस्तेन सर्वथा ॥ ८९ ॥\&[\smallbreak]


	
	  \endgroup
	\leavevmode\marginnote{\textenglish{144/s}}

	  \pstart यस्मादवश्यं परम्परयानुपलब्धेरेव प्रतिषेधः । {\color{DodgerBlue3}“तस्मात्स्वशब्देना”}नुपलम्भस्वरूप वाचकशब्देनो{\color{DodgerBlue3}“क्ता”} स्वभावानुपलम्भादा{\color{DodgerBlue3}“वपि”} शब्दादनुक्तापि विरुद्धोपलब्ध्यादौ {\color{DodgerBlue3}“सानु”}पलब्धिर{\color{DodgerBlue3}“भावसाधिका”} । तस्या एव प्रतिपाद्यत्वात् । {\color{DodgerBlue3}“यस्य”} तु चा र्व्वा क स्य साऽनुपलब्धिर{\color{DodgerBlue3}“प्रमाणं”} प्रतिषेधे साध्ये {\color{DodgerBlue3}“तेन सर्व्वथा निषेधोऽवाच्यः”} प्रत्यक्षस्य भावमात्रविषयत्वात् । प्रमाणान्तरस्य चाभावात् । (८९)
	\pend
      \label{div_pvv.2.90}\edlabel{div_pvv.2.90}
	  
	% new div opening: depth here is 2
	
	  \bigskip
	  \begingroup
	  \large
	
	    
	    \stanza[\smallbreak]
	\label{pv.2.90}\edlabel{pv.2.90}\flagstanza{\tiny\textenglish{...v.2.90}}एतेन तद्विरुद्धार्थकार्योक्तिरुपवर्णिता ।&प्रयोगः केवलं भिन्नः सर्वत्रार्थो न भिद्यते ॥ ९० ॥\&[\smallbreak]


	
	  \endgroup
	

	  \pstart {\color{DodgerBlue3}“एतेन”} स्वभाव\edlabel{pvv.144-1}\footnote{\label{pvv.144-1}  १ विरुद्धोपलब्धि ।} {\color{DodgerBlue3}“त”}त्कारणविरुद्धोपलब्ध्योरनुपलब्धित्वप्रतिपादनेन तयोः \leavevmode\marginnote{\textenglish{27b/MA}} स्वभावतत्कार\edlabel{pvv.144-2}\footnote{\label{pvv.144-2}  २ स्वभावविरुद्धकार्योपलब्धिः । कारणविरुद्धकार्योपलब्धिश्च ।}णयोर्व्विरुद्धार्थस्य {\color{DodgerBlue3}“कार्योक्तिरुपवर्ण्णिता”} बोद्धव्या । यथा नात्र शीतस्पर्शो रोमहर्षवान्वा पुरुषो धूमादिति । अत्रापि निषेध्यानुपलम्भस्य परंपरया प्रतिपादनात् । उपलक्षणञ्चैतद् व्यापकविरुद्धतत्कार्योपलब्धी अपि बोद्धव्ये । यथा नात्र तुषारस्पर्शो वह्नेर्धूमादिति । एतासु स्वभावानुपलब्धिविरुद्धोपलब्ध्यादिषु {\color{DodgerBlue3}“प्रयोगः”} शब्दाभिधाव्यापारः परमनुपलम्भोपलम्भप्रतिपादकत्वेन भिद्यते । {\color{DodgerBlue3}“सर्व्वत्रार्थो”} निषेध्यानुपलम्भलक्षणो {\color{DodgerBlue3}“न भिद्यते”} तस्यैव सर्व्वत्र प्रतिपाद्यत्वात् । (९०)
	\pend
      \label{div_pvv.2.91}\edlabel{div_pvv.2.91}
	  
	% new div opening: depth here is 2
	
	  \bigskip
	  \begingroup
	  \large
	
	    
	    \stanza[\smallbreak]
	\label{pv.2.91}\edlabel{pv.2.91}\flagstanza{\tiny\textenglish{...v.2.91}}विरुद्धं तच्च सोपायमविधायापिधाय च ।&प्रमाणोक्तिर्निषेधे या न साम्नायानुसारिणी ॥ ९१ ॥\&[\smallbreak]


	
	  \endgroup
	

	  \pstart \edlabel{pvv.144-3}\footnote{\label{pvv.144-3}  ३ सांख्येन प्रतिबन्धान्तरं विरोध इष्टस्तादात्म्यतदुत्पत्तिभ्यामस्यासंग्रहादिति सोत्र स्वभावानुपलम्भेऽन्तर्भावितो विरोधस्यापि दृश्यानुपलम्भबलादेव सिद्धेः । तीर्थ्यास्तु विना विरुद्धविधिमनुपलम्भञ्च स्वतो निवृत्तेरित्याहुस्तानाह । या पुनरित्यादि ।}या पुनर्द्विविधेनापि विरोधेन विरुद्धमर्थमभिधा\edlabel{pvv.144-4}\footnote{\label{pvv.144-4}  ४ क्रियते प्रमाणोक्तिः । अकृत्वा ।} य यं कञ्चिदर्थमुपदर्श्यते तन्निषेध्यञ्च सहा\edlabel{pvv.144-5}\footnote{\label{pvv.144-5}  ५ सहानवस्थानपरस्परपरिहारविरोधाभ्यां ।}भावनिश्चयोपायेन कारणव्यापककार्यानुपलम्भे वर्तमानम{\color{DodgerBlue3}“विधाया”}त एवा{\color{DodgerBlue3}“पिधाय”} च पिधानमिव पिधानं स्वरूपप्रतीतिविरोधित्वा\edlabel{pvv.144-6}\footnote{\label{pvv.144-6}  ६ उपलब्धिलक्षणप्राप्तस्यानुपलब्धिमकृत्वाऽप्रदर्श्य ।}दनुपलम्भः। तमकृत्वा कारणाद्यनुपलम्भाप्रदर्शनात्तदुपलम्भसंभावना न व्याहतैव । एवं \leavevmode\marginnote{\textenglish{145/s}} निषेध्योपलम्भसम्भावनामनिवार्यान्यविधिना {\color{DodgerBlue3}“निषेधे”} कर्तव्ये {\color{DodgerBlue3}“प्रमा\edlabel{pvv.145-1}\footnote{\label{pvv.145-1}  १ आगमे सर्व्वविदो वचनकौशलदृष्टेः ।}णोक्तिर्या न साम्नायानुसारिणी”} । (९१)
	\pend
      \label{div_pvv.2.92}\edlabel{div_pvv.2.92}
	  
	% new div opening: depth here is 2
	
	  \bigskip
	  \begingroup
	  \large
	
	    
	    \stanza[\smallbreak]
	\label{pv.2.92a}\edlabel{pv.2.92a}\flagstanza{\tiny\textenglish{....2.92a}}उक्त्यादेः सर्ववित्प्रेत्यभावादिप्रतिषेधवत् ।\&[\smallbreak]


	
	  \endgroup
	

	  \pstart {\color{DodgerBlue3}“उक्त्यादेर्हेतोः । सर्व्वविदः प्रेत्यभावस्य”} परलोकस्य {\color{DodgerBlue3}“निषेधवत्”} । यथा न सर्व्वज्ञत्वमस्य पुरुषस्य वक्तृत्वात् प्रेत्यभावो पु\edlabel{pvv.145-2}\footnote{\label{pvv.145-2}  २ अर्हत्पुरुषवत् । सन्धानं ।}रुषत्वादित्यादि । (।)
	\pend
      

	  \pstart कस्मात्पुनरियं विरुद्धोक्तिरेव नेत्याह (।)
	\pend
      
	  \bigskip
	  \begingroup
	  \large
	
	    
	    \stanza[\smallbreak]
	\label{pv.2.92b}\edlabel{pv.2.92b}\flagstanza{\tiny\textenglish{....2.92b}}अतीन्द्रियाणामर्थानां विरोधस्याप्रसिद्धितः ॥ ९२ ॥\&[\smallbreak]


	
	  \endgroup
	

	  \pstart {\color{DodgerBlue3}“अतीन्द्रियाणां”} सार्व्वज्ञपरलोकादीना{\color{DodgerBlue3}“मर्थानां”} केनचित् वक्तृत्वादिना {\color{DodgerBlue3}“विरोधस्याप्रसिद्धितः”} । ज्ञानोत्कर्षापकर्षयोर्व्वचनापकर्षोत्कर्षादर्शनाद्विपर्ययदर्शनाच्च । (९२)
	\pend
      \label{div_pvv.2.93}\edlabel{div_pvv.2.93}
	  
	% new div opening: depth here is 2
	
	  \bigskip
	  \begingroup
	  \large
	
	    
	    \stanza[\smallbreak]
	\label{pv.2.93}\edlabel{pv.2.93}\flagstanza{\tiny\textenglish{...v.2.93}}बाध्यबाधकभावः कः स्यातां यद्युक्तिसंविदौ ।&तादृशोऽनुपलब्धेश्चेद् उच्यतां सैव साधनम् ॥ ९३ ॥\&[\smallbreak]


	
	  \endgroup
	

	  \pstart यदि चो{\color{DodgerBlue3}“क्तिसम्विदौ”} सह {\color{DodgerBlue3}“स्यातां क”}स्तयोः परस्परं {\color{DodgerBlue3}“बाध्यबाधकभावः”} । न हि ज्ञानवचनयोर्व्विरोधः सिद्धः येनैकसत्वेऽपरस्याभावः । अविरोधोपि न सिद्ध इतिचेत् सत्यं किन्तु संशयेपि न वक्तृत्वमनैकान्तिकं । {\color{DodgerBlue3}“तादृशः”} सर्व्वज्ञस्य {\color{DodgerBlue3}“वक्तुरनुपलब्धे”}रभावश्चेत् {\color{DodgerBlue3}“उच्यतां”} निषेधे साध्ये सैवानुपलब्धिः {\color{DodgerBlue3}“साधनं”} नात्रान्यस्य शक्तिः । (९३)
	\pend
      \label{div_pvv.2.94}\edlabel{div_pvv.2.94}
	  
	% new div opening: depth here is 2
	
	  \bigskip
	  \begingroup
	  \large
	
	    
	    \stanza[\smallbreak]
	\label{pv.2.94}\edlabel{pv.2.94}\flagstanza{\tiny\textenglish{...v.2.94}}अनिश्चयकरं प्रोक्तमीदृक् क्वानुपलम्भनम् ।&तन्नात्यन्तपरोक्षेषु सदसत्ताविनिश्चयौ ॥ ९४ ॥\&[\smallbreak]


	
	  \endgroup
	

	  \pstart किन्त्वीदृगतीन्द्रियार्थविषय{\color{DodgerBlue3}“मनुपलम्भनमनिश्चयकरं प्रोक्तं”} सत्यप्यर्थे सम्भवात् । {\color{DodgerBlue3}“तत्”} तस्मा{\color{DodgerBlue3}“वत्यन्तपरोक्षेषु सदसत्तानिश्चयौ न सतः”} । सत्यपि प्रमाणावृ\edlabel{pvv.145-3}\footnote{\label{pvv.145-3}  ३ सुमेर्व्वादौ ।}त्तेः । प्रमाणनिवृत्तावप्य\edlabel{pvv.145-4}\footnote{\label{pvv.145-4}  ४ पिशाचादेः ।}र्थाभावासिद्धेः । तस्मान्नानुप\edlabel{pvv.145-5}\footnote{\label{pvv.145-5}  ५ अदृश्यानुपलब्धेः ।}लब्धेः कुतश्चिदन्य\edlabel{pvv.145-6}\footnote{\label{pvv.145-6}  ६ अविरुद्धात् ।}स्मादभावसिद्धिः । किन्तु विरुद्धादेव । (९४)
	\pend
      \label{div_pvv.2.95}\edlabel{div_pvv.2.95}
	  
	% new div opening: depth here is 2
	
	  \bigskip
	  \begingroup
	  \large
	
	    
	    \stanza[\smallbreak]
	\label{pv.2.95}\edlabel{pv.2.95}\flagstanza{\tiny\textenglish{...v.2.95}}भिन्नोऽभिन्नोपि वा धर्मः स विरुद्धः प्रयुज्यते ।&यथाऽग्निरहिमे साध्ये सत्ता वा जन्मबाधनी ॥ ९५ ॥\&[\smallbreak]


	
	  \endgroup
	

	  \pstart {\color{DodgerBlue3}“स च विरुद्धो धर्मः”} क्वचिद् {\color{DodgerBlue3}“भिन्नो वा प्रयुज्यते यथाग्निरहिमे”} हिमाभावे {\color{DodgerBlue3}“साध्ये”} (।) {\color{DodgerBlue3}“क्वचिदभिन्नो वा यथा सत्ता जन्मनो बाधनी”} महदादीनां जन्माभावप्रसङ्गः \leavevmode\marginnote{\textenglish{146/s}} सत्त्वादिति सत्ताजन्मनोश्चाभेदः सां-ख्या\edlabel{pvv.146-1}\footnote{\label{pvv.146-1}  १ सदुत्पद्यते आविर्भावतिरोभावः परं वैभाष्योपि सद्वादी ।}भ्युपगमादुच्यते । न तु परमार्थतः तादात्म्यं विरुद्धयोरस्ति । विरुद्धे च सत्ताजन्मनी प्रागसत्तया हि अभिव्यक्तिरूपताय लाभो जन्मोच्यते । सत्ता तु विद्यमानता । ततो येन रूपेण सत्त्वं न तेन जन्म । येन च जन्म तेन न सत्त्वमिति व्यक्तः परस्परपरिहारस्थितिर्व्विरोधः । (९५)
	\pend
      \label{div_pvv.2.96}\edlabel{div_pvv.2.96}
	  
	% new div opening: depth here is 2
	
	  \bigskip
	  \begingroup
	  \large
	
	    
	    \stanza[\smallbreak]
	\label{pv.2.96}\edlabel{pv.2.96}\flagstanza{\tiny\textenglish{...v.2.96}}यथा वस्त्वेव वस्तूनां साधने साधनं मतम् ।&तथा वस्त्वेव वस्तूनां स्वनिवृत्तौ निवर्त्तकम् ॥ ९६ ॥\&[\smallbreak]


	
	  \endgroup
	

	  \pstart {\color{DodgerBlue3}“यथा”} च {\color{DodgerBlue3}“वस्तूनां”} साध्यानां {\color{DodgerBlue3}“साधने साधनं वस्त्वेव मतं”} नावस्तु । {\color{DodgerBlue3}“तथा वस्त्वेव स्वस्य निवृत्तौ वस्तूनां निवर्त्तकं”} नावस्तु । (९६)
	\pend
      \label{div_pvv.2.97}\edlabel{div_pvv.2.97}
	  
	% new div opening: depth here is 2
	
	  \bigskip
	  \begingroup
	  \large
	
	    
	    \stanza[\smallbreak]
	\label{pv.2.97}\edlabel{pv.2.97}\flagstanza{\tiny\textenglish{...v.2.97}}एतेन कल्पनान्यस्तो यत्र क्वचन सम्भवात् ।&धर्मः पक्षसपक्षान्यतरत्वादिरपोदितः ॥ ९७ ॥\&[\smallbreak]


	
	  \endgroup
	

	  \pstart {\color{DodgerBlue3}“एतेन”} वस्तुनः साधनत्वप्रतिपा\edlabel{pvv.146-2}\footnote{\label{pvv.146-2}  २ अनित्यः शब्दः पक्षसपक्षान्यतरत्वाद् घटवदित्यनित्येन प्रत्यासत्तौ दर्शितायां । तदैव वक्तुरिच्छावशान्नित्यः शब्दः पक्षसपक्षान्यतरत्वादाकाशवदिति नित्येन प्रत्यासत्तिरिति कम्बलिप्रोक्तं हेतुमपाकर्त्तुमाह ।}दनेन {\color{DodgerBlue3}“विकल्पनान्यस्तो धर्मः पक्षसपक्षान्यतरत्वादि”} । आदिशब्दान्मदुपगमादिश्चा{\color{DodgerBlue3}“पोदितः”} प्रतिक्षिप्तो बोद्धव्यः । तादृशस्य \leavevmode\marginnote{\textenglish{28a/MA}} साधनस्य {\color{DodgerBlue3}“यत्र क्वचन”} साध्ये {\color{DodgerBlue3}“सम्भवात्”} । न हि पक्षत्वं सपक्षत्वम्वा जातिनियतं किन्तु वाञ्छाधीनमवस्तु । अतः समीहितसिद्धौ सर्व्व सर्व्वस्य सिध्येत् । (९७)
	\pend
      \label{div_pvv.2.98}\edlabel{div_pvv.2.98}
	  
	% new div opening: depth here is 2
	
	  \bigskip
	  \begingroup
	  \large
	
	    
	    \stanza[\smallbreak]
	\label{pv.2.98}\edlabel{pv.2.98}\flagstanza{\tiny\textenglish{...v.2.98}}तत्रापि व्यापको धर्मो निवृत्तेर्गमको मतः ।&व्याप्यस्य स्वनिवृत्तिश्चेत् परिच्छिन्ना कथञ्चन ॥ ९८ ॥\&[\smallbreak]


	
	  \endgroup
	

	  \pstart {\color{DodgerBlue3}“तत्र वस्तुन्यपि व्यापको धर्मः”} कारणं स्वभावो वा । {\color{DodgerBlue3}“व्याप्यस्य”} कार्यस्य {\color{DodgerBlue3}“स्वभा”}वस्य {\color{DodgerBlue3}“निवृत्तेर्गमको मतः”} । व्यापकस्य कारणस्वभावस्य स्वनिवृत्तिः {\color{DodgerBlue3}“कथञ्चन”} साक्षात्परम्परया वा {\color{DodgerBlue3}“परिच्छिन्ना”} चेत् स्यात् । (९८)
	\pend
      \label{div_pvv.2.99}\edlabel{div_pvv.2.99}
	  
	% new div opening: depth here is 2
	
	  \bigskip
	  \begingroup
	  \large
	
	    
	    \stanza[\smallbreak]
	\label{pv.2.99a}\edlabel{pv.2.99a}\flagstanza{\tiny\textenglish{....2.99a}}यदप्रमाणताभावे लिङ्गं तस्यैव कथ्यते ।&तदत्यन्तविमूढार्थं;\&[\smallbreak]


	
	  \endgroup
	

	  \pstart तथा यस्यार्थस्य एकज्ञानसंसर्ग्गिणोऽ{\color{DodgerBlue3}“प्रमाणता”} प्रमाणनिवृत्तिस्तस्यैवा{\color{DodgerBlue3}“भावे-”} ऽभावव्यवहारे च साध्ये सा प्रमाणनिवृत्तिः स्वभावानुपलम्भाख्या {\color{DodgerBlue3}“लिङ्गं कथ्यते । तत्स्व”}भावानुपलब्धिलिङ्गम{\color{DodgerBlue3}“त्यन्तविमूढार्थ”} । ये ह्यनुपलभ्यमानमपि दृश्यमसत्तया मोहान्न व्यवहरन्ति तान् प्रति व्यवहारसाधकं तल्लिङ्गं ।
	\pend
      \leavevmode\marginnote{\textenglish{147/s}}

	  \pstart कस्मात्पुनरमूढार्थमपि तन्नेत्याह (।)
	\pend
      
	  \bigskip
	  \begingroup
	  \large
	
	    
	    \stanza[\smallbreak]
	\label{pv.2.99b}\edlabel{pv.2.99b}\flagstanza{\tiny\textenglish{....2.99b}}आगोपालमसंवृत्तेः ॥ ९९ ॥\&[\smallbreak]


	
	  \endgroup
	

	  \pstart {\color{DodgerBlue3}“आगोपालम”}स्यार्थस्या{\color{DodgerBlue3}“संवृ\edlabel{pvv.147-1}\footnote{\label{pvv.147-1}  १ विवृतौ}त्ते”}रगूढत्वात् प्रसिद्धेरित्यर्थः । (९९)
	\pend
      \label{div_pvv.2.100}\edlabel{div_pvv.2.100}
	  
	% new div opening: depth here is 2
	
	  \bigskip
	  \begingroup
	  \large
	
	    
	    \stanza[\smallbreak]
	\label{pv.2.100}\edlabel{pv.2.100}\flagstanza{\tiny\textenglish{....2.100}}एतावन्निश्चयफलमभावेनुपलम्भनम् ।&तच्च हेतौ स्वभावे वाऽदृश्ये दृश्यतया मते ॥ १०० ॥\&[\smallbreak]


	
	  \endgroup
	

	  \pstart {\color{DodgerBlue3}“एताव”}त्त्रिविधम् । कारणव्यापकस्वभा{\color{DodgerBlue3}“वानुपलम्भनं ।”} अभावेऽभावव्यवहारे च {\color{DodgerBlue3}“निश्चयफलं”} । कारणविरुद्धोपलम्भादयस्तु कारणाद्यनुपलम्भोपलक्षणत्वात् त्रिविध एवान्तर्भूताः । {\color{DodgerBlue3}“तच्चा”}भावनिश्चयफलत्वमनुपलम्भस्य {\color{DodgerBlue3}“हेतौ स्वभावे च”} व्यापके निषेध्यरूपे च {\color{DodgerBlue3}“दृश्यतया”} दर्शनयोग्यतया मतेऽ{\color{DodgerBlue3}“दृश्ये”}ऽनुपलभ्यमाने सति नान्यथा ।
	\pend
      

	  \pstart तदेवं त्रिविधलिङ्गजमनुमानं प्रमाणं वस्तुसम्वादात् । ततस्तत्प्रसिद्धस्यानित्यतादेर्न वस्तुधर्मतेति स्थितं ।\edlabel{pvv.147-2}\footnote{\label{pvv.147-2}  २ आनुषङ्गिकं समाप्य प्रकृतमाह ।}(१००)
	\pend
      \label{div_pvv.2.101}\edlabel{div_pvv.2.101}
	  
	% new div opening: depth here is 2
	

	  \pstart नन्वनित्योऽयं वर्ण्ण इति स्वलक्षणे योजना तत्कथं योजनाद्वर्ण्णसामान्य\cref{pv.2.79b} इत्युक्तं । अत्राह (।)
	\pend
      
	  \bigskip
	  \begingroup
	  \large
	
	    
	    \stanza[\smallbreak]
	\label{pv.2.101}\edlabel{pv.2.101}\flagstanza{\tiny\textenglish{....2.101}}अनुमानादनित्यादेर्ग्रहणेऽयं क्रमो मतः ।&प्रामाण्यमेव नान्यत्र गृहीतग्रहणान्मतम् ॥ १०१ ॥\&[\smallbreak]


	
	  \endgroup
	

	  \pstart {\color{DodgerBlue3}“अनुमाना”}त्परोक्षस्या{\color{DodgerBlue3}“नित्यादे”}र्धर्मस्य {\color{DodgerBlue3}“ग्र\edlabel{pvv.147-3}\footnote{\label{pvv.147-3}  ३ सति नान्यदायं क्रमः ।}हणे”} वस्तुसामान्यमनित्यत्वेन गृह्यत इत्ययं {\color{DodgerBlue3}“क्रमो मतः”} । वर्ण्णसामान्यञ्च प्रत्य\edlabel{pvv.147-4}\footnote{\label{pvv.147-4}  ४ सन्तानोपलम्भावस्थायामनित्यमेतद्ववर्ण्णादिकं । विकारित्वात् । जलादिवदिति स्वभावहेतुरतो वस्तुन एवात्र सिद्धिः ।}क्षतः सिद्धं तद्‏बलभाविना विकल्पेन विजातीयव्यावृत्त्याश्रयेण व्यवस्थापनात् । यस्तु विनश्वरं वर्ण्णं दृष्ट्वाऽनित्योऽयमिति विशेषविषयः अनुमानादन्यत्र । तत्र {\color{DodgerBlue3}“प्रामाण्यमेव न मतं गृहीतग्रहणात्”} (। १०१)
	\pend
      \label{div_pvv.2.102}\edlabel{div_pvv.2.102}
	  
	% new div opening: depth here is 2
	

	  \pstart कथं गृहीतग्राहित्वमित्याह (।)
	\pend
      
	  \bigskip
	  \begingroup
	  \large
	
	    
	    \stanza[\smallbreak]
	\label{pv.2.102a}\edlabel{pv.2.102a}\flagstanza{\tiny\textenglish{...2.102a}}नान्यास्यानित्यता भावात् पूर्वसिद्धः स चैन्द्रियात् ।&नानेकरूपो वाच्योऽसौ;\&[\smallbreak]


	
	  \endgroup
	\leavevmode\marginnote{\textenglish{148/s}}

	  \pstart {\color{DodgerBlue3}“नान्या भावाद्व\edlabel{pvv.148-1}\footnote{\label{pvv.148-1}  १ चलात् ।}र्ण्णादेरनित्यता”} तस्यैव क्षणक्षयिस्वभावत्वात् । {\color{DodgerBlue3}“स च भाव ऐन्द्रियात्प्रत्यक्षत्सिद्धः”} । पूर्व्व विकल्पात् । ततस्तमेव यथागृहीतं विकल्पयन् विकल्पो गृहीतग्राही । यत एव नान्याभावादनित्यता । अत एवाभावोऽसौ धर्मिधर्मरूपतया {\color{DodgerBlue3}“नानेकरूपः”} । तथा न {\color{DodgerBlue3}“वाच्योऽसौ”} शब्दानां धर्मिधर्माभावस्य तैर्व्वचनात् । तस्य च वस्तुन्यसम्भवात् ।
	\pend
      

	  \pstart कस्तर्हि वाच्य इत्याह (।)
	\pend
      
	  \bigskip
	  \begingroup
	  \large
	
	    
	    \stanza[\smallbreak]
	\label{pv.2.102b}\edlabel{pv.2.102b}\flagstanza{\tiny\textenglish{...2.102b}}वाच्यो धर्मो विकल्पजः ॥ १०२ ॥\&[\smallbreak]


	
	  \endgroup
	

	  \pstart {\color{DodgerBlue3}“विकल्पजो”} विजातीयाश्रयेण विकल्पकल्पितो {\color{DodgerBlue3}“धर्मः”} परस्परं यथासंकेतमसंकीर्ण्णः शब्दानां {\color{DodgerBlue3}“वाच्यः”} । (१०२)
	\pend
      \label{div_pvv.2.103}\edlabel{div_pvv.2.103}
	  
	% new div opening: depth here is 2
	
	  \bigskip
	  \begingroup
	  \large
	
	    
	    \stanza[\smallbreak]
	\label{pv.2.103}\edlabel{pv.2.103}\flagstanza{\tiny\textenglish{....2.103}}सामान्याश्रयसंसिद्धौ सामान्यं सिद्धमेव तत् ।&तदसिद्धौ तथास्यैव ह्यनुमानं प्रवर्तते ॥ १०३ ॥\&[\smallbreak]


	
	  \endgroup
	

	  \pstart तस्मात्सामान्यस्यानित्यादेराश्रयस्य वर्ण्णादेः प्रत्यक्षात् {\color{DodgerBlue3}“संसिद्धौ सिद्धमेव तत्सामान्यं”} । यदि त्वभ्यासाद्यभावादनुरूपनिश्चयाभावादज्ञानं तदा प्रत्यक्षा{\color{DodgerBlue3}“त्तदसिद्धावस्यैव”} हि वर्ण्णादेस्तथा नित्यत्वेन तदव्यभिचारिलि{\color{DodgerBlue3}“ङ्गदनुमानं प्रवर्तते”} । (१०३)
	\pend
      \label{div_pvv.2.104}\edlabel{div_pvv.2.104}
	  
	% new div opening: depth here is 2
	

	  \pstart कस्मात्पुनर्गृहीतेप्यपरिज्ञानमित्याह (।)
	\pend
      
	  \bigskip
	  \begingroup
	  \large
	
	    
	    \stanza[\smallbreak]
	\label{pv.2.104}\edlabel{pv.2.104}\flagstanza{\tiny\textenglish{....2.104}}क्वचित्तदपरिज्ञानं सदृशापरसम्भवात् ।&भ्रान्तेरपश्यतो भेदं मायागोलकभेदवत् ॥ १०४ ॥\&[\smallbreak]


	
	  \endgroup
	

	  \pstart {\color{DodgerBlue3}“क्वचिद”}नित्यतादौ गृहीतेप्यनुरूपनिश्चयाभावाद{\color{DodgerBlue3}“परिज्ञानं”} । पूर्व्वक्षणविनाशकाले तत्सदृशस्यापरस्य {\color{DodgerBlue3}“सम्भवात्”} । भावशून्यान्तरा लक्षणाभावात् । स एवायमिति \leavevmode\marginnote{\textenglish{28b/MA}} भ्रान्तेः स्थैर्यग्राहिण्याः क्षणानां {\color{DodgerBlue3}“भेदमपश्यतो”}ऽनिश्चिचन्वतः पुंसो दृष्टान्तमाह (।) {\color{DodgerBlue3}“मायागोलकभेदवत्”} । मायाकारदर्शितौ गोलकौ भिन्नावध्यक्षेण गृहीत्वापि सादृश्याल्लाघवाच्च विप्रलब्धबुद्धिरेकत्वेनाध्यवस्यति यथा तथा स्थिराध्यवसायः । (१०४)
	\pend
      \label{div_pvv.2.105}\edlabel{div_pvv.2.105}
	  
	% new div opening: depth here is 2
	

	  \pstart ननु यदि क्वचिन्नाशो दृश्येंत तदास्या\edlabel{pvv.148-2}\footnote{\label{pvv.148-2}  २ दृश्येतैव दीपादिरिति नित्यं गौः भावी ।}हेतुकस्य स्वभावत्वात् क्षणक्षयिषु भावेषु एकताबुद्धेर्भ्रान्तत्वं भवेदित्याह (।)
	\pend
      
	  \bigskip
	  \begingroup
	  \large
	
	    
	    \stanza[\smallbreak]
	\label{pv.2.105}\edlabel{pv.2.105}\flagstanza{\tiny\textenglish{....2.105}}तथा ह्यलिङ्गमाबालमसंश्लिष्टोत्तरोदयम् ।&पश्यन् परिच्छिनत्त्येव दीपादिं नाशिनं जनः ॥ १०५ ॥\&[\smallbreak]


	
	  \endgroup
	\leavevmode\marginnote{\textenglish{149/s}}

	  \pstart {\color{DodgerBlue3}“तथा”} हि {\color{DodgerBlue3}“दीपादि”}(।) आदिशब्दात्तुरङ्गदिसन्तानविच्छेदकालेऽ{\color{DodgerBlue3}“संश्लिष्टोत्तरो-”} दयमघटित उत्तरस्य उदयो जन्म यस्मिन् तमेकस्य स्वरसनिरोधित्वेऽपरस्य कारणाभावात् । अनुत्पत्तौ {\color{DodgerBlue3}“नाशिनम”}ध्यक्षतः {\color{DodgerBlue3}“पश्यन्न”}लिङ्गङ्गमकलिङ्गरहितमाबालं बालपर्यन्तं जनः {\color{DodgerBlue3}“परिच्छिनत्त्येव”} । ततः प्रागपि सतः स्वरसनिरोधात् सदृशापरापरक्षणप्रचये स एवायमिति बुद्धिभ्रान्तिरेव । (१०५)
	\pend
      \label{div_pvv.2.106}\edlabel{div_pvv.2.106}
	  
	% new div opening: depth here is 2
	

	  \pstart यथा (।)
	\pend
      
	  \bigskip
	  \begingroup
	  \large
	
	    
	    \stanza[\smallbreak]
	\label{pv.2.106}\edlabel{pv.2.106}\flagstanza{\tiny\textenglish{....2.106}}भावस्वभावभूतायामपि शक्तौ फलेऽदृशः ।&अनानन्तर्यतो मोहो विनिश्चेतुरपाटवात् ॥ १०६ ॥\&[\smallbreak]


	
	  \endgroup
	

	  \pstart {\color{DodgerBlue3}“भावस्य”}बीजादेः {\color{DodgerBlue3}“स्वभावभूतायामपि”} अङ्कुरादिजनिकायां {\color{DodgerBlue3}“शक्तौ फलेऽङ्कुरादौ अनान\edlabel{pvv.149-1}\footnote{\label{pvv.149-1}  १ फलस्यानन्तरमभावात्}न्तर्यतोऽदृशो”}ऽदर्शनात् {\color{DodgerBlue3}“विनिश्चेतुः”} पुन्सोऽ{\color{DodgerBlue3}“पाटवात् मोहो”}ऽशक्तभ्रमः । तथा क्षणिकेषु भावेषु सदृशापरापरोत्पत्तेर्भावशून्यक्षणादर्शनाच्च स्थिरभ्रमः । (१०६)
	\pend
      \label{div_pvv.2.107}\edlabel{div_pvv.2.107}
	  
	% new div opening: depth here is 2
	
	  \bigskip
	  \begingroup
	  \large
	
	    
	    \stanza[\smallbreak]
	\label{pv.2.107}\edlabel{pv.2.107}\flagstanza{\tiny\textenglish{....2.107}}तस्यैव विनिवृत्यर्थमनुमानोपवर्णनम् ।&व्यवस्यन्तीक्षणादेव सर्वाकारान् महाधियः ॥ १०७ ॥\&[\smallbreak]


	
	  \endgroup
	

	  \pstart तस्यैकत्वभ्रमस्यैव {\color{DodgerBlue3}“निवृत्त्यर्थ”} सम्वादिलिङ्गजस्य क्षणिकता विषयस्यानु{\color{DodgerBlue3}“मानस्योपवर्ण्णनं”} निश्चयारोपमनसोर्ब्बाध्यबाधकभावत (१।५०) इति न्यायात् । ये तु {\color{DodgerBlue3}“महाधियो”} विपरीतव्यवसायानाक्रान्तप्रत्यक्षा योगिनस्ते पदार्थस्य {\color{DodgerBlue3}“ईक्षणादेव सर्व्वानाकारन् व्यवस्यन्ति”} निश्चिन्वन्ति । (१०७)
	\pend
      \label{div_pvv.2.108}\edlabel{div_pvv.2.108}
	  
	% new div opening: depth here is 2
	

	  \pstart एवञ्च (।)
	\pend
      
	  \bigskip
	  \begingroup
	  \large
	
	    
	    \stanza[\smallbreak]
	\label{pv.2.108}\edlabel{pv.2.108}\flagstanza{\tiny\textenglish{....2.108}}व्यावृत्ते सर्वतस्तस्मिन् व्यावृत्तिविनिबन्धनाः ।&बुद्धयोऽर्थे प्रवर्त्तन्तेऽभिन्ने भिन्नाश्रया इव ॥ १०८ ॥\&[\smallbreak]


	
	  \endgroup
	

	  \pstart {\color{DodgerBlue3}“तस्मिन्नर्थे”} वस्तुतोऽभिन्ने निरंशे {\color{DodgerBlue3}“सर्व्वतः”} सजातीयाद् विजातीयाच्च {\color{DodgerBlue3}“व्यावृत्ते”} यावत्त्यो व्यावृत्तयः सन्ति तावद् {\color{DodgerBlue3}“व्यावृत्तिविनिबन्धना”} निश्चयात्मिका {\color{DodgerBlue3}“बुद्धयो भिन्नाश्रया इव”} तत्तद्‏व्यावृत्तिमात्रविषयत्वेन भिन्नधर्मिधर्म्मादिगोचरा इव {\color{DodgerBlue3}“प्रवर्तन्ते”} । (१०८)
	\pend
      \label{div_pvv.2.109}\edlabel{div_pvv.2.109}
	  
	% new div opening: depth here is 2
	
	  \bigskip
	  \begingroup
	  \large
	
	    
	    \stanza[\smallbreak]
	\label{pv.2.109}\edlabel{pv.2.109}\flagstanza{\tiny\textenglish{....2.109}}यथाचोदनमाख्याश्च । सोऽसति भ्रान्तिकारणे ।&प्रतिभाः प्रतिसन्धत्ते स्वानुरूपाः स्वभावतः ॥ १०९ ॥\&[\smallbreak]


	
	  \endgroup
	

	  \pstart {\color{DodgerBlue3}“यथाचोदनं”} यथासंकेत{\color{DodgerBlue3}“माख्याः”} शब्दा{\color{DodgerBlue3}“श्च”} प्रवर्तन्ते । यत्र व्यावृत्तौ यः शब्दो विनिवेशितः । {\color{DodgerBlue3}“स”} च तस्यां निश्चितायां प्रवर्तते स भावोऽ{\color{DodgerBlue3}“सति भ्रान्तिकारणे”} यथाभ्यासं {\color{DodgerBlue3}“स्वानुरूपाः”} स्वव्यावृत्तिसमुचिताः {\color{DodgerBlue3}“प्रतिभा”} निश्चयबुद्धीः {\color{DodgerBlue3}“स्वभावतः प्रति-”} \leavevmode\marginnote{\textenglish{150/s}} {\color{DodgerBlue3}“संधत्ते”} उत्पादयति । तस्मादभ्यासवतामीक्षणादेवानित्यादिनिश्चयः अन्येषान्तु स्थिरारोपणार्थमनुमानमिति स्थितमेतत् । (१०९)
	\pend
      \label{div_pvv.2.110}\edlabel{div_pvv.2.110}
	  
	% new div opening: depth here is 2
	

	  \pstart अन्ये त्वाचार्याः प्राहुः ।\edlabel{pvv.150-1}\footnote{\label{pvv.150-1}  १ तेषां प्रध्वंसाभावेऽनित्यताभिमता स यथा सिध्यति तद्दर्शयति ।}
	\pend
      
	  \bigskip
	  \begingroup
	  \large
	
	    
	    \stanza[\smallbreak]
	\label{pv.2.110a}\edlabel{pv.2.110a}\flagstanza{\tiny\textenglish{...2.110a}}सिद्धोऽत्राप्यथवा ध्वंसो लिङ्गादनुपलम्भनात् ।\&[\smallbreak]


	
	  \endgroup
	

	  \pstart {\color{DodgerBlue3}“अत्र”} वर्ण्णसन्तानविच्छेदकालेपि सन्तानविच्छेदलक्षणो ध्वंसोऽनित्यता न भावस्वभावात्मिका । स {\color{DodgerBlue3}“चानुप\edlabel{pvv.150-2}\footnote{\label{pvv.150-2}  २ उपलब्धिलक्षणप्राप्तस्यान्त्यक्षणस्य ।}लम्भनाल्लिङ्गत्सिद्धः”}। नाध्यक्षाद् भावविषयत्वात्तस्य ।
	\pend
      

	  \pstart कस्मात्पुनर्ध्वंसोऽनित्यता इत्याह (।)
	\pend
      
	  \bigskip
	  \begingroup
	  \large
	
	    
	    \stanza[\smallbreak]
	\label{pv.2.110b}\edlabel{pv.2.110b}\flagstanza{\tiny\textenglish{...2.110b}}प्राग् भूत्वा ह्यभवद्भावोनित्य इत्यभिधीयते ॥ ११० ॥\&[\smallbreak]


	
	  \endgroup
	

	  \pstart {\color{DodgerBlue3}“प्राग् भूत्वा हि भावः”} पश्चाद{\color{DodgerBlue3}“भवन्ननित्य इत्यभिधीयते”} न तु भाव इत्येव । तथा ध्वंस एवानित्यता सा चानुपलब्धिलिङ्गजाऽनुमानगम्या । (११०)
	\pend
      \label{div_pvv.2.111}\edlabel{div_pvv.2.111}
	  
	% new div opening: depth here is 2
	

	  \pstart ननु प्राक् पश्चादभावयोर्व्यवधायकः सत्तासम्बन्धोऽनित्यता न प्रध्वंसाभाव इत्याह (।)
	\pend
      
	  \bigskip
	  \begingroup
	  \large
	
	    
	    \stanza[\smallbreak]
	\label{pv.2.111}\edlabel{pv.2.111}\flagstanza{\tiny\textenglish{....2.111}}यस्योभयान्तव्यवधिसत्तासम्बन्धवाचिनी ।&अनित्यता-श्रुतिस्तेन तावन्ताविति कौ स्मृतौ ॥ १११ ॥\&[\smallbreak]


	
	  \endgroup
	

	  \pstart {\color{DodgerBlue3}“यस्य”} नै या यि का देरु{\color{DodgerBlue3}“भय”}स्य प्राक् पश्चादभा\edlabel{pvv.150-3}\footnote{\label{pvv.150-3}  ३ आकाशञ्चासदि (ति) विशिनष्टि द्विमध्यस्थेति ।}वस्या{\color{DodgerBlue3}“न्त”}रस्य यो {\color{DodgerBlue3}“व्यवधा”}यकः सत्ता{\color{DodgerBlue3}“सम्बन्ध”}स्तद्वाचिन्यनित्यता {\color{DodgerBlue3}“श्रुति”}रिष्टा {\color{DodgerBlue3}“तेन”} वादिना {\color{DodgerBlue3}“ताव”}न्ताविति कौ\edlabel{pvv.150-4}\footnote{\label{pvv.150-4}  ४ पृच्छत्ययुक्तत्वात् ।} {\color{DodgerBlue3}“स्मृतौ”} । (१११)
	\pend
      \label{div_pvv.2.112}\edlabel{div_pvv.2.112}
	  
	% new div opening: depth here is 2
	
	  \bigskip
	  \begingroup
	  \large
	
	    
	    \stanza[\smallbreak]
	\label{pv.2.112}\edlabel{pv.2.112}\flagstanza{\tiny\textenglish{....2.112}}प्राक् पश्चादप्यभावश्चेत् स एवानित्यता न किम् ।&षष्ठ्याद्ययोगादिति चेद् अन्तयोः स कथं भवेत् ॥ ११२ ॥\&[\smallbreak]


	
	  \endgroup
	

	  \pstart {\color{DodgerBlue3}“प्राग्”}भावः {\color{DodgerBlue3}“पश्चादभावोपि चेत् स एव”} प्रागभावः पश्चादभावोऽ{\color{DodgerBlue3}“नित्यता कि-”} न्नेष्यते । सत्ताविशेषणत्वेनापि तदभ्युपगमस्येष्टत्वात् ॥ पटस्य घटे चा\edlabel{pvv.150-5}\footnote{\label{pvv.150-5}  ५ भावाभावयोर्व्विरोधान्न सम्बन्धषष्ट्यादि ॥}भाव इति \leavevmode\marginnote{\textenglish{29a/MA}} {\color{DodgerBlue3}“षष्ट्या”}दिविभक्त्ययोगात् प्रध्वंसा\edlabel{pvv.150-6}\footnote{\label{pvv.150-6}  ६ कार्यस्य सत्तया सहाशेषः ।}भावो नानित्यता । न हि भावाभावयोः संयो\leavevmode\marginnote{\textenglish{151/s}} गादिः कश्चित्सम्बन्धोस्ति य उच्येत षष्ट्यादिभिरिति {\color{DodgerBlue3}“चेत् एवन्तर्ह्यन्तयोः”} स षष्ट्यादियोगः {\color{DodgerBlue3}“कथं भवेत्”} । अभावयोर्व्यवधिभूता सत्तेत्यादि । (११२)
	\pend
      \label{div_pvv.2.113}\edlabel{div_pvv.2.113}
	  
	% new div opening: depth here is 2
	

	  \pstart किञ्च (।)
	\pend
      
	  \bigskip
	  \begingroup
	  \large
	
	    
	    \stanza[\smallbreak]
	\label{pv.2.113a}\edlabel{pv.2.113a}\flagstanza{\tiny\textenglish{...2.113a}}सत्तासम्बन्धयोर्ध्रौव्यादन्ताभ्यां न विशेषणम् ।\&[\smallbreak]


	
	  \endgroup
	

	  \pstart {\color{DodgerBlue3}“सत्ता तत्सम्बन्धे”} परमते {\color{DodgerBlue3}“ध्रौव्यादन्ताभ्यां”} प्राक्प्रध्वंसाभावाभ्यां {\color{DodgerBlue3}“न विशेषणं”} स्यादन्तद्वयविशिष्टा सत्ता तत्सम्बन्धो वेति (।) न हि निंत्यस्य सर्वकालव्यापिनोऽन्तसम्भवः ।\edlabel{pvv.151-1}\footnote{\label{pvv.151-1}  १ वक्तव्यः ।}\edlabel{pvv.151-1a}\footnote{\label{pvv.151-1a}  1a भावस्य परं ।\begin{english}\par
Placement of note uncertain; marked with a question mark in the edition (see encoding description for details).\end{english}}
	\pend
      
	  \bigskip
	  \begingroup
	  \large
	
	    
	    \stanza[\smallbreak]
	\label{pv.2.113b}\edlabel{pv.2.113b}\flagstanza{\tiny\textenglish{...2.113b}}अविशेषणमेव स्यादन्तौ चेत् कार्यकारणे ॥ ११३ ॥\&[\smallbreak]


	
	  \endgroup
	

	  \pstart {\color{DodgerBlue3}“कार्यकारणे अन्ता”}वभिमते इति {\color{DodgerBlue3}“चेत्”} तदा ताभ्यां सत्तास{\color{DodgerBlue3}“म्बन्धयोरविशेषणमेव स्यात्”} । (११३)
	\pend
      \label{div_pvv.2.114}\edlabel{div_pvv.2.114}
	  
	% new div opening: depth here is 2
	
	  \bigskip
	  \begingroup
	  \large
	
	    
	    \stanza[\smallbreak]
	\label{pv.2.114}\edlabel{pv.2.114}\flagstanza{\tiny\textenglish{....2.114}}असम्बन्धान्न भावस्य प्रागभावं स वांछति ॥&तदुपाधिसमाख्याने तेप्यस्य च न सिध्यतः ॥ ११४ ॥\&[\smallbreak]


	
	  \endgroup
	

	  \pstart यस्मात् स नै या यि का दिर्भावस्या{\color{DodgerBlue3}“सम्बन्धात् प्राग”}भावं सम्बन्धिनं {\color{DodgerBlue3}“न वाञ्छति”} । तथा च {\color{DodgerBlue3}“तदुपाधिसमाख्याने”} प्रागभावविशेषणं कार्यकारणमिति समाख्यानं व्यपदेशो ययोस्ते तथा ते कार्यकारणे{\color{DodgerBlue3}“प्यस्य न सिध्यतः”} । यस्य हि प्रागभावः स भवन् कार्यं स्यात् । न च तत्सम्बन्धो भावस्येति कार्याभावः । कार्याभावाच्च कारणाभावः । तदुत्पादकस्य कारणत्वात् । (११४)
	\pend
      \label{div_pvv.2.115}\edlabel{div_pvv.2.115}
	  
	% new div opening: depth here is 2
	

	  \pstart किञ्च (।)
	\pend
      
	  \bigskip
	  \begingroup
	  \large
	
	    
	    \stanza[\smallbreak]
	\label{pv.2.115}\edlabel{pv.2.115}\flagstanza{\tiny\textenglish{....2.115}}सत्ता स्वकारणश्लेषकरणात् कारणं किल ।&सा सत्ता स च सम्बन्धो नित्यौ कार्यमथेह किम् ॥ ११५ ॥\&[\smallbreak]


	
	  \endgroup
	

	  \pstart जन्म सत्ताश्लेषः {\color{DodgerBlue3}“सत्ता”}समवायः स्वकारणेन समवायिना सह नित्यः । तयोः {\color{DodgerBlue3}“करणात् किल”} त्वयेष्टं {\color{DodgerBlue3}“कारणं । सा च सत्ता स च सम्बन्धो नित्यौ”} द्वावपि {\color{DodgerBlue3}“कार्यम\edlabel{pvv.151-2}\footnote{\label{pvv.151-2}  २ अतो यस्माद् भावः प्रागभावरहितस्ततो न क्रियते ।}थेह”} द्वयोर्मध्ये {\color{DodgerBlue3}“किं”} युक्तं । (११५)
	\pend
      \label{div_pvv.2.116}\edlabel{div_pvv.2.116}
	  
	% new div opening: depth here is 2
	

	  \pstart अपि च(।)
	\pend
      \leavevmode\marginnote{\textenglish{152/s}}
	  \bigskip
	  \begingroup
	  \large
	
	    
	    \stanza[\smallbreak]
	\label{pv.2.116}\edlabel{pv.2.116}\flagstanza{\tiny\textenglish{....2.116}}यस्याभावः क्रियेतासौ न भावः प्रागभाववान् ।&सम्बन्धानभ्युपगमान्नित्यं विश्वमिदं ततः ॥ ११६ ॥\&[\smallbreak]


	
	  \endgroup
	

	  \pstart {\color{DodgerBlue3}“यस्याभाव\edlabel{pvv.152-1}\footnote{\label{pvv.152-1}  १ प्रध्वंसाभावस्य मृतकरणप्रसङ्गः ।}स्यो”}त्पत्तेः प्रागभावः कारणैरसौ {\color{DodgerBlue3}“क्रियते । न\edlabel{pvv.152-2}\footnote{\label{pvv.152-2}  २ प्रागभावश्चेतन्न असम्बन्धात् ।}”} च कश्चिद् {\color{DodgerBlue3}“भावः प्राग्भाववान् सम्बन्ध”}वानन{\color{DodgerBlue3}“भ्युपगमात् । ततः”} कार्याभावा{\color{DodgerBlue3}“न्नित्यं विश्वमिदं”} प्राप्तं । (११६)
	\pend
      \label{div_pvv.2.117}\edlabel{div_pvv.2.117}
	  
	% new div opening: depth here is 2
	

	  \pstart कथन्तर्ह्यभावेन सम्बन्धः । न कथञ्चित् । किन्तु बुद्धिपरिकल्पित एवासावित्याह (।)
	\pend
      
	  \bigskip
	  \begingroup
	  \large
	
	    
	    \stanza[\smallbreak]
	\label{pv.2.117}\edlabel{pv.2.117}\flagstanza{\tiny\textenglish{....2.117}}तस्मादनर्थास्कन्दिन्योऽभिन्नार्थाभिमतेष्वपि ।&शब्देषु वाच्यभेदिन्यो व्यतिरेकास्पदं धियः ॥ ११७ ॥\&[\smallbreak]


	
	  \endgroup
	

	  \pstart यस्माद्वास्तवसम्बन्धाभ्युपगमे दोषस्तस्माद् भावस्य प्रागभाव इत्य{\color{DodgerBlue3}“भिन्नार्थ”}त्वेनाभिमतेष्वपि शब्दाद् भिन्नार्थाभिमतेषु च बीजस्याङ्कुर इत्यादिषु {\color{DodgerBlue3}“शब्देषु धियोऽनर्थास्कन्दिन्यः”} कल्पितसम्बन्धविषया {\color{DodgerBlue3}“वाच्यभेदिन्यः”} । संकेतानुरोधादुपकल्पितसम्बन्धिसम्बन्धलक्षणवाच्यभेदवत्यो {\color{DodgerBlue3}“व्यति”}रेकस्य सम्बन्धिसम्बन्धस्या{\color{DodgerBlue3}“स्पदं”} निमित्तं\edlabel{pvv.152-3}\footnote{\label{pvv.152-3}  ३ धिय एव ।} भवन्ति । अलं वास्तवसम्बन्धानुबन्धेन दोषाश्रयेण । (११७)
	\pend
      \label{div_pvv.2.118}\edlabel{div_pvv.2.118}
	  
	% new div opening: depth here is 2
	

	  \pstart तदेवं मेयबहुत्वाद् (व) हुतापि चेत्यत्र सानुषङ्गं प्रतिविहितं । अनेकस्य वृत्तेरेकत्र वा यथा विशेषदृष्टेनेत्यत्राह (।)
	\pend
      
	  \bigskip
	  \begingroup
	  \large
	
	    
	    \stanza[\smallbreak]
	\label{pv.2.118a}\edlabel{pv.2.118a}\flagstanza{\tiny\textenglish{...2.118a}}विशेषप्रत्यभिज्ञानं न प्रतिक्षणभेदतः ।\&[\smallbreak]


	
	  \endgroup
	

	  \pstart अग्निं दृष्ट्वा क्रमेण धूमाल्लिङ्गात् तस्यैव {\color{DodgerBlue3}“विशेष”}स्य स एवायं वह्निरिति {\color{DodgerBlue3}“यत्प्रत्यभिज्ञानं न”} तत्प्रमा\edlabel{pvv.152-4}\footnote{\label{pvv.152-4}  ४ विशिष्टत्वाभावात् पूर्व्वनाशापरोत्पत्तेः स एवायमिति मिथ्याज्ञानं । अवस्था ।}णं । {\color{DodgerBlue3}“प्रतिक्षणं”} भावस्य {\color{DodgerBlue3}“भेदतः”} एकार्थसाध्यार्थक्रियायाः सम्वादाभावात् ।
	\pend
      

	  \pstart स्यादेतन्न यथा दृष्ट एव विशेषो गृह्यते किन्तु तत्सामान्यमित्याह (।)
	\pend
      
	  \bigskip
	  \begingroup
	  \large
	
	    
	    \stanza[\smallbreak]
	\label{pv.2.118b}\edlabel{pv.2.118b}\flagstanza{\tiny\textenglish{...2.118b}}न वा विशेषविषयं दृष्टसाम्येन तद्ग्रहात् ॥ ११८ ॥\&[\smallbreak]


	
	  \endgroup
	

	  \pstart {\color{DodgerBlue3}“न वा विशेषविषयं”} तद्विशेष{\color{DodgerBlue3}“दृष्ट”}मनुमानं वक्तव्यं प्राग्दृष्टस्य विशेषस्य {\color{DodgerBlue3}“साम्ये”}न तस्योत्तरस्य {\color{DodgerBlue3}“ग्रहणात्”} । (११८)
	\pend
      \label{div_pvv.2.119}\edlabel{div_pvv.2.119}
	  
	% new div opening: depth here is 2
	\leavevmode\marginnote{\textenglish{153/s}}

	  \pstart स्यादेतदत्र (।) यत्र दृष्टान्तदार्ष्टान्तिकयोर्भेदस्तत्र सामान्यतो दृष्टमनुमानं इह तु ।
	\pend
      
	  \bigskip
	  \begingroup
	  \large
	
	    
	    \stanza[\smallbreak]
	\label{pv.2.119}\edlabel{pv.2.119}\flagstanza{\tiny\textenglish{....2.119}}निदर्शनं तदेवेति सामान्याग्रहणं यदि ।&निदर्शनत्वात् सिद्धस्य प्रमाणेनास्य किं पुनः ॥ ११९ ॥\&[\smallbreak]


	
	  \endgroup
	

	  \pstart {\color{DodgerBlue3}“तदेव”} दार्ष्टान्तिकं {\color{DodgerBlue3}“निदर्शनमिति सामान्याग्रहणं”} यद्युच्यते तदा {\color{DodgerBlue3}“निदर्शनत्वात्सिद्धस्य”} निश्चितस्या{\color{DodgerBlue3}“स्य”} दार्ष्टान्तिकस्य {\color{DodgerBlue3}“प्रमाणेन किं”} कर्तव्यं । (११९)
	\pend
      \label{div_pvv.2.120_2.121}\edlabel{div_pvv.2.120_2.121}
	  
	% new div opening: depth here is 2
	
	  \bigskip
	  \begingroup
	  \large
	
	    
	    \stanza[\smallbreak]
	\label{pv.2.120a}\edlabel{pv.2.120a}\flagstanza{\tiny\textenglish{...2.120a}}विस्मृतत्वाददोषश्चेत् तत एवानिदर्शनम् ॥&दृष्टे तद्भावसिद्धिश्चेत् प्रमाणाद्;\&[\smallbreak]


	
	  \endgroup
	

	  \pstart नाप्रसिद्धो दृष्टान्तः स चेत् सिद्धः किमनुमानेन गृहीतस्यापि {\color{DodgerBlue3}“विस्मृतत्वात्”} । पुनरनुमानप्रतीताव{\color{DodgerBlue3}“दोषश्चेत् । ततो”} विस्मृतत्वा{\color{DodgerBlue3}“देवानिदर्शनं”} । न हि गृहीतविस्मृतस्य दृष्टान्तता । पूर्व्वप्रत्ययेन {\color{DodgerBlue3}“दृष्टे”}ऽर्थ {\color{DodgerBlue3}“प्रमाणाद्वि”}शेषदृष्टानुमानाद्य एव प्राग्दृष्टः स एवायमिति {\color{DodgerBlue3}“तद्भाव”}स्य पूर्व्वस्य {\color{DodgerBlue3}“सिद्धिश्चेदि”}ष्यते (।)
	\pend
      

	  \pstart नन्वयं तद्भावः किमन्यवस्तुनि साध्यते उत तत्रैव ।
	\pend
      
	  \bigskip
	  \begingroup
	  \large
	
	    
	    \stanza[\smallbreak]
	\label{pv.2.120b}\edlabel{pv.2.120b}\flagstanza{\tiny\textenglish{...2.120b}}अन्यवस्तुनि ॥ १२० ॥\&[\smallbreak]


	
	  \endgroup
	
	  \bigskip
	  \begingroup
	  \large
	
	    
	    \stanza[\smallbreak]
	\label{pv.2.121}\edlabel{pv.2.121}\flagstanza{\tiny\textenglish{....2.121}}तत्त्वारोपे विपर्यासस्तत्सिद्धरेप्रमाणता ।&प्रत्यक्षेतरयोरैक्यादेकसिद्धिर्द्वयोरपि ॥ १२१ ॥\&[\smallbreak]


	
	  \endgroup
	

	  \pstart तत्रान्यवस्तुनि (१२०) वर्तमाने {\color{DodgerBlue3}“तत्त्व”}स्यातीतवस्त्वात्मक{\color{DodgerBlue3}“स्यारोपे”} स्वीक्रियमाणे {\color{DodgerBlue3}“विपर्यासो”}ऽयथार्थत्वं स्यात् । न ह्यन्यस्यान्यात्मत्वमस्ति तदशक्यप्रापण\leavevmode\marginnote{\textenglish{29b/MA}}मुपदर्शयदप्रमाणं स्यात् । अथ तत्रैव तद्भावसिद्धिरिति द्वितीयः पक्षः । तदा {\color{DodgerBlue3}“तत्सिद्धे”}रेकसिद्धेर्व्विशेषदृष्टस्यानुमानस्या{\color{DodgerBlue3}“प्रमाणता”} गृहीतग्राहित्वात् । न ह्येकस्य निर्भागस्य किञ्चिदगृहीतं नाम । तथा हि {\color{DodgerBlue3}“प्रत्यक्षेतरयो”}रध्यक्षानुमानविषययौरै{\color{DodgerBlue3}“क्यात् द्वयोरपि”} प्रत्यक्षेऽनुमाने च {\color{DodgerBlue3}“एकस्यार्थस्य सिद्धिः”} । (१२१)
	\pend
      \label{div_pvv.2.122}\edlabel{div_pvv.2.122}
	  
	% new div opening: depth here is 2
	

	  \pstart ततश्च दृष्टान्तग्राहिणैवाभ्रष्टस्मृतिसंस्कारेण प्रत्यक्षेण सिद्धत्वात् विफलमनुमानं । प्रत्यक्षसंस्कारभ्रंशे तु नादृष्टान्तमनुमानमस्ति (।) तस्माद् (।)
	\pend
      
	  \bigskip
	  \begingroup
	  \large
	
	    
	    \stanza[\smallbreak]
	\label{pv.2.122}\edlabel{pv.2.122}\flagstanza{\tiny\textenglish{....2.122}}सन्धीयमानं चान्येन व्यवसायं स्मृतिं विदुः ।&तल्लिङ्गापेक्षणान्नो चेत् स्मृतिर्न व्यभिचारतः ॥ १२२ ॥\&[\smallbreak]


	
	  \endgroup
	

	  \pstart अन्येनातीतदर्शनेन एकविषयतया {\color{DodgerBlue3}“सन्धीय\edlabel{pvv.153-1}\footnote{\label{pvv.153-1}  १ अभ्रष्टदर्शनसंस्कारं ।}मानं”} घट्यमानं परं {\color{DodgerBlue3}“व्यवसायं”}\edlabel{pvv.153-2}\footnote{\label{pvv.153-2}  २ निश्चयं ।}\leavevmode\marginnote{\textenglish{154/s}} {\color{DodgerBlue3}“स्मृतिं विदु”}र्व्विद्वांसः । गृहीतार्थविकल्पेन स्मृतित्वं तच्चेहाविकलं । तस्य प्रतिपत्तव्यस्य चिह्नमव्यभिचारि {\color{DodgerBlue3}“लिङ्गं”}\edlabel{pvv.154-1}\footnote{\label{pvv.154-1}  १ स्मृतेर्धूमाद्यपेक्षा नान्त्यस्यास्ति ।}न्त{\color{DodgerBlue3}“दपेक्षणान्नोचे”}द्विशेषदृष्टमनुमानं {\color{DodgerBlue3}“स्मृतिः”} सा तु लिङ्गनिरपेक्ष्या । नैतद्युक्तं {\color{DodgerBlue3}“व्यभिचार\edlabel{pvv.154-2}\footnote{\label{pvv.154-2}  २ न स्मृतिरनिमित्तैव ।}तः”} ।\edlabel{pvv.154-3}\footnote{\label{pvv.154-3}  ३ विशेषदृष्टे हि विशेषेणान्वयाभावात् त्रैरूप्यं । तदेवं लिङ्गजं प्रत्यभिज्ञानं दूषितं प्रत्यक्षजं दूषयिष्यते ।} तथा हि यदि लिङ्गन्त्रिरूपन्तदा व्याप्तिग्रहणविषयत्वेनैव तत्सिद्धेर्व्यर्थमनुमानं । अथ न त्रिरूपन्तदा नाव्य\edlabel{pvv.154-4}\footnote{\label{pvv.154-4}  ४ देशादिकथयापि स्यात् प्रत्यभिज्ञानं न तदव्यभिचारि ।}भिचारनिश्चयः । तस्माद्विशेषदृष्टस्याप्रमाणत्वादेकत्रानेकवृत्तेरपि त्र्येकसंख्यापोहनाभावो निरस्तः । तस्मात्स्थितमेतन्मानं द्विविधं मेयद्वैविध्यादिति ॥ (१२२) ॥
	\pend
      
	  
	% new div opening: depth here is 1
	
\section[{५. प्रत्यक्षचिन्ता}]{५. प्रत्यक्षचिन्ता}

	  \begin{center}%% label @type='head'
	\textbf{(१) प्रत्यक्षलक्षणविप्रतिपत्तिनिरासः}
	\end{center}
	

	  \begin{center}%% label @type='head'
	\textbf{क. कल्पनापोढं प्रत्यक्षम्}
	\end{center}
	\label{div_pvv.2.123}\edlabel{div_pvv.2.123}
	  
	% new div opening: depth here is 2
	

	  \pstart इदानीमवसरप्राप्तां प्रत्यक्षस्य लक्षणविप्रतिपत्तिं निराकर्त्तुमाह ।
	\pend
      
	  \bigskip
	  \begingroup
	  \large
	
	    
	    \stanza[\smallbreak]
	\label{pv.2.123}\edlabel{pv.2.123}\flagstanza{\tiny\textenglish{....2.123}}प्रत्यक्षं कल्पनापोढं प्रत्यक्षेणैव सिध्यति ।&प्रत्यात्मवेद्यः सर्वेषां विकल्पो नामसंश्रयः ॥ १२३ ॥\&[\smallbreak]


	
	  \endgroup
	

	  \pstart यत्त{\color{DodgerBlue3}“त्प्रत्यक्ष”}मिति प्रसिद्धंतत् {\color{DodgerBlue3}“कल्पनाया अपोढं”} द्रष्टव्यं कल्पनार्थरहितमित्यर्थः । तच्चैतदीदृशं {\color{DodgerBlue3}“प्रत्यक्षेणैव”} स्वसम्वेदनेनैव {\color{DodgerBlue3}“सिध्यति”} । कल्पनारहितस्यार्थस्य रूपस्य सम्वेदनस्यापरोक्षत्वात् । यदि तु कल्पनास्वभावत्वमस्य स्यात्तथैव प्रकाशेत । विकल्पस्यापरोक्षत्वात् । तथा हि {\color{DodgerBlue3}“प्रत्यात्मवेद्यः सर्व्वेषां”} प्राणिनां {\color{DodgerBlue3}“विकल्पो नामसंश्रयः”} शब्दसंसर्गवान् । स यदि स्यादुपलभ्य एव भवेत् (। १२३)
	\pend
      \label{div_pvv.2.124}\edlabel{div_pvv.2.124}
	  
	% new div opening: depth here is 2
	

	  \pstart तस्मात् (।)
	\pend
      
	  \bigskip
	  \begingroup
	  \large
	
	    
	    \stanza[\smallbreak]
	\label{pv.2.124}\edlabel{pv.2.124}\flagstanza{\tiny\textenglish{....2.124}}संहृत्य सर्वतश्चिन्तां स्तिमितेनान्तरात्मना ।&स्थितोपि चक्षुषा रूपमीक्षते साक्षजा मतिः ॥ १२४ ॥\&[\smallbreak]


	
	  \endgroup
	

	  \pstart {\color{DodgerBlue3}“संहृत्या”}कृष्य {\color{DodgerBlue3}“सर्व्वतो”} विकल्पनीया{\color{DodgerBlue3}“च्चिन्तां स्तिमितेन”} सर्व्वाविकल्पविगमात् अधिक्षिप्तेना{\color{DodgerBlue3}“न्तरात्मना”} चेतसा {\color{DodgerBlue3}“स्थितोऽपि”} पुरुष{\color{DodgerBlue3}“श्चक्षु”}र्व्विज्ञानेन {\color{DodgerBlue3}“रूपमीक्षते साक्ष”}जा निर्व्विकल्पा {\color{DodgerBlue3}“मतिः”} सर्व सम्विदितैव । (१२४)
	\pend
      \label{div_pvv.2.125}\edlabel{div_pvv.2.125}
	  
	% new div opening: depth here is 2
	\leavevmode\marginnote{\textenglish{155/s}}

	  \pstart सन्त्येवेन्द्रियधियः कल्पनास्तास्तु नोपलभ्यन्त इत्यप्यसत् । तथा हि ।
	\pend
      
	  \bigskip
	  \begingroup
	  \large
	
	    
	    \stanza[\smallbreak]
	\label{pv.2.125}\edlabel{pv.2.125}\flagstanza{\tiny\textenglish{....2.125}}पुनर्विकल्पयन् किञ्चिदासीन्मे कल्पनेदृशी ।&वेत्ति चेति न पूर्व्वोक्तावस्थायामिन्द्रियाद् गतौ ॥ १२५ ॥\&[\smallbreak]


	
	  \endgroup
	

	  \pstart विकल्पावस्थाया ऊर्ध्वं {\color{DodgerBlue3}“पुनर्व्विकल्पयन्”} पुमाना{\color{DodgerBlue3}“सीन्मे कल्पनेदृशीति”} वेति {\color{DodgerBlue3}“नेन्द्रियादु”}त्पन्नायां {\color{DodgerBlue3}“गतौ”} बुद्धौ संहृत्येत्यादिना {\color{DodgerBlue3}“पूर्व्वमुक्तावस्था”} यस्यास्तस्याः कल्पनां वेति । यदि सा तत्र स्यादतत्संस्कारस्य स्मृतिर्जायते । तस्मान्नास्तीति निश्चीयते (। १२५)
	\pend
      \label{div_pvv.2.126}\edlabel{div_pvv.2.126}
	  
	% new div opening: depth here is 2
	

	  \pstart किञ्च (।) वाच्यवाचकाकारसंसर्गवती प्रतीतिः कल्पना । न चेन्द्रियविषयेऽनन्वयात् संकेताऽसम्भवाच्च शब्दयोजनास्ति (।) तथाहि (।)
	\pend
      
	  \bigskip
	  \begingroup
	  \large
	
	    
	    \stanza[\smallbreak]
	\label{pv.2.126}\edlabel{pv.2.126}\flagstanza{\tiny\textenglish{....2.126}}एकत्र दृष्टो भेदो हि क्वचिन्नान्यत्र दृश्यते ॥&न तस्माद् भिन्नमस्त्यन्यत् सामान्यं बुद्धयभेदतः॥ १२६ ॥\&[\smallbreak]


	
	  \endgroup
	

	  \pstart {\color{DodgerBlue3}“एकत्र”} दे\edlabel{pvv.155-1}\footnote{\label{pvv.155-1}  १ शब्दादेकत्र नियुक्तात्सर्व्वत्रार्थप्रतीतिः स्यात् ।}शादौ {\color{DodgerBlue3}“न दृश्यते”} न चाननुयायिनि शब्दसंकेतः । सामान्यमनुयायीति चेत् । {\color{DodgerBlue3}“तस्मा”}द् भेदादन्यद् {\color{DodgerBlue3}“भिन्नं सामान्यं नास्ति बुद्धेरभेदतः”} । (१२६)
	\pend
      \label{div_pvv.2.127}\edlabel{div_pvv.2.127}
	  
	% new div opening: depth here is 2
	

	  \pstart यदि हि सामान्यं सामान्यं स्यात् द्व्याकारा बुद्धिर्भवेत् । विशेषमात्राकारैव तु प्रत्यक्षबुद्धिरुपलभ्यते ।
	\pend
      
	  \bigskip
	  \begingroup
	  \large
	
	    
	    \stanza[\smallbreak]
	\label{pv.2.127}\edlabel{pv.2.127}\flagstanza{\tiny\textenglish{....2.127}}तस्माद् विशेषविषया सर्वैवेन्द्रियजा मतिः ॥&न विशेषेषु शब्दानां प्रवृत्तावस्ति सम्भवः ॥ १२७ ॥\&[\smallbreak]


	
	  \endgroup
	

	  \pstart {\color{DodgerBlue3}“तस्मात्सर्व्वैवेन्द्रियजा मतिर्व्विशेष”}मात्र{\color{DodgerBlue3}“विषया”}ऽन्यस्यानुपलब्धेः । {\color{DodgerBlue3}“न च विशेषेषु शब्दानां प्रवृत्तौ संभवोस्ति”} (१२७)
	\pend
      \label{div_pvv.2.128}\edlabel{div_pvv.2.128}
	  
	% new div opening: depth here is 2
	
	  \bigskip
	  \begingroup
	  \large
	
	    
	    \stanza[\smallbreak]
	\label{pv.2.128}\edlabel{pv.2.128}\flagstanza{\tiny\textenglish{....2.128}}अनन्वयाद् विशेषाणां सङ्केतस्याप्रवृत्तितः ॥&विषयो यश्च शब्दानां संयोज्येत स एव तैः ॥ १२८ ॥\&[\smallbreak]


	
	  \endgroup
	

	  \pstart {\color{DodgerBlue3}“विशेषाणामनन्वयात्”} तत्र {\color{DodgerBlue3}“संकेतस्याप्रवृत्तितः”} । उत्तरकालं शब्दार्थप्रतिपत्त्यर्थं संकेतक्रिया । न च विशेषाः कालान्तरमनुवर्तन्ते । तस्माद्य एव {\color{DodgerBlue3}“शब्दानां विषयो”}\leavevmode\marginnote{\textenglish{30a/MA}} व्यवच्छेदः {\color{DodgerBlue3}“स एव तैः संयोज्येत”} । न स्वलक्षणं ॥ (१२८)
	\pend
      \label{div_pvv.2.129}\edlabel{div_pvv.2.129}
	  
	% new div opening: depth here is 2
	
	  \bigskip
	  \begingroup
	  \large
	
	    
	    \stanza[\smallbreak]
	\label{pv.2.129}\edlabel{pv.2.129}\flagstanza{\tiny\textenglish{....2.129}}अस्येदमिति सम्बन्धे यावर्थौ प्रतिभासिनौ&तयोरेव हि सम्बन्धो न तदेन्द्रियगोचरः ॥ १२९ ॥\&[\smallbreak]


	
	  \endgroup
	

	  \pstart तस्मादस्या{\color{DodgerBlue3}“र्थस्येद”}म्वाचक{\color{DodgerBlue3}“मिति सम्बन्धे”} वाच्यवाचकभावलक्षणे {\color{DodgerBlue3}“यावर्थौ प्रतिभासिनौ तयोरेव हि सम्बन्धो”} वक्तव्यः । यदा चार्थन्दृष्ट्वा सकेतं तत्र प्रवर्तयति । {\color{DodgerBlue3}“तदेन्द्रियगोचरोऽर्थो”} नास्ति (। १२९)
	\pend
      \label{div_pvv.2.130}\edlabel{div_pvv.2.130}
	  
	% new div opening: depth here is 2
	\leavevmode\marginnote{\textenglish{156/s}}
	  \bigskip
	  \begingroup
	  \large
	
	    
	    \stanza[\smallbreak]
	\label{pv.2.130}\edlabel{pv.2.130}\flagstanza{\tiny\textenglish{....2.130}}विशदप्रतिभासस्य तदार्थस्याविभावनात् ।&विज्ञानाभासभेदश्च पदार्थानां विशेषकः ॥ १३० ॥\&[\smallbreak]


	
	  \endgroup
	

	  \pstart संहृतेन्द्रियव्यापारस्य तदा संकेतसंकल्पकाले {\color{DodgerBlue3}“विष (श?)दप्रतिभासस्यार्थस्याविभावनात्”} । यदि तत्रार्थः प्रतिभाति तदेन्द्रियज्ञानवत् स्फुटः प्रतीयते । न च प्रतिभासभेदेपि शब्देन्द्रियज्ञानयोरेकविषयत्वं यस्मा{\color{DodgerBlue3}“द्विज्ञान”}स्या{\color{DodgerBlue3}“भासभेद”} आकारभेदः {\color{DodgerBlue3}“पदार्थानां”} ग्राह्यानां (? णां) {\color{DodgerBlue3}“विशेषको”} भेदकः । यदि तु प्रतिभासभेदेप्यर्थाभेदस्तदा विश्वमेकं द्रव्यं स्यात् ॥(१३०)
	\pend
      \label{div_pvv.2.131}\edlabel{div_pvv.2.131}
	  
	% new div opening: depth here is 2
	

	  \pstart स्यादेतत् । यदा स्वलक्षणमुपदर्श्य शब्दो निवेश्यते तदा स्वलक्षणमेव वाच्यवाचकमित्याह (।)
	\pend
      
	  \bigskip
	  \begingroup
	  \large
	
	    
	    \stanza[\smallbreak]
	\label{pv.2.131}\edlabel{pv.2.131}\flagstanza{\tiny\textenglish{....2.131}}चक्षुषाऽर्थावभासेऽपि यं परोऽस्येति शंसति ।&स एव योज्यते शब्दैर्न खल्विन्द्रियगोचरः ॥ १३१ ॥\&[\smallbreak]


	
	  \endgroup
	

	  \pstart {\color{DodgerBlue3}“चक्षुषार्थवभासेपि”} जाते तथा श्रोत्राच्छाब्दावभासेपि यमर्थमन्यव्यवच्छेदं बुद्धिपरिवर्त्तिनं परः प्रतिपादकोऽस्यार्थस्यायं वाचक इति {\color{DodgerBlue3}“शंसति”} कथयति {\color{DodgerBlue3}“स एवा”}न्यव्यवच्छेदः {\color{DodgerBlue3}“शब्दैर्योज्यते न खल्विन्द्रियगोचरः”} स्वलक्षणमनन्वयात्तस्य । तद्दर्शनान्तरं संकेतसंकल्पे च विनाशाच्च । (१३१)
	\pend
      \label{div_pvv.2.132}\edlabel{div_pvv.2.132}
	  
	% new div opening: depth here is 2
	
	  \bigskip
	  \begingroup
	  \large
	
	    
	    \stanza[\smallbreak]
	\label{pv.2.132a}\edlabel{pv.2.132a}\flagstanza{\tiny\textenglish{...2.132a}}अव्यापृतेन्द्रियस्यान्यवाङ्गमात्रेणाविभावनात् ।\&[\smallbreak]


	
	  \endgroup
	

	  \pstart {\color{DodgerBlue3}“तथाऽव्यापृतेन्द्रियस्यान्यवाङ‏्मात्रेण”} स्वलक्षणा{\color{DodgerBlue3}“विभावनात्”} प्रत्यक्ष इव ।
	\pend
      

	  \pstart स्यादेतत् (।) संकेताविषयत्वेपि शब्दसंसृष्टमेव स्वलक्षणमध्यक्षं प्रकृत्या प्रत्येष्यतीत्याह (।)
	\pend
      
	  \bigskip
	  \begingroup
	  \large
	
	    
	    \stanza[\smallbreak]
	\label{pv.2.132b}\edlabel{pv.2.132b}\flagstanza{\tiny\textenglish{...2.132b}}न चानुदितसंबन्धः स्वयं ज्ञानप्रसङ्गतः ॥ १३२ ॥\&[\smallbreak]


	
	  \endgroup
	

	  \pstart {\color{DodgerBlue3}“न चानुदितसम्बन्धो”} वाच्यवाचकभावो यस्य स शब्दः प्रत्यायको दृष्ट इति शेषः (।) तथाभ्युपगमे तु {\color{DodgerBlue3}“स्वयं”} संकेतमनपेक्ष्यैव श्रुताच्छब्दादर्थस्य {\color{DodgerBlue3}“ज्ञानप्रसङ्गतः”} । (१३२)
	\pend
      \label{div_pvv.2.133}\edlabel{div_pvv.2.133}
	  
	% new div opening: depth here is 2
	

	  \pstart नन्विन्द्रियव्यापारसमकालमहिरहिरिति धारावाहिसविकल्पकमध्यक्षं प्रवर्तते । यदि तु तत्र विकल्पकमविकल्पकञ्च द्वयमिष्यते तदा विकल्पेन निर्व्विकल्पस्य व्यवधानाद् दर्शनं विच्छिन्नं स्यात् । न चैतदस्ति । न युगपज्ज्ञानसम्भवः (।) अत्राह (।)
	\pend
      
	  \bigskip
	  \begingroup
	  \large
	
	    
	    \stanza[\smallbreak]
	\label{pv.2.133}\edlabel{pv.2.133}\flagstanza{\tiny\textenglish{....2.133}}मनसो युगपद्वृत्तेः सविकल्पाविकल्पयोः&विमूढो लघुवृत्तेर्व्वा तयोरैक्यं व्यवस्यति ॥ १३३ ॥\&[\smallbreak]


	
	  \endgroup
	\leavevmode\marginnote{\textenglish{157/s}}

	  \pstart {\color{DodgerBlue3}“मनसोः सविकल्पाविकल्पयो”}रेकस्मात्समनन्तरा{\color{DodgerBlue3}“द्युगपद् वृत्तेः”} कारणात्तयोरैक्यं {\color{DodgerBlue3}“विमूढः”} प्रतिपत्ता व्यवस्यति । निर्व्विकल्पकं हि स्वलक्षणविषयं विकल्पश्च वस्तुतोऽतद्विषयत्वेप्यवसायानुरोधात्तद्विषयः\edlabel{pvv.157-1}\footnote{\label{pvv.157-1}  १ अस्पष्टः ।} । सहोत्पत्तिश्चानुभवसिद्धत्वात् दुरपह्नवा । ततः सहोत्पन्नयोरेकविषययोरैक्यभ्रम एषः । पराभिमतायां युगपदनुत्पत्तावपि सविकल्पाविकल्पयोर्लघुवृत्तेः शीघ्र{\color{DodgerBlue3}“वृत्ते”}र्व्वा कारणात् {\color{DodgerBlue3}“तयोर्मू”}ढमतिः प्रतिपत्ता {\color{DodgerBlue3}“ऐक्यं व्यवस्यति”} अलातभ्रान्तौ चक्रमिव । (१३३)
	\pend
      \label{div_pvv.2.134}\edlabel{div_pvv.2.134}
	  
	% new div opening: depth here is 2
	

	  \pstart यदि दर्शनमविकल्पं तदनन्तरन्तु विकल्पः पुनस्तदनन्तरं दर्शनं तदा (।)
	\pend
      
	  \bigskip
	  \begingroup
	  \large
	
	    
	    \stanza[\smallbreak]
	\label{pv.2.134}\edlabel{pv.2.134}\flagstanza{\tiny\textenglish{....2.134}}विकल्पव्यवधानेन विच्छिन्नं दर्शनम्भवेत् ।&इति चेद् भिन्नजातीयविकल्पेन्यस्य वा कथम् ॥ १३४ ॥\&[\smallbreak]


	
	  \endgroup
	

	  \pstart {\color{DodgerBlue3}“विकल्पेन”} व्यवधाना{\color{DodgerBlue3}“द्विच्छिन्नं दर्शनं भवेत्”} । न धारावाहीति चेत् । {\color{DodgerBlue3}“नन्वस्य”} परस्यापि गां पश्यतो निर्व्विकल्पेन प्रत्यक्षेण भिन्नजातीयस्याश्वादेर्व्विकल्पे जायमाने दर्शनं कथमविच्छिन्नं । न ह्यश्ववाचकशब्देन संयोज्य गौर्गृह्यते । येनैकमेव सविकल्पं तदध्यक्षं भवेत् । एकत्वे वाश्वप्रतीतिर्न स्यात् । गोविषयत्वात्तस्य । (१३४)
	\pend
      \label{div_pvv.2.135}\edlabel{div_pvv.2.135}
	  
	% new div opening: depth here is 2
	

	  \pstart स्यादेतत् ।
	\pend
      
	  \bigskip
	  \begingroup
	  \large
	
	    
	    \stanza[\smallbreak]
	\label{pv.2.135}\edlabel{pv.2.135}\flagstanza{\tiny\textenglish{....2.135}}अलातदृष्टिवद् भावपक्षश्चेद् बलवान् मतः ।&अन्यत्रापि समानं तद् वर्णयोर्व्वा सकृच्छृ तिः ॥ १३५ ॥\&[\smallbreak]


	
	  \endgroup
	

	  \pstart अलातस्य भ्रम्यमानस्य नानादेशेषु लाघवाद् भावपक्षबलवत्वाच्च दर्शनप्रतिसन्धानेन चक्रदृ{\color{DodgerBlue3}“ष्टिवत्”} विजातीयव्यवकीर्यमाणस्य दर्शनस्यान्तरा अभावेपि\leavevmode\marginnote{\textenglish{30b/MA}} {\color{DodgerBlue3}“भावपक्षो बलवान् मत”} इति दर्शनाविच्छेदबुद्धिरति {\color{DodgerBlue3}“चेत् । अन्यत्राभावपक्षेपि”} तद्बलवत्त्वे लाघवसामर्थ्यात् {\color{DodgerBlue3}“समानं\edlabel{pvv.157-2}\footnote{\label{pvv.157-2}  २ दृष्टाविमृष्टादौ ।}”} । ततो दर्शनविच्छेदलाघवाद्विच्छेदधीरेवास्तु । सरो रस इत्यादौ {\color{DodgerBlue3}“वर्ण्णयो”}र्वा लाघवा{\color{DodgerBlue3}“त्स\edlabel{pvv.157-3}\footnote{\label{pvv.157-3}  ३ तथा क्रमाभेदात्सकृत्श्रुतिभेवो न स्यात् ।}कृत्श्रुतिः”} प्राप्ता ॥(१३५)
	\pend
      \label{div_pvv.2.136}\edlabel{div_pvv.2.136}
	  
	% new div opening: depth here is 2
	

	  \begin{center}%% label @type='head'
	\textbf{ख. परमतदूषणम्}
	\end{center}
	

	  \pstart किञ्च (।)
	\pend
      
	  \bigskip
	  \begingroup
	  \large
	
	    
	    \stanza[\smallbreak]
	\label{pv.2.136}\edlabel{pv.2.136}\flagstanza{\tiny\textenglish{....2.136}}सकृत् सङ्गतशब्दार्थेष्विन्द्रियेष्विह सत्स्वपि ।&पञ्चभिर्व्यवधानेपि भात्यव्यवहितेव या ॥ १३६ ॥\&[\smallbreak]


	
	  \endgroup
	\leavevmode\marginnote{\textenglish{158/s}}

	  \pstart {\color{DodgerBlue3}“सकृत्”} यु\edlabel{pvv.158-1}\footnote{\label{pvv.158-1}  १ “युगपद्विज्ञानानुत्पत्तिर्मनसो लिङ्गमि”\href{http://http://sarit.indology.info/?cref=ns.1.1.16}{ (न्यायसूत्रे १।१।१६)} ति दूषयन्नाह ।}गप{\color{DodgerBlue3}“त्संगताः”} स्वस्वगोचरीभूताः सर्व्वेऽ{\color{DodgerBlue3}“र्था”}\edlabel{pvv.158-2}\footnote{\label{pvv.158-2}  २ युगपद्विज्ञानानुत्पत्तिपक्षे ।} येषु ते{\color{DodgerBlue3}“ष्विन्द्रियेषु”} चक्षुरादिषु मनःपर्यन्तेषु {\color{DodgerBlue3}“सत्स्वपीह”} सङ्क्रान्तकान्तावदनप्रतिबिम्बस्य सहकारसुगन्धिनः शीतस्य भ्रमद्भ्रमरोपगीतस्य स्वाद्रुनो मधुनः सार्व्वगुणानुभवकाले प्रसरत्संकल्पजन्मनां यूनां या मतिः {\color{DodgerBlue3}“पञ्चभि”}रिन्द्रियबुद्धि{\color{DodgerBlue3}“भिर्व्यवधानेपि”} त्वत्पक्षे{\color{DodgerBlue3}“ऽव्यव”}हितेव समकालेव {\color{DodgerBlue3}“भाति”} । (१३६)
	\pend
      \label{div_pvv.2.137}\edlabel{div_pvv.2.137}
	  
	% new div opening: depth here is 2
	
	  \bigskip
	  \begingroup
	  \large
	
	    
	    \stanza[\smallbreak]
	\label{pv.2.137}\edlabel{pv.2.137}\flagstanza{\tiny\textenglish{....2.137}}सा मतिर्नामपर्यन्तक्षणिकज्ञानमिश्रणात् ।&विच्छिन्नाभेति तच्चित्रं तस्मात् सन्तु सकृद्धियः ॥ १३७ ॥\&[\smallbreak]


	
	  \endgroup
	

	  \pstart {\color{DodgerBlue3}“सेन्द्रियमतिर्नाम्नः”} शब्दस्य {\color{DodgerBlue3}“पर्यन्तो”} वर्ण्णस्तस्य {\color{DodgerBlue3}“क्षणिकं ज्ञानं”} तेन {\color{DodgerBlue3}“मिश्रणात्”} सरो रस इत्यादिष्वविच्छिन्ना प्राप्नोति । विजातीय\edlabel{pvv.158-3}\footnote{\label{pvv.158-3}  ३ एकलोलीभूतवस्तुग्रहणात्मिका स्यात् ।}विज्ञानान्तराव्यवधानात् । तथापि {\color{DodgerBlue3}“विच्छिन्नाभा”} क्रमवती । यत्तु {\color{DodgerBlue3}“चित्रमा”}श्चर्यं यदि लाघवकृतः सकृद्ग्रहाभिमानः तदा वर्ण्णज्ञाने स नितरां युक्तो विजातीयाव्यवधानात् । इन्द्रियज्ञानेषु तु न युक्तः पञ्चभिर्व्यवधानात् । तस्मा{\color{DodgerBlue3}“त्सकृद्धियः सन्तु”} यथोक्तं मनसो युगपद्वृत्तेरि (२।१३३) ति ॥(१३७)
	\pend
      \label{div_pvv.2.138}\edlabel{div_pvv.2.138}
	  
	% new div opening: depth here is 2
	

	  \pstart अन्य\edlabel{pvv.158-4}\footnote{\label{pvv.158-4}  ४ युगपज्ज्ञानोत्पत्त्यनिष्टौ ।}था (।)
	\pend
      
	  \bigskip
	  \begingroup
	  \large
	
	    
	    \stanza[\smallbreak]
	\label{pv.2.138}\edlabel{pv.2.138}\flagstanza{\tiny\textenglish{....2.138}}प्रतिभासाविशेषश्च सान्तरानन्तरे कथम् ।&शुद्धे मनोविकल्पे च न क्रमग्रहणम्भवेत् ॥ १३८ ॥\&[\smallbreak]


	
	  \endgroup
	

	  \pstart {\color{DodgerBlue3}“सान्तरे”} पञ्चभिरिन्द्रियज्ञानैर्व्यवहितत्वात् । {\color{DodgerBlue3}“अनन्तरे”} सरो रस इत्यादिके ज्ञाने विजातीयाव्यवधानात् {\color{DodgerBlue3}“प्रतिभास”}स्या{\color{DodgerBlue3}“विशेष”}श्चाप्रसक्तः (।) स चानुभवबाधितत्वात् {\color{DodgerBlue3}“कथम”}भ्युपगम इति शेषः । यदि लघुवृत्तित्वात् सकृद्ग्रहस्तदा {\color{DodgerBlue3}“शुद्धे”} विजातीयाव्यवकीर्ण्णे {\color{DodgerBlue3}“मनोविकल्पे च”} प्राबन्धिके {\color{DodgerBlue3}“क्रमग्रहणमनु”}भवसिद्धं न {\color{DodgerBlue3}“भवेत्”} ।(१३८)
	\pend
      \label{div_pvv.2.139}\edlabel{div_pvv.2.139}
	  
	% new div opening: depth here is 2
	
	  \bigskip
	  \begingroup
	  \large
	
	    
	    \stanza[\smallbreak]
	\label{pv.2.139a}\edlabel{pv.2.139a}\flagstanza{\tiny\textenglish{...2.139a}}योऽग्रहः सङ्गतेप्यर्थे क्वचिदासक्तचेतसः ।\&[\smallbreak]


	
	  \endgroup
	

	  \pstart ननु यदि युगपत् ज्ञानोत्पत्तिस्तदैकत्रासक्तं पुनरुत्पत्तिधर्मकं चेतो यस्य तस्या{\color{DodgerBlue3}“सक्तचेतसः क्वचिदर्थे”} स्वेन्द्रियेण {\color{DodgerBlue3}“सङ्गतेपि”} योऽग्रहो विज्ञानानुत्पत्तिः स कथं ।
	\pend
      

	  \pstart आह (।)
	\pend
      
	  \bigskip
	  \begingroup
	  \large
	
	    
	    \stanza[\smallbreak]
	\label{pv.2.139b}\edlabel{pv.2.139b}\flagstanza{\tiny\textenglish{...2.139b}}सक्त्‏या नोत्पत्तिवैगुण्याच्चोद्यं वै तद्द्वयोरपि ॥ १३९ ॥\&[\smallbreak]


	
	  \endgroup
	\leavevmode\marginnote{\textenglish{159/s}}

	  \pstart एकत्र {\color{DodgerBlue3}“सक्त्या”} विषयासञ्चारलक्षणयाऽन्यस्य भिन्नविषयस्य ज्ञानस्यो{\color{DodgerBlue3}“त्पत्तिवैगुण्यात्”} । अविगुणो हि समनन्तरप्रत्ययः स्वकार्यमारभते । न त्वासक्तिविगुणः । {\color{DodgerBlue3}“यच्चैतच्चोद्यं”} परिहृतमस्माभिस्तद् {\color{DodgerBlue3}“द्वयोरपि”} समानं युगपद्विज्ञानानुत्पत्तिवादिनोपि मते सर्व्वत्रेवैन्द्रियसंगमे समाने क्वचिदेव {\color{DodgerBlue3}“ज्ञानं क्वचिन्नेति”} कुतः। (१३९)
	\pend
      \label{div_pvv.2.140}\edlabel{div_pvv.2.140}
	  
	% new div opening: depth here is 2
	

	  \pstart तत्रासक्तिवैगुण्यमेव मनस उत्तरं तच्च समानमस्माकमलातदृष्टिवदिति दृष्टान्तस्यासिद्धिमाह (।)
	\pend
      
	  \bigskip
	  \begingroup
	  \large
	
	    
	    \stanza[\smallbreak]
	\label{pv.2.140}\edlabel{pv.2.140}\flagstanza{\tiny\textenglish{....2.140}}शीघ्रवृत्तेरलातादेरन्वयप्रतिधातिनी ।&चक्रभ्रान्तिं दृगाधत्ते न दृशां घटनेन सा ॥ १४० ॥\&[\smallbreak]


	
	  \endgroup
	

	  \pstart {\color{DodgerBlue3}“शीघ्रा”} प्र{\color{DodgerBlue3}“वृत्तिर्भ्र”}मणं यस्या{\color{DodgerBlue3}“लाता”}देस्त{\color{DodgerBlue3}“स्यान्वयेनानुगमेन प्रतिघात”} उपहतत्वं तद्वती दृग् दृष्टि{\color{DodgerBlue3}“श्चक्रा”}कारां {\color{DodgerBlue3}“भ्रान्ति”}मिन्द्रियजां {\color{DodgerBlue3}“धत्ते ।\edlabel{pvv.159-1}\footnote{\label{pvv.159-1}  १ स्वमतमाख्याय परमतं निषेधति ।}\edlabel{pvv.1591a}\footnote{\label{pvv.1591a}  1a तदालम्बनं ।\begin{english}\par
Placement of note uncertain; marked with a question mark in the edition (see encoding description for details).\end{english}} न दृशां”} भिन्नभिन्नदेशालातदर्शनानां {\color{DodgerBlue3}“घटनेन”} योजनया {\color{DodgerBlue3}“सा”} मानसी भ्रान्तिः स्फुटप्रतिभासत्वात् । {\color{DodgerBlue3}“मानसस्य”} न विपर्ययात् । तस्मादहिरहिरिति विकल्पसमकालमध्यक्षं वस्तु स्फुटमवैति न तु विकल्प इति स्थितं । (१४०)
	\pend
      \label{div_pvv.2.141}\edlabel{div_pvv.2.141}
	  
	% new div opening: depth here is 2
	
	  \bigskip
	  \begingroup
	  \large
	
	    
	    \stanza[\smallbreak]
	\label{pv.2.141}\edlabel{pv.2.141}\flagstanza{\tiny\textenglish{....2.141}}केचिदिन्द्रियजत्वादेर्बालधीवदकल्पनाम् ।&आहुर्बालाविकल्पे च हेतुं सङ्केतमन्दताम् ॥ १४१ ॥\&[\smallbreak]


	
	  \endgroup
	

	  \pstart {\color{DodgerBlue3}“केचि”}दाचार्यीयाः श ङ्क र स्वा मि प्रभृतयः {\color{DodgerBlue3}“इन्द्रियजत्वा”}दादिशब्दादमानसत्वानुभवाकारप्रवृत्तत्त्वादेर्हेतोः प्रत्यक्षबुद्धि{\color{DodgerBlue3}“मकल्पनां बालधीवदाहुः”} । दृष्टान्तसिद्धयर्थं । बालस्या{\color{DodgerBlue3}“विकल्पे”} विकल्पाभावे च {\color{DodgerBlue3}“संकेतमंदतां”} हेतुमाहुः । वाच्यवाचकयोजना हि विकल्पः । सा च संकेतपूर्व्विका तदभावाद् बालस्य कल्पनाभावः । (१४१)
	\pend
      \label{div_pvv.2.142}\edlabel{div_pvv.2.142}
	  
	% new div opening: depth here is 2
	
	  \bigskip
	  \begingroup
	  \large
	
	    
	    \stanza[\smallbreak]
	\label{pv.2.142}\edlabel{pv.2.142}\flagstanza{\tiny\textenglish{....2.142}}तेषां प्रत्यक्षमेव स्याद् बालानामविकल्पनात् ।&सङ्केतोपायविगमात् पश्चादपि भवेन्न सः ॥ १४२ ॥\&[\smallbreak]


	
	  \endgroup
	

	  \pstart {\color{DodgerBlue3}“तेषामेवं”} वादिनां मते {\color{DodgerBlue3}“बालानां प्रत्यक्षमेव”} ज्ञानं {\color{DodgerBlue3}“स्यान्न”} विचारकं (।) किं कारणमविकल्पनात् । भवतु को दोष इति चेत् । आह {\color{DodgerBlue3}“संकेतोपायस्य”} विचारस्य {\color{DodgerBlue3}“विग-\leavevmode\marginnote{\textenglish{31a/MA}} मात्”} बालानां {\color{DodgerBlue3}“पश्चादपि”} स सङ्केतो {\color{DodgerBlue3}“न भवेत्”} । तदभावाद्विकल्पाभावश्च । (१४२)
	\pend
      \label{div_pvv.2.143}\edlabel{div_pvv.2.143}
	  
	% new div opening: depth here is 2
	
	  \bigskip
	  \begingroup
	  \large
	
	    
	    \stanza[\smallbreak]
	\label{pv.2.143}\edlabel{pv.2.143}\flagstanza{\tiny\textenglish{....2.143}}मनो व्युत्पन्नसङ्केतमस्ति तेन स चेन्मतः ।&एवमिन्द्रियजेपि स्याद् शेषवच्चेदमीदृशम् ॥ १४३ ॥\&[\smallbreak]


	
	  \endgroup
	\leavevmode\marginnote{\textenglish{160/s}}

	  \pstart जन्मान्तरागतं {\color{DodgerBlue3}“व्युत्पन्नं-संकेतं मनोस्ति”} बालानां {\color{DodgerBlue3}“तेन”} संकेतः तेषां मतश्चेत् । {\color{DodgerBlue3}“एवं”} सती{\color{DodgerBlue3}“न्द्रियजेपि”} ज्ञाने {\color{DodgerBlue3}“स्यात्”} कल्पना तन्निवर्तकहेत्वभिधानात् ततो\edlabel{pvv.160-1}\footnote{\label{pvv.160-1}  १ असिद्धिर्दृष्टे वस्तुनि ।} दृष्टासिद्धिरेव । {\color{DodgerBlue3}“इद”}मिन्द्रियजत्वादि {\color{DodgerBlue3}“ईदृशं”} निषेध्येन सहासिद्ध\edlabel{pvv.160-2}\footnote{\label{pvv.160-2}  २ विपक्षे बाधनादर्शनात् ।}विरोधं । शेषवच्चोक्तं । (१४३)
	\pend
      \label{div_pvv.2.144}\edlabel{div_pvv.2.144}
	  
	% new div opening: depth here is 2
	

	  \pstart अथान्येन लिङ्गेनाव्यभिचारिणा बालज्ञानमविकल्पनं प्रसाध्य दृष्टान्तीक्रियते तदा (।)
	\pend
      
	  \bigskip
	  \begingroup
	  \large
	
	    
	    \stanza[\smallbreak]
	\label{pv.2.144}\edlabel{pv.2.144}\flagstanza{\tiny\textenglish{....2.144}}यदेव साधनं बाले तदेवात्रापि कथ्यताम् ।&साम्यादक्षधियामुक्तमनेनानुभवादिकम् ॥ १४४ ॥\&[\smallbreak]


	
	  \endgroup
	

	  \pstart {\color{DodgerBlue3}“यदेवा”}व्यभिचारि {\color{DodgerBlue3}“बाले”} बालस्येन्द्रियज्ञाने {\color{DodgerBlue3}“साधनं तदेवात्र”} व्युत्पन्नसंकेतानामिन्द्रियज्ञानेपि {\color{DodgerBlue3}“कथ्यतां”} किमिन्द्रिजत्वादिनोपन्यस्तेन । व्युत्पन्नाव्युत्पन्नयो{\color{DodgerBlue3}“रक्षधियाम”}शब्दसंसृष्टत्वमात्रानुकरणेन {\color{DodgerBlue3}“साम्यात् । अनेने”}न्द्रियजत्वदूषणेना{\color{DodgerBlue3}“नुभव”} आदिर्यस्य मानसत्त्वादेस्तदुक्तं दोषवत्तया बोद्धव्यं । (१४४)
	\pend
      \label{div_pvv.2.145}\edlabel{div_pvv.2.145}
	  
	% new div opening: depth here is 2
	

	  \begin{center}%% label @type='head'
	\textbf{(२) सामान्यनिरासः}
	\end{center}
	

	  \begin{center}%% label @type='head'
	\textbf{क. वर्णसंस्थानराहित्यादसिद्धिः}
	\end{center}
	

	  \pstart अविकल्पसिद्धौ पर\edlabel{pvv.160-3}\footnote{\label{pvv.160-3}  ३ जातिगुणक्रियाद्रव्यसम्बन्धभेदाच्चतुष्टयी शब्दानां प्रवृत्तिरिति सविकल्पवादिनं निषेधति ।}मतं दूषयित्वा स्वयं उपपत्त्यन्तरमाह (।)
	\pend
      
	  \bigskip
	  \begingroup
	  \large
	
	    
	    \stanza[\smallbreak]
	\label{pv.2.145}\edlabel{pv.2.145}\flagstanza{\tiny\textenglish{....2.145}}विशेषणं विशेष्यञ्च सम्बन्धं लौकिकीं स्थितिम् ।&गृहीत्वा सङ्कलय्यैतत् तथा प्रत्येति नान्यथा ॥ १४५ ॥\&[\smallbreak]


	
	  \endgroup
	

	  \pstart {\color{DodgerBlue3}“विशेषणं”} व्यवच्छेदकं {\color{DodgerBlue3}“विशेष्यं”} व्यवच्छेद्यं तयोः {\color{DodgerBlue3}“सम्बन्धं”} यथासम्भवं समवायादिकं {\color{DodgerBlue3}“लौकिकीं”} लोकप्रसिद्धां {\color{DodgerBlue3}“स्थितिं”} व्यवस्थाञ्च जात्यादिकं विशेषणं विशेष्यं द्रव्यादि विशेषणविशेष्यशब्दयोश्च पूर्व्वापरनियम इति पृथक् प्रत्येकं स्वरूपेण {\color{DodgerBlue3}“गृहीत्वा”} तदनन्तर{\color{DodgerBlue3}“मेतत्”} सर्व्व {\color{DodgerBlue3}“सङ्कलय्य”} संयोज्य {\color{DodgerBlue3}“तथा”} विशेषणविशिष्टत्वेन {\color{DodgerBlue3}“प्रत्येति”} विशिष्टबुद्धि{\color{DodgerBlue3}“र्नान्यथा”} । विशेषणाद्यग्रहणे । (१४५)
	\pend
      \label{div_pvv.2.146}\edlabel{div_pvv.2.146}
	  
	% new div opening: depth here is 2
	\leavevmode\marginnote{\textenglish{161/s}}
	  \bigskip
	  \begingroup
	  \large
	
	    
	    \stanza[\smallbreak]
	\label{pv.2.146}\edlabel{pv.2.146}\flagstanza{\tiny\textenglish{....2.146}}यथा दण्डिनि । जात्यादेर्व्विवेकेनानिरूपणात् ।&तद्वता योजना नास्ति कल्पनाप्यत्र नास्त्यतः ॥ १४६ ॥\&[\smallbreak]


	
	  \endgroup
	

	  \pstart {\color{DodgerBlue3}“यथा दण्डिनि”} दण्डीति विशिष्टबुद्धिः दण्डपुरुषतत्सम्बन्धादिग्रहणपूर्व्विका तदग्रहे च न भवति । जातिरादिर्यस्य गुणकर्मादेः {\color{DodgerBlue3}“स्वरूपस्य जात्या\edlabel{pvv.161-1}\footnote{\label{pvv.161-1}  १ इयं जातिरयं जातिमानित्यादिविवेकेन न भाति यतो योजना स्यात् दण्डिवत् ततो न जात्यादयः सन्ति ।}दिमतो विवेकेनानिरूपणात् तद्वता”} जातिमता {\color{DodgerBlue3}“योजना”} विशेषणविशेष्यभावो {\color{DodgerBlue3}“नास्ति । अतो”} योजनाविरहात् {\color{DodgerBlue3}“अत्र”} जातिमदादौ {\color{DodgerBlue3}“कल्पनापि नास्तीति”} तस्माज्जात्यादियोजनात्मिका कल्पना नास्ति । शब्दयोजनात्मिका तु {\color{DodgerBlue3}“सम्भाव्येत”} । सापि स्वलक्षणे संकेताभावान्निरस्ता प्राक्॥ (१४६)
	\pend
      \label{div_pvv.2.147}\edlabel{div_pvv.2.147}
	  
	% new div opening: depth here is 2
	

	  \pstart ननु यदि सामान्याभावस्तदा विभिन्नासु व्यक्तिषु कथमन्वयिप्रत्यय इत्याह (।)
	\pend
      
	  \bigskip
	  \begingroup
	  \large
	
	    
	    \stanza[\smallbreak]
	\label{pv.2.147a}\edlabel{pv.2.147a}\flagstanza{\tiny\textenglish{...2.147a}}यदप्यन्वयिविज्ञानं शब्दव्यक्त्यवभासि तत् ।\&[\smallbreak]


	
	  \endgroup
	

	  \pstart {\color{DodgerBlue3}“यदप्यन्वयिविज्ञानमु”}त्पद्यते, तच्च {\color{DodgerBlue3}“शब्द”}स्य गौरित्यादे{\color{DodgerBlue3}“र्व्यक्ते”}श्च वर्ण्णसंस्थानविशेषस्य आभास आकारस्तद्वत्प्रतीयते न जात्याभासवत् ।
	\pend
      

	  \pstart किं पुनः सामान्याभासमेव नेत्याह (।)
	\pend
      
	  \bigskip
	  \begingroup
	  \large
	
	    
	    \stanza[\smallbreak]
	\label{pv.2.147b}\edlabel{pv.2.147b}\flagstanza{\tiny\textenglish{...2.147b}}वर्ण्णाकृत्यक्षराकारशून्यं गोत्वं हि वर्ण्ण्यते ॥ १४७ ॥\&[\smallbreak]


	
	  \endgroup
	

	  \pstart {\color{DodgerBlue3}“वर्ण्णो”} नीलादिरा{\color{DodgerBlue3}“कृतिः”} संस्थान{\color{DodgerBlue3}“मक्षरं”} गवादिशब्दः । तेषामाकारो यथा प्रतीतः तेन {\color{DodgerBlue3}“शून्यं गोत्वं हि”} सामान्यवादिभि{\color{DodgerBlue3}“र्व्वर्ण्यते”} (१४७)
	\pend
      \label{div_pvv.2.148}\edlabel{div_pvv.2.148}
	  
	% new div opening: depth here is 2
	

	  \pstart अतोऽन्वयिविज्ञाने यद्वर्ण्णसंस्थानादि प्रतिभासते न तत्सामान्यं (।)
	\pend
      
	  \bigskip
	  \begingroup
	  \large
	
	    
	    \stanza[\smallbreak]
	\label{pv.2.148}\edlabel{pv.2.148}\flagstanza{\tiny\textenglish{....2.148}}समानत्वेपि तस्यैव नेक्षणं नेत्रगोचरे ।&प्रतिभासद्वयाभावात् बुद्धे र्भेदश्च दुर्लभः ॥ १४८ ॥\&[\smallbreak]


	
	  \endgroup
	

	  \pstart {\color{DodgerBlue3}“तस्यैव समानत्वे”} वा स्वीक्रियमाणे {\color{DodgerBlue3}“नेत्रगोच”}रेऽर्थे नाङ्गीकर्तव्यमीक्षणं । [विकल्पप्रतिभासिनः] {\color{DodgerBlue3}“प्रतिभासद्वयस्य”} स्फुटास्फुटवर्ण्णसंस्थानवतो{\color{DodgerBlue3}“ऽभावात्”} । एकाकारमेव ज्ञानं यद्युभयाभासमङ्गीक्रियते तदा {\color{DodgerBlue3}“बुद्धे”}(र्भे)दः प्रत्यक्षत्वाप्रत्यक्षत्वादिना दुर्लभः । यदपि स्पष्टप्रतिभासमध्यक्षं तदप्यनक्षजं स्यात् । अस्पष्टप्रतिभासत्वात् । एवमनक्षजमध्यक्षं स्यात् । स्पष्टप्रतिभासत्वात् । तस्माददृष्टेन सामान्यादिना न योजयतीत्यकल्पनमध्यक्षं (। १४८)
	\pend
      \label{div_pvv.2.149}\edlabel{div_pvv.2.149}
	  
	% new div opening: depth here is 2
	\leavevmode\marginnote{\textenglish{162/s}}

	  \begin{center}%% label @type='head'
	\textbf{ख. समवायस्यातीन्द्रियत्वादसिद्धिः}
	\end{center}
	

	  \pstart किञ्च (।)
	\pend
      
	  \bigskip
	  \begingroup
	  \large
	
	    
	    \stanza[\smallbreak]
	\label{pv.2.149a}\edlabel{pv.2.149a}\flagstanza{\tiny\textenglish{...2.149a}}समवायाग्रहादक्षैः सम्बन्धादर्शनं स्थितम् ।\&[\smallbreak]


	
	  \endgroup
	

	  \pstart {\color{DodgerBlue3}“समवा\edlabel{pvv.162-1}\footnote{\label{pvv.162-1}  १ इह बुद्धिनिबन्धनोनुमेय इष्टः ।}य”}स्यातीन्द्रियस्या{\color{DodgerBlue3}“ग्रहादक्षै”}रक्षभवैर्व्विज्ञानैर्जातितद्वतोः {\color{DodgerBlue3}“सम्बन्ध”}स्या\edlabel{pvv.162-2}\footnote{\label{pvv.162-2}  २ सम्बन्धाग्रहे तद्विशिष्टाग्रहात् ।}विशिष्टप्रतीत्या{\color{DodgerBlue3}“ऽदर्शनं स्थितं”} निश्चितं यद्बलेन यत्प्रतीतिस्तदग्रहे न युक्ता सा ।
	\pend
      

	  \pstart य\edlabel{pvv.162-3}\footnote{\label{pvv.162-3}  ३ समवायास्तित्वमाह ।}दि नास्ति समवायस्तदेह तन्तुषु पट इत्यादयो बुद्धयो न स्युरित्याह (।)
	\pend
      
	  \bigskip
	  \begingroup
	  \large
	
	    
	    \stanza[\smallbreak]
	\label{pv.2.149b}\edlabel{pv.2.149b}\flagstanza{\tiny\textenglish{...2.149b}}पटस्तन्तुष्विहेत्यादिशब्दाश्चेमे स्वयं कृताः ॥ १४९ ॥\&[\smallbreak]


	
	  \endgroup
	\leavevmode\marginnote{\textenglish{31b/MA}}

	  \pstart इह {\color{DodgerBlue3}“तन्तुषु पट इत्यादि शब्दाः इमे स्वयं”} समयानुलोचनैः {\color{DodgerBlue3}“कृता”} न वस्तुपराधीनाः । (१४९)
	\pend
      \label{div_pvv.2.150}\edlabel{div_pvv.2.150}
	  
	% new div opening: depth here is 2
	

	  \pstart तथा\edlabel{pvv.162-4}\footnote{\label{pvv.162-4}  ४ नापि लौकिका इत्याह ।} हि (।)
	\pend
      
	  \bigskip
	  \begingroup
	  \large
	
	    
	    \stanza[\smallbreak]
	\label{pv.2.150a}\edlabel{pv.2.150a}\flagstanza{\tiny\textenglish{...2.150a}}शृङ्गं गवीति लोके स्यात् शृङ्गे गौरित्यलौकिकम् ।\&[\smallbreak]


	
	  \endgroup
	

	  \pstart {\color{DodgerBlue3}“शृङ्गं गवि”} तिष्ट{\color{DodgerBlue3}“तीति लोके स्यात्”} प्रमाणप्रसिद्ध्योरनुरोधात् । {\color{DodgerBlue3}“शृङ्गे गौरिति तु”} तदुपकल्पित{\color{DodgerBlue3}“मलौकिकं”} प्रमाणप्रसिद्धिबहिर्भावात् ।
	\pend
      

	  \begin{center}%% label @type='head'
	\textbf{(३) अवयविनिरासः}
	\end{center}
	

	  \pstart यद्यवयवेभ्यो न गौर्भिन्नस्तदा गवि शृङ्गमित्यपि न स्यादित्याह\edlabel{pvv.162-5}\footnote{\label{pvv.162-5}  ५ अतीन्द्रियत्वान्न समवायादयं व्यपदेशः किन्तु ।} (।)
	\pend
      
	  \bigskip
	  \begingroup
	  \large
	
	    
	    \stanza[\smallbreak]
	\label{pv.2.150b}\edlabel{pv.2.150b}\flagstanza{\tiny\textenglish{...2.150b}}गवाख्यपरिशिष्टाङ्गविच्छेदानुपलम्भनात् ॥ १५० ॥\&[\smallbreak]


	
	  \endgroup
	

	  \pstart शृङ्गस्य {\color{DodgerBlue3}“गवाख्यैः परिशिष्टा\edlabel{pvv.162-6}\footnote{\label{pvv.162-6}  ६ सास्नाद्यैः ।}ङ्गैर्व्विच्छेदेस्य”} वियोगस्या{\color{DodgerBlue3}“नुपलम्भनात्”} शृङ्गं गवीत्युच्यते न त्ववयवातिरिक्तगोसद्भावात् । (१५०)
	\pend
      \label{div_pvv.2.151}\edlabel{div_pvv.2.151}
	  
	% new div opening: depth here is 2
	

	  \pstart ननु तन्तुषु पट इति भवत्येव प्रतीतिरिति चेत् । आह (।)
	\pend
      
	  \bigskip
	  \begingroup
	  \large
	
	    
	    \stanza[\smallbreak]
	\label{pv.2.151}\edlabel{pv.2.151}\flagstanza{\tiny\textenglish{....2.151}}तैस्तन्तुभिरियं शाटीत्युत्तरं कार्यमुच्यते ।&तन्तुसंस्कारसम्भूतं नैककालं कथञ्चन ॥ १५१ ॥\&[\smallbreak]


	
	  \endgroup
	

	  \pstart {\color{DodgerBlue3}“तैस्तन्तुभिः”} पटावस्थाप्राग्भाविभि{\color{DodgerBlue3}“रियं शाटीति”} कारणभूततन्तूत्तरकालभावि{\color{DodgerBlue3}“तन्तू”}नां {\color{DodgerBlue3}“सँस्कार”}स्तुरीवेमकुविन्दकरादिसहकारी प्रभवावस्थाविशेषलाभः । तस्मात्स्वरसेन निरुध्यमानात्सं{\color{DodgerBlue3}“भूतंकार्यमुच्यते”} । न तु तन्तुभिः सहै{\color{DodgerBlue3}“ककालं”} कार्यं \leavevmode\marginnote{\textenglish{163/s}} तन्तुपट इति {\color{DodgerBlue3}“कथञ्चन कथ्यते”} कार्यकारणयोः सम\edlabel{pvv.163-1}\footnote{\label{pvv.163-1}  १ समवायस्तु समकालयोरेव ।}कालत्वाभावात् (। १५१)
	\pend
      \label{div_pvv.2.152}\edlabel{div_pvv.2.152}
	  
	% new div opening: depth here is 2
	

	  \pstart यदि नास्ति तन्तुपटयोर्भेदस्तदा कथमेते तन्तवः पटश्चायमिति व्यपदेश इत्याह (।)
	\pend
      
	  \bigskip
	  \begingroup
	  \large
	
	    
	    \stanza[\smallbreak]
	\label{pv.2.152}\edlabel{pv.2.152}\flagstanza{\tiny\textenglish{....2.152}}कारणारोपतः कश्चित् एकापोद्धारतोपि वा ।&तन्त्वाख्यां वर्त्तयेत् कार्ये दर्शयन् नाश्रयं श्रुतेः ॥ १५२ ॥\&[\smallbreak]


	
	  \endgroup
	

	  \pstart {\color{DodgerBlue3}“कारणा”}नान्तन्तूनामा{\color{DodgerBlue3}“रोपतः”} । {\color{DodgerBlue3}“एक”}स्य तन्तोर{\color{DodgerBlue3}“पोद्धारतो”} बुद्ध्या निःकर्षणात् वा {\color{DodgerBlue3}“कश्चिद्”} व्यवहर्त्ता कार्ये पटे {\color{DodgerBlue3}“तन्त्वाख्यां”} तन्तुश्रुतिं {\color{DodgerBlue3}“वर्तयेत्”} । पटश्रुतेराश्रयं कारणं {\color{DodgerBlue3}“दर्शयेन् न”} तन्तुभ्यो व्यतिरिक्तः {\color{DodgerBlue3}“पटोस्ति”} एवमाकारपरिणतास्तन्तवः पट इत्यर्थः । (१५२)
	\pend
      \label{div_pvv.2.153}\edlabel{div_pvv.2.153}
	  
	% new div opening: depth here is 2
	

	  \pstart यदि तन्तवः केवला न तेभ्यः पटोऽन्यस्तदा पटव्यपदेशो निर्निबन्धनः स्यादित्याह (।)
	\pend
      
	  \bigskip
	  \begingroup
	  \large
	
	    
	    \stanza[\smallbreak]
	\label{pv.2.153a}\edlabel{pv.2.153a}\flagstanza{\tiny\textenglish{...2.153a}}उपकार्योपकारित्वं विच्छेदाद् दृष्टिरेव वा ।\&[\smallbreak]


	
	  \endgroup
	

	  \pstart तन्तूनां परस्परं शीताद्यपनोदक्षमं साहित्य{\color{DodgerBlue3}“मुपकार्योपकारित्वं”} । अन्योन्यस्य {\color{DodgerBlue3}“विच्छेदाद् दृष्टिरेव वा”} पटव्यपदेशनिब\edlabel{pvv.163-2}\footnote{\label{pvv.163-2}  २ न समवायः ।}न्धनमिति शेषः ॥
	\pend
      

	  \pstart यदि व्यतिरिक्तं व्यपदेशनिबन्धनं नास्ति तदा पट इति व्यपदेशो न मुख्यः स्यात् । बा ही के गोव्यपदेशवदित्याह (।)
	\pend
      
	  \bigskip
	  \begingroup
	  \large
	
	    
	    \stanza[\smallbreak]
	\label{pv.2.153b}\edlabel{pv.2.153b}\flagstanza{\tiny\textenglish{...2.153b}}मुख्यं यदस्खलज्ज्ञानमादिसंकेतगोचरः ॥ १५३ ॥\&[\smallbreak]


	
	  \endgroup
	

	  \pstart {\color{DodgerBlue3}“मुख्य”}न्तदुच्यते {\color{DodgerBlue3}“यदादिसंकेत”}स्य {\color{DodgerBlue3}“गोचरो”} न व्यतिरिक्त इत्येव परस्पराविच्छेदावस्थेषु च पटश्रुतेः संकेता{\color{DodgerBlue3}“दस्खल”}द्गतिगोचरत्वान्मुख्यत्वं । \edlabel{pvv.163-3}\footnote{\label{pvv.163-3}  ३ यथा सास्नादिमान् गौः ।}तस्मात्प्रत्यक्षतो जातेरनुपलम्भाद्योजनाविरहः । (१५३)
	\pend
      \label{div_pvv.2.154}\edlabel{div_pvv.2.154}
	  
	% new div opening: depth here is 2
	

	  \begin{center}%% label @type='head'
	\textbf{(४) नानुमानतः सामान्यसिद्धिः}
	\end{center}
	

	  \pstart स्यादेतत् (।) विशिष्टप्रतीतिर्व्विशेषणप्रतीतिपूर्व्विका यथा दण्डिप्रतीतिः । विशिष्टप्रतीतिश्च शावलेयादिषु गौरिति विशेषणञ्च शावलेयादिषु गोत्वमेवेत्यनुमानतो जातिसिद्धिरित्याह (।)
	\pend
      \leavevmode\marginnote{\textenglish{164/s}}
	  \bigskip
	  \begingroup
	  \large
	
	    
	    \stanza[\smallbreak]
	\label{pv.2.154a}\edlabel{pv.2.154a}\flagstanza{\tiny\textenglish{...2.154a}}अनुमानञ्च जात्यादौ वस्तुनो नास्ति भेदिनि ।\&[\smallbreak]


	
	  \endgroup
	

	  \pstart {\color{DodgerBlue3}“अनुमानं वस्तुनः”} शाबलेयादे{\color{DodgerBlue3}“र्भेदिनि जात्यादौ नास्ति”} । न हि व्यक्तिव्यतिरिक्तं विशेषणमुपलभ्यते शृङ्गाद्यवयवसन्निवेश एव त्वभिन्नो विशेषणमस्तु । तथा च नाभिमतसिद्धिः ।
	\pend
      

	  \pstart दृष्टान्तासिद्धिमप्याह (।)
	\pend
      
	  \bigskip
	  \begingroup
	  \large
	
	    
	    \stanza[\smallbreak]
	\label{pv.2.154b}\edlabel{pv.2.154b}\flagstanza{\tiny\textenglish{...2.154b}}सर्व्वत्र व्यपदेशो हि दण्डादेरपि सांवृतात् ॥ १५४ ॥\&[\smallbreak]


	
	  \endgroup
	

	  \pstart {\color{DodgerBlue3}“सर्व्वत्र पुरुषादौ”} दण्डीत्यादि{\color{DodgerBlue3}“व्यपदेशो”}पि {\color{DodgerBlue3}“हि”} न {\color{DodgerBlue3}“दण्डादे”}र्व्वस्तुनः किन्तु {\color{DodgerBlue3}“सांवृतात्\edlabel{pvv.164-1}\footnote{\label{pvv.164-1}  १ दण्डदण्डिनोः परस्परोपकार्योपकारकभावोपकल्पितादवस्थाविशेषादर्थान्तरभूतात् ।}”} । दण्डस्वलक्षणस्य व्युपदेशे हेतुत्वे सर्व्वत्र पुरुषे स्यात् । सम्बन्धिन्येवान्यत्रेति चेत् । तर्हि सम्बन्धो दण्डः कारणं दण्डि व्यपदेशस्य सम्बन्धश्च संयोगादिर्नास्तीति परं कार्यकारणभावः परिशिष्यते । तस्य निमित्तत्वे यथा दण्डी पुरुषस्तथा पुरुषी दण्ड इत्यपि स्यात् । समानत्वान्निमित्तस्य (।) तस्मात् कल्पितविशेषणभावनियमो दण्डः सम्बन्धी सांवृत एव विशेषणं । (१५४)
	\pend
      \label{div_pvv.2.155}\edlabel{div_pvv.2.155}
	  
	% new div opening: depth here is 2
	

	  \pstart किञ्च (।)
	\pend
      
	  \bigskip
	  \begingroup
	  \large
	
	    
	    \stanza[\smallbreak]
	\label{pv.2.155}\edlabel{pv.2.155}\flagstanza{\tiny\textenglish{....2.155}}वस्तुप्रासादमालादिशब्दाश्चान्यानपेक्षिणः ।&गेहो यद्यपि संयोगस्तन्माला किन्तु तद्भवेत् ॥ १५५ ॥\&[\smallbreak]


	
	  \endgroup
	

	  \pstart षट्सु पदार्थेषु {\color{DodgerBlue3}“वस्तु”} वस्त्विति सर्व्वानुयायी शब्दः प्रासादेषु {\color{DodgerBlue3}“प्रासादमाले”}ति शब्दो गृहे\edlabel{pvv.164-2}\footnote{\label{pvv.164-2}  २ आदिना ।}षु बहुषु नगरमित्या{\color{DodgerBlue3}“दिशब्दाश्चान्यानपेक्षिणो”}ऽर्थान्तरभूतविशेषणरहिता \leavevmode\marginnote{\textenglish{32a/MA}} इति व्यभिचारिता\edlabel{pvv.164-3}\footnote{\label{pvv.164-3}  ३ अनैकान्तिकता ।} हेतोः । न हि पदार्थेषु व्यतिरिक्तं सामा\edlabel{pvv.164-4}\footnote{\label{pvv.164-4}  ४ वस्तुवस्त्विति ।}न्यं वस्तुतात्वमभ्युपगम्यते वै शे षि कैः । न च प्रासादो द्रव्यं विजातीयानां द्रव्यानारम्भात् । ततश्च मालागु\edlabel{pvv.164-5}\footnote{\label{pvv.164-5}  ५ न द्रव्यं ।}णोपि {\color{DodgerBlue3}“गेहो\edlabel{pvv.164-6}\footnote{\label{pvv.164-6}  ६ अर्थान्तरसंयोगाभावादभ्युपगम्योच्यते ।} यद्यपि संयोगस्त”}स्य {\color{DodgerBlue3}“माला किन्तु तद् भवे\edlabel{pvv.164-7}\footnote{\label{pvv.164-7}  ७ वस्त्वनन्तर्भा(वा)न्न किञ्चित् ।}त्”} । न भावगुणो निर्गुणत्वात् गुणानां । (१५५)
	\pend
      \label{div_pvv.2.156}\edlabel{div_pvv.2.156}
	  
	% new div opening: depth here is 2
	\leavevmode\marginnote{\textenglish{165/s}}

	  \begin{center}%% label @type='head'
	\textbf{क. सामान्यस्वीकारे दोषः}
	\end{center}
	
	  \bigskip
	  \begingroup
	  \large
	
	    
	    \stanza[\smallbreak]
	\label{pv.2.156}\edlabel{pv.2.156}\flagstanza{\tiny\textenglish{....2.156}}जातिश्चेद् गेह एकोपि मालेत्युच्येत वृक्षवत् ।&मालाबहुत्वे तच्छब्दः कथं जातेरजातितः ॥ १५६ ॥\&[\smallbreak]


	
	  \endgroup
	

	  \pstart {\color{DodgerBlue3}“जातिश्चेद”}भ्युपगम्यते {\color{DodgerBlue3}“एको गेहो मालेत्युच्येत वृक्षवत्”} । यथा वृक्षत्वजातियोगादेको वृक्षो वृक्ष इत्युच्यते एवमेकोपि गेहो माला स्यात् । तथा गेहमालानां बहुत्वे {\color{DodgerBlue3}“तच्छब्दो”} मालेत्यनुगामिशब्दः {\color{DodgerBlue3}“कथं जाते\edlabel{pvv.165-1}\footnote{\label{pvv.165-1}  १ निःसामान्यानि सामान्यानीति वचनात् ।}”} र्मालायां {\color{DodgerBlue3}“अजातितो”} जात्यन्तरविरहात् । (१५६)
	\pend
      \label{div_pvv.2.157}\edlabel{div_pvv.2.157}
	  
	% new div opening: depth here is 2
	

	  \pstart किञ्च (।)
	\pend
      
	  \bigskip
	  \begingroup
	  \large
	
	    
	    \stanza[\smallbreak]
	\label{pv.2.157}\edlabel{pv.2.157}\flagstanza{\tiny\textenglish{....2.157}}मालादौ च महत्वादिरिष्टो यश्चौपचारिकः ।&मुख्याविशिष्टविज्ञानग्राह्यत्वान्नौपचारिकः ॥ १५७ ॥\&[\smallbreak]


	
	  \endgroup
	

	  \pstart महती प्रासादमालेति कथं व्यपदेशः । महत्वं परिमाणं गुणः । तस्य मालायां न सत्त्वं(।) न हि प्रासादमाला किञ्चिदित्युक्तं । नापि संयोगलक्षणे प्रासादे महत्त्वं निर्गुणत्वा(त्) गुणानां । काष्ठादिषु द्रव्येषु प्रासादारम्भकेषु महत्त्वसत्त्वाद्वृक्षेषु कुसुमसम्भवाद्\edlabel{pvv.165-2}\footnote{\label{pvv.165-2}  २ अवयव्यभावाद्वनसंख्या ।} वनसंख्यालक्षणं कुसुमितमिति यथोच्यते । तथा प्रासाद{\color{DodgerBlue3}“मालादौ महत्वादिरौपचारिको”} यश्चेष्टः स चायुक्तः काष्ठादिष्वपि तादृशस्य मह\edlabel{pvv.165-3}\footnote{\label{pvv.165-3}  ३ प्रत्यवयदेषु ।}त्त्वस्याभावात् ।
	\pend
      

	  \pstart किञ्च(।) महान् पर्व्वत इति\edlabel{pvv.165-4}\footnote{\label{pvv.165-4}  ४ अस्मन्मते ।} {\color{DodgerBlue3}“मु”}ख्यमहत्त्वग्राहकज्ञानेनास्खल\edlabel{pvv.165-5}\footnote{\label{pvv.165-5}  ५ मुख्यविषयौ यौ शब्दप्रत्ययौ ताभ्यामविशिष्टो यः प्रत्ययस्तेन ग्राह्य एतावन्मात्रनिमित्तत्वान्मुख्यत्वव्यवस्थायाः ।}द्वृत्तित्वादविशिष्टेन ज्ञानेन ग्राह्यत्वात् प्रासादमालामहत्वादि{\color{DodgerBlue3}“रौपचारिको”} न युक्तः (।) न हि माणवक इव सिंहबुद्धिर्महत्त्वबुद्धिश्च मालायां स्खलति । (१५७)
	\pend
      \label{div_pvv.2.158}\edlabel{div_pvv.2.158}
	  
	% new div opening: depth here is 2
	

	  \pstart किञ्च (।) भिन्नविशेषणं {\color{DodgerBlue3}“मुख्य”}मभिन्नविशेषणञ्चामुख्यमिति यदुच्यते तदप्ययुक्तमित्याह (।)
	\pend
      
	  \bigskip
	  \begingroup
	  \large
	
	    
	    \stanza[\smallbreak]
	\label{pv.2.158}\edlabel{pv.2.158}\flagstanza{\tiny\textenglish{....2.158}}अनन्यहेतुता तुल्या सा मुख्याभिमतेष्वपि ॥&पदार्थशब्दः कं हेतुमन्यं षट्कं समीक्षते ॥ १५८ ॥\&[\smallbreak]


	
	  \endgroup
	

	  \pstart {\color{DodgerBlue3}“मुख्याभिमतेष्वपि”} दण्ड्यादिष्व{\color{DodgerBlue3}“नन्यहेतुता”} भिन्नविशेषणनिमित्तरहितता {\color{DodgerBlue3}“तुल्या”} गौणेन \edlabel{pvv.165-6}\footnote{\label{pvv.165-6}  ६ सह मुख्येषु ।}सर्व्वत्र व्यपदेशो हि दण्डा\edlabel{pvv.165-7}\footnote{\label{pvv.165-7}  ७ असत एव विशेषणत्वेनोपनयात् ।}देरपि सांवृतादि(२।१५४)त्युक्तेः । किञ्चा\leavevmode\marginnote{\textenglish{166/s}} नुयायी {\color{DodgerBlue3}“पदार्थशब्दः”} षष्ठ्या पदार्थेषु {\color{DodgerBlue3}“कमन्यं हेतुं”} निमित्तं प्रवृत्तौ {\color{DodgerBlue3}“समीक्ष्यते”} (? क्षते)\edlabel{pvv.166-1}\footnote{\label{pvv.166-1}  १ पूर्व्वमनुमादूषणे उक्तेधुना मुख्योपचारलक्षणदूषणे ।}। न हि षट्पदार्थातिरिक्तं किञ्चिदस्ति (।) (१५८)
	\pend
      \label{div_pvv.2.159}\edlabel{div_pvv.2.159}
	  
	% new div opening: depth here is 2
	
	  \bigskip
	  \begingroup
	  \large
	
	    
	    \stanza[\smallbreak]
	\label{pv.2.159a}\edlabel{pv.2.159a}\flagstanza{\tiny\textenglish{...2.159a}}यो यथा रूढितः सिद्धः तत्साम्याद्यस्तथोच्यते ।\&[\smallbreak]


	
	  \endgroup
	

	  \pstart तस्माद्योऽर्थो येन प्रकारेण रूढितः आदिसंकेतानुसारेण सिद्धः स मुख्यः । यश्च तस्य मुख्यस्य साम्यात्तथामुख्यवाचकशब्देनोच्यते स गौणः (।)
	\pend
      

	  \pstart कुत एतदित्याह (।)
	\pend
      
	  \bigskip
	  \begingroup
	  \large
	
	    
	    \stanza[\smallbreak]
	\label{pv.2.159b}\edlabel{pv.2.159b}\flagstanza{\tiny\textenglish{...2.159b}}यत्र गौणश्च भावेष्यप्यभावस्योपचारतः ॥ १५९ ॥\&[\smallbreak]


	
	  \endgroup
	

	  \pstart {\color{DodgerBlue3}“भावेष्वपि”} कुपुत्रादिषु पुत्रादिरित्यभावो{\color{DodgerBlue3}“पचारतः”} । परमते तु भावे वृत्तत्वान्मुख्यत्वं भवेत् । (१५९)
	\pend
      \label{div_pvv.2.160}\edlabel{div_pvv.2.160}
	  
	% new div opening: depth here is 2
	

	  \begin{center}%% label @type='head'
	\textbf{ख. विवक्षान्वयिसंकेतानुगमत्वाद् रूढेः}
	\end{center}
	

	  \pstart स्यादेतद् (।) रूढ्यैव मुख्यता किन्तु सापि भिन्ने विशेषणे सतीत्याह (।) न (।)
	\pend
      
	  \bigskip
	  \begingroup
	  \large
	
	    
	    \stanza[\smallbreak]
	\label{pv.2.160}\edlabel{pv.2.160}\flagstanza{\tiny\textenglish{....2.160}}संकेतान्वयिनी रूढिर्वक्तुरिच्छान्वयी च सः ।&क्रियते व्यवहारार्थ छन्दः शब्दांशनाभवत् ॥ १६० ॥\&[\smallbreak]


	
	  \endgroup
	

	  \pstart {\color{DodgerBlue3}“संकेतान्वयिनी”} यथासंकेतं {\color{DodgerBlue3}“रूढिः स च”} संकेतो {\color{DodgerBlue3}“वक्तुः\edlabel{pvv.166-2}\footnote{\label{pvv.166-2}  २ नान्तरनिमित्तः ।}”} संकेतयितु{\color{DodgerBlue3}“रिच्छान्वयी व्य\edlabel{pvv.166-3}\footnote{\label{pvv.166-3}  ३ एवं तर्हि किमर्थं क्रियते आह ।}वहारार्थं क्रियते । छन्दसो”} गायत्र्या\edlabel{pvv.166-4}\footnote{\label{pvv.166-4}  ४ अर्थान्तरविशेषणम्विना वृत्तेष्वनष्टवादिनामवत् ।}देः {\color{DodgerBlue3}“शब्दां\edlabel{pvv.166-5}\footnote{\label{pvv.166-5}  ५ अवयव ।}”}शस्य प्रकृतिप्रत्ययादे{\color{DodgerBlue3}“र्नामवत्”} । न हि विशिष्टानुपूर्व्वीकेषु वर्ण्णेषु पृथग्भूतं गायत्र्यादिशूब्दनिमित्तं किञ्चिदस्ति । शब्दांशेषु वाऽपि तु संकेतयितुरिच्छानुरोधादेव तथा व्यपदेशः । (१६०)
	\pend
      \label{div_pvv.2.161}\edlabel{div_pvv.2.161}
	  
	% new div opening: depth here is 2
	

	  \pstart यदि व्यक्तिभ्यो न भिन्नं सामान्यं तदा कथमनुगामी प्रत्यय इत्याह (।)
	\pend
      
	  \bigskip
	  \begingroup
	  \large
	
	    
	    \stanza[\smallbreak]
	\label{pv.2.161}\edlabel{pv.2.161}\flagstanza{\tiny\textenglish{....2.161}}वस्तुधर्मतयैवार्थास्तादृग्विज्ञानकारणम् ।&भेदेपि यत्र तज्ज्ञानांत्तान्तथा प्रतिपद्यते ॥ १६१ ॥\&[\smallbreak]


	
	  \endgroup
	\leavevmode\marginnote{\textenglish{167/s}}

	  \pstart {\color{DodgerBlue3}“वस्तुधर्मतया”} प्रकृ{\color{DodgerBlue3}“त्यैव”} केचि{\color{DodgerBlue3}“दर्थाः”} परस्परं {\color{DodgerBlue3}“भेदेपि तादृ”}शस्यानुगामिनोऽतत्कार्यंव्यावृत्तिविषयस्य {\color{DodgerBlue3}“विज्ञानस्य कारणं”} । यत्र येष्वर्थेष्वनुगामि ज्ञानं तानभेदिनोऽर्थान् {\color{DodgerBlue3}“तथा”} एकत्वे{\color{DodgerBlue3}“न प्रतिपद्यते”} न त्वेकसामान्यं (? न्य-) बलात्तथा ज्ञानं । (१६१)
	\pend
      \label{div_pvv.2.162}\edlabel{div_pvv.2.162}
	  
	% new div opening: depth here is 2
	

	  \pstart स्यादेतत् । प्रतिव्यक्ति ज्ञानान्यपि भिन्नानीति कथमनुगामि ज्ञानमित्याह (।)
	\pend
      
	  \bigskip
	  \begingroup
	  \large
	
	    
	    \stanza[\smallbreak]
	\label{pv.2.162}\edlabel{pv.2.162}\flagstanza{\tiny\textenglish{....2.162}}ज्ञानान्यपि तथा भेदेऽभेदप्रत्यवमर्शने ।&इत्यतत्कार्यविश्लेषस्यान्वयो नैकवस्तुनः ॥ १६२ ॥\&[\smallbreak]


	
	  \endgroup
	

	  \pstart {\color{DodgerBlue3}“ज्ञानान्यपि”} परस्परतो {\color{DodgerBlue3}“भेदे तथा”}ऽर्थवद्वस्तुधर्मितया{\color{DodgerBlue3}“ऽभेदप्रत्यवमर्शने”} निमित्तं ततो ज्ञानान्यपि तदेकपरामर्शगोचरतयाऽनुगामिप्रत्यय उच्यन्ते इत्यनेन प्रकारेण भेदिष्वर्थेष्वत{\color{DodgerBlue3}“त्कार्या”}दर्थस्य {\color{DodgerBlue3}“विश्लेषो”} व्यवच्छेदस्तस्या{\color{DodgerBlue3}“न्वयो”} विद्यते। {\color{DodgerBlue3}“न त्वेकस्य”}\leavevmode\marginnote{\textenglish{32b/MA}} {\color{DodgerBlue3}“वस्तुनः”} सामान्यस्य (। १६२)
	\pend
      \label{div_pvv.2.163}\edlabel{div_pvv.2.163}
	  
	% new div opening: depth here is 2
	
	  \bigskip
	  \begingroup
	  \large
	
	    
	    \stanza[\smallbreak]
	\label{pv.2.163a}\edlabel{pv.2.163a}\flagstanza{\tiny\textenglish{...2.163a}}वस्तूनां विद्यते तस्मात्तन्निष्ठा वस्तुनि श्रुतिः ।\&[\smallbreak]


	
	  \endgroup
	

	  \pstart {\color{DodgerBlue3}“व\edlabel{pvv.167-1}\footnote{\label{pvv.167-1}  १ वस्तुषु ।}स्तूनां”} विशेषाणामन्वयो {\color{DodgerBlue3}“विद्यते”}ऽनुपलम्भबाधितत्वात्तस्य । {\color{DodgerBlue3}“तस्मात् तन्निष्ठा”} व्यावृत्तिविषया {\color{DodgerBlue3}“वस्तु”}नि {\color{DodgerBlue3}“श्रुतिः”} प्रवर्तते ।
	\pend
      

	  \begin{center}%% label @type='head'
	\textbf{(५) अन्यापोहचिन्ता}
	\end{center}
	

	  \begin{center}%% label @type='head'
	\textbf{क. अतत्कार्यव्यावृत्तिः}
	\end{center}
	

	  \pstart नन्वतत्कार्यव्यावृत्तिर्वस्तुनः स्वभावभूता ततो व्यावृत्तिविषयत्वे {\color{DodgerBlue3}“वस्तुविषयतैव शब्दस्य”} स्यादित्याह\edlabel{pvv.167-2}\footnote{\label{pvv.167-2}  २ न च शब्दो वस्तुस्पर्शीत्यनेनास्य विरोधं मन्यते ।} (।)
	\pend
      
	  \bigskip
	  \begingroup
	  \large
	
	    
	    \stanza[\smallbreak]
	\label{pv.2.163b}\edlabel{pv.2.163b}\flagstanza{\tiny\textenglish{...2.163b}}बाह्यशक्तिव्यवच्छेदनिष्ठाभावेपि तच्छ्रुतिः ॥ १६३ ॥\&[\smallbreak]


	
	  \endgroup
	

	  \pstart {\color{DodgerBlue3}“बाह्य”}स्य वस्तुनः {\color{DodgerBlue3}“शक्ते”}रतत्कार्याद्यो {\color{DodgerBlue3}“व्यवच्छेद”}स्तत्र {\color{DodgerBlue3}“निष्ठा”} विषयित्वं तस्याभावेपि {\color{DodgerBlue3}“तच्छ्रुति”}र्व्यवच्छेदवाचिनी श्रु\edlabel{pvv.167-3}\footnote{\label{pvv.167-3}  ३ कथमिति न वृत्तेन सम्बन्धनीयः ।}तिः । (१६३)
	\pend
      \label{div_pvv.2.164}\edlabel{div_pvv.2.164}
	  
	% new div opening: depth here is 2
	
	  \bigskip
	  \begingroup
	  \large
	
	    
	    \stanza[\smallbreak]
	\label{pv.2.164}\edlabel{pv.2.164}\flagstanza{\tiny\textenglish{....2.164}}विकल्पप्रतिबिम्बेषु तन्निष्ठेषु निबध्यते ।&ततोन्यापोहनिष्ठत्वादुक्तान्यापोहकृच्छ्रुतिः ॥ १६४ ॥\&[\smallbreak]


	
	  \endgroup
	

	  \pstart विक{\color{DodgerBlue3}“ल्पानां प्रतिबिम्बे”}ष्वाकारेषु {\color{DodgerBlue3}“तन्निष्ठेषु”} तद्व्यावृत्तिवस्तुत्वेन व्यवस्थाविषयतया तद्व्यवहारव्यवस्थितिषु संके\edlabel{pvv.167-4}\footnote{\label{pvv.167-4}  ४ स्वलक्षणविकल्पविषययोरैक्यभ्रान्त्या ।}तकाले {\color{DodgerBlue3}“निबध्यते । ततो”} विकल्पप्रति\leavevmode\marginnote{\textenglish{168/s}} बिम्बानां बाह्य\edlabel{pvv.168-1}\footnote{\label{pvv.168-1}  १ वस्तुविवेकनिष्ठत्वात् ।}व्यावृत्तात्मत्वेन व्यवहारविषयत्वादन्या{\color{DodgerBlue3}“पोहनिष्ठत्वा”}त्कारणा{\color{DodgerBlue3}“दुक्ता श्रुतिरन्यापोहकृत्”} । अन्यव्यावृत्ताकारविकल्पजननात् अन्यव्यावृत्तेषु प्रव\edlabel{pvv.168-2}\footnote{\label{pvv.168-2}  २ दृश्यविकल्पयोरेकीकरणात् ।}र्तनाच्च शब्दोऽन्यापोहकृदुक्तः । (१६४)
	\pend
      \label{div_pvv.2.165}\edlabel{div_pvv.2.165}
	  
	% new div opening: depth here is 2
	

	  \pstart ननु शाब्दे ज्ञाने ग्राह्यं बाह्यतयैव प्रतीयते न ज्ञानाकारतयेत्याह (।)
	\pend
      
	  \bigskip
	  \begingroup
	  \large
	
	    
	    \stanza[\smallbreak]
	\label{pv.2.165}\edlabel{pv.2.165}\flagstanza{\tiny\textenglish{....2.165}}व्यतिरेकीव यज्ज्ञाने भात्यर्थप्रतिबिम्बकम् ।&शब्दात्तदपि नार्थात्मा भ्रान्तिः सा वासनोद्भवा ॥ १६५ ॥\&[\smallbreak]


	
	  \endgroup
	

	  \pstart {\color{DodgerBlue3}“शब्दादु”}त्पन्न{\color{DodgerBlue3}“ज्ञानेऽर्थप्रतिबिम्बकं व्यतिरेकीव”} भिन्नं बाह्यमिव {\color{DodgerBlue3}“यदाभाति तदपि नार्थात्मा”} बहिरर्थस्वरूपं किन्तु {\color{DodgerBlue3}“भ्रान्तिः सा वासना\edlabel{pvv.168-3}\footnote{\label{pvv.168-3}  ३ स्वलक्षणानुभववासनाहेतु ।}निर्मिता”} । यथा तैमिरिकदृष्टेषु केशादिषु बाह्यभ्रमः । एवं विकल्पाकारेपि बाह्यव्यवहारोऽविद्यावशादित्यर्थः । (१६५)
	\pend
      \label{div_pvv.2.166}\edlabel{div_pvv.2.166}
	  
	% new div opening: depth here is 2
	

	  \pstart ज्ञाना\edlabel{pvv.168-4}\footnote{\label{pvv.168-4}  ४ सावतारः श्लोकः पूर्व्वपक्षः ।}कारस्तर्हि वस्तुभूतो वाच्यः स्यादित्याह (।)
	\pend
      
	  \bigskip
	  \begingroup
	  \large
	
	    
	    \stanza[\smallbreak]
	\label{pv.2.166}\edlabel{pv.2.166}\flagstanza{\tiny\textenglish{....2.166}}तस्याभिधाने श्रुतिभिरर्थे कोंशोवगम्यते ।&तस्यागतौ च संकेतक्रिया व्यर्था तदर्थिका ॥ १६६ ॥\&[\smallbreak]


	
	  \endgroup
	

	  \pstart {\color{DodgerBlue3}“तस्य”} ज्ञानाकारस्य {\color{DodgerBlue3}“श्रुतिभिरभिधानेऽर्थे”}ऽतत्कार्यव्यावृत्ते शब्देनाचोदिते कोंशोवगम्यते न कश्चित्तदकार्यव्यावृ{\color{DodgerBlue3}“त्तस्या”}र्थस्य शब्दा{\color{DodgerBlue3}“दगतौ च”} सत्यां {\color{DodgerBlue3}“संकेतक्रिया व्यर्था”} य\edlabel{pvv.168-5}\footnote{\label{pvv.168-5}  ५ कुत इत्याह ।}स्मा{\color{DodgerBlue3}“त्तदर्थिका”} अतत्कार्यव्यावृत्तार्थप्रतीतिफला सेष्यते । (१६६)
	\pend
      \label{div_pvv.2.167}\edlabel{div_pvv.2.167}
	  
	% new div opening: depth here is 2
	

	  \pstart एवन्तर्ह्यन्यापोहेपि संकेते कृते प्रवृत्तिरर्थेषु न स्यात् । तस्यार्थात्मत्वाभावादित्याह ।\edlabel{pvv.168-6}\footnote{\label{pvv.168-6}  ६ अत्राह सिद्धान्ती ।}
	\pend
      
	  \bigskip
	  \begingroup
	  \large
	
	    
	    \stanza[\smallbreak]
	\label{pv.2.167}\edlabel{pv.2.167}\flagstanza{\tiny\textenglish{....2.167}}शब्दोर्थांशंकमाहेति तत्रान्यापोह उच्यते ।&आकारः स च नार्थेस्ति तं वदन्नर्थभाक् कथम् ॥ १६७ ॥\&[\smallbreak]


	
	  \endgroup
	

	  \pstart {\color{DodgerBlue3}“शब्दोऽर्थांशंकमाहेति”} प्रश्ने {\color{DodgerBlue3}“तत्रान्यापोहो”}ऽतत्कार्यव्यावृत्तिः सर्व्वविशेषसंभविनी वाच्यतयो{\color{DodgerBlue3}“च्यते”} । अतोऽर्थांशात्मन्यन्यापोहे गृहीतसंकेतः शब्दादुच्चरितार्थं प्रतीत्य तत्र प्रवर्त्तित इति युक्तं । यस्तत्रा\edlabel{pvv.168-7}\footnote{\label{pvv.168-7}  ७ बाह्यनीलादिदर्शनाभ्यासायातो ज्ञानीयः ।} {\color{DodgerBlue3}“क्षाकारः स चार्थे नास्ति तं”} बुद्ध्याकारं {\color{DodgerBlue3}“वदन् शब्दोऽर्थभाक्”} बाह्यार्थाभिधायी कथमस्तु । (१६७)
	\pend
      \label{div_pvv.2.168}\edlabel{div_pvv.2.168}
	  
	% new div opening: depth here is 2
	\leavevmode\marginnote{\textenglish{169/s}}

	  \pstart किञ्च (।)
	\pend
      
	  \bigskip
	  \begingroup
	  \large
	
	    
	    \stanza[\smallbreak]
	\label{pv.2.168}\edlabel{pv.2.168}\flagstanza{\tiny\textenglish{....2.168}}शब्दस्यान्वयिनः कार्यमर्थेनान्वयिना स च ।&अनन्वयी धियोऽभेदाद् दर्शनाभ्यासनिर्म्मतः ॥ १६८ ॥\&[\smallbreak]


	
	  \endgroup
	

	  \pstart {\color{DodgerBlue3}“शब्दस्या\edlabel{pvv.169-1}\footnote{\label{pvv.169-1}  १ गौगौ/?/रिति ।}न्वयिनो”}ऽन्वयिना{\color{DodgerBlue3}“र्थेन कार्यं”} व्यवहारकाले प्रतीतिलक्षणं प्रयोजनं । {\color{DodgerBlue3}“स च”} बुद्ध्याकारः स्वलक्षण{\color{DodgerBlue3}“दर्शनाभ्यासेन”} वासना{\color{DodgerBlue3}“निर्म्मितोऽन\edlabel{pvv.169-2}\footnote{\label{pvv.169-2}  २ यत्र बुद्धौ भासते ततोऽभिन्नः ज्ञानवत् ।}न्वयी धियो”}ऽनन्वयिन्या {\color{DodgerBlue3}“अभेदात्”} । (१६८)
	\pend
      \label{div_pvv.2.169}\edlabel{div_pvv.2.169}
	  
	% new div opening: depth here is 2
	

	  \pstart ननु यद्यर्थः शब्दस्य न विषयस्तदा तदं शरूपोप्यन्यापोहः कथं वाच्य इत्याह (।)
	\pend
      
	  \bigskip
	  \begingroup
	  \large
	
	    
	    \stanza[\smallbreak]
	\label{pv.2.169}\edlabel{pv.2.169}\flagstanza{\tiny\textenglish{....2.169}}तद्रूपारोपगत्यान्यव्यावृत्ताधिगतेः पुनः ।&शब्दार्थार्थः स एवेति वचने न विरुध्यते ॥ १६९ ॥\&[\smallbreak]


	
	  \endgroup
	

	  \pstart बुद्ध्या\edlabel{pvv.169-3}\footnote{\label{pvv.169-3}  ३ विनारोपं व्यवहाराभावात् यथा संगतिस्तस्य तथाह ।}कारे तद्रूपस्यापर्थांशापोहस्या{\color{DodgerBlue3}“रोपगत्या”} एकत्वाध्यवसायेना{\color{DodgerBlue3}“न्यव्यावृत्तस्या”}र्थ\edlabel{pvv.169-4}\footnote{\label{pvv.169-4}  ४ स्वलक्षणस्य ।}स्या{\color{DodgerBlue3}“धिगतेः शब्दार्थां”}शापोहः शब्दार्थ उच्यते । न तु सामान्याच्छब्दादर्थप्रतीतेः । यदि पुनर्बुद्ध्याकारस्य व्यावृत्तार्थत्वेन प्रतीतेः स बुद्ध्याकार एव शब्दार्थ इत्युपचारादुच्यते बुद्ध्याकारशब्दार्थवादिना (।) तदैवं {\color{DodgerBlue3}“वचने”} किञ्चिदपि {\color{DodgerBlue3}“न विरुध्यते”} । बुद्ध्याकारस्यान्वयिनः शब्दार्थत्वानिष्टेः । (१६९)
	\pend
      \label{div_pvv.2.170}\edlabel{div_pvv.2.170}
	  
	% new div opening: depth here is 2
	

	  \begin{center}%% label @type='head'
	\textbf{ख. अन्यापोहकृच्छब्दः}
	\end{center}
	
	  \bigskip
	  \begingroup
	  \large
	
	    
	    \stanza[\smallbreak]
	\label{pv.2.170a}\edlabel{pv.2.170a}\flagstanza{\tiny\textenglish{...2.170a}}मिथ्यावभासिनो वैते प्रत्ययाः शब्दनिर्म्मिताः ।\&[\smallbreak]


	
	  \endgroup
	

	  \pstart मिथ्या{\color{DodgerBlue3}“वभासिनो वा शब्दनिर्मिता एते प्रत्ययाः”} । तथा हि न तावदर्थः शब्दबुद्धेर्विषयः । तत्स्वरूपानवभासतः । तत्र शब्दसंकेताभावाच्च । नापि बुद्ध्याकारस्तस्य वेदनेपि विषयत्वेनानध्यवसायात् स्वलक्षणत्वात् संकेताभावाच्च । न हि बुद्ध्याकारस्य बहिष्ट्वं बाह्यस्य वा बुद्ध्याकरत्वमस्ति येन तथेति भासः सत्यप्रतिभासः स्यात् । तस्माद्वस्तुतोऽवस्तुप्रतिभासिनः शाब्दाः प्रत्ययाः ।
	\pend
      

	  \pstart कथन्तर्हीदानीमर्थांशापोहकृच्छ्रुतिरुक्तेत्याह\edlabel{pvv.169-5}\footnote{\label{pvv.169-5}  ५ अनिष्टं परित्यज्य इष्टे प्रवर्तनात् शब्दाः ।} (।)
	\pend
      
	  \bigskip
	  \begingroup
	  \large
	
	    
	    \stanza[\smallbreak]
	\label{pv.2.170b}\edlabel{pv.2.170b}\flagstanza{\tiny\textenglish{...2.170b}}अनुयान्तीममर्थांशमिति वापोहकृच्छ्रुतिः ॥ १७० ॥\&[\smallbreak]


	
	  \endgroup
	\leavevmode\marginnote{\textenglish{170/s}}

	  \pstart {\color{DodgerBlue3}“इमम”}न्यापोहम{\color{DodgerBlue3}“र्थांशं”} शब्दा अतत्प्रतिभासित्वेपि {\color{DodgerBlue3}“अनुयान्ति”} वृत्तिविषयत्वेन व्यवस्थापयन्ति । अर्थदर्शनायातत्वेन परस्परं या तत्प्रतिबन्धनादिति चान्या{\color{DodgerBlue3}“पोहकृच्छ्रुति”}रु\edlabel{pvv.170-1}\footnote{\label{pvv.170-1}  १ नैतेषु विषये प्रवर्तयेयुरिति (दिग्)नागेन ।}क्ता । (१७०)
	\pend
      \label{div_pvv.2.171}\edlabel{div_pvv.2.171}
	  
	% new div opening: depth here is 2
	
	  \bigskip
	  \begingroup
	  \large
	
	    
	    \stanza[\smallbreak]
	\label{pv.2.171a}\edlabel{pv.2.171a}\flagstanza{\tiny\textenglish{...2.171a}}तस्मात् संकेतकालेपि;\&[\smallbreak]


	
	  \endgroup
	\leavevmode\marginnote{\textenglish{33a/MA}}

	  \pstart यस्माद् व्यवहारकालेऽन्यव्यवच्छेदप्रतीतिः शब्दात् {\color{DodgerBlue3}“तस्मात्संकेतकालेप्य”}न्यापोहः श्रुतौ वाच्यतया सम्बध्यते नान्यत् ।
	\pend
      

	  \pstart नन्वर्थमुपदर्श्य संकेतः क्रियते तत्कथमपोह उच्यत इत्याह (।)
	\pend
      
	  \bigskip
	  \begingroup
	  \large
	
	    
	    \stanza[\smallbreak]
	\label{pv.2.171b}\edlabel{pv.2.171b}\flagstanza{\tiny\textenglish{...2.171b}}निर्दिष्टार्थेन संयुतः ।&स्वप्रतीतिफलेनान्यापोहः संबध्यते श्रुतौ ॥ १७१ ॥\&[\smallbreak]


	
	  \endgroup
	

	  \pstart {\color{DodgerBlue3}“निर्दिष्टेनार्थेना”}न्यव्यावृत्तेन व्यवहारकाले {\color{DodgerBlue3}“स्व”}स्य {\color{DodgerBlue3}“प्रतीतिः फलं”} प्रयोजनं यस्य तेन {\color{DodgerBlue3}“संयुतो”}ऽभेदाध्यवसायादेकत्वमुपनीतो{\color{DodgerBlue3}“ऽन्यापोहो”} बुद्ध्याकारस्वभावः {\color{DodgerBlue3}“श्रुतौ सम्बध्यते”} न त्वर्थ एव । (१७१)
	\pend
      \label{div_pvv.2.172}\edlabel{div_pvv.2.172}
	  
	% new div opening: depth here is 2
	

	  \pstart \edlabel{pvv.170-2}\footnote{\label{pvv.170-2}  २ विधिमुखेन सामान्यनिमित्तशब्दप्रवृत्तौ दोषमाह ।}तथा हि (।)
	\pend
      
	  \bigskip
	  \begingroup
	  \large
	
	    
	    \stanza[\smallbreak]
	\label{pv.2.172}\edlabel{pv.2.172}\flagstanza{\tiny\textenglish{....2.172}}अन्यात्रादृष्ट्यपेक्षत्वात् क्वचित्तद्दृष्ट्यपेक्षणात् ।&श्रुतौ संबध्यतेपोहो नैतद् वस्तुनि युज्यते ॥ १७२ ॥\&[\smallbreak]


	
	  \endgroup
	

	  \pstart संकेतस्या{\color{DodgerBlue3}“न्यत्र”} व्यवच्छेद्ये{\color{DodgerBlue3}“ऽवृक्षेऽदर्शनापेक्षत्वात् । क्वचि”}दव्यवच्छेद्ये वृक्षैकदेशे {\color{DodgerBlue3}“दृष्ट्यपेक्षणात् श्रुता”}वपोहः {\color{DodgerBlue3}“सम्बध्यत”} इति निश्चीयते । {\color{DodgerBlue3}“वस्तुनि”}सामान्यादौ संकेतविषये एतत् व्यवच्छेद्याव्यवच्छेद्ययोर्दर्शनादर्शनापेक्षणं {\color{DodgerBlue3}“न युज्यते”} ।\edlabel{pvv.170-3}\footnote{\label{pvv.170-3}  ३ तथा हि सामान्ये ।} वस्तुनि विधिमुखेन प्रतिपाद्ये किमन्यत्रादर्शनापेक्षया । अपेक्ष्यते च ततोऽन्यव्यवच्छेद एव प्रतिपाद्यत इति गम्यते । अन्यव्यवच्छेदः सामान्यादिकं चापेक्ष्यते विपक्षपरिहारेण प्रतिपत्त्यर्थमिति चेत् ।\edlabel{pvv.170-4}\footnote{\label{pvv.170-4}  ४ कल्पनानुविद्धार्थग्रहणं ।}अलं तदा सामान्येन । अन्यव्यवच्छेदेनैव व्यवहारपरिसमाप्तेः । (१७२)
	\pend
      \label{div_pvv.2.173}\edlabel{div_pvv.2.173}
	  
	% new div opening: depth here is 2
	

	  \pstart यतश्च जातिगुणक्रियादीनि विशेषणानि वस्तुग्राहिणि ज्ञाने नाभासन्ते (।)
	\pend
      
	  \bigskip
	  \begingroup
	  \large
	
	    
	    \stanza[\smallbreak]
	\label{pv.2.173}\edlabel{pv.2.173}\flagstanza{\tiny\textenglish{....2.173}}तस्माद् जात्यादितद्योगा नार्थे तेषु च न श्रुतिः ।&संयोज्यतेन्यव्यावृत्तौ शब्दानामेव योजनात् ॥ १७३ ॥\&[\smallbreak]


	
	  \endgroup
	

	  \pstart {\color{DodgerBlue3}“तस्माज्जात्यादयस्तेषां योगा\edlabel{pvv.170-5}\footnote{\label{pvv.170-5}  ५ सम्बन्धाः ।}श्चार्थे\edlabel{pvv.170-6}\footnote{\label{pvv.170-6}  ६ इन्द्रियविषये ।} न”} सन्ति । {\color{DodgerBlue3}“अत”}स्तेषु {\color{DodgerBlue3}“श्रुतिश्च नोप”}युज्यते । {\color{DodgerBlue3}“अन्यव्यावृत्तावेव”} प्रतीतिसिद्धायां {\color{DodgerBlue3}“शब्दानां योजनात्”} । (१७३)
	\pend
      \label{div_pvv.2.174}\edlabel{div_pvv.2.174}
	  
	% new div opening: depth here is 2
	\leavevmode\marginnote{\textenglish{171/s}}

	  \begin{center}%% label @type='head'
	\textbf{(६) क. प्रत्यक्षे शब्दकल्पनानिरासः}
	\end{center}
	

	  \pstart तदेवं जात्यादिकल्पना तत्सम्बन्धकल्पना च नास्तीत्युक्तं । शब्दकल्पनापि न सम्भवतीत्याह (।)
	\pend
      
	  \bigskip
	  \begingroup
	  \large
	
	    
	    \stanza[\smallbreak]
	\label{pv.2.174}\edlabel{pv.2.174}\flagstanza{\tiny\textenglish{....2.174}}संकेतस्मरणोपायं दृष्टसंकलनात्मकम् ।&पूर्वापरपरामर्शशून्ये तच्चाक्षुषे कथम् ॥ १७४ ॥\&[\smallbreak]


	
	  \endgroup
	

	  \pstart शब्दकल्पनं हि पूर्व्वगृहीतस्य {\color{DodgerBlue3}“संकेत”}स्य {\color{DodgerBlue3}“स्मरणमुपायो”} यस्य संकेतस्मरणोपायं वाचकत्वेन {\color{DodgerBlue3}“दृष्ट”}स्य शब्दस्य {\color{DodgerBlue3}“संकलनं”} तथायोजन\edlabel{pvv.171-1}\footnote{\label{pvv.171-1}  १ वर्त्तमानार्थेन ।} {\color{DodgerBlue3}“मात्मा”} यस्य तत् दृष्टसंकलनात्मकं प्रसिद्धं । तच्च {\color{DodgerBlue3}“पूर्व्व”}स्य संकेतकालदृष्टवाचकशब्दस्या{\color{DodgerBlue3}“परस्य”} दृश्यमानार्थस्य {\color{DodgerBlue3}“परामर्शो”} वाच्यवाचकतायोजनन्तेन शून्या शब्दामिश्रवस्तुस्वरूपग्राहिणि {\color{DodgerBlue3}“चाक्षुषे”} ज्ञाने {\color{DodgerBlue3}“कथं”} संभाव्यते । चाक्षुषं चाक्षजमात्रोपलक्षणमिन्द्रियज्ञानमित्यर्थः । (१७४)
	\pend
      \label{div_pvv.2.175}\edlabel{div_pvv.2.175}
	  
	% new div opening: depth here is 2
	

	  \pstart किञ्च (।)
	\pend
      
	  \bigskip
	  \begingroup
	  \large
	
	    
	    \stanza[\smallbreak]
	\label{pv.2.175}\edlabel{pv.2.175}\flagstanza{\tiny\textenglish{....2.175}}अन्यत्र गतचित्तोपि चक्षुषा रूपमीक्षते ।&तत्संकेताग्रहस्तत्र स्पष्टस्तज्जा च कल्पना ॥ १७५ ॥\&[\smallbreak]


	
	  \endgroup
	

	  \pstart दृश्यमानादर्थाद{\color{DodgerBlue3}“न्यत्रा”}तीतादौ विकल्पनीये {\color{DodgerBlue3}“गतचित्तः”} प्रवृत्तविकल्पोपि द्रष्टा {\color{DodgerBlue3}“चक्षुषा”} चक्षुर्व्विज्ञानेन {\color{DodgerBlue3}“रूपमीक्षते”} । तस्य दृश्यमानार्थस्य {\color{DodgerBlue3}“संकेतः”} संकेतविषयो वाचकं नाम तस्या{\color{DodgerBlue3}“ग्रहो”}ऽस्मरणं {\color{DodgerBlue3}“तत्र”} चाक्षुषे ज्ञाने {\color{DodgerBlue3}“स्पष्टः”} । तत{\color{DodgerBlue3}“स्तज्जा”} वाचकनामस्मरणप्रभवा {\color{DodgerBlue3}“कल्पना च”} चाक्षुषे ज्ञाने नास्तीति शेषः । (१७५)
	\pend
      \label{div_pvv.2.176}\edlabel{div_pvv.2.176}
	  
	% new div opening: depth here is 2
	

	  \pstart किञ्च (।)
	\pend
      
	  \bigskip
	  \begingroup
	  \large
	
	    
	    \stanza[\smallbreak]
	\label{pv.2.176}\edlabel{pv.2.176}\flagstanza{\tiny\textenglish{....2.176}}जायन्ते कल्पनास्तत्र यत्र शब्दो निवेशितः ।&तेनेच्छातः प्रवर्त्तेरन् नेक्षेरन् बाह्यमक्षजाः ॥ १७६ ॥\&[\smallbreak]


	
	  \endgroup
	

	  \pstart तत्र विषये शब्दयोजनात्मिकाः {\color{DodgerBlue3}“कल्पना जायन्ते । यत्र”} संकेतकाले {\color{DodgerBlue3}“शब्दो निवेशितः”} । न चेन्द्रियविषये शब्दसंकेत इति न तत् ज्ञानं शब्दयोजनात्मकं (प्राक्) । अथेन्द्रियविषय एव शब्दनिवेशस्तदा {\color{DodgerBlue3}“तेन”} शब्दविषयत्वेन कारणे{\color{DodgerBlue3}“नेच्छातः प्रवर्त्तेरन्नक्षजाः”} प्रत्यया विकल्पवत् । न चैतदस्ति । इच्छाप्रभवत्वे वा बाह्यार्थसंनिधानानपेक्षत्वात् । {\color{DodgerBlue3}“बाह्यमर्थं नेक्षेरन्नक्षजाः”} प्रत्ययविकल्पवत् । (१७६)
	\pend
      \label{div_pvv.2.177}\edlabel{div_pvv.2.177}
	  
	% new div opening: depth here is 2
	

	  \pstart अपि च (।)
	\pend
      
	  \bigskip
	  \begingroup
	  \large
	
	    
	    \stanza[\smallbreak]
	\label{pv.2.177}\edlabel{pv.2.177}\flagstanza{\tiny\textenglish{....2.177}}रूपं रूपमितीक्षेत तद्धियं किमितीक्षते ॥&अस्ति चानुभवस्तस्याःसोविकल्पः कथं भवेत् ॥ १७७ ॥\&[\smallbreak]


	
	  \endgroup
	\leavevmode\marginnote{\textenglish{172/s}}\leavevmode\marginnote{\textenglish{33b/MA}}

	  \pstart सर्व्वविकल्पवादिनो मते {\color{DodgerBlue3}“रूपमिति”} प्रवृत्तविकल्पबुद्धी {\color{DodgerBlue3}“रूपमीक्षेत ।\edlabel{pvv.172-1}\footnote{\label{pvv.172-1}  १ निर्व्विकल्प एवैतद्युक्तं प्राक् रूपबुद्धिस्ततोस्याविकल्पो नात्र ।} तद्धियं”} रूपधियमपि कल्प्यमानां {\color{DodgerBlue3}“किमितीक्षते”} ज्ञाता रूपबुद्ध्य{\color{DodgerBlue3}“नुभवो”} नास्तीति न युक्तं । यस्मादस्ति चानुभवस्तस्याः \edlabel{pvv.172-2}\footnote{\label{pvv.172-2}  २ यदि रूपग्रहणे बुद्धेरननुभवस्तदोत्तरा स्मृतिर्न स्यादस्ति च तदप्रत्यक्षत्वेऽर्थाप्रत्यक्षत्वाच्च ।}सर्व्वेषां प्रतिपत्तृणां । न च रूप इव तद्बुद्धावपि कल्पनाऽनुभूयते । ततश्च रूपबुद्धेरनुभवो{\color{DodgerBlue3}“ऽविकल्पः कथम्भवेत्\edlabel{pvv.172-3}\footnote{\label{pvv.172-3}  ३ त्वन्मतेन ।}”} । (१७७)
	\pend
      \label{div_pvv.2.178}\edlabel{div_pvv.2.178}
	  
	% new div opening: depth here is 2
	
	  \bigskip
	  \begingroup
	  \large
	
	    
	    \stanza[\smallbreak]
	\label{pv.2.178}\edlabel{pv.2.178}\flagstanza{\tiny\textenglish{....2.178}}तयैवानुभवे दृष्टं न विकल्पद्वयं सकृत् ।&एतेन तुल्यकालान्यविज्ञानानुभवो गतः ॥ १७८ ॥\&[\smallbreak]


	
	  \endgroup
	

	  \pstart {\color{DodgerBlue3}“तयै”}व रूपबुद्ध्या रूपस्य स्वात्मनश्चाऽ{\color{DodgerBlue3}“नुभवे”}भ्युपगम्यमाने रूपमिति रूपानुभव इति च {\color{DodgerBlue3}“विकल्पद्वयं सकृत्”} स्यात् । तच्च नास्त्य\edlabel{pvv.172-4}\footnote{\label{pvv.172-4}  ४ अनुपलब्धेः ।}नुभववाधितत्वात् । {\color{DodgerBlue3}“एतेन”} सकृत्कल्पनाद्वयनिषेधेन {\color{DodgerBlue3}“तुल्यकालेनान्येन”}\edlabel{pvv.172-5}\footnote{\label{pvv.172-5}  ५ इन्द्रियज्ञानकालेन विकल्पेन रूपबुद्ध्यनुभवो निरस्तः इति वृत्तिः ।} निर्व्विकल्पज्ञानेना{\color{DodgerBlue3}“नुभवो”} रूपबु्द्धे{\color{DodgerBlue3}“र्गतो”} निर्ण्णीतोत्तरो बोद्धव्यः । (१७८)
	\pend
      \label{div_pvv.2.179}\edlabel{div_pvv.2.179}
	  
	% new div opening: depth here is 2
	

	  \begin{center}%% label @type='head'
	\textbf{ख. उत्तरकालभाविविकल्पज्ञानेनानुभवः}
	\end{center}
	

	  \pstart उत्तरकालभाविना विकल्पज्ञानेनानुभव इति चेदाह (।)
	\pend
      
	  \bigskip
	  \begingroup
	  \large
	
	    
	    \stanza[\smallbreak]
	\label{pv.2.179}\edlabel{pv.2.179}\flagstanza{\tiny\textenglish{....2.179}}स्मृतिर्भवेदतीते च साऽगृहीते कथं भवेत् ।&स्याच्चान्यधीपरिच्छेदाभिन्नरूपा स्वबुद्धिधीः ॥ १७९ ॥\&[\smallbreak]


	
	  \endgroup
	

	  \pstart {\color{DodgerBlue3}“अतीते च”} रूपानुभवे {\color{DodgerBlue3}“स्मृतिः”} पश्चात्तनेन विकल्पेन न भवेन्नानुभवः । {\color{DodgerBlue3}“सा”} स्मृतिर{\color{DodgerBlue3}“गृहीते”}ऽनुभवे {\color{DodgerBlue3}“कथम्भवेत्”} । यदि चातीतबुद्धिर्व्विकल्प्यते तदाऽन्यस्य पुंसो धियः {\color{DodgerBlue3}“परिच्छेदेन”} परोक्षबुद्धिविकल्पात्मकेना{\color{DodgerBlue3}“भिन्नरूपा”} तथात्वेन {\color{DodgerBlue3}“स्वबुद्धिधीः”} स्यात् । अस्ति च परबुद्धिप्रतीतिविलक्षणस्वबुद्ध्यनुभवः । तस्मादविकल्प एवासौ । (१७९)
	\pend
      \label{div_pvv.2.180}\edlabel{div_pvv.2.180}
	  
	% new div opening: depth here is 2
	

	  \pstart किञ्च (।)
	\pend
      
	  \bigskip
	  \begingroup
	  \large
	
	    
	    \stanza[\smallbreak]
	\label{pv.2.180}\edlabel{pv.2.180}\flagstanza{\tiny\textenglish{....2.180}}अतीतमपदृष्टान्तमलिङ्गञ्चार्थवेदनम् ।&सिद्धं तत्केन तस्मिन् हि न प्रत्यक्षं न लैङ्गिकम् ॥ १८० ॥\&[\smallbreak]


	
	  \endgroup
	

	  \pstart सविकल्पकप्रत्यक्षवादिना निर्व्विकल्पस्याप्यस्वसंवेदनवा\edlabel{pvv.172-6}\footnote{\label{pvv.172-6}  ६ वैशेषिकस्य ।}दिनो मतेऽ{\color{DodgerBlue3}“तीतमर्थवेदनं”} न केवलमध्यक्षतो वर्तमानविषय\edlabel{pvv.172-7}\footnote{\label{pvv.172-7}  ७ हेतुः ।}त्वान्न सिध्यति । किन्त्वनुमानादपि यस्माद{\color{DodgerBlue3}“लिङ्गं”} लिङ्गरहितं । तथा हि धर्मिणो ज्ञानस्यासिद्धत्वात् लिङ्गमाश्रया\leavevmode\marginnote{\textenglish{173/s}} {\color{DodgerBlue3}“सिद्धं । अनुमानात् ज्ञानसिद्धिवादिनः कस्यचिज्ज्ञानस्याध्यक्षासिद्धत्वात्”} अनुमानसिद्धावनवस्थानात् {\color{DodgerBlue3}“अपदृष्टान्तं”} । दृष्टान्तासिद्धौ न {\color{DodgerBlue3}“व्याप्तिसिद्धिरिति”} लिङ्गरहितमेवातीतं रूपादिदर्शनं । {\color{DodgerBlue3}“तत्तस्मात्केन प्रमाणेन सिद्धं तस्मिन् हि”} रूपादिदर्शने {\color{DodgerBlue3}“न प्रत्यक्ष”}मभिमतत्वादस्ति । {\color{DodgerBlue3}“न च लैङ्गिकमनुमानमुक्तक्रमादिति”} सकलमप्रतिपत्तिकमन्धमू\edlabel{pvv.173-1}\footnote{\label{pvv.173-1}  १ प्रत्यक्षासिद्ध्या । वचनाभावात् अननुभूतेनाविकल्प्य वचनं ।}कं जगत्प्राप्तमिति ॥ (१८०)
	\pend
      \label{div_pvv.2.181_2.182_2.183_2.184}\edlabel{div_pvv.2.181_2.182_2.183_2.184}
	  
	% new div opening: depth here is 2
	

	  \pstart अथ (।)
	\pend
      
	  \bigskip
	  \begingroup
	  \large
	
	    
	    \stanza[\smallbreak]
	\label{pv.2.181a}\edlabel{pv.2.181a}\flagstanza{\tiny\textenglish{...2.181a}}तत्स्वरूपावभासिन्या बुद्ध्यानन्तरया यदि ।&रूपादिरिव गृह्येत;\&[\smallbreak]


	
	  \endgroup
	

	  \pstart {\color{DodgerBlue3}“तत्स्वरूपावभासिन्या”}ऽतीतरूपादिबुद्धिरूपप्रतिभासिन्या {\color{DodgerBlue3}“तज्जन्ययाऽनन्तया”} धियाऽतीतबुद्धि{\color{DodgerBlue3}“र्यदि गृह्यते”} सौ त्रा न्ति क मते {\color{DodgerBlue3}“रूपादिरिव”} तदनुकारिण्या {\color{DodgerBlue3}“तदनन्तरया”} धिया तदा को {\color{DodgerBlue3}“दोष”} इत्याह (।)
	\pend
      

	  \pstart यदा चिरं बहुषु विषयेषु ज्ञा\edlabel{pvv.173-2}\footnote{\label{pvv.173-2}  २ इन्द्रियज्ञानान्यसन्तानेन ।}नानि प्रवर्तन्ते तदा (।)
	\pend
      
	  \bigskip
	  \begingroup
	  \large
	
	    
	    \stanza[\smallbreak]
	\label{pv.2.181b}\edlabel{pv.2.181b}\flagstanza{\tiny\textenglish{...2.181b}}न स्यात् तत्पूर्वधीग्रहः ॥ १८१ ॥\&[\smallbreak]


	
	  \endgroup
	
	  \bigskip
	  \begingroup
	  \large
	
	    
	    \stanza[\smallbreak]
	\label{pv.2.182}\edlabel{pv.2.182}\flagstanza{\tiny\textenglish{....2.182}}सोविकल्पः स्वविषयो विज्ञानानुभवो यथा ।&अशक्यसमयं तद्वदन्यदप्यविकल्पकम् ॥ १८२ ॥\&[\smallbreak]


	
	  \endgroup
	
	  \bigskip
	  \begingroup
	  \large
	
	    
	    \stanza[\smallbreak]
	\label{pv.2.183}\edlabel{pv.2.183}\flagstanza{\tiny\textenglish{....2.183}}सामान्यवाचिनः शब्दास्तदेकार्था च कल्पना ।&अभावे निर्विकल्पस्य विशेषाधिगमः कथम् ॥ १८३ ॥\&[\smallbreak]


	
	  \endgroup
	
	  \bigskip
	  \begingroup
	  \large
	
	    
	    \stanza[\smallbreak]
	\label{pv.2.184}\edlabel{pv.2.184}\flagstanza{\tiny\textenglish{....2.184}}अस्ति चेन्निर्विकल्पञ्च किञ्चित्तत्तुल्यहेतुकम् ।&सर्वं तथैव हेतोर्हि भेदाद् भेदः फलात्मनाम् ॥ १८४ ॥\&[\smallbreak]


	
	  \endgroup
	

	  \pstart तस्मादन्त्यात् ज्ञानात् याः {\color{DodgerBlue3}“पूर्व्वा”} धियस्तासां {\color{DodgerBlue3}“ग्रहो न स्यादि”}ति दोषः । अन्त्यबुद्धिजनितया हि धिया सैव गृह्यते न त्वन्या इति स्यात् । अस्ति चानुभवस्तासां यद्वलेन चिरमहमद्राक्षमिति भवति द्रष्टुः । तस्मा{\color{DodgerBlue3}“त्स्वविषयः”} स्वरूपालम्बनो {\color{DodgerBlue3}“विज्ञाना”}नां पूर्व्वभाविना{\color{DodgerBlue3}“मनुभवोऽविकल्पः”} । स यथा {\color{DodgerBlue3}“तद्वदन्यदपि”} ज्ञानमन्त्यमप्रतिबद्धवृत्ति {\color{DodgerBlue3}“चाविकल्पकं”} बोद्धव्यं । यस्मात्सकलमेव स्वरूप{\color{DodgerBlue3}“मशक्यसमयं”} शब्दसंकेताविषयः । ततश्च न विकल्पग्राह्यं (।) किञ्च विशेषसंकेताभावात् व्यवहारकालानुयायित्वाच्च {\color{DodgerBlue3}“सामान्यवाचिनः शब्दास्तैः”} शब्दै{\color{DodgerBlue3}“रेकार्था”} एकविषया {\color{DodgerBlue3}“च कल्पना”} शब्दयोजनया शब्दार्थ एव कल्पना । परमते च {\color{DodgerBlue3}“निर्व्विकलपस्य”} ज्ञानस्या{\color{DodgerBlue3}“भावे”} विशेषस्य {\color{DodgerBlue3}“विकल्पा”}विषयस्या{\color{DodgerBlue3}“धिगमः कथं”} न कथञ्चि-\leavevmode\marginnote{\textenglish{34a/MA}} दित्यर्थः । विशेषानुभवदर्शना{\color{DodgerBlue3}“दस्ति किञ्चिन्निर्व्विकल्पं च”} ज्ञानं यथाहुर्मी मां स \leavevmode\marginnote{\textenglish{174/s}} {\color{DodgerBlue3}“कादय इति चेत्”} । एवन्तर्हि तेन निर्व्विकल्पेन {\color{DodgerBlue3}“तुल्यहेतुकं”} चक्षूरूपमनस्कारादिसमानहेतुकं विशेषविषयं {\color{DodgerBlue3}“सर्व्वं”} ज्ञानं {\color{DodgerBlue3}“तथैवा”}विकल्पकमस्तु (।) न तु स्वलक्षणविषयमपि किञ्चित्सविकल्पकं । हिर्यस्मा{\color{DodgerBlue3}“द्धतोर्भेदात्फलात्मनां भेदो”} भवति । हेत्वभेदे तु फलाभेद एव युक्तः । नान्यथा क्वचिदप्येकजातीयता स्यात् ॥\edlabel{pvv.174-1}\footnote{\label{pvv.174-1}  १ “प्रत्यक्षं कल्पनापोढमि” त्यादि “यत्रैषा कल्पना नास्ती”त्यन्तः \href{http://http://sarit.indology.info/?cref=psv.1.3}{[प्रमाण] समुच्चयो व्याख्यातः} ।}(१८१-८४)
	\pend
      \label{div_pvv.2.185}\edlabel{div_pvv.2.185}
	  
	% new div opening: depth here is 2
	

	  \pstart किञ्च (।)
	\pend
      
	  \bigskip
	  \begingroup
	  \large
	
	    
	    \stanza[\smallbreak]
	\label{pv.2.185}\edlabel{pv.2.185}\flagstanza{\tiny\textenglish{....2.185}}अनपेक्षितबाह्यार्था योजना समयस्मृतेः ।&तथानपेक्ष्य समयं वस्तुशक्त्यैव नेत्रधीः ॥ १८५ ॥\&[\smallbreak]


	
	  \endgroup
	

	  \pstart {\color{DodgerBlue3}“योज\edlabel{pvv.174-2}\footnote{\label{pvv.174-2}  २ शब्दार्थयोः ।}ना”} कल्पनाऽन{\color{DodgerBlue3}“पेक्षितबाह्यार्था”} बहिरर्थसंकेतविषयमनपेक्ष्यैव समयस्य प्राग्गृहीतस्य {\color{DodgerBlue3}“स्मृतेः”} सकाशाद् भवति तावत् । {\color{DodgerBlue3}“तथा समयमनपेक्ष्य वस्तुनः”} स्वलक्षणस्य {\color{DodgerBlue3}“शक्त्या”} स्वाकारानुकारिविज्ञानजननसामर्थ्येनैव {\color{DodgerBlue3}“नेत्रधीर्जा”}यते {\color{DodgerBlue3}“यदि”} तदा को विरोधः । (१८५)
	\pend
      \label{div_pvv.2.186}\edlabel{div_pvv.2.186}
	  
	% new div opening: depth here is 2
	

	  \pstart स्यादेतद् (।)
	\pend
      
	  \bigskip
	  \begingroup
	  \large
	
	    
	    \stanza[\smallbreak]
	\label{pv.2.186}\edlabel{pv.2.186}\flagstanza{\tiny\textenglish{....2.186}}संकेतस्मरणापेक्ष रूपं यद्यक्षचेतसि ।&अनपेक्ष्य न चेच्छक्तं स्यात् स्मृतावेव लिङ्गवत् ॥ १८६ ॥\&[\smallbreak]


	
	  \endgroup
	

	  \pstart रूपम{\color{DodgerBlue3}“क्षचेतसि”} कर्तव्ये {\color{DodgerBlue3}“संकेतस्मरणापेक्षं”} तदनपेक्षं पुनर्नाशक्तमिति एवं तर्हि श्रुतावेव रूपं शक्तमिति स्यात् । न त्विन्द्रियबुद्धौ {\color{DodgerBlue3}“लिङ्गवत्”} । यथा हि लिंगं {\color{DodgerBlue3}“न”} लिंगबुद्धौ साक्षाच्छक्तं किन्तु लिङ्गलिङ्गिनोः सम्म्बन्धिस्मृतावेव । तथा संकेतस्मरणे रूपं निमित्तं स्यात् । न चागृहीतं स्मृतिप्रतिबोधकमिति निर्व्विकल्पकमस्य ग्रहणं प्राक् ततः स्मृतिः । ततश्च योजनेति क्रमः । (१८६)
	\pend
      \label{div_pvv.2.187}\edlabel{div_pvv.2.187}
	  
	% new div opening: depth here is 2
	

	  \pstart कथं पुनरर्थसम्मुखीभावात् स्मृतिजन्मेत्याह (।)
	\pend
      
	  \bigskip
	  \begingroup
	  \large
	
	    
	    \stanza[\smallbreak]
	\label{pv.2.187}\edlabel{pv.2.187}\flagstanza{\tiny\textenglish{....2.187}}तस्यास्तत्सङ्गमोत्पत्तेरक्षधीः स्यात् स्मृतेर्न्न वा ।&ततः कालान्तरेपि स्यात् क्वचिद् व्याक्षेपसम्भवात् ॥ १८७ ॥\&[\smallbreak]


	
	  \endgroup
	

	  \pstart {\color{DodgerBlue3}“तस्याः स्मृतेस्त”}स्यार्थस्य {\color{DodgerBlue3}“संगमेन”} सम्मुखीभावे{\color{DodgerBlue3}“नोत्पत्तेः\edlabel{pvv.174-3}\footnote{\label{pvv.174-3}  ३ अर्थात्सैव स्मृतिः स्यात् ।} । न त्वक्षधी”}रर्था{\color{DodgerBlue3}“त्स्यात्”} । सा तु स्मृतेरर्थजनितायाः स्यात् । न वा स्मृतेरपि भवेत् स्मृत्यधीनतायां नार्थाधीनता । यच्च स्मरणं भावि तन्नावश्यं\edlabel{pvv.174-4}\footnote{\label{pvv.174-4}  ४ इच्छावशात् ।} भवतीति कदाचिन्न भवेदपि । {\color{DodgerBlue3}“ततः”} स्मृतेः {\color{DodgerBlue3}“कालान्तरेणापि स्याद”}ध्यक्षधीः स्मृत्यनन्तरं {\color{DodgerBlue3}“क्वचि”}द्विषयान्तरे {\color{DodgerBlue3}“व्याक्षेपस्या-”} शक्तिलक्षणस्य {\color{DodgerBlue3}“संभवात्”} । तन्निवृत्तौ सत्यां क्रमेण भवेत् । (१८७)
	\pend
      \label{div_pvv.2.188}\edlabel{div_pvv.2.188}
	  
	% new div opening: depth here is 2
	\leavevmode\marginnote{\textenglish{175/s}}

	  \pstart स्यादेतत्प्रथममुभिमुखीभवन्नर्थः स्मृ\edlabel{pvv.175-1}\footnote{\label{pvv.175-1}  १ इन्द्रियज्ञानहेतोः ।}तेर्हेतुस्तत इन्द्रियज्ञानस्येति (।)
	\pend
      
	  \bigskip
	  \begingroup
	  \large
	
	    
	    \stanza[\smallbreak]
	\label{pv.2.188}\edlabel{pv.2.188}\flagstanza{\tiny\textenglish{....2.188}}क्रमेणोभयहेतुश्चेत् प्रागेव स्यादभेदतः ॥&अन्योक्षबुद्धिहेतुश्चेत् स्मृतिस्तत्राप्यनर्थिका ॥ १८८ ॥\&[\smallbreak]


	
	  \endgroup
	

	  \pstart {\color{DodgerBlue3}“क्रमेणोभयहेतु”}रभिमत{\color{DodgerBlue3}“श्चेत्”} । यद्येवं पूर्व्वापरै{\color{DodgerBlue3}“कस्वभावसामर्थ्यस्य”} भावस्या{\color{DodgerBlue3}“भेदत-”} स्तज्जन्यं द्वयमपि {\color{DodgerBlue3}“प्रागेव स्यात्”} न क्रमतः ॥\edlabel{pvv.175-2}\footnote{\label{pvv.175-2}  २ अतत्स्वभावत्वेन वा पश्चादपि अक्षणिकपक्षे ।} स्यादेतद्(।) भावानां क्षणिकत्वाच्चान्यः स्मृतिप्रबोधकः क्षणोऽन्यश्चाक्षबुद्धेर्हेतुश्चेत् । {\color{DodgerBlue3}“तत्राद्येपि”} क्षणे वाचक{\color{DodgerBlue3}“शब्दस्मृतिरनर्थिका”} (। १८८)
	\pend
      \label{div_pvv.2.189}\edlabel{div_pvv.2.189}
	  
	% new div opening: depth here is 2
	

	  \pstart यस्माद् (।)
	\pend
      
	  \bigskip
	  \begingroup
	  \large
	
	    
	    \stanza[\smallbreak]
	\label{pv.2.189}\edlabel{pv.2.189}\flagstanza{\tiny\textenglish{....2.189}}यथा समितसिध्यर्थमिष्यते समयस्मृतिः ।&भेदश्चासमितो ग्राह्यः स्मृतिस्तत्र किमर्थिका ॥ १८९ ॥\&[\smallbreak]


	
	  \endgroup
	

	  \pstart {\color{DodgerBlue3}“यथा समित”}स्य शब्दवाच्यतयाऽ\edlabel{pvv.175-3}\footnote{\label{pvv.175-3}  ३ स्मृतिविषयस्याक्षज्ञानेन ग्रहार्थं हि स्मृतिः ।}र्थस्य {\color{DodgerBlue3}“सिद्ध्यर्थं समयस्मृतिरिष्यते ।\edlabel{pvv.175-4}\footnote{\label{pvv.175-4}  ४ यत्र स्मृतिविषयादन्यो ।} भेदो”} विशेषो{\color{DodgerBlue3}“ऽसमितः”} संकेताविषयश्चाक्षधिया {\color{DodgerBlue3}“ग्राह्यः”} । तत्र स्मृतिः समयस्य {\color{DodgerBlue3}“किमर्थिका”} निष्प्रयोजना (। १८९)
	\pend
      \label{div_pvv.2.190}\edlabel{div_pvv.2.190}
	  
	% new div opening: depth here is 2
	

	  \pstart सामान्ये कालान्तरानुवर्त्तिनि संकेतः स एव स्मर्यत इति चेदाह (।) न (।)
	\pend
      
	  \bigskip
	  \begingroup
	  \large
	
	    
	    \stanza[\smallbreak]
	\label{pv.2.190}\edlabel{pv.2.190}\flagstanza{\tiny\textenglish{....2.190}}सामान्यमात्रग्रहणे भेदापेक्षा न युज्यते ॥&तस्माच्चक्षुश्च रूपञ्च प्रतीत्योदेति नेत्रधीः ॥ १९० ॥\&[\smallbreak]


	
	  \endgroup
	

	  \pstart {\color{DodgerBlue3}“सामान्यमात्रस्य ग्रहणे”}ऽभ्युपगम्यमाने {\color{DodgerBlue3}“भेद”}स्य विशेषस्य संकेतविषयस्या{\color{DodgerBlue3}“पेक्षा”} न युज्यते यथा गौरित्युक्ते कीदृशो गौरिति । तस्मात्सामान्यवति विशेषे संकेतस्तेनैवार्थित्वाद् व्यवहारिणां । तथा च भेदश्चासमितो ग्राह्य इत्युक्तं । {\color{DodgerBlue3}“तस्माच्चक्षुश्च रूपञ्च\edlabel{pvv.175-5}\footnote{\label{pvv.175-5}  ५ अथ कस्माद् द्वयाधीनायामुत्पत्तौ प्रत्यक्षमुच्यते न प्रतिविषयमिति[प्रमाण] समुच्चयं व्याख्यातुमुपक्रमते ॥} प्रतीत्यासाद्योदेति नेत्रधी”}रित्यभ्युपगन्तव्यं ॥ (१९०)
	\pend
      \label{div_pvv.2.191ab}\edlabel{div_pvv.2.191ab}
	  
	% new div opening: depth here is 2
	

	  \pstart व्यवधानेनापि रूपकारणता स्यादिति चेत् । आह (।)
	\pend
      
	  \bigskip
	  \begingroup
	  \large
	
	    
	    \stanza[\smallbreak]
	\label{pv.2.191a}\edlabel{pv.2.191a}\flagstanza{\tiny\textenglish{...2.191a}}साक्षाच्चेत् ज्ञानजनने समर्थो विषयोक्षवत् ।\&[\smallbreak]


	
	  \endgroup
	

	  \pstart {\color{DodgerBlue3}“साक्षाच्च विषयो”} रूपादिः स्वग्राहक{\color{DodgerBlue3}“ज्ञानजनने समर्थोऽक्षवत्”} । न व्यवधानेन । स्मृत्यधीनतायां दोषस्योक्तत्वात् (।) तस्मादशब्दसंसृ/?/ष्टार्थबलभावितद्रूपानुकारि\leavevmode\marginnote{\textenglish{34b/MA}} प्रत्यक्षमनाविष्टाभिलापमविकल्पकमेव युक्तं ॥ X X ॥
	\pend
      
	  
	% new div opening: depth here is 1
	
\section[{(६. प्रत्यक्षभेदाः)}]{(६. प्रत्यक्षभेदाः)}

	  \begin{center}%% label @type='head'
	\textbf{(१) इन्द्रियप्रत्यक्षम्}
	\end{center}
	\label{div_pvv.2.191cd}\edlabel{div_pvv.2.191cd}
	  
	% new div opening: depth here is 2
	\leavevmode\marginnote{\textenglish{176/s}}
	  \bigskip
	  \begingroup
	  \large
	
	    
	    \stanza[\smallbreak]
	\label{pv.2.191b}\edlabel{pv.2.191b}\flagstanza{\tiny\textenglish{...2.191b}}अथ कस्माद् द्वयाधीनजन्म तत्तेन नोच्यते ॥ १९१ ॥\&[\smallbreak]


	
	  \endgroup
	

	  \pstart {\color{DodgerBlue3}“अथ द्वयाधीनजन्म”}विषयेन्द्रियोत्पत्ति{\color{DodgerBlue3}“तदि”}न्द्रियज्ञानमिन्द्रियेणो{\color{DodgerBlue3}“च्यते”} व्यपदिश्यते प्रत्यक्षमिति प्रतिगतमक्षम्प्रत्यक्षमिन्द्रियाश्रितमित्यर्थः ः (।) कस्मात्पुनर्व्विषयेण {\color{DodgerBlue3}“नोच्यते”} प्रतिविषयमिति ॥ (१९१)
	\pend
      \label{div_pvv.2.192}\edlabel{div_pvv.2.192}
	  
	% new div opening: depth here is 2
	

	  \pstart न खलु व्यसनितया व्यपदेशो नियुज्यते । अपि तु (।)
	\pend
      
	  \bigskip
	  \begingroup
	  \large
	
	    
	    \stanza[\smallbreak]
	\label{pv.2.192}\edlabel{pv.2.192}\flagstanza{\tiny\textenglish{....2.192}}समीक्ष्य गमकत्वं हि व्यपदेशो न गृह्यते ।&तच्चाक्षव्यपदेशेस्ति तद्धर्मश्च नियोज्यताम् ॥ १९२ ॥\&[\smallbreak]


	
	  \endgroup
	

	  \pstart {\color{DodgerBlue3}“गमकत्वं समीक्ष्य”} परिभाव्य । {\color{DodgerBlue3}“तच्च गमकत्वमक्षेण व्यपदेशे”} प्रत्यक्षमित्यत्रास्ति\edlabel{pvv.176-1}\footnote{\label{pvv.176-1}  १ रूपशब्दादेः ।} तस्य गमकत्वस्य व्यापकस्य {\color{DodgerBlue3}“धर्म्मो”} व्याप्यभूतो {\color{DodgerBlue3}“नियोज्यतां ।”}\edlabel{pvv.176-2}\footnote{\label{pvv.176-2}  २ अर्हता ।}(१९२)
	\pend
      \label{div_pvv.2.193}\edlabel{div_pvv.2.193}
	  
	% new div opening: depth here is 2
	
	  \bigskip
	  \begingroup
	  \large
	
	    
	    \stanza[\smallbreak]
	\label{pv.2.193}\edlabel{pv.2.193}\flagstanza{\tiny\textenglish{....2.193}}ततो लिङ्गस्वभावोत्र व्यपदेशे नियोज्यताम् ।&निवर्त्तते व्यापकस्य स्वभावस्य निवृत्तितः ॥ १९३ ॥\&[\smallbreak]


	
	  \endgroup
	

	  \pstart {\color{DodgerBlue3}“ततो”} व्याप\edlabel{pvv.176-3}\footnote{\label{pvv.176-3}  ३ पूर्व्वार्द्धेनान्वयो द्वितीयेन व्यतिरेकः ।}काभावात् {\color{DodgerBlue3}“व्यपदेशे”} धर्म्मिणि {\color{DodgerBlue3}“अत्र”} गमकत्वे साध्ये {\color{DodgerBlue3}“नियोज्यतां”} लिङ्गं । प्रतिविषयमिति व्यपदेशात् {\color{DodgerBlue3}“व्यापकस्य”} गमकत्वस्य {\color{DodgerBlue3}“निवृत्तितो निवर्तते”} नियोज्यतेति व्यापकानुपलब्ध्या तत्र नियोज्यत्वाभावः सिद्धः । (१९३)
	\pend
      \label{div_pvv.2.194}\edlabel{div_pvv.2.194}
	  
	% new div opening: depth here is 2
	

	  \begin{center}%% label @type='head'
	\textbf{क. अक्षाणां गमकत्वात् प्रत्यक्षम्}
	\end{center}
	
	  \bigskip
	  \begingroup
	  \large
	
	    
	    \stanza[\smallbreak]
	\label{pv.2.194}\edlabel{pv.2.194}\flagstanza{\tiny\textenglish{....2.194}}सञ्चितः समुदायः स सामान्यं तत्र चाक्षधीः ।&सामान्यबुद्धिश्चावश्यं विकल्पेनानुबध्यते ॥ १९४ ॥\&[\smallbreak]


	
	  \endgroup
	

	  \pstart ननु सञ्चितालम्बनाः पञ्च विज्ञानकाया इति सिद्धान्तः । “तत्रा\edlabel{pvv.176-4}\footnote{\label{pvv.176-4}  ४ यच्च वसुबन्धुनोक्तं । आयतनस्वलक्षणं चक्षुर्ग्राह्यत्वादि तत्प्रतिज्ञानानि स्वलक्षणविषयाणि, न द्रव्यं स्वलक्षणं प्रति । एकपरमाणु ।}नेकार्थजन्यत्वात् स्वार्थे सामान्यगोतर” मिति चोक्तं\edlabel{pvv.176-5}\footnote{\label{pvv.176-5}  ५ \href{http://http://sarit.indology.info/?cref=ps.1.4cd}{[प्रमाण] समुच्चये} ।} । तथा च परमाणूनां {\color{DodgerBlue3}“समुदा\edlabel{pvv.176-6}\footnote{\label{pvv.176-6}  ६ रूपशब्दादेः अष्टद्रव्यत्वात् ।}यः”} \leavevmode\marginnote{\textenglish{177/s}} सञ्चित इत्युच्यते । {\color{DodgerBlue3}“स”} एव च सामान्ये मतः {\color{DodgerBlue3}“तत्र च”} सामान्येऽ{\color{DodgerBlue3}“क्षधी”}र्ज्जायते (।) {\color{DodgerBlue3}“सामान्यबुद्धिश्चावश्यं विक\edlabel{pvv.177-1}\footnote{\label{pvv.177-1}  १ सामान्यविषयाऽक्षधीः सविकल्पा परस्य ।}ल्पेनानुबध्यते”} अनुसीव्यते । (१९४)
	\pend
      \label{div_pvv.2.195}\edlabel{div_pvv.2.195}
	  
	% new div opening: depth here is 2
	

	  \pstart तत्कथ\edlabel{pvv.177-2}\footnote{\label{pvv.177-2}  २ यदि रूपशब्दादिसमुदायालम्बना अपि पञ्च विज्ञानकायाः कथमेषां स्वलक्षणविषयत्वं न व्याहृतं कल्पनापोहत्वञ्च ।}मविकल्पं प्रत्यक्षमुच्यते ॥
	\pend
      

	  \pstart अत्रा\edlabel{pvv.177-3}\footnote{\label{pvv.177-3}  ३ पूर्व्वपक्षद्वये बौद्धः ।}ह (।)
	\pend
      
	  \bigskip
	  \begingroup
	  \large
	
	    
	    \stanza[\smallbreak]
	\label{pv.2.195}\edlabel{pv.2.195}\flagstanza{\tiny\textenglish{....2.195}}अर्थान्तराभिसम्बन्धाज्जायन्ते येऽणवोऽपरे ।&उक्तास्ते सञ्चितास्ते हि निमित्तं ज्ञानजन्मनः ॥ १९५ ॥\&[\smallbreak]


	
	  \endgroup
	

	  \pstart {\color{DodgerBlue3}“अर्थान्तराणां परमाण्वन्तराणामभिसम्बन्धात्\edlabel{pvv.177-4}\footnote{\label{pvv.177-4}  ४ विज्ञानजननसमर्थस्वभावोत्पादनप्रत्ययसन्निधानात् ।}”} सन्निधानविशेषेणोपसर्पणप्रत्ययेभ्यः पूर्व्वकेभ्यः\edlabel{pvv.177-5}\footnote{\label{pvv.177-5}  ५ असमर्थेभ्यः ।}परमसन्निहितेभ्योऽपरेन्ये{\color{DodgerBlue3}“येऽण\edlabel{pvv.177-6}\footnote{\label{pvv.177-6}  ६ समर्थाः प्रत्येकं । नान्यदेव सामान्यं ।}वो जायन्ते”} ते सञ्चिता {\color{DodgerBlue3}“उक्ताः सञ्चि”}तालम्बना विज्ञानकाय इत्यादौ । {\color{DodgerBlue3}“ज्ञानजन्म\edlabel{pvv.177-7}\footnote{\label{pvv.177-7}  ७ द्वितीयं परिहरति ।}”}नस्त एव हि {\color{DodgerBlue3}“निमित्त”}मुक्ताः तत्रानेकार्थजन्यत्वादित्यादिना । (१९५)
	\pend
      \label{div_pvv.2.196}\edlabel{div_pvv.2.196}
	  
	% new div opening: depth here is 2
	
	  \bigskip
	  \begingroup
	  \large
	
	    
	    \stanza[\smallbreak]
	\label{pv.2.196}\edlabel{pv.2.196}\flagstanza{\tiny\textenglish{....2.196}}अणूनां स विशेषश्च नान्तरेणापरानणून ।&तदेकानियमाज्ज्ञानमुक्तं सामान्यगोचरम् ॥ १९६ ॥\&[\smallbreak]


	
	  \endgroup
	

	  \pstart {\color{DodgerBlue3}“अणूनां”} स च ज्ञानजननसामर्थ्यलक्षणो {\color{DodgerBlue3}“विशेषोऽपरानणून”}व्यवधानवर्त्तिनो\edlabel{pvv.177-8}\footnote{\label{pvv.177-8}  ८ सर्वेषां तत्साधारणं कार्यमित्यर्थः । न पुनरायतनसामान्यस्य ग्रहणात् ।}ऽन्तरेण विना न भवति । न हि प्रत्येकमणवो दृश्याः किं तु सहिता एव । {\color{DodgerBlue3}“तत्तस्मादेक”}स्मिन्नर्थे परमाणौ ज्ञानस्या{\color{DodgerBlue3}“नियमात्\edlabel{pvv.177-9}\footnote{\label{pvv.177-9}  ९ न द्रव्यस्वलक्षणमिति व्याचष्टे । नियमेनानुत्पत्तेः ।} सामान्यगोचरं”} संचितपरमाणुसंघातविषयं {\color{DodgerBlue3}“ज्ञानमुक्तं”} तत्त्ववादिना । न तु परमाण्वतिरिक्तसामान्यविषयं । तत्कथं सामान्यविषयत्वात् सविकल्पत्वप्रसङ्गः ॥ (१९६)
	\pend
      \label{div_pvv.2.197}\edlabel{div_pvv.2.197}
	  
	% new div opening: depth here is 2
	
	  \bigskip
	  \begingroup
	  \large
	
	    
	    \stanza[\smallbreak]
	\label{pv.2.197}\edlabel{pv.2.197}\flagstanza{\tiny\textenglish{....2.197}}अथैकायतनत्वेपि नानेकं दृश्यते सकृत् ।&सकृद्ग्रहावभासः किं वियुक्तेषु तिलादिषु ॥ १९७ ॥\&[\smallbreak]


	
	  \endgroup
	

	  \pstart अथैकेन्द्रियज्ञानज\edlabel{pvv.177-10}\footnote{\label{pvv.177-10}  १० न विषयव्यपदेशि मनोज्ञानस्यापि विषयज्ञानत्वात् । “असाधारणहेतुत्त्वादक्षैस्तद् व्यपदिश्यते” इति दिग्नागः । अवयविद्रव्यमेकं जनकं न बहवः ।}नकत्वात् नीलपीतादीना{\color{DodgerBlue3}“मेकायतनत्वे”} रूपायतनत्वसंग्रहेपि {\color{DodgerBlue3}“नानेकं”} नीलादि {\color{DodgerBlue3}“सकृद् दृश्यते”} किन्तु क्रमेण तत्कथमणूना {\color{DodgerBlue3}“बहूनामेकदा”} \leavevmode\marginnote{\textenglish{178/s}} {\color{DodgerBlue3}“ग्रहणं ।\edlabel{pvv.178-1}\footnote{\label{pvv.178-1}  १ येन सामान्यविषयत्वं स्यात् ।} अत्रोच्यते”} । यदि नानेकमेकदा गृह्यते तदा {\color{DodgerBlue3}“तिलादिषु वियु\edlabel{pvv.178-2}\footnote{\label{pvv.178-2}  २ माषमुद्गादेः संयोगस्य सहकारिणो नियुक्तत्वेऽभावान्नावयवी ।}क्तेषु”} विभिन्न{\color{DodgerBlue3}“देशेषु सकृद्ग्र”}हाव{\color{DodgerBlue3}“भासो”} युगपद् ग्रहणानुभवः किं कस्माद्धे तोः । (१९७)
	\pend
      \label{div_pvv.2.198}\edlabel{div_pvv.2.198}
	  
	% new div opening: depth here is 2
	

	  \pstart ज्ञानानां लघुवृत्तित्वात् सकृद् ग्रहणभ्रमश्चेत् । आह (।)
	\pend
      
	  \bigskip
	  \begingroup
	  \large
	
	    
	    \stanza[\smallbreak]
	\label{pv.2.198}\edlabel{pv.2.198}\flagstanza{\tiny\textenglish{....2.198}}प्रत्युक्तं लाघवञ्चात्र तेष्वेव क्रमपातिषु ।&किं नाक्रमग्रहस्तुल्यकालाः सर्व्वाश्च बुद्धयः ॥ १९८ ॥\&[\smallbreak]


	
	  \endgroup
	

	  \pstart {\color{DodgerBlue3}“प्रत्युक्तं”} प्रतिक्षिप्तं {\color{DodgerBlue3}“चात्र”} सकृद्ग्रहावभासे {\color{DodgerBlue3}“लाघवं”} बुद्धीनां । अन्यत्रापि समानं तद्वर्ण्णयोर्व्वा सकृच्छ्रुति\edlabel{pvv.178-3}\footnote{\label{pvv.178-3}  ३ युगपत्पञ्चज्ञानसाधने ।}(२।१३५)रित्यादिना । तथाप्युच्यते । {\color{DodgerBlue3}“तेष्वेव”} तिलादिषु हस्तादिभ्यः {\color{DodgerBlue3}“क्रमपातिषु किं”} कस्मा{\color{DodgerBlue3}“न्नाक्रमग्रहणं”} भवति । {\color{DodgerBlue3}“सर्व्वाश्च बुद्धयः\edlabel{pvv.178-4}\footnote{\label{pvv.178-4}  ४ क्षणिकत्वात् ।}”} सहावस्थितेषु सम्भवन्त्य{\color{DodgerBlue3}“स्तुल्यकालाः”} । (१९८)
	\pend
      \label{div_pvv.2.199}\edlabel{div_pvv.2.199}
	  
	% new div opening: depth here is 2
	

	  \pstart ततः (।)
	\pend
      
	  \bigskip
	  \begingroup
	  \large
	
	    
	    \stanza[\smallbreak]
	\label{pv.2.199}\edlabel{pv.2.199}\flagstanza{\tiny\textenglish{....2.199}}काश्चित्तास्वक्रमाभासाः क्रमवत्योपराश्च किम् ।&सर्वार्थग्रहणे तस्मादक्रमोयं प्रसज्यते ॥ १९९ ॥\&[\smallbreak]


	
	  \endgroup
	

	  \pstart {\color{DodgerBlue3}“तासु काश्चिदक्रमाभासायाः”} सहस्थितवस्तुविषयाया अपराश्च बुद्धयः \leavevmode\marginnote{\textenglish{35a/MA}} {\color{DodgerBlue3}“क्रमवत्यो”}ऽयुगपत्प्रतिभासाः {\color{DodgerBlue3}“किं”} भवन्ति याः क्रमपातिवस्तुविषयाः (।)अस्ति चायं भेदः । {\color{DodgerBlue3}“तस्मात्सर्व्व”}स्यार्थस्य क्रमिणोऽक्रमिणश्च {\color{DodgerBlue3}“ग्रहणेऽक्रमोऽयं”} लाघवाविशेषा{\color{DodgerBlue3}“त्प्रसज्यते”} । (१९९)
	\pend
      \label{div_pvv.2.200}\edlabel{div_pvv.2.200}
	  
	% new div opening: depth here is 2
	

	  \pstart किञ्च (।)
	\pend
      
	  \bigskip
	  \begingroup
	  \large
	
	    
	    \stanza[\smallbreak]
	\label{pv.2.200}\edlabel{pv.2.200}\flagstanza{\tiny\textenglish{....2.200}}नैकं चित्रपतङ्गादि रूपं वा दृश्यते कथम् ।&चित्रं तदेकमिति चेदिदं चित्रतरन्ततः ॥ २०० ॥\&[\smallbreak]


	
	  \endgroup
	

	  \pstart {\color{DodgerBlue3}“चित्रपतङ्गादि नैकम”}नेकं नीलादिरूपं {\color{DodgerBlue3}“वा दृश्यते कथं”} यदि नानेकमेकेन गृह्यते । {\color{DodgerBlue3}“चित्रं”} नीलपीताद्यात्मकं तत्पतङ्गादिक{\color{DodgerBlue3}“मेकमिति\edlabel{pvv.178-5}\footnote{\label{pvv.178-5}  ५ सकृद्दर्शनमविरुद्धं ।} चेत्”} इदञ्चित्र{\color{DodgerBlue3}“मेकं”} यदुच्यते त{\color{DodgerBlue3}“त्तत”}श्चित्रपतङ्गादपि {\color{DodgerBlue3}“चित्रतरमा”}श्चर्यतरं । चित्रमिति नानारूपाणि तदेव पुनरेकमुच्यत इत्युपहसति (॥ २००)
	\pend
      \label{div_pvv.2.201}\edlabel{div_pvv.2.201}
	  
	% new div opening: depth here is 2
	

	  \pstart तथा च (।)
	\pend
      
	  \bigskip
	  \begingroup
	  \large
	
	    
	    \stanza[\smallbreak]
	\label{pv.2.201}\edlabel{pv.2.201}\flagstanza{\tiny\textenglish{....2.201}}नैकं स्वभावं चित्रं हि मणिरूपं यथैव तत् ।&नीलादि प्रतिभासश्च तुल्यश्चित्रपटादिषु ॥ २०१ ॥\&[\smallbreak]


	
	  \endgroup
	\leavevmode\marginnote{\textenglish{179/s}}

	  \pstart {\color{DodgerBlue3}“चित्रम”}नेकरूपं {\color{DodgerBlue3}“हि”} यस्मात्तस्मान्नैकं पतङ्गादि । {\color{DodgerBlue3}“यथैव”} संस्थानविशेषेण सन्नि\edlabel{pvv.179-1}\footnote{\label{pvv.179-1}  १ द्रव्याणि द्रव्यान्तरमारभन्ते गुणा गुणान्तरमत्र तु नानारूपा मणयः ।} विष्टानां बहूनां {\color{DodgerBlue3}“मणी”}नां {\color{DodgerBlue3}“रूपं त”}च्चित्रमनेकं नैकमवयवि द्रव्यं विजातीयानां द्रव्यानारम्भात् । चित्रबुद्धिरेकत्वान्मुख्या पतङ्गे\edlabel{pvv.179-2}\footnote{\label{pvv.179-2}  २ एकोत्रावयवी । अवयवास्तु विभिन्नाः दृश्यन्ते ।}मणिरूपादिषु पुनरुपचरितेति चेत् । आह(।) नीलादिप्रतिभासश्चित्रप्रतिभासः स {\color{DodgerBlue3}“चित्रपट आदि”}र्येषां मणिरूपादीनां तेषु चित्रपतङ्गे {\color{DodgerBlue3}“च तुल्यः”} । न त्वग्निमाणवकयोर्दहनबुद्धिरिव स्खलदस्खलद्वृत्तिर्लक्ष्यते । (२०१)
	\pend
      \label{div_pvv.2.202}\edlabel{div_pvv.2.202}
	  
	% new div opening: depth here is 2
	
	  \bigskip
	  \begingroup
	  \large
	
	    
	    \stanza[\smallbreak]
	\label{pv.2.202}\edlabel{pv.2.202}\flagstanza{\tiny\textenglish{....2.202}}तत्रावयवरूपञ्चैत् केवलं दृश्यते तथा ।&नीलादीनि निरस्यान्यच्चित्रं चित्रं यदीक्षसे ॥ २०२ ॥\&[\smallbreak]


	
	  \endgroup
	

	  \pstart {\color{DodgerBlue3}“तत्र”} चित्रपटादिषु केवलमव{\color{DodgerBlue3}“यवरूपं तथा चित्रतया दृश्यते”} नावयवी विजातीयानां द्रव्यानारम्भादिति चेत् । चित्रपतङ्गादावपि {\color{DodgerBlue3}“नीलादीनि निरस्य”} पृथक्कृत्य {\color{DodgerBlue3}“तेभ्योऽन्यच्चित्र”}मवयविरूपं {\color{DodgerBlue3}“यदीक्षसे”} त्वं त{\color{DodgerBlue3}“च्चित्र”}माश्चर्यं । स्वसिद्धान्तानुरागभेषजविशोधितचक्षुरीक्षसे त्वमेव यदीदृशमवयविनं परं नात्रान्येषामधिका\edlabel{pvv.179-3}\footnote{\label{pvv.179-3}  ३ इति स्वभावानुपलम्भ उक्तः ।}रः । (२०२)
	\pend
      \label{div_pvv.2.203}\edlabel{div_pvv.2.203}
	  
	% new div opening: depth here is 2
	

	  \pstart अपि च(।)
	\pend
      
	  \bigskip
	  \begingroup
	  \large
	
	    
	    \stanza[\smallbreak]
	\label{pv.2.203}\edlabel{pv.2.203}\flagstanza{\tiny\textenglish{....2.203}}तुल्यार्थाकारकालत्वेनोपलक्षितयोर्द्वयोः ।&नानार्था क्रमवत्येका किमेकार्थाक्रमापरा ॥ २०३ ॥\&[\smallbreak]


	
	  \endgroup
	

	  \pstart तुल्या\edlabel{pvv.179-4}\footnote{\label{pvv.179-4}  ४ अर्थाकारश्च कालश्च ।}र्थाकारत्वेन तुल्यकालत्त्वेन {\color{DodgerBlue3}“चोपलक्षितयोः”} कृत्रिमाकृत्रिमपतङ्गविशेषणयोर्द्वयोर्म्मध्ये {\color{DodgerBlue3}“एका”} कृत्रिमपतङ्गविषया {\color{DodgerBlue3}“धीर्नानार्था”} विजातीयात्मकद्रव्यानारम्भात् {\color{DodgerBlue3}“क्रमवती”} च नीलानां बहूनां क्रमेण ग्रहणात् । {\color{DodgerBlue3}“अपरा”} अकृत्रिमपतङ्गविषया एकार्थावियविविषया अत एवाक्रमा च किं कस्मादिष्यते । द्वयोरपि समानता युक्ता निमित्तस्य साम्यात् । (२०३)
	\pend
      \label{div_pvv.2.204}\edlabel{div_pvv.2.204}
	  
	% new div opening: depth here is 2
	

	  \pstart किञ्च ।
	\pend
      
	  \bigskip
	  \begingroup
	  \large
	
	    
	    \stanza[\smallbreak]
	\label{pv.2.204}\edlabel{pv.2.204}\flagstanza{\tiny\textenglish{....2.204}}वैश्वरूप्याद्धियामेव भावानां विश्वरूपता ॥&तच्चेदनङ्गं केनेयं सिद्धा भेदव्यवस्थितिः ॥ २०४ ॥\&[\smallbreak]


	
	  \endgroup
	

	  \pstart {\color{DodgerBlue3}“धियामेव वैश्वरू\edlabel{pvv.179-5}\footnote{\label{pvv.179-5}  ५ सम्विनष्टत्वाद्विषयस्थितेः ।}प्यान्ना”}नाकारत्वाद् {\color{DodgerBlue3}“भावानां”} ग्राह्यानां (? णां) {\color{DodgerBlue3}“विश्वरूपता”} व्यवस्थाप्यते । चित्रस्यावयविन\edlabel{pvv.179-6}\footnote{\label{pvv.179-6}  ६ चित्रपतङ्गे ।} एकतास्वीकारे तद्बुद्धिषु प्रतिभासनानात्वं भेदव्यवस्थिता{\color{DodgerBlue3}“वनङ्गं चेत्”} । तदा भावानां {\color{DodgerBlue3}“भेदव्यवस्थिति”}रपहनूयेतैव । {\color{DodgerBlue3}“केना-”} न्येन निबन्धेन {\color{DodgerBlue3}“सिद्धा”} भविष्यति । (२०४)
	\pend
      \label{div_pvv.2.205}\edlabel{div_pvv.2.205}
	  
	% new div opening: depth here is 2
	\leavevmode\marginnote{\textenglish{180/s}}

	  \pstart अपि च\edlabel{pvv.180-1}\footnote{\label{pvv.180-1}  १ एकोऽवयवी यदि ज्ञानजनको न नीलादयस्तदा ।} (।)
	\pend
      
	  \bigskip
	  \begingroup
	  \large
	
	    
	    \stanza[\smallbreak]
	\label{pv.2.205}\edlabel{pv.2.205}\flagstanza{\tiny\textenglish{....2.205}}विजातीनामनारम्भादालेख्यादौ न चित्रधीः ।&अरूपत्वान्न संयोगश्चित्रो भक्तेश्च नाश्रयः ॥ २०५ ॥\&[\smallbreak]


	
	  \endgroup
	

	  \pstart {\color{DodgerBlue3}“विजातीनां”} भिन्नजातीनां रागद्रव्याणां कार्यद्रव्या{\color{DodgerBlue3}“नारम्भात् आलेख्यादौ चित्रधीर्न”} स्यात् ॥ आलेख्यं संयोगस्तस्य चित्रं रूपमिति चेत् (।){\color{DodgerBlue3}“न”} (।){\color{DodgerBlue3}“संयोगो”}पि चित्रस्तस्या{\color{DodgerBlue3}“रूपत्वात्”} ।\edlabel{pvv.180-2}\footnote{\label{pvv.180-2}  २ संयोगस्य रूपत्वे रूपे रूपं स्यात् ।} संयोगो गुणस्तथा रूपञ्च । न च गुणे गुणान्तरमस्ति ॥
	\pend
      

	  \pstart स्यादेतत्(।)यथा तरुषु संख्यालक्षणं वनं कुसुमितत्वं चात एकार्थसमवायात् वने कुसुमितबुद्धिः । तथावयवेषु चित्रसंयोगयोः समवायाच्चित्रं चित्रमिति बुद्धिरुपचारादित्यादि भक्तेरुपचारस्य च संयोग {\color{DodgerBlue3}“आश्रयो न”} युक्तः । (२०५)
	\pend
      \label{div_pvv.2.206}\edlabel{div_pvv.2.206}
	  
	% new div opening: depth here is 2
	
	  \bigskip
	  \begingroup
	  \large
	
	    
	    \stanza[\smallbreak]
	\label{pv.2.206}\edlabel{pv.2.206}\flagstanza{\tiny\textenglish{....2.206}}प्रत्येकमविचित्रत्वाद् गृहीतेषु क्रमेण च ।&न चित्रधीसङ्कलनमनेकस्यैकयाऽग्रहात् ॥ २०६ ॥\&[\smallbreak]


	
	  \endgroup
	

	  \pstart {\color{DodgerBlue3}“प्रत्येक”}मवय(व)ानाम{\color{DodgerBlue3}“विचित्रत्वात्”} । नीलादिषु {\color{DodgerBlue3}“क्रमेण”} स्वबुद्धिभि{\color{DodgerBlue3}“र्गृहीतेषु”}\edlabel{pvv.180-3}\footnote{\label{pvv.180-3}  ३ पश्चात् ।} बुद्ध्यन्तरेण चित्रसंकलनं संक्षिप्य ग्रहणञ्च न युक्तं । {\color{DodgerBlue3}“एकया”} धियाऽ{\color{DodgerBlue3}“नेकस्याग्रहात्”} । (२०६)
	\pend
      \label{div_pvv.2.207}\edlabel{div_pvv.2.207}
	  
	% new div opening: depth here is 2
	
	  \bigskip
	  \begingroup
	  \large
	
	    
	    \stanza[\smallbreak]
	\label{pv.2.207}\edlabel{pv.2.207}\flagstanza{\tiny\textenglish{....2.207}}नानार्थैका भवेत्तस्मात् सिद्धातोप्यविकल्पिका ।&विकल्पयन्नायेकार्थ यतोन्यदपि पश्यति ॥ २०७ ॥\&[\smallbreak]


	
	  \endgroup
	

	  \pstart कथं नीलानामेकबुद्ध्या {\color{DodgerBlue3}“संकलनं”} । इष्टौ वा तस्मात्संकलनस्वीकारादेवैका बुद्धिर्नानार्थाऽनेकविषया भवेत्\edlabel{pvv.180-4}\footnote{\label{pvv.180-4}  ४ यदुक्त“मेकायतनत्वेपि नानेकं दृश्यते सकृदि”ति\cref{pv.2.197} तद् ध्वस्तं स्यात् ।} । अतोऽनेकार्थवेदनादपि बुद्धिर{\color{DodgerBlue3}“विकल्पिका सिद्धा”} । यतः शब्दयोजित{\color{DodgerBlue3}“मेकमर्थं विकल्प”}यन्ना{\color{DodgerBlue3}“न्यद”}संयोजितमर्थान्तर{\color{DodgerBlue3}“मपि पश्यति”} द्रष्टा । न ह्येकदानेकशब्दयोजना तस्मादर्थसञ्चयविषयत्वात् सामान्यविषयत्वं (।) तथापि त्वविकल्पिते तेन विरोधः । (२०७)
	\pend
      \label{div_pvv.2.208}\edlabel{div_pvv.2.208}
	  
	% new div opening: depth here is 2
	

	  \begin{center}%% label @type='head'
	\textbf{ख. चित्रैकत्वचिन्ता}
	\end{center}
	

	  \pstart ननु (।)
	\pend
      
	  \bigskip
	  \begingroup
	  \large
	
	    
	    \stanza[\smallbreak]
	\label{pv.2.208}\edlabel{pv.2.208}\flagstanza{\tiny\textenglish{....2.208}}चित्रावभासेष्वर्थेषु यद्यैकत्वं न युज्यते ।&सैव तावत् कथं बुद्धिरेका चित्रावभासिनी ॥ २०८ ॥\&[\smallbreak]


	
	  \endgroup
	\leavevmode\marginnote{\textenglish{181/s}}

	  \pstart {\color{DodgerBlue3}“चित्रं”} नानाकारोऽ\edlabel{pvv.181-1}\footnote{\label{pvv.181-1}  १ परो दूषयति पूर्व्वोक्तं ।} {\color{DodgerBlue3}“वभासो”} येषां पतङ्गादीनां तेष्वर्थेष्वे{\color{DodgerBlue3}“कत्वं”} न युज्यते । {\color{DodgerBlue3}“यदि सैव”} चित्रार्थग्राहिणी बुद्धिश्चित्रावभासिनी चित्रा\edlabel{pvv.181-2}\footnote{\label{pvv.181-2}  २ परमुखेनोपन्यासो धर्मनैरात्म्यसूचनाय ।}कारा {\color{DodgerBlue3}“कथमेका”} संमता । यथा चित्रत्वेपि बुद्धिरेका तथा कार्यद्रव्यञ्चैकं स्यात् । (२०८)\leavevmode\marginnote{\textenglish{35b/MA}}
	\pend
      \label{div_pvv.2.209}\edlabel{div_pvv.2.209}
	  
	% new div opening: depth here is 2
	

	  \pstart अत्राह (।)
	\pend
      
	  \bigskip
	  \begingroup
	  \large
	
	    
	    \stanza[\smallbreak]
	\label{pv.2.209a}\edlabel{pv.2.209a}\flagstanza{\tiny\textenglish{...2.209a}}इदं वस्तुबलायातं यद् वदन्ति विपश्चितः ।\&[\smallbreak]


	
	  \endgroup
	

	  \pstart इदं वस्तु\edlabel{pvv.181-3}\footnote{\label{pvv.181-3}  ३ तात्विकाकारवादिनोयं दोषो न ममेत्याह ।}नोऽव्यभिचारि लिङ्गस्य {\color{DodgerBlue3}“बलादायातं यद्व\edlabel{pvv.181-4}\footnote{\label{pvv.181-4}  ४ धर्मनैरात्म्यं ।}दन्ति विपश्चितो”} बु द्धा भगवन्तः ।
	\pend
      

	  \pstart किं तदित्याह (।)
	\pend
      
	  \bigskip
	  \begingroup
	  \large
	
	    
	    \stanza[\smallbreak]
	\label{pv.2.209b}\edlabel{pv.2.209b}\flagstanza{\tiny\textenglish{...2.209b}}यथा यथार्था चिन्त्यन्ते विशीर्यन्ते तथा तथा ॥ २०९ ॥\&[\smallbreak]


	
	  \endgroup
	

	  \pstart {\color{DodgerBlue3}“यथा यथा”} येन प्रकारेण एकत्वेनानेकत्वेन {\color{DodgerBlue3}“वाऽर्था”} नीलादयो बाह्यज्ञानात्मानो वा {\color{DodgerBlue3}“विचिन्त्यन्ते तथा”} विशीर्यन्ते क्वचिदपि {\color{DodgerBlue3}“न”} व्यवतिष्ठन्त इति यावत् । न हि ज्ञानमेकं नानाकारत्वात् । तल्लक्षणत्वाच्च भेदस्य । नाप्यनेकं चित्रप्रतिभासानुपपत्तेः । परस्परमवेदनात् । अन्यस्य च ग्राहकस्याभावात् । (२०९)
	\pend
      \label{div_pvv.2.210}\edlabel{div_pvv.2.210}
	  
	% new div opening: depth here is 2
	
	  \bigskip
	  \begingroup
	  \large
	
	    
	    \stanza[\smallbreak]
	\label{pv.2.210a}\edlabel{pv.2.210a}\flagstanza{\tiny\textenglish{...2.210a}}किं स्यात् सा चित्रतैकस्यां;\&[\smallbreak]


	
	  \endgroup
	

	  \pstart ननु यदि {\color{DodgerBlue3}“सा चित्रता”} बुद्धा{\color{DodgerBlue3}“वेकस्यां”} स्यात् तया च चित्रमे\edlabel{pvv.181-5}\footnote{\label{pvv.181-5}  ५ प्रष्टुः साध्यं ।}कं द्रव्यं व्यवस्थाप्येत (।) तदा {\color{DodgerBlue3}“कि”} दूषणं {\color{DodgerBlue3}“स्यात्”} ।
	\pend
      

	  \pstart आह (।)
	\pend
      
	  \bigskip
	  \begingroup
	  \large
	
	    
	    \stanza[\smallbreak]
	\label{pv.2.210b}\edlabel{pv.2.210b}\flagstanza{\tiny\textenglish{...2.210b}}न स्यात्तस्यां मतावपि ।\&[\smallbreak]


	
	  \endgroup
	

	  \pstart {\color{DodgerBlue3}“न”} केवलं द्रव्ये तस्यां मतावप्ये{\color{DodgerBlue3}“कस्यां न स्या च्चित्रता”} । आकारनानात्वलक्षणत्वाद् भेदस्य । नानात्वेपि चित्रता कथम् (।) अनेकपुरुषप्रतीतिवत् ।
	\pend
      

	  \pstart कथन्तर्हि प्रतीतिरित्याह (।)
	\pend
      
	  \bigskip
	  \begingroup
	  \large
	
	    
	    \stanza[\smallbreak]
	\label{pv.2.210c}\edlabel{pv.2.210c}\flagstanza{\tiny\textenglish{...2.210c}}यदीदं स्वयमर्थानां रोचते तत्र के वयम् ॥ २१० ॥\&[\smallbreak]


	
	  \endgroup
	

	  \pstart {\color{DodgerBlue3}“यदीद”}मताद्रूप्येपि ताद्रूप्यप्रथन{\color{DodgerBlue3}“मर्था”}नां भासमानानां नीलादीनां {\color{DodgerBlue3}“स्वयम”}परप्रेरणया {\color{DodgerBlue3}“रोचते । तत्र”} तथाप्रतिभासे {\color{DodgerBlue3}“के वयम”}सहमानाऽ(? अ) पि निषेद्धुं । अवस्तु च प्रति\edlabel{pvv.181-6}\footnote{\label{pvv.181-6}  ६ अविद्यावशादुपलम्भः ।}भासते चेति व्यक्तमालीक्यं । (२१०)
	\pend
      \label{div_pvv.2.211}\edlabel{div_pvv.2.211}
	  
	% new div opening: depth here is 2
	\leavevmode\marginnote{\textenglish{182/s}}
	  \bigskip
	  \begingroup
	  \large
	
	    
	    \stanza[\smallbreak]
	\label{pv.2.211}\edlabel{pv.2.211}\flagstanza{\tiny\textenglish{....2.211}}तस्मान्नार्थेषु न ज्ञाने स्थूलाभासस्तदात्मनः ।&एकत्र प्रतिषिद्धत्वाद् बहुष्वपि न सम्भवः ॥ २११ ॥\&[\smallbreak]


	
	  \endgroup
	

	  \pstart {\color{DodgerBlue3}“तस्मान्ना\edlabel{pvv.182-1}\footnote{\label{pvv.182-1}  १ एकानेकोभयरूपरहिता स्वसम्वित्तिरविरुद्धैव ।}र्थेषु बाह्येषु न ज्ञाने”} तद्ग्राहके {\color{DodgerBlue3}“स्थूलाभासः”} स्थूल आकारः संगच्छते । {\color{DodgerBlue3}“तदात्मनः”} स्थूलस्वरूप{\color{DodgerBlue3}“स्यैकत्रा”}वयवे परमाणौ वा {\color{DodgerBlue3}“प्रतिषिद्धत्वात् । बहुष्वपि”} तेषु {\color{DodgerBlue3}“सम्भवो ना”}स्ति मिलिता अपि हि त\edlabel{pvv.182-2}\footnote{\label{pvv.182-2}  २ परमाणव एव न स्थूलाः स्वरूपहानेः ।} {\color{DodgerBlue3}“एव”} । ते च प्रत्येकं स्थौल्यविकला इति समुदिता अपि तथैव स्युः । तथा नीलाद्याकारेषु प्रत्येकं चित्रस्य स्थौल्यस्याभावात् समुदायेप्यभावः । (२११)
	\pend
      \label{div_pvv.2.212}\edlabel{div_pvv.2.212}
	  
	% new div opening: depth here is 2
	

	  \pstart ननु सुखाद्यात्मकं\edlabel{pvv.182-3}\footnote{\label{pvv.182-3}  ३ तत्त्वतो द्वयप्रतिभासि ।} स्वप्रकाशं विज्ञानमेकमिदमिति यो गा चा र मतमव्याहत{\color{DodgerBlue3}“मित्याह”} (।)
	\pend
      
	  \bigskip
	  \begingroup
	  \large
	
	    
	    \stanza[\smallbreak]
	\label{pv.2.212}\edlabel{pv.2.212}\flagstanza{\tiny\textenglish{....2.212}}परिच्छेदोन्तरन्योऽयं भागो बहिरिव स्थितः ।&ज्ञानस्याभेदिनौ भिन्नौ प्रतिभासो ह्युपप्लवः ॥ २१२ ॥\&[\smallbreak]


	
	  \endgroup
	

	  \pstart {\color{DodgerBlue3}“परिच्छेदो”} ग्राहकाकारः सुखादे{\color{DodgerBlue3}“रन्तर”}बहिर्देशे परिच्छेदा{\color{DodgerBlue3}“दन्योयं”} भागो ग्राह्यो नीलादिर्ब्बहिःस्थित {\color{DodgerBlue3}“इवा”}भाति सर्व्वेषां । {\color{DodgerBlue3}“हि”}र्यस्मात् {\color{DodgerBlue3}“ज्ञानस्याभेदिनौ भि”}न्नावाकारौ तत्त्वतो {\color{DodgerBlue3}“न”} युक्तौ (।) तस्मादन्तर्ब्बहिर्देशसम्बन्धतया {\color{DodgerBlue3}“प्रतिभास उपप्लवो न सत्यः”} । (२१२)
	\pend
      \label{div_pvv.2.213}\edlabel{div_pvv.2.213}
	  
	% new div opening: depth here is 2
	
	  \bigskip
	  \begingroup
	  \large
	
	    
	    \stanza[\smallbreak]
	\label{pv.2.213}\edlabel{pv.2.213}\flagstanza{\tiny\textenglish{....2.213}}तत्रैकस्याप्यभावेन द्वयमप्यवहीयते ।&तस्मात्तदेव तस्यापि तत्त्वं या द्वयशून्यता ॥ २१३ ॥\&[\smallbreak]


	
	  \endgroup
	

	  \pstart {\color{DodgerBlue3}“तत्र”} एकज्ञानात्मनि विरुद्धं द्वयं न युक्तमि{\color{DodgerBlue3}“त्येकस्य”} ग्राह्यत्वस्य ग्राहकत्वस्य वावश्याभ्युपगन्तव्ये{\color{DodgerBlue3}“नाभावेन द्वयमप्यवहीयते”} । अन्योन्यसापेक्षयोरेकाभावेऽपराभावस्य न्यायप्राप्तत्वात् । {\color{DodgerBlue3}“तस्मात्तस्य”} ज्ञानस्या{\color{DodgerBlue3}“पि तत्त्वं तदेव या”} द्वयेन ग्राह्यग्राहकाकारेण शून्यता नाम । (२१३)
	\pend
      \label{div_pvv.2.214}\edlabel{div_pvv.2.214}
	  
	% new div opening: depth here is 2
	
	  \bigskip
	  \begingroup
	  \large
	
	    
	    \stanza[\smallbreak]
	\label{pv.2.214}\edlabel{pv.2.214}\flagstanza{\tiny\textenglish{....2.214}}तद्भेदाश्रयिणी चेयं भावानां भेदसंस्थितिः ।&तदुपप्लवभावे च तेषां भेदोप्युपप्लवः ॥ २१४ ॥\&[\smallbreak]


	
	  \endgroup
	

	  \pstart {\color{DodgerBlue3}“इयं च भावानां”} रूपवेदनादीनां {\color{DodgerBlue3}“भेदसं”}स्थितिः (।) तस्य ग्राह्यग्राहकस्य {\color{DodgerBlue3}“भेदः”} स {\color{DodgerBlue3}“आश्रयो”} यस्याः सा तथा । तस्य ग्राह्यग्राहकभावस्य भेदव्यवस्थानिबन्धनस्यो{\color{DodgerBlue3}“पप्लवभावे”} मिथ्यात्वे च तेषां रूपादीनां {\color{DodgerBlue3}“भेदो”}ऽपि तद्व्यवस्थापित उपप्लवः । न केवलं रूपादीनां भेदाभेदावुपप्लवः । (२१४)
	\pend
      \label{div_pvv.2.215}\edlabel{div_pvv.2.215}
	  
	% new div opening: depth here is 2
	\leavevmode\marginnote{\textenglish{183/s}}

	  \pstart लक्षणशून्यत्वान्निःस्वभावत्वमपीत्याह (।)
	\pend
      
	  \bigskip
	  \begingroup
	  \large
	
	    
	    \stanza[\smallbreak]
	\label{pv.2.215}\edlabel{pv.2.215}\flagstanza{\tiny\textenglish{....2.215}}न ग्राह्यग्राहकाकारबाह्यमस्ति च लक्षणम् ।&अतो लक्षणशून्यत्वान्निःस्वभावाः प्रकाशिताः ॥ २१५ ॥\&[\smallbreak]


	
	  \endgroup
	

	  \pstart रूपादीनां {\color{DodgerBlue3}“ग्राह्यग्राहकाकारा”}भ्यां {\color{DodgerBlue3}“बाह्यं”} भिन्नं {\color{DodgerBlue3}“लक्षणं न चास्ति”} । तथा हि विषयतया किञ्चिन्निर्द्दिश्यते । यथा रूप्यत इति कृत्वा रूपं । किञ्चिद्विषयितया विजानातीत्यादि\edlabel{pvv.183-1}\footnote{\label{pvv.183-1}  १ संजानातीति संज्ञा, अनुभवतीत्यनुभव इत्यादि ।} व्युत्पत्त्या यथा रूपिणः स्कन्धाः । न त्वेतद्व्यतिरिक्तं किञ्चिल्लक्षणमस्ति । {\color{DodgerBlue3}“अतो लक्षणेन”} ग्राह्यग्राहकत्वे\edlabel{pvv.183-2}\footnote{\label{pvv.183-2}  २ न तु सम्वित्त्या ।}न {\color{DodgerBlue3}“शून्यत्वान्निःस्वभावाः”} सर्व्वधर्म्माः {\color{DodgerBlue3}“प्रकाशिता”} भगवद््भिर्बु द्धैः । (२१५)
	\pend
      \label{div_pvv.2.216}\edlabel{div_pvv.2.216}
	  
	% new div opening: depth here is 2
	

	  \pstart किञ्च (।) बहिरर्थवादेपि लक्षणशून्यत्वात् निःस्वभावतां धर्म्माणामाख्यातुमाह (।)
	\pend
      
	  \bigskip
	  \begingroup
	  \large
	
	    
	    \stanza[\smallbreak]
	\label{pv.2.216}\edlabel{pv.2.216}\flagstanza{\tiny\textenglish{....2.216}}व्यापारोपाधिकं सर्वं स्कन्धादीनां विशेषतः ।&लक्षणं स च तत्त्वन्न तेनाप्येते विलक्षणाः ॥ २१६ ॥\&[\smallbreak]


	
	  \endgroup
	

	  \pstart {\color{DodgerBlue3}“स्कन्ध”} आदिर्येषां धात्वायतना{\color{DodgerBlue3}“नां”} तेषां यल्लक्षणं राशीभवन्तीति राश्यर्थः स्कन्धानां (।) कार्योत्पादकत्वेनाकारार्थो धातूनां । आयन्तन्व\edlabel{pvv.183-3}\footnote{\label{pvv.183-3}  ३ चित्तचैत्तानामुत्पत्तिम्विस्तृण्वन्ति ।}न्तीत्यायद्वारार्थ आयतनानां (।) {\color{DodgerBlue3}“तत्सर्वं लक्षणं व्यापारोपाधिकं”} व्यापारविशेषणं । {\color{DodgerBlue3}“स च”} व्यापारो-\leavevmode\marginnote{\textenglish{36a/MA}} ऽप्रतीतेर्भेदपक्षे न तत्त्वं । अभेदेपि प्रागभावी भाव एव स्यात् । स च तत्त्वं न भवति । “अशक्तं सर्वमिति चेदि” \href{http://http://sarit.indology.info/?cref=pv.2.4}{(२।४)} त्यत्रोक्तक्रमात् । {\color{DodgerBlue3}“तेन”} व्यापारोपाधिकलक्षणायोगेनापि {\color{DodgerBlue3}“विलक्षणा”} निःस्व{\color{DodgerBlue3}“भावा एते”} स्कन्धादयः । (२१६)
	\pend
      \label{div_pvv.2.217}\edlabel{div_pvv.2.217}
	  
	% new div opening: depth here is 2
	

	  \pstart कथं तर्हि बाह्यस्कन्धादिदेशना भगवतामित्याह (।)
	\pend
      
	  \bigskip
	  \begingroup
	  \large
	
	    
	    \stanza[\smallbreak]
	\label{pv.2.217}\edlabel{pv.2.217}\flagstanza{\tiny\textenglish{....2.217}}यथास्वंप्रत्ययापेक्षादविद्योपप्लुतात्मनाम् ।&विज्ञप्तिर्वितथाकारा जायते तिमिरादिवत् ॥ २१७ ॥\&[\smallbreak]


	
	  \endgroup
	

	  \pstart अनाद्य{\color{DodgerBlue3}“विद्योपप्लुतात्म”}नामप्रहीणाक्लिष्टज्ञा\edlabel{pvv.183-4}\footnote{\label{pvv.183-4}  ४ ग्राह्यग्राहकाभिनिविष्टानां ।}नानां पुंसां य\edlabel{pvv.183-5}\footnote{\label{pvv.183-5}  ५ बाह्यानपेक्षत्वे सदा स्यादित्याह ।}थास्वं यस्य भ्रमस्य य अत्मीयः\edlabel{pvv.183-6}\footnote{\label{pvv.183-6}  ६ सन्तानपरिणाम उत्पत्तिनिमित्तं ।} प्रत्ययो {\color{DodgerBlue3}“यथास्वंप्रत्य”}यस्तस्या{\color{DodgerBlue3}“पेक्षण”}मपेक्षः । तस्माद्वितथौ ग्राह्यग्राहकाकारौ यस्याः सा तादृशी {\color{DodgerBlue3}“विज्ञप्तिर्जायते । तिमिरादिवत्”} तिमिरादाविव, {\color{DodgerBlue3}“वितथाकार”}चन्द्रद्वयादिविज्ञाप्तिः । (२१७)
	\pend
      \label{div_pvv.2.218}\edlabel{div_pvv.2.218}
	  
	% new div opening: depth here is 2
	\leavevmode\marginnote{\textenglish{184/s}}
	  \bigskip
	  \begingroup
	  \large
	
	    
	    \stanza[\smallbreak]
	\label{pv.2.218a}\edlabel{pv.2.218a}\flagstanza{\tiny\textenglish{...2.218a}}असंविदिततत्त्वा च सा सर्व्वापरदर्शनैः ।\&[\smallbreak]


	
	  \endgroup
	

	  \pstart सा च विज्ञप्तिः सर्व्वैरपरद\edlabel{pvv.184-1}\footnote{\label{pvv.184-1}  १ बुद्धादन्यैः ।}र्शनैरनुत्कृष्टदर्श{\color{DodgerBlue3}“नैरसम्विदिता”} द्वयशून्यतातत्त्वं यस्याः साऽसंविदिततत्वा (।)
	\pend
      

	  \pstart कस्मादित्याह (।)
	\pend
      
	  \bigskip
	  \begingroup
	  \large
	
	    
	    \stanza[\smallbreak]
	\label{pv.2.218b}\edlabel{pv.2.218b}\flagstanza{\tiny\textenglish{...2.218b}}असंभवाद्विना तेषां ग्राह्यग्राहकविप्लवैः ॥ २१८ ॥\&[\smallbreak]


	
	  \endgroup
	

	  \pstart {\color{DodgerBlue3}“ग्राह्यग्राहकविप्लवैर्व्विना तेषा”}मपरदर्शनानां विज्ञप्तेर{\color{DodgerBlue3}“सम्भवात्”} । ग्राह्यग्राहकोपप्लुतदर्शनत्वमेव चानुत्कृष्टदर्शनत्वं । (२१८)
	\pend
      \label{div_pvv.2.219}\edlabel{div_pvv.2.219}
	  
	% new div opening: depth here is 2
	
	  \bigskip
	  \begingroup
	  \large
	
	    
	    \stanza[\smallbreak]
	\label{pv.2.219}\edlabel{pv.2.219}\flagstanza{\tiny\textenglish{....2.219}}तदुपेक्षिततत्त्वार्थैः कृत्वा गजनिमीलनम् ।&केवलं लोकबुध्यैव बाह्यचिन्ता प्रयन्यते ॥ २१९ ॥\&[\smallbreak]


	
	  \endgroup
	

	  \pstart {\color{DodgerBlue3}“तद् भ्रा”}न्तदर्शनानुरोधा{\color{DodgerBlue3}“दुपेक्षिततत्वार्थै”}रवधारिततदद्वयविवेकै{\color{DodgerBlue3}“र्भगवद््भिर्बुद्धैर्ग्गजनिमीलनं कृत्वा”} गजस्येव पार्श्वद्वयं पश्यतोपि नयननिमीलनकलया तत्त्वमपश्यन्तमिवात्मानं दर्शयद््भि\edlabel{pvv.184-2}\footnote{\label{pvv.184-2}  २ केवलं ।} {\color{DodgerBlue3}“र्लोक”}स्याविद्योपहतस्य {\color{DodgerBlue3}“बुद्ध्या”} द्वयग्राहिण्या {\color{DodgerBlue3}“बाह्य”}स्य {\color{DodgerBlue3}“चिन्ता”}स्कन्धायतनादित्वेन {\color{DodgerBlue3}“प्रतन्यते”} । स्कन्धादिदेशनया हि लोकबुद्ध्यनुरोधप्रवर्तितपाषण्डग्रा\edlabel{pvv.184-3}\footnote{\label{pvv.184-3}  ३ आत्मग्रह ।}हनिवृत्तौ मुख्यया येषु मुमुक्षवोऽवतार्यन्त इति देशनाक्रमः । (२१९)
	\pend
      \label{div_pvv.2.220}\edlabel{div_pvv.2.220}
	  
	% new div opening: depth here is 2
	

	  \pstart अथवा चित्रत्वेपि बाह्यमेकं न युक्तं बुद्धिस्तु चित्राप्येकैवेति दर्शयितु\edlabel{pvv.184-4}\footnote{\label{pvv.184-4}  ४ प्रतिबिम्बवदाकाराः पारमार्थिकत्वेऽनेकान्तः स्यात् ।}माह (।)
	\pend
      
	  \bigskip
	  \begingroup
	  \large
	
	    
	    \stanza[\smallbreak]
	\label{pv.2.220a}\edlabel{pv.2.220a}\flagstanza{\tiny\textenglish{...2.220a}}नीलादिश्चित्रविज्ञाने ज्ञानोपाधिरनन्यभाक् ॥&अशक्यदर्शनः;\&[\smallbreak]


	
	  \endgroup
	

	  \pstart {\color{DodgerBlue3}“नीलादिश्चित्रे ज्ञाने ज्ञानोपाधिर”}नुभवस्यात्मभूतः {\color{DodgerBlue3}“अनन्यभाक्”} आकारान्तरासहचरः\edlabel{pvv.184-5}\footnote{\label{pvv.184-5}  ५ तज्ज्ञानवत् ।} केवल इत्यर्थः । तादृशोऽ{\color{DodgerBlue3}“शक्यद\edlabel{pvv.184-6}\footnote{\label{pvv.184-6}  ६ स नीलादिरेकैकं ।}र्शनः”} सहैवाकारान्तरवेदननियमात् । न हि चित्रे विज्ञाने समुत्पन्ने नीलं निरस्य पीतं शक्यदर्शनं । तस्मादशक्यविवेचनत्वं तुल्ययोगक्षेमत्वं सहप्रतिभासनियतत्वं ज्ञानात्मनां नीलादीनामेकत्वं । बाह्यात्मनां तु नैतत्संभवति । एकं पिधायापि द्रष्टुमन्यस्य शक्यत्वात् ।
	\pend
      

	  \pstart ननु ज्ञानाकारोपि नीलः पीतानुभवकाले तदा नानुभूयते तदा शक्यविवेचन एवेत्याह (।)
	\pend
      
	  \bigskip
	  \begingroup
	  \large
	
	    
	    \stanza[\smallbreak]
	\label{pv.2.220b}\edlabel{pv.2.220b}\flagstanza{\tiny\textenglish{...2.220b}}तं हि पतत्यर्थे विवेचयन् ॥ २२० ॥\&[\smallbreak]


	
	  \endgroup
	\leavevmode\marginnote{\textenglish{185/s}}

	  \pstart तमनुभूयमानात् पीतात् {\color{DodgerBlue3}“वि\edlabel{pvv.185-1}\footnote{\label{pvv.185-1}  १ उत्तरज्ञानेन पूर्व्वकं ।}वेचयन्”} भेदेन व्यवस्थापयन् प्रमाता {\color{DodgerBlue3}“अर्थ”} एव नीले {\color{DodgerBlue3}“पतति”} विवेचकत्वेन । परो\edlabel{pvv.185-2}\footnote{\label{pvv.185-2}  २ अतीतत्वात् ।}क्षं तदा नीलमर्थ एव । अपरोक्षतैव तु ज्ञानस्वभावः । अतो यद् विविच्यते तदज्ञानं । यज्ज्ञानं तन्न विवेच्यत एव । (२२०)
	\pend
      \label{div_pvv.2.221}\edlabel{div_pvv.2.221}
	  
	% new div opening: depth here is 2
	

	  \pstart तस्माद् (।)
	\pend
      
	  \bigskip
	  \begingroup
	  \large
	
	    
	    \stanza[\smallbreak]
	\label{pv.2.221}\edlabel{pv.2.221}\flagstanza{\tiny\textenglish{....2.221}}यद् यथा भासते ज्ञानं तत्तथैव प्रकाशते ।&इति नामैकभावः स्याच्चित्राकारस्य चेतसि ॥ २२१ ॥\&[\smallbreak]


	
	  \endgroup
	

	  \pstart {\color{DodgerBlue3}“यज्ज्ञानं यथा”} नीलाद्यात्मतया जातं {\color{DodgerBlue3}“सत्तथा”} भासते {\color{DodgerBlue3}“प्रकाशते”} । भासमानस्वभावत्वात् {\color{DodgerBlue3}“ज्ञानं”} तथा तेनैव स्वरूपेणानुभूयते सर्वेः प्रतिपत्तृभिः । न चोत्पन्नस्यापि ज्ञानस्य स्वप्रकाशकस्याविदितः कश्चिदाकारोस्ति {\color{DodgerBlue3}“इत्य”}शक्यविवेचनत्वात् तुल्ययोगक्षेमत्वात् सहप्रतिभासनियमात् । {\color{DodgerBlue3}“चित्र”}स्य नीलपीताद्या{\color{DodgerBlue3}“कारस्य चे”}तसि बुद्धावेकभावो नाम भवेत्तदा को दोषः ।\edlabel{pvv.185-3}\footnote{\label{pvv.185-3}  ३ न ह्येवं बाह्यश्चित्रोर्थस्तथाहि ।}(२२१)
	\pend
      \label{div_pvv.2.222}\edlabel{div_pvv.2.222}
	  
	% new div opening: depth here is 2
	
	  \bigskip
	  \begingroup
	  \large
	
	    
	    \stanza[\smallbreak]
	\label{pv.2.222a}\edlabel{pv.2.222a}\flagstanza{\tiny\textenglish{...2.222a}}पटादिरूपस्यैकत्वे तथा स्यादविवेकिता ।\&[\smallbreak]


	
	  \endgroup
	

	  \pstart ज्ञानवत् पटादिरूपस्य एकत्वेऽभ्युपगम्यमाने तथा ज्ञानस्येवाविवेकिताऽशक्यविवेचनत्वं स्यात् । न चास्ति (।) नीलादीनां ग्रहणाग्रहणभेदस्य दर्शनात् ।
	\pend
      

	  \pstart स्यादेतद् (।) अवयवा नीलाद्याः परस्परतोऽवयविनश्च भिन्नास्तेषां भेदाद्विवेकेन ग्रहणं । यस्त्वभिन्नोऽवयवी न तस्य विवेकेन ग्रहणमित्याह (।)
	\pend
      
	  \bigskip
	  \begingroup
	  \large
	
	    
	    \stanza[\smallbreak]
	\label{pv.2.222b}\edlabel{pv.2.222b}\flagstanza{\tiny\textenglish{...2.222b}}विवेकीनि निरस्यान्यदा विवेकि च नेक्षते ॥ २२२ ॥\&[\smallbreak]


	
	  \endgroup
	

	  \pstart विवेकीनि नीलाद्यवयवरूपाणि निरस्य पृथक् कृत्वा {\color{DodgerBlue3}“अन्यदा-विवेकिरूपञ्च नेक्षते”} ।\edlabel{pvv.185-4}\footnote{\label{pvv.185-4}  ४ पिहितेष्वप्यवयवेषु निरवयत्वादवयवी दृश्यते ।}दृश्यसम्मतमनुपलभ्यमानं कथमभ्युपगमार्ह । (२२२)
	\pend
      \label{div_pvv.2.223}\edlabel{div_pvv.2.223}
	  
	% new div opening: depth here is 2
	

	  \pstart यच्चोच्यते परमाणवः प्रत्येकम\edlabel{pvv.185-5}\footnote{\label{pvv.185-5}  ५ उद्योतकराद्यैः ।}तीन्द्रियत्वात् सञ्चिता अपि न ज्ञानगोचर इ\edlabel{pvv.185-6}\footnote{\label{pvv.185-6}  ६ नाप्युदकाभ्याहरणं विनावयविनमिति भावः ।}ति तत्राह(।)
	\pend
      
	  \bigskip
	  \begingroup
	  \large
	
	    
	    \stanza[\smallbreak]
	\label{pv.2.223}\edlabel{pv.2.223}\flagstanza{\tiny\textenglish{....2.223}}को वा विरोधो बहवः संजातातिशयाः (पृथक्)।&भवेयुः कारणं बुद्धेर्यदि नात्मेन्द्रियादिवत् ॥ २२३ ॥\&[\smallbreak]


	
	  \endgroup
	\leavevmode\marginnote{\textenglish{186/s}}\leavevmode\marginnote{\textenglish{36b/MA}}

	  \pstart {\color{DodgerBlue3}“यदि बहवः”} परमाणव उपसर्पणप्रत्ययात् {\color{DodgerBlue3}“संजा\edlabel{pvv.186-1}\footnote{\label{pvv.186-1}  १ योग्यदेशत्वाव्यवहितत्वादिना ।} तातिशया”} विज्ञानजननयोग्याः संहता उत्पन्ना स्वग्राहि कार्यञ्च {\color{DodgerBlue3}“बुद्धेः कारणं भवे”}युस्तदा {\color{DodgerBlue3}“को विरोध इन्द्रिया”}\edlabel{pvv.186-2}\footnote{\label{pvv.186-2}  २ अत्मेन्द्रियार्थसन्निकर्षाद् वैशेषिकाः । इन्द्रियार्थसन्निकर्षान्नैयायिकाः ।} दिवत् । इन्द्रियादयः प्रत्ये\edlabel{pvv.186-3}\footnote{\label{pvv.186-3}  ३ पृथगर्थः ।}कं न बुद्धेर्हेतुर्मिलितास्तु भवन्ति तद्वदणवोपि स्युः । न हि परेषामिवास्माकञ्च नित्यैकस्वभावा अणवः । ते हि यथाप्रत्ययमतीन्द्रियाः सन्त ऐन्द्रिया अपि स्युः ॥ (२२३)
	\pend
      \label{div_pvv.2.224}\edlabel{div_pvv.2.224}
	  
	% new div opening: depth here is 2
	

	  \pstart ननु हेतुत्वेपि कथमणवो ग्राह्या इत्याह (।)
	\pend
      
	  \bigskip
	  \begingroup
	  \large
	
	    
	    \stanza[\smallbreak]
	\label{pv.2.224a}\edlabel{pv.2.224a}\flagstanza{\tiny\textenglish{...2.224a}}हेतुभावादृते नान्या ग्राह्यता नाम काचन ॥\&[\smallbreak]


	
	  \endgroup
	

	  \pstart {\color{DodgerBlue3}“हेतुभावदृते विना ग्राह्यता नाम”} या प्रसिद्ध सा नान्या {\color{DodgerBlue3}“काचित्”} । अपि तु हेतुतैन ग्राह्यता ।
	\pend
      

	  \pstart एवन्तर्हीन्द्रियादिकमपि हेतुत्वाद् ग्राह्यं स्यादित्याह (।)
	\pend
      
	  \bigskip
	  \begingroup
	  \large
	
	    
	    \stanza[\smallbreak]
	\label{pv.2.224b}\edlabel{pv.2.224b}\flagstanza{\tiny\textenglish{...2.224b}}तत्र बुद्धिर्यदाकारा तस्यास्तद् ग्राह्यमुच्यते ॥ २२४ ॥\&[\smallbreak]


	
	  \endgroup
	

	  \pstart {\color{DodgerBlue3}“तत्र”} तेषु हेतुषु {\color{DodgerBlue3}“बुद्धिर्यदाकारा”} भवति {\color{DodgerBlue3}“तस्या”} बुद्धेस्तद् ग्राह्य{\color{DodgerBlue3}“मुच्यते”} अणुसञ्चयः । सैव च बुद्धिराकारमनुकरोति नेन्द्रियादेः । (२२४)
	\pend
      \label{div_pvv.2.225}\edlabel{div_pvv.2.225}
	  
	% new div opening: depth here is 2
	
	  \bigskip
	  \begingroup
	  \large
	
	    
	    \stanza[\smallbreak]
	\label{pv.2.225}\edlabel{pv.2.225}\flagstanza{\tiny\textenglish{....2.225}}कथं वाऽवयवी ग्राह्यः सकृत् स्वावयवैः सह ।&न हि गोप्रत्ययो दृष्टः सास्नादीनामदर्शने ॥ २२५ ॥\&[\smallbreak]


	
	  \endgroup
	

	  \pstart {\color{DodgerBlue3}“तत्कथं”} तद्ग्राह्यं । योप्याह (।) नानेकं {\color{DodgerBlue3}“सकृ”}द् गृह्यत इति । तन्मते{\color{DodgerBlue3}“ऽवयवी स्वावयवैः”} सास्नाककुद्लाङ्गूलादिभिः सह कथम्वा ग्राह्यो युक्तः । न गृह्यत एवेति चेत् । {\color{DodgerBlue3}“न हि गो”}रवयविनः {\color{DodgerBlue3}“प्रत्ययः सास्नादीनाम”}वयवा{\color{DodgerBlue3}“नामदर्शने”} क्वापि {\color{DodgerBlue3}“दृष्टः”}\edlabel{pvv.186-4}\footnote{\label{pvv.186-4}  ४ परस्परमनुपाधिभूतानेकद्रव्यग्रहणं नेष्यते न पुनर्गुणप्रधानभूतस्य । विशेषणभूताश्च सास्नादयोऽवयवा गोर्द्रव्यस्येति तेषां सहग्रहो युक्तः । न त्वणूनामने वंत्वात् ।}। (२२५)
	\pend
      \label{div_pvv.2.226}\edlabel{div_pvv.2.226}
	  
	% new div opening: depth here is 2
	
	  \bigskip
	  \begingroup
	  \large
	
	    
	    \stanza[\smallbreak]
	\label{pv.2.226}\edlabel{pv.2.226}\flagstanza{\tiny\textenglish{....2.226}}गुणप्रधानाधिगमः सहाप्यभिमतो यदि ।&सम्पूर्णाङ्गो न गृह्येत सकृन्नापि गुणादिमान् ॥ २२६ ॥\&[\smallbreak]


	
	  \endgroup
	

	  \pstart {\color{DodgerBlue3}“यदि सहाप्यभिमतो गुण”}प्रधानयोर्विशेषणविशेष्ययोर{\color{DodgerBlue3}“धिगमो”} नोपाधीनामन्योन्यं विशेषणविशेष्यभूतानां द्रव्यानां (? णां) वा तादृशानां (? णां) । एवन्तर्हि विषाणी \leavevmode\marginnote{\textenglish{187/s}} सास्नादिमानिति वा यदा गृह्यते तदा तेनैवावयवेन सम्बन्धव्यवसायादितरावयवसम्बन्धानवसायात् {\color{DodgerBlue3}“संपूर्णाङ्गो”}ऽवय\edlabel{pvv.187-1}\footnote{\label{pvv.187-1}  १ न हि विषाणेनान्येप्युपाधयोऽवच्छिद्यन्ते तेषां विशेष्यत्वप्रसङ्गात् ।}वी {\color{DodgerBlue3}“न गृह्येत । सकृद्”} गृह्यते च । {\color{DodgerBlue3}“नापि गुणादिमान्”} गृह्येत । विषाणी गौरिति बुद्ध्या विषाणविशिष्टो गौर्विषयीकृतो न त्वन्ये गुणकर्मसामान्यादयः । ततश्च नीलादिरूपं गुणः परिस्पन्दादि च कर्म्म । वस्तुत्वादि च सामान्यं न गृह्येत । दृष्टविरुद्धञ्चैतत् । (२२६)
	\pend
      \label{div_pvv.2.227}\edlabel{div_pvv.2.227}
	  
	% new div opening: depth here is 2
	

	  \pstart सर्व्वेषां गुणकर्म्मसामान्यावयवादीनां वस्तुतो विशेषणत्वात् {\color{DodgerBlue3}“सर्व्वग्रहणमिति”} चेत् । आह (।)
	\pend
      
	  \bigskip
	  \begingroup
	  \large
	
	    
	    \stanza[\smallbreak]
	\label{pv.2.227}\edlabel{pv.2.227}\flagstanza{\tiny\textenglish{....2.227}}विवक्षा परतन्त्रत्वात् विशेषणविशेष्ययोः ।&यदङ्गभावेनोपात्तन्तत्तेनैव हि गृह्यते ॥ २२७ ॥\&[\smallbreak]


	
	  \endgroup
	

	  \pstart {\color{DodgerBlue3}“विशेषणविशेष्ययोर्व्विवक्षापरतन्त्रत्वान्”} पुरुषेच्छानुरोधात् न पारमार्थिकत्वं । तथा विषाणी गौरिति गोर्विषाणमित्यादौ विपर्ययो विशेषणविशेष्ययोः प्रयोक्तुरिच्छावशेन दृश्यते । तस्माद्यदेव {\color{DodgerBlue3}“ह्यङ्गभावेन”} विशेषणभावेन स्वमनीषिकया प्रतिपादयित्रोपात्तं {\color{DodgerBlue3}“तेनैव”} विशेषणेन विशिष्टं तद्विवक्षितं {\color{DodgerBlue3}“गृह्यते”} न तदितरैः । तेषामविवक्षितत्वेनाविवक्षितत्वात् । ततश्च न संपूर्ण्णाङ्गो नापि गुणादिमान् गृह्येत । (२२७)
	\pend
      \label{div_pvv.2.228_2.229}\edlabel{div_pvv.2.228_2.229}
	  
	% new div opening: depth here is 2
	

	  \pstart किञ्च (।)
	\pend
      
	  \bigskip
	  \begingroup
	  \large
	
	    
	    \stanza[\smallbreak]
	\label{pv.2.228}\edlabel{pv.2.228}\flagstanza{\tiny\textenglish{....2.228}}स्वतो वस्त्वन्तराभेदाद् गुणादेर्भेदकस्य च ।&अग्रहादेकबुद्धिः स्यात् पश्यतोपि परापरम् ॥ २२८ ॥\&[\smallbreak]


	
	  \endgroup
	
	  \bigskip
	  \begingroup
	  \large
	
	    
	    \stanza[\smallbreak]
	\label{pv.2.229}\edlabel{pv.2.229}\flagstanza{\tiny\textenglish{....2.229}}गुणादिभेदग्रहणान्नानात्वप्रतिपद्यदि ।&अस्तु नाम तथाप्येषां भवेत् सम्बन्धिसंकरः ॥ २२९ ॥\&[\smallbreak]


	
	  \endgroup
	

	  \pstart वस्तुनः {\color{DodgerBlue3}“स्वतो वस्त्वन्तराद् भेदाभावात्”} । वदन्ति हि स्वतो हि न गौर्नागौर्ग्गोत्वयोगात् गौः । तथा स्वतो हि शुक्लो नाशुक्लः शुक्लत्वयोगात् शुक्ल इत्यादि । {\color{DodgerBlue3}“भेदकस्य च गुणा\edlabel{pvv.187-2}\footnote{\label{pvv.187-2}  २ सकृदनेकार्थप्रतीतिप्रसङ्गात् ।}”}देरग्रहात् पदार्थैः सह । ततश्च {\color{DodgerBlue3}“परापरम”}र्थजातं {\color{DodgerBlue3}“पश्यतोप्येक”}पदार्थत्व{\color{DodgerBlue3}“बुद्धिः स्यात्”} । न हि द्रव्याणामन्योन्यस्य भेदका गुण\edlabel{pvv.187-3}\footnote{\label{pvv.187-3}  ३ घटदर्शनेपि पटबोध एव विनोपाधिं पदार्थाग्रहात् ।}जात्यादयस्तैः\edlabel{pvv.187-4}\footnote{\label{pvv.187-4}  ४ दीर्घत्वादि ।} सह गृह्यन्ते येन भेदबुद्धिः स्यात् । द्रव्ये गृहीते पश्चाद् गुणादीनां भेदानां विशेषाणां नानात्वग्रहणं द्रव्या\edlabel{pvv.187-5}\footnote{\label{pvv.187-5}  ५ नाना सामान्यादीनि नैकस्येति तद्विशिष्टानां ।}णां यदीष्यते । {\color{DodgerBlue3}“अस्तु नाम”} पश्चाद् गुणादिग्रहणं । {\color{DodgerBlue3}“तथाप्येषां”} \leavevmode\marginnote{\textenglish{188/s}} गुणकर्म्मसामान्यादीनां {\color{DodgerBlue3}“सम्बन्धिनः सांकर्यं”} स्यात् । स्वतो हि द्रव्यं भेदान्नोपलभ्यते । पश्चादुपलभ्यमानैस्तु गुणादिभिः संबन्धितयावगतैर्भेदेन तद् व्यवस्थापनीयं । ततश्चैकत्रैक\edlabel{pvv.188-1}\footnote{\label{pvv.188-1}  १ भिन्नद्रव्यसमवायान्यपि संकलनात् सम्भिन्नग्रहतः ।} द्रव्ये गुणादयः सम्बन्धितया प्रतीयेरन् ॥ भिन्नदेशानां गुणानां भिन्नदेशेषु योजनान्न सम्बन्धिसांकर्यमिति चेत् । न (।) देशभेदस्यापि विशेषणत्वात् सोप्येकस्येति स्यात् । एकस्य विरुद्धधर्मायोगात् प्रतीयमानार्थाः पश्चात् प्रतीयमानै\leavevmode\marginnote{\textenglish{37a/MA}} रूपाधिभिर्भिन्ना व्यवस्थाप्यन्त इति चेत् । नन्वेवमपि स्वतोऽर्थानां भेदाभावात् । भेदनिबन्धनैश्चोपाधिभिः सह वेदनाभावान्नाध्यक्षसिद्धा भेदव्यवस्थितिः स्यात् । किन्तु विरुद्धोपाधिसम्बन्धान्यथानुपपत्त्या\edlabel{pvv.188-2}\footnote{\label{pvv.188-2}  २ भेदस्थितिः ।} कल्पनीया । तथा च प्रत्यक्षविरोधः\edlabel{pvv.188-3}\footnote{\label{pvv.188-3}  ३ अनध्यक्षत्वस्य ।} ।
	\pend
      

	  \pstart अथ दृष्टेर्थे पश्चात्तनमध्यक्षमुपाधिग्राहकं तान् योजयेद् भेदग्राहकं । कथमदृश्यमाने योजनोपाधीनां प्रत्यक्षकृता विकल्पकृतैव तु स्यात् । तथा च व्यक्तं न भेददर्शनं स्यात् । विकल्पविषयस्यास्फुटत्वात् । ृष्टेऽर्थे पश्चादुपाधिः प्रतीयमानस्तस्यैव दृष्ट इति चेत् । तस्येति किमुच्यते । न तावदनेकासूपाधिग्रहणात् पूर्व्वमनेकत्वस्याप्रतीतेः । एकं चेत् कथमुपाधिभिर्भेदनीयं । अनवधृतैकानेकभावं वस्तुमात्रं तदिति चेत् । तत्र तर्हि देशभेदादयोप्युपाधयः पश्चादुपलभ्यमाना योज्यन्ते । तदनन्तरं दृष्टत्वात् तत\edlabel{pvv.188-4}\footnote{\label{pvv.188-4}  ४ उपाधीन् । पूर्व्वके नष्टे । देशभेदादियोगे सति ।}श्चानिवार्यः सम्बन्धिसंकरप्रस\edlabel{pvv.188-5}\footnote{\label{pvv.188-5}  ५ अनेन क्रमेण ।}ङ्गः । (२२८, २२९)
	\pend
      \label{div_pvv.2.230}\edlabel{div_pvv.2.230}
	  
	% new div opening: depth here is 2
	

	  \pstart सां ख्य\edlabel{pvv.188-6}\footnote{\label{pvv.188-6}  ६ येनैकरूपत (ा) इष्टा ।} मतेपि (।)
	\pend
      
	  \bigskip
	  \begingroup
	  \large
	
	    
	    \stanza[\smallbreak]
	\label{pv.2.230}\edlabel{pv.2.230}\flagstanza{\tiny\textenglish{....2.230}}शब्दादीनामनेकत्वात् सिद्धोनेकग्रहः सकृत् ।&सन्निवेशग्रहायोगादग्रहे सन्निवेशिनाम् ॥ २३० ॥\&[\smallbreak]


	
	  \endgroup
	

	  \pstart {\color{DodgerBlue3}“शब्दादीनां”} सु\edlabel{pvv.188-7}\footnote{\label{pvv.188-7}  ७ श्रोत्रादीनां । सत्त्वरजस्तमः ।}खदुःखमोहात्मकतया अनेकत्वात् शब्दादिग्रहे {\color{DodgerBlue3}“सकृदनेकग्रहः सिद्धः । संन्निवेशिना\edlabel{pvv.188-8}\footnote{\label{pvv.188-8}  ८ सुखादीनां ।}मग्रहे सन्निवे\edlabel{pvv.188-9}\footnote{\label{pvv.188-9}  ९ शब्दादेः ।}शस्य ग्रहायोगात्”} ।\edlabel{pvv.188-10}\footnote{\label{pvv.188-10}  १० दृष्टान्तः ।}न ह्यङ्गुल्यग्रहणे मुष्टिग्रहणं । (२३०)
	\pend
      \label{div_pvv.2.231}\edlabel{div_pvv.2.231}
	  
	% new div opening: depth here is 2
	

	  \begin{center}%% label @type='head'
	\textbf{(ग. (क) कल्पनापोढत्वे धर्मधर्म्यादिसंगतिः)}
	\end{center}
	

	  \pstart यदि प्रत्यक्षमविकल्पं तदा कथं धर्म्मधर्म्म्यादिग्रहणमित्याह (।)
	\pend
      
	  \bigskip
	  \begingroup
	  \large
	
	    
	    \stanza[\smallbreak]
	\label{pv.2-231}\edlabel{pv.2-231}\flagstanza{\tiny\textenglish{....2-231}}सर्वतो विनिवृत्तस्य विनिवृत्तिर्यतो यतः ।&तद् भेदोन्नीतभेदा सा धर्म्मिणोऽनेकरूपता ॥ २३१ ॥\&[\smallbreak]


	
	  \endgroup
	\leavevmode\marginnote{\textenglish{189/s}}

	  \pstart {\color{DodgerBlue3}“सर्व्वतः\edlabel{pvv.189-1}\footnote{\label{pvv.189-1}  १ स्वार्थे सामान्यगोचरं व्याख्याय \href{http://http://sarit.indology.info/?cref=ps.1.5}{[प्रमाण] समुच्चये} । “
	    \begin{verse}
	धर्म्मिणो नैकरूपस्य नेन्द्रियात् सर्व्वथा गतिः ।\\
	    स्वसम्वेद्यन्त्वानिर्द्देश्यं रूपमिन्द्रियगोचर\\
	    
	    \end{verse}
	  ” इति व्याचष्टे ।}”} परस्मा{\color{DodgerBlue3}“द्विनिवृत्तस्या”}र्थस्य {\color{DodgerBlue3}“यतो यतः”} परस्मा{\color{DodgerBlue3}“द्विनिवृत्तस्तैर्भेदै”}र्व्यावृत्तिभिरुन्नीता साऽ{\color{DodgerBlue3}“नेकरूपता”} धर्म्मिधर्म्मात्मकतया {\color{DodgerBlue3}“धर्म्मिणो”}ऽर्थस्य । सर्व्वंतो व्यावृत्ते प्रत्यक्षेण गृहीते वस्तुनि तद्व्यावृत्त्यनुकारिणो विकल्पा (:) प्रत्यक्षदृष्टत्वेन धर्म्मिधर्म्मभावं व्यवस्थापयन्तीति प्रत्यक्षकृतः स उच्यते । न तु प्रत्यक्षप्रतिभासमानत्वात् । (२३१)
	\pend
      \label{div_pvv.2.232}\edlabel{div_pvv.2.232}
	  
	% new div opening: depth here is 2
	

	  \pstart तथा हि (।)
	\pend
      
	  \bigskip
	  \begingroup
	  \large
	
	    
	    \stanza[\smallbreak]
	\label{pv.2.232}\edlabel{pv.2.232}\flagstanza{\tiny\textenglish{....2.232}}ते कल्पिता रूपभेदाद् निर्व्विकल्पस्य चेतसः ।&न विचित्रस्य चित्राभाः कादाचित्कस्य गोचरः ॥ २३२ ॥\&[\smallbreak]


	
	  \endgroup
	

	  \pstart {\color{DodgerBlue3}“ते”} धर्मिधर्मादयो {\color{DodgerBlue3}“रूपभेदाद्”} बुद्ध्याकारविशेषा विजातीयव्यावृत्त्याश्रयेण {\color{DodgerBlue3}“कल्पिता”} विचित्राभा वस्तुबलभावित्वात् {\color{DodgerBlue3}“का\edlabel{pvv.189-2}\footnote{\label{pvv.189-2}  २ कादाचित्कत्वादर्थसन्निधिसापेक्षत्वमाहान्यथा कल्पितधर्मभेदज्ञेयत्वादिविषयत्वे सर्व्वदा स्यात् । एतेन सामान्यं वस्तुसत्प्रतिषिद्धं ।}दाचित्कस्य विचित्रस्य”} धर्म्मिधर्म्मभेदप्रतिभासरहितस्य {\color{DodgerBlue3}“निर्व्विकल्पस्य चेतसः”} प्रत्यक्षस्य {\color{DodgerBlue3}“न गोचरः”} । प्रत्यक्षं हि वस्तुसामर्थ्योत्पन्नं तदाकारमनुकुर्यात् न विकल्पितं । (२३२)
	\pend
      \label{div_pvv.2.233}\edlabel{div_pvv.2.233}
	  
	% new div opening: depth here is 2
	

	  \begin{center}%% label @type='head'
	\textbf{(ख) शब्दविकल्पविषयः सामान्यम्}
	\end{center}
	

	  \pstart भवतु वा वस्त्वेव सामान्यं तथापि नासौ शब्दविकल्पविषय इत्याह (।)
	\pend
      
	  \bigskip
	  \begingroup
	  \large
	
	    
	    \stanza[\smallbreak]
	\label{pv.2.233}\edlabel{pv.2.233}\flagstanza{\tiny\textenglish{....2.233}}यद्यप्यस्ति सितत्वादि यादृगिन्द्रियगोचरः ।&न सोभिधीयते शब्दैर्ज्ञानयो रूपभेदतः ॥ २३३ ॥\&[\smallbreak]


	
	  \endgroup
	

	  \pstart {\color{DodgerBlue3}“यद्यपि सितत्वादि”} सामान्यमस्ति पटादौ धर्म्मिणि {\color{DodgerBlue3}“यादृक्”} सितत्वादिविशदाकार {\color{DodgerBlue3}“इन्द्रियस्य गोचरः । स”} इन्द्रियज्ञानगोचरोऽर्थोऽ{\color{DodgerBlue3}“भिधीयते न शब्दैः । ज्ञानयो-”} रिन्द्रियशब्दजनितयो {\color{DodgerBlue3}“रूप”}स्याकारस्य स्फुटास्फुटत्वेन {\color{DodgerBlue3}“भेदतः”} (२३३)
	\pend
      \label{div_pvv.2.234}\edlabel{div_pvv.2.234}
	  
	% new div opening: depth here is 2
	
	  \bigskip
	  \begingroup
	  \large
	
	    
	    \stanza[\smallbreak]
	\label{pv.2.234}\edlabel{pv.2.234}\flagstanza{\tiny\textenglish{....2.234}}एकार्थत्वेपि बुद्धीनां नानाश्रयतया स चेत् ।&श्रोत्रादिचित्तानीदानीं भिन्नार्थानीति तत्कुतः ॥ २३४ ॥\&[\smallbreak]


	
	  \endgroup
	

	  \pstart {\color{DodgerBlue3}“बुद्धीना”}मिन्द्रियशब्दजनिताना{\color{DodgerBlue3}“मेकार्थ”}त्वे एकवि\edlabel{pvv.189-3}\footnote{\label{pvv.189-3}  ३ सामान्यमात्र ।}षयत्वे{\color{DodgerBlue3}“पि नानाश्र\edlabel{pvv.189-4}\footnote{\label{pvv.189-4}  ४ चक्षुरादिमनसी ।}यतया”} कारणभेदा{\color{DodgerBlue3}“त्म”}कस्य आकारभेद{\color{DodgerBlue3}“श्चेत् । इदानी”}मेवं स्थितौ {\color{DodgerBlue3}“श्रोत्रा”}दीन्द्रिय{\color{DodgerBlue3}“चित्तानि”} \leavevmode\marginnote{\textenglish{190/s}} {\color{DodgerBlue3}“भिन्नार्थानि”} शब्दरूपगन्धादिभिन्नविषयाणीति व्यपदिश्यके (।) {\color{DodgerBlue3}“तत्कुतः”} प्रमाणादवधारितं । तान्यपि चितान्यभिन्नविषयत्वेपीन्द्रियाणामाश्रयभूतानां भेदाद् भिन्नाकाराणीति किन्न कल्प्यते । (२३४)
	\pend
      \label{div_pvv.2.235}\edlabel{div_pvv.2.235}
	  
	% new div opening: depth here is 2
	

	  \pstart किञ्च (।)
	\pend
      
	  \bigskip
	  \begingroup
	  \large
	
	    
	    \stanza[\smallbreak]
	\label{pv.2.235}\edlabel{pv.2.235}\flagstanza{\tiny\textenglish{....2.235}}जातो नामाश्रयोन्योन्यः चेतसां तस्य वस्तुनः ।&एकस्यैव कुतो रूपं भिन्नाकारावभासि तत् ॥ २३५ ॥\&[\smallbreak]


	
	  \endgroup
	

	  \pstart {\color{DodgerBlue3}“सामान्यादिचेतसामाश्रयः”} कारणमिन्द्रियं शब्दश्चेति {\color{DodgerBlue3}“अन्योन्यो नाम जातः”} । तथा पि सामान्यादे{\color{DodgerBlue3}“र्व्वस्तुन एकस्यैव रूपं तत्कुतो भिन्नाकारावभासि”} स्फुटास्फुटावभासि स्फुटास्फुटप्रतिभासं । न हि स्वरूपेण भासमानमेकं भिन्नप्रतिभासं युक्तम् । (२३५)
	\pend
      \label{div_pvv.2.236}\edlabel{div_pvv.2.236}
	  
	% new div opening: depth here is 2
	

	  \pstart यदप्युच्यते परैः (।) शाब्देन्द्रियज्ञानयोर्यादि नैकविषयत्वं तदा विषाणादिमन्तमर्थं गौरिति शब्दात्प्रतीत्य कालान्तरे व्यक्तिविशेषं दृष्टवतोऽयमसौ शब्दात् प्राङ्मया प्रतीतो गौरिति प्रत्यभिज्ञानमेकताध्यवसायि यदुत्पद्यते तन्न स्यादिति । तत्राह (।)
	\pend
      
	  \bigskip
	  \begingroup
	  \large
	
	    
	    \stanza[\smallbreak]
	\label{pv.2.236}\edlabel{pv.2.236}\flagstanza{\tiny\textenglish{....2.236}}वृत्तेर्दृश्यापरामर्शेनाभिधानविकल्पयोः ।&दर्शनात् प्रत्यभिज्ञानं गवादीनां निवारितम् ॥ २३६ ॥\&[\smallbreak]


	
	  \endgroup
	\leavevmode\marginnote{\textenglish{37b/MA}}

	  \pstart {\color{DodgerBlue3}“अभिधानविकल्प”}योर्दृ श्यस्याध्यक्षविषयस्या{\color{DodgerBlue3}“परामर्शेन”} विषयीकरणेन {\color{DodgerBlue3}“वृत्तेः”} पश्चाद् {\color{DodgerBlue3}“दर्शनात्”} स एवायं शब्दनिर्दिष्टो गौरिति {\color{DodgerBlue3}“प्रत्य\edlabel{pvv.190-1}\footnote{\label{pvv.190-1}  १ मिथ्यावत् ।} भिज्ञानं गवादीनां”} यदिष्यते {\color{DodgerBlue3}“तन्निवारितं”} बोद्धव्यं । इन्द्रियशब्दज्ञानयोर्भिन्नाकारत्वेनैव विषयत्त्वाभावात् कथं त\edlabel{pvv.190-2}\footnote{\label{pvv.190-2}  २ पूर्व्वापरैकता इन्द्रियज्ञानोत्तरभावि ।}देकताव्यवसायि वस्तुविषयं स्यात् ।(२३६)
	\pend
      \label{div_pvv.2.237}\edlabel{div_pvv.2.237}
	  
	% new div opening: depth here is 2
	
	  \bigskip
	  \begingroup
	  \large
	
	    
	    \stanza[\smallbreak]
	\label{pv.2.237}\edlabel{pv.2.237}\flagstanza{\tiny\textenglish{....2.237}}अन्वयाच्चानुमानं यदभिधानविकल्पयोः ।&दृश्ये गवादौ जात्यादेस्तदप्येतेन दूषितम् ॥ २३७ ॥\&[\smallbreak]


	
	  \endgroup
	

	  \pstart यच्च {\color{DodgerBlue3}“दृश्ये गवादा”}वनेकत्राभिन्नाकारयोर{\color{DodgerBlue3}“भिधानविकल्पयोरन्वयादनुवृत्तेर्जात्यादेरनुमानं परैरुच्यते तदप्येते”}नाभिधानविकल्पयोर्दृश्यत्वापरामर्शेन हेतुना {\color{DodgerBlue3}“दूषितं”} बोद्धव्यं । न ह्यन्वयिनावपि शब्दविकल्पौ वस्तु स्पृशतः । तत्कथं ताभ्यां वस्तुनः सामान्यस्य सिद्धिः । (२३७)
	\pend
      \label{div_pvv.2.238}\edlabel{div_pvv.2.238}
	  
	% new div opening: depth here is 2
	

	  \pstart यदि नास्ति सामान्यं तथा कथमस्तु प्रत्यभिज्ञानमित्याह (।)
	\pend
      \leavevmode\marginnote{\textenglish{191/s}}
	  \bigskip
	  \begingroup
	  \large
	
	    
	    \stanza[\smallbreak]
	\label{pv.2.238}\edlabel{pv.2.238}\flagstanza{\tiny\textenglish{....2.238}}दर्शनान्येव भिन्नान्यप्येकां कुर्वन्ति कल्पनाम् ।&प्रत्यभिज्ञानसंख्यातां स्वभावेनेति वर्णितम् ॥ २३८ ॥\&[\smallbreak]


	
	  \endgroup
	

	  \pstart {\color{DodgerBlue3}“दर्शनानी”}न्द्रियज्ञा{\color{DodgerBlue3}“नान्येव भिन्नानि”} नानाक्षणविषयाण्यनेका{\color{DodgerBlue3}“न्यपि स्वभावेन”} प्रत्यभिज्ञानकारणस्वरूपेण स्वकारणप्रसूतेन कल्पनात्मेकत्वाध्यवसायिनीं {\color{DodgerBlue3}“प्रत्यभिज्ञानसंख्यातां”} प्रत्यभिज्ञाननाम्ना प्रसिद्धां {\color{DodgerBlue3}“कुर्वन्तीति वर्ण्णितं”} प्राक्\edlabel{pvv.191-1}\footnote{\label{pvv.191-1}  १ प्रथमपरिच्छेदे ।} ।
	\pend
      

	  \pstart तस्मात्स्थितमेतत् प्रत्यक्षमनिर्देश्यत्वादविकल्पमिति । उक्तमिन्द्रियप्रत्यक्षं ॥ XX ॥ (२३८)
	\pend
      \label{div_pvv.2.239}\edlabel{div_pvv.2.239}
	  
	% new div opening: depth here is 2
	

	  \begin{center}%% label @type='head'
	\textbf{(२) मानसप्रत्यक्षम्}
	\end{center}
	

	  \pstart मानसमाख्यातुमाह (।) तच्चेन्द्रियज्ञानानन्तरमिष्टं । तेन सहैकविषयं भिन्नविषयं वा स्यात् । उभयथापि तु दोष इत्याह (।)
	\pend
      
	  \bigskip
	  \begingroup
	  \large
	
	    
	    \stanza[\smallbreak]
	\label{pv.2.239}\edlabel{pv.2.239}\flagstanza{\tiny\textenglish{....2.239}}पूर्व्वानुभूतग्रहणे मानसस्याप्रमाणता ।&अदृष्टग्रहणेन्धादेरपि स्यादर्थदर्शनम् ॥ २३९ ॥\&[\smallbreak]


	
	  \endgroup
	

	  \pstart {\color{DodgerBlue3}“पूर्व्वानुभूत”}स्येन्द्रियज्ञानगृहीतस्य {\color{DodgerBlue3}“ग्रहणे मानसस्य”} स्वीक्रियमाणे{\color{DodgerBlue3}“ऽप्रमाणता”} स्यात् । अज्ञाता(र्थ) प्रकाशस्य प्रमाणत्वात् । इन्द्रियज्ञाना{\color{DodgerBlue3}“दृष्टस्य ग्रहणे”} पुनरिष्यमाणे{\color{DodgerBlue3}“ऽन्धादेरपि स्यादर्थ”}स्य रूपादे{\color{DodgerBlue3}“र्दर्शनं”} । (२३९)
	\pend
      \label{div_pvv.2.240}\edlabel{div_pvv.2.240}
	  
	% new div opening: depth here is 2
	

	  \pstart इन्द्रियज्ञानानुभूतविषयत्वेपि द्वौ विकल्पौ सोऽर्थः क्षणिको न वा ।
	\pend
      
	  \bigskip
	  \begingroup
	  \large
	
	    
	    \stanza[\smallbreak]
	\label{pv.2.240}\edlabel{pv.2.240}\flagstanza{\tiny\textenglish{....2.240}}क्षणिकत्वादतीतस्य दर्शने च न सम्भवः ।&वाच्यमक्षणिकत्वे स्याल्लक्षणं सविशेषणम् ॥ २४० ॥\&[\smallbreak]


	
	  \endgroup
	

	  \pstart प्रथमपक्ष इन्द्रियज्ञानविषयस्यार्थ\edlabel{pvv.191-2}\footnote{\label{pvv.191-2}  २ . . . . .... सौत्रान्तिकस्य ।}स्य \edlabel{pvv.191-3}\footnote{\label{pvv.191-3}  ३ “मानसञ्चार्थरागादि स्वसम्वित्तिरकल्पिका” “योगिनां गुरुनिर्देशाव्यतिभिन्नार्थमात्रदृगि” तिसवृत्ति व्याख्यातुं परोक्तदूषणं चतुःश्लोक्याह । वृत्तिर्मानसमपि रूपादिविषयमविकल्पकमनुभवाकारप्रवृत्तमिति ।}पूर्व्वकालस्य सहभु\edlabel{pvv.191-4}\footnote{\label{pvv.191-4}  ४ वैशेषिकस्तुल्यं करामलवद्विषयविषयित्वमाह ।}वो वा {\color{DodgerBlue3}“क्षणिकत्वादतीतस्य”} न\edlabel{pvv.191-5}\footnote{\label{pvv.191-5}  ५ वैभाष्यस्य स ज्ञानविषयनिरोधः ।}ष्टस्य मानसेन {\color{DodgerBlue3}“दर्शने”} दर्शनस्य {\color{DodgerBlue3}“च न सम्भवो”}स्ति । {\color{DodgerBlue3}“अक्षणिकत्वे”} वा अधिगतार्थाधिगन्तृत्वादप्रमाणं स्यात् । अथ मानसमधिगतार्थाधिगन्तृप्रमाणमिष्यते तदापि मानसस्य {\color{DodgerBlue3}“सविशेषणं लक्षणं वाच्यं स्यात्”} । यथाधिगतविषयत्वेपि “कल्पनापोढमभ्रान्त” मानसं प्रत्यक्षमिति (। २४०)
	\pend
      \label{div_pvv.2.241}\edlabel{div_pvv.2.241}
	  
	% new div opening: depth here is 2
	\leavevmode\marginnote{\textenglish{192/s}}

	  \pstart किञ्च (।)
	\pend
      
	  \bigskip
	  \begingroup
	  \large
	
	    
	    \stanza[\smallbreak]
	\label{pv.2.241}\edlabel{pv.2.241}\flagstanza{\tiny\textenglish{....2.241}}निष्पादितक्रिये किञ्चिद् विशेषमसमादधत् ।&कर्म्मण्यैन्द्रियमन्यद् वा साधनं किमितीष्यते ॥ २४१ ॥\&[\smallbreak]


	
	  \endgroup
	

	  \pstart {\color{DodgerBlue3}“निष्पादिता क्रिया”} यस्मिन् तत्र {\color{DodgerBlue3}“कर्मणि विशेषं कञ्चिदसमादधत् ऐन्द्रियमि”}न्द्रियज्ञान{\color{DodgerBlue3}“मन्यद्वा”} प\edlabel{pvv.192-1}\footnote{\label{pvv.192-1}  १ मानसादि ।}रश्वादि {\color{DodgerBlue3}“साधनं किमितीष्यते”} (।) क्रियानिर्व्वर्तनं हि साधनव्यापारः । तच्चेन्निष्पन्नं किमन्यत् कुर्व्वत्तत् साधनं स्यात् । (२४१)
	\pend
      \label{div_pvv.2.242}\edlabel{div_pvv.2.242}
	  
	% new div opening: depth here is 2
	

	  \pstart अपि च (।)
	\pend
      
	  \bigskip
	  \begingroup
	  \large
	
	    
	    \stanza[\smallbreak]
	\label{pv.2.242}\edlabel{pv.2.242}\flagstanza{\tiny\textenglish{....2.242}}सकृद् भावश्च सर्व्वासां धियां तद्भावजन्मनाम् ।&अन्यैरकार्यभेदस्य तदपेक्षाविरोधतः ॥ २४२ ॥\&[\smallbreak]


	
	  \endgroup
	

	  \pstart तस्मात् स्थिराद् {\color{DodgerBlue3}“भावाज्जन्म”} यासान्तासां {\color{DodgerBlue3}“सर्व्वासां धियां सकृद् भावश्च स्यात्”} । न क्रमभावः । क्रमिसहकार्यपेक्षया क्रमेण स्थिरोप्यर्थः करोति बुद्धीरिति चेत् । अतो{\color{DodgerBlue3}“ऽन्यैः”} सहकारिभिर{\color{DodgerBlue3}“कार्यो भेदो”} विशेषो यस्य स्थिरैकरूपस्य तस्य {\color{DodgerBlue3}“तदपेक्षाया विरोधतः”} ॥ (२४२)
	\pend
      \label{div_pvv.2.243}\edlabel{div_pvv.2.243}
	  
	% new div opening: depth here is 2
	

	  \pstart यत एवं (।)
	\pend
      
	  \bigskip
	  \begingroup
	  \large
	
	    
	    \stanza[\smallbreak]
	\label{pv.2.243}\edlabel{pv.2.243}\flagstanza{\tiny\textenglish{....2.243}}तस्मादिन्द्रियविज्ञानान्तरप्रत्ययोद्भवः ।&मनोन्यमेव गृह्णाति विषयं नान्धदृक् ततः ॥ २४३ ॥\&[\smallbreak]


	
	  \endgroup
	

	  \pstart {\color{DodgerBlue3}“तस्मादिन्द्रियविज्ञान”}मेवानन्तर{\color{DodgerBlue3}“प्रत्यय”}स्तस्मा{\color{DodgerBlue3}“दुद्भवो”} यस्य त{\color{DodgerBlue3}“न्मनो”} मानसं प्रत्यक्षं । इन्द्रियप्रत्यक्षग्राह्य (ा) द्विषया{\color{DodgerBlue3}“दन्यमेव विषयं गृह्णाति । तत”} इन्द्रियज्ञानजन्यत्वात् मानसस्यान्धानां चक्षुर्व्विज्ञानविकलानां दृग् दर्शनं रूपस्य न भवति । गृहीतग्राहित्वं च विषयान्तरग्रहणादपास्तं (। २४३)
	\pend
      \label{div_pvv.2.244}\edlabel{div_pvv.2.244}
	  
	% new div opening: depth here is 2
	

	  \pstart यद्यन्यविषयग्राहकं मानसं तदा भूतभविष्यद्ग्राहकमपि स्यादित्याह (।)
	\pend
      
	  \bigskip
	  \begingroup
	  \large
	
	    
	    \stanza[\smallbreak]
	\label{pv.2.244}\edlabel{pv.2.244}\flagstanza{\tiny\textenglish{....2.244}}स्वार्थान्वयार्थापेक्षैव हेतुरिन्द्रियजा मतिः ।&ततोन्यग्रहणेप्यस्य नियतग्राह्यता मता ॥ २४४ ॥\&[\smallbreak]


	
	  \endgroup
	

	  \pstart स्वार्थः स्वकीयो विषयस्तस्माद{\color{DodgerBlue3}“न्वय”} उत्पादो यस्यार्थस्य तद{\color{DodgerBlue3}“पेक्षैव इन्द्रियजा मति”}र्मनोविज्ञानस्य {\color{DodgerBlue3}“हेतु”}रिष्यते । {\color{DodgerBlue3}“ततोऽन्य”}स्य विषयस्य {\color{DodgerBlue3}“ग्रहणेपि”} मनोविज्ञानस्येष्यमाणे नियत इन्द्रियज्ञानग्राह्योपादेयक्षण एव ग्राह्यो यस्य तद्भावो {\color{DodgerBlue3}“नियतग्राह्यता सा मता”}\edlabel{pvv.192-2}\footnote{\label{pvv.192-2}  २ अभिधर्मेऽस्ति मनोविज्ञानसमङ्गी तु नीलमिदमिति च, मनो द्विधा । सविकल्पं निर्व्विकल्पञ्च तु तन्मानसं प्रत्यक्षं विमतिनिरासाय व्युत्पादितं ।}। (२४४)
	\pend
      \label{div_pvv.2.245}\edlabel{div_pvv.2.245}
	  
	% new div opening: depth here is 2
	\leavevmode\marginnote{\textenglish{193/s}}

	  \pstart ननु स्वज्ञानेन स्वालम्बनज्ञानेन एककालिकस्तुल्यकालिकोर्थः ।
	\pend
      
	  \bigskip
	  \begingroup
	  \large
	
	    
	    \stanza[\smallbreak]
	\label{pv.2.245}\edlabel{pv.2.245}\flagstanza{\tiny\textenglish{....2.245}}तदतुल्यक्रियाकालः कथं स्वज्ञानकालिकः ।&सहकारी भवेदर्थ इति चेदक्षचेतसः ॥ २४५ ॥\&[\smallbreak]


	
	  \endgroup
	

	  \pstart तेन सहकारिसम्मतेन्द्रियज्ञाने{\color{DodgerBlue3}“नातुल्यः क्रिया\edlabel{pvv.193-1}\footnote{\label{pvv.193-1}  १ स्वसत्ताकालः ।}कालो”} यस्य भिन्नकालत्वात्\leavevmode\marginnote{\textenglish{38a/MA}} सोऽर्थः {\color{DodgerBlue3}“सहकारी”} कथ{\color{DodgerBlue3}“मक्षचेतसो भवेदिति चेत्”} । (२४५)
	\pend
      \label{div_pvv.2.246}\edlabel{div_pvv.2.246}
	  
	% new div opening: depth here is 2
	

	  \pstart अत्राह (।)
	\pend
      
	  \bigskip
	  \begingroup
	  \large
	
	    
	    \stanza[\smallbreak]
	\label{pv.2.246}\edlabel{pv.2.246}\flagstanza{\tiny\textenglish{....2.246}}असतः प्रागसामर्थ्यात् पश्चाच्चानुपयोगतः ।&प्राग्भावः सर्व्वहेतूनां नातोर्थः स्वधिया सह ॥ २४६ ॥\&[\smallbreak]


	
	  \endgroup
	

	  \pstart कार्योत्पत्तेः प्रागसतस्तत्रासा{\color{DodgerBlue3}“मर्थ्यात्”} । सदधिष्ठानं हि {\color{DodgerBlue3}“सामर्थ्यमसतः कथं”} स्यात् । कार्योत्पत्तेः {\color{DodgerBlue3}“पश्चात्”} सतः कारणव्यापाराद्वा पश्चात् कार्यसमकालस्य सतो वा तत्रा{\color{DodgerBlue3}“नुपयोगतो”} व्यापाराभावात् । कार्या{\color{DodgerBlue3}“त्प्राग्भावः सर्व्वहेतूनामि”}ति स्थितं । विषयश्च ज्ञानानां नाकारणमतिप्रसङ्गात् । {\color{DodgerBlue3}“अतो”} विषयः कारणात्मकः {\color{DodgerBlue3}“स्वधिया”} स्वालम्बनधिया {\color{DodgerBlue3}“सह न”} भवति । पूर्व्वभावित्वे च विषयस्य तत्कालेन्द्रियज्ञानसहकारिता युक्तिमती । (२४६)
	\pend
      \label{div_pvv.2.247}\edlabel{div_pvv.2.247}
	  
	% new div opening: depth here is 2
	

	  \pstart ननु (।)
	\pend
      
	  \bigskip
	  \begingroup
	  \large
	
	    
	    \stanza[\smallbreak]
	\label{pv.2.247}\edlabel{pv.2.247}\flagstanza{\tiny\textenglish{....2.247}}भिन्नकालं कथं ग्राह्यमिति चेद् ग्राह्यतां विदुः ।&हेतुत्वमेव युक्तिज्ञा ज्ञानाकारार्पणक्षमम् ॥ २४७ ॥\&[\smallbreak]


	
	  \endgroup
	

	  \pstart प्राग्भावभावित्वाद् {\color{DodgerBlue3}“भिन्नकालं”} वस्तु {\color{DodgerBlue3}“कथं ग्राह्य”}मिति {\color{DodgerBlue3}“चेत् । हेतुत्वमेव ज्ञानेऽकार”}स्य स्वानुरूपस्या{\color{DodgerBlue3}“र्पणक्षणं ग्राह्यतां युक्तिज्ञा विदुः”} (।) न हि संदंशायोगोलयोरिव ज्ञानपदार्थयोर्ग्राह्यग्राहकभावः । कथन्तर्हि यदाकारमनुकरोति तत् ग्राह्यस्य ग्राहकमित्युच्यते । (२४७)
	\pend
      \label{div_pvv.2.248}\edlabel{div_pvv.2.248}
	  
	% new div opening: depth here is 2
	

	  \pstart ननु यदि कारणं ग्राह्यंतदा समनन्तरप्रत्ययादिकं च तथा स्यात् । इत्याहं (।)
	\pend
      
	  \bigskip
	  \begingroup
	  \large
	
	    
	    \stanza[\smallbreak]
	\label{pv.2.248}\edlabel{pv.2.248}\flagstanza{\tiny\textenglish{....2.248}}कार्यं ह्यनेकहेतुत्वेऽप्यनुकुर्व्वदुदेति यत् ।&तत्तेनाप्यत्र तद्रूपं गृहीतमिति चोच्यते ॥ २४८ ॥\&[\smallbreak]


	
	  \endgroup
	

	  \pstart {\color{DodgerBlue3}“कार्यं”} हि ज्ञानम{\color{DodgerBlue3}“नेकहेतुत्वेऽपि”} यत्कारणमाकारद्वारेणा{\color{DodgerBlue3}“नुकुर्व्वदुदेति”} तत्कारणमर्पि तद्रूपमुपरोपितस्वाकारं {\color{DodgerBlue3}“तेन”} कारणाकारेण {\color{DodgerBlue3}“गृहीतमिति चोच्यते”} । यथा पितृरूपं पुत्रेण गृहीतमिति कथ्यते । उक्तं {\color{DodgerBlue3}“मानसं”} ॥ X X ॥ (२४८)
	\pend
      \label{div_pvv.2.249}\edlabel{div_pvv.2.249}
	  
	% new div opening: depth here is 2
	\leavevmode\marginnote{\textenglish{194/s}}

	  \begin{center}%% label @type='head'
	\textbf{(३) क. स्वसंवेदनप्रत्यक्षम्}
	\end{center}
	

	  \pstart स्वसम्वेदनमाख्यातुमाह (।)
	\pend
      
	  \bigskip
	  \begingroup
	  \large
	
	    
	    \stanza[\smallbreak]
	\label{pv.2.249}\edlabel{pv.2.249}\flagstanza{\tiny\textenglish{....2.249}}अशक्यसमयो ह्यात्मा रागादीनामनन्यभाक् ।&तेषामतः स्वसंवित्तिर्न्नाभिजल्पानुषङ्गिणी ॥ २४९ ॥\&[\smallbreak]


	
	  \endgroup
	

	  \pstart रागद्वेषसुखदुःखादीनां सर्व्वचित्तचैत्तानामात्मसंवेदनं प्रत्यक्षमविकल्पत्वात् ।
	\pend
      

	  \pstart तथा हि । {\color{DodgerBlue3}“रागादीनामात्मा स्वरूपमनन्यभाक्”} ना\edlabel{pvv.194-1}\footnote{\label{pvv.194-1}  १ रागादिसुखादिषु स्वसम्वेदनमिन्द्रियानपेक्षत्वान्मानसं प्रत्यक्षमिति वृद्धिः । अत्र च नात्मसंयोगमात्रभावित्वं सुखादेर्दृष्टमिति दृष्टग्रहणात् ।}न्यं भजते । स्वरूपमात्रावस्थितेः । तस्मा{\color{DodgerBlue3}“दशक्यः समयः”} संकेतोऽस्मिन् {\color{DodgerBlue3}“अतः”} संकेताविषयत्वात् {\color{DodgerBlue3}“तेषां”} रागादीनां {\color{DodgerBlue3}“स्व”}स्यात्मनः {\color{DodgerBlue3}“सम्वित्तिः”} प्रकाशोऽ{\color{DodgerBlue3}“भिजल्पो”} वाचकशब्दोल्लेखस्त{\color{DodgerBlue3}“दनु”}षङ्गो यस्यास्ति सा तथा न भवति । वा\edlabel{pvv.194-2}\footnote{\label{pvv.194-2}  २ वैशेषिका ज्ञानाद् व्यतिरिक्तं सुखादिकमात्मगुणमाहुः । सुखादिज्ञानबाह्यमिति सांख्या एतदपाकरणाद्यर्थं मानसग्रहणं । साततिकं स्ववेदनं सर्व्वेन्द्रियकालिकत्वात् । धारवाहि च प्रत्यक्षसिद्धत्वात् । अविकल्पञ्च स्वप्ने स्पष्टभासात् । कामशोकादिष्वपि सप्रयोजनञ्च । विनानेन स्मृतेरयोगात् । स्वसम्वित्तेर्निर्व्विकल्पत्वं साध्यं सा च ज्ञानस्यापि नास्ति कुतः सुखादीनां ।}च्यं हि वाचकेन संयोज्येत । न च रागाद्यात्मा वा\edlabel{pvv.194-3}\footnote{\label{pvv.194-3}  ३ यादृशो यादृशादुत्पन्नो दृष्टस्तादृशोन्योपि तादृशादेव अन्यादृशस्तु यादृशा इति स्थिते ।}च्यस्ततस्तत्प्रकाशो न शब्दसंगतः । (२४९)
	\pend
      \label{div_pvv.2.250}\edlabel{div_pvv.2.250}
	  
	% new div opening: depth here is 2
	
	  \bigskip
	  \begingroup
	  \large
	
	    
	    \stanza[\smallbreak]
	\label{pv.2.250a}\edlabel{pv.2.250a}\flagstanza{\tiny\textenglish{...2.250a}}अवेदकाः परस्यापि ते स्वरूपं कथं विदुः ।\&[\smallbreak]


	
	  \endgroup
	

	  \pstart ननु रागसुखादय आत्मगुणाः {\color{DodgerBlue3}“परस्य”} बाह्यस्या{\color{DodgerBlue3}“प्यवेदना”}ङ्गज्ञातत्वात् {\color{DodgerBlue3}“ते स्वरूपं कथं विदुः”} । वेदकस्य कदाचित् स्ववेदनं संभाव्येत । अवेदकं च न क्वचिदुपयोगि ॥
	\pend
      

	  \pstart यदि न स्ववेदना रागादयस्तदा कथं वेदयन्त इत्याह (।)
	\pend
      
	  \bigskip
	  \begingroup
	  \large
	
	    
	    \stanza[\smallbreak]
	\label{pv.2.250b}\edlabel{pv.2.250b}\flagstanza{\tiny\textenglish{...2.250b}}एकार्थाश्रयिणा वेद्या विज्ञानेनेति केचन ॥ २५० ॥\&[\smallbreak]


	
	  \endgroup
	

	  \pstart {\color{DodgerBlue3}“एकोर्थ”} आत्मा {\color{DodgerBlue3}“आश्रयो”} रागादिभिः सह यस्यास्ति {\color{DodgerBlue3}“तेनै”}कार्थाश्रयिणा {\color{DodgerBlue3}“ज्ञानेन वेद्या”} रागादय {\color{DodgerBlue3}“इति केचन नै”} या यि का दयः । (२५०)
	\pend
      \label{div_pvv.2.251}\edlabel{div_pvv.2.251}
	  
	% new div opening: depth here is 2
	

	  \pstart अत्राह (।)
	\pend
      \leavevmode\marginnote{\textenglish{195/s}}
	  \bigskip
	  \begingroup
	  \large
	
	    
	    \stanza[\smallbreak]
	\label{pv.2.251}\edlabel{pv.2.251}\flagstanza{\tiny\textenglish{....2.251}}तदतद्रूपिणो भावस्तदतद्रूपहेतुजाः ।&तत्सुखादि किमज्ञानं विज्ञानाभिन्नहेतुजम् ॥ २५१ ॥\&[\smallbreak]


	
	  \endgroup
	

	  \pstart {\color{DodgerBlue3}“तद्रूपिणो”} विवक्षितैकरूपवन्तोऽ{\color{DodgerBlue3}“तद्रू”}पिण इतररूपवन्तो भावा यथाक्रमं तद्रूपाद् दृष्टैकरूपाद्धेतोः सामग्रीलक्षणाज्जाता {\color{DodgerBlue3}“अतद्रूपहेतुजाता”} विलक्षणसामग्रीजाता भवन्तीति तावत् स्थितं । तत् तस्मादिमं न्यायमुल्लङ्ध्य {\color{DodgerBlue3}“विज्ञानेन सहाभिन्न”} एको {\color{DodgerBlue3}“हेतु”}रिन्द्रियविषयमनस्कारादिसामग्रीलक्षणस्तस्मा (त्) जातं {\color{DodgerBlue3}“सुखादिकं”} कस्मादज्ञानं समानसामग्रीप्रसूतत्वात् द्वयमपि ज्ञानं स्यान्न वा किञ्चित् । (२५१)
	\pend
      \label{div_pvv.2.252}\edlabel{div_pvv.2.252}
	  
	% new div opening: depth here is 2
	

	  \pstart अभिन्नहेतुकतामेव समर्थयितुमाह (।)
	\pend
      
	  \bigskip
	  \begingroup
	  \large
	
	    
	    \stanza[\smallbreak]
	\label{pv.2.252}\edlabel{pv.2.252}\flagstanza{\tiny\textenglish{....2.252}}सार्थे सतीन्द्रिये योग्ये यथास्वमपि चेतसि ।&दृष्टं जन्म सुखादीनां तत्तुल्यं मनसामपि ॥ २५२ ॥\&[\smallbreak]


	
	  \endgroup
	

	  \pstart यस्य यदात्मीयं जनकं तस्मि{\color{DodgerBlue3}“न्निन्द्रिये सार्थे”} सविषये योग्ये कार्योत्पादनं प्रति {\color{DodgerBlue3}“चेतसि”} समनन्तरप्रत्यये सति {\color{DodgerBlue3}“सुखादीना”}मपि {\color{DodgerBlue3}“जन्म दृष्टं”} । तद्यथास्वमिन्द्रियादिषु योग्येषु सत्सु जन्म दर्शनं {\color{DodgerBlue3}“मन”}सां ज्ञानाना{\color{DodgerBlue3}“मपि तुल्यं”} । अतस्तुल्यहेतुकत्वात् तुल्यजातीयतैव युक्ता नान्यथा क्वचिदेकता स्यात् । (२५२)
	\pend
      \label{div_pvv.2.253}\edlabel{div_pvv.2.253}
	  
	% new div opening: depth here is 2
	

	  \pstart तुल्यहेतुकत्वेपि संस्कारादेर्नियामकत्वात् सुखादिकमज्ञानं स्यादित्याह (।)
	\pend
      
	  \bigskip
	  \begingroup
	  \large
	
	    
	    \stanza[\smallbreak]
	\label{pv.2.253}\edlabel{pv.2.253}\flagstanza{\tiny\textenglish{....2.253}}असत्सु सत्सु चैतेषु न जन्माजन्म वा क्वचित् ।&दृष्टं सुखादेर्बुद्धेर्वा तत्ततो नान्यतश्च ते ॥ २५३ ॥\&[\smallbreak]


	
	  \endgroup
	

	  \pstart {\color{DodgerBlue3}“अस”}त्स्विन्द्रियादिषु दुःखादेर्बुद्धेर्व्वा जन्म न क्वचिदिष्टं । {\color{DodgerBlue3}“सत्सु”} च एतेष्वि{\color{DodgerBlue3}“न्द्रिया”}दिषु अजन्म वा {\color{DodgerBlue3}“सुखादेर्ब्बुद्धेर्व्वा न क्वचिद् दृष्टं”} (।) तत् तस्मात् तत इन्द्रियादे-\leavevmode\marginnote{\textenglish{38b/MA}} र्दृष्टसामर्थ्यात् कारणात् ते सुखबुद्धी जायेते । {\color{DodgerBlue3}“अन्यतः”} संस्कारादे{\color{DodgerBlue3}“र्न ते”} तस्य सामर्थ्यादर्शनात् । (२५३)
	\pend
      \label{div_pvv.2.254}\edlabel{div_pvv.2.254}
	  
	% new div opening: depth here is 2
	

	  \begin{center}%% label @type='head'
	\textbf{(ख. सुखादिपरवेद्यतानिरस)}
	\end{center}
	

	  \pstart नन्वभिन्नहेतुकत्वे सुखदुःखादिभेदश्च न स्यादित्याह (।)
	\pend
      
	  \bigskip
	  \begingroup
	  \large
	
	    
	    \stanza[\smallbreak]
	\label{pv.2.254}\edlabel{pv.2.254}\flagstanza{\tiny\textenglish{....2.254}}सुखदुःखादिभेदश्च तेषामेव विशेषतः ॥&तस्या एव यथा बुद्धेर्म्मान्द्यपाटवसंश्रयाः ॥ २५४ ॥\&[\smallbreak]


	
	  \endgroup
	

	  \pstart {\color{DodgerBlue3}“सुखदुःखादिभेदश्चा”}वान्तर{\color{DodgerBlue3}“स्तेषामि”}न्द्रियादीना{\color{DodgerBlue3}“मेव”} हेतूनां {\color{DodgerBlue3}“विशेषतो”}ऽवान्तरात् । {\color{DodgerBlue3}“यथा तस्या बुद्धे”}रेवान्तरहेतुविशेषात् {\color{DodgerBlue3}“मान्द्यपाटवसंश्रयाः”} परस्परसंभविनो भवन्ति (।) तस्मात् सुखादयो विज्ञानेनाभिन्नहेतुकत्वाद्विज्ञानस्वभावा एवेति स्थितं । (२५४)
	\pend
      \label{div_pvv.2.255}\edlabel{div_pvv.2.255}
	  
	% new div opening: depth here is 2
	\leavevmode\marginnote{\textenglish{196/s}}

	  \pstart इदानीं परवेद्यतामेषां निषेद्ध्ुमाह (।)
	\pend
      
	  \bigskip
	  \begingroup
	  \large
	
	    
	    \stanza[\smallbreak]
	\label{pv.2.255}\edlabel{pv.2.255}\flagstanza{\tiny\textenglish{....2.255}}यस्यार्थस्य निपातेन ते जाता धीसुखादयः ।&मुक्त्वा तं प्रतिपद्येत सुखादीनेव सा कथम् ॥ २५५ ॥\&[\smallbreak]


	
	  \endgroup
	

	  \pstart {\color{DodgerBlue3}“यस्यार्थस्य”} स्त्र्यादे{\color{DodgerBlue3}“र्निपातेन”} सन्निधानेन {\color{DodgerBlue3}“ते धीसुखादयो जाता”}स्तं कारणभूतमाकारार्पणक्षमं {\color{DodgerBlue3}“मुक्त्वा”} सुखादीनेव सहभाविनो विषयलक्षणरहितान् सा बुद्धिः {\color{DodgerBlue3}“कथं प्रतिपद्येत”} । अकारणस्य विषयत्वेतिप्रसङ्गात् । (२५५)
	\pend
      \label{div_pvv.2.256}\edlabel{div_pvv.2.256}
	  
	% new div opening: depth here is 2
	
	  \bigskip
	  \begingroup
	  \large
	
	    
	    \stanza[\smallbreak]
	\label{pv.2.256}\edlabel{pv.2.256}\flagstanza{\tiny\textenglish{....2.256}}अविछिन्नाथ भासेत तत्संवित्तिः क्रमग्रहे ।&तल्लाघवाच्चेत्तत्तुल्यभित्यसंवेदनं न किम् ॥ २५६ ॥\&[\smallbreak]


	
	  \endgroup
	

	  \pstart {\color{DodgerBlue3}“अथे”}न्द्रियज्ञानेन विषयग्रहस्तदनन्तरं मानसाध्यक्षेण सुखादिग्रह इति {\color{DodgerBlue3}“क्रमग्रहे”} स्वीक्रियमाणे तयोर्व्विषयसुखयोः {\color{DodgerBlue3}“संवित्तिरविच्छिन्ना”} यौगपद्येन न {\color{DodgerBlue3}“भासेत”} । अस्ति च युगपत्प्रतिभासः परिस्फुटः । तयोर्ब्बाह्यसुखग्राहिमनसोर्लाघवात् यौगपद्यप्रतिभास{\color{DodgerBlue3}“श्चेत् । त”}ल्लाघवं प्रतिक्षणमेकैकत्वसंवेदनाभावस्यापि {\color{DodgerBlue3}“तुल्यमित्यसम्वेदनमेव”} साहित्यस्य {\color{DodgerBlue3}“किं न भवति”} । (२५६)
	\pend
      \label{div_pvv.2.257}\edlabel{div_pvv.2.257}
	  
	% new div opening: depth here is 2
	

	  \pstart इन्द्रियबुद्ध्यैवैकया बाह्यसुखयोः सकृद्ग्रहणमिति चेत् । आह ।
	\pend
      
	  \bigskip
	  \begingroup
	  \large
	
	    
	    \stanza[\smallbreak]
	\label{pv.2.257a}\edlabel{pv.2.257a}\flagstanza{\tiny\textenglish{...2.257a}}न चैकया द्वयज्ञानं नियमादक्षचेतसः ।\&[\smallbreak]


	
	  \endgroup
	

	  \pstart {\color{DodgerBlue3}“न चैकया द्वयस्य ज्ञानं”} संभवति । {\color{DodgerBlue3}“अक्षचेतसो”}र्ब्बाह्यरूपादिग्रहण एव {\color{DodgerBlue3}“नियमा न्नि”}यतत्वात् {\color{DodgerBlue3}“सुखा\edlabel{pvv.196-1}\footnote{\label{pvv.196-1}  १ सुखाद्यात्मनः संयोगजमनसा गृह्यते इत्युपगमात् ।} द्या”}त्मगुणग्रहणा{\color{DodgerBlue3}“भा”}वात् ।
	\pend
      

	  \pstart किञ्च (।)
	\pend
      
	  \bigskip
	  \begingroup
	  \large
	
	    
	    \stanza[\smallbreak]
	\label{pv.2.257b}\edlabel{pv.2.257b}\flagstanza{\tiny\textenglish{...2.257b}}सुखाद्यभावेप्यर्थाच्च जातेस्तच्छक्त्यसिद्धितः ॥ २५७ ॥\&[\smallbreak]


	
	  \endgroup
	

	  \pstart नाकारणं विषयः {\color{DodgerBlue3}“सुखाद्यभावेपि”} केवला{\color{DodgerBlue3}“दर्थादि”}न्द्रियबुद्धेर्जातेरुत्पादात् । तस्य सुखादेरिन्द्रियबुद्धिजननं प्रति {\color{DodgerBlue3}“शक्त्यसि”}द्धेरविषयत्वं ।\edlabel{pvv.196-2}\footnote{\label{pvv.196-2}  २ सुखादयो नात्मानं विदुरिति बुद्ध्या वेद्येरन्निति किं सहजया उत्तरकालया वा तत्र । सहिता व्यस्ताश्च यथा नीलादयो ज्ञानहेतुस्तद्वत् सुखादिर्व्विषयश्च स्यादिति द्वयं ग्रहणमविरुद्धमित्याह ।}(२५७)
	\pend
      \label{div_pvv.2.258}\edlabel{div_pvv.2.258}
	  
	% new div opening: depth here is 2
	
	  \bigskip
	  \begingroup
	  \large
	
	    
	    \stanza[\smallbreak]
	\label{pv.2.258a}\edlabel{pv.2.258a}\flagstanza{\tiny\textenglish{...2.258a}}पृथक् पृथक् च सामर्थ्ये द्वयोर्नीलादिवत् सुखम् ।&गृह्येत केवलं;\&[\smallbreak]


	
	  \endgroup
	

	  \pstart {\color{DodgerBlue3}“द्वयो”} रूपिसुखाद्योः {\color{DodgerBlue3}“पृथक् पृथग्ज्ञा”}नोपपत्तौ {\color{DodgerBlue3}“सामर्थ्ये”} वाभ्युपगम्यमाने {\color{DodgerBlue3}“नीलादिवत् केवलं सुखं गृह्येत”} ।
	\pend
      \leavevmode\marginnote{\textenglish{197/s}}

	  \pstart अयुक्तं चैतत् ।
	\pend
      
	  \bigskip
	  \begingroup
	  \large
	
	    
	    \stanza[\smallbreak]
	\label{pv.2.258b}\edlabel{pv.2.258b}\flagstanza{\tiny\textenglish{...2.258b}}तस्य तद्धेत्वर्थमगृह्णतः ॥ २५८ ॥\&[\smallbreak]


	
	  \endgroup
	

	  \pstart {\color{DodgerBlue3}“न हि तस्य”} सुखादे{\color{DodgerBlue3}“र्हेतुमर्थं”} स्त्र्यादि{\color{DodgerBlue3}“मगृह्णतः”}\edlabel{pvv.197-1}\footnote{\label{pvv.197-1}  १ अर्थग्रहणनिरपेक्षं न तत् सिद्धिकल्पं ॥}(। २५८)
	\pend
      \label{div_pvv.2.259}\edlabel{div_pvv.2.259}
	  
	% new div opening: depth here is 2
	
	  \bigskip
	  \begingroup
	  \large
	
	    
	    \stanza[\smallbreak]
	\label{pv.2.259a}\edlabel{pv.2.259a}\flagstanza{\tiny\textenglish{...2.259a}}न हि संवेदनं युक्तमर्थेनैव सह ग्रहे ।\&[\smallbreak]


	
	  \endgroup
	

	  \pstart सुखस्य {\color{DodgerBlue3}“संवेदनं युक्तं । अर्थेनैव सह”} सुखस्येन्द्रियधिया {\color{DodgerBlue3}“ग्रहे”} नास्ति {\color{DodgerBlue3}“दोष”} इति चेत् । न (।)
	\pend
      
	  \bigskip
	  \begingroup
	  \large
	
	    
	    \stanza[\smallbreak]
	\label{pv.2.259b}\edlabel{pv.2.259b}\flagstanza{\tiny\textenglish{...2.259b}}किं सामर्थ्यं सुखादीनां नेष्टा धीर्यत्तदुद्भवा ॥ २५९ ॥\&[\smallbreak]


	
	  \endgroup
	

	  \pstart {\color{DodgerBlue3}“सुखा”}देरिन्द्रियबुद्धिजनने {\color{DodgerBlue3}“किम”}स्ति {\color{DodgerBlue3}“सामर्थ्यं”} । येन तस्याः स विषयः सन् सह गृह्येत । {\color{DodgerBlue3}“यद्य”}स्मा{\color{DodgerBlue3}“त्तदुद्भवा”} सुखादुद्भवा नेन्द्रिय{\color{DodgerBlue3}“धीरिष्टा”} तस्मान्न तद्ग्राहिका । (२५९)
	\pend
      \label{div_pvv.2.260}\edlabel{div_pvv.2.260}
	  
	% new div opening: depth here is 2
	
	  \bigskip
	  \begingroup
	  \large
	
	    
	    \stanza[\smallbreak]
	\label{pv.2.260}\edlabel{pv.2.260}\flagstanza{\tiny\textenglish{....2.260}}विनार्थेन सुखादीनां वेदने चक्षुरादिभिः ।&रूपादिः स्त्र्यादिभेदोऽक्ष्णा न गृह्येत कदाचन ॥ २६० ॥\&[\smallbreak]


	
	  \endgroup
	

	  \pstart ततश्च कथमर्थसुखयोः सहवेदनं । {\color{DodgerBlue3}“अर्थेन विनैव चक्षुरादि”}भिश्चक्षुरादिज्ञानैः {\color{DodgerBlue3}“सुखादीनां”} वेदने चाभ्युपगम्यमाने {\color{DodgerBlue3}“स्त्र्यादेर्भेदो”} विशेषो {\color{DodgerBlue3}“रूपादिः”} सुखहेतुरक्ष्णा चक्षुर्ज्ञानेन (न) {\color{DodgerBlue3}“कदाचन गृह्येत”} । (२६०)
	\pend
      \label{div_pvv.2.261}\edlabel{div_pvv.2.261}
	  
	% new div opening: depth here is 2
	

	  \pstart कस्मादेवमित्याह (।)
	\pend
      
	  \bigskip
	  \begingroup
	  \large
	
	    
	    \stanza[\smallbreak]
	\label{pv.2.261}\edlabel{pv.2.261}\flagstanza{\tiny\textenglish{....2.261}}न हि सत्यन्तरङ्गेर्थे शक्ते धीर्बाह्यदर्शनो ।&अर्थग्रहे सुखादीनां तज्जानां स्यादवेदनम् ॥ २६१ ॥\&[\smallbreak]


	
	  \endgroup
	

	  \pstart {\color{DodgerBlue3}“अन्त”}रङ्गे सन्निहिते {\color{DodgerBlue3}“अर्थे”} क्षेत्रज्ञसमवेते सुखादौ {\color{DodgerBlue3}“शक्ते”} स्वग्राहिज्ञानोत्पादनक्षमे सति न हि {\color{DodgerBlue3}“बाह्यदर्शनी धी”}रर्थग्राहिणी युक्ता । बहिरङ्गो बाह्योर्थ इन्द्रियालोकादिसहकार्यपेक्षणात् । सुखादिस्तु न तत्सापेक्ष इति तस्माज्जाता धीस्तमेव गृह्णीयात् । इन्द्रियधिया{\color{DodgerBlue3}“ऽर्थग्रहे”} स्वीक्रियमाणे {\color{DodgerBlue3}“सुखादीनां”} तज्जानामर्थदर्शनप्रसूतानाम{\color{DodgerBlue3}“र्थमवेदनं स्यात्”} । इन्द्रियबुद्धिरर्थग्रह एवोपयुक्ता तत्कालमन्या च धीर्नास्ति पश्चात् ज्ञानान्तरकाले विषयबलभाविनः सुखादे\edlabel{pvv.197-2}\footnote{\label{pvv.197-2}  २ विषयाभावादेव ।}रभाव इति कथं वेदनं । (२६१)
	\pend
      \label{div_pvv.2.262}\edlabel{div_pvv.2.262}
	  
	% new div opening: depth here is 2
	
	  \bigskip
	  \begingroup
	  \large
	
	    
	    \stanza[\smallbreak]
	\label{pv.2.262}\edlabel{pv.2.262}\flagstanza{\tiny\textenglish{....2.262}}धियोर्युगपदुत्पत्तौ तत्तद्-विषयसम्भवात् ।&सुखदुःखविदौ स्यातां सकृदर्थस्य सम्भवे ॥ २६२ ॥\&[\smallbreak]


	
	  \endgroup
	\leavevmode\marginnote{\textenglish{198/s}}

	  \pstart विषयसुखग्राहिण्योर्द्वयोरिन्द्रियबुद्धिमनोबुद्ध्यात्मिकयोः । तस्य सुखहेतोर्दुःखहेतोश्च {\color{DodgerBlue3}“विषयस्य सम्भवात् युगपदुत्पत्ता”}वभिमतायां {\color{DodgerBlue3}“सकृदर्थस्य”} सुखदुःखहेतोः {\color{DodgerBlue3}“सम्भवे”} सति {\color{DodgerBlue3}“सुखदुःखविदौ स्यातां”} । (२६२)
	\pend
      \label{div_pvv.2.263}\edlabel{div_pvv.2.263}
	  
	% new div opening: depth here is 2
	

	  \pstart स्यादेतत् (।)
	\pend
      
	  \bigskip
	  \begingroup
	  \large
	
	    
	    \stanza[\smallbreak]
	\label{pv.2.263}\edlabel{pv.2.263}\flagstanza{\tiny\textenglish{....2.263}}सत्यान्तरेप्युपादाने ज्ञाने दुःखादिसम्भवः ।&नोपादानं विरुद्धस्य तच्चैकमिति चेन्मतम् ॥ २६३ ॥\&[\smallbreak]


	
	  \endgroup
	

	  \pstart न केवलमर्थे सति किन्तर्ह्या{\color{DodgerBlue3}“न्तरे”} समनन्तरप्रत्यये {\color{DodgerBlue3}“ज्ञान उपादाने सति”} \leavevmode\marginnote{\textenglish{39a/MA}} {\color{DodgerBlue3}“दुःखादिसंभवः”} । तच्चोपादानापेक्षं ज्ञानं {\color{DodgerBlue3}“विरुद्धस्य”} सुखदुःखादेर्न कारणं भवितुमर्हती{\color{DodgerBlue3}“ति मतं चेत्”} (। २६३)
	\pend
      \label{div_pvv.2.264}\edlabel{div_pvv.2.264}
	  
	% new div opening: depth here is 2
	

	  \pstart अत्राह (।)
	\pend
      
	  \bigskip
	  \begingroup
	  \large
	
	    
	    \stanza[\smallbreak]
	\label{pv.2.264}\edlabel{pv.2.264}\flagstanza{\tiny\textenglish{....2.264}}तदज्ञानस्य विज्ञानं केनोपादानकारणं ।&आधिपत्यं तु कुर्वीत तद्‏्विरुद्धेपि दृश्यते ॥ २६४ ॥\&[\smallbreak]


	
	  \endgroup
	

	  \pstart {\color{DodgerBlue3}“तद्विज्ञानमज्ञानस्य”} सुखदुःखादेरु{\color{DodgerBlue3}“पादानकारणं”} केन हेतुना सम्मतं । समानजातीयं कारणमुपादानं नान्यत् । न च सुखादिज्ञानामिष्टं । ज्ञानमज्ञानकार्ये आधिप\edlabel{pvv.198-1}\footnote{\label{pvv.198-1}  १ सिद्धान्त्येवाह (।) आधिपत्यमानं स्यात् ।}त्यं सहकारित्वं कुर्व्वीत । {\color{DodgerBlue3}“तदा”}धिपत्यं {\color{DodgerBlue3}“विरुद्धेपि”} का\edlabel{pvv.198-2}\footnote{\label{pvv.198-2}  २ न च तावतोपादानत्त्वात् ।}र्ये दृश्यते । (२६४)
	\pend
      \label{div_pvv.2.265}\edlabel{div_pvv.2.265}
	  
	% new div opening: depth here is 2
	
	  \bigskip
	  \begingroup
	  \large
	
	    
	    \stanza[\smallbreak]
	\label{pv.2.265}\edlabel{pv.2.265}\flagstanza{\tiny\textenglish{....2.265}}अक्ष्णोर्यथैक आलोको नक्तञ्चरतदन्ययोः ।&रूपदर्शनवैगुण्यावैगुण्ये कुरुते सकृत् ॥ २६५ ॥\&[\smallbreak]


	
	  \endgroup
	

	  \pstart {\color{DodgerBlue3}“यथा एक आलोको नक्तंच”}रस्य मनुष्यादेरक्ष्णो रूपदर्शनस्य वैगुण्यावैगुण्ये यथाक्रमं {\color{DodgerBlue3}“सकृत् कुरुते”} । (२६५)
	\pend
      \label{div_pvv.2.266}\edlabel{div_pvv.2.266}
	  
	% new div opening: depth here is 2
	
	  \bigskip
	  \begingroup
	  \large
	
	    
	    \stanza[\smallbreak]
	\label{pv.2.266}\edlabel{pv.2.266}\flagstanza{\tiny\textenglish{....2.266}}तस्मात् सुखादयोर्थानां स्वसंक्रान्तावभासिनाम् ।&वेदकाः स्वात्मनश्चैषामर्थेभ्यो जन्म केवलम् ॥ २६६ ॥\&[\smallbreak]


	
	  \endgroup
	

	  \pstart यस्माद\edlabel{pvv.198-3}\footnote{\label{pvv.198-3}  ३ महत्तत्त्वादिना ।}न्येन सुखादीनां वेदनं न घटते । {\color{DodgerBlue3}“तस्मात्सुखादयो”} ज्ञानात्मानः स्वस्मिन् स्वरूपे प्रतिबन्धद्वारेण {\color{DodgerBlue3}“संक्रान्तावभासन”}शीलाश्च ये{\color{DodgerBlue3}“ऽर्थानां”} तेषां {\color{DodgerBlue3}“वेदकाः”} । {\color{DodgerBlue3}“आत्मनश्चा”}परोक्षत्वाद्वेदकाः । अर्थसरूपस्यात्मनोऽपरोक्षतैव अर्थवेदनं स्ववेदनञ्च । न त्वन्यः कश्चिद् ग्रहणप्रकारः । तत{\color{DodgerBlue3}“श्चैषां”} सुखादीनामर्थसरूपाणा{\color{DodgerBlue3}“मर्थेभ्यो जन्मैव केवलं”} ग्रहीतृत्वं नापरः कश्चिद् व्यापारः । (२६६)
	\pend
      \label{div_pvv.2.267}\edlabel{div_pvv.2.267}
	  
	% new div opening: depth here is 2
	\leavevmode\marginnote{\textenglish{199/s}}
	  \bigskip
	  \begingroup
	  \large
	
	    
	    \stanza[\smallbreak]
	\label{pv.2.267}\edlabel{pv.2.267}\flagstanza{\tiny\textenglish{....2.267}}अर्थात्मा स्वात्मभूतो हि तेषां तैरनुभूयते ।&तेनार्थानुभवख्यातिरालम्बस्तु तदाभता ॥ २६७ ॥\&[\smallbreak]


	
	  \endgroup
	

	  \pstart अत एवा{\color{DodgerBlue3}“र्थानुभवः”} कथं स्ववेदनमर्थज्ञानयोर्भेदात् । तद्वेदनयोश्च भेदस्य न्यायप्राप्तत्वादिति ब्रुवाणः प्रतिक्षिप्तः । तथा हि तेषां सुखादीना{\color{DodgerBlue3}“मर्थात्मा”} अर्थाकारः प्रतिबिम्बसंक्रान्त्या {\color{DodgerBlue3}“स्वात्मभूतः”} स्वभावभूत{\color{DodgerBlue3}“स्तैः”} सुखादिभि{\color{DodgerBlue3}“रनुभूयते”}ऽपरोक्षत्वात् । तेनोपचारेणार्थस्य परभूतस्य प्रतिबिम्बहेतोरनुभवस्य ख्यातिः प्रसिद्धिः । ततश्च ज्ञानस्यालम्बोऽर्थालम्बनं तदाभता अर्थाकारत्वं विचार्यमाणमवशिष्यते न तु पारमार्थिकमालम्व्यालम्बकत्वं नाम । (२६७)
	\pend
      \label{div_pvv.2.268}\edlabel{div_pvv.2.268}
	  
	% new div opening: depth here is 2
	

	  \begin{center}%% label @type='head'
	\textbf{ग. स्वसंवेदने सांख्यमतनिरासः}
	\end{center}
	

	  \pstart इदानीं सां ख्य म तमुत्थापयन्नाह (।)
	\pend
      
	  \bigskip
	  \begingroup
	  \large
	
	    
	    \stanza[\smallbreak]
	\label{pv.2.268}\edlabel{pv.2.268}\flagstanza{\tiny\textenglish{....2.268}}कश्चिद् बहिःस्थितानेव सुखादीनप्रचेतनान् ।&ग्राह्यानाह न तस्यापि सकृद्युक्तो द्वयग्रहः ॥ २६८ ॥\&[\smallbreak]


	
	  \endgroup
	

	  \pstart {\color{DodgerBlue3}“कश्चिद् बहिःस्थितानेव”} न त्वात्मसमवायिनः । प्रधानपरिणामजत्वेन सुखदुःखमोहस्वभावत्वात् अर्थान{\color{DodgerBlue3}“प्रचेतनात्”} आत्मन एव चेतनत्वाद् ्। बुद्धिदर्पणे जडात्मनि स्वच्छेऽर्थचेतनयोः प्रतिबिम्बसंक्रान्तिद्वारेण च्छायापत्त्या चेतनस्य {\color{DodgerBlue3}“ग्राह्यानाह (।) तस्यापि”} मते {\color{DodgerBlue3}“द्वयस्य”} सुखस्य नीलादेश्च {\color{DodgerBlue3}“सकृद् ग्रहो”} नियमेन {\color{DodgerBlue3}“न युक्तः”} । कदाचिद्रूपनिरपेक्षमपि रूपसुखमनुभूयेत । (२६८)
	\pend
      \label{div_pvv.2.269}\edlabel{div_pvv.2.269}
	  
	% new div opening: depth here is 2
	
	  \bigskip
	  \begingroup
	  \large
	
	    
	    \stanza[\smallbreak]
	\label{pv.2.269}\edlabel{pv.2.269}\flagstanza{\tiny\textenglish{....2.269}}सुखाद्यभिन्नरूपत्वान्नोलादेश्चेत् सकृद् ग्रहः ।&भिन्नावभासिनोर्ग्राह्यं चेतसोस्तदभेदि किम् ॥ २६९ ॥\&[\smallbreak]


	
	  \endgroup
	

	  \pstart {\color{DodgerBlue3}“सुखादेरभिन्नरूपत्वान्नीलादेः सकृन्नियमेन ग्रहश्चेत् । भिन्नावभासिनो”}र्भिन्नाकारयोः सुखनील{\color{DodgerBlue3}“चेतसोर्ग्राह्यं तत् किमभेदि”} इष्यते । सुखदुःखादीनां नीलपीतादीनाञ्च भिन्नमनोग्राह्यानामेकत्वमेवं स्यात् । (२६९)
	\pend
      \label{div_pvv.2.270}\edlabel{div_pvv.2.270}
	  
	% new div opening: depth here is 2
	

	  \pstart किञ्च (।)
	\pend
      
	  \bigskip
	  \begingroup
	  \large
	
	    
	    \stanza[\smallbreak]
	\label{pv.2.270}\edlabel{pv.2.270}\flagstanza{\tiny\textenglish{....2.270}}तस्याविशेषे बाह्यस्य भावनातारतम्यतः ।&तारतम्यञ्च बुद्धौ स्यान्न प्रीतिपरितापयोः ॥ २७० ॥\&[\smallbreak]


	
	  \endgroup
	

	  \pstart यदि नीलाद्येव सुखाद्यात्मकं तदा {\color{DodgerBlue3}“तस्य बाह्यस्य”} सुखदुःखाद्यात्मतयाऽ{\color{DodgerBlue3}“विशेषे”} विशेषाभावे रुच्यरुचिविषयतया {\color{DodgerBlue3}“भावना”}यास्ता{\color{DodgerBlue3}“रतम्यतः । तारतम्यञ्च बुद्धौ प्रीतिपरितापयोर्न”} स्यात् । न हि ग्राह्याविशेषे तज्ज्ञानं विशिष्यते । (२७०)
	\pend
      \label{div_pvv.2.271}\edlabel{div_pvv.2.271}
	  
	% new div opening: depth here is 2
	
	  \bigskip
	  \begingroup
	  \large
	
	    
	    \stanza[\smallbreak]
	\label{pv.2.271}\edlabel{pv.2.271}\flagstanza{\tiny\textenglish{....2.271}}सुखाद्यात्मतया बुद्धेरपि यद्यविरोधिता ।&स इदानीं कथं बाह्यः सुखाद्यात्मेति गम्यते ॥ २७१ ॥\&[\smallbreak]


	
	  \endgroup
	\leavevmode\marginnote{\textenglish{200/s}}

	  \pstart बुद्धेरपि प्रधानपरिणामरूपाया भावनातारतम्यात् सुखाद्यात्म\edlabel{pvv.200-1}\footnote{\label{pvv.200-1}  १ नीलादि सुखादि च भिन्नमेव । नीलसंप्रहर्षणाकारयोर्भेदात् ।}ताविशेषाद् बाह्याविशेषेपि प्रीतिपरितापतारतम्यस्याविरोधितेष्यते । {\color{DodgerBlue3}“यदि इदानी”}मन्याभ्युपगमे {\color{DodgerBlue3}“बाह्योऽर्थः”} स सुखाद्यात्मेति कथं गम्यते । अनुभूयमानं सुखमन्यत्रासंभवद् बाह्ये व्यवस्थापनीयं । (२७१)
	\pend
      \label{div_pvv.2.272}\edlabel{div_pvv.2.272}
	  
	% new div opening: depth here is 2
	
	  \bigskip
	  \begingroup
	  \large
	
	    
	    \stanza[\smallbreak]
	\label{pv.2.272}\edlabel{pv.2.272}\flagstanza{\tiny\textenglish{....2.272}}अग्राह्यग्राहकत्वाच्चेद् भिन्नजातीययोः पुमान् ।&अग्राहकः स्यात् सर्व्वस्य ततो हीयेत भोक्तृता ॥ २७२ ॥\&[\smallbreak]


	
	  \endgroup
	

	  \pstart यदा तु बुद्धिरपि सुखात्मिका तदा किं बाह्यसुखकल्पनया । बाह्यमह\edlabel{pvv.200-2}\footnote{\label{pvv.200-2}  २ प्रधानान्महान्महतोहंकारस्ततः पञ्च शब्दादीनि ततः पञ्चाकाशादीनि पञ्च कर्म्मेन्द्रियाणि वाक्पाणिपादपायूपस्थानि पञ्च बुद्धीन्द्रियाणि मनश्च ... सांख्यस्य । ग्राह्यग्राहकत्वाभावः प्रसज्येद(? ना) स्ति स न स्यात् ।}तोर\leavevmode\marginnote{\textenglish{39b/MA}} सुखरूपत्वाच्च {\color{DodgerBlue3}“भिन्नजातीययोर्ग्राह्यग्राहकत्वा”}भावात् सुखाद्यात्मता बाह्यस्य गम्यत इति चेत्\edlabel{pvv.200-3}\footnote{\label{pvv.200-3}  ३ विषयबुद्धयोः ।} । {\color{DodgerBlue3}“पुमान्”} चेतनो निर्गुणत्वादसुखादिरूपो ग्राह्यस्याचेतनस्य सुखाद्यात्मकस्य सर्व्वस्य भिन्नजातीयत्वात् {\color{DodgerBlue3}“अग्राहकः”} स्यात् । {\color{DodgerBlue3}“ततो”}ऽग्राह\edlabel{pvv.200-4}\footnote{\label{pvv.200-4}  ४ न बाह्यं ज्ञाने संक्रामति स्वरूपेण वेद्यते वाऽग्न्यादाहप्रसङ्गात् ।}कत्वात् {\color{DodgerBlue3}“भोक्तृतास्य हीयते”} । अनुभविता हि भोक्तोच्यतेऽनुभवाभावे कथं भोक्ता । (२७२)
	\pend
      \label{div_pvv.2.273}\edlabel{div_pvv.2.273}
	  
	% new div opening: depth here is 2
	
	  \bigskip
	  \begingroup
	  \large
	
	    
	    \stanza[\smallbreak]
	\label{pv.2.273}\edlabel{pv.2.273}\flagstanza{\tiny\textenglish{....2.273}}कार्यकारणतानेन प्रत्युक्ताऽकार्यकारणे ।&ग्राह्यग्राहकताभावाद् भावेन्यत्रापि सा भवेत् ॥ २७३ ॥\&[\smallbreak]


	
	  \endgroup
	

	  \pstart बुद्धिसुखयोर्ग्राह्यग्राहकाभावात् । नाकारणं विषय इति कार्यकारणता । ततो ग्राह्य\edlabel{pvv.200-5}\footnote{\label{pvv.200-5}  ५ बुद्ध्या कटो विषयाकारः पुमांसमध्यारोहति ततोस्य भोक्तृत्वं ।}स्य सुखाद्यात्मकत्वात् तत्कार्यया बुद्ध्यापि सुखाद्यात्मिकया भवितव्यमिति परैरुक्ता {\color{DodgerBlue3}“कार्यकारणता”} सानेनातिप्रसङ्गेन {\color{DodgerBlue3}“प्रत्युक्ता”} । तथा हि भोक्तापि {\color{DodgerBlue3}“ग्राहक”} इति स च कार्यसुखाद्यात्मकः कथं {\color{DodgerBlue3}“स्यात्”} । अपि चाकार्यकारणे बुद्धिसुखे । ग्राह्यग्राहकतायाः कार्यकारणतानिबन्धनाया अभावात् । {\color{DodgerBlue3}“ग्राह्यग्राहकत्व”}भावात् कार्यकारणभावस्य च {\color{DodgerBlue3}“भावे”}ऽभ्युपगम्यमाने{\color{DodgerBlue3}“ऽन्य”}त्रात्मविषय\edlabel{pvv.200-6}\footnote{\label{pvv.200-6}  ६ विजातीययोरपि ।}योरपि {\color{DodgerBlue3}“सा”} कार्यकारणता स्यात् । ग्राह्यग्राहकताया भावात् । (२७३)
	\pend
      \label{div_pvv.2.274}\edlabel{div_pvv.2.274}
	  
	% new div opening: depth here is 2
	\leavevmode\marginnote{\textenglish{201/s}}

	  \pstart यस्मात्सुखादेर्ब्बाह्यस्य भावनातस्तारतम्यानुयोगयोगः ।
	\pend
      
	  \bigskip
	  \begingroup
	  \large
	
	    
	    \stanza[\smallbreak]
	\label{pv.2.274a}\edlabel{pv.2.274a}\flagstanza{\tiny\textenglish{...2.274a}}तस्मात्त आन्तरा एव,\&[\smallbreak]


	
	  \endgroup
	

	  \pstart {\color{DodgerBlue3}“तस्माद”}सुखादय {\color{DodgerBlue3}“आन्तरा”} ज्ञानस्वभावा अबाह्या अभ्युपगन्तव्याः ।
	\pend
      

	  \pstart किञ्च (।)
	\pend
      
	  \bigskip
	  \begingroup
	  \large
	
	    
	    \stanza[\smallbreak]
	\label{pv.2.274b}\edlabel{pv.2.274b}\flagstanza{\tiny\textenglish{...2.274b}}सम्वेद्यत्वाच्च चेतनाः ।&संवेदनं न यद् रूपं न हि तत्तस्य वेदनम् ॥ २७४ ॥\&[\smallbreak]


	
	  \endgroup
	

	  \pstart {\color{DodgerBlue3}“सम्वेद्यत्वाच्च हेतोस्ते चेतनाः”} । तथा हि सम्वेदनं तद्रूपं यद्विषयस्वरूपं न भवति {\color{DodgerBlue3}“तत्संवेदनं तस्य”} विषयस्य {\color{DodgerBlue3}“वेदनं”} यस्मा{\color{DodgerBlue3}“न्न”} भवति । न ह्यविषयस्वरूपं प्रतिविषयं भेदव्यवस्थां कर्त्तुमर्हतीति साकारं ग्राहकं । तथा च साकारं ज्ञानमेव सुखमस्तु तदुपधायकं तु बाह्यमयुक्तं । भावनाविशेषेण सुखादिवेदनस्याविशेषदर्शनात् । बाह्याधीनत्वे च तदयोगात् । (२७४)
	\pend
      \label{div_pvv.2.275}\edlabel{div_pvv.2.275}
	  
	% new div opening: depth here is 2
	

	  \pstart स्यादेतद् (।)
	\pend
      
	  \bigskip
	  \begingroup
	  \large
	
	    
	    \stanza[\smallbreak]
	\label{pv.2.275}\edlabel{pv.2.275}\flagstanza{\tiny\textenglish{....2.275}}अतत्स्वभावोऽनुभवो बौद्धांस्तान् सन्नवैति चेत् ।&मुक्त्‏वाध्यक्षस्मृताकारां संवित्तिं बुद्धिरत्र का ॥ २७५ ॥\&[\smallbreak]


	
	  \endgroup
	

	  \pstart भावनातस्तारतम्ययोगिनस्तान् बुद्ध्यात्मकसुखादीन् {\color{DodgerBlue3}“अतत्स्वभावो”}ऽसुख्याद्याकारोऽ{\color{DodgerBlue3}“नुभवः सन्नवैति”} प्रत्येतीति {\color{DodgerBlue3}“चेत् ।\edlabel{pvv.201-1}\footnote{\label{pvv.201-1}  १ स्वप्रक्रियामात्रदीपनमेतत् ।} संवित्तिमध्यक्षा”}कारां हर्षविषादाकारां {\color{DodgerBlue3}“स्मृताकाराम”}तीतसुखादिकल्पनारूपां सौमनस्यलक्षणां मुक्त्वा अत्र संवेदनावसरे {\color{DodgerBlue3}“का”} परा बुद्धिरनुभूयते । यस्याः सुखाद्यात्मकत्वमध्य\edlabel{pvv.201-2}\footnote{\label{pvv.201-2}  २ बुद्धिः ।}वसायात्मकत्वञ्चेष्यते । (२७५)
	\pend
      \label{div_pvv.2.276}\edlabel{div_pvv.2.276}
	  
	% new div opening: depth here is 2
	
	  \bigskip
	  \begingroup
	  \large
	
	    
	    \stanza[\smallbreak]
	\label{pv.2.276}\edlabel{pv.2.276}\flagstanza{\tiny\textenglish{....2.276}}तांस्तानर्थानुपादाय सुखदुःखादिवेदनम् ।&एकमाविर्भवद् दृष्टं न दृष्टं त्वन्यदन्तरा ॥ २७६ ॥\&[\smallbreak]


	
	  \endgroup
	

	  \pstart {\color{DodgerBlue3}“ताँस्तानर्थानि”}ष्टाननिष्टा{\color{DodgerBlue3}“नुपादाया”}श्रित्यैकं {\color{DodgerBlue3}“सुखदुःखादिवेदनमाविर्भवद् दृष्टं”} अन्यद् बुद्धिरूप{\color{DodgerBlue3}“मन्तरा”} विषयसंवेदनयोरन्तराले {\color{DodgerBlue3}“न दृष्टं”} । (२७६)
	\pend
      \label{div_pvv.2.277}\edlabel{div_pvv.2.277}
	  
	% new div opening: depth here is 2
	
	  \bigskip
	  \begingroup
	  \large
	
	    
	    \stanza[\smallbreak]
	\label{pv.2.277}\edlabel{pv.2.277}\flagstanza{\tiny\textenglish{....2.277}}संसर्गादविभागश्चेदयोगोलकवह्निवत् ।&भेदाभेदव्यवस्थैवमुच्छिन्ना सर्ववस्तुषु ॥ २७७ ॥\&[\smallbreak]


	
	  \endgroup
	

	  \pstart बुद्धिचेतनयोः {\color{DodgerBlue3}“संसर्ग्गादयोगो\edlabel{pvv.201-3}\footnote{\label{pvv.201-3}  ३ अयःपिण्डमग्निदीप्तं ।}लकवह्न्यो”}रिवाविभागो भेदानुपलब्धिरिति {\color{DodgerBlue3}“चेत्”} । एवं भिन्नाकारानुभवेप्यभेदकल्पनायां\edlabel{pvv.201-4}\footnote{\label{pvv.201-4}  ४ तथा एकाकारेप्ययोगोलकवत्संसर्गकल्पनायां ।} {\color{DodgerBlue3}“सर्व्ववस्तुषु भेदाभेदव्यवस्थोच्छिन्ना”} स्यात् । (२७७)
	\pend
      \label{div_pvv.2.278}\edlabel{div_pvv.2.278}
	  
	% new div opening: depth here is 2
	\leavevmode\marginnote{\textenglish{202/s}}

	  \pstart तथा हि (।)
	\pend
      
	  \bigskip
	  \begingroup
	  \large
	
	    
	    \stanza[\smallbreak]
	\label{pv.2.278}\edlabel{pv.2.278}\flagstanza{\tiny\textenglish{....2.278}}अभिन्नवेदनस्यैक्यं यन्नैवं तद् विभेदवत् ।&सिध्येदसाधनत्वेस्य न सिद्धं भेदसाधनम् ॥ २७८ ॥\&[\smallbreak]


	
	  \endgroup
	

	  \pstart {\color{DodgerBlue3}“अभिन्न”}मेकाकारं {\color{DodgerBlue3}“वेदनं”} यस्य तस्यैक्यं । {\color{DodgerBlue3}“यन्नैवं”} भिन्नवेदनं {\color{DodgerBlue3}“तद्विभेद\edlabel{pvv.202-1}\footnote{\label{pvv.202-1}  १ विभेदोस्यास्तीति कृत्वा ।} वत् सिध्येत्(।)”} अस्यैकाकारज्ञानस्याभेदम्प्र{\color{DodgerBlue3}“त्यसाधनत्वे”} तद्विपरीतभिन्नाकारवेदनं {\color{DodgerBlue3}“भेदसाधनमसिद्धं”} ॥ (२७८)
	\pend
      \label{div_pvv.2.279}\edlabel{div_pvv.2.279}
	  
	% new div opening: depth here is 2
	
	  \bigskip
	  \begingroup
	  \large
	
	    
	    \stanza[\smallbreak]
	\label{pv.2.279}\edlabel{pv.2.279}\flagstanza{\tiny\textenglish{....2.279}}भिन्नाभः सितदुःखादिरभिन्नो बुद्धिवेदने ।&अभिन्नाभे विभिन्ने चेत् भेदाभेदौ किमाश्रयौ ॥ २७९ ॥\&[\smallbreak]


	
	  \endgroup
	

	  \pstart सां\edlabel{pvv.202-2}\footnote{\label{pvv.202-2}  २ भेदाभेदनिबन्धनाभावमाह ।} ख्य स्य तु सितदुःखादिर्भिन्नाकारोऽभिन्न इष्टः । {\color{DodgerBlue3}“बुद्धिवेदने त्वभिन्नाभे विभिन्ने इष्टे चेत् । भेदाभेदौ किमाश्रयौ”} किं निमित्तौ ते व्यवस्थापनीयौ । (२७९)
	\pend
      \label{div_pvv.2.280}\edlabel{div_pvv.2.280}
	  
	% new div opening: depth here is 2
	

	  \pstart ननु वे\edlabel{pvv.202-3}\footnote{\label{pvv.202-3}  ३ वैभा ष्या वेदनादिसंप्रयुक्तविप्रयुक्तवादिनः प्रत्याहुः ।}दनाचेतनासंज्ञादीनां चैत्तानां महाभूमिकादीनां सकृदुत्पन्नानां भेदः परस्परं\edlabel{pvv.202-4}\footnote{\label{pvv.202-4}  ४ एकैकस्योपलब्धेः ।} प्रतीयते । अथ चास्ति । तद्वद्ब्ुद्धिसुखयोरपि स्यादित्याह ।\edlabel{pvv.202-5}\footnote{\label{pvv.202-5}  ५ नैष दोषः ।}
	\pend
      
	  \bigskip
	  \begingroup
	  \large
	
	    
	    \stanza[\smallbreak]
	\label{pv.2.280}\edlabel{pv.2.280}\flagstanza{\tiny\textenglish{....2.280}}तिरस्कृतानां पटुनाप्येकदाऽभेददर्शनात् ।&प्रवाहे वित्तिभेदानां सिद्धा भेदव्यवस्थितिः ॥ २८० ॥\&[\smallbreak]


	
	  \endgroup
	

	  \pstart {\color{DodgerBlue3}“पटुना”} सुखदुःखादीनां वेदनास्कन्धसंगृहीतेन चैत्तेन {\color{DodgerBlue3}“तिरस्कृतानाम”}भिभूतानां संज्ञादी{\color{DodgerBlue3}“नामे”}कदाऽ{\color{DodgerBlue3}“भेददर्शनात् । प्रवाहे”} चित्त\edlabel{pvv.202-6}\footnote{\label{pvv.202-6}  ६ यदा कमनीये तृष्यति अन्यत्र द्वेष्टि रज्यते विरज्यते ।} सन्ताने सुखाद्यभावकाले स्व\edlabel{pvv.202-7}\footnote{\label{pvv.202-7}  ७ कदाचित्तिरस्कारेपि चित्ताभिसंस्कारकाले चेतना वेद्यत एव यतः ।}रूपेणोपलक्षितानां {\color{DodgerBlue3}“भेदव्यवस्थितिः सिद्धा”} । रवेरुदये तारका अनुपलक्षिता अपि निशायामुपलक्ष्यमाणा भेदेन व्यवस्थाप्यन्त एव ।\edlabel{pvv.202-8}\footnote{\label{pvv.202-8}  ८ नैवं प्रवाहेपि बुद्धिसंवेदनयोर्भेदवेदनमस्ति ।}तस्माज्‏ज्ञानात्मानः स्ववेदनाश्च सुखादय इत्याख्यातं स्ववेदनं ॥ X X ॥ (२८०)
	\pend
      \label{div_pvv.2.281}\edlabel{div_pvv.2.281}
	  
	% new div opening: depth here is 2
	

	  \begin{center}%% label @type='head'
	\textbf{(४) योगिज्ञानप्रत्यक्षम्}
	\end{center}
	

	  \pstart योगिज्ञानमाख्यातुमाह (।)
	\pend
      
	  \bigskip
	  \begingroup
	  \large
	
	    
	    \stanza[\smallbreak]
	\label{pv.2.281}\edlabel{pv.2.281}\flagstanza{\tiny\textenglish{....2.281}}प्रागुक्तं योगिनां ज्ञानं तेषां तद् भावनामयम् ।&विधूतकल्पनाजालं स्पष्टमेवावभासते ॥ २८१ ॥\&[\smallbreak]


	
	  \endgroup
	\leavevmode\marginnote{\textenglish{203/s}}

	  \pstart {\color{DodgerBlue3}“प्राक्”} प्रथमपरिच्छेदे\edlabel{pvv.203-1}\footnote{\label{pvv.203-1}  १ योगिनामप्यागमविकल्पाव्यवकीर्णमर्थमात्रदर्शनं प्रत्यक्षमिति व्याचष्टे ।}  {\color{DodgerBlue3}“योगिनां ज्ञानं”} सत्त्यविषयमुक्तं । {\color{DodgerBlue3}“तेषां”} योगिनां {\color{DodgerBlue3}“भावनामयं”} भावनाहेतुनिष्पत्तिकं {\color{DodgerBlue3}“तत्”} ज्ञानं सत्त्यस्वरूपविषयत्वेन {\color{DodgerBlue3}“विधूतकल्पनाजालम-”} विकल्पत्वाच्च {\color{DodgerBlue3}“स्पष्टं”} विशदज्ञेयाकार{\color{DodgerBlue3}“मेवावभासते”} । (२८१)
	\pend
      \label{div_pvv.2.282}\edlabel{div_pvv.2.282}
	  
	% new div opening: depth here is 2
	

	  \pstart भावनाभवं कथं स्पष्टमित्याह (।)
	\pend
      
	  \bigskip
	  \begingroup
	  \large
	
	    
	    \stanza[\smallbreak]
	\label{pv.2.282}\edlabel{pv.2.282}\flagstanza{\tiny\textenglish{....2.282}}कामशोकभयोन्मादचौरस्वप्राद्युपप्लुताः ।&अभूतानपि पश्यन्ति पुरतोवस्थितानिव ॥ २८२ ॥\&[\smallbreak]


	
	  \endgroup
	

	  \pstart {\color{DodgerBlue3}“कामश्च”} शोकश्च भयञ्च तैरुन्मादाश्चौरस्वप्नादयश्चेति {\color{DodgerBlue3}“कामशोकभयोन्मादचौरस्वप्नादिभिरुपप्लुता”} भ्रान्तास्ते{\color{DodgerBlue3}“ऽभूतानप्य”}र्थान् भावना\edlabel{pvv.203-2}\footnote{\label{pvv.203-2}  २ यद् भाव्यते तत् स्फुटं स्यादित्यस्य दृष्टान्तोयं श्लोकः ।}वशात् {\color{DodgerBlue3}“पुरतोऽवस्थिता-\leavevmode\marginnote{\textenglish{40a/MA}} निव पश्य\edlabel{pvv.203-3}\footnote{\label{pvv.203-3}  ३ अनुमेयं ।}न्ति”} । यस्मात्तदनुरूपां प्रवृत्तिं चेष्टन्ते ॥ (२८२)
	\pend
      \label{div_pvv.2.283}\edlabel{div_pvv.2.283}
	  
	% new div opening: depth here is 2
	

	  \pstart भवतु भावनाजं स्पष्टमविकल्पं तु कथमित्याह (।)
	\pend
      
	  \bigskip
	  \begingroup
	  \large
	
	    
	    \stanza[\smallbreak]
	\label{pv.2.283a}\edlabel{pv.2.283a}\flagstanza{\tiny\textenglish{...2.283a}}न विकल्पानुबद्धस्यास्ति स्फुटार्थावभासिता ॥\&[\smallbreak]


	
	  \endgroup
	

	  \pstart {\color{DodgerBlue3}“न विकल्पेनानुबद्धस्य”} संस्तुतस्य ज्ञानस्य {\color{DodgerBlue3}“स्फुटार्थावभासितास्ति”} ॥
	\pend
      

	  \pstart ननु विप्लववशात् विकल्पकमपि स्वप्ने स्पष्टाभं ज्ञानं भवतीत्याह (।)
	\pend
      
	  \bigskip
	  \begingroup
	  \large
	
	    
	    \stanza[\smallbreak]
	\label{pv.2.283b}\edlabel{pv.2.283b}\flagstanza{\tiny\textenglish{...2.283b}}स्वप्नेपि स्मर्यते स्मार्त्तं न च तत्तादृगर्थवत् ॥ २८३ ॥\&[\smallbreak]


	
	  \endgroup
	

	  \pstart {\color{DodgerBlue3}“स्वप्नेपि स्मार्त्तं”} स्मरणं किञ्चिदुत्पद्यते । {\color{DodgerBlue3}“न च”} तत्प्रबोधावस्थायां {\color{DodgerBlue3}“तादृगर्थवद्या”}दृशो निर्व्विकल्पेनानुभूतोऽर्थस्तादृशार्थेन युक्तं स्मर्यते । किन्तर्हि अस्पष्टार्थमेव स्वप्नस्मरणं स्मर्यते । (२८३)
	\pend
      \label{div_pvv.2.284}\edlabel{div_pvv.2.284}
	  
	% new div opening: depth here is 2
	

	  \pstart नन्वभूतार्थभावनाबलजं भवतापि स्पष्टाभमविकल्पञ्च सिद्धान्ते नेष्यते इति तेन सह विरोध इत्या\edlabel{pvv.203-4}\footnote{\label{pvv.203-4}  ४ द्विधा स्वप्नज्ञानं स्पष्टमेकमन्यदतीतस्वप्नाकारं ।} ह (।)
	\pend
      
	  \bigskip
	  \begingroup
	  \large
	
	    
	    \stanza[\smallbreak]
	\label{pv.2.284}\edlabel{pv.2.284}\flagstanza{\tiny\textenglish{....2.284}}अशुभा पृथिवी-कृत्स्नाद्यभूतमपि वर्ण्यते ।&स्पष्टाभं निर्विकल्पञ्च भावनाबलनिर्मितम् ॥ २८४ ॥\&[\smallbreak]


	
	  \endgroup
	

	  \pstart {\color{DodgerBlue3}“अशुभा”} विनीलकविपूयकास्थिसंकलादिका {\color{DodgerBlue3}“पृथ्वीकृत्स्नादि”} भूमयत्वादि {\color{DodgerBlue3}“अभूत”}मसत्यमपि {\color{DodgerBlue3}“भावनाबलेन निर्मितं स्पष्टाभं निर्व्विकल्पकञ्चा”}स्मा{\color{DodgerBlue3}“भिर्व्वर्ण्ण्यत”} इति नास्ति सिद्धान्तविरोधः । (२८४)
	\pend
      \label{div_pvv.2.285}\edlabel{div_pvv.2.285}
	  
	% new div opening: depth here is 2
	\leavevmode\marginnote{\textenglish{204/s}}
	  \bigskip
	  \begingroup
	  \large
	
	    
	    \stanza[\smallbreak]
	\label{pv.2.285}\edlabel{pv.2.285}\flagstanza{\tiny\textenglish{....2.285}}तस्माद् भूतमभूतं वा यद् यदेवाभिभाव्यते ।&भावनापरिनिष्पत्तौ तत् स्फुटाकल्पधीफलं ॥ २८५ ॥\&[\smallbreak]


	
	  \endgroup
	

	  \pstart यतो भावनाया भाव्यस्पष्टतायामाधिपत्यं । {\color{DodgerBlue3}“तस्माद् भूत”}मार्यसत्यादि {\color{DodgerBlue3}“अभूतम”}शुभादि {\color{DodgerBlue3}“यद्यदेवा”}त्यन्तं {\color{DodgerBlue3}“भाव्यते”} । तद् भाव्यमानं {\color{DodgerBlue3}“भावनायाः”} सादरनिरन्तरदीर्घकालप्रवर्त्तितायाः {\color{DodgerBlue3}“परिनिष्पत्तौ स्फुटा कल्पधीः”} सा फलं यस्य तत्तथा । (२८५)
	\pend
      \label{div_pvv.2.286}\edlabel{div_pvv.2.286}
	  
	% new div opening: depth here is 2
	
	  \bigskip
	  \begingroup
	  \large
	
	    
	    \stanza[\smallbreak]
	\label{pv.2.286}\edlabel{pv.2.286}\flagstanza{\tiny\textenglish{....2.286}}तत्र प्रमाणां सम्वादि यत्; प्राङ् निर्णीतवस्तुवत् ।&तद् भावनाजं प्रत्यक्षमिष्टं शेषा उपप्लवाः ॥ २८६ ॥\&[\smallbreak]


	
	  \endgroup
	

	  \pstart {\color{DodgerBlue3}“तत्र”} भावनाबलभाविषु स्पष्टानिर्व्विकल्पेषु {\color{DodgerBlue3}“यत्संवादि”} उपदर्शितार्थप्रापकं तद् भावनाजं प्रत्यक्षं {\color{DodgerBlue3}“प्रमाणमिष्टं”} ।
	\pend
      

	  \pstart किमिवेत्याह (।)
	\pend
      

	  \pstart {\color{DodgerBlue3}“प्राक्”} प्रथमपरिच्छेदे {\color{DodgerBlue3}“निर्ण्णीतं वस्तु”} सत्यचतुष्टयं तस्मिन्निव । यथा आर्यस\edlabel{pvv.204-1}\footnote{\label{pvv.204-1}  १ अव्यतिभिन्नाथो नरव्यापितार्थ उक्तः ।}त्त्यविषयं भावनाबलजं संवादित्वात्प्रत्यक्षं प्रमाणमेवमन्यदपीदृशं । {\color{DodgerBlue3}“शेषा\edlabel{pvv.204-2}\footnote{\label{pvv.204-2}  २ मात्रशब्दव्यवच्छेद्याः ।} अयथार्था उपप्लवा”} भ्रमा यथा अशुभा पृथ्वीकृत्स्नादिप्रत्ययाः । (२८६)
	\pend
      \label{div_pvv.2.287}\edlabel{div_pvv.2.287}
	  
	% new div opening: depth here is 2
	

	  \pstart कल्पनापि स्वसंवित्ताविष्टा नार्थे विकल्पनादिति व्याख्यातुमाह\edlabel{pvv.204-3}\footnote{\label{pvv.204-3}  ३ अत्रापि स्मृतिरस्तीति निर्व्विषयत्वं स्फुटयति ।} (।)
	\pend
      
	  \bigskip
	  \begingroup
	  \large
	
	    
	    \stanza[\smallbreak]
	\label{pv.2.287}\edlabel{pv.2.287}\flagstanza{\tiny\textenglish{....2.287}}शब्दार्थग्राहि यद् यत्र तज्ज्ञानं तत्र कल्पना ।&स्वरूपञ्च न शब्दार्थस्तत्राध्यक्षमतोऽखिलम् ॥ २८७ ॥\&[\smallbreak]


	
	  \endgroup
	

	  \pstart {\color{DodgerBlue3}“शब्दार्थग्राहि”} अन्यव्यवच्छेदस्य ग्राहकं {\color{DodgerBlue3}“यज्ज्ञानं यत्र”} विषये {\color{DodgerBlue3}“तज्ज्ञानं तत्र कल्पनोच्यते”} । ज्ञानानां {\color{DodgerBlue3}“स्वरूपञ्च”} स्वलक्षणात्मकं {\color{DodgerBlue3}“न शब्द”}स्या{\color{DodgerBlue3}“र्थो”} विषयो{\color{DodgerBlue3}“ऽतो”} वाच्यत्वा{\color{DodgerBlue3}“त्तत्र स्वरूपेऽखिलं”} ज्ञानमविकल्पत्वा{\color{DodgerBlue3}“दध्यक्षं”} । उक्तं चतुर्व्विधं प्रत्यक्षं ॥ X X ॥ (२८७)
	\pend
      
	  
	% new div opening: depth here is 1
	
\section[{७. प्रत्यक्षाभासचिन्ता}]{७. प्रत्यक्षाभासचिन्ता}\label{div_pvv.2.288}\edlabel{div_pvv.2.288}
	  
	% new div opening: depth here is 2
	

	  \pstart प्रत्यक्षाभासमिदानीं वक्तव्यं ।
	\pend
      
	  \bigskip
	  \begingroup
	  \large
	
	    
	    \stanza[\smallbreak]
	\label{pv.2.288}\edlabel{pv.2.288}\flagstanza{\tiny\textenglish{....2.288}}त्रिविधं कल्पनाज्ञानमाश्रयोपप्लवोद्‏भवम् ।&अविकल्पकमेकञ्च प्रत्यज्ञाभं चतुर्विधम् ॥ २८८ ॥\&[\smallbreak]


	
	  \endgroup
	\leavevmode\marginnote{\textenglish{205/s}}

	  \pstart {\color{DodgerBlue3}“त्रिविधं कल्पनाज्ञानं\edlabel{pvv.205-1}\footnote{\label{pvv.205-1}  १ “भ्रान्तिसंवृतिसत् ज्ञानमनुमा (ना) नुमानिकं । स्मार्त्ताभिलापिकञ्तेति प्रत्यक्षाभं सतैमिरं” ॥ व्याचष्टे ॥ संकेताश्रयकल्पना घटादि । अर्थान्तरारोपकल्पना मरीचिषु जलं । परोक्षार्थकल्पनाऽनुमानानुमानिकादिषु सम्बन्धकालदृष्टैकत्वेन वृत्ता ॥}”} प्रत्यक्षाभं मरीचिकायां जलाध्यवसायि भ्रान्तिज्ञानं संवृतौ वि\edlabel{pvv.205-2}\footnote{\label{pvv.205-2}  २ सत्त्वं द्रव्यं घटसंख्याक्षेपणसत्ता घटत्वादिषु ।} संवादिव्यवसायसांवृतज्ञानं । पूर्व्वदृष्टैकत्वकल्पनाप्रवृत्तं लिङ्गानुमेयादिज्ञानं । {\color{DodgerBlue3}“अविकल्पकञ्च एकं प्रत्यक्षाभं”} । कीदृशमा{\color{DodgerBlue3}“श्रयस्ये”}न्द्रियस्यो{\color{DodgerBlue3}“पप्लवस्ति”}मिरा द्युपघातस्तस्मा{\color{DodgerBlue3}“द् भवो”} यस्य तत्तथा । एवञ्च {\color{DodgerBlue3}“चतुर्विधञ्च”} प्रत्यक्षाभासं ॥ (२८८)
	\pend
      \label{div_pvv.2.289}\edlabel{div_pvv.2.289}
	  
	% new div opening: depth here is 2
	

	  \pstart नन्वविकल्पकं प्रत्यक्षं । ततस्त्रयमपीदं सविकल्पकत्वादेकः प्रत्यक्षाभासः । तत् किं भ्रान्तिज्ञानं मृगतृष्णिकायां जलावसायि । संवृतिसतो द्रव्यादेर्ज्ञानं । अनुमानं लिङ्गज्ञानं आनुमानिकं लिङ्गिज्ञानं स्मार्त्त स्मृतिः आभिलापिकं चेति विकल्पप्रभेद आचार्य दि ग्ना गे नोक्त इत्याह (।)
	\pend
      
	  \bigskip
	  \begingroup
	  \large
	
	    
	    \stanza[\smallbreak]
	\label{pv.2.289}\edlabel{pv.2.289}\flagstanza{\tiny\textenglish{....2.289}}अनक्षजत्वसिद्ध्यर्थमुक्ते द्वे भ्रान्तिदशनात् ॥&सिद्धानुमादिवचनं साधनायैव पूर्व्वयोः ॥ २८९ ॥\&[\smallbreak]


	
	  \endgroup
	

	  \pstart द्वे सांवृतारोपितयोः कल्पनाज्ञानेऽ{\color{DodgerBlue3}“नक्षजत्व”}स्यानिन्द्रियत्वस्य {\color{DodgerBlue3}“सिद्ध्यर्थं”} भेदे{\color{DodgerBlue3}“नोक्ते”} । परेषां तयोरक्षजत्वभ्रान्तिदर्शनाद् । घटोयं द्वौ\edlabel{pvv.205-3}\footnote{\label{pvv.205-3}  ३ कल्पनमपि स्वसम्वित्तावध्यक्षं स्वसम्वेद्यत्वात् सुखादिवत् ।} कम्पत इत्यादि । जलमिदमिति च व्यवसायात्मक{\color{DodgerBlue3}“मिन्द्रियप्रत्यक्षमेव”} प्रतिपद्यत इति परो मन्यते । तन्निरा सार्थं द्वयोरुपादानं । अनिन्द्रियत्वेन स्मृतिबलभावित्वेन सिद्धं च तद{\color{DodgerBlue3}“नुमादि”} च तस्य {\color{DodgerBlue3}“वचनं पूर्व्वयोः”} संवृतारोपितकल्पनयोरेवा{\color{DodgerBlue3}“नक्षजत्वसाधनाय”} । तथा हि यत्पूर्व्वानुभूतसमयस्मृतिभावि न तत्प्रत्यक्षं । यथाऽनुमानादि । अनुभूतसमयस्मृतिसापेक्षा चावयविजलादिकल्पना इति विरुद्धोपलब्धिरुक्ता । (२८९)
	\pend
      \label{div_pvv.2.290}\edlabel{div_pvv.2.290}
	  
	% new div opening: depth here is 2
	

	  \pstart ननु सांवृतारोपितकल्पनायाः कथं प्रत्यक्षताशङ्केत्याह (।)
	\pend
      
	  \bigskip
	  \begingroup
	  \large
	
	    
	    \stanza[\smallbreak]
	\label{pv.2.290}\edlabel{pv.2.290}\flagstanza{\tiny\textenglish{....2.290}}संकेतसंश्रयान्यार्थसमारोपविकल्पने ।&न प्रत्यक्षानुवृत्तित्वात् कदाचिद् भ्रान्तिकारणम् ॥ २९० ॥\&[\smallbreak]


	
	  \endgroup
	

	  \pstart बहूनां रूपादीनामेकार्थकारित्वख्यापनार्थं घट इत्यादिशब्दनिवेशः स संश्रयो हेतुर्यस्याः सा {\color{DodgerBlue3}“संकेत\edlabel{pvv.205-4}\footnote{\label{pvv.205-4}  ४ संकेतः परमाणुषु । संख्यासमच्चयव्यवच्छेदेन ।}संश्रया”}कल्पना दृश्यमानान्मरीचिनिचयादेरन्यस्य जलादेरा\leavevmode\marginnote{\textenglish{206/s}} रोपितस्य कल्पना । {\color{DodgerBlue3}“अन्यार्थ”}कल्पना । ते एते कल्पने प्रत्यक्षानन्तरभावित्वेन {\color{DodgerBlue3}“प्रत्यक्षानुवृत्तित्वादा”}त्मन्यपि {\color{DodgerBlue3}“कदाचिद”}विमर्शकानां प्रत्यक्षता---{\color{DodgerBlue3}“भ्रान्तिकारणं”} भवतः । (२९०)
	\pend
      \label{div_pvv.2.291}\edlabel{div_pvv.2.291}
	  
	% new div opening: depth here is 2
	
	  \bigskip
	  \begingroup
	  \large
	
	    
	    \stanza[\smallbreak]
	\label{pv.2.291}\edlabel{pv.2.291}\flagstanza{\tiny\textenglish{....2.291}}यथैवेयं परोक्षार्थकल्पना स्मरणात्मिका ।&समयापेक्षिणी नार्थं प्रत्यक्षमध्यवस्यति ॥ २९१ ॥\&[\smallbreak]


	
	  \endgroup
	\leavevmode\marginnote{\textenglish{40b/MA}}

	  \pstart अतो {\color{DodgerBlue3}“यथैवेयम”}नुमानानुमानिकादि{\color{DodgerBlue3}“परोक्षार्थकल्पना”} पूर्व्वं गृहीत{\color{DodgerBlue3}“समयापेक्षिणी”} सती {\color{DodgerBlue3}“स्मरणा”}दिरूपा {\color{DodgerBlue3}“प्रत्यक्षं”} प्रत्यक्षविषय{\color{DodgerBlue3}“मर्थं ना”} ({\color{DodgerBlue3}“ध्य”}) {\color{DodgerBlue3}“वस्यति”} । किन्तु कल्पितमेव गृह्णाति (। २९१)
	\pend
      \label{div_pvv.2.292}\edlabel{div_pvv.2.292}
	  
	% new div opening: depth here is 2
	
	  \bigskip
	  \begingroup
	  \large
	
	    
	    \stanza[\smallbreak]
	\label{pv.2.292}\edlabel{pv.2.292}\flagstanza{\tiny\textenglish{....2.292}}तथानुभूतस्मरणमन्तरेण घटादिषु ।&न प्रत्ययोनुयँस्तच्च प्रत्यक्षात् परिहीयते ॥ २९२ ॥\&[\smallbreak]


	
	  \endgroup
	

	  \pstart {\color{DodgerBlue3}“तथा”} संकेतकाले व्यवहारलघवार्थं कल्पितस्य द्रव्यादेरनु{\color{DodgerBlue3}“मानभूत”}स्य पश्चा{\color{DodgerBlue3}“त्स्मरणमन्तरेण घटादिषु”} घटोऽयमित्याद्येकवस्त्ववसायी {\color{DodgerBlue3}“प्रत्ययो”} न भवति । {\color{DodgerBlue3}“तच्च”} संकेतितमारोपितं चार्थमनुयन् घटोऽयं जलमिति च प्रत्ययः {\color{DodgerBlue3}“प्रत्यक्षात्”} प्रत्यक्ष{\color{DodgerBlue3}“त्वात् परिहीयते”} । तस्य वस्तुविषयत्वात् संकेतापेक्षस्य च विपर्ययात् । (२९२)
	\pend
      \label{div_pvv.2.293}\edlabel{div_pvv.2.293}
	  
	% new div opening: depth here is 2
	
	  \bigskip
	  \begingroup
	  \large
	
	    
	    \stanza[\smallbreak]
	\label{pv.2.293}\edlabel{pv.2.293}\flagstanza{\tiny\textenglish{....2.293}}अपवादश्चतुर्थोत्र तेनोक्तमुपघातजम् ।&केवलन्तत्र तिमिरमुपघातोपलक्षणम् ॥ २९३ ॥\&[\smallbreak]


	
	  \endgroup
	

	  \pstart सतैमिरमिति {\color{DodgerBlue3}“चतुर्थो”} हेत्वाभासः कल्पनारहितत्वेनातिप्रसक्तायाः प्रत्यक्षताया {\color{DodgerBlue3}“अपवादः”} । अभ्रान्तत्वस्य लक्षणैकदेशत्वोपलक्षणत्वं च न तु कल्पनापोढत्वनिराकृतस्योदाहरणमिदं । {\color{DodgerBlue3}“केवलं सतैमिरमि”}त्यत्र तिमिरं सबाह्यो{\color{DodgerBlue3}“पघातस्य”} ज्ञानविकृतिहेतो{\color{DodgerBlue3}“रुपलक्षणं”} । तेनान्ये\edlabel{pvv.206-1}\footnote{\label{pvv.206-1}  १ श्रोत्रादि ।}न्द्रियविकारजमिन्द्रियजञ्च निर्व्विकल्पं प्रत्यक्षाभासं सि\edlabel{pvv.206-2}\footnote{\label{pvv.206-2}  २ उक्तं ।}ध्यति (॥ २९३)
	\pend
      \label{div_pvv.2.294}\edlabel{div_pvv.2.294}
	  
	% new div opening: depth here is 2
	
	  \bigskip
	  \begingroup
	  \large
	
	    
	    \stanza[\smallbreak]
	\label{pv.2.294}\edlabel{pv.2.294}\flagstanza{\tiny\textenglish{....2.294}}मानसं तदपीत्येके तेषां ग्रन्थो विरुध्यते ।&नीलद्विचन्द्रादिधियां हेतुरक्षाण्यपीत्ययम् ॥ २९४ ॥\&[\smallbreak]


	
	  \endgroup
	

	  \pstart {\color{DodgerBlue3}“तद्”} द्विचन्द्रादिज्ञान{\color{DodgerBlue3}“मपि मानसं”} मनोभ्रम {\color{DodgerBlue3}“इत्येके”}\edlabel{pvv.206-3}\footnote{\label{pvv.206-3}  ३ कणादादयः ।} आ चा र्याः । {\color{DodgerBlue3}“तेषा”}मेवं वादिनां {\color{DodgerBlue3}“नीलद्विचन्द्रादिधियामक्षाण्यपि हेतु”}रित्येतदर्थवाचको {\color{DodgerBlue3}“ग्रन्थो विरुध्यते । ग्रन्थः”} पुनरयं यावच्चक्षुरादीनामप्या{\color{DodgerBlue3}“लम्बनत्वप्रसङ्गः”} । तेपि हि परमार्थ\leavevmode\marginnote{\textenglish{207/s}} तोऽन्यथा विद्य\edlabel{pvv.207-1}\footnote{\label{pvv.207-1}  १ अजाजीपुष्पतुल्येनासरूपकत्वेन ।}माना नीलाद्याभासस्य द्विचन्द्राद्याभासस्य च ज्ञानस्य कारणीभवन्तीति । (२९४)
	\pend
      \label{div_pvv.2.295}\edlabel{div_pvv.2.295}
	  
	% new div opening: depth here is 2
	

	  \pstart स्यादेतत् (।)
	\pend
      
	  \bigskip
	  \begingroup
	  \large
	
	    
	    \stanza[\smallbreak]
	\label{pv.2.295}\edlabel{pv.2.295}\flagstanza{\tiny\textenglish{....2.295}}पारम्पर्येण हेतुश्चेदिन्द्रियज्ञानगोचरे ।&विचार्यमाणे प्रस्तावो मानसस्येह कीदृशः ॥ २९५ ॥\&[\smallbreak]


	
	  \endgroup
	

	  \pstart मानसस्य प्रत्य\edlabel{pvv.207-2}\footnote{\label{pvv.207-2}  २ तदनन्तरजत्वात् ।}क्षस्येन्द्रियं {\color{DodgerBlue3}“पारम्पर्येण हेतुः”} । तेन विरोधाभाव{\color{DodgerBlue3}“श्चेत्\edlabel{pvv.207-3}\footnote{\label{pvv.207-3}  ३ यदुक्तं दिग्नागेन बौद्धं प्रत्येतत् ।}”} वा द वि धि प्र करणे {\color{DodgerBlue3}“इन्द्रियज्ञान”}स्य प्रत्यक्षस्य {\color{DodgerBlue3}“गोचरेविचार्यमाणे मानसस्य”} विकल्पस्य इहावसरे कीदृशः {\color{DodgerBlue3}“प्रस्तावः”} । येन परम्परया तद्धेतुरिन्द्रियमुच्यते । (२९५)
	\pend
      \label{div_pvv.2.296}\edlabel{div_pvv.2.296}
	  
	% new div opening: depth here is 2
	
	  \bigskip
	  \begingroup
	  \large
	
	    
	    \stanza[\smallbreak]
	\label{pv.2.296}\edlabel{pv.2.296}\flagstanza{\tiny\textenglish{....2.296}}किम्वैन्द्रियं यदक्षाणां भावाभावानुरोधि चेत् ।&तत्तुल्यं विक्रियावच्चेत् सैवेयं किं निषिध्यते ॥ २९६ ॥\&[\smallbreak]


	
	  \endgroup
	

	  \pstart तिमिरज्ञानं भ्रममनिच्छतोपि {\color{DodgerBlue3}“किं”} कीदृश{\color{DodgerBlue3}“मैन्द्रिय”}ज्ञानमिष्यते । {\color{DodgerBlue3}“यदक्षाणां भावाभावयोरनुरोधि”} स्वभावाभ्यामनुवर्तकं तदैन्द्रियमिति {\color{DodgerBlue3}“चेत्”} । तत् इन्द्रिय{\color{DodgerBlue3}“भावाभा”}वानुरोधित्वं तैमिरिकज्ञानस्यापि {\color{DodgerBlue3}“तुल्यं”} (।) न हीन्द्रियव्यापारमन्तरेण तैमिरिकज्ञानमुत्पद्यते । इन्द्रि\edlabel{pvv.207-4}\footnote{\label{pvv.207-4}  ४ अथ पूर्व्वकं हित्वा ।}यविकारेण विक्रियावत् ज्ञानमैन्द्रियं चेत् द्विचन्द्रादिज्ञानानां तिमिरादीन्द्रियविकारेण {\color{DodgerBlue3}“विक्रिया”} (।) {\color{DodgerBlue3}“सैवेय”}मभूतार्थोपदर्शनान्मिका भ्रान्तत्वनिमित्तमुक्ताऽस्माभिः {\color{DodgerBlue3}“किं निषिध्यते”} । (२९६)
	\pend
      \label{div_pvv.2.297}\edlabel{div_pvv.2.297}
	  
	% new div opening: depth here is 2
	
	  \bigskip
	  \begingroup
	  \large
	
	    
	    \stanza[\smallbreak]
	\label{pv.2.297}\edlabel{pv.2.297}\flagstanza{\tiny\textenglish{....2.297}}सर्प्पादिभ्रान्तिवच्चास्याः स्यादक्षविकृतावपि ।&निवृत्तिर्न निवर्त्तेत निवृत्तेप्यक्षविप्लवे ॥ २९७ ॥\&[\smallbreak]


	
	  \endgroup
	

	  \pstart यदि च द्विचन्द्रादिधीर्मानसी भ्रान्तिस्तदा सर्पोचितदेशे मन्दमन्दालोके रज्वादौ संस्थानसाम्यग्रहात् उत्पन्नायाः सर्पादिभ्रान्तेरिव मरीचिषु जलभ्रान्तेरिव यथा प्रत्यक्षं विचारात् । अस्य(ा) द्विचन्द्रादिभ्रान्तेरक्ष{\color{DodgerBlue3}“विकृतावपि”} सा निवृत्तिः स्यात् । अक्षविप्लवो हि न तस्या हेतुः (।) ततश्च सत्यपि तस्मिन् विमर्शान्निवर्तते सर्पबुद्धिरिव । तथा ह्यक्ष{\color{DodgerBlue3}“विप्लवेपि”} तिमिरादौ {\color{DodgerBlue3}“निवृत्तेपि”} तत्त्वमविचारयतो {\color{DodgerBlue3}“न निवर्तते”} तद्‏द्विचन्द्रबुद्धिः । अक्षविप्लवस्य तत्कारणत्वाभावात् । अकारणनिवृत्तौ च निवृत्त्ययोगात् । (२९७)
	\pend
      \label{div_pvv.2.298}\edlabel{div_pvv.2.298}
	  
	% new div opening: depth here is 2
	

	  \pstart किञ्च (।)
	\pend
      
	  \bigskip
	  \begingroup
	  \large
	
	    
	    \stanza[\smallbreak]
	\label{pv.2.298}\edlabel{pv.2.298}\flagstanza{\tiny\textenglish{....2.298}}कदाचिदन्यसन्ताने तथैवार्प्येत वाचकैः ।&दृष्टस्मृतिमपेक्षेत न भासेत परिस्फुटम् ॥ २९८ ॥\&[\smallbreak]


	
	  \endgroup
	\leavevmode\marginnote{\textenglish{208/s}}

	  \pstart यथानुभवं समदेशकालस्यान्यस्य चित्तसन्ताने यथा अहिरहिरित्युपदर्शनेन सर्पभ्रान्तिरर्प्यते । {\color{DodgerBlue3}“तथा”} द्विचन्द्रादिभ्रान्तिरपि तस्या वाचकैः शब्दैर{\color{DodgerBlue3}“र्प्येत”} सामग्रीतुल्यत्वात् । तिमिरस्य चाहेतुकत्वात् । अपि च(।) यथा मरीचिषु तरङ्गजलसमासु पूर्व्वदृष्टजलस्मरणसापेक्षा जलभ्रान्तिः । तत्र द्विचन्द्रादिभ्रान्तिरपि मानसीत्वात् {\color{DodgerBlue3}“दृष्ट”}चन्द्रद्वय{\color{DodgerBlue3}“स्मृतिमपेक्षेत”} । न चेयं स्मरणसापेक्षा चक्षुर्व्विस्फारणमात्रेण स्फरणात् । \edlabel{pvv.208-1}\footnote{\label{pvv.208-1}  १ “न विकल्पानुबद्धस्य स्पष्टार्थप्रतिभासिता” ।}तथा मानसीत्वात् {\color{DodgerBlue3}“परिस्फुटं”} सुव्यक्तग्राह्याकारा {\color{DodgerBlue3}“न भासेत”} ।\edlabel{pvv.208-2}\footnote{\label{pvv.208-2}  २ कल्पनापोढवचनादेव निवृत्तेः । अभ्रान्तत्वे यत्नवैर्थ्यं स्यात् ।}जलादिभ्रान्तिरिव । (२९८)
	\pend
      \label{div_pvv.2.299}\edlabel{div_pvv.2.299}
	  
	% new div opening: depth here is 2
	
	  \bigskip
	  \begingroup
	  \large
	
	    
	    \stanza[\smallbreak]
	\label{pv.2.299}\edlabel{pv.2.299}\flagstanza{\tiny\textenglish{....2.299}}सुप्तस्य जाग्रतो वापि यैव धीः स्फुटभासिनी ।&सा निर्व्विकल्पोभयथाप्यन्यथैव विकल्पिका ॥ २९९ ॥\&[\smallbreak]


	
	  \endgroup
	

	  \pstart तस्मा{\color{DodgerBlue3}“त्सुप्तस्य जाग्रतोपि वा यैव धीः स्फुटावभासिनी”} व्यक्तग्राह्याकारा सा \leavevmode\marginnote{\textenglish{41a/MA}} {\color{DodgerBlue3}“निर्व्विकल्पा”}ऽभ्युपगन्तव्या {\color{DodgerBlue3}“अन्यथैव”} ह्यस्फुटावभासिनी धीरुभयथा सुप्तस्य जाग्रतो पि वा {\color{DodgerBlue3}“कल्पिका”} युक्ता । (२९९)
	\pend
      \label{div_pvv.2.300}\edlabel{div_pvv.2.300}
	  
	% new div opening: depth here is 2
	

	  \pstart यत एवं (।)
	\pend
      
	  \bigskip
	  \begingroup
	  \large
	
	    
	    \stanza[\smallbreak]
	\label{pv.2.300}\edlabel{pv.2.300}\flagstanza{\tiny\textenglish{....2.300}}तस्मात्तस्याविकल्पेपि प्रामाण्यं प्रतिषिध्यते ।&विसंवादात्तदर्थं च प्रत्यक्षाभं द्विधोदितम् ॥ ३०० ॥\&[\smallbreak]


	
	  \endgroup
	

	  \pstart {\color{DodgerBlue3}“तस्मात्”} कल्पनापोढमित्युक्ते तिमिरज्ञानस्या{\color{DodgerBlue3}“विकल्पे”} निर्व्विकल्पत्वेपि सति {\color{DodgerBlue3}“प्रामाण्यं”} प्रत्यक्षात्मकं प्राप्तं सतैमिरमित्यपवादेन {\color{DodgerBlue3}“विसम्वादात् प्रतिषिध्यते”} । संवादलक्षणत्वात्प्रामाण्यस्य । {\color{DodgerBlue3}“तदर्थमु”}पप्लुतज्ञाननिवृत्त्यर्थ चाचार्य दिङ् ना गे न प्रत्य {\color{DodgerBlue3}“क्षाभं”} संक्षेपतो {\color{DodgerBlue3}“द्विधोक्तं”} सविकल्पमविकल्पञ्च । उक्तं प्रत्यक्षाभं ॥ XX ॥(३००)
	\pend
      
	  
	% new div opening: depth here is 1
	
\section[{८. प्रमाणफलचिन्ता}]{८. प्रमाणफलचिन्ता}\label{div_pvv.2.301}\edlabel{div_pvv.2.301}
	  
	% new div opening: depth here is 2
	

	  \pstart प्रमाणफलव्यवस्थां कर्तुमाह ।
	\pend
      
	  \bigskip
	  \begingroup
	  \large
	
	    
	    \stanza[\smallbreak]
	\label{pv.2.301}\edlabel{pv.2.301}\flagstanza{\tiny\textenglish{....2.301}}क्रियासाधनमित्येव सर्व्वं सर्व्वस्य कर्म्मणः ।&साधनं न हि तत्तस्य साधनं या क्रिया यतः ॥ ३०१ ॥\&[\smallbreak]


	
	  \endgroup
	

	  \pstart {\color{DodgerBlue3}“क्रियायाः साधनं”} हेतुरि{\color{DodgerBlue3}“त्येव न”} हि {\color{DodgerBlue3}“सर्व्वं”} कारणं {\color{DodgerBlue3}“सर्व्वस्य कर्म्मणः”} क्रियायाः {\color{DodgerBlue3}“साधनं”} करणं (।) किन्तर्हि तद्वस्तु {\color{DodgerBlue3}“तस्य”} कर्मणः {\color{DodgerBlue3}“साधनं”} करणं {\color{DodgerBlue3}“या क्रिया यतः”} पदार्थाद\leavevmode\marginnote{\textenglish{209/s}} व्यवधानेन भवति स तस्याः कारणमुच्यते । ततश्चेन्द्रियादेः प्रमितिं प्रत्यव्यवहितसाधकत्वाभावात् न प्रमाणं । (३०१)
	\pend
      \label{div_pvv.2.302}\edlabel{div_pvv.2.302}
	  
	% new div opening: depth here is 2
	

	  \pstart किन्तर्हि प्रमाणमस्तीत्याह (।)
	\pend
      
	  \bigskip
	  \begingroup
	  \large
	
	    
	    \stanza[\smallbreak]
	\label{pv.2.302}\edlabel{pv.2.302}\flagstanza{\tiny\textenglish{....2.302}}तत्रानुभवमात्रेण ज्ञानस्य सदृशात्मनः ।&भाव्यन्तेनात्मना येन प्रतिकर्म्म विभज्यते ॥ ३०२ ॥\&[\smallbreak]


	
	  \endgroup
	

	  \pstart {\color{DodgerBlue3}“तत्र”} रूपादौ कर्मणि {\color{DodgerBlue3}“ज्ञानस्यानुभवमात्रेणा”}नुभवात्मनो {\color{DodgerBlue3}“सदृशात्म”}नस्तुल्यरूपस्य {\color{DodgerBlue3}“तेनात्मना”} स्वरूपेण प्रतिविषयं व्यतिरेकिणा {\color{DodgerBlue3}“भाव्यं येन प्रतिकर्म्म”} प्रतिविषयज्ञानं {\color{DodgerBlue3}“विभज्यते”} । नीलस्येदं पीतस्येदमिति । अन्यथानुभवमात्रतया सर्व्वत्र विषये सदृशं ज्ञानं प्रतिविषयं कथं भेदेन व्यवस्थापयितुं शक्येत । (३०२)
	\pend
      \label{div_pvv.2.303}\edlabel{div_pvv.2.303}
	  
	% new div opening: depth here is 2
	

	  \pstart स्यादेतदिन्द्रियादेर्हेतोः सव्यापा\edlabel{pvv.209-1}\footnote{\label{pvv.209-1}  १ आविलतादि ।} रतादिलक्षणो विशेषो ज्ञानानां भेदेन नियामकमित्याह (।)
	\pend
      
	  \bigskip
	  \begingroup
	  \large
	
	    
	    \stanza[\smallbreak]
	\label{pv.2.303}\edlabel{pv.2.303}\flagstanza{\tiny\textenglish{....2.303}}अनात्मभूतो भेदोस्य विद्यमानोपि हेतुषु ।&भिन्ने कर्मण्यभिन्नस्य न भेदेन नियामकः ॥ ३०३ ॥\&[\smallbreak]


	
	  \endgroup
	

	  \pstart {\color{DodgerBlue3}“हेतु”}ष्विन्द्रियादिषु {\color{DodgerBlue3}“भेदो”} विशेषः सव्यापारतादिलक्षणो {\color{DodgerBlue3}“विद्यमानोप्यस्य”} ज्ञान\edlabel{pvv.209-2}\footnote{\label{pvv.209-2}  २ विषयसारूप्यानधिगमे निराकारवादिनः सर्व्वं ज्ञानं बोधरूपतामात्रेणावशिष्टं विषयाधिगते हेतुरिति स्थितिः ।}स्यानुभवमात्रात्मतया भिन्नस्य {\color{DodgerBlue3}“भिन्ने कर्म्मणि”} नीलादौ ग्राह्ये {\color{DodgerBlue3}“भेदेन नियामको”} न युक्तः । कस्मादित्याह (।) {\color{DodgerBlue3}“अनात्मभूतः”} । ज्ञानास्वरूपत्वादिन्द्रियस्य विशेषः प्रतिकर्म्म न भेत्तुमर्हति । (३०३)
	\pend
      \label{div_pvv.2.304}\edlabel{div_pvv.2.304}
	  
	% new div opening: depth here is 2
	
	  \bigskip
	  \begingroup
	  \large
	
	    
	    \stanza[\smallbreak]
	\label{pv.2.304}\edlabel{pv.2.304}\flagstanza{\tiny\textenglish{....2.304}}तस्माद् यतोस्यात्मभेदादस्याधिगतिरित्ययम् ।&क्रियायाः कर्मनियमः सिद्धा सा तत्प्रसाधना ॥ ३०४ ॥\&[\smallbreak]


	
	  \endgroup
	

	  \pstart {\color{DodgerBlue3}“तस्मादस्य”} ज्ञानस्या{\color{DodgerBlue3}“त्मभेदात् यतोऽस्या”}र्थस्येयम{\color{DodgerBlue3}“धिगतिरिति”} क्रियाया अधिगतेः {\color{DodgerBlue3}“कर्मणि”} वेद्ये {\color{DodgerBlue3}“नियमः सा क्रिया तत्प्रसाधना”} तत्करणाऽभ्युपगन्तव्या । (३०४)
	\pend
      \label{div_pvv.2.305}\edlabel{div_pvv.2.305}
	  
	% new div opening: depth here is 2
	

	  \pstart स्यादेतद् (।) इन्द्रियादिरेव स्वभेदाद् भेदको ज्ञानस्य प्रतिविषयमधिगतेर्नियामकः । ततश्चानुभवात्मत्वादविषय एवासिद्ध इत्याह (।)
	\pend
      
	  \bigskip
	  \begingroup
	  \large
	
	    
	    \stanza[\smallbreak]
	\label{pv.2.305}\edlabel{pv.2.305}\flagstanza{\tiny\textenglish{....2.305}}अर्थेन घटयत्येनां न हि मुक्त्‏वार्थरूपताम् ।&अन्यः स्वभेदाज्ज्ञानस्य भेदकोपि कथञ्चन ॥ ३०५ ॥\&[\smallbreak]


	
	  \endgroup
	\leavevmode\marginnote{\textenglish{210/s}}

	  \pstart {\color{DodgerBlue3}“एनाम”}धिगतिम{\color{DodgerBlue3}“र्थरूपता”}मर्थसरू\edlabel{pvv.210-1}\footnote{\label{pvv.210-1}  १ अत्रैव तादात्म्यात् ।} पतां {\color{DodgerBlue3}“मुक्त्वा न”} ह्यन्यः कश्चिन्द्रियादिः {\color{DodgerBlue3}“स्वभेदात् कथञ्चन”} केनापि प्रकारेण {\color{DodgerBlue3}“ज्ञानस्य भेद\edlabel{pvv.210-2}\footnote{\label{pvv.210-2}  २ इन्द्रियमाविलं ज्ञानञ्च तथा ।}कोप्यर्थेव”} ज्ञेयेन {\color{DodgerBlue3}“घटयति”} योजयति नीलस्येयमधिगतिः पीतस्य चेयमित्यादि । तथा हि यद्यपि प्रत्यर्थं प्रतीन्द्रियञ्च ज्ञानानामस्ति भेदः तथापि विषयसारूप्याभावे स एव विशेषोऽशक्यनिर्देशः । अथ विशेषोप्यधिगतेरेव व्यवस्थाप्यत इति चेत् । ननु तस्या एव\edlabel{pvv.210-3}\footnote{\label{pvv.210-3}  ३ अधिगतेरेव व्यवस्थापकं सारूप्यस्य ।} व्यवस्थापकमिष्यते । न चाव्यस्थिते व्यवस्थापके व्यवस्थाप्यसिद्धिः । ननु न सर्वत्र परशुव्यापारदृष्टिपूर्व्विकाच्छिदासिद्धिः । छिदादर्शनादपि परशुव्यापारव्यवस्थितेः । एवमिहाप्यधिगतिपूर्व्विकायां प्रमाणस्थितौ न दोषः । असमानमेतत् । तथा हि परशुच्छिदयोः कार्यकारणभावविशेषः क्रियाकरणभावः । अधिगतिज्ञानात्मभूतविशेष\edlabel{pvv.210-4}\footnote{\label{pvv.210-4}  ४ सारूप्यं ।}योस्तु व्यवस्थाप्यव्यवस्थाकभाव एषितव्यः । उभयोरपि ज्ञानस्वरूपात्मत्वात् । तत्र कारणमज्ञातमपि स्वकार्यं निर्वर्तयतीति कार्यदर्श--{\color{DodgerBlue3}“नाच्च तद्‏व्यवस्था”} युक्तैव । व्यवस्थापकस्तु नानुपलक्षितो व्यवस्थाप्य व्यवस्थायां क्षमते । न ह्यप्रतीतं कल्प्यमानं आश्वस्त्यं\edlabel{pvv.210-5}\footnote{\label{pvv.210-5}  ५ आश्वतस्तस्य भावः ।}व्यवस्थापयति । (३०५)
	\pend
      \label{div_pvv.2.306}\edlabel{div_pvv.2.306}
	  
	% new div opening: depth here is 2
	
	  \bigskip
	  \begingroup
	  \large
	
	    
	    \stanza[\smallbreak]
	\label{pv.2.306}\edlabel{pv.2.306}\flagstanza{\tiny\textenglish{....2.306}}तस्मात्प्रमेयाधिगतेः साधनं मेयरूपता ।&साधनेऽन्यत्र तत्कर्मसम्बन्धो न प्रसिध्यति ॥ ३०६ ॥\&[\smallbreak]


	
	  \endgroup
	

	  \pstart {\color{DodgerBlue3}“तस्मात्प्रमेयाधिगतेः”} फलभूतायाः व्यवस्थाप्यायाः {\color{DodgerBlue3}“साधनं”} प्रमाणं {\color{DodgerBlue3}“मेयरूपता”} । \leavevmode\marginnote{\textenglish{41b/MA}} अथ न सारूप्यं तस्य प्रतिविषयं भिन्नस्य सूपलक्षणत्वात् सारूप्यात्पुनर {\color{DodgerBlue3}“न्यत्र साधने तस्याः”} क्रियायाः {\color{DodgerBlue3}“कर्म्मसम्बन्धो”} नीलस्येयमधिगतिः पीतस्य चेत्यादि {\color{DodgerBlue3}“न सिध्यति”} । इन्द्रियाधिगतिविशेषस्य सम्भवेप्यनुभवमात्रात्मकज्ञानस्याविशेषकत्वायोगात् । ज्ञानगतस्यापरविशेषस्य लक्षणभेदेनानुपलक्षणात् । (३०६)
	\pend
      \label{div_pvv.2.307}\edlabel{div_pvv.2.307}
	  
	% new div opening: depth here is 2
	
	  \bigskip
	  \begingroup
	  \large
	
	    
	    \stanza[\smallbreak]
	\label{pv.2.307a}\edlabel{pv.2.307a}\flagstanza{\tiny\textenglish{...2.307a}}सा च तस्यात्मभूतैव तेन नार्थान्तरं फलम् ।\&[\smallbreak]


	
	  \endgroup
	

	  \pstart {\color{DodgerBlue3}“सा चा”}धिगतिरनुभवस्वभावा ज्ञानस्या{\color{DodgerBlue3}“त्मभूतैव । तेन”} प्रमाणान्ना{\color{DodgerBlue3}“र्थान्तरं फलं”} । प्रमेयमेव फलमित्यर्थः ।
	\pend
      

	  \pstart ग्राह्यग्राहकभावोपि भाक्त एवेति दर्शयितुमाह (।)
	\pend
      
	  \bigskip
	  \begingroup
	  \large
	
	    
	    \stanza[\smallbreak]
	\label{pv.2.307b}\edlabel{pv.2.307b}\flagstanza{\tiny\textenglish{...2.307b}}दधानं तच्च तामात्मन्यर्थाधिगमनात्मना ॥ ३०७ ॥\&[\smallbreak]


	
	  \endgroup
	\leavevmode\marginnote{\textenglish{211/s}}

	  \pstart {\color{DodgerBlue3}“तच्च”} ज्ञानमा{\color{DodgerBlue3}“त्मनि”} तामर्थ{\color{DodgerBlue3}“सरूपतां दधानं”} बिभ्रद् अर्थस्या{\color{DodgerBlue3}“धिग\edlabel{pvv.211-1}\footnote{\label{pvv.211-1}  १ यदाकारमुत्पद्यते तत्र प्रमाणमुपचर्यते ।} मनात्मना”}ऽधिगमलक्षणेन (३०७)
	\pend
      \label{div_pvv.2.308}\edlabel{div_pvv.2.308}
	  
	% new div opening: depth here is 2
	
	  \bigskip
	  \begingroup
	  \large
	
	    
	    \stanza[\smallbreak]
	\label{pv.2.308}\edlabel{pv.2.308}\flagstanza{\tiny\textenglish{....2.308}}सव्यापारमिवाभाति व्यापारेण स्वकर्मणि ।&तद्वशात्तद्व्यवस्थानादकारकमपि स्वयम् ॥ ३०८ ॥\&[\smallbreak]


	
	  \endgroup
	

	  \pstart व्यापारेण सव्यापार{\color{DodgerBlue3}“मिव”} स्वकर्म्मणि ग्राह्ये {\color{DodgerBlue3}“आभाति\edlabel{pvv.211-2}\footnote{\label{pvv.211-2}  २ विना हि क्रियां न करणं । अथैतत् क्रियाव्याप्यं कर्म वेति विषयाधिगतिः कर्मव्यापारः प्रतिपत्तृधर्मत्वादस्या एतया हि विषयो व्याप्यते । व्यवसायवशाच्च कर्तृगतापि विषयगता व्यवस्थाप्यते यथाध्यवसायं सर्वव्यवहारवृत्तेः (।) तत्र च साधकतमस्वव्यापारेण तदानुकूल्योत्पत्त्या अर्थसारूप्यं करणमिति कर्त्ता कर्म करणं क्रिया चेति चतुष्टयमुक्तं ज्ञेयं ।}”} । व्यापारमपि वस्तुतोऽकारकमपि स्वयं तद्वशान्मेयसारूप्यवशात्तस्याधिगमस्य {\color{DodgerBlue3}“व्यवस्थानात्”} । (३०८)
	\pend
      \label{div_pvv.2.309}\edlabel{div_pvv.2.309}
	  
	% new div opening: depth here is 2
	
	  \bigskip
	  \begingroup
	  \large
	
	    
	    \stanza[\smallbreak]
	\label{pv.2.309}\edlabel{pv.2.309}\flagstanza{\tiny\textenglish{....2.309}}यथा फलस्य हेतूनां सदृशात्मतयोद्भवाद् ।&हेतुरूपग्रहो लोकेऽक्रियावत्वेपि कथ्यते ॥ ३०९ ॥\&[\smallbreak]


	
	  \endgroup
	

	  \pstart {\color{DodgerBlue3}“यथा लोकेपि हेतूनां सदृशात्मतया”} सदृशरूपतयो{\color{DodgerBlue3}“द्भवात् । फलस्याक्रियावत्त्वेपि”} हेतुरूपग्रहणव्यापाराभावेपि {\color{DodgerBlue3}“हेतुरूपग्रहः कथ्यते”} पितू रूपं गृहीतं सुतेनेत्यादि । अतोऽर्थरूपतां मुक्त्वाऽधिगतिसाधनमन्यदयुक्तं । (३०९)
	\pend
      \label{div_pvv.2.310}\edlabel{div_pvv.2.310}
	  
	% new div opening: depth here is 2
	
	  \bigskip
	  \begingroup
	  \large
	
	    
	    \stanza[\smallbreak]
	\label{pv.2.310}\edlabel{pv.2.310}\flagstanza{\tiny\textenglish{....2.310}}आलोचनाक्षसम्बन्धविशेषणधियामतः ।&नेष्टम्प्रामाण्यमेतेषां व्यवधानात् क्रियाम्प्रति ॥ ३१० ॥\&[\smallbreak]


	
	  \endgroup
	

	  \pstart अत आ{\color{DodgerBlue3}“लोचन”}स्यार्थालोचनमा\edlabel{pvv.211-3}\footnote{\label{pvv.211-3}  ३ प्रथमं जगति किमप्येतदित्यालोचनं । नागृहीतविशेषण विशेष्ये धीः (।) विशेषणं जातिगुणक्रियादयः ।}त्रस्य जात्यादिविशिष्टनिश्चयफलं प्रति । {\color{DodgerBlue3}“अक्षसम्बन्धस्य”} इन्द्रियार्थसन्निकर्षस्यालोचनफलं प्रति {\color{DodgerBlue3}“विशेषण”}(स्य) वेद्यविशेष्यबुद्धिफलं प्रति {\color{DodgerBlue3}“प्रामाण्यं”} साधनत्वं {\color{DodgerBlue3}“नेष्टं । एतेषा”}मालोचनार्थसम्बन्धविशेषणज्ञानानां {\color{DodgerBlue3}“क्रिया”}यामधिगतिविषयसारूप्येण {\color{DodgerBlue3}“व्यवधानात्”} । (३१०)
	\pend
      \label{div_pvv.2.311}\edlabel{div_pvv.2.311}
	  
	% new div opening: depth here is 2
	

	  \pstart व्यवस्थानेपि साधकतमत्वं स्यादिति चेत् । आह ।
	\pend
      
	  \bigskip
	  \begingroup
	  \large
	
	    
	    \stanza[\smallbreak]
	\label{pv.2.311}\edlabel{pv.2.311}\flagstanza{\tiny\textenglish{....2.311}}सर्वेषामुपयोगेपि कारकाणां क्रियाम्प्रति ।&यदन्त्यं भेदकं तस्यास्तत् साधकतमं मतम् ॥ ३११ ॥\&[\smallbreak]


	
	  \endgroup
	\leavevmode\marginnote{\textenglish{212/s}}

	  \pstart {\color{DodgerBlue3}“सर्व्वेषां कारकाणां”} साक्षात्पारम्पर्येण {\color{DodgerBlue3}“क्रियां प्रत्युपयोगेपि”} तेषु मध्ये {\color{DodgerBlue3}“यत्कारकमन्त्यं”} कारकान्तरेणाव्यवहितव्याप्यारं सत् क्रिया{\color{DodgerBlue3}“भेदकं तत्तस्याः साधकतमं मतं”} नान्यत् । (३११)
	\pend
      \label{div_pvv.2.312}\edlabel{div_pvv.2.312}
	  
	% new div opening: depth here is 2
	

	  \pstart तथा हि (।)
	\pend
      
	  \bigskip
	  \begingroup
	  \large
	
	    
	    \stanza[\smallbreak]
	\label{pv.2.312}\edlabel{pv.2.312}\flagstanza{\tiny\textenglish{....2.312}}सर्व-सामान्यहेतुत्वादक्षाणामस्ति नेदृशम् ।&तद्भेदेपि ह्यतद्रूपस्यास्येदमिति तत्कुतः ॥ ३१२ ॥\&[\smallbreak]


	
	  \endgroup
	

	  \pstart {\color{DodgerBlue3}“अक्षाणां”} तावदी{\color{DodgerBlue3}“दृशं”} साधकतमत्वं {\color{DodgerBlue3}“नास्ति सर्व्वं---सामान्यहेतुत्वात्”} । \edlabel{pvv.212-1}\footnote{\label{pvv.212-1}  १ नीलादेः ।}सर्व्वज्ञानसाधारणहेतुत्वात् । अवान्तराधिगतिभेदकत्वानुपपत्तेः । तेषामिन्द्रियाणां प्रमादाविलत्वादि{\color{DodgerBlue3}“भेदेपि”} ज्ञानस्यातद्रूपस्य विषयसारूप्यरहितस्य इदमस्य ग्राहकमिति ग्राहकत्वं यदिष्यते {\color{DodgerBlue3}“तत्कुतः”} । (३१२)
	\pend
      \label{div_pvv.2.313}\edlabel{div_pvv.2.313}
	  
	% new div opening: depth here is 2
	
	  \bigskip
	  \begingroup
	  \large
	
	    
	    \stanza[\smallbreak]
	\label{pv.2.313}\edlabel{pv.2.313}\flagstanza{\tiny\textenglish{....2.313}}एतेन शेषं व्याख्यातं विशेषणधियां पुनः ।&अताद्रुप्ये न भेदोपि तद्वदन्यधियोपि वा ॥ ३१३ ॥\&[\smallbreak]


	
	  \endgroup
	

	  \pstart {\color{DodgerBlue3}“एते”}नाक्षाणामसाधनत्वदर्शनेन {\color{DodgerBlue3}“शेष”}मालोचनार्थसम्बन्धादि {\color{DodgerBlue3}“व्याख्यातं”} । तदप्यर्थसारूप्यरहितमसाधनं ।\edlabel{pvv.212-2}\footnote{\label{pvv.212-2}  २ अस्येदमित्यालोचनमित्यसिद्धेः ।} सारूप्ये च स्वीकर्तव्ये तदेवाव्यवहितत्वात् साधनमस्तु । {\color{DodgerBlue3}“विशेषणधियां पुनर”}यमधिको दोषः । {\color{DodgerBlue3}“अताद्रूप्ये”} विशेषणसारूप्याभावे विशेष्यधियः सकाशा{\color{DodgerBlue3}“द्भेदोपि”} न स्यात् । द्वयोरप्याकाररहितत्वेन भेदस्थित्यनुपपत्तेः । ततश्चैका प्रमाणमन्या च फलमिति कुतः । अथ सारूप्यं विशेषणधियो भेदकमिष्यते । एवं च सति तदेव प्रमाणं फलं च स्यात् । अर्थसरूपत्वादधिगतिरूपत्वाच्च । {\color{DodgerBlue3}“तद्वद्”} विशेषणबुद्धेरिव {\color{DodgerBlue3}“अन्य\edlabel{pvv.212-3}\footnote{\label{pvv.212-3}  ३ विशेष्यबुद्धेः ।}”} बुद्धेर्विषयसारूप्यं फलात्मकञ्चेष्यतां । (३१३)
	\pend
      \label{div_pvv.2.314}\edlabel{div_pvv.2.314}
	  
	% new div opening: depth here is 2
	

	  \pstart किञ्च (।)
	\pend
      
	  \bigskip
	  \begingroup
	  \large
	
	    
	    \stanza[\smallbreak]
	\label{pv.2.314}\edlabel{pv.2.314}\flagstanza{\tiny\textenglish{....2.314}}नेष्टो विषयभेदोपि क्रियासाधनयोर्द्वयोः ।&एकार्थत्वे द्वयं व्यर्थं न च स्यात् क्रमभाविता ॥ ३१४ ॥\&[\smallbreak]


	
	  \endgroup
	

	  \pstart {\color{DodgerBlue3}“क्रियासाधनयोर्विषयभेदोपि नेष्टः”} सर्व्वस्य । न ह्यन्य\edlabel{pvv.212-4}\footnote{\label{pvv.212-4}  ४ द्वे ज्ञाने विशेषणविशेष्यभिन्नविषये इति विशेषणे तदधिगमरूपतया व्यापृतं विशेष्ये ।}त्र परशुव्यापारश्छिदा चान्यत्र । इह तु विशेषणे प्रमाणव्यापारः क्रिया च विशेष्य इति भिन्नविषयता कथमिष्टा ।
	\pend
      \leavevmode\marginnote{\textenglish{213/s}}

	  \pstart अथैतद्दोषतया {\color{DodgerBlue3}“द्वयोः”} क्रियासाधनयोः क्रमभाविनो{\color{DodgerBlue3}“रेकार्थत्वे”} एकविषयत्वे च स्वीक्रियमाणे {\color{DodgerBlue3}“द्वयं”} भिन्नं प्रमाणं फलं च {\color{DodgerBlue3}“व्यर्थं”} विशेषणज्ञानं प्रमाणं फलं चास्तु विषयस्वरूपत्वादधिगमस्वभावाच्च । ततः परं तु विशेष्यज्ञानमधिगताधिगन्तृत्वादनुपयुक्तं । {\color{DodgerBlue3}“न चै”}कविषययोर्ज्ञानयोः {\color{DodgerBlue3}“क्रमभाविता”}स्ति विषयस्य तज्जननशक्तस्य क्रमेण स्वकार्यजननविरोधात् । (३१४)
	\pend
      \label{div_pvv.2.315}\edlabel{div_pvv.2.315}
	  
	% new div opening: depth here is 2
	

	  \pstart सकृदुत्पत्तावेव विशेषणविशेष्यधियोः प्रमाणफलता भविष्यतीति चेत् ।\leavevmode\marginnote{\textenglish{42a/MA}} आह (।)
	\pend
      
	  \bigskip
	  \begingroup
	  \large
	
	    
	    \stanza[\smallbreak]
	\label{pv.2.315a}\edlabel{pv.2.315a}\flagstanza{\tiny\textenglish{...2.315a}}साध्यसाधनताभावः सकृद्भावे;\&[\smallbreak]


	
	  \endgroup
	

	  \pstart {\color{DodgerBlue3}“सकृद् भावे साध्यसाधनतायाः अभावः”} । कार्यकारणभावविशेषत्वेन तस्या इ\edlabel{pvv.213-1}\footnote{\label{pvv.213-1}  १ प्रमाणादन्यफलवादिना ।}ष्टत्वात् ।
	\pend
      

	  \pstart नन्वेवमाकाराधिगमयोरेकज्ञानात्मत्वेपि प्रमाणफलताऽनुपपन्नेत्याह (।)
	\pend
      
	  \bigskip
	  \begingroup
	  \large
	
	    
	    \stanza[\smallbreak]
	\label{pv.2.315b}\edlabel{pv.2.315b}\flagstanza{\tiny\textenglish{...2.315b}}धियोंशयोः ।&तद्‏व्यवस्थाश्रयत्वेन साध्यसाधनसंस्थितिः ॥ ३१५ ॥\&[\smallbreak]


	
	  \endgroup
	

	  \pstart {\color{DodgerBlue3}“धियोऽशंयो”}राकाराधिगमलक्षणयोः {\color{DodgerBlue3}“साध्यसाधनसंस्थितिः”} क्रियाकरणव्यवस्था । {\color{DodgerBlue3}“तद्‏व्यवस्थाश्रयत्वे”}नाकारवशेनाधिगतिविशेषव्यवस्थानात् । नास्त्यत्र कार्यकारणात्मकः क्रियाकरणभावः किं तु व्यवस्थाप्यव्यवस्थापकभावः । स च तादात्म्येप्यविरुद्धः । (३१५)
	\pend
      \label{div_pvv.2.316}\edlabel{div_pvv.2.316}
	  
	% new div opening: depth here is 2
	

	  \pstart अर्थसन्निकर्षोपि न प्रमाणमित्याह (।)
	\pend
      
	  \bigskip
	  \begingroup
	  \large
	
	    
	    \stanza[\smallbreak]
	\label{pv.2.316}\edlabel{pv.2.316}\flagstanza{\tiny\textenglish{....2.316}}सर्वात्मनापि सम्बद्धं कैश्चिदेवावगम्यते ।&धर्मैः स नियमो न स्यात् सम्बन्धस्याविशेषतः ॥ ३१६ ॥\&[\smallbreak]


	
	  \endgroup
	

	  \pstart बाह्यं {\color{DodgerBlue3}“सर्व्वात्मना”} सर्व्वैराकारैरिन्द्रियादिभिः {\color{DodgerBlue3}“सम्बद्धमपि कैश्चिदेय धर्म्मै”}र्नीलत्वादिभि{\color{DodgerBlue3}“र्ग्गम्यते”} न त्पणुपुञ्जत्वादिभिः\edlabel{pvv.213-2}\footnote{\label{pvv.213-2}  २ विज्ञानमेव हि विषयप्रतिभासमुत्पद्यमानं ग्राह्यग्राहकभेदवदुपलक्ष्यते तेन बाह्योर्थोस्तीति नार्थाधिगतिरूपा काचित् क्रियास्ति, या प्रमाणफलत्वेन व्यवस्थाप्यत इति वृत्तिः । प्राप्यकारीन्द्रियपक्षे सर्व्वात्मना विषयस्पर्शात् सर्व्वथावसायः स्यात् सारूप्ये तु यावन्मात्रेण प्रतिबिम्बस्तावतो ग्रहान्न दोषः ।} । {\color{DodgerBlue3}“स एष”} ग्रहणस्य {\color{DodgerBlue3}“निय\edlabel{pvv.213-3}\footnote{\label{pvv.213-3}  ३ सन्निकर्षप्रमाणत्वे ।} मो न स्यात् । सम्ब”}\edlabel{pvv.213-4}\footnote{\label{pvv.213-4}  ४ सन्निकर्षस्य ।} {\color{DodgerBlue3}“न्ध”}स्यापि गतिस्थितिहेतोर{\color{DodgerBlue3}“विशेषतः”} । (३१६)
	\pend
      \label{div_pvv.2.317}\edlabel{div_pvv.2.317}
	  
	% new div opening: depth here is 2
	\leavevmode\marginnote{\textenglish{214/s}}
	  \bigskip
	  \begingroup
	  \large
	
	    
	    \stanza[\smallbreak]
	\label{pv.2.317}\edlabel{pv.2.317}\flagstanza{\tiny\textenglish{....2.317}}तदभेदेपि भेदोयं यस्मात्तस्य प्रमाणता ।&संस्काराच्चेदताद्रूप्ये न तस्याप्यव्यवस्थितेः ॥ ३१७ ॥\&[\smallbreak]


	
	  \endgroup
	

	  \pstart {\color{DodgerBlue3}“यस्मा”}दिन्द्रियसन्निकर्षादेः प्रामाण्यमयुक्तं {\color{DodgerBlue3}“तत्त”}स्माद{\color{DodgerBlue3}“भेदेपि”} सन्निकर्षाद्यविशेषेपि यस्मान्नियामकात् तज्ज्ञानस्या{\color{DodgerBlue3}“यं भेदो”} नीलस्येदं ज्ञानं पीतस्य चेदमित्यादि {\color{DodgerBlue3}“तस्य प्रमाण”}ता युक्ता । स च आकार एवेति स एव प्रमाणं ज्ञानजातज्ञानहेतोः {\color{DodgerBlue3}“संस्कारा”}ज्ज्ञानस्य कैश्चिदेव धर्मैर्ग्रहणनियम इति चेत् {\color{DodgerBlue3}“न”} (।) {\color{DodgerBlue3}“तस्य”} संस्कारस्याप्य{\color{DodgerBlue3}“ताद्रूप्ये”} विषयासरूपत्वे{\color{DodgerBlue3}“ऽव्यवस्थितेः”} । (३१७)
	\pend
      \label{div_pvv.2.318_2.319}\edlabel{div_pvv.2.318_2.319}
	  
	% new div opening: depth here is 2
	

	  \pstart यदाकारं ज्ञानं स्यात्तस्यानुभवः संस्कारश्च भवेत् । तस्माच्च ग्रहणप्रतिनियमः । अनुभवस्थित्यभावे तु सर्व्वमव्यवस्थितं ।
	\pend
      
	  \bigskip
	  \begingroup
	  \large
	
	    
	    \stanza[\smallbreak]
	\label{pv.2.318a}\edlabel{pv.2.318a}\flagstanza{\tiny\textenglish{...2.318a}}क्रियाकरणयोरैक्यविरोध इति चेदसत् ।&धर्मभेदाभ्युपगमाद्;\&[\smallbreak]


	
	  \endgroup
	

	  \pstart {\color{DodgerBlue3}“क्रियाकरणयो”}रधिगमाकारयोरैकात्म्ये विरोध {\color{DodgerBlue3}“इति चेत् । असदे”}तत् । {\color{DodgerBlue3}“धर्मभेद”}स्य व्यावृत्त्युपकल्पितस्या{\color{DodgerBlue3}“भ्युपगमात्”} । अनाकारव्यावृत्तिः प्रमाणं । अनधिगतिव्यावृत्तिश्च फलमिति नानयोरैक्यं ।
	\pend
      

	  \pstart कथन्तर्ह्युक्तं (।) “
	    \pend
	  
	    
	    \stanza[\smallbreak]
	सा च तस्यात्मभूतैव तेन नार्थान्तरं फलमि\&[\smallbreak]


	
	    \pstart
	  ” \href{http://http://sarit.indology.info/?cref=pv.2.307}{(२।३०७)} त्याह (।)
	\pend
      
	  \bigskip
	  \begingroup
	  \large
	
	    
	    \stanza[\smallbreak]
	\label{pv.2.318b}\edlabel{pv.2.318b}\flagstanza{\tiny\textenglish{...2.318b}}वस्त्वभिन्नमितीष्यते ॥ ३१८ ॥\&[\smallbreak]


	
	  \endgroup
	

	  \pstart परमार्थतो ज्ञानात्मकं {\color{DodgerBlue3}“वस्त्वभिन्नमितीष्यते”} । (३१८) न तु कल्पितधर्मद्वारेणपि । किं पुनस्तत्त्वत एव साधनाद् भिन्नाऽधिगतिश्छि दादिवन्नेष्यत इत्याह (।)
	\pend
      
	  \bigskip
	  \begingroup
	  \large
	
	    
	    \stanza[\smallbreak]
	\label{pv.2.319}\edlabel{pv.2.319}\flagstanza{\tiny\textenglish{....2.319}}एवं प्रकारा सर्व्वैव क्रियाकारकसंस्थितिः ।&भावस्य भिन्नाभिमतेष्वप्यारोपेण वृत्तितः ॥ ३१९ ॥\&[\smallbreak]


	
	  \endgroup
	

	  \pstart {\color{DodgerBlue3}“सर्वैव क्रियाकारक”}योः {\color{DodgerBlue3}“संस्थिति”}र्व्यवस्था {\color{DodgerBlue3}“एवं-प्रकारा”} कल्पितैव {\color{DodgerBlue3}“भिन्नाभिमतेष्वपि”} दारुपरश्वादिषु क्रियाकरणभावस्या{\color{DodgerBlue3}“रोपेण वृत्तितः”} । न हि तत्रापि कार्यकारणवस्तुद्वयव्यतिरिक्ता क्रियास्ति । द्विधाभूतं काष्ठमेवातद्‏व्यावृत्त्या भेदान्तरप्रतिक्षेपेण छिदेत्युच्यते । तत्कारणेषु च पुरुषकरपरश्वादिषु सामग्र्‏यन्तरवर्त्तिनः कराच्छिदायाः अनुत्पत्तेः परशोरसाधारणं सहकारित्वमुपदर्शयितुमतद्‏व्यावृत्त्या करणव्यपदेशः । न तु क्रियाकरणत्वमन्यदेव कार्यकरणाभ्यां । अधिगमाकारयोस्तु व्यवस्थाप्यव्यवस्थापकभावः क्रियाकरणभावः (।) यथा यः कम्पते सोऽश्व इति (।) उक्ता प्रमाणफलव्यवस्था ॥ X X ॥ (३१९)
	\pend
      
	  
	% new div opening: depth here is 1
	
\section[{९. विक्षप्तिमात्रताचिन्ता}]{९. विक्षप्तिमात्रताचिन्ता}

	  \begin{center}%% label @type='head'
	\textbf{(१) अर्थसंवेदनचिन्ता}
	\end{center}
	

	  \begin{center}%% label @type='head'
	\textbf{क. अर्थसंविद्}
	\end{center}
	\label{div_pvv.2.320}\edlabel{div_pvv.2.320}
	  
	% new div opening: depth here is 2
	

	  \pstart \leavevmode\marginnote{\textenglish{215/s}}इदानीं यो गा चा रो वेद्यवेदकभावमपश्यन् सौ त्रा न्ति कं पृ\edlabel{pvv.215-1}\footnote{\label{pvv.215-1}  १ अन्तर्ब्बहिस्तीर्थ्या दूषिता(ः।) प्रमाद्वयं प्रमेयद्वैविध्यादुक्तं प्रत्यक्षलक्षणञ्च । सौत्रान्तिकप्रमाणं सारूप्यं बाह्योर्थः प्रमेयोधिगतिः फलं व्यवस्थाप्याधुना विज्ञप्तौ प्रमाणफलव्यवस्थां निर्दिदिक्षुः “स्वसम्वित्तिः फलञ्चात्र तद्रूपो ह्यर्थनिश्चय” इति सम्वित्तिं व्याख्यातुमाह सौत्रान्तिकं ।} च्छति ।
	\pend
      
	  \bigskip
	  \begingroup
	  \large
	
	    
	    \stanza[\smallbreak]
	\label{pv.2.320a}\edlabel{pv.2.320a}\flagstanza{\tiny\textenglish{...2.320a}}कार्थसंविद्;\&[\smallbreak]


	
	  \endgroup
	

	  \pstart {\color{DodgerBlue3}“काऽर्थस्य संवित्”} ज्ञानमुच्यते ।
	\pend
      

	  \pstart अत आह (।)
	\pend
      
	  \bigskip
	  \begingroup
	  \large
	
	    
	    \stanza[\smallbreak]
	\label{pv.2.320b}\edlabel{pv.2.320b}\flagstanza{\tiny\textenglish{...2.320b}}यदेवेदं प्रत्यक्षं प्रतिवेदनम् ।\&[\smallbreak]


	
	  \endgroup
	

	  \pstart {\color{DodgerBlue3}“यदेवेदं प्रत्यक्ष”}मनुभवसिद्धं {\color{DodgerBlue3}“प्रतिवेदनं”} नीलाद्याकारेण प्रतिनियतं वेदनं प्रतिसन्ताननियतं वा सैवार्थसंविदुच्यते ।
	\pend
      
	  \bigskip
	  \begingroup
	  \large
	
	    
	    \stanza[\smallbreak]
	\label{pv.2.320c}\edlabel{pv.2.320c}\flagstanza{\tiny\textenglish{...2.320c}}तदर्थवेदनं केन ताद्रूप्याद् व्यभिचारि तत् ॥ ३२० ॥\&[\smallbreak]


	
	  \endgroup
	

	  \pstart ननु {\color{DodgerBlue3}“तत्”} प्रतिनियतं {\color{DodgerBlue3}“वेदन”}मनुभूयमानमर्थंस्य वेदनं {\color{DodgerBlue3}“केन”} हेतुनोच्यते स्वप्रकाशात्मकत्वात् स्ववेदनमेव तद्युक्तं नार्थवेदनं । तस्य सर्व्वदा परोक्षत्वात् । {\color{DodgerBlue3}“ताद्रूप्या”}दर्थसरूपत्वात् ज्ञानमर्थवेदनमिति चेत् । तदर्थसारूप्यं{\color{DodgerBlue3}“व्यभि\edlabel{pvv.215-2}\footnote{\label{pvv.215-2}  २ अतिव्याप्त्या ।} चारि”}, द्विचन्द्रकेशोण्डूकज्ञानाद्याकारस्यार्थमन्तरेणापि भावात् । (३२०)
	\pend
      \label{div_pvv.2.321}\edlabel{div_pvv.2.321}
	  
	% new div opening: depth here is 2
	
	  \bigskip
	  \begingroup
	  \large
	
	    
	    \stanza[\smallbreak]
	\label{pv.2.321a}\edlabel{pv.2.321a}\flagstanza{\tiny\textenglish{...2.321a}}अथ सोनुभवः क्वास्य\&[\smallbreak]


	
	  \endgroup
	

	  \pstart अथानुभाव्याभावे {\color{DodgerBlue3}“सोऽनुभवः”} प्रत्यात्मवेद्यो{\color{DodgerBlue3}“ऽस्य”} ज्ञानस्य न कर्म्मणि व्यवस्थापनीयः । न हि कर्मरहिता क्रिया {\color{DodgerBlue3}“क्व”}चिदस्ति ।
	\pend
      

	  \pstart अत्राह (।)
	\pend
      
	  \bigskip
	  \begingroup
	  \large
	
	    
	    \stanza[\smallbreak]
	\label{pv.2.321b}\edlabel{pv.2.321b}\flagstanza{\tiny\textenglish{...2.321b}}तदेवेदं विचार्यते ।\&[\smallbreak]


	
	  \endgroup
	

	  \pstart यदुच्यते व्यवहर्तृ भिरिदमनेनानुभूयते इति {\color{DodgerBlue3}“तदेवेद”}मस्माभि{\color{DodgerBlue3}“र्व्विचार्यते”} । ज्ञानं स्वप्रकाशमुपलभ्यते प्रकाशश्चोच्यत इत्ययुक्तं ।
	\pend
      \leavevmode\marginnote{\textenglish{42b/MA}}\leavevmode\marginnote{\textenglish{216/s}}

	  \pstart यच्चार्थसारूप्यमनुभवनिबन्धनमुक्तं तदप्यसम्भवीति दर्शयन्नाह (।)
	\pend
      
	  \bigskip
	  \begingroup
	  \large
	
	    
	    \stanza[\smallbreak]
	\label{pv.2.321c}\edlabel{pv.2.321c}\flagstanza{\tiny\textenglish{...2.321c}}सरूपयन्ति तत् केन स्थूलाभासञ्च तेऽणवः ॥ ३२१ ॥\&[\smallbreak]


	
	  \endgroup
	

	  \pstart {\color{DodgerBlue3}“ते”} परस्परं भिन्ना {\color{DodgerBlue3}“अणवः”} तज्ज्ञानं {\color{DodgerBlue3}“स्थूलाभासं”} स्थूलाकारं केन रूपेण स{\color{DodgerBlue3}“रूपयन्ति\edlabel{pvv.216-1}\footnote{\label{pvv.216-1}  १ अवयवी सौन्त्रान्तिकेनैव निरस्तः ।}”} । यदणुस्वरूपमस्थूलमस्ति न तत् ज्ञानारूढं । यच्च ज्ञानारूढं स्थौल्यं नाणुषु तदस्ति । (३२१)
	\pend
      \label{div_pvv.2.322}\edlabel{div_pvv.2.322}
	  
	% new div opening: depth here is 2
	
	  \bigskip
	  \begingroup
	  \large
	
	    
	    \stanza[\smallbreak]
	\label{pv.2.322}\edlabel{pv.2.322}\flagstanza{\tiny\textenglish{....2.322}}तन्नार्थरूपता तस्य सत्यांर्थाव्यभिचारिणी ।&तत्सम्वेदनभावस्य न समर्था प्रसाधने ॥ ३२२ ॥\&[\smallbreak]


	
	  \endgroup
	

	  \pstart तस्मात्तुल्यज्ञानस्य {\color{DodgerBlue3}“नार्थरूपता”}ऽस्ति । सत्यां वार्थरूपतायां व्यभिचारिणी सा द्विचन्द्रज्ञानादिषु । ततश्च {\color{DodgerBlue3}“तत्संवेदनभावस्या”}र्थसंवेदनत्वस्य {\color{DodgerBlue3}“प्रसाधने”}षु साऽर्थरूपता न {\color{DodgerBlue3}“समर्था । न”} केवलादर्थसारूप्यादर्थसंवेदनत्वं येन व्यभिचारः स्यात्(।) किन्तर्हि सारूप्यतदुत्पत्तिभ्यां(।) ते च द्विचन्द्रज्ञानादीनां न स्तः । चन्द्रद्वयस्याभावात् तदुत्पत्तेरयोगात् । (३२२)
	\pend
      \label{div_pvv.2.323}\edlabel{div_pvv.2.323}
	  
	% new div opening: depth here is 2
	

	  \pstart एतदेवाह (।)
	\pend
      
	  \bigskip
	  \begingroup
	  \large
	
	    
	    \stanza[\smallbreak]
	\label{pv.2.323}\edlabel{pv.2.323}\flagstanza{\tiny\textenglish{....2.323}}तत्सारूप्यतदुत्पत्ती यदि सम्वेद्यलक्षणम् ।&सम्वेद्यं स्यात् समानार्थं विज्ञानं समनन्तरम् ॥ ३२३ ॥\&[\smallbreak]


	
	  \endgroup
	

	  \pstart {\color{DodgerBlue3}“तेन”} ग्राह्येण {\color{DodgerBlue3}“सारूप्यं तस्मादुत्पत्तिः”} स्व{\color{DodgerBlue3}“सम्वेद्य”}स्य {\color{DodgerBlue3}“लक्षणं”} यदि संमतं तदापि {\color{DodgerBlue3}“समन\edlabel{pvv.216-2}\footnote{\label{pvv.216-2}  २ पूर्व्वानुभूतस्मरणं ।}न्तरं ज्ञान”}मुत्तरज्ञानेन {\color{DodgerBlue3}“समानार्थं”} समानग्राह्यं {\color{DodgerBlue3}“संवेद्यं स्यात्”} । तत्सरूपतदुत्पत्त्योः संभवात् । (३२३)
	\pend
      \label{div_pvv.2.324}\edlabel{div_pvv.2.324}
	  
	% new div opening: depth here is 2
	

	  \begin{center}%% label @type='head'
	\textbf{ख. दृश्यदर्शने प्रत्यासत्तिविचारः}
	\end{center}
	

	  \pstart स्यादेतत् (।)
	\pend
      
	  \bigskip
	  \begingroup
	  \large
	
	    
	    \stanza[\smallbreak]
	\label{pv.2.324a}\edlabel{pv.2.324a}\flagstanza{\tiny\textenglish{...2.324a}}इदं दृष्टं श्रुतम्वेदमिति यत्रावसायधीः ।&स तस्यानुभवः;\&[\smallbreak]


	
	  \endgroup
	

	  \pstart सारूप्यतदुत्पत्तिमत्त्वेपीदं {\color{DodgerBlue3}“दृष्टं श्रुतम्वे\edlabel{pvv.216-3}\footnote{\label{pvv.216-3}  ३ स्मार्त्तेनेदं प्रयोगः ।}दमिति यत्रावसायधी”}रुत्पद्यते {\color{DodgerBlue3}“तस्य सोऽनुभवो”} नान्यस्य । न च समनन्तरप्रत्यये दृष्टश्रुताद्यवसायो भवति तन्न ग्राह्योऽसौ ।
	\pend
      

	  \pstart अत्राह (।)
	\pend
      
	  \bigskip
	  \begingroup
	  \large
	
	    
	    \stanza[\smallbreak]
	\label{pv.2.324b}\edlabel{pv.2.324b}\flagstanza{\tiny\textenglish{...2.324b}}सैव प्रत्यासत्तिर्व्विचार्यते ॥ ३२४ ॥\&[\smallbreak]


	
	  \endgroup
	\leavevmode\marginnote{\textenglish{217/s}}

	  \pstart दृश्यदर्शनयोः {\color{DodgerBlue3}“सैव”} प्रत्यासत्ति{\color{DodgerBlue3}“र्व्विचार्यते”}ऽस्मा\edlabel{pvv.217-1}\footnote{\label{pvv.217-1}  १ ज्ञानज्ञेययोः पृथगिष्टे न तादात्म्यं तद्रूपेर्थेऽसत्यपि तद्रूपज्ञानात् न तदुत्पत्तिः विषयव्यवस्थाश्रय इदं परामर्शः परामर्शाच्च तद्विषयव्यवस्था ।}भिः । (३२४)
	\pend
      \label{div_pvv.2.325}\edlabel{div_pvv.2.325}
	  
	% new div opening: depth here is 2
	
	  \bigskip
	  \begingroup
	  \large
	
	    
	    \stanza[\smallbreak]
	\label{pv.2.325}\edlabel{pv.2.325}\flagstanza{\tiny\textenglish{....2.325}}दृश्यदर्शनयोर्येन तस्य तद् दर्शनम्मतम् ।&तयोः सम्बन्धमाश्रित्य द्रष्टुरेष विनिश्चयः ॥ ३२५ ॥\&[\smallbreak]


	
	  \endgroup
	

	  \pstart {\color{DodgerBlue3}“येन”} प्रत्यासत्तिसम्भवेन {\color{DodgerBlue3}“तस्य”} बाह्यस्य तज्ज्ञानं {\color{DodgerBlue3}“दर्शनं मतं । तयोर्दृ श्यदर्शनयोः सम्बन्धं”} प्रत्यासत्तिमा{\color{DodgerBlue3}“श्रित्य द्रष्टुः”} पुरुष{\color{DodgerBlue3}“स्येदं दृष्टं श्रुतं वेदमित्यर्थनिश्चयः”} । तत्सारूप्यतदुत्पत्तिलक्षणा च प्रत्यासत्तिः समनन्तरप्रत्ययेपि समानेति तन्निबन्धनो दृष्टश्रुताध्यवसाय(ो) पि तत्र स्यादित्यर्थः । (३२५)
	\pend
      \label{div_pvv.2.326}\edlabel{div_pvv.2.326}
	  
	% new div opening: depth here is 2
	
	  \bigskip
	  \begingroup
	  \large
	
	    
	    \stanza[\smallbreak]
	\label{pv.2.326}\edlabel{pv.2.326}\flagstanza{\tiny\textenglish{....2.326}}आत्मा स तस्यानुभवः स च नान्यस्य कस्यचित् ।&प्रत्यक्षप्रतिवेद्यत्वमपि तस्य तदात्मता ॥ ३२६ ॥\&[\smallbreak]


	
	  \endgroup
	

	  \pstart तस्माद्वेद्यरहितस्तस्य ज्ञानस्य स नीलादिरूप आत्मा{\color{DodgerBlue3}“नुभवः (।) स चानुभवो नान्यस्य कस्यचिद्”} बाह्यस्य । {\color{DodgerBlue3}“तस्य”} ज्ञानस्य {\color{DodgerBlue3}“प्रत्यक्षप्रतिवेद्यत्वमपि”} यदुच्यते । सा तदात्मताऽपरोक्षानुभवात्मता । (३२६)
	\pend
      \label{div_pvv.2.327}\edlabel{div_pvv.2.327}
	  
	% new div opening: depth here is 2
	
	  \bigskip
	  \begingroup
	  \large
	
	    
	    \stanza[\smallbreak]
	\label{pv.2.327}\edlabel{pv.2.327}\flagstanza{\tiny\textenglish{....2.327}}नान्योनुभाव्यस्तेनास्ति तस्य नानुभवोपरः ।&तस्यापि तुल्यचोद्यत्वात् स्वयं सैव प्रकाशते ॥ ३२७ ॥\&[\smallbreak]


	
	  \endgroup
	

	  \pstart यथा च स्वरू\edlabel{pvv.217-2}\footnote{\label{pvv.217-2}  २ ग्राह्यग्राहकत्वायोगात् ।}पाद{\color{DodgerBlue3}“न्यो”} बुद्ध्या {\color{DodgerBlue3}“अनुभाव्यो नास्ति”} । तथा {\color{DodgerBlue3}“तस्य”} ज्ञानस्य चापरोऽ{\color{DodgerBlue3}“नुभवो नास्ति”} । तस्य \edlabel{pvv.217-3}\footnote{\label{pvv.217-3}  ३ ग्राह्यत्वस्य ।}ज्ञानग्रहण\edlabel{pvv.217-4}\footnote{\label{pvv.217-4}  ४ यत् स्वसम्विद्रूपं तन्निरालम्बनं यथा स्वप्नज्ञानं स्वविद्रूपञ्च जागरेपीति तन्मात्रानुबन्धित्वा (त्) स्वभावहेतुरुक्त (ः।) एतेन अप्रत्यक्षोपलम्भस्य नार्थदृष्टिः प्रसिध्यतीति नासिद्धत्वं ।} स्यापि {\color{DodgerBlue3}“तुल्यार्थचोद्यत्वात्”} । स ह्यन्यत्वनिबन्धनो ग्राह्यग्राहकभावः । तच्चानुपपन्नमित्युक्तं । तत्तस्मात्तत् ज्ञानमपरोक्षतयोत्पन्नं {\color{DodgerBlue3}“स्वयं प्रकाशते”} । नान्येन प्रकाश्यते । (३२७)
	\pend
      \label{div_pvv.2.328}\edlabel{div_pvv.2.328}
	  
	% new div opening: depth here is 2
	

	  \begin{center}%% label @type='head'
	\textbf{ग. नीलाद्यनुभवप्रसिद्धिः}
	\end{center}
	

	  \pstart कथं तर्हि नीलाद्यनुभवप्रसिद्धिरित्याह (।)
	\pend
      
	  \bigskip
	  \begingroup
	  \large
	
	    
	    \stanza[\smallbreak]
	\label{pv.2.328}\edlabel{pv.2.328}\flagstanza{\tiny\textenglish{....2.328}}नीलादिरूपस्तस्यासौ स्वभावोनुभवश्च सः ॥&नीलाद्यनुभवात् ख्यातः स्वरूपानुभवोपि सन् ॥ ३२८ ॥\&[\smallbreak]


	
	  \endgroup
	\leavevmode\marginnote{\textenglish{218/s}}

	  \pstart तस्य ज्ञानस्य नीला\edlabel{pvv.218-1}\footnote{\label{pvv.218-1}  १ कल्पितः कर्मकर्त्रादीति वाच्यं ।}दिरूपोऽसौ स्वभावोऽनुभवः प्रकाशात्मकश्च सः । तेन स्वरूपानुभवोपि सन्नीलाद्यनुभवोत् तथा संप्रसिद्धिः । (३२८)
	\pend
      \label{div_pvv.2.329}\edlabel{div_pvv.2.329}
	  
	% new div opening: depth here is 2
	
	  \bigskip
	  \begingroup
	  \large
	
	    
	    \stanza[\smallbreak]
	\label{pv.2.329}\edlabel{pv.2.329}\flagstanza{\tiny\textenglish{....2.329}}प्रकाशमानस्तादात्म्यात् स्वरूपस्य प्रकाशकः ।&यथा प्रकाशोभिमतस्तथा धीरात्मवेदिनी ॥ ३२९ ॥\&[\smallbreak]


	
	  \endgroup
	

	  \pstart {\color{DodgerBlue3}“यथा प्रकाशस्तादात्म्यात्”} प्रकाशात्मकत्वात् परनिरपेक्षः {\color{DodgerBlue3}“प्रकाशमानः स्वरूपस्य प्रकाशकोऽभिमतः । तथा धीः”} परनिरपेक्षा प्रकाशात्मनोत्पन्ना प्रकाशमानाऽ{\color{DodgerBlue3}“त्मवेदिनीति”} उपचारादुच्यते । (३२९)
	\pend
      \label{div_pvv.2.330}\edlabel{div_pvv.2.330}
	  
	% new div opening: depth here is 2
	
	  \bigskip
	  \begingroup
	  \large
	
	    
	    \stanza[\smallbreak]
	\label{pv.2.330}\edlabel{pv.2.330}\flagstanza{\tiny\textenglish{....2.330}}तस्याश्चार्थान्तरे वेद्ये दुर्घटौ वेद्यवेदकौ ।&अवेद्यवेदकाकारा; यथा भ्रान्तैर्निरीक्ष्यते ॥ ३३० ॥\&[\smallbreak]


	
	  \endgroup
	
	  \bigskip
	  \begingroup
	  \large
	
	    
	    \stanza[\smallbreak]
	\label{pv.2.331a}\edlabel{pv.2.331a}\flagstanza{\tiny\textenglish{...2.331a}}विभक्तलक्षणग्राह्यग्राहकाकारविप्लवा ।\&[\smallbreak]


	
	  \endgroup
	

	  \pstart चकारो हेतौ । यस्मा{\color{DodgerBlue3}“त्तस्या”} धियोऽ{\color{DodgerBlue3}“र्थान्तरे वेद्ये वेद्यवे (द) काकारौ दुर्घटौ”} (।) तस्माद् वस्तुतोऽ{\color{DodgerBlue3}“वेद्यवेदकाकारा”} सा बुद्धिर्नीलप्रकाशात्मनोत्पन्ना तथा प्रकाशते । न तु तत्र कश्चिद् ग्राह्यस्य ग्राहकस्य चाकारः समस्ति । कथं तर्हि ग्राह्यग्राहकप्रतिभासव्यवसायावित्याह (।) {\color{DodgerBlue3}“भ्रान्तै”}रप्रहीणद्वयवासनाविप्लवै{\color{DodgerBlue3}“र्यथा विभक्तलक्षणौ ग्राह्यग्राहकाकारावेव विप्लवौ”} यस्याः सा तादृशी {\color{DodgerBlue3}“निरीक्ष्यते विभाव्यते भ्रान्तदर्शनानुरोधेन”} (। ३३०)
	\pend
      \label{div_pvv.2.331}\edlabel{div_pvv.2.331}
	  
	% new div opening: depth here is 2
	

	  \begin{center}%% label @type='head'
	\textbf{घ. ग्राह्यग्राहकप्रतिभासव्यवसाय:}
	\end{center}
	
	  \bigskip
	  \begingroup
	  \large
	
	    
	    \stanza[\smallbreak]
	\label{pv.2.331b}\edlabel{pv.2.331b}\flagstanza{\tiny\textenglish{...2.331b}}तथा कृतव्यवस्थेयं केशादिज्ञानभेदवत् ॥ ३३१ ॥\&[\smallbreak]


	
	  \endgroup
	

	  \pstart {\color{DodgerBlue3}“तथा”} ग्राह्यग्राहकभेदेन {\color{DodgerBlue3}“कृतव्यवस्था”} स नीलादिबहिर्देशं ग्राह्यमान्तरञ्च संवेदनं ग्राहकमिति विप्लव एष {\color{DodgerBlue3}“केशादिज्ञानभेदवत्”} । न हि केशोण्डूकज्ञानविशेषस्य ग्राहकवद् ग्राह्यः केशावयवोस्ति (।) किं तर्हि केशाभासः प्रकाश एव \leavevmode\marginnote{\textenglish{43a/MA}} केवलः । (३३१)
	\pend
      \label{div_pvv.2.332}\edlabel{div_pvv.2.332}
	  
	% new div opening: depth here is 2
	
	  \bigskip
	  \begingroup
	  \large
	
	    
	    \stanza[\smallbreak]
	\label{pv.2.332}\edlabel{pv.2.332}\flagstanza{\tiny\textenglish{....2.332}}यदा तदा न संचोद्यग्राह्यग्राहकलक्षणी ।&तदान्यसम्विदोभावात् स्वसम्वित् फलमिष्यते ॥ ३३२ ॥\&[\smallbreak]


	
	  \endgroup
	

	  \pstart विप्लववशाच्च ग्राह्यग्राहकभेदेन बुद्धि{\color{DodgerBlue3}“र्यदा”} व्यवस्थाप्यते {\color{DodgerBlue3}“तदा न सं\edlabel{pvv.218-2}\footnote{\label{pvv.218-2}  २ न संचोद्ये ग्राह्यग्राहकलक्षणे यस्याः ।} चोद्य-”} \leavevmode\marginnote{\textenglish{219/s}} {\color{DodgerBlue3}“ग्राह्यग्राहकलक्षणा सा”} । न हि वस्तुतो ग्राह्यग्राहकभावः सम्भवति । न च विप्लववशाद्वस्तुव्यवस्था केशद्विचन्द्रादेरपि तत्त्वप्रसङ्गात्(।) यदा च न ग्राह्यग्राहकता विज्ञप्तिमात्रतायां {\color{DodgerBlue3}“तदान्य”}स्य ग्राह्यस्य {\color{DodgerBlue3}“संविदो”} ज्ञानस्या{\color{DodgerBlue3}“भावात् स्वसंवित् फलमिष्यते”} । (३३२)
	\pend
      \label{div_pvv.2.333}\edlabel{div_pvv.2.333}
	  
	% new div opening: depth here is 2
	

	  \begin{center}%% label @type='head'
	\textbf{ङ. बाह्यार्थनिरासः}
	\end{center}
	
	  \bigskip
	  \begingroup
	  \large
	
	    
	    \stanza[\smallbreak]
	\label{pv.2.333}\edlabel{pv.2.333}\flagstanza{\tiny\textenglish{....2.333}}यदि बाह्योनुभूयेत को दोषो नैव कश्चन ।&इदमेव किमुक्तं स्यात् स बाह्योर्थोनुभूयते ॥ ३३३ ॥\&[\smallbreak]


	
	  \endgroup
	

	  \pstart ननु {\color{DodgerBlue3}“यदि बाह्योऽर्थो”} ज्ञानेना{\color{DodgerBlue3}“नुभूयते”} तदा {\color{DodgerBlue3}“को दोषः”} । येन स्वसंवित् फलमिष्यते ।
	\pend
      

	  \pstart आह (।) यद्यनुभूयते तदा {\color{DodgerBlue3}“नैव कश्चन”} दोषः । अनुभव एव तु बाह्यस्य नास्तीत्युच्यते । तथा {\color{DodgerBlue3}“इदमेव किमुक्तं स्या\edlabel{pvv.219-1}\footnote{\label{pvv.219-1}  १ रिक्तं पश्यन् पृच्छति परं ।}त् “बाह्योऽर्थो” ज्ञानेना{\color{DodgerBlue3}“नुभूयत”}”} इति\edlabel{pvv.219-2}\footnote{\label{pvv.219-2}  २ यदि निराकारा बुद्धिर्व्विषयोऽर्थान्तरं तदा सम्बन्धासिद्धेः ।} (३३३ ।)
	\pend
      \label{div_pvv.2.334}\edlabel{div_pvv.2.334}
	  
	% new div opening: depth here is 2
	
	  \bigskip
	  \begingroup
	  \large
	
	    
	    \stanza[\smallbreak]
	\label{pv.2.334}\edlabel{pv.2.334}\flagstanza{\tiny\textenglish{....2.334}}यदि बुद्धिस्तदाकारा सास्त्याकारविशेषिणी ।&सा बाह्यादन्यतो वेति विचारमिदमर्हति ॥ ३३४ ॥\&[\smallbreak]


	
	  \endgroup
	

	  \pstart {\color{DodgerBlue3}“यदि बुद्धिस्तदाकारा”} वा बाह्यसरूपेत्युच्यते । सत्यमस्ति {\color{DodgerBlue3}“सा बुद्धिराकारविशेषिणी”} नीलानीलाद्याकारविशेषयुक्ता । किंतु सा बुद्धि{\color{DodgerBlue3}“र्ब्बाह्याद”}र्थाज्जायेता{\color{DodgerBlue3}“न्यतो”} वासनाप्रतिनियमा{\color{DodgerBlue3}“द्वा इति विचारमिदमर्हति”} । न तावद् बुद्धिव्यतिरेकिणाऽर्थः कश्चिद्धेतुतयोपलभ्यते । बुद्धिस्वरूपमात्रवेदनात् । कादाचित्कतया तु कारणं तस्याः किञ्चिद् व्यवस्थापनीयं तच्च बाह्यं वासना वा स्यात् उभयथाप्युपपत्तेः । (३३४)
	\pend
      \label{div_pvv.2.335}\edlabel{div_pvv.2.335}
	  
	% new div opening: depth here is 2
	
	  \bigskip
	  \begingroup
	  \large
	
	    
	    \stanza[\smallbreak]
	\label{pv.2.335a}\edlabel{pv.2.335a}\flagstanza{\tiny\textenglish{...2.335a}}दर्शनोपाधिरहितस्याग्रहात्तद्‏ग्रहे ग्रहाद् ।&दर्शनं नीलनिर्भासं;\&[\smallbreak]


	
	  \endgroup
	

	  \pstart तत्र {\color{DodgerBlue3}“दर्शनेन”} ज्ञानेनो{\color{DodgerBlue3}“पाधि”}ना विशेषणेन {\color{DodgerBlue3}“रहितस्य”} नीलादेर{\color{DodgerBlue3}“ग्रहात्”}\edlabel{pvv.219-3}\footnote{\label{pvv.219-3}  ३ तथा नीलाकारस्याग्रहे ज्ञानाग्रहात् ।} तस्य ग्रहे च नीलस्य {\color{DodgerBlue3}“ग्र\edlabel{pvv.219-4}\footnote{\label{pvv.219-4}  ४ नीलग्रहे दर्शनाच्च ।}हात्”} सहैव नीलधियोर्व्वेदनात् {\color{DodgerBlue3}“द\edlabel{pvv.219-5}\footnote{\label{pvv.219-5}  ५ न हि ज्ञाने द्वावनुभवौ किन्तु स्वोपलम्भ एव ज्ञानस्यापि ।}र्शनं नीलादिनिर्भासं”} नीलाकारं \leavevmode\marginnote{\textenglish{220/s}} {\color{DodgerBlue3}“व्यवस्थितं”} । यत्तावत् नीलादिकं बाह्यमित्युच्यते । तज्ज्ञानेन सहोपलम्भनियमात् तदभिन्नस्वभावं द्विच\edlabel{pvv.220-1}\footnote{\label{pvv.220-1}  १ भ्रान्त्या ।} न्द्रादिवत् ।
	\pend
      

	  \pstart कस्तर्हि नास्तीत्याह (।)
	\pend
      
	  \bigskip
	  \begingroup
	  \large
	
	    
	    \stanza[\smallbreak]
	\label{pv.2.335b}\edlabel{pv.2.335b}\flagstanza{\tiny\textenglish{...2.335b}}नार्थो बाह्योस्ति केवलम् ॥ ३३५ ॥\&[\smallbreak]


	
	  \endgroup
	

	  \pstart {\color{DodgerBlue3}“बाह्यो”} नीलादि{\color{DodgerBlue3}“रर्थः केवलं नास्ति”} । तत्साधकत्वेनाभिमतस्याध्यक्षस्यासामर्थ्यात् । (३३५)
	\pend
      \label{div_pvv.2.336}\edlabel{div_pvv.2.336}
	  
	% new div opening: depth here is 2
	

	  \pstart एवन्तर्हि ज्ञानस्य गजाद्याकारस्यालोकादिनिमित्तान्तरसद्भावेपि देशकालादिप्रतिनियमदर्शनात् अर्थो \edlabel{pvv.220-2}\footnote{\label{pvv.220-2}  २ प्यस्तीति । अन्यथा सदा सर्वत्र ज्ञानं स्यात् । यथा बीजादि सदा स्थितोपि कार्यकारी ।}व्यवस्यतीत्याह (।)
	\pend
      
	  \bigskip
	  \begingroup
	  \large
	
	    
	    \stanza[\smallbreak]
	\label{pv.2.336}\edlabel{pv.2.336}\flagstanza{\tiny\textenglish{....2.336}}कस्यचित् किञ्चिदेवान्तर्व्वासनायाः प्रबोधकम् ।&ततो धियाम्विनियमो न बाह्यार्थव्यपेक्षया ॥ ३३६ ॥\&[\smallbreak]


	
	  \endgroup
	

	  \pstart {\color{DodgerBlue3}“कस्यचिज्ज्ञा”}नस्य गजाद्याकारस्य {\color{DodgerBlue3}“किञ्चिदेव”} ज्ञानम{\color{DodgerBlue3}“न्तर्व्वासनायाः”} समनन्तरप्रत्यया\edlabel{pvv.220-3}\footnote{\label{pvv.220-3}  ३ आलयाख्य ।}न्तर{\color{DodgerBlue3}“वर्त्तिन्या”} नियतज्ञानजननयोग्यतालक्षणायाः {\color{DodgerBlue3}“प्रबोध\edlabel{pvv.220-4}\footnote{\label{pvv.220-4}  ४ आलयपरिणतेः सहकारिज्ञानं ।}कं”} कार्योत्पादनाभिमुख्यकारकं । {\color{DodgerBlue3}“ततः”} प्रबोधकवशात् {\color{DodgerBlue3}“धियां”} नियताकारतया {\color{DodgerBlue3}“विनियमः। न बाह्यार्थव्यपेक्षया”} (।) को हि विशेषो बाह्यो वा नियामकः प्रतिभासस्य प्रबुद्धवासनाविशेषः समनन्तरप्रत्ययो वा । तत्र वासनायाः सामर्थ्यं स्वप्नादावुपलब्धं । न तु बाह्यस्य नित्यपरोक्षत्वात् । न तथापि परोक्षस्य बाह्यस्य साधकस्याभावेपि नाभावस्थितिरिति चेत् । प्रतिभासमानं ज्ञानं बाह्यं तु न प्रतिभासत एवेति तावतैवाभिमतसिद्धेः । साधकप्रमाणरहितपिशाचायमानबहिरर्थनिषेधेनास्माकमादरः । यदि तु तन्निषेधनिर्ब्बन्धो गरीयान् सांशत्वानंशत्वकल्पनया परमाणुप्रतिषेधे आ चा र्यी यः पर्येशि (? ।षि)तव्यः । (३३६)
	\pend
      \label{div_pvv.2.337}\edlabel{div_pvv.2.337}
	  
	% new div opening: depth here is 2
	

	  \begin{center}%% label @type='head'
	\textbf{(च. विज्ञानद्वैरूप्यम्)}
	\end{center}
	
	  \bigskip
	  \begingroup
	  \large
	
	    
	    \stanza[\smallbreak]
	\label{pv.2.337}\edlabel{pv.2.337}\flagstanza{\tiny\textenglish{....2.337}}तस्माद् द्विरूपमस्त्येकं यदेवमनुभूयते ।&स्मर्यते चोभयाकारस्यास्य सम्वेदनं फलम् ॥ ३३७ ॥\&[\smallbreak]


	
	  \endgroup
	

	  \pstart यस्माद् बाह्योऽर्थो {\color{DodgerBlue3}“नानुभूयते तस्मा”}देकं विज्ञानं अविद्योपप्लुतत्वात् {\color{DodgerBlue3}“द्विरूपं”} बोधरूपं नीलादिरूपञ्चा{\color{DodgerBlue3}“स्ति”} । यत् यस्माज्ज्ञानमेवं द्व्याकारतयाऽनुभूयते स्ववेदनेन यथा\leavevmode\marginnote{\textenglish{221/s}} नुभवं कालान्तरे {\color{DodgerBlue3}“स्मर्यते च”} (।) यथा चान्यस्य संवेदनाभावात् । {\color{DodgerBlue3}“उभयाद्याकारस्य”} नीलाद्यनुभवरूपस्य {\color{DodgerBlue3}“संवेदनं फलं”} । तदेवं प्रमेयो ग्राह्याकारः प्रमाणं ग्राहकाका\edlabel{pvv.221-1}\footnote{\label{pvv.221-1}  १ “
	    \begin{verse}
	सव्यापारप्रतीतत्वात् प्रमाणं फलमेव सत् ।\\
	    स्वसम्वित्तिः फलम्वात्र तद्रूपो ह्यर्थनिश्चयः ।\\
	    विषयाभासतैवास्य प्रमाणन्तेन मीयते ।\\
	    यदाभासं प्रमेयन्तत् प्रमाणफलते पुनः ॥\\
	    ग्राह्यग्राहकसम्वित्ती त्रयन्नातः पृथक्‏कृतमि\\
	    
	    \end{verse}
	  ”ति सूत्रचतुष्टयानुरोधाच्चतुरावर्त्तितः फलविकल्पः । सम्भवमात्रेणामी विकल्पाः समाप्तिस्तु विज्ञानवाद एव कृता ॥}रः फलं स्वसंविदिति दर्शितं भवति ।\edlabel{pvv.221-2}\footnote{\label{pvv.221-2}  २ असमर्थोंयमेषः फलविकल्पः ।}(३३७)
	\pend
      \label{div_pvv.2.338}\edlabel{div_pvv.2.338}
	  
	% new div opening: depth here is 2
	

	  \begin{center}%% label @type='head'
	\textbf{(छ. अर्थसंवित्फलम्)}
	\end{center}
	
	  \bigskip
	  \begingroup
	  \large
	
	    
	    \stanza[\smallbreak]
	\label{pv.2.338}\edlabel{pv.2.338}\flagstanza{\tiny\textenglish{....2.338}}यदा निष्पन्नतद्भाव इष्टोनिष्टोपि वा परः ।&विज्ञप्तिहेतुर्व्विषयस्तस्याश्चानुभवस्तथा ॥ ३३८ ॥\&[\smallbreak]


	
	  \endgroup
	

	  \pstart {\color{DodgerBlue3}“यदा”} बहिरर्थवादेपि {\color{DodgerBlue3}“परो”} वाह्योऽर्थ {\color{DodgerBlue3}“इष्टोऽनिष्टोऽ\edlabel{pvv.221-3}\footnote{\label{pvv.221-3}  ३ रागद्वेषाभ्याम् ।}पि वा निष्यन्नतद्भावो”} भावनावशाद् व्यवस्थितेष्टानिष्टभावः सरूपाया {\color{DodgerBlue3}“विज्ञप्तेर्हेतुः”} सन् {\color{DodgerBlue3}“विषयो”} भवति । तदा {\color{DodgerBlue3}“तस्या”} विज्ञप्तेस्तथा इष्टानिष्टाकारेणानुभवो विषयस्य {\color{DodgerBlue3}“चानुभव”} उच्यते । तेन विषयसारूप्यं प्रमाणमर्थसंवित् फलमु\edlabel{pvv.221-4}\footnote{\label{pvv.221-4}  ४ इति द्वितीयः फलविकल्पः}क्तं । (३३८)
	\pend
      \label{div_pvv.2.339}\edlabel{div_pvv.2.339}
	  
	% new div opening: depth here is 2
	

	  \pstart अथवा विज्ञानवादेप्यविरुद्धमित्याह (।)
	\pend
      
	  \bigskip
	  \begingroup
	  \large
	
	    
	    \stanza[\smallbreak]
	\label{pv.2.339}\edlabel{pv.2.339}\flagstanza{\tiny\textenglish{....2.339}}यदा सविषयं ज्ञानं ज्ञानांशेर्थव्यवस्थितेः ।&तदा य आत्मानुभवः स एवार्थविनिश्चयः ॥ ३३९ ॥\&[\smallbreak]


	
	  \endgroup
	

	  \pstart {\color{DodgerBlue3}“यदा”} ज्ञानस्यांशे आकारे विप्लववशात् {\color{DodgerBlue3}“अर्थ\edlabel{pvv.221-5}\footnote{\label{pvv.221-5}  ५ ग्राह्यस्य ।}स्य व्यवस्थितेर्ज्ञानं सविषय-”} मि\edlabel{pvv.221-6}\footnote{\label{pvv.221-6}  ६ अर्थशून्यं ।}ष्टं । {\color{DodgerBlue3}“तदा य आत्मनो”} ज्ञानाकारस्या{\color{DodgerBlue3}“नुभवः स एवार्थस्य निश्चयः”} संवेदनमिष्यते\leavevmode\marginnote{\textenglish{43b/MA}} (।) ततश्च वि ज्ञा न वा दे प्य\edlabel{pvv.221-7}\footnote{\label{pvv.221-7}  ७ ग्राह्याकारः प्रमेयः ।}र्थाकारः प्रमाणमर्थसंवित्फलमविरुद्धं । (३३९)
	\pend
      \label{div_pvv.2.340}\edlabel{div_pvv.2.340}
	  
	% new div opening: depth here is 2
	
	  \bigskip
	  \begingroup
	  \large
	
	    
	    \stanza[\smallbreak]
	\label{pv.2.340}\edlabel{pv.2.340}\flagstanza{\tiny\textenglish{....2.340}}यदीष्टाकार आत्मा स्यादन्यथा वानुभूयते ।&इष्टोनिष्टोपि वा तेन भवत्यर्थः प्रवेदितः ॥ ३४० ॥\&[\smallbreak]


	
	  \endgroup
	\leavevmode\marginnote{\textenglish{222/s}}

	  \pstart बहिरर्थनयेपि बुद्धिवेदनस्यैवार्थवेदनत्वात् तथा {\color{DodgerBlue3}“यदीष्टाका”}रोऽस्या बुद्धेरा{\color{DodgerBlue3}“त्माऽनुभूयतेऽन्यथा”}निष्टाकारो वा । तदा {\color{DodgerBlue3}“तेन”} ज्ञानेने{\color{DodgerBlue3}“ष्टोऽनिष्टो वार्थः प्रवेदितो भव”}ति नान्यथा । (३४०)
	\pend
      \label{div_pvv.2.341}\edlabel{div_pvv.2.341}
	  
	% new div opening: depth here is 2
	
	  \bigskip
	  \begingroup
	  \large
	
	    
	    \stanza[\smallbreak]
	\label{pv.2.341}\edlabel{pv.2.341}\flagstanza{\tiny\textenglish{....2.341}}विद्यमानेपि बाह्येर्थे यथानुभवमेव सः ।&निश्चितात्मा स्वरूपेण नानेकात्मत्वदोषतः ॥ ३४१ ॥\&[\smallbreak]


	
	  \endgroup
	

	  \pstart यस्मा{\color{DodgerBlue3}“द्विद्यमानेपि बाह्येऽर्थे यथानुभव”}मनुभवाकारानतिक्रमेण स बाह्योऽर्थो {\color{DodgerBlue3}“निश्चितात्मा”} व्यवस्थाप्यते नार्थ{\color{DodgerBlue3}“स्वरूपेण”} । यथा सम्भविना निश्चितात्मा व्यवतिष्ठते । इष्टानिष्टत्वेन पुरुषाभ्यामेकस्यार्थस्य ग्रहणाद{\color{DodgerBlue3}“नेकात्मत्वदोषः”} प्रसज्यते ।\edlabel{pvv.222-1}\footnote{\label{pvv.222-1}  १ अर्थवशाद् बुद्धेर्व्यवस्थाने यावन्त आकारा ज्ञाने तेर्थगताः प्रसजन्तीत्यनेकाकारोर्थः स्यात् ।} (३४१)
	\pend
      \label{div_pvv.2.342_2.343}\edlabel{div_pvv.2.342_2.343}
	  
	% new div opening: depth here is 2
	
	  \bigskip
	  \begingroup
	  \large
	
	    
	    \stanza[\smallbreak]
	\label{pv.2.342}\edlabel{pv.2.342}\flagstanza{\tiny\textenglish{....2.342}}यदि बाह्यं न विद्येत कस्य संवेदनं भवेत् ।&यद्यगत्या स्वरूपस्य बाह्यस्यैव न किं मतम्\edlabel{pvv.222-asterisk}\footnote{\label{pvv.222-asterisk}  * वृत्तिकृता त्वविवृतैषा ॥} ३४२ ॥\&[\smallbreak]


	 
	    
	    \stanza[\smallbreak]
	\label{pv.2.343}\edlabel{pv.2.343}\flagstanza{\tiny\textenglish{....2.343}}अभ्युपायेपि भेदेन न स्यादनुभवो द्वयोः ।&अदृष्टावरणात्स्यात् चेन्न नामार्थवशा गतिः ॥ ३४३ ॥\&[\smallbreak]


	
	  \endgroup
	

	  \pstart अनेका\edlabel{pvv.222-2}\footnote{\label{pvv.222-2}  २ सांख्यस्य बाह्यरूपाः सुखादयः त्रिगुणात्मकत्वाज्जगतः ।}त्मकत्वस्या{\color{DodgerBlue3}“भ्युपाये”} स्वीकारेपि\edlabel{pvv.222-3}\footnote{\label{pvv.222-3}  ३ बाह्यवादिभिरपि यथानुभवमेवार्थनिश्चयोऽभ्युपगत इत्यात्मानुभव एवार्थस्य ।} {\color{DodgerBlue3}“द्वयोः”} पुरुषयो{\color{DodgerBlue3}“र्भेदेने”}ष्टत्वेनैकस्यान्यस्य चानिष्टत्वे{\color{DodgerBlue3}“नानुभवो न स्यात्”} । द्व्याकारत्वाद्वस्तुनस्तथै\edlabel{pvv.222-4}\footnote{\label{pvv.222-4}  ४ द्वयोरपीष्टानिष्टसङ्करप्रतीतिः स्यात् ।}व प्रतीतिप्रसक्तिः । अदृष्टेन सुखदुःखवेदनीयेन कर्मणावरणात् । द्वितीयस्याकारस्य एकात्मत्वेन प्रतिभासः स्यादिति चेत् । एवं सत्यर्थ{\color{DodgerBlue3}“वशा ग\edlabel{pvv.222-5}\footnote{\label{pvv.222-5}  ५ आलोकं चिकीर्षता सर्व्वमेवान्धीकृतं तर्हि ।}ति”}र्ज्ञानमिति नाम प्रसिद्धं न स्यात् । अदृष्टवशेनेष्टानिष्टाकारयोः प्रतीतेरुपदर्शनात् । (३४२, ३४३)
	\pend
      \label{div_pvv.2.344}\edlabel{div_pvv.2.344}
	  
	% new div opening: depth here is 2
	

	  \pstart किञ्च (।)
	\pend
      
	  \bigskip
	  \begingroup
	  \large
	
	    
	    \stanza[\smallbreak]
	\label{pv.2.344}\edlabel{pv.2.344}\flagstanza{\tiny\textenglish{....2.344}}तमनेकात्मकं भावमेकात्मत्वेन दर्शयत् ।&तददृष्टं कथं नाम भवेदर्थस्य दर्शकम् ॥ ३४४ ॥\&[\smallbreak]


	
	  \endgroup
	

	  \pstart भावम{\color{DodgerBlue3}“नेकात्मकमेकात्मकत्वे”}नैकाकारतया दर्श{\color{DodgerBlue3}“यददृष्टं तत्कथमर्थस्य दर्शकं नाम भवेत्”} । तदेव हि दर्शकमस्य यत् तत्स्वरूपं प्रतिभासयति । न चैकाकारोऽर्थः । ततस्तत्प्रतीतिर्नार्थप्रतीतिः । (३४४)
	\pend
      \label{div_pvv.2.345}\edlabel{div_pvv.2.345}
	  
	% new div opening: depth here is 2
	\leavevmode\marginnote{\textenglish{223/s}}
	  \bigskip
	  \begingroup
	  \large
	
	    
	    \stanza[\smallbreak]
	\label{pv.2.345}\edlabel{pv.2.345}\flagstanza{\tiny\textenglish{....2.345}}इष्टानिष्टावभासिन्यः कल्पना नाक्षधीर्यदि ।&अनिष्टादावसन्धानं दृष्टं तत्रापि चेतसाम् ॥ ३४५ ॥\&[\smallbreak]


	
	  \endgroup
	

	  \pstart यथावस्थितवस्तुग्राहिण्य{\color{DodgerBlue3}“क्षधी”}स्तदनन्तर{\color{DodgerBlue3}“मिष्टानिष्टावभासिन्यः कल्पनाऽयथार्था”} । तेनानेकात्मकत्वदोषप्रसङ्ग इति {\color{DodgerBlue3}“यदी”}ष्यते । {\color{DodgerBlue3}“तत्रैन्द्रियत्वेप्यनिष्टादावादि”}शब्दात्कामलादौ {\color{DodgerBlue3}“चेतसामि”}न्द्रियज्ञानानाम{\color{DodgerBlue3}“सन्धा\edlabel{pvv.223-1}\footnote{\label{pvv.223-1}  १ अन्याकारत्वं ।}नमर्था”}काराननुविधानं {\color{DodgerBlue3}“दृष्टं”} । तस्मादिन्द्रियबुद्धिरयथार्थाकारा न भवतीति नास्ति । (३४५)
	\pend
      \label{div_pvv.2.346}\edlabel{div_pvv.2.346}
	  
	% new div opening: depth here is 2
	
	  \bigskip
	  \begingroup
	  \large
	
	    
	    \stanza[\smallbreak]
	\label{pv.2.346}\edlabel{pv.2.346}\flagstanza{\tiny\textenglish{....2.346}}तस्मात् प्रमेये बाह्येपि युक्तं स्वानुभवः फलम् ।&यतः स्वभावोस्य यथा तथैवार्थविनिश्चयः ॥ ३४६ ॥\&[\smallbreak]


	
	  \endgroup
	

	  \pstart {\color{DodgerBlue3}“तस्मान्न”} केवलं स्वरूपे {\color{DodgerBlue3}“बाह्येपि प्रमेये स्वानुभवः फलं युक्तं । यतः कारणात् स्वभावोस्य”} ज्ञानस्य {\color{DodgerBlue3}“यथा”} प्रतिभाति {\color{DodgerBlue3}“तथैव नार्थस्य विनिश्चयः”} सिध्यति । (३४६)
	\pend
      \label{div_pvv.2.347}\edlabel{div_pvv.2.347}
	  
	% new div opening: depth here is 2
	
	  \bigskip
	  \begingroup
	  \large
	
	    
	    \stanza[\smallbreak]
	\label{pv.2.347}\edlabel{pv.2.347}\flagstanza{\tiny\textenglish{....2.347}}तदर्थाभासतैवास्य प्रमाणं न तु सन्नपि ।&ग्राहकाऽत्माऽपरार्थत्वाद् बाह्येष्वर्थेष्वपेक्षते ॥ ३४७ ॥\&[\smallbreak]


	
	  \endgroup
	

	  \pstart {\color{DodgerBlue3}“तत्”} तस्माद् {\color{DodgerBlue3}“बाह्येष्वर्थेषु”} ग्राह्येष्वस्य ज्ञानस्या{\color{DodgerBlue3}“र्थाभासताऽ”}र्थाकारतैव {\color{DodgerBlue3}“प्रमाणम\edlabel{pvv.223-2}\footnote{\label{pvv.223-2}  २ ज्ञानारूढाकारसम्वेदनानुरूपेणार्थावसायव्यवस्थानात् ।}पेक्षते”} न त्वन्वयि{\color{DodgerBlue3}“ग्राहकात्मा”} ग्राहकाकारो{\color{DodgerBlue3}“ऽपरार्थत्वात्”}\edlabel{pvv.223-3}\footnote{\label{pvv.223-3}  ३ परार्थं प्रमाणन्न च ग्राह्याकारवद् ग्राहकाकारो व्यवस्थाकारीति भावः ।}आत्मविषयत्वात्तस्य । (३४७)
	\pend
      \label{div_pvv.2.348}\edlabel{div_pvv.2.348}
	  
	% new div opening: depth here is 2
	

	  \begin{center}%% label @type='head'
	\textbf{(ज. आत्मसंविदेवार्थसंचिद्)}
	\end{center}
	
	  \bigskip
	  \begingroup
	  \large
	
	    
	    \stanza[\smallbreak]
	\label{pv.2.348}\edlabel{pv.2.348}\flagstanza{\tiny\textenglish{....2.348}}यस्माद् यथा निविष्टोसावर्थात्मा प्रत्यये तथा ।&निश्चीयते निविष्टोसावेवमित्यात्मसंविदः ॥ ३४८ ॥\&[\smallbreak]


	
	  \endgroup
	

	  \pstart {\color{DodgerBlue3}“यस्मा”}त्कारण{\color{DodgerBlue3}“द्यथा”} इष्टत्वेनानिष्टत्वेन वार्थस्यात्माकारः {\color{DodgerBlue3}“प्रत्यये निविष्ट”}स्तथाऽर्थों निश्चीयते तस्मादर्थाकारः प्रमाणं । {\color{DodgerBlue3}“असा”}वर्थाकार एवमिष्टानिष्टत्वेन बुद्धौ {\color{DodgerBlue3}“निविष्ट”} इत्यात्मसंविदः स्वसंवेदना{\color{DodgerBlue3}“न्निश्चीयते”} । (३४८)
	\pend
      \label{div_pvv.2.349}\edlabel{div_pvv.2.349}
	  
	% new div opening: depth here is 2
	
	  \bigskip
	  \begingroup
	  \large
	
	    
	    \stanza[\smallbreak]
	\label{pv.2.349}\edlabel{pv.2.349}\flagstanza{\tiny\textenglish{....2.349}}इत्यर्थसंवित् सैवेष्टा यतोर्थात्मा न दृश्यते ।&तस्माद् बुद्धिनिवेश्यार्थः साधनं तस्य सा क्रिया ॥ ३४९ ॥\&[\smallbreak]


	
	  \endgroup
	

	  \pstart इति तस्मात् सैवात्मसंविद{\color{DodgerBlue3}“र्थसंविदिष्टा । यतः”} स्वरूपाद् बहिर्भूतो{\color{DodgerBlue3}“ऽर्थात्मा न दृश्यते”} बुद्ध्याकार एव तु वेद्यते । अतस्तद्वेदनदर्शनमेवार्थवेदनं । यस्माच्चार्थ\leavevmode\marginnote{\textenglish{224/s}} सारूप्यवशेनार्थाधिगतिव्यवस्था {\color{DodgerBlue3}“तस्मा”}द् {\color{DodgerBlue3}“बुद्धि”}निवे{\color{DodgerBlue3}“श्यार्थो”}ऽर्थप्रतिबिम्बं {\color{DodgerBlue3}“साधनं”} प्रमाणं {\color{DodgerBlue3}“तस्य”} प्रमाणस्य {\color{DodgerBlue3}“सा”}धिगतिः {\color{DodgerBlue3}“क्रिया”} फलमिष्यते । (३४९)
	\pend
      \label{div_pvv.2.350}\edlabel{div_pvv.2.350}
	  
	% new div opening: depth here is 2
	
	  \bigskip
	  \begingroup
	  \large
	
	    
	    \stanza[\smallbreak]
	\label{pv.2.350}\edlabel{pv.2.350}\flagstanza{\tiny\textenglish{....2.350}}यथा निविशते सोर्थो यतः सा प्रथते तथा ।&अर्थस्थितेस्तदात्मत्वात् स्वविदप्यर्थविन्मता ॥ ३५० ॥\&[\smallbreak]


	
	  \endgroup
	

	  \pstart यतः कारणात् {\color{DodgerBlue3}“यथार्थो”} ज्ञानात्मनि {\color{DodgerBlue3}“निविशते”} तथा सा स्वसंवित्तिः प्रथते ख्याति (।) तस्मादर्थस्य स्थितेरधिगतेस्तदात्मत्वात् परमार्थतः\edlabel{pvv.224-1}\footnote{\label{pvv.224-1}  १ तद्यद्यर्थसारूप्यं मानं स्ववित्फलं तदा विषयभेदः । आत्मविषया वित्तिर्ब्बाह्यविषयं सारूप्यमित्याह । प्रतिभासमात्रेण तद्व्यवस्थाहेतुत्वात् ।} {\color{DodgerBlue3}“स्वविदपि”} सती {\color{DodgerBlue3}“अर्थविद् मता”} । स्वसंवेदनमेवार्थवेदनमुपचारादुच्यत इति तादात्म्यमनयोः (।) (३५०)
	\pend
      \label{div_pvv.2.351}\edlabel{div_pvv.2.351}
	  
	% new div opening: depth here is 2
	
	  \bigskip
	  \begingroup
	  \large
	
	    
	    \stanza[\smallbreak]
	\label{pv.2.351a}\edlabel{pv.2.351a}\flagstanza{\tiny\textenglish{...2.351a}}तस्माद् विषयभेदोपि न;\&[\smallbreak]


	
	  \endgroup
	

	  \pstart यस्मादर्थाकार एव प्रमाणं प्रतीतिसाधनत्वात् फलञ्च प्रतीयमानं {\color{DodgerBlue3}“तस्मा-”} त्प्रमाणफलयो{\color{DodgerBlue3}“र्व्विषयभेदोपि”} नास्ति । यथा परमते आलोचनविशेषणज्ञानादीनामेकार्थापेक्षिण्यावेव हि सारूप्यप्रतिपत्ती प्रमाणफलत्वं प्रतिलभेते ।
	\pend
      

	  \pstart यद्यर्थसंवेदनं फलं तदा स्ववित्फलं कथमुक्तमित्याह (।)
	\pend
      
	  \bigskip
	  \begingroup
	  \large
	
	    
	    \stanza[\smallbreak]
	\label{pv.2.351b}\edlabel{pv.2.351b}\flagstanza{\tiny\textenglish{...2.351b}}स्वसंवेदनं फलम् ।&उक्तं स्वभावचिन्तायां तादात्म्यादर्थसंविदः ॥ ३५१ ॥\&[\smallbreak]


	
	  \endgroup
	

	  \pstart अर्थसंवेदंनस्य वस्तुतः {\color{DodgerBlue3}“स्वभावचिन्तायां”} स्वसंवेदनं फल{\color{DodgerBlue3}“मुक्तं । अर्थसंविदस्ता”}\leavevmode\marginnote{\textenglish{44a/MA}} {\color{DodgerBlue3}“दात्म्यात्”} \edlabel{pvv.224-2}\footnote{\label{pvv.224-2}  २ “विद्यमानेपि बाह्येर्थे यथानुभवमेव वे ”\cref{pv.2.341}त्यादिना । “यदा त्वबाह्य एवार्थः प्रमेय” इत्यादि \href{http://http://sarit.indology.info/?cref=psv.1.9}{समुच्चयस्य तृतीयः फलविकल्पो व्याख्यातः} ।}स्वसंवेदनात्मत्वात् । न स्वाकारप्रतीतेरन्यास्त्यर्थप्रतीतिः स्वरूपेण तस्याः प्रतीतेः । (३५१)
	\pend
      \label{div_pvv.2.352}\edlabel{div_pvv.2.352}
	  
	% new div opening: depth here is 2
	

	  \pstart ननु यदि यथार्थं नानुभवस्तदा स्ववासनाप्रबोधकमनुवर्तमानस्य ज्ञानस्यार्थों\edlabel{pvv.224-3}\footnote{\label{pvv.224-3}  ३ विना च बाह्यं तत्स्वरूपाग्रहणे कथम्बाह्यार्थस्तृतीयः फलविकल्पः ।} बाह्योस्तीत्येतदेव कुत इत्याह (।)
	\pend
      
	  \bigskip
	  \begingroup
	  \large
	
	    
	    \stanza[\smallbreak]
	\label{pv.2.352}\edlabel{pv.2.352}\flagstanza{\tiny\textenglish{....2.352}}तथावभासमानस्य तादृशोऽन्यादृशोपि वा ।&ज्ञानस्य हेतुरर्थोपीत्यर्थस्येष्टा प्रमेयता ॥ ३५२ ॥\&[\smallbreak]


	
	  \endgroup
	

	  \pstart {\color{DodgerBlue3}“तथा”} इष्टानिष्टाकारेणा{\color{DodgerBlue3}“व\edlabel{pvv.224-4}\footnote{\label{pvv.224-4}  ४ अर्थनिरपेक्षत्वं ।}भासमानस्य ज्ञानस्या”}र्थो{\color{DodgerBlue3}“पि ता\edlabel{pvv.224-5}\footnote{\label{pvv.224-5}  ५ यथावभासोऽन्यादृशो आलोकादिभावेपि नीलादिरहितदेशेऽनुत्पत्तेरिन्द्रियसिद्धिवत् ।}दृशो”}न्यादृशो वा \leavevmode\marginnote{\textenglish{225/s}} हेतुरित्यर्थस्य {\color{DodgerBlue3}“प्रमेयतेष्टा”} सौ त्रा न्ति क मते । सारूप्यस्य परमार्थतः “{\color{DodgerBlue3}“सरूप”}यन्ति तत् केन स्थूलाभासञ्च तेणव” \href{http://http://sarit.indology.info/?cref=pv.2.321}{(२।३२१)} इति प्रतिषेधात्\edlabel{pvv.225-1}\footnote{\label{pvv.225-1}  १ ननु तादृशोऽन्यादृशोपि वार्थः प्रमेयो नेष्टः स्वपरयोस्तत् किमेवमुच्यते । परेण स्ववाचा बाह्यं निषेधयितुमुक्तं तदाह ।}। (३५२)
	\pend
      \label{div_pvv.2.353}\edlabel{div_pvv.2.353}
	  
	% new div opening: depth here is 2
	
	  \bigskip
	  \begingroup
	  \large
	
	    
	    \stanza[\smallbreak]
	\label{pv.2.353a}\edlabel{pv.2.353a}\flagstanza{\tiny\textenglish{...2.353a}}यथा कथञ्चित्तस्यार्थरूपं मुक्त्वावभासिनः ।&अर्थग्रहः कथं;\&[\smallbreak]


	
	  \endgroup
	

	  \pstart यदि सारूप्याभावस्तदा तस्य ज्ञानस्यावभासिनो {\color{DodgerBlue3}“यथा कथञ्चि”}दिष्टानिष््टादिना भासमानमर्थरूपमर्थाकारं मुक्त्वा {\color{DodgerBlue3}“कथं”} केन प्रकारेणार्थस्य {\color{DodgerBlue3}“ग्रहः”} स्यात् । न ह्यर्थः स्वरूपेण दृश्यते । तत् स्वरूपबुद्धिवेदनादर्थग्रहव्यवस्था । सारूप्यमेव चेन्न संभवति कथमर्थग्रह इति मन्यते सौ त्रा न्ति कः
	\pend
      

	  \pstart यो गा चा र स्तु तस्य साहाय्यकं मन्यमान आह (।)
	\pend
      
	  \bigskip
	  \begingroup
	  \large
	
	    
	    \stanza[\smallbreak]
	\label{pv.2.353b}\edlabel{pv.2.353b}\flagstanza{\tiny\textenglish{...2.353b}}सत्यं न जानेहमपोदृशम् ॥ ३५३ ॥\&[\smallbreak]


	
	  \endgroup
	

	  \pstart {\color{DodgerBlue3}“स\edlabel{pvv.225-2}\footnote{\label{pvv.225-2}  २ य एवभिच्छति स पृच्छयतां ।}त्यं न जानेऽहमपीदृशमर्थग्रहः”} कथमिति । (३५३)
	\pend
      \label{div_pvv.2.354}\edlabel{div_pvv.2.354}
	  
	% new div opening: depth here is 2
	

	  \pstart कथं तर्ह्यसत्यर्थे ग्राह्यग्राहकफलभेद इत्याह (।)
	\pend
      
	  \bigskip
	  \begingroup
	  \large
	
	    
	    \stanza[\smallbreak]
	\label{pv.2.354}\edlabel{pv.2.354}\flagstanza{\tiny\textenglish{....2.354}}अविभागोपि बुद्ध्यात्मविपर्यासितदर्शनैः ।&ग्राह्यग्राहकसंवित्तिभेदवानिव लक्ष्यते ॥ ३५४ ॥\&[\smallbreak]


	
	  \endgroup
	

	  \pstart पर\edlabel{pvv.225-3}\footnote{\label{pvv.225-3}  ३ आनुषङ्गिकमुक्त्वा विज्ञप्तौ प्रमादिमाह ।} मार्थतो{\color{DodgerBlue3}“ऽविभा”}गो भेदरहितो{\color{DodgerBlue3}“पि बुद्ध्यात्मद्व”}यवासनया {\color{DodgerBlue3}“विपर्यासितं”} विभागेनोप{\color{DodgerBlue3}“दर्शितं”} दर्शनं येषां तैरतत्त्वदर्शिपुरुषै{\color{DodgerBlue3}“र्ग्राह्यग्राहकसंवित्ती”}नां परस्परं भेदस्तद्वानिव लक्ष्यते । (३५४)
	\pend
      \label{div_pvv.2.355}\edlabel{div_pvv.2.355}
	  
	% new div opening: depth here is 2
	

	  \pstart अत्र दृष्टान्तमाह (।)
	\pend
      
	  \bigskip
	  \begingroup
	  \large
	
	    
	    \stanza[\smallbreak]
	\label{pv.2.355}\edlabel{pv.2.355}\flagstanza{\tiny\textenglish{....2.355}}मन्त्राद्युपप्लुताक्षाणां यथा मृच्छकलादयः ।&अन्यथैवावभासन्ते तद्रूपरहिता अपि ॥ ३५५ ॥\&[\smallbreak]


	
	  \endgroup
	

	  \pstart {\color{DodgerBlue3}“यथा मूच्छकलादयो मन्त्रादिभिरुपप्लुत”}यथार्थज्ञानहेतुकृत{\color{DodgerBlue3}“मक्षं”} येषां तेषामन्यथा सुवर्ण्णादित्वे{\color{DodgerBlue3}“नैवावभासन्ते”} । तेन सुवर्ण्णादिना रूपेण {\color{DodgerBlue3}“रहिता अपि”} वस्तुतः । (३५५)
	\pend
      \leavevmode\marginnote{\textenglish{226/s}}\label{div_pvv.2.356}\edlabel{div_pvv.2.356}
	  
	% new div opening: depth here is 2
	

	  \pstart कस्मात्पुनर्म्मन्त्रादिसामर्थ्यात् सुवर्ण्णादितामेव यातं मृत्खण्डमिति नाभ्युप(ग)म्यते इत्याह (।)
	\pend
      
	  \bigskip
	  \begingroup
	  \large
	
	    
	    \stanza[\smallbreak]
	\label{pv.2.356}\edlabel{pv.2.356}\flagstanza{\tiny\textenglish{....2.356}}तथैवादर्शनात्तेषामनुपप्लुतचक्षुषाम् ।&दूरे यथा वा मरुषु महानल्पोपि दृश्यते ॥ ३५६ ॥\&[\smallbreak]


	
	  \endgroup
	

	  \pstart {\color{DodgerBlue3}“मन्त्रादि”}नाऽनुपप्लुतचक्षुषान्येन पुंसा तेषां मृच्छकलादीनां यथोपहताक्षेण ते दृश्यन्ते {\color{DodgerBlue3}“तथैव दर्शनात्”} मृदात्मत्वेनैव दर्शनात् सुवर्ण्णादित्वेन निष्पत्तिकल्पनमयुक्तं । {\color{DodgerBlue3}“यथा मरुषु दूरेऽल्पोपि महान् दृश्यते”} तद्देशस्थैरल्पत्वेनैव भातस्य दर्शनात् । (३५६)
	\pend
      \label{div_pvv.2.357}\edlabel{div_pvv.2.357}
	  
	% new div opening: depth here is 2
	
	  \bigskip
	  \begingroup
	  \large
	
	    
	    \stanza[\smallbreak]
	\label{pv.2.357}\edlabel{pv.2.357}\flagstanza{\tiny\textenglish{....2.357}}यथानुदर्शनञ्चेयं मेयमानफलस्थितिः ।&क्रियतेऽविद्यमानापि ग्राह्यग्राहकसंविदाम् ॥ ३५७ ॥\&[\smallbreak]


	
	  \endgroup
	

	  \pstart तस्माद् {\color{DodgerBlue3}“ग्राह्यग्राहकसंविदां”} परमार्थतो{\color{DodgerBlue3}“ऽविद्यमानापि मेयमानफलस्थितिर्यथादर्शनं क्रिय”}ते । ग्राह्याका\edlabel{pvv.226-1}\footnote{\label{pvv.226-1}  १ प्रथमे विकल्पे नीलादिपरिच्छेदद्वारेव तज्ज्ञानं फलं आकारः प्रमा । द्वितीये विज्ञप्तौ ग्राह्यांशः प्रमेयो ग्राहकांशः प्रमा । तदुभयवेदनं फलं । तृतीये तु प्रतिपत्तृव्यवसाये नेष्टानिष्टादिरूपेण गृह्यमानोऽर्थः प्रमेयः तथा भानं मानं । तत्संवित् फलं । तदार्थाभासतैवासत्तैवास्य प्रमाणमित्युक्तः चतुर्थः ।}रो मेयः ग्राहकाकारो मानं संवित्तिः फलमिति व्यवस्थाप्यते । (३५७)
	\pend
      \label{div_pvv.2.358}\edlabel{div_pvv.2.358}
	  
	% new div opening: depth here is 2
	
	  \bigskip
	  \begingroup
	  \large
	
	    
	    \stanza[\smallbreak]
	\label{pv.2.358}\edlabel{pv.2.358}\flagstanza{\tiny\textenglish{....2.358}}अन्यथैकस्य भावस्य नानारूपावभासिनः ।&सत्यं कथं स्युराकारास्तदेकत्वस्य हानितः ॥ ३५८ ॥\&[\smallbreak]


	
	  \endgroup
	

	  \pstart {\color{DodgerBlue3}“अन्यथा”} यदि वस्तुतो ग्राहकादिविभाग इष्यते त{\color{DodgerBlue3}“दैकस्य भावस्य”} ज्ञानात्मनो {\color{DodgerBlue3}“नानारूपा(व)भासिनो”}ऽनेकाकारप्रतिभासवन्त आकारा ग्राह्यादिकाः {\color{DodgerBlue3}“कथं सत्यं स्युः”} । तस्यैकज्ञानात्मन {\color{DodgerBlue3}“एकत्वस्य हानितः”} । न ह्येकं प्रतिभासमानानेकाकारात्मकं भवितुमर्हति । प्रतिभासेन सर्व्वेषां भेदेन व्यवस्थापनात् । (३५८)
	\pend
      \label{div_pvv.2.359}\edlabel{div_pvv.2.359}
	  
	% new div opening: depth here is 2
	

	  \pstart अनेकमेव तर्हि ज्ञानं परिच्छेदादित्वेन स्यादित्याह ।
	\pend
      
	  \bigskip
	  \begingroup
	  \large
	
	    
	    \stanza[\smallbreak]
	\label{pv.2.359a}\edlabel{pv.2.359a}\flagstanza{\tiny\textenglish{...2.359a}}अन्यस्यान्यत्वहानेश्च ;\&[\smallbreak]


	
	  \endgroup
	

	  \pstart न चानेकं ज्ञानं एकमेवेष्टव्यं । वस्तुतो भिन्नस्याभेदे इष्यमाणे ग्राह्यादेः परस्परतो{\color{DodgerBlue3}“ऽन्यस्यान्यत्वहानेः”} । यदा हि भेदप्रतिभासेप्येकत्वमिष्टं तदा भेद एव न व्यवस्थितः । ततश्च सुखदुःखादयो नीलपीतादयश्च परस्परं न भिद्येरन् । प्रतिभासभेदस्य तत्साधनत्वात् । अन्यस्य च तत्साधनस्याभावात् ।
	\pend
      \leavevmode\marginnote{\textenglish{227/s}}

	  \pstart एवन्तर्ह्येकं तावत् ज्ञानं नानाकारं स्यादिति शङ्कायां नकारं काकाक्षिवत् संबन्धयन्नाह (।)
	\pend
      
	  \bigskip
	  \begingroup
	  \large
	
	    
	    \stanza[\smallbreak]
	\label{pv.2.359b}\edlabel{pv.2.359b}\flagstanza{\tiny\textenglish{...2.359b}}नाभेदो रूपदर्शनात् ।&रूपाभेदेपि पश्यन्ती धीरभेदं व्यवस्यति ॥ ३५९ ॥\&[\smallbreak]


	
	  \endgroup
	

	  \pstart {\color{DodgerBlue3}“ना\edlabel{pvv.227-1}\footnote{\label{pvv.227-1}  १ अभिन्नरूपदर्शने हि अभेदः स्यात् ।}भेदो”} ज्ञानस्याभिबुद्धस्य {\color{DodgerBlue3}“रूप”}\edlabel{pvv.227-2}\footnote{\label{pvv.227-2}  २ बुद्धेन रूपन्न दृश्यते ।}स्याद{\color{DodgerBlue3}“र्शनात्”} । रूपाभेदञ्चाद्रयादौ {\color{DodgerBlue3}“पश्यन्ती”} धीरभेदं {\color{DodgerBlue3}“व्यवस्यति”} । न च ग्राह्यग्राहकादिषु भिन्नाभासेष्वभेदप्रतिभासः । (३५९)
	\pend
      \label{div_pvv.2.360}\edlabel{div_pvv.2.360}
	  
	% new div opening: depth here is 2
	

	  \pstart तस्याद् (।)
	\pend
      
	  \bigskip
	  \begingroup
	  \large
	
	    
	    \stanza[\smallbreak]
	\label{pv.2.360}\edlabel{pv.2.360}\flagstanza{\tiny\textenglish{....2.360}}भावा येन निरूप्यन्ते तद्रूपं नास्ति तत्त्वतः ।&यस्मादेकमनेकं च रूपं तेषां न विद्यते ॥ ३६० ॥\&[\smallbreak]


	
	  \endgroup
	

	  \pstart {\color{DodgerBlue3}“भावा\edlabel{pvv.227-3}\footnote{\label{pvv.227-3}  ३ चित्तगताः ।}”} ग्राह्यादयो {\color{DodgerBlue3}“येन”} रूपेण ग्राह्यत्वादिना {\color{DodgerBlue3}“निरूप्यन्ते”} अनुभूयन्ते {\color{DodgerBlue3}“तद्रूपं तत्त्वत-”} स्तेषां {\color{DodgerBlue3}“नास्ति । यस्मादेकं”} रूप{\color{DodgerBlue3}“मनेकञ्च रूपं तेषां न विद्यते”} । वस्तुभवदेकमनेकं वा स्यात् । न च ग्राह्याद्याभा एकोऽनेको वा युक्तः । तस्मादुप्पलव एवायं । (३६०)
	\pend
      \label{div_pvv.2.361_2.362_2.363}\edlabel{div_pvv.2.361_2.362_2.363}
	  
	% new div opening: depth here is 2
	

	  \pstart ननु (।)
	\pend
      
	  \bigskip
	  \begingroup
	  \large
	
	    
	    \stanza[\smallbreak]
	\label{pv.2.361}\edlabel{pv.2.361}\flagstanza{\tiny\textenglish{....2.361}}साधर्म्यदर्शनाल्लोके भ्रान्तिर्नामोपजायते ।&अतदात्मनि तादात्म्यव्यवसायेना नेह तत् ॥ ३६१ ॥\&[\smallbreak]


	
	  \endgroup
	
	  \bigskip
	  \begingroup
	  \large
	
	    
	    \stanza[\smallbreak]
	\label{pv.2.362a}\edlabel{pv.2.362a}\flagstanza{\tiny\textenglish{...2.362a}}अदर्शनाज्जगत्यस्मिन्नेकस्यापि तदात्मनः ॥\&[\smallbreak]


	
	  \endgroup
	

	  \pstart {\color{DodgerBlue3}“लोके साधर्म्यदर्शनादतदात्म\edlabel{pvv.227-4}\footnote{\label{pvv.227-4}  ४ मरीचिषु जलवत् ।}नि तादात्म्यव्यवसायेन भ्रान्तिर्व्वि”}तथा-\leavevmode\marginnote{\textenglish{44b/MA}} कारा बुद्धि{\color{DodgerBlue3}“र्जायते”} इति नाम प्रसिद्धं । {\color{DodgerBlue3}“इह”} विज्ञप्तिनये तत् साधर्म्यदर्शनं {\color{DodgerBlue3}“ना”}स्ति । (३६१) {\color{DodgerBlue3}“एकस्यापि तदात्मनो”}ऽभूताकारस्य {\color{DodgerBlue3}“जगत्यदर्शनात्”} ।
	\pend
      

	  \pstart अत्राह (।)
	\pend
      
	  \bigskip
	  \begingroup
	  \large
	
	    
	    \stanza[\smallbreak]
	\label{pv.2.362b}\edlabel{pv.2.362b}\flagstanza{\tiny\textenglish{...2.362b}}अस्तीयमपि या त्वन्तरुपप्लवसमुद्भवा ॥ ३६२ ॥\&[\smallbreak]


	
	  \endgroup
	
	  \bigskip
	  \begingroup
	  \large
	
	    
	    \stanza[\smallbreak]
	\label{pv.2.363}\edlabel{pv.2.363}\flagstanza{\tiny\textenglish{....2.363}}दोषोद्भवा प्रकृत्या सा वितथप्रतिभासिनी ।&अनपेक्षितसाधर्म्यदृगादिस्तैमिरादिवत् ॥ ३६३ ॥\&[\smallbreak]


	
	  \endgroup
	

	  \pstart साधर्म्यदर्शनाद्या मानसी भ्रान्ति{\color{DodgerBlue3}“रियमप्यस्ति ।\edlabel{pvv.227-5}\footnote{\label{pvv.227-5}  ५ इयमेवेति तु न नियमः । अनपेक्षितसाधर्म्याप्यस्ति ।} या पुनरन्तरुपप्लवसमुद्भवा”} अविद्याप्रभवा सा प्रकृत्या दोषोद्भवा । अविद्या तिमिरादिहेतुका वितथप्रतिभासिन्यभूताकारा अनपेक्षितसाधर्म्यदृगादिस्तैमिरादिवत् । यथा तिमिरज्ञान\leavevmode\marginnote{\textenglish{228/s}} मिन्द्रियदोषादनपेक्षितसाधर्म्यमेव जायते तथा ग्राह्यादिभ्रान्तिरपीत्यर्थः, जायमानमेव हि ज्ञानं वासनासामर्थ्यादनुभवाननुभवत्वेनान्तर्ब्बहिर्देशत्वेन सुखादि नीलादि दर्शयति । एवं तर्हि बहिरर्थोपदर्शनशक्तमेव स्वदर्शनहेतुबलात् कस्मान्नेष्यत ज्ञानमिति चेत् । न (।) स्वप्नादावविद्याया असद्ग्राह्याकारोपदर्शनसामर्थ्योपलब्धेः । आकारातिरिक्तबहिरर्थस्यादर्शनान्न ज्ञानस्य तदुपदर्शनसामर्थ्यकल्पना न्याय्येत्यास्तां तावदिदं ॥ (३६२, ३६३)
	\pend
      \label{div_pvv.2.364}\edlabel{div_pvv.2.364}
	  
	% new div opening: depth here is 2
	

	  \begin{center}%% label @type='head'
	\textbf{भ. विज्ञप्तिमात्रतायां प्रमाणफलव्यवस्था}
	\end{center}
	

	  \pstart विज्ञप्तिमात्रतायां प्रमाणफलव्यवस्थाप(ना)र्थमा\edlabel{pvv.228-1}\footnote{\label{pvv.228-1}  १ बाह्यविज्ञानवादौ द्विरावर्त्तितौ भेदकथनार्थ ।}ह (।)
	\pend
      
	  \bigskip
	  \begingroup
	  \large
	
	    
	    \stanza[\smallbreak]
	\label{pv.2.364}\edlabel{pv.2.364}\flagstanza{\tiny\textenglish{....2.364}}तत्र बुद्धेः परिच्छेदो ग्राहकाकारसम्मतः ।&तदात्म्यादात्मवित्तस्य स तस्य साधनं ततः ॥ ३६४ ॥\&[\smallbreak]


	
	  \endgroup
	

	  \pstart {\color{DodgerBlue3}“तत्र”} विज्ञप्तिमात्रतायां {\color{DodgerBlue3}“बुद्धेः”} सग्राहकाकारः {\color{DodgerBlue3}“संमतः परिच्छेद”} आकार {\color{DodgerBlue3}“आत्मवित्”} तस्य ग्राहकाकारस्य फलं । {\color{DodgerBlue3}“तादात्म्यात्”} परिच्छेदस्वभावत्वात् {\color{DodgerBlue3}“स्वसंवित्तिः फलम्वेति सूत्रे स्वाभासं विषयाभासञ्च ज्ञानमुत्पद्यते (।) तत्र यत् स्वसंवेदनन्तत् फलं विषय (:)”} परिच्छेदात्मक एव हि ग्राहकाकारः स एव चात्मवेदनं । {\color{DodgerBlue3}“ततः”} परिच्छेदवशेनात्मवेदनव्यवस्थानात् (।) स परिच्छेद{\color{DodgerBlue3}“स्तस्या”}त्मविदः\edlabel{pvv.228-2}\footnote{\label{pvv.228-2}  २ विज्ञप्तौ ग्राह्यांशो मेयो यः पूर्व्व प्रमाणमुक्तः । ग्राहकांशः प्रमा तदुभयसंवेदनं फलं ।} फलभूतायाः {\color{DodgerBlue3}“साधनं”} प्रमाणं । (३६४)
	\pend
      \label{div_pvv.2.365}\edlabel{div_pvv.2.365}
	  
	% new div opening: depth here is 2
	
	  \bigskip
	  \begingroup
	  \large
	
	    
	    \stanza[\smallbreak]
	\label{pv.2.365}\edlabel{pv.2.365}\flagstanza{\tiny\textenglish{....2.365}}तत्रात्मविषये माने यथा रागादिवेदनम् ।&इयं सर्व्वत्र संयोज्या मानमेयफलस्थितिः ॥ ३६५ ॥\&[\smallbreak]


	
	  \endgroup
	

	  \pstart {\color{DodgerBlue3}“तत्र”} एवं सत्या{\color{DodgerBlue3}“त्मविषये माने”} स्वसंवेदने प्रमाणे {\color{DodgerBlue3}“यथा रागादिवेदनं\edlabel{pvv.228-3}\footnote{\label{pvv.228-3}  ३ स्वविन्निष्ठं ।}”} मानमेयफलात्मकं । {\color{DodgerBlue3}“इयं मानमेयफलस्थितिः”} सर्व्वत्र विज्ञप्तिनयेपि संयोज्या । (३६५)
	\pend
      \label{div_pvv.2.366}\edlabel{div_pvv.2.366}
	  
	% new div opening: depth here is 2
	

	  \pstart तथा हि (।)
	\pend
      
	  \bigskip
	  \begingroup
	  \large
	
	    
	    \stanza[\smallbreak]
	\label{pv.2.366}\edlabel{pv.2.366}\flagstanza{\tiny\textenglish{....2.366}}तत्राप्यनुभवात्मत्वात्ते योग्या स्वात्मसंविदि ।&इति सा योग्यता मानमात्मा मेयः फलं स्ववित् ॥ ३६६ ॥\&[\smallbreak]


	
	  \endgroup
	

	  \pstart {\color{DodgerBlue3}“तत्रापि”} रागादिवेदनेपि {\color{DodgerBlue3}“ते”} रागादयोऽ{\color{DodgerBlue3}“नुभवात्मत्वात् स्वा”}त्मनः {\color{DodgerBlue3}“संविदि योग्या इति”} तस्मात् {\color{DodgerBlue3}“सा”} स्ववेदन{\color{DodgerBlue3}“योग्यता”} रागादीनां {\color{DodgerBlue3}“मानमा\edlabel{pvv.228-4}\footnote{\label{pvv.228-4}  ४ अनुभवनीयः ।}त्मा मेयः फलं स्ववित्”} । (३६६)
	\pend
      \leavevmode\marginnote{\textenglish{229/s}}\label{div_pvv.2.367}\edlabel{div_pvv.2.367}
	  
	% new div opening: depth here is 2
	

	  \pstart ननु रागादिष्टात्मसंवेदनं फलभूतं मानयुक्तमिह तु ग्राहकाकारो योग्यतालक्षणं प्रमाणमुच्यत इति व्या\edlabel{pvv.229-1}\footnote{\label{pvv.229-1}  १ दिग्नागेनोक्तं रागादिषु च स्वसम्वेदनमिन्द्रियानपेक्षत्वात् मानसं प्रत्यक्षं । पुनः ग्राहकाकारसम्वित्ती प्रमाणफलते इति ग्राहकाकारः प्रमाणं । वार्त्तिके तु योग्यतोक्ता । आह एकाभिप्रायतां ।}हतमिति शङ्कायामुत्तरं (।)
	\pend
      
	  \bigskip
	  \begingroup
	  \large
	
	    
	    \stanza[\smallbreak]
	\label{pv.2.367}\edlabel{pv.2.367}\flagstanza{\tiny\textenglish{....2.367}}ग्राहकाकारसंख्याता परिच्छेदात्मतात्मनि ।&सा योग्यतेति च प्रोक्तं प्रमाणं स्वात्मवेदनम् ॥ ३६७ ॥\&[\smallbreak]


	
	  \endgroup
	

	  \pstart या {\color{DodgerBlue3}“ग्राहकाकारसंख्याता परिच्छेदात्मता साऽत्मनि”} संवेदने {\color{DodgerBlue3}“योग्यता चेति”} तस्माद्रागादिषु {\color{DodgerBlue3}“स्वात्मसंवेदनं प्रमाणं प्रोक्तं । न हि स्ववेदनं”} फलभूतमिह विवक्षितं किन्तु योग्यता । तस्मादभिन्नार्थतैव । (३६७)
	\pend
      \label{div_pvv.2.368}\edlabel{div_pvv.2.368}
	  
	% new div opening: depth here is 2
	

	  \pstart ननु स्वाभासं तावदस्तु ज्ञानं परिच्छेदस्व\edlabel{pvv.229-2}\footnote{\label{pvv.229-2}  २ ज्ञानं तज्‏ज्ञानविशेषात्तु द्विरूपता ज्ञानस्य ।}भावत्वात् । विषयाभासं तु कथं । यथा विषयं कारणं तथा चक्षुरादिरपीति तदाकारतापि स्यादित्याह (।)
	\pend
      
	  \bigskip
	  \begingroup
	  \large
	
	    
	    \stanza[\smallbreak]
	\label{pv.2.368}\edlabel{pv.2.368}\flagstanza{\tiny\textenglish{....2.368}}सर्व्वमेव हि विज्ञानं विषयेभ्यः समुद्भवद् ।&तदन्यस्यापि हेतुत्वे कथञ्चिद् विषयाकृति ॥ ३६८ ॥\&[\smallbreak]


	
	  \endgroup
	

	  \pstart {\color{DodgerBlue3}“सर्वमेव हि ज्ञानं विषयेभ्यः समुद्भवदु”}त्पद्यमानं तेभ्यो विषयेभ्योऽन्यस्येन्द्रियादेरपि हेतुत्वे कथञ्चित् स्वकारणायातविषयगतशक्तिभेदा{\color{DodgerBlue3}“द्विषयाकृति”} भ\edlabel{pvv.229-3}\footnote{\label{pvv.229-3}  ३ आलम्बनच्छायमिति वस्तुधर्मता ।}वति नेन्द्रियाद्याकारं । (३६८)
	\pend
      \label{div_pvv.2.369}\edlabel{div_pvv.2.369}
	  
	% new div opening: depth here is 2
	
	  \bigskip
	  \begingroup
	  \large
	
	    
	    \stanza[\smallbreak]
	\label{pv.2.369}\edlabel{pv.2.369}\flagstanza{\tiny\textenglish{....2.369}}यथैवाहारकालादेर्हेतुत्वेऽपत्यजन्मनि ।&पित्रोस्तदेकस्याकारं धत्ते नान्यस्य कस्यचित् ॥ ३६९ ॥\&[\smallbreak]


	
	  \endgroup
	

	  \pstart {\color{DodgerBlue3}“यथैवापत्यजन्मनि”} पित्रोर्माता{\color{DodgerBlue3}“पित्रोराहारकालादे”}\edlabel{pvv.229-4}\footnote{\label{pvv.229-4}  ४ आदिनाऽदृष्टादिः ।}श्च हेतुत्वेपि तदेकस्य पित्रोर्म्मध्ये एकस्य पितुर्म्मातुर्व्वा{\color{DodgerBlue3}“कार”}मपत्यं {\color{DodgerBlue3}“धत्तेऽन्यस्या”}हारकालादेश्च {\color{DodgerBlue3}“कस्यचित्”} (न) । (३६९)
	\pend
      \label{div_pvv.2.370}\edlabel{div_pvv.2.370}
	  
	% new div opening: depth here is 2
	
	  \bigskip
	  \begingroup
	  \large
	
	    
	    \stanza[\smallbreak]
	\label{pv.2.370}\edlabel{pv.2.370}\flagstanza{\tiny\textenglish{....2.370}}तद्धेतुत्वेन तुल्येपि तदन्यैर्विषये मतम् ।&विषयत्वं तदंशेन तदभावे न तद् भवेत् ॥ ३७० ॥\&[\smallbreak]


	
	  \endgroup
	

	  \pstart यत एवं {\color{DodgerBlue3}“तत्”} तस्मात् ततो विषयादन्यैरिन्द्रियादिभि{\color{DodgerBlue3}“र्हेतुत्वेन तुल्ये”} सदृशेपि विषये रूपादौ {\color{DodgerBlue3}“विषयत्वं”} ग्राह्यत्वं तेन स्वाकारार्पकत्वेनां{\color{DodgerBlue3}“शेन”} विशेषणं {\color{DodgerBlue3}“मतं”} । यस्मादा\leavevmode\marginnote{\textenglish{230/s}} कारार्पकत्वं विषयलक्षणमवस्थितं । तस्मात्तस्याकारार्पकत्वस्याभावे तद्विषयत्वमिन्द्रियादिषु {\color{DodgerBlue3}“न भवेत्”} । (३७०)
	\pend
      \label{div_pvv.2.371}\edlabel{div_pvv.2.371}
	  
	% new div opening: depth here is 2
	

	  \pstart किञ्च (।)
	\pend
      
	  \bigskip
	  \begingroup
	  \large
	
	    
	    \stanza[\smallbreak]
	\label{pv.2.371}\edlabel{pv.2.371}\flagstanza{\tiny\textenglish{....2.371}}अनर्थाकारशङ्का स्यादप्यर्थवति चेतसि ।&अतीतार्थग्रहे सिद्धे द्विरूपत्वात्मवेदने ॥ ३७१ ॥\&[\smallbreak]


	
	  \endgroup
	

	  \pstart {\color{DodgerBlue3}“अर्थवति चेत\edlabel{pvv.230-1}\footnote{\label{pvv.230-1}  १ साकारत्वात्ततोन्यं भाविकमर्थं कल्पयतः किमयं ज्ञानीय एवार्थो बहिर्व्वा येन ज्ञानमनर्थकं स्यात् ।}स्यनर्थाकार”}ता अर्थाकाररहितता शङ्का {\color{DodgerBlue3}“स्यादपि”} दृश्यमानस्याकारस्यार्थत्वेनैव सम्भाव्यमानत्वात् । {\color{DodgerBlue3}“अतीत”}स्या{\color{DodgerBlue3}“र्थस्य ग्रहे”} विकल्पात्मके त्वर्थस्या\leavevmode\marginnote{\textenglish{45a/MA}}भा\edlabel{pvv.230-2}\footnote{\label{pvv.230-2}  २ सिद्धा ज्ञानस्यैव द्व्याभासता ।} वात् । अर्थाभासतया च {\color{DodgerBlue3}“द्विरूपत्वं”} तथा चापरोक्षत्वादा{\color{DodgerBlue3}“त्मसंवेदनं”} चेति द्वे अप्यस्तु सिद्धे । न ह्यसन्नेवार्थो दृश्यते । ततोऽर्थाकारता सा ज्ञानस्यैव तथा\edlabel{pvv.230-3}\footnote{\label{pvv.230-3}  ३ बुद्धेद्विरूपत्वमात्मवेदनञ्च ते सिद्धे ।}र्थेन च परोक्षत्वाभावात् स्वाभासञ्च तत् । न चान्येन ज्ञानेन तद् वेद्यते । वेद्यवेदकभावस्य निषिद्धत्वात् । अतः स्ववेदनमेव तत् । (३७१)
	\pend
      \label{div_pvv.2.372}\edlabel{div_pvv.2.372}
	  
	% new div opening: depth here is 2
	

	  \begin{center}%% label @type='head'
	\textbf{ब. सामान्यस्य नित्यानित्यत्वप्रतिषेधः}
	\end{center}
	

	  \pstart स्यादेतद् (।) अतीता व्यक्तिरसत्त्वान्न विकल्पविषयो जातिस्यु सत्त्वात् स्यादित्याह (।)
	\pend
      
	  \bigskip
	  \begingroup
	  \large
	
	    
	    \stanza[\smallbreak]
	\label{pv.2.372}\edlabel{pv.2.372}\flagstanza{\tiny\textenglish{....2.372}}नीलाद्याभासभेदित्वान्नार्थो जातिरतद्वती ।&सा वा-नित्या न जातिः स्यान्नित्या वा जनिका कथम् ॥ ३७२ ॥\&[\smallbreak]


	
	  \endgroup
	

	  \pstart अतीतविकल्पस्य {\color{DodgerBlue3}“नीलाद्याभासभेदित्वात्”} वर्ण्णसंस्थानाद्याकारविशेषवत्त्वात् {\color{DodgerBlue3}“न जातिरतद्वती”} वर्ण्णाद्याकाररहितार्थेर्विषयः । किञ्च (।) जातिर्व्विषयीभवन्ती अनित्या वा भवेत् नित्या वा । {\color{DodgerBlue3}“सा चानित्या जातिर्न स्यात्”} नित्यलक्षणत्वात्तस्याः {\color{DodgerBlue3}“नित्या”} वा बुद्धे{\color{DodgerBlue3}“र्जनिका कथं”} स्यात् । (३७२)
	\pend
      \label{div_pvv.2.373}\edlabel{div_pvv.2.373}
	  
	% new div opening: depth here is 2
	

	  \pstart नित्यस्य क्रमयौगपद्याभ्यां विरहात् ॥ ना\edlabel{pvv.230-4}\footnote{\label{pvv.230-4}  ४ संज्ञा नाम । अर्थसरूपन्निमित्तं ।}मनिमित्ते विप्रयुक्तः संस्कारो विषयश्चेदित्याह (।)
	\pend
      
	  \bigskip
	  \begingroup
	  \large
	
	    
	    \stanza[\smallbreak]
	\label{pv.2.373}\edlabel{pv.2.373}\flagstanza{\tiny\textenglish{....2.373}}नामादिकं निषिद्धं प्राङ् नायमर्थवतां क्रमः ।&इच्छामात्रानुरोधित्वादर्थशक्तिर्न सिध्यति ॥ ३७३ ॥\&[\smallbreak]


	
	  \endgroup
	\leavevmode\marginnote{\textenglish{231/s}}

	  \pstart {\color{DodgerBlue3}“नामादिकं”} विषयत्वेन {\color{DodgerBlue3}“निषिद्धं प्राक्”} \href{http://http://sarit.indology.info/?cref=pv.2.11}{(२।११)} “नामादिवचने वक्तृश्रोतृवाच्यानुबन्धिनी” \href{http://http://sarit.indology.info/?cref=pv.2.11}{(२।११)} त्यादिना । किञ्चा{\color{DodgerBlue3}“र्थवतां”} सविषयाणां चक्षुर्व्विज्ञानादीना{\color{DodgerBlue3}“मयं क्रमो”}ऽर्थसामर्थ्येन विनोत्पाद इति नास्ति । मनोविज्ञानानां त्वर्थसन्निधानानपेक्षाणा{\color{DodgerBlue3}“मिच्छामात्रानुरोधित्वात्”} जनिका{\color{DodgerBlue3}“र्थशक्तिर्न सिध्यति”} । न चाहेतुरर्थी ग्राह्योऽतिप्रसङ्गात् । (३७३)
	\pend
      \label{div_pvv.2.374}\edlabel{div_pvv.2.374}
	  
	% new div opening: depth here is 2
	

	  \begin{center}%% label @type='head'
	\textbf{(२) स्तृतिचिन्ता}
	\end{center}
	

	  \pstart भवत्वतीतार्थलम्बनं विज्ञानं विषयाकारमनुभवस्तु मा भूदित्याह (।)
	\pend
      
	  \bigskip
	  \begingroup
	  \large
	
	    
	    \stanza[\smallbreak]
	\label{pv.2.374}\edlabel{pv.2.374}\flagstanza{\tiny\textenglish{....2.374}}स्मृतिश्चेदृग्‏विधं ज्ञानं तस्याश्चानुभवाद्भवः ।&स चार्थाकाररहितः सेदानीं तद्वती कथम् ॥ ३७४ ॥\&[\smallbreak]


	
	  \endgroup
	

	  \pstart {\color{DodgerBlue3}“ईदृग्‏विधम”}तीतविकल्पनात्मकं {\color{DodgerBlue3}“ज्ञानञ्च”} स्मृतिः । {\color{DodgerBlue3}“तस्याश्चानुभवाद् भव”} उत्पादः । न ह्यनुभवमन्तरेण स्मृतिः । {\color{DodgerBlue3}“स चा”}नुभवो भवन्मते{\color{DodgerBlue3}“ऽर्था\edlabel{pvv.231-1}\footnote{\label{pvv.231-1}  १ निराकारवादित्वात् ।}काररहितः । इदानी”}मस्मिन्नभ्युपगमे सा स्मृति{\color{DodgerBlue3}“स्तद्वत्य”}र्थाकारवती {\color{DodgerBlue3}“कथम”}स्तु । यद्यनुभवारूढो नार्थाकारः कथमसौ विदितः । अविदितस्य का स्मृतिः । (३७४)
	\pend
      \label{div_pvv.2.375}\edlabel{div_pvv.2.375}
	  
	% new div opening: depth here is 2
	

	  \begin{center}%% label @type='head'
	\textbf{क. नार्थात् स्मृतिः}
	\end{center}
	

	  \pstart अर्थात्स्मृतिरुत्पद्यत इति चेत् । आह (।)
	\pend
      
	  \bigskip
	  \begingroup
	  \large
	
	    
	    \stanza[\smallbreak]
	\label{pv.2.375}\edlabel{pv.2.375}\flagstanza{\tiny\textenglish{....2.375}}नार्थाद् भावस्तदाभावात् स्यात्तथानुभवेपि सः ।&आकारः स च नार्थस्य स्पष्टाकारविवेकतः ॥ ३७५ ॥\&[\smallbreak]


	
	  \endgroup
	

	  \pstart {\color{DodgerBlue3}“नार्थाद् भाव”}स्तस्या{\color{DodgerBlue3}“स्तदा”}\edlabel{pvv.231-2}\footnote{\label{pvv.231-2}  २ सिद्धमतः साकारं स्मरणं ।} स्मृतिकाले{\color{DodgerBlue3}“ऽभावा”}दतीतार्थस्य ।\edlabel{pvv.231-3}\footnote{\label{pvv.231-3}  ३ अभ्युपगम्य निराकारवादिनि प्रसङ्गमाह ।} यथा चार्थादुत्पद्यमानायाः स्मृतेरर्थाकारो भवति । {\color{DodgerBlue3}“तथानुभवेप्य”}र्थादुत्पद्यमाने {\color{DodgerBlue3}“सोऽर्थाकारः”} स्यात्(।) अपि च स्मृत्यारूढः {\color{DodgerBlue3}“सोऽ”}स्पष्ट{\color{DodgerBlue3}“श्चा”}कारो {\color{DodgerBlue3}“नार्थ”}स्या\edlabel{pvv.231-4}\footnote{\label{pvv.231-4}  ४ युक्तो (?) स्मृतौ नीलाद्यः ।}नुभवारूढात् {\color{DodgerBlue3}“स्पष्टादर्थाकाराद् विवेकतो”} भेदात् । (३७५)
	\pend
      \label{div_pvv.2.376}\edlabel{div_pvv.2.376}
	  
	% new div opening: depth here is 2
	
	  \bigskip
	  \begingroup
	  \large
	
	    
	    \stanza[\smallbreak]
	\label{pv.2.376}\edlabel{pv.2.376}\flagstanza{\tiny\textenglish{....2.376}}व्यतिरिक्तं तदाकारं प्रतीयादपरस्तथा ।&नित्यमात्मनि सम्बन्धे प्रतीयात् कथितञ्च न ॥ ३७६ ॥\&[\smallbreak]


	
	  \endgroup
	

	  \pstart यदि च बुद्धिव्यतिरिक्तोऽर्थ एव मनोविज्ञानग्राह्यस्तदा बुद्धे{\color{DodgerBlue3}“र्व्यतिरिक्तमे”}केन विकल्प्यमान{\color{DodgerBlue3}“न्तदाकारमपरो”}पि योग्यदेशस्थः प्रमा न प्रतीयात् । परोपलभ्यतां \leavevmode\marginnote{\textenglish{232/s}} निषेद्धुं {\color{DodgerBlue3}“नित्यमात्मन्यस्य”} विकल्प्यस्यार्थस्य {\color{DodgerBlue3}“सम्बन्धे”} वा\edlabel{pvv.232-1}\footnote{\label{pvv.232-1}  १ प्राप्यकारित्वात्कार्यस्य ।} स्वीक्रियमाणे यदा स्वयमनुचिन्तयितुं पर\edlabel{pvv.232-2}\footnote{\label{pvv.232-2}  २ किञ्चिन्तयसीति पृष्टे ।}स्मै कथ्यते तदा कथञ्चित् परो न {\color{DodgerBlue3}“प्रतीयात्”} । प्रत्येति च {\color{DodgerBlue3}“कथितं”} (३७६)
	\pend
      \label{div_pvv.2.377}\edlabel{div_pvv.2.377}
	  
	% new div opening: depth here is 2
	
	  \bigskip
	  \begingroup
	  \large
	
	    
	    \stanza[\smallbreak]
	\label{pv.2.377}\edlabel{pv.2.377}\flagstanza{\tiny\textenglish{....2.377}}एकैकेनाभिसम्बन्धे प्रतिसन्धिर्न युज्यते ।&एकार्थाभिनिवेशात्मा प्रवक्तृश्रोतृचेतसोः ॥ ३७७ ॥\&[\smallbreak]


	
	  \endgroup
	

	  \pstart प्रतीत्यर्थ{\color{DodgerBlue3}“मेकेनैकेन”} वक्त्रा श्रोत्रा च भिन्नस्य भिन्नस्य अर्थस्या{\color{DodgerBlue3}“भिसम्बन्धे-”} ऽत्रा\edlabel{pvv.232-3}\footnote{\label{pvv.232-3}  ३ सर्वः पुरुषः प्रत्यात्मैकैकसंगतः परतः श्रवणेपि स्वार्थाबाधी ।}पि स्वीक्रिमाणे {\color{DodgerBlue3}“प्रवक्तृश्रोतृचेतसोः प्रतिसन्धिरेकार्थाभिनिवे\edlabel{pvv.232-4}\footnote{\label{pvv.232-4}  ४ यमलकयोरिव ।}शात्”} यदेवानेन कथितं तदेव मया प्रतीतं यदेव मया कथितं\edlabel{pvv.232-5}\footnote{\label{pvv.232-5}  ५ किं चिन्तयसीति पृष्टे ।} तदेवानेन ज्ञातमित्यभिन्नार्थाध्यवसायरूपो {\color{DodgerBlue3}“न युज्यते”} । (३७७)
	\pend
      \label{div_pvv.2.378}\edlabel{div_pvv.2.378}
	  
	% new div opening: depth here is 2
	
	  \bigskip
	  \begingroup
	  \large
	
	    
	    \stanza[\smallbreak]
	\label{pv.2.378}\edlabel{pv.2.378}\flagstanza{\tiny\textenglish{....2.378}}तदेकव्यवहारश्चेत् सादृश्यादतदाभयोः ।&भिन्नात्मार्थः कथं ग्राह्यस्तदा स्याद्धीरनर्थिका ॥ ३७८ ॥\&[\smallbreak]


	
	  \endgroup
	

	  \pstart वक्तृश्रोतृसम्बन्धिनोस्तयोरर्थयोरे{\color{DodgerBlue3}“कव्यवहार”} एकत्वावसायः {\color{DodgerBlue3}“सादृश्याच्चेत्”} । तदुभयदर्शने सादृश्यं न चात्र दर्शनं । भवतु वा तथापि अतदाभयो{\color{DodgerBlue3}“र्भिन्नात्म-”} सम्बन्धार्थाप्रतिभासिनोर्व्वक्तृश्रोतृ चेतसो{\color{DodgerBlue3}“र्भिन्नात्माऽर्थः \edlabel{pvv.232-6}\footnote{\label{pvv.232-6}  ६ भेदेनाभानात् ।}कथं ग्राह्यः”} । यदा चार्थाभेदो नास्ति {\color{DodgerBlue3}“तदा”}ऽभिन्नार्थाध्यवसायिनी {\color{DodgerBlue3}“धी”}र्व्वक्तृश्रोत्रो{\color{DodgerBlue3}“रनर्थिका स्यात्”} । यथार्थं बुद्धेरभावात् । यथाबुद्धि चार्थाभावात् । (३७८)
	\pend
      \label{div_pvv.2.379}\edlabel{div_pvv.2.379}
	  
	% new div opening: depth here is 2
	

	  \pstart भवतु तावदेवं स्मृतिर्व्विषयाकाराऽनुभवज्ञानं त्वनाकृति स्यादित्या\edlabel{pvv.232-7}\footnote{\label{pvv.232-7}  ७ विषयज्ञाने तज्ज्ञानाद्याचष्टेनुभवविज्ञानम् ।}ह (।)
	\pend
      
	  \bigskip
	  \begingroup
	  \large
	
	    
	    \stanza[\smallbreak]
	\label{pv.2.379}\edlabel{pv.2.379}\flagstanza{\tiny\textenglish{....2.379}}तच्चानुभवविज्ञानेनुोभयांशावलंबिना ।&एकाकारविशेषेण तज्ज्ञानेनानुबुध्यते ॥ ३७९ ॥\&[\smallbreak]


	
	  \endgroup
	

	  \pstart {\color{DodgerBlue3}“तच्चानुभ\edlabel{pvv.232-8}\footnote{\label{pvv.232-8}  ८ प्राचीन ।}”}विषयेण स्मरणेनानुब \edlabel{pvv.232-9}\footnote{\label{pvv.232-9}  ९ विषयाकारता नेष्टा परेणेति द्विरूपता साध्यते । शरीरान्तर्वर्त्तिनो यथा ।}ध्यतेऽनुगम्यते स्मर्यते इति यावत् ।
	\pend
      

	  \pstart कीदृशेनो{\color{DodgerBlue3}“भयांशावलम्बिना”} ज्ञेयप्राचीनज्ञानगतविषयाकारानुभवरूपधर्मद्वयविषयेण । कथ{\color{DodgerBlue3}“मनुबध्यत”} इत्याह (।)
	\pend
      

	  \pstart {\color{DodgerBlue3}“एक”}स्मादविवादसिद्धादनुभवा{\color{DodgerBlue3}“काराद्विशेंषो”} विषयाकारस्तेन स्मरणज्ञानेन हि विज्ञानमर्याकारानुभवाकारविशिष्टमेव स्मर्यते । ततस्तादृशमेव तत् । (३७९)
	\pend
      \label{div_pvv.2.380}\edlabel{div_pvv.2.380}
	  
	% new div opening: depth here is 2
	\leavevmode\marginnote{\textenglish{233/s}}
	  \bigskip
	  \begingroup
	  \large
	
	    
	    \stanza[\smallbreak]
	\label{pv.2.380}\edlabel{pv.2.380}\flagstanza{\tiny\textenglish{....2.380}}अन्यथा ह्यतथारूपं कथं ज्ञानेधिरोहति ।&एकाकारोत्तरं ज्ञानं तथा ह्युत्तरमुत्तरम् ॥ ३८० ॥\&[\smallbreak]


	
	  \endgroup
	

	  \pstart अन्यथाऽतथारूपमाकारद्वयरहितमनुभवज्ञानं स्वग्राहिणि स्मरणज्ञाने {\color{DodgerBlue3}“कथं”} द्व्याकारमधिरोहति । अधिरूढं च । ततो द्व्याकारं तत् । तथा {\color{DodgerBlue3}“ज्ञानमुत्तरमुत्तरं”} बुद्धिपूर्व्वज्ञानालम्बन{\color{DodgerBlue3}“मेकाकारोत्तर”}मेकेनैकेनाकारेणाधिकं प्रतीयते । अर्थेज्ञानेन तदालम्बक एकोऽर्थाकारः प्रतीयते । तदालम्बनेन\edlabel{pvv.233-1}\footnote{\label{pvv.233-1}  १ प्रथमज्ञानस्य यौ बाह्यग्राहकाकारौ । तौ द्वितीयस्य ज्ञानस्य ग्रह्याकारौ जातौ ।}तु विषयभूतज्ञानाकारस्तदा-\leavevmode\marginnote{\textenglish{45b/MA}} लम्बकश्च स्वाकारः । (३८०)
	\pend
      \label{div_pvv.2.381}\edlabel{div_pvv.2.381}
	  
	% new div opening: depth here is 2
	
	  \bigskip
	  \begingroup
	  \large
	
	    
	    \stanza[\smallbreak]
	\label{pv.2.381}\edlabel{pv.2.381}\flagstanza{\tiny\textenglish{....2.381}}तस्यार्थरूपेणाकारावात्माकारश्च कश्चन ।&द्वितीयस्य तृतीयेन ज्ञानेन हि विविच्यते ॥ ३८१ ॥\&[\smallbreak]


	
	  \endgroup
	

	  \pstart यस्मा{\color{DodgerBlue3}“त्तस्य”} द्वितीयस्य ज्ञानस्य तौ द्वावाकाराव{\color{DodgerBlue3}“र्थरूपेण”} विषयभावेन तदालम्बक {\color{DodgerBlue3}“आत्माकारश्च कश्चन”} तृतीयेन द्वितीयज्ञानालम्बकेन हि यस्माद् विवेच्यते अवधार्यते । तस्मादर्थाकारं स्ववेदनञ्च ज्ञानमभ्युपगन्तव्यं । (३८१)
	\pend
      \label{div_pvv.2.382}\edlabel{div_pvv.2.382}
	  
	% new div opening: depth here is 2
	
	  \bigskip
	  \begingroup
	  \large
	
	    
	    \stanza[\smallbreak]
	\label{pv.2.382a}\edlabel{pv.2.382a}\flagstanza{\tiny\textenglish{...2.382a}}अर्थकार्यतया ज्ञानस्मृतावर्थस्मृतेर्यदि ।&भ्रान्त्या सङ्कलनं;\&[\smallbreak]


	
	  \endgroup
	

	  \pstart अ\edlabel{pvv.233-2}\footnote{\label{pvv.233-2}  २ स्पष्टेप्यर्थे बाह्याभिनिवेशत्यागार्थ परप्रक्रियां दूषयितुं परमतमुपभिपति ।}थ निराकारमेव {\color{DodgerBlue3}“ज्ञानम”}र्थस्य कार्यमनुभवरूप{\color{DodgerBlue3}“मर्थकार्यतया”} ज्ञात\edlabel{pvv.233-3}\footnote{\label{pvv.233-3}  ३ कार्यभूते विषयज्ञाने या स्मृतिः ।} {\color{DodgerBlue3}“स्मृतौ”} विशेषणत्वेना\edlabel{pvv.233-4}\footnote{\label{pvv.233-4}  ४ तत्कारणस्य स्मूतिर्भवति ततः ।} {\color{DodgerBlue3}“र्थस्य स्मृतेः”} स्मर्यमाणस्यार्थस्य {\color{DodgerBlue3}“भ्रान्त्या”} ज्ञानात्म\edlabel{pvv.233-5}\footnote{\label{pvv.233-5}  ५ निराकारेपि साकारवतः ।}नि {\color{DodgerBlue3}“संकलनं”} सम्ब(न्ध)नं यदि स्यात्तदा को दोषः ।
	\pend
      

	  \pstart आह ।
	\pend
      
	  \bigskip
	  \begingroup
	  \large
	
	    
	    \stanza[\smallbreak]
	\label{pv.2.382b}\edlabel{pv.2.382b}\flagstanza{\tiny\textenglish{...2.382b}}ज्योतिर्मनस्कारे च सा भवेत् ॥ ३८२ ॥\&[\smallbreak]


	
	  \endgroup
	

	  \pstart अर्थवत् ज्योतिष आलोकस्य मनस्कारस्य च ज्ञानहेतुत्वात् । तत्कार्यज्ञाने स्मर्यमाणे सा स्मृति{\color{DodgerBlue3}“र्ज्योतिर्मनस्कारे”} आलोकसमनन्तरप्रत्ययोरपि स्यात् । तत्र भ्रान्त्या अर्थाकारमिवालोकमनस्काराकारमपि संकलनीयं । न त्वर्थाकारमेव नियमेन संकलयितुं युक्तं । (३८२)
	\pend
      \label{div_pvv.2.383}\edlabel{div_pvv.2.383}
	  
	% new div opening: depth here is 2
	
	  \bigskip
	  \begingroup
	  \large
	
	    
	    \stanza[\smallbreak]
	\label{pv.2.383}\edlabel{pv.2.383}\flagstanza{\tiny\textenglish{....2.383}}सर्व्वेषामपि कार्य्याणां कारणैः स्यात्तथा ग्रहः ।&कुलालादिविवेकेन न स्मर्येत घटस्ततः ॥ ३८३ ॥\&[\smallbreak]


	
	  \endgroup
	\leavevmode\marginnote{\textenglish{234/s}}

	  \pstart यदि चा\edlabel{pvv.234-1}\footnote{\label{pvv.234-1}  १ इत्यर्थाकारसमावेशितज्ञानम् ।}र्थकार्यं भ्रान्त्या स्मर्यते तदा {\color{DodgerBlue3}“सर्व्वेषामपि कार्य्याणां कारणैः”} सह {\color{DodgerBlue3}“तथा ग्रहः”} कारणात्मत्वेन ग्रहणं {\color{DodgerBlue3}“स्यात्”} । ततश्च धटकुलालादिकार्यं {\color{DodgerBlue3}“कुलालादेर्व्विवेकेन”} भेदेन {\color{DodgerBlue3}“न स्मर्येत”} । (३८३)
	\pend
      \label{div_pvv.2.384}\edlabel{div_pvv.2.384}
	  
	% new div opening: depth here is 2
	

	  \pstart अथास्त्येव कश्चिदालोकादिभ्यो विषयस्य ज्ञानात्मन्यारोपणीयो विशेषस्ततस्तदाकारावग्रहेण स्मर्यते । नान्यथेत्याह (।)
	\pend
      
	  \bigskip
	  \begingroup
	  \large
	
	    
	    \stanza[\smallbreak]
	\label{pv.2.384}\edlabel{pv.2.384}\flagstanza{\tiny\textenglish{....2.384}}यस्मादतिशयाज्‏ज्ञानमर्थसंसर्गभाजनम् ।&सारूप्यात्तत् किमन्यत् स्याद् दृष्टेश्च यमलादिषु ॥ ३८४ ॥\&[\smallbreak]


	
	  \endgroup
	

	  \pstart {\color{DodgerBlue3}“यस्माद”}र्थारोपिताद{\color{DodgerBlue3}“तिशयात् ज्ञानमर्थसंसर्ग”}स्यार्थसंश्लेषस्य {\color{DodgerBlue3}“भाजनं”} पात्रं भवति । अर्थेन सारूप्या{\color{DodgerBlue3}“दन्यत् किन्तत् स्यात्”} । सारूप्यादन्यस्यातिशयस्योपलक्षणत्वायोगात् । {\color{DodgerBlue3}“यमलादिषु”} सारूप्याद् भ्रान्त्या तथार्थनिश्चयस्य {\color{DodgerBlue3}“दृष्टेश्च”} । (३८४)
	\pend
      \label{div_pvv.2.385}\edlabel{div_pvv.2.385}
	  
	% new div opening: depth here is 2
	

	  \pstart किञ्च ।
	\pend
      
	  \bigskip
	  \begingroup
	  \large
	
	    
	    \stanza[\smallbreak]
	\label{pv.2.385}\edlabel{pv.2.385}\flagstanza{\tiny\textenglish{....2.385}}आद्यानुभयरूपत्वे ह्येकरूपं व्यवस्थितम् ।&द्वितीयं व्यतिरिच्येत न परामर्शचेतसा ॥३८५ ॥\&[\smallbreak]


	
	  \endgroup
	

	  \pstart {\color{DodgerBlue3}“आद्य”}स्यार्थज्ञानस्या{\color{DodgerBlue3}“नुभयरूपत्वे”}ऽर्थाकाररहितत्वात् अनुभवैकरूपत्वे तदालम्बकं {\color{DodgerBlue3}“द्वितीयं”} ज्ञान{\color{DodgerBlue3}“मेक”}स्मिन् {\color{DodgerBlue3}“रू\edlabel{pvv.234-2}\footnote{\label{pvv.234-2}  २ विषयज्ञानतज्ज्ञानविशेषात्तु द्विरूपतेति व्याख्याय स्वयमुपपत्तिमाह (।) अद्विरूपं विषयाभासाभावात् ।}पे”}ऽनुभवात्मनि {\color{DodgerBlue3}“व्यवस्थितं परामर्शचेतसा”} ज्ञानज्ञानालम्बकेन तृतीयज्ञानेन न व्यतिरिच्येत । न विषयज्ञानग्राहकतया भेदेन गृह्येत । स्वाभासमात्रत्वेन सर्व्वस्याविशेषात् । (३८५)
	\pend
      \label{div_pvv.2.386}\edlabel{div_pvv.2.386}
	  
	% new div opening: depth here is 2
	
	  \bigskip
	  \begingroup
	  \large
	
	    
	    \stanza[\smallbreak]
	\label{pv.2.386}\edlabel{pv.2.386}\flagstanza{\tiny\textenglish{....2.386}}अर्थसंकलनाश्लेषा धीर्द्वितीयावलम्बते ।&नीलादिरूपेण धियं भासमानां पुरस्ततः ॥ ३८६ ॥\&[\smallbreak]


	
	  \endgroup
	

	  \pstart यतो बुद्धेरनाकारत्वे दोषोऽयं {\color{DodgerBlue3}“ततः”} पु{\color{DodgerBlue3}“रोऽर्थस्य”} धियं {\color{DodgerBlue3}“नीलादिरूपेण भासमानां द्वितीया धीरर्थसंकल”}नस्यार्थाकारावग्रहस्या{\color{DodgerBlue3}“श्लेषः”} संसर्ग्गो यस्याः सा ताम{\color{DodgerBlue3}“वलम्ब”}ते । (३८६)
	\pend
      \label{div_pvv.2.387}\edlabel{div_pvv.2.387}
	  
	% new div opening: depth here is 2
	
	  \bigskip
	  \begingroup
	  \large
	
	    
	    \stanza[\smallbreak]
	\label{pv.2.387}\edlabel{pv.2.387}\flagstanza{\tiny\textenglish{....2.387}}अन्यथा ह्याद्यमेवैकं संयोज्येतार्थसम्भवात् ।&ज्ञानं नादृष्टसम्बन्धं पूर्व्वार्थेनोत्तरोत्तरम् ॥ ३८७ ॥\&[\smallbreak]


	
	  \endgroup
	

	  \pstart {\color{DodgerBlue3}“अन्यथा”} यद्यनाकारं ज्ञानमर्थकार्यतया भ्रान्त्याऽर्थाकारं स्मर्यत इत्याश्रीयते त\edlabel{pvv.234-3}\footnote{\label{pvv.234-3}  ३ न विषयाकारोपधानात् ।} {\color{DodgerBlue3}“दाद्यमेवैक”}मर्थज्ञानं स्मृत्याऽर्थेन {\color{DodgerBlue3}“संयोज्येत । अर्थात्संभवात्”} तस्य । न {\color{DodgerBlue3}“उत्तरोत्तरं ज्ञानं पूर्व्वस्य”} ज्ञानस्या{\color{DodgerBlue3}“र्थेना”}कारणभूतेना{\color{DodgerBlue3}“दृष्टसम्बन्धं”} संयोज्येत ।
	\pend
      \leavevmode\marginnote{\textenglish{235/s}}

	  \pstart तस्मात्स्थितमेतत् ज्ञानानां विषयसारूप्यानुभवरूपत्वाभ्यां द्व्याकारत्वं । ततश्च सहोपलम्भनियमोऽर्थविज्ञानयोः । अर्थाकारताया अर्थोपलम्भात् । अनुभवरूपतायाः स्व{\color{DodgerBlue3}“भे”} (?वे) दनत्वात् ।(३८७)
	\pend
      \label{div_pvv.2.388}\edlabel{div_pvv.2.388}
	  
	% new div opening: depth here is 2
	

	  \pstart तथा च (।)
	\pend
      
	  \bigskip
	  \begingroup
	  \large
	
	    
	    \stanza[\smallbreak]
	\label{pv.2.388}\edlabel{pv.2.388}\flagstanza{\tiny\textenglish{....2.388}}सकृत् सम्वेद्यमानस्य नियमेन धिया सह ।&विषयस्य ततोऽन्यत्वं केनाकारेण सिध्यति ॥ ३८८ ॥\&[\smallbreak]


	
	  \endgroup
	

	  \pstart {\color{DodgerBlue3}“धिया सह नियमेन सकृत्संवेद्यमानस्य विषयस्य ततो”} धियो{\color{DodgerBlue3}“ऽन्यत्वं”} भेदः {\color{DodgerBlue3}“केनाकारे”}ण प्रकारेण {\color{DodgerBlue3}“सिध्यति”} ।\edlabel{pvv.235-1}\footnote{\label{pvv.235-1}  १ विषयस्याभावात्तदभेदो न साध्यः किन्तु बुद्धिरेव तदात्मिका साध्यते ।}भिन्नयोः सहोपलम्भनियमायोगात् । (३८८)
	\pend
      \label{div_pvv.2.389}\edlabel{div_pvv.2.389}
	  
	% new div opening: depth here is 2
	

	  \pstart यदि विषयज्ञानयोरभेदस्तदा ग्राह्यग्राहकतया भेदः कथं प्रतिभातीत्याह (।)
	\pend
      
	  \bigskip
	  \begingroup
	  \large
	
	    
	    \stanza[\smallbreak]
	\label{pv.2.389a}\edlabel{pv.2.389a}\flagstanza{\tiny\textenglish{...2.389a}}भेदश्च भ्रान्तविज्ञानैर्दृश्येतेन्दाविवाद्वये ।\&[\smallbreak]


	
	  \endgroup
	

	  \pstart {\color{DodgerBlue3}“भेदश्च”} वासनावशात् {\color{DodgerBlue3}“भ्रान्त”}मुपप्लुताकारं {\color{DodgerBlue3}“ज्ञानं”} येषां तैरर्वाग्‏दर्शिभि{\color{DodgerBlue3}“र्दृश्येत । इन्दाविवाद्व”}ये एकरूपे द्वैतं तिमिरोपहतबुद्धिभिः ।
	\pend
      

	  \pstart भेदेपि कस्भान्न सहोपलम्भनियम इत्याह (।)
	\pend
      
	  \bigskip
	  \begingroup
	  \large
	
	    
	    \stanza[\smallbreak]
	\label{pv.2.389b}\edlabel{pv.2.389b}\flagstanza{\tiny\textenglish{...2.389b}}संवित्तिनियमो नास्ति भिन्नयोर्नीलपीतयोः ॥ ३८९ ॥\&[\smallbreak]


	
	  \endgroup
	

	  \pstart भिन्नयोर्नीलपीतयोः {\color{DodgerBlue3}“संवित्तिनियमो नास्ति”} । ततो यत्रास्ति तत्राभेदएव (। ३८९)
	\pend
      \label{div_pvv.2.390_2.391_2.392}\edlabel{div_pvv.2.390_2.391_2.392}
	  
	% new div opening: depth here is 2
	

	  \pstart तथा हि (।)
	\pend
      
	  \bigskip
	  \begingroup
	  \large
	
	    
	    \stanza[\smallbreak]
	\label{pv.2.390}\edlabel{pv.2.390}\flagstanza{\tiny\textenglish{....2.390}}नार्थोऽसम्वेदनः कश्चिदनर्थम्वापि वेदनम् ।&दृष्टं सम्वेद्यमानन्तत् तयोर्नास्ति विवेकिता ॥ ३९० ॥\&[\smallbreak]


	
	  \endgroup
	
	  \bigskip
	  \begingroup
	  \large
	
	    
	    \stanza[\smallbreak]
	\label{pv.2.391a}\edlabel{pv.2.391a}\flagstanza{\tiny\textenglish{...2.391a}}तस्मादर्थस्य दुर्वारं ज्ञानकालावभासिनः ।&ज्ञानादव्यतिरेकित्वं;\&[\smallbreak]


	
	  \endgroup
	

	  \pstart नार्थोनुभवमन्तरेण कश्चिद् दृष्टः (।) वेदनञ्चार्थाकारम्विना न दृष्टं सम्वेद्यमानं । तत्तस्मात्तयोरर्थतदुपलम्भयोर्नास्ति विवेधिता । (३९०)तस्मात् ज्ञानकालावभासिनोर्थस्य । अर्थज्ञानयोरपि सहसंवित्तिज्ञानादव्यतिरेकित्वमभिन्न\edlabel{pvv.235-2}\footnote{\label{pvv.235-2}  २ तादात्म्यप्रतिबन्ध उक्तः ।}त्वं दुर्व्वारमित्युपसंहारः ।
	\pend
      

	  \pstart स्यादेतत् ।
	\pend
      
	  \bigskip
	  \begingroup
	  \large
	
	    
	    \stanza[\smallbreak]
	\label{pv.2.391b}\edlabel{pv.2.391b}\flagstanza{\tiny\textenglish{...2.391b}}हेतुभेदानुमा भवेत् ॥ ३९१ ॥\&[\smallbreak]


	
	  \endgroup
	\leavevmode\marginnote{\textenglish{236/s}}
	  \bigskip
	  \begingroup
	  \large
	
	    
	    \stanza[\smallbreak]
	\label{pv.2.392}\edlabel{pv.2.392}\flagstanza{\tiny\textenglish{....2.392}}अभावादक्षबुद्धीनां सत्स्वप्यन्येषु हेतुषु ।&नियमं यदि न ब्रूयाद प्रत्ययात् समनन्तरात् ॥ ३९२ ॥\&[\smallbreak]


	
	  \endgroup
	\leavevmode\marginnote{\textenglish{46a/MA}}

	  \pstart {\color{DodgerBlue3}“सत्स्वप्यन्येष्वि”}न्द्रियादिषु {\color{DodgerBlue3}“हेतुष्वक्षबुद्धीनामभावात्”} । हेतुभेदस्य तदतिरिक्तस्य कारणविशेषस्यानुमा भवेत् । कारणसाकल्ये सति तन्मात्रसाध्यस्य कार्यस्यानुत्पादायोगात् । यश्चासौ कारणभेदः स बाह्योऽर्थो भविष्ययीति मन्यते परः । एवमप्यनुमानगम्यो बहिरर्थो भवेन्न प्रत्यक्षो यथेष्यते । किन्तु व्यतिरेकसामर्थ्यादपि तदाऽर्थःसिध्येत । {\color{DodgerBlue3}“यदि”} यो गा चा रो नानार्थप्रतिभासिनीनां धियामुत्पादक्रमस्य {\color{DodgerBlue3}“नियमं”}यथाप्रत्ययं प्रबुद्धवासनागर्भात् {\color{DodgerBlue3}“समनन्तरप्रत्ययान्न ब्रूयात्”} । (३९१, ३९२)
	\pend
      \label{div_pvv.2.393}\edlabel{div_pvv.2.393}
	  
	% new div opening: depth here is 2
	
	  \bigskip
	  \begingroup
	  \large
	
	    
	    \stanza[\smallbreak]
	\label{pv.2.393}\edlabel{pv.2.393}\flagstanza{\tiny\textenglish{....2.393}}बीजादंकुरजन्मान्नेर्धूमात् सिद्धिरितीदृशी ।&बाह्यार्थाश्रयिणी यापि कारकज्ञापकस्थितिः ॥ ३९३ ॥\&[\smallbreak]


	
	  \endgroup
	

	  \pstart ननु बाह्यार्था\edlabel{pvv.236-1}\footnote{\label{pvv.236-1}  १ “विद्यमानेपि बाह्येर्थे यथानुभवमेव वे” \cref{pv.2.341} त्यादिना “यदा तु बाह्य एवार्थः प्रमेय” इत्यादि \href{http://http://sarit.indology.info/?cref=psv.1.9}{[प्रमाण] समुच्चयस्य तृतीयः फलविकल्पो व्यख्यातः} ।} भावे {\color{DodgerBlue3}“बीजादङ्कुरस्य जन्मे”}तीदृशी एवंजातीया यापि प्रतीतिसिद्धा कारकस्थितिः । {\color{DodgerBlue3}“धूमात्”} कार्यात्कारणस्याग्नेः {\color{DodgerBlue3}“सिद्धिरितीदृशी”} यापि {\color{DodgerBlue3}“ज्ञापकहेतुस्थितिः”} तदुच्छेदः स्यात् । हेतुफलभावाश्रयस्य बाह्यस्यैवाभावात् । (३९३)
	\pend
      \label{div_pvv.2.394_2.395_2.396}\edlabel{div_pvv.2.394_2.395_2.396}
	  
	% new div opening: depth here is 2
	

	  \pstart अत्राह\edlabel{pvv.236-2}\footnote{\label{pvv.236-2}  २ विज्ञप्तौ कार्यकारणत्वस्थापनाय प्रत्यक्षानुपलम्भसाधनं हेतुफलभावं परमाशङ्कते ।} (।)
	\pend
      
	  \bigskip
	  \begingroup
	  \large
	
	    
	    \stanza[\smallbreak]
	\label{pv.2.394}\edlabel{pv.2.394}\flagstanza{\tiny\textenglish{....2.394}}सापि तद्रूपनिर्भासा तथा नियतसङ्गमाः ।&बुद्धीराश्रित्य कल्प्येत यदि किं वा विरुध्यते ॥ ३९४ ॥\&[\smallbreak]


	
	  \endgroup
	

	  \pstart {\color{DodgerBlue3}“सा”} कारकज्ञापकस्थिति{\color{DodgerBlue3}“रपि तद्रूपनिर्भा\edlabel{pvv.236-3}\footnote{\label{pvv.236-3}  ३ बुद्ध्योर्वस्तुत्वान्नावस्तुकं कार्यकारणत्वं ।}सा”} बीजाङ्कुरधूमाग्निप्रतिभासवासनाप्रतिनियमात् । {\color{DodgerBlue3}“तथा”} क्रम\edlabel{pvv.236-4}\footnote{\label{pvv.236-4}  ४ यथा बीजादङ्कुरजन्मनि एवमेव बीजबुद्ध्यनन्तरमङ्कुरबुद्धिः ।}विशेषेण {\color{DodgerBlue3}“नियतः सङ्गम”} उत्पादो यासां ता {\color{DodgerBlue3}“बुद्धीरा\edlabel{pvv.236-5}\footnote{\label{pvv.236-5}  ५ न हि वस्तु केनचिद् गम्यते स्वयमात्मानं गमयति ज्ञानप्रतिभासस्यैव वेदनात् ।}श्रित्य”} यदि {\color{DodgerBlue3}“कल्प्यते”} तदा {\color{DodgerBlue3}“किम्वा विरुध्येत”} न किञ्चित् । हि बीजप्रतिभासं ज्ञानं स्वहेतोः प्रबुद्धाङ्कुरज्ञानवासनापाटवमङ्कुरज्ञानं जनयति । एवं धूमज्ञानमग्निज्ञानमुत्पादयति । तावतैव च ज्ञापकव्यवस्थाया अविरोधः । (३९४)
	\pend
      \leavevmode\marginnote{\textenglish{237/s}}

	  \pstart नन्वस्ति विरोधः । तथा हि (।)
	\pend
      
	  \bigskip
	  \begingroup
	  \large
	
	    
	    \stanza[\smallbreak]
	\label{pv.2.395}\edlabel{pv.2.395}\flagstanza{\tiny\textenglish{....2.395}}अनग्निजन्यो धूमः स्यात् तत्कार्यात् कारणेऽगतिः ।&न स्यात् कारणतायां वा कुत एकान्ततो गतिः ॥ ३९५ ॥\&[\smallbreak]


	
	  \endgroup
	

	  \pstart धूमज्ञानादग्निज्ञानोत्पादे{\color{DodgerBlue3}“ऽनग्निजन्यो धूमः स्यात्”} । अग्निप्रतिभासस्य प्रागविद्यमानत्वात् विपर्ययः स्यात् । {\color{DodgerBlue3}“तत्”} तस्मा{\color{DodgerBlue3}“त्कार्यात् कारणे गतिर्न स्यात्”} । अग्निज्ञानं प्रति धूमज्ञानस्य {\color{DodgerBlue3}“कारणतायां वा”} कारणात् कार्य\edlabel{pvv.237-1}\footnote{\label{pvv.237-1}  १ अनुमातव्ये ।} {\color{DodgerBlue3}“एकान्त\edlabel{pvv.237-2}\footnote{\label{pvv.237-2}  २ नावश्यं कारणानि कार्यवन्ति स्युः ।}तो”}ऽसंदिग्धा {\color{DodgerBlue3}“कुतो गति”}रिति (३९५)
	\pend
      

	  \pstart अत्राह (।)
	\pend
      
	  \bigskip
	  \begingroup
	  \large
	
	    
	    \stanza[\smallbreak]
	\label{pv.2.396}\edlabel{pv.2.396}\flagstanza{\tiny\textenglish{....2.396}}तत्रापि धूमाभासा धीः प्रबोधपटुवासनाम् ।&गमयेदग्निनिर्भासान्धियमेव न पावकम् ॥३९६ ॥\&[\smallbreak]


	
	  \endgroup
	

	  \pstart {\color{DodgerBlue3}“तत्र”} धूमादग्न्यनुमा{\color{DodgerBlue3}“नेपि धूमाभासा धीर”}ग्निवासनाप्रतिबद्धा {\color{DodgerBlue3}“एकसामगग्रय-”} धीनतया{\color{DodgerBlue3}“ऽग्निनिर्भासान्धियमेव”} धूमज्ञानादेव {\color{DodgerBlue3}“प्रबोधेन पटुज”}ननोन्मुखा {\color{DodgerBlue3}“वासना”} शक्तिर्यस्यास्ता{\color{DodgerBlue3}“ङ्गमयेत् । न पा\edlabel{pvv.237-3}\footnote{\label{pvv.237-3}  ३ येन लिङ्गबोधकाले धूमभासज्ञानस्यानग्निजन्यत्वं स्यात् ।}वकं बाह्यरूपं”} सर्व्वदाऽदर्शनात् । (३९६)
	\pend
      \label{div_pvv.2.397}\edlabel{div_pvv.2.397}
	  
	% new div opening: depth here is 2
	

	  \pstart अग्निवासनाधूमज्ञानयोर्हेतुफलतामाख्यातुमाह (।)
	\pend
      
	  \bigskip
	  \begingroup
	  \large
	
	    
	    \stanza[\smallbreak]
	\label{pv.2.397}\edlabel{pv.2.397}\flagstanza{\tiny\textenglish{....2.397}}तद्योग्यवासनागर्भ एव धूमावभासिनीम् ।&व्यनक्ति चित्तसन्तानो धियं धूमोग्नितस्ततः ॥ ३९७ ॥\&[\smallbreak]


	
	  \endgroup
	

	  \pstart {\color{DodgerBlue3}“त”}स्याग्निप्रतिभासस्य {\color{DodgerBlue3}“योग्या”} जननसमर्था {\color{DodgerBlue3}“वासनागर्भे”} स्वभावभूता यस्य चित्तसन्तानस्य स {\color{DodgerBlue3}“चित्तंसन्तानो धूमावभासिनीं धियं व्यनक्ति”} उत्पादयति(।) {\color{DodgerBlue3}“त\edlabel{pvv.237-4}\footnote{\label{pvv.237-4}  ४ बाह्यवादिनोपि तुल्यमेतद्येन वह्निना धूमो जनितः कथं तदनुमानं यश्च भावी तेनासौ न जनित इति भाव्यनुमीयते ।}तोऽग्नित”} एव {\color{DodgerBlue3}“धूमो”} भवतीति न कार्यकारणताविपर्ययः । न च कारणात्कार्यानुमानमग्निवासनाप्रभवत्वात् धूमाग्निज्ञानयोः । धूमज्ञानात् प्रबुद्धाग्निवासनाद्वारेणाग्निज्ञानानुमितिरेकसामग्रयधीना । (३९७)
	\pend
      \label{div_pvv.2.398}\edlabel{div_pvv.2.398}
	  
	% new div opening: depth here is 2
	

	  \begin{center}%% label @type='head'
	\textbf{(ख. ज्ञानद्वयरूपतासिद्धि)}
	\end{center}
	

	  \pstart a. एवन्तर्हि विज्ञाननय एव सर्व्वव्यवस्था नमविरोधात् कथमाचार्येण बहिरर्थापेक्षया ज्ञानद्विरूपतोक्तेत्याह (।)
	\pend
      \leavevmode\marginnote{\textenglish{238/s}}
	  \bigskip
	  \begingroup
	  \large
	
	    
	    \stanza[\smallbreak]
	\label{pv.2.398}\edlabel{pv.2.398}\flagstanza{\tiny\textenglish{....2.398}}अस्त्येष विदुषां वादो बाह्यन्त्वाश्रित्य वर्ण्यते ।&द्वैरूप्यं सहसंवित्तिनियमात्तच्च सिध्यति ॥ ३९८ ॥\&[\smallbreak]


	
	  \endgroup
	

	  \pstart {\color{DodgerBlue3}“अस्त्येष”} सर्व्वव्यवस्थासु विज्ञप्तिमात्रताप्रतिपादको {\color{DodgerBlue3}“विदुषां”} न्यायदर्शिनां {\color{DodgerBlue3}“यो गा”} चा रा णां {\color{DodgerBlue3}“वादः”} । सौ त्रा न्ति कै रिष्टं {\color{DodgerBlue3}“बाह्य”}मर्थमा{\color{DodgerBlue3}“श्रित्य”} ज्ञानस्य द्वैरूप्यमा{\color{DodgerBlue3}“चार्येण वर्ण्यते”} । तच्च {\color{DodgerBlue3}“द्वैरूप्यं”} सहसम्वेदन{\color{DodgerBlue3}“नियमात्”} सहोपलम्भनियमात् {\color{DodgerBlue3}“सिध्यति”} । (३९८) भेदेपि सति तदभावात् ।
	\pend
      \label{div_pvv.2.399}\edlabel{div_pvv.2.399}
	  
	% new div opening: depth here is 2
	

	  \pstart b. द्वैरूप्यसिद्धावुपपत्त्यन्तरं वक्तुमाह (।)
	\pend
      
	  \bigskip
	  \begingroup
	  \large
	
	    
	    \stanza[\smallbreak]
	\label{pv.2.399}\edlabel{pv.2.399}\flagstanza{\tiny\textenglish{....2.399}}ज्ञानमिन्द्रियभेदेन पटुमन्दाविलादिकाम् ।&प्रतिभासभिदामर्थे विभ्रदेकत्र दृश्यते ॥ ३९९ ॥\&[\smallbreak]


	
	  \endgroup
	

	  \pstart इन्द्रियस्य {\color{DodgerBlue3}“भेदेन”} प्रसादोपघातादिना विशेषेण पुरु{\color{DodgerBlue3}“षार्थे ज्ञानं पटुमन्दा”}विलादिकां {\color{DodgerBlue3}“प्रतिभासभिदां विभ्रत्”} दधत् {\color{DodgerBlue3}“दृश्यते”} । (३९९)
	\pend
      \label{div_pvv.2.400}\edlabel{div_pvv.2.400}
	  
	% new div opening: depth here is 2
	
	  \bigskip
	  \begingroup
	  \large
	
	    
	    \stanza[\smallbreak]
	\label{pv.2.400}\edlabel{pv.2.400}\flagstanza{\tiny\textenglish{....2.400}}अर्थस्याभिन्नरूपत्वादेकरूपं भवेन्मनः ।&सर्वं तदर्थमर्थाच्चेत् तस्य नास्ति तदाभता ॥ ४०० ॥\&[\smallbreak]


	
	  \endgroup
	

	  \pstart {\color{DodgerBlue3}“तस्य”} ज्ञानस्यार्था{\color{DodgerBlue3}“च्चेन्नास्ति तदाभताऽ”}\edlabel{pvv.238-1}\footnote{\label{pvv.238-1}  १ निराकारत्वात् ।} र्थाकारता तदा{\color{DodgerBlue3}“र्थस्याभिन्नरूप\edlabel{pvv.238-2}\footnote{\label{pvv.238-2}  २ एकरूप एवार्थः साकारः ।} त्वात् । तदर्थ”} तद्विषयं {\color{DodgerBlue3}“सर्व्वं मनो”}वेदन{\color{DodgerBlue3}“मेकरूपम्भवेत्”} । न पटुमन्दाविलतादिभिन्नं ज्ञानस्य स्वगताकारभेदानभ्युपगमात् । अर्थस्यैकरूपत्वात् प्रतिभास\edlabel{pvv.238-3}\footnote{\label{pvv.238-3}  ३ एकस्मिन्नर्थे नानाकारः ।}भेदविरोधात् । अस्ति \leavevmode\marginnote{\textenglish{46b/MA}} चायं तस्मादर्थरूपताऽनुभवरूपता चेति द्वैरूप्यसिद्धिः । (४००)
	\pend
      \label{div_pvv.2.401}\edlabel{div_pvv.2.401}
	  
	% new div opening: depth here is 2
	

	  \pstart नन्वर्थरूपतायामप्यर्थस्यैकरूपत्वात् तत्सरूपं ज्ञानमेकाकारं स्यात् । न प्रसन्नाविलादिरूपमित्याह (।)
	\pend
      
	  \bigskip
	  \begingroup
	  \large
	
	    
	    \stanza[\smallbreak]
	\label{pv.2.401}\edlabel{pv.2.401}\flagstanza{\tiny\textenglish{....2.401}}अर्थाश्रयेणोद्भवतस्तद्रूपमनुकुर्व्वतः ।&तस्य केनचिदंशेन परतोपि भिदा भवेत् ॥ ४०१ ॥\&[\smallbreak]


	
	  \endgroup
	

	  \pstart {\color{DodgerBlue3}“अर्थ”}स्य सरूपस्या{\color{DodgerBlue3}“श्रयेणोद्भवतस्तस्य”} ज्ञानस्य {\color{DodgerBlue3}“तद्रूपम”}र्थाकार{\color{DodgerBlue3}“मनुकुर्व्वतः केनचिदंशेना”}कारेण पटुमन्दत्वादिना {\color{DodgerBlue3}“परतो”} वासनादेरपि कारणाद्\edlabel{pvv.238-4}\footnote{\label{pvv.238-4}  ४ यस्मिन् सति यत्स्यात् येन विना न भवति तदान्येषु सत्स्वपि तत्तस्य कारणं ।} {\color{DodgerBlue3}“भिदा भवेत्”} । (४०१)
	\pend
      \label{div_pvv.2.402}\edlabel{div_pvv.2.402}
	  
	% new div opening: depth here is 2
	\leavevmode\marginnote{\textenglish{239/s}}
	  \bigskip
	  \begingroup
	  \large
	
	    
	    \stanza[\smallbreak]
	\label{pv.2.402}\edlabel{pv.2.402}\flagstanza{\tiny\textenglish{....2.402}}तथा ह्याश्रित्य पितरं तद्रूपो हि सुतः पितुः ।&भेदं केनचिदंशेन कुतश्चिदवलम्बते ॥ ४०२ ॥\&[\smallbreak]


	
	  \endgroup
	

	  \pstart {\color{DodgerBlue3}“तथा हि पितरमाश्रित्य तद्रूपः”} पित्राकारोपि {\color{DodgerBlue3}“सुत”} उत्पन्नः {\color{DodgerBlue3}“केनचिदंशेनाकारेण कुतश्चित्”} कर्म्मादेर्हेतोः {\color{DodgerBlue3}“पितुः”} शकासात् (? सकाशात्) {\color{DodgerBlue3}“भेदम”}न्यादृशत्व{\color{DodgerBlue3}“मवलम्वते पि”}तापुत्रयोः सर्व्वथा साम्याभावात् । (४०२)
	\pend
      \label{div_pvv.2.403}\edlabel{div_pvv.2.403}
	  
	% new div opening: depth here is 2
	

	  \pstart C. द्वैरूप्यसिद्धावुपपत्त्यन्तरमाह (।)
	\pend
      
	  \bigskip
	  \begingroup
	  \large
	
	    
	    \stanza[\smallbreak]
	\label{pv.2.403}\edlabel{pv.2.403}\flagstanza{\tiny\textenglish{....2.403}}मयूरचन्द्रकाकारं नीललोहितभास्वरम् ।&सम्पश्यन्ति प्रदीपादेर्म्मण्डलं मन्दचक्षुषः ॥ ४०३ ॥\&[\smallbreak]


	
	  \endgroup
	

	  \pstart {\color{DodgerBlue3}“मयूरचन्द्रकाकारमन्तरान्तरा नीललोहितभास्वरं दीप्तं प्रदीपादेर्मण्डलमविद्य-”} मानमेव {\color{DodgerBlue3}“मन्दचक्षुषः\edlabel{pvv.1a}\footnote{\label{pvv.1a}  1a चत्तगताः ।\begin{english}\par
Placement of note uncertain; marked with a question mark in the edition (see encoding description for details)\end{english}}\edlabel{pvv.1b}\footnote{\label{pvv.1b}  1b रूपेण ।\begin{english}\par
Placement of note uncertain; marked with a question mark in the edition (see encoding description for details)\end{english}}\edlabel{pvv.239-1}\footnote{\label{pvv.239-1}  १ विप्लुताक्षः ।} संपश्यन्ति”} । दीपस्य तादृशस्वरूपाभावात् । {\color{DodgerBlue3}“ज्ञानस्यानु-”} भवात्मनः स आकार इति द्वैरूप्यसिद्धिः । (४०३)
	\pend
      \label{div_pvv.2.404_2.405}\edlabel{div_pvv.2.404_2.405}
	  
	% new div opening: depth here is 2
	

	  \pstart अथ तादृशं वस्त्वेवोत्पन्नं दृश्यत इति न ज्ञानाकार इत्याह (।)
	\pend
      
	  \bigskip
	  \begingroup
	  \large
	
	    
	    \stanza[\smallbreak]
	\label{pv.2.404}\edlabel{pv.2.404}\flagstanza{\tiny\textenglish{....2.404}}तस्य तद्बाह्यरूपत्वे का प्रसन्नेक्षणेऽक्षमा ।&भूतं पश्यँश्च तद्दर्शी कथञ्चोपहतेन्द्रियः ॥ ४०४ ॥\&[\smallbreak]


	
	  \endgroup
	
	  \bigskip
	  \begingroup
	  \large
	
	    
	    \stanza[\smallbreak]
	\label{pv.2.405}\edlabel{pv.2.405}\flagstanza{\tiny\textenglish{....2.405}}शोधितं तिमिरेणास्य व्यक्तं चक्षुरतीन्द्रियम् ।&पश्यतोऽन्याक्षदृश्येर्थे तदव्यक्तं कथं पुनः ॥ ४०५ ॥\&[\smallbreak]


	
	  \endgroup
	

	  \pstart {\color{DodgerBlue3}“तस्य”} मण्डलस्य {\color{DodgerBlue3}“तद्बाह्यरूपत्वे”}ऽभ्युपगम्यमाने {\color{DodgerBlue3}“प्रसन्नेक्षणे”} द्रष्टरि {\color{DodgerBlue3}“काऽक्षमा द्वेषो”} येनास्मै नात्मानमुपदर्शयेत् । यद्वस्तुपहतेन्द्रियेण दृश्यते । तदनुपहतेन्द्रियेण सुतरां दृश्यते । {\color{DodgerBlue3}“भूतं”} सत्यं {\color{DodgerBlue3}“च पश्यन् तद्दर्शी”} मण्डलदर्शी {\color{DodgerBlue3}“कथमुपहतेन्द्रियैः”} (४०४) {\color{DodgerBlue3}“अपरैरदृश्यं”} मण्डलं {\color{DodgerBlue3}“पश्य”}तोऽस्य मन्दचक्षुष्ट्वेन नष्टस्य {\color{DodgerBlue3}“तिमिरेण व्यक्तं चक्षुः शोधित”}मित्युपहसति । किन्तु तैमिरिकस्यातीन्द्रियार्थदर्शनक्षमं तच्चक्षुरन्यस्यातैमिरिकस्याक्षदृश्येऽर्थे प्रदीपे {\color{DodgerBlue3}“कथं पुनरव्यक्तम”}स्फुटं यदतीन्द्रियं पश्यति तत् सर्व्वं दृश्यं सुतरां पश्यति । (४०५)
	\pend
      \label{div_pvv.2.406}\edlabel{div_pvv.2.406}
	  
	% new div opening: depth here is 2
	

	  \pstart किञ्च (।)
	\pend
      
	  \bigskip
	  \begingroup
	  \large
	
	    
	    \stanza[\smallbreak]
	\label{pv.2.406}\edlabel{pv.2.406}\flagstanza{\tiny\textenglish{....2.406}}आलोकाक्षमनस्कारादन्यस्यैकस्य गम्यते ।&शक्तिर्हेतुस्ततो नान्योऽहेतुश्च विषयः कथम् ॥ ४०६ ॥\&[\smallbreak]


	
	  \endgroup
	\leavevmode\marginnote{\textenglish{240/s}}

	  \pstart {\color{DodgerBlue3}“आलोकाक्षमनस्कारादन्यस्य दीप\edlabel{pvv.240-1}\footnote{\label{pvv.240-1}  १ एकमेष्टव्यं वस्तुतो भिन्नस्याभेदे इष्यमाणे । मरीचिषु जलवत् ।}स्य एकस्य”} मण्डलज्ञानजनने {\color{DodgerBlue3}“शक्तिर्ग्गम्यते”} तन्मात्रभावेन भावात् । ततो दीपादन्यो मण्डलो न हेतुरहेतुश्चासौ मण्डलज्ञानस्य {\color{DodgerBlue3}“कथं विषयो”}तिप्रसङ्गात् । (४०६)
	\pend
      \label{div_pvv.2.407}\edlabel{div_pvv.2.407}
	  
	% new div opening: depth here is 2
	
	  \bigskip
	  \begingroup
	  \large
	
	    
	    \stanza[\smallbreak]
	\label{pv.2.407a}\edlabel{pv.2.407a}\flagstanza{\tiny\textenglish{...2.407a}}स एव यदि धीहेतुः किम्प्रदीपमपेक्षते ।\&[\smallbreak]


	
	  \endgroup
	

	  \pstart {\color{DodgerBlue3}“स”} मण्डल {\color{DodgerBlue3}“एव”} मण्डलग्राहिण्या {\color{DodgerBlue3}“धियो हेतु”}र्न दीपो {\color{DodgerBlue3}“यदी”}ष्यते तदा मण्डलं {\color{DodgerBlue3}“किं”} कस्मात् {\color{DodgerBlue3}“प्रदीपमपेक्षते”} न ह्यहेतोरपेक्षाऽतिप्रसङ्गात् ।
	\pend
      

	  \pstart दीपो मण्डलञ्च मण्डलबुद्धिहेतुरित्याह (।)
	\pend
      
	  \bigskip
	  \begingroup
	  \large
	
	    
	    \stanza[\smallbreak]
	\label{pv.2.407b}\edlabel{pv.2.407b}\flagstanza{\tiny\textenglish{...2.407b}}दीपमात्रेण धीभावादुभयन्नापि कारणम् ॥ ४०७ ॥\&[\smallbreak]


	
	  \endgroup
	

	  \pstart {\color{DodgerBlue3}“उभयं न कारणं दीपमात्रेण”} मण्डलधियो {\color{DodgerBlue3}“भावात्”} ।\edlabel{pvv.240-2}\footnote{\label{pvv.240-2}  २ बाह्यमभ्युपेत्य द्वैरूप्यमुक्त्वाधुना बाह्याभावमाह ।}(४०७)
	\pend
      \label{div_pvv.2.408}\edlabel{div_pvv.2.408}
	  
	% new div opening: depth here is 2
	
	  \bigskip
	  \begingroup
	  \large
	
	    
	    \stanza[\smallbreak]
	\label{pv.2.408}\edlabel{pv.2.408}\flagstanza{\tiny\textenglish{....2.408}}दूरासन्नादिभेदेन व्यक्ताव्यक्तं न युज्यते ।&तत्स्यादालोकभेदाच्चेत्तत्पिधानापिधानयोः ॥ ४०८ ॥\&[\smallbreak]


	
	  \endgroup
	

	  \pstart यदि चार्थ एव साकारो ग्राह्यो ज्ञानन्त्वनाकारं तदार्थस्य {\color{DodgerBlue3}“दूरासन्नादिना भेदेन विशेषेण व्यक्ताव्यक्तं न युज्यते”} एकात्मनः पदार्थस्य स्वरूपेण दृश्यमानत्वात् । दूरासन्नस्थाभ्यां समानः प्रतीयेत ।\edlabel{pvv.240-3}\footnote{\label{pvv.240-3}  ३ एकस्य नानात्वविरोधात् ।} ज्ञानस्वगताकारभेदानभ्युपगमात् । व्यवघानाव्यवधानयोरा{\color{DodgerBlue3}“लोक”}स्य मान्द्यामान्द्य{\color{DodgerBlue3}“भेदात् । तद्”} व्यक्ताव्यक्तं वस्तु {\color{DodgerBlue3}“स्यादि”}ति चेत् (।) दूरस्थितौ\edlabel{pvv.240-4}\footnote{\label{pvv.240-4}  ४ अत्र द्वौ विकल्पौ ।} तस्यालोकस्य {\color{DodgerBlue3}“पिधानमपिधा”}नञ्चाऽभ्युपगन्तव्यं तयो\leavevmode\marginnote{\textenglish{47a/MA}} {\color{DodgerBlue3}“स्तुल्ययोः”} (।) (४०८)
	\pend
      \label{div_pvv.2.409}\edlabel{div_pvv.2.409}
	  
	% new div opening: depth here is 2
	
	  \bigskip
	  \begingroup
	  \large
	
	    
	    \stanza[\smallbreak]
	\label{pv.2.409}\edlabel{pv.2.409}\flagstanza{\tiny\textenglish{....2.409}}तुल्या दृष्टिरदृष्टिर्वा सूक्ष्मोंशस्तस्य कश्चन ।&आलोकेन च मन्देन दृश्यतेतो भिदा यदि ॥ ४०९ ॥\&[\smallbreak]


	
	  \endgroup
	

	  \pstart सर्व्वस्य प्रतिपत्तु{\color{DodgerBlue3}“र्दृष्टिरदृष्टि”}र्व्वा {\color{DodgerBlue3}“तुल्या”}र्थस्य स्यात् । \edlabel{pvv.240-5}\footnote{\label{pvv.240-5}  ५ साव्यक्ताव्यक्तत्वं ।}दूरस्थस्य रजोनीहारादिभिरुपहतत्वात् {\color{DodgerBlue3}“मन्देनालोकेन तस्य”} दृश्यार्थस्य {\color{DodgerBlue3}“सूक्ष्मोऽशो”}ऽवयवः {\color{DodgerBlue3}“कश्चन”} \leavevmode\marginnote{\textenglish{241/s}} न दृश्यतेऽतः । सन्निकृष्टाद् व्यक्तं दृश्यमानादर्थादव्यक्तत्वेन {\color{DodgerBlue3}“भिदा यद्युच्यते”} । तदापि द्वयी कल्पना । (४०९)
	\pend
      \label{div_pvv.2.410}\edlabel{div_pvv.2.410}
	  
	% new div opening: depth here is 2
	

	  \pstart योसौ स्थवीयान् दृश्यते स एकोऽनेको वा ।
	\pend
      
	  \bigskip
	  \begingroup
	  \large
	
	    
	    \stanza[\smallbreak]
	\label{pv.2.410}\edlabel{pv.2.410}\flagstanza{\tiny\textenglish{....2.410}}एकत्वेर्थस्य बाह्यस्य दृश्यादृश्यभिदा कुतः ॥&अनेकत्वेऽणुशो भिन्ने दृश्यादृश्यभिदा कुतः ॥ ४१० ॥\&[\smallbreak]


	
	  \endgroup
	

	  \pstart तत्रै{\color{DodgerBlue3}“कत्वेऽर्थस्य बाह्यस्या”}भ्युपगम्यमाने {\color{DodgerBlue3}“दृश्यादृश्यभिदा कुतः”} शङ्किता(।) एवं दृश्यमदृश्यमेव वा स्यादेकान्तेन । अनेकत्वे दृश्यस्यार्थस्याभ्युपगम्यमानेऽणुशो भिन्नेऽस्मिन् दृश्यादृश्यभिदा कुतः । न ह्यणुष्वपि स्थूलसूक्ष्मभेदः । येन किञ्चिदुपलभ्येत किञ्चिन्नेति विभागः । (४१०)
	\pend
      \label{div_pvv.2.411}\edlabel{div_pvv.2.411}
	  
	% new div opening: depth here is 2
	
	  \bigskip
	  \begingroup
	  \large
	
	    
	    \stanza[\smallbreak]
	\label{pv.2.411}\edlabel{pv.2.411}\flagstanza{\tiny\textenglish{....2.411}}मान्द्यपाटवभेदेन भासो बुद्धिभिदा यदि ।&भिन्नेऽन्यस्मिन्नभिन्नस्य कुतो भेदेन भासनम् ॥ ४११ ॥\&[\smallbreak]


	
	  \endgroup
	

	  \pstart अथैकरूपेप्यर्थे {\color{DodgerBlue3}“भास”} आलोकस्य {\color{DodgerBlue3}“मान्द्यपाटवभेदेन”} स्पष्टास्पष्टतया {\color{DodgerBlue3}“भेदो बुद्धेर्यदी”}ष्यते । तदाप्य{\color{DodgerBlue3}“न्यस्मिन्ना”}लोके पटुमन्दतया {\color{DodgerBlue3}“भिन्नेऽभिन्नस्या”}र्थस्य स्वरूपेण दृश्यमानस्य {\color{DodgerBlue3}“कुतो भेदेन”} स्पष्टास्पष्टतया {\color{DodgerBlue3}“भासनं”} युक्तं । (४११)
	\pend
      \label{div_pvv.2.412}\edlabel{div_pvv.2.412}
	  
	% new div opening: depth here is 2
	

	  \pstart किञ्च (।)
	\pend
      
	  \bigskip
	  \begingroup
	  \large
	
	    
	    \stanza[\smallbreak]
	\label{pv.2.412}\edlabel{pv.2.412}\flagstanza{\tiny\textenglish{....2.412}}मन्दन्तदपि तेजः किमावृत्तेरिह सा न किम् ।&तनुत्वन्तेजसोप्येतदस्त्यन्यत्राप्यतानवम् ॥ ४१२ ॥\&[\smallbreak]


	
	  \endgroup
	

	  \pstart व्यवहितवस्त्वन्तरालवर्त्ति {\color{DodgerBlue3}“तेजो मन्दं किं”} कस्माद्धेतो रजोनीहारादिभिस्तद्देशवर्त्तिभिरा{\color{DodgerBlue3}“वृत्ते”}रालोको मन्द इति चेत् । इह सन्निहितवस्त्वन्तरालवर्त्तिन्यालोके {\color{DodgerBlue3}“सा”} रजोनीहारादिभिरावृत्तिः {\color{DodgerBlue3}“किन्न”} भवति । {\color{DodgerBlue3}“तनुत्वा”}दावारकस्य नीहारादेर्न्नावृत्तिश्चेत् । {\color{DodgerBlue3}“तत्तनुत्वं तेजसोपि”} सन्निहितवस्त्वन्तरालवर्त्तिनो{\color{DodgerBlue3}“ऽस्तीति”} सन्निहितञ्च वस्तु न स्फुटं प्रतीयते तथा{\color{DodgerBlue3}“ऽन्यत्र”} दूरस्थं वस्तु स्फुटं प्रतीयेत् । दूरस्थे वस्तुन्यावारकालोकयोः समानमतानवं, समीपस्थे च समं तानवमिति न स्यात् प्रतीतिभेदः । (४१२)
	\pend
      \label{div_pvv.2.413}\edlabel{div_pvv.2.413}
	  
	% new div opening: depth here is 2
	

	  \pstart किञ्च (।)
	\pend
      
	  \bigskip
	  \begingroup
	  \large
	
	    
	    \stanza[\smallbreak]
	\label{pv.2.413}\edlabel{pv.2.413}\flagstanza{\tiny\textenglish{....2.413}}अत्यासन्ने च सुव्यक्तं तेजस्तत्स्यादतिस्फुटम् ।&तत्राप्यदृष्टमाश्रित्य भवेद्रूपान्तरं यदि ॥ ४१३ ॥\&[\smallbreak]


	
	  \endgroup
	

	  \pstart लोचनस्या{\color{DodgerBlue3}“त्यासन्ने”} शलाकादौ {\color{DodgerBlue3}“सुव्यक्तं तेज”} आवारकस्य तनुत्वादिति तदत्यासन्नं शलाकादिक{\color{DodgerBlue3}“मतिस्फुटं स्यात्”} । न च मनागव्यवहितमिवात्यासन्नं वस्तु स्फुट\leavevmode\marginnote{\textenglish{242/s}} मीक्ष्यते । {\color{DodgerBlue3}“तत्र”} दूरात्यासत्तिभेदेन व्यक्ताव्यक्तदर्शने{\color{DodgerBlue3}“प्यदृष्टं”} धर्माधर्म{\color{DodgerBlue3}“माश्रि”}त्यापेक्ष्य {\color{DodgerBlue3}“रूपान्तरं”} व्यक्ताव्यक्तं जायत इति यद्युच्यते (। ४१३)
	\pend
      \label{div_pvv.2.414}\edlabel{div_pvv.2.414}
	  
	% new div opening: depth here is 2
	
	  \bigskip
	  \begingroup
	  \large
	
	    
	    \stanza[\smallbreak]
	\label{pv.2.414}\edlabel{pv.2.414}\flagstanza{\tiny\textenglish{....2.414}}अन्योन्यावरणात्तेषां स्यात्तेजोविहतिस्ततः ।&तत्रैकमेव दृश्येत तस्यानावरणे सकृत् ॥ ४१४ ॥\&[\smallbreak]


	
	  \endgroup
	

	  \pstart तदा {\color{DodgerBlue3}“तेषां”} व्यक्ताव्यक्तानां रूपाणा{\color{DodgerBlue3}“मन्योन्य”}स्या{\color{DodgerBlue3}“वरणात्”} कदाचित्कस्यचिदुपलम्भो भवतीति वक्तव्यं । {\color{DodgerBlue3}“तत”} एकोपलम्भकालेऽपरोपलम्भहेतो{\color{DodgerBlue3}“स्तेजसोविहति”}रावृतिरिति च {\color{DodgerBlue3}“स्यात्”} । अन्यथा नावृत्ते तस्मिन्नपरस्याप्युपलब्धिः स्यादवैकल्यात्सामग्र्याः । {\color{DodgerBlue3}“तत्रा”}परोपलम्भहेतोरालोकस्यावरणे सति {\color{DodgerBlue3}“एकमेव”} व्यक्तमव्यक्तस्वरूपं दूरासन्नादिदेशस्थितैः प्रतिपत्तृभिः सर्व्वैर्दृश्येत । न त्वेकेन व्यक्तमितरेण चाव्यक्तमिति स्यात् । {\color{DodgerBlue3}“तस्या”}दृष्टोत्पन्नरूपस्य परस्परमनावरणे तेजसश्चा{\color{DodgerBlue3}“नावरणे सकृत्”} (। ४१४)
	\pend
      \label{div_pvv.2.415}\edlabel{div_pvv.2.415}
	  
	% new div opening: depth here is 2
	
	  \bigskip
	  \begingroup
	  \large
	
	    
	    \stanza[\smallbreak]
	\label{pv.2.415}\edlabel{pv.2.415}\flagstanza{\tiny\textenglish{....2.415}}पश्येत् स्फुटास्फुटं रूपमेकोऽदृष्टेन वारणे ।&अर्थानर्थौ न येन स्तस्तददृष्टं करोति किम् ॥ ४१५ ॥\&[\smallbreak]


	
	  \endgroup
	\leavevmode\marginnote{\textenglish{47b/MA}}

	  \pstart {\color{DodgerBlue3}“स्फुटास्फुटं रूपं पश्येदे”}कः\edlabel{pvv.242-1}\footnote{\label{pvv.242-1}  १ एकैको दूरासन्नस्थः सर्व्वः ।} प्रतिपत्ता । दृश्यरूपद्व\edlabel{pvv.242-2}\footnote{\label{pvv.242-2}  २ युगपत् स्फुटानि भवन्ति रूपाणि ॥}यस्य दर्शनहेतोश्चालोकस्यानावरणाददृष्टेन द्वितीयस्य रूपस्य । वरणे व्यक्तमव्यक्तमेव वा रूपमेकं दृश्यत इति\edlabel{pvv.242-3}\footnote{\label{pvv.242-3}  ३ सौमनस्योत्पादनेनानुग्राहकं धर्मी दर्शयति । दुःखोत्पादनेने(?नो)पपातकमावृणोति विपर्य्ययादधर्मः ।} चेत् । {\color{DodgerBlue3}“येन”} द्वितीयरूपावरणेन कृतेन पुंसो{\color{DodgerBlue3}“ऽर्थानर्थाव”}दृष्टकार्यौ {\color{DodgerBlue3}“न स्तः”} सम्भवत{\color{DodgerBlue3}“स्तदावरणमदृष्टं”} कर्त्तृ {\color{DodgerBlue3}“किं”} कस्मात् {\color{DodgerBlue3}“करोति”} शुभाशुभलक्षणं ह्यदृष्टमर्थानर्थफलं(।) यत्पुनरनुभवस्वभावं तत् दृष्टफलमेव भवति (। ४१५)
	\pend
      \label{div_pvv.2.416}\edlabel{div_pvv.2.416}
	  
	% new div opening: depth here is 2
	

	  \pstart यस्मात् सर्व्वमनन्तरोक्तमसंगतं ।
	\pend
      
	  \bigskip
	  \begingroup
	  \large
	
	    
	    \stanza[\smallbreak]
	\label{pv.2.416}\edlabel{pv.2.416}\flagstanza{\tiny\textenglish{....2.416}}तस्मात् संविद् यथाहेतु जायमानार्थसंश्रयात् ।&प्रतिभासभिदां धत्ते शेषाः कुमति-दुर्न्नयाः ॥ ४१६ ॥\&[\smallbreak]


	
	  \endgroup
	

	  \pstart {\color{DodgerBlue3}“तस्मादर्थसंश्रया\edlabel{pvv.242-4}\footnote{\label{pvv.242-4}  ४ अभ्युपगमेप्यर्थस्य ।}ज्‏जायमाना संवित्”} बुद्धिर्यथाहेतु वासनाप्रबोधहेत्वनतिक्रमेण {\color{DodgerBlue3}“प्रतिभास”}स्याकारस्य व्यक्ताव्यक्तादे{\color{DodgerBlue3}“र्भिदां धत्ते”} विभर्त्तीति\edlabel{pvv.242-5}\footnote{\label{pvv.242-5}  ५ द्वैरूप्यं} न्याय्यं । तदितरे पुनरालोकभेदोपन्यासाद्याः {\color{DodgerBlue3}“शेषाः कुमतिर्दुर्नयाः”} परवादिनां कुमतीनां दुर्व्विमर्शाः\edlabel{pvv.242-6}\footnote{\label{pvv.242-6}  ६ अर्थव्यक्त्यसंभवन्दर्शयन् बुद्धेर्द्वैरूप्यमाह ।}(। ४१६)
	\pend
      \label{div_pvv.2.417}\edlabel{div_pvv.2.417}
	  
	% new div opening: depth here is 2
	\leavevmode\marginnote{\textenglish{243/s}}

	  \pstart किञ्च(।) अनाकारेण ज्ञानेनार्थःक्षणि\edlabel{pvv.243-1}\footnote{\label{pvv.243-1}  १ नित्यो दीपः कस्यचित् । शब्दो नित्यो वैयाकरणादेः बुद्धिर्नित्या सांख्यस्य ।}कोऽक्षणिको वा व्यज्येत(।) तत्र(।)
	\pend
      
	  \bigskip
	  \begingroup
	  \large
	
	    
	    \stanza[\smallbreak]
	\label{pv.2.417}\edlabel{pv.2.417}\flagstanza{\tiny\textenglish{....2.417}}ज्ञानशब्दप्रदीपानां प्रत्यक्षस्येतरस्य वा ।&जनकत्वेन पूर्वेषां क्षणिकानां विनाशतः ॥ ४१७ ॥\&[\smallbreak]


	
	  \endgroup
	

	  \pstart {\color{DodgerBlue3}“क्षणिकानां ज्ञानशब्दप्रदीपादीनां”} स्वविषयस्य {\color{DodgerBlue3}“प्र\edlabel{pvv.243-2}\footnote{\label{pvv.243-2}  २ यदा स्वरूपानुकारि ज्ञानमव्यवहितं विकल्पज्ञानजनने ।}त्यक्ष”}स्याप्रत्यक्षस्य {\color{DodgerBlue3}“वा”} ज्ञानस्य {\color{DodgerBlue3}“जनकत्वेन”} हे\edlabel{pvv.243-3}\footnote{\label{pvv.243-3}  ३ ज्ञानादीनां ।}तूनां {\color{DodgerBlue3}“पू\edlabel{pvv.243-4}\footnote{\label{pvv.243-4}  ४ नाकारणं विषयः ।}र्व्वेषां”} ज्ञानकाले {\color{DodgerBlue3}“विनाशतः”} (४१७)
	\pend
      \label{div_pvv.2.418}\edlabel{div_pvv.2.418}
	  
	% new div opening: depth here is 2
	
	  \bigskip
	  \begingroup
	  \large
	
	    
	    \stanza[\smallbreak]
	\label{pv.2.418a}\edlabel{pv.2.418a}\flagstanza{\tiny\textenglish{...2.418a}}व्यक्तिः कुतोऽसतां ज्ञानाद्;\&[\smallbreak]


	
	  \endgroup
	

	  \pstart {\color{DodgerBlue3}“असतां ज्ञानात् कुतो व्यक्तिः”} । यदाऽर्थस्तदा न ज्ञानं यदा ज्ञानं {\color{DodgerBlue3}“तदा नार्थ इति”} कुतो व्य\edlabel{pvv.243-5}\footnote{\label{pvv.243-5}  ५ ज्ञानस्यैव विषयाकारः सिद्धः ।}ङ्ग्यव्यञ्जकभावस्तयोः । अथ शब्दादयो ज्ञानेन सह द्वितीयं स्वोपादेयक्षणं जनयन्ति स एव तेन ज्ञानेन व्यज्यते नेतरदित्याह (।)
	\pend
      
	  \bigskip
	  \begingroup
	  \large
	
	    
	    \stanza[\smallbreak]
	\label{pv.2.418b}\edlabel{pv.2.418b}\flagstanza{\tiny\textenglish{...2.418b}}अन्यस्यानुपकारिणः ।&व्यक्तौ व्यज्येत सर्व्वोर्थस्तद्धेतोर्नियमो यदि ॥ ४१८ ॥\&[\smallbreak]


	
	  \endgroup
	

	  \pstart स्वकारणाद{\color{DodgerBlue3}“न्यस्य”} सहोत्पन्नस्या{\color{DodgerBlue3}“नुपकारिणो व्यक्ति”}विवक्षायां सर्व्वोऽर्थः समानकालेन ज्ञानेन {\color{DodgerBlue3}“व्यज्य”}तामनुपकारकत्वाविशेषात् । {\color{DodgerBlue3}“तस्मा”}त्सहोत्पादकात् {\color{DodgerBlue3}“हेतोर”}नुपकारकत्वविशेषेष्वपि सहोत्पन्न एवार्थो {\color{DodgerBlue3}“ज्ञानेन”} व्यज्यत इति {\color{DodgerBlue3}“नियमो”} यदि कल्प्यते(४१८)
	\pend
      \label{div_pvv.2.419}\edlabel{div_pvv.2.419}
	  
	% new div opening: depth here is 2
	
	  \bigskip
	  \begingroup
	  \large
	
	    
	    \stanza[\smallbreak]
	\label{pv.2.419}\edlabel{pv.2.419}\flagstanza{\tiny\textenglish{....2.419}}नैषापि कल्पना ज्ञाने ज्ञानन्त्वर्थावभासतः ।&तं व्यनक्तीतिं कथ्येत तदभावेपि तत्कृतम् ॥ ४१९ ॥\&[\smallbreak]


	
	  \endgroup
	

	  \pstart तदैषापि {\color{DodgerBlue3}“कल्पना न”} युक्ता ज्ञाने व्यञ्जकत्वस्य स\edlabel{pvv.243-6}\footnote{\label{pvv.243-6}  ६ इन्द्रिंयेणापि ज्ञानहेतुना यज्जनितमिन्द्रियन्तद्विषयः स्यात् ।}होत्पन्नेन्द्रियादिव्यञ्जकत्वप्रसङ्गात् । अस्माकं मते तु साकारं तेनार्थेन कृतं {\color{DodgerBlue3}“ज्ञानमिति”} ज्ञानस्य काले {\color{DodgerBlue3}“तस्यार्थ”}स्या{\color{DodgerBlue3}“भावे”}प्यर्था{\color{DodgerBlue3}“वभासतो”}र्थाकारात्संवेद्यमानात् {\color{DodgerBlue3}“तमर्थं व्यनक्तीति कथ्यते”}ऽन्यस्यार्थव्यक्तिप्रकारस्यायोगात् (। ४१९)
	\pend
      \label{div_pvv.2.420}\edlabel{div_pvv.2.420}
	  
	% new div opening: depth here is 2
	

	  \pstart सहोत्पन्नस्यापि तर्हि स्वाकारज्ञानेन व्यक्तिः स्यादित्याह (।)
	\pend
      
	  \bigskip
	  \begingroup
	  \large
	
	    
	    \stanza[\smallbreak]
	\label{pv.2.420}\edlabel{pv.2.420}\flagstanza{\tiny\textenglish{....2.420}}नाकारयति चान्योर्थोऽनुपकारात् सहोदितः ।&व्यक्तोनाकारयज्ज्ञानं स्वाकारेण कथं भवेत् ॥ ४२० ॥\&[\smallbreak]


	
	  \endgroup
	

	  \pstart कारणादन्यस्या{\color{DodgerBlue3}“न्यश्चार्थः सहो”}त्पन्नो ज्ञानं {\color{DodgerBlue3}“नाकारयति स्वाकारेण”} विशेषयति {\color{DodgerBlue3}“अनुपकारात्”} । न ह्यनुपकार(का)कारेण विशिष्यतेऽतिप्रसङ्गात् । यश्चार्थो {\color{DodgerBlue3}“ज्ञानं नाकारयति स कथं व्यक्तो भवेत्”} । (४२०)
	\pend
      \label{div_pvv.2.421}\edlabel{div_pvv.2.421}
	  
	% new div opening: depth here is 2
	\leavevmode\marginnote{\textenglish{244/s}}

	  \begin{center}%% label @type='head'
	\textbf{ग. अक्षणिकस्य व्यक्तिरसम्भवा}
	\end{center}
	

	  \pstart अक्षणिकस्याप्यर्थस्य व्यक्तिं निषेद्ध्ुमाह (।)
	\pend
      
	  \bigskip
	  \begingroup
	  \large
	
	    
	    \stanza[\smallbreak]
	\label{pv.2.421}\edlabel{pv.2.421}\flagstanza{\tiny\textenglish{....2.421}}वज्रोपलादिरप्यर्थः स्थिरः सोन्यानपेक्षणात् ।&सकृत् सर्वस्य जनयेज्ज्ञानानि जगतः समम् ॥ ४२१ ॥\&[\smallbreak]


	
	  \endgroup
	

	  \pstart यो {\color{DodgerBlue3}“वज्रोपलादिः”} स्थिरो{\color{DodgerBlue3}“ऽर्थः सोऽपि”} ज्ञानोत्पादनस्वभावत्वे{\color{DodgerBlue3}“नान्यस्य”} सहकारिणोऽनुपकारकस्या{\color{DodgerBlue3}“नपेक्षणात् सकृत्सर्व्वस्य जगतः”} स्वग्राहकाणि {\color{DodgerBlue3}“ज्ञाना”}नि {\color{DodgerBlue3}“सममे”}ककालं {\color{DodgerBlue3}“जनयेत्”} । (४२१)
	\pend
      \label{div_pvv.2.422}\edlabel{div_pvv.2.422}
	  
	% new div opening: depth here is 2
	

	  \pstart न चैतदस्ति (।)
	\pend
      
	  \bigskip
	  \begingroup
	  \large
	
	    
	    \stanza[\smallbreak]
	\label{pv.2.422}\edlabel{pv.2.422}\flagstanza{\tiny\textenglish{....2.422}}क्रमाद् भवन्ति तान्यस्य सहकार्युपकार्यतः ।&आहुः प्रतिक्षणं भेदं स दोषोऽत्रापि पूर्ववत् ॥ ४२२ ॥\&[\smallbreak]


	
	  \endgroup
	\leavevmode\marginnote{\textenglish{48a/MA}}

	  \pstart क्रमेणोत्पादात् {\color{DodgerBlue3}“क्रमाद्”} भवन्ति जायमानानि {\color{DodgerBlue3}“तानि”} वज्रादिज्ञानान्यस्य वज्रोपलादेः । {\color{DodgerBlue3}“सहकारि”}णामुपकारात् स्वभावान्तरलक्षणात् {\color{DodgerBlue3}“प्रतिक्षणं भेद”}\edlabel{pvv.244-1}\footnote{\label{pvv.244-1}  १ कारकाधारकत्वेन ।} मन्यस्वभावता\edlabel{pvv.244-2}\footnote{\label{pvv.244-2}  २ क्षणिकत्वं ।}{\color{DodgerBlue3}“माहुः”}, यथा च {\color{DodgerBlue3}“पू\edlabel{pvv.244-3}\footnote{\label{pvv.244-3}  ३ विनाशादसतां कुतो ज्ञानाद् व्यक्तिरिति ग्राहिकया ।}र्व्ववत्”} शब्दादिष्वि{\color{DodgerBlue3}“वात्र”} वज्रादिष्वपि क्षणिकेषु विज्ञानात् पूर्व्वकालभाविषु {\color{DodgerBlue3}“सो”}ऽव्यक्ति{\color{DodgerBlue3}“दोष”}प्रसङ्गस्तदवस्थः । (४२२)
	\pend
      \label{div_pvv.2.423}\edlabel{div_pvv.2.423}
	  
	% new div opening: depth here is 2
	

	  \begin{center}%% label @type='head'
	\textbf{(३) a. स्वसंवेदनचिन्ता}
	\end{center}
	

	  \begin{center}%% label @type='head'
	\textbf{क. बुद्धिरर्थाकारा}
	\end{center}
	

	  \pstart पुनर्बुद्धेरर्थाकारसिद्ध्यर्थमाह (।)
	\pend
      
	  \bigskip
	  \begingroup
	  \large
	
	    
	    \stanza[\smallbreak]
	\label{pv.2.423}\edlabel{pv.2.423}\flagstanza{\tiny\textenglish{....2.423}}संवेदनस्य तादात्म्ये न विवादोस्ति कस्यचित् ।&तस्यार्थरूपताऽसिद्धा सापि सिध्यते संस्मृतेः ॥ ४२३ ॥\&[\smallbreak]


	
	  \endgroup
	

	  \pstart {\color{DodgerBlue3}“सम्वेद\edlabel{pvv.244-4}\footnote{\label{pvv.244-4}  ४ यस्माच्चानुभवोत्तरकालं विषय इव ज्ञाने स्मृतिरुत्पद्यते, तस्मादस्ति द्विरूपता ज्ञानस्येत्यादि व्याचष्टे ॥}नस्य तादात्म्ये”}ऽनुभवरूपत्वे {\color{DodgerBlue3}“कस्यचि”}द्विदुषो {\color{DodgerBlue3}“वा\edlabel{pvv.244-5}\footnote{\label{pvv.244-5}  ५ ज्ञानं न ज्ञानरूपमितिवत् ।}दो नास्ति । तस्य”} संवेदनस्या{\color{DodgerBlue3}“र्थरूपता”} विवादाद{\color{DodgerBlue3}“सिद्धा”}\edlabel{pvv.244-6}\footnote{\label{pvv.244-6}  ६ विषयज्ञानतज्ज्ञानेत्यादौ एकत्र ज्ञाने विषयाकारोल्लेखेन द्वैरूप्यमुक्तमत्र ।} {\color{DodgerBlue3}“साऽर्था”}कारता{\color{DodgerBlue3}“पि संस्मृतेर”}र्थाभासानुभवस्य सम्य\edlabel{pvv.244-7}\footnote{\label{pvv.244-7}  ७ यथा परस्परविलक्षणेषु रूपादिष्वनुभूतेष्वन्योन्यविवेकेन स्मृतिः स्यात्तथा ज्ञानेपि स्मृतिरुत्पद्यते तदस्ति द्वैरूप्यमर्थवेदनम्विनार्थस्मृतेरयोगादस्ति च सात्र ।}क् स्मरणात् {\color{DodgerBlue3}“सिध्य\edlabel{pvv.244-8}\footnote{\label{pvv.244-8}  ८ कुत एतद्यस्मात् ।}ति”} । (४२३)
	\pend
      \label{div_pvv.2.424}\edlabel{div_pvv.2.424}
	  
	% new div opening: depth here is 2
	\leavevmode\marginnote{\textenglish{245/s}}
	  \bigskip
	  \begingroup
	  \large
	
	    
	    \stanza[\smallbreak]
	\label{pv.2.424}\edlabel{pv.2.424}\flagstanza{\tiny\textenglish{....2.424}}भेदेनाननुभूतेस्मिन्नविभक्ते स्वगोचरैः ।&एवमेतन्न खल्वेवमिति सा स्यान्न भेदिनि ॥ ४२४ ॥\&[\smallbreak]


	
	  \endgroup
	

	  \pstart {\color{DodgerBlue3}“अस्मिन्न”}र्थसंवेदनेंन {\color{DodgerBlue3}“भेदेनानुभूते”} स्व स्य {\color{DodgerBlue3}“गोचरैरर्थेः”} स्वाकारसमर्पणद्वारेणा{\color{DodgerBlue3}“विभक्ते”} परस्परतो भेदेन व्यवस्थापिते {\color{DodgerBlue3}“एतज्ज्ञा”}नमेवं घटग्रा\edlabel{pvv.245-1}\footnote{\label{pvv.245-1}  १ स्वसत्ताकाले}हकं । {\color{DodgerBlue3}“न खल्वेवं नैव”} घटग्राहक{\color{DodgerBlue3}“मिति । सा”} सम्वेदनस्मृति{\color{DodgerBlue3}“र्भेदिनी”} विभागवती {\color{DodgerBlue3}“न स्यात्”} ज्ञानेन सह (। ४२४)
	\pend
      \label{div_pvv.2.425}\edlabel{div_pvv.2.425}
	  
	% new div opening: depth here is 2
	
	  \bigskip
	  \begingroup
	  \large
	
	    
	    \stanza[\smallbreak]
	\label{pv.2.425}\edlabel{pv.2.425}\flagstanza{\tiny\textenglish{....2.425}}न चानुभवमात्रेण कश्चिद् भेदो विवेचकः ।&विवेकिनी न चास्पष्टभेदे धीर्यमलादिवत् ॥ ४२५ ॥\&[\smallbreak]


	
	  \endgroup
	

	  \pstart {\color{DodgerBlue3}“न चानुभवमात्रेणा”}वान्तरभिन्नेन प्रतिज्ञानं {\color{DodgerBlue3}“कश्चिद् भेदो”} विद्यमानोपि परस्परं {\color{DodgerBlue3}“विवेचको”} भेदव्यवस्थापनहेतुः । तथाविधा{\color{DodgerBlue3}“ऽस्पष्टे भेदे”} सत्यपि {\color{DodgerBlue3}“धीः”} स्मृतिरूपा {\color{DodgerBlue3}“विवेकिनी”} न भवति । किन्त्वे\edlabel{pvv.245-2}\footnote{\label{pvv.245-2}  २ सारूप्यादन्य इन्द्रियादिभेदात् ।}कबोधाध्यवसायिनी प्रत्यभिज्ञैव स्या{\color{DodgerBlue3}“द्यमलादि\edlabel{pvv.245-3}\footnote{\label{pvv.245-3}  ३ सामग्रीभेदात् सुखादिभेदवद्विवेकेन स्मृतिः सेत्स्यतीत्यपि न । यतः सदार्था स्पष्टानुभवपूर्व्वा साकारा नैवं सुखादिरन्तः प्रीत्यादिरूपः ।}वद्य-”} मलयोरर्थान्तरभेदसद्‏भावेप्येको दृश्यमानो नापरस्माद् भेदेनावसीयते किन्त्वेकत्वेनैव प्रत्यभिज्ञायते । (४२५)
	\pend
      \label{div_pvv.2.426}\edlabel{div_pvv.2.426}
	  
	% new div opening: depth here is 2
	

	  \pstart तस्मादर्थाकारानुभवाकारतया बुद्धिर्द्विरूपैव ॥
	\pend
      
	  \bigskip
	  \begingroup
	  \large
	
	    
	    \stanza[\smallbreak]
	\label{pv.2.426}\edlabel{pv.2.426}\flagstanza{\tiny\textenglish{....2.426}}द्वैरूप्यसाधनेनापि प्रायः सिद्धं स्ववेदनम् ।&स्वरूपभूताभासस्य तदा संवेदनेक्षणात् ॥ ४२६ ॥\&[\smallbreak]


	
	  \endgroup
	

	  \pstart ज्ञानानां {\color{DodgerBlue3}“द्वैरूप्यसाधनेनापि \edlabel{pvv.245-4}\footnote{\label{pvv.245-4}  ४ यथापरस्परविलक्षणेषु रूपादिष्वनुभूतेष्वन्योन्यविवेकेन ।}प्रायो”} बाहुल्येन\edlabel{pvv.245-5}\footnote{\label{pvv.245-5}  ५ ग्राहिकया } {\color{DodgerBlue3}“स्ववे\edlabel{pvv.245-6}\footnote{\label{pvv.245-6}  ६ तच्च वेद्यत इत्यर्थादात्मवेदनं न साक्षात्प्रायः शब्दाः ।}द”}नं ज्ञानं {\color{DodgerBlue3}“सिद्धं । तथा”} ज्ञानस्य {\color{DodgerBlue3}“स्वरूपभूतस्याभासस्या”}कारस्य {\color{DodgerBlue3}“तदा”} द्विरूपज्ञानोत्पत्तिकाले {\color{DodgerBlue3}“संवेदनादनु”}भूतेरीक्षणात् (। ४२६)
	\pend
      \label{div_pvv.2.427}\edlabel{div_pvv.2.427}
	  
	% new div opening: depth here is 2
	

	  \begin{center}%% label @type='head'
	\textbf{ख. अर्थानुभवाकारा}
	\end{center}
	

	  \pstart ज्ञानान्तरेण सरूपेण ज्ञानमर्थवद्वेद्यते इति चेत् । तदा (।)
	\pend
      
	  \bigskip
	  \begingroup
	  \large
	
	    
	    \stanza[\smallbreak]
	\label{pv.2.427}\edlabel{pv.2.427}\flagstanza{\tiny\textenglish{....2.427}}धियाऽतद्रूपया ज्ञाने निरुद्धेऽनुभवः कथम् ॥&स्वञ्च रूपं न सा वेत्तीत्युत्सन्नोनुभवोऽखिलः ॥ ४२७ ॥\&[\smallbreak]


	
	  \endgroup
	

	  \pstart {\color{DodgerBlue3}“धि\edlabel{pvv.245-7}\footnote{\label{pvv.245-7}  ७ अनुत्तरेण द्वैरूप्ये विषयसारूप्यमात्मभूतं ज्ञानस्य सिद्धं ।}याऽतद्रूपया”}ऽग्राह्यज्ञानस्वरूपया {\color{DodgerBlue3}“निरुद्धे”} ग्राह्ये {\color{DodgerBlue3}“ज्ञाने कथमनुभवः । स्वकाले”}\leavevmode\marginnote{\textenglish{246/s}} ज्ञानं न वे\edlabel{pvv.246-1}\footnote{\label{pvv.246-1}  १ तुल्यकालयोर्न ग्राह्यग्राहकत्वं स्वसम्वेदनं नाभ्युपेतं ।}द्यते ग्राहककाले ग्राह्यस्यैवाभाव इति कथं बुद्धिवेदनं । स्वञ्च रूपं त्वन्मते सा बुद्धिर्न {\color{DodgerBlue3}“वेत्तीत्यनुभवोऽखिलो”}ऽर्थस्य ज्ञानस्य चो{\color{DodgerBlue3}“त्सन्नः”} स्यात् । ज्ञानप्रकाशो ह्यर्थप्रकाशः । स च स्वपरकालयोर्नास्तीति प्रकाशो न स्यात् सर्व्वस्य । (४२७)
	\pend
      \label{div_pvv.2.428}\edlabel{div_pvv.2.428}
	  
	% new div opening: depth here is 2
	

	  \pstart किञ्च (।)
	\pend
      
	  \bigskip
	  \begingroup
	  \large
	
	    
	    \stanza[\smallbreak]
	\label{pv.2.428}\edlabel{pv.2.428}\flagstanza{\tiny\textenglish{....2.428}}बहिर्मुखञ्च तज्ज्ञानं भात्यर्थप्रतिभासवत् ।&बुद्धेश्च ग्राहिका वित्तिर्न्नित्यमन्तर्मुखात्मनि ॥ ४२८ ॥\&[\smallbreak]


	
	  \endgroup
	

	  \pstart अस्यार्थस्य ग्राह्यस्य प्रतिभासवदाकारवत्तद् बाह्यग्राहकं ज्ञानं {\color{DodgerBlue3}“बहिर्मुखं”} बाह्यतया प्रतिभाति यथा नीलादिज्ञानं । {\color{DodgerBlue3}“बुद्धेश्चात्मनि ग्राहिका”} वित्ति\edlabel{pvv.246-2}\footnote{\label{pvv.246-2}  २ प्रत्यक्षविरुद्धत्वं स्ववेदनाभावस्याह एतेन ।} {\color{DodgerBlue3}“र्नित्यं”} सर्व्वकाल{\color{DodgerBlue3}“मन्तर्मुखा”}ऽबाह्यतया ग्राहकत्वेन प्रतिभाति । तदेतत् स्ववेदनतायामेवोपपन्नं । (४२८)-
	\pend
      \label{div_pvv.2.429}\edlabel{div_pvv.2.429}
	  
	% new div opening: depth here is 2
	

	  \pstart --- यदि तु बुद्ध्यन्तरग्राहिका बुद्धिः स्यात्तदा नीलादिवत् स्मर्यमाणाऽतीतस्वबुद्धि\edlabel{pvv.246-3}\footnote{\label{pvv.246-3}  ३ या विषयग्राहिका पूर्व्वा ।}वच्च ग्राहकाकारत्वाद् बहिष्ट्वेनावभासेत तथा स्ववेदनताऽभावे (।)
	\pend
      
	  \bigskip
	  \begingroup
	  \large
	
	    
	    \stanza[\smallbreak]
	\label{pv.2.429}\edlabel{pv.2.429}\flagstanza{\tiny\textenglish{....2.429}}यो यस्य विषयाभासस्तं वेत्ति न तदित्यपि ।&प्राप्तां का संविदन्यास्ति ताद्रूप्यादिति चेन्मतम् ॥ ४२९ ॥\&[\smallbreak]


	
	  \endgroup
	

	  \pstart {\color{DodgerBlue3}“यो विषयस्याभास”} आकारो {\color{DodgerBlue3}“यस्य”} ज्ञानस्य {\color{DodgerBlue3}“तं”} स्वाकारार्पकं विषयं {\color{DodgerBlue3}“तदा”}कारवत् ज्ञानं न {\color{DodgerBlue3}“वेत्तीति”} प्राप्तं विषयस्वरूपस्यात्मनो वेदने हि विषयवेदनं तत्परोक्षतया अर्थोपि परोक्षः स्यात् । यतोऽर्थस्वरूपधीवेदनादन्या {\color{DodgerBlue3}“का संविदर्थ”}स्यास्ति । {\color{DodgerBlue3}“ताद्रूप्याद्वि”}षयसारूप्या\edlabel{pvv.246-4}\footnote{\label{pvv.246-4}  ४ नैयायिकः न स्वरूपबोधं मन्यते ।}दस्वसंवेदनादर्थस्य संविदि{\color{DodgerBlue3}“ति चेन्मतं”} । (४२९)
	\pend
      \label{div_pvv.2.430}\edlabel{div_pvv.2.430}
	  
	% new div opening: depth here is 2
	

	  \pstart एवं सति (।)
	\pend
      
	  \bigskip
	  \begingroup
	  \large
	
	    
	    \stanza[\smallbreak]
	\label{pv.2.430}\edlabel{pv.2.430}\flagstanza{\tiny\textenglish{....2.430}}प्राप्तं संवेदनं सर्व्वसदृशानां परस्परम् ।&बुद्धिः सरूपा तद्विच्चेत् नेदानीं वित् सरूपिका ॥ ४३० ॥\&[\smallbreak]


	
	  \endgroup
	

	  \pstart {\color{DodgerBlue3}“सर्व्वे”}षां यमलकादीनां {\color{DodgerBlue3}“सदृशानां परस्परं सम्वेदनं प्राप्तं”} न सदृश इत्येवानुभवः {\color{DodgerBlue3}“सरूपा तद्विद”}र्थस्य संवेदन{\color{DodgerBlue3}“ञ्चेत्”} । {\color{DodgerBlue3}“इदानीम”}स्मिन्नभ्युपगमे {\color{DodgerBlue3}“सरूपेका”} विन्न भवति । सारूप्यं वेदनलक्षणं न भवति । किन्त्वनुभवरूपता । सत्यपि सारूप्ये यमलकादीनामनुभवत्वात् । नापि सारूप्यवाहितानुभवमात्रं वेदनं, किन्तु स्वसंवेदनं \leavevmode\marginnote{\textenglish{48b/MA}} सारूप्यं । (४३०)
	\pend
      \label{div_pvv.2.431}\edlabel{div_pvv.2.431}
	  
	% new div opening: depth here is 2
	\leavevmode\marginnote{\textenglish{247/s}}

	  \pstart यदि सारूप्यवशाद्वेदनं तदा बुद्ध्यात्मनापि सारूप्याद्वेदनं स्यात् । तथा ग्राह्यग्राहकयोर्भेद एव प्राप्त इत्याह (।)
	\pend
      
	  \bigskip
	  \begingroup
	  \large
	
	    
	    \stanza[\smallbreak]
	\label{pv.2.431a}\edlabel{pv.2.431a}\flagstanza{\tiny\textenglish{...2.431a}}स्वयं सोनुभवस्तस्या न स सारूप्यकारणः ।\&[\smallbreak]


	
	  \endgroup
	

	  \pstart {\color{DodgerBlue3}“तस्या”} बुद्धेः {\color{DodgerBlue3}“सोऽनुभवो”}ऽपरोक्षत्वं स्वयं स्वरूपेण तथोत्पत्तेर्न {\color{DodgerBlue3}“सारूप्य\edlabel{pvv.247-1}\footnote{\label{pvv.247-1}  १ सारूप्यकृतः ।}कारणः”} सोऽनुभवो बुद्धेः । एवं तर्हि बाह्येप्यर्थे बुद्धिसारूप्यं निष्फलमित्याह (।)
	\pend
      
	  \bigskip
	  \begingroup
	  \large
	
	    
	    \stanza[\smallbreak]
	\label{pv.2.431b}\edlabel{pv.2.431b}\flagstanza{\tiny\textenglish{...2.431b}}क्रियाकर्म्मव्यवस्थायास्तल्लोके स्यान्निबन्धनम् ॥ ४३१ ॥\&[\smallbreak]


	
	  \endgroup
	

	  \pstart {\color{DodgerBlue3}“तदर्थ”}सारूप्यं {\color{DodgerBlue3}“क्रियाया”} अर्थादनुभूतेः । {\color{DodgerBlue3}“कर्मणो”} बाह्यस्य {\color{DodgerBlue3}“व्यवस्थाया निबन्धनं लोके”} बहिरध्यवसायिनि स्यात् । (४३१)
	\pend
      \label{div_pvv.2.432}\edlabel{div_pvv.2.432}
	  
	% new div opening: depth here is 2
	

	  \pstart न ह्यस्य सारूप्यमन्तरेणेंयमस्य संवित्तिरिति शक्यं व्यवस्थापयितुं\edlabel{pvv.247-2}\footnote{\label{pvv.247-2}  २ यदि स्वानुभवात्मतयैव प्रकाशो नार्थानुभवात्मतया तदा सम्बन्धाभावादर्थानुभवव्यपदेशो न युक्त इत्याह(।)} (॥)
	\pend
      
	  \bigskip
	  \begingroup
	  \large
	
	    
	    \stanza[\smallbreak]
	\label{pv.2.432}\edlabel{pv.2.432}\flagstanza{\tiny\textenglish{....2.432}}स्वभावभूततद्रूपसंविदारोपविप्लवात् ।&नीलादेरनुभूताख्या नानुभूतेः परात्मनः ॥ ४३२ ॥\&[\smallbreak]


	
	  \endgroup
	

	  \pstart यस्या बुद्धेः {\color{DodgerBlue3}“स्वभावभूतस्य”} रूपस्य विषयाकारस्य {\color{DodgerBlue3}“संविदो”} बहिरर्थेष्वा\edlabel{pvv.247-3}\footnote{\label{pvv.247-3}  ३ असतोपि ।} {\color{DodgerBlue3}“रोपः”} स एव {\color{DodgerBlue3}“विप्लवो”} भ्रान्त्युपनीतत्वात् तस्मा{\color{DodgerBlue3}“न्नीलादे”}र्व्वस्तुतोऽदृश्यमानस्याप्य{\color{DodgerBlue3}“नुभूताख्या”}ऽनुभवव्यवहारो लोकस्य {\color{DodgerBlue3}“न”} पुनर्ज्ञाना{\color{DodgerBlue3}“त्परात्मनो”}र्थस्य साक्षाद{\color{DodgerBlue3}“नुभूते”}रर्थानुभवव्यवहारः । (४३२)
	\pend
      \label{div_pvv.2.433}\edlabel{div_pvv.2.433}
	  
	% new div opening: depth here is 2
	
	  \bigskip
	  \begingroup
	  \large
	
	    
	    \stanza[\smallbreak]
	\label{pv.2.433a}\edlabel{pv.2.433a}\flagstanza{\tiny\textenglish{...2.433a}}धियो नीलादिरूपत्वे बाह्योर्थः किं प्रमाणकः ।\&[\smallbreak]


	
	  \endgroup
	

	  \pstart परमार्थतस्तु {\color{DodgerBlue3}“धियो”} नीला{\color{DodgerBlue3}“दिरूपत्वे”} स्वसम्वेद्ये तदाकारार्प्पको {\color{DodgerBlue3}“बाह्योऽर्थः”} स्वरूपेणादृश्यमानः {\color{DodgerBlue3}“किं प्रमाणकः”} । न ह्याकारद्वयं वेद्यते येनैको बाह्यस्यापरो ज्ञानस्येति स्यात् ।
	\pend
      

	  \pstart बाह्य एवाकारवान् धीस्तु निराकारेति प्रत्यक्षसिद्धोऽर्थः स्यादित्याह (।)
	\pend
      
	  \bigskip
	  \begingroup
	  \large
	
	    
	    \stanza[\smallbreak]
	\label{pv.2.433b}\edlabel{pv.2.433b}\flagstanza{\tiny\textenglish{...2.433b}}धियोऽनीलादिरूपत्वे स तस्यानुभवः कथम् ॥ ४३३ ॥\&[\smallbreak]


	
	  \endgroup
	

	  \pstart {\color{DodgerBlue3}“धियोऽनीलादिरूपत्वे सो”}ऽर्थाकाररहितो{\color{DodgerBlue3}“नुभवस्तस्य”} नीलस्य ग्राहक इति {\color{DodgerBlue3}“कथं”} शक्यव्यवस्थापनः । (४३३)
	\pend
      \label{div_pvv.2.434}\edlabel{div_pvv.2.434}
	  
	% new div opening: depth here is 2
	
	  \bigskip
	  \begingroup
	  \large
	
	    
	    \stanza[\smallbreak]
	\label{pv.2.434}\edlabel{pv.2.434}\flagstanza{\tiny\textenglish{....2.434}}यदा संवेदनात्मत्वं न सारूप्यनिबन्धनम् ।&सिद्धं तत् स्वत एवास्य किमर्थेनोपनीयते ॥ ४३४ ॥\&[\smallbreak]


	
	  \endgroup
	

	  \pstart अनुभवमात्रात्मतया सर्व्वत्र ज्ञानेऽविशेषात् विेशेषव्यवस्थानशक्तया ।
	\pend
      \leavevmode\marginnote{\textenglish{248/s}}

	  \pstart {\color{DodgerBlue3}“यदा सम्वेदनात्मकत्व”}मपरोक्षत्वं {\color{DodgerBlue3}“न सारूप्यनिबन्धनं तदा स्वत एव”} प्रकाशात्मतयोत्पत्तेस्तत्संवेदनात्मत्वं {\color{DodgerBlue3}“सिद्धं”} । ततश्चार्थेनास्य ज्ञानस्य {\color{DodgerBlue3}“किमुन्नीयते”} येन तस्य तद्वेदनमित्युच्यते(।) न हि ज्ञानस्य स्वप्रकाशेऽर्थापेक्षा । न च स्वस्माद् व्यतिरिक्तं तेन वेद्यते तत्कथमर्थवेदनमनेनेत्युच्यते (। ४३४)
	\pend
      \label{div_pvv.2.435}\edlabel{div_pvv.2.435}
	  
	% new div opening: depth here is 2
	

	  \pstart किञ्च (।)
	\pend
      
	  \bigskip
	  \begingroup
	  \large
	
	    
	    \stanza[\smallbreak]
	\label{pv.2.435}\edlabel{pv.2.435}\flagstanza{\tiny\textenglish{....2.435}}नच सर्वात्मना साम्यमज्ञानत्वप्रसङ्गतः ।&न च केनचिदंशेन सर्वं सर्वस्य वेदनम् ॥ ४३५ ॥\&[\smallbreak]


	
	  \endgroup
	

	  \pstart ज्ञानं {\color{DodgerBlue3}“सर्व्वात्मना”} वा एकदेशेन वाऽर्थस्य सरूपं यत्तद्‏ग्राहकं स्यात् । तत्र न तावत्सर्व्वेण जडत्वादिना {\color{DodgerBlue3}“साम्यमज्ञानत्वप्रसङ्गतः । न च”} जडयोर्ग्राह्यग्राहकभावः {\color{DodgerBlue3}“केनचिदंशेन”} वस्तुत्वनीलत्वादीना {\color{DodgerBlue3}“सर्व्वं”} ज्ञानं सर्व्वस्यार्थस्य संवेदनं स्यात् । {\color{DodgerBlue3}“सर्व्वं”} वा नीलज्ञानं सर्व्वस्य\edlabel{pvv.248-1}\footnote{\label{pvv.248-1}  १ भिन्नाकारस्य न वेदनमिति शङ्का स्यात् नीलाकारस्तु सर्व्वनीलानुकारीति सर्व्वग्रहः ।} नीलस्य {\color{DodgerBlue3}“वेदनं स्यात्”} । (४३५)
	\pend
      \label{div_pvv.2.436}\edlabel{div_pvv.2.436}
	  
	% new div opening: depth here is 2
	
	  \bigskip
	  \begingroup
	  \large
	
	    
	    \stanza[\smallbreak]
	\label{pv.2.436}\edlabel{pv.2.436}\flagstanza{\tiny\textenglish{....2.436}}यथा नीलादिरूपत्वान्नीलाद्यनुभवो मतः ।&तथानुभवरूपत्वात्तस्याप्यनुभवो भवेत् ॥ ४३६ ॥\&[\smallbreak]


	
	  \endgroup
	

	  \pstart {\color{DodgerBlue3}“यथा नीलादिरूपत्वात्”} ज्ञानं {\color{DodgerBlue3}“नीलादी”}नाम{\color{DodgerBlue3}“नुभवो मतः”} तथा किमनुभव{\color{DodgerBlue3}“रूपत्वात्”} तस्यानुभवस्यार्थविषयस्यापि पूर्व्वकस्योत्तरं ज्ञानम{\color{DodgerBlue3}“नुभवो न\edlabel{pvv.248-2}\footnote{\label{pvv.248-2}  २ ताद्रूप्यमभिव्यापित्वाद्विषाणित्वमिव गौर्नानुभवलक्षणः ।} भवेत्”} । (४३६)
	\pend
      \label{div_pvv.2.437}\edlabel{div_pvv.2.437}
	  
	% new div opening: depth here is 2
	
	  \bigskip
	  \begingroup
	  \large
	
	    
	    \stanza[\smallbreak]
	\label{pv.2.437}\edlabel{pv.2.437}\flagstanza{\tiny\textenglish{....2.437}}नानुभूतोनुभव इत्यर्थवद् (हि) विनिश्चयः ।&तस्माददोष इति चेत् नार्थेप्यस्त्येष सर्वदा ॥ ४३७ ॥\&[\smallbreak]


	
	  \endgroup
	

	  \pstart {\color{DodgerBlue3}“ना”}\edlabel{pvv.248-3}\footnote{\label{pvv.248-3}  ३ न चेष्यते ।}नुभवेऽ{\color{DodgerBlue3}“नुभूतोऽनुभव इत्यर्थवदर्थ”} इव गृहीते {\color{DodgerBlue3}“विनिश्चयो”} भवति । {\color{DodgerBlue3}“तस्मा”}दनुभवस्याप्यनुभवो ग्राह्यः कः स्यादित्यय{\color{DodgerBlue3}“मदोष इति चेत्”} ।\edlabel{pvv.248-4}\footnote{\label{pvv.248-4}  ४ तदुत्पत्तिसारूप्ययोः सतोरपीदं मयानुभूतमिति निश्चयोऽर्थानुभवः । नैवमनुभवविषयोऽनुभवः स्मृतिरेव तु स्यात् ।} न केवलमनुभवे{\color{DodgerBlue3}“ऽर्थेऽप्येषो”}\edlabel{pvv.248-5}\footnote{\label{pvv.248-5}  ५ परामर्शयोगी ।}नुभूतत्वनिश्चयः {\color{DodgerBlue3}“सर्व्वदा ना”}स्ति । न हि धारावाहिन्यर्थज्ञानेऽर्थेषु प्रतिक्षणमनुभूतनिश्चयः\edlabel{pvv.248-6}\footnote{\label{pvv.248-6}  ६ पश्यत एवार्थं विषयान्तरव्याक्षेपान्नानुभूतनिश्चयः ।}। ततश्चार्थोपि नानुभूतः स्यात् । (४३७)
	\pend
      \label{div_pvv.2.438_2.439}\edlabel{div_pvv.2.438_2.439}
	  
	% new div opening: depth here is 2
	
	  \bigskip
	  \begingroup
	  \large
	
	    
	    \stanza[\smallbreak]
	\label{pv.2.438}\edlabel{pv.2.438}\flagstanza{\tiny\textenglish{....2.438}}कस्माद्वाऽनुभवे नास्ति सति सत्तानिबन्धने ।&अपि चेदं यदाभाति दृश्यमाने सितादिके ॥ ४३८ ॥\&[\smallbreak]


	
	  \endgroup
	\leavevmode\marginnote{\textenglish{249/s}}
	  \bigskip
	  \begingroup
	  \large
	
	    
	    \stanza[\smallbreak]
	\label{pv.2.439}\edlabel{pv.2.439}\flagstanza{\tiny\textenglish{....2.439}}पुंसः सिताद्यभिव्यक्तिरूपं संवेदनं स्फुटम् ।&तत्किं सिताद्यभिव्यक्तेः पररूपमथात्मनः ॥ ४३९ ॥\&[\smallbreak]


	
	  \endgroup
	

	  \pstart {\color{DodgerBlue3}“कस्माद्वाऽनुभवे”}ऽनुभूतनिश्चयस्य {\color{DodgerBlue3}“सत्तानिबन्धने”} सारूप्ये तदुत्पादे च {\color{DodgerBlue3}“सति नास्ति”} सत्तानिश्चयः । अर्थेप्यनुभूतनिबन्धने सारूप्यतदुत्पत्ती एव ते तज्ज्ञानेपि समाने । {\color{DodgerBlue3}“अपि च”} स्वसंवेदनानभ्युपगमे {\color{DodgerBlue3}“सितादिके \edlabel{pvv.249-1}\footnote{\label{pvv.249-1}  १ विच्छिन्ने ।}दृश्यमाने, यदिदं\edlabel{pvv.249-2}\footnote{\label{pvv.249-2}  २ तदैवावेदनात् ।}”} (४३८) {\color{DodgerBlue3}“सिताद्यभिव्यक्तिरूपमन्तः”} प्रकाशमानं {\color{DodgerBlue3}“संवेदनं स्फुटं पुंसः”} प्रतिपत्तु{\color{DodgerBlue3}“राभाति । तत् किं सिताद्यभिव्यक्तेः”} पररूप{\color{DodgerBlue3}“मथात्म”}भूतमिति विकल्पौ, सतस्तत्त्वान्यत्वाव्यतिक्रमात् । (४३९)\leavevmode\marginnote{\textenglish{49a/MA}}
	\pend
      \label{div_pvv.2.440}\edlabel{div_pvv.2.440}
	  
	% new div opening: depth here is 2
	
	  \bigskip
	  \begingroup
	  \large
	
	    
	    \stanza[\smallbreak]
	\label{pv.2.440}\edlabel{pv.2.440}\flagstanza{\tiny\textenglish{....2.440}}पररूपेऽप्रकाशायां व्यक्तौ व्यक्तं कथं सितम् ।&ज्ञानं व्यक्तिर्न सा व्यक्तेत्यव्यक्तमखिलं जगत् ॥ ४४० ॥\&[\smallbreak]


	
	  \endgroup
	

	  \pstart {\color{DodgerBlue3}“पररूपे”}ऽभ्युपगम्यमाने सम्वेदनं यत्प्रकाशते तन्न सितादिव्यक्तिरूपमित्य{\color{DodgerBlue3}“प्रका-”} शायां शुक्लादि{\color{DodgerBlue3}“व्यक्तौ सितं कथं व्यक्तं । ज्ञानं”} हि {\color{DodgerBlue3}“व्यक्तिर्न च सा व्यक्ता”} सितादिके दृश्यमाने इष्यत {\color{DodgerBlue3}“इत्यव्यक्तमखिलं जग”}त्प्राप्तं । सिताद्यभिव्यक्तिरेषितव्या । (४४०)
	\pend
      \label{div_pvv.2.441}\edlabel{div_pvv.2.441}
	  
	% new div opening: depth here is 2
	

	  \pstart तथा हि (।)
	\pend
      
	  \bigskip
	  \begingroup
	  \large
	
	    
	    \stanza[\smallbreak]
	\label{pv.2.441a}\edlabel{pv.2.441a}\flagstanza{\tiny\textenglish{...2.441a}}व्यक्तेर्व्यक्त्यन्तरव्यक्तावपि दोषप्रसङ्गतः ।\&[\smallbreak]


	
	  \endgroup
	

	  \pstart अर्थ{\color{DodgerBlue3}“व्यक्तेर्व्यक्तिरा”}पद्यमाना न व्यक्ता स्यात् । अथ व्यक्तिरेव न सिध्येत् । तस्याः स्वप्रकाशत्वेऽर्थव्यक्तिरपि तथास्तु । अथार्थव्यक्तिर्व्यक्तेर्व्यक्त्यन्तराद् उत्तरकालभाविवेदनाद् व्यक्तिरेवं तस्याश्चा\edlabel{pvv.249-3}\footnote{\label{pvv.249-3}  ३ विषयज्ञानस्यापि स्वयमव्यक्तेः ।}न्यत इत्यनवस्था दुर्व्वारा ।
	\pend
      

	  \pstart किञ्च (।)
	\pend
      
	  \bigskip
	  \begingroup
	  \large
	
	    
	    \stanza[\smallbreak]
	\label{pv.2.441b}\edlabel{pv.2.441b}\flagstanza{\tiny\textenglish{...2.441b}}दृष्ट्या वाज्ञातसम्बन्धं विशिनष्टि तया कथं ॥ ४४१ ॥\&[\smallbreak]


	
	  \endgroup
	

	  \pstart {\color{DodgerBlue3}“दृष्ट्या”} धि याऽ\edlabel{pvv.249-4}\footnote{\label{pvv.249-4}  ४ तत्समकालं, आह}सम्विदितया सहा{\color{DodgerBlue3}“ज्ञातसम्बन्धमर्थं कथं विशिनष्टि”} प्रतिपत्ता दृष्टोयमिति । (४४१)
	\pend
      \label{div_pvv.2.442}\edlabel{div_pvv.2.442}
	  
	% new div opening: depth here is 2
	

	  \pstart कथमर्थो दृष्ट्याऽज्ञातसम्बन्ध इत्याह (।)
	\pend
      
	  \bigskip
	  \begingroup
	  \large
	
	    
	    \stanza[\smallbreak]
	\label{pv.2.442}\edlabel{pv.2.442}\flagstanza{\tiny\textenglish{....2.442}}यस्माद् द्वयोरेकगतौ न द्वितीयस्य दर्शनम् ।&द्वयोः संसृष्टयोर्दृष्टौ स्याद् दृष्टमिति निश्चयः ॥ ४४२ ॥\&[\smallbreak]


	
	  \endgroup
	

	  \pstart {\color{DodgerBlue3}“यस्माद्”} द्वयोरर्थज्ञानयोर्मध्ये {\color{DodgerBlue3}“एकस्य गतौ”} दर्शनकाले {\color{DodgerBlue3}“न द्वितीयस्य दर्शनमस्ति”} । तथा हि न पदार्थो दृश्यते न तदा बुद्धिरुपलभ्यते तदुपलम्भस्य भावित्वात् ।
	\pend
      \leavevmode\marginnote{\textenglish{250/s}}

	  \pstart यदा च बुद्धिरुपलभ्यते न तदाऽन्योऽतीतत्वात् । तस्माद् {\color{DodgerBlue3}“द्वयो”}रर्थज्ञानयोः {\color{DodgerBlue3}“संसृष्टयोरे”}कोपलम्भात् {\color{DodgerBlue3}“दृष्टौ”} सत्यां {\color{DodgerBlue3}“दृष्टमिदमिति निश्चयः”} ततोऽन्योपलब्धिः स्वोपलब्धिरूपेव {\color{DodgerBlue3}“स्यादे”}तत् । (४४२)
	\pend
      \label{div_pvv.2.443}\edlabel{div_pvv.2.443}
	  
	% new div opening: depth here is 2
	
	  \bigskip
	  \begingroup
	  \large
	
	    
	    \stanza[\smallbreak]
	\label{pv.2.443a}\edlabel{pv.2.443a}\flagstanza{\tiny\textenglish{...2.443a}}सरूपं दर्शनं यस्य दृश्यतेन्येन चेतसा ।&दृष्टाख्येति न चेत्;\&[\smallbreak]


	
	  \endgroup
	

	  \pstart {\color{DodgerBlue3}“यस्यार्थस्य सरूपं”} समानाकारं {\color{DodgerBlue3}“दर्शनं”} ज्ञान{\color{DodgerBlue3}“मन्येन चेतसा दृश्यते”} तत्रार्थे {\color{DodgerBlue3}“दृष्टाख्या”} दृष्टव्यवहार {\color{DodgerBlue3}“इति चेत्”} ।
	\pend
      

	  \pstart अत्राह (।)
	\pend
      
	  \bigskip
	  \begingroup
	  \large
	
	    
	    \stanza[\smallbreak]
	\label{pv.2.443b}\edlabel{pv.2.443b}\flagstanza{\tiny\textenglish{...2.443b}}सिद्धं सारूप्येऽस्य स्ववेदनम् ॥ ४४३ ॥\&[\smallbreak]


	
	  \endgroup
	

	  \pstart अस्य ज्ञानस्यार्थेन सह {\color{DodgerBlue3}“सारूप्ये”} सिद्धे {\color{DodgerBlue3}“स्वसंवेदनं”} ज्ञानं सिद्धं । तथा ह्यर्थाकारस्तावत् ज्ञानकाले परिस्फुटं वेद्यमानो ज्ञानस्यात्मा चेत् ज्ञानमप्यर्थाकारवदपरोक्षमेव स्वभावत इति नान्यवेद्यं । (४४३)
	\pend
      \label{div_pvv.2.444}\edlabel{div_pvv.2.444}
	  
	% new div opening: depth here is 2
	
	  \bigskip
	  \begingroup
	  \large
	
	    
	    \stanza[\smallbreak]
	\label{pv.2.444a}\edlabel{pv.2.444a}\flagstanza{\tiny\textenglish{...2.444a}}अथात्मरूपं नो वेत्ति पररूपस्य वित् कथम् ।\&[\smallbreak]


	
	  \endgroup
	

	  \pstart {\color{DodgerBlue3}“अथ”} ज्ञानमात्मरूपं {\color{DodgerBlue3}“न वेत्ति पररूपस्य”} बाह्यरूपस्य {\color{DodgerBlue3}“वित्\edlabel{pvv.250-1}\footnote{\label{pvv.250-1}  १ वेदकमप्रत्यक्षोपलम्भात् ।} कथं”} । न ह्यर्थाकारज्ञानवेदनमन्तरेणार्थवेदनमित्युक्तं ।
	\pend
      

	  \pstart सारूप्यमात्रेणार्थवित्तिर्भविष्यतीति चेत् (।)
	\pend
      
	  \bigskip
	  \begingroup
	  \large
	
	    
	    \stanza[\smallbreak]
	\label{pv.2.444b}\edlabel{pv.2.444b}\flagstanza{\tiny\textenglish{...2.444b}}सारूप्याद् वेदनाख्या च प्रागेव प्रतिवर्ण्णिता ॥ ४४४ ॥\&[\smallbreak]


	
	  \endgroup
	

	  \pstart {\color{DodgerBlue3}“सारूप्यादनु”}भवात्मतारहिता{\color{DodgerBlue3}“द्वेदनाख्या”} वेदनव्यवहृतिश्च {\color{DodgerBlue3}“प्रागेव”} \cref{pv.2.430} “प्राप्तं संवेदनं सर्वसदृशानां परस्परभि”त्यनेन प्रतिवर्ण्णिता {\color{DodgerBlue3}“प्रत्युक्ता”} । (४४४)
	\pend
      \label{div_pvv.2.445}\edlabel{div_pvv.2.445}
	  
	% new div opening: depth here is 2
	

	  \pstart किञ्च\edlabel{pvv.250-2}\footnote{\label{pvv.250-2}  २ सारूप्यमेव नास्तीत्याह(।)} (।)
	\pend
      
	  \bigskip
	  \begingroup
	  \large
	
	    
	    \stanza[\smallbreak]
	\label{pv.2.445}\edlabel{pv.2.445}\flagstanza{\tiny\textenglish{....2.445}}दृष्टयोरेव सारूप्यग्रहोर्थञ्च न दृष्टवान् ।&प्राक् कथं दर्शनेनास्य सारूप्यं सोध्यवस्यति ॥ ४४५ ॥\&[\smallbreak]


	
	  \endgroup
	

	  \pstart {\color{DodgerBlue3}“दृष्टयोरेव”} कयोश्चित् {\color{DodgerBlue3}“सारूप्यग्रहो”} दृष्टो यथा यमलकयोः । न च कश्चिद् द्रष्टा ज्ञानात्\edlabel{pvv.250-3}\footnote{\label{pvv.250-3}  ३ प्रथमज्ञाने ।} प्रागर्थं दृष्टवान् । तत्कथन्दर्शनेन सहास्यादृष्टस्यार्थस्य स द्रष्टा {\color{DodgerBlue3}“सारूप्यमध्यवस्यति”} निश्चिनोति । (४४५)
	\pend
      \label{div_pvv.2.446}\edlabel{div_pvv.2.446}
	  
	% new div opening: depth here is 2
	

	  \pstart किञ्च (।)
	\pend
      \leavevmode\marginnote{\textenglish{251/s}}
	  \bigskip
	  \begingroup
	  \large
	
	    
	    \stanza[\smallbreak]
	\label{pv.2.446}\edlabel{pv.2.446}\flagstanza{\tiny\textenglish{....2.446}}सारूप्यभपि नेच्छेद्यः तस्य नोभयदर्शनम् ।&तदार्थो ज्ञानमिति च ज्ञाते चेति गता कथा ॥ ४४६ ॥\&[\smallbreak]


	
	  \endgroup
	

	  \pstart {\color{DodgerBlue3}“सारूप्यमपि”}\edlabel{pvv.251-1}\footnote{\label{pvv.251-1}  १ अनाकारवादी वैभाष्यादिः । साकारवाद्यपि नैयायिकादिर्यः सम्वेदनं नेच्छति ।} शब्दात् स्वसम्वेद{\color{DodgerBlue3}“नं यो”} वादी {\color{DodgerBlue3}“नेच्छेत् न तस्योभय”}स्यार्थस्य ज्ञानस्य च {\color{DodgerBlue3}“दर्शनं”} संगच्छते । सारूप्याभावेऽर्थवेदनाऽयोगात् । स्ववेदनाभावे च न ज्ञानसम्वेदनं । अन्येन तद्ग्रहस्य निषिद्धत्वात्\edlabel{pvv.251-2}\footnote{\label{pvv.251-2}  २ ज्ञानान्तरेण स्वयमविदिते नान्यग्रहायोगात् ।} । यदा चैवं {\color{DodgerBlue3}“तदार्थो ज्ञानमिति”} भेदः । {\color{DodgerBlue3}“ते च ज्ञाते चोति कथापि गतेति”} कृत्स्नं जगदन्धमूकं भवेत् । प्रतीतिनिबन्धनत्वादस्य व्यवहारस्य । (४४६)
	\pend
      \label{div_pvv.2.447}\edlabel{div_pvv.2.447}
	  
	% new div opening: depth here is 2
	
	  \bigskip
	  \begingroup
	  \large
	
	    
	    \stanza[\smallbreak]
	\label{pv.2.447}\edlabel{pv.2.447}\flagstanza{\tiny\textenglish{....2.447}}अथ स्वरूपं सा तर्हि स्वयमेव प्रकाशते ।&यत्तस्यामप्रकाशायामर्थः स्यादप्रकाशितः ॥ ४४७ ॥\&[\smallbreak]


	
	  \endgroup
	

	  \pstart {\color{DodgerBlue3}“अथ”} यदेतत्सिताद्यभिव्यक्तिरूपं स्फृटसम्वेदनमाभाति तद् बुद्धेः {\color{DodgerBlue3}“स्वरूपं”} (।) सा बुद्धि{\color{DodgerBlue3}“स्तर्हि स्वयमेवा”}परोक्षतया {\color{DodgerBlue3}“प्रकाशते यद्यस्मादस्यां”} बुद्धाव{\color{DodgerBlue3}“प्रकाशायां”} परोक्षाया{\color{DodgerBlue3}“मर्थोऽप्रकाशितः स्यात्”} । प्रकाशते चार्थ इति बुद्धिरप्यपरोक्षस्वभावेति स्वसम्वेदनसिद्धिः । (४४७)
	\pend
      \label{div_pvv.2.448}\edlabel{div_pvv.2.448}
	  
	% new div opening: depth here is 2
	
	  \bigskip
	  \begingroup
	  \large
	
	    
	    \stanza[\smallbreak]
	\label{pv.2.448}\edlabel{pv.2.448}\flagstanza{\tiny\textenglish{....2.448}}एतेनानात्मवित्‏पक्षे सर्व्वार्थादर्शनेन ये ।&अप्रत्यक्षां धियं प्राहुस्तेपि निर्व्वर्णितोत्तराः ॥ ४४८ ॥\&[\smallbreak]


	
	  \endgroup
	

	  \pstart {\color{DodgerBlue3}“अनात्मवित्पक्षे”} स्वसंवेदनाभावे\edlabel{pvv.251-3}\footnote{\label{pvv.251-3}  ३ यत्सर्व्वार्थादर्शनमुक्तं पूर्व्व अर्थो ज्ञानमिति ते ।} एतेनानन्तरमुपदर्शितेन {\color{DodgerBlue3}“सर्व्व”}स्या{\color{DodgerBlue3}“र्थ”}स्या{\color{DodgerBlue3}“दर्शनेन”} दर्शनाभावप्रसङ्गेन {\color{DodgerBlue3}“ये”} जै मि नी या {\color{DodgerBlue3}“अप्रत्यक्षां धिय”}मर्थापत्तिगम्यामा{\color{DodgerBlue3}“हुः तेपि निर्व्वर्ण्णितोत्तरा”} दत्तोत्तरा बोद्धव्याः । तथा ह्यर्थदर्शनान्यथानुपपत्त्या बुद्धिर्व्यवस्थापनीया । अर्थदर्शनमेव तु बुद्धिपरोक्षतायामसङ्गतमिति न तदन्यथानुपपद्यमानं बृद्धिं कल्पयितुमलं । (४४८)
	\pend
      \label{div_pvv.2.449}\edlabel{div_pvv.2.449}
	  
	% new div opening: depth here is 2
	

	  \pstart अपि च (।)
	\pend
      
	  \bigskip
	  \begingroup
	  \large
	
	    
	    \stanza[\smallbreak]
	\label{pv.2.449}\edlabel{pv.2.449}\flagstanza{\tiny\textenglish{....2.449}}आश्रयालम्बनाभ्यासभेदाद् भिन्नप्रवृत्तयः ।&सुखदुःखाभिलाषादिभेदा बुद्ध्य एव ताः ॥ ४४९ ॥\&[\smallbreak]


	
	  \endgroup
	

	  \pstart {\color{DodgerBlue3}“आश्रयस्ये”}न्द्रियस्या{\color{DodgerBlue3}“लम्बन”}स्य सुखादिवेदनीयस्या{\color{DodgerBlue3}“भ्यास”}स्य च यथा वृत्तस्य {\color{DodgerBlue3}“भेदा”}द्विशेषात् {\color{DodgerBlue3}“सुखदुःखभिलाषादि\edlabel{pvv.251-4}\footnote{\label{pvv.251-4}  ४ द्वेषप्रत्यत्नादिरादिना ।}भेदा भिन्नप्रवृत्तयो”} नानाकाराः सम्वि\leavevmode\marginnote{\textenglish{252/s}} \leavevmode\marginnote{\textenglish{49b/MA}} दितरूपा जायन्ते {\color{DodgerBlue3}“बुद्ध्य\edlabel{pvv.252-1}\footnote{\label{pvv.252-1}  १ सुखादिजातं प्रतिभासे ।} एव”} च ता बोधस्वभावत्वात् । ज्ञानेनाभिन्नहेतुकत्वा\edlabel{pvv.252-2}\footnote{\label{pvv.252-2}  २ पूर्व्वं बृद्धिरूपाः स्थापिताः ।}च्च । (४४९)
	\pend
      \label{div_pvv.2.450}\edlabel{div_pvv.2.450}
	  
	% new div opening: depth here is 2
	
	  \bigskip
	  \begingroup
	  \large
	
	    
	    \stanza[\smallbreak]
	\label{pv.2.450}\edlabel{pv.2.450}\flagstanza{\tiny\textenglish{....2.450}}प्रत्यक्षास्तद्‏विविक्तञ्च नान्यत् किञ्चिद् विभाव्यते ।&यत्तज्ज्ञानं । परोप्येतान् भुञ्चीतांन्येन विद्यदि ॥ ४५० ॥\&[\smallbreak]


	
	  \endgroup
	

	  \pstart ततः {\color{DodgerBlue3}“प्रत्यक्षाः”} । न च सामान्येन वेदनमिति स्ववेदनतैव । {\color{DodgerBlue3}“न च ते”}भ्यः सुखादिभ्यो {\color{DodgerBlue3}“विविक्तं”} भिन्न{\color{DodgerBlue3}“मन्यत् किञ्चिद्”} बृद्धिस्वरूपं {\color{DodgerBlue3}“विभाव्यते”} उपलभ्यते {\color{DodgerBlue3}“यत्तत्”} प्रत्यक्षं {\color{DodgerBlue3}“ज्ञानं”} स्यात् । कि{\color{DodgerBlue3}“ञ्चान्येन”} ज्ञानेनान्यस्य ज्ञानस्य {\color{DodgerBlue3}“वित्”} वेदनं यदीष्यते तदा भोक्तृसन्तानवर्त्तिन {\color{DodgerBlue3}“एता”}न् सुखादीन् {\color{DodgerBlue3}“परः”} प्रतिपत्ताऽल\edlabel{pvv.252-3}\footnote{\label{pvv.252-3}  ३ पाश्चात्यज्ञानं तदालम्बते न स्ववेदनं तदनुमातृसन्तानेप्यस्ति ।} भ्यमानो {\color{DodgerBlue3}“भुञ्जीत”} सुखाद्युपभोगवान्भवेत् भोक्तृपुरुषवत् । (४५०)
	\pend
      \label{div_pvv.2.451}\edlabel{div_pvv.2.451}
	  
	% new div opening: depth here is 2
	
	  \bigskip
	  \begingroup
	  \large
	
	    
	    \stanza[\smallbreak]
	\label{pv.2.451}\edlabel{pv.2.451}\flagstanza{\tiny\textenglish{....2.451}}तज्जा तत्प्रतिभासा वा यदि धीर्वेत्ति नापरा ।&आलम्बमानस्यान्यस्याप्यस्त्यवश्यमिदं द्वयम् ॥ ४५१ ॥\&[\smallbreak]


	
	  \endgroup
	

	  \pstart {\color{DodgerBlue3}“तस्मा”}त्सुखादेर्जाता {\color{DodgerBlue3}“तत्प्रतिभासा”} सुखादिप्रतिभासा {\color{DodgerBlue3}“वा”} तान् सुखादीन् {\color{DodgerBlue3}“वेत्ति”} भोक्तृत्वेन {\color{DodgerBlue3}“नाप”}रायाः काश्िचद् बु\edlabel{pvv.252-4}\footnote{\label{pvv.252-4}  ४ तन्नान्यस्य भोगः ।}द्धिरिति यदीष्यते तदा भोक्तृसन्तानवर्त्तिनः सुखादीना{\color{DodgerBlue3}“लम्बमानस्यान्यस्य”} पुरुषान्तरज्ञान{\color{DodgerBlue3}“स्येदं”} तज्जत्वं तत्प्रतिभासि\edlabel{pvv.252-5}\footnote{\label{pvv.252-5}  ५ उत्पत्तिसारूप्याभ्यामालम्बनव्यवस्थानात् ।}त्वं {\color{DodgerBlue3}“द्वय”}मालम्बनीयसुखाद्यपेक्षयाप्य{\color{DodgerBlue3}“दश्यमस्ति”} ततः सोपि भोक्ता स्यात् । (४५१)
	\pend
      \label{div_pvv.2.452_2.453}\edlabel{div_pvv.2.452_2.453}
	  
	% new div opening: depth here is 2
	
	  \bigskip
	  \begingroup
	  \large
	
	    
	    \stanza[\smallbreak]
	\label{pv.2.452a}\edlabel{pv.2.452a}\flagstanza{\tiny\textenglish{...2.452a}}अथ नोत्पद्यते तस्मान्न च तत्प्रतिभासिनी ।&सा धोर्निर्विषया प्राप्ता;\&[\smallbreak]


	
	  \endgroup
	

	  \pstart {\color{DodgerBlue3}“अथा”}न्यस्य धीर्भोक्तृसुखादेः सकाशा{\color{DodgerBlue3}“न्नोत्पद्यते नापि तत्प्रतिभासिनी”}ष्य\edlabel{pvv.252-6}\footnote{\label{pvv.252-6}  ६ सुखाद्यनालम्बनत्वात् ।}ते तदा {\color{DodgerBlue3}“सा धीर्निर्विषया प्राप्ता”} । ग्राह्यस्य पररूपस्याभावेपि प्रकाशमाना स्वप्रकाशैव स्यात् ।
	\pend
      

	  \pstart स्यादेतत् (।) भोक्तुः सुखं यद्यपि स्वरूपेण परबुद्ध्या न गृह्यते तत्सामान्यमात्रं तु गृह्यते इति भोक्तृत्वनिरालम्बनत्वयोरभाव इत्याह (।)
	\pend
      
	  \bigskip
	  \begingroup
	  \large
	
	    
	    \stanza[\smallbreak]
	\label{pv.2.452b}\edlabel{pv.2.452b}\flagstanza{\tiny\textenglish{...2.452b}}सामान्यं च तदग्रहे ॥ ४५२ ॥\&[\smallbreak]


	
	  \endgroup
	
	  \bigskip
	  \begingroup
	  \large
	
	    
	    \stanza[\smallbreak]
	\label{pv.2.453a}\edlabel{pv.2.453a}\flagstanza{\tiny\textenglish{...2.453a}}न गृह्यत इति प्रोक्तं;\&[\smallbreak]


	
	  \endgroup
	\leavevmode\marginnote{\textenglish{253/s}}

	  \pstart {\color{DodgerBlue3}“तस्य”} भोक्तृसुखविशेषस्या{\color{DodgerBlue3}“ग्रहे”} (४५२) तत्समवायि {\color{DodgerBlue3}“सामान्यं न गृह्यत इति प्रोक्तं”} । “अतत्समानता व्यक्ती तेन नित्योपलम्भनमि” \cref{pv.2.20} त्यादिना ।
	\pend
      
	  \bigskip
	  \begingroup
	  \large
	
	    
	    \stanza[\smallbreak]
	\label{pv.2.453b}\edlabel{pv.2.453b}\flagstanza{\tiny\textenglish{...2.453b}}न च तद्वस्तु किञ्चन ।&तस्मादर्थावभासोसौ नान्यस्तस्या धियस्ततः ॥ ४५३ ॥\&[\smallbreak]


	
	  \endgroup
	

	  \pstart न च तत्सामान्यं {\color{DodgerBlue3}“किञ्चन \edlabel{pvv.253-1}\footnote{\label{pvv.253-1}  १ तदा व्यतिरेकाव्यतिरेकोभयरूपेणायोगः प्रागृक्त एव ।}वस्तु”} (।)यथा ह्युपगतस्यानुपलम्भबाधितत्वात् ।
	\pend
      

	  \pstart {\color{DodgerBlue3}“तस्माद”}नन्तरोक्ताद् युक्तिकलापात् {\color{DodgerBlue3}“अर्थावभासोसौ”} स्फृटं प्रकाशमान{\color{DodgerBlue3}“स्तस्याः”} परोक्षत्वेनेष्टाया {\color{DodgerBlue3}“धियो नान्यः”} किन्तु तद्रूप एव (। ४५३)
	\pend
      \label{div_pvv.2.454}\edlabel{div_pvv.2.454}
	  
	% new div opening: depth here is 2
	
	  \bigskip
	  \begingroup
	  \large
	
	    
	    \stanza[\smallbreak]
	\label{pv.2.454}\edlabel{pv.2.454}\flagstanza{\tiny\textenglish{....2.454}}सिद्धे प्रत्यक्षभावात्मविदौ । गृह्णाति तत्पुनः ।&नाध्यक्षमिति चेदेष कुतो भेदः समार्थयोः ॥ ४५४ ॥\&[\smallbreak]


	
	  \endgroup
	

	  \pstart ततोऽर्थाभासज्ञानयोस्तादात्म्यात् {\color{DodgerBlue3}“प्रत्यक्षभावात्मविदौ”} प्रत्यक्षत्वस्वसं{\color{DodgerBlue3}“वित्ती सिद्धे”} । ज्ञानस्यापरोक्षतया परनिरपेक्षप्रकाशत्वाच्च ।
	\pend
      

	  \pstart स्यादेतत् (।) स्वसन्तानवर्त्तिनः सुखादीनध्य{\color{DodgerBlue3}“क्षमालम्बते”} ततः प्रीतिपरितापादियोगाद् भोक्तृता (।)
	\pend
      

	  \pstart अन्यस्य {\color{DodgerBlue3}“पुन”}स्तान् सुखादीन् {\color{DodgerBlue3}“नाध्यक्षं\edlabel{pvv.253-2}\footnote{\label{pvv.253-2}  २ अन्यस्य ।} गृह्णाति”} किन्तु बुद्धिमात्रं\edlabel{pvv.253-3}\footnote{\label{pvv.253-3}  ३ अनुमा ।} ततः प्रीतिपरितापाद्यभावात् न भोक्तृत्त्वमिति चेत् । {\color{DodgerBlue3}“समार्थयो”}रेकविषययोः स्वपरसन्तानवर्त्तिनोर्ज्ञानयोरेष प्रत्यक्षाप्रत्यक्षलक्षणः {\color{DodgerBlue3}“कुतो\edlabel{pvv.253-4}\footnote{\label{pvv.253-4}  ४ अस्ववेदने ।} भेदः”}। सुखस्वरूपविषयत्वात् द्वयमपि प्रत्यक्षमप्रत्यक्षम्वा स्यात् । (४५४)
	\pend
      \label{div_pvv.2.455}\edlabel{div_pvv.2.455}
	  
	% new div opening: depth here is 2
	
	  \bigskip
	  \begingroup
	  \large
	
	    
	    \stanza[\smallbreak]
	\label{pv.2.455}\edlabel{pv.2.455}\flagstanza{\tiny\textenglish{....2.455}}अदृष्टैकार्थयोगादेः सम्विदो नियमो यदि ।&सर्वथान्यो न गृह्णीयात्सम्विद्भेदोप्यपोदितः ॥ ४५५ ॥\&[\smallbreak]


	
	  \endgroup
	

	  \pstart {\color{DodgerBlue3}“अदृष्टा”}च्छुभाशुभादिलक्षणादे{\color{DodgerBlue3}“कार्थ”}समवायादेर्व्वा निमित्तात्स्वसन्तानवर्त्तिसुखग्राहिकायाः {\color{DodgerBlue3}“सम्विदो नियमो”}\edlabel{pvv.253-5}\footnote{\label{pvv.253-5}  ५ नान्यस्यैकात्मसमवायः । ईश्वरप्रसाद आदिना ।} भोगरूपत्वावधारणं । तेनान्यस्य न भोक्तृतेति यदीष्यते तदाऽदृष्टेनैकार्थसमवायेन वा नियमितं सुखाद्य{\color{DodgerBlue3}“न्यो न गृह्णीया”}देवेत्यस्तु स्वरूपप्रतिभासे\edlabel{pvv.253-6}\footnote{\label{pvv.253-6}  ६ परस्य स्वीकृते ।} भोक्तृत्वस्याप्रतिषेधात् । न ह्यात्मसमवायितामात्रेण सुखादेर्भोगः किन्तर्ह्युपलम्भेन । स च परस्याप्यस्तीति भोक्ता स्यात् । एकस्य विषयस्य {\color{DodgerBlue3}“सम्विद्भेदो”} ग्रहणभेदे{\color{DodgerBlue3}“पि”} व्यक्ताव्यक्ततया {\color{DodgerBlue3}“उदितो”} निराकृतः । ततः स्वरूपप्रतिभासस्यैकप्रकारत्वात् । (४५५)
	\pend
      \leavevmode\marginnote{\textenglish{254/s}}\label{div_pvv.2.456}\edlabel{div_pvv.2.456}
	  
	% new div opening: depth here is 2
	
	  \bigskip
	  \begingroup
	  \large
	
	    
	    \stanza[\smallbreak]
	\label{pv.2.456}\edlabel{pv.2.456}\flagstanza{\tiny\textenglish{....2.456}}येषाञ्च योगिनोन्यस्य प्रत्यक्षेण सुखादिकम् ।&विदन्ति तुल्यानुभवास्तद्वत्तेपि स्युरातुराः ॥ ४५६ ॥\&[\smallbreak]


	
	  \endgroup
	

	  \pstart {\color{DodgerBlue3}“येषाञ्च”}\edlabel{pvv.254-1}\footnote{\label{pvv.254-1}  १ वैभाष्यादीनां ।} परेषां कारणादीनां {\color{DodgerBlue3}“योगिनोन्यस्य सुखादिकं प्रत्यक्षेण”} योगबलोत्पन्नेन {\color{DodgerBlue3}“विदन्ती”}ति मतं । तेषां मते परेण सुखिना दुःखिना च सह {\color{DodgerBlue3}“तुल्यानुभवा”} योगिन इति {\color{DodgerBlue3}“तद्वत्”} दुःखिपुरुषवत् योगिनो{\color{DodgerBlue3}“प्यातुरा”} दुःखपीडिताः स्युः । (४५६)
	\pend
      \label{div_pvv.2.457}\edlabel{div_pvv.2.457}
	  
	% new div opening: depth here is 2
	
	  \bigskip
	  \begingroup
	  \large
	
	    
	    \stanza[\smallbreak]
	\label{pv.2.457}\edlabel{pv.2.457}\flagstanza{\tiny\textenglish{....2.457}}विषयेन्द्रियसम्पाताभावात्तेषां तदुद्भवम् ।&नोदेति दुःखमिति चेत् न वै दुःखसमुद्भवः ॥ ४५७ ॥\&[\smallbreak]


	
	  \endgroup
	\leavevmode\marginnote{\textenglish{50a/MA}}

	  \pstart {\color{DodgerBlue3}“विषयेन्द्रिययोः सम्पात”}स्य संसर्गस्या{\color{DodgerBlue3}“भावात् । तदुद्भवं”} विषयेन्द्रियसंसर्गजं {\color{DodgerBlue3}“दुःखं तेषां”} योगिनां {\color{DodgerBlue3}“नोदेतीति चेत् । न वै”} नैव {\color{DodgerBlue3}“दुःख”}स्य {\color{DodgerBlue3}“समुद्भव”} उत्पत्तिः (४५७)
	\pend
      \label{div_pvv.2.458}\edlabel{div_pvv.2.458}
	  
	% new div opening: depth here is 2
	
	  \bigskip
	  \begingroup
	  \large
	
	    
	    \stanza[\smallbreak]
	\label{pv.2.458}\edlabel{pv.2.458}\flagstanza{\tiny\textenglish{....2.458}}दुःखस्य वेदनं किन्तु दुःखज्ञानसमुद्भवः ।&न हि दुःखाद्यसंवेद्यं पीडानुग्रहकारणम् ॥ ४५८ ॥\&[\smallbreak]


	
	  \endgroup
	

	  \pstart {\color{DodgerBlue3}“दुःखस्य वेदनं”} दुःखित्वं । {\color{DodgerBlue3}“किन्तु दुःख”}विषय{\color{DodgerBlue3}“ज्ञानसमुद्भवो”} दुःखिता (।) न हि {\color{DodgerBlue3}“दुःखं आदि”}शब्दात् सुख{\color{DodgerBlue3}“मसम्वेद्यं”} अज्ञायमानं {\color{DodgerBlue3}“पीडानुग्रहयोः कारणं”} भवति येन दुःखसुखयोरूत्पत्ती दुःखितासुखिते स्यातां । (४५८)
	\pend
      \label{div_pvv.2.459}\edlabel{div_pvv.2.459}
	  
	% new div opening: depth here is 2
	

	  \begin{center}%% label @type='head'
	\textbf{(ग. स्वसंवेदननये योगिनामनातुरता)}
	\end{center}
	

	  \pstart ननु बौद्धस्यापि मते योगिनः सुखाद्याकारेण ज्ञानेन परदुःखमालम्बमानाः कस्मादातुरा न भवन्ति दुःखिन इव योगिनोपि दुःखा\edlabel{pvv.254-2}\footnote{\label{pvv.254-2}  २ परचित्ताभिज्ञया मनसा सुखादियुतं मनो वेत्ति योगी यथाग्न्यादयः स्वयमसञ्चरन्तः स्वज्ञानेध्यक्षास्तदाभासमात्रेण । न हि दुःखादीति वृत्तौ सिद्धान्ते योजितं ।}कारं स्वसम्वेदनञ्च ज्ञानमिति न कश्चिद्विशेष इत्याह (।)
	\pend
      
	  \bigskip
	  \begingroup
	  \large
	
	    
	    \stanza[\smallbreak]
	\label{pv.2.459}\edlabel{pv.2.459}\flagstanza{\tiny\textenglish{....2.459}}भासमानं स्वरूपेण पीडा दुःखं स्वयं यदा ।&न तदालम्बनं ज्ञानं न तदैवं प्रयुज्यते ॥ ४५९ ॥\&[\smallbreak]


	
	  \endgroup
	

	  \pstart {\color{DodgerBlue3}“दुःखं स्वयं”}\edlabel{pvv.254-3}\footnote{\label{pvv.254-3}  ३ स्वसन्ततिजं ।} परनिरपेक्षप्रकाशं {\color{DodgerBlue3}“स्वरूपेण”} प्रकाशस्वभावेन {\color{DodgerBlue3}“भासमानं पीडा”} (।) तत्तस्मादुत्पन्नं तत्सरूपं {\color{DodgerBlue3}“तदालम्बनं”} योगिनो {\color{DodgerBlue3}“ज्ञान”}न्न पीडेति यदा बौद्धैरिष्यते {\color{DodgerBlue3}“तदैवं”} योगिनोपि परदुःखालम्बका दुःखिनः स्युरिति {\color{DodgerBlue3}“न युज्यते”} येनायं भेदः (४५९)
	\pend
      \leavevmode\marginnote{\textenglish{255/s}}\label{div_pvv.2.460}\edlabel{div_pvv.2.460}
	  
	% new div opening: depth here is 2
	
	  \bigskip
	  \begingroup
	  \large
	
	    
	    \stanza[\smallbreak]
	\label{pv.2.460}\edlabel{pv.2.460}\flagstanza{\tiny\textenglish{....2.460}}भिन्ने ज्ञानस्य सर्व्वस्य तेनालम्बनवेदने ।&अर्थसारूप्यमालम्ब आत्मा वित्तिः स्वयं स्फुटा ॥ ४६० ॥\&[\smallbreak]


	
	  \endgroup
	

	  \pstart तेन {\color{DodgerBlue3}“सर्व्वस्य ज्ञानस्यालम्बनवेदने भिन्ने”} भिन्नलक्षणे तथा {\color{DodgerBlue3}“ह्यर्थ\edlabel{pvv.255-1}\footnote{\label{pvv.255-1}  १ अर्थान्तरात्सारूप्येणोत्पत्तिः ।}सारूप्यमालम्ब”} आलम्बनार्थः । {\color{DodgerBlue3}“आत्मा स्वयं”} प\edlabel{pvv.255-2}\footnote{\label{pvv.255-2}  २ ग्राह्यग्राहकत्वादिजात्यादिरहितः ।}रनिरपेक्षः {\color{DodgerBlue3}“स्फुटा वित्ति”}\edlabel{pvv.255-3}\footnote{\label{pvv.255-3}  ३ स्वसम्वेदनरूपोत्पत्तिः}र्व्वेदनार्थः । (४६०)
	\pend
      \label{div_pvv.2.461}\edlabel{div_pvv.2.461}
	  
	% new div opening: depth here is 2
	
	  \bigskip
	  \begingroup
	  \large
	
	    
	    \stanza[\smallbreak]
	\label{pv.2.461}\edlabel{pv.2.461}\flagstanza{\tiny\textenglish{....2.461}}अपि चाध्यक्षताऽभावे धियः स्याल्लिङ्गतो गतिः ॥&तच्चाक्षमर्थो धीः पूर्व्वो मनस्कारोपि वा भवेत् ॥ ४६१ ॥\&[\smallbreak]


	
	  \endgroup
	

	  \pstart {\color{DodgerBlue3}“अपि च”} धियो{\color{DodgerBlue3}“ऽध्यक्षताऽभावे लिङ्गतो गतिः स्यात् । तच्च”} लिङ्गम्भवदक्षमिन्द्रिय{\color{DodgerBlue3}“मर्थो”} विषयो {\color{DodgerBlue3}“धी”}\edlabel{pvv.255-4}\footnote{\label{pvv.255-4}  ४ धीः स्वयमिति वृत्तिः । यानुमेया सैव प्रकारान्तरेण लिङ्गं कल्प्येत यथा यत्कृतकं तदेवानित्यं । अनन्तरानुमानमेव युक्तं ।}रनन्तरा {\color{DodgerBlue3}“पूर्व्व”}को {\color{DodgerBlue3}“मनस्कारो वा भवेत् ।”} (४६१)
	\pend
      \label{div_pvv.2.462_2.463}\edlabel{div_pvv.2.462_2.463}
	  
	% new div opening: depth here is 2
	
	  \bigskip
	  \begingroup
	  \large
	
	    
	    \stanza[\smallbreak]
	\label{pv.2.462}\edlabel{pv.2.462}\flagstanza{\tiny\textenglish{....2.462}}कार्यकारणसामग्रयामस्यां सम्बन्धि नापराम् ।&सामर्थ्यादर्शनात्तत्र नेन्द्रियं व्यभिचारतः ॥ ४६२ ॥\&[\smallbreak]


	
	  \endgroup
	
	  \bigskip
	  \begingroup
	  \large
	
	    
	    \stanza[\smallbreak]
	\label{pv.2.463}\edlabel{pv.2.463}\flagstanza{\tiny\textenglish{....2.463}}तथार्थो धीमनस्कारौ ज्ञानं तौ च न सिध्यतः ।&नाप्रसिद्धस्य लिङ्गत्वं व्यक्तिरर्थस्य चेन्मता ॥ ४६३ ॥\&[\smallbreak]


	
	  \endgroup
	
	  \bigskip
	  \begingroup
	  \large
	
	    
	    \stanza[\smallbreak]
	\label{pv.2.464a}\edlabel{pv.2.464a}\flagstanza{\tiny\textenglish{...2.464a}}लिङ्गं;\&[\smallbreak]


	
	  \endgroup
	

	  \pstart यस्मा{\color{DodgerBlue3}“त्कार्यकारणसामग्रयामस्या”}मेभ्यो{\color{DodgerBlue3}“ऽपरमा”}त्मनः संयोगादि{\color{DodgerBlue3}“सम्बन्धि नास्ति सामर्थ्यादर्शनात् । तत्र”} तेष्वि{\color{DodgerBlue3}“\edlabel{pvv.255-5}\footnote{\label{pvv.255-5}  ५ सुप्तमूर्च्छादौ ।}न्द्रियं”} तावन्न लिङ्गं {\color{DodgerBlue3}“व्यभिचारतः”} सत्यपि तस्मिन् ज्ञानाभावात् । (४६२) {\color{DodgerBlue3}“तथार्थोपि”} ज्ञानव्यभिचारान्न लिङ्गम् । {\color{DodgerBlue3}“धीमनस्कारौ”} बोधस्वभावत्वात् {\color{DodgerBlue3}“ज्ञानं तौ च”} लीङ्गज्ञाना\edlabel{pvv.255-6}\footnote{\label{pvv.255-6}  ६ सर्व्वं लिङ्गमात्मनि ज्ञानापेक्षं लिङ्गिबोधोपायो भवति ।}त्प्राङ् {\color{DodgerBlue3}“न सिध्यतः”} ज्ञानस्यानुमेयत्वात् (।) न चा{\color{DodgerBlue3}“प्रसिद्धस्य”} निश्चितस्य {\color{DodgerBlue3}“लिङ्गत्वं”} सत्तामात्रेण लिङ्गत्वेऽतिप्रसङ्गात् । {\color{DodgerBlue3}“अर्थस्य व्यक्तिः”} स्फुटता बुर्द्धर्लिङ्गं {\color{DodgerBlue3}“मता चेत्”} (४६३)
	\pend
      \label{div_pvv.2.464}\edlabel{div_pvv.2.464}
	  
	% new div opening: depth here is 2
	
	  \bigskip
	  \begingroup
	  \large
	
	    
	    \stanza[\smallbreak]
	\label{pv.2.464b}\edlabel{pv.2.464b}\flagstanza{\tiny\textenglish{...2.464b}}सैव ननु ज्ञानं व्यक्तोर्थोनेन वर्ण्णितः ।&व्यक्तावननुभूतायां तद्व्यक्तत्वाविनिश्चयात् ॥ ४६४ ॥\&[\smallbreak]


	
	  \endgroup
	

	  \pstart {\color{DodgerBlue3}“ननु सैव”} व्यक्तिर्ज्ञानमर्थप्रकाशलक्षणत्वात् । न च तदेव लिङ्गि चेति युक्तं । अनेन व्यक्तेर्लिङ्गत्वकथनेन {\color{DodgerBlue3}“व्यक्तोऽर्थो वर्ण्णितः”} प्रतिक्षिप्तः । तथा हि {\color{DodgerBlue3}“व्यक्तौ”} \leavevmode\marginnote{\textenglish{256/s}} बृद्धिरूपायाम{\color{DodgerBlue3}“ननुभूताया”}मर्थसम्बन्धिन{\color{DodgerBlue3}“स्तद्ब्यक्तत्वस्य”} व्यक्तिव्यक्तत्वस्या{\color{DodgerBlue3}“विनिश्चयान्न”} लि\edlabel{pvv.256-1}\footnote{\label{pvv.256-1}  १ व्यक्तोऽर्थो लिङ्गन्नार्थमात्रमिति यत्कल्प्यते तस्य ।}ङ्गत्वं सत्तामात्रेण लिङ्गत्वेऽतिप्रसङ्गात् । (४६४)
	\pend
      \label{div_pvv.2.465}\edlabel{div_pvv.2.465}
	  
	% new div opening: depth here is 2
	
	  \bigskip
	  \begingroup
	  \large
	
	    
	    \stanza[\smallbreak]
	\label{pv.2.465a}\edlabel{pv.2.465a}\flagstanza{\tiny\textenglish{...2.465a}}अथार्थस्यैव कश्चित्स विशेषो व्यक्तिरिष्यते ।\&[\smallbreak]


	
	  \endgroup
	

	  \pstart {\color{DodgerBlue3}“अथार्थस्यैव”} स्वभावभूतः {\color{DodgerBlue3}“स कश्चित्”} स्वभाव{\color{DodgerBlue3}“विशेषो व्यक्तिरिष्यते”} न ज्ञानं ।
	\pend
      
	  \bigskip
	  \begingroup
	  \large
	
	    
	    \stanza[\smallbreak]
	\label{pv.2.465b}\edlabel{pv.2.465b}\flagstanza{\tiny\textenglish{...2.465b}}नानुत्पादव्ययवतो विशेषोऽर्थस्य कश्चन ॥ ४६५ ॥\&[\smallbreak]


	
	  \endgroup
	

	  \pstart तद{\color{DodgerBlue3}“प्यर्थस्य”} स्थिरैकरूपत्वाद{\color{DodgerBlue3}“नुत्पादव्ययवतो\edlabel{pvv.256-2}\footnote{\label{pvv.256-2}  २ यत् केनचिज्ज्ञानेन गृह्यमाणेस्य व्यक्तिर्व्यवस्थितिः ।} विशेषो”} व्यक्तिरूपः {\color{DodgerBlue3}“कश्चन”} न सङ्गतः । (४६५)
	\pend
      \label{div_pvv.2.466}\edlabel{div_pvv.2.466}
	  
	% new div opening: depth here is 2
	
	  \bigskip
	  \begingroup
	  \large
	
	    
	    \stanza[\smallbreak]
	\label{pv.2.466}\edlabel{pv.2.466}\flagstanza{\tiny\textenglish{....2.466}}तदिष्टौ वा प्रतिज्ञानं क्षणभङ्गः प्रसज्यते ।&स च ज्ञातोऽथवाऽज्ञातो भवेज्ज्ञातस्य लिङ्गता ॥ ४६६ ॥\&[\smallbreak]


	
	  \endgroup
	

	  \pstart तस्य विशेष{\color{DodgerBlue3}“स्येष्टौ वा प्रतिज्ञान”}मर्थस्य पूर्व्वस्वभावनाशे सति स्वभावान्तरोत्पादात् {\color{DodgerBlue3}“क्षणभङ्गः प्रसज्यते”} (।) {\color{DodgerBlue3}“स चा”}र्थस्वभावविशेषो {\color{DodgerBlue3}“ज्ञातोऽज्ञातो वा भवेत्”} लिङ्गं ज्ञानस्य तत्र {\color{DodgerBlue3}“ज्ञातस्य यदि लीङ्गते”}ष्यते (। ४६६)
	\pend
      \label{div_pvv.2.467}\edlabel{div_pvv.2.467}
	  
	% new div opening: depth here is 2
	

	  \pstart तदा (।)
	\pend
      
	  \bigskip
	  \begingroup
	  \large
	
	    
	    \stanza[\smallbreak]
	\label{pv.2.467}\edlabel{pv.2.467}\flagstanza{\tiny\textenglish{....2.467}}यदि ज्ञानेऽपरिच्छिन्ने ज्ञातोसाविति तत्कुतः ।&ज्ञातत्वेनापरिच्छिन्नमपि तद् गमकं कथम् ॥ ४६७ ॥\&[\smallbreak]


	
	  \endgroup
	

	  \pstart {\color{DodgerBlue3}“ज्ञानेऽपरिच्छिन्ने”} तदुपाधिर्ज्ञातोसावर्थों लिङ्ग\edlabel{pvv.256-3}\footnote{\label{pvv.256-3}  ३ न कारकवज्‏ज्ञापकोऽज्ञातोपि कार्यकारी ।}मिति यदिष्टं तत्कुत उपपद्यते (।) अथाज्ञातस्य लिङ्गता तदा {\color{DodgerBlue3}“ज्ञातत्वेनापरिच्छिन्नमपि”} तद्वस्तु {\color{DodgerBlue3}“कथं गमकं”} लिङ्गं सत्तामात्रेण गमकत्वेऽतिप्रसङ्गादित्युक्तं । न च ज्ञानादर्शने दृष्टता युक्ता । (४६७)
	\pend
      \label{div_pvv.2.468_2.469}\edlabel{div_pvv.2.468_2.469}
	  
	% new div opening: depth here is 2
	
	  \bigskip
	  \begingroup
	  \large
	
	    
	    \stanza[\smallbreak]
	\label{pv.2.468a}\edlabel{pv.2.468a}\flagstanza{\tiny\textenglish{...2.468a}}अदृष्टादृष्टयोन्येन द्रष्ट्रा दृष्टा न हि क्वचित् ।\&[\smallbreak]


	
	  \endgroup
	

	  \pstart {\color{DodgerBlue3}“हिर्य”}स्माद{\color{DodgerBlue3}“दृष्टा दृष्टि”}र्ज्ञानं येषां तेऽर्थाः {\color{DodgerBlue3}“क्वचि”}दन्येन {\color{DodgerBlue3}“द्रष्ट्रा दृष्टा”} इति न दृष्टा निश्चयविषयाः स्युः ।
	\pend
      

	  \pstart अथार्थस्यै (व) वि\edlabel{pvv.256-4}\footnote{\label{pvv.256-4}  ४ एकदृष्टावप्यन्यस्यास्त्यर्थस्याज्ञातो विशेषः ।}शेषः कश्चिद् बुद्धिकृत आस्ते तेन बुद्ध्यनुमानमित्याह (।)
	\pend
      
	  \bigskip
	  \begingroup
	  \large
	
	    
	    \stanza[\smallbreak]
	\label{pv.2.468b}\edlabel{pv.2.468b}\flagstanza{\tiny\textenglish{...2.468b}}विशेषः सोन्यदृष्टावप्यस्तीति स्यात्स्वधीगतिः ॥ ४६८ ॥\&[\smallbreak]


	
	  \endgroup
	\leavevmode\marginnote{\textenglish{257/s}}

	  \pstart {\color{DodgerBlue3}“स विशेषो”}ऽर्थस्यान्येन पुरुषेण {\color{DodgerBlue3}“दृष्टावप्यस्तीति”} पुरुषान्तरस्यातद्व्यापृतेन्द्रियस्य तस्मादर्थगतविशेषात् {\color{DodgerBlue3}“स्वधीग\edlabel{pvv.257-1}\footnote{\label{pvv.257-1}  १ तमर्थमपश्यतोपि स्वबुद्ध्यनुमानं स्यान्न च युक्तं ।}तिः स्यात्”} । (४६८)
	\pend
      \leavevmode\marginnote{\textenglish{50b/MA}}

	  \pstart अथ धर्मस्य साधारणत्वात् तस्य बुद्ध्यव्यभिचारात् ।
	\pend
      
	  \bigskip
	  \begingroup
	  \large
	
	    
	    \stanza[\smallbreak]
	\label{pv.2.469}\edlabel{pv.2.469}\flagstanza{\tiny\textenglish{....2.469}}तस्मादनुमितिर्बुद्धेः स्वधर्मनिरपेक्षिणः ।&केवलान्नार्थधर्मात्कः; स्वधर्मः स्वधियो परः ॥ ४६९ ॥\&[\smallbreak]


	
	  \endgroup
	

	  \pstart {\color{DodgerBlue3}“तस्मात् केवलादर्थधर्मात् स्वधर्मनिरपेक्षिणोऽ”}नुमातृपुरुषात्मभूतज्ञाननिरपेक्षाद्व्ुद्धेरनुमितिर्न सम्भवति सर्व्वस्यैव तस्मात् स्वबुद्ध्यनुमानप्रसङ्गात् । अथात्मधर्म ए\edlabel{pvv.257-2}\footnote{\label{pvv.257-2}  २ पुरुषस्य ।}व स कश्चिद् बुद्धेर्गमक इति चेत् । आह (।) {\color{DodgerBlue3}“स्वस्या”}त्मनो {\color{DodgerBlue3}“धर्मः स्व”}बुद्धेरात्मसम्बन्धिन्या बुद्धेरपरोन्यः कः (। ४६९)
	\pend
      \label{div_pvv.2.470}\edlabel{div_pvv.2.470}
	  
	% new div opening: depth here is 2
	
	  \bigskip
	  \begingroup
	  \large
	
	    
	    \stanza[\smallbreak]
	\label{pv.2.470}\edlabel{pv.2.470}\flagstanza{\tiny\textenglish{....2.470}}प्रत्यक्षाधिगतो हेतुः तुल्यकारणजन्मनः ।&तस्य भेदः कुतो बुद्धेर्व्यभिचार्यन्यजश्च सः ॥ ४७० ॥\&[\smallbreak]


	
	  \endgroup
	

	  \pstart {\color{DodgerBlue3}“प्रत्यक्षाधिगतो हेतुः”} स्यात् । न ह्यप्रतीतस्य हेतुता । न च बुद्धेः प्रत्यक्षतेष्यते तद्व्यतिरिक्तिश्च कश्चिदात्मधर्मो न प्रत्यक्ष इति न स्याद् बुद्ध्यनुमानं (।) किञ्चात्मधर्मोसौ बुद्ध्या सममेककारणो वा स्यात् भिन्नकारणो वा (।) तत्र बुद्ध्या सह {\color{DodgerBlue3}“तुल्यात् कारणाज्जन्म”} यस्य {\color{DodgerBlue3}“तस्या”}त्मधर्मस्य {\color{DodgerBlue3}“बुद्धेः”} सकाशात् {\color{DodgerBlue3}“कुतो भेदः”} । अभिन्नहेतुकत्वेऽभिन्नतैव युक्ता । अथान्यहेतुकोसौ तदा{\color{DodgerBlue3}“न्यजश्च स व्य\edlabel{pvv.257-3}\footnote{\label{pvv.257-3}  ३ बुद्ध्यसिद्धेर्न तादात्म्यं नापि तदुत्पत्तिरनुमेयधिया ।}भिचारी”} स्यात् । एकसामग्रयधीनयोरेकदर्शनादपरानुमानमव्यभिचारि नान्यथा ।\edlabel{pvv.257-4}\footnote{\label{pvv.257-4}  ४ बुद्धेर्लिङ्गादनुमाने तल्लिङ्गेन कार्येण भाव्यमिति ।}(४७०)
	\pend
      \label{div_pvv.2.471}\edlabel{div_pvv.2.471}
	  
	% new div opening: depth here is 2
	

	  \pstart उक्तमेवार्थं संगृह्णन्नाह (।)
	\pend
      
	  \bigskip
	  \begingroup
	  \large
	
	    
	    \stanza[\smallbreak]
	\label{pv.2.471}\edlabel{pv.2.471}\flagstanza{\tiny\textenglish{....2.471}}रूपादीन् पञ्चविषयानिन्द्रियाण्युपलम्भनम् ।&मुक्त्त्वा न कार्यमपरं तस्याः समुपलभ्यते ॥ ४७१ ॥\&[\smallbreak]


	
	  \endgroup
	

	  \pstart {\color{DodgerBlue3}“रूप”}मादिर्येषां तान् शब्दगन्धरसस्पर्शन् {\color{DodgerBlue3}“पञ्चविषयान्”} पञ्चे{\color{DodgerBlue3}“न्द्रियाणि”} चक्षुःश्रोत्रादीनि {\color{DodgerBlue3}“उपलम्भनं”} ज्ञानं {\color{DodgerBlue3}“मुक्त्वा तस्या”} बुद्धेर्न {\color{DodgerBlue3}“कार्यमपरं समुपलभ्यते ।”} इयतैव सर्व्वस्य संग्रहात् । (४७१)
	\pend
      \leavevmode\marginnote{\textenglish{258/s}}\label{div_pvv.2.472}\edlabel{div_pvv.2.472}
	  
	% new div opening: depth here is 2
	
	  \bigskip
	  \begingroup
	  \large
	
	    
	    \stanza[\smallbreak]
	\label{pv.2.472}\edlabel{pv.2.472}\flagstanza{\tiny\textenglish{....2.472}}तत्रात्यक्षं द्वयं । पञ्चस्वर्थेष्वेकोपि नेक्ष्यते ।&रूपदर्शनतो जातो योन्यथा-व्यस्तसम्भवः ॥ ४७२ ॥\&[\smallbreak]


	
	  \endgroup
	

	  \pstart {\color{DodgerBlue3}“तत्र”} तेषु मध्ये {\color{DodgerBlue3}“द्वय”}मिन्द्रियं ज्ञानञ्चा{\color{DodgerBlue3}“त्यक्ष”}मतीन्द्रियं इन्द्रियस्य ज्ञानान्यथानुपपत्त्या व्यवस्थापनात् । ज्ञानस्य त्वन्मतेऽप्रत्यक्षत्वात् । रूपादिषु {\color{DodgerBlue3}“पञ्चस्वर्थेषु एकोपि नेक्ष्यते”} बुद्धेरप्रत्यक्षत्वात् । {\color{DodgerBlue3}“यो या”}वद् दृश्यमानोऽ{\color{DodgerBlue3}“न्यथा”} ज्ञानमन्तरेण व्य{\color{DodgerBlue3}“स्तसम्भवः”} प्रतिक्षिप्तसत्त्वो विषयस्य {\color{DodgerBlue3}“रूपदर्शनतो”} बुद्धे{\color{DodgerBlue3}“र्जातो”}ऽभ्युपगम्यते । (४७२)
	\pend
      \label{div_pvv.2.473}\edlabel{div_pvv.2.473}
	  
	% new div opening: depth here is 2
	
	  \bigskip
	  \begingroup
	  \large
	
	    
	    \stanza[\smallbreak]
	\label{pv.2.473}\edlabel{pv.2.473}\flagstanza{\tiny\textenglish{....2.473}}यदेवमप्रतीतं तल्लिङ्गमित्यतिलौकिकम् ।&विद्यमानेपि लिङ्गे तान्तेन सार्द्धमपश्यतः ॥ ४७३ ॥\&[\smallbreak]


	
	  \endgroup
	

	  \pstart {\color{DodgerBlue3}“यच्चैवं”} बुद्धिनान्तरयीकतया{\color{DodgerBlue3}“ऽप्रतीतं”} तद् बुद्धे{\color{DodgerBlue3}“र्लिङ्गमि”}त्यति{\color{DodgerBlue3}“लौकिकं”} लो \edlabel{pvv.258-1}\footnote{\label{pvv.258-1}  १ बालोपि सम्बद्धमेव लिङ्गमाह वह्नेरिव धूमः ।} कातिक्रान्तं । किञ्चाभ्युपगम्योच्यते । {\color{DodgerBlue3}“विद्यमानेपि”} कस्मिंश्चिल्लि{\color{DodgerBlue3}“ङ्गे”} कदाचिद् बुद्धिं {\color{DodgerBlue3}“तां तेन”} लि\edlabel{pvv.258-2}\footnote{\label{pvv.258-2}  २ आत्मनि बुद्धेरन्वयादृष्टेः ।}ङ्गेन {\color{DodgerBlue3}“सार्धमपश्यतो”}ऽप्रतिपत्तेः । (४७३)
	\pend
      \label{div_pvv.2.474}\edlabel{div_pvv.2.474}
	  
	% new div opening: depth here is 2
	
	  \bigskip
	  \begingroup
	  \large
	
	    
	    \stanza[\smallbreak]
	\label{pv.2.474}\edlabel{pv.2.474}\flagstanza{\tiny\textenglish{....2.474}}कथं प्रतीतिर्लिङ्गं हि नादृष्टस्य प्रकाशकम् ।&तत एवास्य लिङ्गात्प्राक् प्रसिद्धेरुपवर्ण्णने ॥ ४७४ ॥\&[\smallbreak]


	
	  \endgroup
	

	  \pstart तस्माल्लिङ्गात् {\color{DodgerBlue3}“कथं”} बुद्धि{\color{DodgerBlue3}“प्रतीतिः । लिङ्गं ह्यन्व”}यरहितम{\color{DodgerBlue3}“दृष्ट”}स्यार्थस्य {\color{DodgerBlue3}“न प्रकाश”}कं यु\edlabel{pvv.258-3}\footnote{\label{pvv.258-3}  ३ स्यादेतद् (।) येन कायस्पन्दादिनात्मनि बुर्द्धि साधयितुमिच्छति प्रमाता तत एव लिङ्गात् सपक्षे बृद्धिः सेत्स्यति तया सिद्ध्यात्मन्यनुमानमिति सपक्षेऽनुमानं विना दृष्टान्तबलेनेत्याह ।}क्तं (।) {\color{DodgerBlue3}“तत एव लिङ्गादस्य”} ज्ञानस्यान्वयसिद्ध्यर्थमात्मन्यनुमाना{\color{DodgerBlue3}“त्प्राक् सिद्धेर्निश्चयस्योपवर्ण्णने”} वाभिधीयमाने (४७४)
	\pend
      \label{div_pvv.2.475}\edlabel{div_pvv.2.475}
	  
	% new div opening: depth here is 2
	
	  \bigskip
	  \begingroup
	  \large
	
	    
	    \stanza[\smallbreak]
	\label{pv.2.475}\edlabel{pv.2.475}\flagstanza{\tiny\textenglish{....2.475}}दृष्टान्तान्तरसाध्यत्वं तस्यापीत्यनवस्थितिः ।&इत्यर्थस्य धियः सिद्धिः नार्थात्तस्याः कथञ्चन ॥ ४७५ ॥\&[\smallbreak]


	
	  \endgroup
	

	  \pstart {\color{DodgerBlue3}“तस्या”}न्वयसाधकस्याप्यनुमानस्य {\color{DodgerBlue3}“दृष्टान्तान्तरे”}णानुमानसाध्येन {\color{DodgerBlue3}“साध्यत्व\edlabel{pvv.258-4}\footnote{\label{pvv.258-4}  ४ अन्यथा सपक्षे यदि दृष्टान्तं विना सिद्धिरात्मन्यपि किन्न सिद्धिः ।}मित्यनवस्थितिः”} स्यात् । तथा चैकस्यासिद्धौ सर्व्वस्यासिद्धिः प्रसज्यते ।
	\pend
      

	  \pstart {\color{DodgerBlue3}“इति”} तस्मा{\color{DodgerBlue3}“दर्थस्य धियः”} सकाशात् {\color{DodgerBlue3}“सिद्धिर्नार्थात् तस्या”} धियः {\color{DodgerBlue3}“कथं च न”} सिद्धिरिति न्याय्यं (। ४७५)
	\pend
      \label{div_pvv.2.476}\edlabel{div_pvv.2.476}
	  
	% new div opening: depth here is 2
	
	  \bigskip
	  \begingroup
	  \large
	
	    
	    \stanza[\smallbreak]
	\label{pv.2.476}\edlabel{pv.2.476}\flagstanza{\tiny\textenglish{....2.476}}तदप्रसिद्धावर्थस्य स्वयमेवाप्रसिद्धितः ।&प्रत्यक्षाञ्च धियं दृष्ट्वा तस्याश्चेष्टाभिधादिकम् ॥ ४७६ ॥\&[\smallbreak]


	
	  \endgroup
	\leavevmode\marginnote{\textenglish{259/s}}

	  \pstart यस्मात्तस्या धियो{\color{DodgerBlue3}“ऽसिद्धावर्थस्य स्वयमेवाप्रसिद्धितः”} कथं लिङ्गता ।
	\pend
      

	  \pstart स्वप्रकाशत्वात् {\color{DodgerBlue3}“प्रत्यक्षां धियं तस्याश्च चेष्टाऽभिधाऽदिर्यस्य”} सुखप्रसादवैवर्ण्ण्यादेस्तं {\color{DodgerBlue3}“दृष्ट्वा”} गृहीतव्याप्तिकस्यान्यसम्बन्धिचेष्टादिदर्शनात् । (४७६)
	\pend
      \label{div_pvv.2.477}\edlabel{div_pvv.2.477}
	  
	% new div opening: depth here is 2
	
	  \bigskip
	  \begingroup
	  \large
	
	    
	    \stanza[\smallbreak]
	\label{pv.2.477}\edlabel{pv.2.477}\flagstanza{\tiny\textenglish{....2.477}}परचित्तानुमानञ्च न स्यादात्मन्यदर्शनात् ।&सबन्धस्य मनोबुद्धावर्थलिङ्गाप्रसिद्धितः ॥ ४७७ ॥\&[\smallbreak]


	
	  \endgroup
	

	  \pstart {\color{DodgerBlue3}“परचित्तानुमानञ्चेष्टं न स्यात्”} । बुद्धेरा{\color{DodgerBlue3}“त्मनि”} स्वसन्ततौ चेष्टादिभिः सह {\color{DodgerBlue3}“सम्बन्ध”}स्या{\color{DodgerBlue3}“दर्शनात्”} । अपि च वासनामात्रबलभाविन्या {\color{DodgerBlue3}“मनोबृद्धौ”} विकल्पबुद्धौ विषयभूतस्यार्थस्याभावात्\edlabel{pvv.259-1}\footnote{\label{pvv.259-1}  १ न विकल्पोऽर्थापेक्षः ।} {\color{DodgerBlue3}“अर्थस्य लिङ्गस्यासिद्धितो”}\edlabel{pvv.259-2}\footnote{\label{pvv.259-2}  २ अनुमानं न स्यात् । ज्ञानान्तरवेद्यपक्षेपि ।} बुद्ध्यन्तराल्लिङ्गादनुमानं स्यात् ।\edlabel{pvv.259-3}\footnote{\label{pvv.259-3}  ३ तत्रापि}(४७७)
	\pend
      \label{div_pvv.2.478}\edlabel{div_pvv.2.478}
	  
	% new div opening: depth here is 2
	
	  \bigskip
	  \begingroup
	  \large
	
	    
	    \stanza[\smallbreak]
	\label{pv.2.478}\edlabel{pv.2.478}\flagstanza{\tiny\textenglish{....2.478}}प्रकाशिता कथं वा स्यात् बुद्धिर्बुद्ध्यन्तरेण वः ।&अप्रकाशात्मनोः साम्याद् व्यङ्ग्यव्यञ्जकता कुतः ॥ ४७८ ॥\&[\smallbreak]


	
	  \endgroup
	

	  \pstart {\color{DodgerBlue3}“कथम्वा बुद्ध्यन्तरेणा”}प्रत्यक्षेण {\color{DodgerBlue3}“बुद्धिः प्रकाशिता स्यात् । वो”} युष्माकं दर्शने\edlabel{pvv.259-4}\footnote{\label{pvv.259-4}  ४ पूर्व्वात्मरूपयोत्तरबुद्ध्येति चेदाह । एकस्य विकल्पाविकल्पजननविरोधात् पूर्व्वधिया परधीबोधजनने स्मृतेरजननात्तु द्वितीयतया धिया स्वविषया(व) बोधाद्यवधानं ।} ।\leavevmode\marginnote{\textenglish{51a/MA}} यस्माद{\color{DodgerBlue3}“प्रकाशात्मनो”}र्ल्लिङ्गलिङ्गिनोरसिद्धत्वेन {\color{DodgerBlue3}“साम्यात् व्यङ्ग्यव्यञ्जकता कुतः”} । यद्यप्रकाशात्मनोर्न व्यङ्ग्यव्यञ्जकता तदार्थज्ञानयोरपि कथं व्यङ्ग्यव्यञ्जकताभाव इति । (४७८)
	\pend
      \label{div_pvv.2.479}\edlabel{div_pvv.2.479}
	  
	% new div opening: depth here is 2
	
	  \bigskip
	  \begingroup
	  \large
	
	    
	    \stanza[\smallbreak]
	\label{pv.2.479a}\edlabel{pv.2.479a}\flagstanza{\tiny\textenglish{...2.479a}}विषयस्य कथं व्यक्तिः;\&[\smallbreak]


	
	  \endgroup
	

	  \pstart {\color{DodgerBlue3}“विषयस्य कथं व्यक्ति”}रिति (।)
	\pend
      

	  \pstart उत्तरमाह (।)
	\pend
      
	  \bigskip
	  \begingroup
	  \large
	
	    
	    \stanza[\smallbreak]
	\label{pv.2.479b}\edlabel{pv.2.479b}\flagstanza{\tiny\textenglish{...2.479b}}प्रकाशे रूपसंक्रमात् ।&स च प्रकाशस्तद्रूपः स्वयमेव प्रकाशते ॥ ४७९ ॥\&[\smallbreak]


	
	  \endgroup
	

	  \pstart {\color{DodgerBlue3}“प्रकाशे”} स्वसम्विदिते ज्ञाने विषयस्य {\color{DodgerBlue3}“रूपसंक्रमात्”} सारूप्यसंभवात् ज्ञानेनार्थप्रकाशित इत्युच्यते । {\color{DodgerBlue3}“स च प्रकाशस्तद्रूपो”} विषयस्वरुपः {\color{DodgerBlue3}“स्वयमेवा”}परोक्षप्रकाशात्मनोत्पन्नः {\color{DodgerBlue3}“प्रकाशते”} न त्वन्येन प्रकाश्यते । (४७९)
	\pend
      \label{div_pvv.2.480}\edlabel{div_pvv.2.480}
	  
	% new div opening: depth here is 2
	\leavevmode\marginnote{\textenglish{260/s}}

	  \pstart स्यादेतद् (।) बुद्धिरपि बुद्ध्यन्तरसरूपोत्पन्ना प्रकाशमाना बुद्धेर्व्यञ्जिका मतेति चेत् । आह (।)
	\pend
      
	  \bigskip
	  \begingroup
	  \large
	
	    
	    \stanza[\smallbreak]
	\label{pv.2.480}\edlabel{pv.2.480}\flagstanza{\tiny\textenglish{....2.480}}तथाभ्युपगमे बुद्धेः बुद्धौ बुद्धिः स्ववेदिका ।&सिद्धान्यथा तुल्यधर्मा विषयोपि धिया सह ॥ ४८० ॥\&[\smallbreak]


	
	  \endgroup
	

	  \pstart {\color{DodgerBlue3}“बुद्धेः”} प्रकाश्यायाः {\color{DodgerBlue3}“बुद्धौ”} व्यञ्जिकायां {\color{DodgerBlue3}“तथा”} संक्रान्तसारूप्यप्रकाशद्वारेण {\color{DodgerBlue3}“वेदनाभ्युपगमे”} व्यञ्जिका {\color{DodgerBlue3}“बुद्धिः स्वसंवेदिका सिद्धा”} (।) धीसरूपाया {\color{DodgerBlue3}“बुद्धेः”} स्वप्रकाशत्वे पूर्व्वबुद्धिः प्रकाशिता स्यात् । {\color{DodgerBlue3}“अन्यथा”} स्वप्रकाशत्वानभ्युपगमे {\color{DodgerBlue3}“विषयो”}प्यप्रकाशस्वभावतया {\color{DodgerBlue3}“धिया सह तुल्यधर्मेति”} सोपि बुद्धेर्व्यञ्जकः स्यात् । सरूपयोर्धीविषययोरन्योन्यं व्यञ्जकता भवेत् । (४८०)
	\pend
      \label{div_pvv.2.481}\edlabel{div_pvv.2.481}
	  
	% new div opening: depth here is 2
	
	  \bigskip
	  \begingroup
	  \large
	
	    
	    \stanza[\smallbreak]
	\label{pv.2.481}\edlabel{pv.2.481}\flagstanza{\tiny\textenglish{....2.481}}इति प्रकाशरूपा नः स्वयं धीः संप्रकाशते ।&अन्योस्यां रूपसंक्रान्त्या प्रकाशः सन् प्रकाशते ॥ ४८१ ॥\&[\smallbreak]


	
	  \endgroup
	

	  \pstart {\color{DodgerBlue3}“इति”} तस्मान्नोऽस्माकं मते {\color{DodgerBlue3}“धीः स्वय”}मात्मना {\color{DodgerBlue3}“प्रकाशरूपो”}त्पन्ना सती {\color{DodgerBlue3}“प्रकाशते”} । न त्वन्येन प्रकाश्यते इति युक्तं । अन्यः पुनर{\color{DodgerBlue3}“न्योस्यां”} बुद्धौ प्रकाशायां {\color{DodgerBlue3}“रूपसं”}\edlabel{pvv.260-1}\footnote{\label{pvv.260-1}  १ स्फटिकमणाविव जवा (कु) सुमं ।}क्रान्त्या {\color{DodgerBlue3}“प्रकाशः सन् प्रकाशते”} । ततोऽर्थवेदनव्यवहारः । (४८१)
	\pend
      \label{div_pvv.2.482}\edlabel{div_pvv.2.482}
	  
	% new div opening: depth here is 2
	
	  \bigskip
	  \begingroup
	  \large
	
	    
	    \stanza[\smallbreak]
	\label{pv.2.482a}\edlabel{pv.2.482a}\flagstanza{\tiny\textenglish{...2.482a}}सादृश्येपि हि धीरन्या प्रकाश्या न तया मता ।&स्वयं प्रकाशमाना ;\&[\smallbreak]


	
	  \endgroup
	

	  \pstart {\color{DodgerBlue3}“सादृश्ये”} सारूप्येपि सति {\color{DodgerBlue3}“तया”} सरूपया धियाऽ{\color{DodgerBlue3}“न्या”} पूर्व्विका {\color{DodgerBlue3}“धीर्न प्रकाश्या मता”} प्रकाशस्वभावस्य परेण प्रकाशायोगात् । किन्तु {\color{DodgerBlue3}“स्वयं प्रकाश”}स्वभावतया प्रकाशमाना प्रकाशत इत्यभ्युपेयं ।
	\pend
      

	  \begin{center}%% label @type='head'
	\textbf{घ. अर्थस्य ज्ञानरूपेण प्रकाशकता}
	\end{center}
	

	  \pstart एवन्तर्ह्यर्थस्याप्रकाशात्मनः कथं प्रकाशत इत्याह (।)
	\pend
      
	  \bigskip
	  \begingroup
	  \large
	
	    
	    \stanza[\smallbreak]
	\label{pv.2.482b}\edlabel{pv.2.482b}\flagstanza{\tiny\textenglish{...2.482b}}अर्थस्तद्रूपेण प्रकाशते ॥ ४८२ ॥\&[\smallbreak]


	
	  \endgroup
	

	  \pstart {\color{DodgerBlue3}“अर्थः”} सारूप्यसंक्रान्ते{\color{DodgerBlue3}“स्तद्रूपे”}ण ज्ञानरूपेण {\color{DodgerBlue3}“प्रकाशते”} न तु साक्षात् स्वरूपेण । (४८२)
	\pend
      \label{div_pvv.2.483}\edlabel{div_pvv.2.483}
	  
	% new div opening: depth here is 2
	

	  \pstart दृष्टान्तमाह (।)
	\pend
      
	  \bigskip
	  \begingroup
	  \large
	
	    
	    \stanza[\smallbreak]
	\label{pv.2.483}\edlabel{pv.2.483}\flagstanza{\tiny\textenglish{....2.483}}यथा प्रदीपयोर्द्दीपघटयोश्च तदाश्रयः ।&व्यङ्ग्यव्यञ्जकभेदेन व्यवहारः प्रतन्यते ॥ ४८३ ॥\&[\smallbreak]


	
	  \endgroup
	\leavevmode\marginnote{\textenglish{261/s}}

	  \pstart {\color{DodgerBlue3}“यथा प्रदीपयोः”} प्रकाशात्मनोर्न प्रकश्यप्रकाशकभावः । तथा बुद्ध्योरपि । यथा {\color{DodgerBlue3}“दीपघटयोः”} प्रकाशाप्रकाशस्वभावयोरेकः प्रकाशकोऽन्यः प्रकाश्यः । तथा ज्ञानार्थयोरपि । {\color{DodgerBlue3}“तदाश्रयो”}ऽप्रकाशप्रकाशात्मनिष्ठो {\color{DodgerBlue3}“व्यङ्ग्यव्यञ्जकभेदेन व्यवहारो”} लोके {\color{DodgerBlue3}“प्रतन्य”}ते (। ४८३)
	\pend
      \label{div_pvv.2.484}\edlabel{div_pvv.2.484}
	  
	% new div opening: depth here is 2
	
	  \bigskip
	  \begingroup
	  \large
	
	    
	    \stanza[\smallbreak]
	\label{pv.2.484}\edlabel{pv.2.484}\flagstanza{\tiny\textenglish{....2.484}}विषयेन्द्रियमात्रेण न दृष्टमिति निश्चयः ।&तस्माद्यतोयं तस्यापि वाच्यमन्यस्य दर्शनम् ॥ ४८४ ॥\&[\smallbreak]


	
	  \endgroup
	

	  \pstart यतश्च {\color{DodgerBlue3}“विषय”}मात्रेण {\color{DodgerBlue3}“इन्द्रियमात्रेण”} वे\edlabel{pvv.261-1}\footnote{\label{pvv.261-1}  १ नैयायिकजैमिनीयादेर्बुद्धिपरोक्षत्वात् न स्यादेव ।} दं {\color{DodgerBlue3}“दृष्टमिति न निश्यस्त”} स्माद्यतस्तद्व्यतिरिक्ताज्ज्ञानादिदं दृष्टमिति निश्चयस्तस्यापि विषयेन्द्रियाभ्या{\color{DodgerBlue3}“मन्यस्य”} ज्ञानस्य {\color{DodgerBlue3}“दर्शनम”}परोक्षत्वं {\color{DodgerBlue3}“वाच्य”}मिति बुद्धिपरोक्षतावादो न युक्तः । (४८४)
	\pend
      \label{div_pvv.2.485}\edlabel{div_pvv.2.485}
	  
	% new div opening: depth here is 2
	

	  \begin{center}%% label @type='head'
	\textbf{(३) व. स्वसंवित्तिसिद्धिः}
	\end{center}
	

	  \begin{center}%% label @type='head'
	\textbf{क. स्मृतेः स्वसंवित्तिः}
	\end{center}
	

	  \pstart स्व\edlabel{pvv.261-2}\footnote{\label{pvv.261-2}  २ स्वोपपत्तिभिः स्ववेदं प्रसाध्याचार्योपपत्तिमाह ।}संवित्तिसिद्ध्यर्थमुपपत्त्यन्तरमाह (।)
	\pend
      
	  \bigskip
	  \begingroup
	  \large
	
	    
	    \stanza[\smallbreak]
	\label{pv.2.485a}\edlabel{pv.2.485a}\flagstanza{\tiny\textenglish{...2.485a}}स्मृतेरप्यात्मवित्सिद्धा ज्ञानस्य;\&[\smallbreak]


	
	  \endgroup
	

	  \pstart ज्ञानस्यातीतस्य {\color{DodgerBlue3}“स्मृतेरप्यात्मवित्”} सुसंवित्तिः {\color{DodgerBlue3}“सिद्धा”} । प्रतीतमेव हि स्मर्यते यथार्थः ।\edlabel{pvv.261-3}\footnote{\label{pvv.261-3}  ३ स्मर्यते च बुद्धिः ।}
	\pend
      

	  \pstart स्यादेतत् ज्ञानं प्रतीतमन्येन चेतसा न स्वसंवेदनेनेत्याह (।)
	\pend
      
	  \bigskip
	  \begingroup
	  \large
	
	    
	    \stanza[\smallbreak]
	\label{pv.2.485b}\edlabel{pv.2.485b}\flagstanza{\tiny\textenglish{...2.485b}}अन्येन वेदने ।&दीर्घादिग्रहणन्न स्याद् बहुमात्रानवस्थितेः ॥ ४८५ ॥\&[\smallbreak]


	
	  \endgroup
	

	  \pstart अन्येन ज्ञानेन पूर्व्वकस्य ज्ञानस्य\edlabel{pvv.261-4}\footnote{\label{pvv.261-4}  ४ परोक्तं सिद्धसाधनत्वं स्वयं परिह (र)ति ज्ञानान्तरेणानुभवेऽनिष्टा ।} {\color{DodgerBlue3}“वेदने”}ऽभिधीयमाने \edlabel{pvv.261-5}\footnote{\label{pvv.261-5}  ५ तत्रापि हि स्मृतिविषयान्तरसञ्चारस्तथा न स्यात् स चेक्षते इत्याद्याचार्यसिद्धान्तं मुक्त्वाधिकदोषाभिधानाय ।} {\color{DodgerBlue3}“दीर्घ\edlabel{pvv.261-6}\footnote{\label{pvv.261-6}  ६ ह्रस्वप्लुतादेः ।}देः”} स्वरस्य \leavevmode\marginnote{\textenglish{262/s}} {\color{DodgerBlue3}“ग्रहणं न स्यात्”} । क्षणिकस्य ज्ञानस्य एका\edlabel{pvv.262-1}\footnote{\label{pvv.262-1}  १ यावता कालेन परमाणुरिष्टः परमाण्वन्तरमेकमतिक्रामति तावत्कालः क्षणः (।) क्षणिका श्रोत्रधीः सर्व्वेषामनेकक्षणात्मकमेकवर्प्णनिष्पत्तिकालं न तिष्ठति ।}ण्वत्ययकालमात्रस्थायि\edlabel{pvv.262-2}\footnote{\label{pvv.262-2}  २ एकवर्णभागग्राहिबुद्धौ नष्टायामनन्तरन्तदनुभवबुद्धिरिति व्यवहितं तदपरवर्ण्णभागज्ञानं तद्धिया तदज्ञानात् दीर्घग्रहो न स्यात् स्मृतिरपि न विकल्पाविकल्पयोरेकेन जननविरोधात् ।}नोऽ\edlabel{pvv.262-3}\footnote{\label{pvv.262-3}  ३ सर्व्वेषु वर्ण्णभागेषु बुद्धयः स्वसंविदिता इति स्मृतिसंकलं स्यात् न च स्ववेदनं मन्यते ।}नेकक्षणनिर्व्वर्त्त्यासु बह्वीषु {\color{DodgerBlue3}“मात्रासु”} दीर्घादिनिर्व्वर्त्तनिकासु ग्राहकत्वेना{\color{DodgerBlue3}“नवस्थितेः”} । न ह्येकक्षणमात्रस्थापि ज्ञानमनेकक्षणकलापनिर्व्वर्तनीयमात्रासञ्चयात्मकं दीर्घा\leavevmode\marginnote{\textenglish{51b/MA}} दिकं शक्नोति ग्रहीतुं । किन्त्वकाराद्येकैकमात्रावयवलेशं गृह्णाति । तद्‏ग्राहकत्वेन द्वितीयज्ञानेन तत् प्रतीयते एवमपरापरैर्ज्ञानैरवयवलेशग्राहकैः स्वस्व\edlabel{pvv.262-4}\footnote{\label{pvv.262-4}  ४ पूर्व्वपूर्व्वेण ।}ग्राहकज्ञानान्तरितैर्ग्रहणक्रमे केन मात्राग्रहणं । मात्राप्रचयदीर्घादिग्रहणं वा स्यात् । (४८५)
	\pend
      \label{div_pvv.2.486}\edlabel{div_pvv.2.486}
	  
	% new div opening: depth here is 2
	
	  \bigskip
	  \begingroup
	  \large
	
	    
	    \stanza[\smallbreak]
	\label{pv.2.486a}\edlabel{pv.2.486a}\flagstanza{\tiny\textenglish{...2.486a}}अवस्थितावक्रमायां सकृदाभासनान्मतौ ।&वर्ण्णाः स्यादक्रमोऽदीर्घः;\&[\smallbreak]


	
	  \endgroup
	

	  \pstart अथा\edlabel{pvv.262-5}\footnote{\label{pvv.262-5}  ५ आसर्गप्रलयादेरेकैव बुद्धिः सांख्यस्य तमाह तां ।}नेकमात्राकालमेकैव बुद्धिरस्तीत्युच्यते तदानेककाल{\color{DodgerBlue3}“मवस्थितौ”} सत्या{\color{DodgerBlue3}“मक्रमायां मतौ”} सर्व्वमात्राणां {\color{DodgerBlue3}“सकृदाभासनाद्दीर्घादिर्व्वर्ण्णोऽक्रमः स्यात्”} । प्रतीतिनिबन्धनत्वाद्वस्तुव्यवस्थायाः । तथा चादीर्घो \edlabel{pvv.262-6}\footnote{\label{pvv.262-6}  ६ अपरापरबुद्धावपरापरमात्राभागप्रतिभासक्रमेण दीर्घभानं यतः ।}भवेत् । न ह्येककालमुच्चैरुच्चार्यमाणोप्येकमात्रिको दीर्घः ।
	\pend
      

	  \pstart ननु क्रमवन्तो वर्ण्णा अवयवक्रमेणोत्पद्यमाना दीर्घादिबुद्धिमुत्पादयिष्यन्तीत्याह (।)
	\pend
      
	  \bigskip
	  \begingroup
	  \large
	
	    
	    \stanza[\smallbreak]
	\label{pv.2.486b}\edlabel{pv.2.486b}\flagstanza{\tiny\textenglish{...2.486b}}क्रमवानक्रमां कथम् ॥ ४८६ ॥\&[\smallbreak]


	
	  \endgroup
	
	  \bigskip
	  \begingroup
	  \large
	
	    
	    \stanza[\smallbreak]
	\label{pv.2.487a}\edlabel{pv.2.487a}\flagstanza{\tiny\textenglish{...2.487a}}उपकुर्यादसंश्लिष्यन्वर्ण्णभागः परस्परम् (।)\&[\smallbreak]


	
	  \endgroup
	\leavevmode\marginnote{\textenglish{263/s}}

	  \pstart {\color{DodgerBlue3}“क्रमवान् वर्ण्णभागः”} स्वस्वकालस्थायी {\color{DodgerBlue3}“प\edlabel{pvv.263-1}\footnote{\label{pvv.263-1}  १ उच्चारितैकवर्णभागनाशेऽपरोच्चारणमिति साहित्यं नास्ति ।}रस्परमसंश्लिष्यन्न”}सम्बध्यमानो दीर्घबुद्धि{\color{DodgerBlue3}“मक्रमां कथमुपकुर्य्या”}दुत्पादये\edlabel{pvv.263-2}\footnote{\label{pvv.263-2}  २ पूर्वभागस्य बुद्धिजनकत्वे परभागानामनुपयोगात् ।}त् ।\edlabel{pvv.263-3}\footnote{\label{pvv.263-3}  ३ स्वाकारबुद्धिजननेन ।}क्रमवति ज्ञेये ज्ञानमपि तथैव युक्तं । (४८६)
	\pend
      \label{div_pvv.2.487}\edlabel{div_pvv.2.487}
	  
	% new div opening: depth here is 2
	

	  \begin{center}%% label @type='head'
	\textbf{ख. क्रमभाविनां वर्णानां स्फोटेनासंगतिः}
	\end{center}
	

	  \pstart अथोत्पन्ना वर्ण्णावयवा अन्त्यावयवोत्पत्तिपर्यन्तमनुवर्तन्ते । ततः क्रमग्रहणं सर्व्वग्रहणञ्चास्तीति युक्तं दीर्घादिग्रहणमित्याह (।)
	\pend
      
	  \bigskip
	  \begingroup
	  \large
	
	    
	    \stanza[\smallbreak]
	\label{pv.2.487b}\edlabel{pv.2.487b}\flagstanza{\tiny\textenglish{...2.487b}}आन्त्यं पूर्वस्थितादूर्ध्वं वर्धमानो ध्वनिर्भवेत् ।\&[\smallbreak]


	
	  \endgroup
	

	  \pstart {\color{DodgerBlue3}“आ”} अन्त्य{\color{DodgerBlue3}“मन्त्यं”} वर्ण्णावयवं यावत् {\color{DodgerBlue3}“पूर्व्व”}पूर्व्वेषां वर्ण्णावयवानां क्रमोत्पन्नानां {\color{DodgerBlue3}“स्थितौ”} सत्यां प्रथमवर्ण्णावयवा{\color{DodgerBlue3}“दूर्ध्व”} पूर्व्वोत्पन्नस्यानुवृत्तावपूर्व्वस्य चापरस्योत्पत्तौ {\color{DodgerBlue3}“वर्द्धमानो ध्वनिर्भवेत्”}\edlabel{pvv.263-4}\footnote{\label{pvv.263-4}  ४ न द्विमात्र इति प्रत्यक्षविरोधः ।} । न चैतदस्ति । अवयवक्रमग्रहणेन दीर्घबुद्धेरुत्पा दात् ।
	\pend
      

	  \pstart स्यादेतत् (।) क्रमेणोत्पन्नानामवयवानामन्त्यावयव\edlabel{pvv.263-5}\footnote{\label{pvv.263-5}  ५ “
	    \begin{verse}
	नादैराहितबीजायामन्त्येन ध्वनिना सह\\
	    आवृत्तपरिपाकायां बुद्धौ शब्दोवधार्यत\\
	    
	    \end{verse}
	  ” इति वै या क र णा स्तानाह ।}काले ग्रहणमिति न वर्द्धमानध्वनिर्भवति पूर्व्वं कस्यचिद् ग्रहणाभावादित्याह (।)
	\pend
      
	  \bigskip
	  \begingroup
	  \large
	
	    
	    \stanza[\smallbreak]
	\label{pv.2.487c}\edlabel{pv.2.487c}\flagstanza{\tiny\textenglish{...2.487c}}अक्रमेण ग्रहादन्ते क्रमवद्धीश्च नो भवेत् ॥ ४८७ ॥\&[\smallbreak]


	
	  \endgroup
	

	  \pstart {\color{DodgerBlue3}“अक्रमेण”} ग्रहणादन्त्यवर्णनिष्पत्तिकाले च तद्‏ग्राहिका {\color{DodgerBlue3}“क्रमवती धीर्नो भवेत्”} । ततश्च न दीर्घग्रहणं । न ह्येककालमनेकैरुच्चार्यमाणेऽनेकस्मिन्नकारादौ दीर्घबुद्धिर्भवेत् । (४८७)
	\pend
      \label{div_pvv.2.488}\edlabel{div_pvv.2.488}
	  
	% new div opening: depth here is 2
	

	  \pstart एकैकबुद्धिः सकृदुत्पन्ना क्रमेणावयवान् गृह्णात्यन्त्यावयवग्रहणकाले दीर्घग्राहिकेति चेत् आह (।)
	\pend
      
	  \bigskip
	  \begingroup
	  \large
	
	    
	    \stanza[\smallbreak]
	\label{pv.2.488}\edlabel{pv.2.488}\flagstanza{\tiny\textenglish{....2.488}}धियः स्वयञ्च न स्थानं तदूर्ध्वविषयास्थितेः ॥ ४८८ ॥\&[\smallbreak]


	
	  \endgroup
	\leavevmode\marginnote{\textenglish{264/s}}

	  \pstart {\color{DodgerBlue3}“धियो”}ऽवयवग्राहिकाया{\color{DodgerBlue3}“स्तस्मा”}देकायवग्रहणा{\color{DodgerBlue3}“दूर्ध्वं”} पूर्वगृहीतस्य विषय\edlabel{pvv.264-1}\footnote{\label{pvv.264-1}  १ भागानामस्थितेरिति तुल्यकालम्विष (य) विषयित्वेऽयं येन दीर्घादिबुद्धिकाले न स्यात् ।}स्या{\color{DodgerBlue3}“स्थितेर्हेतोर्न स्थानं”} स्थितिर्युक्ता\edlabel{pvv.264-2}\footnote{\label{pvv.264-2}  २ शब्दपरमाणून् तद्देशविभागेनापरापरदेशसंयोगेन श्रोत्रपथमानयत्यतो वर्ण्णाभिव्यक्तिरिति प्रत्यभिज्ञा च ।}। (४८८)
	\pend
      \label{div_pvv.2.489}\edlabel{div_pvv.2.489}
	  
	% new div opening: depth here is 2
	
	  \bigskip
	  \begingroup
	  \large
	
	    
	    \stanza[\smallbreak]
	\label{pv.2.489}\edlabel{pv.2.489}\flagstanza{\tiny\textenglish{....2.489}}स्थाने स्वयन्न नश्येत् सा पश्चादप्यविशेषतः ।&दोषोयं सकृदुत्पन्नाक्रमवर्ण्णस्थितावपि ॥ ४८९ ॥\&[\smallbreak]


	
	  \endgroup
	

	  \pstart अथ विषयानवस्थानेपि {\color{DodgerBlue3}“स्वयम”}विनश्वरस्वभावतया बुद्धेः {\color{DodgerBlue3}“स्थाने”} वा स्वीक्रियमाणे {\color{DodgerBlue3}“पश्चाद”}न्तावयवग्रहणानन्तरमप्यविनश्वरस्वभावतया{\color{DodgerBlue3}“ऽविशेषतो न नश्येत्”} ॥ \edlabel{pvv.264-3}\footnote{\label{pvv.264-3}  ३ नैयायिकस्य ।}बुद्धेः सकृदुत्पन्नायाश्चिरावस्थाने यो {\color{DodgerBlue3}“दोषः”} सर्व्वदाऽविनाशप्रसङ्ग उक्तोऽयं {\color{DodgerBlue3}“सकृदुत्पन्ना”}नामक्रमाणां {\color{DodgerBlue3}“वर्ण्णानां”} क्रमग्राहिविज्ञानोत्पत्तिकालं यावद{\color{DodgerBlue3}“वस्थिताव”}प्युच्यमानायां बोद्धव्यः । यदि सकृदुत्पन्ना अप्यविनश्वरस्वभावतया कञ्चित्कालमनुवर्तते तदा चिरमपि तत्स्वभावाप्रच्यवादनुवर्तेरन् । (४८९)
	\pend
      \label{div_pvv.2.490}\edlabel{div_pvv.2.490}
	  
	% new div opening: depth here is 2
	

	  \pstart किञ्च (।)
	\pend
      
	  \bigskip
	  \begingroup
	  \large
	
	    
	    \stanza[\smallbreak]
	\label{pv.2.490a}\edlabel{pv.2.490a}\flagstanza{\tiny\textenglish{...2.490a}}सकृद्यत्नोद्भवाद् व्यर्थः स्याद्यत्नश्चोत्तरोत्तरः ।\&[\smallbreak]


	
	  \endgroup
	

	  \pstart {\color{DodgerBlue3}“सकृत् कृताद्यत्ना”}त्ताल्वादिव्यापारात् वर्ण्णा\edlabel{pvv.264-4}\footnote{\label{pvv.264-4}  ४ मीमांसको वर्णस्फोटवादी न वर्ण्णातिरिक्तं पदं वाक्यं वाचकमस्तीत्याह श्रोत्रप्राप्यकारि । वर्ण्णाश्च नित्या देशकालनरान्तरेष्वेकाकारबुद्धिग्राह्यत्वात् यद्देशः शब्दस्तद्देशसन्निहितो वायुरदुष्टचोदितः ।}ना{\color{DodgerBlue3}“मुत्पन्न”}त्वादुत्तरोत्तरो यत्नश्च {\color{DodgerBlue3}“व्यर्थः”} स्यात् ।
	\pend
      

	  \begin{center}%% label @type='head'
	\textbf{ग. न संयोगविभागद्वारेण शब्दाभिव्यक्तिः}
	\end{center}
	

	  \pstart योपि मन्यते (।) प्रयत्नप्रेरितेन वायुना स्तिमितस्य वायोराकाशसंयुक्तस्य संयोगविभागकृता शब्दस्याभिव्यक्तिर्भवतीति तं प्रत्याह (।)
	\pend
      
	  \bigskip
	  \begingroup
	  \large
	
	    
	    \stanza[\smallbreak]
	\label{pv.2.490b}\edlabel{pv.2.490b}\flagstanza{\tiny\textenglish{...2.490b}}व्यक्तावप्येष वर्ण्णानां दोषः समनुषज्यते ॥ ४९० ॥\&[\smallbreak]


	
	  \endgroup
	\leavevmode\marginnote{\textenglish{265/s}}

	  \pstart {\color{DodgerBlue3}“वर्ण्णाना”}म्वायवीयसंयोगविभागद्वारेणाभि{\color{DodgerBlue3}“व्यक्ताव”}पीष्यमाणायामयमनन्तरोक्तोऽक्रमाणां सकृदभिव्यक्तेरुत्तरो यत्नः प्राणप्रेरणादि\edlabel{pvv.265-1}\footnote{\label{pvv.265-1}  १ सकृदभिव्यक्तस्य सदा स्थितेर्नित्यत्वात्}को व्यर्थः स्यादिति {\color{DodgerBlue3}“दोषः”}\leavevmode\marginnote{\textenglish{52a/MA}} {\color{DodgerBlue3}“समनु”}\edlabel{pvv.265-2}\footnote{\label{pvv.265-2}  २ उक्तमप्युत्तरप्रबन्धावतारार्थमाह ।}षज्यते । (४९०)
	\pend
      \label{div_pvv.2.491}\edlabel{div_pvv.2.491}
	  
	% new div opening: depth here is 2
	

	  \pstart स्या\edlabel{pvv.265-3}\footnote{\label{pvv.265-3}  ३ सावयववादी मीमांसक आह ।}देतद् (।)
	\pend
      
	  \bigskip
	  \begingroup
	  \large
	
	    
	    \stanza[\smallbreak]
	\label{pv.2.491}\edlabel{pv.2.491}\flagstanza{\tiny\textenglish{....2.491}}अनेकया तद्ग्रहणे यान्त्या धीः सानुभूयते ।&न दीर्घग्राहिका सा च तन्न स्याद्दीर्घधीस्मृतिः ॥ ४९१ ॥\&[\smallbreak]


	
	  \endgroup
	

	  \pstart {\color{DodgerBlue3}“अनेक\edlabel{pvv.265-4}\footnote{\label{pvv.265-4}  ४ निरन्तरया इति पूर्व्वतो विशेषः ।}या”} धिया तस्यानेकाश्रयात्मकस्य वर्ण्णस्य {\color{DodgerBlue3}“ग्रहणे”} कृते पश्चाद् {\color{DodgerBlue3}“यान्त्या धी”}रन्त्यावयवग्राहिका {\color{DodgerBlue3}“सा”} बुद्ध्यन्तरेणा{\color{DodgerBlue3}“नुभूयते”} तया दीर्घादिबुंद्धिर्भविष्यतीति चेत् । नन्वेवं पूर्व्वासां बुद्धीनामनेकावयवग्राहिकाणां न बुद्ध्यन्तरेण ग्रहो नापि स्वसम्वेदनेन वित्तिः । या चानुभूयते साऽन्त्यावयवग्राहिका धीः {\color{DodgerBlue3}“सापि”} हि {\color{DodgerBlue3}“न दीर्घा”}\edlabel{pvv.265-5}\footnote{\label{pvv.265-5}  ५ पूर्व्वाग्रहादेव ।}दिवर्ण्ण{\color{DodgerBlue3}“ग्राहिका त”}त्तस्मा{\color{DodgerBlue3}“द्दीर्घधीस्मृतिर्न स्यात्”} । (४९१)
	\pend
      \label{div_pvv.2.492}\edlabel{div_pvv.2.492}
	  
	% new div opening: depth here is 2
	
	  \bigskip
	  \begingroup
	  \large
	
	    
	    \stanza[\smallbreak]
	\label{pv.2.492}\edlabel{pv.2.492}\flagstanza{\tiny\textenglish{....2.492}}पृथक् पृथक् च बुद्धीनां सम्वित्तौ तद् ध्वनिः श्रुतेः ।&अविच्छिन्नाभता न स्याद् घटनञ्च निराकृतम् ॥ ४९२ ॥\&[\smallbreak]


	
	  \endgroup
	

	  \pstart न ह्यगृहीतमेव स्मर्यते वर्ण्णावयवग्राहिकाणां {\color{DodgerBlue3}“बुद्धीनां पृथक् पृथक्”} स्व\edlabel{pvv.265-6}\footnote{\label{pvv.265-6}  ६ पूर्व्वानुभवाभिः ।}स्वग्राहिणीभिः बुद्धिभिः {\color{DodgerBlue3}“संवित्ता”}विष्यमाणायाञ्च तद् {\color{DodgerBlue3}“ध्वनिश्रुते”}र्दीर्घादिवर्ण्णबुद्धेर{\color{DodgerBlue3}“विच्छिन्ना”}भता निरन्तरप्रतिभासता {\color{DodgerBlue3}“न स्यात्”} । अन्तरान्तरा बुद्धिग्राहिणीभिर्बुद्धिभिरवयवग्राहिबुद्धीनां व्यवधानात् । स्वग्राहिकाभिर्बुद्धिभिर्व्यवधानेपि बुद्धीनां लघुवृत्तित्वात् {\color{DodgerBlue3}“घटन”}मव्यवधानाध्यवसानं {\color{DodgerBlue3}“निराकृत”}मन्यत्रापि समानं तदि\edlabel{pvv.265-7}\footnote{\label{pvv.265-7}  ७ तद्वर्ण्णयोर्व्वा सकृत् श्रुतिः ।}त्यादिना (२।१३५) यथावयवग्राहिका बुद्धयो लघुवृत्तयः तथा तद्‏ग्राहिबुद्धिकाले तदभावोंपि लघुवृत्तिरिति बुद्ध्यवच्छेदवर्ण्णविच्छेदोऽपि मन्येतेत्युक्तं प्राक् । (४९२)
	\pend
      \label{div_pvv.2.493}\edlabel{div_pvv.2.493}
	  
	% new div opening: depth here is 2
	

	  \pstart किञ्च (।)
	\pend
      
	  \bigskip
	  \begingroup
	  \large
	
	    
	    \stanza[\smallbreak]
	\label{pv.2.493}\edlabel{pv.2.493}\flagstanza{\tiny\textenglish{....2.493}}विच्छिन्नं श्रृण्वतोप्यस्य यद्यविच्छिन्नविभ्रमः ।&ह्रस्वद्वयोच्चारणेपि स्यादविच्छिन्नविभ्रमः ॥ ४९३ ॥\&[\smallbreak]


	
	  \endgroup
	

	  \pstart {\color{DodgerBlue3}“विच्छिन्नं शृण्वतोपि”} लघुवृत्ते{\color{DodgerBlue3}“र्यद्यविच्छिन्नभ्रम”}स्तदा {\color{DodgerBlue3}“ह्रस्वद्वयोच्चारणेपि”} लघुवृत्तेर{\color{DodgerBlue3}“विच्छिन्नविभ्रमः स्यादि”}ति दीर्घबुद्धिर्भवेत् । (४९३)
	\pend
      \leavevmode\marginnote{\textenglish{266/s}}\label{div_pvv.2.494}\edlabel{div_pvv.2.494}
	  
	% new div opening: depth here is 2
	
	  \bigskip
	  \begingroup
	  \large
	
	    
	    \stanza[\smallbreak]
	\label{pv.2.494}\edlabel{pv.2.494}\flagstanza{\tiny\textenglish{....2.494}}विच्छिन्ने दर्शने चाक्षादविच्छन्नाधिरोपणम् ।&नाक्षात्सर्व्वाक्षबुद्धीनां वितथत्वप्रसङ्गतः ॥ ४९४ ॥\&[\smallbreak]


	
	  \endgroup
	

	  \pstart वर्ण्णावयवानाम\edlabel{pvv.266-1}\footnote{\label{pvv.266-1}  १ श्रोत्रात् श्रवणे ।}क्षादिन्द्रिया{\color{DodgerBlue3}“द्विच्छिन्ने”} स्वबुद्धिभिर्व्यवहिते च वस्तुतो {\color{DodgerBlue3}“दर्शनेऽक्षादविच्छिन्न”}स्य दर्शनस्या{\color{DodgerBlue3}“धिरोपणमारोप”} इति न युक्तं । एव {\color{DodgerBlue3}“सर्व्वा”}सा{\color{DodgerBlue3}“मक्षबुद्धीनां”} रूपादिग्राहिणीनां {\color{DodgerBlue3}“वितथत्वप्रसङ्ग\edlabel{pvv.266-2}\footnote{\label{pvv.266-2}  २ व्यवधानेप्यव्यवहितभानात् ।}तो”} न किञ्चिदभ्रान्तं स्यात् । (४९४)
	\pend
      \label{div_pvv.2.495}\edlabel{div_pvv.2.495}
	  
	% new div opening: depth here is 2
	

	  \begin{center}%% label @type='head'
	\textbf{घ. बुद्धेर्बुद्ध्यन्तरेण गृहीतो नाविच्छेदप्रतिभासः}
	\end{center}
	

	  \pstart बुद्धेर्बुद्ध्यन्तरेण गृहीतावविच्छेदप्रतिभासो नास्तीति सोपपत्तिकमाख्यातुमाह (।)
	\pend
      
	  \bigskip
	  \begingroup
	  \large
	
	    
	    \stanza[\smallbreak]
	\label{pv.2.495}\edlabel{pv.2.495}\flagstanza{\tiny\textenglish{....2.495}}सर्व्वान्त्योपि हि वर्ण्णात्मा निमेषतुलितस्थितिः ।&स च क्रमादनेकाणुसम्बन्धेन नितिष्ठति ॥ ४९५ ॥\&[\smallbreak]


	
	  \endgroup
	

	  \pstart {\color{DodgerBlue3}“सर्व्वे”}षां दीर्घादीनामेकमात्रिकत्वा{\color{DodgerBlue3}“दन्त्यो वर्ण्णात्मा”}ऽकारादि{\color{DodgerBlue3}“र्निमेषे”}ण नयननि मीलनेन {\color{DodgerBlue3}“तुलिता”} \edlabel{pvv.266-3}\footnote{\label{pvv.266-3}  ३ निमेषमात्रेणाभिनिर्वृत्तिधर्म्मेति यावत् ।}परिमिता {\color{DodgerBlue3}“स्थिति”}र्यस्य स तथा । स चाक्षिनिमेषकालस्थाय्यकारादि{\color{DodgerBlue3}“रनेके”}षामाकाशतमःसंज्ञितानां पक्ष्ममालाद्वयवर्त्तिना{\color{DodgerBlue3}“मणू”}नां {\color{DodgerBlue3}“सम्बन्धे”}नाक्रमणेन निमेषकालसंज्ञितेन {\color{DodgerBlue3}“क्रमाद् भूयः”} क्षणव्यवधाना{\color{DodgerBlue3}“न्नितिष्ठति”} परिसमाप्यते । एतेनान्त्योपि वर्ण्णोऽनेकक्षणस्थायीत्युक्तं । (४९५)
	\pend
      \label{div_pvv.2.496}\edlabel{div_pvv.2.496}
	  
	% new div opening: depth here is 2
	
	  \bigskip
	  \begingroup
	  \large
	
	    
	    \stanza[\smallbreak]
	\label{pv.2.496}\edlabel{pv.2.496}\flagstanza{\tiny\textenglish{....2.496}}एकाण्वत्ययकालश्च कालोल्पोयान् क्षणो मतः ।&बुद्धिश्च क्षणिका तस्मात्क्रमाद्वर्ण्णान्प्रपद्यते ॥ ४९६ ॥\&[\smallbreak]


	
	  \endgroup
	

	  \pstart {\color{DodgerBlue3}“एकस्याणोरत्यय”} आक्रमणं स {\color{DodgerBlue3}“कालः”} परिमाणं यस्य तादृशश्च {\color{DodgerBlue3}“कालो”}ल्पीयान् लघुतमो भेत्तुमशक्यः {\color{DodgerBlue3}“क्ष\edlabel{pvv.266-4}\footnote{\label{pvv.266-4}  ४ अन्यस्य द्वित्रिपरमाणुव्यतिक्रमादिभेदेन कालो भेत्तुं शक्यः ।}णो”} भवेत् । तावत्कालमात्रस्थाष्णुतया {\color{DodgerBlue3}“बुद्धिश्च क्षणिका\edlabel{pvv.266-5}\footnote{\label{pvv.266-5}  ५ अनेकक्षणाभिनिर्वृत्तिधर्म्माश्च वर्ण्णाः ।} तस्मा”}त्क्षणिकत्वाद् बुद्धिः {\color{DodgerBlue3}“क्रमा”}त्पुन\edlabel{pvv.266-6}\footnote{\label{pvv.266-6}  ६ क्रमग्राहिकापि स्वग्राहिकयान्तरितेत्यविच्छिन्नग्रहः कुतः ।} र्धीर्व्यवहिता भूयसी भवन्ती {\color{DodgerBlue3}“वर्ण्णान् प्रतिपद्यते”} तत्कथमविच्छिन्नवर्ण्णग्रहः सङ्गतः । (४९६)
	\pend
      \leavevmode\marginnote{\textenglish{267/s}}\label{div_pvv.2.497}\edlabel{div_pvv.2.497}
	  
	% new div opening: depth here is 2
	
	  \bigskip
	  \begingroup
	  \large
	
	    
	    \stanza[\smallbreak]
	\label{pv.2.497}\edlabel{pv.2.497}\flagstanza{\tiny\textenglish{....2.497}}इति वर्ण्णेपि रूपादावविच्छिन्नावभासिनी ।&विच्छिन्नाप्यन्यया बुद्धिः सर्व्वा स्याद्वितथार्थिका ॥ ४९७ ॥\&[\smallbreak]


	
	  \endgroup
	

	  \pstart {\color{DodgerBlue3}“इति”} तस्माद् {\color{DodgerBlue3}“वर्ण्णेपि रू\edlabel{pvv.267-1}\footnote{\label{pvv.267-1}  १ यथा वर्ण्णेषु प्रत्येकं विभ्रम एवं रूपादावपीति दार्ष्टान्तिककथनं ।}पादा”}वपि बुद्धिस्तद्‏ग्राहिकयाऽन्यया धियाऽन्तरिता वस्तुतो {\color{DodgerBlue3}“विच्छिन्नाप्यविच्छिन्नावभासिनी”} कु\edlabel{pvv.267-2}\footnote{\label{pvv.267-2}  २ तन्माभूत् सर्व्वेन्द्रियज्ञानवितथत्वमिति नाविच्छिन्नभ्रमः स्वीकर्तव्यस्तदा च निरन्तरभानन्न स्यादस्ति चातः स्ववित्तिर्ज्ञानानां सिद्धा ।}तश्चिन्निमिताद् भवन्ती {\color{DodgerBlue3}“सर्व्वा\leavevmode\marginnote{\textenglish{52b/MA}} वितथार्थिका”} शून्या भ्रान्तिः {\color{DodgerBlue3}“स्यात्”} । (४९७)
	\pend
      \label{div_pvv.2.498}\edlabel{div_pvv.2.498}
	  
	% new div opening: depth here is 2
	
	  \bigskip
	  \begingroup
	  \large
	
	    
	    \stanza[\smallbreak]
	\label{pv.2.498}\edlabel{pv.2.498}\flagstanza{\tiny\textenglish{....2.498}}घटनं यच्च भावानामन्यत्रेन्द्रियविभ्रमात् ।&भेदालक्षणविभ्रान्तं स्मरणन्तद्विकल्पकं ॥ ४९८ ॥\&[\smallbreak]


	
	  \endgroup
	

	  \pstart {\color{DodgerBlue3}“यच्च भावानां\edlabel{pvv.267-3}\footnote{\label{pvv.267-3}  ३ क्षणिकानां निरन्वयविनाशिनां । असति बाधके भानादेवानयोः प्रामाण्यं ।}”} सदृशापरापरेषामुत्पद्यमानानां स एवायमित्येकत्व{\color{DodgerBlue3}“घटनं”} तद् भावानां परस्परतो {\color{DodgerBlue3}“भेदालक्षणेन विभ्रान्तं विकल्पकं”} ज्ञानं {\color{DodgerBlue3}“स्मरणं । अ\edlabel{pvv.267-4}\footnote{\label{pvv.267-4}  ४ इन्द्रियवैगुण्ये तु न घटनं सकृदसदर्थप्रतिभासं स्पष्टमेव ज्ञानमत्र ।}न्यत्रेन्द्रियविभ्रमात्”} । अतः पुनरलातादिषु भ्राम्यमाणेषु चक्राद्याकारं ज्ञानमुत्पद्यतेऽसाविन्द्रियक्रमो निर्व्विकल्पक एव । (४९८)
	\pend
      \label{div_pvv.2.499}\edlabel{div_pvv.2.499}
	  
	% new div opening: depth here is 2
	
	  \bigskip
	  \begingroup
	  \large
	
	    
	    \stanza[\smallbreak]
	\label{pv.2.499}\edlabel{pv.2.499}\flagstanza{\tiny\textenglish{....2.499}}तस्य स्पष्टावभासित्वं जल्पसंसर्गिणः कुतः ॥&नाक्षग्राह्येस्ति शब्दानां योजनेति विवेचितम् ॥ ४९९ ॥\&[\smallbreak]


	
	  \endgroup
	

	  \pstart {\color{DodgerBlue3}“तस्य”} विकल्पस्य {\color{DodgerBlue3}“जल्पसंसर्गिणः”} शब्दसंसर्गवतः {\color{DodgerBlue3}“स्पष्टाव\edlabel{pvv.267-5}\footnote{\label{pvv.267-5}  ५ यतोनेन दीर्घादिवर्ण्णस्य प्रतिभासो गृह्येत ।}भासित्वं”} ज्ञेयाकारवैशद्यं {\color{DodgerBlue3}“कुतः”} ।\edlabel{pvv.267-6}\footnote{\label{pvv.267-6}  ६ यश्चाक्षप्रत्ययः स्पष्टप्रतिभासस्तस्य वर्णभागेष्वनुसन्धायी दीर्घादिप्रत्ययो नास्ति ।} यस्मा{\color{DodgerBlue3}“दक्षग्राह्ये”} स्वलक्षणे {\color{DodgerBlue3}“शब्दानां”} संकेताग्रहणात् वाचकत्वेन {\color{DodgerBlue3}“योजना”} नास्तीति {\color{DodgerBlue3}“विवेचितं”} प्राक् । (४९९)
	\pend
      \label{div_pvv.2.500}\edlabel{div_pvv.2.500}
	  
	% new div opening: depth here is 2
	

	  \begin{center}%% label @type='head'
	\textbf{(ङ. तदा न विच्छिन्नं दर्शनम्)}
	\end{center}
	
	  \bigskip
	  \begingroup
	  \large
	
	    
	    \stanza[\smallbreak]
	\label{pv.2.500a}\edlabel{pv.2.500a}\flagstanza{\tiny\textenglish{...2.500a}}विच्छिन्नं पश्यतोप्यक्षैर्घटयेद्यदि कल्पना ।\&[\smallbreak]


	
	  \endgroup
	

	  \pstart अथाक्षैरिन्द्रियज्ञानैः स्वज्ञानव्यवहितैरपरापरमर्थं {\color{DodgerBlue3}“विच्छिन्नं पश्यतोपि”} ज्ञानानुचरी {\color{DodgerBlue3}“कल्प\edlabel{pvv.267-7}\footnote{\label{pvv.267-7}  ७ ग्राहकज्ञानानुभवस्यानन्तरजा तृतीया ।}ना”} तदेवेदमित्येकत्वेन {\color{DodgerBlue3}“घटयेदिति यद्यु”}च्यते (।)
	\pend
      \leavevmode\marginnote{\textenglish{268/s}}

	  \pstart तदा विच्छिन्नदर्शनमेव नास्तीति वक्तुमाह (।)
	\pend
      
	  \bigskip
	  \begingroup
	  \large
	
	    
	    \stanza[\smallbreak]
	\label{pv.2.500b}\edlabel{pv.2.500b}\flagstanza{\tiny\textenglish{...2.500b}}अर्थस्य तत्संवित्तेश्च सततं भासमानयोः ॥ ५०० ॥\&[\smallbreak]


	
	  \endgroup
	

	  \pstart {\color{DodgerBlue3}“अर्थस्य”} ग्राह्यस्य {\color{DodgerBlue3}“तत्संवित्तेश्च”} विजातीयाव्यव\edlabel{pvv.268-1}\footnote{\label{pvv.268-1}  १ इन्द्रियज्ञाने अर्थोपि निरन्तरं भासते तज्ज्ञानमपि तदाकारं सम्वेद्यतेऽक्षसिद्धं ।}कीर्ण्णत्वात् । {\color{DodgerBlue3}“सततं भासमानयोर”}नयोरविच्छेद\edlabel{pvv.268-2}\footnote{\label{pvv.268-2}  २ प्रत्यभिज्ञा नेह विच्छेदावसायसाधकं किञ्चित्प्रमाणमस्ति ।}भासनस्य (। ५००)
	\pend
      \label{div_pvv.2.501}\edlabel{div_pvv.2.501}
	  
	% new div opening: depth here is 2
	
	  \bigskip
	  \begingroup
	  \large
	
	    
	    \stanza[\smallbreak]
	\label{pv.2.501}\edlabel{pv.2.501}\flagstanza{\tiny\textenglish{....2.501}}बाधकेऽसति सन्न्याये विच्छिन्ने इति तत्कुतः ।&बुद्धीनां शक्तिनियमादिति चेत्स कुतो मतः ॥ ५०१ ॥\&[\smallbreak]


	
	  \endgroup
	

	  \pstart {\color{DodgerBlue3}“बाधके”} संश्चासौ {\color{DodgerBlue3}“न्यायश्च”} तस्मिन् प्रमाणभूते बाधके{\color{DodgerBlue3}“ऽसती”}त्यर्थः । तेऽर्थसंवित्ती ज्ञानज्ञानेन {\color{DodgerBlue3}“विच्छिन्ने इति”} यदुच्यते {\color{DodgerBlue3}“तत्कुतः\edlabel{pvv.268-3}\footnote{\label{pvv.268-3}  ३ यस्मिन्विच्छेदे सति कल्पिकया घटनं ।} । बुद्धीनामे”}कैकबुद्धिजनन{\color{DodgerBlue3}“शक्तिनियमात्”} न ग्राह्यबुद्ध्या सह तद्‏ग्राहिका बुद्धिर्भवति । ततो विच्छिन्नतयाऽर्थदर्शन{\color{DodgerBlue3}“मिति चेत् । स”} युगपत् ज्ञानानुत्पादः {\color{DodgerBlue3}“कुतो”} हेतोर्मतः । (५०१)
	\pend
      \label{div_pvv.2.502}\edlabel{div_pvv.2.502}
	  
	% new div opening: depth here is 2
	
	  \bigskip
	  \begingroup
	  \large
	
	    
	    \stanza[\smallbreak]
	\label{pv.2.502a}\edlabel{pv.2.502a}\flagstanza{\tiny\textenglish{...2.502a}}युगपद् बुद्ध्यदृष्टेश्चेत् तदेवेदं विचार्यते ।\&[\smallbreak]


	
	  \endgroup
	

	  \pstart {\color{DodgerBlue3}“युगपद् बु (द्) ध्यो”}रुत्पन्नयोर{\color{DodgerBlue3}“दृष्टेश्चेत्”} यद्यदर्शनं प्रमाणं विच्छेदस्यापि तदस्तीति सोप्ययुक्ताभ्युपगमः । अथ विच्छेदादर्शनेपि विचारात् सा सिध्यति तदा {\color{DodgerBlue3}“तदेवेदं युगपद् बुद्ध्य”}दर्शनमपि बाध्यतया न युगपद् बुद्ध्युत्पादबाधनासमर्थमिति {\color{DodgerBlue3}“विचार्यते”} । तत् किमस्योपन्यासेन ।
	\pend
      

	  \begin{center}%% label @type='head'
	\textbf{च. न युगपत् चित्तद्वयसंप्रतिपत्तिः}
	\end{center}
	

	  \pstart ननूक्तं भ ग व ता “ऽस्था\edlabel{pvv.268-4}\footnote{\label{pvv.268-4}  ४ वैभाष्यः स्ववित्तिं नेच्छति ।} नमेतत् यत् द्वे चित्ते युगपत् संप्रतिपद्येयातामि” ति । तत्कथमित्याह (।)
	\pend
      
	  \bigskip
	  \begingroup
	  \large
	
	    
	    \stanza[\smallbreak]
	\label{pv.2.502b}\edlabel{pv.2.502b}\flagstanza{\tiny\textenglish{...2.502b}}तासां समानजातीये सामर्थ्यनियमो भवेत् ॥ ५०२ ॥\&[\smallbreak]


	
	  \endgroup
	

	  \pstart {\color{DodgerBlue3}“ता\edlabel{pvv.268-5}\footnote{\label{pvv.268-5}  ५ आभिप्रायिकमेतत् ।}सां”} बुद्धीनां विकल्पिकानामिन्द्रियजानाञ्च {\color{DodgerBlue3}“समानजातीय”} एकस्मिन् ज्ञाने कर्त्तव्ये {\color{DodgerBlue3}“सामर्थ्यनियम”}स्तथा भगवता {\color{DodgerBlue3}“प्रतिपादितो भवेत्”} । ततो ग्राह्य\edlabel{pvv.268-6}\footnote{\label{pvv.268-6}  ६ विषयज्ञानन्तदनुभवज्ञानञ्च ।}ग्राहकमसमानजातीयं बुद्धिद्वयमपि सह जायेत । समानजातीयन्तु न सह जायते । (५०२)
	\pend
      \leavevmode\marginnote{\textenglish{269/s}}\label{div_pvv.2.503_2.504}\edlabel{div_pvv.2.503_2.504}
	  
	% new div opening: depth here is 2
	
	  \bigskip
	  \begingroup
	  \large
	
	    
	    \stanza[\smallbreak]
	\label{pv.2.503a}\edlabel{pv.2.503a}\flagstanza{\tiny\textenglish{...2.503a}}तथा हि सम्यक् लक्ष्यन्ते विकल्पाः क्रमभाविनः ।\&[\smallbreak]


	
	  \endgroup
	

	  \pstart {\color{DodgerBlue3}“तथा हि विकल्पाः क्रमभाविनः सम्यग् लक्ष्यन्ते”} । इन्द्रियज्ञानानि च समानजातीयानि क्रमवन्ति च दृश्यन्ते । विकल्पेन्द्रियज्ञाने चक्षुःश्रोत्रादिज्ञाने च सहोत्पद्यते (न्ते) ।
	\pend
      

	  \begin{center}%% label @type='head'
	\textbf{(४) प्रत्यभिज्ञाचिन्ता}
	\end{center}
	

	  \begin{center}%% label @type='head'
	\textbf{क. एकत्वमन्तरेण प्रत्यभिज्ञानम्}
	\end{center}
	
	  \bigskip
	  \begingroup
	  \large
	
	    
	    \stanza[\smallbreak]
	\label{pv.2.503b}\edlabel{pv.2.503b}\flagstanza{\tiny\textenglish{...2.503b}}एतेन यः समक्षेर्थे प्रत्यभिज्ञानकल्पनाम् ॥ ५०३ ॥\&[\smallbreak]


	
	  \endgroup
	
	  \bigskip
	  \begingroup
	  \large
	
	    
	    \stanza[\smallbreak]
	\label{pv.2.504a}\edlabel{pv.2.504a}\flagstanza{\tiny\textenglish{...2.504a}}स्पष्टावभासां प्रत्यक्षां कल्पयेत् सोपि वारितः ॥\&[\smallbreak]


	
	  \endgroup
	

	  \pstart {\color{DodgerBlue3}“एतेन”} विकल्पाविकल्पयोः सहोत्पादेन यो {\color{DodgerBlue3}“मीमांस\edlabel{pvv.269-1}\footnote{\label{pvv.269-1}  १ जैमिनीयः ।}कः समक्षे”} प्रत्यक्षे{\color{DodgerBlue3}“ऽर्थे स”} एवायमित्येक\edlabel{pvv.269-2}\footnote{\label{pvv.269-2}  २ विकल्प एव प्रत्यभिज्ञा ।}त्वविषयां {\color{DodgerBlue3}“प्रत्यभिज्ञानाख्यां कल्पनां”} (५०३) {\color{DodgerBlue3}“स्पष्टप्रतिभा\edlabel{pvv.269-3}\footnote{\label{pvv.269-3}  ३ विषयज्ञानन्तदनुभवज्ञानञ्च परो भ्रान्त्या ।}सां प्रत्यक्षां”} प्रमाणं {\color{DodgerBlue3}“कल्पयेत् सोपि निवारितः”} । विकल्पो हि विकल्पितार्थगोचरत्वादस्पष्टप्रतिभास एव । विष (?श) दप्रतिभासनेन्द्रियज्ञानेन सहभावित्वात् स्पष्ट-\leavevmode\marginnote{\textenglish{53a/MA}} ग्राह्यारोपोऽध्यवसीयते ।
	\pend
      

	  \pstart किञ्चैकत्वमन्तरेणापि प्रत्यभिज्ञानं दृश्यते इति दर्शयन्नाह (।)
	\pend
      
	  \bigskip
	  \begingroup
	  \large
	
	    
	    \stanza[\smallbreak]
	\label{pv.2.504b}\edlabel{pv.2.504b}\flagstanza{\tiny\textenglish{...2.504b}}केशगोलकदीपादावपि स्वष्टावभासनात् ॥ ५०४ ॥\&[\smallbreak]


	
	  \endgroup
	

	  \pstart लूनपुनर्जाते {\color{DodgerBlue3}“केशादौ”} मायाकारदर्शिते {\color{DodgerBlue3}“गोलकादौ”} क्षणविनाशि{\color{DodgerBlue3}“दीपादौ । स्पष्टावभासनात्”} । (५०४)
	\pend
      \label{div_pvv.2.505}\edlabel{div_pvv.2.505}
	  
	% new div opening: depth here is 2
	
	  \bigskip
	  \begingroup
	  \large
	
	    
	    \stanza[\smallbreak]
	\label{pv.2.505}\edlabel{pv.2.505}\flagstanza{\tiny\textenglish{....2.505}}प्रतीतभेदेप्यध्यक्षा धीः कथन्तादृशी भवेत् ॥&तस्मान्न प्रत्यभिज्ञानाद्वर्ण्णाद्येकत्वनिश्चयः ॥ ५०५ ॥\&[\smallbreak]


	
	  \endgroup
	

	  \pstart प्रत्यक्षात् {\color{DodgerBlue3}“प्रतीते भेदे”} सा प्रत्यभिज्ञा {\color{DodgerBlue3}“तादृशी”} एकत्वाध्यवसायिनी दृश्यमाना{\color{DodgerBlue3}“ध्यक्षा धीः कथं\edlabel{pvv.269-4}\footnote{\label{pvv.269-4}  ४ एकत्वसाधनाभिमतस्यानेकत्रापि दर्शनात् ।} भवेत्”} । प्राप्नोति च प्रत्यभिज्ञा प्रत्यक्षवादिनो म\edlabel{pvv.269-5}\footnote{\label{pvv.269-5}  ५ इति साध्यं हेतुः परं ।}ते । {\color{DodgerBlue3}“तस्मान्नास्ति प्रत्यभिज्ञानाद् वर्ण्णाद्ये \edlabel{pvv.269-6}\footnote{\label{pvv.269-6}  ६ रूपादिः ।} कत्वनिश्चयः”} । (५०५)
	\pend
      \leavevmode\marginnote{\textenglish{270/s}}\label{div_pvv.2.506}\edlabel{div_pvv.2.506}
	  
	% new div opening: depth here is 2
	
	  \bigskip
	  \begingroup
	  \large
	
	    
	    \stanza[\smallbreak]
	\label{pv.2.506}\edlabel{pv.2.506}\flagstanza{\tiny\textenglish{....2.506}}पूर्व्वानुभूतस्मरणात्तद्धर्मारोपणाद्विना ।&स एवायमिति ज्ञानं नास्ति तच्चाक्षजे कुतः ॥ ५०६ ॥\&[\smallbreak]


	
	  \endgroup
	

	  \pstart यस\edlabel{pvv.270-1}\footnote{\label{pvv.270-1}  १ स्वीकृत्येन्द्रियत्वं दूषयित्वा ऐन्द्रियस्वभावमधुनाह ।}मा{\color{DodgerBlue3}“त्पूर्व्वानुभूतस्यार्थस्य”} स्मरणा{\color{DodgerBlue3}“त्तद्धर्मस्य”} विद्यमानत्वे{\color{DodgerBlue3}“नारोप\edlabel{pvv.270-2}\footnote{\label{pvv.270-2}  २ अतीतारोपं विना स एवायमिति नास्ति ।}णात् विना स एवायमिति”} ज्ञानमेकत्वविषयं नास्ति(।) तच्च पूर्व्वानुभूतस्मरणं तद्धर्मारोपणञ्चाक्षजे वर्तमानवस्तुबलभाविनि कुतः सम्भवति । (५०६)
	\pend
      \label{div_pvv.2.507_2.508}\edlabel{div_pvv.2.507_2.508}
	  
	% new div opening: depth here is 2
	

	  \begin{center}%% label @type='head'
	\textbf{(ख. अविच्छिन्नं वर्णादिदर्शनम्)}
	\end{center}
	

	  \pstart स्यादेतद् (।) अर्थज्ञानयोर्युगपत्स\edlabel{pvv.270-3}\footnote{\label{pvv.270-3}  ३ बौद्धेन स्वीकृतयुगपज्ज्ञानसम्भवे सति ।}म्भवे सत्यविच्छिन्नं वर्ण्णादिदर्शनं स्यादित्याह (।)
	\pend
      
	  \bigskip
	  \begingroup
	  \large
	
	    
	    \stanza[\smallbreak]
	\label{pv.2.507}\edlabel{pv.2.507}\flagstanza{\tiny\textenglish{....2.507}}न चार्थज्ञानसम्वित्योर्युगपत्सम्भवो यतः ।&लक्ष्येते प्रतिभासौ द्वौ नार्थार्थज्ञानयोः पृथक् ॥ ५०७ ॥\&[\smallbreak]


	
	  \endgroup
	
	  \bigskip
	  \begingroup
	  \large
	
	    
	    \stanza[\smallbreak]
	\label{pv.2.508a}\edlabel{pv.2.508a}\flagstanza{\tiny\textenglish{...2.508a}}न ह्यर्थाभासि च ज्ञानमर्थो बाह्यश्च केवलः ।\&[\smallbreak]


	
	  \endgroup
	

	  \pstart न चार्थज्ञानयोर्ये सम्वित्ती तयो{\color{DodgerBlue3}“र्युगपत्सम्भवो”}स्ति (।) यतोर्थस्या{\color{DodgerBlue3}“र्थज्ञान”}स्य द्वौ {\color{DodgerBlue3}“प्रतिभा\edlabel{pvv.270-4}\footnote{\label{pvv.270-4}  ४ बौद्धेन स्वीकृतयुगपज्ज्ञानसम्भवे सति ।}सौ पृथग्न”} भेदन {\color{DodgerBlue3}“ल\edlabel{pvv.270-5}\footnote{\label{pvv.270-5}  ५ उपलब्धिलक्षणप्राप्तस्यानुपलब्धेः ।}क्ष्यते”} । (५०७) न ह्य{\color{DodgerBlue3}“र्थाभासि”} तत् {\color{DodgerBlue3}“ज्ञानं । अर्थो बाह्यश्च केवेलो”} बुद्धिव्यतिरिक्त इति द्वौ प्रतिभासौ सम्भवतः ।
	\pend
      

	  \pstart स्यादेतत् । अर्थज्ञानज्ञानयोर्भिन्नावेव प्रतिभासौ केवलं तदुत्तरया विकल्पबुद्ध्या एकत्वेन गृह्येते अत्राह (।)
	\pend
      
	  \bigskip
	  \begingroup
	  \large
	
	    
	    \stanza[\smallbreak]
	\label{pv.2.508b}\edlabel{pv.2.508b}\flagstanza{\tiny\textenglish{...2.508b}}एकाकारमतिग्राह्ये भेदाभावप्रसङ्गतः ॥ ५०८ ॥\&[\smallbreak]


	
	  \endgroup
	

	  \pstart {\color{DodgerBlue3}“एकाकारया मत्या”} विकल्पिकया {\color{DodgerBlue3}“ग्राह्ये”}ऽर्थज्ञानयोर्ज्ञाने यदि तदा तयोर्भेद\edlabel{pvv.270-6}\footnote{\label{pvv.270-6}  ६ बुद्ध्याकाराभेदादर्थाभेदेन ।}स्या{\color{DodgerBlue3}“भावप्रसङ्गत”} एकत्वमेवाभ्युपगन्तव्यं । अर्थज्ञानज्ञानयोः प्रतिभास{\color{DodgerBlue3}“भेदाभावात्”} । (५०८)
	\pend
      \label{div_pvv.2.509}\edlabel{div_pvv.2.509}
	  
	% new div opening: depth here is 2
	
	  \bigskip
	  \begingroup
	  \large
	
	    
	    \stanza[\smallbreak]
	\label{pv.2.509}\edlabel{pv.2.509}\flagstanza{\tiny\textenglish{....2.509}}सूपलक्षेण भेदेन यौ सम्वित्तौ न लक्षितौ ।&अर्थार्थप्रत्ययौ पश्चात् स्मर्येते तौ पृथक् कथम् ॥ ५०९ ॥\&[\smallbreak]


	
	  \endgroup
	\leavevmode\marginnote{\textenglish{271/s}}

	  \pstart अर्थश्चार्थप्रत्ययश्चा{\color{DodgerBlue3}“र्थार्थप्रत्ययौ सूपलक्षेण भेदेन संवित्तावनुभवे, न लक्षितौ । तौ पश्चात् कालान्तरे पृथग्”} भेदेनायमर्थोऽर्थज्ञा\edlabel{pvv.271-1}\footnote{\label{pvv.271-1}  १ स्मर्येते च तस्मान्न युगपद् द्वयं ।} नञ्चेदमिति {\color{DodgerBlue3}“कथं स्मर्येते”} युगपदर्थंज्ञानतज्ज्ञानयोरुत्पादे प्रतिभासनानात्वप्रसङ्गात् । (५०९)
	\pend
      \label{div_pvv.2.510}\edlabel{div_pvv.2.510}
	  
	% new div opening: depth here is 2
	
	  \bigskip
	  \begingroup
	  \large
	
	    
	    \stanza[\smallbreak]
	\label{pv.2.510}\edlabel{pv.2.510}\flagstanza{\tiny\textenglish{....2.510}}क्रमेणानुभवोत्पादेप्यर्थार्थमनसोरयम् ।&प्रतिभासस्य नानात्वचोद्यदोषो दुरुद्धरः ॥ ५१० ॥\&[\smallbreak]


	
	  \endgroup
	

	  \pstart अर्थार्थमनसोः {\color{DodgerBlue3}“क्रमेणानुभवोत्पादेपीष्यमाणेऽयं प्रतिभासस्य नाना\edlabel{pvv.271-2}\footnote{\label{pvv.271-2}  २ पूर्व्वेन्द्रियज्ञानस्य मनसानुभवेपि युगपद् भानम्मा भूद्विच्छिन्नमुत्पद्येत ।}त्वचोद्य”}लक्षणो {\color{DodgerBlue3}“दोषो दुरुद्धरः”} । द्वयप्रतिभासस्य क्रमेणाभ्युपगमात् । (५१०)
	\pend
      \label{div_pvv.2.511_2.512}\edlabel{div_pvv.2.511_2.512}
	  
	% new div opening: depth here is 2
	
	  \bigskip
	  \begingroup
	  \large
	
	    
	    \stanza[\smallbreak]
	\label{pv.2.511}\edlabel{pv.2.511}\flagstanza{\tiny\textenglish{....2.511}}अर्थसंवेदनं तावत्ततोर्थाभासवेदनम् ।&न हि संवेदनं शुद्धं भवेदर्थस्य वेदनम्\edlabel{pvv.271-asterisk}\footnote{\label{pvv.271-asterisk}  * व्त्तिकृतात्त्वविवृतैषा ।} ॥ ५११ ॥\&[\smallbreak]


	
	  \endgroup
	
	  \bigskip
	  \begingroup
	  \large
	
	    
	    \stanza[\smallbreak]
	\label{pv.2.512}\edlabel{pv.2.512}\flagstanza{\tiny\textenglish{....2.512}}तथा हि नीलाद्याकार एक एकं च वेदनम् ।&लक्ष्यते न तु नीलाभे वेदने वेदनं परम् ॥ ५१२ ॥\&[\smallbreak]


	
	  \endgroup
	

	  \pstart न चार्थज्ञानं ज्ञानञ्च कदाचिद् भेदेन प्रतीयते\edlabel{pvv.271-3}\footnote{\label{pvv.271-3}  ३ अर्थज्ञानं ज्ञानज्ञानारूढमिति ।}। (५११) {\color{DodgerBlue3}“तथा हि नीलाद्याकार एको”} ग्राह्यतया {\color{DodgerBlue3}“एकञ्च”} तद्ग्राहकं {\color{DodgerBlue3}“वेदनं लक्ष्यते न तु नीलाभे वेदने”} पृथग् ग्राहक{\color{DodgerBlue3}“मपरं वेदनं”} लक्ष्यते । तस्माज्‏ज्ञानस्य विदितस्यान्येनानुभवासम्भवे स्ववेद\edlabel{pvv.271-4}\footnote{\label{pvv.271-4}  ४ एकोऽर्थाकारः सम्वेदनाकारश्चेत्यप्रसङ्गोऽत्र ।}नमेव तदिति व्यवतिष्ठते ॥ (५१२)
	\pend
      \label{div_pvv.2.513_2.514}\edlabel{div_pvv.2.513_2.514}
	  
	% new div opening: depth here is 2
	

	  \pstart किञ्च (।)
	\pend
      
	  \bigskip
	  \begingroup
	  \large
	
	    
	    \stanza[\smallbreak]
	\label{pv.2.513a}\edlabel{pv.2.513a}\flagstanza{\tiny\textenglish{...2.513a}}ज्ञानान्तरेणानुभवो भवेत्तत्रापि च स्मृतिः ।&दृष्टा; तद्वेदनं केन तस्याप्यन्येन चेत्;\&[\smallbreak]


	
	  \endgroup
	

	  \pstart नीलादिविषयस्य ज्ञानस्य {\color{DodgerBlue3}“ज्ञानान्तरेणानुभवो भवेत्\edlabel{pvv.271-5}\footnote{\label{pvv.271-5}  ५ शास्त्रकृत् स्वयं परित्यज्याचार्यीयमाह (।) न ह्यसौ स्मृतिरभावितेषु जात इत्याचार्येण स्वसम्वित्तिसाधनायोक्तं । परेणात्र सिद्धसाधनमुक्तं विनापि स्वसंवित्तिं ज्ञानान्तरेण संविद् भविष्यतीति सिद्धान्तितमत्र । ज्ञानान्तरेणानुभवेऽनिष्टा तत्रापि हि स्मृतिः विषयान्तरसञ्चारस्तथा स्यात् स चेत् स इति व्याचष्टे ।} । तत्रापि ज्ञान-”} ज्ञानेपि हि\edlabel{pvv.271-6}\footnote{\label{pvv.271-6}  ६ भवतु नाम, तथापि चात्र स्मृतिर्दष्टासीन् मे ज्ञानज्ञानमिति सा न स्यात् स्वयन्तस्याननुभवात् ।} {\color{DodgerBlue3}“स्मृति”}र्दृष्ट । यदा ज्ञानान्तरालम्बकं ज्ञानं क्रमेण स्मर्यते न चागु\leavevmode\marginnote{\textenglish{272/s}} हीतं स्मर्यते इति तस्य ज्ञानज्ञानस्य वेदनं वक्तव्यं । {\color{DodgerBlue3}“तत् केनास्तु वेदनं”} यदि स्वसम्वेदनेन {\color{DodgerBlue3}“तदा पूर्व्वकस्यापि”} तथा स्थितिर्व्यर्थमन्येन वेदनाङ्गीकरणं (।) {\color{DodgerBlue3}“तस्याप्यन्येन\edlabel{pvv.272-1}\footnote{\label{pvv.272-1}  १ अनवस्था स्यादेवं ।} चेद्वेदनं”} तदा (।)
	\pend
      
	  \bigskip
	  \begingroup
	  \large
	
	    
	    \stanza[\smallbreak]
	\label{pv.2.513b}\edlabel{pv.2.513b}\flagstanza{\tiny\textenglish{...2.513b}}इमाम् ॥ ५१३ ॥\&[\smallbreak]


	
	  \endgroup
	
	  \bigskip
	  \begingroup
	  \large
	
	    
	    \stanza[\smallbreak]
	\label{pv.2.514}\edlabel{pv.2.514}\flagstanza{\tiny\textenglish{....2.514}}मालां ज्ञानविदां कोयं जनयत्यनुबन्धिनीम् ॥&पूर्व्वा धीः सैव चेन्न स्यात्सञ्चारो विषयान्तरे ॥ ५१४ ॥\&[\smallbreak]


	
	  \endgroup
	

	  \pstart ज्ञानवृत्तीनामि{\color{DodgerBlue3}“मां”} (५१३) {\color{DodgerBlue3}“मा\edlabel{pvv.272-2}\footnote{\label{pvv.272-2}  २ स्वकं विषयान्तरासञ्चारन्तावदाह त्यक्त्वाचार्यीयं ।}लामनुबन्धिनीं”} प्रबन्धप्रवृत्तां {\color{DodgerBlue3}“को जनयति । न”} तावदर्थेन्द्रियादिसामग्री तस्या अर्थज्ञानमात्रजननव्यापारत्वात् । {\color{DodgerBlue3}“सैव”} ग्राह्या {\color{DodgerBlue3}“धीः”} पूर्व्विका स्वग्राहिकां धियं जनयति सा च स्वग्राहिकामिति । प्रबन्धप्रवृत्तिरिति {\color{DodgerBlue3}“चेत्”} । एवं सति स्वकारणग्रहणप्रवणत्वा{\color{DodgerBlue3}“द्विषयान्तरे”} ज्ञानादन्यस्मिन्नर्थे {\color{DodgerBlue3}“सञ्चारो”} ग्राहकत्वेन प्रवृत्तिर्न {\color{DodgerBlue3}“स्यात्\edlabel{pvv.272-3}\footnote{\label{pvv.272-3}  ३ एकार्थविषयैव ज्ञानपरम्पराऽसञ्चारं स्यात् ।}”} । (५१४)
	\pend
      \label{div_pvv.2.515}\edlabel{div_pvv.2.515}
	  
	% new div opening: depth here is 2
	

	  \pstart तथा हि (।)
	\pend
      
	  \bigskip
	  \begingroup
	  \large
	
	    
	    \stanza[\smallbreak]
	\label{pv.2.515}\edlabel{pv.2.515}\flagstanza{\tiny\textenglish{....2.515}}तां ग्राह्यलक्षणप्राप्तामासन्नां जनिकां धियम् ।&अगृहीत्वोत्तरं ज्ञानं गृह्णीयादपरं कथम् ॥ ५१५ ॥\&[\smallbreak]


	
	  \endgroup
	\leavevmode\marginnote{\textenglish{53b/MA}}

	  \pstart {\color{DodgerBlue3}“जनिकां धियमासन्नां ग्राह्यलक्षणप्राप्ता”}मगृहीत्वोत्तरं {\color{DodgerBlue3}“ज्ञानमपरं”} विषयं {\color{DodgerBlue3}“कथं गृह्णीयात्”} । (५१५)
	\pend
      \label{div_pvv.2.516}\edlabel{div_pvv.2.516}
	  
	% new div opening: depth here is 2
	

	  \pstart प्रत्यासन्नेनार्थेन प्रतिबद्धशक्तित्वात् पूर्व्वा धीरर्थग्राहकमेव ज्ञानं जनयति न स्वग्राहकमिति चेत् । आह (।)
	\pend
      
	  \bigskip
	  \begingroup
	  \large
	
	    
	    \stanza[\smallbreak]
	\label{pv.2.516}\edlabel{pv.2.516}\flagstanza{\tiny\textenglish{....2.516}}आत्मनि ज्ञानजनने स्वभावे नियताञ्च ताम् ।&को नामान्यो विबध्नीयाद् बहिरङ्गेऽन्तरङ्‏गिकाम् ॥ ५१६ ॥\&[\smallbreak]


	
	  \endgroup
	

	  \pstart {\color{DodgerBlue3}“आत्मनि विज्ञानजनने”} स्वग्राहकज्ञानोत्पादके {\color{DodgerBlue3}“स्वभावो नियतां”} व्यवस्थितां च तामिमाम{\color{DodgerBlue3}“न्तरङ्गिकामन्य”}निरपेक्षत्वात् {\color{DodgerBlue3}“को नामान्योर्थो बहिरङ्गः”} स्वज्ञानजनने चक्षुर्मनस्काराद्यपेक्षित्वा{\color{DodgerBlue3}“द्विबध्नीयात्”} । स्वग्राहकज्ञानजननप्रवृत्तां तिरस्कुर्य्यात् । येन विषयान्तरसञ्चारो धियः स्यात् । (५१६)
	\pend
      \label{div_pvv.2.517}\edlabel{div_pvv.2.517}
	  
	% new div opening: depth here is 2
	
	  \bigskip
	  \begingroup
	  \large
	
	    
	    \stanza[\smallbreak]
	\label{pv.2.517}\edlabel{pv.2.517}\flagstanza{\tiny\textenglish{....2.517}}बाह्यः सन्निहितोप्यर्थः तां विबध्नन् हि न प्रभुः ।&धियं नानुभवेत् कश्चिदन्यथार्थस्य सन्निधौ ॥ ५१७ ॥\&[\smallbreak]


	
	  \endgroup
	\leavevmode\marginnote{\textenglish{273/s}}

	  \pstart तस्माद्वा{\color{DodgerBlue3}“ह्यः सन्निहितोप्यर्थस्तां”} स्वग्राहकज्ञानजनने शक्तां {\color{DodgerBlue3}“धियं विबध्नन्”} प्रतिहातुं {\color{DodgerBlue3}“न”} प्रभुः शक्तः । {\color{DodgerBlue3}“अन्यथा”} यद्येवं नाभ्युपगम्यते {\color{DodgerBlue3}“तदार्थस्य सन्निधौ”} तत् ज्ञानस्यैवोत्पादनात् ज्ञानज्ञानस्यानुत्पादाद्धियं कश्चिन्नानुभवेत् । (५१७)
	\pend
      \label{div_pvv.2.518}\edlabel{div_pvv.2.518}
	  
	% new div opening: depth here is 2
	
	  \bigskip
	  \begingroup
	  \large
	
	    
	    \stanza[\smallbreak]
	\label{pv.2.518}\edlabel{pv.2.518}\flagstanza{\tiny\textenglish{....2.518}}न चासन्निहितार्थास्ति दशा काचिदतो धियः ।&उत्खातमूला स्मृतिरप्युत्सन्नेत्युज्ज्वलं मतम् ॥ ५१८ ॥\&[\smallbreak]


	
	  \endgroup
	

	  \pstart {\color{DodgerBlue3}“न च काचिद्दशा सन्निहितार्था/?/\edlabel{pvv.273-1}\footnote{\label{pvv.273-1}  १ पञ्चस्कन्धके भवे सदैवार्थः ।} /?/स्ति”} यत्र ज्ञानज्ञानमुत्पद्येत नार्थज्ञानमिति न स्यादनुभवो बुद्धेः । तस्माद् {\color{DodgerBlue3}“धियः स्मृतिरप्यु त्खातमूला उत्सन्ना”} बुद्धिस्मरणस्य हि मूलं बुद्ध्यनुभव {\color{DodgerBlue3}“इति”} ज्ञानाननुभवे कुतः स्मरणमिति ज्ञानान्तरेण ज्ञानानुभववादिना{\color{DodgerBlue3}“मुज्ज्वलं मतमि”}त्युपहसति । (५१८) किञ्च (।)
	\pend
      \label{div_pvv.2.519}\edlabel{div_pvv.2.519}
	  
	% new div opening: depth here is 2
	
	  \bigskip
	  \begingroup
	  \large
	
	    
	    \stanza[\smallbreak]
	\label{pv.2.519}\edlabel{pv.2.519}\flagstanza{\tiny\textenglish{....2.519}}अतीतादिविकल्पानां येषां नार्थस्य सन्निधिः ।&सञ्चारकरणाभावादुत्सीदेदर्थचिन्तनम् ॥ ५१९ ॥\&[\smallbreak]


	
	  \endgroup
	

	  \pstart येषामतीताद्यर्थं\edlabel{pvv.273-2}\footnote{\label{pvv.273-2}  २ भवतु नामाध्यक्षेषु विषयेषु सञ्चारोऽतीतानागतविकल्पे तु नार्थो यः ।}विषयाणां विकल्पानामर्थस्य सन्निधिर्नास्ति यः पूर्व्वकस्य ज्ञानस्य स्वग्राहकज्ञानजननशक्तिं प्रतिबध्नीयात् येन विषयान्तरग्राहिणो विकल्पाः स्युः । तत्र सञ्चारकारणस्यार्थस्य विकल्पविषयस्याभावात् कस्यचि{\color{DodgerBlue3}“दतीतादेरर्थ”}स्य चिन्तनं {\color{DodgerBlue3}“विकल्पनमुत्सीदे”}त् । (५१९)
	\pend
      \label{div_pvv.2.520_2.521}\edlabel{div_pvv.2.520_2.521}
	  
	% new div opening: depth here is 2
	

	  \pstart स्यादेतत् (।)
	\pend
      
	  \bigskip
	  \begingroup
	  \large
	
	    
	    \stanza[\smallbreak]
	\label{pv.2.520a}\edlabel{pv.2.520a}\flagstanza{\tiny\textenglish{...2.520a}}आत्मविज्ञानजनने शक्तिसंक्षयतः शनैः ।&विषयान्तरसञ्चारो यदि;\&[\smallbreak]


	
	  \endgroup
	

	  \pstart तज्ज्ञानानामा{\color{DodgerBlue3}“त्मनि”} ग्राहक{\color{DodgerBlue3}“ज्ञानजनने शनैः”} क्रमेण स्वग्राहकज्ञानजनिकायाः {\color{DodgerBlue3}“शक्तेः संक्षयतः”} तथाभूतज्ञानानुत्पत्तौ {\color{DodgerBlue3}“विषयान्तरे सञ्चारो”} ज्ञानप्रवृत्ति{\color{DodgerBlue3}“र्यदि”} कथ्यते ।
	\pend
      
	  \bigskip
	  \begingroup
	  \large
	
	    
	    \stanza[\smallbreak]
	\label{pv.2.520b}\edlabel{pv.2.520b}\flagstanza{\tiny\textenglish{...2.520b}}सैवार्थधीः कुतः ॥ ५२० ॥\&[\smallbreak]


	
	  \endgroup
	
	  \bigskip
	  \begingroup
	  \large
	
	    
	    \stanza[\smallbreak]
	\label{pv.2.521a}\edlabel{pv.2.521a}\flagstanza{\tiny\textenglish{...2.521a}}शक्तिक्षये पूर्वधियो न हि धीः प्राग्धियां विना ।\&[\smallbreak]


	
	  \endgroup
	

	  \pstart तदा पूर्व्वधियः शक्तिक्षये सति {\color{DodgerBlue3}“सैवार्थ”}ग्राहिका {\color{DodgerBlue3}“धीः कुतो”} जा\edlabel{pvv.273-3}\footnote{\label{pvv.273-3}  ३ अर्थबुद्धिजननापेक्षया पूर्व्वाऽन्त्या तस्याः क्षये सर्व्वा बुद्धिं न जनयेदविशेषात् ।}यते । (५२०) न हि प्राग्धिया पूर्व्विकां धियं समर्थां विनोत्तरमर्थज्ञानमुत्पद्यते ।
	\pend
      
	  \bigskip
	  \begingroup
	  \large
	
	    
	    \stanza[\smallbreak]
	\label{pv.2.521b}\edlabel{pv.2.521b}\flagstanza{\tiny\textenglish{...2.521b}}अन्यार्थाशक्तिविगुणे ज्ञाने ज्ञानोदयागतेः ॥ ५२१ ॥\&[\smallbreak]


	
	  \endgroup
	\leavevmode\marginnote{\textenglish{274/s}}

	  \pstart यदि च सा-शक्तत्वात् स्वग्राहिकाम्धियं कर्त्तुमसमर्थार्थधियमपि न कुर्य्यात् ।\edlabel{pvv.274-1}\footnote{\label{pvv.274-1}  १ कुत एतदिति युक्तिमाह । सन्निहितेपि विषये ज्ञानोत्पादानुपलब्धेः ।} तथा ह्यन्यस्मि{\color{DodgerBlue3}“न्नर्थे आसक्ति”}रभिष्वङ्गस्तया {\color{DodgerBlue3}“विगुणे”} पूर्व्वे {\color{DodgerBlue3}“विज्ञाने”} तदुत्तरस्य {\color{DodgerBlue3}“ज्ञानस्योदयागतेः”} जन्माप्रतीतेः पूर्व्वबुद्धेः सामर्थ्यादुत्तरबुद्धेर्जन्मेति निश्चीयते (।) ततः समर्था पूर्व्वबुद्धिः स्वग्राहिणीमेव धियं जनयेत् । तत्र परापेक्षाविरहात् । (५२१)
	\pend
      \label{div_pvv.2.522}\edlabel{div_pvv.2.522}
	  
	% new div opening: depth here is 2
	

	  \pstart न\edlabel{pvv.274-2}\footnote{\label{pvv.274-2}  २ विज्ञानवादिन आलय (विज्ञानं) विविधवासनाभावितं नदीस्योतोवदविरतं इन्द्रियज्ञानानां प्रवर्तकं ।}न्वा ल य वि ज्ञा नात् सकृत् षट् प्रवृत्तिविज्ञानानि जायन्त इतीष्यते ततत्सान्यर्थाशक्तिवैगुण्येपि पू\edlabel{pvv.274-3}\footnote{\label{pvv.274-3}  ३ मनसः ।}र्व्वचेतस आलयज्ञानान्तरं जायेरन् आलयज्ञानस्याव्याहतशक्तित्वादित्याह\edlabel{pvv.274-4}\footnote{\label{pvv.274-4}  ४ न च प्रवर्तन्ते इति मन एवोपादानं नालयं ।} (।)
	\pend
      
	  \bigskip
	  \begingroup
	  \large
	
	    
	    \stanza[\smallbreak]
	\label{pv.2.522}\edlabel{pv.2.522}\flagstanza{\tiny\textenglish{....2.522}}सकृद्विजातीयजातावप्येंकेन पटीयसा ।&चित्तेनाहितवैगुण्यादालयान्नान्यसम्भवः ॥ ५२२ ॥\&[\smallbreak]


	
	  \endgroup
	

	  \pstart सकृद्विजातीया\edlabel{pvv.274-5}\footnote{\label{pvv.274-5}  ५ आलयाद्विजातीयत्वं । मनसो व्याकृतत्वेन, ऐन्द्रियाणां च कादाचित्कत्वेन ।}नां प्रवृतिज्ञानानां जातावप्यालयज्ञानादिष्टायामेकेन चित्तेन {\color{DodgerBlue3}“स्वविषयासक्तेन”} पटीयसा विषयान्तरज्ञानजननं प्रत्याहितमारोपितं {\color{DodgerBlue3}“वैगुण्यं यस्य तस्मादालयादन्य”}स्य विषयान्तरग्राहिज्ञानस्य {\color{DodgerBlue3}“सम्भवो न”} भवति । (५२२)
	\pend
      \label{div_pvv.2.523}\edlabel{div_pvv.2.523}
	  
	% new div opening: depth here is 2
	

	  \pstart न ह्यालयज्ञानमित्येव प्रवृत्तिज्ञानानि भवन्ति (।) किन्तु मनस्कारसाद्गुण्यम{\color{DodgerBlue3}“पेक्षन्तेऽन्यथा”} यद्येवं नेष्यते तदा (।)
	\pend
      
	  \bigskip
	  \begingroup
	  \large
	
	    
	    \stanza[\smallbreak]
	\label{pv.2.523}\edlabel{pv.2.523}\flagstanza{\tiny\textenglish{....2.523}}नापेक्षेतान्यथा साम्यं मनोवृत्तेर्मनोन्तरम् ।&मनोज्ञानक्रमोत्पत्तिरप्यपेक्षा-प्रसाधनी ॥ ५२३ ॥\&[\smallbreak]


	
	  \endgroup
	\leavevmode\marginnote{\textenglish{54a/MA}}

	  \pstart {\color{DodgerBlue3}“मनसः समनन्तर”}प्रत्ययस्य वृत्तेः {\color{DodgerBlue3}“साम्य\edlabel{pvv.274-6}\footnote{\label{pvv.274-6}  ६ आलयाद्विजातीयत्वं । अवैगुण्यं साम्यं ।}म”}र्थान्तरानासक्तत्वेनानुकूल्यमुत्पित्सु {\color{DodgerBlue3}“मनोऽन्तरं नापेक्षे”}त । अपेक्षते च । ततो मनस्कारानुकूलता-सापेक्षादालयात् ज्ञानोत्पत्तिः । तथा {\color{DodgerBlue3}“मनोज्ञानानां”} विकल्पानां {\color{DodgerBlue3}“क्रमेणोत्पत्ति”}र्युगपदनुत्पादो{\color{DodgerBlue3}“प्यु”}त्तरज्ञानानां {\color{DodgerBlue3}“पूर्व्वज्ञानापेक्षा प्रसाधनी”} । (५२३)
	\pend
      \label{div_pvv.2.524}\edlabel{div_pvv.2.524}
	  
	% new div opening: depth here is 2
	
	  \bigskip
	  \begingroup
	  \large
	
	    
	    \stanza[\smallbreak]
	\label{pv.2.524}\edlabel{pv.2.524}\flagstanza{\tiny\textenglish{....2.524}}एकत्वान्मनसोन्यस्मिन्सक्तस्यान्यागतेर्यदि ।&ज्ञानान्तरस्यानुदयो न कदाचित्सहोदयात् ॥ ५२४ ॥\&[\smallbreak]


	
	  \endgroup
	\leavevmode\marginnote{\textenglish{275/s}}

	  \pstart अविगुणपूर्व्वज्ञानानपेक्षायामालयादिन्द्रियज्ञानानीव विकल्पज्ञानान्यपि सह जायेरन् ।\edlabel{pvv.275-1}\footnote{\label{pvv.275-1}  १ यतो विषयान्तरादिदुक्षुः स्यात् ।} अणुपरिमाणस्य नित्यस्यैकस्य {\color{DodgerBlue3}“मनसोऽन्यस्मिन्नि”}न्द्रिये {\color{DodgerBlue3}“सक्तस्यान्यत्रेन्द्रियान्त”}रेऽगतेरगमनादिन्द्रियान्तरजस्य {\color{DodgerBlue3}“ज्ञानान्तरस्यानुदयो”} यदि सम्मतः । सो पि न युक्तः । {\color{DodgerBlue3}“कदाचिद्”} नर्तकीदृष्ट्यवस्थादिष्वनेकविषयसन्निपाते चक्षुरादि\edlabel{pvv.275-2}\footnote{\label{pvv.275-2}  २ सकृदपि दीर्घशष्कुलीसमर्मरर्ध्वानं समुद्वहद्बहलामोदां धवलादिरसान्वितां अतिस्पर्शवतीं कुड्यां मिश्रतो (? ता) नेकेन्द्रियज्ञानस्य सम्वेदनात् स्फटिकतुल्ये समनन्तरे सकृत्संगतसर्व्वार्थेष्विन्द्रियेष्वसत्स्वपीत्यादौ साधितं प्राक् ।}ज्ञानानां सहोदयात् ।(५२४)
	\pend
      \label{div_pvv.2.525}\edlabel{div_pvv.2.525}
	  
	% new div opening: depth here is 2
	

	  \pstart यदा च न पटीयान् कश्चिद्विषयः प्रत्युपतिष्ठते पुरुषस्य च न क्वचिद्विशेषेणेच्छा भवति तदा\edlabel{pvv.275-3}\footnote{\label{pvv.275-3}  ३ वैशेषिक आह (।) अणु मनोद्रव्यं पृथिव्यादिद्रव्येषु पतितं । ज्ञानन्त्वात्मगुणो नवस्वात्मगुणेषु मध्ये पाठात् ।}(।)
	\pend
      
	  \bigskip
	  \begingroup
	  \large
	
	    
	    \stanza[\smallbreak]
	\label{pv.2.525}\edlabel{pv.2.525}\flagstanza{\tiny\textenglish{....2.525}}समवृत्तौ च तुल्यत्त्वात्सर्व्वदान्यागतिर्भवेत् ।&जन्म चात्ममनोयोगमात्रजानां सकृद् भवेत् ॥ ५२५ ॥\&[\smallbreak]


	
	  \endgroup
	

	  \pstart अर्थपुरुषेच्छायाः {\color{DodgerBlue3}“समा”}यां साधारणायां {\color{DodgerBlue3}“वृत्तौ च”} मनस एकेन्द्रियसम्बद्धस्य तुल्यत्वाद्विषयान्तरे प्रेरकाभावात् उदासीनत्वात् {\color{DodgerBlue3}“सर्व्वदा”} तदिन्द्रियज्ञानोत्पत्तौ सत्या{\color{DodgerBlue3}“मन्य”}स्येन्द्रियान्तरज्ञानज्ञेयस्या{\color{DodgerBlue3}“गतिर्भवेत्”} प्रतीतिर्न स्यात् । अस्ति च क्वचिदनासक्तस्य विषयान्तरेष्वेकस्याप्यनेकार्थदर्शनं ।\edlabel{pvv.275-4}\footnote{\label{pvv.275-4}  ४ यत्रात्मा मनसा तदिन्द्रियेण तद्विषयेण तत्र क्रमोत्पत्तिकल्पनैवं स्यादपि । आत्ममनसोर्नित्यत्वादविशेषात्तत्प्रतिबद्धसन्निकर्षोप्यविशिष्टः ।} {\color{DodgerBlue3}“आत्ममनोयोगमात्रजानां”} विषयानिरपेक्षाणां सुखादिज्ञानानामिन्द्रियमनोयोगविशेषस्य निया\edlabel{pvv.275-5}\footnote{\label{pvv.275-5}  ५ न क्रमनियामकं मन एकत्वात् सकृदात्मसंयोगात् ।}मकस्याभावात् {\color{DodgerBlue3}“सकृद्”} वा {\color{DodgerBlue3}“जन्म”} स्या त् । (२२५)
	\pend
      \label{div_pvv.2.526}\edlabel{div_pvv.2.526}
	  
	% new div opening: depth here is 2
	
	  \bigskip
	  \begingroup
	  \large
	
	    
	    \stanza[\smallbreak]
	\label{pv.2.526}\edlabel{pv.2.526}\flagstanza{\tiny\textenglish{....2.526}}एकैव चैत्क्रियैकः स्यात् किन्दीपोऽनेकदर्शनः ।&क्रमेणापि न शक्तं स्यात्पश्चादप्यविशेषतः ॥ ५२६ ॥\&[\smallbreak]


	
	  \endgroup
	

	  \pstart न हि रूपादिबुद्धीनामेव सुखादिबुद्धीनामिन्द्रियमनोयोगविशेषात् प्रतिनियमः शक्यो वक्तुं एकस्मान्मन\edlabel{pvv.275-6}\footnote{\label{pvv.275-6}  ६ आत्मानं कर्त्तारमपेक्ष्य कारणभूतात् ।}स {\color{DodgerBlue3}“एकैव”} सुखादिबुद्धिलक्षणा {\color{DodgerBlue3}“क्रिया”} जायते न च द्वे इति {\color{DodgerBlue3}“चेत्”} । यद्येवं {\color{DodgerBlue3}“किं”} कस्मा{\color{DodgerBlue3}“द्दीप”} एकोऽ{\color{DodgerBlue3}“नेकदर्शनो”}ऽनेकद्रष्टृज्ञानजनकः । द्रष्टुर\leavevmode\marginnote{\textenglish{276/s}} नेकत्वादेको\edlabel{pvv.276-1}\footnote{\label{pvv.276-1}  १ मनसः करणानां न स्वातन्त्र्यं कर्तृवशवृत्तेः कर्त्रोरैक्यै (इ) न्द्रियैक्यमनेकत्वे तु एककरणेनाप्यनेका क्रिया परस्य ।}प्यनेकज्ञानजनक इति चेत् । ज्ञेयस्यानेकत्वात्तद्ग्राहकानेक\edlabel{pvv.276-2}\footnote{\label{pvv.276-2}  २ पूर्व्वानुवाद (:) । न तर्हि मनो नियामकं ज्ञानानामात्मैव सवासनो नियामकः स्यात् ।}ज्ञान जन\edlabel{pvv.276-3}\footnote{\label{pvv.276-3}  ३ युगपत्}कोपि स्यादिति समानं । किञ्च (।) मनो नित्यमेकदापि सर्व्वज्ञानजननशक्तं न वा । शक्तञ्चेत् सर्व्वज्ञानानि सकृत् कुर्यात् । अशक्तञ्चेत् । {\color{DodgerBlue3}“क्रमेणापि”} तज्जनने {\color{DodgerBlue3}“न शक्तं”} स्यात् । पूर्व्वावस्थितस्याशक्तस्य रूपस्य {\color{DodgerBlue3}“पश्चादप्यविशेषतो विशेषाभावात्\edlabel{pvv.276-4}\footnote{\label{pvv.276-4}  ४ न सहकार्यपेक्षाप्यनाधेयातिशयस्य ।}”} । (५२६)
	\pend
      \label{div_pvv.2.527}\edlabel{div_pvv.2.527}
	  
	% new div opening: depth here is 2
	
	  \bigskip
	  \begingroup
	  \large
	
	    
	    \stanza[\smallbreak]
	\label{pv.2.527}\edlabel{pv.2.527}\flagstanza{\tiny\textenglish{....2.527}}अनेन देहपुरुषावुक्तौ संस्कारतो यदि ।&नियमः स कुतः पश्चात् बुद्धेश्चेदस्तु सम्मतम् ॥ ५२७ ॥\&[\smallbreak]


	
	  \endgroup
	

	  \pstart {\color{DodgerBlue3}“अनेन”} मनसः शक्तत्वाशक्तत्वविकल्पाद्दोषद्वयेन {\color{DodgerBlue3}“देहपुरुषौ”} शरीरात्माना{\color{DodgerBlue3}“वुक्तावु”}क्तोत्तरौ । तयोरपि तथा\edlabel{pvv.276-5}\footnote{\label{pvv.276-5}  ५ एकदेहादिकज्ञानं वैभाष्यस्य पुद्गलनामा पुरुषोपि । स सांख्यस्यात्मा इष्टः ।}विकल्पे तथादोषसद्भावात् ज्ञानजातज्ञानहेतोः {\color{DodgerBlue3}“संस्कारतो”} बुद्धीनामसकृद्-{\color{DodgerBlue3}“भावनियमो यदीष्ट”}स्तदा संस्कारस्य {\color{DodgerBlue3}“कुतः पश्चा\edlabel{pvv.276-6}\footnote{\label{pvv.276-6}  ६ नित्यत्वे सकृद्भाव एव अनित्यत्वे तु प्रश्नः ।} दु”}त्पन्नो बुद्धीर्नियमयेत् । बुद्धेः पूर्व्विकाया उपपद्यत इति चेत् (।) अस्तु पूर्व्वस्या {\color{DodgerBlue3}“बुद्धे”}र्व्वासनासंज्ञकस्य संस्कारः स बुद्ध्यात्मनो जन्म {\color{DodgerBlue3}“सम्मत”}मस्माकं (।) तथा च यदुक्तं “न हि धीः प्राग् धिया विने”\href{http://http://sarit.indology.info/?cref=pv.2.521}{(२।५२१)}ति तदेव प्रसाधितं स्यात् । ततश्च बुद्धेर्बुद्ध्यन्तरजन्मनः सामर्थ्यादुत्तरया धिया पूर्व्वबुद्धिर्गृह्येतेति न स्याद्विषयान्तरसञ्चारोस्याः । (५२७)
	\pend
      \label{div_pvv.2.528}\edlabel{div_pvv.2.528}
	  
	% new div opening: depth here is 2
	

	  \pstart स्यादेतत् । बुद्धेरुपादानतामात्रं बुद्ध्यन्तरं प्रति न तद्ग्राह्यता ततो विषयान्तरसञ्चारो भवेदित्याह (।)
	\pend
      
	  \bigskip
	  \begingroup
	  \large
	
	    
	    \stanza[\smallbreak]
	\label{pv.2.528a}\edlabel{pv.2.528a}\flagstanza{\tiny\textenglish{...2.528a}}न ग्राह्यतान्या जननाज्जननं ग्राह्यलक्षणम् ।&अग्राह्यं न हि तेजोस्ति;\&[\smallbreak]


	
	  \endgroup
	

	  \pstart {\color{DodgerBlue3}“न”} बुद्ध्यन्तर{\color{DodgerBlue3}“जननादन्या”} तद्ग्राह्यता । किन्तर्हि जननमेव {\color{DodgerBlue3}“ग्राह्यस्य लक्षणं”} । तच्चेदस्ति कुतो ग्राह्यताया अभावः (।) {\color{DodgerBlue3}“न हि तेजो”} ज्योतिर्बुद्धेर्जनकतयाऽ\leavevmode\marginnote{\textenglish{277/s}} {\color{DodgerBlue3}“ग्राह्यमस्ति”} । तस्मा\edlabel{pvv.277-1}\footnote{\label{pvv.277-1}  १ आलोको यदि न ग्राह्यो न जनयेदेव बुद्धिं ।}दिदमेव बुद्धिं प्रति ग्राह्यत्वं ग्राह्यस्य यत्तत् ज्ञानं नाम ॥
	\pend
      \leavevmode\marginnote{\textenglish{54b/MA}}

	  \pstart ननु च तेजसो बुद्धिजनकस्य सूक्ष्मो व्यवहितः सन्निहिततरश्च कश्चिदवयवो यथा न गृह्यते तथा जनिकापि बुद्धिर्न गृह्येतेत्याह\edlabel{pvv.277-2}\footnote{\label{pvv.277-2}  २ आलोकेप्यग्राह्यत्वं जनकत्वेन दृष्टमेव । अभ्युपगम्याह । आलोकः सावयव इति स्यादप्यग्राह्यता न त्वमूर्त्ते ज्ञाने ।} ।
	\pend
      
	  \bigskip
	  \begingroup
	  \large
	
	    
	    \stanza[\smallbreak]
	\label{pv.2.528b}\edlabel{pv.2.528b}\flagstanza{\tiny\textenglish{...2.528b}}न च सौक्ष्म्याद्यनंशके ॥ ५२८ ॥\&[\smallbreak]


	
	  \endgroup
	

	  \pstart {\color{DodgerBlue3}“अनंशके”} निरवयवे ज्ञाने {\color{DodgerBlue3}“सौक्ष्म्यं”} सूक्ष्मत्वं नास्ति । आदिशब्दात् स्वसन्तानवर्तित्वाव्यवधानात्यासत्त्यादयश्चानुपल (म्भ) हेतवो न सन्ति । (५२८)
	\pend
      \label{div_pvv.2.529}\edlabel{div_pvv.2.529}
	  
	% new div opening: depth here is 2
	

	  \pstart स्यादेतत् (।) ज्ञानस्य द्वे शक्ती ज्ञानजननशक्तिर्ग्राह्यताशक्तिश्चतत्र क्रमेण ग्राह्यताशक्तिहानौ जननशक्तिमात्रमवतिष्ठत इत्याह (।)
	\pend
      
	  \bigskip
	  \begingroup
	  \large
	
	    
	    \stanza[\smallbreak]
	\label{pv.2.529}\edlabel{pv.2.529}\flagstanza{\tiny\textenglish{....2.529}}ग्राह्यताशक्तिहानिः स्यात् नान्यस्य जननात्मनः ।&ग्राह्यताया न खल्वन्यज्जननं ग्राह्यलक्षणे ॥ ५२९ ॥\&[\smallbreak]


	
	  \endgroup
	

	  \pstart {\color{DodgerBlue3}“अन्यस्य”} ज्ञानकार्यभूत{\color{DodgerBlue3}“जननात्मनो”} जनकस्वभावस्य पूर्व्वज्ञानस्य {\color{DodgerBlue3}“ग्राह्यताशक्तिहानिर्न स्यात्”} । जननस्य ग्राह्यलक्षणत्वात्तस्य च सत्त्वात् । तथाहि {\color{DodgerBlue3}“ग्राह्यलक्षणे”}ऽक्षवि\edlabel{pvv.277-3}\footnote{\label{pvv.277-3}  ३ ग्राहके रूपज्ञानादौ सति ।}द्युपलभ्यमाने\edlabel{pvv.277-4}\footnote{\label{pvv.277-4}  ४ व्यवधानाभावात् स्वभावविशेषसंमुखीभावात् कारणान्तरसाकल्याच्च ।} {\color{DodgerBlue3}“न खलु ग्राह्यताया अन्यत्”} साक्षा{\color{DodgerBlue3}“ज्ज्जननं”} । यदेवोत्तरोत्तरज्ञानस्य साक्षाज्जननं तदेव तद्ग्राह्यत्वं (।) ततो जननसत्त्वे नास्ति ग्राह्यत्वहानिः । (५२९)
	\pend
      \label{div_pvv.2.530}\edlabel{div_pvv.2.530}
	  
	% new div opening: depth here is 2
	
	  \bigskip
	  \begingroup
	  \large
	
	    
	    \stanza[\smallbreak]
	\label{pv.2.530}\edlabel{pv.2.530}\flagstanza{\tiny\textenglish{....2.530}}साक्षान्न ह्यन्यथा बुद्धे रूपादिरुपकारकः ।&ग्राह्यतालक्षणादन्यस्तद्भावनियमोस्य कः ॥ ५३० ॥\&[\smallbreak]


	
	  \endgroup
	

	  \pstart न {\color{DodgerBlue3}“ह्यन्यथा साक्षा”}ज्जननादन्येन प्रकारेण {\color{DodgerBlue3}“रूपादिर्दृ”}श्यमानो {\color{DodgerBlue3}“बुद्धे”}र्ग्राहिकाया {\color{DodgerBlue3}“उपकारकः”} । परस्परोपकारिणोपि ग्राह्यत्वे रूपकारणग्रहणमपि स्यात् । अनुपकारकस्य ग्राह्यत्वे सर्व्वग्रहणप्रसङ्गः । तस्मा{\color{DodgerBlue3}“दस्य”} रूपादे{\color{DodgerBlue3}“र्ग्राह्यतालक्षणाद्दे”}शाद्यवि\edlabel{pvv.277-5}\footnote{\label{pvv.277-5}  ५ न बुद्धिमात्रं परबुद्धिसत्त्वात् । नोपादानमात्रमिन्द्रियत्वात् परमते । तस्माद् द्वयं ।}प्रकर्षिणः साक्षाज्जनकत्वादन्यः {\color{DodgerBlue3}“कस्तद्भावनियमो”} ग्राह्यत्वनियम: । (५३०)
	\pend
      \label{div_pvv.2.531}\edlabel{div_pvv.2.531}
	  
	% new div opening: depth here is 2
	
	  \bigskip
	  \begingroup
	  \large
	
	    
	    \stanza[\smallbreak]
	\label{pv.2.531}\edlabel{pv.2.531}\flagstanza{\tiny\textenglish{....2.531}}बुद्धेरपि तदस्तीति सापि सत्त्वे व्यवस्थिता ।&ग्राह्युोपादानसंवित्ती चेतसो ग्राह्यलक्षणम् ॥ ५३१ ॥\&[\smallbreak]


	
	  \endgroup
	\leavevmode\marginnote{\textenglish{278/s}}

	  \pstart {\color{DodgerBlue3}“तत्साक्षा”}ज्जनकत्वलक्षणं ग्राह्यत्वं {\color{DodgerBlue3}“बुद्धेरप्यस्तीति सा”}प्युत्तरबुद्धिजनिका {\color{DodgerBlue3}“तथैव”} तद्ग्राह्यत्वे {\color{DodgerBlue3}“व्यवस्थिता”} रूपादिषु कारणत्वेपि देशाद्यविप्रकर्षश्चक्षुराद्युपयोगञ्चापेक्ष्य तद्ग्राह्यतेष्यते । {\color{DodgerBlue3}“चेत”}सस्तु देशादिविप्रकर्षाभावाच्चश्रुराद्युपयोगाभावाच्च ग्राहिणो ज्ञानस्योपादानमुपादानकारणता सम्वित्तिरनुभवात्मता ते {\color{DodgerBlue3}“ग्राह्यलक्षणं”} । स्वसम्वेदनस्वभावतायां सत्यामुपादानकारणतोत्तरबुद्धिग्राह्यतेत्यर्थः । (५३१)
	\pend
      \label{div_pvv.2.532}\edlabel{div_pvv.2.532}
	  
	% new div opening: depth here is 2
	

	  \begin{center}%% label @type='head'
	\textbf{(५) क. योगिनां ज्ञानम्}
	\end{center}
	

	  \pstart ननु सूक्ष्मव्यवहितासन्नरूपाद्यनुपादानञ्च ज्ञानञ्चेन्न ग्राह्यं तदा कथं यो\edlabel{pvv.278-1}\footnote{\label{pvv.278-1}  १ यदि सौक्ष्म्यादेरग्राह्यं रूपादिरचित्तञ्चानुपादानभूतत्त्वात्तदा योगिनां तदुभयं न ग्राह्यं स्यात् (।) भवति च तद् व्यभिचारि लक्षणं ।}गिनां तेषां ग्रहणमित्याह (।)्
	\pend
      
	  \bigskip
	  \begingroup
	  \large
	
	    
	    \stanza[\smallbreak]
	\label{pv.2.532}\edlabel{pv.2.532}\flagstanza{\tiny\textenglish{....2.532}}रूपादेश्चेतसश्चैवमविशुद्धधियं प्रति ।&ग्राह्यलक्षणचिन्तेयमचिन्त्या योगिनां गतिः ॥ ५३२ ॥\&[\smallbreak]


	
	  \endgroup
	

	  \pstart रूपादेरसूक्ष्मस्याव्यवधानादिविशिष्टस्य बुद्धिकारणत्वं । {\color{DodgerBlue3}“चेतसश्चानुभ”}वात्मत्वे सत्युपादानता ग्राह्यत्वमिति । एवमनेन प्रकारेण {\color{DodgerBlue3}“ग्राह्यलक्षणचिन्तेयं”} प्रक्रान्ता {\color{DodgerBlue3}“अशुद्धधियं”} वासनोपप्लुतबुद्धिमदर्व्वाग्दर्शिनं जनं प्रति न तु विशिष्टबुद्धीन् योगिनः प्रति (।) यस्मा\edlabel{pvv.278-2}\footnote{\label{pvv.278-2}  २ समाधिबलेनैकजन्मकुशलेनापि देवेष्वष्टगुणैश्वर्यादिः किं पुनश्चिरेण वीतावरणानां ।} {\color{DodgerBlue3}“द्योगिनां”} सूक्ष्मव्यवहितपरचित्तदिगतिरचिन्त्या । भावसत्तामात्रं योगिज्ञानापेक्ष्यं ग्राह्यलक्षणमित्यर्थः । (५३२)
	\pend
      \label{div_pvv.2.533}\edlabel{div_pvv.2.533}
	  
	% new div opening: depth here is 2
	
	  \bigskip
	  \begingroup
	  \large
	
	    
	    \stanza[\smallbreak]
	\label{pv.2.533}\edlabel{pv.2.533}\flagstanza{\tiny\textenglish{....2.533}}तत्र सूक्ष्मादिभावेन ग्राह्यमग्राह्यतां व्रजेत् ।&रूपादि बुद्धेः कि जातं पश्चाद् यत् प्राङ् न विद्यते ॥ ५३३ ॥\&[\smallbreak]


	
	  \endgroup
	

	  \pstart {\color{DodgerBlue3}“तत्रै”}तादृशे ग्राह्यलक्षणद्वये स्थिते सति रूपं स्थूलमव्यवधानादिभावे सति {\color{DodgerBlue3}“ग्राह्यं”} सत्पश्चात् {\color{DodgerBlue3}“सूक्ष्मादिभावेनाग्राह्यतां व्र”}जेदिति युज्यत एवैतत् ।\edlabel{pvv.278-3}\footnote{\label{pvv.278-3}  ३ रूपमग्राह्यं स्यादपि बुद्धेस्त्वेतत् सर्व्वं नास्तीति कथमग्राह्यता ।} बुद्धेरेकदा ग्राह्यायाः {\color{DodgerBlue3}“पश्चाद”}ग्रहणकाले {\color{DodgerBlue3}“किं”} सूक्ष्मत्वाद्यग्राह्यताकारणं {\color{DodgerBlue3}“जातं यत् प्रागु”}पलम्भकाले {\color{DodgerBlue3}“न विद्यते”} । न हि स्वसन्तानवर्तितनश्चेसो निरवयवस्य सर्व्वदा किञ्चि\leavevmode\marginnote{\textenglish{279/s}} दुपलम्भकारणमस्ति । तादुशस्य यद्यनुपलम्भः कदाचिन्नोपलभ्येत । उपलभ्यते चेह कदाचित्सर्व्वदोपलम्भप्रसङ्गः ॥ (५३३)
	\pend
      \label{div_pvv.2.534}\edlabel{div_pvv.2.534}
	  
	% new div opening: depth here is 2
	
	  \bigskip
	  \begingroup
	  \large
	
	    
	    \stanza[\smallbreak]
	\label{pv.2.534}\edlabel{pv.2.534}\flagstanza{\tiny\textenglish{....2.534}}सति स्वधीग्रहे तस्मात्सैवानन्तरहेतुता ।&चेतसो ग्राह्यता सैव ततो नार्थान्तरे गतिः ॥ ५३४ ॥\&[\smallbreak]


	
	  \endgroup
	

	  \pstart {\color{DodgerBlue3}“तस्मा”}च्चित्तान्तरेण {\color{DodgerBlue3}“स्वधीग्रहे सति चेतसो”} यैवान{\color{DodgerBlue3}“न्तरहेतुता”} बुद्ध्यन्तरं प्रति {\color{DodgerBlue3}“सैव ग्राह्यता”} सा च नापैति {\color{DodgerBlue3}“तत”} उत्तरोत्तरबुद्धेः पूर्व्वपूर्व्वबुद्धिग्रहणमेव व्यापार\leavevmode\marginnote{\textenglish{55a/MA}} इत्यर्था{\color{DodgerBlue3}“न्तरे”} न स्या{\color{DodgerBlue3}“द् ग\edlabel{pvv.279-1}\footnote{\label{pvv.279-1}  १ न विषयान्तसञ्चार इत्युपसंहारः ।}तिः”} ॥ (५३४)
	\pend
      \label{div_pvv.2.535}\edlabel{div_pvv.2.535}
	  
	% new div opening: depth here is 2
	

	  \begin{center}%% label @type='head'
	\textbf{ख. ग्राह्यताशक्तिहानादपि न जननशक्तिः}
	\end{center}
	

	  \pstart ननु भावा अनेकशक्तियो\edlabel{pvv.279-2}\footnote{\label{pvv.279-2}  २ दर्शकवादी ।}गादनेककार्यकारिण एकशक्तियोगादेककार्यकारिणः तत् ज्ञानस्य ग्राह्यताश\edlabel{pvv.279-3}\footnote{\label{pvv.279-3}  ३ कर्मरूपा ।}क्तिरप्यन्या । अन्या च जनन\edlabel{pvv.279-4}\footnote{\label{pvv.279-4}  ४ कर्तृरूपा ।}शक्तिः । ततो ग्राह्यताशक्तिहानादपि जननशक्तिर्भविष्यतीत्याह (।)
	\pend
      
	  \bigskip
	  \begingroup
	  \large
	
	    
	    \stanza[\smallbreak]
	\label{pv.2.535}\edlabel{pv.2.535}\flagstanza{\tiny\textenglish{....2.535}}नानैकशक्त्यभावेपि भावो नानैककार्यकृत् ।&प्रकृत्यैवेति गदितं; नानैकस्मान्न चेद्भवेत् ॥ ५३५ ॥\&[\smallbreak]


	
	  \endgroup
	

	  \pstart {\color{DodgerBlue3}“नानैकशक्त्यभा\edlabel{pvv.279-5}\footnote{\label{pvv.279-5}  ५ भिन्ना शक्तिरयुक्ता ।}वेपि”} भावव्यतिरिक्ताऽनेकासां शक्तीनामेकस्याश्च शक्तेरभावेपि भावः स्वहेतोर्नानैककार्यकारणस्वभावतयोत्पन्नो {\color{DodgerBlue3}“नानैककार्यकृत् प्रकृत्यैव”} भवती{\color{DodgerBlue3}“ति गदि\edlabel{pvv.279-6}\footnote{\label{pvv.279-6}  ६ प्राक् ।} तं”} ।
	\pend
      

	  \pstart ननु नानैकं कार्यमेकस्मात्कारणान्न भवेत्सिद्धा\edlabel{pvv.279-7}\footnote{\label{pvv.279-7}  ७ स्वयूथ्याः ।}न्तविरोधादिति चेत् । (५३५)
	\pend
      \label{div_pvv.2.536}\edlabel{div_pvv.2.536}
	  
	% new div opening: depth here is 2
	

	  \begin{center}%% label @type='head'
	\textbf{(६) हेतुसामग्र्‏याः सर्वसम्भवः}
	\end{center}
	
	  \bigskip
	  \begingroup
	  \large
	
	    
	    \stanza[\smallbreak]
	\label{pv.2.536a}\edlabel{pv.2.536a}\flagstanza{\tiny\textenglish{...2.536a}}न किञ्चिदेकमेकस्मात्सामग्रयाः सर्व्वसम्भवः ।\&[\smallbreak]


	
	  \endgroup
	

	  \pstart सत्त्यमेतन्न किञ्चि{\color{DodgerBlue3}“त्कार्य”}मेकस्मात्कारणाज्जायते किन्तु सामग्र्‏या जन्मानेकोपादानसहकारिभावसञ्चयात्मिकायाः सर्व्वस्यैकस्यानेकस्य च कार्यस्य सम्भवः ।
	\pend
      

	  \pstart कथन्तर्ह्येकमनेकस्यं\edlabel{pvv.279-8}\footnote{\label{pvv.279-8}  ८ तथानेककृदेकोपि तद्भावपरिदीपन इत्यत्र ।} हेतुरुच्यत इत्याह (।)
	\pend
      
	  \bigskip
	  \begingroup
	  \large
	
	    
	    \stanza[\smallbreak]
	\label{pv.2.536b}\edlabel{pv.2.536b}\flagstanza{\tiny\textenglish{...2.536b}}एकं स्यादपि सामग्र्योरित्युक्तं तदनेककृत् ॥ ५३६ ॥\&[\smallbreak]


	
	  \endgroup
	\leavevmode\marginnote{\textenglish{280/s}}

	  \pstart चक्षुरूपालोकमनस्कारादिषु जायमानरूपक्षणेन्द्रियज्ञानकार्यद्वयापेक्षयाऽवान्तरभिन्तसामग्र्‏योरेकस्य रूपस्योपादानसहकारिभावेनोपयोगात् एकस्मादप्यनेकं कार्यं जायत इति तत एकमनेककार्यकृदित्युच्यते (।) न त्वेकस्मादसहायादनेकं कार्यं जायते इत्यभिप्रायात् (५३६)
	\pend
      \label{div_pvv.2.537_2.538}\edlabel{div_pvv.2.537_2.538}
	  
	% new div opening: depth here is 2
	

	  \pstart स्यादेतत् (।)
	\pend
      
	  \bigskip
	  \begingroup
	  \large
	
	    
	    \stanza[\smallbreak]
	\label{pv.2.537a}\edlabel{pv.2.537a}\flagstanza{\tiny\textenglish{...2.537a}}अर्थं पूर्व्वञ्च विज्ञानं गृह्णीयाद् यदि धीः परा ।\&[\smallbreak]


	
	  \endgroup
	

	  \pstart {\color{DodgerBlue3}“धी”}र्जायमाना {\color{DodgerBlue3}“पूर्व्वञ्च विज्ञानमर्थञ्च”} तत्कालसन्निपतितं यदि {\color{DodgerBlue3}“गृह्णीयात्”} तदा विषयान्तरसञ्चारः सम्भवेत् । अत्राह (।)
	\pend
      
	  \bigskip
	  \begingroup
	  \large
	
	    
	    \stanza[\smallbreak]
	\label{pv.2.537b}\edlabel{pv.2.537b}\flagstanza{\tiny\textenglish{...2.537b}}अभिलापद्वयं नित्यं स्याद् दृष्टक्रममक्रमम् ॥ ५३७ ॥\&[\smallbreak]


	
	  \endgroup
	

	  \pstart पूर्व्वगृहीतस्य वर्ण्णस्य चिन्तादावादिशब्दात् प्रत्यक्षेप्येकस्मिन् {\color{DodgerBlue3}“चेतसि”} दृष्टः क्रमो यस्य तत् दृष्टक्रम{\color{DodgerBlue3}“मभिलापद्वयं”} वर्ण्णद्वयं {\color{DodgerBlue3}“नित्यम”}क्रमं {\color{DodgerBlue3}“स्यात्”} युगपत्प्रतीयेतेत्यर्थः । (५३७) किं कारणमित्याह (।) {\color{DodgerBlue3}“पूर्व्वापरार्थभासित्वा\edlabel{pvv.280-1}\footnote{\label{pvv.280-1}  १ प्रसङ्गसाधनमिदं यदि पूर्व्वापरावर्थावेकत्र भासेते इत्यर्थः ।}त् चिन्तादि”}ज्ञानेन हि पूर्व्ववर्ण्णग्राहकज्ञानं गृह्येत इति तत्प्रतिभासी वर्ण्णो गृह्येत । तत्का\edlabel{pvv.280-2}\footnote{\label{pvv.280-2}  २ अवर्ण्णचिन्तने अग्राहिज्ञानग्राहकं यदुत्तरमिवर्ण्णग्राहकं तत्रेकारश्च पूर्व्वज्ञानारूढाकारश्च भासत इत्यभिलापद्वयमेकचित्तं स्यात् । न चास्ति ।}लोपनिपतितश्चान्यो वर्ण्ण इति क्रमोपलभ्ययोर्युगपद् ग्रहणप्रसङ्गः ।
	\pend
      
	  \bigskip
	  \begingroup
	  \large
	
	    
	    \stanza[\smallbreak]
	\label{pv.2.538}\edlabel{pv.2.538}\flagstanza{\tiny\textenglish{....2.538}}द्विर्द्विरेकं च भासेत भासनादात्मतद्धियोः ॥ ५३८ ॥\&[\smallbreak]


	
	  \endgroup
	

	  \pstart एकञ्च वर्ण्णादि{\color{DodgerBlue3}“र्द्विर्द्विर्भासेत”} । आ\edlabel{pvv.280-3}\footnote{\label{pvv.280-3}  ३ विषयरूपमात्मा, तदिति तदालम्वनं ज्ञानं तयोः । एकदा ह्यकारः स्वग्राहिज्ञाने भाति पुनरिकारज्ञाने स्वज्ञानारूढः इत्येकवर्ण्णचिन्तैव न स्यात् ।}त्मतद्धियोर्भासनात् । तद्दर्शने तत् ज्ञानदर्शने तु तदृष्टमिति द्विधा दर्शनप्रसङ्गः\edlabel{pvv.280-4}\footnote{\label{pvv.280-4}  ४ स्वयं सिद्धसाधनं परिहृत्याचार्यीयं समन्वयते परिहारं ।}। (५३८)
	\pend
      \label{div_pvv.2.539}\edlabel{div_pvv.2.539}
	  
	% new div opening: depth here is 2
	

	  \begin{center}%% label @type='head'
	\textbf{(७) आत्मानुभूतं प्रत्यक्षम्}
	\end{center}
	
	  \bigskip
	  \begingroup
	  \large
	
	    
	    \stanza[\smallbreak]
	\label{pv.2.539}\edlabel{pv.2.539}\flagstanza{\tiny\textenglish{....2.539}}विषयान्तरसञ्चारे यद्यन्त्यन्नानुभूयते ।&परानुभूतवत्सर्व्वाननुभूतिः प्रसज्यते ॥ ५३९ ॥\&[\smallbreak]


	
	  \endgroup
	

	  \pstart अथ विषयज्ञाने वृत्ते तज्ज्ञाने चोत्पन्ने विषयदर्शनस्य निष्पन्नत्वात् \leavevmode\marginnote{\textenglish{281/s}} ज्ञानान्तरं विषयान्तरग्राहकमिति स्या{\color{DodgerBlue3}“द्विषया\edlabel{pvv.281-1}\footnote{\label{pvv.281-1}  १ ज्ञानज्ञानादन्त्यवर्तिनो ज्ञानान्तरेणाननुभूताद्विषयान्तरसञ्चारस्तस्य स्ववेदनमवश्याभ्युपेयं । तस्याननुभवे प्राक्तनमगृहीतमिति सर्व्वलोपः ।}न्तरसञ्चार”} इत्यभिमते {\color{DodgerBlue3}“यद्यन्त्यं”} ज्ञानं ज्ञानानुभवितृ {\color{DodgerBlue3}“नानुभूयते”} तदा {\color{DodgerBlue3}“सर्व्व”}स्यार्थस्य तत् ज्ञानस्य {\color{DodgerBlue3}“चाननुभूतिः प्रसज्यते”} परानुभूतवत् । पुरुषान्तरानुभूतार्थतज्ज्ञानयोरिव (५३९) ।
	\pend
      \label{div_pvv.2.540}\edlabel{div_pvv.2.540}
	  
	% new div opening: depth here is 2
	

	  \pstart ननु आत्मनाऽनुभूतमर्थं-तत्-ज्ञानं-प्रत्यक्षं न परैरनुभूतमिति यद्युच्यते ।
	\pend
      
	  \bigskip
	  \begingroup
	  \large
	
	    
	    \stanza[\smallbreak]
	\label{pv.2.540}\edlabel{pv.2.540}\flagstanza{\tiny\textenglish{....2.540}}आत्मानुभूतं प्रत्यक्षं नानुभूतं परैः यदि ।&आत्मानुभूतिः सा सिद्धा कुतो येनैवमुच्यते ॥ ५४० ॥\&[\smallbreak]


	
	  \endgroup
	

	  \pstart नन्वन्त्यज्ञानाननुभवेऽर्थज्ञानस्याननुभवात् । तदनुभवाभावे {\color{DodgerBlue3}“चार्थानुभवासिद्धेरात्मानुभूतिः सा कुतः सिद्धा । येनैवमुच्यते । आत्मानुभूतं प्रत्य\edlabel{pvv.281-2}\footnote{\label{pvv.281-2}  २ अनुभवसिद्धौ ह्यात्मपरविभागः ।}क्षं”} न परानु-\leavevmode\marginnote{\textenglish{55b/MA}} भूतमिति । (५४०)
	\pend
      \label{div_pvv.2.541}\edlabel{div_pvv.2.541}
	  
	% new div opening: depth here is 2
	

	  \pstart ननु चक्षुरादावनर्थभूते चक्षुरादिना रूपाद्यनुभूतमिति यथा तथा ज्ञानाननुभवेप्यर्थो ज्ञात इति भविष्यतीत्याह (।)
	\pend
      
	  \bigskip
	  \begingroup
	  \large
	
	    
	    \stanza[\smallbreak]
	\label{pv.2.541}\edlabel{pv.2.541}\flagstanza{\tiny\textenglish{....2.541}}व्यक्तिहेत्वप्रसिद्धिः स्यात् न व्यक्तेर्व्यक्तमिच्छतः ।&व्यक्त्यसिद्धावपि व्यक्तं यदि व्यक्तमिदं जगत् ॥ ५४१ ॥\&[\smallbreak]


	
	  \endgroup
	

	  \pstart अर्थव्य{\color{DodgerBlue3}“क्तिहेतो”}श्चक्षुरादेरर्थदर्शनेप्य{\color{DodgerBlue3}“प्रसिद्धिर”}व्यक्तिः स्यात्\edlabel{pvv.281-3}\footnote{\label{pvv.281-3}  ३ अज्ञातस्यापि बीजस्याङकुरजननदृष्टेः ।} यतो न कारणदर्शनपूर्व्वकं कार्यदर्शनं । {\color{DodgerBlue3}“न तु व्यक्ते”}रूपलब्धेर्व्य{\color{DodgerBlue3}“क्तमर्थमिच्छतो”} व्यक्त्यसिद्धिर्युक्ता । {\color{DodgerBlue3}“यदि”} पुन{\color{DodgerBlue3}“र्व्यक्तेरसिद्धावपि व्यक्तं”} वस्तूच्यते तदा सर्व्व{\color{DodgerBlue3}“मिदं जगद् व्यक्तं्”}\edlabel{pvv.281-4}\footnote{\label{pvv.281-4}  ४ सर्व्वे सर्व्वज्ञाः स्युः ।} स्यात् । अव्यक्तव्यक्तिकत्वेन विशेषाभावात्\edlabel{pvv.281-5}\footnote{\label{pvv.281-5}  ५ न च भवति तस्मात् स्ववेदनमेष्टव्यम् ।}तथा ह्यर्थो न सत्तामात्रेण प्रतीत उच्यते । प्रतिपत्तिसंयोगात् (।) नापि ज्ञानसत्तामात्रेण परचित्तज्ञानेनापि संवेदनप्रसङ्गात् किन्तु ज्ञानानुभवेनार्थप्रतीतिर्व्वाच्या । न चान्येन सम्वेदनं ज्ञानस्योपपपद्यत इति स्वप्रकाशमेव स्वभावतस्तद्वक्तव्यमिति स्वसम्वेदनसिद्धिरिति । (५४१)
	\pend
      
	    
	    \pstart
	    \begin{center}
	  आचार्यश्री म नो र थ न न्दि कृतायां वार्त्तिकवृत्तौ प्रत्यक्षपरिच्छेदो द्वितीयः ॥
	    \end{center}
	    \pend
	  
	  \leavevmode\marginnote{\textenglish{282/s}}
	    
	    \endnumbering% ending numbering from div
	    \endgroup
	    
	  
	  
	% new div opening: depth here is 0
	
	    
	    \begingroup
	    \beginnumbering% beginning numbering from div depth=0
	    
	  
\chapter[{तृतीयः परिच्छेदः: स्वार्थानुमानं}]{तृतीयः परिच्छेदः: स्वार्थानुमानं}
	  
	% new div opening: depth here is 1
	

	  \pstart प्रत्यक्षमाख्यायावसरप्राप्तमनु\edlabel{pvv.282-1}\footnote{\label{pvv.282-1}  १ निरोधमार्गप्राप्तिदुःखसमुदयपरिहारार्थत्वादस्यैव ।}मानमिदानीम्वक्तव्यं (।) तच्च द्विविधं (।) स्वार्थं परार्थञ्च । तत्र स्वार्थमिदानीं वक्तव्यं एतत्पूर्वकत्वात् परार्थस्य ।
	\pend
      
	  
	% new div opening: depth here is 1
	
\section[{१---हेतुचिन्ता}]{१---हेतुचिन्ता}

	  \begin{center}%% label @type='head'
	\textbf{(१) हेतुलक्षणम्}
	\end{center}
	\label{div_pvv.3.1}\edlabel{div_pvv.3.1}
	  
	% new div opening: depth here is 2
	

	  \pstart तल्लिङ्गे विप्रतिपत्तयः सन्तीति तासां निराकरणेन तद्व्यवस्थापनार्थमाह ।
	\pend
      
	  \bigskip
	  \begingroup
	  \large
	
	    
	    \stanza[\smallbreak]
	\label{pv.3.1a}\edlabel{pv.3.1a}\flagstanza{\tiny\textenglish{...v.3.1a}}पक्ष\edlabel{pvv.282-2}\footnote{\label{pvv.282-2}  २ श्लोकेत्र लिङ्गलक्षणं संख्या(त्रिधैव)नियमः (अविनाभावनियमात्) संख्यानियमकारणं विपक्षनिवृत्ति(हेत्वाभास)श्चोक्तो}धर्मस्तदंशेन व्याप्तो हेतुः ;\&[\smallbreak]


	
	  \endgroup
	

	  \pstart पक्षो धर्मिधर्मसमुदायोनुमेयः । तदेकदेशत्वादुपचारेण\edlabel{pvv.282-3}\footnote{\label{pvv.282-3}  ३ तदंशत्वेन सम्बन्ध उक्तः ।} धर्मी पक्ष उक्तः । तस्य धर्मः (।) अनेन पक्षधर्मत्वमुक्तं । धर्मिधर्मो हेतुरित्यनुमाने सर्व्वस्य धर्म्मिधर्म्मो हेतुः स्यादित्युपचाराश्रयणं । तथा चाक्षु\edlabel{pvv.282-4}\footnote{\label{pvv.282-4}  ४ यथा रूपमिति दृष्टान्तधर्मिधर्म एव हेतुः स्यादन्यथा ।}षत्वादिः शब्दे धर्मिणि न हेतुः । तदंशेन पक्षस्य धर्मेण\edlabel{pvv.282-5}\footnote{\label{pvv.282-5}  ५ धर्मिमात्रस्य पक्षत्वाद् धर्मी च वन्ह्यादिः साध्य एव तदंशो विवक्षावशाद् यदा तु समुदायः पक्षस्तदा तदंश एकदेश एव ।} सिसाधयिषितेन व्याप्तो हेतुर्बोद्धव्यः । द्विविधा चेयं व्याप्तिर्व्यापकव्याप्यधर्मतया । तत्र व्याप्ये सति व्यापकस्यावश्यम्भाव\edlabel{pvv.282-6}\footnote{\label{pvv.282-6}  ६ इति व्याप्यधर्म उक्तो व्याप्ये धर्मिणि व्याप्तिर्द्धर्मः ।}स्तस्य व्याप्तिः । व्याप्यस्य च व्यापक एव\edlabel{pvv.282-7}\footnote{\label{pvv.282-7}  ७ व्यापकधर्मोऽत्र व्याप्तिमत्वात् धर्मत्वं व्याप्तेः ।} सति भावो नाम तस्य व्याप्तिः । आभ्यां \leavevmode\marginnote{\textenglish{283/s}} यथाक्रममन्वयव्यतिरेकावुक्तौ । व्याप्यसद्‏भावे व्यापकस्य सत्वनियमस्यान्वयरूपत्वात् । व्यापकाभावे व्याप्याभावस्य च व्यतिरेकरूपत्वात् (।)
	\pend
      

	  \begin{center}%% label @type='head'
	\textbf{(२) हेतुस्त्रिधा एतेन त्रिरूपत्वं हेतोर्लक्षणमुक्तं ॥}
	\end{center}
	
	  \bigskip
	  \begingroup
	  \large
	
	    
	    \stanza[\smallbreak]
	\label{pv.3.1b}\edlabel{pv.3.1b}\flagstanza{\tiny\textenglish{...v.3.1b}}त्रिधैव सः ।\&[\smallbreak]


	
	  \endgroup
	

	  \pstart {\color{DodgerBlue3}“त्रिधैव”} त्रिप्रकार एव कार्यस्वभावानुपलम्भभेदेन {\color{DodgerBlue3}“स\edlabel{pvv.283-1}\footnote{\label{pvv.283-1}  १ पक्षधर्मत्त्वेन यथोक्तव्याप्त्या च युक्तः ।} हेतुः । यथाग्निरत्र”} धमात् । वृक्षव्यवहारयोग्योयं शिंशपात्वात् । नेह प्रदेशे घट उपलब्धिलक्षणप्राप्तस्यानुपलब्धेरिति संख्यानियम उक्तः ।
	\pend
      

	  \pstart कस्मात् पुनस्त्रिविध एव हेतुरित्याह ।
	\pend
      
	  \bigskip
	  \begingroup
	  \large
	
	    
	    \stanza[\smallbreak]
	\label{pv.3.1c}\edlabel{pv.3.1c}\flagstanza{\tiny\textenglish{...v.3.1c}}अविनाभावनियमात्;\&[\smallbreak]


	
	  \endgroup
	

	  \pstart {\color{DodgerBlue3}“अविनाभावस्य”} साध्याव्यभिचारित्वस्य त्रिविध एव हेतौ {\color{DodgerBlue3}“नियमात्”} नियतत्वात् । संयोग्यादिषु चाभावात् सत्येवाविनाभावे हेतुत्वं स च व्याप्त्या कथितः । अनेन संख्यानियमकारणमुक्तं ।
	\pend
      

	  \begin{center}%% label @type='head'
	\textbf{(३) हेत्वाभासाः}
	\end{center}
	

	  \pstart हेतव उक्ताः । के पुनर्हेत्वाभासा इत्याह (।)
	\pend
      
	  \bigskip
	  \begingroup
	  \large
	
	    
	    \stanza[\smallbreak]
	\label{pv.3.1d}\edlabel{pv.3.1d}\flagstanza{\tiny\textenglish{...v.3.1d}}हेत्वाभासास्ततोऽपरे ॥ १ ॥ \leavevmode\marginnote{\textenglish{56a/MA}}\&[\smallbreak]


	
	  \endgroup
	

	  \pstart हेतुवदाभासन्त इति {\color{DodgerBlue3}“हेत्वाभासा”} हेतुप्रतिरूपकाः (।) {\color{DodgerBlue3}“तत”}स्त्रिविधाद्धेतोरपरेऽन्ये संयोग्यादयः । अविनाभावे सति हेतुत्वं स च तादात्म्यात् तदुत्पत्तेश्च । ये तु तद्विकलास्तेऽविनाभावविरहात् हेत्वाभासा इत्यर्थः (। १)
	\pend
      \label{div_pvv.3.2ab}\edlabel{div_pvv.3.2ab}
	  
	% new div opening: depth here is 2
	

	  \pstart यदि\edlabel{pvv.283-2}\footnote{\label{pvv.283-2}  २ दिङ्नागोक्तं परं प्रति सर्व्वथा गम्यगमकत्वं कार्यहेतौ शङ्कते ।} तदुत्पत्त्या गम्यगमकभावस्तदा धूमत्वविशेषवत् सामान्यधर्माः पार्थिवत्वादयोपि गमकाः स्युः । तथाऽग्नेः सामान्यधर्मवच्चान्दनत्वादयोपि विशेषधर्म्मा गम्याः स्युः । सर्व्वथा कार्यकारणभावादित्याह ।
	\pend
      
	  \bigskip
	  \begingroup
	  \large
	
	    
	    \stanza[\smallbreak]
	\label{pv.3.2a}\edlabel{pv.3.2a}\flagstanza{\tiny\textenglish{...v.3.2a}}कार्यं स्वभावैर्यावद्भिरविनाभावि कार्यवत् ।\&[\smallbreak]


	
	  \endgroup
	\leavevmode\marginnote{\textenglish{284/s}}

	  \pstart {\color{DodgerBlue3}“कार्य स्वभावैर्याव”}द्भिर्व्विशिष्टैर्द्‏धूम\edlabel{pvv.284-1}\footnote{\label{pvv.284-1}  १ इत्थंभूतलक्षणा तृतीया । स्वगतैः सामान्यैः धर्मेः कार्य्यस्यास्तु विशेषधर्मा गमकाः स्वभावे कृतकत्वेन प्रमेयत्वं गमकं ।}वत्वादिभि{\color{DodgerBlue3}“रविनाभावि”} विना न भवतीति कारणे कारणविषयेऽवधारितं तैरेव विशिष्टैः स्वभावैर्हेतुर्न साधारणैः पार्थिवत्वादिभिः । अग्निमन्तरेणापि तेषां भावात् । तथा कारणेऽधिकरणे याव\edlabel{pvv.284-2}\footnote{\label{pvv.284-2}  २ कारणस्थैः ।}द्‏भिः स्वभावैरग्नित्वादिभिः सामान्यधर्मैरविनाभावि कार्य धूमादिनिश्चितं तेषां सामन्यधर्माणां हेतुर्न विशेषधर्माणां चान्दनत्वादीनां । तान्यन्तरेणापि धूमादेर्दर्शनात् । यदि धूमत्वविशेषितं पार्थिवत्वं हेतुः क्रियते तदेष्टमेव व्यभिचाराभावात् (।) यदि च कारण\edlabel{pvv.284-3}\footnote{\label{pvv.284-3}  ३ एवमङ्गेन जन्यजन(क)त्वं स्यादिष्यते च सर्व्वथा । आह ज्ञापकहेतुरयं ।}गतचान्दनत्वादिविशेषजनितो धूमस्य विशेषः शक्यो निश्चेतुं तदा चान्दनत्वादयो गम्या इष्यन्ते । न हि कार्यकारणभावः सत्तामात्रेण गभ्यगमकभावनिमित्तं किन्तु निश्चयापेक्षः । स च यावन्निश्चीयते तथागम्यगमकभावः ।
	\pend
      

	  \pstart {\color{DodgerBlue3}“भावोपि”} स्वभावोपि हेतुः स्वभावे साध्ये कीदृशे हेतोर्भावः केवलो भावमात्रं तदनुरोद्‏धुमनुवर्त्तितुं शीलमस्येति भावमात्रानुरोध तस्मिन् । यस्य सत्तामात्रेण यो धर्मोऽवश्यं भवति न हेत्वन्तरमपेक्षते । तस्मिन् साध्ये स्वभावाख्यो हेतुर्नान्यत्र यथा वस्त्रत्वं रागे(।) एवञ्च विधिसाधनत्वं कार्यस्वभावयोर्दर्शितं ।
	\pend
      
	  
	% new div opening: depth here is 1
	
\section[{२ अनुपलब्धिचिन्ता}]{२ अनुपलब्धिचिन्ता}\label{div_pvv.3.2cd}\edlabel{div_pvv.3.2cd}
	  
	% new div opening: depth here is 2
	

	  \begin{center}%% label @type='head'
	\textbf{(१) दृश्यानुपलब्धिफलम्}
	\end{center}
	

	  \pstart इदानीं तृतीयहेतोः प्रतिषेधफलत्वमाह ।
	\pend
      
	  \bigskip
	  \begingroup
	  \large
	
	    
	    \stanza[\smallbreak]
	\label{pv.3.2b}\edlabel{pv.3.2b}\flagstanza{\tiny\textenglish{...v.3.2b}}अप्रवृत्तिः प्रमाणानां\edlabel{pvv.284-4}\footnote{\label{pvv.284-4}  ४ प्रत्यक्षपृष्ठभाविना निश्चयेनाधूमव्यावृत्तिरूपावधारणेन धूमादिस्वलक्षणं भासमानं तार्ण्णादिविशेषानवधारणेन चानेकस्वलक्षणरूपं सामान्यलक्षणं लिङ्गं प्रत्यक्षविषयो व्यवस्थाप्येत लिङ्गि च । \par
व्यक्तिभेदेन मानबहुत्वात् आगमापेक्षया वा आगमस्यापि निवृत्तिर्नाभावं साधयतीति वक्ष्यते ।};\&[\smallbreak]


	
	  \endgroup
	\leavevmode\marginnote{\textenglish{285/s}}

	  \pstart प्रमाणनिवृत्तिरूपाऽनुपलब्धिरसति सद्व्यवहारातिक्रान्तत्वादसदविशिष्टे देशकालस्वभावविप्रकृष्टे सु-मे-र्व्वा-दौ (।) सज्ज्ञान\edlabel{pvv.285-1}\footnote{\label{pvv.285-1}  १ उपलब्धिः कर्मधर्मश्चेद्वस्तुयोग्यता । कर्तृधर्मश्चेज्ज्ञानं । उपलि(? लब्धि): सत्वमेव । असताञ्चानुपलब्धिरसत्वं । मुख्यं सत्वमनेन दृश्यानुपलब्ध्याऽक्षेपः । अदृश्यानुपलब्धिः । दृश्यानुपलब्धौ वस्तुवशान्निवृत्तं सत्वं स्वनिमित्तं ज्ञानादि निवर्त्तयति । अदृश्येप्यभीष्टकार्याकरणादसत्कल्पे प्रतिपत्त्यध्यवसायान्निवृत्तं सत्वं स्वनिबंधनं निवर्तयति (।)}शब्दव्यवहाराणां प्रवृत्तिप्रतिषेधोऽप्रवृत्तिस्तत्फला (।) एतच्चाप्रवृत्तिफलत्वं दृश्यादृश्यानुपलब्ध्योः साधारणंज्ञानपूर्वकत्वात्\edlabel{pvv.285-2}\footnote{\label{pvv.285-2}  २ उपलब्धिः कारणं सद्व्यवहारस्य । तदभावे कार्याभाव उक्तः ।} (।) सत् ज्ञानशब्दव्यवहारस्य ज्ञानाभावे तदभावस्य न्यायप्राप्तत्वात् ।
	\pend
      

	  \pstart दृश्यानुपलब्धेः फलान्तरमाह । काचित् प्रमाणनिवृत्तिरसत्‏ज्ञानमभावज्ञानं तत्फला । कथमित्याह ।
	\pend
      \leavevmode\marginnote{\textenglish{56b/MA}}
	  \bigskip
	  \begingroup
	  \large
	
	    
	    \stanza[\smallbreak]
	\label{pv.3.2c}\edlabel{pv.3.2c}\flagstanza{\tiny\textenglish{...v.3.2c}}हेतुभेदव्यपेक्षया ॥ २ ॥\&[\smallbreak]


	
	  \endgroup
	

	  \pstart {\color{DodgerBlue3}“हेतु”}रनुपलम्भस्तस्य {\color{DodgerBlue3}“भेदो”} विशेषणमुपलब्धिलक्षणप्राप्तविषयत्वं तस्य {\color{DodgerBlue3}“व्यपेक्षया”} कांक्षया उपलब्धिलक्षणप्राप्तानुपलब्धिरित्यर्थः ॥
	\pend
      

	  \pstart ननु\edlabel{pvv.285-3}\footnote{\label{pvv.285-3}  ३ अभावप्रमाणस्वीकारे यद् दूषणं तदिहातिदिश्य परिह(र)ति ।} यद्यनुपलब्ध्याऽभावः साध्यते तदोपलब्ध्यभावोपि तत एवेत्यन\edlabel{pvv.285-4}\footnote{\label{pvv.285-4}  ४ यस्य विषयस्याभावः साध्यते तदुपलब्धेरप्यभावोन्ययानुपलब्ध्या तस्या अप्यन्ययेति ।}वस्थानादभावाप्रतिपत्तिः स्यात् । दृष्टान्तश्च न स्यादन्यस्याभावसाधनस्याभावादनुपलब्धेश्चान\edlabel{pvv.285-5}\footnote{\label{pvv.285-5}  ५ येनैव पक्षधर्मेण साध्यधर्मिण्यभावः साध्यते । तेनैव दृष्टान्तधर्मिण्यपि । तत्राप्यपर(ो) दृष्टान्त इति ।}वस्थाप्राप्तत्वात् । अथ पदार्थान्तरोपलब्धिरेवानुपलब्धिस्तयाऽध्यक्षसिद्ध्याऽभावः साध्यते तस्मान्न दोषः । यद्येवमर्थान्तरादपि किन्नाभावः साध्यते (।) तदपि ह्यनुपलब्धिरूपं प्रत्यक्षसिद्धञ्च । को वान्योपलम्भाभावेन सम्बन्धः ।
	\pend
      

	  \pstart उच्यते । एकज्ञानसंसर्ग्गि वस्त्वन्तरं तदुपलब्धिश्चानुपलब्धिर्व्विवक्षितोपलब्धेरन्यत्वादभक्षास्पर्शनीयवत् । स एवाभावः तदतिरिक्तस्य विग्रहवतोऽभावस्याभावात् । तस्मात्तादात्म्यमभावानुपलम्भयोः सम्बन्धः ।
	\pend
      

	  \pstart एवं तर्ह्यनुपलब्धिसिद्धिरेवाभावसिद्धिः किं साध्यते ॥
	\pend
      \leavevmode\marginnote{\textenglish{286/s}}

	  \pstart सत्यं (।) नाभावः साध्यः सिद्धत्वादस्य । किन्तर्हि (।) विषयोपदर्शनेन विषयी व्यवहारः साध्यते । यथा गोव्यवहारविषयोऽयं सास्नादिसमुदाया\edlabel{pvv.286-1}\footnote{\label{pvv.286-1}  १ सिद्धमेतदस्यैव गोत्वात् ।}त्मकत्वात् । (२)
	\pend
      \label{div_pvv.3.3}\edlabel{div_pvv.3.3}
	  
	% new div opening: depth here is 2
	

	  \begin{center}%% label @type='head'
	\textbf{(२) अनुपलब्धिश्चतुर्विधा}
	\end{center}
	

	  \pstart सा च प्रयोगभेदाद{\color{DodgerBlue3}“नुपलब्धिश्चतुर्व्विधा”} कथमित्याह ।
	\pend
      
	  \bigskip
	  \begingroup
	  \large
	
	    
	    \stanza[\smallbreak]
	\label{pv.3.3}\edlabel{pv.3.3}\flagstanza{\tiny\textenglish{...pv.3.3}}विरुद्धकार्ययोः सिद्धिर/?/ हेतु/?/भायोः ।&दृश्यात्मनोरभावार्थानुपलब्धिश्चतुर्व्विधा ॥ ३ ॥\&[\smallbreak]


	
	  \endgroup
	

	  \pstart विरुद्धञ्च कार्यञ्च विरुद्धकार्ये कार्यञ्च । प्रत्यासत्तेर्व्विरुद्धस्यैव बोद्धव्यं । {\color{DodgerBlue3}“तयोर्दृश्यात्मनोर्हेतुभावयोः”} कारणस्वभावयोश्च दृश्यात्मनोर्यथाक्रमं सिद्धिरुपलब्धिरसिद्धिरनुपलब्धिश्च ।
	\pend
      

	  \pstart {\color{DodgerBlue3}“विरुद्धो”}पलब्धिर्यथा नात्र शीतस्पर्शोऽग्नेः । व्याप्यव्यापकयोर्वस्तुतस्तादात्म्यात् । \edlabel{pvv.286-2}\footnote{\label{pvv.286-2}  २ कारणाभावस्यैव ख्यापनात् ।} {\color{DodgerBlue3}“व्यापक”}विरुद्धोपलब्धिरप्यनेनैवोक्ता भवति । यथा नात्र तु\edlabel{pvv.286-3}\footnote{\label{pvv.286-3}  ३ अस्य व्यापकं शीतं ।}षारस्परर्शोऽग्नेः । {\color{DodgerBlue3}“विरु\edlabel{pvv.286-4}\footnote{\label{pvv.286-4}  ४ एकप्रकारैवेयं ।}द्धकार्यो”}पलब्धिर्यथा नात्र शीतस्पर्शो धूमात् । कारणा\edlabel{pvv.286-5}\footnote{\label{pvv.286-5}  ५ प्रभेदोस्याः कारणविरुद्धोपलब्धिः कारणविरुद्धकार्य्योपलब्धिश्च प्रतिषेध्यकारणानुपलम्भसाधनात् । वह्निर्विरुद्धं शीतं निवर्त्तयन् तत्कार्यं निवर्त्तयति (।)}नुपलब्धिर्यथा नात्र {\color{DodgerBlue3}“धूमो”}ऽनग्नेः । {\color{DodgerBlue3}“स्वभावा”}नुपलब्धिर्यथा नात्र धूमोनुपलब्धेः । अनेन {\color{DodgerBlue3}“व्यापकानु”}पलब्धिरप्युक्ता\edlabel{pvv.286-6}\footnote{\label{pvv.286-6}  ६ स्वभावाभावस्यैव प्रतिपादनात्} यथा नात्र शिंशपा वृक्षाभावात् ।
	\pend
      

	  \pstart चतुर्व्विधाप्यनुपलब्धिरभावार्था प्रतिषेधफला (।) तत्र स्वभावानुपलब्धिः स्वयमेव प्रतिषेध्याभावरूपतया सिद्धाऽभावव्यवहारसाधनी । इतरास्तु निषेध्याभावाव्यभिचारिण्यःतदभावमभावव्यवहारञ्च साधयन्ति । तस्यासिद्धत्वात् । (३)
	\pend
      \label{div_pvv.3.4}\edlabel{div_pvv.3.4}
	  
	% new div opening: depth here is 2
	

	  \pstart यदि विरुद्धकार्योपलब्ध्याऽभावसिद्धिस्तदा विरुद्धकारणोपलब्ध्यापि किं न साध्यत इत्याह ।
	\pend
      
	  \bigskip
	  \begingroup
	  \large
	
	    
	    \stanza[\smallbreak]
	\label{pv.3.4}\edlabel{pv.3.4}\flagstanza{\tiny\textenglish{...pv.3.4}}तद्विरुद्धनिमित्तस्य योपलब्धिः प्रयुज्यते ।&निमित्तयोर्विरुद्धत्वाभावो हि व्यभिचारवान् ॥ ४ ॥\&[\smallbreak]


	
	  \endgroup
	\leavevmode\marginnote{\textenglish{287/s}}

	  \pstart {\color{DodgerBlue3}“तद्विरुद्धनिमित्तस्य”} निषेध्य(=शीतस्पर्शवह्नि)विरुद्धकारण(=काष्ठ)स्य {\color{DodgerBlue3}“योपलब्धिः प्रयुज्यते”} (।) यथा न शीतस्पर्शोत्र काष्ठादिति सा निषेध्यस्य विरुद्धस्य च ये निमित्ते तयो\edlabel{pvv.287-1}\footnote{\label{pvv.287-1}  १ काष्ठतुषारयोः}र्व्विरुद्धत्वाभावे व्यभिचारिणी अनै\edlabel{pvv.287-2}\footnote{\label{pvv.287-2}  २ यद्यप्यग्निजनकान्त्यकाष्ठेन शीतनिमित्तस्य विरोधस्तथापि तस्य कार्यदर्शनादेव निश्चय इति कार्यविरोध एव ।}कान्तिकी । विरोधे तु निमित्तयोरिष्यत एव\edlabel{pvv.287-3}\footnote{\label{pvv.287-3}  ३ कारणविरुद्धोपलब्धिः ।} (।) यथा नास्य\edlabel{pvv.287-4}\footnote{\label{pvv.287-4}  ४ दन्तवीणादि ।} रोमहर्षादि\edlabel{pvv.287-5}\footnote{\label{pvv.287-5}  ५ शीतकार्या न पिशाचादिकृताः ।}विशेषाः\edlabel{pvv.287-6}\footnote{\label{pvv.287-6}  ६ रोमहर्षापनयक्षम ।} सन्ति सन्निहितदहनविशेष\edlabel{pvv.287-7}\footnote{\label{pvv.287-7}  ७ संताप ।}त्वात् (।) रोमहर्षतदभावयोर्व्विरोधः । तत्कारणयोश्च शीत\edlabel{pvv.287-8}\footnote{\label{pvv.287-8}  ८ रोमाञ्चतापनिमित्तयोः}-\leavevmode\marginnote{\textenglish{57a/MA}} दहनयोरिति युक्तेयमनुपलब्धिः । तथा निषेध्य(=रोमहर्ष)विरुद्ध(=ताप)कारण(=दहन)काय(=धूर्मो/?/)पलब्धिर्यथा । {\color{DodgerBlue3}“न रोम\edlabel{pvv.287-9}\footnote{\label{pvv.287-9}  ९ कारणविरुद्धकार्योपलब्धिरियं ।}हर्षयुक्तपुरुषवानयं”} प्रदेशो\edlabel{pvv.287-10}\footnote{\label{pvv.287-10}  १० धूमादित्यस्य पक्षधर्मत्वदर्शनाय पुरुषे धर्मिणि न स्यात् । धूमस्य प्रदेशधर्मत्वात् । अत्र धूमोपलब्ध्याऽग्निविरुद्धशीतकार्यरोमहर्षाद्यभावः ।}धूमादित्युक्ताष्टविधानुपलब्धिः । (४)
	\pend
      \label{div_pvv.3.5}\edlabel{div_pvv.3.5}
	  
	% new div opening: depth here is 2
	

	  \pstart अनया दिशाऽपरा अपि बोद्धव्याः ।
	\pend
      
	  \bigskip
	  \begingroup
	  \large
	
	    
	    \stanza[\smallbreak]
	\label{pv.3.5}\edlabel{pv.3.5}\flagstanza{\tiny\textenglish{...pv.3.5}}इष्टं विरुद्धकार्येपि देशकालाद्यपेक्षणाम् ।&अन्यथा व्यभिचारी स्यात् भस्मेवाशीतसाधने ॥ ५ ॥\&[\smallbreak]


	
	  \endgroup
	

	  \pstart {\color{DodgerBlue3}“वि\edlabel{pvv.287-11}\footnote{\label{pvv.287-11}  ११ चिरविनष्टेऽप्यग्नौ धूमसंभवात् कथं न व्यभिचार इत्याह ।}रुपद्धकार्ये”} विरुद्धकार्योपलब्धावपि\edlabel{pvv.287-12}\footnote{\label{pvv.287-12}  १२ अपिशब्दात् कार्यहेतावपि (।)}विषये निर्देशात् (।) देशस्य सन्निहितस्य कालस्य वर्त्तमानस्य क्वचिदतीतापेक्षण{\color{DodgerBlue3}“मिष्टं”}\edlabel{pvv.287-13}\footnote{\label{pvv.287-13}  १३ आदिशब्दादवस्थाविशेषापेक्षा ।}(।) अन्यथा व्यभिचारि स्याल्लिङ्गं (।) उदाहरणमाह (।) {\color{DodgerBlue3}“भस्मेवाशीतस्य”} शीताभावस्य {\color{DodgerBlue3}“साधने”} (।) यथा नासीत् क्वचित्\edlabel{pvv.287-14}\footnote{\label{pvv.287-14}  १४ देशानपेक्षा ।}शीतस्पर्शो, नास्ति वात्र भस्मन इति भस्मानैकान्तिकं। देशविशेष्येऽतीतकालापेक्षया तु स्याद्धेतुर्नासीदत्र शीतस्पर्शो भस्मन इति ॥ (५)
	\pend
      \label{div_pvv.3.6}\edlabel{div_pvv.3.6}
	  
	% new div opening: depth here is 2
	

	  \pstart ननु (।)
	\pend
      \leavevmode\marginnote{\textenglish{288/s}}
	  \bigskip
	  \begingroup
	  \large
	
	    
	    \stanza[\smallbreak]
	\label{pv.3.6}\edlabel{pv.3.6}\flagstanza{\tiny\textenglish{...pv.3.6}}हेतुना यः समग्रेण कार्योत्पादोनुमीयते ।&अर्थान्तरानपेक्षत्वात् स स्वभावोनुवर्ण्णितः ॥ ६ ॥\&[\smallbreak]


	
	  \endgroup
	

	  \pstart {\color{DodgerBlue3}“हेतुना समग्रेण यः कार्योत्पादोनुमीयते”} (।) स कस्मिन् हेताव\edlabel{pvv.288-1}\footnote{\label{pvv.288-1}  १ अप्रतिबद्धसामर्थ्याच्चेत् कार्यमुत्पन्नं दुश्येतैव । कारणकारणात्तु व्यभिचार इत्याह ।}न्तर्भवतीत्याह (।) {\color{DodgerBlue3}“स स्वभाव”}हेतुर{\color{DodgerBlue3}“नुवर्ण्णितः”} । न हि समग्राद्धेतोः कार्यसम्भवोनुमीयते (।) किन्तर्हि (।) कार्यार्जनयोग्यत्वं (।) तच्चार्थान्तरानपेक्षत्वाद्धेतुसाकल्यमात्रानुबन्ध्येवेति तस्मिन् साध्ये हेतुसाकल्यं स्वभावहेतुरेव । (६)
	\pend
      \label{div_pvv.3.7}\edlabel{div_pvv.3.7}
	  
	% new div opening: depth here is 2
	

	  \pstart कस्मात् पुनः कार्यमेव नानुमीयत इत्याह ।
	\pend
      
	  \bigskip
	  \begingroup
	  \large
	
	    
	    \stanza[\smallbreak]
	\label{pv.3.7}\edlabel{pv.3.7}\flagstanza{\tiny\textenglish{...pv.3.7}}सामग्रोफलशक्तीनां परिणामानुबन्धिनि ।&अनैकान्तिकता कार्ये प्रतिबन्धस्य सम्भवात् ॥ ७ ॥\&[\smallbreak]


	
	  \endgroup
	

	  \pstart कार्येऽनुमेयेऽनैकान्तिकता हेतुसाकल्यस्य । कीदृशे (।) {\color{DodgerBlue3}“सामग्र्‏याः”} फलञ्च ताः [उत्तरक्षणं] (।) {\color{DodgerBlue3}“शक्त”}यश्च तासां {\color{DodgerBlue3}“परिणामः”}\edlabel{pvv.288-2}\footnote{\label{pvv.288-2}  २ यथा तानवितानवतस्त(न्तू)न् व्यापारवत्पुंसाधिष्ठितान् दृष्ट्‏वा पटानुमानं । कारणाख्यं नानुपलब्धि(त्रिधैवेति इष्यते)र्विधिसाधनात् । कारणस्वभावत्वान्न कार्यं । अन्येनान्य(ानु)मानान्न स्वभावोपि हेतुरित्याह । नात्र कार्योत्पादानुमानं किन्तु समग्राणां कार्योत्पादनयोग्यता । स्वभावविशेषः कार्यमुत्पद्यतेऽस्मादिति श्लोकेप्यर्थः प्रयोगो या समग्रा सोत्तरपरिणामात् कार्योत्पादनयोग्या यथोत्पादितकार्यसामग्री (।)} कार्योत्पादानुगुणस्तारतम्ययोगी क्षणप्रबन्धः । {\color{DodgerBlue3}“तदनुबन्धिनि”} तदपेक्षिणि । अत एव प्रतिबन्धस्य सम्भवादित्युक्तं । (७)
	\pend
      \label{div_pvv.3.8}\edlabel{div_pvv.3.8}
	  
	% new div opening: depth here is 2
	

	  \pstart न खलु कार्यं हेतवो मिलिता इत्येव भवति (।) किं त्वेषां विशेषमपेक्षतेऽत्रान्तरे च शक्तिव्याघातो मन्त्रतन्त्रादिना वैकल्यञ्च सम्भवतीति कार्यस्यावश्यभावान्न तदनुमानं ।
	\pend
      
	  \bigskip
	  \begingroup
	  \large
	
	    
	    \stanza[\smallbreak]
	\label{pv.3.8}\edlabel{pv.3.8}\flagstanza{\tiny\textenglish{...pv.3.8}}एकसामग्र्यधीनस्य रूपादे रसतो गतिः ।&हेतुधर्मानुमानेन धूमेन्धनविकारवत् ॥ ८ ॥\&[\smallbreak]


	
	  \endgroup
	

	  \pstart या च {\color{DodgerBlue3}“रसतो”} मधुरादिकात् (।) {\color{DodgerBlue3}“रूपादे”}रादिशब्दाद् गन्धस्य स्पर्शस्य च {\color{DodgerBlue3}“एकसामग्र्‏यधीनस्य”} रसादिना \edlabel{pvv.288-3}\footnote{\label{pvv.288-3}  ३ पुनस्त्रिधैवेति त्रस्यति (।) अकार्यकारणभूते नानुमेयादन्येनान्धकारे मातुलुङ्गादिरसमास्वाद्य चम्पकगन्धमाघ्राय वह्निस्पर्शमनुभूय तेषां रूपसामान्यमनुमीयते । सा कथमनुमेयादन्येनानुमानान्न स्वभावहेतुः ।}सहैकसामग्रयायत्तस्य गतिः सा कथमित्याह(।) \leavevmode\marginnote{\textenglish{289/s}} {\color{DodgerBlue3}“हेतुधर्मानुमानेन”} रसकारणस्य {\color{DodgerBlue3}“धर्मो रसादिसहचररूपजनकत्वं (।)तदनुमानेन रसाद्”} रूपादिगतिः । न हि कार्यं रसः कारणमन्तरेण । कारणञ्चास्य\edlabel{pvv.289-1}\footnote{\label{pvv.289-1}  १ इति कार्यात् कारणविशेषानुमानं ।} रससहकारि रूपजनकं पुञ्जात् पुञ्जोत्पत्तेः । अतः तस्मिन्ननुमितेऽनुमितमेव रूपं {\color{DodgerBlue3}“धूमेन्धनविकार-”} वत् । धूमाद्धेतुधर्मानुमानेनेन्धनविकारस्याङ्गारादेर्द्धूमसहचरस्योवानुमानं । (८)
	\pend
      \label{div_pvv.3.9}\edlabel{div_pvv.3.9}
	  
	% new div opening: depth here is 2
	

	  \pstart एतदेव स्फुटयन्ना (।)
	\pend
      
	  \bigskip
	  \begingroup
	  \large
	
	    
	    \stanza[\smallbreak]
	\label{pv.3.9}\edlabel{pv.3.9}\flagstanza{\tiny\textenglish{...pv.3.9}}शक्तिप्रवृत्त्या न विना रसः सैवान्यकारणम् ।&इत्यतीतैककालानां जायते कार्यलिङ्गतः ॥ ९ ॥\&[\smallbreak]


	
	  \endgroup
	

	  \pstart रसहेतोः {\color{DodgerBlue3}“शक्तिप्रवृत्त्या”} शक्त्याभिमुख्येन {\color{DodgerBlue3}“विना न रस”} उत्पद्यते (।) सैव रसहेतोः शक्तिप्रवृत्तिरन्यस्य रूपादेः {\color{DodgerBlue3}“कारणं”} इतरसमानकालरूपादिजनकत्वेन {\color{DodgerBlue3}“रस”}हेतोर\edlabel{pvv.289-2}\footnote{\label{pvv.289-2}  २ रसोपादानसमानकालभाविनोतीताः । लिङ्गभूतरससहभाविन एककालाः । इति अनेन द्वारेण । नानागतगतिः(।)} {\color{DodgerBlue3}“तीतस्य”} हेतो{\color{DodgerBlue3}“रेककालानां”} रससहचराणां च रूपादीनां {\color{DodgerBlue3}“गतिः”} तस्य रसरूपहेतोः {\color{DodgerBlue3}“कार्याद्”} रसाल्लिङ्गाज्जाता ॥
	\pend
      

	  \pstart एवं पिपीलिकोत्सरणादि\edlabel{pvv.289-3}\footnote{\label{pvv.289-3}  ३ योग्यतानुमानं ।}दर्शनात् वर्ष(ा) पिपी\edlabel{pvv.289-4}\footnote{\label{pvv.289-4}  ४ मत्स्योत्पतनात् । गृहीताण्डस्यान्यत्र गतिः बलाकातस्तोयं (।)}लिकादिक्षोभहेतोर्भूतपरिणामविशेषस्यानुमानात् वर्षानुमानं कार्यलिङ्गजं वेदितव्यं । नदीपूरादेर्व्वर्षकार्यत्वं व्यक्तमेव ॥ (९)
	\pend
      \label{div_pvv.3.10}\edlabel{div_pvv.3.10}
	  
	% new div opening: depth here is 2
	
	  \bigskip
	  \begingroup
	  \large
	
	    
	    \stanza[\smallbreak]
	\label{pv.3.10}\edlabel{pv.3.10}\flagstanza{\tiny\textenglish{...v.3.10}}हेतुना योऽसमग्रेण कार्योत्पादोऽनुमीयते ।&तच्छेषवदसामर्थ्याद् देहाद् रागानुमानवत् ॥ १० ॥\&[\smallbreak]


	
	  \endgroup
	

	  \pstart {\color{DodgerBlue3}“यः”} पुन{\color{DodgerBlue3}“र्हेतुना”}ऽसमग्रेण कार्योत्पादः कार्योत्पादनयोग्यत्वमनुमीयते\edlabel{pvv.289-5}\footnote{\label{pvv.289-5}  ५ मीमांसाकादिभिः ।} तच्छेषवदनैकान्तिकमसामर्थ्यात् । समर्थं हि कार्योत्पादनयोग्यं न च व्यग्राणां समर्थता ।\leavevmode\marginnote{\textenglish{57b/MA}} देहात्\edlabel{pvv.289-6}\footnote{\label{pvv.289-6}  ६ कार्यं सर्वथाऽसम्बद्धं योग्यतापि नास्ति क्षणपरिणामेनापि रागादिमानयं देह इन्द्रियबुद्धिमत्वात् ।} बुद्धीन्द्रियादेश्च हेतो रागानुमानवत् । देहादीनां कथञ्चिद्\edlabel{pvv.289-7}\footnote{\label{pvv.289-7}  ७ अदेहादे रागादृष्टेः} रागहेतुत्वेपि\edlabel{pvv.289-8}\footnote{\label{pvv.289-8}  ८ अहं ममेत्यात्मात्मीयाभिनिवेशपूर्व्वकत्वात् (।)}\leavevmode\marginnote{\textenglish{290/s}} नायोनिशोमनस्कार\edlabel{pvv.290-1}\footnote{\label{pvv.290-1}  १ योनिशो मनो नैरात्म्यज्ञानं (।) तद्विरुद्धमात्मादिज्ञानं ।}रहितानां कारणत्वं सहितानामेव\edlabel{pvv.290-2}\footnote{\label{pvv.290-2}  २ रागे साध्ये तत्र हितोपलंबिपक्षस्तत्र देहाद्यदृष्टेरस्ति (।)}हेतुत्वात् । ततःकेवलाद् देहाद् रागानुमानमनैकान्तिकं । (१०)
	\pend
      \label{div_pvv.3.11}\edlabel{div_pvv.3.11}
	  
	% new div opening: depth here is 2
	

	  \pstart तथा (।)
	\pend
      
	  \bigskip
	  \begingroup
	  \large
	
	    
	    \stanza[\smallbreak]
	\label{pv.3.11}\edlabel{pv.3.11}\flagstanza{\tiny\textenglish{...v.3.11}}विपक्षेऽदृष्टिमात्रेण कार्यसामान्यदर्शनात् ।&हेतुज्ञानप्रमाणामं वचनाद् रागितादिवत् ॥ ११ ॥\&[\smallbreak]


	
	  \endgroup
	

	  \pstart {\color{DodgerBlue3}“विपक्षेऽदृष्टिमात्रेण कार्यसामान्यस्य\edlabel{pvv.290-3}\footnote{\label{pvv.290-3}  ३ वक्तुकामताजस्य राग्यरागिगतवचनस्य (।) वक्तुकामतैव रागश्चेदिष्टन्तत् । अभिष्वङ्गो नो रागः ।}”} क्वचिद् धर्मिणि {\color{DodgerBlue3}“दर्शनात् । हेतुज्ञानं”}\edlabel{pvv.290-4}\footnote{\label{pvv.290-4}  ४ नैवं करुणाधर्मालम्बनत्वात् । न वाग्विशेषात् सरागोपि वीतरागवदित्यनिश्चयात् ।} यत् {\color{DodgerBlue3}“प्रमाणाभं”} सन्दिग्धविपक्षव्यावृत्तिकं तत् {\color{DodgerBlue3}“वचनाद्”} रागितादिवत्\edlabel{pvv.290-5}\footnote{\label{pvv.290-5}  ५ द्वेषमोहादि ।} । यथा वचनस्य वीतरागेऽदर्शनात् क्वचिद् दर्शनाद् रागानुमानमनैकान्तिकं (।) न हि रागवचनयोः कार्यकारणभावः सिद्धः । वक्तुकामता\edlabel{pvv.290-6}\footnote{\label{pvv.290-6}  ६ अविशिष्टं विवक्षामात्रं ।}हेतुत्वात् तस्य । सा च वीतरागस्य करुणया\edlabel{pvv.290-7}\footnote{\label{pvv.290-7}  ७ नेयमभिष्वङ्गो धर्मालम्बनत्वात् ।}संभवति । यद्यपि वक्तरि रागो दृश्यते तथापि क्वचिद् वक्तुकामतासत्वेपि रागाभावे वचनाभावासिद्धेर्न तत्कार्यतासिद्धिरुपलखण्डाद् वक्तुकामतानिवृत्तेरेव वचनाभावो न तु रागाभावात् (।) ततो वीतरागात् सन्दिग्धोस्य व्यतिरेकः ॥ (११)
	\pend
      \label{div_pvv.3.12}\edlabel{div_pvv.3.12}
	  
	% new div opening: depth here is 2
	
	  \bigskip
	  \begingroup
	  \large
	
	    
	    \stanza[\smallbreak]
	\label{pv.3.12}\edlabel{pv.3.12}\flagstanza{\tiny\textenglish{...v.3.12}}न चादर्शनमात्रेण विपक्षेऽव्यभिचारिता ।&संभाव्य व्यभिचारित्वात् स्थालीतंडुलपाकवत् ॥ १२ ॥\&[\smallbreak]


	
	  \endgroup
	

	  \pstart {\color{DodgerBlue3}“न च विपक्षे”}\edlabel{pvv.290-8}\footnote{\label{pvv.290-8}  ८ विपक्षे दृष्ट्या व्यभिचारी न च विरागे वचनं दृष्टमित्याह । अनैकान्तिकस्यादर्शनमात्रेणानिरासादनैकान्तिकविपक्षेण यद्वचनं तेन वैशेषिकेण वायोः सत्वसाधनार्थं स्पर्शश्च न च दृष्टानामिति (वैशे॰ सू १० । अ॰ २ । आ॰ १)} हेतोरदर्शन{\color{DodgerBlue3}“मात्रेण”} साध्याभावप्रयुक्तसाधनाभावनिश्चयरहितेन साध्या{\color{DodgerBlue3}“व्यभिचारिता”} साध्यते सम्भाव्यव्यभिचा(ि)रत्वात् । यदि विपक्षाद्धेतुनिवृत्तिनिश्चय एवं व्यभिचारशङ्कानिरासः । स तु नास्तीति तस्याप्यभावः । \leavevmode\marginnote{\textenglish{291/s}} {\color{DodgerBlue3}“स्थालीतण्डुलपाकवत्”} । स्थाल्यन्तर्गतानां तण्डुलानां\edlabel{pvv.291-1}\footnote{\label{pvv.291-1}  १ बाहुल्येन पक्षदर्शनेपि पक्षसमहेतवः पक्षा इति युक्तं वैधर्म्ये ।}पाकस्येवानुमानं । तदितरेषां पाकादर्शनात् शेषवत् ॥ (१२)
	\pend
      \label{div_pvv.3.13}\edlabel{div_pvv.3.13}
	  
	% new div opening: depth here is 2
	

	  \begin{center}%% label @type='head'
	\textbf{(क) शेषवदनुमाननिरासः}
	\end{center}
	

	  \pstart किं पुनः शेषवदित्युच्यते\edlabel{pvv.291-2}\footnote{\label{pvv.291-2}  २ पूर्व्ववच्छेषवत्सामान्यतो दृष्टमिति (न्यायसू॰१।१।५) शेषवल्लक्षणं नैयायिकस्य विरुद्धं कारणात् कार्यानुमानं पूर्व्ववत् । शेषः कार्यं यस्यास्ति तच्छेषवत् (।) कार्यात् कारणानुमानं । अतः पृच्छति किमिति (।)} । इत्याह ।
	\pend
      
	  \bigskip
	  \begingroup
	  \large
	
	    
	    \stanza[\smallbreak]
	\label{pv.3.13}\edlabel{pv.3.13}\flagstanza{\tiny\textenglish{...v.3.13}}यस्यादर्शनमात्रेण व्यतिरेकः प्रदर्श्यते ।&तस्य संशयहेतुत्वाच्छेषवत्तदुदाहृतम् ॥ १३ ॥\&[\smallbreak]


	
	  \endgroup
	

	  \pstart {\color{DodgerBlue3}“यस्य”} लिङ्गस्या{\color{DodgerBlue3}“दर्शनमा”}त्रेण विपक्षे\edlabel{pvv.291-3}\footnote{\label{pvv.291-3}  ३ क्वचित् ।} {\color{DodgerBlue3}“व्यतिरेकः प्रदर्श्यते”} । न तु कारणव्यापकनिवृत्त्या तस्य संशयहेतुत्वात् साध्यानिश्चायकत्वात् शेषवत् तदुदाहृतं । सन्दिग्धविपक्षव्यावृत्तिकत्वं शेषवदुच्यते प्रतिबन्धाभावादित्यर्थः ।(१३)
	\pend
      \label{div_pvv.3.14}\edlabel{div_pvv.3.14}
	  
	% new div opening: depth here is 2
	

	  \begin{center}%% label @type='head'
	\textbf{(ख) त्रिषु हेतुरूपेषु निश्चयः}
	\end{center}
	
	  \bigskip
	  \begingroup
	  \large
	
	    
	    \stanza[\smallbreak]
	\label{pv.3.14}\edlabel{pv.3.14}\flagstanza{\tiny\textenglish{...v.3.14}}हेतोस्त्रिष्वपि रूपेषु निश्चयस्तेन वर्णितः ।&असिद्धविपरीतार्थव्यभिचारिविपक्षतः ॥ १४ ॥\&[\smallbreak]


	
	  \endgroup
	

	  \pstart सूत्रमुक्तं (।) अस्यायमर्थः (।) यो गुणः स द्रव्याश्रयी तद्यथा रूपादिः । अपाकजानुष्णाशीतस्पर्शश्य (? स्य) गुणः तस्मात् तस्याश्रयभूतेन द्रव्येण भवितव्यं । (।) न चायं दृष्टानां पृथिव्यादीनां गुणः (।) तेषां पाकजानुष्णाशीतस्पर्शादिगुणत्वात् (।) ततो यस्यायं गुणः स वायुर्भविष्यतीत्युक्ते वैशेषिकेण (।) तत्राचार्य दि ग्ना गेनोक्तं (।) यद्यदर्शनमात्रेण दृष्टेभ्यः (स्पर्शस्य) प्रतिषेधः क्रियते न च सोपि युक्त इति दृश्यादृश्यसमुदायस्य सामान्येन निषेधात् (।) यदेतदुक्तं तद्विरुध्यत इति वा र्ति क का रो दर्शयन्नाह दृष्टेत्यादि । यद्यदृष्ट्या निवृतिः स्यात् तदाऽदृष्टेन दर्शनात् कारणादपाकजस्यानुष्णाशीतस्पर्शस्य दृष्टाऽयुक्तिः (।) दृष्टेषु पृथिव्यादिष्वसङ्गतिर्यां वर्णिता वै शे षि कैः यस्या आचार्येणायुक्तत्वमुक्तं सा स्यादविरोधिनी युक्तैव स्यादित्यर्थः । वायुप्रकरणे पृथिव्यादिभ्यः यदि स्पर्शादेर्गुणत्वं सिद्धं स्यात् ततो वायुद्रव्यानुमानं स्यात् । सैव त्वसिद्धा स्वातन्त्र्येण प्रतीतेः स्पर्शाविशेषोस्माकं वायुः (।) आचार्येण तु स्पर्शव्यतिरिक्तं वायुमभ्युपगम्यायुक्तत्वमुक्तं परे ।
	\pend
      \leavevmode\marginnote{\textenglish{292/s}}

	  \pstart {\color{DodgerBlue3}“तेन”} प्रतिबन्धस्यावश्याभ्युपगन्तव्यत्वेन {\color{DodgerBlue3}“हेतोस्त्रिष्वपि”} पक्षधर्मा\edlabel{pvv.292-1}\footnote{\label{pvv.292-1}  १ प्रसिद्धस्तु द्वयोरपि साधनमित्यादिना वादिविवादिनोः ।}न्वयव्यतिरेकेषु {\color{DodgerBlue3}“निश्चय”} आचार्य {\color{DodgerBlue3}“दि ग्ना गे न”} वार्ण्णितः । कथमुक्त इत्याह । {\color{DodgerBlue3}“असिद्धस्य विपरीता”}र्थस्य {\color{DodgerBlue3}“विरुद्धस्य व्यभिचारिणो”}ऽनैकान्तिकस्य {\color{DodgerBlue3}“विपक्षेण”}\edlabel{pvv.292-2}\footnote{\label{pvv.292-2}  २ तृतीया आद्यदिग्धातु सिविपक्षत (?) इत्यत्र ।}। तत्रासिद्धत्वविपक्षेण पक्षधर्मत्वस्य निश्चय उक्तः । विरुद्धविपक्षेणान्वयस्य । अनैकान्तिकविपक्षेण व्यतिरेकस्य ॥ (१४)
	\pend
      \label{div_pvv.3.15}\edlabel{div_pvv.3.15}
	  
	% new div opening: depth here is 2
	

	  \begin{center}%% label @type='head'
	\textbf{(३) व्याप्तिचिन्ता}
	\end{center}
	

	  \begin{center}%% label @type='head'
	\textbf{(क) प्रतिबन्धो दिग्नागेष्टः}
	\end{center}
	

	  \pstart न ह्य सति प्रतिबन्धे विपक्षाद् व्यतिरेकः शक्यनिश्चयः । उक्तश्च निश्चयस्ततः प्रतिबन्धोप्याचार्येणेष्ट इति प्रतीयते । अन्यथा (।)
	\pend
      
	  \bigskip
	  \begingroup
	  \large
	
	    
	    \stanza[\smallbreak]
	\label{pv.3.15}\edlabel{pv.3.15}\flagstanza{\tiny\textenglish{...v.3.15}}व्यभिचारिविपक्षेण वैधर्म्यवचनं च यत् ।&यद्यदृष्टिफलं तच्च तदनुक्तेपि गम्यते ॥ १५ ॥\&[\smallbreak]


	
	  \endgroup
	

	  \pstart प्रतिबन्धानिष्टौ व्यभिचारिणोऽनैकान्तिकस्य विपक्षेण वैधर्म्यस्य व्यतिरेकस्य वचनं यदाचार्यस्य\edlabel{pvv.292-3}\footnote{\label{pvv.292-3}  ३ तद्व्यर्थमित्याकूतं} (।) एष तावन्न्यायः यदुभयं वक्तव्यं\edlabel{pvv.292-4}\footnote{\label{pvv.292-4}  ४ न्यायमुख आचार्येणोक्तं साधर्म्यं वैधर्म्यं चोभयं ।} विरुद्धानैकान्तिकप्रतिपक्षेणेत्युभयमन्वयव्यतिरेकौ । तच्च यद्यदृष्टिफलमदर्शनमात्रफलं तददर्शनमात्र{\color{DodgerBlue3}“मनुक्तेपि”}\edlabel{pvv.292-5}\footnote{\label{pvv.292-5}  ५ वैधर्म्यस्य ।} व्यभिचारिविपक्षेण व्यतिरेको गम्यते (।) हेतुत्रैरूप्यनिर्द्देशादेव विपक्षेऽदर्शनमात्रस्य गतत्वात् । तस्माद् {\color{DodgerBlue3}“वैधर्म्यवचनेन”}\edlabel{pvv.292-6}\footnote{\label{pvv.292-6}  ६ अनैकान्तिकस्यादर्शनमात्रेणानिरासादनैकान्तिकविपक्षेण यद्वचनं तेन ।}विपक्षे हेत्वभावः कथ्यते । स चादर्शनमात्रेण न सिध्यति ॥ (१५)
	\pend
      \label{div_pvv.3.16}\edlabel{div_pvv.3.16}
	  
	% new div opening: depth here is 2
	

	  \pstart नास्तीति वचनादेव सिध्यतीति चेत् ।
	\pend
      
	  \bigskip
	  \begingroup
	  \large
	
	    
	    \stanza[\smallbreak]
	\label{pv.3.16}\edlabel{pv.3.16}\flagstanza{\tiny\textenglish{...v.3.16}}न च नास्तीति वचनात् तन्नास्त्येव यथा यदि ।&नास्ति स ख्याप्यते चायमुक्तौ नेति गतिस्तदा ॥ १६ ॥\&[\smallbreak]


	
	  \endgroup
	\leavevmode\marginnote{\textenglish{293/s}}

	  \pstart {\color{DodgerBlue3}“न च नास्तीति वचनादेव”} तत्साधनं विपक्षे {\color{DodgerBlue3}“नास्त्येवेति”} युक्तं । तदपि ह्यनुपलम्भमेव ख्यापयति\edlabel{pvv.293-1}\footnote{\label{pvv.293-1}  १ विशिष्टकारणानुमानं ।}(।) स च सर्व्वस्माद्विपक्षाद्धेत्वभावप्रतीतावशक्तः ।\leavevmode\marginnote{\textenglish{58a/MA}} (कथन्तर्हीत्याह) यथा येन प्रकारेण साध्यसाधनयोः प्रतिबन्धो सति साध्याभावेन साधनं विपक्षे नास्ति स न्यायो यदि ख्याप्यते व्यभिचारिविपक्षेण वैधर्म्योक्त्या तदा नास्तीति गम्यते नान्यथा ॥ (१६)
	\pend
      \label{div_pvv.3.17}\edlabel{div_pvv.3.17}
	  
	% new div opening: depth here is 2
	

	  \pstart किञ्च (।)
	\pend
      
	  \bigskip
	  \begingroup
	  \large
	
	    
	    \stanza[\smallbreak]
	\label{pv.3.17}\edlabel{pv.3.17}\flagstanza{\tiny\textenglish{...v.3.17}}यद्यदृष्टौ निवृत्तिः स्याच्छेषवद् व्यभिचारि किम् ।&व्यतिरेक्यपि हेतुः स्यान्न वाच्याऽसिद्धियोजना ॥ १७ ॥\&[\smallbreak]


	
	  \endgroup
	

	  \pstart {\color{DodgerBlue3}“यद्यदृष्टौ”} विपक्षाद्धेतो{\color{DodgerBlue3}“र्निवृत्तिः”} स्यात् {\color{DodgerBlue3}“शेषवत्”} सन्दिग्धविपक्षव्यतिरेकं\edlabel{pvv.293-2}\footnote{\label{pvv.293-2}  २ उपभुक्तान्यफलानि पक्वानि मधुरादिरसानि वा । रक्तादिरूपाविशेषादुपभुक्तफलवत् । एकशाखाप्रभवत्वाद्वा ।} {\color{DodgerBlue3}“व्यभिचारि”} किमिष्टमा चा र्ये ण । एवञ्च सरसान्येतानि फलानि रूपाविशेषादिति । सर्व्वस्य तादृग्रूपस्य पक्षीकृतत्वात् न व्यभिचारदर्शनं विपक्षे चादर्शनमस्तीति प्राप्तमदर्शनमात्रद् व्यतिरेकवादिनो हेतुत्वमस्य ॥
	\pend
      

	  \pstart {\color{DodgerBlue3}“व्यतिरेक्यपि हेतुः स्यात्”} । नेदं निरात्मकं जीवच्छरीरमप्राणादिमत्वप्रसङ्गादि ति । बौ द्धं प्रति सात्मकस्य कस्यचिदसिद्धेरन्वयाभावात् । निरात्मकेभ्यश्च पटादिभ्यो निवृत्तेः प्राणादि\edlabel{pvv.293-3}\footnote{\label{pvv.293-3}  ३ अपानोन्मेषनिमेषादि ।}र्व्यतिरेकी स हेतुः स्यात् । अदर्शनाद् व्यतिरेकसिद्धेरनिष्ट\edlabel{pvv.293-4}\footnote{\label{pvv.293-4}  ४ घटादावपि नैरात्म्यासिद्धेर्बौद्धाभ्युपगमाच्चेत् आत्मा न स्यात् ।}श्चाचार्येण ॥
	\pend
      

	  \pstart किञ्च (।) वादिप्रतिवादिनोरसिद्धस्यान्यतरासिद्धस्य सन्दिग्धस्याश्रया\edlabel{pvv.293-5}\footnote{\label{pvv.293-5}  ५ आचार्येण ह्युक्तं पक्षधर्मों वादिप्रतिवादिनिश्चितो गृह्यते । तेनैषां निरासः ।} सिद्धस्य सन्दिग्धाश्रयासिद्धस्य असिद्धस्य च निरासं पक्षधर्मत्वनिश्चयेन प्रतिपाद्यांन्वयव्यतिरेकनिश्चयवचनेन सपक्षविपक्षयोर्हेतोरन्वयस्य व्यतिरेकस्य च विपरीताया असिद्धेश्चतुर्व्विधाया योजना आचार्येण सपक्षे सन्नसन्नित्येवमादिष्वपि यथायो\edlabel{pvv.293-6}\footnote{\label{pvv.293-6}  ६ अन्यतरासिद्धादीनां सपक्षादिष्वसम्भवात् यथायोगं ।}गमुदाहार्य्यमित्यादिना निर्दिष्टा (।) सापि न वाच्या । अदर्शनस्य व्यतिरेकनिश्चयहेतुत्वे सन्दिग्धव्यतिरेकस्य हेतुत्वात् ॥ (१७)
	\pend
      \label{div_pvv.3.18}\edlabel{div_pvv.3.18}
	  
	% new div opening: depth here is 2
	

	  \pstart अपि च (।)
	\pend
      \leavevmode\marginnote{\textenglish{294/s}}
	  \bigskip
	  \begingroup
	  \large
	
	    
	    \stanza[\smallbreak]
	\label{pv.3.18}\edlabel{pv.3.18}\flagstanza{\tiny\textenglish{...v.3.18}}विशेषस्य व्यवच्छेदहेतुता स्याददर्शनात् ।&प्रमाणान्तरवाधा चेन्नेदानीन्नास्ति-तादृशः ॥ १८ ॥\&[\smallbreak]


	
	  \endgroup
	

	  \pstart विशेषस्यासाधारणस्य\edlabel{pvv.294-1}\footnote{\label{pvv.294-1}  १ अदर्शनमात्राद् व्यावृत्तिरिष्टाऽस्ति च नित्यानित्ययोरदर्शनं ।} श्रावणत्वादेर्नित्येऽनित्ये चादर्शनादुभयतो व्यावृत्तेः शब्दे धर्मिणि द्वयव्यवच्छेदन\edlabel{pvv.294-2}\footnote{\label{pvv.294-2}  २ प्रतिषेधहेतुत्वं ।} हेतुता स्यात् । {\color{DodgerBlue3}“प्रमाणान्तरबाधा चेत्”}\edlabel{pvv.294-3}\footnote{\label{pvv.294-3}  ३ श्रावणत्वस्य कृतकत्ववद् वस्तुधर्मत्वान्नित्यादस्त्येव व्यतिरेको बाधकात् । परे तु नित्यमपि वस्त्विच्छन्ति तदपेक्षयोच्यते नित्यानित्यव्यावृत्तिः ।} अन्योन्यव्यवच्छेदरूपयोरेकप्रतिषेधस्यापरविधिनान्तरीयकत्वात् उभयव्यवच्छेदो\edlabel{pvv.294-4}\footnote{\label{pvv.294-4}  ४ वस्तु भवत्तत्वमन्यत्वम्वा नातिक्रामति (।)}नुमानबाधितः । {\color{DodgerBlue3}“नेदानीं”}\edlabel{pvv.294-5}\footnote{\label{pvv.294-5}  ५ यद् विरुद्धं न तस्यैकत्र युगपत्सम्भवो यथा शीतोष्णयोर्विरुद्धश्च नित्यानित्ययो र्युगपदेकत्र धर्मिणि व्यवच्छेदो यथोक्तविधानेनेति व्यापकविरुद्धं ।}(बाधसंभवे) नास्ति तादृशः । एवं तर्ह्यदृष्टेरभावनिश्चयो नास्तीति वक्तव्यं । यथा श्रावणत्वेनोभयव्यावृत्ततया निश्चितेन प्रसाध्यमानस्योभयव्यवच्छेदस्य प्रमा (णा) न्तरेण बाधा । (१८)
	\pend
      \label{div_pvv.3.19}\edlabel{div_pvv.3.19}
	  
	% new div opening: depth here is 2
	

	  \begin{center}%% label @type='head'
	\textbf{(ख) आचार्योयमतनिरासः}
	\end{center}
	
	  \bigskip
	  \begingroup
	  \large
	
	    
	    \stanza[\smallbreak]
	\label{pv.3.19}\edlabel{pv.3.19}\flagstanza{\tiny\textenglish{...v.3.19}}तथान्यत्रापि संभाव्यं प्रमाणान्तरबाधनम् ।&दृष्टाऽयुक्तिरदृष्टेश्च स्यात् स्पर्शस्याविरोधनी ॥ १९ ॥\&[\smallbreak]


	
	  \endgroup
	

	  \pstart तथान्यत्रापि विपक्षाद्धेतो\edlabel{pvv.294-6}\footnote{\label{pvv.294-6}  ६ लक्षणयुक्ते बाधासम्भवे तल्लक्षणमेव दूषितं स्यादिति सर्व्वत्रानाश्वासः । नैवं मानद्वये ।}र्व्यतिरिकेपि {\color{DodgerBlue3}“संभाव्यं प्रमाणान्तरबाधनं”} । अदर्शनमात्रस्य निमित्तस्य समानत्वात् । तथा दृष्टाऽयुक्तिरदृष्टेश्च\edlabel{pvv.294-7}\footnote{\label{pvv.294-7}  ७ अध्यक्षं किञ्चिद् दृष्ट्वा यत् किञ्चिद् पृथिव्यादि तत् सर्व्वमनुष्णाशीतरहितं तन्न तूलोपलपर्ण्णवादिवद् भेदसम्भवात् । अदर्शनमात्रे प्रत्यक्षवाधा ।} स्यात् स्पर्शस्याविरोधिनी । दृष्टेषु पृथिव्यादिष्वपाकजस्यानुष्णाशीतस्पर्शस्यायुक्ति\edlabel{pvv.294-8}\footnote{\label{pvv.294-8}  ८ आसङ्गतिः ।}र्योगाभावः (।) अदृष्टेरदर्शनात् वै शे षि क स्येष्टाऽनुपलम्भाद् व्याप्त्या निवृत्त्या निश्चयादा चार्येण प्रतिक्षिप्ता च । \edlabel{pvv.294-9}\footnote{\label{pvv.294-9}  ९ यद्यदृष्ट्या निवृत्तिस्तदाऽदृष्टेरदर्शनादनुष्णाशीतस्य स्पर्शंस्य दृष्टो युक्तिः दृष्टेषु पृथिव्यादिष्वसङ्गतिर्या वैशेषिकोक्ता । यस्याचार्येणायुक्तत्वमुक्तं स्यादविरोधिनी, युक्तैव ।} “यद्यदर्शनमात्रणे दृष्टेभ्यः\edlabel{pvv.294-10}\footnote{\label{pvv.294-10}  १० दृष्टमेव स्वीकृत्यादृष्टेपि ।}प्रतिषेधः क्रियते न च सोपि युक्त” \leavevmode\marginnote{\textenglish{295/s}} इत्यादिनाऽ{\color{DodgerBlue3}“विरोधीनो”} विरोधरहिता स्यात् । एवमा चा र्यी यः\edlabel{pvv.295-1}\footnote{\label{pvv.295-1}  १ शिष्योऽज्ञोऽदर्शनेनाभाववादी ।}कश्चन प्रतिबन्धमनभ्युपगच्छन् विरोधादुपालब्धः ॥ (१९)
	\pend
      \label{div_pvv.3.20}\edlabel{div_pvv.3.20}
	  
	% new div opening: depth here is 2
	

	  \pstart संप्रति परान् प्रत्याह (।)
	\pend
      
	  \bigskip
	  \begingroup
	  \large
	
	    
	    \stanza[\smallbreak]
	\label{pv.3.20}\edlabel{pv.3.20}\flagstanza{\tiny\textenglish{...v.3.20}}देशादिभेदाद् दृश्यन्ते भिन्ना द्रव्येषु शक्तयः ।&तत्रैकदृष्ट्या नान्यत्र युक्तस्तद्भावनिश्चयः ॥ २० ॥\&[\smallbreak]


	
	  \endgroup
	

	  \pstart {\color{DodgerBlue3}“देशादिभेदात्”} देशकालसहकारिभेदात् {\color{DodgerBlue3}“दृश्यन्ते भिन्ना”} नानाप्रकारा {\color{DodgerBlue3}“द्रव्येषु शक्तयः”}\edlabel{pvv.295-2}\footnote{\label{pvv.295-2}  २ अनेकशक्तिद्रव्येषु तत्र ।} (।) {\color{DodgerBlue3}“तत्रैक”}स्मिन् देशादावेकस्य द्रव्यस्य शक्तिविशेषवतो {\color{DodgerBlue3}“दृष्ट्याऽन्यत्र”} देशादौ\edlabel{pvv.295-3}\footnote{\label{pvv.295-3}  ३ हेतोर्व्विरोधादनुयुक्ते ।} {\color{DodgerBlue3}“तस्य”} शक्तिविशेषवतो द्रव्यस्य {\color{DodgerBlue3}“भावनिश्चयो न युक्तः”} प्रतिबन्धमन्तरेण । न ह्येकत्र यथौषधयो दृष्टास्तथैवान्यत्रापि ताः शक्यन्ते व्यवस्थापयितुं क्षेत्रादिविशेषाद् विशिष्टतररस\edlabel{pvv.295-4}\footnote{\label{pvv.295-4}  ४ वीर्यदोषापनयशक्तिः परिणामो विपाकः संस्कारः क्षीरसेकादि ।}वीर्यादिदर्शनात् । (२०)\leavevmode\marginnote{\textenglish{58b/MA}}
	\pend
      \label{div_pvv.3.21}\edlabel{div_pvv.3.21}
	  
	% new div opening: depth here is 2
	

	  \begin{center}%% label @type='head'
	\textbf{(क) वैशेषिकमतनिरासः}
	\end{center}
	

	  \pstart किञ्च ।
	\pend
      
	  \bigskip
	  \begingroup
	  \large
	
	    
	    \stanza[\smallbreak]
	\label{pv.3.21}\edlabel{pv.3.21}\flagstanza{\tiny\textenglish{...v.3.21}}आत्ममृच्चेतनादीनां योऽभावस्याप्रसाधकः ।&स एवानुपलम्भः कि हेत्वभावस्य साधकः ॥ २१ ॥\&[\smallbreak]


	
	  \endgroup
	

	  \pstart {\color{DodgerBlue3}“आत्ममृच्चेतनादीनां योऽभावस्याप्रसाधकः”} । आत्मनोनुपलभ्यमानस्य च योऽनुपलम्भोऽभावस्याप्रसाधको वै शे षि क स्ये ष्टः (।) तथा मुदश्चेतनायाः स्वभावभूताया योऽनुपलम्भोऽभावाप्रसाधकश्चा र्व्वा क स्येष्टः ।\edlabel{pvv.295-5}\footnote{\label{pvv.295-5}  ५ लोकायतिकस्य ।} \edlabel{pvv.295-6}\footnote{\label{pvv.295-6}  ६ आदिना ।}तथा क्षीरादौ दध्याद्यभावस्य योनुपलम्भोऽप्रसाधक इष्टः सां ख्य स्य । स एवानुपलम्भः किं कस्माद् विपक्षे हेत्वभावस्येष्टो वै शे षिका दिभिः (।) यथा घटादौ प्राणादिमत्वाभावो वैशेषिकस्य । वक्तृत्वाभावो विपक्षे\edlabel{pvv.295-7}\footnote{\label{pvv.295-7}  ७ असर्व्वज्ञाविरागयोः ।} चा र्व्वा क स्य । संघात\edlabel{pvv.295-8}\footnote{\label{pvv.295-8}  ८ अपरार्थेषु च शशविषाणादिषु (परार्थाश्चक्षुरादय इत्यभिधाय) संघातत्वस्यादर्शनाद् व्यतिरेकः को हि संहतस्य परार्थत्वे नियमः ।} त्वाभावो वा विपक्षे सां ख्य स्यानुपलम्भादिष्टः । \leavevmode\marginnote{\textenglish{296/s}} उक्तः परेषाञ्च न्यायव्याघातः\edlabel{pvv.296-1}\footnote{\label{pvv.296-1}  १ भिन्ना द्रव्येष्विति ।} परस्परविरोधश्च\edlabel{pvv.296-2}\footnote{\label{pvv.296-2}  २ आत्मन्यस्वीकृतिः ।}(।) (२१)
	\pend
      \label{div_pvv.3.22}\edlabel{div_pvv.3.22}
	  
	% new div opening: depth here is 2
	

	  \pstart यस्माददर्शनमात्रत् न व्यतिरेकसिद्धिः ।
	\pend
      
	  \bigskip
	  \begingroup
	  \large
	
	    
	    \stanza[\smallbreak]
	\label{pv.3.22}\edlabel{pv.3.22}\flagstanza{\tiny\textenglish{...v.3.22}}तस्मात् तन्मात्रसम्बन्धः स्वभावो भावमेव वा ।&निवर्तयेत् कारणं वा कार्यमव्यभिचारतः ॥ २२ ॥\&[\smallbreak]


	
	  \endgroup
	

	  \pstart {\color{DodgerBlue3}“तस्मात् तन्मात्रसम्बन्ध”}स्तत्साधनं केवलं तन्मात्रं तेन सम्बद्धः । हेतु\edlabel{pvv.296-3}\footnote{\label{pvv.296-3}  ३ ज्ञापको लिङ्गं ।}सत्तामात्रेण कारणान्तरापेक्षारहितेन सम्बन्धो यस्येत्यर्थः\edlabel{pvv.296-4}\footnote{\label{pvv.296-4}  ४ साध्यस्य ।} । {\color{DodgerBlue3}“स्वभावो”} व्यापको धर्मो निवर्त्तमानो भावं व्याप्यमेव वा निवर्त्तयेत् । यथा वृक्षत्वं शिंशपां वृक्षविशेषत्वाच्छिंशपायाः । {\color{DodgerBlue3}“कारणं वा”} निवर्त्तमानं {\color{DodgerBlue3}“कार्यं निवर्त्तयति अव्यभिचारतः”} (।) न हि व्याप्यं कार्यं च व्यापकेन कारणेन विना भवतः तत्स्वभावत्वात् तदधीनत्वाच्चेति । (२२)
	\pend
      \label{div_pvv.3.23}\edlabel{div_pvv.3.23}
	  
	% new div opening: depth here is 2
	

	  \pstart तादात्म्यतदुत्पत्ती अवश्याश्रयणीये ॥
	\pend
      
	  \bigskip
	  \begingroup
	  \large
	
	    
	    \stanza[\smallbreak]
	\label{pv.3.23}\edlabel{pv.3.23}\flagstanza{\tiny\textenglish{...v.3.23}}अन्यथैकनिवृत्यान्यविनिवृत्तिः कथं भवेत् ।&नाश्ववानिति संदेहे किं न वा पशुसंशयः ॥ २३ ॥\&[\smallbreak]


	
	  \endgroup
	

	  \pstart {\color{DodgerBlue3}“अन्यथैकस्य”} साध्यस्य {\color{DodgerBlue3}“निवृत्याऽन्यस्य”} साधनस्य {\color{DodgerBlue3}“विनिवृत्तिः कथम्भवेत्”} स्वतन्त्रत्वात् । न हि मर्त्योऽश्वरहित इति गोरहितोपि भवति । (२३)
	\pend
      \label{div_pvv.3.24}\edlabel{div_pvv.3.24}
	  
	% new div opening: depth here is 2
	

	  \pstart तथा (।)
	\pend
      
	  \bigskip
	  \begingroup
	  \large
	
	    
	    \stanza[\smallbreak]
	\label{pv.3.24}\edlabel{pv.3.24}\flagstanza{\tiny\textenglish{...v.3.24}}सन्निधानात् तथैकस्य कथमन्यस्य सन्निधिः ।&गोमानित्येव मर्त्येन भाव्यमश्ववतापि किम् ॥ २४ ॥\&[\smallbreak]


	
	  \endgroup
	

	  \pstart तथा {\color{DodgerBlue3}“सन्निधानादे”}कस्य हेतोः\edlabel{pvv.296-5}\footnote{\label{pvv.296-5}  ५ स्वभावेनासम्बद्धस्य ।} {\color{DodgerBlue3}“कथमन्यस्य”} साध्यस्य {\color{DodgerBlue3}“सन्निधिः । गोमा\edlabel{pvv.296-6}\footnote{\label{pvv.296-6}  ६ यस्मात् ।}नित्येव मर्त्येन भाव्यमश्ववतापि किं”} । गवाश्वस्य परस्परमप्रतिबद्धत्वादेकभावे नान्यभावः । तथा शिंशपाऽवृक्षादावपि स्यात् (।) यस्मात् कारणव्यापकनिवृत्त्या कार्यव्यापकनिवृत्तिर्दर्शनीया वि\edlabel{pvv.296-7}\footnote{\label{pvv.296-7}  ७ यस्माद् व्याप्तिग्राहकं प्रमाणं नान्यत् स्वभावप्रतिबन्धग्राहकात् । तेनैव साधनस्य साध्यायत्तत्वग्रहात् । साध्याभावेऽभावो गृहीत एव । केवलं तदविनाभावग्राहकं प्रमाणं विस्मृतत्वाद् दृष्टान्ताभ्याँ साधर्म्यवैधर्म्ययोः प्रदर्श्यते ।}पक्षे (। २४)
	\pend
      \leavevmode\marginnote{\textenglish{297/s}}\label{div_pvv.3.25}\edlabel{div_pvv.3.25}
	  
	% new div opening: depth here is 2
	
	  \bigskip
	  \begingroup
	  \large
	
	    
	    \stanza[\smallbreak]
	\label{pv.3.25}\edlabel{pv.3.25}\flagstanza{\tiny\textenglish{...v.3.25}}तस्माद् वैधर्म्यदृष्टान्ते नेष्टोवश्यमिहाश्रयः&तदभावे च तन्नेति वचनादपि तद्गतिः ॥ २५ ॥\&[\smallbreak]


	
	  \endgroup
	

	  \pstart {\color{DodgerBlue3}“तस्माद् वैधर्म्यदृष्टान्ते नेष्टो\edlabel{pvv.297-1}\footnote{\label{pvv.297-1}  १ नियमेन कार्यस्वभावे हेतौ ।}ऽवश्यमिहाश्रयः”} वस्तुभूतो धर्मी । तस्य कारणव्यापकस्य साध्यस्याभावे च तत्कार्यवायप्यं {\color{DodgerBlue3}“साधनं नेति वचनादपि”} तस्य साध्याभावे साधनाभावस्य गतिः । वस्तुनि वस्तुनिवृत्तिः सन्दिग्‏धा वस्तुसम्बन्धाविरोधात् । अवस्तुनि तु वस्तुसत्ता विरुध्यते निवृत्तिस्तु युक्ता । अतः स धर्मिणमन्तरेणापि वोङ्मात्रतोपि गम्यत एव (। २५)
	\pend
      \label{div_pvv.3.26}\edlabel{div_pvv.3.26}
	  
	% new div opening: depth here is 2
	

	  \pstart यतः ।
	\pend
      
	  \bigskip
	  \begingroup
	  \large
	
	    
	    \stanza[\smallbreak]
	\label{pv.3.26}\edlabel{pv.3.26}\flagstanza{\tiny\textenglish{...v.3.26}}तद्‏भावहेतुभावौ हि दृष्टान्ते तदवेदिनः ।&ख्याप्येते विदुषां वाच्यो हेतुरेव हि केवलः ॥ २६ ॥\&[\smallbreak]


	
	  \endgroup
	

	  \pstart तद्‏भावहेतुभावौ {\color{DodgerBlue3}“दृष्टान्ते ख्याप्येते”} (।) तस्य साधनस्य भावस्तादात्म्यं साध्यस्य हेतुभा\edlabel{pvv.297-2}\footnote{\label{pvv.297-2}  २ स्वभावहेतौ साध्यस्य तद्‏भावः साधनव्यापकत्वं । कार्यहेतौ साध्यस्य हेतुभावः कारणत्वं ।}वश्च । ख्याप्यते वैधर्म्यदृष्टान्ते साध्यनिवृत्त्या {\color{DodgerBlue3}“साधननिवृत्ति”}कथनेन तयोस्तादात्म्यतदुत्पत्तिबन्धमजानतो याह्यन्यनिवृत्त्येतरनिवृत्तिः । सा वस्तुभूतमाश्रयमन्तरेणापि शक्यते निर्देष्टुं । ये तु प्रतिबन्धं विदन्ति\edlabel{pvv.297-3}\footnote{\label{pvv.297-3}  ३ अविस्मृतेः ।} तेषां {\color{DodgerBlue3}“विदुषां हेतुरेव केवलो वाच्यः”}\edlabel{pvv.297-4}\footnote{\label{pvv.297-4}  ४ पक्षधर्ममात्रनिश्चयार्थं ।} । न दृष्टान्तः तत्र दर्शनीयस्य\edlabel{pvv.297-5}\footnote{\label{pvv.297-5}  ५ अन्वयव्यतिरेकनिश्चायकस्य ।}प्रतिबन्धस्य सिद्धत्वात् । यस्माद् दृष्टान्ते प्रतिबन्धः कथ्यते । (२६)
	\pend
      \label{div_pvv.3.27}\edlabel{div_pvv.3.27}
	  
	% new div opening: depth here is 2
	
	  \bigskip
	  \begingroup
	  \large
	
	    
	    \stanza[\smallbreak]
	\label{pv.3.27}\edlabel{pv.3.27}\flagstanza{\tiny\textenglish{...v.3.27}}तेनैव ज्ञातसम्बन्धे द्वयोरन्यतरोक्तितः ।&अर्थापत्या द्वितीयेपि स्मृतिः समुपजायते ॥ २७ ॥\&[\smallbreak]


	
	  \endgroup
	

	  \pstart {\color{DodgerBlue3}“तेनैव”} कारणेन {\color{DodgerBlue3}“ज्ञातसम्बन्धे”} हेतौ साधर्म्यवैधर्म्यदृष्टान्तयो{\color{DodgerBlue3}“रन्यतरस्योक्तितोऽर्थापत्त्या”} सामर्थ्येन {\color{DodgerBlue3}“द्वितीयेपि स्मृतिः समुपजा”}यते इति न तस्य निर्देशः कर्त्तव्यः (।) तथा हि यत् कृतकं तदनित्यं\edlabel{pvv.297-6}\footnote{\label{pvv.297-6}  ६ यदैककार्यकरणं प्रति सामर्थ्यं तत् तदैव न पूर्वं न पश्चात् तत्कार्याभावात् (।) सामर्थ्यञ्च तदव्यतिरिक्तमेवमुत्तरोत्तरकार्योत्पत्तावपि सामर्थ्यभेदेन पदार्थभेदात् क्षणिक एव क्रमाक्रमनियमः । अन्यथा एकदैककार्यकरणेऽनेककृतौ वान्यदाऽवस्तुत्वं कार्याभावात् पुनः कृतोक्रम एव । अथ प्रकारान्तरेण नैकदा नापि पुनः पुनः करोति तदास्यावस्तुत्वं नित्यस्याकर्तृत्वात् ।} यथा घट इति (।) तादात्म्ये निर्ज्ञाते नित्यत्वाभावे\leavevmode\marginnote{\textenglish{59a/MA}} \leavevmode\marginnote{\textenglish{298/s}} कृतकत्वाभावः सामर्थ्यादाकाशादौ गम्यते । तथा वैधर्म्यदृष्टान्तेन तादात्म्ये कथिते सामर्थ्यादन्वयो घटादौ गम्यते ॥
	\pend
      

	  \pstart एवं साधनभावे साध्यस्यावश्यं भावो यदि साध्याभावे हेत्वभावः । तथा साध्याभावे हेत्वभावस्तदा भवति । (२७)
	\pend
      \label{div_pvv.3.28}\edlabel{div_pvv.3.28}
	  
	% new div opening: depth here is 2
	

	  \pstart यदि साधनभावेऽवश्यं साध्यभावः । एवं कार्यानुपलम्भयोरपि योज्यं ॥
	\pend
      
	  \bigskip
	  \begingroup
	  \large
	
	    
	    \stanza[\smallbreak]
	\label{pv.3.28}\edlabel{pv.3.28}\flagstanza{\tiny\textenglish{...v.3.28}}हेतुस्वभावाभावोतः प्रतिषेधे च कस्यचित् ।&हेतुर्युक्तोपलम्भस्य तस्य चानुपलम्भनम् ॥ २८ ॥\&[\smallbreak]


	
	  \endgroup
	

	  \pstart यतः कारणव्यापकनिवृत्तिभ्यां कार्यव्याप्यनिवृत्तिः अतो हेतोः\edlabel{pvv.298-1}\footnote{\label{pvv.298-1}  १ कारणस्य ।} {\color{DodgerBlue3}“स्वभावस्य”} व्यापकस्या{\color{DodgerBlue3}“भावः । कस्यचित्”} कार्यस्य व्याप्यस्य च {\color{DodgerBlue3}“प्रतिषेधे”}ऽभावे\edlabel{pvv.298-2}\footnote{\label{pvv.298-2}  २ च शब्दात् ।}ऽभावव्यवहारे च हेतुः । तथा {\color{DodgerBlue3}“तस्य”} प्रतिषेध्यस्य\edlabel{pvv.298-3}\footnote{\label{pvv.298-3}  ३ स्वभावस्य ।} {\color{DodgerBlue3}“युक्तो\edlabel{pvv.298-4}\footnote{\label{pvv.298-4}  ४ न्याय्य ।}पलम्भस्य”} उपलब्धिलक्षणप्राप्तस्या{\color{DodgerBlue3}“नुपलम्भनं”} प्रति\edlabel{pvv.298-5}\footnote{\label{pvv.298-5}  ५ कारणव्यापकानुपलब्धी तु प्रतिषेधतद्‏व्यवहारयोः ।}षेधे\edlabel{pvv.298-6}\footnote{\label{pvv.298-6}  ६ इत्यर्थः परः ।} प्रतिषेधव्यवहारे हेतुः\edlabel{pvv.298-7}\footnote{\label{pvv.298-7}  ७ किं कारणं प्रतिषेधस्य न हेतुरित्याह ।}। तस्य स्वयमेवाभावरूपत्वात् ॥ (२८)
	\pend
      \label{div_pvv.3.29}\edlabel{div_pvv.3.29}
	  
	% new div opening: depth here is 2
	
	  \bigskip
	  \begingroup
	  \large
	
	    
	    \stanza[\smallbreak]
	\label{pv.3.29}\edlabel{pv.3.29}\flagstanza{\tiny\textenglish{...v.3.29}}इतीयन्त्रिविधोक्ताप्यनुपलब्धिरनेकधा ।&तत्तद्‏विरुद्धाद्यगतिगतिभेदप्रयोगतः ॥ २९ ॥\&[\smallbreak]


	
	  \endgroup
	

	  \pstart {\color{DodgerBlue3}“इति”} निर्दिष्टक्रमेणे{\color{DodgerBlue3}“यम”}नुपलब्धिः कारणव्यापकस्वभावानुपलब्धिभेदेन {\color{DodgerBlue3}“त्रिविधा\edlabel{pvv.298-8}\footnote{\label{pvv.298-8}  ८ समासात् ।}प्यनेकधा”} बहुप्रकारा {\color{DodgerBlue3}“तत्तद्विरुद्धाद्यगतिगतिभेदप्रयोगतः । ते च”} कारणव्यापकस्वभावा{\color{DodgerBlue3}“स्तेषां”} विरुद्धादयश्च तत्तद्विरुद्धादय आदि\edlabel{pvv.298-9}\footnote{\label{pvv.298-9}  ९ कारणविरुद्धकार्योपलब्धिरादिना ।}शब्दाद् विरुद्धकार्यादयश्च तेषां यथाक्रममगतिगतयः कारणव्यापकस्वभावानामनुपलब्धय (:।) तेषां विरुद्धादीनामुपलब्धयश्च तासामन्योन्यं भेदो नानात्वं तस्य प्रयोगतः । शब्दस्याभिधाव्यापारात् । यथोक्तं\edlabel{pvv.298-10}\footnote{\label{pvv.298-10}  १० प्रागत्राष्टधोक्तवत् ।}प्राक् । सैषाऽनेकप्रकारापि त्रिविधानुपलब्धिः संगृहीतेत्यर्थः ॥ (२९)
	\pend
      \label{div_pvv.3.30}\edlabel{div_pvv.3.30}
	  
	% new div opening: depth here is 2
	

	  \pstart उक्तमर्थ श्लोकाभ्यां संगृह्णन्नाह ।
	\pend
      
	  \bigskip
	  \begingroup
	  \large
	
	    
	    \stanza[\smallbreak]
	\label{pv.3.30}\edlabel{pv.3.30}\flagstanza{\tiny\textenglish{...v.3.30}}कार्यकारणभावाद्वा स्वभावाद्वा नियामकात् ।&अविनाभावनियमोऽदर्शनान्न न दर्शनात् ॥ ३० ॥\&[\smallbreak]


	
	  \endgroup
	\leavevmode\marginnote{\textenglish{299/s}}

	  \pstart {\color{DodgerBlue3}“कार्यकारणभावात्”} तदुत्पत्तेर्व्वा {\color{DodgerBlue3}“नियामकात्”} साधनस्य साध्याव्यभिचारकारणात् {\color{DodgerBlue3}“स्वभावात्”} तादात्म्याद्वा नियाम{\color{DodgerBlue3}“कादवि\edlabel{pvv.299-1}\footnote{\label{pvv.299-1}  १ अविनाभावेनैव सिद्धे पुनर्नियमग्रहणं परनिरासाय । स ह्यन्यं नियममिच्छति ।}नाभावनियमः”} साध्याव्यभिचारित्वनियमः साधनस्य । विपक्षे हेतो{\color{DodgerBlue3}“रदर्शनात्”} न सपक्षे {\color{DodgerBlue3}“दर्शनात्”} । दर्शनादर्शनयोर्व्यभिचारिण्यपि हेतौ सम्भवात् नियमहेत्वभावाच्च ॥ (३०)
	\pend
      \label{div_pvv.3.31}\edlabel{div_pvv.3.31}
	  
	% new div opening: depth here is 2
	
	  \bigskip
	  \begingroup
	  \large
	
	    
	    \stanza[\smallbreak]
	\label{pv.3.31}\edlabel{pv.3.31}\flagstanza{\tiny\textenglish{...v.3.31}}अवश्यंभावनियमोऽन्यथा परस्य कः परैः ।&अर्थान्तरनिमित्ते वा धर्मे वाससि रागवत् ॥ ३१ ॥\&[\smallbreak]


	
	  \endgroup
	

	  \pstart {\color{DodgerBlue3}“अन्यथा”} तादात्म्यतदुत्पत्त्योरव्यभिचारनिबन्धनयोरस्वीकारे {\color{DodgerBlue3}“परस्य”} साध्यस्य {\color{DodgerBlue3}“परैः”} साधनैः {\color{DodgerBlue3}“कोऽवश्यंभावनियमो”} न कश्चित् ।उत्पादकादन्योर्थो{\color{DodgerBlue3}“ऽर्थान्तरं तन्निमित्ते वा धर्मे”}\edlabel{pvv.299-2}\footnote{\label{pvv.299-2}  २ कृतकस्य हेतोरर्थान्तरस्य मृद्गादेर्निमित्तत्वमनित्यं प्रतीष्यते यत् तस्य कुतो नियमः ।}ऽस्वभावभूते कोऽवश्यम्भावनियमः । {\color{DodgerBlue3}“वाससि रागवत्”}। यथा\edlabel{pvv.299-3}\footnote{\label{pvv.299-3}  ३ कुसुम्भादि ।}र्थान्तरनिमित्तस्य रागस्य वस्त्रे\edlabel{pvv.299-4}\footnote{\label{pvv.299-4}  ४ पूर्व्वनिष्पन्ने ।} नावश्यंभावनि\edlabel{pvv.299-5}\footnote{\label{pvv.299-5}  ५ दूषणान्तरमाह । अर्थान्तरनिमित्तमनित्यत्वं कृतकादन्यदेव स्यात् ।}यमः ॥ (३१)
	\pend
      \label{div_pvv.3.32}\edlabel{div_pvv.3.32}
	  
	% new div opening: depth here is 2
	

	  \begin{center}%% label @type='head'
	\textbf{(ख) अविनाभावनियमः}
	\end{center}
	
	  \bigskip
	  \begingroup
	  \large
	
	    
	    \stanza[\smallbreak]
	\label{pv.3.32}\edlabel{pv.3.32}\flagstanza{\tiny\textenglish{...v.3.32}}अर्थान्तरनिमित्तो हे धर्मः स्यादन्य एव सः ।&पश्चाद्भावान्न हेतुत्वं फलेप्येकान्तता कुतः ॥ ३२ ॥\&[\smallbreak]


	
	  \endgroup
	

	  \pstart वस्त्रोत्पादका{\color{DodgerBlue3}“दर्थान्तरनिमित्तो”} रागद्रव्यकारणो {\color{DodgerBlue3}“हि धर्मो”} रागः\edlabel{pvv.299-6}\footnote{\label{pvv.299-6}  ६ अयमेव हि भेदो भेदहेतुर्व्वा यदुत विरुद्धधर्माध्यासः कारणभेदश्च ।} प्रागुत्पन्नाद् वस्त्रा{\color{DodgerBlue3}“दन्य एव स्याद्”} भिन्नकालत्वाद् भिन्नहेतुकत्वाच्च (।) इत्थ\edlabel{pvv.299-7}\footnote{\label{pvv.299-7}  ७ त्रैगुण्यादैक्यमिति चेदाह ।}मपि यद्येकत्वं\leavevmode\marginnote{\textenglish{59b/MA}} विश्वमेकं भवेत् । ततश्च सहोत्पत्तिनाशौ स्यातां । सर्वत्र सर्व्वं चोपयुज्यते । एकत्वाभिमानस्तु सदृशापरापरोत्पत्तेर्भ्रान्त्या । तस्य\edlabel{pvv.299-8}\footnote{\label{pvv.299-8}  ८ अहेतुफलस्य साध्यस्यासंबन्धाद्धेतुः फलम्वा स्यादित्याह ।} चा\edlabel{pvv.299-9}\footnote{\label{pvv.299-9}  ९ कृतकादर्थान्तरनिमित्तस्यानित्यस्य न हेतुत्वं ।}र्थान्तरनिमित्त\edlabel{pvv.299-10}\footnote{\label{pvv.299-10}  १० फले नित्ये पश्चाद्भाविनि च नैकान्तता कृतकस्य ज्ञापकहेतोः ।}स्य धर्मस्य वस्त्रोत्पादात् {\color{DodgerBlue3}“पश्चाद्भावान्न हेतुत्वं”} वस्त्रं प्रति । अतः कारणतयापि नास्यानुमानं रागोत्पत्तौ वस्त्रं सहकारिकारणं । अतः कार्यतया रागानुमानं चेत् \leavevmode\marginnote{\textenglish{300/s}} {\color{DodgerBlue3}“फलेपि”} कारणादनुमीयमाने {\color{DodgerBlue3}“एकान्तता कुतः”} । न ह्यवश्यं कारणानि कार्यवन्ति भवन्ति ॥ (३२)
	\pend
      \label{div_pvv.3.33}\edlabel{div_pvv.3.33}
	  
	% new div opening: depth here is 2
	

	  \pstart यदि दर्शनादर्शने नान्वयव्यतिरेकबुद्धिहेतुर्द्धूमोग्निन्न व्यभिचरतीति न स्यात प्रतिपत्तिरित्याह ।
	\pend
      
	  \bigskip
	  \begingroup
	  \large
	
	    
	    \stanza[\smallbreak]
	\label{pv.3.33}\edlabel{pv.3.33}\flagstanza{\tiny\textenglish{...v.3.33}}कार्यं धूमो हुतभुजः कार्यधर्मानुवुत्तितः ।&तस्याभावे तु स भवन् हेतुमत्तां विलंघयेत् ॥ ३३ ॥\&[\smallbreak]


	
	  \endgroup
	

	  \pstart {\color{DodgerBlue3}“कार्यं धूमो हुतभुजः”} कार्यधर्मस्य कारणान्वयव्यतिरेकानुविधायित्वस्य त्रिविध\edlabel{pvv.300-1}\footnote{\label{pvv.300-1}  १ प्रागदृष्टौ क्रमात् पश्यन् वेत्ति हेतुफलस्थितिं । दृष्टौ वा क्रमशोऽपश्यन्नन्यथा त्वनवस्थितिः ॥}दर्शनादर्शननिश्चितस्यानुवृत्तितः स\edlabel{pvv.300-2}\footnote{\label{pvv.300-2}  २ धूमस्यान्यत्राग्निजन्यत्वे न किञ्चिद् बाधकमस्ति । तदेवेदमिति च प्रतीतेस्तत्सामान्यं प्रतीतमुच्यते ।}धूमस्तस्याग्नेर{\color{DodgerBlue3}“भावे तु”} भवन् {\color{DodgerBlue3}“हेतुमत्ताम्वि\edlabel{pvv.300-3}\footnote{\label{pvv.300-3}  ३ न वाग्निसम्बन्धितयात्यक्षमीक्षेताकारणान् सकृदप्यनुदयात् ।}लंघयेदति”}क्रामयेत् ॥ (३३)
	\pend
      \label{div_pvv.3.34}\edlabel{div_pvv.3.34}
	  
	% new div opening: depth here is 2
	
	  \bigskip
	  \begingroup
	  \large
	
	    
	    \stanza[\smallbreak]
	\label{pv.3.34}\edlabel{pv.3.34}\flagstanza{\tiny\textenglish{...v.3.34}}नित्यं सत्त्वमसत्त्वं वाऽहेतोरन्यानपेक्षणात् ।&अपेक्षातश्च भावानां कादाचित्कस्य सम्भवः ॥ ३४ ॥\&[\smallbreak]


	
	  \endgroup
	

	  \pstart अहेतुत्वे च धूमस्य {\color{DodgerBlue3}“नित्यं सत्त्वमा”}काशस्येव स्यात् । {\color{DodgerBlue3}“असत्त्वं”} शशविषाणादेरिव । {\color{DodgerBlue3}“अहेतोरन्यापेक्षणा”}भावात् {\color{DodgerBlue3}“अपेक्षातश्च भावानां कादाचित्कस्य सम्भवः”} । ततो यद्यहेतुर्भावस्तदा नित्यं स्यात् । असदेव वा स्याद्धेत्वभावात् (।) तस्माद् धूमस्य\edlabel{pvv.300-4}\footnote{\label{pvv.300-4}  ४ नापि स्वभावतो भवति । अनिष्पन्नस्यासत्वादेव ।} कादाचित्कत्वदर्शनात् हेतुमत्वं । अग्न्यन्वयव्यतिरेकानुविधानदर्शनात् तत्कार्यत्वञ्च । यश्च यस्य कार्यं स तं न व्यभिचरति । तदधीनस्वरूपत्वात्\edlabel{pvv.300-5}\footnote{\label{pvv.300-5}  ५ तद्विकारणमकारणं ।}॥ (३४)
	\pend
      \label{div_pvv.3.35}\edlabel{div_pvv.3.35}
	  
	% new div opening: depth here is 2
	

	  \pstart अतश्च (।)
	\pend
      
	  \bigskip
	  \begingroup
	  \large
	
	    
	    \stanza[\smallbreak]
	\label{pv.3.35}\edlabel{pv.3.35}\flagstanza{\tiny\textenglish{...v.3.35}}अग्निस्वभावः शक्रस्य मूर्धा यद्यग्निरेव सः ।&अथानग्निस्वभावोसौ धूमस्तत्र कथं भवेत् ॥ ३५ ॥\&[\smallbreak]


	
	  \endgroup
	

	  \pstart {\color{DodgerBlue3}“अग्निस्वभावः शक्रस्य मूर्द्धा”} वल्मीको {\color{DodgerBlue3}“यद्यग्निरेव स”} तदा न हि वह्निस्वरूपतां विहायान्यद् वह्ने रूपं । {\color{DodgerBlue3}“अथा”}न्यथा प्रतीयमानत्वाद{\color{DodgerBlue3}“नग्निस्वभावोसौ”} तदा {\color{DodgerBlue3}“धूमो”} वह्नेर्जन्यस्वभाव{\color{DodgerBlue3}“स्तत्र शक्रमूर्ध्नि कथं भवेत्”} । न हि वह्निजन्योन्यस्माद् भवितु\leavevmode\marginnote{\textenglish{301/s}} मर्हति तदधीनत्वात् । ततः शक्रमूर्ध्नो धूमोत्पत्तिरिति भ्रान्तिरेषा\edlabel{pvv.301-1}\footnote{\label{pvv.301-1}  १ वाष्पे वर्षासु । ..........}। वह्नेरेव तद्देशवर्तिनोऽनुपलक्षितादुत्पत्तिः ॥(३५)
	\pend
      \label{div_pvv.3.36}\edlabel{div_pvv.3.36}
	  
	% new div opening: depth here is 2
	

	  \pstart किञ्च (।)
	\pend
      
	  \bigskip
	  \begingroup
	  \large
	
	    
	    \stanza[\smallbreak]
	\label{pv.3.36}\edlabel{pv.3.36}\flagstanza{\tiny\textenglish{...v.3.36}}धूमहेतुस्वभावो हि वह्निस्तच्छक्तिभेदवान् ।&अधूमहेतोर्धूमस्य भावे स स्यादहेतुकः ॥ ३६ ॥\&[\smallbreak]


	
	  \endgroup
	

	  \pstart {\color{DodgerBlue3}“धूमहेतुस्वभावस्तच्छक्ति\edlabel{pvv.301-2}\footnote{\label{pvv.301-2}  २ खद्योतादिसकाशात् ।}भेदवान्”} धूमजननशक्तिविशेषयुक्तो {\color{DodgerBlue3}“वह्निः”} प्रतीतः । {\color{DodgerBlue3}“अधूमहेतो”}रदहनात् {\color{DodgerBlue3}“धूमस्य भावे स”} धूमो{\color{DodgerBlue3}“ऽहेतुकः स्यात्”} । हेतुं प्रमाणनिश्चितमन्तरेणैवोत्पादात् । अहेतुत्वे च नित्यं सत्त्वमसत्त्वम्वां स्यादित्युक्तं ॥(३६)
	\pend
      \label{div_pvv.3.37}\edlabel{div_pvv.3.37}
	  
	% new div opening: depth here is 2
	

	  \pstart स्यादेतत्(।) शालूकादि स्वबीजाद् विजाती{\color{DodgerBlue3}“याच्च”} गोमयादेर्दृश्यते (।) तत्कथमवह्नेर्द्धूमोत्पादेऽहेतुत्वप्रसङ्ग इत्याह (।)
	\pend
      
	  \bigskip
	  \begingroup
	  \large
	
	    
	    \stanza[\smallbreak]
	\label{pv.3.37}\edlabel{pv.3.37}\flagstanza{\tiny\textenglish{...v.3.37}}अन्वयव्यतिरेकाद्यः यस्य दृष्टोनुवर्तकः ।&स्वभावस्तस्य तद्धेतुरतो भिन्नान्न सम्भवः ॥ ३७ ॥\&[\smallbreak]


	
	  \endgroup
	

	  \pstart {\color{DodgerBlue3}“अन्वय\edlabel{pvv.301-3}\footnote{\label{pvv.301-3}  ३ कार्यहेतुमधि ।}व्यतिरेकाद् यः स्वभावो यस्यानुवर्त्तको”}ऽपेक्षको दृष्टस्तंस्य स्वभावस्य तदनुवर्त्त्यमानं हेतुरतो भिन्नाद् विजातीयान्न संभवः कस्यचि\edlabel{pvv.301-4}\footnote{\label{pvv.301-4}  ४ उत्पद्यमानस्य पूर्व्वापररूपविविक्तस्य प्रत्यक्षेण ग्रहात् क्षणिकग्रह एव । न त्वक्षणिकग्रहः पूर्व्वापरकालयोरभासात् तत्सम्बन्धितयाऽभानमिदानीं पूर्व्वस्य विनाशः । अन्यस्वभावस्य भानमेवोत्पाद इति कथमुच्यते पूर्वोत्तरक्षणानां विनाशोत्पादादृष्टेरक्षणिक इति । नाप्यनेकक्षणरूप इदानीन्तनः तत्त्वे ह्यतीतादित्वमस्य स्यात् । तत्र चासत्त्वादेवाभानं । भासश्च वर्त्तमानस्यैव सत्त्वात् । तेन स्पष्टवस्तुभानमेव क्षणिकत्वं भ्रान्तेस्तु नाध्यवसायः प्रथममनुपलब्ध्या प्राक् सत्त्वं । अन्यत आगमनं तत्सिन्निहि (त) कुड्यादेर्हेतुत्वनिषेधः । एतावद्भिर्धूमो नाग्निजन्यः स्यात् । प्रथमप्रत्यक्षपक्षे यत्संनिधानात् कार्यप्रवृत्तिस्तन्मध्ये यदभावे कार्याभावस्तत्करणं ज्ञेयं गर्दभादेः ।}द् । यद्धि शालूकादि स्वबीजप्रभवं यच्च विजातीयप्रभवं तयोः सदृशत्वप्रतीतावपि न तादृशत्वं रसवीर्यविपाक\edlabel{pvv.301-5}\footnote{\label{pvv.301-5}  ५ यदा कदलीबीजकन्दोद्भवा स्फुटमेव तादृशो लोको विवेचयत्याकारभेदात् ।}भेदात् । यदि तु हेतुभेदेप्यभेदो विश्वात्मकं द्रव्यं स्यादित्याद्युक्तं\leavevmode\marginnote{\textenglish{60a/MA}} प्रसज्येत ॥(३७)
	\pend
      \label{div_pvv.3.38}\edlabel{div_pvv.3.38}
	  
	% new div opening: depth here is 2
	

	  \begin{center}%% label @type='head'
	\textbf{(४) सामान्यचिन्ता}
	\end{center}
	
	  \bigskip
	  \begingroup
	  \large
	
	    
	    \stanza[\smallbreak]
	\label{pv.3.38}\edlabel{pv.3.38}\flagstanza{\tiny\textenglish{...v.3.38}}स्वभावेप्यविनाभावो भावमात्रानुरोधिनि ।&तदभावे स्वयम्भावस्याभावः स्यादभेदतः ॥ ३८ ॥\&[\smallbreak]


	
	  \endgroup
	\leavevmode\marginnote{\textenglish{302/s}}

	  \pstart स्वभावे\edlabel{pvv.302-1}\footnote{\label{pvv.302-1}  १ स्वभावे भावोपीत्युक्तेपि प्राक् तदनूद्य पक्षैकदेशासिद्धिनिवृत्त्यर्थं पुनराह ।}पि हेतावविनाभावः । क्व (।) साध्ये {\color{DodgerBlue3}“भावमात्रानुरोधिनि”} साधनस्वरूपमात्रानुवर्त्तिनि व्यापके यथा कृतकत्वस्यानित्यत्वे (।) यत{\color{DodgerBlue3}“स्त”}स्य व्यापकस्या{\color{DodgerBlue3}“भावे भावस्य”} व्याप्यस्य कृतकत्वादेः {\color{DodgerBlue3}“स्वयमभावः स्यात् । अभेदत”} ऐकात्म्याद् वस्तुतः ॥ (३८)
	\pend
      \label{div_pvv.3.39}\edlabel{div_pvv.3.39}
	  
	% new div opening: depth here is 2
	

	  \pstart यदि य एव कृतकः स एवानित्यो भेदाभावात् तदा\edlabel{pvv.302-2}\footnote{\label{pvv.302-2}  २ एकं सन्धित्सतोऽपरं प्रबाधते ।}  प्रतिज्ञा\edlabel{pvv.302-3}\footnote{\label{pvv.302-3}  ३ पक्षनिर्देशः प्रतिज्ञा तस्या अर्थो धर्म्मर्धाम्मसमुदायः । तदैकदेशः साध्यधर्मात्मको हेतुः स्यादसिद्धः (।) अत्र च पूर्ववृत्तेन धर्मकल्पनाबीजं द्वितीयेन धर्मकल्पना तृतीयेन प्रतिज्ञार्थैकदेशगतापरिहार इति समासार्थः ।}र्थैकदेशो हेतुः स्यादित्याह ।
	\pend
      
	  \bigskip
	  \begingroup
	  \large
	
	    
	    \stanza[\smallbreak]
	\label{pv.3.39}\edlabel{pv.3.39}\flagstanza{\tiny\textenglish{...v.3.39}}सर्वे भावाः स्वभावेन स्वस्वभावव्यवस्थितेः ।&स्वभावपरभावाभ्यां व्यावृत्तिभागिनो यतः ॥ ३९ ॥\&[\smallbreak]


	
	  \endgroup
	

	  \pstart {\color{DodgerBlue3}“सर्व्वे भावाः स्वभावेन\edlabel{pvv.302-4}\footnote{\label{pvv.302-4}  ४ जात्यादिभिर्भावा भिद्यन्ते न स्वभावेनेति परः (।) तन्न (।) भावा भिन्नाभिन्नभेदकारणे भावस्य न कि (ञ्चि)त् ॥ सर्व्वभावा भिन्ना इति नाध्यक्षानुगम्यमिति बहवः । तन्न । “
	    \begin{verse}
	तत्वमुक्तं प्रतिद्रव्यं भिन्नरूपोपलम्भनात् ।\\
	    न ह्याख्यातुमशक्यत्वाद् भेदो नास्तीति गम्यते ॥\\
	    गवाश्वादीनां स्वरूपेणोत्पत्तिरेव भेदः ।\\
	    
	    \end{verse}
	  ”} स्वस्वभावव्यवस्थितेः”} । आत्मात्मीयरूपवस्थितत्वात् । स्वभावपरभावाभ्यां सजातीयाद् विजातीयाच्च व्यावृत्ति\edlabel{pvv.302-5}\footnote{\label{pvv.302-5}  ५ यदि न वस्तुतो धर्मधर्मिभावः कथम्बुद्धिभेद इत्याह ।}भागिनो यस्मा\edlabel{pvv.302-6}\footnote{\label{pvv.302-6}  ६ नित्याकृतकादेः ।}न्न केनचिन्मिश्राः । (३९)
	\pend
      \label{div_pvv.3.40}\edlabel{div_pvv.3.40}
	  
	% new div opening: depth here is 2
	
	  \bigskip
	  \begingroup
	  \large
	
	    
	    \stanza[\smallbreak]
	\label{pv.3.40}\edlabel{pv.3.40}\flagstanza{\tiny\textenglish{...v.3.40}}तस्माद् व्यावृत्तिरर्थानां यतश्च तन्निबन्धनाः ।&जातिभेदाः प्रकल्प्यन्ते तद्विशेषावगाहिनः ॥ ४० ॥\&[\smallbreak]


	
	  \endgroup
	

	  \pstart तस्मात् {\color{DodgerBlue3}“यतो”} यतोऽस्वरूपादर्था\edlabel{pvv.302-7}\footnote{\label{pvv.302-7}  ७ शब्दादयः ।}द् {\color{DodgerBlue3}“व्यावृत्तिरर्थानां तन्निबन्धनाः”}\edlabel{pvv.302-8}\footnote{\label{pvv.302-8}  ८ नित्यो नेति वासनातोऽनित्यबुद्धिरेवेति बुद्धिभेदः ।} तत्तद्व्या-
	\pend
      \leavevmode\marginnote{\textenglish{303/s}}

	  \pstart वृत्तिनिमित्ता {\color{DodgerBlue3}“जाति\edlabel{pvv.303-1}\footnote{\label{pvv.303-1}  १ अनित्यकृतकादयः ।}भेदास्त\edlabel{pvv.303-2}\footnote{\label{pvv.303-2}  २ स्वलक्षणस्य ये विशेषा अकृतकादिव्यावृत्ताः ।}द्विशेषावगाहिनः”} तत्स्वलक्षणाश्रयाः {\color{DodgerBlue3}“कल्प्यन्ते”} शब्दैः स्ववाच्यतया ॥ (४०)
	\pend
      \label{div_pvv.3.41}\edlabel{div_pvv.3.41}
	  
	% new div opening: depth here is 2
	

	  \pstart यतः शब्दः काश्चिद् अर्थक्रियाः कुर्वन्नतत्कारिणोऽशब्दादकृतकान्नित्यादेश्च व्यावृत्त इति तत्तद्वयवच्छेदप्रतिपादनार्थं शब्दकृतकानित्यत्वाद्या जातयोऽनन्यशब्दवाच्यतया कल्प्यन्ते ।
	\pend
      
	  \bigskip
	  \begingroup
	  \large
	
	    
	    \stanza[\smallbreak]
	\label{pv.3.41}\edlabel{pv.3.41}\flagstanza{\tiny\textenglish{...v.3.41}}तस्माद् विशेषो यो येन धर्मेण संप्रतीयते ।&न स शक्यस्ततोन्येन तेन भिन्ना व्यवस्थितिः ॥ ४१ ॥\&[\smallbreak]


	
	  \endgroup
	

	  \pstart {\color{DodgerBlue3}“तस्मात्”} स्वभावाभेदेपि कृतकत्वानित्यत्वादीनां {\color{DodgerBlue3}“येन”} धर्मेण नाम्नाऽनित्य इत्यनेन {\color{DodgerBlue3}“यो विशेषो”} नित्याद् व्यावृत्तिः {\color{DodgerBlue3}“संप्रतीयते न”} स विशेषः {\color{DodgerBlue3}“शक्यः ततो”}ऽनित्यशब्दा{\color{DodgerBlue3}“दन्येन”} कृतकशब्देन प्रत्येतुं (।)। {\color{DodgerBlue3}“तेन”} कारणेन साध्यसाधनयो{\color{DodgerBlue3}“र्भिन्ना व्यवस्थितिः”} । यद्यपि शब्द एव कृतकोऽनित्यश्च तथाप्यशब्दव्यावृत्ततया निश्चितो धर्मी कृतकतया च हेतुः । अनित्यतया चासिद्धः साध्यः\edlabel{pvv.303-3}\footnote{\label{pvv.303-3}  ३ अतो भिन्नविषया विकल्पाः शब्दाश्च तत्समविषया अपर्यायाः ।} । ततो धर्मिसाध्यसाधनानां भेदः कल्पितो न तु तान्येव कल्पितानि तेषां सत्त्वात् । ततः कल्पितधर्मिणि कल्पितात् साधनात् कल्पितस्य साध्यस्य\edlabel{pvv.303-4}\footnote{\label{pvv.303-4}  ४ सर्व्वगत आत्मा सर्व्वत्रोपलभ्यमानगुणत्वादित्यादि ।} सिद्धिमाचक्षाणः प्रत्याख्यातः (।) {\color{DodgerBlue3}“तस्मान्न”} प्रतिज्ञार्थैकदेशो हेतुः । साध्यसाधनयोर्भिन्नव्यवच्छेदरूपत्वात् ॥ (४१)
	\pend
      \label{div_pvv.3.42}\edlabel{div_pvv.3.42}
	  
	% new div opening: depth here is 2
	

	  \pstart कस्मात् पुनर्व्यवच्छेदः शब्दलिङ्गाभ्यां प्रतिपाद्यत इतीष्यते न तु {\color{DodgerBlue3}“वस्त्वेव”} विधिरूपेणेत्याह\edlabel{pvv.303-5}\footnote{\label{pvv.303-5}  ५ सामान्यं खलु नात्माधि (ष्ठि)तोपि कल्पितो धर्मधर्मिभावोन्यथापि साध्यते ।} (।)
	\pend
      
	  \bigskip
	  \begingroup
	  \large
	
	    
	    \stanza[\smallbreak]
	\label{pv.3.42}\edlabel{pv.3.42}\flagstanza{\tiny\textenglish{...v.3.42}}एकस्यार्थस्वभावस्य प्रत्यक्षस्य सतः स्वयम् ।&कोन्यो भागो न दृष्टः स्यात् यः प्रमाणैः परीक्ष्यते ॥ ४२ ॥\&[\smallbreak]


	
	  \endgroup
	

	  \pstart {\color{DodgerBlue3}“एकस्थार्थस्वभावस्य स्वयमात्मना प्रत्यक्षस्य सतः कोन्यो भागो न दृष्टः स्यात् यः प्रमाणैः\edlabel{pvv.303-6}\footnote{\label{pvv.303-6}  ६ धर्मधर्मिनिरंशस्य दृष्टस्यानुमानैर्व्यक्त्यपेक्षया बहुवचनं ।} परीक्ष्यते”} निर्ण्णीयते । यदि प्रमाणान्तरैः शब्दान्तरैश्च वस्त्वेव विषयीकर्त्तव्यं\edlabel{pvv.303-7}\footnote{\label{pvv.303-7}  ७ यथाऽनित्ये साध्ये शब्दो धर्मी प्रत्यक्षतः सर्व्वांकारं सिद्धः स्यात् ।}तदा तत् प्रत्यक्षेण शब्दान्तरेण च दृष्टमेवेति व्यर्थानि प्रमाणानि स्युः । (४२)
	\pend
      \leavevmode\marginnote{\textenglish{304/s}}\label{div_pvv.3.43}\edlabel{div_pvv.3.43}
	  
	% new div opening: depth here is 2
	

	  \pstart त्वन्मतेपि प्रत्यक्षेण दृष्टे धर्मिणि प्रमाणान्तरवैफल्यं स्यात् (।) नेत्याह ।
	\pend
      
	  \bigskip
	  \begingroup
	  \large
	
	    
	    \stanza[\smallbreak]
	\label{pv.3.43}\edlabel{pv.3.43}\flagstanza{\tiny\textenglish{...v.3.43}}नो चेद् भ्रान्तिनिमित्तेन संयोज्येत गुणान्तरम् ।&शुक्तौ वा रजताकारो रूपसाधर्म्यदर्शनात् ॥ ४३ ॥\&[\smallbreak]


	
	  \endgroup
	

	  \pstart दृष्टे सर्वथा वस्तुनि {\color{DodgerBlue3}“नो चेद् भ्रान्तिनिमित्तेन”} सादृश्यादिना {\color{DodgerBlue3}“संयोज्ये”}तारोप्येत {\color{DodgerBlue3}“गुणान्तरं”} धर्म्मान्तरमसदेव\edlabel{pvv.304-1}\footnote{\label{pvv.304-1}  १ विकल्पेन सदृशापरोत्पत्या भ्रान्त्या नारोप्येत स्थिरत्वादि (ति) चेत् । अनुभूतानिश्चिते तु प्रमाणान्तरं प्रवर्त्तत एवारोपनिवृत्तये । तच्च लिङ्गं स्वव्यापकं विधिरूपेण निश्चितत्वादन्यसमारोपं निषेधति । अन्यथा समारोप एव व्यावृतेरल्पवृत्तिर्न स्यात् ।} शुक्तौ वा शुक्ताविव {\color{DodgerBlue3}“रूपसाधर्म्यदर्शनात्”} भ्रान्ति\leavevmode\marginnote{\textenglish{60b/MA}} निमित्ताद् रजताकारस्तदा सर्व्वथा वस्तुनिश्चयात् प्रमाणान्तरशब्दान्तरवैफल्यं स्यात् । (४३)
	\pend
      \label{div_pvv.3.44}\edlabel{div_pvv.3.44}
	  
	% new div opening: depth here is 2
	

	  \pstart एतदेवाह (।)
	\pend
      
	  \bigskip
	  \begingroup
	  \large
	
	    
	    \stanza[\smallbreak]
	\label{pv.3.44}\edlabel{pv.3.44}\flagstanza{\tiny\textenglish{...v.3.44}}तस्माद् दृष्टस्य भावस्य दृष्ट एवाखिलो गुणः ।&भ्रान्तेर्न निश्चय इति साधनं संप्रवर्त्तते ॥ ४४ ॥\&[\smallbreak]


	
	  \endgroup
	

	  \pstart {\color{DodgerBlue3}“तस्माद्\edlabel{pvv.304-2}\footnote{\label{pvv.304-2}  २ अनंशस्यैकदेशदर्शनायोगात् ।} दृष्टस्य भावस्य दृष्ट एवाखिलो गुणः”} धर्मः । तथापि भ्रान्तेर्व्विपरीताकारारोपिकाया न निश्ची\edlabel{pvv.304-3}\footnote{\label{pvv.304-3}  ३ दृष्टोपि ।}यत इत्यारोपितव्यवच्छेदार्थं {\color{DodgerBlue3}“साधनं”} शब्दान्तरञ्च {\color{DodgerBlue3}“संप्रवर्त्तते ॥”} (४४)
	\pend
      \label{div_pvv.3.45}\edlabel{div_pvv.3.45}
	  
	% new div opening: depth here is 2
	

	  \pstart न केवलं प्रत्यक्षाद् वस्तुग्रहे साधनान्तरशब्दान्तरवैफल्यं किन्त्वनुमानादपीत्याह ।
	\pend
      
	  \bigskip
	  \begingroup
	  \large
	
	    
	    \stanza[\smallbreak]
	\label{pv.3.45}\edlabel{pv.3.45}\flagstanza{\tiny\textenglish{...v.3.45}}वस्तुग्रहेनुमानाच्च धर्मस्यैकस्य निश्चये ।&सर्वग्रहो ह्यपोहे तु नायं दोषः प्रसज्यते ॥ ४५ ॥\&[\smallbreak]


	
	  \endgroup
	

	  \pstart {\color{DodgerBlue3}“वस्तुग्रहेनुमानाच्च धर्मस्यैकस्य”} कृतकत्वादेः प्रत्ययभेदाभेदित्वादिकृते {\color{DodgerBlue3}“निश्चये सर्व”}स्यानित्यत्वादे{\color{DodgerBlue3}“र्ग्रहः”} स्यात् । एकरूपत्वात् । कृतकत्वादिसाधनान्तरवैयर्थ्यं । {\color{DodgerBlue3}“अपोहे\edlabel{pvv.304-4}\footnote{\label{pvv.304-4}  ४ अपोह्यतेऽनेनेति बाह्यतया आरोपित आकारः । अपोह्यतेस्मिन् स्वलक्षणम्वाऽपोहः ।} त्व”}न्यव्यवच्छेदे शब्दलिङ्गविषये स्वीक्रियमाणे {\color{DodgerBlue3}“नायं”} प्रमाणान्तरादिवैफल्य\leavevmode\marginnote{\textenglish{305/s}} {\color{DodgerBlue3}“षः प्रसज्यते”}\edlabel{pvv.305-1}\footnote{\label{pvv.305-1}  १ न विकल्पानां स्वरूपेण बाह्यो ग्राह्योपि तु स्वाकारेण सहैकीकृत एव बाह्यो विषयः स चासत्योऽपोह्यतेऽनेनान्यदित्यपोहः ।} । न खलु {\color{DodgerBlue3}“प्रत्यग्रदृष्टं वस्त्वन्तरं शब्दलिङ्गाभ्यां विषयीक्रियते\edlabel{pvv.305-2}\footnote{\label{pvv.305-2}  २ अकस्माद् धूमादग्निप्रतिपत्तावपि विपर्यासोस्ति । अग्निमन्तं प्रदेशमनग्निमत्वेन (धूमदर्शनात् प्राक् दर्शने साग्निदेशादन्योयमिति) ग्रहात् तस्मान्न धर्म्मबुद्धिर्वस्तुग्राहिणीत्यपोहविषया लिङ्गादनुमेयमनुसरन्नवश्यं संशयितो विपर्यस्तो वा स्यात् ।}”} किन्त्वन्यव्यवच्छेदः । एकस्मिँश्च व्यवच्छेदे सिद्धेप्यसिद्धं व्यवच्छेदान्तरं लिङ्गान्तरतः शब्दान्तरतश्च साध्यत इति न कश्चिद् दोषः ॥ (४५)
	\pend
      \label{div_pvv.3.46}\edlabel{div_pvv.3.46}
	  
	% new div opening: depth here is 2
	
	  \bigskip
	  \begingroup
	  \large
	
	    
	    \stanza[\smallbreak]
	\label{pv.3.46}\edlabel{pv.3.46}\flagstanza{\tiny\textenglish{...v.3.46}}तस्मादपोहविषयं लिङ्गमिति प्रकीर्तितम् ।&अन्यथा धर्मिणः सिद्धौ किमतः साधकं परम् ॥ ४६ ॥\&[\smallbreak]


	
	  \endgroup
	

	  \pstart {\color{DodgerBlue3}“तस्मा”}दाचार्येणा{\color{DodgerBlue3}“पोहविषयं लिङ्गमि\edlabel{pvv.305-3}\footnote{\label{pvv.305-3}  ३ उक्तेन प्रकारेण न साक्षात् ।}ति प्रकीर्तितं”} । लिङ्गमुपलक्षणं शब्दश्च । {\color{DodgerBlue3}“अन्यथा”} यदि वस्तुविषयं\edlabel{pvv.305-4}\footnote{\label{pvv.305-4}  ४ नापोहविषयं ।} लिङ्गं तदा {\color{DodgerBlue3}“धर्म्मिणः”} सर्वात्मना प्रत्यक्षतः {\color{DodgerBlue3}“सिद्धौ किमतो”} धर्मिणः {\color{DodgerBlue3}“परम”}सिद्धमस्ति यत्प्र{\color{DodgerBlue3}“साधकं”} लिङ्गम्प्रमाणं स्यात् ॥ (४६)
	\pend
      \label{div_pvv.3.47}\edlabel{div_pvv.3.47}
	  
	% new div opening: depth here is 2
	
	  \bigskip
	  \begingroup
	  \large
	
	    
	    \stanza[\smallbreak]
	\label{pv.3.47}\edlabel{pv.3.47}\flagstanza{\tiny\textenglish{...v.3.47}}क्वचित् सामान्यविषयं दृष्टे ज्ञानमलिङ्गजम् ।&कथमन्यापोहविषयं तन्मात्रापोहगोचरः ॥ ४७ ॥\&[\smallbreak]


	
	  \endgroup
	

	  \pstart ननु {\color{DodgerBlue3}“क्वचि”}न्नीलादावसमारोपितोऽन्यो विपरीतांशो यस्मिन् । तस्मिन् प्रत्यक्षेण {\color{DodgerBlue3}“दृष्टे यज्ज्ञानमलिङ्गजं”} विकल्पकं {\color{DodgerBlue3}“सामान्यविषयं”} भवति । तदारोपाभावात् । कथमन्यापोहविषयं । आह\edlabel{pvv.305-5}\footnote{\label{pvv.305-5}  ५ स्वलक्षणं स्वाकाराभेदेन गृह्णत् प्रत्यक्षव्यापारमात्मन्यारोपयति । बाह्याध्यवसाय एव योज्यते तेन नीलमिति नीलसजातीये विहितं सर्व्वमन्यद् व्यवच्छिनत्ति । रूपनिश्चयेपि क्षणि (कत्वा) निश्चयात् ।} {\color{DodgerBlue3}“तन्मात्र”}स्यानीलमात्रस्या{\color{DodgerBlue3}“पोहो”} विजातीयाद् व्यावृत्तिर्व्यवच्छेदः स {\color{DodgerBlue3}“गोचरो”} यस्य तत्तथा । नीलविकल्पस्यानीलव्यवच्छेद एव विषय इत्यर्थः ॥ (४७)
	\pend
      \label{div_pvv.3.48}\edlabel{div_pvv.3.48}
	  
	% new div opening: depth here is 2
	

	  \pstart कस्मादेवमित्याह (।)
	\pend
      
	  \bigskip
	  \begingroup
	  \large
	
	    
	    \stanza[\smallbreak]
	\label{pv.3.48}\edlabel{pv.3.48}\flagstanza{\tiny\textenglish{...v.3.48}}निश्चयारोपमनसोर्बाध्यबाधकभावतः ।&समारोपविवेकेऽस्य प्रवृत्तिरिति गम्यते ॥ ४८ ॥\&[\smallbreak]


	
	  \endgroup
	\leavevmode\marginnote{\textenglish{306/s}}

	  \pstart {\color{DodgerBlue3}“निश्चयारो\edlabel{pvv.306-1}\footnote{\label{pvv.306-1}  १ प्रत्यक्षपृष्ठभाविन्यारोपि ।}पमनसोर्बाष्यबाधकभावतः ।”} समारोपस्य {\color{DodgerBlue3}“विवेके”} व्यवच्छेदेऽस्य निश्चयस्य {\color{DodgerBlue3}“प्रवृत्तिरिति गम्यते”} ॥ (४८)
	\pend
      \label{div_pvv.3.49}\edlabel{div_pvv.3.49}
	  
	% new div opening: depth here is 2
	

	  \pstart तस्माद् (।)
	\pend
      
	  \bigskip
	  \begingroup
	  \large
	
	    
	    \stanza[\smallbreak]
	\label{pv.3.49}\edlabel{pv.3.49}\flagstanza{\tiny\textenglish{...v.3.49}}यावन्तोंऽशसमारोपा निश्चयास्तन्निरासतः ।&तावन्त एव शब्दा भिन्नव्यवच्छेदगोचराः ॥ ४९ ॥\&[\smallbreak]


	
	  \endgroup
	

	  \pstart {\color{DodgerBlue3}“यावन्तोऽ”}शस्य\edlabel{pvv.306-2}\footnote{\label{pvv.306-2}  २ यतो वस्त्वध्यवसायेन शब्दादेर्वुत्तः न वस्तुस्वरूपग्रहेण ततः ।} धर्मस्य {\color{DodgerBlue3}“समारोपास्तस्य निरास”}निमित्तं वि{\color{DodgerBlue3}“निश्चयाः”} शब्दाश्च {\color{DodgerBlue3}“तावन्त”} एव (।) तेन ते निश्चयाः शब्दाश्च {\color{DodgerBlue3}“भिन्नव्यवच्छेदगोचराः”} ॥ (४९)
	\pend
      \label{div_pvv.3.50}\edlabel{div_pvv.3.50}
	  
	% new div opening: depth here is 2
	
	  \bigskip
	  \begingroup
	  \large
	
	    
	    \stanza[\smallbreak]
	\label{pv.3.50}\edlabel{pv.3.50}\flagstanza{\tiny\textenglish{...v.3.50}}अन्यथा वस्तु विषयीकुरुत एकेन वा धिया ।&एकत्र नान्यो विषयोस्तीति स्यात् पर्यायता ॥ ५० ॥\&[\smallbreak]


	
	  \endgroup
	

	  \pstart {\color{DodgerBlue3}“अन्यथा”} यदि शब्दनिश्चयौ सर्वात्मना वस्तु विषयीकुरुतः तदेकेन शब्देन {\color{DodgerBlue3}“बुद्ध्या वा”} विकल्पिकया व्याप्ते सर्वात्मना विषयीकृते वस्तु{\color{DodgerBlue3}“न्येकत्र नान्यो”}ऽप्रतिपन्नो {\color{DodgerBlue3}“विष”}यो{\color{DodgerBlue3}“स्ती”}ति {\color{DodgerBlue3}“पर्यायता”} शब्दानां {\color{DodgerBlue3}“स्यात्”} विकल्पानाञ्चैकविषयता भवेत् ॥ (५०)
	\pend
      \label{div_pvv.3.51}\edlabel{div_pvv.3.51}
	  
	% new div opening: depth here is 2
	

	  \pstart ननु द्रव्यादुपाधयः परस्परञ्च भिन्नास्तन्निमित्ता विकल्पाः शब्दाश्च तेषु तदाधारे वा द्रव्ये वर्त्तन्त इत्याह ।
	\pend
      
	  \bigskip
	  \begingroup
	  \large
	
	    
	    \stanza[\smallbreak]
	\label{pv.3.51}\edlabel{pv.3.51}\flagstanza{\tiny\textenglish{...v.3.51}}नानोपाधिविशिष्टस्य भेदिनोर्थस्य ग्राहिका ।&उपकाराङ्गशक्तिभ्योऽभिन्नात्मनिश्चयग्रहे ॥ ५१ ॥\&[\smallbreak]


	
	  \endgroup
	

	  \pstart {\color{DodgerBlue3}“यस्यापि नै या यि का दे”}\edlabel{pvv.306-3}\footnote{\label{pvv.306-3}  ३ वैशेषिकादेः भिन्नं पारमार्थिकं धर्मधर्मित्वं निन्दति ।}र्मते {\color{DodgerBlue3}“नानोपाधे”}र्द्रव्यत्वाद्यनेकधर्म{\color{DodgerBlue3}“विशिष्ठ”}\edlabel{pvv.306-4}\footnote{\label{pvv.306-4}  ४ घटादेः ।} स्यात \leavevmode\marginnote{\textenglish{61a/MA}} एवोपाघि\edlabel{pvv.306-5}\footnote{\label{pvv.306-5}  ५ उपाधिषु वा शब्दधियौ वर्त्तेते इति न पर्यायताप्रसङ्गः ।} भेदाद् {\color{DodgerBlue3}“भेदिनोर्थस्य\edlabel{pvv.306-6}\footnote{\label{pvv.306-6}  ६ नानाभेदयोगात् स्वयमभिन्नस्य ।} ग्राहिका”} विधिरूपेण\edlabel{pvv.306-7}\footnote{\label{pvv.306-7}  ७ प्रत्युपाधिभिन्ना ।} धीः (।) तस्यापि\edlabel{pvv.306-8}\footnote{\label{pvv.306-8}  ८ दूषणमाशङ्क्य परिहरति} मतेन नानाप्रकाराणामुपाधी{\color{DodgerBlue3}“नामुपकारस्याङ्गं\edlabel{pvv.306-9}\footnote{\label{pvv.306-9}  ९ कारणं ।}”} याः {\color{DodgerBlue3}“शक्त”}यस्ताभ्यो{\color{DodgerBlue3}“ऽभिन्नात्मन”} उपाधिमतो द्रव्यस्य {\color{DodgerBlue3}“निश्चय”}ज्ञानेन {\color{DodgerBlue3}“ग्रहे”} ॥ (५१)
	\pend
      \leavevmode\marginnote{\textenglish{307/s}}\label{div_pvv.3.52}\edlabel{div_pvv.3.52}
	  
	% new div opening: depth here is 2
	
	  \bigskip
	  \begingroup
	  \large
	
	    
	    \stanza[\smallbreak]
	\label{pv.3.52}\edlabel{pv.3.52}\flagstanza{\tiny\textenglish{...v.3.52}}सर्वात्मना कृते ग्राहे को भेदः स्यादनिश्चितः ।&तयोरात्मनि सम्बन्धादेकज्ञाने द्वयग्रहः ॥ ५२ ॥\&[\smallbreak]


	
	  \endgroup
	

	  \pstart {\color{DodgerBlue3}“सर्व्वात्मना”} कृते सत्युपकार्यस्योपाधिकलापस्य मध्ये को {\color{DodgerBlue3}“भेद”} उपाधिविशेषः {\color{DodgerBlue3}“स्यादनिश्चितः ।”} यथा द्रव्यमेकमुपाधिमुपकरोति । तथा परानपि (।) तत एकोपाध्युपकारकत्वे विधिरूपेण गृह्यमाणे\edlabel{pvv.307-1}\footnote{\label{pvv.307-1}  १ सति सम्बन्धाद् द्वयग्रह उपाध्युपाधिमतोः} सर्व्वोपाध्युपकारकत्वं गृह्येत । तथा च सर्व्वोपाधिग्रहणप्रसङ्गः । न हि यत्सापेक्षं यद्रूपं तद\edlabel{pvv.307-2}\footnote{\label{pvv.307-2}  २ स्वाग्रहणे स्वाम्यग्रहवत् ।}ग्रहणे तद् गृहीतुं शक्यं ॥
	\pend
      

	  \pstart ननु धूमापेक्षं वह्नेः कारणत्वं न च तद्ग्रहे धूमग्रहः । नैतद् युक्तं (।) तथाहि (।)
	\pend
      

	  \pstart किं पुनः कारणत्वमिष्टं वह्नेर्यद्यविद्यमानत्वाद् धूमात् पूर्व्वकालभाविता । धूमासाहित्यं वर्त्तमानकालसत्ता । न तद् धूमापेक्षं स्वकारणात् तथोत्पत्तेः । धूममन्तरेणैव च भावात् । न च तद्ग्रहे धूमग्रहो विरुद्धत्वात् । न हि धूमग्रहणे तस्मात् प्राग्भावित्वस्य तदसाहित्यस्य च ग्रहणं । अथ धूमस्यायं हेतुरिति निश्चयविषयत्वं कारणत्वमिष्टन्तदाऽस्त्येवेदृशकारणत्वस्य ग्रहे धूमप्रतीतिरुभयप्रतीत्याकारत्वादस्य निश्चयस्य द्वयान्वयव्यतिरेकग्रहणसापेक्षत्वाच्च ॥
	\pend
      

	  \pstart ननु च गृहीततदुत्पत्तेरग्निमात्रदर्शनात् कारणत्वप्रतीतिर्न च तदा {\color{DodgerBlue3}“धूमप्रतीतिः”} ॥
	\pend
      

	  \pstart (सिद्धान्ती ।) न च तत्र प्रत्यक्षात् कारणत्वस्य प्रतीतिः किन्त्वग्निरू[8]पमात्रस्य (।) अन्यथा सर्व्वस्य तद्दर्शिनः कारणत्वप्रतीतिप्रसङ्गः ।\edlabel{pvv.307-3}\footnote{\label{pvv.307-3}  ३ कथन्तर्हि तत्प्रतीतिरित्याह ।} किन्तु व्याप्तिग्रहसापेक्षा आनुमानिकी तत्प्रतीतिः । तथा हि गृहीततदुत्पत्तेरेवेयं भवति नान्यस्य । तदुत्पत्तिग्रहे च यदेवं रूपं तदेतत्कारणमिति गृहीता व्याप्तिरनुमानप्रतीतौ च धूमकारणं वह्निरिति व्यवच्छेदद्वयमन्योन्यसापेक्षं प्रतीयत एव इत्यलं तत्प्रसङ्गेन । न ह्यनुत्खातितकार्यः कारणत्वनिश्चयः । अत्यन्तमभ्यासाच्च व्याप्तिग्रहसंस्कारस्य प्रबुद्धत्वादपेक्षा नास्तीत्यतत्वविवेकिनां नानुमानत्वप्रबोधः । न ह्यनुमानमहमित्युत्पन्नमनुमानमुच्यते । किन्तु त्रिरूपलिङ्गजमनुमेयविषयं कारणत्वज्ञानञ्च तादृशमेव । द्रव्यन्तु येनैव रूपेण एकमुपाधिमुपकरोति तेनैवापरानपीत्येकोपोध्युपकारकत्वे गृह्यमाणे नानेकोपाध्युपकारकत्वस्य ग्रहणात् सर्व्वोपाधिग्रहणप्रसङ्गः परस्परसापेक्षत्वात् । ततश्च {\color{DodgerBlue3}“तयो”}रुपाधिकलापतदुपकारकत्वयो{\color{DodgerBlue3}“रात्मनि सम्बन्धात्”} । \leavevmode\marginnote{\textenglish{308/s}} द्रव्यस्य सम्बन्धा{\color{DodgerBlue3}“देक”}स्योपाध्युपकारकत्वस्य {\color{DodgerBlue3}“ज्ञाने द्वयस्य ग्रहः”} स्यात् । एकनियता प्रतीतिर्न भवेदित्यर्थः ॥ (५२)
	\pend
      \label{div_pvv.3.53}\edlabel{div_pvv.3.53}
	  
	% new div opening: depth here is 2
	

	  \pstart अथ द्रव्यादुपाध्युपकारिकाः शक्तयो भिन्ना एव द्रव्यग्रहणे तासामग्रहणात् न सर्व्वोपाधिग्रहणप्रसङ्ग इत्याह ।
	\pend
      
	  \bigskip
	  \begingroup
	  \large
	
	    
	    \stanza[\smallbreak]
	\label{pv.3.53}\edlabel{pv.3.53}\flagstanza{\tiny\textenglish{...v.3.53}}धर्मोपकारशक्तीनां भेदे ताः तस्य किं यदि ।&नोपकारस्ततः तासां तथा स्यादनवस्थितिः ॥ ५३ ॥\&[\smallbreak]


	
	  \endgroup
	

	  \pstart {\color{DodgerBlue3}“धर्मा”}णामुपाधीना {\color{DodgerBlue3}“उपकार”}स्य निमित्तभूतानां {\color{DodgerBlue3}“शक्ती”}नां द्रव्याद् {\color{DodgerBlue3}“भेदे”} स्वीक्रियमाणे {\color{DodgerBlue3}“ताः”} शक्तय{\color{DodgerBlue3}“स्तस्य”} द्रव्यस्य {\color{DodgerBlue3}“किं”} कस्मात् (।) यदि {\color{DodgerBlue3}“नोपकारस्ततो”} द्रव्यात् {\color{DodgerBlue3}“तासां”} शक्तीनामुपकारमन्तरेण सम्बन्धेऽतिप्रसङ्गात् (।) अथ तासां शक्तीनां द्रव्येणोपकारः क्रियते स्वरूपेण तदा शक्त्युपकारत्वस्य ग्रहणाच्छक्तीनां ग्रहः । तद्ग्रहाच्च सर्व्वोपाधिग्रहप्रसङ्गः तदवस्थः ।
	\pend
      \leavevmode\marginnote{\textenglish{61b/MA}}

	  \pstart स्यादेतत् (।) शक्तीरपि भिन्नाभिःशक्तिभिरुपकरोति {\color{DodgerBlue3}“तथा स्यादनवस्थितिः”} । तथा हि यास्ताः शक्त्युपकारिकाः शक्त्यपस्ता द्रव्यस्योपकारद्वारा\edlabel{pvv.308-1}\footnote{\label{pvv.308-1}  १ यदोपाधिष्वेव शब्दादिवृत्तिस्तदा नोक्तो दोषः किन्तु शब्दाद्यैरनाक्षेपादुपाधिमति प्रवृत्तिर्न स्यादिति व्यर्थः शब्दप्रयोगः । अर्थक्रियाश्रयो हि व्यवहार उपाध (य) श्चात्र व्यवहारेऽसमर्थाः (।) समर्थश्चोपाधिमान्नोच्यते (।) किञ्च (।) निश्चयगृहीतेप्यर्थे भ्रान्तिनिवृत्यर्थं प्रमाणान्तरमिच्छता निश्चयविषयश्च न च निश्चित इत्यभ्युपेतं स्यात् अन्यथा भ्रान्त्ययोगात् (।) तच्चायुक्तमित्याह ।}द् (?) यदीष्टास्तदास्वरूपेणोपकारकत्वे सर्व्वोपाधिग्रहप्रसङ्गभयादन्याः शक्तय एष्टव्याः । तथा चानवस्था व्यक्ता । (५३)
	\pend
      \label{div_pvv.3.54}\edlabel{div_pvv.3.54}
	  
	% new div opening: depth here is 2
	

	  \pstart उक्तमर्थं संगृह्णन्नाह ।
	\pend
      
	  \bigskip
	  \begingroup
	  \large
	
	    
	    \stanza[\smallbreak]
	\label{pv.3.54}\edlabel{pv.3.54}\flagstanza{\tiny\textenglish{...v.3.54}}एकोपकारके ग्राह्येऽदृष्टाः तस्मिन्न सन्ति ते ।&सर्वोपकारकं ह्येकं तद्ग्रहे सकलग्रहः ॥ ५४ ॥\&[\smallbreak]


	
	  \endgroup
	

	  \pstart {\color{DodgerBlue3}“एक”}स्योपाधे{\color{DodgerBlue3}“रुपकारके”} द्रव्ये {\color{DodgerBlue3}“ग्राह्ये”}ऽभिमते तत एकोपाध्युपकारकद्रव्यस्वभावादपरे उपाध्यन्तराणामुपकारका उपकारशक्तिभेदा दृष्टे {\color{DodgerBlue3}“तस्मिन्ने”}कोपाध्युपकारके द्रव्येऽदृष्टा ये {\color{DodgerBlue3}“ते न सन्ति”} (।) एकमेव हि रूपं सर्व्वोपाध्युपकारकमतस्त{\color{DodgerBlue3}“स्यैको”}पाधिमतो {\color{DodgerBlue3}“ग्रहे सकलोपाधिग्रहः”} स्यात् ॥ (५४)
	\pend
      \leavevmode\marginnote{\textenglish{309/s}}\label{div_pvv.3.55}\edlabel{div_pvv.3.55}
	  
	% new div opening: depth here is 2
	

	  \begin{center}%% label @type='head'
	\textbf{क. न्यायमीमांसामतनिरासः}
	\end{center}
	

	  \begin{center}%% label @type='head'
	\textbf{(क) व्यावृत्तस्वभावा भावाः}
	\end{center}
	
	  \bigskip
	  \begingroup
	  \large
	
	    
	    \stanza[\smallbreak]
	\label{pv.3.55}\edlabel{pv.3.55}\flagstanza{\tiny\textenglish{...v.3.55}}यदि भ्रान्तिनिवृत्त्यर्थं गृहीतेप्यन्यदिष्यते ।&तद्व्यवच्छेदविषयं सिद्धन्तद्वत् ततोऽपरम् ॥ ५५ ॥\&[\smallbreak]


	
	  \endgroup
	

	  \pstart सर्व्वात्मना विकल्पेन {\color{DodgerBlue3}“गृहीतेपि”} वस्तुनि भ्रान्त्या तथा न निश्चय इति {\color{DodgerBlue3}“यदि भ्रान्तिनिवृत्त्यर्थमन्यत्”} प्रमाणान्तर{\color{DodgerBlue3}“मिष्यते”} (।) तद्भ्रान्तिनिवर्त्तकं प्रमाणान्तरं {\color{DodgerBlue3}“व्यवच्छेदविषयं सिद्धं”}\edlabel{pvv.309-1}\footnote{\label{pvv.309-1}  १ उत्पित्सुसमारोपनिषेधद्वारेण ।} भ्रमारोपितत्वापोहविषयत्वाद् । {\color{DodgerBlue3}“ततो”} भ्रान्तिनिवर्त्तकाद{\color{DodgerBlue3}“परं”} यत् पूर्व्वमुत्पन्नम्वस्तुविषयमिष्टं {\color{DodgerBlue3}“तद्व”}दपोहविषयं सिद्धं ॥ (५५)
	\pend
      \label{div_pvv.3.56}\edlabel{div_pvv.3.56}
	  
	% new div opening: depth here is 2
	

	  \pstart कस्मादित्याह (।)
	\pend
      
	  \bigskip
	  \begingroup
	  \large
	
	    
	    \stanza[\smallbreak]
	\label{pv.3.56}\edlabel{pv.3.56}\flagstanza{\tiny\textenglish{...v.3.56}}तद्विपक्षसमारोपविषये यदि निश्चयैः ।&निश्चीयते न यद् रूपं तत्तेषां विषयः कथम् ॥ ५६ ॥\&[\smallbreak]


	
	  \endgroup
	

	  \pstart अविद्यमानान्यसमारोपे {\color{DodgerBlue3}“विषये वृत्तेः”} । विकल्पो हि व्यवच्छेदविषयं निश्चिन्वन् तद्विपक्षसमारोपविषये न भवति । किन्तु तद्व्यवच्छेदनिश्चयारोपमनसोर्ब्बाध्यबाधकभावत इत्युक्तं ।
	\pend
      

	  \pstart अपि\edlabel{pvv.309-2}\footnote{\label{pvv.309-2}  २ किञ्च ।} च निश्चयैरेकाकारप्रवृत्तै\edlabel{pvv.309-3}\footnote{\label{pvv.309-3}  ३ यदेषां स्वविषयनिश्चयनमेव स्वार्थे वृत्तिः ।}{\color{DodgerBlue3}“र्यद् रूपं न निश्चीयते तत्तेषाम्विषयः क”}थमुच्यते । निश्चितञ्चाप्रतिपन्नं चेति विप्रति{\color{DodgerBlue3}“सि”} (? षि) द्धं । यदि कृतकत्वनिश्चयेऽनित्यत्वाद्यपि निश्चितं कथं तस्याप्रतिपत्तिः । कृत (क) त्वस्यापि वा माभूत् ॥ (५६)
	\pend
      \label{div_pvv.3.57}\edlabel{div_pvv.3.57}
	  
	% new div opening: depth here is 2
	

	  \pstart एवन्तर्हि प्रत्यक्षगृहीते वस्तुनि निश्चयानिश्चयौ न स्यातामित्याह ।
	\pend
      
	  \bigskip
	  \begingroup
	  \large
	
	    
	    \stanza[\smallbreak]
	\label{pv.3.57}\edlabel{pv.3.57}\flagstanza{\tiny\textenglish{...v.3.57}}प्रत्यक्षेण गृहीतेपि विशेषेंशविवर्जिते ।&यद्विशेषावसायेस्ति प्रत्ययः स प्रतीयते ॥ ५७ ॥\&[\smallbreak]


	
	  \endgroup
	

	  \pstart {\color{DodgerBlue3}“प्रत्यक्षेण गृहीतेपि विशेषे”} स्वलक्षणे सर्व्वतो व्यावृत्तें{\color{DodgerBlue3}“ऽशै”}र्भागै{\color{DodgerBlue3}“र्व्विवर्जिते”} यस्य {\color{DodgerBlue3}“विशेष”}स्य व्यवच्छेदस्या{\color{DodgerBlue3}“वसाये”}स्ति {\color{DodgerBlue3}“प्रत्ययः”} सहकारी प्रकरणाभ्यासपाटवादिः {\color{DodgerBlue3}“स”} \leavevmode\marginnote{\textenglish{310/s}} {\color{DodgerBlue3}“प्रतीयते”} नेतरः । न खल्वस्मन्मते प्रत्यक्षं निश्च\edlabel{pvv.310-1}\footnote{\label{pvv.310-1}  १ कथमिदानीमनिश्चीयमानं प्रत्यक्षेणापि ग्रहीतमिति चेन्न प्रत्यक्षं निश्चयेन गृह्णाति किन्तु तत्प्रतिभासेन (।) शब्दप्रतिपत्योर्भेदस्तु संकेतभेदान्न तत्त्वतो वाच्यभेदः । संकेतभेदश्चानन्तरं भेदान्तरेत्याद्युक्तः ।}यात्मकं नाप्यनुभवमात्राधीनो निश्चयो\edlabel{pvv.310-2}\footnote{\label{pvv.310-2}  २ अनुभवो हि पटीयान् स्मृतिबीजमाधत्ते निश्चयः स्मृतिरूपः ।} येन सर्व्वथा निश्चयप्रसङ्गः । किन्तु यत्र व्यवच्छेदे\edlabel{pvv.310-3}\footnote{\label{pvv.310-3}  ३ यथोपाध्याये पितर्यागच्छति पिता मे आगच्छतीति तत्रापि तारतम्यात् पूर्व्वपरविकल्पजननशक्तिः ।}ऽभ्यासादयः\edlabel{pvv.310-4}\footnote{\label{pvv.310-4}  ४ अथ गोरश्वादव्यावृत्तिर्व्यावृत्तो गौरिति वा यदा क्रियते तदा व्यावृत्तितद्वतोर्व्वस्तुतो भेदे सामान्यं नामान्तरेणोक्तं स्यात् व्यावृत्तिरिति । अभेदेपि ज्ञानशब्दयोर्भेदो (स्वलक्षणात्) न स्यादित्याह (।) एकमानेन सर्वसिद्धौ मानान्तरादिवैयर्थ्यं तदवस्थं ।}सहकारिणः प्रत्यक्षस्य सन्ति स निश्चीयते नान्य इति युक्तो विभागः । (५७)
	\pend
      \label{div_pvv.3.58}\edlabel{div_pvv.3.58}
	  
	% new div opening: depth here is 2
	
	  \bigskip
	  \begingroup
	  \large
	
	    
	    \stanza[\smallbreak]
	\label{pv.3.58}\edlabel{pv.3.58}\flagstanza{\tiny\textenglish{...v.3.58}}तत्रापि चान्यव्यावृत्तिरन्यव्यावृत्त इत्यपि ।&शब्दाश्च निश्चयाश्चैव निमित्तमनुरुन्धते ॥ ५८ ॥\&[\smallbreak]


	
	  \endgroup
	

	  \pstart {\color{DodgerBlue3}“तत्रापि”} चान्यापोहेपि शब्दविकल्पविषये{\color{DodgerBlue3}“ऽन्यव्यावृत्तिरन्यव्यावृत्त इत्यपि (।) ये शब्दाश्च निश्चयाश्चैव”} भिन्नविषया इव प्रवर्तन्ते न ते व्यावृत्तिव्यावृत्तयोर्व्वास्तवभेदनिबन्धनाः किन्तर्हि संकेतमेव भेदव्यवहारव्यस्थापकं वक्ष्यमाण{\color{DodgerBlue3}“निमित्तमनुरुन्धते”}ऽनुवर्त्तन्ते । यदि गोरन्याऽगोव्यावृत्तिस्तदा यथा गौरगोरश्वादेर्व्यावृत्तस्तथाऽगोव्यावृत्तेरपि व्यावृत्तः । ततश्चाश्वादिवद् गोरगौरेव स्यात् । यो ह्यगोव्यावृत्तेर्व्यावृत्तः सोऽगौर्यथाश्वादिर्गौश्च तथेति स्यात् । तथाऽगोव्यावृत्ति[6]रपि न प्राप्नोति । सर्व्वस्यैवागोत्वात् । गोरगोरन्यत्वे हि तस्माद् व्यावृत्तिः स्यात् । यदा तु गौरेव नास्ति तदा कस्य कस्माद् व्यावृत्तिः । तस्मान्न व्यावृत्तिव्यावृत्तयोर्व्वस्तुतो भेदः । (५८)
	\pend
      \label{div_pvv.3.59}\edlabel{div_pvv.3.59}
	  
	% new div opening: depth here is 2
	

	  \pstart ततश्च ॥
	\pend
      
	  \bigskip
	  \begingroup
	  \large
	
	    
	    \stanza[\smallbreak]
	\label{pv.3.59}\edlabel{pv.3.59}\flagstanza{\tiny\textenglish{...v.3.59}}द्वयोरेकाभिधानेपि विभक्तिर्व्यतिरेकिणी ।&भिन्नमर्थमिवान्वेति वाच्यते स विशेषतः ॥ ५९ ॥\&[\smallbreak]


	
	  \endgroup
	

	  \pstart {\color{DodgerBlue3}“द्वयो”}र्व्यावृत्तिव्यावृत्तयोरेकस्यार्थस्या{\color{DodgerBlue3}“भिधानेपि”} वस्तुतो {\color{DodgerBlue3}“विभक्तिः”} षष्ठ्यादि{\color{DodgerBlue3}“र्व्यतिरेको”} वाच्यभेदः तद्वती । व्यावृत्तस्य व्यावृत्तिरिति {\color{DodgerBlue3}“भिन्नमिवार्थमन्वे”}त्यनु\leavevmode\marginnote{\textenglish{311/s}} गच्छति वाचकत्वेन । वाच्यस्य लेशविशेषतो\edlabel{pvv.311-1}\footnote{\label{pvv.311-1}  १ शब्दा हीच्छाधीना न वस्त्वधीनाः ते यथा प्रयोक्तुमिष्यन्ते भेदेऽभेदे वा तथा वाचकाः (।) यथा राज्ञः पुरुष आत्मात्मनो द्रष्ट्रिति ।}ऽल्पभेदात् साङ्केतिकादभिन्ने-\leavevmode\marginnote{\textenglish{62a/MA}} प्यर्थे ॥ (५९)
	\pend
      \label{div_pvv.3.60}\edlabel{div_pvv.3.60}
	  
	% new div opening: depth here is 2
	

	  \pstart किमर्थं संकेतभेदः कश्च वाच्यविशेष इत्याह ।
	\pend
      
	  \bigskip
	  \begingroup
	  \large
	
	    
	    \stanza[\smallbreak]
	\label{pv.3.60}\edlabel{pv.3.60}\flagstanza{\tiny\textenglish{...v.3.60}}भेदान्तरप्रतिक्षेपाप्रतिक्षेपौ तयोर्द्वयोः ।&पदं सङ्केतभेदस्य ज्ञातृवाञ्छानुरोधिनः ॥ ६० ॥\&[\smallbreak]


	
	  \endgroup
	

	  \pstart {\color{DodgerBlue3}“भेदान्त”}रस्य\edlabel{pvv.311-2}\footnote{\label{pvv.311-2}  २ गोत्वापेक्षया द्रव्यत्वपार्थिवत्वादेः ।} प्रतिपाद्यमानाद् व्यवच्छेदादन्यस्य व्यवच्छेदस्य {\color{DodgerBlue3}“प्रतिक्षेपः\edlabel{pvv.311-3}\footnote{\label{pvv.311-3}  ३ अस्वीकारः ।}”} । सामानाधिकरण्यस्य स्वभावोऽ{\color{DodgerBlue3}“प्रतिक्षेप”}स्तत्सम्भवः । {\color{DodgerBlue3}“तौ\edlabel{pvv.311-4}\footnote{\label{pvv.311-4}  ४ यथाक्रमं ।} तयोर्द्वयो”}र्व्यावृत्तिव्यावृत्त\edlabel{pvv.311-5}\footnote{\label{pvv.311-5}  ५ धर्मधर्मिवाचिनोः ।}शब्दयोः\edlabel{pvv.311-6}\footnote{\label{pvv.311-6}  ६ प्रयोजनं ।} {\color{DodgerBlue3}“संकेतभेदस्य ज्ञातृवाञ्छानुरोधिनः”} पदं कारणं (।) ज्ञाता\edlabel{pvv.311-7}\footnote{\label{pvv.311-7}  ७ श्रोता ।} हि कदाचित् गौरनश्वत्वं निष्कृष्टधर्मान्तर\edlabel{pvv.311-8}\footnote{\label{pvv.311-8}  ८ (अमहिषत्वादि) किमस्याश्वाद् व्यावृतं रूपमस्तीति ।} सम्बन्धयोग्यत्वं जिज्ञासते\edlabel{pvv.311-9}\footnote{\label{pvv.311-9}  ९ न सामानाधिकरण्यम्बिशेषणविशेष्यभावो वा त्यक्त्वा भेदान्तरत्वादेव गोत्वशुक्लत्वाभ्यां युक्तमेकं धर्मिणं गृहीत्वा बुद्धेरप्रतिभासनात् ।} (।) तदा गोत्वमस्येत्युच्यते न तु गोत्वमस्य शुक्लमिति । यदा तु व्यावृत्ति\edlabel{pvv.311-10}\footnote{\label{pvv.311-10}  १० कुतोस्य व्यावृत्तिरिति तदा भेदान्तराक्षेपात् तत्साकांक्षत्वाच्च ।}रेवानिष्कृष्टधर्मान्तरसम्बन्धयोग्या जिज्ञासिता भवति (।) तदा व्यावृत्तशब्दसंकेतो यथा गौरयमिति धर्म्मान्तरसामानाधिकरण्यञ्च गौः\edlabel{pvv.311-11}\footnote{\label{pvv.311-11}  ११ अनेकधर्मवन्तं धर्मिणमेकमिव दर्शयन्ती बुद्धिरत्र यस्मात् ।}शुक्ल इत्यादि\edlabel{pvv.311-12}\footnote{\label{pvv.311-12}  १२ अयं नीलादि ।}॥ (६०)
	\pend
      \label{div_pvv.3.61}\edlabel{div_pvv.3.61}
	  
	% new div opening: depth here is 2
	
	  \bigskip
	  \begingroup
	  \large
	
	    
	    \stanza[\smallbreak]
	\label{pv.3.61}\edlabel{pv.3.61}\flagstanza{\tiny\textenglish{...v.3.61}}भेदोयमेव सर्वत्र द्रव्यभावाभिधायिनोः ।&शब्दयोर्न तयोर्वाच्ये विशेषस्तेन कश्चन ॥ ६१ ॥\&[\smallbreak]


	
	  \endgroup
	

	  \pstart {\color{DodgerBlue3}“भेदोयमेव”} संकेतकृतो धर्मान्तर\edlabel{pvv.311-13}\footnote{\label{pvv.311-13}  १३ नापरं सामान्यगुणादिकं बाधितत्वात् ।}प्रतिक्षेपाप्रतिक्षेपप्रतिपत्तिफलो {\color{DodgerBlue3}“द्रव्य\edlabel{pvv.311-14}\footnote{\label{pvv.311-14}  १४ सर्व्वत्र सामान्यतद्वति गुणतद्वति क्रियातद्वति ।}भावाभिधायिनो”}र्द्धर्मिधर्मवाचिनोः {\color{DodgerBlue3}“शब्दयोर्न”} वास्तवः । {\color{DodgerBlue3}“तेन तयोर्व्वाच्ये”} निश्चयविषये {\color{DodgerBlue3}“न क”}श्चिद् {\color{DodgerBlue3}“विशेषः”} । भेदान्तरप्रतिक्षेपाप्रतिक्षेपाभ्यामेकस्यैव प्रत्यायनात् ॥ (६१)
	\pend
      \leavevmode\marginnote{\textenglish{312/s}}\label{div_pvv.3.62}\edlabel{div_pvv.3.62}
	  
	% new div opening: depth here is 2
	
	  \bigskip
	  \begingroup
	  \large
	
	    
	    \stanza[\smallbreak]
	\label{pv.3.62}\edlabel{pv.3.62}\flagstanza{\tiny\textenglish{...v.3.62}}जिज्ञापयिषुरर्थं तं तद्धितेन कृतापि वा ।&अन्येन वा यदि ब्रूयात् भेदो नास्ति ततः परः ॥ ६२ ॥\&[\smallbreak]


	
	  \endgroup
	

	  \pstart तथा व्यवहर्त्ता {\color{DodgerBlue3}“जिज्ञापयिषुरर्थं तं”} साङ्केतिकं भेदं {\color{DodgerBlue3}“तद्धितेन कृतापि वा\edlabel{pvv.312-1}\footnote{\label{pvv.312-1}  १ जातिगुणक्रियासम्बन्धैश्चतुष्टयी वृत्तिरुक्तानेनैव ।}”} पाचकत्वमस्य पाचकोयम्पाकः पाक्यो वेत्यादि । {\color{DodgerBlue3}“अन्येन वा”} स्वयंकृतेन समयेन {\color{DodgerBlue3}“यदि ब्रूयात्”} (।) तथापि {\color{DodgerBlue3}“ततो”} भेदप्रतिपादकात् तद्धितादे{\color{DodgerBlue3}“र्भेदः”} परो वास्तवो {\color{DodgerBlue3}“नास्ति”}॥ (६२)
	\pend
      \label{div_pvv.3.63}\edlabel{div_pvv.3.63}
	  
	% new div opening: depth here is 2
	
	  \bigskip
	  \begingroup
	  \large
	
	    
	    \stanza[\smallbreak]
	\label{pv.3.63}\edlabel{pv.3.63}\flagstanza{\tiny\textenglish{...v.3.63}}तेनान्यापोहविषये तद्दोषोपवर्ण्णनम् ।\edlabel{pvv.312-asterisk}\footnote{\label{pvv.312-asterisk}  * तद्वत्पक्षोपवर्ण्णनम् ।}&प्रत्याख्यातं पृथक्त्वे हि स्याद् दोषो जातितद्वतोः ॥ ६३ ॥\&[\smallbreak]


	
	  \endgroup
	

	  \pstart {\color{DodgerBlue3}“तेन”} व्यावृत्तिव्यावृत्तयोरभेदे{\color{DodgerBlue3}“नान्यापोहविषये”} जातिमान् शब्दवाच्य इति पक्षः तद्वत् पक्षभूर्दोषोऽन्यापोहेपि स्यादिति । {\color{DodgerBlue3}“तद्दोषोपवर्ण्णनं प्रत्याख्यातं”} । तद्वत् पक्षो हि जातिमभिधाय शब्दस्तद्वति वर्त्तत इति तद्वचने\edlabel{pvv.312-2}\footnote{\label{pvv.312-2}  २ जाति ।} स्वातन्त्र्यम\edlabel{pvv.312-3}\footnote{\label{pvv.312-3}  ३ शब्दस्य ।}स्य न स्यात् सामानाधिकरण्यञ्च न भवेत् (।) गौः शुक्ल इति जातेरशुक्लत्वात् (।) न चोपचाराश्रयेण स्वातन्त्र्यं सामान्याधिकरण्यञ्चास्खलद्गतियुक्तमित्यादि {\color{DodgerBlue3}“जातितद्वतोः पृथक्त्वे हि स्याद् दोष एषः”} । न तु व्यावृत्तिव्यावृत्तिमतोर्भेद इति नात्र तत्पक्षोक्तदोषः ॥ (६३)
	\pend
      \label{div_pvv.3.64}\edlabel{div_pvv.3.64}
	  
	% new div opening: depth here is 2
	

	  \pstart यदि\edlabel{pvv.312-4}\footnote{\label{pvv.312-4}  ४ कथमिदानीमेकस्याननुगमादन्यव्यावृत्तिः सामान्यं । स्वलक्षणानुभवोत्तरमेककार्येष्वेकमाकारमादर्श (य) न्विकल्पः सामान्यं अतत्कार्येभ्यो भिद्यमाना अर्थाः सामान्यादिव्यवहारविषय इत्यन्यापोहविषयत्वं ।} व्यावृत्तितद्वतोर्न भवेस्तदा गोर्गोत्वं शुक्लत्वसास्नादिमत्वादयश्चेति षष्ठीवचनभेदादि न प्राप्नोतीत्याह ।
	\pend
      
	  \bigskip
	  \begingroup
	  \large
	
	    
	    \stanza[\smallbreak]
	\label{pv.3.64}\edlabel{pv.3.64}\flagstanza{\tiny\textenglish{...v.3.64}}येषां वस्तुवशा वाचो न विवक्षापराश्रयाः ।&षष्ठीवचनभेदादि चोद्यं तान् प्रति युक्तिमत् ॥ ६४ ॥\&[\smallbreak]


	
	  \endgroup
	

	  \pstart {\color{DodgerBlue3}“येषां”} बाह्यानां मते {\color{DodgerBlue3}“वस्तुवशा”} वस्त्वायत्ता {\color{DodgerBlue3}“वाचो न विवक्षापर आश्रयः”} कारणं यासां तास्तथा । {\color{DodgerBlue3}“षष्ठीवचनभेदादि चोद्यं तान् प्रति युक्तिमत्”} ॥ (६४)
	\pend
      \label{div_pvv.3.65}\edlabel{div_pvv.3.65}
	  
	% new div opening: depth here is 2
	
	  \bigskip
	  \begingroup
	  \large
	
	    
	    \stanza[\smallbreak]
	\label{pv.3.65}\edlabel{pv.3.65}\flagstanza{\tiny\textenglish{...v.3.65}}यद् यथा वाचकत्वेन वक्‏तृभिर्विनियम्यते ।&अनपेक्षितबाह्यार्थं तत् तथा वाचकं वचः ॥ ६५ ॥\&[\smallbreak]


	
	  \endgroup
	\leavevmode\marginnote{\textenglish{313/s}}

	  \pstart अस्माकं यद्वचो {\color{DodgerBlue3}“यथा”} धर्मान्तरस्य प्रतिक्षेपेणाप्रति{\color{DodgerBlue3}“वाचकत्वेन वक्तृभिर्नियम्यते”} विशेषेण व्यवस्थाप्यते{\color{DodgerBlue3}“ऽनपेक्षितवाह्यार्थ”} संकेतमात्रानुरोधित्वात् {\color{DodgerBlue3}“तद्वचस्तथा वाचकमि”}ष्टमिति न चोद्यावकाशः ॥ (६५)
	\pend
      \label{div_pvv.3.66}\edlabel{div_pvv.3.66}
	  
	% new div opening: depth here is 2
	
	  \bigskip
	  \begingroup
	  \large
	
	    
	    \stanza[\smallbreak]
	\label{pv.3.66}\edlabel{pv.3.66}\flagstanza{\tiny\textenglish{...v.3.66}}दाराः षण्णगरीत्यादौ भेदाभेदव्यवस्थितेः ।&खस्य स्वभावः खत्वं वेत्यत्र वा किं निबन्धनम् ॥ ६६ ॥\&[\smallbreak]


	
	  \endgroup
	

	  \pstart यस्य तु वास्तव एव शाब्दो व्यवहारस्तस्य {\color{DodgerBlue3}“दाराः षण्णगरी\edlabel{pvv.313-1}\footnote{\label{pvv.313-1}  १ क्रियातो गुणतो वा समाहारो द्रव्याश्रितः । नगरन्तु विजातीयानारम्भादद्रव्यं ।}त्यादा”}वादिशब्दात् गृहा विंशतिरित्यादौ च यथाक्रममभिन्ने भिन्ने च वस्तुतो {\color{DodgerBlue3}“भेदाभेदयो”}र्बहुवचनैकवचननिमित्तयो{\color{DodgerBlue3}“र्व्यवस्थितेः । खस्य स्वभावः खत्वं वेत्यत्र”} धर्मिधर्मभेदस्य {\color{DodgerBlue3}“किम्वा”} निमित्तं न किञ्चन । न ह्येकस्या (:) स्त्रिया बहुत्वं षण्णां नगराणां वा एकत्वमाकाशस्य स्वभावो भि\edlabel{pvv.313-2}\footnote{\label{pvv.313-2}  २ भेदान्तरादिना एतदुक्तम्भवति । अतश्वत्वामहिषत्वादिषु भेदान्तरेष्वेकपिण्डनिष्ठेषु सत्स्वप्यनपेक्षितभेदान्तरमश्वव्यवच्छेदलक्षणमनश्वत्वमात्रमनश्वत्वशब्दस्य धर्मवचनस्य संकेतकारणं । अत्यक्तभेदान्तरन्तु तदेवानश्वत्वं अनश्व इति धर्मिवचनस्य संकेतभेदे कारणं (।) एतच्च पूर्व्वश्लोकावतारणेन ज्ञेयं ।}न्नः सामान्यं वास्ति । अथ चास्ति शब्दवृत्तिस्ततः कल्पित एव तद्विषयो वक्तव्यः ॥ (६६)
	\pend
      \label{div_pvv.3.67}\edlabel{div_pvv.3.67}
	  
	% new div opening: depth here is 2
	

	  \pstart यदि सर्वतो व्यावृत्तस्वभावा भावा न तेषु सामान्यमस्ति कथं गोत्वमित्यादिसामान्यप्रतीतिरित्याह ।
	\pend
      
	  \bigskip
	  \begingroup
	  \large
	
	    
	    \stanza[\smallbreak]
	\label{pv.3.67}\edlabel{pv.3.67}\flagstanza{\tiny\textenglish{...v.3.67}}एकार्थप्रतिभासिन्या भावानाश्रित्य भेदिनः ।&रूपं परेषां व्यावृत्तं सा धीः संवृतिरुच्यते ॥ ६७ ॥\&[\smallbreak]


	
	  \endgroup
	

	  \pstart {\color{DodgerBlue3}“भेदिनः”} सर्व्वतो व्यावृत्तान् {\color{DodgerBlue3}“भावानाश्रित्य”} परम्परापातेभ्य\edlabel{pvv.313-3}\footnote{\label{pvv.313-3}  ३ तदन्यव्यतिरेकिणः पदार्थानाश्रित्य ।} उत्पद्य {\color{DodgerBlue3}“यया”} धिया {\color{DodgerBlue3}“एकार्थप्रतिभासिन्या”} एकार्थाध्यवसायस्वाकारया स्वरूपेण स्वप्रतिभासेन {\color{DodgerBlue3}“परेषां”} स्वलक्षणानां {\color{DodgerBlue3}“रूपं”} सर्व्वतो व्यावृत्तं\edlabel{pvv.313-4}\footnote{\label{pvv.313-4}  ४ तदेवं समारोपपक्षे परोक्तं दूषणं परिहृत्यान्यव्यावृत्तिपक्षे तानभ्युपगमादेवं व्यावृत्ताभावा एकत्वेनाध्यवसिताः सामान्यमित्युक्त्वा बुद्ध्याकारे सामान्ये परदूषणमपनयति । तत्तु बुद्ध्याकारश्च बुद्धिस्थो नार्थबुद्ध्यन्तरानुगः । नाभिप्रेतार्थकारी च सोपि वाच्यो न तत्वतः ॥ अनुप्रवेशे सामान्यं न स्यादित्यव्यतिरिक्तदूषणं ।} संव्रियते प्रच्छाद्यते {\color{DodgerBlue3}“सा\edlabel{pvv.313-5}\footnote{\label{pvv.313-5}  ५ प्रकृत्या एकाकारपरामर्शहेतून् भावानाश्रित्य विकल्पबुद्धिरेकाकारोत्पद्यमाना यं एकमाकारं भावेष्वर्प्ययति स एव बुद्ध्याकारः शब्दप्रवृत्यङगः सामान्यं सिद्धान्तिनापि बीजमस्य वाच्यमनाश्रयस्यानुत्पत्तेरित्याह ।} बुद्धिः”}\leavevmode\marginnote{\textenglish{62b/MA}} {\color{DodgerBlue3}“संवृतिरुच्यते”} ॥ (६७)
	\pend
      \leavevmode\marginnote{\textenglish{314/s}}\label{div_pvv.3.68}\edlabel{div_pvv.3.68}
	  
	% new div opening: depth here is 2
	
	  \bigskip
	  \begingroup
	  \large
	
	    
	    \stanza[\smallbreak]
	\label{pv.3.68}\edlabel{pv.3.68}\flagstanza{\tiny\textenglish{...v.3.68}}तया संवृतनानात्वाः संवृत्या भेदिनः स्वयम् ।&अभेदिन इवाभान्ति भेदा\edlabel{pvv.314-1}\footnote{\label{pvv.314-1}  १ PH. भावाः}रूपेण केनचित् ॥ ६८ ॥\&[\smallbreak]


	
	  \endgroup
	

	  \pstart {\color{DodgerBlue3}“तया”} संवृत्त्या {\color{DodgerBlue3}“स्वय”}मात्मना {\color{DodgerBlue3}“भेदिनः”} स्वस्वभावव्यवस्थिता भावाः {\color{DodgerBlue3}“संवृतनानात्वाः”} स्थगितभेदाः {\color{DodgerBlue3}“केनचिद् रूपेण”} विजातीयव्यावृत्त्युपकल्पितेन गोत्वादिनाऽ{\color{DodgerBlue3}“भेदिन इवाभान्ति”} ॥ (६८)
	\pend
      \label{div_pvv.3.69}\edlabel{div_pvv.3.69}
	  
	% new div opening: depth here is 2
	
	  \bigskip
	  \begingroup
	  \large
	
	    
	    \stanza[\smallbreak]
	\label{pv.3.69}\edlabel{pv.3.69}\flagstanza{\tiny\textenglish{...v.3.69}}तस्या अभिप्रायवशात् सामान्यं सत्प्रकीर्तितम् ।&यथा तयोपकल्पितं तदसत परमार्थतः ॥ ६९ ॥\&[\smallbreak]


	
	  \endgroup
	

	  \pstart {\color{DodgerBlue3}“तस्या”} बुद्धेः सामान्यरूपतयाऽध्यवसिताकारायाऽ (?{\color{DodgerBlue3}“अ”}) {\color{DodgerBlue3}“भिप्रायवशात्”} सामान्यं {\color{DodgerBlue3}“सत् प्रकीर्त्तितं”} । विजातीयव्यावृत्तेर्व्वस्तुष्वभावात् । तदुपकल्पितं गोत्वादि सामान्य रूपेण बुद्ध्याकारमध्यव{\color{DodgerBlue3}“श्य”} (? स्य) न्ति व्यवहर्त्तारः । अध्यवसायानुरोधेन च सामान्यं सदित्युच्यते (।) {\color{DodgerBlue3}“यथा”} वस्तुत्वेन {\color{DodgerBlue3}“तत्”} सामान्यं {\color{DodgerBlue3}“तया”} संवृत्तिबुद्ध्या कल्पितं {\color{DodgerBlue3}“तथाऽसत् परमार्थतः”} ॥ (६९)
	\pend
      \label{div_pvv.3.70}\edlabel{div_pvv.3.70}
	  
	% new div opening: depth here is 2
	

	  \pstart तदेवासत्त्वमाह ।
	\pend
      
	  \bigskip
	  \begingroup
	  \large
	
	    
	    \stanza[\smallbreak]
	\label{pv.3.70}\edlabel{pv.3.70}\flagstanza{\tiny\textenglish{...v.3.70}}व्यक्तयो नानुयन्त्यन्यदनुयायि न भासते ।&ज्ञानादव्यतिरिक्तं वा कथमर्थान्तरं व्रजेत् ॥ ७० ॥\&[\smallbreak]


	
	  \endgroup
	

	  \pstart {\color{DodgerBlue3}“व्यक्तय”}स्ताव{\color{DodgerBlue3}“न्न”} परस्पर{\color{DodgerBlue3}“मनुयन्ति”} भेदात्\edlabel{pvv.314-2}\footnote{\label{pvv.314-2}  २ व्यतिरिक्ते ।} । तास्व{\color{DodgerBlue3}“नु\edlabel{pvv.314-3}\footnote{\label{pvv.314-3}  ३ विकल्पप्रतिभासेपि दोषश्चेत् समानाकारो भात्येव यथा प्रतिभासञ्चास्त्येवावस्तुत्वात् ।}यायि”} च किञ्चिन्न {\color{DodgerBlue3}“भासते”} व्यक्तिमात्रवेदनात् । यच्च {\color{DodgerBlue3}“ज्ञानादव्यतिरिक्त”}माकार\edlabel{pvv.314-4}\footnote{\label{pvv.314-4}  ४ विकल्पबुद्ध्याकारोस्तु सामान्यं स च ज्ञानवद् वस्तुसन्नित्यत्राह ।}स्वरूपं तत्क{\color{DodgerBlue3}“थमर्थान्तरं”} ज्ञानान्तरं\edlabel{pvv.314-5}\footnote{\label{pvv.314-5}  ५ बाह्यव्यक्तेर्ज्ञानस्य चाव्यापने स्यात् सामान्यं ।} {\color{DodgerBlue3}“वा व्रजेत्”} स्वलक्षणरूपत्वादस्य ॥ (७०)
	\pend
      \label{div_pvv.3.71}\edlabel{div_pvv.3.71}
	  
	% new div opening: depth here is 2
	
	  \bigskip
	  \begingroup
	  \large
	
	    
	    \stanza[\smallbreak]
	\label{pv.3.71}\edlabel{pv.3.71}\flagstanza{\tiny\textenglish{...v.3.71}}तस्मान्मिथ्याविकल्पोयमर्थेष्वेकात्मताग्रहः ।&इतरेतरभेदोस्य बीजं संज्ञा यदर्थिका ॥ ७१ ॥\&[\smallbreak]


	
	  \endgroup
	

	  \pstart {\color{DodgerBlue3}“तस्मादर्थरूप”}स्य ज्ञानरूपस्य च सामान्यस्य योगा{\color{DodgerBlue3}“न्मिथ्याविकल्पोयम”}र्थशून्य एष विकल्पोयम{\color{DodgerBlue3}“र्थेष्वेकात्मतायाः”} सामान्यरूपताया {\color{DodgerBlue3}“ग्रहः । अस्य”}\edlabel{pvv.314-6}\footnote{\label{pvv.314-6}  ६ एकप्रत्यवमर्षज्ञानसाधने नियता इति साध्यमुक्तम् ।} चैकात्मतया प्रति\leavevmode\marginnote{\textenglish{315/s}} भासिनो मिथ्याविकल्पस्य {\color{DodgerBlue3}“बीजं”} हेतुरितरस्मादतत्कार्य\edlabel{pvv.315-1}\footnote{\label{pvv.315-1}  १ वाहदोहाद्यकारिणः ।} कारिणो {\color{DodgerBlue3}“भेदो”} व्यावृत्तिः । {\color{DodgerBlue3}“यदर्थिका संज्ञा”} शब्दोपि विजातीयव्यावृत्तौ संकेत्यते तद्विषयश्च ॥ (७१)
	\pend
      \label{div_pvv.3.72}\edlabel{div_pvv.3.72}
	  
	% new div opening: depth here is 2
	

	  \begin{center}%% label @type='head'
	\textbf{(ख) भिन्नानामभिन्नं कार्यम्}
	\end{center}
	

	  \pstart कथं पुनर्भिन्नानामभिन्नं कार्यमित्याह ।
	\pend
      
	  \bigskip
	  \begingroup
	  \large
	
	    
	    \stanza[\smallbreak]
	\label{pv.3.72}\edlabel{pv.3.72}\flagstanza{\tiny\textenglish{...v.3.72}}एकप्रत्यवमर्शार्थज्ञानाद्येकार्थसाधने ।&केचिद् भेदेपि नियताः स्वभावेनेन्द्रियादिवत् ॥ ७२ ॥\&[\smallbreak]


	
	  \endgroup
	

	  \pstart {\color{DodgerBlue3}“एकप्रत्यवमर्श”} एकाकाराध्यवसायोऽ{\color{DodgerBlue3}“र्थज्ञानं”} रूपादि{\color{DodgerBlue3}“ज्ञानं”}\edlabel{pvv.315-2}\footnote{\label{pvv.315-2}  २ पूर्व्वेण सह द्वन्द्वः ।} च त{\color{DodgerBlue3}“दादि”}र्यस्य ज्वरहरणादेः । तस्यैकार्थस्य {\color{DodgerBlue3}“साधने”} करणे {\color{DodgerBlue3}“स्वभावेन”} प्रकृत्या स्वहेतुदत्तया {\color{DodgerBlue3}“भेदेपि”} भेदाविशेषेपि {\color{DodgerBlue3}“केचिद्”} बाहुलेयादयो {\color{DodgerBlue3}“नियता”} न कर्कादय {\color{DodgerBlue3}“इन्द्रियादिवत्\edlabel{pvv.315-3}\footnote{\label{pvv.315-3}  ३ अर्थज्ञानादिदृष्टान्तः ।}”} । यथा चक्षूरूपालोकमनस्कारादय\edlabel{pvv.315-4}\footnote{\label{pvv.315-4}  ४ असत्यप्येककार्यनियते सामान्ये ।}एव भेदाविशेषेपि रूपविज्ञानं जनयन्ति न श्रोत्रशब्दादयः ॥ (७२)
	\pend
      \label{div_pvv.3.73}\edlabel{div_pvv.3.73}
	  
	% new div opening: depth here is 2
	

	  \pstart एतदेव दृष्टान्तरेण दृढयन्नाह ।
	\pend
      
	  \bigskip
	  \begingroup
	  \large
	
	    
	    \stanza[\smallbreak]
	\label{pv.3.73}\edlabel{pv.3.73}\flagstanza{\tiny\textenglish{...v.3.73}}ज्वरादिशमने काश्चित् सह प्रत्येकमेव वा ।&दृष्टा यथा वौषधयो नानात्वेपि न चापराः ॥ ७३ ॥\&[\smallbreak]


	
	  \endgroup
	

	  \pstart {\color{DodgerBlue3}“ज्वरादिशमने”} कर्त्तव्ये {\color{DodgerBlue3}“काश्चित्”} गूडूच्यादयः {\color{DodgerBlue3}“सह”} परस्परं {\color{DodgerBlue3}“प्रत्येकम्वा दृष्टा । नानात्वेपि न चापराः । यथा”} गूडूचीमुस्तादयो भिन्नास्तथा त्रपुषादयोपि । तथापि काश्चिज्ज्वरं शमयन्ति न सर्व्वाः । एवं शाबलेयादय एकप्रत्यवमर्शं कुर्व्वन्ति । {\color{DodgerBlue3}“न”} कर्कादयः ॥ (७३)
	\pend
      \label{div_pvv.3.74}\edlabel{div_pvv.3.74}
	  
	% new div opening: depth here is 2
	

	  \pstart स्यादेतत् । तास्वोषधीषु सामान्यं किञ्चिदस्ति तत् ज्वरादिशमनं करोति । तन्नैष दृष्टान्त (: ।) इत्याह ।
	\pend
      
	  \bigskip
	  \begingroup
	  \large
	
	    
	    \stanza[\smallbreak]
	\label{pv.3.74}\edlabel{pv.3.74}\flagstanza{\tiny\textenglish{...v.3.74}}अविशेषान्न सामान्यमविशेषप्रसङ्गतः ।&तासां क्षेत्रादिभेदेपि ध्रौव्याच्चानुपकारतः ॥ ७४ ॥\&[\smallbreak]


	
	  \endgroup
	

	  \pstart {\color{DodgerBlue3}“अविशेषात्”} सामान्यस्याक्रियावस्थातः क्रियावस्थायां {\color{DodgerBlue3}“न सामान्यं”} काञ्चिदर्थक्रियामुपकल्पयति । {\color{DodgerBlue3}“तासां”} गूडूच्यादिव्यक्तीनां {\color{DodgerBlue3}“क्षेत्रादिभेदेपि”} सामान्यस्य कार्यिणोऽविशिष्टत्वात् ज्वरशमनादेः कार्यस्या{\color{DodgerBlue3}“विशेषप्रसङ्गतः”} चिरक्षिप्रप्रशमनाद्यभावासक्तेः ।\edlabel{pvv.315-5}\footnote{\label{pvv.315-5}  ५ प्रसङ्गात् ।} {\color{DodgerBlue3}“ध्रौव्याच्चानुपकारतः”} । सामान्यस्य नित्यत्वात् । अन्येभ्यः सहकारिभ्य उपकाराभावात् सकृत् तत्कार्याणि स्युः ॥ (७४)
	\pend
      \leavevmode\marginnote{\textenglish{316/s}}\label{div_pvv.3.75}\edlabel{div_pvv.3.75}
	  
	% new div opening: depth here is 2
	

	  \begin{center}%% label @type='head'
	\textbf{(ग) अपोहस्य विजातीयव्यावर्त्तकत्वम्}
	\end{center}
	

	  \pstart अपोहविषयत्वे शब्दविकल्पयोः सामान्यं विशेषणविशेष्यभावं धर्मिधर्मभावञ्च व्यवस्थापयितुमाह ।
	\pend
      
	  \bigskip
	  \begingroup
	  \large
	
	    
	    \stanza[\smallbreak]
	\label{pv.3.75}\edlabel{pv.3.75}\flagstanza{\tiny\textenglish{...v.3.75}}तत्स्वभावविकल्पा धीस्तदर्थे वाप्यनर्थिका ।&विकल्पिकाऽतत्कार्यार्थभेदनिष्ठा प्रजायते ॥ ७५ ॥\&[\smallbreak]


	
	  \endgroup
	

	  \pstart {\color{DodgerBlue3}“तस्यो”}त्पलादेः शब्दादेश्च स्वलक्षणस्य {\color{DodgerBlue3}“स्वभाव”}ग्रहादूर्ध्वं या {\color{DodgerBlue3}“विकल्पिका धीः प्रजायते”} वस्तुतो{\color{DodgerBlue3}“ऽनर्थि”}कापि {\color{DodgerBlue3}“तदर्थेव”} स्वलक्षणविषयेवाऽध्यवसायानुरोधात् परमार्थता\edlabel{pvv.316-1}\footnote{\label{pvv.316-1}  १ तेन यद् [कुमारिल] भ ट्टः (।) “अन्य निवृत्तिमात्रापोहे अनीलादिव्यावृत्तावनुत्पलादिव्यावृत्त्यभावः । एवमनुत्पले ततो न विशेषणविशेष्यता । नापि सामानाधिकरण्यमपोहयोर्भेदात् (।) न च स्वलक्षणं शब्दविषयः । अपोहयोर्व्वाऽप्रतीतेः ।” तन्निरस्तं । बाह्याभिन्नस्य स्वाकारस्य शब्दादिविषयत्वेनेष्टत्वात् (।) तेन नीलोत्पलादिशब्दे शबलार्थाभिधानमेव ।} {\color{DodgerBlue3}“ऽतत्कार्येभ्यो अर्थेभ्यो भेदो”} व्यावृत्तिस्तत्र {\color{DodgerBlue3}“निष्ठा”}ऽवस्थानं यस्या सा तथा विजातीयव्यावृत्तिविषयेत्यर्थः ॥ (७५)
	\pend
      \label{div_pvv.3.76_3.77}\edlabel{div_pvv.3.76_3.77}
	  
	% new div opening: depth here is 2
	
	  \bigskip
	  \begingroup
	  \large
	
	    
	    \stanza[\smallbreak]
	\label{pv.3.76}\edlabel{pv.3.76}\flagstanza{\tiny\textenglish{...v.3.76}}तस्यां यद्रूपमाभाति बाह्यमेकमिवान्यतः ।&व्यावृत्तमिव निस्तत्वं परीक्षानङ्गभावतः ॥ ७६ ॥\&[\smallbreak]


	
	  \endgroup
	
	  \bigskip
	  \begingroup
	  \large
	
	    
	    \stanza[\smallbreak]
	\label{pv.3.77}\edlabel{pv.3.77}\flagstanza{\tiny\textenglish{...v.3.77}}अर्था ज्ञाननिविष्टास्त एवं व्यावृत्तरूपकाः ।&अभिन्ना इव चाभान्ति व्यावृत्ताः पुनरन्यतः ॥ ७७ ॥\&[\smallbreak]


	
	  \endgroup
	\leavevmode\marginnote{\textenglish{63a/MA}}

	  \pstart {\color{DodgerBlue3}“तस्यां”} विकल्पबुद्धौ {\color{DodgerBlue3}“यद् रूपं”} य आकारो दृश्यविकल्पयोरेकत्वाध्यवसायाभ्यासदाढ्‏र्यादबाह्यमपि बाह्यमिवासाधारणमप्येकमिव सर्व्वव्यक्तिषु सदृशवृत्तेः । यथा यथा व्यक्तयो दृश्यन्ते तथा तथैवाध्यवसायात् व्यावृत्तमिव विजातीयव्यावृत्तवस्त्वभेदेन निश्चयात् । न च तद्व्यावृत्तं गोरूपत्वप्रसङ्गात् । अत एव निस्तत्वं निःस्वरूपं यथाभूतरूपतिरोधानेनान्यथाध्यवसायात् । तथा चापरीक्षाया विचारस्या{\color{DodgerBlue3}“नङ्गभावा”}दनाश्रयान्निस्तत्वं तत्\edlabel{pvv.316-2}\footnote{\label{pvv.316-2}  २ परीक्षाङ्गं ।} द्विविधं त्वर्थो ज्ञानं वा । न चैतत् तथा । यतः\edlabel{pvv.316-3}\footnote{\label{pvv.316-3}  ३ यतो बुद्धिप्रतिभासि रूपं निस्तत्वमतस्तद्विषयो व्यवहारोपि मिथ्यार्थ इत्याह ।} कारणात् तेन कल्पितेन सामान्यरूपेण {\color{DodgerBlue3}“तेऽर्था”} एकार्थक्रियाकारिणोऽतत्कार्ये\edlabel{pvv.316-4}\footnote{\label{pvv.316-4}  ४ यतो विजातीयाद् व्यावृत्तिरूपवन्तः । यथाऽनुत्पलाद् व्यावृत्तिरुपिण उत्पलार्थाः ।}भ्यो \leavevmode\marginnote{\textenglish{317/s}} {\color{DodgerBlue3}“व्यावृ”}त्तस्वभावा {\color{DodgerBlue3}“ज्ञाननिविष्टा”} विकल्पबुद्ध्यारूढा {\color{DodgerBlue3}“अभिन्ना इवा\edlabel{pvv.317-1}\footnote{\label{pvv.317-1}  १ न वस्तुतो बुद्धिरूपस्यालीकत्वात् ।} भान्त्युत्पलत्वादिना”} शब्दत्वादिना {\color{DodgerBlue3}“च”} ।
	\pend
      

	  \pstart एतेन सामान्यव्यवस्थो\edlabel{pvv.317-2}\footnote{\label{pvv.317-2}  २ व्यवहारनिमित्तं ।}क्ता (।) त\edlabel{pvv.317-3}\footnote{\label{pvv.317-3}  ३ त एवेति परिकारिकाया आकृष्टं ।} एवैकजा\edlabel{pvv.317-4}\footnote{\label{pvv.317-4}  ४ त एव ज्ञान (नि) विष्टा व्यावृत्ताः सन्तः पुनरन्यतः सजातीयादपि व्यावृत्ता भान्ति यथा नीला अनीलादतश्च व्यावृत्तिद्वयस्य भानात् सामानाधिकरण्यबीजमुक्तं ।}त्यवसिताऽन्यतोऽनीलात् नित्याच्च नीलमित्यादिविकल्पाकारेण एकेन तत्कारिता\edlabel{pvv.317-5}\footnote{\label{pvv.317-5}  ५ व्यावृत्तभाव ।} व्यावृत्तरूपताऽध्यवसायविषयेण विशेषिता व्यावृत्ता आभान्ति नीलोत्पलमिति शब्द\edlabel{pvv.317-6}\footnote{\label{pvv.317-6}  ६ उत्पलानित्यमात्रजाती बुद्ध्या नीलशब्देन विशेषिते ।}स्यानित्यत्वमिति ॥ (७६, ७७)
	\pend
      \label{div_pvv.3.78}\edlabel{div_pvv.3.78}
	  
	% new div opening: depth here is 2
	
	  \bigskip
	  \begingroup
	  \large
	
	    
	    \stanza[\smallbreak]
	\label{pv.3.78}\edlabel{pv.3.78}\flagstanza{\tiny\textenglish{...v.3.78}}त एव तेषां सामान्यसमानाधारगोचरैः ।&ज्ञानाभिधानैर्व्यवहारो मिथ्यार्थः प्रतन्यते ॥ ७८ ॥\&[\smallbreak]


	
	  \endgroup
	

	  \pstart {\color{DodgerBlue3}“तेषा”}मुभयव्यावृत्तिविशेषितानां विकल्पारूढानामर्थानां {\color{DodgerBlue3}“सामान्ययो”}र्व्विशेषणविशेष्यभूतयोर्द्धर्मधर्मिरूपयोश्च सामानाधिकरण्यं {\color{DodgerBlue3}“समानाधारो”} भावप्रधानत्वान्निर्देशस्य । स {\color{DodgerBlue3}“गोचरो”} येषान्तै{\color{DodgerBlue3}“र्ज्ञानाभिधानै”}र्व्विकल्पशब्दैर्व्विशेषणविशेष्यभावस्य धर्मिधर्मभावस्य {\color{DodgerBlue3}“व्यवहारो मिथ्यार्थः प्रतन्यते”} विस्तार्यते ॥ (७८)
	\pend
      \label{div_pvv.3.79_3.80}\edlabel{div_pvv.3.79_3.80}
	  
	% new div opening: depth here is 2
	
	  \bigskip
	  \begingroup
	  \large
	
	    
	    \stanza[\smallbreak]
	\label{pv.3.79}\edlabel{pv.3.79}\flagstanza{\tiny\textenglish{...v.3.79}}स च सर्वः पदार्थानामन्योन्याभावसंश्रयः ।&तेनान्यापोहविषयः तदतत्कार्यकारिणाम् ॥ ७९ ॥\&[\smallbreak]


	
	  \endgroup
	
	  \bigskip
	  \begingroup
	  \large
	
	    
	    \stanza[\smallbreak]
	\label{pv.3.80a}\edlabel{pv.3.80a}\flagstanza{\tiny\textenglish{....3.80a}}वस्तुलाभाश्रयो;\&[\smallbreak]


	
	  \endgroup
	

	  \pstart {\color{DodgerBlue3}“स च”} ज्ञानाभिधानलक्षणो विशेषणविशेष्यभावादिव्यवहारः {\color{DodgerBlue3}“सर्व्वः पदार्थानां”} तदतत्कार्यकारिणा{\color{DodgerBlue3}“मन्योत्यस्य”} इतरेतरस्या{\color{DodgerBlue3}“भावो”} व्यवच्छेदः\edlabel{pvv.317-7}\footnote{\label{pvv.317-7}  ७ व्यावृत्तपदार्थानुभवद्वारेणोत्पत्तेः ।} {\color{DodgerBlue3}“संश्रयो”} विषयो यस्य स तथा । {\color{DodgerBlue3}“तेन”} व्यवच्छेदविषयत्वेना{\color{DodgerBlue3}“पोहविषयो”} व्यवस्थाप्यते । {\color{DodgerBlue3}“वस्तुनो\edlabel{pvv.317-8}\footnote{\label{pvv.317-8}  ८ विधिविषयत्वादस्य ।} लाभस्य”} प्राप्तेश्चा{\color{DodgerBlue3}“श्रयो”} निमित्तं स व्यवहारो भवति ॥\edlabel{pvv.317-9}\footnote{\label{pvv.317-9}  ९ न सर्व्वः किन्तु ।}(७९)
	\pend
      
	  \bigskip
	  \begingroup
	  \large
	
	    
	    \stanza[\smallbreak]
	\label{pv.3.80b}\edlabel{pv.3.80b}\flagstanza{\tiny\textenglish{....3.80b}}यत्र यथोक्तानुमित्तौ यथा ।&नान्यत्र भ्रान्तिसाम्येपि दीपतेजो मणौ यथा ॥ ८० ॥\&[\smallbreak]


	
	  \endgroup
	

	  \pstart यत्र व्यवहारे वस्तुनः सम्बन्धः परंपरया तदुत्पत्तेरस्ति\edlabel{pvv.317-10}\footnote{\label{pvv.317-10}  १० उदाहरणमाह ।} {\color{DodgerBlue3}“यथोक्तानुमितौ”} \leavevmode\marginnote{\textenglish{318/s}} यादृशी साध्यप्रतिबद्धकार्यस्वभावानुपलम्भलिङ्गजानुमितिरुक्ता । तत्र {\color{DodgerBlue3}“यथा”} परंपरया वस्तुसम्बन्धाद् वस्तुप्रा{\color{DodgerBlue3}“प्तिर्नान्यत्र\edlabel{pvv.318-1}\footnote{\label{pvv.318-1}  १ स्थिरादिविकल्पे तत्र प (।) रंपर्येणापि वस्त्वसम्बन्धात् वस्तुनोऽस्थिरादित्वाद् ।} भ्रान्तिसाम्येपि दीपतेजो मणौ यथा”} । कुञ्चिकाविवरदेशस्थे दीपतेजसि भ्रान्त्या समारोपिते मणौ\edlabel{pvv.318-2}\footnote{\label{pvv.318-2}  २ अधिगन्तव्ये ।} परम्परयापि वस्तुसम्बन्धाभावान्न प्राप्तिः । एवं यद्यपि सर्व्वस्य विकल्पस्य स्वप्रतिभासेऽनर्थेर्थाध्यवसायेन वृत्तेर्भ्रान्तत्वं तथापि यो वस्तुसम्बन्धवान् स तत्प्रापको नेतर इति युक्तो विभागः ॥ (८०)
	\pend
      \label{div_pvv.3.81}\edlabel{div_pvv.3.81}
	  
	% new div opening: depth here is 2
	

	  \pstart ननु यदि ज्ञाननिविष्टानामर्थानां सामर्थ्यादिव्यवहारस्तदा बाह्येषु स न स्यादित्याह ।
	\pend
      
	  \bigskip
	  \begingroup
	  \large
	
	    
	    \stanza[\smallbreak]
	\label{pv.3.81}\edlabel{pv.3.81}\flagstanza{\tiny\textenglish{...v.3.81}}तत्रानेकोपि कार्यैका न तत्कार्यपराश्रयैः ।&ज्ञानाभिधानैरेकत्वात् व्यवहारः प्रतार्यते ॥ ८१ ॥\&[\smallbreak]


	
	  \endgroup
	

	  \pstart {\color{DodgerBlue3}“तत्र”} गवाश्वादिव्यक्तिषु मध्ये{\color{DodgerBlue3}“ऽनेकोपि”} सा (?शा) वलेयबाहुलेयादिरेकमभेदावसायादिकार्यं यस्य {\color{DodgerBlue3}“ज्ञानाभिधानैः”} कीदृशैस्तद्वाहाद्यकार्यं {\color{DodgerBlue3}“कार्यं”} न भवति येषां कर्कादीनां तेभ्योऽन्यता तद्व्यवच्छेदः स {\color{DodgerBlue3}“आश्रयो”} विषयो येषां {\color{DodgerBlue3}“तैरेकत्वेन व्यवहारं प्रतार्यते”} प्राप्यते । शावलेयादिर्विजातीयव्यावृतौ व्यावृत्त्याश्रयैः शब्दज्ञानैरेकत्वेन व्यवह्नियते इत्यर्थः ॥ (८१)
	\pend
      \label{div_pvv.3.82}\edlabel{div_pvv.3.82}
	  
	% new div opening: depth here is 2
	
	  \bigskip
	  \begingroup
	  \large
	
	    
	    \stanza[\smallbreak]
	\label{pv.3.82}\edlabel{pv.3.82}\flagstanza{\tiny\textenglish{...v.3.82}}ततश्चैकोप्यनेककृत् तद्भावपरिदीपनात् ।&अतत्कार्यार्थभेदेन नानाधर्मः प्रतीयते ॥ ८२ ॥\&[\smallbreak]


	
	  \endgroup
	

	  \pstart {\color{DodgerBlue3}“ततश्\edlabel{pvv.318-3}\footnote{\label{pvv.318-3}  ३ तथेत्यन्तरं सामान्यव्यपेक्षया सामानाधिकरण्यविशेषणविशेष्यभावव्यवहारश्च बाह्येष्वेवेति दर्शयन्नाह ।}चैकोपि दीपादिरा”}लोकान्धकारापनयनवर्तिदाहा{\color{DodgerBlue3}“द्यनेककार्यकृत्”} (।) \leavevmode\marginnote{\textenglish{63b/MA}} {\color{DodgerBlue3}“तद्भावस्या”}नेककार्यकारित्वस्य {\color{DodgerBlue3}“परिदीपन”}निमित्त{\color{DodgerBlue3}“मतत्कार्यार्थे”}भ्य एकैककार्यसमर्थेभ्यो {\color{DodgerBlue3}“भेदेन”} व्यवच्छेदेन {\color{DodgerBlue3}“नानाधर्मः प्रतीयते”} शब्दविकल्पैः । यथाऽनालोककारिभ्यो भेदादालोककृदनन्धकारहन्तृभ्यो भेदादन्धकारहन्ता दीप उच्यत इत्यादि ॥ (८२)
	\pend
      \label{div_pvv.3.83}\edlabel{div_pvv.3.83}
	  
	% new div opening: depth here is 2
	
	  \bigskip
	  \begingroup
	  \large
	
	    
	    \stanza[\smallbreak]
	\label{pv.3.83}\edlabel{pv.3.83}\flagstanza{\tiny\textenglish{...v.3.83}}यथाप्रतीति कथितः शब्दार्थोसावसन्नपि ।&सामानाधिकरण्यं च वस्तुन्यस्य न सम्भवः ॥ ८३ ॥\&[\smallbreak]


	
	  \endgroup
	\leavevmode\marginnote{\textenglish{319/s}}

	  \pstart तदेवं {\color{DodgerBlue3}“यथा\edlabel{pvv.319-1}\footnote{\label{pvv.319-1}  १ यदि बाह्ये सामान्यादिव्यवहारस्तर्हि वास्तवः प्राप्त इत्याह ।}प्रतीति”} संव्यवहारानतिक्रमेण\edlabel{pvv.319-2}\footnote{\label{pvv.319-2}  २ विकल्पबुद्ध्यनुरोधेन ।} {\color{DodgerBlue3}“शब्दार्थः”} सामान्यलक्षणः सामानाधिकरण्यं {\color{DodgerBlue3}“विशेषणविशेष्यभावश्च च शब्दात् कथितोऽसन्नपि परमार्थतः । यतो वस्तुन्यस्य शब्दार्थस्य सामान्यादेः पारमार्थिकस्यासंभवः । सर्व्वतो व्यावृत्तस्य वस्तुमात्रस्याध्यक्षेणोपलम्भात् । तद्व्यावुत्त्याश्रयेण कल्प्यमानं सामान्यं तत्सामानाधिकरण्यं”} चावस्त्वेव ॥ (८३)
	\pend
      \label{div_pvv.3.84}\edlabel{div_pvv.3.84}
	  
	% new div opening: depth here is 2
	
	  \bigskip
	  \begingroup
	  \large
	
	    
	    \stanza[\smallbreak]
	\label{pv.3.84}\edlabel{pv.3.84}\flagstanza{\tiny\textenglish{...v.3.84}}धर्म्मधर्म्मिव्यवस्थानं भेदोऽभेदश्च यादृशः ।&असमीक्षिततत्वोर्थो यथा लोके प्रतीयते ॥ ८४ ॥\&[\smallbreak]


	
	  \endgroup
	

	  \pstart तथा {\color{DodgerBlue3}“धर्म\edlabel{pvv.319-3}\footnote{\label{pvv.319-3}  ३ अयमपि ज्ञानप्रतिभासिन्यर्थ इत्याह ।} धर्म्मिणोर्व्यवस्थानियमः”} शब्दो धर्म्येव कृतकत्वं {\color{DodgerBlue3}“धर्म एव । तयोर्भेदः\edlabel{pvv.319-4}\footnote{\label{pvv.319-4}  ४ शब्दस्य कृतकमिति ।}”} शब्दो धर्मी कृतकत्वं धर्म इत्य{\color{DodgerBlue3}“भेद”}श्च कृतकोऽनित्यश्च शब्द {\color{DodgerBlue3}“इत्यादि (।) यादृशो”} धर्मान्तरप्रतिक्षेपाप्रतिक्षेपाभ्यामुक्तोऽ{\color{DodgerBlue3}“समीक्षित\edlabel{pvv.319-5}\footnote{\label{pvv.319-5}  ५ विकल्पारोपितत्वात् ।}तत्वोर्थो”}ऽलक्षिततत्वो यथा\edlabel{pvv.319-6}\footnote{\label{pvv.319-6}  ६ बुद्ध्यारूढोप्यध्यवसिततद्भावतया ।} लोके {\color{DodgerBlue3}“प्रतीयते”} ॥ (८४)
	\pend
      \label{div_pvv.3.85}\edlabel{div_pvv.3.85}
	  
	% new div opening: depth here is 2
	
	  \bigskip
	  \begingroup
	  \large
	
	    
	    \stanza[\smallbreak]
	\label{pv.3.85}\edlabel{pv.3.85}\flagstanza{\tiny\textenglish{...v.3.85}}तं तथैव समाश्रित्य साध्यसाधनसंस्थितिः ।&परमार्थावताराय विद्वद्भिरवकल्प्यते ॥ ८५ ॥\&[\smallbreak]


	
	  \endgroup
	

	  \pstart {\color{DodgerBlue3}“तं”} धर्म्मिधर्म्मादिविभागं {\color{DodgerBlue3}“तथैव समाश्रित्य साध्यसाधनसंस्थितिर्व्विद्वद्‏भिरवकल्प्यते”} (।) {\color{DodgerBlue3}“परमार्थे”} वस्तुस्वभावभूते क्षणिकत्वादा{\color{DodgerBlue3}“ववताराय”} लोकस्य वस्तुत्वक्षणिकत्वयोर्भेदः परः कल्पितः वस्तु तु क्षणिकमेव । तच्च साध्यसाधनकल्पनया शक्यं प्रत्येतुं ॥(८५)
	\pend
      \label{div_pvv.3.86}\edlabel{div_pvv.3.86}
	  
	% new div opening: depth here is 2
	

	  \pstart कस्मात् पुनर्व्वस्तुनि सामान्यधर्मिधर्मादि नास्तीत्याह (।)
	\pend
      
	  \bigskip
	  \begingroup
	  \large
	
	    
	    \stanza[\smallbreak]
	\label{pv.3.86}\edlabel{pv.3.86}\flagstanza{\tiny\textenglish{...v.3.86}}संसृज्यन्ते न भिद्यन्ते स्वतोर्थाः पारमार्थिकाः ।&रूपमेकमनेकञ्च तेषु बुद्धेरुपप्लवः ॥ ८६ ॥\&[\smallbreak]


	
	  \endgroup
	

	  \pstart {\color{DodgerBlue3}“पारमार्थिका अर्थाः स्वतः”} स्वरूपेण {\color{DodgerBlue3}“न संसृज्यन्ते”} (।) यतः सामान्यं वस्तु स्यात् । {\color{DodgerBlue3}“नापि भिद्यन्ते”} कृतकत्वशब्दत्वादिना यतो धर्मिधर्मभेदो भवेत् । यत्तु {\color{DodgerBlue3}“तेषु”}\edlabel{pvv.319-7}\footnote{\label{pvv.319-7}  ७ बहुष्वर्थेषु ।} {\color{DodgerBlue3}“रूपमेकं”} गोत्वाद्यनुयायि । {\color{DodgerBlue3}“अनेक”}ञ्च \edlabel{pvv.319-8}\footnote{\label{pvv.319-8}  ८ एकत्रार्थ ।}शब्दकृतकत्वादि व्यवह्रियतेऽसौ बुद्धेरनादिवासनोपहताया {\color{DodgerBlue3}“उपल्पवो”} मिथ्योपदर्शनं ॥ (८६)
	\pend
      \leavevmode\marginnote{\textenglish{320/s}}\label{div_pvv.3.87}\edlabel{div_pvv.3.87}
	  
	% new div opening: depth here is 2
	
	  \bigskip
	  \begingroup
	  \large
	
	    
	    \stanza[\smallbreak]
	\label{pv.3.87}\edlabel{pv.3.87}\flagstanza{\tiny\textenglish{...v.3.87}}भेदस्ततोपि बौद्धेऽर्थे सामान्यं भेद इत्यपि ।&तस्यैव चान्यव्यावृत्त्या धर्मभेदः प्रकल्प्यते ॥ ८७ ॥\&[\smallbreak]


	
	  \endgroup
	

	  \pstart {\color{DodgerBlue3}“ततो”}यं विशेष इदं {\color{DodgerBlue3}“सामान्य”}मिति {\color{DodgerBlue3}“भेद”} इदं साध्यमिदं साधनमित्यपि भेदो {\color{DodgerBlue3}“बौद्धे”} बुद्धिपरिकल्पितेऽ{\color{DodgerBlue3}“र्थे”} न वस्तुनि । कथन्तर्हि स्वलक्षणे कृतकत्वादि{\color{DodgerBlue3}“भेद”} इत्याह । {\color{DodgerBlue3}“तस्यैव”} स्वलक्षणस्यान्यस्मादकृतकादे{\color{DodgerBlue3}“र्व्यावृत्या”} तदाश्रयेण {\color{DodgerBlue3}“धर्मभेदः प्रकल्प्यते ॥”}(८७)
	\pend
      \label{div_pvv.3.88}\edlabel{div_pvv.3.88}
	  
	% new div opening: depth here is 2
	

	  \pstart कस्मात् कल्पितभेदद्वारेण साध्यसाधनभाव इष्टो न वस्तुभेदेनेत्याह ।
	\pend
      
	  \bigskip
	  \begingroup
	  \large
	
	    
	    \stanza[\smallbreak]
	\label{pv.3.88}\edlabel{pv.3.88}\flagstanza{\tiny\textenglish{...v.3.88}}साध्यसाधनसंकल्पे वस्तुदर्शनहानितः ।&भेदः सामान्यसंसृष्टो ग्राह्यो नात्र स्वलक्षणम् ॥ ८८ ॥\&[\smallbreak]


	
	  \endgroup
	

	  \pstart {\color{DodgerBlue3}“साध्यसाधनसंकल्पे”} इदं साध्यमिदं साधनमिति विकल्पे क्रियमाणे {\color{DodgerBlue3}“वस्तुदर्शनस्य हानितः”} कल्पित एव भेदः । न खलु विकल्पे वस्तुदर्शनमस्ति कल्पितगोचर\edlabel{pvv.320-1}\footnote{\label{pvv.320-1}  १ कुतः स्वलक्षणस्य सामान्यसहितस्य ग्रह इति वा ।}त्वात् तस्य । आलोचनाज्ञानं वस्तुविषयं न किञ्चित् तद् विभजति ।
	\pend
      

	  \pstart नन्वाचार्यादि ग्ना स्य {\color{DodgerBlue3}“भेदः सामान्यसंसृष्टो ग्राह्य”} इष्टः । {\color{DodgerBlue3}“अत्र”} भेदः सामान्यसंसृष्टः प्रतीयते इत्यस्मिन् वचने {\color{DodgerBlue3}“न स्वलक्षणं”} ग्राह्यतया निर्दिष्टं किन्त्वध्यवसेयतया\edlabel{pvv.320-2}\footnote{\label{pvv.320-2}  २ बाह्य एव भेदास्तेनापोहलक्षणेन सामान्येन संसृष्टा अध्यवसीयन्ते ।}॥ (८८)
	\pend
      \label{div_pvv.3.89}\edlabel{div_pvv.3.89}
	  
	% new div opening: depth here is 2
	

	  \pstart कस्मादेव\edlabel{pvv.320-3}\footnote{\label{pvv.320-3}  ३ तत्र बोद्धव्यं ।}मित्याह ।
	\pend
      
	  \bigskip
	  \begingroup
	  \large
	
	    
	    \stanza[\smallbreak]
	\label{pv.3.89}\edlabel{pv.3.89}\flagstanza{\tiny\textenglish{...v.3.89}}समानभिन्नाद्याकारैर्न तद् ग्राह्यं कथञ्चन ।&भेदानां बहुभेदानां तत्रैकस्मिन्नयोगतः ॥ ८९ ॥\&[\smallbreak]


	
	  \endgroup
	

	  \pstart {\color{DodgerBlue3}“समानभिन्नाद्याकारैः”} सामान्याकारधर्मिधर्मभेदाकारसामानाधिकरण्याद्याकारैश्च {\color{DodgerBlue3}“तत्”} स्वलक्षणं {\color{DodgerBlue3}“कथञ्चन ग्राह्यं”} न भवति । किं कारणमित्याह । {\color{DodgerBlue3}“भेदानां”}\edlabel{pvv.320-4}\footnote{\label{pvv.320-4}  ४ वस्तुरूपाणां ।} धर्माणां कृतकत्वादीनां {\color{DodgerBlue3}“बहुभेदा”}नामनेकप्रकाराणां {\color{DodgerBlue3}“तत्र”} स्वल\edlabel{pvv.320-5}\footnote{\label{pvv.320-5}  ५ निरंशत्वात् ।}क्षण {\color{DodgerBlue3}“एकस्मिन्नयोगतः”} ॥ (८९)
	\pend
      \leavevmode\marginnote{\textenglish{321/s}}\label{div_pvv.3.90}\edlabel{div_pvv.3.90}
	  
	% new div opening: depth here is 2
	
	  \bigskip
	  \begingroup
	  \large
	
	    
	    \stanza[\smallbreak]
	\label{pv.3.90}\edlabel{pv.3.90}\flagstanza{\tiny\textenglish{...v.3.90}}तद्रूपं सर्वतो भिन्नं तथा तत्प्रतिपादिका ।&न श्रुतिः कल्पना वास्ति सामान्येनैव वृत्तितः ॥ ९० ॥\&[\smallbreak]


	
	  \endgroup
	

	  \pstart तस्मात् तस्य स्वलक्षणस्य {\color{DodgerBlue3}“रूपं सर्व्वतः”} सजातीयविजातीयाद् {\color{DodgerBlue3}“भिन्नं तथा”} तेन सर्व्वतो भिन्नेन रूपेण {\color{DodgerBlue3}“तत्प्रतिपादिका श्रुतिः कल्पना वा नास्ति । सामान्येनैव”} कल्पितेन रूपेण शब्दविकल्पयो{\color{DodgerBlue3}“र्वृत्तितः”} ॥ (९०)
	\pend
      \label{div_pvv.3.91}\edlabel{div_pvv.3.91}
	  
	% new div opening: depth here is 2
	

	  \pstart किं पुनः स्वलक्षणमेव शब्दैर्नोच्यते इत्याह\edlabel{pvv.321-1}\footnote{\label{pvv.321-1}  १ येन तत्प्रतिपादिका न श्रुतिः ।} ।
	\pend
      
	  \bigskip
	  \begingroup
	  \large
	
	    
	    \stanza[\smallbreak]
	\label{pv.3.91}\edlabel{pv.3.91}\flagstanza{\tiny\textenglish{...v.3.91}}शब्दाः संकेतितं प्राहुर्व्यवहाराय स स्मृतः ।&तदा स्वलक्षणं नास्ति सङ्केतस्तेन तत्र न ॥ ९१ ॥\&[\smallbreak]


	
	  \endgroup
	\leavevmode\marginnote{\textenglish{64a/MA}}

	  \pstart {\color{DodgerBlue3}“शब्दाः संकेतित”}मर्थमा{\color{DodgerBlue3}“हुर्न”} यं कञ्चित् ।\edlabel{pvv.321-2}\footnote{\label{pvv.321-2}  २ सामान्येनैवेत्युक्तेप्यधिकपरिहारायोपन्यासः ।} {\color{DodgerBlue3}“स”} संकेतो {\color{DodgerBlue3}“व्यवहाराय स्मृतः”} (।) संकेतित\edlabel{pvv.321-3}\footnote{\label{pvv.321-3}  ३ संकेतविषयीकृतं ।}मर्थं शब्दादुच्चरितात् प्रतिपद्येयमिति संकेतग्रहणं । {\color{DodgerBlue3}“तदा”} व्यवहारकाले च {\color{DodgerBlue3}“स्वलक्षणं”}\edlabel{pvv.321-4}\footnote{\label{pvv.321-4}  ४ एकस्वलक्षणस्यापि क्षणिकत्वान्नानुगमोऽक्षणिकस्यापि संकेतज्ञानाजनकत्वात् । किमुत देशकालभिन्नेषु ।} संकेतविषयो {\color{DodgerBlue3}“नास्ति तेन तत्र”} स्वलक्षणे {\color{DodgerBlue3}“संकेतो न”} युक्तः ॥ (९१)
	\pend
      \label{div_pvv.3.92}\edlabel{div_pvv.3.92}
	  
	% new div opening: depth here is 2
	

	  \pstart एवन्तर्हि सामान्ये\edlabel{pvv.321-5}\footnote{\label{pvv.321-5}  ५ वै शे षि क स्य व्यतिरिक्ते सां ख्य स्याव्यतिरिक्ते ।} व्यवहारकालानुयायिनि संकेतः \edlabel{pvv.321-6}\footnote{\label{pvv.321-6}  ६ भवतु ।} स्यादित्याह ।
	\pend
      
	  \bigskip
	  \begingroup
	  \large
	
	    
	    \stanza[\smallbreak]
	\label{pv.3.92}\edlabel{pv.3.92}\flagstanza{\tiny\textenglish{...v.3.92}}अपि प्रवर्त्तेत पुमान् विज्ञायार्थक्रियाक्षमान् ।&तत्साधनायेत्यर्थेषु संयोज्यन्तेऽभिधाक्रियाः ॥ ९२ ॥\&[\smallbreak]


	
	  \endgroup
	

	  \pstart {\color{DodgerBlue3}“अपि\edlabel{pvv.321-7}\footnote{\label{pvv.321-7}  ७ कथं नाम ।}प्रवर्त्तेत पुमान्”} शब्दाद् {\color{DodgerBlue3}“विज्ञायार्थक्रियाक्षमान्”} । तस्या अर्थक्रियायाः {\color{DodgerBlue3}“साधनायेति”} । एतदर्थम{\color{DodgerBlue3}“र्थेषु संयोज्यन्ते”} संकेत्यन्ते शब्दा न व्यसनितया ॥ (९२)
	\pend
      \label{div_pvv.3.93}\edlabel{div_pvv.3.93}
	  
	% new div opening: depth here is 2
	
	  \bigskip
	  \begingroup
	  \large
	
	    
	    \stanza[\smallbreak]
	\label{pv.3.93}\edlabel{pv.3.93}\flagstanza{\tiny\textenglish{...v.3.93}}अत्रानर्थक्रियायोग्यो नास्ति तद्वानलं स च ।&साक्षान्न योज्यते कस्मादानन्त्याच्चेदिदं समम् ॥ ९३ ॥\&[\smallbreak]


	
	  \endgroup
	

	  \pstart तत्रैवं सत्यनर्थक्रियायां योग्योऽर्थक्रियायां शक्ता जातिरिति न तत्र सङ्केतो युक्तः । {\color{DodgerBlue3}“तद्वान्”} जातिमान् विशेषो{\color{DodgerBlue3}“ऽलं”} शक्तो\edlabel{pvv.321-8}\footnote{\label{pvv.321-8}  ८ सामान्येन शब्दलक्षितेन सम्बन्धाद् व्यक्तिरपि लक्ष्यते}र्थक्रियायामिति तत्र संकेत इति चेत् । {\color{DodgerBlue3}“स”} विशेषश्च {\color{DodgerBlue3}“साक्षात्”} संङ्केते {\color{DodgerBlue3}“कस्मान्न योज्यते”} (।) किं जातिव्यवधि\leavevmode\marginnote{\textenglish{322/s}} स्वीकारेण अनर्थक्रियाकारित्वादस्याः स्वलक्षणस्य च विपर्ययात् । व्यक्तीना{\color{DodgerBlue3}“मानन्त्यान्न”} तत्र शक्य इति {\color{DodgerBlue3}“चेत् । इद”}मानन्त्यं {\color{DodgerBlue3}“समं”}\edlabel{pvv.322-1}\footnote{\label{pvv.322-1}  १ न हि गोशब्दाद् गोत्वबुद्ध्या व्यक्तिबोधोस्ति । सामान्ये वातोदिते कथं व्यक्तौ वृतिः । न हि सम्बन्धेपि दण्डं (ि) च्छन्धीति दण्डिनं छिनत्ति कश्चित् ।} जातिमद्व्यक्तिष्वपि\edlabel{pvv.322-2}\footnote{\label{pvv.322-2}  २ जातौ कृते संकेते व्यक्तावप्रतीतिर्न च जातितद्वतोः सम्बन्धोऽतदुत्पत्तेः ।}॥ (९३)
	\pend
      \label{div_pvv.3.94}\edlabel{div_pvv.3.94}
	  
	% new div opening: depth here is 2
	
	  \bigskip
	  \begingroup
	  \large
	
	    
	    \stanza[\smallbreak]
	\label{pv.3.94}\edlabel{pv.3.94}\flagstanza{\tiny\textenglish{...v.3.94}}तत्कारिणामतत्कारिभेदसाम्ये न किं कृतः ।&तद्वद्दोषस्य साम्याच्चेदस्तु जातिरलं परा ॥ ९४ ॥\&[\smallbreak]


	
	  \endgroup
	

	  \pstart किञ्च यामर्थक्रियामुद्दिश्य शब्दनियोग{\color{DodgerBlue3}“स्तत्कारिणा”}मर्थाना{\color{DodgerBlue3}“मतत्कारि”}भ्यो यो {\color{DodgerBlue3}“भेदो”} व्यवच्छेदस्तदेव {\color{DodgerBlue3}“साम्यं”} सामान्यमन्यापोहः साधारणत्वात् । तत्र {\color{DodgerBlue3}“किं”} संकेतो न कृतः सामान्यवदपोहोपि साधारणोऽनर्थक्रियाकारी स्वलक्षणसम्बन्धात् तदुपलभ्यरूपश्चः । {\color{DodgerBlue3}“तद्वद्दोषस्य”} जाति\edlabel{pvv.322-3}\footnote{\label{pvv.322-3}  ३ जातिमत्पक्षें यो [दिग्] नागोक्तदोषः । तद्वतो नास्वतन्त्रत्वादित्यादिना तद्दोषावताराद् भेदेन्यव्यावृत्तिलक्षणे न शब्दनियोगः जातिकल्पनमनर्थनिर्व्वन्ध एव नित्यव्यापिताद्ययोगादयमाशयो व्यावृत्त्यमिधाने नागस्य जातिर्न प्रवृत्तियोग्या तद्द्वातोदितेपि न वृत्तिरित्युक्तेः अर्थाशक्तेः ।}मति संकेतविषये आनन्त्यात् संकेताकरणस्य {\color{DodgerBlue3}“साम्याच्चेदस्तु”} दोषः समानत्वात् (।) द्वयोरस्त {\color{DodgerBlue3}“जातिः पराऽलमनु”}पयुक्ता जातिमभ्युपेत्यापि विजातीयव्यवच्छेदोऽ\edlabel{pvv.322-4}\footnote{\label{pvv.322-4}  ४ भावानामन्यव्यावृत्त्यभावे वस्तुभूताऽनेकासमवेता जातिरपि न स्यात् ।} वश्याश्रयणीयः । यदि गौरश्वादिभ्यो न व्यवच्छिन्नस्तदाऽश्व एव स्यात् । यश्चैकस्य व्यवच्छेदः\edlabel{pvv.322-5}\footnote{\label{pvv.322-5}  ५ विजातीयाद् भेद एव सजातीयाभेदः इति आनन्त्यादिदोषः खण्डितः स्यात् जातिधर्मश्च दर्शित इति व्यवच्छेदं त्यक्त्वा ।} स सर्व्वस्य (।) स एव सङ्केतविषयो\edlabel{pvv.322-6}\footnote{\label{pvv.322-6}  ६ व्यवच्छेदविशिष्टोर्थः ।}ऽस्तु किं प्रमाणबाधितजातिस्वीकारेण । अत एव तद्वद्दोषोपि न सम्भवति । अभ्युपगम्य तु सांप्रतमापादितं ॥ (९४) किञ्च (।)
	\pend
      \label{div_pvv.3.95}\edlabel{div_pvv.3.95}
	  
	% new div opening: depth here is 2
	
	  \bigskip
	  \begingroup
	  \large
	
	    
	    \stanza[\smallbreak]
	\label{pv.3.95}\edlabel{pv.3.95}\flagstanza{\tiny\textenglish{...v.3.95}}तदन्यपरिहारेण प्रवर्त्तेतेति च ध्वनिः ।&तेन तेभ्योऽव्यवच्छेदे प्रवर्त्तेत कथञ्च सः ॥ ९५ ॥\&[\smallbreak]


	
	  \endgroup
	

	  \pstart तस्मा\edlabel{pvv.322-7}\footnote{\label{pvv.322-7}  ७ अधुना शब्देनावश्यं व्यावृत्तिश्चोदनीयेत्याह ।}देकार्थक्रियाकारिणो\edlabel{pvv.322-8}\footnote{\label{pvv.322-8}  ८ श्रोत्रा प्रतिपादकेन ।}{\color{DodgerBlue3}“ऽन्य”}स्य {\color{DodgerBlue3}“परिहा”}रेण शब्दात् {\color{DodgerBlue3}“प्रवर्त्ततेति ध्व”}निरुच्यते । {\color{DodgerBlue3}“तेन”} ध्वनिना {\color{DodgerBlue3}“तेभ्यो”}ऽतत्कारिभ्योऽ{\color{DodgerBlue3}“स्य”} तत्कारिणोऽव्यवच्छेदे \leavevmode\marginnote{\textenglish{323/s}} व्यवच्छेदेऽक्रियमाणे {\color{DodgerBlue3}“कथं”} स श्रोता प्रतिनियतपदार्थार्थी\edlabel{pvv.323-1}\footnote{\label{pvv.323-1}  १ आनयेति सर्व्वस्य अग्निमिति व्यवच्छेदवैयर्थ्यं अव्यवच्छेदेनाभिधाने ।} प्रवर्त्तेत {\color{DodgerBlue3}“विषयानभिधानात्”} ॥ (९५)
	\pend
      \label{div_pvv.3.96}\edlabel{div_pvv.3.96}
	  
	% new div opening: depth here is 2
	

	  \pstart अथ शब्दैर्व्यवच्छेदः क्रियत एव प्रवृत्तिविषयस्तु जातिरुच्यत इत्याह ।
	\pend
      
	  \bigskip
	  \begingroup
	  \large
	
	    
	    \stanza[\smallbreak]
	\label{pv.3.96}\edlabel{pv.3.96}\flagstanza{\tiny\textenglish{...v.3.96}}व्यवच्छेदोस्ति चेदस्य नन्वेतावत् प्रयोजनम् ।&शब्दानामिति किं तत्र सामान्येनापरेण वः ॥ ९६ ॥\&[\smallbreak]


	
	  \endgroup
	

	  \pstart {\color{DodgerBlue3}“अस्य”} जातिमतो {\color{DodgerBlue3}“व्यवच्छेदोऽस्ति चेत् । नन्वेतावदन्यव्यवच्छेदे\edlabel{pvv.323-2}\footnote{\label{pvv.323-2}  २ तदस्वीकारे न जातिरित्युक्तं ।}न प्रवर्त्तनं शब्दानां प्रयोजन”}मिष्ट{\color{DodgerBlue3}“मिति सामान्येनापरेण”} किं कार्यं वः । तदन्तरेण च शब्दादन्यव्यवच्छेदाभिधानेपि प्रवृत्तिसंभवात् ॥ (९६)
	\pend
      \label{div_pvv.3.97}\edlabel{div_pvv.3.97}
	  
	% new div opening: depth here is 2
	

	  \begin{center}%% label @type='head'
	\textbf{(६) सामान्याभावे प्रत्यभिज्ञासंगतिः}
	\end{center}
	

	  \pstart यदि नास्ति जातिस्तदा भिन्नस्वभावेषु भावेषु स एवायं गौरित्यादि प्रत्यभिज्ञानं न स्यादित्याह ।
	\pend
      
	  \bigskip
	  \begingroup
	  \large
	
	    
	    \stanza[\smallbreak]
	\label{pv.3.97}\edlabel{pv.3.97}\flagstanza{\tiny\textenglish{...v.3.97}}ज्ञानाद्यर्थक्रियां तांस्तां दृष्ट्वा भेदेन कुर्वतः ।&अर्थान्तदन्यविश्लेषविषयैर्ध्वनिभिः सह ॥ ९७ ॥\&[\smallbreak]


	
	  \endgroup
	

	  \pstart {\color{DodgerBlue3}“ज्ञानमादि”}र्यस्या बाहा{\color{DodgerBlue3}“द्यर्थक्रिया”}यास्तामेकाकारपरामर्शविषयां {\color{DodgerBlue3}“भेदेन”} नानात्वेपि {\color{DodgerBlue3}“कुर्व्वतो”}र्थान् {\color{DodgerBlue3}“दृष्ट्वा तदन्य”}स्माद् यो {\color{DodgerBlue3}“विश्लेषः”} स {\color{DodgerBlue3}“विषयो”} येषां {\color{DodgerBlue3}“तैर्द्ध‏्् वनिभिः सह”} ॥ (९७)
	\pend
      \label{div_pvv.3.98}\edlabel{div_pvv.3.98}
	  
	% new div opening: depth here is 2
	
	  \bigskip
	  \begingroup
	  \large
	
	    
	    \stanza[\smallbreak]
	\label{pv.3.98}\edlabel{pv.3.98}\flagstanza{\tiny\textenglish{...v.3.98}}संयोज्य प्रत्यभिज्ञानं पूर्वदृष्टान्यदर्शने ।&परस्यापि न सा बुद्धिः सामान्यादेव केवलात् ॥ ९८ ॥\&[\smallbreak]


	
	  \endgroup
	

	  \pstart {\color{DodgerBlue3}“संयोज्य”} स एवायं गौरित्यादि{\color{DodgerBlue3}“प्रत्यभिज्ञानं पूर्वदृष्टा”}दर्थाद् विलक्षणस्य {\color{DodgerBlue3}“दर्शनेपि”} कुर्यात् । तदन्यविश्लेषस्य सर्व्वत्र साम्यात् ॥
	\pend
      

	  \pstart किञ्च (।) यदि नानात्वात् प्रत्यभिज्ञानमयुक्तं तदा {\color{DodgerBlue3}“परस्यापि न सा”} प्रत्यभिज्ञा  {\color{DodgerBlue3}“बुद्धिः सामान्यादेव केवलादिष्टा”} किन्तु सामान्यविशेषाभ्यां\edlabel{pvv.323-3}\footnote{\label{pvv.323-3}  ३ मी मां स क स्य मतेन ।}। (९८)
	\pend
      \label{div_pvv.3.99_3.100_3.101}\edlabel{div_pvv.3.99_3.100_3.101}
	  
	% new div opening: depth here is 2
	

	  \pstart कस्मादित्याह (।)
	\pend
      
	  \bigskip
	  \begingroup
	  \large
	
	    
	    \stanza[\smallbreak]
	\label{pv.3.99a}\edlabel{pv.3.99a}\flagstanza{\tiny\textenglish{....3.99a}}नित्यं तन्मात्रविज्ञाने व्यक्त‏्यज्ञानप्रसङ्गतः ।\&[\smallbreak]


	
	  \endgroup
	

	  \pstart {\color{DodgerBlue3}“नित्य”}तया {\color{DodgerBlue3}“तस्य”} सामान्य{\color{DodgerBlue3}“मात्रस्य विज्ञाने व्यक्त्यज्ञानप्रसङ्गतः”} ॥
	\pend
      \leavevmode\marginnote{\textenglish{324/s}}

	  \begin{center}%% label @type='head'
	\textbf{(ङ) तद्वत्तानिश्चयः}
	\end{center}
	
	  \bigskip
	  \begingroup
	  \large
	
	    
	    \stanza[\smallbreak]
	\label{pv.3.99b}\edlabel{pv.3.99b}\flagstanza{\tiny\textenglish{....3.99b}}तदा कदाचित् सम्बद्धस्यागृहीतस्य तद्वतः ॥ ९९ ॥\&[\smallbreak]


	
	  \endgroup
	
	  \bigskip
	  \begingroup
	  \large
	
	    
	    \stanza[\smallbreak]
	\label{pv.3.100a}\edlabel{pv.3.100a}\flagstanza{\tiny\textenglish{...3.100a}}तद्वत्तानिश्चयो न स्याद् व्यवहारस्ततः कथम् ।\&[\smallbreak]


	
	  \endgroup
	

	  \pstart यदा सामान्यज्ञानस्य न विशेषो विषय{\color{DodgerBlue3}“स्तदा कदाचिदगृहीतस्य सम्बद्धस्य”} सामान्येन {\color{DodgerBlue3}“तद्वतो”} विशेषस्य (९९) {\color{DodgerBlue3}“तद्वत्तानिश्चय”} इदमस्य सामान्यमिति ज्ञानं {\color{DodgerBlue3}“न स्यात्”} । ततस्तद्वत्ताव्यवहारः कथं त्वन्मते ॥
	\pend
      
	  \bigskip
	  \begingroup
	  \large
	
	    
	    \stanza[\smallbreak]
	\label{pv.3.100b}\edlabel{pv.3.100b}\flagstanza{\tiny\textenglish{...3.100b}}एकवस्तुसहायाश्चेद् व्यक्तयो ज्ञानकारणम् ॥ १०० ॥\&[\smallbreak]


	
	  \endgroup
	
	  \bigskip
	  \begingroup
	  \large
	
	    
	    \stanza[\smallbreak]
	\label{pv.3.101}\edlabel{pv.3.101}\flagstanza{\tiny\textenglish{....3.101}}तदेकं वस्तु किं तासां नानात्वं समपोहति ।&नानात्वाच्चैकविज्ञानहेतुता तासु नेष्यते ॥ १०१ ॥\&[\smallbreak]


	
	  \endgroup
	\leavevmode\marginnote{\textenglish{64b/MA}}

	  \pstart एकं वस्तु सामान्यं तत्सहायाश्चेद् व्यक्तयो ज्ञानस्य प्रत्यभिज्ञानस्य कारणमिष्यन्ते (।) {\color{DodgerBlue3}“तदेकं”} सामान्यं {\color{DodgerBlue3}“वस्तु किन्तासां”} व्यक्तीनाममिश्रस्वभावानां {\color{DodgerBlue3}“नानात्वं समपोहति”} (।) येन प्रत्यभिज्ञानहेतुत्वे ।\edlabel{pvv.324-1}\footnote{\label{pvv.324-1}  १ किं पुनस्तासां नानात्वापोहः भाष्यत इत्याह ।} {\color{DodgerBlue3}“नानात्वाच्चैकस्य\edlabel{pvv.324-2}\footnote{\label{pvv.324-2}  २ ह्यर्थे चः ।}”} प्रत्यभिज्ञा{\color{DodgerBlue3}“ज्ञानस्य हेतुता तासु”} त्वया {\color{DodgerBlue3}“नेष्यते”} । तच्चेत् तथैव\edlabel{pvv.324-3}\footnote{\label{pvv.324-3}  ३ व्यक्तिषु ।} न\edlabel{pvv.324-4}\footnote{\label{pvv.324-4}  ४ पूर्ववद् व्यक्तिग्रहः ।} स्यात्\edlabel{pvv.324-5}\footnote{\label{pvv.324-5}  ५ च एवार्थे ।}प्रत्यभिज्ञानम् ॥ (१००, १०१)
	\pend
      \label{div_pvv.3.102}\edlabel{div_pvv.3.102}
	  
	% new div opening: depth here is 2
	
	  \bigskip
	  \begingroup
	  \large
	
	    
	    \stanza[\smallbreak]
	\label{pv.3.102}\edlabel{pv.3.102}\flagstanza{\tiny\textenglish{....3.102}}अनेकमपि यद्येकमपेक्ष्याभिन्नबुद्धिकृत् ।&ताभिर्विनापि प्रत्येकं क्रियमाणान्धियं प्रति ॥ १०२ ॥\&[\smallbreak]


	
	  \endgroup
	
	  \bigskip
	  \begingroup
	  \large
	
	    
	    \stanza[\smallbreak]
	\label{pv.3.103a}\edlabel{pv.3.103a}\flagstanza{\tiny\textenglish{...3.103a}}तेनैकेनापि सामान्यात् तासां नेत्यग्रहे धिया ।\&[\smallbreak]


	
	  \endgroup
	

	  \pstart {\color{DodgerBlue3}“अनेकमपि”} व्यक्तिरूपं {\color{DodgerBlue3}“यद्येकं”} सामान्य{\color{DodgerBlue3}“मपेक्ष्याभिन्नबुद्धिकृत्”} प्रत्यभिज्ञानकारीष्यते । {\color{DodgerBlue3}“ताभि”}र्व्यक्तिभिः प्रत्येकं\edlabel{pvv.324-6}\footnote{\label{pvv.324-6}  ६ न समस्ताभिः किन्तु प्रत्येकं शावलेयाभावे बाहुलेये गोबुद्धिस्तदभावेन्यत्र (।) एवं सर्व्वासां प्रत्येकमभावेपि ।} {\color{DodgerBlue3}“विनैकै”}कया प्रेरितेन सामान्येनैकेन {\color{DodgerBlue3}“क्रियमाणां धियं”} प्रत्यभिज्ञां प्रतिभासां व्यक्तीनां\edlabel{pvv.324-7}\footnote{\label{pvv.324-7}  ७ प्रत्येकं व्यक्त्यभावेपि ज्ञानभावात् ।}सामर्थ्यं नेति तया धिया तासामग्रहः । केवलं सामान्यग्रहणे च तद्वत्ता निश्चयो न स्यादिति दुःपरिहरं ॥ (१०२)
	\pend
      \label{div_pvv.3.103}\edlabel{div_pvv.3.103}
	  
	% new div opening: depth here is 2
	

	  \pstart ननु यथा नीलादीनामेकैकापायेपि चक्षुर्व्विज्ञानं दृष्टं तत्समुदायेपि दृश्यते । एवं यद्यपि प्रत्येकं व्यक्तीनां व्यभिचारस्तथापि यदा सत्वं\edlabel{pvv.324-8}\footnote{\label{pvv.324-8}  ८ बाह्यमध्यवस्यतीत्यलीकेति न सम्वाद इत्याह ।} तदा विषयत्वमित्याह ॥
	\pend
      \leavevmode\marginnote{\textenglish{325/s}}
	  \bigskip
	  \begingroup
	  \large
	
	    
	    \stanza[\smallbreak]
	\label{pv.3.103b}\edlabel{pv.3.103b}\flagstanza{\tiny\textenglish{...3.103b}}नीलादेर्नेत्रविज्ञाने पृथक् सामर्थ्यदर्शनात् ॥ १०३ ॥\&[\smallbreak]


	
	  \endgroup
	

	  \pstart {\color{DodgerBlue3}“नीलादेर्नेत्रविज्ञाने”} कार्ये {\color{DodgerBlue3}“पृथक्”} प्रत्येकं {\color{DodgerBlue3}“सामर्थ्यदर्शनात्”} । (१०३)
	\pend
      \label{div_pvv.3.104}\edlabel{div_pvv.3.104}
	  
	% new div opening: depth here is 2
	
	  \bigskip
	  \begingroup
	  \large
	
	    
	    \stanza[\smallbreak]
	\label{pv.3.104a}\edlabel{pv.3.104a}\flagstanza{\tiny\textenglish{...3.104a}}शक्तिसिद्धिः समूहेपि नैवं व्यक्तेः कथञ्चन ।\&[\smallbreak]


	
	  \endgroup
	

	  \pstart {\color{DodgerBlue3}“शक्तिसिद्धिः समूहेपि युक्ता । एवं व्यक्तेर्न्न”} प्रत्यभिज्ञानजनने सामर्थ्यं कथञ्चन दृष्टं येन समुदायेपि सामर्थ्यकल्पना स्यात्\edlabel{pvv.325-1}\footnote{\label{pvv.325-1}  १ नीलादिस्वस्वरूपभेदवत् ज्ञानान्यपि भिन्नान्याकारभेदात् । नैवं शुक्लादिसहितसामान्यजनिते ज्ञाने भेदः ।} ॥
	\pend
      
	  \bigskip
	  \begingroup
	  \large
	
	    
	    \stanza[\smallbreak]
	\label{pv.3.104b}\edlabel{pv.3.104b}\flagstanza{\tiny\textenglish{...3.104b}}तासामन्यतमापेक्ष्यं तच्चेच्छक्तं न केवलम् ॥ १०४ ॥\&[\smallbreak]


	
	  \endgroup
	

	  \pstart {\color{DodgerBlue3}“तासां”} व्यक्ती{\color{DodgerBlue3}“नामन्यतमापेक्ष्यं”} तत् सामान्यं प्रत्यभिज्ञाने {\color{DodgerBlue3}“शक्तं”} न {\color{DodgerBlue3}“केवलमिति चेत्”}\edlabel{pvv.325-2}\footnote{\label{pvv.325-2}  २ एवं सति ।}॥ (१०४)
	\pend
      \label{div_pvv.3.105}\edlabel{div_pvv.3.105}
	  
	% new div opening: depth here is 2
	
	  \bigskip
	  \begingroup
	  \large
	
	    
	    \stanza[\smallbreak]
	\label{pv.3.105}\edlabel{pv.3.105}\flagstanza{\tiny\textenglish{....3.105}}तदेकमुपकुर्युस्ताः कथमेकान्धियञ्च न ।&कार्यश्च तासां प्राप्तोसौ जननं यदुपक्रिया ॥ १०५ ॥\&[\smallbreak]


	
	  \endgroup
	

	  \pstart तत्सामान्य{\color{DodgerBlue3}“मेकं”} कथन्ता व्यक्तय {\color{DodgerBlue3}“उपकुर्युः”} (।) {\color{DodgerBlue3}“न त्वेका”}न्धियमनुपक्रियमाणस्य समवेतत्वेऽतिप्रसङ्गाद् व्यक्तिभिः सामान्यमुपकर्त्तव्यं । तथा धीरप्येकोपकर्त्तव्या । यच्च सामान्यादिरुपक्रियते व्यक्तिभिः {\color{DodgerBlue3}“कार्यश्च तासां प्राप्तोसौ यद्य”}स्मा{\color{DodgerBlue3}“दुपक्रिया”} जननमेव । न ह्युपक्रियमाणादन्यस्मिन्नुपकाराख्ये वस्तुनि कृतेपि तस्य किञ्चित् । तत्सम्बन्धाच्चेत् । सम्बन्धो भिन्न एवेति किन्तस्य जातं । {\color{DodgerBlue3}“अभिन्ने”} तूपकारे स एव कृतः स्यात् । तथा जननमेवोपकारः ॥ (१०५)
	\pend
      \label{div_pvv.3.106}\edlabel{div_pvv.3.106}
	  
	% new div opening: depth here is 2
	
	  \bigskip
	  \begingroup
	  \large
	
	    
	    \stanza[\smallbreak]
	\label{pv.3.106}\edlabel{pv.3.106}\flagstanza{\tiny\textenglish{....3.106}}अभिन्नप्रतिभासा धीर्न भिन्नेष्विति चेन्मतम् ।&प्रतिभासो धिया भिन्नः समाना इति तद्ग्रहात् ॥ १०६ ॥\&[\smallbreak]


	
	  \endgroup
	

	  \pstart एकसामान्याभावे{\color{DodgerBlue3}“ऽभिन्नप्रतिभासा”} एकाकारा {\color{DodgerBlue3}“धीर्न भिन्नेष्विति चेन्मतं”} । ननु \edlabel{pvv.325-3}\footnote{\label{pvv.325-3}  ३ व्यतिरिक्ताव्यतिरिक्तसामान्ययोगाद् भ्रान्तिरेव व्यक्तिष्वेकाकारः प्रतिभास इत्युक्त्वा नैवास्त्येकप्रतिभासो व्यक्तिष्वित्याहाधुना ।} {\color{DodgerBlue3}“प्रतिभासो धिया भिन्नो”} नैकाकारः {\color{DodgerBlue3}“समाना”} इमा {\color{DodgerBlue3}“इति ता”}सां व्यक्तीनां {\color{DodgerBlue3}“ग्रहात्”} ।\edlabel{pvv.325-4}\footnote{\label{pvv.325-4}  ४ न ह्येकस्मिन् प्रतिभासे समाना इति युक्तं किन्तु तदेवेति ।} न हि भूत\edlabel{pvv.325-5}\footnote{\label{pvv.325-5}  ५ नवग्रहं ।} कण्ठगुणवदेकं सामान्यं\edlabel{pvv.325-6}\footnote{\label{pvv.325-6}  ६ व्यक्तिविशिष्टसामान्यग्रहोपि न युक्त इत्याह ।}सर्व्वानुपा (?या) यि व्यक्तिव्यतिरिक्तं प्रतिभाति । किन्तु गौर्गौरिति सामान्यमवसीयते तच्च भेदाधिष्ठानमेव ॥ (१०६)
	\pend
      \label{div_pvv.3.107}\edlabel{div_pvv.3.107}
	  
	% new div opening: depth here is 2
	\leavevmode\marginnote{\textenglish{326/s}}
	  \bigskip
	  \begingroup
	  \large
	
	    
	    \stanza[\smallbreak]
	\label{pv.3.107}\edlabel{pv.3.107}\flagstanza{\tiny\textenglish{....3.107}}कथन्ता भिन्नधीग्राह्याः समाश्चेदेककार्यता ।&सादृश्यं ननु धीः कार्यं तासां सा च विभिद्यते ॥ १०७ ॥\&[\smallbreak]


	
	  \endgroup
	

	  \pstart \edlabel{pvv.326-1}\footnote{\label{pvv.326-1}  १ परोत्र विरोधमाह समाश्चेत् (।) कथम्भिन्नधीग्राह्या एकाभिन्नधीग्राह्या अपि स्युर्व्यक्तिसामान्ये समाना इति अन्यथा घटपटादिवदभेद एव स्यात् ।}ननु समाश्चेद् व्यक्तयोध्यवसीयन्ते । {\color{DodgerBlue3}“कथं ता भिन्नधीग्राह्याः”} । \edlabel{pvv.326-2}\footnote{\label{pvv.326-2}  २ न समानतान्यथानुपपत्त्या स्वहेतुभ्यस्तथोत्पत्तेः केषाञ्चित् ।}न खलु {\color{DodgerBlue3}“सम”}त्वमेकत्वं कि{\color{DodgerBlue3}“न्त्वेककार्यतासादृश्यं”} (।) न हि स एवायं तदत्र वेति नि{\color{DodgerBlue3}“श्चयः”} । अपि त्वयमपि गौरित्यध्यवसायः । तथा चैकार्थक्रियाकारित्वमेव सादृश्यं । {\color{DodgerBlue3}“ननु धीः कार्यं तासां”} व्यक्तीनां {\color{DodgerBlue3}“सा च”} प्रतिव्यक्ति {\color{DodgerBlue3}“भिद्यते”} तत्कथमेकार्थक्रियाकारित्वं सादृश्यं\edlabel{pvv.326-3}\footnote{\label{pvv.326-3}  ३ उदकाहरणादि प्रतिव्यक्ति भिन्नं अनुभवज्ञानमेव व्यक्तिकार्यं तदपि भिन्नमेव न विकल्पकं व्यक्त्यभावेपि भावात् धीरविकल्पापि ।}। (१०७)
	\pend
      \label{div_pvv.3.108}\edlabel{div_pvv.3.108}
	  
	% new div opening: depth here is 2
	

	  \pstart अत आह ।
	\pend
      
	  \bigskip
	  \begingroup
	  \large
	
	    
	    \stanza[\smallbreak]
	\label{pv.3.108}\edlabel{pv.3.108}\flagstanza{\tiny\textenglish{....3.108}}एकप्रत्यवमर्शस्य हेतुत्वाद् धीरभेदिनी ।&एकधीहेतुभावेन व्यक्तीनामप्यभिन्नता ॥ १०८ ॥\&[\smallbreak]


	
	  \endgroup
	

	  \pstart यद्यपि प्रतिव्यक्ति भिन्ना तथाप्येक{\color{DodgerBlue3}“प्रत्यवमर्शस्य”}\edlabel{pvv.326-4}\footnote{\label{pvv.326-4}  ४ स्वविषयस्यैकाकारप्रत्ययस्य ।} हेतुत्वाद् धीरभेदिन्येकाऽभिधीयते (।) तथाविधायाश्चैकस्या धियो\edlabel{pvv.326-5}\footnote{\label{pvv.326-5}  ५ अध्यवसितैकरूपायाः प्रत्येकं व्यक्तिग्राहिधियां भेदेपि प्रत्यभिज्ञया तासामेकत्वमध्यवसीयत इत्यर्थः एतदेवैककार्यंतासादृश्यं ।} हेतुभावेन {\color{DodgerBlue3}“व्यक्तीनामभिन्नतो”}च्यते ॥ (१०८)
	\pend
      \label{div_pvv.3.109}\edlabel{div_pvv.3.109}
	  
	% new div opening: depth here is 2
	
	  \bigskip
	  \begingroup
	  \large
	
	    
	    \stanza[\smallbreak]
	\label{pv.3.109}\edlabel{pv.3.109}\flagstanza{\tiny\textenglish{....3.109}}सा चातत्कार्यविश्लेषस्तदन्यस्यानुवर्तिनः ।&अदृष्टेः प्रतिषेधाच्च संकेतस्तद्विदर्थिकः ॥ १०९ ॥\&[\smallbreak]


	
	  \endgroup
	

	  \pstart {\color{DodgerBlue3}“सा चा”}भिन्नता{\color{DodgerBlue3}“ऽतत्का”}र्येभ्यः पदार्थेभ्यो {\color{DodgerBlue3}“विश्लेषो”} व्यवच्छेदस्तत्कारिणां वस्तुभूतजातिरेव किन्नेष्यते इत्याह । {\color{DodgerBlue3}“ततो”} व्यवच्छेदा{\color{DodgerBlue3}“दन्यस्य”} (अनुवर्त्तिनः) सामान्यस्य {\color{DodgerBlue3}“वस्तुसतोऽदृष्टेः”} स्वभावानुपलब्ध्या {\color{DodgerBlue3}“प्रतिषेधाच्च\edlabel{pvv.326-6}\footnote{\label{pvv.326-6}  ६ पूर्व्वोक्तात् ।}”} । ततश्च {\color{DodgerBlue3}“संकेतस्तद्विदर्थिक”}\edlabel{pvv.326-7}\footnote{\label{pvv.326-7}  ७ विकल्पाध्यस्तबाह्यप्रतिपत्त्यर्थः ।}स्तस्य व्यवच्छेदस्य विदर्थो यस्यास्ति स तथा ॥ (१०९)
	\pend
      \label{div_pvv.3.110}\edlabel{div_pvv.3.110}
	  
	% new div opening: depth here is 2
	
	  \bigskip
	  \begingroup
	  \large
	
	    
	    \stanza[\smallbreak]
	\label{pv.3.110}\edlabel{pv.3.110}\flagstanza{\tiny\textenglish{....3.110}}अतत्कारिविवेकेन प्रवृत्यर्थतया श्रुतिः ।&अकार्यकृतितत्कारितुल्यरूपावभासिनीम् ॥ ११० ॥\&[\smallbreak]


	
	  \endgroup
	\leavevmode\marginnote{\textenglish{327/s}}

	  \pstart संकेत{\color{DodgerBlue3}“श्चातत्कारिविवेकेन प्रवृत्त्यर्यतया”} प्रवृत्तिप्रयोजनत्वेन कृतः ।\edlabel{pvv.327-1}\footnote{\label{pvv.327-1}  १ शब्दजनिता धीः स्वाकारं बाह्यमध्यवस्यतीत्यलीकेति न सम्वादः इत्याह ।} ततश्च श्रुतिरपि धियं जनयन्त्यर्थेन विसंवादिकेति सम्बन्धनीयं । कीदृशीं {\color{DodgerBlue3}“धियं”} (।) स्वाकारे{\color{DodgerBlue3}“ऽकार्यकृति”} । कार्यकारणासमर्थे {\color{DodgerBlue3}“तत्कारि”} तुल्यरूपावभासिनीमेकार्थक्रियाकारि-\leavevmode\marginnote{\textenglish{65a/MA}} सर्व्ववस्तुसाधारणैकरूपाध्यवसायिनीं ॥ (११०)
	\pend
      \label{div_pvv.3.111}\edlabel{div_pvv.3.111}
	  
	% new div opening: depth here is 2
	
	  \bigskip
	  \begingroup
	  \large
	
	    
	    \stanza[\smallbreak]
	\label{pv.3.111}\edlabel{pv.3.111}\flagstanza{\tiny\textenglish{....3.111}}धियं वस्तुपृथग्भावमात्रबीजामनर्थिकाम् ।&जनयन्त्यप्यतत्कारिपरिहाराङ्गभावातः ॥ १११ ॥\&[\smallbreak]


	
	  \endgroup
	

	  \pstart {\color{DodgerBlue3}“वस्तु”}\edlabel{pvv.327-2}\footnote{\label{pvv.327-2}  २ एतेन बहिः प्रवृत्यङ्गत्वं श्रुतेः ।} नस्तत्कारिणोऽतत्कारिभ्यश्च {\color{DodgerBlue3}“पृथग्भाव”} एव केवलः {\color{DodgerBlue3}“स बीजं कारणं यस्या”}स्ता{\color{DodgerBlue3}“मनर्थिका”}मर्थशून्यां\edlabel{pvv.327-3}\footnote{\label{pvv.327-3}  ३ विधित्वेन वस्त्वध्यवसा (या)त् कथमन्यापोहविषयतेत्याह ततोऽतत्कारिपरिहारे हेतुत्वात् ।} परमार्थतः ।
	\pend
      

	  \pstart कथमविसम्वादिका तर्हि सेत्याह ।
	\pend
      

	  \pstart {\color{DodgerBlue3}“अतत्का”}रिणां विजातीयानां {\color{DodgerBlue3}“परिहार”}स्या{\color{DodgerBlue3}“ङ्गभावतः”} कारणत्वात् {\color{DodgerBlue3}“तद्व्यवच्छे”}दस्य विषयत्वात्\edlabel{pvv.327-4}\footnote{\label{pvv.327-4}  ४ अन्यव्यावृत्तवस्त्वध्यवसायिबुद्धिजननात् स्वलक्षणे प्रवर्त्तयति ।}तत्र प्रवृत्तिः ॥ (१११)
	\pend
      \label{div_pvv.3.112}\edlabel{div_pvv.3.112}
	  
	% new div opening: depth here is 2
	
	  \bigskip
	  \begingroup
	  \large
	
	    
	    \stanza[\smallbreak]
	\label{pv.3.112}\edlabel{pv.3.112}\flagstanza{\tiny\textenglish{....3.112}}वस्तुभेदाश्रयाच्चार्थे न विसंवादिका मता ।&ततोन्यापोहविषया तत्कर्त्राश्रितभावतः ॥ ११२ ॥\&[\smallbreak]


	
	  \endgroup
	

	  \pstart वस्तुभेदाश्रयाच्च वस्तुविषयस्य परंपरया हेतुत्वात् तत्र प्रवर्त्तयन्ती श्रुतिरर्थे न विसंवादिका मता ।\edlabel{pvv.327-5}\footnote{\label{pvv.327-5}  ५ विधित्वेन वस्त्वध्यवसा (या) त् कथमन्यापोहविषयतेत्याह (।) ततोऽतत्कारिपरिहारहेतुः ।} ततोन्यायोहविषया तत्कर्त्राश्रितभावतः । तत्कार्यकर्तृ वस्त्वाश्रितत्वात्\edlabel{pvv.327-6}\footnote{\label{pvv.327-6}  ६ स्वार्थाभिधानाच्छु तेरपोहे कर्त्तृत्वं व्यवहारे संकेते वस्तुभेदाश्रयत्वादपोहाश्रितत्वं ।}॥ (११२)
	\pend
      \label{div_pvv.3.113}\edlabel{div_pvv.3.113}
	  
	% new div opening: depth here is 2
	
	  \bigskip
	  \begingroup
	  \large
	
	    
	    \stanza[\smallbreak]
	\label{pv.3.113}\edlabel{pv.3.113}\flagstanza{\tiny\textenglish{....3.113}}अवृक्षव्यतिरेकेण वृक्षार्थग्रहणे द्वयम् ।&अन्योन्याश्रयमित्येकग्रहाभावे द्वयाग्रहः ॥ ११३ ॥\&[\smallbreak]


	
	  \endgroup
	

	  \pstart ननु संकेतकाले{\color{DodgerBlue3}“ऽवृक्षव्यतिरेकेण वृक्षार्थ”}स्य {\color{DodgerBlue3}“ग्रहणे द्वयमन्योन्याश्रयं”} वृक्षो\edlabel{pvv.327-7}\footnote{\label{pvv.327-7}  ७ यावद् गां नाबुद्ध तावद् गां न बुध्यते इति परः तत्सामन्यमेष्टव्यं ।}ऽवृक्षश्चान्योन्यापेक्ष इत्येकस्य वृक्षस्यावृक्षस्य चाव्यवस्थितत्वात् {\color{DodgerBlue3}“ग्रहाभावे द्वयाग्रहः”} \leavevmode\marginnote{\textenglish{328/s}} प्राप्तः । निश्चिते हि वृक्षे तदभावः शक्यो निश्चेतुं । अवृक्षत्वे च निश्चिते वृक्षो निश्चेय इत्येकानिश्चयाद् द्वयानिश्चयः ॥ (११३)
	\pend
      \label{div_pvv.3.114_3.115}\edlabel{div_pvv.3.114_3.115}
	  
	% new div opening: depth here is 2
	
	  \bigskip
	  \begingroup
	  \large
	
	    
	    \stanza[\smallbreak]
	\label{pv.3.114a}\edlabel{pv.3.114a}\flagstanza{\tiny\textenglish{...3.114a}}सङ्केतासम्भवस्तस्मादिति केचित् प्रचक्षते ।&तेषामवृक्षास्सङ्केते व्यवच्छिन्ना न वा;\&[\smallbreak]


	
	  \endgroup
	

	  \pstart {\color{DodgerBlue3}“सेङ्केतासम्भवस्तस्मादिति केचि”} ज्जै मि नी याः {\color{DodgerBlue3}“प्रचक्षते । तेषामेव”}\edlabel{pvv.328-1}\footnote{\label{pvv.328-1}  १ तुल्यदोषतामाह ।} वादिनां जातावपि {\color{DodgerBlue3}“सङ्केते”} क्रियमाणेऽ{\color{DodgerBlue3}“वृक्षा व्यवच्छिन्ना न वा”} ।
	\pend
      
	  \bigskip
	  \begingroup
	  \large
	
	    
	    \stanza[\smallbreak]
	\label{pv.3.114b}\edlabel{pv.3.114b}\flagstanza{\tiny\textenglish{...3.114b}}यदि ॥ ११४ ॥\&[\smallbreak]


	
	  \endgroup
	
	  \bigskip
	  \begingroup
	  \large
	
	    
	    \stanza[\smallbreak]
	\label{pv.3.115}\edlabel{pv.3.115}\flagstanza{\tiny\textenglish{....3.115}}व्यवच्छिन्नाः कथं ज्ञाताः प्राग्वृक्षग्रहणादृते ।&अनिराकरणे तेषां सङ्केते व्यवहारिणाम् ॥ ११५ ॥\&[\smallbreak]


	
	  \endgroup
	

	  \pstart {\color{DodgerBlue3}“यदि व्यवच्छिन्ना”} इष्यन्ते {\color{DodgerBlue3}“कथं”} संकेतात् प्राक् {\color{DodgerBlue3}“ज्ञाता वृक्षग्रहणा”}दृते संकेताद् वृक्षस्याज्ञातत्वाद् अवृक्षाः कथं ज्ञाताः । अथ न व्यवच्छिन्नाः तदाऽनिराकरणे {\color{DodgerBlue3}“तेषां”} संकेते {\color{DodgerBlue3}“व्यवहारिणां”} व्यवहारकाले\edlabel{pvv.328-2}\footnote{\label{pvv.328-2}  २ न हि संकेते पराव्यवच्छेदेन निवेशिच्छब्दाद् व्यवहारे तत्परिहारेण प्रवृत्तिः ।}॥ (११५)
	\pend
      \label{div_pvv.3.116_3.117_3.118_3.119_3.120}\edlabel{div_pvv.3.116_3.117_3.118_3.119_3.120}
	  
	% new div opening: depth here is 2
	
	  \bigskip
	  \begingroup
	  \large
	
	    
	    \stanza[\smallbreak]
	\label{pv.3.116}\edlabel{pv.3.116}\flagstanza{\tiny\textenglish{....3.116}}न स्यात् तत्परिहारेण प्रवृत्तिर्वृक्षभेदवत् ।&अविधाय निषिध्यान्यत् प्रदर्श्यैकं पुरः स्थितम् ॥ ११६ ॥\&[\smallbreak]


	
	  \endgroup
	
	  \bigskip
	  \begingroup
	  \large
	
	    
	    \stanza[\smallbreak]
	\label{pv.3.117a}\edlabel{pv.3.117a}\flagstanza{\tiny\textenglish{...3.117a}}वृक्षोयमिति संकेतः क्रियते तत् प्रपद्यते ।&ब्यवहारेपि तेनायमदोष इति चेत्;\&[\smallbreak]


	
	  \endgroup
	

	  \pstart तेषामवृक्षाणां {\color{DodgerBlue3}“परिहारेण न स्यात् प्रवृत्तिः वृक्षभेदवद्”} (।) यथा वृक्षविशेषाणां वृक्षसंकेतेऽव्यवच्छिन्नत्वात् प्रवृत्तिविषयत्वमेवमवृक्षाणामपि स्यात् । नन्वे\edlabel{pvv.328-3}\footnote{\label{pvv.328-3}  ३ वस्तुसामान्यवादी प्राह ।} {\color{DodgerBlue3}“कं”} शाखादिमन्तं {\color{DodgerBlue3}“पुर (:) स्थितं प्रदर्श्य”} तस्मा{\color{DodgerBlue3}“दन्यदविधाय निषिध्य”} च । (११६) {\color{DodgerBlue3}“वृक्षोयमिति च संकेतः क्रियते”} वस्तुसामान्यादिभिः {\color{DodgerBlue3}“तत्”} संकेतविषयं सामान्यं\edlabel{pvv.328-4}\footnote{\label{pvv.328-4}  ४ सामान्ये कृतो व्यक्तिषु व्यापकः स्यात् ।} तत्सम्बन्धिबाह्यलक्षणं {\color{DodgerBlue3}“व्यवहारे प्रपद्यते तेनायम”}नन्तरोक्त{\color{DodgerBlue3}“दोषो न”} भवतीति {\color{DodgerBlue3}“चेत्”} ॥
	\pend
      
	  \bigskip
	  \begingroup
	  \large
	
	    
	    \stanza[\smallbreak]
	\label{pv.3.117b}\edlabel{pv.3.117b}\flagstanza{\tiny\textenglish{...3.117b}}तरुः ॥ ११७ ॥\&[\smallbreak]


	
	  \endgroup
	
	  \bigskip
	  \begingroup
	  \large
	
	    
	    \stanza[\smallbreak]
	\label{pv.3.118a}\edlabel{pv.3.118a}\flagstanza{\tiny\textenglish{...3.118a}}अयमप्ययमेवेति प्रसङ्गो न निवर्त्तते ।&एकप्रत्यवमर्शाख्ये ज्ञाने;\&[\smallbreak]


	
	  \endgroup
	\leavevmode\marginnote{\textenglish{329/s}}

	  \pstart अत्र चा{\color{DodgerBlue3}“य\edlabel{pvv.329-1}\footnote{\label{pvv.329-1}  १ तरुरयमपीति पक्षेऽन्यस्यापि तरुत्वमनिषिद्धमिति व्यवहारो न नियतः । अयमेवेति पक्षेऽतरुव्यवच्छेदः स्यात् । तत्र चोक्तं प्रतिपाद्येन संकेते वृक्षावृक्षौ कथं ज्ञाताविति तदवस्थो दोषः ।}मपि”} शाखादिमांस्तरुर{\color{DodgerBlue3}“यमेव”} तरु{\color{DodgerBlue3}“रिति”} पूर्व्वकोऽवृक्षाव्यवच्छेद{\color{DodgerBlue3}“प्रसङ्गो न”} निवर्त्ततेऽत्र चोक्त एव दोष इति द्वयोरप्येंकाग्रहात् द्वयासम्भवात् सङ्केतासम्भवः समानः । इदानीं परिहर्तुमाह ।\edlabel{pvv.329-2}\footnote{\label{pvv.329-2}  २ न दृष्टविपरीतस्य सुज्ञानात् । व्यवच्छेदे तु नैवमनन्वयात् एकत्र दृष्टस्य क्व (चि) दपि । एतत्तुल्यं यस्मादित्याह ।} {\color{DodgerBlue3}“एकप्रत्यवमर्शा”}ख्ये {\color{DodgerBlue3}“ज्ञाने”} भेदाविशेषेपि केचिदेव भावा एकाकाराध्यवसायहेतवो नेतरे ॥ (११७)
	\pend
      
	  \bigskip
	  \begingroup
	  \large
	
	    
	    \stanza[\smallbreak]
	\label{pv.3.118b}\edlabel{pv.3.118b}\flagstanza{\tiny\textenglish{...3.118b}}एकत्र हि स्थितः ॥ ११८ ॥\&[\smallbreak]


	
	  \endgroup
	
	  \bigskip
	  \begingroup
	  \large
	
	    
	    \stanza[\smallbreak]
	\label{pv.3.119}\edlabel{pv.3.119}\flagstanza{\tiny\textenglish{....3.119}}प्रपत्ता तदतद्धेतूनर्थान् विभजते स्वयम् ।&तद्बुद्धिवर्तिनो भावान् भातो हेतुतया धियः ॥ ११९ ॥\&[\smallbreak]


	
	  \endgroup
	
	  \bigskip
	  \begingroup
	  \large
	
	    
	    \stanza[\smallbreak]
	\label{pv.3.120}\edlabel{pv.3.120}\flagstanza{\tiny\textenglish{....3.120}}अहेतुरूपविकलानेकरूपानिव स्वयम् ।&भेदेन प्रतिपद्येतेत्युक्तिर्भेदे नियुज्यते ॥ १२० ॥\&[\smallbreak]


	
	  \endgroup
	

	  \pstart त{\color{DodgerBlue3}“त्रैक”}स्मिन्नेकाकारज्ञानाध्यवसाये {\color{DodgerBlue3}“स्थितः प्रतिपत्ता”} पुरुष{\color{DodgerBlue3}“स्तदतद्धेतूनर्थान् विभजते स्वयं”} । यानेकाकारपरामर्शविषयबुद्ध्यादिहेतूनध्यवस्यति । याँश्चान्यथा तान् यथाक्रमं तद्धेतूनतद्धेतूश्चं भेदेन व्यवस्थापयति संकेतात् {\color{DodgerBlue3}“प्रागेव\edlabel{pvv.329-3}\footnote{\label{pvv.329-3}  ३ अतद्धेतुभ्यस्तद्धेतून् विभज्य स्थापयति या बुद्धिः ।} तद्बुद्धिवर्त्तिन”} एकाकारपरामर्शविषयान् {\color{DodgerBlue3}“भवान्”} शाखादिमतो {\color{DodgerBlue3}“धिय”} एकाकाराया {\color{DodgerBlue3}“हेतुतया भातः”} प्रतिभासमाना {\color{DodgerBlue3}“न हेतो”}रेकाकारबुद्ध्यकारणस्य शाखादिमत्वरहितस्य {\color{DodgerBlue3}“रूपेण विकलान्”} परमार्थभिन्ना{\color{DodgerBlue3}“नप्येकरूपानिव\edlabel{pvv.329-4}\footnote{\label{pvv.329-4}  ४ नियुंक्ते} स्वयमा”}त्मना संकेतयिता मन्यमानः प्रतिपाद्योपि तानतत्कारिभ्यः शाखादिमत्वरहितेभ्यो {\color{DodgerBlue3}“भेदे”}\edlabel{pvv.329-5}\footnote{\label{pvv.329-5}  ५ एकव्यक्तौ गोशब्दसङ्केते सत्यपि संकेतविषयस्य व्यक्त्यन्तरेऽनुगमात् स एवायं गौरिति स्यात् प्रतीतिः ।}नासंकरेण {\color{DodgerBlue3}“प्रतिपद्येतेति । उक्तिः”} शब्दो {\color{DodgerBlue3}“भेदेऽ”}न्यापोहे {\color{DodgerBlue3}“नियुज्यते”} ।\edlabel{pvv.329-6}\footnote{\label{pvv.329-6}  ६ सङ्केत्यते ।}(११८-२०)
	\pend
      \label{div_pvv.3.121_3.122abc}\edlabel{div_pvv.3.121_3.122abc}
	  
	% new div opening: depth here is 2
	
	  \bigskip
	  \begingroup
	  \large
	
	    
	    \stanza[\smallbreak]
	\label{pv.3.121}\edlabel{pv.3.121}\flagstanza{\tiny\textenglish{....3.121}}तं तस्या धीर्विकल्पिका भ्रान्त्यैकं वस्त्विवेक्षते ।&क्वचिन्निवेशनायार्थे विनिवर्त्य कुतश्चन ॥ १२१ ॥\&[\smallbreak]


	
	  \endgroup
	
	  \bigskip
	  \begingroup
	  \large
	
	    
	    \stanza[\smallbreak]
	\label{pv.3.122a}\edlabel{pv.3.122a}\flagstanza{\tiny\textenglish{...3.122a}}बुद्धेः प्रयुज्यते शब्दस्तदर्थस्यावधारणात् ।&व्यर्थोन्यथा प्रयोगः स्यात्;\&[\smallbreak]


	
	  \endgroup
	\leavevmode\marginnote{\textenglish{330/s}}

	  \pstart {\color{DodgerBlue3}“तं”} भेदं {\color{DodgerBlue3}“तस्या”} उक्तेरुच्चारिताया वाच्यतया प्रतियती {\color{DodgerBlue3}“धीर्व्विकल्पिका”} प्रकृति{\color{DodgerBlue3}“भ्रान्त्या एकमिव वस्त्वीक्षते”} । तस्मात् कुतश्चनाकार्यकारिणो\edlabel{pvv.330-1}\footnote{\label{pvv.330-1}  १ ज्ञेयादिदोषे व्यतिरेक उक्तो राद्धान्तः ।}ऽर्थाद् विनिवर्त्त्य \leavevmode\marginnote{\textenglish{65b/MA}} {\color{DodgerBlue3}“क्वचिदे”}कार्थक्रियाकारिण्यर्थे {\color{DodgerBlue3}“बुद्धेर्निवेशनाय शब्दः प्रयुज्यते”} (।) {\color{DodgerBlue3}“तदर्थस्य”} शब्दार्थस्या{\color{DodgerBlue3}“वधारणात्”} । घटेनोदकमानयेत्यादौ प्रतिपदमवधारणमिष्टं । {\color{DodgerBlue3}“व्यर्थोऽन्यथा प्रयोगः स्यात्”} । यदि येन केनचिदानयनमिष्टमुदकमानयेत्युच्येत । यदि\edlabel{pvv.330-2}\footnote{\label{pvv.330-2}  २ अन्यापोहे शब्दार्थे परोऽव्यापित्वमाह । अज्ञेयाद् विज्ञेयस्य भेदेन विषयीकरणं वाच्यं ततोऽज्ञेयोपि ज्ञेयः स्यादविषयीकृताद् व्यवच्छेदाशक्तेः । एकाद्यसर्वञ्चेन्न व्यतिरिक्तसमुदायास्वीकृतेरनर्थकता स्यात् । यत्परः शब्द स शब्दार्थ इति विधायकस्य व्यवच्छेदोप्यर्थ इत्यर्थः ।} शब्दानां व्यवच्छेदो वाच्यस्तदा {\color{DodgerBlue3}“ज्ञेयादिपदानां”} सर्व्वस्य ज्ञेयत्वेन व्यवच्छेद्याभावात् । (१२१, १२२)
	\pend
      \label{div_pvv.3.122d_3.123_3.124}\edlabel{div_pvv.3.122d_3.123_3.124}
	  
	% new div opening: depth here is 2
	

	  \pstart अर्थो न स्यादित्याह\edlabel{pvv.330-3}\footnote{\label{pvv.330-3}  ३ अन्यथा यदि शब्देन कश्चिदर्थो न व्यवच्छिद्यते व्यर्थः शब्दप्रयोगः स्यादुक्तयुक्त्या ।}
	\pend
      
	  \bigskip
	  \begingroup
	  \large
	
	    
	    \stanza[\smallbreak]
	\label{pv.3.122b}\edlabel{pv.3.122b}\flagstanza{\tiny\textenglish{...3.122b}}तज्ज्ञेयादिपदेष्वपि ॥ १२२ ॥\&[\smallbreak]


	
	  \endgroup
	
	  \bigskip
	  \begingroup
	  \large
	
	    
	    \stanza[\smallbreak]
	\label{pv.3.123a}\edlabel{pv.3.123a}\flagstanza{\tiny\textenglish{...3.123a}}व्यवहारोपनीतेषु व्यवच्छेद्योस्ति कश्चन ।\&[\smallbreak]


	
	  \endgroup
	

	  \pstart यस्माद् व्यवच्छेदमन्तरेण न शब्दप्रयोगः तत् तस्माज्ज्ञेया\edlabel{pvv.330-4}\footnote{\label{pvv.330-4}  ४ ज्ञेयाः सर्व्वपदार्थाः सर्वज्ञज्ञानस्येत्यत्रापि यदज्ञेयत्वमाशङ्कितन्तद्व्यवच्छेद्यं ।}दिपदेष्वपि कुतश्चित् प्रकरणात् {\color{DodgerBlue3}“व्यवहारोपनीतेषु”}\edlabel{pvv.330-5}\footnote{\label{pvv.330-5}  ५ विधिप्रतिषेधप्रयोगस्थेषु ।} विशेषविषयेषु {\color{DodgerBlue3}“कश्च”}न तदितरोस्ति कल्पितो वा ।\edlabel{pvv.330-6}\footnote{\label{pvv.330-6}  ६ यदि विधिशब्दार्थोर्थादन्यनिषेधस्तर्हि नागोक्तम्विरुद्धमित्यत आह वृक्षस्यायं भेदो न स्याद् घटवत् ।}
	\pend
      
	  \bigskip
	  \begingroup
	  \large
	
	    
	    \stanza[\smallbreak]
	\label{pv.3.123b}\edlabel{pv.3.123b}\flagstanza{\tiny\textenglish{...3.123b}}निवेशनं च यो यस्माद् भिद्यते तन्निवर्तनात् ॥ १२३ ॥\&[\smallbreak]


	
	  \endgroup
	
	  \bigskip
	  \begingroup
	  \large
	
	    
	    \stanza[\smallbreak]
	\label{pv.3.124a}\edlabel{pv.3.124a}\flagstanza{\tiny\textenglish{...3.124a}}तद्भेदे भिद्यमानानां समानाकारभासिनि ।\&[\smallbreak]


	
	  \endgroup
	

	  \pstart {\color{DodgerBlue3}“यो यस्माद”}तत्कारिणो भिद्यते (।) तमतत्कारिणम्विनिवर्त्य {\color{DodgerBlue3}“भिद्यमानानां तद्भेदे”}ऽतत्कारिभेदे {\color{DodgerBlue3}“समानाकार\edlabel{pvv.330-7}\footnote{\label{pvv.330-7}  ७ विकल्पेनैकत्वेनारोप्य सर्व्वत्र ।} भासिनि निवेशनञ्च शब्दानां\edlabel{pvv.330-8}\footnote{\label{pvv.330-8}  ८ सङ्केतेपि विधिरुक्तोऽनेन ।}”} ॥
	\pend
      
	  \bigskip
	  \begingroup
	  \large
	
	    
	    \stanza[\smallbreak]
	\label{pv.3.124b}\edlabel{pv.3.124b}\flagstanza{\tiny\textenglish{...3.124b}}स चायमन्यव्यावृत्या गम्यते तस्य वस्तुनः ॥ १२४ ॥\&[\smallbreak]


	
	  \endgroup
	\leavevmode\marginnote{\textenglish{331/s}}

	  \pstart {\color{DodgerBlue3}“स चायमन्य”}व्यवच्छेदः प्रोक्त आ चा र्ये ण {\color{DodgerBlue3}“अन्यव्यावृत्त्या”}ऽन्यव्यावृत्तत्वेन {\color{DodgerBlue3}“गम्यते तस्य वस्तुनः”} (॥ १२४)
	\pend
      \label{div_pvv.3.125}\edlabel{div_pvv.3.125}
	  
	% new div opening: depth here is 2
	
	  \bigskip
	  \begingroup
	  \large
	
	    
	    \stanza[\smallbreak]
	\label{pv.3.125a}\edlabel{pv.3.125a}\flagstanza{\tiny\textenglish{...3.125a}}कश्चिद् भाग इति प्रोक्तो रूपं नास्यापि किञ्चन ।\&[\smallbreak]


	
	  \endgroup
	

	  \pstart {\color{DodgerBlue3}“कश्चिद् भागो”} धर्म {\color{DodgerBlue3}“इत्य”}र्थवाचकेन शब्दोऽर्थान्तरनिवृत्तिविशिष्टानेव भावानाहेत्यादिना ग्रन्थेन\edlabel{pvv.331-1}\footnote{\label{pvv.331-1}  १ प्रोक्तो निर्द्दिष्टो गम्यत इति सम्बन्धः ।} । न त्वन्यव्यावृत्तिर्नाम काचिदन्या । तद्विशिष्टञ्च वस्तु वाच्यमिति किन्त्वाक्षिप्तव्यावृत्तिको धर्म एव कश्चित् कल्पितभेदो बहिरध्यवसायविषयः शब्दवाच्यः । वस्तुतो {\color{DodgerBlue3}“रूपं नास्यापि किञ्चनास्ति”} ।
	\pend
      
	  \bigskip
	  \begingroup
	  \large
	
	    
	    \stanza[\smallbreak]
	\label{pv.3.125b}\edlabel{pv.3.125b}\flagstanza{\tiny\textenglish{...3.125b}}तद्गतावेव शब्देभ्यो गम्यतेन्यनिवर्त्तनम् ॥ १२५ ॥\&[\smallbreak]


	
	  \endgroup
	

	  \pstart {\color{DodgerBlue3}“शब्देभ्यस्तस्य”} धर्मस्य नीला{\color{DodgerBlue3}“देर्गतावेव\edlabel{pvv.331-2}\footnote{\label{pvv.331-2}  २ यदि भेदस्य नीरूपत्वं कथं तर्हि निवृत्तिविशिष्टं वाच्यमित्याह तद्गतेति ।} गम्यते अन्य”}स्य {\color{DodgerBlue3}“निवर्त्त”}नमसंकरप्रतीतिसामर्थ्यात् । (१२५)
	\pend
      \label{div_pvv.3.126}\edlabel{div_pvv.3.126}
	  
	% new div opening: depth here is 2
	
	  \bigskip
	  \begingroup
	  \large
	
	    
	    \stanza[\smallbreak]
	\label{pv.3.126}\edlabel{pv.3.126}\flagstanza{\tiny\textenglish{....3.126}}न तत्र गम्यते कश्चित् केनचिद् भेदवान् परः ।&न चापि शब्दो द्वयकृदन्योन्याभाव इत्यसौ ॥ १२६ ॥\&[\smallbreak]


	
	  \endgroup
	

	  \pstart {\color{DodgerBlue3}“न तत्र”} शाब्द्यां बुद्धौ {\color{DodgerBlue3}“कश्चि”}न्नीला\edlabel{pvv.331-3}\footnote{\label{pvv.331-3}  ३ अन्यतो भेदस्यानुयायिनः शब्दवाच्यत्वे सामान्यं तदेव स्यादित्याह असाविति शब्दविशेषयोर्भेदः ।}दिः {\color{DodgerBlue3}“केनचिद्”} व्यावृत्त्यादिना विशिष्टः {\color{DodgerBlue3}“परो गम्यते । न चापि\edlabel{pvv.331-4}\footnote{\label{pvv.331-4}  ४ नन्वेकः शब्दो विधिप्रतिषेधकृत् कथमित्याह ।} शब्दो”} द्वयस्य व्यावृत्तिवचनस्य तद्विशिष्टवचनस्य मुख्यतः कृत् कर्त्ताऽसंकीर्ण्णधर्म वदन् सामर्थ्याद् व्यावृत्तिञ्चाह । तदेवाह (।) {\color{DodgerBlue3}“अन्योन्याभाव इति”} (।) यस्मात् परस्पराभावरूपः सर्व्वो धर्मस्तस्मात् तद्वचने व्यावृत्तिरपि सामर्थ्यादुक्ता । (१२६)
	\pend
      \label{div_pvv.3.127}\edlabel{div_pvv.3.127}
	  
	% new div opening: depth here is 2
	
	  \bigskip
	  \begingroup
	  \large
	
	    
	    \stanza[\smallbreak]
	\label{pv.3.127}\edlabel{pv.3.127}\flagstanza{\tiny\textenglish{....3.127}}अरूपो रूपवत्त्वेन दर्शनं बुद्धिविप्लवः ।&तेनैवापरमार्थोसावन्यथा न हि वस्तुनः ॥ १२७ ॥\&[\smallbreak]


	
	  \endgroup
	

	  \pstart यश्चायं भेदोसावरूपो {\color{DodgerBlue3}“रूपवत्वेन”} यत् {\color{DodgerBlue3}“तद्दर्शन”}\edlabel{pvv.331-5}\footnote{\label{pvv.331-5}  ५ एतावन्मात्रेण व्यावृत्तिविशिष्टत्वं न दण्डिवत्} मस्य स {\color{DodgerBlue3}“बुद्धिविप्लवः ।\edlabel{pvv.331-6}\footnote{\label{pvv.331-6}  ६ दृश्यविकल्पैक्येन वक्तूश्रोत्रोः ।} तेनैव”} बुद्धिविप्लवेनापरमा{\color{DodgerBlue3}“परमार्थो”}ऽसत्योसौ भेदः । {\color{DodgerBlue3}“अन्यथा”} परमार्थत्वे व्यावृत्तिर्व्व{\color{DodgerBlue3}“स्तुनो”} न वस्तु स्यात् । (१२७)
	\pend
      \label{div_pvv.3.128_3.129_3.130_3.131}\edlabel{div_pvv.3.128_3.129_3.130_3.131}
	  
	% new div opening: depth here is 2
	

	  \pstart तच्चायुक्तं ।
	\pend
      \leavevmode\marginnote{\textenglish{332/s}}
	  \bigskip
	  \begingroup
	  \large
	
	    
	    \stanza[\smallbreak]
	\label{pv.3.128a}\edlabel{pv.3.128a}\flagstanza{\tiny\textenglish{...3.128a}}व्यावृत्तिवस्तु भवति भेदोस्यास्मादितीरणात् ।\&[\smallbreak]


	
	  \endgroup
	

	  \pstart न हि वस्तुनो {\color{DodgerBlue3}“व्यावृ\edlabel{pvv.332-1}\footnote{\label{pvv.332-1}  १ अभिन्ना वाऽवृक्षाद् वृक्षस्य भिन्ना वा उभयथापि दूषणमाह निवृत्तिररूपत्वेनाबुद्धत्वात् सङ्केतावृत्तेश्च न शब्दविषयत्वमनुभवात्तु वृक्षोयं नावृक्ष इति निश्चय (:।) तेनान्यनिवृत्तिः प्रतिषेधविकल्पेन कल्पितावृक्षादौ च वृत्तेः शब्दोन्यनिवृत्तमाक्षिपति । तेनान्यनिवृत्तिविशिष्टत्वमुक्तमाचार्येण ।}त्तिर्व्वस्तु”} भवितुमर्हति । {\color{DodgerBlue3}“अस्मा”}देव वृक्षा{\color{DodgerBlue3}“दस्य”} वृक्षस्य {\color{DodgerBlue3}“भेद इतीरणाद्”} विकल्पनात् भेदस्य वस्तुत्वे भिद्यमाना शिंशपैव वा भेदः स्याद् वस्त्वन्तरं वा । न तावच्छिंशपा धवादेरवृक्षाद् भेदाभावप्रसङ्गात् । न हि शिंशपास्वभावलक्षणो भेदो धवादेरस्ति येन तेप्यवृक्षस्य भेदाः स्युः ॥
	\pend
      

	  \pstart सर्व्वत्र एव हि भिद्यमानाभावास्तदात्मन इति चेत् । न तर्ह्येको भेदस्तत्कारिणामतत्कारिभ्यो यः शब्दवाच्यः । व्यक्तिस्वभावस्य तु भेदस्य संकेताविषयत्वादवाच्यतैव । अथ वस्त्वन्तरं भेदस्तदा तस्माद् भेदाख्या (ना) द् वस्त्वन्तराद् भिद्यमानस्य शिंशपादेर्भेदो\edlabel{pvv.332-2}\footnote{\label{pvv.332-2}  २ भेदशिंशपयोर्मध्येऽपरः ।} वक्तव्यः अन्यथा वस्त्वन्तरत्वायोगात् । एवञ्चावृक्षव्यावृत्तेद्वर्यावृत्तत्वात् शिंशपादिरवृक्षः कर्कादिवत् । भेदस्य च वृक्षाद् भिन्नत्वं भेदान्तरोपाधिकमेवेति द्रव्यान्तरवर्ण्णभेदः स्यात् । न ह्यन्योन्यस्य भेदः सम्बन्धाभावेनातिप्रसङ्गात् । सति सम्बन्धे कार्यकारणभाव एवासौ । ततः सर्व्वं कार्यं \leavevmode\marginnote{\textenglish{66a/MA}} स्वकारणस्य भेदः (व्यावृत्तिः) स्यात् ।\edlabel{pvv.332-3}\footnote{\label{pvv.332-3}  ३ अतदो व्यावृत्तिस्तेनैवं न चैवं तद्भेदाभिमतेपि मा भूत् ।} तस्माद् योऽयं भेदः शब्दव्यवहारविषयोऽनुयायी स कल्पितोऽपरमार्थः । स्वस्वभावव्यवस्थितास्तु भावाः पारमार्थिको भेदः ॥
	\pend
      

	  \pstart ननु यदि वस्त्वेक\edlabel{pvv.332-4}\footnote{\label{pvv.332-4}  ४ पूर्व्वयुक्त्या न व्यतिरिक्ता व्यावृर्त्तिर्नापि सामान्यं ।}रूपं तदैकेन शब्देन लिङ्गेन वा वस्तुनि प्रतिपादिते\edlabel{pvv.332-5}\footnote{\label{pvv.332-5}  ५ अखण्डे स्वीक्रियमाणे तस्माद् यो येन धर्मेणेत्युक्तेपि प्रागधिकविधानायाह (।) अ (? आ) कारान्तरसमारोपोत्र (?) श्लेषः (।) स च प्रतिपत्तिभेदेनानेकः ।} शब्दप्रमाणान्तरवृत्तिर्न स्यात् ॥
	\pend
      
	  \bigskip
	  \begingroup
	  \large
	
	    
	    \stanza[\smallbreak]
	\label{pv.3.128b}\edlabel{pv.3.128b}\flagstanza{\tiny\textenglish{...3.128b}}एकार्थश्लेषविच्छेद एको व्याप्रियते ध्वनिः ॥ १२८ ॥\&[\smallbreak]


	
	  \endgroup
	
	  \bigskip
	  \begingroup
	  \large
	
	    
	    \stanza[\smallbreak]
	\label{pv.3.129a}\edlabel{pv.3.129a}\flagstanza{\tiny\textenglish{...3.129a}}लिङ्गं वा तत्र विच्छिन्नं वाच्यं वस्तु न किञ्चन ।\&[\smallbreak]


	
	  \endgroup
	

	  \pstart {\color{DodgerBlue3}“एकस्य”} नित्यत्वादेरर्थस्य {\color{DodgerBlue3}“श्लेषः”} सम्बन्धस्तस्य {\color{DodgerBlue3}“विच्छेदे”} व्यावृत्ता\edlabel{pvv.332-6}\footnote{\label{pvv.332-6}  ६ बुद्धिप्रतिभासिनि धर्मिणि बाह्यभिन्नतयाध्यस्ते शब्दस्य (? मा) त्रेण वृत्तेरिति तत्राभिप्रायः ।}वेको {\color{DodgerBlue3}“ध्वनि-”} \leavevmode\marginnote{\textenglish{333/s}} {\color{DodgerBlue3}“र्लिङ्गं”} वा व्याप्रियते\edlabel{pvv.333-1}\footnote{\label{pvv.333-1}  १ स्वार्थाभिधाद्वाराऽरोपखण्डेन शब्दप्रमान्तरसाफल्यमित्यर्थः ।} । तत्र ध्वनौ {\color{DodgerBlue3}“लिङ्गो”} वा {\color{DodgerBlue3}“विच्छिन्नं”} सर्व्वतो व्यावृत्तं {\color{DodgerBlue3}“न किञ्चन वाच्यमस्ति”} एका व्यावृत्तिः शब्देन {\color{DodgerBlue3}“लिङ्गेन”} वा प्रतिपाद्यते न वस्त्वित्यर्थः ।
	\pend
      
	  \bigskip
	  \begingroup
	  \large
	
	    
	    \stanza[\smallbreak]
	\label{pv.3.129b}\edlabel{pv.3.129b}\flagstanza{\tiny\textenglish{...3.129b}}यस्याभिधानतो वस्तुसामर्थ्यादखिले गतिः ॥ १२९ ॥\&[\smallbreak]


	
	  \endgroup
	
	  \bigskip
	  \begingroup
	  \large
	
	    
	    \stanza[\smallbreak]
	\label{pv.3.130a}\edlabel{pv.3.130a}\flagstanza{\tiny\textenglish{...3.130a}}भवेन्नानाफलः शब्द एकाधारो भवत्यतः ॥\&[\smallbreak]


	
	  \endgroup
	

	  \pstart {\color{DodgerBlue3}“यस्य”} वस्तुनोऽ{\color{DodgerBlue3}“भिधानतो वस्तुसामर्थ्यादखिले”} कृतकानित्यादिवस्तुरूपे {\color{DodgerBlue3}“गतिर्भवेत्”} । येन शब्दप्रमाणान्तरवैयर्थ्यं स्यात् । {\color{DodgerBlue3}“अतः शब्दो”} भूयान्नानाफलोऽनेकधर्मप्रतीतिफल {\color{DodgerBlue3}“एकाधार”} एकधर्मिनिष्टो {\color{DodgerBlue3}“भवति”} ।
	\pend
      

	  \pstart साध्यसाधनभावादिमाख्याय सामानाधिकरण्यन्दर्शयितुमाह ।\edlabel{pvv.333-2}\footnote{\label{pvv.333-2}  २ यदि वस्त्वेव शब्दादेर्विषयस्तदा सर्वाकारप्रतीतिप्रसंगोऽसमानाधिकरण्यादयश्चेत्यन्यापोहे विषयत्वमाह ।}
	\pend
      
	  \bigskip
	  \begingroup
	  \large
	
	    
	    \stanza[\smallbreak]
	\label{pv.3.130b}\edlabel{pv.3.130b}\flagstanza{\tiny\textenglish{...3.130b}}विच्छेदं सूचयन्नेकमप्रतिक्षिप्य वर्त्तते ॥ १३० ॥\&[\smallbreak]


	
	  \endgroup
	
	  \bigskip
	  \begingroup
	  \large
	
	    
	    \stanza[\smallbreak]
	\label{pv.3.131a}\edlabel{pv.3.131a}\flagstanza{\tiny\textenglish{...3.131a}}यदान्यत्; तेन स व्याप्त एकत्वेन न भासते ।\&[\smallbreak]


	
	  \endgroup
	

	  \pstart {\color{DodgerBlue3}“विच्छेद”}मनीलादिव्यवच्छेद{\color{DodgerBlue3}“मेकैकं”} नीलादिशब्दः {\color{DodgerBlue3}“सूचयन् यदाऽप्रतिक्षिप्यान्यदनुत्पलव्यवच्छेदादौ वर्त्तते (।)”} तद्वाचक\edlabel{pvv.333-3}\footnote{\label{pvv.333-3}  ३ अनुत्पलप्रयोगेऽनेन व्याप्तो नीलशब्दः ।}शब्दप्रयोगे सति {\color{DodgerBlue3}“तेन व्याप्तः”} शिष्ट एकत्वेन च भासते ।
	\pend
      
	  \bigskip
	  \begingroup
	  \large
	
	    
	    \stanza[\smallbreak]
	\label{pv.3.131b}\edlabel{pv.3.131b}\flagstanza{\tiny\textenglish{...3.131b}}सामानाधिकरण्यं स्यात् तदा बुद्ध्यनुरोधतः ॥ १३१ ॥\&[\smallbreak]


	
	  \endgroup
	

	  \pstart {\color{DodgerBlue3}“तदा सामानाधिकरण्यं स्या”}न्नीलोत्पलमित्युभयव्यावृत्तिविशिष्टैकवस्तुव्यवसायिकाया {\color{DodgerBlue3}“बुद्धेरनुरोधतः”} ॥ (१३१)
	\pend
      \label{div_pvv.3.132}\edlabel{div_pvv.3.132}
	  
	% new div opening: depth here is 2
	

	  \pstart किञ्च ।
	\pend
      
	  \bigskip
	  \begingroup
	  \large
	
	    
	    \stanza[\smallbreak]
	\label{pv.3.132}\edlabel{pv.3.132}\flagstanza{\tiny\textenglish{....3.132}}वस्तुधर्मस्य संस्पर्शो विच्छेदकरणे ध्वनेः ।&स्यात् सत्यं सति तत्त्वे हि नैकवस्त्वभिधायिनि ॥ १३२ ॥\&[\smallbreak]


	
	  \endgroup
	

	  \pstart {\color{DodgerBlue3}“ध्वनेर्विच्छेदकरणे”} व्यावृत्ति\edlabel{pvv.333-4}\footnote{\label{pvv.333-4}  ४ स्वार्थांभिधानद्वारेण ।}प्रतिपादकत्वेऽवस्थिते {\color{DodgerBlue3}“वस्तुधर्मस्य”} नीलादेः {\color{DodgerBlue3}“संस्पर्शः”} प्रवृत्तिविषयत्वं {\color{DodgerBlue3}“स्यात्”} । स विच्छेदो हि यस्मात् तत्र वस्तुनि {\color{DodgerBlue3}“सत्यं”} सन् तस्मात् सूचने तद्वती प्रवृत्तिर्युक्ता । {\color{DodgerBlue3}“एकं”} सामान्यं {\color{DodgerBlue3}“वस्तु तदभिधायिनि”} तु ध्वनौ न वस्तुसंस्पर्शः ॥ (१३२)
	\pend
      \label{div_pvv.3.133_3.134_3.135}\edlabel{div_pvv.3.133_3.134_3.135}
	  
	% new div opening: depth here is 2
	

	  \pstart हेतुमाह (।)
	\pend
      \leavevmode\marginnote{\textenglish{334/s}}
	  \bigskip
	  \begingroup
	  \large
	
	    
	    \stanza[\smallbreak]
	\label{pv.3.133}\edlabel{pv.3.133}\flagstanza{\tiny\textenglish{....3.133}}बुद्धावभासमानस्य दृश्यस्याभावनिश्चयात् ।&तेनान्यापोहविषयाः प्रोक्ताः सामान्यगोचराः ॥ १३३ ॥\&[\smallbreak]


	
	  \endgroup
	
	  \bigskip
	  \begingroup
	  \large
	
	    
	    \stanza[\smallbreak]
	\label{pv.3.134a}\edlabel{pv.3.134a}\flagstanza{\tiny\textenglish{...3.134a}}शब्दाश्च बुद्धयश्चैव वस्तुन्येषामसंभवात् ।\&[\smallbreak]


	
	  \endgroup
	

	  \pstart {\color{DodgerBlue3}“बुद्धावभासमान्यस्य दृश्यस्य”} सम्मतस्य तस्य स्वभावानुपलब्धेर{\color{DodgerBlue3}“भावनिश्चयात्”} । न वस्तुनि सत्त्वमिति कथं तत्प्रतिपादिकायाः श्रुतेर्व्वस्तुनि वृत्तिः । {\color{DodgerBlue3}“तेन”}\edlabel{pvv.334-1}\footnote{\label{pvv.334-1}  १ यतो वस्तुनि शब्दार्थे दोषः ।} व्यवच्छेदस्य वस्तुनि सत्त्वेन {\color{DodgerBlue3}“सामान्य \edlabel{pvv.334-2}\footnote{\label{pvv.334-2}  २ विकल्पिका इत्यर्थः ।}गोचराश्शब्दाः बुद्धयश्च”} कल्पिका {\color{DodgerBlue3}“अन्या\edlabel{pvv.334-3}\footnote{\label{pvv.334-3}  ३ अन्योऽपोह्यतेऽनेंनेति विकल्पाकारोपोहः ।}पोहविषया”} आ चा र्ये ण प्रोक्ता (:) । अपोहः शब्दलिङ्गाभ्यां प्रतिपाद्यत इति ब्रुवता । न तु भूतसामान्यविषया । {\color{DodgerBlue3}“वस्तु”}न्येषां नीलत्वादीनामनुपलब्धिबाधितत्वेना{\color{DodgerBlue3}“सम्भवात्”} ।
	\pend
      
	  \bigskip
	  \begingroup
	  \large
	
	    
	    \stanza[\smallbreak]
	\label{pv.3.134b}\edlabel{pv.3.134b}\flagstanza{\tiny\textenglish{...3.134b}}एकत्वाद् वस्तुरूपस्य भिन्नरूपा मतिः कुतः ॥ १३४ ॥\&[\smallbreak]


	
	  \endgroup
	
	  \bigskip
	  \begingroup
	  \large
	
	    
	    \stanza[\smallbreak]
	\label{pv.3.135a}\edlabel{pv.3.135a}\flagstanza{\tiny\textenglish{...3.135a}}अन्वयव्यतिरेकौ वा नैकस्यैकार्थगोचरौ ।\&[\smallbreak]


	
	  \endgroup
	

	  \pstart यदि वाच्यं {\color{DodgerBlue3}“तदेक\edlabel{pvv.334-4}\footnote{\label{pvv.334-4}  ४ तच्छब्दवाच्यं सामान्यं स्वलक्षणाद् भिन्नमभिन्नं वाऽतः प्राह । वैशेषिकसाङ्ख्यादेः ।} त्वाद् वस्तुरूपस्य”} । तस्मिन् {\color{DodgerBlue3}“भिन्नरूपा”}\edlabel{pvv.334-5}\footnote{\label{pvv.334-5}  ५ स्वसामान्याकारा ।} अनित्यकृतकत्वाद्या {\color{DodgerBlue3}“मतिः कुतः”} ।\edlabel{pvv.334-6}\footnote{\label{pvv.334-6}  ६ अखण्डं यदि शब्दवाच्यं ।} एकत्वाद् विषयस्यैकरूपवद् बुद्धिर्युक्ता विशेषश्च सर्वतो व्यावृत्त इति तदात्मभूतं सामान्यमपि तथा स्यात् । ततश्चैकस्य सामान्यस्यान्वयव्यतिरेकानुवृत्त्यननुवृत्ती {\color{DodgerBlue3}“एकार्थगोचरौ”} व्यक्त्यन्तरविषयौ न सम्भवतः । तथा हि सामान्यं व्यक्त्यन्तरानुयायि न व्यक्तिरिति वदता व्यक्त्यभिन्नात्मनः सामान्यस्यान्वय{\color{DodgerBlue3}“व्यतिरेकौ”} विरुद्धावभ्युगतौ स्यातां । तच्चायुक्तं ।
	\pend
      
	  \bigskip
	  \begingroup
	  \large
	
	    
	    \stanza[\smallbreak]
	\label{pv.3.135b}\edlabel{pv.3.135b}\flagstanza{\tiny\textenglish{...3.135b}}अभेदव्यवहाराश्च भेदे स्युरनिबन्धनाः ॥ १३५ ॥\&[\smallbreak]


	
	  \endgroup
	

	  \pstart अथवा\edlabel{pvv.334-7}\footnote{\label{pvv.334-7}  ७ भेदपक्षे दोषमाह ॥} व्यक्त्यात्मनो व्यक्तिरूपवद् {\color{DodgerBlue3}“भेदे”} च सामान्यस्या{\color{DodgerBlue3}“भेद\edlabel{pvv.334-8}\footnote{\label{pvv.334-8}  ८ सामान्याधिकरण्यादयः ।}व्यवहारा अनिबन्धनाः स्युरे”}कस्यानुयायिनोऽभावात् ।
	\pend
      

	  \pstart अस्मन्मते तु (।)
	\pend
      
	  \bigskip
	  \begingroup
	  \large
	
	    
	    \stanza[\smallbreak]
	\label{pv.3.136a}\edlabel{pv.3.136a}\flagstanza{\tiny\textenglish{...3.136a}}सर्वत्र भावाद् व्यावृत्तेर्नैते दोषाः प्रसङ्गिनः ।\&[\smallbreak]


	
	  \endgroup
	\leavevmode\marginnote{\textenglish{335/s}}

	  \pstart {\color{DodgerBlue3}“सर्व्वत्र”} व्यक्तिषु {\color{DodgerBlue3}“भावात् व्यावृत्तेः”} सजातीयाद्\edlabel{pvv.335-1}\footnote{\label{pvv.335-1}  १ सजातिभावो विजातिव्यावृत्तिरिति सम्बन्धः ।} विजातीयाच्चैते {\color{DodgerBlue3}“दोषा”} भिन्नाभास\edlabel{pvv.335-2}\footnote{\label{pvv.335-2}  २ अनित्यकृतत्वादि ।}बुद्धिविषत्वाभावान्वय\edlabel{pvv.335-3}\footnote{\label{pvv.335-3}  ३ व्यक्तिरूपत्वे ।}व्यतिरेकादि\edlabel{pvv.335-4}\footnote{\label{pvv.335-4}  ४ अभेद्यव्यवहाराः ।} विरुद्धधर्माध्यासा {\color{DodgerBlue3}“अप्रसङ्गिनो”}\edlabel{pvv.335-5}\footnote{\label{pvv.335-5}  ५ यथैको र्गौरगोर्भिन्नस्तथान्येपीति नासामान्यतादोषः ।} भवन्ति । शब्दविकल्पानां भिन्नभिन्नव्यावृत्तिविषयत्वात् विजातीयव्यावृत्त्याश्रयेणान्वयबुद्धिविषयत्वात् । सजातीयव्यावृत्त्याश्रयेण व्यतिरेकबुद्धिविषयत्वाच्चेति ॥
	\pend
      

	  \pstart कस्मात् पुनरन्यव्यावृत्तौ शब्दसङ्केतो न स्वलक्षण इत्याह ।\leavevmode\marginnote{\textenglish{66b/MA}}
	\pend
      \label{div_pvv.3.136_3.137}\edlabel{div_pvv.3.136_3.137}
	  
	% new div opening: depth here is 2
	
	  \bigskip
	  \begingroup
	  \large
	
	    
	    \stanza[\smallbreak]
	\label{pv.3.136b}\edlabel{pv.3.136b}\flagstanza{\tiny\textenglish{...3.136b}}एककार्येषु भावेषु तत्कार्यपरिचोदने ॥ १३६ ॥\&[\smallbreak]


	
	  \endgroup
	
	  \bigskip
	  \begingroup
	  \large
	
	    
	    \stanza[\smallbreak]
	\label{pv.3.137}\edlabel{pv.3.137}\flagstanza{\tiny\textenglish{....3.137}}गौरवाशक्तिवैफल्याद् भेदाख्यायाः समा श्रुतिः ।&कृता वृद्धैरतत्कार्यव्यावृत्तिविनिबन्धना ॥ १३७ ॥\&[\smallbreak]


	
	  \endgroup
	

	  \pstart {\color{DodgerBlue3}“एक\edlabel{pvv.335-6}\footnote{\label{pvv.335-6}  ६ विजातीयव्यावृत्तं भावं सर्व्वत्र बुद्ध्या स्वाकाराभेदेनाध्यस्तमेकं शब्दाभिधेयमुक्त्वाधुनाऽभिन्नाकारम्विनाप्येककार्येषु च भावेषु कः शब्दो नियोज्यत इत्याह ।}कार्येष्वे”}कत्वाध्यवसायविषयकार्यकारिषु भावेषु {\color{DodgerBlue3}“तत्कार्यपरि\edlabel{pvv.335-7}\footnote{\label{pvv.335-7}  ७ एककार्याणां चोदनार्थं त्रिकालस्थानां ।}चोद”}ननिमित्तं प्रतिव्यक्ति {\color{DodgerBlue3}“भेदाख्याया”} भिन्नस्य शब्दस्य योजनेन संकेतक्रियाया व्यक्त्यानन्त्याद् {\color{DodgerBlue3}“गौरवाद”}शक्ते\edlabel{pvv.335-8}\footnote{\label{pvv.335-8}  ८ स्वलक्षणस्याबाध्यत्वात्}वैफल्याच्च । {\color{DodgerBlue3}“समा”} एका {\color{DodgerBlue3}“श्रुति”}रतत्कार्येभ्यो वा व्यावृत्तिस्तन्निवन्धना तदाश्रया {\color{DodgerBlue3}“वृद्धैः\edlabel{pvv.335-9}\footnote{\label{pvv.335-9}  ९ व्यवहारज्ञैः ।} कृता”} संकेतिता । (१३६, १३७)
	\pend
      \label{div_pvv.3.138}\edlabel{div_pvv.3.138}
	  
	% new div opening: depth here is 2
	
	  \bigskip
	  \begingroup
	  \large
	
	    
	    \stanza[\smallbreak]
	\label{pv.3.138}\edlabel{pv.3.138}\flagstanza{\tiny\textenglish{....3.138}}न भावे सर्वभावानां स्वस्वभावव्यवस्थितेः ।&यद् रूपं शावलेयस्य बाहुलेयस्य नास्ति तत् ॥ १३८ ॥\&[\smallbreak]


	
	  \endgroup
	

	  \pstart {\color{DodgerBlue3}“न भावे”} स्वलक्षणे कस्मादित्याह । {\color{DodgerBlue3}“सर्व्वभावानां स्वस्वभावव्यवस्थितेः\edlabel{pvv.335-10}\footnote{\label{pvv.335-10}  १० असांकर्यात् ।}”} कारणतः । {\color{DodgerBlue3}“यद्रूपं शावलेयस्य बाहुलेयस्य नास्ति तद्”} रूपं । (१३८)
	\pend
      \label{div_pvv.3.139}\edlabel{div_pvv.3.139}
	  
	% new div opening: depth here is 2
	
	  \bigskip
	  \begingroup
	  \large
	
	    
	    \stanza[\smallbreak]
	\label{pv.3.139a}\edlabel{pv.3.139a}\flagstanza{\tiny\textenglish{...3.139a}}अतत्कार्यपरावृत्तिर्द्वयोरपि च विद्यते ।\&[\smallbreak]


	
	  \endgroup
	

	  \pstart {\color{DodgerBlue3}“अतत्कार्येभ्यः परावृत्तिश्च द्वयोः”} शावलेयबाहुलेययो{\color{DodgerBlue3}“रपि विद्यते”} ।\edlabel{pvv.335-11}\footnote{\label{pvv.335-11}  ११ सवार्थाभेदः शब्दाभेदस्य कारणमेष्टव्यं ।}ततस्तत्रैव संकेतो न स्वलक्षणे ।
	\pend
      

	  \pstart स्वलक्षणान्येव तर्ह्यभिन्नस्य\edlabel{pvv.335-12}\footnote{\label{pvv.335-12}  १२ एकस्य सर्व्वसजातीये}शब्दस्य वाच्यानि स्युरित्याह ।
	\pend
      \leavevmode\marginnote{\textenglish{336/s}}
	  \bigskip
	  \begingroup
	  \large
	
	    
	    \stanza[\smallbreak]
	\label{pv.3.139b}\edlabel{pv.3.139b}\flagstanza{\tiny\textenglish{...3.139b}}अर्थाभेदेन च विना शब्दाभेदो न युज्यते ॥ १३९ ॥\&[\smallbreak]


	
	  \endgroup
	

	  \pstart {\color{DodgerBlue3}“अर्थ”}स्या{\color{DodgerBlue3}“भेदेन”} च {\color{DodgerBlue3}“विना शब्द”}स्या{\color{DodgerBlue3}“भेदो न युज्यते”}ऽतिप्रसङ्गात् । (१३९)
	\pend
      \label{div_pvv.3.140_3.141_3.142_3.143_3.144}\edlabel{div_pvv.3.140_3.141_3.142_3.143_3.144}
	  
	% new div opening: depth here is 2
	

	  \pstart एवन्तर्ह्येककार्यतैव भिन्नशब्दवाच्या स्यादित्याह ।
	\pend
      
	  \bigskip
	  \begingroup
	  \large
	
	    
	    \stanza[\smallbreak]
	\label{pv.3.140a}\edlabel{pv.3.140a}\flagstanza{\tiny\textenglish{...3.140a}}तस्मात् तत्कार्यतापीष्टाऽतत्कार्यादेव भिन्नता ।\&[\smallbreak]


	
	  \endgroup
	

	  \pstart यस्मादर्थाभेदाच्छब्दाभेदः कार्यञ्च प्रतिव्यक्ति भिन्नमिति तत्कार्यताप्येककार्योच्यते । साप्यतत्कार्याद् भिन्न\edlabel{pvv.336-1}\footnote{\label{pvv.336-1}  १ न तु तत्कार्यता नाम सामान्यमस्ति ।}तैव स चापोहः ।
	\pend
      

	  \pstart अतत्कार्यव्यावृत्तौ संकेतक्रियां द्रढयितुं दृष्टान्तमाह ।
	\pend
      
	  \bigskip
	  \begingroup
	  \large
	
	    
	    \stanza[\smallbreak]
	\label{pv.3.140b}\edlabel{pv.3.140b}\flagstanza{\tiny\textenglish{...3.140b}}चक्षुरादावनेकत्र रूपविज्ञानके क्वचित् ॥ १४० ॥\&[\smallbreak]


	
	  \endgroup
	
	  \bigskip
	  \begingroup
	  \large
	
	    
	    \stanza[\smallbreak]
	\label{pv.3.141}\edlabel{pv.3.141}\flagstanza{\tiny\textenglish{....3.141}}अविशेषेण तत्कार्यचोदनासंभवे सति ।&सकृत् सर्व्वप्रतीत्यर्थं कश्चित् साङ्केतिकीं श्रुतिम् ॥ १४१ ॥\&[\smallbreak]


	
	  \endgroup
	
	  \bigskip
	  \begingroup
	  \large
	
	    
	    \stanza[\smallbreak]
	\label{pv.3.142a}\edlabel{pv.3.142a}\flagstanza{\tiny\textenglish{...3.142a}}कुर्यादृतेपि तद्रूपसामान्याद् व्यतिरेकिणः ।\&[\smallbreak]


	
	  \endgroup
	

	  \pstart {\color{DodgerBlue3}“यथा चक्षुरादा”}वनेकत्र {\color{DodgerBlue3}“क्वचिद् रूपविज्ञान”}फले विषयभूते{\color{DodgerBlue3}“ऽविशेषेण”} सामान्येन\edlabel{pvv.336-2}\footnote{\label{pvv.336-2}  २ असाधारणकारणचोदने तु चक्षुरादीनां नैकाश्रुतिः ।} {\color{DodgerBlue3}“तस्य”} चक्षुरादि{\color{DodgerBlue3}“कार्य”}स्य कारणवाचकैकशब्दद्वारेण {\color{DodgerBlue3}“चोदना”}याः परेभ्यः प्रकाशनायाश्चक्षुरालोकादिष्वेकसामान्यभावेपि {\color{DodgerBlue3}“संभवे सति कश्चित्”} सन्धेयव्यवहाररुचिः {\color{DodgerBlue3}“सकृत् सर्व्वस्य”} चक्षुरादेः {\color{DodgerBlue3}“प्रतीत्यर्थं\edlabel{pvv.336-3}\footnote{\label{pvv.336-3}  ३ व्यवहारलाघवार्थं । अविशेषेण सामग्रीप्रश्ने ।} सांकेतिकीं”} श्रुतिं {\color{DodgerBlue3}“कुर्यात्”} (।) कुतो रूपं\edlabel{pvv.336-4}\footnote{\label{pvv.336-4}  ४ परो रूपविज्ञानहेतुः सरो वेति ।} विज्ञानमिति {\color{DodgerBlue3}“ऋते”}विना{\color{DodgerBlue3}“पि तद्रूपाद्”} रूपविज्ञानजनकात् {\color{DodgerBlue3}“सामान्या”}च्चक्षुरादि{\color{DodgerBlue3}“व्यतिरेकिणः”} ।
	\pend
      
	  \bigskip
	  \begingroup
	  \large
	
	    
	    \stanza[\smallbreak]
	\label{pv.3.142b}\edlabel{pv.3.142b}\flagstanza{\tiny\textenglish{...3.142b}}एकवृत्तेरनेकोपि यद्येकश्रुतिमान् भवेत् ॥ १४२ ॥\&[\smallbreak]


	
	  \endgroup
	
	  \bigskip
	  \begingroup
	  \large
	
	    
	    \stanza[\smallbreak]
	\label{pv.3.143a}\edlabel{pv.3.143a}\flagstanza{\tiny\textenglish{...3.143a}}वृत्तिराधेयता व्यक्तिरिति तस्मिन्न युज्यते ।&नित्यस्यानुपकार्यत्वात् नाधारः;\&[\smallbreak]


	
	  \endgroup
	

	  \pstart अथैकस्य सामान्यस्य {\color{DodgerBlue3}“वृत्तेरनेकोपि”}\edlabel{pvv.336-5}\footnote{\label{pvv.336-5}  ५ एककार्यत्वेनैकः शब्दो बहुषु सामान्येन वेत्यविशेषं मन्यते परः ।} विषयो {\color{DodgerBlue3}“यद्येकश्रुतिमान् भवेत्”} तदा को दोषः ।\edlabel{pvv.336-6}\footnote{\label{pvv.336-6}  ६ उपलभ्यत्वेऽनुपलब्धिरनुपलभ्यत्वे च बहुष्वेकशब्दः तुल्यज्ञानञ्च नेत्युक्तं ।} ननु केयम्वृत्तिरिष्टा (।) किमाधेयता उत व्यक्तिरभिव्यक्तिः । इत्येतद् द्वयमपि तस्मिन् सामान्ये न {\color{DodgerBlue3}“युज्यते”} । कथमित्याह । {\color{DodgerBlue3}“नित्य”}स्यानाधेयाति\leavevmode\marginnote{\textenglish{337/s}} शयस्या\edlabel{pvv.337-1}\footnote{\label{pvv.337-1}  १ इह गोत्वमिति समवायश्चेन्न स समवेतस्य क्वचित् स्यात् तच्च तदायत्तत्वे तदुत्पत्त्यान्यस्य ।}विशिष्टरूपस्य केनचिद{\color{DodgerBlue3}“नुपकार्यत्वात् नाधारः”} कश्चित् न ह्यनुपकारक आधारोऽतिप्रसङ्गात् ।\edlabel{pvv.337-2}\footnote{\label{pvv.337-2}  २ पूर्व्वत्रानेकान्तमाह परः ।}
	\pend
      

	  \pstart ननु वदरस्यानुपकारकमपि कुण्डमाधार इत्याह ।
	\pend
      
	  \bigskip
	  \begingroup
	  \large
	
	    
	    \stanza[\smallbreak]
	\label{pv.3.143b}\edlabel{pv.3.143b}\flagstanza{\tiny\textenglish{...3.143b}}प्रविसर्प्पतः ॥ १४३ ॥\&[\smallbreak]


	
	  \endgroup
	
	  \bigskip
	  \begingroup
	  \large
	
	    
	    \stanza[\smallbreak]
	\label{pv.3.144}\edlabel{pv.3.144}\flagstanza{\tiny\textenglish{....3.144}}शक्तिस्तद्देशजननं कुण्डादेर्ब्बदरादिषु ।&न संभवति साप्यत्र तदभावेप्यवस्थितेः ॥ १४४ ॥\&[\smallbreak]


	
	  \endgroup
	

	  \pstart {\color{DodgerBlue3}“बदरादिषु”} गुरुतया {\color{DodgerBlue3}“प्रविसर्प्पतो”} विसर्प्पणात् । प्रतिक्षणमसमान\edlabel{pvv.337-3}\footnote{\label{pvv.337-3}  ३ कुण्डात् तुल्यदेशे ।}देशोत्पत्तेः {\color{DodgerBlue3}“कुण्डादेः”} सहकारिणः {\color{DodgerBlue3}“तद्देशजननं”}\edlabel{pvv.337-4}\footnote{\label{pvv.337-4}  ४ कुण्डवदरक्षणाद् विशिष्टदेशजत्वेनैकसामग्रयधीनत्वमुक्तं ।} बदरोपादानदेशजननं {\color{DodgerBlue3}“शक्तिः”} कारणत्वमाधारता (।) सापि शक्तिलक्षणाधारतात्र सामान्ये {\color{DodgerBlue3}“न”} संभवति जन्यत्वाभावात् । सामान्यस्य स्थितिं कुर्व्वन् विशेष आधारः स्यादित्याह । {\color{DodgerBlue3}“त”}स्य विशेषस्या{\color{DodgerBlue3}“भावेप्यवस्थितेः”} (। १४३,१४४)
	\pend
      \label{div_pvv.3.145_3.146}\edlabel{div_pvv.3.145_3.146}
	  
	% new div opening: depth here is 2
	
	  \bigskip
	  \begingroup
	  \large
	
	    
	    \stanza[\smallbreak]
	\label{pv.3.145a}\edlabel{pv.3.145a}\flagstanza{\tiny\textenglish{...3.145a}}न स्थितिः साप्ययुक्तैव भेदाभेदविवेचने ।\&[\smallbreak]


	
	  \endgroup
	

	  \pstart न स्थितिर्व्विशेषात् सामान्यस्य स्वरूपातिरिक्ताया स्थितेर्निष्क्रियत्वेनाभावात् स्वरूपस्थितिः । तच्च प्रत्येकं व्यक्त्यपायेप्यस्तीति नासौ ततः । अस्तु वा स्थितिः {\color{DodgerBlue3}“साप्ययुक्तैव भेदाभेदविवेचने”} क्रियमाणे । तथा ह्याश्रयहेतुका स्थितिः सामान्याभिन्नाऽभिन्ना वा स्यात् । भिन्ना चेत् तस्याः कारणत्वादाश्रय स्यात् । न तु सामान्यस्य स्थितिरसम्बन्धात् । सामान्यस्यापि चेद् भेदे सति कोऽनयोः सम्बन्धः । न कार्यकारणभावः स्थितेराश्रयादुत्पत्तेः । सामान्यादुत्पत्तौ वा न व्यक्तिराधारः स्यात्(।) स्थितिहेतुत्वाभावात् । सामान्यञ्च\edlabel{pvv.337-5}\footnote{\label{pvv.337-5}  ५ अभेदपक्षे स्थितिक्रिया सामान्यक्रियैव सा चायुक्ता ।} नित्यत्वादकार्यमेव । न चान्यः\edlabel{pvv.337-6}\footnote{\label{pvv.337-6}  ६ व्यक्तिसामान्ययो (:) स्थितेरन्यः ।} सम्बन्धोस्ति निराकरणात् ।
	\pend
      

	  \pstart एवं गमनप्रतिबन्धादिष्वपि वाच्यं । तदेवं न तावदाधेयता वृत्तिः ।
	\pend
      

	  \pstart व्यक्तिरपि न युक्ता (।)
	\pend
      \leavevmode\marginnote{\textenglish{338/s}}
	  \bigskip
	  \begingroup
	  \large
	
	    
	    \stanza[\smallbreak]
	\label{pv.3.145b}\edlabel{pv.3.145b}\flagstanza{\tiny\textenglish{...3.145b}}विज्ञानात्पत्तियोग्यत्वायात्मन्यन्यानुरोधि यत् ॥ १४५ ॥\&[\smallbreak]


	
	  \endgroup
	
	  \bigskip
	  \begingroup
	  \large
	
	    
	    \stanza[\smallbreak]
	\label{pv.3.146}\edlabel{pv.3.146}\flagstanza{\tiny\textenglish{....3.146}}तद् व्यङ्ग्यं योग्यतायाश्च कारणं कारकं मतम् ।&प्रागेवास्य च योग्यत्वे तदपेक्षा न युज्यते ॥ १४६ ॥\&[\smallbreak]


	
	  \endgroup
	\leavevmode\marginnote{\textenglish{67a/MA}}

	  \pstart {\color{DodgerBlue3}“यत्”} स्वरूपेण स्थितमेव (सामान्यं) स्वविषय{\color{DodgerBlue3}“विज्ञानोत्पत्तियोग्यत्वायात्मन्यन्यानुरोधि”} परापेक्षं {\color{DodgerBlue3}“तद् व्यङ्ग्यमु”}च्यते । यच्च योग्यतायाः स्वरूपभूतायाः कारकं\edlabel{pvv.338-1}\footnote{\label{pvv.338-1}  १ दीपादि । पूर्वमयोग्यस्य योग्यत्त्वेनोत्पादनात् ।} तत् कारणं मतं । व्यतिरिक्तयोग्यताकरणे तु न स्यात् सामान्य\edlabel{pvv.338-2}\footnote{\label{pvv.338-2}  २ व्यञ्जकात्तु कारकस्य विशेषो जननमात्रं न ज्ञानजननयोग्यता धूमेनाजनकेनापि न व्यभिचारोत्र वह्निर्जनको धूमजत्वाज्ज्ञानस्यान्यथाग्निस्वलक्षणाकारत्वं स्यात् ।} स्योपलब्धिः । उपलम्भायोग्यस्वभावत्वात् । योग्यतासम्बन्धेप्यनुपलभ्यस्वभावानपायात् तदवस्थोऽनुपलम्भः । अथ व्यञ्जकाभिमतैर्व्विशेषैर्न योग्यता क्रियते तदा सा प्रागेवास्तीति स्यात् । तथा {\color{DodgerBlue3}“च प्रागेवास्य योग्यत्वे तदपेक्षा”} व्यञ्जकापेक्षा {\color{DodgerBlue3}“न युज्यते”} (। १४५, १४६)
	\pend
      \label{div_pvv.3.147}\edlabel{div_pvv.3.147}
	  
	% new div opening: depth here is 2
	
	  \bigskip
	  \begingroup
	  \large
	
	    
	    \stanza[\smallbreak]
	\label{pv.3.147a}\edlabel{pv.3.147a}\flagstanza{\tiny\textenglish{...3.147a}}सामान्यस्याविकार्यस्य तत्सामान्यवतः कुतः ।\&[\smallbreak]


	
	  \endgroup
	

	  \pstart {\color{DodgerBlue3}“तत्”} तस्मात् {\color{DodgerBlue3}“सामान्यवतो”} विशेषस्य सकाशात् {\color{DodgerBlue3}“सामान्यस्याविकार्यस्य”} व्यक्तिरपि कुतः सम्भवति ।
	\pend
      

	  \pstart अथ न सामान्यस्य संस्काराद् व्यक्तिर्व्यञ्जिका किन्त्विन्द्रियस्याञ्जनवदित्याह ।
	\pend
      
	  \bigskip
	  \begingroup
	  \large
	
	    
	    \stanza[\smallbreak]
	\label{pv.3.147b}\edlabel{pv.3.147b}\flagstanza{\tiny\textenglish{...3.147b}}अञ्जनादेरिव व्यक्तेः संस्कारोऽक्षस्य न भवेत् ॥ १४७ ॥\&[\smallbreak]


	
	  \endgroup
	
	  \bigskip
	  \begingroup
	  \large
	
	    
	    \stanza[\smallbreak]
	\label{pv.3.148a}\edlabel{pv.3.148a}\flagstanza{\tiny\textenglish{...3.148a}}तत्प्रतिपत्तेरभिन्नत्वात् तद्भावाभावकालयोः ।\&[\smallbreak]


	
	  \endgroup
	

	  \pstart {\color{DodgerBlue3}“अञ्जनादेरिवे”}न्द्रियस्य {\color{DodgerBlue3}“न”} संस्कारो {\color{DodgerBlue3}“व्यक्तेः”} सकाशात् । {\color{DodgerBlue3}“त\edlabel{pvv.338-3}\footnote{\label{pvv.338-3}  ३ ज्ञानस्य व्यक्त्यसत्वेप्यभेदात् स्पष्टत्वेन ।}त्प्रतिपत्ते”}स्तस्या व्यक्ते\edlabel{pvv.338-4}\footnote{\label{pvv.338-4}  ४ व्यञ्जिकायाः ।} {\color{DodgerBlue3}“र्भावाभावकालयोरभिन्नत्वात्”} । न हि यथेन्द्रियस्याञ्जनसंस्कारभावाभावयोरर्थप्रतिपत्तेः स्पष्टास्पष्टलक्षणो विशेषः तथा व्यक्तिभावाभावकालयोः सामान्यप्रतीतेः सर्व्वदा तुल्याकारत्वात् तस्याः ॥
	\pend
      \label{div_pvv.3.148_3.149_3.150_3.151_3.152_3.153_3.154_3.155_3.156}\edlabel{div_pvv.3.148_3.149_3.150_3.151_3.152_3.153_3.154_3.155_3.156}
	  
	% new div opening: depth here is 2
	

	  \pstart किञ्च ।
	\pend
      
	  \bigskip
	  \begingroup
	  \large
	
	    
	    \stanza[\smallbreak]
	\label{pv.3.148b}\edlabel{pv.3.148b}\flagstanza{\tiny\textenglish{...3.148b}}व्यञ्जकस्य च जातीनां जातिमत्ता यदीष्यते ॥ १४८ ॥\&[\smallbreak]


	
	  \endgroup
	
	  \bigskip
	  \begingroup
	  \large
	
	    
	    \stanza[\smallbreak]
	\label{pv.3.149a}\edlabel{pv.3.149a}\flagstanza{\tiny\textenglish{...3.149a}}प्राप्तो गोत्वादिना तद्वान् प्रदीपादिप्रकाशकः ।\&[\smallbreak]


	
	  \endgroup
	\leavevmode\marginnote{\textenglish{339/s}}

	  \pstart {\color{DodgerBlue3}“व्यञ्जकस्य”}\edlabel{pvv.339-1}\footnote{\label{pvv.339-1}  १ व्यञ्जिकात्वेपि व्यक्तेर्जातिमत्वमयुक्तमतिप्रसङ्गादित्याह ।} विशेषस्य {\color{DodgerBlue3}“जातीनां जातिमत्ता यदीष्यते”} (।) तदा {\color{DodgerBlue3}“गोत्वादिना”} सामान्येन {\color{DodgerBlue3}“प्रदीपादि”}\edlabel{pvv.339-2}\footnote{\label{pvv.339-2}  २ आदिनेन्द्रियमनस्कारादि ।}स्तस्य {\color{DodgerBlue3}“प्रकाशकः तद्वान्”} सामान्यवान् प्राप्तः\edlabel{pvv.339-3}\footnote{\label{pvv.339-3}  ३ शावलेयादिवत् ।} व्यञ्जकलक्षणत्वात् तद्वतः । तस्मान्न वृत्तिरभिव्यक्तिरपि तत्कथमेकवृत्तेरनेकोप्येकशब्दवान् स्यात् ॥
	\pend
      

	  \pstart किञ्च (।) \edlabel{pvv.339-4}\footnote{\label{pvv.339-4}  ४ यद्वा सर्व्वगतत्वेपि व्यक्तिः सत्तानुरोधतः । व्यक्तिशून्येऽदर्शनं व्यञ्जकाभावात् अत एव ।}
	\pend
      
	  \bigskip
	  \begingroup
	  \large
	
	    
	    \stanza[\smallbreak]
	\label{pv.3.149b}\edlabel{pv.3.149b}\flagstanza{\tiny\textenglish{...3.149b}}व्यक्तेरन्याथवानन्या येषां जातिस्तु विद्यते ॥ १४९ ॥\&[\smallbreak]


	
	  \endgroup
	
	  \bigskip
	  \begingroup
	  \large
	
	    
	    \stanza[\smallbreak]
	\label{pv.3.150a}\edlabel{pv.3.150a}\flagstanza{\tiny\textenglish{...3.150a}}तेषां व्यक्तिष्वपूर्वासु कथं सामान्यबुद्धयः ।\&[\smallbreak]


	
	  \endgroup
	

	  \pstart व्यक्तेरन्याऽथवाऽनन्या अव्यतिरिक्ता येषां वै शे षि क सां ख्या दीनां जातिर्व्विद्यते\edlabel{pvv.339-5}\footnote{\label{pvv.339-5}  ५ वस्तुसतीत्यर्थः ।} एवेति मतं (।) तेषां मते व्यक्तिष्वपूर्व्वासु कथं सामान्यबुद्धयः स्युः ।
	\pend
      

	  \pstart तथा हि (।)
	\pend
      
	  \bigskip
	  \begingroup
	  \large
	
	    
	    \stanza[\smallbreak]
	\label{pv.3.150b}\edlabel{pv.3.150b}\flagstanza{\tiny\textenglish{...3.150b}}एकत्र तत्सतोन्यत्र दर्शनासम्भवात् सतः ॥ १५० ॥\&[\smallbreak]


	
	  \endgroup
	
	  \bigskip
	  \begingroup
	  \large
	
	    
	    \stanza[\smallbreak]
	\label{pv.3.151a}\edlabel{pv.3.151a}\flagstanza{\tiny\textenglish{...3.151a}}अनन्यत्वेऽन्वयाभावादन्यत्वेप्यनपाश्रयात् ।\&[\smallbreak]


	
	  \endgroup
	

	  \pstart अनन्यत्वे एकत्र व्यक्तौ तदात्मनः सतः सामान्यस्यान्यत्र व्यक्त्यन्तरेऽ{\color{DodgerBlue3}“न्वयाभावात्”} दर्शनासम्भवात् । कथं सामान्यबुद्धिः । अन्यत्वेपि जातेर{\color{DodgerBlue3}“नपाश्रया”}दुक्तयुक्त्या\edlabel{pvv.339-6}\footnote{\label{pvv.339-6}  ६ आधाराधयेत्वनिषेधात् ।}ऽश्रयत्वाभावात् न व्यक्तिष्वपूर्व्वासु तन्निबन्धना धीः स्यात् ।
	\pend
      

	  \pstart अपि च\edlabel{pvv.339-7}\footnote{\label{pvv.339-7}  ७ “केनचिच्चात्मनैकत्वं नानात्वञ्चास्य केनचिदि”ति \href{http://http://sarit.indology.info/?cref=śv.560}{(श्लोकवा॰ ५६०)} भिन्नाभिन्नपक्षोप्ययुक्त इत्याह । यद्वस्तुत्वे सत्यतद्रूपं तस्य ततोन्यत्वमेव । यथा सुखाद् दुःखस्य । वस्तुत्वे सत्याशक्तिरूपञ्चेष्यते सामान्यमित्येतद्रूपत्वेनान्यत्वे व्यवहारस्य साध्यत्वात् स्वभावहेतुः ॥ तच्चेत् सामान्यस्य रूपमनन्यत्तदेव तद् भवति । अतत्वेऽन्यत्वात् । अनन्यत्वेन्वयाभावस्यैव दोषः ।} सामान्यं स्वाश्रयगतं वा\edlabel{pvv.339-8}\footnote{\label{pvv.339-8}  ८ “पिण्डेष्वेव च सामान्यं नान्तरा गृह्यते यत” इति \href{http://http://sarit.indology.info/?cref=śv.551}{(कुमारिल) भट्टः (श्लोकवार्तिके ५५१)} ।} कल्प्यते सर्व्वगतं वा आकाशादिवत् । तत्र प्रथमपक्षं दूषयितुमाह (।)
	\pend
      
	  \bigskip
	  \begingroup
	  \large
	
	    
	    \stanza[\smallbreak]
	\label{pv.3.151b}\edlabel{pv.3.151b}\flagstanza{\tiny\textenglish{...3.151b}}न याति न च तत्रासीदस्ति पश्चान्न चांशवत् ॥ १५१ ॥\&[\smallbreak]


	
	  \endgroup
	
	  \bigskip
	  \begingroup
	  \large
	
	    
	    \stanza[\smallbreak]
	\label{pv.3.152a}\edlabel{pv.3.152a}\flagstanza{\tiny\textenglish{...3.152a}}जहाति पूर्व्वं नाधारमहो व्यसनसन्ततिः ।\&[\smallbreak]


	
	  \endgroup
	

	  \pstart \leavevmode\marginnote{\textenglish{340/s}}एकव्यक्तिस्थितं सामान्यं व्यक्त्यन्तरमुत्पद्यमानं निष्क्रियत्वा{\color{DodgerBlue3}“न्न याति ।\edlabel{pvv.340-1}\footnote{\label{pvv.340-1}  १ न प्रतिबिम्बवदवस्तुत्वात् “
	    \begin{verse}
	विरुद्धपरिमाणेषु वज्रादर्शतलादिषु ।\\
	    पर्वतादिस्वभावानां भावानां नास्ति सम्भवो\\
	    
	    \end{verse}
	  .” यतः ।} न तत्र”} व्यक्त्युत्पत्तिदेशे प्रागा{\color{DodgerBlue3}“सीत्”} । व्यक्तिमात्रनिष्ठत्वात् । {\color{DodgerBlue3}“अस्ति”} च {\color{DodgerBlue3}“पश्चादु”}त्पत्तेः सामान्यं\edlabel{pvv.340-2}\footnote{\label{pvv.340-2}  २ यैः प्रकारैर्व्यक्त्यन्तरे सम्भवस्तेनेष्यन्ते सामान्यञ्चात्रेति व्याघातः (।) व्यक्तिसमवेतभानमनुत्पादेपीति किमुत्पादेनेति चेन्न । प्रतिभासोऽलीकः सामान्याभावेपीति न तन्मात्रात् सत्त्यत्वमिति नेष्टसिद्धिः ।} सामान्य\edlabel{pvv.340-3}\footnote{\label{pvv.340-3}  ३ न व्यक्तिसहोत्पन्नं नित्यत्वात् । न व्यक्त्युत्पाद एव तदुत्पादाभेदात् ।}शून्याया व्यक्तेः स्थित्यनुपगमा{\color{DodgerBlue3}“न्न चांशवत्”} (।) न हि सामान्यं सावयवं येन क्वचिदेकेनावयवेन समवेतं सद् अवयवान्तरैरुत्पद्यमानव्यक्तिभिः सम्बध्यते । {\color{DodgerBlue3}“न च पूर्व्वो”}त्पन्न{\color{DodgerBlue3}“माधारं जहाति”}\edlabel{pvv.340-4}\footnote{\label{pvv.340-4}  ४ अनंशम्वा} (।) उत्पित्सुव्यक्त्यन्तरेण सम्बन्धार्थमन्यत्र स्थितस्यापरसम्बन्धस्तद्देशागमनं किल न युज्यत {\color{DodgerBlue3}“इत्यहो व्यसनसन्ततिः”} स्वाश्रयगतसामान्यवादिनां ।
	\pend
      

	  \pstart किञ्च (।)
	\pend
      
	  \bigskip
	  \begingroup
	  \large
	
	    
	    \stanza[\smallbreak]
	\label{pv.3.152b}\edlabel{pv.3.152b}\flagstanza{\tiny\textenglish{...3.152b}}अन्यत्र वर्त्तमानस्य ततोन्यस्थानजन्मनि ॥ १५२ ॥\&[\smallbreak]


	
	  \endgroup
	
	  \bigskip
	  \begingroup
	  \large
	
	    
	    \stanza[\smallbreak]
	\label{pv.3.153a}\edlabel{pv.3.153a}\flagstanza{\tiny\textenglish{...3.153a}}स्वस्थानादचलतोन्यत्र वृत्तिरयुक्तिमत् ।\&[\smallbreak]


	
	  \endgroup
	

	  \pstart {\color{DodgerBlue3}“अन्यत्र”} पूर्व्वस्थितायां व्यक्तौ {\color{DodgerBlue3}“वर्त्तमानस्य”} सामान्यस्य स्वस्मात् स्था{\color{DodgerBlue3}“नादा-”} श्रयाद{\color{DodgerBlue3}“चलतः ततः”} पूर्व्वव्यक्ते{\color{DodgerBlue3}“रन्यत्र”} स्थाने {\color{DodgerBlue3}“जन्म”} यस्य तत्र द्रव्ये {\color{DodgerBlue3}“वृत्तिरित्ययुक्तिमत्”} । न हि व्यक्त्यन्तरस्थितस्य व्यक्त्यन्तरमनागच्छतस्तेन सहानुत्पद्यमानस्य च तत्सम्बन्धो युक्तः ।
	\pend
      

	  \pstart तथा\edlabel{pvv.340-5}\footnote{\label{pvv.340-5}  ५ पूर्व्वव्यक्तिदेशादविचलदपि ततोन्यदेशद्रव्यं व्याप्नोतीत्यत्राह ।} (।)
	\pend
      
	  \bigskip
	  \begingroup
	  \large
	
	    
	    \stanza[\smallbreak]
	\label{pv.3.153b}\edlabel{pv.3.153b}\flagstanza{\tiny\textenglish{...3.153b}}यत्रासौ वर्त्तते भावस्ते न संबध्यतेपि न ॥ १५३ ॥\&[\smallbreak]


	
	  \endgroup
	
	  \bigskip
	  \begingroup
	  \large
	
	    
	    \stanza[\smallbreak]
	\label{pv.3.154a}\edlabel{pv.3.154a}\flagstanza{\tiny\textenglish{...3.154a}}तद्देशिनञ्च व्याप्नोति किमप्येतन्महाद्भ्ुतम् ।\&[\smallbreak]


	
	  \endgroup
	

	  \pstart {\color{DodgerBlue3}“यत्र”} देशेऽसौ\edlabel{pvv.340-6}\footnote{\label{pvv.340-6}  ६ पश्चाद्भावी ।} शावलेयादि{\color{DodgerBlue3}“र्भावो वर्त्तते”} तेन देशेन सामान्यं न सम्बध्यते स्वव्यक्तिनिष्ठत्वात् तस्य । {\color{DodgerBlue3}“तद्देशिनं”} सामान्यसम्बन्धरहितो देशो यस्य तं विशेषञ्च \leavevmode\marginnote{\textenglish{67b/MA}} {\color{DodgerBlue3}“व्याप्नोती”}ति {\color{DodgerBlue3}“किमपि महाद्भुतमेतत्”} दुर्बुद्धिविलपितेषु ॥
	\pend
      \leavevmode\marginnote{\textenglish{341/s}}
	  \bigskip
	  \begingroup
	  \large
	
	    
	    \stanza[\smallbreak]
	\label{pv.3.154b}\edlabel{pv.3.154b}\flagstanza{\tiny\textenglish{...3.154b}}व्यक्त्यवैकत्र व्यक्ताऽथ सर्वगा जातिरिष्यते ॥ १५४ ॥\&[\smallbreak]


	
	  \endgroup
	
	  \bigskip
	  \begingroup
	  \large
	
	    
	    \stanza[\smallbreak]
	\label{pv.3.155a}\edlabel{pv.3.155a}\flagstanza{\tiny\textenglish{...3.155a}}सर्वत्र दृश्येताभेदात् सापि न व्यक्त्यपेक्षिणी ।\&[\smallbreak]


	
	  \endgroup
	

	  \pstart {\color{DodgerBlue3}“अथ\edlabel{pvv.341-1}\footnote{\label{pvv.341-1}  १ स्वाश्रयेन्द्रियसंयोगात् पक्षत्वात्तच्छून्येन दृश्यते जातिरिति समाधातुरपि ।} सर्व्वत्रगा”} जातिरिष्यते {\color{DodgerBlue3}“तदैकत्रैव”} देशे {\color{DodgerBlue3}“व्यक्त्या व्यक्ता”} सा जातिः {\color{DodgerBlue3}“सर्व्वत्र”}\edlabel{pvv.341-2}\footnote{\label{pvv.341-2}  २ व्यक्तिशून्येपि नो चेत् स्वभावनानात्वाप्रसङ्गः ।} देशे {\color{DodgerBlue3}“दृश्येताभेदात्”} । एकव्यक्तिव्यक्तं रूपं {\color{DodgerBlue3}“सर्व्वत्र”} विद्यमानमभिन्नमित्युपलब्धिप्रसङ्गः । व्यङ्ग्यव्यञ्जकभावश्चायुक्त इत्युक्तं\edlabel{pvv.341-3}\footnote{\label{pvv.341-3}  ३ वृत्तिराधेयतेत्यादिना ।} (।) भवतु वा तथापि न {\color{DodgerBlue3}“सा”} जाति{\color{DodgerBlue3}“र्व्यक्त्यपेक्षिणी”} ।
	\pend
      

	  \pstart तथा हि ।
	\pend
      
	  \bigskip
	  \begingroup
	  \large
	
	    
	    \stanza[\smallbreak]
	\label{pv.3.155b}\edlabel{pv.3.155b}\flagstanza{\tiny\textenglish{...3.155b}}व्यञ्जकस्याप्रतीतौ न व्यङ्ग्यं सम्यक् प्रतीयते ॥ १५५ ॥\&[\smallbreak]


	
	  \endgroup
	
	  \bigskip
	  \begingroup
	  \large
	
	    
	    \stanza[\smallbreak]
	\label{pv.3.156a}\edlabel{pv.3.156a}\flagstanza{\tiny\textenglish{...3.156a}}विपर्ययः पुनः कस्मादिष्टः सामान्यतद्वतोः ।\&[\smallbreak]


	
	  \endgroup
	

	  \pstart व्यञ्जकस्य\edlabel{pvv.341-4}\footnote{\label{pvv.341-4}  ४ नागृहीतविशेषणा विशेष्ये बुद्धिरिति सामान्यग्रहद्वारा व्यक्तिग्रह इष्टः (।) स च न्यायविपरीतः ।} प्रदीपादेर{\color{DodgerBlue3}“प्रतीतौ न व्यङ्ग्यं”} घटादिः {\color{DodgerBlue3}“संप्रतीयत”} इति तावत् स्थितं । {\color{DodgerBlue3}“सामान्यतद्वतोः पुनः कस्माद् विपर्यय इष्टः”} । यदि व्यक्तिः सामान्यस्य व्यञ्जिका तदा व्यक्तेर्ग्रहे सामान्यं न गृह्येत । इह तु सामान्यस्य व्यङ्ग्यस्य प्राग् ग्रहणं (।) ततस्तद्विशिष्टत्वेन व्यक्तिर्गह्यत इति किमिष्यते । व्यञ्जकाग्रहे व्यङ्ग्याग्रह्णात् ।
	\pend
      

	  \pstart किञ्च (।)
	\pend
      
	  \bigskip
	  \begingroup
	  \large
	
	    
	    \stanza[\smallbreak]
	\label{pv.3.156b}\edlabel{pv.3.156b}\flagstanza{\tiny\textenglish{...3.156b}}पाचकादिष्वभिन्नेन विनाप्यर्थेन वाचकः ॥ १५६ ॥\&[\smallbreak]


	
	  \endgroup
	

	  \pstart यदि सामान्यमेवाभिन्नाभिधानहेतुर्न व्यक्तयः तदा {\color{DodgerBlue3}“पाचकादिष्वभिन्नेनार्थेन”} पाचकत्वादि\edlabel{pvv.341-5}\footnote{\label{pvv.341-5}  ५ पाचकादिरन्यत्रापीत्यनुवृत्तप्रत्ययः ।}सामान्येन {\color{DodgerBlue3}“विनैव”} त्वन्मतेपि पाचकादिशब्दः कथं {\color{DodgerBlue3}“वाचकः”} ।
	\pend
      

	  \pstart प्रत्ययश्चानुयायी । कर्म पचनादि\edlabel{pvv.341-6}\footnote{\label{pvv.341-6}  ६ अध्ययनादि ।} तद्धेतुरिति चेत्\edlabel{pvv.341-7}\footnote{\label{pvv.341-7}  ७ व्यक्तिभ्य एव प्रत्ययोस्तु किं कर्मादिना ।} । आह ।
	\pend
      \label{div_pvv.3.157_3.158_3.159_3.160_3.161_3.162_3.163_3.164_3.165_3.166_3.167_3.168_3.169_3.170_3.171_3.172_3.173}\edlabel{div_pvv.3.157_3.158_3.159_3.160_3.161_3.162_3.163_3.164_3.165_3.166_3.167_3.168_3.169_3.170_3.171_3.172_3.173}
	  
	% new div opening: depth here is 2
	
	  \bigskip
	  \begingroup
	  \large
	
	    
	    \stanza[\smallbreak]
	\label{pv.3.157a}\edlabel{pv.3.157a}\flagstanza{\tiny\textenglish{...3.157a}}भेदान्न हेतुः कर्मास्य न जातिः कर्मसंश्रयात् ।\&[\smallbreak]


	
	  \endgroup
	\leavevmode\marginnote{\textenglish{342/s}}

	  \pstart भेदान्न हेतुः कर्मादि । कर्मापि प्रतिव्यक्त्येकस्यामपि पौर्व्वापर्येण भिन्न\edlabel{pvv.342-1}\footnote{\label{pvv.342-1}  १ भिन्नमभिन्नप्रत्ययहेतुर्न भवतीत्येकं सामान्यमिष्टं तद् यदि भिन्नमपि कर्माभिन्नं प्रत्ययं जनयेत् व्यक्तिभिः कोपराधः कृतः ।}मेव तत्कथं व्यक्तिवदेकाभिधानहेतुः । कर्मसामान्यं\edlabel{pvv.342-2}\footnote{\label{pvv.342-2}  २ कर्मणः पाकक्रियासमवेतं ।} हेतुश्चेदाह । अस्य कर्मणो जातिर्नाभिन्नाऽभिधानहेतुः। तस्याः कर्मसंश्रयात् । कर्मणि\edlabel{pvv.342-3}\footnote{\label{pvv.342-3}  ३ द्रव्यादर्थान्तरभूते ।} समवेतत्वात् तत्रैवाभिधानहेतुता स्यान्न द्रव्ये\edlabel{pvv.342-4}\footnote{\label{pvv.342-4}  ४ पाचकशब्दादिना द्रव्यस्यैव परिच्छेदः ।} ।
	\pend
      

	  \pstart अथ कर्मसामान्यं कर्मणि सम्बद्धं तच्च द्रव्य इति परम्परया तत्सम्बद्धं तन्निमित्तः शब्दो द्रव्ये वर्त्तत इत्याह ।
	\pend
      
	  \bigskip
	  \begingroup
	  \large
	
	    
	    \stanza[\smallbreak]
	\label{pv.3.157b}\edlabel{pv.3.157b}\flagstanza{\tiny\textenglish{...3.157b}}श्रुत्यन्तरनिमित्तत्वात् कर्मणो न निमित्तता ॥ १५७ ॥\&[\smallbreak]


	
	  \endgroup
	
	  \bigskip
	  \begingroup
	  \large
	
	    
	    \stanza[\smallbreak]
	\label{pv.3.158a}\edlabel{pv.3.158a}\flagstanza{\tiny\textenglish{...3.158a}}असंबंधान्न सामान्यं नायुक्तं शब्दकारणात् ।&अतिप्रसंगात् ;\&[\smallbreak]


	
	  \endgroup
	

	  \pstart {\color{DodgerBlue3}“श्रुत्यन्तरस्य”} पाचकादिशब्दादन्यस्य पाक\edlabel{pvv.342-5}\footnote{\label{pvv.342-5}  ५ पाकः पाक इति हि ततः कर्मजातेः स्यान्न पाचक इति ।} इत्यादिशब्दस्य {\color{DodgerBlue3}“निमितत्वा”}त् {\color{DodgerBlue3}“कर्मणो”} द्रव्ये पाचकशब्दवृत्तौ नास्ति निमित्तता परंपरयापि । तथाऽनित्यत्वात् स्थित्यभावाच्च\edlabel{pvv.342-6}\footnote{\label{pvv.342-6}  ६ पचत एव कर्मास्ति न सर्वदा कर्म ।} कर्मणस्तस्मि\edlabel{pvv.342-7}\footnote{\label{pvv.342-7}  ७ कर्मणि}न्निवृत्ते पाचकादिप्रतिपत्ति\edlabel{pvv.342-8}\footnote{\label{pvv.342-8}  ८ कर्मनिमित्तत्वे सति पाचक इति नोच्येत उच्यते च तन्न वस्तुभूतक्रियानिमित्तको व्यपदेशः ।}र्न्न स्यात् । अपचत्यपि तद्व्यपदेशो दृश्यते । नित्यत्वात् कर्मसामान्यं पाचकव्यपदेशे हेतुश्चेदाह (।) असम्बन्धात्\edlabel{pvv.342-9}\footnote{\label{pvv.342-9}  ९ नष्टे कर्मणि सामान्यं न कर्मणि न कर्त्तरीति सम्बद्धसम्बन्धोपि नेत्यसम्बन्धान्न शब्दज्ञानहेतुः ।} । द्रव्येण कर्मसामान्यस्य न सम्बन्धः साक्षात् तत्रासमवायात् । नापि परंपरया कर्मणा तत्समवेतस्य नष्टत्वात्\edlabel{pvv.342-10}\footnote{\label{pvv.342-10}  १० कर्मणः ।}समवायिना सामान्येन सम्बन्धाभावात् । न\edlabel{pvv.342-11}\footnote{\label{pvv.342-11}  ११ असम्बद्धमपि हेतुरित्याह ।}सामान्यं {\color{DodgerBlue3}“शब्द”}स्य पाचकादेः {\color{DodgerBlue3}“कारणान्नायुक्तं”} किंत्वयुक्तमेवाति{\color{DodgerBlue3}“प्रसङ्गात्”} । कर्मत्वस्य सम्बन्धव्यतिरेकेण व्यपदेशहेतुत्वे सर्व्वत्र तथात्वप्र\edlabel{pvv.342-12}\footnote{\label{pvv.342-12}  १२ गोत्वमप्यश्वज्ञानहेतुः स्यात् ।}सङ्गः (१५७) ॥
	\pend
      

	  \pstart अपचत्यपि पुरुषेऽतीतानागतं कर्म तद्व्यपदेशनिमित्तमिति चेत् । आह (।)
	\pend
      \leavevmode\marginnote{\textenglish{343/s}}
	  \bigskip
	  \begingroup
	  \large
	
	    
	    \stanza[\smallbreak]
	\label{pv.3.158b}\edlabel{pv.3.158b}\flagstanza{\tiny\textenglish{...3.158b}}कर्मापि नासज्ज्ञानाभिधानयोः ॥ १५८ ॥\&[\smallbreak]


	
	  \endgroup
	
	  \bigskip
	  \begingroup
	  \large
	
	    
	    \stanza[\smallbreak]
	\label{pv.3.159a}\edlabel{pv.3.159a}\flagstanza{\tiny\textenglish{...3.159a}}अनैमित्तिकतापत्तेः ;\&[\smallbreak]


	
	  \endgroup
	

	  \pstart न कर्मासदपि ज्ञानाभिधानयोर्हेतुरनैमित्तिकताया\edlabel{pvv.343-1}\footnote{\label{pvv.343-1}  १ अहेतुकौ ज्ञानशब्दौ स्यातां ।} आपत्तेः । हेतुं विना भवतोऽहेतुकत्वप्रसङ्गात् (।) अतीतानागतञ्च कर्माविद्यमानत्वादसदेव ।
	\pend
      

	  \pstart पाकादिनिर्वर्त्तिका शक्तिः पाचकादिव्यपदेशहे\edlabel{pvv.343-2}\footnote{\label{pvv.343-2}  २ पाचकस्था ।}तुरिति चेत् । आह (।)
	\pend
      
	  \bigskip
	  \begingroup
	  \large
	
	    
	    \stanza[\smallbreak]
	\label{pv.3.159b}\edlabel{pv.3.159b}\flagstanza{\tiny\textenglish{...3.159b}}न च शक्तिरनन्वयात् ।\&[\smallbreak]


	
	  \endgroup
	

	  \pstart {\color{DodgerBlue3}“शक्ति”}र्हि द्रव्यादभिन्ना भिन्ना वा प्रतिपन्ना । अभिन्ना चेत् तदा द्रव्यव{\color{DodgerBlue3}“दनन्वयान्न”} चान्वयिनोऽन्वयिशब्दहेतुता\edlabel{pvv.343-3}\footnote{\label{pvv.343-3}  ३ अन्वयित्वात् शब्दज्ञानादेरपि ।} । अथ भिन्ना तदा\edlabel{pvv.343-4}\footnote{\label{pvv.343-4}  ४ शक्तेरेवोपयोगाद् द्रव्यानुपयोगासङ्गः ।}ऽनुपकारकयोः सम्बन्धानुपपत्तेर्द्रव्येणोपक्रियमाणा शक्तिस्तत्सम्बन्धिनीति वक्तव्यं (।) अत्रापि द्रव्यं शक्त्यन्तरेण स्वयमेवासमर्थस्वभावतया शक्तिमुपकरोतीति वाच्यं । आद्येऽनवस्था द्वितीये तु कार्यमेव किन्न करोतीत्यलं शक्तिस्वीकारेण\edlabel{pvv.343-5}\footnote{\label{pvv.343-5}  ५ द्रव्यञ्च नान्वेतीत्यन्वयी शब्दो न स्यात् ।} ॥
	\pend
      \leavevmode\marginnote{\textenglish{68a/MA}}
	  \bigskip
	  \begingroup
	  \large
	
	    
	    \stanza[\smallbreak]
	\label{pv.3.159c}\edlabel{pv.3.159c}\flagstanza{\tiny\textenglish{...3.159c}}सामान्यं पाचकत्वादि यदि प्रागेव तद्भवेत् ॥ १५९ ॥\&[\smallbreak]


	
	  \endgroup
	
	  \bigskip
	  \begingroup
	  \large
	
	    
	    \stanza[\smallbreak]
	\label{pv.3.160}\edlabel{pv.3.160}\flagstanza{\tiny\textenglish{....3.160}}व्यक्तं सत्तादिवन्नो चेन्न पश्चादविशेषतः ।&क्रियोपकारापेक्षस्य व्यञ्जकत्वेऽविकारिणः ॥ १६० ॥\&[\smallbreak]


	
	  \endgroup
	
	  \bigskip
	  \begingroup
	  \large
	
	    
	    \stanza[\smallbreak]
	\label{pv.3.161a}\edlabel{pv.3.161a}\flagstanza{\tiny\textenglish{...3.161a}}अतिशये वा ह्यप्यस्य क्षणिकत्वात् क्रिया कुतः ।\&[\smallbreak]


	
	  \endgroup
	

	  \pstart अभिन्नाभिधाने निमित्तं {\color{DodgerBlue3}“सामान्य”}मेव\edlabel{pvv.343-6}\footnote{\label{pvv.343-6}  ६ पाचकादिद्रव्येषु ।} {\color{DodgerBlue3}“पाचकत्वादि यदी”}ष्यते तदा पाकादिक्रियातः {\color{DodgerBlue3}“प्रागेव तत्”} पाचकत्वादि सामान्यं द्रव्योत्पत्तावेव {\color{DodgerBlue3}“व्यक्तं भवेत्”}\edlabel{pvv.343-7}\footnote{\label{pvv.343-7}  ७ यदैव वस्तु तदैव गोत्वादिसत्तादियोगात् यथा सत्ता द्रव्यत्वादि यावद्द्रव्यभावि ।} (।) नो चेत् प्रागव्यक्तं न {\color{DodgerBlue3}“पश्चा”}दपि व्यक्तं {\color{DodgerBlue3}“स्याद\edlabel{pvv.343-8}\footnote{\label{pvv.343-8}  ८ सामान्याश्रयद्रव्यस्य ।}विशेष”}तो विशेषाभावात् (।) नित्यैकरूपस्य सामान्यस्य क्रियायाः कर्मणः पाकादि न {\color{DodgerBlue3}“उपकार”}स्तदपेक्षस्य द्रव्यस्य पाचकत्वादिसामान्य{\color{DodgerBlue3}“व्यञ्जकत्वे”} चेष्यमाणे{\color{DodgerBlue3}“ऽविकारिणः”} स्थिरस्य\edlabel{pvv.343-9}\footnote{\label{pvv.343-9}  ९ अक्षणिकत्वान्न सहकार्यपेक्षा ।} द्रव्यस्यापेक्षा नास्ति । {\color{DodgerBlue3}“अतिशये वा”} प्रागवस्थातः स्वीक्रियमाणे क्षणिकत्वं स्यात् । {\color{DodgerBlue3}“क्षणिकत्वा”}\leavevmode\marginnote{\textenglish{344/s}} दुत्पत्त्यनन्तरविनाशित्वात् {\color{DodgerBlue3}“क्रिया”} कर्म कुतः सम्भवति । उत्पत्त्यनन्तरं स्थितौ हि कर्म कुर्यात् । सा च क्षणिकस्य नास्तीति न तेनोपकारोपि ।
	\pend
      
	  \bigskip
	  \begingroup
	  \large
	
	    
	    \stanza[\smallbreak]
	\label{pv.3.161b}\edlabel{pv.3.161b}\flagstanza{\tiny\textenglish{...3.161b}}तुल्ये भेदे यया जातिः प्रत्यासत्त्या प्रसर्पति ॥ १६१ ॥\&[\smallbreak]


	
	  \endgroup
	
	  \bigskip
	  \begingroup
	  \large
	
	    
	    \stanza[\smallbreak]
	\label{pv.3.162a}\edlabel{pv.3.162a}\flagstanza{\tiny\textenglish{...3.162a}}क्वचिन्नान्यत्र सैवास्तु शब्दज्ञाननिबन्धनम् ।\&[\smallbreak]


	
	  \endgroup
	

	  \pstart वस्तुभूतसामान्यवादिनोपि व्यक्तीनां {\color{DodgerBlue3}“तुल्ये भेदे”} परस्परमेकार्थकरणादिकया {\color{DodgerBlue3}“यया प्रत्यासत्त्या क्वचिद्”} व्यक्तौ शावलेयबाहुलेययो{\color{DodgerBlue3}“र्जातिः प्रसर्प्पति\edlabel{pvv.344-1}\footnote{\label{pvv.344-1}  १ व्याप्य वर्त्तते विना सामान्यं ।} । नान्यत्र कर्कादौ\edlabel{pvv.344-2}\footnote{\label{pvv.344-2}  २ नाश्वविशेषादौ ।}”} । सैव प्रत्यासत्तिरभिन्नस्य {\color{DodgerBlue3}“शब्दज्ञानस्य निबन्धम”}स्तु किं प्रमाणान्तरबाधितजातिस्वीकारेण ॥
	\pend
      

	  \begin{center}%% label @type='head'
	\textbf{ख. सांख्यमतनिरासः}
	\end{center}
	

	  \pstart सां ख्या\edlabel{pvv.344-3}\footnote{\label{pvv.344-3}  ३ तस्माद्व्यावृत्तेरेवैकत्वाध्यवसायाद्भावेष्यन्वय इति स्थितम् ।} प्राहुः ।
	\pend
      
	  \bigskip
	  \begingroup
	  \large
	
	    
	    \stanza[\smallbreak]
	\label{pv.3.162b}\edlabel{pv.3.162b}\flagstanza{\tiny\textenglish{...3.162b}}न निवृत्तिं विहायास्ति यदि भावान्वयोपरः ॥ १६२ ॥\&[\smallbreak]


	
	  \endgroup
	
	  \bigskip
	  \begingroup
	  \large
	
	    
	    \stanza[\smallbreak]
	\label{pv.3.163a}\edlabel{pv.3.163a}\flagstanza{\tiny\textenglish{...3.163a}}एकस्य कार्यमन्यस्य न स्यादत्यन्तभेदतः ।\&[\smallbreak]


	
	  \endgroup
	

	  \pstart {\color{DodgerBlue3}“यदि निवृत्तिं”} विजातीय\edlabel{pvv.344-4}\footnote{\label{pvv.344-4}  ४ यद् बीजस्याङ्कुरजनकं रूपं न तत्पृथिव्यादावस्ति यदस्ति बुद्ध्यारोपितव्यावृत्तत्वं न तज्जनकं निःस्वभावत्वात् । सां ख्य स्तु यज्जनकं रूपं बीजे तच्चेदन्यस्यापि स्यात् मतेन स्वभावेन ततोऽभिन्न इत्यन्वयमाह । भेदेपि केचित् नियता इत्यत्र व्यभिचारः ।}व्यवच्छेदं {\color{DodgerBlue3}“विहाय भावानां”} सत्वरजस्तमःसाम्यावस्थास्वभावप्रकृत्यात्मनाऽद्वय एकरूपोऽ{\color{DodgerBlue3}“न्वयोऽपरो नास्ति”} (।) {\color{DodgerBlue3}“तदेकस्य कार्य”}मङ्कुरो{\color{DodgerBlue3}“ऽन्यस्य”} क्षित्यादेर्न {\color{DodgerBlue3}“स्यात्”} । बीजक्षित्यादीनाम{\color{DodgerBlue3}“त्यन्तभेदतः”} । अङ्कुरकारकं यद् रूपं बीजस्य तच्चेत् क्षित्यादेर्न्नास्ति न स्यादसौ तत्कारकः । अस्तित्वे चैकरूपान्वयः (।)
	\pend
      

	  \pstart सिद्धान्तवाद्याह ।
	\pend
      
	  \bigskip
	  \begingroup
	  \large
	
	    
	    \stanza[\smallbreak]
	\label{pv.3.163b}\edlabel{pv.3.163b}\flagstanza{\tiny\textenglish{...3.163b}}यद्येकात्मतयानेकः कार्यस्यैकस्य कारकः ॥ १६३ ॥\&[\smallbreak]


	
	  \endgroup
	
	  \bigskip
	  \begingroup
	  \large
	
	    
	    \stanza[\smallbreak]
	\label{pv.3.164a}\edlabel{pv.3.164a}\flagstanza{\tiny\textenglish{...3.164a}}आत्मैकत्रापि वास्तीति व्यर्थाः स्युः सहकारिणः ।\&[\smallbreak]


	
	  \endgroup
	

	  \pstart {\color{DodgerBlue3}“यद्येकात्मतयाऽनेको”} बीजश(?स)लिलादिः {\color{DodgerBlue3}“कार्यस्या”}ङ्कुर{\color{DodgerBlue3}“स्यैकस्य कारकस्त-”} दाङ्कुरकारक {\color{DodgerBlue3}“आत्मा”}ऽनुयायी । {\color{DodgerBlue3}“एकत्रापि”} बीजेऽन्यत्र वा{\color{DodgerBlue3}“स्तीति व्यर्थाः सहकारिणः स्युः”} एकत्र विद्यमानेनैव तेनात्मना कार्योत्पत्तेः ।
	\pend
      \leavevmode\marginnote{\textenglish{345/s}}
	  \bigskip
	  \begingroup
	  \large
	
	    
	    \stanza[\smallbreak]
	\label{pv.3.164b}\edlabel{pv.3.164b}\flagstanza{\tiny\textenglish{...3.164b}}नापैत्यभिन्नं तद् रूपं विशेषाः खल्वपायिनः ॥ १६४ ॥\&[\smallbreak]


	
	  \endgroup
	
	  \bigskip
	  \begingroup
	  \large
	
	    
	    \stanza[\smallbreak]
	\label{pv.3.165a}\edlabel{pv.3.165a}\flagstanza{\tiny\textenglish{...3.165a}}एकापाये फलाभावाद् विशेषेभ्यस्तदुद्भवः ।\&[\smallbreak]


	
	  \endgroup
	

	  \pstart सह\edlabel{pvv.345-1}\footnote{\label{pvv.345-1}  १ एतदेव द्रढयन्नाह ।}कारिणामभावेपि बीजसत्वे{\color{DodgerBlue3}“ऽभिन्नं तत्”} प्रधानं\edlabel{pvv.345-2}\footnote{\label{pvv.345-2}  २ सामान्यं ।} कर्त्तृ{\color{DodgerBlue3}“रूपं नापैति”}\edlabel{pvv.345-3}\footnote{\label{pvv.345-3}  ३ अस्य न विशेषो विशेषेऽभेदहानेः ।} नित्यत्वात्\edlabel{pvv.345-4}\footnote{\label{pvv.345-4}  ४ एकस्य अप्यन्ते ।} व्यक्ति\edlabel{pvv.345-5}\footnote{\label{pvv.345-5}  ५ त्रैगुण्यस्य सर्व्वात्मना सर्व्वत्र सर्वदा सत्त्वात् ।}सत्वाच्च । {\color{DodgerBlue3}“विशे\edlabel{pvv.345-6}\footnote{\label{pvv.345-6}  ६ व्यक्तिभेदाः ।}षाः खलु नूनमपायिनो”} न प्रधानं ।\edlabel{pvv.345-7}\footnote{\label{pvv.345-7}  ७ अत एकव्यक्तिसत्त्वेपि कार्यं स्यान्न चास्त्यतः ।} {\color{DodgerBlue3}“एकस्य”} सहकारिणो विशेषस्या{\color{DodgerBlue3}“पाये”}पि फलस्या{\color{DodgerBlue3}“भावात् । विशेषेभ्यस्त”}स्य कार्यस्य {\color{DodgerBlue3}“उद्भवो”} निश्चीयते न प्रधानात् (।) विशेषस्यान्वय\edlabel{pvv.345-8}\footnote{\label{pvv.345-8}  ८ पाथपृथिव्यादेरङ्कुरेण ।}व्यतिरेकानुविधानात् । सामान्यस्य\edlabel{pvv.345-9}\footnote{\label{pvv.345-9}  ९ एकविशेषस्थितावप्यविकलस्य ।} विपर्ययात् ।
	\pend
      
	  \bigskip
	  \begingroup
	  \large
	
	    
	    \stanza[\smallbreak]
	\label{pv.3.165b}\edlabel{pv.3.165b}\flagstanza{\tiny\textenglish{...3.165b}}स पारमार्थिको भावो य एवार्थक्रियाक्षमः ॥ १६५ ॥\&[\smallbreak]


	
	  \endgroup
	
	  \bigskip
	  \begingroup
	  \large
	
	    
	    \stanza[\smallbreak]
	\label{pv.3.166a}\edlabel{pv.3.166a}\flagstanza{\tiny\textenglish{...3.166a}}स च नान्वेति योन्वेति न तस्मात् कार्यसम्भवः ।\&[\smallbreak]


	
	  \endgroup
	

	  \pstart स एव {\color{DodgerBlue3}“पारमार्थिको भावो योर्थक्रियाक्षमः । स च”} विशेषोर्थक्रियाक्षमो {\color{DodgerBlue3}“नान्वेति”} परस्परं भेदात् (।) {\color{DodgerBlue3}“योऽन्वेति”} प्रधानाख्यो भावो {\color{DodgerBlue3}“न तस्मात् कार्यस्य सम्भवः”} ।
	\pend
      

	  \pstart यद्येकरूपा\edlabel{pvv.345-10}\footnote{\label{pvv.345-10}  १० प्रधानस्य ।}ननुगमस्तदा विशेषेष्वपि केचिज्जनयन्ति नापर इति कुतोयं\edlabel{pvv.345-11}\footnote{\label{pvv.345-11}  ११ सर्व्व सामान्यं निरस्य पृथिव्यादेर्जनकत्वे न कोप्यजनक इत्याशयः ।} विभाग इत्याह ।
	\pend
      
	  \bigskip
	  \begingroup
	  \large
	
	    
	    \stanza[\smallbreak]
	\label{pv.3.166b}\edlabel{pv.3.166b}\flagstanza{\tiny\textenglish{...3.166b}}तेनात्मनापि भेदे हि हेतुः कश्चिन्न चापरः ॥ १६६ ॥\&[\smallbreak]


	
	  \endgroup
	
	  \bigskip
	  \begingroup
	  \large
	
	    
	    \stanza[\smallbreak]
	\label{pv.3.167a}\edlabel{pv.3.167a}\flagstanza{\tiny\textenglish{...3.167a}}स्वभावोयमभेदे तु स्यातां नाशोद्भवौ सकृत् ।\&[\smallbreak]


	
	  \endgroup
	

	  \pstart येनैक\edlabel{pvv.345-12}\footnote{\label{pvv.345-12}  १२ “एकस्य कार्यमन्यस्य न स्यादत्यन्तभेवत” \href{http://http://sarit.indology.info/?cref=pv.3.163}{(३।१६३)} इत्यत्राह । “ज्वरादिशमन” \cref{pv.3.73} इत्याद्युक्तेप्यधिकार्थं ।}रूपान्वये सहकारिवैकल्यप्रसङ्ग{\color{DodgerBlue3}“स्तेन”} कारणेना{\color{DodgerBlue3}“त्मना”} स्वरूपेण बीजजलशिलानलादीनां {\color{DodgerBlue3}“भेदेपि कश्चित्”} क्षितिश(? स)लिलबीजादिरङ्कुरस्य {\color{DodgerBlue3}“हेतुर्न चापरः”} शिलानलादिः । {\color{DodgerBlue3}“हि”} यस्मात् {\color{DodgerBlue3}“स्वभावोयम”}ङ्कुरजननशक्तः । {\color{DodgerBlue3}“तदितरश्च”} प्रमाणदृष्टो बीजादीनां शिलादीनाञ्च नात्र पर्यनुयोगावतारः । कुत उत्पन्न इति चेत् । स्वस्वहेतोः । सोपि कस्मात् तज्जनक इति चेत् । स्वहेतोः (।) {\color{DodgerBlue3}“तत्रापि”} \leavevmode\marginnote{\textenglish{346/s}} {\color{DodgerBlue3}“प्रश्ने तदेवोत्तरं”} । अनादिश्च हेतुफलपरंपरैकरूपानुगमाद्(।) विशेषाणा{\color{DodgerBlue3}“मभेदे\edlabel{pvv.346-1}\footnote{\label{pvv.346-1}  १ भिन्नानां कश्चिद्धेतुर्नान्यः स्वभावादित्यत्र न किञ्चिद् बाधकं भवतु बाधेत्याह ।}तु”} \leavevmode\marginnote{\textenglish{68b/MA}} स्वीक्रियमाणे {\color{DodgerBlue3}“सकृदे”}कस्य {\color{DodgerBlue3}“नाशोद्भवौ\edlabel{pvv.346-2}\footnote{\label{pvv.346-2}  २ उपलक्षणमेतत् सूत्रे सर्व्वत्र सर्व्वोत्पादिरपि ज्ञेयः ।} स्यातां”} (।) विशेषान्तरयोर्यथासंभवं नाशो जन्मनि च तदपरस्य विशेषस्य तदभिन्नरूपतया नाशोद्भवौ स्याता\edlabel{pvv.346-3}\footnote{\label{pvv.346-3}  ३ व्यपदेशोपि सर्व्वं हि वस्तु रूपेण भिद्यते \par
न परस्परं । तन्मतं सामान्यं सर्व्वत्र स्थितं विशेषा नश्यन्ति ।} (।)
	\pend
      
	  \bigskip
	  \begingroup
	  \large
	
	    
	    \stanza[\smallbreak]
	\label{pv.3.167b}\edlabel{pv.3.167b}\flagstanza{\tiny\textenglish{...3.167b}}भेदोपि तेन नैवं चेद् य एकस्मिन् विनश्यति ॥ १६७ ॥\&[\smallbreak]


	
	  \endgroup
	
	  \bigskip
	  \begingroup
	  \large
	
	    
	    \stanza[\smallbreak]
	\label{pv.3.168a}\edlabel{pv.3.168a}\flagstanza{\tiny\textenglish{...3.168a}}तिष्ठत्यात्मा न तस्यातो न स्यात् सामान्यभेदधीः ।\&[\smallbreak]


	
	  \endgroup
	

	  \pstart अथ परस्परं विशेषाणामेकान्तेन नाभेदः (।) किन्तर्हि {\color{DodgerBlue3}“भेदो”} विशेषरूपतया । {\color{DodgerBlue3}“तेन”} भेदे{\color{DodgerBlue3}“नैवं”} सकृन्नाशोद्भवप्रसङ्गो {\color{DodgerBlue3}“न चेत्”} । यद्येक{\color{DodgerBlue3}“स्मिन्”} विशेषे {\color{DodgerBlue3}“विनश्यति”} (।) {\color{DodgerBlue3}“तिष्ठत्यात्मा”}नुयायी प्रधानाख्यस्तदा स {\color{DodgerBlue3}“तस्य”} विशेषस्यात्मा न भवति (।) न हि यस्मिन् विनश्यति यो {\color{DodgerBlue3}“न”} नष्टः स तस्य स्वभावो विरुद्धधर्माध्यासलक्षणत्वादभेदस्य । {\color{DodgerBlue3}“अतः”} परस्परं सर्व्वथा भेदा{\color{DodgerBlue3}“न्न स्यात् सामान्यभेदधीः”}\edlabel{pvv.346-4}\footnote{\label{pvv.346-4}  ४ परस्परसम्बद्धा} । अन्वयि सामान्यमनन्वयी भेद उच्यते (।) यदा च सामान्यं भेदाद् भिन्नं तदाऽनुगामिव्यक्तिस्वरूपं न सामान्यं (।) न च तदात्मको भेदः । अतस्तद्बुद्धिरपि न भवेत् ।
	\pend
      

	  \pstart नन्वन्यनिवृत्तावपि समानमेतत् । भावे नश्यति अन्यनिवृत्तिर्नश्यति न वा । यदि नश्यति न स्याद् व्यक्त्यन्तरे तद्बुद्धिः (।) अथ न नश्यति तदाऽत्यन्तभेदात सामान्यभेदबुद्धिर्न स्यादित्याह ।
	\pend
      
	  \bigskip
	  \begingroup
	  \large
	
	    
	    \stanza[\smallbreak]
	\label{pv.3.168b}\edlabel{pv.3.168b}\flagstanza{\tiny\textenglish{...3.168b}}निवृत्तेर्निःस्वभावत्वात् न स्थानास्थानकल्पना ॥ १६८ ॥\&[\smallbreak]


	
	  \endgroup
	
	  \bigskip
	  \begingroup
	  \large
	
	    
	    \stanza[\smallbreak]
	\label{pv.3.169a}\edlabel{pv.3.169a}\flagstanza{\tiny\textenglish{...3.169a}}उपप्लवश्च सामान्यधियस्तेनाप्यदूषणा ।\&[\smallbreak]


	
	  \endgroup
	

	  \pstart {\color{DodgerBlue3}“निवृत्ते\edlabel{pvv.346-5}\footnote{\label{pvv.346-5}  ५ विजातिव्यावृत्तदाह्यं स्वाकाराभेदेनाध्यस्तं शब्दविकल्पविषयमपोहं मुक्त्वाऽन्यनिवृत्तिमात्रे प्राह (दिग्)नागाभिमते वा ।}र्निःस्वभावत्वात् न स्थानास्थानयोः”} स्थितिनिवृत्त्योः {\color{DodgerBlue3}“कल्पना”} युक्ता । कल्पिता ह्यन्यनिवृत्तिः निःस्वभावा सा किं भावाद् भिन्नाऽभिन्ना वेति न युक्ता कल्पना । न हि शशविषाणं भिन्नमभिन्नम्वेति युक्तं कल्पियितुं\edlabel{pvv.346-6}\footnote{\label{pvv.346-6}  ६ विकल्पबुद्ध्यारोपितसामान्येपि विशेषे नश्यतीत्यादिनेत्याह यः सामान्याकारोऽनेकपदार्थाभिन्नः स भ्रान्तो बहिर्वास्ति ।} । उपप्लवो \leavevmode\marginnote{\textenglish{347/s}} मिथ्यात्वञ्च सामान्यधियो विषयाभावात् (।) तेन\edlabel{pvv.347-1}\footnote{\label{pvv.347-1}  १ न प्लवत्वेन हेतुना नास्या युक्तदूषणं सामान्यबुद्धौ ।} नाप्यदूषणा ॥
	\pend
      
	  \bigskip
	  \begingroup
	  \large
	
	    
	    \stanza[\smallbreak]
	\label{pv.3.169b}\edlabel{pv.3.169b}\flagstanza{\tiny\textenglish{...3.169b}}यत्तस्य जनकं रूपं ततोन्यो जनकः कथम् ॥ १६९ ॥\&[\smallbreak]


	
	  \endgroup
	
	  \bigskip
	  \begingroup
	  \large
	
	    
	    \stanza[\smallbreak]
	\label{pv.3.170a}\edlabel{pv.3.170a}\flagstanza{\tiny\textenglish{...3.170a}}भिन्ना विशेषा जनकाः ;\&[\smallbreak]


	
	  \endgroup
	

	  \pstart \edlabel{pvv.347-2}\footnote{\label{pvv.347-2}  २ तुल्यदोषत्वमपनीय प्रकारान्तरेण प्रक्रान्तचोद्यापनया(या)ह ।}ननूक्तं यदि भावाः सर्व्वथा भिन्नास्तदा {\color{DodgerBlue3}“यत्तस्य”} बीजस्य {\color{DodgerBlue3}“जनकं रूपं न”} तदन्यस्य क्षित्यादेरिति {\color{DodgerBlue3}“ततोन्यो जनकः कथ”}मिति । अत्रोत्तरमप्यु\edlabel{pvv.347-3}\footnote{\label{pvv.347-3}  ३ “एकापाये फलाभावाद् विशेषेभ्यस्तदुद्भव” \href{http://http://sarit.indology.info/?cref=pv.3.165}{(३।१६५)} इति प्रसाधनं प्रागुक्तं स्मारयति ।}क्तमभिन्नरूपान्वयव्यतिरेकानुविधानाभावाद् भिन्ना\edlabel{pvv.347-4}\footnote{\label{pvv.347-4}  ४ नात्र विरोधोस्ति ।} {\color{DodgerBlue3}“विशेषा”} एव {\color{DodgerBlue3}“जनका”}स्तदन्वयव्यतिरेकानुविधानात् कार्यस्य सर्व्व एव ते तत्कार्यस्योत्पादकतयोत्पन्नास्तदुत्पादयन्ति ।
	\pend
      
	  \bigskip
	  \begingroup
	  \large
	
	    
	    \stanza[\smallbreak]
	\label{pv.3.170b}\edlabel{pv.3.170b}\flagstanza{\tiny\textenglish{...3.170b}}अप्यभेदोपि तेषु चेत् ।&तेन तेऽजनकाः प्रोक्ताः ;\&[\smallbreak]


	
	  \endgroup
	

	  \pstart तेषु विशेषेष्वन्वयिना रूपेणाभेदोप्यस्ति तेनैव ते जनका इति चेत्(।) तेनान्वयिरूपेण ते विशेषा अजनकाः प्रोक्ताः\edlabel{pvv.347-5}\footnote{\label{pvv.347-5}  ५ सामान्यान्वयव्यतिरेकाननुविधानादिति नापैत्यभिन्नमित्यादिना ।} एकत्रापि विशेषे तद्भावात् सहकारिवैफल्यप्रसक्तेः । तदन्वयव्यतिरेकानर्थविधानाद् विशेषस्य विपर्ययाच्चेति\edlabel{pvv.347-6}\footnote{\label{pvv.347-6}  ६ “स्यातां नाशोद्भवौ सकृदि”त्यादिना \href{http://http://sarit.indology.info/?cref=pv.3.167}{(३।१६७)} भेदं प्रसाध्य प्रतिभासभेदेनापि भेदमाह ।} ।
	\pend
      

	  \pstart किञ्च (।)
	\pend
      
	  \bigskip
	  \begingroup
	  \large
	
	    
	    \stanza[\smallbreak]
	\label{pv.3.170c}\edlabel{pv.3.170c}\flagstanza{\tiny\textenglish{...3.170c}}प्रतिभासोपि भेदकः ॥ १७० ॥\&[\smallbreak]


	
	  \endgroup
	
	  \bigskip
	  \begingroup
	  \large
	
	    
	    \stanza[\smallbreak]
	\label{pv.3.171a}\edlabel{pv.3.171a}\flagstanza{\tiny\textenglish{...3.171a}}अनन्यभाक् स एवार्थस्तस्य व्यावृत्तयोपरे ।\&[\smallbreak]


	
	  \endgroup
	

	  \pstart {\color{DodgerBlue3}“प्रतिभासोप्यनन्यभाग्\edlabel{pvv.347-7}\footnote{\label{pvv.347-7}  ७ प्रतिव्यक्तिभिन्नः ।} भेदको”} विशेषाणामसंकीर्ण्णरूपव्यवस्थापनात् । अपि शब्दादुत्पत्तिस्थितिविनाशादयश्च समानाः । ततश्च विशेष एवार्थः पारमार्थिकः । ये {\color{DodgerBlue3}“त्वपरे”} सामान्यादयो धर्मास्ते {\color{DodgerBlue3}“तस्य”} शेषस्य विजातीयाद् {\color{DodgerBlue3}“व्यावृत्तयः”} कल्पिता अनर्थक्रियाकारिणः ।
	\pend
      \leavevmode\marginnote{\textenglish{348/s}}
	  \bigskip
	  \begingroup
	  \large
	
	    
	    \stanza[\smallbreak]
	\label{pv.3.171b}\edlabel{pv.3.171b}\flagstanza{\tiny\textenglish{...3.171b}}तत् कार्यं कारणञ्चोक्तं तत्स्वलक्षणमिष्यते ॥ १७१ ॥\&[\smallbreak]


	
	  \endgroup
	
	  \bigskip
	  \begingroup
	  \large
	
	    
	    \stanza[\smallbreak]
	\label{pv.3.172a}\edlabel{pv.3.172a}\flagstanza{\tiny\textenglish{...3.172a}}तत्त्यागाप्तिफलाः सर्वाः पुरुषाणां प्रवृत्तयः ।\&[\smallbreak]


	
	  \endgroup
	

	  \pstart अर्थक्रियाकारि तु\edlabel{pvv.348-1}\footnote{\label{pvv.348-1}  १ विशेषरूपं ।} यद् रूपं तत्कार्य {\color{DodgerBlue3}“कारणञ्चोक्तं तत्स्वलक्षणमिष्यते (।) तस्य त्यागाप्तिस्तत्फलाः पुरुषाणाम”}र्थानर्थप्राप्तिपरिहारैषिणां {\color{DodgerBlue3}“प्रवृत्तयः सर्व्वाः”} ।
	\pend
      

	  \pstart किञ्च\edlabel{pvv.348-2}\footnote{\label{pvv.348-2}  २ एवं मी मां स का दि मतेन प्रातिभासिकं सामान्यं निरस्यानुमानिकं पूर्व्वोक्तं प्रत्याह ।} (।)
	\pend
      
	  \bigskip
	  \begingroup
	  \large
	
	    
	    \stanza[\smallbreak]
	\label{pv.3.172b}\edlabel{pv.3.172b}\flagstanza{\tiny\textenglish{...3.172b}}यथाऽभेदाविशेषेपि न सर्वं सर्व्वसाधनम् ॥ १७२ ॥\&[\smallbreak]


	
	  \endgroup
	
	  \bigskip
	  \begingroup
	  \large
	
	    
	    \stanza[\smallbreak]
	\label{pv.3.173a}\edlabel{pv.3.173a}\flagstanza{\tiny\textenglish{...3.173a}}तथाऽभेदाविशेषेपि न सर्व्वं सर्व्वसाधनम् ।\&[\smallbreak]


	
	  \endgroup
	

	  \pstart सां ख्य स्यापि मते {\color{DodgerBlue3}“यथाऽभेद”}स्य प्रधानात्मतया सर्व्वभावेष्व{\color{DodgerBlue3}“विशेषेपि न सर्व्व”} व्यक्तं {\color{DodgerBlue3}“सर्व्व”}स्य कार्यस्य {\color{DodgerBlue3}“साधनं”} हेतुः । {\color{DodgerBlue3}“तथा भेदाविशेषेपि\edlabel{pvv.348-3}\footnote{\label{pvv.348-3}  ३ बौद्धस्य ।} न सर्व्व”} क्षितिबीजानलशिलादिकं {\color{DodgerBlue3}“सर्व्व”}स्य तापाङ्कुरादेः {\color{DodgerBlue3}“साधनं”} ।
	\pend
      
	  \bigskip
	  \begingroup
	  \large
	
	    
	    \stanza[\smallbreak]
	\label{pv.3.173b}\edlabel{pv.3.173b}\flagstanza{\tiny\textenglish{...3.173b}}भेदे हि कारकं किञ्चिद् वस्तुधर्मतया भवेत् ॥ १७३ ॥\&[\smallbreak]


	
	  \endgroup
	

	  \pstart विशेषान्तराद् {\color{DodgerBlue3}“भेदे हि”} सति {\color{DodgerBlue3}“वस्तुधर्मतया”}\edlabel{pvv.348-4}\footnote{\label{pvv.348-4}  ४ अभेदे तु तस्य सर्व्वत्राभिन्नत्वात् क्रियाक्रिये विरुद्धे ।} वस्तुस्वभावत्वात् {\color{DodgerBlue3}“किञ्चिद्”} वस्तु {\color{DodgerBlue3}“कारकं भवेत्”} । न सर्व्वमिति युक्तं स्वहेतुबलायातत्वाद् भिन्नशक्तिकत्वस्य ।(१७३)
	\pend
      \label{div_pvv.3.174}\edlabel{div_pvv.3.174}
	  
	% new div opening: depth here is 2
	
	  \bigskip
	  \begingroup
	  \large
	
	    
	    \stanza[\smallbreak]
	\label{pv.3.174a}\edlabel{pv.3.174a}\flagstanza{\tiny\textenglish{...3.174a}}अभेदे तु विरुध्येते तस्यैकस्य क्रियाक्रिये ।\&[\smallbreak]


	
	  \endgroup
	

	  \pstart {\color{DodgerBlue3}“अभेदे तु”} सर्व्वभवानां {\color{DodgerBlue3}“विरुध्येते तस्यै\edlabel{pvv.348-5}\footnote{\label{pvv.348-5}  ५ त्रैगुण्यस्य बीजदहनाद्यात्मकस्य ।}कस्या”}ङ्कुरादेः {\color{DodgerBlue3}“क्रियाक्रिये”} बीजरूपतया प्रधानमङ्कुरकारकमकारकञ्च दहनरूपतया । विप्रतिषिद्धञ्चैतत् \leavevmode\marginnote{\textenglish{69a/MA}} कथमेकस्य युक्तं ॥
	\pend
      
	  \bigskip
	  \begingroup
	  \large
	
	    
	    \stanza[\smallbreak]
	\label{pv.3.174b}\edlabel{pv.3.174b}\flagstanza{\tiny\textenglish{...3.174b}}भेदोप्यस्त्यक्रियातश्चेत् न कुर्युः सहकारिणः ॥ १७४ ॥\&[\smallbreak]


	
	  \endgroup
	

	  \pstart व्यक्तीनां मिथो व्यक्तिरूपतया {\color{DodgerBlue3}“भेदोप्यस्ति”} । व्यक्त्यन्तरसाध्यस्य कार्यस्या{\color{DodgerBlue3}“क्रियातश्चेत्”} । तदा भेदात् {\color{DodgerBlue3}“सहकारिणः”} कार्यं {\color{DodgerBlue3}“न कुर्युः”} । एकस्य यः कारकः स्वभावस्तस्यान्यत्राभावात् । (१७४)
	\pend
      \label{div_pvv.3.175}\edlabel{div_pvv.3.175}
	  
	% new div opening: depth here is 2
	
	  \bigskip
	  \begingroup
	  \large
	
	    
	    \stanza[\smallbreak]
	\label{pv.3.175a}\edlabel{pv.3.175a}\flagstanza{\tiny\textenglish{...3.175a}}पर्यायेणाथ कर्त्तृत्वं स किं तस्यैव वस्तुनः ।\&[\smallbreak]


	
	  \endgroup
	\leavevmode\marginnote{\textenglish{349/s}}

	  \pstart अथ\edlabel{pvv.349-1}\footnote{\label{pvv.349-1}  १ यवबीजञ्चे (त्) शाल्यङ्कुरकारणं पर्यायेण यदा तच्छालिबीजत्वेन परिणमेत् । प्रधानशक्त्यधिष्ठितभेदपरिणामे प्रधानशक्तेर्भेदत्वेन परिणामे वा सर्व्वत्र सर्व्वोपयोगात् ।} कारकैकस्वभावानुगमात् सहकारिणां {\color{DodgerBlue3}“पर्यायेण कर्त्तृत्वं”} । स पर्यायस्तस्यान्वयिन एकस्य \edlabel{pvv.349-2}\footnote{\label{pvv.349-2}  २ शालिबीजस्य यवबीजत्वेन परिणामो न युक्त इत्यर्थः भेदाधिष्ठानत्वात् पर्यायस्य ।} {\color{DodgerBlue3}“वस्तुनः किं”} (कस्मात्) युक्तः । न ह्येकं वस्तु कार्य कुर्व्वत पर्यायेण न करोतीति युक्तं व्यपदेष्टुं तेनैव क्रियमाणत्वात् । सहकारिणां बहूनां पर्यायेण क्रिया सम्भाव्यते । तेषु चैकस्य कारकं रूपमन्यत्र नास्तीति न ते कुर्युः । सर्व्वेषाञ्च स्वहेतुवलायात एककार्यकारकः स्वभावोऽस्माभिरिव सां ख्यै र्नेष्यते॥
	\pend
      

	  \pstart किञ्च\edlabel{pvv.349-3}\footnote{\label{pvv.349-3}  ३ परिणामं निरस्याभिन्नं भिन्नं भिन्नाभिन्नं सर्व्वसु चोत्तरोत्तरावस्यास्वनुयायित्वादूर्ध्ववृत्तिर्वा समं सर्व्वासु व्यक्तिष्वनुयायित्वात् तिर्यग्वृत्ति वा सामान्यं सांख्यमीमांसकनैयायिकादेर्दू षयितुमाह किञ्चेति ।} ।
	\pend
      
	  \bigskip
	  \begingroup
	  \large
	
	    
	    \stanza[\smallbreak]
	\label{pv.3.175b}\edlabel{pv.3.175b}\flagstanza{\tiny\textenglish{...3.175b}}अत्यन्तभेदाभेदौ तु स्यातां तद्वति वस्तुनि ॥ १७५ ॥\&[\smallbreak]


	
	  \endgroup
	

	  \pstart {\color{DodgerBlue3}“तद्वति”} सामान्यविशेषवति {\color{DodgerBlue3}“वस्तुनि”} स्वीक्रियमाणे{\color{DodgerBlue3}“ऽत्यन्तमे”}कान्तेन सामान्यविशेषयोर्भेदाभेदो {\color{DodgerBlue3}“स्यातां”} । सामान्यस्वरूपत्वाद् भेदस्य सामान्यमेव भवेन्न भेदो भेदात्मत्वात् सामान्यस्य भेद एव भवेत् सामान्यं ।\edlabel{pvv.349-4}\footnote{\label{pvv.349-4}  ४ निरस्तमभिन्नं ।}(१७५)
	\pend
      \label{div_pvv.3.176_3.177_3.178_3.179_3.180_3.181_3.182_3.183_3.184ab}\edlabel{div_pvv.3.176_3.177_3.178_3.179_3.180_3.181_3.182_3.183_3.184ab}
	  
	% new div opening: depth here is 2
	

	  \pstart अथ तयोः कथञ्चन\edlabel{pvv.349-5}\footnote{\label{pvv.349-5}  ५ स्वभावाभेदेपि लक्षणभेदात् अवस्थायाः ।} भेदोप्यस्तीति चेदाह (।)
	\pend
      
	  \bigskip
	  \begingroup
	  \large
	
	    
	    \stanza[\smallbreak]
	\label{pv.3.176a}\edlabel{pv.3.176a}\flagstanza{\tiny\textenglish{...3.176a}}अन्योन्यं वा तयोर्भेदः सदृशासदृशात्मनोः ।\&[\smallbreak]


	
	  \endgroup
	

	  \pstart {\color{DodgerBlue3}“तयोः”} सामान्यविशेषयोः {\color{DodgerBlue3}“सदृशासदृशात्मनोः \edlabel{pvv.349-6}\footnote{\label{pvv.349-6}  ६ अनुगतव्यावृत्तयोरश्लेषात् ।} साधार”}णासाधारणस्वरूपयोरन्योन्यं {\color{DodgerBlue3}“भेद”} एव {\color{DodgerBlue3}“वा”} भवेत् । न कथञ्चिदेकत्वं ।
	\pend
      

	  \pstart एतदेव स्फुटयितुं पूर्व्वपक्षयति (।)
	\pend
      
	  \bigskip
	  \begingroup
	  \large
	
	    
	    \stanza[\smallbreak]
	\label{pv.3.176b}\edlabel{pv.3.176b}\flagstanza{\tiny\textenglish{...3.176b}}तयोरपि भवेद् भेदो यदि येनात्मना तयोः ॥ १७६ ॥\&[\smallbreak]


	
	  \endgroup
	
	  \bigskip
	  \begingroup
	  \large
	
	    
	    \stanza[\smallbreak]
	\label{pv.3.177a}\edlabel{pv.3.177a}\flagstanza{\tiny\textenglish{...3.177a}}भेदः सामान्यमित्येतद् यदि भेदस्तदात्मना ।&भेद एव;\&[\smallbreak]


	
	  \endgroup
	\leavevmode\marginnote{\textenglish{350/s}}

	  \pstart न सर्व्वात्मनाऽभेद\edlabel{pvv.350-1}\footnote{\label{pvv.350-1}  १ सामान्यं विशेष इति भेदस्थापनात् ।} {\color{DodgerBlue3}“स्तयोः”} सामान्यविशेषो{\color{DodgerBlue3}“रपि”} कथञ्चिद् {\color{DodgerBlue3}“भेदो  भवेद् यदि”} (।) अन्यथा सामान्यविशेषभावानुपपत्तिः । अत्राह । {\color{DodgerBlue3}“येनात्मना”} स्वरूपेण साधारणेन चासाधारणेन च {\color{DodgerBlue3}“तयोः”} सामान्यविशेषयो{\color{DodgerBlue3}“र्भेदः सामान्यं”} विशेष {\color{DodgerBlue3}“इत्येतद्”} व्यवस्थाप्यते । तेनात्मना साधारणासाधारणेन {\color{DodgerBlue3}“यदि भेदस्तदा”} भेद एव {\color{DodgerBlue3}“तयोः”} स्यात् भिन्नलक्षणत्वात् ।
	\pend
      
	  \bigskip
	  \begingroup
	  \large
	
	    
	    \stanza[\smallbreak]
	\label{pv.3.177b}\edlabel{pv.3.177b}\flagstanza{\tiny\textenglish{...3.177b}}तथा च स्यान्निःसामान्यविशेषता ॥ १७७ ॥\&[\smallbreak]


	
	  \endgroup
	
	  \bigskip
	  \begingroup
	  \large
	
	    
	    \stanza[\smallbreak]
	\label{pv.3.178a}\edlabel{pv.3.178a}\flagstanza{\tiny\textenglish{...3.178a}}भेदसामान्ययोर्यद्वद् घटादीनां परस्परम् ।\&[\smallbreak]


	
	  \endgroup
	

	  \pstart {\color{DodgerBlue3}“तथा च निःसामान्यविशेषता”} समान्यविशेषरूपताऽभावः सामान्यविशेषयो\edlabel{pvv.350-2}\footnote{\label{pvv.350-2}  २ भेदो निःसामान्यः सामान्यं निर्व्विशेषः ।}रभितयोः {\color{DodgerBlue3}“स्यात् । यद्वद् घटादीनां”} विशेषाणां {\color{DodgerBlue3}“परस्परं”} न सामान्यविशेषता ।\edlabel{pvv.350-3}\footnote{\label{pvv.350-3}  ३ अजन्यजनकत्वेन सम्बन्धाभावात् ।} विशेषः स्व(लक्षण)रूपमनुगामि सामान्यं तदेवानुगामि विशेष उच्यते (।) यदि तु तयोर्भेद एव न सामान्यविशेषभावः स्यात् ।\edlabel{pvv.350-4}\footnote{\label{pvv.350-4}  ४ दिगम्बरस्योर्ध्वतासामान्यं सांख्यस्य तिर्यक् एकदोषेण (देशिनं) निरस्य तिर्यक्सामान्यं पुनराह ।}
	\pend
      
	  \bigskip
	  \begingroup
	  \large
	
	    
	    \stanza[\smallbreak]
	\label{pv.3.178b}\edlabel{pv.3.178b}\flagstanza{\tiny\textenglish{...3.178b}}यमात्मानं पुरस्कृत्य पुरुषोयं प्रवर्तते ॥ १७८ ॥\&[\smallbreak]


	
	  \endgroup
	
	  \bigskip
	  \begingroup
	  \large
	
	    
	    \stanza[\smallbreak]
	\label{pv.3.179a}\edlabel{pv.3.179a}\flagstanza{\tiny\textenglish{...3.179a}}तत्साध्यफलवाञ्छावान् भेदाभेदौ तदाश्रयौ ।&चिन्त्यते स्वात्मना भेदः ;\&[\smallbreak]


	
	  \endgroup
	

	  \pstart अपि चायं व्यवहारी {\color{DodgerBlue3}“पुरुषो यम”}र्थस्या{\color{DodgerBlue3}“त्मानं”} स्वभावं {\color{DodgerBlue3}“पुरस्कृत्य”} प्रवृत्तिविषयत्वेनाग्रहं कृत्वा {\color{DodgerBlue3}“तत्साध्यफलवाञ्छावान्”} अर्थसाध्यफलसमीहायुक्तः सन \edlabel{pvv.350-5}\footnote{\label{pvv.350-5}  ५ अर्थार्थिनः सामान्यस्य भेदाभेदचिन्तया न कि(ञ्चि)दनर्थत्वात् ।} {\color{DodgerBlue3}“प्रवर्त्तते (।) तदाश्रयौ”} तदर्थविषयौ {\color{DodgerBlue3}“भेदाभेदौ”} शास्त्रकारैश्चिन्त्येते । न त्वर्थक्रियानुपयुक्तसामान्यविषयौ (।) तेषाञ्चार्थक्रियाकारिणां पुरुषप्रवृत्तिविषयाणामर्थानां {\color{DodgerBlue3}“स्वात्मना”} स्वरूपेण {\color{DodgerBlue3}“भेदः”} ।\edlabel{pvv.350-6}\footnote{\label{pvv.350-6}  ६ आत्यन्तिकोस्त्येव ।}
	\pend
      

	  \pstart कथं तर्ह्येकबुद्धिशब्दविषयतेत्याह ।
	\pend
      
	  \bigskip
	  \begingroup
	  \large
	
	    
	    \stanza[\smallbreak]
	\label{pv.3.179b}\edlabel{pv.3.179b}\flagstanza{\tiny\textenglish{...3.179b}}व्यावृत्त्या च समानता ॥ १७९ ॥\&[\smallbreak]


	
	  \endgroup
	
	  \bigskip
	  \begingroup
	  \large
	
	    
	    \stanza[\smallbreak]
	\label{pv.3.180a}\edlabel{pv.3.180a}\flagstanza{\tiny\textenglish{...3.180a}}अस्त्येव वस्तु नान्वेति प्रवृत्त्यादिप्रसङ्गतः ।\&[\smallbreak]


	
	  \endgroup
	\leavevmode\marginnote{\textenglish{351/s}}

	  \pstart {\color{DodgerBlue3}“विजातीयाद् व्यावृत्या च सा समानताऽस्त्येव”} । स्वस्वभावनियतन्तु स्वलक्षणं {\color{DodgerBlue3}“वस्तु नान्वेति”} सर्व्वत्र {\color{DodgerBlue3}“प्रवृत्त्यादिप्रसङ्गतः”} । अग्निरपि {\color{DodgerBlue3}“श”} (=स) लिल\edlabel{pvv.351-1}\footnote{\label{pvv.351-1}  १ अथ स्वलक्षणमन्वेष्यतीति किं कल्पितव्यावृत्त्यापीत्यपि न परस्परं भेदात् । यदि घटः पटे स्यादुदकार्थी पटेपि प्रवर्त्तेत न चास्तीति न प्रवृत्त्यादिना तुल्योत्पत्तिनिरोधादि ।}स्वभाव एवेति {\color{DodgerBlue3}“श”} (=स) लिलार्थी तत्रापि प्रवर्त्तेत । तस्मात् स्थितमेतत् (।) न किञ्चित् किमप्यन्वेतीति\edlabel{pvv.351-2}\footnote{\label{pvv.351-2}  २ सामान्यविशेषयोर्वस्तुत एकत्वात् कृतकानित्यत्वानि वा व्यभिचारो व्यावृत्तस्वलक्षणस्यैव सामान्यत्वात् बौद्धे ।} ।
	\pend
      

	  \begin{center}%% label @type='head'
	\textbf{ग. जैनमतनिरासः}
	\end{center}
	
	  \bigskip
	  \begingroup
	  \large
	
	    
	    \stanza[\smallbreak]
	\label{pv.3.180b}\edlabel{pv.3.180b}\flagstanza{\tiny\textenglish{...3.180b}}एतेनैव यदाह्रीकाः किमप्ययुक्तमाकुलम् ॥ १८० ॥\&[\smallbreak]


	
	  \endgroup
	
	  \bigskip
	  \begingroup
	  \large
	
	    
	    \stanza[\smallbreak]
	\label{pv.3.181a}\edlabel{pv.3.181a}\flagstanza{\tiny\textenglish{...3.181a}}प्रलपन्ति प्रतिक्षिप्तं तदप्येकान्तसम्भवात् ।\&[\smallbreak]


	
	  \endgroup
	

	  \pstart {\color{DodgerBlue3}“एतेन सां ख्य”} मतनिराकरणे{\color{DodgerBlue3}“नैवाह्रीका”} दि ग म्ब रा\edlabel{pvv.351-3}\footnote{\label{pvv.351-3}  ३ सर्वः सर्वस्वभावो न च सर्वस्वभावः ।} {\color{DodgerBlue3}“यत्”} “स्यादुष्ट्रो दधि वस्तुत्वात् । न वा स्यादुष्ट्रो\edlabel{pvv.351-4}\footnote{\label{pvv.351-4}  ४ दध्यवस्थायां ।} विशेषरूपतये” ति । {\color{DodgerBlue3}“किमप्ययुक्त”}तया हेयोपादेयविषयापरिनिष्ठाना{\color{DodgerBlue3}“दाकुलं प्रलपन्ति तदपि प्रतिक्षिप्तमेकान्तस्य भेदस्य सम्भवात्”} ।
	\pend
      

	  \pstart आकुलत्वमेवाख्यातुमाह ।
	\pend
      
	  \bigskip
	  \begingroup
	  \large
	
	    
	    \stanza[\smallbreak]
	\label{pv.3.181b}\edlabel{pv.3.181b}\flagstanza{\tiny\textenglish{...3.181b}}सर्वंस्योभयरूपत्वे तद्विशेषनिराकृतेः ॥ १८१ ॥\&[\smallbreak]


	
	  \endgroup
	
	  \bigskip
	  \begingroup
	  \large
	
	    
	    \stanza[\smallbreak]
	\label{pv.3.182a}\edlabel{pv.3.182a}\flagstanza{\tiny\textenglish{...3.182a}}चोदितो “दधि खादे” ति किमुष्ट्रं नाभिधावति ।\&[\smallbreak]


	
	  \endgroup
	

	  \pstart {\color{DodgerBlue3}“सर्व्वस्य”} वस्तुन {\color{DodgerBlue3}“उभयरुपत्वे”}\edlabel{pvv.351-5}\footnote{\label{pvv.351-5}  ५ दध्यादेरुष्ट्रादिषु तादात्म्यानुगमात् ।} स्वपररूपत्वे सति {\color{DodgerBlue3}“तद्विशेष”}स्य दध्येव दधिनोष्ट्र\leavevmode\marginnote{\textenglish{69b/MA}} उष्ट्र एवोष्ट्रो न दधीत्यस्य भेदस्य {\color{DodgerBlue3}“निराकृतेः । दधि खादेति चोदितो”} नियोज्यः {\color{DodgerBlue3}“किमुष्ट्रं”} प्रति {\color{DodgerBlue3}“नाभिधावति”} ।
	\pend
      
	  \bigskip
	  \begingroup
	  \large
	
	    
	    \stanza[\smallbreak]
	\label{pv.3.182b}\edlabel{pv.3.182b}\flagstanza{\tiny\textenglish{...3.182b}}अथारस्त्यतिशयः कशिचद् येन भेदेन वर्त्तते ॥ १८२ ॥\&[\smallbreak]


	
	  \endgroup
	
	  \bigskip
	  \begingroup
	  \large
	
	    
	    \stanza[\smallbreak]
	\label{pv.3.183a}\edlabel{pv.3.183a}\flagstanza{\tiny\textenglish{...3.183a}}स एव विशेषोऽन्यत्र नास्तीत्यनुभयं परम् ।\&[\smallbreak]


	
	  \endgroup
	

	  \pstart {\color{DodgerBlue3}“अथास्ति”} दध्नः सकाशाद् उष्ट्रस्या{\color{DodgerBlue3}“तिशयो”} विशेषः {\color{DodgerBlue3}“कश्चिद् येन”} विशेषेण चोदितेन {\color{DodgerBlue3}“भेदेन”} प्रतिनियमेन दधिशब्दाद् दध्न्येव उष्ट्रशब्दादुष्ट्र एव {\color{DodgerBlue3}“प्रवर्त्तते”} (।) एवन्तर्हि {\color{DodgerBlue3}“स”} विशेष एवान्यत्रासम्भवी उष्ट्रो विशेषो दधिलक्ष{\color{DodgerBlue3}“णोऽन्यत्र”} वस्तुनि {\color{DodgerBlue3}“नास्तीति”} सर्व्वं व{\color{DodgerBlue3}“स्त्वनुभयं”}\edlabel{pvv.351-6}\footnote{\label{pvv.351-6}  ६ स्वरूपभेदवल्लक्षणभेदश्चेत् न च द्रव्यत्वं द्रव्यादिव्यतिरिक्तं भावि ।} न स्वपररूपं किन्तु {\color{DodgerBlue3}“पर”}मेव परस्मात् ।
	\pend
      \leavevmode\marginnote{\textenglish{352/s}}

	  \pstart किञ्च (।)
	\pend
      
	  \bigskip
	  \begingroup
	  \large
	
	    
	    \stanza[\smallbreak]
	\label{pv.3.183b}\edlabel{pv.3.183b}\flagstanza{\tiny\textenglish{...3.183b}}सर्व्वात्मत्वे च सर्व्वेषां भिन्नौ स्यातां न धीध्वनी ॥ १८३ ॥\&[\smallbreak]


	
	  \endgroup
	
	  \bigskip
	  \begingroup
	  \large
	
	    
	    \stanza[\smallbreak]
	\label{pv.3.184a}\edlabel{pv.3.184a}\flagstanza{\tiny\textenglish{...3.184a}}भेदसंहारवादस्य तदभेदादसम्भवः ।\&[\smallbreak]


	
	  \endgroup
	

	  \pstart {\color{DodgerBlue3}“सर्व्वेषां”} भावानां {\color{DodgerBlue3}“सर्व्वात्मत्वे च भिन्नौ धीध्वनी”} न स्यातामेकविषयत्वात् । तयोर्धीध्वन्यो{\color{DodgerBlue3}“रभेदात् भेदसंहारवादस्यासम्भवः”} स्यात् । उष्ट्राद् भिन्नं दधीति भेदव्यवहारो दध्येवोष्ट्र इति च तदात्मतोपसंहारव्यवहारश्च बुद्धिशब्दयोरभेदान्न स्यात् । न हि बुद्धिशब्दयोर्भेदव्यवहारो युक्तः । तन्निबन्धनत्वात् तस्य । तदभावेपि भावे चातिप्रसङ्गात् । भेदप्रतीत्योर्भावात् तादात्म्योपसंहारश्च कथं तदधीनत्वात् तस्य ।
	\pend
      
	  
	% new div opening: depth here is 1
	
\section[{३. शब्दचिन्ता}]{३. शब्दचिन्ता}\label{div_pvv.3.184cd_3.185_3.186_3.187_3.188_3.189_3.190_3.191_3.192_3.193_3.194_3.195_3.196_3.197_3.198_3.199_3.200}\edlabel{div_pvv.3.184cd_3.185_3.186_3.187_3.188_3.189_3.190_3.191_3.192_3.193_3.194_3.195_3.196_3.197_3.198_3.199_3.200}
	  
	% new div opening: depth here is 2
	
	  \bigskip
	  \begingroup
	  \large
	
	    
	    \stanza[\smallbreak]
	\label{pv.3.184b}\edlabel{pv.3.184b}\flagstanza{\tiny\textenglish{...3.184b}}द्रव्याभावादभावस्य शब्दा रूपाभिधायिनः ॥ १८४ ॥\&[\smallbreak]


	
	  \endgroup
	
	  \bigskip
	  \begingroup
	  \large
	
	    
	    \stanza[\smallbreak]
	\label{pv.3.185a}\edlabel{pv.3.185a}\flagstanza{\tiny\textenglish{...3.185a}}नाशङ्क्या एव सिद्धास्तेऽतोव्यवच्छेदवाचकाः ।\&[\smallbreak]


	
	  \endgroup
	

	  \pstart उक्तं तावद् भावादिशब्दानां व्यवच्छेदविषयत्वं । येप्यभावादिशब्दा {\color{DodgerBlue3}“अभावस्य”} स्वरूपा{\color{DodgerBlue3}“भावात्”} तेपि {\color{DodgerBlue3}“रूपाभिधायिनो”} वस्तुवाचका {\color{DodgerBlue3}“नाशङक्या एव”} । सम्भवद् वस्तु वाच्यं स्यान्न वेति चिन्त्येतापि ।\edlabel{pvv.352-1}\footnote{\label{pvv.352-1}  १ यथाऽविषयशब्दाभावाग्राहकत्वादपोहविषयाः सिद्धा इति दृष्टान्तेन परः साध्यस्तथाभावविषया अप्यपोहविषयाभावस्वरूपाग्राहकत्वात् (।) केवलमध्यवसायात्तृ तद्विषयत्वं ।} यत्र तु वस्त्वेव नास्ति तत्र का चिन्ता । {\color{DodgerBlue3}“अतस्ते”}ऽभावादिशब्दा {\color{DodgerBlue3}“व्यवच्छेदस्य”} भावव्यावृत्ते{\color{DodgerBlue3}“र्व्वाचकाः सिद्धाः”} । तस्मात् स्थितमेतत् स्वभावहेतोर्व्वस्तुतः साध्यात्मकत्वेपि न प्रतिज्ञार्थैकदेशो हेतुः । साध्यसाधनधर्मिध्वनिविकल्पानां भिन्न\edlabel{pvv.352-2}\footnote{\label{pvv.352-2}  २ कारणैर्न कृत इति द्वितीयादिक्षणस्थायीति शब्दे भिन्नभिन्नारोपव्यवच्छेदकत्वेन ।}व्यवच्छेदविषय\edlabel{pvv.352-3}\footnote{\label{pvv.352-3}  ३ वस्तुविषयत्वे स्यात् प्रतिज्ञार्थैकदेशता ।}त्वात् । न च साध्यादीनां कल्पितत्वं भेदस्य कल्पनात् । शब्दत्वेन निश्चितो धर्मी सत्त्वेन च हेतुः (।) क्षणिकत्वेन निश्चितः साध्यः । इत्यकल्पिता एव धर्म्यादय इत्युक्तं ।
	\pend
      

	  \pstart स चायं स्वभावो हेतुः ॥
	\pend
      
	  \bigskip
	  \begingroup
	  \large
	
	    
	    \stanza[\smallbreak]
	\label{pv.3.185b}\edlabel{pv.3.185b}\flagstanza{\tiny\textenglish{...3.185b}}उपाधिभेदापेक्षो वा स्वभावः केवलोऽथवा ॥ १८५ ॥\&[\smallbreak]


	
	  \endgroup
	
	  \bigskip
	  \begingroup
	  \large
	
	    
	    \stanza[\smallbreak]
	\label{pv.3.186a}\edlabel{pv.3.186a}\flagstanza{\tiny\textenglish{...3.186a}}उच्यते साध्यसिध्यर्थं नाशे कार्यत्वसत्त्ववत् ।\&[\smallbreak]


	
	  \endgroup
	\leavevmode\marginnote{\textenglish{353/s}}

	  \pstart क्वचिदुपाधिभेदो विशेषणविशेषो भिन्नो\edlabel{pvv.353-1}\footnote{\label{pvv.353-1}  १ यद्यपि कृतके सत्त्वमस्ति तथापि हेतुकृतोयं कृतक एतन्मात्रम्विवक्षितं न सामर्थ्यं ।}ऽभिन्नो\edlabel{pvv.353-2}\footnote{\label{pvv.353-2}  २ स्थानकारणादिप्रत्ययभेदित्वपुरूषप्रयत्नजत्वं स्वभावभूतधर्मभेदेनोत्पत्तिमत्त्वं ।} वा तदपेक्षः । {\color{DodgerBlue3}“केवलो”} विशेषणरहितः शुद्धो{\color{DodgerBlue3}“थवा साध्यसिध्यर्थमुच्यते\edlabel{pvv.353-3}\footnote{\label{pvv.353-3}  ३ अनित्ये कृतकमुपाधिभेदापेक्षं सत्त्वमनपेक्षं ।} नाशे कार्यत्वसत्त्ववत्”} । नाशे साध्ये कार्यत्वं भिन्नविशेषणा{\color{DodgerBlue3}“पेक्षः स्वभावः”} तज्जन्मन्यपेक्षितपरव्यापारस्य कार्यत्वात् । एवं प्रत्ययभेदभेदित्वादयो द्रष्टव्याः । उत्पत्तिमत्वं पुनरभिन्नविशेषणमुत्पत्त्या स्वभावभूतया कल्पितभेदया विशेषणात् । सत्त्वन्तु केवलं नाश एव साध्ये स्वभावो विशेषणानुपादानात् ।
	\pend
      
	  \bigskip
	  \begingroup
	  \large
	
	    
	    \stanza[\smallbreak]
	\label{pv.3.186b}\edlabel{pv.3.186b}\flagstanza{\tiny\textenglish{...3.186b}}सत्तास्वभावो हेतुश्चेत् सा सत्ता साध्यते कथम् ॥ १८६ ॥\&[\smallbreak]


	
	  \endgroup
	
	  \bigskip
	  \begingroup
	  \large
	
	    
	    \stanza[\smallbreak]
	\label{pv.3.187a}\edlabel{pv.3.187a}\flagstanza{\tiny\textenglish{...3.187a}}भेदेनानन्वयात् सोयं व्याहतो हेतुसाध्ययोः ।\&[\smallbreak]


	
	  \endgroup
	

	  \pstart ननु {\color{DodgerBlue3}“स\edlabel{pvv.353-4}\footnote{\label{pvv.353-4}  ४ सत्त्वे}त्तास्वभावो हेतुश्चेद”}भिमतः सामान्यरूपो विशेषस्यानन्वयात् । {\color{DodgerBlue3}“तदा सा सत्ता”} प्रधाना\edlabel{pvv.353-5}\footnote{\label{pvv.353-5}  ५ सांख्यस्यास्ति प्रधानमिति साध्यं ।}ख्या सर्व्वव्यक्तिव्यापिनी {\color{DodgerBlue3}“कथं साध्यते”}ऽचे\edlabel{pvv.353-6}\footnote{\label{pvv.353-6}  ६ अथ सत्तायाः सामान्ये साध्ये सिद्धसाध्यताऽतो विशेषः साध्यः तत्रान्वयात् साध्यशून्यो दृष्टान्तोतो न सत्ता ।}तनत्वादिहेतोः । अथ भेदानां परस्परमत्यन्तं {\color{DodgerBlue3}“भेदेनानन्वयान्न”} साध्यते तदा {\color{DodgerBlue3}“सोयं”} भेदानामनन्वयो हि {\color{DodgerBlue3}“हे\edlabel{pvv.353-7}\footnote{\label{pvv.353-7}  ७ सत्त्वहेतावसाध्यशून्यो दृष्टान्तविशेषाणां दुष्टो हेतौ साध्यं च ।}तुसाध्ययोर्व्याहितः”} ।
	\pend
      

	  \pstart अर्थानन्वयिनः साध्यता न युक्ता तथा साधनतापि । तत् कथं सत्त्वं साधनं (।) तस्मात् सत्ता सामान्यं साध्यञ्च \edlabel{pvv.353-8}\footnote{\label{pvv.353-8}  ८ असिद्धत्वात् ।} साधनञ्चानन्वयात् स्यात् \edlabel{pvv.353-9}\footnote{\label{pvv.353-9}  ९ इति सांख्योक्तौ बौद्धो दोषमाह ।} ।
	\pend
      

	  \pstart अत्राह ।
	\pend
      
	  \bigskip
	  \begingroup
	  \large
	
	    
	    \stanza[\smallbreak]
	\label{pv.3.187b}\edlabel{pv.3.187b}\flagstanza{\tiny\textenglish{...3.187b}}भावोपादानमात्रे तु साध्ये सामान्यधर्मिणि ॥ १८७ ॥\&[\smallbreak]


	
	  \endgroup
	
	  \bigskip
	  \begingroup
	  \large
	
	    
	    \stanza[\smallbreak]
	\label{pv.3.188a}\edlabel{pv.3.188a}\flagstanza{\tiny\textenglish{...3.188a}}न कश्चिदर्थः सिद्धः स्यादनिषिद्धञ्च तादृशम् ।\&[\smallbreak]


	
	  \endgroup
	

	  \pstart {\color{DodgerBlue3}“भावः”} सत्ता सा {\color{DodgerBlue3}“उपादानं”} विशेषणं यस्य स भावोपादानः । स एव केवलस्त{\color{DodgerBlue3}“न्मात्रं”} तस्मिन्  {\color{DodgerBlue3}“साध्ये सामान्यधर्मिणि”} सामान्यधर्मवति {\color{DodgerBlue3}“न कश्चिदर्थः”} प्रधान-\leavevmode\marginnote{\textenglish{70a/MA}} \leavevmode\marginnote{\textenglish{354/s}} सिद्धिलक्षणः सां ख्य स्य \edlabel{pvv.354-1}\footnote{\label{pvv.354-1}  १ त्रैगुण्यादिर्यतो नित्यः} {\color{DodgerBlue3}“सिद्धः स्यात्”} (।) सत्वमात्रविशेषणस्य धर्मिणः साधनात् । {\color{DodgerBlue3}“अनिषिद्धञ्च तादृशं”} साध्यं भेदानां सत्वस्येष्टत्वात्\edlabel{pvv.354-2}\footnote{\label{pvv.354-2}  २ सिद्धसाध्यतोक्ता} ॥
	\pend
      
	  \bigskip
	  \begingroup
	  \large
	
	    
	    \stanza[\smallbreak]
	\label{pv.3.188b}\edlabel{pv.3.188b}\flagstanza{\tiny\textenglish{...3.188b}}उपात्तभेदे साध्येस्मिन् भवेद्धेतुरनन्वयः ॥ १८८ ॥\&[\smallbreak]


	
	  \endgroup
	
	  \bigskip
	  \begingroup
	  \large
	
	    
	    \stanza[\smallbreak]
	\label{pv.3.189a}\edlabel{pv.3.189a}\flagstanza{\tiny\textenglish{...3.189a}}सत्तायां तेन साध्यायां विशेषः साधितो भवेत् ।\&[\smallbreak]


	
	  \endgroup
	

	  \pstart अथैकसुखाद्यात्मकमित्यप्रधानविशेष (ण) विशेषितं सत्त्वं साध्यं । तदो{\color{DodgerBlue3}“पात्तभेदे साध्येऽस्मिन्”}\edlabel{pvv.354-3}\footnote{\label{pvv.354-3}  ३ निरन्वयविनाशाभावान्नित्ये त्रिगुणत्वात् सुखदुःखमोहात्मके कर्तृत्वादियुक्ते धूमाग्न्योस्त्वयोगव्यवच्छेदेन व्याप्तिः प्रदेशनिष्ठताऽसिद्धा साध्येति न दोषः ।} सत्त्वे भवेद्धेतुरनन्वयोऽन्वयरहितः प्रधानस्य क्वचिदन्वयासिद्धेः (।) सामान्यमेव  पुनः किन्न साध्यते । सामान्ये साध्ये सिद्धसाधनदोषात् । {\color{DodgerBlue3}“सत्तायां साध्याया”}मिष्टायां {\color{DodgerBlue3}“तेन”} वादिना {\color{DodgerBlue3}“विशेष”} एवाभिमतः {\color{DodgerBlue3}“साधितो भवेत्”} ।
	\pend
      

	  \pstart अत्र चानन्वयदोष उक्तः ।
	\pend
      

	  \pstart अस्माकन्तु\edlabel{pvv.354-4}\footnote{\label{pvv.354-4}  ४ दोषमाह ।}(।)
	\pend
      
	  \bigskip
	  \begingroup
	  \large
	
	    
	    \stanza[\smallbreak]
	\label{pv.3.189b}\edlabel{pv.3.189b}\flagstanza{\tiny\textenglish{...3.189b}}अपरामृष्टतद्भेदे वस्तुमात्रे तु साधने ॥ १८९ ॥\&[\smallbreak]


	
	  \endgroup
	
	  \bigskip
	  \begingroup
	  \large
	
	    
	    \stanza[\smallbreak]
	\label{pv.3.190a}\edlabel{pv.3.190a}\flagstanza{\tiny\textenglish{...3.190a}}तन्मात्रव्यापिनः साध्यस्यान्वयो न विहन्यते ।\&[\smallbreak]


	
	  \endgroup
	

	  \pstart {\color{DodgerBlue3}“अपरामुष्टो”}ऽनध्यवसितः तस्य वस्तुनो\edlabel{pvv.354-5}\footnote{\label{pvv.354-5}  ५ सत्तायाः ।} {\color{DodgerBlue3}“भेदो”} यस्मिन् तस्मिन् {\color{DodgerBlue3}“वस्तुमात्रे तु साधने तन्मात्रव्यापिनः साध्यस्या”}नित्यत्वस्या{\color{DodgerBlue3}“न्वयो न विहन्यते”}ऽतः साधनत्वं सत्त्वस्य युक्तं ।
	\pend
      

	  \pstart ननु धूमादग्निसाधनेपि समानमेतत् । तथा हि यदि धूमादग्निसत्तामात्रं साध्यते तदा सिद्धसाध्यता । अथ पर्व्वतेऽस्तीति साध्यते तदाऽन्वयासिद्धिः । नैतदस्ति । \edlabel{pvv.354-6}\footnote{\label{pvv.354-6}  ६ यत्र धूमस्तत्राग्निरित्यग्निमात्रेण व्याप्तोग्निनान्तरीयो धूमः सिद्धो यत्रैव दृश्यते तत्रैवाग्निर्बुद्धिं जनयति तत्र च साध्यनिर्देशेन न किञ्चित् ।} न हि पक्षायोगव्यवच्छेदस्तमग्निं विशेषी\edlabel{pvv.354-7}\footnote{\label{pvv.354-7}  ७ पक्षोग्नेर्न विशेषणं ।}करोति । अनग्निव्यावृत्तस्य साधनात् । सत्तासाधने तु तद् (सत्त्व) विशेष एव प्रधानाख्यः साध्यः ।
	\pend
      

	  \pstart किञ्च (।)
	\pend
      
	  \bigskip
	  \begingroup
	  \large
	
	    
	    \stanza[\smallbreak]
	\label{pv.3.190b}\edlabel{pv.3.190b}\flagstanza{\tiny\textenglish{...3.190b}}नासिद्धे भावधर्मोस्ति व्यभिचार्युभयाश्रयः ॥ १९० ॥\&[\smallbreak]


	
	  \endgroup
	
	  \bigskip
	  \begingroup
	  \large
	
	    
	    \stanza[\smallbreak]
	\label{pv.3.191a}\edlabel{pv.3.191a}\flagstanza{\tiny\textenglish{...3.191a}}धर्मो विरुद्धोऽभावस्य सा सत्ता साध्यते कथम् ।\&[\smallbreak]


	
	  \endgroup
	\leavevmode\marginnote{\textenglish{355/s}}

	  \pstart सत्तायां साधनमचेतनत्वादीष्टं सद्भावधर्मोऽभावधर्म उभय धर्मो वा भवेत् । तत्र साधनात् प्राग{\color{DodgerBlue3}“सिद्धे”} भावे {\color{DodgerBlue3}“भावधर्मो”} नास्तीत्यसिद्धासौ । {\color{DodgerBlue3}“उभयाश्रयो”} भावाभावधर्मश्च {\color{DodgerBlue3}“व्यभिचार्य”}नौकान्तिकः । न ह्युभयधर्म एका\edlabel{pvv.355-1}\footnote{\label{pvv.355-1}  १ यो भावस्य धर्मः स्वभावः स कथं न भावस्य ।}न्तेनैकसत्तां गमयति । अमूर्तत्वमिव वस्तुतां\edlabel{pvv.355-2}\footnote{\label{pvv.355-2}  २ प्रसङ्गस्याश्रितत्वादत्र मूर्त्तनिवृत्तिमात्रं भावे विज्ञानेपि निरूपाख्येपीति विपक्षादव्यावृत्तेरगमकं ।} । {\color{DodgerBlue3}“अभावस्य”} तु {\color{DodgerBlue3}“धर्मो विरुद्धो”}ऽसत्त्वसाधनात्\edlabel{pvv.355-3}\footnote{\label{pvv.355-3}  ३ हेतुसिद्धिं स्वीकृत्यानेकान्तविरुद्धतोक्तिरिहासिद्धावभावात् ।} । ततस्त्रिविधदोषदुष्टत्वात् साधनस्य\edlabel{pvv.355-4}\footnote{\label{pvv.355-4}  ४ यत् किंञ्चित् साधनमुपादीयते तस्य ।} {\color{DodgerBlue3}“सा सत्ता कथं साध्यते”} ।
	\pend
      

	  \pstart साधनपक्षेतु सत्ता धर्मिणि सिद्धत्वान्नासिद्धा । अनित्यताव्याप्तिप्राप्तेः\edlabel{pvv.355-5}\footnote{\label{pvv.355-5}  ५ निश्चयात् ।}विरोधव्यभिचारौ चापास्तौ । तस्माद् ।
	\pend
      
	  \bigskip
	  \begingroup
	  \large
	
	    
	    \stanza[\smallbreak]
	\label{pv.3.191b}\edlabel{pv.3.191b}\flagstanza{\tiny\textenglish{...3.191b}}सिद्धः स्वभावो गमकोऽतो गम्यस्तस्य व्यापकः ॥ १९१ ॥\&[\smallbreak]


	
	  \endgroup
	
	  \bigskip
	  \begingroup
	  \large
	
	    
	    \stanza[\smallbreak]
	\label{pv.3.192a}\edlabel{pv.3.192a}\flagstanza{\tiny\textenglish{...3.192a}}सिद्धः स्वभावनियतः स्वनिवृत्तौ निवर्तकः ।\&[\smallbreak]


	
	  \endgroup
	

	  \pstart {\color{DodgerBlue3}“व्याप्यस्वभावः”} साध्यात्मतया {\color{DodgerBlue3}“सिद्धो”} (निश्चितो) गमकः । तस्य व्याप्यस्य व्यापकः स्वभावः सिद्धो गम्यः । व्यापकस्य तत्र भाव एव । व्याप्यस्य च तत्रैव भाव इत्युभयधर्मरूपाया व्याप्तेः सिद्धत्वात् ।
	\pend
      

	  \pstart {\color{DodgerBlue3}“अतश्चायं”} \edlabel{pvv.355-6}\footnote{\label{pvv.355-6}  ६ गम्यगमकत्वमुक्त्वा निवर्त्त्यनिवर्त्तकमाह ।}व्यापकः {\color{DodgerBlue3}“स्वनिवृत्तौ”} तस्य व्याप्यस्य {\color{DodgerBlue3}“निवर्त्तकः”} । अनेन साधर्म्यवैधर्म्यप्रयोगावुद्दिष्टौ ।
	\pend
      

	  \begin{center}%% label @type='head'
	\textbf{(१) निर्हेतुकविनाशः}
	\end{center}
	

	  \pstart उदाहरणमाह ।
	\pend
      
	  \bigskip
	  \begingroup
	  \large
	
	    
	    \stanza[\smallbreak]
	\label{pv.3.192b}\edlabel{pv.3.192b}\flagstanza{\tiny\textenglish{...3.192b}}अनित्यत्वे यथा कार्यमकार्यं वाऽविनाशिनि ॥ १९२ ॥\&[\smallbreak]


	
	  \endgroup
	
	  \bigskip
	  \begingroup
	  \large
	
	    
	    \stanza[\smallbreak]
	\label{pv.3.193a}\edlabel{pv.3.193a}\flagstanza{\tiny\textenglish{...3.193a}}अहेतुत्वाद् विनाशस्य स्वभावादनुबन्धिता ।\&[\smallbreak]


	
	  \endgroup
	

	  \pstart {\color{DodgerBlue3}“अनित्यत्वे”} साध्ये {\color{DodgerBlue3}“कार्य”}\edlabel{pvv.355-7}\footnote{\label{pvv.355-7}  ७ कृतकत्वं ।} हेतु{\color{DodgerBlue3}“र्यथा”} यत् कृतकं तदनित्यं तथा घटः कृतकश्च शब्द इति साधर्म्यप्रयोगः । अकार्यमकार्यस्वभावो वा{\color{DodgerBlue3}“ऽविनाशिनि”} नाशाभाव इति । अनित्यत्वनिवृत्तौ कृतकत्वनिवृत्तिर्यथाकाशे कृतकश्च शब्द इति वैधर्म्यप्रयोगः । कथं पुनर्गम्यते सत्त्वमात्रानुबन्धिनी नश्वरतेत्याह (।) {\color{DodgerBlue3}“अहेतुत्वात्”} अ\edlabel{pvv.355-8}\footnote{\label{pvv.355-8}  ८ व्याप्तिविषयं प्रमाणं विपक्षे आधकमाह पुष्टः परेण ।}हेतुकृतत्वाद् \leavevmode\marginnote{\textenglish{356/s}} {\color{DodgerBlue3}“विनाशस्य स्वभावाद्”} वस्तुसत्तामात्रेणा{\color{DodgerBlue3}“नुबन्धिता”}\edlabel{pvv.356-1}\footnote{\label{pvv.356-1}  १ भावस्तावदित्यादिना निर्हेतुकत्वे सिद्धे स्वरसते (ा) निवर्त्तमाने घटे मुद्‏गरादिसहाये कपालजनके सदृशक्षणानारम्भान्मन्दमतीनां सहेतुत्वावसायो मुद्गराच्छेदात् संतानस्य न प्रतीतिबाधापि ।} ।
	\pend
      

	  \pstart कस्मादेवमित्याह ।
	\pend
      
	  \bigskip
	  \begingroup
	  \large
	
	    
	    \stanza[\smallbreak]
	\label{pv.3.193b}\edlabel{pv.3.193b}\flagstanza{\tiny\textenglish{...3.193b}}सापेक्षाणां हि भावानां नावश्यम्भावितेंक्ष्यते ॥ १९३ ॥\&[\smallbreak]


	
	  \endgroup
	
	  \bigskip
	  \begingroup
	  \large
	
	    
	    \stanza[\smallbreak]
	\label{pv.3.194a}\edlabel{pv.3.194a}\flagstanza{\tiny\textenglish{...3.194a}}बाहुल्येपीति चेत् तस्य हेतोः क्वचिदसम्भवः ।\&[\smallbreak]


	
	  \endgroup
	

	  \pstart {\color{DodgerBlue3}“सापेक्षाणां भावानां हि”} यस्माद{\color{DodgerBlue3}“वश्यंभाविता\edlabel{pvv.356-2}\footnote{\label{pvv.356-2}  २ नित्योपि स्यात् कश्चित् ।} नेक्ष्यते”} रागस्येव वाससि । ततश्च कश्चिद् घटो न विनश्येदपि । विनाशकानां हेतूनां बाहुल्यादवश्यं इति चेत् । {\color{DodgerBlue3}“बाहुल्येपि”} तस्य नाशकस्य हेतोः\edlabel{pvv.356-3}\footnote{\label{pvv.356-3}  ३ सर्व्वज्ञसत्वे तु न दोषो यतः यदुपदिश्यते तज्ज्ञानपूर्व्वं यथान्यत् उपदिश्यते च चतुरार्यसत्त्यं स ज्ञानी सर्व्ववित्} क्वचिद् घटादावसम्भवः स्यात् । तद्व्या\leavevmode\marginnote{\textenglish{70b/MA}} घातकानामपि बाहुल्यात् ।
	\pend
      
	  \bigskip
	  \begingroup
	  \large
	
	    
	    \stanza[\smallbreak]
	\label{pv.3.194b}\edlabel{pv.3.194b}\flagstanza{\tiny\textenglish{...3.194b}}एतेन व्यभिचारित्वमुक्तं कार्याव्यवस्थितेः ॥ १९४ ॥\&[\smallbreak]


	
	  \endgroup
	
	  \bigskip
	  \begingroup
	  \large
	
	    
	    \stanza[\smallbreak]
	\label{pv.3.195a}\edlabel{pv.3.195a}\flagstanza{\tiny\textenglish{...3.195a}}सर्वेषां नाशहेतूनां हेतुमन्नाशवादिनाम् ।\&[\smallbreak]


	
	  \endgroup
	

	  \pstart {\color{DodgerBlue3}“एतेन”} नाश\edlabel{pvv.356-4}\footnote{\label{pvv.356-4}  ४ लिङ्गत्वेनोपात्तानां ।}हेतूनां प्रतिरोधसम्भवेन हेतु\edlabel{pvv.356-5}\footnote{\label{pvv.356-5}  ५ ये यद्भावं प्रत्यनपेक्षास्ते तद्भावनियता यथाऽसम्भवत्‏प्रतिबन्धान्त्या सामग्री कार्योत्पादनेऽन्यानपेक्षश्च कृतको विनाशे इति स्वभावहेतुः ।}मन्नाशवादिनां मतेन {\color{DodgerBlue3}“नाशहेतूनां”} मुद्‏गरादीनां {\color{DodgerBlue3}“सर्व्वेषां”} नाशे कार्येऽनुमापयितव्ये {\color{DodgerBlue3}“व्यभिचारित्वमुक्तं”} बोद्धव्यं । {\color{DodgerBlue3}“कार्याव्यवस्थितेः”} । विनाशहेतोर्विनाशस्योत्पत्तिनियमाभावात् ।
	\pend
      

	  \pstart कथं पुनर्व्विनाशस्याहेतुतेत्याह ।
	\pend
      
	  \bigskip
	  \begingroup
	  \large
	
	    
	    \stanza[\smallbreak]
	\label{pv.3.195b}\edlabel{pv.3.195b}\flagstanza{\tiny\textenglish{...3.195b}}असामर्थ्याच्च तद्धेतोर्भवत्येव स्वभावतः ॥ १९५ ॥\&[\smallbreak]


	
	  \endgroup
	
	  \bigskip
	  \begingroup
	  \large
	
	    
	    \stanza[\smallbreak]
	\label{pv.3.196a}\edlabel{pv.3.196a}\flagstanza{\tiny\textenglish{...3.196a}}यत्र नाम भवत्यस्मादन्यत्रापि स्वभावतः ।\&[\smallbreak]


	
	  \endgroup
	

	  \pstart {\color{DodgerBlue3}“असामर्थ्याच्च तद्धेतो”}र्व्विनश्वरस्य वा भावस्य विनाशः क्रियते नाशहेतुना । तत्र विनश्वरस्य स्वयमेव विनाशादलं नाशहेतुना । अविनश्वरस्य नाशं कर्त्तुं न कश्चित् समर्थः । च शब्दात् क्रिया\edlabel{pvv.356-6}\footnote{\label{pvv.356-6}  ६ कारकत्वं ।}प्रतिषेधः स्यात् । तथा ह्यभावो यदि पर्युदासो \leavevmode\marginnote{\textenglish{357/s}} भावान्तरं कपालादिकं तदा तस्य मुद्गरादिहेतुतेष्यत एव । तदुत्पादेपि घटस्य न किञ्चिदिति प्राग्वदुपलब्ध्यादिप्रसङ्गः । तस्मादभावं करोति. भावं न करोतीति स्यात् । तथा चाकर्त्तुरहेतुतैव । तस्माद् विनाशहेतोरयोगात् एव विनाशः स्वभावतः स्वहेतोर्भवति {\color{DodgerBlue3}“अस्मादि”}ति\edlabel{pvv.357-1}\footnote{\label{pvv.357-1}  १ स्वभावमात्रभावात् ।} भावो हेतुः । {\color{DodgerBlue3}“यत्र नाम”} विनाशो {\color{DodgerBlue3}“भव”}तीति मुद्‏गरात् कपालोत्पत्तौ लोकाभिमानस्तत्र विनाशकायोगात् स्वहेतोरेव विनश्वरस्वभावतयोत्पत्तेर्द्वितीये क्षणे न भवतीति वक्तव्यं । तस्मादहेतु\edlabel{pvv.357-2}\footnote{\label{pvv.357-2}  २ हेतुनिरपेक्षत्वात् ।}त्वव्यवस्थानादन्यत्रापि यत्र विसदृशानुत्पत्त्या विनाशोत्पत्त्यभिमानो नास्ति तत्रापि {\color{DodgerBlue3}“स्वभावतः”} स्वहेतोरेव विनश्यतीति विनाश एकक्षणस्थायी भावो जायते ।
	\pend
      

	  \pstart तथाविधे च भावमात्रानुरोधिनि विनाशे सत्वं हेतुरव्यभिचारः कार्यस्य कारणे तदधीनत्वादित्युक्तं । तदेवाह\edlabel{pvv.357-3}\footnote{\label{pvv.357-3}  ३ अत्र द्वौ वस्तुसाधनाविति प्रागुक्तमुपसंहरति ।} ।
	\pend
      
	  \bigskip
	  \begingroup
	  \large
	
	    
	    \stanza[\smallbreak]
	\label{pv.3.196b}\edlabel{pv.3.196b}\flagstanza{\tiny\textenglish{...3.196b}}या काचिद् भावविषयाऽनुमितिर्द्विविधैव सा ॥ १९६ ॥\&[\smallbreak]


	
	  \endgroup
	
	  \bigskip
	  \begingroup
	  \large
	
	    
	    \stanza[\smallbreak]
	\label{pv.3.197a}\edlabel{pv.3.197a}\flagstanza{\tiny\textenglish{...3.197a}}स्वसाध्ये कार्यभावाभ्यां सम्बन्धनियमात् तयोः ।\&[\smallbreak]


	
	  \endgroup
	

	  \pstart {\color{DodgerBlue3}“या काचि\edlabel{pvv.357-4}\footnote{\label{pvv.357-4}  ४ विधि ।}द् भावविषयानुमितिः”} सा {\color{DodgerBlue3}“द्विविधैव कार्यभावाभ्यां”} हेतुभ्यां कारणव्यापकविषया भवन्ती {\color{DodgerBlue3}“तयोः”} कार्यस्वभावयोः एव {\color{DodgerBlue3}“स्वसाध्ये सम्बन्धस्य नियमात् ।”} अन्यस्य तु साध्येऽनापत्तेरतदात्मत्वाच्च नाव्यभिचारनियम: ।
	\pend
      

	  \begin{center}%% label @type='head'
	\textbf{(२) अनुपलब्धिचिन्ता}
	\end{center}
	

	  \begin{center}%% label @type='head'
	\textbf{क. अनुपलब्धेः प्रामाण्यम्}
	\end{center}
	

	  \pstart अनुपलब्धेरप्युपलब्धिनिवृत्तिमात्रलक्षणायाः प्रामाण्यमाख्यातुमाह ।
	\pend
      
	  \bigskip
	  \begingroup
	  \large
	
	    
	    \stanza[\smallbreak]
	\label{pv.3.197b}\edlabel{pv.3.197b}\flagstanza{\tiny\textenglish{...3.197b}}प्रवृत्तेर्बुद्धिपूर्वत्वात् तद्भावानुपलम्भने ॥ १९७ ॥\&[\smallbreak]


	
	  \endgroup
	
	  \bigskip
	  \begingroup
	  \large
	
	    
	    \stanza[\smallbreak]
	\label{pv.3.198a}\edlabel{pv.3.198a}\flagstanza{\tiny\textenglish{...3.198a}}प्रवर्त्तितव्यं नेत्युक्ताऽनुपलब्धेः प्रमाणता ।\&[\smallbreak]


	
	  \endgroup
	

	  \pstart सज्ज्ञानशब्दव्यवहाराणां{\color{DodgerBlue3}“प्रवृत्तेर्बुद्धिपूर्व्वकत्वात् त”}स्या बुद्धे\edlabel{pvv.357-5}\footnote{\label{pvv.357-5}  ५ तस्य प्रवृत्तिविषयस्य भावस्यानुपलम्भने प्रत्यक्षानुमानाभ्यां प्रेक्षावता ।} {\color{DodgerBlue3}“र्भावानुपलम्भने”} कारणाभावात् प्रवर्तितव्यं नेति सामर्थ्यात् सिध्यति (।) न हि कारणाभावं कार्यं युक्तं । अतोऽनुपलब्धेरुपलब्धिनिवृत्तिरुपायाः प्रवृत्तिनिषेधे साध्ये प्रमाणतोक्ता चा र्ये ण । न पिशाचादिकं घटादिकं सदिति वक्तव्यमनुपलब्धेरिति सद्व्यवहारप्रतिषेधमात्रं साध्यते न त्वसत्त्वव्यवहारः ।
	\pend
      \leavevmode\marginnote{\textenglish{358/s}}

	  \pstart यत्र तर्हि प्रत्यक्षानुमानयोः शास्त्रस्य निवृत्तिस्तस्याभाव एव साधयितुं युक्त इत्याह ।
	\pend
      
	  \bigskip
	  \begingroup
	  \large
	
	    
	    \stanza[\smallbreak]
	\label{pv.3.198b}\edlabel{pv.3.198b}\flagstanza{\tiny\textenglish{...3.198b}}शास्त्राधिकारासम्बद्धा बहवोर्था अतीन्द्रियाः ॥ १९८ ॥\&[\smallbreak]


	
	  \endgroup
	
	  \bigskip
	  \begingroup
	  \large
	
	    
	    \stanza[\smallbreak]
	\label{pv.3.199a}\edlabel{pv.3.199a}\flagstanza{\tiny\textenglish{...3.199a}}अलिङ्गाश्च कथन्तेषामभावोनुपलब्धितः ।\&[\smallbreak]


	
	  \endgroup
	

	  \pstart {\color{DodgerBlue3}“शास्त्राधिकारे\edlabel{pvv.358-1}\footnote{\label{pvv.358-1}  १ शास्त्राधिकारो यत्र प्रकरणे तत्रान्तरीयकाः ।}ऽसम्बद्धा”} अनेन शास्त्रविषयत्वमाह (।) {\color{DodgerBlue3}“बहवोऽर्था”} अनियतकारणोपनिपातजन्याः सूक्ष्मा दुर्ल्लक्षभेदा मनोवृत्तयो जन्मिनां । देशफलव्यवहिता वाऽनुत्पन्ना द्रव्यविशेषा {\color{DodgerBlue3}“अतीन्द्रियाः”} । अनेन प्रत्यक्षाविषयतामाह । लिङ्गाच्चाननुमेयतामाह । {\color{DodgerBlue3}“तेषाम”}र्थानां प्रत्यक्षानुमानशास्त्रनिवृत्तिलक्षणाया {\color{DodgerBlue3}“अनुपलब्धितः कथमभावः”} साधयितुं युक्तः । सत्यप्यनुपलम्भे तेषां सत्त्वसम्भवात् ।
	\pend
      

	  \pstart अत ईदृश्यनुपलब्धिः ।
	\pend
      
	  \bigskip
	  \begingroup
	  \large
	
	    
	    \stanza[\smallbreak]
	\label{pv.3.199b}\edlabel{pv.3.199b}\flagstanza{\tiny\textenglish{...3.199b}}सदसन्निश्चयफला नेति स्याद् वाऽप्रमाणता ॥ १९९ ॥\&[\smallbreak]


	
	  \endgroup
	
	  \bigskip
	  \begingroup
	  \large
	
	    
	    \stanza[\smallbreak]
	\label{pv.3.200a}\edlabel{pv.3.200a}\flagstanza{\tiny\textenglish{...3.200a}}प्रमाणमपि काचित् स्याद् लिङ्गातिशयभाविनी ।\&[\smallbreak]


	
	  \endgroup
	

	  \pstart {\color{DodgerBlue3}“सतोऽसन्निश्चयफला नेति अप्रमाणता\edlabel{pvv.358-2}\footnote{\label{pvv.358-2}  २ सन्निश्चयफला न सद्व्यवहारनिमित्ताभावात् । नाप्यसन्निश्चयफला सन्देहात् । इति हेतोः स्याद्वाऽस्या अप्रमाणता । निश्चयफलं हि प्रमाणं ।}”} वाऽस्याः स्यात् । सत्त्वप्रतिषेधे साध्ये {\color{DodgerBlue3}“काचित्”} त्वनुपलब्धिर्ल्लिङ्गजा प्रतीतिरस्मिन्नेव\edlabel{pvv.358-3}\footnote{\label{pvv.358-3}  ३ शास्त्रं हि पुरुषार्थमधिवृत्तं तत्र च न सर्व्वमधिकृतं पुरुषचेतोवृत्तीनां प्रत्येकमानन्त्येनाशक्यवचनत्वादनियतान्निमित्ताद् भवनशीलत्वाच्च बहुत्वं । देशादिविप्रकृष्टाः पुरुषानुपयोगिनो न निर्देश्याः । लिङ्गस्यानुपलब्धेरतिशये उपलब्धिलक्षणप्राप्तत्वं तद्भावो यत्रास्ति ।} साध्ये {\color{DodgerBlue3}“प्रमाणमपि स्याल्लिङ्गाति\leavevmode\marginnote{\textenglish{71a/MA}} शयभाविनी”} लिङ्गविशेषप्रभवा । यथोक्तं प्राग्घेतुभेदव्यपेक्षयेति (।) उपलब्धिलक्षणप्राप्तानुपलब्धिलिङ्गजेत्यर्थः ।
	\pend
      
	  \bigskip
	  \begingroup
	  \large
	
	    
	    \stanza[\smallbreak]
	\label{pv.3.200b}\edlabel{pv.3.200b}\flagstanza{\tiny\textenglish{...3.200b}}स्वभावज्ञापकाज्ञानस्यायं न्याय उदाहृतः ॥ २०० ॥\&[\smallbreak]


	
	  \endgroup
	

	  \pstart यत् पुनरुक्तमप्रमाणमनुपलब्धिरिति ।\edlabel{pvv.358-4}\footnote{\label{pvv.358-4}  ४ भस्मविशेषेण किन्तु पिशाचादेः ।} यस्य कस्यचित् {\color{DodgerBlue3}“स्वभा”}वस्य ज्ञापकस्य\edlabel{pvv.358-5}\footnote{\label{pvv.358-5}  ५ स्वभावज्ञापकयोरज्ञानं तस्य ।} लिङ्गस्य चा{\color{DodgerBlue3}“ज्ञानस्या”}नुपलब्धे\edlabel{pvv.358-6}\footnote{\label{pvv.358-6}  ६ अदृश्यविषयायाः ।} {\color{DodgerBlue3}“रयं न्याय उदाहृतः”} । न हि स्वभावो नोपलभ्यत\edlabel{pvv.358-7}\footnote{\label{pvv.358-7}  ७ स्वभावाज्ञानेन प्रत्यक्षनिवृत्तिरुक्ता ।} इत्येव नास्ति । देशकालस्वभावविप्रकृष्टानामदर्शनेपि सत्त्वाविरोधात् । ततो \leavevmode\marginnote{\textenglish{359/s}} नास्ति विरक्त\edlabel{pvv.359-1}\footnote{\label{pvv.359-1}  १ अनुमानिवृत्तिरुक्ता लिङ्गाज्ञानेन ।}चेत इत्याद्य\edlabel{pvv.359-2}\footnote{\label{pvv.359-2}  २ हिंसाविरतिचेतनादेर्नाभ्युदयहेतुतादि ।}युक्तं (।) न च कारणमित्येव कार्याण्यव्यवधानतो भवन्ति (।) ततो नास्ति दानहिंसाविरतिचेतनानामभ्युदयहेतुता {\color{DodgerBlue3}“फलानन्तर्या”}भावादित्ययुक्तं\edlabel{pvv.359-3}\footnote{\label{pvv.359-3}  ३ मूषिकादिविषविकारवद् व्यवहितं फलं स्यात् ।}। (२००)
	\pend
      \label{div_pvv.3.201_3.202_3.203_3.204_3.205_3.206_3.207_3.208_3.209_3.210_3.211_3.212ab}\edlabel{div_pvv.3.201_3.202_3.203_3.204_3.205_3.206_3.207_3.208_3.209_3.210_3.211_3.212ab}
	  
	% new div opening: depth here is 2
	
	  \bigskip
	  \begingroup
	  \large
	
	    
	    \stanza[\smallbreak]
	\label{pv.3.201a}\edlabel{pv.3.201a}\flagstanza{\tiny\textenglish{...3.201a}}कार्ये तु कारकाज्ञानमभावस्यैव साधकम् ।\&[\smallbreak]


	
	  \endgroup
	

	  \pstart कारकाज्ञान\edlabel{pvv.359-4}\footnote{\label{pvv.359-4}  ४ कारणानुपलम्भः साध्येऽनुपलब्धिर्या दृश्योपि ।}न्तु कार्येऽभावस्यैव साधकं (।) न ह्यसति कारणे कार्यसंभवः\edlabel{pvv.359-5}\footnote{\label{pvv.359-5}  ५ यदि कथञ्चित् कारणाभावः सिध्येत् तदा कार्याभावः साध्यः यथा नात्र धूमोनग्नेः ।} ॥
	\pend
      
	  \bigskip
	  \begingroup
	  \large
	
	    
	    \stanza[\smallbreak]
	\label{pv.3.201b}\edlabel{pv.3.201b}\flagstanza{\tiny\textenglish{...3.201b}}स्वभावानुपलम्भश्च स्वभावेर्थस्य लिङ्गिनि ॥ २०१ ॥\&[\smallbreak]


	
	  \endgroup
	
	  \bigskip
	  \begingroup
	  \large
	
	    
	    \stanza[\smallbreak]
	\label{pv.3.202a}\edlabel{pv.3.202a}\flagstanza{\tiny\textenglish{...3.202a}}तदभावः प्रतीयेत हेतुना यदि केनचित् ॥\&[\smallbreak]


	
	  \endgroup
	

	  \pstart तथा{\color{DodgerBlue3}“र्थस्य”} व्यापकतया निश्चितस्य {\color{DodgerBlue3}“स्वभावस्यानुपलम्भश्च स्वभावे”} व्याप्ये लिङ्गिन्यसत्तया साध्ये साधनंतदा च कारणव्यापकानुपलब्धी गमिके {\color{DodgerBlue3}“यदि केन”} चिद्धेतुनोपलब्धिलक्षणप्राप्तानुपलम्भेनान्येन वा तयोः कारणव्यापकयोरभावः प्रतीयते (।) न तूपलम्भाभावमात्रेण सन्दिग्धासिद्धत्वात् ।
	\pend
      

	  \begin{center}%% label @type='head'
	\textbf{ख. स्वभावानुपलब्धिः}
	\end{center}
	

	  \pstart स्वभावानुपलम्भमाह ।
	\pend
      
	  \bigskip
	  \begingroup
	  \large
	
	    
	    \stanza[\smallbreak]
	\label{pv.3.202b}\edlabel{pv.3.202b}\flagstanza{\tiny\textenglish{...3.202b}}दृश्यस्य दर्शनाभावकारणाऽसम्भवे सति ॥ २०२ ॥\&[\smallbreak]


	
	  \endgroup
	
	  \bigskip
	  \begingroup
	  \large
	
	    
	    \stanza[\smallbreak]
	\label{pv.3.203a}\edlabel{pv.3.203a}\flagstanza{\tiny\textenglish{...3.203a}}भावस्यानुपलम्भस्य भावाभावः प्रतीयते ।\&[\smallbreak]


	
	  \endgroup
	

	  \pstart {\color{DodgerBlue3}“दृश्यस्य”} वस्तुनो {\color{DodgerBlue3}“दर्शनाभाव”}स्य यत् {\color{DodgerBlue3}“कारणं”} व्यवधानेन्द्रियवैकल्यादि तस्या{\color{DodgerBlue3}“सम्भवे”} सति {\color{DodgerBlue3}“भावस्या”}नुपलब्धस्य {\color{DodgerBlue3}“भाव”} (={\color{DodgerBlue3}“सत्ता”}){\color{DodgerBlue3}“ाभावः”} स्वभावानुपलब्धेः {\color{DodgerBlue3}“प्रतीयते”} ॥
	\pend
      

	  \begin{center}%% label @type='head'
	\textbf{ग. अनुपलब्धिरेवाभावः}
	\end{center}
	
	  \bigskip
	  \begingroup
	  \large
	
	    
	    \stanza[\smallbreak]
	\label{pv.3.203b}\edlabel{pv.3.203b}\flagstanza{\tiny\textenglish{...3.203b}}विरुद्धस्य च भावस्य भावे तद्भावबाधनात् ॥ २०३ ॥\&[\smallbreak]


	
	  \endgroup
	
	  \bigskip
	  \begingroup
	  \large
	
	    
	    \stanza[\smallbreak]
	\label{pv.3.204a}\edlabel{pv.3.204a}\flagstanza{\tiny\textenglish{...3.204a}}तद्विरुद्धोपलब्धौ स्यादसत्ताया विनिश्चयः ।\&[\smallbreak]


	
	  \endgroup
	\leavevmode\marginnote{\textenglish{360/s}}

	  \pstart {\color{DodgerBlue3}“विरुद्धस्य”}\edlabel{pvv.360-1}\footnote{\label{pvv.360-1}  १ यदभावः साध्यस्तद्विरुद्धस्यानयोरुपादानयोरन्योन्यवैगुण्यस्याश्रयत्वेनारम्भविरोधात् ।} निवर्त्तकस्य वह्नयादे{\color{DodgerBlue3}“र्भावस्य भावे त”}स्य {\color{DodgerBlue3}“श”}(?स)लिलादेर्निवर्त्त्यस्य {\color{DodgerBlue3}“भावबाधनात् । तद्विरुद्धोपलब्धौ”} सत्याम{\color{DodgerBlue3}“सत्ता\edlabel{pvv.360-2}\footnote{\label{pvv.360-2}  २ प्रतिषेध्यस्य ।}या विनिश्चयः”} स्यात् ।
	\pend
      

	  \pstart नन्वनुपलब्धेरभावसाधने को दृष्टान्तः । व्योमकुसुमादिरिति चेत् । अत्रापि यद्यनुपलब्धेरभावसिद्धिस्तदा दृष्टान्तान्तरापेक्षायामनवस्थाप्रसङ्गः । तदनपेक्षायां तदनुपलब्धिरभावाख्यं प्रमाणमस्तु ।
	\pend
      

	  \pstart असम्बद्धमेतत् । न ह्यभावोऽनुपलब्ध्या साध्यते । अनुपलब्धि\edlabel{pvv.360-3}\footnote{\label{pvv.360-3}  ३ उपलब्धिलक्षणप्राप्तस्य ।}रेव ह्यभावः स च सिद्ध एव । तथापि तु मूढः\edlabel{pvv.360-4}\footnote{\label{pvv.360-4}  ४ अभावतत्स्वभावानुपलब्धावभावव्यवहार एव साध्यते ।} तमव्यवहरन् निमित्तो\edlabel{pvv.360-5}\footnote{\label{pvv.360-5}  ५ अनुपलम्भेन ।}पदर्शनेन व्यवहार्यते\edlabel{pvv.360-6}\footnote{\label{pvv.360-6}  ६ व्यवहारं ।} । तथा च भावो\edlabel{pvv.360-7}\footnote{\label{pvv.360-7}  ७ निरुपाख्योपि ।}ऽभावस्य दष्टान्तः\edlabel{pvv.360-8}\footnote{\label{pvv.360-8}  ८ यथा दृश्यः सन् दृश्यते तथा दृश्योऽसन्ननुपलब्धेः ।} । तयोः स्वनैमित्तिक\edlabel{pvv.360-9}\footnote{\label{pvv.360-9}  ९ उपलब्ध्यनुपलब्ध्योर्भावाभावव्यवहारप्रवर्त्तनस्य । यस्य यत्र निमित्तं सकलमप्रतिबद्धमस्ति तत्र तेन भवितव्यं यथाऽङ्कुरादि । अस्ति चोपलब्धिलक्षणप्राप्तस्यानुपलब्धावसद्व्यवहारनिमित्तमिति स्वभावहेतुः अनुपलब्धिः ।}प्रवर्त्तनस्य सिद्धत्वात् ।
	\pend
      

	  \pstart ननु यदिदं\edlabel{pvv.360-10}\footnote{\label{pvv.360-10}  १० (दिग्) नागादिनोक्तं ।}न सन्ति प्रधानादयोऽनुपलब्धेरिति तत्र कथमसद्वयवहारविधिः सद्व्यवहारप्रतिषेधो वा । प्रधानादिशब्दवाच्यस्य प्रतिषेधे त\edlabel{pvv.360-11}\footnote{\label{pvv.360-11}  ११ वाच्यम्विना वाचकाप्रयोगात् प्रधानं ।}च्छब्दाप्रयोगात्\edlabel{pvv.360-12}\footnote{\label{pvv.360-12}  १२ प्रधानशब्दाप्रयोगे प्रतिषेधोपि प्रतिषेध्याकीर्त्तनान्निर्व्विषयस्याप्रयोगादयुक्त इति चोदकाशयः ।}।
	\pend
      

	  \pstart अत्राह ।
	\pend
      
	  \bigskip
	  \begingroup
	  \large
	
	    
	    \stanza[\smallbreak]
	\label{pv.3.204b}\edlabel{pv.3.204b}\flagstanza{\tiny\textenglish{...3.204b}}अनादिवासनोद्भूतविकल्पपरिनिष्ठितः ॥ २०४ ॥\&[\smallbreak]


	
	  \endgroup
	
	  \bigskip
	  \begingroup
	  \large
	
	    
	    \stanza[\smallbreak]
	\label{pv.3.205a}\edlabel{pv.3.205a}\flagstanza{\tiny\textenglish{...3.205a}}शब्दार्थस्त्रिविधो धर्मो भावाभावोभयाश्रयः ।\&[\smallbreak]


	
	  \endgroup
	

	  \pstart {\color{DodgerBlue3}“अनादिवि”}कल्पाभ्यास{\color{DodgerBlue3}“वासनाया उद्भूतविकल्पे परिनिष्ठितः”} प्रतिभासमानः शब्दार्थो धर्मो त्रिविधः । कथ\edlabel{pvv.360-13}\footnote{\label{pvv.360-13}  १३ कथं भावाश्रयोर्थजन्यत्वेनाविकल्पत्वप्रसङ्गात् कथमभावाश्रयस्तस्याकारणत्वात् । उभयेऽहेतुकत्वात् त्यक्तः ।}मित्याह (।) {\color{DodgerBlue3}“भावाभावोभयाश्रयः”} । सदसदुभय\leavevmode\marginnote{\textenglish{361/s}} विकल्पवासनाप्रभवत्वात् । तदध्यवसायेन तद्विषयत्वात् । {\color{DodgerBlue3}“तत्र भावोपादानो”} विकल्पः पटादिरभावोपादानः शशवि{\color{DodgerBlue3}“शा”} (?षा) णादिः । {\color{DodgerBlue3}“उभयोपादानः प्रधाने”}श्वरादिः ।
	\pend
      
	  \bigskip
	  \begingroup
	  \large
	
	    
	    \stanza[\smallbreak]
	\label{pv.3.205b}\edlabel{pv.3.205b}\flagstanza{\tiny\textenglish{...3.205b}}तस्मिन् भावानुपादाने साध्येऽस्यानुपलम्भनम् ॥ २०५ ॥\&[\smallbreak]


	
	  \endgroup
	
	  \bigskip
	  \begingroup
	  \large
	
	    
	    \stanza[\smallbreak]
	\label{pv.3.206a}\edlabel{pv.3.206a}\flagstanza{\tiny\textenglish{...3.206a}}तथा हेतुर्न तस्यैवाभावः शब्दप्रयोगतः ।\&[\smallbreak]


	
	  \endgroup
	

	  \pstart {\color{DodgerBlue3}“तस्मिन्”} शब्दार्थे प्रधानादौ {\color{DodgerBlue3}“भावानुपादाने”} भावभूतप्रधानाश्रये {\color{DodgerBlue3}“साध्ये”}ऽस्य\leavevmode\marginnote{\textenglish{71b/MA}} प्रधाना\edlabel{pvv.361-1}\footnote{\label{pvv.361-1}  १ बुद्धिवर्त्तिनो धर्मिणः ।}दे{\color{DodgerBlue3}“स्तथा”} (बाह्य) भावाश्रयत्वेना{\color{DodgerBlue3}“नुपलम्भनं हेतु\edlabel{pvv.361-2}\footnote{\label{pvv.361-2}  २ प्रधानादिविकल्पप्रतिभासस्य बाह्योपादानत्वानुपलम्भोस्तीति नापक्षधर्मः ।}”}र्व्यवहारसाधनः । न तु {\color{DodgerBlue3}“तस्य”}\edlabel{pvv.361-3}\footnote{\label{pvv.361-3}  ३ विकल्पस्थस्य ॥} शब्दार्थस्यैवाभावः प्रधानादि{\color{DodgerBlue3}“शब्द”}स्य तत्प्रतिपादकस्य {\color{DodgerBlue3}“प्रयोगतः”}\edlabel{pvv.361-4}\footnote{\label{pvv.361-4}  ४ यदि स्वलक्षणमभिधेयं स्यात् तदा तत्प्रतिषेधेऽनर्थकस्य शब्दस्याप्रयोगः स्यात् न चैतत् ।} ॥
	\pend
      

	  \pstart यदि तु वस्त्वेव शब्दविषयस्तदा (।)
	\pend
      
	  \bigskip
	  \begingroup
	  \large
	
	    
	    \stanza[\smallbreak]
	\label{pv.3.206b}\edlabel{pv.3.206b}\flagstanza{\tiny\textenglish{...3.206b}}परमार्थैकतानत्वे शब्दानामनिबन्धना ॥ २०६ ॥\&[\smallbreak]


	
	  \endgroup
	
	  \bigskip
	  \begingroup
	  \large
	
	    
	    \stanza[\smallbreak]
	\label{pv.3.207a}\edlabel{pv.3.207a}\flagstanza{\tiny\textenglish{...3.207a}}न स्यात् प्रवृत्तिरर्थेषु दर्शनान्तरभेदिषु ।\&[\smallbreak]


	
	  \endgroup
	

	  \pstart {\color{DodgerBlue3}“परमार्थै\edlabel{pvv.361-5}\footnote{\label{pvv.361-5}  ५ स्वलक्ष (त्वे)}कतानत्वे”} परमार्थैकपर(वृत्ति)त्वे {\color{DodgerBlue3}“शब्दानामर्थेषु दर्शनान्तरभेदिषु”} प्रतिदर्शनं भिन्नाभ्युप\edlabel{pvv.361-6}\footnote{\label{pvv.361-6}  ६ सिद्धान्त (त्वे) ।}गमेन नित्यत्वानित्यत्वत्रिगुणीमयत्वादिकल्पि{\color{DodgerBlue3}“तभेदेषु अनिबन्धना”} परमार्थनिबन्धनरहिता {\color{DodgerBlue3}“प्रवृत्तिर्न स्यात्”} । न हि परस्परविरुद्धा बहवो धर्मा एकत्र सन्ति ।
	\pend
      
	  \bigskip
	  \begingroup
	  \large
	
	    
	    \stanza[\smallbreak]
	\label{pv.3.207b}\edlabel{pv.3.207b}\flagstanza{\tiny\textenglish{...3.207b}}अतीताजातयोर्वापि न च स्यादनृतार्थता ॥ २०७ ॥\&[\smallbreak]


	
	  \endgroup
	
	  \bigskip
	  \begingroup
	  \large
	
	    
	    \stanza[\smallbreak]
	\label{pv.3.208a}\edlabel{pv.3.208a}\flagstanza{\tiny\textenglish{...3.208a}}वाचः कस्याश्चिदित्येषा बौद्धार्थविषया मता ।\&[\smallbreak]


	
	  \endgroup
	

	  \pstart {\color{DodgerBlue3}“अतीताजातयोर्व्वप्यसतोर्न”} स्याच्छब्दवृत्तिः । {\color{DodgerBlue3}“न च कस्याश्चिद् वाचोऽनृतार्थता स्यात्”} । अर्थमन्तरेण शब्दाभावात् । यस्मादेते दोषा वस्तुविषयत्वे वाच इति तस्मादेषा {\color{DodgerBlue3}“बौद्धा\edlabel{pvv.361-7}\footnote{\label{pvv.361-7}  ७ वाक् ।}र्थविषया”} कल्पि\edlabel{pvv.361-8}\footnote{\label{pvv.361-8}  ८ विकल्पभासी ।}तार्थगोचरा मता ।
	\pend
      

	  \pstart यश्च शब्दार्थः तस्य भावानुपादानत्वं साध्यते न तु स एव निषिध्यतेऽन्यथा (।)
	\pend
      
	  \bigskip
	  \begingroup
	  \large
	
	    
	    \stanza[\smallbreak]
	\label{pv.3.208b}\edlabel{pv.3.208b}\flagstanza{\tiny\textenglish{...3.208b}}शब्दार्थापह्नवे साध्ये धर्माधारनिराकृतेः ॥ २०८ ॥\&[\smallbreak]


	
	  \endgroup
	
	  \bigskip
	  \begingroup
	  \large
	
	    
	    \stanza[\smallbreak]
	\label{pv.3.209a}\edlabel{pv.3.209a}\flagstanza{\tiny\textenglish{...3.209a}}न साध्यः समुदायः स्यात् सिद्धो धर्मश्च केवलः ।\&[\smallbreak]


	
	  \endgroup
	\leavevmode\marginnote{\textenglish{362/s}}

	  \pstart {\color{DodgerBlue3}“शब्दार्थस्यापह्नवे साध्ये धर्माधार”}\edlabel{pvv.362-1}\footnote{\label{pvv.362-1}  १ नास्तित्वं साध्यो धर्मः सिद्धान्ती ।}स्य धर्मिणो\edlabel{pvv.362-2}\footnote{\label{pvv.362-2}  २ प्रधानादिशब्दार्थस्य ।} {\color{DodgerBlue3}“निराकृतेः समुदायः साध्यो न स्यात्”} । धर्मिधर्मसमुदायश्चानुमेयः । धर्म एव केवलः साध्यते इति चेत् {\color{DodgerBlue3}“सिद्धो\edlabel{pvv.362-3}\footnote{\label{pvv.362-3}  ३ नास्तित्वमात्रस्य क्वचित् सिद्धत्वात् ।} धर्मश्चा”}भावादिः {\color{DodgerBlue3}“केवलः”} (।) किमर्थं साधनीयः प्रधानादिविकल्पस्य भावानुपादानत्वं तु न सिद्धं तदेव साध्यं युक्तं ।
	\pend
      

	  \pstart किञ्च\edlabel{pvv.362-4}\footnote{\label{pvv.362-4}  ४ प्रतिज्ञा ।} (।)
	\pend
      
	  \bigskip
	  \begingroup
	  \large
	
	    
	    \stanza[\smallbreak]
	\label{pv.3.209b}\edlabel{pv.3.209b}\flagstanza{\tiny\textenglish{...3.209b}}सदसत्पक्षभेदेन शब्दार्थानपवादिभिः ॥ २०९ ॥\&[\smallbreak]


	
	  \endgroup
	
	  \bigskip
	  \begingroup
	  \large
	
	    
	    \stanza[\smallbreak]
	\label{pv.3.210a}\edlabel{pv.3.210a}\flagstanza{\tiny\textenglish{...3.210a}}वस्त्वेव चिन्त्यते ह्यत्र प्रतिबद्धः फलोदयः ।\&[\smallbreak]


	
	  \endgroup
	

	  \pstart {\color{DodgerBlue3}“सदसत्पक्षभेदेन वस्त्वेव”} व्यवहारिभिः शब्दार्थानपवादिभिः चिन्त्यते नावस्तु । हि यस्माद् {\color{DodgerBlue3}“अत्र”} वस्तुनि {\color{DodgerBlue3}“फल”}स्यार्थक्रियाया {\color{DodgerBlue3}“उदयः प्रतिबद्धः”} ।
	\pend
      

	  \pstart ततश्च (।)
	\pend
      
	  \bigskip
	  \begingroup
	  \large
	
	    
	    \stanza[\smallbreak]
	\label{pv.3.210b}\edlabel{pv.3.210b}\flagstanza{\tiny\textenglish{...3.210b}}अर्थक्रियाऽसमर्थस्य विचारैः किं तद््र्थिनाम् ॥ २१० ॥\&[\smallbreak]


	
	  \endgroup
	
	  \bigskip
	  \begingroup
	  \large
	
	    
	    \stanza[\smallbreak]
	\label{pv.3.211a}\edlabel{pv.3.211a}\flagstanza{\tiny\textenglish{...3.211a}}षण्ढस्य रूपे वैरूप्ये कामिन्या किं परीक्षया ।\&[\smallbreak]


	
	  \endgroup
	

	  \pstart {\color{DodgerBlue3}“अर्थक्रियायामसमर्थस्य”} शब्दार्थादे{\color{DodgerBlue3}“र्व्विचारैः”} सदसत्पक्षचिन्ताभि{\color{DodgerBlue3}“स्तदर्थिना”}मर्थक्रियार्थिनां {\color{DodgerBlue3}“किं”} न किञ्चित् प्रयोजनं\edlabel{pvv.362-5}\footnote{\label{pvv.362-5}  ५ दृष्टान्तमाह ।} (।) {\color{DodgerBlue3}“षण्ढस्य”} नपुन्सकस्य {\color{DodgerBlue3}“रूपे वैरूप्ये”} वा कामिन्या वृषस्यन्त्या योषितः {\color{DodgerBlue3}“किं परीक्षया”} ।\edlabel{pvv.362-6}\footnote{\label{pvv.362-6}  ६ उ द्यो त क रा द्युक्तदोषनिरासाय पृच्छति नागेन (? दिग्नागः ) ।}
	\pend
      

	  \begin{center}%% label @type='head'
	\textbf{घ. कल्पितस्यानुपलब्धिर्धर्मः}
	\end{center}
	

	  \pstart यत् पुनराचार्येणोक्तं कल्पितस्यानुपलब्धिर्द्धर्म्म इति तस्य कोर्थः ।\edlabel{pvv.362-7}\footnote{\label{pvv.362-7}  ७ उत्तरमाह ।}
	\pend
      
	  \bigskip
	  \begingroup
	  \large
	
	    
	    \stanza[\smallbreak]
	\label{pv.3.211b}\edlabel{pv.3.211b}\flagstanza{\tiny\textenglish{...3.211b}}शब्दार्थः कल्पनाज्ञानविषयत्वेन कल्पितः ॥ २११ ॥\&[\smallbreak]


	
	  \endgroup
	
	  \bigskip
	  \begingroup
	  \large
	
	    
	    \stanza[\smallbreak]
	\label{pv.3.212a}\edlabel{pv.3.212a}\flagstanza{\tiny\textenglish{...3.212a}}धर्मो वस्त्वाश्रयाऽसिद्धिरस्योक्ता न्यायवादिना ।\&[\smallbreak]


	
	  \endgroup
	

	  \pstart {\color{DodgerBlue3}“कल्पनाज्ञान”}स्य {\color{DodgerBlue3}“विषयत्वेन कल्पित”} इष्टः प्रधानादि{\color{DodgerBlue3}“शब्दार्थः । अस्य\edlabel{pvv.362-8}\footnote{\label{pvv.362-8}  ८ प्रधानादिशब्दार्थस्य ।}”} कल्पितस्य {\color{DodgerBlue3}“वस्त्वा\edlabel{pvv.362-9}\footnote{\label{pvv.362-9}  ९ वस्तुनो बाह्यस्य प्रधानस्याश्रयणं तेनासिद्धिरनुपलब्धिर्द्धर्म उक्तो लिङ्गभूतो भावानुपादानत्वे साध्ये प्रमाणेन चेत् सत्त्वं प्रधानस्य । नानुपलब्धिः धर्मः । असत्त्वेप्यसत्त्वादिति परोक्तमपास्तमनेन शब्दार्थस्यैव कल्पितत्वात् ।}श्रयाऽसिद्धि”}र्व्वस्त्वाधिष्ठानत्वानुपलब्धि{\color{DodgerBlue3}“र्द्धर्मो न्यायवादिना”}चार्येणोक्ता ।
	\pend
      
	  
	% new div opening: depth here is 1
	
\section[{४. आगमचिन्ता}]{४. आगमचिन्ता}\label{div_pvv.3.212cd_3.213_3.214_3.215_3.216}\edlabel{div_pvv.3.212cd_3.213_3.214_3.215_3.216}
	  
	% new div opening: depth here is 2
	

	  \pstart \leavevmode\marginnote{\textenglish{363/s}}ननु यदुक्तं (।) प्रमाणत्रय\edlabel{pvv.363-1}\footnote{\label{pvv.363-1}  १ आगमेन सह ।}निवृत्तावपि नार्थाभावनिश्चय इति तन्मा {\color{DodgerBlue3}“भूत्”} प्रत्यक्षानुमानयोरसर्व्वविषयत्वात् तन्निवृत्त्या (ऽ) भावनिश्चयः । आगमस्तु सर्व्वविषय इति तन्निवृत्तौ युक्तोऽर्थासत्त्वनिश्चय इत्याह\edlabel{pvv.363-2}\footnote{\label{pvv.363-2}  २ सिद्धान्ती ।} ।
	\pend
      
	  \bigskip
	  \begingroup
	  \large
	
	    
	    \stanza[\smallbreak]
	\label{pv.3.212b}\edlabel{pv.3.212b}\flagstanza{\tiny\textenglish{...3.212b}}नान्तरीयकताऽभावाच्छब्दानां वस्तुभिस्सह ॥ २१२ ॥\&[\smallbreak]


	
	  \endgroup
	
	  \bigskip
	  \begingroup
	  \large
	
	    
	    \stanza[\smallbreak]
	\label{pv.3.213a}\edlabel{pv.3.213a}\flagstanza{\tiny\textenglish{...3.213a}}नार्थसिद्धिस्ततस्ते हि वक्त्त्रभिप्रायसूचकाः ॥\&[\smallbreak]


	
	  \endgroup
	

	  \pstart {\color{DodgerBlue3}“नान्तरीयकता”}या अविनाभावस्या{\color{DodgerBlue3}“भावाद् वस्तुभिः सह शब्दानां । ततः”} शब्देभ्यो {\color{DodgerBlue3}“नार्थस्य सिद्धि”}र्निश्चयः । किन्तर्हि तेभ्यो गम्यत इत्याह\edlabel{pvv.363-3}\footnote{\label{pvv.363-3}  ३ ...“शास्त्राधिकार” \href{http://http://sarit.indology.info/?cref=pv.3.198}{(३।१९८)} इत्यत्र नागमे सर्व्वार्थनिबन्धनमुक्तं यत् तद् बाह्यार्थे यद्यपि घटविवक्षातः पटशब्दस्योत्पत्तिस्तथापि स्थानकरणाभिधातादेरेव साक्षात्कारणादुत्पत्तेर्व्यभिचाराभावान्नाहेतुकत्वं ।} । {\color{DodgerBlue3}“वक्तुरभिप्रायस्य”} विवक्षाया{\color{DodgerBlue3}“स्ते”} शब्दाः {\color{DodgerBlue3}“सूचका”}स्तदन्वयव्यतिरेकानुविधायित्वात् । न च विवक्षा यथार्थं भवति\edlabel{pvv.363-4}\footnote{\label{pvv.363-4}  ४ आगमप्रामाण्यमभ्युपगम्यात्र तु नैव बाह्यार्थेऽस्य प्रामाण्यमित्युक्तम् ।} येन परंपरया तत्सम्वादः स्यात् । विसम्वादाभिप्रायादज्ञानाद् वाऽन्यथापि विवक्षासम्भवात् ।
	\pend
      \leavevmode\marginnote{\textenglish{72a/MA}}
	  \bigskip
	  \begingroup
	  \large
	
	    
	    \stanza[\smallbreak]
	\label{pv.3.213b}\edlabel{pv.3.213b}\flagstanza{\tiny\textenglish{...3.213b}}आप्तवादाविसंवादसामान्यादनुमानता ॥ २१३ ॥\&[\smallbreak]


	
	  \endgroup
	

	  \pstart यद्येवं सर्व्वमेव वचनं प्रवृत्तिकामानां परीक्षार्हं स्यात् । कश्च सम्वादार्थः कथञ्चाप्तवादसामान्यादनुमानतास्या चा र्ये णोक्तेत्याह ।\edlabel{pvv.363-5}\footnote{\label{pvv.363-5}  ५ यो य आप्तवादः सोऽविसंवादी यथा क्षणिकाः सर्व्वसंस्कार इत्यादिकः । आप्तवादश्चायमत्यन्तपरोक्षेप्यर्थे इत्यविसम्वादसामान्यादागमस्य बाह्येर्थे (दिग्)नागेनानुमानमुक्तमित्यभ्युपेतबाधामाह । उच्यते न पुरुषोऽनाश्रित्यागमप्रामाण्यमसितुं समर्थः । प्रत्यक्षफलाया हिंसादिविरतेः स्वर्गादिश्रुतेरविरतेर्नकादिश्रुतेः । तद्भावे विरोधाभावच्च । तत् सति प्रवर्त्तितव्ये वरमेवं प्रवृत्त इति परीक्षया प्रामाण्यमाह (।) तच्च शास्त्रं न सर्व्वमधिकृतं किन्तु ।}
	\pend
      
	  \bigskip
	  \begingroup
	  \large
	
	    
	    \stanza[\smallbreak]
	\label{pv.3.214}\edlabel{pv.3.214}\flagstanza{\tiny\textenglish{....3.214}}सम्बद्धानुगुणोपायं पुरुषार्थाभिधायकम् ।&परीक्षाधिकृतं वाक्यमतोनधिकृतम्परम् ॥ २१४ ॥\&[\smallbreak]


	
	  \endgroup
	

	  \pstart {\color{DodgerBlue3}“सम्बद्ध”}वाक्यानां परस्पराभिसम्बद्धानामेकार्थोपसंहारात् (।) न च दशदाडिमादिवाक्यमिवैकार्थानभिधायि । {\color{DodgerBlue3}“अनुगुणोणायं”} शक्यानुष्ठानोपेयसाधनं न तु \leavevmode\marginnote{\textenglish{364/s}} विषशमनतक्षकफणारत्नालङ्कारोपदेशकमिव {\color{DodgerBlue3}“पुरुषार्थस्य”} स्वर्गापवर्गस्या{\color{DodgerBlue3}“भिधायकं”} न तु काकदन्त{\color{DodgerBlue3}“परीक्षो”}पदेशकमिव परीक्षायां प्रवृत्त्यर्हविषयम{\color{DodgerBlue3}“धिकृतं वाक्यं । अतोऽपरमनधिकृतम”}नवधानार्हत्वात् ।
	\pend
      
	  \bigskip
	  \begingroup
	  \large
	
	    
	    \stanza[\smallbreak]
	\label{pv.3.215}\edlabel{pv.3.215}\flagstanza{\tiny\textenglish{....3.215}}प्रत्यक्षेणानुमानेन द्विविधेनाप्यबाधनम् ।&दृष्टादृष्टार्थयोरस्याविसंवादः तदर्थयोः ॥ २१५ ॥\&[\smallbreak]


	
	  \endgroup
	

	  \pstart अस्य च परीक्षार्हस्य वाक्यस्याविसम्वादः {\color{DodgerBlue3}“तदर्थयो”}\edlabel{pvv.364-1}\footnote{\label{pvv.364-1}  १ गुणत्रययुक्तञ्च यद्यविसंवादि तदा प्रवर्तित्तव्यमित्यविसम्वादमाह ।} रागमाभिधेययो{\color{DodgerBlue3}“र्दृष्टादृष्टयोः”}प्रत्यक्षाप्रत्यक्ष\edlabel{pvv.364-2}\footnote{\label{pvv.364-2}  २ प्रत्यक्षानुमानविषययोः ।} योरर्थयोः {\color{DodgerBlue3}“प्रत्यक्षेणानुमानेन”} च {\color{DodgerBlue3}“द्विविधेने”}ति वस्तुबलभाविनाऽगमाश्रयेण चा{\color{DodgerBlue3}“बाधन”}मन्येषाञ्च बाधनं नाम । यथा प्रत्यक्षत्वेन सम्मतानां पञ्चानां स्कन्धानां प्रत्यक्षेणाबाधनं सिद्धिरेव अप्रत्यक्षत्वेनेष्टानां शब्दादि\edlabel{pvv.364-3}\footnote{\label{pvv.364-3}  ३ सां ख्ये शब्दादिस्वभावानां सुखदुःखमोहानां प्रत्यक्षेणाप्रतीतेः । वैशेषिकादेर्द्रव्यमाकाशादि । अनेकद्रव्यञ्च द्रव्यमवयवी । कर्मोत्क्षेपणादि । सामान्यं सत्ता गोत्वादि । आदिना विभागादि ।}त्रिगुणमयत्वद्रव्यकर्मसामान्यसंयोगादीनाञ्च तेन बाधनं\edlabel{pvv.364-4}\footnote{\label{pvv.364-4}  ४ नीलादिव्यतिरेकेणानुपलब्धेः ।} । अनुमेयत्वेनेष्टानां चतुरार्यसत्यानां वस्तुबलप्रवृत्ते\edlabel{pvv.364-5}\footnote{\label{pvv.364-5}  ५ आगमापेक्षेण ।}नानुमानेनाबाधनं सिद्धिरेव । अननुमेयत्वेनेष्टानाञ्चात्मेश्वरादीनामनुमानेन\edlabel{pvv.364-6}\footnote{\label{pvv.364-6}  ६ लिङ्गाभावान्नानुमेयता ।} बाध एव । अत्यन्तपरोक्षाणां रागादिहेतुकाऽधर्मप्रहाणादीनामागमाश्रयानुमानेनाबाधनं सिद्धिरेवं\edlabel{pvv.364-7}\footnote{\label{pvv.364-7}  ७ रागद्वेषादिस्वभावमधर्मन्तदुत्थं कायकर्मादि चाधर्ममभ्युपेत्य स्नानाद्यनुक्तेः ।} रागादिहेतुत्वेनेष्टस्य हेतुत्वानुपरोधिनः स्नोनोवासाग्निहोत्रादेः प्रहाणोपायतयाऽनुपदेशात् हेतुव्याघातस्योपायत्वेनोपदेशाच्चैवं विधमबाधनम{\color{DodgerBlue3}“विसम्वाद”} इष्टः ।
	\pend
      
	  \bigskip
	  \begingroup
	  \large
	
	    
	    \stanza[\smallbreak]
	\label{pv.3.216}\edlabel{pv.3.216}\flagstanza{\tiny\textenglish{....3.216}}आप्तवादाविसंवादसामान्यादनुमानता ।&बुद्धेरगत्याऽभिहिता निषिद्धाप्यस्य गोचरे ॥ २१६ ॥\&[\smallbreak]


	
	  \endgroup
	

	  \pstart तस्यैवं भूतस्या{\color{DodgerBlue3}“प्तवाद”}स्यादृष्टव्यभिचारस्या{\color{DodgerBlue3}“विसम्वादसामान्यात्”}\edlabel{pvv.364-8}\footnote{\label{pvv.364-8}  ८ शक्यपरिच्छेदेऽविसम्वादवत्परोक्षेप्याप्तवादाल्लिङ्गादुत्पन्नायाः सम्वादबुद्धेरनुमानता ।} प्रत्यक्षानुमानागम्येप्यर्थे उत्पन्नाया {\color{DodgerBlue3}“बुद्धे”}रविसम्वादा{\color{DodgerBlue3}“दनुमानता”} चा चार्य दि ग्ना गे ना{\color{DodgerBlue3}“भिहिता”}\leavevmode\marginnote{\textenglish{365/s}} {\color{DodgerBlue3}“गत्या”} । अत्यन्तपरोक्षेष्वर्थेषु दानहिंसाचेतनादिष्वर्थानर्थश्रवणादागमप्रामाण्यमनाश्रित्य स्थातुमसामर्थ्यादेतद्भावे विरोधाभावाच्च सत्यां प्रवृत्तौ\edlabel{pvv.365-1}\footnote{\label{pvv.365-1}  १ न तु प्रमाणगम्य एवार्थे विसंवादकात् ।} वरमेवं प्रवृत्तिरित्यगत्या\edlabel{pvv.365-2}\footnote{\label{pvv.365-2}  २ अतोन्यथापरोक्षे प्रवृत्त्यसंभवात् ।}ऽनुमानतोक्ता । न तु वस्तुतो वचनानामर्थेषु नान्तरीयकत्वाभावात् । (२१६)
	\pend
      \label{div_pvv.3.217}\edlabel{div_pvv.3.217}
	  
	% new div opening: depth here is 2
	

	  \pstart किम्वा\edlabel{pvv.365-3}\footnote{\label{pvv.365-3}  ३ अथवा ।} (।)
	\pend
      
	  \bigskip
	  \begingroup
	  \large
	
	    
	    \stanza[\smallbreak]
	\label{pv.3.217}\edlabel{pv.3.217}\flagstanza{\tiny\textenglish{....3.217}}हेयोपादेयतत्त्वस्य सोपायस्य प्रसिद्धितः ।&प्रधानार्थाविसम्वादादनुमानम्परत्र वा ॥ २१७ ॥\&[\smallbreak]


	
	  \endgroup
	

	  \pstart {\color{DodgerBlue3}“हेय”}स्य दुःखसत्यस्य {\color{DodgerBlue3}“उपादे”}यस्य निरोधसत्यस्य\edlabel{pvv.365-4}\footnote{\label{pvv.365-4}  ४ तत्त्वमविपरीतं रूपं ।} {\color{DodgerBlue3}“सोपायस्य”} यथाक्रमं समुदयसत्यस्य समार्गसत्यस्य चागमोक्तवस्तुबलप्रवृत्तेनानुमाने {\color{DodgerBlue3}“प्रसिद्धितो”} निश्चयात् । सत्यचतुष्टयाधिगमस्य निर्व्वाणहेतुत्वेन {\color{DodgerBlue3}“प्रधानार्थ”}स्या{\color{DodgerBlue3}“विसम्वादात्”} । {\color{DodgerBlue3}“परत्रा”}त्यन्तपरोक्षेप्यर्थे भग वद्वचनादुत्पन्नं ज्ञानम{\color{DodgerBlue3}“नुमानं”} युक्तमिति वा पक्षान्तरं । (२१७)
	\pend
      \label{div_pvv.3.218}\edlabel{div_pvv.3.218}
	  
	% new div opening: depth here is 2
	
	  \bigskip
	  \begingroup
	  \large
	
	    
	    \stanza[\smallbreak]
	\label{pv.3.218}\edlabel{pv.3.218}\flagstanza{\tiny\textenglish{....3.218}}पुरुषातिशयापेक्षं यथार्थमपरे विदुः ।&इष्टोयमर्थः प्रत्येतुं शक्यः सोतिशयो यदि ॥ २१८ ॥\&[\smallbreak]


	
	  \endgroup
	

	  \pstart {\color{DodgerBlue3}“अपरे”} नै या यि का दयः {\color{DodgerBlue3}“पुरुष”}स्या{\color{DodgerBlue3}“तिशयापेक्षं”} यथाभूतार्थदर्शि तदाख्यात् पुरुषप्रणीतं वचनं {\color{DodgerBlue3}“यथार्थं”} सत्यार्थं प्रतिजानीयुः (।)
	\pend
      

	  \pstart सिद्धान्तमाह (।) पुरुषातिशयप्रणीतं वचनम प्रमाणमिती{\color{DodgerBlue3}“ष्टोऽयमर्थो यदि”} पुरुषाणां {\color{DodgerBlue3}“सोतिश”}यो ज्ञातुं शक्यः स्यात् ।
	\pend
      \label{div_pvv.3.219_3.220_3.221_3.222_3.223}\edlabel{div_pvv.3.219_3.220_3.221_3.222_3.223}
	  
	% new div opening: depth here is 2
	

	  \begin{center}%% label @type='head'
	\textbf{(१) पौरुषेयत्वे}
	\end{center}
	

	  \begin{center}%% label @type='head'
	\textbf{क. पुरुषातिशयप्रणीतं वचनं प्रमाणम्}
	\end{center}
	

	  \pstart किन्तु (।)
	\pend
      
	  \bigskip
	  \begingroup
	  \large
	
	    
	    \stanza[\smallbreak]
	\label{pv.3.219}\edlabel{pv.3.219}\flagstanza{\tiny\textenglish{....3.219}}अयमेवं न वेत्यन्यदोषानिर्दोषतापि वा ।&दुर्लभत्वात् प्रमाणानां दुर्बोधेत्यपरे विदुः ॥ २१९ ॥\&[\smallbreak]


	
	  \endgroup
	

	  \pstart {\color{DodgerBlue3}“अयं”} पुमा{\color{DodgerBlue3}“नेवं”}दोषवान् {\color{DodgerBlue3}“न वा”} निर्दोष इत्यन्यस्य {\color{DodgerBlue3}“दोषा निर्दोषतापि वा प्रमाणानां\edlabel{pvv.365-5}\footnote{\label{pvv.365-5}  ५ अन्यगुणदोषनिश्चायकानां ।} दुर्लभत्वाद् दुर्बोधे\edlabel{pvv.365-6}\footnote{\label{pvv.365-6}  ६ चैतसानामतीन्द्रियत्वात् नापि रागाद्यनुमेयाः कायवचसां प्रतिसंख्ययान्यथात्वस्य शक्यत्वात् ।}त्यपरे”} सौ ग ता {\color{DodgerBlue3}“विदुः”} । दुर्ब्बोधेत्यन्यदोषा इत्यनेन\leavevmode\marginnote{\textenglish{72b/MA}} लिङ्ग\edlabel{pvv.365-7}\footnote{\label{pvv.365-7}  ७ पुल्लिङ्गबहुवचनं कृत्वा ।} वचनविपरिणामेन सम्बन्धनीयं ।
	\pend
      

	  \pstart \leavevmode\marginnote{\textenglish{366/s}}तत्किमशक्योच्छेदा दोषा नेत्याह ।
	\pend
      
	  \bigskip
	  \begingroup
	  \large
	
	    
	    \stanza[\smallbreak]
	\label{pv.3.220}\edlabel{pv.3.220}\flagstanza{\tiny\textenglish{....3.220}}सर्व्वेषां सविपक्षत्वान्निर्ह्नासातिशयं श्रितः ।&सात्मीभावात् तदभ्यासाद् हीयेरन्नास्रवाः क्वचित् ॥ २२० ॥\&[\smallbreak]


	
	  \endgroup
	

	  \pstart {\color{DodgerBlue3}“सर्व्वेषा”}मास्रवाणां रागादीनांप्रतिपक्षसंमुखीभावाभावयोर्न्निर्ह्रासातिशयावुपचयापचयौ श्रयन्त इति {\color{DodgerBlue3}“निर्ह्रासातिशयं श्रित”}स्तेषां {\color{DodgerBlue3}“सविपक्षत्वात्”} प्रतिपक्ष\edlabel{pvv.366-1}\footnote{\label{pvv.366-1}  १ नैरात्म्यस्य ।}सम्भवात् तस्य प्रतिपक्षस्या{\color{DodgerBlue3}“भ्यासात् सात्मीभावादास्रवा”} क्व1चिच्चित्तसन्ताने {\color{DodgerBlue3}“हीयेरन्नि”}ति खलु निर्दोषपुरुषापलापः क्रियते किन्तु तदवधारणोपायो नास्तीत्युच्यते । इच्छाधीनस्य व्याहारस्यान्यथापि कर्त्तुं शक्यत्वात् । न ततस्तथार्थनिश्चयः ।
	\pend
      

	  \pstart अथ सात्मीयभूतप्रतिपक्षस्य मार्गाभ्यासान्निर्दोषतायामपि सत्यां विपक्षाभ्यासात् पुनर्दोषोत्पत्तिरित्याह ।
	\pend
      
	  \bigskip
	  \begingroup
	  \large
	
	    
	    \stanza[\smallbreak]
	\label{pv.3.221}\edlabel{pv.3.221}\flagstanza{\tiny\textenglish{....3.221}}निरुपद्रवभूतार्थस्वभावस्य विपर्ययैः ।&न बाधा यत्नवत्वेऽपि बुद्धेस्तत्पक्षपाततः ॥ २२१ ॥\&[\smallbreak]


	
	  \endgroup
	

	  \pstart {\color{DodgerBlue3}“निरुपद्रव”}स्य दोषराशेरुद्वेजकस्य प्रहाणात् । {\color{DodgerBlue3}“भूतार्थ”}स्य प्रमाणपरिदृष्टा\edlabel{pvv.366-2}\footnote{\label{pvv.366-2}  २ भूतार्थत्वादेव मार्गश्चित्तस्य स्वभाव उक्तः ।}र्थविषयत्वात् । {\color{DodgerBlue3}“स्व\edlabel{pvv.366-3}\footnote{\label{pvv.366-3}  ३ दोषप्रतिपक्षस्य विपर्ययैः सो यत्राभूतार्था स्वभावैर्दोषैः न बाधार्थः श्रोत्रियः सन् कापालिको भवति तस्य पूर्व्वा घृणा यथाऽयत्नमशक्यनिवर्त्त्या तद्वत् ।}भाव”}स्यानारोपितत्वात् (।) मार्गसात्म्यस्य विपक्षेण {\color{DodgerBlue3}“न बाधा यत्नवत्वेपि”} । यत्न एव तावन्न सम्भवति विपक्षाभ्यासे दोषदर्शनात् । {\color{DodgerBlue3}“यत्नवत्वेपि”} तु {\color{DodgerBlue3}“बुद्धेस्तत्र”} मार्गसात्म्येऽभिरुचिविषयत्वेन {\color{DodgerBlue3}“पक्षपाततो\edlabel{pvv.366-4}\footnote{\label{pvv.366-4}  ४ बहुमानतः ।} न बाधा”} । न हि रज्वां निवृत्तसर्प्पभ्रमः सर्प्पं भावयितुं यतते कश्चित् (।) भूतार्थस्य दर्शनात् ।
	\pend
      

	  \begin{center}%% label @type='head'
	\textbf{ख. सत्कायदर्शनं दोषकारणम्}
	\end{center}
	

	  \pstart कः पुनर्दोषाणां हेतुर्यत्प्रहाणादमी प्रहीयन्त इत्या3ह ।
	\pend
      
	  \bigskip
	  \begingroup
	  \large
	
	    
	    \stanza[\smallbreak]
	\label{pv.3.222a}\edlabel{pv.3.222a}\flagstanza{\tiny\textenglish{...3.222a}}सर्व्वासां दोषजातीनां जातिः सत्कायदर्शनात् ।\&[\smallbreak]


	
	  \endgroup
	

	  \pstart {\color{DodgerBlue3}“सर्व्वासां दोषजातीनां”} दोषप्रकाराणां {\color{DodgerBlue3}“जाति”}र्जन्म {\color{DodgerBlue3}“सत्कायदर्शनादा”}त्माभिनिवेशात् ।
	\pend
      

	  \pstart नन्वविद्याहेतुकाः क्लेशा भ ग व तो क्तो इत्याह ।
	\pend
      
	  \bigskip
	  \begingroup
	  \large
	
	    
	    \stanza[\smallbreak]
	\label{pv.3.222b}\edlabel{pv.3.222b}\flagstanza{\tiny\textenglish{...3.222b}}साऽविद्या तत्र तत्स्नेहस्तस्माद् द्वेषादिसम्भवः ॥ २२२ ॥\&[\smallbreak]


	
	  \endgroup
	\leavevmode\marginnote{\textenglish{367/s}}

	  \pstart {\color{DodgerBlue3}“साऽविद्या”} सत्कायदर्शनमेवाविद्याऽन्यत्रोच्यत इति नास्ति विरोधः । {\color{DodgerBlue3}“तत्र”} सत्कायदर्शने सति तेष्वात्मीयेषु {\color{DodgerBlue3}“स्नेहस्तस्मा”}दात्मीयेषु स्नेहात् तदपकारिषु {\color{DodgerBlue3}“द्वेषादी”}नां {\color{DodgerBlue3}“सम्भव”} इति दोषोत्पत्तिक्रमः । अतो नैरात्म्यदर्शनं मार्गो युक्तः सत्कायदृष्टिप्रतिपक्षत्वात् ।
	\pend
      

	  \pstart यतश्च सत्त्वदृष्टिरविद्या ।
	\pend
      
	  \bigskip
	  \begingroup
	  \large
	
	    
	    \stanza[\smallbreak]
	\label{pv.3.223a}\edlabel{pv.3.223a}\flagstanza{\tiny\textenglish{...3.223a}}मोहो निदानं दोषाणां अत्र एवाभिधीयते ।&सत्कायदृष्टिरन्यत्र;\&[\smallbreak]


	
	  \endgroup
	

	  \pstart {\color{DodgerBlue3}“अत एव मोहो”}ऽविद्या {\color{DodgerBlue3}“दोषाणां निदानमभिधीयते”} । भगवता“ऽविद्या हेतुकाः सर्वै क्लेशा” इति पुनरन्यत्र प्रदेशे “सत्कायदृष्टिर्दोषनिदानम” {\color{DodgerBlue3}“भिधीयते”} ।
	\pend
      

	  \pstart नन्वन्येपीन्द्रियविषया योनिशोमनस्कारादयो दोषहेतवस्तत्किमविद्यासत्कायदृष्टी एवाभिहिते इत्याह ।
	\pend
      
	  \bigskip
	  \begingroup
	  \large
	
	    
	    \stanza[\smallbreak]
	\label{pv.3.223b}\edlabel{pv.3.223b}\flagstanza{\tiny\textenglish{...3.223b}}तत्प्रहाणे प्रहाणतः ॥ २२३ ॥\&[\smallbreak]


	
	  \endgroup
	

	  \pstart तस्य मोहस्य {\color{DodgerBlue3}“सत्कायदृष्टि”}लक्षणस्य {\color{DodgerBlue3}“प्रहाणे”} दोषाणां {\color{DodgerBlue3}“प्रहाणतः”} प्राधान्यात् स एवोक्तो नेतर इत्यर्थः ॥ (२२३)
	\pend
      
	  
	% new div opening: depth here is 1
	
\section[{५. अपौरुषेयत्वे}]{५. अपौरुषेयत्वे}\label{div_pvv.3.224}\edlabel{div_pvv.3.224}
	  
	% new div opening: depth here is 2
	

	  \begin{center}%% label @type='head'
	\textbf{(१) वेदप्रामाण्यनिरासः}
	\end{center}
	

	  \pstart वेदप्रमाण्यं निराचिकीर्षन् परमतमुत्थापयति ।
	\pend
      
	  \bigskip
	  \begingroup
	  \large
	
	    
	    \stanza[\smallbreak]
	\label{pv.3.224}\edlabel{pv.3.224}\flagstanza{\tiny\textenglish{....3.224}}गिराम्मिथ्यात्वहेतूनां दोषाणां पुरुषाश्रयात् ।&अपौरुषेयं सत्यार्थमिति केचित् प्रचक्षते ॥ २२४ ॥\&[\smallbreak]


	
	  \endgroup
	

	  \pstart {\color{DodgerBlue3}“गिरां”} वाचां {\color{DodgerBlue3}“मिथ्यात्वं”}\edlabel{pvv.367-1}\footnote{\label{pvv.367-1}  १ मृष (ा)र्थत्वस्य ये हेतवो दोषा रागादयस्तेषां वाचश्च पुरुष आश्रयस्तैः पुरुषस्य परिगृहीतत्वादप्रमाणत्वं (।) द्विधा शब्दार्थो निसर्गसिद्धो वेदादौ औपाधिकः पुरुषाधीनोन्यत्र । न मिथ्यात्वं वेदे पुरुषनिवृत्तेः । न संशयोऽप्रतिभासात् । नाज्ञानं वेदादर्थगतेः (।) पुरुषकारणाभावान्मिथ्यात्वकार्याभावसिद्धिः ।}स्य {\color{DodgerBlue3}“हेतूनां दोषाणाम”}ज्ञानविसम्वादाभिप्रायादीनां वा {\color{DodgerBlue3}“पुरुष”}स्या{\color{DodgerBlue3}“श्रया”}दाश्रयणत्वात् {\color{DodgerBlue3}“अपौरुषेयं”} वाक्यं मिथ्यात्वहेतोः पुरुषदोषस्याभावात् {\color{DodgerBlue3}“सत्यार्थमिति केचित्”} जै मि नी याः {\color{DodgerBlue3}“प्रचक्षते”} ॥ (२२४)
	\pend
      \label{div_pvv.3.225}\edlabel{div_pvv.3.225}
	  
	% new div opening: depth here is 2
	\leavevmode\marginnote{\textenglish{368/s}}
	  \bigskip
	  \begingroup
	  \large
	
	    
	    \stanza[\smallbreak]
	\label{pv.3.225}\edlabel{pv.3.225}\flagstanza{\tiny\textenglish{....3.225}}गिरां सत्यत्वहेतूनां गुणानां पुरुषाश्रयात् ।&अपौरुषेयं मिथ्यार्थं किं नेत्यन्ये प्रचक्षते ॥ २२५ ॥\&[\smallbreak]


	
	  \endgroup
	

	  \pstart तानेव\edlabel{pvv.368-1}\footnote{\label{pvv.368-1}  १ मी मां स कानेव शास्त्रकारः परमुखेनाह ।} प्रति गिरां {\color{DodgerBlue3}“सत्यत्वस्य हेतूनां\edlabel{pvv.368-2}\footnote{\label{pvv.368-2}  २ दयादीनां ।}”} दयाधर्मपरत्वादीनां {\color{DodgerBlue3}“गुणानां पुरुष”}स्या{\color{DodgerBlue3}“श्रयादपौरुषेयं”} वाक्यं सत्यता\edlabel{pvv.368-3}\footnote{\label{pvv.368-3}  ३ पुरुषनिवृत्त्या सत्त्वकारणनिवृत्तेः कार्यस्यापि सत्त्वस्य निवृत्तितः शब्दे सत्यत्वमिथ्यात्वयोः पुरुषायत्तत्वात् पुरुषनिवृत्तौ सत्त्यवन्मिथ्यात्वं च स्यात् ।}हेतोः पुरुषगुणस्याभावात् {\color{DodgerBlue3}“मिथ्यार्थं किं न”} भवतीत्यन्ये सौ ग ताः प्रचक्षते ॥ (२२५)
	\pend
      \label{div_pvv.3.226}\edlabel{div_pvv.3.226}
	  
	% new div opening: depth here is 2
	

	  \pstart किञ्च (।)
	\pend
      
	  \bigskip
	  \begingroup
	  \large
	
	    
	    \stanza[\smallbreak]
	\label{pv.3.226}\edlabel{pv.3.226}\flagstanza{\tiny\textenglish{....3.226}}अर्थज्ञापनहेतुर्हि सङ्केतः पुरुषाश्रयः ।&गिरामपौरुषेयत्वेप्यतो मिथ्यात्वसम्भवः ॥ २२६ ॥\&[\smallbreak]


	
	  \endgroup
	

	  \pstart सङ्केतमन्तरेणापौरुषे\edlabel{pvv.368-4}\footnote{\label{pvv.368-4}  ४ वस्तुतः पुंनिवृत्त्या सत्त्यमिथ्यार्थत्वनिवृत्तेरानर्थक्यादनुत्पत्तिलक्षणमानर्थंक्यं सङ्केतकाभावात् स्वभावतोर्थबोधानुपपत्तिः ।}यादपि वाक्यादर्थप्रतीतेरभावात् । {\color{DodgerBlue3}“अर्थज्ञापनहेतु”}रिह{\color{DodgerBlue3}“सङ्केतः”} स्वीकर्तव्यः । स च पुरुषकृतत्वात् {\color{DodgerBlue3}“पुरुषाश्रयः । अतः”} संकेतस्य\leavevmode\marginnote{\textenglish{73a/MA}} पुरुषाश्रयत्वात् {\color{DodgerBlue3}“गिरामपौरुषेयत्वेपि मिथ्यात्वस्य सम्भवः”} । संकेतवशेन वाचोऽर्थं ब्रुवते । स च दोषाश्रयेण पुरुषेण क्रियत इति तासां न विसंवादशङ्कानिरासः । पौरुषेयवाक्यवदिति व्यर्थमपौरुषेयत्वकल्पनं । (२२६)
	\pend
      \label{div_pvv.3.227}\edlabel{div_pvv.3.227}
	  
	% new div opening: depth here is 2
	

	  \begin{center}%% label @type='head'
	\textbf{क. अपौरुषेयत्वेप्यप्रामाण्यम्}
	\end{center}
	

	  \pstart अथ शब्दार्थ\edlabel{pvv.368-5}\footnote{\label{pvv.368-5}  ५ शब्दार्थानादितेव सम्बन्धोनादिः (।) स च त्रिप्रमाणकः श्रोतुर्ब्बाधायै केन शब्दे प्रयुक्ते पार्श्वस्थः प्रयोक्तारं वाच्यं वाचकञ्च बुध्यतेऽध्यक्षेण । श्रोतुश्च प्रतिपन्नत्वं प्रवृत्तिलिङ्गानुमानेन । अर्थप्रतिपत्त्यन्यथानुपपत्त्या च शब्दार्थाश्रितां वाच्यवाचकशक्तिमवगच्छत्यर्थापत्त्येति त्रीणि प्रमाणानि सम्बन्धस्य बोधे ।}योः सम्बन्धो न पौरुषेयः किन्तु स्वाभाविकः ततो न मिथ्यात्वसम्भवः । तदा (।)
	\pend
      
	  \bigskip
	  \begingroup
	  \large
	
	    
	    \stanza[\smallbreak]
	\label{pv.3.227a}\edlabel{pv.3.227a}\flagstanza{\tiny\textenglish{...3.227a}}सम्बन्धापौरुषेयत्वे स्यात् प्रतीतिरसंविदः ।\&[\smallbreak]


	
	  \endgroup
	

	  \pstart {\color{DodgerBlue3}“सम्बन्धापौरुषेयत्वे”}पीष्यमाणे1 {\color{DodgerBlue3}“स्याद”}र्थानां {\color{DodgerBlue3}“प्रतीतिरसंविदो”}ऽविद्यमानसङ्केतप्रतीतेः पुंसः । न चेच्छब्दार्थयोः सांकेतिको वाच्यवाचकतासम्बन्धः किन्तु स्वाभाविकः । तदाऽगृहीतश (?स) ङ्केतोपि श्रुताच्छब्दादर्थं प्रतिपद्येतेति ।
	\pend
      \leavevmode\marginnote{\textenglish{p369/s}}

	  \pstart अथ संकेतात् सतोपि तस्य सम्बन्धस्याभिव्यक्ति (ः) प्रदीपादिवद् घटादेरतो नागृहीतव्यञ्जकस्य व्यङ्ग्यप्रतीतिः । तदा (।)
	\pend
      
	  \bigskip
	  \begingroup
	  \large
	
	    
	    \stanza[\smallbreak]
	\label{pv.3.227b}\edlabel{pv.3.227b}\flagstanza{\tiny\textenglish{...3.227b}}संकेतात् तदभिव्यक्तावसमर्थान्यकल्पना ॥ २२७ ॥\&[\smallbreak]


	
	  \endgroup
	

	  \pstart {\color{DodgerBlue3}“संङ्केतात् त”}स्य सम्बन्धस्या{\color{DodgerBlue3}“भिव्यक्ता”}विष्यमा\edlabel{pvv.369-1}\footnote{\label{pvv.369-1}  १ प्रतीत्यन्यथानुपपत्त्या सम्बन्धकल्पनमत्यर्थेपि चेन्न प्रतीतिः किन्तत्कल्पनया संकेत एवान्वयव्यतिरेकात् ।}णायां संकेता{\color{DodgerBlue3}“दन्य”}स्य सम्बन्धस्य {\color{DodgerBlue3}“कल्पनाऽसमर्था”} सम्बन्धव्यवस्थापनाय । संकेतादेव वाच्यवाचकभावस्य कल्पितस्य घटमानत्वात् हस्तसंज्ञादेरिवार्थप्रतिपादनस्य । (२२७)
	\pend
      \label{div_pvv.3.228}\edlabel{div_pvv.3.228}
	  
	% new div opening: depth here is 2
	

	  \pstart किञ्च (।) वाचा किमेकेनार्थेन सह वाच्यवाचकसम्बन्धः । अथानेकैः । तत्र\edlabel{pvv.369-2}\footnote{\label{pvv.369-2}  २ नित्ये सम्बन्धदोषमाह ।} (।)
	\pend
      
	  \bigskip
	  \begingroup
	  \large
	
	    
	    \stanza[\smallbreak]
	\label{pv.3.228a}\edlabel{pv.3.228a}\flagstanza{\tiny\textenglish{...3.228a}}गिरामेकार्थनियमे न स्यादर्थान्तरे गतिः ।\&[\smallbreak]


	
	  \endgroup
	

	  \pstart {\color{DodgerBlue3}“गिरामेक”}स्मिन्नर्थे वाचकतया {\color{DodgerBlue3}“नियमे”} सति संकेतवशादन्य{\color{DodgerBlue3}“त्रार्थे न स्याद् गतिः”} । दृश्यते च विवक्षातोऽनेकार्थाभिधानम्\edlabel{pvv.369-3}\footnote{\label{pvv.369-3}  ३ अनेकार्थप्रतिपादनस्य दर्शनात् सर्व्वे सर्व्वार्थवाचकाश्चेत् ।} ॥
	\pend
      
	  \bigskip
	  \begingroup
	  \large
	
	    
	    \stanza[\smallbreak]
	\label{pv.3.228b}\edlabel{pv.3.228b}\flagstanza{\tiny\textenglish{...3.228b}}अनेकार्थाभिसम्बन्धे विरुद्धव्यक्तिसम्भवः ॥ २२८ ॥\&[\smallbreak]


	
	  \endgroup
	

	  \pstart {\color{DodgerBlue3}“अने”}कैर{\color{DodgerBlue3}“र्थै”}र्व्वाचकत्वा{\color{DodgerBlue3}“भिसम्बन्धे विरुद्ध”}स्यार्थस्य {\color{DodgerBlue3}“व्यक्तेः\edlabel{pvv.369-4}\footnote{\label{pvv.369-4}  ४ अभिमत एव समय इत्यनियमात् सर्व्ववाचकत्वे किं स्वर्गसाधन एवाग्निहोत्रादिसंकेतः किम्वा तद्विरुद्धे बुद्धिमान्द्यादिति संशयात् ।}”} प्रतीतेः {\color{DodgerBlue3}“सम्भवः”} स्यात् । अग्निष्टोमः स्वर्गस्य साधनमिति विपर्ययोप्यवसीयेत\edlabel{pvv.369-5}\footnote{\label{pvv.369-5}  ५ द्विधा शब्दविषयः साक्षाज्जातिस्तल्लक्षिता च व्यक्तिरिति व्यक्त्या सम्बन्धे सम्बन्धीत्यादि ।}(।) ततश्चाप्रवृत्तिरेव स्यात् स्वर्गीर्थिनः । (२२८)
	\pend
      \label{div_pvv.3.229}\edlabel{div_pvv.3.229}
	  
	% new div opening: depth here is 2
	

	  \pstart अथानेकार्थाभिधाय्यपि शब्दः पुरुषेण सङ्केतादभिमतार्थाभिधायित्वेन नियम्यते तदा (।)
	\pend
      
	  \bigskip
	  \begingroup
	  \large
	
	    
	    \stanza[\smallbreak]
	\label{pv.3.229}\edlabel{pv.3.229}\flagstanza{\tiny\textenglish{....3.229}}अपौरुषेयतायाञ्च व्यर्था स्यात् परिकल्पना ।&वाच्यश्च हेतुर्भिन्नानां सम्बन्धस्य व्यवस्थितेः ॥ २२९ ॥\&[\smallbreak]


	
	  \endgroup
	

	  \pstart {\color{DodgerBlue3}“अपौरुषेयतायाञ्च व्यर्था परिकल्पना स्यात्”} । तदभ्युपगमेपि पुरुषस्वातन्त्र्याभ्युपगमात् । तदार्थेभ्यो {\color{DodgerBlue3}“भिन्नानां”} शब्दानां तैः सह {\color{DodgerBlue3}“सम्बन्धस्य व्यव”}स्थितेः\edlabel{pvv.369-6}\footnote{\label{pvv.369-6}  व्यवस्थाय (ा)ः षष्ठी ।} । हेतुश्च \leavevmode\marginnote{\textenglish{370/s}} वाच्यो\edlabel{pvv.370-1}\footnote{\label{pvv.370-1}  १ शब्दार्थसम्बन्धवादिना ।}येनाव्यभिचारः । न हि शब्दार्थयोस्तादात्म्यं भेदात् । नापि तदुत्पत्तिरर्थमन्तरेणापि विवक्षातः शब्दोत्पत्तेः । अन्यथा चाव्यभिचाराभावात् । (२२९)
	\pend
      \label{div_pvv.3.230}\edlabel{div_pvv.3.230}
	  
	% new div opening: depth here is 2
	

	  \pstart उक्तमर्थं संगृह्णान्नाह ।
	\pend
      
	  \bigskip
	  \begingroup
	  \large
	
	    
	    \stanza[\smallbreak]
	\label{pv.3.230}\edlabel{pv.3.230}\flagstanza{\tiny\textenglish{....3.230}}असंस्कार्यतया पुंभिः सर्व्वथा स्यान्निरर्थता ।&संस्कारोपगमे मुख्यं गजस्नाननिभं भवेत् ॥ २३० ॥\&[\smallbreak]


	
	  \endgroup
	

	  \pstart यदि सङ्केतनिरपेक्षाणां स्वत एव वाचकत्वं शब्दानां तदा\edlabel{pvv.370-2}\footnote{\label{pvv.370-2}  २ सत्यभिथ्यार्थत्वयोः पुरुषसंस्कारप्रतिबद्धत्वात् ।} {\color{DodgerBlue3}“पुंभिरसंस्कार्यतया”} सङ्केतद्वारेण नियम्यतया {\color{DodgerBlue3}“सर्व्वथा निरर्थता स्यात्”} । पुरुषसङ्केतनिरपेक्षाच्छब्दादर्थंप्रतीतेरभावात् । एतद्दोषभयात् {\color{DodgerBlue3}“संस्कारस्योपगमे”} स्वीकारे इदमपौरुषेयत्वं मुख्यमनुपचरितं {\color{DodgerBlue3}“गजस्नान\edlabel{pvv.370-3}\footnote{\label{pvv.370-3}  ३ मिथ्यार्थतापि स्यात् । जातिचोदनेपि न प्रयोजनं निर्ल्लोठितमेतदन्यापोह “अपि प्रवर्त्तेत पुमान् विज्ञायार्थक्रियाक्षमान् ।” \cref{pv.3.92} ।}म्भवेत्”} । गजो हि स्नाने पङ्कमपनीय पुनस्तेनात्मानं लिम्पति । तथाऽपौरुषेयत्वं सम्बन्धस्य स्वीकृत्यापि पुनः सङ्केते पुरुषापेक्षेति व्यक्तं साम्यं । (२३०)
	\pend
      \label{div_pvv.3.231_3.232_3.233_3.234_3.235_3.236ab}\edlabel{div_pvv.3.231_3.232_3.233_3.234_3.235_3.236ab}
	  
	% new div opening: depth here is 2
	

	  \begin{center}%% label @type='head'
	\textbf{ख. सम्बन्धचिन्ता}
	\end{center}
	

	  \begin{center}%% label @type='head'
	\textbf{(क) सम्बन्ध्यनित्यत्वे सम्बन्धनित्यता}
	\end{center}
	
	  \bigskip
	  \begingroup
	  \large
	
	    
	    \stanza[\smallbreak]
	\label{pv.3.231a}\edlabel{pv.3.231a}\flagstanza{\tiny\textenglish{...3.231a}}सम्बन्धिनामनित्यत्वान्न संबन्धेस्ति नित्यता ।\&[\smallbreak]


	
	  \endgroup
	

	  \pstart तथा {\color{DodgerBlue3}“सम्बन्धिनाम”}र्थाना{\color{DodgerBlue3}“मनित्यत्वात् संबन्धे नित्यता नास्ति”} न ह्याश्रयापाये भवत्याश्रितं ।
	\pend
      

	  \pstart किञ्च (।)
	\pend
      
	  \bigskip
	  \begingroup
	  \large
	
	    
	    \stanza[\smallbreak]
	\label{pv.3.231b}\edlabel{pv.3.231b}\flagstanza{\tiny\textenglish{...3.231b}}नित्यस्यानुपकार्यत्वादकुर्वाणश्च नाश्रयः ॥ २३१ ॥\&[\smallbreak]


	
	  \endgroup
	
	  \bigskip
	  \begingroup
	  \large
	
	    
	    \stanza[\smallbreak]
	\label{pv.3.232a}\edlabel{pv.3.232a}\flagstanza{\tiny\textenglish{...3.232a}}अर्थैरतः स शब्दानां संस्कार्यः पुरुषैर्धिया ।\&[\smallbreak]


	
	  \endgroup
	

	  \pstart {\color{DodgerBlue3}“नित्यस्य”} सम्बन्धस्या{\color{DodgerBlue3}“नुपकार्यत्वात् । अकुर्व्वाणो”}ऽनुपकुर्व्वाणः शब्दोऽर्थश्चा{\color{DodgerBlue3}“श्रयो न”} युक्तः । यतः स्वाभाविकसम्बन्धानुपपत्ति{\color{DodgerBlue3}“रतः”} सम्बन्धो{\color{DodgerBlue3}“ऽर्थैः”} सह {\color{DodgerBlue3}“शब्दानां”}\leavevmode\marginnote{\textenglish{73b/MA}} पुरुषैर्व्यवहर्तृभिरर्थप्रतिपादनाभिप्रायान्वयव्यतिरेकानुविधानमाश्रित्य {\color{DodgerBlue3}“धिया”} कल्पिकया\edlabel{pvv.370-4}\footnote{\label{pvv.370-4}  ४ शब्दार्थावसंबन्धिनावपि सम्बद्धौ पुरुषस्य प्रतिभासेते विकल्पबुद्धौ अनादिव्यवहाराभ्यासादर्थकार्यः शब्दास्तद्भानुविधायित्वादित्यध्यवसायवशात् सम्बन्धव्यवस्था ।} {\color{DodgerBlue3}“संस्कार्यो”} व्यवस्थाप्यः ।
	\pend
      \leavevmode\marginnote{\textenglish{371/s}}

	  \pstart अथानित्य एव सम्बन्धस्तदा सम्बन्धिनां नाशे सम्बन्धस्य नष्टत्वान्निरर्थकः शब्दः स्यात् ।
	\pend
      

	  \pstart अथवा (।)
	\pend
      
	  \bigskip
	  \begingroup
	  \large
	
	    
	    \stanza[\smallbreak]
	\label{pv.3.232b}\edlabel{pv.3.232b}\flagstanza{\tiny\textenglish{...3.232b}}अर्थैरेव सहोत्पादे न स्वभावविपर्ययः ॥ २३२ ॥\&[\smallbreak]


	
	  \endgroup
	
	  \bigskip
	  \begingroup
	  \large
	
	    
	    \stanza[\smallbreak]
	\label{pv.3.233a}\edlabel{pv.3.233a}\flagstanza{\tiny\textenglish{...3.233a}}शब्देषु युक्तः ;\&[\smallbreak]


	
	  \endgroup
	

	  \pstart वाच्यै{\color{DodgerBlue3}“रर्थै”}रेव {\color{DodgerBlue3}“सह”} सम्बन्धस्यो{\color{DodgerBlue3}“त्पाद”} इष्यते (।) तदोत्पाद इष्यमाणेपि पूर्व्वमर्थेन सह सम्बन्धस्य विनष्टत्वात् । अर्थसम्बन्धरहितात्मसु {\color{DodgerBlue3}“शब्दे”}षु स्वभावस्य सम्बन्धविकलस्य {\color{DodgerBlue3}“विपर्ययः”} सम्बन्धयोगी {\color{DodgerBlue3}“न युक्तः”}\edlabel{pvv.371-1}\footnote{\label{pvv.371-1}  १ अर्थेन सहोत्पन्नस्यानुपकारिणि शब्देनाश्रयणाच्च ।} । न हि नित्यस्य पूर्व्वापरैकस्वभावस्यान्यथात्वं युक्तं । अन्यथा नित्यताहानिप्रसङ्गात् ।
	\pend
      

	  \begin{center}%% label @type='head'
	\textbf{(ख) सम्बन्धः कल्पितः}
	\end{center}
	

	  \pstart अस्मन्मते तु (।)
	\pend
      
	  \bigskip
	  \begingroup
	  \large
	
	    
	    \stanza[\smallbreak]
	\label{pv.3.233b}\edlabel{pv.3.233b}\flagstanza{\tiny\textenglish{...3.233b}}सम्बन्धे नायं दोषो विकल्पिते ।\&[\smallbreak]


	
	  \endgroup
	

	  \pstart विकल्पिते कल्पनानिर्मिते सम्बन्धेऽयं स्वभावान्यत्वप्रसङ्गदोषो न भवति । न हि कल्पनाक्लृप्तो धर्मः स्वभावं वस्तुतः स्पृशति ।
	\pend
      
	  \bigskip
	  \begingroup
	  \large
	
	    
	    \stanza[\smallbreak]
	\label{pv.3.233c}\edlabel{pv.3.233c}\flagstanza{\tiny\textenglish{...3.233c}}नित्यत्वादाश्रयापायेप्यनाशो यदि सम्मतः ॥ २३३ ॥\&[\smallbreak]


	
	  \endgroup
	
	  \bigskip
	  \begingroup
	  \large
	
	    
	    \stanza[\smallbreak]
	\label{pv.3.234a}\edlabel{pv.3.234a}\flagstanza{\tiny\textenglish{...3.234a}}नित्येष्वाश्रयसामर्थ्यं किं येनेष्टः स चाश्रयः ।\&[\smallbreak]


	
	  \endgroup
	

	  \pstart सम्बन्धस्य\edlabel{pvv.371-2}\footnote{\label{pvv.371-2}  २ सम्बन्धिनामनित्यत्वादित्यादौ परः ।} {\color{DodgerBlue3}“नित्यत्वात् आश्रय”}स्य वाच्यस्या{\color{DodgerBlue3}“पाये”}प्य{\color{DodgerBlue3}“नाशो”} यदि जाते\edlabel{pvv.371-3}\footnote{\label{pvv.371-3}  ३ नित्यत्वादाश्रयनाशेप्यनाशवत् ।}रिव {\color{DodgerBlue3}“सम्मतः”} तदा {\color{DodgerBlue3}“नित्येषु”} जाति\edlabel{pvv.371-4}\footnote{\label{pvv.371-4}  ४प्रसिद्धिमात्रं तन्निर्व्वस्तुकं ।}सम्बन्धादिष्वा{\color{DodgerBlue3}“श्रय”}स्य वाच्यस्या वाचकस्य {\color{DodgerBlue3}“च किं सामर्थ्य-”} मुपकारविशेषाधायकं {\color{DodgerBlue3}“येन”} सामर्थ्येन स वाच्यदिरा{\color{DodgerBlue3}“श्रय इष्टः”} । न ह्यनुपकार्यमाश्रितमतिप्रसङ्गात् । नित्यस्य चोपकारासम्भवः । भेदाभेदकल्पनायामयुक्तत्वात्
	\pend
      

	  \pstart अथ नित्यस्यापि जातिसम्बन्धादेराश्रयेणाभिव्यक्तिलक्षण उपकारः क्रियते । न चाव्यक्तिहेतुः कारको दीपादिवत् घटादेरित्याह ।
	\pend
      
	  \bigskip
	  \begingroup
	  \large
	
	    
	    \stanza[\smallbreak]
	\label{pv.3.234b}\edlabel{pv.3.234b}\flagstanza{\tiny\textenglish{...3.234b}}ज्ञानोत्पादेन हेतूनां सम्बन्धात् सहकारिणाम् ॥ २३४ ॥\&[\smallbreak]


	
	  \endgroup
	
	  \bigskip
	  \begingroup
	  \large
	
	    
	    \stanza[\smallbreak]
	\label{pv.3.235a}\edlabel{pv.3.235a}\flagstanza{\tiny\textenglish{...3.235a}}तदुत्पादनयोग्यत्वेनोत्पत्तिर्व्यक्तिरिष्यते ।&घटादिष्वपि युक्तिज्ञैः ;\&[\smallbreak]


	
	  \endgroup
	\leavevmode\marginnote{\textenglish{372/s}}

	  \pstart {\color{DodgerBlue3}“ज्ञानोत्पादेन हेतूनां”} दीपादीनां {\color{DodgerBlue3}“सहकारिणां\edlabel{pvv.372-1}\footnote{\label{pvv.372-1}  १ योग्यदेशावस्थानात् ।}”} सम्बन्धात् {\color{DodgerBlue3}“तदुत्पादनयोग्यत्वेन”} ज्ञानोत्पादनसमर्थत्वेनोत्पत्ति{\color{DodgerBlue3}“र्घटादिष्वपि”} भावेषु {\color{DodgerBlue3}“युक्तिज्ञै”}र्न्यायविद्भि{\color{DodgerBlue3}“र्व्यक्तिरिष्यते-”} ऽन्यथा ज्ञानोत्पादनयोग्यस्य स्वभावस्यानुत्पत्तौ ज्ञानोत्पादनं न स्यात् (।)
	\pend
      
	  \bigskip
	  \begingroup
	  \large
	
	    
	    \stanza[\smallbreak]
	\label{pv.3.235b}\edlabel{pv.3.235b}\flagstanza{\tiny\textenglish{...3.235b}}अविशेषेऽविकारिणाम् ॥ २३५ ॥\&[\smallbreak]


	
	  \endgroup
	
	  \bigskip
	  \begingroup
	  \large
	
	    
	    \stanza[\smallbreak]
	\label{pv.3.236a}\edlabel{pv.3.236a}\flagstanza{\tiny\textenglish{...3.236a}}व्यञ्जकैः स्वैः कुतः कोर्थो व्यक्तास्तैस्ते यतो मताः ।\&[\smallbreak]


	
	  \endgroup
	

	  \pstart नित्यानां जातिसम्बन्धादीनाम{\color{DodgerBlue3}“विकारिणां”} कुतश्चिदविशेषे विशेषासम्भवे स्वैर्व्यञ्जकैराश्रयाभिमतैः {\color{DodgerBlue3}“कोर्थः”} स्वभावान्यथात्वादिः {\color{DodgerBlue3}“कुतो”} न कश्चित् । {\color{DodgerBlue3}“यतो”}ऽर्थात् कृतात् तै{\color{DodgerBlue3}“र्व्यञ्ज”}कैस्ते  जात्यादयो {\color{DodgerBlue3}“व्यक्ता मताः”} ।
	\pend
      \label{div_pvv.3.236bc_3.237_3.238_3.239_3.240_3.241_3.242_3.243_3.244_3.245_3.246}\edlabel{div_pvv.3.236bc_3.237_3.238_3.239_3.240_3.241_3.242_3.243_3.244_3.245_3.246}
	  
	% new div opening: depth here is 2
	

	  \begin{center}%% label @type='head'
	\textbf{(ग) भेदाभेदव्यवस्थातोऽपि सम्बन्धस्यावस्तुत्वम्}
	\end{center}
	

	  \pstart किञ्च \edlabel{pvv.372-2}\footnote{\label{pvv.372-2}  २ न च सम्बन्धस्त्रिप्रमाणक इति दर्शयन्नाह वर्ण्णा न वाचकास्तेन वाच्यवाचकसम्बन्धस्यावृत्तिर्वर्ण्णेषु निरर्थकत्वात् तद्वृतौ सम्बन्धस्य वाचकाङ्गत्वं स्यात् ।} (।)
	\pend
      
	  \bigskip
	  \begingroup
	  \large
	
	    
	    \stanza[\smallbreak]
	\label{pv.3.236b}\edlabel{pv.3.236b}\flagstanza{\tiny\textenglish{...3.236b}}सम्बन्धस्य च वस्तुत्वे स्याद् भेदाद् बुद्धिचित्रता ॥ २३६ ॥\&[\smallbreak]


	
	  \endgroup
	
	  \bigskip
	  \begingroup
	  \large
	
	    
	    \stanza[\smallbreak]
	\label{pv.3.237a}\edlabel{pv.3.237a}\flagstanza{\tiny\textenglish{...3.237a}}ताभ्यामभेदे तावेव नातोन्या वस्तुनो गतिः ।\&[\smallbreak]


	
	  \endgroup
	

	  \pstart यदि {\color{DodgerBlue3}“सम्बन्धस्य वस्तुत्व”}न्तदा वस्तुत्वे सति भेदोऽभेदो वाऽभ्युपगन्तव्यः । तत्र {\color{DodgerBlue3}“भेदाद्\edlabel{pvv.372-3}\footnote{\label{pvv.372-3}  ३ सम्बन्धिभ्यां सम्बन्धस्य भेदात् ।} बुद्धेश्चित्रता”} स्यात् (।) वाच्यवाचकौ सम्बन्धश्चेति त्रितयं दृश्येत\edlabel{pvv.372-4}\footnote{\label{pvv.372-4}  ४ यत्सम्बन्धाभ्यां(?)भेदेन नोपलभ्यते तत्ततो नान्यत् यद् दृश्यं नोपलभ्यते तन्नास्ति ।} । न चेक्ष्यते । अथ द्वितीयः पक्षः तदा {\color{DodgerBlue3}“ताभ्यामभेदे”} सम्बन्धस्य {\color{DodgerBlue3}“तौ”} वाच्यवाचका{\color{DodgerBlue3}“वेव”} स्यातां न तु सम्बन्धो नाम कश्चित् । अथ सम्बन्धो न भिन्नो नाप्यभिन्नः\edlabel{pvv.372-5}\footnote{\label{pvv.372-5}  ५ तत्त्वान्यत्त्वरहितः सम्बन्ध इत्यत्राह ।} । {\color{DodgerBlue3}“अतो”} भेदाभेदाभ्या{\color{DodgerBlue3}“मन्या”} वस्तुनो {\color{DodgerBlue3}“गति”}र्नास्ति । अन्योन्यवच्छेदात्मकत्वादनयो राश्यन्तरासम्भवान्नान्यः प्रकारोस्ति वस्तुनः ।
	\pend
      

	  \pstart तस्माद् (।)
	\pend
      
	  \bigskip
	  \begingroup
	  \large
	
	    
	    \stanza[\smallbreak]
	\label{pv.3.237b}\edlabel{pv.3.237b}\flagstanza{\tiny\textenglish{...3.237b}}भिन्नत्वाद् वस्तुरूपस्य सम्बन्धः कल्पनाकृतः ॥ २३७ ॥\&[\smallbreak]


	
	  \endgroup
	
	  \bigskip
	  \begingroup
	  \large
	
	    
	    \stanza[\smallbreak]
	\label{pv.3.238a}\edlabel{pv.3.238a}\flagstanza{\tiny\textenglish{...3.238a}}सद्द्रव्यं स्यात् पराधीनं सम्बन्धोन्यस्य वा कथम् ।\&[\smallbreak]


	
	  \endgroup
	\leavevmode\marginnote{\textenglish{373/s}}

	  \pstart भिन्नत्वाद् {\color{DodgerBlue3}“वस्तुनोः”} सम्बन्धिनोः {\color{DodgerBlue3}“रूपस्य सम्बन्धः”} श्लेषलक्षणः {\color{DodgerBlue3}“कल्पनया कृतो”} न वास्तवः । अन्यथा {\color{DodgerBlue3}“सद्द्रव्यं”} सम्बन्धाख्यं {\color{DodgerBlue3}“पराधीनं”} सम्बन्धायत्तं {\color{DodgerBlue3}“कथं स्यात् ।\edlabel{pvv.373-1}\footnote{\label{pvv.373-1}  १ एतेनार्थान्तरत्वे सम्बन्धस्याश्रितत्वं श्लेषञ्च गतः ।} कथम्वा”} भिन्नयोः सम्बन्धिनोः श्लेषलक्षणः {\color{DodgerBlue3}“सम्बन्धः”} परस्परममिश्रस्वभावत्वात् सर्व्वस्य तथापि सम्बन्धेतिप्रसङ्गात् ।
	\pend
      

	  \begin{center}%% label @type='head'
	\textbf{(६) वर्णपदादिषु सम्बन्धस्यासद्भावः}
	\end{center}
	

	  \pstart किञ्च (।)
	\pend
      
	  \bigskip
	  \begingroup
	  \large
	
	    
	    \stanza[\smallbreak]
	\label{pv.3.238b}\edlabel{pv.3.238b}\flagstanza{\tiny\textenglish{...3.238b}}वर्ण्णा निरर्थकाः सन्तः;\&[\smallbreak]


	
	  \endgroup
	

	  \pstart अयं सम्बन्धो वर्त्तमानो वर्ण्णेषु पदादिषु वा वर्त्तत । तत्र {\color{DodgerBlue3}“वर्ण्णाः सन्तो\edlabel{pvv.373-2}\footnote{\label{pvv.373-2}  २ वस्तुसन्तो विद्यमाना अपि ।} निरर्थकाः”} प्रत्येकं तेषामर्थप्रतिपादकत्वाभावात् । नानाप्रयोक्तृप्रयुक्तेभ्योऽर्थाप्रतिपत्तेश्च व्यतिक्रमप्रयुक्तेभ्यश्च स\edlabel{pvv.373-3}\footnote{\label{pvv.373-3}  ३ साहित्याभावेन । नानुमानाल्लिङ्गाभावान्न हि केचिद् दृष्टान्ते सम्बन्धं कार्याऽर्थप्रतीतिः सम्बन्धस्यातीन्द्रियत्वेनेन्द्रियादिवत् साधनापेक्षणात् ।}रो रस इत्यादिभ्य\edlabel{pvv.373-4}\footnote{\label{pvv.373-4}  ४ क्रमविशेषेणैकप्रयोक्तृप्रयुक्ता वर्ण्णा एव वाचका इति न दोषश्चेत् । न क्रमस्य नार्थान्तत्वेन यद् रूपं सरे तद्रूपं रसेपीति तुल्या प्रतीतिः स्यात् ।}स्तुल्या स्यात् प्रतिपत्तिस्तत्समुदायस्य चासम्भवः क्रमेणोपलम्भात् । न च समुदायो नाम समुदायिभ्यो भिन्नोऽनुपलम्भबाधितत्वात् । तेषाञ्च वाचकत्वादेकस्मादपि\leavevmode\marginnote{\textenglish{74a/MA}} प्रतीतिः स्यात् । प्रत्येकं न चेद् वाचकाः समुदितेभ्योपि तेभ्यो न स्यात् प्रतीतिस्तदाप्यन्यस्याभावात् ।
	\pend
      

	  \pstart अथ क्रमेण वर्ण्णेषु गृहीतेषु तत्संस्कारसहायेनाध्यक्षेण गृहीतादन्त्यवर्ण्णादर्थप्रतीतिः\edlabel{pvv.373-5}\footnote{\label{pvv.373-5}  ५ न वर्ण्णानुभवाहितसंस्कारस्य वर्ण्णेष्वेव स्मृतिहेतुत्वान्नार्थे न हि गवानुभवाहितसंस्कारोऽश्वस्मरणमादधति । दृष्टत्वादित्यपि न संकेतं विनाऽसत्यात् । संकेतश्च सामान्यविषयो न वर्ण्णस्वलक्षणे ।} । तत्किमन्त्य एव वर्ण्णो वाचको नान्ये । तथा तेद् व्यर्थं तेषामुच्चारणं । सव्वेषु प्रतीतेष्वर्थप्रतीतिरिति चेत् । किमन्त्यवर्ण्णग्राहिकया बुद्ध्या1 सर्व्वत्र ग्रहणं । अन्यान्यबुद्ध्यैव चेत् । ताः किं बुद्धयोऽन्त्यवर्ण्णबुद्धिकाले भवन्ति । येन तदार्थप्रतीतिरुच्यते । अन्यान्यकाल एवेति चेत् । यदि ताभिर्व्वाचका वर्ण्णा गृह्यन्ते (।) एकैकवर्ण्णग्रहणेप्यर्थप्रतीतिः स्यात् । वाचकेषु सर्व्वषु गृहीतेषु प्रतीतिरिति चेत् । तदा तु न प्रत्येकं वाचकस्तदतिरिक्तश्च समुदायो नास्ति ॥
	\pend
      \leavevmode\marginnote{\textenglish{374/s}}

	  \pstart प्रत्येकं समर्थाः स्थितिबीजादयोऽङ्कुरजनने न च केवलाजनयन्तीति चेत् । ये समर्था न तेषां क्षणिकत्वात् पृथग्भाव इत्यसमानं । समुदिता एव तु समर्थाः । न त्वेवं वर्ण्णानां क्वापि समुदायः क्रमोपलभ्यत्वात् । तदा चेद् प्रतिपादका अवाचका एव । पूर्व्ववर्ण्णग्रहणसंस्कारेपि किमन्त्यवर्ण्णबुद्धौ सर्व्वे प्रतिभान्ति न वा । न तावदुपलभ्यन्ते ततस्तदुपदर्शमपि व्यर्थं सर्व्वेषु क्रमात् प्रतीतेषु स्मृतिः समुदायविषया भवतीति चेत् । किम्वर्ण्णानां समुदायोस्ति प्रतीतो वा यः स्मर्यते केवलं कल्प्यते । कल्पितस्य वाचकत्वाभ्युपगमे न विवादः । तस्मान्न वर्ण्णे सम्बन्धवृत्तिः ।
	\pend
      

	  \pstart पदवाक्यादिषु तर्हि स्यादिति चेत् ।
	\pend
      
	  \bigskip
	  \begingroup
	  \large
	
	    
	    \stanza[\smallbreak]
	\label{pv.3.238c}\edlabel{pv.3.238c}\flagstanza{\tiny\textenglish{...3.238c}}पदादिपरिकल्पितम् ॥ २३८ ॥\&[\smallbreak]


	
	  \endgroup
	
	  \bigskip
	  \begingroup
	  \large
	
	    
	    \stanza[\smallbreak]
	\label{pv.3.239a}\edlabel{pv.3.239a}\flagstanza{\tiny\textenglish{...3.239a}}अवस्तुनि कथं वृत्तिः सम्बन्धस्यास्य वस्तुनः ।\&[\smallbreak]


	
	  \endgroup
	

	  \pstart \edlabel{pvv.374-1}\footnote{\label{pvv.374-1}  १ वैयाकरणानां वर्ण्णादिव्यतिरिक्तं पदादि निरस्यते ।} न हि क्रमोच्चारितेभ्यो वर्ण्णेभ्यो व्यतिरिक्तं पदादिकमुपलभ्यते (।) केवलं कल्पनाबुद्ध्या\edlabel{pvv.374-2}\footnote{\label{pvv.374-2}  २ भिन्नवर्ण्णानुभवात् कथमेकपदाद्यवभासौ विकल्पः । अस्ति चेत्यनुभवोस्तीत्याह क्रमवर्ण्णानुभवदृष्टभाविमनोविज्ञानं वर्ण्णान्यदादित्वेनैकस्वभावानध्यवस्यति मिथ्याविभ्रमोऽनादिः ।} क्रमोच्चारितानां वर्ण्णानां समुदायः कल्पितः पदं । पदानाञ्च समुदायः कल्पितो वाक्यमुच्यते । तच्च कल्पितत्वादवस्तु । {\color{DodgerBlue3}“अवस्तुनि सम्बन्धस्य वस्तुनः कथम्वृत्तिः”} । न हि शशवि{\color{DodgerBlue3}“शा”} (?षा)णस्य नीलादिर्द्धर्मो युक्तः । तदेवं न सम्बन्धो नित्योऽनित्यो वा युक्त इति स्थितं ॥
	\pend
      

	  \begin{center}%% label @type='head'
	\textbf{ग. नापौरुषेयता}
	\end{center}
	
	  \bigskip
	  \begingroup
	  \large
	
	    
	    \stanza[\smallbreak]
	\label{pv.3.239b}\edlabel{pv.3.239b}\flagstanza{\tiny\textenglish{...3.239b}}अपौरुषेयतापीष्टा कर्त्तृणामस्मृतेः किल ॥ २३९ ॥\&[\smallbreak]


	
	  \endgroup
	
	  \bigskip
	  \begingroup
	  \large
	
	    
	    \stanza[\smallbreak]
	\label{pv.3.240a}\edlabel{pv.3.240a}\flagstanza{\tiny\textenglish{...3.240a}}सन्त्यस्याप्यनुवक्तार इति धिग्व्यापकं तमः ।\&[\smallbreak]


	
	  \endgroup
	

	  \pstart वेदवाक्यानाम{\color{DodgerBlue3}“पौरुषेयतापि”} केनचि न्मी मां स क\edlabel{pvv.374-3}\footnote{\label{pvv.374-3}  ३ बहूनां जीर्ण्णकूपादीनां कर्त्ता न स्मर्यते न च तावताऽकर्त्तृता । इति व्यभिचारादयुक्तं लिङ्गं जैमि(नि)नोक्तं ।} प्रवरेणेष्टा {\color{DodgerBlue3}“कर्त्तृणामस्मृतेः”} । लिङ्गात् {\color{DodgerBlue3}“किल”} । अक्षमायां किल शब्दः । अस्याप्यर्थस्य न्यायाद् दूरमायातस्या\leavevmode\marginnote{\textenglish{375/s}} नुवक्तारः पण्डितंमन्याः\edlabel{pvv.375-1}\footnote{\label{pvv.375-1}  १ कुमारिलादयः ।} सन्ति । तस्यैव\edlabel{pvv.375-2}\footnote{\label{pvv.375-2}  २ यः कर्त्तुरस्मरणादपौरुषेयतामाह जै मि नि (ः) एवमेतदिति निःकृपमाक्रान्तं जगत् येन तमसा । अज्ञानस्यैव धिग्वादो न प्राणिनः ।} तावदीदृशं\edlabel{pvv.375-3}\footnote{\label{pvv.375-3}  ३ यत इदं साधनमसिद्धमनेकञ्चेत्याह ।} प्रज्ञास्खलितं कथम्वृत्तमिति सविस्मयानुकम्पन्नश्चेतः । तदप्यपरेऽनुवदन्ती\edlabel{pvv.375-4}\footnote{\label{pvv.375-4}  ४ अतिस्थूलं सह विस्मयेनानुकम्पया च वर्त्तते श्रुतवतोपीदृशमविद्याविलसितमिति सविस्मयं । गाढेनाविद्याबन्धेन सत्वाः पीड्यन्त इति सानुकम्पं अपरे तन्मतानुगाः ।}ति निर्दयाक्रान्तभुवनं धिगव्यापकन्तमः । तथा हि कर्त्तुः स्मरणमसिद्धं स्मरन्ति सौगता मन्त्राणां कर्तॄन् अष्टकादीन् । का णा\edlabel{pvv.375-5}\footnote{\label{pvv.375-5}  ५ वैशेषिकाः ।}दा श्च वि धा ता रं । मिथ्या तत्स्मरणञ्चेत् । कु मा रस म्भ वा देरपि का लि दा सादिकर्त्तृस्मरणं मिथ्येति तदप्यपौरुषेयं । तत्रैकं स्मरणमप्रमाणमन्यच्चान्यथेति नात्र विभागकारणं । बहूनां संप्रतिपत्तिविप्रतिपत्तयश्च न प्रमाणेतरलक्षणे सम्वादसत्वेन सिध्यतः । ततश्चास्मृतकर्त्तृकमित्यशक्यनिश्चयं सन्दिग्धविपक्षव्यावृत्तिकत्वात्\edlabel{pvv.375-6}\footnote{\label{pvv.375-6}  ६ साधनान्तरं निरस्यति । “
	    \begin{verse}
	वेदस्याध्ययनं सर्व्वं गुर्व्वध्ययनपूर्व्वकम् ।\\
	    विद्या (?वेदा) ध्ययनवाच्यत्वादधुनाध्ययनं यथे\\
	    
	    \end{verse}
	  ”ति \href{http://http://sarit.indology.info/?cref=śv.949}{(श्लोकवार्तिके ९४९)} दूषयन्नाह ।} ।
	\pend
      

	  \begin{center}%% label @type='head'
	\textbf{घ. न नित्यता}
	\end{center}
	

	  \begin{center}%% label @type='head'
	\textbf{(क) a गुर्वध्ययनपूर्वकत्वादपि न}
	\end{center}
	
	  \bigskip
	  \begingroup
	  \large
	
	    
	    \stanza[\smallbreak]
	\label{pv.3.240b}\edlabel{pv.3.240b}\flagstanza{\tiny\textenglish{...3.240b}}यथायमन्यतोऽश्रुत्वा नेमं वर्णपदक्रमम् ॥ २४० ॥\&[\smallbreak]


	
	  \endgroup
	
	  \bigskip
	  \begingroup
	  \large
	
	    
	    \stanza[\smallbreak]
	\label{pv.3.241a}\edlabel{pv.3.241a}\flagstanza{\tiny\textenglish{...3.241a}}वक्तुं समर्थः पुरुषस्तथान्योपीति कश्चन ।\&[\smallbreak]


	
	  \endgroup
	

	  \pstart यदापि वेदानधीयानो यथायमिदानीन्तनो माणवकोऽ{\color{DodgerBlue3}“न्यत”} उपाध्याया{\color{DodgerBlue3}“दश्रुत्वा इम”}मुच्यमानं {\color{DodgerBlue3}“वर्ण्णपदयोः क्रम”}मानुपूर्व्वीविशिष्टां वेदाख्यां {\color{DodgerBlue3}“वक्तुमसमर्थो”} वेदकपाठकत्वात् । {\color{DodgerBlue3}“तथाऽन्य”} उपाध्यायस्तदुपाध्यायोपीत्यनादिरेष क्रम इति\leavevmode\marginnote{\textenglish{74b/MA}} कश्चि {\color{DodgerBlue3}“न्मी मां स कः”} । सर्व्वस्यैव वेदपाठः परोपदेशादिति नित्यतैव वेदानां ।
	\pend
      

	  \pstart तत्राप्याह ।
	\pend
      
	  \bigskip
	  \begingroup
	  \large
	
	    
	    \stanza[\smallbreak]
	\label{pv.3.241b}\edlabel{pv.3.241b}\flagstanza{\tiny\textenglish{...3.241b}}\leavevmode\marginnote{\textenglish{376/s}}अन्यो वा रचितो ग्रन्थः सम्प्रदायाद् ऋते परैः ॥ २४१ ॥\&[\smallbreak]


	
	  \endgroup
	
	  \bigskip
	  \begingroup
	  \large
	
	    
	    \stanza[\smallbreak]
	\label{pv.3.242a}\edlabel{pv.3.242a}\flagstanza{\tiny\textenglish{...3.242a}}दृष्टः कोऽभिहितो येन सोप्येवं नानुमीयते ।\&[\smallbreak]


	
	  \endgroup
	

	  \pstart {\color{DodgerBlue3}“अन्यो वा”} वेदादितरः काव्यादि{\color{DodgerBlue3}“र्ग्रन्थो रचितः”} कविप्रभृतिभिः {\color{DodgerBlue3}“संप्रदायाद् ऋते”} उपदेशाद्विना परैरध्येतृभिर{\color{DodgerBlue3}“भिहितः को दृष्टो”} न कश्चित् (।) {\color{DodgerBlue3}“येन”} परोपदेशेऽसत्यशक्याध्ययनत्वेन {\color{DodgerBlue3}“सोपि”} काव्यादि{\color{DodgerBlue3}“रेवं”} नित्यं {\color{DodgerBlue3}“नानुमी”}यते\edlabel{pvv.376-1}\footnote{\label{pvv.376-1}  १ विनोपदेशं पाठाशक्तेः काव्येपि सत्त्वात् ।} । तत्रापि हेतुरयं सिद्ध एव । अथ पुरुषेण तत्करणाविरोधात् सन्दिग्धव्यतिरेकताऽस्य हेतोस्तदा वेदेपि दुष्टत्वमस्या (ः) कथं निवार्यं ।
	\pend
      

	  \pstart अपि च (।)
	\pend
      
	  \bigskip
	  \begingroup
	  \large
	
	    
	    \stanza[\smallbreak]
	\label{pv.3.242b}\edlabel{pv.3.242b}\flagstanza{\tiny\textenglish{...3.242b}}यज्जातीयो यतः सिद्धः सोऽविशिष्टोग्निकाष्ठवत् ॥ २४२ ॥\&[\smallbreak]


	
	  \endgroup
	
	  \bigskip
	  \begingroup
	  \large
	
	    
	    \stanza[\smallbreak]
	\label{pv.3.243a}\edlabel{pv.3.243a}\flagstanza{\tiny\textenglish{...3.243a}}अदृष्टहेतुरप्यन्यस्तद्भवः संप्रतीयते ।\&[\smallbreak]


	
	  \endgroup
	

	  \pstart {\color{DodgerBlue3}“यज्जातीयो”} यद्द्रव्यसमानजातीयो {\color{DodgerBlue3}“यः”} पदार्थो {\color{DodgerBlue3}“यतो”} हेतोः {\color{DodgerBlue3}“सिद्धो”}ऽन्वयव्यतिरेकाभ्यां निश्चितः {\color{DodgerBlue3}“स”} तज्जातीयत्वेना{\color{DodgerBlue3}“विशिष्टो अन्योऽदृष्टहेतुरपि”} तद्धेतुकत्वेन संप्रतीयते (।) किमिव । {\color{DodgerBlue3}“अग्निकाष्ठवत्”} । काष्ठकार्यत्वेन वह्नेर्निश्चितत्वात् (।) (।) वह्निदर्शनाददृष्टमपि काष्ठमनुमीयते\edlabel{pvv.376-2}\footnote{\label{pvv.376-2}  २ अपौरुषेयत्वप्रतिज्ञाया अनुमानबाधोक्तानेन ।} । न च\edlabel{pvv.376-3}\footnote{\label{pvv.376-3}  ३ काष्ठहेतुकोयमग्निरिति ।} वैदिकपौरुषेयवाक्यानां कश्चिद्भेदः (।) सर्व्वेषां दुर्भणत्वादीनां मन्त्रादिसामार्थ्यानाञ्च साधारणत्वात् ।
	\pend
      

	  \pstart इदानीम्वेदकरणसमर्थपुरुषादर्शनं भा र तादिष्वपि समानं\edlabel{pvv.376-4}\footnote{\label{pvv.376-4}  ४ इदानीन्तदकारणात् ।} । ततोऽसत्यवान्तरभेदे भेदाद्धेतूपन्यासो न युक्तः । सति तु वस्तुतः (।)
	\pend
      
	  \bigskip
	  \begingroup
	  \large
	
	    
	    \stanza[\smallbreak]
	\label{pv.3.243b}\edlabel{pv.3.243b}\flagstanza{\tiny\textenglish{...3.243b}}तत्राप्रदर्श्य ये भेदं कार्यसामान्यदर्शनात् ॥ २४३ ॥\&[\smallbreak]


	
	  \endgroup
	
	  \bigskip
	  \begingroup
	  \large
	
	    
	    \stanza[\smallbreak]
	\label{pv.3.244a}\edlabel{pv.3.244a}\flagstanza{\tiny\textenglish{...3.244a}}हेतवः प्रवितन्यन्ते सर्वे ते व्यभिचारिणः ।\&[\smallbreak]


	
	  \endgroup
	

	  \pstart तत्र साधनीकृते\edlabel{pvv.376-5}\footnote{\label{pvv.376-5}  ५ एतस्मिन्न्याये स्थिते ।} वस्तुन्यवान्तरभेदे भेद\edlabel{pvv.376-6}\footnote{\label{pvv.376-6}  ६ लौकिकवैदिकयोः ।} {\color{DodgerBlue3}“मप्रदर्श्य कार्यसा\edlabel{pvv.376-7}\footnote{\label{pvv.376-7}  ७ पुरुषकार्यैः शब्दैः सामान्यस्य तुल्यस्य वैदिकशब्देषु दर्शनात् ।}मान्य”}स्य विजातीयव्यावृत्तिमात्रस्य {\color{DodgerBlue3}“दर्शनात्”} । ये {\color{DodgerBlue3}“हेतवो”} यद्वेदाध्ययनं तद्वेदाध्ययनपूर्व्वकं न करणपूर्व्वकं\edlabel{pvv.376-8}\footnote{\label{pvv.376-8}  ८ स्वयमकृत्वा वेदस्य अध्ययनं दार्ष्टान्तिकमग्नेर्दृष्टान्तो नैकान्तिकत्वसाधनस्य ।} यथा यः पथिकाग्निः\edlabel{pvv.376-9}\footnote{\label{pvv.376-9}  ९ आद्योपि पथिककृतोग्निरदृष्टहेतुत्वात् कालान्तरहेतुकः पथिकाग्नित्वात् ज्वालान्तरसंभूतदृश्यमानाग्निवत् ज्वालारणिजन्मनोरबाध्यबाधकत्वादनैकान्तिको वह्निसामान्ये ।} स ज्वालापूर्व्वको न वारणिनिर्मथनपूर्व्वक \leavevmode\marginnote{\textenglish{377/s}} इत्यादयः {\color{DodgerBlue3}“प्रवितन्यन्ते”} विस्तार्यन्ते {\color{DodgerBlue3}“सर्व्व ते व्यभिचारिणोऽनैकान्तिकाः न हि\edlabel{pvv.377-1}\footnote{\label{pvv.377-1}  १ वेदक्रियाशक्तिरहितस्योपदेष्ट्टपूर्व्वकाध्यायदृष्टिर्न्न ब्रह्मादेरपि तथात्वं स्वयं रचयित्वाध्ययनसंभवाविरोधात् ।}”} वेदाध्ययनमित्येवाध्ययनपूर्व्वकं {\color{DodgerBlue3}“कृत्वा”} करण\edlabel{pvv.377-2}\footnote{\label{pvv.377-2}  २ यद्वेदाध्ययनमित्यादौ वा । ७ स्वयं कृत्वा ।}पूर्व्वकस्याप्यध्ययनस्योपपत्तेः । अरणिनिर्मथनस्य च वह्नेर्भावाविरोधात् ।
	\pend
      

	  \pstart भवतु वा
	\pend
      
	  \bigskip
	  \begingroup
	  \large
	
	    
	    \stanza[\smallbreak]
	\label{pv.3.244b}\edlabel{pv.3.244b}\flagstanza{\tiny\textenglish{...3.244b}}सर्व्वथाऽनादिता सिद्ध्येदेवं नापुरुषाश्रयः ॥ २४४ ॥\&[\smallbreak]


	
	  \endgroup
	
	  \bigskip
	  \begingroup
	  \large
	
	    
	    \stanza[\smallbreak]
	\label{pv.3.245a}\edlabel{pv.3.245a}\flagstanza{\tiny\textenglish{...3.245a}}तस्मादपौरुषेयत्वे स्यादन्योप्यनराश्रयः ।\&[\smallbreak]


	
	  \endgroup
	

	  \pstart एवं वेदाध्ययनमध्ययनपूर्व्वतासा धनं\edlabel{pvv.377-3}\footnote{\label{pvv.377-3}  ३ अभ्युपगम्याह ।} तथापि {\color{DodgerBlue3}“सर्व्वथाऽनादिता”} वेदाध्ययनस्य {\color{DodgerBlue3}“सिध्येत्”} डिम्भकपांशुक्रीडादीनामिव {\color{DodgerBlue3}“नापुरुषाश्रयः”} पुरुषाश्रयणाभावस्तु {\color{DodgerBlue3}“न”} सिध्येत् । डिम्भकपांशुक्रीडा\edlabel{pvv.377-4}\footnote{\label{pvv.377-4}  ४ आदिना भोजनादि बालस्य ।}दयो हि दर्शनपूर्व्वका अनादयश्च न चापौरुषेयाः डिम्भकैरेव क्रियमाणत्वात् । एवं शब्दा अप्यध्येतृभिरेव क्रियन्ते न तु स्वयमात्मानं ध्वनयन्ति येनापौरुषेयाः स्युः । अनादिस्तादृक् करणक्रमः पुरुषपरंपराया अनादेरागतत्वात् । अथानादित्वादेवापौरुषेयतेत्याह । तस्यानादित्वा{\color{DodgerBlue3}“दपौरुषेयत्वे साध्ये स्यादन्यो\edlabel{pvv.377-5}\footnote{\label{pvv.377-5}  ५ लोकव्यवहारः ।}”} प्यपौरुषेयो{\color{DodgerBlue3}“ऽनराश्रय”} इति प्रसङ्गः ।
	\pend
      

	  \pstart b. तमेवाह ।
	\pend
      
	  \bigskip
	  \begingroup
	  \large
	
	    
	    \stanza[\smallbreak]
	\label{pv.3.245b}\edlabel{pv.3.245b}\flagstanza{\tiny\textenglish{...3.245b}}म्लेच्छादिव्यवहाराणां नास्तिक्यवचसामपि ॥ २४५ ॥\&[\smallbreak]


	
	  \endgroup
	
	  \bigskip
	  \begingroup
	  \large
	
	    
	    \stanza[\smallbreak]
	\label{pv.3.246a}\edlabel{pv.3.246a}\flagstanza{\tiny\textenglish{...3.246a}}अनादित्वाद् तथाभावः ;\&[\smallbreak]


	
	  \endgroup
	

	  \pstart {\color{DodgerBlue3}“म्लेच्छादेर्व्यवहाराणां”} मा\edlabel{pvv.377-6}\footnote{\label{pvv.377-6}  ६ मृते भर्त्तरि पुत्रेण मातृविवाहः कार्यः । वृद्धादीनां मारणं संसारमोचनार्थं आदिना मदनत्रयोदश्यां मदनोत्सवः पुत्रादिजन्मोत्सवः ।}तृविवाहमुक्तिप्रापणमारणादीनां {\color{DodgerBlue3}“नास्तिक्यवचसा\edlabel{pvv.377-7}\footnote{\label{pvv.377-7}  ७ लोकायतिकानां ।}मपि”} परलोककर्मफलाद्यपवादिनामनादित्वात् तथाभावोऽपौरुषेयत्वं स्यात् ।
	\pend
      

	  \pstart यदि पौरुषेयाः कथमनादय इत्याह ।
	\pend
      
	  \bigskip
	  \begingroup
	  \large
	
	    
	    \stanza[\smallbreak]
	\label{pv.3.246b}\edlabel{pv.3.246b}\flagstanza{\tiny\textenglish{...3.246b}}पूर्वसंस्कारसन्ततेः ।\&[\smallbreak]


	
	  \endgroup
	\leavevmode\marginnote{\textenglish{378/s}}

	  \pstart {\color{DodgerBlue3}“पूर्व्वसंस्काराद”}नादेः {\color{DodgerBlue3}“सन्ततेः”} सन्तानेन प्रवृत्तेः ।
	\pend
      

	  \pstart अथानादित्वात् म्लेच्छादिव्यवहाराणाञ्चापौरुषेयतास्तित्वाह ।
	\pend
      
	  \bigskip
	  \begingroup
	  \large
	
	    
	    \stanza[\smallbreak]
	\label{pv.3.246c}\edlabel{pv.3.246c}\flagstanza{\tiny\textenglish{...3.246c}}तादृशेऽपौरुषेयत्वे कः सिद्धेपि गुणो भवेत् ॥ २४६ ॥\&[\smallbreak]


	
	  \endgroup
	

	  \pstart {\color{DodgerBlue3}“तादृशे”} म्लेच्छादिव्यवहारसाधारणेऽ{\color{DodgerBlue3}“पौरुषे”}यत्वे {\color{DodgerBlue3}“सिद्धेपि को गुणः”} अविसम्वा\leavevmode\marginnote{\textenglish{75a/MA}}दकलक्षणो भवेत् । पौरुषेयवाक्यानां विसम्वादादर्शनादपौरुषेयत्वमिष्टं । स च तस्मिन्नपि सति म्लेच्छादिव्यवहाराणामिव दुष्परिहरः ।
	\pend
      \label{div_pvv.3.247}\edlabel{div_pvv.3.247}
	  
	% new div opening: depth here is 2
	

	  \begin{center}%% label @type='head'
	\textbf{(ख) a. अनादित्वेऽर्थसंस्कारभेदेन संशयः}
	\end{center}
	

	  \pstart अथ वेदवाक्यानामेवानादित्वादपौरुषेयत्वं तदा (।)
	\pend
      
	  \bigskip
	  \begingroup
	  \large
	
	    
	    \stanza[\smallbreak]
	\label{pv.3.247a}\edlabel{pv.3.247a}\flagstanza{\tiny\textenglish{...3.247a}}अर्थसंस्कारभेदानां दर्शनात् संशयः पुनः ।\&[\smallbreak]


	
	  \endgroup
	

	  \pstart अर्थस्य {\color{DodgerBlue3}“संस्कारो”} व्याख्यानं तस्य {\color{DodgerBlue3}“भेदानां”} विकल्पानां प्रकृतिप्रत्ययानेकार्थ\edlabel{pvv.378-1}\footnote{\label{pvv.378-1}  १ अपौरुषेयत्वं कल्पयित्वापि पुनः संशय एव यतो वेदार्थव्याख्याविकल्पानामाचार्यभेदेन भेदः ।}त्वात् रूढेर्निरुक्तादिभ्यश्च यथाप्रतिभं पुंसान्द{\color{DodgerBlue3}“र्शनात् संशयो”}ऽर्थनिश्चयाभावः ।
	\pend
      

	  \pstart b. कस्य चापौरुषेयत्वमिष्टं किम्वर्ण्णानामुत पदवाक्यानामित्याह\edlabel{pvv.378-2}\footnote{\label{pvv.378-2}  २ वर्ण्णविकल्पमधिकृत्य ।} ।
	\pend
      
	  \bigskip
	  \begingroup
	  \large
	
	    
	    \stanza[\smallbreak]
	\label{pv.3.247b}\edlabel{pv.3.247b}\flagstanza{\tiny\textenglish{...3.247b}}अन्याविशेषाद् वर्ण्णानां साधने किं फलं भवेत् ॥ २४७ ॥\&[\smallbreak]


	
	  \endgroup
	

	  \pstart {\color{DodgerBlue3}“अन्यै”}र्लोकिकैर्वर्ण्णैर्वैदिकानाम{\color{DodgerBlue3}“विशेषात्”}प्रत्य{\color{DodgerBlue3}“भिज्ञा”}यमानत्वेनैकत्वादपौरुषेयत्वस्य {\color{DodgerBlue3}“साधने किं फलं भवेत्”} (।) तथात्वे लौकिकानामर्थव्यभिचारात् ।
	\pend
      \label{div_pvv.3.248_3.249_3.250}\edlabel{div_pvv.3.248_3.249_3.250}
	  
	% new div opening: depth here is 2
	
	  \bigskip
	  \begingroup
	  \large
	
	    
	    \stanza[\smallbreak]
	\label{pv.3.348}\edlabel{pv.3.348}\flagstanza{\tiny\textenglish{....3.348}}c. वाक्यं भिन्नं न वर्णेभ्यो विद्यतेऽनुपलम्भतः ।&अनेकावयवात्मत्वे पृथक् तेषां निरर्थता ॥ २४८ ॥\&[\smallbreak]


	
	  \endgroup
	
	  \bigskip
	  \begingroup
	  \large
	
	    
	    \stanza[\smallbreak]
	\label{pv.3.249a}\edlabel{pv.3.249a}\flagstanza{\tiny\textenglish{...3.249a}}अतद्रूपे च ताद्रूप्यं कल्पितं सिंहतादिवत् ।\&[\smallbreak]


	
	  \endgroup
	

	  \pstart {\color{DodgerBlue3}“वाक्यं”} पदञ्च {\color{DodgerBlue3}“वर्ण्णेभ्यो भिन्नं न विद्यतेऽनुपलम्भात्”} (।) तत्कथमस्यापौरुषेयत्वं\edlabel{pvv.378-3}\footnote{\label{pvv.378-3}  ३ प्रत्येकं वर्ण्णानर्थक्यादर्थप्रतीतिकार्यान्यथानुपपत्त्यास्ति पदादीति चेन्न यावान्वर्ण्णसमुदायोर्थप्रतीतये संकेतस्तावतोर्थप्रतीत्यभाव एतस्मान्न चैवं दृश्यानुपलम्भाच्च पदादेः ।} साध्यं । अभिन्नं चेत् तदा{\color{DodgerBlue3}“नेकावयवात्मत्वे”} वाक्यस्य तेषामवयवानां पृथक् प्रत्येकं निरर्थकतेति पदात्मकमनर्थकमेव स्यात् । ततश्चातद्रूपेऽवाचक\leavevmode\marginnote{\textenglish{379/s}} रूपे ताद्रूप्यं वाचकत्वं कल्पितं कल्पनाबुद्धिनिर्मितं माणवकादाविव सिंहतादि । ततः\edlabel{pvv.379-1}\footnote{\label{pvv.379-1}  १ वाचकत्वस्य पुरुषेण कल्पितत्वात् ।} पौरुषेयमेव वाचकत्वं स्यादिति प्रस्तुतक्षतिः ।
	\pend
      

	  \pstart d. अथ प्रत्येकमवयवानां सार्थकत्वं तदा (।)
	\pend
      
	  \bigskip
	  \begingroup
	  \large
	
	    
	    \stanza[\smallbreak]
	\label{pv.3.249b}\edlabel{pv.3.249b}\flagstanza{\tiny\textenglish{...3.249b}}प्रत्येकं सार्थकत्वेपि मिथ्यानेकत्वकल्पना ॥ २४९ ॥\&[\smallbreak]


	
	  \endgroup
	
	  \bigskip
	  \begingroup
	  \large
	
	    
	    \stanza[\smallbreak]
	\label{pv.3.250a}\edlabel{pv.3.250a}\flagstanza{\tiny\textenglish{...3.250a}}एकावयवगत्या च वाक्यार्थप्रतिपद् भवेत् ।\&[\smallbreak]


	
	  \endgroup
	

	  \pstart प्रत्येकं सार्थकत्वे\edlabel{pvv.379-2}\footnote{\label{pvv.379-2}  २ वाक्यार्थेन ।}पि मिथ्यानेकत्वस्यानेकावयवात्मकस्य कल्पना\edlabel{pvv.379-3}\footnote{\label{pvv.379-3}  ३ एकस्याप्यवयवस्य वाक्यार्थवत्वादवयवान्तरापेक्षा न स्यात् ।} एकस्मादर्थप्रतीतेः ।\edlabel{pvv.379-4}\footnote{\label{pvv.379-4}  ४ यदा चैकोप्यवयवोर्थवान् तदा ।} तथैकस्यावयवस्य गत्या वाक्यार्थस्य प्रतिपत् प्रतीतिर्भवेत् (।) एकस्यापि वाचकत्वात् । तत्समुदायो वाचक इति चेत् । स किन्तेभ्यो भिन्नः । स च प्रत्युक्तः (।) अवयवा मिलिताः समुदाय इति चेत् । स च\edlabel{pvv.379-5}\footnote{\label{pvv.379-5}  ५ परवर्ण्णोच्चारणे पूर्व्वध्वंसात् ।} न सम्भवत्येव । तद् यदि प्रत्येकं वाचकत्वमेव तदैषां प्रत्येकं वाचकत्वे चैकस्मादपि स्यादर्थप्रतीतिः ।
	\pend
      

	  \pstart अथ । (।)
	\pend
      
	  \bigskip
	  \begingroup
	  \large
	
	    
	    \stanza[\smallbreak]
	\label{pv.3.250b}\edlabel{pv.3.250b}\flagstanza{\tiny\textenglish{...3.250b}}सकृच्छ्रुतौ च सर्वेषां कालभेदो न युज्यते ॥ २५० ॥\&[\smallbreak]


	
	  \endgroup
	

	  \pstart {\color{DodgerBlue3}“एकावयवगत्याऽर्थप्रतिपत्ते”}रवयवान्तरवैफल्यदोषात् सर्व्वावयवानां सकृच्छ्रुतिरिष्यते । तदा सर्व्वेषामवयवानां {\color{DodgerBlue3}“सकृच्छ्रुतौ”} चाभिमतायामवयवश्रवणस्य {\color{DodgerBlue3}“कालभेदो\edlabel{pvv.379-6}\footnote{\label{pvv.379-6}  ६ एकावयवबोधकाले सर्व्वत्र श्रवणात् ।} न युज्यते”} । दृश्येत च ।
	\pend
      \label{div_pvv.3.251_3.252_3.253_3.254_3.255_3.256_3.257_3.258_3.259_3.260}\edlabel{div_pvv.3.251_3.252_3.253_3.254_3.255_3.256_3.257_3.258_3.259_3.260}
	  
	% new div opening: depth here is 2
	

	  \begin{center}%% label @type='head'
	\textbf{(ग) स्फोटनिरासः}
	\end{center}
	

	  \pstart अथानवयवमेकं वर्ण्णेभ्यो व्यतिरिक्तं स्फोटरूपं वाक्यं तच्च क्रमवद्भिर्न्नियतानुपूर्व्वीकैर्ध्वनिभिः क्रमेण व्यज्यते । व्यक्त्यनुक्रमेणैव च क्रमवत् प्रतीयते तद्रूपाविभागेन नाद\edlabel{pvv.379-7}\footnote{\label{pvv.379-7}  ७ नादानां भेदात् ।}रूपाणाम्वर्ण्णानां ग्रहणात् वर्ण्णविभागवच्च लक्ष्यते । वस्तुतस्तथारहितमपीति केचित् ।
	\pend
      
	  \bigskip
	  \begingroup
	  \large
	
	    
	    \stanza[\smallbreak]
	\label{pv.3.251a}\edlabel{pv.3.251a}\flagstanza{\tiny\textenglish{...3.251a}}एकत्वेपि ह्यभिन्नस्य क्रमशो गत्यसम्भवात् ।\&[\smallbreak]


	
	  \endgroup
	

	  \pstart {\color{DodgerBlue3}“तदेकत्वेपि ह्यभिन्नस्या”}नवयवस्य स्फोटरूपस्य वाक्यस्य प्रथमध्वनिनापि व्यक्तत्वात् {\color{DodgerBlue3}“क्रमशो गतेः\edlabel{pvv.379-8}\footnote{\label{pvv.379-8}  ८ गृहीतागृहीतयोरभेदात् ।}”} प्रतीतेर{\color{DodgerBlue3}“सम्भवात्”} सकृत् प्रतीतिप्रसङ्गः\edlabel{pvv.379-9}\footnote{\label{pvv.379-9}  ९ जातिस्फोटस्तु जात्यभावादेव निरस्तः ।} । यदि प्रथमव्यक्तौ न प्रतीतिरपरास्वपि न स्यात् । तदेकव्यञ्चकत्वाद् व्यक्तीनां ।
	\pend
      \leavevmode\marginnote{\textenglish{380/s}}

	  \pstart किञ्च (।) वाक्यन्तन्नित्यमनित्यम्वा स्यात् ।
	\pend
      
	  \bigskip
	  \begingroup
	  \large
	
	    
	    \stanza[\smallbreak]
	\label{pv.3.251b}\edlabel{pv.3.251b}\flagstanza{\tiny\textenglish{...3.251b}}अनित्यं यत्नसम्भूतं पौरुषेयं कथन्न तत् ॥ २५१ ॥\&[\smallbreak]


	
	  \endgroup
	
	  \bigskip
	  \begingroup
	  \large
	
	    
	    \stanza[\smallbreak]
	\label{pv.3.252a}\edlabel{pv.3.252a}\flagstanza{\tiny\textenglish{...3.252a}}नित्योपलब्धिनित्यत्वेऽप्यनावरणसम्भवात् ।\&[\smallbreak]


	
	  \endgroup
	

	  \pstart {\color{DodgerBlue3}“यद्यनित्यं पौरुषेयं कथं न तत् यत्नसम्भ”}वात्\edlabel{pvv.380-1}\footnote{\label{pvv.380-1}  १ अवश्यं ह्युत्पत्तिमदनित्यं कुतश्चित् स्यान्निर्हेतुकत्वे देशादिनियमाभावात् तच्च पुरुषप्रयत्नान्वयव्यतिरेकानुविधायिपौरुषेयमेव स्यात् । किञ्चिदेव प्रतिनियतं वस्तुस्थित्यास्ति तत्कदाचित् कस्यचिद् भवति तेन क्विचित् कदाचिच्छ्रवणं ।} घटादिवत् । {\color{DodgerBlue3}“नित्यत्वेप्युपलब्धिर्नित्या”} भवेत् । वाक्यानाम{\color{DodgerBlue3}“नावरण”}सम्भवात् । आवरणायोगात् । यदि नित्यं कदाचिदुपलभ्यते तदा तस्यानावृतस्वभावताऽभ्युपगन्तव्या । तादृशं नित्यमुपलभ्येताभिमतकाल इव । आवृतस्वभावे तु सर्व्वदाऽनुपलम्भः स्यात् । तस्मात् कालभेदेनोपलभ्यानुपलभ्यस्वभावत्वादेकस्य नाशादन्यस्योदय इति स्यात् । तथा च नित्यत्वक्षतिः ।
	\pend
      

	  \pstart एकस्वभाव एव शब्दः परं (।)
	\pend
      
	  \bigskip
	  \begingroup
	  \large
	
	    
	    \stanza[\smallbreak]
	\label{pv.3.252b}\edlabel{pv.3.252b}\flagstanza{\tiny\textenglish{...3.252b}}अश्रुतिर्व्विकलत्वाच्चेत् कस्यचित् सहकारिणः ॥ २५२ ॥\&[\smallbreak]


	
	  \endgroup
	
	  \bigskip
	  \begingroup
	  \large
	
	    
	    \stanza[\smallbreak]
	\label{pv.3.253a}\edlabel{pv.3.253a}\flagstanza{\tiny\textenglish{...3.253a}}काममन्यप्रतीक्षोक्तिर्नियमस्तु विरुध्यते ।\leavevmode\marginnote{\textenglish{75b/MA}}\&[\smallbreak]


	
	  \endgroup
	

	  \pstart {\color{DodgerBlue3}“कस्यचित् सहकारिणो”} ज्ञानजनकस्य {\color{DodgerBlue3}“वैकल्यादश्रुति”}रिति {\color{DodgerBlue3}“चेत् । काममन्य”}स्य सहकारिण उपकारकस्य प्रतीक्षापेक्षणमस्तु नियमः पूर्व्वापरैकस्वभावतावरणन्तु शब्दानां {\color{DodgerBlue3}“विरुध्यते”} । न ह्येकस्वभावः परमपेक्षते चेति क्षमं । उपकारकस्यापेक्षणीयत्वात् । उपकारान्तरस्य च स्वभावान्तरलक्षणत्वात् ।
	\pend
      

	  \pstart अपि च शब्दा अव्यापिनो व्यापिनो वा स्युः । तत्र (।)
	\pend
      
	  \bigskip
	  \begingroup
	  \large
	
	    
	    \stanza[\smallbreak]
	\label{pv.3.253b}\edlabel{pv.3.253b}\flagstanza{\tiny\textenglish{...3.253b}}सर्वत्रानुपलम्भः स्यात् तेषामव्यापिता यदि ॥ २५३ ॥\&[\smallbreak]


	
	  \endgroup
	
	  \bigskip
	  \begingroup
	  \large
	
	    
	    \stanza[\smallbreak]
	\label{pv.3.254a}\edlabel{pv.3.254a}\flagstanza{\tiny\textenglish{...3.254a}}सर्वेषामुपलम्भः स्यात् युगपद् व्यापिता यदि ।\&[\smallbreak]


	
	  \endgroup
	

	  \pstart {\color{DodgerBlue3}“यद्यव्यापिता तेषां सर्व्वत्र देशेऽऽनुपलम्भः स्यात्”} । न ह्येकदेशस्थितः शैलः सर्व्वत्रोपलभ्यते । योग्यतातिशयलाभाद् व्यञ्चकेभ्यो दृश्यते एवेति चेत् । एवन्तर्हि सर्व्वैः सर्व्वदेशस्थैरुपलभ्येत । तथा {\color{DodgerBlue3}“व्यापिता यदि\edlabel{pvv.380-2}\footnote{\label{pvv.380-2}  २ श्रोत्रजप्रतिज्ञया नित्यत्वं व्यापित्वञ्च शब्दानामित्याह ।} सर्व्वेषां”} शब्दानां {\color{DodgerBlue3}“युगपदुपलम्भः स्यात्”} । व्यापित्वात् सर्व्वेषां सर्व्वत्र भावात् ।
	\pend
      

	  \pstart अथ व्यापित्वेपि य एवाभिव्यक्त्या संस्कृतः\edlabel{pvv.380-3}\footnote{\label{pvv.380-3}  ३ प्रयत्नाभिहतवायुना संस्कृतस्य शब्दस्य संस्कृतेनैवेन्द्रियेणोपलब्धेर्न प्रसङ्गः ।} स एवोपलभ्यते नेतरः ।
	\pend
      \leavevmode\marginnote{\textenglish{381/s}}
	  \bigskip
	  \begingroup
	  \large
	
	    
	    \stanza[\smallbreak]
	\label{pv.3.254b}\edlabel{pv.3.254b}\flagstanza{\tiny\textenglish{...3.254b}}संस्कृतस्योपलम्भे च कः संस्कर्त्ताऽविकारिणः ॥ २५४ ॥\&[\smallbreak]


	
	  \endgroup
	
	  \bigskip
	  \begingroup
	  \large
	
	    
	    \stanza[\smallbreak]
	\label{pv.3.255a}\edlabel{pv.3.255a}\flagstanza{\tiny\textenglish{...3.255a}}इन्द्रियस्य स्यात् संस्कारः शृणुयान्निखिलञ्च तत् ।\&[\smallbreak]


	
	  \endgroup
	

	  \pstart {\color{DodgerBlue3}“संस्कृतस्य चोपलम्भे च”} स्वीक्रियमाणे नित्यत्वाद{\color{DodgerBlue3}“विकारिणः कः संस्कर्त्ता”} नाम । न ह्यनुपकुर्व्वन् संस्कर्त्ता अनुपकार्यो वा संस्कार्यः । अथेन्द्रियमनित्यत्वात् संस्कार्य ततः संस्कृत एव पश्यति नान्य इति विभागः । सत्यमिन्द्रियस्य स्यात् संस्कारः । किन्तु तदिन्द्रियं {\color{DodgerBlue3}“निखिलं च”} शब्दग्रामं {\color{DodgerBlue3}“शृणुयात्”} न त्वेकशब्दं ।\edlabel{pvv.381-1}\footnote{\label{pvv.381-1}  १ अथा यथा शब्दप्रतिपत्त्यन्यथानुपपत्त्येन्द्रियस्य संस्कारकल्पना तथा संस्कारभेदो यतः प्रतिविषयं भिन्नत्वादिन्द्रियस्यैकशब्दस्य ग्रहणं । प्रतिनियता हि संस्काराः शब्दानां तत्र केनचित् संस्कृतमिन्द्रियं कस्य(चि)देव ग्राहकं ।}
	\pend
      
	  \bigskip
	  \begingroup
	  \large
	
	    
	    \stanza[\smallbreak]
	\label{pv.3.255b}\edlabel{pv.3.255b}\flagstanza{\tiny\textenglish{...3.255b}}संस्कारभेदभिन्नत्वादेकार्थनियमो यदि ॥ २५५ ॥\&[\smallbreak]


	
	  \endgroup
	
	  \bigskip
	  \begingroup
	  \large
	
	    
	    \stanza[\smallbreak]
	\label{pv.3.256a}\edlabel{pv.3.256a}\flagstanza{\tiny\textenglish{...3.256a}}अनेकशब्दसंघाते श्रुतिः कलकले कथम् ।\&[\smallbreak]


	
	  \endgroup
	

	  \pstart {\color{DodgerBlue3}“संस्कार”}स्य शब्दविषयस्य {\color{DodgerBlue3}“भेदा”}त् प्रतिनियमाद् {\color{DodgerBlue3}“भिन्नत्वा”}दिन्द्रियसंस्काराणा{\color{DodgerBlue3}“मेक”}स्मि{\color{DodgerBlue3}“न्नर्थे”} शब्दे {\color{DodgerBlue3}“नियमः”} श्रुति{\color{DodgerBlue3}“र्यदी”}ष्यते तदा{\color{DodgerBlue3}“नेकशब्दसंघाते कलकले श्रुति”}रने{\color{DodgerBlue3}“केषां”} शब्दानां {\color{DodgerBlue3}“कथंसंस्कारप्रतिनियमादिन्द्रियं”} नानेकशब्दग्राहि स्यात् \edlabel{pvv.381-2}\footnote{\label{pvv.381-2}  २ तस्मात् ताल्वादिना शब्दकरणमेव तेन यावन्तः श्रूयन्ते ।} ।
	\pend
      

	  \pstart अथ (।)
	\pend
      
	  \bigskip
	  \begingroup
	  \large
	
	    
	    \stanza[\smallbreak]
	\label{pv.3.256b}\edlabel{pv.3.256b}\flagstanza{\tiny\textenglish{...3.256b}}ध्वनयः केवलं तत्र श्रूयन्ते चेन्न वाचकाः ॥ २५६ ॥\&[\smallbreak]


	
	  \endgroup
	
	  \bigskip
	  \begingroup
	  \large
	
	    
	    \stanza[\smallbreak]
	\label{pv.3.257a}\edlabel{pv.3.257a}\flagstanza{\tiny\textenglish{...3.257a}}ध्वनिभ्यो भिन्नमस्तीति श्रद्धेयमविवक्षितम् ।\&[\smallbreak]


	
	  \endgroup
	

	  \pstart {\color{DodgerBlue3}“तत्र”} कलकले {\color{DodgerBlue3}“ध्वनयः”} शब्दव्यञ्जकाः {\color{DodgerBlue3}“केवलं श्रूयन्ते न वाचकाः”} शब्दाः\edlabel{pvv.381-3}\footnote{\label{pvv.381-3}  ३ वाचक एव प्रतिनियतशक्तीन्द्रियं न ध्वनिषु ।} (।) {\color{DodgerBlue3}“ध्वनिभ्यः”} श्रूयमाणेभ्यो {\color{DodgerBlue3}“भिन्नं\edlabel{pvv.381-4}\footnote{\label{pvv.381-4}  ४ वर्ण्णपदादिशब्दस्य ध्वनिविशेषत्वात् ।}”} शब्दरूप{\color{DodgerBlue3}“मस्तीति”} यत्किञ्जिदिदमति{\color{DodgerBlue3}“श्रद्धेयं”} । श्रद्धावशाद् यद्येतदङ्गीक्रियते न तु प्रमाणबलात् ।
	\pend
      

	  \pstart किञ्ज (।)
	\pend
      
	  \bigskip
	  \begingroup
	  \large
	
	    
	    \stanza[\smallbreak]
	\label{pv.3.257b}\edlabel{pv.3.257b}\flagstanza{\tiny\textenglish{...3.257b}}स्थितेष्वन्येषु शब्देषु श्रूयते वाचकः कथम् ॥ २५७ ॥\&[\smallbreak]


	
	  \endgroup
	

	  \pstart कलकले ध्वनिमात्रं यदि श्रूयते न शब्दः तदा बहूनां व्याहर्त्तॄणां तूष्णीम्भा{\color{DodgerBlue3}“वादन्येष्वप्येषु शब्देषु स्थितेषु”} एकस्मिन् पुरुषे व्याहरति {\color{DodgerBlue3}“कथं वाचकः श्रूयते”} \leavevmode\marginnote{\textenglish{382/s}} \edlabel{pvv.382-1}\footnote{\label{pvv.382-1}  १ अवाचकश्रवणादेव । वाचकेन सह पृथग्वा ध्वनेरपि श्रवणं स्यादित्यर्थः ।}कलकलइ व ध्वनिमा\edlabel{pvv.382-2}\footnote{\label{pvv.382-2}  २ कलकलेपि वाचकश्रुतिः स्यात् पदवाक्यच्छेदबाधात् ।}त्रश्रुतिस्तदापि स्यात् । अथ वाचकस्योलब्धिप्रत्ययभावा\edlabel{pvv.382-3}\footnote{\label{pvv.382-3}  ३ पूर्वपूर्वध्वनिभागस्योत्तरोत्तरेण ध्वनिभागेनासन्धानाद् वैयाकरणाद्यैरवाचका इष्टा ते दोषाः । तस्मादनुपलब्धिवादितो ध्वनिः}दुपलब्धिस्तदा कलकलेपि स्याद् विशेषाभावात् (।)
	\pend
      

	  \pstart यदि चेन्द्रियाणां संस्कारविशेषाच्छब्दविशेषोपलब्धिप्रतिनियमस्तदा (।)
	\pend
      
	  \bigskip
	  \begingroup
	  \large
	
	    
	    \stanza[\smallbreak]
	\label{pv.3.258a}\edlabel{pv.3.258a}\flagstanza{\tiny\textenglish{...3.258a}}कथं वा शक्तिनियमाद् भिन्नध्वनिगतिर्भवेत् ।\&[\smallbreak]


	
	  \endgroup
	

	  \pstart शक्तिनियमस्तदा {\color{DodgerBlue3}“शक्तिनियमादि”}न्द्रियाणां {\color{DodgerBlue3}“कथं भिन्नध्वनिगतिर्भवेत् ।\edlabel{pvv.382-4}\footnote{\label{pvv.382-4}  ४ गतौ वा युगपन्नानारूपध्वनिश्रुतिवच्छब्दानामपि स्यात् न हि तैः किञ्चिदपराद्धं ।}”} शब्दविशेषवत् ध्वनिविशेषस्यैव ग्राहकमिन्द्रियं स्यात् तत्कथं कलकलध्वनिप्रतीतिः ।
	\pend
      

	  \pstart किञ्च (।)
	\pend
      
	  \bigskip
	  \begingroup
	  \large
	
	    
	    \stanza[\smallbreak]
	\label{pv.3.258b}\edlabel{pv.3.258b}\flagstanza{\tiny\textenglish{...3.258b}}ध्वनयः संमता यैस्ते दोषैः कैरप्यवाचकाः ॥ २५८ ॥\&[\smallbreak]


	
	  \endgroup
	
	  \bigskip
	  \begingroup
	  \large
	
	    
	    \stanza[\smallbreak]
	\label{pv.3.259a}\edlabel{pv.3.259a}\flagstanza{\tiny\textenglish{...3.259a}}ध्वनिभिर्ब्यज्यमानेस्मिन् वाचकेपि कथं न ते ।\&[\smallbreak]


	
	  \endgroup
	

	  \pstart {\color{DodgerBlue3}“यैः कैरपि दोषैर्ध्वनयोऽवाचकाः सम्मतास्ते”} दोषा {\color{DodgerBlue3}“व्यज्यमानेपि”} वाचके{\color{DodgerBlue3}“स्मिन् कथन्न”} भवन्ति । तथाहि । यथा प्रत्येकमवाचकत्वाद्वाच(क)त्वे {\color{DodgerBlue3}“वा”} ध्वन्यन्तरवैफल्यात् साहित्याभावाच्च ध्वनयोऽवाचकास्तथा ध्वनिभिः प्रत्येकं वाचकानभिव्यक्तेरभिव्यक्तौ वा ध्वन्यन्तरे वैफल्यात् साहित्याभावाच्च नाभिव्यज्येत\edlabel{pvv.382-5}\footnote{\label{pvv.382-5}  ५ तन्न वर्ण्णाद्यपौरुषेयता ।} शब्दः । अनभिव्यक्तश्च कमर्थ प्रतिपादयेत् ।
	\pend
      

	  \begin{center}%% label @type='head'
	\textbf{(घ) a. वर्णानुपूर्विचिन्ता}
	\end{center}
	
	  \bigskip
	  \begingroup
	  \large
	
	    
	    \stanza[\smallbreak]
	\label{pv.3.259b}\edlabel{pv.3.259b}\flagstanza{\tiny\textenglish{...3.259b}}वर्ण्णानुपूर्वी वाक्यं चेन्न वर्ण्णानामभेदतः ॥ २५९ ॥\&[\smallbreak]


	
	  \endgroup
	
	  \bigskip
	  \begingroup
	  \large
	
	    
	    \stanza[\smallbreak]
	\label{pv.3.260a}\edlabel{pv.3.260a}\flagstanza{\tiny\textenglish{...3.260a}}तेषाञ्च न व्यवस्थानं क्रमान्तरविरोधतः ।\&[\smallbreak]


	
	  \endgroup
	

	  \pstart {\color{DodgerBlue3}“वर्ण्णानामानुपूर्व्वी”} परिपाटिविशेषो {\color{DodgerBlue3}“वाक्यं\edlabel{pvv.382-6}\footnote{\label{pvv.382-6}  ६ न वर्णव्यतिरिक्तं तच्चापौरुषेयं ।}”} । तच्चोपलभ्यत एवेति {\color{DodgerBlue3}“चेत् । न वर्ण्णाना”}मानुपूर्व्याया {\color{DodgerBlue3}“अभेदतः”} । न हि वर्ण्णेभ्यो भिन्ना आनुपूर्व्वी प्रतीति\leavevmode\marginnote{\textenglish{383/s}} विषयः । ततश्च वर्ण्णा एव वाक्यमिति स्यात् । तेषाञ्च लोकवेदयोर्न विशेष इति सर्व्वत्र प्रामाण्यं । अथाप्रामाण्यम्वा स्यात् । अथ विशेषानुपूर्व्वीका वर्ण्णा एव वेदवाक्यं नेतर इत्याह \edlabel{pvv.383-1}\footnote{\label{pvv.383-1}  १ अथ क्रमो वर्ण्णानां धर्ममात्रं न वस्त्वन्तरं ततोयमदोष इत्याह ।} {\color{DodgerBlue3}“तेषाञ्च”} वर्ण्णानां {\color{DodgerBlue3}“न व्यवस्थान”}\edlabel{pvv.383-2}\footnote{\label{pvv.383-2}  २ न व्यवस्थितक्रमत्वं}क्रमनियमः ।\leavevmode\marginnote{\textenglish{76a/MA}} {\color{DodgerBlue3}“क्रमान्तरस्य”} लौकिकस्य {\color{DodgerBlue3}“विरोधतः”} । तथा हि वर्ण्णवदानुपूर्व्वी नित्या न च वर्ण्णा बहवः \edlabel{pvv.383-3}\footnote{\label{pvv.383-3}  ३ येन केनचिद् व्यवस्थि(त)क्रमा वैदिकाः स्युरन्ये लौकिका यथेष्टपरावृत्तयः ।} सन्ति । समानजातीया वा एकत्वाद् वर्ण्णस्य । ततश्चाग्निरित्येवा\edlabel{pvv.383-4}\footnote{\label{pvv.383-4}  ४ कर्ण्ण एकत्र एवाकारगकारो गकारश्च करणं ।}कारगकारनकाराणामानुपूर्व्वीविशिष्टः स्यात् नगमित्यन्यथा न भवेत् । कृतकानामपि बीजाङ्त्कुरपत्रादीना\edlabel{pvv.383-5}\footnote{\label{pvv.383-5}  ५ हेमन्तादि ।}मृतुसं\edlabel{pvv.383-6}\footnote{\label{pvv.383-6}  ६ शौक्रवार्हस्पत्यादि आदिना ग्रहादि ।}वत्सरादीनां विशिष्टानुपूर्वी नान्यथा भवति किम्पुनर्नित्यानां ।
	\pend
      

	  \pstart सा चेयमानुपूर्व्वी वर्ण्णानां (।)
	\pend
      
	  \bigskip
	  \begingroup
	  \large
	
	    
	    \stanza[\smallbreak]
	\label{pv.3.260b}\edlabel{pv.3.260b}\flagstanza{\tiny\textenglish{...3.260b}}देशकालक्रमाभावो व्याप्तिनित्यत्ववर्ण्णनात् ॥ २६० ॥\&[\smallbreak]


	
	  \endgroup
	

	  \pstart {\color{DodgerBlue3}“देश”}कृता वा \edlabel{pvv.383-7}\footnote{\label{pvv.383-7}  ७ देशकालाभ्यां कृतो यः क्रमस्तस्य वर्ण्णेष्वभावः ।} पिपीलिकानामिव पंक्ती स्यात् । {\color{DodgerBlue3}“काल”}कृता वा \edlabel{pvv.383-8}\footnote{\label{pvv.383-8}  ८ बीजकाले नाङ्कुरस्तत्काले न पत्रादि ।} बीजाङ्कुरादीनामिव । द्वयोरपि {\color{DodgerBlue3}“देशकालक्रमयोरभावो”} वर्ण्णानां {\color{DodgerBlue3}“व्याप्तिनित्यत्वयोर्व्वर्ण्णनात्”} । अन्योन्यदेशपरिहारेण । वृत्तिर्हि देशपौर्वापर्य । तच्च सर्व्वगानामसम्भवि । तथान्योन्यकालपरिहारेण वृत्तिः कालपौर्वापर्यञ्च नित्यानामसम्भवि । (२६०)
	\pend
      \label{div_pvv.3.261_3.262_3.263_3.264_3.265_3.266_3.267_3.268_3.269_3.270_3.271_3.272}\edlabel{div_pvv.3.261_3.262_3.263_3.264_3.265_3.266_3.267_3.268_3.269_3.270_3.271_3.272}
	  
	% new div opening: depth here is 2
	

	  \pstart अथानुपूर्व्वी समर्थनार्थमनित्यताऽव्यापितेष्यते ।
	\pend
      
	  \bigskip
	  \begingroup
	  \large
	
	    
	    \stanza[\smallbreak]
	\label{pv.3.261a}\edlabel{pv.3.261a}\flagstanza{\tiny\textenglish{...3.261a}}अनित्याव्यापितायाञ्च दोषः प्रागेव कीर्त्तितः ।\&[\smallbreak]


	
	  \endgroup
	

	  \pstart {\color{DodgerBlue3}“अनित्या\edlabel{pvv.383-9}\footnote{\label{pvv.383-9}  ९ “अनित्यं यत्नसंभूतं पौरुषेयं कथन्न तदि”\href{http://http://sarit.indology.info/?cref=pv.3.251}{(३।२५१)}त्यादिना ।}व्यापितायाञ्च दोषः प्रागेवोक्तः”} । अनित्यत्वे पौरुषेयता । अव्यापित्वे च सर्व्वत्रोपलब्धिश्च न स्यात्\edlabel{pvv.383-10}\footnote{\label{pvv.383-10}  १० न वर्ण्णानुपूर्व्वी वाक्यं येनायं दोषः स्यात् किन्तु वर्ण्णव्यक्तेरित्याह ।}।
	\pend
      
	  \bigskip
	  \begingroup
	  \large
	
	    
	    \stanza[\smallbreak]
	\label{pv.3.261b}\edlabel{pv.3.261b}\flagstanza{\tiny\textenglish{...3.261b}}व्यक्तिक्रमोपि वाक्यं न नित्यव्यक्तिनिराकृतेः ॥ २६१ ॥\&[\smallbreak]


	
	  \endgroup
	
	  \bigskip
	  \begingroup
	  \large
	
	    
	    \stanza[\smallbreak]
	\label{pv.3.262a}\edlabel{pv.3.262a}\flagstanza{\tiny\textenglish{...3.262a}}व्यापारादेव तत्सिद्धेः करणानां च कार्यता ।\&[\smallbreak]


	
	  \endgroup
	\leavevmode\marginnote{\textenglish{384/s}}

	  \pstart इति नित्यव्यापिनामपि शब्दानां {\color{DodgerBlue3}“व्यक्ते”}रभिव्यक्ते\edlabel{pvv.384-1}\footnote{\label{pvv.384-1}  १ यदा कार्यस्याक्रिया व्यक्तिः ।} प्रतिनियतदेशकालायाः {\color{DodgerBlue3}“क्रमो वाक्यं न”} युक्तः । {\color{DodgerBlue3}“नित्यस्य व्यक्तिनि\edlabel{pvv.384-2}\footnote{\label{pvv.384-2}  २ व्यञ्जककृतेन साक्षाज्जननक्त्युपधानेन ज्ञानजननासमर्थानां कार्यविशेष एव व्यक्तिः ।}राकृते”}र्ज्ञानोत्पादनहेतूनामित्यादिना । तस्मात् {\color{DodgerBlue3}“करणानां व्यापरादेव”} {\color{DodgerBlue3}“ते”}षाम्वर्ण्णानां सिद्धेः {\color{DodgerBlue3}“कार्य”}तैषां युक्ता न व्यङ्ग्यता ।
	\pend
      

	  \pstart b. किञ्च (।)
	\pend
      
	  \bigskip
	  \begingroup
	  \large
	
	    
	    \stanza[\smallbreak]
	\label{pv.3.262b}\edlabel{pv.3.262b}\flagstanza{\tiny\textenglish{...3.262b}}स्वज्ञानेनान्यधीहेतुः सिद्धेर्थे व्यञ्चको मतः ॥ २६२ ॥\&[\smallbreak]


	
	  \endgroup
	
	  \bigskip
	  \begingroup
	  \large
	
	    
	    \stanza[\smallbreak]
	\label{pv.3.263a}\edlabel{pv.3.263a}\flagstanza{\tiny\textenglish{...3.263a}}यथा दीपोन्यथा वापि को विशेषोस्य कारकात् ।\&[\smallbreak]


	
	  \endgroup
	

	  \pstart {\color{DodgerBlue3}“सिद्धे”} विद्यमानेऽर्थे \edlabel{pvv.384-3}\footnote{\label{pvv.384-3}  ३ स्वकारणादुत्पन्ने व्यङ्ग्ये ।} {\color{DodgerBlue3}“स्वज्ञानेन”} कारणेना\edlabel{pvv.384-4}\footnote{\label{pvv.384-4}  ४ दीपः स्वज्ञानद्वारा धटं बोधयति ।} {\color{DodgerBlue3}“न्य”}स्य ज्ञानहेतु{\color{DodgerBlue3}“र्व्यञ्जको मतः । प्रदीपो”} घटस्य {\color{DodgerBlue3}“यथा । अन्यथा वापीति”} यदि व्यङ्ग्यः प्राक् सिद्धो न भवेत् तदास्य \edlabel{pvv.384-5}\footnote{\label{pvv.384-5}  ५ शब्दो वा घटादिवत् कार्यवत् कार्य एव ? यः परार्थं प्रयुज्यते स प्रयोगात् प्राग् विद्यमानो यथा वाश्या(?वास्या)दि च्छिदायां । प्रयुज्यते च शब्दः परप्रत्यायनाय । प्रत्यभिज्ञयापि सिद्धः ॥ ग्रहे क्षणिकेपि कर्मणि प्रयोगो दीपादौ च प्रत्यभिज्ञेत्यनेकान्ता एते ।}व्यञ्जकस्य {\color{DodgerBlue3}“कारका”}द्धेतोः {\color{DodgerBlue3}“को विशेषो”} न कश्चित् । अपूर्व्वप्रतिपत्तिहेतुत्वाविशेषात् ।
	\pend
      

	  \pstart c. तथा (।)
	\pend
      
	  \bigskip
	  \begingroup
	  \large
	
	    
	    \stanza[\smallbreak]
	\label{pv.3.263b}\edlabel{pv.3.263b}\flagstanza{\tiny\textenglish{...3.263b}}करणानां समग्राणां व्यापारादुपलब्धितः ॥ २६३ ॥\&[\smallbreak]


	
	  \endgroup
	
	  \bigskip
	  \begingroup
	  \large
	
	    
	    \stanza[\smallbreak]
	\label{pv.3.264a}\edlabel{pv.3.264a}\flagstanza{\tiny\textenglish{...3.264a}}नियमेन च कार्यत्वं व्यञ्चके तदसम्भवात् ।\&[\smallbreak]


	
	  \endgroup
	

	  \pstart करणानां समग्राणां व्यापारात् नियमेनोपलब्धितश्च कार्यत्वमेव वर्ण्णानां व्यञ्चके दीपादौ तस्य व्यङ्ग्योपलब्धिनियमस्यासम्भवात् । न हि दीप इत्येव घटप्रतीतिः \edlabel{pvv.384-6}\footnote{\label{pvv.384-6}  ६ घटशून्ये देशेऽनुपलब्धे ।} करणसामग्र्यन्तु कार्यमवश्यम्भावयतीति करणसामग्र्ये नियतोपलम्भस्य कार्यतैव ।
	\pend
      
	  \bigskip
	  \begingroup
	  \large
	
	    
	    \stanza[\smallbreak]
	\label{pv.3.264b}\edlabel{pv.3.264b}\flagstanza{\tiny\textenglish{...3.264b}}d. तद्रूपावरणनां च व्यक्तिस्ते विगमो यदि ॥ २६४ ॥\&[\smallbreak]


	
	  \endgroup
	
	  \bigskip
	  \begingroup
	  \large
	
	    
	    \stanza[\smallbreak]
	\label{pv.3.265a}\edlabel{pv.3.265a}\flagstanza{\tiny\textenglish{...3.265a}}अभावे करणग्रामसामर्थ्यं किं नु तद्भवेत् ।\&[\smallbreak]


	
	  \endgroup
	\leavevmode\marginnote{\textenglish{385/s}}

	  \pstart {\color{DodgerBlue3}“तेषा”}म्वर्ण्णानां {\color{DodgerBlue3}“रूप”}स्य स्तिमितवायवीयावयवसंयोगरूपाणामा{\color{DodgerBlue3}“वरणानां”} प्रयत्नप्रेरितेन वायुना {\color{DodgerBlue3}“विगमो”} यदि {\color{DodgerBlue3}“व्यक्तिस्ते मी मां स क”} स्येष्टा तदा पूर्व्वावस्थात्यागे नातिशयो न व्यक्तिरनित्यत्वासक्तेः । उपलम्भावरणविगमो वा शब्दालम्बनं ज्ञानम्वा व्यक्तिः स्यादित्यत्राह कार्यं व्यक्तिरशक्या यस्मादावरणविगमेऽभावे नीरूपे {\color{DodgerBlue3}“करणग्रामस्य किं नु”} तत् {\color{DodgerBlue3}“सामर्थ्यं भवेत्”} । क्वचित् कर्त्तव्ये सामर्थ्यं स्यात् । न तु कर्त्तव्याभावे \edlabel{pvv.385-1}\footnote{\label{pvv.385-1}  १ नित्यस्यानाधेयातिशयत्वान्नावरणमित्युक्तेश्च ।} । यदि समस्तका{\color{DodgerBlue3}“त्वर्य”} (? र्यत्व)सम्भवेपि शब्दानां न कार्यता तदा (।)
	\pend
      
	  \bigskip
	  \begingroup
	  \large
	
	    
	    \stanza[\smallbreak]
	\label{pv.3.265b}\edlabel{pv.3.265b}\flagstanza{\tiny\textenglish{...3.265b}}शब्दाविशेषादन्येषामपि व्यक्तिः प्रसज्यते ॥ २६५ ॥\&[\smallbreak]


	
	  \endgroup
	
	  \bigskip
	  \begingroup
	  \large
	
	    
	    \stanza[\smallbreak]
	\label{pv.3.266a}\edlabel{pv.3.266a}\flagstanza{\tiny\textenglish{...3.266a}}तथाभ्युपगमे सर्व्वकारकाणां निरर्थता ।\&[\smallbreak]


	
	  \endgroup
	

	  \pstart {\color{DodgerBlue3}“शब्दादविशेषा\edlabel{pvv.385-2}\footnote{\label{pvv.385-2}  २ न किञ्चिदपि कार्यं स्यात् ।}दन्य”}व्यापारान्तरादुपलभ्यमानत्वनियमेनान्येषां घटादीना{\color{DodgerBlue3}“मपि”} {\color{DodgerBlue3}“व्यक्तिः प्रसज्येत”} कार्यता न स्यात् । \edlabel{pvv.385-3}\footnote{\label{pvv.385-3}  ३ सर्व्वस्य व्यङ्ग्यत्वमिष्टञ्चेत् ।} {\color{DodgerBlue3}“तथाभ्युपगमे”} च \edlabel{pvv.385-4}\footnote{\label{pvv.385-4}  ४ व्यक्तेः पक्षत्रयानतिक्रमात् तस्यात्रासंभवाज्ञानस्य सिद्धत्वात् ।} {\color{DodgerBlue3}“सर्व्वेषां कारकाणां”} व्यञ्च\edlabel{pvv.385-5}\footnote{\label{pvv.385-5}  ५ व्यञ्जकविकल्पत्रये द्वयं निरस्तं । न ज्ञानकरणात् सत्कार्यका (?वा) दे वस्तुवत् । असत्कार्ये तु सर्व्वस्य कार्यतासक्तिर्ज्ञानवत् स्यात् ।}काभिमतानां {\color{DodgerBlue3}“निरर्थकता”} । उत्पादकं हि कारणभिष्यते न तु व्यञ्जकं शब्दानामिव ।
	\pend
      
	  \bigskip
	  \begingroup
	  \large
	
	    
	    \stanza[\smallbreak]
	\label{pv.3.266b}\edlabel{pv.3.266b}\flagstanza{\tiny\textenglish{...3.266b}}e. साधनं प्रत्यभिज्ञानं सत्प्रयोगादि यन्मतम् ॥ २६६ ॥\&[\smallbreak]


	
	  \endgroup
	
	  \bigskip
	  \begingroup
	  \large
	
	    
	    \stanza[\smallbreak]
	\label{pv.3.267a}\edlabel{pv.3.267a}\flagstanza{\tiny\textenglish{...3.267a}}अनुदाहरणं सर्व्वभावानां क्षणभङ्गतः ।\&[\smallbreak]


	
	  \endgroup
	

	  \pstart यच्च\edlabel{pvv.385-6}\footnote{\label{pvv.385-6}  ६ किञ्च ।} नित्यत्वसिद्धये वर्ण्णानां {\color{DodgerBlue3}“प्रत्यभिज्ञानं\edlabel{pvv.385-7}\footnote{\label{pvv.385-7}  ७ अष्टकृत्वो गोशब्द उच्चारितो नाष्टौ गोशब्दाः । यत्प्रयुज्यते तत्प्राक् सद् यथा वास्यादि । यत्परार्थ प्रत्यायत्तमुच्चार्यते । उच्चरितप्रध्वंसित्वे परप्रत्यायनं न स्यात् ।} सत्प्रयोग”} आदिशब्दादुच्चार्यमाणत्वादि {\color{DodgerBlue3}“साधनं मतं”} सन्नेव हि प्रयुज्यते । यथा वास्यादि च्छिदायां (।) ततः प्रयोगात् प्रयोगेपि विद्यमानत्वाद् वर्ण्णा नित्या एव । त{\color{DodgerBlue3}“दनुदाहरणं”} दृष्टान्तविकलं\edlabel{pvv.385-8}\footnote{\label{pvv.385-8}  ८ विनाशस्याकारणत्वादिना ।} {\color{DodgerBlue3}“सर्व्वे”}षा{\color{DodgerBlue3}“म्भावानां क्षणभङ्गतः”} ।
	\pend
      
	  \bigskip
	  \begingroup
	  \large
	
	    
	    \stanza[\smallbreak]
	\label{pv.3.267b}\edlabel{pv.3.267b}\flagstanza{\tiny\textenglish{...3.267b}}दूष्यः कुहेतुरन्योपि ;\&[\smallbreak]


	
	  \endgroup
	

	  \pstart अनयैव दिशा परैरुच्यमानः {\color{DodgerBlue3}“कुहेतुर”}न्यो\edlabel{pvv.385-9}\footnote{\label{pvv.385-9}  ९ ........... साधनार्थः ।}पि दूष्यः ।\leavevmode\marginnote{\textenglish{76b/MA}}
	\pend
      \leavevmode\marginnote{\textenglish{386/s}}

	  \pstart f. अथ बुद्धिरभिव्यक्तिर्व्वर्ण्णानां सा च क्रमवती वाक्यमिष्टं तदा\edlabel{pvv.386-1}\footnote{\label{pvv.386-1}  १ ......ज्ञानं व्यक्तिरित्यत्र बुद्धीनां अनुक्रमो वाक्याव्यक्तीनां क्रमवत्वात् । तच्चायुक्तन्न बुद्धिरूपत्वाद् वाक्यस्य अभ्युपगम्य दोषमाह ।} वाक्यस्यापौरुषेयत्वेनेष्टत्वाद् बुद्धिरपौरुषेयी स्यात् । तत्र (।)
	\pend
      
	  \bigskip
	  \begingroup
	  \large
	
	    
	    \stanza[\smallbreak]
	\label{pv.3.267c}\edlabel{pv.3.267c}\flagstanza{\tiny\textenglish{...3.267c}}बुद्धेरपुरुषाश्रये ॥ २६७ ॥\&[\smallbreak]


	
	  \endgroup
	
	  \bigskip
	  \begingroup
	  \large
	
	    
	    \stanza[\smallbreak]
	\label{pv.3.268a}\edlabel{pv.3.268a}\flagstanza{\tiny\textenglish{...3.268a}}बाधाभ्युपेतप्रत्यक्षप्रतीतानुमितैः समम् ।\&[\smallbreak]


	
	  \endgroup
	

	  \pstart {\color{DodgerBlue3}“बुद्धेरपुरुषाश्रये”} पुरुषानाश्रयणे{\color{DodgerBlue3}“\edlabel{pvv.386-2}\footnote{\label{pvv.386-2}  २ साध्ये प्रतिज्ञायाः ।}बाधा”} । कैरित्याह । {\color{DodgerBlue3}“अभ्युपेतप्रत्यक्षप्रतीतानुमितैः समं”} एककालमेव बुद्धेरपुरुषाश्रयत्वस्याभ्युपेतेन पुरुष\edlabel{pvv.386-3}\footnote{\label{pvv.386-3}  ३ तीर्थस्य ।}गुणत्वाभ्युपगमेन प्रत्यक्षप्रतीतेन च पुरुषकार्य\edlabel{pvv.386-4}\footnote{\label{pvv.386-4}  ४ भावस्यैव कार्यत्वात् ।}त्वेन । तत्प्रयत्नका\edlabel{pvv.386-5}\footnote{\label{pvv.386-5}  ५ तद्भावभावित्वात् ।}  र्यत्वेन वा कादाचित्कत्वानुमितेन कार्यत्वेन च बाधा ।
	\pend
      

	  \pstart g. वर्ण्णानामानुपूर्व्वी वाक्यं स्यादित्याह ।
	\pend
      
	  \bigskip
	  \begingroup
	  \large
	
	    
	    \stanza[\smallbreak]
	\label{pv.3.268b}\edlabel{pv.3.268b}\flagstanza{\tiny\textenglish{...3.268b}}आनुपूर्व्याश्च वर्ण्णेभ्यो भेदः स्फोटेन चिन्तितः ॥ २६८ ॥\&[\smallbreak]


	
	  \endgroup
	
	  \bigskip
	  \begingroup
	  \large
	
	    
	    \stanza[\smallbreak]
	\label{pv.3.269a}\edlabel{pv.3.269a}\flagstanza{\tiny\textenglish{...3.269a}}कल्पनारोपिता सा स्यात् कथं वाऽपुरुषाश्रया ।\&[\smallbreak]


	
	  \endgroup
	

	  \pstart {\color{DodgerBlue3}“वर्ण्णेभ्य आनुपूर्व्या भेदश्च स्फोटेन”} चिन्तितेन एकत्वेन ह्यभिन्नस्य क्रमश इत्यादिना {\color{DodgerBlue3}“चिन्तितः”} । न हि वर्ण्णेभ्यो व्यतिरिक्ता आनुपूर्व्वी काचिदुपलभ्यते । अभेदपक्षे च सरो रस इत्यादौ प्रतिपत्तेर्भेदाभावप्रसङ्गः । प्रकारान्तरस्य चाभावः । तस्माद् वस्तुभूतानुपूर्व्याऽयोगात् {\color{DodgerBlue3}“कल्पना”}{\color{DodgerBlue3}“रोपिता”} सा {\color{DodgerBlue3}“स्यात्”} । तथा {\color{DodgerBlue3}“चापुरुषाश्रया कथमुच्यते”} ।
	\pend
      

	  \begin{center}%% label @type='head'
	\textbf{(ङ) निर्हेतुको विनाशः}
	\end{center}
	

	  \pstart कथं पुनरवगम्यते ध्वनिरवश्यमनित्य इत्याह ।
	\pend
      
	  \bigskip
	  \begingroup
	  \large
	
	    
	    \stanza[\smallbreak]
	\label{pv.3.269b}\edlabel{pv.3.269b}\flagstanza{\tiny\textenglish{...3.269b}}सत्तामात्रानुबन्धित्वात् नाशस्यानित्यता ध्वनेः ॥ २६९ ॥\&[\smallbreak]


	
	  \endgroup
	
	  \bigskip
	  \begingroup
	  \large
	
	    
	    \stanza[\smallbreak]
	\label{pv.3.270a}\edlabel{pv.3.270a}\flagstanza{\tiny\textenglish{...3.270a}}अविनाशात् स एवास्य विनाश इति चेत् कथम् ।&अन्योर्थोन्यस्य नाशोस्तु काष्ठं कस्मान्न दृश्यते ॥ २७० ॥\&[\smallbreak]


	
	  \endgroup
	

	  \pstart {\color{DodgerBlue3}“सत्तामात्रानुबन्धित्वान्नाशस्य ध्वनेरनित्यता”} । न खल्वसतामन्यस्मान्ना{\color{DodgerBlue3}“शोत्पत्तिः”} स्वहेतोरेव तु विनश्वरस्वभावतयोत्पन्ना भावा विनश्यन्ति । \edlabel{pvv.386-6}\footnote{\label{pvv.386-6}  ६ अभ्युपगम्याह ।} विनाशो हि क्रियमाणो भावाद् व्यतिरिक्तो वा भवेत् । अव्यतिरेकपक्षे भाव एव क्रियत \leavevmode\marginnote{\textenglish{387/s}} इति स्यात् । तच्चाशक्यक्रियमुत्पन्नत्वात् । व्यतिरेकेप्यग्नेः सकाशात् अर्थान्तरस्य विनाशाख्यस्योत्पत्तौ काष्ठस्य दर्शनम्भ{\color{DodgerBlue3}“वेदविनाशात्”} ।
	\pend
      

	  \pstart स एवाग्निजन्माऽर्थोऽस्य\edlabel{pvv.387-1}\footnote{\label{pvv.387-1}  १ काष्ठस्याभावोऽभूतत्वान्न दृश्यते ।} {\color{DodgerBlue3}“विनाशस्तेनादर्शनमिति चेत्\edlabel{pvv.387-2}\footnote{\label{pvv.387-2}  २ भवत्वग्निजार्थस्याभाव इति नाम तथापि ।} कथमन्योर्थोन्यस्य”} विनाशो युक्तः । एवं ह्यति\edlabel{pvv.387-3}\footnote{\label{pvv.387-3}  ३ सर्व्वे पदार्थाः काष्ठस्य विनाशः स्युः ।}प्रसङ्गः स्यात् । काष्ठेऽग्निकृतः स्वभावो\edlabel{pvv.387-4}\footnote{\label{pvv.387-4}  ४ ततो नातिप्रसङ्गः ।} विनाशो न सर्व्व इति चेत् । काष्ठविनाशयोः कः सम्बन्धः । नाश्रयाश्रयिभावो निषेत्स्यमानत्वात् । कार्यकारणभावश्चेत् अग्नेरपि स विनाशः स्यात् । तस्यापि कार्यत्वात् । तस्मान्न भावान्तरमर्थस्य नाशः (।) {\color{DodgerBlue3}“अस्तु वा नाशः काष्ठञ्चेद-”} प्रच्युतप्राचीनस्वभावं {\color{DodgerBlue3}“कस्मान्न दृश्यते”} ।
	\pend
      

	  \pstart ननु योसावर्थान्तरस्वभावो वह्निकृतः\edlabel{pvv.387-5}\footnote{\label{pvv.387-5}  ५ तेन परिग्रहात् स्वीकारात् काष्ठं न दृश्यते ।} स काष्ठस्य नाशो नाशरूपतया प्रतीतेः । विनाशश्चाभावो यश्चाभावः स काष्ठविरोधिरूप एव क्रियते । न चायमर्थान्तरत्वाद् घटवद् विरोधिरूपता कुर्त्तुमशक्यः । न हि घटवदर्थान्तरत्वाद् धूमोऽग्निकार्यो न भवति । तस्माद् यथान्तररूपोपि धूमोऽग्निना क्रियते तथा विरोधिरूपोपि विनाशीक्रियेत । ययोश्च परस्परपरिहारेण विरोधस्तयोरेकभाव एवापरस्परस्यादर्शनमिति कथमग्निकृतस्यार्थान्तरस्य विनाशसंज्ञितस्य विरोधिनो भावे काष्ठस्य दर्शनमुच्यते ।
	\pend
      

	  \pstart अत्रोच्यते । योसावर्थान्तरस्वभावो नाशस्तेन सह काष्ठस्य को विरोधः ।\leavevmode\marginnote{\textenglish{77a/MA}} यदि सहानवस्थानलक्षणः सम्भाव्यत एव । तद्भावयोर्निवर्त्त्यनिवर्त्तकभावदर्शनादग्निशीतयोरिव । किन्तु भावान्तरस्य यदि काष्ठविनाशकत्वं । स च नाशः किमर्थान्तरनिवृत्तिः । अर्थान्तरत्वे तुल्यः प्रसङ्गः ।\edlabel{pvv.387-6}\footnote{\label{pvv.387-6}  ६ तदुत्पादेपि काष्ठं तथैव दृश्यते ।} निवृत्तिश्च निःस्वभावा न तत्र हेतुव्यापार इति वक्ष्यते । परस्परपरिहारस्थितिलक्षणश्चेद् विरोधः । एवमप्यतिप्रसङ्गः । यथा काष्ठाद् भिन्नमभिमतपदार्थान्तरं तथान्यदपि घटादिकमिति तदपि काष्ठस्य विनाशः स्यात् ।
	\pend
      

	  \pstart ननु य एव काष्ठस्य नाशरूपतया विरोधिरूपतया च स्वहेतुभिः क्रियते स एव नाशो न तु यः कश्चिदर्थ इति कथमतिप्रसक्तिः ।
	\pend
      \leavevmode\marginnote{\textenglish{388/s}}

	  \pstart तदप्ययुक्तं । परस्परपरिहारेण हि विरोधी नाश इष्टं । स च परस्परपरिहारोऽभिमतपदार्थवद् घटादीनामपीति न विशेषः । अथ य एव काष्ठनिवृत्तिरूपः स एव तन्नाशो नान्यः । किमिदं निवृत्तिरूपत्वं काष्ठादन्यत्वं निवृत्तिमात्रात्मकत्वम्वा । अन्यत्वञ्चेत् तदितरस्यापि समानं । निवृत्तिमात्रात्मकत्वञ्च भावान्तरस्यायुक्तं स्वभावाविशेषवत्वात् । यदुत्पत्तौ यन्निवृत्तिः स विरोधी नाशश्चेति चेत् । अन्या तर्हि विनाशान्निवृत्तिः । तत्र च समानः सर्व्व एव प्रसङ्गः । तस्मात् काष्ठं स्वरसनिरोधितया निवर्त्ततेऽग्निकाष्ठादिसामग्रयास्त्वङ्गारादिकं जायत इति युक्तं । अर्थान्तरस्वभावे तु सति नाशे काष्ठं कस्मान्न दृश्यत इत्यनिवार्यः प्रसङ्गः ।
	\pend
      
	  \bigskip
	  \begingroup
	  \large
	
	    
	    \stanza[\smallbreak]
	\label{pv.3.271a}\edlabel{pv.3.271a}\flagstanza{\tiny\textenglish{...3.271a}}तत्परिग्रहतश्चेन्न तेनानावरणं यतः ।&विनाशस्य विनाशित्वं ;\&[\smallbreak]


	
	  \endgroup
	

	  \pstart {\color{DodgerBlue3}“तेन”} विनाशाख्येन भावान्तरेण {\color{DodgerBlue3}“परिग्रहतो”}वष्टब्धत्वात् काष्ठादर्शनमिति {\color{DodgerBlue3}“चेत् । न तेन”} भावान्तरेण काष्ठस्या{\color{DodgerBlue3}“नावरणं यत”}स्ततः काष्ठस्यापरिग्रहः । न हि नाशो वस्त्वावरणम{\color{DodgerBlue3}“विनाशित्वप्र”}सङ्गात् । न च पूर्व्वापरैकस्वभावस्यवरणं युक्तमित्युक्तं । किञ्च (।)
	\pend
      
	  \bigskip
	  \begingroup
	  \large
	
	    
	    \stanza[\smallbreak]
	\label{pv.3.271b}\edlabel{pv.3.271b}\flagstanza{\tiny\textenglish{...3.271b}}स्यादुत्पत्तेस्ततः पुनः ॥ २७१ ॥\&[\smallbreak]


	
	  \endgroup
	
	  \bigskip
	  \begingroup
	  \large
	
	    
	    \stanza[\smallbreak]
	\label{pv.3.272a}\edlabel{pv.3.272a}\flagstanza{\tiny\textenglish{...3.272a}}काष्ठस्य दर्शनं ;\&[\smallbreak]


	
	  \endgroup
	

	  \pstart भावान्तरभूतस्य नाशस्य यद्युत्पत्तिरिष्यते तदोत्पत्तिलिङ्गाद् विनाशित्वं नाश\edlabel{pvv.388-1}\footnote{\label{pvv.388-1}  १ यदुत्पत्तिमत् तद्विनश्वरं ।}स्य स्यात् घटादिवत् । ततः काष्ठनाशस्य नाशात् पुनः काष्ठदर्शनं भवेत् ।
	\pend
      
	  \bigskip
	  \begingroup
	  \large
	
	    
	    \stanza[\smallbreak]
	\label{pv.3.272b}\edlabel{pv.3.272b}\flagstanza{\tiny\textenglish{...3.272b}}हन्तृघाते चैत्रापुनर्भवः ।&यथात्राप्येवमिति चेत् हन्तुर्न्नामरणत्वतः ॥ २७२ ॥\&[\smallbreak]


	
	  \endgroup
	

	  \pstart ननु {\color{DodgerBlue3}“चैत्रस्य”} हन्तुर्व्याघाते कृते चैत्रस्य {\color{DodgerBlue3}“पुनर्भावो”} नास्ति {\color{DodgerBlue3}“यथा”}, तथा{\color{DodgerBlue3}“त्रापि”} काष्ठनाशनिवृत्त्या न काष्ठपुनर्भाव {\color{DodgerBlue3}“इति चेत् । न हन्तुरमरणत्वतः ।”} न हि हन्ता मरणं चैत्रस्य । किन्त्वन्य एवेन्द्रियायुर्निरोधः । येन हन्तृमरणे चैत्रपुनरुज्जीवनप्रसङ्गः । यदि त्विन्द्रियादिनिरोधनिवृत्तिः स्यात् स्यादेवोज्जीवनं तच्च त्वन्मते प्राप्तं । निरोधस्योत्पत्तिभावयोरिष्टः । (२७२)
	\pend
      \label{div_pvv.3.273}\edlabel{div_pvv.3.273}
	  
	% new div opening: depth here is 2
	\leavevmode\marginnote{\textenglish{389/s}}
	  \bigskip
	  \begingroup
	  \large
	
	    
	    \stanza[\smallbreak]
	\label{pv.3.273}\edlabel{pv.3.273}\flagstanza{\tiny\textenglish{....3.273}}अनन्यत्वे विनाशस्य स्यान्नाशः काष्ठमेव तु ।&तस्यासत्वादहेतुत्वं नातोन्या विद्येते गतिः ॥ २७३ ॥\&[\smallbreak]


	
	  \endgroup
	

	  \pstart अथ न भावाद्भिन्नो नाशः किन्तर्ह्यभिन्नः । अनन्यत्वे विनाशस्य नाशः काष्ठमेव तु स्यात् (।) तस्य काष्ठ\edlabel{pvv.389-1}\footnote{\label{pvv.389-1}  १ अहेतोरुत्पन्नस्य न वहन्यादिभिः किञ्चित्कर्त्तव्यमिति नाशोऽस्तु ।}स्याग्निसन्निधानात् (।) प्रागेव सत्वादहेतुत्वं  नाशकाभिमतस्य । नातो भेदाभेदप्रकारादन्या गतिरुत्पत्तिमतोस्ति । द्विधापि च नाशहेत्वयोगः \edlabel{pvv.389-2}\footnote{\label{pvv.389-2}  २ प्रध्वंसाभावं नाशं गृहीत्वा परतोद्यमाशङ्कतेऽहेतुहिते क्षणिकवादिनः नित्यं भवेत् भावकालेपीति सहभावः स्यात् ।}। (२७३)
	\pend
      \label{div_pvv.3.274}\edlabel{div_pvv.3.274}
	  
	% new div opening: depth here is 2
	
	  \bigskip
	  \begingroup
	  \large
	
	    
	    \stanza[\smallbreak]
	\label{pv.3.274}\edlabel{pv.3.274}\flagstanza{\tiny\textenglish{....3.274}}अहेतुत्वेपि नाशस्य नित्यत्वाद् भावनाशयोः ।&सहभावप्रसङ्गश्चेदसतो नित्यता कुतः ॥ २७४ ॥\&[\smallbreak]


	
	  \endgroup
	

	  \pstart {\color{DodgerBlue3}“अहेतुत्वेपि नाशस्य नित्यत्वादा”}काशादिवत् {\color{DodgerBlue3}“भावनाशयो”}रन्योन्याभावस्वभावयोः {\color{DodgerBlue3}“सहभावप्रसङ्गश्चेत्”} (।) ननु नाशस्यासतो नीरूपत्वान्नित्यता कुतः । वस्तु हि नित्यमनित्यम्वा स्यात् । यत्तु न किञ्चित् तत्कथमु\edlabel{pvv.389-3}\footnote{\label{pvv.389-3}  ३ विनाशस्य भावनिवृत्तिलक्षणत्वात् ।}च्यतां (। २७४)
	\pend
      \label{div_pvv.3.275}\edlabel{div_pvv.3.275}
	  
	% new div opening: depth here is 2
	
	  \bigskip
	  \begingroup
	  \large
	
	    
	    \stanza[\smallbreak]
	\label{pv.3.275}\edlabel{pv.3.275}\flagstanza{\tiny\textenglish{....3.275}}असत्वेऽभावनाशित्वप्रसङ्गोपि न युज्यते ।&नाशेन यस्माद् भावस्य न विनाशनमिष्यते ॥ २७५ ॥\&[\smallbreak]


	
	  \endgroup
	

	  \pstart अतश्चानित्यत्वा{\color{DodgerBlue3}“दसत्वे”} नाशस्याभाव{\color{DodgerBlue3}“नाशित्व”}स्य  {\color{DodgerBlue3}“प्रसङ्गो\edlabel{pvv.389-4}\footnote{\label{pvv.389-4}  ४ बौद्धस्य यदि नाशोऽसन्निष्टस्तदा भावस्य नाशित्वं न स्यात् कथमसन् विनाशो भावं नाशयेदिति असतो व्यापारायोगात् ।}पि न युज्यते यस्माद् भावस्य”} काष्ठादे{\color{DodgerBlue3}“र्नाशेन”} हेतुना {\color{DodgerBlue3}“नाशनं नेष्यते”} । यदि हि नाशेन नाशः क्रियते इतीष्यते तदा नाशाभावे वस्तुनाशो न स्यात् । किन्तु \edlabel{pvv.389-5}\footnote{\label{pvv.389-5}  ५ कथन्तर्हि भावो नष्ट इति व्यपदेश इत्याह ।}स्वहेतुत एव भावा एकक्षणस्थितिधर्माण उत्पन्ना द्वितीये क्षणे न भवन्ति न तु नाम कश्चिद् भवति । यस्य नित्यत्वानित्य(त्व)योर्दोषावकाशः । (२७५)
	\pend
      \label{div_pvv.3.276}\edlabel{div_pvv.3.276}
	  
	% new div opening: depth here is 2
	

	  \pstart a. कथन्तर्हीदानीमहेतुको नाशो भवतीत्युच्यत \edlabel{pvv.389-6}\footnote{\label{pvv.389-6}  ६ भावस्य नाश इति व्यतिरेकः कथं यस्य स्वभाव एव नास्ति तस्य किमहेतुकः सहेतुको वेति चिन्तया ।} इत्याह (।)
	\pend
      
	  \bigskip
	  \begingroup
	  \large
	
	    
	    \stanza[\smallbreak]
	\label{pv.3.276}\edlabel{pv.3.276}\flagstanza{\tiny\textenglish{....3.276}}नश्यन् भावः परापेक्षो न तस्य ज्ञापनाय सा ।&अवस्थाऽहेतुरुक्तास्या भेदमारोप्य चेतसा ॥ २७६ ॥\&[\smallbreak]


	
	  \endgroup
	\leavevmode\marginnote{\textenglish{390/s}}

	  \pstart \leavevmode\marginnote{\textenglish{77b/MA}}{\color{DodgerBlue3}“नश्यन् भाव”} एव क्षणस्थितिधर्मतया स्वहेतोरुत्पत्तेर्द्वितीये क्षणेऽभवन्न {\color{DodgerBlue3}“परापेक्षः”} कारणान्तरनिरपेक्ष इति । {\color{DodgerBlue3}“तस्य”} कारणान्तरानपेक्षनाशित्वस्य {\color{DodgerBlue3}“ज्ञापनाय सा”} भावानाम{\color{DodgerBlue3}“वस्थाऽहेतुरुक्ता”} । अहेतुको विनाशो भवतीत्यादि वचसैवा{\color{DodgerBlue3}“स्या”} अवस्थाया धर्मिणः सकाशात्\edlabel{pvv.390-1}\footnote{\label{pvv.390-1}  १ अर्थान्तरमिव नाशं ।} भेदान्तरप्रतिक्षेपेण {\color{DodgerBlue3}“चेतसा”} विकल्पकेन\edlabel{pvv.390-2}\footnote{\label{pvv.390-2}  २ भावस्य नाशः किमन्यस्मान्न वेति जिज्ञासायां ।} {\color{DodgerBlue3}“भेदमारोप्य”} न तु वस्तुतो नाशो नाम कश्चित् भावाद् भिन्नस्वभावो भवति । (२७६)
	\pend
      \label{div_pvv.3.277}\edlabel{div_pvv.3.277}
	  
	% new div opening: depth here is 2
	
	  \bigskip
	  \begingroup
	  \large
	
	    
	    \stanza[\smallbreak]
	\label{pv.3.277}\edlabel{pv.3.277}\flagstanza{\tiny\textenglish{....3.277}}स्वतोपि भावेऽभावस्य विकल्पश्चेदयं समः ।&न तस्य किञ्चिद् भवति न भवत्येव केवलम् ॥ २७७ ॥\&[\smallbreak]


	
	  \endgroup
	

	  \pstart \edlabel{pvv.390-3}\footnote{\label{pvv.390-3}  ३ ननु यस्याहेतुको नाशो बौद्धस्य तन्मते ।}स्वतोप्य{\color{DodgerBlue3}“भाव\edlabel{pvv.390-4}\footnote{\label{pvv.390-4}  ४ नाशस्य ।}स्य भावेऽय”}न्तत्वान्यत्व{\color{DodgerBlue3}“विकल्पः समश्चेत्”} । तथा हि यदि भावाद् भिन्नोप्यभावो ज्ञातो भावः किमिति न दृश्यतेऽभेद तु भाव एव नाश इति कथं नष्टः । अत्राह (।) {\color{DodgerBlue3}“न तस्य”} भावस्य {\color{DodgerBlue3}“किञ्चिद्\edlabel{pvv.390-5}\footnote{\label{pvv.390-5}  ५ व्यतिरिक्तमव्यतिरिक्तम्वा ।}”} विनाशोऽन्यो\edlabel{pvv.390-6}\footnote{\label{pvv.390-6}  ६ स्थित्यन्यथात्वादिर्द्धर्मः ।}वा {\color{DodgerBlue3}“भवति”} । किन्तर्हि स {\color{DodgerBlue3}“एव केवलं”} न भवति । व्यवहर्त्तव्यैकरूपत्वात् तस्य । तत्र च भेदाभेदविकल्पानवतारः । (२७७)
	\pend
      \label{div_pvv.3.278}\edlabel{div_pvv.3.278}
	  
	% new div opening: depth here is 2
	
	  \bigskip
	  \begingroup
	  \large
	
	    
	    \stanza[\smallbreak]
	\label{pv.3.278a}\edlabel{pv.3.278a}\flagstanza{\tiny\textenglish{...3.278a}}भावे ह्येष विकल्पः स्याद् विधेर्वस्त्वनुरोधतः ।\&[\smallbreak]


	
	  \endgroup
	

	  \pstart हि यस्मात् भावे विकल्प एष भेदाभेदात्मकः स्यात् विधेर्व्वस्त्वनुरोधतः । {\color{DodgerBlue3}“नाशस्तु”} प्रसज्यप्रतिषेधरूपो निःस्वभावत्वात् भेदाभेदविकल्पाक्षमः । यदि च प्रसज्यप्रतिषेधेपि वस्त्वन्तरविधिस्तदा पर्युदासान्न भिद्येतोभयत्रापि विधेः प्राधान्यात् । पर्युदासो वा न सिध्येत् एकनिवृत्तावपरविधाने स स्यात् निवृत्त्यसिद्धौ तु कथं युक्तः\edlabel{pvv.390-7}\footnote{\label{pvv.390-7}  ७ इति पर्युदासोपि न स्यात् ।} ।
	\pend
      
	  \bigskip
	  \begingroup
	  \large
	
	    
	    \stanza[\smallbreak]
	\label{pv.3.278b}\edlabel{pv.3.278b}\flagstanza{\tiny\textenglish{...3.278b}}न भावो भवतीत्युक्तमभावो भवतीति न ॥ २७८ ॥\&[\smallbreak]


	
	  \endgroup
	

	  \pstart यदा च भावनिवृत्तिर्विनाशार्थः तदा{\color{DodgerBlue3}“ऽभावो भवतीत्यादि”} वाक्येन {\color{DodgerBlue3}“भावो न भवती”}त्युक्तं\edlabel{pvv.390-8}\footnote{\label{pvv.390-8}  ८ एवं हि भावो निवर्त्तितो भवति यदि किञ्चिन्न विधीयते ।}। अभावस्य भावायोगात् । अतश्च हेतुरपि नाशस्य न कश्चित् । (२७८)
	\pend
      \label{div_pvv.3.279}\edlabel{div_pvv.3.279}
	  
	% new div opening: depth here is 2
	
	  \bigskip
	  \begingroup
	  \large
	
	    
	    \stanza[\smallbreak]
	\label{pv.3.279}\edlabel{pv.3.279}\flagstanza{\tiny\textenglish{....3.279}}b. अपेक्षेत परः कार्यं यदि विद्येत किञ्चन ।&यदकिञ्चित्करं वस्तु किं केनचिदपेक्ष्यते ॥ २७९ ॥\&[\smallbreak]


	
	  \endgroup
	\leavevmode\marginnote{\textenglish{391/s}}

	  \pstart यस्मात् परः कारणाभिमतोऽपेक्ष्यत\edlabel{pvv.391-1}\footnote{\label{pvv.391-1}  १ विनश्यता भावेनापेक्षेत यदि भावस्य कर्तव्यं स्यात् ।}यदि किञ्चन कार्यं विद्येत । अन्यथा यदकिञ्चित्करं वस्तु तत् केनचित् किमपेक्ष्यते । (२७९)
	\pend
      \label{div_pvv.3.280}\edlabel{div_pvv.3.280}
	  
	% new div opening: depth here is 2
	
	  \bigskip
	  \begingroup
	  \large
	
	    
	    \stanza[\smallbreak]
	\label{pv.3.280}\edlabel{pv.3.280}\flagstanza{\tiny\textenglish{....3.280}}एतेनाहेतुकत्वेपि ह्यभूत्वा नाशभावतः ।&सत्तानाशित्वदोषस्य प्रत्याख्यातं प्रसञ्चनम् ॥ २८० ॥\&[\smallbreak]


	
	  \endgroup
	

	  \pstart {\color{DodgerBlue3}“\edlabel{pvv.391-2}\footnote{\label{pvv.391-2}  २ परता हेतुकेपि नाशेऽभूत्वा भावात् सत्ताऽनित्यत्वञ्च दुर्न्निवारं अभूत्वा भवन्नहेतुकं इत्यपि विरुद्धं कादाचित्कस्य सहेतुत्वात् ।}एतेन”} नाशस्य निःस्वभावत्वकथनेना{\color{DodgerBlue3}“हेतुकत्वेपि\edlabel{pvv.391-3}\footnote{\label{pvv.391-3}  ३ नाशस्य स्वीकृते ।}”} सत्य{\color{DodgerBlue3}“भूत्वा नाशस्य भावतः\edlabel{pvv.391-4}\footnote{\label{pvv.391-4}  ४ स नाशो अभूत्वा भवतीतु हेतोः सत्ता च नाशित्वञ्चेति ।} । सत्तानाशित्वदोषस्य प्रसञ्जनं\edlabel{pvv.391-5}\footnote{\label{pvv.391-5}  ५ यत्परेण ।}”} घटादाविव\edlabel{pvv.391-6}\footnote{\label{pvv.391-6}  ६ तदेतेनैव ।} प्रत्याख्यातं\edlabel{pvv.391-7}\footnote{\label{pvv.391-7}  ७ नाशे कस्यापि भावानभ्युप (ग) मात् ।}। न ह्यभावो नाम कश्चिद् भवति यस्याभूत्वा भावात् सत्त्वं नाशित्वम्वा स्यात् ॥ (२८०)
	\pend
      \label{div_pvv.3.281}\edlabel{div_pvv.3.281}
	  
	% new div opening: depth here is 2
	
	  \bigskip
	  \begingroup
	  \large
	
	    
	    \stanza[\smallbreak]
	\label{pv.3.281}\edlabel{pv.3.281}\flagstanza{\tiny\textenglish{....3.281}}c. यथा केषाञ्चिदेवेष्टः प्रतिघो जन्मिनां तथा ।&नाश(ः)स्वभावो भावानां नानुत्पत्तिमतां यदि ॥ २८१ ॥\&[\smallbreak]


	
	  \endgroup
	

	  \pstart ननु {\color{DodgerBlue3}“यथा जन्मिनां”} बुद्ध्यादीनां मध्ये {\color{DodgerBlue3}“केषाञ्चिदेव”} घटादीनां प्रतिघः स्वदेशे वस्त्वन्तरोत्पत्तिव्याघात {\color{DodgerBlue3}“इष्टः”} । न बुद्ध्यादीनां । तथा {\color{DodgerBlue3}“यदि”} सतामुत्पत्तिमतामेव {\color{DodgerBlue3}“नाशः स्वभावः”} स्यात् {\color{DodgerBlue3}“नानुत्पत्तिमतां”} शब्दाकाशादीनां तदा\edlabel{pvv.391-8}\footnote{\label{pvv.391-8}  ८ एतेन सत्वञ्चानश्वरञ्चेति व्यभिचारः ।} कथमुक्तं सत्ता\edlabel{pvv.391-9}\footnote{\label{pvv.391-9}  ९ ननु नाशस्वभावो भावानां नानुत्पत्तिमतां यदीति प्रकृतं चोद्यं न चात्र नानित्येत्यादिः परिहारः । नाकाशादेः स्वहेतुतो नश्वरताऽनुत्पत्तिमत्वात् ॥ अत्र सिद्धान्त्येवम्मन्यते । यथा सत्वं व्यभिचार्युक्त(ः)तथा कृतकोपि कश्चिन्नश्वरः कश्चिन्नेत्याशङ्क्यते । तेनादौ कृतकव्यभिचारं परिहरति ॥ सत्वेप्युच्यते । ये क्वचित् कदाचित् केनचिदज्ञाता ज्ञायन्ते पुनर्न ज्ञायन्ते तेषां सत्तानुबन्धी नाश इति व्याप्तिः । अन्यथा नित्त्यत्वात् सदा ज्ञानजननप्रसङ्गः । सहकार्यपेक्षापि न । नित्त्यत्वात् पूर्व्वमेव निष्पत्तेर्ज्ञानज (न) कस्य ।}मात्रानुबन्धित्वात् नाशस्यानित्यता ध्वनेरिति । (२८१)
	\pend
      \label{div_pvv.3.282}\edlabel{div_pvv.3.282}
	  
	% new div opening: depth here is 2
	

	  \pstart अत्राह (।)
	\pend
      
	  \bigskip
	  \begingroup
	  \large
	
	    
	    \stanza[\smallbreak]
	\label{pv.3.282}\edlabel{pv.3.282}\flagstanza{\tiny\textenglish{....3.282}}स्वभावनियमाद्धेतोः स्वभावनियमः फले ।&नानित्ये रूपभेदोस्ति भेदकानामभावतः ॥ २८२ ॥\&[\smallbreak]


	
	  \endgroup
	\leavevmode\marginnote{\textenglish{392/s}}

	  \pstart {\color{DodgerBlue3}“हेतोः स्वभाव”}स्य विशिष्ट\edlabel{pvv.392-1}\footnote{\label{pvv.392-1}  १ अयं सप्रतिघस्य जनकोऽयं नेति ।} कार्योत्पादनयोग्यताया {\color{DodgerBlue3}“नियमात् । फले”} कार्ये {\color{DodgerBlue3}“स्वभावस्य”} प्रतिघाप्रतिघादे{\color{DodgerBlue3}“र्नियमः”} । ततो नित्यत्वाभिमतानाञ्च हेतुमन्तरेण स्वभावनियमायोगात् हेतुरेष्टव्यः । कृतके {\color{DodgerBlue3}“पुनरनित्ये”} वा भावे {\color{DodgerBlue3}“रूप”}स्य नश्वरानश्वरस्य {\color{DodgerBlue3}“भेदो नास्ति”} । कस्मादित्याह । नित्यानित्यस्वभावतया कृतकस्य {\color{DodgerBlue3}“भेदकानां”} हेतूनाम{\color{DodgerBlue3}“भावतः”} । न हि कश्चिदेव हेतुरनित्यं जनयति\edlabel{pvv.392-2}\footnote{\label{pvv.392-2}  २ सर्व्वेषां विनश्वरस्वभावस्यैव जननात् ।}नापरः सर्व्वेषां कृतकानामर्थक्रियाकारित्वात् । तस्य चानित्यताव्याप्तेः । (२८२)
	\pend
      \label{div_pvv.3.283}\edlabel{div_pvv.3.283}
	  
	% new div opening: depth here is 2
	
	  \bigskip
	  \begingroup
	  \large
	
	    
	    \stanza[\smallbreak]
	\label{pv.3.283}\edlabel{pv.3.283}\flagstanza{\tiny\textenglish{....3.283}}d. प्रत्याख्येयाऽत एवैषां सम्बन्धस्यापि नित्यता ।&सम्बन्धदोषैः प्रागुक्तैः शब्दशक्तिश्च दूषिता ॥ २८३ ॥\&[\smallbreak]


	
	  \endgroup
	

	  \pstart {\color{DodgerBlue3}“अतः\edlabel{pvv.392-3}\footnote{\label{pvv.392-3}  ३ अनन्तरोक्तात् वस्तुमात्रानुबन्धात् ।}”} सर्व्वभावक्षणिकत्वसाधकात् प्रमाणादे{\color{DodgerBlue3}“वैषां”} शब्दानामर्थ{\color{DodgerBlue3}“सम्बन्धस्यापि नित्यता प्रत्याख्येया”} । सम्बन्धस्य वस्तुत्वे क्षणिकत्वात् । या च शब्दशक्ति\leavevmode\marginnote{\textenglish{78a/MA}}र्योग्यताख्याऽर्थप्रतिपत्त्याश्रयो वर्ण्ण्यते\edlabel{pvv.392-4}\footnote{\label{pvv.392-4}  ४ जैमिनीयैः ।} सा\edlabel{pvv.392-5}\footnote{\label{pvv.392-5}  ५ व्यतिरिक्त एव नास्तीत्यतोधुना ।} च शब्दाद् व्यतिरिक्तैवेति तद्व\edlabel{pvv.392-6}\footnote{\label{pvv.392-6}  ६ शब्दवत्}दनित्या । अथ भिन्ना\edlabel{pvv.392-7}\footnote{\label{pvv.392-7}  ७ शब्दशक्तिः सम्बन्ध ।} तादृशी च {\color{DodgerBlue3}“सम्बन्धदोषैः”} “सम्बन्धिनामनित्यत्वान्न सम्बन्धे\edlabel{pvv.392-8}\footnote{\label{pvv.392-8}  ८ अव्यतिरेकात् ।}अस्ति नित्यते” \href{http://http://sarit.indology.info/?cref=pv.3.231}{(३।२३१)}त्यादिना {\color{DodgerBlue3}“प्रागुक्तैर्दूषिते”}ति न पुनरुच्यते ।\edlabel{pvv.392-9}\footnote{\label{pvv.392-9}  ९ तदेवं ।}(२८३)
	\pend
      \label{div_pvv.3.284}\edlabel{div_pvv.3.284}
	  
	% new div opening: depth here is 2
	

	  \begin{center}%% label @type='head'
	\textbf{(२) a नव्यमीमांसक(भादृ)मतनिरासः}
	\end{center}
	

	  \begin{center}%% label @type='head'
	\textbf{क. अपौरुषेयत्वेऽपि दोषाः}
	\end{center}
	

	  \begin{center}%% label @type='head'
	\textbf{(क) अपौरुषेयत्वान्न याथार्थ्यसिद्धिः}
	\end{center}
	

	  \pstart {\color{DodgerBlue3}“न”} तावदपौरुषेयं वचनमस्तीत्युक्तं । भवतु वा । तथापि (।)
	\pend
      
	  \bigskip
	  \begingroup
	  \large
	
	    
	    \stanza[\smallbreak]
	\label{pv.3.284}\edlabel{pv.3.284}\flagstanza{\tiny\textenglish{....3.284}}नाऽपौरुषेयमित्येव यथार्थज्ञानसाधनम् ।&दृष्टोऽन्यथापि वह्न्यादिरदुष्टः पुरुषागसा ॥ २८४ ॥\&[\smallbreak]


	
	  \endgroup
	\leavevmode\marginnote{\textenglish{392/s}}

	  \pstart {\color{DodgerBlue3}“नापौरुषेयमित्येव वचनं यथाऽर्थ”}स्याविसम्वादिनो {\color{DodgerBlue3}“ज्ञान”}स्य {\color{DodgerBlue3}“साधनं”}\edlabel{pvv.393-1}\footnote{\label{pvv.393-1}  १ शक्यनिश्चयः ।} यस्मात् पुरुषस्या{\color{DodgerBlue3}“गसा”} दोषेणा{\color{DodgerBlue3}“दुष्टोपि वह्न्यादि\edlabel{pvv.393-2}\footnote{\label{pvv.393-2}  २ ज्योत्स्नादि ।}”}र्नीलोत्पला\edlabel{pvv.393-3}\footnote{\label{pvv.393-3}  ३ न हि रागादि पुंदोषसंस्कारादेवार्थेषु ज्ञाय्येषु ज्ञापकानां शब्दानां ज्ञानभ्रमः प्रकृत्यापि मिथ्याज्ञानजननस्य संभाव्यत्वाद् वने दवो रात्रौ नीलोत्पलो रक्तप्रतिभासि ज्ञानहेतु ज्योत्स्ना पीते शुक्लज्ञानहेतुरिति विना पुरुषञ्च दृष्टेर्व्वनदवादेः;} {\color{DodgerBlue3}“३दावन्यथा अपरार्थज्ञानहेतु”}र्दृष्टः ।\edlabel{pvv.393-4}\footnote{\label{pvv.393-4}  ४ वह्न्यादेः सहकारि बलादस्त्वन्यथात्वं नित्ये तु नैवमिति चेन्न शब्देपि सङ्केतसहकार्यपेक्षत्वादनपेक्ष्य ज्ञानजननस्वभावत्वे सङ्केतकरणव्यापारम्विनापि वेदादर्थज्ञानं सर्व्वस्य सदा स्यान्न चैवं तन्न स्थितस्वभावतेति मिथ्याज्ञानहेतुत्वं तदवस्थं ।}(२८४)
	\pend
      \label{div_pvv.3.285}\edlabel{div_pvv.3.285}
	  
	% new div opening: depth here is 2
	

	  \begin{center}%% label @type='head'
	\textbf{(ख) अकृतकत्वे ज्ञानहेतुत्वाभावः}
	\end{center}
	

	  \pstart किञ्च (।)
	\pend
      
	  \bigskip
	  \begingroup
	  \large
	
	    
	    \stanza[\smallbreak]
	\label{pv.3.285}\edlabel{pv.3.285}\flagstanza{\tiny\textenglish{....3.285}}न ज्ञानहेतुतैव स्यात् तस्मिन्नकृतके मते ।&नित्येभ्यो वस्तुसामर्थ्यात् न च जन्मास्ति कस्यचित् ॥ २८५ ॥\&[\smallbreak]


	
	  \endgroup
	

	  \pstart {\color{DodgerBlue3}“तस्मिन्”} शब्देऽ{\color{DodgerBlue3}“कृतके मते\edlabel{pvv.393-5}\footnote{\label{pvv.393-5}  ५ इष्टे सति ।} ज्ञानहेतुतैव न स्यात् न”} हि {\color{DodgerBlue3}“नित्येभ्यो वस्तु\edlabel{pvv.393-6}\footnote{\label{pvv.393-6}  ६ प्रतीत्य जन्मकाले यत् तदजनकं रूपं तत्स्थस्याजनकत्वात् ।}सामर्थ्यात्”} कार्यजन्माशक्तेः {\color{DodgerBlue3}“कस्यचित्”} ज्ञानस्यान्यस्य वा\edlabel{pvv.393-7}\footnote{\label{pvv.393-7}  ७ एतत्परिहाराय नित्यं स्वकार्यजननञ्च मिथ्याऽदर्शनात्} {\color{DodgerBlue3}“जन्मास्ति”} (।) क्रमाक्रमयोर्व्यापकयोर्नित्यान्निवृत्तेः । तद्व्याप्यस्य सामर्थ्यस्याभावात्\edlabel{pvv.393-8}\footnote{\label{pvv.393-8}  ८ नित्याकाशादिभ्यो बुद्धयो भवन्तीत्यपि मृषा न ता (ः) तद्भावभाविन्यः क्रमयौगपद्यार्थक्रियाविरोधात् ।}। (२८५)
	\pend
      \label{div_pvv.3.286}\edlabel{div_pvv.3.286}
	  
	% new div opening: depth here is 2
	

	  \begin{center}%% label @type='head'
	\textbf{(ग) शब्दे समारोपितगोचरा बुद्धयः}
	\end{center}
	

	  \pstart तस्मात् (।)
	\pend
      
	  \bigskip
	  \begingroup
	  \large
	
	    
	    \stanza[\smallbreak]
	\label{pv.3.286}\edlabel{pv.3.286}\flagstanza{\tiny\textenglish{....3.286}}विकल्पवासनोद्भूताः समारोपितगोचराः ।&जायन्ते बुद्धयस्तत्र केवलं नार्थगोचराः ॥ २८६ ॥\&[\smallbreak]


	
	  \endgroup
	

	  \pstart तत्रानित्यात्मन्युच्चरिते शब्देऽनाद्यन्तेन {\color{DodgerBlue3}“विकल्पने”} चित्तसन्ततावारोपिताया {\color{DodgerBlue3}“वासनाया”} विकल्पिका {\color{DodgerBlue3}“बुद्धयो जायन्ते\edlabel{pvv.393-9}\footnote{\label{pvv.393-9}  ९ स्वागमसंस्कारकल्पनाया बाह्यत्वेन ततः ।} समारोपित\edlabel{pvv.393-10}\footnote{\label{pvv.393-10}  १० आकाशाद्याकारः ।}गोचराः”} कल्पितार्थ\leavevmode\marginnote{\textenglish{394/s}} {\color{DodgerBlue3}“विषया नार्थगोचरा”} न स्व\edlabel{pvv.394-1}\footnote{\label{pvv.394-1}  १ आकाशादि ।}लक्षणविषयाः । तथार्थत्वे शब्दप्रमाणान्तरवैफल्यस्योक्तेः ।\edlabel{pvv.394-2}\footnote{\label{pvv.394-2}  २ सत्यार्थं वेदवाक्यमकृतकत्वादित्यत्रान्वयाभावात् व्यतिरेकिप्रयोगमाह ।}(२८६)
	\pend
      \label{div_pvv.3.287_3.288}\edlabel{div_pvv.3.287_3.288}
	  
	% new div opening: depth here is 2
	

	  \begin{center}%% label @type='head'
	\textbf{ख. कृतकत्वेऽपि न दोषः}
	\end{center}
	

	  \begin{center}%% label @type='head'
	\textbf{(क) कृतकत्वान्न मिथ्यात्वनियमः}
	\end{center}
	

	  \pstart ननु (।)
	\pend
      
	  \bigskip
	  \begingroup
	  \large
	
	    
	    \stanza[\smallbreak]
	\label{pv.3.287}\edlabel{pv.3.287}\flagstanza{\tiny\textenglish{....3.287}}मिथ्यात्वं कृतकेष्वेव दृष्टमित्यकृतं वचः ।&सत्यार्थं व्यतिरेकस्य विरोधिव्यापनाद् यदि ॥ २८७ ॥\&[\smallbreak]


	
	  \endgroup
	
	  \bigskip
	  \begingroup
	  \large
	
	    
	    \stanza[\smallbreak]
	\label{pv.3.288}\edlabel{pv.3.288}\flagstanza{\tiny\textenglish{....3.288}}हेतावसम्भवे भावस्तत् तस्यापि शङ्क्यते ।&विरुद्धानाम्पदार्थानामपि व्यापकदर्शनात् ॥ २८८ ॥\&[\smallbreak]


	
	  \endgroup
	

	  \pstart {\color{DodgerBlue3}“मिथ्यात्व”}मर्थशून्यत्वं {\color{DodgerBlue3}“कृतकेष्वेव”} वाक्येषु {\color{DodgerBlue3}“दृष्टमि\edlabel{pvv.394-3}\footnote{\label{pvv.394-3}  ३ हेतोः ।}त्यकृतं वचः”} (।) {\color{DodgerBlue3}“सत्यार्थं यदि”} स्यात् तदा को दोषः । कस्मादेवमित्याह परः । अकृतकत्वस्य साधनस्य विरोधिना कृतकत्वेन साध्य\edlabel{pvv.394-4}\footnote{\label{pvv.394-4}  ४ सत्यार्थत्वस्य ।}विपर्ययस्य मिथ्यात्वस्य व्यापनात्\edlabel{pvv.394-5}\footnote{\label{pvv.394-5}  ५ वेदे कृतकनिवृत्तौ मिथ्यानिवृत्तेः सत्यता सेष्टैव ।} । अत्राह । विपक्षान्मिथ्यात्वादकृतकस्य हेतोरसंभवेऽसंभवनिमित्तं हेतौ बाधकप्रमाणेऽनुक्ते तस्याकृतकत्वस्यापि मिथ्यात्वे विपक्षे भावः शंक्यते बाधकप्रमाणादर्शनात् ।
	\pend
      

	  \pstart ननु मिथ्यात्वं कृतकेषु दृष्टं तदकृतकेषु विरोधिषु कथं स्यादित्याह । {\color{DodgerBlue3}“विरुद्धानामपि”} हि व्याप्यानां {\color{DodgerBlue3}“पदार्थानामे”}कस्य {\color{DodgerBlue3}“व्यापक”}स्य {\color{DodgerBlue3}“दर्शनात्”} । यथा प्रयत्नानन्तरीय{\color{DodgerBlue3}“कत्वयोरेकेन”} कृतकत्वेनानित्यत्वेन वा व्याप्तिः । (२८७,२८८)
	\pend
      \label{div_pvv.3.289}\edlabel{div_pvv.3.289}
	  
	% new div opening: depth here is 2
	

	  \pstart किञ्च (।)
	\pend
      
	  \bigskip
	  \begingroup
	  \large
	
	    
	    \stanza[\smallbreak]
	\label{pv.3.289}\edlabel{pv.3.289}\flagstanza{\tiny\textenglish{....3.289}}नासत्ता सिद्धिरित्युक्तं सर्वतोनुपलम्भनात् ।&असिद्धायामसत्तायां सन्दिग्धा व्यतिरेकिता ॥ २८९ ॥\&[\smallbreak]


	
	  \endgroup
	

	  \pstart {\color{DodgerBlue3}“सर्व्वतो”} विपक्षाद्धेतोरसत्ताया {\color{DodgerBlue3}“अनुपलम्भात्”} प्रतिब्धमन्तरेण {\color{DodgerBlue3}“न सिद्धिरित्युक्तं प्राक्”} । “न चादर्शनमात्रेण विपक्षेऽव्यभिचारिते” \cref{pv.3.12}त्यादिना । {\color{DodgerBlue3}“असिद्धायाम्वासत्यायां”} विपक्षाद् {\color{DodgerBlue3}“व्यतिरेकिता सन्दिग्धा”} (।)
	\pend
      \leavevmode\marginnote{\textenglish{395/s}}

	  \pstart कस्मात् पुनर्व्यतिरेकनिश्चयापेक्षा । न हि साधनतयोपात्तो {\color{DodgerBlue3}“धर्मं”} इत्येव साधनं । (२८९)
	\pend
      \label{div_pvv.3.290}\edlabel{div_pvv.3.290}
	  
	% new div opening: depth here is 2
	

	  \pstart किन्तर्हि (।)
	\pend
      
	  \bigskip
	  \begingroup
	  \large
	
	    
	    \stanza[\smallbreak]
	\label{pv.3.290}\edlabel{pv.3.290}\flagstanza{\tiny\textenglish{....3.290}}अन्वयो व्यतिरेको वा सत्त्वं वा साध्यधर्मिणि ।&तन्निश्चयफलैर्ज्ञानैः सिद्ध्यन्ति यदि साधनम् ॥ २९० ॥\&[\smallbreak]


	
	  \endgroup
	

	  \pstart {\color{DodgerBlue3}“अन्वयः”} सपक्षे\edlabel{pvv.395-1}\footnote{\label{pvv.395-1}  १ स्वसाध्येन हेतोर्व्याप्तिः ।} {\color{DodgerBlue3}“व्यतिरेको”} विपक्षात्\edlabel{pvv.395-2}\footnote{\label{pvv.395-2}  २ व्यावृत्तिर्व्वा ।} । {\color{DodgerBlue3}“साध्यधर्मिणि सत्व\edlabel{pvv.395-3}\footnote{\label{pvv.395-3}  ३ हेतोः पक्षधर्मत्वं ।}म्वा”} त{\color{DodgerBlue3}“न्निश्चयफलै”}रन्वयादिनिश्चयप्रयोजनै{\color{DodgerBlue3}“र्ज्ञानैः”} प्रमाणात्म{\color{DodgerBlue3}“भिर्यदि सिध्यन्ति”} तदा {\color{DodgerBlue3}“साधनं भवति”} । निश्चितत्रैरूप्यस्यैव हेतुत्वात् । (२९०)
	\pend
      \label{div_pvv.3.291}\edlabel{div_pvv.3.291}
	  
	% new div opening: depth here is 2
	

	  \pstart यदपि व्यतिरेकी हेतुरित्युक्तं तच्चायुक्तमित्याह \edlabel{pvv.395-4}\footnote{\label{pvv.395-4}  ४ य एव मिथ्यात्वव्यवच्छेदस्य विषयः ।} ।
	\pend
      
	  \bigskip
	  \begingroup
	  \large
	
	    
	    \stanza[\smallbreak]
	\label{pv.3.291}\edlabel{pv.3.291}\flagstanza{\tiny\textenglish{....3.291}}यत्र साध्यविपक्षस्य वर्ण्णते व्यतिरेकिता ।&स एवास्य सपक्षः स्यात् सर्वो हेतुरनन्वयी ॥ २९१ ॥\&[\smallbreak]


	
	  \endgroup
	

	  \pstart {\color{DodgerBlue3}“यत्र”} धर्मिणि {\color{DodgerBlue3}“साध्यविपक्षस्य”} मिथ्यात्वस्य साधनाविपक्षव्यतिरेकाद् {\color{DodgerBlue3}“व्यतिरेकिता वर्ण्ण्यते”} । यत् कृतकं न भवति तन्मिथ्यार्थञ्च न भवतीति {\color{DodgerBlue3}“स एव”} साध्यसाधनविपर्ययनिवृत्तिदर्शनविषयो धर्मी {\color{DodgerBlue3}“सपक्षः स्यादस्या”}कृतकत्वस्य हेतोः । विपर्ययनिषेधेन विधेरेव प्रतिपादनात् । यत्र च साध्यसाधनसत्तवनिश्चयः स एव पक्षः । {\color{DodgerBlue3}“अतः सर्व्वो हेतुरनन्वयी”} न केवलव्यतिरेकी । (२९१)
	\pend
      \label{div_pvv.3.292}\edlabel{div_pvv.3.292}
	  
	% new div opening: depth here is 2
	

	  \begin{center}%% label @type='head'
	\textbf{(ख) समयत्वान्मन्त्राणां कृतकारिता}
	\end{center}
	

	  \pstart ननु \edlabel{pvv.395-5}\footnote{\label{pvv.395-5}  ५ साध्यधर्मसामान्येन समानोर्थः सपक्षः ।} पक्षस्य वेदस्य कथं सपक्षता सपक्षलक्षण\edlabel{pvv.395-6}\footnote{\label{pvv.395-6}  ६ साधर्म्यदृष्टान्त उच्यते न चायमिहास्ति साध्यत्वेनान्वय एव सपक्ष उच्यते ।} योगात् । एवं तर्हि सर्व्वः पक्षः सपक्षः स्यात् । भवत्येव किमन्येनेति चेत् । अन्यस्यापि लाक्षणिकं तत्कथं त्यज्यतां । किञ्च (।)
	\pend
      
	  \bigskip
	  \begingroup
	  \large
	
	    
	    \stanza[\smallbreak]
	\label{pv.3.292a}\edlabel{pv.3.292a}\flagstanza{\tiny\textenglish{...3.292a}}समयत्वे हि मन्त्राणां कस्यचित् कार्यदर्शनम् ।\&[\smallbreak]


	
	  \endgroup
	

	  \pstart किञ्चित्कार्य\edlabel{pvv.395-7}\footnote{\label{pvv.395-7}  ७ सङ्केतत्वादेव पौरुषेयत्वमाह ।}कारिणो मन्त्रा अपौरुषेयाश्चेति व्याहतं (।) तथा हि\leavevmode\marginnote{\textenglish{78b/MA}} कस्यचित् सत्यतपःप्रभावतः पुंसः समयत्वे संकेतत्वे मन्त्राणामभ्युपगम्यमाने \leavevmode\marginnote{\textenglish{396/s}} कार्यस्य मन्त्रप्रयोगनिस्प (? ष्प)ाद्यत्वेनेष्टस्य साधनं सिद्धिः स्यात् । केनचिच्छक्तिविशेषवता मत्प्रणीतं मन्त्रमेवंप्रयुञ्जानस्यायमर्थः सेत्स्यतीति समयस्य कृतत्वात् सिद्धिः स्यात् । कविसमयादिवत् काव्यापाठकस्या\edlabel{pvv.396-1}\footnote{\label{pvv.396-1}  १ मत्काव्यं यः पठति तस्मै मयेदं देयमिति समयात् कर्त्तुः ।}र्थसिद्धिः ।
	\pend
      
	  \bigskip
	  \begingroup
	  \large
	
	    
	    \stanza[\smallbreak]
	\label{pv.3.292b}\edlabel{pv.3.292b}\flagstanza{\tiny\textenglish{...3.292b}}अथापि भावशक्तिः स्यादन्यथाप्यविशेषतः ॥ २९२ ॥\&[\smallbreak]


	
	  \endgroup
	

	  \pstart {\color{DodgerBlue3}“अथापि भावशक्तिः सादृश्यपौरुषेयाणां स्यात्”} । यया-\edlabel{pvv.396-2}\footnote{\label{pvv.396-2}  २ विधिना नियोगं ।} भिमतसिद्धिः । एवन्तर्ह्य{\color{DodgerBlue3}“न्यथा”} प्रयुक्ते{\color{DodgerBlue3}“पि”} मन्त्रे \edlabel{pvv.396-3}\footnote{\label{pvv.396-3}  ३ तस्य तद्रूपविशिष्टं विधिशून्ये विपरीतपाठादौ च ।} स्यादभिमतम{\color{DodgerBlue3}“विशेषतो”} वर्ण्णात्मकस्य मन्त्रस्य । (२९२)
	\pend
      \label{div_pvv.3.293}\edlabel{div_pvv.3.293}
	  
	% new div opening: depth here is 2
	

	  \begin{center}%% label @type='head'
	\textbf{(ग) वर्णक्रमो मन्त्रेष्वकिञ्चित्करः}
	\end{center}
	

	  \pstart वर्ण्णानां क्रमविशेषो मन्त्रः ततस्तद्ग्रहणेन फलमिति चेत् ।
	\pend
      
	  \bigskip
	  \begingroup
	  \large
	
	    
	    \stanza[\smallbreak]
	\label{pv.3.293}\edlabel{pv.3.293}\flagstanza{\tiny\textenglish{....3.293}}क्रमस्यार्थान्तरत्वञ्च पूर्व्वमेव निराकृतम् ।&नित्त्यं तदर्थसिद्धिः स्यादसामर्थ्यमपेक्ष्यते ॥ २९३ ॥\&[\smallbreak]


	
	  \endgroup
	

	  \pstart {\color{DodgerBlue3}“क्रमस्य”} वर्ण्णेभ्यो{\color{DodgerBlue3}“ऽर्थान्तरत्व”}ञ्च {\color{DodgerBlue3}“पूर्व्वमेव निराकृतं”} । “वर्ण्णानुपूर्व्वी वाक्यञ्चेदि”त्य \href{http://http://sarit.indology.info/?cref=pv.3.259}{(३।२५९)}त्रान्तरे । अस्तु वा क्रमः तस्य नित्यत्वात् तदर्थस्य तन्नि{\color{DodgerBlue3}“स्पा”} (? ष्पा) द्यस्या{\color{DodgerBlue3}“र्थस्य नित्यं सिद्धिः”} स्यात् । न ह्यविकले \edlabel{pvv.396-4}\footnote{\label{pvv.396-4}  ४ आधेयातिशयस्य ।}कारणे कार्यक्षेपो युक्तः । अथ प्रयोगविधानाद्यपेक्षास्ति साप्ययुक्ता । अपेक्षणे कस्यचित् सहकारिणो{\color{DodgerBlue3}“ऽसामर्थ्यं”} स्वभावतः स्यात् । समर्थस्यान्यापेक्षाऽयोगात् । अपेक्षणीयात् समर्थस्वभावोत्पत्तौ तु स्यादपेक्षा । (२९३)
	\pend
      \label{div_pvv.3.294}\edlabel{div_pvv.3.294}
	  
	% new div opening: depth here is 2
	

	  \pstart किञ्च (।)
	\pend
      
	  \bigskip
	  \begingroup
	  \large
	
	    
	    \stanza[\smallbreak]
	\label{pv.3.294a}\edlabel{pv.3.294a}\flagstanza{\tiny\textenglish{...3.294a}}सर्वस्य साधनं ते स्युर्भावशक्तिर्यदीदृशी ।\&[\smallbreak]


	
	  \endgroup
	

	  \pstart मन्त्राणां भावशक्तिर्यदीदृशी कार्यविशेषसाधिका तदा सर्व्वस्य यजमानस्य इतरस्य\edlabel{pvv.396-5}\footnote{\label{pvv.396-5}  ५ पातक्यादेश्च ।} च ते मन्त्रा अभिमतार्थसिद्धेः साधनं स्युः । न हि कार्यकारिता तेषां कञ्चिदेव प्रति नेतरान् साधारणत्वाद् भावस्वभावस्य ।
	\pend
      

	  \begin{center}%% label @type='head'
	\textbf{ग. नित्यत्वे दोषः}
	\end{center}
	

	  \begin{center}%% label @type='head'
	\textbf{(क) असंस्कार्यस्य न प्रयोक्तृभेदापेक्षा}
	\end{center}
	

	  \pstart अथ प्रयोक्तुर्विशेषमपेक्ष्य फलप्रायाः । तच्चासत् ।
	\pend
      \leavevmode\marginnote{\textenglish{397/s}}
	  \bigskip
	  \begingroup
	  \large
	
	    
	    \stanza[\smallbreak]
	\label{pv.3.294b}\edlabel{pv.3.294b}\flagstanza{\tiny\textenglish{...3.294b}}प्रयोक्तृभेदापेक्षा च नासंस्कार्यस्य युज्यते ॥ २९४ ॥\&[\smallbreak]


	
	  \endgroup
	

	  \pstart प्रयोक्तुर्भेदो यजमानत्वं तदपेक्षा च नित्यस्य परैरसंस्कार्यस्य न युज्यते । (२९४)
	\pend
      \label{div_pvv.3.295}\edlabel{div_pvv.3.295}
	  
	% new div opening: depth here is 2
	
	  \bigskip
	  \begingroup
	  \large
	
	    
	    \stanza[\smallbreak]
	\label{pv.3.295}\edlabel{pv.3.295}\flagstanza{\tiny\textenglish{....3.295}}संस्कार्यस्यापि भावस्य वस्तुभेदो हि भेदकः ।&प्रयोक्तृभेदान्नियमः शक्तौ न समये भवेत् ॥ २९५ ॥\&[\smallbreak]


	
	  \endgroup
	

	  \pstart संस्कार्यस्यापि भावस्य\edlabel{pvv.397-1}\footnote{\label{pvv.397-1}  १ आधेयातिशयस्य ।} वस्तुनः\edlabel{pvv.397-2}\footnote{\label{pvv.397-2}  २ कारणस्य ।} संस्कर्त्तुर्भेदो हि भेदको भवितुमर्हति । न तु ब्राह्मणशूद्रादीनां वस्तुतो जातिभेदः कश्चिदस्ति व्यवहा/?/रमात्रत्वात् तस्य । ततश्च प्रयोक्तुर्भेदादपि मन्त्राणां शक्तौ नियमो न सम्भवति । ननूपलभ्यते नियता मन्त्राणां शक्तिः । सत्यमुपलभ्यते किन्तु सा पुरुषकृते समयेऽभ्युपगम्यमाने भवेत् न त्वकृतत्वे । \edlabel{pvv.397-3}\footnote{\label{pvv.397-3}  ३ यदा तु समयो मन्त्रस्तदा समयकर्त्ता वस्त्वनपेक्षः समयं करोतीत्याह ।}यो हि ब्राह्मण इति प्रसिद्धः तस्यैवायं विधिप्रयुक्तो मन्त्रः फलप्रदो नेतरस्येति कर्त्रा पुरुषेण शक्तिविशेषवता समितत्वान्मन्त्रस्य । (२९५)
	\pend
      \label{div_pvv.3.296}\edlabel{div_pvv.3.296}
	  
	% new div opening: depth here is 2
	

	  \begin{center}%% label @type='head'
	\textbf{(ख) नित्त्यानां मन्त्राणां प्रयोजको निरर्थकः}
	\end{center}
	

	  \pstart किञ्च (।)
	\pend
      
	  \bigskip
	  \begingroup
	  \large
	
	    
	    \stanza[\smallbreak]
	\label{pv.3.296}\edlabel{pv.3.296}\flagstanza{\tiny\textenglish{....3.296}}अनाधेयविशेषाणां किंकुर्वाणः प्रयोजकः ।&प्रयोगो यद्यभिव्यक्तिः सा प्रागेव निराकृता ॥ २९६ ॥\&[\smallbreak]


	
	  \endgroup
	

	  \pstart नित्यानामनाधेयविशेषाणां प्रयोक्ता\edlabel{pvv.397-4}\footnote{\label{pvv.397-4}  ४ अन्यथापि न कश्चित् फलमश्नुतेऽन्यो न शूद्रादिरिति कुतोयं विभागः ।} पुरुषः किंकुर्व्वाणः प्रयोजक इष्टः । अनुपकारकस्य प्रयोजकत्वे सर्व्वस्य तथात्वप्रसङ्गात् । यद्यभिव्यक्तिः प्रयोग उच्यते\edlabel{pvv.397-5}\footnote{\label{pvv.397-5}  ५ नोत्पादनं ।} साऽभिव्यक्तिर्नित्यानां प्रागेव सामान्यप्रस्तावे निराकृता । न हि स्वरूपपरिणाम\edlabel{pvv.397-6}\footnote{\label{pvv.397-6}  ६ नित्त्यस्य तदनुपपत्तेः नित्त्यस्य तु युक्तापेक्षा ।} आवरणविगमो वा सा\edlabel{pvv.397-7}\footnote{\label{pvv.397-7}  ७ अभिव्यक्तिर्न घटते ।}नित्यातां घटते । (२९६)
	\pend
      \label{div_pvv.3.297}\edlabel{div_pvv.3.297}
	  
	% new div opening: depth here is 2
	

	  \begin{center}%% label @type='head'
	\textbf{(ग) नित्यस्य व्यक्तिरसिद्धा}
	\end{center}
	

	  \pstart बुद्धिः कदाचित् सम्भाव्यते\edlabel{pvv.397-8}\footnote{\label{pvv.397-8}  ८ नित्त्यत्वान्न योग्योत्पत्तिः किन्तु शब्दविषया बुद्धिः ।} । तदा (।)
	\pend
      
	  \bigskip
	  \begingroup
	  \large
	
	    
	    \stanza[\smallbreak]
	\label{pv.3.297}\edlabel{pv.3.297}\flagstanza{\tiny\textenglish{....3.297}}व्यक्तिश्च बुद्धिः सा यस्मात् स फलैर्यदि युज्यते ।&स्याच्छ्रोतुः फलसंबन्धो वक्ता हि व्यक्तिकारणम् ॥ २९७ ॥\&[\smallbreak]


	
	  \endgroup
	\leavevmode\marginnote{\textenglish{398/s}}

	  \pstart तद्{\color{DodgerBlue3}“व्यक्तिश्च बुद्धि”}रुच्येत । {\color{DodgerBlue3}“सा\edlabel{pvv.398-1}\footnote{\label{pvv.398-1}  १ व्यक्तिः ।}यस्माद्”} वक्तु \edlabel{pvv.398-2}\footnote{\label{pvv.398-2}  २ पुरुषात् ।} र्भवति {\color{DodgerBlue3}“स फ लेन युज्यते यदि”} तदा नातिप्रसङ्गः । नन्वेमपि {\color{DodgerBlue3}“श्रोतुः फलसम्बन्धः \edlabel{pvv.398-3}\footnote{\label{pvv.398-3}  ३ यो वक्त्रा पठ्यमानं मंत्रं शृणोति ।}”} स्यान्न वक्तुरेव । \edlabel{pvv.398-4}\footnote{\label{pvv.398-4}  ४ यस्मात् ।} {\color{DodgerBlue3}“वक्ता हि व्यक्ते”}र्ज्ञानस्य {\color{DodgerBlue3}“कारण\edlabel{pvv.398-5}\footnote{\label{pvv.398-5}  ५ नित्ये शब्दे बुद्धिजन्मार्थं पुंसो व्यापाराभा (वा) दनुपकार्योपकारकत्वान्न वक्ता श्रोतुरुपकार इति विशेषोनयोर्न्नास्ति ।}मिति”} फले युज्यते । तच्च \edlabel{pvv.398-6}\footnote{\label{pvv.398-6}  ६ मन्त्रविषयं ।}ज्ञानहेतुत्वं श्रोतुरप्यस्त्येव ।\edlabel{pvv.398-7}\footnote{\label{pvv.398-7}  ७ किं न फलेन युज्यते ।} (२९७)
	\pend
      \label{div_pvv.3.298}\edlabel{div_pvv.3.298}
	  
	% new div opening: depth here is 2
	

	  \pstart किञ्च (।)
	\pend
      
	  \bigskip
	  \begingroup
	  \large
	
	    
	    \stanza[\smallbreak]
	\label{pv.3.298}\edlabel{pv.3.298}\flagstanza{\tiny\textenglish{....3.298}}अनभिव्यक्तशब्दानां करणानां प्रयोजनम् ।&मनोजपो वा व्यर्थः स्याच्छब्दो हि श्रोत्रगोचरः ॥ २९८ ॥\&[\smallbreak]


	
	  \endgroup
	

	  \pstart {\color{DodgerBlue3}“अनभिव्य\edlabel{pvv.398-8}\footnote{\label{pvv.398-8}  ८ श्रोत्रविषयं न नीतः यैस्ताल्वादिभिः ।}क्तो”}ऽनिवेदितः \edlabel{pvv.398-9}\footnote{\label{pvv.398-9}  ९ प्रयोगो व्यर्थः स्यादिति संबन्धः । यत्रौष्ठप्रस्पन्दमात्रेण उपांशुजपः क्रियते सोपि व्यर्थः स्यादित्यादावुत्पत्य व्यक्तिं नित्येषु अनाधेयातिशयेषु ।} {\color{DodgerBlue3}“शब्दो”} यैस्तेषां {\color{DodgerBlue3}“करणा”}दीनां {\color{DodgerBlue3}“प्रयोजनं”} प्रयोगः \leavevmode\marginnote{\textenglish{79a/MA}} यदा । तदा ताल्वादिकरणप्रस्पन्दमात्रयो\edlabel{pvv.398-10}\footnote{\label{pvv.398-10}  १० यदपि विशिष्टः प्रयोक्ता मन्त्रफलमश्नुत इति तत्रापि समीरितार्थोत्पादयोग्योत्पादनमुत्पन्नस्य उत्तरोत्तरविशेषोत्पादनम्वार्थः सविशेषजन्मनि स्यात् ।}पांशुजपः क्रियते न तु व्यक्तमुच्यते । यदा च तामपि  विना \edlabel{pvv.398-11}\footnote{\label{pvv.398-11}  ११ ताल्वादिकरणप्रस्यन्दमात्रां ।}मनोमात्रेण {\color{DodgerBlue3}“मनोजपो वा”} क्रियते तदा द्वावपि जपाविमौ {\color{DodgerBlue3}“व्यर्थो”} स्यातां । \edlabel{pvv.398-12}\footnote{\label{pvv.398-12}  १२ यस्माच्छ्रोत्रग्राह्य एव शब्दः तत्स्वभावश्च मन्त्रः । उपांशुमनोजपयोश्च श्रोत्रग्रहणाभावादशब्दत्वं तत्त्वादमन्त्रः कथं फलवान् । अयमाशयः शब्दात्मनां मन्त्राणां व्यक्तिहेतुः शब्दग्राहिज्ञानहेतुपुरुषः प्रयोक्ता तस्य फलेन सम्बन्धः चेत् । तदोपांशुमनोजापी न फलयोगी स्यादभिव्यक्त्यहेतुत्वात् । मन्त्रस्य श्रोत्रग्राह्यापौरुषेयं नित्यत्वाभ्युपगमात् ।} {\color{DodgerBlue3}“शब्दो हि श्रोत्रगोचर”} उच्यते न च जपयोरनयोः श्रोत्रगोचरः कश्चिदस्ति । (२९८)
	\pend
      \label{div_pvv.3.299}\edlabel{div_pvv.3.299}
	  
	% new div opening: depth here is 2
	
	  \bigskip
	  \begingroup
	  \large
	
	    
	    \stanza[\smallbreak]
	\label{pv.3.299}\edlabel{pv.3.299}\flagstanza{\tiny\textenglish{....3.299}}पारम्पर्येण तज्जत्वात् तद्व्यक्तिः सापि चेन्मतिः ।&ते तथा स्युस्तदर्था चेदसिद्धं कल्पनान्वयात् ॥ २९९ ॥\&[\smallbreak]


	
	  \endgroup
	\leavevmode\marginnote{\textenglish{399/s}}

	  \pstart ननु याप्युपांशुमनोजपकाले शब्दाभासा धीः सापि\edlabel{pvv.399-1}\footnote{\label{pvv.399-1}  १ तद्व्यक्तिस्तस्य शब्दस्य व्यक्तिर्ज्ञानं पूर्व्वशब्दजज्ञाना हि संस्कारस्य मनोजपे शब्दप्रतिभासात् मन्त्रत्वं ।} मतिः {\color{DodgerBlue3}“पारम्पर्येण तज्जत्वात्”} तस्य {\color{DodgerBlue3}“व्यक्ति”}रिति चेत् । यद्येवं शब्दाविकल्पवदर्थविकल्पा \edlabel{pvv.399-2}\footnote{\label{pvv.399-2}  २ संकेतात् किल वादिनो मनोजपादिसाफल्यमाह दृश्यविकल्प्यैक्यात् ।} अपि तत्प्रभवा इति {\color{DodgerBlue3}“ते”}पि \edlabel{pvv.399-3}\footnote{\label{pvv.399-3}  ३ विकल्पा अपि मन्त्रा इति तत्प्रयोक्तापि फलभाग् स्यात् । एतेनातिप्रसङ्ग उक्तः ।} {\color{DodgerBlue3}“तथा”} शब्दव्यक्तयः {\color{DodgerBlue3}“स्युः”} । परः न केवलाच्छब्दप्रभवत्वात् तद्व्यक्तिः\edlabel{pvv.399-4}\footnote{\label{pvv.399-4}  ४ यतस्तद्वान् प्रयोक्ता स्यात् ।} किन्तु {\color{DodgerBlue3}“तदर्था”} शब्दविषया सतीति {\color{DodgerBlue3}“चेत्”} न चार्थविकल्पः शब्दविषयः । शब्दविकल्पस्य \edlabel{pvv.399-5}\footnote{\label{pvv.399-5}  ५ स्वलक्षणाविषयत्वाद् विकल्पस्य समयकाराभिप्रायसंवादनात् फलप्राप्तेः । न तु वस्तुन्यविरोधः शब्दात् फले मननमशब्दो यतः ।} शब्दविषयत्वमसिद्धं {\color{DodgerBlue3}“कल्पनाया”} वाच्यवाचकयोजनाया {\color{DodgerBlue3}“अन्वयात्”} सम्बन्धात् न च कल्पना वस्तुविषयेति \edlabel{pvv.399-6}\footnote{\label{pvv.399-6}  ६ वासनाप्रबाधादुत्पत्तेर्ब्बाह्यासत्त्वेपि ।}कथ्यते । (२९९)
	\pend
      \label{div_pvv.3.300}\edlabel{div_pvv.3.300}
	  
	% new div opening: depth here is 2
	

	  \begin{center}%% label @type='head'
	\textbf{घ. समयकाण्णामुक्त्या फलविशेषः}
	\end{center}
	

	  \pstart अस्माकन्तु मते (।)
	\pend
      
	  \bigskip
	  \begingroup
	  \large
	
	    
	    \stanza[\smallbreak]
	\label{pv.3.300}\edlabel{pv.3.300}\flagstanza{\tiny\textenglish{....3.300}}स्वसामान्यस्वभावानामेकभावविवक्षया ।&उक्तेः समयकाण्णामविरोधो न वस्तुनि ॥ ३०० ॥\&[\smallbreak]


	
	  \endgroup
	

	  \pstart स्वसामान्यस्वभावानां शब्दस्वलक्षणसामान्यलक्षणानां प्रत्यक्षविकल्पबुद्धिविषयाणामेक\edlabel{pvv.399-7}\footnote{\label{pvv.399-7}  ७ दृश्यविकल्पैक्यात् ।}त्वाध्यवसायवशादेकभावस्यैकत्वस्य । विवक्षया समयकाराणां समयकर्त्तृभिः शक्तिमत्पुंभिर्मन्त्राणामुक्तेरविरोधः । उपांशुजपमनोजपयोर्व्वैफल्य\edlabel{pvv.399-8}\footnote{\label{pvv.399-8}  ८ समयकाराभिप्रायसम्पादनात् फलप्राप्तेः न तु वस्तुन्यविरोधः शब्दात् फले मननमशब्दो यतः ।}विरोधाभावो दृश्यविकल्प्यावेकाध्यवसायादेककार्यकारित्वेनाधिष्ठितत्वात् तत्कुरुतः । ये तु वस्तुभूतं मन्त्रमिच्छन्ति तेषाम्विरोध एव तदाह । न वस्तुनि विरोधाभावः । न ह्युपांशुजपादिविषयो वस्तुकल्पितत्वात्\edlabel{pvv.399-9}\footnote{\label{pvv.399-9}  ९ यदुक्तं वर्ण्णा एव मन्त्रस्तत्र ।}। (३००)
	\pend
      \label{div_pvv.3.301}\edlabel{div_pvv.3.301}
	  
	% new div opening: depth here is 2
	\leavevmode\marginnote{\textenglish{400/s}}

	  \begin{center}%% label @type='head'
	\textbf{ङ वर्ण्णानुपूर्विचिन्ता}
	\end{center}
	

	  \begin{center}%% label @type='head'
	\textbf{(क) आनुपूर्व्यभावे नार्थभेदः}
	\end{center}
	

	  \pstart अथ त्वन्मतेपि (।)
	\pend
      
	  \bigskip
	  \begingroup
	  \large
	
	    
	    \stanza[\smallbreak]
	\label{pv.3.301}\edlabel{pv.3.301}\flagstanza{\tiny\textenglish{....3.301}}आनुपूर्व्यामसत्यां स्यात् सरो रस इति श्रुतौ ।&न कार्यभेद इति चेद् अस्ति सा पुरुषाश्रया ॥ ३०१ ॥\&[\smallbreak]


	
	  \endgroup
	

	  \pstart आनुपूर्व्या वर्ण्णव्यतिरिक्तायामसत्यां सरो रस इति प्रसिद्धानुलोमविलोम क्रमायां श्रुतौ कार्यस्य स्वज्ञानस्य भेदो न स्यादिति चेत् । अस्त्यस्मन्मते सानुपूर्व्वी पुरुषाश्रया\edlabel{pvv.400-1}\footnote{\label{pvv.400-1}  १ वर्ण्णाव्यतिरिक्ता ।} पुरुषकृता प्रतिपदं\edlabel{pvv.400-2}\footnote{\label{pvv.400-2}  २ तत्रैकत्वाध्यवसायः परं मन्दानां ।}भिन्ना ततो न ज्ञानभेदप्रसङ्गः । (३०१)
	\pend
      \label{div_pvv.3.302}\edlabel{div_pvv.3.302}
	  
	% new div opening: depth here is 2
	

	  \pstart तथा हि (।)
	\pend
      
	  \bigskip
	  \begingroup
	  \large
	
	    
	    \stanza[\smallbreak]
	\label{pv.3.302}\edlabel{pv.3.302}\flagstanza{\tiny\textenglish{....3.302}}यो यद्वर्णसमुत्थानज्ञानजाज्ज्ञानतो ध्वनिः ।&जायते तदुपाधिः स श्रुत्या समवसीयते ॥ ३०२ ॥\&[\smallbreak]


	
	  \endgroup
	

	  \pstart यो ध्वनिर्जायते तद्वर्ण्णसमुत्थो ज्ञानतः । यश्चासौ वर्ण्णश्च यद्वर्ण्णस्तस्य समुत्थानं कारणं तच्च तत् ज्ञानञ्च यद्वर्ण्णसमुत्थानज्ञानं तस्माज्जातं यद्वर्ण्णसमुत्थानज्ञानजं तस्मात् ज्ञानमतः । अयमर्थ आद्यस्य वर्ण्ण\edlabel{pvv.400-3}\footnote{\label{pvv.400-3}  ३ वक्तृस्थं पूर्वपूर्ववर्ण्णसमुत्थापकहेतु ।}स्य यत्समुत्थापकं विवक्षात्मकं ज्ञानं तेन समनन्तरप्रत्ययेन सता द्वितीयवर्ण्णसमुत्थापकं ज्ञानं जन्यते तेन च द्वितीयो वर्ण्ण एवं द्वितीयवर्ण्णसमुत्थापकात् ज्ञानात् तृतीयवर्ण्णेत्थापकज्ञानोत्पत्तौ तृतीयवर्णोत्पत्तिरिति कारणक्रमाद्वर्णोत्पत्तिक्रम उक्तः\edlabel{pvv.400-4}\footnote{\label{pvv.400-4}  ४ कारणभेदात्कार्यभेदमुक्त्वा ।} । पुनः कार्यक्रमेण क्रमं दर्शयितु माह\edlabel{pvv.400-5}\footnote{\label{pvv.400-5}  ५ वर्णाश्च क्रमेणोत्पन्नाः श्रोतृसंतानस्थानां स्वविषयज्ञानानां क्रमेण हेतवो भवन्तो जायन्त इति दर्शयन्नाह ।}। स उत्तरो वर्णस्तदुपाधिः पूर्णवर्णविशेषणस्तदनन्तर इत्यर्थः । श्रुत्या श्रवणज्ञानेन ग्राह्यवर्णकार्येण श्रोतृसन्तानवर्त्तिना समवसीयते । क्रमोत्पन्ना वर्णाः स्वस्वजनितज्ञानैरसहभाविन एव गृह्यन्ते । (३०२)
	\pend
      \label{div_pvv.3.303}\edlabel{div_pvv.3.303}
	  
	% new div opening: depth here is 2
	

	  \pstart ननु क्रमभाविनां सहदर्शनाभावात्कथं पूर्ववर्ण्णोपाधि\edlabel{pvv.400-6}\footnote{\label{pvv.400-6}  ६ परकाले पूर्ववर्णध्वंसात् ।} ग्रहणमित्याह (।)
	\pend
      \leavevmode\marginnote{\textenglish{401/s}}
	  \bigskip
	  \begingroup
	  \large
	
	    
	    \stanza[\smallbreak]
	\label{pv.3.303}\edlabel{pv.3.303}\flagstanza{\tiny\textenglish{....3.303}}तज्ज्ञानजनितज्ञानः स श्रुतावपटुश्रुतिः ।&अपेक्ष्य तत्स्मृतिं पश्चात् स्मृतिमाधत्त आत्मनि ॥ ३०३ ॥\&[\smallbreak]


	
	  \endgroup
	

	  \pstart तस्य पूर्व्ववर्णस्य {\color{DodgerBlue3}“ज्ञानेन”} ग्राहकेणोत्तरवर्ण्णसहकारिणा {\color{DodgerBlue3}“जनितं”} ग्राहकं\edlabel{pvv.401-1}\footnote{\label{pvv.401-1}  १ साकारालम्बनं ज्ञानकाल एवाकारसमुत्थापकचित्तेनाकारो जनित इति समकालता ।} {\color{DodgerBlue3}“ज्ञानं”} यस्मिन् स तत् ज्ञानजनित उत्तरो वर्ण्णः मन्दमुच्चार्यमाणत्वात्\edlabel{pvv.401-2}\footnote{\label{pvv.401-2}  २ शनैरुच्चारितो यदा वर्ण्णः ।} श्रुतौ श्रवणज्ञानेऽ{\color{DodgerBlue3}“पटश्रुति”}र्मन्दचारि\edlabel{pvv.401-3}\footnote{\label{pvv.401-3}  ३ यो मनसापि जपेत् तदर्थकरोहमिति यत्र विभक्ता वर्ण्णा अवधार्यन्तेवस्थायां ।}श्रवणज्ञानः । तस्य पूर्व्ववर्ण्णस्य {\color{DodgerBlue3}“स्मृतिमपेक्ष्या\edlabel{pvv.401-4}\footnote{\label{pvv.401-4}  ४ त्वरिते क्रमश्रुतिः कुतः ।}त्मनि”} स्मृतिं पूर्व्ववणनिन्तरत्वेनाधत्ते (।) यस्मात् पूर्व्ववर्णानन्तरत्वेनोत्तरः स्मर्यते (।) तस्मात्तदनन्तर एवासौ गृहीत इत्यर्थः । (३०३)
	\pend
      \label{div_pvv.3.304}\edlabel{div_pvv.3.304}
	  
	% new div opening: depth here is 2
	

	  \begin{center}%% label @type='head'
	\textbf{(ख) आनुपूर्वी पौरुषेयी}
	\end{center}
	
	  \bigskip
	  \begingroup
	  \large
	
	    
	    \stanza[\smallbreak]
	\label{pv.3.304}\edlabel{pv.3.304}\flagstanza{\tiny\textenglish{....3.304}}इत्येषा पौरुषेय्येव तद्धेतुग्रीहिचेतसाम् ।&कार्यकारणता वर्णे ह्यानुपूर्वीति कथ्यते ॥ ३०४ ॥\&[\smallbreak]


	
	  \endgroup
	

	  \pstart इत्येवमुक्तेन क्रमेण \edlabel{pvv.401-5}\footnote{\label{pvv.401-5}  ५ एवं रेफाकारविसर्जनीयोत्थापकानि पूर्वपूर्वसमनन्तरप्रत्ययानि ।} वर्ण्णेषु क्रमभाविषु तद्धेतुचेतसां वर्ण्णो\edlabel{pvv.401-6}\footnote{\label{pvv.401-6}  ६ कार्यजन्यत्वात् ।}त्थापकचेतसां\leavevmode\marginnote{\textenglish{79b/MA}} वक्तृसन्तानवर्तिनां । {\color{DodgerBlue3}“तद्ग्राहिचेतसां”} वर्ण्णग्राहकचेतसां श्रोतृसन्तानवर्तिनां {\color{DodgerBlue3}“कार्यकारणतैषा”} यथायोगं वर्ण्णापेक्षा क्रमवती कारणता कार्यता चा{\color{DodgerBlue3}“नुपूर्वी”} कथ्यते (।) सा च पुरुषनिर्वर्त्त्यत्वात् {\color{DodgerBlue3}“पौरुषेय्येव”} । (३०४)
	\pend
      \label{div_pvv.3.305}\edlabel{div_pvv.3.305}
	  
	% new div opening: depth here is 2
	
	  \bigskip
	  \begingroup
	  \large
	
	    
	    \stanza[\smallbreak]
	\label{pv.3.305}\edlabel{pv.3.305}\flagstanza{\tiny\textenglish{....3.305}}अन्यदेव ततो रूपं तद्वर्ण्णानां पदे पदे ।&कर्त्तृसंस्कारतो भिन्नं सहितं कार्यभेदकृत् ॥ ३०५ ॥\&[\smallbreak]


	
	  \endgroup
	

	  \pstart यतो न नित्यत्वं {\color{DodgerBlue3}“ततोन्यदेव वर्ण्णानां तद्रूपं पदे पदे”} प्रतिपदमेकाध्यवसायविषयत्वेपि\edlabel{pvv.401-7}\footnote{\label{pvv.401-7}  ७ अन्यथा देशाद्यनियमान्नियमे च देशादेरेवेन्धनत्व ।} {\color{DodgerBlue3}“कर्त्तृ\edlabel{pvv.401-8}\footnote{\label{pvv.401-8}  ८ क्रमभेद एव वर्ण्णभेदः केवलं रसपदात्र स पदान्तरस्याभेदो नावधार्यते ।}संस्कारतो”} वर्ण्णसमुत्थापकचित्तशक्तिभेदाज्जातं {\color{DodgerBlue3}“भिन्नं\edlabel{pvv.401-9}\footnote{\label{pvv.401-9}  ९ यतः ।}”} । क्रमेण चानुभूय स्मृत्या {\color{DodgerBlue3}“सहितं”} स्मृतं सत् {\color{DodgerBlue3}“कार्यभेदकृदर्थ”}प्रतिपत्तिविशेषकारि । (३०५)
	\pend
      \label{div_pvv.3.306_3.307}\edlabel{div_pvv.3.306_3.307}
	  
	% new div opening: depth here is 2
	\leavevmode\marginnote{\textenglish{402/s}}
	  \bigskip
	  \begingroup
	  \large
	
	    
	    \stanza[\smallbreak]
	\label{pv.3.306}\edlabel{pv.3.306}\flagstanza{\tiny\textenglish{....3.306}}सा चानुपूर्वी वर्ण्णानां तद्धेतुग्राहिचेतसाम् ।&इच्छाऽविरुद्धसिद्धीनां स्थितिक्रमविरोधतः ॥ ३०६ ॥\&[\smallbreak]


	
	  \endgroup
	
	  \bigskip
	  \begingroup
	  \large
	
	    
	    \stanza[\smallbreak]
	\label{pv.3.307}\edlabel{pv.3.307}\flagstanza{\tiny\textenglish{....3.307}}कार्यकारणतासिद्धेः पुंभ्यो वर्णक्रमस्य च ।&सर्वो वर्णक्रमः पुंभ्यो दहनेन्धनयुक्तिवत् ॥ ३०७ ॥\&[\smallbreak]


	
	  \endgroup
	

	  \pstart {\color{DodgerBlue3}“सा च वर्णानामानुपूर्व्वी\edlabel{pvv.402-1}\footnote{\label{pvv.402-1}  १ (वर्ण्णनिरर्थकत्वेपि एकविकल्पेन विषयीकृताः क्रमिणो वर्ण्णाः सहिता उक्ताः) ।}”} तद्धेतुग्राहिचेतसां रचनाकृतः पुरुषात्प्रवृत्तेति न स्थितक्रमा वर्णाः पुरुषेच्छयाऽविरुद्धसिद्धीनां वर्ण्णानां स्थितस्य क्रमस्य विरोधात् (।) न हि स्थितक्रमाणां हिमवद्विन्ध्यमलयानामिच्छया विपरीतक्रमः शक्यः (।) वर्ण्णास्त्विच्छया विपर्य्यास्यन्तें विकल्पक्रमानुविधायित्वाच्च (।) तद्वत् कल्पितोप्यर्थो न प्रमाणं {\color{DodgerBlue3}“कार्यकारणतायाः सिद्धेः (।) सर्व्वो”} वैदिकोऽन्यश्च {\color{DodgerBlue3}“वर्ण्णक्रमः\edlabel{pvv.402-2}\footnote{\label{pvv.402-2}  २ (पुरुषः कारणं वर्ण्णक्रमस्येति) ।} पुंभ्यो\edlabel{pvv.402-3}\footnote{\label{pvv.402-3}  ३ पुंविकल्पानुक्रमे सति भावादसति वा भावात् ।}”} भवतीत्यवधारणं । {\color{DodgerBlue3}“दहनेन्धनयुक्तिवत्”} (।) \edlabel{pvv.402-4}\footnote{\label{pvv.402-4}  ४ तथा लोकवर्ण्णक्रमवद् वैदिकः साध्यते ।}यथैकस्य दहनस्य इन्धनपूर्व्वकत्वदृष्ट्या सर्व्वोऽग्निरिन्धनपूर्व्व इति न्यायः । (३०६, ३०७)
	\pend
      \label{div_pvv.3.308}\edlabel{div_pvv.3.308}
	  
	% new div opening: depth here is 2
	
	  \bigskip
	  \begingroup
	  \large
	
	    
	    \stanza[\smallbreak]
	\label{pv.3.308}\edlabel{pv.3.308}\flagstanza{\tiny\textenglish{....3.308}}असाधारणता सिद्धा पुंसां च क्रमकारिणां ।&अतो ज्ञानप्रभावाभ्यामन्येषां तदभावतः ॥ ३०८ ॥\&[\smallbreak]


	
	  \endgroup
	

	  \pstart अतएव च मन्त्राख्य वर्ण्णक्रमकारिणां पुंसां ज्ञान\edlabel{pvv.402-5}\footnote{\label{pvv.402-5}  ५ समीहितफलसाधनवर्ण्णक्रमज्ञानं समीहितसंपादनशक्तिप्रभावः ।}प्रभावाभ्यां पुरुषान्तरैः सहासा{\color{DodgerBlue3}“धारणता”}ऽसमानता\edlabel{pvv.402-6}\footnote{\label{pvv.402-6}  ६ तदयं क्रमवत्त्वेन ज्ञातुः स परोक्षदृश्यस्ति ।} {\color{DodgerBlue3}“सिद्धा । अन्येषा\edlabel{pvv.402-7}\footnote{\label{pvv.402-7}  ७ प्राक्तनानाम् ।}न्तयोर्ज्ञानप्रभा”}वयो{\color{DodgerBlue3}“रभावतः”} । (३०८)
	\pend
      \label{div_pvv.3.309}\edlabel{div_pvv.3.309}
	  
	% new div opening: depth here is 2
	

	  \begin{center}%% label @type='head'
	\textbf{च. आप्तचिन्ता}
	\end{center}
	

	  \begin{center}%% label @type='head'
	\textbf{(क) आप्तसिद्धिः}
	\end{center}
	

	  \pstart a. ननु तन्त्रज्ञा रत्थापुरुषा अपि मन्त्रं किञ्चित् कार्यसमर्थ प्रणयन्तो दृश्यन्ते । ततो मन्त्रकरणा\edlabel{pvv.402-8}\footnote{\label{pvv.402-8}  ८ एवमन्योपि स्यात् कथं पुरुषातिशयसिद्धिः ।}न्नातिशयसिद्धिरित्याह ।
	\pend
      
	  \bigskip
	  \begingroup
	  \large
	
	    
	    \stanza[\smallbreak]
	\label{pv.3.309}\edlabel{pv.3.309}\flagstanza{\tiny\textenglish{....3.309}}येपि मन्त्रविदः केचिन् मन्त्रान् कांश्चन कुर्व्वते ।&प्रभोः प्रभावस्तेषां स तदुक्तन्यायवृत्तितः ॥ ३०९ ॥\&[\smallbreak]


	
	  \endgroup
	\leavevmode\marginnote{\textenglish{403/s}}

	  \pstart {\color{DodgerBlue3}“यपि”} केचित् साधारणा {\color{DodgerBlue3}“मन्त्रविदो”} मन्त्रशास्त्रज्ञाः {\color{DodgerBlue3}“कांश्चन मन्त्रान्”} विषादिशमनान् {\color{DodgerBlue3}“कुर्व्वते तेषां”} प्रभोर्मन्त्रप्रणेतुरतिशयितशक्तेः {\color{DodgerBlue3}“स प्रभावः”} सामर्थ्यं\edlabel{pvv.403-1}\footnote{\label{pvv.403-1}  १ प्रभुस्तुष्टस्तत्प्रणीतानप्यधितिष्ठतीति भावः ।} {\color{DodgerBlue3}“तदुक्त”}स्य न्यायस्य समयानुष्ठानादे{\color{DodgerBlue3}“र्वृत्तितः”} । आराधितस्य प्रभोः प्रभावादीदृगक् शक्तिलाभ इत्यर्थः\edlabel{pvv.403-2}\footnote{\label{pvv.403-2}  २ अपि च केचित् मन्त्र(ान्) कुर्व्वते न सर्व्व इति वदता पुरुषातिशय एव समर्थितः स्यात् ।}। (३०९)
	\pend
      \label{div_pvv.3.310}\edlabel{div_pvv.3.310}
	  
	% new div opening: depth here is 2
	

	  \pstart तस्मात् (।)
	\pend
      
	  \bigskip
	  \begingroup
	  \large
	
	    
	    \stanza[\smallbreak]
	\label{pv.3.310}\edlabel{pv.3.310}\flagstanza{\tiny\textenglish{....3.310}}कृतकः पौरुषेयाश्च मन्त्रा वाच्याः फलेप्सुना ।&अशक्तिसाधनं पुंसामनेनैव निराकृतम् ॥ ३१० ॥\&[\smallbreak]


	
	  \endgroup
	

	  \pstart कृतकाः पौरुषेयाश्च मन्त्रा वाच्याः । {\color{DodgerBlue3}“फलेप्सुना”} सर्व्वेण येन पुरुषाः शक्तिविशेषवन्तो मन्त्रान् प्रणेतुमीशते (।) \edlabel{pvv.403-3}\footnote{\label{pvv.403-3}  ३ मन्त्रकर्त्तुर्ज्ञानातिशयसाधनेन वस्तुबलायातस्य निवारणसाधनाभावात् ।} {\color{DodgerBlue3}“अनेनैव”} मन्त्रादिप्रणयनं प्रति {\color{DodgerBlue3}“पुंसामशक्तिसाधनं”} यत् किमपि मी मां स कैरिष्टं त{\color{DodgerBlue3}“न्निराकृतं”} बोद्धव्यं । (३१०)
	\pend
      \label{div_pvv.3.311}\edlabel{div_pvv.3.311}
	  
	% new div opening: depth here is 2
	

	  \pstart एतदेव स्फुटयितुमाह (।)
	\pend
      
	  \bigskip
	  \begingroup
	  \large
	
	    
	    \stanza[\smallbreak]
	\label{pv.3.311}\edlabel{pv.3.311}\flagstanza{\tiny\textenglish{....3.311}}बुद्धीन्द्रियोक्तिपुंस्त्वादिसाधनं यत्तु वर्ण्यते ।&प्रमाणाभं यथार्थास्ति न हि शेषवतो गतिः ॥ ३११ ॥\&[\smallbreak]


	
	  \endgroup
	

	  \pstart {\color{DodgerBlue3}“\edlabel{pvv.403-4}\footnote{\label{pvv.403-4}  ४ सत्त्वादिन्द्रियत्वाद् वचनात् पुंस्त्वन्तस्या(?)पुरुषवदित्यत आह । आदिना प्राण्यादिमत्वात् ।}बुद्धीन्द्रियोक्तिपुंस्त्वादिसाधनं”} शक्तिविशेषनिराकरणं {\color{DodgerBlue3}“यत्तु वर्ण्यते”} तत् {\color{DodgerBlue3}“प्रमाणाभमनै”}कान्तिकं (।) न हि बुद्धिमत एकस्य शक्तिविशेषो न दृष्ट इत्यन्यस्यापि तथा । अभ्यासाधीनो हि प्रकर्षो गुणेषु तदभिलाषादभ्यासोपि सम्भवी (।) दृश्यते च प्रज्ञादिगुणानामतिशयितः प्रकर्ष इति बुद्धिमत्वाद्यसाधनं ।\edlabel{pvv.403-5}\footnote{\label{pvv.403-5}  ५ विपक्षवृत्तेः सन्दे(हे)न सर्वस्य शेषवत्वात् ।} {\color{DodgerBlue3}“न हि शेषवतो”} लिङ्गात् (।) {\color{DodgerBlue3}“यथार्था”}नुमेयस्य {\color{DodgerBlue3}“गति”}र्भवितुमर्हति । (३११)
	\pend
      \label{div_pvv.3.312}\edlabel{div_pvv.3.312}
	  
	% new div opening: depth here is 2
	

	  \pstart b. अपि च (।) पुरुषातिशयमनिच्छतां जैमिनीयानां महत् दुःश्लिष्टं । तथा हि (।)
	\pend
      
	  \bigskip
	  \begingroup
	  \large
	
	    
	    \stanza[\smallbreak]
	\label{pv.3.312}\edlabel{pv.3.312}\flagstanza{\tiny\textenglish{....3.312}}अर्थोयं नायमर्थो न इति शब्दा वदन्ति न ।&कल्प्योयमर्थः पुरुषैस्ते च रागादिसंयुताः ॥ ३१२ ॥\&[\smallbreak]


	
	  \endgroup
	\leavevmode\marginnote{\textenglish{404/s}}

	  \pstart {\color{DodgerBlue3}“अयमर्थो”}ऽस्य शब्दस्यायं {\color{DodgerBlue3}“नेति”} न ताव{\color{DodgerBlue3}“च्छब्दाः”} स्वयं {\color{DodgerBlue3}“वदन्ति”} । तस्मात् {\color{DodgerBlue3}“पुरुषैरयमर्थः कल्पः । ते च रागादियुक्ता”} नावधार\edlabel{pvv.404-1}\footnote{\label{pvv.404-1}  १ तत्कल्पितो न प्रमाणं पुरुषातिशयाभावप्रतिज्ञापि विरोधिता त्वयं ।}णपटवः । (३१२)
	\pend
      \label{div_pvv.3.313}\edlabel{div_pvv.3.313}
	  
	% new div opening: depth here is 2
	
	  \bigskip
	  \begingroup
	  \large
	
	    
	    \stanza[\smallbreak]
	\label{pv.3.313}\edlabel{pv.3.313}\flagstanza{\tiny\textenglish{....3.313}}स एकस्तत्त्वविप्न्नान्य इति भेदश्च किंकृतः ।&तद्वत् पुंस्त्वे कथमपि ज्ञानी कश्चित् कथं न वः ॥ ३१३ ॥\&[\smallbreak]


	
	  \endgroup
	

	  \pstart अथ रागिष्वपि कश्चि ज्जै मि न्या दिर्जानात्येव । तत्रै{\color{DodgerBlue3}“कस्त\edlabel{pvv.404-2}\footnote{\label{pvv.404-2}  २ ज्ञानवानन्यो प्रसक्तः ।}त्त्वविदन्यो नेति भेदश्च किंकृत”} एषः । रागित्वाविशेषात् सर्व्व एवाज्ञः प्राप्तो न वा कश्चित् । अथ पुरुषत्वसाम्येपि जैमिन्यादिरतीन्द्रियार्थदर्शित्वाद् वैदिकशब्दानामतीन्द्रियार्थसम्बन्धवेत्ता कल्प्यते (।) तदा {\color{DodgerBlue3}“पुंस्त्वे”} सत्यपि कश्चिदन्योपि {\color{DodgerBlue3}“ज्ञानी”} ज्ञानातिशयवान् {\color{DodgerBlue3}“कथमप्य-”} भ्यासादिना {\color{DodgerBlue3}“तद्वज्जैमिन्या”}दिवत् । {\color{DodgerBlue3}“कथम्वो\edlabel{pvv.404-3}\footnote{\label{pvv.404-3}  ३ मन्त्रकर्त्तुर्ज्ञानातिशयसाधनेन वस्तुबलागतस्य निवारणे को दोषः (?) ।}”} मीमांसकानां नाभिमतः । (३१३)
	\pend
      \label{div_pvv.3.314}\edlabel{div_pvv.3.314}
	  
	% new div opening: depth here is 2
	

	  \begin{center}%% label @type='head'
	\textbf{(ख) आप्तलक्षणम्}
	\end{center}
	

	  \pstart स्यादेतन्न वयं पुरुषप्रमाण्याद् व्याख्यानमनुमन्यामहे । किन्तु (।)
	\pend
      
	  \bigskip
	  \begingroup
	  \large
	
	    
	    \stanza[\smallbreak]
	\label{pv.3.314}\edlabel{pv.3.314}\flagstanza{\tiny\textenglish{....3.314}}प्रमाणमविसंवादि वचनं सोर्थविद् यदि ।&न ह्यत्यन्तपरोक्षेषु प्रमाणस्यास्ति सम्भवः ॥ ३१४ ॥\&[\smallbreak]


	
	  \endgroup
	

	  \pstart {\color{DodgerBlue3}“यस्य प्रमाणमविसम्वादि वचनं सोर्थवित्”} वेदार्थज्ञाता {\color{DodgerBlue3}“यदी”}ष्यते । नन्व{\color{DodgerBlue3}“त्यन्त”}\leavevmode\marginnote{\textenglish{80a/MA}} {\color{DodgerBlue3}“परोक्षेषु”} वेदार्थेषु {\color{DodgerBlue3}“प्रमाणस्य सम्भवो न हि”} कस्यचि{\color{DodgerBlue3}“दस्ति”} तत्कथं सम्वादादर्थविद् व्यवस्थाप्यते । (३१४)
	\pend
      \label{div_pvv.3.315}\edlabel{div_pvv.3.315}
	  
	% new div opening: depth here is 2
	

	  \pstart अपि च\edlabel{pvv.404-4}\footnote{\label{pvv.404-4}  ४ बहुषु व्याख्यातृषु यः प्रमाणं प्रत्यक्षादि संस्यन्दयति तद्भाषितं गृह्यत इति ब्रुवताऽपौरुषेयत्वादागमलक्षणादन्यदेवमागमलक्षणं स्यादित्याह ।} (।)
	\pend
      
	  \bigskip
	  \begingroup
	  \large
	
	    
	    \stanza[\smallbreak]
	\label{pv.3.315}\edlabel{pv.3.315}\flagstanza{\tiny\textenglish{....3.315}}यस्य प्रमाणसंवादि वचनं तत्कृतं वचः ।&स आगम इति प्राप्तं निरर्थाऽपौरुषेयता ॥ ३१५ ॥\&[\smallbreak]


	
	  \endgroup
	

	  \pstart {\color{DodgerBlue3}“यस्य प्रमाणसम्वादि वचनं”} स यदि व्याख्याता तदा सम्वाद आगमलक्षणमिति {\color{DodgerBlue3}“तेन संस्कृतं\edlabel{pvv.404-5}\footnote{\label{pvv.404-5}  ५ प्रमाणानुगृहीतत्वे ख्यापनं संस्कृतत्वं ।} वच आगम इति\edlabel{pvv.404-6}\footnote{\label{pvv.404-6}  ६ न प्रकाशयति जैमिनिशवरस्वाम्यादिवैयर्थ्यासङ्गात् ।} प्राप्त”}मिति {\color{DodgerBlue3}“निरर्थाऽपौरुषेयता”} वेदानां कल्पिता । (३१५)
	\pend
      \label{div_pvv.3.316}\edlabel{div_pvv.3.316}
	  
	% new div opening: depth here is 2
	\leavevmode\marginnote{\textenglish{405/s}}

	  \begin{center}%% label @type='head'
	\textbf{(ग) परपक्षे दोषाः}
	\end{center}
	

	  \pstart a. तथा (।)
	\pend
      
	  \bigskip
	  \begingroup
	  \large
	
	    
	    \stanza[\smallbreak]
	\label{pv.3.316}\edlabel{pv.3.316}\flagstanza{\tiny\textenglish{....3.316}}यद्यत्यन्तपरोक्षेर्थेऽनागमज्ञानसम्भवः ।&अतीन्द्रियार्थवित् कश्चिदस्तीत्यभिमतं भवेत् ॥ ३१६ ॥\&[\smallbreak]


	
	  \endgroup
	

	  \pstart {\color{DodgerBlue3}“यद्यत्यन्तपरोक्षेर्थे”} स्वर्गसम्बन्धादौ {\color{DodgerBlue3}“जै मि न्या देरनागम”}स्यागमनिरपेक्ष\edlabel{pvv.405-1}\footnote{\label{pvv.405-1}  १ परोक्षे ।}स्य {\color{DodgerBlue3}“ज्ञानस्य सम्भवः”} । तदाऽ{\color{DodgerBlue3}“तीन्द्रियार्थ”}दर्शी\edlabel{pvv.405-2}\footnote{\label{pvv.405-2}  २ अनुमानस्याध्यक्षपूर्वत्वात् ।} {\color{DodgerBlue3}“कश्चिदस्तीत्यभिमतं भवेत्”} । ततस्तत्प्रतिक्षेपो न युक्तः । (३१६)
	\pend
      \label{div_pvv.3.317}\edlabel{div_pvv.3.317}
	  
	% new div opening: depth here is 2
	

	  \pstart यदि तु न कश्चिदतीन्द्रियार्थदर्शी तदा (।)
	\pend
      
	  \bigskip
	  \begingroup
	  \large
	
	    
	    \stanza[\smallbreak]
	\label{pv.3.317}\edlabel{pv.3.317}\flagstanza{\tiny\textenglish{....3.317}}स्वयं रागादिमान्नार्थं वेत्ति वेदस्य नान्यतः ।&न वेदयति वेदोपि वेदार्थस्य कुतो गतिः ॥ ३१७ ॥\&[\smallbreak]


	
	  \endgroup
	

	  \pstart {\color{DodgerBlue3}“स्वयं रागादिमान्”} पुमान् {\color{DodgerBlue3}“वेदस्यार्थं न वेत्ति न चान्यतः\edlabel{pvv.405-3}\footnote{\label{pvv.405-3}  ३ न ह्यन्धेनाकृष्यमाणोन्धोवगच्छति वर्त्म ।}”} अन्यस्यापि रागादिमत्वेऽज्ञानत्वात् । {\color{DodgerBlue3}“वेदोपि”} स्वार्थ {\color{DodgerBlue3}“न \edlabel{pvv.405-4}\footnote{\label{pvv.405-4}  ४ अज्ञातार्थत्वेन वेदस्य ।} वेदयति”} ततश्च {\color{DodgerBlue3}“कुतो वेदार्थस्य गतिः”} । न कुतश्चिदिति पुरुषा एव यथाप्रतिभं कल्पयेयुः । (३१७)
	\pend
      \label{div_pvv.3.318}\edlabel{div_pvv.3.318}
	  
	% new div opening: depth here is 2
	
	  \bigskip
	  \begingroup
	  \large
	
	    
	    \stanza[\smallbreak]
	\label{pv.3.318}\edlabel{pv.3.318}\flagstanza{\tiny\textenglish{....3.318}}तेनाग्निहोत्रं जुहुयात् स्वर्गकाम इति श्रुतौ ।&खादेच्छवमांसमित्येष नार्थ इत्यत्र का प्रमा ॥ ३१८ ॥\&[\smallbreak]


	
	  \endgroup
	

	  \pstart {\color{DodgerBlue3}“तेना“ग्निहोत्रं जुहुयात् स्वर्गकाम” इति श्रुतौ\edlabel{pvv.405-5}\footnote{\label{pvv.405-5}  ५ क्वचिदप्यर्थे प्रत्यासत्तिविप्रकर्षरहितायां ।}”} वेदवाक्ये {\color{DodgerBlue3}““श्वमांसं स्वादेदि”त्येष नार्थ इत्यत्र का प्रमा”} । इत्यपि कल्पयितुं शक्यत्वात्\edlabel{pvv.405-6}\footnote{\label{pvv.405-6}  ६ अनुमानस्याप्यध्यक्षपूर्व्वत्वात् ।}। (३१८)
	\pend
      \label{div_pvv.3.319}\edlabel{div_pvv.3.319}
	  
	% new div opening: depth here is 2
	 {\ b.}
	  \bigskip
	  \begingroup
	  \large
	
	    
	    \stanza[\smallbreak]
	\label{pv.3.319}\edlabel{pv.3.319}\flagstanza{\tiny\textenglish{....3.319}}प्रसिद्धो लोकवादश्चेत्तत्र कोतीन्द्रियार्थदृक् ।&अनेकार्थेषु शब्देषु येनार्थोयं विवेचितः ॥ ३१९ ॥\&[\smallbreak]


	
	  \endgroup
	

	  \pstart अग्निर्दाहादिसमर्थः तस्मिन् घृतादिप्रक्षेपश्च हवनमिति {\color{DodgerBlue3}“प्रसिद्धो लोकवादः”} ततो नान्यार्थकल्पनेति चेत् । {\color{DodgerBlue3}“तत्र”} लोके रागादिमति {\color{DodgerBlue3}“कोऽतीन्द्रियार्थदृगस्ति येनानेकार्थे”}ष्वनेकार्थाभिधानयोग्येषु {\color{DodgerBlue3}“शब्देष्व”}यमभिमतो{\color{DodgerBlue3}“\edlabel{pvv.405-7}\footnote{\label{pvv.405-7}  ७ अयमर्थो वेदस्य नान्य इति विभक्तो ।}र्थो विवेचितो”} वाच्यतया  विशेषणव्यवस्थापितः । यदि दाहकद्रव्ये घृतप्रक्षेपस्य स्वर्गेण सम्बन्धः श्वमांसभक्षणेन \leavevmode\marginnote{\textenglish{406/s}} च नास्तीति केनापि दृष्टं स्यात् स्यादयमर्थविवेकः । तद्दर्शी तु रागादिमत्वात् कश्चिन्नेष्ट इति नास्ति विपरीतकल्पनानिरासः । (३१९)
	\pend
      \label{div_pvv.3.320}\edlabel{div_pvv.3.320}
	  
	% new div opening: depth here is 2
	

	  \pstart न च लोकप्रसिद्धानुसाराद् वेदार्थव्यवस्था । तथा हि (।)
	\pend
      
	  \bigskip
	  \begingroup
	  \large
	
	    
	    \stanza[\smallbreak]
	\label{pv.3.320}\edlabel{pv.3.320}\flagstanza{\tiny\textenglish{....3.320}}स्वर्गोर्वश्यादिशब्दश्च दृष्टोरूढार्थवाचकः ।&शब्दान्तरेषु तादृक्षु तादृश्येवास्तु कल्पना ॥ ३२० ॥\&[\smallbreak]


	
	  \endgroup
	

	  \pstart {\color{DodgerBlue3}“स्वर्गोर्वश्यादिशब्दश्चारूढार्थवाचको\edlabel{pvv.406-1}\footnote{\label{pvv.406-1}  १ वेदवादिना कृतो ।} दृष्टः”} । मनुष्यातिशायिपुरुषविशेषनिकेतोऽतिमानुषसुखाधिष्ठानो नानोपकारणः स्वर्गः । तत्स्थाऽप्सरा उ र्व्व शी ति प्रसिद्धो लोकवादः\edlabel{pvv.406-2}\footnote{\label{pvv.406-2}  २ पौरुषेये नायं दोषः संप्रदायसम्भवात् । लोकसंकेतप्रसिद्धशब्दानुविधानात् । शब्दानामनेकार्थत्वेपि । पुरुषोपयोगिनमेवागमार्थं चतुःसत्यं निश्चिन्वन्ति सौगता न प्रवादमात्रं वृद्धस्येत्यतोप्यदोषः ॥ एतत् त्रयं वेदेनेति त्याज्यन्तदित्याह ।} । तमुल्लंध्य दुःखेनासम्भिन््ना निरतिशया प्रीतिः स्वर्गः । उर्व्वशी चारणिः पात्री वेत्यप्रसिद्धार्थकल्पनेति । {\color{DodgerBlue3}“शब्दान्तरे\edlabel{pvv.406-3}\footnote{\label{pvv.406-3}  ३ प्रदेशान्तरे वन्हौ घृतहवनमुक्तं ततो निश्चय इत्यपि न तस्याप्यसिद्धार्थत्वात् । श्वमांसकल्पनैवास्तु ।}”}ष्वग्निहोत्रादिषु {\color{DodgerBlue3}“तादृक्षु”} सर्व्वार्थयोग्येषु {\color{DodgerBlue3}“तादृश्येवा”}प्रसिद्धार्थैव {\color{DodgerBlue3}“कल्पनास्तु”} न्यायस्य तुल्यत्वात् । (३२०)
	\pend
      \label{div_pvv.3.321}\edlabel{div_pvv.3.321}
	  
	% new div opening: depth here is 2
	

	  \pstart किञ्च (।)
	\pend
      
	  \bigskip
	  \begingroup
	  \large
	
	    
	    \stanza[\smallbreak]
	\label{pv.3.321}\edlabel{pv.3.321}\flagstanza{\tiny\textenglish{....3.321}}प्रसिद्धश्च नृणां वादः प्रमाणं स च नेष्यते ।&ततश्च भूयोर्थगतिः किमेतद् द्विष्ठकामितम् ॥ ३२१ ॥\&[\smallbreak]


	
	  \endgroup
	

	  \pstart {\color{DodgerBlue3}“प्रसिद्धश्च नृणाम्वाद”} एव न त्वन्यथा काचित् । {\color{DodgerBlue3}“स  च”} पुरुषप्रवर्त्तितत्वात् {\color{DodgerBlue3}“प्रमाणं नेष्यते”} भवता । {\color{DodgerBlue3}“भूयः”} पुन{\color{DodgerBlue3}“स्ततो”} जनवादा{\color{DodgerBlue3}“दर्थ”}स्य {\color{DodgerBlue3}“गति”}रिति {\color{DodgerBlue3}“किमेतत्”} द्विष्ठकामितं । लोकवादोऽप्रमाणत्वात् द्विष्ठः संप्रति काम्यते इति व्यक्तो विरोधः । (३२१)
	\pend
      \label{div_pvv.3.322}\edlabel{div_pvv.3.322}
	  
	% new div opening: depth here is 2
	
	  \bigskip
	  \begingroup
	  \large
	
	    
	    \stanza[\smallbreak]
	\label{pv.3.322}\edlabel{pv.3.322}\flagstanza{\tiny\textenglish{....3.322}}अथ प्रसिद्धिमुल्लंध्य कल्पने न निबन्धनम् ।&प्रसिद्धेरप्रमाणत्वात् तद्ग्रहे किं निबन्धनम् ॥ ३२२ ॥\&[\smallbreak]


	
	  \endgroup
	

	  \pstart {\color{DodgerBlue3}“अथ”} लोकस्य {\color{DodgerBlue3}“प्रसिद्धिमुल्लंध्य”} श्वमांसभक्षणाद्यर्थ{\color{DodgerBlue3}“कल्पने न निबन्धनमस्ति । ननु प्रसिद्धेरप्रमाणत्वात्”} तस्या {\color{DodgerBlue3}“ग्रहे\edlabel{pvv.406-4}\footnote{\label{pvv.406-4}  ४ प्रसिद्धिग्रहोपि मा भूत् ।} किं निबन्धनं”} विचारकस्य न किञ्चित् । (३२२)
	\pend
      \label{div_pvv.3.323}\edlabel{div_pvv.3.323}
	  
	% new div opening: depth here is 2
	

	  \pstart किञ्च (।)
	\pend
      
	  \bigskip
	  \begingroup
	  \large
	
	    
	    \stanza[\smallbreak]
	\label{pv.3.323}\edlabel{pv.3.323}\flagstanza{\tiny\textenglish{....3.323}}उत्पादिता प्रसिद्ध्यैव शङ्का शब्दार्थनिश्चये ।&यस्मान्नानार्थवृत्तित्वं शब्दानां तत्र दृश्यते ॥ ३२३ ॥\&[\smallbreak]


	
	  \endgroup
	\leavevmode\marginnote{\textenglish{407/s}}

	  \pstart प्रसिद्ध्यैव शब्दार्थनिश्चये विपरीतार्थकल्पनाया शङ्कोत्पादिता । यस्मान्नानार्थवृत्तित्वं शब्दानां गवाक्षप्रभृतीनां तत्र प्रसिद्धौ दृश्यते (। ३२३)
	\pend
      \label{div_pvv.3.324}\edlabel{div_pvv.3.324}
	  
	% new div opening: depth here is 2
	

	  \pstart c. तथा (।)
	\pend
      
	  \bigskip
	  \begingroup
	  \large
	
	    
	    \stanza[\smallbreak]
	\label{pv.3.324}\edlabel{pv.3.324}\flagstanza{\tiny\textenglish{....3.324}}अन्यथासम्भवाभावात् नानाशक्तेः स्वयं ध्वनेः ।&अवश्यं शङ्कया भाव्यं नियामकमपश्यताम् ॥ ३२४ ॥\&[\smallbreak]


	
	  \endgroup
	

	  \pstart {\color{DodgerBlue3}“नानाशक्ते”}रनेकार्थप्रतिपादनयोग्यस्य {\color{DodgerBlue3}“ध्वनेः स्वय”}मात्मनाऽन्यथासम्भवस्य एकार्थप्रतिपादन\edlabel{pvv.407-1}\footnote{\label{pvv.407-1}  १ साध्यार्थादन्यत्र वृत्तिरन्यथा । तदसम्भवो यस्तस्याभावादिति द्वौ निषेधौ सम्भवोन्यथापि ।} योग्यतासम्भवस्यासम्भवात् । {\color{DodgerBlue3}“अवश्य”}मर्थान्तर{\color{DodgerBlue3}“शङ्कया भाव्यं”} । कथमित्याह । {\color{DodgerBlue3}“नियामकमपश्यताम”}नेकार्थस्य शब्दस्य एकवृत्तिनियमकारणमपश्यतां पुंसां ।\edlabel{pvv.407-2}\footnote{\label{pvv.407-2}  २ यत एव तस्मादज्ञातार्थशब्देष्वप्रमाणकार्यारोपन्निश्चित्य व्याचक्षाणो जैमिनिस्तद्व्याजेन स्वमतमेवाहेत्यर्थं दर्शयन्नाह ।}(३२४)
	\pend
      \label{div_pvv.3.325}\edlabel{div_pvv.3.325}
	  
	% new div opening: depth here is 2
	

	  \pstart d. किञ्च (।)
	\pend
      
	  \bigskip
	  \begingroup
	  \large
	
	    
	    \stanza[\smallbreak]
	\label{pv.3.325}\edlabel{pv.3.325}\flagstanza{\tiny\textenglish{....3.325}}एष स्थाणुरयं मार्ग इति वक्तीति कश्चन ।&अन्यः स्वयं ब्रवीमीति तयोर्भेदः परीक्ष्यताम् ॥ ३२५ ॥\&[\smallbreak]


	
	  \endgroup
	

	  \pstart स्वयमर्थप्रतिपादनशून्यो नेता\edlabel{pvv.407-3}\footnote{\label{pvv.407-3}  ३ वैदिकशब्दान्...वैकत्वे नियमः ।}नभिमतार्थद्योतकत्वेन प्रतिपादयन् कर्त्तुर्न्न\leavevmode\marginnote{\textenglish{80b/MA}} भिद्यते । तस्याप्येवं व्यापारत्वात् । यथा केनचित् {\color{DodgerBlue3}“कश्चन\edlabel{pvv.407-4}\footnote{\label{pvv.407-4}  ४ पाटलिपुत्रस्य ।}”} पन्थानं पृष्ठ आह । नाहं जाने किन्तु {\color{DodgerBlue3}“स्थाणुरेष\edlabel{pvv.407-5}\footnote{\label{pvv.407-5}  ५ मार्गोपदेशसामर्थ्यशून्यस्थाणुव्याजेन मार्गमाह ।} वक्ति (।) अयं मार्ग इति । अन्यः”} पुनः प्रत्याह (।) अहं {\color{DodgerBlue3}“स्वयं ब्रवीमि”} (।) {\color{DodgerBlue3}“अयं”} मार्ग {\color{DodgerBlue3}“इति”} (।) {\color{DodgerBlue3}“तयोरेवं”} वादिनोः प्रतिपादकत्वस्य {\color{DodgerBlue3}“भेदः परीक्ष्यतां”} (।) एको निरभिप्राय\edlabel{pvv.407-6}\footnote{\label{pvv.407-6}  ६ स्थाणोः ।}वचनं वक्तृत्वेन व्यपदिशति । अन्यस्तु नेति वचनवैदग्ध्य\edlabel{pvv.407-7}\footnote{\label{pvv.407-7}  ७ जडस्य प्रतिपत्तिमान्द्यादन्यत्र विशेषो नानयोः ।}मेवानयोर्भिद्यते नोपदेशप्रवृत्तिनिवृत्ती । तथा स्वयमर्थ प्रतिपादयतोपि वेदार्थविशेषाभिधायिनोऽभिदधत् कर्त्तंव वाचकतायाः । (३२५)
	\pend
      \label{div_pvv.3.326}\edlabel{div_pvv.3.326}
	  
	% new div opening: depth here is 2
	

	  \pstart e. अथ तदर्थप्रतिपादने योग्या एव शब्दाः । नन्वेवं (।)
	\pend
      
	  \bigskip
	  \begingroup
	  \large
	
	    
	    \stanza[\smallbreak]
	\label{pv.3.326}\edlabel{pv.3.326}\flagstanza{\tiny\textenglish{....3.326}}सर्व्वत्र योग्यस्यैकार्थद्योतने नियमः कुतः ।&ज्ञाता वातीन्द्रियाः केन विवक्षावचनादृते ॥ ३२६ ॥\&[\smallbreak]


	
	  \endgroup
	\leavevmode\marginnote{\textenglish{408/s}}

	  \pstart {\color{DodgerBlue3}“सर्व्वत्रार्थे योग्यस्य”} शब्दस्य {\color{DodgerBlue3}“एकार्थद्योतेने”} नियमः {\color{DodgerBlue3}“कुतो”} जातः (।) पुरुषश्चेन्नियामको न भवति । स्यादेतत् (।) प्रतिनियता एवार्थास्तेषां न ते पुरुषेण नियम्यन्ते । एवमपि \edlabel{pvv.408-1}\footnote{\label{pvv.408-1}  १ तथापि तं ज्ञातुमशक्तः पुरुषः ।} {\color{DodgerBlue3}“ज्ञाता वा अतीन्द्रियाः”} प्रतिनियता अर्थाः\edlabel{pvv.408-2}\footnote{\label{pvv.408-2}  २ पुरुषातिशयानिष्टेः ।} {\color{DodgerBlue3}“केन”} पुरुषेण रागादिमता {\color{DodgerBlue3}“विवक्षा”}या(ः) प्रकाशकात् {\color{DodgerBlue3}“वचनात्”} प्रतिपादना{\color{DodgerBlue3}“दृते (।) न”} हि वेदेषु कस्यचिद् विवक्षास्ति (।) तदवगतिमन्तरेण च प्रतिनियतार्थता न शक्या बोद्धुं । न हि शब्दाः स्वयं स्वार्थनियमं कथयन्ति संकेतमन्तरेण(।) स च पौरुषेयः । न चार्थनियमावगमं विना संकेतकरणं अर्थनियमप्रतीतिश्च  विना संकेतं नास्तीति {\color{DodgerBlue3}“व्यक्त”}मितरेतराश्रयत्वं । (३२६)
	\pend
      \label{div_pvv.3.327}\edlabel{div_pvv.3.327}
	  
	% new div opening: depth here is 2
	

	  \pstart f. अपि च (।)
	\pend
      
	  \bigskip
	  \begingroup
	  \large
	
	    
	    \stanza[\smallbreak]
	\label{pv.3.327}\edlabel{pv.3.327}\flagstanza{\tiny\textenglish{....3.327}}विवक्षानियमे हेतुः सङ्केतस्तत्प्रकाशनः ।&अपौरुषेये सा नास्ति तस्य सैकार्थता कुतः ॥ ३२७ ॥\&[\smallbreak]


	
	  \endgroup
	

	  \pstart {\color{DodgerBlue3}“विवक्षार्थ”}स्य {\color{DodgerBlue3}“नियमे हेतुः संकेतः तत्प्रकाशनो”}ऽर्थनियमबोधनः । {\color{DodgerBlue3}“अपौरुषेये च”} वेदे {\color{DodgerBlue3}“सा”} विवक्षा नास्ति पुरुषप्रयोज्यत्वात् तस्याः । ततस्तत्प्रसाध्या सा {\color{DodgerBlue3}“एकार्थता”} वेदस्य {\color{DodgerBlue3}“कुतः”} । (२२७)
	\pend
      \label{div_pvv.3.328}\edlabel{div_pvv.3.328}
	  
	% new div opening: depth here is 2
	

	  \pstart अथ (।)
	\pend
      
	  \bigskip
	  \begingroup
	  \large
	
	    
	    \stanza[\smallbreak]
	\label{pv.3.328}\edlabel{pv.3.328}\flagstanza{\tiny\textenglish{....3.328}}स्वभावनियमेन्यत्र न योज्येत तया पुनः ।&सङ्केतश्च निरर्थः स्याद् व्यक्तौ च नियमः कुतः ॥ ३२८ ॥\&[\smallbreak]


	
	  \endgroup
	

	  \pstart {\color{DodgerBlue3}“स्वभावतः”} शब्दाः क्वचिदर्थे नियता\edlabel{pvv.408-3}\footnote{\label{pvv.408-3}  ३ न विवक्षातः ।}स्तदा स्वभावनियमे\edlabel{pvv.408-4}\footnote{\label{pvv.408-4}  ४ स शब्दो यथेष्टं सम्वादितं तुल्यरसमिति ।} {\color{DodgerBlue3}“ऽन्यत्रार्थे”} वाचकत्वेन {\color{DodgerBlue3}“तया”} विवक्षया {\color{DodgerBlue3}“न योज्येत पुनः”} । न हि स्वभावेन प्रतिनियताविषयोन्यथा कर्तुं शक्यते\edlabel{pvv.408-5}\footnote{\label{pvv.408-5}  ५ चक्षुरिव रूपे ।} । {\color{DodgerBlue3}“संकेतश्च निरर्थः स्यात्”} स्वभावप्रतिनियमे सति न हि चक्षुरिन्द्रियं रूपग्रहणप्रतिनियतं तत्र सङ्के\edlabel{pvv.408-6}\footnote{\label{pvv.408-6}  ६ संकेतापेक्षप्रतीतयस्तु स्वप्रतीत्यर्थसंकेतितराजचिह्नवत् ।}तमपेक्षते ।\edlabel{pvv.408-7}\footnote{\label{pvv.408-7}  ७ वैदिकार्थो निसर्गसिद्धः संकेतेन व्यज्यत इत्याह ।} सङ्केताद् वाचकताया\edlabel{pvv.408-8}\footnote{\label{pvv.408-8}  ८ स्वभावविशेषस्य ।} {\color{DodgerBlue3}“व्यक्ता”}विष्यमाणायां {\color{DodgerBlue3}“नियमः कुतः”} (।) अयमेवास्य शब्दस्यार्थ इति नियमो न युज्यते पुरुषाधीनसङ्केतापेक्षायां । (३२८)
	\pend
      \label{div_pvv.3.329}\edlabel{div_pvv.3.329}
	  
	% new div opening: depth here is 2
	

	  \pstart तथाहि (।)
	\pend
      \leavevmode\marginnote{\textenglish{409/s}}
	  \bigskip
	  \begingroup
	  \large
	
	    
	    \stanza[\smallbreak]
	\label{pv.3.329}\edlabel{pv.3.329}\flagstanza{\tiny\textenglish{....3.329}}यत्र स्वातन्त्र्यमिच्छाया नियमो नाम तत्र कः ।&द्योतयेन् तेन सङ्केतो नेष्टामेवास्य योग्यताम् ॥ ३२९ ॥\&[\smallbreak]


	
	  \endgroup
	

	  \pstart यत्र\edlabel{pvv.409-1}\footnote{\label{pvv.409-1}  १ संकेते ।} पुरुषस्य स्वात न्त्र्यमिच्छायास्तत्र नियमो नाम कः सङ्गतः । \edlabel{pvv.409-2}\footnote{\label{pvv.409-2}  २ अनियतत्वेन ।}तेनेच्छाधीनत्वेन संकेतो नेष्टामेव\edlabel{pvv.409-3}\footnote{\label{pvv.409-3}  ३ वैदिकशब्दस्य ।}योग्यतामुद्द्योतयेत् । अनभिमतामपि व्यञ्जयेदित्यर्थः । तदेवमपौरुषेयत्वमागमलक्षणं ब्रुवाणा मी मां स काः प्रतिक्षिप्ताः । (३२९)
	\pend
      \label{div_pvv.3.330}\edlabel{div_pvv.3.330}
	  
	% new div opening: depth here is 2
	

	  \begin{center}%% label @type='head'
	\textbf{(२ ) b. बुद्धमीमांसक(जैमिनि)मतनिरासः}
	\end{center}
	

	  \begin{center}%% label @type='head'
	\textbf{क. वेदैंकदेशसंवादित्वे न सर्वस्य प्रामाण्यम्}
	\end{center}
	

	  \pstart \edlabel{pvv.409-4}\footnote{\label{pvv.409-4}  ४ तदेवमपौरुषेयत्वं नागमलक्षणमित्युक्त्वा एकदेशाविसम्वादनमागमलक्षणं दूषयितुमाह ।}संप्रति बृ द्ध मी मां स कानां मतं दूषयितुमुत्थापयति ।
	\pend
      
	  \bigskip
	  \begingroup
	  \large
	
	    
	    \stanza[\smallbreak]
	\label{pv.3.330}\edlabel{pv.3.330}\flagstanza{\tiny\textenglish{....3.330}}यस्मात् किलेदृशं सत्यं यथाग्निः शीतनोदनः ।&वाक्यं वेदैकदेशत्वादन्यदप्यपरोऽव्रवीत् ॥ ३३० ॥\&[\smallbreak]


	
	  \endgroup
	

	  \pstart \edlabel{pvv.409-5}\footnote{\label{pvv.409-5}  ५ यस्मात् सत्यं तन्न दूष्यमिति परः} यथा वैदिकं वाक्य“मग्निर्हिमस्य भेषजमि”ति {\color{DodgerBlue3}“सत्यं शीतनोदनत्वेन वह्नेः”} प्रमाणसिद्धत्वात् तथाऽन्न्यदपि\edlabel{pvv.409-6}\footnote{\label{pvv.409-6}  ६ विशेषस्य पक्षीकरणात् साध्यमाह । यत्र बौद्धस्य विप्रतिपत्तिरेकदेशे तदवितथमिति ।} {\color{DodgerBlue3}“वाक्य”}“मग्निष्टोमेन जुहुयात् स्वर्गकाम” इत्यादि (।) {\color{DodgerBlue3}“वेदैकदेशत्वात्\edlabel{pvv.409-7}\footnote{\label{pvv.409-7}  ७ हेतुः सामान्यं प्रतिज्ञार्थतानिरासाय ।}”} {\color{DodgerBlue3}“सत्य”}मित्य{\color{DodgerBlue3}“परो”} मी मां स को\edlabel{pvv.409-8}\footnote{\label{pvv.409-8}  ८ वृद्धः चक्षुर्दोषोपहतत्वात् ।}ऽब्रवीदुक्तवान् (।) {\color{DodgerBlue3}“किल”} शब्दोऽक्षमायां । (३३०)
	\pend
      \label{div_pvv.3.331}\edlabel{div_pvv.3.331}
	  
	% new div opening: depth here is 2
	
	  \bigskip
	  \begingroup
	  \large
	
	    
	    \stanza[\smallbreak]
	\label{pv.3.331}\edlabel{pv.3.331}\flagstanza{\tiny\textenglish{....3.331}}रसवत् तुल्यरूपत्वादेकभाण्डे च पाकवत् ।&शेषवद् व्यभिचारित्वात् क्षिप्तं न्यायविदेदृशम् ॥ ३३१ ॥\&[\smallbreak]


	
	  \endgroup
	

	  \pstart ईदृशमनुमानं {\color{DodgerBlue3}“शेषवद”}नैकान्तिकं {\color{DodgerBlue3}“व्यभिचारित्वान्न्यायविदाचार्य”}\edlabel{pvv.409-9}\footnote{\label{pvv.409-9}  ९ नैयायिकशेषवदनुमानव्यभिचारमुद्भावयता (प्रमाण) समुच्चये ।}दि ग्नागे न प्रति{\color{DodgerBlue3}“क्षिप्तं रसवत् तुल्यरूपत्वात्”} । स्वादितफलेन तुल्यरूपत्वात् फलान्तरस्य तादृग्\edlabel{pvv.409-10}\footnote{\label{pvv.409-10}  १० अस्वादितं तुल्यरसमिति ।}रसानुमानवत् । {\color{DodgerBlue3}“एकभाण्डा”}न्तर्गत्वात् दृष्ट{\color{DodgerBlue3}“पाक”}तण्डुला\edlabel{pvv.409-11}\footnote{\label{pvv.409-11}  ११ एकभाण्डे पचनात् । अदृष्टा अपि तण्डुलाः पक्वा इति साध्यं ।}नुमानवत् ।\leavevmode\marginnote{\textenglish{81a/MA}} \leavevmode\marginnote{\textenglish{410/s}} न ह्येतादृशस्य हेतोः साध्यप्रतिबन्धोस्ति विपर्ययबाधकप्रमाणादिति विपञ्चितं प्राक् \edlabel{pvv.410-1}\footnote{\label{pvv.410-1}  १ “यस्यादर्शनमात्रेण व्यतिरेकः प्रदर्श्यत” \cref{pv.3.13} इत्यादिना ।}। (३३१)
	\pend
      \label{div_pvv.3.332_3.333_3.334}\edlabel{div_pvv.3.332_3.333_3.334}
	  
	% new div opening: depth here is 2
	

	  \begin{center}%% label @type='head'
	\textbf{ख. वेदप्रामाण्यघोषणा जैमिनेर्धार्ष्ट्यम्}
	\end{center}
	

	  \pstart दृश्यते च बहुतरविषये विसम्वादो वेदस्य कथमेकसत्यतया सर्व्वत्र तथात्वं । तथा \edlabel{pvv.410-2}\footnote{\label{pvv.410-2}  २ शक्यपरिच्छेदेपि विषये प्रमाणविरोधादयुक्तत्वेमेवाह ।} च (।)
	\pend
      
	  \bigskip
	  \begingroup
	  \large
	
	    
	    \stanza[\smallbreak]
	\label{pv.3.332}\edlabel{pv.3.332}\flagstanza{\tiny\textenglish{....3.332}}नित्यस्य पुंसः कर्तृत्वं नित्यान् भावानतीन्द्रियान् ।&ऐन्द्रियान्विषमं हेतुं भावानां विषमां स्थितिम् ॥ ३३२ ॥\&[\smallbreak]


	
	  \endgroup
	
	  \bigskip
	  \begingroup
	  \large
	
	    
	    \stanza[\smallbreak]
	\label{pv.3.333}\edlabel{pv.3.333}\flagstanza{\tiny\textenglish{....3.333}}निवृत्तिं च प्रमाणाभ्यामन्यद् वा व्यस्तगोचरम् ।&विरुद्धमागमापेक्षेणानुमानेन वा वदत् ॥ ३३३ ॥\&[\smallbreak]


	
	  \endgroup
	
	  \bigskip
	  \begingroup
	  \large
	
	    
	    \stanza[\smallbreak]
	\label{pv.3.334}\edlabel{pv.3.334}\flagstanza{\tiny\textenglish{....3.334}}विरोधमसमाधाय शास्त्रार्थं चाप्रदर्श्य सः ।&सत्यार्थं प्रतिज्ञानानो जयेद् धाष्टर्येन बन्धकीम् ॥ ३३४ ॥\&[\smallbreak]


	
	  \endgroup
	

	  \pstart परैरनाधेयविशेषस्य {\color{DodgerBlue3}“पुंस”} आत्मनः क्रमजन्मसु कर्मादिषु \edlabel{pvv.410-3}\footnote{\label{pvv.410-3}  ३ तत्फलभोक्तृत्वं ।} {\color{DodgerBlue3}“कर्तृत्वं”} वद\edlabel{pvv.410-4}\footnote{\label{pvv.410-4}  ४ वददिति सर्वत्र योज्यं त्रिभिः श्लोकैरत्र सम्बन्धः । “सुखदुःखादिसम्वित्तिसमवायस्तु भोक्तृता” । तन्न नित्यानां कार्यकरणत्वस्य निरस्तत्वात् ।}च्छास्त्रं । तथा {\color{DodgerBlue3}“नित्यान् भावान्”} दिक्कालाकाशादीनर्थक्रियारहितान् सतो \edlabel{pvv.410-5}\footnote{\label{pvv.410-5}  ५ अक्षणिकस्य क्रमयौगपद्यार्थक्रियाविरोधने वस्तुधर्मातिक्रमादयुक्तं तत् ।} वदत् । वस्तुतो{\color{DodgerBlue3}“ऽतीन्द्रियान्”} गुणकर्मसामान्यादीन् {\color{DodgerBlue3}“ऐन्द्रियान्”} \edlabel{pvv.410-6}\footnote{\label{pvv.410-6}  ६ तदयुक्तमनध्यक्षस्याध्यक्षत्वविरोधात् ।} प्रत्यक्षान् वदत् । तथा {\color{DodgerBlue3}“भावानां विषमं हेतुं”} प्राग\edlabel{pvv.410-7}\footnote{\label{pvv.410-7}  ७ अर्थक्रियाः प्रागजनकं पश्चात् सहकार्यपेक्षया जनकं तदयुक्तं नित्यस्याविशेषात् ।}जनकं वदत् । तथा {\color{DodgerBlue3}“विषमां स्थितिं निस्प”} (? ष्प) न्नानां भावानामाश्रयवशेन \edlabel{pvv.410-8}\footnote{\label{pvv.410-8}  ८ संश्च सर्व्वनिराशंसो भावः कथमपेक्षते परं ।} {\color{DodgerBlue3}“स्थितिं”} वदत् । {\color{DodgerBlue3}“निवृत्तिञ्च विषमां”} स्वतोऽनश्वरस्वभावा\edlabel{pvv.410-9}\footnote{\label{pvv.410-9}  ९ विनाशस्याहेतुत्वोक्तेः ।}नामन्यकृतां निवृत्तिं वदत् । {\color{DodgerBlue3}“अन्यद्वा”} \edlabel{pvv.410-10}\footnote{\label{pvv.410-10}  १० अन्यदप्यभावादिकं सदित्याह ।}वस्तु {\color{DodgerBlue3}“व्यस्तगोचरं प्रमाणाभ्यां”} प्रत्यक्षानुमानाभ्यां निरस्तावकाशं वदत् । {\color{DodgerBlue3}“आगमापेक्षेणा”}गमसिद्धलिङ्ग\leavevmode\marginnote{\textenglish{411/s}} त्रैरूप्येणा{\color{DodgerBlue3}“नुमानेन”}  च विरुद्धमग्निहोत्रस्नानादेः पापशमनं {\color{DodgerBlue3}“वदत्”} । \edlabel{pvv.411-1}\footnote{\label{pvv.411-1}  १ आगमाश्रयानुमानबाधितमेतदित्याह ।} न हि रागादिप्रभवो धर्मस्तदपनयनमन्तरेण स्नानादेर्निवृत्तिमर्हति । एवम्विधविसम्वादभाजमर्थं वदत् शास्त्रं {\color{DodgerBlue3}“विरोधं”} \edlabel{pvv.411-2}\footnote{\label{pvv.411-2}  २ शक्यविचारे वस्तुनि ।} प्रामाणिक{\color{DodgerBlue3}“मसमाधाय”} अपरिह्यत्य प्रवृत्तिकामोचितं {\color{DodgerBlue3}“शास्त्रार्थमनु”}\edlabel{pvv.411-3}\footnote{\label{pvv.411-3}  ३ पुरुषप्रवृत्तिनिमित्तं ।}गुणोपायं पुरुषार्थलक्षण{\color{DodgerBlue3}“ञ्चाप्रदर्श्य स”} जर न्मी मां स क एकदेशसम्वाददर्शनात् {\color{DodgerBlue3}“सत्यार्थं प्रतिजानानो धार्ष्ट्येन बन्धकीं”} साक्षाद्दृष्टव्यलीकां स्वपतिं स्वशीलप्रामाण्योद्भावनेन भ्रान्तमावेदयन्तीं {\color{DodgerBlue3}“जयेत्”} । (३३२-३३४)
	\pend
      \label{div_pvv.3.335}\edlabel{div_pvv.3.335}
	  
	% new div opening: depth here is 2
	

	  \pstart किञ्च (।)
	\pend
      
	  \bigskip
	  \begingroup
	  \large
	
	    
	    \stanza[\smallbreak]
	\label{pv.3.335}\edlabel{pv.3.335}\flagstanza{\tiny\textenglish{....3.335}}सिध्येत् प्रमाणं यद्येवमप्रमाणमथेह किम् ।&न ह्येकं नास्ति सत्यार्थं पुरुषे बहुभाषिणि ॥ ३३५ ॥\&[\smallbreak]


	
	  \endgroup
	

	  \pstart {\color{DodgerBlue3}“यद्येवं\edlabel{pvv.411-4}\footnote{\label{pvv.411-4}  ४ सर्व्वः पुरुषः सर्व्वत्रार्थे प्रमाणं स्यादित्याह ।}”} दृष्टैकदेशसम्वादस्यावयवत्वादन्यदपि {\color{DodgerBlue3}“प्रमाणं सिध्येत् । अथेह”} न्याये {\color{DodgerBlue3}“किमप्रमाणं”} शास्त्रं भविष्यति । {\color{DodgerBlue3}“न हि बहुभाषिणि पुरुषे”} सत्यार्थ{\color{DodgerBlue3}“मेकं”} वचे {\color{DodgerBlue3}“नास्ति”} किन्त्वस्त्येव । तेनैव दृष्टान्तेन तद्वचनं प्रमाणं स्यात् । (३३५)
	\pend
      \label{div_pvv.3.336}\edlabel{div_pvv.3.336}
	  
	% new div opening: depth here is 2
	
	  \bigskip
	  \begingroup
	  \large
	
	    
	    \stanza[\smallbreak]
	\label{pv.3.336}\edlabel{pv.3.336}\flagstanza{\tiny\textenglish{....3.336}}नायं स्वभावः कार्यं वा वस्तूनां वक्तरि ध्वनिः ।&न च तद्व्यतिरिक्तस्य विद्यतेव्यभिचारिता ॥ ३३६ ॥\&[\smallbreak]


	
	  \endgroup
	

	  \pstart {\color{DodgerBlue3}“नायं”} ध्वनिर्व्व{\color{DodgerBlue3}“स्तूनां स्वभावः । कार्यम्वा \edlabel{pvv.411-5}\footnote{\label{pvv.411-5}  ५ विना वाच्यं न शब्दवृत्तिश्चेत् ।}”} यस्माद् {\color{DodgerBlue3}“वक्तरि ध्वनिर्भवति”} । न हि वस्तुनः स्वभावोन्यत्र धर्मिणि वर्त्तते । कार्यम्वाऽन्यतो भवितुमर्हति । {\color{DodgerBlue3}“न च तद्व्यति रिक्तस्य”} कार्यस्वभावाभ्यामपरस्या{\color{DodgerBlue3}“व्यभिचारि”}ता {\color{DodgerBlue3}“विद्यत”} इति निवेदितं । (३३६)
	\pend
      \label{div_pvv.3.337}\edlabel{div_pvv.3.337}
	  
	% new div opening: depth here is 2
	

	  \pstart स्यादेतद् (।)
	\pend
      
	  \bigskip
	  \begingroup
	  \large
	
	    
	    \stanza[\smallbreak]
	\label{pv.3.337}\edlabel{pv.3.337}\flagstanza{\tiny\textenglish{....3.337}}प्रवृत्तिर्वाचकानाञ्च वाच्यदृष्टिकृतेति चेत् ।&परस्परविरुद्धार्था कथमेकत्र सा भवेत् ॥ ३३७ ॥\&[\smallbreak]


	
	  \endgroup
	

	  \pstart {\color{DodgerBlue3}“वाचकानां”} शब्दानां {\color{DodgerBlue3}“प्रवृत्तिर्व्वाच्यदृष्टिकृता”} अभिधेयदर्शनागता ततः परंपरया\edlabel{pvv.411-6}\footnote{\label{pvv.411-6}  ६ पदानां सङ्गतिः सम्बन्धः । शक्यसाधन उपायोनुगुणः । अभ्युदयादिः पुरुषार्थ इति शास्त्रधर्मा प्रदर्श्य विरोधमसमाधाय चात्यन्तप्रसिद्धसत्यार्थतामात्रेण प्रज्ञाप्रकर्षेणापि दुरवगाहेपि सत्यार्थतां साधयन् दुश्चारिणीं जयेत् । “सा स्वामिना परेण सङ्गता त्वमित्युपालब्धाऽह । पश्यत पुंसो वैपरीत्यं धर्मपत्न्या प्रत्ययमकृत्वा स्वनेत्रबुद्बुदयोः प्रत्येति । जरत्काणग्राम्यकाष्टहारेण प्रार्थिताऽसङ्गता । रूपगुणानुरागेण किल मन्त्रिमुख्यदारकं कामयेहमिति तुल्यं दृष्टविरोधस्यातिपरोक्षेऽविसम्वादानुमानेन” \href{http://http://sarit.indology.info/?cref=pvsv}{(स्ववृत्तौ)} ।}\leavevmode\marginnote{\textenglish{412/s}} तत्कार्यतैवैषामि{\color{DodgerBlue3}“ति चेत्”} । एवं तर्हि सा शब्दप्रवृत्ति{\color{DodgerBlue3}“रेकत्र”} वस्तुनि {\color{DodgerBlue3}“परस्परविरुद्धार्था कथम्भवेत्”} । (३३७)
	\pend
      \label{div_pvv.3.338}\edlabel{div_pvv.3.338}
	  
	% new div opening: depth here is 2
	

	  \pstart यदि यथा\edlabel{pvv.412-1}\footnote{\label{pvv.412-1}  १ वाच्यार्थस्य गमको हि तज्जन्यस्तत्स्वभावो वा स्यादित्याह शब्दस्त्विच्छायत्तो न बहिरधीनः (।) सति वाच्ये तद्दर्शनं दृष्टेर्विवक्षा । ततो वचनं परम्परा(?)} वस्त्वेव शब्दस्तदा वस्तुत एकरूपेण\edlabel{pvv.412-2}\footnote{\label{pvv.412-2}  २ आगम्यते तेन ।} शब्दे नित्यः किमयमनित्यो वेत्यादि शब्दसन्दर्भो न भवेत् ।
	\pend
      
	  \bigskip
	  \begingroup
	  \large
	
	    
	    \stanza[\smallbreak]
	\label{pv.3.338}\edlabel{pv.3.338}\flagstanza{\tiny\textenglish{....3.338}}वस्तुभिर्नागमास्तेन कथञ्चिन्नान्तरीयकाः ।&प्रतिपत्तुर्न सिध्यन्ति कुतस्तेभ्योऽर्थनिश्चयः ॥ ३३८ ॥\&[\smallbreak]


	
	  \endgroup
	

	  \pstart {\color{DodgerBlue3}“तेन”}\edlabel{pvv.412-3}\footnote{\label{pvv.412-3}  ३ शब्दानां वस्त्वसम्बन्धेन ।} प्रतिबन्धाभावेन {\color{DodgerBlue3}“वस्तुभिः”} सह {\color{DodgerBlue3}“नान्तरीयका आगमा प्रतिपत्तु”}रर्थं शब्दात् प्रतिपद्यमानस्य {\color{DodgerBlue3}“कथञ्चिन्न”} सिध्यन्ति । तत् कुतस्तेभ्य आगमार्थनिश्चयः । (३३८)
	\pend
      \label{div_pvv.3.339}\edlabel{div_pvv.3.339}
	  
	% new div opening: depth here is 2
	
	  \bigskip
	  \begingroup
	  \large
	
	    
	    \stanza[\smallbreak]
	\label{pv.3.339}\edlabel{pv.3.339}\flagstanza{\tiny\textenglish{....3.339}}तस्मान्न तन्निवृत्त्यापि भावाभावः प्रसिध्यति ।&तेनासन्निश्चयफलाऽनुपलब्धिर्न सिध्यति ॥ ३३९ ॥\&[\smallbreak]


	
	  \endgroup
	

	  \pstart यस्मात् प्रवर्त्तमानादागमान्नास्त्यर्थसिद्धिस्त्{\color{DodgerBlue3}“स्मात् त”}स्यागमस्य {\color{DodgerBlue3}“निवृत्त्यापि भावस्याभावो न सिध्यति । तेन”} प्रमाणत्रयनिवृत्तिलक्षणाप्य{\color{DodgerBlue3}“नुपलब्धि”}रर्थाना{\color{DodgerBlue3}“मसन्निश्चयफला न सिध्यति”} । ततो युक्तमुक्तं “सदसन्निश्चय\edlabel{pvv.412-4}\footnote{\label{pvv.412-4}  ४ सर्व्वविषयत्वादागमस्य सति वस्तुन्यविसम्वादेनावृत्तेस्तन्निवृत्तिलक्षणानुपलब्धिरभावसाधनमित्ययुक्तमेव परस्य । विनापि वस्त्वागमप्रवृत्तेः सर्व्वविषयत्वञ्च निरस्तमप्रस्तुतावचनात् ।}फला नेति स्याद् वाऽप्रमाणतेति” । सद्व्यवहारसाधने चाधिकृते व्यवस्थितेत्यवस्थितं ॥ (३३९)
	\pend
      
	    
	    \pstart
	    \begin{center}
	  आचार्य म नो र थ न न्दि कृतायां वार्तिकवृत्तौ तृतीयः परिच्छेदः ॥
	    \end{center}
	    \pend
	  
	  
	    
	    \endnumbering% ending numbering from div
	    \endgroup
	    
	  
	  
	% new div opening: depth here is 0
	
	    
	    \begingroup
	    \beginnumbering% beginning numbering from div depth=0
	    
	  
\chapter[{चतुर्थः परिच्छेदः: परार्थानुमानं}]{चतुर्थः परिच्छेदः\footnote{१ चतुर्थो व्याख्यायते । तत्र सम्बन्धमन्तरेण वाक्यार्थावृत्तेरिति तृतीयपरिच्छेदं व्याख्यायानेनोत्तरसम्बन्धं कथयन् प्रतिपर्तृसुखार्थमन्त्यपरिच्छेदे सर्वमेव शास्त्रशरीरमनुक्रमेण वृत्तिकृत् कथयति “अनुमानमि”त्यादि ।}: परार्थानुमानं}\label{div_pvv.iv.0}\edlabel{div_pvv.iv.0}
	  
	% new div opening: depth here is 1
	\leavevmode\marginnote{\textenglish{413/s}}

	  \pstart स्वार्थानुमानं व्याख्याय परार्थानुमानं व्याख्यातुमाह (।)
	\pend
      \leavevmode\marginnote{\textenglish{81b/MA}}

	  \pstart तत्र परार्थानुमानं\edlabel{pvv.413-2}\footnote{\label{pvv.413-2}  २ [तत्रैव]---परार्थमनुमानं तु ।} स्वदृष्टार्थप्रकाशनमित्याचार्यीयलक्षणं । स्वेन\edlabel{pvv.413-3}\footnote{\label{pvv.413-3}  ३ वादिप्रतिवादिभ्यामिति प्रकरणात् ।}दृष्टः स्वदृष्टः । स्वदृष्टश्चासावर्थश्चेति त्रिरूपो हेतुः । तस्य प्रकाशनम्वचनं अनुमानहेतुत्वादित्यर्थः ।
	\pend
      
	  
	% new div opening: depth here is 1
	
\section[{१. परार्थानुमानलक्षणम् (दिग्नागस्य)}]{१. परार्थानुमानलक्षणम् (दिग्नागस्य)}\label{div_pvv.4.1}\edlabel{div_pvv.4.1}
	  
	% new div opening: depth here is 2
	

	  \begin{center}%% label @type='head'
	\textbf{(१) तत्र स्वदृष्टग्रहणफलम्}
	\end{center}
	

	  \begin{center}%% label @type='head'
	\textbf{क. परमतनिरासाय}
	\end{center}
	

	  \pstart ननु त्रिरूपलिङ्गाख्यानमित्येवास्तु किं स्वदृष्टग्रहणेनेति शङ्कायां परमतनिरासार्थतामस्य दर्शयितुमाह ।
	\pend
      
	  \bigskip
	  \begingroup
	  \large
	
	    
	    \stanza[\smallbreak]
	\label{pv.4.1}\edlabel{pv.4.1}\flagstanza{\tiny\textenglish{...pv.4.1}}परस्य प्रतिपाद्यत्वात् अदृष्टोपि स्वयं परैः ।&दृष्टसाधनमित्येके तत्क्षेपायात्मदृग्वचः ॥ १ ॥\&[\smallbreak]


	
	  \endgroup
	

	  \pstart {\color{DodgerBlue3}“परस्य”} परार्थानुमानेन {\color{DodgerBlue3}“प्रतिपाद्यत्वात्”} साधनवादिना स्वयम{\color{DodgerBlue3}“दृष्टोपि प्रमाणेन परैः”} प्रतिवादिभिरागमाद् {\color{DodgerBlue3}“दृष्टसाधनं”} लिङ्ग{\color{DodgerBlue3}“मित्येके”} सां ख्याः । ते हि सुखादीनामुत्पत्तिमत्वादनित्यादचेतनत्वं रूपादीनामिव बौद्धं प्रत्याहुः । {\color{DodgerBlue3}“न ह्यसत”} उत्पत्तिमत्वं सतश्च निरन्वयविनाशोऽनित्यत्वं हेतुः सांख्यासिद्धः । बौद्धस्य पुनरागमात् \leavevmode\marginnote{\textenglish{414/s}} सिद्धः । तावतैव हेतुरिति मन्यन्ते । तस्य परमतस्य {\color{DodgerBlue3}“क्षेपाय”} प्रतिषेधाया{\color{DodgerBlue3}“त्मदृश्र/?/वचः”} अदृष्टवचनं सूत्रे । न प्रतिवादिमात्रसिद्धस्य हेतुत्वं किन्तूभयसिद्धस्यैवत्यर्थः ॥ (१)
	\pend
      \label{div_pvv.4.2}\edlabel{div_pvv.4.2}
	  
	% new div opening: depth here is 2
	

	  \begin{center}%% label @type='head'
	\textbf{ख. अनुमानविषये नागमस्य प्रामाण्यम्}
	\end{center}
	

	  \pstart न चागमात् प्रतिवादिनोपि साधनसिद्धिर्युक्तेति वक्तुमाह ।
	\pend
      
	  \bigskip
	  \begingroup
	  \large
	
	    
	    \stanza[\smallbreak]
	\label{pv.4.2}\edlabel{pv.4.2}\flagstanza{\tiny\textenglish{...pv.4.2}}अनुमाविषये नेष्टं परीक्षितपरिग्रहात् ।&वाचः प्रामाण्यमस्मिन् हि नानुमानं प्रवर्तते ॥ २ ॥\&[\smallbreak]


	
	  \endgroup
	

	  \pstart {\color{DodgerBlue3}“अनुमानस्य”} वस्तुबलप्रवृत्तस्य {\color{DodgerBlue3}“विषये”} यस्मादनुमानज्ञानमुत्पद्यते (।) स च त्रिरूपो हेतुः (।) तत्र प्रतिपाद्ये {\color{DodgerBlue3}“वाच”} आगमात्मिकायाः {\color{DodgerBlue3}“प्रामाण्यं”} नेष्टमर्थप्रतिबन्धाभावादित्युक्तं ।
	\pend
      

	  \pstart किञ्च (।) प्रमाणान्तरसम्वादात् {\color{DodgerBlue3}“परीक्षितस्य”} प्रमाणोपपन्नस्यागमार्थस्य {\color{DodgerBlue3}“परिग्रहात्”} स्वीकारान्नेष्टं वचनप्रामाण्यं । यदि वचनमित्येव प्रमाणन्तदा प्रतिज्ञापदादेव साध्यस्य सिद्धेर्निष्फलं हेतुदृष्टन्तादिवचनं स्यात् । प्रमाणान्तरसम्वादापेक्षा च न भवेत् । सम्वादज्ञानस्यैवार्थभावानुविधायित्वात् तस्मिन्नागमार्थे प्रामाण्यं । वचनस्य तु विपर्ययात् । यदि त्वागम इत्येव प्रमाणं (।) तदा प्रमाणान्तरसम्वादापेक्षा न स्यात् । हि यस्मात् । {\color{DodgerBlue3}“अस्मिन्ना”}गमार्थे प्रमाणप्रतिपादितत्वान्निश्चितेऽ{\color{DodgerBlue3}“नुमानं न प्रवृर्त्तते”} (।) यदि ह्यागमार्थः सन्दिह्येत तदा तन्निश्चयार्था प्रमाणान्तरवृत्तिरपेक्ष्येत ॥ (२)
	\pend
      \label{div_pvv.4.3}\edlabel{div_pvv.4.3}
	  
	% new div opening: depth here is 2
	
	  \bigskip
	  \begingroup
	  \large
	
	    
	    \stanza[\smallbreak]
	\label{pv.4.3}\edlabel{pv.4.3}\flagstanza{\tiny\textenglish{...pv.4.3}}बाधनायागमस्योक्तेः साधनस्य परं प्रति ।&सोप्रमाणं तदाऽसिद्धं तत्सिद्धमखिलन्ततः ॥ ३ ॥\&[\smallbreak]


	
	  \endgroup
	

	  \pstart यदि चाग{\color{DodgerBlue3}“मस्य”} सुखादिचैतन्यप्रतिपादकस्य {\color{DodgerBlue3}“बाधनाय साधन”}स्योत्पत्तिमत्वादेः सां ख्ये न {\color{DodgerBlue3}“परं बौद्धं प्रत्युक्तिः”} ततः कारणात् {\color{DodgerBlue3}“स”} आगमोऽ{\color{DodgerBlue3}“प्रमाणं । तदा”} । न हि प्रमाणस्य बाधो युक्तः । तत आगमाप्रामाण्यात् तेनागमेन साक्षात् पारंपर्याभ्यां {\color{DodgerBlue3}“सिद्ध”}मुत्पत्तिमत्वादि साध्यं चाचैतन्य{\color{DodgerBlue3}“मखिलमिदमसिद्धं”} ॥ (३)
	\pend
      \label{div_pvv.4.4}\edlabel{div_pvv.4.4}
	  
	% new div opening: depth here is 2
	

	  \begin{center}%% label @type='head'
	\textbf{ग. प्रमाणेन बाध्यमानस्यागमस्य न सिद्धिः}
	\end{center}
	
	  \bigskip
	  \begingroup
	  \large
	
	    
	    \stanza[\smallbreak]
	\label{pv.4.4}\edlabel{pv.4.4}\flagstanza{\tiny\textenglish{...pv.4.4}}तदागमवतः सिद्धं यदि कस्य क आगमः ।&बाध्यमानः प्रमाणेन स सिद्धः कथमागमः ॥ ४ ॥\&[\smallbreak]


	
	  \endgroup
	

	  \pstart स च सुखाद्युत्पत्तिमत्वप्रतिपादक आगमः (च) {\color{DodgerBlue3}“तदागमः”} तद्वतस्तत्सम्बन्धिनो {\color{DodgerBlue3}“बौद्धस्य सिद्ध”}मुत्पत्तिमत्त्वादिलिङ्गमिति चेत् । {\color{DodgerBlue3}“कस्य”} पुरुषस्य क आगमः सम्बन्धी । न \leavevmode\marginnote{\textenglish{415/s}} तावदवयवत् आगमोपि पुरुषस्य सहजसम्बन्धेन सम्बद्धः\edlabel{pvv.415-1}\footnote{\label{pvv.415-1}  १ तादात्मा(?त्म्या)भाव उक्तः ।} । नापि युक्त्युपपन्नयतो\edlabel{pvv.415-2}\footnote{\label{pvv.415-2}  २ तदुत्पत्तेः ।}पाधिना\edlabel{pvv.415-3}\footnote{\label{pvv.415-3}  ३ अस्येदमिति ।} प्रत्यक्षानुमानवत् सम्बन्धः । तथा हि {\color{DodgerBlue3}“प्रमाणे”}नोत्पत्तिमत्वादिलिङ्गजेनानुमानेनागमप्रतिपादितस्य सुखादिचैतन्यस्य बाधनात् {\color{DodgerBlue3}“बाध्यमान”} आगमः हेतुः । {\color{DodgerBlue3}“स कथमागमसिद्धो”} येन युक्तिसम्वादोपाधिनापि पुरुषेण सम्बन्धमनुभवेत् । (४)
	\pend
      \label{div_pvv.4.5}\edlabel{div_pvv.4.5}
	  
	% new div opening: depth here is 2
	

	  \pstart स्यादेतत् । सुखादीनामचैतन्यं तेनैवागमेन प्रदेशान्तरे दर्शितमतोऽबाध्यत्वात् प्र7माणादस्मात् साधनविधिर्युक्त इत्याह ।\leavevmode\marginnote{\textenglish{82a/MA}}
	\pend
      
	  \bigskip
	  \begingroup
	  \large
	
	    
	    \stanza[\smallbreak]
	\label{pv.4.5}\edlabel{pv.4.5}\flagstanza{\tiny\textenglish{...pv.4.5}}तद्विरुद्धाभ्युपगमस्तेनैव च कथं भवेत् ।&तदन्योपगमे तस्य त्यागांगस्याप्रमाणता ॥ ५ ॥\&[\smallbreak]


	
	  \endgroup
	

	  \pstart तस्माच्चैतन्यात् प्रतिपादिताद् {\color{DodgerBlue3}“विरुद्ध”}स्याचैतन्यस्या{\color{DodgerBlue3}“भ्युपगमस्तेन”} चैतन्यप्रतिपादकेनैव चागमेन {\color{DodgerBlue3}“कथम्भवेत्”} । सम्भवे वा विरुद्धार्थाभिधायित्वेनाप्रमाणादागमाद्धेतुसिद्धिरयुक्तैव (।) आगमप्रतीतेनोत्पत्तिमत्त्वादिना । तस्मादागमोक्त {\color{DodgerBlue3}“चैतन्यादन्य”}स्याचैतन्यस्यो{\color{DodgerBlue3}“पगमे”} स्वीकारे वा प्रतिवादीना क्रियमाणे {\color{DodgerBlue3}“तस्या”}गमस्य {\color{DodgerBlue3}“त्यागांगस्याप्रमाणता”}ऽभ्युपगता स्यात् (।) यदैवागमप्रामाण्यस्य बाधके हेतावाद्रियते तदैव तत्र संशयितः । न चाप्रमाणाद्धेतुसिद्धिः ॥ (५)
	\pend
      \label{div_pvv.4.6}\edlabel{div_pvv.4.6}
	  
	% new div opening: depth here is 2
	

	  \pstart किञ्च (।)
	\pend
      
	  \bigskip
	  \begingroup
	  \large
	
	    
	    \stanza[\smallbreak]
	\label{pv.4.6}\edlabel{pv.4.6}\flagstanza{\tiny\textenglish{...pv.4.6}}तत् कस्मात् साधनं नोक्तं स्वप्रतोतिर्यदुद्भवा ।&युक्त्या ययागमो ग्राह्यः परस्यापि च सा न किम् ॥ ६ ॥\&[\smallbreak]


	
	  \endgroup
	

	  \pstart अयम्वादी तावत् स्वयम्प्रतीतमेवाचैतन्यं {\color{DodgerBlue3}“पर”}\edlabel{pvv.415-4}\footnote{\label{pvv.415-4}  ४ अज्ञातं ममेति वचने उपहसनीयः स्यात् ।}स्मै प्रतिपादयति । {\color{DodgerBlue3}“व च साधन”}मन्तरेणैव प्रतीतिः । ततः प्रतिपादयितुः\edlabel{pvv.415-5}\footnote{\label{pvv.415-5}  ५ सांख्यस्य ।} स्वस्य {\color{DodgerBlue3}“प्रतीतिश्चै”}तन्यविषया यस्मात् साधनादुद्भवो यस्याःसा यदुद्भवा । {\color{DodgerBlue3}“तत्साधनं”} स्वागमादिकं कस्मात्\edlabel{pvv.415-6}\footnote{\label{pvv.415-6}  ६ तदुत्पत्तेः ।} प्रतिपाद्यं प्रति {\color{DodgerBlue3}“नोक्तं”} । स्वयं प्रतिपन्नसामर्थ्यमेव साधनं वक्तुमुचितं । तथा निर्युक्तिकस्यागमस्याप्रामाण्यात् {\color{DodgerBlue3}“यया युक्त्यो”}पपन्न {\color{DodgerBlue3}“आगमः”} साधनत्वेन {\color{DodgerBlue3}“गाह्यो”} वादिना सा युक्तिः {\color{DodgerBlue3}“परस्य”} प्रतिपाद्यस्यापि किन्न साधनयेन तत्परित्यज्यान्यदप्रमाणसिद्धमुत्पत्त्याद्युच्यते । (६)
	\pend
      \label{div_pvv.4.7}\edlabel{div_pvv.4.7}
	  
	% new div opening: depth here is 2
	

	  \pstart अथ योगबलजेन प्रत्यक्षेण सुखादीनामचैतन्यं प्रतीतं\edlabel{pvv.415-7}\footnote{\label{pvv.415-7}  ७ कपिलऋषिणा ।} न तदन्यथा प्रतिपादयितुं परस्य शक्यत इति परोपगतं साधनमुच्यते ॥
	\pend
      \leavevmode\marginnote{\textenglish{416/s}}

	  \pstart ननु तथापि नाकस्मिको बुद्धिसुखाद्यचैतन्यविषयस्य निश्चयः । तत्साक्षात्कारिप्रत्यक्षतत्साधनयोगानुष्ठानयोः \edlabel{pvv.416-1}\footnote{\label{pvv.416-1}  १ ज्ञात्वा हि लिङ्गैस्तदुपायोनुष्ठीयतेऽन्यथा भावनानुपपत्तेः ।} साध्यसाधनभावावधारणञ्चावश्यकर्त्तव्यं । तन्निश्चयमन्तरेण योगानुष्ठानायोगात् । तथा च (।)
	\pend
      
	  \bigskip
	  \begingroup
	  \large
	
	    
	    \stanza[\smallbreak]
	\label{pv.4.7}\edlabel{pv.4.7}\flagstanza{\tiny\textenglish{...pv.4.7}}प्राकृतस्य सतः प्राग् यैः प्रतिपत्त्यक्षसंभवौ ।&साधनैः साधनान्यर्थशक्तिज्ञानेस्य तान्यलम् ॥ ७ ॥\&[\smallbreak]


	
	  \endgroup
	

	  \pstart {\color{DodgerBlue3}“प्राकृत”}स्यार्व्वाग्दर्शनस्य सतो यैः {\color{DodgerBlue3}“साधनैः”} सुखाचैतन्यादिविषयं तद्ग्राहि {\color{DodgerBlue3}“प्रत्यक्ष”}तत्साधनयोः सम्बन्धञ्च निश्चित्य प्रवृत्तस्योपाये प्रतिपत्तुरनुष्ठानस्य । तत्फलस्याक्षस्य प्रत्यक्षस्याचैतन्यादिविषयग्राहिणः सम्भवौ भवतः (।) {\color{DodgerBlue3}“तानि”} {\color{DodgerBlue3}“साधनान्यर्थ”}स्योपायाभ्यासस्य {\color{DodgerBlue3}“शक्ते”}रचैतन्यग्राहिप्रत्यक्षजनिकाया {\color{DodgerBlue3}“ज्ञाने”} कर्त्तव्ये{\color{DodgerBlue3}“ऽस्य”} प्रतिपाद्यस्यालं समर्थानि । ततस्यान्येव साधनान्युच्यन्तां \edlabel{pvv.416-2}\footnote{\label{pvv.416-2}  २ यथास्माभिर्द्वितीयपरिच्छेदे उक्तं परो मोक्षेऽविद्या(सत्काय)तृष्णाभ्यांक्षेपि जन्मेति सम्बन्धः शून्यतादृष्टिस्तयोः प्रतिपक्ष इति यथेह तृष्णया प्रवृत्तिरविद्याप्रेरणया ।} किं स्वयमदृष्टसाध्य\edlabel{pvv.416-3}\footnote{\label{pvv.416-3}  ३ यत्रैव स्कन्धेऽविद्यादयस्तत्रानारोपितेऽविद्याशून्यमिति सम्बन्धः ।}प्रतिपादन{\color{DodgerBlue3}“सामर्थ्येनोत्प”}त्तिमत्वादिनोक्तेन ॥ (७)
	\pend
      \label{div_pvv.4.8}\edlabel{div_pvv.4.8}
	  
	% new div opening: depth here is 2
	

	  \pstart स्यादेतत् । प्रत्यक्षस्याचैतन्यादिविषयस्य योगाद्यभ्यासेन सह साध्यसाधनसम्बन्धो विप्रकृष्टात्वात् सामान्याकारेणापि न प्रतीयते ततोस्यानुपदर्शनमित्याह ।
	\pend
      
	  \bigskip
	  \begingroup
	  \large
	
	    
	    \stanza[\smallbreak]
	\label{pv.4.8}\edlabel{pv.4.8}\flagstanza{\tiny\textenglish{...pv.4.8}}विच्छिन्नानुगमा येपि सामान्येनाप्यगोचराः ।&साध्यसाधनचिन्तास्ति न तेष्वर्थेषु काचन ॥ ८ ॥\&[\smallbreak]


	
	  \endgroup
	

	  \pstart {\color{DodgerBlue3}“येपि विच्छिन्नो”} विप्रकृष्टोऽ{\color{DodgerBlue3}“नुगमः”} सम्बन्धो येषां ते प्रत्यक्षतदुपायादयः {\color{DodgerBlue3}“सामान्ये”}नाविशेषाकारेणा{\color{DodgerBlue3}“प्यगोचरास्तेष्वर्थेष्विदं”} साधनमिदं साध्यमिति च {\color{DodgerBlue3}“साध्यसाधनचिन्ता काचन नास्ति”} । ततोऽर्व्वाग्दर्शनस्याबुद्धिविषयीकृतयो रूपयोरुपयोपेययोरेकत्रानुष्ठानावपरत्र निष्पादनमिति कुतः ॥ (८)
	\pend
      \label{div_pvv.4.9}\edlabel{div_pvv.4.9}
	  
	% new div opening: depth here is 2
	

	  \pstart {\color{DodgerBlue3}“किञ्च”} (।) अप्रमाणकादागमादिच्छामात्रेण प्रतिपाद्यो हेतुं कुर्व्वन् पुनरिच्छया तमेव परिहरन् हेतुतदाभासयोरिच्छाधीनतां स्वीकुर्यादित्याख्यातुमाह ।
	\pend
      
	  \bigskip
	  \begingroup
	  \large
	
	    
	    \stanza[\smallbreak]
	\label{pv.4.9}\edlabel{pv.4.9}\flagstanza{\tiny\textenglish{...pv.4.9}}पुंसामभिप्रायवशात् तत्त्वातत्त्वव्यवस्थितौ ।&लुप्तौ हेतुतदाभासौ तस्य वस्त्वसमाश्रयात् ॥ ९ ॥\&[\smallbreak]


	
	  \endgroup
	\leavevmode\marginnote{\textenglish{82b/MA}}

	  \pstart {\color{DodgerBlue3}“पुंसामभिप्रायवशादि”}च्छानुरोधात् {\color{DodgerBlue3}“तत्त्वातत्त्व”}योर्हेतुतदाभासत्वयो{\color{DodgerBlue3}“र्व्यवस्थिता”}विष्यमाणायां {\color{DodgerBlue3}“हेतुतदाभासौ लुप्तौ”} स्यातां अव्यवस्थितत्वात् {\color{DodgerBlue3}“तस्य”} पुरुषाभिप्रा\leavevmode\marginnote{\textenglish{417/s}} यस्य {\color{DodgerBlue3}“वस्त्व”}संश्रयाद् यथावस्तुप्रवृत्तिनियमाभावात् ॥ (९)
	\pend
      \label{div_pvv.4.10}\edlabel{div_pvv.4.10}
	  
	% new div opening: depth here is 2
	

	  \pstart अपि च (।)
	\pend
      
	  \bigskip
	  \begingroup
	  \large
	
	    
	    \stanza[\smallbreak]
	\label{pv.4.10}\edlabel{pv.4.10}\flagstanza{\tiny\textenglish{...v.4.10}}सन्नर्थो ज्ञानसापेक्षो नासन् ज्ञानेन साधकः ।&सतोपि वस्त्वसंश्लिष्टाऽसंगत्या सदृशी गतिः ॥ १० ॥\&[\smallbreak]


	
	  \endgroup
	

	  \pstart {\color{DodgerBlue3}“सन्नर्थो ज्ञानसापेक्षः”} साध्य{\color{DodgerBlue3}“साधकः”} प्रतीतो यथा धूमादिः । {\color{DodgerBlue3}“नासन्न”}र्थो {\color{DodgerBlue3}“ज्ञानेन”} प्रतीतिमात्रेण {\color{DodgerBlue3}“साधकः”} साध्यस्य यथा कल्पितो धूमः । अथाचैतन्यम्वस्तुतोस्त्येव तद् यथा कथञ्चित् परस्मै प्रतिपादनीयं । अतः पराभ्युपगतो हेतुः क्रियत इत्याह । {\color{DodgerBlue3}“सतोप्य”}चैतन्यस्य {\color{DodgerBlue3}“वस्त्वसंश्लिष्टा”} वस्तुभूतलिङ्गाप्रतिबद्धा गतिरस{\color{DodgerBlue3}“ङ्गत्या”}ऽसतः प्रतीत्या {\color{DodgerBlue3}“सदृशी”} सम्यक् प्रतीतत्वाभावत् । अन्येत्वसद्गत्या दोषवत्प्रतीत्या सदृशीति व्याचक्षते (।) तेषां सतोप्यवस्तुकृता प्रतिपत्तिरसत्प्रतिपत्तिं नातिशेते । अप्रत्ययत्वादिति विनिश्चयग्रन्थेन सह एकवाक्यता न स्यात् ॥ (१०)
	\pend
      \label{div_pvv.4.11}\edlabel{div_pvv.4.11}
	  
	% new div opening: depth here is 2
	

	  \pstart किञ्च (।)
	\pend
      
	  \bigskip
	  \begingroup
	  \large
	
	    
	    \stanza[\smallbreak]
	\label{pv.4.11}\edlabel{pv.4.11}\flagstanza{\tiny\textenglish{...v.4.11}}लिङ्गं स्वभावः कार्यं वा दृश्यादर्शनमेव वा ।&सम्बद्धं वस्तुतस्सिद्धं तदसिद्धं किमात्मनः ॥ ११ ॥\&[\smallbreak]


	
	  \endgroup
	

	  \pstart लिङ्गं साध्यार्थ{\color{DodgerBlue3}“सम्बद्धं”} नान्यत् व्यभिचारात् । तच्च {\color{DodgerBlue3}“स्वभावः कार्यं दृश्यादर्शन”}मनुपलब्धिरेवान्यस्य प्रतिबन्धाभावादित्युक्तं । तदुत्पत्त्यादिकं यदि त्रिषु हेतुष्वन्तर्भूतं तदा {\color{DodgerBlue3}“वस्तुतः”} परमार्थतः {\color{DodgerBlue3}“सिद्धं”} प्रतिपाद्यस्य {\color{DodgerBlue3}“किं”} कस्मा{\color{DodgerBlue3}“दात्मनः”} सां ख्य स्य वादिनो\edlabel{pvv.417-1}\footnote{\label{pvv.417-1}  १ उभयसिद्धमस्तु ।} {\color{DodgerBlue3}“ऽसिद्धं”} । उत्पत्तिमत्वादिस्वभावहेतुः स्यात् । स च धर्मिग्राहकात् प्रमाणादन्यतो वा शिंशपा\edlabel{pvv.417-2}\footnote{\label{pvv.417-2}  २ वृक्षं ग्राहयति ।}त्ववत् कृतकत्वादिवदर्थान् प्रतिवादिन इव वादिनोपि सि३ध्येत् । (११)
	\pend
      \label{div_pvv.4.12}\edlabel{div_pvv.4.12}
	  
	% new div opening: depth here is 2
	

	  \pstart अथ\edlabel{pvv.417-3}\footnote{\label{pvv.417-3}  ३ पराभ्युपगतेन दूषयित्वा प्रमाऽनन्तर्भावमाह ।} त्रिविधे हेतौ नान्तर्भवति उत्पत्त्यादि तदा (।)
	\pend
      
	  \bigskip
	  \begingroup
	  \large
	
	    
	    \stanza[\smallbreak]
	\label{pv.4.12a}\edlabel{pv.4.12a}\flagstanza{\tiny\textenglish{....4.12a}}परेणाप्यन्यतो गन्तुमयुक्तं;\&[\smallbreak]


	
	  \endgroup
	

	  \pstart {\color{DodgerBlue3}“परेण”} प्रतिवादिनापि त्रिविधाद्धेतो{\color{DodgerBlue3}“रन्यतो”} हेतोरचैतन्यं {\color{DodgerBlue3}“गन्तुं”} प्रत्येतु{\color{DodgerBlue3}“मयुक्तं”} (।) यदि पराभ्युपगमसिद्धमसाधनं तदा प्रसङ्गहेतुरहेतुः स्यादित्याह ।
	\pend
      
	  \bigskip
	  \begingroup
	  \large
	
	    
	    \stanza[\smallbreak]
	\label{pv.4.12b}\edlabel{pv.4.12b}\flagstanza{\tiny\textenglish{....4.12b}}परकल्पितैः&प्रसङ्गो द्वयसम्बन्धादेकापायेन्यहानये ॥ १२ ॥\&[\smallbreak]


	
	  \endgroup
	\leavevmode\marginnote{\textenglish{418/s}}

	  \pstart {\color{DodgerBlue3}“परकल्पितैः”} साधनैः {\color{DodgerBlue3}“प्रसङ्गः”} क्रियते यथा सामान्यस्य परोपगतानेकवृत्तित्वाद् अनेकत्वमापाद्यते न त्वयं पारमार्थिको हेतुस्त्रैरूप्याभावात् । यद्ययं न हेतुः  तदा किमर्थमुच्यत इत्याह । {\color{DodgerBlue3}“द्वयोः”} साध्यसाधनयोः {\color{DodgerBlue3}“सम्बन्धाद्”} व्याप्यव्यापकभावात् स्फारिता{\color{DodgerBlue3}“देकस्य”} साध्यस्यापायेऽन्यस्य साधनस्य {\color{DodgerBlue3}“हान”}ये । यथा चानेकं\edlabel{pvv.418-1}\footnote{\label{pvv.418-1}  १ पक्षधर्मोपसंहार एषः ।}सामान्यं तस्मान्नानेकवृत्तीति विपर्ययप्रयोगे साध्याभावे साधनाभावः कथ्यते । प्रसङ्गविपर्ययोत्र मौलो हेतुः साध्यसाधन्यव्याप्तिग्राहकप्रमाणस्मारकस्तु प्रसङ्गे प्रयोग इत्यर्थः । भिन्नदेशकालादिष्वनेकासु व्यक्तिषु वृत्तस्य तदतद्देशत्वादिविरुद्धधर्माध्यासादनेकत्वसिद्धेरनेकवृत्तत्वानेकत्वयोर्व्याप्तिसिद्धिर्बोद्धव्या । (१)
	\pend
      \label{div_pvv.4.13}\edlabel{div_pvv.4.13}
	  
	% new div opening: depth here is 2
	

	  \pstart उक्तं स्वदृष्टग्रहणस्य साफल्यं \edlabel{pvv.418-2}\footnote{\label{pvv.418-2}  २ उभयसिद्धो हेतुः सूचितः प्रमोपपत्तिः ।} ॥
	\pend
      

	  \begin{center}%% label @type='head'
	\textbf{(२) अर्थग्रहणफलम्}
	\end{center}
	

	  \pstart अर्थग्रहणस्येदानीं वक्तव्यं । अदृष्टार्थप्रकाशनमित्यत्र सूत्रे यदुपात्तं (।)
	\pend
      
	  \bigskip
	  \begingroup
	  \large
	
	    
	    \stanza[\smallbreak]
	\label{pv.4.13}\edlabel{pv.4.13}\flagstanza{\tiny\textenglish{...v.4.13}}तदर्थग्रहणं शब्दकल्पनारोपितात्मनाम् ।&अलिङ्गत्वप्रसिद्ध्यर्थमर्थादर्थस्य सिद्धितः ॥ १३ ॥\&[\smallbreak]


	
	  \endgroup
	

	  \pstart {\color{DodgerBlue3}“तदर्थग्रहणं शब्देन \edlabel{pvv.418-3}\footnote{\label{pvv.418-3}  ३ यथा सर्व्वगत आत्मा सर्व्वत्रोपलभ्यमानगुणत्वादाकाशवत् ॥ नित्योऽनित्यो वा शब्दः पक्षसपक्षान्यतरत्वात्परमाणुवद् घटवद्वा ।} कल्पनया चारोपित आत्मा”} येषां पक्षसपक्षान्यतरत्वादीनां {\color{DodgerBlue3}“तेषामलिङ्गत्वप्रसिद्ध्यर्थं”} बोद्धव्यं । कस्मात् पुनः कल्पितस्यालिङ्गत्वमित्याह । {\color{DodgerBlue3}“अर्थाद्”} वस्तुभूताल्लिङ्गादर्थस्य साध्यस्य {\color{DodgerBlue3}“सिद्धितः”} ॥ (१)
	\pend
      \label{div_pvv.4.14}\edlabel{div_pvv.4.14}
	  
	% new div opening: depth here is 2
	
	  \bigskip
	  \begingroup
	  \large
	
	    
	    \stanza[\smallbreak]
	\label{pv.4.14}\edlabel{pv.4.14}\flagstanza{\tiny\textenglish{...v.4.14}}कल्पनागमयोः कर्त्तुरिच्छामात्रानुवृत्तितः ।&वस्तुनश्चान्यथाभावात् तत्कृता व्यभिचारिणः ॥ १४ ॥\&[\smallbreak]


	
	  \endgroup
	

	  \pstart {\color{DodgerBlue3}“कल्पनाया आगमस्य”} शब्दस्य च {\color{DodgerBlue3}“कर्त्तुः”} पुरुष{\color{DodgerBlue3}“स्येच्छामात्र”}स्या{\color{DodgerBlue3}“नुवृत्तितो”}नुरोधात् । {\color{DodgerBlue3}“वस्तुनश्चान्यथाभावात्”} । कर्त्तुरिच्छानुवृत्ते{\color{DodgerBlue3}“स्ताभ्यां”} शब्दकल्पनाभ्यां {\color{DodgerBlue3}“कृता”} हेतवो {\color{DodgerBlue3}“व्यभिचारिणो”} नैकान्तिकाः । (१४)
	\pend
      

	  \pstart उक्तमर्थग्रहणप्रयोजनं ॥
	\pend
      

	  \pstart स्वदृष्टार्थप्रकाशनशब्देन त्रिरूपलिङ्गवचनमिष्टं । न पक्षवचनमपीति वक्तव्यं ॥
	\pend
      
	  
	% new div opening: depth here is 1
	
\section[{२. पक्षचिन्ता}]{२. पक्षचिन्ता}\label{div_pvv.4.15}\edlabel{div_pvv.4.15}
	  
	% new div opening: depth here is 2
	\leavevmode\marginnote{\textenglish{419/s}}

	  \begin{center}%% label @type='head'
	\textbf{(२) पक्षहेतुवचनमसाधनम्}
	\end{center}
	

	  \begin{center}%% label @type='head'
	\textbf{क. हेतुवचनं न साधनम्}
	\end{center}
	

	  \pstart यदि साक्षात् पारम्पर्येण वा पक्षवचनं साध्यप्रत्तिपत्तावुपयुज्यते । तदो-\leavevmode\marginnote{\textenglish{83a/MA}} च्येत किन्त्वेतन्नास्तीत्याह ।
	\pend
      
	  \bigskip
	  \begingroup
	  \large
	
	    
	    \stanza[\smallbreak]
	\label{pv.4.15}\edlabel{pv.4.15}\flagstanza{\tiny\textenglish{...v.4.15}}अर्थादर्थगतेः शक्तिः पक्षहेत्वभिधानयोः ।&नार्थे तेन तयोर्न्नास्ति स्वतः साधनसंस्थिति ॥ १५ ॥\&[\smallbreak]


	
	  \endgroup
	

	  \pstart \edlabel{pvv.419-1}\footnote{\label{pvv.419-1}  १ पञ्चावयवत्वात् परे पक्षहेतुवचनयोः साधनत्वमाहुः । तन्निषेधायाह । साधनं भवत् साक्षात् पारंपर्येण वा स्यात् तत्र प्रतिज्ञाहेतुदृष्टान्तउपनयनिगमनाख्यं ।} साक्षात् तावत पक्षाभिधानस्य हेत्वभिधानस्य च {\color{DodgerBlue3}“प्रतिपाद्येऽर्थे शक्तिर्न”} विद्यते (।) किं कारणमित्याह ।
	\pend
      

	  \pstart {\color{DodgerBlue3}“अर्थाद्”} वचनप्रतिपाद्या{\color{DodgerBlue3}“दर्थ”} साध्यस्य {\color{DodgerBlue3}“गतेर्न”} वचनात् । {\color{DodgerBlue3}“तेन”} साक्षादर्थप्रतिपादकत्वाभावेन तयोः {\color{DodgerBlue3}“पक्षहेत्वभिधानयोः स्वतः”} स्वरूपेण {\color{DodgerBlue3}“साधनसंस्थितिः”} । साधनत्वव्यवस्था {\color{DodgerBlue3}“नास्ति”} \edlabel{pvv.419-2}\footnote{\label{pvv.419-2}  २ ननु आचार्येण शाब्दं प्रमाणमिष्टं कथन्ततो नार्थ इत्याह ।}यतश्च पक्षवचनं साक्षादर्थे न प्रमाणं ॥ (१५)
	\pend
      \label{div_pvv.4.16}\edlabel{div_pvv.4.16}
	  
	% new div opening: depth here is 2
	

	  \begin{center}%% label @type='head'
	\textbf{ख. पक्षवचनमसाधनम्}
	\end{center}
	
	  \bigskip
	  \begingroup
	  \large
	
	    
	    \stanza[\smallbreak]
	\label{pv.4.16}\edlabel{pv.4.16}\flagstanza{\tiny\textenglish{...v.4.16}}तत् पक्षवचनं वक्तुरभिप्रायनिवेदने ।&प्रमाणं संशयोत्पत्तेस्ततः साक्षान्न साधनम् ॥ १६ ॥\&[\smallbreak]


	
	  \endgroup
	

	  \pstart तत्तस्मात् {\color{DodgerBlue3}“पक्षवचनं”} {\color{DodgerBlue3}“वक्तुरभिप्रायनिवेदने प्रमाणं”} शब्दप्रामाण्यमा चा र्य स्य वदतोऽभिमतमिति बोद्धव्यं । तत्पक्षवचनात् साध्येर्थ {\color{DodgerBlue3}“संशयोत्पत्तेरनिश्चयान्न साक्षात्”} साधनमर्थस्य तत् ॥ (१६)
	\pend
      \label{div_pvv.4.17}\edlabel{div_pvv.4.17}
	  
	% new div opening: depth here is 2
	

	  \pstart परंपरया साध्यसाधनात् प्रमाणं पक्षवचनं वक्तुरभिप्रायनिवेदने प्रमाणन्तर्हि स्यादित्याह ।
	\pend
      
	  \bigskip
	  \begingroup
	  \large
	
	    
	    \stanza[\smallbreak]
	\label{pv.4.17}\edlabel{pv.4.17}\flagstanza{\tiny\textenglish{...v.4.17}}साध्यस्यैवाभिधानेन पारंपर्येण नाप्यलम् ।&शक्तस्य सूचकं हेतुवचोऽशक्तमपि स्वयम् ॥ १७ ॥\&[\smallbreak]


	
	  \endgroup
	\leavevmode\marginnote{\textenglish{420/s}}

	  \pstart {\color{DodgerBlue3}“पारम्पर्येणापि”} पक्षवचन{\color{DodgerBlue3}“मलं”} समर्थं {\color{DodgerBlue3}“न”} साध्यसिद्धौ । {\color{DodgerBlue3}“साध्यस्यैव”} केवल{\color{DodgerBlue3}“स्याभिधानात्”} । न हि पक्षवचसा साधकं किञ्चिदुच्यते साध्यमात्रस्यैवाभिधानात् । \edlabel{pvv.420-1}\footnote{\label{pvv.420-1}  १ हेतुवचोवत् पारंपर्येण स्यादौपचारिकं साधनं तत्तु साध्यस्यासिद्धस्याभिधायकमिति नोपचरितोपि साधनं उक्तञ्च । \par
तत्रानुमेयनिर्देशो हेत्वर्थविषयो मत इति ।}त्रिरूपस्य {\color{DodgerBlue3}“हेतोर्व्वचो”} वचनन्तु {\color{DodgerBlue3}“स्वयं”} साक्षात् सिद्धाव{\color{DodgerBlue3}“शक्तमपि”} (।) {\color{DodgerBlue3}“शक्तस्य”} त्रिरूपलिङ्गस्य {\color{DodgerBlue3}“सूचकं”} प्रतिपादकमिति साधनमुचितं ॥ (१७)
	\pend
      \label{div_pvv.4.18}\edlabel{div_pvv.4.18}
	  
	% new div opening: depth here is 2
	

	  \begin{center}%% label @type='head'
	\textbf{ग. (दिग्नागस्य) पक्षवचनमसाधनत्वेनेष्टम्}
	\end{center}
	

	  \pstart नन्वाचार्यस्य पक्षवचनमसाधनत्वेनेष्टमिति कथं ज्ञायत इत्याह ।
	\pend
      
	  \bigskip
	  \begingroup
	  \large
	
	    
	    \stanza[\smallbreak]
	\label{pv.4.18}\edlabel{pv.4.18}\flagstanza{\tiny\textenglish{...v.4.18}}हेत्वर्थविषयत्वेन तदशक्तोक्तिरीरिता ।&शक्तिस्तस्यापि चेद्धेतुवचनस्य प्रवृर्त्तनात् ॥ १८ ॥\&[\smallbreak]


	
	  \endgroup
	

	  \pstart हेतोरर्थः साध्यः स विषयोस्येति {\color{DodgerBlue3}“हेत्वर्थविषयः”} तत्वेन साध्यार्थोपदर्शकत्वेन तस्य पक्षवचनस्य साध्यसाधनं प्रत्य{\color{DodgerBlue3}“शक्त”}स्यो{\color{DodgerBlue3}“क्तिरीरिता”} निर्दिष्टाचार्येण (।) “तत्रानुमेयनिर्देशो हेत्वर्थविषयो मत” इत्यनेन ग्रन्थेन । ततो ज्ञायते पक्षवचनमसाधनमिष्टमाचार्यस्येति ।
	\pend
      

	  \pstart ननु {\color{DodgerBlue3}“तस्य”} पक्षवचनस्यापि साध्यसिद्धौ {\color{DodgerBlue3}“शक्तिरस्ति”} । तत्साधकस्य {\color{DodgerBlue3}“हेतुवचनस्य”} {\color{DodgerBlue3}“प्रवर्त्तनात्”} । नह्यनुद्दिष्टेर्थे साधनप्रस्तावः । ततः साधनप्रस्तावनाहेतुत्वेन पक्षवचनस्य साधकत्वमस्तीति चेत् । एवं (। १८)
	\pend
      \label{div_pvv.4.19}\edlabel{div_pvv.4.19}
	  
	% new div opening: depth here is 2
	
	  \bigskip
	  \begingroup
	  \large
	
	    
	    \stanza[\smallbreak]
	\label{pv.4.19a}\edlabel{pv.4.19a}\flagstanza{\tiny\textenglish{....4.19a}}तत्संशयेन जिज्ञासोर्भवेत् प्रकरणाश्रयः ।\&[\smallbreak]


	
	  \endgroup
	

	  \pstart तस्य साध्य{\color{DodgerBlue3}“संशयेन”} जिज्ञासा तस्याञ्च सत्यां साधनमुच्यत इति\edlabel{pvv.420-2}\footnote{\label{pvv.420-2}  २ परस्य दशावयवं वाक्यं । तत्र जिज्ञासासंशयप्रयोजनशक्यप्राप्तिसंशयष्युदासाः पञ्च प्रवृत्यङ्गानि । प्रतिज्ञादिपञ्चावयवाः । प्रवृत्त्यङ्गैरनेकान्तमाह साधनवचनाश्रवत्वेन संशयादीनामपि साधनत्वं स्यात् ।} {\color{DodgerBlue3}“जिज्ञासोः”} {\color{DodgerBlue3}“पुंसः संशयो”} जिज्ञासा च {\color{DodgerBlue3}“प्रकरणस्य”} साधनप्रस्तावस्या{\color{DodgerBlue3}“श्रयो”} निमित्तमिति {\color{DodgerBlue3}“भवेत्”} साधनं पक्षवचनवत् ॥
	\pend
      

	  \pstart ननु संशयजिज्ञासे प्रतिपाद्यप्रवर्तिते पक्षवचनन्तु वादिप्रवर्तितं तत्कथं तत्समुदायस्य साधनस्य वादिना निर्देशसंभव इत्याह ।
	\pend
      \leavevmode\marginnote{\textenglish{421/s}}
	  \bigskip
	  \begingroup
	  \large
	
	    
	    \stanza[\smallbreak]
	\label{pv.4.19b}\edlabel{pv.4.19b}\flagstanza{\tiny\textenglish{....4.19b}}विपक्षोपगमेप्येतत् तुल्यमित्यनवस्थितिः ॥ १९ ॥\&[\smallbreak]


	
	  \endgroup
	

	  \pstart एवन्तर्हि {\color{DodgerBlue3}“विपक्ष”}स्य साध्यविरुद्धस्य धर्म{\color{DodgerBlue3}“स्योपगमे\edlabel{pvv.421-1}\footnote{\label{pvv.421-1}  १ वादीना ।}”} पराभिप्रायेण {\color{DodgerBlue3}“नित्यशब्द”} इति साध्यनिर्देशकृते पार्श्वस्थानां तथा संशयनिरासार्थं यत्कृतकं तदनित्यं यथा घटः कृतकश्च शब्द इति पुनर्वादिनैवोच्यते । तदा नित्यत्वप्रतिज्ञायास्तदनित्यत्वाव्यभिचारिकृतकत्वहेतुप्रवर्त्तकत्वं {\color{DodgerBlue3}“तुल्यमिति”} नित्यत्वप्रतिज्ञाप्यनित्यप्रतिज्ञावत्\leavevmode\marginnote{\textenglish{83b/MA}} साधनं स्या{\color{DodgerBlue3}“दित्यनवस्थितिः”} साधनावयवानां\edlabel{pvv.421-2}\footnote{\label{pvv.421-2}  २ न चैवं तस्मान्न हेतुप्रवर्त्तकत्वेन प्रतिज्ञायाः साधनत्वं ।}॥ (१९)
	\pend
      \label{div_pvv.4.20}\edlabel{div_pvv.4.20}
	  
	% new div opening: depth here is 2
	

	  \pstart यतश्च पक्षवचनस्य साध्यसिद्धौ साक्षात् पारम्पर्येण वा {\color{DodgerBlue3}“सामर्थ्यं नास्ति ततः(।)”}
	\pend
      
	  \bigskip
	  \begingroup
	  \large
	
	    
	    \stanza[\smallbreak]
	\label{pv.4.20}\edlabel{pv.4.20}\flagstanza{\tiny\textenglish{...v.4.20}}अन्तरङ्गं तु सामर्थ्यं त्रिषु रूपेषु संस्थितम् ।&तत्र स्मृतिसमाधानं तद्वचस्येव संस्थितम् ॥ २० ॥\&[\smallbreak]


	
	  \endgroup
	

	  \pstart अन्तरङ्गं {\color{DodgerBlue3}“सामर्थ्यं तु”} शब्दोऽवधारणे भिन्नक्रमश्चेति । {\color{DodgerBlue3}“त्रिष्वेव”} पक्षधर्मतादिषु रूपेषु संस्थितं । {\color{DodgerBlue3}“तत्र”} त्रिरूपलिङ्गे साध्यसाधनशक्ति{\color{DodgerBlue3}“स्मृतेः समा\edlabel{pvv.421-3}\footnote{\label{pvv.421-3}  ३ उत्पादनं ।}धानमारोपणं तद्वचसि”} त्रिरूपलिङ्गप्रतिपादकवचन {\color{DodgerBlue3}“एव संस्थितं”} । अतस्तदेव पारम्पर्येण साध्यसिद्धेरङ्गत्वात् प्रमाणन्न पक्षवचनं ॥ (२०)
	\pend
      \label{div_pvv.4.21_4.22}\edlabel{div_pvv.4.21_4.22}
	  
	% new div opening: depth here is 2
	

	  \pstart ननु (।)
	\pend
      
	  \bigskip
	  \begingroup
	  \large
	
	    
	    \stanza[\smallbreak]
	\label{pv.4.21}\edlabel{pv.4.21}\flagstanza{\tiny\textenglish{...v.4.21}}अख्यापिते हि विषये हेतुवृत्तेरसंभवात् ।&विषयख्यापनादेव सिद्धौ चेत्तस्य शक्तता ॥ २१ ॥\&[\smallbreak]


	
	  \endgroup
	
	  \bigskip
	  \begingroup
	  \large
	
	    
	    \stanza[\smallbreak]
	\label{pv.4.22a}\edlabel{pv.4.22a}\flagstanza{\tiny\textenglish{....4.22a}}उक्तमत्र;\&[\smallbreak]


	
	  \endgroup
	

	  \pstart {\color{DodgerBlue3}“अख्यापिते”}ऽप्रतिपादिते साधनस्य {\color{DodgerBlue3}“विषये”} साध्ये {\color{DodgerBlue3}“हेतोर्वृत्ते”}रेव {\color{DodgerBlue3}“ह्यसम्भवात्”} तस्य पक्षवचनस्य {\color{DodgerBlue3}“विषयख्यापनादेव”} साध्यस्य सिद्धौ पारम्पर्येण शक्ततेति {\color{DodgerBlue3}“चेत्”} । (२१) उक्तमत्र संशयजिज्ञासयोरपि साधनप्रवृर्त्तकत्वात्\edlabel{pvv.421-2-bis}\footnote{\label{pvv.421-2-bis}  २ न चैवं तस्मान्न हेतुप्रवर्त्तकत्वेन प्रतिज्ञायाः साधनत्वं ।} साधनत्वप्रसङ्ग इति ।
	\pend
      

	  \pstart किञ्च (।)
	\pend
      
	  \bigskip
	  \begingroup
	  \large
	
	    
	    \stanza[\smallbreak]
	\label{pv.4.22b}\edlabel{pv.4.22b}\flagstanza{\tiny\textenglish{....4.22b}}विनाप्यस्मात् कृतकः शब्द ईदृशाः ।&सर्वेऽनित्या इति प्रोक्तेप्यर्थात् तन्नाशधोर्भवेत् ॥ २२ ॥\&[\smallbreak]


	
	  \endgroup
	

	  \pstart {\color{DodgerBlue3}“अस्मात्”} पक्षवचनाद् {\color{DodgerBlue3}“विनापि कृतकः शब्द ईदशा”} ये कृतकास्ते {\color{DodgerBlue3}“सर्व्वेऽनित्या इति”} पक्षधर्मताव्याप्तिवचने {\color{DodgerBlue3}“प्रोक्तेप्यर्थात्”} तस्य शब्दस्य {\color{DodgerBlue3}“नाशधीर्भवेत्”} । अनित्य\leavevmode\marginnote{\textenglish{422/s}} त्वाव्यभिचारि कृतकत्वं शब्दे वर्त्तमानमनित्यतां तत्र गमयत्येवेति निष्फलं \edlabel{pvv.422-1}\footnote{\label{pvv.422-1}  १ द्वितीयकृतकशब्दवदुच्यमाने निग्रहस्थानं ।} {\color{DodgerBlue3}“पक्षवचनं”} । (२२)
	\pend
      \label{div_pvv.4.23}\edlabel{div_pvv.4.23}
	  
	% new div opening: depth here is 2
	

	  \begin{center}%% label @type='head'
	\textbf{(२) प्रतिज्ञा न साधनावयवः}
	\end{center}
	

	  \pstart तस्मात् (।)
	\pend
      
	  \bigskip
	  \begingroup
	  \large
	
	    
	    \stanza[\smallbreak]
	\label{pv.4.23}\edlabel{pv.4.23}\flagstanza{\tiny\textenglish{...v.4.23}}अनुक्तावपि पक्षस्य सिद्धेरप्रतिबन्धतः ।&त्रिष्वन्यतमरूपस्यैवानुक्तिर्न्यूनतोदिता ॥ २३ ॥\&[\smallbreak]


	
	  \endgroup
	

	  \pstart {\color{DodgerBlue3}“पक्षस्यानुक्तावपि साध्यसिद्धेरप्रतिबन्धतः”} अविरोधात् {\color{DodgerBlue3}“त्रि”} पक्षधर्मतादिषु {\color{DodgerBlue3}“रूपेष्वन्यतमस्य”} एकस्या{\color{DodgerBlue3}“नुक्तिर्न्यूनतोदिता”} साधनदोषो न तु पक्षाद्यवचनं । (२३)
	\pend
      \label{div_pvv.4.24}\edlabel{div_pvv.4.24}
	  
	% new div opening: depth here is 2
	

	  \pstart यदि \edlabel{pvv.422-2}\footnote{\label{pvv.422-2}  २ पक्षस्यासाधनत्व गुणमाह ।} च प्रतिज्ञा साधनमिष्यते तदा साध्यनिर्देशः प्रतिज्ञेति प्रतिज्ञालक्षणमतिव्यापि स्यात् । असिद्धस्य हेतोर्दृष्टान्तस्य चासिद्धस्य साध्यत्वं साधनत्वञ्चास्तीति प्रतिज्ञार्थं स्यात् । यस्य तु मते प्रतिज्ञा न साधनं (।)
	\pend
      
	  \bigskip
	  \begingroup
	  \large
	
	    
	    \stanza[\smallbreak]
	\label{pv.4.24}\edlabel{pv.4.24}\flagstanza{\tiny\textenglish{...v.4.24}}साध्योक्तिं वा प्रतिज्ञां स वदन् दोषैर्न युज्यते ।&साधनाधिकृतेरेव हेत्वाभासाप्रसङ्गतः ॥ २४ ॥\&[\smallbreak]


	
	  \endgroup
	

	  \pstart {\color{DodgerBlue3}“स साध्योक्तिं”} साध्यनिर्देशं {\color{DodgerBlue3}“प्रतिज्ञां वदन्न”}नन्तरोक्तै{\color{DodgerBlue3}“र्दोषैर्न युज्यते । साधनाधिकृतेः”} साधनत्वेनाधिकार{\color{DodgerBlue3}“देव”} प्रतिज्ञार्थस्य \edlabel{pvv.422-3}\footnote{\label{pvv.422-3}  ३ साधनस्य विजातीयत्वात् ।} {\color{DodgerBlue3}“हेत्वाभासेष्वप्रसङ्गतः”} (२४)
	\pend
      \label{div_pvv.4.25}\edlabel{div_pvv.4.25}
	  
	% new div opening: depth here is 2
	

	  \pstart यस्माद् (।)
	\pend
      
	  \bigskip
	  \begingroup
	  \large
	
	    
	    \stanza[\smallbreak]
	\label{pv.4.25}\edlabel{pv.4.25}\flagstanza{\tiny\textenglish{...v.4.25}}अविशेषोक्तिरप्येकजातीये संशयावहा ।&अन्यथा सर्व्वसाध्योक्तेः प्रतिज्ञात्वं प्रसज्यते ॥ २५ ॥\&[\smallbreak]


	
	  \endgroup
	

	  \pstart {\color{DodgerBlue3}“अविशेषोक्तिः”} \edlabel{pvv.422-4}\footnote{\label{pvv.422-4}  ४ साधनासाधनविभागं विनोक्तिः ।}सामान्याभिधानम{\color{DodgerBlue3}“प्येकजातीये \edlabel{pvv.422-5}\footnote{\label{pvv.422-5}  ५ साध्य एव न साधने ।} संशयावहा”} तत्त्वार्थशङ्कोपाधिका न सर्व्वत्रेति न्यायः । ततोऽसाधनमेव साध्यं प्रतिज्ञा । नत्वसिद्धहेतुदृष्टान्तादिकं तस्य साधनत्वेनेष्टत्वात् । {\color{DodgerBlue3}“अन्यथा”} यद्येवं नाभ्युपगम्यते तदा {\color{DodgerBlue3}“सर्व्वस्याः साध्योक्ते”}र्घटं \edlabel{pvv.422-6}\footnote{\label{pvv.422-6}  ६ असिद्धस्य करणात् ।}करोतीत्यादेः {\color{DodgerBlue3}“प्रतिज्ञात्वं प्रसज्यते”} । ज्ञापकहेत्वधिकारात् तत्साध्यस्यैव प्रतिज्ञात्वं न कारकहेतुसाध्यस्येति चेत् । यद्येवमसाधनभूतसाध्यनिर्देशः प्रतिज्ञा न साधननिर्देश इति सिद्धं । (२५)
	\pend
      \label{div_pvv.4.26}\edlabel{div_pvv.4.26}
	  
	% new div opening: depth here is 2
	

	  \pstart \leavevmode\marginnote{\textenglish{423/s}}स्यादेतत् \edlabel{pvv.423-1}\footnote{\label{pvv.423-1}  १ सिद्धो हि शब्दादिकः साध्यधर्मी । अन्यथाऽश्रयासिद्धो हेतुरपक्षधर्मत्वात् स्यात् कथं साधकः ।} (।)
	\pend
      
	  \bigskip
	  \begingroup
	  \large
	
	    
	    \stanza[\smallbreak]
	\label{pv.4.26}\edlabel{pv.4.26}\flagstanza{\tiny\textenglish{...v.4.26}}सिद्धोक्तेः साधनत्वाच्च परस्यापि न दुष्यति ।&इदानीं साध्यनिर्देशः साधनावयवः कथम् ॥ २६ ॥\&[\smallbreak]


	
	  \endgroup
	

	  \pstart {\color{DodgerBlue3}“परस्यापि \edlabel{pvv.423-2}\footnote{\label{pvv.423-2}  २ योपि प्रतिज्ञासाधनमाह । तस्यापि हेत्वाभासवचने न प्रतिज्ञात्वं ।} सिद्धोक्तेः”} सिद्धार्थप्रतिपादकत्वात् । {\color{DodgerBlue3}“साधनत्वाच्च”} प्रतिज्ञात्वमिष्टं । अतो हेत्वाभासादि प्रतिज्ञात्वप्रसङ्गेन {\color{DodgerBlue3}“न दुष्यति । हेत्वाभासाद्यसि”}-\leavevmode\marginnote{\textenglish{84a/MA}} द्धत्वादसाधनमंसाधनत्वाच्च न प्रतिज्ञा । {\color{DodgerBlue3}“इदानी\edlabel{pvv.423-3}\footnote{\label{pvv.423-3}  ३ यद्यसिद्धाभिधानाद्धेत्वाभासा न साधनं तदा ।}”}मस्मिन्नभ्युपगमे {\color{DodgerBlue3}“साध्यनिर्देशो”}ऽसिद्धत्वात् {\color{DodgerBlue3}“साधनावयवः कथं”} (।) न हि प्रतिज्ञार्थः सिद्धः । तदर्थमेव साधनोपन्यासात् । असिद्धश्च न साधनं हेत्वाभासवत् ॥ (२६)
	\pend
      \label{div_pvv.4.27}\edlabel{div_pvv.4.27}
	  
	% new div opening: depth here is 2
	

	  \pstart या च स्व\edlabel{pvv.423-4}\footnote{\label{pvv.423-4}  ४ न्यायमुखटीकाकारादेः ।}यूथ्यानां पूर्व्वपक्षपरिहारोक्तिः (।)
	\pend
      
	  \bigskip
	  \begingroup
	  \large
	
	    
	    \stanza[\smallbreak]
	\label{pv.4.27}\edlabel{pv.4.27}\flagstanza{\tiny\textenglish{...v.4.27}}साभासोक्याद्युपक्षेपपरिहारविडम्बना ।&असम्बद्धा तथा ह्येष न न्याय्य इति सूचितम् ॥ २७ ॥\&[\smallbreak]


	
	  \endgroup
	

	  \pstart पक्ष\edlabel{pvv.423-5}\footnote{\label{pvv.423-5}  ५ प्रयोगस्तु पक्षवचनं साधनं साधनकाले उपादानात् हेतुदृष्टान्तवत् । पूर्व्वपक्षः ।}वचनं साधनं साभास(साभासार्थ)त्वादिति चेत् । न प्रत्यक्षेणा-\edlabel{pvv.423-6}\footnote{\label{pvv.423-6}  ६ परिहारः ।} नेकान्तात् प्रत्यक्षं साभासमपि न कस्यचित् प्रमाणस्य साधनं/?/ वचना\edlabel{pvv.423-7}\footnote{\label{pvv.423-7}  ७ तेनैव पर आशङ्क्यते ।}त्मत्वे सति साभास\edlabel{pvv.423-8}\footnote{\label{pvv.423-8}  ८ प्रत्यक्षमवचनात्मकं ।}त्वात् साधनत्वमिति चेत् । न दूषणेना\edlabel{pvv.423-9}\footnote{\label{pvv.423-9}  ९ परिहारः । साधनकाले दूषणमप्युपादीयते न च तत्साधनं ।}नेकान्तात् । दूषणं साभासंवचनात्मत्वेपि न साधनमितिसापि साभासोक्तिरा\edlabel{pvv.423-10}\footnote{\label{pvv.423-10}  १० अत्र यदि परः प्रदूषयति अदूषणत्वे सति साभासोक्तिर्हेतुस्तदा नैवानेकान्तः । एवं यत्र यत्र न्या य मुखटीकाकृता व्यभिचारा उच्यते । तत्र तत्र परेण विशेषणमुच्यत इति परंपरा । विरुद्धाव्यभिचार्युपक्षेपे च पक्षवचनं न साधनमसिद्धोक्तेरसिद्धदृष्टान्तवचनवत् । इत्युक्त एव बाधितः स्यात् ।}दिर्यस्य तौ{\color{DodgerBlue3}“साभासोक्त्यादी”} \leavevmode\marginnote{\textenglish{424/s}} {\color{DodgerBlue3}“उपक्षेपपरिहारौ”} । तावेव {\color{DodgerBlue3}“विडम्वना”}ऽयुक्ततया । अत एवाह (।) {\color{DodgerBlue3}“असम्बद्धा । तथा हि”} साभासत्वस्य विपर्यये बाधक\edlabel{pvv.424-1}\footnote{\label{pvv.424-1}  १ वचने साभासत्वासाधनत्वयोरविरोधात् ।}प्रमाणभावादेवाहेतुत्वादसम्बन्धः पूर्व्वपक्षः । ततस्तमनुम\edlabel{pvv.424-2}\footnote{\label{pvv.424-2}  २ विरुद्धाव्यभिचार्युपक्षेपादिशोभनं स्यात् सन्दिग्धं व्यतिरेकाद् भावनं न्याय्यं ।}त्य प्रत्यक्षेणानेकान्ततापादनमशो\edlabel{pvv.424-2-bis}\footnote{\label{pvv.424-2-bis}  २ विरुद्धाव्यभिचार्युपक्षेपादिशोभनं स्यात् सन्दिग्धं व्यतिरेकाद् भावनं न्याय्यं ।}भनं । पुनर्वचनात्मत्वं विशेषणं परोक्तमप्रतिक्षिप्य दूषणाभासेनानेकान्ततापादनं चायुक्तं । तदेव\edlabel{pvv.424-3}\footnote{\label{pvv.424-3}  ३ तत्प्रतिक्षेपोपायमाह ।} हि हेतोर्व्विशेषणमुपयुक्तं(।)यद्विपक्षाद्धेतुं व्यावर्त्तयति न च वचनात्मत्वाऽसाधनत्वयोः कश्चिद्विरोधो येनासाधनाद् वचनात्मत्वनिवृत्तेर्व्विशेषणसाफल्यं स्यात् । तथा ह्येष विपक्षादव्यावर्त्तकहेतुविशेषणोपन्यासो {\color{DodgerBlue3}“न न्याय्य”} इति वर्ण्णितं\edlabel{pvv.424-4}\footnote{\label{pvv.424-4}  ४ युक्तं तु विशेषणं वेदचिन्तायां अपौरुषेयं ।}प्राक् वेदनित्यतासिध्यर्थंमध्ययनपूर्व्वकमित्युक्ते भारताध्ययनेनानेकान्ततामापादितां प्रतिषेद्धुं वेदाध्ययनत्वे सतीति विशेषणं मी मां स केनोपन्यस्तं तदपि करण(कृति)पूर्व्वकं भारताध्ययनवद् स्यात् न कश्चिद् विरोधः । ततो विपक्षादव्यावर्त्तकं विशेषणमयुक्तमित्युक्तं प्राक् ॥ (२७)
	\pend
      \label{div_pvv.4.28}\edlabel{div_pvv.4.28}
	  
	% new div opening: depth here is 2
	

	  \begin{center}%% label @type='head'
	\textbf{(३) पक्षलक्षणकरणे प्रयोजनम्}
	\end{center}
	

	  \pstart ननु यदि पक्षवचनमसाधनं सामर्थ्यगम्याभिधेयञ्च तदाचा र्ये ण पक्षलक्षणं कृतं किमर्थमित्याह (।)
	\pend
      
	  \bigskip
	  \begingroup
	  \large
	
	    
	    \stanza[\smallbreak]
	\label{pv.4.28}\edlabel{pv.4.28}\flagstanza{\tiny\textenglish{...v.4.28}}गम्यार्थत्वेपि साध्योक्तेरसंमोहाय लक्षणम् ।&तच्चतुर्लक्षणं रूपनिपातेषु स्वयं पदैः ॥ २८ ॥\&[\smallbreak]


	
	  \endgroup
	

	  \pstart {\color{DodgerBlue3}“साध्यव्या”}प्तपक्षवचनसामर्थ्याद् {\color{DodgerBlue3}“गम्यार्थत्वेपि साध्योक्तेः”} पक्षवचनस्य {\color{DodgerBlue3}“लक्षण”}मुक्तम{\color{DodgerBlue3}“संमोहाय”} विप्रतिपत्तिनिराकरणेन साध्यप्रतिपत्त्यर्थं । तथा ह्यात्मार्थत्वं {\color{DodgerBlue3}“साध्यमपि सां ख्या”} असाध्यमाचक्षते परार्थत्वमसाध्यमपि साध्यमिति सन्ति विप्रतिपत्तयः । तच्च साध्यं {\color{DodgerBlue3}“चतुर्लक्षण”}मुक्तं । स्वरूपेणैव निर्देश्यः स्वयमिष्टोऽनिराकृतः पक्ष इत्यत्र लक्षणचतुष्टयप्रतिपादकै रुपनिपातेषु {\color{DodgerBlue3}“स्वयं पदै”}र्यथाक्रमम् (। २८)
	\pend
      \label{div_pvv.4.29}\edlabel{div_pvv.4.29}
	  
	% new div opening: depth here is 2
	
	  \bigskip
	  \begingroup
	  \large
	
	    
	    \stanza[\smallbreak]
	\label{pv.4.29}\edlabel{pv.4.29}\flagstanza{\tiny\textenglish{...v.4.29}}असिद्धासाधनार्थोक्तवाद्यभ्युपगतग्रहः ।&अनुक्तोपीच्छया व्याप्तः साध्य आत्मार्थवन्मतः ॥ २९ ॥\&[\smallbreak]


	
	  \endgroup
	

	  \pstart {\color{DodgerBlue3}“असिद्ध”}स्या{\color{DodgerBlue3}“साधन”}स्यार्थो{\color{DodgerBlue3}“क्त”}स्य {\color{DodgerBlue3}“वाद्यभ्युपगतस्य ग्रहः”} । असिद्धस्वभावत्वात् साध्यस्य न सिद्धस्य ग्रहणं । ततः सिद्धं चाक्षुषत्वादि रूपादेर्न साध्यं । निपातैव \leavevmode\marginnote{\textenglish{425/s}} कारकरणेनासाधनस्य ग्रहणं ततश्चाक्षुषत्वाद्यसिद्धमपि साधनत्वेन शब्देभिधीयमानं न साध्यं । दृष्टशब्देनार्थोक्तस्यापि ग्रहणं । ततोऽनुक्तमप्यात्मार्थत्वं संघातत्वाच्चक्षुरादेः सांख्यस्य साध्यं । स्वयं शब्देन वाद्यभ्युपगतस्य ग्रहणं । ततः शास्त्राभ्युपगतस्याकाशगुणत्वादेः शब्दे धर्मिणि वादिनाऽनित्यत्वे साधयितुमारब्धे-\leavevmode\marginnote{\textenglish{84b/MA}} ऽसाध्यता । अल्पवक्तव्यतयाऽर्थोक्तस्य तावत् साध्यतां समर्थयितुमाह । वादिनो{\color{DodgerBlue3}“ऽनुक्तोपीच्छया व्याप्तः साध्यो मतः । आत्मार्थवत्”} । यथा आत्मास्ति न वेति विवादे तत्साधनार्थं सां ख्ये न परार्थाश्चक्षुरादयः संघातत्वात् शयनाश(? स)नाद्यङ्गवत् । इत्युक्तस्य साधनस्यात्मार्थत्वमनुक्तमपि साध्यमिच्छाविषयत्वात् ॥ (२९)
	\pend
      \label{div_pvv.4.30}\edlabel{div_pvv.4.30}
	  
	% new div opening: depth here is 2
	

	  \pstart नन्विष्ट शब्देनानिष्टस्य सर्व्वस्य निरासात् । शास्त्रोपगतस्यापि वाद्यनिष्ठस्यासाध्यत्वं सिद्धं । तन्निष्फलं स्वयंपदमित्याह । व्यवच्छेदफलत्वाच्छब्दा- नामिष्टशब्दात् (।)
	\pend
      
	  \bigskip
	  \begingroup
	  \large
	
	    
	    \stanza[\smallbreak]
	\label{pv.4.30}\edlabel{pv.4.30}\flagstanza{\tiny\textenglish{...v.4.30}}सर्वान्येष्टनिवृत्तावप्याशङ्कास्थानवारणम् ।&वृत्तौ स्वयंश्रुतेनाह कृता चैषा तदर्थिका ॥ ३० ॥\&[\smallbreak]


	
	  \endgroup
	

	  \pstart {\color{DodgerBlue3}“सर्व्व”}स्य वादिनोऽ{\color{DodgerBlue3}“न्येन”} शास्त्रोदिना इ{\color{DodgerBlue3}“ष्ट”}स्य {\color{DodgerBlue3}“निवृत्तौ”} सिद्धाया{\color{DodgerBlue3}“मपि”} शास्त्रेणेष्टं वादिनोपीष्टमेवेति {\color{DodgerBlue3}“शङ्कास्थान”}स्य विप्रतिपत्तिविषयस्य {\color{DodgerBlue3}“वारणं”} फलं {\color{DodgerBlue3}“स्वयंश्रुते”}नाचार्यो {\color{DodgerBlue3}“वृत्ता”}वाह । स्वयमिति शास्त्रानपेक्षमभ्युपगमन्दर्शयति । {\color{DodgerBlue3}“एवा”} स्वयंश्रुति{\color{DodgerBlue3}“स्तदर्थिका”} विप्रतिपत्तिनिराकरणार्था {\color{DodgerBlue3}“कृता”} ॥ (३०)
	\pend
      \label{div_pvv.4.31}\edlabel{div_pvv.4.31}
	  
	% new div opening: depth here is 2
	

	  \begin{center}%% label @type='head'
	\textbf{(४) आत्मार्थत्वविवादे दोषः ।}
	\end{center}
	

	  \pstart \edlabel{pvv.425-1}\footnote{\label{pvv.425-1}  १ यदि वादिनेष्ट एव साध्यस्तदा धर्मविशेषविपरीतसाधनादीनां विरुद्धानामसम्भव एवेत्याह ।} य एवेच्छया विषयीकृतः स (।)
	\pend
      
	  \bigskip
	  \begingroup
	  \large
	
	    
	    \stanza[\smallbreak]
	\label{pv.4.31}\edlabel{pv.4.31}\flagstanza{\tiny\textenglish{...v.4.31}}विशेषस्तद्व्यपेक्षत्वात् कथितो धर्मधर्मिणोः ।&अनुक्तावपि वाञ्छाया भवेत् प्रकरणाद् गतिः ॥ ३१ ॥\&[\smallbreak]


	
	  \endgroup
	

	  \pstart विशेषो {\color{DodgerBlue3}“धर्मधर्मिणोः”} सम्बन्धी 1/?/ {\color{DodgerBlue3}“व्यपेक्ष”}त्वात् । इच्छारचितात् सम्बन्धात् साध्यत्वेन {\color{DodgerBlue3}“कथितः”} । चक्षुरादीनां संहतविषयं पारार्थ्यमिति धर्मस्य {\color{DodgerBlue3}“विशेषः”} साध्यः । परार्थस्य साध्यत्वात् । परार्थाः सन्तश्चक्षुरादयोऽसंहतार्था इति धर्मिणो विशेषः साध्यः । चक्षुरादयोऽनेकाणुसञ्चयात्मिकाः क्रमेणैककालञ्च संहताः । ज्ञानादि तु \leavevmode\marginnote{\textenglish{426/s}} कालभेदेनानेकत्वात् संहतानि । तेषां परार्थानां सतामसंहतविषयत्वमेवेच्छाविषयत्वात् साध्यं । आत्मनः सर्व्वकालमेकत्वे नासंहतत्वात् । कथं पुनरात्मार्थत्वस्यानुक्तौ तद्विषयाया वाञ्छायाः प्रतीतिरित्याह । {\color{DodgerBlue3}“वाञ्छाया अनुक्तावपि”} मुख्यं शब्देन {\color{DodgerBlue3}“प्रकरणा”}दात्मास्ति नास्तीति संशये सति तत्साधनोपन्यासप्रस्तावाद् {\color{DodgerBlue3}“गतिः”} प्रतीति{\color{DodgerBlue3}“र्भवति”} ॥ (३१)
	\pend
      \label{div_pvv.4.32}\edlabel{div_pvv.4.32}
	  
	% new div opening: depth here is 2
	

	  \pstart आत्मार्थत्वस्य विवादे को दोष इत्याह (।)
	\pend
      
	  \bigskip
	  \begingroup
	  \large
	
	    
	    \stanza[\smallbreak]
	\label{pv.4.32}\edlabel{pv.4.32}\flagstanza{\tiny\textenglish{...v.4.32}}अनन्वयोपि दृष्टान्ते दोषस्तस्य यथोदितम् ।&आत्मा परश्चेत् सोऽसिद्ध इति तत्रेष्टघातवत् ॥ ३२ ॥\&[\smallbreak]


	
	  \endgroup
	

	  \pstart {\color{DodgerBlue3}“तस्ये”}च्छाविषयस्यात्मार्थत्वस्य साध्या{\color{DodgerBlue3}“नन्वयो दृष्टान्ते दोषः\edlabel{pvv.426-1}\footnote{\label{pvv.426-1}  १ उक्त विरुद्धत्वं ।}”} । अपिशब्दाद्वक्ष्यमाण इष्टविधातश्च । {\color{DodgerBlue3}“यथोदित”}माचार्यव सु ब न्धु ना । परार्थाश्चक्षुरादय इत्यत्र {\color{DodgerBlue3}“परश्चेदात्मा”} विवक्षितः {\color{DodgerBlue3}“सोऽसिद्धो”} दृष्टान्त {\color{DodgerBlue3}“इति । तत्रा”}न्वये सती{\color{DodgerBlue3}“ष्टविधातवत्”} । साधनं इष्टात्मार्थत्वविपर्ययेणान्वयात् तत्साधकत्वात् ॥ (३२)
	\pend
      \label{div_pvv.4.33}\edlabel{div_pvv.4.33}
	  
	% new div opening: depth here is 2
	

	  \pstart अथात्मार्थत्व न साध्यमित्याह (।)
	\pend
      
	  \bigskip
	  \begingroup
	  \large
	
	    
	    \stanza[\smallbreak]
	\label{pv.4.33}\edlabel{pv.4.33}\flagstanza{\tiny\textenglish{...v.4.33}}साधनं यद्विवादे न न्यस्तं तच्चेन्न साध्यते ।&किं साध्यमन्यथानिष्टं भवेद् वैफल्यमेव वा ॥ ३३ ॥\&[\smallbreak]


	
	  \endgroup
	

	  \pstart {\color{DodgerBlue3}“य”}स्यात्मनोर्थस्य {\color{DodgerBlue3}“विवादे”}ऽस्ति नास्तीति सन्देहे न साधनं {\color{DodgerBlue3}“न्यस्त”}मुपन्यस्तं {\color{DodgerBlue3}“तच्चेन्न साध्यते कि”}मिदानीं {\color{DodgerBlue3}“साध्यं”} स्यात् । {\color{DodgerBlue3}“अन्यथा”} विवादविषयो यदि न साध्यं तदा{\color{DodgerBlue3}“निष्टं”} विपर्ययसिद्धिः स्यात् । यथा व्युत्पन्नसर्व्वशब्दवादिनं\edlabel{pvv.426-2}\footnote{\label{pvv.426-2}  २ वैयाकरणं ।} प्रत्यव्युत्पन्नसंज्ञाशब्दवादिना तदर्थवत्वसिद्ध्यर्थं साधनमुच्यते ॥
	\pend
      

	  \pstart संज्ञिसम्बन्धात् प्रागर्थव\edlabel{pvv.426-3}\footnote{\label{pvv.426-3}  ३ संज्ञी नास्तीति न संज्ञाशब्दस्तदर्थवानिति मत्वायं ।}च्छब्दरूपं विभक्तिदर्शनात् तदन्यशब्दवदिति । अत्र \edlabel{pvv.426-4}\footnote{\label{pvv.426-4}  ४ नित्ये शब्दार्थसम्बन्धे ।} गच्छतीति गौरित्येकार्थसमवायात् क्रियोपलक्षितेन बाह्यसामान्येनार्थवान् गोशब्दः सिद्धोऽव्युत्पन्नवादिनः । नतु स्वरूपमात्रेणा\edlabel{pvv.426-5}\footnote{\label{pvv.426-5}  ५ तस्य साध्यत्वेनानभ्युपगमात् ।}र्थेनार्थवान् । अर्थमात्रजञ्च साध्यत्वेनोद्दिष्टं न तु स्वरूपेणार्थेनार्थवत्वं । ततो दृष्टान्ते विभक्त्यन्तस्य वाक्यार्थवत्वेन व्याप्तिसिद्धेर्देवदत्तादावपि पक्षीकृते संज्ञाशब्दे देवैर्दत्तो देवदत्त इति वाक्यार्थवत्वं स्वरूपार्थवत्वे विरुद्धं सिध्यति ॥
	\pend
      \leavevmode\marginnote{\textenglish{427/s}}

	  \pstart अथवेष्टस्य साध्यत्वाभावे परार्थाश्चक्षुरादयः संहतत्वादित्यत्रात्मार्थत्वस्यासंहतपारार्थ्यस्यासाध्यत्वात् । ज्ञानहेतुत्वेन संहतपारार्थस्य बौद्धेनापीष्टे । साध-\leavevmode\marginnote{\textenglish{85a/MA}} नवैफल्यमेव वा स्यात् । (३३)
	\pend
      \label{div_pvv.4.34_4.35}\edlabel{div_pvv.4.34_4.35}
	  
	% new div opening: depth here is 2
	
	  \bigskip
	  \begingroup
	  \large
	
	    
	    \stanza[\smallbreak]
	\label{pv.4.34a}\edlabel{pv.4.34a}\flagstanza{\tiny\textenglish{....4.34a}}सद्वितीयप्रयोगेषु निरन्वयविरुद्धते ।&एतेन कथिते साध्यं;\&[\smallbreak]


	
	  \endgroup
	

	  \pstart {\color{DodgerBlue3}“एतेन”} साध्यत्वेनेष्टस्यानन्वयदोषदर्शनेन {\color{DodgerBlue3}“सद्वितीयप्रयोगेषु”} चा र्व्वा ककृतेषु यथाभिव्यक्तचैतन्यशरीरलक्षणपुरुष\edlabel{pvv.427-1}\footnote{\label{pvv.427-1}  १ घटयोरन्यतरेण ।}सद्वितीयो घटः । अनुत्पन्नत्वात् । कुड्यवदिति शरीरमेवाभिव्यक्तचैतन्यं पुरुषो नात्मा कश्चित् परलोकी तेन सद्वितीयत्वं (ससहायत्त्वं) घटस्य साध्यत इति प्रयोगफलं । तत्र च {\color{DodgerBlue3}“निरन्वयविरुद्धते कथिते”} । तथा ह्यभिव्यक्तचैतन्यदेहलक्षणपुरुषेण सद्वितीयत्वं {\color{DodgerBlue3}“साध्यं”} ।\edlabel{pvv.427-2}\footnote{\label{pvv.427-2}  २ प्रत्यक्षविषयत्वादनुमानमुक्तं ।} तेन च कुड्येन्वयो न दृष्ट इति निरन्वयता (।) घटस्य तु कुड्येऽन्वयो दृष्ट इति तेन सद्वितीयत्वसाधनात् विरुद्धता स्यात् ।
	\pend
      

	  \begin{center}%% label @type='head'
	\textbf{(व्यक्त्यसिद्धौ न सामान्यम्)}
	\end{center}
	
	  \bigskip
	  \begingroup
	  \large
	
	    
	    \stanza[\smallbreak]
	\label{pv.4.34b}\edlabel{pv.4.34b}\flagstanza{\tiny\textenglish{....4.34b}}सामान्येनाथ सम्मतम् ॥ ३४ ॥\&[\smallbreak]


	
	  \endgroup
	
	  \bigskip
	  \begingroup
	  \large
	
	    
	    \stanza[\smallbreak]
	\label{pv.4.35}\edlabel{pv.4.35}\flagstanza{\tiny\textenglish{...v.4.35}}तदेवार्थान्तराभावाद् देहानाप्तौ न सिध्यति ।&वाच्यशून्य प्रलपतां तदेतज्जाड्यवर्ण्णितम् ॥ ३५ ॥\&[\smallbreak]


	
	  \endgroup
	

	  \pstart {\color{DodgerBlue3}“अथ सामान्येन”}\edlabel{pvv.427-3}\footnote{\label{pvv.427-3}  ३ अन्यतरार्थान्तरभावः सामान्यं साध्यं ।} विशेषमनुल्लिख्य सद्वितीयत्वं साध्यं कुड्ये सद्वितीयत्वमात्रेणान्वयात् । एवमपि {\color{DodgerBlue3}“तत्सा”}मान्य{\color{DodgerBlue3}“मेव”} न सिध्यति प्रतिवादिनः । घटादभिव्यक्तचैतन्यस्वभावतया{\color{DodgerBlue3}“ऽर्थान्तराभावात्”} । अर्थान्तरत्वासम्भवात् । {\color{DodgerBlue3}“देह”}स्या{\color{DodgerBlue3}“नाप्ता”}वर्थान्तरत्वेनासिद्धौ द्वयोर्भिन्नयोरन्यतरसद्वितीयत्वं सामान्यं स्यात् ।\edlabel{pvv.427-4}\footnote{\label{pvv.427-4}  ४ न हि गोव्यक्त्यभावे सामान्यं ।} न हि घटः स्वरुपेणैवान्यतरसद्वितीयः । देहन्तु नाभिव्यक्तचैतन्यलक्षणपुरुषमिच्छति प्रतिवादिति भेदाभावात् । तेनापि नान्यतरसद्वितीयत्वसिद्धिः । यतश्च न घटस्य स्वरूपेणैव सद्वितीयत्वसम्भवः । भेदाधिष्ठानत्वात् तस्य, नापि देहेनान्वयाभावात् । ततो वाच्यशून्यत्वमर्थशून्यत्वमन्यतरसद्वितीयत्वं {\color{DodgerBlue3}“प्रलपतां”} परलोकापवादीनां {\color{DodgerBlue3}“तदेतद”}न्यतरसद्वितीयत्वसाध्यवचनं {\color{DodgerBlue3}“जाड्यस्य वर्ण्णितं”} चेष्टितं ॥ (३४,३५)
	\pend
      \label{div_pvv.4.36}\edlabel{div_pvv.4.36}
	  
	% new div opening: depth here is 2
	\leavevmode\marginnote{\textenglish{428/s}}
	  \bigskip
	  \begingroup
	  \large
	
	    
	    \stanza[\smallbreak]
	\label{pv.4.36}\edlabel{pv.4.36}\flagstanza{\tiny\textenglish{...v.4.36}}तुल्यं नाशेपि चेच्छब्दघटभेदेन कल्पने ।&न सिद्धेन विनाशेन तद्वतः साधनाद् ध्वनेः ॥ ३६ ॥\&[\smallbreak]


	
	  \endgroup
	

	  \pstart {\color{DodgerBlue3}“नाशेपि”} साध्ये {\color{DodgerBlue3}“शब्दघटयोः”} साध्यदृष्टान्तधर्मिणोः सम्बन्धितया {\color{DodgerBlue3}“भेदेन\edlabel{pvv.428-1}\footnote{\label{pvv.428-1}  १ शब्दानित्यता घटनित्यतेति ।} कल्पने”} शब्दसम्बन्धिनो नाशस्य घटेन्वयाभावादसाध्यत्वं । घटसम्बन्धिनश्च शब्देऽसम्भवादसाध्यतेति तुल्यमिदमिति {\color{DodgerBlue3}“चेत् । न तुल्यं वानाशेन”} प्रध्वंसलक्षणेन {\color{DodgerBlue3}“सिद्धेन”} निश्चितेन {\color{DodgerBlue3}“ध्वनेस्तद्वतो”} विनाशवतः {\color{DodgerBlue3}“साधना”}द्विनाशसामान्यं साध्यं सिद्धं केवलं तद्वत्ता शब्दस्य न सिद्धेति साध्यते । यथा विनाशे सामन्येन सिद्धे सत्यसिद्धस्तद्वान् शब्दः साध्यते\edlabel{pvv.428-2}\footnote{\label{pvv.428-2}  २ एवमर्थान्तभावः सिद्धो नास्ति ।}। (३६)
	\pend
      \label{div_pvv.4.37}\edlabel{div_pvv.4.37}
	  
	% new div opening: depth here is 2
	
	  \bigskip
	  \begingroup
	  \large
	
	    
	    \stanza[\smallbreak]
	\label{pv.4.37a}\edlabel{pv.4.37a}\flagstanza{\tiny\textenglish{....4.37a}}तथार्थान्तरभावे स्यात् तद्वान् कुम्भोपि ।\&[\smallbreak]


	
	  \endgroup
	

	  \pstart {\color{DodgerBlue3}“तथार्थान्तरभावे”}ऽभि\edlabel{pvv.428-3}\footnote{\label{pvv.428-3}  ३ घटव्यक्त्या ।}व्यक्तचैतन्यस्वभावतया देहस्य घटात् वैजात्ये सिद्धे सति {\color{DodgerBlue3}“तद्वान् कुम्भोपि”} सिध्येत् । न चैतत् प्रतिवादी बोधियितुं शक्यते । तेनाचैतन्यस्य भूतव्यतिरिक्तस्यैव स्वीकारात् । यदि पुनरचेतनस्वभावतया घटजातीयेनैव देहेन\edlabel{pvv.428-3-bis}\footnote{\label{pvv.428-3-bis}  ३ घटव्यक्त्या ।} सद्वितीयत्वं घटस्य साध्यते तदा सिध्यत्येव । तथाविधस्य सद्वितीयत्वस्य सिद्धत्वाद् विनाशवत् । किन्तु वादिनो नेष्टिसिद्धिः । देहस्य चेतनस्वभावतयाऽसिद्धेः ॥
	\pend
      
	  \bigskip
	  \begingroup
	  \large
	
	    
	    \stanza[\smallbreak]
	\label{pv.4.37b}\edlabel{pv.4.37b}\flagstanza{\tiny\textenglish{....4.37b}}अनित्यता ।&विशिष्टा ध्वनिनान्वेति नो चेन्नायोगवाण्णात् ॥ ३७ ॥\&[\smallbreak]


	
	  \endgroup
	\leavevmode\marginnote{\textenglish{85b/MA}}

	  \pstart अथ ध्वनिना स्वसम्बन्धितया विशिष्टाऽनित्यता दृष्टान्तं {\color{DodgerBlue3}“ना”}न्वेतीति {\color{DodgerBlue3}“चेत्”} । {\color{DodgerBlue3}“ना”}नन्वयदोषो विशेषणेना{\color{DodgerBlue3}“योग”}स्यासम्बन्धस्य {\color{DodgerBlue3}“वारणात्”} ॥ (३७)
	\pend
      \label{div_pvv.4.38}\edlabel{div_pvv.4.38}
	  
	% new div opening: depth here is 2
	
	  \bigskip
	  \begingroup
	  \large
	
	    
	    \stanza[\smallbreak]
	\label{pv.4.38}\edlabel{pv.4.38}\flagstanza{\tiny\textenglish{...v.4.38}}द्विविधो हि व्यवच्छेदो वियोगापरयोगयोः ।&व्यवच्छेदादयोगे तु वार्ये नानन्वयागमः ॥ ३८ ॥\&[\smallbreak]


	
	  \endgroup
	

	  \pstart {\color{DodgerBlue3}“द्विविधो हि व्यवच्छेदो”} विशेषणेन दृष्टो {\color{DodgerBlue3}“वियोगापरयोगयोर”}योगान्ययोगयो{\color{DodgerBlue3}“र्व्यवच्छेदात्”} । यथा चैत्रो धनुर्द्धरः पा र्थो धनुर्द्धर इति । तत्र धर्मिणा विशेषणेनायोगे । वार्येऽनित्यताया {\color{DodgerBlue3}“नानन्वयागमो”}ऽनन्वयापत्तिर्न भवति । शब्दोऽनित्यो न वेत्ययोगः शङ्कितो विशेषणेनव्यावर्त्त्यते शब्दोऽनित्य इति । एवम्विधा चानित्यता नान्यसम्बन्धेन विरुध्यत इति नानन्वयो दृष्टान्ते ॥ (३८)
	\pend
      \label{div_pvv.4.39}\edlabel{div_pvv.4.39}
	  
	% new div opening: depth here is 2
	

	  \pstart उक्तार्थसंग्रहमाह ।
	\pend
      \leavevmode\marginnote{\textenglish{429/s}}
	  \bigskip
	  \begingroup
	  \large
	
	    
	    \stanza[\smallbreak]
	\label{pv.4.39}\edlabel{pv.4.39}\flagstanza{\tiny\textenglish{...v.4.39}}सामान्यमेव तत्साध्यं न च सिद्धप्रसाधम् ।&विशिष्टं धर्मिणा तच्च न निरन्वयदोषवत् ॥ ३९ ॥\&[\smallbreak]


	
	  \endgroup
	

	  \pstart {\color{DodgerBlue3}“तद”}नित्यतादि {\color{DodgerBlue3}“सामान्यमेव साध्यं”} न विशेषो येनानन्वयदोषः स्यात् । नन्वनित्यतादि सामान्यं {\color{DodgerBlue3}“सिद्धमेव क्व\edlabel{pvv.429-1}\footnote{\label{pvv.429-1}  १ विद्युदादौ ।}चित्”} साधने वैयर्थ्यमित्याह । {\color{DodgerBlue3}“न च सिद्धस्य”} क्वचित् सत्तामात्रेणानित्यत्वस्य {\color{DodgerBlue3}“प्रसाधनं”} । धर्मिण्ययोगव्यवच्छेदस्यासिद्धस्य प्रसाधनात् । न च {\color{DodgerBlue3}“धर्मिणा”}ऽयोगव्यवच्छेदतो {\color{DodgerBlue3}“विशिष्टं तच्चा”}नित्यतादि दृष्टान्ते {\color{DodgerBlue3}“निरन्वयदोषवत्”} (।) (न च)अयोगव्यवच्छेदेन धर्मिविशेषितस्य धर्म्यन्तरसम्बन्धाविरोधात् ॥ (३९)
	\pend
      \label{div_pvv.4.40}\edlabel{div_pvv.4.40}
	  
	% new div opening: depth here is 2
	
	  \bigskip
	  \begingroup
	  \large
	
	    
	    \stanza[\smallbreak]
	\label{pv.4.40}\edlabel{pv.4.40}\flagstanza{\tiny\textenglish{...v.4.40}}एतेन धर्मिधर्माभ्यां विशिष्टौ धर्मधर्मिणौ ।&प्रत्याख्यातौ निराकुर्वन् धर्मिण्येवमसाधनात् ॥ ४० ॥\&[\smallbreak]


	
	  \endgroup
	

	  \pstart {\color{DodgerBlue3}“एतेने”}ष्टस्य साध्यत्ववचनेन {\color{DodgerBlue3}“धर्मिधर्माभ्यां विशिष्टौ धर्मधर्मिणा”}वनन्वया{\color{DodgerBlue3}“न्निराकुर्वन्”} चार्व्वाको यथा न शब्दानित्यत्ववान् शब्दो नानित्यशब्दवान् वा शब्द इति । न हि शब्दानित्यत्वेनानित्यशब्देन वा क्वचिद् घटादौ दृष्टान्ते कृतकत्वस्यान्वयोस्ति तत इष्टविपर्यासिनाद् विरुद्धं कृतकत्वमिति स एवं वदन् प्रत्याख्यातः कथमित्याह । {\color{DodgerBlue3}“धर्मिणि”} शब्दे {\color{DodgerBlue3}“एवं”} धर्म्मिविशिष्टस्य धर्मस्य धर्माविशिष्टस्य वा धर्मिणोऽ{\color{DodgerBlue3}“साधनाद”}नित्यत्वमात्रस्य शब्दे साध्यत्वेनेष्टत्वात् । अन्यथाऽनित्यशब्दवति शब्दे सिद्धेपि शब्दो नानित्यः स्यात् । यदि धर्ममात्रं साध्यं तदा समुदायः साध्यो न स्यात्\edlabel{pvv.429-2}\footnote{\label{pvv.429-2}  २ न धर्मिणा सह समुदायसाधनात् । न हि शब्दे परः शब्दानित्यत्वसमुदायः साध्यः ततः तस्य निराकरणेपि न दोषः ।}। हेतोस्तदपवादोविरुद्धस्य न स्यादित्याह । धर्ममात्रस्य धर्मि\edlabel{pvv.429-3}\footnote{\label{pvv.429-3}  ३ आश्रयासिद्धिरन्यथा ।}साध्यत्वात् । (४०)
	\pend
      \label{div_pvv.4.41}\edlabel{div_pvv.4.41}
	  
	% new div opening: depth here is 2
	
	  \bigskip
	  \begingroup
	  \large
	
	    
	    \stanza[\smallbreak]
	\label{pv.4.41}\edlabel{pv.4.41}\flagstanza{\tiny\textenglish{...v.4.41}}समुदायापवादो हि न धर्मिणि विरुध्यते ।&साध्यं यतस्तथा नेष्टं साध्यो धर्मोत्र केवलः ॥ ४१ ॥\&[\smallbreak]


	
	  \endgroup
	

	  \pstart समुदाय एव साध्यो {\color{DodgerBlue3}“हि”} यस्मात् तस्माद् विरुद्धस्य हेतोरिष्ट{\color{DodgerBlue3}“समुदाया\edlabel{pvv.429-4}\footnote{\label{pvv.429-4}  ४ ॥ ॰ ॥ स्वयं शब्दात् । स च द्वयोरेकाभावे समुदाय एव निराकृतः स्यात् ।}पवादो न विरुध्यते”} । ननु समुदायस्य साध्यत्वेऽनन्वयदोष इष्टविघातो वा स्यादित्याह (।) तथा धर्मिधर्मसमुदायोऽन्यधर्मिसम्बन्धितया {\color{DodgerBlue3}“साध्यं यतो नेष्टं”} (।) तस्मान्नानन्वयो विरुद्धता वा । {\color{DodgerBlue3}“तथा ह्यत्र”}  शब्दादौ धर्मिणि {\color{DodgerBlue3}“धर्मो”}ऽनित्यतादिः {\color{DodgerBlue3}“केवलः साध्य”} इष्टः । तस्य चान्वयोस्तीति न दोषः । (४१)
	\pend
      \label{div_pvv.4.42}\edlabel{div_pvv.4.42}
	  
	% new div opening: depth here is 2
	

	  \pstart \leavevmode\marginnote{\textenglish{430/s}}उक्तमिष्टग्रहणस्य प्रयोजनं ॥
	\pend
      

	  \begin{center}%% label @type='head'
	\textbf{(५) स्वयंशब्दग्रहणफलम् ॥}
	\end{center}
	

	  \pstart स्वयं शब्दस्येदानीं वक्तुमाह ।
	\pend
      
	  \bigskip
	  \begingroup
	  \large
	
	    
	    \stanza[\smallbreak]
	\label{pv.4.42}\edlabel{pv.4.42}\flagstanza{\tiny\textenglish{...v.4.42}}एकस्य धर्मिणः शास्त्रें नानाधर्मस्थितावपि ।&साध्यः स्यादात्मनैवेष्ट इत्युपात्ता स्वयं श्रुतिः ॥ ४२ ॥\&[\smallbreak]


	
	  \endgroup
	

	  \pstart {\color{DodgerBlue3}“एकस्य”} शब्दादे{\color{DodgerBlue3}“र्द्धमिणः शास्त्रे नानाधर्मा”}णाममूर्त्तत्वानित्यताकाशगुणत्वादीनां\leavevmode\marginnote{\textenglish{86a/MA}} {\color{DodgerBlue3}“स्थिताव”}भ्युपगमेपि वादिना {\color{DodgerBlue3}“आत्मनैव”} साधनोपन्यासकाले साधयितुमिष्टो धर्मः {\color{DodgerBlue3}“साध्यः स्यात्”} । नान्य {\color{DodgerBlue3}“इति स्वयं”} श्रुतितिराचार्येणो{\color{DodgerBlue3}“पात्ता”} । (४२)
	\pend
      \label{div_pvv.4.43}\edlabel{div_pvv.4.43}
	  
	% new div opening: depth here is 2
	

	  \pstart यदि पुनः (।)
	\pend
      
	  \bigskip
	  \begingroup
	  \large
	
	    
	    \stanza[\smallbreak]
	\label{pv.4.43}\edlabel{pv.4.43}\flagstanza{\tiny\textenglish{...v.4.43}}शास्त्राभ्युपगमादेव सर्वादानात् प्रबाधने ।&तत्रैकस्यापि दोषः स्याद् यदि हेतुप्रतिज्ञयोः ॥ ४३ ॥\&[\smallbreak]


	
	  \endgroup
	

	  \pstart {\color{DodgerBlue3}“शास्त्रे”}णा{\color{DodgerBlue3}“भ्युपगमादेव”} सर्व्वेषां धर्माणा{\color{DodgerBlue3}“मादानात्”} परिग्रहात् वादिना {\color{DodgerBlue3}“तत्र”} तेषु मध्ये {\color{DodgerBlue3}“एकस्यापि”} धर्मस्योपन्यस्तहेतुना बाधने {\color{DodgerBlue3}“हेतुप्रतिज्ञयो”}र्व्विरुद्धता {\color{DodgerBlue3}“दोष”} उच्यते\edlabel{pvv.430-1}\footnote{\label{pvv.430-1}  १ यदि तदापरः श्लोकः ।}॥ (४३)
	\pend
      \label{div_pvv.4.44}\edlabel{div_pvv.4.44}
	  
	% new div opening: depth here is 2
	

	  \pstart यथा शब्दे शास्त्रेष्टमाकाशाश्रयत्वं बाधमानस्य\edlabel{pvv.430-2}\footnote{\label{pvv.430-2}  २ आकाशस्य नित्यत्वात् तदाश्रितञ्च नित्यं स्यात् । तदनित्यत्वेन बाध्यते ।} वादीष्टमनित्यत्वं साधयतोपि कृतकत्वस्य विरुद्धत्वं प्रतिज्ञाविरोधो वाभिधीयते । कृतकं हि क्षणिकं । न च क्षणिकमुत्पादानन्नरं क्षणमप्यस्ति । ततः  कृतकत्वमाकाशाश्रयत्वबाधनं ।
	\pend
      
	  \bigskip
	  \begingroup
	  \large
	
	    
	    \stanza[\smallbreak]
	\label{pv.4.44}\edlabel{pv.4.44}\flagstanza{\tiny\textenglish{...v.4.44}}शब्दनाशे प्रसाध्ये स्याद् गन्धभूगुणताक्षतेः ।&हेतुर्व्विरुद्धोप्रकृतेर्न्नोचेदन्यत्र सा समा ॥ ४४ ॥\&[\smallbreak]


	
	  \endgroup
	

	  \pstart तथा कृतकत्वात् {\color{DodgerBlue3}“शब्द”}स्य {\color{DodgerBlue3}“नाशे सा”}ध्य\edlabel{pvv.430-3}\footnote{\label{pvv.430-3}  ३ वैशेषिककृता यथा शब्दे आकाशगुणत्वमिष्टं एवं गन्धे पृथिवीगुणत्वमपीति कोत्र विशेषो येनैकत्र हेतुप्रतिज्ञादोषो नान्यत्रेति भावः अप्रकृतत्वाच्चेदत्रापि समानं । अनेन वेदापौरुषेयवादीनं प्रत्युक्तनित्यः शब्दः कृतकत्वात् जलादिवत् । गन्धेपि कृतकत्वादनित्यत्वे भूगुणत्वक्षतेः ।}माने {\color{DodgerBlue3}“गन्धस्य”}\edlabel{pvv.430-4}\footnote{\label{pvv.430-4}  ४ धर्मिणः} कृतकत्वान्नश्वरस्य {\color{DodgerBlue3}“भूगुणता”}या ः पृथिव्याश्रिततायाः शास्त्रेष्टायाश्च {\color{DodgerBlue3}“क्षतेः”} । विपर्यासनात् {\color{DodgerBlue3}“हेतुः”} प्रय\leavevmode\marginnote{\textenglish{431/s}} त्नानन्तरीयकत्वादि\edlabel{pvv.431-1}\footnote{\label{pvv.431-1}  १ कृतकत्वादिरादिना ।} {\color{DodgerBlue3}“विरुद्धः”} स्यात् । प्रतिज्ञा --- विरुद्धा स्यात् । उपात्तो हेतुः वादिनानिष्टं शास्त्रेष्टं बाधत इत्येव यदि विरुद्धः तदा गन्धभूगुणतां बाधमानस्य कृतकत्वस्यं शब्दे विरुद्धता स्यात् । शास्त्रेष्टबाधकताया अविशेषात् । (४४)
	\pend
      \label{div_pvv.4.45}\edlabel{div_pvv.4.45}
	  
	% new div opening: depth here is 2
	

	  \pstart अथ गन्धभूगुणताया साध्यत्वेनाप्रकृतेरप्रस्तुते तद्वाधनेपि शब्दे कृतकत्वं विरुद्धं नो चेत् सा प्रकृतिरन्यत्राकाशगुणत्वेपि समा । न हि वादिना आकाशगुणत्वं साधयितुमिष्टं किन्त्वनित्यत्वं । अतोऽप्रकृतस्यास्य बाधने न विरुद्धः स्याद्धेतुः ।
	\pend
      
	  \bigskip
	  \begingroup
	  \large
	
	    
	    \stanza[\smallbreak]
	\label{pv.4.45}\edlabel{pv.4.45}\flagstanza{\tiny\textenglish{...v.4.45}}अथात्र धर्मी प्रकृतस्तत्र शास्त्रार्थबाधनम् ।&अथ वादीष्टतां ब्रूयाद् धर्मिधर्मादिसाधनैः ॥ ४५ ॥\&[\smallbreak]


	
	  \endgroup
	

	  \pstart {\color{DodgerBlue3}“अथात्रा”}काशगुणत्वादौ {\color{DodgerBlue3}“धर्मी”} शब्दः {\color{DodgerBlue3}“प्रकृतः”} । {\color{DodgerBlue3}“तत्र शास्त्रार्थ”}स्याकाशगुणत्वादेः साधनं तद्बाधने च विरुद्धता हेतोः । भूगुणत्वे तु गन्धो धर्म्यप्रकृत इति तद्बाधनेपि न विरोधः । {\color{DodgerBlue3}“नैष परिहारः । तथाहि”} न वादीष्टविपर्यासनेन दोष उक्तः । किन्तु शास्त्रार्थविरोधेन तथा च प्रकृतत्वमनुपयुक्तं । {\color{DodgerBlue3}“अथ”} वाद्यनिष्टतयाऽप्रकृतत्वं तच्चाकाशगुणत्वयोः समानं । अथाकाशगुणत्वस्य {\color{DodgerBlue3}“वादीष्टतां”} परो {\color{DodgerBlue3}“ब्रूयात् धर्म्मिधर्म्मादिसाधनैः”} \edlabel{pvv.431-2}\footnote{\label{pvv.431-2}  २ यो धर्मिणो विशेषः साध्यसमुदायैकदेशविशेषो वा स साध्यः ।}। साध्यधर्मिधर्मत्वात् तदेकदेशत्वाद्वाऽकाशगुणत्वमिष्टं वादिनोऽनित्यत्ववदिति । (४५)
	\pend
      \label{div_pvv.4.46}\edlabel{div_pvv.4.46}
	  
	% new div opening: depth here is 2
	

	  \pstart ननु (।)
	\pend
      
	  \bigskip
	  \begingroup
	  \large
	
	    
	    \stanza[\smallbreak]
	\label{pv.4.46}\edlabel{pv.4.46}\flagstanza{\tiny\textenglish{...v.4.46}}कैश्चित् प्रकरणैरिच्छा भवेत् सा गम्यते च तैः ।&बलात् तवेच्छेयमिति व्यक्तमीश्वरचेष्टितम् ॥ ४६ ॥\&[\smallbreak]


	
	  \endgroup
	

	  \pstart {\color{DodgerBlue3}“कैश्चित् प्रकरणै”}र्व्विवादादि{\color{DodgerBlue3}“भिरिच्छा”}वादिनः कस्मिश्चिद् धर्मे {\color{DodgerBlue3}“भवेत्”} । तैरेव च प्रकरणैः {\color{DodgerBlue3}“सा”} इच्छा {\color{DodgerBlue3}“गम्यते”} परेणापि । न तु तस्य धर्मिणो धर्म इत्येव वादिनेष्यते । ततो धमिधर्मत्वादिसन्दिग्धविपक्षव्यतिरेकित्वात् शेषवत् । स्वयमनिच्छतश्च वादिनः साधनस्य {\color{DodgerBlue3}“बलात् तवेच्छेयमिति”} यदुच्यते {\color{DodgerBlue3}“व्यक्त”}मिद{\color{DodgerBlue3}“मीश्वरचेश्टित”}मित्युपहसति । (४६)
	\pend
      \label{div_pvv.4.47}\edlabel{div_pvv.4.47}
	  
	% new div opening: depth here is 2
	

	  \pstart किञ्च (।)
	\pend
      
	  \bigskip
	  \begingroup
	  \large
	
	    
	    \stanza[\smallbreak]
	\label{pv.4.47}\edlabel{pv.4.47}\flagstanza{\tiny\textenglish{...v.4.47}}वदन्नकार्यलिङ्गां तां व्यभिचारेण बाध्यते ।&अनान्तरीयके चार्थे बाधितेन्यस्य का क्षतिः ॥ ४७ ॥\&[\smallbreak]


	
	  \endgroup
	\leavevmode\marginnote{\textenglish{432/s}}

	  \pstart {\color{DodgerBlue3}“ता”}मिच्छाम{\color{DodgerBlue3}“कार्यलिङ्गजां”} कार्येतरलिङ्गामनुमेयत्वेन वदन् परो {\color{DodgerBlue3}“व्यभिचारेण बाध्यते”} । न ह्यन्योऽकार्योऽन्यं न व्यभिचरतीति नियमोस्ति । अपि च साध्यसानित्यत्वस्या{\color{DodgerBlue3}“नन्तरीयकेऽर्थे”} आकाशगुणत्वादौ {\color{DodgerBlue3}“बाधित”}त्वेप्य{\color{DodgerBlue3}“न्यस्य”} साध्यस्य {\color{DodgerBlue3}“का क्षतिः”} । न ह्यनित्यत्वमाकाशगुणत्वनान्तरीयकं येन तदभावे तदपि न स्यात् । कृतकत्वन्त्वनित्वत्यताऽव्यभिचारीति तस्मान्नानुमानमनैकान्तिकं । (४७)
	\pend
      
	  
	% new div opening: depth here is 1
	
\section[{३. शब्दाप्रामाण्यचिन्ता}]{३. शब्दाप्रामाण्यचिन्ता}\label{div_pvv.4.48}\edlabel{div_pvv.4.48}
	  
	% new div opening: depth here is 2
	

	  \begin{center}%% label @type='head'
	\textbf{(१) शास्त्रविरोधोऽकिञ्चित्करः}
	\end{center}
	
	  \bigskip
	  \begingroup
	  \large
	
	    
	    \stanza[\smallbreak]
	\label{pv.4.48}\edlabel{pv.4.48}\flagstanza{\tiny\textenglish{...v.4.48}}उक्तञ्च नागमापेक्षमनुमानं स्वगोचरे ।&सिद्धं तेन सुसिद्ध तन्न तदा शास्त्रमीक्ष्यते ॥ ४८ ॥\&[\smallbreak]


	
	  \endgroup
	

	  \pstart अनुमाविषये नेष्टं वाचः प्रामाण्य (३।३१०) मित्यादि{\color{DodgerBlue3}“नोक्तं”} प्राक् । वस्तुबलप्रवृत्त{\color{DodgerBlue3}“मनुमानं नागमापेक्षं स्व”}स्य {\color{DodgerBlue3}“गोचरे”} साध्य इति । तस्मात् {\color{DodgerBlue3}“तेन”} वस्तुबलप्रवृत्तेनागमानपेक्षिणाऽनुमाने यत् {\color{DodgerBlue3}“सिद्धं सुसिद्ध”}न्त{\color{DodgerBlue3}“त्तदा”} च {\color{DodgerBlue3}“न शास्त्रमीक्ष्यते”} बाधितं न वेति वस्तुबलप्रवृत्तानुमानेन तदपेक्षाभावात् । तदानुकाले शास्त्रस्यानाश्रयणात् । (४८)
	\pend
      \label{div_pvv.4.49}\edlabel{div_pvv.4.49}
	  
	% new div opening: depth here is 2
	
	  \bigskip
	  \begingroup
	  \large
	
	    
	    \stanza[\smallbreak]
	\label{pv.4.49}\edlabel{pv.4.49}\flagstanza{\tiny\textenglish{...v.4.49}}वादत्यागस्तदा स्याच्चेन्न तदानभ्युपायतः ।&उपायो ह्यभ्युपायेऽयमनङ्गं स तदापि सन् ॥ ४९ ॥\&[\smallbreak]


	
	  \endgroup
	\leavevmode\marginnote{\textenglish{86b/MA}}

	  \pstart {\color{DodgerBlue3}“वादत्यागः स्याच्चेत्”} । न वादत्यागः सिसाधयिषितैकधर्मादपरस्य शास्त्राभ्युपेतस्य {\color{DodgerBlue3}“तदा”} साधनोपन्यासका\edlabel{pvv.432-1}\footnote{\label{pvv.432-1}  १ वस्तुबलप्रवृत्ते ।}ले साध्यतयाऽ{\color{DodgerBlue3}“नभ्युपायतो”}ऽस्वीकारात् । ननु शास्त्राभ्युपगमाद् यदा वादः क्रियते तदा शास्त्रार्थबाधनात् वादत्यागः स्यादेवेत्याह । शास्त्रस्या{\color{DodgerBlue3}“भ्युपाये\edlabel{pvv.432-2}\footnote{\label{pvv.432-2}  २ स्वीकारे}ऽय”}म्विचार\edlabel{pvv.432-3}\footnote{\label{pvv.432-3}  ३ विचार्यग्रहात् ।} {\color{DodgerBlue3}“उपायः”} । तत{\color{DodgerBlue3}“स्तदा”} विचारकाले {\color{DodgerBlue3}“सन्नप्य”}भ्युपगमोऽ{\color{DodgerBlue3}“नङ्गं”} शास्त्रार्थग्रहणे गृहीतस्य त्यागः स्यात् । न विचारात् प्राग् ग्रहणमभ्युपगमान्न्याय्यं । (४८)
	\pend
      \label{div_pvv.4.50}\edlabel{div_pvv.4.50}
	  
	% new div opening: depth here is 2
	

	  \pstart कदा तर्हि शास्त्रेण बाधेष्यते इत्याह ।
	\pend
      
	  \bigskip
	  \begingroup
	  \large
	
	    
	    \stanza[\smallbreak]
	\label{pv.4.50}\edlabel{pv.4.50}\flagstanza{\tiny\textenglish{...v.4.50}}तदा विशुद्धे विषयद्वये शास्त्रपरिग्रहम् ।&चिकीर्षोः स हि कालः स्यात् तदा शास्त्रेण बाधनम् ॥ ५० ॥\&[\smallbreak]


	
	  \endgroup
	\leavevmode\marginnote{\textenglish{433/s}}

	  \pstart {\color{DodgerBlue3}“शास्त्रो”}पदर्शिते {\color{DodgerBlue3}“विषय\edlabel{pvv.433-1}\footnote{\label{pvv.433-1}  १ विषयास्त्रयस्तत्र प्रत्यक्षानुमानविषयो प्राग्विचारेण विशोधनीयौ । तथा विशुद्धे पश्चाच्छास्त्रग्रहं चिकीर्षा ।}द्वये”} प्रत्यक्षपरोक्षे रूप\edlabel{pvv.433-2}\footnote{\label{pvv.433-2}  २ रूपाद्यक्षस्यान्यदनुमायाः ।}नैरात्म्यादौ तदा प्रमाणप्रवृत्त्या विशुद्धे निर्ण्णीते सति पश्चादत्यन्तपरोक्षे स्वर्गादौ शास्त्रेण शास्त्राश्रयणेनानुमानं चिकीर्षोः सतः स हि कालोऽभ्युगम्य यदि शास्त्रबाधो न भवेत् । अत{\color{DodgerBlue3}“स्तदा शास्त्रेण”} {\color{DodgerBlue3}“बाधनं”} साध्यसाधनादेरिष्यते । (५०)
	\pend
      \label{div_pvv.4.51}\edlabel{div_pvv.4.51}
	  
	% new div opening: depth here is 2
	

	  \pstart तदपि करमाच्छास्त्रबाधेष्यत इत्याह । (।)
	\pend
      
	  \bigskip
	  \begingroup
	  \large
	
	    
	    \stanza[\smallbreak]
	\label{pv.4.51}\edlabel{pv.4.51}\flagstanza{\tiny\textenglish{...v.4.51}}तद्विरोधेन चिन्तायास्तत्सिद्धार्थेष्वयोगतः ।&तृतीयस्थानसंक्रान्तौ न्याय्यः शास्त्रपरिग्रहः ॥ ५१ ॥\&[\smallbreak]


	
	  \endgroup
	

	  \pstart {\color{DodgerBlue3}“त”}स्य शास्त्रस्य {\color{DodgerBlue3}“विरोधेन तत्सिद्धेष्वर्थेषु”} लिङ्गादिष्वसिद्धकल्पेषु गमक{\color{DodgerBlue3}“चिन्ताया अयोगतः”} । यस्मात् प्रत्यक्षपरोक्षार्थयोर्नागमाधिकारः तस्मात् {\color{DodgerBlue3}“तृतीयस्थाने”}ऽतीन्द्रिये विषये विचार{\color{DodgerBlue3}“संक्रान्तो शास्त्रपरिग्रहो न्याय्यः”} । प्रकारान्तराभावात् । (५१)
	\pend
      \label{div_pvv.4.52}\edlabel{div_pvv.4.52}
	  
	% new div opening: depth here is 2
	
	  \bigskip
	  \begingroup
	  \large
	
	    
	    \stanza[\smallbreak]
	\label{pv.4.52}\edlabel{pv.4.52}\flagstanza{\tiny\textenglish{...v.4.52}}तत्रापि साध्यधर्मस्य नान्तरीयकबाधनम् ।&परिहार्यं न चान्येषामनवस्थाप्रसङ्गतः ॥ ५२ ॥\&[\smallbreak]


	
	  \endgroup
	

	  \pstart {\color{DodgerBlue3}“तत्र”} शास्त्रे परि{\color{DodgerBlue3}“ग्रहेपि”} तदा साधयितुमारब्धस्य {\color{DodgerBlue3}“साध्यधर्मस्य यन्नान्तरीयकं”} सम्बद्धं यथा क्षणिकत्वस्य नैरात्म्यं तस्य {\color{DodgerBlue3}“बाधनं”} परिहार्यं । वस्तुतस्तादात्म्यादनयोर्नैरात्म्यबाधने क्षणिकत्वबाधनप्रसङ्गात् । {\color{DodgerBlue3}“न त्वन्येषां”} साध्यस्याकारणाव्यापकभूतानां धर्माणां बाधनं परिहार्यमनवस्थाप्रसङ्गतः । न हि शास्त्रदर्शितसम्भवधर्मव्याप्तिलिङ्गस्य दृष्टान्ते दर्शयितुं शक्यते येन कश्चिदागमाश्रयो हेतुरवतिष्ठेत । (५२)
	\pend
      \label{div_pvv.4.53}\edlabel{div_pvv.4.53}
	  
	% new div opening: depth here is 2
	

	  \pstart ननु शास्त्रमनपेक्ष्य न वादः कर्त्तव्य इति वस्तुबलप्रवृत्तानुमानेपि शास्त्रापेक्षेत्याह ।
	\pend
      
	  \bigskip
	  \begingroup
	  \large
	
	    
	    \stanza[\smallbreak]
	\label{pv.4.53}\edlabel{pv.4.53}\flagstanza{\tiny\textenglish{...v.4.53}}केनेयं सर्वचिन्तासु शास्त्रं ग्राह्यमिति स्थितिः ।&कृतेदानीमसिद्धान्तैर्ग्राह्यो धूमेन नानलः ॥ ५३ ॥\&[\smallbreak]


	
	  \endgroup
	

	  \pstart {\color{DodgerBlue3}“सर्व्वा”}सु परोक्षात्यन्तपरोक्षार्थ{\color{DodgerBlue3}“चिन्तासु शास्त्रं ग्राह्यमिति केनेयं स्थितिः”} कृता (।) नैतदनुमन्यन्ते विद्वान्सः । {\color{DodgerBlue3}“इदानीम”}विदुषामस्मिन्नभ्युपगमेऽ{\color{DodgerBlue3}“सिद्धान्तैः”} सिद्धान्त\leavevmode\marginnote{\textenglish{434/s}} विशेषाश्रयरहितैर्गोपालकादि{\color{DodgerBlue3}“भिर्द्धू मेन”} लिङ्गेन {\color{DodgerBlue3}“नानलो ग्राह्यः\edlabel{pvv.434-1}\footnote{\label{pvv.434-1}  १ शास्त्रापरिग्रहेण सिद्धिप्रतिबन्धात् ।}”} । (५३)
	\pend
      \label{div_pvv.4.54}\edlabel{div_pvv.4.54}
	  
	% new div opening: depth here is 2
	

	  \pstart न च कस्यचित् सिद्धान्तसम्बन्धो युक्तः । तथा हि सम्बन्धो भवन् सहजो वा भवेदौपाधिको वा (।) द्वयमपि निषेद्ध्ुमाह ।
	\pend
      
	  \bigskip
	  \begingroup
	  \large
	
	    
	    \stanza[\smallbreak]
	\label{pv.4.54}\edlabel{pv.4.54}\flagstanza{\tiny\textenglish{...v.4.54}}रिक्तस्य जन्तोर्ज्जातस्य गुणदोषमपश्यतः ।&विलब्धा वत केनामी सिद्धान्तविषमग्रहाः ॥ ५४ ॥\&[\smallbreak]


	
	  \endgroup
	

	  \pstart {\color{DodgerBlue3}“रिक्तस्य”} तुच्छस्य सिद्धान्तरहितस्य {\color{DodgerBlue3}“जन्तोर्जातस्या”}नेन सहज\edlabel{pvv.434-2}\footnote{\label{pvv.434-2}  २ सहजः कर्णादिवत् । औपाधिकः स्वयं गुणदोषपरीक्षयाभ्युपगच्छतः ।}सम्बन्धाभावनिमित्तमुक्तं । {\color{DodgerBlue3}“गुणदोषं”} प्रामाण्याप्रामाण्यनिबन्धन{\color{DodgerBlue3}“मपश्यतः\edlabel{pvv.434-3}\footnote{\label{pvv.434-3}  ३ आगमभात्रेण ।}”} । अनेनौपाधिकसम्बन्धनिमित्ताभाव उक्तः । \edlabel{pvv.434-4}\footnote{\label{pvv.434-4}  ४ उपहसति ।} {\color{DodgerBlue3}“केनान”}र्थपटीयसा {\color{DodgerBlue3}“सिद्धान्ता”}\edlabel{pvv.434-5}\footnote{\label{pvv.434-5}  ५ त्याजयितुमशक्यत्वात् ।}एव {\color{DodgerBlue3}“विषमग्रहा”} दुष्परिहारत्वा{\color{DodgerBlue3}“दिमे विलब्धा”} \edlabel{pvv.434-6}\footnote{\label{pvv.434-6}  ६ अस्यायमागमो नास्येति ।} {\color{DodgerBlue3}“वत”} येषु स्वामित्वेन जन्तवो व्यवहरन्ति । न हि कर्ण्णनाश(? स)मिव सिद्धान्तः प्राणसहजः । नापि दोषोर्ज्जितगुणोपपन्नः प्रमाणमिव विचारात् प्राक् सिद्धान्तः सिद्धो येन समर्थविषयः स्यात् । (५४)
	\pend
      \label{div_pvv.4.55}\edlabel{div_pvv.4.55}
	  
	% new div opening: depth here is 2
	

	  \pstart किञ्च (।)
	\pend
      
	  \bigskip
	  \begingroup
	  \large
	
	    
	    \stanza[\smallbreak]
	\label{pv.4.55}\edlabel{pv.4.55}\flagstanza{\tiny\textenglish{...v.4.55}}यदि साधन एकत्र सर्वं शास्त्रं निदर्शने ।&दर्शयेत् साधनं स्यादित्येषा लोकोत्तरा स्थितिः ॥ ५५ ॥\&[\smallbreak]


	
	  \endgroup
	

	  \pstart {\color{DodgerBlue3}“यदि साधन एकत्र सर्व्वं शास्त्रं”} शास्त्रार्थं {\color{DodgerBlue3}“निदर्शने”} दृष्टान्ते वादी {\color{DodgerBlue3}“दर्शयेत्”} तदा तत् {\color{DodgerBlue3}“साधनं स्यात्”} (।) न त्वेकस्य शास्त्रदर्शितधर्मस्यान्वये । एतच्च न क्वचित् \leavevmode\marginnote{\textenglish{87a/MA}} साधने कर्त्तुं शक्य{\color{DodgerBlue3}“मिति लोका”}तिक्रान्ता {\color{DodgerBlue3}“स्थितिरेषा”} । (५५)
	\pend
      \label{div_pvv.4.56}\edlabel{div_pvv.4.56}
	  
	% new div opening: depth here is 2
	

	  \pstart अपि च (।)
	\pend
      
	  \bigskip
	  \begingroup
	  \large
	
	    
	    \stanza[\smallbreak]
	\label{pv.4.56}\edlabel{pv.4.56}\flagstanza{\tiny\textenglish{...v.4.56}}असम्बद्धस्य धर्मस्य किमसिद्धौ न सिद्धति ।&हेतुस्तत्साधनायोक्तः किं दुष्टस्तत्र सिध्यति ॥ ५६ ॥\&[\smallbreak]


	
	  \endgroup
	

	  \pstart {\color{DodgerBlue3}“असम्बद्धस्य”} साधयितुमप्रवृत्तस्य {\color{DodgerBlue3}“धर्म”}स्याकाशगुणत्वादेः कृतकत्वाद्धेतोर्व्विपर्यासनाद{\color{DodgerBlue3}“सिद्धौ”} सत्यां सिसाधयिषितं हेतुव्यापकमनित्यत्वं {\color{DodgerBlue3}“किं न सिध्यति”} । न हि व्यापकमन्तरेण व्याप्यं भवति । {\color{DodgerBlue3}“हेतुस्त”}स्य व्यापकस्य {\color{DodgerBlue3}“साधनायोक्तः किन्दुष्टः”} \leavevmode\marginnote{\textenglish{435/s}} {\color{DodgerBlue3}“तत्र”} स्वसाध्ये {\color{DodgerBlue3}“सिध्यति”} साध्यप्रतिपादनं साध्यव्यापारः{\color{DodgerBlue3}“तच्चेदस्ति कथं दुष्टः”} ।(५६)
	\pend
      \label{div_pvv.4.57}\edlabel{div_pvv.4.57}
	  
	% new div opening: depth here is 2
	

	  \pstart शास्त्रार्थबाधनेऽभिमतस्यापि न सिद्धिरिति चेत् । {\color{DodgerBlue3}“आह”} ॥
	\pend
      
	  \bigskip
	  \begingroup
	  \large
	
	    
	    \stanza[\smallbreak]
	\label{pv.4.57}\edlabel{pv.4.57}\flagstanza{\tiny\textenglish{...v.4.57}}धर्माननुपनीयैव दृष्टान्ते धर्मिणोऽखिलान् ।&वाग्धूमादेर्जनोन्वेति चैतन्यदहनादिकम् ॥ ५७ ॥\&[\smallbreak]


	
	  \endgroup
	

	  \pstart धर्मिणो{\color{DodgerBlue3}“धर्मान्”} शास्त्रदर्शिता{\color{DodgerBlue3}“नखिलान्”} हेतुव्यापकत्वे{\color{DodgerBlue3}“नानुपनीया”}प्रदर्श्यं {\color{DodgerBlue3}“वाग्धूमा”}देर्हेतो{\color{DodgerBlue3}“श्चैतन्यदहनादिकं”} यथाक्रमं स्वसन्तान\edlabel{pvv.435-1}\footnote{\label{pvv.435-1}  १ चेतनोयं वचनादहमिव ।}वन्महानसवच्च {\color{DodgerBlue3}“जनोऽन्वेति”} प्रतिपद्यते ॥ (५७)
	\pend
      \label{div_pvv.4.58}\edlabel{div_pvv.4.58}
	  
	% new div opening: depth here is 2
	

	  \pstart किञ्च (।)
	\pend
      
	  \bigskip
	  \begingroup
	  \large
	
	    
	    \stanza[\smallbreak]
	\label{pv.4.58}\edlabel{pv.4.58}\flagstanza{\tiny\textenglish{...v.4.58}}स्वभावं कारणं चार्थोऽव्यभिचारेण साधयन् ।&कस्यचिद् वादबाधायां स्वभावान्न निवर्तते ॥ ५८ ॥\&[\smallbreak]


	
	  \endgroup
	

	  \pstart {\color{DodgerBlue3}“स्वभावं”} व्यापकं वृक्षत्वादि {\color{DodgerBlue3}“कारणं”} वह्न्यादि {\color{DodgerBlue3}“चार्थो”} व्याप्यः शिंशपादिः {\color{DodgerBlue3}“कार्यं”} धूमादिर{\color{DodgerBlue3}“व्यभिचारेणा”}विनाभावित्वात् {\color{DodgerBlue3}“साधयन् कस्यचित्”} शास्त्रपराधीनस्य {\color{DodgerBlue3}“वा”}दिनो वादस्य {\color{DodgerBlue3}“बाधायां स्वभावात्”} व्यापककारणगमकान्न {\color{DodgerBlue3}“निवर्त्तते”} ततः {\color{DodgerBlue3}“शास्त्रेषु”} धर्मान्तरव्याहतावपि हेतुः साध्यीकृतं स्वसम्बद्धमर्थं प्रतिपादयति । (५८)
	\pend
      \label{div_pvv.4.59}\edlabel{div_pvv.4.59}
	  
	% new div opening: depth here is 2
	

	  \pstart ततश्च ।
	\pend
      
	  \bigskip
	  \begingroup
	  \large
	
	    
	    \stanza[\smallbreak]
	\label{pv.4.59}\edlabel{pv.4.59}\flagstanza{\tiny\textenglish{...v.4.59}}प्रपद्यमानश्चान्यस्तं नान्तरीयकमीप्सितैः ।&साध्यार्थैहेतुना तेन कथमप्रतिपादितः ॥ ५९ ॥\&[\smallbreak]


	
	  \endgroup
	

	  \pstart {\color{DodgerBlue3}“साध्यैरर्थैरीप्सितै”}राप्तं प्रत्येतुमष्टै{\color{DodgerBlue3}“र्न्नान्तिरीय”}कमविनाभाविनं {\color{DodgerBlue3}“तं”} हेतुं {\color{DodgerBlue3}“प्रपद्यमानः”} प्रतिपद्यमानो{\color{DodgerBlue3}“न्यः”} प्रतिवादी {\color{DodgerBlue3}“कथन्तेन हेतुनाऽप्रतिपादितः”} साध्यनान्तरीयकतया क्वचिद्धर्मिणि साधनप्रतीतिरेव हि साध्यप्रतीतिः सा चास्ति प्रतिवादिनः ॥ (५९)
	\pend
      \label{div_pvv.4.60}\edlabel{div_pvv.4.60}
	  
	% new div opening: depth here is 2
	

	  \pstart किञ्च (।)
	\pend
      

	  \pstart हेतुना\edlabel{pvv.435-2}\footnote{\label{pvv.435-2}  २ प्रकृतसाधकेन यथा कृतकत्वेनाकाशगुणत्वादि ।} यः शास्त्रार्थो बाध्यते किन्तस्मिन् साध्ये वादीना हेतुरसाधूक्तः । आहोश्वि(? स्वि)त् तत्र साध्ये हेतुरुच्यतां मा वा वस्तुतस्तद्वाधकोसौ हेतुरिति दुष्टता । तत्राद्यपक्षे भवत्येव दोषो यद्येवमिष्यते (।)
	\pend
      

	  \pstart अथ (।)
	\pend
      
	  \bigskip
	  \begingroup
	  \large
	
	    
	    \stanza[\smallbreak]
	\label{pv.4.60}\edlabel{pv.4.60}\flagstanza{\tiny\textenglish{...v.4.60}}उक्तोऽनुक्तोपि वा हेतुर्विरोद्धा वादिनोत्र किम् ।&न हि तस्योक्तिदोषेण स जातः शास्त्रबाधनः ॥ ६० ॥\&[\smallbreak]


	
	  \endgroup
	\leavevmode\marginnote{\textenglish{436/s}}

	  \pstart {\color{DodgerBlue3}“उक्तोऽनुक्तोपि”} वा {\color{DodgerBlue3}“हेतुः”} वस्तुत एव तस्य विरोद्धा प्रतिघातकस्तदा तत्र शास्त्रार्थबाधने {\color{DodgerBlue3}“वादिनः किन्दू”}षणं न किञ्चित् । {\color{DodgerBlue3}“हि”} यस्मात् {\color{DodgerBlue3}“तस्य”} वादिन {\color{DodgerBlue3}“उक्तिदोषेण”} स कृतकत्वादिहेतुः {\color{DodgerBlue3}“शास्त्र”}स्य शास्त्रार्थस्याकाशगुणत्वादेः {\color{DodgerBlue3}“बाधनो बाधको न जातः”} ॥ (६०)
	\pend
      \label{div_pvv.4.61}\edlabel{div_pvv.4.61}
	  
	% new div opening: depth here is 2
	
	  \bigskip
	  \begingroup
	  \large
	
	    
	    \stanza[\smallbreak]
	\label{pv.4.61}\edlabel{pv.4.61}\flagstanza{\tiny\textenglish{...v.4.61}}बाधकस्याभिधानाच्चेद् दोषो यदि वदेन्न सः ।&किन्न बाधेत सोऽकुर्वन्नयुक्तं केन दुष्यति ॥ ६१ ॥\&[\smallbreak]


	
	  \endgroup
	

	  \pstart शास्त्रार्थ{\color{DodgerBlue3}“बाधकस्य”} हेतोर{\color{DodgerBlue3}“भिधानात्”} वादिनोपि {\color{DodgerBlue3}“दोषश्चेत् यदि”} तं हेतुं {\color{DodgerBlue3}“न वदेत् स”} वादी । तदा किमसौ हेतुः शास्त्रार्थ न {\color{DodgerBlue3}“बाधेत”} । वस्तुतस्तद्विरोधित्वादवश्यं बाधते । ततश्च स वाद्य{\color{DodgerBlue3}“कुर्वन्नयुक्तं केन”} कारणेन {\color{DodgerBlue3}“दुष्यति”} पराजितः स्यात् ॥ (६९)
	\pend
      \label{div_pvv.4.62}\edlabel{div_pvv.4.62}
	  
	% new div opening: depth here is 2
	

	  \pstart ननु यदि दुष्टहेतुवचनेपि न वादिनो दुष्टता । तदाऽसिद्धादिवचनेपि न दोषः स्यादित्याह ।
	\pend
      
	  \bigskip
	  \begingroup
	  \large
	
	    
	    \stanza[\smallbreak]
	\label{pv.4.62}\edlabel{pv.4.62}\flagstanza{\tiny\textenglish{...v.4.62}}अन्येषु हेत्वाभासेषु स्वेष्टस्येवाप्रसाधनात् ।&दुष्येद् व्यर्थाभिधानेन नात्र तस्य प्रसाधनात् ॥ ६२ ॥\&[\smallbreak]


	
	  \endgroup
	

	  \pstart {\color{DodgerBlue3}“अन्येष्व”}सिद्धादिषु {\color{DodgerBlue3}“हेत्वाभासेषु”} वाद्युक्तेषु {\color{DodgerBlue3}“स्वेष्टस्य”} वादीष्टस्यै{\color{DodgerBlue3}“वाप्रसाधनाद्”} वादी {\color{DodgerBlue3}“दुष्यति । व्यर्थ”}स्य साध्यसाधनानुपयुक्तस्य साधनस्या{\color{DodgerBlue3}“भिधानात्”}\edlabel{pvv.436-1}\footnote{\label{pvv.436-1}  १ प्रयोगवैफल्येन ।} । {\color{DodgerBlue3}“अत्र”} कृतकत्वे तु वाद्युक्ते वाञ्छितस्यानित्यत्वस्य {\color{DodgerBlue3}“प्रसाधनान्न”} वादी दुष्यति । शास्त्रार्थे तु वाद्यनिष्टे बाध्यमाने शास्त्रमेव दुष्टं भविष्यति ॥ (६२)
	\pend
      \label{div_pvv.4.63}\edlabel{div_pvv.4.63}
	  
	% new div opening: depth here is 2
	

	  \pstart यद्यपि स्वेष्टस्य तेन साधनं तथापि शास्त्रार्थस्य बाधनमिति दुष्ट एवेत्याह ।
	\pend
      
	  \bigskip
	  \begingroup
	  \large
	
	    
	    \stanza[\smallbreak]
	\label{pv.4.63}\edlabel{pv.4.63}\flagstanza{\tiny\textenglish{...v.4.63}}यदि किञ्चित् क्वचित् शास्त्रे न युक्तं प्रतिषिध्यते ।&ब्रुवाणो युक्तमप्यन्यदिति राजकुलस्थितिः ॥ ६३ ॥\&[\smallbreak]


	
	  \endgroup
	

	  \pstart {\color{DodgerBlue3}“क्वचिद्”} वै शे षि का दि{\color{DodgerBlue3}“शास्त्रे”} निर्दिष्टं {\color{DodgerBlue3}“यदि किञ्चि”}दाकाशगुणत्वादि बाध्यमानत्वा{\color{DodgerBlue3}“दयुक्तं”} । तावताऽ{\color{DodgerBlue3}“न्यद”}नित्यत्वादि{\color{DodgerBlue3}“युक्तमपि”} कृतकत्वहेतुना {\color{DodgerBlue3}“ब्रुवाणः”} प्रतिपादयन् {\color{DodgerBlue3}“प्रतिषिध्यते”} शास्त्रार्थबाधनात् विरोधो\edlabel{pvv.436-2}\footnote{\label{pvv.436-2}  २ अयुक्तं त्वयोक्तमिति ।}पन्यासेनेति व्यक्तमियं {\color{DodgerBlue3}“राजकुल”}\leavevmode\marginnote{\textenglish{87b/MA}} {\color{DodgerBlue3}“स्थितिः”} । राजशासनस्यैव बलप्रवृत्तस्य युक्तायुक्तविचारणाबहिर्भावात् । (६३)
	\pend
      \label{div_pvv.4.64}\edlabel{div_pvv.4.64}
	  
	% new div opening: depth here is 2
	

	  \pstart किञ्च (।)
	\pend
      
	  \bigskip
	  \begingroup
	  \large
	
	    
	    \stanza[\smallbreak]
	\label{pv.4.64}\edlabel{pv.4.64}\flagstanza{\tiny\textenglish{...v.4.64}}सर्वानर्थान् समीकृत्य वक्तुं शक्यं न साधनम् ।&सर्वत्र तेन सुच्छन्नेयं साध्यसाधनसंस्थितिः ॥ ६४ ॥\&[\smallbreak]


	
	  \endgroup
	\leavevmode\marginnote{\textenglish{437/s}}

	  \pstart {\color{DodgerBlue3}“सर्व्वान्”} शास्त्रदृष्टा{\color{DodgerBlue3}“नर्थान्”} साध्यत्वेन {\color{DodgerBlue3}“समीकृत्य”} किञ्चित् {\color{DodgerBlue3}“साधनं वक्तुमशक्यं”} दृष्टान्ते शास्त्रदृष्टाखिलधर्मव्याप्त्यनुपलम्भात् । {\color{DodgerBlue3}“तेन”} कारणेन {\color{DodgerBlue3}“सर्व्वत्र”} धर्मिणि {\color{DodgerBlue3}“साध्यसाधनयोः संस्थिति”}र्व्यव{\color{DodgerBlue3}“स्थेयं सुच्छन्ना”} स्यात् ॥ (६४)
	\pend
      \label{div_pvv.4.65}\edlabel{div_pvv.4.65}
	  
	% new div opening: depth here is 2
	

	  \begin{center}%% label @type='head'
	\textbf{(शब्दस्य नाकाशगुणत्वम्)}
	\end{center}
	

	  \pstart यदि तर्ह्याकाशगुणत्वाभावेप्यनित्यत्वं सिध्यदबाध्यं शब्दे तदा श्रावणत्वादिहेतुना नित्यत्वमपि साध्यमानमबाध्यं स्यादित्याह ।
	\pend
      
	  \bigskip
	  \begingroup
	  \large
	
	    
	    \stanza[\smallbreak]
	\label{pv.4.65}\edlabel{pv.4.65}\flagstanza{\tiny\textenglish{...v.4.65}}विरुद्धयोरेकधर्मिण्ययोगादस्तु बाधनम् ।&विरुद्धैकान्तिके नात्र तद्वदस्ति विरोधिता ॥ ६५ ॥\&[\smallbreak]


	
	  \endgroup
	

	  \pstart {\color{DodgerBlue3}“विरुद्धयो”}र्न्नित्यत्वानित्यत्वयो{\color{DodgerBlue3}“रेकत्र”} शब्दे {\color{DodgerBlue3}“धर्मिण्ययोगाद् विरुद्धैकान्तिके”} विरुद्धाव्यभिचारिणि श्रावणत्वादौ नित्यत्वसाधके {\color{DodgerBlue3}“बाधनमस्तु”} । नह्येकत्र धर्मिणि विरुद्धौ धर्मो भवितुमर्हतः । {\color{DodgerBlue3}“तद्वन्नि”}त्यत्वयोरिवात्रानयोः प्रकृताप्रकृतयोरनित्यत्वाकाशगुणत्वाभावयो{\color{DodgerBlue3}“र्व्विरोधिता”} नास्ति । ततः कृतकत्वाच्छब्देऽनित्यत्वसिद्धावाकाशगुणत्वाभावो न बाध्यते ॥ (६५)
	\pend
      \label{div_pvv.4.66}\edlabel{div_pvv.4.66}
	  
	% new div opening: depth here is 2
	

	  \pstart स्यादेतत् । प्रकृताप्रकृतयोरनित्यत्वाकाशगुणत्वाभावयोः परस्परं (।)
	\pend
      
	  \bigskip
	  \begingroup
	  \large
	
	    
	    \stanza[\smallbreak]
	\label{pv.4.66}\edlabel{pv.4.66}\flagstanza{\tiny\textenglish{...v.4.66}}अबाध्यबाधकत्वेपि तयोः शास्त्रार्थविल्पवात् ।&असम्बन्धेपि बाधा चेत् स्यात् सर्वं सर्वबाधनम् ॥ ६६ ॥\&[\smallbreak]


	
	  \endgroup
	

	  \pstart {\color{DodgerBlue3}“बाध्यबाधक”}त्वाभा{\color{DodgerBlue3}“वेप्ये”}कस्मिन्ननित्यत्वे कृतकत्वात् सिध्यति {\color{DodgerBlue3}“शब्दे धर्मिणि शास्त्रार्थस्य शास्त्रा”}भ्युपगतस्याकाशगुणत्वस्य {\color{DodgerBlue3}“विप्लवात्”} कारणाद् असम्बद्धे (।) अप्रकृताकाशगुणत्व{\color{DodgerBlue3}“सम्बन्धरहि”}तेऽनित्यत्वे{\color{DodgerBlue3}“पि बाधा”} भवतीति {\color{DodgerBlue3}“चेत्”} । एवन्तर्हि प्रयन्तानन्तरीयकत्वाद् गन्धे पृथिवीगुणत्वबाधने {\color{DodgerBlue3}“सर्व्वं”} कृतकत्वादि {\color{DodgerBlue3}“सर्व्व”}स्यानित्यत्वादेः साध्यस्य {\color{DodgerBlue3}“बाधनं”}  {\color{DodgerBlue3}“स्यात्”} । शब्दादौ धर्मिण्यप्रकृतशास्त्रार्थबाधनस्य तुल्यत्वात् ॥ (६६)
	\pend
      \label{div_pvv.4.67}\edlabel{div_pvv.4.67}
	  
	% new div opening: depth here is 2
	
	  \bigskip
	  \begingroup
	  \large
	
	    
	    \stanza[\smallbreak]
	\label{pv.4.67}\edlabel{pv.4.67}\flagstanza{\tiny\textenglish{...v.4.67}}सम्बन्धस्तेन तस्यैव बाधनादस्ति चेदसत् ।&हेतोः सर्व्वस्य चिन्त्यत्वात् स्वसाध्ये गुणदोषयोः ॥ ६७ ॥\&[\smallbreak]


	
	  \endgroup
	

	  \pstart अथ तत्र शब्द एव धर्मिणि आकाशगुणत्वस्य सत्त्वात् {\color{DodgerBlue3}“सम्बन्धोस्ति तेन”} कृतकत्वात् {\color{DodgerBlue3}“तस्यैव बाधनाद्”} विरोधः । पृथिवीगुणत्वन्तु शब्दे धर्मिण्यसम्बद्धं । ततस्तद्बाधनेपि शब्दे कृतकत्वमविरुद्धमिति {\color{DodgerBlue3}“चेत् । असदे”}तत् । {\color{DodgerBlue3}“सर्व्वस्य हेतोः स्वसाध्ये”} प्रकृते {\color{DodgerBlue3}“गुणदोषयोश्चिन्त्यत्वात्”} । यत्पुनरप्रकृतं धर्मिसम्बद्धमपि  न तत् साध्यं । तद्वाधनेपि न काचित् । क्षतिः ॥ (६७)
	\pend
      \label{div_pvv.4.68}\edlabel{div_pvv.4.68}
	  
	% new div opening: depth here is 2
	

	  \pstart \leavevmode\marginnote{\textenglish{438/s}}किञ्च (।) धर्मिणि सत्तामात्रं न सम्बन्धः । किन्तु (।)
	\pend
      
	  \bigskip
	  \begingroup
	  \large
	
	    
	    \stanza[\smallbreak]
	\label{pv.4.68}\edlabel{pv.4.68}\flagstanza{\tiny\textenglish{...v.4.68}}नान्तरीयकता साध्ये सम्बन्धः सेह नेक्ष्यते ।&केवलं शास्त्रपीडेति दोषः सान्यकृते समा ॥ ६८ ॥\&[\smallbreak]


	
	  \endgroup
	

	  \pstart {\color{DodgerBlue3}“साध्ये नान्तरीयकता\edlabel{pvv.438-1}\footnote{\label{pvv.438-1}  १ साध्यस्य तदन्यनान्तरीयकता यथाऽनित्यत्वस्य दुःखादिनान्तरीयकता ।}”} साध्याविनाभावित्वं {\color{DodgerBlue3}“सम्बन्ध उच्यते (।) सा”} साध्यनान्तरीयता {\color{DodgerBlue3}“इह”} प्रकृताकाशगुणत्वबाधने सति नेक्ष्यते \edlabel{pvv.438-2}\footnote{\label{pvv.438-2}  २ केवलं संयोगादिविपर्यासनाच्छास्त्रवाचा ।} (।) यद्यपि सत्वनान्तरीयकमाकाशगुणत्वं शब्दे स्यात् न बाध्येत । {\color{DodgerBlue3}“केवलं”} शास्त्राभ्युपगतधर्मबाधना{\color{DodgerBlue3}“च्छास्त्रपीडे”}ति दोषः । {\color{DodgerBlue3}“सा”} च कृतकत्वादनित्यत्वसिद्धौ दृश्यते शास्त्रपीडा{\color{DodgerBlue3}“ऽन्ये”}न प्रयत्नानन्तरीयकत्वादिना गन्धे पृथिवीगुणत्वबाधनेपि {\color{DodgerBlue3}“समेति”} कृतकत्वं शब्दे विरुद्धं स्यात् ॥ (६८)
	\pend
      \label{div_pvv.4.69}\edlabel{div_pvv.4.69}
	  
	% new div opening: depth here is 2
	

	  \pstart यदप्याहुरा{\color{DodgerBlue3}“चार्यीयाः”} शास्त्रमभ्युपगम्य यदा वादः क्रियते तदा शास्त्रदृष्टस्य सकलस्य धर्मस्य साध्यतेत्यत्राह ।
	\pend
      
	  \bigskip
	  \begingroup
	  \large
	
	    
	    \stanza[\smallbreak]
	\label{pv.4.69}\edlabel{pv.4.69}\flagstanza{\tiny\textenglish{...v.4.69}}शास्त्राभ्युपगमात् साध्यः शास्त्रदृष्टोऽखिलो यदि ।&प्रतिज्ञाऽसिद्धदृष्टान्तहेतुवादः प्रसज्यते ॥ ६९ ॥\&[\smallbreak]


	
	  \endgroup
	

	  \pstart {\color{DodgerBlue3}“शास्त्राभ्युपगमाच्छास्त्रदृष्टोऽखिलो”} धर्मो {\color{DodgerBlue3}“यदि साध्य”} इष्यते तदाऽ{\color{DodgerBlue3}“सिद्धयोः”} शास्त्रोद्दिष्टयो{\color{DodgerBlue3}“र्हेतुदृष्टान्तयोर्व्वादः”}  {\color{DodgerBlue3}“प्रतिज्ञा”} साध्यं {\color{DodgerBlue3}“प्रसज्यते”} । शास्त्रे दृष्टस्यासिद्धस्य साध्यत्वात् ॥ (६९)
	\pend
      \label{div_pvv.4.70}\edlabel{div_pvv.4.70}
	  
	% new div opening: depth here is 2
	

	  \pstart स्यादेतत् । किन्तु (।)
	\pend
      
	  \bigskip
	  \begingroup
	  \large
	
	    
	    \stanza[\smallbreak]
	\label{pv.4.70}\edlabel{pv.4.70}\flagstanza{\tiny\textenglish{...v.4.70}}उक्तयोः साधनत्वेन नो चेदीप्सितवादतः ।&न्यायप्राप्तं न साध्यत्वं वचनाद् विनिवर्त्तते ॥ ७० ॥\&[\smallbreak]


	
	  \endgroup
	

	  \pstart {\color{DodgerBlue3}“वादिनेप्सित”}स्य साध्यत्वेन वादतः स्वयं साध्यत्वेनेप्सितः पक्षो विरुद्धत्वान्निराकृत इत्यादिकात् \edlabel{pvv.438-3}\footnote{\label{pvv.438-3}  ३ साध्यत्वेनेप्सित इति कृतं अन्यत्र स्वरूपेणैवेत्यवधारणमतः ।} {\color{DodgerBlue3}“साधनत्वेनोक्तयो”}रसिद्धहेतुदृष्टान्तयोः साध्यता {\color{DodgerBlue3}“नो चेत्”} । नन्वसिद्धस्य शास्त्राभ्युपगतस्य {\color{DodgerBlue3}“साध्यत्वं न्यायप्राप्तं वचन”}मात्रादीप्सितसाध्यत्वं प्रतिपादक{\color{DodgerBlue3}“त्वान्न विनिवर्त्तते”} \edlabel{pvv.438-4}\footnote{\label{pvv.438-4}  ४ और्ण्यमिवाग्नेः ।}॥ (७०)
	\pend
      \label{div_pvv.4.71}\edlabel{div_pvv.4.71}
	  
	% new div opening: depth here is 2
	\leavevmode\marginnote{\textenglish{439/s}}
	  \bigskip
	  \begingroup
	  \large
	
	    
	    \stanza[\smallbreak]
	\label{pv.4.71}\edlabel{pv.4.71}\flagstanza{\tiny\textenglish{...v.4.71}}अनीप्सितमसाध्यञ्चेद् वादिनान्योप्यनीप्सितः ।&धर्मोऽसाध्यस्तदाऽसाध्यं बाधमानं विरोधि किम् ॥ ७१ ॥\&[\smallbreak]


	
	  \endgroup
	

	  \pstart शास्त्राभ्युपगमेपि वादिनाऽ{\color{DodgerBlue3}“नीप्सितमसाध्यं चेत्”} । एवन्तर्ह्याकाशगुण-\leavevmode\marginnote{\textenglish{88a/MA}} त्वादिरपि {\color{DodgerBlue3}“धर्मो वादिना”} साध्यत्वेना{\color{DodgerBlue3}“नीप्सितोऽसाध्यः”} स्यात् । {\color{DodgerBlue3}“तदा तदसाध्य”}माकाशगुणत्वं {\color{DodgerBlue3}“बाधमानं”} कृतकत्वं {\color{DodgerBlue3}“किं”} कस्माद् {\color{DodgerBlue3}“विरोधि”} । (७१)
	\pend
      \label{div_pvv.4.72}\edlabel{div_pvv.4.72}
	  
	% new div opening: depth here is 2
	

	  \begin{center}%% label @type='head'
	\textbf{(२) अन्यथा स्वयंशब्दोऽनर्थकः}
	\end{center}
	

	  \pstart किञ्च (।)
	\pend
      
	  \bigskip
	  \begingroup
	  \large
	
	    
	    \stanza[\smallbreak]
	\label{pv.4.72}\edlabel{pv.4.72}\flagstanza{\tiny\textenglish{...v.4.72}}पक्षलक्षणबाह्यार्थः स्वयंशब्दोप्यनर्थकः ।&शास्त्रेष्विच्छाप्रवृत्त्यर्थो यदि शङ्का कुतोन्वियम् ॥ ७२ ॥\&[\smallbreak]


	
	  \endgroup
	

	  \pstart यदि शास्त्राभ्युपगतत्वं पक्ष\edlabel{pvv.439-1}\footnote{\label{pvv.439-1}  १ शास्त्रार्थः सर्व्वः साध्यः ।}लक्षणं तदा {\color{DodgerBlue3}“स्वयंशब्दोपि पक्षलक्षणबाह्यार्थो”} भिन्नाभिधेयोऽ{\color{DodgerBlue3}“नर्थकः”} स्यात् । शास्त्राभ्युपगमे शास्त्रेष्टस्य साध्यताप्राप्तौ वादीष्टमात्रं साध्यं नान्यदिति हि स्वयंशब्दस्य प्रयोजनं । शास्त्रेष्टमात्रस्य तु {\color{DodgerBlue3}“साध्यत्वे”} निष्फलमेव तत् । {\color{DodgerBlue3}“शास्त्रेष्विच्छया प्रवृत्त्यर्थः”} \edlabel{pvv.439-2}\footnote{\label{pvv.439-2}  २ स्वीकृतशास्त्रं मुक्त्वा वादकाले शास्त्रान्तरमिच्छया लभ्यतेङ्गीकर्त्तुं ।} स्वयंशब्दो {\color{DodgerBlue3}“यदि”} कथ्यते स्वयंशब्दमन्तरेण शास्त्रमिच्छया न ग्रहीतव्यमिति {\color{DodgerBlue3}“शङ्केयं कुतो”} नु हेतोर्जाता । {\color{DodgerBlue3}“येन”} तन्निवृत्त्यर्था स्वयंश्रुतिर्व्वर्ण्यते ॥ (७२)
	\pend
      \label{div_pvv.4.73}\edlabel{div_pvv.4.73}
	  
	% new div opening: depth here is 2
	
	  \bigskip
	  \begingroup
	  \large
	
	    
	    \stanza[\smallbreak]
	\label{pv.4.73}\edlabel{pv.4.73}\flagstanza{\tiny\textenglish{...v.4.73}}सोनिषिद्धः प्रमाणेन गृह्णन् केन निवार्यते ।&निषिद्धश्चेत् प्रमाणेन वाचा केन प्रवृर्त्त्यते ॥ ७३ ॥\&[\smallbreak]


	
	  \endgroup
	

	  \pstart {\color{DodgerBlue3}“स”} वादी {\color{DodgerBlue3}“प्रमाणेन”} शास्त्रार्थबाधकेना{\color{DodgerBlue3}“निषिद्धः”} शास्त्रं {\color{DodgerBlue3}“गृह्णन् केन निवार्यते”} न केनचित् । यतः स्वयंग्रहणं शास्त्रं ग्राहयत् सफलं स्यात् । {\color{DodgerBlue3}“प्रमाणेन”} चेच्छास्त्रार्थबाधकेन {\color{DodgerBlue3}“निषिद्धो”} वादी {\color{DodgerBlue3}“वाचा”} स्वयंशब्देन शास्त्राभ्युपगमे {\color{DodgerBlue3}“केन”} लक्षणकर्त्रा {\color{DodgerBlue3}“प्रवर्त्त्यते”} न केनचित् ॥ (७३)
	\pend
      \label{div_pvv.4.74}\edlabel{div_pvv.4.74}
	  
	% new div opening: depth here is 2
	

	  \pstart किञ्च (।)
	\pend
      
	  \bigskip
	  \begingroup
	  \large
	
	    
	    \stanza[\smallbreak]
	\label{pv.4.74}\edlabel{pv.4.74}\flagstanza{\tiny\textenglish{...v.4.74}}पूर्वमप्येष सिद्धान्तं स्वेच्छयैव गृहीतवान् ।&किञ्चिदन्यं स (तु) पुनर्ग्रहीतुं लभते न किंम् ॥ ७४ ॥\&[\smallbreak]


	
	  \endgroup
	

	  \pstart {\color{DodgerBlue3}“एष”} वादी {\color{DodgerBlue3}“पूर्व्वमपि स्वेच्छेयैव सिद्धान्तं”} क णा दा दिप्रणीतं {\color{DodgerBlue3}“गृहीतवान्”} । न त्वन्यपुराणादिबलात् । स कथमिच्छया शास्त्रोद्दिष्ट(ङ्) {\color{DodgerBlue3}“किञ्चिद्”} धर्मव्यभि\leavevmode\marginnote{\textenglish{440/s}} चारदर्शनादिना शास्त्रेषु सफलधर्मकलापसाध्यत्वादन्यं सिद्धान्तमाकाशगुणत्वरहितानित्यतादिकं {\color{DodgerBlue3}“ग्रहीतुं किन्न लभते”} (।) इच्छाधीनत्वे नियमायोगात् । तस्मात् स्वयंग्रहणं शास्त्रेच्छाप्रवृत्त्यर्थमित्ययुक्तं ॥ (७४)
	\pend
      \label{div_pvv.4.75}\edlabel{div_pvv.4.75}
	  
	% new div opening: depth here is 2
	

	  \pstart नन्विष्टस्यापि स्वेच्छयैव साध्यतापरिग्रहः सिद्ध इति व्यर्थं स्वयंग्रहणमित्याह ।
	\pend
      
	  \bigskip
	  \begingroup
	  \large
	
	    
	    \stanza[\smallbreak]
	\label{pv.4.75}\edlabel{pv.4.75}\flagstanza{\tiny\textenglish{...v.4.75}}दृष्टेर्विप्रतिपत्तीनामत्राकार्षीत् स्वयंश्रुतिम् ।&इष्टाक्षतिमसाध्यत्वमनवस्थाञ्च दर्शयन् ॥ ७५ ॥\&[\smallbreak]


	
	  \endgroup
	

	  \pstart शास्त्रकारस्येष्टं साध्यमिति {\color{DodgerBlue3}“विप्रतिपत्तीनां दृष्टे”}स्तन्निराकणार्थ{\color{DodgerBlue3}“मत्र”} पक्षलक्षणे {\color{DodgerBlue3}“स्वयंश्रुति”}माचार्योऽ{\color{DodgerBlue3}“कार्षीत्”} । शास्त्रेष्वाकाशगुणत्वासिद्धावपि वादी{\color{DodgerBlue3}“ष्टस्याक्षतिं”} शास्त्रेष्टस्या{\color{DodgerBlue3}“साध्यत्वं”} शास्त्रेष्टधर्मान्तरासिद्धौ हेतुबलप्रसिद्धसाध्यबाधने \edlabel{pvv.440-1}\footnote{\label{pvv.440-1}  १ तदसम्बद्धानित्यत्वे बाधाभ्युपगमे ।}गन्धे भूतगुणताबाधा\edlabel{pvv.440-2}\footnote{\label{pvv.440-2}  २ तत्परिहृतावप्येवं विरोधः स्यादित्यनवस्था ।}यां शब्दे कृतकत्वमनित्यत्वसाधनं विरुद्धं स्यादित्यादि कामन{\color{DodgerBlue3}“वस्थाञ्च”} परस्य {\color{DodgerBlue3}“दर्शयन्नाचा”} र्यः स्वयंश्रुतिमकार्षीदिति पूर्व्वेण सम्बन्धः ॥ (७५)
	\pend
      \label{div_pvv.4.76}\edlabel{div_pvv.4.76}
	  
	% new div opening: depth here is 2
	
	  \bigskip
	  \begingroup
	  \large
	
	    
	    \stanza[\smallbreak]
	\label{pv.4.76}\edlabel{pv.4.76}\flagstanza{\tiny\textenglish{...v.4.76}}समयाहितभेदस्य परिहारेण धर्मिणः ।&प्रसिद्धस्य गृहीत्यर्था जगादान्यः स्वयंश्रुतिम् ॥ ७६ ॥\&[\smallbreak]


	
	  \endgroup
	

	  \pstart {\color{DodgerBlue3}“समयेन”} \edlabel{pvv.440-3}\footnote{\label{pvv.440-3}  ३ टीकाकारकल्पितार्थदूषणायाह ।} सिद्धान्ते{\color{DodgerBlue3}“नाहित”} आरोपितो {\color{DodgerBlue3}“भेदो”} विशेष आकाशगुणत्वादिर्यस्य {\color{DodgerBlue3}“तस्य धर्मिणः”} परिहारेणागमनिरपेक्षप्रमाणबलात् {\color{DodgerBlue3}“प्रसिद्धस्य”} धर्मिणः शब्दमात्रा{\color{DodgerBlue3}“देर्गृहीतिरित्यर्थः”} प्रयोजनं यस्तास्तां {\color{DodgerBlue3}“स्वयंश्रुतिमन्यो  जगाद”} । स्वं प्रसिद्धो धर्मी कार्यो नागमसिद्ध इत्यर्थः । (७६)
	\pend
      \label{div_pvv.4.77}\edlabel{div_pvv.4.77}
	  
	% new div opening: depth here is 2
	

	  \pstart अत्राह ।
	\pend
      
	  \bigskip
	  \begingroup
	  \large
	
	    
	    \stanza[\smallbreak]
	\label{pv.4.77}\edlabel{pv.4.77}\flagstanza{\tiny\textenglish{...v.4.77}}विचारप्रस्तुतेरेव प्रसिद्धःसिद्ध आश्रयः ।&स्वेच्छाकल्पितभेदेषु पदार्थेष्वविवादतः ॥ ७७ ॥\&[\smallbreak]


	
	  \endgroup
	

	  \pstart धर्मिणि साध्यधर्मस्य भावाभाव{\color{DodgerBlue3}“विचारप्रस्तुतेरे”}वागममनपेक्ष्य {\color{DodgerBlue3}“प्रसिद्ध आश्रयो”} धर्मी सिद्धः स्वेच्छया {\color{DodgerBlue3}“कल्पितो भेदो”} विशेषो येषां तेषु {\color{DodgerBlue3}“पदार्थेष्वविवादतो”} विवादाभावात् । न हि कल्पितभेदे धर्मिणि कश्चित् प्रेक्षावान् कस्यचिद् धर्मस्य साधनं बाधनं वेहते (।) किन्तु प्रमाणप्रतीते वस्तुनि । अतस्तदर्थमपि स्वयंग्रहणमनुपयुक्तं ॥ (७७)
	\pend
      \leavevmode\marginnote{\textenglish{441/s}}\label{div_pvv.4.78}\edlabel{div_pvv.4.78}
	  
	% new div opening: depth here is 2
	
	  \bigskip
	  \begingroup
	  \large
	
	    
	    \stanza[\smallbreak]
	\label{pv.4.78}\edlabel{pv.4.78}\flagstanza{\tiny\textenglish{...v.4.78}}असाध्यतामथ प्राह सिद्धादेशेन धर्मिणः ।&स्वरूपेणैव निर्देश्य इत्यनेनैव तद्गतं ॥ ७८ ॥\&[\smallbreak]


	
	  \endgroup
	

	  \pstart {\color{DodgerBlue3}“अथ सिद्धादेशेन”} प्रसिद्धार्थवाचकेन स्वयंशब्देन {\color{DodgerBlue3}“धर्मिणो\edlabel{pvv.441-1}\footnote{\label{pvv.441-1}  १ असिद्धधर्मिपरिहारेण सिद्धधर्मिग्रहमाह ।}ऽसाध्यतां प्राह”} । यथाऽस्ति प्रधानं भेदाना\edlabel{pvv.441-2}\footnote{\label{pvv.441-2}  २ धर्म्यैव प्रधानमसिद्धः क्वास्तित्वं साध्यं ।}मनुपदर्शनादिति । इदमप्ययुक्तं । यस्मात् {\color{DodgerBlue3}“स्वरूपेण”} साध्यत्वेनैव {\color{DodgerBlue3}“निर्देश्य इत्यनेन”} पक्षलक्षणावयवे{\color{DodgerBlue3}“नैव”} च {\color{DodgerBlue3}“तद्ध”}र्मिणः सिद्धस्यासाध्यत्वं {\color{DodgerBlue3}“गतं”} प्रतीतं ॥ (७८)
	\pend
      \label{div_pvv.4.79}\edlabel{div_pvv.4.79}
	  
	% new div opening: depth here is 2
	

	  \pstart तथा ह्ययमिष्टोऽनिराकृतः पक्ष इत्यनेन (।)
	\pend
      
	  \bigskip
	  \begingroup
	  \large
	
	    
	    \stanza[\smallbreak]
	\label{pv.4.79}\edlabel{pv.4.79}\flagstanza{\tiny\textenglish{...v.4.79}}सिद्धसाधनरूपेण निर्देशस्य हि सम्भवे ।&साध्यत्वैनेव निर्देश्य इतीदं फलवद् भवेत् ॥ ७९ ॥\&[\smallbreak]


	
	  \endgroup
	

	  \pstart सिद्धस्य सिद्ध\edlabel{pvv.441-3}\footnote{\label{pvv.441-3}  ३ तद्ध्याकार एव ।}रूपेण निर्देशस्य धर्मवचनस्यासिद्धस्यापि {\color{DodgerBlue3}“साधनरूपेण”}\leavevmode\marginnote{\textenglish{88b/MA}} {\color{DodgerBlue3}“निर्देश”}स्यासिद्धवचनस्य पक्षत्व{\color{DodgerBlue3}“सम्भवे”} हि तत्प्रतिषेधं विदधत् {\color{DodgerBlue3}“साध्यत्वेनैव निर्देश्य”} इतीदं {\color{DodgerBlue3}“फलवद् भवेत्”} । साध्यस्यैव निर्देशः पक्ष इति सिद्धस्य धर्मिणोऽसिद्धस्य च साधनत्वेनोक्तस्य निरासः ॥ (७९)
	\pend
      \label{div_pvv.4.80}\edlabel{div_pvv.4.80}
	  
	% new div opening: depth here is 2
	

	  \pstart किञ्च (।)
	\pend
      
	  \bigskip
	  \begingroup
	  \large
	
	    
	    \stanza[\smallbreak]
	\label{pv.4.80}\edlabel{pv.4.80}\flagstanza{\tiny\textenglish{...v.4.80}}अनुमानस्य सामान्यविषयत्वञ्च वर्ण्णितम् ।&इहैवं न ह्यनुक्तोपि किञ्चित् पक्षे विरुध्यते ॥ ८० ॥\&[\smallbreak]


	
	  \endgroup
	

	  \pstart {\color{DodgerBlue3}“अनुमानस्य सामान्यविषयत्वं”} स्वयमाचार्येण वर्ण्णितं (।) यदि च {\color{DodgerBlue3}“धर्मी”} पक्षः तदा तस्य स्वलक्षणत्वात् सामन्यविषयता व्याहन्येत । कि{\color{DodgerBlue3}“ञ्चेह”} पक्षलक्षण {\color{DodgerBlue3}“एवमुक्त”}क्र(मे)णा{\color{DodgerBlue3}“नुक्तेपि”} स्वयंशब्दे सिद्धधर्म्यसिद्धसाधनव्यवच्छेदार्थे {\color{DodgerBlue3}“किञ्चित्”} {\color{DodgerBlue3}“पक्षे”} प्रतिपाद्ये {\color{DodgerBlue3}“न विरुध्यते”} ॥ (८०)
	\pend
      \label{div_pvv.4.81}\edlabel{div_pvv.4.81}
	  
	% new div opening: depth here is 2
	

	  \pstart नन्वनुक्ते स्वयंशब्दे पक्षलक्षणं (।)
	\pend
      
	  \bigskip
	  \begingroup
	  \large
	
	    
	    \stanza[\smallbreak]
	\label{pv.4.81}\edlabel{pv.4.81}\flagstanza{\tiny\textenglish{...v.4.81}}कुर्याच्चेद् धर्मिणं साध्यं ततः किन्तन्न शक्यते ।&कस्माद्धेत्वन्वयाभावान्न च दोषस्तयोरपि ॥ ८१ ॥\&[\smallbreak]


	
	  \endgroup
	

	  \pstart {\color{DodgerBlue3}“धर्मिणं साध्यं कुर्या”}दिति दोषः । {\color{DodgerBlue3}“ततो”} धर्मिणः पक्षत्वप्रसङ्गात् {\color{DodgerBlue3}“किन्दूषणं”} ॥ तद्धर्म्मिपक्षत्वं कर्त्तुं {\color{DodgerBlue3}“न शक्यत”} इति अशक्यतादूषणं । {\color{DodgerBlue3}“कस्मात्”} कारणाद् {\color{DodgerBlue3}“धर्मी”} \leavevmode\marginnote{\textenglish{442/s}} पक्षो भवितु नार्हति धर्मिणः साध्यत्वेनासिद्धतायां \edlabel{pvv.442-1}\footnote{\label{pvv.442-1}  १ आश्रयासिद्ध्या ।} {\color{DodgerBlue3}“हेतो”}रभावात् । विशेषस्य धर्मिणो दृष्टान्तेऽसम्भवात् (।) {\color{DodgerBlue3}“अन्वयाभावा”}च्च धर्मी पक्षः कर्त्तुं {\color{DodgerBlue3}“न”} शक्यते । (८१)
	\pend
      \label{div_pvv.4.82}\edlabel{div_pvv.4.82}
	  
	% new div opening: depth here is 2
	

	  \pstart नन्वयं दोषस्तयोर्हेतुदृष्टान्तयोर्न तु पक्षस्य ॥ तथा हि (।)
	\pend
      
	  \bigskip
	  \begingroup
	  \large
	
	    
	    \stanza[\smallbreak]
	\label{pv.4.82}\edlabel{pv.4.82}\flagstanza{\tiny\textenglish{...v.4.82}}उत्तरावयवापेक्षो न दोषः पक्ष इष्यते ।&तथा हेत्वादिदोषोपि पक्षदोषः प्रसज्यते ॥ ८२ ॥\&[\smallbreak]


	
	  \endgroup
	

	  \pstart साधनवाक्यस्य पक्षादु{\color{DodgerBlue3}“त्तरेऽवयवे”} हेतुदृष्टान्तादिकेऽ{\color{DodgerBlue3}“पेक्षा”} यस्यासौ {\color{DodgerBlue3}“दोषः”} {\color{DodgerBlue3}“पक्षे नेष्यते”} हेतुदृष्टान्तसम्बन्धित्वात् तस्य । यदि तूत्तरावयवापेक्षोपि पक्षस्य बाधनात् पक्षदोष उच्यते {\color{DodgerBlue3}“तथा”} सति {\color{DodgerBlue3}“हेत्वादिदोषोपि पक्षदोषः प्रसज्यते”} (। ८२)
	\pend
      \label{div_pvv.4.83}\edlabel{div_pvv.4.83}
	  
	% new div opening: depth here is 2
	
	  \bigskip
	  \begingroup
	  \large
	
	    
	    \stanza[\smallbreak]
	\label{pv.4.83}\edlabel{pv.4.83}\flagstanza{\tiny\textenglish{...v.4.83}}सर्वैः पक्षस्य बाधातस्तस्मात् तन्मात्रसङ्गिनः ।&पक्षदोषा मता नान्ये प्रत्यक्षादिविरोधवत् ॥ ८३ ॥\&[\smallbreak]


	
	  \endgroup
	

	  \pstart {\color{DodgerBlue3}“सर्व्वै”}र्हेत्वादिदोषैः  {\color{DodgerBlue3}“पक्षस्य बाधात् तस्मात् तन्मात्रङ्गिनः”} पक्षमात्रसम्बद्धा दोषाः {\color{DodgerBlue3}“पक्षदोषा मताः”} । {\color{DodgerBlue3}“नान्ये”}ऽवयवान्तरापेक्षाः {\color{DodgerBlue3}“प्रत्यक्षादिविरोधवत्”} । यथा प्रत्यक्षादिबाधितत्वमवयवान्तरानपेक्षं पक्षदोषः । (८३)
	\pend
      \label{div_pvv.4.84}\edlabel{div_pvv.4.84}
	  
	% new div opening: depth here is 2
	

	  \pstart तस्माद् (।)
	\pend
      
	  \bigskip
	  \begingroup
	  \large
	
	    
	    \stanza[\smallbreak]
	\label{pv.4.84}\edlabel{pv.4.84}\flagstanza{\tiny\textenglish{...v.4.84}}हेत्वादिलक्षणैर्ब्बाध्यं मुक्त्वा पक्षस्य लक्षणम् ।&उच्यते परिहारार्थमव्याप्तिव्यतिरेकयोः ॥ ८४ ॥\&[\smallbreak]


	
	  \endgroup
	

	  \pstart {\color{DodgerBlue3}“हेत्वादीनां लक्षणैर्ब्बाध्यं”} परिहर्त्तव्यं दोषमन्वयविरोधादिकं {\color{DodgerBlue3}“मुक्त्वा”} पक्षमात्रानुषङ्गिणो{\color{DodgerBlue3}“रव्याप्तिव्यतिरेकयोः”} परिहारार्थं {\color{DodgerBlue3}“पक्षलक्षणमुच्यते”}\edlabel{pvv.442-2}\footnote{\label{pvv.442-2}  २ अन्यथा हेत्वादिलक्षणं निर्विषयं स्यात् ।}व्यतिरेके आधि क्यमभिव्याप्तिरित्यर्थः ॥ (८४)
	\pend
      \label{div_pvv.4.85}\edlabel{div_pvv.4.85}
	  
	% new div opening: depth here is 2
	

	  \pstart तत्र येन पदेन यद् दूषणं परिह्रियते तदाह ।
	\pend
      
	  \bigskip
	  \begingroup
	  \large
	
	    
	    \stanza[\smallbreak]
	\label{pv.4.85}\edlabel{pv.4.85}\flagstanza{\tiny\textenglish{...v.4.85}}स्वयन्निपातरूपाख्या व्यतिरेकस्य बाधिकाः ।&सहानिराकृतेनेष्टश्रुतिरव्याप्तिबाधनी ॥ ८५ ॥\&[\smallbreak]


	
	  \endgroup
	

	  \pstart {\color{DodgerBlue3}“स्वयञ्च निपातञ्च”} एवं {\color{DodgerBlue3}“रूपं”} स्वरूपञ्चा{\color{DodgerBlue3}“ख्या”} श्रुतयः सहानिराकृतेन पदेन {\color{DodgerBlue3}“व्यतिरेक”}स्यातिव्याप्ते{\color{DodgerBlue3}“र्ब्बाधिकाः”} । स्वयंशब्देन शास्त्रेष्टस्य निपातेनासिद्धस्यापि साधनत्वेनोक्तस्य स्वरूपशब्देन सिद्धस्य । {\color{DodgerBlue3}“निराकृत”}शब्देन प्रत्यक्षादि\leavevmode\marginnote{\textenglish{443/s}} निराकृतस्य पक्षत्वं सक्तं निषिध्यते । {\color{DodgerBlue3}“इष्टश्रुतिरव्याप्तेर्ब्बाधनी । इष्टशब्दे ह्यक्रि”}यमाणे निर्द्दिष्टमेव साध्यं साध्यं स्यात् न प्रकरणापन्नमिष्टं ॥ (८५)
	\pend
      \label{div_pvv.4.86}\edlabel{div_pvv.4.86}
	  
	% new div opening: depth here is 2
	

	  \pstart यदि “स्वयंनिपातरूपाख्या व्यतिरेकस्य वाधिकाः” प्र मा ण स मु च्च यल क्ष णे निर्द्दिष्टास्तदा न्यायमुखे “साध्यत्वेनेप्सितः पक्षो विरुद्धार्थानिराकृत” इति पक्षलक्षणे ता न सन्तीति कथन्तेनाव्याप्तिव्यतिरेकयोः परिहार {\color{DodgerBlue3}“इत्याह”} साध्यत्वेनेप्सितः पक्ष इति (।)
	\pend
      
	  \bigskip
	  \begingroup
	  \large
	
	    
	    \stanza[\smallbreak]
	\label{pv.4.86}\edlabel{pv.4.86}\flagstanza{\tiny\textenglish{...v.4.86}}साध्याभ्युपगमः पक्षलक्षणं तेष्वपक्षता ।&निराकृते बाधनतः शेषेऽलक्षणवृत्तितः ॥ ८६ ॥\&[\smallbreak]


	
	  \endgroup
	

	  \pstart {\color{DodgerBlue3}“साध्याभ्युपगमः”} पक्ष इति {\color{DodgerBlue3}“पक्षलक्षणम”}वतिष्ठते । तथा च {\color{DodgerBlue3}“तेषु”} शास्त्रेष्टा{\color{DodgerBlue3}“दिषु”} पञ्चसु व्यावर्त्त्येषु मध्ये {\color{DodgerBlue3}“निराकृते”} प्रत्यक्षादिबाधिते {\color{DodgerBlue3}“बाधनतोऽपक्षता”} विरुद्धार्था निराकृतस्य पक्षविधानात् । {\color{DodgerBlue3}“शेषे”} शास्त्रेष्टे वादिनाऽनिष्टे साधने च सिद्धे साधयितुमिष्ट एष्यमाणे सिद्धे च साध्यविपरीतेऽप्रस्तुते चोक्तमात्रे {\color{DodgerBlue3}“लक्षण”}स्य साध्यत्वेनेप्सितत्वस्या{\color{DodgerBlue3}“वृत्तितो”}ऽपक्षता सिद्धेति परिपूर्ण्णमिदमपि लक्षणं ॥ (८६)
	\pend
      \label{div_pvv.4.87}\edlabel{div_pvv.4.87}
	  
	% new div opening: depth here is 2
	

	  \pstart ननु यथा सत्यर्थेभ्यो \edlabel{pvv.443-1}\footnote{\label{pvv.443-1}  १ मतिबुद्धिपूजार्थेभ्यः ।}वर्त्तमाने कुविधानादीप्सितशब्दो वर्त्तमानमिच्छा-\leavevmode\marginnote{\textenglish{89a/MA}} माह । तथेष्टशब्दोपि तत्र एषिष्यमाणे पक्षत्वमप्रसक्तमेव तत्किं प्र मा णस मु च्च य लक्षणेऽवधारणं कृतमित्याह ।
	\pend
      
	  \bigskip
	  \begingroup
	  \large
	
	    
	    \stanza[\smallbreak]
	\label{pv.4.87}\edlabel{pv.4.87}\flagstanza{\tiny\textenglish{...v.4.87}}स्वयमिष्टाभिधानेन गतार्थेप्यवधारणे ।&कृत्यान्तेनाभिसम्बन्धादुक्तं कालान्तरच्छिदे ॥ ८७ ॥\&[\smallbreak]


	
	  \endgroup
	

	  \pstart {\color{DodgerBlue3}“स्वयमिष्ट”} इत्यनयोः पदयोर{\color{DodgerBlue3}“भिधानेनावधारणे”} निपातार्थे गते प्रतीतेपि {\color{DodgerBlue3}“कृत्यान्तेन”} निर्देश्य शब्देन सर्व्वकालसम्बन्धयोग्याभिधायिनाऽ{\color{DodgerBlue3}“भिसम्बन्धा”}दिष्टशब्दस्यावर्त्तमानकालेच्छाविषयस्यापि पक्षत्वं स्यात् । यथा  आगतो देवदत्तो द्रष्टव्य इति (।) यदा गमिष्यति तदा द्रक्ष्यत इत्यर्थः । अतो वर्तमानकालात् {\color{DodgerBlue3}“कालान्तर”}स्य भविष्यदादेः साध्ये साध्येच्छाविषयस्य {\color{DodgerBlue3}“च्छिदे”} प्रतिषेधार्थ{\color{DodgerBlue3}“मुक्त”}मवधारणं स्वरूपेणैवेति । (८७)
	\pend
      \label{div_pvv.4.88}\edlabel{div_pvv.4.88}
	  
	% new div opening: depth here is 2
	

	  \pstart यस्मात् कृत्यान्तेनाभिसम्बन्धात् कालसामान्यवृत्तिः (।)
	\pend
      
	  \bigskip
	  \begingroup
	  \large
	
	    
	    \stanza[\smallbreak]
	\label{pv.4.88}\edlabel{pv.4.88}\flagstanza{\tiny\textenglish{...v.4.88}}इहानङ्गमिषेर्न्निष्ठा तेनेप्सितपदे पुनः ।&अङ्गमेव तयाऽसिद्धहेत्वादि प्रतिषिध्यते ॥ ८८ ॥\&[\smallbreak]


	
	  \endgroup
	\leavevmode\marginnote{\textenglish{444/s}}

	  \pstart {\color{DodgerBlue3}“इह”} प्र मा ण स मु च्च (य)लक्षणे निष्ठा वर्त्तमानसाध्यत्वेष्टिप्रतिपादन{\color{DodgerBlue3}“मत्यनङ्ग”}हेतुः । न्या य मु खे {\color{DodgerBlue3}“तेन”} कृत्यान्तेन सम्बन्धाभावेनेप्सितपदे पुनरङ्गमेव निष्ठा वर्त्तमानसाध्येच्छाबोधने {\color{DodgerBlue3}“तया”}\edlabel{pvv.444-1}\footnote{\label{pvv.444-1}  १ निष्ठया ।}वर्त्तमानसाध्येच्छाबोधिकयाऽ{\color{DodgerBlue3}“सिद्धहेत्वाद्य”}पीष्यमाणं साध्यत्वेन {\color{DodgerBlue3}“प्रतिषिध्यते”} । तस्मान्न धर्मिणः साध्यता\edlabel{pvv.444-2}\footnote{\label{pvv.444-2}  २ समयाहिता ह्युक्ता ।}प्रतिक्षेपार्थं स्वयंग्रहणं सिद्धत्वे\edlabel{pvv.444-3}\footnote{\label{pvv.444-3}  ३ विचारप्रस्तुतेरेव प्रसिद्धः सिद्ध आश्रयः ।}नैव तत्परिहारस्य लब्धत्वात् ॥ (८८)
	\pend
      \label{div_pvv.4.89}\edlabel{div_pvv.4.89}
	  
	% new div opening: depth here is 2
	

	  \pstart नापि शास्त्रेष्विच्छाप्रवृत्त्यर्थ\edlabel{pvv.444-4}\footnote{\label{pvv.444-4}  ४ असाध्यतामित्याद्युक्त ।}मिच्छामात्रेणैव तद्ग्रहणस्य सिद्धत्वादित्युक्तं ।
	\pend
      
	  \bigskip
	  \begingroup
	  \large
	
	    
	    \stanza[\smallbreak]
	\label{pv.4.89}\edlabel{pv.4.89}\flagstanza{\tiny\textenglish{...v.4.89}}अवाचकत्वच्चायुक्तं तेनेष्टं स्वयमात्मना ।&अनपेक्ष्याखिलं शास्त्रं तद्वादीष्टस्य साध्यता ॥ ८९ ॥\&[\smallbreak]


	
	  \endgroup
	

	  \pstart {\color{DodgerBlue3}“अवाचकत्वाच्चायुक्तं”} । तदर्थं {\color{DodgerBlue3}“स्वय”}ग्रहणं । न हि स्वयंशब्दः स्वेच्छया शास्त्रं ग्राह्यामित्येतदर्थवाचकं किन्तु वादिन एव वाचकः । {\color{DodgerBlue3}“तेन”} तद्वाचकत्वेन स्वयं वादिनाऽ{\color{DodgerBlue3}“त्मना शास्त्रमखिलमनपेक्ष्य यदिष्टं त”}स्य वादी{\color{DodgerBlue3}“ष्टस्य साध्यते”}ष्यते न शास्त्रेष्टस्येत्युपसंहारः । (८९)
	\pend
      \label{div_pvv.4.90}\edlabel{div_pvv.4.90}
	  
	% new div opening: depth here is 2
	
	  \bigskip
	  \begingroup
	  \large
	
	    
	    \stanza[\smallbreak]
	\label{pv.4.90}\edlabel{pv.4.90}\flagstanza{\tiny\textenglish{...v.4.90}}तेनानभीष्टसंसृष्टस्येष्टस्यापि हि बाधने ।&यथा साध्यमबाधतः पक्षहेतून दुष्यतः ॥ ९० ॥\&[\smallbreak]


	
	  \endgroup
	

	  \pstart {\color{DodgerBlue3}“तेन”} \edlabel{pvv.444-5}\footnote{\label{pvv.444-5}  ५ दृष्टानुमेयवचनेन ।}कारणेन वादिनो{\color{DodgerBlue3}“नभीष्टे”}नाकाशगुणत्वेन {\color{DodgerBlue3}“संसृष्टस्येष्टस्या”}नित्यत्वस्या{\color{DodgerBlue3}“पि हि बाधने”}ऽभिधीयमाने {\color{DodgerBlue3}“पक्षहेतू न दुष्यतः”} । किङ्गाकारणमित्याह । {\color{DodgerBlue3}“यथा साध्यमबाधातः”} । न हि वादिनाऽकाशगुणत्वैकार्थसमवाय्यनित्यत्वं साधयितुमिष्टं येनास्य बाधः स्यात् किन्त्वनित्यत्वमात्रं । न चास्य प्रत्यक्षादिबाधास्ति । हेतोर्व्वा तदपेक्षया विरुद्ध{\color{DodgerBlue3}“तादिकं”} । तदेवं स्वयं-निपातरूपाख्या व्यतिरेकस्य बाधिकाः सहानिराकृतेनेति व्याख्यातं ॥ (९०)
	\pend
      \label{div_pvv.4.91}\edlabel{div_pvv.4.91}
	  
	% new div opening: depth here is 2
	

	  \begin{center}%% label @type='head'
	\textbf{(३) सहानिराकृतग्रहणफलम्}
	\end{center}
	

	  \pstart अनिराकृतपदं व्याख्यातुमाह ।
	\pend
      
	  \bigskip
	  \begingroup
	  \large
	
	    
	    \stanza[\smallbreak]
	\label{pv.4.91}\edlabel{pv.4.91}\flagstanza{\tiny\textenglish{...v.4.91}}अनिषिद्धः प्रमाणाभ्यां स चोपगम इष्यते ।&सन्दिग्धे हेतुवचनाद् व्यस्तो हेतोरनाश्रयः ॥ ९१ ॥\&[\smallbreak]


	
	  \endgroup
	\leavevmode\marginnote{\textenglish{445/s}}

	  \pstart {\color{DodgerBlue3}“स चो”}क्तलक्षणः साध्य{\color{DodgerBlue3}“स्योपगमः”} पक्षः । {\color{DodgerBlue3}“प्रमाणाभ्यां”} प्रत्यक्षानुमानाभ्यामनिषिद्ध इष्यते । कस्मादित्याह । संदिग्धेऽर्थे साधकबाधकप्रमाणविषये {\color{DodgerBlue3}“हेतोर्व्वचनाद् व्यस्तः”} प्रमाणप्रतिक्षिप्तो {\color{DodgerBlue3}“हेतोरनाश्रयो”}ऽविषयः । (९१)
	\pend
      \label{div_pvv.4.92}\edlabel{div_pvv.4.92}
	  
	% new div opening: depth here is 2
	

	  \begin{center}%% label @type='head'
	\textbf{(४) बाधा चतुर्विधा}
	\end{center}
	

	  \pstart यदि द्विविधौ पक्षबाधौ तदा प्रत्यक्षानुमानाप्तैः प्रसिद्धेनेति कथ मा चा र्ये ण चतुर्व्विधा सा दर्शितेत्याह ।
	\pend
      
	  \bigskip
	  \begingroup
	  \large
	
	    
	    \stanza[\smallbreak]
	\label{pv.4.92}\edlabel{pv.4.92}\flagstanza{\tiny\textenglish{...v.4.92}}अनुमानस्य भेदेन सा बाधोक्ता चतुर्विधा ।&तत्राभ्युपायः कार्याङ्गं स्वभावाङ्गं जगस्त्थितिः ॥ ९२ ॥\&[\smallbreak]


	
	  \endgroup
	

	  \pstart {\color{DodgerBlue3}“अनुमानस्य भेदेन”} त्रैविध्येन प्रत्यक्षेण चैकेन {\color{DodgerBlue3}“सह सा बाधा चतुर्व्विधोक्ता (।) तत्र”} तेषु बाधकेष्व{\color{DodgerBlue3}“भ्युपाय”} आप्तस्ववचने {\color{DodgerBlue3}“कार्यमङ्गं जगतः स्थिति”}र्व्यवहृतिः प्रसिद्धिः {\color{DodgerBlue3}“स्वभावोऽङ्गं”} हेतुः ॥ (९२)
	\pend
      \label{div_pvv.4.93}\edlabel{div_pvv.4.93}
	  
	% new div opening: depth here is 2
	

	  \begin{center}%% label @type='head'
	\textbf{(५) आगमस्ववचनयोस्तुल्यबलता}
	\end{center}
	

	  \pstart कस्मात् पुनराप्तवचनं स्ववचनञ्चाभ्युपाय उच्यत इत्याह (।)
	\pend
      
	  \bigskip
	  \begingroup
	  \large
	
	    
	    \stanza[\smallbreak]
	\label{pv.4.93}\edlabel{pv.4.93}\flagstanza{\tiny\textenglish{...v.4.93}}आत्मापरोधाभिमतो भूतनिश्चययुक्तवाक् ।&आप्तः स्ववचनं शास्त्रं चैकमुक्तं समत्वतः ॥ ९३ ॥\&[\smallbreak]


	
	  \endgroup
	

	  \pstart {\color{DodgerBlue3}“आत्मापरोधाभिमतो भूत”}स्यार्थस्य {\color{DodgerBlue3}“निश्चयेन युक्ता”} प्रयुक्ता {\color{DodgerBlue3}“वाग्”} यस्य {\color{DodgerBlue3}“स आप्त”} उच्यते (।) एवं परंपरयाऽर्थकार्यत्वेन {\color{DodgerBlue3}“स्ववचनं शास्त्रञ्च समत्वतो”}ऽभ्युपाय इति {\color{DodgerBlue3}“सम”}स्य कार्यलिङ्ग{\color{DodgerBlue3}“मेक\edlabel{pvv.445-1}\footnote{\label{pvv.445-1}  १ आप्तवाच्यं ।}मुक्तं”} । (९३)
	\pend
      \label{div_pvv.4.94}\edlabel{div_pvv.4.94}
	  
	% new div opening: depth here is 2
	

	  \pstart किञ्च (।)
	\pend
      
	  \bigskip
	  \begingroup
	  \large
	
	    
	    \stanza[\smallbreak]
	\label{pv.4.94}\edlabel{pv.4.94}\flagstanza{\tiny\textenglish{...v.4.94}}यथात्मनोऽप्रमाणत्वे वचनं न प्रवर्त्तते ।&शास्त्रसिद्धे तथा नार्थे विचारस्तदनाश्रये ॥ ९४ ॥\&[\smallbreak]


	
	  \endgroup
	

	  \pstart वक्तु{\color{DodgerBlue3}“रात्मनोऽप्रमाणत्वे”} प्रामाण्यनिमित्ताभावात् {\color{DodgerBlue3}“वचनं”}प्रामाण्यं {\color{DodgerBlue3}“न प्रवर्त्तते”} । न\leavevmode\marginnote{\textenglish{89b/MA}} ह्यसत्यार्थेन वचनेन परः प्रतिपादयितुं शक्यते । विसम्वादनाश्रयस्य वचनादर्थापत्तेः । ततो यथा प्रतिपादयितुः प्रामाण्य एव सति वचनं {\color{DodgerBlue3}“प्रवर्त्तते”} (।) {\color{DodgerBlue3}“तथा शास्त्रसिद्धेऽर्थे त”}स्य शास्त्रप्रामाण्यस्या{\color{DodgerBlue3}“नाश्रयेण न विचारः”} प्रवर्त्तते (।) प्रमाणविषयो \leavevmode\marginnote{\textenglish{446/s}} लब्धः साक्षादस्यैव प्रमाणस्य शास्त्रस्यात्यन्तपरोक्षार्थे विचारः प्रस्तूयते प्रेक्षैर्नानस्य । ततः शास्त्रस्ववचनयोः प्रामाण्येऽभ्युपगतेपि साम्यमुक्तं ॥ (९४)
	\pend
      \label{div_pvv.4.95}\edlabel{div_pvv.4.95}
	  
	% new div opening: depth here is 2
	

	  \pstart साम्यमेव पुनः किमर्थमुपदर्शितामित्याह (।)
	\pend
      
	  \bigskip
	  \begingroup
	  \large
	
	    
	    \stanza[\smallbreak]
	\label{pv.4.95}\edlabel{pv.4.95}\flagstanza{\tiny\textenglish{...v.4.95}}तत्प्रस्तावाश्रयत्वे हि शास्त्रं बाधकमित्यमुम् ।&वक्तुमर्थं स्ववाचास्य सहोक्तिः साम्यदृष्टये ॥ ९५ ॥\&[\smallbreak]


	
	  \endgroup
	

	  \pstart {\color{DodgerBlue3}“हि”} यस्मात् {\color{DodgerBlue3}“तत्प्रस्ताव”}स्य विचारप्रक्रमस्या{\color{DodgerBlue3}“श्रयत्वेऽधि”}करणत्वे सति {\color{DodgerBlue3}“शास्त्रं सिद्धे”} धर्मिणि शास्त्रं प्रतिज्ञार्थविरुद्धं{\color{DodgerBlue3}“बाधकं”} न वस्तुबलप्रवृत्तानुमान{\color{DodgerBlue3}“मित्यमुमर्थं वक्तु”}मस्य शास्त्रस्य {\color{DodgerBlue3}“स्ववाचा साम्यस्य दृष्टये”} दर्शनार्थं {\color{DodgerBlue3}“सहोक्ति”}रनयोरभ्युपायतानिर्देशे । स्ववचनमपि ह्युच्चारणसामर्थ्यादु\edlabel{pvv.446-1}\footnote{\label{pvv.446-1}  १ स्वाप्रामाण्यबोधेनुच्चारणात् ।}पगतप्रामाण्यं सिद्धे धर्मिणि विचारप्रक्रमे प्रतिज्ञार्थविषयबाधकं {\color{DodgerBlue3}“बाधकं”} दृष्टं (।) न वस्तुबलप्रवृत्तानुमानेन इदमनयोः साम्यदर्शनप्रयोजनं । (९५)
	\pend
      \label{div_pvv.4.96_4.97}\edlabel{div_pvv.4.96_4.97}
	  
	% new div opening: depth here is 2
	
	  \bigskip
	  \begingroup
	  \large
	
	    
	    \stanza[\smallbreak]
	\label{pv.4.96}\edlabel{pv.4.96}\flagstanza{\tiny\textenglish{...v.4.96}}उदाहारणमप्यत्र सदृशं तेन वर्णितम् ।&प्रमाणानामभावे हि शास्त्रवाचोरयोगातः ॥ ९६ ॥\&[\smallbreak]


	
	  \endgroup
	

	  \pstart अत्र शास्त्रवचनयोः व्याघातेना चा र्ये {\color{DodgerBlue3}“णोदाहरणमपि”} सदृशमभिन्नं {\color{DodgerBlue3}“वर्ण्णितं”} । {\color{DodgerBlue3}“यथा न सन्ति प्रमाणानि”} प्रमेयार्थानीति । कथं पुनरत्र शास्त्रार्थस्ववचनाभ्यां व्याघात इत्याह । {\color{DodgerBlue3}“प्रमाणानामभावे हि शास्त्रवाचोरयोगातः”} । अनुपपत्तेः प्रमाणसम्भवे हि शास्त्रवचनयोः परप्रतिपादनार्थमुक्तिर्युक्ता । तस्मादुच्चारण{\color{DodgerBlue3}“सामर्थ्यादभ्यु”}पगतप्रामाण्यात् प्रयोगवचनादेव प्रमाणाभावप्रतिज्ञा बाध्यते ।
	\pend
      

	  \pstart एतच्चोदाहरणम् (।)
	\pend
      
	  \bigskip
	  \begingroup
	  \large
	
	    
	    \stanza[\smallbreak]
	\label{pv.4.97}\edlabel{pv.4.97}\flagstanza{\tiny\textenglish{...v.4.97}}स्ववाग्विरोधे विस्पष्टमुदाहरणमागमे ।&दिङ्मात्रदर्शनं तत्र प्रत्येधर्मोसुखप्रदः ॥ ९७ ॥\&[\smallbreak]


	
	  \endgroup
	

	  \pstart {\color{DodgerBlue3}“स्ववाग्विरोधे”} स्ववचनव्याहतौ  {\color{DodgerBlue3}“विस्पष्टं”} तथा हि प्रमाणाभावप्रतिज्ञावचनोच्चारणसामर्थ्याभ्युपगतप्रामाण्येन वचनेनैव बाध्यते । आगमे शास्त्रे पुनरुदाहरणस्य {\color{DodgerBlue3}“दिङ्मात्रदर्शन”}मुपलक्षणमात्रमेतत् । न तु मुख्यमुदाहरणं । प्रमाणस्य धर्मिण आगमसिद्धत्वाभावात् । इदं पुनर्मुख्यमुदाहरणं आगमे {\color{DodgerBlue3}“प्रेत्य”} परलोके {\color{DodgerBlue3}“धर्मोऽसुखप्रदः”} (।) आगमसिद्धे धर्मिणि धर्मे सामान्येऽसुखप्रदत्वस्य विशेषस्य सुखप्रदत्वेन विरुद्धेनागमसिद्धेन बाधनात् । (९७)
	\pend
      \label{div_pvv.4.98}\edlabel{div_pvv.4.98}
	  
	% new div opening: depth here is 2
	

	  \pstart {\color{DodgerBlue3}“किन्तु (।)”}
	\pend
      \leavevmode\marginnote{\textenglish{447/s}}
	  \bigskip
	  \begingroup
	  \large
	
	    
	    \stanza[\smallbreak]
	\label{pv.4.98}\edlabel{pv.4.98}\flagstanza{\tiny\textenglish{...v.4.98}}शास्त्रिणोप्यतदालम्बे विरुद्धोक्तौ तु वस्तुनि ।&न बाधा प्रतिबन्धः स्यात् तुल्यकक्षतया तयोः ॥ ९८ ॥\&[\smallbreak]


	
	  \endgroup
	

	  \pstart {\color{DodgerBlue3}“शास्त्रिणो”}ऽभ्युपगतशास्त्रस्या{\color{DodgerBlue3}“तदालम्बे”} शास्त्रासिद्धे\edlabel{pvv.447-1}\footnote{\label{pvv.447-1}  १ यो न शास्त्रेण दर्शितः ।} प्रमाणसिद्धे {\color{DodgerBlue3}“वस्तुनि”} धर्मिणि शास्त्रप्रतिज्ञा{\color{DodgerBlue3}“विरुद्धस्य धर्मस्योक्तौ न सा बाधा”} । यथा मी मां स क स्य गृहीतशास्त्रस्य प्रत्यक्षसिद्धे शब्दे धर्मिणि कृतकत्वादनित्यत्वोक्तावपि {\color{DodgerBlue3}“शास्त्र”}प्रतिज्ञातेन नित्यत्वेन न बाधा । यदि बाधा किन्तर्हि भवतीत्याह । {\color{DodgerBlue3}“प्रतिबन्धः स्यात्”} । कस्मादित्याह । {\color{DodgerBlue3}“तुल्यकक्ष”}त्वात् । समबलत्वात् । यथा माता मे बन्ध्या चेति स्ववाचि तुल्यकक्षत्वात् पदयोः परस्परं प्रतिबन्धः । (९८)
	\pend
      \label{div_pvv.4.99}\edlabel{div_pvv.4.99}
	  
	% new div opening: depth here is 2
	

	  \pstart ननु स्ववचनयोस्तुल्यकक्षत्वाद् युक्तः प्रतिबन्धः । आगमस्ववचनयोस्तुल्यबलतैव कथमित्याह ।
	\pend
      
	  \bigskip
	  \begingroup
	  \large
	
	    
	    \stanza[\smallbreak]
	\label{pv.4.99}\edlabel{pv.4.99}\flagstanza{\tiny\textenglish{...v.4.99}}यथा स्ववाचि तच्चास्य तदा स्ववचनात्मकम् ।&तयोः प्रमाणं यस्यास्ति तत् स्यादन्यस्य बाधकम् ॥ ९९ ॥\&[\smallbreak]


	
	  \endgroup
	

	  \pstart {\color{DodgerBlue3}“तच्च”} शास्त्रं नित्यत्वप्रतिज्ञातस्य वादिनः {\color{DodgerBlue3}“तदा”} प्रसिद्धे धर्मिणि शास्त्रविरुद्ध-\leavevmode\marginnote{\textenglish{90a/MA}} प्रतिज्ञासमये स्वोपगमस्वीकृतप्रमाणत्वात् {\color{DodgerBlue3}“स्ववचनात्मकं”} \edlabel{pvv.447-2}\footnote{\label{pvv.447-2}  २ इति वचनात्मत्वाविशेषः ।} जातं वचनं शास्त्रञ्च स्वयमभ्युपगतप्रामाण्ययुक्तं तुल्यकक्षं {\color{DodgerBlue3}“यथा स्ववाचि”} माता मे बन्ध्येति वचनमात्रयोः प्रतिबन्धोन्योन्यं । {\color{DodgerBlue3}“तयोः”} शास्त्रवचनयोर्व्विरूद्धार्थाभिधायिनोर्मध्ये {\color{DodgerBlue3}“यस्य प्रमाण”}मनुवर्त्तक{\color{DodgerBlue3}“मस्ति तत्”} प्रमाणवत् । {\color{DodgerBlue3}“अन्यस्या”}प्रमाणकस्य {\color{DodgerBlue3}“बाधकं”} भवति । यथाऽनित्यत्वं नित्यस्य शब्दे । (९९)
	\pend
      \label{div_pvv.4.100}\edlabel{div_pvv.4.100}
	  
	% new div opening: depth here is 2
	
	  \bigskip
	  \begingroup
	  \large
	
	    
	    \stanza[\smallbreak]
	\label{pv.4.100}\edlabel{pv.4.100}\flagstanza{\tiny\textenglish{....4.100}}प्रतिज्ञामनुमानं वा प्रतिज्ञाऽपेतयुक्तिका ।&तुल्यकक्षा यथार्थम्वा बाधेत कथमन्यथा ॥ १०० ॥\&[\smallbreak]


	
	  \endgroup
	

	  \pstart अन्यथा यदि प्रमाणसिद्धे धर्मिणि प्रमाणाननुगृहीतयोः परस्परं बाधा । प्रमाणाभावे केनेतरस्य नाबाधेतीष्यते तदा {\color{DodgerBlue3}“प्रतिज्ञा”}विपरीता शास्त्रवचनाख्या{\color{DodgerBlue3}“ऽपेतयुक्तिका”} प्रतिज्ञा प्रमाण्याभ्युपगमात् {\color{DodgerBlue3}“तुल्यकक्षा”} सती कथं बाधेत । {\color{DodgerBlue3}“यथा”} शब्दे नित्यत्वप्रतिज्ञा शास्त्रोक्ता अनित्यत्व{\color{DodgerBlue3}“प्रतिज्ञां”} विद्यमान{\color{DodgerBlue3}“मनुमानम्वा”} यथार्थं । वस्तुभूतकृतकत्वलि{\color{DodgerBlue3}“ङ्ग”}समुत्थं नित्यत्वप्रतिज्ञया {\color{DodgerBlue3}“कथं”} बाधितं \edlabel{pvv.447-3}\footnote{\label{pvv.447-3}  ३ अपि च आगमानां प्रामाण्यं प्रागेव निवारितमिति कुतो बाधा । तस्माच्छास्त्रमभ्युपगम्य धर्मविचारेष्वयन्दोषः शास्त्रबाधाख्य इष्यते ।}स्यादनित्यत्वात् । (१००)
	\pend
      \label{div_pvv.4.101_4.102}\edlabel{div_pvv.4.101_4.102}
	  
	% new div opening: depth here is 2
	\leavevmode\marginnote{\textenglish{448/s}}
	  \bigskip
	  \begingroup
	  \large
	
	    
	    \stanza[\smallbreak]
	\label{pv.4.101}\edlabel{pv.4.101}\flagstanza{\tiny\textenglish{....4.101}}प्रामाण्यमागमानाञ्च प्रागेव विनिवारितम् ।&अभ्युपायविचारेषु तस्माद् दोषोयमिष्यते\edlabel{pvv.448-asterisk}\footnote{\label{pvv.448-asterisk}  * नास्ति वृत्तौ ॥} ॥ १०१ ॥\&[\smallbreak]


	
	  \endgroup
	
	  \bigskip
	  \begingroup
	  \large
	
	    
	    \stanza[\smallbreak]
	\label{pv.4.102}\edlabel{pv.4.102}\flagstanza{\tiny\textenglish{....4.102}}तस्माद् विषयभेदस्य दर्शनार्थं पृथक्कृतः ।&अनुमानाबर्हिर्भूतोप्यभ्युपायः प्रबाधनात् ॥ १०२ ॥\&[\smallbreak]


	
	  \endgroup
	

	  \pstart {\color{DodgerBlue3}“तस्मात्”} कार्यलिङ्गत्वादनुमानादबहिर्भूतोप्यभ्युपायो {\color{DodgerBlue3}“विषयस्य भेदो”} नानात्वं {\color{DodgerBlue3}“तद्दर्शनार्थ”}मनुमानात् {\color{DodgerBlue3}“पृथक्कृतः”} । अनुमानं सर्व्वत्र बाधकं शास्त्रन्तु शास्त्राश्रये धर्मिणीति बाधकत्वविषयभेदोपदर्शनं पृथक्करणफलं । कथं ज्ञायतेऽ{\color{DodgerBlue3}“नुमानाबहिर्भूतं”} शास्त्रमित्याह । स्वसिद्धे धर्मिणि स्वोपगमविरुद्धस्य धर्मस्य {\color{DodgerBlue3}“प्रबाधनात्”} न ह्यप्रमाणम्बाधकं । प्रमाणञ्चाप्रत्यक्षत्वात् अनुमानमेव ॥ (१०१,१०२)
	\pend
      \label{div_pvv.4.103}\edlabel{div_pvv.4.103}
	  
	% new div opening: depth here is 2
	
	  \bigskip
	  \begingroup
	  \large
	
	    
	    \stanza[\smallbreak]
	\label{pv.4.103a}\edlabel{pv.4.103a}\flagstanza{\tiny\textenglish{...4.103a}}अन्यथाऽतिप्रसङ्गः स्याद् व्यर्थता वा पृथक्कृतेः ।\&[\smallbreak]


	
	  \endgroup
	

	  \pstart {\color{DodgerBlue3}“अन्यथा”} यदि विषयभेदोपदर्शनफलं पृथक्करणं नेष्यते तदा प्रभेददर्शनार्थं वक्तव्यं । तथा  च कार्यस्वभावानुपलम्भानां प्रभेदो यावत् \edlabel{pvv.448-1}\footnote{\label{pvv.448-1}  १ तैरपि बाधनात् ।} सम्भवं वाच्य इत्यति{\color{DodgerBlue3}“प्रसङ्गः”} स्यात् ।
	\pend
      

	  \pstart अथ सदपि भेदान्तरं नोच्यते तदाऽनुमानाच्छास्त्रस्य {\color{DodgerBlue3}“पृथक् कृतेर्व्यर्थता वा”} स्यात् । प्रभेदवचनस्याविवक्षितत्वात् । सामान्यवचनस्यानुमानेनैव सिद्धत्वात् । यदि विषयभेदोपदर्शनार्थमनुमानात् पृथग्वचनं शास्त्रस्य तदाभ्युपगमात् स्ववचनमाचार्येण किमर्थं पृथक्कृतं (न्यायमु खे) । यथा सर्व्वमुक्तं मृषेति (।) तथा औलूक्यस्य नित्यः शब्द इति ।
	\pend
      

	  \pstart अत्राह (।)
	\pend
      
	  \bigskip
	  \begingroup
	  \large
	
	    
	    \stanza[\smallbreak]
	\label{pv.4.103b}\edlabel{pv.4.103b}\flagstanza{\tiny\textenglish{...4.103b}}भेदो वाङ्मात्रवचने प्रतिबन्धः स्ववाच्यपि ॥ १०३ ॥\&[\smallbreak]


	
	  \endgroup
	

	  \pstart {\color{DodgerBlue3}“स्ववाच्यपि भेदः”} स्ववचनस्यापि शास्त्रात् पृथक्करणं अप्रमाणकं वचनं {\color{DodgerBlue3}“वाङ्मात्रं”} तस्मिन् {\color{DodgerBlue3}“प्रतिबन्धः”} । उच्चारणसामर्थ्यादभ्युपगतप्रामाण्यं स्ववचनं ।
	\pend
      

	  \pstart अन्यथोच्चारणमेव न स्यात् । सत्यार्थता च प्रामाण्यं तन्मृषार्थतया वाच्यया निषेध्यत इति तुल्यकक्षतया प्रतिबन्ध एवानयोर्न बाधा । तदेनं वाक्यं स्वार्थं प्रतिबध्नाति । वाक्यान्तरनिर्दिष्टम्वस्तु शास्त्रमित्यनयोर्भेद इत्युक्तं । (१०३)
	\pend
      \label{div_pvv.4.104}\edlabel{div_pvv.4.104}
	  
	% new div opening: depth here is 2
	

	  \pstart स्यादेतत् (।)
	\pend
      \leavevmode\marginnote{\textenglish{449/s}}
	  \bigskip
	  \begingroup
	  \large
	
	    
	    \stanza[\smallbreak]
	\label{pv.4.104}\edlabel{pv.4.104}\flagstanza{\tiny\textenglish{....4.104}}तेनाभ्युपागमाच्छास्त्रं प्रमाणं सर्ववस्तुषु ।&बाधकं, यदि नेच्छेत् स बाधकं किम्पुनर्भवेत् ॥ १०४ ॥\&[\smallbreak]


	
	  \endgroup
	

	  \pstart शास्त्रस्य {\color{DodgerBlue3}“तेन”} वादिना{\color{DodgerBlue3}“भ्युपगमात् सर्व्वत्र वस्तुषु”} धर्मिषु शास्त्रसिद्धे वस्तुबलप्रवृत्ते प्रमाणनिश्चितेषु च {\color{DodgerBlue3}“शास्त्रं प्रमाणं”} सत् {\color{DodgerBlue3}“बाधकमेव”} स्यात् (।) विपरीतप्रतिज्ञाया न प्रतिबन्धकं । तत्कथं प्रमाणीकृतशास्त्रस्य मीमान्सकस्य शब्दे प्रत्यक्षसिद्धे कृतकत्वादनित्यत्वप्रतिज्ञायाः साध्यमानायाः शास्त्रेण प्रतिबन्धो न बाधेत्युक्तं । \edlabel{pvv.449-1}\footnote{\label{pvv.449-1}  १ सिद्धान्ती बन्धाति ।} अथ {\color{DodgerBlue3}“स”} वादी {\color{DodgerBlue3}“यदि”} शास्त्रं प्रमाणं {\color{DodgerBlue3}“नेच्छेत्”} तदा {\color{DodgerBlue3}“किं पुनर्द्धर्मस्यासुख”}प्रदत्वस्य बाधायां प्रमाणं {\color{DodgerBlue3}“भवेत्”} । न ह्यप्रमाणं क्वचित् प्रमाणीकृतशास्त्रस्य शब्दे प्रत्यक्षसिद्धे भवितुमर्हति । (१०४)
	\pend
      \label{div_pvv.4.105}\edlabel{div_pvv.4.105}
	  
	% new div opening: depth here is 2
	

	  \pstart धर्मस्य सुखप्रदत्वेन निर्देशादसुखप्रदत्वे साध्ये स्ववचनविरोध एवेति चेत् । एवं तर्हि (।)
	\pend
      
	  \bigskip
	  \begingroup
	  \large
	
	    
	    \stanza[\smallbreak]
	\label{pv.4.105}\edlabel{pv.4.105}\flagstanza{\tiny\textenglish{....4.105}}स्ववाग्विरोधेऽभेदः स्यात् स्ववाक्शास्त्रविरोधयोः ।&पुरुषेच्छाकृता चास्य परिपूर्ण्णा प्रमाणता ॥ १०५ ॥\&[\smallbreak]


	
	  \endgroup
	

	  \pstart {\color{DodgerBlue3}“स्ववाग्विरोधे”}ऽभ्युपगम्यमाने {\color{DodgerBlue3}“स्ववाक्शास्त्रविरोधयोरभेद”} एव {\color{DodgerBlue3}“स्यात्”} । द्वयोरभ्युपगमसिद्धप्रमाणत्वात् । \leavevmode\marginnote{\textenglish{90b/MA}}{\color{DodgerBlue3}“पुरुषेच्छाकृता चास्य”} शास्त्रस्य {\color{DodgerBlue3}“परिपूर्ण्णा प्रमाणते”}त्युपहसति । सत्यप्यागमस्ववचनयोरभ्युपगमाहितप्रामाण्यादभेदे {\color{DodgerBlue3}“भेद”}दर्शननिमित्तञ्चोक्तं । (१०५)
	\pend
      \label{div_pvv.4.106}\edlabel{div_pvv.4.106}
	  
	% new div opening: depth here is 2
	

	  \pstart यस्माच्छास्त्रं तत्सिद्ध एव धर्मिणि लिङ्गे च बाधकं न तु प्रमाणसिद्धेपि ।
	\pend
      
	  \bigskip
	  \begingroup
	  \large
	
	    
	    \stanza[\smallbreak]
	\label{pv.4.106}\edlabel{pv.4.106}\flagstanza{\tiny\textenglish{....4.106}}तस्मात् प्रसिद्धष्वर्थेषु शास्त्रत्यागेपि न क्षतिः ।&परोक्षेष्वागमाऽनिष्टौ न चिन्तैव प्रवर्त्तते ॥ १०६ ॥\&[\smallbreak]


	
	  \endgroup
	

	  \pstart {\color{DodgerBlue3}“तस्मात्”} प्रत्यक्षानुमानाभ्यां {\color{DodgerBlue3}“प्रसिद्धेषु”} धर्मिलिङ्गसाध्यसम्बन्धादिषु सत्सु {\color{DodgerBlue3}“शास्त्रत्यागेपि न क्षतिः\edlabel{pvv.449-2}\footnote{\label{pvv.449-2}  २ इष्टाप्रतिबन्धात् ।}”} । यथा शब्दकृतत्वानित्यत्वसम्बन्धादिषु प्रमाणसि- द्धेषु शास्त्रस्याकाशगुणत्वप्रतिपादकस्य त्यागेपि नानिष्टं । {\color{DodgerBlue3}“परोक्षेषु”} धर्माधर्मादिषु \edlabel{pvv.449-3}\footnote{\label{pvv.449-3}  ३ आगमो ग्राह्य एव ।} पुनरागमस्य प्रमाणत्वेनानिष्टौ {\color{DodgerBlue3}“चिन्तैव न प्रवर्त्तते(।) नह्यसिद्धे धर्मिणि”} विचारः । (१०६)
	\pend
      \label{div_pvv.4.107}\edlabel{div_pvv.4.107}
	  
	% new div opening: depth here is 2
	

	  \pstart ननु शास्त्रञ्चेन्न प्रमाणं कथन्तत्सिद्धे धर्मिणि लिङ्गादौ वा विचार इत्याह ।
	\pend
      \leavevmode\marginnote{\textenglish{450/s}}
	  \bigskip
	  \begingroup
	  \large
	
	    
	    \stanza[\smallbreak]
	\label{pv.4.107}\edlabel{pv.4.107}\flagstanza{\tiny\textenglish{....4.107}}विरोधोद्भावनप्राया परीक्षाप्यत्र तद्यथा ।&अधर्ममूलं रागादि स्नानञ्चाधर्मनाशनम् ॥ १०७ ॥\&[\smallbreak]


	
	  \endgroup
	

	  \pstart {\color{DodgerBlue3}“अत्र”} शास्त्रे {\color{DodgerBlue3}“परीक्षापि”} या क्रियते सा पूर्व्वापराभ्यां {\color{DodgerBlue3}“विरोधोद्भावनप्राया न”} वास्तवी {\color{DodgerBlue3}“तद्यथाऽधर्मस्य मूलं रागादीति”} क्वचिदुक्तं {\color{DodgerBlue3}“स्नानञ्चा\edlabel{pvv.450-1}\footnote{\label{pvv.450-1}  १ जपहोमादि ।}धर्मनाशन-\edlabel{pvv.450-2}\footnote{\label{pvv.450-2}  २ चित्तमन्तर्गतं दुष्टं ती(र्थ)स्नानैर्न शुध्यति । \par
शतशोपि तद्धौतं क्षुराभाण्डमिवाशुचि ॥ \par
पुनः \par
गङ्गाद्वारे कुशावर्ते विल्वकीनीलपर्वते । \par
स्नात्वा कनखले तीर्थे सम्भवेन्न पुनर्भवे ॥}”} मित्यत्रोच्यमानम्विरुणद्धि । रागादयो हि पापनिदानं न च निदानाविरोधे निदानिनो बाधा । तत्कथं रागाद्यविरोधि स्नानं पापविरोधि स्यात् ।\edlabel{pvv.450-3}\footnote{\label{pvv.450-3}  ३ यद्यन्निदानं न बाधते तन्न तद्बाधकं यथा श्लेष्मणो मधुरादि ॥ येन यन्निदान बाधनास्तेभ्यः तन्निवृत्तिर्यथा शीलं दुश्चरितस्य समाधिः पर्यवस्थानस्य । प्रज्ञाऽनुशयस्य बाधिका ।} अप्रमाणे शास्त्रे विरोधोद्भावनप्रायापि चिन्ता कस्मात् प्रवर्त्त्यते इति चेत् । दानादिचेतनानां प्रवृत्तेर्महानु{\color{DodgerBlue3}“संशा”}(? शंसा)श्रवणात् (।) हिंसादिचेतनानां महापापश्रवणाच्च (।) अपेक्षितफलेषु दानादिष्वयं पुरुषः प्रवृत्तिकामो नागमप्रामाण्यमनाश्रित्यासितुं समर्थः । (१०७)
	\pend
      \label{div_pvv.4.108}\edlabel{div_pvv.4.108}
	  
	% new div opening: depth here is 2
	

	  \begin{center}%% label @type='head'
	\textbf{(६) प्रतीतिबाधा ।}
	\end{center}
	

	  \pstart ततः (।)
	\pend
      
	  \bigskip
	  \begingroup
	  \large
	
	    
	    \stanza[\smallbreak]
	\label{pv.4.108}\edlabel{pv.4.108}\flagstanza{\tiny\textenglish{....4.108}}शास्त्रं यत्सिद्धया युक्त्या स्ववाचा च न बाध्यते ।&दृष्टेऽदृष्टेपि तद् ग्राह्यमिति चिन्ता प्रवर्त्त्यते ॥ १०८ ॥\&[\smallbreak]


	
	  \endgroup
	

	  \pstart {\color{DodgerBlue3}“यच्छास्त्रं दृष्टे”} प्रमाणे-विषये {\color{DodgerBlue3}“युक्त्या”} प्रत्यक्षाद्याख्यया {\color{DodgerBlue3}“न बाध्यते । अदृष्टे”} प्रमाणविषये च स्ववाचाऽगमाश्रये\edlabel{pvv.450-4}\footnote{\label{pvv.450-4}  ४ अत्यन्तपरोक्षे ।}णानुमानेन न बाध्यते (।) तत्प्रमाणत्वेनादृष्टे विषये प्रवृत्तिकामस्य {\color{DodgerBlue3}“ग्राह्यं”} न तु यत्किञ्चि{\color{DodgerBlue3}“दित्यनेन”} प्रयोजनेन शास्त्रे विरोधोद्भावनप्राया {\color{DodgerBlue3}“चिन्ता प्रवर्त्त्यते”} । (१०८)
	\pend
      \label{div_pvv.4.109}\edlabel{div_pvv.4.109}
	  
	% new div opening: depth here is 2
	

	  \pstart शास्त्रस्ववचनविरोधौ व्याख्यातौ ।
	\pend
      \leavevmode\marginnote{\textenglish{451/s}}

	  \begin{center}%% label @type='head'
	\textbf{(क. आप्तलक्षणम्)}
	\end{center}
	

	  \pstart प्रतीतिबाधां व्याख्यातुमाह ।
	\pend
      
	  \bigskip
	  \begingroup
	  \large
	
	    
	    \stanza[\smallbreak]
	\label{pv.4.109}\edlabel{pv.4.109}\flagstanza{\tiny\textenglish{....4.109}}अर्थेष्वप्रतिषिद्धत्वात् पुरुषेच्छानुरोधिनः ।&इष्टशब्दाभिधेयस्याप्तो वाक्षतवाग्जनः ॥ १०९ ॥\&[\smallbreak]


	
	  \endgroup
	

	  \pstart इष्टशब्दाभिधेयत्वस्याभिमतवाच्यत्वस्य {\color{DodgerBlue3}“पुरुषेच्छानुरोधिनः”} पुरुषेच्छाधीनस्या{\color{DodgerBlue3}“र्थेष्वप्रतिषिद्धत्वात्”} (।) न हि पुरुषेच्छायामपि शशी चन्द्रशब्दं वाचकतया न स्वीकरोति । ततश्चात्रेष्ट{\color{DodgerBlue3}“शब्दाभिधेय”}त्वे विषये {\color{DodgerBlue3}“आप्तो”} व्यवहर्त्ता {\color{DodgerBlue3}“जनोऽक्षतवाग”}प्रतिषिद्धेष्टवचनः । अनेन शब्देनायमर्थो मयाऽ{\color{DodgerBlue3}“भिधातव्य इति कल्पनाविषयत्व”}मिष्टशब्दाभिधेयत्वं । तत्र च पुरुषस्यारोपेण स्वेच्छाधीना वचनप्रवृत्तिः । तस्मादिष्टशब्दाभिधेयत्वबाधः पक्षीक्रियमास्तेनैव स्वसम्वेदनसिद्धेन बाध्यते । (१०९)
	\pend
      \label{div_pvv.4.110}\edlabel{div_pvv.4.110}
	  
	% new div opening: depth here is 2
	
	  \bigskip
	  \begingroup
	  \large
	
	    
	    \stanza[\smallbreak]
	\label{pv.4.110}\edlabel{pv.4.110}\flagstanza{\tiny\textenglish{....4.110}}उक्तः प्रसिद्धशब्देन धर्मस्तद्व्यवहारजः ।&प्रत्यक्षादिमिता मानश्रुत्यारोपेण सूचिताः ॥ ११० ॥\&[\smallbreak]


	
	  \endgroup
	

	  \pstart चन्द्रश्चन्द्र इत्यादिशब्द{\color{DodgerBlue3}“व्यवहाराज्जातो धर्मः”} कल्पनाविषयो योग्यताख्य आचार्येण {\color{DodgerBlue3}“प्रसिद्धशब्देन”} तद्यथा शाब्दप्रसिद्धेनेत्यादिनोक्तः । शाब्दी प्रसिद्धिर्व्यवहारः शाब्दप्रसिद्धिः तद्भवो विषयः शाब्दप्रसिद्धः तेन बाधेत्यर्थः । न केवलमिहैव प्रत्यक्षादिबाधास्वपि {\color{DodgerBlue3}“मानश्रतौ\edlabel{pvv.451-1}\footnote{\label{pvv.451-1}  १ नेह प्रत्यक्षाबाधकं किं त्वेतैर्मिताः प्रत्यक्षानुमानशास्त्रं परिच्छिन्नाः श्रावणत्वादयः प्रतिज्ञार्थस्य । प्रत्यक्षातिश्रुतौ श्रावणत्वाद्यारोपेण ।}”} मेयस्या{\color{DodgerBlue3}“रोपेण”} प्रत्यक्षादिभ्यामनुमानागमाभ्यां मितार्था एव विरोधिनो वाचकत्वे पक्षस्य {\color{DodgerBlue3}“सूचिताः”} । (११०)
	\pend
      \label{div_pvv.4.111}\edlabel{div_pvv.4.111}
	  
	% new div opening: depth here is 2
	

	  \pstart न हि प्रत्यक्षानुमाने पुरुषचित्तवर्तिनी आगमश्च बाध्यमाने धर्मिणि सम्भवन्ति\edlabel{pvv.451-2}\footnote{\label{pvv.451-2}  २ एषः (?।) तत्रोपलब्धात् तस्य च धर्मा (ः) ।} । तदुपलब्धतद्धर्म्माः सन्तो युज्यन्ते बाधकास्तस्मात्(।)
	\pend
      
	  \bigskip
	  \begingroup
	  \large
	
	    
	    \stanza[\smallbreak]
	\label{pv.4.111}\edlabel{pv.4.111}\flagstanza{\tiny\textenglish{....4.111}}तदाश्रयभुवामिच्छानुरोधादनिषेधिनाम् ।&कृतानामकृतानां च योग्यं विश्वं स्वभावतः ॥ १११ ॥\&[\smallbreak]


	
	  \endgroup
	

	  \pstart तस्य लोकस्याश्रयेण भुवां भवतां शब्दानां व्यावहारिकाणांइच्छानुरोधिनां(अ)\leavevmode\marginnote{\textenglish{91a/MA}} ऽर्थेषु वाचकत्वप्रवृत्तेः कारणात् क्वचिदपि विषयेऽनिषेधिनां निषेधरहितानां कृतानां संकेतितानामसंकेतितानाञ्च विश्वमिदं वाच्यत्वेन स्वभावादेव योग्यं ॥ (१११)
	\pend
      \label{div_pvv.4.112}\edlabel{div_pvv.4.112}
	  
	% new div opening: depth here is 2
	

	  \begin{center}%% label @type='head'
	\textbf{ख. योग्यता प्रसिद्धिशब्दार्थ;}
	\end{center}
	

	  \pstart यतश्चेष्टशब्दाभिधेयत्वयोग्यता सर्व्वत्र सम्भवति । अतो विशेषानपेक्षणाद् (।)
	\pend
      \leavevmode\marginnote{\textenglish{452/s}}
	  \bigskip
	  \begingroup
	  \large
	
	    
	    \stanza[\smallbreak]
	\label{pv.4.112}\edlabel{pv.4.112}\flagstanza{\tiny\textenglish{....4.112}}अर्थमात्रानुरोधिन्या भाविन्या भूतयापि वा ।&बाध्यते प्रतिरुन्धानः शब्दयोग्यतया तया ॥ ११२ ॥\&[\smallbreak]


	
	  \endgroup
	

	  \pstart {\color{DodgerBlue3}“अर्थमात्रानुरोधिन्या तया शब्दयोग्यतया भा”}विसंकेतापेक्षया {\color{DodgerBlue3}“भाविन्या”} । अतीतसंकेतापेक्षया {\color{DodgerBlue3}“भूतयापि वा”} व्यवस्थितया तामेव ({\color{DodgerBlue3}“योग्यतां प्रतिरुन्धानः”} । {\color{DodgerBlue3}“यथाऽचन्द्रः”} शशी सत्वादिति {\color{DodgerBlue3}“बाध्यते”}) । (११२)
	\pend
      \label{div_pvv.4.113}\edlabel{div_pvv.4.113}
	  
	% new div opening: depth here is 2
	
	  \bigskip
	  \begingroup
	  \large
	
	    
	    \stanza[\smallbreak]
	\label{pv.4.113}\edlabel{pv.4.113}\flagstanza{\tiny\textenglish{....4.113}}तद्योग्यताबलादेव वस्तुतो घटितो ध्वनिः ।&सर्व्वोस्यामप्रतीतेपि तस्मिंस्तत्सिद्धता ततः ॥ ११३ ॥\&[\smallbreak]


	
	  \endgroup
	

	  \pstart योग्यताबलादेव वस्तुतः सामर्थ्यात् सर्व्वः संकेतितोऽसंकिततश्च ध्वनिरस्यां योग्यतायां घटितः सम्बद्धः साक्षात् वाचकत्वेन तस्मिन् शब्देऽप्रतीतेपि वस्तुनि अप्रातिकूल्यलक्षणस्य योग्यत्वस्य सर्व्वदा स्थितेः । यत एवं ततस्तस्या योग्यतायः शब्दप्रसिद्धताचार्येणोक्ता । यद् यत्र समर्थ तदसंमुखीभावेपि तत् तेन व्यपदिष्य (? श्य)ते यथा पाचक इति । समर्थञ्च वस्त्विष्टशब्दाभिधेयत्व इति कृत्वा शाब्दप्रसिद्ध इति । (११३)
	\pend
      \label{div_pvv.4.114}\edlabel{div_pvv.4.114}
	  
	% new div opening: depth here is 2
	
	  \bigskip
	  \begingroup
	  \large
	
	    
	    \stanza[\smallbreak]
	\label{pv.4.114}\edlabel{pv.4.114}\flagstanza{\tiny\textenglish{....4.114}}असाधारणता न स्यात् बाधाहेतोरिहान्यथा ।&तन्निषेधोऽनुमानात् स्याच्छब्दार्थेऽनक्षवृत्तितः ॥ ११४ ॥\&[\smallbreak]


	
	  \endgroup
	

	  \pstart {\color{DodgerBlue3}“अन्यथा”} यदि शब्दोऽसंमुखीभवन्नपि योग्यतायां न सम्बद्धः । तदेह शब्दयोग्यताप्रतिषेधे कर्त्तव्ये {\color{DodgerBlue3}“बाधाहेतोरचन्द्रः”} शशी सत्वादेरित्यादेर{\color{DodgerBlue3}“साधारणतोक्ता न स्यात्”} । सर्व्वस्य चन्द्रशब्दयोग्यत्वे सपक्षाभावात्मत्वसाधारणं स्यान्नान्यथा । {\color{DodgerBlue3}“शब्दार्थे”} योग्यतालक्षणे कल्पितेऽ{\color{DodgerBlue3}“नक्षवृत्तितः”} । अक्षवृत्त्यभावात् । प्रत्यक्षबाधकमिति तस्या योग्यताया {\color{DodgerBlue3}“निषेधो”}नुमानात् {\color{DodgerBlue3}“स्यात्”} । (११४)
	\pend
      \label{div_pvv.4.115}\edlabel{div_pvv.4.115}
	  
	% new div opening: depth here is 2
	
	  \bigskip
	  \begingroup
	  \large
	
	    
	    \stanza[\smallbreak]
	\label{pv.4.115}\edlabel{pv.4.115}\flagstanza{\tiny\textenglish{....4.115}}असाधारणता तत्र हेतूनां यत्र नान्वयि ।&सत्त्वमित्यस्योदाहारो हेतोरेवं कुतो मतः ॥ ११५ ॥\&[\smallbreak]


	
	  \endgroup
	

	  \pstart {\color{DodgerBlue3}“तत्र”} योग्यताप्रतिषेधे कर्त्तव्ये सर्व्वेषां {\color{DodgerBlue3}“हेतूनां”} सपक्षाभावात् {\color{DodgerBlue3}“असाधारणता\edlabel{pvv.452-1}\footnote{\label{pvv.452-1}  १ सर्वः शब्दो योग्यतायां स्वभावतो घटत इति न कश्चिदचन्द्रोस्ति सपक्षो यत्र वृत्तं लिङ्गमन्वयि स्यात् ।}”} कथमेतदित्याह (।) {\color{DodgerBlue3}“यत्र”} साध्ये {\color{DodgerBlue3}“सत्त्व”}मपि लिङ्गं सर्व्ववस्तुव्यापि {\color{DodgerBlue3}“नाऽन्वयि”} साधारणं भवति तत्रान्यस्य का कथेति {\color{DodgerBlue3}“हेतोः”} कृतक\edlabel{pvv.452-2}\footnote{\label{pvv.452-2}  २ अथवाऽसाधारणत्वादनुमानाभाव इत्यस्यामन्योर्थः । योऽचन्द्रत्वं प्रतिजानीते तं प्रति ब्रुवतो लोकस्यानुमानेत्युच्यते तावन्तावनिष्टौ कथमन्यत्रेष्टिः स्यात् ॥}त्वस्यो{\color{DodgerBlue3}“दाहार”} आचार्येस्यैवं फलः {\color{DodgerBlue3}“सर्व्वहेत्वसाधारणत्वप्रतिपादनप्रयोदनो मतः”} । (११५)
	\pend
      \label{div_pvv.4.116}\edlabel{div_pvv.4.116}
	  
	% new div opening: depth here is 2
	

	  \pstart \leavevmode\marginnote{\textenglish{453/s}}कथं गम्यते  सर्व्वेषां शब्दानां सर्व्वत्रार्थे सिद्धिरित्याह ।
	\pend
      
	  \bigskip
	  \begingroup
	  \large
	
	    
	    \stanza[\smallbreak]
	\label{pv.4.116}\edlabel{pv.4.116}\flagstanza{\tiny\textenglish{....4.116}}संकेतसंश्रयाः शब्दाः स चेच्छामात्रसंश्रयः ।&नासिद्धिः शब्दसिद्धानामिति शाब्दप्रसिद्धिवाक् ॥ ११६ ॥\&[\smallbreak]


	
	  \endgroup
	

	  \pstart सङ्केतमन्तरेण वाचकादृष्टेः {\color{DodgerBlue3}“संकेतसंश्रयाः शब्दाः स च”} संकेतः पुरुषे{\color{DodgerBlue3}“च्छामात्रसंश्रयः”} (।) तदतिरिक्तस्यापेक्षणीयस्याभावात् । तस्मा{\color{DodgerBlue3}“च्छब्दसिद्धानाम”}भिधेयत्वादीनां क्वचिदप्यर्थे {\color{DodgerBlue3}“नासिद्धिः”} । इति हेतोः {\color{DodgerBlue3}“शाब्दप्रसिद्धिराचार्यस्य”} ॥ (११६)
	\pend
      \label{div_pvv.4.117}\edlabel{div_pvv.4.117}
	  
	% new div opening: depth here is 2
	

	  \pstart एतच्च शाब्दप्रसिद्धिवचनम् (।)
	\pend
      
	  \bigskip
	  \begingroup
	  \large
	
	    
	    \stanza[\smallbreak]
	\label{pv.4.117}\edlabel{pv.4.117}\flagstanza{\tiny\textenglish{....4.117}}अनुमानप्रसिद्धेषु विरुद्धाव्यभिचारिणः ।&अभावः दर्शयत्येवं प्रतीतेरनुमात्वतः ॥ ११७ ॥\&[\smallbreak]


	
	  \endgroup
	

	  \pstart वस्तुबलप्रवृत्तेना{\color{DodgerBlue3}“नुमानेन प्रसिद्धे”}ष्वर्थेषु विपरीतधर्मोपस्थापकस्य {\color{DodgerBlue3}“विरुद्धाव्यभिचारिणः”} साधनान्तरस्या{\color{DodgerBlue3}“भावन्दर्शयति”} । कस्मादित्याह । {\color{DodgerBlue3}“एवमी\edlabel{pvv.453-1}\footnote{\label{pvv.453-1}  १ लक्षणयुक्ते बाधासंभवे लक्षणमेन दूषितं स्यात् ।}”}दृश्याः शब्दसिद्धाया योग्यतायाः {\color{DodgerBlue3}“प्रतीते \edlabel{pvv.453-2}\footnote{\label{pvv.453-2}  २ अनुमीयतेऽनयेति प्रसिद्धेरनुमात्वात् ।}”} स्वभावलिङ्गसमुत्थत्वात् {\color{DodgerBlue3}“अनुमात्वतः”} । यथा शब्दसिद्धा योग्यताऽनुमानसिद्धेति बाध्या सत्वादिहेतुना (।) तथाऽन्योपि वस्तुबलप्रवृत्तानुमानविषयः समानत्वात् न्यायस्येत्यर्थः । (११७)
	\pend
      \label{div_pvv.4.118}\edlabel{div_pvv.4.118}
	  
	% new div opening: depth here is 2
	
	  \bigskip
	  \begingroup
	  \large
	
	    
	    \stanza[\smallbreak]
	\label{pv.4.118}\edlabel{pv.4.118}\flagstanza{\tiny\textenglish{....4.118}}अथवा ब्रुवतो लोकस्यानुमाऽभाव उच्यते ।&किन्तेन भिन्नविषया प्रतीतिरनुमानतः ॥ ११८ ॥\&[\smallbreak]


	
	  \endgroup
	

	  \pstart अथवाऽचन्द्रःशशी सत्वादिति विप्रति\edlabel{pvv.453-3}\footnote{\label{pvv.453-3}  ३ लोकश्चन्द्रः शशी ह्लादनाद्भासनाद्वा कर्पूरादिवत् न तत्तस्यासिद्धं ।}पद्यमानं प्रतिपत् प्रतिपादनार्थं {\color{DodgerBlue3}“लोकस्य ब्रुवतः”} शाब्दप्रसिद्धेनासाधारणत्वाद{\color{DodgerBlue3}“नुमानाभाव”} आचार्येणो{\color{DodgerBlue3}“च्यते”} । पारमार्थिकस्य बाध्यत्वस्याभावात् \edlabel{pvv.453-4}\footnote{\label{pvv.453-4}  ४ किं फलमित्याह ? वस्तुतः शशिनि चन्द्रत्वमपि शब्दबलात् ।} । कल्पितं निषेध्यं तच्च पुरुषेच्छामात्राधीनत्वात् सर्व्वत्र सम्भवतीति सर्व्वस्य चन्द्रशब्दयोग्यतायोगान्न कश्चिदचन्द्रः पक्षोस्ति यत्र वर्त्तमानंसत्त्वमसाधारणतां जह्या/?/ । एवन्दर्शिते {\color{DodgerBlue3}“किम्भ”}वतीति चेत् । {\color{DodgerBlue3}“तेना”}नुमानाभावाभिधा- यिना शब्दप्रसिद्धाभिधानेन शब्दसिद्धा {\color{DodgerBlue3}“प्रतीति”}र्व्वस्तुबलप्रवृत्ता{\color{DodgerBlue3}“नुमान”}तो {\color{DodgerBlue3}“भिन्नविष”}-\leavevmode\marginnote{\textenglish{91b/MA}} {\color{DodgerBlue3}“यो”}क्ता भवति । वस्तुविषयं ह्यनुमानं कल्पितगोचरान्तरा शाब्दी प्रतीतिरित्यर्थः ॥ (११८)
	\pend
      \label{div_pvv.4.119}\edlabel{div_pvv.4.119}
	  
	% new div opening: depth here is 2
	\leavevmode\marginnote{\textenglish{454/s}}

	  \begin{center}%% label @type='head'
	\textbf{(ग. वस्तुबलप्रवृत्तमनुमानम्)}
	\end{center}
	
	  \bigskip
	  \begingroup
	  \large
	
	    
	    \stanza[\smallbreak]
	\label{pv.4.119}\edlabel{pv.4.119}\flagstanza{\tiny\textenglish{....4.119}}तेनानुमानाद् वस्तूनां सदसत्तानुरोधिनः ।&भिन्नस्यातद्वशा वृत्तिस्तदिच्छाजेति सूचितम् ॥ ११९ ॥\&[\smallbreak]


	
	  \endgroup
	

	  \pstart {\color{DodgerBlue3}“तेन\edlabel{pvv.454-1}\footnote{\label{pvv.454-1}  १ भेददर्शनेपि किं फलमित्याह ।}”} विषयभेदेन {\color{DodgerBlue3}“वस्तूनां सदसत्तानुरोधिनोऽनुमानात् भिन्नस्य”} कल्पितस्यार्थस्य शब्दयोग्यत्वस्य {\color{DodgerBlue3}“वृत्तिरतद्वशा”} वस्त्वनायत्ता {\color{DodgerBlue3}“तस्य”} पुरुष{\color{DodgerBlue3}“स्येच्छाजेति सूचितं”} भवति । (११९)
	\pend
      \label{div_pvv.4.120}\edlabel{div_pvv.4.120}
	  
	% new div opening: depth here is 2
	

	  \pstart किञ्च (।)
	\pend
      
	  \bigskip
	  \begingroup
	  \large
	
	    
	    \stanza[\smallbreak]
	\label{pv.4.120}\edlabel{pv.4.120}\flagstanza{\tiny\textenglish{....4.120}}चन्द्रतां शशिनोऽनिच्छान् कां प्रतीति स वाञ्छति ।&इति तं प्रत्यदृष्टान्तं तदसाधारणं मतम् ॥ १२० ॥\&[\smallbreak]


	
	  \endgroup
	

	  \pstart {\color{DodgerBlue3}“शशिनः”} सर्व्वजनसिद्धां व्यावहारिकी {\color{DodgerBlue3}“चन्द्रता”} चन्द्रशब्दवाच्यतां {\color{DodgerBlue3}“अनिच्छन्”} परः{\color{DodgerBlue3}“कामन्यां प्रतीतिं स वाञ्छति”} (।) यया वाच्यतासिद्धिः क्वचित् स्यात् । मता च पारमार्थिकी वाच्यताप्रतीतिर्न क्वचिदस्ति । वस्तुतः सर्व्वस्यावाच्यताप्रदर्शनात् । कल्पिवाच्यताप्रतीतिस्तु पुरुषेच्छामात्रप्रभवत्वात् सर्व्वत्राव्याहतैव । अतः सर्व्वस्य चन्द्रशब्दवाच्यतायोगात् सपक्षो नास्तीति तं चन्द्रतापलापिनं वादिनं {\color{DodgerBlue3}“प्रति”} सत्वं {\color{DodgerBlue3}“लिङ्गमदृष्टान्तमसाधारण”}मुक्तमाचार्येण न तु चन्द्रस्यैकस्यान्यत्रासम्भवात् सपक्षविपक्षयोरभावादसाधारणत्वमभिप्रेतमा चा र्य स्य । अचन्द्रत्वे साध्ये घटादेः सपक्षस्य सत्वात् । चन्द्रस्तु विपक्षो मा भूत् (।) तथापि हेतुनिवृत्तिरस्मादव्याहतैव । असतोपि हेतुनिवृत्तेः साधनात् । (१२०)
	\pend
      \label{div_pvv.4.121}\edlabel{div_pvv.4.121}
	  
	% new div opening: depth here is 2
	

	  \pstart अपिच (।)
	\pend
      
	  \bigskip
	  \begingroup
	  \large
	
	    
	    \stanza[\smallbreak]
	\label{pv.4.121}\edlabel{pv.4.121}\flagstanza{\tiny\textenglish{....4.121}}नोदाहरणमेवेदमधिकृत्येदमुच्यते ।&लक्षणत्वात् तथा वृक्षोऽधात्रीत्युक्तौ च बाधनात् ॥ १२१ ॥\&[\smallbreak]


	
	  \endgroup
	

	  \pstart अचन्द्रः शशी सत्त्वादित्येतदेवोदा{\color{DodgerBlue3}“हरणमधिकृत्येदं”} शब्दवाच्यत्वप्रतिक्षेपहेतोरसाधारणत्वं {\color{DodgerBlue3}“नोच्यते । लक्षणत्वात् \edlabel{pvv.454-2}\footnote{\label{pvv.454-2}  २ सर्व्वप्रतीतिविरोधानां सामान्येन लक्षणत्वात् ।}”} । लक्षणेन हि लक्ष्यं व्याप्तं दर्शनीयं न च द्वितीयचन्द्राभावेनोसाधरणतोक्तिरुदाहरणान्तरं व्याप्नोति । यथा वा चन्द्रतायाः प्रतीत्या बाधेष्यते {\color{DodgerBlue3}“तथाऽधात्री वृक्षः”} सत्त्वात् \edlabel{pvv.454-3}\footnote{\label{pvv.454-3}  ३ पार्थिवत्वात् ।} घटादिवदित्युदाहरणो{\color{DodgerBlue3}“क्तौ”} \leavevmode\marginnote{\textenglish{455/s}} सर्व्वलोकसिद्धया पुरुषेच्छाधीनया घटादावपि वृक्षशब्दयोग्यताप्रतीत्या {\color{DodgerBlue3}“सपक्षाभावे-”} नासाधारणत्वाद् वृक्षशब्दवाच्यत्वाभावस्य साधनात् (।) {\color{DodgerBlue3}“यथोक्तमेवा”}साधारणत्वमाचार्यस्येष्टं । (१२१)
	\pend
      \label{div_pvv.4.122}\edlabel{div_pvv.4.122}
	  
	% new div opening: depth here is 2
	

	  \pstart यदप्यु\edlabel{pvv.455-1}\footnote{\label{pvv.455-1}  १ न्या य मु ख टीकाकारमुपक्षिपति ।}च्यते द्वितीयस्य चन्द्रस्याभावादसाधारणतेति तत्राह (।)
	\pend
      
	  \bigskip
	  \begingroup
	  \large
	
	    
	    \stanza[\smallbreak]
	\label{pv.4.122}\edlabel{pv.4.122}\flagstanza{\tiny\textenglish{....4.122}}अत्रापि लोके दृष्टत्वात् कर्पूररजतादिषु ।&समयाद्वर्तमानस्य काऽसाधारणतापि वा ॥ १२२ ॥\&[\smallbreak]


	
	  \endgroup
	

	  \pstart {\color{DodgerBlue3}“लोके कर्प्पूररजतादिषु”} गान्धिकवाचिकादीनां {\color{DodgerBlue3}“समयाद्वर्त्तमानस्य दृष्टत्वा-”} दत्राचन्द्रः शशी सत्त्वादित्युदाहरणे हेतोर{\color{DodgerBlue3}“साधारणतापि का वा”} । यदि द्वितीयचन्द्रो न भवेत् । एवमसाधारणता  स्याद् वस्तुत्वस्य । (१२२)
	\pend
      \label{div_pvv.4.123}\edlabel{div_pvv.4.123}
	  
	% new div opening: depth here is 2
	

	  \pstart स्यादेतत् (।) तत्समयादपि वर्त्तमानस्य चन्द्रत्वादेर्न शब्दवाच्यता । ततो यथा न वह्निशब्दवाच्यता कर्प्पूरस्य तथा चन्द्रशब्दवाच्यता च न स्यादित्यसाधारणतैवेत्याह ।
	\pend
      
	  \bigskip
	  \begingroup
	  \large
	
	    
	    \stanza[\smallbreak]
	\label{pv.4.123a}\edlabel{pv.4.123a}\flagstanza{\tiny\textenglish{...4.123a}}यदि तस्य क्वचित् सिध्येत् सिद्धं वस्तुबलेन तत् ।\&[\smallbreak]


	
	  \endgroup
	

	  \pstart {\color{DodgerBlue3}“तस्यैवं”}वादिनः सत्यपि सामयिके चन्द्रे {\color{DodgerBlue3}“क्वचिदर्थ”}विशेषे चन्द्रशब्दवाच्यत्वं यदि सिध्येत् न तु सर्व्वत्र कर्पूररजतादौ तदा {\color{DodgerBlue3}“तच्च”}न्द्रशब्दवाच्यत्व {\color{DodgerBlue3}“वस्तुबलेन सिद्धं”} स्यात् । न त्वेवं दृश्यते सर्व्वस्य कर्पूरादेश्च शब्दाभिधेयत्वदर्शनात् ।
	\pend
      

	  \pstart अथ यत्रैव कल्प्यते चन्द्रत्वं तदेव तच्छब्दवाच्यं प्रतीयत इति तथाभ्युपगम्यते तदा ।
	\pend
      
	  \bigskip
	  \begingroup
	  \large
	
	    
	    \stanza[\smallbreak]
	\label{pv.4.123b}\edlabel{pv.4.123b}\flagstanza{\tiny\textenglish{...4.123b}}प्रतीतिसिद्धपगमेऽशशिन्यप्यनिवारणम् ॥ १२३ ॥\&[\smallbreak]


	
	  \endgroup
	

	  \pstart सामयिके चन्द्रत्वे क्वापि प्रतीत्या सिद्धस्य चन्द्रशब्दवाच्यत्वस्योपगमे । श{\color{DodgerBlue3}“शिन्य”}पि तच्छब्दवाच्यस्या{\color{DodgerBlue3}“निवारणं”} । (१२३)
	\pend
      \label{div_pvv.4.124}\edlabel{div_pvv.4.124}
	  
	% new div opening: depth here is 2
	
	  \bigskip
	  \begingroup
	  \large
	
	    
	    \stanza[\smallbreak]
	\label{pv.4.124}\edlabel{pv.4.124}\flagstanza{\tiny\textenglish{....4.124}}तस्य वस्तुनि सिद्धस्य शशिन्यप्यनिवारणम् ।&तद्वस्त्त्वभावे शशिनि वारणेपि न दुष्यति ॥ १२४ ॥\&[\smallbreak]


	
	  \endgroup
	

	  \pstart शुक्लतादिके निमित्तभूते वस्तुनि । {\color{DodgerBlue3}“तस्य”} चन्द्रशब्दाभिधेयत्वस्य {\color{DodgerBlue3}“सिद्धस्य”} {\color{DodgerBlue3}“शशिन्यपि”} निमित्तसद्भावाद{\color{DodgerBlue3}“निवारणं”} तस्य निमित्तभूतस्य वस्तुनः {\color{DodgerBlue3}“शशिन्यभावे”} तु\leavevmode\marginnote{\textenglish{92a/MA}} चन्द्रशब्दवाच्यत्वस्य वस्तुत्वाद्धेतोर्व्वारणेपि न किञ्चिद् {\color{DodgerBlue3}“दुष्यति”} । निमित्ताभावे \leavevmode\marginnote{\textenglish{456/s}} नैमित्तिकाभावास्येष्टत्वात् । सामयिकन्तु सर्व्वत्राशक्यवारणमिति {\color{DodgerBlue3}“तच्छब्दयोग्यताप्यबाध्या”} । (१२४)
	\pend
      \label{div_pvv.4.125}\edlabel{div_pvv.4.125}
	  
	% new div opening: depth here is 2
	
	  \bigskip
	  \begingroup
	  \large
	
	    
	    \stanza[\smallbreak]
	\label{pv.4.125}\edlabel{pv.4.125}\flagstanza{\tiny\textenglish{....4.125}}तस्मादवस्तुनियतसंकेतबलभाविनाम् ।&योग्याः पदार्था धर्माणामिच्छाया अनिरोधानात् ॥ १२५ ॥\&[\smallbreak]


	
	  \endgroup
	

	  \pstart {\color{DodgerBlue3}“तस्मादवस्तुनियतो”} वस्तुन्यनियत इच्छाधीनत्वात् {\color{DodgerBlue3}“संकेत”}स्तस्य बलाद् {\color{DodgerBlue3}“भाविनां धर्म्माणां”} वाच्यत्वादीनां {\color{DodgerBlue3}“पदार्थाः”} सर्व्वे धर्मित्वेन {\color{DodgerBlue3}“योग्याः”} । इच्छातः {\color{DodgerBlue3}“पुंसः केनचिद् वाच्यत्वाद्युत्था”}पिकाया {\color{DodgerBlue3}“इच्छाया अनिरोधात्”} । (१२५)
	\pend
      \label{div_pvv.4.126}\edlabel{div_pvv.4.126}
	  
	% new div opening: depth here is 2
	
	  \bigskip
	  \begingroup
	  \large
	
	    
	    \stanza[\smallbreak]
	\label{pv.4.126}\edlabel{pv.4.126}\flagstanza{\tiny\textenglish{....4.126}}तां योग्यतां विरुन्धानं संकेताप्रतिषेधजा ।&प्रतिहन्ति प्रतीत्याख्या योग्यताविषयेऽनुमा ॥ १२६ ॥\&[\smallbreak]


	
	  \endgroup
	

	  \pstart {\color{DodgerBlue3}“तामि”}ष्टशब्दाभिधेयत्व{\color{DodgerBlue3}“योग्यतां”} पदार्थानां सत्त्वादिकाद्धेतोर्व्वि{\color{DodgerBlue3}“रुन्धानं प्रतिक्षिपन्तं”} वादिनं {\color{DodgerBlue3}“प्रतीत्या”}ख्या प्रतीतिसंज्ञिताऽ{\color{DodgerBlue3}“नुमा प्रतिहन्ति । संकेताप्रतिषेधजेति”} स्वभावलिङ्गजत्वमाह । इच्छाधीनत्वात् योग्यताविषये विपरीतधर्मोपस्थानमाह । प्रयोगः पुनः (।) यः पुरुषेच्छानुभिधायी स सर्व्वत्र सम्भवी तद्यथा विकल्पः पुरुषेच्छानुविधायि चार्थेष्विष्टशब्दाभिधेयत्वमिति । (१२६)
	\pend
      \label{div_pvv.4.127}\edlabel{div_pvv.4.127}
	  
	% new div opening: depth here is 2
	
	  \bigskip
	  \begingroup
	  \large
	
	    
	    \stanza[\smallbreak]
	\label{pv.4.127}\edlabel{pv.4.127}\flagstanza{\tiny\textenglish{....4.127}}शब्दानामर्थनियमः संकेतानुविधायिनाम् ।&नेत्यनेनोक्तमत्रैषां प्रतिषेधो विरुध्यते ॥ १२७ ॥\&[\smallbreak]


	
	  \endgroup
	

	  \pstart {\color{DodgerBlue3}“अनेन”} चेष्टशब्दाभिधेयत्वयोग्यताप्रतिषेधबाधनेन दर्शितेन {\color{DodgerBlue3}“शब्दानां संकेतानुविधायि”}नाम{\color{DodgerBlue3}“र्थनियमः”} प्रतिनियतवाचकत्वं {\color{DodgerBlue3}“नेत्युक्त”}म्भवति । ततोस्य स्वेच्छाकल्पितोऽर्थे {\color{DodgerBlue3}“एषां”} शब्दानां वाचकत्वस्य {\color{DodgerBlue3}“प्रतिषेधः”} सत्त्वादिहेतोः {\color{DodgerBlue3}“क्रियमाणो विरुध्य\edlabel{pvv.456-1}\footnote{\label{pvv.456-1}  १ प्रतिज्ञादोषो भवति ।}ते”} । (१२७)
	\pend
      \label{div_pvv.4.128}\edlabel{div_pvv.4.128}
	  
	% new div opening: depth here is 2
	

	  \pstart यद्येवन्तदा क्वचिदर्थे निषेधं कुर्व्वता शब्दो बाध्यः स्यात् । ततश्च गुणा दिकं निमित्तभूतं गुणादि\edlabel{pvv.456-2}\footnote{\label{pvv.456-2}  २ जात्यादि ।}शब्दानां गुणगुणिसम्बन्धादिपारमार्थिकमर्थं गुणिशब्दादीनां निषेधन् बाध्यः स्यादित्याह । येन कल्पितमर्थं शब्दानां बाधमानः प्रतिक्षिप्यते । न तु गुणादिकं तेन (।)
	\pend
      
	  \bigskip
	  \begingroup
	  \large
	
	    
	    \stanza[\smallbreak]
	\label{pv.4.128}\edlabel{pv.4.128}\flagstanza{\tiny\textenglish{....4.128}}नैमित्तिक्याः श्रुतेरर्थमर्थम्वा पारमार्थिकम् ।&शब्दानां प्रतिरुन्धानोऽबाधनार्हो हि वर्ण्णितः ॥ १२८ ॥\&[\smallbreak]


	
	  \endgroup
	\leavevmode\marginnote{\textenglish{457/s}}

	  \pstart {\color{DodgerBlue3}“नैमित्तिक्या”}\edlabel{pvv.457-1}\footnote{\label{pvv.457-1}  १ निमित्ताभावे ।} वस्तुभूतगुणादिनिमित्तवत्याः {\color{DodgerBlue3}“श्रुतेरर्थं गुणादिकं पारमार्थिकमर्थं”} \edlabel{pvv.457-2}\footnote{\label{pvv.457-2}  २ न चन्द्रत्वं परमार्थतोस्तीति ।} गुणिगुणादिसम्बन्धं {\color{DodgerBlue3}“शब्दानां”} गुण्यादिवाचिनां {\color{DodgerBlue3}“प्रतिरुन्धानो”}ऽबाधनार्हो बाधां नार्हतीत्युक्तो \edlabel{pvv.457-3}\footnote{\label{pvv.457-3}  ३ अबाधनार्हः ।}भवति ॥ (१२८)
	\pend
      \label{div_pvv.4.129}\edlabel{div_pvv.4.129}
	  
	% new div opening: depth here is 2
	

	  \pstart यस्माच्च सांकेतिकार्थनिराकरणे प्रतीतिबाधा (।)
	\pend
      
	  \bigskip
	  \begingroup
	  \large
	
	    
	    \stanza[\smallbreak]
	\label{pv.4.129}\edlabel{pv.4.129}\flagstanza{\tiny\textenglish{....4.129}}तस्माद् विषयभेदस्य दर्शनाय पृथक्कृता ।&अनुमानाबहिर्भूता प्रतीतिरपि पूर्व्ववत् ॥ १२९ ॥\&[\smallbreak]


	
	  \endgroup
	

	  \pstart {\color{DodgerBlue3}“तस्मात्”} स्वभावलिङ्गजत्वेना{\color{DodgerBlue3}“नुमानादबहिर्भूता प्रतीतिरपि”} तस्मात्\edlabel{pvv.457-4}\footnote{\label{pvv.457-4}  ४ अनुमानात् ।} {\color{DodgerBlue3}“पृथक्कृता”} (।) किमर्थमित्याह । {\color{DodgerBlue3}“विषयस्य भेदः”} कल्पिताकल्पितत्वं तस्य {\color{DodgerBlue3}“दर्शनाय”} । कल्पितार्थविषया प्रतीतिः । वस्तुविषयन्त्वनुमानमित्यर्थः । {\color{DodgerBlue3}“पूर्व्ववदिति”} । यथा आगमस्ववचनेऽभ्युपगतप्रामाण्येऽप्रत्यक्षत्वादनुमानान्तर्गतेपि विषयभेददर्शनाय पृथग्दर्शिते वस्तुबलप्रवृत्तेनुमानं सर्व्वविषयविचारे त्वागमस्ववचने अधिकृते । तथा प्रतीत्यनुमाने अपि भिन्नविषये इत्यर्थः । (१२९)
	\pend
      \label{div_pvv.4.130}\edlabel{div_pvv.4.130}
	  
	% new div opening: depth here is 2
	

	  \pstart अनुमानबाधायामन्तर्भावादनयोरभ्यु\edlabel{pvv.457-5}\footnote{\label{pvv.457-5}  ५ स्ववचनाप्तवचनयोर्ग्रहोनेन ।}पगमप्रतीतिबाधयोः (।)
	\pend
      
	  \bigskip
	  \begingroup
	  \large
	
	    
	    \stanza[\smallbreak]
	\label{pv.4.130}\edlabel{pv.4.130}\flagstanza{\tiny\textenglish{....4.130}}सिद्धयोः पृथगाख्याने दर्शयंश्च प्रयोजनम् ।&एते सहेतुके प्राह नानुमाध्यक्षबाधने ॥ १३० ॥\&[\smallbreak]


	
	  \endgroup
	

	  \pstart {\color{DodgerBlue3}“सिद्धयो\edlabel{pvv.457-6}\footnote{\label{pvv.457-6}  ६ अनुमानापृथ(?)क्त्वेन निश्चितयोः ।}”}रपि {\color{DodgerBlue3}“पृथगाख्याने”} विषयभेदलक्षणं {\color{DodgerBlue3}“प्रयोजनन्दर्शयन्ना”} चा र्य {\color{DodgerBlue3}“एते”} अभ्युपगमप्रतीतिबाधे {\color{DodgerBlue3}“सहेतुके प्राह\edlabel{pvv.457-7}\footnote{\label{pvv.457-7}  ७ नागः ।}”} । न सन्ति \edlabel{pvv.457-8}\footnote{\label{pvv.457-8}  ८ समुच्चयमाह प्राक् प्रामाण्यमाज्ञाय विरोधमनेन ।} प्रमाणानि प्रमेयार्थानीति\edlabel{pvv.457-9}\footnote{\label{pvv.457-9}  ९ हेतुनानेन सहेतुकमाह ।} प्रतिज्ञामात्रे\edlabel{pvv.457-10}\footnote{\label{pvv.457-10}  १० शास्त्रस्ववचनप्रामाण्याख्येन ।}णेति (।) अत्र प्रतिज्ञामात्रं शास्त्रस्ववचनयोः सिद्धयोरप्रामाण्यप्रतिज्ञाबाधकमुक्तम् (।) अतोप्य\edlabel{pvv.457-11}\footnote{\label{pvv.457-11}  ११ दृष्टान्ताभावात् ।}साधारणत्वादनुमानाभावे शाब्दप्रसिद्धेनापोद्यते न सपक्ष इति । अत्र शाब्दप्रसिद्धेन शशिनश्चन्द्रत्वेनाचन्द्रत्वप्रतिज्ञाया बाधनमुक्तं । {\color{DodgerBlue3}“अनुमाध्यक्षबाधने”} तु न सहेतुके प्राह ।\edlabel{pvv.457-12}\footnote{\label{pvv.457-12}  १२ उद्योतकरादिनोक्तः । सम्बन्धो नाध्यक्ष इत्युक्त्वा ।}अश्रावणः शब्दो नित्यो घट इति तस्माद् विषयभेदोपलक्षणार्थ सहेतुत्वाहेतुत्वदर्शनं । प्रत्यक्षानुमान बाधे सर्व्वाविषये ।\leavevmode\marginnote{\textenglish{92b/MA}} अभ्युपगमप्रतीतिबाधे तु नियतविषये इत्यर्थः ॥ (१३०)
	\pend
      \label{div_pvv.4.131_4.132}\edlabel{div_pvv.4.131_4.132}
	  
	% new div opening: depth here is 2
	

	  \pstart उक्ता प्रतीतिबाधा ॥
	\pend
      \leavevmode\marginnote{\textenglish{458/s}}

	  \begin{center}%% label @type='head'
	\textbf{(७) प्रत्यक्षबाधा}
	\end{center}
	

	  \pstart प्रत्यक्षबाधा वक्तव्या । न केवलं शाब्दप्रसिद्धे व्यवहारधर्मप्रसिद्धौ तत्प्रतिरोद्धा बाध्यते । किन्तु (।)
	\pend
      
	  \bigskip
	  \begingroup
	  \large
	
	    
	    \stanza[\smallbreak]
	\label{pv.4.131a}\edlabel{pv.4.131a}\flagstanza{\tiny\textenglish{...4.131a}}अत्राप्यध्यक्षबाधायां नानारूपतया ध्वनौ ।&प्रसिद्धस्य श्रुतौ;\&[\smallbreak]


	
	  \endgroup
	

	  \pstart {\color{DodgerBlue3}“अत्राप्यध्यक्षबाधायां”} व्यावहारिककल्पनावशात् {\color{DodgerBlue3}“नानारूपतया”} लोके {\color{DodgerBlue3}“प्रसिद्धस्य”} ख्यातस्य {\color{DodgerBlue3}“ध्वनौ श्रुतौ”} श्रवणज्ञाने ॥
	\pend
      
	  \bigskip
	  \begingroup
	  \large
	
	    
	    \stanza[\smallbreak]
	\label{pv.4.131b}\edlabel{pv.4.131b}\flagstanza{\tiny\textenglish{...4.131b}}रूपं यदेव प्रतिभासते ॥ १३१ ॥\&[\smallbreak]


	
	  \endgroup
	
	  \bigskip
	  \begingroup
	  \large
	
	    
	    \stanza[\smallbreak]
	\label{pv.4.132}\edlabel{pv.4.132}\flagstanza{\tiny\textenglish{....4.132}}अद्वयं शबलाभासस्यादृष्टेर्बुद्धिजन्मनः ।&तदर्थार्थोक्तिरस्यैव क्षेपेऽध्यक्षेण बाधनम् ॥ १३२ ॥\&[\smallbreak]


	
	  \endgroup
	

	  \pstart {\color{DodgerBlue3}“यदेव रूपमद्वयं”} धर्मादिद्वयशून्यं {\color{DodgerBlue3}“प्रतिभासते शबलाभासस्य”} नानाकारस्य {\color{DodgerBlue3}“बुद्धिजन्मनोऽदृष्टेः”} । यदि नानाकरता शब्दस्य वास्तवी स्यात् (।) तथैव श्रुतिज्ञाने प्रतिभासेत । {\color{DodgerBlue3}“तदर्था”} तत्प्रतिपादनफलाऽचार्यस्य प्रत्यक्षानुमानार्थप्रसिद्धेन निराकृतइत्यत्रा{\color{DodgerBlue3}“र्थोक्तिरर्थ\edlabel{pvv.458-1}\footnote{\label{pvv.458-1}  १ श्रोत्रशब्दयोर्यः सम्बन्धो ग्राह्यगाहकलक्षणस्तद्धितवाच्यः स श्रावणशब्दस्यार्थः ।}”}ग्रहणमस्याध्यक्ष\edlabel{pvv.458-2}\footnote{\label{pvv.458-2}  २ स्वलक्षणस्यैव ।}सिद्धस्यैव रूपस्य \edlabel{pvv.458-3}\footnote{\label{pvv.458-3}  ३ न ग्राह्यग्राहकत्वप्रतिषेधे सामान्यस्य वा ।} {\color{DodgerBlue3}“क्षेपेऽध्यक्षेण बाधन”}मिष्टं । (१३१,१३२)
	\pend
      
	  
	% new div opening: depth here is 1
	
\section[{४. सामान्यचिन्ता}]{४. सामान्यचिन्ता}

	  \begin{center}%% label @type='head'
	\textbf{(१) सामान्यं व्यावृत्तिलक्षणम्}
	\end{center}
	\label{div_pvv.4.133}\edlabel{div_pvv.4.133}
	  
	% new div opening: depth here is 2
	
	  \bigskip
	  \begingroup
	  \large
	
	    
	    \stanza[\smallbreak]
	\label{pv.4.133}\edlabel{pv.4.133}\flagstanza{\tiny\textenglish{....4.133}}तदेव रूपं तत्रार्थः शेषं व्यावृत्तिलक्षणम् ।&अवस्तुभूतं सामान्यमतस्तन्नाक्षगोचरः ॥ १३३ ॥\&[\smallbreak]


	
	  \endgroup
	

	  \pstart {\color{DodgerBlue3}“तत्र”} श्रुतिज्ञाने {\color{DodgerBlue3}“भास”}मानं {\color{DodgerBlue3}“तद्रूप”}मर्थः स्वलक्षणं । तदतिरिक्तं {\color{DodgerBlue3}“शेषं”} । धर्मिधर्मादि\edlabel{pvv.458-4}\footnote{\label{pvv.458-4}  ४ जात्यादि ।} {\color{DodgerBlue3}“व्यावृत्तिलक्षण”}मन्यव्यवच्छेदस्वभावम{\color{DodgerBlue3}“वस्तु”}भूतं सर्व्वत्र सम्भवात् {\color{DodgerBlue3}“सामा”}न्यञ्च । {\color{DodgerBlue3}“अतोऽ”}वस्तुत्वादेस्तद्गुणजात्या{\color{DodgerBlue3}“दि नाक्षगोचरः”} । (१३३)
	\pend
      \label{div_pvv.4.134}\edlabel{div_pvv.4.134}
	  
	% new div opening: depth here is 2
	\leavevmode\marginnote{\textenglish{459/s}}
	  \bigskip
	  \begingroup
	  \large
	
	    
	    \stanza[\smallbreak]
	\label{pv.4.134}\edlabel{pv.4.134}\flagstanza{\tiny\textenglish{....4.134}}तेन समान्यधर्माणामप्रत्यक्षत्वसिद्धितः ।&प्रतिक्षेपेप्यबाधेति श्रावणोक्त्या प्रकाशितम् ॥ १३४ ॥\&[\smallbreak]


	
	  \endgroup
	

	  \pstart {\color{DodgerBlue3}“तेना”}वस्तुत्वेन कारणेन {\color{DodgerBlue3}“सामान्यधर्माणां”} प्रमेयत्वादीनाम{\color{DodgerBlue3}“प्रत्यक्षत्वस्य सिद्धितः”} । केनचिद्वादिना प्रत्यक्षसिद्धितः {\color{DodgerBlue3}“प्रतिक्षेपेपि”} क्रियमाणे न बाध्यबाधकभाव इत्यश्रावण इत्यत्र {\color{DodgerBlue3}“श्रावणोक्त्या”} निषेध्यदर्शिकया प्रकाशितं स्वलक्षणबाधने प्रत्यक्ष{\color{DodgerBlue3}“बाधेत्यर्थः”} ॥ (१३४)
	\pend
      \label{div_pvv.4.135}\edlabel{div_pvv.4.135}
	  
	% new div opening: depth here is 2
	

	  \pstart \edlabel{pvv.459-1}\footnote{\label{pvv.459-1}  १ इन्द्रियविषयस्वभावःस्वलक्षणं स यदि श्रावणशब्देनाभिमतः ।} एवं तर्हि शब्दस्वलक्षणं नास्तीत्येव कस्मान्नोच्यते । किं श्रावणत्व\edlabel{pvv.459-2}\footnote{\label{pvv.459-2}  २ किं क्रियानिमित्तेनोच्यते ।}मुख्यं निषेध्य युक्तिरित्याह ।
	\pend
      
	  \bigskip
	  \begingroup
	  \large
	
	    
	    \stanza[\smallbreak]
	\label{pv.4.135}\edlabel{pv.4.135}\flagstanza{\tiny\textenglish{....4.135}}सर्व्वथाऽवाच्यरूपत्वात् सिद्ध्या तस्य समाश्रयात् ।&बाधनात् तद्बलेनोक्तः श्रावणेनाक्षगोचरः ॥ १३५ ॥\&[\smallbreak]


	
	  \endgroup
	

	  \pstart स्वलक्षणस्य {\color{DodgerBlue3}“सर्वथा”} केनचिच्छब्देना{\color{DodgerBlue3}“वाच्य”}त्वात्\edlabel{pvv.459-3}\footnote{\label{pvv.459-3}  ३ स्वलक्षणस्य ।} मुख्यमभिधानं नास्त्येव । अथ सामान्यवृत्तिरपि स्वलक्षणशब्दोऽध्यवसायानुरोधात् स्वलक्षणमुपलक्षयति । एवं श्रावणशब्दोपीन्द्रियग्राह्यतोपलक्षणं शब्दस्वलक्षणमुपलक्षयिष्यतीति न कश्चिद्विशेषः । अथवास्त्येव श्रावणशब्देनाभिधाने प्रयोजनमित्याह । श्रोत्रेन्द्रियविषस्य या सिद्धिस्तथा भावः । तया सिद्ध्या {\color{DodgerBlue3}“तस्ये”}न्द्रियज्ञानस्य {\color{DodgerBlue3}“समाश्रया”}च्छब्दस्य स्वरूपव्यवस्थित्या हेतुना तेन व्यपदेश इन्द्रियज्ञानबलेन शब्दस्वरूपपावस्थितः (।) तथाभिधानमित्यर्थः । किञ्च तस्येन्द्रियज्ञानस्य {\color{DodgerBlue3}“बलेन”} शब्दस्वलक्षण{\color{DodgerBlue3}“विपर्ययभावस्य\edlabel{pvv.459-4}\footnote{\label{pvv.459-4}  ४ तथाभूतार्थप्रतिषेधकस्य पुरुषस्य ।}”} {\color{DodgerBlue3}“बाधानात्”} कारणात् {\color{DodgerBlue3}“श्रावणेनाक्षगोचरः”} स्वलक्षण{\color{DodgerBlue3}“मुक्तः”} । नित्यो घट इत्यनुमाने नित्यत्वविषयेण कृतकत्वलिङ्गभुवा बाधितः पक्षः । स चासकृद्दर्शित एवेति नेह विपञ्चितः ॥ (१३५)
	\pend
      \label{div_pvv.4.136}\edlabel{div_pvv.4.136}
	  
	% new div opening: depth here is 2
	

	  \begin{center}%% label @type='head'
	\textbf{(२) स्वधर्मिग्रहणप्रयोजनम्}
	\end{center}
	

	  \pstart इदानीं प्रत्यक्षानुमानाप्तप्रसिद्धेन स्व\edlabel{pvv.459-5}\footnote{\label{pvv.459-5}  ५ वादिनेष्टस्य स्वस्य धर्मी स्वधर्मी तत्र ।}धर्मिणीति स्वधर्मिग्रहणस्य {\color{DodgerBlue3}“साफल्य”}माख्यातुमाह ।
	\pend
      
	  \bigskip
	  \begingroup
	  \large
	
	    
	    \stanza[\smallbreak]
	\label{pv.4.136}\edlabel{pv.4.136}\flagstanza{\tiny\textenglish{....4.136}}सर्व्वत्र वादिनो धर्मो यः स्वसाध्यतयोप्सितः ।&तद्धर्मवति बाधा स्यान्नान्यधर्मेण धर्मिणि ॥ १३६ ॥\&[\smallbreak]


	
	  \endgroup
	\leavevmode\marginnote{\textenglish{460/s}}

	  \pstart {\color{DodgerBlue3}“सर्व्वत्र”} वादकाले साध्यकाले वा {\color{DodgerBlue3}“वादिनः स्वसाध्यतया यो धर्म ईप्सितः तद्धर्भवति”} धर्मिणि {\color{DodgerBlue3}“बाधा स्यात्”} । यथा श्रावणत्ववति शब्दे बाधिते बाधा पक्षस्य । {\color{DodgerBlue3}“न”} तु वादीष्टाद् धर्माद{\color{DodgerBlue3}“न्येन धर्मेण”} धर्मवति {\color{DodgerBlue3}“धर्मिणि”} बाधिते बाधा पक्षस्य स्यादिति धर्मिग्रहप्रयोजनं यथाकाशगुणत्ववति शब्दे बाधितेनं पक्षबाधा । (१३६)
	\pend
      \label{div_pvv.4.137}\edlabel{div_pvv.4.137}
	  
	% new div opening: depth here is 2
	
	  \bigskip
	  \begingroup
	  \large
	
	    
	    \stanza[\smallbreak]
	\label{pv.4.137}\edlabel{pv.4.137}\flagstanza{\tiny\textenglish{....4.137}}अन्यथास्योपरोधः को बाधितेन्यत्र धर्मिणि ।&गतार्थे लक्षणेनास्मिन् स्वधर्मिवचनं पुनः ॥ १३७ ॥\&[\smallbreak]


	
	  \endgroup
	

	  \pstart {\color{DodgerBlue3}“अन्यथा”} यद्येवं नेष्यते त{\color{DodgerBlue3}“दान्यत्र”} धर्मे {\color{DodgerBlue3}“धर्मिणि बाधितेऽस्य”} प्रकृतधर्मविशिष्टस्य धर्मिणः क {\color{DodgerBlue3}“उपरोधो”} बाध\edlabel{pvv.460-1}\footnote{\label{pvv.460-1}  १ येन तदाशङ्कानिवृत्त्यर्थः स्यात् ।} ॥ ननु स्वरूपेणैव निर्देश्यः स्वयमिष्टोऽनिराकृतः पक्ष इति पक्षलक्षणेनैवानिष्टधर्मवतो धर्मिणो बाधा न पक्षबाधेति लभ्यत \leavevmode\marginnote{\textenglish{93a/MA}} एवेत्याह । पक्षस्य {\color{DodgerBlue3}“लक्षणेनास्मिन्”} वादीष्टधर्मवति धर्मिणि बाध्यत्वेन {\color{DodgerBlue3}“गतार्थे पुनः”} {\color{DodgerBlue3}“स्वधर्मिवचनं”} यत्कृतं । (१३७)
	\pend
      \label{div_pvv.4.138}\edlabel{div_pvv.4.138}
	  
	% new div opening: depth here is 2
	
	  \bigskip
	  \begingroup
	  \large
	
	    
	    \stanza[\smallbreak]
	\label{pv.4.138}\edlabel{pv.4.138}\flagstanza{\tiny\textenglish{....4.138}}बाधायां धर्मिणोपि स्यात् बाधेत्यस्य प्रसिद्धये ।&आश्रयस्य विरोधेन तदाश्रितविरोधनात् ॥ १३८ ॥\&[\smallbreak]


	
	  \endgroup
	

	  \pstart {\color{DodgerBlue3}“धर्मिणोपि\edlabel{pvv.460-2}\footnote{\label{pvv.460-2}  २ तन्न साध्यस्यैव बाधायां बाधा किन्तु धर्मिणोपि ।} बाधायां”} धर्मस्य बाधने पक्ष{\color{DodgerBlue3}“बाधा”} यथा {\color{DodgerBlue3}“स्यादित्यस्या”}र्थस्य {\color{DodgerBlue3}“प्रसिद्धये”} क्वचिदा{\color{DodgerBlue3}“श्रयस्य”} धर्मिणो {\color{DodgerBlue3}“विरोधेन”} प्रतिक्षेपेण तदा{\color{DodgerBlue3}“श्रितस्य”} धर्मस्य {\color{DodgerBlue3}“विरोधनात्”} धर्मद्वारेण धर्मिद्वारेण वा समुदायबाधायां पक्षबाधेत्यर्थः । (१३८)
	\pend
      \label{div_pvv.4.139}\edlabel{div_pvv.4.139}
	  
	% new div opening: depth here is 2
	
	  \bigskip
	  \begingroup
	  \large
	
	    
	    \stanza[\smallbreak]
	\label{pv.4.139}\edlabel{pv.4.139}\flagstanza{\tiny\textenglish{....4.139}}अन्यथैवंविधो धर्मः साध्य इत्यभिधानतः ।&तद्बाधामेव मन्येत स्वधर्मिग्रहणन्ततः ॥ १३९ ॥\&[\smallbreak]


	
	  \endgroup
	

	  \pstart {\color{DodgerBlue3}“अन्यथा”} धर्मिद्वारेण समुदायबाधायाः संग्रहार्थं स्वधर्मिग्रहणं यदि न क्रियते {\color{DodgerBlue3}“तदैवम्विधः”} साध्यत्वेनैवेष्टो {\color{DodgerBlue3}“धर्मः”} साध्य {\color{DodgerBlue3}“इत्यभिधानतः”} । {\color{DodgerBlue3}“त”}स्यं धर्ममात्रस्यैव {\color{DodgerBlue3}“बाधां मन्येत”} प्रतिपत्ता न तु धर्मिबाधामपि । ततः (।) उभयसंग्रहार्थं {\color{DodgerBlue3}“स्वधर्मि”}ग्रहणमा चा र्य स्य ॥ (१३९)
	\pend
      \label{div_pvv.4.140}\edlabel{div_pvv.4.140}
	  
	% new div opening: depth here is 2
	
	  \bigskip
	  \begingroup
	  \large
	
	    
	    \stanza[\smallbreak]
	\label{pv.4.140}\edlabel{pv.4.140}\flagstanza{\tiny\textenglish{....4.140}}नन्वेतदप्यर्थसिद्धं सत्यं केचित्तु धर्मिणः ।&केवलस्योपरोधेपि दोषवत्तामुपागताः ॥ १४० ॥\&[\smallbreak]


	
	  \endgroup
	\leavevmode\marginnote{\textenglish{461/s}}

	  \pstart {\color{DodgerBlue3}“नन्वे”}तदुभयसंग्रहण{\color{DodgerBlue3}“मप्यर्थतः”} \edlabel{pvv.461-1}\footnote{\label{pvv.461-1}  १ साक्षाद् धर्मिद्वारेण चेत्यविशेषात् ।} सामर्थ्यंतः {\color{DodgerBlue3}“सिद्धं”} । न हि केवलो धर्मोस्ति क्वचित् । तद्वाधने तद्विशिष्टस्य {\color{DodgerBlue3}“धर्मिणो”}पि बाधनात् समुदायबाधैवैषितव्या । सा च धर्मिद्वारेण वा भवतु धर्मद्वारेण वा न कश्चिद् विशेष इत्यर्थः ।
	\pend
      

	  \pstart अत्राह सत्यमेतत् {\color{DodgerBlue3}“केचित्तु”} वादिनो \edlabel{pvv.461-2}\footnote{\label{pvv.461-2}  २ यत्र धर्मिबाधेपि साध्यस्य न क्षतिः तत्रापि दोषः ।} {\color{DodgerBlue3}“धर्मिणोः केवलस्योपरोधे साध्य-”} धर्मस्याबाधायामपि पक्षस्य {\color{DodgerBlue3}“दोषवत्तामुपागताः”} प्रतिपन्नाः । (१४०)
	\pend
      \label{div_pvv.4.141}\edlabel{div_pvv.4.141}
	  
	% new div opening: depth here is 2
	
	  \bigskip
	  \begingroup
	  \large
	
	    
	    \stanza[\smallbreak]
	\label{pv.1.141}\edlabel{pv.1.141}\flagstanza{\tiny\textenglish{....1.141}}यथा परैरनुत्पाद्यापूर्व्वरूपन्न खादिकम् ।&सकृच्छब्दाद्यहेतुत्वादित्युक्ते प्राह दूषकः ॥ १४१ ॥\&[\smallbreak]


	
	  \endgroup
	

	  \pstart {\color{DodgerBlue3}“यथा परैः”} सहकारि\edlabel{pvv.461-3}\footnote{\label{pvv.461-3}  ३ वै शे षि (कं) प्रति सौ त्रा न्ति केन ।}भिर{\color{DodgerBlue3}“नुत्पाद्यापूर्व्वरूपं\edlabel{pvv.461-4}\footnote{\label{pvv.461-4}  ४ न क्षतस्वभावमिति साध्यं धर्मी ।} न खादिक”}माकाशदिक्कालादि न भवति । अपि तूत्पाद्यपूर्व्वरूपमेव भवति । सकृदेककालन्तदुत्पाद्यस्य\edlabel{pvv.461-5}\footnote{\label{pvv.461-5}  ५ शब्दहेतुराकाशं तद् यदि नित्य एकदोत्पत्तिप्रसङ्गः सदा सन्निधानात् हेतोः ।} कार्यकालपस्य {\color{DodgerBlue3}“शब्दादेरहेतुत्वाद”}विशिष्टैकरूपात् कारणात् सकृत् सर्व्वकार्योत्पत्तिप्रसङ्गा{\color{DodgerBlue3}“दिति”} वादि{\color{DodgerBlue3}“नोक्ते”} {\color{DodgerBlue3}“दूषकः”} प्रतिवाद्याह\edlabel{pvv.461-6}\footnote{\label{pvv.461-6}  ६ विरुद्धतां ।}। (१४१)
	\pend
      \label{div_pvv.4.142}\edlabel{div_pvv.4.142}
	  
	% new div opening: depth here is 2
	
	  \bigskip
	  \begingroup
	  \large
	
	    
	    \stanza[\smallbreak]
	\label{pv.4.142}\edlabel{pv.4.142}\flagstanza{\tiny\textenglish{....4.142}}तद्वद् वस्तुस्वभावोऽसन् धर्मी व्योमादिरित्यपि ।&नैवमिष्टस्य साध्यस्य बाधा काचन विद्यते ॥ १४२ ॥\&[\smallbreak]


	
	  \endgroup
	

	  \pstart {\color{DodgerBlue3}“तद्वद्य”}थानुत्पाद्यापूर्व्वरूप आकाशादिर्न भवति ।\edlabel{pvv.461-7}\footnote{\label{pvv.461-7}  ७ वस्तुस्वभाव आकाशादिर्द्धर्मो न वेति विवक्षामनङ्गीकृत्याकाशसत्त्ववादिना सौ त्रा न्ति के न सामान्येन प्रकृते आकाशादौ स्थिररूपत्वधर्मव्यवच्छेदमात्रे साध्ये युगपद्धेतुत्वव्यवच्छेदरूपे हेतौ यद्याह परः वस्तुभूताकाशाभावं । तादृशे निराकृतेपि नैवेष्टस्य व्यवच्छेदमात्रस्य बाधा प्रज्ञप्तिमति धर्मिणि व्यवच्छेद मात्रस्याव्याघातात् ।} तथा वस्तु{\color{DodgerBlue3}“स्वभावो धर्मी व्योमादिरसन्नित्यपि”} स्यात् । अर्थक्रियाऽसमर्थस्य वस्तुत्वाभावादिति धर्मिणः केवलस्य बाधनं न धर्मस्य । {\color{DodgerBlue3}“एवं”} धर्मिबाधनेपी{\color{DodgerBlue3}“ष्टस्य साध्यस्य काचन बाधा न विद्यते”} । हेतोर्व्वाऽसिद्धिः । असत्यपि कार्यानुत्पादस्य व्यवच्छेदस्य सिद्धेः । (१४२)
	\pend
      \label{div_pvv.4.143}\edlabel{div_pvv.4.143}
	  
	% new div opening: depth here is 2
	

	  \pstart ततः (।)
	\pend
      \leavevmode\marginnote{\textenglish{462/s}}
	  \bigskip
	  \begingroup
	  \large
	
	    
	    \stanza[\smallbreak]
	\label{pv.4.143}\edlabel{pv.4.143}\flagstanza{\tiny\textenglish{....4.143}}द्वयस्यापि हि साध्यत्वे साध्यधर्मोपरोधि यत् ।&बाधनं धर्मिणस्तत्र बाधेत्येतेन वर्ण्णितम् ॥ १४३ ॥\&[\smallbreak]


	
	  \endgroup
	

	  \pstart {\color{DodgerBlue3}“द्वयस्य”} धर्मिधर्मसमुदायस्यापि हि {\color{DodgerBlue3}“साध्यत्वे”} सति यत्र {\color{DodgerBlue3}“धर्मिणो बाधनं साध्यधर्मोपरोधि”} तत्र पक्ष{\color{DodgerBlue3}“बाधा”} युक्तेत्ये{\color{DodgerBlue3}“तेन”} स्वधर्मिग्रहणेन {\color{DodgerBlue3}“वर्ण्णितं”} । न हि साध्यधर्ममात्रसाधनार्थं कश्चित् साधनमन्वेषते । तस्य जगति क्वचित् सत्तायां विवादाभावात् । तथा निश्चये प्रवृत्त्ययोगाच्च । किन्तु धर्मिविशेषनिष्ठं साध्यमिष्टं ॥ (१४३)
	\pend
      \label{div_pvv.4.144_4.145}\edlabel{div_pvv.4.144_4.145}
	  
	% new div opening: depth here is 2
	
	  \bigskip
	  \begingroup
	  \large
	
	    
	    \stanza[\smallbreak]
	\label{pv.4.144a}\edlabel{pv.4.144a}\flagstanza{\tiny\textenglish{...4.144a}}तथैव धर्मिणोप्यत्र साध्यत्वात् केवलस्य न ।&यद्येवमत्र बाधा स्यात्;\&[\smallbreak]


	
	  \endgroup
	

	  \pstart तत्र यथा धर्मिविशिष्टस्य साध्यत्वं धर्मस्य {\color{DodgerBlue3}“तथैव”} साध्यत्वविशिष्टत्वेन {\color{DodgerBlue3}“धर्मिणोपि साध्यत्वात् केवलस्य”} धर्मस्य न क्वचित् साध्यता । ततो न केवलस्य धर्मिणो बाधने पक्षबाधा ॥ {\color{DodgerBlue3}“यदि”} साध्यधर्मोपरोधिनि धर्मिणि बाधिते पक्षबाधेष्यते (।) {\color{DodgerBlue3}“एवं”} सत्यत्र हेतौ सां ख्यं प्रति बौद्धेनोक्ते धर्मिबाधाद्वारेण धर्म{\color{DodgerBlue3}“बाधा स्यात्”} ।
	\pend
      
	  \bigskip
	  \begingroup
	  \large
	
	    
	    \stanza[\smallbreak]
	\label{pv.4.144b}\edlabel{pv.4.144b}\flagstanza{\tiny\textenglish{...4.144b}}नान्यानुत्पाद्यशक्तिकः ॥ १४४ ॥\&[\smallbreak]


	
	  \endgroup
	
	  \bigskip
	  \begingroup
	  \large
	
	    
	    \stanza[\smallbreak]
	\label{pv.4.145}\edlabel{pv.4.145}\flagstanza{\tiny\textenglish{....4.145}}सकृच्छब्दाद्यहेतुत्वात् सुखादिरिति पूर्व्ववत् ।&विरोधिता भवेदत्र हेतुरैकान्तिको यदि ॥ १४५ ॥\&[\smallbreak]


	
	  \endgroup
	

	  \pstart तद्यथा {\color{DodgerBlue3}“सुखादिः”} सुखदुःखमोहात्मकं {\color{DodgerBlue3}“प्रधानं नान्येन”} सहाकारिणाऽ{\color{DodgerBlue3}“नुत्पाद्यशक्तिकोऽ”}नाधेयसामर्थ्यः {\color{DodgerBlue3}“सकृच्छब्दा”}दीनां कार्याणामहे{\color{DodgerBlue3}“हेतुत्वा”}दिति । {\color{DodgerBlue3}“अत्र पूर्व्ववत्”} परैरनुत्पाद्येत्यादिप्रयोगत्वाच्च । सुखाद्यात्मकस्य नित्यत्वस्य बाधनात् सुखादौ धर्म्मिणि बाधिते तद्धर्म्मस्य नित्यत्वस्य विरोधने विपर्ययसाधने बाधा स्यात् ।\edlabel{pvv.462-1}\footnote{\label{pvv.462-1}  १ सामर्थ्यादन्याभावात् ।} अनित्यस्वभावो हि सुखादिः साधयितुमिष्टः । सुखादिस्वभावभूतनित्यत्वबाधने च सुखादिरेव बाधित इति धर्मोपरोधिनि धर्मिणि बाधिते पक्षबाधा स्यात् । अत्राह । {\color{DodgerBlue3}“भवेदत्र”} हेतौ पक्षबाधा । {\color{DodgerBlue3}“यदि”} न सकृच्छब्दाद्य\edlabel{pvv.462-2}\footnote{\label{pvv.462-2}  २ अादिना स्पर्शरूपरसगन्धग्रहः ।}नुत्पादादिति\leavevmode\marginnote{\textenglish{93b/MA}} {\color{DodgerBlue3}“हेतुः”} साध्यस्य । वस्तुभूतसुखाद्यनित्यत्वस्य । विपर्यये सुखादिधर्म्यभावादनि- त्यत्वाभावे{\color{DodgerBlue3}“नैकान्तिको”}ऽव्यभिचारी भवेत् । (१४४,१४५)
	\pend
      \label{div_pvv.4.146}\edlabel{div_pvv.4.146}
	  
	% new div opening: depth here is 2
	

	  \pstart यावता या च (।)
	\pend
      \leavevmode\marginnote{\textenglish{463/s}}
	  \bigskip
	  \begingroup
	  \large
	
	    
	    \stanza[\smallbreak]
	\label{pv.4.146}\edlabel{pv.4.146}\flagstanza{\tiny\textenglish{....4.146}}क्रमक्रियाऽनित्यतयोरविरोधाद् विपक्षत्तः ।&व्यावृत्तेः संशयान्नायं शेषवद् भेद इष्यते ॥ १४६ ॥\&[\smallbreak]


	
	  \endgroup
	

	  \pstart {\color{DodgerBlue3}“क्रमक्रिया”} हेतुः । या च साध्य{\color{DodgerBlue3}“ऽनित्यता तयोरविरोधात्”} । अव{\color{DodgerBlue3}“स्तुभूत”} ({\color{DodgerBlue3}“धर्मि”}) सुखादिधर्मानित्यत्वे विपरीते साध्ये वस्तुभूतसुखादिधर्मानित्यत्वं {\color{DodgerBlue3}“विपक्षस्ततो”} हेतोः क्रमकरण{\color{DodgerBlue3}“व्यावृत्तेः संशयान्नायं”} विरुद्धो \edlabel{pvv.463-1}\footnote{\label{pvv.463-1}  १ धर्मोपरोधाद् विरुद्धः सांख्यस्याचार्यविरुद्धत्वं नेह ।} हेतुः । {\color{DodgerBlue3}“शेषवद् भेदोऽ”}नैकान्तिकविशेष{\color{DodgerBlue3}“स्त्विष्यते”} । सति च विरुद्धत्वे धर्मिबाधाद्वारेण धर्मबाधा स्यात् । (१४६)
	\pend
      \label{div_pvv.4.147}\edlabel{div_pvv.4.147}
	  
	% new div opening: depth here is 2
	

	  \begin{center}%% label @type='head'
	\textbf{(३) धर्मिस्वरूपनिरासः}
	\end{center}
	
	  \bigskip
	  \begingroup
	  \large
	
	    
	    \stanza[\smallbreak]
	\label{pv.4.147}\edlabel{pv.4.147}\flagstanza{\tiny\textenglish{....4.147}}स्वयमिष्टो यतो धर्मः साध्यस्तस्मात् तदाश्रयः ।&बाध्यो न केवलो नान्यसंश्रयो वेति सूचितम् ॥ १४७ ॥\&[\smallbreak]


	
	  \endgroup
	

	  \pstart {\color{DodgerBlue3}“यतः”} कारणात् {\color{DodgerBlue3}“स्वय”}म्वादि{\color{DodgerBlue3}“नेष्टो धर्मः साध्यस्तस्मात्”} साध्यधर्मसा{\color{DodgerBlue3}“श्रयः”} यः स एव बाध्यः {\color{DodgerBlue3}“केवलो न बाध्यः”} । यथा वस्तुभूताकाशबाधायामपि नित्यैकरूपत्वाभावस्य साध्यधर्मस्य न क्षतिः । साध्यधर्मा{\color{DodgerBlue3}“दन्य”}स्य च धर्मस्या{\color{DodgerBlue3}“श्रयो न बाध्य इति”} स्वयंशब्देन {\color{DodgerBlue3}“सूचितं”} । यथाऽनित्यत्वे साध्ये शब्दे आकाशगुणत्वा{\color{DodgerBlue3}“श्रयत्वेन”} बाधायामपि न दोषः । (१४७)
	\pend
      \label{div_pvv.4.148}\edlabel{div_pvv.4.148}
	  
	% new div opening: depth here is 2
	

	  \pstart तस्मात् (।)
	\pend
      
	  \bigskip
	  \begingroup
	  \large
	
	    
	    \stanza[\smallbreak]
	\label{pv.4.148}\edlabel{pv.4.148}\flagstanza{\tiny\textenglish{....4.148}}स्वयंश्रुत्यान्यधर्माणां बाधाऽबाधेति कथ्यते ।&तथा स्वधर्मिणान्यस्य धर्मिणोपीति कथ्यते ॥ १४८ ॥\&[\smallbreak]


	
	  \endgroup
	

	  \pstart {\color{DodgerBlue3}“स्वयंश्रुत्या”} साध्याद् धर्मादन्येषां {\color{DodgerBlue3}“धर्माणां”} यथा {\color{DodgerBlue3}“बाधा”} या सा ऽ{\color{DodgerBlue3}“बाधेति कथ्यते । तथा स्वधर्मिणा”} स्वधर्मिवचनेन साध्या{\color{DodgerBlue3}“दन्यस्य”} धर्मस्य {\color{DodgerBlue3}“धर्मिणो”} बाधाऽ{\color{DodgerBlue3}“बाधेति”} कथ्यते ॥ (१४८)
	\pend
      
	  
	% new div opening: depth here is 1
	
\section[{५---पक्षदोषाः}]{५---पक्षदोषाः}

	  \begin{center}%% label @type='head'
	\textbf{(१) हेतुनिरपेक्षः पक्षदोषः}
	\end{center}
	\label{div_pvv.4.149}\edlabel{div_pvv.4.149}
	  
	% new div opening: depth here is 2
	

	  \pstart तथाऽपरेपि पक्षाभासाः सन्ति ते कस्मान्नोच्यन्ते । तथा ह्यप्रसिद्धविशेष्यः कश्चित् पक्षो यथा\edlabel{pvv.463-2}\footnote{\label{pvv.463-2}  २ वैशेषिकस्य ।} विभुरात्मा । \edlabel{pvv.463-3}\footnote{\label{pvv.463-3}  ३ सिद्धान्ती सति धर्मिणि धर्माश्चिन्त्यन्ते ।} आत्मन एव बौद्ध स्यासिद्धत्वात् । कश्चिद्\leavevmode\marginnote{\textenglish{464/s}} -प्रसिद्धविशेषणः । यथा विनाशी शब्दः ॥ सां ख्यं प्रमाणं\edlabel{pvv.464-1}\footnote{\label{pvv.464-1}  १ बौद्धस्य ।} प्रति तस्य विनाशासिद्धेः । कश्चिदप्रसिद्धोभयः (।) यथा समवायिकारणमात्मा । बौद्धस्योभयासिद्धेरिति ॥
	\pend
      

	  \pstart नन्वसिद्धोप्यात्मा पक्षी भविष्यति तल्लक्षणयोगात् (।) नह्यसिद्धस्य पक्षता निरस्ता । आश्रयासिद्धत्वाद्धेतोर्न पक्ष इति चेत् । न ।\edlabel{pvv.464-2}\footnote{\label{pvv.464-2}  २ अत्राह सिद्धान्ती ।}
	\pend
      

	  \pstart साधनदोषोयं न पक्षदोषः । तथा शब्देऽनित्यत्वं विषेशणमसिद्धमिति गुण एवायं न पक्षदोषः । असिद्धस्यैव साध्यत्वात् । अथ विपर्ययसिद्ध्याऽसिद्धमुच्यते । तथापि मूढस्य विपर्ययसिद्धावपि नायं पक्षदोषः । विरोधो नाम हेतुदोष ए- वायं । प्रमाणेन च विपर्ययसिद्धौ प्रमाणबाधितत्वमेव पक्षदोषोस्तु । अलमप्रसिद्धविशेषणत्वाभिधानेन । अप्रसिद्धोभयस्य तूभयदोषाच्च सर्व्वेऽमी हेतुदोषा एवेति किं पक्षदोषा वक्तव्याः । अथ (।)
	\pend
      
	  \bigskip
	  \begingroup
	  \large
	
	    
	    \stanza[\smallbreak]
	\label{pv.4.149}\edlabel{pv.4.149}\flagstanza{\tiny\textenglish{....4.149}}सर्वसाधनदोषेण पक्ष एवोपरुध्यते ।&तथापि पक्षदोषत्वं प्रतिज्ञामात्रभाविनः ॥ १४९ ॥\&[\smallbreak]


	
	  \endgroup
	

	  \pstart {\color{DodgerBlue3}“सर्व्वेण साधन”}स्य {\color{DodgerBlue3}“दोषे”}णासिद्धत्वादिना {\color{DodgerBlue3}“पक्ष एवोपरुध्यते”} तेन पक्षदोषा असिद्धविशेष्यादयोऽभिधीयन्ते । यद्यपि पक्षोपरोधफलाः सर्व्वे दोषास्त{\color{DodgerBlue3}“थापि प्रतिज्ञामा”}त्रेण \edlabel{pvv.464-3}\footnote{\label{pvv.464-3}  ३ नोत्तरत्वेत्यादिदोषेण ।} भवनशीलस्य दोषस्य {\color{DodgerBlue3}“पक्षवोषत्वमु”}क्तमिष्टं । (१४९)
	\pend
      \label{div_pvv.4.150}\edlabel{div_pvv.4.150}
	  
	% new div opening: depth here is 2
	

	  \pstart यस्मात् साक्षात् (।)
	\pend
      
	  \bigskip
	  \begingroup
	  \large
	
	    
	    \stanza[\smallbreak]
	\label{pv.4.150}\edlabel{pv.4.150}\flagstanza{\tiny\textenglish{....4.150}}उत्तरावयवापेक्षो यो दोषः सोनुबध्यते ।&तेनेयुक्तमतोऽपक्षदोषोऽसिद्धाश्रयादिकः ॥ १५० ॥\&[\smallbreak]


	
	  \endgroup
	

	  \pstart {\color{DodgerBlue3}“उत्तरोऽवयवो”} हेतुदृष्टान्तादिस्तद{\color{DodgerBlue3}“पेक्षो यो दोषः स तेन”} हेत्वादिनानु{\color{DodgerBlue3}“बध्यते”} । आत्मनि सम्बध्यते {\color{DodgerBlue3}“इत्युक्तं”} प्राक् । उत्तरावयवापेक्षो न दोषः पक्ष इष्यत इत्या{\color{DodgerBlue3}“दिना । अतोऽसिद्धाश्रयादिक”} आश्रयासिद्धत्वादिरुत्तरावयवापेक्षो {\color{DodgerBlue3}“न पक्षदोषो मतः”} ॥ (१५०)
	\pend
      \label{div_pvv.4.151}\edlabel{div_pvv.4.151}
	  
	% new div opening: depth here is 2
	

	  \pstart नन्वश्रावणः शब्दो नित्यो घटः । नानुमानं प्रमाणं  (।) अचन्द्रः शशीत्युदाहरणैरेभिर्द्धर्मस्वरूप\edlabel{pvv.464-4}\footnote{\label{pvv.464-4}  ४ नित्यो घट इति प्रतिज्ञायां नित्यत्वस्य बाधितत्वात् ।}निराकरणेन बाधा दर्शिता यथा-प्रतिज्ञातधर्ममात्रस्य विपरीतधर्मोपस्थापनेन निराकरणात् । धर्मिविशेषस्य धर्मविशेषस्य धर्मिस्वरूपस्य च बाधनेन पक्षबाधास्ति \edlabel{pvv.464-5}\footnote{\label{pvv.464-5}  ५ अव्याख्याता स मु च्च ये ।} सा कथमवगन्तव्येत्याह ।
	\pend
      \leavevmode\marginnote{\textenglish{465/s}}
	  \bigskip
	  \begingroup
	  \large
	
	    
	    \stanza[\smallbreak]
	\label{pv.4.151}\edlabel{pv.4.151}\flagstanza{\tiny\textenglish{....4.151}}धर्मिधर्मविशेषाणां स्वरूपस्य च धर्म्मिणः ।&बाधा साध्याङ्गभूतानामनेनैवोपदर्शिता ॥ १५१ ॥\&[\smallbreak]


	
	  \endgroup
	

	  \pstart {\color{DodgerBlue3}“धर्मिधर्म”}यो{\color{DodgerBlue3}“र्व्विशेषाणां”} व्यक्तिभेदापेक्षया बहुवचनं । {\color{DodgerBlue3}“धर्मिणः स्वरूपस्य च”} सर्व्वेषामेषां {\color{DodgerBlue3}“साध्यं”} प्रत्य{\color{DodgerBlue3}“ङ्गभूतानां बाधा । अनेनैव”} धर्मस्वरूपनिराकरणपरेणो-\leavevmode\marginnote{\textenglish{94a/MA}} दाहरणेन साध्यते । पक्षलक्षणत्वाद्वाधो{\color{DodgerBlue3}“पदर्शिता”} (।) (१५१)
	\pend
      \label{div_pvv.4.152}\edlabel{div_pvv.4.152}
	  
	% new div opening: depth here is 2
	

	  \begin{center}%% label @type='head'
	\textbf{(२) वा/?/क्यविनिरास}
	\end{center}
	
	  \bigskip
	  \begingroup
	  \large
	
	    
	    \stanza[\smallbreak]
	\label{pv.4.152}\edlabel{pv.4.152}\flagstanza{\tiny\textenglish{....4.152}}तत्रोदाहृतिदिङ् मात्रमुच्यतेऽर्थस्य दृष्टये ।&द्रव्यलक्षणयुक्तोन्यः संयोगेर्थोस्ति दृष्टिभाक् ॥ १५२ ॥\&[\smallbreak]


	
	  \endgroup
	

	  \pstart \edlabel{pvv.465-1}\footnote{\label{pvv.465-1}  १ यद्येवमिदमेवोदाहरणमस्तु किमर्थं नान्याऽव(य)व्यवयवेभ्य इत्यादिकमा चा र्ये णो क्तमित्याह ।} {\color{DodgerBlue3}“तत्रा”}श्रावणः शब्द इत्यादिषू{\color{DodgerBlue3}“दाहरणदिङमात्रमुच्यते”} । साध्याङगभू\edlabel{pvv.465-2}\footnote{\label{pvv.465-2}  २ नान्तरीयकस्य ।}तस्य सर्व्वस्यैव बाधा भवतीत्य{\color{DodgerBlue3}“र्थस्य दृष्टये”} दर्शनार्थ । तत्र परस्यावयवेभ्योऽवयविनो गुरुत्वादिगुणयोगिनोऽन्यत्वेऽभिमते यदोच्यते नान्योऽवयव्यवयवेभ्यस्तुलानतिविशेषाग्रहणादिति (।) एतद्धर्मविशेषनिराकरणेनोदाहरणं बोद्धव्यं । तथाहि नात्रान्यत्वमात्रं निषेद्ध्ुमिष्टं । तथात्वे धर्मस्वरूपनिराकरणोदाहरणमेतत् स्यात् । तस्मादन्यत्वस्य साध्यधर्मस्य नान्तरीयका गौरवादयो विशेषा निराकर्त्तुमिष्टाः । तथा च धर्मविशेषोदाहरणमेव तत् । तथाहि परैरेकस्यावयवस्यान्यान्यावयव{\color{DodgerBlue3}“संयोगे”} सति तदन्योऽर्थोऽवयविसंज्ञितो {\color{DodgerBlue3}“द्रव्यस्य लक्ष”}णेन क्रियावद् {\color{DodgerBlue3}“गुणवत्समवायिकारणञ्चे”} त्यनेन युक्तो {\color{DodgerBlue3}“दृष्टिभाक्”} दृश्योस्तीतीष्टमेव । (१५२)
	\pend
      \label{div_pvv.4.153}\edlabel{div_pvv.4.153}
	  
	% new div opening: depth here is 2
	

	  \pstart अन्यथा (।)
	\pend
      
	  \bigskip
	  \begingroup
	  \large
	
	    
	    \stanza[\smallbreak]
	\label{pv.4.153}\edlabel{pv.4.153}\flagstanza{\tiny\textenglish{....4.153}}अदृश्यस्या विशिष्टस्य प्रतिज्ञा निष्प्रयोजना ।&इष्टो ह्यवयवी कार्यं दृष्ट्वाऽदृश्येष्वसम्भवि ॥ १५३ ॥\&[\smallbreak]


	
	  \endgroup
	

	  \pstart {\color{DodgerBlue3}“अदृश्यस्य”} संव्यवहाराविषयस्य द्रव्यलक्षणेन {\color{DodgerBlue3}“विशिष्ट”}सामान्यमात्रस्यास्तित्व{\color{DodgerBlue3}“प्रतिज्ञा निष्प्रयोजना”} भवेत् । तथा हि परमाणुष्वदृश्येष्बदर्शनावरणप्रतिघातादि{\color{DodgerBlue3}“कार्यमसंभवि”} दृष्ट्वाऽ{\color{DodgerBlue3}“वयवी चेष्टः\edlabel{pvv.465-3}\footnote{\label{pvv.465-3}  ३ वैशेषिकेण ।}”} । तस्मिन् दर्शनादिकार्ययोगात् । (१५३)
	\pend
      \label{div_pvv.4.154}\edlabel{div_pvv.4.154}
	  
	% new div opening: depth here is 2
	
	  \bigskip
	  \begingroup
	  \large
	
	    
	    \stanza[\smallbreak]
	\label{pv.4.154}\edlabel{pv.4.154}\flagstanza{\tiny\textenglish{....4.154}}अविशिष्टस्य चान्यस्य साधने सिद्धसाधनम् ।&गुरुत्वाधोगती स्यातां यद्यस्य स्यात् तुलानतिः ॥ १५४ ॥\&[\smallbreak]


	
	  \endgroup
	\leavevmode\marginnote{\textenglish{466/s}}

	  \pstart यदि पुनरनिन्द्रियग्राह्यत्वं द्रव्यलक्षणेनाविशिष्टन्तदवयवि द्रव्यं साध्यते तदा{\color{DodgerBlue3}“ऽविशिष्ट”}स्यान्यस्य च {\color{DodgerBlue3}“साधने”} बौद्धस्य न काचित् क्षतिरिति {\color{DodgerBlue3}“सिद्धसाधनं”} स्यात् । तस्माद् दृश्यो द्रव्यगुणवान् भावोऽवयवीति साध्यं । ततश्चास्यावयविनो {\color{DodgerBlue3}“गुरुत्वं गुणोऽधोगतिश्च कर्म यदि स्यातां”} तदा मृदादिखण्डयोः सहतोलितयोर्यावती {\color{DodgerBlue3}“तुलानति”}गौरववशाद् दृष्टा ततोधिका तुलानतिः स्यात् । यदा तयोर्मृदादिखण्डयोः संयोगे सति द्रव्यान्तरमुत्पद्य ते तदा तयोः पूर्व्वावस्थितयोः पूर्व्वावस्थितं गौरवं तदोत्पन्नस्य च द्रव्यस्याधिकगौरवविशेषात् तुलानतिविशषो दृश्येत\edlabel{pvv.466-1}\footnote{\label{pvv.466-1}  १ परमाणवोऽदृश्या नोदकाहरणक्षमा अतोवयवी वै शे षि कस्वीकृतः । स धर्मी अन्यत्वं साध्यं तत्र नान्योवयवीत्युक्तं मृत्पिण्डयोस्तोल्ययोस्तृतीयक्षेपे गौरवान्यत्ववत् तृतीयावयविनि नेति ।}। न चैवं तस्मान्न तत्र कार्यद्रव्यसम्भव इत्यवयवी निर्गुणो निष्क्रियश्च स्यात् (। १५४)
	\pend
      \label{div_pvv.4.155}\edlabel{div_pvv.4.155}
	  
	% new div opening: depth here is 2
	
	  \bigskip
	  \begingroup
	  \large
	
	    
	    \stanza[\smallbreak]
	\label{pv.4.155}\edlabel{pv.4.155}\flagstanza{\tiny\textenglish{....4.155}}तन्निर्गुणक्रियस्तस्मात् समवायि न कारणम् ।&तत एव न दृश्योसावदृष्टेः कार्यरूपयोः ॥ १५५ ॥\&[\smallbreak]


	
	  \endgroup
	

	  \pstart {\color{DodgerBlue3}“तस्मान्निर्गुण”}क्रियत्वात् गुणकर्मणोर्न {\color{DodgerBlue3}“समवायि कारण”}मवयवी । {\color{DodgerBlue3}“तत एव”} द्रव्यलक्षणयोगान्न तस्यावरणादि {\color{DodgerBlue3}“कार्यं स्वरूप”}म्वा किञ्चित् दृश्यते (।) अदर्श{\color{DodgerBlue3}“नाच्च न”} दृश्योऽवयवी (। १५५)
	\pend
      \label{div_pvv.4.156_4.157_4.158}\edlabel{div_pvv.4.156_4.157_4.158}
	  
	% new div opening: depth here is 2
	
	  \bigskip
	  \begingroup
	  \large
	
	    
	    \stanza[\smallbreak]
	\label{pv.4.156a}\edlabel{pv.4.156a}\flagstanza{\tiny\textenglish{...4.156a}}[तद्बाधान्यविशेषस्य नान्तरीयकभाविनः ।\&[\smallbreak]


	
	  \endgroup
	

	  \pstart {\color{DodgerBlue3}“तत्”} तस्मादसाववयविनो {\color{DodgerBlue3}“बाधाऽन्य”}स्य \edlabel{pvv.466-2}\footnote{\label{pvv.466-2}  २ कीदृशस्येत्याह ।} {\color{DodgerBlue3}“विशेष”}स्य गौरवाधोगत्यादे{\color{DodgerBlue3}“र्नान्तरीयकभाविनः”} साध्या\edlabel{pvv.466-3}\footnote{\label{pvv.466-3}  ३ यत्साद्यमन्यत्वन्तदविना ।}न्यत्वाविनाभाविनः । बाधेति धर्मविशेषनिराकरणे निर्देशो युक्तः ।
	\pend
      
	  \bigskip
	  \begingroup
	  \large
	
	    
	    \stanza[\smallbreak]
	\label{pv.4.156b}\edlabel{pv.4.156b}\flagstanza{\tiny\textenglish{...4.156b}}आसूक्ष्माद् द्रव्यमालायास्तोल्यत्वादंशुपातवत् ॥ १५६ ॥\&[\smallbreak]


	
	  \endgroup
	
	  \bigskip
	  \begingroup
	  \large
	
	    
	    \stanza[\smallbreak]
	\label{pv.4.157a}\edlabel{pv.4.157a}\flagstanza{\tiny\textenglish{...4.157a}}द्रव्यान्तरगुरुत्वस्य गतिर्नेत्यपरोऽब्रवीत् ।\&[\smallbreak]


	
	  \endgroup
	

	  \pstart {\color{DodgerBlue3}“आसूक्ष्माद्”} द्व्यणुका\edlabel{pvv.466-4}\footnote{\label{pvv.466-4}  ४ त्र्यणुकादौ महत्वोक्तेः ।}दारभ्य {\color{DodgerBlue3}“द्रव्यमालायाः”}  स्थूलावयविपर्यन्तायास्तो{\color{DodgerBlue3}“ल्यत्वा”}त् तद्दद्रव्यमालावर्त्तिनः स्थूलस्य {\color{DodgerBlue3}“द्रव्यान्तर”}स्य {\color{DodgerBlue3}“न गति”}र्भवति । तुलाया{\color{DodgerBlue3}“मंशुपातवत्”} । कर्पासभारपतितस्यांशोरेकस्य यथा गुरुत्वं सदपि न प्रतीयते (।) तथा द्रव्यमालावर्त्तिनः स्थूलद्रव्यस्य {\color{DodgerBlue3}“इत्यपरो\edlabel{pvv.466-5}\footnote{\label{pvv.466-5}  ५ उ द्यो त क रा दिः ।}ऽब्रवीत्”} ।
	\pend
      \leavevmode\marginnote{\textenglish{467/s}}
	  \bigskip
	  \begingroup
	  \large
	
	    
	    \stanza[\smallbreak]
	\label{pv.4.157b}\edlabel{pv.4.157b}\flagstanza{\tiny\textenglish{...4.157b}}तस्य क्रमेण संयुक्ते पांशुराशौ सकृद् युते ॥ १५७ ॥\&[\smallbreak]


	
	  \endgroup
	
	  \bigskip
	  \begingroup
	  \large
	
	    
	    \stanza[\smallbreak]
	\label{pv.4.158}\edlabel{pv.4.158}\flagstanza{\tiny\textenglish{....4.158}}भेदः स्याद् गौरवे तस्मात् पृथक् सह च तोलिते ।&सुवर्णमाषकादीनां संख्यासाम्यं न युज्यते ॥ १५८ ॥\&[\smallbreak]


	
	  \endgroup
	

	  \pstart {\color{DodgerBlue3}“तस्यै”}वंवादिनो मते {\color{DodgerBlue3}“क्रमेण”} सूक्ष्मावयवसंयोगाभिवृद्धिपरिपाट्या {\color{DodgerBlue3}“संयुक्ते”} स्थूलावयवितां गते {\color{DodgerBlue3}“पांशुराशा”}ववयवसंयोगाभिवृद्धिक्रममनपेक्ष्य  {\color{DodgerBlue3}“सकृदे”}ककालं {\color{DodgerBlue3}“युक्ते”} स्थूलावयवितां {\color{DodgerBlue3}“गते भेदो गौरवे स्यात्”} । तथा हि द्व्यणुकादिक्रमेण कार्यद्रव्यसंयोगपरंपरया च द्रव्यमुत्पद्यते । तत्रानेकद्रव्यभारसद्भावात् महद्गौरवं भवेत् । यत्र त्वेक एव सकृत् पांशुराशिः संयोगाज्जायते तत्रैकस्य द्रव्यस्याल्पीयो गौरवं भवति न चास्त्येतत् । किञ्च (।) {\color{DodgerBlue3}“तस्माद्”} गौरवभेदात् {\color{DodgerBlue3}“पृथक्”} प्रत्येकं माषकादौ {\color{DodgerBlue3}“तोलिते”} पिण्डावस्थयोः {\color{DodgerBlue3}“सह चा”}परैर्मास (? ष) कादिभिस्तोलिते सुवर्ण्णमास-\leavevmode\marginnote{\textenglish{94b/MA}} कादिभिन्नायाः संख्यायाः {\color{DodgerBlue3}“साम्यं \edlabel{pvv.467-1}\footnote{\label{pvv.467-1}  १ रक्तिकशतेऽवयविराशेः कथमेकावयविना साम्यं । एकानेकपलपिण्डवत् ।} न युज्यते”} (।) मासकावयविनां विनाशे तत्संख्यानां गौरवाणां च नाशादेकं सुवर्ण्णमित्येव स्यात् । दृश्यते च प्रत्येकं माषकादीनां तुलायां यावती संख्या तावत्येव सहतोलितानामपि ॥ (१५७, १५८)
	\pend
      \label{div_pvv.4.159}\edlabel{div_pvv.4.159}
	  
	% new div opening: depth here is 2
	
	  \bigskip
	  \begingroup
	  \large
	
	    
	    \stanza[\smallbreak]
	\label{pv.4.159}\edlabel{pv.4.159}\flagstanza{\tiny\textenglish{....4.159}}सर्षपादामहाराशेरुत्तरोत्तरवृद्धिमत् ।&गौरवं कार्यमालाया यदि नैवोपलभ्यते ॥ १५९ ॥\&[\smallbreak]


	
	  \endgroup
	

	  \pstart अथ\edlabel{pvv.467-2}\footnote{\label{pvv.467-2}  २ मा भूत् संख्यावैषम्यमिति ।} रक्तिकायाश्चतुर्थो भागः {\color{DodgerBlue3}“सर्षप”}स्तस्मादारभ्य {\color{DodgerBlue3}“आमहाराशेः”} स्थूलावयविनं यावत् क्रमवृद्धिमतां {\color{DodgerBlue3}“का1र्याणां मालाया उत्तरोत्तरवृद्धिमद् गौरवं सदपि”} नोपलभ्यत इति यदुच्यते (। १५९) तदा (।)
	\pend
      \label{div_pvv.4.160}\edlabel{div_pvv.4.160}
	  
	% new div opening: depth here is 2
	
	  \bigskip
	  \begingroup
	  \large
	
	    
	    \stanza[\smallbreak]
	\label{pv.4.160}\edlabel{pv.4.160}\flagstanza{\tiny\textenglish{....4.160}}आसर्षपाद् गौरवन्तु दुर्लक्षतमनल्पकम् ।&तोल्यं तत्कारणं कार्यगौरवानुपलक्षणात् ॥ १६० ॥\&[\smallbreak]


	
	  \endgroup
	

	  \pstart {\color{DodgerBlue3}“आसर्षपात्”} सर्षपादारभ्य द्ब्यणुकं यावत् पूर्व्वं पूर्व्वं {\color{DodgerBlue3}“गौरवं”} तत् {\color{DodgerBlue3}“दुर्लक्षतमनल्पकमिति”} सुतरामनुपलभ्यं स्यात् । सर्षपादुत्तरन्तु गौरवं स्वयमेव दुर्ल्लक्षमिष्टमिति कार्यद्रव्यगौरवानुपलक्षणात् तस्य कार्यद्रव्यस्य {\color{DodgerBlue3}“कारणं”} परमाणवः पारिशेष्यात् {\color{DodgerBlue3}“तोल्यं”} स्यात् । न च परमाणूनां\edlabel{pvv.467-3}\footnote{\label{pvv.467-3}  ३ तदनुपलम्भादेवावयविकल्पनात् ।} गुरुत्वादिगुणोपलम्भोस्ति {\color{DodgerBlue3}“तदनुपलक्षणेन”} च नावयविनः परमाणूनाम्वोपलम्भोस्तीति सर्व्वथार्थानामप्रतिपतिः स्यात् । (१६०)
	\pend
      \label{div_pvv.4.161}\edlabel{div_pvv.4.161}
	  
	% new div opening: depth here is 2
	
	  \bigskip
	  \begingroup
	  \large
	
	    
	    \stanza[\smallbreak]
	\label{pv.4.161a}\edlabel{pv.4.161a}\flagstanza{\tiny\textenglish{...4.161a}}नन्वदृष्टोंऽशुवत् सोर्थो न च तत्कार्यमीक्ष्यते ।\&[\smallbreak]


	
	  \endgroup
	\leavevmode\marginnote{\textenglish{468/s}}

	  \pstart {\color{DodgerBlue3}“ननु”} गुरुत्वाप्रतीतावपि अंशुः प्रतीयते तद्वदवयव्यादिकमपि प्रत्येष्यत इत्याह । न च गुरुत्वानुपलक्षणे{\color{DodgerBlue3}“प्यंशु”}रिव {\color{DodgerBlue3}“सोऽर्थो”}ऽवयव्यादिरीक्ष्यते । {\color{DodgerBlue3}“न च तत्कार्यं”} गौरवा{\color{DodgerBlue3}“वरणादिरीक्ष्यते”} । तस्मादप्रत्यक्षतैव सर्व्वथा स्यात् ।
	\pend
      

	  \pstart किञ्च (।)
	\pend
      
	  \bigskip
	  \begingroup
	  \large
	
	    
	    \stanza[\smallbreak]
	\label{pv.4.161b}\edlabel{pv.4.161b}\flagstanza{\tiny\textenglish{...4.161b}}गुरुत्वागतिवत् सर्वतद्गुणानुपलक्षाणात् ॥ १६१ ॥\&[\smallbreak]


	
	  \endgroup
	
	  \bigskip
	  \begingroup
	  \large
	
	    
	    \stanza[\smallbreak]
	\label{pv.4.162a}\edlabel{pv.4.162a}\flagstanza{\tiny\textenglish{...4.162a}}माषकादेरनाधिक्यम्;\&[\smallbreak]


	
	  \endgroup
	

	  \pstart {\color{DodgerBlue3}“गुरुत्वा”}दिगुणा{\color{DodgerBlue3}“गतिवत् सर्व्वे”}षां रूपादीनामधिकानां तस्य द्रव्यस्य {\color{DodgerBlue3}“गुणाना”}म{\color{DodgerBlue3}“नुपलक्षणाद”}वयवेभ्यो {\color{DodgerBlue3}“भाषकादे”}रवयविनोर{\color{DodgerBlue3}“नाधिक्यं”} । यदि ह्यवयवेभ्योधिकं तदा गुरुत्वादिवत् रूपादिवत् अवयवेषु वर्द्धमानेषु वर्द्धमानं दृश्येत । (१६१)
	\pend
      \label{div_pvv.4.162}\edlabel{div_pvv.4.162}
	  
	% new div opening: depth here is 2
	

	  \pstart ननु तुलानतिविशेषाग्रहणादित्युक्तमा चा र्येण तत्किं रूपादिग्रहणाभाव उच्यत इत्याह ।
	\pend
      
	  \bigskip
	  \begingroup
	  \large
	
	    
	    \stanza[\smallbreak]
	\label{pv.4.162b}\edlabel{pv.4.162b}\flagstanza{\tiny\textenglish{...4.162b}}अनतिः सोपलक्षणम् ।&यथास्वमक्षेणादृष्टे रूपादावधिकाधिके ॥ १६२ ॥\&[\smallbreak]


	
	  \endgroup
	

	  \pstart {\color{DodgerBlue3}“अनति”}राचार्येण या निर्द्दिष्टा {\color{DodgerBlue3}“सोपलक्षण”}मात्रं न तु नियमः । अनतिः क्वोपलक्षणमित्याह । {\color{DodgerBlue3}“रूपादौ”} द्रव्याभिवृद्ध्या{\color{DodgerBlue3}“ऽधिकाधिके यथास्वं”} यस्य यदात्मीयं ग्राहकं {\color{DodgerBlue3}“तेनाक्षेणे”}न्द्रियज्ञानेना{\color{DodgerBlue3}“दृष्टेः”} \edlabel{pvv.468-1}\footnote{\label{pvv.468-1}  १ अनुपलब्धिसाधनमुक्तं ।}। यद्रूपादिविषयं यदिन्द्रियप्रत्यक्षं तदेव तस्य धर्मिविशेषस्य पक्षबाधकमित्यर्थः । तस्माद् गुणक्रियावत् दृश्यावयव्यभाव एवेति धर्मविशेषवाग्द्वारेणोदाहरणमुक्तं । (१६२)
	\pend
      \label{div_pvv.4.163}\edlabel{div_pvv.4.163}
	  
	% new div opening: depth here is 2
	
	  \bigskip
	  \begingroup
	  \large
	
	    
	    \stanza[\smallbreak]
	\label{pv.4.163}\edlabel{pv.4.163}\flagstanza{\tiny\textenglish{....4.163}}अभ्युपायस्ववागाद्यबाधायाः संभवेन तु ।&उदाहरणमप्यन्यद्दिशा गम्यं यथोक्तया ॥ १६३ ॥\&[\smallbreak]


	
	  \endgroup
	

	  \pstart {\color{DodgerBlue3}“अभ्युपायो”}ऽभ्युपगमः {\color{DodgerBlue3}“स्ववागादि\edlabel{pvv.468-2}\footnote{\label{pvv.468-2}  २ आदिना प्रत्यक्षानुमानप्रतीतिसंभवग्रहः ।}”}र्यस्य तेना{\color{DodgerBlue3}“बाधायाः संभवेन त्व”}न्यदप्युदाहरणं । धर्मिविशेषनिराकरणेन धर्मिस्वरूपनिराकरणेन {\color{DodgerBlue3}“यथोक्तया”}ऽनयोदा{\color{DodgerBlue3}“हरणदिशा गम्यं”} ॥ (१६३)
	\pend
      \label{div_pvv.4.164}\edlabel{div_pvv.4.164}
	  
	% new div opening: depth here is 2
	

	  \pstart तत्र \edlabel{pvv.468-3}\footnote{\label{pvv.468-3}  ३ तदाचार्योक्तमाह ।} परेणावयविनः सकाशादवयवानामन्यत्वे प्रतिज्ञाते यदुच्यते (।) नान्येऽवयवा अवयविनः अप्रत्यक्षत्वप्रसङ्गा\edlabel{pvv.468-4}\footnote{\label{pvv.468-4}  ४ अवयवा धर्मिणः अन्यत्वं साध्यं परेण तन्मध्ये केषाञ्चित् प्रत्यक्षत्वविशेषोभिमतस्तेन । स चान्यत्वेनुपपन्नं तस्य भेदेनाभासनात् । यद्भेदेन यतो न भाति न स्वप्रत्यक्षो । यथा प्रथमस्यामहतोवयविनोऽवयवाः । न भान्ति चान्यस्याप्यवयविनोऽवयवाभेदेनेति व्यापकाभावः ।}दिति । तद्धर्मिविशेषनिराकरणो\leavevmode\marginnote{\textenglish{469/s}} दाहरणं । तथा ह्यवयवानां भेदमिच्छन् प्रत्यक्षतामपीच्छति । अन्यत्वे च निराकृते प्रत्यक्षतायाश्च निरासात् । अवयवानां धर्मिविशेषनिराकरणोदाहरणत्वं व्यक्तं । अभ्युपगम एव चात्र बाधकः । अवयवादर्शने द्रव्यादर्शनस्वीकारात् ॥ गुणव्यतिरिक्तं द्रव्यमस्तीति परेणोक्ते यदोच्यते {\color{DodgerBlue3}“नास्ति द्रव्यं गुणद्रव्याणां द्रव्या-”} द्रव्यत्वप्रस\edlabel{pvv.469-1}\footnote{\label{pvv.469-1}  १ द्रव्यं पृथिव्यादि गुणो गुरुत्वादि ।}ङ्गात् । तद्धर्मिस्वरूपनिराकरणोदाहरणं  (।)
	\pend
      

	  \pstart \edlabel{pvv.469-2}\footnote{\label{pvv.469-2}  २ सत्तायोगात् क ण भु जो द्रव्यादित्रयं सत् । पदार्थसत्करी सत्तेति वचनात् । अतो गुणानामपि द्रव्यतापत्तिः । दृष्टस्थितसत्वेन सम्बन्धात् । यद् द्रव्यसमवायिस्वभावं सत्त्वं तन्न गुण इति एकत्वहानिः सा च नेष्टेति गुणेपि द्रव्यस्थितरूपस्यैव वृत्तिरिति द्रव्यत्वप्रसङ्गः इति दृष्टान्तसूचनं । यत् सत्तावन्न तद् द्रव्यं यथा गुणः । सत्तावच्च द्रव्यमिति विरुद्धव्याप्तेन धर्मिस्वरूपनिकारणात् प्रतिज्ञादोष एवं गुणेपि योज्यं ।} तथाहि धर्मिण एव द्रव्यस्य स्वरूपमात्रं निराक्रियते (।) गुणद्रव्याणाम\leavevmode\marginnote{\textenglish{95a/MA}}न्योन्यं भेदः (स्वीकृतः) गुणोपि द्रव्यं स्यात् । द्रव्यञ्च गुणः । भेदाविशेषादित्यभ्युपायस्य बाधकत्वं ।
	\pend
      

	  \begin{center}%% label @type='head'
	\textbf{(३) नेयायिकपक्षलक्षणे दोषः}
	\end{center}
	

	  \pstart ननु \edlabel{pvv.469-3}\footnote{\label{pvv.469-3}  ३ व्याख्येयान्तर्गतत्वेन परमतमाह ।} साध्य\edlabel{pvv.469-4}\footnote{\label{pvv.469-4}  ४ सिद्धहेत्वादिनिवृत्त्यर्थः ।}निर्देशः प्रतिज्ञेति पक्षलक्षणं नै या यि का नां तत्र को दोषः । असिद्धहेतुदृष्टान्तस्यापि पक्षत्वप्रसङ्ग इत्युक्तं । ननु साध्यत इति साध्यं । हेतुदृष्टान्तौ तु साधयिष्यते ततो नानयोः पक्षत्वप्रसङ्ग इत्याह ।
	\pend
      
	  \bigskip
	  \begingroup
	  \large
	
	    
	    \stanza[\smallbreak]
	\label{pv.4.164}\edlabel{pv.4.164}\flagstanza{\tiny\textenglish{....4.164}}त्रिकालविषयत्वात्तु कृत्यानामतथात्मकम् ।&तथा परं प्रति न्यस्तं साध्यं नेष्टं तदापि तत् ॥ १६४ ॥\&[\smallbreak]


	
	  \endgroup
	

	  \pstart कृत्यानां प्रत्ययानां कालसामान्यविहितत्वेन {\color{DodgerBlue3}“त्रिकालविषयत्वात् साध्य”}शब्देन {\color{DodgerBlue3}“कृत्या”}न्तेन न साध्यमात्रस्य ग्रहणं साधयिष्यमाणस्यापि ग्रहणात् । तथा च नित्यः शब्दो मूर्तत्वात् बुद्धिवदित्या\edlabel{pvv.469-5}\footnote{\label{pvv.469-5}  ५ सिद्धहेत्वादिनिवृत्त्यर्थं ।}दि प्रयोगे तदा वाद\edlabel{pvv.469-6}\footnote{\label{pvv.469-6}  ६ नित्यत्वसाधक ।}काले तद्धेतुदृष्टान्तादिकम{\color{DodgerBlue3}“तथात्मकं”} वस्तु\edlabel{pvv.469-7}\footnote{\label{pvv.469-7}  ७ नित्ये साध्येऽनित्यं ।}तोऽतत्स्वभावात्मकं (।) {\color{DodgerBlue3}“परं प्रति तथा”}ऽतद्रूपत्वेन {\color{DodgerBlue3}“न्यस्त”}\leavevmode\marginnote{\textenglish{470/s}} {\color{DodgerBlue3}“मुपन्यस्तं”} वादिना यत्नतः {\color{DodgerBlue3}“साध्यं”} । यद्यपि शब्दे मूर्त्तत्वं नास्ति ({\color{DodgerBlue3}“अनिष्ट”}ञ्च साध्यत्वेन) तथापि हेतुदृष्टान्तयोरुपन्यासादवश्यं साध्यं । साध्यनिर्देशश्च प्रतिज्ञेति प्रतिज्ञात्वञ्च दुर्व्वारं । (१६४)
	\pend
      \label{div_pvv.4.165}\edlabel{div_pvv.4.165}
	  
	% new div opening: depth here is 2
	

	  \begin{center}%% label @type='head'
	\textbf{(साध्यशब्दचिन्ता)}
	\end{center}
	
	  \bigskip
	  \begingroup
	  \large
	
	    
	    \stanza[\smallbreak]
	\label{pv.4.165}\edlabel{pv.4.165}\flagstanza{\tiny\textenglish{....4.165}}प्रत्यायनाधिकारे तु सर्वासिद्धावरोधिनी ।&यस्मात् साध्यश्रुतिर्नेष्टं विशेषमवलम्बते ॥ १६५ ॥\&[\smallbreak]


	
	  \endgroup
	

	  \pstart {\color{DodgerBlue3}“प्रत्यायन”}स्य ज्ञापकस्य हेतो{\color{DodgerBlue3}“रधिकारे तु”} साधनस्य {\color{DodgerBlue3}“सर्व्व”}स्याभ्युपगमहेतुदृष्टान्तादे{\color{DodgerBlue3}“रसिद्धावरोधः”} संग्रहः (।) तद्वती यस्मात् {\color{DodgerBlue3}“साध्यश्रुतिरिष्टं विशेषं न”} साध्य{\color{DodgerBlue3}“त्वेनावलम्बते”} परिगृह्णाति ॥ (१६५)
	\pend
      \label{div_pvv.4.166}\edlabel{div_pvv.4.166}
	  
	% new div opening: depth here is 2
	

	  \pstart येन प्रत्यायनाधिकारे\edlabel{pvv.470-1}\footnote{\label{pvv.470-1}  १ यदि साध्य इति सामान्यशब्दः किमिति हेतुदृष्टान्त एव सर्व्वमसिद्धमस्तीत्याह ।} साधनस्यासिद्धस्य पक्षत्वप्रसङ्गः (।)
	\pend
      
	  \bigskip
	  \begingroup
	  \large
	
	    
	    \stanza[\smallbreak]
	\label{pv.4.166}\edlabel{pv.4.166}\flagstanza{\tiny\textenglish{....4.166}}तेनाप्रसिद्धदृष्टान्तहेतूदाहरणं कृतम् ।&अन्यथा शशशृङ्गादौ सर्वासिद्धेऽपि साध्यता ॥ १६६ ॥\&[\smallbreak]


	
	  \endgroup
	

	  \pstart {\color{DodgerBlue3}“तेनाप्रसिद्धाभ्यां दृष्टान्तहेतु”}भ्यां प्रतिज्ञाप्रसङ्गादुदाहरणं कृतं प्राक् । तथा चासिद्धदृष्टान्तहेतुवादः प्रसज्यत इत्यनेन । {\color{DodgerBlue3}“अन्यथा”} यदि साधनमसिद्धं विवक्षितं न स्यात् तदा {\color{DodgerBlue3}“शशशृङ्गादा”}वपि {\color{DodgerBlue3}“सर्वस्यासिद्ध”}स्य पक्षत्वप्रसङ्गः ॥ (१६६)
	\pend
      \label{div_pvv.4.167}\edlabel{div_pvv.4.167}
	  
	% new div opening: depth here is 2
	

	  \pstart ननु साध्यं कर्म (।) कर्मणि कृत्यविधानात् । कर्म चेप्सिततमं (।) तच्च क्रियाप्यमात्रं पयसोदनं भुक्तं इत्यत्र क्रियाप्यत्वेप्यनीप्सिततमत्वात् पयः करणं । ततः साधनीयत्वेपि वादिनोऽनीप्सिततमत्वात् हेतुदृष्टान्तादिकं न प्रतिज्ञा भविष्यति । तथा \edlabel{pvv.470-2}\footnote{\label{pvv.470-2}  २ द्वितीयः सन्ति प्रमाणानि प्रमेयार्थानि सर्व्वसंमत्या ।} चतुर्व्विधः सिद्धान्तः \edlabel{pvv.470-3}\footnote{\label{pvv.470-3}  ३ यथा स्वं नित्यमनित्यम्वा ।} । सर्व्वतन्त्रसिद्धान्तः । \edlabel{pvv.470-4}\footnote{\label{pvv.470-4}  ४ यत् प्रसिद्धावन्यसिद्धिः सांख्यस्य वान्तराभवनिषेधे आत्मैव सञ्चरत्यशरीराऽपरीक्षिताभ्युपगमात् ।}प्रतितन्त्रसिद्धान्तः । अधिकरणसिद्धान्तः । अभ्युपगमसिद्धान्तश्च । तत्राभ्युपगमसिद्धान्तत्वं\edlabel{pvv.470-5}\footnote{\label{pvv.470-5}  ५ परीक्षात्मास्तित्ववत् न च तदा हेत्वादयोभ्युपगताः ।} शास्त्रदृष्टस्येति । तदाश्रयेण साध्यस्य निर्देशः प्रतिज्ञा न हेतुदृष्टान्तादेः पक्षत्वप्रसङ्ग इत्याह ।
	\pend
      \leavevmode\marginnote{\textenglish{471/s}}
	  \bigskip
	  \begingroup
	  \large
	
	    
	    \stanza[\smallbreak]
	\label{pv.4.167}\edlabel{pv.4.167}\flagstanza{\tiny\textenglish{....4.167}}सर्वस्य चाप्रसिद्धत्वात् कथञ्चित् तेन न क्षमाः ।&कर्मादिभेदोपक्षेपपरिहाराविवेचने ॥ १६७ ॥\&[\smallbreak]


	
	  \endgroup
	

	  \pstart {\color{DodgerBlue3}“सर्व्वस्य”} प्रतिज्ञाहेतुदृष्टान्तादे{\color{DodgerBlue3}“श्चाप्रसिद्धत्वात्”} । साधयितुमीप्सिततमत्वेनाभ्युपगमात् कर्मत्वं (।) कर्म च साध्यं प्रतिज्ञेति प्रतिज्ञात्वप्रसङ्गो दुर्व्वारः । अभ्युपगमश्च यथा नित्यत्वे तथा मूर्त्तत्त्वादावपि शास्त्राभ्युपगमयोश्च भेदः प्रागुक्तस्तेन कर्मत्वाविशेषेण साध्यं कर्म साधनं करणमिति {\color{DodgerBlue3}“कर्मादिभेदस्योपक्षेपेणोपन्यासेन परिहाराविवेचने”} साध्यविशेषावगमापने कथ{\color{DodgerBlue3}“ञ्चिन्न क्षमाः”} । (१६७)
	\pend
      \label{div_pvv.4.168}\edlabel{div_pvv.4.168}
	  
	% new div opening: depth here is 2
	

	  \pstart यद्यप्युच्यते यः साध्योऽवयवः \edlabel{pvv.471-1}\footnote{\label{pvv.471-1}  १ प्रतिज्ञादिषु प्रक्रान्तेषु साधनविषय एव यः साध्यावयवः ।} तस्य निर्देशः प्रतिज्ञा । ततो हेत्वादेः सिद्धस्यावयवस्य न प्रतिज्ञात्वमिति । तदप्यसत् \edlabel{pvv.471-2}\footnote{\label{pvv.471-2}  २ तथा यदि साध्यं वस्तुत एव साधनाद्भेदेन स्यात् न चैवं ।} ।
	\pend
      

	  \pstart किञ्च
	\pend
      
	  \bigskip
	  \begingroup
	  \large
	
	    
	    \stanza[\smallbreak]
	\label{pv.4.168}\edlabel{pv.4.168}\flagstanza{\tiny\textenglish{....4.168}}प्रागसिद्धस्वभावत्वात् साध्योवयव इत्यसत् ।&तुल्या सिद्धान्तता ते हि येनोपगमलक्षणाः ॥ १६८ ॥\&[\smallbreak]


	
	  \endgroup
	

	  \pstart लक्षणवचनात् {\color{DodgerBlue3}“प्राक्”}  साध्यासाध्ययो{\color{DodgerBlue3}“रसिद्धस्वभावत्वात्”} । लक्षणेन हि साध्यता प्रतिपत्तव्या त्रिकालविषयत्वात् (।) कृत्यप्रत्ययस्य साधयिष्यमाणेपि साध्य इति (।) साधनं च साध्यं स्यात् । {\color{DodgerBlue3}“सिद्धान्तता”} हेतुदृष्टान्तादेरपि {\color{DodgerBlue3}“तुल्या । ते हि”} प्रतितन्त्रादि{\color{DodgerBlue3}“सिद्धान्ता येन”} कारणेन {\color{DodgerBlue3}“उपगमलक्षणा”} अभ्युपगमस्वभावाः । यथा हि नित्यत्वमभ्युपगम्यते\edlabel{pvv.471-3}\footnote{\label{pvv.471-3}  ३ यद्यपि तदा नाभ्युपगता सिद्धान्तता तथापि कृत्यप्रत्यनिर्देशादेवाभ्युपगन्तव्येन्तर्भावः ।}तथा मूर्त्तत्वादिकमपीति न विशेषः । (१६८)
	\pend
      \label{div_pvv.4.169}\edlabel{div_pvv.4.169}
	  
	% new div opening: depth here is 2
	

	  \pstart किञ्च साध्यो धर्मो धर्मी द्वयम्वा स्यात् \edlabel{pvv.471-4}\footnote{\label{pvv.471-4}  ४ अतिव्याप्तिमुक्त्वान्यदप्याह ।} । यदि धर्मस्तदा साध्यसाधर्म्यात् \leavevmode\marginnote{\textenglish{95b/MA}} तद्धर्मभवीदृष्टान्त उदाहरणमिति \edlabel{pvv.471-5}\footnote{\label{pvv.471-5}  ५ नैयायिकस्य ।} दृष्टान्तलक्षणं विरुध्यते । न हि साध्यधर्मस्यानित्यत्वादेर्द्धर्म उत्पत्तिमत्वादिः (।) किन्तु \edlabel{pvv.471-6}\footnote{\label{pvv.471-6}  ६ उत्पत्तिमान् शब्दो घटश्चेति साध्यसाधर्म्यं ततः साधयितुमिष्टो धर्मस्तद्धर्मः तस्य भावः सत्ता । सोस्यास्तीति धर्मस्य धर्मान्तरासम्बन्धात् स्वरूपहानिप्रसङ्गाच्च । दृष्टान्तस्यापि तद्धर्मभावित्वं नास्ति ।} शब्दस्य (।) ततः साध्यधर्मेण साधर्म्यात् दृष्टान्तस्य तद्धर्मभावित्वं साध्यधर्मोत्पत्तिमत्वादिभाववत्वं नास्ति ।
	\pend
      

	  \pstart \leavevmode\marginnote{\textenglish{472/s}}अथ धर्मी साध्यः तदोदाहरणसाधर्म्यात् साध्यसाधनं हेतुरिति हेतुलक्षणं न युज्येत । न हि शब्दो धर्म्मसिद्धो येन तत्साधनात् साध्यसाधनो हेतुः स्यात् ।
	\pend
      

	  \pstart अथो\edlabel{pvv.472-1}\footnote{\label{pvv.472-1}  १ पृथक् पृथक् चतुर्थः समुदायपक्षो वक्ष्यमाणः ।}भयं साध्यं तदोभयपक्षभाविदोषप्रसङ्गः । समुदायस्यासिद्धत्वात् स एव साध्य इत्याह ।
	\pend
      
	  \bigskip
	  \begingroup
	  \large
	
	    
	    \stanza[\smallbreak]
	\label{pv.4.169}\edlabel{pv.4.169}\flagstanza{\tiny\textenglish{....4.169}}समुदायस्य साध्यत्वेप्यन्योन्यस्य विशेषणम् ।&साध्यं द्वयं तदाऽसिद्धं हेतुदृष्टान्तलक्षणम् ॥ १६९ ॥\&[\smallbreak]


	
	  \endgroup
	

	  \pstart {\color{DodgerBlue3}“समुदायस्य साध्यत्वे”}पीष्यमाणे{\color{DodgerBlue3}“ऽन्योन्यस्य”} परस्परं {\color{DodgerBlue3}“विशेषण”}म्वक्तव्यं । धर्मवि\leavevmode\marginnote{\textenglish{96b/MA}} शिष्टो\edlabel{pvv.472-asterisk}\footnote{\label{pvv.472-asterisk}  * अत्र\edlabel{pvv.472-asterisk-1}[[१ [आदर्शपुस्तकस्थकुण्डलीकृतो धर्मिविशिष्टो वा साध्य इत्यादि “यदि नित्यः शब्दः सर्वस्यानित्यत्वादिति वैधर्म्यदृष्टान्तोपदर्शनमेतत्” इत्यन्तग्रन्थांशोऽत्र यथानुक्रममेवोद्ध्ृत्य योजितः ।]]]पतितमधःपत्रपृष्टेस्ति कुण्डलीकृतं । \par
\leavevmode\marginnote{\textenglish{95a-1/MA}}तदा\edlabel{pvv.472-asterisk-2}[[२ १७०-१७७ कारिकाणां व्याख्यान्तरमादर्शपुस्तकस्थमधो विन्यस्तम् ।]]चासिद्धं हेतुलक्षणं धर्मिधर्मसमुदायधर्मेणोत्पत्तिमत्वादिनोदाहरणसाध र्म्याभावात् । \textbf{दृष्टान्तलक्षण}मप्य\textbf{सिद्धं} दृष्टान्ते धर्मधर्मिसमुदायेन साधर्म्याभावात् ॥ 
	  \bigskip
	  \begingroup
	  \large
	
	    \begin{verse}
	असंभवात् साध्यशब्दो धर्मिवृत्तिर्यदीष्यते ।\\
	    शास्त्रेणालं यथायोगं लोक एव प्रवर्त्तताम् ॥ १७० ॥\\
	    
	    \end{verse}
	  
	  \endgroup
	 \par
अथोत्पत्तिमत्वादेः समुदायधर्मत्वाभावात् \textbf{साध्यशब्दो धर्मिवृत्तिरिष्यते} यथा बौ द्धे नोक्त उपचारात् । तदा \textbf{शास्त्रेण} लक्षणे\textbf{नालं} किं प्रयोजनं प्रसिद्धत्वात् \textbf{यथायोगं} बुद्धा \textbf{लोक एव प्रवर्त्ततां} (।) सविकल्पकं हि ज्ञानं तस्य प्रसिद्धं । बौद्धस्तेन लोकसिद्धमनुवदति किन्तु न्यायमिति तस्य पक्षैकदेशयोग्यधर्मिणो  धर्मसामान्यस्य दृष्टान्तेपि कथनार्थं जिज्ञासितविशेषो धर्मीति लक्षणं युक्तं । न च साध्यशब्दस्यासिद्धदृष्टान्तादौ प्रसङ्गात् पक्षत्वन्तूपचारात् ॥ \par
किञ्च (।) अत्र साध्यनिर्देश एव प्रतिज्ञेति पूर्व्वावधारणन्तावन्न भवति । 
	  \bigskip
	  \begingroup
	  \large
	
	    \begin{verse}
	साधनाख्यानसामर्थ्यात् तदर्थे साध्यता गता ।\\
	    हेत्वादिवचनैर्व्याप्तेरनाशङ्क्यं च साधनम् ॥ १७१ ॥\\
	    
	    \end{verse}
	  
	  \endgroup
	 \par
यस्मात् \textbf{तदर्थे} सिद्धनिवृत्त्याऽसिद्धग्रहार्थे \textbf{साध्यता} गम्यत एव (।) \textbf{सिद्धसाधन}वैयर्थ्येन साधनञ्चोक्तमत्रेति \textbf{तत्सामर्थ्यात्} । \par
अपि च साध्यनिर्देश एव प्रतिज्ञेति पूर्व्वावधारणे \textbf{साधनं} न प्रतिज्ञेति साध्यं (।) तच्चा\textbf{नाशङ्क्यं} तस्य हेतूदाहरणोपनयाद्यैरेव व्याप्तत्वात् (।) 
	  \bigskip
	  \begingroup
	  \large
	
	    \begin{verse}
	पूर्व्वावधारणे तेन प्रतिज्ञालक्षणाभिधा ।\\
	    व्यर्था व्याप्तिफला सोक्तिः सामर्थ्याद् गम्यते ततः ॥ १७२ ॥\\
	    
	    \end{verse}
	  
	  \endgroup
	 \par
तेन पूर्व्वावधारणे प्रतिज्ञालक्षणाभिधा व्यर्था । ततः कारणात् \textbf{सामर्थ्यात्} साध्यनिर्देशः प्रतिज्ञैवेत्युत्तरावधार\textbf{णोक्तिर्व्याप्तिफला}ऽयोगव्यवच्छेदफला । इत्यसिद्धहेत्वा(दे)रपि प्रतिज्ञात्वप्रसङ्गः ॥ \par
प्रतिज्ञाहेत्वोर्व्विरोधो नोक्तः पक्षाभाषे (? से)ष्वित्यव्याप्तिलक्षणमिति नै या यिकावकाशं भत्वा चा र्यो3 क्तं व्याख्यातुमाह । प्रतिज्ञाहेत्वोर्व्विरोधः प्रतिज्ञादोष इति यथा नित्यः शब्दः सर्व्वस्यानित्यत्वात् । अत्र यदि शब्दो नित्यः सर्व्वस्यानित्यत्वं विरुद्धमिति प्रतिज्ञया हेतोर्व्विरोधः । सर्व्वानित्यत्वे तदन्तर्गतत्वाच्छब्दस्येति हेतुर्न प्रतिज्ञाविरोधः । उत्तरमाह (।) 
	  \bigskip
	  \begingroup
	  \large
	
	    \begin{verse}
	विरुद्धतेष्टासम्बन्धोऽनुपकारसहास्थिती ।\\
	    एवं सर्व्वाङ्गदोषणां प्रतिज्ञादोषता भवेत् ॥ १७३ ॥\\
	    
	    \end{verse}
	  
	  \endgroup
	 \par
\textbf{इष्टेनार्थेना\edlabel{pvv.472-asterisk-1-bis}[[१ साध्यधर्मेण ।]]सम्बन्धो विरुद्धतात्र} स्यात् । सा चा\textbf{नुपकारसहास्थिती} । हेतुस्तत्राभावात् साध्यं नोपकरोतीति वा स्थितहेतुना इष्टधर्मविरोधात् सहानवस्थानम्वा स्यात् । तत्र नानुपकारः । यदि हेतौ दुष्टे साध्योपरोधात् प्रतिज्ञादोष एवं \textbf{सर्व्व}लिङ्ग\textbf{दोषाणाम}सिद्धादीनां \textbf{प्रतिज्ञादोषता} स्यात् । 
	  \bigskip
	  \begingroup
	  \large
	
	    \begin{verse}
	पक्षदोषः परापेक्षो नेति च प्रतिपादितम् ।\\
	    इष्टासम्भव्यसिद्धश्च स एव स्यात् निराकृतः ॥ १७४ ॥\\
	    
	    \end{verse}
	  
	  \endgroup
	 \par
उत्तरावयवापेक्षः पक्षदोषो न भवतीति प्रागेवोक्तं । \par
अथेष्टेन साध्यधर्मेणासम्भवी सहानवस्थायी हेतुर्व्विरुद्धः । तन्न (।) \textbf{इष्टासम्भव्यसिद्धश्च स्याद्} हेतुदोषो न पक्षे । \textbf{निराकृतो} हेत्वाभासेष्वेवोक्तत्वात् । अधुना हेतुप्रयोग एवायं नित्यात् । न सर्व्वानित्यत्वेन शब्दे नित्यत्वनिराक्रिया यतो विरोधः स्यात् (।) यस्मादेवं स साध्यधर्मो निराकृतः स्यात् । 
	  \bigskip
	  \begingroup
	  \large
	
	    \begin{verse}
	अनित्यतवसहेतुत्वे शब्द एवं प्रकीर्त्तयेत् ॥\\
	    दृष्टान्ताख्यानतोऽन्यत् किमस्त्यत्रार्थानुदर्शनम् ॥ १७५ ॥\\
	    
	    \end{verse}
	  
	  \endgroup
	 \par
यद्यनित्यत्वं शब्दे स्यात् । ततोऽ\textbf{नित्यत्वसहेतुत्वे} सति \textbf{शब्दे एवं} नित्यः शब्दः \par
सर्व्वस्यानित्यत्वादिति \textbf{प्रकीर्तयेत्} । न चैवं (।)हेतुलक्षणाभावात् । \textbf{वैधर्म्य}दृष्टान्तः परमत्र । \textbf{अतोन्यत्र किञ्चिदर्थस्य प्रदर्शनं} । \par
तथाहि (।) 
	  \bigskip
	  \begingroup
	  \large
	
	    \begin{verse}
	विशेषे भिन्नमाख्याय सामान्यस्यानुवर्त्तने ।\\
	    न तद् व्याप्तिः फलं वा किं सामान्येनानुवर्त्तने ॥ १७६ ॥\\
	    
	    \end{verse}
	  
	  \endgroup
	 \par
\textbf{विशे}षे शब्द एव \textbf{भिन्नं} विरुद्धं नित्यत्व\textbf{माख्याय} सर्व्वस्यानित्यत्वादिति \textbf{सामान्यस्यानुवर्त्तने । न तस्य} शब्दस्य विशेषस्य व्याप्तिरनित्यताप्राप्तिर्व्विशेषपरिहारणैव वृत्तेः तक्रं कौण्डिन्याय ब्राह्मणेभ्यो दधि दीयतामितिवत् । शब्दे नित्यत्वं विधाय सर्व्वानित्यत्वमुच्यमानं न तमास्कन्दति । अतः शब्देऽनित्यत्वमभवत् कथं स्वविरुद्धमपाकुर्यात् । \par
यदि शब्दव्यतिरिक्तस्यानित्यत्वं नेष्टं स्या(त्) तदा \textbf{फलं वा किं सामान्यानुवर्त्तने} सर्व्वस्यानित्यत्वादिति । शब्दस्यानित्यत्वादित्येव वाच्यं एवं हि स्फुटो विरोधः स्यात् (।) तस्मान्न शब्देऽनित्यत्वं (।) । 
	  \bigskip
	  \begingroup
	  \large
	
	    \begin{verse}
	स्यान्निराकरणं शब्दे स्थिते नैवेत्यतोब्रवीत् ।\\
	    
	    \end{verse}
	  
	  \endgroup
	 \par
अत एवा चार्योऽब्रवीत् समुच्चये “स्यान्निराकरणं शब्दानित्यत्वेनेति ।” xx} धर्मिविशिष्टो वा\edlabel{pvv.472-2}\footnote{\label{pvv.472-2}  २ यथा सर्व्वमनित्यं न च सर्व्वं शब्दः इत्यसर्वत्वान्नित्यः ।}साध्यः । यथा शब्दविशिष्टमनित्यमिति तथाच द्वयं साध्यं स्यात् । तदा द्वयसाध्यत्वाभ्युपगमे तु हेतुदृष्टान्तयोर्लक्षणमसिद्धं स्यात् । न हि धर्मिधर्मविशिष्टेन धर्मेणानित्यशब्दसम्बन्धिना उत्पत्तिमत्वादिना घटादेः साधर्म्यमस्ति येन दृष्टान्तता भवेत् । तथा धर्मविशिष्टे धर्मिणि प्रागनुमानाद्धेतुरपि न कस्यचित् सिद्ध इति हेतुलक्षणं च न स्यात् । (१६९)
	\pend
      \label{div_pvv.4.170}\edlabel{div_pvv.4.170}
	  
	% new div opening: depth here is 2
	\leavevmode\marginnote{\textenglish{473/s}}
	  \bigskip
	  \begingroup
	  \large
	
	    
	    \stanza[\smallbreak]
	\label{pv.4.170}\edlabel{pv.4.170}\flagstanza{\tiny\textenglish{....4.170}}असंभवात् साध्यशब्दो धर्मिवृत्तिर्यदीष्यते ।&शास्त्रेणालं यथायोगं लोक एव प्रवर्त्तताम् ॥ १७० ॥\&[\smallbreak]


	
	  \endgroup
	

	  \pstart यथोत्पत्तिमत्वादेः समुदायधर्मत्वासंभवाद्धर्मिधर्मत्वसंभवाच्च साध्यशब्दो यदि धर्मिवृत्तिरुपचारादिष्यते तदा शास्त्रेण दुर्विहितेनालं लोक एव यथाक्रमं यथायोगं यस्य यादृशं लक्षणं युक्तं तदवधार्य प्रतिज्ञाहेत्वादिषु प्रवर्त्ततां । यत्तु पक्षो धर्मी अवयवसमुदायोपचारादित्यस्माभिरुच्यते । तत्सर्वं धर्मिधर्मप्रतिषेधार्थं उपचारयोग्यपरिग्रहार्थं । तस्मादयुक्तं परस्य प्रतिज्ञालक्षणं । (१७०)
	\pend
      \label{div_pvv.4.171}\edlabel{div_pvv.4.171}
	  
	% new div opening: depth here is 2
	

	  \pstart किञ्च ।
	\pend
      

	  \pstart सर्ववाक्यानामवधारणफलत्वात्साध्यनिर्देश एव प्रतिज्ञेति पूर्वपदावधारणं वा स्यात् । साध्यनिर्देशः प्रतिज्ञैवेत्युत्तरपदावधारणं वा स्यात् । तत्र प्रथमपक्षे सिद्धनिवृत्याऽसिद्धपरिग्रहः प्रयोजनं । तच्चान्यथापि लभ्यते । तथा हि ।
	\pend
      
	  \bigskip
	  \begingroup
	  \large
	
	    
	    \stanza[\smallbreak]
	\label{pv.4.171}\edlabel{pv.4.171}\flagstanza{\tiny\textenglish{....4.171}}साधनाख्यानसामर्थ्यात्तदर्थे साध्यता गता ।&हेत्वादिवचनैर्व्याप्तेरनाशङ्क्यं च साधनम् ॥ १७१ ॥\&[\smallbreak]


	
	  \endgroup
	

	  \pstart सिद्धेऽपि साधनोपन्यासोऽनुपयुक्त इति साधनाख्यानसामर्थ्यात् तस्याः प्रतिज्ञाया अर्थेऽसिद्धे साध्यता {\color{DodgerBlue3}“प्रतीता”} प्रतिज्ञालक्षणं विनापि
	\pend
      \leavevmode\marginnote{\textenglish{474/s}}

	  \begin{center}%% label @type='head'
	\textbf{(४) प्रतिज्ञालक्षणे दोषः}
	\end{center}
	

	  \pstart नचासिद्धे हेतुदृष्टान्तादिके प्रतिज्ञार्थप्रसंगः । तथा हि । हेत्वादिवचनैः पृथक् लक्षणप्रतिपादकैर्व्याप्तेर्विषयिकृतत्वात् साधनं हेतुदृष्टान्तं असिद्धं च प्रतिज्ञार्थे वाऽनाशङ्क्यं । (१७१)
	\pend
      \label{div_pvv.4.172}\edlabel{div_pvv.4.172}
	  
	% new div opening: depth here is 2
	\leavevmode\marginnote{\textenglish{475/s}}
	  \bigskip
	  \begingroup
	  \large
	
	    
	    \stanza[\smallbreak]
	\label{pv.4.172}\edlabel{pv.4.172}\flagstanza{\tiny\textenglish{....4.172}}पूर्वावधारणे तेन प्रतिज्ञालक्षणाभिधा ।&व्यर्था व्याप्तिफला सोक्तिः सामर्थ्याद्गम्यते ततः ॥ १७२ ॥\&[\smallbreak]


	
	  \endgroup
	

	  \pstart तेनातिप्रसंगाभावेन पूर्वस्य पदस्यावधारणे प्रतिज्ञालक्षणाभिधा व्यर्था । ततः पारिशेष्यात् सामर्थ्यात्साध्यनिर्देशः प्रतिज्ञैवेत्ययोगव्यवच्छेदफलमुत्तरपदावधारणं स्यात् । तथाचासिद्धहेतुदृष्टान्तादावपि प्रतिज्ञात्वं दुर्वारं । असिद्धस्य साधनाङ्गस्य कथमपि प्रतिज्ञात्वायोगताविरहात् निरस्तं प्रतिज्ञालक्षणं । (१७२)
	\pend
      \label{div_pvv.4.173_4.174_4.175}\edlabel{div_pvv.4.173_4.174_4.175}
	  
	% new div opening: depth here is 2
	

	  \pstart परस्य प्रतिज्ञाभासलक्षणं संप्रति निराकरणीयं (।) हेतुप्रतिज्ञयोर्व्याघातः प्रतिज्ञादोषो मतः । यथा नित्यः शब्दः सर्वस्य नित्यत्वात् । यदि सर्वमनित्यं तदा शब्दस्यापि सर्वत्रान्तर्भावात् कुतो नित्यता । अथ शब्दो नित्यः कथं सर्वस्यानित्यतेति प्रतिज्ञाहेत्वोर्विरोधात् । प्रतिज्ञाविरुद्धतादोषः ।
	\pend
      
	  \bigskip
	  \begingroup
	  \large
	
	    
	    \stanza[\smallbreak]
	\label{pv.4.173}\edlabel{pv.4.173}\flagstanza{\tiny\textenglish{....4.173}}विरुद्धतेष्टासम्बन्धोऽनुपकारसहास्थिती ।&एवं सर्वाङ्गदोषाणां प्रतिज्ञादोषता भवेत् ॥ १७३ ॥\&[\smallbreak]


	
	  \endgroup
	

	  \pstart विरुद्धता चेष्टस्या साध्यधर्मस्य धर्मिण्यसम्बन्धो नाम । स च विचार्यमाणो \leavevmode\marginnote{\textenglish{476/s}} हेतुना साध्यधर्मस्यानुपकारोऽनिश्चायनं वा स्यात् । धर्मिणि साध्येन सहास्थितिर्वा स्यात् । तत्र यदि हेतोः साध्ये प्रतिपादकत्वाभावात् प्रतिज्ञादोष उच्यते (।) एवं सति सर्वेषामङ्गस्य हेतोर्दोषाणां प्रतिज्ञादोषता भवेत् सर्वैर्हेंतुदोषैः प्रतिज्ञाया एव व्याहननात् ।
	\pend
      
	  \bigskip
	  \begingroup
	  \large
	
	    
	    \stanza[\smallbreak]
	\label{pv.4.174}\edlabel{pv.4.174}\flagstanza{\tiny\textenglish{....4.174}}पक्षदोषः परापक्षो नेति च प्रतिपादितम् ।&इष्टासम्भव्यसिद्धश्च स एवं स्यान्निराकृतः ॥ १७४ ॥\&[\smallbreak]


	
	  \endgroup
	

	  \pstart प्रतिज्ञामात्रभागी च पक्षदोषः । परापेक्षः साधनादिसापेक्षः न दोष इति च प्रतिपादितं प्रागुत्तरावयवापेक्षो न दोषः पक्ष इष्यत इत्यादिना । अथ सहास्थितिस्वभावो विरोधः । तदेष्टे पक्षेऽसम्भवी हेतुदोष एवायं न पक्षदोषः । अथ तेन हेतुना प्रतिज्ञार्थनिराकरणात् प्रतिज्ञाविरोधः पक्ष-दोष एव (।) तच्चायुक्तं तथा हि (।) साध्यधर्मो धर्मिण्येवं निराकृतः स्यात् ।
	\pend
      
	  \bigskip
	  \begingroup
	  \large
	
	    
	    \stanza[\smallbreak]
	\label{pv.4.175}\edlabel{pv.4.175}\flagstanza{\tiny\textenglish{....4.175}}अनित्यत्वसहेतुत्वे शब्द एवं प्रकीर्त्तयेत् ।&दृष्टान्ताख्यानतोऽन्यत् किमस्त्यत्रार्थानुदर्शम् ॥ १७५ ॥\&[\smallbreak]


	
	  \endgroup
	

	  \pstart यदि शब्दे धर्मिण्यनित्यत्वेन धर्मेण सहेतुत्वे प्रतिपाद्ये एवं सर्वस्यानित्यत्वादि \leavevmode\marginnote{\textenglish{97a/MA}} प्रकीर्त्तयेत् । न चेदृशं वादिनो विवक्षितं सर्व्वस्य परस्यानित्यत्वात् । शब्दो नित्य इति विवक्षितत्वात् । तथा च सामान्यविशेषभावाद् विरोधभावः । भवतु वा शब्देऽनित्यत्वनिराकरणं विवक्षितं (।) तथापि प्रमाणबाधैवापक्षतेति न प्रतिज्ञाविरोधो नाम पक्षदोषः । तस्माद्धेत्वर्थतानुपपत्तेः सर्व्वस्यानित्यत्वादित्यत्र वैधर्म्येण दृष्टान्ताख्यानतोऽन्यदर्थानुदर्शनं किमस्ति । वैधर्म्यदृष्टान्त एव  सुशिक्षितैरित्थमाख्यातः । सर्व्वस्य नित्यत्वे व्याप्तिदर्शनार्थं यदि पुनर्व्वैधर्म्यदृष्टान्तोपदर्शनमेतन्न भवति (।) तदा ।
	\pend
      \label{div_pvv.4.176_4.177_4.178_4.179_4.180_4.181_4.182_4.183_4.184_4.185_4.186_4.187_4.188}\edlabel{div_pvv.4.176_4.177_4.178_4.179_4.180_4.181_4.182_4.183_4.184_4.185_4.186_4.187_4.188}
	  
	% new div opening: depth here is 2
	

	  \begin{center}%% label @type='head'
	\textbf{(५) सामान्यचिन्ता}
	\end{center}
	

	  \begin{center}%% label @type='head'
	\textbf{क. सामान्यानुवर्त्तने निष्फलम्}
	\end{center}
	
	  \bigskip
	  \begingroup
	  \large
	
	    
	    \stanza[\smallbreak]
	\label{pv.4.176}\edlabel{pv.4.176}\flagstanza{\tiny\textenglish{....4.176}}विशेषेभिन्नमाख्याय सामान्यस्यानुवर्त्तने ।&न तद्व्याप्तिः फलं वा किं सामान्येनानुवर्त्तने ॥ १७६ ॥\&[\smallbreak]


	
	  \endgroup
	

	  \pstart विशेषेण शब्दे भिन्नं नित्यत्वमाख्याय सर्व्वस्यानित्यत्वादिति व्यापित्वात् सामान्यस्यानित्यत्वस्यानुवर्त्तने क्रियमाणे तस्यानित्यत्वस्य व्याप्तिरशेषपदार्थग्रहो न भवति । यथा कौण्डिन्यस्य तक्रदानं विहितं ब्राह्मणेभ्यः सामान्येन विहितदधिदानेन न बाध्यते । प्रकल्प्यापवादविषयमुत्सर्गस्य प्रवृत्तेः । तथा शब्दे नित्यत्वस्य \leavevmode\marginnote{\textenglish{477/s}} विशेषविहितस्य सर्व्वानित्यत्वेन सामान्यविहितेन बाधाशङ्का नास्तीति कथं प्रतिज्ञाहेत्वोर्व्विरोधः । यदि शब्दव्यतिरिक्तस्य सर्व्वस्यानित्यत्वमिष्टं न स्यात् (।) तदा सर्व्वस्यानित्यत्वादिति सामान्येनानुवर्त्तने किं फलं स्यात् । नित्यः शब्दः शब्दस्यानित्यत्वादित्येव वाच्यं । एवं विरोधस्य वक्तव्यत्वात् शब्द एवोदाहरणम्भविष्य तीति चेत् । नानुन्मत्त एवं ब्रूयात् । यद्यात्मनोऽनित्यत्वं हेतुः सिद्धः । कथं तद्विरुद्धं साध्यं (।) तत्रैव प्रतिजानीयात् । अथासिद्धस्तदा हेतुदोष एवासौ न प्रतिज्ञादोषः (।) तस्मान्नास्ति शब्दे नित्यत्वं । अतः स्वविरोधिनमपि निराकर्त्तुमशक्तं ।
	\pend
      
	  \bigskip
	  \begingroup
	  \large
	
	    
	    \stanza[\smallbreak]
	\label{pv.4.177a}\edlabel{pv.4.177a}\flagstanza{\tiny\textenglish{...4.177a}}स्यान्निराकरणं शब्दे स्थितेनैवेत्यतोब्रवीत् ।\&[\smallbreak]


	
	  \endgroup
	

	  \pstart अत एवाचार्य {\color{DodgerBlue3}“शब्देस्थितेनैवा”}नित्यत्वेन {\color{DodgerBlue3}“निराकरणं”} नित्यत्वस्य {\color{DodgerBlue3}“स्यादित्यब्रवीत्”} । तदा च स्यादत्र प्रतिज्ञार्थस्य निराकरणं ॥
	\pend
      

	  \pstart यदि नित्यः  शब्दः सर्व्वस्यानित्यत्वादिति वैधर्म्यदृष्टान्तोपदर्शनमेतत् । यथा नित्यत्वविशिष्ट शब्द इति । तदा हेतु-निर्देशोन स्यादित्याह (।) अ\edlabel{pvv.477-1}\footnote{\label{pvv.477-1}  १ साधर्म्यवैधर्म्योदाहरणापेक्षः तथा न तथेति पक्षधर्मोपसंहार उपनय इह तु वैधर्म्योपनयः । यदाह तस्मात् सर्वत्वान्नानित्य इति ।}सर्वश्च शब्द इत्युपनयाद्धेतुर्वक्तव्यः ।
	\pend
      
	  \bigskip
	  \begingroup
	  \large
	
	    
	    \stanza[\smallbreak]
	\label{pv.4.177b}\edlabel{pv.4.177b}\flagstanza{\tiny\textenglish{...4.177b}}विरुद्धविषयेन्यस्मिन् वदन्नाहान्यतां श्रुतेः ॥ १७७ ॥\&[\smallbreak]


	
	  \endgroup
	

	  \pstart तथाहि नित्यत्वस्य विरुद्धं शब्दादन्यस्मिन् वदन् सर्व्वस्मादन्यतां श्रुतेः 95b शब्द\edlabel{pvv.477-2}\footnote{\label{pvv.477-2}  २ सर्वमनित्यत्वेन व्याप्तं शब्दमनित्यत्वात्ततोऽन्य इत्यसर्वत्वं हेतुः स्यात् ।}स्याह ।
	\pend
      
	  \bigskip
	  \begingroup
	  \large
	
	    
	    \stanza[\smallbreak]
	\label{pv.4.178a}\edlabel{pv.4.178a}\flagstanza{\tiny\textenglish{...4.178a}}स च भेदोप्रतिक्षेपात् सामान्यानान्न विद्यते ।\&[\smallbreak]


	
	  \endgroup
	

	  \pstart स च सर्व्वस्माद् भेदोऽसर्व्वलक्षणः शब्दस्य न विद्यते (।) सामान्यानां व्यापिनां भेदस्याप्रतिक्षेपात् स्वीकारात् (।) निः शेषार्थसंगृहीतत्वात् सर्व्वशब्दो न किञ्चित् परिहरति ।
	\pend
      

	  \pstart ननु सामान्यानां विशेषप्रतिक्षेपो दृश्यते एव । यथा किं शिंशपैव वृक्षो न वेति प्रश्ने कथ्यते न शिंशपैव वृक्षः । तदा शिंशपावृक्षत्वप्रतिक्षेपो भवत्येव ।
	\pend
      

	  \pstart असदेतत् । न हि तत्र शिंशपावृक्षत्वं निषिध्यते किन्तु शिंशपैव वृक्ष इति नियमः प्रतिक्षिप्यते (।) तदितरस्यापि वृक्षत्वात् ।
	\pend
      
	  \bigskip
	  \begingroup
	  \large
	
	    
	    \stanza[\smallbreak]
	\label{pv.4.178b}\edlabel{pv.4.178b}\flagstanza{\tiny\textenglish{...4.178b}}वृक्षो न शिंशपैवेति यथा प्रकरणे क्वचित् ॥ १७८ ॥\&[\smallbreak]


	
	  \endgroup
	\leavevmode\marginnote{\textenglish{478/s}}
	  \bigskip
	  \begingroup
	  \large
	
	    
	    \stanza[\smallbreak]
	\label{pv.4.179}\edlabel{pv.4.179}\flagstanza{\tiny\textenglish{....4.179}}सर्वश्रुतेरेकवृत्तिनिषेधः स्यान्न चेयता ।&सोसर्वः सर्वभेदानामतत्त्वे तदसम्भवात् ॥ १७९ ॥\&[\smallbreak]


	
	  \endgroup
	

	  \pstart यथा च क्वचित् प्रकरणे शिंशपामात्रवृक्षत्वप्रश्नहेतौ यथा वृक्षो न शिंशपैवेति शिंशपामात्रवृक्षत्वनिषेध इष्टः । तथा निःशेषवस्तुसंग्राहिकायाः सर्व्वश्रुतेरेकत्र शब्दमात्रवृत्तिनिषेधः स्यात् (।) न चेयता स शब्दोऽसर्व्वः सर्व्वान्तर्गमात् तस्य । प्रत्येकं सर्व्वेषां भेदानां विशेषाणामतत्वेऽसर्व्वत्वे तस्य सर्व्वस्यासम्भवात् ।
	\pend
      

	  \pstart अथ पारिभाषिकं सर्व्वत्वं शब्दादितरत्वं तथाऽसर्व्वत्वं शब्दे सिद्धमेवेति चेत् । तदयुक्तं\edlabel{pvv.478-1}\footnote{\label{pvv.478-1}  १ शब्दोऽसर्व्व इति पर्यायाः । तथा च यथा नित्यः शब्दः शब्दत्वादिति प्रतिज्ञार्थैकदेशो हेतुरसिद्धस्तथेहाप्यसिद्धिरिति समुदायार्थः ।} (।) एवं हि सर्व्वत्वमेव सर्व्वशब्देनोक्तं स्यात् । तथाऽप्रसिद्धतैव । तथा हि ।
	\pend
      
	  \bigskip
	  \begingroup
	  \large
	
	    
	    \stanza[\smallbreak]
	\label{pv.4.180}\edlabel{pv.4.180}\flagstanza{\tiny\textenglish{....4.180}}ज्ञाप्यज्ञापकयोर्भेदात् धर्मिणो हेतुभाविनः ।&असिद्धेर्ज्ञापिकत्वस्य धर्म्यसिद्धः स्वसाधने ॥ १८० ॥\&[\smallbreak]


	
	  \endgroup
	

	  \pstart ज्ञाप्यज्ञापकयोर्भेदात् कारणा  त् साध्याभिन्नं साधनं वक्तव्यं । ततो धर्मिणो हेतुत्वेन भाविनो हेतुर्भविष्यतो नित्यत्वेन साध्यत्वात् । ज्ञापकस्यासिद्धेः कारणात् । धर्मी साध्यत्वात् स्वस्य साधनेऽसिद्धः । असिद्धं हि साध्यं । सिद्धञ्च साधनं । अनयोः कथमैकात्म्यं ॥
	\pend
      

	  \pstart ननु शब्दत्वं सिद्धमेव । नित्यवत्तया तु तदसिद्धं साध्यते । न च तथैव तत्साधनं । शब्दत्वमात्रेण साधनत्वात् । तत्कथन्धर्मी स्वसाधनेऽसिद्ध उच्यते ॥ अयमभिप्रायः । न शब्दत्वमित्येव गमकत्वं किन्तु साध्यप्याप्तं तस्य चासाधारणत्वेन नान्यत्र व्याप्त्युपलम्भः । ततश्चान्यत्रानित्यत्वनियमात् शब्दे शब्दत्वं नित्यताव्याप्तं सद्धेतुर्व्वक्तव्यः । तथा च य एव साध्यः स एव हेतुरिति साध्यसिद्धेर्हेत्वसिद्धिश्चेति युक्तमुक्तं धर्म्यसिद्धः प्रसाधन इति ॥
	\pend
      

	  \begin{center}%% label @type='head'
	\textbf{ख. सामान्यनिरासः}
	\end{center}
	
	  \bigskip
	  \begingroup
	  \large
	
	    
	    \stanza[\smallbreak]
	\label{pv.4.181}\edlabel{pv.4.181}\flagstanza{\tiny\textenglish{....4.181}}धर्मधर्मिविवेकस्य सर्वभावेष्वसिद्धितः ।&सर्वत्र दोषस्तुल्यश्चेन्न संवृत्या विशेषतः ॥ १८१ ॥\&[\smallbreak]


	
	  \endgroup
	

	  \pstart नन्वेवं सति सर्व्वत्र भावेषु धर्मयाः साध्यसाधनयोर्द्धर्मिणश्च विवेकस्य भेदस्यासिद्धितः \edlabel{pvv.478-2}\footnote{\label{pvv.478-2}  २ तत्रैव---उपरिमस्य पतितमेतत् ।}  ।
	\pend
      \leavevmode\marginnote{\textenglish{479/s}}

	  \pstart \edlabel{pvv.479-1}\footnote{\label{pvv.479-1}  १ स्वभावहेतौ ।} {\color{DodgerBlue3}“सर्व्वत्र तुल्यो दोषः”} । नहि शब्दादन्यत् नित्यत्वं कृतकत्वम्वा । अनित्य-\leavevmode\marginnote{\textenglish{96a/MA}} त्वासिद्धौ च यथा नित्यः साध्यस्तथा तदात्मवान् कृतकोपि शब्दोऽसिद्ध ऐवति चेत् । नैष दोषः (।) {\color{DodgerBlue3}“संवृत्त्या”} भिन्नव्यावृत्तिविषयया साध्यसाधनधर्मिणो {\color{DodgerBlue3}“विशेषतो”} भेदात् । यथा स्वं विकल्पैः संकेतवासनानुगमनियमितैकैकवृत्तिमात्रविषयार्थैः शब्दत्वकृतकत्वनित्यत्वाद्यसंकीर्ण्णान्येव धर्मिसाधनसाध्यतया व्यवस्थाप्यन्त इति न दोषः ।
	\pend
      

	  \pstart यदि धर्मधर्मिविवेकोऽस्त्येव तदा तत्त्वान्यत्वप्रतिषेधः कथं कृत इत्याह ।
	\pend
      
	  \bigskip
	  \begingroup
	  \large
	
	    
	    \stanza[\smallbreak]
	\label{pv.4.182}\edlabel{pv.4.182}\flagstanza{\tiny\textenglish{....4.182}}परमार्थविचारेषु तथाभूताप्रसिद्धितः ॥&तत्त्वान्यत्वं पदार्थेषु सांबृतेषु निषिध्यते ॥ १८२ ॥\&[\smallbreak]


	
	  \endgroup
	

	  \pstart सांवृतेषु कल्पनाविषयेषु धर्मिधर्मादिषु तत्त्वान्यत्वं {\color{DodgerBlue3}“परमार्थस्य”} तत्त्वस्य {\color{DodgerBlue3}“विचारेषु”} निषिध्यते {\color{DodgerBlue3}“तथाभूतस्य”} यथाकल्पनं परस्परतो भिन्नस्य धर्मिधर्मादेः प्रमाणेना{\color{DodgerBlue3}“प्रसिद्धितः”} । न तु सांवृतोपि तेषां भेदाभावः । न चेयता कल्पिते धर्मिणि कल्पितात् साधनात् कल्पितस्य साध्यस्य सिद्धिरित्यनुमानादवस्तुप्रतीत्यभावप्रसङ्ग । अशब्दव्यावृत्त्या निश्चितस्य वस्तुन एव धर्मित्वात् । एवं कृतकत्वानित्यत्वाभ्यां निश्चितस्य तस्यैव साधनत्वात् साध्यत्वाच्चेत्युक्तेः । अन्योन्यस्य तेषां भेदः पुनः कल्पित एव ।
	\pend
      
	  \bigskip
	  \begingroup
	  \large
	
	    
	    \stanza[\smallbreak]
	\label{pv.4.183}\edlabel{pv.4.183}\flagstanza{\tiny\textenglish{....4.183}}अनुमानानुमेयार्थव्यवहारस्थितिस्त्वियम् ।&भेदं प्रत्ययसंसिद्धमवलम्ब्य च कल्प्यते ॥ १८३ ॥\&[\smallbreak]


	
	  \endgroup
	

	  \pstart अतो{\color{DodgerBlue3}“नुमान”}हेतुत्वादनुमानस्य लिङ्गस्या{\color{DodgerBlue3}“नुमेयार्थ”}स्यानयोरुपलक्षणत्वात् (।)
	\pend
      

	  \pstart धर्मिणश्च {\color{DodgerBlue3}“व्यवहारस्थितिस्त्विय”}मनित्यः कृतकत्वादित्यादौ क्रियमाणा तेषां परस्परतो {\color{DodgerBlue3}“भेद प्रत्यतेन”} विकल्पकेनैकव्यावृत्तिमात्रविषयेण {\color{DodgerBlue3}“संसिद्धं”} निश्चितं । अवलम्व्याश्रित्य च कल्प्यते ॥
	\pend
      
	  \bigskip
	  \begingroup
	  \large
	
	    
	    \stanza[\smallbreak]
	\label{pv.4.184}\edlabel{pv.4.184}\flagstanza{\tiny\textenglish{....4.184}}यथा स्वं भेदनिष्ठेषु प्रत्ययेषु विवेकिनः ।&धर्मी धर्माश्च भासन्ते व्यवहारस्तदाश्रयः ॥ १८४ ॥\&[\smallbreak]


	
	  \endgroup
	

	  \pstart तथाहि {\color{DodgerBlue3}“यथा स्वं”} यस्य य आत्मीयो ग्राह्यो {\color{DodgerBlue3}“भेदो”} व्यावृत्तिस्त{\color{DodgerBlue3}“न्निष्ठेषु”} विकल्पेषु {\color{DodgerBlue3}“विवेकिनो धर्मी धर्माश्च”} साध्यसाधनादयो {\color{DodgerBlue3}“भासन्ते । तदाश्रयो”} विकल्पगोचराश्रयो धर्मिधर्मादिभेदस्य {\color{DodgerBlue3}“व्यवहारः”} प्रवर्त्तते ॥
	\pend
      \leavevmode\marginnote{\textenglish{480/s}}

	  \pstart एवन्तर्हि नित्यः शब्दोऽसर्व्वत्वादिति । अत्रापि व्यावृत्तिभेदादेव साध्यसाधनभावो भविष्यतीत्याह ।
	\pend
      
	  \bigskip
	  \begingroup
	  \large
	
	    
	    \stanza[\smallbreak]
	\label{pv.4.185}\edlabel{pv.4.185}\flagstanza{\tiny\textenglish{....4.185}}व्यवहारोपनीतोत्र स एवाश्लिष्टभेदधीः ।&साध्यः साधनतां नीतस्तेनासिद्धः प्रकाशितः ॥ १८५ ॥\&[\smallbreak]


	
	  \endgroup
	

	  \pstart {\color{DodgerBlue3}“अत्र”} प्रयोगे {\color{DodgerBlue3}“स”} शब्द {\color{DodgerBlue3}“एव साध्योऽश्लिष्टभेदधी”}रसंस्पृष्टान्यादृशबुद्धिः । द्वाभ्यामपि शब्दाभ्यामेकस्या व्यावृत्तेः प्रतिपादनात् {\color{DodgerBlue3}“व्यवहारेण”} व्यावृत्तिसमाश्रयेणो{\color{DodgerBlue3}“पनीतः”} प्रत्युपस्थापितः येन कारणेन {\color{DodgerBlue3}“साधनतां नीतः तेनासिद्धः प्रकाशितः”} ।
	\pend
      

	  \pstart ननु संस्कृतशब्दोऽनित्यः संस्कृतत्वादिति प्रतिज्ञार्थैकदेशस्य यथा हेतुत्वं तथा नित्यः शब्दः शब्दत्वादित्यस्यापि स्यादित्याह ।
	\pend
      
	  \bigskip
	  \begingroup
	  \large
	
	    
	    \stanza[\smallbreak]
	\label{pv.4.186}\edlabel{pv.4.186}\flagstanza{\tiny\textenglish{....4.186}}भेदसामान्ययोर्द्धर्मभेदादंगांगिता ततः ।&यथाऽनित्यः प्रयत्नोत्थः प्रयत्नोत्थतया ध्वनिः ॥ १८६ ॥\&[\smallbreak]


	
	  \endgroup
	

	  \pstart साध्यधर्मिमात्रनिष्ठत्वात् सर्व्वधर्मगोचरत्वाच्च {\color{DodgerBlue3}“भेदसामान्ययोर्द्धर्मभेदाद्”} व्यावृत्तिभेदात् । साध्य\edlabel{pvv.480-1}\footnote{\label{pvv.480-1}  १ अनित्यं शब्दे ।}धर्मो हि साध्यधर्मिनिष्ठत्वेन सजातीयाद् विजातीयाच्च व्यावृत्तत्वाद् विशेषः । साधनधर्मस्तु विजातीयमात्रव्यावृत्तत्वेन सामान्यं । ततो भेदसामान्यभावेन {\color{DodgerBlue3}“भेदादङ्गाङ्गिता”} हेतुसाध्यता युक्ता । विशेषः साध्यः सामान्यं हेतुरिति कुतः प्रतिज्ञार्थैकदेशता । {\color{DodgerBlue3}“यथाऽनित्यः प्रयत्नोत्थो ध्वनिरिति”} प्रतिज्ञा {\color{DodgerBlue3}“प्रयत्नोत्थतयेति”} हेतुः । शब्दः पुनरभिन्न\edlabel{pvv.480-2}\footnote{\label{pvv.480-2}  २ नित्यः शब्दः शब्दत्त्वादित्यत्र ।}विषयो हेतुः साध्यश्चेति प्रतिज्ञार्थैकदेशत्वं ॥
	\pend
      

	  \pstart प्रयत्नानन्तरीयकत्वस्य धर्मिविशेषणत्वात् प्रतिज्ञार्थैकदेशत्वमस्त्येवेत्याह ।
	\pend
      
	  \bigskip
	  \begingroup
	  \large
	
	    
	    \stanza[\smallbreak]
	\label{pv.4.187}\edlabel{pv.4.187}\flagstanza{\tiny\textenglish{....4.187}}पक्षाङ्गत्वेप्यबाधत्वान्नासिद्धिर्भिन्नधर्मिणि ।&यथाश्वो न विषाणित्वादेष पिंडो विषाणवान् ॥ १८७ ॥\&[\smallbreak]


	
	  \endgroup
	\leavevmode\marginnote{\textenglish{97a/MA}}

	  \pstart पक्षाङ्गत्वे विशेषणत्वेपि नास्त्येव तावत् साधनस्य पक्षाङ्गत्वं विशेषसाध्यत्वात् । भवतु वा तथापि नासिद्धस्य । तेन विशेषणेन {\color{DodgerBlue3}“भिन्ने”} विशेषिते {\color{DodgerBlue3}“धर्मिणि”} धर्म्म्यन्तरव्या\edlabel{pvv.480-3}\footnote{\label{pvv.480-3}  ३ साध्यस्य धर्मिणो धर्मस्य वा साधनतया विरोध एकस्य ज्ञाप्यज्ञापकत्वविरोधात् (।) अत्र तु विशेषः साध्यः सामान्यं साधनं तत्त्वे प्रतिज्ञया सिद्धिप्रसक्तेः ।}वृत्ते प्रयत्नानन्तरीयकत्वस्याविरोधादबाधत्वात् धर्मिणं विशेषयदपि \leavevmode\marginnote{\textenglish{481/s}} प्रयत्नोत्थत्वं शब्दे प्रसिद्धमेव । {\color{DodgerBlue3}“यथा”} बहुषु पिण्डेषु दृश्यमानेषु {\color{DodgerBlue3}“किमयमश्वो”} नवेति संशयोऽभिधीयते {\color{DodgerBlue3}“विशा( ? षा)णवानेष पिण्डो”} धर्मी {\color{DodgerBlue3}“नाश्वो”} विषाणित्वादिति । विषाणित्वं धर्मिणं विशेषयदिप प्रमाणप्रतीतत्वान्नासिद्धो हेतुः ।
	\pend
      
	  \bigskip
	  \begingroup
	  \large
	
	    
	    \stanza[\smallbreak]
	\label{pv.4.188}\edlabel{pv.4.188}\flagstanza{\tiny\textenglish{....4.188}}साध्यकालाङ्गता वा न निवृत्तेरुपलक्ष्य तत् ।&विशेषोपि प्रतिज्ञार्थो धर्मभेदान्न युज्यते ॥ १८८ ॥\&[\smallbreak]


	
	  \endgroup
	

	  \pstart अथवा विशेषणस्य प्रयत्नोत्थत्वादेस्तच्छब्दरूपं विशेष्यत्वेनोपलक्ष्यानुमानात् प्रागेव निवृत्तेः (।) {\color{DodgerBlue3}“साध्यका”}लेऽनुमेयप्रतीते कालेऽ{\color{DodgerBlue3}“ङ्गता”} विशेषणता नास्त्येव अप्रतीतं ह्यनुमानात् प्रत्येतव्यं । प्रयत्नोत्थत्वादिति च प्रतीतं । न च तत्र विवादः । ततो धर्मिमात्रोपलक्षणन्तत् । यथा काको देवदत्तगृहोपलक्षणत्वान्न कार्योपयोगी (।) यद्यपि\edlabel{pvv.481-1}\footnote{\label{pvv.481-1}  १ यदि च ।} विशिष्टे धर्मिणि प्रतिज्ञार्थैकदेशत्वं हेतोः । तथा नित्यः शब्दः श्रावणत्वादिति शब्दस्वभावभूतस्य श्रावणत्वस्य शब्दत्ववत् प्रतिज्ञार्थैकदेशता स्यात् नासाधारणतेत्याह । न केवलं विषाणित्वादि{\color{DodgerBlue3}“र्व्विशेषः”} श्रावणत्वादि\leavevmode\marginnote{\textenglish{97b/MA}}रपि {\color{DodgerBlue3}“प्रतिज्ञार्थै”}कदेशो {\color{DodgerBlue3}“न युज्यते । धर्म”}स्य व्यावृत्ते{\color{DodgerBlue3}“र्भेदात्”} । श्रावणत्वं श्रवणग्राह्यताऽश्रावणव्यावृत्तिः (।) तच्च शब्देपि क्वचित् {\color{DodgerBlue3}“कञ्चित् पुरुषमपेक्ष्य भवति”} । अशब्दव्यावृत्तिस्तु शब्दत्वं तच्च सर्व्वत्रास्ति ततो व्यावृत्तिभेदात् शब्दे सिद्धस्य श्रावणत्वस्यान्यत्राननुवृत्तेरसाधारणतैव युक्तेत्युक्तं सपरिकरं पक्षलक्षणम् ॥
	\pend
      

	  \pstart इति पक्षलक्षण(म् । )
	\pend
      
	  
	% new div opening: depth here is 1
	
\section[{६. हेतुचिन्ता}]{६. हेतुचिन्ता}

	  \begin{center}%% label @type='head'
	\textbf{(२) हेतुलक्षणम्}
	\end{center}
	

	  \begin{center}%% label @type='head'
	\textbf{म. पक्षधर्मप्रभेदकथने कारणम्}
	\end{center}
	\label{div_pvv.4.189_4.190_4.191_4.192_4.193_4.194}\edlabel{div_pvv.4.189_4.190_4.191_4.192_4.193_4.194}
	  
	% new div opening: depth here is 2
	

	  \pstart हेतुलक्षणमिदानीम्वक्तव्यं । तत्र हेतुलक्षणमेव “ तत्र यः सन् सजातीये”इत्यादिकं युक्तं  वक्तुं । “सपक्षे सन्नसन् द्वेधा पक्षधर्मः पुनस्त्रिधे”त्यादिना नवधा पक्षधर्मप्रभेदस्तु कस्मादुक्त \edlabel{pvv.481-2}\footnote{\label{pvv.481-2}  २ स मु च्च ये ।} इत्याह ।
	\pend
      
	  \bigskip
	  \begingroup
	  \large
	
	    
	    \stanza[\smallbreak]
	\label{pv.4.189}\edlabel{pv.4.189}\flagstanza{\tiny\textenglish{....4.189}}पक्षधर्मप्रभेदेन सुखग्रहणसिद्धये ।&हेतुप्रकरणार्थस्य सूत्रसंक्षेप उच्यते ॥ १८९ ॥\&[\smallbreak]


	
	  \endgroup
	\leavevmode\marginnote{\textenglish{482/s}}

	  \pstart {\color{DodgerBlue3}“पक्षधर्म”}स्य नवधा {\color{DodgerBlue3}“प्रभेदेन हेतुप्रकरणार्थस्य”} हेतुहेत्वाभासलक्षणात्मकस्य {\color{DodgerBlue3}“सुखेन ग्रहण”}स्य {\color{DodgerBlue3}“सिद्धये”} तदर्थवाचकानां {\color{DodgerBlue3}“सूत्रा”}णां {\color{DodgerBlue3}“संक्षेपतः”} संग्रह {\color{DodgerBlue3}“उच्यते”} ॥ सपक्षे सन्नित्यादिना (।)
	\pend
      

	  \begin{center}%% label @type='head'
	\textbf{ख. सूत्रे निपातग्रहणफलम्}
	\end{center}
	

	  \pstart यदि पक्षस्य धर्मो हेतुस्तदा तद्विशेषणापेक्षस्य धर्मस्यान्यत्र धर्मिण्यननुवृत्तेरसाधारणता स्यात् । अथ पक्षेण न विशिष्यते तदा न पक्षधर्मो हेतुः स्यात् । असदेतत् । न ह्यन्ययोगव्यवच्छेदेनैव विशेषणमन्यथापि सम्भवादिति दर्शयितुमाह ।
	\pend
      
	  \bigskip
	  \begingroup
	  \large
	
	    
	    \stanza[\smallbreak]
	\label{pv.4.190}\edlabel{pv.4.190}\flagstanza{\tiny\textenglish{....4.190}}अयोगं योगमपरैरत्यन्तायोगमेव च ।&व्यवच्छिनत्ति धर्मस्य निपातो व्यतिरेचकः ॥ १९० ॥\&[\smallbreak]


	
	  \endgroup
	

	  \pstart {\color{DodgerBlue3}“निपात”} एवकारो {\color{DodgerBlue3}“व्यतिरेचकः”} नियामकः क्वचिद् {\color{DodgerBlue3}“धर्मस्य”} विशेषणस्या{\color{DodgerBlue3}“योगं व्यवच्छिनत्ति”} क्वचिदपरैर्व्विशेष्या{\color{DodgerBlue3}“दन्यैर्योगं”} व्यवच्छिनत्ति । क्वचि{\color{DodgerBlue3}“दत्यन्तायोगं”} व्यवच्छिनत्ति ॥
	\pend
      

	  \pstart ननु निपातो न स्वयं वाचकः किन्तु द्योतकः । तदस्य कथमयमर्थप्रभेद इत्याह ।
	\pend
      
	  \bigskip
	  \begingroup
	  \large
	
	    
	    \stanza[\smallbreak]
	\label{pv.4.191a}\edlabel{pv.4.191a}\flagstanza{\tiny\textenglish{...4.191a}}विशेषणविशेषाभ्यां क्रियया च सहोदितः ।\&[\smallbreak]


	
	  \endgroup
	

	  \pstart द्योतकत्वादेव निपातो {\color{DodgerBlue3}“विशेषणेन”} सहोदितोऽयोगस्य व्यवच्छेदकः । {\color{DodgerBlue3}“विशेष्येण”} सहोक्तोन्ययोगस्य । {\color{DodgerBlue3}“क्रियया च सहोक्तो”}ऽत्यन्तायोगस्येति विशेषणादिपदवाच्य एवायोगव्यवच्छेदादिस्तत्सहोक्तनिपातद्योत्य इत्यर्थः ।
	\pend
      

	  \pstart भवतु तावन्निपातप्रयोगे व्यवच्छेदविशेषस्य प्रतीतिः पक्ष\edlabel{pvv.482-1}\footnote{\label{pvv.482-1}  १ ...पक्षधर्मः ।} इत्यादौ तु कथमित्याह ।
	\pend
      
	  \bigskip
	  \begingroup
	  \large
	
	    
	    \stanza[\smallbreak]
	\label{pv.4.191b}\edlabel{pv.4.191b}\flagstanza{\tiny\textenglish{...4.191b}}विवक्षातोऽप्रयोगेपि सर्वोऽर्थोयं प्रतीयते ॥ १९१ ॥\&[\smallbreak]


	
	  \endgroup
	
	  \bigskip
	  \begingroup
	  \large
	
	    
	    \stanza[\smallbreak]
	\label{pv.4.192}\edlabel{pv.4.192}\flagstanza{\tiny\textenglish{....4.192}}व्यवच्छेदफलं वाक्यं यतश्चैत्रो धनुर्धरः ।&पार्थो धनुर्द्धरो नीलं सरोजमिति वा यथा ॥ १९२ ॥\&[\smallbreak]


	
	  \endgroup
	

	  \pstart {\color{DodgerBlue3}“अप्रयोगेपि”} निपातस्य वक्तु{\color{DodgerBlue3}“र्व्विवक्षातः”} सर्व्वोंयमयोगव्यवच्छेदा{\color{DodgerBlue3}“दिरर्थः प्रतीयते । यतो व्यवच्छेदफलं वाक्य”}मित्युक्तं प्राक् । वाक्यञ्चोपलक्षणं पदमपि व्यवच्छेदफलं । न हि घटेनोदकमानयेति प्रतिपादमनवधारणेऽघटेनानयनप्रतिषेधः । पक्ष इत्यादौ तु कथमित्याह । अप्रयोगेपि निपातस्य वक्तुर्व्विवक्षातः सर्व्वोयमयोगव्यवच्छेदोऽनुदकानयनप्रतिषेधः  अनानयननिवृत्तिर्व्वा शक्योपदर्शना । अयोग\leavevmode\marginnote{\textenglish{483/s}} व्यवच्छेदादीनामुदाहरणमाह । {\color{DodgerBlue3}“यथा चैत्रो धनुर्द्धरः । पार्थो धनुर्द्धरः । नीलं सरोजमिति”} (।) चैत्रे धनुर्द्धरत्वसन्देहाद् विशेषणेनायोगमात्रं व्यवच्छिद्यते । {\color{DodgerBlue3}“पार्थे”} धनुर्द्धरत्वं प्रसिद्धमेव (।) किन्तु तादृशमन्यस्यापि किमस्तीति सन्देहेऽन्ययोगव्यवच्छेदफलं विशेषणं । न खलु सर्व्वमेव नीलं सरोजं येनायोगव्यवच्छेदः स्यात् । नापि सरोजमेव नीलं येनान्ययोगव्यवच्छेदो भवेत् । किन्तु नीलं सरोजं संभवति न वेत्यन्तायोगसंदेहे विशेषणेन स एव व्यवच्छिद्यते ।
	\pend
      

	  \pstart ननु भवतु तावच्चैत्रो धनुर्द्धर एव । पार्थ एव धनुर्द्धरः । सरोजं नीलं संभवत्येवेति निपातप्रयोगे विवक्षावशात् विशेषणादिपदानामेव वा व्यवच्छेदप्रतिपादकत्वादप्रयोगेऽपि निपातस्य चैत्रो धनुर्द्धर इत्यादिप्रयोगेषु योगव्यवच्छेदादीनां \leavevmode\marginnote{\textenglish{98a/MA}} प्रतीतिः । पक्षधर्म इत्यत्र पुनःपक्षो विशेषणं । धर्मो विशेष्यः । तद् यदि पक्षस्यैव धर्म इति विशेषणेन सह निपात उच्यते अप्रयोगेपि वा विशेषणस्य तदर्थवृत्तितेष्यते । उप(?) भयथाप्यन्ययोगव्यवच्छेदप्रतिपादकत्वमेव स्यात् ।
	\pend
      

	  \pstart अथ धर्म एव विशेष्यो निपातसहचरः । तदर्थ वृत्तिर्व्वेष्टः तदाऽयोगव्यवच्छेदो लभ्यत एव परं  विशेषणसहितोऽन्ययोगं व्यवच्छिनत्तीत्युक्तेर्व्विरुध्यते । अत्रोच्यते । बाह्यो निपातः श्रूयमाणोपि विशेष्यादिभिः सह वक्तृविवक्षावशाद् गम्यमानो वाऽयोगव्यवच्छेदको भवतीति दृष्टान्तप्रदर्शनार्थमुक्तं । तथैव चैत्रो धनुर्द्धर इत्याद्युदाहरणप्रदर्शनात् । व्यवच्छेदफलं वाक्यमित्युक्तेश्च । समासे तु विशेषणमेवायोगादिव्यवच्छेदकं विवक्षावशादिष्टं । अयोगव्यवच्छेदेन विशेषणादित्युक्तेः । पक्षधर्म इति पक्षशब्दोऽयोगव्यवच्छेदकः । पक्षासम्बद्धो न भवतीत्यर्थः । चाक्षुषं रूपमिति चाक्षुषत्वस्य रूपे विवादाभावात् (।) शब्दादीनां विशेषणेन व्यवच्छेदः क्रियते । नीलोत्पलमिति । उत्पले नीलत्वनियमाभावात् (।) परेष्वभावादयोगान्ययोगयोर्व्यवच्छेदाभावात् नीलशब्देनात्यन्तासम्भवमात्रं व्यवच्छिद्यते । इति न कश्चिद्विरोधः ।
	\pend
      

	  \pstart ननु यथा पार्थ एव धनुर्द्धर इति विशेषणस्य सन्निधानात् निपातस्य विशेष्यान्तरव्यवच्छेदः । तथा विशेषणसन्निधानादवधारणस्य चैत्रो धनुर्द्धर एवेति गुणान्तरव्यवच्छेदः स्यादित्याह ।
	\pend
      
	  \bigskip
	  \begingroup
	  \large
	
	    
	    \stanza[\smallbreak]
	\label{pv.4.193}\edlabel{pv.4.193}\flagstanza{\tiny\textenglish{....4.193}}प्रतियोगिव्यवच्छेदस्तत्राप्यर्थेषु गम्यते ।&तथा प्रसिद्धेः सामर्थ्याद् विवक्षानुगमाद् ध्वनेः ॥ १९३ ॥\&[\smallbreak]


	
	  \endgroup
	

	  \pstart {\color{DodgerBlue3}“तत्र”} विशेषणादिष्व{\color{DodgerBlue3}“र्थेषु”} व्यवच्छेदेपि क्रियमाणे प्रकरणाद् बुद्धिविषयीकृतस्य {\color{DodgerBlue3}“प्रतियोगि”}न {\color{DodgerBlue3}“व्यवच्छेदो”} विशेषणेन {\color{DodgerBlue3}“गम्यते”} नान्यस्य । {\color{DodgerBlue3}“तथा प्रसिद्धेः”} प्रतियोगिन \leavevmode\marginnote{\textenglish{484/s}} एव बुद्धिस्थीकृत्य विशेषणेन व्यवच्छेदो नेतरस्येति लोकप्रसिद्धेः । {\color{DodgerBlue3}“विवक्षाया अनुगमात् ध्वने”}र्व्यवच्छेदादौ {\color{DodgerBlue3}“सामर्थ्या”}न्नाविवक्षितव्यवच्छेदः । यत एवायोगव्यवच्छेदोप्यस्ति (।)
	\pend
      
	  \bigskip
	  \begingroup
	  \large
	
	    
	    \stanza[\smallbreak]
	\label{pv.4.194}\edlabel{pv.4.194}\flagstanza{\tiny\textenglish{....4.194}}तदयोगव्यवच्छेदाद् धर्मी-धर्मविशेषणम् ।&तद्विशिष्टतया धर्मो न निरन्वयदोषभाक् ॥ १९४ ॥\&[\smallbreak]


	
	  \endgroup
	

	  \pstart {\color{DodgerBlue3}“तत्”} तस्माद् धर्मी पक्षो धर्मस्या{\color{DodgerBlue3}“योगव्यवच्छेदाद् विशेषणं”} धर्मिणो नाधर्मो\edlabel{pvv.484-1}\footnote{\label{pvv.484-1}  १ अपि तु धर्म एव ।} हेतुरित्यर्थः । अयोगव्यवच्छेदात् तेन धर्मिणा {\color{DodgerBlue3}“विशिष्टतया धर्मो निरन्वयदोषभाग् न”} भवति ।
	\pend
      

	  \pstart सपक्षे घटादौ सन् शब्दानित्यत्वे साध्ये कृतकत्वं हेतुः । असन् सपक्षे व्योमादौ शब्दानित्यत्वे साध्ये कृतकत्वं हेतुः । सपक्षे द्वेधा । सन्नसँश्च (।) शब्दानित्यत्वे साध्ये यत्नजत्वं हेतुः । घटादौ सपक्षे सन् विद्युदादौ चासन् । पुनस्त्रिधा (।) सपक्षे सन् असन् सदसँश्च (।) शब्दस्य यत्नजत्वे साध्ये सपक्षे घटादावनित्यत्वं हेतुः सन् । शब्दानित्यत्वे साध्ये यत्नजत्वं हेतुरसन् विद्युदादौ सपक्षे ॥ शब्देऽयत्नजत्वे साध्येऽनित्यत्वं हेतुः सपक्षे विद्युदादौ सन् । व्योमादौ चासन् । एवं प्रत्येकमसपक्षेपि सन् असन् द्वेधा चेति योज्यं । शब्दनित्यत्वे साध्ये प्रमेयत्वं हेतुः (।) असपक्षे घटादौ सन् (।) शब्दनित्यत्वे साध्ये श्रावणत्वं हेतुरसपक्षेऽसन् शब्दनित्यत्वे साध्येऽस्पर्शवत्वं हेतुरसपक्षे घटादावसन् (।) बुद्ध्यादौ सन्निति द्विधा \leavevmode\marginnote{\textenglish{98b/MA}} पक्षधर्मनिर्देशः । किमर्थ हेतुप्रकरणे नवधा पक्षधर्मनिर्देशः ।---
	\pend
      \label{div_pvv.4.195_4.196_4.197_4.198}\edlabel{div_pvv.4.195_4.196_4.197_4.198}
	  
	% new div opening: depth here is 2
	

	  \begin{center}%% label @type='head'
	\textbf{(२) हेतुभेदा}
	\end{center}
	

	  \pstart ---सम्यग्धेतुरसिद्धविरुद्धानैकान्तिकहेत्वाभासा एव युक्तनिर्देशा इत्याह ।
	\pend
      
	  \bigskip
	  \begingroup
	  \large
	
	    
	    \stanza[\smallbreak]
	\label{pv.4.195}\edlabel{pv.4.195}\flagstanza{\tiny\textenglish{....4.195}}स्वभावकार्यसिध्यर्थं द्वौ द्वौ हेतुविपर्ययौ ।&विवादाद् भेदसामान्ये शेषो व्यावृत्तिसाधनः ॥ १९५ ॥\&[\smallbreak]


	
	  \endgroup
	

	  \pstart {\color{DodgerBlue3}“स्वभावकार्य”}योरेव हेतुत्वेन {\color{DodgerBlue3}“सिद्ध्यर्थं”} तत्र {\color{DodgerBlue3}“द्वौ”} शब्देऽनित्यत्वसिद्ध्यर्थं कृतकत्वप्रयत्नानन्तरीयकत्वाख्यौ हेतू निर्दिष्टौ । तथा शब्द एव नित्यत्वसाधने द्वौ {\color{DodgerBlue3}“हेतुविपर्ययौ”} विरुद्धौ हेतुभावे चोक्तौ । यथा हि सम्यग्धेतोः स्वसाध्ये व्याप्यकार्यतया प्रतिबद्धस्य गमकत्वं । तथा साध्यविपर्यये व्याप्यकार्यतया प्रतिबद्धस्यैव \leavevmode\marginnote{\textenglish{485/s}} तद्गमकत्वेन विरुद्धता नान्यस्येत्यर्थः । व्यतिरेकी\edlabel{pvv.485-1}\footnote{\label{pvv.485-1}  १ नैयायिकस्य ।}अन्वयी च हेतुरिति परेषां {\color{DodgerBlue3}“विवादात्”} तत्प्रतिषेधार्थं {\color{DodgerBlue3}“भेदसामान्ये”}ऽसाधारणसाधारणे श्रावणत्वप्रमेयत्वे निर्द्दिष्टे । यदि विपक्षे नास्तीति सात्मकत्वे साध्ये प्राणादिमत्वं हेतुस्तदा श्रावणत्वमपि स्यात् । न चैवं (।) तस्मान्न व्यतिरेकी हेतुः । यदि च केवलान्वयिनो दर्शनमात्राध्यवसिताव्यभिचारस्य हेतुत्वं तदा प्रमेयत्वस्याकाशादौ दर्शनादव्यभिचारनिश्चये सति शब्दे नित्यत्वगमकत्वं स्यान्न चैवं । ततो न केवलान्वयी हेतुः । {\color{DodgerBlue3}“शेषो”}ऽप्रयत्नोत्थः शब्दोऽनित्यत्वात् । नित्यः शब्दोऽस्पर्शवत्वात् । प्रयत्नानन्तरीयः शब्दोऽनित्यत्वादिति हेतुत्रयं विपक्षाद्धेतो{\color{DodgerBlue3}“र्व्यावृत्तिसाधनः”} ।---
	\pend
      

	  \pstart ---यदि हि सपक्षे दर्शनमात्रेण गमकत्वं तदैते हेतवः प्राप्ताः (।) {\color{DodgerBlue3}“सर्व्वेषां सपक्षे”} सत्वात् (।) तस्मान्नान्वयसम्बन्धमात्रेण गमकत्वं (।) किन्तु विपक्षाद् व्यतिरेकनिश्चये न चास्ति व्यतिरेकनिश्चयः (।) प्रधानं हेतुत्वनिबन्धनमित्यर्थः । कथं पुनर्ज्ञायते स्वभावहेतुः कृतकत्वं । कार्यहेतुः प्रयत्नानन्तरीयकमित्याह ।
	\pend
      
	  \bigskip
	  \begingroup
	  \large
	
	    
	    \stanza[\smallbreak]
	\label{pv.4.196}\edlabel{pv.4.196}\flagstanza{\tiny\textenglish{....4.196}}न हि स्वभावादन्येन व्याप्तिर्गम्यस्य कारणे ।&सम्भवाद् व्यभिचारस्य द्विधावृत्तिफलं ततः ॥ १९६ ॥\&[\smallbreak]


	
	  \endgroup
	

	  \pstart {\color{DodgerBlue3}“न हि स्वभावा”}द्धेतो{\color{DodgerBlue3}“रन्येन”} हेतुना {\color{DodgerBlue3}“गम्यस्य”} साध्यस्य {\color{DodgerBlue3}“व्याप्तिः”} । यत्र यत्रानित्यत्वं तत्र कृतकत्वमिति दृश्यते च व्याप्तिः (।) तस्मात् स्वभावहेतुरेव\edlabel{pvv.485-2}\footnote{\label{pvv.485-2}  २ कारणेपि व्याप्तिः स्यादित्याह ।} । नावश्यं कारणानि कार्यवन्ति भवन्तीति कारणे कार्यस्य {\color{DodgerBlue3}“व्यभिचारसम्भवात्”} । कार्यं कारणव्यापकं न भवति । \edlabel{pvv.485-3}\footnote{\label{pvv.485-3}  ३ स्वभावहेतौ ।} तत्र सपक्षे {\color{DodgerBlue3}“द्विधावृत्ति”}भावाभावात् प्रत्यत्ना\edlabel{pvv.485-4}\footnote{\label{pvv.485-4}  ४ शब्दोच्चारणप्रयत्नानन्तरजं शब्दालम्बनं ज्ञानं ।}नन्तरीयत्वं {\color{DodgerBlue3}“फलं”} कार्यहेतुः\edlabel{pvv.485-5}\footnote{\label{pvv.485-5}  ५ भवार्थे गहादित्वाच्छः प्रयत्नानन्तरीयकत्वं घटादौ सत् विद्युदादौ चासदिति सपक्षे द्विधावृत्तिर्यतः ।} प्रयत्नकार्यस्य तथाभिधानात् ।
	\pend
      

	  \pstart ननु प्रयत्नजेन ज्ञानेनानित्यः शब्दोऽनुमेय इष्टः (।) शब्दाश्च नित्या एव तद्विषयज्ञानोत्पादादिना यत्नेन ते व्यज्यन्ते(।)तत्कथं कार्यहेतूदाहरणमिदमित्याह ।
	\pend
      
	  \bigskip
	  \begingroup
	  \large
	
	    
	    \stanza[\smallbreak]
	\label{pv.4.197}\edlabel{pv.4.197}\flagstanza{\tiny\textenglish{....4.197}}प्रयत्नानन्तरं ज्ञानं प्राक्सतो नियमेन न ।&तस्यावृत्यक्षशब्देषु सर्वथाऽनुपयोगतः ॥ १९७ ॥\&[\smallbreak]


	
	  \endgroup
	

	  \pstart प्रयत्नात् {\color{DodgerBlue3}“प्राक्”} सतः शब्दस्य {\color{DodgerBlue3}“नियमेन प्रयत्नानन्तरं ज्ञानं न”} यज्यते । प्रयत्नं विनापि कदाचिदुपलभ्येत । न च प्रयत्नव्यज्यता शब्दानां युक्ता । {\color{DodgerBlue3}“तस्य”} प्रयत्नस्य \leavevmode\marginnote{\textenglish{486/s}} {\color{DodgerBlue3}“आवृता”}वुपलम्भावरणे शब्दविषयज्ञानजनके श्रोत्रे {\color{DodgerBlue3}“शब्देषु”} च विषयेषु {\color{DodgerBlue3}“सर्व्वथा”}ऽकिञ्चित्करत्वे{\color{DodgerBlue3}“नानुपयोगत”} इत्युक्तं प्राक्\edlabel{pvv.486-1}\footnote{\label{pvv.486-1}  १ श्रुतिपरीक्षायां ।} ।
	\pend
      

	  \pstart किञ्च (।)
	\pend
      
	  \bigskip
	  \begingroup
	  \large
	
	    
	    \stanza[\smallbreak]
	\label{pv.4.198}\edlabel{pv.4.198}\flagstanza{\tiny\textenglish{....4.198}}कदाचिन्निरपेक्षस्य कार्याऽकृतिविरोधतः ।&कादाचित्कफलं सिद्धं तल्लिङ्गं ज्ञानमीदृशम् ॥ १९८ ॥\&[\smallbreak]


	
	  \endgroup
	

	  \pstart सहकारिभिरनाधेयातिशयत्वेन्नित्यस्य {\color{DodgerBlue3}“कदाचित्”} प्रयत्नकाले कार्यस्य ज्ञानस्य(ा)कृतिविरोधतः कारणात् तच्छ्रुतिविषयं ज्ञानं {\color{DodgerBlue3}“कादाचित्क”}स्यानित्यस्य शब्दस्य {\color{DodgerBlue3}“फलं लिङ्गं”} कार्य{\color{DodgerBlue3}“मीदृशं”} नियमेन प्रयत्नानन्तरभावि {\color{DodgerBlue3}“सिद्धम्”} ॥ (१९८)
	\pend
      \label{div_pvv.4.199}\edlabel{div_pvv.4.199}
	  
	% new div opening: depth here is 2
	

	  \begin{center}%% label @type='head'
	\textbf{(३) कार्यस्वभावहेत्वोर्निर्देशस्य फलम्}
	\end{center}
	
	  \bigskip
	  \begingroup
	  \large
	
	    
	    \stanza[\smallbreak]
	\label{pv.4.199}\edlabel{pv.4.199}\flagstanza{\tiny\textenglish{....4.199}}एतावतैव सिद्धेपि स्वभावस्य पृथक् कृतिः ।&कार्येण सह निर्द्देशे मा ज्ञासीत् सर्वमीदृशम् ॥ १९९ ॥\&[\smallbreak]


	
	  \endgroup
	

	  \pstart एतावता प्रयत्नानन्तरीयकत्वेनैव स्वभावहेतुनिर्देशेपि सिद्धे प्रयत्नानन्तरमुत्पादस्याभिव्यक्तेश्च तथाभिधानात् । तथापि स्वभावस्य कृतकत्वस्य हेतोर्या पृथक्कृतिः सा कार्येण सह श्लेषेण निर्देशे मा ज्ञासीत् प्रतिपत्ता सर्व्वं स्वभावहेतुमी\leavevmode\marginnote{\textenglish{99a/MA}}दृशं सपक्षे द्विधावृत्तिं कृतकत्वादेः विपर्ययव्याप्तिसम्भवात् । (१९९)
	\pend
      \label{div_pvv.4.200}\edlabel{div_pvv.4.200}
	  
	% new div opening: depth here is 2
	

	  \pstart किञ्च (।)
	\pend
      
	  \bigskip
	  \begingroup
	  \large
	
	    
	    \stanza[\smallbreak]
	\label{pv.4.200}\edlabel{pv.4.200}\flagstanza{\tiny\textenglish{....4.200}}व्युत्पत्त्यर्था च हेतूक्तिरुक्तार्थानुमितौ कृता ।&अत्र प्रभेद आख्यातः लक्षणन्तु न भिद्यते ॥ २०० ॥\&[\smallbreak]


	
	  \endgroup
	

	  \pstart अनुमितौ स्वार्थानुमाने {\color{DodgerBlue3}“हेतूक्तिरुक्तार्था”}ऽभिहितलक्षणापि प्रतिपत्तॄणां व्युत्प{\color{DodgerBlue3}“त्त्यर्था च”} परार्थानुमाने {\color{DodgerBlue3}“कृता”} । अत्र च {\color{DodgerBlue3}“प्रभेद आख्यातः । लक्षणं”} पुनर्हेतोर्न {\color{DodgerBlue3}“भिद्यते”} ।\edlabel{pvv.486-2}\footnote{\label{pvv.486-2}  २ स्वार्थानुमानोक्तात् ।}तथाविधलक्षणहेतुवचनस्य परार्थानुमानत्वात् । (२००)
	\pend
      \label{div_pvv.4.201}\edlabel{div_pvv.4.201}
	  
	% new div opening: depth here is 2
	
	  \bigskip
	  \begingroup
	  \large
	
	    
	    \stanza[\smallbreak]
	\label{pv.4.201}\edlabel{pv.4.201}\flagstanza{\tiny\textenglish{....4.201}}तेनात्र कार्यलिङ्गेन स्वभावोप्येकदेशभाक् ।&सदृशोदाहृतिश्चातः प्रयत्नाद् व्यक्तिजन्मनः ॥ २०१ ॥\&[\smallbreak]


	
	  \endgroup
	

	  \pstart {\color{DodgerBlue3}“तेन”} प्रयत्नान्तरीयकत्वेन {\color{DodgerBlue3}“कार्यलिङ्गेन”} श्लेषनिर्देशात् । स्वभावहेतुधर्मभाजा {\color{DodgerBlue3}“स्वभावोपि”} सप{\color{DodgerBlue3}“क्षैकदेशभाक्”} भवति इत्यक्तो भवति । {\color{DodgerBlue3}“अत”} एव कार्य\leavevmode\marginnote{\textenglish{487/s}} स्वभावतया {\color{DodgerBlue3}“सदृश”}स्य प्रयत्नानन्तरीयकत्व{\color{DodgerBlue3}“स्योदाहृति”}रा चा र्ये ण कृता । {\color{DodgerBlue3}“प्रयत्नाद्\edlabel{pvv.487-1}\footnote{\label{pvv.487-1}  १ उपलब्धेः कार्यत्वं ज्ञानत्वात् । उत्पत्तेः स्वभावत्वं शब्दरूपत्वात् ज्ञानस्य ।} व्यक्तेर्जन्म”}नश्च भावात् । यदा प्रयत्नाद् व्यक्तिस्तदा कार्यहेतुः । यदा जन्म\edlabel{pvv.487-2}\footnote{\label{pvv.487-2}  २ शब्दस्य ।}तदा स्वभावहेतुः । (२०१)
	\pend
      \label{div_pvv.4.202}\edlabel{div_pvv.4.202}
	  
	% new div opening: depth here is 2
	

	  \pstart कार्यस्वभावयोः प्रभेदनिर्देशस्य किं फलमित्याह ।
	\pend
      
	  \bigskip
	  \begingroup
	  \large
	
	    
	    \stanza[\smallbreak]
	\label{pv.4.202}\edlabel{pv.4.202}\flagstanza{\tiny\textenglish{....4.202}}यन्नान्तरीयका सत्ता यो वात्मन्यविभागवान् ।&स तेनाव्यभिचारी स्यादित्यर्थं तत्प्रभेदनम् ॥ २०२ ॥\&[\smallbreak]


	
	  \endgroup
	

	  \pstart {\color{DodgerBlue3}“यन्नान्तरीयका सत्ता”} भेदे सति यमन्तरेण हेतु\edlabel{pvv.487-3}\footnote{\label{pvv.487-3}  ३ कार्यहेतुः ।}र्न भवति । {\color{DodgerBlue3}“यो वा”} स्व आत्मीयः \edlabel{pvv.487-4}\footnote{\label{pvv.487-4}  ४ स्वभावः ।}साध्या{\color{DodgerBlue3}“दविभागवान्”} (।)\edlabel{pvv.487-5}\footnote{\label{pvv.487-5}  ५ हेत्वन्तरानपेक्षत्वार्थं ।} आत्मा स्वभावः {\color{DodgerBlue3}“स तेन”} कारणेन {\color{DodgerBlue3}“व्यापकेन”} वाऽ{\color{DodgerBlue3}“व्यभिचारी”} व्यभिचाररहितः {\color{DodgerBlue3}“स्यात्”} नान्य {\color{DodgerBlue3}“इत्यर्थ”}मेतत् प्रयोजनं (।) तयोः कार्यस्वभावयोः प्रभेदनं । (२०२)
	\pend
      \label{div_pvv.4.203}\edlabel{div_pvv.4.203}
	  
	% new div opening: depth here is 2
	

	  \pstart एवञ्च सति (।)
	\pend
      
	  \bigskip
	  \begingroup
	  \large
	
	    
	    \stanza[\smallbreak]
	\label{pv.4.203}\edlabel{pv.4.203}\flagstanza{\tiny\textenglish{....4.203}}संयोग्यादिषु येष्वस्ति प्रतिबन्धो न तादृशः ।&न ते हेतव इत्युक्तं व्यभिचारस्य सम्भवात् ॥ २०३ ॥\&[\smallbreak]


	
	  \endgroup
	

	  \pstart {\color{DodgerBlue3}“संयोगि”}\edlabel{pvv.487-6}\footnote{\label{pvv.487-6}  ६ वैशेषिकादिकल्पिताः ।} समवायि एकार्थसमवायि आकाशा{\color{DodgerBlue3}“दिषु”} पराभिमतेषु हेतुषु {\color{DodgerBlue3}“येषु प्रतिबन्धः तादृश”}स्तादात्म्यतदुत्पत्तिलक्षणो {\color{DodgerBlue3}“नास्ति (।) न ते हेतव इत्युक्तं भवति”} । अतदात्मनोऽतदुत्पत्तेश्च साध्य{\color{DodgerBlue3}“व्यभिचारस्य सम्भवात्”} । (२०३)
	\pend
      \label{div_pvv.4.204}\edlabel{div_pvv.4.204}
	  
	% new div opening: depth here is 2
	

	  \pstart अथ संयोग्यादिषु तदुत्पत्तिप्रतिबन्धोस्ति तदा (।)
	\pend
      
	  \bigskip
	  \begingroup
	  \large
	
	    
	    \stanza[\smallbreak]
	\label{pv.4.204}\edlabel{pv.4.204}\flagstanza{\tiny\textenglish{....4.204}}सति वा प्रतिबन्धेस्तु स एव गतिसाधनः ।&नियमो ह्यविनाभावोऽनियतश्च न साधनम् ॥ २०४ ॥\&[\smallbreak]


	
	  \endgroup
	

	  \pstart {\color{DodgerBlue3}“सति प्रतिबन्धे स एव गतिसाधनः”} साध्यप्रतिपत्तिहेतुरस्तु निष्फला संयोग्यादिकल्पना (।) {\color{DodgerBlue3}“नियमो”} नियतत्वं {\color{DodgerBlue3}“हि”} साधनस्य साक्षा{\color{DodgerBlue3}“दविनाभाव”} उच्यते (।) स च तादात्म्यतदुत्पत्तिभ्यां नान्यथा (।) {\color{DodgerBlue3}“स च”} तद्विकलत्वात् साध्ये{\color{DodgerBlue3}“ऽनियतः”} । न स साधनं । यथा संयोगित्वेपि स वह्निर्द्धूमस्येति“स्वभावकार्यसिद्ध्यर्थं द्वौ द्वौ हेतुविपर्ययावि” ति व्याख्यातं । (२०४)
	\pend
      \label{div_pvv.4.205}\edlabel{div_pvv.4.205}
	  
	% new div opening: depth here is 2
	\leavevmode\marginnote{\textenglish{488/s}}

	  \begin{center}%% label @type='head'
	\textbf{(४) विवादाद् भेदसामान्य इत्यस्य व्याख्यानम्}
	\end{center}
	

	  \pstart विवादाद् भेदसामान्य इति व्याख्यातव्यं ।
	\pend
      

	  \pstart स्यात् प्राणादिमत्वं हेतुर्यदि विपक्षाद्धेतुव्यतिरेकः स्यात् (।) स एव तु न सिध्यतीत्याह ।
	\pend
      
	  \bigskip
	  \begingroup
	  \large
	
	    
	    \stanza[\smallbreak]
	\label{pv.4.205}\edlabel{pv.4.205}\flagstanza{\tiny\textenglish{....4.205}}ऐकान्तिकत्वं व्यावृत्तेरविनाभाव उच्यते ।&तच्च नाप्रतिबद्धेषु तत एवान्वयस्थितिः ॥ २०५ ॥\&[\smallbreak]


	
	  \endgroup
	

	  \pstart विपक्षाद्भेदो {\color{DodgerBlue3}“व्यावृत्ते”}र्ब्बाधकप्रमाणनिश्चितत्वात् । {\color{DodgerBlue3}“ऐकान्तिकत्वमविनाभाव उच्यते (।) तच्च”} व्यावृत्तेरैकान्तिकत्वं स्वसाध्या{\color{DodgerBlue3}“प्रतिबद्धेषु”} प्राणादिषु नास्ति । यदि ह्यात्मनि प्रतिबद्धाः प्राणादयः तदात्मनिवृत्तौ निवृत्ता विपक्षाद् गम्येरन् (।) अन्यथा तु पक्ष एव सन्देहः (।) किममी आत्माभावेपि वर्त्तन्ते उत नेति (।) यथा च प्रतिबन्धाद् व्यतिरेकनिश्चयः । तथा ततः प्रतिबन्धादेवा{\color{DodgerBlue3}“न्वयस्य स्थितिः”} । न तु सहदर्शनमात्रेण । (२०५)
	\pend
      \label{div_pvv.4.206}\edlabel{div_pvv.4.206}
	  
	% new div opening: depth here is 2
	

	  \pstart तस्मात् साध्यस्य (।)
	\pend
      
	  \bigskip
	  \begingroup
	  \large
	
	    
	    \stanza[\smallbreak]
	\label{pv.4.206}\edlabel{pv.4.206}\flagstanza{\tiny\textenglish{....4.206}}स्वात्मत्वे हेतुभावे वा सिद्धे हि व्यतिरेकिता ।&सिध्येदतो विशेषे न व्यतिरेको न चान्वयः ॥ २०६ ॥\&[\smallbreak]


	
	  \endgroup
	

	  \pstart {\color{DodgerBlue3}“स्वात्मत्वे”} हेतुस्वभावात्मकत्वे {\color{DodgerBlue3}“हेतुभावे वा सिद्धे”} सति (तदभावे नियमेन व्यतिरेकात्) विपक्षाद् व्यापककारणव्यतिरेके स्वभावकार्यहेत्वो{\color{DodgerBlue3}“र्व्यतिरेकिता”} सिध्येत् । नान्यथेति न्याय एषः । यतः प्रतिबन्धेनैवान्वयव्यतिरेकसिद्धिः । अतो \leavevmode\marginnote{\textenglish{99b/MA}}विशेषेऽसाधारणे हेतौ नान्वयो न वा व्यतिरेकः सिध्यति ॥ (२०६)
	\pend
      \label{div_pvv.4.207}\edlabel{div_pvv.4.207}
	  
	% new div opening: depth here is 2
	

	  \pstart एवं तर्हि सपक्षविपक्षव्यतिरेकाभ्यां व्यावृत्तेर्विशेषः कथ मा चा र्ये णोक्तः इत्याह ।
	\pend
      
	  \bigskip
	  \begingroup
	  \large
	
	    
	    \stanza[\smallbreak]
	\label{pv.4.207}\edlabel{pv.4.207}\flagstanza{\tiny\textenglish{....4.207}}अदृष्टिमात्रमादाय केवलं व्यतिरेकिता ।&उक्ताऽनैकान्तिकस्तस्मादन्यथा गमको भवेत् ॥ २०७ ॥\&[\smallbreak]


	
	  \endgroup
	

	  \pstart {\color{DodgerBlue3}“अदृष्टिमात्रं”} बादकप्रमाणरहित{\color{DodgerBlue3}“मादाय”} पराभिप्रायेण सपक्षाद् {\color{DodgerBlue3}“व्यतिरेकितोक्ता”} । तस्यादर्शनमात्रेण व्यतिरेकानिश्चया{\color{DodgerBlue3}“दनैकान्तिक”} आचार्येणोक्तः । {\color{DodgerBlue3}“अन्यथा”} विपक्षाद् व्यतिरेकनिश्चये {\color{DodgerBlue3}“गमको”} हेतु{\color{DodgerBlue3}“र्भवेत्”} । ततश्चानैकान्तिकवर्गे न प्रक्षिप्येत । (२०७)
	\pend
      \label{div_pvv.4.208}\edlabel{div_pvv.4.208}
	  
	% new div opening: depth here is 2
	\leavevmode\marginnote{\textenglish{489/s}}

	  \begin{center}%% label @type='head'
	\textbf{(५) साध्याभावस्य साधनाभावेन न व्याप्तता}
	\end{center}
	

	  \begin{center}%% label @type='head'
	\textbf{क. जीवच्छरीरं प्राणादिमत्त्वादित्यत्र दोषः}
	\end{center}
	

	  \pstart ननु साध्यनिवृत्तौ\edlabel{pvv.489-1}\footnote{\label{pvv.489-1}  १ तादात्म्यनिवृत्ति ।} नियमेन साधनं निवर्त्तत इति {\color{DodgerBlue3}“साध्याभावः साधनाभावेन”} व्याप्तः । ततश्च (।)
	\pend
      
	  \bigskip
	  \begingroup
	  \large
	
	    
	    \stanza[\smallbreak]
	\label{pv.4.208}\edlabel{pv.4.208}\flagstanza{\tiny\textenglish{....4.208}}प्राणाद्यभावो नैरात्म्यव्यापीति विनिवर्त्तने ।&आत्मनो विनिवर्त्तेत प्राणादिर्यदि तच्च न ॥ २०८ ॥\&[\smallbreak]


	
	  \endgroup
	

	  \pstart {\color{DodgerBlue3}“प्राणादेः”} साधनस्या{\color{DodgerBlue3}“भावो नैरात्म्य”}स्य सात्मकत्वसाध्याभावस्य {\color{DodgerBlue3}“व्यापीति”} हेतो{\color{DodgerBlue3}“रात्मनो”} घटादेर्व्विपक्षाद् {\color{DodgerBlue3}“विनिवर्त्तने”} सति {\color{DodgerBlue3}“प्राणादिर्व्विनिवर्त्त”}ते । न ह्यन्यथा साधनाभावः साध्याभावस्य च व्यापको भवति । ततो यत्र प्राणादिभावनिवृत्तिस्तत्र सात्मकत्वभावनिवृत्तिरपीति यद्युच्यते {\color{DodgerBlue3}“तच्च न”} युक्तं । (२०८)
	\pend
      \label{div_pvv.4.209_4.210}\edlabel{div_pvv.4.209_4.210}
	  
	% new div opening: depth here is 2
	
	  \bigskip
	  \begingroup
	  \large
	
	    
	    \stanza[\smallbreak]
	\label{pv.4.209a}\edlabel{pv.4.209a}\flagstanza{\tiny\textenglish{...4.209a}}अन्यस्य विनिवृत्यान्यविनिवृत्त्येरयोगतः ।&तदात्मा तत्प्रसूतिश्चेन्नैतत्;\&[\smallbreak]


	
	  \endgroup
	

	  \pstart {\color{DodgerBlue3}“अन्यस्य”} कारणस्यात्मनो {\color{DodgerBlue3}“विनिवृत्त्या”} प्राणादे{\color{DodgerBlue3}“र्व्विनिवृत्तेरयोगतः”} । तादात्म्यतदुत्पत्तावेव हि साध्यनिवृत्तौ साधननिवृत्तिर्युक्ता तत्स्वभावत्वात् तदायत्तत्वाच्च । सति च साध्यसाधनयोर्नियमेन निवर्त्त्यनिवर्त्तकभावे साध्याभावः साधनाभावेन व्याप्यते । प्राणादि{\color{DodgerBlue3}“स्त”}स्यात्मन {\color{DodgerBlue3}“आत्मा”} स्वभावः । {\color{DodgerBlue3}“त”}स्मात् {\color{DodgerBlue3}“प्र”}सूतिर्व्वास्येति चेत् । {\color{DodgerBlue3}“नैतद”}स्ति युक्तं\edlabel{pvv.489-2}\footnote{\label{pvv.489-2}  २ न चेष्टं परस्य केवलमेतावानेव प्रतिबन्धे इत्युपन्यासः ।} । तथा हि (।)
	\pend
      
	  \bigskip
	  \begingroup
	  \large
	
	    
	    \stanza[\smallbreak]
	\label{pv.4.209b}\edlabel{pv.4.209b}\flagstanza{\tiny\textenglish{...4.209b}}आत्मोपलम्भने ॥ २०९ ॥\&[\smallbreak]


	
	  \endgroup
	
	  \bigskip
	  \begingroup
	  \large
	
	    
	    \stanza[\smallbreak]
	\label{pv.4.210}\edlabel{pv.4.210}\flagstanza{\tiny\textenglish{....4.210}}तस्योपलब्धावगतावगतौ च प्रसिध्यति ।&ते चात्यन्तपरोक्षस्य दृष्ट्यदृष्टी न सिध्यतः ॥ २१० ॥\&[\smallbreak]


	
	  \endgroup
	

	  \pstart {\color{DodgerBlue3}“आत्मन उपलम्भने”} सति {\color{DodgerBlue3}“तस्य”} प्राणादे{\color{DodgerBlue3}“रुपलब्धौ”} सत्यामात्मनोऽ{\color{DodgerBlue3}“गतौ प्राणा”}देर{\color{DodgerBlue3}“गतौ च”} सत्यां कार्यकारणभावः {\color{DodgerBlue3}“प्रसिध्यति । ते च दृष्ट्यदृष्टी”} कार्यकारणभावसाधनेऽ{\color{DodgerBlue3}“त्यन्तप”}रो{\color{DodgerBlue3}“क्ष”}स्यात्मनो {\color{DodgerBlue3}“न सिध्यतः”} । अत्यन्तपरोक्षस्य कथमदृष्टिरपि न सिध्यतीति चेत् । अभावसाधिका दृश्यानुपलब्धिर्न सिध्यत्येव । अदृष्टिमात्रन्तु नाभावसाधकं । ततः प्राणादेरात्मना सह प्रतिबन्धासिद्धेर्नात्मनिवृत्त्या प्राणादिनिवृत्तिसिद्धिः । (२०९, २१०)
	\pend
      \label{div_pvv.4.211}\edlabel{div_pvv.4.211}
	  
	% new div opening: depth here is 2
	

	  \pstart किञ्च (।)
	\pend
      \leavevmode\marginnote{\textenglish{490/s}}
	  \bigskip
	  \begingroup
	  \large
	
	    
	    \stanza[\smallbreak]
	\label{pv.4.211}\edlabel{pv.4.211}\flagstanza{\tiny\textenglish{....4.211}}अन्यत्रादृष्टरूपस्य घटादौ नेति वा कुतः ।&अज्ञातव्यतिरेकस्य व्यावृत्तेर्व्यापिता कुतः ॥ २११ ॥\&[\smallbreak]


	
	  \endgroup
	

	  \pstart आत्मनो{\color{DodgerBlue3}“न्यत्र”} जीवच्छरीरे{\color{DodgerBlue3}“ऽदृष्टरूप”}स्य {\color{DodgerBlue3}“घटादौ न”} सत्त्व{\color{DodgerBlue3}“मिति”} यदुच्यते त{\color{DodgerBlue3}“त्कुतः”} सिद्धं । दृश्यानुपलब्ध्या ह्यभावसिद्धिः । न चादृष्टचरे तत्सम्भवः । {\color{DodgerBlue3}“अज्ञातो व्यतिरेको”}ऽभावो यस्य तस्यात्मनो या {\color{DodgerBlue3}“व्यावृत्ति”}र्घटादौ तस्याः प्राणादिनिवृत्त्या {\color{DodgerBlue3}“व्यापिता”} व्यापनं {\color{DodgerBlue3}“कुतः”} सम्भवि । येन जीवच्छरीरे प्राणादिनिवृत्त्यभावात् सात्मकत्वनिवृत्त्यभावे सत्यात्मसिद्धिः स्यात् ॥ (२११)
	\pend
      \label{div_pvv.4.212}\edlabel{div_pvv.4.212}
	  
	% new div opening: depth here is 2
	

	  \pstart ननु यथा घटादौ प्राणाद्यभावनिश्चयस्तथात्माभावनिश्चयोपि किं न भवतीत्याह ।
	\pend
      
	  \bigskip
	  \begingroup
	  \large
	
	    
	    \stanza[\smallbreak]
	\label{pv.4.212}\edlabel{pv.4.212}\flagstanza{\tiny\textenglish{....4.212}}प्राणादेश्च क्वचिद् दृष्ट्या सत्त्वासत्त्वं प्रतीयते ।&तथात्मा यदि दृश्येत सत्त्वासत्त्वं प्रतीयते ॥ २१२ ॥\&[\smallbreak]


	
	  \endgroup
	

	  \pstart {\color{DodgerBlue3}“प्राणादेः क्वचि”}ज्जीवच्छरीरे {\color{DodgerBlue3}“दृष्ट्या सत्त्वं”} प्रतीयते । मृतदेहे चोपलब्धिलक्षणप्राप्तस्यास्यादृष्ट्या{\color{DodgerBlue3}“ऽसत्त्वं”} प्रतीयते । {\color{DodgerBlue3}“तथा यदि”} क्वचिद् देहे {\color{DodgerBlue3}“आत्मा दृश्येत”} तदा {\color{DodgerBlue3}“तत्र सत्त्व”}मस्य प्रतीयेत । अन्यत्रोपलभ्यस्वरूपस्यास्यानुपलब्धेर{\color{DodgerBlue3}“सत्त्वं प्रतीयेत । न”} चात्मोपलब्धिरस्ति क्वचिदिति न तदभावनिश्चयः । (२१२)
	\pend
      \label{div_pvv.4.213}\edlabel{div_pvv.4.213}
	  
	% new div opening: depth here is 2
	

	  \pstart यदप्युच्यते (।) यदि न सात्मकं जीवच्छरीरं तदास्य प्राणादिविरहप्रसङ्गो नरात्म्याद् घटवदिति प्रसङ्गसाधनं । तच्चायुक्तं दर्शयितुमाह ।
	\pend
      
	  \bigskip
	  \begingroup
	  \large
	
	    
	    \stanza[\smallbreak]
	\label{pv.4.213}\edlabel{pv.4.213}\flagstanza{\tiny\textenglish{....4.213}}यस्य हेतोरभावेन घटे प्राणो न दृश्यते ।&देहेपि यद्यसौ न स्याद् युक्तो देहे न सम्भवः ॥ २१३ ॥\&[\smallbreak]


	
	  \endgroup
	

	  \pstart {\color{DodgerBlue3}“यस्य”} बुद्धिदेहातिशयप्रयत्ना{\color{DodgerBlue3}“देर्हेतोरभावेन घटे प्राणो न दृश्यते (।) देहेपि यद्यसौ”} बुद्धिदेहातिशयप्रयत्नादि{\color{DodgerBlue3}“र्न स्यात्”} (।) तदा देहेपि हेतुविरहात् प्राणादेर्न\edlabel{pvv.490-1}\footnote{\label{pvv.490-1}  १ शाश्वत आत्मा तद्विपरीताः प्राणादय इति न तादात्म्यं ।} \leavevmode\marginnote{\textenglish{100a/MA}}{\color{DodgerBlue3}“संभवो युक्तो”} यथा मृतशरीरे । जीवद्देहे तु बुद्ध्यादिसद्भावादेव प्राणादिरुचितसंभव इति नात्माभावे तदभावप्रसङ्गः सङ्गतः । (२१३)
	\pend
      \label{div_pvv.4.214}\edlabel{div_pvv.4.214}
	  
	% new div opening: depth here is 2
	

	  \pstart अथ घटजीवच्छरीरादीनां नैरात्म्येन यथा साधर्म्यं तथा प्राणादिरहिततयापि स्यात् । न च भवति (।) तस्मान्निरात्मके घटादौ प्राणादिरदृष्टो यत्र वर्त्तते तत्र सात्मकत्वं साधयतीत्याह ।
	\pend
      
	  \bigskip
	  \begingroup
	  \large
	
	    
	    \stanza[\smallbreak]
	\label{pv.4.214}\edlabel{pv.4.214}\flagstanza{\tiny\textenglish{....4.214}}भिन्नेपि किञ्चित् साधर्म्याद् यदि तत्त्वं प्रतीयते ।&प्रमेयत्वाद् घटादीनां सात्मत्वं किन्न मीयते ॥ २१४ ॥\&[\smallbreak]


	
	  \endgroup
	\leavevmode\marginnote{\textenglish{491/s}}

	  \pstart घटभिन्नेपि जीवद्देहे {\color{DodgerBlue3}“किञ्चि”}न्मात्रेण निरात्मकत्वेन {\color{DodgerBlue3}“साधर्म्यात् यदि तत्त्वं”} प्राणादिविरहिततया घटसदृशत्वं {\color{DodgerBlue3}“प्रतीयते”} तदा {\color{DodgerBlue3}“प्रमेयत्वा”}द्धेतो{\color{DodgerBlue3}“र्घटादीनां”} जीवच्छरीर{\color{DodgerBlue3}“सात्मकत्वं किन्न मीयते”} (। २१४)
	\pend
      \label{div_pvv.4.215}\edlabel{div_pvv.4.215}
	  
	% new div opening: depth here is 2
	
	  \bigskip
	  \begingroup
	  \large
	
	    
	    \stanza[\smallbreak]
	\label{pv.4.215}\edlabel{pv.4.215}\flagstanza{\tiny\textenglish{....4.215}}अनिष्टेश्चेत् प्रमाणं हि सर्वेष्टीनां निबन्धनम् ।&भावाभावव्यवक्थां कः ? कर्त्तुं तेन विना प्रभुः ॥ २१५ ॥\&[\smallbreak]


	
	  \endgroup
	

	  \pstart {\color{DodgerBlue3}“अनिष्टेश्चेत्”} (।) ननु {\color{DodgerBlue3}“प्रमाणं हि सर्व्वेष्टीनां निबन्धनं”} । तद्वशेनार्थान्न स्थितेः । {\color{DodgerBlue3}“तेन”} प्रमाणेन {\color{DodgerBlue3}“भावाभाव”}यो{\color{DodgerBlue3}“र्व्यवस्थां कर्त्तुं कः प्रभुः”} । यदि च किञ्चित् साधर्म्यमात्रात् साधनं साध्यसाधकं । तदा प्रमेयत्वाद् घटस्यापि सात्मकत्वसाधनादनिष्टिरनुपयुक्ता । (२१५)
	\pend
      \label{div_pvv.4.216_4.217}\edlabel{div_pvv.4.216_4.217}
	  
	% new div opening: depth here is 2
	

	  \begin{center}%% label @type='head'
	\textbf{ख. स्मृतीच्छादयः प्राणादिहेतुः}
	\end{center}
	

	  \pstart यदि तर्हि नात्मा प्राणादेर्हेतुः (।) कस्तर्हि भविष्यतीत्याह । यथा योगं (।)
	\pend
      
	  \bigskip
	  \begingroup
	  \large
	
	    
	    \stanza[\smallbreak]
	\label{pv.4.216a}\edlabel{pv.4.216a}\flagstanza{\tiny\textenglish{...4.216a}}स्मृतीच्छायत्नजः प्राणनिमेषादिस्तदुद्भवः ।&विषयेन्द्रियचित्तेभ्यः ;\&[\smallbreak]


	
	  \endgroup
	

	  \pstart {\color{DodgerBlue3}“स्मृतीच्छा”}प्र{\color{DodgerBlue3}“यत्ने”}\edlabel{pvv.491-1}\footnote{\label{pvv.491-1}  १ प्राणवायोः प्रेरकोऽन्तर्व्यापारः प्रयत्नः ।}भ्यो जातः {\color{DodgerBlue3}“प्राणनिमेषादिः”} समाहितस्य निरुद्धवायोः समाधिव्युत्थितस्य स्मरणात् प्राणवृत्तिर्गाढप्रहारादिभिर्व्याहतप्राणस्य इच्छाप्रयत्नाभ्यां प्राणप्रवृत्तिः । स्वस्थस्य साद्गुण्यात्\edlabel{pvv.491-2}\footnote{\label{pvv.491-2}  २ किञ्चित् पश्यतोऽक्षान्तरप्रस्फुरणे शरीरसाद्गुण्यबुद्धिप्रयत्नादयः प्राणादिषु ।} निमेषादेरिच्छाप्रयत्नजत्वं व्यक्तं । तेषां च स्मृतीच्छायत्नाना{\color{DodgerBlue3}“मुद्भवो विषयेन्द्रियचित्तेभ्यो”} यथा\edlabel{pvv.491-3}\footnote{\label{pvv.491-3}  ३ स्मृतिरनुभवात् तत इन्द्रियविकारस्फुरणे । अनुभूतस्मृत्या स्फुरणं नानास्रवः । मातुलुङ्गदर्शने ।}योगं (।) क्वचिद् विषयाद् वदरादे रसनास्रवादिहेतो रसादिस्मृतिर्भवति । इन्द्रियाद्वा विलादेरपटुजानहेतोः प्रदीपमण्डलादिस्मरणं भवति बुद्धेरेवातद्विषयस्यातीतानुकूलतादिस्मृतिरुद्भवति ।
	\pend
      

	  \pstart ननु षट् प्रवृत्तिबुद्धय एवात्मजाः प्रवृत्तिबुद्धिजाश्च स्मृत्यादयः । तद्भवाश्च प्राणादयः इति परंपराऽत्महेतुका एवेत्याह । ताः षड्बुद्धयः (।)
	\pend
      
	  \bigskip
	  \begingroup
	  \large
	
	    
	    \stanza[\smallbreak]
	\label{pv.4.216b}\edlabel{pv.4.216b}\flagstanza{\tiny\textenglish{...4.216b}}ताः स्वजातिसमुद्भवाः ॥ २१६ ॥\&[\smallbreak]


	
	  \endgroup
	
	  \bigskip
	  \begingroup
	  \large
	
	    
	    \stanza[\smallbreak]
	\label{pv.4.217}\edlabel{pv.4.217}\flagstanza{\tiny\textenglish{....4.217}}अन्योन्यप्रत्ययापेक्षा अन्वयव्यतिरेकभाक् ।&एतावत्यात्मभावोयमनवस्थान्यकल्पने ॥ २१७ ॥\&[\smallbreak]


	
	  \endgroup
	\leavevmode\marginnote{\textenglish{492/s}}

	  \pstart {\color{DodgerBlue3}“अन्योन्यप्रत्ययापेक्षा”} यस्या य आत्मीयः प्रत्ययः सहकारिकारणं इन्द्रियादि तस्मिन्नपेक्षाऽयत्तिर्यासां तास्तथा सत्यः स्वजातिसमुद्भवाः समनन्तरप्रभवा न तु परंपरयाप्यात्मापेक्षिण्य इत्यर्थः । {\color{DodgerBlue3}“एताव”}ति कारणकलापे{\color{DodgerBlue3}“ऽयमात्मभावः”} षडायतनं गृहीतो{\color{DodgerBlue3}“ऽन्वयव्यतिरेकभाक्”} प्रतिबद्धः कारणत्वेन ततोन्यस्य\edlabel{pvv.492-1}\footnote{\label{pvv.492-1}  १ आत्मादेः ।} {\color{DodgerBlue3}“कल्पनेऽनवस्था”} कारणानां । तस्मान्नादृष्टसामर्थ्यस्यात्मनो निवृत्तौ प्राणादिनिवृत्तिर्युक्ता ॥ (२१६, २१७)
	\pend
      \label{div_pvv.4.218}\edlabel{div_pvv.4.218}
	  
	% new div opening: depth here is 2
	

	  \pstart किञ्च (।)
	\pend
      
	  \bigskip
	  \begingroup
	  \large
	
	    
	    \stanza[\smallbreak]
	\label{pv.4.218}\edlabel{pv.4.218}\flagstanza{\tiny\textenglish{....4.218}}श्रावणत्वेन तत् तुल्यं प्राणादि व्यभिचारतः ।&न तस्य व्यभिचारित्वाद् व्यतिरेकेपि चेत् कथम् ॥ २१८ ॥\&[\smallbreak]


	
	  \endgroup
	

	  \pstart {\color{DodgerBlue3}“तत् प्राणादि”}साधनं {\color{DodgerBlue3}“श्रावणत्वेन”} हेतुना {\color{DodgerBlue3}“व्यभिचारतो”}ऽनैकान्तिकत्वात्\edlabel{pvv.492-2}\footnote{\label{pvv.492-2}  २ उभयत्र व्यतिरेकव्यभिचारस्य तुल्यत्वात् ।} {\color{DodgerBlue3}“तुल्यं”} । नैत\edlabel{pvv.492-3}\footnote{\label{pvv.492-3}  ३ परः ।}दस्ति । तस्य श्रावणत्वस्य विपक्षाद् बहुलं व्यतिरेकेपि व्यतिरेकस्य {\color{DodgerBlue3}“व्यभिचारतः”} । न हि श्राव(ण)त्वमनित्येभ्यः प्रायो व्यावृत्तमित्यनित्यत्वव्यतिरेकाव्यभिचारि शक्यमवसातुं ।\edlabel{pvv.492-4}\footnote{\label{pvv.492-4}  ४ यद्यपि कुड्यादेरनित्याद् व्यावृत्तः सपक्षाकाशात् तथापि व्यभिचारोऽच्छटादिशब्देऽनित्यत्वादिति ।}पक्ष एवानित्येन सहाकार्यविरोधात् । प्राणादि तु नियमेन विपक्षाद् व्यतिरेकाव्यभिचारीति न युक्तं श्रावणत्वेन तुल्यमिति चेत् । पृच्छत्या चा र्यः । प्राणादिश्रावणत्वेनातुल्यं कथमिति । (२१८)
	\pend
      \label{div_pvv.4.219}\edlabel{div_pvv.4.219}
	  
	% new div opening: depth here is 2
	

	  \pstart इतर आह ।
	\pend
      
	  \bigskip
	  \begingroup
	  \large
	
	    
	    \stanza[\smallbreak]
	\label{pv.4.219}\edlabel{pv.4.219}\flagstanza{\tiny\textenglish{....4.219}}नासाध्यादेव विश्लेषस्तस्य नन्वेवमुच्यते ।&साध्येनुवृत्यभावोर्थात् तस्यान्यत्राप्यसौ समः ॥ २१९ ॥\&[\smallbreak]


	
	  \endgroup
	

	  \pstart तस्य श्रावणत्वस्यासाध्याद् विपक्षादेव न विश्लेषो व्यावृत्तिः किन्तु सपक्षादपि । प्राणादेस्तु सर्व्वस्य जीवच्छरीरस्य सात्मत्वेन साध्यत्वात् (।) सपक्ष एव नास्तीति कथन्ततो व्यावृत्तिरिति (।) अत्राह । {\color{DodgerBlue3}“नन्वेवमसाध्यादेव”} श्रावणत्वस्य {\color{DodgerBlue3}“न विश्लेष”} \leavevmode\marginnote{\textenglish{100b/MA}}इत्याख्याने {\color{DodgerBlue3}“साध्य”} साध्यवति सपक्षे{\color{DodgerBlue3}“ऽनुवृत्ते”}रन्वयस्या{\color{DodgerBlue3}“भावोऽर्था”}दुक्तः स्यात् । यो हि विपक्षमात्रादव्यावृत्तः स सपक्षादपि व्यावृत्तेरन्वयरहित उक्तो भवति । तथा च प्राणादेरसौ सपक्षानुवृत्त्यभावः {\color{DodgerBlue3}“समः”} । न हि प्राणादिसपक्षे क्वचित् सिद्धं । सर्व्वंस्य जीवच्छरीरस्य पक्षत्वात् । (२१९)
	\pend
      \label{div_pvv.4.220}\edlabel{div_pvv.4.220}
	  
	% new div opening: depth here is 2
	
	  \bigskip
	  \begingroup
	  \large
	
	    
	    \stanza[\smallbreak]
	\label{pv.4.220}\edlabel{pv.4.220}\flagstanza{\tiny\textenglish{....4.220}}असाध्यादेव विच्छेद इति साध्येस्तितोच्यते ।&अर्थापत्याऽत एवोक्तमेकेनोभयदर्शनम् ॥ २२० ॥\&[\smallbreak]


	
	  \endgroup
	\leavevmode\marginnote{\textenglish{493/s}}

	  \pstart {\color{DodgerBlue3}“असाध्याद्”} विपक्षा{\color{DodgerBlue3}“देव”} साधनस्य {\color{DodgerBlue3}“विच्छेदो”} व्यावृत्ति{\color{DodgerBlue3}“रित्य”}भिधानेनार्थापत्त्यसामर्थ्येन {\color{DodgerBlue3}“साध्ये”} सपक्षे{\color{DodgerBlue3}“ऽस्तितोच्यते”} । यदि तु  सपक्षादपि व्यावृत्तिस्तदा विपक्षादेव व्यावृत्तिरित्यसङ्गतं । यतो विपक्षादेव व्यावृत्तिरिति व्यतिरेक आक्षिप्तान्वय एव भवति । अत एवा चा र्ये ण {\color{DodgerBlue3}“एकेन”} व्यतिरेकेणान्वयेन वा नियमवता {\color{DodgerBlue3}“दृष्टेनोभयदर्शन”}मुक्तं । अर्थापत्त्याऽन्यतरेणोभयदर्शनादिति । (२२०)
	\pend
      \label{div_pvv.4.221}\edlabel{div_pvv.4.221}
	  
	% new div opening: depth here is 2
	

	  \pstart यस्मादैकान्तिको व्यतिरेक आक्षिप्तान्वय एव भवति । अतः (।)
	\pend
      
	  \bigskip
	  \begingroup
	  \large
	
	    
	    \stanza[\smallbreak]
	\label{pv.4.221}\edlabel{pv.4.221}\flagstanza{\tiny\textenglish{....4.221}}ईदृगव्यभिचारोतोऽनन्वयिषु न सिध्यति ।&प्रतिषेधनिषेधश्च विधानात् कीदृशोऽपरः ॥ २२१ ॥\&[\smallbreak]


	
	  \endgroup
	

	  \pstart {\color{DodgerBlue3}“ईदृग्”} विपक्षे व्यतिरेको (ऽ) {\color{DodgerBlue3}“व्यभिचारः अनन्वयिषु”} हेतुषु {\color{DodgerBlue3}“न सिध्यति”} । य एव साध्येनान्वितो हेतुस्तस्यैव विपक्षादेव व्यतिरेकः । यदि तु सत्यपि साध्ये हेत्वभावस्तदा सपक्षादपि व्यतिरेकात् कथं विपक्षादेव निवृत्तिः । किञ्च (।) प्राणादेः सपक्षे {\color{DodgerBlue3}“प्रतिषेध”}स्य निवृत्ते\edlabel{pvv.493-1}\footnote{\label{pvv.493-1}  १ प्रतिषेधव्याख्या (।) प्राणादेः सपक्षेऽभावो नास्तीति ।} {\color{DodgerBlue3}“निषेधो विधानादपरः कीदृशः”} । प्रतिषेधनिषेधो हि विधिरेव परस्परपरिहारेणावस्थितेः । ततः प्राणादेः {\color{DodgerBlue3}“सपक्षान्निवृत्तिर्नास्तीत्य”}र्थाद् वृत्तिरेवोक्ता स्यादिति न व्यतिरेकित्वं । (२२१)
	\pend
      \label{div_pvv.4.222}\edlabel{div_pvv.4.222}
	  
	% new div opening: depth here is 2
	

	  \pstart स्यादेवं यदि सपक्षो भवति । किन्तु (।)
	\pend
      
	  \bigskip
	  \begingroup
	  \large
	
	    
	    \stanza[\smallbreak]
	\label{pv.4.222}\edlabel{pv.4.222}\flagstanza{\tiny\textenglish{....4.222}}निवृत्तिर्नासतः साध्यादसाध्येष्वेव नो ततः ।&नेति सैव निवृत्तिः किं निवृत्तेरसतो मता ॥ २२२ ॥\&[\smallbreak]


	
	  \endgroup
	

	  \pstart सात्मकत्वे साध्ये सपक्षो नास्तीत्य{\color{DodgerBlue3}“सतः साध्यात्”} सपक्षात् प्राणादे{\color{DodgerBlue3}“र्निवृत्तिर्नास्ति ततोऽसाध्येषु”} विपक्षेष्वेव {\color{DodgerBlue3}“नो”} वृत्तिरिति व्यतिरेकित्वमिष्टं । एवन्तर्हि सपक्षाद{\color{DodgerBlue3}“सतो”} हेतो{\color{DodgerBlue3}“र्निवृत्तेर्निवृत्ति”}रस्माकं अस्तित्वेन येष्टा सैव {\color{DodgerBlue3}“किन्ने\edlabel{pvv.493-2}\footnote{\label{pvv.493-2}  २ कस्मात् ।}ति”} भवतो {\color{DodgerBlue3}“मता”} । यदि ह्यसन् निवृत्तेर्नाधिकरणं तदा निवृत्तनिवृत्तेः कथं भविष्यति । (२२२)
	\pend
      \label{div_pvv.4.223}\edlabel{div_pvv.4.223}
	  
	% new div opening: depth here is 2
	

	  \pstart किञ्च (।)
	\pend
      
	  \bigskip
	  \begingroup
	  \large
	
	    
	    \stanza[\smallbreak]
	\label{pv.4.223}\edlabel{pv.4.223}\flagstanza{\tiny\textenglish{....4.223}}निवृत्त्यभावस्तु विधिर्व्वस्तुभावोऽसतोपि सन् ।&वस्त्वभावस्तु नास्तीति पश्य बान्ध्यविजृम्भितम् ॥ २२३ ॥\&[\smallbreak]


	
	  \endgroup
	

	  \pstart {\color{DodgerBlue3}“असतोपि”} सपक्षाद्धेतु{\color{DodgerBlue3}“निवृत्ते”}र्नीरूपाया अ{\color{DodgerBlue3}“भावस्तु विधिर्वस्तुभावो”} हेतुसम्भवः स नेष्यते (।) यद्यसति सपक्षे हेतुनिवृत्तिर्नास्ति तदा हेतुरेवास्तीत्युक्तं स्यात् । \leavevmode\marginnote{\textenglish{494/s}} {\color{DodgerBlue3}“वस्तुनो”} हेतो{\color{DodgerBlue3}“रभावस्तु”} निवृत्तिशब्दवाच्यो {\color{DodgerBlue3}“नास्तीति पश्य बान्ध्यविजृम्भितं”} परेषां । असत्यसत्तवमविरुद्धं । विरुद्धन्तु सत्त्वमित्यर्थः । (२२३)
	\pend
      \label{div_pvv.4.224}\edlabel{div_pvv.4.224}
	  
	% new div opening: depth here is 2
	

	  \pstart अपि च (।)
	\pend
      
	  \bigskip
	  \begingroup
	  \large
	
	    
	    \stanza[\smallbreak]
	\label{pv.4.224}\edlabel{pv.4.224}\flagstanza{\tiny\textenglish{....4.224}}निवृत्तिर्यदि तस्मिन्न हेतोर्वृत्तिः किमिष्यते ।&सापि न प्रतिषेधोयं निवृत्तिः किं निषिध्यते ॥ २२४ ॥\&[\smallbreak]


	
	  \endgroup
	

	  \pstart {\color{DodgerBlue3}“हेतोस्तस्मिन्नस”}ति सपक्षे {\color{DodgerBlue3}“निवृत्तिर्यदि”} नास्ति तदा {\color{DodgerBlue3}“किं वृत्तिरिष्यते”} विधिनिषेधयोरन्योन्यव्यवच्छेदात्मत्वादेकप्रतिषेधस्य तदपरविधिनान्तरीयकत्वात् । ततश्चान्वयव्यतिरेकित्वात् प्राणादिर्न व्यतिरेकी । {\color{DodgerBlue3}“सा”} वृत्तिर{\color{DodgerBlue3}“पि ना”}सति सपक्ष इति चेत् । ननु वृत्तिनिषेधः {\color{DodgerBlue3}“प्रतिषेधोयं”} निवृत्त्यात्मकः । ततश्चासतो वृत्तिनिवृत्तेरधिकरणत्वात् । तस्मि{\color{DodgerBlue3}“न्निवृत्तिः किं निषिध्यते”} । सत्याञ्च निवृत्तौ श्रावणत्ववदसाधारणं प्राणादि स्यात् । (२२४)
	\pend
      \label{div_pvv.4.225}\edlabel{div_pvv.4.225}
	  
	% new div opening: depth here is 2
	

	  \begin{center}%% label @type='head'
	\textbf{ग. शाब्दो व्यवहारो विधिप्रतिषेधप्रयोजनः}
	\end{center}
	

	  \pstart अथाधिकरणतयाऽपादानतया वा सतः प्रतीतिर्नास्तीति न तस्मान्निवृत्तिरित्युच्यते । एतदर्थं हि (।)
	\pend
      
	  \bigskip
	  \begingroup
	  \large
	
	    
	    \stanza[\smallbreak]
	\label{pv.4.225}\edlabel{pv.4.225}\flagstanza{\tiny\textenglish{....4.225}}विधानं प्रतिषेधञ्च मुक्त्वा शाब्दोस्ति नापरः ।&व्यवहारः स चासत्सु नेति प्राप्तात्र मूकता ॥ २२५ ॥\&[\smallbreak]


	
	  \endgroup
	

	  \pstart {\color{DodgerBlue3}“विधानं प्रतिषेधञ्च मुक्त्वाऽपरः श(ा)ब्दो”} व्यवहारो {\color{DodgerBlue3}“नास्ति”} तयोरेव शब्दप्रतिपाद्यत्वात् । {\color{DodgerBlue3}“स च”} विधिप्रतिषेध{\color{DodgerBlue3}“व्यवहारोऽसत्सु नास्तीति । अत्रा”}सत्सु {\color{DodgerBlue3}“मूकतैव प्राप्ता”} । तथा ह्यसति विधिव्यवहारस्तावन्नास्त्येव । प्रतिषेधोपि त्वन्मते नास्तीति युक्तं मूकत्वं । (२२५)
	\pend
      \label{div_pvv.4.226}\edlabel{div_pvv.4.226}
	  
	% new div opening: depth here is 2
	

	  \pstart किञ्च (।)
	\pend
      
	  \bigskip
	  \begingroup
	  \large
	
	    
	    \stanza[\smallbreak]
	\label{pv.4.226}\edlabel{pv.4.226}\flagstanza{\tiny\textenglish{....4.226}}सताञ्च न निषेधोऽस्ति सोऽसत्सु च न विद्यते ।&जगत्यनेन न्यायेन नञर्थः प्रलयं गतः ॥ २२६ ॥\&[\smallbreak]


	
	  \endgroup
	

	  \pstart {\color{DodgerBlue3}“सताञ्च”} पदार्थानां {\color{DodgerBlue3}“न निषेधोस्ति”} विद्यमानत्वात् । स प्रतिषेधोऽ{\color{DodgerBlue3}“सत्सु न”} त्वदभिप्रायाद् {\color{DodgerBlue3}“विद्यते । अनेन न्यायेन नञोर्थः”} प्रतिषेधो {\color{DodgerBlue3}“जगति”} विषये {\color{DodgerBlue3}“प्रलयं”} \leavevmode\marginnote{\textenglish{101a/MA}}{\color{DodgerBlue3}“गतः”} । (२२६)
	\pend
      \label{div_pvv.4.227}\edlabel{div_pvv.4.227}
	  
	% new div opening: depth here is 2
	
	  \bigskip
	  \begingroup
	  \large
	
	    
	    \stanza[\smallbreak]
	\label{pv.4.227}\edlabel{pv.4.227}\flagstanza{\tiny\textenglish{....4.227}}देशकालनिषेधश्चेद् यथास्ति स निषिध्यते ।&न तथा न यथा सोस्ति तथापि न निषिध्यते ॥ २२७ ॥\&[\smallbreak]


	
	  \endgroup
	

	  \pstart क्वचित् सतामेवार्थानामन्ययो{\color{DodgerBlue3}“र्देशकाल”}योर्नञादिना {\color{DodgerBlue3}“निषेध”} इष्ट{\color{DodgerBlue3}“श्चेत् । यथा”} \leavevmode\marginnote{\textenglish{495/s}} यद्देशकालसम्बद्धत्वेन {\color{DodgerBlue3}“सोऽर्थो”} नास्ति तथा {\color{DodgerBlue3}“निषिध्यते”} त्वदभिप्रायादतो {\color{DodgerBlue3}“निषेधायो”}गात् । {\color{DodgerBlue3}“यथा”} चास्ति स {\color{DodgerBlue3}“तथा”}पि न निषिध्यते सत्वात् । (२२७)
	\pend
      \label{div_pvv.4.228}\edlabel{div_pvv.4.228}
	  
	% new div opening: depth here is 2
	
	  \bigskip
	  \begingroup
	  \large
	
	    
	    \stanza[\smallbreak]
	\label{pv.4.228}\edlabel{pv.4.228}\flagstanza{\tiny\textenglish{....4.228}}तस्मादाश्रित्य शब्दार्थं भवाभावसमाश्रयम् ।&अबाह्याश्रयमत्रेष्टं सर्वं विधिनिषेधनम् ॥ २२८ ॥\&[\smallbreak]


	
	  \endgroup
	

	  \pstart {\color{DodgerBlue3}“तस्माच्छब्द”}स्यार्थमारोपितबहीरूपमन्यव्यवच्छेदम{\color{DodgerBlue3}“बाह्याश्र”}यं बाह्यविषयरहितं य एव {\color{DodgerBlue3}“भावाभाव”}योर्व्विधिप्रतिषेधविकल्पप्रतिपाद्ययोः {\color{DodgerBlue3}“समाश्रय”}स्तमा{\color{DodgerBlue3}“श्रित्य”} व्यवहारे {\color{DodgerBlue3}“सर्व्वं विधिनिषेध\edlabel{pvv.495-1}\footnote{\label{pvv.495-1}  १ बौद्धार्थे ।}नमिष्टं”} । (२२८)
	\pend
      \label{div_pvv.4.229}\edlabel{div_pvv.4.229}
	  
	% new div opening: depth here is 2
	

	  \pstart यदि वस्तु शब्दविकल्पाभ्यां विषयीक्रियते सर्व्वथा प्रतीतेः शब्दप्रमाणान्तरवैफल्यप्रसङ्गः । ततोन्यापोह एव विधिनिषेधसम्बन्धयोर्व्विषयः । तथा शब्दात् सत्त्वासत्त्वाभ्यां वस्तुनः प्रतीतौ विधिनिषेधयोरवैफल्यप्रसङ्गः । तस्मात् भावाभावसाधारणोऽन्वयव्यवच्छेदो विधिप्रतिषेधाभ्यां सम्बध्यत इति ।
	\pend
      
	  \bigskip
	  \begingroup
	  \large
	
	    
	    \stanza[\smallbreak]
	\label{pv.4.229}\edlabel{pv.4.229}\flagstanza{\tiny\textenglish{....4.229}}ताभ्यां स धर्मी सम्बद्धः ख्यात्यभावेपि तादृशः ।&शब्दप्रवृत्तेरस्तीति सोपीष्टो व्यवहारभाक् ॥ २२९ ॥\&[\smallbreak]


	
	  \endgroup
	

	  \pstart {\color{DodgerBlue3}“ताभ्यां”} विधिप्रतिषेधाभ्यां {\color{DodgerBlue3}“स”} शब्दार्थो {\color{DodgerBlue3}“धर्मी सम्बद्धः ख्याति”}वस्तुतो{\color{DodgerBlue3}“ऽभावेपि तादृशो”} विधिप्रतिषेधसम्बद्ध\edlabel{pvv.495-2}\footnote{\label{pvv.495-2}  २ लक्षणस्य ।}स्यार्थस्य (।) न हि शब्दार्थ एव कश्चित् बाह्याबाह्यार्थयो\edlabel{pvv.495-3}\footnote{\label{pvv.495-3}  ३ द्वयं नास्ति ।}रयोगात् । तदितर\edlabel{pvv.495-4}\footnote{\label{pvv.495-4}  ४ तयोरन्यस्य ।}स्य चाभावादित्युक्तेः । कुत एव विधिनिषेधाभ्यां तस्या सम्बन्धः स्यात् । तथा{\color{DodgerBlue3}“पि स”} विधिः प्रतिषमधश्च {\color{DodgerBlue3}“शब्दप्रवृत्तेरस्तीति व्यवहारभागिष्टः”} । शब्दो हि प्रवर्त्तमानो विधिप्रतिषेधव्यवहारं वस्तुतोऽसन्तमप्यविद्याभ्यासतो वासनावशादुपदर्शयतीति तदनुरोधात् सन्नुच्यते\edlabel{pvv.495-5}\footnote{\label{pvv.495-5}  ५ यदि न बौद्धो विधिप्रतिषेधः धर्मिणा धर्माणामभेदो भेदो वेति दूषयति ।}। (२२९)
	\pend
      \label{div_pvv.4.230_4.231}\edlabel{div_pvv.4.230_4.231}
	  
	% new div opening: depth here is 2
	
	  \bigskip
	  \begingroup
	  \large
	
	    
	    \stanza[\smallbreak]
	\label{pv.4.230}\edlabel{pv.4.230}\flagstanza{\tiny\textenglish{....4.230}}अन्यथा स्यात् पदार्थानां विधानप्रतिषेधने ।&एकधर्मस्य सर्वात्मविधानप्रतिषेधनम् ॥ २३० ॥\&[\smallbreak]


	
	  \endgroup
	
	  \bigskip
	  \begingroup
	  \large
	
	    
	    \stanza[\smallbreak]
	\label{pv.4.231a}\edlabel{pv.4.231a}\flagstanza{\tiny\textenglish{...4.231a}}अनानात्मतया;\&[\smallbreak]


	
	  \endgroup
	

	  \pstart {\color{DodgerBlue3}“अन्यथा”} यद्येवं नेष्यते तदा {\color{DodgerBlue3}“पदार्थानां विधानप्रतिषेधने”}ऽभ्युपगम्यमाने यदि धर्मा धर्मिणोऽभिन्नास्तदैकस्य {\color{DodgerBlue3}“धर्मस्य”} विधाने प्रतिषेधने वा कृते {\color{DodgerBlue3}“सर्व्वेषां”} धर्माणां ध{\color{DodgerBlue3}“र्म्यात्म”}भूतानां {\color{DodgerBlue3}“विधानप्रतिषेधनं स्यात्”} । (२३०) \edlabel{pvv.495-6}\footnote{\label{pvv.495-6}  ६ किं कारणं ।}तेषां धर्माणामेकधर्मिस्वभावत्वेनानानात्मतयैकस्वभावतयैकस्य धर्मस्य विधिः प्रतिषेधो वा सर्व्वस्य भवेत् ॥
	\pend
      \leavevmode\marginnote{\textenglish{496/s}}
	  \bigskip
	  \begingroup
	  \large
	
	    
	    \stanza[\smallbreak]
	\label{pv.4.231b}\edlabel{pv.4.231b}\flagstanza{\tiny\textenglish{...4.231b}}भेदे नानाविधिनिषेधवत् ।&एकधर्मिण्यसंहारो विधानप्रतिषेधयोः ॥ २३१ ॥\&[\smallbreak]


	
	  \endgroup
	

	  \pstart धर्मिणः सकाशाद् धर्माणां {\color{DodgerBlue3}“भेदे”}ऽभ्युपगम्यमाने {\color{DodgerBlue3}“एक”}स्मिन् {\color{DodgerBlue3}“धर्मिणि”} धर्माणां {\color{DodgerBlue3}“विधानप्रतिषेधयोरसंहारः”} सामानाधिकरण्यं न स्यात् । {\color{DodgerBlue3}“नानाविधिनिषेधवत्”} । स्वतन्त्रानेकपदार्थविधिनिषेधाविव नैकत्र वस्तुनि सामान्याधिकरण्यभाजौ । (२३१)
	\pend
      \label{div_pvv.4.232_4.233}\edlabel{div_pvv.4.232_4.233}
	  
	% new div opening: depth here is 2
	

	  \pstart अस्मन्मते तु (।)
	\pend
      
	  \bigskip
	  \begingroup
	  \large
	
	    
	    \stanza[\smallbreak]
	\label{pv.4.232}\edlabel{pv.4.232}\flagstanza{\tiny\textenglish{....4.232}}एकधर्मिणमुद्दिश्य नानाधर्मसमाश्रयम् ।&विधावेकस्य तद्भाजमिवान्येषामुपेक्षकम् ॥ २३२ ॥\&[\smallbreak]


	
	  \endgroup
	
	  \bigskip
	  \begingroup
	  \large
	
	    
	    \stanza[\smallbreak]
	\label{pv.4.233}\edlabel{pv.4.233}\flagstanza{\tiny\textenglish{....4.233}}निषेधे तद्विविक्तञ्ज तदन्येषामपेक्षकम् ।&व्यवहारमसत्यार्थं प्रकल्पयति धीर्यथा ॥ २३३ ॥\&[\smallbreak]


	
	  \endgroup
	

	  \pstart शब्दार्थ धर्मिणमेकं {\color{DodgerBlue3}“नानाधर्मसमाश्रयमेकस्य”} धर्मस्य {\color{DodgerBlue3}“विधौ”} क्रियमाणे {\color{DodgerBlue3}“तद्भाजमेक”}धर्मसम्बद्धमि{\color{DodgerBlue3}“वान्येषां”} विधिप्रतिषेधाभ्यामपरामृष्टानां धर्माणामु{\color{DodgerBlue3}“पेक्षकं”} (।) तथैकधर्मस्य {\color{DodgerBlue3}“निषेधे”} क्रियमाणे एकं धर्मिणं नानाधर्मसमाश्रयणं {\color{DodgerBlue3}“तेन”} निषिध्यमानेन धर्मेण {\color{DodgerBlue3}“विविक्त”}मेव {\color{DodgerBlue3}“तदन्येषां”} निषिध्यमानधर्मेतरेषां धर्माणा{\color{DodgerBlue3}“मपेक्षकमु”}द्दिश्य {\color{DodgerBlue3}“यथा”} कल्पिका {\color{DodgerBlue3}“धीर्व्यवहारं”} परमार्थतो{\color{DodgerBlue3}“ऽसत्यार्थं कल्पयति”} । (२३२, २३३)
	\pend
      \label{div_pvv.4.234}\edlabel{div_pvv.4.234}
	  
	% new div opening: depth here is 2
	
	  \bigskip
	  \begingroup
	  \large
	
	    
	    \stanza[\smallbreak]
	\label{pv.4.234}\edlabel{pv.4.234}\flagstanza{\tiny\textenglish{....4.234}}तं तथैवाविकल्पार्थं-भेदाश्रयमुपागताः ।&अनादिवासनोद्भूतं बाधन्तेऽर्थं न लौकिकम् ॥ २३४ ॥\&[\smallbreak]


	
	  \endgroup
	

	  \pstart {\color{DodgerBlue3}“तथैव”} कल्पनानतिक्रमेण {\color{DodgerBlue3}“त”}म्व्यवहारमर्थ{\color{DodgerBlue3}“भेदाश्रयम”}न्यव्यावृत्तिविषय{\color{DodgerBlue3}“मनादि\leavevmode\marginnote{\textenglish{101b/MA}}वासनाया उद्भूतं”} । यथा तत्त्वेन विकल्पोपगताः प्रतिपन्ना व्यवहर्त्तारो लौकिकं बौद्ध{\color{DodgerBlue3}“मर्थं”} शब्दप्रतिपादित{\color{DodgerBlue3}“न्न बाधन्ते”} व्यवहारोच्छेदप्रसङ्गात् । (२३४)
	\pend
      \label{div_pvv.4.235}\edlabel{div_pvv.4.235}
	  
	% new div opening: depth here is 2
	

	  \begin{center}%% label @type='head'
	\textbf{घ. धर्मभेदव्यवहारविचारः}
	\end{center}
	

	  \pstart कथं पुनर्द्धर्मभेदो व्यवह्रियते इत्याह ।
	\pend
      
	  \bigskip
	  \begingroup
	  \large
	
	    
	    \stanza[\smallbreak]
	\label{pv.4.235}\edlabel{pv.4.235}\flagstanza{\tiny\textenglish{....4.235}}तत्फलोऽतत्फलश्चार्थो भिन्न एकस्ततस्ततः ।&तैस्तैरुपप्लवैर्नीतसञ्चयापचयैरिव ॥ २३५ ॥\&[\smallbreak]


	
	  \endgroup
	

	  \pstart अर्थः शब्दादिस्त{\color{DodgerBlue3}“त्फलः”} श्रोत्रविज्ञानादिहेतु{\color{DodgerBlue3}“रतत्फलश्च”}क्षुर्व्विज्ञानाद्यहेतुस्ततो गन्धादेः श्रोत्रज्ञानाहेतोः (।) {\color{DodgerBlue3}“ततो”} रूपादिकाच्चक्षुर्व्विज्ञानहेतो{\color{DodgerBlue3}“र्भिन्नो”} व्यावृत्तस्तत्तद्व्यावृत्तिविशिष्टतया कल्पितधर्मिधर्मनानात्वः परमार्थत {\color{DodgerBlue3}“एकस्तैस्तैरुपप्लवैः”} \leavevmode\marginnote{\textenglish{497/s}} कल्पितधर्मभेदविधिनिषेधयोर्गोचरैर्व्विकल्पैर्यथाक्रमं धर्मिधर्माणां {\color{DodgerBlue3}“नीचः प्रापितः सञ्चयोऽपच”}यश्च यैस्तै{\color{DodgerBlue3}“रिव”} यथाध्यवसायं । (२३५)
	\pend
      \label{div_pvv.4.236}\edlabel{div_pvv.4.236}
	  
	% new div opening: depth here is 2
	

	  \pstart वस्तुतः (।)
	\pend
      
	  \bigskip
	  \begingroup
	  \large
	
	    
	    \stanza[\smallbreak]
	\label{pv.4.236}\edlabel{pv.4.236}\flagstanza{\tiny\textenglish{....4.236}}अतद्वानपि सम्बन्धात् कुतश्चिदुपनीयते ।&दृष्टिं भेदाश्रयैस्तेपि तस्मादज्ञातविप्लवाः ॥ २३६ ॥\&[\smallbreak]


	
	  \endgroup
	

	  \pstart {\color{DodgerBlue3}“भेदाश्रयै”}र्व्यावृत्तिविषयै{\color{DodgerBlue3}“रतद्वान्”} धर्मभेदेन तत्सञ्चयापचयाभ्यां रहितोपि शब्दादि{\color{DodgerBlue3}“दृष्टिं”} प्रतीतव्यवहारमु{\color{DodgerBlue3}“पनी”}यते । विकल्पास्तावद् वासनावशात् तथोपदर्शयन्ति । व्यवहर्त्तारस्तु कस्माद् विमृश्य न निवर्त्तन्त इत्याह । अर्थे {\color{DodgerBlue3}“तेषां व्यावृत्त्याश्रयेण”} कल्पितधर्माणां {\color{DodgerBlue3}“कुतश्चित् सम्बन्धात्”} तदर्थक्रियाप्राप्तेः परितोषादर्थक्रियार्थिनां व्यवहारिणां धर्मभेदाद् यथार्थत्वेन विमर्शादादर इति {\color{DodgerBlue3}“ते”} व्यवहर्त्तारो{\color{DodgerBlue3}“प्यज्ञातविप्लवा”} धर्मभेदविधिप्रतिषेधव्यवहारस्येति यथार्थमेव तत् मन्यन्त इति । तस्मात् परमार्थतोऽसत्येव धर्मिणि विधिप्रतिषेधव्यवहारात् असतोपि सपक्षाद् विपक्षाद् वा हेतुनिवृत्तिः सन्दिग्धा । (२३६)
	\pend
      \label{div_pvv.4.237}\edlabel{div_pvv.4.237}
	  
	% new div opening: depth here is 2
	

	  \pstart किञ्च (।) प्राणादेः सपक्षो नास्तीति ततो न व्यतिरेक इति यद्युच्यते तदसङ्गतमित्याह ।
	\pend
      
	  \bigskip
	  \begingroup
	  \large
	
	    
	    \stanza[\smallbreak]
	\label{pv.4.237}\edlabel{pv.4.237}\flagstanza{\tiny\textenglish{....4.237}}सत्तासाधनवृत्तेश्च सन्दिग्धः स्यादसन्न सः ।&असत्वञ्चाभ्युपगमादप्रमाणन्न यज्यते ॥ २३७ ॥\&[\smallbreak]


	
	  \endgroup
	

	  \pstart आत्मनः {\color{DodgerBlue3}“सत्तायां साधन”}स्य प्राणादे{\color{DodgerBlue3}“र्वृत्तेः”} कारणात् वादिप्रतिवादिनोरात्मा {\color{DodgerBlue3}“सन्दिग्ध स्यात्”} । न ह्यसत्तया निश्चितेऽर्थे कश्चित् साधनमाह । प्रतिवादी च तत्साधनं शृण्वन् कथमसन्दिग्धो नाम । आत्मसन्देहाच्च सपक्षो {\color{DodgerBlue3}“नासन्”} तत् कथमुच्यते सपक्षासत्वात् न ततो व्यावृत्त इति व्यतिरेकी प्राणादिः । स्यादेतत् (।) प्रतिवादिनः सात्मत्वेन कस्याश्चिदनिष्टेः सपक्षाभाव उच्यते ।
	\pend
      

	  \pstart ननु पराभ्युपगमप्रमाणमप्रमाणम्वा । प्रमाणञ्चेन्नैरात्म्यमेव तर्हि सिद्धं ।) अलमात्मसाधनोपन्यासप्रयासेन (।) अथा{\color{DodgerBlue3}“प्रमाणं”} (।) परस्याभ्युपगमात् अप्रमाणकात् सपक्षसत्वं च {\color{DodgerBlue3}“न युज्यते”} । (२३७) तस्माद् (।)
	\pend
      \label{div_pvv.4.238}\edlabel{div_pvv.4.238}
	  
	% new div opening: depth here is 2
	
	  \bigskip
	  \begingroup
	  \large
	
	    
	    \stanza[\smallbreak]
	\label{pv.4.238}\edlabel{pv.4.238}\flagstanza{\tiny\textenglish{....4.238}}असतो व्यतिरेकेपि सपक्षाद् विनिवर्त्तनम् ।&सन्दिग्धं तस्य सन्देहाद् विपक्षाद् विनिवर्त्तनम् ॥ २३८ ॥\&[\smallbreak]


	
	  \endgroup
	

	  \pstart अ{\color{DodgerBlue3}“सतः सपक्षा”}द्धेतो{\color{DodgerBlue3}“र्व्य”}ति{\color{DodgerBlue3}“रेकेप्यु”}पगम्यमाने प्राणादेः सपक्षाद् {\color{DodgerBlue3}“विनिवर्त्तनं सन्दिग्धं तस्य”} सपक्ष{\color{DodgerBlue3}“सन्देहात्”} । यदि सपक्षाभावेाऽसन्दिग्ध एवं हेतुव्यावृत्तिभावो\edlabel{pvv.497-1}\footnote{\label{pvv.497-1}  १ विपक्षात् ।}प्यसन्दिग्धः \leavevmode\marginnote{\textenglish{498/s}} स्यात् । तत्सन्देहे तस्यापि सन्देहः । सपक्षाद् व्यावृत्तिसन्देहे {\color{DodgerBlue3}“विपक्षाद् विनिवर्त्तनं”} प्राणादेः सन्दिग्धं । (२३८)
	\pend
      \label{div_pvv.4.239}\edlabel{div_pvv.4.239}
	  
	% new div opening: depth here is 2
	

	  \pstart कथमित्याह (।)
	\pend
      
	  \bigskip
	  \begingroup
	  \large
	
	    
	    \stanza[\smallbreak]
	\label{pv.4.239}\edlabel{pv.4.239}\flagstanza{\tiny\textenglish{....4.239}}एकत्र नियमे सिद्धे सिध्यत्यन्यनिवर्त्तनम् ।&द्वैराश्ये सति दृष्टेषु स्याददृष्टेपि संशयः ॥ २३९ ॥\&[\smallbreak]


	
	  \endgroup
	

	  \pstart {\color{DodgerBlue3}“एकत्र”} सपक्षे सत्त्वस्य {\color{DodgerBlue3}“नियमे सिद्धे”} सति {\color{DodgerBlue3}“अन्य”}तो विपक्षा{\color{DodgerBlue3}“न्निवर्त्तनं सिध्यति”} । यो हि पक्ष एव भवति स नियमाद् विपक्षे न भवति । यदि तु विपक्षेपि स्यात् सपक्षे सत्ता नियमो व्याहन्येत ।
	\pend
      

	  \pstart ननु सात्मकत्वानात्मकत्वाभ्यां द्वैराश्यं भावानां । तत्र निरात्मकेषु घटादिषु दृष्टः प्राणादिरर्थात् सात्मकेषु व्यवतिष्ठत इति प्राणादेरतो युक्तं सात्मकत्वानुमानमित्याह । द्वैराश्ये सति भावानां घटादिषु निरात्मकेषु दृष्टेषु बहुलं प्राणादाव \leavevmode\marginnote{\textenglish{102a/MA}}{\color{DodgerBlue3}“दृष्टेपि”} देशादिविप्रकृष्टेषु प्राणादिसत्ता{\color{DodgerBlue3}“संशयः”} कदाचित् क्वचिन्निरात्मका अपि प्राणादियुक्ताः स्युः बाधकाभावात् । (२३९) तथाहि ।
	\pend
      \label{div_pvv.4.240}\edlabel{div_pvv.4.240}
	  
	% new div opening: depth here is 2
	
	  \bigskip
	  \begingroup
	  \large
	
	    
	    \stanza[\smallbreak]
	\label{pv.4.240}\edlabel{pv.4.240}\flagstanza{\tiny\textenglish{....4.240}}अव्यक्तिव्यापिनोप्यर्थाः सन्ति तज्जातिभाविनः ।&क्वचिन्न नियमोदृष्ट्या पार्थिवालोहलेख्यवत् ॥ २४० ॥\&[\smallbreak]


	
	  \endgroup
	

	  \pstart {\color{DodgerBlue3}“तस्या”}मेकस्यां {\color{DodgerBlue3}“जातौ”} सम्भ{\color{DodgerBlue3}“विनोप्यर्था”} धर्मा {\color{DodgerBlue3}“अव्यक्तिव्यापिनो”} निःशेषतद्व्यक्त्य{\color{DodgerBlue3}“सम्भविनः सन्तिः”} ततः क्वचिद् घटादौ निरात्मके प्राणादिर्नांस्तीत्य{\color{DodgerBlue3}“दृष्ट्या”}दर्शनमात्रेण समस्तेषु निरात्मकेषु न प्राणाद्यभावस्य {\color{DodgerBlue3}“नियमः”} । बहुषु {\color{DodgerBlue3}“पार्थि”}वेषु काष्ठपाषाणादिषु लोहलेख्यत्वदर्शनेपि पार्थिव एव\edlabel{pvv.498-1-bis}\footnote{\label{pvv.498-1-bis}  १ लोहलेख्यं वज्रं पार्थिवत्वात् काष्ठवत् ।} वज्रेऽलोहलेख्यत्वदर्शनेपि पार्थिव एव\edlabel{pvv.498-1}\footnote{\label{pvv.498-1}  १ लोहलेख्यं वज्रं पार्थिवत्वात् काष्ठवत् ।}वज्रे{\color{DodgerBlue3}“ऽलोहलेख्यत्ववत्”} । अलोहलेख्यस्येव नासम्भवस्य नियमः । (२४०)
	\pend
      \label{div_pvv.4.241}\edlabel{div_pvv.4.241}
	  
	% new div opening: depth here is 2
	
	  \bigskip
	  \begingroup
	  \large
	
	    
	    \stanza[\smallbreak]
	\label{pv.4.241a}\edlabel{pv.4.241a}\flagstanza{\tiny\textenglish{...4.241a}}भावे विरोधस्यादृष्टेः कः सन्तेहं निवारयेत् ।\&[\smallbreak]


	
	  \endgroup
	

	  \pstart निरात्मकत्वेन सह प्राणादे{\color{DodgerBlue3}“र्व्विरोधस्यादृष्टे”}र्निरात्मकेष्वपि भावेषु प्राणादेर्भावे सत्तायां {\color{DodgerBlue3}“कः सन्देहं निवारयेत्”} । न हि प्राणादेर्नैरात्म्येन सहानवस्थानलक्षणो विरोधः (।) क्वचिन्निवर्त्त्यनिवर्त्तकभावानुपलब्धेः नाप्यन्योन्यपरिहारस्थितिलक्षणः परस्परव्यवच्छेदात्मकत्वाभावात् ।
	\pend
      

	  \pstart स्यादेतत् (।) नैरात्म्यविरुद्धेनात्मना व्याप्तत्वात् प्राणादेः परम्परया नैरात्म्येन विरोध इत्याह ।
	\pend
      
	  \bigskip
	  \begingroup
	  \large
	
	    
	    \stanza[\smallbreak]
	\label{pv.4.241b}\edlabel{pv.4.241b}\flagstanza{\tiny\textenglish{...4.241b}}क्वचिद् विनियमात् कोन्यस्तत्कार्यात्मतया स च ॥ २४१ ॥\&[\smallbreak]


	
	  \endgroup
	\leavevmode\marginnote{\textenglish{499/s}}

	  \pstart तस्यात्मनः {\color{DodgerBlue3}“कार्यतया”} आत्मतया प्राणादेः {\color{DodgerBlue3}“क्वचिदा”}त्मनि {\color{DodgerBlue3}“विनियमा”}न्नियतत्वात् {\color{DodgerBlue3}“कोन्यः”} परम्परया विरोधश्चोक्तः स्यात् । तादात्म्यतदुत्पत्तिभ्यामेव केनचित् किञ्चिद् व्याप्यते नान्यथा । (२४१)
	\pend
      \label{div_pvv.4.242}\edlabel{div_pvv.4.242}
	  
	% new div opening: depth here is 2
	

	  \pstart न चात्मनोऽत्यन्तपरोक्षतया ते सिध्यतः । तत्कथन्तन्निमित्तकविरोधस्थितिः ।
	\pend
      
	  \bigskip
	  \begingroup
	  \large
	
	    
	    \stanza[\smallbreak]
	\label{pv.4.242a}\edlabel{pv.4.242a}\flagstanza{\tiny\textenglish{...4.242a}}नैरात्म्यादपि तेनास्य सन्दिग्धं विनिवर्तनम् ।\&[\smallbreak]


	
	  \endgroup
	

	  \pstart {\color{DodgerBlue3}“तेन”} साक्षात्परम्परया च विरोधाभावेन {\color{DodgerBlue3}“नैरात्म्याद्”} विपक्षा{\color{DodgerBlue3}“दपि”} शब्दान्न केवलं सपक्षा{\color{DodgerBlue3}“दस्य”} प्राणादे{\color{DodgerBlue3}“र्निवर्त्तनं सन्दिग्धं”} । सन्दिग्धान्वयव्यतिरेकः प्राणादिरित्यर्थः ।
	\pend
      

	  \pstart नास्ति तावदुक्तक्रमेण नैरात्म्यात् प्राणादिनिवृत्तिः (।)
	\pend
      
	  \bigskip
	  \begingroup
	  \large
	
	    
	    \stanza[\smallbreak]
	\label{pv.4.242b}\edlabel{pv.4.242b}\flagstanza{\tiny\textenglish{...4.242b}}अस्तु नाम तथाप्यात्मा नानैरात्म्यात् प्रसिध्यति ॥ २४२ ॥\&[\smallbreak]


	
	  \endgroup
	

	  \pstart {\color{DodgerBlue3}“अस्तु नामा”}ङ्गीकारात् {\color{DodgerBlue3}“तथापि”} जीवच्छरीरे{\color{DodgerBlue3}“ऽनैरात्म्यान्नै”}रात्म्याभावात् प्राणादिमत्वसाधिता{\color{DodgerBlue3}“दात्मा न सिध्यति”} । (२४२)
	\pend
      \label{div_pvv.4.243}\edlabel{div_pvv.4.243}
	  
	% new div opening: depth here is 2
	

	  \pstart यन्न विरुद्धयोरेकाभावादन्यतरस्यावश्यं सिद्धिर्भत्येवेत्याह ।
	\pend
      
	  \bigskip
	  \begingroup
	  \large
	
	    
	    \stanza[\smallbreak]
	\label{pv.4.243}\edlabel{pv.4.243}\flagstanza{\tiny\textenglish{....4.243}}येनासौ व्यतिरेकस्य नाभावं भावमिच्छति ।&यथा नाव्यतिरेकेपि प्राणादिर्न्न सपक्षतः ॥ २४३ ॥\&[\smallbreak]


	
	  \endgroup
	

	  \pstart {\color{DodgerBlue3}“येन”} कारणे{\color{DodgerBlue3}“नासौ”} वादी व्यतिरेकस्या{\color{DodgerBlue3}“भावं भाव”}मन्वयं {\color{DodgerBlue3}“नेच्छति । यथा”} प्राणादिरसतः सपक्षतः । {\color{DodgerBlue3}“अव्यतिरेके”} व्यतिरेकाभावे{\color{DodgerBlue3}“पि न सपक्षे”} सत्वेनेष्टः । तथा नैरात्म्यनिवृत्तावपि आत्मभावो जीवच्छरीरे न स्यात् । (२४३)
	\pend
      \label{div_pvv.4.244}\edlabel{div_pvv.4.244}
	  
	% new div opening: depth here is 2
	

	  \pstart तस्मात् (।)
	\pend
      
	  \bigskip
	  \begingroup
	  \large
	
	    
	    \stanza[\smallbreak]
	\label{pv.4.244}\edlabel{pv.4.244}\flagstanza{\tiny\textenglish{....4.244}}सपक्षाव्यतिरेकी चेद्धेतुर्हेतुरतोन्वयी ।&नान्वय्यव्यतिरेकी चेदनैराम्त्यं न सात्मकम् ॥ २४४ ॥\&[\smallbreak]


	
	  \endgroup
	

	  \pstart {\color{DodgerBlue3}“सपक्षाव्यतिरेकी”} प्राणादिर्नैरात्म्ये निवर्त्तमानः सत्तामात्मनो गमयन् {\color{DodgerBlue3}“हेतुश्चेदिष्ट”}स्तदाऽत एव न्यायात् प्राणादि{\color{DodgerBlue3}“हेतुरन्वयी”} स्यात् । सपक्षात् प्राणादेर्निवृत्त्यभाव एव भावः । स चान्वयः (।) सपक्षा{\color{DodgerBlue3}“दव्यतिरेकी”} व्यतिरेकरहितः प्राणादि{\color{DodgerBlue3}“र्नान्वयी चेदि”}ष्यते तदा तद्वद{\color{DodgerBlue3}“नैरात्म्यं”} नैरात्म्यरहितं शरीरं\edlabel{pvv.499-1}\footnote{\label{pvv.499-1}  १ नात्र नैरात्म्यमस्तीति ।} {\color{DodgerBlue3}“न सात्मकं”} स्यात् । (२४४)
	\pend
      \label{div_pvv.4.245}\edlabel{div_pvv.4.245}
	  
	% new div opening: depth here is 2
	

	  \pstart किञ्च (।)
	\pend
      
	  \bigskip
	  \begingroup
	  \large
	
	    
	    \stanza[\smallbreak]
	\label{pv.4.245}\edlabel{pv.4.245}\flagstanza{\tiny\textenglish{....4.245}}यन्नान्तरीयकः स्वात्मा यस्य सिद्धः प्रवृत्तिषु ।&निवर्त्तकः स एवातः प्रवृत्तौ च प्रवर्त्तकः ॥ २४५ ॥\&[\smallbreak]


	
	  \endgroup
	

	  \pstart {\color{DodgerBlue3}“यस्य”} स्वात्म(ा) स्वभावो {\color{DodgerBlue3}“यन्नान्तरीयकः\edlabel{pvv.499-2}\footnote{\label{pvv.499-2}  २ लिङ्गं धूमादिः ।}”} यदन्वयव्यतिरेकानुविधायी प्रतिबन्ध\leavevmode\marginnote{\textenglish{500/s}} कस्य वह्न्यादेः {\color{DodgerBlue3}“प्रवृत्तिषु”} विधिषु सिद्धः । {\color{DodgerBlue3}“स एव”} निवर्त्तमानः तस्यानुविहितान्वयव्यतिरेकस्य {\color{DodgerBlue3}“निवर्त्तकः”} । {\color{DodgerBlue3}“अतः”} प्रवृत्तिविषयत्वात् {\color{DodgerBlue3}“प्रवृत्तौ”} स्वसत्तायाञ्च तस्यावश्यं प्रवृत्तेः कर्त्ता हेतुर्भवति । यथा धूमो दहननान्तरीयकतया सिद्धो दहननिवृत्तिप्रवृत्तिभ्यां निवर्त्तते प्रवर्त्तते च (। २४५)
	\pend
      \label{div_pvv.4.246}\edlabel{div_pvv.4.246}
	  
	% new div opening: depth here is 2
	
	  \bigskip
	  \begingroup
	  \large
	
	    
	    \stanza[\smallbreak]
	\label{pv.4.246}\edlabel{pv.4.246}\flagstanza{\tiny\textenglish{....4.246}}नान्तरीयकता सा च साधनं समपेक्षते ।&कार्ये दृष्टिरदृष्टिश्च कार्यकारणता हिता ॥ २४६ ॥\&[\smallbreak]


	
	  \endgroup
	

	  \pstart {\color{DodgerBlue3}“सा च नान्तरीयकता साधनं”} निश्चायकं मान{\color{DodgerBlue3}“मपेक्षते”} गमकत्वहेत्वधिकारेऽनिश्चितगमकत्वनिबन्धस्याहेतुत्वात् । साधनञ्च {\color{DodgerBlue3}“कार्ये”} कारणान्वयवति {\color{DodgerBlue3}“दृष्टि”}\leavevmode\marginnote{\textenglish{102b/MA}}स्तद्व्यतिरेकवति {\color{DodgerBlue3}“चादृष्टिः”} । न तु सपक्षविपक्षयोर्दर्शनादर्शनमात्रकं । हि यस्मात् तत्कार्यदृष्ट्यदृष्टी कार्यकारणता कारणभावाभावप्रयुक्ते कार्यभावाभावदर्शने कार्यकारणतोक्ता (।) अन्यनिश्चयोपायतादर्शनार्थं । (२४६)
	\pend
      \label{div_pvv.4.247}\edlabel{div_pvv.4.247}
	  
	% new div opening: depth here is 2
	

	  \pstart ततश्च (।)
	\pend
      
	  \bigskip
	  \begingroup
	  \large
	
	    
	    \stanza[\smallbreak]
	\label{pv.4.247a}\edlabel{pv.4.247a}\flagstanza{\tiny\textenglish{...4.247a}}अर्थान्तरस्य तद्भावेऽभावानियमतोऽगतिः ।\&[\smallbreak]


	
	  \endgroup
	

	  \pstart अकारणस्या{\color{DodgerBlue3}“र्थान्तरस्या”}त्मन{\color{DodgerBlue3}“स्त”}स्य प्राणादे{\color{DodgerBlue3}“र्भावे”}ऽभावा{\color{DodgerBlue3}“नियम”}तोऽवश्यम्भावा{\color{DodgerBlue3}“भावात् अग”}तिरप्रतीतिः ।
	\pend
      

	  \pstart नन्वात्मान्वयव्यतिरेकानुविधानात् प्राणादय आत्मानं तत्कार्यतयानुमापयिष्यन्तीत्याह ।
	\pend
      
	  \bigskip
	  \begingroup
	  \large
	
	    
	    \stanza[\smallbreak]
	\label{pv.4.247b}\edlabel{pv.4.247b}\flagstanza{\tiny\textenglish{...4.247b}}अभावासम्भवात् तेषामभावे नित्यभाविनः ॥ २४७ ॥\&[\smallbreak]


	
	  \endgroup
	

	  \pstart तस्यात्मनो {\color{DodgerBlue3}“नित्यभाविनः”} सर्व्वकालस्थायिनः कदाचिदभावे सति {\color{DodgerBlue3}“तेषां”} प्राणादीना{\color{DodgerBlue3}“मभाव”}स्या{\color{DodgerBlue3}“सम्भवात्”} (।) यदि व्यतिरेकमन्तरेणान्वयमात्रादात्मकार्यता प्राणादीनां तदा कालाकालादिकार्यतापि स्यात् (।) या यस्य समानत्वात् (।) ततश्च तानप्यनुमापयेयुः कालादिनिवृत्तिश्च प्राणादिनिवृत्त्या व्याप्यते । ततो यत्र प्राणादिनिवृत्तिरस्ति तत्र कालादिनिवृत्तिरपि स्यात् । एवं व्याप्यसद्भावात् । न च कालरहितः कश्चिदस्ति । (२४७)
	\pend
      \label{div_pvv.4.248}\edlabel{div_pvv.4.248}
	  
	% new div opening: depth here is 2
	

	  \begin{center}%% label @type='head'
	\textbf{(६) क. सामग्रीशक्तिभेदाद् विश्वरूपता}
	\end{center}
	

	  \pstart किञ्च (।)
	\pend
      
	  \bigskip
	  \begingroup
	  \large
	
	    
	    \stanza[\smallbreak]
	\label{pv.4.248}\edlabel{pv.4.248}\flagstanza{\tiny\textenglish{....4.248}}कार्यस्वभावभेदानां कारणेभ्यः समुद्भवात् ।&तैर्विना भवतोन्यस्मात् तज्जं रूपं कथं भवेत् ॥ २४८ ॥\&[\smallbreak]


	
	  \endgroup
	

	  \pstart \leavevmode\marginnote{\textenglish{501/s}}{\color{DodgerBlue3}“कार्यस्वभावभेदानां\edlabel{pvv.501-1}\footnote{\label{pvv.501-1}  १ अन्वयमात्रात् परो मन्यते तन्निषेधति ।}”} यथा स्वं कारणेभ्यः\edlabel{pvv.501-2}\footnote{\label{pvv.501-2}  २ आकस्मिकानुपपत्तेः ।} {\color{DodgerBlue3}“समुद्भवात्”} कारणात् केवलादन्वयात् कार्यकारणभावेपि कार्यभावाङ्गीकारात् तैः कारणै{\color{DodgerBlue3}“र्व्विना भवतः”} कार्यस्य {\color{DodgerBlue3}“रूपं तज्जं कथं भवेत्”} । न हि वह्निजं युक्तं प्रत्येकं व्यभिचारादेर्हेतुत्वप्रसङ्गात् ॥ (२४८)
	\pend
      \label{div_pvv.4.249}\edlabel{div_pvv.4.249}
	  
	% new div opening: depth here is 2
	
	  \bigskip
	  \begingroup
	  \large
	
	    
	    \stanza[\smallbreak]
	\label{pv.4.249}\edlabel{pv.4.249}\flagstanza{\tiny\textenglish{....4.249}}सामग्रीशक्तिभेदाद्धि वस्तूनां विश्वरूपता ।&सा चेन्न भेदिका प्राप्तमेकरूपमिदं जगत् ॥ २४९ ॥\&[\smallbreak]


	
	  \endgroup
	

	  \pstart {\color{DodgerBlue3}“सामग्रीणां शक्तिभेदाद्धि वस्तूनां”} कार्याणां {\color{DodgerBlue3}“विश्वरूपता”} नानात्मता । {\color{DodgerBlue3}“सामग्री चेत्”} स्वभेदेन कार्याणां {\color{DodgerBlue3}“न”} भेदिका । {\color{DodgerBlue3}“जगदिदमेकरूपं प्राप्तं”} । (२४९)
	\pend
      \label{div_pvv.4.250}\edlabel{div_pvv.4.250}
	  
	% new div opening: depth here is 2
	

	  \pstart ननु\edlabel{pvv.501-3}\footnote{\label{pvv.501-3}  ३ अग्नितोप्यग्निधूमादि । अग्निस्वभावकाशक्रमूर्ध्नोपि धूमादिति व्यभिचार इत्यन्यत उत्पादेऽहेतुत्वं नास्ति । अग्न्यादिविलक्षणसामग्रयाश्च भेदकमभेदकञ्च रूपमस्ति येन धूमञ्जनयति तदभेदकं ।} कारणानि कार्यणानि कार्यमात्राणि जनयन्ति । न तेषां परस्परतो भेदमपि\edlabel{pvv.501-4}\footnote{\label{pvv.501-4}  ४ जनयन्ति ।} ततो धूमः पावकादिव शक्रमूर्ध्नोपि जायेत । इत्याह ।
	\pend
      
	  \bigskip
	  \begingroup
	  \large
	
	    
	    \stanza[\smallbreak]
	\label{pv.4.250}\edlabel{pv.4.250}\flagstanza{\tiny\textenglish{....4.250}}भेदकाभेदकत्वे स्याद् व्याहता भिन्नरूपता ।&एकस्य नानारूपत्वे द्वे रूपे पावकेतरौ ॥ २५० ॥\&[\smallbreak]


	
	  \endgroup
	

	  \pstart अस्याः सामग्रया {\color{DodgerBlue3}“भेदकत्वे”} भिन्नत्वे भिन्नसामग्रीकार्यविलक्षणकार्यजनकत्वेऽभेदकत्वे च सामग्रयन्तरकार्यजनकत्वेऽभ्युपगम्यमाने एकस्याः सामग्रया {\color{DodgerBlue3}“भिन्नरूपता”} नानात्माऽभ्युपगता {\color{DodgerBlue3}“स्यात्”} । सा च {\color{DodgerBlue3}“व्याहता”} । तथा हि {\color{DodgerBlue3}“धूमाग्न्यादिसामग्री”} सामग्रयन्तरकार्यविलक्षणं जनयन्ती तस्य भेदिका प्रतीता । यदि च सामग्रयन्तरकार्यं च सा जनयेत् {\color{DodgerBlue3}“तदाऽभेदिका”} च स्यात् । तथा {\color{DodgerBlue3}“चैकस्य”} शक्रमूर्ध्नो {\color{DodgerBlue3}“नानारूपत्वे द्वे रूपे पावकेतरौ”} स्यातां । धूमजनकत्वाद् वह्नित्वं विलक्षणरूपत्वाच्चावह्निरूपत्वं । न चैते एकस्य युक्ते रूपभेदलक्षणत्वाद् वस्तुभेदस्य । (२५०)
	\pend
      \label{div_pvv.4.251}\edlabel{div_pvv.4.251}
	  
	% new div opening: depth here is 2
	
	  \bigskip
	  \begingroup
	  \large
	
	    
	    \stanza[\smallbreak]
	\label{pv.4.251a}\edlabel{pv.4.251a}\flagstanza{\tiny\textenglish{...4.251a}}तत् तस्या जननं रूपमन्यस्य यदि सैव सा ।\&[\smallbreak]


	
	  \endgroup
	

	  \pstart {\color{DodgerBlue3}“तत्”} तस्मात् {\color{DodgerBlue3}“तस्याः”} शक्रमूर्द्धादिसामग्रया {\color{DodgerBlue3}“जननं”} धूमोत्पादकं {\color{DodgerBlue3}“रूपं यदि”} विद्यते तदा {\color{DodgerBlue3}“सैव”} दहनात्मिकैव {\color{DodgerBlue3}“सा”} शक्रमूर्द्धादिसामग्री । अग्निधूमयोरन्वयव्यतिरेकदर्शनात् स एवाग्निर्यो धूमजनकः\edlabel{pvv.501-5}\footnote{\label{pvv.501-5}  ५ इति नास्त्येव व्यभिचार इत्याशयः अवह्नेरनुत्पत्तेः ।} । स एव धूमो योऽग्निजन्य इति निश्चितत्वात् ।
	\pend
      
	  \bigskip
	  \begingroup
	  \large
	
	    
	    \stanza[\smallbreak]
	\label{pv.4.251b}\edlabel{pv.4.251b}\flagstanza{\tiny\textenglish{...4.251b}}न तस्या जननं रूपं तत् तस्याः संभवेत् कथम् ॥ २५१ ॥\&[\smallbreak]


	
	  \endgroup
	\leavevmode\marginnote{\textenglish{502/s}}

	  \pstart अथ {\color{DodgerBlue3}“तस्याः”} शक्रमूर्द्धादिसामग्रया {\color{DodgerBlue3}“जननं रूपं”} नास्तीति तदा {\color{DodgerBlue3}“तद्”} धूमाख्यं रूपं वह्निजन्य{\color{DodgerBlue3}“मस्याः कथं संभवेत्”} (।) न ह्यधूमहेतोर्द्धूमजन्म युक्तं (।) यतो धूमजनक एव वह्निर्वह्निजन्यश्च धूमः । (२५१)
	\pend
      \label{div_pvv.4.252}\edlabel{div_pvv.4.252}
	  
	% new div opening: depth here is 2
	
	  \bigskip
	  \begingroup
	  \large
	
	    
	    \stanza[\smallbreak]
	\label{pv.4.252}\edlabel{pv.4.252}\flagstanza{\tiny\textenglish{....4.252}}ततः स्वभावौ नियतावन्योन्यं हेतुकार्ययोः ।&तस्मात् स्वदृष्टाविव तद् दृष्टे कार्येपि गम्यते ॥ २५२ ॥\&[\smallbreak]


	
	  \endgroup
	

	  \pstart {\color{DodgerBlue3}“ततो हेतुकार्ययो”}रग्निधूमाद्योः {\color{DodgerBlue3}“स्वभावौ नियतौ अन्योन्यं”} तज्जनकत्वतद्भा\leavevmode\marginnote{\textenglish{103a/MA}} वित्वाभ्यां (।) {\color{DodgerBlue3}“तस्मात्”} कारणकार्ययोर्नियमादव्यभिचारहेतोः यथा {\color{DodgerBlue3}“स्वस्य”} कारणात्मने(ा)वह्नेः {\color{DodgerBlue3}“दृष्टौ”} तत्कारणत्वं गम्यते तद्वत् साक्षाद् दर्शनाभावे {\color{DodgerBlue3}“कार्येपि”} धूमे {\color{DodgerBlue3}“दृष्टे”} तत्कारणे परोक्षमनुमानबुद्ध्या {\color{DodgerBlue3}“गम्यते\edlabel{pvv.502-1}\footnote{\label{pvv.502-1}  १ इत्यव्यभिचारिकार्यलिङ्गम् ।}”} साक्षात् परंपरया वा तदुत्पन्नेन ज्ञानेन प्रतीयमानस्य व्यभिचाराभावात् ॥ (२५२)
	\pend
      \label{div_pvv.4.253}\edlabel{div_pvv.4.253}
	  
	% new div opening: depth here is 2
	
	  \bigskip
	  \begingroup
	  \large
	
	    
	    \stanza[\smallbreak]
	\label{pv.4.253a}\edlabel{pv.4.253a}\flagstanza{\tiny\textenglish{...4.253a}}एकं कथमनेकस्मात् क्लेदवद्दुग्धवारिणः ।\&[\smallbreak]


	
	  \endgroup
	

	  \pstart (पर आह ।) न यद्येकमेकस्मादेव भवति त{\color{DodgerBlue3}“दैकं”} कार्यं {\color{DodgerBlue3}“कथमनेकस्माद्”} भवति (।) उदाहरणमाह । {\color{DodgerBlue3}“क्लेदवत्”} तण्डुलाद्यवयवशैथिल्यमिव {\color{DodgerBlue3}“दुग्धाद्वारिणश्च”} । अग्निसहकारि दुग्धं वारि च स्वयं पृथक् तण्डुलादेर्व्विक्लित्तिं जनयद् दृश्यते ॥
	\pend
      
	  \bigskip
	  \begingroup
	  \large
	
	    
	    \stanza[\smallbreak]
	\label{pv.4.253b}\edlabel{pv.4.253b}\flagstanza{\tiny\textenglish{...4.253b}}द्रवशक्तेः यतः क्लेदः सा त्वकैव द्वयोरपि ॥ २५३ ॥\&[\smallbreak]


	
	  \endgroup
	

	  \pstart (सिद्धान्ती) अयुक्तमेतत् । अन्यो हि दुग्धेन जनितः क्लेदोऽन्यश्च वारिणा । रसवीर्यपरिणामाकारभेदात् । तत् कथमेकस्मादनेकोत्पादः ।
	\pend
      

	  \pstart अथ सत्यप्यवान्तरभेदे विजातीयव्यावृत्त्या {\color{DodgerBlue3}“क्लेद”} एक उच्यते तदा दुग्धवारिणोर्यतो यस्या {\color{DodgerBlue3}“द्रव्य”}जनिकायाः {\color{DodgerBlue3}“शक्तेः”} क्लेदो जायते {\color{DodgerBlue3}“सा”} शक्तिस्तु {\color{DodgerBlue3}“द्वयोरपि”} विजातीयव्यावृत्ते{\color{DodgerBlue3}“रेकैव”} । तत् कथमनेकस्मादेकोत्पत्तिः ॥ (२५३)
	\pend
      \label{div_pvv.4.254}\edlabel{div_pvv.4.254}
	  
	% new div opening: depth here is 2
	
	  \bigskip
	  \begingroup
	  \large
	
	    
	    \stanza[\smallbreak]
	\label{pv.4.254}\edlabel{pv.4.254}\flagstanza{\tiny\textenglish{....4.254}}भिन्नाभिन्नः किमस्यात्मा भिन्नोथ द्रवता कथम् ।&अभिन्नेत्युच्यते बुद्धेस्तद्रूपाया अभेदतः ॥ २५४ ॥\&[\smallbreak]


	
	  \endgroup
	

	  \pstart (परः) यदि दुग्धस्य क्लेदजननशक्त्यात्मतया वारिणा सहाभेद इष्यते तदा {\color{DodgerBlue3}“भिन्नाभिन्नोऽस्यात्मा किम”}भ्युपगन्तव्यः । शक्त्यात्मतयाऽभेदात् । आकारभेदाच्च भेदात् । उत्तरमाह (।)
	\pend
      

	  \pstart न भिन्नाभिन्न आत्मा किन्तु {\color{DodgerBlue3}“भिन्न”} एव । {\color{DodgerBlue3}“कथन्तर्हि”} द्वयोरपि {\color{DodgerBlue3}“द्रवता”}ऽभिन्ना क्लेदहेतुरित्युच्यते द्रवत्वस्याभेदात् । अभेदोक्तिर्भेदोक्त्या विरुध्यते ।
	\pend
      \leavevmode\marginnote{\textenglish{503/s}}

	  \pstart यदि नाम स्वस्वभावस्थितत्वाद् भावानां भेद एव पारमार्थिकः तथापि केचिद् भावा भिन्ना अपि स्वहेतुबलायातस्वरूपविशेषात् एककार्यकृत इति {\color{DodgerBlue3}“बुद्धेर्विक”}ल्पिकाया{\color{DodgerBlue3}“स्तद्रूपाया”} एकप्रत्यवमर्शाकाराया {\color{DodgerBlue3}“अभेदतः”} कारणात् द्रवताक्लेदहेतु{\color{DodgerBlue3}“रभिन्नेत्युच्यते”} । तदेकव्यावृत्तिविषयस्यावसायस्यानुरोधात् । एकत्वव्यवहारो न पारमार्थिक इत्यर्थः । (२५४)
	\pend
      \label{div_pvv.4.255}\edlabel{div_pvv.4.255}
	  
	% new div opening: depth here is 2
	
	  \bigskip
	  \begingroup
	  \large
	
	    
	    \stanza[\smallbreak]
	\label{pv.4.255}\edlabel{pv.4.255}\flagstanza{\tiny\textenglish{....4.255}}तद्वद् भेदेपि दहनो दहनप्रत्ययाश्रयः ।&येनांशेनादधद् धूमं तेनांशेन तथा गतिः ॥ २५५ ॥\&[\smallbreak]


	
	  \endgroup
	

	  \pstart {\color{DodgerBlue3}“तद्वत्”} क्षीरोदकवत् । अवान्तर{\color{DodgerBlue3}“भेदे”} सत्यपि {\color{DodgerBlue3}“दहनो येनांशेन”} स्वभावेन स्वरूपोष्णस्पर्शाद्यात्मकेनानग्निव्यावृत्तेन {\color{DodgerBlue3}“दहन”} इति {\color{DodgerBlue3}“प्रत्यय”}स्य व्यवहारस्य{\color{DodgerBlue3}“श्रयो”} विषयस्तेन स्वभावेन धूममादधत् जनयत् धूमा{\color{DodgerBlue3}“श्र”}यो भवति (।) {\color{DodgerBlue3}“तेन धूमाश्रयत्वेन”} कारणेन धूमालिङ्गाद् दहनो हेतुः । {\color{DodgerBlue3}“तथा”}ऽनग्निव्यावृत्तित्वेन गम्यते । (२५५)
	\pend
      \label{div_pvv.4.256}\edlabel{div_pvv.4.256}
	  
	% new div opening: depth here is 2
	

	  \pstart तथा हि (।)
	\pend
      
	  \bigskip
	  \begingroup
	  \large
	
	    
	    \stanza[\smallbreak]
	\label{pv.4.256}\edlabel{pv.4.256}\flagstanza{\tiny\textenglish{....4.256}}दहनप्रत्ययाङ्गादेवान्यापेक्षात् समुद्भवात् ।&धूमोतद्व्यभिचारीति तद्वत् कार्यं तथापरम् ॥ २५६ ॥\&[\smallbreak]


	
	  \endgroup
	

	  \pstart {\color{DodgerBlue3}“दहनप्रत्यय”}स्या{\color{DodgerBlue3}“ङ्गा”}न्निमित्ता{\color{DodgerBlue3}“देवा”}ग्निलक्षणाद{\color{DodgerBlue3}“न्यापेक्षादा”}र्द्रेन्धनादिसहायात् {\color{DodgerBlue3}“समुदभवा”}दुत्पत्तेः कारणाद् {\color{DodgerBlue3}“धूमोऽतद्व्यभिचारीति”} तद्वत् धूमवत् । {\color{DodgerBlue3}“अपरमपि\edlabel{pvv.503-1}\footnote{\label{pvv.503-1}  १ देशादिभिन्नं विजातीयव्यावृत्त्या त्वभिन्नं ।} कार्यं”} तथा स्वकारणाव्यभिचारि बोद्धव्यं । (२५६)
	\pend
      \label{div_pvv.4.257}\edlabel{div_pvv.4.257}
	  
	% new div opening: depth here is 2
	

	  \pstart किं पुनर्द्धूमो वह्रिं न व्यभिचरतीत्याह\edlabel{pvv.503-2}\footnote{\label{pvv.503-2}  २ एतदेव स्थिरयितुमाह ।} ।
	\pend
      
	  \bigskip
	  \begingroup
	  \large
	
	    
	    \stanza[\smallbreak]
	\label{pv.4.257}\edlabel{pv.4.257}\flagstanza{\tiny\textenglish{....4.257}}धूमेन्धनविकाराङ्गतापदे दहनस्थितेः ।&अनग्निश्चेदधूमोऽसौ सधूमश्चेत् स पावकः ॥ २५७ ॥\&[\smallbreak]


	
	  \endgroup
	

	  \pstart {\color{DodgerBlue3}“धूम”}स्येन्धन{\color{DodgerBlue3}“विकारा”}देर{\color{DodgerBlue3}“ङ्गं”} हेतुस्तस्य भावो धूमेन्धनविकाराङ्गता तस्याः पदे आश्रयवस्तुनि {\color{DodgerBlue3}“दहनस्थिते”}रग्निताव्यवहाराच्छक्रमूर्धा{\color{DodgerBlue3}“ऽनग्नि”}र्द्धूमेन्धनविकारकारी न {\color{DodgerBlue3}“चेदि”}ष्यते (।) ततो जायमानः पदार्थो{\color{DodgerBlue3}“सौ”} दृश्यमानोऽ{\color{DodgerBlue3}“धू”}मो वाष्पादिरेव । {\color{DodgerBlue3}“न”} ह्यग्निजन्योऽनग्निर्भवति । सर्व्वथा साम्यात् {\color{DodgerBlue3}“सधूमश्चेदि”}ष्यते तर्हि {\color{DodgerBlue3}“स”} वल्मीकः {\color{DodgerBlue3}“पावक”} एव धूमजनकस्यैव पावकत्वात् । देशकालस्वभावप्रतीतिनियमो हि भावानां नाकस्मिकः । तथात्वेऽतिप्रसङ्गात् । न च नानाहेतुकः प्रत्येकः व्यभिचारेऽहेतुत्वप्रसङ्गात् तस्मान्नियतहेतुकः । यदि चान्वयव्यतिरेकाभ्यामुपलब्धाग्निकारण\leavevmode\marginnote{\textenglish{504/s}} \leavevmode\marginnote{\textenglish{103b/MA}}भावो धूमः शक्रमूर्ध्नो जायते । तदा वह्नेरेव शक्रमूर्द्धा । अथानग्निस्तथाऽदृश्यमानत्वात् तथा सोपि न धूमो वास्यादिरेव आकारसाम्यात्तु धूमाभः । तस्मात्\edlabel{pvv.504-1}\footnote{\label{pvv.504-1}  १ एवं तावत् कार्यस्य नियमः ।} कार्यकारणभावस्य नान्तरीयक\edlabel{pvv.504-2}\footnote{\label{pvv.504-2}  २ नियमः ।}तेति स्थितं ॥ (२५७)
	\pend
      \label{div_pvv.4.258}\edlabel{div_pvv.4.258}
	  
	% new div opening: depth here is 2
	
	  \bigskip
	  \begingroup
	  \large
	
	    
	    \stanza[\smallbreak]
	\label{pv.4.258}\edlabel{pv.4.258}\flagstanza{\tiny\textenglish{....4.258}}नान्तरीयकता ज्ञेया यथास्वं हेत्वपेक्षया ।&स्वभावस्य यथोक्तं प्राक् विनाशकृतकत्वयोः ॥ २५८ ॥\&[\smallbreak]


	
	  \endgroup
	

	  \pstart {\color{DodgerBlue3}“स्वभावस्य”} च हेतो{\color{DodgerBlue3}“र्नान्तरीयकता”} साध्या अविनाभाविता {\color{DodgerBlue3}“ज्ञेया । यथास्वं”} यस्य स्वभाव{\color{DodgerBlue3}“हेतोर्य”} आत्मीयस्तादात्म्यसाधको हेतुः साधनं तस्या{\color{DodgerBlue3}“पेक्षया । यथा प्रागुक्तं विनाशकृतकत्वयो”}स्तादात्म्यसाधनं अवश्यं हि कृतकानाम्विनाशः । न चार्थसापेक्षाणामवश्यं भावः । ततः स्वभावत एव कृतकानन्नश्वराणां विनाशं प्रत्यनपेक्षित्वादिति दर्शितं । तदेव स्वभावकार्ययोरव्यभिचारित्वाद्धेतुत्वं । न च प्राणादेरात्मकार्यत्वं सिद्धमिति कथं हेतुता । (२५८)
	\pend
      \label{div_pvv.4.259}\edlabel{div_pvv.4.259}
	  
	% new div opening: depth here is 2
	

	  \begin{center}%% label @type='head'
	\textbf{ख. प्राणादेरुक्तो दोष आचार्येण}
	\end{center}
	

	  \pstart नन्वयं प्राणादेरुक्तो दोषकलापि आ चा र्ये णे ष्ट इति कथं गम्यत इत्याह ।
	\pend
      
	  \bigskip
	  \begingroup
	  \large
	
	    
	    \stanza[\smallbreak]
	\label{pv.4.259}\edlabel{pv.4.259}\flagstanza{\tiny\textenglish{....4.259}}अहेतुत्वगतिन्यायः सर्वोयं व्यतिरेकिणः ।&अभ्यूह्यः श्रावणत्वोक्तेः कृतायाः साम्यदृष्टये ॥ २५९ ॥\&[\smallbreak]


	
	  \endgroup
	

	  \pstart सर्व्वस्यासादारणस्य दोषदुष्टत्वेन {\color{DodgerBlue3}“साम्यदृष्टये”} तुल्यतोपदर्शनाया चा र्ये ण {\color{DodgerBlue3}“श्रावणत्व”}स्यासाधारणस्य {\color{DodgerBlue3}“योक्तिः कृता”} तस्या {\color{DodgerBlue3}“एवायं व्यतिरेकिणः”} प्राणादेरन्यस्य च हेतोर{\color{DodgerBlue3}“हेतुत्वगतिन्यायः सर्व्वोऽभ्यूह्यः”} । (२५९)
	\pend
      
	  
	% new div opening: depth here is 1
	
\section[{(७) अनुपलब्धिचिन्ता}]{(७) अनुपलब्धिचिन्ता}

	  \begin{center}%% label @type='head'
	\textbf{(१) अनुपलब्धेः पृथगग्रहणे कारणम्}
	\end{center}
	\label{div_pvv.4.260}\edlabel{div_pvv.4.260}
	  
	% new div opening: depth here is 2
	

	  \pstart ननु यथा स्वभावकार्यसिद्ध्यर्थं द्वौ हेतू उक्तौ तथाऽनुपलब्धिरपि वक्तुं युक्ता हेतुत्वात् । स्वभावानुपलब्धिस्तावत् तादात्म्यप्रतिबन्धात् स्वभावहेतोर्न भिद्यत इति स्वभावहेतुर्निर्देशादेव निर्द्दिष्टा । कारणव्यापकानुपलब्धिभ्याञ्च निषेध्यानुपलब्धिरेव प्रतिपाद्यत इति न तेपि स्वभावानुपलब्धेर्भिद्यन्ते । उदाहरणन्तर्हि कस्मान्नोक्तमित्याह ।
	\pend
      \leavevmode\marginnote{\textenglish{505/s}}
	  \bigskip
	  \begingroup
	  \large
	
	    
	    \stanza[\smallbreak]
	\label{pv.4.260}\edlabel{pv.4.260}\flagstanza{\tiny\textenglish{....4.260}}हेतुस्वभावनिवृत्त्यैवार्थनिवृत्तिवर्णनात् ।&सिद्धोदाहरणेत्युक्तानुपलब्धिः पृथग् न तु ॥ २६० ॥\&[\smallbreak]


	
	  \endgroup
	

	  \pstart त्रिविधाप्यनुपलब्धिर्व्विपक्षा{\color{DodgerBlue3}“द्धेतोः”} कारणस्य {\color{DodgerBlue3}“स्वभाव”}स्य व्यापकस्य निवृत्यैवार्थस्य कार्यस्य व्याप्यस्य च {\color{DodgerBlue3}“निवृत्तेर्व्वर्णनात् सिद्धोदाहरणेति न पृथगुक्ता”} । स्वभा{\color{DodgerBlue3}“वानुप(ल)ब्धिः”} कारणव्यापकाभावसाधिका । तयोरनुपलब्धिस्तु कार्यव्याप्याभावसाधनीति त्रिविधानुपलम्भोदाहरणं सिद्धं । (२६०)
	\pend
      \label{div_pvv.4.261}\edlabel{div_pvv.4.261}
	  
	% new div opening: depth here is 2
	

	  \pstart उपलब्धिलक्षणप्राप्तविषयत्वमनुपलब्धेर्व्विशेषणं कथं {\color{DodgerBlue3}“सिद्धमिति चेदाह”} ।
	\pend
      
	  \bigskip
	  \begingroup
	  \large
	
	    
	    \stanza[\smallbreak]
	\label{pv.4.261}\edlabel{pv.4.261}\flagstanza{\tiny\textenglish{....4.261}}तत्राप्यदृश्यात् पुरुषात् प्राणादेरनिवर्तनात् ।&सन्देहहेतुताख्यात्या दृश्येर्थे सेति सूचितम् ॥ २६१ ॥\&[\smallbreak]


	
	  \endgroup
	

	  \pstart {\color{DodgerBlue3}“प्राणादे”}र्हेतोर{\color{DodgerBlue3}“दृश्यात् पुरुषादा”}त्म\edlabel{pvv.505-1}\footnote{\label{pvv.505-1}  १ अन्यत्रादृष्टरूपस्येत्यादिनात्मनोऽदृश्यस्याव्यतिरेकेण ।}नोऽ{\color{DodgerBlue3}“निवृत्ते”}र्निवृत्त्यसिद्धे जीवच्छरीरे {\color{DodgerBlue3}“सन्देहहेतु”}तया {\color{DodgerBlue3}“आख्यात्या”} कथनेन तत्रानुपलब्धावपि {\color{DodgerBlue3}“दृश्येऽर्थे”} याऽनुपलब्धिः {\color{DodgerBlue3}“सा”} भावसाधिका नान्ये{\color{DodgerBlue3}“ति सूचितमा”} चा र्ये ण । यदि त्वनुपलब्धिरित्येव प्रमाणं तदाऽत्मनि प्राणादेर्निवृत्तिः सिध्येत् । ततश्च जीवच्छरीरे वर्त्तमानमात्मानं गमयेदिति न सन्देहहेतुः स्यात् । (२६१)
	\pend
      \label{div_pvv.4.262}\edlabel{div_pvv.4.262}
	  
	% new div opening: depth here is 2
	

	  \pstart ननु यथा व्यवच्छेदविषयाऽनुपलब्धिः तथा कार्यस्वभावावपीति अनुपलब्धिरेवैको हेतुः स्यादित्याह ।
	\pend
      
	  \bigskip
	  \begingroup
	  \large
	
	    
	    \stanza[\smallbreak]
	\label{pv.4.262}\edlabel{pv.4.262}\flagstanza{\tiny\textenglish{....4.262}}अनङ्गीकृतवस्त्वंशो निषेधः साध्यतेनया ।&वस्तुन्यपि तु पूर्वाभ्यां पर्युदासो विधानतः ॥ २६२ ॥\&[\smallbreak]


	
	  \endgroup
	

	  \pstart {\color{DodgerBlue3}“अनया”}नुपलब्ध्या न केवल{\color{DodgerBlue3}“म्वस्तुनि”} अवस्तुन्यपि {\color{DodgerBlue3}“निषेधः”} केवलो {\color{DodgerBlue3}“नाङ्गीकृतवस्त्वंशः”} कारणव्यापकानुपलब्धिभ्यां वस्तुव्यवच्छेदमात्रं अभावव्यवहारश्च साध्यते (।) यथा प्रदेशे धर्मिणि घटाभावः । {\color{DodgerBlue3}“पूर्व्वाभ्यां”} कार्यस्वभावाभ्यां त्वन्यव्यवच्छेदे नैकस्य {\color{DodgerBlue3}“विधानतः पर्युदासः”} साध्यते (।) यथाऽनग्निव्यवच्छेदेनाग्निः । नित्यव्यवच्छेदेनानित्यत्वं । (२६२)
	\pend
      \label{div_pvv.4.263}\edlabel{div_pvv.4.263}
	  
	% new div opening: depth here is 2
	

	  \pstart ननु स्वभावानुपलब्धौ कथं तादात्म्यं प्रतिबन्ध इत्याह । \leavevmode\marginnote{\textenglish{104a/MA}}
	\pend
      
	  \bigskip
	  \begingroup
	  \large
	
	    
	    \stanza[\smallbreak]
	\label{pv.4.263a}\edlabel{pv.4.263a}\flagstanza{\tiny\textenglish{...4.263a}}तत्रोपलभ्येष्वस्तित्वमुपलब्धेर्न चापरम् ।\&[\smallbreak]


	
	  \endgroup
	

	  \pstart {\color{DodgerBlue3}“तत्रोपलभ्येषु”} भावेषु विचार्यमाणम{\color{DodgerBlue3}“स्तित्वमुपलब्धेरपरं न”} भवति । यदि ह्युपलब्धिः कर्मधर्मस्तदोपलभ्यमानतास्तित्वं । अथ कर्तृधर्मो ज्ञानं । तदा \leavevmode\marginnote{\textenglish{506/s}} तदन्वयव्यतिरेकानुविधानाद् भावसत्ताव्यवस्थानस्योपचारात् सैव सत्ता । यथा चोपलब्धिरेव सत्ता तथाऽनुपलब्धिरेवासत्तेति तादात्म्यमनयोः सम्बन्धः ।
	\pend
      

	  \begin{center}%% label @type='head'
	\textbf{अनुपलब्धेरभावसिद्धौ लिङ्गलिङ्गिग्रहणप्रयोजनम्}
	\end{center}
	

	  \pstart एवं तर्ह्यनुपलब्धिरेवाभावसिद्धिरित्यलं लिङ्गिलिङ्गभावेनेत्याह ।
	\pend
      
	  \bigskip
	  \begingroup
	  \large
	
	    
	    \stanza[\smallbreak]
	\label{pv.4.263b}\edlabel{pv.4.263b}\flagstanza{\tiny\textenglish{...4.263b}}इत्यज्ञज्ञापनायैकानुपाख्योदाहृतिर्मता ॥ २६३ ॥\&[\smallbreak]


	
	  \endgroup
	

	  \pstart अभावः सिद्धत्वान्न साधनार्हः । तस्मादभावं पश्यतोप्यव्यवहरतो{\color{DodgerBlue3}“ऽज्ञ”}स्य मूढस्य {\color{DodgerBlue3}“ज्ञापना”}याभावव्यवहाराय {\color{DodgerBlue3}“एका”} स्वभावानुपलम्भो{\color{DodgerBlue3}“दाहृतिर्मता”} (।) सा चानुपलब्धिर{\color{DodgerBlue3}“नुपाख्या”} वस्तुतोऽभावात्मिका । (२६३)
	\pend
      \label{div_pvv.4.264_4.265}\edlabel{div_pvv.4.264_4.265}
	  
	% new div opening: depth here is 2
	

	  \begin{center}%% label @type='head'
	\textbf{(३) स्वभावानुपलब्ध्या विषयिणः प्रतिषेधः}
	\end{center}
	

	  \pstart यद्यभावः स्वभावानुपलब्ध्या न साध्यते किन्तर्हि साध्यत इत्याह ।
	\pend
      
	  \bigskip
	  \begingroup
	  \large
	
	    
	    \stanza[\smallbreak]
	\label{pv.4.264}\edlabel{pv.4.264}\flagstanza{\tiny\textenglish{....4.264}}विषयासत्त्वतस्तत्र विषयि प्रतिषिध्यते ।&ज्ञानाभिधानसन्देहं यथाऽदाहादपावकः ॥ २६४ ॥\&[\smallbreak]


	
	  \endgroup
	
	  \bigskip
	  \begingroup
	  \large
	
	    
	    \stanza[\smallbreak]
	\label{pv.4.265}\edlabel{pv.4.265}\flagstanza{\tiny\textenglish{....4.265}}तथान्या नोपलभ्येषु नास्तितानुपलम्भनात् ।&तज्ज्ञानशब्दाः साध्यन्ते तद्भावात् तन्निबन्धनाः ॥ २६५ ॥\&[\smallbreak]


	
	  \endgroup
	

	  \pstart {\color{DodgerBlue3}“तत्र”} स्वभावानु(प)लम्भे {\color{DodgerBlue3}“विषयासत्वतो”} ज्ञानाभिधान\edlabel{pvv.506-1}\footnote{\label{pvv.506-1}  १ सत्ताज्ञानं । तदभिधानं । अस्ति न वेति सन्देहः ।}सन्देहानां विषयस्यासत्त्वतः  कारणाद् {\color{DodgerBlue3}“विषयि प्रतिषिध्यते”} । किन्तद्विषयीत्याह । ज्ञानमभिधानञ्च सन्देहश्च {\color{DodgerBlue3}“ज्ञानाभिधानसंदेहं”} । सदिति ज्ञानं सदित्यभिधानं । अस्ति न वेति सन्देहश्चानुपलब्धिलक्षणादभावान्निषिध्यते (।) सत्त्वविषया हि सत्-ज्ञानादयः । तदभावे निषिध्यन्त इति न्याय्यं । दृष्टान्तमाह । {\color{DodgerBlue3}“यथा”} गुञ्जादौ वह्नित्वसन्देहे कश्चिदाह पावकोयमिति (।) दाहादिनिमित्तो हि वह्नित्वव्यवहारः तदभावात् प्रतिषिद्धः । यथोपलब्धेरन्या नास्तिता तथोपलभ्येषूपलब्धिलक्षणप्राप्तेष्वनुपलम्भादन्या नास्तिता न भवति । किन्त्वनुपलब्धिरेव नास्तित्वं (।) तस्मात् तन्निबन्धना अनुपलब्धिनिमित्ताः तत् ज्ञानाभिधानव्यवहारास्तस्यानुपलम्भनस्य भावात् साध्यन्ते । (२६४,२६५)
	\pend
      \label{div_pvv.4.266}\edlabel{div_pvv.4.266}
	  
	% new div opening: depth here is 2
	

	  \pstart यद्यनुपलम्भेन निमित्तेन नैमित्तिकोऽसत् ज्ञानादिः साध्यते तदा निमित्ते \leavevmode\marginnote{\textenglish{507/s}}  सति नैमित्तिकभावनियमाभावात् सत्यनुपलम्भेऽसत्‌ज्ञानादिव्यवहार ऐकान्तिको न स्यादित्याह (।)
	\pend
      
	  \bigskip
	  \begingroup
	  \large
	
	    
	    \stanza[\smallbreak]
	\label{pv.4.266}\edlabel{pv.4.266}\flagstanza{\tiny\textenglish{....4.266}}सिद्धो हि व्यवहारोयं दृश्यादृष्टावसन्निति ।&तस्याः सिद्धावसन्दिग्धौ तत्कार्यत्वेपि धीध्वनी ॥ २६६ ॥\&[\smallbreak]


	
	  \endgroup
	

	  \pstart सर्व्वस्यैव {\color{DodgerBlue3}“हि दृश्यादृष्टावसन्निति व्यवहारोयं”} निमित्तान्तरनिरपेक्षः {\color{DodgerBlue3}“सिद्धः । तस्या”} दृश्यादृष्टेः {\color{DodgerBlue3}“सिद्धौ”} सत्यां असदिति {\color{DodgerBlue3}“धीध्वनी कार्यत्वे”}प्य{\color{DodgerBlue3}“सदिग्धौ”} नियत{\color{DodgerBlue3}“प्रव”}र्त्तनौ सिद्धौ । यद्यपि कार्यं न कारणेन नियतभावं तथापि यदि क्वचिद् भावव्यवहारः प्रवर्त्त्यते तदानुपलब्धिमाननिमित्तत्वादन्यत्राप्यनुपलब्धौ सत्यां स प्रवर्त्तनीय इत्यर्थः । (२६६)
	\pend
      \label{div_pvv.4.267}\edlabel{div_pvv.4.267}
	  
	% new div opening: depth here is 2
	
	  \bigskip
	  \begingroup
	  \large
	
	    
	    \stanza[\smallbreak]
	\label{pv.4.267}\edlabel{pv.4.267}\flagstanza{\tiny\textenglish{....4.267}}विद्यमानेपि विषये मोहादत्राननुब्रुवन् ।&केवलं सिद्धसाधर्म्यात् स्मार्यते समयं परः ॥ २६७ ॥\&[\smallbreak]


	
	  \endgroup
	

	  \pstart अभावव्यवहारस्य {\color{DodgerBlue3}“विषये”} दृश्यादर्शने {\color{DodgerBlue3}“विद्यमाने”}पि {\color{DodgerBlue3}“केवलं मोहादसत् ज्ञान”}शब्दव्यवहारान{\color{DodgerBlue3}“ननुब्रुवन्”} नानुवदन् परोऽसद्व्यवहारविषयतया {\color{DodgerBlue3}“सिद्धे”}न घटेन दृष्टान्तेन दृश्यादर्शनवत्तायाः {\color{DodgerBlue3}“साधर्म्यात् समयं”} व्यवहारं {\color{DodgerBlue3}“स्मार्यते”} । पूर्व्वमपि त्वया दृश्यादर्शनमात्रकोऽसद्व्यवहारः प्रवर्त्तितः । तत्स{\color{DodgerBlue3}“द्भावा”}दिहापि प्रवर्त्तयेति परः प्रतिपाद्यते । (२६७)
	\pend
      \label{div_pvv.4.268}\edlabel{div_pvv.4.268}
	  
	% new div opening: depth here is 2
	

	  \pstart दृष्टान्तमाह ।
	\pend
      
	  \bigskip
	  \begingroup
	  \large
	
	    
	    \stanza[\smallbreak]
	\label{pv.4.268}\edlabel{pv.4.268}\flagstanza{\tiny\textenglish{....4.268}}कार्यकारणता यद्वत् साध्यते दृष्ट्यदृष्टितः ।&कार्यादिशब्दा हि तयोर्व्यवहाराय कल्पिताः ॥ २६८ ॥\&[\smallbreak]


	
	  \endgroup
	

	  \pstart {\color{DodgerBlue3}“कार्यकारणता दृष्ट्यदृष्टितो”} दर्शनादर्शनाभ्यामन्वयव्यतिरेकग्राहकाभ्यां {\color{DodgerBlue3}“यद्वत् साध्य”}ते । न हि वस्तुतो दर्शनादर्शनाभ्यामन्या कार्यकारणता । किन्तु {\color{DodgerBlue3}“तयो”}र्दर्शनादर्शनयो{\color{DodgerBlue3}“र्हि”} संक्षेपेण {\color{DodgerBlue3}“व्यवहाराय कार्यादिशब्दाः कल्पिताः”} । (२६८) ततश्च (।)
	\pend
      \label{div_pvv.4.269}\edlabel{div_pvv.4.269}
	  
	% new div opening: depth here is 2
	
	  \bigskip
	  \begingroup
	  \large
	
	    
	    \stanza[\smallbreak]
	\label{pv.4.269}\edlabel{pv.4.269}\flagstanza{\tiny\textenglish{....4.269}}कारणात् कार्यसंसिद्धिः स्वभावान्तर्गमादियम् ।&हेतुप्रभेदाख्याने न दर्शितोदाहृतिः पृथक् ॥ २६९ ॥\&[\smallbreak]


	
	  \endgroup
	

	  \pstart {\color{DodgerBlue3}“कारणात्”} दृश्यादर्शनात् असत्ज्ञानशब्दव्यवहार{\color{DodgerBlue3}“कार्य”}योग्यता{\color{DodgerBlue3}“संसिद्धिः”} 104b {\color{DodgerBlue3}“स्वभाव”}हेताव{\color{DodgerBlue3}“न्तर्गमात्र”} हेत्वन्तरं । तेनेयं स्वभावानुपलब्धि{\color{DodgerBlue3}“र्हेतु”}ना {\color{DodgerBlue3}“प्रभेदस्याख्याने”} क्रियमाणे स्वभावहेतुनैव {\color{DodgerBlue3}“दर्शितोदाहृतिर्न पृथक्”} निर्द्दिष्टा । (२६९)
	\pend
      \label{div_pvv.4.270}\edlabel{div_pvv.4.270}
	  
	% new div opening: depth here is 2
	

	  \pstart भवतु तावद् दृश्यानुपलब्धेरसद्व्यवहारयोग्यतासिद्धिः । सैव तु कथं सिध्य- तीत्याह ।
	\pend
      \leavevmode\marginnote{\textenglish{508/s}}
	  \bigskip
	  \begingroup
	  \large
	
	    
	    \stanza[\smallbreak]
	\label{pv.4.270}\edlabel{pv.4.270}\flagstanza{\tiny\textenglish{....4.270}}एकोपलम्भानुभवादिदं नोपलभे इति ।&बुद्धेरुपलभे वेति कल्पिकायाः समुद्भवः ॥ २७० ॥\&[\smallbreak]


	
	  \endgroup
	

	  \pstart {\color{DodgerBlue3}“एक”}ज्ञानसंसर्गिण एकस्य प्रदेशस\edlabel{pvv.508-1}\footnote{\label{pvv.508-1}  १ अनन्यसंसर्गिणः ।}यो{\color{DodgerBlue3}“पलम्भानुभवात्”} स्वसम्वेदनादनन्तरं घटादि नोपलभ्यते । {\color{DodgerBlue3}“इदं”} प्रदेशोभ्युपल  भ्यत इति {\color{DodgerBlue3}“कल्पिकाया बुद्धेः”} स्ववेदन- विषयीकृतविधिप्रतिषेधानुकारिण्याः {\color{DodgerBlue3}“समुद्भवो”} भवतीति स्वसम्वेदनादेवानुपलम्भ- सिद्धिः । (२७०)
	\pend
      \label{div_pvv.4.271}\edlabel{div_pvv.4.271}
	  
	% new div opening: depth here is 2
	

	  \begin{center}%% label @type='head'
	\textbf{(४) विशिष्टवेदनादर्थाकां विशेषावगमः}
	\end{center}
	

	  \pstart भवतु ज्ञानं स्वसम्वेदनविषययोर्ज्ञानयोरन्योन्यभेदस्तु केन ज्ञायते इत्याह\edlabel{pvv.508-2}\footnote{\label{pvv.508-2}  २ एकोपलम्भादिति वाच्येऽनुभवग्रहणप्रयोजनमाह ।} ।
	\pend
      
	  \bigskip
	  \begingroup
	  \large
	
	    
	    \stanza[\smallbreak]
	\label{pv.4.271}\edlabel{pv.4.271}\flagstanza{\tiny\textenglish{....4.271}}विशेषो गम्यतेऽर्थानां विशिष्टादेव वेदनात् ।&तथाभूतात्मसंपत्तिर्भेदधीहेतुरस्य च ॥ २७१ ॥\&[\smallbreak]


	
	  \endgroup
	

	  \pstart {\color{DodgerBlue3}“अर्थानां विषेशो”} (? शेषो)ऽन्योन्यं भेदो गम्यते (।) {\color{DodgerBlue3}“विशिष्ट”}प्रतिनियता- कारा{\color{DodgerBlue3}“देव \edlabel{pvv.508-3}\footnote{\label{pvv.508-3}  ३ नार्थसत्तामात्रेण ।} वेदनात्”} । अस्य सम्वेदनस्य तयो{\color{DodgerBlue3}“र्भेदा”}नियताकारत्वं तस्य धियः सम्वेदनस्य तु साधनं\edlabel{pvv.508-4}\footnote{\label{pvv.508-4}  ४ किमित्याह ।} {\color{DodgerBlue3}“तथाभूता”} प्रतिनियताकारा परनिरपेक्षप्रकाशा{\color{DodgerBlue3}“त्मसंपत्ति”}- रपरोक्षता । (२७१)
	\pend
      \label{div_pvv.4.272}\edlabel{div_pvv.4.272}
	  
	% new div opening: depth here is 2
	
	  \bigskip
	  \begingroup
	  \large
	
	    
	    \stanza[\smallbreak]
	\label{pv.4.272}\edlabel{pv.4.272}\flagstanza{\tiny\textenglish{....4.272}}तस्मात् स्वतो धियोर्भेदसिद्धिस्ताभ्यां तदर्थयोः ।&अन्यथा ह्यनवस्थातो भेदः सिध्येन्न कस्यचित् ॥ २७२ ॥\&[\smallbreak]


	
	  \endgroup
	

	  \pstart {\color{DodgerBlue3}“तस्माद् धियो भेदस्य स्वतः”} प्रतिनियतात् अपरोक्षप्रकाशात् स्वरूपत\edlabel{pvv.508-5}\footnote{\label{pvv.508-5}  ५ नार्थस्यैवान्यतो ज्ञानात् ।}एव {\color{DodgerBlue3}“सिद्धिः । ताभ्यां”} भिन्नतया सिद्धाभ्यां तयो{\color{DodgerBlue3}“रर्थयोः”} सारूप्यजनकयोर्भेदसिद्धिः । {\color{DodgerBlue3}“अन्यथा”} यद्येवं नेष्यते तदाऽपराभ्यां भेदसिद्धिर्व्वक्तव्या । तयोश्च भेदसिद्धौ भेदग्राहकता युक्तेति तद्भेदग्राहकमपरं द्वयमेष्टव्यं । एवमपरापरापेक्षाया- {\color{DodgerBlue3}“मनवस्थातः न कस्यचिद् भेदः सिध्येत्”} । (२७२)
	\pend
      \label{div_pvv.4.273}\edlabel{div_pvv.4.273}
	  
	% new div opening: depth here is 2
	
	  \bigskip
	  \begingroup
	  \large
	
	    
	    \stanza[\smallbreak]
	\label{pv.4.273}\edlabel{pv.4.273}\flagstanza{\tiny\textenglish{....4.273}}विशिष्टरूपानुभवादन्यथान्यनिराक्रिया ।&तद्विशिष्टोपलम्भोतः तस्याप्यनुपलम्भनम् ॥ २७३ ॥\&[\smallbreak]


	
	  \endgroup
	

	  \pstart अतो {\color{DodgerBlue3}“विशिष्ट”}स्य नियतस्य {\color{DodgerBlue3}“रूप”}स्या{\color{DodgerBlue3}“नुभवाद”}न्य{\color{DodgerBlue3}“थान्य”}स्य {\color{DodgerBlue3}“निराक्रिया”} न भवति । {\color{DodgerBlue3}“अतस्त”}स्मात् प्रतिषेध्याद् {\color{DodgerBlue3}“विशिष्ट”}स्य भिन्नस्य प्रदेशादे{\color{DodgerBlue3}“रुपलम्भस्तस्य”} प्रतिषे- ध्यस्या{\color{DodgerBlue3}“प्यनुपलम्भनं”} । (२७३)
	\pend
      \label{div_pvv.4.274}\edlabel{div_pvv.4.274}
	  
	% new div opening: depth here is 2
	\leavevmode\marginnote{\textenglish{509/s}}
	  \bigskip
	  \begingroup
	  \large
	
	    
	    \stanza[\smallbreak]
	\label{pv.4.274}\edlabel{pv.4.274}\flagstanza{\tiny\textenglish{....4.274}}तस्मादनुपलम्भोयं स्वयं प्रत्यक्षतो गतः ।&स्वमात्रवृत्तेर्गमकस्तदभावव्यवस्थितेः ॥ २७४ ॥\&[\smallbreak]


	
	  \endgroup
	

	  \pstart {\color{DodgerBlue3}“तस्मादयं”} ज्ञानात्मकाऽ{\color{DodgerBlue3}“नुपलम्भः स्वय”}मात्मना {\color{DodgerBlue3}“प्रत्यक्षतः”} स्वसम्वेदनाद् {\color{DodgerBlue3}“गतः”} प्रतीतः सन् {\color{DodgerBlue3}“स्वमात्रवृत्ते”}रात्ममात्रप्रतीताया{\color{DodgerBlue3}“स्त”}स्यानुपलभ्यमानस्या{\color{DodgerBlue3}“भावव्यवस्थिते- र्गमकः”} । अनुपलम्भो हि निमित्तं \edlabel{pvv.509-1}\footnote{\label{pvv.509-1}  १ ज्ञानेन निमित्तेन नैमित्तिको व्यवहारः साध्यः । अर्थो विषयोनुपलम्भश्चेद् विषयी व्यवहारः ।}विषयो वाऽभावव्यवहारस्येत्युभयथापि स्वसत्तामात्रेणाभावव्यवहारहेतुः । (२७४)
	\pend
      \label{div_pvv.4.275}\edlabel{div_pvv.4.275}
	  
	% new div opening: depth here is 2
	
	  \bigskip
	  \begingroup
	  \large
	
	    
	    \stanza[\smallbreak]
	\label{pv.4.275}\edlabel{pv.4.275}\flagstanza{\tiny\textenglish{....4.275}}[अन्यथार्थस्य नास्तित्वं गम्यतेनुपलम्भतः ।&उपलम्भस्य नास्तित्वमन्येनेत्यनवस्थितिः ॥ २७५ ॥\&[\smallbreak]


	
	  \endgroup
	

	  \pstart {\color{DodgerBlue3}“अन्यथा”} यदि स्वतो नानुपलम्भसिद्धिस्तदाऽ{\color{DodgerBlue3}“र्थस्य नास्तित्वमनुपलम्भस्ततो गम्यते”} (।) {\color{DodgerBlue3}“उपलम्भस्य नास्तित्वम”}नुपलब्धिर{\color{DodgerBlue3}“न्येना”}नुपलम्भेन गम्यते । सोप्य- नपलम्भान्तरेणे{\color{DodgerBlue3}“ति”} अपेक्षाया{\color{DodgerBlue3}“मनवस्थितिः”} स्यात् । (२७५)
	\pend
      \label{div_pvv.4.276}\edlabel{div_pvv.4.276}
	  
	% new div opening: depth here is 2
	

	  \begin{center}%% label @type='head'
	\textbf{(५) दृश्यानुपलब्धिः सद्व्यवहारबाधिका}
	\end{center}
	

	  \pstart भवतु तावद् दृश्येष्वनुपलब्धावभावप्रतीतिरदृश्ये पुनः कथमित्याह ।
	\pend
      
	  \bigskip
	  \begingroup
	  \large
	
	    
	    \stanza[\smallbreak]
	\label{pv.4.276}\edlabel{pv.4.276}\flagstanza{\tiny\textenglish{....4.276}}अदृश्ये निश्चयायोगात् स्थितिरन्यत्र बाध्यते ।&यथाऽलिङ्गोऽन्यसत्त्वेषु विकल्पादिर्न सिध्यति ॥ २७६ ॥\&[\smallbreak]


	
	  \endgroup
	

	  \pstart {\color{DodgerBlue3}“अन्यत्र”} दृश्यानुपलब्धौ सत्याम{\color{DodgerBlue3}“दृश्ये”} विषयेऽभाव{\color{DodgerBlue3}“निश्चयायोगात्”} । सद्व्यव- हारस्य {\color{DodgerBlue3}“स्थितिर्ब्बाध्यते”} उपलम्भपूर्व्वकत्वात् तस्याः । {\color{DodgerBlue3}“यथाऽन्येषु सत्त्वेषु प्राणिषु”} रागादिविषयो {\color{DodgerBlue3}“विकल्पो”} परचित्तज्ञानादिरादिनाऽलिङ्गो लिङ्गरहितः सद्व्यव- हारविषयत्वेन न सिध्यति । (२७६)
	\pend
      \label{div_pvv.4.277}\edlabel{div_pvv.4.277}
	  
	% new div opening: depth here is 2
	

	  \pstart किं पुनरदृश्यानुपलब्धौ सत्यामपि न सिध्यतीत्याह ।
	\pend
      
	  \bigskip
	  \begingroup
	  \large
	
	    
	    \stanza[\smallbreak]
	\label{pv.4.277a}\edlabel{pv.4.277a}\flagstanza{\tiny\textenglish{...4.277a}}अनिश्चयफला ह्येषा नालं व्यावृत्तिसाधने ।\&[\smallbreak]


	
	  \endgroup
	

	  \pstart {\color{DodgerBlue3}“एषा हि”} सत्यप्यर्थे सम्भवन्ती अभावस्या{\color{DodgerBlue3}“निश्चयफला”} (।) तस्माद् {\color{DodgerBlue3}“व्यावृत्ते”}- रभावस्य {\color{DodgerBlue3}“साधने नालं”} शक्ता ।
	\pend
      
	  \bigskip
	  \begingroup
	  \large
	
	    
	    \stanza[\smallbreak]
	\label{pv.4.277b}\edlabel{pv.4.277b}\flagstanza{\tiny\textenglish{...4.277b}}आद्याधिक्रियते हेतोर्न्निश्चितेनैव साधने ॥ २७७ ॥\&[\smallbreak]


	
	  \endgroup
	

	  \pstart {\color{DodgerBlue3}“आद्या”} दृश्यानुपलब्धिः पुनर्व्यावृत्तिसाधने{\color{DodgerBlue3}“ऽधिक्रियते”} । कस्मादित्याह । \leavevmode\marginnote{\textenglish{510/s}} \leavevmode\marginnote{\textenglish{105a/MA}}{\color{DodgerBlue3}“हेतो”}र्व्विपक्षात् कारणव्यापकानुपलब्धिभ्यां व्यावृत्ते{\color{DodgerBlue3}“र्निश्चिते\edlabel{pvv.510-1}\footnote{\label{pvv.510-1}  १ तृतीयेन रूपेण ।}नैव”} साध्यार्थ{\color{DodgerBlue3}“साधने”} ऽधिकारात् । न हि विपक्षादनिश्चितव्यावृत्तिको हेतुर्गमकः । न हि कारण- व्यापकानुपलब्धिभ्यामन्यो व्यावृत्तिसाधनः कारणविरुद्धोपलम्भादिष्वपि कारणा- भावादभावः प्रतिपाद्यः । (२७७ )
	\pend
      \label{div_pvv.4.278}\edlabel{div_pvv.4.278}
	  
	% new div opening: depth here is 2
	
	  \bigskip
	  \begingroup
	  \large
	
	    
	    \stanza[\smallbreak]
	\label{pv.4.278}\edlabel{pv.4.278}\flagstanza{\tiny\textenglish{....4.278}}तस्याः स्वयं प्रयोगेषु स्वरूपं वा प्रयुज्यते ।&अर्थबाधनरूपम्वा भावे भावादभावतः ॥ २७८ ॥\&[\smallbreak]


	
	  \endgroup
	

	  \pstart {\color{DodgerBlue3}“तस्या”} अनुपलब्धेः {\color{DodgerBlue3}“प्रयोगेषु स्वयं”} शब्दप्रतिपादितं वा {\color{DodgerBlue3}“स्वरूपं प्रयुज्यते”} । यथा स्वभावकारणव्यापकानुपलब्ध्यादिषु । निषेध्यस्यार्थस्य {\color{DodgerBlue3}“बाधनं”} विरुद्धं {\color{DodgerBlue3}“वा”} प्रयुज्यते यथा स्वभावकारणव्यापकविरुद्धोपलब्ध्यादिषु प्रयुज्यत इत्याह । अविकलकारणतया शीतस्य {\color{DodgerBlue3}“भावे”} सत्तायां सत्यां दहनस्य {\color{DodgerBlue3}“भावात् । अभावतो”} निवृत्तेः । यथा शीताभावसाधने सहानवस्थानविरुद्धो वह्नेः प्रयुज्यते नात्र शीतस्पर्शो वह्रेरिति । (२७८)
	\pend
      \label{div_pvv.4.279}\edlabel{div_pvv.4.279}
	  
	% new div opening: depth here is 2
	
	  \bigskip
	  \begingroup
	  \large
	
	    
	    \stanza[\smallbreak]
	\label{pv.4.279}\edlabel{pv.4.279}\flagstanza{\tiny\textenglish{....4.279}}अन्योन्यभेदसिद्धेर्वा ध्रुवभावविनाशवत् ।&प्रमाणान्तरबाधाद् वा सापेक्षध्रुवभाववत् ॥ २७९ ॥\&[\smallbreak]


	
	  \endgroup
	

	  \pstart निषेध्यविधीयमानयोर{\color{DodgerBlue3}“न्योन्य”}स्याभावात्मतया {\color{DodgerBlue3}“भेदसिद्धेर्व्वा”}ऽर्थबाधनरूपं प्रयुज्यते । {\color{DodgerBlue3}“ध्रुवभावविनाशवत्”} । नित्यत्वानित्यत्वयोरन्योन्याभावात्मकत्वेन भेदसिद्धौ परस्परपरिहारस्थितिलक्षणतया विरुद्धं विनाशित्वं नित्यत्वबाधने प्रयुज्यते । यथा नित्यः शब्दो विनाशित्वात् । स चायं परस्परपरिहारस्थितिविरुद्धः क्वचित् साक्षाद् वा प्रयुज्यते यथा नित्यत्वस्य विनाशित्वं । क्वचिद् धर्मान्तर- विरोधग्राहिणा {\color{DodgerBlue3}“प्रमाणान्त”}रेण परम्परया {\color{DodgerBlue3}“बा”}धनात् । अबाधितविरुद्धभावो {\color{DodgerBlue3}“वा”} प्रयुज्यते । {\color{DodgerBlue3}“सापेक्षध्रुवभावित्ववत्”} । यथा सापेक्षध्रुवभावयोः साक्षाद् विरो- धाभावेपि ध्रुवभावित्वस्य यद् व्यापकं तेन विरुद्धं सापेक्षत्वं व्याप्यव्यापक- योश्च वस्तुतस्तादात्म्यादयो यद्व्यापके विरुध्यते स तद्व्याप्येनापीति प्रमाणान्तर- बाधनादेवावधृतविरुद्धभावं सापेक्षत्वं ध्रुवभावित्वबाधने प्रयुज्यते (।) यथा न ध्रुवभावी कृतकस्य विनाशो हेत्वन्तरसापेक्षत्वादिति । (२७९)
	\pend
      
	  
	% new div opening: depth here is 2
	

	  \pstart कस्मात् पुनर्हेत्वन्तरसापेक्षो न ध्रुवभावीत्याह ।
	\pend
      \leavevmode\marginnote{\textenglish{511/s}}
	  
	% new div opening: depth here is 1
	
\section[{८. भावस्वभावचिन्ता}]{८. भावस्वभावचिन्ता}

	  \begin{center}%% label @type='head'
	\textbf{(१) हेत्वन्तरसापेक्षो न ध्रुवभावः}
	\end{center}
	\label{div_pvv.4.280}\edlabel{div_pvv.4.280}
	  
	% new div opening: depth here is 2
	
	  \bigskip
	  \begingroup
	  \large
	
	    
	    \stanza[\smallbreak]
	\label{pv.4.280a}\edlabel{pv.4.280a}\flagstanza{\tiny\textenglish{...4.280a}}हेत्वन्तरसमुत्थस्य सन्निधौ नियतः कुतः ।\&[\smallbreak]


	
	  \endgroup
	

	  \pstart उत्पादकाद्धेतो{\color{DodgerBlue3}“र्हेत्वन्तरसमुत्थस्य”} धर्मस्य {\color{DodgerBlue3}“सन्निधौ”} सन्निधाने {\color{DodgerBlue3}“नियतः कुतः”} । यथा कारणान्तरसापेक्षस्य वाससि नावश्यंभावनियमो रागस्य ।
	\pend
      

	  \begin{center}%% label @type='head'
	\textbf{(२) न भावनश्वरस्वभावनियतो भावः}
	\end{center}
	

	  \pstart स्यादेतत् । भावहेतुरेवानित्यत्वाख्यं धर्मं भावनाशकं जनयति तेन नश्वर- स्वभावनियतो भाव इत्याह ।
	\pend
      
	  \bigskip
	  \begingroup
	  \large
	
	    
	    \stanza[\smallbreak]
	\label{pv.4.280b}\edlabel{pv.4.280b}\flagstanza{\tiny\textenglish{...4.280b}}भावहेतुभवत्वे किं पारम्पर्यपरिश्रमैः ॥ २८० ॥\&[\smallbreak]


	
	  \endgroup
	

	  \pstart {\color{DodgerBlue3}“भावहेतुभवत्वे”}ऽनित्यत्वाख्यस्य धर्मस्य भावनाशकस्येष्यमाणे {\color{DodgerBlue3}“पारम्पर्ये परिश्रमै”}रेभिः स्वीकृतैः {\color{DodgerBlue3}“किं”} प्रयोजनं । (२८०)
	\pend
      \label{div_pvv.4.281}\edlabel{div_pvv.4.281}
	  
	% new div opening: depth here is 2
	

	  \pstart तथा  हि (।)
	\pend
      
	  \bigskip
	  \begingroup
	  \large
	
	    
	    \stanza[\smallbreak]
	\label{pv.4.281}\edlabel{pv.4.281}\flagstanza{\tiny\textenglish{....4.281}}नाशनं जनयित्वान्यं स हेतुस्तस्य नाशकः&तमेव नश्वरं भावं जनयेद् यदि किम्भवेत् ॥ २८१ ॥\&[\smallbreak]


	
	  \endgroup
	

	  \pstart {\color{DodgerBlue3}“स”} भाव{\color{DodgerBlue3}“हेतुरन्यम”}नित्यत्वाख्यं धर्मं {\color{DodgerBlue3}“भावनाशनं जनयित्वा तस्या”}भावस्य {\color{DodgerBlue3}“नाशक”} इष्यते परंपरया । {\color{DodgerBlue3}“यदि”} तु {\color{DodgerBlue3}“तमेव”} न{\color{DodgerBlue3}“श्वरं भावं”} साक्षाद् भावहेतु{\color{DodgerBlue3}“र्जनयेत् । तदा- किं”}न्दूषणं {\color{DodgerBlue3}“भवेत्”} । न किञ्चित् । (२८१)
	\pend
      \label{div_pvv.4.282}\edlabel{div_pvv.4.282}
	  
	% new div opening: depth here is 2
	

	  \pstart अपि च (।)
	\pend
      
	  \bigskip
	  \begingroup
	  \large
	
	    
	    \stanza[\smallbreak]
	\label{pv.4.282}\edlabel{pv.4.282}\flagstanza{\tiny\textenglish{....4.282}}आत्मोपकारकः कः स्यात् तस्य सिद्धात्मनः सतः ।&नात्मोपकारकः कः स्यात् तेन यः समपेक्ष्यते ॥ २८२ ॥\&[\smallbreak]


	
	  \endgroup
	

	  \pstart यो यावाननित्यत्वाख्यो धर्म {\color{DodgerBlue3}“आत्मा”} भावस्य स किमु{\color{DodgerBlue3}“पकारकः”} सन् {\color{DodgerBlue3}“विनाशको”} भावस्य उतानुपकारक एव । तत्राद्ये पक्षे {\color{DodgerBlue3}“सिद्धात्मनो”} भावस्य {\color{DodgerBlue3}“सतः”} सर्व्वतो निरा- शंस्यस्य {\color{DodgerBlue3}“क आत्मा”}ऽनित्याख्यो धर्मोऽन्यो वा {\color{DodgerBlue3}“उपकारकः स्यात्”} । अथ द्वितीयः पक्षः (।) तदाऽ{\color{DodgerBlue3}“नात्मा उपकारकः”} स्यात् । {\color{DodgerBlue3}“तेन”} भावेन {\color{DodgerBlue3}“यो”} नाशकत्वेन {\color{DodgerBlue3}“समपेक्ष्येत”} । उपकारक एव ह्यपेक्ष्यते अनुपकारकत्वे का तस्यापेक्षा नाशकता वा ।\edlabel{pvv.511-1}\footnote{\label{pvv.511-1}  १ व्यर्था नाशहेतुकल्पना}(२८२)
	\pend
      \label{div_pvv.4.283}\edlabel{div_pvv.4.283}
	  
	% new div opening: depth here is 2
	\leavevmode\marginnote{\textenglish{512/s}}

	  \begin{center}%% label @type='head'
	\textbf{(३) अनपेक्ष्य एव भावो नश्वरत्वे}
	\end{center}
	

	  \pstart अथ नाशहेत्वयोगात् । अनपेक्ष्य एव भावो नश्वरतायां, तदा (।)
	\pend
      
	  \bigskip
	  \begingroup
	  \large
	
	    
	    \stanza[\smallbreak]
	\label{pv.4.283}\edlabel{pv.4.283}\flagstanza{\tiny\textenglish{....4.283}}अनपेक्षश्च किम्भावोऽतथाभूतः कदाचन ।&यथा न क्षेपभागिष्टः स एवोद्भूतनाशकः ॥ २८३ ॥\&[\smallbreak]


	
	  \endgroup
	

	  \pstart \leavevmode\marginnote{\textenglish{105b/MA}}{\color{DodgerBlue3}“अनपेक्षश्च”} नाशे {\color{DodgerBlue3}“भावः किं”} कस्मात् {\color{DodgerBlue3}“कदाचनातथाभूतो”}  अनश्वरस्वभावः । सर्वदैव नश्वरस्वभावताऽस्य युक्ता । {\color{DodgerBlue3}“यथा”} त्वन्मते {\color{DodgerBlue3}“स एव”} कृतको भाव {\color{DodgerBlue3}“उद्भूत- नाशक”} उन्मुखनाशकानित्यत्वस्वभावो नाशकालेऽ{\color{DodgerBlue3}“क्षेप”}भागचिरविनाशी इष्टः । तथोत्पादानन्तरं नश्वरस्वभावतया विनश्येदिति । (२८३)
	\pend
      \label{div_pvv.4.284}\edlabel{div_pvv.4.284}
	  
	% new div opening: depth here is 2
	
	  \bigskip
	  \begingroup
	  \large
	
	    
	    \stanza[\smallbreak]
	\label{pv.4.284}\edlabel{pv.4.284}\flagstanza{\tiny\textenglish{....4.284}}क्षणमप्यनपेक्षत्वे भावो भावस्य नेति चेत् ।&भावो हि स तथा भूतोऽभावे भावस्तथा कथम् ॥ २८४ ॥\&[\smallbreak]


	
	  \endgroup
	

	  \pstart क्षणक्षयिस्वभावा भावाः स्वहेतोरेव जायन्ते । विनाशं प्रत्यन{\color{DodgerBlue3}“पेक्षत्वे भावस्य”} यथा द्वितीये क्षणे {\color{DodgerBlue3}“भावो ना”}स्ति तथा प्रथमेपि क्षणे न स्यादिति । क्षणमपि भावस्य भावो न स्यादिति चेत् । अयुक्तमेतत् हि यस्मात् {\color{DodgerBlue3}“स भाव”}स्त{\color{DodgerBlue3}“था”} नश्वरस्वभाव इष्यते । यदा तु भाव एव नास्ति तदाऽ{\color{DodgerBlue3}“भावे”} भावस्य {\color{DodgerBlue3}“भावस्तथा”} नश्वरः {\color{DodgerBlue3}“कथ”}मुच्यते (।) ततो लब्धजन्मतो भावस्य क्षणान्तरा- ननुवृत्तेर्नश्वरता ॥ (२८३) ।
	\pend
      

	  \pstart तस्मात् ।
	\pend
      
	  \bigskip
	  \begingroup
	  \large
	
	    
	    \stanza[\smallbreak]
	\label{pv.4.285}\edlabel{pv.4.285}\flagstanza{\tiny\textenglish{....4.285}}येऽपरापेक्षतद्भावास्तद्भावनियता हि ते ।&असम्भवद्विबन्धा च सामग्री कार्यकर्मणि ॥ २८५ ॥\&[\smallbreak]


	
	  \endgroup
	

	  \pstart {\color{DodgerBlue3}“ये”} भावा अनपेक्ष{\color{DodgerBlue3}“तद्भावाः परापेक्षां”} विना सम्भवद्धर्मविशेषसंभवाः । {\color{DodgerBlue3}“ते तस्य भावे”} धर्मस्य {\color{DodgerBlue3}“नियताः”} । कारणसामग्रीव {\color{DodgerBlue3}“असम्भवद्विबन्धा”} विबन्धकारण- रहिता\edlabel{pvv.512-1}\footnote{\label{pvv.512-1}  १ अन्त्या सामग्री} {\color{DodgerBlue3}“कार्यस्य कर्मणि”} क्रियायां नियता\edlabel{pvv.512-2}\footnote{\label{pvv.512-2}  २ दृष्टान्तः} । परनिरपेक्षाश्च भावाः स्वनाश इति । “क्षणिकाः सर्व्वसंस्कारा” इत्यकम्प्यः सौ ग तः सिंहनादः ॥
	\pend
      

	  \pstart न यदिह तन्न न्याय्यं तेनोदितेन च किं फलं । यदिं बहुशस्तस्या वृत्तौ गुणः कथं कस्य कः ।
	\pend
      \leavevmode\marginnote{\textenglish{513/s}}

	  \pstart यदि परमसौ व्याख्येयार्थग्रहस्य विरोधिनी (।) विवृत्तिरचनामात्रे तस्मात् कृतोत्र भयादरः ॥
	\pend
      
	    
	    \stanza[\smallbreak]
	आचार्यश्रीमनोरथनन्दिकृतायां वार्त्तिकवृत्तौ चतुर्थः परिच्छेदः समाप्तः ॥\&[\smallbreak]


	

	  \pstart लिखितेयं पंडि (त) वि भू ति च न्द्रे ण यदत्र पुण्यं तद्भवत्वाचार्योपाध्याय- पूर्व्वङ्गमसकलसत्वराशेरनुत्तरज्ञानफलावाप्तय इति ।
	\pend
      \edlabel{pvv.513-1a}[[1a साध्यहेतुदृष्टान्तोपनयनिगमनानि पञ्च । \begin{english}\par
Placement of note uncertain; marked with a question mark in the edition (see encoding description for details)\end{english}]]\edlabel{pvv.513-1b}[[1b पा ... कस्य इत्यादौ अवयव्येकत्वं हेतुः । तेन सर्वकस्य प्रसंगः साध्यः । विपर्यये सर्वकस्य साध्यस्याभावः साधनं । तेनावयव्येकत्वस्य हेतोरभावः साध्यः । (एवं वक्षमाणेत्यादि) । अवयव्येकरूप(हेतु)त्वे एकस्यावृत्तौ सर्वस्यावृत्तिः (साध्यप्रसंगः) स्यात् । विपर्यये सर्वावृत्ते साध्यस्याभावः साधनं । तेनावयव्येकत्वस्य हेतोरभावः साध्यः । अवयव्येकरूपहेतुत्वे एकस्य रक्तत्वे सर्वस्य (प्रसंगं साध्यं) रक्तत्वं स्यात् । विपर्यये सर्वरक्तत्वस्य साध्यस्याभावः साधनं । तेनावयव्येकत्वस्य हेतोरभावः साध्यः । \begin{english}\par
Placement of note uncertain; marked with a question mark in the edition (see encoding description for details)\end{english}]]
	    
	    \endnumbering% ending numbering from div
	    \endgroup
	    
	  % running endDocumentHook
     \backmatter 
	 \chapter{The TEI Header}
	 \begin{minted}[fontfamily=rmfamily,fontsize=\footnotesize,breaklines=true]{xml}
       <teiHeader xmlns="http://www.tei-c.org/ns/1.0" xml:lang="en">
   <fileDesc>
      <titleStmt>
         <title type="main" subtype="base-text">Pramāṇavārttika</title>
         <title type="sub" subtype="commentary">Pramāṇavārttikavṛtti</title>
         <author role="base-author">Dharmakīrti</author>
         <author role="commentator">Manorathanandin</author>
         <funder>Deutsche Forschungsgemeinschaft</funder>
         <funder>The National Endowment for the Humanities</funder>
         <principal>
					       <persName>Birgit Kellner</persName>
				     </principal>
         <respStmt>
            <resp>data entry by</resp>
            <name key="name_swift">SWIFT Information Technologies, Mumbai</name>
         </respStmt>
         <respStmt>
            <resp>prepared for SARIT by</resp>
            <persName key="name_person_lo">Liudmila Olalde</persName>
         </respStmt>
      </titleStmt>
      <editionStmt>
         <p> </p>
      </editionStmt>
      <publicationStmt>
         <publisher>SARIT: Search and Retrieval of Indic Texts. DFG/NEH Project (NEH-No. HG5004113), 2013-2016 </publisher>
         <idno>2014-04-15</idno>
         <availability status="restricted">
            <p>Copyright Notice:</p>
            <p>Copyright 2014 SARIT</p>
            <licence> 
						         <p>Distributed under a <ref target="https://creativecommons.org/licenses/by-sa/4.0/">Creative Commons Attribution-ShareAlike 4.0 International licence.</ref> Under this licence, you are free to:</p>
						         <list>
                  <item>Share — copy and redistribute the material in any medium or format.</item>
                  <item>Adapt — remix, transform, and build upon the material for any purpose, even commercially.</item>
               </list>
						         <p>The licensor cannot revoke these freedoms as long as you follow the license terms.</p>
						         <p>Under the following terms:</p>
						         <list>
                  <item>Attribution — You must give appropriate credit, provide a link to the license, and indicate if changes were made. You may do so in any reasonable manner, but not in any way that suggests the licensor endorses you or your use.</item>
                  <item>ShareAlike — If you remix, transform, or build upon the material, you must distribute your contributions under the same license as the original.</item>
               </list>
						         <p>More information and fuller details of this license are given on the Creative Commons website.</p>
					       </licence>
            <p>SARIT assumes no responsibility for unauthorised use that infringes the rights of any copyright owners, known or unknown.</p>
         </availability>
         <date>2014</date>
      </publicationStmt>
      <sourceDesc>
         <bibl xml:id="pvv-sankrtyayana-book">
					       <title type="main">
               <persName>Dharmakīrti</persName>'s Pramāṇavārttika</title>
					       <title type="sub">with a commentary by <persName>Manorathanandin</persName>
            </title>
					       <author>Dharmakīrti</author>
					       <author>Manorathanandin</author>
					       <editor key="name_person_rs">Rāhula Sāṅkṛtyāyana</editor>
					       <publisher>Bihar and Orissa Research Society</publisher>
					       <pubPlace>Patna</pubPlace>
					       <date>1938-1940</date>
					       <note>Appendix to the Journal of the Bihar and Orissa Research Society</note>
					       <note>The manuscript consulted by Sāṅkṛtyāyana is described below.</note>
				     </bibl>
         <msDesc>
            <msIdentifier>
               <idno/>
               <altIdentifier>
                  <idno>Manuscript nr. 237 (henceforth MA).</idno>
                  <note>In: Sāṅkṛtyāyana, "Second Search of Sanskrit Palm-leaf Mss. in Tibet". JBORS 23,1 (1937) 1-57.</note>
               </altIdentifier>
            </msIdentifier>
            <msContents>
               <msItem>
                  <author>Manorathanandin</author>
                  <title>Pramāṇavārttikavṛtti</title>
               </msItem>
            </msContents>
            <physDesc>
               <objectDesc>
                  <p>Written by Vibhūticandra in early old Bengali script (Sāṅkṛtyāyana refers to the script as Kuṭilā), the manuscript comprises 105 leaves of seven lines each that according to Sāṅkṛtyāyana measure 67.31 x 5.80 cm.</p>
               </objectDesc>
            </physDesc>
            <history>
               <p>On July 28, 1936, Sāṅkṛtyāyana found this paper manuscript of Manorathanandin’s Pramāṇavārttikavṛtti in the hermitage Zha lu ri phug.</p>
            </history>
         </msDesc>
      </sourceDesc>
   </fileDesc>
   <encodingDesc>
      <p>In the source file, there were two types of line breaks: returns (and possible surrounding space) and hyphens+returns. These were replaced with lb-elements. I didn't check whether the source was consequent in this respect. The ed-attribute "s" refers to Sāṅkṛtyāyana's edition<ref sameAs="#pvv-sankrtyayana-book"/>.</p>
      <p>The folio numbers on the margins were encoded as pb-elements. The ed-attribute "MA" refers to the manuscript used by Sāṅkṛtyāyana. The line numbers in the manuscript were encoded as lb-elements with the ed-attribute "MA".</p>
      <p>The text is structured in three div-levels:<list>
            <item>Four chapters encoded as: div n="..." type="chapter" subtype="pariccheda"</item>
            <item>Subchapters encoded as: div n="[roman numbers]" type="subchapter". The subchapter number is not reflected in the verse numbers.</item>
            <item>The lowest div-level encloses a verse (or a group of verses) and its corresponding commentary, e.g.: div n="3.121 3.122abc" for a div enclosing verses 121 and the first three padas of 122 of the first pariccheda as well their commentary.</item>
         </list>
      </p>
      <p>The notes represent marginal notes on the manuscript MA, written by Vibhūticandra. The editor Rāhula Sāṅkṛtyāyana is responsible for linking them with particular passages in the text. His linkages await further study. In some places his notes are prefixed with question marks, which we interpret as indicating Sāṅkṛtyāyana's uncertainty regarding where they belong. We placed them right after the immediately preceding note in the text, as a convention.</p>
      <p>The verse numbers are those of Sāṅkṛtyāyana's edition<ref sameAs="#pvv-sankrtyayana-book"/>.</p>
      <p>Abbreviations used in the attributes ed, cRef and xml:id's in this file: <!-- this is a provisory list and has to be replaced by a refsDecl -->
				<list>
            <item>Divy = Divyāvadāna; the page number refers to P.L. Vaidya, Divyāvadāna, The Mithila Institute of Post-Graduate Studies and Research in Sanskrit Learning, Darbhanga 1959 (Buddhist Sanskrit Texts, 20)</item>
            <item>MA = Sanskrit manuscript of Manorathanandin's Pramāṇavārttikavṛtti used by Sāṅkṛtyāyana</item>
            <item>ps = Pramāṇasamuccaya; the verse numbers correspond to <ref target="http://www.ikga.oeaw.ac.at/Mat/dignaga_PS_1.pdf">Steinkellner, Dignāga's Pramāṇasamuccaya, Chapter 1</ref>
            </item>
            <item>psv = Pramāṇasamuccayavṛtti; the verse numbers correspond to to <ref target="http://www.ikga.oeaw.ac.at/Mat/dignaga_PS_1.pdf">Steinkellner, Dignāga's Pramāṇasamuccaya, Chapter 1</ref>
            </item>
            <item>pv = Dharmakīrti's Pramāṇavārttika</item>
            <item>pvv = Manorathanandin's Pramāṇavārttikavṛtti</item>
            <item>ns = Gautama's Nyayasutra</item>
            <item>VāPa = Bhartṛhari's Vākyapadīya</item>
            <item>śv = Kumārila's Ślokavārttika</item>
            <item>pvsv = Dharmakīrti's Pramāṇavārttikasvavṛtti</item>
            <item>ts = Tattvasaṅgraha; the verse numbers correspond to Krishnamacharya's edition of Tattvasaṅgrahapañjikā</item>
         </list>
      </p>
   </encodingDesc>
   <profileDesc><!-- ... --></profileDesc>
   <revisionDesc>
      <change who="#lo" when="2014-05-18">I corrected the verse number of verse 2.450.</change>
      <change who="#lo" when="2014-06-04">I corrected the verse number of verse 4.116.</change>
      <change who="#lo" when="2014-06-04">I corrected the folio number 65b, which was wrong in the printed edition.</change>
      <change who="#lo" when="2014-06-30">I changed फलभि to फलमि on p. 214, line 15 from below.</change>
      <change who="#lo" when="2014-07-23">I corrected the verse number (३।१६३) in footnote 345-12. In the printed edition it reads (३।१६५).</change>
      <change who="#lo" when="2014-07-23">In footnote 370-3 I changed प्रवर्त्तत to प्रवर्त्तेत.</change>
      <change who="#lo" when="2015-08-18">Added TOC.</change>
      <change who="#lo" when="2015-12-30">Added @xml:lang to the front-element.</change>
   </revisionDesc>
</teiHeader>
	 \end{minted}
       
      \clearpage
      \begin{english}
      \printshorthands
      \printbibliography
      \end{english}
    
\end{document}
