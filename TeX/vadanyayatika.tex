%% require snapshot package to record versions to log files
    \RequirePackage[log]{snapshot}
    \documentclass[article,12pt,a4paper]{memoir}%
    
      %% useful for debugging
      %% \usepackage{syntonly}%
      %%\syntaxonly%
    
	  \usepackage[normalem]{ulem}
	  \usepackage{eulervm}
	  \usepackage{xltxtra}
  \usepackage{polyglossia}
  \PolyglossiaSetup{sanskrit}{
  hyphenmins={2,3},% default is {1,3}
  }
  \setdefaultlanguage{sanskrit}
  % english etc. should also be available, notes and bib
  \setotherlanguages{english,german,italian,french}
  
	\setotherlanguage[numerals=arabic]{tibetan}
      
  \usepackage{fontspec}
  %% redefine some chars (either changed by parsing, or not commonly in font)
  \catcode`⃥=\active \def⃥{\textbackslash}
  \catcode`‿=\active \def‿{\textunderscore}
  \catcode`❴=\active \def❴{\{}
  \catcode`❵=\active \def❵{\}}
  \catcode`〔=\active \def〔{{[}}% translate 〔OPENING TORTOISE SHELL BRACKET
  \catcode`〕=\active \def〕{{]}}% translate 〕CLOSING TORTOISE SHELL BRACKET
  \catcode` =\active \def {\,}
  \catcode`·=\active \def·{\textbullet}
  %% BREAK PERMITTED HERE: \discretionary{-}{}{}\nobreak\hspace{0pt}
  \catcode`‚=\active \def‚{\-}
  \catcode`ꣵ=\active \defꣵ{%
  म्\textsuperscript{cb}%for candrabindu
  }
  %% show a lot of tolerance
  \tolerance=9000
  \def\textJapanese{\fontspec{Kochi Mincho}}
  \def\textChinese{\fontspec{HAN NOM A}}
  \def\textKorean{\fontspec{Baekmuk Gulim} }
  % make sure English font is there
  \newfontfamily\englishfont[Mapping=tex-text]{TeX Gyre Schola}
    % set up a devanagari font
  \newfontfamily\devanagarifont[Script=Devanagari,Mapping=devanagarinumerals,AutoFakeBold=1.5,AutoFakeSlant=0.3]{Chandas}
	\newfontfamily\rmlatinfont[Mapping=tex-text]{TeX Gyre Pagella}
	\newfontfamily\tibetanfont[Script=Tibetan,Scale=1.2]{Tibetan Machine Uni}
  \newcommand\bo\tibetanfont
  
    \defaultfontfeatures{Scale=MatchLowercase,Mapping=tex-text}
	\setmainfont{Chandas}
    \setsansfont{TeX Gyre Bonum}
  
  \setmonofont{DejaVu Sans Mono}
	  %% page layout start: fit to a4 and US letterpaper (example in memoir.pdf)
	  %% page layout start
	  % stocksize (actual size of paper in the printer) is a4 as per class
	  % options;
	  
	  % trimming, i.e., which part should be cut out of the stock (this also
	  % sets \paperheight and \paperwidth):
	  % \settrimmedsize{0.9\stockheight}{0.9\stockwidth}{*}
	  % \settrimmedsize{225mm}{150mm}{*}
	  % % say where you want to trim
	  \setlength{\trimtop}{\stockheight}    % \trimtop = \stockheight
	  \addtolength{\trimtop}{-\paperheight} %           - \paperheight
	  \setlength{\trimedge}{\stockwidth}    % \trimedge = \stockwidth
	  \addtolength{\trimedge}{-\paperwidth} %           - \paperwidth
	  % % this makes trims equal on top and bottom (which means you must cut
	  % % twice). if in doubt, cut on top, so that dust won't settle when book
	  % % is in shelf
	  \settrims{0.5\trimtop}{0.5\trimedge}

	  % figure out which font you're using
	  \setxlvchars
	  \setlxvchars
	  % \typeout{LENGTH: lxvchars: \the\lxvchars}
	  % \typeout{LENGTH: xlvchars: \the\xlvchars}

	  % set the size of the text block next:
	  % this sets \textheight and \textwidth (not the whole page including
	  % headers and footers)
	  \settypeblocksize{230mm}{130mm}{*}

	  % left and right margins:
	  % this way spine and edge margins are the same
	  % \setlrmargins{*}{*}{*}
	  \setlrmargins{*}{*}{1.5}

	  % upper and lower, same logic as before
	  % \setulmargins{*}{*}{*}% upper = lower margin
	  % \uppermargin = \topmargin + \headheight + \headsep
	  %\setulmargins{*}{*}{1.5}% 1.5*upper = lower margin
	  \setulmargins{*}{*}{1.5}% 

	  % header and footer spacings
	  \setheadfoot{2\baselineskip}{2\baselineskip}

	  % \setheaderspaces{ headdrop }{ headsep }{ ratio }
	  \setheaderspaces{*}{*}{1.5}

	  % see memman p. 51 for this solution to widows/orphans 
	  \setlength{\topskip}{1.6\topskip}
	  % fix up layout
	  \checkandfixthelayout
	  %% page layout end
	
	  \sloppybottom
	
	    % numbering depth
	    \maxtocdepth{section}
	    % set up layout of toc
	    \setpnumwidth{4em}
	    \setrmarg{5em}
	    \setsecnumdepth{all}
	    \newenvironment{docImprint}{\vskip 6pt}{\ifvmode\par\fi }
	    \newenvironment{docDate}{}{\ifvmode\par\fi }
	    \newenvironment{docAuthor}{\ifvmode\vskip4pt\fontsize{16pt}{18pt}\selectfont\fi\itshape}{\ifvmode\par\fi }
	    % \newenvironment{docTitle}{\vskip6pt\bfseries\fontsize{18pt}{22pt}\selectfont}{\par }
	    \newcommand{\docTitle}[1]{#1}
	    \newenvironment{titlePart}{ }{ }
	    \newenvironment{byline}{\vskip6pt\itshape\fontsize{16pt}{18pt}\selectfont}{\par }
	    % setup title page; see CTAN /info/latex-samples/TitlePages/, and memoir
	  \newcommand*{\plogo}{\fbox{$\mathcal{SARIT}$}}
	  \newcommand*{\makeCustomTitle}{\begin{english}\begingroup% from example titleTH, T&H Typography
	  \thispagestyle{empty}
	  \raggedleft
	  \vspace*{\baselineskip}
	  
	      % author(s)
	    {\Large Śāntarakṣita}\\[0.167\textheight]
	    % maintitle
	    {\Huge Vādanyāyaṭīkā Vipañcitārthā}\\[\baselineskip]
	    {\Large SARIT}\\\vspace*{\baselineskip}\plogo\par
	  \vspace*{3\baselineskip}
	  \endgroup
	  \end{english}}
	  \newcommand{\gap}[1]{}
	  \newcommand{\corr}[1]{($^{x}$#1)}
	  \newcommand{\sic}[1]{($^{!}$#1)}
	  \newcommand{\reg}[1]{#1}
	  \newcommand{\orig}[1]{#1}
	  \newcommand{\abbr}[1]{#1}
	  \newcommand{\expan}[1]{#1}
	  \newcommand{\unclear}[1]{($^{?}$#1)}
	  \newcommand{\add}[1]{($^{+}$#1)}
	  \newcommand{\deletion}[1]{($^{-}$#1)}
	  \newcommand{\quotelemma}[1]{\textcolor{cyan}{#1}}
	  \newcommand{\name}[1]{#1}
	  \newcommand{\persName}[1]{#1}
	  \newcommand{\placeName}[1]{#1}
	  % running latexPackages template
     \usepackage[x11names]{xcolor}
     \definecolor{shadecolor}{gray}{0.95}
     \usepackage{longtable}
     \usepackage{ctable}
     \usepackage{rotating}
     \usepackage{lscape}
     \usepackage{ragged2e}
     
	 \usepackage{titling}
	 \usepackage{marginnote}
	 \renewcommand*{\marginfont}{\color{black}\rmlatinfont\scriptsize}
	 \setlength\marginparwidth{.75in}
	 \usepackage{graphicx}
	 \graphicspath{{images/}}
	 \usepackage{csquotes}
       
	 \def\Gin@extensions{.pdf,.png,.jpg,.mps,.tif}
       
      \usepackage[noend,series={A,B}]{reledmac}
       % simplify what ledmac does with fonts, because it breaks. From the documentation of ledmac:
       % The notes are actually given seven parameters: the page, line, and sub-line num-
       % ber for the start of the lemma; the same three numbers for the end of the lemma;
       % and the font specifier for the lemma. 
       \makeatletter
       \def\select@lemmafont#1|#2|#3|#4|#5|#6|#7|%
       {}
       \makeatother
       \AtEveryPstart{\refstepcounter{parCount}}
       \setlength{\stanzaindentbase}{20pt}
     \setstanzaindents{3,2,2,2,2,2,2,2,2,}
     % \setstanzapenalties{1,5000,10500}
     \lineation{page}
     % \linenummargin{inner}
     \linenumberstyle{arabic}
     \firstlinenum{5}
    \linenumincrement{5}
    \renewcommand*{\numlabfont}{\normalfont\scriptsize\color{black}}
    \addtolength{\skip\Afootins}{1.5mm}
    \Xnotenumfont{\bfseries\footnotesize}
    \sidenotemargin{outer}
    \linenummargin{inner}
    \Xarrangement{twocol}
    \arrangementX{twocol}
    %% biblatex stuff start
	 \usepackage[backend=biber,%
	 citestyle=authoryear,%
	 bibstyle=authoryear,%
	 language=english,%
	 sortlocale=en_US,%
	 ]{biblatex}
	 
		 \addbibresource[location=remote]{https://raw.githubusercontent.com/paddymcall/Stylesheets/HEAD/profiles/sarit/latex/bib/sarit.bib}
	 \renewcommand*{\citesetup}{%
	 \rmlatinfont
	 \biburlsetup
	 \frenchspacing}
	 \renewcommand{\bibfont}{\rmlatinfont}
	 \DeclareFieldFormat{postnote}{:#1}
	 \renewcommand{\postnotedelim}{}
	 %% biblatex stuff end
	 
	 \setcounter{errorcontextlines}{400}
       
	 \usepackage{lscape}
	 \usepackage{minted}
       
	   % pagestyles
	   \pagestyle{ruled}
	   \makeoddhead{ruled}{{Vādanyāyaṭīkā Vipañcitārthā}}{}{          Śāntarakṣita}
	   \makeoddfoot{ruled}{{\tiny\rmlatinfont \textit{Compiled: \today}}}{%
	   {\tiny\rmlatinfont \textit{Revision: \href{https://github.com/paddymcall/SARIT-pdf-conversions/commit/a0c8ae0}{a0c8ae0}}}%
	   }{\rmlatinfont\thepage}
	   \makeevenfoot{ruled}{\rmlatinfont\thepage}{%
	   {\tiny\rmlatinfont \textit{Revision: \href{https://github.com/paddymcall/SARIT-pdf-conversions/commit/a0c8ae0}{a0c8ae0}}}%
	   }{{\tiny\rmlatinfont \textit{Compiled: \today}}}
	   
	 
	   \usepackage{perpage}
           \MakePerPage{footnote}
	 
       \usepackage[destlabel=true,% use labels as destination names; ; see dvipdfmx.cfg, option 0x0010, if using xelatex
       pdftitle={Vādanyāyaṭīkā Vipañcitārthā // Śāntarakṣita},
       pdfauthor={SARIT: Search and Retrieval of Indic Texts. DFG/NEH Project (NEH-No. HG5004113), 2013-2016 },
       unicode=true]{hyperref}
       
       \renewcommand\UrlFont{\rmlatinfont}
       \newcounter{parCount}
       \setcounter{parCount}{0}
       % cleveref should come last; note: also consider zref, this could become more useful than cleveref?
       \usepackage[english]{cleveref}% clashes with eledmac < 1.10.1 standard
       \crefname{parCount}{§}{§§}
     
\begin{document}
    
     \makeCustomTitle
     \let\tabcellsep&
	\frontmatter
	\tableofcontents
	% \listoffigures
	% \listoftables
	\cleardoublepage
         \mainmatter 
	  
	% new div opening: depth here is 0
	
	    
	    \beginnumbering% beginning numbering from div depth=0
	    
	  
\chapter[{१. निग्र‚ह‚स्थान‚ल‚क्ष‚ण‚म्}][{१. निग्र‚ह‚स्थान‚ल‚क्ष‚ण‚म्}]{१. निग्र‚ह‚स्थान‚ल‚क्ष‚ण‚म्}\textsuperscript{\textenglish{1/s}}

	  
	  \pstart \leavevmode% starting standard par
	न‚मो विघ्घ्न‚प्र‚म‚थ‚नाय ॥ \leavevmode\ledsidenote{\textenglish{1b/msK}}
	{\color{gray}{\rmlatinfont\textsuperscript{§~\theparCount}}}
	\pend% ending standard par
      ‚{\tiny $_{lb}$}‚
	  \bigskip
	  \begingroup
	
	    
	    \stanza[\smallbreak]
	  \flagstanza{\tiny\textenglish{...1}}{\normalfontlatin\large ``\qquad}नानास‚द्गुण‚र‚त्न‚राशिकिर‚ण‚ध्व‚स्तान्ध‚कार‚स्स‚दा [,]&‚{\tiny $_{lb}$}‚यो नानाविध‚स‚त्त्व‚वांछित‚फ‚ल‚प्राप्त्य‚र्थ‚म‚त्त्युद्य‚तः ।&‚{\tiny $_{lb}$}‚त‚न्निःशेष‚ज‚ग‚द्धितोद‚य‚प‚र‚न्न‚त्वार्य‚म‚ञ्जुश्रियं । &‚{\tiny $_{lb}$}‚वाद‚न्याय‚विभाग एष विम‚लः स‚ङ्क्षिप्त आर‚भ्य‚ते ॥ [१]{\normalfontlatin\large\qquad{}"}\&[\smallbreak]
	  
	  
	  
	  \endgroup
	‚{\tiny $_{lb}$}‚

	  
	  \pstart \leavevmode% starting standard par
	य‚त्प्र‚योज‚न‚र‚हितं त‚त्प्रेक्षापूर्व्व‚कारिभिर्न्नार‚भ्य‚ते । य‚था ब‚लित्व‚ग्द‚र्श‚न‚{\tiny $_{lb}$}‚विनिश्च‚यादिकं । अप्र‚योज‚न‚ञ्चेदं प्र‚क‚र‚ण ‚{\tiny $_{2}$}‚ मित्याशंकाव‚त‚स्त‚दाशंकाप‚रिजिहीर्ष‚या ‚{\tiny $_{lb}$}‚प्र‚योज‚न‚प्र‚द‚र्श‚नाय \quotelemma{न्याय‚वादिन} \cite[1b1]{vn-msN} मित्यादिवाक्य‚मुप‚न्य‚स्त‚वान् । क‚थ‚म्पु‚{\tiny $_{lb}$}‚न‚र‚नेन वाक्येनास्य प्र‚योज‚न‚मुप‚द‚र्श्य‚त इत्यास्तां ‚{\tiny $_{3}$}‚ ताव‚देत‚द् । अर्थ‚स्तु ‚{\tiny $_{lb}$}‚व्याख्याय‚ते ॥ न्याय‚स्त्रिरूप‚लिङ्ग‚ल‚क्ष‚णा युक्तिः । नीय‚ते प्राप्य‚ते विव‚क्षितार्थ‚{\tiny $_{lb}$}‚सिद्धिर‚नेनेति कृत्त्वा अत एव \quotelemma{त्रिविधं लिङ्ग‚मि}\cite[1b2]{vn-msN} त्यादिना त्रि ‚{\tiny $_{4}$}‚ रूप‚मेव ‚{\tiny $_{lb}$}‚लिङ्ग‚म‚न‚न्त‚र‚म्व‚क्ष्य‚ति । त‚द‚भिधायि व‚च‚न‚मित्य‚प‚रे । त‚म्व‚दितुं शीलं य‚स्य स ‚{\tiny $_{lb}$}‚त‚थोक्तः । त‚म‚पि निगृह्ण‚न्ति प‚राज‚य‚न्त इत्य‚र्थः । निगृह्ण‚न्तु नामान्याय ‚{\tiny $_{5}$}‚ ‚{\tiny $_{lb}$}‚वादिन‚म्प‚राज‚याधिक‚र‚ण‚त्त्वादेव । न्याय‚वादिन‚न्त्व‚निग्र‚हार्ह‚म‚पि य‚न्निगृह्ण‚न्त्ये‚{\tiny $_{lb}$}‚त‚न्न स‚म्भाव्य‚ते । प‚रोत्क‚र्ष‚व्यारोप‚धिय‚स्तु त‚म‚पि प‚राज‚य‚न्त इति स‚म्भाव‚ना ‚{\tiny $_{6}$}‚ ‚{\tiny $_{lb}$}‚याम‚पि श‚ब्दः स‚मुच्च‚यार्थोऽतिश‚य‚द्योत‚नार्थो वा केषु [।] निगृह्ण‚न्तीत्याह । ‚{\tiny $_{lb}$}‚ \quotelemma{वादेषु} \cite[1b1]{vn-msN} साध‚न‚दूष‚ण‚संश‚ब्दितेषु विचारेष्विति याव‚त् । \quotelemma{विवादेष्विति} ‚{\tiny $_{lb}$}‚क्व‚चित्पाठः । ‚{\tiny $_{7}$}‚ त‚त्र विरुद्धा वादा \quotelemma{विवादा} स्तेष्विति व्याख्येयं । ‚{\tiny $_{lb}$}‚विरुद्वाश्च क‚थं [।] साध‚न‚दूष‚ण‚संश‚ब्दितानाम्विचाराणान्त‚द्विरुद्धार्थ‚साध‚न‚{\tiny $_{lb}$}‚प्र‚वृत्त‚त्त्वात् ॥ क‚थ‚म्पुन‚र्न्याय‚वादिन‚म‚स‚त्स्व‚सि ‚{\tiny $_{8}$}‚ द्ध्यादिषु हेतुदोषेषु निगृह्ण‚न्ती ‚{\tiny $_{lb}$}‚त्याह ॥ \quotelemma{अस‚द्व्य‚य‚व‚स्थोप‚न्यासैः} । \cite[1b1]{vn-msN} अस‚ताम‚साधूनाम्व्य‚व‚स्थाः । अस‚त्यो ‚{\tiny $_{lb}$}‚ \leavevmode\ledsidenote{\textenglish{2/s}} वा व्य‚व‚स्थाः स‚द्भिः कुत्सित‚त्त्वात् । ताश्च फ‚ल‚जात्य‚स‚न्निग्र‚ह‚स्थान‚{\tiny $_{lb}$}‚ल‚क्ष ‚{\tiny $_{9}$}‚\leavevmode\ledsidenote{\textenglish{2a/msK}} णास्तासामुप‚न्यासाः । प्र‚योग‚स्तैरिति विग्र‚हः । के पुन‚र‚ह्रीकास्त एवं ‚{\tiny $_{lb}$}‚विधा इत्याह ॥ श‚ठा धूर्ता मायाविनः प‚र‚स‚म्प‚त्तावीर्ष्याल‚व इति याव‚त् । ‚{\tiny $_{lb}$}‚य‚स्मात्तं त‚था निगृह्ण‚न्ति ‚{\tiny $_{1}$}‚ इति त‚स्मात्त‚न्निषेधार्थं तेषां श‚ठानान्तेषाम्वा ‚{\tiny $_{lb}$}‚ऽस‚द्व्य‚व‚स्थोप‚न्यासानान्त‚स्य \quotelemma{वा} निग्र‚ह‚स्य त्र‚याणां \quotelemma{वा} निषेधो निरास‚स्त‚द‚र्थ‚न्त‚न्नि‚{\tiny $_{lb}$}‚मित्त‚मिद‚म्प्र‚क‚र‚ण‚मार‚भ्य‚ते ॥ \quotelemma{स ए} वार्थोऽस्ये ‚{\tiny $_{2}$}‚ ति विग्र‚हीत‚व्यं । त‚न्निषेधे च ‚{\tiny $_{lb}$}‚कृते स‚म्य‚ग्विचारः प्र‚व‚र्त‚ते त‚त्पूर्व‚क‚श्च स‚र्वः पुरुषार्थ इत्य‚भिप्रायः । आस‚न्न‚{\tiny $_{lb}$}‚विष‚यिणा त्व‚न्त‚र्विप‚रिव‚र्त्ति प्र‚क‚र‚ण‚मिद‚माप‚रा ‚{\tiny $_{3}$}‚ मृष‚ति । अन्त‚स्त‚न्त्वात्म‚ना ‚{\tiny $_{lb}$}‚प‚रिनिष्प‚न्न‚त्वात् । अन्य‚थाऽप‚रिनिष्प‚न्नात्म‚त‚याऽस‚न्न‚त्वाभावादिद‚म् श‚ब्द‚{\tiny $_{lb}$}‚प्र‚योगो न स्यात् । \quotelemma{आर‚भ्य‚त} \cite[1b1]{vn-msN} इति व‚र्त‚मान‚काल‚निर्देशः [।] \quotelemma{क ‚{\tiny $_{4}$}‚} ‚{\tiny $_{lb}$}‚थ‚मिति चेत् । व‚र्त‚मान‚सामीप्ये व‚र्त‚मान‚व‚द्वे \href{http://sarit.indology.info/?cref=P\%C4\%81.3.3.31}{पाणिनिः ३।३।३१} ति व‚च‚नात् ‚{\tiny $_{lb}$}‚स‚म्ब‚न्धोऽप्य‚भिधानीय एवान्य‚था बालोन्म‚त्त‚प्र‚लाप‚व‚द‚ग्राह्य‚मिदं प्रेक्षापूर्व‚कारिणा‚{\tiny $_{lb}$}‚म्भ‚वेदिति चेत् ‚{\tiny $_{5}$}‚ स‚त्त्य‚मेत‚त् । प्र‚योज‚नान्त‚र्ग‚त‚त्त्वात् पृथ \quotelemma{ग} सौ नाभिहितः । त‚थाहि ‚{\tiny $_{lb}$}‚ \quotelemma{त‚न्निषेधार्थ‚मिद‚मार‚भ्य‚ते} \cite[1b1]{vn-msN} त‚त‚श्चैत‚त्प्र‚योज‚न‚म‚नेन प्र‚क‚र‚णेन साध्य‚ते । ‚{\tiny $_{lb}$}‚त‚था च प्र‚योज‚न‚प्र‚क‚र‚ण ‚{\tiny $_{6}$}‚ योः साध्य‚साध‚न‚ल‚क्ष‚णः स‚म्ब‚न्ध इति सूचितं । ये त्व‚न्ये ‚{\tiny $_{lb}$}‚क्रियान‚न्त‚र्यादिल‚क्ष‚णाः स‚म्ब‚न्धास्ते न वाच्या एव प्र‚क‚र‚ण‚क्रियायाम‚न‚ङ्ग‚भू‚{\tiny $_{lb}$}‚त‚त्त्वात् । त‚थाहि तेषु स‚त्स्व‚पि प्र‚यो ‚{\tiny $_{7}$}‚ ज‚नाभावे नार‚भ्य‚त एव प्र‚क‚र‚णं । ‚{\tiny $_{lb}$}‚अस‚त्स्व‚पि च तेषु स‚ति प्र‚योज‚ने प्रार‚भ्य‚त एव । त‚स्मात्प्र‚क‚र‚णार‚म्भ‚स्य ‚{\tiny $_{lb}$}‚प्र‚योज‚नान्व‚य‚व्य‚तिरेकानुविधानात्प्र‚योज‚न‚मेवाभिधानी ‚{\tiny $_{8}$}‚ य‚म्प्रेक्षापूर्व‚कारिणा । त‚स्मिं‚{\tiny $_{lb}$}‚श्चाभिहिते स‚म्ब‚न्धोप्युक्त एव भ‚व‚तीति म‚न्य‚ते । नान‚व‚धारित‚प्र‚क‚र‚ण ‚{\tiny $_{lb}$}‚श‚रीराः प्र‚व‚र्त‚न्ते प्रेक्षाव‚न्त‚स्त‚स्मात्प्र‚योज‚न‚व‚त्प्र‚वृत्त्य‚ङ्ग‚त्वात्प्र‚क‚र‚ण ‚{\tiny $_{9}$}‚\leavevmode\ledsidenote{\textenglish{2b/msK}} श‚रीर‚म‚पि ‚{\tiny $_{lb}$}‚व‚क्त्व्य‚मेवेति चेत् । एव‚मेत‚त्प्र‚योज‚न‚वाक्येन त्व‚भिहित‚त्वान्नैव त‚द‚पि स‚म्ब‚न्ध‚व‚{\tiny $_{lb}$}‚त्पृथ‚ग‚भिधान‚म‚र्ह‚ति । य‚त‚स्त‚न्निषेधार्थ‚मिद‚मार‚भ्य‚त इत्युक्त‚म‚त‚श्च त‚त् ‚{\tiny $_{1}$}‚ निषे ‚{\tiny $_{lb}$}‚धोऽस्य श‚रीर‚मित्युक्त‚म्भ‚व‚ति । य‚दा चैत‚न्निषेधार्थ‚मिद‚मार‚भ्य‚ते त‚दा पूर्व‚को हेतु‚{\tiny $_{lb}$}‚र‚सिद्धः । प्र‚तिप्र‚माण‚द्व‚य‚ञ्चानेन सूचितं प्रार‚ब्ध‚व्य‚मिद‚म्प्र‚क‚र‚ण‚म्प्रेक्षाव‚ता ‚{\tiny $_{2}$}‚ स‚ति ‚{\tiny $_{lb}$}‚साम‚र्थ्ये । ग्राह्य‚म्वा प्र‚योज‚न‚व‚त्वात् । स‚म्ब‚न्ध‚व‚त्वाच्च त‚द‚न्य‚शास्त्र‚व‚दिति ‚{\tiny $_{lb}$}‚स्व‚भाव‚हेतुः ।
	{\color{gray}{\rmlatinfont\textsuperscript{§~\theparCount}}}
	\pend% ending standard par
      ‚{\tiny $_{lb}$}‚\textsuperscript{\textenglish{3/s}}

	  
	  \pstart \leavevmode% starting standard par
	एव‚म‚भिधाय प्र‚योज‚नं स‚क‚ल‚प्र‚क‚र‚णार्थ‚संग्राह‚कं श्लोक‚माह ‚{\tiny $_{3}$}‚ \quotelemma{असाध‚नाङ्व‚च‚न} ‚{\tiny $_{lb}$}‚ \cite[1b1]{vn-msN} मित्यादिना । असाध‚नाङ्ग‚व‚च‚न‚म‚दोषोद्भाव‚न‚ञ्च द्व‚योर्वादिप्र‚तिवादि ‚{\tiny $_{lb}$}‚नोर्य‚थाक्र‚मं निग्र‚ह‚स्थानं प‚राज‚याधिक‚र‚णं । अन्य‚त्त्वित्येत ‚{\tiny $_{4}$}‚ द्धेय‚व्य‚तिरिक्त‚म‚{\tiny $_{lb}$}‚ \quotelemma{क्ष‚पाद‚प} रिक‚ल्पितं प्र‚तिज्ञासंन्यासादिकं व‚क्ष्य‚माणं निग्र‚ह‚स्थानं न \quotelemma{युक्त‚मिति} ‚{\tiny $_{lb}$}‚कृत्वा \quotelemma{नेष्य‚ते} [।] निग्र‚ह‚स्थान‚मिति व‚र्त्त‚ते । अयं ताव‚त् स ‚{\tiny $_{5}$}‚ मासेन श्लोकार्थः । ‚{\tiny $_{lb}$}‚ \quotelemma{इष्ट‚स्ये} \cite[1b1]{vn-msN} त्यादिना विभाग‚मार‚भ‚ते । इष्टोऽर्थोऽनित्त्यः श‚ब्द इत्यादि ‚{\tiny $_{lb}$}‚साध्य‚त्त्वेनेप्सितः । त‚स्य \quotelemma{सिद्धिः} प्र‚तिप‚त्तिः साध‚नं । त‚द‚नेन भाव‚स्य ‚{\tiny $_{6}$}‚ ‚{\tiny $_{lb}$}‚साध‚नोयं साध‚न‚श‚ब्द‚स्ताव‚द‚स्मिन्व्याख्यानेऽभिप्रेतो न तु क‚र‚ण‚साध‚न इति ‚{\tiny $_{lb}$}‚द‚र्श‚य‚ति ॥ त‚स्य साध‚न‚स्येष्टार्थ‚सिद्धिल‚क्ष‚ण‚म‚स्याङ्गं किन्त‚दित्याह । \quotelemma{निर्व‚र्त्त‚कं} ‚{\tiny $_{lb}$}‚ज‚न‚कं । अने ‚{\tiny $_{7}$}‚ नाङ्ग‚श‚ब्दं व्याच‚ष्टे । कार‚ण‚प‚र्यायोय‚म‚त्राङ्ग‚श‚ब्दो नाव‚य‚व‚प‚र्याय ‚{\tiny $_{lb}$}‚इत्य‚र्थः । त‚च्च साध‚नाङ्ग‚मिह निश्चित‚त्रैरूप्यं लिङ्ग‚मुच्य‚ते । अस्य साध‚नाङ्ग‚स्य ‚{\tiny $_{lb}$}‚व‚च‚नं त्रिरूप‚लिङ्गा ‚{\tiny $_{8}$}‚ ख्यानं । त‚स्य \quotelemma{साध‚नाङ्ग‚स्याव‚च‚न‚म‚नुच्चार‚ण‚म} \cite[1b2]{vn-msN} ‚{\tiny $_{lb}$}‚न‚भिधानं य‚त्त‚द‚साध‚नाङ्ग‚व‚च‚नं । अनेनैत‚त्क‚थ‚य‚ति असाध‚नाङ्ग‚स्य प‚क्षोप‚न‚य‚ना‚{\tiny $_{lb}$}‚देर्व‚च‚न‚म‚साध‚नाङ्ग‚व‚च‚न‚मिति नै ‚{\tiny $_{9}$}‚\leavevmode\ledsidenote{\textenglish{3a/msK}} व प्र‚तिप‚त्त‚व्य‚म‚स्मिन्व्याख्याने । किन्तु साध‚नाङ्ग‚{\tiny $_{lb}$}‚स्यैवाव‚च‚न‚म‚साध‚नाङ्ग‚व‚च‚न‚मिति । त‚द‚साध‚नाङ्ग‚व‚च‚नं वादिनो निग्र‚ह‚स्थानं । ‚{\tiny $_{lb}$}‚त‚देतेन श्लोक‚स्य पूर्व्व‚भाग‚म्विवृणोति । क‚थ‚म्पुनः साध‚नाङ्ग‚स्यानुच्चार‚ण‚म्भ‚व‚ति ‚{\tiny $_{lb}$}‚ \quotelemma{निग्र‚ह‚स्थानं} चेत्याह । \quotelemma{त‚द‚भ्युप‚ग‚म्येति} \cite[1b2]{vn-msN} त‚दितीष्टं साध्य‚म‚भ्युप‚ग‚म्याह‚{\tiny $_{lb}$}‚मेत‚त्साध‚यामीति प्र‚तिज्ञ‚याप्र‚तिभ‚या क‚र‚ण ‚{\tiny $_{2}$}‚ भूत‚या तूष्णीम्भावात् । अप्र‚तिभाऽत्र ‚{\tiny $_{lb}$}‚पूर्व्वाधिग‚तार्थ‚विस्म‚र‚णं स्त‚म्भित‚त्त्व‚ञ्च गृह्य‚ते ।
	{\color{gray}{\rmlatinfont\textsuperscript{§~\theparCount}}}
	\pend% ending standard par
      ‚{\tiny $_{lb}$}‚

	  
	  \pstart \leavevmode% starting standard par
	अनेन स‚र्व्व‚था साध‚नाङ्ग‚स्याव‚च‚न‚माह अभिधाने वा य‚दि न स‚म‚र्थितं त‚दोक्त‚{\tiny $_{lb}$}‚म‚प्यऽ ‚{\tiny $_{3}$}‚ नूक्त‚मेव स्व‚कार्याक‚र‚णात् । इत्य‚भिप्राय‚वानाह [—] \quotelemma{साध‚नाङ्ग‚स्या‚{\tiny $_{lb}$}‚स‚म‚र्थ‚नाद्वेति} \cite[1b2]{vn-msN} । त‚द‚भ्युप‚ग‚म्येतिव‚र्त‚ते । वा श‚ब्दः पुर्व्वापेक्ष‚या ‚{\tiny $_{lb}$}‚विक‚ल्पार्थः । साध‚ना ‚{\tiny $_{4}$}‚ ङ्ग‚स्यास‚म‚र्थ‚नं त्रिष्व‚पि रूपेषु निश्च‚याप्र‚द‚र्श‚नं । त‚स्मात्तू‚{\tiny $_{lb}$}‚ष्णीम्भावाद‚स‚म‚र्थ‚नाच्च साध‚नाङ्ग‚स्यानुच्चार‚णं । त‚त‚श्च प्र‚तिज्ञातार्थाकार‚णात् ‚{\tiny $_{lb}$}‚वादिनो ‚{\tiny $_{5}$}‚ \quotelemma{निग्र‚हाधिक‚र‚ण} मिति \cite[1b2]{vn-msN} प्र‚कृतेन स‚म्ब‚न्धः । क्व‚चित्तु \quotelemma{वादिन} ‚{\tiny $_{lb}$}‚इति पाठः । त‚त्र त‚च्छ‚ब्देन प्र‚कृत‚म‚नुच्चार‚णं संब‚ध्य‚ते ।
	{\color{gray}{\rmlatinfont\textsuperscript{§~\theparCount}}}
	\pend% ending standard par
      ‚{\tiny $_{lb}$}‚

	  
	  \pstart \leavevmode% starting standard par
	क‚थ‚म्पुनः साध‚नाङ्गास‚म‚र्थ‚न‚म्भ ‚{\tiny $_{6}$}‚ व‚ति । येन त‚द्विप‚र्य‚येनास‚म‚र्थ‚नात्प्र‚ति‚{\tiny $_{lb}$}‚ज्ञातार्थाकार‚णाद्वादिनो निग्र‚हाधिक‚र‚ण‚त्त्व‚मिति क‚दाचित्क‚श्चिद् ब्रूयादित्येत‚त्प‚रि‚{\tiny $_{lb}$}‚जिहीर्षुरादिप्र‚स्थान‚मार‚च‚य‚ति ‚{\tiny $_{7}$}‚ \quotelemma{त्रिविध‚मेवे} \cite[1b2]{vn-msN} त्यादिना त्रिप्र‚कार‚मेव लिङ्ग‚{\tiny $_{lb}$}‚ \leavevmode\ledsidenote{\textenglish{4/s}} म‚ङ्ग‚ङ्क‚र‚णं क‚स्यासिद्धेः प्र‚तिप‚त्तिरूपायाः । क‚स्य सिद्धेरित्याह । \quotelemma{अप्र‚त्य‚{\tiny $_{lb}$}‚क्ष‚स्या} प‚रोक्ष‚स्यानुमेय‚भूत‚स्य व‚स्तुन इति ‚{\tiny $_{8}$}‚ याव‚त् । अव‚धार‚ण‚ञ्च‚तुरादि ‚{\tiny $_{lb}$}‚व्य‚व‚च्छेद‚माच‚ष्टे । \quotelemma{त्रिविध‚मेवेति} निय‚मः क‚थ‚म‚य‚मिति चेत् । य‚स्माद्विधि ‚{\tiny $_{lb}$}‚प्र‚तिषेध‚रूप‚त‚या द्विधा साध्यं व्य‚व‚स्थितं । विधिरूप‚ञ्च स‚द्भा ‚{\tiny $_{9}$}‚ \leavevmode\ledsidenote{\textenglish{3b/msK}} व‚रूप‚ङ्कार‚ण‚रू ‚{\tiny $_{lb}$}‚प‚म्वा भ‚व‚द् भ‚वेत् । नान्य‚त् । त‚त्र हेतोः प्र‚तिब‚न्धायोगात् । न‚ह्य‚र्थान्त‚र‚स्याका‚{\tiny $_{lb}$}‚र्य‚स्य स‚द्भावेऽप‚र‚स्य स‚द्भावो युक्तः । प‚ट‚स‚द्भाव इवोष्ट्र‚स्य । नाप्य‚र्थां ‚{\tiny $_{1}$}‚ ‚{\tiny $_{lb}$}‚त‚र‚स्याकार‚ण‚स्य निवृत्ताव‚कार्य‚स्यान्य‚स्य च निवृत्तिर्युक्तिम‚ती । उष्ट्रे निवृत्ताविव ‚{\tiny $_{lb}$}‚प‚ट‚स्य । न चान्व‚य‚व्य‚तिरेक‚विक‚ल‚स्याग‚म‚क‚त्त्वं युक्तं । प‚ट‚स्याप्युष्ट्र‚ग‚म‚क‚त्त्व ‚{\tiny $_{2}$}‚ ‚{\tiny $_{lb}$}‚प्र‚स‚ङ्गात् । स्व‚भाव‚भूत‚ध‚र्म‚स‚द्भावे तु स्व‚भाव‚भूत‚स्यान्य‚स्य स‚द्भावो युक्तो न‚हि ‚{\tiny $_{lb}$}‚स्व‚भावः स्व‚म्भाव‚म्प‚रित्य‚ज्य व‚र्त्ते । त‚त्स्व‚भाव‚त्त्वाभाव‚प्र‚स‚ङ्गात् । त‚न्निवृत्तौ‚{\tiny $_{lb}$}‚च निवृ ‚{\tiny $_{3}$}‚ त्तिः ॥ कार्य‚स्यापि भावे कार‚ण‚स्य भावो युक्तः । त‚न्निवृत्तौ च ‚{\tiny $_{lb}$}‚निवृत्तिर‚न्य‚था तेन विनापि भावात्कार्य‚मेव त‚त्त‚स्य न स्यात्त‚द‚न्य‚व‚त् । ‚{\tiny $_{lb}$}‚त‚स्मात्स्व‚भाव ‚{\tiny $_{4}$}‚ कार‚ण‚भूत‚साध्य‚भेदात् । द्विधैव विधिरूपं साध्यं त‚त्रैव हेतोः ‚{\tiny $_{lb}$}‚प्र‚तिब‚न्धात् । स्व‚भाव‚भूत‚ञ्च साध्यं स्व‚भाव‚हेतुः साध‚य‚ति । कार‚ण‚भूत‚ञ्च ‚{\tiny $_{lb}$}‚कार्य‚हेतु ‚{\tiny $_{5}$}‚ रित्य‚भ्युपेयं । प्र‚तिषेध‚म‚प्युप‚ल‚ब्धिल‚क्ष‚ण‚प्राप्तानुप‚ल‚ब्धिरेव साध‚य‚ति । ‚{\tiny $_{lb}$}‚नान्या य‚था त‚था विस्त‚रेण प्र‚तिपाद‚यिष्य‚ति । त‚दिद‚मिह संक्षिप्त‚म‚र्थ ‚{\tiny $_{6}$}‚ त‚त्त्वं ‚{\tiny $_{lb}$}‚विष‚य‚व्य‚पेक्ष‚या विष‚यिणो लिंग‚स्य व्य‚व‚स्था । विष‚य‚श्च विधिः प्र‚तिषेधो वा ‚{\tiny $_{lb}$}‚भ‚वेत् । विधाव‚प्य‚र्थान्त‚र‚म्वा विधीये तान‚र्थान्त‚रं वा अर्थान्त‚र‚विधाव‚पि का ‚{\tiny $_{7}$}‚ ‚{\tiny $_{lb}$}‚र्य‚कार‚ण‚म‚नुभ‚य‚म्वा साध्य‚ते । कार्य‚प्र‚तिपाद‚नेपि कार‚ण‚सामान्य‚म्वाऽमेघादि‚{\tiny $_{lb}$}‚व्यावृत्तं व‚स्तुमात्रं लिङ्ग‚त्वेनोच्य‚ते । कार‚ण‚विशेषो वा योऽप्र‚ति‚{\tiny $_{lb}$}‚ब‚द्ध‚साम‚र्थ्ये मे ‚{\tiny $_{8}$}‚ घादिः । प्र‚तिषेधोपि निषेध्याभिम‚त‚स्यानुप‚ल‚म्भेनोप‚ल‚म्भेन ‚{\tiny $_{lb}$}‚वा प्र‚तिपाद्येत । अनुप‚ल‚म्भेपि उप‚ल‚म्भ‚निवृत्तिमात्र‚ल‚क्ष‚णो वा भ‚वेत् । ‚{\tiny $_{lb}$}‚त‚त्तुल्य‚योगाव‚स्थ[ः] केव‚ला ‚{\tiny $_{9}$}‚\leavevmode\ledsidenote{\textenglish{4a/msK}} प‚र‚प‚दार्थोप‚ल‚म्भ‚रूपो वेति विक‚ल्पाः । त‚त्र‚{\tiny $_{lb}$}‚नाव‚श्यं ग‚म्भीर‚ध्वानादियुक्त‚म‚पि मेघादिकार‚ण‚मात्रं वृष्ट्यादिकार्याविर्भाव‚क‚{\tiny $_{lb}$}‚म‚न्त‚रा प्र‚तिब‚न्ध‚स‚म्भ‚वेन व्य ‚{\tiny $_{1}$}‚ भिचारात् । अतो न कार‚ण‚मात्रं ग‚म‚कं । कार‚ण‚{\tiny $_{lb}$}‚विशेषाद‚प्य‚प्र‚तिह‚त‚श‚क्तेर‚न‚न्त‚रं स‚म्ब‚न्ध‚स्मृतिव्य‚व‚हिताद‚नुमेय‚विज्ञानात्प्राक् ‚{\tiny $_{lb}$}‚कार्य‚मेवोद्भूत‚म‚क्ष‚ज्ञान‚ग्राह्यं ‚{\tiny $_{2}$}‚ भ‚व‚तीति न त‚स्यापि लिङ्ग‚त्वं । कार्य‚न्तु कार‚ण‚{\tiny $_{lb}$}‚लिङ्गं युक्तं । त‚द‚विनाभावात् । त‚द‚तिक्र‚मे वा हेतुम‚त्ताम्विलंघ‚येत् । अनुभ‚य‚म‚प्य‚{\tiny $_{lb}$}‚स‚म्ब‚न्धानुग‚म‚क‚म‚तिप्र‚स‚ङ्गो ‚{\tiny $_{3}$}‚ वा । अन‚र्थान्त‚र‚म‚प्य‚व्य‚भिचाराङ्ग‚म्य‚त इति युक्ति‚{\tiny $_{lb}$}‚म‚त् । निषेधोप्युप‚ल‚म्भेन न युज्य‚ते विरोधात् । क‚थं हि नामोप‚ल‚भ्य‚ते च नास्ति ‚{\tiny $_{lb}$}‚ \leavevmode\ledsidenote{\textenglish{5/s}} चेत्युप‚प‚द्य‚ते । त‚ज्ज्ञान ‚{\tiny $_{4}$}‚ निवृत्तिमात्र‚म‚पि व्य‚भिचारि । एक‚ज्ञान‚संस‚र्गिण‚स्तु कैव‚ल्य‚{\tiny $_{lb}$}‚दृष्टेर‚स‚त्ताव्य‚व‚हारो युक्तो य‚दि हि स्यादुप‚ल‚भ्य स‚त्त्व एव भ‚वेत् । प्र‚माण‚ञ्च य‚द‚{\tiny $_{lb}$}‚नुप‚प‚द्य ‚{\tiny $_{5}$}‚ मान‚विष‚यं न त‚द्विष‚यि युक्तं । य‚था वाजिविषाणं । अनुप‚प‚द्य‚मान‚विष‚य‚{\tiny $_{lb}$}‚ञ्चोक्त‚प्र‚कारेण स्व‚भावादि स्याद‚न्य‚संयोग्यादिप‚र‚प‚रिक‚ल्पितं लिं ‚{\tiny $_{6}$}‚ ग‚मिति व्याप‚{\tiny $_{lb}$}‚कानुप‚ल‚म्भः । वैध‚र्म्येण नीलादिज्ञानं । त‚देव वा कार्यादिलिङ्गं । नैमित्तिक‚श‚ब्दा‚{\tiny $_{lb}$}‚र्थानुप‚प‚त्तिर्बाधिका । किमेवं सिद्ध‚म‚ग‚म‚क‚त्त्व‚म‚न्येषां । त‚थाहि विष ‚{\tiny $_{7}$}‚ यित्व‚ङ्ग‚{\tiny $_{lb}$}‚म‚क‚त्त्वं प्र‚काश‚क‚त्त्व‚मित्याद‚यः प‚र्यायाः । त‚स्मात्साध्य‚स्य त्रिविध‚त्त्वात्त‚ङ्ग‚म‚को ‚{\tiny $_{lb}$}‚हेतुर‚पि त्रिविध एवेति स्थित‚मेत‚त् ।
	{\color{gray}{\rmlatinfont\textsuperscript{§~\theparCount}}}
	\pend% ending standard par
      ‚{\tiny $_{lb}$}‚

	  
	  \pstart \leavevmode% starting standard par
	\hphantom{.}क‚थ‚न्त्रिविध‚मित्याह । \quotelemma{स्व‚भाव} \cite[1b2]{vn-msN} इत्यादि ‚{\tiny $_{8}$}‚ [।] च श‚ब्दः ‚{\tiny $_{lb}$}‚स्व‚भाव‚कार्यापेक्ष‚या स‚मुच्च‚यार्थः । एत‚च्चाभिम‚त‚हेतुप्र‚द‚र्श‚नं स्व‚भाव‚कार्यानुप‚{\tiny $_{lb}$}‚ल‚म्भ‚भेदात्त्रिविध‚म‚प्र‚त्य‚क्ष‚स्य सिद्धेर‚ङ्गं । न‚तु कार‚णैकार्थ‚स‚म‚वायिविरो ‚{\tiny $_{9}$}‚\leavevmode\ledsidenote{\textenglish{4b/msK}} ध्यादि ‚{\tiny $_{lb}$}‚भ‚दादिति द‚र्श‚नार्थं । न‚नु चोप‚ल‚ब्धिल‚क्ष‚ण‚प्राप्तानुप‚ल‚म्भ‚श्चेति व‚क्त‚व्यं । त‚स्यैवा ‚{\tiny $_{lb}$}‚प्र‚त्य‚क्ष‚स्य सिद्धेर्व‚क्ष्य‚माणेन न्यायेनाङ्ग‚त्त्वान्नानुप‚ल‚म्भ‚मात्र‚स्य त‚त्क‚थं सामान्ये‚{\tiny $_{lb}$}‚तो ‚{\tiny $_{1}$}‚ क्त‚मिति चेत् । एव‚मेत‚त्स‚म‚र्थित‚साध‚नाङ्गाधिकारात्तु सामान्य‚श‚ब्दोप्य‚य‚{\tiny $_{lb}$}‚म‚नुप‚ल‚म्भ‚श‚ब्दौप‚ल‚ब्धिल‚क्ष‚ण‚प्राप्तानुप‚ल‚म्भ एव व‚र्त्त‚ते । त‚थाहि सामान्य‚श‚ब्दा ‚{\tiny $_{lb}$}‚अपि श‚ब्दा ‚{\tiny $_{2}$}‚ न्त‚र‚स‚न्निधानात्प्र‚क‚र‚ण‚साम‚र्थ्याच्च विशेषेष्व‚व‚तिष्ठ‚न्ते । य‚दाह । \quotelemma{न‚हि ‚{\tiny $_{lb}$}‚विशेष‚श‚ब्द‚स‚न्निधेरेव श‚ब्दानां विशेषाव‚स्थितिहेतुर‚पि तु प्र‚क‚र‚ण‚साम‚र्थ्यादि‚{\tiny $_{lb}$}‚क‚म‚पीति} ‚{\tiny $_{3}$}‚ य‚था चानुप‚ल‚म्भ‚मात्र‚स्य स‚म‚र्थित‚साध‚नाङ्ग‚त्त्वं न स‚म्भ‚व‚ति त‚थोत्त‚र‚त्र ‚{\tiny $_{lb}$}‚प्र‚तिपाद‚यिष्य‚ति । अप्र‚त्य‚क्ष‚सिद्धिहेतुलिङ्गाधिकाराद्वाऽचोद्य‚मेवैत‚त् । ‚{\tiny $_{4}$}‚ किम्पुन‚{\tiny $_{lb}$}‚र‚स्य त्रिविध‚स्य साध‚नाङ्ग‚स्य स‚म‚र्थ‚नं । य‚द्विप‚र्य‚याद‚स‚म‚र्थ‚न‚म्भ‚विष्य‚तीत्याह । ‚{\tiny $_{lb}$}‚ \quotelemma{त‚स्ये} \cite[1b3]{vn-msN} त्यादि । साध्य‚श‚ब्दोत्रानित्य‚त्त्वादिध‚र्म‚मात्र‚स्य वाच‚कः । ‚{\tiny $_{5}$}‚ अव‚य‚व‚{\tiny $_{lb}$}‚स‚मुदायोप‚चारात् न तु साध्य‚ध‚र्मिध‚र्म‚स‚मुदाय‚स्य व्याप्तेरेवाभाव‚प्र‚स‚ङ्गात । ‚{\tiny $_{lb}$}‚श‚ब्दादिध‚र्म्मिविशिष्ट‚स्यानित्त्य‚त्त्वादेर्दृष्टान्त‚ध‚र्मिण्य‚भावात् । ‚{\tiny $_{6}$}‚ व्याप्तिं प्र‚साध्या‚{\tiny $_{lb}$}‚न्व‚य‚व्य‚तिरेक‚साध‚केन प्र‚माणेन अनेनान्व‚य‚व्य‚तिरेक‚निश्च‚यावुक्तौ । \quotelemma{ध‚र्मिणि} ‚{\tiny $_{lb}$}‚ \cite[1b3]{vn-msN} जिज्ञासित‚विशेषे श‚ब्दादौ भाव‚साध‚नं \cite[1b2]{vn-msN} । प‚क्ष‚ध‚र्म‚त्त्व‚साध‚केन ‚{\tiny $_{lb}$}‚प्र‚मा ‚{\tiny $_{7}$}‚ णेन स‚त्त्व‚क‚थ‚न‚मित्य‚र्थः । अनेनापि प‚क्ष‚ध‚र्म‚त्त्व‚निश्च‚य उक्तः । क‚थ‚म्पुनः ‚{\tiny $_{lb}$}‚स‚र्व्वोप‚संहारेण व्याप्तिं प्र‚साध्य ध‚र्मिणि भावः क‚थ्य‚त इत्य‚त्रोदाह‚र‚ण‚माह । ‚{\tiny $_{lb}$}‚ \quotelemma{य‚थेत्या} \cite[1b3]{vn-msN} ‚{\tiny $_{8}$}‚ दि । \quotelemma{त‚त्स‚र्व्व‚मि} त्य‚नेन स‚र्व्वोप‚संहारेण व्याप्तिप्र‚द‚र्श‚न‚ङ्क‚{\tiny $_{lb}$}‚थ‚य‚ति । किम‚र्थं [।] विप्र‚तिप‚त्तिनिरासार्थं [।] त‚थाहि प‚क्ष‚स‚प‚क्षा‚{\tiny $_{lb}$}‚ \leavevmode\ledsidenote{\textenglish{6/s}} पेक्ष‚यान्त‚र्व्याप्तिर्ब्ब‚हिर्व्याप्तिश्च प्र‚द‚र्श्य ‚{\tiny $_{9}$}‚\leavevmode\ledsidenote{\textenglish{5a/msK}} त इत्येके विप्र‚तिप‚न्नाः । त‚च्च ‚{\tiny $_{lb}$}‚न युक्तं व‚स्तुब‚लायात‚त्त्वाद्व्याप्तेः । पूर्व्व साध्येन व्यार्प्तिं प्र‚साध्य प‚श्चा‚{\tiny $_{lb}$}‚द् ध‚र्म्मिणि स‚त्त्वं क‚थ‚यित‚व्य‚मित्य‚य‚मीदृशः क्र‚म‚निय‚मः किम‚त्रास्ति न वेत्या ‚{\tiny $_{1}$}‚ ह । ‚{\tiny $_{lb}$}‚ \quotelemma{अत्रापी} \cite[1b3]{vn-msN} त्यादि । अत्रेति म‚न्म‚ते साध‚नाङ्ग‚स‚म‚र्थ‚ने वा ऽपिश‚ब्दोऽव‚धार‚णे‚{\tiny $_{lb}$}‚प्र‚तिषेधे न च स‚म्ब‚न्ध‚नीयः । नैव क‚श्चिद‚य‚मीदृशः क्र‚म‚निय‚मः \cite[1b3]{vn-msN} प‚रिपाटि‚{\tiny $_{lb}$}‚निय‚म इति या ‚{\tiny $_{2}$}‚ व‚त् । किंकार‚ण‚मित्याह । इष्टार्थ‚सिद्धेरुभ‚य‚त्राविशेषा \cite[1b3]{vn-msN} ‚{\tiny $_{lb}$}‚दिति । व्याप्तिसाध‚नाभिधान‚पूर्व्व‚क‚ध‚र्मिभाव‚साध‚नाभिधाने ध‚र्मिभाव‚साध‚नाभिधाने ‚{\tiny $_{lb}$}‚ध‚र्मिभाव‚साध‚नाभिधान‚पूर्व्व‚के वा व्याप्तिसा ‚{\tiny $_{3}$}‚ ध‚नाभिधाने साध्यार्थ‚सिद्धेर्विशेषाभावा‚{\tiny $_{lb}$}‚दित्य‚र्थः । एव‚म कूतं क्र‚म‚निय‚मो हि किम‚र्थ‚माश्रीय‚ते [।] साध्य‚सिध्य‚र्थं । ‚{\tiny $_{lb}$}‚य‚था साध‚र्म\edtext{}{\lemma{र्म}\Bfootnote{? र्म्य}}व‚ति दृष्टान्त‚प्र‚योगे सा ‚{\tiny $_{4}$}‚ ध्येनैव हेतोर‚विनाभावः प्र‚द‚र्श्य‚ते । ‚{\tiny $_{lb}$}‚न हेतुना साध्य‚स्य । त‚था वैध‚र्म्य‚व‚ति साध्याभाव एव हेतोर‚भावः क‚थ्य‚ते । न तु ‚{\tiny $_{lb}$}‚हेत्व‚भावे साध्य‚स्य । किम‚र्थं । माभूद्धेतोः ‚{\tiny $_{5}$}‚ साध्येनाविनाभावित्त्वाप्र‚द‚र्श‚नेनेष्टार्थ‚{\tiny $_{lb}$}‚सिद्धेर‚सिद्धिर्विप‚र्य‚य‚सिद्धिश्चेति । य‚दाह । एवं हि हेतोः स‚प‚क्ष एव स‚त्त्वं । साध्या‚{\tiny $_{lb}$}‚भावे चास‚त्त्व‚मेव श‚क्यं द ‚{\tiny $_{6}$}‚ र्श‚यितुं । न विप‚र्य‚यादिति । य‚था नित्य‚ताऽकृत‚क‚त्त्वेन ‚{\tiny $_{lb}$}‚नाशित्त्वाद्वाऽत्र कार्य‚ता । स्याद‚नुक्ता कृता व्यापित्त्व‚निष्ठ‚श्च स‚म‚न्व‚य इति । ‚{\tiny $_{lb}$}‚त‚द‚त्र युक्तं क्र‚म‚स‚माश्र‚य‚णं ‚{\tiny $_{7}$}‚ इह तु विनाप्य‚नेनाभिम‚तार्थ‚सिद्धिः स‚म्प‚द्य‚त इति ‚{\tiny $_{lb}$}‚सूव‚त‚न्न क‚श्चित्क्र‚म‚निय‚म इति । इष्टार्थ‚सिद्धेरुभ[य]त्राविशेषादित्येत‚देवात्र कुत ‚{\tiny $_{lb}$}‚इति चेदाह । य‚स्माद् ध‚र्मिणी ‚{\tiny $_{8}$}‚ त्यादि । साध्येन व्याप्तिं प्र‚साध्येत्युक्तं प्राक् । किम्पु‚{\tiny $_{lb}$}‚न‚स्त‚द्व्याप्तिसाध‚न‚मित्याह । \quotelemma{अत्रेत्यादि} \cite[1b3]{vn-msN} । अत्रेति स्य‚भाव‚हेतौ । कार्या‚{\tiny $_{lb}$}‚नुप‚ल‚म्भ‚योस्तु प‚श्चाद्व्याप्तिसाध‚न‚म‚भिधा ‚{\tiny $_{9}$}‚ \leavevmode\ledsidenote{\textenglish{5b/msK}} स्यात् । विप‚र्य‚ये साध्य‚स्य हेतोर्व ‚{\tiny $_{lb}$}‚र्त्त‚मान‚स्य स‚त इति शेषः । बाध‚कं प्र‚माणं येन साध्य‚विप‚र्य‚ये व‚र्त्त‚मानो हेतुर्बाध्य‚ते ‚{\tiny $_{lb}$}‚त‚स्य क‚थ‚नं य‚त्त‚द्व्याप्तिसाध‚न‚मित्य‚र्थः ।
	{\color{gray}{\rmlatinfont\textsuperscript{§~\theparCount}}}
	\pend% ending standard par
      ‚{\tiny $_{lb}$}‚

	  
	  \pstart \leavevmode% starting standard par
	किं ‚{\tiny $_{1}$}‚ पुन‚स्त‚द्वाध‚क‚प्र‚माणोप‚द‚र्श‚न‚मित्याह [—] य‚दि न स‚र्व्वं व‚स्तु स‚त्कृत‚कं ‚{\tiny $_{lb}$}‚ \leavevmode\ledsidenote{\textenglish{7/s}} चेति \cite[1b5]{vn-msN} पूर्व्वोक्तिहेतुद्व‚यं प‚रामृष‚ति । प्र‚तिक्ष‚ण‚विनाशि \cite[1b5]{vn-msN} स्यात्त‚दा‚{\tiny $_{lb}$}‚ऽस‚देव स्यादिति स‚म्ब‚न्धः । कुतः ‚{\tiny $_{2}$}‚ अक्ष‚णिक‚स्य प‚दार्थ‚स्य क्र‚म‚यौग‚प‚द्याभ्याम‚र्थ‚{\tiny $_{lb}$}‚क्रिया \cite[1b5]{vn-msN} ऽयोगात् । त‚थाह [—] क्ष‚णिक‚त्त्वेनाभिम‚त‚स्य भाव‚स्य क्र‚मेण ‚{\tiny $_{lb}$}‚ताव‚द‚र्थ‚क्रिया न युज्य‚ते । कार्य‚निर्व‚र्त्त‚न‚योग्य‚स्य ‚{\tiny $_{3}$}‚ स्व‚भाव‚स्य स‚दा स‚त्त्वात् । अन्य‚था ‚{\tiny $_{lb}$}‚प‚श्चाद‚पि न कुर्यात् । पूर्व्व‚स्व‚भावाप्र‚च्युतः पुराव‚त् । स‚ह‚कारिण‚मासाद्य क‚रोतीति ‚{\tiny $_{lb}$}‚चेत् । न अनाधेयात्माति ‚{\tiny $_{4}$}‚ श‚य‚स्य पूर्व्व‚स्व‚भावाप‚रित्यागिनः स‚ह‚कारिष्व‚पेक्षायो‚{\tiny $_{lb}$}‚गात् । आद‚ध‚त्त्येव स‚ह‚कारिण‚स्त‚स्यात्मातिश‚य‚मिति चेत् । न स‚ह‚कारिभिराहि‚{\tiny $_{lb}$}‚त‚स्यातिश ‚{\tiny $_{5}$}‚ य‚स्य त‚त्त्वान्य‚त्त्वायोगात् । त‚थाहि न ताव‚द‚य‚मात्मातिश‚य‚स्त‚स्यात्म‚{\tiny $_{lb}$}‚भूतः । त‚स्यैव त‚द‚व्य‚तिरेकादात्मातिश‚य‚व‚त्स‚ह‚कारिब‚लादुत्प‚त्तिप्र‚स‚ङ्गात् ‚{\tiny $_{6}$}‚ [।] ‚{\tiny $_{lb}$}‚एव‚ञ्चाभ्युपेत‚म‚स्याक्ष‚णिक‚त्त्व‚म‚व‚हीय‚ते व्य‚तिरिक्त एव स त‚स्मादिति चेत् । ‚{\tiny $_{lb}$}‚भ‚व‚तु किन्तु त‚स्मादेवात्मातिश‚यात्कार्योत्प‚त्तेस्त‚द‚व‚स्थ‚म‚स्यार्थ‚क्रियास्व‚साम‚र्थ\edtext{}{\lemma{र्थ}\Bfootnote{‚{\tiny $_{lb}$}‚? र्थ्य}}मि ‚{\tiny $_{7}$}‚ ति दुर्न्निवारः प्र‚स‚ङ्गः स‚माप‚त‚ति । स‚म्ब‚न्धोप्य‚नेन क‚थ‚मिति व‚श्चिन्ता ‚{\tiny $_{lb}$}‚विष‚य‚म‚व‚त‚र‚त्येव । अतिश‚य‚ब‚लात्क‚रोतीत्य‚त्रापि स‚ह‚कार्य‚पेक्षाप‚क्षोदितो दोषः ॥ ‚{\tiny $_{8}$}‚ ‚{\tiny $_{lb}$}‚स‚म‚र्थ‚स्व‚भाव‚त्त्वाद‚नाधेयातिश‚य‚त्त्वेपि कुविन्दादिव‚त् किञ्चिद‚पेक्ष्य कार्य‚{\tiny $_{lb}$}‚ज‚न‚क इति चेत् । न त‚त् सारं । न‚हि स‚ह‚कारिणः प्र‚त्य‚यास्त‚स्य ताव‚द‚तिश‚य‚मा‚{\tiny $_{lb}$}‚धातुं ‚{\tiny $_{9}$}‚ \leavevmode\ledsidenote{\textenglish{6a/msK}} क्ष‚माः । न चाप्य‚नुप‚कार‚के भावेऽपेक्षा युक्तिम‚नुप‚त‚त्य‚तिप्र‚सा\edtext{}{\lemma{सा}\Bfootnote{? स}}ङ्गात् । ‚{\tiny $_{lb}$}‚एव‚ञ्च स‚र्व्व‚काल‚म‚स्याकार्य‚ज‚न‚क‚त्त्व‚प्र‚स‚ङ्गः । कुविन्दादीनाम‚पि त‚त्स्व‚भाव‚स्य ‚{\tiny $_{lb}$}‚क‚र‚णाद‚का ‚{\tiny $_{1}$}‚ र‚क‚स्य वा [।] \quotelemma{त‚त्स्व‚भाव‚त्त्वादित्यादि हेतुविन्दा}\edtext{\textsuperscript{*}}{\lemma{*}\Bfootnote{न्याय‚विन्दाव‚पि - त‚त्स्व‚भाव‚त्त्वात्त‚त्स्व‚भाव‚स्य च हेतुत्त्वात् \href{http://sarit.indology.info/?cref=nb.3.69}{तृ० प‚रि० पृ० ६९} \begin{english}\textit{Kashi Sanskrit Series.}\end{english}}} ‚{\tiny $_{1}$}‚ वुक्त‚मिति ‚{\tiny $_{lb}$}‚नेहोच्य‚ते । एव‚न्ताव‚त्क्र‚मेणास्यार्थ‚क्रिया न युज्य‚ते नापि युग‚प‚त् । त‚थाहि ‚{\tiny $_{lb}$}‚अर्थ‚क्रियानिव‚र्त्त‚न‚योग्य‚स्व ‚{\tiny $_{2}$}‚ भावाध्यासित‚मूर्तिः स‚हैवासाव‚त‚श्च प‚श्चाद‚पि त‚द्रूप‚{\tiny $_{lb}$}‚वियोगात्कार्य‚मुत्पाद‚येद‚न्य‚थोद‚यान‚न्त‚र‚मेवास्य क्ष‚यः स्यात् । न चाप्य‚क्ष‚णि‚{\tiny $_{lb}$}‚क‚त्त्वेनोप‚ग‚त‚स्य स‚कृ ‚{\tiny $_{3}$}‚ त् कार्य‚मुत्प‚द्य‚मान‚मुप‚ल‚भ्य‚ते क्र‚म‚स‚म्भ‚व‚द‚र्श‚नात् । त‚दे‚{\tiny $_{lb}$}‚व‚म‚य‚म‚क्ष‚णिकः प‚दार्थः क्र‚मेण युग‚प‚द्वा न काञ्चिद‚प्य‚द‚र्थ‚क्रियामात्रामंश‚तोपि क्ष ‚{\tiny $_{4}$}‚ मो ‚{\tiny $_{lb}$}‚निर्व‚र्त‚यितुमित्त्य‚स‚त्त्व‚मेवास्य । य‚दि नामार्थ‚क्रिया सा न युक्ता त‚क्तिमित्य‚स‚त्त्व‚मे‚{\tiny $_{lb}$}‚वास्य स्यादित्याह । \quotelemma{अर्थ‚क्रियेत्यादि} \cite[1b5]{vn-msN} । अर्थ‚क्रियायाः साम ‚{\tiny $_{5}$}‚ र्थ्य‚न्त‚देव ल‚क्ष‚णं ‚{\tiny $_{lb}$}‚य‚स्य स‚त्त्व‚स्येति विग्र‚हः । अतोऽर्थ‚क्रियासाम‚र्थ्य‚ल‚क्ष‚णात्स‚त्त्वाद्व्यावृत्त‚म्व्य‚व‚च्छिन्नं । ‚{\tiny $_{lb}$}‚ \leavevmode\ledsidenote{\textenglish{8/s}} \quotelemma{इति} श्रुतेर्हेतौ । त‚स्माद‚र्थे वा [।] क‚थ‚म्पुन‚रि ‚{\tiny $_{6}$}‚ द‚म‚व‚सीय‚ते [।] अर्थ‚क्रियासाम‚र्थं\edtext{}{\lemma{र्थं}\Bfootnote{‚{\tiny $_{lb}$}‚? र्थ्यं}}स‚त्त्व‚ल‚क्ष‚ण‚मित्य‚त आह । \quotelemma{स‚र्व्वे} \cite[1b5]{vn-msN} त्यादि । स‚र्व्वेषां साम‚र्थ्याना‚{\tiny $_{lb}$}‚मुपाख्या श्रुतिः । उपाख्याय‚ते अन‚येति कृत्वा त‚स्याविर‚होऽभावः । ‚{\tiny $_{7}$}‚ स एव ल‚क्ष‚णं ‚{\tiny $_{lb}$}‚य‚स्य निरुपाख्य‚स्य स त‚था ख्याय‚ते । अत्रापीति श्रुतिहेतौ । स‚र्व्व‚ग्र‚ह‚णं घ‚टादीना‚{\tiny $_{lb}$}‚म‚पि क्ष‚णिक‚त्त्वेनाभिम‚तानाम्विष‚य‚भेदेनार्थ‚क्रियासाम‚र्थ्य ‚{\tiny $_{8}$}‚ निर्वृत्तेर‚स्तीति तेषाम‚स‚त्त्व‚{\tiny $_{lb}$}‚व्य‚व‚च्छेदाय । नैवार्थ‚क्रियासाम‚र्थ्यं स‚त्त्व‚ल‚क्ष‚ण‚म‚पि तु स‚त्तायोग इति चेत् । न । ‚{\tiny $_{lb}$}‚स‚त्ताया अभा ‚{\tiny $_{9}$}‚ \leavevmode\ledsidenote{\textenglish{6b/msK}} वात् । त‚द‚भाव‚श्चान्य‚त्र प्र‚तिपादित इति नेहोच्य‚ते । स‚त्तायाश्च ‚{\tiny $_{lb}$}‚नैव स‚त्त्वं प्राप्नोति स‚त्तायोगाभावात् । निःसामान्या ‚{\tiny $_{1}$}‚ नि सामान्यानीति स‚म‚यात् । ‚{\tiny $_{lb}$}‚न च स्व‚य‚म‚त‚द्रूपाः प‚दार्थात्मानः स्व‚भावान्त‚र‚स‚म्प‚र्क्क‚मासाद‚य‚न्तोपि ताद्रूप्य‚म्प्र‚ति‚{\tiny $_{lb}$}‚प‚द्य‚न्ते । स्फ‚टिकाभ्र‚प‚ट‚लाद‚य इव ‚{\tiny $_{2}$}‚ ज‚वाकुसुमादिरूप‚मिति य‚क्तिञ्चिदेत‚त् । ‚{\tiny $_{lb}$}‚त‚त्वेताव‚देवाभिधानीयं । \quotelemma{स‚र्व्व‚साम‚र्थ्य‚विर‚ह‚ल‚क्ष‚णं निरुपाख्य‚मि} \cite[1b6]{vn-msN} ति ‚{\tiny $_{lb}$}‚त‚क्तिम‚नेनोपाख्याग्र‚ह‚णेनेति चेत् । सूक्त ‚{\tiny $_{3}$}‚ मेत‚त्स‚र्व्व‚साम‚र्थ्य‚र‚हित‚स्य तु साम‚र्थ्य‚{\tiny $_{lb}$}‚निब‚न्ध‚न‚स्य क‚स्य‚चिद‚पि श‚ब्द‚स्यावृत्तेर‚स‚द्व्य‚व‚हार‚विष‚य‚त्त्व‚ख्याप‚नाय । संज्ञा‚{\tiny $_{lb}$}‚याश्चानुग‚तार्थ‚त्त्व‚सिद्ध्य‚र्थ ‚{\tiny $_{4}$}‚ मिद‚मुक्त‚मिति ग‚म्य‚ते । य‚दि त्वेवं साध्य‚विप‚र्य‚ये हेतो‚{\tiny $_{lb}$}‚त्बाध‚क‚प्र‚माणोप‚द‚र्श‚नं न क्रिय‚ते त‚तः किं स्यादितिचेदाह । एवं \quotelemma{साध‚न‚स्य साध्य‚{\tiny $_{lb}$}‚विप ‚{\tiny $_{5}$}‚ र्य‚ये बाध‚क‚प्र‚माणानुप‚द‚र्श‚ने} \cite[1b6]{vn-msN} स‚त्य‚निवृत्तिरेवाशंकाया इति स‚म्ब‚न्धः । ‚{\tiny $_{lb}$}‚का पुनः सा शंका । स कृत‚को वा स्यान्नित्य‚श्चैवं प्र‚कारा ।
	{\color{gray}{\rmlatinfont\textsuperscript{§~\theparCount}}}
	\pend% ending standard par
      ‚{\tiny $_{lb}$}‚

	  
	  \pstart \leavevmode% starting standard par
	न‚नु विप ‚{\tiny $_{6}$}‚ क्ष‚हेतोर्वृत्तिर्नोप‚ल‚भ्य‚ते त‚त्क‚थ‚म‚निवृत्तिरित्याह । \quotelemma{अद‚र्श‚ने \cite[1b6]{vn-msN} पि} ‚{\tiny $_{lb}$}‚वृत्तेर‚स्य साध‚न‚स्याक्ष‚विप‚र्य‚ये साध्य‚स्य ह‚युप‚स्कारः । किमित्येव‚म‚प्य‚निवृत्तिरित्याह । ‚{\tiny $_{lb}$}‚ \quotelemma{वि ‚{\tiny $_{7}$}‚ रोधाभावादि} \cite[1b6]{vn-msN} ति । साध‚न‚साध्य‚विप‚क्ष‚योरिति शेषः । अय‚म‚स्याभिप्रायो य‚दि ‚{\tiny $_{lb}$}‚साध‚न‚स्य साध्य‚विप‚र्य‚स्य च प‚र‚स्प‚र‚म्विरोधः सिद्धः स्यात् । भ‚वेद‚द‚र्श‚न‚मात्रे ‚{\tiny $_{8}$}‚ \leavevmode\ledsidenote{\textenglish{7a/msK}} ते अन्य‚था ‚{\tiny $_{lb}$}‚बाध‚कासिद्धौ संश‚यो दुर्न्निवारः स्यादित्याश‚ङ्क्याह । न च नैव स‚र्व्वानुप‚ल‚ब्धिर्बा‚{\tiny $_{lb}$}‚धिका प्र‚तिषेधिका युक्तेत्युप‚स्कारः । क‚स्य भाव‚स्य स‚त्त्व‚स्य साध्य‚विप‚र्य‚ये ‚{\tiny $_{1}$}‚ हेतोरिति ‚{\tiny $_{lb}$}‚शेषः । प्र‚क‚र‚णाद्वैत‚द् ग‚म्य‚त एव । इद‚मुक्त‚म्भ‚व‚ति । व्याप‚कानुप‚ल‚ब्धिरेव स‚ह‚भाव ‚{\tiny $_{lb}$}‚ \leavevmode\ledsidenote{\textenglish{9/s}} बाध‚ते हेतोः साध्याभावेन । य‚था प्र‚तिपादित‚म्प्राक् । \quotelemma{नाप्य‚द‚र्श ‚{\tiny $_{2}$}‚ न‚मात्राद्व्यावृत्तिर्वि‚{\tiny $_{lb}$}‚प्र‚कृष्ट‚स‚र्व्व‚द‚र्शिनोऽद‚र्श‚न‚स्याभावासाध‚नादि} \cite[1b7]{vn-msN} त्यादिना । त‚स्मात्सूक्त‚म‚न्य‚था ‚{\tiny $_{lb}$}‚बाध‚कासिद्धौ संश‚यो दुर्न्निवारः स्यादिति । अनेन पूर्व्वोक्त ‚{\tiny $_{3}$}‚ मेव स्मार‚य‚ति । एव‚ञ्चै ‚{\tiny $_{lb}$}‚त‚द‚र्थ‚व‚त् । अभाव‚साध‚न‚स्याद‚र्श‚न‚स्याप्र‚तिषेधादित्युक्तं । त‚त्र भ‚वेत्क‚स्य‚चिदाशंका ‚{\tiny $_{lb}$}‚किम‚नेनाभाव‚साध‚न‚स्येति विशे ‚{\tiny $_{4}$}‚ ष‚णेन याव‚ता स‚र्व्व‚मेवाद‚र्श‚न‚म‚भाव‚साध‚न‚मित्य‚{\tiny $_{lb}$}‚त‚स्त‚दाश‚ङ्काविनिवृत्त्य‚र्थ‚मिद‚माह । \quotelemma{न चेत्यादि} \cite[1b11]{vn-msN} । अत्र च श‚ब्दो हि ‚{\tiny $_{lb}$}‚श‚ब्दार्थे प्र‚तिप‚त्त‚व्यः । त‚स्मिं ‚{\tiny $_{5}$}‚ नैव वाऽव‚धार‚णे व्याख्यान‚न्तु पूर्व्व‚व‚त् ।
	{\color{gray}{\rmlatinfont\textsuperscript{§~\theparCount}}}
	\pend% ending standard par
      ‚{\tiny $_{lb}$}‚

	  
	  \pstart \leavevmode% starting standard par
	न‚नु च बाध‚क‚स्यैव प्र‚माण‚स्य क्र‚माक्र‚मायोग‚स्यासाम‚र्थ्येन व्याप्तिर्न सिद्धा त‚त्क‚थं ‚{\tiny $_{lb}$}‚त‚त्स्व‚यं असिद्ध‚व्याप्तिकं स ‚{\tiny $_{6}$}‚ द‚प‚र‚स्य स‚त्त्वादेर्हेतोर्व्याप्तिसाध‚ने प‚र्याप्तं । प्र‚मा‚{\tiny $_{lb}$}‚णान्त‚रेण त‚त्र व्याप्तिः साध्य‚त इति चेत् । सैव त‚र्हीय‚म‚न‚व‚स्थादोषादिवानुब‚ध्ना‚{\tiny $_{lb}$}‚तीति क‚दाचित्प‚रो ब्रूयादित्या ‚{\tiny $_{7}$}‚ श‚ङ्क्य स‚र्व्व‚मिद‚माह । \quotelemma{त‚त्रेत्यादि} \cite[1b11]{vn-msN} । त‚त्र ‚{\tiny $_{lb}$}‚श‚ब्दो वाक्योप‚न्यासार्थः । साम‚र्थ्यं य‚द्व‚स्तुल‚क्ष‚णं \quotelemma{त‚त्क्र‚माक्र‚म‚योगेन व्याप्तं सिद्धं} । ‚{\tiny $_{lb}$}‚य‚त्र साम‚र्थ्यं त‚त्र क्र‚माक्र‚माभ्याम‚र्थ‚क्रिय‚या ‚{\tiny $_{8}$}‚ भ‚वित‚व्य‚मित्य‚नेनाकारेण त‚देव कुत ‚{\tiny $_{lb}$}‚इति चेदाह । \quotelemma{प्र‚कारान्त‚रास‚ग्भ‚वात्} \cite[1b11]{vn-msN} । य‚स्माद‚न्य‚त् क्र‚माक्र‚म‚व्य‚तिरिक्तं ‚{\tiny $_{lb}$}‚प्र‚कारान्त‚रं नास्ति । त‚स्माद्य‚त्रेदं स‚त्त्व‚ल‚क्ष‚ण‚म‚र्थ‚क्रिया ‚{\tiny $_{9}$}‚ \leavevmode\ledsidenote{\textenglish{7b/msK}} साम‚र्थ्यं त‚त्राव‚श्यं च ‚{\tiny $_{lb}$}‚क्र‚माक्र‚माभ्याम‚र्थ‚क्रिय‚या भ‚वित‚व्यं [।] न‚नु च क्र‚माक्र‚माभ्याम‚न्यो रासि ‚{\tiny $_{lb}$}‚\leavevmode\ledsidenote{\textenglish{10/s}}\edtext{}{\lemma{रासि}\Bfootnote{? शि}}र्नास्तीत्येत‚देव क‚थं सिद्धं । क्र‚माक्र‚म‚योर‚न्योन्य‚प‚रिहार‚स्थित‚ल‚क्ष‚ण‚त्त्वेन ‚{\tiny $_{lb}$}‚तृती ‚{\tiny $_{1}$}‚ य‚प्र‚कार‚व्य‚तिरेक‚त्वात् । भावाभाव‚व‚दिति ब्रूमः । भ‚व‚त्वेवं स तु क्र‚म‚यौग‚प‚द्य‚{\tiny $_{lb}$}‚रूप एवेति कुत इति चेत् । क‚स्त‚ह‚र्य‚न्यो भ‚व‚तु । क‚श्चिद् भ‚वेदिति चेत् । किम‚र्थ‚{\tiny $_{lb}$}‚न्त‚र्हि म‚ह ‚{\tiny $_{2}$}‚ त्य‚न‚र्थ‚स‚ङ्क‚टे प‚तितोसि य‚दिद‚न्त‚या त‚द्रूपाभिधानेप्य‚स‚म‚र्थोसि । य‚दि ‚{\tiny $_{lb}$}‚नाम साम‚र्थ्यं क्र‚माक्र‚म‚योगेन व्याप्तं सिद्धं त‚थापिकिं सिद्ध‚मिति चेदाह । तेन ‚{\tiny $_{lb}$}‚साम‚र्थ्य ‚{\tiny $_{3}$}‚ स्य क्र‚माक्र‚म‚योगेन व्याप्त‚त्वेन \quotelemma{व्याप‚क‚स्य ध‚र्म‚स्य} क्र‚माक्र‚म‚योग‚स्यानुप‚{\tiny $_{lb}$}‚ल‚ब्धिः \cite[2a1]{vn-msN} । \quotelemma{अक्ष‚णिके} प‚दार्थेऽभ्युप‚ग‚ते \quotelemma{साम‚र्थ्यं बाध‚ते} निराक‚रोतीत्य‚र्थः ॥ ‚{\tiny $_{4}$}‚ ‚{\tiny $_{lb}$}‚त‚था ह्य‚यं क्र‚माक्र‚म‚योग‚स्त‚स्य साम‚र्थ्य‚स्य व्याप‚कः । त‚त‚श्चास्य निवृत्ताव‚श्य‚{\tiny $_{lb}$}‚मेव साम‚र्थ्य‚स्यापि निवृत्तिर‚न्य‚थाऽय‚म‚स्य व्याप‚क एव न प्राप्नोति । य‚स्मा ‚{\tiny $_{5}$}‚ त्त‚द्वा‚{\tiny $_{lb}$}‚ध‚ते इति त‚स्मा \quotelemma{त्क्र‚म‚यौग‚प‚द्यायोग‚स्य} व्याप‚काभाव‚स्य क‚र्म‚भूत‚स्य \quotelemma{साम‚र्थ्याभावेन} ‚{\tiny $_{lb}$}‚व्याप्याभावेन क‚र्त्तृभूतेन व्याप्तिसिद्धेः कार‚णान्नान‚व‚स्थाप्र ‚{\tiny $_{6}$}‚ स‚ङ्गः ।
	{\color{gray}{\rmlatinfont\textsuperscript{§~\theparCount}}}
	\pend% ending standard par
      ‚{\tiny $_{lb}$}‚

	  
	  \pstart \leavevmode% starting standard par
	\hphantom{.}एवं स्व‚भाव‚हेतोः साध‚नाङ्ग‚स‚म‚र्थ‚न‚म‚भिधायाधुना निग‚म‚य‚ति । \quotelemma{एव‚मि} \cite[2a1]{vn-msN} ‚{\tiny $_{lb}$}‚त्यादिना । एव‚ञ्च य‚दि न स‚म‚र्थ‚नं क्रिय‚ते त‚देव त‚द्वादिनः प‚राज‚य‚माव‚ह‚ती ‚{\tiny $_{7}$}‚ ति प्राक्‚{\tiny $_{lb}$}‚प्र‚तिज्ञात‚मेवायोज‚य‚ति । \quotelemma{त‚स्यास‚म‚र्थ‚न} \cite[2a2]{vn-msN} मित्यादिना । क‚स्मादेवं प्रार‚ब्धार्था‚{\tiny $_{lb}$}‚साध‚नादिति । न ह्य‚स‚म‚र्थितं साध‚न‚मार‚ब्ध‚म‚र्थं साध‚यितुं स‚म‚र्थं । विवादाभा ‚{\tiny $_{8}$}‚ \leavevmode\ledsidenote{\textenglish{8a/msK}} व‚{\tiny $_{lb}$}‚प्र‚स‚ङ्गात् । त‚थाहि सार्व्व‚ज्ञ‚ज्ञान‚साध‚ने संस्कारोत्क‚र्ष‚भेदेन स‚म्भ‚वे प्र‚क‚र्ष‚प‚र्य‚न्त‚वृ‚{\tiny $_{lb}$}‚त्त‚यः प्र‚ज्ञाद‚यो गुणाः स्थिराश्र‚य‚व‚र्त्ति स‚कृद्य‚थाक‚थंचिदाहित‚विशेषं विना ‚{\tiny $_{9}$}‚ हेतु‚{\tiny $_{lb}$}‚रात्म‚नीति । य‚दि त‚र्हि बाध‚क‚प्र‚माणोप‚द‚र्श‚नेन हेतोर्व्याप्तिः प्र‚साध्य‚ते । त‚था स‚त्त्य‚{\tiny $_{lb}$}‚न‚व‚स्था भ‚व‚तः प्राप्नोतीत्याशंकाप‚नोद‚नाय पूर्व्व‚प‚क्ष‚मार‚च[य]न्नाह । \quotelemma{अत्रापी} ‚{\tiny $_{1}$}‚ \cite[1b9]{vn-msN} ‚{\tiny $_{lb}$}‚त्यादि । अत्रेति बाध‚के प्र‚माणे । अद‚र्श‚न‚म‚प्र‚माण‚य‚त‚स्त‚व नाद‚र्श‚न‚मात्राद्धेतोर्व्य‚ति‚{\tiny $_{lb}$}‚रेक‚निश्च‚य इत्य‚नेनाकारेण न केव‚लं मौले हेतावित्य‚पि श‚ब्दः किं पूर्व्व ‚{\tiny $_{2}$}‚ स्यापि ‚{\tiny $_{lb}$}‚मौल‚स्य हेतोर‚व्याप्तिः प्राप्नोतीति क्रियाप‚दं । किङ्कार‚णं । क्र‚म‚यौग‚प‚द्यायोग‚स्य वा ‚{\tiny $_{lb}$}‚साम‚र्थ्येन व्याप्त्य‚सिद्धः । त‚थाहि य‚द्य‚द‚र्श‚न‚मात्रेण न व्य‚तिरेक‚निश्च‚य ‚{\tiny $_{3}$}‚ स्त‚था स‚ति ‚{\tiny $_{lb}$}‚क्र‚म‚यौग‚प‚द्यायोग‚श्च भ‚विष्य‚ति स‚ल्ल‚क्ष‚णं साम‚र्थ्य‚ञ्च भ‚विष्य‚ति । कोऽन‚योर्वि‚{\tiny $_{lb}$}‚रोध इत्य‚त्रैव बाध‚के प्र‚माणे व्याप्तिर्न सिद्धा । य‚दि नामात्र ‚{\tiny $_{4}$}‚ न सिद्धा मौल‚हेता‚{\tiny $_{lb}$}‚वेत‚द्व‚लेन व्याप्तिः सेत्स्य‚तीति चेदाह । पूर्व्व‚स्यापि मौल‚स्यापि हेतोर‚व्याप्तिः प्राप्नो‚{\tiny $_{lb}$}‚तीत्य‚ध्याहारः । त‚थाहि न स्व‚य‚म‚प्र‚माण‚कं बा ‚{\tiny $_{5}$}‚ ध‚कं प्र‚माण‚म‚न्य‚स्य प्र‚माण‚मुप‚क‚ल्प‚यि‚{\tiny $_{lb}$}‚ \leavevmode\ledsidenote{\textenglish{11/s}} तुम‚लं प्रामाण्य‚प्र‚स‚ङ्गात् ।
	{\color{gray}{\rmlatinfont\textsuperscript{§~\theparCount}}}
	\pend% ending standard par
      ‚{\tiny $_{lb}$}‚

	  
	  \pstart \leavevmode% starting standard par
	य‚द्येव‚म‚त्रापि त‚र्हि व्याप‚के प्र‚माणेऽन्येन व्याप‚केन प्र‚माणेन व्याप्तिर्निश्ची ‚{\tiny $_{6}$}‚ य‚त ‚{\tiny $_{lb}$}‚इत्याह । \quotelemma{इहापि} \cite[1b9]{vn-msN} न केव‚लं मौल‚हेतावित्य‚पिनाह । पुनः साध‚नोप‚क्र‚मे ‚{\tiny $_{lb}$}‚स‚त्य‚न‚व‚स्था भ‚वेत् । त‚थाहि य‚त्त‚द्वाध‚के प्र‚माणे व्याप्तिप्र‚साध‚नार्थं बाध‚कं प्र‚मा ‚{\tiny $_{7}$}‚ ण‚{\tiny $_{lb}$}‚मुच्य‚ते । त‚त्रापि व्याप्तिर‚न्येन बाध‚केन प्र‚माणेन साध्या । य‚स्मान्न त‚द‚पि स्व‚य‚म‚{\tiny $_{lb}$}‚प्र‚माण‚मित‚र‚स्य प्रामाण्यं क‚र्त्तु स‚म‚र्थ‚मित्येत‚त्त‚त्रापि श‚क्य‚म्व‚क्तुं । त‚स्याप्य‚न्येन ‚{\tiny $_{8}$}‚ ‚{\tiny $_{lb}$}‚व्याप्तिः साध्य‚त इति चेत् । य‚द्येवं त‚त्रापीय‚मेव वार्त्तेंत्य‚न‚व‚स्था भ‚व‚त‚स्त‚था ‚{\tiny $_{lb}$}‚स‚ति प्र‚स‚ज‚ति । एवं स‚मार‚चित‚पूर्व्व‚प‚क्षः साम्प्र‚त‚म‚त्र प्र‚तिविधान‚माह । \quotelemma{नाभा‚{\tiny $_{lb}$}‚व‚सा ‚{\tiny $_{9}$}‚ \leavevmode\ledsidenote{\textenglish{8b/msK}} ध‚न‚स्ये} \cite[1b10]{vn-msN} त्यादि । व्य‚तिरेक‚साध‚न‚त्त्वेनेत्युप‚स्कार[ः ।] इद‚मुक्त‚म्भ‚व‚ति । ‚{\tiny $_{lb}$}‚न स‚र्व्व‚म‚त्राद‚र्श‚नं प्र‚तिक्षिप्य‚ते । व्य‚तिरेक‚निश्चाय‚क‚स्य व्याप‚कानुप‚ल‚ब्धिसंज्ञ‚{\tiny $_{lb}$}‚क‚स्यानिषेधात् । किन्त्व ‚{\tiny $_{1}$}‚ द‚र्श‚न‚मात्र‚मिति । य‚दाह । \quotelemma{य‚द‚द‚र्श‚न‚म्विप‚र्य‚य‚म‚भावं साध‚{\tiny $_{lb}$}‚य‚ती} \cite[1b10]{vn-msN} \quotelemma{ति} । क‚स्य हेतोः कुत्र साध्य‚विप‚र्य‚ये त‚द‚द‚र्श‚न‚म‚स्य हेतोर्बाध‚क‚म्प्र‚माण‚{\tiny $_{lb}$}‚मुच्य‚ते । क‚स्माद्विरुद्ध‚प्र‚त्यु ‚{\tiny $_{2}$}‚ प‚स्थाप‚नात् अस्येति व‚र्त्त‚ते । त‚थाहि य‚स्य क्र‚म‚यौग‚प‚द्या‚{\tiny $_{lb}$}‚भ्याम‚र्थ‚क्रियायोग‚स्त‚स्य साम‚र्थ्य‚ल‚क्ष‚णं स‚त्त्वं नास्ति । य‚था ब‚न्ध्यात‚न‚यादीनान्त‚था ‚{\tiny $_{lb}$}‚वा क्ष‚णिकानाम‚पि क्र‚म‚यौ ‚{\tiny $_{3}$}‚ ग‚प‚द्याभ्याम‚र्थ‚क्रियाऽयोग इति । क्र‚माक्र‚माभ्याम‚र्थ‚{\tiny $_{lb}$}‚क्रियाऽयोगादित्य‚यं व्याप‚कानुप‚ल‚म्भः स‚त्त्वादित्य‚स्य हेतोर्विरुद्ध‚म‚स‚त्त्वं साध्य‚वि‚{\tiny $_{lb}$}‚प‚र्य‚ये प्र‚त्यु ‚{\tiny $_{4}$}‚ प‚स्थाप‚य‚द्वाध‚कं प्र‚माण‚मुच्य‚ते । एव‚ञ्च कृत‚क‚त्वादाव‚पि य‚थायोग्य‚{\tiny $_{lb}$}‚म्वाच्यं ।
	{\color{gray}{\rmlatinfont\textsuperscript{§~\theparCount}}}
	\pend% ending standard par
      ‚{\tiny $_{lb}$}‚

	  
	  \pstart \leavevmode% starting standard par
	\hphantom{.}क‚स्माद्विरुद्ध‚प्र‚त्युप‚स्थाप‚नाद‚स्य त‚द्वाध‚क‚म्प्र‚माण‚मुच्य‚त इत्याह । \quotelemma{एवं ‚{\tiny $_{5}$}‚ हि ‚{\tiny $_{lb}$}‚स} हेतुः स‚त्त्वादिल‚क्ष‚णः साध्याभावे त‚स्मिन्न‚स‚न्निति सिध्येत् य‚दि त‚त्र साध्याभावे ‚{\tiny $_{lb}$}‚बाध्य‚ते निराक्रिय‚ता केन स्व‚विरुद्धेन स्व‚रूप‚विरुद्धेनास‚त्त्वादि ‚{\tiny $_{6}$}‚ नेति याव‚त् । ‚{\tiny $_{lb}$}‚किम्भूतेन प्र‚माण‚व‚ता प्र‚माण‚युक्तेन । क‚स्मादेव‚म‚सौ त‚त्रास‚स्तिध्य‚तीत्याह । \quotelemma{अन्य‚था ‚{\tiny $_{lb}$}‚त‚त्र \cite[1b11]{vn-msN} साध्य‚विप‚र्य‚येऽस्य हेतोर्बाध‚क‚स्यासिद्धौ स‚त्त्यां संश‚यः} । सं ‚{\tiny $_{7}$}‚ श्च ‚{\tiny $_{lb}$}‚स्यान्नित्य‚श्चेत्यादि दुर्निवारः स्यादिति शेषः । दुःखेन निवार्य‚त इति दुर्निवारो ‚{\tiny $_{lb}$}‚दुर्न्निषेध इत्य‚र्थः । बाध‚क‚ग्र‚ह‚णेनात्र विरुद्ध‚स्य प्र‚त्युप‚स्थाप‚क‚म्प्र‚माणं ‚{\tiny $_{8}$}‚ गृह्य‚ते । ‚{\tiny $_{lb}$}‚वाध‚क‚प्र‚माण‚प्र‚त्युप‚स्थापित‚म्वा हेतुविरुद्धं । न‚नु चानुप‚ल‚ब्धिमात्रादेव साध्य‚विप‚र्य‚ये ‚{\tiny $_{lb}$}‚हेतोर्व्यावृत्तिनिश्च‚याद‚स‚न्दिग्धो व्य‚तिरेको भ‚विष्य‚ति । त‚क्तिमुच्य ‚{\tiny $_{9}$}‚ \leavevmode\ledsidenote{\textenglish{9a/msK}} से श‚ङ्काया ‚{\tiny $_{lb}$}‚व्यावृत्तिः । बाध‚क‚प्र‚माणानुप‚द‚र्श‚ने तु स एव न सिध्य‚ति त‚त्क‚थ‚मियं निव‚र्त्ते‚{\tiny $_{lb}$}‚तेति । य‚दि नामेय‚माश‚ङ्का न व्याव‚र्त्त‚ते । त‚तः किमित्याह । त‚तः आशंकाया ‚{\tiny $_{1}$}‚ ‚{\tiny $_{lb}$}‚ \leavevmode\ledsidenote{\textenglish{12/s}} अनिवृत्ते \quotelemma{र‚नेकान्तिकः स्याद्धेत्त्वाभासः} \cite[1b7]{vn-msN} । क‚स्माद्व्य‚तिरेक‚स्य‚स्या\edtext{}{\lemma{स्या}\Bfootnote{? सा}} ‚{\tiny $_{lb}$}‚ ध्याभावे हेतोर‚भाव‚ल‚क्ष‚ण‚स्य स‚न्देहात्कार‚णात्स‚न्दिग्ध‚विप‚क्ष‚व्यावृत्तिकः स्याद्धे‚{\tiny $_{lb}$}‚त्वाभास इत्य‚र्थः ।
	{\color{gray}{\rmlatinfont\textsuperscript{§~\theparCount}}}
	\pend% ending standard par
      ‚{\tiny $_{lb}$}‚

	  
	  \pstart \leavevmode% starting standard par
	किम्पुन ‚{\tiny $_{2}$}‚ र‚द‚र्श‚नेप्य‚निवृत्तिराशंकाया याव‚ता त‚द‚द‚र्श‚न‚म‚भावं साध‚य‚तीत्याह । ‚{\tiny $_{lb}$}‚ \quotelemma{नाप्य‚द‚र्श‚न‚मात्राद्व्यावृत्तिः} \cite[1b7]{vn-msN} साध्याभावे हेतोः सिध्य‚तीति वाक्याध्याहारः । ‚{\tiny $_{lb}$}‚अपिश‚ब्दो य ‚{\tiny $_{3}$}‚ स्माद‚र्थे । मात्र‚ग्र‚ह‚ण‚मुप‚ल‚ब्धिल‚क्ष‚ण‚प्राप्ताद‚र्श‚न‚स्य व्य‚व‚च्छेदार्थं । ‚{\tiny $_{lb}$}‚कुत एत‚त् । विप्र‚कृष्टेषु देश‚काल‚स्व‚भाव‚विप्र‚क‚र्षैः प‚दार्थेषु चीन‚दा ‚{\tiny $_{4}$}‚ श‚र‚थिपिशाच ‚{\tiny $_{lb}$}‚प्र‚भृतिषु य‚द‚द‚र्श‚नं त‚स्याभावासाध‚नात् । क‚स्याभावं साध‚य‚तीति चेत् । प्र‚कृत‚त्त्वा‚{\tiny $_{lb}$}‚द्विप्र‚कृष्टानामिति ग‚म्य‚ते ।
	{\color{gray}{\rmlatinfont\textsuperscript{§~\theparCount}}}
	\pend% ending standard par
      ‚{\tiny $_{lb}$}‚

	  
	  \pstart \leavevmode% starting standard par
	न‚नु स‚मासादि ‚{\tiny $_{5}$}‚ त‚स‚क‚ल‚प‚दार्थ‚व्यापि जानाति स य‚स्याद‚र्श‚न‚म‚भाव‚म्विप्र‚कृष्टाना‚{\tiny $_{lb}$}‚म‚पि साध‚य‚ति त‚त्क‚थ‚मिद‚मुक्त‚मित्याह । \quotelemma{अस‚र्व्व‚द‚र्शिन} \cite[1b7]{vn-msN} इति । स‚र्व्व‚न्द्र‚ष्टुं ‚{\tiny $_{lb}$}‚शी ‚{\tiny $_{6}$}‚ ल‚म‚स्य त‚तो न‚ञा स‚मासः । क‚स्मात्त‚स्याप्य‚द‚र्श‚न‚म‚भाव‚न्न साध‚य‚तीति । अर्व्वा‚{\tiny $_{lb}$}‚ग्द‚र्श‚नेन पुंसा स‚ताम‚पि केषाञ्चिद‚र्थानाम्विप्र‚कृष्टानाम‚द‚र्श‚नात् । इद‚मागूरितं । ‚{\tiny $_{7}$}‚ नेह ‚{\tiny $_{lb}$}‚स‚र्व्व‚द‚र्शिद‚र्श‚नं स‚म‚स्त‚व‚स्तुस‚त्तां प्राप्नोति । येन त‚न्निव‚र्त्त‚मान‚म‚र्थ‚स‚त्ताम्वृक्ष‚व‚च्छिं‚{\tiny $_{lb}$}‚स‚पां निव‚र्त्त‚येद् भेदात् । नापि त‚त्त‚स्याः कार‚णं येन व‚ह्निव‚द्धूमं निव‚र्त्त‚मानं नि ‚{\tiny $_{8}$}‚ व‚{\tiny $_{lb}$}‚र्त्त‚येत् । त‚द‚भावेपि भावादिति । बाध‚कं पुनः प्र‚माण‚मित्यादि । अत्र केचिदेवं ‚{\tiny $_{lb}$}‚पूर्व्व‚प‚क्ष‚य‚न्ति । किम्पुन‚र्बाध‚कं प्र‚माणं य‚स्योप‚द‚र्श‚नेन मौल‚स्य हेतोर्व्याप्तिप्र ‚{\tiny $_{9}$}‚ \leavevmode\ledsidenote{\textenglish{9b/msK}} ती‚{\tiny $_{lb}$}‚तिर्भ‚व‚तीत्याह । \quotelemma{बाध‚क} म्पुन \cite[1b7]{vn-msN} रित्यादि । तेषाङ्क‚थ‚म‚त्र व्याप्तिसाध‚न‚म्विप‚र्य‚ये ‚{\tiny $_{lb}$}‚बाध‚क‚प्र‚माणोप‚द‚र्श‚नं य‚दि न स‚र्व्वं स‚त्कृत‚कं वा प्र‚तिक्ष‚ण‚म्विनाशि स्यादित्यादि‚{\tiny $_{lb}$}‚नाऽत्रैव ‚{\tiny $_{1}$}‚ प्राग‚र्थ‚स्याभिहित‚त्त्वात्पुन‚रुक्त‚दोष‚प्र‚स‚क्तिर्न भ‚व‚तीति चिन्त्य‚मेत‚त् ‚{\tiny $_{lb}$}‚तैरेवेत्य‚लं प‚र‚दोष‚संकीर्त्त‚नेन । त‚स्माद‚न्य‚था पूर्व्व‚प‚क्ष्य‚ते । य‚स्यापि त‚र्हि बाध‚क‚म्प्र‚{\tiny $_{lb}$}‚माण‚म‚स्ति त ‚{\tiny $_{2}$}‚ स्य क‚थ‚म‚य‚म‚दोष इत्य‚त आह । \quotelemma{बाध‚कं पुनः प्र‚माणं} \cite[1b7]{vn-msN} प्र‚व‚र्त्त‚{\tiny $_{lb}$}‚मान‚म‚साम‚र्थ्य‚माक‚र्ष‚तीति क्रियाप‚दं । कीदृश‚म‚स‚ल्ल‚क्ष‚णं । क‚थं प्र‚माणं य‚स्य प‚दार्थ‚स्य ‚{\tiny $_{lb}$}‚क्र‚म‚यौग ‚{\tiny $_{3}$}‚ प‚द्यायोगः । अर्थ‚क्रियाया इत्य‚ध्याहार्यं । न त‚स्य क्व‚चित्कार्ये साम‚र्थ्यं ‚{\tiny $_{lb}$}‚य‚था न‚भ‚स्त‚लार‚विन्द‚स्येत्य‚ध्याहार्यो दृष्टान्तः । अस्ति चाक्ष‚णिके भावे स क्र ‚{\tiny $_{4}$}‚ म‚{\tiny $_{lb}$}‚यौग‚प‚द्याभ्याम‚र्थ‚क्रियाया अयोग इत्येव‚म्प्र‚व‚र्त्त‚मानं । त‚तः किञ्जात‚मिति चेदाह । ‚{\tiny $_{lb}$}‚ \quotelemma{तेन} \cite[1b8]{vn-msN} कार‚णेन येन त‚त्प्र‚व‚र्त्त‚मान‚म‚स‚ल्ल‚क्ष‚ण‚म‚साम ‚{\tiny $_{5}$}‚ र्थ्य‚माक‚र्ष‚य‚ति । य‚त्स‚{\tiny $_{lb}$}‚त्कृत‚क‚म्वा त‚द‚नित्य‚मेवेति सिध्य‚ति ।
	{\color{gray}{\rmlatinfont\textsuperscript{§~\theparCount}}}
	\pend% ending standard par
      ‚{\tiny $_{lb}$}‚

	  
	  \pstart \leavevmode% starting standard par
	\hphantom{.}एव‚म‚पि किं सिद्ध‚म्भ‚व‚तीत्याह । \quotelemma{ताव‚ता} च वाध‚क‚प्र‚माणोप‚द‚र्श‚न‚मात्रेण ‚{\tiny $_{6}$}‚ \quotelemma{साध‚न-} ‚{\tiny $_{lb}$}‚ \leavevmode\ledsidenote{\textenglish{13/s}} \quotelemma{ध‚र्म‚मात्रान्व‚यः} \cite[1b8]{vn-msN} सिध्य‚तीति व‚र्त्त‚ते । केनेत्याह । साध्य‚ध‚र्म‚स्य क‚र्त्त‚रि चेयं ‚{\tiny $_{lb}$}‚ष‚ष्ठी प्र‚तिप‚त्त‚व्या । तेन साध्य‚ध‚र्मेण साध‚न‚ध‚र्म‚मात्र‚स्यान‚पेक्षित‚हेत्व‚न्त‚र‚व्यापा ‚{\tiny $_{7}$}‚ र‚स्या‚{\tiny $_{lb}$}‚न्व‚यः सिध्य‚तीति वाक्यार्थः स‚न्तिष्ठ‚ते [।] न‚त्वेव‚ङ्क‚र‚णीयं साध‚न‚ध‚र्म‚मात्रेणान्व‚यः ‚{\tiny $_{lb}$}‚साध्य‚ध‚र्मेति । एवं हि साध्य‚मेव हेतुना ऽविनाभूतं जातं न हेतुरिति हेतो ‚{\tiny $_{8}$}‚ र‚ग‚म‚{\tiny $_{lb}$}‚क‚त्त्व‚म्भ‚वेत् । त‚त‚श्च को गुणो ल‚भ्य‚त इत्याह । \quotelemma{स्व‚भाव‚हेतुल‚क्ष‚ण‚ञ्च सिद्ध‚म्भ‚{\tiny $_{lb}$}‚व‚ति} \cite[1b9]{vn-msN} ताव‚ता चेति व‚र्त्त‚ते । स्व‚भाव‚हेतुल‚क्ष‚ण‚ञ्च त‚द्भाव‚मात्राद्ध‚र्मिनिस्व‚भा‚{\tiny $_{lb}$}‚वो ‚{\tiny $_{9}$}‚ \leavevmode\ledsidenote{\textenglish{10a/msK}} पि प्र‚त्य‚याभावेऽपुन‚र्य‚त्नापेक्षित‚त्त्वात् । क‚ल‚धौत‚म‚ल‚विशुद्धिव‚दित्येव‚माद‚यो हेत‚वः ‚{\tiny $_{lb}$}‚प्र‚तिल‚ब्ध‚साम‚र्थ्यातिश‚याः स‚न्त्येव ते च य‚द्य‚स‚म‚र्थिता एव ज्ञाप्य‚स‚म‚र्थं ज्ञाप ‚{\tiny $_{1}$}‚ येयुस्त‚दा ‚{\tiny $_{lb}$}‚जैमिनिप्र‚भृतीनां विवादाभाव एव भ‚वेत् । न‚न्वेत‚देव न स‚म्भाव्य‚ते । य‚त्प‚र‚मार्थ‚तः ‚{\tiny $_{lb}$}‚स‚म‚र्थ‚स्यापि हेतोर‚भिधाने निग्र‚हार्होऽसावित्याश‚ङ्कायां स्वाभिप्रा ‚{\tiny $_{2}$}‚ यं प्र‚क‚ट‚य‚ति । ‚{\tiny $_{lb}$}‚व‚स्तुतः स‚म‚र्थ‚स्य हेतोरुपादानेपि साम‚र्थ्य‚प्र‚तिपाद‚नादिति । अय‚म‚स्य भावो य‚दि ‚{\tiny $_{lb}$}‚नामानेन प‚र‚मार्थ‚तः स‚म‚र्थो हेतुरुपात्त‚स्त‚थापि त‚स्य सा ‚{\tiny $_{3}$}‚ म‚र्थ्यं \quotelemma{साध‚नाङ्गास‚म‚र्थ‚नान्न} ‚{\tiny $_{lb}$}‚प्र‚तिपादित‚म‚नेनेति अस‚म‚र्थ‚क‚ल्प एवासौ । न ह्य‚र्थ‚स्य प‚रार्थानुमाने गुण‚दोषा‚{\tiny $_{lb}$}‚व‚धिक्रियेते । किन्त‚र्हि व‚च‚न‚स्य व‚क्तुर‚य ‚{\tiny $_{4}$}‚ थार्थाभिधानेनोपाल‚म्भात् । अत एव ‚{\tiny $_{lb}$}‚य‚त्राप्य‚र्थ‚स्य गुण‚दोषाव‚धिक्रियेते त‚त्रापि व‚च‚न‚द्वारेणैव । एव‚मेत‚द‚भ्युप‚ग‚न्त‚व्य‚{\tiny $_{lb}$}‚म‚न्य‚था क्ष‚णिकः श‚ब्द इ ‚{\tiny $_{5}$}‚ त्येताव‚न्मात्र‚मेव प्र‚तिज्ञाव‚च‚न‚म‚भिधाय स्थात‚व्यं । त‚था‚{\tiny $_{lb}$}‚भिधानादेवाभिम‚तार्थ‚सिद्धेरिति ॥ ० ॥
	{\color{gray}{\rmlatinfont\textsuperscript{§~\theparCount}}}
	\pend% ending standard par
      ‚{\tiny $_{lb}$}‚

	  
	  \pstart \leavevmode% starting standard par
	एवं स्व‚भाव‚हेतावुप‚द‚र्श्य साध‚नाङ्ग‚स‚म‚र्थ‚न‚मिदा ‚{\tiny $_{6}$}‚ नीङ्कार्य‚हेतावाह [।] ‚{\tiny $_{lb}$}‚ \quotelemma{कार्य‚हेताव‚पी} \cite[2a2]{vn-msN} त्यादिना । किम्पुन‚स्त‚दित्याह [।] \quotelemma{य‚त्कार्यं लिङ्गं} धूमादि‚{\tiny $_{lb}$}‚संज्ञ‚कं \quotelemma{कार‚ण‚स्य} द‚ह‚नादेः \quotelemma{साध‚नायोपादीय‚ते} \cite[2a3]{vn-msN} । त‚स्य धूमादेस्तेन द‚ह‚ना‚{\tiny $_{lb}$}‚दिना ‚{\tiny $_{7}$}‚ स‚ह कार्य‚कार‚ण‚भाव‚प्र‚साध‚न‚लिङ्गिलिङ्ग‚योर्हेतुफ‚ल‚भाव‚साध‚न‚मेव य‚त्त‚देव ‚{\tiny $_{lb}$}‚कार्य‚हेतौ साध‚नाङ्ग‚स‚म‚र्थ‚न‚मित्य‚र्थः । केन पुन‚स्त‚योः कार्य‚कार‚ण‚भावः प्र‚सा ‚{\tiny $_{8}$}‚ ‚{\tiny $_{lb}$}‚ध्य‚त इत्याह । \quotelemma{भावाभाव‚साध‚न‚प्र‚माणाभ्या} \cite[2a3]{vn-msN} मिति । भावाभावौ कार्य‚{\tiny $_{lb}$}‚कार‚ण‚योः स‚द‚स‚त्ते त‚योः साध‚ने ते च ते प्र‚माणे चेति व्यु ‚{\tiny $_{9}$}‚ \leavevmode\ledsidenote{\textenglish{10b/msK}} त्प‚त्तिक्र‚मः [।] साध‚न- ‚{\tiny $_{lb}$}‚श‚ब्द‚श्च क‚र‚ण‚साध‚नः भावाभाव‚साध‚न‚प्र‚माणे च प्र‚त्य‚क्षानुप‚ल‚म्भो य‚थाक्र‚मं [।] ‚{\tiny $_{lb}$}‚क‚थ‚म्पुनः प्र‚त्य‚क्षानुप‚ल‚म्भाभ्यां कार्य‚कार‚ण‚भावः प्र‚साध्य‚त इत्याह ‚{\tiny $_{1}$}‚ [।] य‚थे‚{\tiny $_{lb}$}‚ \leavevmode\ledsidenote{\textenglish{14/s}} \cite[2a3]{vn-msN} त्यादि । इद‚न्धूमादिसंज्ञ‚कं कार्य‚म‚स्मिन्द‚ह‚ने स‚ति भ‚व‚ति । उप‚ल‚ब्धिल‚क्ष‚ण‚{\tiny $_{lb}$}‚प्राप्तं स‚द‚नुप‚ल‚ब्धं प्रागिति वाक्य‚शेषः कार्यः । अन्य‚था द‚ह‚न‚स्य त‚त्र धूमे व्यापार ‚{\tiny $_{lb}$}‚एव ‚{\tiny $_{2}$}‚ न क‚थितः स्यात् द‚ह‚न‚स‚न्निधानात्प्राग‚प्येत‚दासीदित्याश‚ङ्कास‚म्भ‚वात् । य‚थो‚{\tiny $_{lb}$}‚क्तानुप‚ल‚म्भ‚ग्र‚ह‚णे तु नैवेय‚म‚व‚त‚र‚ति । अत एव च \quotelemma{प्र‚माण‚विनिश्चि\edtext{}{\lemma{विनिश्चि}\Bfootnote{? श्च}}या} दा‚{\tiny $_{lb}$}‚व‚प्येव‚मेवा ‚{\tiny $_{3}$}‚ भिहित‚मिति । अनेन च प्र‚त्य‚क्ष‚प्र‚माण‚व्यापार उक्तः । त‚थाहि अस्मि‚{\tiny $_{lb}$}‚न्स‚तीदं भ‚व‚तीति प्र‚त्य‚क्षेणैत‚द् ग‚म्य‚ते । स‚म्प्र‚त्य‚नुप‚ल‚म्भ‚स्य व्यापारं निर्दिदिक्षु ‚{\tiny $_{4}$}‚ ‚{\tiny $_{lb}$}‚राह । \quotelemma{स‚त्स्व‚पी} \cite[2a3]{vn-msN} त्यादि । त‚स्माद्द‚ह‚नाद‚न्येषु विद्य‚मानेष्व‚पि । त‚था त‚स्य ‚{\tiny $_{lb}$}‚धूम‚स्य हेतुष्विन्ध‚नानिलादिषु स‚म‚र्थेषु स‚त्स्व‚पीति व‚र्त्त‚ते । च‚कार‚श्चात्र लुप्त ‚{\tiny $_{5}$}‚ ‚{\tiny $_{lb}$}‚निर्दिष्टः प्र‚तिप‚त्त‚व्यः स‚म‚र्थेषु चेति । त‚स्य द‚ह‚न‚स्याभावे न भ‚व‚ति । इद‚मित्य‚धि‚{\tiny $_{lb}$}‚कृतं । इति एव‚मित्य‚स्यार्थे व‚र्त्त‚ते व्य‚व‚च्छेदे वा । त‚द‚नेन ग‚वाश्वादी ‚{\tiny $_{6}$}‚ नां ‚{\tiny $_{lb}$}‚त‚द्देश‚काल‚स[न्]निहितानाम‚पि धूम‚ज‚न‚नं प्र‚ति कार‚ण‚त्त्व‚न्निसि\edtext{}{\lemma{न्निसि}\Bfootnote{? षि}}द्धं । य‚तो ‚{\tiny $_{lb}$}‚य‚दि ते ग‚वाश्वाद‚य‚स्त‚स्य कार‚ण‚म्भ‚वेयुस्त‚दा व्य‚तीतेप्य‚ग्नौ तेषां स‚न्निहित‚त्त्वाद् ‚{\tiny $_{lb}$}‚धूमो ‚{\tiny $_{7}$}‚ त्प‚त्तिप्र‚स‚ङ्गः । इन्ध‚नादिकार‚णान्त‚रापेक्षास्ते त‚स्य ज‚न‚का भ‚व‚न्ति त‚तोय‚म‚{\tiny $_{lb}$}‚प्र‚स‚ङ्ग इति चेत् । य‚द्येवं तेपि त‚त्र स‚न्निहिता एवेति न व्याव‚र्त्त‚ते प्र‚स‚ङ्ग इति ‚{\tiny $_{lb}$}‚द ‚{\tiny $_{8}$}‚ ह‚न‚म‚पि स‚ह‚कारिण‚म‚पेक्ष्य तं ज‚न‚य‚न्ति त‚तो न युक्त‚मिद‚मिति चेत् । न‚न्वेवं ‚{\tiny $_{lb}$}‚स‚त्यायात‚न्द‚ह‚न‚स्य धूमोर्त्प‚त्तिं प्र‚ति कार‚ण‚त्त्वं । त‚क्तिमिद‚मुच्य‚ते पूर्व्वाप‚र‚व्याह‚{\tiny $_{lb}$}‚त‚ङ्ग ‚{\tiny $_{9}$}‚ \leavevmode\ledsidenote{\textenglish{11a/msK}} वाश्वाद‚य एव त‚स्य कार‚ण‚न्न द‚ह‚न इति । अस्तु त‚र्हि त‚स्यापि द‚ह‚न‚स्य ‚{\tiny $_{lb}$}‚कार‚ण‚त्त्व‚ङ्ग‚वाश्वादीनाम‚पीति चेत् । न युक्त‚मेत‚त् । त‚त्र व्य‚तिरेक‚ग‚तेर्दुर्घ‚ट‚त्वात् । ‚{\tiny $_{lb}$}‚त‚था ह्य‚प ‚{\tiny $_{1}$}‚ ग‚तेष्व‚पि ग‚वादिषु स‚ति च द‚ह‚ने त‚त्रेन्ध‚नादिक‚लापे भ‚व‚त्येव हुत‚भुग्धे‚{\tiny $_{lb}$}‚तोरुत्प‚त्तिः । य‚तः कार‚ण‚प्र‚ब‚न्ध‚ङ्कार्य‚प्र‚ब‚न्ध‚ञ्चाश्रित्य हेतुफ‚ल‚भाव‚श्चिन्त्य‚ते ‚{\tiny $_{lb}$}‚भावानाङ्का ‚{\tiny $_{2}$}‚ र‚ण‚प्र‚ब‚न्ध‚पूर्व्वः कार्य‚प्र‚ब‚न्ध इति । त‚त्तु क्ष‚ण‚भेदं । न‚हि स‚मासादित‚{\tiny $_{lb}$}‚ज्ञानातिश‚यानाम‚यं पूर्व्वः क्ष‚णोऽय‚मुत्त‚र इति विशेषाव‚ल‚म्बि ज्ञान‚मुदेति [।] ‚{\tiny $_{lb}$}‚अर्व्वाग्द‚र्शि ‚{\tiny $_{3}$}‚ भिश्चाधिकृत्य प्र‚माण‚ल‚क्ष‚णं प्र‚ण‚यितं कृपाव‚द्भिः । य‚थोक्तं ‚{\tiny $_{lb}$}‚सांव्य‚व‚हारिक‚स्यैत‚त् प्र‚माण‚स्य रूप‚मुक्त‚म‚त्रापि प‚रे विमूढा विस‚म्वाद‚य‚न्ति लो ‚{\tiny $_{4}$}‚ क‚{\tiny $_{lb}$}‚मिति । वास‚गृहादिषु त‚र्हि द‚ह‚नाभावेपि धूम‚स‚द्भावाद्व्य‚भिचार इति चेत् । ‚{\tiny $_{lb}$}‚भूत‚स्यापि द‚ह‚न‚प्र‚ब‚न्ध‚पूर्व्व‚क‚त्त्व‚म‚स्त्येव । साक्षात्पार‚म्प ‚{\tiny $_{5}$}‚ र्य‚कृत‚स्तु विशेषः । ‚{\tiny $_{lb}$}‚अव‚य‚न्ति च विच्छिन्नाविच्छिन्न‚द‚र्श‚न‚प्र‚ब‚न्ध‚योर्द्धूम‚प्र‚ब‚न्ध‚योर्वास‚गृहादिर‚स‚व‚तीप्र‚दे‚{\tiny $_{lb}$}‚शादिभाविनोः स्फुट‚मेव भेद‚म्विचित्र ‚{\tiny $_{6}$}‚ भाव‚स्व‚भाव‚विवेकाभ्यास‚ब‚लोप‚जात‚विद‚ग्ध‚{\tiny $_{lb}$}‚ \leavevmode\ledsidenote{\textenglish{15/s}} बुद्ध‚य इति भेदेनाप्य‚नुमान‚म‚विरुद्ध‚म‚त एव देश‚कालाद्य‚पेक्ष‚म‚नुमानं कार्य‚हेतौ ‚{\tiny $_{lb}$}‚विरुद्ध‚कार्योप‚ल‚म्भे चोक्तं ।‚{\tiny $_{7}$}‚ ‚{\tiny $_{lb}$}‚ 
	    \pend% close preceding par
	  
	    
	    \stanza[\smallbreak]
	  \flagstanza{\tiny\textenglish{...2}}{\normalfontlatin\large ``\qquad}इष्ट‚म्विरुद्ध‚कार्येपि देश‚कालाद्य‚पेक्ष‚णं ।&‚{\tiny $_{lb}$}‚अन्य‚था व्य‚भिचारि स्याद् ग‚त्येवासीत \edtext{}{\lemma{त्येवासीत}\Bfootnote{?}} साध‚नं [॥ २]{\normalfontlatin\large\qquad{}"}\&[\smallbreak]
	  
	  
	  
	    \pstart  \leavevmode% new par for following
	    \hphantom{.}
	   इति ।
	{\color{gray}{\rmlatinfont\textsuperscript{§~\theparCount}}}
	\pend% ending standard par
      ‚{\tiny $_{lb}$}‚

	  
	  \pstart \leavevmode% starting standard par
	अथापि क‚श्चिद्विविध‚भाव‚भेद‚प्र‚विच‚य‚चातुर्यातिश‚य‚श‚लाकोन्मी ‚{\tiny $_{8}$}‚ लित‚प्र‚ज्ञाच‚क्षु‚{\tiny $_{lb}$}‚ष्ट्वाद‚यं ज्व‚ल‚न‚ज‚नितो धूमोऽयं धूम‚ज‚नित इति विवेच‚य‚ति । त‚थापि न सुत‚रां ‚{\tiny $_{lb}$}‚व्य‚भिचारः । त‚थाहि नाग्निज‚न्यो धूमो धूमाद् भ‚व‚ति निर्हेतुक‚त्त्व‚प्र‚स‚ङ्गात् । ‚{\tiny $_{9}$}‚ \leavevmode\ledsidenote{\textenglish{11b/msK}} ‚{\tiny $_{lb}$}‚त‚था च य‚दुक्तं [।] ‚{\tiny $_{lb}$}‚ 
	    \pend% close preceding par
	  
	    
	    \stanza[\smallbreak]
	  \flagstanza{\tiny\textenglish{...3}}{\normalfontlatin\large ``\qquad}अत‚श्चान‚ग्नितो धूमो य‚दि धूम‚स्य स‚म्भ‚वः ।&‚{\tiny $_{lb}$}‚श‚क्र‚मूर्ध्न‚स्त‚थास्त\edtext{}{\lemma{थास्त}\Bfootnote{? त}}स्य केन वार्येत स‚म्भ‚वः [॥ ३]{\normalfontlatin\large\qquad{}"}\&[\smallbreak]
	  
	  
	  
	    \pstart  \leavevmode% new par for following
	    \hphantom{.}
	   इत्यादि
	{\color{gray}{\rmlatinfont\textsuperscript{§~\theparCount}}}
	\pend% ending standard par
      ‚{\tiny $_{lb}$}‚

	  
	  \pstart \leavevmode% starting standard par
	त‚द‚सार‚मित्य‚प्युपेक्ष‚ते [।] त‚स्मान्न तेषां ग‚वाश्वादीनां त‚त्र ‚{\tiny $_{1}$}‚ कार‚ण‚त्त्व‚म‚स्तीति ‚{\tiny $_{lb}$}‚निश्च‚यः स‚माधीय‚तां । अत‚श्च द‚ह‚न एव त‚स्य कार‚णं नाश्वाद‚य इति स्थित‚मेत‚त् । ‚{\tiny $_{lb}$}‚त‚था च द‚ह‚न‚स्य कार‚ण‚त्त्वं योजित‚म‚न्व‚य‚व्य‚तिरेकाभ्यां य‚थो ‚{\tiny $_{2}$}‚ क्त‚प्र‚काराभ्यामेव‚{\tiny $_{lb}$}‚मिंध‚नादिसाम‚ग्र्याः स‚र्व‚स्याः कार‚णं योज‚यित‚व्यं । य‚दि वैक‚वाक्य‚त‚यैव व्याख्या‚{\tiny $_{lb}$}‚य‚ते । स‚त्स्व‚पि त‚स्माद्द‚ह‚नाद‚न्येषु स‚म‚र्थेषु त‚द्धेतुष्विं ‚{\tiny $_{3}$}‚ ध‚नादिष्व‚स्याभावे न भ‚व‚{\tiny $_{lb}$}‚तीति । ग‚वाश्वादीनां त्व‚हेतुत्त्व‚म्व्य‚तिरेकाभाव‚त‚या य‚थोक्तेन विधिना बोद्ध‚व्यं ।
	{\color{gray}{\rmlatinfont\textsuperscript{§~\theparCount}}}
	\pend% ending standard par
      ‚{\tiny $_{lb}$}‚

	  
	  \pstart \leavevmode% starting standard par
	न‚नु चैत‚देव युक्त‚म्व‚क्तुं त‚द‚भावेन भ‚व‚ती ‚{\tiny $_{4}$}‚ त्य‚थ किम‚र्थं स‚त्स्व‚पि त‚द‚न्येषु ‚{\tiny $_{lb}$}‚स‚म‚र्थेषु त‚द्धेतुष्वित्युच्य‚त इति चेदाह । \quotelemma{एवं ही} \cite[2a4]{vn-msN} त्यादि । य‚स्मादेवं स‚त्स्व‚{\tiny $_{lb}$}‚पीत्यादिनाऽभिधीय‚मानोऽस्य धूम‚स्य त‚त्कार्य ‚{\tiny $_{5}$}‚ त्व‚म‚ग्निकार्य‚त्त्वं स‚म‚र्थितं निश्चि‚{\tiny $_{lb}$}‚त‚म‚स‚न्दिग्ध‚म्भ‚व‚ति । अन्य‚था य‚द्येवं नोप‚द‚र्श्य‚ते । केव‚लं त‚द‚भावेन भ‚व‚तीत्युप‚{\tiny $_{lb}$}‚द‚र्श्य‚ते त‚दा त‚द‚भावेन भ‚व‚तीत्युप ‚{\tiny $_{6}$}‚ द‚र्श‚ने क्रिय‚माणेन्य‚स्यापि तु ग‚वाश्वादेरिन्ध‚{\tiny $_{lb}$}‚नादेश्च त‚त्राग्निशून्य‚भूभागेऽभावे स‚ति स‚न्दिग्ध‚म‚स्याग्नेः साम‚र्थ्य‚म्भ‚व‚तीति कुतः ‚{\tiny $_{lb}$}‚कार‚ण‚भाव‚निश्च‚य इति स‚मुदाया ‚{\tiny $_{7}$}‚ र्थः । त‚त एवाह [।] \quotelemma{स‚त्सु हि स‚म‚र्थेषु त‚द्धे‚{\tiny $_{lb}$}‚तुषु \cite[2a3]{vn-msN} कार्यानुत्प‚त्तिः कार‚णान्त‚र‚विक‚ल्पं सूच‚य‚ती} ति स‚न्दिग्ध‚म‚स्यान्य‚था ‚{\tiny $_{lb}$}‚साम‚र्थ्य‚मित्येत‚देवान्य \quotelemma{त्त‚त्रेत्या} \cite[2a4]{vn-msN} दिना सूच‚य‚ति । त‚त्र धू ‚{\tiny $_{8}$}‚ माश्व‚कार्येऽन्य‚{\tiny $_{lb}$}‚देवाश्वादि । य‚दिन्ध‚नादिस‚म‚र्थ‚न्त‚द‚भावात्त‚न्न भूत‚न्द‚ह‚न‚शून्य‚देशे । अस्य स्व‚भावा‚{\tiny $_{lb}$}‚त्त‚न्न जात‚मिति कुतोयं निश्च‚य इत्य‚र्थः ।
	{\color{gray}{\rmlatinfont\textsuperscript{§~\theparCount}}}
	\pend% ending standard par
      ‚{\tiny $_{lb}$}‚\textsuperscript{\textenglish{16/s}}

	  
	  \pstart \leavevmode% starting standard par
	 \leavevmode\ledsidenote{\textenglish{12a/msK}}य‚द्य‚न्य‚त्त‚त्र स‚म‚र्थ‚न्त‚द भावात्त‚न्न जात‚मेत‚न्निवृत्तौ निवृत्तिस्त‚र्ह्य‚स्य क‚थ‚मिति ‚{\tiny $_{lb}$}‚चेदाह [।] \quotelemma{एत‚न्निवृत्तावित्या \cite[2a5]{vn-msN} दि} । एत‚स्याग्नेर्न्निवृत्तौ धूम‚निवृत्तिर्येयं ‚{\tiny $_{lb}$}‚धूम‚स्य सा य‚दृच्छास‚म्वादः । \quotelemma{काक‚ता ‚{\tiny $_{1}$}‚ ली} य‚न्यायेनेत्य‚र्थः । य‚दा तु स‚त्स्व‚पीति क्रिय‚ते ‚{\tiny $_{lb}$}‚त‚दा स‚र्व्वेषां त‚त्र स‚न्निपातादेत‚स्यैव निवृत्ताव‚स्य निवृत्तिरिति निश्च‚यान्न य‚दृच्छा‚{\tiny $_{lb}$}‚स‚म्वाद इत्य‚भिप्रायः ।
	{\color{gray}{\rmlatinfont\textsuperscript{§~\theparCount}}}
	\pend% ending standard par
      ‚{\tiny $_{lb}$}‚

	  
	  \pstart \leavevmode% starting standard par
	किम्व‚दे ‚{\tiny $_{2}$}‚ त‚न्निवृत्तौ निवृत्तिर्य‚दृच्छास‚म्वाद इत्याह [।] \quotelemma{मातृविवाह} \cite[2a5]{vn-msN} ‚{\tiny $_{lb}$}‚इत्यादि । मातुर्विवाह उचित आच‚रितो य‚स्मिन्देशे स त‚था । त‚तो देश‚{\tiny $_{lb}$}‚श‚ब्देन स‚ह विशेष‚ण‚स‚मास ‚{\tiny $_{3}$}‚ [ः ।] त‚त्र स च ज‚न्माश्र‚य‚त्त्वादुप‚चारात् । ज‚न्म ‚{\tiny $_{lb}$}‚उत्प‚त्तिर्य‚स्य त‚स्य \quotelemma{पार‚सीक} देश‚भावि न याव‚त् । देशान्त‚रे \quotelemma{माल‚व‚का} दिदेशे य‚था‚{\tiny $_{lb}$}‚ऽभावो मातृविवा ‚{\tiny $_{4}$}‚ हाभावे य‚दृच्छास‚म्वाद‚स्त‚द्व‚द‚त्रापि । त‚थाहि मृद्विशेषाभावा‚{\tiny $_{lb}$}‚द्देशान्त‚रे त‚स्याभावो न तु मातृविवाहाभावादिति काक‚तालीय‚स्त‚द‚भावे त‚स्या‚{\tiny $_{lb}$}‚भा ‚{\tiny $_{2}$}‚ व इति । एव‚ञ्चैत‚त् ।
	{\color{gray}{\rmlatinfont\textsuperscript{§~\theparCount}}}
	\pend% ending standard par
      ‚{\tiny $_{lb}$}‚

	  
	  \pstart \leavevmode% starting standard par
	अथ‚वा अन्य‚था व्याख्याय‚ते । य‚थेदं धूमादिकार्य‚म‚स्मिन्न‚ग्नीन्ध‚नादिकार‚ण‚{\tiny $_{lb}$}‚क‚लापे स‚ति भ‚व‚ति । वाक्याध्याहार‚स्तु पूर्व्व ‚{\tiny $_{6}$}‚ व‚त्कार्यः । त‚स्य प्र‚योज‚नं त‚देवाव‚{\tiny $_{lb}$}‚ग‚न्त‚व्यं । इद‚म्प्र‚त्य‚क्ष‚व्यापार‚स‚ङ्कीर्त्त‚नं । स‚त्स्व‚पीत्यादिनाऽनुप‚ल‚म्भ‚स्य त‚द‚न्येषु ‚{\tiny $_{lb}$}‚पुन‚स्त‚स्माद‚ग्न्यादिकार‚ण‚क‚लापात् । अन्ये ‚{\tiny $_{7}$}‚ षु ग‚वाश्वादिषु स‚म‚र्थेषु त‚द्धेतुष्व‚स्या‚{\tiny $_{lb}$}‚ग्न्यादिकार‚ण‚क‚लाप‚स्याभावे न भ‚व‚ति । एत‚च्च प‚र‚म‚तापेक्ष‚मुक्तं । न तु तेषान्त‚द्धे‚{\tiny $_{lb}$}‚तुत्त्व‚म‚स्ति । य‚दि पुन‚स्ते त‚स्य हेत‚वः स्यु ‚{\tiny $_{8}$}‚ स्त‚दा त‚त्क‚लाप‚स‚न्निधेः प्राग‚पि प‚श्चा‚{\tiny $_{lb}$}‚दिव धूमोत्पाद‚प्र‚स‚ङ्गः । त‚त्सापेक्ष‚त‚या त‚त्कृत‚क‚त्त्वं तेषामिति चेत् । आयातं त‚र्हि ‚{\tiny $_{lb}$}‚त‚स्य क‚लाप‚स्य कार‚ण‚त्त्वं ।
	{\color{gray}{\rmlatinfont\textsuperscript{§~\theparCount}}}
	\pend% ending standard par
      ‚{\tiny $_{lb}$}‚

	  
	  \pstart \leavevmode% starting standard par
	भ‚व‚तु ‚{\tiny $_{9}$}‚\leavevmode\ledsidenote{\textenglish{12b/msK}} त‚र्ह्युभ‚योर‚पि न नः काचित् क्ष‚तिरिति चेत् । न । व्य‚तिरेक‚ग‚त‚स्त‚त्र ‚{\tiny $_{lb}$}‚दुर्घ‚ट‚त्वादित्युक्तं । य‚थाऽप‚ग‚तेष्व‚पि स‚र्व्वेषु तेषु त‚स्मिं क‚लापे स‚ति भ‚व‚त्येव त‚स्य ‚{\tiny $_{lb}$}‚स‚म्भ‚व इति त‚द ‚{\tiny $_{1}$}‚ भावे न भ‚व‚तीति वाच्यं । त‚त्किम‚र्थं स‚त्स्व‚पीत्याद्युक्त‚मिति ‚{\tiny $_{lb}$}‚चेदाह । \quotelemma{एवं ही \cite[2a4]{vn-msN} त्यादि} । अन्य‚था त‚स्य क‚लाप‚स्याभावे न भ‚व‚तीत्युप‚द‚र्श‚ने ‚{\tiny $_{lb}$}‚त‚स्यापि ग‚वाश्वादेस्त‚त्राभावे स‚ति ‚{\tiny $_{2}$}‚ स‚न्दिग्ध‚म‚स्य क‚लाप‚स्य साम‚र्थ्य‚म्भ‚वेत् । ‚{\tiny $_{lb}$}‚य‚तोऽन्य‚द् ग‚वाश्वादि त‚त्र श‚क्तं त‚द‚भावात्त‚न्न भूत‚मेत‚स्य क‚लाप‚स्य निवृत्तौ निवृ‚{\tiny $_{lb}$}‚त्तिर्य‚दृच्छास‚म्वादः शेषं पूर्व्व‚य‚त् ।
	{\color{gray}{\rmlatinfont\textsuperscript{§~\theparCount}}}
	\pend% ending standard par
      ‚{\tiny $_{lb}$}‚\textsuperscript{\textenglish{17/s}}

	  
	  \pstart \leavevmode% starting standard par
	एवाङ्कार्य ‚{\tiny $_{3}$}‚ कार‚ण‚भाव‚निश्च‚योपाय‚विधिमुक्त्वा प्र‚कृत‚मुप‚संह‚र‚ति । \quotelemma{एव‚{\tiny $_{lb}$}‚मि} \cite[2a5]{vn-msN} त्यादिना । एवं य‚थोक्तेन विधिना त‚द्धूमादि त‚स्य व‚ह्न्यादेः \quotelemma{कार्यं ‚{\tiny $_{lb}$}‚स‚म‚र्थितं} निश्चितं ‚{\tiny $_{4}$}‚ सिध्य‚ति भ‚व‚ति । य‚देवास‚म‚र्थित‚म‚स‚न्दिग्धं सिध्य‚ति निश्ची‚{\tiny $_{lb}$}‚य‚ते । अथ‚वा एवं प्र‚त्य‚क्षानुप‚ल‚म्भाभ्यां स‚म र्थित\edtext{}{\lemma{र्थित}\Bfootnote{? तं}}स‚त्त‚त्त‚स्य कार्यं सिध्य‚ति ।
	{\color{gray}{\rmlatinfont\textsuperscript{§~\theparCount}}}
	\pend% ending standard par
      ‚{\tiny $_{lb}$}‚

	  
	  \pstart \leavevmode% starting standard par
	य‚दि नाम सिध्य‚ति त ‚{\tiny $_{5}$}‚ त् किभित्याह । \quotelemma{सिद्धं स‚त् स्व‚स‚म्भ‚वेन} आत्म‚स‚न्निधानेन ‚{\tiny $_{lb}$}‚ \quotelemma{त‚त्स‚म्भ‚वं} त‚स्य कार‚ण‚स्य स‚न्निधानं साध‚य‚ति \cite[2a6]{vn-msN} । देश‚कालाद्य‚पेक्ष‚येत्या‚{\tiny $_{lb}$}‚ध्याह‚र्त्त‚व्यं । एत‚दुक्त‚म्भ ‚{\tiny $_{6}$}‚ व‚ति [।] कार्य‚कार‚ण‚भाव‚निश्च‚यास्तिद्धं त‚दुत्प‚त्तिल‚क्ष‚ण ‚{\tiny $_{lb}$}‚प्र‚तिव‚स्तु य‚त्रैव‚म‚तादृश उप‚ल‚भ्य‚ते त‚त्रैव स्व‚स‚त्तामात्रेण देश‚कालाद्य‚पेक्ष‚या ‚{\tiny $_{lb}$}‚त‚त् स्व‚कार‚ण‚ङ्ग‚म‚य‚तीति । किं ‚{\tiny $_{7}$}‚ कार‚ण‚मित्याह [।] \quotelemma{कार्य‚स्य कार‚णाव्य‚भि‚{\tiny $_{lb}$}‚चारादि} \cite[2a6]{vn-msN} ति । अन्य‚था हि त‚त्त‚स्य कार्य‚मेव न स्यात् । त‚द्व्य‚भिचारात् । ‚{\tiny $_{lb}$}‚न‚हि य‚द्व्य‚तिरेकेन\edtext{}{\lemma{तिरेकेन}\Bfootnote{? ण}}य‚द् भ‚व‚ति त‚त्त‚स्य कार्यं युक्तं । कुण्ड‚ल‚मिव केयूर ‚{\tiny $_{8}$}‚ स्येत्य‚भि‚{\tiny $_{lb}$}‚स‚न्धिः ।
	{\color{gray}{\rmlatinfont\textsuperscript{§~\theparCount}}}
	\pend% ending standard par
      ‚{\tiny $_{lb}$}‚

	  
	  \pstart \leavevmode% starting standard par
	न‚नु य‚दि नाम धूमोऽग्निकार्य‚त्त्व‚न्न व्य‚भिच‚र‚ति । अन्य‚स्य त्व‚ङ्कुरादेः ‚{\tiny $_{lb}$}‚स्व‚कार‚णैरेव बीजादिभिर‚व्य‚भिचार इति कुत एत‚दित्याह । \quotelemma{अव्य‚भिचारे चे} ‚{\tiny $_{lb}$}‚ \cite[2a6]{vn-msN} त्यादि । इ ‚{\tiny $_{9}$}‚ \leavevmode\ledsidenote{\textenglish{13a/msK}} द‚मुक्त‚म्भ‚व‚ति । य‚दा धूम‚स्य स्व‚कार‚णाव्य‚भिचार‚स्त- ‚{\tiny $_{lb}$}‚दुत्प‚त्तेः सिद्ध‚स्त‚दा बीजादिभिरात्मीयैः कार‚णैः स‚हाङ्कुरादीनां स‚र्व्व‚कार्याणां ‚{\tiny $_{lb}$}‚स‚दृशोऽव्य‚भिचार‚न्यायः । त‚थाहि ‚{\tiny $_{1}$}‚ तेपि य‚थोक्त‚प्र‚काराभ्यां प्र‚त्य‚क्षानुप‚ल‚म्भाभ्यां ‚{\tiny $_{lb}$}‚त‚त्कार्य‚त‚या सिद्धाः स‚न्त‚स्त‚द‚व्य‚भिचारिणः सिद्ध्य‚न्ति । एत‚त्साध‚नाङ्ग‚स‚{\tiny $_{lb}$}‚म‚र्थ‚नं कार्य‚हेतौ । एत‚द्विप‚रीत‚ञ्चास‚म‚र्थ ‚{\tiny $_{2}$}‚ नं । त‚द्वादिनः प‚राज‚याधिक‚र‚ण‚मिति द‚र्श‚{\tiny $_{lb}$}‚य‚न्नाह । \quotelemma{एव‚मि} \cite[2a6]{vn-msN} त्यादि । एव‚मिति प्र‚त्य‚क्षानुप‚ल‚म्भाभ्यां य‚थोक्त‚प्र‚कारा‚{\tiny $_{lb}$}‚भ्याङ्कार्य‚हेताव‚पि न केव‚लं पूर्व्वोक्तेन प्र‚कारे ‚{\tiny $_{3}$}‚ ण स्व‚भाव‚हेताव‚स‚म‚र्थ‚न‚म्वादिनः ‚{\tiny $_{lb}$}‚प‚राज‚य‚स्थान‚मित्य‚पि श‚ब्दः । क‚स्मादेव‚मित्याह । \quotelemma{अस‚म‚र्थित} \cite[2a7]{vn-msN} मित्यादि । ‚{\tiny $_{lb}$}‚अस‚म‚र्थिते त‚स्मिन्कार्य‚कार‚ण‚भावे ‚{\tiny $_{4}$}‚ लिङ्ग‚लिङ्गिनो[र्]लिङ्ग‚स्य वा त‚त्का‚{\tiny $_{lb}$}‚र्य‚त्त्वे आर‚ब्धार्थासिद्धेरिति क्रियाप‚दं ॥ आर‚ब्धोऽर्थः कार‚ण‚स्य स‚त्तासाध‚नं । ‚{\tiny $_{lb}$}‚त‚स्यासिद्धेस्त‚त्प‚राज‚य‚स्थान‚मिति ‚{\tiny $_{5}$}‚ प्र‚कृतेन स‚म्ब‚न्धः ।
	{\color{gray}{\rmlatinfont\textsuperscript{§~\theparCount}}}
	\pend% ending standard par
      ‚{\tiny $_{lb}$}‚

	  
	  \pstart \leavevmode% starting standard par
	\hphantom{.}एत‚देव कुत इत्याह । \quotelemma{अर्थान्त‚र‚स्य} \cite[2a7]{vn-msN} धूमादेर्भावे स‚त्त्वे त‚स्या‚{\tiny $_{lb}$}‚ \leavevmode\ledsidenote{\textenglish{18/s}} न्यादेर्भाव‚निय‚माभावात् । निय‚म‚ग्र‚ह‚णं य‚दृच्छास‚म्वाद‚निरा ‚{\tiny $_{6}$}‚ सार्थं । अर्थान्त‚र‚{\tiny $_{lb}$}‚स्यापि त‚द्भावे प्र‚तिब‚द्ध‚स्व‚भाव‚स्य भावे भ‚व‚त्येव त‚द्भाव‚निय‚म इत्य‚त आह । ‚{\tiny $_{lb}$}‚ \quotelemma{त‚द्भावाप्र‚तिब‚द्ध‚स्व‚भाव‚स्ये} \cite[2a7]{vn-msN} ति । त‚द्भावेऽग्न्यादिभावेऽप्र‚तिब‚द्धोऽना‚{\tiny $_{lb}$}‚य ‚{\tiny $_{7}$}‚ त्तः स्व‚भावोऽस्येति विग्र‚हः । एत‚च्चार्थान्त‚र‚स्य भावे त‚द्भाव‚निय‚माभावादि‚{\tiny $_{lb}$}‚त्येत‚स्य कार‚ण‚म‚व‚ग‚न्त‚व्यं । प्र‚योगः पुन‚र्योऽर्थान्त‚र‚भूतो य‚स्मिन्न प्र‚तिब‚द्ध‚स्व‚भाव ‚{\tiny $_{8}$}‚ ‚{\tiny $_{lb}$}‚स्त‚स्य भावे न त‚द्भाव‚निय‚मः । त‚द्य‚था नूपुर‚स्य भावे मुकुट‚स्य । अर्थान्त‚र‚भूत‚{\tiny $_{lb}$}‚श्चायं धूमादिर‚प्र‚तिब‚द्ध‚स्व‚भाव‚स्त‚स्मिन्न‚ग्न्यादाविति व्याप‚कानुप‚ल‚ब्धिः । व्याप‚क‚{\tiny $_{lb}$}‚विरु ‚{\tiny $_{9}$}‚ \leavevmode\ledsidenote{\textenglish{13b/msK}} द्धोप‚ल‚ब्धिविधिना वा हेत्व‚र्थ‚क‚ल्प‚नात‚द्भावाप्र‚तिब‚द्ध‚स्व‚भाव‚त्त्व‚मेव कुत ‚{\tiny $_{lb}$}‚इत्याह [।] \quotelemma{कार्य‚त्त्वासिद्धे} \cite[2a7]{vn-msN} रिति । एत‚त्पुन‚र‚स‚म‚र्थिते त‚स्मि‚{\tiny $_{lb}$}‚नि\edtext{}{\lemma{नि}\Bfootnote{? न्नि}}ति बोद्ध‚व्यं । त‚त् च प‚र‚मार्थ‚त‚स्ते ‚{\tiny $_{1}$}‚ न कार्य‚हेतुरेवोपात्त‚स्त्त‚द्य‚दि नाम त‚{\tiny $_{lb}$}‚त्कार्य‚न्तेन न स‚म‚र्थितं त‚थाहि क‚थ‚म‚सौ निगृहीत इत्याह । \quotelemma{व‚स्तुतः कार्य‚स्याप्युपादान‚{\tiny $_{lb}$}‚प्र‚तिपाद‚नादि} \cite[2a8]{vn-msN} ति त‚त्कार्य‚त्व‚स्येति वि ‚{\tiny $_{2}$}‚ शेषः ॥ ० ॥
	{\color{gray}{\rmlatinfont\textsuperscript{§~\theparCount}}}
	\pend% ending standard par
      ‚{\tiny $_{lb}$}‚

	  
	  \pstart \leavevmode% starting standard par
	\hphantom{.}एवं कार्य‚हेताव‚पि साध‚नाङ्ग‚स‚म‚र्थ‚न‚म‚भिधायानुप‚ल‚ब्धावाह । \quotelemma{अनुप‚ल‚ब्धा‚{\tiny $_{lb}$}‚व‚पि स‚म‚र्थ‚न} \cite[2a8]{vn-msN} मिति स‚म्ब‚न्धः । साध‚नाङ्स्येत्य‚ध्याहार्य । किं पुन‚स्त‚{\tiny $_{lb}$}‚दित्या ‚{\tiny $_{3}$}‚ ह । अनुप‚ल‚ब्धिसाध‚नं । किं य‚स्य क‚स्य‚चिन्नेत्याह । \quotelemma{उप‚ल‚ब्धिल‚क्ष‚ण‚प्रा‚{\tiny $_{lb}$}‚प्त‚स्येति} \cite[2a8]{vn-msN} । दृश्य‚स्व‚भाव‚स्य नान्य‚स्येति याव‚त् । उप‚ल‚ब्धिर्ज्ञानं । उप‚ल‚{\tiny $_{lb}$}‚ब्धिश‚ब्द‚स्य भा ‚{\tiny $_{4}$}‚ व‚क‚र‚ण‚साध‚न‚त‚या ज्ञान‚प‚र्याय‚त्त्वात्त‚स्या ल‚क्ष‚ण‚ङ्कार‚णं । ल‚क्ष‚ण‚{\tiny $_{lb}$}‚श‚ब्द‚स्य क‚र‚ण‚साध‚न‚त्वेन कृत‚काभिधायित्त्वात् । त‚च्च प्र‚त्य‚यान्त‚र‚साक‚ल्यं स्व‚भाव‚{\tiny $_{lb}$}‚वि ‚{\tiny $_{5}$}‚ शेष‚श्च । त‚द्व्याप्त‚स्यानुप‚ल‚ब्धिः । त‚स्याः साध‚नं प्र‚तिपाद‚न‚मिति व्युत्प‚त्ति‚{\tiny $_{lb}$}‚क्र‚मः ।
	{\color{gray}{\rmlatinfont\textsuperscript{§~\theparCount}}}
	\pend% ending standard par
      ‚{\tiny $_{lb}$}‚

	  
	  \pstart \leavevmode% starting standard par
	\hphantom{.}क‚थ‚मेवंविध‚स्यानुप‚ल‚ब्धिरिति चेत् । \quotelemma{नोद्य‚ते त‚स्य त‚त्रैवेत्} य‚पित्व‚न्य ‚{\tiny $_{6}$}‚ त्र ‚{\tiny $_{lb}$}‚त‚ज्जातीय‚स्य । क‚स्य पुन‚रेवं विध‚स्यानुप‚ल‚ब्धिः प्र‚साध्य‚त इत्याह । \quotelemma{प्र‚तिप‚त्तुः} ‚{\tiny $_{lb}$}‚ \cite[2a9]{vn-msN} प्र‚तिवादिनः । य‚दि चोप‚ल‚ब्धिल‚क्ष‚ण‚प्राप्ताः पिशाचाद‚योपि भ‚व‚न्ति ‚{\tiny $_{lb}$}‚ \leavevmode\ledsidenote{\textenglish{19/s}} त‚ज्जातीयानां ‚{\tiny $_{7}$}‚ अन्येषाम्प्र‚भाव‚व‚ता वा । त‚क्तिन्तेषाम‚प्य‚नुप‚ल‚ब्धिसाध‚नं साध‚{\tiny $_{lb}$}‚नाङ्ग‚स‚म‚र्थ‚न‚म्भ‚व‚ति । नेत्याह । प्र‚तिप‚त्तुः । एत‚दुक्त‚म्भ‚व‚ति [।] य एवासौ प्र‚ति‚{\tiny $_{lb}$}‚पाद्य‚स्त‚स्यैव य‚दुप‚ल‚ब्धिल‚क्ष‚ण‚ञ्चा ‚{\tiny $_{8}$}‚ याति त‚स्यैवानुप‚ल‚ब्धिसाध‚नं नान्य‚स्येति । किं ‚{\tiny $_{lb}$}‚पुनः कार‚ण‚मेव‚म्प्र‚कार‚स्यैवानुप‚ल‚ब्धिसाध‚नं । नान्य‚स्येत्याह । \quotelemma{तादृश्या एवा- ‚{\tiny $_{lb}$}‚नुप‚ल‚ब्धेर‚स‚द्व्य‚व‚हार‚सिद्धेरिति} \cite[2a8]{vn-msN} । अ ‚{\tiny $_{9}$}‚ \leavevmode\ledsidenote{\textenglish{14a/msK}} नुप‚ल‚ब्धिल‚क्ष‚ण‚प्राप्तानुप‚ल‚ब्धेः ‚{\tiny $_{lb}$}‚संश‚य‚हेतुत‚या अग‚म‚क‚त्त्वादिति भावः । अस‚द्व्य‚व‚हार‚सिद्धेरिति व‚च‚न‚म‚स‚द्व्य‚व‚{\tiny $_{lb}$}‚हार एव त‚या साध्य‚ते न त्व‚भावः स्व‚भावानु ‚{\tiny $_{1}$}‚ प‚ल‚ब्धेः स्व‚य‚म‚भाव‚रूप‚त्वादिति ‚{\tiny $_{lb}$}‚प्र‚द‚र्श‚नार्थं । अस‚द्व्य‚व‚हार‚ग्र‚ह‚ण‚ञ्चोप‚ल‚क्ष‚णार्थं । तेनास‚ज्ज्ञान‚श‚ब्दाव‚पि ग्राह्यौ । ‚{\tiny $_{lb}$}‚एत‚त्पुनः कुतोऽव‚सीय‚ते इति चेदाह ॥ ‚{\tiny $_{2}$}‚ \quotelemma{अनुप‚ल‚ब्धिल‚क्ष‚ण‚प्राप्त‚स्यार्थ‚स्य ‚{\tiny $_{lb}$}‚प्र‚तिप‚त्तुः प्र‚त्य‚क्षं त‚देवोप‚ल‚ब्धिस्त‚स्यानिवृत्ताव‚पि स‚त्याम‚भावासिद्धेः} \cite[2a9]{vn-msN} । ‚{\tiny $_{lb}$}‚अभाव‚ग्र‚ह‚ण‚म‚भाव‚व्य‚व‚हार‚श‚ब्द‚ज्ञानोप‚ल‚क्ष ‚{\tiny $_{3}$}‚ णं । उप‚ल‚ब्धिल‚क्ष‚ण‚ञ्च ज्ञान‚प‚रिग्र‚हेण ‚{\tiny $_{lb}$}‚त‚त्प्र‚मित‚व‚स्तुव्युदासाय । का पुन‚रिय‚मुप‚ल‚ब्धिल‚क्ष‚ण‚प्राप्तिर्य‚द्योगादुप‚ल‚ब्धि‚{\tiny $_{lb}$}‚ल‚क्ष‚ण‚प्राप्त इत्युच्य‚त इ ‚{\tiny $_{4}$}‚ त्याह । \quotelemma{त‚त्रेत्यादि} \cite[2a9]{vn-msN} । त‚त्र श्रुतिव‚च‚नोप‚न्या‚{\tiny $_{lb}$}‚सार्था । स्व‚भाव‚विशेषः । किमिय‚देव । नेत्याह । कार‚णान्त‚र‚साक‚ल्य‚ञ्च । त‚स्मा‚{\tiny $_{lb}$}‚त्स्व‚भाव‚विशेषाद्यान्य‚न्यानि ‚{\tiny $_{5}$}‚ कार‚णानीन्द्रिय‚म‚न‚स्कारादीनि तानि कार‚णान्त‚{\tiny $_{lb}$}‚राणि तेषां । साक‚ल्यं साम‚ग्र्यं । स्व‚भाव‚विशेषापेक्ष‚या स‚मुच्च‚यार्थ‚श्च‚कारः । ‚{\tiny $_{lb}$}‚कः पुन‚र‚यं स्व‚भाव ‚{\tiny $_{6}$}‚ विशेष इत्याह । \quotelemma{स्व‚भाव इत्यादि} \cite[2a9]{vn-msN} । य‚द्य‚यं त्रिविधेन ‚{\tiny $_{lb}$}‚विप्र‚क‚र्षेण व्य‚व‚धानेन देश‚काल‚स्व‚भाव‚ल‚क्ष‚णेन न विप्र‚कृष्टं मेरुराम‚सुरादिरूप‚व‚त् ‚{\tiny $_{lb}$}‚स्व‚भाव‚वि ‚{\tiny $_{7}$}‚ शेष उच्य‚ते । त‚मेव स्प‚ष्ट‚य‚ति । \quotelemma{य‚दि} \cite[2a10]{vn-msN} त्यादिना । न आत्म‚{\tiny $_{lb}$}‚रूपोऽनात्म‚रूपः प‚र‚रूप इत्य‚र्थः । स चासौ प्र‚तिभास‚श्च त‚स्य विवेकोऽभाव‚स्तेना‚{\tiny $_{lb}$}‚कारेण य‚त्प्र‚तिप ‚{\tiny $_{8}$}‚ त्तुः प्र‚त्य‚क्ष‚न्त‚त्राप्र‚तिभासितुं शीलं य‚स्य रूप‚स्य स्व‚भाव‚स्य ‚{\tiny $_{lb}$}‚त‚द्रूप‚न्त‚थोक्तं । अथ‚वा रूप‚श‚ब्देण\edtext{}{\lemma{ब्देण}\Bfootnote{? न}}स‚ह विशेष‚ण‚स‚मासः कार्यः । स‚त्युप‚{\tiny $_{lb}$}‚ल‚म्भ‚प्र‚त्य‚यान्त‚र‚सा ‚{\tiny $_{9}$}‚ \leavevmode\ledsidenote{\textenglish{14b/msK}} क‚ल्य इत्युप‚स्क्रिय‚ते । यः स‚जातीय‚विजातीय‚र‚हितेनात्म‚ना ‚{\tiny $_{lb}$}‚प्र‚तिभास‚ते स्व‚ज्ञाने त‚द‚न्य‚कार‚ण‚स‚म‚व‚धाने स‚ति स स्व‚भाव इति याव‚त् । तादृश ‚{\tiny $_{lb}$}‚इति त्रिविध ‚{\tiny $_{1}$}‚ विप्र‚क‚र्षाविप्र‚कृष्ट‚रूपः प‚दार्थ‚स्त‚थाऽनात्म‚रूप‚प्र‚तिभास‚विवेकेन प्र‚ति‚{\tiny $_{lb}$}‚प‚त्तृप्र‚त्य‚क्ष‚प्र‚तिभासेनाश‚येनानुप‚ल‚ब्धः स न अस‚द्व्य‚व‚हार‚स्य विष‚यो भ‚व‚ति । ‚{\tiny $_{2}$}‚ ‚{\tiny $_{lb}$}‚अस‚द्व्य‚व‚हार‚प्र‚तिप‚त्तियोग्यो भ‚व‚तीत्य‚र्थः । विद्य‚मानोपीन्द्रिय‚स्यालोक‚स्य म‚न‚स्का‚{\tiny $_{lb}$}‚ \leavevmode\ledsidenote{\textenglish{20/s}} र‚स्य वाऽभावान्नोप‚ल‚भ्य‚ते तादृश‚स्त‚त्क‚थ‚म‚स‚द्व्य‚व‚हार‚विष‚यो भ‚व‚तीति ‚{\tiny $_{3}$}‚ चेदाह । ‚{\tiny $_{lb}$}‚ \quotelemma{स‚त्स्व‚न्येषूप‚ल‚म्भ} \cite[2a10]{vn-msN} कार‚णेष्विति ।
	{\color{gray}{\rmlatinfont\textsuperscript{§~\theparCount}}}
	\pend% ending standard par
      ‚{\tiny $_{lb}$}‚

	  
	  \pstart \leavevmode% starting standard par
	न‚न्व‚विप्र‚कृष्टोपि घ‚टादिरुप‚ल‚म्भ‚कार‚णान्त‚र‚स‚म‚व‚धानेपि च स‚न्तान‚विप‚{\tiny $_{lb}$}‚रिणामापेक्ष‚त्वान्नोप‚ल‚भ्य‚ते । ‚{\tiny $_{4}$}‚ न‚हि हेत्व‚न्त‚र‚स‚न्निधान‚मिति स्व‚फ‚लोत्पाद‚नानु‚{\tiny $_{lb}$}‚गुणः प‚रिणामो भ‚व‚ति कार‚ण‚स्य [।] त‚थाहि स‚त्याम‚पि पृथिवीबीज‚ज‚लादि‚{\tiny $_{lb}$}‚साम‚ग्र्याम‚तिब‚हुनैव ‚{\tiny $_{5}$}‚ कालेन ताल‚बीज‚स्य स्व‚कार्योद‚यानुकूला प‚रिण‚तिर्भ‚व‚ति । ‚{\tiny $_{lb}$}‚श‚णादिबीज‚स्य त्व‚न‚न्त‚र‚मेव त‚थात्रापि भ‚विष्य‚तीति । किञ्चान्य‚त्प्र‚भाव‚व‚ता‚{\tiny $_{lb}$}‚योगे पि ‚{\tiny $_{6}$}‚ शाच‚मायाकारादिनाऽधिष्ठितो भ‚व‚ति य‚दायं भाव‚स्त‚दा विद्य‚मानोपि नोप‚{\tiny $_{lb}$}‚ल‚भ्य‚ते त‚त्क‚थ‚मुक्तं तादृशः स‚त्स्व‚न्येषूप‚ल‚म्भ‚कार‚णेष्व‚नुप‚ल‚ब्धोऽस‚द्व्य‚व‚हार‚विष ‚{\tiny $_{7}$}‚ य ‚{\tiny $_{lb}$}‚इति । नून‚म्भ‚वा \quotelemma{न्न्याय‚विन्दा} [व]\edtext{\textsuperscript{*}}{\lemma{*}\Bfootnote{न्याय‚विन्दौ द्वितीय‚प‚रिच्छेदे लिङ्ग‚स्य त्रिषु भेदेष्वेकः ।}} ‚{\tiny $_{1}$}‚ प्य‚कृत‚प‚रिश्र‚मः । त‚थाहि अत्रोक्तं \quotelemma{स्व‚भावो यः ‚{\tiny $_{lb}$}‚स‚त्स्व‚न्येषूप‚ल‚म्भ‚कार‚णेषु स‚न्प्र‚त्य‚क्ष एव भ‚व‚ती} ति । य‚श्चायं स‚न्तान‚प‚रिणाम‚म ‚{\tiny $_{8}$}‚ पेक्ष‚ते ‚{\tiny $_{lb}$}‚य‚श्च प्र‚भाव‚व‚ताधिष्ठितः स स्व‚भाव‚विशेष एव न भ‚व‚ति । स‚क‚ल‚त‚द‚न्योप‚ल‚म्भ‚{\tiny $_{lb}$}‚प्र‚त्य‚य‚स‚म‚व‚धानेपि स्व‚रूप‚विष‚योप‚ल‚म्भ‚ज‚न‚क‚त्वात्त‚थैवंविध‚स्य पि ‚{\tiny $_{9}$}‚ \leavevmode\ledsidenote{\textenglish{15a/msK}} शाचादि‚{\tiny $_{lb}$}‚स्व‚भावाविशिष्ट‚रूप‚स्याभाव‚व्य‚व‚हार‚विष‚य‚ता साध्य‚ते । किन्त‚र्हीन्द्रियाण्युप‚ल‚म्भ‚{\tiny $_{lb}$}‚प्र‚त्य‚यान्त‚र‚स‚न्निधाने यः स‚न्प्र‚त्य‚क्ष एव भ‚व‚ति त‚स्य । नैव‚न्त‚र्हि स‚र्व‚थाऽ ‚{\tiny $_{1}$}‚ भावः ‚{\tiny $_{lb}$}‚साधितो भ‚व‚तीति चेत् सुष्ट्व‚नुकूल‚माच‚र‚सि । य‚तोऽन‚न्त‚र‚मेवोक्तं । \quotelemma{य एवाय‚म‚नु‚{\tiny $_{lb}$}‚प‚ह‚तेन्द्रियादिसाक‚ल्ये द‚र्श‚न‚प‚थ‚मुप‚याति । त‚स्य} च त‚त्साक‚ल्येऽनुप ‚{\tiny $_{2}$}‚ ल‚म्भेस्य च ‚{\tiny $_{lb}$}‚व्य‚व‚हार‚विष‚य‚ता साध्य‚ते न तु पिशाचादिस्व‚भावाविशिष्ट‚रूप‚स्येति । न च त‚था‚{\tiny $_{lb}$}‚विध‚स्यापि स‚क‚ल‚त‚द‚न्योप‚ल‚म्भ‚प्र‚त्य‚य‚स‚म‚व‚धानेऽनुप‚ल‚ब्ध‚स्या ‚{\tiny $_{3}$}‚ स्तित्वं युक्त‚{\tiny $_{lb}$}‚म‚नुप‚ल‚ब्धेरेवायोगात् । उप‚ल‚म्भ‚ज‚न‚ने क‚स्य‚चिद‚पेक्ष‚णीय‚स्याभावात् । \quotelemma{प्र‚माण‚{\tiny $_{lb}$}‚विनिश्च‚ये} तु स्प‚ष्टीकृत‚मेवेदं । \quotelemma{न कार्य‚कालेऽभाव‚प्र‚ति ‚{\tiny $_{4}$}‚ प‚त्ते} रित्यादिना । एतेनैव ‚{\tiny $_{lb}$}‚य‚देकेनाव‚श्यं साम‚ग्रिसाक‚ल्येपि प‚रिणाम‚स्ताल‚ग‚ण‚बीज‚व‚दित्यादिना स त‚म‚ति‚{\tiny $_{lb}$}‚श‚य‚व‚त् म‚तिम‚तो म‚नाग‚प्य‚न‚व ‚{\tiny $_{5}$}‚ ग‚च्छ‚न्त‚श्चोद्य‚चुञ्च‚व‚श्चोचुदुस्त‚त्र स‚र्वंम‚यं ‚{\tiny $_{lb}$}‚दुःस्थितं वेदित‚व्य‚मित्य‚ल‚म‚प्र‚तिष्ठित‚बाल‚प्र‚लापैरिति विर‚म्य‚ते । त‚स्मादुप‚ल‚ब्धि‚{\tiny $_{lb}$}‚ल‚क्ष‚ण‚प्राप्तानुप‚ल ‚{\tiny $_{6}$}‚ ब्धिरेवाभाव‚व्य‚व‚हार‚साध‚नीति स्थित‚मेत‚त् । य‚त‚श्चैत‚देवं ‚{\tiny $_{lb}$}‚त‚त‚स्त‚स्मात्कार‚णाद‚न्य‚था स‚ति लिङ्गे स‚म‚वाय उप‚ल‚ब्धिल‚क्ष‚ण‚प्राप्तानुप‚ल‚ब्धि‚{\tiny $_{lb}$}‚ \leavevmode\ledsidenote{\textenglish{21/s}} मुक्त्वा य‚द‚न्य‚द‚स‚द्व्य ‚{\tiny $_{7}$}‚ व‚हार‚साध‚न‚म‚नुप‚ल‚ब्धिमात्रं लिङ्ग‚मुपादीय‚ते । त‚दा ‚{\tiny $_{lb}$}‚त‚स्मिन्स‚ति संश‚यो भ‚व‚ति नास्त्य‚स‚द्व्य‚व‚हार‚निश्च‚यः उप‚ल‚ब्धिनिवृत्ताव‚प्य‚र्था‚{\tiny $_{lb}$}‚भावासिद्धेरिति स‚मुदायार्थ ‚{\tiny $_{8}$}‚ [ः ।] य‚दि वा त‚तो दृश्यानुप‚ल‚म्भाल्लिङ्गात् ‚{\tiny $_{lb}$}‚स‚काशाद‚न्य‚था स‚ति लिङ्गे संश‚य इति व्याख्यात‚व्यं । त‚स्माच्छ‚ब्द‚स्तु पूर्व‚म‚ध्या ‚{\tiny $_{lb}$}‚ह‚र्त्त‚व्यं । अथ‚वा त‚त उप‚ल‚ब्धिल‚क्ष‚ण‚प्राप्ताद ‚{\tiny $_{9}$}‚ \leavevmode\ledsidenote{\textenglish{15b/msK}} न्य‚था त‚द्विहीने संश‚ये स‚ति त‚ल्लिङ्ग ‚{\tiny $_{lb}$}‚इति व्याख्येयं । का पुन‚र‚त्रानुप‚ल‚ब्धौ व्याप्तिरित्याह । \quotelemma{अत्रापीत्यादि} \cite[2b11]{vn-msN} । ‚{\tiny $_{lb}$}‚एवं विध‚मिति दृश्यं स‚द‚नुप‚ल‚ब्धं [।] स‚र्व्व‚ग्र‚ह‚णं स‚र्व्वोप‚सं ‚{\tiny $_{1}$}‚ हारेण व्याप्तिप्र‚द‚र्श‚नार्थं ॥
	{\color{gray}{\rmlatinfont\textsuperscript{§~\theparCount}}}
	\pend% ending standard par
      ‚{\tiny $_{lb}$}‚

	  
	  \pstart \leavevmode% starting standard par
	न‚नु य‚दि नाम क‚स्य‚चिद्विषाणादेः श‚श‚म‚स्त‚कादावुप‚ल‚ब्धिल‚क्ष‚ण‚प्राप्तानु‚{\tiny $_{lb}$}‚प‚ल‚ब्ध‚स्यास‚द्व्य‚व‚हार‚विष‚य‚ता । अन्येनापि सामान्ये वि ‚{\tiny $_{2}$}‚ शेष्येव‚य‚विद्र‚व्य‚संयोग‚{\tiny $_{lb}$}‚विभागादिना त‚थाविधेन त‚था भ‚वित‚व्य‚मिति कुतोऽयं निय‚म इत्य‚त आह । \quotelemma{क‚स्य} ‚{\tiny $_{lb}$}‚चिदि \cite[2b1]{vn-msN} त्यादि । क‚स्य‚चिदुप‚ल‚ब्धिल‚क्ष‚ण‚प्राप्त‚स्या ‚{\tiny $_{3}$}‚ नुप‚ल‚ब्ध‚स्य श‚श‚विषा‚{\tiny $_{lb}$}‚णादेर‚स‚तोऽस‚द्व्य‚व‚हार‚विष‚येस्त्य‚भ्युप‚ग‚मेऽस‚द्व्य‚व‚हारादिविष‚योऽस‚न्नित्युक्तः । ‚{\tiny $_{lb}$}‚ \quotelemma{त‚ल्ल‚क्ष‚णाविशेषादि \cite[2b1]{vn-msN} ति} । त‚स्या ‚{\tiny $_{4}$}‚ स‚तो ल‚क्ष‚णं निमित्तं य‚थोक्तानुप‚ल‚ब्धि‚{\tiny $_{lb}$}‚र्ल‚क्ष‚ण‚श‚ब्द‚श्च क‚र‚ण‚साध‚न‚स्त‚स्याविशिष्ट‚त्वात् सामान्य‚विशेषाव‚य‚विद्र‚व्यादा‚{\tiny $_{lb}$}‚विति वाक्य‚शेषः । एत ‚{\tiny $_{5}$}‚ दुक्त‚म्भ‚व‚ति श‚श‚विषाणादेर‚प्य‚स‚द्व्य‚व‚हार‚विष[य] ‚{\tiny $_{lb}$}‚त्त्वं क‚स्मादिष्य‚ते । य‚थोक्तानुप‚ल‚म्भ‚स्य त‚न्निमित्त‚स्य स‚द्भावादिति चेत् । य‚द्येवं ‚{\tiny $_{lb}$}‚सामान्य‚विशे ‚{\tiny $_{6}$}‚ ष‚ता त‚स्यास्तीति क‚स्मात्त‚था स‚द्व्य‚व‚हार‚विष‚य‚त्व‚न्नाभ्युप‚ग‚म्य‚ते [।] ‚{\tiny $_{lb}$}‚अन्य‚था त‚त्रापि त‚त्स्याच्चेत् । न‚हि पुरुषेच्छाव‚शाद्धेतौ विष‚य‚प्र‚विभागो युक्त ‚{\tiny $_{lb}$}‚इति । \quotelemma{न‚ही} \cite[2b1]{vn-msN} त्या ‚{\tiny $_{7}$}‚ दिनैत‚देव व्य‚न‚क्ति । एवंविध‚स्य दृश्य‚स्य स‚त्त्वेऽनुप‚ल‚ब्ध‚स्या ‚{\tiny $_{lb}$}‚ \quotelemma{स‚त्वान‚भ्युप‚ग‚म} \cite[2b1]{vn-msN} इति । अस‚द्व्य‚व‚हारादिविष‚य‚त्वान्नाभ्युप‚ग‚म इत्य‚र्थः ‚{\tiny $_{lb}$}‚अस‚त्व‚श‚ब्देना ‚{\tiny $_{8}$}‚ स‚द्व्य‚व‚हारो विनिश्च‚य‚स्त‚स्योप‚ल‚क्ष‚ण‚म् । युक्तोप‚ल‚म्भ‚स्य त‚स्यैवा ‚{\tiny $_{lb}$}‚नुप‚ल‚म्भ‚नं प्र‚तिषेध‚हेतुरित्यादि चेत् । अन्य‚त्र श‚श‚शृङ्गाभावे द‚ण्डेन पुरुष‚स्य \edtext{}{\lemma{स्य}\Bfootnote{?}} ‚{\tiny $_{lb}$}‚योगः स ‚{\tiny $_{9}$}‚ \leavevmode\ledsidenote{\textenglish{16a/msK}} एव इत्य‚र्थः [।] न‚ह्येवंविध‚स्य दृश्य‚स्य च‚क्षुरादिशून्येषूप‚ल‚म्भ‚कार‚णेषु ‚{\tiny $_{lb}$}‚स अनुप‚ल‚ब्धिर्भ‚व‚ति ‚{\tiny $_{1}$}‚ ॥
	{\color{gray}{\rmlatinfont\textsuperscript{§~\theparCount}}}
	\pend% ending standard par
      ‚{\tiny $_{lb}$}‚

	  
	  \pstart \leavevmode% starting standard par
	किन्त‚र्ह्युप‚ल‚ब्धिरेव भ‚व‚तीति प्र‚तिषेध‚द्व‚येनाह । अन्य‚स्योप‚ल‚ब्धिप्र‚त्य‚य‚स्य ‚{\tiny $_{lb}$}‚क‚स्य‚चिव‚पेक्ष‚णीय‚स्याभावादिति भावः । त‚द‚नेन प्र‚कृत‚मेव स्प‚ष्ट‚य‚ति । अनुप‚ल ‚{\tiny $_{2}$}‚ ‚{\tiny $_{lb}$}‚भ्य‚मानं त्वीदृश‚मित्युप‚ल‚ब्धिल‚क्ष‚ण‚प्राप्त‚न्नास्ति त‚स्मादेताव‚त्सात्र उप‚ल‚ब्धिल‚क्ष‚ण‚{\tiny $_{lb}$}‚प्राप्तानुप‚ल‚ब्धिमात्र‚न्निमित्तं य‚स्यास‚द्व्य‚व‚हार‚स्य स त‚था ख्यातः त‚द‚नेनास‚त् ‚{\tiny $_{3}$}‚ ‚{\tiny $_{lb}$}‚ \leavevmode\ledsidenote{\textenglish{22/s}} व्य‚व‚हार‚स्यान‚न्य‚निमित्त‚तामाह । एत‚देव कुत इत्याह । \quotelemma{अन्य‚स्ये} \cite[2b2]{vn-msN} त्यादि । ‚{\tiny $_{lb}$}‚य‚थोक्तानुप‚ल‚ब्धिम‚पास्यान्य‚स्य नास्तित्व‚व्य‚व‚हार‚निमित्त‚स्याभावादिति स‚{\tiny $_{4}$}‚म\edtext{}{\lemma{म}\Bfootnote{? मु}} ‚{\tiny $_{lb}$}‚च्च‚यार्थः ।
	{\color{gray}{\rmlatinfont\textsuperscript{§~\theparCount}}}
	\pend% ending standard par
      ‚{\tiny $_{lb}$}‚

	  
	  \pstart \leavevmode% starting standard par
	न‚नु च य‚स्य य‚त्र न किञ्चित्साम‚र्थ्य‚म‚स्ति त‚द‚स‚द्व्य‚व‚हार‚विष‚यो य‚था न‚भ‚स्त‚ले ‚{\tiny $_{lb}$}‚क‚म‚लं । त‚थाभिम‚तेपि देशादाव‚भिम‚त‚स्य भाव‚स्य ‚{\tiny $_{5}$}‚ न किञ्चित्साम‚र्थ्य‚म‚स्तीति ‚{\tiny $_{lb}$}‚स‚र्व्व‚साम‚र्थ्य‚विवेक एव नास्तित्व‚व्य‚व‚हार‚स्य निमित्तं भ‚विष्य‚ति । एत‚क्तिम‚सं‚{\tiny $_{lb}$}‚ब‚द्ध‚मेवोद्घाटित‚शिरोभिर‚भिधीय‚ते । अ ‚{\tiny $_{6}$}‚ न्य‚स्य त‚न्निमित्त‚स्याभावादिति क‚दाचित् ‚{\tiny $_{lb}$}‚क‚श्चित् ब्रूयादिति त‚न्म‚त‚माशंक‚ते । \quotelemma{स‚र्व्व‚साम‚र्थ्य‚विवेको निमित्त‚मिति} \cite[2b2]{vn-msN} चेदि ‚{\tiny $_{lb}$}‚ति । अत्र स‚माधिमाह । \quotelemma{एव‚मि} \cite[2b2]{vn-msN} त्यादिना । एवं ‚{\tiny $_{7}$}‚ म‚न्य‚ते सूक्त‚मेत‚त् स‚र्व्व‚{\tiny $_{lb}$}‚साम‚र्थ्य‚विवेको निमित्त‚मिति किन्तु स एव स‚र्व्व‚साम‚र्थ्य‚विवेकोयं प‚दार्थः क‚थ‚म‚व‚{\tiny $_{lb}$}‚ग‚तो य‚थोक्ताम‚नुप‚ल‚ब्धिम‚पास्य येन स‚र्व्व‚साम‚र्थ्य‚वि ‚{\tiny $_{8}$}‚ वेकोऽस्यास‚द्व्य‚व‚हार‚स्य ‚{\tiny $_{lb}$}‚निमित्त‚म्भ‚विष्य‚ति । न चासाव‚ज्ञात एव त‚स्य निमित्त‚म्भ‚वितुम‚र्ह‚ति । ज्ञाप‚क‚हेत्व‚{\tiny $_{lb}$}‚धिकारात् । क‚स्मात्स‚र्व्व‚साम‚र्थ्य‚विवेकिनो य‚थोक्तानु ‚{\tiny $_{9}$}‚ \leavevmode\ledsidenote{\textenglish{16b/msK}} प‚ल‚म्भेने\edtext{}{\lemma{म्भेने}\Bfootnote{नै ?}}व प्र‚तीति‚{\tiny $_{lb}$}‚रित्याह । \quotelemma{अन्य‚स्य त‚त्प्र‚तिप‚त्युपाय‚स्याभावा} \cite[2b3]{vn-msN} दिति । य‚दा तु य‚थोक्तानुप‚{\tiny $_{lb}$}‚ल‚ब्ध्या त‚स्य स‚र्व्व‚साम‚र्थ्य‚विवेकिनः प्र‚तीतिर्भ‚व‚ति । त‚दा त‚त्प्र‚तिप‚त्तौ ‚{\tiny $_{1}$}‚ स‚त्याम‚{\tiny $_{lb}$}‚स‚द्व्य‚व‚हारो भ‚व‚ति । इति त‚स्मादिदं य‚थोक्तानुप‚ल‚म्भ‚नं त‚स्यास‚द्व्य‚व‚हार‚स्या‚{\tiny $_{lb}$}‚निमित्त‚मुच्य‚ते ।
	{\color{gray}{\rmlatinfont\textsuperscript{§~\theparCount}}}
	\pend% ending standard par
      ‚{\tiny $_{lb}$}‚

	  
	  \pstart \leavevmode% starting standard par
	पुन‚र‚पि प‚रोन्य‚स्य त‚न्निमित्त‚स्याभावादित्य‚स्य क‚दाचि ‚{\tiny $_{2}$}‚ द‚युक्त‚ताम्ब्रूयादि‚{\tiny $_{lb}$}‚त्याश‚ङ्क‚ते \quotelemma{बुद्धिव्य‚प‚देश} \cite[2b3]{vn-msN} इत्यादिना । अय‚म‚स्याभिस‚न्धिर्बुद्धिव्य‚प‚दे‚{\tiny $_{lb}$}‚शार्थ‚क्रियाभ्यः स‚कासा\edtext{}{\lemma{कासा}\Bfootnote{? शा}}त्त‚द्व्य‚व‚हारो भ‚व‚ति । त‚था हि ताः प्र‚व‚र्त्त‚माना ‚{\tiny $_{3}$}‚ ‚{\tiny $_{lb}$}‚व‚स्तुस‚त्तां साध‚य‚न्ति । त‚द्भेदाभेदौ च व‚स्तुभेदाभेदावित्याश‚यः । ‚{\tiny $_{lb}$}‚ताश्च निव‚र्त्त‚मानाः स्व‚निमित्तं स‚द्व्य‚व‚हारं निव‚र्त‚य‚न्त्य‚ग्निरिव धूमं । त‚न्निवृत्तौ ‚{\tiny $_{4}$}‚ ‚{\tiny $_{lb}$}‚चास‚द्व्य‚व‚हारः । स‚द्व्य‚व‚हारास‚द्व्य‚व‚हार‚योर‚न्योन्य‚व्य‚व‚च्छेद‚स्थित‚रूप‚त्वेन ‚{\tiny $_{lb}$}‚एक‚त्याग‚स्याप‚रोपादानेनान्त‚रीय‚क‚त्वात् । त‚त‚श्च बुद्ध्यादिनिवृत्तौ ‚{\tiny $_{5}$}‚ चास‚द्व्य‚व‚{\tiny $_{lb}$}‚हार‚निमित्त‚मिति नेदं युक्तं व‚क्तुम‚न्य‚स्येत्यादि । अत्रापि प्र‚तिविधान‚माह । \quotelemma{भ‚व}‚{\tiny $_{lb}$}‚ \leavevmode\ledsidenote{\textenglish{23/s}} ती \cite[2b4]{vn-msN} त्यादि । य‚थोक्त‚प्र‚तिभाषा\edtext{}{\lemma{तिभाषा}\Bfootnote{? सा}}बुद्धिः प्र‚तिप‚त्तृप्र‚त्य‚क्ष‚प्र ‚{\tiny $_{6}$}‚ तिभासि‚{\tiny $_{lb}$}‚रूप‚निर्भासा य‚थोक्तः प्र‚तिभासो य‚स्या इति विग्र‚हः । उक्त‚ञ्च य‚द‚नात्मेत्यादिना । ‚{\tiny $_{lb}$}‚त‚स्याः स‚कासा\edtext{}{\lemma{कासा}\Bfootnote{? शा}}त्स‚द्व्य‚व‚हारो भ‚व‚ति साक्षा[द्] व‚स्तुग्र‚ह‚णात् । त‚स्या ‚{\tiny $_{7}$}‚ ञ्च ‚{\tiny $_{lb}$}‚विप‚र्य‚योऽभाव‚स्त‚स्मिन् स‚त्य‚स‚द्व्य‚व‚हारो भ‚व‚ति । स‚त्स्व‚न्येषूप‚ल‚म्भ‚कार‚णेष्विति ‚{\tiny $_{lb}$}‚वाक्य‚प‚रिस‚माप्तिः कार्याः\edtext{}{\lemma{कार्याः}\Bfootnote{? र्या}}। अन्य‚था संस\edtext{}{\lemma{संस}\Bfootnote{? श}}योत्प‚त्तेः । न‚हि व‚स्तु‚{\tiny $_{lb}$}‚स‚त्व उप‚लं ‚{\tiny $_{8}$}‚ भ‚प्र‚त्य‚यान्त‚र‚साक‚ल्ये च सा निव‚र्त्त‚त इति ‚{\tiny $_{9}$}‚ \leavevmode\ledsidenote{\textenglish{17a/msK}} निवेदित‚मेत‚त् पुरोऽस्मा- ‚{\tiny $_{lb}$}‚भिरितिभावः । त‚व‚नेन य‚द्येवंविधा बुद्धिर‚भिम‚ता त्व‚या त‚दाव‚योरैक‚म‚त्य‚मेव ‚{\tiny $_{lb}$}‚त‚थापि न नः किञ्चिद‚निष्ट‚मुक्तं स्यात । अथान्या\edtext{}{\lemma{अथान्या}\Bfootnote{? न्यः}}त‚दा व्य ‚{\tiny $_{1}$}‚ भिचार ‚{\tiny $_{lb}$}‚इति द‚र्श‚य‚ति । त‚मेव व्य‚भिचार‚न्द‚र्श‚य‚न्नाह । \quotelemma{प्र‚त्य‚क्षाविष‚ये} \cite[2b4]{vn-msN} त्यादिना । ‚{\tiny $_{lb}$}‚लिङ्गाज्जाता लिङ्ग‚जाः । अनुमान‚मित्य‚र्थः । त‚स्याः स‚कासा\edtext{}{\lemma{कासा}\Bfootnote{? शा}}त्स‚द्व्य‚व‚हारः ‚{\tiny $_{lb}$}‚स्यात्प‚रो ‚{\tiny $_{2}$}‚ क्षेऽर्थेन केव‚ल‚म‚न‚न्त‚रोदित‚रूपायाः स्व‚ग्राह्य इत्य‚पिनाह । किं ‚{\tiny $_{lb}$}‚लिङ्ग‚जायाः स‚र्व्व‚स्याः स‚म्भ‚व‚ति नेत्याह । \quotelemma{कुत‚श्चिदि} \cite[2b4]{vn-msN} ति [।] स्व‚भाव‚{\tiny $_{lb}$}‚कार्य‚लिङ्ग‚द्व‚य‚ब‚लोप‚जाता ‚{\tiny $_{3}$}‚ या इत्य‚र्थः । अनुप‚ल‚म्भ‚स्यास‚त्ताऽस‚द्व्य‚व‚हार‚{\tiny $_{lb}$}‚साध‚क‚त्वादिति भावः । य‚दि नामे\edtext{}{\lemma{नामे}\Bfootnote{? मै}}वं त‚तः क‚थ‚म्व्य‚भिचार इति चेदाह । ‚{\tiny $_{lb}$}‚ \quotelemma{अस‚द्व्य‚व‚हार‚स्त्वि} \cite[2b4]{vn-msN} त्या ‚{\tiny $_{4}$}‚ दि । त‚द्विप‚र्य इति त‚स्या य‚थोक्त‚लिङ्ग‚जाया ‚{\tiny $_{lb}$}‚बुद्धेर्विनिवृत्ताव‚नैकान्तिकः स‚न्दिग्ध इत्य‚र्थः । किं कार‚णं विप्र‚कृष्टेर्थे देशादि‚{\tiny $_{lb}$}‚विप्र‚क‚र्षैः ‚{\tiny $_{5}$}‚ प्र‚तिप‚त्तृप्र‚त्य‚क्ष‚स्य प्र‚माण‚स्य निवृत्ताव‚पि संश‚यात्कार‚णात् । अर्था‚{\tiny $_{lb}$}‚भाव इति शेषः [।] प्र‚तिप‚त्तुः प्र‚त्य‚क्ष‚मिति ष‚ष्ठीस‚मासः । इद‚ञ्च प्र‚ही ‚{\tiny $_{6}$}‚ ण‚स‚क‚ल‚ज्ञे‚{\tiny $_{lb}$}‚याव‚र‚ण‚स्य प्र‚त्य‚क्ष‚निवृत्तौ त्व‚संदेह एवेति क‚थ‚नायोपात्तं । अन्य‚स्य चेत्य‚नुमान‚स्या‚{\tiny $_{lb}$}‚ग‚म‚स्य च । एत‚च्चाग‚म‚स्य प्रामाण्य‚म‚भ्युप‚ग‚म्याभिधीय ‚{\tiny $_{7}$}‚ ते । न तु त‚स्य प्रामाण्य‚{\tiny $_{lb}$}‚म‚स्ति [।]
	{\color{gray}{\rmlatinfont\textsuperscript{§~\theparCount}}}
	\pend% ending standard par
      ‚{\tiny $_{lb}$}‚
	  \bigskip
	  \begingroup
	
	    
	    \stanza[\smallbreak]
	  \flagstanza{\tiny\textenglish{...4}}{\normalfontlatin\large ``\qquad}नान्त‚रीय‚क‚ताभावाच्छ‚ब्दानाम्व‚स्तुभिः स‚ह [।]&‚{\tiny $_{lb}$}‚नार्थ‚सिद्धिस्त‚त‚स्ते हि व‚क्त्र‚भिप्राय‚सूच‚का [ः ॥ ४]{\normalfontlatin\large\qquad{}"}\&[\smallbreak]
	  
	  
	  
	  \endgroup
	‚{\tiny $_{lb}$}‚

	  
	  \pstart \leavevmode% starting standard par
	इत्यादिव‚च‚नात् ।
	{\color{gray}{\rmlatinfont\textsuperscript{§~\theparCount}}}
	\pend% ending standard par
      ‚{\tiny $_{lb}$}‚

	  
	  \pstart \leavevmode% starting standard par
	अय‚म‚स्याभिप्रायो य‚दि ‚{\tiny $_{8}$}‚ नाम प्र‚माण‚त्र‚य‚न्निवृत्त‚म‚प्र‚त्य‚क्ष‚व‚स्तुनि त‚थापि ‚{\tiny $_{lb}$}‚त‚न्नास्तीति कुतोय‚न्निश्च‚यः । त‚थाहि म‚ल‚य‚न‚ग‚निकुञ्ज‚व‚र्त्तिदूर्व्वाप्र‚वाल‚प‚त्र‚{\tiny $_{lb}$}‚प्र‚भृत‚यः प्र‚माण‚त्र‚य‚गोच ‚{\tiny $_{9}$}‚\leavevmode\ledsidenote{\textenglish{17b/msK}} र‚भावातिक्रान्तां मूर्तिमुद्व‚ह‚न्त‚स्तिष्ठ‚न्ति । न च ते न ‚{\tiny $_{lb}$}‚ \leavevmode\ledsidenote{\textenglish{24/s}} स‚न्तीति श‚क्य‚म‚भिधातुं । प्र‚माण‚भाव‚स्य स‚क‚ल‚विष‚य‚स‚त्व‚व्याप‚क‚त्व‚कार‚ण‚त्वा‚{\tiny $_{lb}$}‚भावात् । न च त‚द्रूप‚विक‚ल‚प‚दार्थ‚नि ‚{\tiny $_{1}$}‚ वृत्ताव‚न्य‚निवृत्तिनिय‚मेनातिप्र‚स‚ङ्ग‚दोषो‚{\tiny $_{lb}$}‚प‚निपातादित्यावेदित‚मेतः\edtext{}{\lemma{मेतः}\Bfootnote{? त‚त्}}पुर‚स्तात् । य‚दाह । ‚{\tiny $_{lb}$}‚ 
	    \pend% close preceding par
	  
	    
	    \stanza[\smallbreak]
	  \flagstanza{\tiny\textenglish{...5}}{\normalfontlatin\large ``\qquad}शास्त्राधिकारास‚म्ब‚द्धा ब‚ह‚वोऽर्थाकृतेन्द्रियाः ।&‚{\tiny $_{lb}$}‚अलिङ्गाश्च क‚थ‚न्तेषाम‚भावोऽनुप‚ल‚म्भ‚ता [॥ ५]{\normalfontlatin\large\qquad{}"}\&[\smallbreak]
	  
	  
	  
	    \pstart  \leavevmode% new par for following
	    \hphantom{.}
	   इति ।
	{\color{gray}{\rmlatinfont\textsuperscript{§~\theparCount}}}
	\pend% ending standard par
      ‚{\tiny $_{lb}$}‚

	  
	  \pstart \leavevmode% starting standard par
	न‚नु चात्र लिङ्ग‚जाया म‚तेर‚स‚द्व्य‚व‚हार‚हेतुत्त्वं निषेद्धुमार‚ब्ध‚न्त‚द्विप‚र्य्य‚य ‚{\tiny $_{lb}$}‚इत्याद्य‚भिधानात् । त‚त्क‚स्माद‚प्र‚स्तुत‚स्यैव प्र‚त्य‚क्ष‚स्य चाग‚म ‚{\tiny $_{3}$}‚ स्य चोप‚क्षेपः कृत इति‚{\tiny $_{lb}$}‚चेत् । युक्त‚मेत‚त् । स‚र्व‚प्र‚माण‚निवृत्तेस्त्व‚ग‚म‚क‚त्व‚प्र‚द‚र्श‚नेनैत‚देकान्तास‚म्भ‚व‚द‚र्श‚{\tiny $_{lb}$}‚नायोक्त‚मिति ल‚क्ष्य‚तेऽस्य सुधियोऽ ‚{\tiny $_{4}$}‚ भिस‚न्धिः ।
	{\color{gray}{\rmlatinfont\textsuperscript{§~\theparCount}}}
	\pend% ending standard par
      ‚{\tiny $_{lb}$}‚

	  
	  \pstart \leavevmode% starting standard par
	\hphantom{.}न‚न्विद‚मुक्तं \quotelemma{स‚द्व्य‚व‚हारास‚द्व्य‚व‚हार‚योर‚न्योन्य‚व्य‚व‚च्छेद} स्थित‚ल‚क्ष‚ण‚त्वे ‚{\tiny $_{lb}$}‚नैकाभाव‚स्याप‚र‚भाव‚नान्त‚रीय‚क‚त्वात् विप्र‚कृष्टे त‚न्निमित्ता ‚{\tiny $_{5}$}‚ भावात् स‚द्व्य‚व‚{\tiny $_{lb}$}‚हार‚निवृत्त्याऽस‚द्व्य‚व‚हार इति [।] स‚त्य‚मुक्त‚मेवैत‚न्न पुन‚र्युक्तं । त‚थाहि प‚रोक्षेऽर्थे ‚{\tiny $_{lb}$}‚स‚द्व्य‚व‚हार‚निवृत्तिः क‚थ‚न्त‚न्निमित्ताभावेपि ‚{\tiny $_{6}$}‚ द्व‚योर‚प्य‚न‚योर‚नुप‚ल‚ब्ध्योः स्व‚विप‚र्य‚य‚{\tiny $_{lb}$}‚हेत्व‚भाव‚भावाभ्यां स‚द्व्य‚व‚हार‚प्र‚तिषेध‚फ‚ल‚त्व‚न्तुल्य‚मेक‚त्र संश‚याद‚प‚र‚त्र विप‚{\tiny $_{lb}$}‚र्य‚यादिति व‚च‚नात् संश‚येनेति चेत् । ‚{\tiny $_{7}$}‚ य‚द्येवं क‚थ‚न्त‚र्ह्य‚स‚द्व्य‚व‚हार‚निश्च‚य‚स्त‚त्र ‚{\tiny $_{lb}$}‚युक्तिम‚नुप‚त‚ति । य‚त्र तु निश्च‚येन स‚द्व्य‚व‚हार‚निवृत्तिस्त‚त्रास‚द्व्य‚व‚हारोपि ‚{\tiny $_{lb}$}‚पुक्त एवान्यः प्र‚व‚र्त‚न‚फ‚लोसीत्युक्तेः ‚{\tiny $_{8}$}‚ न चाभाव‚रूप‚व्य‚व‚च्छेदे भावानुष‚ङ्गोस्ति ‚{\tiny $_{lb}$}‚निय‚मेन । न‚हि ब‚न्ध्यात‚न‚य‚न‚भःप‚ङ्क‚जादिष्व‚स‚द‚व‚स्थ‚ता भ‚व‚ति प्र‚तिषेधात् ‚{\tiny $_{lb}$}‚स‚द‚व‚स्थ‚ता भ‚व‚ति प्र‚तिषेध‚मात्र‚न्तु स्यात् । त‚यो ‚{\tiny $_{9}$}‚\leavevmode\ledsidenote{\textenglish{18a/msK}} र‚न्योन्य‚व्य‚व‚च्छेदेनाव‚स्थानात् । ‚{\tiny $_{lb}$}‚त‚थात्राप्य‚प्र‚त्य‚क्षे स‚द्व्य‚व‚हार‚प्र‚तिषेधान्न विधिभूता स‚द्व्य‚व‚हारानुस\edtext{}{\lemma{हारानुस}\Bfootnote{? ष}}ङ्ग‚{\tiny $_{lb}$}‚स्त‚द्व्य‚व‚च्छेद‚मात्र‚न्तु स्यात् । त‚द्भाव‚स्य त‚द्भाव‚स्यान्योन्य‚प‚रिहारेण ‚{\tiny $_{1}$}‚ अव‚स्थि‚{\tiny $_{lb}$}‚त‚त्वात् । उक्त‚ञ्चैत‚द‚म‚ल‚न्याय‚त‚त्व‚प्र‚बोधोद्ग‚त‚प्र‚ज्ञालोक‚तिर‚स्कृताशेष‚प‚र‚तीर्थ्य ‚{\tiny $_{lb}$}‚प्र‚वाद‚ध्वान्तेन \quotelemma{ध‚र्म‚कीर्तिनैवानित्य‚निरात्म‚तादिव्य‚व‚च्छेदेपि त‚च्च स्यादित्यादि} - ‚{\tiny $_{2}$}‚ ‚{\tiny $_{lb}$}‚नेत्यास्तान्ताव‚त् ।
	{\color{gray}{\rmlatinfont\textsuperscript{§~\theparCount}}}
	\pend% ending standard par
      ‚{\tiny $_{lb}$}‚

	  
	  \pstart \leavevmode% starting standard par
	अधुना सामान्य‚भूतानां बुद्धिव्य‚प‚देशानां स‚द्व्य‚व‚हार‚हेतुत्व‚म‚पि नास्तीति ‚{\tiny $_{lb}$}‚क‚थ‚य‚न्नाह । \quotelemma{न चे} \cite[2b5]{vn-msN} त्यादि । त‚त्र च य‚थाक्र‚म‚म‚भिस‚म्ब‚न्धः । ते स‚र्व्वे बुद्धि‚{\tiny $_{lb}$}‚व्य‚प‚दे ‚{\tiny $_{3}$}‚ शा न व‚स्तुस‚त्तां साध‚य‚न्ति । तेषाम्वा भेदाभेदौ न व‚स्तुभेदाभेद‚योः स‚त्ता‚{\tiny $_{lb}$}‚ \leavevmode\ledsidenote{\textenglish{25/s}} स‚द्भाव‚मिति । स‚र्व्व‚ग्र‚ह‚णं केचित्तु साध‚य‚न्त्येवेति प्र‚द‚र्श‚नाय । कुत एत‚दित्याह । ‚{\tiny $_{lb}$}‚ \quotelemma{अस ‚{\tiny $_{4}$}‚ त्स्व‚प्य‚तीतानाग‚तादिषु} \cite[2b5]{vn-msN} वृत्तेरिति क्रियाप‚दं । आदिश‚ब्देन व्योमो‚{\tiny $_{lb}$}‚त्प‚लाद‚यः प‚रिगृह्य‚न्ते । क‚थ‚म्पुन‚र्विष‚य‚म‚न्त‚रेण तेषु तेषां वृत्तिर्युक्तेति चेदाह । ‚{\tiny $_{lb}$}‚ \quotelemma{क‚थं ‚{\tiny $_{5}$}‚ चित्त} \cite[2b6]{vn-msN} द्रूपोनुभ‚वाहित‚वास‚नाप‚रिपाक‚प्र‚भावादित्य‚र्थः । तेषाञ्च ‚{\tiny $_{lb}$}‚व‚स्तुप्र‚तिब‚न्धाभावादिति भावः । \quotelemma{श‚ङ्ख‚च‚क्र‚व‚र्ती} \cite[2b7]{vn-msN} त्यादिना प्र‚कारेण ‚{\tiny $_{lb}$}‚त‚द‚नेन व‚स्तुस‚त्तां साध‚य‚न्तीत्येत‚स्य कार‚ण‚माह । \quotelemma{नानैकाम‚र्थ‚क्रिया} \cite[2b6]{vn-msN} ‚{\tiny $_{lb}$}‚ङ्क‚र्तुंसी\edtext{}{\lemma{र्तुंसी}\Bfootnote{? शी}}लं येषां ते त‚थोक्ताः । तेष्व‚पि च वृत्तेः कार‚णात् । किम‚र्थं तेषु ‚{\tiny $_{lb}$}‚तेषाम्वृत्तिस्त‚द्भाव‚ख्याप‚नाय । तेषान्नाऽ ‚{\tiny $_{7}$}‚ नैकार्थ‚क्रियाकारिणाम्भाव‚स्त‚स्य ख्या‚{\tiny $_{lb}$}‚प‚नाय । नानार्थ‚क्रियाकारित्व‚स्यैकार्थ‚क्रियाकारित्व‚स्य च क‚थ‚नार्थ‚मिति याव‚त् । ‚{\tiny $_{lb}$}‚अस्त्येव त‚र्हि त‚स्य व‚स्तुन‚स्त‚त्व‚मित्य‚त आह [।] ‚{\tiny $_{8}$}‚ \quotelemma{नानैकात्म‚ताया} \cite[2b6]{vn-msN} ‚{\tiny $_{lb}$}‚अभावेपि त‚स्य व‚स्तुन इत्य‚धा\edtext{}{\lemma{धा}\Bfootnote{? ध्या}}ह‚र्त‚व्यं । नानैक‚रूपाणाम्बुद्धिव्य‚प‚देशानां ‚{\tiny $_{lb}$}‚त‚द‚नेन न व‚स्तुभेदाभेदौ साध‚य‚न्तीति साध‚य‚ति । इद‚मेव निद‚र्श‚न ‚{\tiny $_{9}$}‚ \leavevmode\ledsidenote{\textenglish{18b/msK}} प्र‚द‚र्श‚नेन स‚फ‚ली- ‚{\tiny $_{lb}$}‚क‚रोति । \quotelemma{राजा म‚हास‚म्म‚त} \cite[2b6]{vn-msN} इद‚म‚तीत‚वृत्तेरुदाह‚र‚णं । य‚थेति चाध्याहार्यं । ‚{\tiny $_{lb}$}‚ \quotelemma{श‚ङ्ख} च‚क्र‚व‚र्तीत्याद्य‚नुत्प‚न्न‚वृत्तेः श‚ब्दैर्विषाण‚मित्यादि य‚दोपात्त‚स्य रूपं स‚निद‚र्श‚न‚{\tiny $_{lb}$}‚ञ्च‚क्षुर्विज्ञान‚ज‚न‚क‚त्वात् । स‚प्र‚तिघ‚श्च स्व‚देशे प‚रोत्प‚त्तिप्र‚तिब‚न्धात् । एत‚न्ना‚{\tiny $_{lb}$}‚नार्थ‚क्रियाकारिषु वृत्तेरित्येत‚स्य निद‚र्श‚नं । य‚स्मात्त‚च्च‚क्षुर्विज्ञानादि ‚{\tiny $_{2}$}‚ कार्य‚{\tiny $_{lb}$}‚ज‚न‚क‚त्वादेक‚रूप‚म‚पि नानारूपैः स‚निद‚र्श‚नादि श‚ब्दैर्विष‚यीक्रिय‚ते । त‚स्मान्न ते ‚{\tiny $_{lb}$}‚व‚स्तुभेद‚साध‚नायालं । घ‚ट‚श्चेत्येत‚देकार्थ‚क्रियाकारिष्वित्येत‚स्योदा ‚{\tiny $_{3}$}‚ ह‚र‚ण‚न्त‚थाहि ‚{\tiny $_{lb}$}‚ब‚ह‚वो रूप‚ग‚न्ध‚र‚स‚स्प‚र्शा उद‚क‚धार‚ण‚विशेषादिकार्य‚निर्व‚र्त्त‚न‚स‚म‚र्थ‚त्वाद‚भिन्न‚{\tiny $_{lb}$}‚स‚मैस्तै[ः] विष‚य‚त्वेनात्म‚साक्त्रिय‚न्ते । त‚त‚स्ते नाभेदं सा ‚{\tiny $_{4}$}‚ ध‚यितुं क्ष‚माः । ‚{\tiny $_{lb}$}‚त‚च्च[ा]तीतानाग‚त‚श‚श‚विषाणादिषु त‚त्प्र‚तिप‚त्तिर्न व‚स्तु साध‚य‚तीत्य‚तिप्र‚तीत‚{\tiny $_{lb}$}‚मेत‚त् । अथ क‚थ‚मिद‚ङ्ग‚म्य‚ते स‚निद‚र्श‚नादिबुद्धिश‚ब्दा ‚{\tiny $_{5}$}‚ न व‚स्तुभेदं साध‚य‚न्तीत्य‚तः ‚{\tiny $_{lb}$}‚प्राह । \quotelemma{न‚हीत्या} \cite[2b7]{vn-msN} दि । क‚स्मादेवात्र व‚स्तुनि रूपादावुप‚संहारात्स‚निद‚र्श‚नं ‚{\tiny $_{lb}$}‚स‚प्र‚तिघं रूप‚मित्येव स‚मानाधिक‚र‚ण‚त्वादिति ‚{\tiny $_{6}$}‚ याव‚त् । अन्य‚था भिन्नाधिक‚र‚ण‚{\tiny $_{lb}$}‚त्वाद्व‚कुलोत्प‚ल‚क‚म‚ल‚माल‚तीम‚ल्लिकादिश‚ब्दानामेव सामानाधिक‚र‚ण्य‚मेव न ‚{\tiny $_{lb}$}‚ \leavevmode\ledsidenote{\textenglish{26/s}} भ‚वेदिति भावः । \quotelemma{क‚ण‚भ‚क्षाक्ष‚पाद} म‚तानुसारिण ‚{\tiny $_{7}$}‚ स्तु मिथ्याद‚र्श‚नानुराग‚ज‚निता‚{\tiny $_{lb}$}‚स‚द्विक‚ल्प‚म‚लोप‚लिप्तान्त‚र्लोच‚नाः स‚ञ्च‚क्ष‚ते [।] नानाविष‚य‚त्वेप्य‚भ्युप‚ग‚म्य‚माने ‚{\tiny $_{lb}$}‚तेषामेक‚त्रोप‚संहारोऽविरुद्ध एव । त‚न्निमित्तानां ‚{\tiny $_{8}$}‚ स‚निद‚र्श‚नादीनान्त‚त्र रूपादौ ‚{\tiny $_{lb}$}‚स‚म‚वायादिति ।
	{\color{gray}{\rmlatinfont\textsuperscript{§~\theparCount}}}
	\pend% ending standard par
      ‚{\tiny $_{lb}$}‚

	  
	  \pstart \leavevmode% starting standard par
	त‚देत‚त्स‚र्व‚मेषाम‚विचारित‚र‚म‚णीय‚त‚या विचार‚विम‚र्दीक्ष‚म‚त्वात् प‚ण्डित‚ज‚न‚{\tiny $_{lb}$}‚हास‚कारि द‚र्श‚न‚मित्य‚भिप्राय‚वा ‚{\tiny $_{9}$}‚ \leavevmode\ledsidenote{\textenglish{19a/msK}} नाह । \quotelemma{आयासे व‚ताय} मित्या \cite[2b8]{vn-msN}दि [।] ‚{\tiny $_{lb}$}‚व‚त‚श‚ब्दोऽनुकंपायाङ् कासावित्याह । \quotelemma{अनेकं स‚म्ब‚न्धिनं} स‚निद‚र्श‚न‚त्वादिक‚मुप‚{\tiny $_{lb}$}‚कृत्यानुप‚कारे तेन तेषान्त‚त्र स‚म्ब‚न्धित्वायोगादित्य‚भि ‚{\tiny $_{1}$}‚ प्रायः । अनेकं स‚निद‚र्श‚नादि‚{\tiny $_{lb}$}‚ \quotelemma{श‚ब्दं} \cite[2b9]{vn-msN} तेभ्यः स‚म्ब‚न्धिभ्यःश‚कासा\edtext{}{\lemma{कासा}\Bfootnote{? स‚काशा}}दात्म‚नि संमार्ग‚य‚न्निद‚{\tiny $_{lb}$}‚मायास‚प‚त‚ने कार‚णं ।
	{\color{gray}{\rmlatinfont\textsuperscript{§~\theparCount}}}
	\pend% ending standard par
      ‚{\tiny $_{lb}$}‚

	  
	  \pstart \leavevmode% starting standard par
	न‚नु स‚रूपादिभावो यैः श‚क्तिभेदैर‚नेक‚स‚म्ब‚न्धिन ‚{\tiny $_{2}$}‚ मुप‚क‚रोति । तैरेव श‚क्ति‚{\tiny $_{lb}$}‚भेदैर‚नेकं बुद्ध्यादिश‚ब्दं किन्नोत्थाप‚य‚ति । य‚दि पुन‚रेवं भ‚वेत्त‚दाको गुणो ल‚भ्य‚त ‚{\tiny $_{lb}$}‚इत्याह । \quotelemma{एवं ह‚य‚नेन प‚र‚म्प‚रानुसार‚श्र‚मः प‚रिहृ ‚{\tiny $_{3}$}‚ तो भ‚व‚ती} \cite[2b9]{vn-msN} ति । श‚क्ति‚{\tiny $_{lb}$}‚भेदैः स‚म्ब‚न्धिन‚मुप‚क‚रोति तेभ्य‚श्च श‚ब्दाः प्र‚व‚र्त‚न्त इत्य‚य‚म्प‚र‚म्प‚रानुस‚र‚णाया‚{\tiny $_{lb}$}‚सोऽनेन‚त‚प‚श्विना\edtext{}{\lemma{श्विना}\Bfootnote{? त‚प‚स्विना}}रूपादिना त्य‚क्तो भ‚व‚तीत्य‚र्थः । 
	{\color{gray}{\rmlatinfont\textsuperscript{§~\theparCount}}}
	\pend% ending standard par
      ‚{\tiny $_{lb}$}‚

	  
	  \pstart \leavevmode% starting standard par
	न‚नु च प्र‚तिनिय‚तोपि कार्य‚श‚क्तिम‚न्तः स‚र्व्व एव भावास्त्व‚याप्येत‚द‚व‚स्य\edtext{}{\lemma{स्य}\Bfootnote{‚{\tiny $_{lb}$}‚? श्य}}मेवाभ्युपेय‚म‚न्य‚था क‚स्माच्छालिबीजंसा\edtext{}{\lemma{स्माच्छालिबीजंसा}\Bfootnote{? शा}}ल्य‚ङ्कुर‚मेवोत्पाद‚य‚ति न ‚{\tiny $_{lb}$}‚य‚वाङ्कुर‚मिति ‚{\tiny $_{5}$}‚ प‚रेणाभियुक्तेन किम‚भिधानीयं भाव‚प्र‚कृतिं मुक्त्वा [।] त‚स्मात्त‚व ‚{\tiny $_{lb}$}‚प‚दार्थ‚प्र‚कृतिस‚माश्र‚य‚ण‚मेव श‚र‚ण‚म‚न्य‚थास्य दोष‚स्य प‚रिह‚र्त्तुम‚श‚क्य‚त्वात् ‚{\tiny $_{6}$}‚ [।] ‚{\tiny $_{lb}$}‚एत‚च्च न म‚मापि राज‚कुल‚निवारितं । त‚थाहि श‚क्य‚मे[त]त्म‚याप्य‚भिधातुम‚नेक‚{\tiny $_{lb}$}‚स‚म्ब‚न्ध्युप‚कार एव त‚स्य साम‚र्थ्यं नानैक‚श‚ब्दोत्थाप‚न‚मिति चेत् । स‚त्य‚मेव‚मेत ‚{\tiny $_{7}$}‚ त् । ‚{\tiny $_{lb}$}‚एव‚न्तु म‚न्य‚ते । न ताव‚त् स‚निद‚र्श‚न‚त्वाद‚यः स‚न्ति । क्र‚म‚यौग‚प‚द्याभ्याम‚र्थ‚क्रिया त्व‚नु‚{\tiny $_{lb}$}‚प्र‚योगात् । उप‚ल‚ब्धिल‚क्ष‚ण‚प्राप्तानाञ्चानुप‚ल‚म्भात् न चोप‚ल‚ब्धिल‚क्ष ‚{\tiny $_{8}$}‚ ण‚प्राप्तं ‚{\tiny $_{lb}$}‚स‚द‚नुप‚ल‚भ्य‚मान‚म‚स्तीति श‚क्य‚ते व‚क्तुम‚तिप्र‚स‚ङ्गात् । अनुप‚ल‚ब्धिल‚क्ष‚ण‚प्राप्त‚तायां ‚{\tiny $_{lb}$}‚ \leavevmode\ledsidenote{\textenglish{27/s}} वा क‚थं त‚न्निब‚न्ध‚नाः प्र‚त्य‚य‚व्य‚प‚देशाः प्र‚व‚र्त्त‚न्ते गोग‚व‚यादिषु [।] ‚{\tiny $_{9}$}‚ \leavevmode\ledsidenote{\textenglish{19b/msK}} एतेनोप‚ल‚ब्धा- ‚{\tiny $_{lb}$}‚नाम‚पि क्ष‚णिक‚त्वादिव‚त् व्य‚क्तिव्य‚तिरिक्तेणा\edtext{}{\lemma{तिरिक्तेणा}\Bfootnote{? ना}}नुप‚ल‚क्ष‚णं प्र‚त्य‚क्षं य‚स्मात्सा‚{\tiny $_{lb}$}‚मान्यं य‚दि दृष्ट‚म‚प्य‚विक‚लं भिन्नं न संल‚क्ष‚ते । भावे त‚द्ब‚ल‚भाविनी भ‚व‚ति ‚{\tiny $_{lb}$}‚सा ‚{\tiny $_{1}$}‚ या श‚ब्द‚वृत्तिः क‚थं । द‚ण्ड्यादौ न निब‚न्ध‚न‚स्य न ग‚तौ धीश‚ब्द‚योर‚स्ति सा ‚{\tiny $_{lb}$}‚त‚स्माद‚स्य क‚थ‚ञ्चिदेव त‚द‚पि ते युक्त्या न स‚ङ्ग‚च्छ‚ते ‚{\tiny $_{1}$}‚ ।
	{\color{gray}{\rmlatinfont\textsuperscript{§~\theparCount}}}
	\pend% ending standard par
      ‚{\tiny $_{lb}$}‚

	  
	  \pstart \leavevmode% starting standard par
	किञ्च ॥
	{\color{gray}{\rmlatinfont\textsuperscript{§~\theparCount}}}
	\pend% ending standard par
      ‚{\tiny $_{lb}$}‚
	  \bigskip
	  \begingroup
	
	    
	    \stanza[\smallbreak]
	  \flagstanza{\tiny\textenglish{...6}}{\normalfontlatin\large ``\qquad}भावानामैक‚देश्यं प्र‚स‚ज‚ति भ‚व‚तो ‚{\tiny $_{2}$}‚ द‚र्श‚ने स‚र्व‚थैषां&‚{\tiny $_{lb}$}‚स‚त्तादेसा\edtext{}{\lemma{त्तादेसा}\Bfootnote{? शा}}द‚भेदात्स‚कृदिद‚म‚थ‚वा भिन्न‚देशे निवृत्तं ।&‚{\tiny $_{lb}$}‚वृत्तौ वानेक‚मेत‚न्न‚हि भ‚व‚ति स‚कृत्स‚र्व्व‚था वृत्तिभाजां&‚{\tiny $_{lb}$}‚तालादीनां फ‚लानां ब‚हुषु ब‚हुवि‚{\tiny $_{3}$}‚धेष्वाश्र‚येष्वेक‚भावः ॥ [६]{\normalfontlatin\large\qquad{}"}\&[\smallbreak]
	  
	  
	  
	  \endgroup
	‚{\tiny $_{lb}$}‚

	  
	  \pstart \leavevmode% starting standard par
	स‚त्वे वा तेषान्न बुद्धिश‚ब्दोत्थाप‚न‚साम‚र्थ्य‚म‚न्व‚य‚व्य‚तिरेकाभ्याम‚व‚धार्य‚ते । ‚{\tiny $_{lb}$}‚अत एवानुप‚ल‚भ्य‚मान‚त्वात्पाच‚कादिष्व‚पि च ‚{\tiny $_{4}$}‚ त‚द्व्य‚तिरेकेणापि तेषाम्भावात् । ‚{\tiny $_{lb}$}‚उप‚पादित‚ञ्चैत‚द् \href{http://sarit.indology.info/?cref=pv.1.160}{प्र‚माण–व‚र्तिके १।१६० ।}
	{\color{gray}{\rmlatinfont\textsuperscript{§~\theparCount}}}
	\pend% ending standard par
      ‚{\tiny $_{lb}$}‚
	  \bigskip
	  \begingroup
	
	    
	    \stanza[\smallbreak]
	  \flagstanza{\tiny\textenglish{...7}}{\normalfontlatin\large ``\qquad}पाच‚कादिष्व‚भिन्नेन विनाप्य‚र्थेन वाच‚कः ।&‚{\tiny $_{lb}$}‚भेदान्न हेतुः क‚र्मास्य न जातिः क‚र्म‚संश्र‚{\tiny $_{5}$}‚या [॥ ७]{\normalfontlatin\large\qquad{}"}\&[\smallbreak]
	  
	  
	  
	  \endgroup
	‚{\tiny $_{lb}$}‚

	  
	  \pstart \leavevmode% starting standard par
	इत्यादिनेति नेहोच्य‚ते । न च तेषामुप‚कार्य‚त्व‚म‚स्ति नित्य‚त‚याऽनाधेयातिश‚य‚त्वात् । ‚{\tiny $_{lb}$}‚न च पुरुषाभिप्रायान‚पेक्षो व्य‚क्त्युप‚कृत‚स‚प्र‚तिघ‚त्वा ‚{\tiny $_{6}$}‚ दिसामान्य‚साम‚र्थ्य‚भावे‚{\tiny $_{lb}$}‚नोप‚नेयो विव‚क्षायान्तु संमुखीभावेपि प्र‚वृत्तिप्राप्तः । पुरुषाभिप्रायानुरूपे वा स ए‚{\tiny $_{lb}$}‚वास्तु नियाम‚कः किम‚न्त‚र्गंतृभिः सामान्यैः ‚{\tiny $_{7}$}‚ [।] त‚थाहि त‚द्ग‚तान्व‚य‚व्य‚तिरेकानुविधान ‚{\tiny $_{lb}$}‚मेव ल‚क्ष्य‚ते सामान्यानां न नित्यानाम‚व्य‚तिरेक‚त्वात् ।
	{\color{gray}{\rmlatinfont\textsuperscript{§~\theparCount}}}
	\pend% ending standard par
      ‚{\tiny $_{lb}$}‚

	  
	  \pstart \leavevmode% starting standard par
	त‚स्मात् । स‚र्व्व‚मेत‚त् कुद‚र्श‚न‚स‚माश्र‚येण क‚ल्प‚नामात्रं । क‚ल्प‚ना च ‚{\tiny $_{8}$}‚ सैव ‚{\tiny $_{lb}$}‚क‚र्त‚व्या या पुन‚र्न्न प‚र्य‚नुयोग‚म‚र्ह‚ति त‚त्र‚त्यां क‚ल्प‚नायां व‚र‚मेव क‚ल्प‚नादोषाभावात् । ‚{\tiny $_{lb}$}‚गुण‚स‚द्भावाच्चेति ।सात्म‚त\edtext{}{\lemma{त}\Bfootnote{? स्यान्म‚तं}}युक्तैवेयं क‚ल्प‚ना । न‚हि एक‚स्य नाना ‚{\tiny $_{lb}$}‚सु\edtext{}{\lemma{सु}\Bfootnote{? र्थ}}क ‚{\tiny $_{9}$}‚ \leavevmode\ledsidenote{\textenglish{20a/msK}} श‚ब्दोत्थाप‚ने साम‚र्थ्य‚म‚स्तीति । य‚द्येव‚म‚त्रापि ब्रूम इत्याह । नानाश‚ब्दो- ‚{\tiny $_{lb}$}‚त्थाप‚नासाम‚र्थ्ये \quotelemma{नानास‚म्ब‚न्ध्युप‚कारोपि माभूदि} \cite[2b10]{vn-msN} ति । एक‚स्यानेकोप‚कार‚{\tiny $_{lb}$}‚क‚त्व‚विरोधाभ्युप‚ग‚मादित्य‚भिस‚न्धिः । नित्य‚त्वात्स‚म्ब‚न्धिनाम‚नुप‚कारोऽभ्युपेत एवेति ‚{\tiny $_{lb}$}‚चेदाह । \quotelemma{अनुप‚कारे हि तेषां} \cite[2b10]{vn-msN} स‚निद‚र्श‚न ‚{\tiny $_{2}$}‚ त्वादीनान्तेनाङ्गी \quotelemma{क्रिय‚माणे ‚{\tiny $_{lb}$}‚त‚त्स‚म्ब‚न्धिता न सिध्य‚ति} । न‚हि यो येन नोप‚क्रिय‚ते हेदुः स त‚स्य स‚म्व‚न्धियुक्तो ‚{\tiny $_{lb}$}‚ \leavevmode\ledsidenote{\textenglish{28/s}} \quotelemma{हिम‚वा} निव \quotelemma{म‚ल‚य‚गिरेरिति} भावः । एव‚न्ताव‚त्प‚र ‚{\tiny $_{3}$}‚ प‚क्ष‚निराक‚र‚णेन स‚निद‚र्श‚नादि‚{\tiny $_{lb}$}‚श‚ब्दानाम‚भिन्न‚विष‚य‚त्वं साधितं । त एव ज‚डिम्नः प‚द‚मुद्व‚ह‚न्तः पुन‚र‚पि प‚र्य‚{\tiny $_{lb}$}‚नुयुञ्ज‚ते ।
	{\color{gray}{\rmlatinfont\textsuperscript{§~\theparCount}}}
	\pend% ending standard par
      ‚{\tiny $_{lb}$}‚

	  
	  \pstart \leavevmode% starting standard par
	न‚नु भ‚व‚तु नाम ‚{\tiny $_{4}$}‚ स‚निद‚र्श‚नादिश‚ब्दानाम‚भिन्न‚विष‚य‚त्वं । अथ क‚थ‚म‚व‚सीय‚ते । ‚{\tiny $_{lb}$}‚घ‚ट‚प‚टादिश‚ब्दानाम‚नेकार्थ‚विष‚य‚त्व‚मिति याव‚ता रूपादिव्य‚तिरिक्त ‚{\tiny $_{5}$}‚ म‚न्य‚दे‚{\tiny $_{lb}$}‚वाव‚य‚वि द्र‚व्य‚म‚स्ति । त‚देव च घ‚ट‚प‚टादिश‚ब्दैर्विष‚यीक्रिय‚ते । त‚थाहि विचार‚{\tiny $_{lb}$}‚विष‚याप‚न्नः प‚ट‚स्त‚न्तुभ्यो व्य‚तिरिच्य‚ते भिन्न‚क‚र्तृक‚त्वात् ‚{\tiny $_{6}$}‚ घ‚टादिव‚त् । त‚था ‚{\tiny $_{lb}$}‚स‚म‚स्त‚व्य‚स्त‚प्र‚त्य‚याविष‚य‚त्वाद् ग‚वादिव‚त् । न‚हि त‚न्त‚वः त‚न्तुस‚मुदाय इति वा प‚टे ‚{\tiny $_{lb}$}‚प्र‚त्य‚यो दुष्टः । उपायान्त‚र‚साध्य‚त्त्वाच्च घ‚टादिव‚त् । भिन्न‚देशाव‚स्थितैश्च क्रिय‚{\tiny $_{lb}$}‚माण‚त्त्वात् । घ‚टादिव‚देव भिन्न‚प‚रिमाण‚त्त्वाच्च । व‚कुलाम‚ल‚क‚बिम्बादिव‚त् । अत‚श्च ‚{\tiny $_{lb}$}‚ \leavevmode\ledsidenote{\textenglish{20b/msK}} पूर्व्वोत्त‚र काल‚भावित्त्वाद् बीजाङ्कुरादिव‚त् । अथ‚वा प‚टाद‚न्ये त‚न्त‚व‚स्त‚त्कार‚ण‚त्त्वात् ‚{\tiny $_{lb}$}‚तुर्यादिव‚त् । त‚न्तुप‚ट‚योर्वाऽन्य‚त्त्वं भिन्न‚श‚क्तिम‚त्त्वात् ज‚लान‚लादिव‚त् । त‚थेद‚म‚{\tiny $_{lb}$}‚प‚र‚म्विचार‚विष‚याप‚न्न‚मिन्दीव‚र‚ङ्ग‚न्धादि ‚{\tiny $_{1}$}‚ भ्योऽत्य‚न्त‚भिन्न‚न्तेषाम्व्य‚व‚च्छेद‚क‚त्वात् । ‚{\tiny $_{lb}$}‚चैत्रादिव‚त् । इह य‚द्य‚स्य व्य‚व‚च्छेद‚क‚न्त‚त्त‚स्माद‚न्य‚त्त‚द्य‚था गोपिण्डाच्चैत्र ‚{\tiny $_{lb}$}‚इत्येतानि त‚द्व्य‚तिरेक‚साध‚न‚प्र‚माणानि स‚न्ति । त‚त्क ‚{\tiny $_{1}$}‚ थ‚न्तेषाम्भिन्न‚विष‚य‚त्व‚म्भ‚वि‚{\tiny $_{lb}$}‚ष्य‚तीति ॥ ० ॥
	{\color{gray}{\rmlatinfont\textsuperscript{§~\theparCount}}}
	\pend% ending standard par
      ‚{\tiny $_{lb}$}‚

	  
	  \pstart \leavevmode% starting standard par
	त‚देत‚द‚प्येषाम‚स‚द्द‚र्श‚नाभिनिवेश‚प‚ट‚ल‚प्र‚च्छादितान्तःक‚र‚णानां नाल्पीय‚स‚स्त‚म‚सो ‚{\tiny $_{lb}$}‚दुर्विल‚सित‚मित्यागूर्या ‚{\tiny $_{2}$}‚ ह । \quotelemma{घ‚ट इत्य‚पि च रूपाद‚य एवैकार्थ‚क्रियाकारिण ‚{\tiny $_{lb}$}‚एक‚श‚ब्द‚वाच्या भ‚व‚न्तु किम‚र्थान्त‚र‚क‚ल्प‚न‚ये} \cite[2b10]{vn-msN} ति कार्य‚मित्युप‚स्क्रिय‚ते । नैव ‚{\tiny $_{lb}$}‚किञ्चित्त‚स्य तादृश‚स्य नीलादि ‚{\tiny $_{3}$}‚ व्य‚तिरेकेणानुप‚ल‚क्ष‚णादित्याकूत‚म‚स्य । यानि त्वे‚{\tiny $_{lb}$}‚तानि त‚त्प्र‚तिपाद‚नाय प्र‚माणान्युक्तानि तान्य‚सिद्ध‚तादिदोष‚दुष्ट‚त्वान्नालं त‚द्भेद‚{\tiny $_{lb}$}‚साध‚नायेति भो ‚{\tiny $_{4}$}‚ ता \edtext{}{\lemma{ता}\Bfootnote{?}} नामेव पुर‚तः छायान्द‚ध‚तीति म‚न्य‚ते । त‚थाहि नेदं ताव‚{\tiny $_{lb}$}‚दाद्यं प्र‚माणं प‚रीक्ष्य‚माणं पूर्व्वाम‚पि प‚रीक्षां क्ष‚म‚ते । य‚तोत्र विक‚ल्प‚द्व‚य‚माविर्भ‚व‚ति । ‚{\tiny $_{lb}$}‚अ ‚{\tiny $_{5}$}‚ न्याव‚स्थाव‚स्थितेभ्यो वा त‚न्तुभ्यः प‚ट‚स्यान्य‚त्वं साध्य‚ते विशिष्ट‚संस्थानाव‚स्थिते‚{\tiny $_{lb}$}‚भ्यो वेति । त‚त्र न ताव‚द‚य‚माद्यः प्र‚कारः स‚ह‚ते विचार‚भार‚गौ ‚{\tiny $_{6}$}‚ र‚वं । सिद्ध‚साध‚न‚ता‚{\tiny $_{lb}$}‚दोषानुष‚ङ्गात् । य‚स्मात्स‚म‚धिग‚त‚स‚म‚स्त‚व‚स्तुयाथात‚थ्य‚सुग‚त‚स‚म‚य‚न‚य‚स‚माश्र‚य‚{\tiny $_{lb}$}‚प्र‚वृत्तिब‚लासादिताव‚दात‚म‚त‚यः प्र‚स‚वान‚न्त‚र‚नि ‚{\tiny $_{7}$}‚ रोध‚भाजः स‚र्व्व‚भावाइति प्र‚क‚ल्प‚{\tiny $_{lb}$}‚य‚न्ति । त‚त‚श्च तेभ्योन्य‚त्त्व‚मिष्ट‚मेवेति सिद्ध‚साध्य‚ताप्र‚स‚ङ्गोप‚निपात‚पिशाचः ‚{\tiny $_{lb}$}‚क‚थ‚मिव भ‚व‚न्तं न गृह‚णाति । द्वितीयोपि विक ‚{\tiny $_{8}$}‚ ल्पः तीव्रान‚लोप‚त‚प्त इवोप‚ल‚त‚ले ‚{\tiny $_{lb}$}‚त‚लानि पादानां न प्र‚तिष्ठां स‚मासाद‚प\edtext{}{\lemma{प}\Bfootnote{? यं}}ति । हेतोः प‚रं प्र‚त्य‚सिद्ध‚त्वात् । ‚{\tiny $_{lb}$}‚न‚हि विशिष्ट‚स्थानाव‚स्थितेभ्यः प‚ट‚स्य भिन्न‚क‚र्तृक‚त्वं प‚रं ‚{\tiny $_{9}$}‚ \leavevmode\ledsidenote{\textenglish{21a/msK}} प्र‚ति सिद्ध‚प‚द्ध‚तिम‚व‚त‚र‚ति ‚{\tiny $_{lb}$}‚ \leavevmode\ledsidenote{\textenglish{29/s}} योहि तादृक्प्र‚कारेभ्योन्य‚त्व‚म‚भावादेव नाभ्युपैति स क‚थ‚मिव भिन्न‚क‚र्तृक‚त्व‚{\tiny $_{lb}$}‚म‚भ्युप‚ग‚मिष्य‚तीति । त‚द‚न‚न्त‚राभिहित‚म‚पि प्र‚मा ‚{\tiny $_{1}$}‚ णेन स‚म‚भिल‚सित\edtext{}{\lemma{सित}\Bfootnote{? षित}}‚{\tiny $_{lb}$}‚म‚नोर‚थ‚प‚रिपूर‚णायालं हेत्व‚सिद्धेः । य‚तः प‚ट इति त‚न्तुष्वेव स‚न्निवेश‚विशेषेणा‚{\tiny $_{lb}$}‚व‚स्थितेषु प्र‚त्य‚यो व‚र्त‚ते । त‚द्विविक्त‚रूप‚स्यात्य‚न्त‚मु ‚{\tiny $_{2}$}‚ न्मिषित‚च‚क्षुषाप्य‚द‚र्श‚नात्स्फ‚{\tiny $_{lb}$}‚टिकादौ दृष्ट‚मिति चेत् । एत‚दुत्त‚र‚त्र निषेत्स्यामः । य‚त्त्विद‚मुपायान्त‚र‚साध्य‚त्वाद् ‚{\tiny $_{lb}$}‚भिन्न‚देशाव‚स्थितैः क्रिय‚माण‚त्वात् भिन्न‚प‚रिमा ‚{\tiny $_{3}$}‚ ण‚त्वात्पूर्वोत्त‚र‚काल‚भावित्त्वाद् ‚{\tiny $_{lb}$}‚भिन्न‚श‚क्तिम‚त्वाच्चेति ॥
	{\color{gray}{\rmlatinfont\textsuperscript{§~\theparCount}}}
	\pend% ending standard par
      ‚{\tiny $_{lb}$}‚

	  
	  \pstart \leavevmode% starting standard par
	अत्र प्र‚थ‚म‚साध‚नाभिहित‚विक‚ल्प‚दोष‚स्तीव्राम‚र्श‚विर‚क्त‚लोच‚न इवारातिस्त‚{\tiny $_{lb}$}‚त्स‚म्प‚द‚न्न स ‚{\tiny $_{4}$}‚ ह‚ते ॥
	{\color{gray}{\rmlatinfont\textsuperscript{§~\theparCount}}}
	\pend% ending standard par
      ‚{\tiny $_{lb}$}‚
	  \bigskip
	  \begingroup
	
	    
	    \stanza[\smallbreak]
	  \flagstanza{\tiny\textenglish{...8}}{\normalfontlatin\large ``\qquad}प्र‚थ‚मे सिद्ध‚साध्य‚त्वं द्वितीये हेत्व‚सिद्ध‚ता ।&‚{\tiny $_{lb}$}‚क्ष‚णिक‚त्वाद्विशिष्टानामुत्पादोभिम‚तो य‚त [ः ॥ ८]{\normalfontlatin\large\qquad{}"}\&[\smallbreak]
	  
	  
	  
	  \endgroup
	‚{\tiny $_{lb}$}‚

	  
	  \pstart \leavevmode% starting standard par
	इति स‚ङ्ग्र‚ह‚श्लोकः । ‚{\tiny $_{lb}$}‚य‚च्चेद‚मुक्तं विचार‚विष‚या पं ‚{\tiny $_{5}$}‚ न \edtext{}{\lemma{न}\Bfootnote{? प‚न्न}}मिन्दीव‚र‚मित्यादि । त‚द‚पि न स‚ङ्ग‚{\tiny $_{lb}$}‚च्छ‚ते । य‚स्मादिन्दीव‚र‚स्य ग‚न्धाद‚यइतीन्दिव‚र\edtext{}{\lemma{र}\Bfootnote{? इतीन्दीव‚र}}स्व‚भावा ग‚न्धाद‚यो ‚{\tiny $_{lb}$}‚म‚धुभाव‚नाविशेषादिकार्य‚निव‚र्त्त‚न\edtext{}{\lemma{न}\Bfootnote{? निर्व‚र्त्त‚न}}स ‚{\tiny $_{6}$}‚ म‚र्था इति याव‚त् । अवि‚{\tiny $_{lb}$}‚शिष्ट‚कार्य‚साध‚नात्म‚ना सामान्य‚भूत‚ग‚न्धादिश‚ब्दैः प्र‚सिद्धाविशिष्ट‚कार्य‚साध‚ना‚{\tiny $_{lb}$}‚ख्येन विशेषेण ये विशिष्टास्त एव‚मुच्य‚न्ते । न पुन‚र‚त्रान्य‚त् ‚{\tiny $_{7}$}‚ किञ्चिदित्य‚र्था‚{\tiny $_{lb}$}‚व‚र्णित‚ल‚क्ष‚णं द्र‚व्य‚म‚स्ति त‚स्य तादृशोऽनुप‚ल‚ब्धेरित्युक्त‚प्रायं । त‚था चानेन प्र‚कारेण ‚{\tiny $_{lb}$}‚तेषान्त‚द्व्य‚व‚च्छेद‚कं भ‚व‚तीति । तेषान्त‚द्व्य‚च्छेद‚कं च न चात्य‚न्तं ‚{\tiny $_{8}$}‚ भिन्न‚मिति ‚{\tiny $_{lb}$}‚कोऽन‚योर्विरोध इति । स‚न्दिग्ध‚विप‚क्ष‚व्यावृत्तिको हेतुः \quotelemma{प्र‚तिब‚न्धासिद्धः} । न‚हि दृष्टान्त‚{\tiny $_{lb}$}‚मात्रास्तिद्धिर‚स्ति । स‚र्व्व‚सिद्धिप्र‚स‚ङ्गात् । अपि च शिलापु ‚{\tiny $_{9}$}‚ \leavevmode\ledsidenote{\textenglish{21b/msK}} त्र‚क‚स्य श‚रीरं । ‚{\tiny $_{lb}$}‚रूप‚स्य स्व‚भाव इत्य‚त्रापि शिलापुत्र‚क‚रूप‚योः श‚रीर‚स्व‚भाव‚व्य‚व‚च्छेद‚क‚त्व‚म‚स्तीति ‚{\tiny $_{lb}$}‚भेद‚स्त‚योर‚पि त‚तः प्र‚स‚ज‚ते न च भ‚व‚ति । नैःस्वाभाव्य‚प्र‚स ‚{\tiny $_{1}$}‚ ङ्गात् । त‚स्मा‚{\tiny $_{lb}$}‚द‚य‚मेतेनानैकान्तिकः स्फुट‚मेव भ‚व‚द्भिर‚भिधानीयः । किञ्चेद‚म‚तिविक‚लैर्मि‚{\tiny $_{lb}$}‚थ्याद‚र्श‚न‚संराग‚पिशाचाविष्ट‚बुद्धिभिः किमिन्दीव‚र‚स्य ग‚न्धाद‚य इत्येते श ‚{\tiny $_{2}$}‚ ब्दा[ः] ‚{\tiny $_{lb}$}‚पुरुषाभिप्राय‚व्यापार‚निर‚पेक्षा एव व‚स्तुत‚त्व‚निब‚न्ध‚नाः प्र‚व‚र्त्त‚न्ते । किम्वा य‚थैव तैः ‚{\tiny $_{lb}$}‚प्र‚युज्य‚न्ते त‚थैव व‚स्तुत‚त्व‚म‚न‚पेक्ष्य त‚म‚र्थ‚म‚स‚त्कारेण प्र‚तिपाद‚य‚न्तीति [।] य ‚{\tiny $_{3}$}‚ द्याद्यः ‚{\tiny $_{lb}$}‚प‚क्ष‚स्त‚दा स‚दाध्व‚न‚न‚प्र‚स‚ङ्गोऽतीतादिष्व‚न्य‚त्र च पुरुषे च्छाव‚सा\edtext{}{\lemma{सा}\Bfootnote{? व‚शा}}न्नियो‚{\tiny $_{lb}$}‚ज‚न‚न्न‚भ‚वेत् । न च प्र‚व‚च‚नान्त‚र‚भेदेष्व‚र्थेषु प्र‚वृत्तिः प्राप्नोति । न च क ‚{\tiny $_{4}$}‚ स्याश्चि‚{\tiny $_{lb}$}‚द्वाचोऽस‚त्यार्थ‚ता स्यात् । अथोत्त‚र‚स्त‚दा । ‚{\tiny $_{lb}$}‚ 
	    \pend% close preceding par
	  
	    
	    \stanza[\smallbreak]
	  \flagstanza{\tiny\textenglish{...9}}{\normalfontlatin\large ``\qquad}येषाम्व‚स्तुव‚सा\edtext{}{\lemma{सा}\Bfootnote{? व‚शा}}वाचो न विव‚क्षाप‚राश्र‚या[ः] ।&‚{\tiny $_{lb}$}‚\leavevmode\ledsidenote{\textenglish{30/s}}ष‚ष्ठिव‚च‚न‚भेदादि चोद्य‚न्तान्प्र‚ति युक्तिम‚त् ॥ [९]{\normalfontlatin\large\qquad{}"}\&[\smallbreak]
	  
	  
	  
	    \pstart  \leavevmode% new par for following
	    \hphantom{.}
	   ‚{\tiny $_{lb}$}‚य‚दाहुः ॥
	{\color{gray}{\rmlatinfont\textsuperscript{§~\theparCount}}}
	\pend% ending standard par
      ‚{\tiny $_{lb}$}‚
	  \bigskip
	  \begingroup
	
	    
	    \stanza[\smallbreak]
	  \flagstanza{\tiny\textenglish{...10}}{\normalfontlatin\large ``\qquad}य‚द्य ‚{\tiny $_{5}$}‚ था वाच‚क‚त्वेन व‚क्तृभिर्विनिय‚म्य‚ते ।&‚{\tiny $_{lb}$}‚अन‚पेक्षित‚वाह‚यार्थ‚न्त‚त्त‚थावाच‚क‚म्व‚चः । [१०]{\normalfontlatin\large\qquad{}"}\&[\smallbreak]
	  
	  
	  
	  \endgroup
	‚{\tiny $_{lb}$}‚

	  
	  \pstart \leavevmode% starting standard par
	त‚दा न पुरुषेच्छाब‚ल‚प्र‚वृत्त‚श‚ब्द‚व‚सा\edtext{}{\lemma{सा}\Bfootnote{? व‚शा}}द‚र्थ‚त‚त्वं व्य‚व‚तिष्ठ‚त इति त‚द ‚{\tiny $_{lb}$}‚व‚स्थं ‚{\tiny $_{6}$}‚ स‚न्दिग्ध‚विप‚क्ष‚व्य‚तिरेक‚त्वं हेतोरिति । एतेनैत‚द‚पि प्र‚त्युक्तं विप्र‚तिप‚त्ति‚{\tiny $_{lb}$}‚विष‚याप‚न्नाच्च‚न्द‚नाद‚न्ये रूप‚र‚स‚ग‚न्ध‚स्प‚र्शा हेय‚त्वाद‚य‚श्च\edtext{}{\lemma{श्च}\Bfootnote{? य‚श्चे}}ति प्र‚तिजानी‚{\tiny $_{lb}$}‚म‚हे न ‚{\tiny $_{7}$}‚ व्य‚प‚दिश्य‚माण\edtext{}{\lemma{माण}\Bfootnote{? न}}त्वात् । शिलातुलाढ‚क‚प्र‚सेविकाव‚दिति । त‚स्मात्त‚द्भाव‚{\tiny $_{lb}$}‚प्र‚तिपाद‚नाय न किञ्चित्प्र‚माण‚म‚स्तीति स्थित‚मेत‚त् । अस्माक‚न्तु त‚द‚भाव‚प्र‚माण‚{\tiny $_{lb}$}‚साध‚कं प्र‚मा ‚{\tiny $_{8}$}‚ \leavevmode\ledsidenote{\textenglish{22a/msK}} ण‚मेत‚त् । ये प‚र‚स्प‚र‚व्याव‚र्त्त‚मान‚स्व‚भावाव‚स्थितिस‚मालिङ्गित ‚{\tiny $_{lb}$}‚स‚रीरा\edtext{}{\lemma{रीरा}\Bfootnote{? श‚रीरा}}स्ते व्य‚तिरिक्ताव‚य‚विद्र‚व्यानुग‚त‚मूर्त्तिमात्मातिश‚यं नात्म‚{\tiny $_{lb}$}‚सात्\edtext{}{\lemma{सात्}\Bfootnote{? शात्}}कुर्व‚न्ति । य‚था ब‚ह‚वो भ‚स्माधा ‚{\tiny $_{1}$}‚ र‚न \edtext{}{\lemma{न}\Bfootnote{?}} लालाथूकाद‚य‚स्त‚था ‚{\tiny $_{lb}$}‚च य‚थोप‚दिष्ट‚ध‚र्म‚व‚न्त‚स्त‚न्त्वाद‚य इति स्व‚भाव‚हेतुः । वैध‚र्म्येण न‚भःप‚ङ्क‚जाद‚य‚स्ते‚{\tiny $_{lb}$}‚षान्नि[ः]स्व‚भाव‚त्वात् । प‚र‚स्प‚र‚व्याव‚र्त्त‚मानानाम‚पि ‚{\tiny $_{1}$}‚ य‚द्येक‚स्व‚भावान‚भ्युप‚ग‚मे त‚स्य ‚{\tiny $_{lb}$}‚तेषु स‚र्व्वात्म‚नाऽन्य‚था वा वृत्त्य‚योगो बाध‚क‚म्प्र‚माणं । कुत‚स्त‚द्धि युग‚प‚द‚नेक‚त्र ‚{\tiny $_{lb}$}‚स‚र्व्वात्म‚ना व‚र्त्त‚मान‚म‚नेकाधार‚स्थिताधेय‚व ‚{\tiny $_{2}$}‚ द‚नेक‚त्त्व‚मात्म‚नोऽनुमाप‚य‚तीति ‚{\tiny $_{lb}$}‚क‚थ‚म‚स्याभिन्न‚स्व‚भाव‚ता योज्य‚ते । एकाव‚य‚वोप‚ल‚म्भ‚वेलायाञ्च स‚क‚ल‚स्य त‚त्र ‚{\tiny $_{lb}$}‚प‚रिस‚माप्त‚त्वादुप‚ल‚ब्धिप्र‚स‚ङ्गः । अनेकाव ‚{\tiny $_{3}$}‚ य‚वोप‚ल‚ब्धिद्वारेणोप‚ल‚म्भ‚क‚ति‚{\tiny $_{lb}$}‚प‚याव‚य‚व‚द‚र्श‚नेपि स्यात् स‚म‚स्ताव‚य‚वोप‚ल‚म्भ‚द्वारेण उप‚ल‚ब्धौ स‚र्व्व‚काल‚म‚द‚र्श‚न‚{\tiny $_{lb}$}‚प्र‚स‚ङ्गः । त‚स्याम्भास्व‚र‚म‚ध्य ‚{\tiny $_{4}$}‚ भागानां स‚कृद‚नुप‚ल‚म्भात् । एकाव‚य‚व‚क‚म्पे च ‚{\tiny $_{lb}$}‚स‚र्व्व‚क‚म्पादिप्र‚स‚ङ्ग‚श्च वाच्यः । नाप्येक‚देशेन साव‚य‚व‚त्व‚प्र‚स‚ङ्गात् । एक‚देशा‚{\tiny $_{lb}$}‚नाञ्चान‚व‚स्थाप्र‚स‚ङ् ‚{\tiny $_{5}$}‚ गात् । तेपि हि त‚स्याव‚य‚वा इति पाण्य‚व‚य‚व‚वृत्तेष्व‚पि अन्येन ‚{\tiny $_{lb}$}‚व‚र्त्तित‚व्य‚मित्यादिना त‚द‚न्यैक‚देशाभाव‚वानेकः क‚श्चिद‚व‚य‚वी विद्य‚ते । त‚था चा ‚{\tiny $_{6}$}‚ ‚{\tiny $_{lb}$}‚ण्वादिस‚मुदाय एवास्तु कोनुरोधः स्वात्म‚भूतेष्व‚व‚य‚वेष्विति । न वा क्व‚चिद‚प्य‚सौ ‚{\tiny $_{lb}$}‚वृत्तो न ह्येक‚देशाः प्र‚त्येक‚म‚व‚य‚वीत्य‚लं प्र‚तिष्ठित‚मिथ्याप्र‚लापैरिति विर‚म्य ‚{\tiny $_{7}$}‚ ते ।
	{\color{gray}{\rmlatinfont\textsuperscript{§~\theparCount}}}
	\pend% ending standard par
      ‚{\tiny $_{lb}$}‚

	  
	  \pstart \leavevmode% starting standard par
	त‚देव‚मेत‚त् प‚र‚म‚त‚म‚ल‚मालोच्य‚मान‚तीव्र‚त‚रार्क्क‚र‚श्मिसंपात‚योगिहिम‚शैल‚शिला‚{\tiny $_{lb}$}‚श‚क‚ल‚व‚द्विल‚य‚मुप‚यातीति म‚न्य‚मानः प्राह । \quotelemma{किम‚र्थान्त‚र‚क‚ल्प‚न} \cite[3a1]{vn-msN} ‚{\tiny $_{8}$}‚ येति । ‚{\tiny $_{lb}$}‚स्यादिय‚त्त‚राशाप‚र‚स्य नैवानेक‚स्यैकार्थ‚क्रियाकारित्व‚म‚स्तीत्य‚त आह । \quotelemma{ब‚ह‚वो} पि ही‚{\tiny $_{lb}$}‚ \cite[3a1]{vn-msN} \quotelemma{त्यादि} । किंव‚त् । च‚क्षुरादिव‚त् । य‚था रूपालोक‚म‚न‚स्कार‚च‚क्षुराद‚{\tiny $_{9}$}‚‚{\tiny $_{lb}$}‚\leavevmode\ledsidenote{\textenglish{22b/msK}} \leavevmode\ledsidenote{\textenglish{31/s}}य‚श्च‚क्षुरादिविज्ञान‚मेकंर्व कुव्न्ति त‚था रूपा[द]योप्युद‚क‚धार‚ण‚विशेषादिकामे- ‚{\tiny $_{lb}$}‚काम‚र्थ‚क्रियाङ्क‚रिष्य‚न्तीत्य‚र्थः । य‚त‚श्चैत‚देवं त‚स्मात्त‚स्यैकार्थ‚क्रियासाम‚र्थ्य‚स्य ‚{\tiny $_{lb}$}‚ख्याप‚नाय ‚{\tiny $_{1}$}‚ त‚त्र रूपादावेक‚स्य प‚टादिश‚ब्द‚स्य नियोगोपि स्यादिति एत‚द्युक्तं प‚श्यामः । ‚{\tiny $_{lb}$}‚न केव‚ल‚मेकार्थ‚क्रियाकारित्वं तेषामित्य‚पिश‚ब्देनाह । क‚थं युक्त‚मिति चेत् । एवं ‚{\tiny $_{lb}$}‚म‚न्य‚ते ‚{\tiny $_{2}$}‚ केन‚चित्प्र‚योज‚नेन केचिच्छ‚ब्दाः क्व‚चिन्निवेश्य‚न्ते त‚त्र य‚द‚नेक‚मेक‚त्रोप‚{\tiny $_{lb}$}‚युज्य‚ते त‚द‚व‚श्य‚न्त‚त्र चोद‚नीयं । त‚स्य च पृथ‚क्क‚थ‚ञ्चोद‚नेऽतिगौर‚वं स्यात् । न चा‚{\tiny $_{lb}$}‚स्यान‚न्य‚साधार ‚{\tiny $_{3}$}‚ णं रूपं श‚क्यं चोद‚यितुं । नाप्य‚स्यायास‚स्य किञ्चित्साफ‚ल्यं ‚{\tiny $_{lb}$}‚केव‚ल‚म‚नेन योग्यास्त‚त्र तेर्थाश्चोद‚नीयास्त एकेन वा श‚ब्देन चोद्येर‚न् ब‚हुभिर्वेति ‚{\tiny $_{lb}$}‚स्वात‚न्त्र्य‚म‚त्र ‚{\tiny $_{4}$}‚ व‚क्तुः । त‚दिय‚मेका श्रुतिर्ब‚हुषु व‚क्त्र‚भिप्राय‚व‚सा\edtext{}{\lemma{सा}\Bfootnote{? व‚शा}}त्प्र‚व‚र्त्त‚{\tiny $_{lb}$}‚माना नोपाल‚म्भ‚म‚र्ह‚ति । न चेय‚म‚श‚क्य‚प्र‚व‚र्त्त‚माना इच्छाधीन‚त्वात् । य‚दि हि न ‚{\tiny $_{lb}$}‚प्र‚योक्तु ‚{\tiny $_{5}$}‚ रिच्छा क‚थ‚मिय‚मेक‚त्रापि व‚र्त्तेत । इच्छायाम्वा क एनां ब‚हुष्व‚पि ‚{\tiny $_{lb}$}‚प्र‚तिब‚द्धुं स‚म‚र्थः । प्र‚योज‚नाभावाद‚प्र‚व‚र्त्त‚न‚मित्य‚पि नाश‚ङ्क‚नीयं । भिन्नेष्व ‚{\tiny $_{6}$}‚ प्ये‚{\tiny $_{lb}$}‚क‚स्माच्छ‚ब्दात्प्र‚तीतिर‚त‚त्प्र‚योज‚न‚भेदेन य‚था स्यादित्युक्त‚त्वात् प्र‚योज‚न‚स्य त‚स्मा‚{\tiny $_{lb}$}‚त्सूक्त‚म‚स्माभिर्युक्तं प‚श्याम इति । य‚था क‚थ‚ञ्चिद्विनैव प्र‚योज‚नेन लोकः श‚ब्दं ‚{\tiny $_{7}$}‚ ‚{\tiny $_{lb}$}‚प्र‚युंक्ते । त‚तो न युक्त‚मेत‚दिति चेदाह । न च \quotelemma{निःप्र‚योज‚ना लोक‚स्यार्थेषु} श‚ब्द‚योज‚ने‚{\tiny $_{lb}$}‚ \cite[3a1]{vn-msN} ति । न हि व्य‚स‚न‚मेवैत‚ल्लोक‚स्य य‚द‚य‚म‚स‚ङ्ग‚तं य‚न्न प्र‚युज्जानो वा श‚ब्दान्तः ‚{\tiny $_{lb}$}‚ख‚ल्वा ‚{\tiny $_{8}$}‚ सीत् । किन्त‚र्हि स‚र्व्व एवास्यार‚म्भः प्र‚योज‚न‚सापेक्षः प्र‚योज‚न‚ञ्चेत‚दुक्त ‚{\tiny $_{lb}$}‚मिति म‚न्य‚ते त‚त्र प्र‚योज‚न‚व‚त्वेनेति । य‚था रूप‚ग‚न्ध‚र‚साद‚यः स‚हैक‚प्र‚योज‚नाः ‚{\tiny $_{lb}$}‚स‚ङ्क‚लि ‚{\tiny $_{9}$}‚ \leavevmode\ledsidenote{\textenglish{23a/msK}} ता एक‚कार्य‚कारिण इत्य‚र्थः । स‚म‚व‚हितानाम‚पि क‚दाचित्क‚स्य‚चिदेव ‚{\tiny $_{lb}$}‚कार्ये व्यापारोन्य‚स्य त्वौदासीन्य‚मिति ‚{\tiny $_{1}$}‚ स्यादाश‚ङ्कासंभ‚व‚स्त‚त आह । \quotelemma{पृथ‚ग्वेति} ‚{\tiny $_{lb}$}‚ \cite[3a2]{vn-msN} [।] वा श‚ब्द‚श्चार्थे । स‚र्व्व एव व्यापार‚व‚न्त इत्य‚र्थः । अन्ये त्व‚न्य‚था ‚{\tiny $_{lb}$}‚व्याख्यान‚य‚न्ति । व्याख्यान‚ञ्चादो दूष‚य‚न्ति । त‚त्रैत‚स्मिन् श‚ब्दैर ‚{\tiny $_{2}$}‚ र्थ प्र‚त्याय‚न‚{\tiny $_{lb}$}‚क्र‚मे येर्था रूपाद‚यः स‚ह पृथ‚ग्वैक‚प्र‚योज‚नास्तेषां रूपादीनां संहितानां पृथुबुध्नोद‚{\tiny $_{lb}$}‚राकार‚संस्थापितानामेकं प्र‚योज‚नं । य‚दुत म‚धूद‚काद्याह‚र‚णं पृथ‚ग्वा ‚{\tiny $_{3}$}‚ तेषामेव ‚{\tiny $_{lb}$}‚प्र‚त्येकं स्वाकार‚ज्ञान‚ज‚न‚नं ॥ एक‚ञ्च त‚त्प्र‚योज‚न‚मेक‚त्र दृष्टं य‚त्त‚द‚न्य‚त्र नास्तीति ‚{\tiny $_{lb}$}‚स‚ह‚भूतानाम‚पि क‚दाचिदौदासीन्य‚द‚र्श‚नात्स‚र्वेषां स‚व्यापा ‚{\tiny $_{4}$}‚ र‚तामाद‚र्श‚यितुं पृथ‚{\tiny $_{lb}$}‚ग्वेत्य‚भिहितं [।] वा श‚ब्द‚श्च स‚मुच्च‚य इत्य‚न्ये । केव‚ल‚म‚त्रैक‚प्र‚योज‚ना इत्य‚भिधाना‚{\tiny $_{lb}$}‚ \leavevmode\ledsidenote{\textenglish{32/s}} त्स‚र्वेषान्त‚थाभाव‚प्र‚तीतिर‚स्त्येवेति व्य‚र्थ‚म्पृ ‚{\tiny $_{5}$}‚ थ‚ग्वेति स्यात् न चायं श‚ब्दार्थ इति ‚{\tiny $_{lb}$}‚य‚क्तिञ्चिदेत‚त् । तैः प्र‚क‚र‚णं न ल‚क्षितं त‚था ह्य‚त्र स‚मुदाय‚श‚ब्द‚स्यैक‚व‚च‚न‚प्र‚वृ‚{\tiny $_{lb}$}‚त्य‚विरोधः क‚थ‚यितुमार‚ब्धः । त‚त्र कः प्र‚स्तावः पृथ‚ग्वेत्य‚भिधान‚स्य ॥
	{\color{gray}{\rmlatinfont\textsuperscript{§~\theparCount}}}
	\pend% ending standard par
      ‚{\tiny $_{lb}$}‚

	  
	  \pstart \leavevmode% starting standard par
	केव‚लं रूपादिश‚ब्द‚श्चाय‚ञ्जातिश‚ब्दः । त‚त्र चान्यादृश्येव प्र‚क्रिया भ‚विष्य‚ति । ‚{\tiny $_{lb}$}‚य‚त्त्विद‚मुक्तं केव‚ल‚म‚त्रैक‚प्र‚योज‚ना इत्य‚भिधानात्स‚र्व्वे ‚{\tiny $_{7}$}‚ षां त‚थाभाव‚प्र‚तीतिर‚स्त्येवेति ‚{\tiny $_{lb}$}‚त‚द‚पि न युक्तिस‚ङ्ग‚तं । त‚थाहि प‚र‚ब‚ल‚प‚राज‚योद्य‚तानामेक‚प्र‚योज‚न‚व‚त्वेपि न त‚त्र स‚र्व्वे ‚{\tiny $_{lb}$}‚व्यापार‚व‚न्तो भ‚व‚न्ति । त‚द्व‚द‚त्रापि भ ‚{\tiny $_{8}$}‚ वेत् । अत एव च स्यादाश‚ङ्कास‚म्भ‚व इति ‚{\tiny $_{lb}$}‚व्याख्यातं । य‚दा तु स‚र्व्वेषामेव स‚व्यापार‚ताख्याप‚नाय पृथ‚ग्वेत्येत‚दुच्य‚ते त‚दाऽप‚ह्नु ‚{\tiny $_{lb}$}‚र\edtext{}{\lemma{र}\Bfootnote{?}}त‚मुत्सार्य‚ते । त‚देतेनैवाश‚ब्दार्थ ‚{\tiny $_{9}$}‚ \leavevmode\ledsidenote{\textenglish{23b/msK}} त्व‚म‚पि प्र‚त्युक्त‚मिति य‚क्तिञ्चिदेत‚देव ।
	{\color{gray}{\rmlatinfont\textsuperscript{§~\theparCount}}}
	\pend% ending standard par
      ‚{\tiny $_{lb}$}‚

	  
	  \pstart \leavevmode% starting standard par
	\hphantom{.}अस्तु वैत‚द‚पि व्याख्यानं य‚दि क‚थ‚ञ्चिद्व्य‚व‚स्थापितुं पार्य‚ते । \quotelemma{तेषा} मेवं विधा‚{\tiny $_{lb}$}‚नाम‚र्थानान्त‚स्यैकार्थ‚क्रियाकारिणो भाव‚स्य ख्याप‚ना ‚{\tiny $_{1}$}‚ यैकोघ‚टादिश‚ब्दो य‚दि ‚{\tiny $_{lb}$}‚नियुज्येत त‚दा किं स्यान्न क‚श्चिद्दोषः स्यात् । गुण एव तु केव‚लो ल‚भ्य‚त इत्याह [।] ‚{\tiny $_{lb}$}‚ \quotelemma{त‚द‚र्थ‚क्रियास\edtext{}{\lemma{क्रियास}\Bfootnote{? श}}} \quotelemma{क्ते} र‚भिन्नाया[ः] ख्याप‚नाय \quotelemma{नियुक्त‚स्य स‚मुदाय‚श ‚{\tiny $_{2}$}‚ ब्द‚स्यैक‚व‚च‚न‚{\tiny $_{lb}$}‚विरोधोपि नास्त्येव} \cite[3a2]{vn-msN} । कुतः [।] य‚स्मात्स \quotelemma{हितानां सा श‚क्तिरेका} \cite[3a3]{vn-msN} ‚{\tiny $_{lb}$}‚ऽभिन्ना न प्र‚त्येकं न तु पृथ‚ग्भूतानामित्य‚र्थः । इति त‚स्मात्स‚मुदाय‚श‚ब्दे त‚स्मिन्नै‚{\tiny $_{lb}$}‚क‚स्मिन्घ‚टादौ ‚{\tiny $_{3}$}‚ स‚मुदाये वाच्ये एक‚व‚च‚नं घ‚ट इति भ‚व‚तीति शेषः । स्यादिति वा ‚{\tiny $_{lb}$}‚व‚क्ष्य‚माणं क्रियाप‚दं । न‚न्व‚य‚ङ्घ‚टादिश‚ब्दो ग‚वादिश‚ब्द‚व‚ज्जातिश‚ब्द‚स्त‚त्क‚थ‚मे‚{\tiny $_{lb}$}‚त‚दु ‚{\tiny $_{4}$}‚ क्त‚मिति चेत् । स‚त्यं स‚मुदा[या]न्त‚र‚वृत्य‚पेक्ष‚या जातिश‚ब्दोयं रूपादिस‚मु‚{\tiny $_{lb}$}‚दाय्य‚पेक्ष‚या तु स‚मुदाय‚श‚ब्दोपीत्य‚भिस‚न्धेर‚दोषः । त‚थाहि त्र‚य्येव‚ग‚तिः श ‚{\tiny $_{5}$}‚ ब्दा‚{\tiny $_{lb}$}‚नाङ्केचिज्जातिश‚ब्दा एव । य‚था सुखादिश‚ब्दाः सुखादेर‚न‚व‚य‚व‚त्वात् । केचित्तु ‚{\tiny $_{lb}$}‚स‚मुदाय‚श‚ब्दा एव य‚था \quotelemma{विन्ध्य‚हिम‚व‚त्सुमेर्वादि} श‚ब्दाः । त‚ज्जातीय‚स ‚{\tiny $_{6}$}‚ मुदाया‚{\tiny $_{lb}$}‚न्त‚राभावात् । अप‚रे पुन‚र्जातिस‚मुदाय‚श‚ब्दाः । य‚थैत एव घ‚टादिश‚ब्दाः स‚मुदाया‚{\tiny $_{lb}$}‚न्त‚र‚स‚मुदाय्य‚पेक्ष‚येति । एव‚न्ताव‚त्स‚मुदाय‚श‚ब्देषु व‚च‚न‚प्र‚व‚त्य‚वि ‚{\tiny $_{7}$}‚ रोध उक्तः । ‚{\tiny $_{lb}$}‚अथ क‚थ‚ञ्जातिश‚ब्देष्वित्याह । \quotelemma{जातिश‚ब्देष्वित्यादि} \cite[3a3]{vn-msN} । अर्थानां घ‚टा‚{\tiny $_{lb}$}‚वीनां प्र‚त्येकं स‚हितानाञ्च श‚क्तेः कार‚णात् नानाश‚क्तिरेका च । एत‚दुक्तं भ‚व‚ति ‚{\tiny $_{lb}$}‚य‚स्मादेको ‚{\tiny $_{8}$}‚ \leavevmode\ledsidenote{\textenglish{24a/msK}} पि वृक्षो गृह‚क‚र‚णाद्य‚र्थ‚क्रियानिव‚र्त्त‚ने\edtext{}{\lemma{ने}\Bfootnote{? निर्व‚र्त्त‚ने}}पि योग्यो ब‚ह‚{\tiny $_{lb}$}‚ \leavevmode\ledsidenote{\textenglish{33/s}} वोपि वृक्षाः । त‚त‚श्च तेषाङ्केव‚लानाम‚पि योग्य‚त्वाद‚नेका श‚क्तिः स‚म‚व‚हिता‚{\tiny $_{lb}$}‚नाम‚पि योग्य‚त्वादेका श‚क्तिरेक‚प्र‚त्य‚व‚म‚र्श‚प्र‚त्य‚य‚निब‚न्ध‚न‚त्वेनैक‚त्वोप‚चारात् । ‚{\tiny $_{lb}$}‚य‚त‚श्चैव ‚{\tiny $_{1}$}‚ मिति त‚स्माद्य‚थाक्र‚मं नानाश‚क्तिविव‚क्षायां स‚त्यां ब‚हुव‚च‚न‚म‚नेक‚{\tiny $_{lb}$}‚त्वाच्छ‚क्तेर्वृक्षा इति भ‚व‚ति । एक‚श‚क्तिविव‚क्षायान्तु एक‚त्वाच्छ‚ब्द एक‚व‚च‚न‚{\tiny $_{lb}$}‚मुक्त इति स्यात् । त ‚{\tiny $_{2}$}‚ दा य‚द्येष निय‚मो भ‚व‚द्भिर‚स‚द्ग्र‚ह‚ग्र‚हावेश‚व्याकुलित‚चेतोभि‚{\tiny $_{lb}$}‚रिष्य‚ते । ब‚हुष्वेव वाच्येषु ब‚हुव‚च‚नं भ‚व‚ति । एक‚स्मिन्नेव चैक‚व‚च‚न‚मिति । त‚द‚नेन ‚{\tiny $_{lb}$}‚य‚दाप्येत‚द्द ‚{\tiny $_{3}$}‚ र्श‚न‚माश्रीय‚ते ब‚हुषु [व‚हु]व‚च‚न‚म्भ‚व‚ति । \quotelemma{द्व्येक‚योर्द्विव‚च‚नैक‚व‚च‚ने} ‚{\tiny $_{lb}$}‚ \href{http://sarit.indology.info/?cref=P\%C4\%81.1.4.22}{पाणिनिः १।४।२२} इति त‚दापि न क‚श्चिद्दोष इति द‚र्श‚य‚ति । भ‚व‚तान्तु क‚थ‚{\tiny $_{lb}$}‚मित्याह । \quotelemma{अस्माक‚मि} \cite[3a4]{vn-msN} त्यादि । ‚{\tiny $_{4}$}‚ संकेत‚ब‚सा\edtext{}{\lemma{सा}\Bfootnote{? ब‚शा}}च्छ‚ब्दानाम्ब‚हुव‚च‚{\tiny $_{lb}$}‚नान्तानान्दाराः सिक‚ताः पादाः । गुर‚व इत्यादिनाऽस‚त्य‚पि ब‚हुत्वेऽभिधेय‚स्य वृत्तिः ॥
	{\color{gray}{\rmlatinfont\textsuperscript{§~\theparCount}}}
	\pend% ending standard par
      ‚{\tiny $_{lb}$}‚

	  
	  \pstart \leavevmode% starting standard par
	त‚थास‚त्य‚प्य‚नेक‚त्वेष‚ण्ण‚ग‚री ‚{\tiny $_{5}$}‚ ष‚ट्पू\edtext{}{\lemma{ट्पू}\Bfootnote{?}}लीव‚न‚मित्यादिनैक‚व‚च‚नान्तानाम्वृत्तिरि‚{\tiny $_{lb}$}‚त्य‚न‚भिनिवेश एव । को हि नाम स‚चेत‚नः पुरुषाभिप्राय‚मात्राधीन‚वृत्तिषु श‚ब्देष्व‚भिनि‚{\tiny $_{lb}$}‚वेशं क‚र्त्तुमु ‚{\tiny $_{6}$}‚ त्स‚ह‚त इति भावः । प‚र‚प‚क्षं पूर्व‚प‚क्ष‚य‚ति । \quotelemma{नानेको रूपादिरेक‚श‚ब्दो‚{\tiny $_{lb}$}‚त्थाप‚ने स‚म‚र्थ इति चे} \cite[3a5]{vn-msN} दिति । न‚हि अनेक‚स्यैकेन स‚म्ब‚न्धो युज्य‚त इति । ‚{\tiny $_{lb}$}‚किमि \cite[3a5]{vn-msN} त्यादिना प‚रि ‚{\tiny $_{7}$}‚ हारः । पुरुषाणाम्वृत्तिरिच्छा त‚त्रान‚पेक्षाः स‚न्तोऽर्थाः ‚{\tiny $_{lb}$}‚किं स्व‚यं श‚ब्दानुत्थाप‚य‚न्ति । आहोस्वित्पुरुषैस्ते व्य‚व‚हारार्थ‚म‚र्थेषु य‚था क‚थ‚ञ्चि‚{\tiny $_{lb}$}‚न्नियुज्य‚न्त इति विक‚ल्प‚द्व ‚{\tiny $_{8}$}‚ यं । त‚त्र पुरुषैरेव ते य‚थेष्टं नियुज्य‚न्तेऽन्य‚थाऽतीता ‚{\tiny $_{lb}$}‚जात‚योर्द‚र्श‚नान्त‚र‚भेदिष्व‚न्य‚त्र वा नियोज‚न‚न्न भ ‚{\tiny $_{9}$}‚ \leavevmode\ledsidenote{\textenglish{24b/msK}} वेदिति भावः । त‚त‚श्च \quotelemma{स्व‚यं} ‚{\tiny $_{lb}$}‚ \cite[3a6]{vn-msN} पुरुषेच्छाऽन‚पेक्षाणाम‚र्थानां श‚ब्दोत्थाप‚ने स‚ति भाव‚स्य श‚क्तिर‚स‚क्ति\edtext{}{\lemma{क्ति}\Bfootnote{? श‚क्ति}} ‚{\tiny $_{lb}$}‚ र्वा चिन्त्येत \quotelemma{नामैक} इत्यादिना । अस्त्येव त‚र्हि स्व‚य‚मुत्त्थाप‚न‚मिति चेदा ‚{\tiny $_{1}$}‚ ह । \quotelemma{न च ‚{\tiny $_{lb}$}‚त‚द्युक्तं} \cite[3a6]{vn-msN} । अन‚न्त‚रोक्तात् कार‚ण‚त्र‚यादित्य‚भिप्रायः । त‚स्मात्पुरुषैस्तेषां ‚{\tiny $_{lb}$}‚श‚ब्दानां नियोगोर्थेषु विनाप्येक‚त्वादिना ते पुरुषाः य‚थेष्ट‚मेक‚त्रापि व‚हुव‚च‚नान्त‚म ‚{\tiny $_{2}$}‚ ‚{\tiny $_{lb}$}‚नेक‚त्राप्येक‚व‚च‚नान्तं श‚ब्दं नियुञ्जीर‚न्निति क‚स्त‚त्र तेषु श‚ब्देषूपाल‚म्भो नानेको ‚{\tiny $_{lb}$}‚ \leavevmode\ledsidenote{\textenglish{34/s}} रूपादिरेक‚श‚ब्दोत्थाप‚ने स‚म‚र्थ इत्य‚यं नैव क‚श्चित् केव‚ल‚म‚तिब‚हुल‚व्यामोह‚विजृ‚{\tiny $_{lb}$}‚म्भित‚मिति म‚न्य‚ते । स्यान्म‚त‚ङ्किमित्येकं श‚ब्द‚म‚नेक‚त्र नियुञ्ज‚त इत्याह । \quotelemma{निमि‚{\tiny $_{lb}$}‚त‚ञ्च नियोग‚स्योक्त‚मेवेति} \cite[3a6]{vn-msN} त‚त्साम‚र्थ्य‚ख्याप‚नाय त‚त्रैक ‚{\tiny $_{4}$}‚ श‚ब्द‚नियोगोऽपि ‚{\tiny $_{lb}$}‚स्यादित्य‚त्राव‚स‚रे । उप‚च‚य‚माह । \quotelemma{अपि चे} \cite[3a7]{vn-msN} त्यादिना । आश्र‚याभिम‚तेने‚{\tiny $_{lb}$}‚त्य‚व‚य‚विद्र‚व्येण । तेषान्त‚त्र स‚म‚वाय‚स‚म्ब‚न्धेन स‚म्ब‚न्धात् । क ‚{\tiny $_{5}$}‚ थं स‚म्ब‚न्धो नैवाने‚{\tiny $_{lb}$}‚क‚स्य एकेन स‚ह स‚म्ब‚न्धो विरोधाभ्युप‚ग‚मात् । अन्य‚थैकेन श‚ब्देनापि स‚ह प्राप्नो‚{\tiny $_{lb}$}‚तीत्य‚भिस‚न्धिः । प‚रः प्राह । \quotelemma{न चे} \cite[3a7]{vn-msN} द‚य‚मेकेन स ‚{\tiny $_{6}$}‚ ह \quotelemma{स‚म्ब‚न्ध‚विरोधात्} ‚{\tiny $_{lb}$}‚कार‚णा \quotelemma{देक‚श‚ब्दं} रूपादिषु \quotelemma{नेच्छामः} । किन्त्व‚भिन्नानाम‚विशिष्टानां रूपादीनां‚{\tiny $_{lb}$}‚घ‚ट‚क‚म्ब‚ल‚प‚र्य‚ङ्कादिषु । नानाविधा येय‚म‚र्थ‚क्रिया ज‚ल‚धा ‚{\tiny $_{7}$}‚ र‚ण‚प्राव‚र‚णादिस्त‚स्या ‚{\tiny $_{lb}$}‚विरोधः । त‚था च त‚त्साम‚र्थ्य‚ख्याप‚नाय श‚ब्द‚स्य विरोधात् । तेषाञ्चाभेद‚स्त‚दाश्र‚य‚{\tiny $_{lb}$}‚द्र‚व्य‚भेदाभावात् । एत‚देव स्फुट‚य‚ति । ते रूपाद‚य ए ‚{\tiny $_{8}$}‚ \leavevmode\ledsidenote{\textenglish{25a/msK}} क‚स्व‚भावाः स‚न्तः ‚{\tiny $_{lb}$}‚स‚मुदायान्त‚रे क‚म्व‚लादौ येय‚म‚स‚म्भाविनी उद‚क धार‚ण विशेषाद्य‚र्थ‚क्रिया ‚{\tiny $_{lb}$}‚तामेव कुर्युस्तेन कार‚णेन त‚स्या अर्थ‚क्रियायाः प्र‚काश‚ना ‚{\tiny $_{1}$}‚ येमामेतेऽर्थ‚क्रियां न ते ‚{\tiny $_{lb}$}‚त‚द‚स‚म्भाविनीम‚र्थ‚क्रियाङ्कुर्व‚न्ति [।] य‚था त एव क‚म्ब‚ल‚ग ‚{\tiny $_{2}$}‚ ता रूपाद‚यः ‚{\tiny $_{lb}$}‚स‚जातीयेभ्यः । त‚था च घ‚ट‚ग‚ता अपि रूपाद‚यः क‚म्ब‚ल‚ग‚तेभ्यो रूपादि‚{\tiny $_{lb}$}‚भ्योऽविशिष्ट‚स्व‚भावा इति व्याप‚कानुप‚ल‚ब्धिः । एव‚म‚न्य‚त्रापि ‚{\tiny $_{3}$}‚ योज्य‚मितीयं पूर्व्व‚{\tiny $_{lb}$}‚प‚क्ष‚र‚च‚ना । अत्रोत्त‚र‚माह । \quotelemma{भ‚व‚तु नामेत्यादिना} \cite[3a9]{vn-msN} । त‚द‚नेन हेतोर‚सिद्धि‚{\tiny $_{lb}$}‚मुद्भाव‚य‚ति । अय‚म‚त्रार्थो न‚हि रूपादीनाङ् क‚म्ब‚ला ‚{\tiny $_{4}$}‚ दिष्व‚भेदोस्ति । प‚र‚स्प‚र‚{\tiny $_{lb}$}‚रूप‚विविक्तानामेव प्र‚त्य‚क्ष‚प्र‚माण‚प‚रिच्छेद्य‚त्त्वात् । एव‚ञ्च स‚तीदं प्र‚त्य‚क्षं किमे‚{\tiny $_{lb}$}‚नाम्वाञ्छामुपेक्ष‚ते । किम्प्र‚श्ने क्षेपे वा ‚{\tiny $_{5}$}‚ नैव क्ष‚न्तुम‚र्ह‚त्य‚पाक‚रोतीति । ‚{\tiny $_{lb}$}‚किंञ्चानिष्ट‚ञ्चेद \cite[3a10]{vn-msN} म‚स्माभिर्घ‚ट‚क‚म्ब‚लादिष्व‚भिन्ना रूपाद‚य इति कुतो ‚{\tiny $_{lb}$}‚ \leavevmode\ledsidenote{\textenglish{35/s}} रूपादीनाम्प्र‚तिस‚मुदाय‚त्त्वे हेतुब‚लाद‚न‚पेक्षित‚द्र‚व्याणां स्व‚भेदाभ्युप‚ग‚मात् । त‚द‚{\tiny $_{lb}$}‚नेनाभ्युप‚ग‚म‚द्वारेणाप्य‚सिद्ध‚ताञ्चोद‚य‚ति । पूर्व्वेण प्र‚त्य‚क्ष‚द्वारेणेति विशेषः । ‚{\tiny $_{lb}$}‚पुन‚र‚पीर्ष्यालुः प‚रः प्रा ‚{\tiny $_{7}$}‚ ह । य‚द्य‚न्य एव रूपादिभ्यो घ‚टः स्यात् किं ‚{\tiny $_{lb}$}‚स्यादिति [।] न क‚श्चिद्दोषः स्यादित्याकूतं । न व‚यं मात्स‚र्यात्तं नेच्छामः । ‚{\tiny $_{lb}$}‚किन्तु भ‚व‚त्येताव‚त्त्व‚त्र व‚क्त‚व्य‚स्तीत्याह । त‚स्याव‚य ‚{\tiny $_{8}$}‚ विनः \quotelemma{प्र‚त्य‚क्ष‚स्य स‚तः} ‚{\tiny $_{lb}$}‚ \cite[3a10]{vn-msN} च‚क्षुः स्प‚र्श‚नेन्द्रिय‚ग्राह्य‚त‚याभ्युप‚ग‚त‚त्त्वात् । \quotelemma{अरूपादिरूप‚स्य} ‚{\tiny $_{lb}$}‚ \cite[3a10]{vn-msN} रूप‚ग‚न्धादिस्व‚भाव‚र‚हित‚स्येत्य‚र्थः । गुण‚द्र‚व्य‚योर्भेदाभ्युप‚ग‚मात् ॥ ‚{\tiny $_{9}$}‚ \leavevmode\ledsidenote{\textenglish{25b/msK}} ‚{\tiny $_{lb}$}‚त‚द्विवेकेन रूपादिविवेकेन बुद्धौ च‚क्षुः स्प[र्श]नेन्द्रिय‚जायां प्र‚तिभास‚ने किमा‚{\tiny $_{lb}$}‚व‚र‚ण‚न्न क‚श्चित्प्र‚तिब‚न्ध इत्य‚र्थः । न च क‚श्चिद‚त्याद‚रेणाप्र‚तिह‚त‚क‚र‚णोपि नि ‚{\tiny $_{1}$}‚ ‚{\tiny $_{lb}$}‚रूप‚य‚न्नील‚म‚धुर‚सुर‚भिक‚र्क्क‚शादिव्य‚तिरेकेण त‚द्रूप‚म्विविक्त‚रूपं घ‚टादिद्र‚व्य‚{\tiny $_{lb}$}‚मुप‚ल‚ब्धुमीश इति म‚न्य‚ते । \quotelemma{अबिद्ध‚क‚र्ण्ण‚स्त्वाह} [।] रूपाद्य‚ग्र‚हेपि द्र‚व्य‚ग्र ‚{\tiny $_{2}$}‚ ह‚ण‚म‚स्त्येव ‚{\tiny $_{lb}$}‚य‚तो म‚न्द‚म‚न्द‚प्र‚काशेऽनुप‚ल‚भ्य‚मान‚रूपादिकं द्र‚व्य‚मुप‚ल‚भ्य‚तेऽनिश्चित‚रूपं गौर‚श्वो ‚{\tiny $_{lb}$}‚वेति । न‚नु च त‚त्रापि संस्थान‚मात्र‚मुप‚ल‚भ्य‚ते । स‚त्य‚मुप ‚{\tiny $_{3}$}‚ ल‚भ्य‚ते न तु त‚द्रूपा‚{\tiny $_{lb}$}‚द्यात्म‚कं । रूपाद्यात्म‚क‚त्वे वा नील‚पीतादिविशेष‚ग्र‚ह‚ण‚प्र‚स‚ङ्गः । त‚थाय‚स्क‚ञ्चु‚{\tiny $_{lb}$}‚कान्त‚र्ग‚ते पुरुषे पुरुष‚रूपाद्य‚ग्र‚हे ‚{\tiny $_{4}$}‚ पि पुरुष‚प्र‚त्य‚यो दृष्टः । रात्रौ च व‚लाकाव्यामुक्त ‚{\tiny $_{lb}$}‚रूपाद्य‚ग्र‚हेपि प‚क्षिप्र‚त्य‚यो दृष्टः । त‚थानीलाद्युप‚धान‚भेदानुविधायिनः स्फ‚टिक‚म ‚{\tiny $_{5}$}‚ णेः ‚{\tiny $_{lb}$}‚स्फ‚टिक‚रूपाद्य‚ग्र‚हेपि स्फ‚टिक‚प्र‚त्य‚यः । त‚था क‚षाय‚रूपेण प‚ट‚रूपाभिभ‚वे प‚ट‚रूपाद्य‚{\tiny $_{lb}$}‚ग्र‚हेऽपि प‚ट‚प्र‚त्य‚यो दृष्ट इति । त‚देत‚त्स ‚{\tiny $_{6}$}‚ र्व्व‚म‚स्यान‚ल्प‚कालोप‚चित‚कुद‚र्श‚नाभ्या‚{\tiny $_{lb}$}‚सोप‚जात‚बुद्धिमान्द्य‚विजृम्भित‚मेव प्र‚क‚ट‚य‚ति व‚चः । त‚थाहि य‚त्ताव‚दिद‚मुक्तं म‚न्द‚{\tiny $_{lb}$}‚म‚न्दालोके रात्रौ च नीलाद्युप‚धान ‚{\tiny $_{7}$}‚ स‚द्भावे च त‚द्रूपाद्य‚ग्र‚हेपि द्र‚व्य‚मुप‚ल‚भ्य‚त इति ‚{\tiny $_{lb}$}‚त‚त्र व‚क्त‚व्यं कीदृशं त‚त्र द्र‚व्य‚मुप‚ल‚भ्य‚त इति । दृश्य‚त एव त‚द्यादृश‚मुप‚ल‚भ्य‚त इति ‚{\tiny $_{lb}$}‚चेत् । न‚नु श्याम‚रूपं ‚{\tiny $_{8}$}‚ म‚न्द‚म‚न्दालोके रात्रौ च त‚त्र त‚द‚पुल‚भ्य‚ते उप‚धानं रूप‚ञ्च । ‚{\tiny $_{lb}$}‚न च त‚द्रूप‚न्त‚त् । ताद्रूप्येऽन‚न्त‚रोदित‚प‚क्ष‚क्ष‚य‚प्र‚स‚ङ्गात् । त‚त्स‚मीप‚पार्श्व‚व‚र्त्तिभिश्च ‚{\tiny $_{lb}$}‚त‚थानुप ‚{\tiny $_{9}$}‚ \leavevmode\ledsidenote{\textenglish{26a/msK}} ल‚म्भात् । न चाप्य‚ण्या\edtext{}{\lemma{ण्या}\Bfootnote{? न्या}}कारेण बोधेन व‚स्तुनोऽव‚ग‚तिः य‚स्य क‚स्य- ‚{\tiny $_{lb}$}‚चिज्ज्ञान‚स्य स‚र्व्व‚व‚स्तुप‚रिच्छेद‚क‚त्त्व‚प्र‚स‚ङ्गात् । त‚स्माद् भ्रान्त‚मेत‚त् ज्ञान‚म्भ्रान्ति‚{\tiny $_{lb}$}‚वीजात्स्वोपादानाद‚नादि ‚{\tiny $_{1}$}‚ कालीनान्निर्विष‚य‚मेव त‚था प्र‚तिभासि द्विच‚न्द्रादिप्र‚त्य‚य‚{\tiny $_{lb}$}‚व‚दुप‚जाय‚ते [।] निर्विष‚य‚त्त्वेपि प्र‚तिनिय‚त‚देश‚काल‚भावि भ‚व‚ति । स्वोपादान‚{\tiny $_{lb}$}‚वास‚नाप्र‚बोध‚क‚बाह्या ‚{\tiny $_{2}$}‚ धिप‚तिप्र‚त्य‚यापेक्ष‚ना\edtext{}{\lemma{ना}\Bfootnote{? णा}}त् । द्विच‚न्द्रादिज्ञान‚व‚देव । ‚{\tiny $_{lb}$}‚भ्रान्त‚त्त्वेप्य‚र्थाविष\edtext{}{\lemma{र्थाविष}\Bfootnote{? स}}म्वादो विशिष्टाधिप‚तिप्र‚त्य‚य‚स‚द्भावात् । म‚णि‚{\tiny $_{lb}$}‚ \leavevmode\ledsidenote{\textenglish{36/s}} प्र‚भायां म‚णिभ्रान्तिरिव । न चार्थाविस‚म्वाद ‚{\tiny $_{3}$}‚ नादेवास्य स‚विष‚य‚त्वं युक्त‚म‚नु ‚{\tiny $_{lb}$}‚मानेन व्य‚भिचारात् । म‚णिभ्रान्त्या च । त‚देव द्र‚व्य‚न्त‚था गृह‚णाति त‚तोन्य‚स्य निर्वि‚{\tiny $_{lb}$}‚ष‚य‚त्त्व‚मिति चेत् । न‚नु न त‚द् द्र‚व्य‚न्त ‚{\tiny $_{4}$}‚ द्रूप‚न्न वान्याकारानुस्यूतः प्र‚त्य‚योऽन्य‚{\tiny $_{lb}$}‚स्य प‚रिच्छेद‚क इत्युक्तं । एव‚ञ्च स‚ति स‚द्विष‚य‚त्त्वे स‚त्य‚पेक्षेपि स‚द्विष‚य‚त्त्व‚म‚स्त्येव । ‚{\tiny $_{lb}$}‚त‚था हि स‚माप्येत‚च्छ‚क्य‚म्व‚क्तुं ‚{\tiny $_{5}$}‚ त एव नीलाद‚य‚स्त‚था प्र‚तिभास‚न्त इति । अस‚ति ‚{\tiny $_{lb}$}‚भ्रांतिस‚न्देह‚कार‚णे सालोकाव‚स्थायां योग्य‚देशाव‚स्थाने च निरुप‚धानाव‚स्थायाञ्च ‚{\tiny $_{lb}$}‚नोप‚ल‚भ्य‚ते ‚{\tiny $_{6}$}‚ [।] त‚त् द्र‚व्य‚म‚नात्म‚रूप‚प्र‚तिभासि विवेकेनान्य‚दा तु स‚ति भ्रान्ति‚{\tiny $_{lb}$}‚स‚न्देह‚कार‚णे निशान्ध‚काराव‚च्छादित‚लोच‚नाव‚स्थायां दूर‚देशाव‚स्थाने सोप‚धा‚{\tiny $_{lb}$}‚नाव‚स्थायाञ्च त‚द ‚{\tiny $_{7}$}‚ न्याकार‚विवेकेन प्र‚तिभास‚त इति कोन्यो भौतिकाद्व‚क्तु‚{\tiny $_{lb}$}‚म‚र्ह‚ति । अय‚स्क‚ञ्चुकान्त‚र्ग‚ते पुरुष‚प्र‚त्य‚यो न प्र‚त्य‚क्षः । किन्त‚र्हि [।] लैगिकः । ‚{\tiny $_{lb}$}‚त‚था हि पुरुष‚श ‚{\tiny $_{8}$}‚ रीराव‚य‚व‚स‚माश्र‚य‚ब‚लोद्भूत‚विशिष्ट‚संस्थानाव‚स्थित‚क‚ञ्चुक‚{\tiny $_{lb}$}‚द‚र्श‚नात्कार्य‚लिंग‚ज्ञानात् स‚म्ब‚न्ध‚स्म‚र‚णापेक्षिणः कार‚ण‚भूते त‚थाविधे पुंसि ‚{\tiny $_{lb}$}‚पुरुषोय‚मित्य‚न‚न्त‚र ‚{\tiny $_{9}$}‚ \leavevmode\ledsidenote{\textenglish{26b/msK}} मेव प्र‚त्य‚यः स‚मुद्भूतिमासाद‚य‚ति । अत एव चास्प‚ष्टाकारा सा ‚{\tiny $_{lb}$}‚प्र‚तीतिः क‚श्चाय‚मिति संश‚य‚श्च भ‚व‚ति । त‚थाविध‚संस्थान‚स्य च क‚ञ्चुक‚स्योत्त्प‚त्तेः ‚{\tiny $_{lb}$}‚पुरुष‚रूपाद‚य ए ‚{\tiny $_{1}$}‚ व हेत‚वो भ‚व‚न्ति न‚त्व‚न्य‚द‚व‚य‚वि द्र‚व्यं । त‚स्यासिद्धेर‚सिद्ध‚स्य च ‚{\tiny $_{lb}$}‚कार‚ण‚त्वाभ्युप‚ग‚मायोगात् । रूपादिभिस्तु प्र‚त्य‚क्षानुप‚ल‚म्भाभ्याङ्कार्य‚कार‚ण‚भाव‚{\tiny $_{lb}$}‚सिद्धेः । तेषा ‚{\tiny $_{2}$}‚ मेव केव‚लानान्त‚था स‚न्निविष्टानामुप‚ल‚म्भात् । प‚टे तु क‚षाय‚म‚ञ्जिष्ठा‚{\tiny $_{lb}$}‚दिस‚म्प‚र्काद‚र्थान्त‚र‚मेव केव‚ल‚न्त‚त्त‚था जात‚मीक्ष्य‚ते । न‚तु नानारूप‚योर्द्र‚व्य‚योः सं ‚{\tiny $_{3}$}‚ स‚र्गा‚{\tiny $_{lb}$}‚द‚विभागात् त‚थोप‚ल‚म्भः । पुन‚स्त‚द्द्र‚व्य‚संस्थान‚स्थितिकार‚ण‚विच्छेदात्त‚न्निवृत्तिः । ‚{\tiny $_{lb}$}‚त‚दुपादान‚कार‚णापेक्षिण‚श्च ज‚ल‚पाव‚कादेर‚प‚रोत्त्प‚त्तिरिति ‚{\tiny $_{4}$}‚ । एतेनायोगोल‚क‚{\tiny $_{lb}$}‚त‚द्रूप‚ग्र‚ह‚णेपि त‚त्प्र‚त्य‚यो दृष्ट इत्येत‚द‚पि प्र‚तिस्फुटं । त‚देवं द्र‚व्य‚स्य प्र‚त्य‚{\tiny $_{lb}$}‚क्ष‚त्वासिद्धेर्य‚दुक्त‚ङ्ग‚वाश्व‚म‚हिष‚व‚राह‚मात‚ङ्गा ‚{\tiny $_{5}$}‚ विम‚त्य‚धिक‚र‚ण‚भावाप‚न्ना रूपादि‚{\tiny $_{lb}$}‚व्य‚तिरिक्ता इत्येव घोष‚णा । ऐन्द्रि[य‚क]त्वे स‚ति स‚म‚स्त‚रूपादिग्राह‚क‚वाक्येन्द्रियान‚{\tiny $_{lb}$}‚व‚च्छेद्य‚त्वात्प्रीत्यादिव‚दिति त ‚{\tiny $_{6}$}‚ द‚प‚ह‚स्तितं । प्र‚योगाः पुनः । य‚द् दृश्यं स‚त्स‚द्व्य‚तिरेकेण ‚{\tiny $_{lb}$}‚नोप‚ल‚भ्य‚ते त‚त्त‚तो भिन्न‚न्नाभ्युपेय‚न्नास्तीति वाभ्युप‚ग‚न्त‚व्यं । य‚था न‚र‚शिर‚सि ‚{\tiny $_{lb}$}‚विषाण‚न्नोप‚ल‚भ्य‚ते च दृश्यं ‚{\tiny $_{7}$}‚ स‚न्नीलादिषु त‚द्व्य‚तिरेकेण सामान्य‚विशेष‚संयोग‚{\tiny $_{lb}$}‚विभाग‚प‚र‚त्वाप‚र‚त्वादिक‚मिति स्व‚भावानुप‚ल‚ब्धिः । नास्य सिद्धिः । दृश्य‚त्वेन ‚{\tiny $_{lb}$}‚स्व‚य‚म‚भ्युप‚ग‚मात् । त‚था ‚{\tiny $_{8}$}‚ रूपाद्य‚र्थ‚प‚ञ्च‚क‚व्य‚तिरिक्त‚त्वेनोप‚ग‚तं द्र‚व्य‚न्न‚च‚क्षुःप्र‚त्य‚या‚{\tiny $_{lb}$}‚व‚सेय‚मुप‚ल‚ब्धिल‚क्ष‚ण‚प्राप्त‚त्वेनोप‚ग‚त‚त्वे स‚ति नीलादिव‚स्तुरूप‚विर‚हात् । श‚ब्द ‚{\tiny $_{lb}$}‚ \leavevmode\ledsidenote{\textenglish{37/s}} ग‚न्ध‚र‚स‚व‚त् । न च ‚{\tiny $_{9}$}‚ \leavevmode\ledsidenote{\textenglish{27a/msK}} त‚स्माद‚व्य‚तिरिक्त एवायं त‚त्र चाय‚न्दोष इत्यागूर्याह । सोतिश‚यो ‚{\tiny $_{lb}$}‚व्य‚व‚च्छेद‚ल‚क्ष‚ण‚स्त‚स्यातिश‚य‚व‚तोऽव‚स्थातुरात्म‚भूतोऽन‚न्व‚य इत्येकान्तेन निव‚र्त्त्त‚{\tiny $_{lb}$}‚मानः । व्या ‚{\tiny $_{1}$}‚ प‚क‚त्वाभावात्प्र‚व‚र्त‚मानोऽस‚न्नेव क‚थ‚न्न स्व‚भाव‚नानात्वं सुख‚दुःख‚{\tiny $_{lb}$}‚धीरिवाक‚र्ष‚ति । अन्वाक‚र्ष‚त्येवेत्य‚र्थः ।प्र‚योगो\edtext{}{\lemma{योगो}\Bfootnote{? प्र‚योगः}}पुन‚र्यो य‚स्यात्म‚भूतः ‚{\tiny $_{lb}$}‚स त‚न्निवृ ‚{\tiny $_{2}$}‚ त्तावेकान्तेन निव‚र्त्त‚ते प्र‚वृत्तौ चास‚न्नेव प्र‚व‚र्त्त‚ते य‚था त‚स्यैवातिश‚य‚स्यात्मा । ‚{\tiny $_{lb}$}‚आत्म‚भूत‚श्चातिश‚य‚स्यातिश‚य‚वानिति स्व‚भाव‚हेतुः । त‚त‚श्च त‚योर‚व‚स्थ ‚{\tiny $_{3}$}‚ योर‚व‚स्था‚{\tiny $_{lb}$}‚तुर्न्नानात्व‚म्प‚र‚स्प‚र‚विरोधिप‚र्य्याध्यासित‚त्वात् सुख‚दुःख‚व‚दितिस्व‚भाव‚हेतुरेव । न वा ‚{\tiny $_{lb}$}‚साव‚तिश‚योऽन‚न्व‚यः प्र‚व‚र्त्त‚ते निव ‚{\tiny $_{4}$}‚ र्त्त‚ते वाऽतः पुर्व‚स्मिन् प्र‚माणे साध्य‚विक‚ल‚{\tiny $_{lb}$}‚त्व‚न्दृष्टान्त‚स्य । उत्त‚र‚त्र त्व‚सिद्धिर्हेतोरिति चेदाह । \quotelemma{सान्व‚य‚त्वे} चातिश‚य‚स्य ‚{\tiny $_{lb}$}‚निवृत्तिप्र‚वृत्योरं ‚{\tiny $_{5}$}‚ गीक्रिय‚माणे \quotelemma{का क‚स्य निवृत्तिः प्र‚वृत्तिर्वेति} \cite[4b4]{vn-msN} । नैव ‚{\tiny $_{lb}$}‚काचित् क‚स्य‚चिन्निवृत्तिः प्र‚वृत्तिर्वा । स‚र्व‚स्य स‚र्व‚दा स‚त्वात् । त‚था च स‚र्वं स‚र्व‚त्र ‚{\tiny $_{lb}$}‚स‚मुप‚यु ‚{\tiny $_{6}$}‚ ज्येतेत्यादिना पुरोनुक्रान्तो दोषोनुप‚युज्य‚त इत्य‚भिप्रायः । उप‚च‚य‚माह । ‚{\tiny $_{lb}$}‚य‚दि च क‚स्य‚चित् स्व‚भाव‚स्यातिश‚याख्य‚स्य प्र‚वृत्तिर्निवृत्तिर्वेति स्व‚य‚म‚भ्य‚नुज्ञा ‚{\tiny $_{7}$}‚ ‚{\tiny $_{lb}$}‚य‚ते त्व‚या । एकातिश‚य‚निवृत्याऽप‚रातिश‚योत्त्प‚त्या व्य‚व‚हार‚भेदोप‚ग‚मादित्य‚भि‚{\tiny $_{lb}$}‚धानात् त‚देत‚देव प‚र‚स्त‚थाग‚त‚व‚चोऽभ्यासोप‚जाताव‚दात‚म‚ति ‚{\tiny $_{8}$}‚ र्ब्रुवाणः । नानु‚{\tiny $_{lb}$}‚म‚न्य‚ते भ‚द्र‚मुखेण\edtext{}{\lemma{मुखेण}\Bfootnote{? मुखेन}}भ‚वेदेवं य‚दि य‚थाम‚या प्र‚वृत्तिनिवृत्ती अभ्य‚नुज्ञायेते त‚था ‚{\tiny $_{lb}$}‚तेनापि । याव‚तास्य निर‚न्व‚योप‚ज‚न‚न‚विनाशोप‚ग‚म‚स‚म ‚{\tiny $_{9}$}‚ \leavevmode\ledsidenote{\textenglish{27b/msK}} त्वाद् द्र‚व्य‚स्यालोक‚नीलादि- ‚{\tiny $_{lb}$}‚व‚त्त‚द्व्य‚तिरेकेणाप्र‚तिभास‚न‚म‚भिन्नेंद्रिय‚ग्राह्य‚त्वाद्वा त‚द्व‚देवेत्य‚तआह । \quotelemma{प्र‚तिभास‚{\tiny $_{lb}$}‚मानाश्च विवेकेने} \cite[3b1]{vn-msN} दं नील‚मिदं सुर‚भि म‚धुरं क‚र्क‚श‚मिद ‚{\tiny $_{1}$}‚ मिति चेति प्र‚त्य‚क्षा ‚{\tiny $_{lb}$}‚अर्था दृश्य‚न्तेऽपृथ‚ग्देश‚त्वेपि स‚ति के ते रूपाद‚यः । लोक‚प्र‚सिद्ध्या चेद‚मुक्त‚न्न तु ‚{\tiny $_{lb}$}‚तेषाम‚भिन्न‚देश‚त्व‚म‚स्ति । स‚प्र‚तिघा द‚श रूपिण \href{http://sarit.indology.info/?cref=ak.1.29}{अभिध‚र्म‚कोशे १।२९ } इति व‚च‚नात् । ‚{\tiny $_{lb}$}‚त ‚{\tiny $_{2}$}‚ था ऽभिन्नेन्द्रिय \quotelemma{ग्राह‚य‚त्वेपि वातात‚प‚स्प‚र्शाद‚य} \cite[3ba]{vn-msN} इति य‚थाक्र‚मं चैत‚दुत्त‚रं । ‚{\tiny $_{lb}$}‚अनेनैकान्तिक‚त्वं हेतोरुद्भाव‚य‚ति । उपेत्य च ध‚र्मिस‚म्ब‚न्धं अभिन्नेन्द्रिय‚ग्राह्य‚त्व‚स्य ‚{\tiny $_{lb}$}‚व्य‚भि ‚{\tiny $_{3}$}‚ चार उक्तो न‚त्व‚साव‚स्यास्त्येक‚देशासिद्धेः । क‚थं । य‚तो न सुर‚भिम‚धुराद‚यो ‚{\tiny $_{lb}$}‚द्र‚व्य‚ग्राह‚केन्द्रिय‚ग्राह्याः । सार्वेन्द्रिय‚त्व‚प्र‚स‚ङ्गाद् द्र‚व्य‚स्य । आलोक‚नीलादीनां ‚{\tiny $_{4}$}‚ त्व‚भेद ‚{\tiny $_{lb}$}‚एव य‚तः प्र‚दीपादिस‚न्निधानात् प्र‚काश‚रूपा एव त‚थाविध‚स्व‚भावाध्यासित‚{\tiny $_{lb}$}‚व‚पुष‚स्ते स‚मुद्भ‚व‚न्ति न तु तेषां भेदोऽस्तीति साध‚न‚विक‚ल्प‚ताऽपि दृ ‚{\tiny $_{5}$}‚ ष्टान्त‚स्येति ‚{\tiny $_{lb}$}‚म‚न्य‚ते । त‚स्माद‚स्य प्र‚त्य‚क्ष‚त्व‚म‚भ्युप‚ग‚च्छ‚द्भिर्न ब‚हिर‚व‚श्यं रूपादिविवेकेन प्र‚ति‚{\tiny $_{lb}$}‚ \leavevmode\ledsidenote{\textenglish{38/s}} भास‚न‚म‚भ्युप‚ग‚न्त‚व्य‚म‚न्य‚था प्र‚त्य‚क्ष‚त्वासिद्धेः । कुतो य ‚{\tiny $_{6}$}‚ स्मादिद‚मेवेत्यादि सुबोधं । ‚{\tiny $_{lb}$}‚अय‚म्पुन‚र्घ‚टादिर्भ‚व‚द्भीरूपादिव्य‚तिरेकेणाभ्युप‚ग‚तः । को साव‚मूल्य‚दान‚क्र‚यीयः । ‚{\tiny $_{lb}$}‚त‚देतेन नाय‚मीदृशो लोक‚व्य‚व‚हार‚प‚द्ध‚ति ‚{\tiny $_{7}$}‚ म‚व‚त‚र‚तीत्याच‚ष्टे । स स्व‚रूप‚ञ्च ‚{\tiny $_{lb}$}‚नोत्क‚र्षेण द‚र्श‚य‚त्य‚प्र‚तिभास‚मान‚त्वाप्र‚त्य‚क्ष‚ताञ्च स्वीक‚र्त्तुमिच्छ‚ति दार्श‚नं स्पार्श‚{\tiny $_{lb}$}‚न‚ञ्च द्र‚व्य‚मिति सिद्धान्ते पाठात् । इत्येत‚दात्म‚नि ‚{\tiny $_{8}$}‚ र‚न्त‚र‚प्रेमाणः सुहृदः प्र‚त्येष्य‚न्ती‚{\tiny $_{lb}$}‚त्त्य‚ध्याह‚र्त्त‚व्यं । मूल्य‚दान‚क्र‚या विद्य‚न्तेस्येति मूल्य‚दान‚क्र‚यी न त‚थेति वृत्तिः । अथ‚वा ‚{\tiny $_{lb}$}‚क्रेतुं शीलं य‚स्यासौ क्र‚यी मूल्य‚दानेन क्र ‚{\tiny $_{9}$}‚ \leavevmode\ledsidenote{\textenglish{28a/msK}} यी न त‚था । क‚थ‚मेत‚दित्याह । \quotelemma{यः प्र‚त्य‚क्ष‚ता} ‚{\tiny $_{lb}$}‚ \cite[3b2]{vn-msN} मित्यादि । मूल्य‚दान‚ञ्चात्र स्व‚रूपार्प‚ण‚मित्युप‚ह‚स‚ति ।
	{\color{gray}{\rmlatinfont\textsuperscript{§~\theparCount}}}
	\pend% ending standard par
      ‚{\tiny $_{lb}$}‚

	  
	  \pstart \leavevmode% starting standard par
	न‚नु चैको घ‚ट इति प्र‚त्य‚य‚व्य‚प‚देश‚स‚द्भावाद्रूपादिव‚त् त‚द‚स्त्येव । त‚त्क‚थ‚{\tiny $_{lb}$}‚म‚स्यास‚त्व‚मिति चेदाह \quotelemma{बुद्धिश‚ब्दाद‚योपि व्याख्या} ता न च स‚र्व्व इत्यादिना । आदि‚{\tiny $_{lb}$}‚श‚ब्देन त‚द्भेदाभेदोपादानं । य‚दि तैर्वुद्धिव्य‚प‚देशादिभिस्त‚स्य साध‚न‚सिद्धि ‚{\tiny $_{2}$}‚ ‚{\tiny $_{lb}$}‚रिष्य‚ते । स्यादेत‚त्प्र‚तिभास‚मान‚म‚पि द्र‚व्यं ल‚व‚ण‚र‚साभिभ‚वे ख‚ण्ड‚र‚स‚व‚न्नोप‚ल‚क्ष्य‚ते । ‚{\tiny $_{lb}$}‚त‚त‚स्त‚त्प्र‚साध‚नाय लिङ्ग‚मुच्य‚त इत्य‚त आह । \quotelemma{न च प्र‚त्य‚क्ष‚स्यार्थ‚स्य रूपानुप ‚{\tiny $_{3}$}‚ ल‚{\tiny $_{lb}$}‚क्ष‚णं} \cite[3b3]{vn-msN} युक्तं । द्र‚व्यान्त‚रेणान‚भिभ‚वे स‚ति । अभिभ‚वे तु युक्त‚मेव । य‚था ‚{\tiny $_{lb}$}‚ख‚ण्डादिर‚स‚स्य । न चात्र केन‚चिद‚भिभ‚वोस्ति । नीलादिभिर‚स्तीति चेति । न ‚{\tiny $_{lb}$}‚म‚ह‚त्य‚ने ‚{\tiny $_{4}$}‚ क‚द्र‚व्य‚व‚त्ताद्रूपाच्चोप‚ल‚द्धिः । त‚था रूप‚संस्काराभावाद्या वानुप‚ल‚ब्धि‚{\tiny $_{lb}$}‚रित्युक्तं । त‚स्य चानुप‚ल‚क्ष‚णे तेषाम‚प्य‚नु[प]ल‚क्ष‚ण‚प्र‚स‚ङ्गः । त‚त‚श्च स ‚{\tiny $_{5}$}‚ र्व्व‚{\tiny $_{lb}$}‚प‚दार्थानाम‚नुप‚ल‚क्ष‚णाल्लोक‚व्य‚व‚हारोच्छेद एव भ‚वेदिति म‚न्य‚ते । येनानुप‚ल‚क्ष‚णेन ‚{\tiny $_{lb}$}‚त‚स्याव‚य‚विनः साध‚नाय लिङ्ग‚मुच्य‚ते । त‚द्भाव‚साध ‚{\tiny $_{6}$}‚ न‚ञ्च लिङ्ग‚म‚भ्युप‚ग‚म्येत ‚{\tiny $_{lb}$}‚त‚द्भाष्य‚ते न तु त‚द्ग‚म‚कं लिङ्गं किञ्चिद‚प्य‚स्ति । य‚थोक्त‚म्प्राक् । त‚त्प्र‚तिपाद‚क‚{\tiny $_{lb}$}‚प्र‚माणाभावेपि त‚द‚स्त्येवेति चेदाह [।] \quotelemma{अप्र‚त्य‚क्ष‚त्वेप्य‚प्र‚माण ‚{\tiny $_{7}$}‚ स्य स‚त्वोप‚ग‚मो‚{\tiny $_{lb}$}‚ऽयुक्त} \cite[3b3]{vn-msN} इति । अप्र‚माण‚स्येत्य‚नेन प्र‚त्य‚क्ष‚व्य‚तिरिक्त‚त‚त्प्र‚साध‚क‚प्र‚माणा‚{\tiny $_{lb}$}‚भाव‚माह । य‚स्य \quotelemma{स‚द्भाव‚साध‚कं प्र‚माणं नास्ति न त‚द‚स्ती} त्य‚ङ्गीक‚र्त्त ‚{\tiny $_{8}$}‚ व्यं । य‚था‚{\tiny $_{lb}$}‚न‚भ‚स्त‚ले क‚म‚लं नास्ति चाव‚य‚विनोऽस्तित्व‚साध‚कं प्र‚माण‚मिति स‚द्व्य‚व‚हार‚प्र‚ति‚{\tiny $_{lb}$}‚षेध‚फ‚लाम‚नुप‚ल‚ब्धिं म‚न्य‚ते । एवं विस्त‚रेणैक‚द्र‚व्याभावं प्र ‚{\tiny $_{9}$}‚ \leavevmode\ledsidenote{\textenglish{28b/msK}} तिपाद्य प्र‚कृत‚मुप‚{\tiny $_{lb}$}‚ \leavevmode\ledsidenote{\textenglish{39/s}} संह‚र‚ति । \quotelemma{त‚दि} त्यादिना । मूल‚प्र‚क‚र‚ण‚म‚पि निग‚म‚य‚ति । \quotelemma{एव‚न्ताव} दि \cite[3b3]{vn-msN} ‚{\tiny $_{lb}$}‚त्यादिना । अत एव न तेषाम्बुद्ध्यादीनाम्विप‚र्य‚यात्तेषां स‚त्तादीनाम्वि ‚{\tiny $_{1}$}‚ प‚र्य‚योऽभावः । ‚{\tiny $_{lb}$}‚योहि य‚स्य भाव‚मेव न साध‚य‚ति स क‚थ‚मिव व‚र्त्त‚मान‚स्त‚द‚भावं साध‚य‚तीत्याकूतं । ‚{\tiny $_{lb}$}‚य‚दि नाम बुद्ध्याद‚यः स‚त्ताम्भेदाभेदौ वा न साध‚य‚न्त्य‚र्थ‚क्रि ‚{\tiny $_{2}$}‚ या तु तान्साध‚यिष्य‚तीत्य‚त ‚{\tiny $_{lb}$}‚आह । \quotelemma{अर्थंक्रियात‚स्तु स‚त्ताव्य‚व‚हारः स्यादि} \cite[3b4]{vn-msN} ति त‚ल्ल‚क्ष‚ण‚त्वात् स‚त्त्व‚स्येति ‚{\tiny $_{lb}$}‚भावः । अनेनाव‚योर‚त्र साम्य‚मेवेति द‚र्श‚य‚ति । अन्य ‚{\tiny $_{3}$}‚ त्र तु विवाद इत्याह । \quotelemma{न स‚त्ता‚{\tiny $_{lb}$}‚भेदाभेद‚व्य‚व‚हार} \cite[3b4]{vn-msN} इति । कुत एक‚स्याप्य‚नेकार्थ‚क्रियाद‚र्श‚नात् । त‚त्र नैक‚{\tiny $_{lb}$}‚प्र‚त्य‚य‚ज‚नितं किञ्चिद‚स्ति त‚त्क‚थ ‚{\tiny $_{4}$}‚ मेव‚मुच्य‚ते । स‚त्य‚मेत‚देकं तु ब‚ह‚वीषु साम‚ग्रीषु ‚{\tiny $_{lb}$}‚व‚र्त्त‚त इत्य‚नेकार्थ‚कृदित्युच्य‚ते । य‚दाह ॥
	{\color{gray}{\rmlatinfont\textsuperscript{§~\theparCount}}}
	\pend% ending standard par
      ‚{\tiny $_{lb}$}‚
	  \bigskip
	  \begingroup
	
	    
	    \stanza[\smallbreak]
	  \flagstanza{\tiny\textenglish{...11}}{\normalfontlatin\large ``\qquad}न किञ्च‚देक‚मेक‚स्मात् साम‚ग्र्या स‚र्व्व‚स‚म्भ‚वः ।&‚{\tiny $_{lb}$}‚एकं‚{\tiny $_{5}$}‚स्याद‚पि साम‚ग्र्योरित्युक्तं त‚द‚नेक‚कृदिति ॥ [११]{\normalfontlatin\large\qquad{}"}\&[\smallbreak]
	  
	  
	  
	  \endgroup
	‚{\tiny $_{lb}$}‚

	  
	  \pstart \leavevmode% starting standard par
	किम्व‚त् । य‚था प्र‚दीप‚स्य विज्ञान‚स्य व‚र्त्तिविकार‚स्य ज्वालान्त‚र‚स्य च स्व‚प‚र‚{\tiny $_{lb}$}‚स‚न्तान‚स‚म्ब‚न्धिक्ष‚णान्त‚र‚स्यो ‚{\tiny $_{6}$}‚ त्पाद‚नानि त‚देवं स‚त्ताभेद‚व्य‚व‚हाराभावे कार‚णं । ‚{\tiny $_{lb}$}‚क‚थ‚मेक‚म‚नेकं कार्य‚मुत्पाद‚य‚तीति चेति । एक‚स्यैव ईदृश‚स्यानेक‚कार्य‚ज‚न‚नात्तुर्या‚{\tiny $_{lb}$}‚तिश‚य‚क्रोडी ‚{\tiny $_{7}$}‚ कृतं रूप‚व‚तः स्व‚हेतुभ्यः संजात‚त्त्वादिति भावो न्याय‚त‚त्त्व‚विदः । ‚{\tiny $_{lb}$}‚त‚थानेक‚स्यापि च‚क्षुरादेरेक‚विज्ञान‚क्रियाद‚र्श‚नात् । अभेद‚व्य‚व‚हाराभावे कार ‚{\tiny $_{8}$}‚ ‚{\tiny $_{lb}$}‚ण‚मेत‚त् । \quotelemma{क‚ण‚भु} ग्म‚त‚विप‚र्यासित‚धिय‚स्त्वाहुः । \quotelemma{न ब्रूम} इत्यादि । किन्त‚र्ह्य‚दृष्टार्थ ‚{\tiny $_{lb}$}‚क्रियाभेदेन स‚त्ताभेद इति व‚र्त्त‚ते । त‚देव व्य‚न‚क्ति । यार्थ‚क्रिया \cite[3b6]{vn-msN} म‚ध्वाद्या ‚{\tiny $_{9}$}‚ \leavevmode\ledsidenote{\textenglish{29a/msK}} ‚{\tiny $_{lb}$}‚ह‚र‚णादिल‚क्ष‚णा त‚स्मिन्घ‚टादाव‚दृष्टा स‚ती पुन‚र्दृश्य‚ते । अन्य‚त्र घ‚टादौ । ‚{\tiny $_{lb}$}‚सैव‚म्विधा स‚त्ताभेदं साध‚य‚ति । तेषां घ‚टादीनाम‚नेन व्याप्तिः क‚थिता । किमिव । ‚{\tiny $_{1}$}‚ ‚{\tiny $_{lb}$}‚ \quotelemma{य‚था प‚टेऽदृष्टा स‚त्युद‚क‚धार‚णाद्य‚र्थ‚क्रिया घ‚टे दृश्य‚माना} \cite[3b7]{vn-msN} स‚त्ताभेदं साध‚य‚{\tiny $_{lb}$}‚तीति प्र‚कृतेनाभिस‚म्ब‚न्धः । दृष्टान्त‚क‚थ‚नं चैत‚त् । \quotelemma{अदृष्टा च त‚न्तुषु प्राव‚र‚ण‚द्य ‚{\tiny $_{2}$}‚ र्थ‚{\tiny $_{lb}$}‚\leavevmode\ledsidenote{\textenglish{40/s}} क्रिया प‚टे दृश्य‚त इति} \cite[3b7]{vn-msN} प‚क्ष‚ध‚र्मोप‚द‚र्श‚न‚मिति । त‚स्मात् स‚त्ताभेद‚स्त‚न्तु‚{\tiny $_{lb}$}‚प‚ट‚योः सिद्ध इति शेषः । त‚द‚नेन साध‚न‚फ‚लं स‚ङ्कीर्तितं । स्व‚भाव‚हेतुश्चायं ‚{\tiny $_{3}$}‚ य‚स्मात्त‚द‚{\tiny $_{lb}$}‚दृष्टार्थ‚क्रियाक‚र‚ण‚मात्रानुब‚न्धी स‚त्ताभेद इति । त‚देतेन त‚न्तुभ्यः प‚ट‚स्यान्य‚त्वं ‚{\tiny $_{lb}$}‚साध‚य‚न्न‚र्थान्त‚र‚भूताव‚य‚विसिद्धिं म‚न्य‚ते ॥ \quotelemma{आचार्य‚स्त्} वाह ॥ \quotelemma{सिध्य‚त्येवं त‚न्तु‚{\tiny $_{lb}$}‚प‚ट‚योः स‚त्ताभेद} इति प्र‚कृतं ॥ वांछितार्थ‚सिद्धिस्तु भ‚व‚तो नैवास्तीत्य‚भिप्राय‚वा ‚{\tiny $_{lb}$}‚नाह । \quotelemma{अर्थान्त‚र‚न्त‚थाप्य‚व‚य‚वी न ‚{\tiny $_{5}$}‚ सिध्य‚तीति} \cite[3b7]{vn-msN} कुत एत‚द्य‚तो य‚थाप्र‚त्य‚य‚{\tiny $_{lb}$}‚म‚स्यां संस्कार‚संत‚तौ स्व‚भाव‚भेदोत्प‚त्तेः कार‚णाद‚र्थ‚क्रियाभेदः प्राव‚र‚णादिल‚क्ष‚णो ‚{\tiny $_{lb}$}‚भ‚व‚ति । ‚{\tiny $_{6}$}‚ एत‚दुक्तं भ‚व‚ति पिण्डीकृतेभ्य‚स्त‚न्तुभ्य उपादान‚कार‚ण‚भूतेभ्यः ‚{\tiny $_{lb}$}‚कुविन्दादिस‚ह‚कारिप्र‚त्य‚य‚स‚न्निधानाच्च विशिष्ट‚स‚न्निवेशाव‚च्छिन्ना एव ‚{\tiny $_{lb}$}‚त‚न्त‚वो जायं ‚{\tiny $_{7}$}‚ ते । ये प्राव‚र‚णाद्य‚र्थ‚क्रिया[या]मुप‚युज्य‚न्ते । तेभ्य‚श्च पूर्व्वेभ्यः प‚ट‚स्या‚{\tiny $_{lb}$}‚न्य‚त्व‚मिष्ट‚मेवास्माभिर‚पि । न तु विशिष्ट‚संस्थानाव‚च्छिन्नेभ्य इति त्य‚ज्य‚तामिय‚म‚र्था‚{\tiny $_{lb}$}‚न्त‚रा ‚{\tiny $_{8}$}‚ \leavevmode\ledsidenote{\textenglish{29b/msK}} व‚य‚विसिद्धिप्र‚त्याशेति । त‚देतेनैवा \quotelemma{विद्ध‚क‚र्ण्णो} क्तं पूर्व्वोत्त‚र‚काल‚भावित्वादित्यादि ‚{\tiny $_{lb}$}‚त‚त्साध‚न‚म‚प‚ह‚स्तितं वेदित‚व्यं । अस्माभिस्तु विस्त‚रेण प्राक् प्र‚युक्त‚मेवेति । ‚{\tiny $_{lb}$}‚न पुन‚र्योज्य‚ते । किम्व‚त् प्र‚त्य‚य‚व‚सात्\edtext{}{\lemma{सात्}\Bfootnote{? व‚शात्}}स्व‚भाव‚विशोषोत्प‚त्ते \quotelemma{र‚र्थ‚किया‚{\tiny $_{lb}$}‚भेद} \cite[3b7]{vn-msN} इत्याह । \quotelemma{अर‚णिनिर्म‚थ‚ना} \cite[3b7]{vn-msN} दित्यादि सुज्ञानं [।] दृष्टान्तं प्र‚द‚र्श्य‚{\tiny $_{lb}$}‚दार्ष्टान्तिक‚माह । \quotelemma{त‚था य‚थे} \cite[3b8]{vn-msN} त्यादि ‚{\tiny $_{1}$}‚ । अनेनैव य‚थाप्र‚त्य‚य‚स्व‚भाव‚भेदेन‚{\tiny $_{lb}$}‚य‚देके चोद‚य‚न्ति । न‚नु च त‚न्त‚वः प‚ट इति बुद्धिव्य‚प‚देश‚भेदात् । क‚थ‚म‚स्यान्य‚त्वं ‚{\tiny $_{lb}$}‚नास्तीति त‚त्प्र‚तिक्षिप्त‚मित्याकूत‚वानाह ‚{\tiny $_{2}$}‚ । \quotelemma{एतेन बुद्धिव्य‚प‚देश‚भेदौ व्याख्याता‚{\tiny $_{lb}$}‚विति} \cite[3b8]{vn-msN} । त‚त्रैवं स्थिते य‚दुक्तं प्राक्त्व‚याऽर्थ‚क्रियातः स‚द्व्य‚व‚हार‚सिद्धिर्भ‚व‚ति ‚{\tiny $_{lb}$}‚विप‚र्य‚याच्चार्थ‚क्रिया निवृत्ते विप‚र्य‚योऽस‚द्व्य‚व‚हार ‚{\tiny $_{3}$}‚ इति स‚त्य‚मेत‚त् । एताव‚त्तु‚{\tiny $_{lb}$}‚ब्रूमः । स एव विप‚र्य‚योऽर्थ‚क्रियाया अनुप‚ल‚ब्धिल‚क्ष‚ण‚प्राप्तेषु न सिध्य‚ति । त‚थाहि ‚{\tiny $_{lb}$}‚य‚द्य‚य‚मुप‚ल‚ब्धिल‚क्ष‚ण‚प्राप्तानुप ‚{\tiny $_{4}$}‚ ल‚म्भो नेष्य‚ते त‚दास्यार्थ‚क्रियासाम‚र्थ्यं नास्तीति ‚{\tiny $_{lb}$}‚क‚थ‚म‚धिग‚तं भ‚व‚ता [।] न चानुप‚ल‚ब्धिमात्रादिति युक्त‚म्व‚क्तुँ । त‚स्य व्य‚भिचारात् । ‚{\tiny $_{lb}$}‚ \leavevmode\ledsidenote{\textenglish{41/s}} त‚त्र पुन‚र‚निच्छ‚तो ‚{\tiny $_{5}$}‚ प्यायातं त‚व । य‚स्येद‚म‚र्थ‚क्रियासाम‚र्थ्य‚मुप‚ल‚ब्धिल‚क्ष‚ण‚प्राप्तं ‚{\tiny $_{lb}$}‚स‚न्नोप‚ल‚भ्य‚ते सोऽस‚द्व्य‚व‚हार‚विष‚य इति । कुतः साम‚र्थ्य‚ल‚क्ष‚ण‚त्वात् स‚त्व‚स्य ‚{\tiny $_{6}$}‚ । ‚{\tiny $_{lb}$}‚भ‚व‚त्वेव‚ङ्को दोष इति चेदाह । \quotelemma{त‚थापि कोतिश‚यः} \cite[3b10]{vn-msN} पूर्व्व‚काद‚स्मादुप‚{\tiny $_{lb}$}‚व‚र्णितादुप‚ल‚भ्यानुप‚ल‚म्भात् । अस्य साम‚र्थ्यानुप‚ल‚म्भ‚स्य भ‚व‚त्प‚रिक‚ल्पित‚स्य । ‚{\tiny $_{lb}$}‚स्यात् म ‚{\tiny $_{7}$}‚ तं स स्व‚भाव‚स्यैवानुप‚ल‚म्भोऽयं तु पुनः साम‚र्थ्य‚स्येत्य‚त आह । \quotelemma{न ही}\cite[3b10]{vn-msN} ‚{\tiny $_{lb}$}‚त्यादि । त‚था च त‚स्य साम‚र्थ्य‚स्य योनुप‚ल‚म्भः स स्व‚भाव‚स्यैव इति त‚स्मात् ‚{\tiny $_{lb}$}‚पूर्व‚कैव स्व‚भावा ‚{\tiny $_{8}$}‚ नुप‚ल‚ब्धिरेवेयं साम‚र्थ्यानुप‚ल‚ब्धिः । त‚स्माद् दृश्यानुप‚ल‚ब्धिरेवा‚{\tiny $_{lb}$}‚स‚द्व्य‚व‚हार‚साध‚नेति स्थित‚मेत‚त् । य‚स्माच्चैव‚न्त‚स्माद‚नेन वादिना क्व‚चिच्छ‚श‚{\tiny $_{lb}$}‚विषाणादाव‚स‚द्व्य‚व ‚{\tiny $_{9}$}‚ \leavevmode\ledsidenote{\textenglish{30a/msK}} व‚हार‚म‚भ्युप‚ग‚च्छ‚ता दृश्यानुप‚ल‚म्भाद‚भ्युप‚ग‚न्त‚व्यो- ‚{\tiny $_{lb}$}‚ऽन्य‚स्य त‚त्प्र‚तिप‚त्त्युपाय‚स्याभावादिति भावः । त‚तः सोनुप‚ल‚म्भोऽन्य‚त्रापि सामान्य‚{\tiny $_{lb}$}‚विशेष‚संयोगाव‚य ‚{\tiny $_{1}$}‚ विद्र‚व्यादौ त‚थाविधे उप‚ल‚ब्धिल‚क्ष‚ण‚प्राप्ते अविशेष इति ‚{\tiny $_{lb}$}‚सोपि सामान्य‚विशेषादिस्त‚थास्त्व‚स‚द्व्य‚व‚हार‚विष‚य‚त्त्वेनास्तीत्य‚र्थः । स्यादेत‚न्नैव ‚{\tiny $_{lb}$}‚सामान्य‚विशे ‚{\tiny $_{2}$}‚ षादिस्त‚थाविधोनुप‚ल‚ब्धोस्य स‚द्व्य‚व‚हार‚विष‚य इत्य‚त आह । न वा ‚{\tiny $_{lb}$}‚क्व‚चिच्छ‚श‚विषाणादाव‚स‚द्व्य‚व‚हारो ऽभ्युप‚ग‚न्त‚व्यः । कुतो विशेषाभावाद‚नुप ‚{\tiny $_{lb}$}‚ल‚म्भ‚स्य ‚{\tiny $_{3}$}‚ । अय‚म‚त्रार्थो द्व्योर‚नुप‚ल‚म्भे त‚न्निमित्ते तुल्येपि य‚द्य‚स‚द्व्य‚व‚हारः ‚{\tiny $_{lb}$}‚सामान्यादौ नोत्प‚द्य‚ते । अन्य‚त्रापि त‚र्हि स नाभ्युपेयो विशेष‚हेत्व‚भावादि ‚{\tiny $_{4}$}‚ त्य‚नेन च ‚{\tiny $_{lb}$}‚पूर्व्वोक्त‚मेव क‚स्य चिद‚स‚तोभ्युप‚ग‚मे त‚ल्ल‚क्ष‚णाविशेषादिति स्मार‚य‚ति । त‚स्मा ‚{\tiny $_{lb}$}‚ \quotelemma{त्स‚र्व्व एवंविधो दृश्यानुप‚ल‚म्भोऽस‚द्व्य‚व‚हार‚स्य वि ‚{\tiny $_{5}$}‚ ष‚य} \cite[4a2]{vn-msN} इति व्याप्तिः । ‚{\tiny $_{lb}$}‚अनुप‚ल‚ब्धौ सिद्धेतिशेषः ॥ ० ॥
	{\color{gray}{\rmlatinfont\textsuperscript{§~\theparCount}}}
	\pend% ending standard par
      ‚{\tiny $_{lb}$}‚\textsuperscript{\textenglish{42/s}}

	  
	  \pstart \leavevmode% starting standard par
	\hphantom{.}\quotelemma{कापिला} स्त्वाहुः । स‚र्व्व‚मेव स‚र्व्वात्म‚कं अन्य‚था य‚दि मृत्पिण्ड‚दुग्ध‚बीजादिषु ‚{\tiny $_{lb}$}‚घ‚ट‚द‚ध्य‚ङ्कुराद ‚{\tiny $_{6}$}‚ यो न विद्य‚न्त एव श‚क्त्यात्म‚ना त‚दा क‚थं पुन‚स्तेभ्य‚स्तेषामुत्प‚त्तिः । ‚{\tiny $_{lb}$}‚न‚हि श‚श‚विषाण‚म‚विद्य‚मान‚न्त‚त्रोदेति । एव‚ञ्च स‚ति दृश्यः स‚न्न‚नुप‚ल‚ब्धोपि क‚श्चिद् ‚{\tiny $_{lb}$}‚घ‚टा ‚{\tiny $_{7}$}‚ दिः क्व‚चिद्देशादौ क‚थ‚ञ्चिच्च संस्थान‚विशेषादिना नैवाभाव‚व्य‚व‚हार‚विष‚यो ‚{\tiny $_{lb}$}‚भ‚व‚तीति तेषां म‚त‚मास‚ङ्क\edtext{}{\lemma{ङ्क}\Bfootnote{? श‚ङ्क}}ते । \quotelemma{नैवे} \cite[4a3]{vn-msN} त्यादिना ।
	{\color{gray}{\rmlatinfont\textsuperscript{§~\theparCount}}}
	\pend% ending standard par
      ‚{\tiny $_{lb}$}‚
	  \bigskip
	  \begingroup
	
	    
	    \stanza[\smallbreak]
	  \flagstanza{\tiny\textenglish{...12}}{\normalfontlatin\large ``\qquad}एत‚त् साङ्ख्य‚प‚शोः कोन्यः स‚ल‚ज्जो व‚क्तुमीह‚ते ।&‚{\tiny $_{lb}$}‚अदृष्ट‚पूर्व्व‚म‚स्तीति तृणाग्रे क‚रिणां श‚त‚म् [। १२] \href{http://sarit.indology.info/?cref=pv.2.0}{प्र‚माण‚वार्त्तिके २ प‚रि०}{\normalfontlatin\large\qquad{}"}\&[\smallbreak]
	  
	  
	  
	  \endgroup
	‚{\tiny $_{lb}$}‚

	  
	  \pstart \leavevmode% starting standard par
	\hphantom{.}इत्य‚भिप्राय‚वानाह । \quotelemma{स‚र्व्व‚स्ये} \cite[4a3]{vn-msN} त्यादि । य‚द्य‚दृष्ट‚म‚पि त‚त्रास्ति त‚दा‚{\tiny $_{lb}$}‚स‚र्व्व एव क्षीराद‚यः स‚र्व्वे र्घ‚टादिरूपै ‚{\tiny $_{9}$}‚ \leavevmode\ledsidenote{\textenglish{30b/msK}} र‚नुम‚त‚त्वात् त‚त्साध्याम‚र्थ‚क्रियाङ्कुर्युरित्य‚र्थः । ‚{\tiny $_{lb}$}‚किञ्चेद‚म‚प‚रं न स्यादिद‚न्द‚ध्यादिकार्य‚म‚तः क्षीरादेर्भ‚व‚ति नान्य‚तो ज‚लादेः । ‚{\tiny $_{lb}$}‚य‚दि वा नातः क्षीरादेरिदं म‚ध्वादि ‚{\tiny $_{1}$}‚ कं त‚थेद‚ङ्कुङ्कुमादिक‚मिह \quotelemma{क‚स्मीरा\edtext{}{\lemma{स्मीरा}\Bfootnote{? ‚{\tiny $_{lb}$}‚क‚श्मीरा}}} दिदेशे नेद‚मिह \quotelemma{माल‚व‚कादि} देशे य‚दि वा नेदं च‚न्द‚नादिक‚मिह । त‚थेद‚ङ्कु‚{\tiny $_{lb}$}‚न्दादिक‚मिदानीं शिशिर‚स‚म‚ये न‚त्विद‚मि ‚{\tiny $_{2}$}‚ दानीन्निदाघ‚काले । अथ‚वा नेदं क‚म‚लादिक‚{\tiny $_{lb}$}‚मिदानीन्त‚थेदं ख‚ण्डादिक‚मेवं माधुर्यादिगुण‚विशिष्टं । नेद‚मेव‚ङ्क‚टुकादिरूपं । ‚{\tiny $_{lb}$}‚य‚द्वा नेदं निम्वादिक‚मे ‚{\tiny $_{3}$}‚ व‚मिति व्याख्यात‚व्यं [।] किङ्कार‚ण‚मेव‚मेत‚दित्य‚त आह । ‚{\tiny $_{lb}$}‚ \quotelemma{क‚स्य‚चिद‚पी} \cite[4a4]{vn-msN} त्यादि । इद‚मेव‚म्बिध‚म‚धुनास्य रूप‚न्नास्तीति योय‚म्विवेको‚{\tiny $_{lb}$}‚ऽभाव‚स्त ‚{\tiny $_{4}$}‚ स्य हेतोर‚भावात् । स‚र्व्व‚स्य स‚र्व्व‚रूपाणां स‚र्व्व‚दानुवृत्तेरिति म‚तिः । न‚न्विद‚{\tiny $_{lb}$}‚म‚न‚न्त‚र‚मेव व‚स्तुतोऽभिहित‚मेव स‚र्व्व स‚र्व्व‚त्र स‚र्व्व‚दा स‚मुप ‚{\tiny $_{5}$}‚ युज्येतेत्य‚त्र त‚क्तिमिदं ‚{\tiny $_{lb}$}‚पुन‚श्च‚र्वित‚च‚र्व्व‚ण‚मास्थीय‚त इति चेत् । स‚त्यं पूर्वं कार‚ण‚ग‚तो व्यापारः क‚थितो‚{\tiny $_{lb}$}‚ऽधुना तु कार्य‚ग‚त इति विशेषा ‚{\tiny $_{6}$}‚ द‚दोषः । इद‚ञ्चान्य‚त‚र‚मुखेण\edtext{}{\lemma{मुखेण}\Bfootnote{? मुखेन}}दूष‚ण‚{\tiny $_{lb}$}‚व‚च‚नं शिष्य‚व्युत्पाद‚नाय । त‚त‚श्च भेदाभावान्न विद्येते अन्व‚य‚व्य‚तिरेकौ य‚स्मिं ‚{\tiny $_{lb}$}‚निति विग्र‚हः । इद‚म‚त्रास्तीत्या ‚{\tiny $_{7}$}‚ द्य‚न्व‚यो नास्तीति व्य‚तिरेकः । प‚रः प्राह [।] ‚{\tiny $_{lb}$}‚ \quotelemma{अव‚स्थेत्या} दिना । एत‚दुक्त‚म्भ‚व‚ति । य‚त्र य‚द् व्य‚क्त‚न्त‚त्त‚त्रास्तीत्यादि व्य‚व‚हिर‚य‚ते ‚{\tiny $_{lb}$}‚ \leavevmode\ledsidenote{\textenglish{43/s}} य‚त्र तु य‚न्नैव व्य‚क्त‚न्त‚त्र त‚न्नास्तीत्य‚तो ‚{\tiny $_{8}$}‚ ऽय‚म‚दोष इति । \quotelemma{न‚त्वि} \cite[4a5]{vn-msN} त्याद्याचार्यः । ‚{\tiny $_{lb}$}‚‘त एवाव‚स्थानिवृत्तिप्र‚वृत्तिभेदा न स‚म्भ‚व‚न्ति ताव‚कीने द‚र्श‚ने । कुतः स‚र्व्व‚विष‚य‚स्या‚{\tiny $_{lb}$}‚स‚द्व्य‚व‚हार‚स्याभावात् । अथापि क्व‚चि ‚{\tiny $_{9}$}‚ \leavevmode\ledsidenote{\textenglish{31a/msK}} द्विष‚येऽस‚द‚व्य‚व‚हार इष्य‚ते त‚दा त‚स्य कार‚णं ‚{\tiny $_{lb}$}‚भ‚व‚द्भिर्व‚क्त‚व्य‚मित्याह । क् \quotelemma{व‚चिदि} \cite[4a5]{vn-msN} त्यादि । य‚दि वाव‚श्य‚म‚नेन क्व‚चित्प‚रि‚{\tiny $_{lb}$}‚क‚ल्पिते व्य‚तिरिक्ताव‚य‚व्यादाव‚स‚द्व्य‚व‚हारों ‚{\tiny $_{1}$}‚ गीक‚र्त्त‚व्यः । स चास्य न युक्तो‚{\tiny $_{lb}$}‚हेत्व‚भावादित्याह । \quotelemma{क्व‚चिदि} त्यादि । कुतः । य‚स्मान्न‚हि अनुप‚ल‚म्भाद‚न्यो व्य‚व‚च्छे‚{\tiny $_{lb}$}‚द‚स्याभाव‚स्य हेतुर‚स्ति प्र‚साध‚क इति शेषः । स च त्व‚या ‚{\tiny $_{2}$}‚ नेष्ट‚क इति भावः । क‚स्मा‚{\tiny $_{lb}$}‚देवं य‚तो विधिना स्व‚भाव‚विरुद्धोप‚ल‚म्भादौ प्र‚तिषेधेन व्याप‚कानुप‚ल‚म्भादौ व्य‚व‚च्छेदे ‚{\tiny $_{lb}$}‚साध्येऽनुप‚ल‚म्भ‚स्यैव स‚र्व्व‚दा साध‚क‚त्वात् । अथा ‚{\tiny $_{3}$}‚ ह‚म‚प्य‚स्मादेवानुप‚ल‚म्भाद् व्य‚व‚{\tiny $_{lb}$}‚च्छेदं साध‚यामीति ब्रूषे त‚द‚त्रापि ब्रूम इत्याह । सोनुप‚ल‚म्भो \quotelemma{य‚त्रैवास्ति स स‚र्व्वो ‚{\tiny $_{lb}$}‚ऽस‚द्व्य‚व‚हार‚विष‚य} \cite[4a6]{vn-msN} इति व‚क्त‚व्यं ॥ ‚{\tiny $_{4}$}‚ किमिति विशेषाभावात् । त‚था ‚{\tiny $_{lb}$}‚च‚घ‚टादि\edtext{}{\lemma{टादि}\Bfootnote{? घ‚टादे}}र‚पि क्व‚चित्प्र‚देश‚विशेषादाव‚स‚द्व्य‚व‚हार‚विष‚य‚त्वं सिद्धं । ‚{\tiny $_{lb}$}‚त‚थाविध‚स्यानुप‚ल‚म्भ‚स्यात्रापि भावात् त‚त्किं ‚{\tiny $_{5}$}‚ ब्रूषे । नैव क्व‚चित् क‚श्चिद् दृष्टो‚{\tiny $_{lb}$}‚प्य‚स‚द्व्य‚व‚हार‚विष‚य इति अभिप्रायः । अन्य‚था अत्रापि व्य‚तिरिक्ताव‚य‚व्यादौ ‚{\tiny $_{lb}$}‚मा भूद‚स‚द्व्य‚व‚हार इति याव‚त् ।
	{\color{gray}{\rmlatinfont\textsuperscript{§~\theparCount}}}
	\pend% ending standard par
      ‚{\tiny $_{lb}$}‚

	  
	  \pstart \leavevmode% starting standard par
	\hphantom{.}\quotelemma{स‚र्व्व ‚{\tiny $_{6}$}‚ प्र‚माण‚निवृत्तिरि} \cite[4a7]{vn-msN} त्यादि प‚रः । त‚था चायुक्तं उक्तं । न‚हि ‚{\tiny $_{lb}$}‚अनुप‚ल‚म्भाद‚न्यो व्य‚व‚च्छेद‚हेतुर‚स्तीति । एव‚ञ्च स‚ति न घ‚ट‚स्यापि क्व‚चिद‚स‚द्व्य‚व ‚{\tiny $_{lb}$}‚हार‚विष‚य‚त्व‚मा ‚{\tiny $_{7}$}‚ ग‚मानुमान‚भावेन स‚र्व्व‚प्र‚माण‚निवृत्तेरेवाभावादिति म‚न्य‚ते । ‚{\tiny $_{lb}$}‚कुतः पुन‚रिद‚म‚तिप्र‚ज्ञाकौश‚ल‚मासादितं भ‚व‚तेत्यागूर्योप‚ह‚स‚ति । \quotelemma{सुकुमार‚प्र‚ज्ञ} \cite[4a7]{vn-msN} ‚{\tiny $_{lb}$}‚इत्या ‚{\tiny $_{8}$}‚ दिना । \quotelemma{न प्र‚स‚ह‚ते प्र‚माण‚चिन्ताप‚रिक्लेश} \cite[4a7]{vn-msN} मिति सुकुमार‚प्र‚ज्ञ‚त्वे ‚{\tiny $_{lb}$}‚कार‚णं ।
	{\color{gray}{\rmlatinfont\textsuperscript{§~\theparCount}}}
	\pend% ending standard par
      ‚{\tiny $_{lb}$}‚

	  
	  \pstart \leavevmode% starting standard par
	\hphantom{.}न‚नु च किम‚त्रायुक्त‚मुक्त‚म‚स्माभिर्येनोप‚ह‚स‚सीत्याह । \quotelemma{न‚ही} \cite[4a8]{vn-msN} त्यादि । ‚{\tiny $_{lb}$}‚अदिश ‚{\tiny $_{9}$}‚ \leavevmode\ledsidenote{\textenglish{31b/msK}} ब्देनाग‚म‚प‚रिग्र‚हः । व्य‚भिचार‚श्च पूर्व‚मेव प्र‚तिपादितः । स‚र्व्व‚प्राणि ‚{\tiny $_{lb}$}‚प्र‚त्य‚क्ष‚निवृत्तिस्त‚र्हि ग‚म‚यिष्य‚तीत्याह । \quotelemma{न स‚र्व्व‚प्र‚त्य‚क्ष‚निवृत्तिरि} \cite[4a8]{vn-msN} ति । ‚{\tiny $_{lb}$}‚ \leavevmode\ledsidenote{\textenglish{44/s}} कुतोऽसिद्धेः । आत्म‚प‚र ‚{\tiny $_{1}$}‚ योर‚प्र‚तिप‚त्तेरित्य‚र्थः । न ह्य‚त्र स‚र्वेषाम्प्र‚त्य‚क्ष‚न्निवृत्त‚मिति ‚{\tiny $_{lb}$}‚निश्च‚ये प्र‚माण‚म‚स्ति किञ्चित् । आत्म‚प्र‚त्य‚क्ष‚निवृत्तिरेव त‚र्हि ग‚म‚य‚तीति चेदाह । ‚{\tiny $_{lb}$}‚ \quotelemma{नात्म‚प्र‚त्य‚क्षा विशे ‚{\tiny $_{2}$}‚ ष‚निवृत्तिर‚पी} \cite[4a8]{vn-msN} ति [।] न केव‚लं पूर्वोक्तेत्य‚पि श‚ब्दः । ‚{\tiny $_{lb}$}‚अविशेषेण निवृत्तिर‚विशेष‚निवृत्तिः । आत्म‚प्र‚त्य‚क्ष‚स्याविशेष‚निवृत्तिरात्म‚प्र‚त्य‚क्षा ‚{\tiny $_{lb}$}‚विशेष‚निवृत्त‚रिति ‚{\tiny $_{3}$}‚ व्युत्प‚त्तिक्र‚मः । स‚न्निहित‚स‚क‚ल‚त‚द‚न्य‚कार‚ण‚स्य त्वात्म‚{\tiny $_{lb}$}‚प्र‚त्य‚क्ष‚स्य निवृत्तिस्त्रिविध‚विप्र‚क‚र्षाविप्र‚कृष्टेऽभाव‚ङ्ग‚म‚य‚त्येवेति क‚थ‚नीयाविशेष ‚{\tiny $_{lb}$}‚वि ‚{\tiny $_{4}$}‚ प्र‚कृष्ट‚व‚च‚नं । य‚स्मात् स‚र्व्व‚प्र‚माण‚निवृत्तिर्न्नास‚द्व्य‚व‚हार‚हेतुस्त‚स्मात् स ‚{\tiny $_{lb}$}‚स्व‚भाव‚विशेष‚स्त्रिविध‚विप्र‚क‚र्षाविप्र‚कृष्ट‚रूपो भावो य‚तः प्र‚माणात्सं ‚{\tiny $_{5}$}‚ निहित‚{\tiny $_{lb}$}‚स‚म‚स्त‚त‚द‚न्य‚त्क्रियादिकार‚णात् प्र‚त्य‚क्षान्निय‚मेन स‚द्व्य‚व‚हार‚म्प्र‚तिप‚द्य‚ते स‚मासाद‚{\tiny $_{lb}$}‚य‚ति । त‚स्यैव य‚थोक्त‚स्य प्र‚माण‚स्य निवृत्तिस्त‚स्य स्व ‚{\tiny $_{6}$}‚ भाव‚विशेष‚स्यास‚द्व्य‚व‚हारं ‚{\tiny $_{lb}$}‚प्र‚साध‚य‚ति । अव‚धार‚ण‚म‚नुमानाव‚ग‚मादिनिवृत्तेर्व्य‚व‚च्छेदाय । किङ्कार‚ण‚न्त‚स्य ‚{\tiny $_{lb}$}‚स्व‚भाव‚विशेष‚स्य स्व‚भाव‚स‚त्तायास्त‚स्य य‚थो ‚{\tiny $_{7}$}‚ क्त‚स्य प्र‚माण‚स्य येयं स‚त्ता त‚या व्याप्तः ‚{\tiny $_{lb}$}‚कार‚णात् । त‚थाहि य‚त्र स तादृग्विधः प‚दार्थ‚स्त‚त्राव‚श्यं तेनापि प्र‚माणेन भ‚वित‚व्यं । ‚{\tiny $_{lb}$}‚स‚म‚र्थ‚स्य कार‚ण‚स्य कार्या ‚{\tiny $_{8}$}‚ व्य‚भिचारात् । एव‚ञ्चैत‚त्प्र‚माणं त‚द्व्याप‚क‚त्वान्निव‚र्त्त‚{\tiny $_{lb}$}‚मानं ताम‚पि वृक्ष‚व‚च्छिंश‚पां निव‚र्त्त‚य‚ति । अनेन च य‚थोक्ताद‚नुप‚ल‚म्भादित्याद्युक्त‚{\tiny $_{lb}$}‚मुप‚संह‚र‚ति स्यादेत ‚{\tiny $_{9}$}‚ \leavevmode\ledsidenote{\textenglish{32a/msK}} दुप‚ल‚ब्धिल‚क्ष‚ण‚प्राप्त‚म‚पि क्षीरादिषु द‚ध्यादिकं न प्र‚त्य‚क्षे ‚{\tiny $_{1}$}‚ ‚{\tiny $_{lb}$}‚णोप‚ल‚भ्य‚तेऽपि त्व‚नुमानेनाश‚क्ताद‚नुत्प‚त्तिरिति । अतो न त‚न्निवृत्याप्य‚स‚द्व्य‚व‚हार‚{\tiny $_{lb}$}‚विष‚य‚त्व‚न्त‚स्येत्य‚त आह । \quotelemma{न चे} \cite[4a9]{vn-msN} त्यादि । येनान्योप‚ल‚ब्धित्वेनानुमाना‚{\tiny $_{lb}$}‚द‚स्यो ‚{\tiny $_{2}$}‚ प‚ल‚ब्धिः स्यात् । किम्पुन‚र‚न्योप‚ल‚ब्धिर्न युज्य‚त इत्याह । \quotelemma{न चेत्या} दि । य‚स्माद‚र्थे‚{\tiny $_{lb}$}‚च‚कारः । त‚स्य रूप‚स्योप‚ल‚ब्धिल‚क्ष‚ण‚प्राप्त‚स्यान्य‚थाभाव‚प्र‚च्युतिर्न्न त‚म‚न्त ‚{\tiny $_{3}$}‚ रेणा‚{\tiny $_{lb}$}‚प्र‚त्य‚क्षः स भावो युक्त इति शेषः । त‚देतेन प्र‚त्य‚क्ष‚मेव त‚स्योप‚ल‚ब्धिर‚पेक्ष‚णीय‚स्य ‚{\tiny $_{lb}$}‚क‚स्य‚चिद्धेतोर‚भावादिति प्र‚साध‚य‚ति । प्र‚यो ‚{\tiny $_{4}$}‚ गः पुनः । य‚द्य‚दास‚न्निहित‚स‚क‚ला‚{\tiny $_{lb}$}‚प्र‚तिब‚द्ध‚साम‚र्थ्य‚कार‚ण‚न्त‚त्त‚दा भ‚व‚त्येव न चाक्षेप‚कारि । य‚था स‚म‚ग्राप्र‚तिह‚त‚{\tiny $_{lb}$}‚साम‚र्थ्य‚कार‚ण‚सा ‚{\tiny $_{5}$}‚ म‚ग्रीकोङ्कुरः । त‚था च क्षीराद्य‚व‚स्थासु य‚थोप‚दिष्ट‚प‚क्ष‚ध‚र्म‚व‚द् ‚{\tiny $_{lb}$}‚द‚ध्यादिविष‚यं ज्ञान‚मिति स्व‚भाव‚हेतुः । द्वितीय‚साध्यापेक्ष‚या व्याप‚क‚वि ‚{\tiny $_{6}$}‚ रुद्धोप‚{\tiny $_{lb}$}‚ \leavevmode\ledsidenote{\textenglish{45/s}} ल‚ब्धिः । अन्य‚था त्व‚न्त‚स्य भ‚व‚त्येवातोयं हेतुर‚सिद्ध इति चेदाह [।] \quotelemma{अन्य‚था भावे} ‚{\tiny $_{lb}$}‚चेष्य‚माणे \quotelemma{त‚देवो} प‚ल‚ब्धिल‚क्ष‚ण‚प्राप्तं द‚ध्यादि \quotelemma{न स्यात्} \cite[4b1]{vn-msN} प्राच्य‚रूपात् प्र‚च्यु ‚{\tiny $_{7}$}‚ तेः । ‚{\tiny $_{lb}$}‚त‚था च त‚द्रूप‚तायां निर‚न्व‚य‚विनास\edtext{}{\lemma{विनास}\Bfootnote{? विनाश}}प्र‚स‚ङ्ग इति भावः । \quotelemma{अपि चे}‚{\tiny $_{lb}$}‚\cite[4b1]{vn-msN}त्यादिनोप‚च‚य‚माह । य‚द‚य‚म्भावः । अजातोऽन‚ष्ट‚श्च रूपातिश‚यो‚{\tiny $_{lb}$}‚स्येति विग्र‚हः ‚{\tiny $_{8}$}‚ [।] नित्य‚मेक‚त्व‚रूप‚त्वाद् द्र‚व्यान्त‚रेण व्य‚व‚धाने दूर‚देश‚स्थितौ च भ‚वे‚{\tiny $_{lb}$}‚युर‚पि प्र‚त्य‚क्षाप्र‚त्य‚क्ष‚त्वाद‚य इत्य‚त आह । \quotelemma{अव्य‚व‚धान‚दूर‚स्थान} \cite[4b1]{vn-msN} इति । ‚{\tiny $_{lb}$}‚न विद्य‚ते व्य‚व‚धान‚दू ‚{\tiny $_{9}$}‚ \leavevmode\ledsidenote{\textenglish{32b/msK}} र‚स्थाने चास्येति विग्र‚हः । क्व‚चिद‚व्य‚व‚धानादूर‚स्थान ‚{\tiny $_{lb}$}‚इति प‚ठ्य‚ते । त‚त्र व्य‚व‚धान‚दूर‚स्थान‚श‚ब्द‚योः प्र‚त्येकं न‚ञा स‚मासं कृत्वा प‚श्चा[द्] ‚{\tiny $_{lb}$}‚विशेष‚ण‚स‚मासः कार्यः । क‚ञ्चित्पुरुष‚म‚पेक्ष्य कोपि प्र‚त्य‚क्षोऽन्य‚ञ्चापेक्ष्य प्र‚त्य‚क्ष ‚{\tiny $_{lb}$}‚इति न विरोध इत्याह । \quotelemma{त‚स्यैव} \cite[4b2]{vn-msN} । त‚स्याप्युन्मीलित‚लोच‚नाद्य‚व‚स्थायां ‚{\tiny $_{lb}$}‚प्र‚त्य‚क्षोऽन्य‚दा चाप्र‚त्य‚क्ष इति न ‚{\tiny $_{2}$}‚ काचित् क्ष‚तिरित्याह । \quotelemma{त‚द‚व‚स्थेन्द्रियादेरेव} ‚{\tiny $_{lb}$}‚ \cite[4b2]{vn-msN} त‚द‚व‚स्थ‚म‚विकृत‚मिन्द्रिय‚म‚स्येति विग्र‚हः । आदिग्र‚ह‚णं म‚न‚स्काराद्याक्षेपाय । ‚{\tiny $_{lb}$}‚क‚दाचिद‚भिव्य‚क्त‚वेलायां ‚{\tiny $_{3}$}‚ प्र‚त्य‚क्षो भ‚व‚ति क‚दाचिच्चान‚भिव्य‚क्त‚क्षीरादिवेला‚{\tiny $_{lb}$}‚याम‚प्र‚त्य‚क्ष‚श्चेति येन प्र‚त्य‚क्षाप्र‚त्य‚क्ष‚त्वेन क‚दाचिद‚नुमान‚स्योप‚ल‚ब्धिर‚श‚क्ताद‚नुत्प‚{\tiny $_{lb}$}‚त्ते ‚{\tiny $_{4}$}‚ रिति । क‚दाचित्तु व्य‚क्ताव‚स्थायां प्र‚त्य‚क्षं । किं पुन‚र‚त्रायुक्तं । येनैवं ब्रूष ‚{\tiny $_{lb}$}‚इति चेदाह । एक‚स्मिन्नेवान‚तिश‚ये द‚ध्यादाव‚मीषां प्र‚काराणाम्प्र ‚{\tiny $_{5}$}‚ त्य‚क्षाप्र‚त्य‚क्ष‚त्त्वा‚{\tiny $_{lb}$}‚दीनां \quotelemma{विरोधादिति} \cite[4b2]{vn-msN} । ये प‚र‚स्प‚र‚विरुद्ध‚रूपा न तेषामेक‚त्रान‚तिश‚ये स‚म्भ‚वः । ‚{\tiny $_{lb}$}‚त‚द्य‚था शीतोष्ण‚स्प‚र्शादीनां । प‚र‚स्प‚र‚विरु ‚{\tiny $_{6}$}‚ द्धाश्च प्र‚त्य‚क्षाप्र‚त्य‚क्ष‚त्वाद‚यः । इति ‚{\tiny $_{lb}$}‚व्याप‚क‚विरुद्धोप‚ल‚ब्धिम्म‚न्य‚ते ॥
	{\color{gray}{\rmlatinfont\textsuperscript{§~\theparCount}}}
	\pend% ending standard par
      ‚{\tiny $_{lb}$}‚

	  
	  \pstart \leavevmode% starting standard par
	\hphantom{.}प‚रः प्राह ॥ \quotelemma{नान‚तिश‚य} \cite[4b3]{vn-msN} इति । एक‚स्याव्य‚क्ताव‚स्थाल‚क्ष‚ण‚स्या‚{\tiny $_{lb}$}‚तिश‚य‚स्य निवृत्याऽप‚र ‚{\tiny $_{7}$}‚ स्य व्य‚क्ताव‚स्थाल‚क्ष‚ण‚स्योत्प‚त्या च क्षीरं द‚धीति ‚{\tiny $_{lb}$}‚व्य‚व‚हार‚स्योप‚ग‚मात् । अनेनाजातान‚ष्ट‚रूपातिश‚य इत्यादेर‚सिद्ध‚त्व‚माह । न ‚{\tiny $_{lb}$}‚ताव‚द‚य‚म‚तिश‚यो भ‚व‚द्भिर‚ति ‚{\tiny $_{8}$}‚ श‚य‚व‚द् भाव‚व्य‚तिरिक्तोऽभ्युप‚ग‚तोऽभ्युप‚ग‚मे ‚{\tiny $_{lb}$}‚वा त‚द‚व‚स्थोऽन‚न्त‚राभिहितो दोषः स्यात् ‚{\tiny $_{9}$}‚ \leavevmode\ledsidenote{\textenglish{33a/msK}} [।] त‚स्माद‚व्य‚तिरिक्त एवायं ‚{\tiny $_{lb}$}‚त‚त्र चाय‚न्दोष इत्यागूर्याह । सोतिश‚यो व्य‚व‚स्थाल‚क्ष‚ण‚स्त‚स्यातिश‚य‚व‚तोऽव‚स्थातु‚{\tiny $_{lb}$}‚ \leavevmode\ledsidenote{\textenglish{46/s}} रात्म‚भूतोऽन‚न्व‚य इत्येकान्तेन निव‚र्त्त‚मानः । व्या ‚{\tiny $_{1}$}‚ प‚क‚स्व‚भावात्प्र‚व‚र्त्त‚मानोऽस‚न्नेव ‚{\tiny $_{lb}$}‚क‚थ‚न्न स्व‚भाव‚नानात्वं । सुख‚दुःख‚योरिवाक‚र्ष‚ति । अन्वाक‚र्ष‚त्येवेत्य‚र्थः ।प्र‚योगो\edtext{}{\lemma{योगो}\Bfootnote{‚{\tiny $_{lb}$}‚? प्र‚योगः}}पुन‚र्यो य‚स्या \quotelemma{त्म‚भूतः} \cite[4b3]{vn-msN} स त‚न्निवृ ‚{\tiny $_{2}$}‚ त्तावेकान्तेन निव‚र्त्त‚ते प्र‚वृ‚{\tiny $_{lb}$}‚त्तौ चास‚न्नेव प्र‚व‚र्त्त‚ते य‚था त‚स्यैवातिश‚य‚स्यात्मा । आत्म‚भूत‚श्चातिश‚य‚स्यातिश‚य‚{\tiny $_{lb}$}‚वानिति स्व‚भाव‚हेतुः । त‚त‚श्च त‚योर‚व‚स्थ ‚{\tiny $_{3}$}‚ योर‚व‚स्थातुर्न्नानात्वं प‚र‚स्प‚र‚विरोधि‚{\tiny $_{lb}$}‚ध‚र्माध्यासित‚त्वात् सुख‚दुःख‚व‚दिति स्व‚भाव‚हेतुरेव । नैवासाव‚तिश‚योऽन‚न्व‚यः ‚{\tiny $_{lb}$}‚प्र‚व‚र्त्त‚ते निव ‚{\tiny $_{4}$}‚ र्त्त‚ते वाऽतः पूर्व‚स्मिन्प्र‚माणे साध्य‚विक‚ल‚त्व‚न्दृष्टान्त‚स्य । उत्त‚र‚त्र त्व‚{\tiny $_{lb}$}‚सिद्धिर्हेतोरिति चेदाह । \quotelemma{सान्व‚य‚त्वेचा} तिश‚य‚स्य निवृत्तिप्र‚वृत्योरं ‚{\tiny $_{5}$}‚ गीक्रिय‚माणे ‚{\tiny $_{lb}$}‚का क‚स्य \quotelemma{निवृत्तिः प्र‚वृत्तिर्वेति} \cite[4b3]{vn-msN} । नैव काचित्क‚स्य‚चिन्निवृत्तिः प्र‚वृतिर्वा । ‚{\tiny $_{lb}$}‚स‚र्व्व‚स्य स‚र्व्व‚दा स‚त्त्वात् । त‚था च स‚र्व्वं स‚र्व्व‚त्र स‚मुप‚यु ‚{\tiny $_{6}$}‚जे\edtext{}{\lemma{जे}\Bfootnote{? ज्ये}}तेत्यादिना ‚{\tiny $_{lb}$}‚पुरोनुक्रान्तो दोषोनुप‚युज्य‚त इत्य‚भिप्रायः । उप‚च‚य‚माह । य‚दि च क‚स्य‚चित् ‚{\tiny $_{lb}$}‚स्व‚भाव‚स्यातिश‚याख्य‚स्य प्र‚वृत्तिर्निवृत्तिर्वेति स्व‚य‚म‚भ्य‚नुज्ञाय‚ते ‚{\tiny $_{7}$}‚ त्व‚या । एकाति‚{\tiny $_{lb}$}‚श‚य‚निवृत्याऽप‚रातिश‚योत्प‚त्या व्य‚व‚हार‚भेदोप‚ग‚मादित्य‚विधानात् । त‚देत‚देव ‚{\tiny $_{lb}$}‚प‚र‚स्त‚थाग‚त‚व‚चोऽभ्यासोप‚जाताव‚दात‚म‚ति ‚{\tiny $_{8}$}‚ र्ब्रुवाणः । नानुम‚न्य‚ते भ‚द्र‚मुखेण\edtext{}{\lemma{मुखेण}\Bfootnote{‚{\tiny $_{lb}$}‚? मुखेन}}भ‚वेदेवं य‚दि य‚था म‚या प्र‚वृत्तिनिवृत्ती अभ्य‚नुज्ञायेते त‚था तेनापि । ‚{\tiny $_{lb}$}‚याव‚तास्य निर‚न्व‚योप‚ज‚न‚न‚विनाशोप‚ग‚मो म‚म ‚{\tiny $_{9}$}‚ \leavevmode\ledsidenote{\textenglish{33b/msK}} त्वाविर्भाव‚तिरोभाव‚मात्र‚न्त‚त्क‚थ‚मिवा‚{\tiny $_{lb}$}‚नुम‚न्य‚त इति क‚दाचिद् ब्रूयात्प‚र इति त‚न्म‚त‚माश‚ङ्क‚ते । \quotelemma{त‚स्ये} \cite[4b4]{vn-msN} त्यादिना । ‚{\tiny $_{1}$}‚ ‚{\tiny $_{lb}$}‚स‚दैव भ‚व‚द्भिः शून्य‚हृद‚यैर‚य‚म‚न्व‚यो घोष्य‚ते । त‚त्र व‚क्त‚व्य‚ङ्कोय‚म‚न्व‚यो नाम ‚{\tiny $_{lb}$}‚भाव‚स्य ज‚न्म‚विनाश‚योरिति स‚त्त्यं एषा । प‚रः प्राह [।] \quotelemma{किम‚त्राभिधानीयं} याव‚{\tiny $_{lb}$}‚ता श‚क्तिर‚न्व‚यो भाव‚स्य ज‚न्म‚विनाश‚योरिति व ‚{\tiny $_{2}$}‚ र्त्त‚ते । क‚थ‚म्पुनः सान्व‚य इत्याह । ‚{\tiny $_{lb}$}‚य‚तोस्त्येव \quotelemma{प्राग‚पि ज‚न्म‚नो निरोधाद‚प्यूर्ध्वं} \cite[4b5]{vn-msN} सा श‚क्तिर‚व‚स्थातृल‚क्ष‚णा ‚{\tiny $_{lb}$}‚येनैत‚देव‚न्तेनाय‚म्भावो नापूर्वः स‚न् स‚र्व‚था जाय‚ते ‚{\tiny $_{3}$}‚ अपि तु श‚क्तिरूपेण पूर्व्वं व्य‚व‚स्थित ‚{\tiny $_{lb}$}‚एव केव‚ल‚माविर्भ‚व‚तीति स‚र्व्व‚थाग्र‚ह‚णेन ज्ञाप‚य‚ति । त‚था न पूर्व्वो विन‚श्य‚त्येकान्ते‚{\tiny $_{lb}$}‚नापि तु तिरोभ‚व‚ति । ‚{\tiny $_{4}$}‚ अस‚तो नास्त्युद‚यः स‚त‚श्च नास्ति विनाश इति याव‚त् । ‚{\tiny $_{lb}$}‚ \leavevmode\ledsidenote{\textenglish{47/s}} आचार्य आह । \quotelemma{य‚दि सा श‚क्तिः स‚र्व्व‚दा} \cite[4b6]{vn-msN} तिरोभावाविर्भाव‚कालेऽ \quotelemma{न‚तिश‚याति‚{\tiny $_{lb}$}‚श‚य ‚{\tiny $_{5}$}‚ र‚हिता एक‚रूपे} ति याव‚त् । त‚दा \quotelemma{किमिदानीं} अतिश‚य‚व‚द्विद्य‚ते । य‚त[ः] ‚{\tiny $_{lb}$}‚कुतोयं व्य‚व‚हार‚विभागः क्षीर‚न्द‚धित‚क्र‚मित्यादि । \quotelemma{साङ्ख्य} आह । ‚{\tiny $_{6}$}‚ अव‚स्था \quotelemma{अति‚{\tiny $_{lb}$}‚श‚य‚व‚त्य} \cite[4b6]{vn-msN} इति । \quotelemma{ता} \cite[4b6]{vn-msN} इत्याद्या \quotelemma{चार्यः} । विक‚ल्प‚द्व‚य‚ञ्च प्र‚कारान्त‚रा‚{\tiny $_{lb}$}‚स‚म्भ‚वात्कृतं । न वाह‚रीक‚वादो युज्य‚ते त‚त्स्वान्य‚त्व‚योः प‚र‚स्प‚र‚प‚रिहार‚स्थि ‚{\tiny $_{7}$}‚ ति‚{\tiny $_{lb}$}‚ल‚क्ष‚ण‚त‚या तृतीय‚राशिव्य‚तिरेच‚क‚त्वात् । एक‚त्वे को दोष इति चेदाह [।] \quotelemma{एक‚श्चे‚{\tiny $_{lb}$}‚त्त‚दा क‚थ‚मिद‚मेक‚त्राविभ‚क्ता} त्म‚न्य‚विभ‚क्त‚स्व‚रूपे \quotelemma{योक्ष्य‚ते} \cite[4b7]{vn-msN} । व्य‚पेक्ष‚या ‚{\tiny $_{lb}$}‚भ ‚{\tiny $_{8}$}‚ वेद‚पीत्याह । \quotelemma{निष्प‚र्यायं} किम्पुन‚स्त‚त्प‚र‚स्प‚र‚व्याह‚त इत्याह \quotelemma{ज‚न्म अव‚स्थानाम‚{\tiny $_{lb}$}‚ज‚न्म‚श‚क्तेः} । त‚थार्थ‚क्रियायामुप‚योगोऽव‚स्थानां श‚क्त‚स्त्व‚नुप‚योग \cite[4b7]{vn-msN} इ ‚{\tiny $_{9}$}‚ \leavevmode\ledsidenote{\textenglish{34a/msK}} ति । ‚{\tiny $_{lb}$}‚प्र‚योगाः पुनः [।] श‚क्तेर‚पि ज‚न्मास्ति । अव‚स्थाभ्योऽव्य‚तिरेकात् । अव‚स्था‚{\tiny $_{lb}$}‚स्व‚रूप‚व‚त् । अव‚स्थानाम्वा न ज‚न्म श‚क्तेर‚व्य‚तिरेकात् । श‚क्तिस्व‚रूप‚व‚त् । ‚{\tiny $_{lb}$}‚स्व‚भाव‚हेतुविरुद्ध‚व्याप्तो ‚{\tiny $_{1}$}‚ प‚ल‚ब्धि[ः] । एतेनैव प्र‚कारेणार्थ‚क्रियोप‚योगानुप‚योग‚निवृत्य‚{\tiny $_{lb}$}‚निवृत्यादिषु स्व‚भाव‚हेतुविरुद्ध‚व्याप्तोप‚ल‚ब्ध‚यो योज्याः । आदिग्र‚ह‚णेन प‚त‚नाप‚त‚न‚{\tiny $_{lb}$}‚पोर‚पि प‚रिग्र‚हः ।
	{\color{gray}{\rmlatinfont\textsuperscript{§~\theparCount}}}
	\pend% ending standard par
      ‚{\tiny $_{lb}$}‚

	  
	  \pstart \leavevmode% starting standard par
	पु ‚{\tiny $_{2}$}‚ न‚र‚पि \quotelemma{साङ्खीय} म्म‚त‚माश‚ङ्क‚ते । \quotelemma{अस्ती} त्या \cite[4b8]{vn-msN} दिना । केन‚चित्प‚र्या‚{\tiny $_{lb}$}‚येण अव‚स्थाश‚क्त्योर‚न‚न्य‚त्व‚म्प‚र‚मार्थ‚त‚स्तु भेद एव तेन ज‚न्मादीनाम‚विरोध इति । ‚{\tiny $_{lb}$}‚नून‚म्भ‚व‚तः स्व‚प‚क्ष‚र‚क्ष‚णा ‚{\tiny $_{3}$}‚ कुल‚बुद्धेरात्मापि विस्मृतः । इत्याकूत‚वानाह । \quotelemma{विस्म‚र‚ण‚{\tiny $_{lb}$}‚शील} \cite[4b8]{vn-msN} इत्यादि । य‚तोऽन‚न्य‚त्व‚प‚क्षेऽय‚न्दोषोस्माभिरुक्तोऽन्य‚त्व‚प‚क्षेत्व‚न्य ‚{\tiny $_{lb}$}‚एव भ‚विष्य‚ति । कः पुन‚र‚साव‚न्य ‚{\tiny $_{4}$}‚ इति त‚मेव‚द‚र्श‚यितुमुप‚क्र‚म‚ते । अथाप्य‚न‚योः ‚{\tiny $_{lb}$}‚श‚क्त्य‚व‚स्थ‚योर्विभागोऽन्य‚त्व‚न्त‚दा न क‚श्चिद्विरोधः । केव‚लं सान्व‚यो भाव‚स्य ज‚न्म ‚{\tiny $_{lb}$}‚विनासा\edtext{}{\lemma{विनासा}\Bfootnote{? विनाशा}}दिति न स्यात् । किं ‚{\tiny $_{5}$}‚ कार‚णं । य‚स्मात् य‚स्यान्व‚यः श‚क्ति‚{\tiny $_{lb}$}‚त्वेनाभिम‚त‚स्य न त‚स्य ज‚न्म‚विनाशौ नित्य‚मेक‚स्मिन्नेव स्व‚भावे व्य‚व‚स्थानात् । य‚स्य ‚{\tiny $_{lb}$}‚ \leavevmode\ledsidenote{\textenglish{48/s}} वा ता उत्त्पाद‚विनाशाव‚व‚स्थात्वेनाभी ‚{\tiny $_{6}$}‚ ष्ट‚स्य न त‚स्यान्व‚यः । अप‚राप‚राव‚स्थो‚{\tiny $_{lb}$}‚द‚यास्त‚म‚येनाव‚स्थित‚रूपाभावात् । त‚योः श‚क्तिव्य‚क्त‚योर‚भेदाद‚दोष इति \quotelemma{कापिलः । ‚{\tiny $_{lb}$}‚अनुत्त‚र} \cite[4b10]{vn-msN} \quotelemma{मित्याद्याचार्यः} । किम‚त्रायुज्य‚मा ‚{\tiny $_{7}$}‚ न‚कं येनैवं व‚द‚सीत्याह । \quotelemma{अभे‚{\tiny $_{lb}$}‚दो हि नामैक्य‚मुच्य‚ते} \cite[4b10]{vn-msN} । तौ श‚क्तिव्य‚क्तिभेदा \quotelemma{वित्य‚य‚ञ्च} भेदाधिष्ठानो‚{\tiny $_{lb}$}‚ \quotelemma{न्य‚त्व‚निब‚न्ध‚नो व्य‚व‚हारो भाविक} इति क‚ल्प‚नाविर‚चित‚स्याप्र‚तिक्षे ‚{\tiny $_{8}$}‚ पात् । किञ्च ‚{\tiny $_{lb}$}‚निवृत्तिप्रादुर्भाव‚योः स‚तोर‚निवृत्तिप्रादुर्भावौ त‚था स्थितौ स‚त्याम‚स्थितिः । आदि‚{\tiny $_{lb}$}‚ग्र‚ह‚णाद् ग‚ताव‚ग‚तिरित्यादि योज्यं । एत‚द् भेद‚ल‚क्ष‚ण‚ङ्क‚थं योज्य‚ते भ ‚{\tiny $_{9}$}‚ \leavevmode\ledsidenote{\textenglish{34b/msK}} व‚ता । त‚था‚{\tiny $_{lb}$}‚ह्य‚व‚स्थानिवृत्तिप्रादुर्भावाभ्याम‚निवृत्तिप्रादुर्भाव‚व‚त्याः श‚क्तेर‚भेदो नेष्य‚ते त्व‚या । ‚{\tiny $_{lb}$}‚त‚था श‚क्तेर‚व‚स्थानेपि नाव‚स्थानाम‚व‚स्थानं । न च श‚क्तेस्तासाम‚न्य‚त्व‚मिष्टं ‚{\tiny $_{1}$}‚ । ‚{\tiny $_{lb}$}‚त‚स्मादेवं रूपं नानात्व‚मित्याह । एष हि \quotelemma{निवृत्तिप्रादुर्भाव‚योर‚निवृत्तिप्रादुर्भाव} \cite[5a1]{vn-msN} ‚{\tiny $_{lb}$}‚इत्यादिभेदः । त‚था हि य‚न्निवृत्यादिना न य‚स्य निवृत्याद‚य‚स्त‚त्त‚स्माद् भिन्नं य‚था ‚{\tiny $_{lb}$}‚ता ‚{\tiny $_{2}$}‚ ल‚त‚रुस्त‚मालादित्य‚तिप्र‚तीत‚मेत‚त् । एत‚द्विर‚ह‚श्चाभेद इति य‚न्निवृत्या य‚स्य ‚{\tiny $_{lb}$}‚निवृत्तिरित्यादि । न‚नु च भूत‚भौतिक‚चित्त‚चैत्तादीनाम्प्र‚तिनिय‚त‚स‚होत्पाद‚निरो‚{\tiny $_{lb}$}‚ध ‚{\tiny $_{3}$}‚ स्थितीनामेत‚द्विद्य‚ते । न च तेषाम‚भेद‚स्त‚त् क‚थ‚मुक्त‚मेत‚द्विर‚ह‚श्चाभेद इति चेत् । ‚{\tiny $_{lb}$}‚न तेषाम्भिन्नोत्पादादिम‚त्वात् । य‚थाक्र‚म‚मुदाह‚र‚ण‚द्व‚य‚माह । \quotelemma{य‚थे} ‚{\tiny $_{4}$}‚ त्यादि । अन्य‚{\tiny $_{lb}$}‚थे \cite[5a1]{vn-msN} ति । य‚द्य‚न‚न्त‚रोक्त‚म्भेदाभेद‚ल‚क्ष‚ण‚न्नाश्रीय‚ते त‚दा भेद‚योर्ल‚क्ष‚णा‚{\tiny $_{lb}$}‚भावात्कार‚णाद् भेदाभेद‚योर‚व्य‚व‚स्था स्यात् । स‚र्व्व‚त्रेति सुखादीनाम्प‚र‚स्प‚रं ‚{\tiny $_{5}$}‚ चैत‚न्या‚{\tiny $_{lb}$}‚नाञ्च । सुखादिभ्य‚श्चैत‚न्यानां अभेदः । सुखादीनाम्प्र‚त्येक‚म्भेदो न भ‚वेदिति ‚{\tiny $_{lb}$}‚याव‚त् । त‚दात्म‚नीत्यादिना प‚रः स्व‚स‚म‚य‚प्र‚तीत‚म्भेदाभेद‚योर्ल‚क्ष‚ण ‚{\tiny $_{6}$}‚ माह । \quotelemma{तेना‚{\tiny $_{lb}$}‚\leavevmode\ledsidenote{\textenglish{49/s}} विरोध} \cite[5a3]{vn-msN} इति ज‚न्माज‚न्मादीनां । अन‚न्त‚रोक्त‚स्य वा । न वै मृदात्म‚नीत्या‚{\tiny $_{lb}$}‚दिना मृप्तिण्ड‚घ‚ट‚योराधाराधेय‚भावं प्र‚तिक्षिप‚ति’ । किन्त‚र्हि मृदात्मैव क‚श्चित् ‚{\tiny $_{7}$}‚ ‚{\tiny $_{lb}$}‚विशिष्ट‚ग्रीवादिस‚न्निवेशाव‚च्छिन्नो घ‚ट इत्य‚भीधीय‚ते [।] न‚न्वेक‚मेव मृद्द्र‚व्यं स‚र्व्व‚त्र ‚{\tiny $_{lb}$}‚त‚त्क‚थ‚मिद‚म‚भिहित‚मित्य‚त आह । \quotelemma{न‚हि एक‚स्त्रैलोक्य‚मृदात्मे} \cite[5a3]{vn-msN} ति । कुतः ‚{\tiny $_{lb}$}‚प्र‚तिविज्ञ ‚{\tiny $_{8}$}‚ प्तिप्र‚तिभास‚भेद‚न्द्र‚व्य‚स्व‚भाव‚भेदादिति स‚म्ब‚न्धः । अन्योन्य‚भिन्नानामेव ‚{\tiny $_{lb}$}‚द्र‚व्याणाम्विज्ञाने प्र‚तिभास‚नादित्य‚र्थः । त‚था प्र‚त्य‚व‚स्थाभेद‚भिन्नाव‚स्थ‚त्वात् । ‚{\tiny $_{lb}$}‚प्र‚त्य‚र्थ ‚{\tiny $_{9}$}‚ \leavevmode\ledsidenote{\textenglish{35a/msK}} क्रियाभेदं चाश्रित्य द्र‚व्य‚स्व‚भाव‚भेदात् । प‚र‚स्प‚रास‚म्भ‚विकार्य‚कार‚णा- ‚{\tiny $_{lb}$}‚दिति याव‚त् । प‚र‚स्यापि स‚त्व‚र‚ज‚स्त‚म‚श्चैत‚न्येषु भेदाभ्युप‚ग‚म इद‚मेव कार‚णं ‚{\tiny $_{lb}$}‚युक्त‚मिति क‚थ‚य ‚{\tiny $_{1}$}‚ [न्] नाह । \quotelemma{एवं हीति} \cite[5a4]{vn-msN} । य‚दि प्र‚तिविज्ञ‚प्तिप्र‚तिभास‚{\tiny $_{lb}$}‚भेदादिना भेद इष्य‚ते । चैत‚न्येषु चेति ब‚हुव‚च‚नं ब‚ह‚वः पुमांस इति सिद्धान्तात् । य‚द्येव‚{\tiny $_{lb}$}‚मिति प्र‚तिविज्ञ‚प्ति प्र ‚{\tiny $_{2}$}‚ तिभास‚भेदादिना । पुन‚र‚प्याह । \quotelemma{स‚त्य‚प्येत‚स्मिन्} प्र‚तिविज्ञ‚प्ति‚{\tiny $_{lb}$}‚प्र‚तिभास‚भेदादौ क‚स्य‚चि \quotelemma{दात्म‚न} \cite[5a4]{vn-msN} इति श‚क्तेर‚नुग‚मादैक्य‚म‚व‚स्थानामिति । ‚{\tiny $_{lb}$}‚ \quotelemma{आचार्य} आह । \quotelemma{य‚द्ये ‚{\tiny $_{3}$}‚ वं सुखादिष्व‚प्य‚य‚मेवाभेद‚प्र‚स‚ङ्} ग‚श्चैत‚न्येषु च । सुखादिष्व‚पि ‚{\tiny $_{lb}$}‚हि गुण‚त्वाद् भोक्तृत्व‚क‚र्त्तृत्वादीनाम‚नुग‚माच्चैत‚न्येषु च भोक्तृत्वाक‚र्त्तृत्वागुण‚{\tiny $_{lb}$}‚त्वादी ‚{\tiny $_{4}$}‚ नान्त‚था सुखादि चैत‚न्येषु स‚त्व‚ज्ञेय‚त्वादीनाम‚न्व‚यादित्य‚भिप्रायः । प्र‚योगो\edtext{}{\lemma{योगो}\Bfootnote{? गः}} ‚{\tiny $_{lb}$}‚ पुन‚र‚भिन्नाः पुरुष‚सुखाद‚यः प‚र‚स्प‚र‚म‚न्व‚यान्व‚य‚भाक्त्वात् । श ‚{\tiny $_{5}$}‚ क्तिव्य‚क्तिव‚त् । श‚क्ति‚{\tiny $_{lb}$}‚व्य‚क्ती वा भिन्नेऽन्व‚योन‚न्व‚य‚भाक्त्वादेव । सुखादिचैत‚न्य‚व‚दिति स्व‚भाव‚हेतु[ः] । ‚{\tiny $_{lb}$}‚अथापि स्याद्य‚त्र स‚र्व्वात्म‚नैवान्व‚य‚स्त‚त्राभेदो न ‚{\tiny $_{6}$}‚ तु य‚त्र केन‚चिद्रूपेण । घ‚टादिषु च ‚{\tiny $_{lb}$}‚स‚र्व्वात्म‚नान्व‚य‚स्त‚तोय‚म‚दोष इत्य‚त आह । \quotelemma{न च घ‚टादिष्व‚पि स‚र्व्वात्म‚नान्व‚यो ‚{\tiny $_{lb}$}‚ \cite[5a5]{vn-msN} पि तु केन‚चिद्रूपेणेति} न केव‚लं सुखादि ‚{\tiny $_{7}$}‚ ष्वित्य‚पि श‚ब्दः । कुतोऽवैश्व‚रूप्य‚{\tiny $_{lb}$}‚स‚होत्पादादिप्र‚स‚ङ्गात् । त‚थाहि स‚र्व्वासाम‚व‚स्थानां स‚र्व्व‚प्र‚कारेणान्व‚ये स‚त्यैक्य‚{\tiny $_{lb}$}‚म्प्राप्नोति । त‚त‚श्च विशिष्ट‚रूप‚र‚स‚ग‚न्ध‚श ‚{\tiny $_{8}$}‚ वीर्य‚विपाकाभावात् । ‚{\tiny $_{lb}$}‚वैचित्र्य‚न्न भ‚वेत् । एव‚ञ्च प‚ञ्च‚भूताभाव‚प्र‚स‚ङ्गोऽध्य‚क्षादिवाधाप्र‚सं ‚{\tiny $_{9}$}‚ \leavevmode\ledsidenote{\textenglish{35b/msK}} ग‚श्चेति ‚{\tiny $_{lb}$}‚भावः । स‚होत्प‚त्तिश्च स‚र्व्वासाम‚व‚स्थानाम्प्र‚स‚ज्य‚ते । आदिश‚ब्देन ह्य‚निरो‚{\tiny $_{lb}$}‚धार्थ‚क्रियाव्यापार‚विकाराद‚य ‚{\tiny $_{1}$}‚ उपादीय‚न्ते । प्र‚योगाः पुन‚र्य‚द्विशिष्ट‚रूप‚र‚स‚ग‚न्ध‚{\tiny $_{lb}$}‚ \leavevmode\ledsidenote{\textenglish{50/s}} श‚ब्दादिभिर‚नेक‚प्र‚कारं न भ‚व‚ति । न त‚स्य वैश्व‚रूप्य‚म‚स्ति । य‚थैक‚स्य सुखाद्या‚{\tiny $_{lb}$}‚त्म‚नः । त‚था स‚ति \edtext{}{\lemma{ति}\Bfootnote{?}} म‚ताना ‚{\tiny $_{2}$}‚ म‚प्य‚व‚स्थानाम‚न‚न्त‚रोक्तो ध‚र्मो नास्ति न चासिद्धो ‚{\tiny $_{lb}$}‚हेतुर्य‚तो य‚द्य‚स्मान्न व्य‚तिरिच्य‚ते न त‚द्विशिष्ट‚रूपादिभिर‚नेक‚प्र‚कारं य‚था त‚स्यै‚{\tiny $_{lb}$}‚वात्मा । न व्य‚तिरिच्य‚न्ते चा ‚{\tiny $_{3}$}‚ व‚स्था अभीष्टा इति व्याप‚क‚विरुद्धोप‚ल‚ब्धी । त‚था ‚{\tiny $_{lb}$}‚य‚द्य‚स्माद‚पृथ‚ग्भूतं त‚त्त‚दुत्पादादिभिरुत्पादादिम‚त् । य‚था त‚स्यैव स्व‚रूपं । अपृथ‚ग्भू‚{\tiny $_{lb}$}‚ताश्चा ‚{\tiny $_{4}$}‚ भिम‚ता अव‚स्थास्ताभ्योऽन्य‚स्या इति स्व‚भाव‚हेतुः । अन्य‚था घ‚टोय‚मित्य‚न‚{\tiny $_{lb}$}‚न्य‚त्व‚मेवायुक्तं । नामान्त‚र‚म्वा अर्थ‚भेद‚म‚भ्युप‚ग‚म्य त‚थाभिधाना ‚{\tiny $_{5}$}‚ त् । उप‚च‚य‚माह । ‚{\tiny $_{lb}$}‚ \quotelemma{न च घ‚टं मृदात्मान‚ञ्च क‚श्चिद \cite[5a5]{vn-msN} त्य‚र्थ} मुन्मीलित‚लोच‚नोप्य‚यं घ‚टोयं च ‚{\tiny $_{lb}$}‚मृदात्मेति विवेकेनोप‚ल‚क्ष‚य‚ति । येनैवं स्यादिद‚मि ‚{\tiny $_{6}$}‚ ह प्रादुर्भूत‚मिति त‚द‚नेनाभेद‚ल‚क्ष‚ण‚{\tiny $_{lb}$}‚म‚त्य‚न्तास‚म्ब‚द्ध‚मेवेत्याह । न‚नु च पिण्ड‚रूपात्मृदात्म‚नो घ‚ट‚स्य विवेकेनोप‚ल‚क्ष‚ण‚म‚स्त्येव ‚{\tiny $_{lb}$}‚त‚क्तिमेव‚मुक्त‚मिति चे ‚{\tiny $_{7}$}‚ त् । स‚त्य‚म‚स्ति । न तु घ‚टाद् भिन्न‚न्तं प‚रोभिम‚न्य‚त इत्य‚भि‚{\tiny $_{lb}$}‚प्रायाद‚दोषः । य‚दि नाम भेदेनानुप‚ल‚क्ष‚ण‚न्त‚योस्त‚थापि क‚स्मादेवं न स्यादिति चेदाह ॥ ‚{\tiny $_{lb}$}‚ \quotelemma{न‚ह्य‚धि ‚{\tiny $_{8}$}‚ ष्ठानाधिष्ठानिनोराधाराधेय‚योः कुण्डेव‚द‚र‚योर्विवेकेनानुन‚प‚ल‚क्ष‚णे स‚त्ये‚{\tiny $_{lb}$}‚व‚म्भ‚व‚तीद‚मिह प्रादूर्भूत‚मिति} \cite[4a6]{vn-msN} । त‚द‚नेन घ‚ट‚मृदात्म‚नोराधाराधेय‚भावो ‚{\tiny $_{9}$}‚ ‚{\tiny $_{lb}$}‚ \leavevmode\ledsidenote{\textenglish{36a/msK}} नास्ति विवेकेनानुप‚ल‚क्ष‚णात्स‚त्वादित‚त्स्व‚भाव‚योरेवेति व्याप‚कानुप‚ल‚ब्धिं म‚न्य‚ते । ‚{\tiny $_{lb}$}‚अधुना य‚द्य‚स्मिन्प्रादुर्भ‚व‚ति त‚त्त‚तोऽभिन्न‚मित्य‚स्याभेद‚ल‚क्ष‚ण‚स्याव्या ‚{\tiny $_{1}$}‚ पितासा\edtext{}{\lemma{पितासा}\Bfootnote{? शा}} ‚{\tiny $_{lb}$}‚ चिख्यासुराह । न च श‚क्त्यात्म‚नि प्रादुर्भाव‚स्त‚स्या नित्य‚म‚व‚स्थानाभ्युप‚ग‚मात् ॥ ‚{\tiny $_{lb}$}‚अन्य‚थाव‚स्थैव सा स्यात् । त‚था च त‚स्याः स्वात्म‚नःस‚कासा\edtext{}{\lemma{कासा}\Bfootnote{? स‚काशा}}द‚भ्यु ‚{\tiny $_{2}$}‚ पेतो‚{\tiny $_{lb}$}‚ऽभेदो न स्यात् । अभेद‚ल‚क्ष‚णाभावात् उप‚ल‚क्ष‚ण‚ञ्चैत‚द्व्य‚क्तौ सुखादिषु पुरुषेषु च ‚{\tiny $_{lb}$}‚तुल्य‚दोष‚त्वात् । अन्ये तु स्व‚द‚र्श‚नाप‚राध‚म‚लीम‚स‚धिय[ः] ‚{\tiny $_{3}$}‚ केचित् \quotelemma{सांख्या} एव‚माहुः । ‚{\tiny $_{lb}$}‚यो य‚स्य प‚रिणाम‚स्स त‚स्माद‚भिन्नः । त‚द्य‚था हेम्नः कुण्ड‚लाद्य‚व‚स्थाविशेष इति ‚{\tiny $_{lb}$}‚तेप्य‚नेनैव पूर्व्व‚स्याभेद‚ल‚क्ष‚ण ‚{\tiny $_{4}$}‚ स्याव्यापिताप्र‚द‚र्श‚नेनाप‚ह‚स्तिता इति चेत‚स्या‚{\tiny $_{lb}$}‚रोप्याह । \quotelemma{एतेनैवे} \cite[5a6]{vn-msN} त्यादि । युष्म‚द्द‚र्श‚न‚प‚रिणामोपि न युक्त इत्य‚भिप्राय‚वा‚{\tiny $_{lb}$}‚न‚प‚क्षेप ‚{\tiny $_{5}$}‚ ङ्क‚रोति । \quotelemma{किञ्चेद} \cite[5a7]{vn-msN} मित्यादिना । प‚रेणापि किम‚त्र व‚क्त‚व्यं याव‚ता ‚{\tiny $_{lb}$}‚ \leavevmode\ledsidenote{\textenglish{51/s}} भ‚ग‚व‚ता \quotelemma{क‚पिलेन} स्प‚ष्ट‚मिद‚मुक्त‚मित्य‚भिस‚न्धायाह । अव‚स्थित‚स्य ‚{\tiny $_{6}$}‚ द्र‚व्य‚स्य य‚था ‚{\tiny $_{lb}$}‚काञ्च‚न‚स्य ध‚र्मान्त‚र‚स्य केयूर‚स्य निवृत्तिः । ध‚र्मान्त‚र‚स्य च कुण्ड‚लादेः प्रादुर्भावः ‚{\tiny $_{lb}$}‚प‚रिणाम इति । \quotelemma{आचार्य‚स्त‚स्यैव} ताव‚दिद‚मीदृशं ‚{\tiny $_{7}$}‚ प्र‚ज्ञास्ख‚लित‚ङ्क‚थं वृत्त‚मिति स‚वि‚{\tiny $_{lb}$}‚स्म‚यानुकंप‚न्न‚श्चेतः । त‚द‚प‚रेप्य‚नुव‚द‚न्तीति निर्द‚याक्रान्त‚भुव‚नं दिग्व्याप‚क‚न्त‚मः । कः ‚{\tiny $_{lb}$}‚प्राणिनो हितेच्छा विपुल‚त्व ‚{\tiny $_{8}$}‚ स्याप‚राध इति म‚न्य‚मान[ः] प्राह ।
	{\color{gray}{\rmlatinfont\textsuperscript{§~\theparCount}}}
	\pend% ending standard par
      ‚{\tiny $_{lb}$}‚

	  
	  \pstart \leavevmode% starting standard par
	\hphantom{.}\quotelemma{न‚नु य‚दि नाम तेनैव‚मुक्तं} । भ‚व‚द्भिस्तु निभाल‚नीय‚मेत‚त् य‚त्त‚द्ध‚र्मान्त‚रं ‚{\tiny $_{lb}$}‚कुण्डादिकं निव‚र्त्त‚ते प्रादुर्भ‚व‚ति च किं त ‚{\tiny $_{9}$}‚ \leavevmode\ledsidenote{\textenglish{36b/msK}} देवाव‚स्थितं काञ्च‚न‚द्र‚व्यं स्यात्त‚तोर्था- ‚{\tiny $_{lb}$}‚न्त‚र‚म्वेति । क‚स्माद्विक‚ल्प‚द्व‚य‚मेव कृत‚मितिचेदाह [।] \quotelemma{अन्य‚विक‚ल्पाभावात्} ‚{\tiny $_{lb}$}‚ \cite[5a8]{vn-msN} । \quotelemma{निर्ग्र‚न्थ‚वाद} स्यायोगादित्य‚भिस‚न्धिः । य‚द्याद्यो ‚{\tiny $_{1}$}‚ विक‚ल्प‚स्त‚दा को‚{\tiny $_{lb}$}‚दोष इति चेदाह । य‚दि त‚द्ध‚र्मान्त‚र‚न्त‚देवाव‚स्थितं द्र‚व्यं \cite[5a8]{vn-msN} । त‚दा त‚स्याव ‚{\tiny $_{lb}$}‚स्थानान्न निवृत्तिप्रादुर्भावावाविर्भाव‚तिरोभाव‚ल‚क्ष‚णाविति त ‚{\tiny $_{2}$}‚ स्मात्क‚स्य ‚{\tiny $_{lb}$}‚ताविति व‚क्त‚व्य‚म्भ‚व‚द्भिः । \quotelemma{प्र‚योगो\edtext{}{\lemma{योगो}\Bfootnote{? प्र‚योगः}}} पुनः य‚स्याव‚स्थानं न त‚स्य ‚{\tiny $_{lb}$}‚निवृत्तिप्रादुर्भावौ । य‚थाव‚स्थातुर्द्र‚व्य‚स्य । त‚था चाव‚स्थान‚न्त‚स्य ध‚र्मा ‚{\tiny $_{3}$}‚ न्त‚र‚{\tiny $_{lb}$}‚स्येति व्याप‚क‚विरुद्धोप‚ल‚ब्धिः । न चासिद्धो हेतुर्य‚तो य‚द‚व‚स्थातुर‚न‚न्य‚त्त‚स्या‚{\tiny $_{lb}$}‚व‚स्थानं य‚था त‚त्स्व‚रूप‚स्यैव । अन‚न्य‚च्चैत‚द्ध‚र्मान्त‚रं त‚स्मा ‚{\tiny $_{4}$}‚ दिति स्व‚भाव‚हेतुः ‚{\tiny $_{lb}$}‚किञ्च य‚द्य‚व‚स्थित‚मेव द्र‚व्यं त‚द्ध‚र्मान्त‚रं त‚दाव‚स्थित‚स्य द्र‚व्य‚स्य ध‚र्मान्त‚र‚मिति । ‚{\tiny $_{lb}$}‚व‚च‚नं सिध्य‚ति ॥ किङ्कार‚ण‚मित्याह । \quotelemma{न ‚{\tiny $_{5}$}‚ हि त‚देव त‚स्य ध‚र्मान्त‚र‚म्भ‚व‚ती} \cite[5a9]{vn-msN} ‚{\tiny $_{lb}$}‚ति भ‚व‚त्येव त‚देव त‚स्य ध‚र्मान्त‚रं य‚था कृत‚क‚त्वं श‚ब्द‚स्याव्य‚तिरिक्त‚म‚पि त‚स्मादिति ‚{\tiny $_{lb}$}‚चेदाह । \quotelemma{अन‚पाश्रित‚व्य ‚{\tiny $_{6}$}‚ पेक्षाभेदं} \cite[5a9]{vn-msN} । एत‚दुक्त‚म्भ‚व‚ति । अत्र हि स एव ‚{\tiny $_{lb}$}‚स‚ब्दो\edtext{}{\lemma{ब्दो}\Bfootnote{? श‚ब्दो}}ऽकृत‚कादिभ्योव्यावृत्त‚त्त्वात् त‚द्व्यावृत्त्य‚पेक्ष‚या त‚न्मात्र‚जिज्ञासायां ‚{\tiny $_{lb}$}‚प्र‚तिक्षिप्त‚भेदान्त‚रेण श‚ब्देन ध‚र्म‚त्वे ‚{\tiny $_{7}$}‚ न व्य‚प‚दिस्य‚ते\edtext{}{\lemma{ते}\Bfootnote{? व्य‚प‚दिश्य‚ते}}इह तु पुन‚र्व्य‚{\tiny $_{lb}$}‚पेक्षाभेदोपि नास्ति त‚त्क‚थ‚न्त‚देव त‚स्य ध‚र्मान्त‚र‚म्भ‚विष्य‚तीति ।
	{\color{gray}{\rmlatinfont\textsuperscript{§~\theparCount}}}
	\pend% ending standard par
      ‚{\tiny $_{lb}$}‚

	  
	  \pstart \leavevmode% starting standard par
	\hphantom{.}ध‚र्म‚स्य द्र‚व्याद‚र्थान्त‚र‚प‚क्षे त‚र्हि को दोष इत्याह ॥ \quotelemma{अथेत्या} \cite[5a9]{vn-msN} ‚{\tiny $_{8}$}‚ दि ॥ क‚स्मा‚{\tiny $_{lb}$}‚द्ध‚र्म‚निवृत्तिप्रादुर्भावाभ्यां न द्र‚व्य‚स्य प‚रिणामो य‚स्मा[न्] न ह्य‚र्थान्त‚र‚ग‚ताभ्यां निवृ‚{\tiny $_{lb}$}‚ \leavevmode\ledsidenote{\textenglish{52/s}} त्तिप्रादुर्भावाभ्याम‚र्थान्त‚र‚स्य प‚रिण‚तिर्भ‚व‚ति । त‚देव कुत‚श्चैत ‚{\tiny $_{9}$}‚ \leavevmode\ledsidenote{\textenglish{37a/msK}} न्येपि प‚रिण‚तेः प्र‚स‚ङ्‚{\tiny $_{lb}$}‚गात् । न च चैत‚न्य‚स्य प‚रिण‚तिरिष्य‚ते । प्र‚धान‚पुरुष‚योरैक्याप‚त्तेर‚क‚र्तृता चेति व‚च‚{\tiny $_{lb}$}‚नात् । प्र‚योगः पुनः । य‚द्य‚तोर्थान्त‚र‚न्न त‚द्ग‚ताभ्यां निवृ ‚{\tiny $_{1}$}‚ त्तिप्रादुर्भावाभ्यां त‚स्य प‚रि‚{\tiny $_{lb}$}‚ण‚तिः । त‚द्य‚था चैत‚न्य‚भिन्न‚स्व‚भाव‚स्याङ्कुर‚स्य निवृत्तिप्रादुर्भावाभ्यां चैत‚न्य‚स्य ध‚र्मान्त‚{\tiny $_{lb}$}‚र‚ञ्च द्र‚व्यादिति व्याप‚क‚विरुद्धोप‚ल‚ब्धिः । भ‚वे ‚{\tiny $_{2}$}‚ देत‚न्न य‚स्य क‚स्य‚चिद‚र्थान्त‚र‚स्या‚{\tiny $_{lb}$}‚स‚म्ब‚द्ध‚स्यापि निवृत्तिप्रादुर्भावाभ्याम‚न्य‚स्य प‚रिण‚तिर‚पि कुत[ः] चास‚म्ब‚द्ध‚स्यैव । ‚{\tiny $_{lb}$}‚य‚था त‚स्यैवाङ्कुर‚स्य बीज‚स‚म्ब‚द्ध‚स्य निवृत्तिप्रादुर्भावाभ्याम्बी ‚{\tiny $_{3}$}‚ ज‚स्य तेन सामान्येन ‚{\tiny $_{lb}$}‚साध‚ने सिद्ध‚साध‚नं । ध‚र्म‚स्य द्र‚व्य‚स‚म्ब‚न्धात् [त]द्विशेषेण तु साध‚न‚विक‚ल‚ता ‚{\tiny $_{lb}$}‚निद‚र्श‚न‚स्य । अङ्कुर‚स्य चैत‚न्येन स‚ह स‚म्ब‚न्धाभावादिति ‚{\tiny $_{4}$}‚ चेदाह । \quotelemma{द्र‚व्य‚स्य ध‚र्म ‚{\tiny $_{lb}$}‚इति \cite[5a10]{vn-msN} व्य‚प‚देशो न सिध्य‚ति} \cite[5a10]{vn-msN} । कुतः स‚म्ब‚न्धाभावात् । एवं ‚{\tiny $_{lb}$}‚म‚न्य‚ते द्र‚व्य‚स‚म्ब‚न्धोयं ध‚र्म इत्येत‚देव न विद्य‚ते । त‚त् कुतो व्यावृत्ति ‚{\tiny $_{5}$}‚ प्र‚स‚ङ्ग‚स्येति । ‚{\tiny $_{lb}$}‚अस्त्येव त‚र्हि द्र‚व्य‚ध‚र्म‚योराधाराधेय‚भाव‚ल‚क्ष‚ण‚स्स‚म्ब‚न्ध‚स्त‚त‚श्च स‚विशेष‚णेपि हेतौ ‚{\tiny $_{lb}$}‚असिद्धिरित्य‚त आह । \quotelemma{न‚हि कार्य‚कार‚ण‚भा ‚{\tiny $_{6}$}‚ वाद‚न्यो व‚स्तुभूतः स‚म्ब‚न्धोस्ती} \cite[5b1]{vn-msN} ‚{\tiny $_{lb}$}‚ति । आधाराधेय‚भावोऽपि कार्य‚कार‚ण‚भाव‚विशेषादेव व्य‚व‚स्थाप्य‚ते । य‚था निर्णीत‚मा‚{\tiny $_{lb}$}‚धार‚तोभिनिवृत्तेरात्म‚न‚स्तादृ ‚{\tiny $_{7}$}‚ शो नु\edtext{}{\lemma{नु}\Bfootnote{?}}...यः कार्य‚न्त‚स्येत्य‚त्र प्र‚क‚र‚णे \quotelemma{प्र‚माण‚विनिश्च‚य} ‚{\tiny $_{lb}$}‚इत्य‚भिप्रायः । अस्तु त‚र्हि कार्य‚कार‚ण‚भाव‚स्त‚योरिति चेदाह । \quotelemma{न चान‚योर्द्र‚व्य‚ध‚र्म‚योः ‚{\tiny $_{lb}$}‚कार्य‚कार‚ण‚भाव ‚{\tiny $_{8}$}‚} \cite[5b1]{vn-msN} इति । कुतः स्व‚य‚म‚त‚दात्म‚नोऽत‚त्कार‚ण‚त्वात् । य‚द्धि य‚त्स्व‚{\tiny $_{lb}$}‚भावं न भ‚व‚ति न त‚त्त‚त्कार‚ण‚त‚या भ‚व‚द्भिर‚भ्युपेयं य‚था र‚ज‚स्त‚म‚सः । त‚था चेद‚म‚पि ‚{\tiny $_{lb}$}‚द्र‚व्य‚ध‚र्म‚स्स्व‚भावो ‚{\tiny $_{9}$}‚ \leavevmode\ledsidenote{\textenglish{37b/msK}} भ‚व‚ति त‚त्क‚थ‚मिव त‚स्य कार‚ण‚त्व‚मुपेयादिति व्याप‚कानुप‚ल‚ब्धि‚{\tiny $_{lb}$}‚प्र‚स‚ङ्गं म‚न्य‚ते । न चाय‚म‚सिद्धो हेतुरिति म‚न्त‚व्यं । अर्थाभाव‚प‚क्षं स‚माश्रित्य दोषा‚{\tiny $_{lb}$}‚भिधान‚स्य प्र‚कृत ‚{\tiny $_{1}$}‚ त्त्वात् । य‚दाह । \quotelemma{ध‚र्म‚स्य द्र‚व्याद‚र्थान्त‚र‚त्वं \cite[5b1]{vn-msN} स्यादिति} । अथा‚{\tiny $_{lb}$}‚प्य‚स्म‚द्वै[फ]ल्ये स्यात् पूर्व्व‚कान् कापिलान‚तिप‚त्य साङ्ख्यानां श‚क‚माध‚व‚व‚त् । ‚{\tiny $_{lb}$}‚द्र‚व्य‚स्य व्य‚तिरेकेपि ध‚र्म‚कार ‚{\tiny $_{2}$}‚ ण‚त्व‚मिष्य‚ते त‚दापि ब्रूम इत्याह । \quotelemma{अर्थान्त‚र‚त्त्वेपि द्र‚व्य‚स्य ‚{\tiny $_{lb}$}‚ध‚र्म‚कार‚ण‚त्वे} ऽङ्गीक्रिय \quotelemma{माणेऽर्थान्त‚र‚स्य कार्य‚स्योत्पाद‚नात्} \cite[5b2]{vn-msN} कार‚णात् । द्र‚व्य‚स्य ‚{\tiny $_{lb}$}‚ \leavevmode\ledsidenote{\textenglish{53/s}} \quotelemma{प‚रिणाम इतीष्टं स्या} \cite[5b2]{vn-msN} द् भ‚व‚ता त‚तः किं स्यात् इत्याह [।] \quotelemma{त‚द्विरुद्ध‚स्यापि त‚था‚{\tiny $_{lb}$}‚ग‚ता} नुसारिणः [।] किङ्कार‚ण‚न्तेनापि \quotelemma{हेतुफ‚ल‚संतानं} मृद्द्र‚व्याख्ये पूर्व्व‚कात् \quotelemma{मृप्तिण्डा} ‚{\tiny $_{lb}$}‚त्कार‚ण‚भूता ‚{\tiny $_{4}$}‚ दुत्त‚र‚स्य \quotelemma{घ‚ट‚द्र‚व्य‚स्य कार्य‚स्योत्प‚त्तौ} स‚त्यां \quotelemma{मृद्द्र‚व्यं प‚रिण‚त‚मिति व्य‚व‚हार‚{\tiny $_{lb}$}‚भेद‚स्योप‚ग‚मात्} \cite[5b3]{vn-msN} कार‚णात् । स्यात् म‚तं । य‚दि नाम प्र‚कार‚द्व‚येनापि प‚रि‚{\tiny $_{lb}$}‚णामो ‚{\tiny $_{5}$}‚ न युज्य‚ते प्र‚कारान्त‚रेण तु भ‚विष्य‚तीत्येत‚दाह । \quotelemma{न चे} \cite[5b3]{vn-msN} त्यादि । ‚{\tiny $_{lb}$}‚त‚स्मादुभ[य]थापि न प‚रिणाम इत्युप‚संहारः । \quotelemma{न निर्विवेकं} निर्विशेषं \quotelemma{द्र‚व्य‚मेव} ‚{\tiny $_{lb}$}‚ \cite[5b3]{vn-msN} प‚रो ना ‚{\tiny $_{6}$}‚ पि \quotelemma{द्र‚व्याद‚र्थान्त‚र} मेकान्तेनैव \quotelemma{किन्त‚र्हि द्र‚व्य‚स‚न्निवेशोऽव‚स्था‚{\tiny $_{lb}$}‚न्त} र‚न्नान्यः \quotelemma{य‚थाङ्गुलीनां} स‚न्निवेशोऽव‚स्थान्त‚र‚म्मुष्टिः । य‚थाङ्गुलीनां स‚न्निवेशो‚{\tiny $_{lb}$}‚ऽव‚स्थान्त‚र‚न्त‚त्वान्य‚त्वाभ्याम‚निर्व‚च‚नीय‚म्मुष्टिः क‚स्माद्धेतोर्न ह्य‚ङ्गुल्य ‚{\tiny $_{7}$}‚ एव ‚{\tiny $_{lb}$}‚निर्विवेका मुष्टिः । कुतः \quotelemma{प्र‚सारितानाम‚मुष्टित्वात्} \cite[5b4]{vn-msN} । अन्य‚था प्र‚सारिता‚{\tiny $_{lb}$}‚नाम‚पि विशेषाभावात् मुष्ट्य‚व‚स्थायामिव मुष्टित्त्व‚प्र‚स‚ङ्ग इति । अभाव‚हेतुकाले ‚{\tiny $_{lb}$}‚रूप ‚{\tiny $_{8}$}‚ कः । \quotelemma{नाप्य‚र्थान्त‚रं} मुष्टिर‚ङ्गुलिव्य‚तिरेकेणाप्र‚तिह‚त‚कार‚णेन प्र‚य‚त्न‚व‚तापि ‚{\tiny $_{lb}$}‚मुष्टेर‚नुप‚ल‚ब्धेरिति । क‚दाचि \quotelemma{त्कापिला} एवं ब्रूयुरिति त‚न्म‚तं श‚ङ्क‚ते [।] \quotelemma{न निर्वि‚{\tiny $_{lb}$}‚वेक} ‚{\tiny $_{9}$}‚ \leavevmode\ledsidenote{\textenglish{38a/msK}} मित्यादिना । ग‚तार्थ‚मेत‚द् । \quotelemma{न‚हि मुष्टेर‚ङ्गुलिविशेष‚त्वादि} \cite[5b5]{vn-msN} ति प‚रिह- ‚{\tiny $_{lb}$}‚र‚ति । अस‚क्ताङ्गुल्य एव च निर्विवेका मुष्टिरिति क‚थ‚य‚न् दृष्टान्तायोग‚माह । \quotelemma{अतोपि ‚{\tiny $_{lb}$}‚य‚दुत । ‚{\tiny $_{1}$}‚ प्र‚सारितानाम‚मुष्टित्वादिति} त‚द‚प्य‚युक्त‚मेव । किङ्कार‚णं [।] य‚तोङ्गु \quotelemma{ल्य ‚{\tiny $_{lb}$}‚एव हि} विशिष्ट‚हेतुप्र‚त्य‚य‚ब‚लेन त‚थोत्प‚न्ना \quotelemma{काश्च‚न मुष्टिर्न्न तु स‚र्वाः} । त‚देव कुत ‚{\tiny $_{lb}$}‚इत्याह ‚{\tiny $_{2}$}‚ । \quotelemma{न प्र‚सारिता} अङ्गुल्यो \quotelemma{निर्विवेक‚स्व‚भावा मुष्ट्य‚ङ्गुल्य‚श्चेति} [।] च श‚ब्दोत्र ‚{\tiny $_{lb}$}‚लुप्त‚निर्दिष्टो ज्ञेयः । अथ‚वा मुष्ट्यात्मिका अङ्गुल्यः प्र‚सारिताः स‚त्यो न‚हि निर्विशिष्ट‚{\tiny $_{lb}$}‚रूपा इ ‚{\tiny $_{3}$}‚ ति व्याख्येयं । क‚स्मात् । अव‚स्थादूयेपि प्र‚सारिताप्र‚सारित‚रूपे । उभ‚योर‚{\tiny $_{lb}$}‚प्र‚सारित‚प्र‚सारिताव‚स्थ‚योर्य‚थाक्र‚मं \quotelemma{प्र‚तिप‚त्तिप्र‚स‚ङ्गात्} \cite[5b5]{vn-msN} । प्र‚योगः [पुनः ।] ‚{\tiny $_{lb}$}‚प्र‚सारिताव‚स्थायाम‚प्र‚सारिताव‚स्थायाः प्र‚तिप‚त्तिर्भ‚वेत् अङ्गुलीनां विवेकाभा‚{\tiny $_{lb}$}‚ \leavevmode\ledsidenote{\textenglish{54/s}} वात् । अप्र‚सारिताव‚स्थायामिव स्व‚भाव‚हे ‚{\tiny $_{5}$}‚ तुः । एव‚म‚प्र‚सारिताव‚स्थायां प्र‚सारिता‚{\tiny $_{lb}$}‚व‚स्थायां त‚त्प्र‚ति ‚{\tiny $_{7}$}‚ प‚त्तिः स्यादित्य‚प‚रो योज्यः । य‚त्तूभ‚य‚स्येति व्याप‚कानुप‚ल‚ब्धिः ‚{\tiny $_{lb}$}‚योज्या । अथाऽ ‚{\tiny $_{6}$}‚ पि क‚थ‚ञ्चित्क‚श्चिद्विवेको स्थित‚योर[व]स्थ‚योस्त‚दा स विवेक‚श्चा‚{\tiny $_{lb}$}‚साम‚ङ्गुलीनां स्व‚भाव‚भूतो वा भ‚वेन्न‚वेति विक‚ल्प‚द्व‚यं [।] प्र‚थ‚मे ताव‚द् दोष‚माह । \quotelemma{य ‚{\tiny $_{lb}$}‚एव} ख‚लु \quotelemma{विवेकः स्व ‚{\tiny $_{7}$}‚ भाव‚भूतः । स एव स्व‚भेद‚ल‚क्ष‚णं सुख‚दुःख‚व‚दिति} \cite[5b6]{vn-msN} । ‚{\tiny $_{lb}$}‚द्वितीयेप्याह । \quotelemma{प‚र‚भूते च विवेकोत्पादेऽङ्गुल्यः प्र‚सारिता एवोप‚ल‚भ्येर‚न्} \cite[5b6]{vn-msN} ‚{\tiny $_{lb}$}‚मुष्ट्य‚व‚स्थायाम‚पीति शेषः ‚{\tiny $_{8}$}‚ [।] किमिति । य‚तो न‚हि स्व‚भावाद‚प्र‚च्युत‚स्यार्थान्त‚रो‚{\tiny $_{lb}$}‚त्पादे स‚त्य‚न्य‚थोप‚ल‚ब्धिर्भ‚व‚त्य‚तिप्र‚स‚ङ्गात् । उष्ट्र‚स्याप्य‚र्थान्त‚र‚स्य क‚ल‚भ‚स्योत्पादे‚{\tiny $_{lb}$}‚ऽत्य‚थोप‚ल‚ब्धिः स्या ‚{\tiny $_{9}$}‚ \leavevmode\ledsidenote{\textenglish{38b/msK}} दित्य‚तिप्र‚स‚ङ्गो व‚क्त‚व्यः । प्र‚योगः पुनः । य‚त्र‚स्व‚स्यात्म‚{\tiny $_{lb}$}‚भावाद‚प्र‚च्युतं न त‚स्यार्थान्त‚रोत्पादेपि अन्य‚थोप‚ल‚ब्धिः । य‚थोष्ट्र‚स्य क‚ल‚भ‚{\tiny $_{lb}$}‚प्रादुर्भावे । अप्र‚च्युताश्च स्व ‚{\tiny $_{1}$}‚ स्मात्स्व‚भावाद‚ङ्गुल्यो विवेकोत्पादेपीति विधि‚{\tiny $_{lb}$}‚प्र‚तिषेधाभ्यां हेत्व‚व‚क‚ल्प‚नायां कार‚ण‚विरुद्ध‚कार‚णानुप‚ल‚ब्धी । न‚न्वित्यादि ‚{\tiny $_{lb}$}‚प‚रः । त‚त्वान्य‚त्वाभ्याम‚निर्व‚च‚नीयं त‚दुक्त‚मिति वाक्यार्थः । उक्त‚मेत‚न्न पुन‚र्युक्त‚{\tiny $_{lb}$}‚मित्या \quotelemma{चार्यः} । क‚थ‚म‚युक्त‚मित्याह । \quotelemma{न‚हि स‚तो व‚स्तुन‚स्त‚त्त्वान्य‚त्वे मुक्त्वान्य[ः]प्र‚कारः ‚{\tiny $_{lb}$}‚स‚म्भ‚व‚ती} \cite[5b7]{vn-msN} ति स‚द्व‚स्तुग्र‚ह‚णं क‚ल्प ‚{\tiny $_{3}$}‚ नाशिल्पोप‚रिच‚त‚स्यान्यापोहादेः स‚म्भ‚व‚{\tiny $_{lb}$}‚तीति प्र‚तिपाद‚नाय । कुत इत्याह । \quotelemma{त‚योरि} \cite[5b8]{vn-msN} त्यादि । प्र‚योगः पुनः । यौ प‚र‚स्प‚र‚{\tiny $_{lb}$}‚प‚रिहार‚स्थित‚ल‚क्ष‚णौ ‚{\tiny $_{4}$}‚ त‚योरेक‚त्यागोऽप‚रोपादान‚नान्त‚रीय‚कः । एकोपादान‚ञ्चा‚{\tiny $_{lb}$}‚प‚र‚त्याग‚नान्त‚रीय‚कं त‚द्य‚था भावाभावौ । य‚थोक्त‚ध‚र्म‚व‚न्तौ च त‚त्वान्य‚त्व‚{\tiny $_{lb}$}‚प्र‚कारावि ‚{\tiny $_{5}$}‚ ति स्व‚भाव‚हेतुः । न‚न्व‚ङ्गुलीभ्यो मुष्टेस्त‚त्वान्य‚त्व‚प्र‚कारौ मुक्त्त्वाप्य‚{\tiny $_{lb}$}‚न्यः प्र‚कारः संभ‚व‚त्येव । न ह्य‚ङ्गुल्य एव मुष्टिः प्र‚सारितानाम‚मुष्टित्वात् । नाप्य‚{\tiny $_{lb}$}‚ ‚{\tiny $_{6}$}‚ र्थान्त‚रं पृथ‚क्स्व‚भावानुप‚ल‚ब्धेरिति चेदाह । \quotelemma{अङ्गुलीषु पुन‚रि} \cite[5b8]{vn-msN} ‚{\tiny $_{lb}$}‚त्यादि । प्र‚तिक्ष‚णं विनाशो विद्य‚ते यासां इति विग्र‚हः । ता एव क्ष‚णिक‚त्वात् ‚{\tiny $_{lb}$}‚त‚थाविधा जाय ‚{\tiny $_{7}$}‚ न्ते येन मुष्ट्यादिवाच्या भ‚व‚न्तीत्य‚र्थः ॥ त‚देत‚च्च व‚स्तुतो न ‚{\tiny $_{lb}$}‚ \leavevmode\ledsidenote{\textenglish{55/s}} मुष्टेर‚ङ्गुलिविशेषादित्य‚त्रोक्त‚म‚पि प्र‚स‚ङ्गात् युक्त‚मुक्त‚मित्य‚व‚सेयं । अन्य‚था कि‚{\tiny $_{lb}$}‚म‚नेन य‚द्ये ‚{\tiny $_{8}$}‚ वं क‚थ‚न्त‚र्हि मुष्टिर‚ङ्गुलीति च व्य‚प‚देश‚भेद इत्य‚त आह । \quotelemma{त‚त्र मुष्ट्यादि‚{\tiny $_{lb}$}‚श‚ब्दा विशिष्ट‚विष‚या} \cite[5b9]{vn-msN} विशिष्टाव‚स्थानामेवाङ्गुलीनां वाच‚क‚त्वात् । ‚{\tiny $_{lb}$}‚अङ्गुलीश‚ब्द‚स्तु सामा ‚{\tiny $_{9}$}‚ \leavevmode\ledsidenote{\textenglish{39a/msK}} न्य‚श‚ब्दः स‚र्व्वाव‚स्थानां तासाम‚भिधाय‚क‚त्वात् । य‚थाक्र‚म- ‚{\tiny $_{lb}$}‚मुदाह‚र‚ण‚द्व‚य‚माह । \quotelemma{वीजाङ्कुरादिश‚ब्द‚व‚द् ब्रीह्यादिश‚ब्द‚व‚च्चे} \cite[5b9]{vn-msN} ति । एवं ‚{\tiny $_{lb}$}‚श‚क‚लीकृत‚स‚क‚ल‚प‚र‚प‚क्षः कु ‚{\tiny $_{1}$}‚ चोद्य‚शेषं प‚रोप‚न्य‚स्तं प‚रिजिहीषुः । प‚र‚मुखेन चोद्य‚{\tiny $_{lb}$}‚मुप‚स्थाप‚य‚ति [।] \quotelemma{त‚द्य‚दीत्या} \cite[5b9]{vn-msN} दिना । इद‚म‚स्याकूतं य‚था हि तिलेष्व‚विद्य‚मानं ‚{\tiny $_{lb}$}‚घृतं । त‚था तैल‚म‚पि ‚{\tiny $_{2}$}‚ । त‚द्य‚दि प्राग‚स‚देव कार‚णे कार्य‚मुत्प‚द्य‚ते त‚था घृत‚स्यापि तिलेभ्य ‚{\tiny $_{lb}$}‚उत्प‚त्तिः स्यात् । अस‚त्वात् तैल‚व‚त् । न वा तैल‚स्यापि त‚त एव घृत‚व‚त् । न‚हि अस‚त्ये ‚{\tiny $_{lb}$}‚क‚श्चिद् वि ‚{\tiny $_{3}$}‚ शेष इति स्व‚भाव‚हेतुव्याप‚कानुप‚ल‚ब्धित्वेनाभिम‚त‚योर्व्याप्य‚व्याप‚क‚भाव‚{\tiny $_{lb}$}‚प्र‚साध‚न‚प्र‚कार एषः ।
	{\color{gray}{\rmlatinfont\textsuperscript{§~\theparCount}}}
	\pend% ending standard par
      ‚{\tiny $_{lb}$}‚

	  
	  \pstart \leavevmode% starting standard par
	त‚देत‚त् स‚र्वं[म]भ्य‚व‚धाय कृत्योत्थाप‚न‚म्भ‚व‚त इ ‚{\tiny $_{4}$}‚ ति म‚न्य‚मानः प्राह । \quotelemma{न‚नु स‚र्व्व‚त्र ‚{\tiny $_{lb}$}‚स‚र्व्व‚स्यास‚त्वेप्य‚य‚न्तुल्यो दोषः} \cite[5b10]{vn-msN} । न‚हि स‚त्वे क‚श्चिद्विशेष इतिप्र‚योगो\edtext{}{\lemma{योगो}\Bfootnote{‚{\tiny $_{lb}$}‚? प्र‚योगः}}पुन‚स्तावेव स‚त्वादिति हेतुविप‚र्य‚यं ‚{\tiny $_{5}$}‚ कृत्वा कार्यौ । अथापि क‚श्चिद्वि‚{\tiny $_{lb}$}‚शेषोस्ति तेन स‚त्वेपि न स‚र्व्वं स‚र्व‚स्मात् जाय‚ते तेन संदिग्ध‚विप‚क्ष‚व्यावृत्तिक‚त्वं ‚{\tiny $_{lb}$}‚प्र‚माण‚योरिति चेदाह । विशेषे चाऽ ‚{\tiny $_{6}$}‚ \quotelemma{भ्युप‚ग‚म्य} माने स‚विशेष‚स्त्रैगुण्यात् स‚त्व‚र‚{\tiny $_{lb}$}‚ज‚स्त‚मोरूपाद् भिन्नः स्यात् । क‚स्मात्त‚स्य त्रैगुण्य‚स्य भावेपि विशेष‚स्यान‚नुवृत्तेः ‚{\tiny $_{lb}$}‚कार‚णात् । प्र‚योगः पुनः । य‚द्भावेपि ‚{\tiny $_{7}$}‚ य‚न्नानुव‚र्त्त‚ते त‚त्त‚स्माद‚त्य‚न्तं भिन्नं । य‚था ‚{\tiny $_{lb}$}‚श‚ब्द‚स्प‚र्श‚रूप‚र‚स‚ग‚न्धेभ्य‚श्चैत‚न्य‚न्नानुव‚र्त्त‚ते च विशेष‚स्त्रैगुण्य‚भावेपीति स्व‚भाव‚हेतुः । ‚{\tiny $_{lb}$}‚एत‚च्चाभ्युप‚ग‚म्योद्ग्राहितं ‚{\tiny $_{8}$}‚ । अधुना स‚त्कार्य‚वादे ज‚न्मार्थ एव न युक्त इत्याह । ‚{\tiny $_{lb}$}‚ \quotelemma{स‚त‚श्चे} \cite[5b10]{vn-msN} त्यादि । नैव त‚स्य चास‚त्वेनाभिम‚त‚स्य ज‚न्मास्ति । स‚त्वात् । ‚{\tiny $_{lb}$}‚निष्प‚न्नाव‚स्थायामिवेति विरुद्ध‚व्याप्तोप‚ल‚ब्धिर‚स्य ‚{\tiny $_{9}$}‚ \leavevmode\ledsidenote{\textenglish{39b/msK}} म‚न‚सि व‚र्त्त‚ते । अन्य‚था ‚{\tiny $_{lb}$}‚पुन‚र्जात‚स्यापि पुन‚र्जातिः प्र‚स‚ज्य‚त इत्य‚न‚व‚स्था स्यात् । य‚दाह । ‚{\tiny $_{lb}$}‚ 
	    \pend% close preceding par
	  
	    
	    \stanza[\smallbreak]
	  \flagstanza{\tiny\textenglish{...13}}{\normalfontlatin\large ``\qquad}स‚तो य‚दि भ‚वेज्ज‚न्म जात‚स्यापि भ‚वेद् भ‚व [१३]{\normalfontlatin\large\qquad{}"}\&[\smallbreak]
	  
	  
	  
	    \pstart  \leavevmode% new par for following
	    \hphantom{.}
	   इति ।
	{\color{gray}{\rmlatinfont\textsuperscript{§~\theparCount}}}
	\pend% ending standard par
      ‚{\tiny $_{lb}$}‚\textsuperscript{\textenglish{56/s}}

	  
	  \pstart \leavevmode% starting standard par
	किञ्च साध‚नानां कार‚णानाम्बीज‚तेजोज‚लादी ‚{\tiny $_{1}$}‚ नां वैफ‚ल्यं प्र‚स‚ज्येत साध्य‚स्य ‚{\tiny $_{lb}$}‚क‚र्त्त‚व्य‚स्य क‚स्य‚चिद्रूप‚स्याभावादिति प्र‚योगः । य‚त्र साध्य‚न्न किञ्चिद‚प्य‚स्ति त‚त्र ‚{\tiny $_{lb}$}‚साध‚न‚साफ‚ल्यं विद्य‚ते य‚था न‚भ‚स्य‚नाधेयातिश‚ये । न च साध्य‚ङ्कि ‚{\tiny $_{2}$}‚ ञ्चि‚{\tiny $_{lb}$}‚द‚प्य‚स्ति कार‚णे व्य‚व‚स्थिते स‚ति कार्यं इति व्याप‚कानुप‚ल‚ब्धिः । न चाय‚म‚सिद्धो ‚{\tiny $_{lb}$}‚हेतुरिति म‚न्त‚व्यं । य‚स्माद् य‚स्य क‚स्य‚चिद‚तिश‚य‚स्य त‚त्र कार‚णे स्थिते कार्ये क‚थ‚ञ्चि‚{\tiny $_{lb}$}‚दुत्प‚त्ता ‚{\tiny $_{3}$}‚ विष्य‚माणायां सोतिश‚य‚स्त‚त्रास‚न् क‚थ‚ञ्जायेत नैव जायेतास‚त्वात् । ‚{\tiny $_{lb}$}‚व्योमोत्प‚ल‚मिव दुग्ध इति व्याप‚कानुप‚ल‚ब्धिर‚स्य चेत‚सि स्थिता । अथास‚न्न‚प्य‚ति‚{\tiny $_{lb}$}‚श‚यो जा ‚{\tiny $_{4}$}‚ य‚ते । त‚दा जातौ वा त‚स्यास‚तोपि स‚र्व्वोतिश‚यः स‚र्व्व‚स्माज्जायेतेति तुल्यः ‚{\tiny $_{lb}$}‚प‚र्य‚नुयोग इति । भ‚व‚तोपि घृतातिश‚योपि तिलेभ्य उत्प‚द्येतास‚त्वात् । तेनाति ‚{\tiny $_{5}$}‚ ‚{\tiny $_{lb}$}‚श‚य‚व‚दित्य‚र्थः । स्व‚भाव‚हेतुप्र‚स‚ङ्गः । प‚र‚म‚त‚माश‚ङ्क‚ते । \quotelemma{नातिश‚य‚स्त‚त्रे} \cite[6a2]{vn-msN} ‚{\tiny $_{lb}$}‚त्यादिना । य‚था नास्ति स क‚थ‚न्त‚त्रास‚न् प्र‚कारो जायेतेति प्र‚क्षिप‚ति । जातो ‚{\tiny $_{6}$}‚ वा ‚{\tiny $_{lb}$}‚स‚र्व्वः स‚र्व्व‚स्माज्जायेतेति तुल्यः प‚र्य‚नुयोग इति पूर्वोक्तो दोषो न युज्य‚त ‚{\tiny $_{lb}$}‚इत्य‚भिप्रायः । स‚र्व‚प्र‚कारेणैव त‚र्हि निष्प‚न्न‚रूपातिश‚योस्तीति चेदाह । ‚{\tiny $_{7}$}‚ \quotelemma{न चेत्यादि} [।] ‚{\tiny $_{lb}$}‚एव‚न्ताव‚त्स‚द‚स‚त्कार्य‚वादिनोः स‚र्व्व‚स्मात्स‚र्व्व‚स्योत्प‚त्तिदोष‚स्तुल्य इति प्र‚तिपादितं ।
	{\color{gray}{\rmlatinfont\textsuperscript{§~\theparCount}}}
	\pend% ending standard par
      ‚{\tiny $_{lb}$}‚

	  
	  \pstart \leavevmode% starting standard par
	न च त‚योर‚पि तुल्य‚ञ्चोद्य‚न्न त‚देको व‚क्तुम‚र्ह‚ति । स‚त्कार्य‚वादे च न ‚{\tiny $_{8}$}‚ क‚श्चिज्ज‚{\tiny $_{lb}$}‚न्मार्थ इति प्र‚साधितं तेनाय‚म‚स्तीत्य‚धिको दोषः । त‚देव‚ङ्क‚दाचित्प‚रोऽभिद‚ध्यान्न‚नु भो ‚{\tiny $_{lb}$}‚य‚दि नाम म‚यैत‚न्न प‚रिहृतं भ‚व‚ता त्व‚व‚स्यं\edtext{}{\lemma{स्यं}\Bfootnote{? श्यं}}स्थितेः किञ्चित् स्व‚प‚क्ष‚स्य र‚क्ष‚णाय ‚{\tiny $_{9}$}‚ \leavevmode\ledsidenote{\textenglish{40a/msK}} ‚{\tiny $_{lb}$}‚वाच्यं । न‚हि प‚र‚स्य प‚क्षं दूष‚य‚ता स्व‚प‚क्ष‚स्थितिर‚न‚व‚द्या ल‚भ्य‚ते । न भ‚व‚ति नित्यः ‚{\tiny $_{lb}$}‚श‚ब्दो मूर्त्त‚त्वात् । सुखादिभिर्व्य‚भिचारेणेत्यादाव‚नित्य‚त्वा ‚{\tiny $_{1}$}‚ सिद्ध‚व‚दित्य‚त आह । ‚{\tiny $_{lb}$}‚ \quotelemma{अस‚तोपि कार्य‚स्य कार‚णादुत्पादे यो य‚ज्ज‚न‚न‚स्व‚भाव‚स्त‚त एव त‚स्य ज‚न्म ज‚न्म नान्य‚{\tiny $_{lb}$}‚स्मादिति निय‚म} \cite[6a3]{vn-msN} इति । अपि श‚ब्दः स‚म्भाव‚नायां । इदं अत्रा ‚{\tiny $_{2}$}‚ र्थ‚त‚त्व‚{\tiny $_{lb}$}‚म‚विद्य‚मान‚म‚पि तैलं तिलेभ्य एवोत्प‚द्य‚ते । त‚दुत्पाद‚न‚श‚क्तियुक्त‚त्वात् तिलानां ‚{\tiny $_{lb}$}‚नान्य‚स्मात् त‚ज्ज‚न‚न‚श‚क्तिविक‚ल‚त्वात्त‚स्य । श‚क्तिप्र‚तिनिय‚म एव च क‚थ‚मि ‚{\tiny $_{3}$}‚ ति च ‚{\tiny $_{lb}$}‚प‚र्य‚नुयोगे व‚स्तुस्व‚भावैरुत्त‚रं वाच्यं । य एव‚म्भ‚व‚न्ति य‚था वा त‚थैव प्र‚धाना‚{\tiny $_{lb}$}‚ \leavevmode\ledsidenote{\textenglish{57/s}} न्म‚हान् एव जाय‚ते नाह‚ङ्कारो म‚ह‚तो ऽह‚ङ्कारो न प‚ञ्च‚त‚न्मात्राणीत्या ‚{\tiny $_{4}$}‚ दि प्र‚क्रिया । ‚{\tiny $_{lb}$}‚त‚त्र च भ‚व‚तः श‚क्तिप्र‚तिनिय‚माव‚ल‚म्ब‚न‚मेव‚स‚र‚णं\edtext{}{\lemma{णं}\Bfootnote{? श‚र‚णं}}। अन्य‚स्य ‚{\tiny $_{lb}$}‚प‚रिहारोपाय‚स्याभावात् [।] त‚देत‚च्च न म‚मापि काकेन भ‚क्षितं । तेन ‚{\tiny $_{5}$}‚ य‚दुक्त‚न्न‚{\tiny $_{lb}$}‚ह्य‚स‚त्वे क‚श्चिद् विशेष इति त‚द‚युक्तिम‚त् । कार‚ण‚साम‚र्थ्यासाम‚र्थ्य‚कृत‚त्वात् ‚{\tiny $_{lb}$}‚कार्योत्प‚त्य‚नुत्प‚त्योः । त‚स्मात् पुरोनुक्रान्त‚योः प्र‚माण‚योः स‚न्दिग्ध‚वि ‚{\tiny $_{6}$}‚ प‚क्ष‚व्यावृत्ति‚{\tiny $_{lb}$}‚क‚त्व‚साध‚न‚क‚ल‚ङ्काङ्कितो हेतुरिति । भ‚वेदेत‚त्त‚स्यापि हेतो [ः] त‚ज्ज‚न‚न‚{\tiny $_{lb}$}‚स्व‚भाव‚निय‚मः । कुतो जात इत्य‚त आह । \quotelemma{त‚स्यापि स स्व‚भाव‚निय‚मः । स्व‚हेतो ‚{\tiny $_{7}$}‚} ‚{\tiny $_{lb}$}‚रिति \cite[6a4]{vn-msN} । त‚स्यापि स कुत इति चेदाह । इत्येव‚म‚नादिभाव‚स्व‚भाव‚निय‚म ‚{\tiny $_{lb}$}‚इति । न विद्य‚ते आदिर‚स्येति विग्र‚हः अनादित्वाभ्युप‚ग‚माद्धेतुफ‚ल‚प्र‚कृति ‚{\tiny $_{lb}$}‚प‚रं ‚{\tiny $_{8}$}‚ प‚राया नान‚व‚स्थादोषो ल‚घीय‚सीम‚पि क्ष‚तिमाव‚ह‚त्य‚न्य‚थाऽदौ क‚ल्प्य‚माने ‚{\tiny $_{lb}$}‚त‚स्याहेतुक‚त्व‚प्र‚स‚ङ्ग‚स्तेनास्थान एवेय‚माश‚ङ्का भ‚व‚त इति भावः । अथ‚वान्य‚था‚{\tiny $_{lb}$}‚ऽ ‚{\tiny $_{9}$}‚ \leavevmode\ledsidenote{\textenglish{40b/msK}} य‚ङ्ग्र‚न्थो व्याख्याय‚ते [।] निष्प‚र्यायेणास‚न्नेव त‚र्ह्य‚तिश‚यो जाय‚ते । न च ‚{\tiny $_{lb}$}‚स‚र्व्वं स‚र्व्व‚स्माज्जायेतेति प‚र्य‚नुयोज्यं । यो य‚ज्ज‚न‚न‚स्व‚भाव‚स्त‚त एव त‚स्यातिश‚यो‚{\tiny $_{lb}$}‚त्प‚त्तिरिति श‚क्तिनिय‚म‚स‚माश्र‚यादिति क‚दाचित्स्व‚सिद्धान्त‚म‚नादृत्यापि प‚रोभि‚{\tiny $_{lb}$}‚द‚धात्याश‚ङ्कायां न म‚माप्येत‚च्छ‚क्तिप्र‚तिनिय‚माव‚ल‚म्व‚न‚ङ्केन‚चिद्द‚ण्डेन निवारित‚{\tiny $_{lb}$}‚मित्यागूर्याह । \quotelemma{अस‚तोपी} \cite[6a3]{vn-msN} त्यादि । प‚द‚वि ‚{\tiny $_{2}$}‚ भाग‚स्तु पूर्व‚व‚त् ।प्र‚योगो\edtext{}{\lemma{योगो}\Bfootnote{? ‚{\tiny $_{lb}$}‚प्र‚योगः}}पुन‚र्य‚स्य य‚ज्ज‚न‚नाय स‚म‚र्थं कार‚ण‚म‚स्ति सोस‚न्न‚पि जाय‚त एव य‚थातिश‚य‚{\tiny $_{lb}$}‚विशेषः । त‚ज्ज‚न‚नाय स‚म‚र्थ‚ङ्कार‚ण‚म‚स्ति च कार्य‚विशेष‚स्येति ‚{\tiny $_{3}$}‚ स्व‚भाव‚हेतुः । ‚{\tiny $_{lb}$}‚त‚था यो य‚त्राविद्य‚मान‚त‚ज्ज‚न‚न‚स‚म‚र्थ‚कार‚णः स त‚त्रास‚त्वेपि नोदेति । य‚था ‚{\tiny $_{lb}$}‚तिलेषु घृतातिश‚य‚स्त‚था चाविद्य‚मान‚त‚ज्ज‚न‚न‚स‚म‚र्थ‚कार‚णः ‚{\tiny $_{4}$}‚ कार्य‚विशेषः कार‚ण‚{\tiny $_{lb}$}‚विशेष इति व्याप‚कानुप‚ल‚ब्धिः । अप‚रः प‚र्य्यायः । साध‚न‚स्य लिङ्ग‚स्य ।
	{\color{gray}{\rmlatinfont\textsuperscript{§~\theparCount}}}
	\pend% ending standard par
      ‚{\tiny $_{lb}$}‚
	  \bigskip
	  \begingroup
	
	    
	    \stanza[\smallbreak]
	  \flagstanza{\tiny\textenglish{...14}}{\normalfontlatin\large ``\qquad}स‚द‚कार‚णादुपादान‚ग्र‚ह‚णात् स‚र्व्व‚स‚म्भ‚वाभावात् ।&‚{\tiny $_{lb}$}‚श‚क्य‚स्य श ‚{\tiny $_{5}$}‚ क्य‚क‚र‚णात् कार‚ण‚भावाच्च स‚त्का\edtext{}{\lemma{त्का}\Bfootnote{सांख्य‚कारिका ।}} र्यं [॥ १४]{\normalfontlatin\large\qquad{}"}\&[\smallbreak]
	  
	  
	  
	  \endgroup
	‚{\tiny $_{lb}$}‚

	  
	  \pstart \leavevmode% starting standard par
	इत्येव‚मादेर्वैफ‚ल्यं । साध्य‚स्य क‚र्त‚व्य‚स्य क‚स्य‚चित्संश‚य‚विप‚र्यास‚व्य‚व‚च्छेद‚स्य ‚{\tiny $_{lb}$}‚निश्च‚य‚प्र‚त्य‚य‚ज‚न्म‚न‚श्चाभावात् । स‚र्व्वं हि ‚{\tiny $_{6}$}‚ साध‚नं विवाद‚प‚दे व‚स्तुनि संश‚य‚{\tiny $_{lb}$}‚विप‚र्यासाव‚प‚न‚य‚न्त‚द्विष‚य‚न्निश्च‚य‚प्र‚त्य‚य‚मुत्पाद‚य‚द्विभ‚र्त्ति नामानुरूपं न द्व‚य‚म‚प्येत‚त् ‚{\tiny $_{lb}$}‚ \quotelemma{कापिल‚म‚ते} स‚म्भ‚व‚ति । स‚दाव‚स्थित‚स्य का ‚{\tiny $_{7}$}‚ र्य‚स्य हान्युप‚ज‚न‚नायोगात् । अथ ‚{\tiny $_{lb}$}‚ \leavevmode\ledsidenote{\textenglish{58/s}} स‚न्न‚प्य‚यं निश्च‚यः साध‚न‚व‚च‚नाद‚न‚भिव्य‚क्तं । पूर्व्व‚म‚भिव्य‚क्तिमुप‚यात्य‚तो न ‚{\tiny $_{lb}$}‚वैफ‚ल्य‚मिति म‚त‚म‚त आह । \quotelemma{य‚स्य क‚स्य‚चिद‚ति}स‚य\edtext{}{\lemma{य}\Bfootnote{? श‚य}}‚{\tiny $_{8}$}‚ स्या \cite[6a1]{vn-msN} भि‚{\tiny $_{lb}$}‚व्य‚क्तिल‚क्ष‚ण‚स्य त‚त्र साध्ये निश्च‚य‚रूपे क‚थ‚ञ्चिद‚स‚त उत्प‚त्तौ प्राप्तात्साध‚नात् ‚{\tiny $_{lb}$}‚सोऽतिश‚य‚स्त‚त्रास‚न् ‚{\tiny $_{9}$}‚ \leavevmode\ledsidenote{\textenglish{41a/msK}} क‚थ‚ञ्जाय‚ते । जातो वा स‚र्व्वातिश‚यः । स‚म‚स्त‚साध्य‚{\tiny $_{lb}$}‚निश्च‚याभिव्य‚क्तिल‚क्ष‚णः स‚र्व्व‚स्माद‚न्य‚साध‚नात् साध‚नाभासात् वोत्प‚द्येतेति तुल्यः ‚{\tiny $_{lb}$}‚प्र‚स‚ङ्गः । पाव‚कादिप्र‚तिप‚त्तिहेत‚वो ‚{\tiny $_{1}$}‚ धूमाद‚यः स‚त्कार्य‚विनिश्च‚याद्य‚भिव्य‚क्तिङ्कुर्यु‚{\tiny $_{lb}$}‚रित्य‚र्थः । उत्प‚त्त्य चाभिव्य‚क्तिमेत‚दुच्य‚ते । न‚त्विय‚म‚विकृत‚रूपेषु कृतास्प‚दा सा ‚{\tiny $_{lb}$}‚हि त‚त्स्व‚रूप‚ल‚क्ष‚णा त‚द्विष‚य‚ज्ञान‚ल‚क्ष ‚{\tiny $_{2}$}‚ णा । रूपान्त‚र‚प्रादुर्भाव‚ल‚क्ष‚णाभावा भ‚वेत्स्व‚रूपं ‚{\tiny $_{lb}$}‚ताव‚त् अविकार्य‚मिति न साध‚नैर‚न्यैर्वा क‚र्तुं श‚क्य‚ते । विकारे वा पूर्व्व‚स्व‚भाव‚वानिव ‚{\tiny $_{lb}$}‚पूर्व्व‚रूप‚प्रादुर्भाव‚श्चेत्य‚स‚त्कार्य ‚{\tiny $_{3}$}‚ वाद एव स‚म‚र्थितः पूर्वाप‚र‚रूप‚त्यागावाप्तिल‚क्ष‚ण‚{\tiny $_{lb}$}‚त्वात् विकार‚स्य । चैत‚न्य‚स्यैक‚त्वाद‚प‚र‚स्त‚द्विष‚यः प्र‚त्य‚यो न भ‚व‚ति प‚र‚स्येति ‚{\tiny $_{lb}$}‚त‚द्रूपाभि ‚{\tiny $_{4}$}‚ व्य‚क्तिर‚नुप‚प‚न्ना । रूपान्त‚र‚प्रादुर्भावे च नान्य‚स्य किंचिद‚प्युप‚जाय‚ते ‚{\tiny $_{lb}$}‚विल‚क्ष‚ण‚त्वादिति तृतीयापि व्य‚क्तिर‚स‚म्भ‚विनी द्वितीयायाम‚प्य‚य‚म‚निवारितो ‚{\tiny $_{5}$}‚ दोषः । ‚{\tiny $_{lb}$}‚त‚द्विष‚य‚प्र‚त्य‚योद‚येप्य‚र्थान्त‚र‚स्याभूत‚भाव‚वैप‚रीत्य‚स्य व्य‚क्तेर‚योगात् । न चानुप‚{\tiny $_{lb}$}‚कार‚कः प्र‚त्य‚य‚स्य विष‚यः स‚म्भ‚वी । त‚दुप‚कार‚क‚त्वे वा त ‚{\tiny $_{6}$}‚ स्मादेवास्योत्प‚त्तिरिति ‚{\tiny $_{lb}$}‚लिङ्गान‚पेक्षा । स्व‚त एव साध्य‚निश्च‚योस्याभिव्य‚क्तिरिति प्राप्तं । साध‚नापेक्षादेव ‚{\tiny $_{lb}$}‚साध्य‚निश्च‚यात् स्व‚विष‚य‚ज्ञानोत्पादेनैवापेक्षातिश ‚{\tiny $_{7}$}‚ योत्प‚त्तिल‚क्ष‚णास्थिरेषु ल‚ब्धा‚{\tiny $_{lb}$}‚स्प‚देति प्र‚तिपादितं स‚र्व्व‚दा वा भ‚वेत् । लिङ्ग‚स्यापि स‚दा स‚न्निहित‚रूप‚त्वात् । ‚{\tiny $_{lb}$}‚लिङ्ग‚ज्ञानापेक्षायाम‚पि तुल्यः । त‚स्यापि स‚त्वे वादिनः स ‚{\tiny $_{8}$}‚ र्व्व‚कालास्तित्वादिति ।
	{\color{gray}{\rmlatinfont\textsuperscript{§~\theparCount}}}
	\pend% ending standard par
      ‚{\tiny $_{lb}$}‚

	  
	  \pstart \leavevmode% starting standard par
	\hphantom{.}अपि चेत्यादिना स‚त्कार्य‚वाद‚निराक‚र‚णे कार‚णान्त‚र‚माह । \quotelemma{त‚द‚व‚स्थाया} मिति ‚{\tiny $_{lb}$}‚ \cite[6a4]{vn-msN} [।] मृत्पिण्डाव‚स्थायां प‚श्चाद्व‚द‚भिव्य‚क्ताव‚स्थायामिव त‚द‚र्थ‚क्रियेति ‚{\tiny $_{9}$}‚ \leavevmode\ledsidenote{\textenglish{41b/msK}} घ‚ट‚{\tiny $_{lb}$}‚साध्योद‚क‚धार‚ण‚विशेषाद्य‚र्थ‚क्रिया । व्य‚क्तेर‚विशिष्ट‚संस्थानाया अप्रादुर्भावादिति ‚{\tiny $_{lb}$}‚चेत् । प‚र‚म‚तास‚ङ्का\edtext{}{\lemma{ङ्का}\Bfootnote{? श‚ङ्का}}त‚स्या एवेत्यादि प्र‚तिविधानं । एत‚दुक्त‚म्भ‚व‚ति । ‚{\tiny $_{lb}$}‚ग्रीवादिस‚न्निवे ‚{\tiny $_{1}$}‚ श‚विशेषाव‚च्छिन्न एषोर्थ‚क्रियाविशेष‚कारी क‚श्चित् मृद्विकारो ‚{\tiny $_{lb}$}‚घ‚ट इत्युच्य‚ते नान्यः । स चेत् प्राग‚पि मृप्तिण्डाव‚स्थायाम‚पि त‚दाव्य‚क्ताव‚स्थायामिव ‚{\tiny $_{lb}$}‚त‚द‚र्थ‚क्रियोप‚ल्ब्धौ स्या ‚{\tiny $_{2}$}‚ तां । न च भ‚व‚त‚स्त‚स्मान्नास्त्येवासाविति निश्च‚यः स‚माधी‚{\tiny $_{lb}$}‚ \leavevmode\ledsidenote{\textenglish{59/s}} य‚तां किम‚लीक‚निर्ब‚न्धेनेति । अव‚स्थातुर्भावाद‚साव‚प्य‚स्तीति चेदाह । \quotelemma{न‚हि रूपान्त‚{\tiny $_{lb}$}‚र‚स्य भावे रूपान्त‚र‚म‚स्ति} \cite[6a5]{vn-msN} । ‚{\tiny $_{3}$}‚ पीत इव नील‚मिति विरुद्ध‚व्याप्तोप‚ल‚ब्धिरा‚{\tiny $_{lb}$}‚कूता । न चाव‚स्थाव‚स्थात्रोर‚भेदाद‚सिद्धो हेतुरिति ग‚र्जित‚व्यं । य‚स्मान्न च रूप‚प्र‚ति‚{\tiny $_{lb}$}‚भास‚भेदेपि व‚स्तुभेदो युक्त ‚{\tiny $_{4}$}‚ [ः,] अतिप्र‚स‚ङ्गात् । रूप‚प्र‚तिभास‚भेद‚ग्र‚ह‚ण‚मुप‚ल‚क्ष‚णार्थं । ‚{\tiny $_{lb}$}‚तेनार्थ‚क्रियाभेदोप्य‚भ्युप‚ग‚न्त‚व्यः । एवं म‚न्य‚ते । य‚दि भिन्न‚प्र‚तिभासि ज्ञानं भेदं ‚{\tiny $_{lb}$}‚साध‚य‚ति त‚दा ‚{\tiny $_{5}$}‚ सुख‚दुःख‚मोहानां अस‚ङ्कीर्णा भेद‚व्य‚व‚स्था भ‚वेत् । नान्य‚था त‚था ‚{\tiny $_{lb}$}‚च मृप्तिण्ड‚घ‚ट‚योर‚पि प‚र‚स्प‚र‚म‚त्यंत‚म्भेद इति प्र‚तिजानीम‚हे । भिन्नाकार‚ज्ञान‚प‚रि ‚{\tiny $_{6}$}‚ ‚{\tiny $_{lb}$}‚च्छेद्य‚त्वात् । प‚र‚स्प‚रास‚म्भ‚विकार्य‚कारित्वाच्च‚सुखादिव‚दिति स्व‚भाव‚हेतू । अन्य‚था ‚{\tiny $_{lb}$}‚सुखादीनाम‚पि प‚र‚स्प‚र‚म‚भेद‚प्र‚स‚ङ्गः श‚क्तिव्य‚क्तिव‚त् । विशेषो वा वाच्य इति । ‚{\tiny $_{7}$}‚ ‚{\tiny $_{lb}$}‚त‚स्मादित्युप‚संहारः । न‚हि त‚स्य घ‚टादेस्त‚स्मिन्नुप‚ल‚ब्धिल‚क्ष‚ण‚प्राप्ते स्व‚भावे स्थितौ ‚{\tiny $_{lb}$}‚स‚त्याम‚नुप‚ल‚ब्धिर्युज्य‚ते । अथापि भ‚व‚ति त‚दाऽस्थितिश्च त‚स्मिन्स्व‚भावेऽत‚त्व‚म‚त ‚{\tiny $_{8}$}‚ ‚{\tiny $_{lb}$}‚त्स्व‚भाव‚त्व‚मुप‚ल‚ब्धिल‚क्ष‚ण‚प्राप्तात्स्व‚भावादेकान्तेन भेद इति याव‚त् ।
	{\color{gray}{\rmlatinfont\textsuperscript{§~\theparCount}}}
	\pend% ending standard par
      ‚{\tiny $_{lb}$}‚

	  
	  \pstart \leavevmode% starting standard par
	एवं स्व‚भाव[ा]नुप‚ल‚ब्धौ साध‚नाङ्ग‚स‚म‚र्थ‚नं प्र‚प‚ञ्चेनाभिधाय प‚रिशिष्टास्व‚नुप‚ल‚{\tiny $_{lb}$}‚ब्धिष्वाचिख्यासुराह ‚{\tiny $_{9}$}‚ \leavevmode\ledsidenote{\textenglish{42a/msK}} । \quotelemma{व्याप‚कानुप‚ल‚ब्धावि} \cite[6a8]{vn-msN} त्यादि । ध‚र्म‚योर्य‚था शिंश‚पा- ‚{\tiny $_{lb}$}‚त्व‚वृक्ष‚त्व‚योर्व्याप्य‚व्याप‚क‚भावं केन‚चित्प्र‚माणेन प्र‚साध्य व्याप‚क‚स्य वृक्ष‚त्वादेर्न्निवृत्ति‚{\tiny $_{lb}$}‚प्र‚साध‚नं स‚म‚र्थ‚नं साध‚नाङ्ग‚स्येत्य‚ध्याहारः । य‚था नास्त्य‚त्र शिंश‚पा वृक्षाभावादिति । ‚{\tiny $_{lb}$}‚न‚नु त‚त्र स्व‚भावानुप‚ल‚ब्ध्यैव त‚द‚भावः सिध्य‚ति त‚क्तिम‚न‚या । न‚हि निष्पादित‚क्रिये ‚{\tiny $_{lb}$}‚क‚र्म‚णां ‚{\tiny $_{2}$}‚ विशेषाधायि साधु साध‚न‚म्भ‚व‚ति । \quotelemma{साध‚क‚त‚म‚ङ्क‚र‚ण‚मिति} \href{http://sarit.indology.info/?cref=P\%C4\%81.1.4}{पाणिनिः १।४। ४२} व‚च‚नात् अन‚धिग‚तार्थाधिग‚म‚रूप‚ञ्च प्र‚माण‚मुक्त‚म‚ज्ञातार्थ‚प्र‚काशो वेति । ‚{\tiny $_{lb}$}‚स‚त्य‚मेत ‚{\tiny $_{3}$}‚ त् । त‚थाहि नेयं स‚र्व्व‚त्र प्र‚युज्य‚ते । किन्त‚र्हि व्योम‚ग‚त‚त्र‚पादिमात्रे य‚त्र ‚{\tiny $_{lb}$}‚साल‚स‚र‚ल‚प‚लार्शाशंश‚पादिपाद‚प‚भेदाव‚धार‚ण‚न्नास्ति त‚त्र । स‚र्व्व‚था य‚त्रैव ‚{\tiny $_{4}$}‚ व्याप्या‚{\tiny $_{lb}$}‚ \leavevmode\ledsidenote{\textenglish{60/s}} भावो न निश्चीय‚ते क्व‚चित् कुत‚श्चिद् भ्रान्तिनिमित्तात् त‚त्रैवेयं प्र‚युज्य‚ते । कार‚णा‚{\tiny $_{lb}$}‚नुप‚ल‚ब्धिर‚पि य‚त्र कार्याभावो न निश्चीय‚ते त‚त्रैव प्र‚योक्त‚व्या ना ‚{\tiny $_{5}$}‚ न्य‚त्र वैय‚र्थ्यात् । ‚{\tiny $_{lb}$}‚य‚था स‚न्त‚म‚से धूमाभावानिश्च‚ये नास्त्य‚त्र धूमोऽग्न्य‚भावादिति । कार्याभावे संश‚यात् । ‚{\tiny $_{lb}$}‚कार‚णाभावे च निश्च‚यात् । स्व‚भाव‚विरुद्धोप‚ल‚ब्धिर‚पि सं ‚{\tiny $_{6}$}‚ ग‚विष‚य‚भावाव‚स्थित‚{\tiny $_{lb}$}‚गात्र‚स्प‚र्श‚वालाक‚लापाकुलान‚लालीढ एव व्योमादिमात्र‚व‚र्तिनिर्देशे प्र‚योक्त‚व्या । ‚{\tiny $_{lb}$}‚कार‚ण‚विरुद्धोप‚ल‚ब्धिश्चाप्य‚दृश्य‚मान‚क‚मारोमोद्ग ‚{\tiny $_{7}$}‚ \leavevmode\ledsidenote{\textenglish{42b/msK}}म‚द‚न्त‚वीणादिभेद‚भावाभा वाक्य‚{\tiny $_{lb}$}‚श‚क्य‚गानुस‚मीपाव‚स्थित‚पुरुष‚स‚माक्रान्त‚भूत‚ल एव प्र‚कृतेनान्य‚त्र वैफ‚ल्यात् । अन‚या‚{\tiny $_{lb}$}‚दिसा\edtext{}{\lemma{दिसा}\Bfootnote{? दिशा}}ऽन्यासाम‚प्य‚नुप‚ल‚ब्धीनाम्प्र‚यो ‚{\tiny $_{8}$}‚ ग‚विष‚योऽनुस‚र्त्त‚व्य इति । तेषां स्व‚भाव‚{\tiny $_{lb}$}‚व्याप‚क‚कार‚णानां । विरुद्धास्तेषामुप‚ल‚ब्ध‚य‚स्तास्विति विग्र‚हः । द्व‚योर्विरोध‚योर्म‚{\tiny $_{lb}$}‚ध्ये एक‚स्योप‚द‚र्श‚नं । द्वौ पुन‚र्विरोधाव‚विक‚ल‚का ‚{\tiny $_{1}$}‚ र‚ण‚स्य भ‚व‚तोन्य‚भावे भावः । ‚{\tiny $_{lb}$}‚प‚र‚स्प‚र‚प‚रिहार‚स्थित‚ल‚क्ष‚ण‚श्च । अन‚या दिशा स्व‚भाव‚विरुद्ध‚कार्योप‚ल‚ब्ध्यादिष्व‚पि ‚{\tiny $_{lb}$}‚साध‚नाङ्ग‚स‚म‚र्थ‚नं सुज्ञान‚मेवेति नोक्तं । त‚था ‚{\tiny $_{2}$}‚ पि किञ्चिन्मात्र‚प्र‚योग‚भेदादेका‚{\tiny $_{lb}$}‚द‚शानुप‚ल‚ब्धिव्य‚तिरिक्तास्व‚पि कार‚ण‚विरुद्ध‚व्याप्तोप‚ल‚ब्धिकार्य‚विरुद्ध‚व्याप्तोप‚{\tiny $_{lb}$}‚ल‚ब्धिव्याप‚क‚विरुद्ध‚कार्योप‚ल‚ब्धिकार्य‚विरुद्ध‚का ‚{\tiny $_{3}$}‚ र्योप‚ल‚ब्ध्यादिषु साध‚नाङ्ग‚स‚म‚{\tiny $_{lb}$}‚र्थ‚न‚मुक्त‚म्वेदित‚व्यं । तासां पुन‚रुदाह‚र‚णानि य‚थाक्र‚मं । नात्र धूम‚स्तुषार‚स्प‚र्शात् । ‚{\tiny $_{lb}$}‚नेहाप्र‚तिव‚द्ध‚साम‚र्थ्यान्य‚ग्निकार ‚{\tiny $_{4}$}‚ णानि स‚न्ति तुषार‚स्प‚र्शात् । न तुषार‚स्प‚र्शोऽत्र ‚{\tiny $_{lb}$}‚धूमात् । नेहाप्र‚तिब‚द्ध‚साम‚र्थ्यानि शीत‚कार‚णानि स‚न्ति धूमादिति ।
	{\color{gray}{\rmlatinfont\textsuperscript{§~\theparCount}}}
	\pend% ending standard par
      ‚{\tiny $_{lb}$}‚
	  \bigskip
	  \begingroup
	
	    
	    \stanza[\smallbreak]
	  \flagstanza{\tiny\textenglish{...15}}{\normalfontlatin\large ``\qquad}हेतुकार्य‚विरुद्धाप्त‚भावो व्याप‚क ‚{\tiny $_{5}$}‚ कार्य‚योः ।&‚{\tiny $_{lb}$}‚विरुद्ध‚कार्य‚योर‚न्यः प्र‚तिषेध‚स्य साध‚कः ॥ [१५]{\normalfontlatin\large\qquad{}"}\&[\smallbreak]
	  
	  
	  
	  \endgroup
	
	  \bigskip
	  \begingroup
	
	    
	    \stanza[\smallbreak]
	  \flagstanza{\tiny\textenglish{...16}}{\normalfontlatin\large ``\qquad}नेह धूमो हिम‚स्प‚र्शात् स‚म‚र्थ‚न्नाग्निकार‚णं ।&‚{\tiny $_{lb}$}‚नेह धूमाद्धिम‚स्प‚र्शो न श‚क्यं शीत‚कार‚ण [१६] मिति{\normalfontlatin\large\qquad{}"}\&[\smallbreak]
	  
	  
	  
	  \endgroup
	‚{\tiny $_{lb}$}‚

	  
	  \pstart \leavevmode% starting standard par
	[—] ‚{\tiny $_{6}$}‚ स‚ङ्ग्र‚ह‚श्लोकौ ।
	{\color{gray}{\rmlatinfont\textsuperscript{§~\theparCount}}}
	\pend% ending standard par
      ‚{\tiny $_{lb}$}‚

	  
	  \pstart \leavevmode% starting standard par
	एवं ताव‚देकेन प्र‚कारेणासाध‚नाङ्ग‚व‚च‚न‚त्वादिनो निग्र‚ह‚स्थान‚मिति प्र‚तिपादितं । ‚{\tiny $_{lb}$}‚प्र‚कारान्त‚रेणापि त‚देवोप‚पाद‚य‚ति । \quotelemma{अथ‚वेत्यादि ‚{\tiny $_{7}$}‚} \cite[6b1]{vn-msN} न चेति स‚मुदाय‚श्चाय‚म‚त्रा‚{\tiny $_{lb}$}‚वृत्या पूर्वोदितार्थ‚प‚रित्यागेनार्थान्त‚र‚स‚मुच्च‚ये व‚र्त्त‚ते ॥ न‚तु ध‚व‚स्थित्याथ‚वा ख‚दिर‚{\tiny $_{lb}$}‚मित्यादाविव पूर्व्वार्थ‚प‚रित्यागेन विक‚ल्पादिवि ‚{\tiny $_{8}$}‚ \leavevmode\ledsidenote{\textenglish{43a/msK}} ध‚स्याप्य‚र्थ‚स्य विव‚क्षित‚त्वात् । इह ‚{\tiny $_{lb}$}‚ \leavevmode\ledsidenote{\textenglish{61/s}} च प‚र्याये साध‚न‚श‚ब्दः क‚र‚ण‚साध‚नः । इहाङ्ग‚श‚ब्दोऽव‚य‚व‚व‚च‚नः पूर्व‚स्मिन्कार‚ण‚{\tiny $_{lb}$}‚व‚च‚न इति विशेषः । त्रिरूप‚हेतुव‚च‚न ‚{\tiny $_{1}$}‚ स‚मुदाय‚ग्र‚ह‚णेन तेषु ल‚क्ष‚णादिव‚च‚नानां ‚{\tiny $_{lb}$}‚साध‚न‚त्वं तिर‚य‚ति । स्याद् बुद्धिः साध‚र्म्य‚व‚ति प्र‚योगेनास‚प‚क्षे हेतोर‚स‚त्व‚मुच्य‚ते । ‚{\tiny $_{lb}$}‚वैध‚र्म्य‚व‚ति च न स‚प‚क्ष‚स‚त्व‚म‚न‚न्त‚र ‚{\tiny $_{2}$}‚ मेव निषेध्य‚मान‚त्वात् । त‚त्क‚थ‚न्त‚स्यैक‚स्याप्य‚{\tiny $_{lb}$}‚व‚च‚न‚म‚साध‚नाङ्ग‚व‚च‚न‚मित्येत‚न्न व‚क्ष्य‚माणे व्याह‚त‚मिति । एत‚च्च नैव‚मेव हि ‚{\tiny $_{lb}$}‚व्याख्याय‚ते । त्रिरूपो हेतुर‚र्थात्म‚कः । ‚{\tiny $_{3}$}‚ प‚र‚मार्थ‚तोव‚स्थित‚स्त‚स्य व‚च‚ने ये प्र‚काश‚के ‚{\tiny $_{lb}$}‚प‚क्ष‚ध‚र्म‚व‚च‚नं स‚प‚क्ष‚स‚त्व‚व‚च‚ने प‚क्ष‚ध‚र्म‚व‚च‚नं विप‚क्ष‚स‚त्व‚व‚च‚ने वा त‚योस्स‚मुदायः त‚स्य ‚{\tiny $_{lb}$}‚व ‚{\tiny $_{4}$}‚ च‚न‚द्व‚य‚स‚मुदाय‚स्याङ्ग‚म्प‚क्ष‚ध‚र्मादिव‚च‚न‚मिति प‚क्ष‚ध‚र्म‚व‚द‚न‚न्ताव‚द‚विच‚ल‚मित‚र‚योः ‚{\tiny $_{lb}$}‚त्व‚न्य‚त‚रान्य‚त‚र‚त् कादाचित्कं । तेन व‚च‚न‚द्व‚य‚स‚मु ‚{\tiny $_{5}$}‚ दाय‚रूप‚स्याङ्गिनोङ्गं द्विविध‚मेव ‚{\tiny $_{lb}$}‚स‚दा त‚स्येदानीम‚ङ्ग‚स्यैक‚स्याप्य‚व‚च‚न‚म‚साध‚नाङ्ग‚म्व‚च‚नं । न केव‚लं द्व‚योः प्र‚थ‚म‚{\tiny $_{lb}$}‚व्याख्यानुसारेणेत्य‚पि श ‚{\tiny $_{6}$}‚ ब्दात् । द्व‚योर्ह्य‚व‚च‚नं तूष्णीम्भावः । स चोक्तोऽप्र‚तिभ‚या ‚{\tiny $_{lb}$}‚तूष्णींभावादिति प‚र्यायान्त‚र‚म‚प्याह ।
	{\color{gray}{\rmlatinfont\textsuperscript{§~\theparCount}}}
	\pend% ending standard par
      ‚{\tiny $_{lb}$}‚

	  
	  \pstart \leavevmode% starting standard par
	\hphantom{.}\quotelemma{अथ‚वे} \cite[6b1]{vn-msN} त्यादि । त‚स्यैवेति त्रिरूप‚व‚च‚न‚स‚मुदाय‚स्य य‚न्नाङ्गं ना ‚{\tiny $_{7}$}‚ व‚य‚वः । ‚{\tiny $_{lb}$}‚क‚थं पुनः प्र‚तिज्ञादीनाम‚साध‚नाङ्ग‚त्व‚मिति चेत् । उच्य‚ते । प्र‚तिज्ञाव‚च‚न‚{\tiny $_{lb}$}‚साध‚नं । साक्षात् पारंप‚र्येण वा त‚स्याः सिद्धेर‚नुत्प‚त्तेः त‚थाह्य‚र्थ ए ‚{\tiny $_{8}$}‚ व प्र‚तिब‚न्धा‚{\tiny $_{lb}$}‚र्थ‚ङ्ग‚म‚य‚ति । नाभिधान‚म‚र्थ‚प्र‚तिब‚न्ध‚विक‚ल‚त्वात् त‚स्मात् प्र‚तिज्ञाव‚च‚नं हेतु‚{\tiny $_{lb}$}‚व‚च‚नं वा न साक्षात्साध‚न‚म‚र्थ‚सिद्धौ । संश‚य‚श्च प‚क्ष‚व‚च‚नाद‚र्यें दृष्टो ‚{\tiny $_{9}$}‚ \leavevmode\ledsidenote{\textenglish{43b/msK}} न ‚{\tiny $_{lb}$}‚निश्च‚य‚स्त‚द‚तोपि न साक्षात् साध‚नं । स्यान्म‚तं संश‚य एवासिद्धः प‚क्ष‚व‚च‚नाद्वादि‚{\tiny $_{lb}$}‚प्र‚तिवादिनोर्निश्चित‚त्वाद‚थान्येषां भ‚व‚ति । एवं स‚ति कृत‚क‚त्वादिव‚च‚नेप्य‚व्युत्पं\edtext{}{\lemma{व्युत्पं}\Bfootnote{‚{\tiny $_{lb}$}‚? व्युत्प‚न्}}‚{\tiny $_{1}$}‚ नानां संश‚यो भ‚व‚तीत्य‚नेकान्तः । त‚देत‚द‚स‚म्ब‚द्धं । वादिप्र‚तिवादिनो ‚{\tiny $_{lb}$}‚र्हि निश्चित‚त्व‚मेक‚स्मिन् वा ध‚र्मेऽनित्य‚त्वादिके प्र‚त्याय‚यितुमार‚ब्धे भ‚वेत् प्र‚त्य‚नीक‚{\tiny $_{lb}$}‚ध‚र्म‚द्व‚ये वा ॥ ‚{\tiny $_{2}$}‚ न ताव‚देक‚स्मिन् विवादाभाव‚तः । साध‚न‚प्र‚योगान‚र्थ‚क्य‚प्र‚स‚ङ्गात् [।] ‚{\tiny $_{lb}$}‚नापि प्र‚त्य‚नीक‚ध‚र्म‚द्व‚ये व‚स्तुनो विरुद्ध‚ध‚र्म‚द्व‚याध्यास‚प्र‚स‚ङ्गात् । य‚दाह्येक‚स्मिन्व‚{\tiny $_{lb}$}‚स्तुनि प्र‚मा ‚{\tiny $_{3}$}‚ ण‚ब‚लेन विरुद्धौ ध‚र्मौ वादिप्र‚तिवादिभ्यां निश्चितौ भ‚व‚त‚स्त‚दा त‚द्व‚स्तु ‚{\tiny $_{lb}$}‚ \leavevmode\ledsidenote{\textenglish{62/s}} द्व्यात्म‚कं प्राप्तं । अथ न प्र‚माण‚साम‚र्थ्यात् तौ निश्चिताव‚पि तु स्व‚स्मात्स्व‚{\tiny $_{lb}$}‚स्मादाग ‚{\tiny $_{4}$}‚ मात् । एव‚म‚पि तु ध‚र्म‚योः प्र‚माणेन निश्च‚यात् क‚थ‚न्न प‚क्ष‚व‚च‚नात् संश‚यो ‚{\tiny $_{lb}$}‚भ‚व‚तीति वाच्यं । त‚स्मात् प‚क्ष‚व‚च‚नं न साक्षात् साध‚नं । नापि पार‚म्प‚र्य्येण सा ‚{\tiny $_{5}$}‚ ध्या‚{\tiny $_{lb}$}‚भिधाय‚क‚त्वेनासिद्धे हेतुदृष्टान्ताभासोक्तिव‚द‚श‚क्य‚सूच‚क‚त्वात् । हेतुव‚च‚न‚न्तु श‚क्य‚{\tiny $_{lb}$}‚सूच‚क‚त्वात् श‚क्तितः साध‚न‚निष्टं स‚दोच्य‚ते साध‚नाङ्ग‚म्प्र‚तिज्ञाव ‚{\tiny $_{6}$}‚ च‚न‚त्वे स‚ति ‚{\tiny $_{lb}$}‚साध‚नोप‚कार‚क‚त्वाद्धेतुव‚च‚न‚व‚त् । साध‚न‚विष‚य‚प्र‚काश‚न‚द्वारेण च प्र‚तिज्ञासाध‚न‚म‚{\tiny $_{lb}$}‚नुगृह्णाति । अन्य‚थाह्य‚विष‚यं त‚त्साध‚नं प्र‚व‚र्त्त‚ते । ज्ञा ‚{\tiny $_{7}$}‚ नात्म‚म‚नःस‚न्निक‚र्षादीनाम‚पि ‚{\tiny $_{lb}$}‚साध‚नोप‚कार‚क‚त्व‚म‚तो व‚च‚न‚त्वे स‚तीति विशेष‚णं । इत‚श्च साध‚नाङ्ग‚साध्य‚साध‚न‚{\tiny $_{lb}$}‚विष‚य‚प्र‚काश‚नात् दृष्टान्त‚व‚च‚न‚व‚दिति । ‚{\tiny $_{8}$}‚ इद‚म‚प्य‚त्य‚र्थ‚म‚सारं । य‚स्माद‚नित्यं श‚ब्दं ‚{\tiny $_{lb}$}‚साध‚येत्य‚भ्य‚र्थंना वाह्यं व‚च‚न‚त्वे स‚ति साध‚नोप‚कार‚कं साध्य‚साध‚क‚विप‚र्य‚य‚प्र‚काश‚{\tiny $_{lb}$}‚क‚ञ्च न च त‚द‚न्त‚र‚ङ्गं साध्य‚सिद्धा ‚{\tiny $_{9}$}‚ \leavevmode\ledsidenote{\textenglish{44a/msK}} वाङ्गं । को वा विष‚योप‚द‚र्श‚न‚स्योप‚योगो य‚दि ‚{\tiny $_{lb}$}‚ह्य‚नेन विना न साध्य‚सिद्धिः स्यात् । \quotelemma{स‚र्व्व}सोभेत\edtext{}{\lemma{सोभेत}\Bfootnote{? शोभेत}}याव‚ता विनाप्य‚नेन ‚{\tiny $_{lb}$}‚याव‚त् । [यः] क‚श्चित्कृत‚कः स स‚र्वोऽनित्यो य‚था कुम्भादिः ‚{\tiny $_{1}$}‚ श‚ब्द‚श्च कृत‚क इत्य‚नु‚{\tiny $_{lb}$}‚क्तेपि प‚क्ष‚श‚ब्दोऽनित्य इत्य‚र्थाङ्ग‚मात्र एव । त‚स्माद‚स्य निर्देशो निर‚र्थ‚क एव । स्याद‚यं ‚{\tiny $_{lb}$}‚विप‚र्यासो य‚दि हेतुव्यापार‚विष‚योप‚द‚र्श‚नाय प‚क्ष‚व‚द् ‚{\tiny $_{2}$}‚ व‚च‚न‚न्नैव प्र‚युज्य‚ते त‚दा क‚थ‚म्प‚क्ष‚{\tiny $_{lb}$}‚स‚माश्र‚य‚ल‚ब्ध‚व्य‚प‚देशा । प‚क्ष‚ध‚र्म‚त्वाद‚यः स‚म्प‚द्य‚न्ते । तेषु वा निश्रितात्म‚सुस‚म्भूत‚{\tiny $_{lb}$}‚साम‚र्थ्यात् प‚क्ष‚ग‚तिर‚स‚म्भाव्यैव । साम‚र्थ्य ‚{\tiny $_{3}$}‚ ल‚भ्य‚प‚क्ष‚ब‚लेन प‚क्ष‚ध‚र्म‚त्वाद‚यः स‚म्प‚द्य‚न्त ‚{\tiny $_{lb}$}‚इत्य‚प्य‚युक्तं । तेष्व‚स‚त्सु साम‚र्थ्य‚ल‚भ्य‚स्यैव प‚क्ष‚स्यास‚म्भ‚वात् । अन्योन्याश्र‚यं चेद‚म्प‚क्ष‚{\tiny $_{lb}$}‚ध‚र्म‚त्वादिसाम‚र्थ्या ‚{\tiny $_{4}$}‚ यात‚प‚क्ष‚व‚शेन प‚क्ष‚ध‚र्म‚त्वाद‚यः स‚म्प‚द्य‚न्ते । प‚क्ष‚ध‚र्म‚त्वादिब‚लेन ‚{\tiny $_{lb}$}‚च प‚क्ष इति । त‚द‚त्रोच्य‚ते । न ख‚लु साध‚न‚काले प‚क्ष‚ध‚र्म‚त्वादिविक‚ल्पोऽस्ति के ‚{\tiny $_{5}$}‚ व‚लं ‚{\tiny $_{lb}$}‚य‚त्रैव जिज्ञासित‚विशेषे ध‚र्मिणि श‚ब्दादौ तु च क‚रीशादिस्थ‚गित‚तेज‚सि वा कुण्डादौ ‚{\tiny $_{lb}$}‚यो ध‚र्मः कृत‚क‚त्व‚धूम‚त्वादिल‚क्ष‚णोनुमान‚तः प्र‚त्य‚क्ष‚तो ‚{\tiny $_{6}$}‚ वा प्र‚तीय‚ते । प्र‚त्याय्य‚ते ‚{\tiny $_{lb}$}‚वा । त‚द्विशेष‚योगित‚या वा निश्चितेऽप‚र‚स्मिन्घ‚ट‚म‚हान‚सादाव‚स्थित‚त्वेन स्म‚र्य‚ते ‚{\tiny $_{lb}$}‚त‚द्विशेष‚विर‚हिणि वा ग‚ग‚न‚साग‚रादौ नास्तित्वेनै ‚{\tiny $_{7}$}‚ व स्म‚र्य‚ते । स व‚स्तुध‚र्म‚त‚यैव ‚{\tiny $_{lb}$}‚विनापि प‚क्ष‚ध‚र्म‚त्वादिव्य‚प‚देशेन त‚त् ध‚र्मिणं जिज्ञासित‚ध‚र्म‚विशिष्टं साम‚र्थ्यादेव ‚{\tiny $_{lb}$}‚प्र‚तिपाद‚य‚ति । स चास्य साम‚र्थ्य‚विष‚यः प‚क्ष ‚{\tiny $_{8}$}‚ इति गीय‚ते । त‚तः प‚श्चात् त‚त्स‚माश्र‚य‚{\tiny $_{lb}$}‚भाविन्यो य‚थेष्ट‚प‚क्ष‚ध‚र्म‚त्वादिसंज्ञाः शास्त्रेषु संव्य‚व‚हारार्थ‚म्प्र‚त‚न्य‚न्ते । य‚दि वा प्र‚त्या‚{\tiny $_{lb}$}‚लोच‚न‚प्र‚क‚र‚ण‚ब‚लात् साध‚न‚का ‚{\tiny $_{9}$}‚ \leavevmode\ledsidenote{\textenglish{44b/msK}} लेपि भ‚व‚न्तु प‚क्ष‚ध‚र्म‚त्वादिविक‚ल्पाः । क‚थं योहि ‚{\tiny $_{lb}$}‚व‚स्तुनो ध‚र्मो वादिना विवा[दा]स्प‚दीभूत‚ध‚र्मिविशिष्ट‚त‚या साध‚यितुमिष्टः स प‚क्ष‚स्त‚स्य ‚{\tiny $_{lb}$}‚योन्यो ध‚र्मः स प‚क्ष‚ध‚र्मः । ‚{\tiny $_{1}$}‚ प्र‚कृत‚साध्य‚ध‚र्म‚सामान्येन च स‚मानोर्थः स‚प‚क्षः । त‚द्विर‚ही ‚{\tiny $_{lb}$}‚ \leavevmode\ledsidenote{\textenglish{63/s}} वास‚प‚क्ष इति । य‚स्यापि हि साध‚न‚काले प‚क्ष‚प्र‚योगोस्ति त‚स्यापि न वाद्य‚काण्ड‚मेव ‚{\tiny $_{lb}$}‚प‚क्षं जातेऽनि ‚{\tiny $_{2}$}‚ त्यः श‚ब्द इति । क‚स्तु प्र‚स्तावान्त‚रेना\edtext{}{\lemma{रेना}\Bfootnote{? रेणा}}पि प्र‚क‚र‚ण‚ब‚लेनैव ‚{\tiny $_{lb}$}‚प‚क्ष‚ध‚र्म‚त्वाद‚योपि व‚क्त‚व्या [ः] । त‚च्च प‚क्ष‚प्र‚योग‚दूष‚क‚स्यापि स‚मानं । त‚स्मा‚{\tiny $_{lb}$}‚त्प्र‚तिज्ञाव‚च‚नं न साध‚नां ‚{\tiny $_{3}$}‚ गं ।
	{\color{gray}{\rmlatinfont\textsuperscript{§~\theparCount}}}
	\pend% ending standard par
      ‚{\tiny $_{lb}$}‚

	  
	  \pstart \leavevmode% starting standard par
	\hphantom{.}उप‚न‚य‚निग‚म‚न‚व‚च‚न‚न्तु य‚था न साध‚नाङ्ग‚न्त‚थोच्य‚ते ॥ त‚त्र‚ताव‚दु दाह‚र‚णा‚{\tiny $_{lb}$}‚पेक्ष‚स्त‚थेत्युप‚संहारो न त‚थेति वा साध‚न‚स्योप‚न‚यः \href{http://sarit.indology.info/?cref=ns\%C5\%AB.1.2.38}{न्या० सू०  १।२।३८ } । य‚था ‚{\tiny $_{lb}$}‚त ‚{\tiny $_{4}$}‚ थेतिप्र‚तिबिम्ब‚नार्थं । किम्पुन‚र‚त्र प्र‚तिबिम्ब‚न‚न्दृष्टान्त‚ग‚त‚स्य ध‚र्म‚स्याव्य‚भिचार‚त्त्वे ‚{\tiny $_{lb}$}‚सिद्धे । तेन साध्य‚ग‚त‚स्य तुल्य‚ध‚र्म‚ता । एव‚ञ्चाय‚ङ्कृत‚क इति सा ‚{\tiny $_{5}$}‚ ध्येन स‚ह ‚{\tiny $_{lb}$}‚स‚म्भ‚व उप‚न‚यार्थः । न‚नु च कृत‚क‚त्वादित्य‚नेन स‚म्भ‚व उक्तः । नोक्तः । साध्य‚साध‚न‚{\tiny $_{lb}$}‚ध‚र्म‚मात्र‚निर्देशात् । साध्य‚साध‚न‚ध‚र्म‚मात्र‚निर्देशः कृ ‚{\tiny $_{6}$}‚ त‚क‚त्वाद‚नित्यः श‚ब्दो भ‚व‚ति । ‚{\tiny $_{lb}$}‚त‚त्पुनः श‚ब्दे कृत‚क‚त्त्व‚म‚स्ति । नास्तीत्युप‚न‚येन स‚म्भ‚वो ग‚म्य‚ते । अस्ति च श‚ब्दे कृत‚{\tiny $_{lb}$}‚क‚त्व‚मिति । त‚था च हेतुव‚च‚नाद् भिन्नार्थ‚प्र‚तिपाद ‚{\tiny $_{7}$}‚ क‚त्व‚मुप‚न‚य‚स्याभिन्न‚रूप‚त्वे प्र‚सिद्ध‚{\tiny $_{lb}$}‚प‚र्याय‚व्य‚तिरिक्त‚त्वे च स‚ति हेतुव‚च‚नोत्त‚र‚काल‚मुपादीय‚मान‚त्वात् दृष्टान्त‚व‚च‚न‚{\tiny $_{lb}$}‚व‚दिति श‚क्येत अनुमातुं ।
	{\color{gray}{\rmlatinfont\textsuperscript{§~\theparCount}}}
	\pend% ending standard par
      ‚{\tiny $_{lb}$}‚

	  
	  \pstart \leavevmode% starting standard par
	\hphantom{.}हेत्व‚प‚देशात् ‚{\tiny $_{8}$}‚ प्र‚तिज्ञायाः पून‚र्व‚च‚न‚न्निग‚म‚नं \href{http://sarit.indology.info/?cref=ns\%C5\%AB.1.1.39}{न्या० सू० १।१।३९ } । ‚{\tiny $_{lb}$}‚प्र‚तिज्ञायाः पुन‚र्व‚च‚न‚मिति प्र‚तिज्ञाविष‚य‚स्यार्थ‚स्याशेष‚प्र‚माणोप‚प‚त्तौ विप‚रीत‚{\tiny $_{lb}$}‚प्र‚स‚ङ्ग‚प्र‚तिषेधार्थं य‚त्पुन‚र‚भिधानं ‚{\tiny $_{9}$}‚ \leavevmode\ledsidenote{\textenglish{45a/msK}} त‚न्निग‚म‚नं । न पुनः प्र‚तिज्ञाया एव ‚{\tiny $_{lb}$}‚पुन‚र्व‚च‚नं । किङ्कार‚णं य‚स्मात्प्र‚तिज्ञासाध्य‚निर्देशः सिद्ध‚निर्देशो निग‚म‚न‚मिति । ‚{\tiny $_{lb}$}‚पुनः श‚ब्द‚श्च नानात्वे दृष्टः [।] पुन‚रिय‚म‚चिर‚प्र‚भा नि ‚{\tiny $_{1}$}‚ श्च‚र‚ति । पुन‚रिद‚ङ्ग‚न्ध‚{\tiny $_{lb}$}‚र्व्व‚न‚ग‚रं दृश्य‚त इति । अत्र च साम‚र्थ्यादुप‚न‚यान‚न्त‚र‚भावी हेत्व‚प‚देशो गृह्य‚ते । न ‚{\tiny $_{lb}$}‚प्र‚तिज्ञान‚न्त‚र‚भावी । अस‚म्भ‚वात् । न‚हि क‚श्चित्प्र‚तिज्ञा ‚{\tiny $_{2}$}‚ न‚न्त‚रं हेत्व‚प‚देशान्निग‚म‚नं ‚{\tiny $_{lb}$}‚प्र‚युंक्ते । अनित्यः श‚ब्दः कृत‚क‚त्वात् । त‚स्माद‚नित्यः श‚ब्द इति । अत‚श्च प्र‚तिज्ञार्थ‚{\tiny $_{lb}$}‚वाक्याद् भिन्नार्थं निग‚म‚न‚व‚च‚नं । प्र‚तिज्ञावाक्याद् भिन्न ‚{\tiny $_{3}$}‚ रूप‚त्वे स‚ति हेतुव‚च‚नोत्त‚र‚{\tiny $_{lb}$}‚काल‚म‚भिधीय‚मान‚त्वात् \quotelemma{दृष्टान्त‚व‚च‚न‚व‚त्} । न च साध्यार्थ‚प्र‚तिपाद‚क‚न्निग‚म‚नं । ‚{\tiny $_{lb}$}‚श‚ब्दान्त‚रोपात्त‚स्याव‚धार‚ण‚रूपेण प्र‚वृत्त‚त्वात् । योय ‚{\tiny $_{4}$}‚ माग‚च्छ‚त्य‚यं विषाणीति ‚{\tiny $_{lb}$}‚केन‚चिदुक्ते त‚स्माद‚न‚श्व इत्यादिव‚च‚न‚व‚त् । त‚स्माच्छ‚ब्द‚स‚हितं वाक्य‚म्विचार‚{\tiny $_{lb}$}‚विष‚याय प्र‚साध्यार्थ‚प्र‚तिपाद‚क‚न्न भ‚व‚ति । का ‚{\tiny $_{5}$}‚ र‚णोप‚देशोत्त‚र‚काल‚मुपात्त‚त्त्वात् । ‚{\tiny $_{lb}$}‚दृष्टान्तः पूर्व‚व‚त् । त‚देत‚त् प्र‚तिषिध्य‚ते न ख‚ल्वेवं प्र‚योगः क्रिय‚ते । अनित्यः श‚ब्दः ‚{\tiny $_{lb}$}‚कृत‚क‚त्वात् । प्र‚तिज्ञाप्र‚योग ‚{\tiny $_{6}$}‚ स्यान‚न्त‚रं निराकृत‚त्वात् । अपि तु कृत‚कः श‚ब्दः । ‚{\tiny $_{lb}$}‚प‚श्चैवं स स‚र्वोऽनित्यो य‚था क‚ल‚शादिः । यो वा कृत‚कः स स‚र्व्वोऽनित्यो य‚था घ‚टादिः । ‚{\tiny $_{lb}$}‚ \leavevmode\ledsidenote{\textenglish{64/s}} त‚था च कृत‚कः श‚ब्द इ ‚{\tiny $_{7}$}‚ त्येव‚मुभ‚य‚था य‚थेष्टं प्र‚योग[ः] क्रिय‚ते । साध्य‚सिद्धेरुभ‚{\tiny $_{lb}$}‚य‚थापि भावात् । त‚त्र य‚दि कृत‚कः श‚ब्दो य‚श्चैवं स स‚र्व्वोऽनित्यो य‚था घ‚टादि‚{\tiny $_{lb}$}‚रित्य‚भिधाय त‚था कृत ‚{\tiny $_{8}$}‚ कः श‚ब्द इति प्र‚तिबिंब‚नार्थ ‚{\tiny $_{9}$}‚ \leavevmode\ledsidenote{\textenglish{45b/msK}} मुप‚न‚य‚व‚च‚न‚मुच्य‚ते । त‚दे\edtext{}{\lemma{दे}\Bfootnote{? दि}}‚{\tiny $_{lb}$}‚द‚म‚न‚र्थ‚कं ‚{\tiny $_{1}$}‚ विनाप्य‚नेन प्र‚तिबिंब‚नेनान‚न्त‚रोक्त‚प्र‚योग‚मात्रात् प्र‚तीतिभावात् । ‚{\tiny $_{lb}$}‚साध‚न‚ञ्च य‚द‚न‚र्थ‚कं न त‚त्साध‚न‚वाक्ये विद्व‚द्भिरुपादेयं । त‚द्य‚था द‚श‚दाडिमादि ‚{\tiny $_{lb}$}‚वाक्यं त‚था ‚{\tiny $_{2}$}‚ चान‚र्थ‚कं प्र‚तिबिंब‚नार्थ‚मुप‚न‚य‚व‚च‚न‚मिति व्याप‚क‚विरुद्धोप‚ल‚ब्धेः । ‚{\tiny $_{lb}$}‚स्वार्थानुमिताव‚प्य‚य‚मेव न्यायो दृष्टो न‚हि क‚श्चित्स‚चेत‚नः कृत‚क‚त्त्व‚स्य भावं श‚ब्दे ‚{\tiny $_{lb}$}‚गृही ‚{\tiny $_{3}$}‚ त्वा त‚स्य चाविनाभावित्व‚म‚नुस्मृत्य त‚था च कृत‚कः श‚ब्द इति प्र‚तिबिम्ब‚नार्थ‚{\tiny $_{lb}$}‚क‚रोति । अथापि यः कृत‚कः स स‚र्व्वोऽनित्यो य‚था घ‚टः । त‚था च कृत‚कः श‚ब्द ‚{\tiny $_{lb}$}‚इ ‚{\tiny $_{4}$}‚ ति स‚म्भ‚व‚प्र‚द‚र्श‚नार्थ‚मुप‚न‚य‚व‚च‚न‚मुच्य‚ते । त‚देत‚द् द्व‚य‚म‚प्य‚ङ्गीकुर्मः । प्र‚तिज्ञा‚{\tiny $_{lb}$}‚न‚न्त‚र‚भाविन‚स्तु साध‚न‚मात्र‚निर्देश‚म‚नित्यः श‚ब्दः कृत‚क‚त्वादित्येव ‚{\tiny $_{5}$}‚ न प्र‚तिप‚द्याम‚हे । ‚{\tiny $_{lb}$}‚प्र‚तिज्ञायाः प्र‚योगाभावात् । त‚त‚श्चोप‚न‚य‚स्याव‚य‚वान्त‚र‚त्व‚प्र‚तिपाद‚नायोक्तो ‚{\tiny $_{lb}$}‚हेतुर‚सिद्ध‚तोर‚ग‚द‚ष्ट‚त्वाङ्ग‚ताव‚श‚क्त एव । य‚त् ‚{\tiny $_{6}$}‚ पुन‚रिदं सिद्धार्थ‚निर्देश‚ल‚क्ष‚णं निग‚{\tiny $_{lb}$}‚म‚नं पौन‚रुक्त्य‚प‚रिहाराय व‚र्ण्य‚ते त‚न्नैवोप‚प‚द्य‚ते विना निग‚म‚नेनार्थ‚सिद्धेरेव प‚ञ्चा‚{\tiny $_{lb}$}‚व‚य‚व‚साध‚न‚वादिनोऽनुप‚प‚त्तेः ‚{\tiny $_{7}$}‚ अन्य‚था निग‚म‚नात् प्रागेवार्थ‚स्य सिद्ध‚त्वात् व्य‚र्थ‚त‚या ‚{\tiny $_{lb}$}‚न साध‚नाङ्ग‚न्निग‚म‚न‚म्प्राप्नोति । त‚त‚श्च नेद‚मुपादेयं साध‚न‚वाक्ये सिद्ध‚मित्य‚{\tiny $_{lb}$}‚प्र‚तिज्ञा । भ‚वेद्व्यामोहो विप्र ‚{\tiny $_{8}$}‚ तिप‚न्न‚स्य प्र‚माणान्त‚र‚व्य‚पेक्षा नास्तीति सिद्ध‚म‚नित्य‚त्व‚{\tiny $_{lb}$}‚मुच्य‚ते । निग‚म‚नं तु प्र‚तिविष‚य‚स्यार्थ‚स्याशेष‚प्र‚माणोप‚प‚त्ताव‚शेषाव‚य‚व‚प‚राम‚र्शेनाव‚{\tiny $_{lb}$}‚धार‚णार्थ‚म ‚{\tiny $_{9}$}‚ \leavevmode\ledsidenote{\textenglish{46a/msK}} नित्य एवेति प्र‚व‚र्त्त‚त इति । य‚दि त‚र्हि प्र‚माणान्त‚र‚व्य‚पेक्षा नास्ति त‚त्त‚र्हि ‚{\tiny $_{lb}$}‚[साध्यं] साम‚र्थ्याद‚व‚धार्य‚त एव । त‚थाहि य‚द‚कृत‚क‚न्त‚द‚नित्य‚मेव । य‚था कुण्डादि‚{\tiny $_{lb}$}‚श‚ब्द‚श्च कृत‚क ‚{\tiny $_{1}$}‚ इत्येव‚म‚नि[त्य]त्वाविनाभाविनः कृत‚क‚त्व‚स्य श‚ब्दे भाव‚ख्यातौ ‚{\tiny $_{lb}$}‚त‚त्साम‚र्थ्यादेवानित्यः श‚ब्द इति निश्च‚यो भ‚व‚ति [।] त‚द‚स्य व‚च‚नं साम‚र्थ्यं प्र‚तीता‚{\tiny $_{lb}$}‚र्थ‚प्र‚त्याय‚क‚त्वात् पुन‚रुक्त‚म‚नु ‚{\tiny $_{2}$}‚ पादानार्ह‚ञ्च । न चात्र विप‚र्य‚य‚प्र‚स‚ङ्ग‚स्य लेशोप्या‚{\tiny $_{lb}$}‚श‚ङ्क्य‚ते । येन त‚द्व्य‚व‚च्छेदाय स‚फ‚ल‚मेत‚स्योपादानं स्यात् । अनित्य‚त्वेनैव कृत‚क‚त्व‚स्य ‚{\tiny $_{lb}$}‚व्याप्तिप्र‚साध‚नात् । प्र‚योग‚स्तु ‚{\tiny $_{3}$}‚ [।] य‚त्साम‚र्थ्यात् प्र‚तीय‚ते न त‚स्य व‚च‚न‚म्प्रेक्षाव‚ता ‚{\tiny $_{lb}$}‚क‚र्त्त‚व्यं । त‚द्व‚च‚न‚म्पुन‚रुक्त‚म्वा त‚द्य‚था गेहे नास्ति कुमारो जीव‚ति चेत्येत‚त्साम‚र्थ्यात् ‚{\tiny $_{lb}$}‚प्र‚तीयंमान‚स्य त‚द्व‚हि ‚{\tiny $_{4}$}‚ र्भाव‚स्य व‚च‚नं । प‚क्ष‚ध‚र्मान्व‚य‚व्य‚तिरेक‚त‚द्व‚च‚न‚साम‚र्थ्याच्च ‚{\tiny $_{lb}$}‚प्र‚तीय‚ते त‚स्माद‚नित्य एवेत्येव‚मादिना पुनः सिसाध‚यिषितोर्थः प्र‚थ‚म‚साध्यापेक्ष‚या ‚{\tiny $_{5}$}‚ ‚{\tiny $_{lb}$}‚व्याप‚क‚विरुद्धोप‚ल‚ब्धिर्द्वितीय‚साध्यापेक्ष‚या च स्व‚भाव‚हेतुः । अत एव निग‚म‚न‚स्या‚{\tiny $_{lb}$}‚व‚य‚वान्त‚र‚त्व‚प्र‚तिपाद‚नायोक्ता हेत‚वोऽसिद्धाः । त‚द‚प्येतेनैव प्र ‚{\tiny $_{6}$}‚ त्युक्तं । य‚दाह ।
	{\color{gray}{\rmlatinfont\textsuperscript{§~\theparCount}}}
	\pend% ending standard par
      ‚{\tiny $_{lb}$}‚
	  \bigskip
	  \begingroup
	
	    
	    \stanza[\smallbreak]
	  \flagstanza{\tiny\textenglish{...17}}{\normalfontlatin\large ``\qquad}प्र‚त्य‚येक्ष \edtext{}{\lemma{येक्ष}\Bfootnote{?}} प्र‚तिज्ञादीन्वाक्यार्थ‚प्र‚तिप‚त्त‚ये ।&‚{\tiny $_{lb}$}‚\leavevmode\ledsidenote{\textenglish{65/s}}प्रोच्य‚मान‚न्निग‚म‚नं पुन‚रुक्त‚न्न जाय‚ते ॥ [१७]{\normalfontlatin\large\qquad{}"}\&[\smallbreak]
	  
	  
	  
	  \endgroup
	
	  \bigskip
	  \begingroup
	
	    
	    \stanza[\smallbreak]
	  \flagstanza{\tiny\textenglish{...18}}{\normalfontlatin\large ``\qquad}विप्र‚कीर्णौश्च व‚च‚नैर्नैकोर्थः प्र‚तिपाद्य‚ते ।&‚{\tiny $_{lb}$}‚तेन स‚म्ब‚न्ध‚सि ‚{\tiny $_{7}$}‚ ध्य‚र्थ‚म्वाच्य‚न्निग‚म‚नं पृथ‚ग् [॥ १८]{\normalfontlatin\large\qquad{}"}\&[\smallbreak]
	  
	  
	  
	  \endgroup
	‚{\tiny $_{lb}$}‚

	  
	  \pstart \leavevmode% starting standard par
	इत्य‚ल‚म‚तिप्र‚सारिण्या क‚थ‚या ॥ ० ॥
	{\color{gray}{\rmlatinfont\textsuperscript{§~\theparCount}}}
	\pend% ending standard par
      ‚{\tiny $_{lb}$}‚

	  
	  \pstart \leavevmode% starting standard par
	अन्व‚य‚व्य‚तिरेक‚योर्वेति प‚र्य्यायान्त‚र‚क‚थ‚न‚मुपादान‚मिति \href{http://sarit.indology.info/?cref=ns\%C5\%AB.2.1.12}{न्या० सू० २।१।१२ } ‚{\tiny $_{lb}$}‚व‚र्त‚ते द्वितीय‚स्यासाम‚र्थ्यं जातायाः ‚{\tiny $_{8}$}‚ सिद्धेः पुन‚र‚ज‚न्य‚त्वात् । [प्र‚माण-] स‚मुच्च‚य ‚{\tiny $_{lb}$}‚ टीकाकारास्त्वाहुः न‚न्वि \cite[6b4]{vn-msN} त्यादि । ने \cite[6b4]{vn-msN} त्याद्युत्त‚रः । य‚दि चेत्युप‚च‚य‚{\tiny $_{lb}$}‚हेतुः । साध‚नाव‚य‚वः प्र‚तिज्ञां प्राप्नोति निय‚मेन साध्य‚प्र‚तीतिनि ‚{\tiny $_{9}$}‚ \leavevmode\ledsidenote{\textenglish{46b/msK}} मित्त‚त्वात् ‚{\tiny $_{lb}$}‚प‚क्ष‚ध‚र्मादिव‚च‚न‚व‚त् । स‚न्दिग्ध‚व्य‚तिरेको हेतुरिति चेदाह । \quotelemma{न‚हि प‚क्ष‚ध‚र्म‚व‚च‚न‚स्या‚{\tiny $_{lb}$}‚पीति} \cite[6b6]{vn-msN} । त‚त्तुल्य‚मिति विरुद्धानैकान्तिक‚योः प‚क्ष‚ध‚र्म‚स‚द्भावेप्य‚ग‚म‚क‚त्वात् ‚{\tiny $_{1}$}‚ । ‚{\tiny $_{lb}$}‚त‚त एव‚संस‚यो\edtext{}{\lemma{यो}\Bfootnote{? संश‚यो}}त्प‚त्तेः प‚क्ष‚ध‚र्म‚व‚च‚न‚न्न साध‚नं साधार‚णादिव‚च‚न‚व‚दिति ‚{\tiny $_{lb}$}‚चेदाह । \quotelemma{ऐतेन} \cite[6b7]{vn-msN} त‚त्तुल्य‚मित्यादिना संश‚योत्प‚त्तिः प्र‚त्युक्तेति । एत‚देव व्य‚न‚क्ति ‚{\tiny $_{lb}$}‚प‚क्ष‚ध‚र्म ‚{\tiny $_{2}$}‚ व‚च‚नाद‚पीति । त‚द‚नेनान‚न्त‚र‚स्य हेतोर्व्य‚भिचार‚ङ्क‚थ‚य‚ति ।
	{\color{gray}{\rmlatinfont\textsuperscript{§~\theparCount}}}
	\pend% ending standard par
      ‚{\tiny $_{lb}$}‚

	  
	  \pstart \leavevmode% starting standard par
	न‚नु च प‚क्ष‚ध‚र्म‚स्य श्राव‚ण‚त्वादेर‚प्र‚द‚र्शिते स‚म्ब‚न्धेनैव साध‚नाव‚य‚व‚त्व‚म‚तो विप‚क्ष‚{\tiny $_{lb}$}‚त्वाभावान्न व्य‚भि ‚{\tiny $_{3}$}‚ चारः । प्र‚द‚र्शिते तु स‚म्ब‚न्धे साध‚नाव‚य‚व‚त्वं त‚दा च त‚स्मात् संश‚यो ‚{\tiny $_{lb}$}‚नास्तीति सुत‚रान्नानेकान्त इति ॥ एवं म‚न्य‚ते । \quotelemma{प‚क्ष‚व‚च‚नेपि तुल्य‚मे} \cite[6b6]{vn-msN} त‚दिति ‚{\tiny $_{lb}$}‚त‚द‚पि सा ‚{\tiny $_{4}$}‚ ध‚नं स्यात् । अथ प्र‚तिप‚द्येथा स‚त्यं स्याद्य‚दि साध्यं स्यान्न चास्त्य‚न्य‚तः ‚{\tiny $_{lb}$}‚ \leavevmode\ledsidenote{\textenglish{66/s}} साध्य‚सिद्धः । न च निष्पादित‚क्रिये दारुणि दात्राद‚यः क‚ञ्च‚नार्थं पुष्य‚न्ति । अप्र‚द‚र्शिते ‚{\tiny $_{lb}$}‚तु संब ‚{\tiny $_{5}$}‚ न्धे संश‚योत्प‚त्तिहेतुत्वादिद‚मुक्त‚न्त‚त एव‚संस‚यो\edtext{}{\lemma{यो}\Bfootnote{? संश‚यो}}त्प‚त्तेरिति । ‚{\tiny $_{lb}$}‚य‚द्येवं न त‚र्हि त‚त्प्र‚योग‚म‚न्त‚रेण साध्य‚सिद्धेर‚भाव इति व्य‚र्थ एव त‚त्प्र‚योगः स्यात् ‚{\tiny $_{6}$}‚ ‚{\tiny $_{lb}$}‚अन्य‚था कः प‚क्ष‚व‚च‚नं साध‚नाद‚पाक‚र्त्तुं स‚म‚र्थः । त‚त‚श्च \quotelemma{त्रिरूप‚लिङ्गा} ख्यानं \quotelemma{प‚रार्थ} ‚{\tiny $_{lb}$}‚म‚नुमान‚मित्याद्याचार्य‚व‚चो व्याह‚न्येत । क‚थं त‚र्ह्युक्तं ।
	{\color{gray}{\rmlatinfont\textsuperscript{§~\theparCount}}}
	\pend% ending standard par
      ‚{\tiny $_{lb}$}‚

	  
	  \pstart \leavevmode% starting standard par
	प‚क्ष‚ध‚र्म‚त्व‚स‚म्ब‚न्ध‚साध्योक्तेर‚न्य‚व‚र्ज‚न‚मिति नास्ति विरोधः । प‚क्ष‚ध‚र्म‚त्व‚संब‚न्धा‚{\tiny $_{lb}$}‚भ्यां साध्य‚स्योक्तिप्र‚कास\edtext{}{\lemma{कास}\Bfootnote{? प्र‚काश}}न‚माक्षेप‚स्त‚स्माद‚न्येषां प‚क्षोप‚न‚य‚व‚च‚नादीना‚{\tiny $_{lb}$}‚मुपादेय‚त्वेन साध‚न‚वाक्य‚व‚र्ज‚न‚मि ‚{\tiny $_{8}$}‚ \leavevmode\ledsidenote{\textenglish{47a/msK}} ति व्याख्यानात् । विव‚र‚णेप्य‚य‚म‚र्थो य‚स्मात् ‚{\tiny $_{lb}$}‚प‚क्ष‚ध‚र्म‚त्व‚स‚म्ब‚न्ध‚व‚च‚न‚मेवान्व‚य‚व्य‚तिरेकाभ्याम्विव‚क्षितार्थ‚सिद्धिकार‚णं युक्तं नान्य‚त् । ‚{\tiny $_{lb}$}‚त‚स्माद‚नुमेय‚स्योप‚द‚र्श‚नार्थ ‚{\tiny $_{9}$}‚ सिद्ध्य‚र्थं प‚क्ष‚व‚च‚न‚मुपादेयं नान्य‚दित्युप‚स्कारः । प‚क्ष ‚{\tiny $_{lb}$}‚उच्य‚ते आक्षिप्य‚ते प्र‚काश्य‚ते अनेनेति प‚क्ष‚व‚च‚न‚न्त्रिरूपं लिङ्गं । आक्षेपो ह्य‚भि‚{\tiny $_{lb}$}‚धान‚तुल्य इति व‚च‚न‚मित्युक्तं व‚चेर ‚{\tiny $_{1}$}‚ नेकार्थ‚त्वाद्वा । अस्माकं तु [।]
	{\color{gray}{\rmlatinfont\textsuperscript{§~\theparCount}}}
	\pend% ending standard par
      ‚{\tiny $_{lb}$}‚
	  \bigskip
	  \begingroup
	
	    
	    \stanza[\smallbreak]
	  \flagstanza{\tiny\textenglish{...19}}{\normalfontlatin\large ``\qquad}त‚त्रानुमेय‚निर्देशो हेत्व‚र्थ‚विष‚यो म‚त [१९]{\normalfontlatin\large\qquad{}"}\&[\smallbreak]
	  
	  
	  
	  \endgroup
	‚{\tiny $_{lb}$}‚

	  
	  \pstart \leavevmode% starting standard par
	इत्य‚पि व‚च‚नं विरुध्य‚ते । य‚स्मा
	{\color{gray}{\rmlatinfont\textsuperscript{§~\theparCount}}}
	\pend% ending standard par
      ‚{\tiny $_{lb}$}‚
	  \bigskip
	  \begingroup
	
	    
	    \stanza[\smallbreak]
	  \flagstanza{\tiny\textenglish{...20}}{\normalfontlatin\large ``\qquad}त‚त्रेति त‚र्क‚शास्त्र‚स्य स‚म्ब‚न्धोत्राभिधीय‚ते ।&‚{\tiny $_{lb}$}‚प्र‚योग‚स्य तु स‚म्ब‚न्धे ब‚हु स्याद‚स‚मं ‚{\tiny $_{2}$}‚ ज‚सं ॥ [२०]{\normalfontlatin\large\qquad{}"}\&[\smallbreak]
	  
	  
	  
	  \endgroup
	‚{\tiny $_{lb}$}‚
	  \bigskip
	  \begingroup
	
	    
	    \stanza[\smallbreak]
	  \flagstanza{\tiny\textenglish{...21}}{\normalfontlatin\large ``\qquad}त‚स्यैव प्र‚कृतेरुक्त‚मेत‚च्चास्यैव ल‚क्ष‚णे ।&‚{\tiny $_{lb}$}‚प‚र‚विप्र‚तिप‚त्तीनान्निषेधाय विशेष‚त [॥ २१]{\normalfontlatin\large\qquad{}"}\&[\smallbreak]
	  
	  
	  
	  \endgroup
	‚{\tiny $_{lb}$}‚

	  
	  \pstart \leavevmode% starting standard par
	इत्य‚लं प्र‚स‚ङ्गेन ॥ ० ॥
	{\color{gray}{\rmlatinfont\textsuperscript{§~\theparCount}}}
	\pend% ending standard par
      ‚{\tiny $_{lb}$}‚

	  
	  \pstart \leavevmode% starting standard par
	त‚द्भाव‚रूपं साध‚न‚म‚ङ्ग‚न्ध‚र्मो विष‚यित्वेन । य‚स्यार्थ‚स्य ‚{\tiny $_{3}$}‚ प्र‚स्तुत‚स्य स साध‚{\tiny $_{lb}$}‚ \leavevmode\ledsidenote{\textenglish{67/s}} नाङ्ग‚स्त‚स्यैवाभिव्य‚क्तिरुत्त‚रेण प‚द‚द्व‚येन ॥ अजिज्ञासितं प्र‚तिवादिनाऽशास्त्राश्र‚य‚{\tiny $_{lb}$}‚व्याजादिभिरित्यादिप‚देनास‚म्ब‚द्ध‚प्र‚स‚ङ्ग‚प‚रिग्र‚हः । प्र‚क्षे ‚{\tiny $_{4}$}‚ पो नाम‚मात्रेण घोष‚णं ‚{\tiny $_{lb}$}‚विस्त‚रेण । य‚था बुद्धीन्द्रिय‚देह‚क‚लाप‚व्य‚तिरेकात्मास्ति नास्तीत्येताव‚त् मात्रे ‚{\tiny $_{lb}$}‚वुभुस्तिते नैयायिकाः प्र‚माण‚य‚न्ति । स‚दाद्य‚विशेष‚वि ‚{\tiny $_{5}$}‚ ष‚या विष‚य‚ज्ञेय‚विष‚या म‚दीयाः ‚{\tiny $_{lb}$}‚प्र‚त्य‚क्षाद‚यः प्र‚त्य‚या म‚दीय‚श‚रीरादिव्य‚तिरिक्त‚स‚म्वेद‚क‚स‚म्वेद्याः स्व‚कार‚णाय‚त्त‚{\tiny $_{lb}$}‚ज‚न्म‚व‚त्वादिभ्यः पुरुषान्त‚र ‚{\tiny $_{6}$}‚ प्र‚त्य‚य‚व‚दिति त‚तः स‚द‚नित्य‚न्द्र‚व्य‚व‚त् कार्य‚कार‚णं ‚{\tiny $_{lb}$}‚सामान्य‚विशेष‚व‚दिति द्र‚व्य‚गुण‚क‚र्म‚णाम‚विशेष इति म‚ह‚ता व्यासेन स‚दाद्य‚विशेषाद् ‚{\tiny $_{lb}$}‚व्याच‚क्ष‚ते । न‚ह्य‚त्र स ‚{\tiny $_{7}$}‚ दाद्य‚विशेष‚विष‚या विष‚य‚ज्ञेय‚विष‚य‚त्व‚न्ध‚र्म‚विशेष‚णं क‚थं‚{\tiny $_{lb}$}‚चिद‚पि प्र‚कृत‚साध्य‚सिध्युप‚कारि । प‚र‚व्यामोह‚नानुभाष‚ण‚श‚क्तिविघातादिहेतो‚{\tiny $_{lb}$}‚रित्य‚त्रादिश ‚{\tiny $_{8}$}‚ ब्देनोत्त‚र‚प्र‚तिप‚त्तिश‚क्तिविघात‚हेतोः प‚रिग्र‚हः क्रिय‚माणः प्र‚स‚ङ्गो‚{\tiny $_{lb}$}‚ \leavevmode\ledsidenote{\textenglish{68/s}} य‚स्येति विग्र‚हः । नैरात्म्य‚वाद्युदाह‚र‚णेन किं ज्ञाप‚य‚ति । \quotelemma{य‚त्र नाम} विहित‚प्र‚तिसिद्धो\edtext{}{\lemma{तिसिद्धो}\Bfootnote{? प्र‚तिषिद्धो }}वा ‚{\tiny $_{9}$}‚ \leavevmode\ledsidenote{\textenglish{47b/msK}} दिदोष‚गुण‚सौग‚त‚ध‚र्म‚विन‚य‚स्याप्य‚ह‚ङ्कार‚निमित्त‚स‚क‚लोद्ध ‚{\tiny $_{lb}$}‚वादिम‚ल‚क्षाल‚नायोद्य‚त‚म‚त्रैव नात्म‚वादिन‚स्त‚त्साध‚ने नृत्य‚गीतादेः प्र‚स‚ङ्गः । ‚{\tiny $_{lb}$}‚त‚त्रान्येषाम‚न्य‚स्य च का ग ‚{\tiny $_{1}$}‚ ण‚ना । न‚नु च व‚यं \quotelemma{बौद्धा} ब्रूम इति क‚थं याव‚ता ‚{\tiny $_{lb}$}‚स‚विशेष‚ण‚स्य प्र‚तिषेधाभिधानात् । अह‚म्बौद्धो ब्र‚वीमीति भ‚वित‚व्यं । य‚थाहंगार्गो\edtext{}{\lemma{थाहंगार्गो}\Bfootnote{‚{\tiny $_{lb}$}‚? गार्ग्यो}}ब्र‚वीम्य‚हं प‚टु ब्र‚वीमि इति न च ब‚हु ‚{\tiny $_{2}$}‚ ष्वेवेत‚द्व‚हुव‚च‚न‚मिति\href{http://sarit.indology.info/?cref=P\%C4\%81.2.4.21}{पाणिनिः  २।४।२१} श‚क्य‚म‚भिधातुँ क‚श्चिदिति व‚च‚नात् । नैव य‚स्माद‚सावात्म‚नि प‚रान् ‚{\tiny $_{lb}$}‚स्व‚यूथ्यान‚प्य‚न्यान्ब‚हून‚पेक्ष्य त‚था प्र‚युक्त‚वान् । ईदृश्यामेव च वादिनो विव‚क्षा ‚{\tiny $_{3}$}‚ यामिद‚{\tiny $_{lb}$}‚मुक्त‚मुदाह‚र‚णं नान्य‚स्यामिति प्र‚तिप‚त्त‚व्यं [।] अथ‚वा ज‚ड‚शाब्दिकाभिनिवेश‚निवा‚{\tiny $_{lb}$}‚र‚णायेद‚मेव‚मुक्तं त‚था च व्य‚र्थ‚ता श‚ब्दानुसास‚न\edtext{}{\lemma{न}\Bfootnote{? नुशास‚न}}स्य प्र‚तिपाद‚यि ‚{\tiny $_{4}$}‚ ष्य‚ति । ‚{\tiny $_{lb}$}‚अत एवान्येन म‚हार‚थेनापीदं प्र‚युक्तं ॥ ‚{\tiny $_{lb}$}‚ 
	    \pend% close preceding par
	  
	    
	    \stanza[\smallbreak]
	  \flagstanza{\tiny\textenglish{...22}}{\normalfontlatin\large ``\qquad}त्वं राजा व‚य‚म‚प्युपासित‚गुरुप्र‚ज्ञाभिमानोन्न‚ता । [२२]{\normalfontlatin\large\qquad{}"}\&[\smallbreak]
	  
	  
	  
	    \pstart  \leavevmode% new par for following
	    \hphantom{.}
	  इति ।
	{\color{gray}{\rmlatinfont\textsuperscript{§~\theparCount}}}
	\pend% ending standard par
      ‚{\tiny $_{lb}$}‚

	  
	  \pstart \leavevmode% starting standard par
	\hphantom{.}स‚भ्यः साधुसंम‚तानामित्युप‚ह‚स‚ति । \quotelemma{अहो} ‚{\tiny $_{5}$}‚ श‚ब्द‚श्चेहाध्याह्रिय‚ते । द्वाद‚शा‚{\tiny $_{lb}$}‚नाम्प्र‚माणादिल‚क्ष‚णानां यः प्र‚प‚ञ्चो विस्त‚र‚स्त‚स्य प्र‚काश‚नाय य‚च्छास्त्रं मीमांसा‚{\tiny $_{lb}$}‚ख्यं त‚स्य प्र‚णेता स चासौ जैमिनिश्च तेन प्र ‚{\tiny $_{6}$}‚ तिज्ञातं य‚त्त‚त्वं नित्य‚ताभिधानं । ‚{\tiny $_{lb}$}‚ \leavevmode\ledsidenote{\textenglish{69/s}} त‚स्याधिक‚र‚णं यः श‚ब्दः स च घ‚ट‚श्च त‚योर‚न्य‚त‚र‚स्तेन स द्वितीयो घ‚ट इतीत्थं ‚{\tiny $_{lb}$}‚प्र‚तिज्ञामुप‚र‚च‚य्य द्वाद‚श‚ल‚क्ष‚णादिव्याख्यान‚ङ्क‚रो ‚{\tiny $_{7}$}‚ ति । प्र‚माण‚ल‚क्ष‚ण‚मेव ताव‚देकं ‚{\tiny $_{lb}$}‚म‚ह‚ता कालेन व्याच‚ष्टे । चोद‚नाल‚क्ष‚णो ध‚र्म \href{http://sarit.indology.info/?cref=M\%C4\%ABS\%C5\%AB.1.1.2}{मीमांसा सू० १।१।२} श्चोद‚{\tiny $_{lb}$}‚नेति क्रियायाः प्र‚व‚र्त‚क‚म्व‚च‚न‚माहुश् \quotelemma{चोद‚ना हि भूतं भ‚व‚न्तं भ‚विष्य‚न्तं सूक्ष्मं व्य‚व ‚{\tiny $_{lb}$}‚हि ‚{\tiny $_{8}$}‚ \leavevmode\ledsidenote{\textenglish{48a/msK}} तं विप्र‚कृष्ट{...} म‚र्थं श‚क्नोत्य‚व‚ग‚म‚यितुं नान्य‚त् किञ्च‚नेन्द्रियं}\edtext{}{\lemma{माहुश्}\Bfootnote{मीमांसाश‚ब‚र‚भाष्ये १।१।२}} [।] त‚थाहि ‚{\tiny $_{lb}$}‚ [स‚त्] संप्र‚योगे पुरुष‚स्येन्द्रियाणाम्बुद्धिज‚न्म त‚त्प्र‚त्य‚क्षं । अनिमित्तं विद्य‚मानोप‚{\tiny $_{lb}$}‚ल‚म्भ‚न‚त्वादि \href{http://sarit.indology.info/?cref=M\%C4\%ABS\%C5\%AB.1.1.4}{मीमांसा सू० १।१।४ } त्यादिना । संस्कार‚दुःख‚तासिद्धिम‚न्त‚रेण ‚{\tiny $_{lb}$}‚नानित्य‚तासिद्धिर‚प्र‚तीत्य स‚मुत्प‚न्न‚स्य क्ष‚णिक‚त्वायोगात् । स त‚र्हि तादृशो ध‚र्मः ‚{\tiny $_{lb}$}‚पृथ‚ग्वाच्यो नेत्याह । \quotelemma{त‚थाविध‚स्त्वित्या} ‚{\tiny $_{1}$}‚ दि । एवंविध‚स्यापि प्र‚स्तुत‚साध्य‚ध‚र्म‚{\tiny $_{lb}$}‚नान्त‚रीय‚क‚स्य प्र‚तिवादिनाऽजिज्ञासित‚स्य त‚द्व्य‚तिरेकेण प्र‚तिज्ञायाम‚न्य‚त्र चाहेतु‚{\tiny $_{lb}$}‚दृष्टान्त‚योः क‚दा पुन‚रेत‚द‚साध‚ना ‚{\tiny $_{2}$}‚ ङ्ग‚व‚च‚नं य‚थोक्तं निग्र‚ह‚स्थान‚मित्याह । \quotelemma{प्र‚तिवादिना} ‚{\tiny $_{lb}$}‚त‚थाभावेऽसाध‚नाङ्ग‚त्वे प्र‚तिपादिते स‚ति । य‚दा तु न प्र‚तिपाद‚य‚ति त‚दा द्व‚योरेक‚स्यापि ‚{\tiny $_{lb}$}‚न ज‚य‚प‚राज‚यौ भ‚व‚त[ः] ‚{\tiny $_{3}$}‚ कुतः साध‚नान‚भिधानान्न वादिनो ज‚यः । प्र‚तिवादिना ‚{\tiny $_{lb}$}‚त‚थाभाव‚स्याप्र‚तिपादित‚त्वाच्च प‚राज‚योपि नास्त्येव । त‚स्य प्र‚तिप‚न्नापेक्ष‚त्वात् । ‚{\tiny $_{lb}$}‚अत एव प्र ‚{\tiny $_{4}$}‚ तिवादिन्य‚पि त‚योर‚भावः ॥ ४ ॥
	{\color{gray}{\rmlatinfont\textsuperscript{§~\theparCount}}}
	\pend% ending standard par
      ‚{\tiny $_{lb}$}‚

	  
	  \pstart \leavevmode% starting standard par
	\hphantom{.}स‚म्प्र‚ति प्र‚तिवादिनो निग्र‚ह‚स्थान‚म‚धिकृत्याह । \quotelemma{अदोषोदूभाव‚न‚मित्यादि} \cite[7b6]{vn-msN} । ‚{\tiny $_{lb}$}‚य‚त्र विष‚ये जिज्ञासिते अजिज्ञासिते ‚{\tiny $_{5}$}‚ पुन‚र्दोष‚स्यानुद्भाव‚नेपि नाप‚राध इत्य‚भिप्रायः । ‚{\tiny $_{lb}$}‚के पुन‚स्ते साध‚न‚स्य दोषा इत्याह । न्यून‚त्वं ष‚ट्प्र‚कार‚मेकैक‚द्विद्विरूपानुक्तौ [।] ‚{\tiny $_{lb}$}‚ स्यान्म‚तिः स‚प‚क्ष ‚{\tiny $_{6}$}‚ विप‚क्ष‚योः स‚द‚स‚त्त्व‚योर्यौग‚प‚द्येनाप्र‚योगे क‚थ‚ञ्च प्र‚कारात् न्यून‚तोच्य‚ते ‚{\tiny $_{lb}$}‚ [।] य‚दा स‚र्व्वोप‚संहारेण व्याप्तिव्य‚तिरेकाभ्यान्त‚दाक्षेपोपि नास्ति त‚देयं व्य‚व‚स्थाप्य‚ते । ‚{\tiny $_{lb}$}‚ \leavevmode\ledsidenote{\textenglish{70/s}} अ ‚{\tiny $_{7}$}‚ थोच्य‚ते त‚दाप्य‚प्र‚द‚र्शितान्व‚य‚व्य‚तिरेकादिदृष्टान्त‚दोषो भ‚व‚ति । भ‚व‚त्व‚{\tiny $_{lb}$}‚य‚म‚प‚रोस्याप‚राधो न ह्येक‚दोषालीढान्येव साध‚नानि भ‚व‚न्ति त्र‚यो हेत्वाभासा दृष्टा‚{\tiny $_{lb}$}‚न्ता ‚{\tiny $_{8}$}‚ \leavevmode\ledsidenote{\textenglish{48b/msK}} भासाश्चाष्टाद‚श \quotelemma{न्याय‚विन्दौ} [तृतीये प‚रिच्छेदे] सोदाह‚र‚णा[ः] प्र‚प‚ञ्चेन ‚{\tiny $_{lb}$}‚द्र‚ष्ट‚व्याः । तेषाम‚नुद्भाव‚नं प‚र्याय‚श‚ब्द‚द्व‚येन व्याच‚ष्टे । त‚च्चानुद्भाव‚नं त्रिभिः ‚{\tiny $_{lb}$}‚कार‚णैरित्याह । त‚तः पुनः \quotelemma{‚{\tiny $_{9}$}‚ साध‚न‚स्य निर्दोष‚त्वादित्यादि} ।
	{\color{gray}{\rmlatinfont\textsuperscript{§~\theparCount}}}
	\pend% ending standard par
      ‚{\tiny $_{lb}$}‚

	  
	  \pstart \leavevmode% starting standard par
	न‚नु च युक्तो निर्दोषे साध‚ने प्र‚तिवादिनो दोषानुद्भाव‚नान्निग्र‚हः । ‚{\tiny $_{lb}$}‚स‚दोषे त्व‚ज्ञानासाम‚र्थ्याभ्याम‚नुद्भाव‚नेपि दोष‚स्य दुष्ट‚साध‚न‚प्र‚यो ‚{\tiny $_{1}$}‚ गाद्वादिन एव ‚{\tiny $_{lb}$}‚प‚राज‚यो युक्तो न प्र‚तिवादिन इति । अत्राह । \quotelemma{न हि दुष्ट‚साध‚नाभिधानेपीति} \cite[7b8]{vn-msN} । ‚{\tiny $_{lb}$}‚य‚द्येवं दुष्टेनापि साध‚नेन वादिना प्र‚तिवादिन‚स्तिर‚स्कृत‚त्वात् क‚स्माज्ज ‚{\tiny $_{2}$}‚ यो ‚{\tiny $_{lb}$}‚न भ‚व‚ति त‚स्येत्याह । \quotelemma{केव‚ल‚मित्यादि} \cite[7b9]{vn-msN} । य‚द्येवं किन्न प‚राज‚यः । त‚त्व‚सिद्धि‚{\tiny $_{lb}$}‚भ्रंशादिति चोद्यं । नानिराक‚र‚णादित्याद्युत्त‚रं । दुर्ज‚नानाम्विप्र‚तिप‚त्तिर‚शोभ‚नो ‚{\tiny $_{lb}$}‚व्य ‚{\tiny $_{3}$}‚ व‚हारः त‚स्मान्न योग‚विहितो न्याय्यः क‚श्चिद्विज‚गीषुवादो नाम य‚च्छ‚ला‚{\tiny $_{lb}$}‚दिभिः क्रिय‚त इत्य‚ध्याहारः । उक्ते स‚ति न्याये त‚त्वार्थी चेत् प्र‚तिवादी प्र ‚{\tiny $_{4}$}‚ तिप‚द्येत ‚{\tiny $_{lb}$}‚त‚म‚र्थं न्यायोपेतं । अथ स्व‚प‚क्ष‚राग‚स्य व‚लीय‚स्त्वादुक्तेपि न्याये न प्र‚तिप‚द्येत । त‚दा ‚{\tiny $_{lb}$}‚तेन प्र‚तिवादिना त‚स्य न्याय‚स्यार्थ‚स्याप्र‚तिप‚त्ताव‚न्य ‚{\tiny $_{5}$}‚ स‚मीप‚व‚र्त्त्यात्म‚ज्ञो ज‚न‚कायो न ‚{\tiny $_{lb}$}‚विप्र‚तिप‚द्येतेति कृत्वा न्यायानुस‚र‚ण‚मेव स‚तां वाद इति व‚र्त्त‚ते । त‚त्व‚र‚क्ष‚णार्थ‚मिति‚{\tiny $_{lb}$}‚प‚रः । य‚थोक्तं त‚त्वा[ध्य]व‚साय ‚{\tiny $_{6}$}‚ संर‚क्ष‚णार्थ‚ञ्ज‚ल्प‚वित‚ण्डे बीज‚प्र‚रोह‚संर‚क्ष‚णार्थं ‚{\tiny $_{lb}$}‚क‚ण्ट‚क‚शाखाव‚र‚ण‚व‚दिति \href{http://sarit.indology.info/?cref=ns\%C5\%AB.4.2.50}{न्या० सू० ४।२।५०} । नेत्याद्याचार्यः । एव‚न्त‚त्वं ‚{\tiny $_{lb}$}‚सुर‚क्षित‚म्भ‚व‚ति । एकान्तेन प्र‚तिद्व‚न्द्युन्मूल‚नादिति भाव[ः ।] ‚{\tiny $_{7}$}‚ त‚द‚भाव इति साध‚न‚{\tiny $_{lb}$}‚प्र‚ख्याप‚न‚साध‚नाभास‚दूष‚ण‚योर‚भावे । अन्य‚थापीति मिथ्याप्र‚लापाद्य‚भावेपि ॥ ० ॥
	{\color{gray}{\rmlatinfont\textsuperscript{§~\theparCount}}}
	\pend% ending standard par
      ‚{\tiny $_{lb}$}‚

	  
	  \pstart \leavevmode% starting standard par
	\hphantom{.}क‚थ‚म‚सौ न दोषः साध‚न‚स्येत्याह । \quotelemma{त‚स्य दोष‚त्वे} ‚{\tiny $_{8}$}‚ नाभिम‚त‚स्य भावेपि सिद्धेर्वि‚{\tiny $_{lb}$}‚\leavevmode\ledsidenote{\textenglish{71/s}} धाताभावात् । साध‚यितुम‚निष्टोप्याकाश‚गुण‚त्वादिकार्य‚त्वेनानित्य‚त्व‚मात्र‚साध‚ने ‚{\tiny $_{lb}$}‚ध्व‚नौ विव‚क्षिते स‚ति \quotelemma{काणादाः} केचिच्चोद‚य‚न्ति ‚{\tiny $_{9}$}‚ \leavevmode\ledsidenote{\textenglish{49a/msK}} न्यायान‚भिज्ञाः । शास्त्रोप‚ग‚मात् ‚{\tiny $_{lb}$}‚स‚र्व्व‚स्त‚दिष्टः साध्यः । त‚त्प्र‚धाने च हेतुप्र‚तिज्ञ‚योर्दोष इति त‚च्चायुज्य‚मानं शास्त्रा‚{\tiny $_{lb}$}‚श्र‚येप्य‚स्त्य‚प‚ग‚त‚मात्र‚स्यैव साध्य‚त्वात् । अन्य‚था ग‚न्धे भूगुण‚ताविप‚र्य‚य‚साध‚नाद‚य‚{\tiny $_{lb}$}‚मेव हेतुर‚स्यामेव प्र‚तिज्ञायां विरुद्धः प्राप्नोति [।] त‚थेद‚म‚प‚र‚म‚दोषोद्भाव‚नं । ‚{\tiny $_{lb}$}‚य‚थाह \quotelemma{भार‚द्वाजो नास्त्यात्मेति प्र‚तिज्ञाप‚द‚योः} प ‚{\tiny $_{2}$}‚ र‚स्प‚र‚विरोध इति । य‚स्मादात्मेति ‚{\tiny $_{lb}$}‚व‚स्त्व‚भिधीय‚ते नास्तीति त‚स्य प्र‚तिषेधः । इद‚म‚प्य‚युक्त‚म‚नादिवास‚नोद्भूतात्म‚विक‚{\tiny $_{lb}$}‚ल्प‚प‚रिनिष्ठित‚प्र‚तिभास‚भेद‚स्य श ‚{\tiny $_{3}$}‚ ब्दार्थ‚स्य प‚रेष्टानित्य‚चित्त‚त्वादिविशेष‚णात्म‚{\tiny $_{lb}$}‚\leavevmode\ledsidenote{\textenglish{72/s}} ल‚क्ष‚ण‚भावोपादान‚त्व‚स्य निराचिकीर्षित‚त्वात् । अत्रैव हि ध‚र्मिणि व्य‚व‚स्थिताः ‚{\tiny $_{lb}$}‚स‚द‚स‚त्व‚ञ्चिन्त‚य‚न्ति स‚न्त ‚{\tiny $_{4}$}‚ [ः] किम‚य‚मात्म‚विक‚ल्प‚प्र‚तिभास्य‚र्थो य‚थाभिम‚त‚{\tiny $_{lb}$}‚भावोपादानो न वेति । न तु पुन‚र‚त्राय‚मेव विक‚ल्प‚प्र‚तिभास्येवार्थोऽप‚ह्नूय‚ते त‚स्यैव ‚{\tiny $_{lb}$}‚बुद्धावुप‚स्थाप ‚{\tiny $_{5}$}‚ नाय श‚ब्द‚प्र‚योगात् प्र‚त्यात्म‚वेद्य‚त्वाच्च । विक‚ल्प‚प्र‚तिबिम्ब‚व्य‚तिरिक्तं ‚{\tiny $_{lb}$}‚तु बाह्यं स्व‚ल‚क्ष‚णं नैव श‚ब्दार्थ इति न त‚स्य विधिर्नापि प्र‚तिषेध‚णं\edtext{}{\lemma{णं}\Bfootnote{? नं}}। ‚{\tiny $_{lb}$}‚अन्य‚था
	{\color{gray}{\rmlatinfont\textsuperscript{§~\theparCount}}}
	\pend% ending standard par
      ‚{\tiny $_{lb}$}‚
	  \bigskip
	  \begingroup
	
	    
	    \stanza[\smallbreak]
	  \flagstanza{\tiny\textenglish{...23}}{\normalfontlatin\large ``\qquad}प‚र‚मा चै ‚{\tiny $_{6}$}‚ क‚तान‚त्वे श‚ब्दानाम‚निब‚न्ध‚ना [।]&‚{\tiny $_{lb}$}‚न स्यात् प्र‚वृतिर‚थेषु द‚र्श‚नान्त‚र‚भेदिषु ॥ [२३]{\normalfontlatin\large\qquad{}"}\&[\smallbreak]
	  
	  
	  
	  \endgroup
	‚{\tiny $_{lb}$}‚
	  \bigskip
	  \begingroup
	
	    
	    \stanza[\smallbreak]
	  \flagstanza{\tiny\textenglish{...24}}{\normalfontlatin\large ``\qquad}अतीताजात‚योर्वापि न च स्याद नृतार्थ‚ता ।&‚{\tiny $_{lb}$}‚वाचः क‚स्याश्चिदित्येषा बौद्धार्थ‚विष‚या म‚ता ‚{\tiny $_{7}$}‚ ॥ [२४]{\normalfontlatin\large\qquad{}"}\&[\smallbreak]
	  
	  
	  
	  \endgroup
	‚{\tiny $_{lb}$}‚\textsuperscript{\textenglish{73/s}}

	  
	  \pstart \leavevmode% starting standard par
	स चाय‚म्विक‚ल्पो भावोपादान‚त्वेन निराचीकीर्षितो देश‚काल‚प्र‚तिनिय‚तिम‚न‚{\tiny $_{lb}$}‚पेक्ष्य विक‚ल्प‚प्र‚तिबिंब‚विष‚य‚त्वादेव चात्म‚श‚ब्द‚स्य न निर्विष‚य‚त्व‚म‚स्ति । त‚त‚श्च ‚{\tiny $_{lb}$}‚य‚दु ‚{\tiny $_{8}$}‚ क्तं य‚च्च य‚त्र प्र‚तिषिद्ध्य‚ते त‚त् त‚स्माद‚न्य‚त्रास्ति । य‚था नास्ति नास‚माना‚{\tiny $_{lb}$}‚धिक‚र‚णो घ‚ट‚श‚ब्दो न घ‚टाभावं प्र‚तिपाद‚यितुँ श‚क्नोति । अपि तु देश‚काल‚विशेषात् ‚{\tiny $_{lb}$}‚प्र‚तिषेधाग ‚{\tiny $_{9}$}‚ \leavevmode\ledsidenote{\textenglish{49b/msK}} ति[ः] । नास्ति घ‚ट इति देश‚विशेषे प्र‚तिषेधो गेहे नास्ति इति । काल‚विशेषे ‚{\tiny $_{lb}$}‚वा प्र‚तिषेधः । इदानीं नास्ति । प्राग्नास्ति । ऊर्ध्वं नास्ति । स‚र्व्व‚स्यायं प्र‚तिषेधो ‚{\tiny $_{lb}$}‚नान‚नुभूत‚घ‚ट‚स‚त्व ‚{\tiny $_{1}$}‚ स्य युक्तः । त‚था नास्त्यात्मेति किम‚य‚न्देश‚विशेषः प्र‚तिषिध्य‚ते । ‚{\tiny $_{lb}$}‚उत्त‚र‚काल‚विशेष इति । य‚दि ताव‚द्देश‚विशेष‚प्र‚तिषेधः । स आत्म‚नि न युक्तोऽदेश‚{\tiny $_{lb}$}‚त्वादात्म‚नः । न च देश‚विशेष‚प्र‚तिषेधादात्मा प्र‚तिषिद्धो भ‚व‚ति । न चाय‚म्भ‚व‚ता‚{\tiny $_{lb}$}‚म‚भिप्रायः । श‚रीर‚मात्मा न भ‚व‚तीति चेत् । क‚स्य वा श‚रीर‚मात्मा यं प्र‚ति प्र‚तिषेधः । ‚{\tiny $_{lb}$}‚श‚रीरे नास्त्या ‚{\tiny $_{3}$}‚ त्मेत्येवं प्र‚तिषेध इति चेत् । क‚स्य श‚रीरे आत्मा यं प्र‚ति प्र‚तिषेधः । ‚{\tiny $_{lb}$}‚क्व त‚र्ह्यात्मा । न क्व‚चिदात्मा । किम‚यं नास्त्येव । न नास्ति विशेष‚प्र‚तिषेधात् । ‚{\tiny $_{lb}$}‚केयं वाचो युक्ति ‚{\tiny $_{4}$}‚ र्न्न श‚रीरे नान्य‚त्र । न च नास्ति । एषैवेषा वाचो युक्तिः । य‚द्य‚था ‚{\tiny $_{lb}$}‚भूत‚न्त‚त्त‚था निर्दिश्य‚त इति न चाय‚मात्मा क्व‚चिद‚पीति । त‚स्मात्त‚थैव निर्देशः । न च ‚{\tiny $_{lb}$}‚काल‚विशेषे ‚{\tiny $_{5}$}‚ प्र‚तिषेधो युक्तः । आत्म‚नि त्रैकाल्य‚स्यान‚भिव्य‚क्तेरात्म‚प्र‚तिषेध‚ञ्च ‚{\tiny $_{lb}$}‚कुर्वाणेनात्म‚श‚ब्द‚स्य विष‚यो व‚क्त‚व्यः । न ह्येकं प‚दं निर‚र्थ‚कं प‚श्यामः ॥ अथापि ‚{\tiny $_{lb}$}‚श‚री ‚{\tiny $_{6}$}‚ रादिषु आत्म‚श‚ब्दं प्र‚तिप‚द्येथाः । एव‚म‚प्य‚निवृत्तौ व्याघातः क‚थ‚मिति । नास्त्या‚{\tiny $_{lb}$}‚त्मेत्य‚स्य वाक्य‚स्य त‚दानीम‚य‚म‚र्थो भ‚व‚ति श‚रीराद‚यो न स‚न्तीति । एव‚मादि ब‚ह्व‚{\tiny $_{lb}$}‚सं ‚{\tiny $_{7}$}‚ ब‚द्धं त‚द‚प‚ह‚स्तित‚म्भ‚व‚ति । प्र‚तिज्ञार्थैक‚देश इत्येत‚द‚प्य‚स‚त् सामान्य‚विशेष‚स्याभा‚{\tiny $_{lb}$}‚वात् । य‚द्वा न प्र‚य‚त्नान‚न्त‚रीय‚क‚त्व‚स्य प्र‚तिज्ञार्थैक‚देश‚ता ध‚र्मिण‚मुप‚ल‚क्ष्य निवृत्त‚{\tiny $_{lb}$}‚त्वात् ‚{\tiny $_{8}$}‚ \leavevmode\ledsidenote{\textenglish{50a/msK}} [।] य‚स्य हि य‚दुप‚ल‚क्ष‚कं न त‚स्य त‚देक‚देश‚त्वं य‚था न काक‚स्य गृद्धैक‚देश- ‚{\tiny $_{lb}$}‚त्व‚मिति ॥
	{\color{gray}{\rmlatinfont\textsuperscript{§~\theparCount}}}
	\pend% ending standard par
      
	    
	    \endnumbering% ending numbering from div
	    
	  \textsuperscript{\textenglish{74/s}}
	  
	% new div opening: depth here is 0
	
	    
	    \beginnumbering% beginning numbering from div depth=0
	    
	  
\chapter[{२. न्याय‚म‚त‚खंड‚न‚म्}][{२. न्याय‚म‚त‚खंड‚न‚म्}]{२. न्याय‚म‚त‚खंड‚न‚म्}\textsuperscript{\textenglish{75/s}}

	  
	  \pstart \leavevmode% starting standard par
	अन्य‚त्तु न युक्त‚मिति \cite[8b5]{vn-msN} य‚दुक्त‚म‚क्ष‚पादेन द्वाविंश‚तिविधं निग्र‚ह‚स्थानं । ‚{\tiny $_{lb}$}‚ प्र‚तिज्ञाहानिः । प्र‚तिज्ञान्त‚रं । प्र‚तिज्ञाविरोधः । प्र‚तिज्ञासंन्यासो हेत्व‚न्त‚र‚म‚र्थान्त‚र‚{\tiny $_{lb}$}‚न्निर‚र्थ‚क‚म‚विज्ञातार्थ‚म‚पार्थ‚क‚म‚प्राप्त‚कालं न्यून म‚धिकं पुन‚रुक्त‚म‚न‚नुभाष‚ण‚म‚ज्ञा‚{\tiny $_{lb}$}‚न‚म‚प्र‚तिभा विक्षेपो म‚तानुज्ञा प‚र्य‚नुयोज्योपेक्ष‚ण‚न्निर‚नुयोज्यानुयोगोप‚सिद्धान्तो ‚{\tiny $_{lb}$}‚हेत्वाभासाश्च निग्र‚ह‚स्थानानि \href{http://sarit.indology.info/?cref=ns\%C5\%AB.5.2.1}{न्या० सू० ५।२।१} । तानीमानि द्वाविंश‚ति‚{\tiny $_{lb}$}‚विधानि विभ‚ज्य व‚क्ष्य‚न्ते\edtext{\textsuperscript{*}}{\lemma{*}\Bfootnote{\href{http://sarit.indology.info/?cref=nbh.5.2.1}{न्याय‚वात्स्याय‚न‚भाष्ये ५।२।१} ।}}। प्र‚तिदृ‚{\tiny $_{2}$}‚ष्टान्त‚ध‚र्मानुज्ञा स्व‚दृष्टान्ते प्र‚तिज्ञाहानिः । ‚{\tiny $_{lb}$}‚ \href{http://sarit.indology.info/?cref=ns\%C5\%AB.5.2.}{न्या० सू० ५।२। } त‚त्र भाष्य‚कार‚म‚तं दूष‚यित्वा वार्त्तिक‚कारोयं स्थित‚प‚क्ष‚माह । ‚{\tiny $_{lb}$}‚त‚मेव ब्रूम इति । भाष्य‚कार‚म‚त‚स्य \quotelemma{भार‚द्वाजे}नैव दूषि‚{\tiny $_{3}$}‚ त‚त्वाद‚स्माक‚म‚र्द्ध‚न्ताव‚द‚व‚सितं ‚{\tiny $_{lb}$}‚भार‚स्येति भावः । त‚त्रेद‚म्भाष्य‚कार‚स्य म‚तं । साध्य‚ध‚र्म‚प्र‚त्य‚नीकेन‚ध‚र्मेण ‚{\tiny $_{lb}$}‚प्र‚त्य‚व‚स्थितः प्र‚तिदृष्टान्त‚ध‚र्म स्व‚दृष्टां ‚{\tiny $_{4}$}‚ तेनुजान‚न् प्र‚तिज्ञां ज‚हातीति प्र‚तिज्ञाहानिः । ‚{\tiny $_{lb}$}‚निद‚र्श‚न‚म‚नित्यः श‚ब्द ऐन्द्रिय‚क‚त्वात् घ‚ट‚व‚दिति कृते प‚र आह । दृष्ट‚मैन्द्रिय‚कं ‚{\tiny $_{lb}$}‚सामान्यं नित्य‚ङ्क‚स्मात् ‚{\tiny $_{5}$}‚ न त‚था श‚ब्द इति प्र‚त्य‚व‚स्थित इद‚माह य‚द्यैन्द्रिय‚कं ‚{\tiny $_{lb}$}‚सामान्यं कामं घ‚टोपि नित्योस्त्विति । स ख‚ल्व‚यं साध‚न‚स्य दृष्टान्त‚स्य नित्य‚त्वं ‚{\tiny $_{lb}$}‚प्र‚स‚ञ्ज‚य‚न्निग‚म‚ना ‚{\tiny $_{6}$}‚ न‚न्त[र]मेव प‚क्ष‚ञ्ज‚हाति प‚क्ष‚ञ्च ज‚ह‚तः प्र‚तिज्ञाहानिरित्युच्य‚ते । ‚{\tiny $_{lb}$}‚प्र‚तिज्ञाश्र‚य‚त्वात् प‚क्ष‚स्येति ।\edtext{\textsuperscript{*}}{\lemma{*}\Bfootnote{त‚त्रैव \href{http://sarit.indology.info/?cref=nbh.5.2.12}{५।२।२} ।}} \quotelemma{वार्तिक‚कारेण} चैव‚मेत‚द् दूषितं । \edtext{\textsuperscript{*}}{\lemma{*}\Bfootnote{\href{http://sarit.indology.info/?cref=nv.5.2.2}{न्याय‚वात्तिंके ५।२।२} ।}}एत‚त्तु न बुद्ध्या‚{\tiny $_{lb}$}‚म‚हे क‚थ‚म‚त्र प्र‚ति ‚{\tiny $_{7}$}‚ ज्ञा हीय‚त इति हेतोर‚नैकान्तिक‚त्वं सामान्य‚दृष्टान्तेन प‚रेण चोद्य‚ते । ‚{\tiny $_{lb}$}‚ \leavevmode\ledsidenote{\textenglish{76/s}} त‚स्यानैकान्तिक‚दोषोद्धार‚म‚नुक्त्वा स्व‚दृष्टान्ते नित्य‚तां प्र‚तिप‚द्य‚ते । नित्य‚ताप्र‚तिप‚त्ते‚{\tiny $_{lb}$}‚श्चा ‚{\tiny $_{8}$}‚ सिद्ध‚तादृष्टान्त‚दोषो भ‚व‚तीति सोयं दृष्टान्त‚दोषेण वा हेतुदोषेण वा निग्र‚हो ‚{\tiny $_{lb}$}‚न प्र‚तिज्ञाहानिरिति । दृष्टान्त‚ञ्च ज‚ह‚त् प्र‚तिज्ञाञ्ज‚हातीति उप‚चारेण निग्र‚ह‚स्थानं । ‚{\tiny $_{9}$}‚ \leavevmode\ledsidenote{\textenglish{50b/msK}} ‚{\tiny $_{lb}$}‚न च प्र‚धानास‚म्भ‚वे उप‚चारो ल‚भ्य‚त इति प्र‚तिज्ञाहानेर्मुख्यो विष‚यो व‚क्त‚व्य इति । ‚{\tiny $_{lb}$}‚इदानीम्वार्तिक‚कार‚म‚तं स्व‚य‚मेवोप‚न्य‚स्य‚ति । \quotelemma{प्र‚तिदृष्टान्त‚स्येत्यादिना} \cite[8b6]{vn-msN} कः ‚{\tiny $_{lb}$}‚पुन‚र ‚{\tiny $_{1}$}‚ त्र दृष्टान्तोऽभिम‚तो य‚दि ताव‚त् य‚त्र ‚{\tiny $_{1}$}‚ लौकिक‚प‚रीक्ष‚कानो\edtext{}{\lemma{कानो}\Bfootnote{? णां}}बुद्धि‚{\tiny $_{lb}$}‚साम्यं स दुष्टान्त \href{http://sarit.indology.info/?cref=ns\%C5\%AB.1.1.25}{न्या० सू० १।१।२५} इति पारिभाषिक‚स्त‚दा भाष्य‚कार‚म‚ताद‚{\tiny $_{lb}$}‚विषेश‚स्त‚त्र च प्र‚तिविहितं । अथान्यः स न ग ‚{\tiny $_{2}$}‚ म्य‚त इत्याह । त‚त्र दृष्ट‚श्चासौ प‚ञ्चाव‚{\tiny $_{lb}$}‚य‚वेन साध‚नेनान्ते च निग‚म‚न‚स्य व्य‚व‚स्थित इति दृष्टान्तः प‚क्षः । त‚तः स्व‚श‚ब्देन स‚ह ‚{\tiny $_{lb}$}‚विशेष‚ण‚स‚मासः । त‚द्विप‚रीतः प्र‚तिदृष्टान्त ‚{\tiny $_{3}$}‚ [ः] । य‚थाऽनित्यः श‚ब्दः ऐन्द्रिय‚क‚त्वादिति ‚{\tiny $_{lb}$}‚ब्रुव‚न्वादी प्र‚तिप‚क्ष‚वादिनि सामान्यादिक‚मैन्द्रिय‚कं नित्यं च । त‚तोविप‚क्षेपि वृत्तेर्व्य‚{\tiny $_{lb}$}‚भिचार्य‚यं हेतुरित्येवं सामा ‚{\tiny $_{4}$}‚ न्येन प्र‚त्य‚व‚स्थिते स‚त्याह य‚द्येवं श‚ब्दोप्येव‚म‚स्त्विति ‚{\tiny $_{lb}$}‚एषा प्र‚तिज्ञाहानिर्नाम निग्र‚ह‚स्थानं । क‚स्मात् । प्राग्प्र‚तिज्ञात‚स्य श‚ब्दानित्य‚त्व‚स्य ‚{\tiny $_{lb}$}‚त्यागात् । प्र‚तिज्ञाश ‚{\tiny $_{5}$}‚ ब्देन ध‚र्मिविशेष‚ण‚भूतो ध‚र्म उच्य‚ते स‚मुदायाव‚य‚व‚त्वात् । एत‚त् ‚{\tiny $_{lb}$}‚प्र‚तिक्षिप‚ति । अत्र \quotelemma{भार‚द्वाज} म‚ते उप‚ग‚तायाः प्र‚तिज्ञायास्त्यागात् कार‚णात् । येयं ‚{\tiny $_{lb}$}‚प्र‚ति ‚{\tiny $_{6}$}‚ ज्ञाहानिर्व्य‚व‚स्थापिता त‚स्यां विशेष‚निय‚मः किङ्कृतः । कोसाव‚नेन प्र‚कारेण ‚{\tiny $_{lb}$}‚स्व‚प‚क्षे प्र‚तिप‚क्ष‚ध‚र्मानुज्ञास्व‚रूपेण प्र‚तिज्ञाहानिरिति । स्यात् म‚त‚म‚य‚मेव प्र‚ति ‚{\tiny $_{7}$}‚ ज्ञा‚{\tiny $_{lb}$}‚हानिः प्र‚कारो नान्योस्ति त‚तो निय‚मार्थ‚मुच्य‚त इति । \quotelemma{स‚म्भ‚व‚ति ह्य‚न्येनापी} ति । अथ ‚{\tiny $_{lb}$}‚ \leavevmode\ledsidenote{\textenglish{77/s}} म‚तिः प्र‚धान‚मेत‚न्निमित्तं त‚स्यास्त‚तोस्मिन् प्र‚द‚र्शितेऽन्योपि प्र‚काशित एव भ‚व‚ती ‚{\tiny $_{8}$}‚ ति । ‚{\tiny $_{lb}$}‚त‚द‚त्राप्याह । इद‚मेव च हेतुदोषोद्भाव‚नादिक‚ङ्कार‚णं य‚स्मादेवं हेतुदोषोद्भाव‚ना‚{\tiny $_{lb}$}‚दिना प्र‚तिपादितेन प्र‚तिवादिना प्र‚तिज्ञा हात‚व्या स‚म्य‚ग्दूष‚णाभिधानात् । ‚{\tiny $_{9}$}‚ \leavevmode\ledsidenote{\textenglish{51a/msK}} य‚च्चेद- ‚{\tiny $_{lb}$}‚म‚भ्य‚धायि सामान्यं नित्य‚मैन्द्र्यिक‚मित्युक्ते श‚ब्दोप्येव‚म‚स्त्वित्य‚त्र प्र‚तिविध‚त्ते । \quotelemma{इद‚{\tiny $_{lb}$}‚म्पुन‚र‚स‚म्ब‚द्ध‚मेव} \cite[9a1]{vn-msN} । य‚स्मात् कः स्व‚स्थात्मा सामान्योप‚द‚र्श‚न‚मात्रेण सामान्य‚{\tiny $_{lb}$}‚म‚स्ति ‚{\tiny $_{1}$}‚ न चैन्द्रिय‚क‚न्नित्य‚ञ्चेत्येत‚द‚विचार्य श‚ब्दं नित्यं प्र‚तिप‚द्येत । एताव‚त्तु भ‚वेत् ‚{\tiny $_{lb}$}‚साम‚न्य‚स्यापि नित्य‚स्यैन्द्रिय‚क‚त्वे त‚स्य ऐन्द्रिय‚क‚त्व‚स्यानित्येपि घ‚टे द‚र्श‚नात् संश‚यितः ‚{\tiny $_{lb}$}‚स्यात् ‚{\tiny $_{2}$}‚ [।] अपि च प्र‚तिदृष्टान्त‚ध‚र्मानुज्ञैवात्र न युक्तेत्याह । न च त‚द्ध‚मं त‚स्य सामा‚{\tiny $_{lb}$}‚न्य‚स्य ध‚र्म‚न्नित्य‚त्वं य‚तोऽनित्यः श‚ब्द इति व‚द‚ता क‚स्य‚चिन्नित्यः श‚ब्द इत्य‚य‚म‚ञ्ज‚शो\edtext{}{\lemma{शो}\Bfootnote{‚{\tiny $_{lb}$}‚? से}}‚{\tiny $_{3}$}‚ ति प्र‚त्यास‚न्नः प्र‚तिप‚क्षः स्यान्न सामान्य‚न्त‚स्य ध‚र्म्य‚न्त‚र‚त्वात् । त‚था ह्येका‚{\tiny $_{lb}$}‚धिक‚र‚ण‚योरेव नित्य‚त्वानित्य‚त्व‚योर्विरोधो न नानाधिक‚र‚ण‚योः । आञ्ज‚स ‚{\tiny $_{4}$}‚ ग्र‚ह‚ण‚म‚य‚{\tiny $_{lb}$}‚म‚पि विरुद्ध‚ध‚र्माधिक‚र‚ण‚त्वात् प्र‚तिप‚क्षो न त्व‚तिनिक‚टो य‚था नित्यः श‚ब्द इत्य‚य‚मिति ‚{\tiny $_{lb}$}‚प‚रिदीप‚नार्थं । नानेन प्र‚कारेण प्र‚तिज्ञाहाने ‚{\tiny $_{5}$}‚ र्निग्र‚हार्ह इति व‚र्त‚ते । केनानेनेत्याह । ‚{\tiny $_{lb}$}‚प्र‚तिप‚क्ष‚ध‚र्मानुज्ञ‚या । अथ‚वा अनेनेत्य‚साध‚नाङ्ग‚व‚च‚नेन । य‚थोक्त‚मिद‚मेव प्र‚धानं ‚{\tiny $_{lb}$}‚निमित्त‚मिति ॥
	{\color{gray}{\rmlatinfont\textsuperscript{§~\theparCount}}}
	\pend% ending standard par
      ‚{\tiny $_{lb}$}‚\textsuperscript{\textenglish{78/s}}

	  
	  \pstart \leavevmode% starting standard par
	\hphantom{.}प्र‚तिज्ञा ‚{\tiny $_{6}$}‚ तार्थ‚प्र‚तिषेधे ध‚र्म‚विक‚ल्पात्त‚द‚र्थ‚निर्देशः प्र‚तिज्ञान्त‚र‚मिति \cite[9a5]{vn-msN} \href{http://sarit.indology.info/?cref=ns\%C5\%AB.5.2.3}{न्या० सू० ५।२।३ } द्वितीय‚ल‚क्ष‚ण‚सूत्रं [।] निग्र‚ह‚स्थान‚मिति स‚र्व्व‚त्रानुव‚र्त‚ते । अस्यार्थः ‚{\tiny $_{lb}$}‚प्र‚तिषेधो विप‚क्षे हेतुस‚द्भाव ‚{\tiny $_{7}$}‚ क‚थ‚नं त‚स्मिन्स‚ति स‚प‚क्ष‚विप‚क्ष‚योर्द्ध‚र्म‚भेदेन क‚र‚ण‚भू‚{\tiny $_{lb}$}‚तेन पूर्व्व‚प्र‚तिज्ञार्थ‚प्र‚तिप‚त्य‚र्थं प्र‚तिज्ञान्त‚र‚ङ्क‚रोति । य‚था घ‚टोऽस‚र्व‚ग‚त एवं श‚ब्दोप्य‚{\tiny $_{lb}$}‚स‚र्व्व‚ग‚तो घ‚ट‚व‚देवा ‚{\tiny $_{8}$}‚ नित्यः श‚ब्द इति शेषः सुज्ञानः । इदं निराक‚रोति \quotelemma{अत्रापी} \cite[9a8]{vn-msN} ‚{\tiny $_{lb}$}‚त्यादिना । \quotelemma{अबिद्ध‚क‚र्ण} स्तु भाष्य‚टीकायामिद‚माश‚ङ्क्य‚प‚रिजिहीर्ष‚ति [।] \quotelemma{न‚नु चास‚र्व्व‚{\tiny $_{lb}$}‚ग‚त‚त्वे} स‚तीति । हे ‚{\tiny $_{9}$}‚ \leavevmode\ledsidenote{\textenglish{51b/msK}} तुविशेष‚ण‚मुक्तं । स‚विशेष‚ण‚श्च हेतुर्विप‚क्षे नास्तीति न प्र‚तिज्ञान्त‚रं ‚{\tiny $_{lb}$}‚निग्र‚ह‚स्थानं । न‚हि त‚देव‚म‚स‚र्व्व‚ग‚तः श‚ब्द इति प्र‚तिज्ञान्त‚रोपादानात् । हेतुविशेष‚णो‚{\tiny $_{lb}$}‚पादाने हेत्व‚न्त‚रं निग्र‚ह‚स्थान‚मिति । एत‚च्चातिस्थूलं । स ह्येवं प‚क्ष‚ध‚र्म‚मेव विद‚ग्ध‚{\tiny $_{lb}$}‚बुद्धिर्विशिन‚ष्टि न तु प्र‚तिज्ञान्त‚र‚मुपाद‚त्ते सिद्ध‚त्वात् । य‚द‚पि हेतुविशेष‚णोपादाने ‚{\tiny $_{lb}$}‚हेत्व ‚{\tiny $_{2}$}‚ न्त‚र‚न्निग्र‚ह‚स्थान‚मित्य‚भ्य‚धायि त‚द‚प्य‚तिपेल‚वं । य‚स्मादेवं त‚देव नामास्तु प्र‚तिज्ञा‚{\tiny $_{lb}$}‚न्त‚र‚त्व‚स‚म्ब‚द्धं । उदाह‚र‚ण‚साध‚र्म्यादेश्चेति । उदाह‚र‚ण‚साध‚र्म्यात्साध्य‚साध‚नं हेतु ‚{\tiny $_{lb}$}‚ \href{http://sarit.indology.info/?cref=ns\%C5\%AB.1.1.34}{न्या० सू० १।१।३४} रित्येत‚स्य प्र‚तिज्ञाल‚क्ष‚ण‚स्य साध्य‚निर्देशः प्र‚तिज्ञेत्येत‚स्याभावात् । ‚{\tiny $_{lb}$}‚उपाद‚व‚ता चानेन प्र‚तिज्ञां प्र‚तिज्ञासाध‚नाय प्र‚तिज्ञामात्रेण युक्तिर‚हिते ‚{\tiny $_{4}$}‚ न सिद्धिरिष्टा ‚{\tiny $_{lb}$}‚ \leavevmode\ledsidenote{\textenglish{79/s}} भ‚व‚ति । त‚त‚श्च प्राग‚पि प्र‚थ‚म‚प्र‚तिज्ञान‚न्त‚र‚म‚पि हेतुमैन्द्रिय‚क‚त्व‚न्न ब्रूयात् । त‚स्मादेवं ‚{\tiny $_{lb}$}‚प्र‚काराणाम्बाल‚प्र‚लापानां प्र‚तिज्ञासाध‚नाय प्र‚ति ‚{\tiny $_{5}$}‚ ज्ञान्त‚र‚मुच्य‚त इत्येवं रूपाणां ‚{\tiny $_{lb}$}‚प रिस ङ्ख्यातुम‚स\edtext{}{\lemma{स}\Bfootnote{? श}}क्य‚त्वात् ल‚क्ष‚ण‚निय‚मोप्य‚स‚म्ब‚द्ध एव । कोसौ । प्र‚तिज्ञान्त‚रा‚{\tiny $_{lb}$}‚भिधाने प्र‚तिज्ञान्त‚रं नाम निग्र‚ह‚स्थान‚मिति ‚{\tiny $_{6}$}‚ ।
	{\color{gray}{\rmlatinfont\textsuperscript{§~\theparCount}}}
	\pend% ending standard par
      ‚{\tiny $_{lb}$}‚

	  
	  \pstart \leavevmode% starting standard par
	न‚नु नाय‚मीदृशो ल‚क्ष‚ण‚निय‚मः प्र‚तिज्ञातार्थ‚प्र‚तिषेधे ध‚र्म‚विक‚ल्पात्त‚द‚र्थ‚निर्देश ‚{\tiny $_{lb}$}‚इत्येवं कृत‚त्वात् । नास्ति दोष‚स्त‚स्यैव प‚र्य्यायान्त‚रेण क‚थ‚नात् । अथोच्य‚ते य ‚{\tiny $_{7}$}‚ था ‚{\tiny $_{lb}$}‚विद्वांसो न प्र‚तिज्ञां प्र‚तिज्ञासाधान‚याहुस्त‚था साध्य‚सिध्य‚र्थ‚म‚सिद्ध‚विरुद्धानैकान्त‚कादी‚{\tiny $_{lb}$}‚न‚पि प्र‚युञ्ज‚ते त‚त‚श्चासाध‚नाङ्ग‚व‚च‚न‚मित्यादि त्व‚यापि न वाच्यं ‚{\tiny $_{8}$}‚ भ‚वेद‚तः प्राह [।] ‚{\tiny $_{lb}$}‚विदुषाम‚पी \cite[9b6]{vn-msN} ति । अनुद्दिश्याप्र‚माण‚कं शास्त्रोप‚ग‚म‚मिति माम‚कीने त‚न्त्रे ‚{\tiny $_{lb}$}‚ \leavevmode\ledsidenote{\textenglish{80/s}} सामान्यं य‚था भूतं सिद्ध‚मित्येव न प्र‚द‚र्श्य‚त[इ]त्य‚र्थः । त‚थाहि व्युत्थित ‚{\tiny $_{9}$}‚ \leavevmode\ledsidenote{\textenglish{52a/msK}} चेत‚सो ‚{\tiny $_{lb}$}‚न प‚र‚स‚म‚य‚व्य‚व‚स्थोप‚रोध‚माद्रिय‚न्ते त‚त्व‚द‚र्श‚नाध्य‚व‚साय‚शूराः शू\edtext{}{\lemma{शू}\Bfootnote{? सू}}र‚यः । ‚{\tiny $_{lb}$}‚अप्र‚माण‚क‚म्व‚च‚नं प्र‚माणोपेत‚स्याभ्युप‚ग‚म‚स्य विद्व‚द्भिर‚ल‚ङ्घ‚नीय‚त्वात् । एत‚च्च ‚{\tiny $_{lb}$}‚स्या ‚{\tiny $_{1}$}‚ त् प्र‚माणैर‚स‚म‚र्थित‚साध‚नाभिधानाद्वाद्य‚पि जेता न भ‚व‚ति प्र‚तिप‚क्ष‚स्य ‚{\tiny $_{lb}$}‚निराक‚र‚णात् ॥ ४ ॥
	{\color{gray}{\rmlatinfont\textsuperscript{§~\theparCount}}}
	\pend% ending standard par
      ‚{\tiny $_{lb}$}‚

	  
	  \pstart \leavevmode% starting standard par
	\hphantom{.}प्र‚तिज्ञाहेत्वोर्विरोधः प्र‚तिज्ञाविरोधो \href{http://sarit.indology.info/?cref=ns\%C5\%AB.5.2.4}{न्या० सू० ५।२।४ }\cite[9b10]{vn-msN} नाम ‚{\tiny $_{lb}$}‚निग्र‚ह‚स्थानं । गुण‚व्य‚ति ‚{\tiny $_{2}$}‚ रिक्तं द्र‚व्य‚मिति प्र‚तिज्ञा । रूपादिभ्योर्थान्त‚र‚स्यानुप‚{\tiny $_{lb}$}‚ल‚ब्धेरिति हेतुः । सोय‚म्प्र‚तिज्ञाहेत्वोर्विरोधः । य‚दि गुण‚व्य‚तिरिक्तं द्र‚व्यं रूपादिभ्ये‚{\tiny $_{lb}$}‚ऽर्थान्त‚र‚स्यानुप‚ल‚ब्धिर्नोप‚प‚द्य‚ते । अथ रूपादिभ्योर्थान्त‚र‚स्यानुप‚ल‚ब्धिर्गुण‚व्य‚तिरिक्तं ‚{\tiny $_{lb}$}‚द्र‚व्य‚मिति नोप‚प‚द्य‚ते । एतेनैव प्र‚तिज्ञाहेत्वोर्विरोधेन प्र‚तिज्ञाविरोधः स्व‚व‚च‚नेन ‚{\tiny $_{lb}$}‚व्याख्यात [।] ‚{\tiny $_{4}$}‚ सूत्र‚कारेणास्योप‚ल‚क्ष‚णार्थ‚मुक्त‚मेत‚त् । श्र‚म‚णा \cite[10a1]{vn-msN} प्र‚तिविर‚त‚{\tiny $_{lb}$}‚पुरुष‚स‚म्भोगा ग‚र्भ‚श्च नान्त‚रेण पुरुष‚स‚म्भोग‚मिति स्व‚व‚च‚न‚व्याह‚तिः । हेतुविरोध ‚{\tiny $_{lb}$}‚एतेन ‚{\tiny $_{5}$}‚ चोक्त इति व‚र्त‚ते । स‚र्व्व पृथ‚ग् नाना नास्त्येको भाव इति याव‚त् । स‚मूहे ‚{\tiny $_{lb}$}‚भाव‚स‚ब्द\edtext{}{\lemma{ब्द}\Bfootnote{? श‚ब्द}}प्र‚योगात् स‚मूह‚वाच‚क‚घ‚टादिभाव‚श‚ब्द‚वाच्य‚त्वादित्य‚र्थः । य‚स्मात् ‚{\tiny $_{lb}$}‚स‚मूह इ ‚{\tiny $_{6}$}‚ ति ब्रुवाणेन एकोभ्युप‚ग‚तो भ‚व‚ति । एक‚स‚मुच्च‚यो हि स‚मूह इति । त‚था हि ‚{\tiny $_{lb}$}‚ग‚वादिद्र‚व्याणि स‚मुदितानि प्र‚तिप‚द्य‚मानेन स‚मूहोभ्युपेयः । स चायं स‚मूह‚य‚न्ति ‚{\tiny $_{lb}$}‚ \leavevmode\ledsidenote{\textenglish{81/s}} द्र ‚{\tiny $_{7}$}‚ व्याण्येतानि ग‚वादिभावेन व्य‚व‚स्थितानीति न व्य‚व‚तिष्ठ‚ते । भेदोप्य‚ल्प‚त‚र‚त‚म‚त्वेन ‚{\tiny $_{lb}$}‚य‚त्त‚त्र प‚र‚माल्पं य‚द‚भेद्यं त‚तो निव‚र्त्त‚ते य‚त‚श्चायं भेदो निव‚र्त्त‚ते त‚देकं । अथ म ‚{\tiny $_{8}$}‚ न्य‚से ‚{\tiny $_{lb}$}‚यं त‚म‚भेद्यं प‚र‚माणुं म‚न्य‚से सोपि रूपादीनां स‚मुदाय इति । एत‚स्मिन्वै द‚र्श‚ने ये ‚{\tiny $_{lb}$}‚रूपाद‚यः स‚मुदितास्ते प‚र‚माणुरिति प‚र‚माणौ रूपं स क‚स्य स‚मुदाय इ ‚{\tiny $_{9}$}‚ \leavevmode\ledsidenote{\textenglish{52b/msK}} ति व‚क्त‚व्यं । ‚{\tiny $_{lb}$}‚एवं शेषेषु गुणेषु । अथ न तं स‚मुदाय‚म्प्र‚तिप‚द्य‚से । अष्टौ द्र‚व्याणि स‚मुदितानि ‚{\tiny $_{lb}$}‚प‚र‚माणुरिति शास्त्रं व्याह‚तं । \quotelemma{कामेऽष्ट‚द्र‚व्य‚कोऽश‚ब्दः प‚र‚माणुरिति \href{http://sarit.indology.info/?cref=ak.2.22}{अभिध‚र्म‚कोशे २।२२}} । त‚स्मा ‚{\tiny $_{1}$}‚ द‚नुप‚प‚त्ताव‚नेकोप‚प‚त्तिरित्य‚तिमौढ्यं । असिद्ध‚श्चायं हेतुः । ‚{\tiny $_{lb}$}‚य‚स्माद‚नेक‚विध‚ल‚क्ष‚णैर्ग‚न्धादिभिर्गुणैर्बुध्नादिभिश्चाव‚य‚वैः स‚म्ब‚द्ध एको भाव ‚{\tiny $_{lb}$}‚उप‚प‚द्य‚ते । अतः ‚{\tiny $_{2}$}‚ श‚ब्दादेकार्थाधिग‚तौ शेषोनुस‚क्तो\edtext{}{\lemma{क्तो}\Bfootnote{? ष‚क्तो}}र्थो ग‚म्य‚त इति ।
	{\color{gray}{\rmlatinfont\textsuperscript{§~\theparCount}}}
	\pend% ending standard par
      ‚{\tiny $_{lb}$}‚

	  
	  \pstart \leavevmode% starting standard par
	न‚नु चाय‚म‚पि प्र‚तिज्ञाहेत्वोर्विरोध इति प्र‚थ‚माद‚स्याविशेषः । मैव‚मुभ‚याश्रित ‚{\tiny $_{lb}$}‚त्वात् विरोध‚स्य । विव‚क्षातो ‚{\tiny $_{3}$}‚ ऽन्य‚त‚र‚निर्देश इति \quotelemma{भार‚द्वाजे} नैवोक्त‚त्त्वात् । प्र‚ति‚{\tiny $_{lb}$}‚ज्ञाया दृष्टान्त‚विरोधो य‚था व्य‚क्त‚मेक‚प्र‚कृतिकं प‚रिमित‚त्वात् श‚रावादिव‚दिति ‚{\tiny $_{lb}$}‚श‚रावादिर्दृष्टान्त ए ‚{\tiny $_{4}$}‚ क‚प्र‚कृतित्वं बाध‚ते । दृष्टान्त‚भूतायाः प्र‚कृतेः प्र‚कृत्यंत‚र‚त्वात् । ‚{\tiny $_{lb}$}‚एक‚प्र‚कृतित्वे वा श‚रावादिर्दृष्टान्तोऽयुक्तः । हेतोश्च दृष्ट‚न्तादिभिर्विरोधो य‚था ‚{\tiny $_{lb}$}‚गुण ‚{\tiny $_{5}$}‚ व्य‚तिरिक्तं द्र‚व्य‚म‚र्थान्त‚र‚त्वेनानुप‚ल‚भ्य‚मान‚त्त्वात् । घ‚टादिव‚दिति । घ‚टादी‚{\tiny $_{lb}$}‚नाम्भेदेन ग्र‚ह‚णाद्धेतुं बाध‚ते दृष्टान्तः । आदिग्र‚ह‚णेन हेतोरुप‚न‚य‚निग‚म[न]आभ्यां ‚{\tiny $_{lb}$}‚विरोधो गृह्य‚ते । अन‚योरुदाह‚र‚ण‚म‚नित्यः श‚ब्दः कृत‚क‚त्वात् । य‚त्कृत‚क‚न्त‚द‚नित्यं ‚{\tiny $_{lb}$}‚य‚थाकाश‚न्त‚था च कृत‚कः श‚ब्द इत्युप‚न‚येन हेतोर्विरोधः । त‚था ह्युदा ‚{\tiny $_{7}$}‚ ह‚र‚णा ‚{\tiny $_{lb}$}‚पेक्ष‚स्त‚थेत्युयुप‚संहारो न त‚थेति चेति \href{http://sarit.indology.info/?cref=ns\%C5\%AB.1.1.38}{न्या० सू० १।१।३८} साध्य‚स्योप‚न‚य उक्तः । ‚{\tiny $_{lb}$}‚इह च विप‚रीत‚मुदाह‚र‚ण‚मित्येत‚द‚पेक्षोप‚न‚येन हेतोर्विरोधः । ईदृशे च प्र‚योगे ‚{\tiny $_{lb}$}‚त ‚{\tiny $_{8}$}‚ स्माद‚नित्य इत्युप‚संहारे निग‚म‚नेन । प्र‚माण‚विरोध‚श्च प्र‚तिज्ञाहेतोर्य‚थाऽनुष्णो‚{\tiny $_{lb}$}‚ग्निर्द्र‚व्य‚त्वाज्ज‚ल‚व‚दिति प्र‚त्य‚क्ष‚म्बाध‚ते । \quotelemma{प‚र‚प‚क्ष} \cite[10a2]{vn-msN} इत्यादि । एत‚च्च ‚{\tiny $_{lb}$}‚य‚च्च \quotelemma{स्व‚प‚क्षा ‚{\tiny $_{9}$}‚\leavevmode\ledsidenote{\textenglish{53a/msK}} न‚पेक्ष‚ञ्} चेत्यादि \cite[10a3]{vn-msN} । एत‚द‚प्युभ‚य‚म्प्र‚तिज्ञाहेतोर्विरोध ‚{\tiny $_{lb}$}‚इत्य‚नेनैव स‚ङ्गृहीत‚त्वात् पृथ‚ग् निग्र‚ह‚स्थान‚त्वेन नैव व‚क्त‚व्य‚मिति द‚र्श‚य‚ति । ‚{\tiny $_{1}$}‚ ‚{\tiny $_{lb}$}‚प‚र‚प‚क्ष इत्य‚त्र प‚रेण‚प्र माणे कृते \quotelemma{क‚णा} दोऽनैकान्तिक‚मुद्भाव‚य‚ति । स्व‚प‚क्षान‚पेक्ष‚ञ्चेत्य‚त्र ‚{\tiny $_{lb}$}‚ \leavevmode\ledsidenote{\textenglish{82/s}} तु \quotelemma{वैशेषिक} एव प्र‚माण‚ङ्क‚रोति । प‚र‚स्तं व्य‚भिचार‚य‚तीति भेदः । य‚दि त‚र्हि ‚{\tiny $_{lb}$}‚गो ‚{\tiny $_{2}$}‚ त्वादिना व्य‚भिचारे कृते विरुद्ध‚मुत्त‚रं त‚था स‚त्य‚नैकान्तिको निर्विष‚य इत्याह । ‚{\tiny $_{lb}$}‚ \quotelemma{उभ‚येत्या \cite[10a4]{vn-msN} दि} । वादिप्र‚तिवादिप्र‚सिद्ध उभ‚य‚प‚क्ष‚संप्र‚तिप‚न्नः सोऽनैकान्तिक‚स्त‚{\tiny $_{lb}$}‚द्विष‚य‚त्वादुप ‚{\tiny $_{3}$}‚ चारेण त‚था च वृत्तिस्तेनानैकान्तिक‚चोद‚नेति । \quotelemma{अत्रापी} \cite[10a3]{vn-msN} त्यादि । ‚{\tiny $_{lb}$}‚नैत‚द‚पि प्र‚तिक्षिप‚ति त‚दाश्र‚यः सा प्र‚तिज्ञाऽश्र‚यो य‚स्य विरोध‚स्य स त‚था । त‚त्कृतो ‚{\tiny $_{lb}$}‚वेति त ‚{\tiny $_{4}$}‚ या प्र‚तिज्ञ‚या कृतः । प‚रिशिष्ट‚म‚तिस्फुटं । व्य‚तिरिक्तानाम‚पि कुत‚श्चित् प‚र्व‚{\tiny $_{lb}$}‚तादेः स‚काशाद्विप्र‚क‚र्षिणाम्पिसा\edtext{}{\lemma{र्षिणाम्पिसा}\Bfootnote{? शा}}चादीनां त‚त्रेद‚मेव निग्र‚हाधिक‚र‚णं । ‚{\tiny $_{5}$}‚ य‚दुत ‚{\tiny $_{lb}$}‚प्र‚तिज्ञायाः प्र‚योगः । न विरोधः प्र‚तिज्ञायाः निग्र‚हाधिक‚र‚ण‚मिति व‚र्त‚ते । किमिति । ‚{\tiny $_{lb}$}‚त‚द‚धिक‚र‚ण‚त्वात् । प्र‚तिज्ञाश्र‚य‚त्वात् इत्य‚र्थः । य‚दि पु ‚{\tiny $_{6}$}‚ न‚स्त‚द‚धिक‚र‚णो न भ‚वेद् ‚{\tiny $_{lb}$}‚ \leavevmode\ledsidenote{\textenglish{83/s}} भ‚वेन्निग्र‚हाधिक‚र‚ण‚मित्याह । \quotelemma{य‚दी} \cite[10b1]{vn-msN} त्यादि । प्र‚स्ताव‚स्य वाद‚स्योप‚संहारः प‚रि‚{\tiny $_{lb}$}‚स‚माप्तिस्त‚स्याव‚सान‚न्निमितं प्र‚तिज्ञाप्र‚योगः । त‚न्मात्रेणै ‚{\tiny $_{7}$}‚ वासाध‚नाङ्गाभिधानात् ‚{\tiny $_{lb}$}‚वादिनोभ‚ङ्गात् । क्व‚चित्प्र‚स्तावोप‚संहाराव‚स‚र‚त्वादिति प‚ठ्य‚ते । त‚त्रापि ‚{\tiny $_{lb}$}‚वाद‚प‚रिस‚माप्तेः प्र‚तिज्ञाप‚द‚प्र‚योगे स‚त्य‚व‚स‚रोऽधिकार इत्य‚र्थः । अथ बुद्धिर्य‚था ‚{\tiny $_{lb}$}‚भ‚व‚द्भिः क‚स्य‚चिद‚र्थ‚स्य क्ष‚णिक‚त्वादिक‚मेक‚मेव साध्यं ब‚हुभिः स‚त्वोत्प‚त्तिम‚त्व‚{\tiny $_{lb}$}‚प्र‚त्य‚य‚भेद‚भेदित्वादिभिर्हेतुभिः प्र‚तिपाद्य‚ते त‚थैक ‚{\tiny $_{9}$}‚‚{\tiny $_{53b}$}‚ म‚पि दूष्य‚म्प‚रोप‚न्य‚स्तं साध‚न‚{\tiny $_{lb}$}‚वाक्यं प्र‚तिज्ञोपादान‚द्वारेण त‚द्विरोध‚द्वारेणान्य‚था वा दूष्य‚ते । त‚था च नाय‚न्दोषः ‚{\tiny $_{lb}$}‚प‚राजित‚प‚राज‚याभावादिति । त‚द‚त्राह । ये तु हेत‚वः ‚{\tiny $_{1}$}‚ उच्य‚न्ते \cite[10b2]{vn-msN} तेषाम्विक‚ल्पेन ‚{\tiny $_{lb}$}‚पूर्व्व‚हेत्व‚न‚पेक्ष‚या । एवं वैत‚त् । अथ‚वान्य‚था साध‚यामीत्येत‚त् साध्य‚साध‚नाय वृत्तेः ‚{\tiny $_{lb}$}‚कार‚णात्साम‚र्थ्य‚म‚स्ति [।] किं पुनः कार‚णं न स‚मुच्च‚ये नैव प्र‚योग ‚{\tiny $_{2}$}‚ इत्याह । \quotelemma{अन्य‚था ‚{\tiny $_{lb}$}‚य‚दि} \cite[10b3]{vn-msN} स‚मुच्च‚ये नैवाप‚र‚हेत्व‚न्त‚र‚प्र‚योगोभीष्ट‚स्त‚दा द्वितीय‚स्य वैय‚र्थ्यात् विक‚ल्पेन ‚{\tiny $_{lb}$}‚सामान्य‚मिति व‚र्त‚ते । वैय‚र्थ्य‚मेव प्र‚तिपाद‚य‚ति । य‚दि हि त‚त्रा ‚{\tiny $_{3}$}‚ प्येक‚प्र‚योग‚म‚न्त‚रेणा‚{\tiny $_{lb}$}‚प‚र‚स्य प्र‚योगो न स‚म्भ‚वेत् । उभ‚य‚प्र‚तिषेधेन विध्य‚व‚सायात् । य‚द्येक‚स्य प्र‚योगे‚{\tiny $_{lb}$}‚ऽप‚र‚स्य स‚मुच्च‚येन प्र‚योगः स‚म्भ‚वेदि ‚{\tiny $_{4}$}‚ त्य‚र्थः । त‚दा न द्वितीय‚स्य क‚श्चित् साधानार्थो ‚{\tiny $_{lb}$}‚प्र‚तीत‚प्र‚तिपाद‚नाभावात् । प्र‚थ‚म‚हेतुप्र‚तिपादित एवार्थे व्यापृत‚त्वान्निष्पादित‚{\tiny $_{lb}$}‚क्रिये दारुणि प्र‚वृत्त ‚{\tiny $_{5}$}‚ स्यैव दात्रादेर्न क‚श्चित्साध‚क‚त‚म‚त्वार्थ इति याव‚त् । न‚नु च ‚{\tiny $_{lb}$}‚साध‚न‚व‚द्विक‚ल्पेनैव दूष‚ण‚म‚पि भ‚विष्य‚ति । एवं म‚न्य‚ते । नैवं प‚रोभ्युप‚ग‚न्तुर्म‚ह‚ति । ‚{\tiny $_{lb}$}‚ए ‚{\tiny $_{6}$}‚ वं हि तेन स्व‚य‚मेव प्र‚तिज्ञाया असाध‚नाङ्ग‚त्व‚म्प्र‚तिप‚न्न‚म्भ‚वेत् । त‚त‚श्चैत‚द् व्याह‚{\tiny $_{lb}$}‚न्य‚ते । प्र‚तिज्ञाहेतूदाह‚र‚णोप‚न‚य‚निग‚म‚नान्य‚व‚य‚वा \href{http://sarit.indology.info/?cref=ns\%C5\%AB.1.1.32}{न्या० सू० १।१।३२ } इति । ‚{\tiny $_{lb}$}‚अन्यैरेव हेतुभिरित्य‚व ‚{\tiny $_{7}$}‚ य‚विद्र‚व्य‚निषेध‚कैः पूर्व्वोक्त‚प्र‚कारैः कुम्भादिश‚ब्द‚स्यैक‚{\tiny $_{lb}$}‚घ‚टाद्य‚व‚य‚विद्र‚व्य‚ल‚क्ष‚ण‚विशेषान‚भिधान‚म‚नेक‚स्य चार्थ‚स्य रूपादेर्य‚त्सामान्य‚मेकार्थ‚{\tiny $_{lb}$}‚ \leavevmode\ledsidenote{\textenglish{84/s}} क्रियासाम‚र्थ्यात्म ‚{\tiny $_{8}$}‚ क‚न्त‚द‚भिधान‚ञ्च प्र‚तिपाद्य स‚र्व्व‚स्य श‚ब्दार्थ‚स्य रूपादेरेकार्थ‚{\tiny $_{lb}$}‚क्रियास‚म‚र्थ‚स्य नानार्थ‚रूप‚त‚या क‚र‚ण‚भूत‚या । एक‚श्चासौ व‚स्तुविशेष‚स्व‚भाव‚श्चा‚{\tiny $_{lb}$}‚व‚य‚विद्र‚व्य‚रूप‚स्त ‚{\tiny $_{9}$}‚ \leavevmode\ledsidenote{\textenglish{54a/msK}} स्य भाव एक‚व‚स्तुविशेष‚स्व‚भाव‚ता त‚स्या अभाव‚मुप‚द‚र्श‚य‚न्नास्त्येको ‚{\tiny $_{lb}$}‚भाव इत्य‚भिद \quotelemma{ध्याद्} बौद्धो न तु रूपाणीन्द्रियार्थान् प्र‚तिक्षिप‚न् । स्यात् म‚ती ‚{\tiny $_{lb}$}‚रूपाद्य‚व्य‚तिरेकात् साम‚र्थ्य‚म‚प्य‚नेकं त‚त्क‚थ‚न्त‚देक‚मित्युच्य‚ते क‚थं वा त‚स्य श‚ब्दार्थ‚त्वं । ‚{\tiny $_{lb}$}‚न‚हि स्व‚ल‚क्ष‚णं श‚ब्दार्थ इत्युच्य‚ते । नानाभूत‚म‚पि साम‚र्थ्य‚भिन्न‚व‚त्स्व‚व्य‚तिरेकादेकार्थ‚{\tiny $_{lb}$}‚क्रियाकारित‚यैक‚प्र‚त्य‚व‚म ‚{\tiny $_{2}$}‚ र्ष‚हेतुत्वात् प‚र‚म्प‚र‚यैक‚मित्याख्याय‚ते । य‚थोक्त‚म् ।
	{\color{gray}{\rmlatinfont\textsuperscript{§~\theparCount}}}
	\pend% ending standard par
      ‚{\tiny $_{lb}$}‚
	  \bigskip
	  \begingroup
	
	    
	    \stanza[\smallbreak]
	  \flagstanza{\tiny\textenglish{...25}}{\normalfontlatin\large ``\qquad}एक‚प्र‚त्य‚व‚म‚र्ष‚स्य हेतुत्वाद्धीर‚भेदिनी [।]&‚{\tiny $_{lb}$}‚एक‚धा हेतुभावेन व्य‚क्तीनाम‚प्य‚भिन्न‚तेति ॥ [२५]{\normalfontlatin\large\qquad{}"}\&[\smallbreak]
	  
	  
	  
	  \endgroup
	‚{\tiny $_{lb}$}‚

	  
	  \pstart \leavevmode% starting standard par
	पुरुषाध्य‚व‚सायानिरोधे ‚{\tiny $_{3}$}‚ न श‚ब्दार्थ‚त्वं त‚स्य व्य‚व‚स्थाप्य‚ते । पुरुषोह्य‚नादिमिथ्या‚{\tiny $_{lb}$}‚भ्यास‚वास‚नाप‚रिपाक‚प्र‚भावाद‚न्त‚र्मात्राविप‚रिव‚र्तिन‚माकारं बाह‚येष्वेवारोप्य दृश्य‚{\tiny $_{lb}$}‚विक‚ल्प‚यो ‚{\tiny $_{4}$}‚ रेक‚त्व‚म्प्र‚तिप‚न्नः प‚र‚मार्थ‚त‚स्तु निर्विष‚या एव ध्व‚न‚यः । व्य‚क्तीनाम्विज्ञाना‚{\tiny $_{lb}$}‚कार‚स्य चार्थान्त‚रानुग‚माभावेनाभिलापागोच‚र‚त्वात् । य‚थाध्य‚व‚साय‚ञ्चाका ‚{\tiny $_{5}$}‚ र‚स्य ‚{\tiny $_{lb}$}‚स‚त्वात् । य‚थोक्तं \quotelemma{सूत्रे} ॥ ‚{\tiny $_{lb}$}‚ 
	    \pend% close preceding par
	  
	    
	    \stanza[\smallbreak]
	  \flagstanza{\tiny\textenglish{...26}}{\normalfontlatin\large ``\qquad}येन येन हि नाम्ना वै यो यो ध‚र्मोभिल‚प्य‚ते ।&‚{\tiny $_{lb}$}‚न स स‚म्विद्य‚ते त‚त्र ध‚र्माणां सा हि ध‚र्म‚तेति ॥ [२६]{\normalfontlatin\large\qquad{}"}\&[\smallbreak]
	  
	  
	  
	    \pstart  \leavevmode% new par for following
	    \hphantom{.}
	   ‚{\tiny $_{lb}$}‚त‚द‚य‚म‚त्र स‚म‚दायार्थो रूपादी ‚{\tiny $_{6}$}‚ नाङ्घ‚ट‚स्य च य‚था क्र‚म‚म‚नेक‚त्व‚मेक‚त्व‚ञ्च व‚हुव‚च‚नैक‚{\tiny $_{lb}$}‚व‚च‚नाभिध‚य‚त्वात् [।] त‚द्य‚था न‚क्ष‚त्राणि श‚शीत्येव‚मादिभिर‚नुमानाभासैः प‚रेण ‚{\tiny $_{lb}$}‚घ‚टादिश‚ब्द‚स्य विष‚यो ‚{\tiny $_{7}$}‚ योय‚मेकार्थोऽव‚य‚व्य‚भिधानोभ्युप‚ग‚तः स एव प्र‚तिक्षिप्य‚ते । न‚तु ‚{\tiny $_{lb}$}‚रूप‚र‚साद‚यः प‚र‚माणुस्व‚भावास्त‚था हि तेषाम्प्र‚त्येक‚मेकैकात्म‚क‚त्व‚मिष्ट‚मेव । केव‚ला ‚{\tiny $_{8}$}‚ ‚{\tiny $_{lb}$}‚स्त‚दातिस‚फ‚ल‚बीज‚व‚न्न स‚मुदाय‚मासाद‚य‚न्तीति निय‚त‚स‚होत्पाद‚त्व‚प‚रिदीप‚नायोक्तं ॥ ‚{\tiny $_{lb}$}‚ 
	    \pend% close preceding par
	  
	    
	    \stanza[\smallbreak]
	  \flagstanza{\tiny\textenglish{...27}}{\normalfontlatin\large ``\qquad}कामेष्ट‚द्र‚व्य‚कोऽश‚ब्दः प‚र‚माणुर‚तीन्द्रियः [।]&‚{\tiny $_{lb}$}‚ \leavevmode\ledsidenote{\textenglish{54b/msK}}कायेन्द्रियो न‚व‚द्र‚व्यो द‚श‚द्र‚व्योऽप‚रेन्द्रिय इति । [२७]{\normalfontlatin\large\qquad{}"}\&[\smallbreak]
	  
	  
	  
	    \pstart  \leavevmode% new par for following
	    \hphantom{.}
	  \href{http://sarit.indology.info/?cref=ak.2.22}{अभिध‚र्म‚कोशे २।२२}
	{\color{gray}{\rmlatinfont\textsuperscript{§~\theparCount}}}
	\pend% ending standard par
      ‚{\tiny $_{lb}$}‚

	  
	  \pstart \leavevmode% starting standard par
	य‚था तु प‚र‚माणूनामैन्द्रिय‚क‚त्व‚म‚नित्य‚त्व‚ञ्च त‚द्विस्त‚रेणोक्त‚म‚न्य‚त्रास्माभिः । ‚{\tiny $_{lb}$}‚य‚त्पुन‚रेत‚द्व‚हुव‚च‚नैक‚व‚च‚नाभिधेय‚त्वादिति त‚द्व्य‚भिचा ‚{\tiny $_{1}$}‚ रि । त‚थाहि य‚दैक‚स्याम‚पि ‚{\tiny $_{lb}$}‚ \leavevmode\ledsidenote{\textenglish{85/s}} योषिति ज‚ले सिक‚ताद्र‚व्ये वा दारा आपः सिक‚ता इति व्य‚व‚हारः । त‚दा किन्त‚त्र ‚{\tiny $_{lb}$}‚बाहुल्यं येनैवं भ‚व‚ति श‚क्तिभेद इति चेत् । स‚र्व्व‚त्रोच्छि ‚{\tiny $_{2}$}‚ न्न‚मिदानीमेक‚व‚च‚न‚मेक‚{\tiny $_{lb}$}‚श‚क्तेर‚भावात् । व‚स्त्व‚भेदाद‚न्य‚त्रैक‚व‚च‚न‚मिति चेत् । इहाप्य‚स्तु । त‚द‚य‚न्निर्व‚स्तुको ‚{\tiny $_{lb}$}‚निय‚मः क्रिय‚माणः स्वात‚न्त्र्य‚मिच्छायाः श‚ब्द‚प्र‚यो ‚{\tiny $_{3}$}‚ गे ख्याप‚य‚ति । एतेन त‚द‚पि ‚{\tiny $_{lb}$}‚प्र‚त्युक्तं य‚दाह \quotelemma{कुमारिलः} [।] ‚{\tiny $_{lb}$}‚ 
	    \pend% close preceding par
	  
	    
	    \stanza[\smallbreak]
	  \flagstanza{\tiny\textenglish{...28}}{\normalfontlatin\large ``\qquad}त‚त्र व्य‚क्तौ च जातौ च दारादिश्चेत्प्र‚युज्य‚ते ।&‚{\tiny $_{lb}$}‚व्य‚क्तेर‚व‚य‚वानाम्वा संख्यामादाय व‚र्त‚त [२८]{\normalfontlatin\large\qquad{}"}\&[\smallbreak]
	  
	  
	  
	    \pstart  \leavevmode% new par for following
	    \hphantom{.}
	  इति ॥ ‚{\tiny $_{lb}$}‚ष‚ण्ण‚ग‚रीति च क‚थ‚म्व‚हुष्वेक‚व‚च‚नं । न‚हि न‚ग‚राण्येव किञ्चित् कुत‚स्तेषां स‚माहारः । ‚{\tiny $_{lb}$}‚प्रासाद‚पुरुषादीनां विजातीयानाम‚नार‚म्भात् कुत‚स्त‚त्स‚मु ‚{\tiny $_{5}$}‚ दायो द्र‚व्यं असंयोगाच्च ‚{\tiny $_{lb}$}‚नापि संयोगः । प्रासादादीनां प‚र‚स्प‚र‚संयोगात् । प्रासाद‚स्य स्व‚यं संयोगात्म‚क‚स्य ‚{\tiny $_{lb}$}‚निर्गुण‚त‚याप‚रेणासंयोगाच्च । त‚त ‚{\tiny $_{6}$}‚ एव च संख्याभावः । त‚त्संयोग‚पुरुष‚विशिष्टा ‚{\tiny $_{lb}$}‚स‚त्ता न‚ग‚र‚मिति चेत् । किम‚स्यानिर‚तिस\edtext{}{\lemma{तिस}\Bfootnote{? श}}याया विशेष‚णं स‚त्तायाश्चैक‚त्वात् ‚{\tiny $_{lb}$}‚न‚ग‚र‚ब‚हुत्वेपि न‚ग‚राणीति ब‚हु ‚{\tiny $_{7}$}‚ व‚च‚नं स्यात् [।] द्व‚य‚स्य प‚र‚स्प‚र‚स‚हित‚तेति चेत् । ‚{\tiny $_{lb}$}‚अनुप‚कार‚क‚योः कः स‚हायीभावः । पुरुष‚संयोग‚स‚त्तानां च व‚हुत्वान्न‚ग‚र‚मिति क‚थ‚मेक‚{\tiny $_{lb}$}‚व‚च‚नं । त‚था भू ‚{\tiny $_{8}$}‚ तानां क्व‚चिद‚भिन्ना श‚क्तिः सा निमित्त‚मिति चेन्न । श‚क्तेर्व‚स्तु‚{\tiny $_{lb}$}‚रूपाव्य‚तिरेकात् । व्य‚तिरेके चानुप‚कार्य‚स्य पार‚त‚न्त्र्यायोगात् । उप‚कारे वा श‚{\tiny $_{lb}$}‚क्त्युप‚कारिण्या अपि श ‚{\tiny $_{9}$}‚ \leavevmode\ledsidenote{\textenglish{55a/msK}} क्तेर्व्य‚तिरेक इत्य‚व‚स्थितेर‚प्र‚तिप‚त्तिः । त‚द‚व्य‚तिरेके अन्यासा- ‚{\tiny $_{lb}$}‚म‚पि प्र‚संग इति य‚त्किञ्चिदेत‚त् । प्र‚कारान्त‚र‚म‚प्याह । \quotelemma{दृष्टोप‚द‚र्श‚न} \uline{श्चै} त‚दिति । ‚{\tiny $_{lb}$}‚किं पुनः प‚ञ्च‚म्य‚न्त ‚{\tiny $_{1}$}‚ निर्देशेपि दृष्टान्तो भ‚व‚तीत्याह । \quotelemma{कृत‚कानित्य‚त्वादि \cite[10b7]{vn-msN} ‚{\tiny $_{lb}$}‚ति} य‚था येनोक्तं । ‚{\tiny $_{lb}$}‚ 
	    \pend% close preceding par
	  
	    
	    \stanza[\smallbreak]
	  \flagstanza{\tiny\textenglish{...29}}{\normalfontlatin\large ``\qquad}हेतोः साध्यान्व‚यो य‚त्राभावेभाव‚श्च‚क‚थ्य‚ते ।&‚{\tiny $_{lb}$}‚प‚ञ्च‚म्या त‚त्र दृष्टान्तो हेतुस्तूप‚न‚याऽ ‚{\tiny $_{2}$}‚ त्म‚क [२९]{\normalfontlatin\large\qquad{}"}\&[\smallbreak]
	  
	  
	  
	    \pstart  \leavevmode% new par for following
	    \hphantom{.}
	   इति ॥ ‚{\tiny $_{lb}$}‚क्व‚चिद‚र्थे घ‚टादिद्र‚व्ये विप्र‚तिप‚त्तौ स‚त्यां रूपादिव्य‚तिरिक्त‚म‚स्ति नास्तीत्य‚ने‚{\tiny $_{lb}$}‚क‚स्यार्थ‚स्य प‚र‚स्प‚र‚व्यावृत्त‚स्य न‚ग‚रादेः सामान्यं ष‚ण्ण‚ग‚रीत्यादि ‚{\tiny $_{3}$}‚ य‚द् बुध्यारोपितं ‚{\tiny $_{lb}$}‚ \leavevmode\ledsidenote{\textenglish{86/s}} त‚त्र प्र‚सिद्धं श‚ब्द‚प्र‚योग‚माद‚र्श्य प‚र‚स्प‚र‚व्यावृत्तानामेकार्थान‚नुग‚तानां बुद्धिस‚माकृते ‚{\tiny $_{lb}$}‚स‚मूहे भाव‚श‚ब्द‚प्र‚योगादित्य‚नेन प‚श्चा ‚{\tiny $_{4}$}‚ दुप‚न‚येन प‚क्ष‚ध‚र्मोप‚संहार‚मागूर्य प्र‚तिपादित‚{\tiny $_{lb}$}‚विप्र‚तिप‚त्तिस्थानः स‚न्सामान्येनोप‚संह‚र‚ति । \quotelemma{स‚र्व्वं पृथ‚गि \cite[10b8]{vn-msN} ति} । प्र‚तिपादितं ‚{\tiny $_{lb}$}‚प्र‚तिप‚त्तिस्थान‚म ‚{\tiny $_{5}$}‚ नेनेति विग्र‚हः । एत‚दुक्त‚म्भ‚व‚ति । क‚पालादिव्य‚तिरेकेना\edtext{}{\lemma{तिरेकेना}\Bfootnote{? णा}}‚{\tiny $_{lb}$}‚व‚य‚व्य‚स्ति नास्तीति विवादे स‚त्य‚यं त्रिल‚क्ष‚ण‚हेतुसूच‚न‚प‚रो दृष्टान्त उप‚न्य‚स्तो न ‚{\tiny $_{lb}$}‚हेतुः ‚{\tiny $_{6}$}‚ । प्र‚योग‚स्त्व‚त्रैवं क्रिय‚ते । ये प‚र‚स्प‚र‚व्यावृत्ता न ते व्य‚तिरिक्तैकाव‚य‚विद्र‚व्यानु‚{\tiny $_{lb}$}‚ग‚त‚मूर्त्त‚यः । त‚द्य‚था ष‚ण्ण‚ग‚र्याद‚यः । त‚था च प‚र‚स्प‚र‚व्यावृत्ताः क‚पालाद‚य इ ‚{\tiny $_{7}$}‚ ति ॥
	{\color{gray}{\rmlatinfont\textsuperscript{§~\theparCount}}}
	\pend% ending standard par
      ‚{\tiny $_{lb}$}‚

	  
	  \pstart \leavevmode% starting standard par
	न‚नु च य‚द्य‚यं दृष्टान्त‚प्र‚योग‚स्त‚क्तिमृजुनैव त‚त्प्र‚योग‚क्र‚मेण न प्र‚युक्तो य‚था ‚{\tiny $_{lb}$}‚य‚त्स‚त् त‚त्क्ष‚णिकं य‚था घ‚ट इत्यादौ । किम्प‚ञ्च‚म्य‚न्त‚निर्देशेन । विप्र‚तिप‚त्ति ‚{\tiny $_{8}$}‚ ‚{\tiny $_{lb}$}‚विष‚य‚श्च किन्न द‚र्शितः क‚पालादेर‚व‚य‚विप्र‚तिषेध‚विशिष्टः । य‚थान्य‚त्रानित्यः श‚ब्दः ‚{\tiny $_{lb}$}‚कृत‚कानित्य‚त्वादिति । च‚कारात् स्प‚ष्ट‚श्च क‚स्मात् हेतुः साध्यानुग‚तो न प्र‚द ‚{\tiny $_{9}$}‚ \leavevmode\ledsidenote{\textenglish{55b/msK}} र्शितः । ‚{\tiny $_{lb}$}‚त‚थाह्य‚त्र प‚र‚स्प‚र‚व्यावृत्तानामेकार्थान‚नुग‚तानां बुद्ध्या स‚माहिते स‚मूह‚भाव‚श‚ब्द‚{\tiny $_{lb}$}‚प्र‚योगादित्य‚भ्यूह्य वाक्य‚प‚रिस‚माप्तिः क्रिय‚ते । अत्रोत्त‚रं न स‚मास‚निर्दे ‚{\tiny $_{1}$}‚ शात् ‚{\tiny $_{lb}$}‚संक्षेपाभिधानादित्य‚र्थः । एव‚म‚पि प्र‚योग‚द‚र्श‚नात् कृत‚कानित्य‚त्वादित्यादौ । ‚{\tiny $_{lb}$}‚असाध‚नं वाक्य‚त्वाच्च साध‚न‚प्र‚योगोत्प्रेक्षासूच‚कं वाक्य‚मेत‚त् । न‚त्विदं साध‚न‚{\tiny $_{lb}$}‚वा ‚{\tiny $_{2}$}‚ क्य‚मित्य‚र्थः । अत एवेति दृष्टान्त‚वाक्य‚त्वादेवेति । य‚श्चायं हेतुस्त‚न्तुप‚ट‚रूपे ‚{\tiny $_{lb}$}‚भिन्न‚कार‚णे विशेष‚व‚त्वाद्रूप‚स्प‚र्श‚व‚दिति ॥ अय‚म‚पि त‚न्तुप‚ट‚योर्भेदासिद्धौ त‚दा ‚{\tiny $_{3}$}‚ ‚{\tiny $_{lb}$}‚श्रित‚स्यापि गुण‚स्य विभागासिद्धेर‚सिद्धाश्र‚य इति नाल‚मिष्ट‚सिद्ध‚ये । त‚था हि सूक्ष्म‚स्‚{\tiny $_{lb}$}‚थूल‚द्र‚व्य‚स‚म‚वायो विशेष‚व‚त्वं भिन्न‚कालोत्प‚न्न‚द्र‚व्य‚संवाया ‚{\tiny $_{4}$}‚ वेति व्याच‚क्ष‚ते ।
	{\color{gray}{\rmlatinfont\textsuperscript{§~\theparCount}}}
	\pend% ending standard par
      ‚{\tiny $_{lb}$}‚

	  
	  \pstart \leavevmode% starting standard par
	प‚रे । न‚नु विचित्राभिस‚न्ध‚यः योक्तारः । त‚त्र ये केचिद्धेत्व‚भिप्रायेनैव\edtext{}{\lemma{भिप्रायेनैव}\Bfootnote{? णैव}} ‚{\tiny $_{lb}$}‚ वाचः प्र‚युञ्ज‚ते तान्प्र‚त्य‚स्माभिः प्र‚तिज्ञ‚या हेतोर्बाध‚न‚मु ‚{\tiny $_{5}$}‚ च्य‚ते न तु ये दृष्टान्ताभि‚{\tiny $_{lb}$}‚मानिन इत्य‚त्राह [।] \quotelemma{न‚चे \cite[10b9]{vn-msN} त्यादि} । भ‚ग‚व \quotelemma{त्त‚थाग‚त} म‚ताव‚ल‚म्बिनामुप‚र्य‚य‚मु‚{\tiny $_{lb}$}‚प‚क्षिप्तो विरोधो भ‚व‚द्भि \quotelemma{राक्ष‚पादै} र्न च नः स्व‚प्न ‚{\tiny $_{6}$}‚ व्ये तादृशोस्तीति पिण्डार्थः । ‚{\tiny $_{lb}$}‚स्यात् म‚त‚म‚स्त्येव \quotelemma{योगाचारो यः} [—]
	{\color{gray}{\rmlatinfont\textsuperscript{§~\theparCount}}}
	\pend% ending standard par
      ‚{\tiny $_{lb}$}‚

	  
	  \pstart \leavevmode% starting standard par
	\leavevmode\ledsidenote{\textenglish{87/s}} 
	    \pend% close preceding par
	  
	    
	    \stanza[\smallbreak]
	  \flagstanza{\tiny\textenglish{...30}}{\normalfontlatin\large ``\qquad}प‚ङ्केन युग‚प‚द्योगात् प‚र‚माणोः प‚त‚ङ्ग‚तां ।&‚{\tiny $_{lb}$}‚ष‚ण्णां स‚मान‚देश‚त्वात् पिण्डः स्याद‚णुमात्र‚क [ः ॥ ३०]{\normalfontlatin\large\qquad{}"}\&[\smallbreak]
	  
	  
	  
	    \pstart  \leavevmode% new par for following
	    \hphantom{.}
	   ‚{\tiny $_{lb}$}‚इ‚{\tiny $_{7}$}‚ त्यादिना प‚र‚माणोरेक‚त्व‚म‚न‚भ्युप‚ग‚च्छ‚न्न‚पि पिण्डं स‚मूहाप‚र‚प‚र्याय‚मिच्छ‚ती‚{\tiny $_{lb}$}‚त्येत‚दुच्य‚ते । \quotelemma{योपी \cite[11a1]{vn-msN} त्यादि} किन्त‚र्ह्य‚भाव एवाणोर‚नेन प्र‚कारेण साध‚यितु ‚{\tiny $_{8}$}‚ ‚{\tiny $_{lb}$}‚मिष्टः । क‚थं । एकानेक‚प्र‚तिषेधात् । प‚ङ्कायोगादिना ताव‚देक‚त्वं प्र‚तिसि\edtext{}{\lemma{तिसि}\Bfootnote{? षि}}द्धं । ‚{\tiny $_{lb}$}‚त‚त्स‚मुदाय‚रूप‚म‚नेक‚त्व‚म‚पि त‚द‚भावादेव न विद्य‚ते । य‚थोक्त‚न् न‚नु  [।] ‚{\tiny $_{lb}$}‚ 
	    \pend% close preceding par
	  
	    
	    \stanza[\smallbreak]
	  \flagstanza{\tiny\textenglish{...31}}{\normalfontlatin\large ``\qquad}त‚स्य त‚स्यै ‚{\tiny $_{9}$}‚\leavevmode\ledsidenote{\textenglish{56a/msK}} क‚ता नास्ति यो यो भावः प‚रीक्ष्य‚ते ।&‚{\tiny $_{lb}$}‚न स‚न्ति तेनानेकेपि येनैकोपि न विद्य‚त [३१]{\normalfontlatin\large\qquad{}"}\&[\smallbreak]
	  
	  
	  
	    \pstart  \leavevmode% new par for following
	    \hphantom{.}
	   इति ॥ ‚{\tiny $_{lb}$}‚न‚नु ‚{\tiny $_{1}$}‚ प‚ङ्क‚योगादिना क‚थ‚मेक‚त्व‚म‚पोदितं । याव‚ता त‚त्र त‚स्य साव‚य‚व‚{\tiny $_{lb}$}‚त्व‚मापादितं ॥ त एव चाव‚य‚वास्त‚स्याल्पीयांसः प‚र‚माण‚वो वि ‚{\tiny $_{2}$}‚ भाग‚प‚र्य‚व‚सान‚ल‚क्ष‚{\tiny $_{lb}$}‚ण‚त्वात् प‚र‚माणूनां । अथ तेषाम‚प्य‚ङ्गानाम‚नेनैव विधानेन साव‚य‚व‚त्व‚मापाद्य‚ते । त‚था ‚{\tiny $_{lb}$}‚स‚ति त‚त्राप्येत‚देवोत्त‚र‚मित्य‚नेनैव प्र‚कारे ‚{\tiny $_{3}$}‚ ण न श‚क्य‚ते प‚र‚माणोरेक‚त्व‚निषेधं क‚र्त्तु । ‚{\tiny $_{lb}$}‚विभाग‚स्य विभ‚ज्य‚माण\edtext{}{\lemma{माण}\Bfootnote{? न}}त‚न्त्र‚त्त्वात् । क‚थ‚ञ्चान‚भ्युप‚ग‚ताणुस्त‚स्य प‚ङ्क‚योगा‚{\tiny $_{lb}$}‚दिक‚म‚भ्युप‚ग‚च्छ‚तीति त ‚{\tiny $_{4}$}‚ द‚स‚त्व‚प्र‚तिपाद‚ने स‚र्वे हेत‚वः स्व‚त एवाश्र‚यासिद्धा इति । ‚{\tiny $_{lb}$}‚एत‚च्च नैवं य‚स्मात्स‚म‚र्था वादिनोऽप‚ग‚ताव‚य‚व‚विभाग‚मासादिताप‚क‚र्ष‚य‚न्तं ‚{\tiny $_{lb}$}‚भाव ‚{\tiny $_{5}$}‚ म‚णुरित्याच‚क्ष‚ते त‚स्य तेन प‚ङ्कायोगादिनैक‚त्व‚म‚पाक्रिय‚ते । ते च य‚द्येवं ‚{\tiny $_{lb}$}‚निराकृताः स‚न्तो य‚थोप‚ग‚त‚स्य साव‚य‚व‚त्वं प्र‚तिप‚द्य‚न्ते त‚दा स्व ‚{\tiny $_{6}$}‚ प्र‚तिज्ञायाश्च्य‚वे‚{\tiny $_{lb}$}‚र‚न् । न हि अन‚ङ्गीकृत‚साव‚य‚व‚त्वास्त‚था प्र‚त्य‚व‚स्थान‚म‚र्ह‚न्ति । त एवाव‚य‚वाः स‚न्तु ‚{\tiny $_{lb}$}‚प‚र‚माण‚व इति । तैरेव च त‚ल्ल‚क्ष‚ण‚म्व्य‚व‚स्थाप‚नीयं \quotelemma{योगाचा ‚{\tiny $_{7}$}‚ रे} ण च निषेध्य‚मिति ‚{\tiny $_{lb}$}‚निगृह्य‚न्ते । अत एव नान‚व‚स्था । प्र‚स‚ङ्ग‚साध‚न‚त्वाच्चासिद्ध‚तादोषोपि नास्तीत्य‚ल‚{\tiny $_{lb}$}‚मेतेन । अथोच्य‚ते न व‚यं भ‚व‚न्तं प्र‚तीदं ब्रूमो य‚स्तु क‚श्चिद‚धौ ‚{\tiny $_{8}$}‚ त‚पादो वाद्येवं ‚{\tiny $_{lb}$}‚ \leavevmode\ledsidenote{\textenglish{88/s}} प्राह त‚म्प्र‚तीति । त‚च्चास‚त्स‚र्व्वं पृथ‚ग्भाव‚ल‚क्ष‚ण‚पृथ‚ग्त्वात् नानेक‚ल‚क्ष‚णे‚{\tiny $_{lb}$}‚नैक‚भाव‚निष्प‚त्तेरित्य‚त्र प्र‚स्तावे \quotelemma{भार‚द्वाजे} नास्मान्प्र‚त्येव \quotelemma{कामेष्ट‚द्र‚व्य ‚{\tiny $_{9}$}‚ \leavevmode\ledsidenote{\textenglish{56b/msK}} क} इत्यादिना ‚{\tiny $_{lb}$}‚सिद्धान्त‚म‚स्माकीन‚मुप‚क्षिप्याप्य‚भिधानात् । त‚थाप्य‚भ्युप‚ग‚म्य दोषान्त‚र‚माह । न ‚{\tiny $_{lb}$}‚चाय‚म्पूर्व‚काद् गुण‚व्य‚तिरिक्त‚मित्यादिप‚द‚सूचितात् प‚र‚स्प‚रार्थ‚मा ‚{\tiny $_{1}$}‚ धाय भिद्य‚ते । ‚{\tiny $_{lb}$}‚हेतुप्र‚तिज्ञ‚योः स‚म्ब‚न्धिन्योः बाध‚योरुदाह‚र‚णोपेत‚योः पृथ‚ग्बाधोदाह‚र‚ण‚योर्न क‚श्चि‚{\tiny $_{lb}$}‚द‚र्थ‚भेदः श‚ब्द‚भेद‚स्तु केव‚लः । त‚थाविध‚स्य च पृथ‚गुदा ‚{\tiny $_{2}$}‚ ह‚र‚णेऽतिप्र‚स‚ङ्गोऽकृत‚कः ‚{\tiny $_{lb}$}‚श‚ब्दः कृत‚क‚त्वादित्याद्य‚प्युदाह‚र्त्त‚व्य‚म्भ‚वेत । स‚ह पृथ‚ग्वेति क्व‚चित्पाठः । त‚त्राय‚{\tiny $_{lb}$}‚म‚र्थः स‚ह यौग‚प‚द्येन य‚था प्र‚थ‚मे पृथ‚क् प्र‚त्येकं । य ‚{\tiny $_{3}$}‚ थेह अथ‚वा विरोध‚चिन्ताप्य‚त्रा‚{\tiny $_{lb}$}‚युक्तेत्याह [।] \quotelemma{अपिचे} \cite[11a5]{vn-msN} त्यादि । स‚र्व्वं पृथ‚क् स‚मूहे भाव‚श‚ब्द‚प्र‚यो[गा] ‚{\tiny $_{lb}$}‚दित्य‚यं हेतुः । स‚र्व्व‚स्य ध‚र्मिणो ध‚र्म एव न भ‚व‚ति श‚ब्द‚ध‚र्म ‚{\tiny $_{4}$}‚ त्वादित्य‚सिद्धः । ‚{\tiny $_{lb}$}‚त‚था च व्य‚धिक‚र‚ण‚त्वाद‚सिद्ध‚तैव दोषो गुडो म‚धुरः काक‚स्य कार्ष्ण्यादिति य‚था । ‚{\tiny $_{lb}$}‚त‚त्र न विरोधो भिन्नाधिक‚र‚ण‚त्वाद्धेतुप्र‚ति ‚{\tiny $_{5}$}‚ ज्ञार्थ‚योः । स्याद् बुद्धिः स‚मूह‚वाच‚क‚{\tiny $_{lb}$}‚श‚ब्द‚वाच्य‚त्वादित्येवं \quotelemma{भाविविक्तेन} भाष्य‚टीकायां प्र‚योगाद् व्य‚धिक‚र‚ण‚त्वं नास्ति । ‚{\tiny $_{lb}$}‚एवं म‚न्य‚ते न ताव‚द‚य \quotelemma{मु ‚{\tiny $_{6}$}‚ द्योत‚क‚रे} णैवं प्र‚युक्त‚स्य वाय‚म‚स्माभिर्दोषोभिधातुमार‚ब्धो ‚{\tiny $_{lb}$}‚येपि स‚म्प्र‚त्य‚न्य‚था प्र‚युञ्ज‚ते तेषाम‚पि य‚द्य‚यं दोषो न भ‚व‚ति । भ‚व‚तु अन‚न्त‚रोक्त‚स्तु ‚{\tiny $_{lb}$}‚दोषो व‚क्ष्य ‚{\tiny $_{7}$}‚ माण‚श्च ब्र‚ह्म‚णाऽपि न श‚क्य‚ते प‚रिह‚र्तुमिति । प्र‚तिज्ञाहेत्वोर्विरोध‚स्य च ‚{\tiny $_{lb}$}‚निग्र‚ह‚स्थानान्त‚र‚त्व‚म‚ङ्गीकृत्य म‚येद‚म‚भ्य‚धायि । न त्व‚स्य त‚द्युक्तं । हेत्वाभासा‚{\tiny $_{lb}$}‚श्च निग्र‚ह ‚{\tiny $_{8}$}‚ स्थानानी \href{http://sarit.indology.info/?cref=ns\%C5\%AB.5.2.24}{न्या० सू० ५।२।२४ } त्य‚नेनैव स‚ङ्गृहीत‚त्वादित्येत‚द्विभ‚णि‚{\tiny $_{lb}$}‚षुराह । \quotelemma{अपिचे} \cite[11a5]{vn-msN} त्यादि । द्वाव‚व‚य‚वौ य‚स्या दोष‚जातेर्दो ‚{\tiny $_{9}$}‚ \leavevmode\ledsidenote{\textenglish{57a/msK}} ष‚प्र‚कार‚स्य सा द्व‚यी । ‚{\tiny $_{lb}$}‚कामित्याह । विरुद्ध‚ताम‚सिद्ध‚ताञ्च । क‚थ‚म्पुन‚र्विरुद्ध‚तेत्याह । विरुद्ध‚तेत्यादि । अय‚म‚त्र ‚{\tiny $_{lb}$}‚संक्षेपार्थः । प्र‚तिज्ञाहेत्वोर्य‚त्र प्र‚योगेंविरोध‚श्चोद्य‚ते त‚त्रा ‚{\tiny $_{1}$}‚ व‚श्यं सिद्धेन ध‚र्मिणा भाव्यं । ‚{\tiny $_{lb}$}‚सिद्धे च त‚स्मिन्ध‚र्म‚णि\edtext{}{\lemma{णि}\Bfootnote{? ध‚र्मिणि}}हेतोर्वा स‚त्व‚म्भ‚वेत् साध्य‚ध‚र्म‚स्य । द्व‚योर्वा । त‚त्र न ‚{\tiny $_{lb}$}‚ताव‚त् द्व‚योर‚पि स‚त्वं प ‚{\tiny $_{2}$}‚ र‚स्प‚र‚विरोधित्वेन शीतोष्ण‚योरिव एकाधिक‚र‚ण‚त्वाभावात् । ‚{\tiny $_{lb}$}‚अन्य‚था स‚हैक‚त्राव‚स्थानाद्र‚स‚रूप‚व‚द‚विरोध एव भ‚वेदिति प्र‚तिज्ञाहेत्वोर्विरोधो दूर‚त‚र ‚{\tiny $_{lb}$}‚एव ‚{\tiny $_{3}$}‚ प्र‚स‚ज्य‚ते । त‚द्व‚क्ष्य‚ति । विरुद्ध‚योः स्व‚भाव‚योरेक‚त्रास‚म्भ‚वान्न चान्य‚था विरोध ‚{\tiny $_{lb}$}‚इति । अथ हेतोस्त‚त्र स‚त्वं । एव‚म‚पि य‚त्र हेतुस्त‚त्र न साध्य‚ध‚र्म‚स्त‚द्विप ‚{\tiny $_{4}$}‚ र्य‚य‚स्तु विद्य‚त ‚{\tiny $_{lb}$}‚ \leavevmode\ledsidenote{\textenglish{89/s}} इति व्य‚क्त‚म‚स्य विरुद्ध‚त्वं । नित्यः श‚ब्दः कृत‚क‚त्वादिव‚त् । त‚दाह [।] \quotelemma{विरुद्ध‚ता ‚{\tiny $_{lb}$}‚सिद्धेर्हेत्वोर्ध‚र्मिणि भाव} \cite[11a6]{vn-msN} इति । य‚दा पुन‚स्त‚स्मिन्ध‚र्मिणि प्र‚मा ‚{\tiny $_{5}$}‚ णान्त‚रेण ‚{\tiny $_{lb}$}‚साध्य‚ध‚र्म‚स्य स‚त्वं निश्चितं त‚दा त‚त्र हेतोर‚वृत्तिर्विरोधिना क्रोडीकृत‚त्त्वात् । ‚{\tiny $_{lb}$}‚अत‚श्चासिद्ध‚त्वं हेतोः । कृत‚कः श‚ब्दोऽकार्य‚त्वादिति य‚था । त‚ज्जा ‚{\tiny $_{6}$}‚ ते असिद्ध‚ता ‚{\tiny $_{lb}$}‚पुन‚र्द्ध‚र्मिणीत्यादि । अथ‚म‚न्य‚से । प्र‚माणेन सिद्ध एव गुण‚व्य‚तिरिक्ते द्र‚व्यादौ ‚{\tiny $_{lb}$}‚ध‚र्मिणि प्र‚तिज्ञाहेतोर्विरोधो व्य‚व‚स्थाप्य‚ते त‚तो नायं दोष इत्य ‚{\tiny $_{7}$}‚ त इद‚मास‚ङ्क‚ते\edtext{}{\lemma{ते}\Bfootnote{‚{\tiny $_{lb}$}‚? श‚ङ्क‚ते}}\quotelemma{असिद्ध \cite[11b8]{vn-msN} इत्यादिना} । एव‚म‚पि य‚दि नाम ध‚र्म्य‚भावेन ‚{\tiny $_{lb}$}‚प‚क्ष‚ध‚र्म‚स्यास‚म्भ‚वात् विरुद्ध‚त्वं प‚रिहृतं । असिद्ध‚त्वं पुन‚स्त‚द‚व‚स्थ‚मेवेति म‚न‚स्या ‚{\tiny $_{8}$}‚ ‚{\tiny $_{lb}$}‚धायाह । प्र‚माण‚योगे तूभ‚योर्वादिप्र‚तिवादिनोर्ध‚र्मिणि हेतोर्वृत्तिसंश‚यः । प्र‚माण ‚{\tiny $_{lb}$}‚निवृत्ताव‚प्य‚र्थाभावासिद्धेः । अत‚श्चासिद्ध‚तैव स‚न्दिग्धाश्र‚य‚त्वात् । इह नि ‚{\tiny $_{9}$}‚ \leavevmode\ledsidenote{\textenglish{57b/msK}} कुञ्जे ‚{\tiny $_{lb}$}‚म‚यूरः केकायित‚त्त्वादित्यादिव‚त् । तु श‚ब्दः प्र‚तिपाद‚क‚प्र‚माणायोगे ध‚र्मिणः ‚{\tiny $_{lb}$}‚स‚न्दिग्धाश्र‚य‚ताहेतोर्ध‚र्मिबाध‚क‚प्र‚माण‚वृत्तौ स्फुट‚मेवाश्र‚यासिद्ध‚त‚त्वं । स‚र्व्व‚ग‚त आ ‚{\tiny $_{1}$}‚ त्म‚नि ‚{\tiny $_{lb}$}‚साध्ये स‚र्व‚त्रोप‚ल‚भ्य‚माण\edtext{}{\lemma{माण}\Bfootnote{? न}}गुण‚त्व‚व‚दित्य‚स्य स‚मुच्च‚यार्थः । त‚था ह्य‚सिद्धेः ‚{\tiny $_{lb}$}‚ध‚र्मिस्व‚भाव इत्य‚त्र प्र‚तिपाद‚क‚प्र‚माणावृत्तेर‚सिद्धो ध‚र्मी विव‚क्षितः स्यात् । ‚{\tiny $_{lb}$}‚ \leavevmode\ledsidenote{\textenglish{90/s}} बाध‚क ‚{\tiny $_{2}$}‚ प्र‚माण‚वृत्तेर्वा । पूर्व‚स्मिन्प‚क्षे क‚ण्ठेनैवोक्तो दोष उत्त‚र‚त्र श‚ब्देन स‚मुच्चितः । ‚{\tiny $_{lb}$}‚अत्रौद्योत‚क‚र‚मुत्त‚र‚माश‚ङ्क‚ते । \quotelemma{उभ‚याश्र‚य‚त्वा \cite[11a9]{vn-msN} दित्यादिना} । ग‚तार्थ‚त्वात् ‚{\tiny $_{lb}$}‚सुज्ञानं ‚{\tiny $_{3}$}‚ स‚र्व्व‚मेत‚त् । न स‚र्व्व‚त्रेत्यादिना निराक‚रोति । य‚थोक्तं प्राग् न द्व‚यीं दोष‚{\tiny $_{lb}$}‚जातिमित्य‚त्र । अथ प्र‚तिज्ञामात्र‚भाव्येव हेत्व‚न‚पेक्षः प्र‚तिज्ञाविरोधो व्य‚व‚स्था ‚{\tiny $_{4}$}‚ प्य‚ते ‚{\tiny $_{lb}$}‚य‚था नास्त्यात्मा श्र‚म‚णा ग‚र्भिणीत्य‚त्रेत्य‚त आह [।] \quotelemma{अन‚पेक्षे च हेतुग्र‚ह‚ण‚म‚स‚म्व‚द्धं} ‚{\tiny $_{lb}$}‚ \cite[11b3]{vn-msN} । अनुप‚कार‚क‚त्वात् । य‚द‚पीदं हेतुविरोध‚स्योदाह‚र ‚{\tiny $_{5}$}‚ णं द‚त्तं नित्यः श‚ब्द ‚{\tiny $_{lb}$}‚इत्यादिना त‚त्प्र‚तिज्ञाविरोध‚स्य हेतुनायुक्त‚मिति क‚थ‚नायाह । न \quotelemma{चेदि \cite[11b3]{vn-msN} ‚{\tiny $_{lb}$}‚ त्यादि} । स्यात् म‚त‚मुभ‚याश्र‚य‚त्वाद्विरोध‚स्यैव‚म‚पि न हेतुत ‚{\tiny $_{6}$}‚ एवेत्य‚त उच्य‚ते ‚{\tiny $_{lb}$}‚ \quotelemma{उभ‚याश्र‚येपी \cite[11b4]{vn-msN} त्यादि} । एव‚मुप‚द‚र्शितान्युदाह‚र‚णानि प्र‚क्षिप्यातिदिष्ट‚{\tiny $_{lb}$}‚दूष‚णायाह । य‚च्चोक्त‚मेतेन प्र‚तिज्ञायाः दृष्टान्त‚विरोधाद‚योपि ‚{\tiny $_{7}$}‚ व‚क्त‚व्या \quotelemma{भ‚ण्डा‚{\tiny $_{lb}$}‚लेख्य‚न्या} येने \cite[11b5]{vn-msN} ति । इति श‚ब्दो व‚क्त‚व्य इत्य‚त्र प्र‚तिप‚त्त‚व्योऽन्य‚थाप‚रे‚{\tiny $_{lb}$}‚णोत्त‚र‚स्याप्र‚युक्त‚त्त्वात् दुःश्लिष्टो भ‚वेत । भ‚ण्ड‚ग्र‚ह‚ण‚न्नित्य‚पुरुषोप‚ल‚क्ष‚णार्थं । ‚{\tiny $_{lb}$}‚य‚था हि भ‚ण्डा प्राकृतान् विस्माप‚य‚न्त‚श्चित्र‚ल‚क्ष‚णोपेत‚क‚पिशाल‚भ‚ञ्जिकादिप्र‚ति‚{\tiny $_{lb}$}‚च्छ‚न्द‚क‚मालिख्य विचित्र‚शिल्प‚क‚लाकौश‚ल‚सादि\edtext{}{\lemma{सादि}\Bfootnote{? शालि}}नोऽतिदि ‚{\tiny $_{9}$}‚ \leavevmode\ledsidenote{\textenglish{58a/msK}} शंत्येवं प्र‚का‚{\tiny $_{lb}$}‚राण्य‚प्य‚स्म‚त्कौश‚ल‚निर्मितान्येक‚ताल‚मात्रेण ह‚स्त्यादिरूप‚क‚स्थानानि प्र‚तिप‚त्त‚व्या‚{\tiny $_{lb}$}‚नीति त‚था जातीय‚क‚मेत‚दु \quotelemma{द्योत‚क‚र‚स्य} । त‚था ह्येत‚देव भाव उप ‚{\tiny $_{1}$}‚ द‚र्शित‚हेतुविरोधा‚{\tiny $_{lb}$}‚दिकं हेत्वाभास‚व्य‚तिरिक्त‚ल‚क्ष‚णोपेतं । त‚द‚तिदिष्टे पुनः कैव चिन्ता । तामेव चाति‚{\tiny $_{lb}$}‚दिष्ट‚स्य दृष्टान्त‚विरोधादेर्हेत्वाभास‚व्य‚तिरिक्त ‚{\tiny $_{2}$}‚ ल‚क्ष‚णापेत‚ताम‚भिधातुमुप‚क्र‚म‚ते । ‚{\tiny $_{lb}$}‚ \quotelemma{त‚त्रापी} \cite[11b5]{vn-msN} त्यादिना । य‚त्र प्र‚तिज्ञायाः दृष्टान्त‚विरोध‚स्त‚त्रापि प‚क्षीकृत‚ध‚र्म‚वि‚{\tiny $_{lb}$}‚प‚र्य‚य‚व‚ति दृष्टान्ते स‚ति विरोधः स्यात् प्र ‚{\tiny $_{3}$}‚ तिज्ञायाः दृष्टान्तेनेति शेषः । प‚क्षीकृत‚श्चा‚{\tiny $_{lb}$}‚ \leavevmode\ledsidenote{\textenglish{91/s}} सौ ध‚र्म‚श्च त‚स्य विप‚र्य‚यः स विद्य‚ते य‚स्मिन्निति विग्र‚हः । दृष्टान्त इति च साध‚र्म्य‚दृ‚{\tiny $_{lb}$}‚ष्टान्तो । अभिप्रेतः । य‚स्मा ‚{\tiny $_{4}$}‚ द्वैध‚र्म्य‚दृष्टान्तः साध्य‚ध‚र्म‚विप‚र्य‚य‚वानेव त‚त्र को विरोधः । ‚{\tiny $_{lb}$}‚त‚त्रोदाह‚र‚णं । नित्यः श‚ब्दो घ‚ट‚व‚दिति । विरुद्धे च दृष्टान्ते स‚ति य‚दि प‚क्ष‚ध‚र्म‚स्य वृत्ति ‚{\tiny $_{5}$}‚ ‚{\tiny $_{lb}$}‚र‚न‚न्य‚साधार‚णा प्र‚साध्य‚ते प्र‚माणेन विरुद्ध‚स्त‚दा हेत्वाभासः । नान्य‚साधार‚णेत्य‚न‚न्य‚{\tiny $_{lb}$}‚साधार‚णा । अन्य‚श‚ब्देन प‚क्षीकृत‚ध‚र्म‚विप‚र्य‚य‚व‚तः ‚{\tiny $_{6}$}‚ पृथ‚ग्भूतः प‚क्षीकृत‚ध‚र्म‚वान‚भि‚{\tiny $_{lb}$}‚प्रेतः प‚क्षीकृत‚ध‚र्म‚विप‚र्य‚य‚व‚त्येव‚व‚र्त‚ते इत्येवं य‚दि साध्य‚त इत्य‚र्थः । य‚थान‚योरेव ‚{\tiny $_{lb}$}‚साध्य‚दृष्टान्त‚योः कार्य‚त्वादि ‚{\tiny $_{7}$}‚ ति त‚द्विप‚क्षीकृत‚ध‚र्म‚ब‚हिर्व्योमादौ न व‚र्त‚ते त‚द्विप‚रीते ‚{\tiny $_{lb}$}‚पुन‚र्घ‚टे व‚र्त‚त इति । साधार‚णायाम्वृत्तौ साधितायां स‚प‚क्ष‚विप‚क्ष‚योरिति शेषः । अनै‚{\tiny $_{lb}$}‚कान्तिकः ‚{\tiny $_{8}$}‚ साधार‚णाख्यः । य‚थान‚योरेव साध्योदाह‚र‚ण‚योः प्र‚मेय‚त्वादिति । अप्र‚सा‚{\tiny $_{lb}$}‚धिते चात‚द्वृत्तिनिय‚मे त‚योः स‚प‚क्ष‚विप‚क्ष‚योर्वृत्तिनिय‚मे स‚प‚क्ष एव व‚र्त्त‚ते वि ‚{\tiny $_{9}$}‚ \leavevmode\ledsidenote{\textenglish{58b/msK}} प‚क्ष ‚{\tiny $_{lb}$}‚एवेति अनैकान्तिक एव स‚न्दिग्धान्व‚यः स‚न्दिग्ध‚व्य‚तिरेको वा । य‚था स‚र्व्व‚विद्वीत‚{\tiny $_{lb}$}‚रागो वा विव‚क्षितः पुरुषो न वा त‚था व‚क्तृत्वाद्र‚थ्यान‚र‚व‚दिति । त‚योरेव स‚प‚क्ष ‚{\tiny $_{1}$}‚ ‚{\tiny $_{lb}$}‚विप‚क्ष‚योर‚वृत्तौ वा स‚त्याम‚साधार‚णः । नित्यः श‚ब्दः श्राव‚ण‚त्वादिति य‚था । प‚रः प्राह ‚{\tiny $_{lb}$}‚विरुद्ध‚दृष्टान्तावृत्तौ हेतोर्विप‚र्य‚य‚वृत्तौ च स‚त्यान्न क‚श्चिद्धेतुदोषः ‚{\tiny $_{2}$}‚ त‚द्य‚थाऽनित्यः ‚{\tiny $_{lb}$}‚श‚ब्दः प्र‚त्य‚य‚भेद‚भेदित्वात् न‚भोव‚दिति साध‚र्म्येण । वैध‚र्म्येण च घ‚ट‚व‚दिति । अत्र ‚{\tiny $_{lb}$}‚नासिद्ध‚त्वं ध[ि]र्म‚णि हेतोः स‚द्भावात् । नाप्य‚नैकान्तिक‚त्व‚मुभ ‚{\tiny $_{3}$}‚ य‚त्रावृत्तेः । ‚{\tiny $_{lb}$}‚प्र‚तिब‚न्ध‚स‚द्भावाच्च । न च विरुद्ध‚त्वं स‚प‚क्ष‚विप‚क्ष‚योर्वैप‚रीत्येन वृत्य‚भावात् । ‚{\tiny $_{lb}$}‚दृष्टान्तेन तु विरोधः प्र‚तिज्ञायाः इत्य‚यं हेतुदोषान‚ति ‚{\tiny $_{4}$}‚ क्रान्तो विष‚यः प्र‚तिज्ञायाः ‚{\tiny $_{lb}$}‚दृष्टान्तेन च विरोध‚स्येति । इद‚म‚प‚नुद‚ति । न । त‚दापि संश‚य‚हेतुत्वान‚तिवृत्तेः । ‚{\tiny $_{lb}$}‚य‚स्माद् दृष्टान्ते न प्र‚तिज्ञाया विरोधः सा ‚{\tiny $_{5}$}‚ ध‚र्म्ये दृष्टान्ते दोषो न वैध‚र्म्ये । क‚स्माद‚{\tiny $_{lb}$}‚भिम‚त‚त्वाद् विरोध‚स्य । प‚क्षीकृत‚ध‚र्म‚विप‚र्य‚य‚वानेव हि वैध‚र्म्यंदृष्टान्त उच्य‚त इत्य‚{\tiny $_{lb}$}‚भिप्रायः । य‚दि ना ‚{\tiny $_{6}$}‚ मैवं त‚थापि क‚थं हेत्वाभासान‚तिवृत्तिरित्याह साध‚र्म्य‚दृष्टान्ते ‚{\tiny $_{lb}$}‚च विप‚रीत‚ध‚र्म‚व‚ति न‚भ‚सि नाऽव्य‚भिचार‚ध‚र्म‚ता श‚क्या द‚र्श‚यितुँ । त‚द‚र्थ‚श्च दृष्टान्तः ‚{\tiny $_{lb}$}‚प्र‚द‚र्श‚ते ॥ ‚{\tiny $_{7}$}‚ य‚दाह ‚{\tiny $_{lb}$}‚ \leavevmode\ledsidenote{\textenglish{92/s}} 
	    \pend% close preceding par
	  
	    
	    \stanza[\smallbreak]
	  \flagstanza{\tiny\textenglish{...32}}{\normalfontlatin\large ``\qquad}त्रिरूपो हेतुरित्युक्तं प‚क्ष‚ध‚र्मे च संस्थितः ।&‚{\tiny $_{lb}$}‚रूढे रूप‚द्व‚यं शेषं दृष्टान्तेन प्र‚द‚र्श्य‚त [३२]{\normalfontlatin\large\qquad{}"}\&[\smallbreak]
	  
	  
	  
	    \pstart  \leavevmode% new par for following
	    \hphantom{.}
	  इति । ‚{\tiny $_{lb}$}‚न‚नु च क‚थ‚म‚श‚क्या याव‚ता प्र‚त्य‚य‚भेद‚भेदित्व‚म‚नित्य‚त्वाव्य‚भि ‚{\tiny $_{8}$}‚ चार्येव त‚त्व‚त ‚{\tiny $_{lb}$}‚इत्य‚त आह । \quotelemma{व‚स्तुतः साध्याव्य‚भिचारेपी \cite[11b7]{vn-msN} ति} । विद्य‚मानोप्य‚व्य‚भिचारः ‚{\tiny $_{lb}$}‚प्र‚माणेनाप्र‚तिपादित‚त्वाद‚स‚त्क‚ल्प इति भावः । त‚देत‚न्नाप्र‚द‚र्शितावि ‚{\tiny $_{9}$}‚ \leavevmode\ledsidenote{\textenglish{59a/msK}} नाभाव‚स‚म्ब‚द्धा‚{\tiny $_{lb}$}‚द्धेतोः साध्य‚निश्च‚यः । त‚त्त‚स्मान्न प्र‚तिज्ञाया दृष्टान्त‚विरोधोपि हेत्वाभासान‚तिव‚र्त‚ते । ‚{\tiny $_{lb}$}‚अस्यापि त‚दानीं संदिग्ध‚विप‚क्ष‚व्यावृत्तिक‚त्वादित्यागूरितं ‚{\tiny $_{1}$}‚ । न केव‚ल‚हेतुविरोध ‚{\tiny $_{lb}$}‚इत्य‚पि श‚ब्दः प‚र‚म‚त‚मास\edtext{}{\lemma{मास}\Bfootnote{? श}}ङ्क‚ते । उभ‚य‚थापि हेतुद्वारेण दृष्टान्त‚द्वारेण च । न ‚{\tiny $_{lb}$}‚हेतुद्वारेण प्राग्दृष्टान्त‚दोषात् प्र‚स‚ङ्गेन प‚राजित‚स्य वादि ‚{\tiny $_{2}$}‚ नो दोषान्त‚र‚स्य दृष्टान्त‚{\tiny $_{lb}$}‚विरोध‚स्य वाच्य‚स्य वान‚पेक्ष‚णात् प‚राजित‚प‚राज‚याभावादित्याकूतं । विशेषेण साध‚{\tiny $_{lb}$}‚नाव‚य‚वानुक्र‚म‚वादिनो नैयायिक‚स्य स हि ‚{\tiny $_{3}$}‚ प्र‚तिज्ञाहेतूदाह‚र‚णोप‚न‚य‚निग‚म‚नानामानु‚{\tiny $_{lb}$}‚पूर्वीं प्र‚तिप‚न्नः । कः पुनः त‚स्यातिश‚य इत्याह । \quotelemma{उदाह‚र‚ण‚साध‚र्म्य‚मि} \cite[11b9]{vn-msN}‚{\tiny $_{lb}$}‚त्यादि । अङ्गीकृत्य चेद‚म‚वादि ‚{\tiny $_{4}$}‚ न तु दृष्टान्त‚विरोधो हेत्वाभास‚रूपासंस्प‚र्श्य‚स्ति । ‚{\tiny $_{lb}$}‚य‚थोक्त‚म‚न‚न्त‚र‚मिति । एतेन विक‚ल्प‚तो दोष‚विधानं प्र‚त्युक्तं । एव‚न्ताव‚द्व्य‚व‚स्थित‚{\tiny $_{lb}$}‚मेत‚द्य‚था प्र ‚{\tiny $_{5}$}‚ तिज्ञाया दृष्टान्त‚विरोधो हेत्वाभासान्नातिव‚र्त‚त इति । य‚त्पुन‚रुदाहृत‚{\tiny $_{lb}$}‚\quotelemma{म‚विद्ध‚क‚र‚णेन} भाष्य‚टीकायां व्य‚क्त‚मेक‚प्र‚कृतिकं प‚रिमित‚त्वाच्छ‚रावादि ‚{\tiny $_{6}$}‚ व‚दिति । ‚{\tiny $_{lb}$}‚त‚त्रापि विरुद्धो हेतुः प‚रिमित‚त्त्व‚स्य हेतोः स‚प‚क्षेऽभावे वा वृत्तेः । विप‚क्षे चानेक‚प्र‚कृति‚{\tiny $_{lb}$}‚के श‚रावादौ वृत्तेः । मृदः प्र‚तिक्ष‚णं प्र‚त्य‚व‚य‚व‚ञ्च भिद्य ‚{\tiny $_{7}$}‚ मान‚त्वात् । संप्र‚ति हेतोर‚पि ‚{\tiny $_{lb}$}‚दृष्टान्तेन विरोधो हेत्वाभासान्त‚र्ग‚त इति क‚थ‚य‚ति । हेतोर‚पि दृष्टान्त‚विरोधे ‚{\tiny $_{lb}$}‚स‚त्य‚सा[धा]र‚ण‚त्व‚मुभ‚य‚त्रावृत्तेः । विरुद्ध‚त्व‚म्वा । क‚दा ‚{\tiny $_{8}$}‚ विरुद्ध‚त्त्व‚मित्याह । वैध‚र्म्ये ‚{\tiny $_{lb}$}‚य‚दि स्याद‚प्य‚त्रोदाह‚र‚ण‚मुक्तं तेनैव गुण‚व्य‚तिरिक्तं द्र‚व्य‚म‚र्थान्त‚र‚त्वेनानुप‚ल‚भ्य‚मा‚{\tiny $_{lb}$}‚न‚त्वाद् घ‚ट‚व‚दिति अत्रापि दृश्य‚त्वे स‚तीति ‚{\tiny $_{9}$}‚ \leavevmode\ledsidenote{\textenglish{59b/msK}} हेतुविशेष‚णे विरुद्धः स‚प‚क्षे अव‚र्त‚मान‚{\tiny $_{lb}$}‚त्वात् । विप‚क्षे च रूपादीनां स्व‚रूपे व‚र्त‚माग‚त्त्वात् । विशेष‚णानुपादाने तु व्य‚भिचारो‚{\tiny $_{lb}$}‚\leavevmode\ledsidenote{\textenglish{93/s}} र्थान्त‚र‚त्त्वेनानुप‚ल‚ब्धानाम‚पि पि ‚{\tiny $_{1}$}‚ शाचादीनां प‚र‚स्प‚र‚व्य‚तिरेकित्वात् । न चात्र घ‚ट‚{\tiny $_{lb}$}‚व‚दिति दृष्टान्तो युक्त‚स्त‚स्यैव द्र‚व्यान्त‚र‚त्वेन प‚क्षीकृत‚त्वात् । त‚स्य रूपादिभ्यो भेदेन ‚{\tiny $_{lb}$}‚ग्र‚ह‚णं पूर्व्व‚मेव प्र‚तिसि\edtext{}{\lemma{तिसि}\Bfootnote{? षि}}द्धं ‚{\tiny $_{2}$}‚ ग्र‚ह‚णे चासिद्धो हेत्वाभास इत्य‚स्म‚न्म‚त‚मेव ‚{\tiny $_{lb}$}‚स्थितं । अथ हेतोः प्र‚माण‚विरोधे को हेत्वाभास इत्याह । असिद्धोग्नेः शैत्य‚स्या‚{\tiny $_{lb}$}‚विद्य‚मान‚त्वात् । य‚त्पुन‚र‚त्रो ‚{\tiny $_{3}$}‚ दाह‚र‚ण‚म‚न्य‚द‚नुष्णोग्निर्द्र‚व्य‚त्वाज्ज‚ल‚व‚दिति त‚द‚युक्तं । ‚{\tiny $_{lb}$}‚न‚हि प्र‚त्य‚क्षं द्र‚व्यं हेतुं बाध‚ते । त‚स्य ध‚र्मिणि सिद्ध‚त्वात् । किन्तु प्र‚तिज्ञार्थ‚म‚नुष्ण‚त्वं ॥ ‚{\tiny $_{4}$}‚ ‚{\tiny $_{lb}$}‚ अथ प्र‚तिज्ञार्थ‚स्य प्र‚त्य‚क्षेण बाधित‚त्वाद्धेतोस्तेन व्याप्तिर्न्नास्तीति हेतोः प्र‚माण‚विरोध ‚{\tiny $_{lb}$}‚उच्य‚ते । एव‚न्त‚र्हि विरुद्धेन साध्य‚ध‚र्मेणाव्याप्तेः स‚न्दि ‚{\tiny $_{5}$}‚ ग्ध‚व्य‚तिरेको हेत्वाभास ‚{\tiny $_{lb}$}‚इत्य‚स्म‚त्प‚क्ष एव स‚म‚र्थितः ।
	{\color{gray}{\rmlatinfont\textsuperscript{§~\theparCount}}}
	\pend% ending standard par
      ‚{\tiny $_{lb}$}‚
	  \bigskip
	  \begingroup
	
	    
	    \stanza[\smallbreak]
	  \flagstanza{\tiny\textenglish{...33}}{\normalfontlatin\large ``\qquad}हेतोःप्र‚माण\edtext{}{\lemma{माण}\Bfootnote{? मान}}विरोघ‚स्य हेत्वाभासान‚तिक्र‚मात् ॥ [३३]{\normalfontlatin\large\qquad{}"}\&[\smallbreak]
	  
	  
	  
	  \endgroup
	‚{\tiny $_{lb}$}‚

	  
	  \pstart \leavevmode% starting standard par
	त‚दुक्त‚म्
	{\color{gray}{\rmlatinfont\textsuperscript{§~\theparCount}}}
	\pend% ending standard par
      ‚{\tiny $_{lb}$}‚
	  \bigskip
	  \begingroup
	
	    
	    \stanza[\smallbreak]
	  \flagstanza{\tiny\textenglish{...34}}{\normalfontlatin\large ``\qquad}प्र‚त्य‚क्षादि[वि]रोधा ये व्याप्त‚कालो ‚{\tiny $_{6}$}‚ प‚पातिनः ।&‚{\tiny $_{lb}$}‚ते स‚र्वे न विरुद्धेन व्याप्तिध‚र्मेण युञ्ज‚त [३४] इति ॥{\normalfontlatin\large\qquad{}"}\&[\smallbreak]
	  
	  
	  
	  \endgroup
	‚{\tiny $_{lb}$}‚

	  
	  \pstart \leavevmode% starting standard par
	स्यान्म‚त‚म्प्र‚तिज्ञायाः प्र‚माण‚विरोध‚स्त‚न्मात्र‚भावित्वाद्धेत्वाभासेऽन्त‚र्ग‚म‚यितुं न ‚{\tiny $_{lb}$}‚पार्य‚त इत्य‚त आह । ‚{\tiny $_{7}$}‚ \quotelemma{प्र‚तिज्ञायाः प्र‚माण‚विरोधः स्व‚व‚च‚न‚विरोधेन व्याख्यातः} \cite[12a2]{vn-msN} ‚{\tiny $_{lb}$}‚ कृत‚प्र‚तिक्रिय‚स्त‚त्रेद‚मेव निग्र‚हाधिक‚र‚ण‚म‚साध‚नाङ्ग‚भूतायाः प्र‚तिज्ञायाः साध‚न‚वाक्ये ‚{\tiny $_{lb}$}‚प्र ‚{\tiny $_{8}$}‚ योग इत्यादिना । इति त‚स्मात् स‚र्व्व एवेत्युप‚संह‚र‚ति । य‚त्तु विरुद्ध‚मुत्त‚र‚मिति ‚{\tiny $_{lb}$}‚पूर्व्व‚प‚क्षोक्त‚म‚प‚र‚मुप‚क्षिप‚ति त‚द‚स‚म्ब‚द्ध‚मेव । \quotelemma{य‚स्माद्य‚दि ही \cite[12a3]{vn-msN} त्य‚दि} । अनि ‚{\tiny $_{9}$}‚ \leavevmode\ledsidenote{\textenglish{60a/msK}} त्यः ‚{\tiny $_{lb}$}‚श‚ब्द ऐन्द्रिय‚क‚त्वाद् घ‚ट‚व‚दित्येकं बौद्धेनान्येन वा कृते \quotelemma{मीमांस‚कः काणादोन्यो} वा ‚{\tiny $_{lb}$}‚स्व‚प‚क्ष‚सिद्धेन गोत्वादिना सामान्ये ‚{\tiny $_{1}$}‚ न प‚र‚स्य साध‚न‚वादिनो बौद्ध‚स्य हेतोर्व्य‚भिचार‚{\tiny $_{lb}$}‚सिद्धिमाकांक्षेत गोत्व‚म‚प्यैन्द्रिय‚कं त‚द‚पि भ‚व‚तोऽनित्यं प्र‚स‚ज्य‚त इत्येव य‚दि प‚रं प्र‚त्ये‚{\tiny $_{lb}$}‚\leavevmode\ledsidenote{\textenglish{94/s}} वाध्यारोप्याभिद‚ध्याद् व्य‚भि ‚{\tiny $_{2}$}‚ चारं त‚दा त‚स्य \quotelemma{बौद्ध} स्य त‚त्सामान्य‚मैन्द्रिय‚कं नित्य‚ञ्च ‚{\tiny $_{lb}$}‚स्व‚प‚क्ष‚विरुद्धं नित्य‚प‚दार्थान‚भ्युप‚गामान्नाभिम‚त‚म‚त‚श्च क‚थं व्य‚भिचार इति ‚{\tiny $_{lb}$}‚विरोधो व्याह‚तिर‚युक्त‚त्त्व‚मिति ‚{\tiny $_{3}$}‚ याव‚त् युज्य‚त उत्त‚र‚स्येत्य‚ध्याह‚र्त‚व्यं । न तु पुन‚रेव‚{\tiny $_{lb}$}‚म‚सौ प‚र‚स्येवोप‚रि भार‚मुप‚क्षिप्य व्य‚भिचार‚मुद्भाव‚य‚ति त‚त्क‚थ‚मुत्त‚र‚स्य विरोधः ‚{\tiny $_{lb}$}‚य‚तः स ‚{\tiny $_{4}$}‚ ह्युत्त‚र‚वादी स्व‚यं प्र‚तिप‚न्ने नित्य‚त्वेन गोत्वे हेतोरैन्द्रिय‚क‚त्व‚स्य वृत्तेः संश‚{\tiny $_{lb}$}‚यानः स‚न् किङ्घ‚ट‚व‚दैन्द्रिय‚क‚त्वाद‚नित्यः श‚ब्दो भ‚व‚तु किम्वा गोत्वा ‚{\tiny $_{5}$}‚ दिव‚न्नित्य इत्य‚{\tiny $_{lb}$}‚प्र‚तिप‚त्तिम‚निश्च‚य‚मात्म‚न‚स्त‚था ब्रुवाणः ख्याप‚य‚ति स‚त्प‚क्षे ख‚ल्वेन्द्रिय‚क‚म‚पि गोत्वं ‚{\tiny $_{lb}$}‚नित्यं त‚स्माद‚यं सांप्र‚त्य‚नैकान्तिक इती ‚{\tiny $_{6}$}‚ त्थ‚मात्मीय‚मेवाभ्युप‚ग‚मं पुर‚स्कृत्यानेकान्त‚{\tiny $_{lb}$}‚ञ्चोद‚य‚ति । त‚तः साध्विवोत्त‚र‚मिति स‚मुदायार्थः । स्यात् म‚त‚म्बौद्ध‚स्य नास्त्येव‚{\tiny $_{lb}$}‚गोत्वं नित्यं त‚तो व्याह‚त‚मेवोत्त‚र ‚{\tiny $_{7}$}‚ मित्य‚त आह । स च हेतु \cite[12a5]{vn-msN} रैन्द्रिय‚क‚त्वादिति ‚{\tiny $_{lb}$}‚स‚त्य‚स‚ति वा गोत्वे प‚र‚मार्थ‚तः । अप्र‚साधित‚साध‚न‚साम‚र्थ्यः स‚न् विप‚र्य‚ये बाध‚क‚{\tiny $_{lb}$}‚प्र‚माणावृत्या संश‚य‚हेतुत्वाद ‚{\tiny $_{8}$}‚ नैकान्तिक एव । अप्र‚साधितं साध‚नाय साम‚र्थ्यं साध्या ‚{\tiny $_{lb}$}‚विनाभावित्व‚ल‚क्ष‚ण‚म‚स्येति विग्र‚हः । साध‚न‚श‚ब्दो भाव‚साध‚नः । य‚दा तु बाध‚क‚{\tiny $_{lb}$}‚प्र‚माण‚ब‚लेन हेतोर‚वि ‚{\tiny $_{9}$}‚ \leavevmode\ledsidenote{\textenglish{60b/msK}} नाभावं स‚र्व्वोप‚संहारेण साध‚य‚ति य‚त्किञ्चिदिन्द्रिय‚ज्ञान‚ग्राह्यं ‚{\tiny $_{lb}$}‚स्व‚निर्भास‚ज्ञान‚ज‚न‚क‚त्वात्त‚त्र स‚र्व्व‚म‚नित्यं नित्य‚त्वे स‚र्व्व‚दा त‚द्विष‚यं ज्ञानं प्र‚स‚ञ्ज‚ते न ‚{\tiny $_{lb}$}‚वा क‚दाचिद‚पि ‚{\tiny $_{1}$}‚ त‚थाहि । ‚{\tiny $_{lb}$}‚ 
	    \pend% close preceding par
	  
	    
	    \stanza[\smallbreak]
	  \flagstanza{\tiny\textenglish{...35}}{\normalfontlatin\large ``\qquad}स्वात्म‚नि ज्ञान‚ज‚न‚ने य‚च्छ‚क्तं श‚क्त‚मेव त‚त् ।&‚{\tiny $_{lb}$}‚अथ‚वाऽश‚क्तं क‚दाचिच्चेद‚श‚क्तं स‚र्व‚दैव त‚त् ॥ [३५]{\normalfontlatin\large\qquad{}"}\&[\smallbreak]
	  
	  
	  
	    \pstart  \leavevmode% new par for following
	    \hphantom{.}
	   ‚{\tiny $_{lb}$}‚ 
	    \pend% close preceding par
	  
	    
	    \stanza[\smallbreak]
	  \flagstanza{\tiny\textenglish{...36}}{\normalfontlatin\large ``\qquad}त‚स्य श‚क्तिर‚श‚क्तिर्वा या स्व‚भावेन संस्थिता ।&‚{\tiny $_{lb}$}‚नित्य‚त्वाद‚चिकित्स्य‚{\tiny $_{2}$}‚स्य क‚स्तां क्ष‚प‚यितुं क्ष‚म [३६]{\normalfontlatin\large\qquad{}"}\&[\smallbreak]
	  
	  
	  
	    \pstart  \leavevmode% new par for following
	    \hphantom{.}
	   इति ॥
	{\color{gray}{\rmlatinfont\textsuperscript{§~\theparCount}}}
	\pend% ending standard par
      ‚{\tiny $_{lb}$}‚

	  
	  \pstart \leavevmode% starting standard par
	त‚दानीं गोत्वादीनाम‚पि नित्यानामेक‚प्र‚घ‚टेन इव पाटित‚त्वात् गोत्वे हेतोर‚{\tiny $_{lb}$}‚वृत्तेर्न संश‚य एव भ‚व‚ति । \quotelemma{एतेने \cite[12a6]{vn-msN} त्यादि} सुज्ञानं । त‚त्रा ‚{\tiny $_{3}$}‚ प्य‚नैकान्तिक‚हेत्वा‚{\tiny $_{lb}$}‚भास‚त्वान‚तिवृत्तिरिति संक्षेपः । त‚त्संश‚य‚हेतुत्व‚मुखेनानैकान्तिक‚त्व‚म‚स‚म‚र्थिते स‚ति ‚{\tiny $_{lb}$}‚हेतौ । \quotelemma{अन्य‚त्रापी} त्येक‚प‚क्ष‚प्र‚तिप‚न्नेपि ‚{\tiny $_{4}$}‚ व‚स्तुनि तुल्य‚मिति नोभ‚य‚सिद्धेत‚र‚योर्व‚स्तुनोर‚नै‚{\tiny $_{lb}$}‚कान्तिक‚त्व‚विशेषः । य‚था क‚थित‚म‚न‚न्त‚र‚मेव । स च हेतुः स‚त्य‚स‚ति वेत्यादिना । इत‚र‚{\tiny $_{lb}$}‚\leavevmode\ledsidenote{\textenglish{95/s}} देक‚प‚क्ष ‚{\tiny $_{5}$}‚ प्र‚तिप‚न्न‚म‚नैकान्तिक‚विष‚य‚त्वाच्चानैकान्तिक‚मिति व्याख्यातं । स्याच्चित्त‚न्ना‚{\tiny $_{lb}$}‚निष्टेर्दूष‚णं स‚र्व्व‚प्र‚सिद्ध‚स्तु द्व‚योर‚पि साध‚नं । दूष‚ण‚म्वेत्येत‚त्क‚थ‚मेव‚न्न ‚{\tiny $_{6}$}‚ व्याह‚न्य‚त इति ‚{\tiny $_{lb}$}‚त‚च्च नैवं । निश्चित‚दूष‚णाभिस‚न्धिव‚च‚नात् । त‚त एव त‚द‚न‚न्त‚र‚माहान्यः पुनः साध्य‚{\tiny $_{lb}$}‚त्व‚मीक्ष‚त इति । एत‚त्तु स्यात् । त‚दा द्व‚योरेक‚स्यापि न ज‚य ‚{\tiny $_{7}$}‚ प‚राज‚यौ । य‚द‚प्युक्त \quotelemma{मुद्योत ‚{\tiny $_{lb}$}‚क‚रेण} प्र‚तिज्ञाविरोध‚सूत्र‚मेव विवृण्व‚ता दृष्टान्ताभासा इत्यादि त‚द‚प्य‚व‚य‚वान्त‚र‚वा‚{\tiny $_{lb}$}‚दिनो नैयायिक‚स्यायुक्तं । बौद्ध एवैवं ब्रुवा ‚{\tiny $_{8}$}‚ णः शोभ‚त इत्य‚भिप्रेतं [।] त‚द्व‚च‚नेन हेत्वा‚{\tiny $_{lb}$}‚भास‚व‚च‚नेन ग‚म्य‚मान‚स्य दृष्टान्ताभास‚स्य त‚स्माद्धेतोः स‚काशात् सा ‚{\tiny $_{9}$}‚ \leavevmode\ledsidenote{\textenglish{61a/msK}} ध‚नान्त‚र‚त्वा- ‚{\tiny $_{lb}$}‚भाव‚प्र‚स‚ङ्गात् । दृष्टान्त‚स्येति शेषः ।
	{\color{gray}{\rmlatinfont\textsuperscript{§~\theparCount}}}
	\pend% ending standard par
      ‚{\tiny $_{lb}$}‚

	  
	  \pstart \leavevmode% starting standard par
	न‚नु च दृष्टान्ताभासानां हेत्वाभासेष्व‚न्त‚र्भावेऽतिदिष्टे हेतोर्दृष्टान्तेऽव‚य‚वान्त‚रं ‚{\tiny $_{lb}$}‚न प्राप्नोतीति व‚च‚न ‚{\tiny $_{1}$}‚ म‚स‚म्ब‚द्ध‚मेवेत्य‚त आह । \quotelemma{दृष्टान्ताभासाना} मि \cite[12a9]{vn-msN} त्यादि । ‚{\tiny $_{lb}$}‚अय‚म‚स्य प्र‚योगो म‚न‚सि विजृम्भ‚ते । य‚द्य‚तोऽर्थान्त‚र‚भूतं न त‚दाभास‚व‚च‚नेन त‚दाभास‚{\tiny $_{lb}$}‚व‚च‚नं न्याय्यं ‚{\tiny $_{2}$}‚ न च त‚दाभासेषु त‚दाभासानाम‚न्त‚र्भावः । त‚द्य‚था प्र‚त्य‚क्षाभासानाम‚{\tiny $_{lb}$}‚नुमानाभासेषु । त‚था च भ‚व‚तो हेतोर्दृष्टान्तोर्थान्त‚र‚भूत इति व्याप‚क‚विरुद्धोप‚{\tiny $_{lb}$}‚ल‚ब्धिः ‚{\tiny $_{3}$}‚ अतोऽव‚श्यं दृष्टान्त‚स्य हेताव‚न्त‚र्भाव एष्ट‚व्यः । त‚त्र च न दृष्टान्तः पृथ‚क् ‚{\tiny $_{lb}$}‚साध‚नाव‚य‚वः स्यात् । अपृथ‚ग्वृत्तेः एक‚व्यापार‚त्त्वादित्य‚र्थः । एत‚देव व्या ‚{\tiny $_{4}$}‚ च‚ष्टे \quotelemma{यो ‚{\tiny $_{lb}$}‚दृष्टान्त} \cite[12b1]{vn-msN} इत्यादिना । एवं प्र‚तिज्ञाहेत्वोर्विरोध‚स्य प्र‚प‚ञ्च‚स्य हेत्वाभासैः ‚{\tiny $_{lb}$}‚स‚ङ्गृहीत‚त्वान्न पृथ‚ग्व‚च‚नं क‚र्त्त‚व्य‚मित्य‚भिधायाधुना प्र‚तिघ‚हा ‚{\tiny $_{5}$}‚ न्यादीनाम‚पीय‚मेव ‚{\tiny $_{lb}$}‚ग‚तिरित्यावेद‚नायाह । \quotelemma{अपि चे} \cite[12b1]{vn-msN} त्यादि । पूर्व्व‚प‚क्ष‚वादिग्र‚ह‚ण‚मुत्त‚र‚प‚क्ष‚वादि‚{\tiny $_{lb}$}‚\leavevmode\ledsidenote{\textenglish{96/s}} नोऽज्ञानादीनि हेत्वाभास‚स्प‚र्शानि संतीति ‚{\tiny $_{6}$}‚ क‚थ‚नार्थं । त‚त्स‚म्ब‚न्धीनीति हेत्वाभास‚{\tiny $_{lb}$}‚पूर्व्व‚प‚क्ष‚वादिस‚म्ब‚न्धीनि वा । अथोच्य‚ते । अर्थान्त‚र‚ग‚म‚नादीनां हेत्वाभासासंस्प‚र्शित्त्वा ‚{\tiny $_{lb}$}‚न्न‚तेस्व\edtext{}{\lemma{तेस्व}\Bfootnote{? ष्व}}न्त‚र्भाव इ ‚{\tiny $_{7}$}‚ ति । त‚च्चास‚त् । अर्थान्त‚र‚ग‚म‚नादेर‚पि हेतोर‚स‚{\tiny $_{lb}$}‚म‚र्थ एव‚म‚तिस‚म्भ‚वात् । कुतः अस‚म‚र्थ‚स्य न्याय‚ब‚लेन साध्य‚प्र‚तिपाद‚ने वादिन इति ‚{\tiny $_{lb}$}‚शेषः । मिथ्याप्र‚वृ ‚{\tiny $_{8}$}‚ त्तेर‚र्थान्त‚र‚ग‚म‚नादिनेत्य‚भिप्रायः ॥ ४ ॥
	{\color{gray}{\rmlatinfont\textsuperscript{§~\theparCount}}}
	\pend% ending standard par
      ‚{\tiny $_{lb}$}‚

	  
	  \pstart \leavevmode% starting standard par
	उत्त‚रः प‚श्चाद् फ‚ल‚भावी स चासौ प्र‚तिज्ञास‚न्यास‚श्च त‚स्यापेक्ष‚या किन्न किञ्चि‚{\tiny $_{lb}$}‚दित्य‚र्थः । अश‚क्तः प‚रिच्छेदः सं ‚{\tiny $_{9}$}‚ \leavevmode\ledsidenote{\textenglish{61b/msK}} ख्ये येषां क्लीव‚प्र‚लाप‚चेष्टितानां तानि त‚था क्लीवा‚{\tiny $_{lb}$}‚दीनां प्र‚लापा येषां वादिनान्तेषां चेष्टितानि प्र‚तिज्ञासंन्यासादीनि वै किमुप‚न्य‚स्तैः [।] ‚{\tiny $_{lb}$}‚ क[ः] पुन‚रेवं स‚ति दोष इत्याह । \quotelemma{एवं ह्य‚तिप्र‚स‚ङ्गः} \cite[11b6]{vn-msN} स्यात् । एव‚माद्य‚पीति ‚{\tiny $_{lb}$}‚मूर्च्छावेप‚थुत्र‚स‚त्त्वादीनामादिश‚ब्देनाव‚रोधः । त‚स्मादेत‚द‚प्य‚स‚म्ब‚द्धं विद्व‚त्स‚द‚स्येवं ‚{\tiny $_{lb}$}‚प्र‚कार‚स्य स्थूल‚त्वादित्त्याभिप्रायः त‚द‚त्र \quotelemma{भाविविक्तः} स्व‚य‚माशंक्य किल प्र‚तिविध‚त्ते ‚{\tiny $_{lb}$}‚स्थूल‚त्वेनेदं निग्र‚ह‚स्थान‚मिति चेत् । प्राश्निक‚प्र‚तिवादिस‚न्निधौ प्र‚तिज्ञातार्थाप‚ह्न‚व‚{\tiny $_{lb}$}‚\leavevmode\ledsidenote{\textenglish{97/s}} ङ्क‚रोतीति । अस‚म्ब‚द्ध ‚{\tiny $_{3}$}‚ मुच्य‚ते त‚न्नाभिप्रायाप‚रिज्ञानात् । न ब्रूमो ध्वंसी श‚ब्द इति ‚{\tiny $_{lb}$}‚किन्तु संयोग‚विभागाभ्यां न व्य‚क्त इत्य‚यं प्र‚तिज्ञातार्थ इत्याह सामान्य‚स्य च स्वाश्र‚य‚{\tiny $_{lb}$}‚व्य‚ङ्ग्य ‚{\tiny $_{4}$}‚ त्वात् विवादाभाव इति । निग्र‚ह‚स्थान‚न्तु पूर्व‚म‚प्र‚तिज्ञातार्थ‚त्वात् । अनैकान्ति‚{\tiny $_{lb}$}‚क‚दोषेण प्र‚तिषेधे हेतौ प्र‚तिज्ञातार्थाप‚ह्न‚व‚ङ्क‚रोतीति निगृह्य‚त इति ‚{\tiny $_{5}$}‚ त‚त्र‚वाच्यं ‚{\tiny $_{lb}$}‚य‚दि वादी साकांक्ष एवान्त‚राले केन‚चिद् दुर्व्विद‚ग्धेनानैकान्तिक‚दोषेण चोदितः ‚{\tiny $_{lb}$}‚स‚न्प्र‚तिज्ञातार्थ‚फ‚लीक‚र‚णेन स्वाभिप्राय‚माविष्क‚रोति । त‚दा ‚{\tiny $_{6}$}‚ न्योपि न क‚श्चि‚{\tiny $_{lb}$}‚द्दोषः । किम‚ङ्ग पुनः प्र‚तिज्ञासंन्यासः । अथ निराकांक्षः स‚न् प‚श्चाच्चोदितः ‚{\tiny $_{lb}$}‚प्र‚तिज्ञां विशिन‚ष्टि । त‚द‚प्य‚नैकान्तिक‚दोषेणैव निगृह्य‚त इति कि\edtext{}{\lemma{कि}\Bfootnote{\href{http://sarit.indology.info/?cref=nbh.259-60}{न्याय‚भाष्ये २५९-६०} अल्प‚भेदेन ।}}मुत्त‚र‚प्र‚तिज्ञा‚{\tiny $_{lb}$}‚संन्यासापेक्ष‚येति न किञ्चित्प‚रिहृतं किञ्च स्फुट‚मिदं प्र‚तिज्ञान्त‚रेन्त‚र्भ‚व‚तीति नः ‚{\tiny $_{lb}$}‚पृथ‚ग्वाच्य‚मिति ॥ ४ ॥
	{\color{gray}{\rmlatinfont\textsuperscript{§~\theparCount}}}
	\pend% ending standard par
      ‚{\tiny $_{lb}$}‚

	  
	  \pstart \leavevmode% starting standard par
	\hphantom{.}\quotelemma{अविशेषोक्ते हेतावि} त्यादि सूत्रं ‚{\tiny $_{8}$}‚ अत्र निद‚र्श‚न‚मुदाह‚र‚ण‚मित्य‚र्थः । \quotelemma{कापिलः} ‚{\tiny $_{lb}$}‚ प्र‚माण‚य‚ति प्र‚धान‚सिद्धिप्र‚त्याश‚या । एक‚प्र‚कृतीदं व्य‚क्तं व्य‚क्तादिप‚रिमित‚त्वाद् घ‚ट‚श‚{\tiny $_{lb}$}‚रावादिव‚दिति । एका प्र ‚{\tiny $_{9}$}‚ \leavevmode\ledsidenote{\textenglish{62a/msK}} कृतिर‚स्येति विग्र‚हः । प्र‚कृतिरुपादान‚कार‚णं । या च ‚{\tiny $_{lb}$}‚किल सा प्र‚कृतिर्विकार‚ग्राम‚स्य त‚त्प्र‚धान‚मितीय‚म‚लीक‚प्र‚त्यासा\edtext{}{\lemma{त्यासा}\Bfootnote{? शा}}साङ्ख्य‚स्या‚{\tiny $_{lb}$}‚\leavevmode\ledsidenote{\textenglish{98/s}} प‚रिमाण‚ञ्च‚तुर‚स्र‚म्प‚रिम‚ण्ड‚ल‚मित्यादि । मृत्पूर्व्व‚काणामित्य‚न्व‚य‚माह । अस्य हेतो‚{\tiny $_{lb}$}‚र्व्य‚भिचारेण प्र‚त्य‚व‚स्थानं प्र‚तिवादिना क्रिय‚ते । नानाप्र‚कृतीनाङ्ग‚वाश्वादीनामेक‚{\tiny $_{lb}$}‚प्र‚कृतीनाञ्च कुम्भोद‚ञ्च‚ना ‚{\tiny $_{2}$}‚ दीनान्दृष्ट‚म्प‚रिमाण‚मित्येवं प्र‚त्य‚व‚स्थिते स‚ति प्र‚ति‚{\tiny $_{lb}$}‚वादिनि । य‚दि वा प्र‚त्य‚व‚स्थितः प्र‚तिषिद्धः प्र‚धान‚वाद्याह । एक‚प्र‚कृतिस‚म‚न्व‚ये स‚ति ‚{\tiny $_{lb}$}‚प‚रिमाण‚द‚र्श‚ना ‚{\tiny $_{3}$}‚ दिति स‚विशेष‚ण‚त्वाद्धेतोर्व्य‚भिचाराभाव इति म‚तिः । क‚थं पुन‚रेक‚प्र‚{\tiny $_{lb}$}‚कृतिस‚म‚न्व‚य इत्याह । सुख‚दुःख \quotelemma{मोह‚स‚म‚न्वितं} हीदं व्य‚क्तं प‚रिमितं गृह्य ‚{\tiny $_{4}$}‚ ते । स‚र्व्व‚त्र ‚{\tiny $_{lb}$}‚त‚त्कार्य‚द‚र्श‚नादित्याकूतं । त‚थाहि सुख‚ब‚हुलानाम्प्र‚साद‚लाघ‚व‚प्र‚स‚वाभिष्व‚ङ्गाद्ध‚र्ष ‚{\tiny $_{lb}$}‚प्रीत‚यः कार्यं । र‚जोब‚हुलानां शोष‚ताप‚भेद‚स्तं ‚{\tiny $_{5}$}‚ भोद्वेगाप‚द्वेषाः । त‚मोब‚हुलानां साव‚{\tiny $_{lb}$}‚र‚ण‚माद‚नाय‚ध्वंस‚वीभ‚त्स‚दैन्य‚गौर‚वाणि । एतानि च स‚र्व्वाणि स‚र्व्व‚त्रैव य‚थोत्क‚र्षाप‚{\tiny $_{lb}$}‚क‚र्ष‚भेद‚मुप‚ल‚भ्य‚न्ते । त ‚{\tiny $_{6}$}‚ स्मात्त्रैगुण्य‚प्र‚कृतीदं विश्वं । त‚दिद‚मित्यादिना निग्र‚ह‚स्थान‚त्वे ‚{\tiny $_{lb}$}‚कार‚ण‚माह । \quotelemma{अत्रापी} त्याद्य‚स्य प्र‚तिषेधः सुज्ञानः । अविरामाद‚च्छेदाद‚प‚रिस‚माप्त‚{\tiny $_{lb}$}‚त्वात् ‚{\tiny $_{7}$}‚ साध‚नाभिधान‚स्येत्य‚र्थः ॥ ० ॥
	{\color{gray}{\rmlatinfont\textsuperscript{§~\theparCount}}}
	\pend% ending standard par
      ‚{\tiny $_{lb}$}‚

	  
	  \pstart \leavevmode% starting standard par
	य‚थोक्त‚ल‚क्ष‚ण इत्येकाधिक‚र‚णौ विरुद्धौ ध‚र्माविति प‚क्ष‚प्र‚तिप‚क्ष‚ल‚क्ष‚णं स्म‚{\tiny $_{lb}$}‚र‚य‚ति । प‚रिग्र‚हे वादिप्र‚तिवादिभ्यां कृते स‚ति हेतुतः ‚{\tiny $_{8}$}‚ साध्य‚सिद्धौ प्र‚कृतायां ‚{\tiny $_{lb}$}‚हेतुव‚सा\edtext{}{\lemma{सा}\Bfootnote{? शा}}त्साध्य‚सिद्धिरित्येत‚स्मिन्प्र‚क‚र‚णे स‚ति प्र‚कृतोर्थः श‚ब्द‚नित्य‚त्वं । तेना‚{\tiny $_{lb}$}‚स‚ङ्ग‚त‚त्वात् । त‚द‚स‚म्ब‚द्ध‚त्त्वात्त‚द‚नुप‚कार‚क‚त्वादित्य‚र्थः [।] त‚था ‚{\tiny $_{9}$}‚ \leavevmode\ledsidenote{\textenglish{62b/msK}} हि विनापिरूप‚सि‚{\tiny $_{lb}$}‚द्ध्या प्रातिप‚दिकादिव्याख्यानं य‚था क‚थ‚ञ्चित्प्र‚तिपादिताद‚र्थादेवार्थः सिध्य‚ति । ‚{\tiny $_{lb}$}‚न्याय्य‚मेत‚दिति स्व‚म‚तेनाविरुद्ध‚त्वाद‚भ्य‚नुजानाति । क‚दा च पूर्वो ‚{\tiny $_{1}$}‚ त्त‚र‚प‚क्ष‚वादिनो‚{\tiny $_{lb}$}‚र्न्याय्यं निग्र‚ह‚स्थान‚मित्याह । प्र‚तिपादिते दोषे स‚ति वादिप्र[ति]वादिभ्याम‚न्यो‚{\tiny $_{lb}$}‚न्य‚म‚साध‚नाङ्ग‚व‚च‚न‚मेत‚द‚दोष‚द्भाव‚न‚ञ्च भ‚वेदिति अन्य‚था न ह्य ‚{\tiny $_{2}$}‚ [न]‚{\tiny $_{lb}$}‚\leavevmode\ledsidenote{\textenglish{99/s}} योरेक‚स्यापि ज‚य‚प‚राज‚यावित्युक्तं । प्र‚कृतं प‚रित्य‚ज्येति न्याय्य‚तामेवास्य ‚{\tiny $_{lb}$}‚प्र‚तिपाद‚य‚ति । प्र‚कृत‚म‚त्र साध्य‚साध‚न‚हेत्व‚भिधानं त‚द‚कृत्वेति उप‚न्य‚स्ते दोषे ‚{\tiny $_{3}$}‚ न ‚{\tiny $_{lb}$}‚स‚म‚र्थ‚नं । \quotelemma{अप‚र‚स्य} \cite[13a7]{vn-msN} रूप‚सिध्यादेः । अत‚न्नान्त‚रीय‚क‚स्यापीति । उप‚न्य‚स्त‚{\tiny $_{lb}$}‚साव‚न‚स‚म‚र्थ‚नाङ्ग‚स्येत्य‚र्थः । अप‚र‚स्य नामादिव्याख्यानादेरुप‚क्षे ‚{\tiny $_{4}$}‚ पः प‚राज‚य‚स्थान‚{\tiny $_{lb}$}‚मिति व‚र्त्त‚ते ॥ ४ ॥
	{\color{gray}{\rmlatinfont\textsuperscript{§~\theparCount}}}
	\pend% ending standard par
      ‚{\tiny $_{lb}$}‚

	  
	  \pstart \leavevmode% starting standard par
	\hphantom{.}व‚र्ण‚क्र‚म‚निदेश[व]न्निर‚र्थ‚कं \href{http://sarit.indology.info/?cref=ns\%C5\%AB.2.1.8}{न्या० सू० २।१।८ } य‚त्र व‚र्णा एव केव‚लं क्र‚मेण ‚{\tiny $_{lb}$}‚निर्दिश्य‚न्ते । न प‚द‚न्नापि वाक्यं । अर्थान्त‚रे किलाप्र ‚{\tiny $_{5}$}‚ कृतार्थ‚क‚थ‚न‚मिह व‚र्ण‚मात्रोच्चा‚{\tiny $_{lb}$}‚र‚ण‚मिति शेषः ॥ अस‚म्व‚द्ध‚तामेवाह । \quotelemma{न‚हि व‚र्ण्ण‚क्र‚म‚निर्देशादेव} \cite[13a9]{vn-msN} केव‚लादान‚र्थ‚{\tiny $_{lb}$}‚क्य‚म‚पि तु य‚देव किञ्चिद‚साध ‚{\tiny $_{6}$}‚ नाङ्ग‚स्यासिद्ध‚विरुद्धादेः श‚ब्द‚रूप‚सिध्यादेश्च व‚च‚न‚{\tiny $_{lb}$}‚न्त‚देवान‚र्थ‚कं । किं कार‚णं । साध्य‚सिद्ध्युप‚योगिनोऽभिधेय‚स्याभावात् । साध्य‚सिद्ध्यु‚{\tiny $_{lb}$}‚प‚योगिनोऽभावे ‚{\tiny $_{7}$}‚ पि क‚स्यान्य‚त्प्र‚योज‚न‚म‚स्तीत्य‚पि न म‚न्त‚व्यं इति क‚थ‚य‚ति । ‚{\tiny $_{lb}$}‚निष्प्र‚योज‚न‚त्वाच्चेति । साध्य‚सिद्धेरेव प्र‚स्तुत‚त्वाद‚न्य‚प्र‚योज‚न‚व‚त्वेपि आन‚र्थ‚क्य‚मेव ‚{\tiny $_{lb}$}‚त‚त्र प्र‚स्ता ‚{\tiny $_{8}$}‚ व इत्य‚भिप्रायः । त‚स्मात्प्र‚कार‚विशेषोपादान‚व‚र्ण‚क्र‚म‚निर्देश‚व‚दित्य‚स‚म्ब‚द्धं । ‚{\tiny $_{lb}$}‚प‚रः प्राह । न साध्य‚सिद्धौ य‚द‚न‚र्थ‚क‚म‚न‚ङ्ग‚न्त‚न्निर‚र्थ‚क‚म‚भिप्रेत‚म‚पि ‚{\tiny $_{9}$}‚ \leavevmode\ledsidenote{\textenglish{63a/msK}} तु य‚स्य व‚च‚न‚स्य ‚{\tiny $_{lb}$}‚काक‚वासितादेरिव नैव क‚श्चिद‚र्थः । त‚था च नार्थान्त‚रापार्थ‚कादीनाम‚नेनैव संग्र‚ह‚स्त‚त्र‚{\tiny $_{lb}$}‚\leavevmode\ledsidenote{\textenglish{100/s}} क‚स्य‚चिद‚र्थ‚लेश‚स्य स‚द्भावात् । \quotelemma{आचार्य} आह ‚{\tiny $_{1}$}‚ [।] \quotelemma{य‚स्य क‚स्य चिद} \cite[13b3]{vn-msN} प्यादिनोपि ‚{\tiny $_{lb}$}‚निर‚र्थ‚काभिधाने वाहित इव किन्न निग्र‚हो भ‚व‚ति । क‚थं स्यादित्याह [।] \quotelemma{निग्र‚ह‚निमित्त ‚{\tiny $_{lb}$}‚त्त‚स्य} निर‚र्थ‚काभिधान‚स्य वाद्य‚वादिनोर‚वि ‚{\tiny $_{2}$}‚ शेषात् । नेति प‚र‚न्त‚स्य वादिन इह ‚{\tiny $_{lb}$}‚वाद‚प्र‚क‚र‚णे । आयात‚मित्याचार्यः । त‚स्य तेनैव निर‚र्थ‚काभिधानेन । त‚त्रैवं स्थिते ‚{\tiny $_{lb}$}‚वादे तुल्यं । स‚र्व‚स्यासाध‚नाङ्ग ‚{\tiny $_{3}$}‚ वादिनो निर‚र्थ‚काभिधायित्व‚मित्य‚ध्याह‚र्त्त‚व्यं । क्व‚{\tiny $_{lb}$}‚चित्त‚वेतिपाठः । त‚त्र नोप‚स्कारेण किञ्चित् । अनेनैव निर‚र्थ‚काभिधानेन । प्र‚त्यु‚{\tiny $_{lb}$}‚च्य‚ते । य‚स्य नैव ‚{\tiny $_{4}$}‚ क‚श्चिद‚र्थ इति । एत‚द‚प्य‚स‚म्ब‚द्धं । य‚स्मान्न च व‚र्ण‚क्र‚म‚निर्देशोपि ‚{\tiny $_{lb}$}‚निर‚र्थ‚कः क्व‚चित्प्र‚क‚र‚णे प्र‚त्याहारादाव‚र्थ‚व‚त्वाच्च । त‚स्माद‚त्रैव वादेस्य व‚र्ण्ण‚क्र ‚{\tiny $_{5}$}‚ म‚{\tiny $_{lb}$}‚स्यान‚र्थ‚क्यं । त‚च्चार्थान्त‚रादेर‚पि तुल्य‚मिति चित्तं क‚क्क‚ङ्पिङ्गित‚मित्य‚त्रादिश‚ब्देन ‚{\tiny $_{lb}$}‚उत्प्लुत्य ग‚म‚नं ताल‚दान‚नृत्त\edtext{}{\lemma{नृत्त}\Bfootnote{? नृत्य}}आदीनाङ्ग‚ह‚णं ॥ ४ ॥
	{\color{gray}{\rmlatinfont\textsuperscript{§~\theparCount}}}
	\pend% ending standard par
      ‚{\tiny $_{lb}$}‚

	  
	  \pstart \leavevmode% starting standard par
	त्रिर ‚{\tiny $_{6}$}‚ भिहित‚मिति त्रिव‚च‚न‚ङ्कार्य‚मिति न्याय‚त्वं द‚र्श‚य‚ति । स‚कृदुक्तं स्प‚ष्टार्थ‚{\tiny $_{lb}$}‚म‚पि क‚दाचिन्न ज्ञाय‚त इति त्रिरुच्चार‚ण‚ङ्कार्यं । क‚स्मात्पुनः प‚द‚वाक्य‚प्र‚माण‚वि ‚{\tiny $_{7}$}‚ द्‚{\tiny $_{lb}$}‚\leavevmode\ledsidenote{\textenglish{101/s}} भिर्वाक्य‚न्न ज्ञाय‚त इत्याह । \quotelemma{क्लिष्ट‚श‚ब्द‚मित्यादि} । क्लिष्ट‚श‚ब्दं म‚नागुच्चारित‚त्वात् । ‚{\tiny $_{lb}$}‚अप‚श‚ब्द‚त्वादित्य‚प‚रे । क‚स्मादेवं प्र‚युक्त‚मित्याह । \quotelemma{असाम‚र्थ्य‚स‚म्व‚र‚णा ‚{\tiny $_{8}$}‚ ये \cite[13b9]{vn-msN} ‚{\tiny $_{lb}$}‚ति} । स्प‚ष्टार्थ‚स्य प्र‚योगे दूष‚ण‚म्व‚क्ष्य‚तीति भ‚यात्प्र‚युंक्ते । इद‚ञ्च साध‚न‚दूष‚ण‚{\tiny $_{lb}$}‚वादिनोः स‚मानं । दूष‚ण‚वाक्य‚म‚पि ह्येवंभूत‚निग्र‚ह‚प्राप्तिकार‚णं । नेदं ‚{\tiny $_{9}$}‚ \leavevmode\ledsidenote{\textenglish{63b/msK}} निर‚र्थ‚काद्- ‚{\tiny $_{lb}$}‚भिद्य‚ते । त‚था हि श्लिष्ट‚श‚ब्दादिभिः प्र‚कृतार्थ‚स‚म्ब‚द्ध‚ङ्ग‚म‚क‚मेव ब्रूयात् । एत‚द्विप‚{\tiny $_{lb}$}‚रीत‚म्वा । प्राक्त‚ने प्र‚कारे नास्यासाम‚र्थ्य‚न्त‚त्र तु प‚रिष‚दाद‚यो जाड्या ‚{\tiny $_{1}$}‚ त्त‚दुक्त‚न्न ‚{\tiny $_{lb}$}‚प्र‚तिप‚द्यंत इति नेय‚ता विद्वान्वादी निग्र‚ह‚म‚र्ह‚ति ।
	{\color{gray}{\rmlatinfont\textsuperscript{§~\theparCount}}}
	\pend% ending standard par
      ‚{\tiny $_{lb}$}‚
	  \bigskip
	  \begingroup
	
	    
	    \stanza[\smallbreak]
	  \flagstanza{\tiny\textenglish{...37}}{\normalfontlatin\large ``\qquad}व‚क्तुरेव हि त‚ज्जाड्यं य‚च्छ्रोत्रा नाव‚बुद्ध्य‚तें । [३७]{\normalfontlatin\large\qquad{}"}\&[\smallbreak]
	  
	  
	  
	  \endgroup
	‚{\tiny $_{lb}$}‚

	  
	  \pstart \leavevmode% starting standard par
	त‚तोसौ निग्र‚हार्ह एवेत्याकूत‚वानाह प‚रः । प‚रिष‚त् ‚{\tiny $_{2}$}‚ प्र‚ज्ञामिति । \quotelemma{न्याय‚वादिन} ‚{\tiny $_{lb}$}‚ \cite[13b9]{vn-msN} इति प‚रिह‚र‚ति । न्याय‚वादिनः उक्त‚मिति स‚म्ब‚न्धः । वादी तु जाड्या‚{\tiny $_{lb}$}‚त्प‚रिष‚दादेर‚विज्ञात‚साध‚न‚साम‚र्थ्य इति कृत्वा विजेता न स्यात् । प‚रिष‚त्प्र‚तिवादि‚{\tiny $_{lb}$}‚प्र‚त्याय‚नेन ज‚य‚व्य‚व‚स्थाप‚नात् । अविज्ञातं प्र‚तिपाद‚न‚साम‚र्थ्यं प‚रिष‚त्प्र‚तिवादिभ्यां ‚{\tiny $_{lb}$}‚य‚स्येति कार्यं । द्वितीय‚न्तु विक‚ल्प‚म‚धि ‚{\tiny $_{4}$}‚ कृत्याह । \quotelemma{अस‚म्ब‚द्धाभिधाने निर‚र्थ‚क‚मेवे} ‚{\tiny $_{lb}$}‚ \cite[14a1]{vn-msN} \quotelemma{ति} ॥ ४ ॥
	{\color{gray}{\rmlatinfont\textsuperscript{§~\theparCount}}}
	\pend% ending standard par
      ‚{\tiny $_{lb}$}‚

	  
	  \pstart \leavevmode% starting standard par
	\hphantom{.}\quotelemma{अनेक‚स्य प‚द‚स्येति} । य‚दानीम‚स‚म्ब‚द्धार्थ‚प्र‚तिपाद‚क‚त्त्वे वाक्यार्थ‚प्र‚तिपाद‚क‚त्वं ‚{\tiny $_{lb}$}‚निराक‚रोति ‚{\tiny $_{5}$}‚ वाक्य‚स्यास‚म्ब‚द्धार्थ‚प्र‚तिपाद‚क‚त्वे प्र‚क‚र‚णाध्याय‚प्र‚तिप‚त्य‚भावः । ‚{\tiny $_{lb}$}‚स‚मुदाय‚प्र‚तिप‚त्य‚भावाच्च निग्र‚ह‚स्थानं । उदाह‚र‚णं द‚श डा\edtext{}{\lemma{डा}\Bfootnote{? दा}}डिमाः ष‚ड‚पू ‚{\tiny $_{6}$}‚ पाः ‚{\tiny $_{lb}$}‚कुण्ड‚म‚जाजिनं प‚ल‚ल‚पिण्डं । अथ रौरुक‚मेत‚त् कुमार्यः स्फैय‚कृत‚स्य पिता प्र‚तिशीन ‚{\tiny $_{lb}$}‚इति अत्र च \quotelemma{भार‚द्वाजेन} निर‚र्थ‚कापार्थ‚क‚योर‚भेद इत्याश‚ङ्क‚य ‚{\tiny $_{7}$}‚ प्र‚तिविहितं त‚त्र हि ‚{\tiny $_{lb}$}‚व‚र्ण्ण‚मात्र‚मिह य‚दान्य‚स‚म्ब‚द्धानीति । त‚देवाचार्योप्युप‚क्षिप‚ति । \quotelemma{इदं किले} \cite[14a2]{vn-msN} त्या‚{\tiny $_{lb}$}‚दिना । अस‚म्ब‚द्धा व‚र्ण्णा य‚स्मिन्निर‚र्थ‚क इति विग्र‚हः । कि ‚{\tiny $_{8}$}‚ ल श‚ब्दोऽन‚भिम‚त‚त्व‚{\tiny $_{lb}$}‚\leavevmode\ledsidenote{\textenglish{102/s}} प्र‚द‚र्श‚नार्थः । अन‚भिम‚त‚त्व‚मेवाह । न‚न्व‚यं प‚दानाम‚स‚म्ब‚न्धाद‚पार्थ‚क‚व‚द‚स‚म्ब‚न्ध‚{\tiny $_{lb}$}‚वाक्य‚म‚पि निर‚र्थ‚कात् पृथ‚ग् वाच्यं स्यात् । स्यात्म‚त‚म‚पार्थ‚कं ‚{\tiny $_{9}$}‚ \leavevmode\ledsidenote{\textenglish{64a/msK}} नैवास‚म्ब‚द्ध‚प‚दार्था‚{\tiny $_{lb}$}‚स‚म्ब‚द्ध‚वाक्यार्थ‚योः स‚ङ्गृहीत‚त्वात् पृथ‚ग् न वाच्य‚मित्य‚त उच्य‚ते । \quotelemma{नोभ‚य‚{\tiny $_{lb}$}‚स‚ङ्ग्र‚हाद} \cite[14a3]{vn-msN} पार्थ‚कं युक्तं । क‚स्माद‚स‚म्ब‚द्ध‚प‚दार्थेनापार्थ‚केनैवास‚म्ब‚द्ध‚वा ‚{\tiny $_{1}$}‚ क्य‚स्येव ‚{\tiny $_{lb}$}‚निर‚र्थ‚क‚स्यापि व‚र्ण्ण‚क्र‚म‚मात्र‚ल‚क्ष‚ण‚स्य स‚ङ्ग्र‚ह‚प्र‚स‚ङ्गात् । अथोच्य‚ते । निर‚र्थ‚कं किमु‚{\tiny $_{lb}$}‚च्य‚ते । य‚स्यार्थ एव नास्ति केव‚लं व‚र्ण्ण‚क्र‚म‚मात्रं । अस‚म्ब‚द्ध‚प‚द ‚{\tiny $_{2}$}‚ वाक्य‚योस्तु साध्य‚{\tiny $_{lb}$}‚सिद्ध्य‚नुप‚योगेपि न स‚र्व‚था नैर‚र्थ‚क्य‚म‚तोऽर्थ‚त‚त्वे साम्यात् द्व‚योरेवैकीक‚र‚ण‚मित्य‚त ‚{\tiny $_{lb}$}‚आह । एवं विधाच्चेत्यादि । क‚पोल‚वादितादीनाम‚पि ‚{\tiny $_{3}$}‚ पृथ‚ग‚भिधान‚प्र‚स‚ङ्ग ‚{\tiny $_{lb}$}‚इत्य‚त्रातिप्र‚स‚ङ्ग उक्तः । न‚हि किञ्चित्मात्रेण विशेषो न श‚क्य‚ते क्व‚चित्प्र‚द‚र्श‚यितु‚{\tiny $_{lb}$}‚मित्य‚भिस‚न्धिः अथ निर‚र्थ‚कापार्थ‚क‚योः ‚{\tiny $_{4}$}‚ स‚ङ्ग्र‚ह‚निर्देश‚दोषं भेद‚निर्देशे च गुण‚म्प‚{\tiny $_{lb}$}‚श्य‚ताऽ \quotelemma{क्ष‚पादेन} न स‚ङ्ग्र‚ह‚निर्देशः कृत इति म‚न्य‚से । न साधु म‚न्य‚स इत्याह । \quotelemma{न च ‚{\tiny $_{lb}$}‚स‚ङ्ग्र‚ह} \cite[14a4]{vn-msN} इत्यादि ॥ ४ ॥ 
	{\color{gray}{\rmlatinfont\textsuperscript{§~\theparCount}}}
	\pend% ending standard par
      ‚{\tiny $_{lb}$}‚

	  
	  \pstart \leavevmode% starting standard par
	य‚था ल‚क्ष‚ण‚म‚र्थ‚व‚सा\edtext{}{\lemma{सा}\Bfootnote{? शा}}दित्य‚र्थः साम‚र्थ्यं । अनुप‚द‚र्शिते हि विष‚ये निर्विष‚या ‚{\tiny $_{lb}$}‚साध‚न‚प्र‚वृत्तिर्मा भूदिति साध्य‚निर्देश‚ल‚क्ष‚णा प्र‚तिज्ञा पूर्व्व‚मुच्य ‚{\tiny $_{6}$}‚ ते । त‚द‚न‚न्त‚र मुदाह‚र‚ण‚{\tiny $_{lb}$}‚साध‚र्म्यांत्साध्य‚साध‚नं हेतु रित्येवं ल‚क्ष‚णो हेतुस्त‚त्साध‚नायोच्य‚ते । त‚तो हेतोर्व‚हिर्व्या‚{\tiny $_{lb}$}‚प्तिप्र‚द‚र्श‚नार्थंसाध्य‚साध‚र्म्यात्त‚द्ध‚र्म‚भाविदृ ‚{\tiny $_{7}$}‚ ष्टान्त उदाह‚र‚ण‚मि \href{http://sarit.indology.info/?cref=ns\%C5\%AB.1.1.36}{न्या० सू० १।१।३६} ‚{\tiny $_{lb}$}‚ त्येवं ल‚क्ष‚ण‚मुदाह‚र‚णं । त‚तः प्र‚तिबिंब‚नार्थं साध्य‚ध‚र्मिणि स‚म्भ‚व‚प्र‚द‚र्श‚नार्थ‚म्वा‚{\tiny $_{lb}$}‚उदाह‚र‚णापेक्ष‚स्त‚थेत्युप‚संहारो न त‚थेति वेति ‚{\tiny $_{8}$}‚ साध‚न‚स्योप‚न‚य\href{http://sarit.indology.info/?cref=ns\%C5\%AB.1.1.38}{न्या० सू० १।१।३८ } ‚{\tiny $_{lb}$}‚ इत्येवंल‚क्ष‚ण उप‚न‚यः । त‚त उत्त‚र‚कालं स‚र्व्वाव‚य‚व‚प‚राम‚र्षेण\edtext{}{\lemma{र्षेण}\Bfootnote{? र्शेन}}विप‚रीत‚प्र‚स‚ङ्ग‚{\tiny $_{lb}$}‚निवृत्य‚र्थं हेत्व‚प‚देशात् प्र‚तिज्ञायाः पुन‚र्व‚च‚नं निग ‚{\tiny $_{9}$}‚ \leavevmode\ledsidenote{\textenglish{64b/msK}} म‚न‚मि \href{http://sarit.indology.info/?cref=ns\%C5\%AB.1.1.39}{न्या० सू० १।१।३९ } त्येवं ‚{\tiny $_{lb}$}‚ल‚क्ष‚णं निग‚म‚न‚मिति । अय‚म‚सौ य‚थाल‚क्ष‚ण‚म‚र्थ‚व‚सा\edtext{}{\lemma{सा}\Bfootnote{? शा}}त्क्र‚मः । त‚थाहि लोकेपि ‚{\tiny $_{lb}$}‚पूर्व्व‚ङ्कार्यं मृत्पिण्डाद्युपादीय‚ते प‚श्चात्तु क‚र‚ण‚ञ्च‚क्र‚द‚ण्डादिक‚मिति ‚{\tiny $_{1}$}‚ न्यायः । ‚{\tiny $_{lb}$}‚त‚त्रैत‚स्मिन‚क्र‚म\edtext{}{\lemma{म}\Bfootnote{? न्क्र‚मे}}न्याय‚तः । स्थितेऽव‚य‚वानां प्र‚तिज्ञादीनां विप‚र्य‚येणाभिधानं ‚{\tiny $_{lb}$}‚ \leavevmode\ledsidenote{\textenglish{103/s}} निग्र‚ह‚स्थानं । य‚था घ‚ट‚व‚त्कृत‚क‚त्वाद‚नित्य इति । नैव‚म‚पि सिद्धेरिति \quotelemma{भार ‚{\tiny $_{2}$}‚ द्वाजः} ‚{\tiny $_{lb}$}‚ स्व‚य‚मेवाश‚ङ्क्य प‚रिह‚र‚ति । \quotelemma{न प्र‚योगापेत‚श‚ब्द‚व‚देत‚त्स्यादिति} अनेनेति गोणीप‚देन । ‚{\tiny $_{lb}$}‚य‚था [।]
	{\color{gray}{\rmlatinfont\textsuperscript{§~\theparCount}}}
	\pend% ending standard par
      ‚{\tiny $_{lb}$}‚
	  \bigskip
	  \begingroup
	
	    
	    \stanza[\smallbreak]
	  \flagstanza{\tiny\textenglish{...38}}{\normalfontlatin\large ``\qquad}अम्ब‚म्बिति य‚था वालः शिक्ष्य‚माणः प्र‚भास‚ते\edtext{}{\lemma{ते}\Bfootnote{? ष‚ते ।}}&‚{\tiny $_{lb}$}‚अव्य‚क्तं ‚{\tiny $_{3}$}‚ त‚द्विदान्तेन व्य‚क्ते भ‚व‚ति निश्च‚यः । [३८]{\normalfontlatin\large\qquad{}"}\&[\smallbreak]
	  
	  
	  
	  \endgroup
	‚{\tiny $_{lb}$}‚

	  
	  \pstart \leavevmode% starting standard par
	त‚था किल गोण्याद‚यः श‚ब्दाः ते साधुष्व‚नुमाणे\edtext{}{\lemma{नुमाणे}\Bfootnote{? ने}}न प्र‚त्य‚योत्प‚त्तिहेत‚व ‚{\tiny $_{lb}$}‚इति । त‚देत‚दुन्म‚त्त‚क‚स्य वैया ‚{\tiny $_{4}$}‚ क‚र‚ण‚स्योन्म‚त्त‚क‚संव‚र्ण्ण‚न‚मुन्म‚त्त‚केनो \quotelemma{द्योत‚क‚रेण} संव‚र्ण्ण‚नं ‚{\tiny $_{lb}$}‚य‚था ह्येक उन्म‚त्तो द्वितीय‚मुन्म‚त्त‚कं स‚म्व‚र्ण्ण‚य‚ति त‚था भूत‚मेत‚द‚पीति या ‚{\tiny $_{5}$}‚ व‚त् । य‚दि ‚{\tiny $_{lb}$}‚चोन्म‚त्त‚क‚स्यो \quotelemma{द्योत‚क‚र} स्योन्म‚त्त‚क‚स्य वैयाक‚र‚ण‚स्य स‚म्व‚र्ण्ण‚नं । त‚था हि शाब्दिक एव ‚{\tiny $_{lb}$}‚ताव‚दुन्म‚त्तः प्र‚माण‚विरुद्ध‚व‚त्त्वाभिधायित्वात् । त‚त[ः] कु ‚{\tiny $_{6}$}‚ त‚स्त‚त्प्र‚क्रियायाः प्र‚माण‚{\tiny $_{lb}$}‚चिन्ताया ज्ञाप‚क‚त्व‚मित्य‚भिप्रेतं । क‚थ‚म्पुनः शाब्दिक‚स्यायुक्ताभिधायित्व‚मित्याह । ‚{\tiny $_{lb}$}‚ \quotelemma{य‚दि \cite[14a4]{vn-msN} त्यादि} सुबोधं । स्त्रीशूद्र‚श‚ब्दो मूर्ख‚व‚च‚नः । ‚{\tiny $_{7}$}‚ य‚स्तु \quotelemma{न‚क्क} श‚ब्दं \quotelemma{मुक्क} श‚ब्द‚{\tiny $_{lb}$}‚मेव नासाप‚र्याय‚म्वेत्ति । स क‚थ‚म‚प‚श‚ब्दाच्छ‚ब्दं साधुं प्र‚तिप‚द्यातः साधोः श‚ब्दा‚{\tiny $_{lb}$}‚द‚र्थ‚म्प्र‚तिप‚द्येत । किमुच्य‚ते नैवासौ त‚था विवोध‚म्प्र‚तिप ‚{\tiny $_{8}$}‚ द्य‚त इत्याह । \quotelemma{दृष्टाचानुभ‚य‚{\tiny $_{lb}$}‚वेदिनोपि \cite[14b3]{vn-msN} स‚न‚का} देः प्र‚तीतिरिति त‚स्मान्न प‚र‚म्प‚र‚या प्र‚तीतिर‚र्थ‚स्य । अय‚म‚त्र ‚{\tiny $_{lb}$}‚संक्षेपः । स्यादेव‚म‚साधूनां साध्व‚नुमाप‚क‚त्व‚म् । य ‚{\tiny $_{9}$}‚ \leavevmode\ledsidenote{\textenglish{65a/msK}} द्येषान्धूमादीनामिव त्रैरूप्य‚म्भ‚वे- ‚{\tiny $_{lb}$}‚ \leavevmode\ledsidenote{\textenglish{104/s}} न्निश्चितं । त‚च्च न स‚म्भ‚व‚ति । य‚स्मादेताव‚द‚नुभ‚य‚वेदिनः \quotelemma{स‚न‚का} द‚य‚स्ते स‚न्त‚म‚पि ‚{\tiny $_{lb}$}‚व्याप्य‚व्याप‚क‚भाव‚न्न प्र‚तिप‚द्य‚न्ते । न चासाव ‚{\tiny $_{1}$}‚ ज्ञातो ग‚म‚को ज्ञाप‚क‚त्वात् । येपि ‚{\tiny $_{lb}$}‚श‚ब्दाप‚श‚ब्द‚प्र‚विभाग‚कुश‚लास्तेप्य‚विद्य‚मान‚त्वादेव भाव‚य‚न्ति । त‚थाह्य‚साधूनां साधुभिः ‚{\tiny $_{lb}$}‚स‚म्ब‚न्ध‚स्तादात्म्यं कार्य‚कार‚ण‚भावो वा ‚{\tiny $_{2}$}‚ भ‚वेत । त‚दुभ‚य‚विक‚ल‚स्याव्य‚भिचार‚निय‚मा‚{\tiny $_{lb}$}‚भावात् । त‚त्र च ताव‚न्न तादात्म्य‚म‚भ्युपेयं पार‚मार्थिक‚स्यैव भेद‚स्य स्फुटं प्र‚त्य‚क्ष‚तः ‚{\tiny $_{lb}$}‚प्र‚तीतेः । श‚ब्द‚व‚द ‚{\tiny $_{3}$}‚ साधोर‚प्य‚व्य‚तिरेक‚तो वाच‚क‚त्व‚प्र‚स‚ङ्गाच्च । त‚दुत्प‚त्तिर‚पि दूरो‚{\tiny $_{lb}$}‚त्सारितेव । य‚तो नासाध‚वः साधुभ्यो जाय‚न्ते । क[ा]र‚ण‚गुण‚व‚क्तुकाम‚तामात्र‚हेतुत्वा ‚{\tiny $_{4}$}‚‚{\tiny $_{lb}$}‚त्तेषां । न च तेषान्नित्य‚त्व‚ङ्कादाचित्कोप‚ल‚म्भ‚तः । त‚त्वे वा सुत‚रान्त‚दुत्प‚त्तेर‚भावः ‚{\tiny $_{lb}$}‚स‚त्य‚पि वा व्याप्य‚व्याप‚क‚भावे त‚त्प‚रिज्ञाने च प‚क्ष‚ध‚र्म‚त्व‚वैक ‚{\tiny $_{5}$}‚ ल्याच्चाक्षुष‚त्वादे‚{\tiny $_{lb}$}‚रिवासाधुभ्यो नानुमानं । न‚ह्य‚त्र ध‚र्मे विद्य‚ते । य‚तः प‚क्ष‚ध‚र्म‚त्वं निष्प‚द्य‚ते । न‚हि ‚{\tiny $_{lb}$}‚साधूनामेव ध‚र्मित्व‚न्तेषामेवानुमीय‚मान‚त्वात् । न च ध‚र्मिसा ‚{\tiny $_{6}$}‚ ध‚नं युक्तिम‚तः । भावा‚{\tiny $_{lb}$}‚भावोभ‚य‚ध‚र्म‚स्यासिद्ध‚विरुद्धानैकान्तिक‚दोष‚दुष्ट‚त्व‚तः । क‚थं वा साधूनां त‚त्ध‚र्म‚त्वं । ‚{\tiny $_{lb}$}‚न‚हि त‚त्काले ते स‚न्ति । अस‚ताञ्च ध‚र्मित्वं वाच‚क‚त्वं ‚{\tiny $_{7}$}‚ चेति सुभाषितं । किमुच्य‚ते ‚{\tiny $_{lb}$}‚पुरुषो ध‚र्मी साधुश‚ब्द‚विव‚क्षा साध्य‚ध‚र्मः प‚क्ष‚ध‚र्म‚श्चासाधुरिति त‚द‚प्य‚स‚म्ब‚द्धं । ‚{\tiny $_{lb}$}‚व्याप्य‚व्याप‚क‚भावाभावादेव । य‚स्मान्न च ‚{\tiny $_{8}$}‚ \leavevmode\ledsidenote{\textenglish{65b/msK}} गोणीश‚ब्द‚प्र‚योग‚काले गोश‚ब्द‚विव‚क्षामु‚{\tiny $_{lb}$}‚प‚ल‚भाम‚हे । अथ प्र‚त्य‚व‚स्थीय‚ते । य‚था प‚क्ष‚ध‚र्म‚त्वादिवैक‚ल्येप्य‚व्य‚क्तं । बाल‚व‚चोव्य‚क्त‚{\tiny $_{lb}$}‚म‚नुमाप‚य‚ति । त‚थैवासाध‚वोपि ‚{\tiny $_{1}$}‚ साधूनिति [।] त‚द‚युक्तं त‚त्रापि तुल्य‚प‚र्य‚नुयोग‚त्वा‚{\tiny $_{lb}$}‚त् । व‚य‚न्तु प्र‚तिप‚द्याम‚हे साक्षादेव त‚स्माद‚प्य‚व्य‚क्तान्मात्राद्य‚र्थः प्र‚तीय‚त इति । ‚{\tiny $_{lb}$}‚त‚त्र संज्ञासंज्ञिस‚म्ब‚न्ध‚स्यान‚नुभूत ‚{\tiny $_{2}$}‚ त्वाद‚युक्ताप्र‚तीतावित्य‚पि न म‚न्त‚व्यं । अनादि‚{\tiny $_{lb}$}‚म‚ति संसारे व्य‚व‚हार‚प‚र‚म्प‚रायास्त‚थाभूतायाः स‚म्ब‚न्ध‚स्योल्लिङ्गित‚त्वात् । त‚थाहि ‚{\tiny $_{lb}$}‚न ग‚वादिश‚ब्दानाम‚पि प्रा ‚{\tiny $_{3}$}‚ यः शृङ्ग‚ङ्ग्राहिक‚यार्थ‚निय‚मः स‚ङ्केत्य‚तेपि तु व्य‚व‚हार‚{\tiny $_{lb}$}‚पार‚म्प‚र्य‚तो विद‚ग्धा निश्चिन्व‚न्ति । त‚च्चेहापि स‚मान‚मेव । त‚स्मादेत‚द‚र‚ण्य‚रुदितं । ‚{\tiny $_{lb}$}‚ 
	    \pend% close preceding par
	  
	    
	    \stanza[\smallbreak]
	  \flagstanza{\tiny\textenglish{...39}}{\normalfontlatin\large ``\qquad}अम्ब‚म्विति य‚था बालः शिक्ष्य‚माणः प्र‚भास‚ते ।&‚{\tiny $_{lb}$}‚अव्य‚क्त‚न्त‚द्विदान्तेन व्य‚क्तेन भ‚व‚ति निश्च‚यः ॥ [३९]{\normalfontlatin\large\qquad{}"}\&[\smallbreak]
	  
	  
	  
	    \pstart  \leavevmode% new par for following
	    \hphantom{.}
	   
	    \pend% close preceding par
	  
	    
	    \stanza[\smallbreak]
	  \flagstanza{\tiny\textenglish{...40}}{\normalfontlatin\large ``\qquad}एवं साधौ प्र‚योक्त‚व्ये यो य‚द्भ्रंशः प्र‚युज्य‚ते ।&‚{\tiny $_{lb}$}‚तेन साधु व्य‚व‚हितः क‚श्चिद ‚{\tiny $_{5}$}‚ र्थोव‚सीय‚त [४०]{\normalfontlatin\large\qquad{}"}\&[\smallbreak]
	  
	  
	  
	    \pstart  \leavevmode% new par for following
	    \hphantom{.}
	   इति ॥
	{\color{gray}{\rmlatinfont\textsuperscript{§~\theparCount}}}
	\pend% ending standard par
      ‚{\tiny $_{lb}$}‚

	  
	  \pstart \leavevmode% starting standard par
	\hphantom{.}य‚द‚प्य‚भ्य‚धायि\quotelemma{कुमारिलेन ।} ‚{\tiny $_{lb}$}‚ \leavevmode\ledsidenote{\textenglish{105/s}} 
	    \pend% close preceding par
	  
	    
	    \stanza[\smallbreak]
	  \flagstanza{\tiny\textenglish{...41}}{\normalfontlatin\large ``\qquad}गोश‚ब्देऽव‚स्थितेस्माक‚न्त‚द‚श‚क्तिज‚कारिता ।&‚{\tiny $_{lb}$}‚गाव्यादेर‚पि गोबुद्धिर्मूल‚श‚ब्दानुसारिणी [४१]{\normalfontlatin\large\qquad{}"}\&[\smallbreak]
	  
	  
	  
	    \pstart  \leavevmode% new par for following
	    \hphantom{.}
	   ति ॥
	{\color{gray}{\rmlatinfont\textsuperscript{§~\theparCount}}}
	\pend% ending standard par
      ‚{\tiny $_{lb}$}‚

	  
	  \pstart \leavevmode% starting standard par
	त‚स्यापीद ‚{\tiny $_{6}$}‚ मेव प्र‚तिविधान‚मिद‚ञ्च स‚र्व‚मागूर्य्य निग‚म‚य‚ति । न \quotelemma{प‚र‚म्प‚र‚या ‚{\tiny $_{lb}$}‚प्र‚तीति} रिति । अत्रैवोप‚च‚य‚माह । अर्थे प्र‚तिपाद‚नायास‚म‚र्थ‚स्यासाधोः श‚ब्देपि ‚{\tiny $_{lb}$}‚साधौ प्र‚तीतिज ‚{\tiny $_{7}$}‚ न‚नासाम‚र्थ्याच्च । त‚त्रैत‚त्स्यान्न व‚य‚म‚साधूनाम‚र्थेषु प्र‚तीति‚{\tiny $_{lb}$}‚ज‚न‚क‚त्वं निराकुर्मः । किन्तु वाच‚क‚त्वं । श‚ब्दे चासाधुः प्र‚तीतिज‚न‚क एव ‚{\tiny $_{lb}$}‚न वाच‚क एव इत्य‚त आह । न \quotelemma{ह्य ‚{\tiny $_{8}$}‚ \leavevmode\ledsidenote{\textenglish{66a/msK}} र्थेपि श‚ब्द‚स्य वाच‚क‚त्व‚म‚न्य‚देवे} \cite[14b4]{vn-msN}- ‚{\tiny $_{lb}$}‚त्यादि । य‚द्य‚साधोर‚र्थे प्र‚तीतिज‚न‚क‚त्व‚मिष्य‚ते । त‚दैताव‚ता व‚य‚माहित‚प‚रितोषाः । ‚{\tiny $_{lb}$}‚किम‚स्माक‚म‚भिधानान्त‚र‚क‚ल्पितेन वाच‚क ‚{\tiny $_{1}$}‚ त्वेनेत्याकूतं । नैव त‚र्ह्य‚साव‚र्थ‚प्र‚तीतिं ‚{\tiny $_{lb}$}‚ज‚न‚यितुँ क्ष‚मोऽपि तु श‚ब्द एवेति चेदाह । \quotelemma{अप‚श‚ब्द‚श्चेदि \cite[14b4]{vn-msN} ति} । अथोच्य‚ते ‚{\tiny $_{lb}$}‚श‚ब्देन त‚स्य स्वाभाविकः स‚म्ब‚न्धो नार्थेन त‚त‚स्त‚मे ‚{\tiny $_{2}$}‚ व प्र‚तिपाद‚य‚ति नार्थ‚न्त‚द्य‚था ‚{\tiny $_{lb}$}‚स्व‚भाव‚त‚श्च‚क्षूरूपं प्र‚काश[य]ति न श‚ब्दादीन‚त आह । \quotelemma{अकृत‚स‚म‚य‚स्ये \cite[14b5]{vn-msN} ‚{\tiny $_{lb}$}‚ त्यादि} । अद‚र्श‚नादिति । न ह्य‚प्र‚तीत‚स‚म्ब‚न्धाः \quotelemma{सिंह‚ल} श‚ब्दा आर्य ‚{\tiny $_{3}$}‚ ज‚न‚व्य‚व‚हा‚{\tiny $_{lb}$}‚राय व‚र्त्त‚न्ते । स‚म‚य एव तु ज‚न‚येत् प्र‚तीतिं । साम‚यिके च त‚त्र स‚म्ब‚न्धे सोर्थेप्य‚{\tiny $_{lb}$}‚निवार्यः । स‚म‚य‚व‚सा\edtext{}{\lemma{सा}\Bfootnote{? शा}}द‚साधुः साधौ व‚र्त्त‚मानोर्थ एव ग‚वादौ ‚{\tiny $_{4}$}‚ किन्न ‚{\tiny $_{lb}$}‚प्र‚व‚र्त‚ते । न‚हि किञ्चित्त‚था दोषो गुण‚स्तु केव‚ल इत्याह । \quotelemma{एवं ही} त्या \cite[14b6]{vn-msN} ‚{\tiny $_{lb}$}‚दि । एत‚दुक्त‚म्भ‚व‚ति । ये स्व‚भाव‚तः प्र‚काश‚का न ते स‚म‚य‚म‚पेक्ष‚न्ते । य‚था च‚क्षुर्दी‚{\tiny $_{lb}$}‚पा ‚{\tiny $_{5}$}‚ द‚यो रूपादीनां । स्व‚भाव‚त‚श्चाप‚श‚ब्दो य‚दि श‚ब्द‚स्य प्र‚काश‚को भ‚वेत् । त‚त‚स्ते‚{\tiny $_{lb}$}‚नापि स‚म्ब‚न्धोनापेक्षः स्यात् । अपेक्ष्य‚ते च त‚तो नास्य श‚ब्दे स्वा[भा]विकं साम‚र्थ्यं । ‚{\tiny $_{6}$}‚ ‚{\tiny $_{lb}$}‚त‚था चेद‚म‚पि श‚क्य‚म‚नुमातुं । ये स‚म‚याक्षेक्ष\edtext{}{\lemma{याक्षेक्ष}\Bfootnote{? पेक्ष}}प्र‚वृत्त‚य‚स्ते स‚र्व‚त्र य‚थास‚म‚य‚{\tiny $_{lb}$}‚म‚निवारित‚प्र‚स‚राः साक्षादेव प्र‚तिपाद‚का भ‚व‚न्ति । य‚थाकाय‚विज्ञ‚प्त्याद‚यः । त ‚{\tiny $_{7}$}‚ था ‚{\tiny $_{lb}$}‚चाप‚श‚ब्दा अपि स‚म‚यापेक्ष‚प्र‚वृत्त‚य इति सिद्ध‚मेषाम‚व्य‚व‚धान‚त एवार्थ‚प्र‚ति‚{\tiny $_{lb}$}‚ \leavevmode\ledsidenote{\textenglish{106/s}} पाद‚क‚त्व‚मिति । विप‚र्य‚य‚द‚र्श‚नाच्चेत्युप‚च‚यान्त‚रं । त‚थाहि वृक्षोग्निरुत्प‚ल‚मित्युक्ते ‚{\tiny $_{8}$}‚ ‚{\tiny $_{lb}$}‚ऽव्युत्प‚न्न‚धियो वालाः प्र‚श्नोप‚क्र‚मं स‚न्तिष्ठ‚न्ते । कोयं वृक्ष इत्यादिना । ते चान्य‚स्य ‚{\tiny $_{lb}$}‚व्युत्पाद‚नोपाय‚स्याभावाद‚प‚श‚ब्दैरेव व्युत्पाद्य‚न्ते रुक्ख अग्गी उप्प‚ल‚मिति ॥ त ‚{\tiny $_{9}$}‚ \leavevmode\ledsidenote{\textenglish{66b/msK}} देव‚{\tiny $_{lb}$}‚म‚त्रासाध‚व एव वाच‚का न साध‚वः स‚न्तोपीति विप‚र्य‚यो दृश्य‚ते [।] अथ प्र‚तिप‚द्य‚से ‚{\tiny $_{lb}$}‚ध‚र्म‚साध‚न‚ता श‚ब्द‚संस्कारो य‚थोव‚तं ।
	{\color{gray}{\rmlatinfont\textsuperscript{§~\theparCount}}}
	\pend% ending standard par
      ‚{\tiny $_{lb}$}‚

	  
	  \pstart \leavevmode% starting standard par
	\hphantom{.}
	    \pend% close preceding par
	  
	    
	    \stanza[\smallbreak]
	  \flagstanza{\tiny\textenglish{...42}}{\normalfontlatin\large ``\qquad}शिष्टेभ्य आग‚मात् सिद्धं साध‚नो ध‚र्म‚साध‚नं &‚{\tiny $_{lb}$}‚अर्थ‚प्र‚त्याय‚नाभेदे विप‚रीतास्त्व‚साध‚व [४२]{\normalfontlatin\large\qquad{}"}\&[\smallbreak]
	  
	  
	  
	    \pstart  \leavevmode% new par for following
	    \hphantom{.}
	   इति । त‚था
	{\color{gray}{\rmlatinfont\textsuperscript{§~\theparCount}}}
	\pend% ending standard par
      ‚{\tiny $_{lb}$}‚
	  \bigskip
	  \begingroup
	

	  
	  \pstart \leavevmode% starting standard par
	\hphantom{.}
	    \pend% close preceding par
	  
	    
	    \stanza[\smallbreak]
	  \flagstanza{\tiny\textenglish{...43}}म‚न्त्र[ो] हीनः स्व‚र‚तो व‚र्ण‚तो[वा]मिथ्याप्र‚युक्तो न त‚म‚र्थ‚माह ।&‚{\tiny $_{lb}$}‚स वाग्व‚ज्रो य‚ज‚मानं हिन‚स्ति य‚थेन्द्र‚श‚त्रुः स्व‚र‚तोप‚राधात्\edtext{}{\lemma{राधात्}\Bfootnote{व्याक‚र‚ण‚म‚हाभाष्ये प‚स्प‚शाह्निके ।}}। ‚{\tiny $_{2}$}‚ [४३]\&[\smallbreak]
	  
	  
	  
	    \pstart  \leavevmode% new par for following
	    \hphantom{.}
	   ‚{\tiny $_{lb}$}‚ते सुरा हेऽल‚यो[हे]ऽल‚य इत्युक्त‚वंतः प‚राव‚भूवुः । एकोपि श‚ब्दः स‚म्य‚क्प्र‚युक्तः सुकृ‚{\tiny $_{lb}$}‚तिनां लोक‚ङ्ग‚म‚य‚ति । आहिताग्निर‚प‚श‚ब्द‚म‚भिधाय प्राय‚श्चित्तीयामिष्टिं नि ‚{\tiny $_{3}$}‚ र्व‚पे‚{\tiny $_{lb}$}‚ [दि]त्यादि ।
	{\color{gray}{\rmlatinfont\textsuperscript{§~\theparCount}}}
	\pend% ending standard par
      
	  \endgroup
	

	  
	  \pstart \leavevmode% starting standard par
	\href{http://sarit.indology.info/?cref=MaBh\%C4\%81.1}{म‚हाभाष्ये आह्निके १} इद‚म‚प‚सार‚य‚ति \quotelemma{न ध‚र्म‚साध‚न‚ता} \cite[14b9]{vn-msN} ‚{\tiny $_{lb}$}‚श‚ब्दानां संस्कार इति व‚र्त‚ते किङ्कार‚ण‚मित्याह । \quotelemma{मिथ्यावृत्तिचोद‚नेभ्य} \cite[14b9]{vn-msN} ‚{\tiny $_{lb}$}‚इत्यादि । मिथ्यावृत्तिश्चोद्य‚ते यैरि ‚{\tiny $_{4}$}‚ ति कार्यं । य‚था ह्य‚स्याभिन‚व‚विद्रुमाङ्कुर‚प्र‚क‚रा‚{\tiny $_{lb}$}‚भिराम‚किश‚ल‚य‚म‚ञ्जुम‚ञ्ज‚रीराजीविराजित‚त‚रोर‚शोक‚व‚न‚स्प‚तेर‚धः श‚यित‚स्य द्विज‚{\tiny $_{lb}$}‚न्म‚नो नील‚नीर‚ज ‚{\tiny $_{5}$}‚ नील‚तातिशायिना म‚ण्ड‚लाग्रेण शिर‚श्छित्त्वेत्युक्तेपि भ‚व‚त्येव ब्र‚ह्म‚ह‚{\tiny $_{lb}$}‚त्य‚या स‚म्ब‚न्धः प्र‚योज‚क‚स्य । अन्येभ्य इत्य‚स‚म्भूतेभ्यो विप‚र्य‚येण स‚म्य‚क्त्व‚वृत्तिचोद‚{\tiny $_{lb}$}‚ने ‚{\tiny $_{6}$}‚ न । य‚था अस्स ब‚म्भ‚ण‚स्स गावी दीअदि । स‚र्व‚ञ्चेद‚म‚प्र‚माण‚त्वाद्व‚च‚न‚मात्रं ‚{\tiny $_{lb}$}‚भ‚व‚ताविम‚त्याकूत‚वानुप‚च‚य‚माह । \quotelemma{श‚ब्द‚स्य सुप्र‚योगादेवेत्यादि} \cite[14b9]{vn-msN} । एवं ‚{\tiny $_{lb}$}‚विधानित्य‚प्र‚मा ‚{\tiny $_{7}$}‚ ण‚कान् ।
	{\color{gray}{\rmlatinfont\textsuperscript{§~\theparCount}}}
	\pend% ending standard par
      ‚{\tiny $_{lb}$}‚

	  
	  \pstart \leavevmode% starting standard par
	\hphantom{.}न‚नु च प्र‚तिष्ठिते भूप्र‚देशे चैत्य‚ङ्कार‚य‚ति ब्राह्म्यं पुण्यं प्र‚स‚व‚ति क‚ल्पं स्व‚ग‚षु ‚{\tiny $_{lb}$}‚मोद‚त इत्यादाव‚पि प्र‚माणाभावाद‚यं तुल्यः प्र‚स‚ङ्गो भ‚व‚ताम‚पि । न तुल्यो ‚{\tiny $_{8}$}‚ य‚स्माद‚त्र ‚{\tiny $_{lb}$}‚ \leavevmode\ledsidenote{\textenglish{107/s}} विष‚य‚द्व‚य‚प‚रिशुद्धिः प‚रिनार्था \edtext{}{\lemma{रिनार्था}\Bfootnote{?}} विस‚म्वाद‚श्चास्तीति तृतीयेपि राशावाहित‚प‚रि‚{\tiny $_{lb}$}‚तोषाः प्रेक्षाव‚न्तः प्र‚व‚र्त्त‚न्ते । न‚त्वेवं भ‚व‚न्म‚तेऽन‚न्त‚रोदित‚द्व‚य‚म‚पि ‚{\tiny $_{8}$}‚ \leavevmode\ledsidenote{\textenglish{67a/msK}} प्र‚माण‚व्याह‚त‚त्वात् ‚{\tiny $_{lb}$}‚प्र‚माण‚व्याह‚तिश्चान‚न्त‚र‚मेवावेदिता । त‚स्माद्द‚रिद्रेश्व‚र‚स्प‚र्धास‚मान‚मेत‚त् [।]
	{\color{gray}{\rmlatinfont\textsuperscript{§~\theparCount}}}
	\pend% ending standard par
      ‚{\tiny $_{lb}$}‚

	  
	  \pstart \leavevmode% starting standard par
	विदित‚वेद्यादिगुण‚प्र‚युक्ता \cite[15a2]{vn-msN} इत्य‚न्त‚र्भावित‚भाव‚प्र‚त्य‚य‚यो\edtext{}{\lemma{यो}\Bfootnote{? य}} ‚{\tiny $_{lb}$}‚ \quotelemma{निर्देश ‚{\tiny $_{1}$}‚ ः} । विदितं वेद्यं हेयोपादेय‚त्वं यैस्ते त‚थोक्ताः । आदिग्र‚ह‚णात् क‚रुणादिप‚{\tiny $_{lb}$}‚रिग्र‚हः । अमूने \cite[5a2]{vn-msN} व संस्कृतान‚प‚रान‚संस्कृतान् । एत‚दुक्त‚म्भ‚व‚ति [।] ‚{\tiny $_{lb}$}‚श‚ब्दो हि व्य‚व‚हारोर्थ‚प्र ‚{\tiny $_{2}$}‚ त्याय‚न‚फ‚लः । त‚च्च य‚था संस्कृतेभ्यः स‚ङ्केत‚व‚सा\edtext{}{\lemma{सा}\Bfootnote{? शा}} ‚{\tiny $_{lb}$}‚ त्स‚म्प‚द्य‚ते त‚थाऽसंस्कृतेभ्योपीति किम‚स्थानेभिनिविष्टाः शिष्टाः । अत एव च ‚{\tiny $_{lb}$}‚म‚न्ये प्रेक्षाव‚द्भ्योन्य‚त्वाद‚नुग‚ता ‚{\tiny $_{3}$}‚ र्थ‚मेवं नामामीषामिति । अथ‚वा किम‚स्माक‚{\tiny $_{lb}$}‚म‚र्ह‚षितिः\edtext{}{\lemma{षितिः}\Bfootnote{?}} प्र‚त्याख्यानैः ॥ ते अमून्नैव प्र‚युञ्ज‚ते नाप‚रानित्य‚त्रैव निश्च‚या‚{\tiny $_{lb}$}‚भावात् । य‚दाह [।] न चात्र श ‚{\tiny $_{4}$}‚ ब्दे प‚रोक्षः साक्षी य‚तः साक्षिण इद‚मेवामूनेव ‚{\tiny $_{lb}$}‚प्र‚युञ्ज‚ते नाप‚रानिति निश्चिनुमः \cite[15a3]{vn-msN} ।
	{\color{gray}{\rmlatinfont\textsuperscript{§~\theparCount}}}
	\pend% ending standard par
      ‚{\tiny $_{lb}$}‚

	  
	  \pstart \leavevmode% starting standard par
	\hphantom{.}न‚नु चोक्त‚न्त‚द‚न्वाख्यान‚स्य प्र‚योज‚नं \quotelemma{र‚क्षोहाग‚म‚ल‚घ्व‚स‚न्दे ‚{\tiny $_{5}$}‚ हा} [म‚हाभाष्येआह्निके १] इति । त‚त्क‚थं गुणातिश‚याभावादित्युच्य‚त इत्याह । \quotelemma{वेद‚र‚क्षादिक‚ञ्चाप्र‚यो‚{\tiny $_{lb}$}‚ज‚न‚मेवात‚त्स‚म‚य‚स्थायिन} \cite[15a4]{vn-msN} स्ताथाग‚त‚स्य । न्यायानुपायित्वात् । त‚त्स्व‚भा ‚{\tiny $_{6}$}‚ ‚{\tiny $_{lb}$}‚व‚स्य \cite[15a4]{vn-msN} साधुश‚ब्द‚रूप‚स्य । अन्य‚तोपीति \cite[15a4]{vn-msN} बृद्ध‚प्र‚वाद‚पार‚म्प‚र्यात् । ‚{\tiny $_{lb}$}‚एत‚देव दृष्टान्तोप्र‚क्र‚मं व्य‚न‚क्ति । \quotelemma{प्राकृतेत्यादिना} \cite[15a4]{vn-msN} ।
	{\color{gray}{\rmlatinfont\textsuperscript{§~\theparCount}}}
	\pend% ending standard par
      ‚{\tiny $_{lb}$}‚\textsuperscript{\textenglish{108/s}}

	  
	  \pstart \leavevmode% starting standard par
	इ[त्]थं शा[ि]ब्द‚क‚स्योन्म‚त्त‚क‚तामुप‚द‚र्श्याधुना \quotelemma{भा ‚{\tiny $_{7}$}‚ र‚द्वाज} स्याह ।
	{\color{gray}{\rmlatinfont\textsuperscript{§~\theparCount}}}
	\pend% ending standard par
      ‚{\tiny $_{lb}$}‚

	  
	  \pstart \leavevmode% starting standard par
	\hphantom{.}\quotelemma{अव‚य‚व‚विप‚र्य‚येपीत्यादि} \cite[15a6]{vn-msN} । स‚म्व‚न्ध‚प्र‚तीतिरिति स‚म्ब‚न्धः प‚र‚स्प‚र‚मुप‚{\tiny $_{lb}$}‚कार्योप‚कार‚क‚भावः । साम‚र्थ्याद्विव‚क्षित‚प्र‚तिपाद‚न इति शेषः । ‚{\tiny $_{8}$}‚ अथ स्या \quotelemma{द‚क्ष‚पाद} ‚{\tiny $_{lb}$}‚सिद्धान्त‚नीतिपाल‚नाय न प्र‚तिज्ञादीनां क्र‚म‚व्य‚त्य‚यः क्रिय‚त इत्य‚त्राह । \quotelemma{न‚ह्य‚त्र ‚{\tiny $_{lb}$}‚क‚श्चित्स} म‚यः \cite[15a6]{vn-msN} सिद्धान्तो निय‚मो वा प्र‚माणोपेत इत्य‚प्याह । न ‚{\tiny $_{9}$}‚ \leavevmode\ledsidenote{\textenglish{67b/msK}} प‚र आह । ‚{\tiny $_{lb}$}‚ \quotelemma{न विप‚र्य‚यात्प्र‚तीतिः} \cite[15a7]{vn-msN} साध्य‚स्य । किन्तु त‚तो विप‚र्य‚यादानुपूर्व्या प्र‚तीति‚{\tiny $_{lb}$}‚रिति । अस्य प्र‚तिषेधः । \quotelemma{नाप्र‚तीय‚मान‚स‚म्ब‚न्धेभ्य आनुपूर्वी प्र‚तीतिरिति \cite[15a8]{vn-msN} । ‚{\tiny $_{1}$}‚ ‚{\tiny $_{lb}$}‚ येषामित्यादि} नै \cite[15a8]{vn-msN} त‚देव व्याच‚ष्टे ॥ अपि च प्र‚तिज्ञोप‚न‚य‚निग‚म‚नानां ‚{\tiny $_{lb}$}‚पूर्व‚मेवास्माभिः साध‚न‚वाक्ये प्र‚योगः प्र‚तिक्षिप्तः । त‚त्कुत‚स्त‚त्कृतो विप‚र्य‚य इत्येत‚{\tiny $_{lb}$}‚त्क‚थ‚य‚ति ‚{\tiny $_{2}$}‚ [।] \quotelemma{प्र‚तिपादित} \cite[15a10]{vn-msN} मित्यादिना । प्र‚तिज्ञाग्र‚ह‚ण‚मुप‚ल‚क्ष‚णार्थं । ‚{\tiny $_{lb}$}‚अथ साम‚र्थ्य‚ल‚भ्यापि प्र‚युज्य‚ते त‚दातिप्र‚स‚ङ्ग इत्येत‚दाह । \quotelemma{प्र‚तीय‚मानार्थ‚स्य च ‚{\tiny $_{lb}$}‚प्र‚योगेति} \cite[15b1]{vn-msN} प्र‚स‚ङ्गः सा ‚{\tiny $_{3}$}‚ ध‚र्म्य‚व‚ति प्र‚योगे वैध‚र्म\edtext{}{\lemma{र्म}\Bfootnote{? र्म्य}}स्यापि प्र‚योग‚{\tiny $_{lb}$}‚प्र‚स‚ङ्गः । न चेष्य‚ते । अर्थादाप‚न्न‚स्य स्व‚श‚ब्देन पुन‚र्व‚च‚न‚ञ्चेति \href{http://sarit.indology.info/?cref=ns\%C5\%AB.5.2.15}{न्या० सू० ५।२।१५ } ‚{\tiny $_{lb}$}‚निग्र‚ह‚स्थान‚व‚च‚नात् । प‚क्ष‚ध‚र्मान्व‚य‚व्य‚तिरे ‚{\tiny $_{4}$}‚ केषु त‚र्हि प्र‚तिज्ञाद्य‚भावेपि क्र‚म‚निय‚मो‚{\tiny $_{lb}$}‚ \leavevmode\ledsidenote{\textenglish{109/s}} भ‚विष्य‚तीत्य‚त आह । \quotelemma{प‚रिशिष्टे} \cite[15b1]{vn-msN} ष्वित्यादि । अप्र‚तीय‚मान‚स‚म्ब‚न्ध‚प‚क्षे ‚{\tiny $_{lb}$}‚दोषान्त‚रं ब्रूते [।] \quotelemma{नेद‚म‚पार्थ‚काद् भि ‚{\tiny $_{5}$}‚ द्य‚त} \cite[15b2]{vn-msN} इति न पृथ‚ग्वाच्यं स्यादिति ॥ ४ ॥
	{\color{gray}{\rmlatinfont\textsuperscript{§~\theparCount}}}
	\pend% ending standard par
      ‚{\tiny $_{lb}$}‚

	  
	  \pstart \leavevmode% starting standard par
	\hphantom{.}\quotelemma{य‚स्मिन्वाक्ये प्र‚तिज्ञादीनान्निग‚म‚न‚प‚र्य‚न्तानाम‚न्य‚त‚मोऽव‚य‚वो न भ‚व‚ति । त‚द्वा‚{\tiny $_{lb}$}‚क्यं हीनं} \cite[15b3]{vn-msN} निग्र‚ह‚स्था ‚{\tiny $_{6}$}‚ न‚त्वे कार‚ण‚माह । \quotelemma{साध‚नाभावे साध्यासिद्धे‚{\tiny $_{lb}$}‚रिति} \cite[15b3]{vn-msN} । इद‚न्निराक‚रोति [।] न \quotelemma{प्र‚तिज्ञादीनामित्यादिना} \cite[15b3]{vn-msN} । ‚{\tiny $_{lb}$}‚प्र‚तिज्ञाग्र‚ह‚ण‚मुप‚ल‚क्ष‚णार्थं तेनोप‚न‚य‚निग‚म‚न‚योर‚पि ‚{\tiny $_{7}$}‚ प‚रिग्र‚हः । \quotelemma{उद्योत‚क‚र‚स्य} म‚त‚मुप‚{\tiny $_{lb}$}‚न्य‚स्य‚ति । \quotelemma{हीन‚मेव त‚त्} \cite[15b3]{vn-msN} । प्र‚तिज्ञान्यूनं । त‚स्याः प्र‚तिज्ञायाः न्यून‚तायाम‚पि ‚{\tiny $_{lb}$}‚निग्र‚हादिति अस्यायुक्त‚तामाह । \quotelemma{यः} साध‚न‚सा ‚{\tiny $_{8}$}‚ म‚र्थ्या \quotelemma{त्प्र‚तीय‚मानार्थ‚म‚न‚र्थ‚कं श‚ब्दं} ‚{\tiny $_{lb}$}‚साध्याभिधायिनं साध‚ने \quotelemma{प्र‚युङ्क्ते स निग्र‚ह‚म‚र्हेत्} \cite[15b4]{vn-msN} । त‚था हि श‚ब्द‚स्या‚{\tiny $_{lb}$}‚नित्य‚त्त्व‚विचारे प्र‚स्तुते य‚दा ब्र‚वीति । कृत‚कानाम‚नित्य‚त्वं ‚{\tiny $_{9}$}‚ \leavevmode\ledsidenote{\textenglish{68a/msK}} दृष्ट‚ङ्कृत‚क‚श्च श‚ब्द ‚{\tiny $_{lb}$}‚इति । त‚दा व‚च‚न‚द्व‚या देव‚साध्यार्थः प्र‚तीय‚त इति निर‚र्थ‚क‚म्प्र‚तिज्ञाव‚च‚नं । नार्थोप‚सं ‚{\tiny $_{1}$}‚ ‚{\tiny $_{lb}$}‚हित‚स्य युक्तियुक्त‚स्य प‚क्ष‚ध‚र्म‚स‚म्ब‚न्ध‚मात्र‚स्याभिधानेत्य \quotelemma{स‚मीक्षिताभिधान‚मेत} \cite[15b4]{vn-msN} ‚{\tiny $_{lb}$}‚द्वार्तिक‚कार‚स्य । अत एव चेति \cite[15b4]{vn-msN} य‚तः प्र‚तीय‚मानार्थें श‚ब्दे ‚{\tiny $_{lb}$}‚प्र‚युक्ते ‚{\tiny $_{2}$}‚ निग्र‚ह‚म‚र्ह‚ति । त‚द \quotelemma{त्राबिद्ध‚क‚र्णः} प्र‚तिब‚न्ध‚क‚न्यायेन प्र‚त्य‚व‚तिष्ठ‚ते । ‚{\tiny $_{lb}$}‚य‚द्येव‚ङ्कृत‚क‚श्च श‚ब्द इत्येत‚द‚पि न व‚क्त‚व्यं किंकार‚ण‚नी\edtext{}{\lemma{नी}\Bfootnote{? निमित्त}}म‚नित्य‚त्व‚{\tiny $_{lb}$}‚मित्येतेनैव श‚ब्दे ‚{\tiny $_{3}$}‚ पि कृत‚क‚त्व‚म‚नित्य‚त्व‚ञ्चोभ‚यं प्र‚तिप‚द्य‚ते । य‚स्मात्पूर्व‚म‚पि श‚ब्दे ‚{\tiny $_{lb}$}‚कृत‚क‚त्व‚म्प‚रेण प्र‚तिप‚न्न‚मेव क‚र‚णाच्छ‚ब्दोपि बुद्धौ व्य‚व‚स्थितः । अतोन्व‚य‚{\tiny $_{lb}$}‚वा ‚{\tiny $_{4}$}‚ क्येन स्मृतिमात्र‚क‚मुत्पाद्य‚ते । अप्र‚तिप‚न्न‚कृत‚क‚त्त्व‚स्य पुनः कृत‚क‚श्च श‚ब्द इत्ये‚{\tiny $_{lb}$}‚ \leavevmode\ledsidenote{\textenglish{110/s}} त‚स्माद‚पि नैव भ‚व‚ति । य‚द्वा कृत‚कः श‚ब्द इत्येताव‚द्व‚क्त‚व्यं । कृत‚क ‚{\tiny $_{5}$}‚ त्व‚स्य त्व‚नित्य‚त्वे‚{\tiny $_{lb}$}‚नाविनाभावित्वं प‚र‚स्य प्र‚सिद्ध‚मिति श‚ब्देप्य‚नित्य‚त्वं प्र‚तिप‚द्य‚त इति तेनानुकूल‚{\tiny $_{lb}$}‚मेवाच‚रितं । त‚था हि य‚दि वादिना क‚थ‚ञ्चिन्निश्चि ‚{\tiny $_{6}$}‚ त‚म्भ‚व‚ति प्र‚तिप‚न्न‚म‚नेन वादिना ‚{\tiny $_{lb}$}‚कृत‚क‚त्वं श‚ब्द इति त‚दा नैव तेन प‚क्ष‚ध‚र्मोप‚संहारः क‚र्त‚व्यो निष्फ‚ल‚त्वात् । प्र‚ति‚{\tiny $_{lb}$}‚ब‚न्ध‚मात्र‚न्तु प्र‚द‚र्श‚नीयं । अथ त‚था न ‚{\tiny $_{7}$}‚ निश्चितं । त‚थापि य‚द्य‚यं प‚रः प‚क्ष‚ध‚र्मोप‚संहारे ‚{\tiny $_{lb}$}‚म‚या कृते त‚स्यासिद्धिञ्चोद‚यिष्य‚ति । त‚दाह‚न्तां प्र‚त्य‚य‚भेद‚भेदित्वादिभिरुपायैः ‚{\tiny $_{lb}$}‚प्र‚तिनिवार‚यिष्यामि स्व ‚{\tiny $_{8}$}‚ य‚मेव वा ऽचोदित एवाश‚ङ्क्यैत‚च्चेत‚स्याधाय प‚क्ष‚ध‚र्म‚त्व‚मुप‚{\tiny $_{lb}$}‚संह‚र्त‚व्य‚मेव कृत‚क‚श्च श‚ब्द इति । य‚दाप्येवं वादी निश्चित‚वान् कृत‚क‚त्व‚स्यानित्य‚त्वे‚{\tiny $_{lb}$}‚नाविनाभा ‚{\tiny $_{9}$}‚ \leavevmode\ledsidenote{\textenglish{68b/msK}} वित्वं प‚र‚स्य प्र‚सिद्ध‚मिति त‚स्याम‚प्य‚व‚स्थायां कृत‚कः श‚ब्द इत्येतावेद‚व ‚{\tiny $_{lb}$}‚व‚क्त‚व्यं । य‚थोक्त‚न् ‚{\tiny $_{lb}$}‚ 
	    \pend% close preceding par
	  
	    
	    \stanza[\smallbreak]
	  \flagstanza{\tiny\textenglish{...44}}{\normalfontlatin\large ``\qquad}त‚द्भान‚हेतुभावौ हि दृष्टान्ते त‚द‚वेदिनः&‚{\tiny $_{lb}$}‚ख्याप्य‚ते विदुषाम्वाच्यो हेतुरे ‚{\tiny $_{1}$}‚ व हि केव‚ल इति ॥{\normalfontlatin\large\qquad{}"}\&[\smallbreak]
	  
	  
	  
	    \pstart  \leavevmode% new par for following
	    \hphantom{.}
	   [४४] ॥
	{\color{gray}{\rmlatinfont\textsuperscript{§~\theparCount}}}
	\pend% ending standard par
      ‚{\tiny $_{lb}$}‚

	  
	  \pstart \leavevmode% starting standard par
	त‚देत‚न्निय‚माभ्युप‚ग‚म इत्य‚धिकं निग्र‚ह‚स्थानं । विशिष्टे विष‚ये स्थाप‚य‚ति त‚ञ्च ‚{\tiny $_{lb}$}‚विशिष्टं विष‚य‚माह । \quotelemma{य‚त्रेत्यादिना} \cite[15b5]{vn-msN} । न‚नु चेदं निय ‚{\tiny $_{2}$}‚ माभ्युप‚ग‚मे वेदि‚{\tiny $_{lb}$}‚त‚व्य‚मिति भाष्य‚कारेणैवोक्तं । त‚त्किम‚त्र दूष‚ण‚माचार्येणोक्तं स‚त्य‚न्न किञ्चिदुक्तं । ‚{\tiny $_{lb}$}‚आ[चा]र्येण तु \quotelemma{प‚क्षिलो} क्त‚मेव‚नूद्य‚तेऽभ्य‚नुज्ञानार्थ‚म् ॥ ४ ॥
	{\color{gray}{\rmlatinfont\textsuperscript{§~\theparCount}}}
	\pend% ending standard par
      ‚{\tiny $_{lb}$}‚

	  
	  \pstart \leavevmode% starting standard par
	\hphantom{.}\quotelemma{श‚ब्दार्थ‚योः पुन‚र्व‚च‚नं पुन‚रुक्त} मित्य‚स्याप‚वाद‚माह \quotelemma{अन्य‚त्रानुवादादिति} । अनु‚{\tiny $_{lb}$}‚वादो निग‚म‚नं । अनुवादो हि न पुन‚रुक्त‚व्य‚प‚दे ‚{\tiny $_{4}$}‚ शं ल‚भ‚ते । श‚ब्दाभ्याम‚र्थ‚विशेषोत्प‚त्तिः । ‚{\tiny $_{lb}$}‚य‚स्मात्साध्य‚निर्देशः । प्र‚तिज्ञासिद्ध‚निर्देशो निग‚म‚न‚मित्युक्तं । पुनः श‚ब्द‚श्च नानात्वे ‚{\tiny $_{lb}$}‚दृष्टः । पुन‚रि ‚{\tiny $_{5}$}‚ य‚म‚चिर‚प्र‚भा निश्च‚र‚तीत्य‚प्यावेदित‚मेव । य‚द्येव‚न्त‚त्र त‚र्हि पुन‚{\tiny $_{lb}$}‚\leavevmode\ledsidenote{\textenglish{111/s}} रुक्त‚तायाः प्राप्तिरेव नास्तीति किम‚र्थ‚म‚य‚म‚प‚वादः प्रार‚भ्य‚ते । स‚त्य‚मेव‚मे ‚{\tiny $_{6}$}‚ त‚त् । ‚{\tiny $_{lb}$}‚त एव तु प्र‚कृष्ट‚तार्किकाः प्र‚ष्ट‚व्याः । क‚थ‚मेत‚दिति । अस्माक‚न्तु किं प‚र‚कीयाभिर्गृह‚{\tiny $_{lb}$}‚चिन्ताभिश्चिन्तिताभिरित्य‚ल‚म्प्र‚स‚ङ्गेन । अत्र चेद‚म‚पि द्वितीय‚सूत्र‚म ‚{\tiny $_{7}$}‚ स्ति अर्था‚{\tiny $_{lb}$}‚दाप‚न्न‚स्य स्व‚श‚ब्देन पुन‚र्व‚च‚न‚मिति \href{http://sarit.indology.info/?cref=ns\%C5\%AB.5.2.15}{न्या० सू० ५।२।१५ } त‚दाचार्येण नोप‚न्य‚स्त‚{\tiny $_{lb}$}‚मुप‚ल‚क्ष‚णार्थ‚त्वात् । त‚द्भाष्य[म]प‚क्षिप्य निराक‚रिष्य‚ति । \quotelemma{ग‚म्य‚मानार्थं} पुन‚र्व‚च‚न‚म‚{\tiny $_{lb}$}‚पी ‚{\tiny $_{8}$}‚ त्यादिना \cite[15b9]{vn-msN} । \quotelemma{अत्रेत्यादिना} \cite[15b7]{vn-msN} दूष‚ण‚मार‚भ‚ते । एत‚दुक्त‚म्भ‚व‚ति । ‚{\tiny $_{lb}$}‚य‚त्र श‚ब्द‚साम्येप्य‚र्थो न भिद्य‚ते त‚त्रार्थ‚पुन‚रुक्ते ‚{\tiny $_{9}$}‚ \leavevmode\ledsidenote{\textenglish{69a/msK}} न ग‚तं य‚त्र तु श‚ब्द‚साम्येप्य‚र्थ‚भेद‚स्त‚त्र ‚{\tiny $_{lb}$}‚श‚ब्द‚पुन‚रुक्त‚तायाम‚पि न किञ्चित्कृतं । किम‚स्त्य‚य‚मीदृशः स‚म्भ‚वो य‚च्छ‚ब्द‚पुन‚रुक्त‚{\tiny $_{lb}$}‚तायाम‚प्य‚र्थ‚भेदोस्तीत्य‚त आह । \quotelemma{य‚था ह‚स ‚{\tiny $_{1}$}‚ ति ह‚स‚तीत्यादि} \cite[15b7]{vn-msN} । अत्र हि ‚{\tiny $_{lb}$}‚पूर्वो ह‚स‚तिश‚ब्दः स‚प्त‚म्य‚न्तो द्वितीय‚श्च तिङ्न्त इत्य‚र्थ‚भेदः । एव‚मुत्त‚र‚त्रापि । काव्य ‚{\tiny $_{lb}$}‚ईदृशः स‚म्भ‚वो न तु वाद इत्याश‚ङ्कायामुदाह‚र ‚{\tiny $_{2}$}‚ ति । य‚था चेत्यादि \cite[15b8]{vn-msN} । ‚{\tiny $_{lb}$}‚ \quotelemma{न‚नु} चेहाप्य‚र्थ‚भेद‚व‚च्छ‚ब्दो\edtext{}{\lemma{ब्दो}\Bfootnote{? ब्द}}भेदोप्य‚स्ति सुब‚न्त‚तिङ‚न्त‚त‚या । स‚त्त्य‚न्न केव‚ल‚म‚{\tiny $_{lb}$}‚त्रापि । अत्राप्य‚नित्त्यः श‚ब्दोऽनित्यःश‚ब्द इत्य‚त्रास्त्येव श‚ब्द‚भेदः स्व‚ल‚क्ष ‚{\tiny $_{3}$}‚ ण‚भेदात् । ‚{\tiny $_{lb}$}‚अन्य‚था न क्र‚म‚भावि श्र‚व‚णं स्यात् । स‚मान‚श्रुतिस‚माश्र‚य‚मिह पौन‚रुक्त्यं य‚दि ‚{\tiny $_{lb}$}‚व्य‚व‚स्थाप्य‚ते त‚द‚त्रापि तुल्य‚मेव । अर्थ‚भेद एवायं । क्रि ‚{\tiny $_{4}$}‚ याभेदादिवाच्य‚भेदात् । ‚{\tiny $_{lb}$}‚त‚द्व‚ल‚क‚ल्पित एव हि प‚द‚भेदः । \quotelemma{ग‚म्य‚मानार्थं पुन‚र्व‚च‚न‚म‚पि पुन‚रुक्त‚मिति} \cite[15b9]{vn-msN} ‚{\tiny $_{lb}$}‚ द्वितीय‚म्पुन‚रुक्त‚ल‚क्ष‚ण‚सूत्र‚मुप‚ल‚क्ष‚य‚ति । अस्य ‚{\tiny $_{5}$}‚ चोदाह‚र‚णं \quotelemma{वात्स्याय‚नेन} न्याय‚भाष्य ‚{\tiny $_{lb}$}‚उक्तं । साध‚र्म्य‚व‚ति प्र‚योगे वैध‚र्म्य‚स्य । आचार्य‚स्तु प्र‚तिज्ञायाम‚प्येत‚त्स‚मान‚मित्यागूर्य ‚{\tiny $_{lb}$}‚प्र‚तिज्ञायाः साध‚न‚वा ‚{\tiny $_{6}$}‚ क्येऽनुप‚न्यासं प्र‚तिपाद‚यितुकामो व‚क्रोक्त्या प्र‚तिज्ञाव‚च‚न‚मेवो‚{\tiny $_{lb}$}‚\leavevmode\ledsidenote{\textenglish{112/s}} दाह‚र‚ण‚त्वेनोप‚न्य‚स्य‚ति । \quotelemma{निय‚त‚प‚द‚प्र‚योगे साध‚न‚वाक्ये य‚था प्र‚तिज्ञाव‚च‚न‚मिति} ‚{\tiny $_{lb}$}‚ \cite[15b9]{vn-msN} निय‚तानां ‚{\tiny $_{7}$}‚ प‚दानां प्र‚योगो य‚स्मिन्निति कार्यं । इद‚म्प्र‚तिक्षिप‚ति ‚{\tiny $_{lb}$}‚ \quotelemma{अर्थ‚पुन‚रुक्तेनैव ग‚त[ार्थ]त्वान्न पृथ‚ग्वाच्य‚मिति} । य‚था ह्येक‚श‚ब्द‚प्र‚तिपादितेर्थे ‚{\tiny $_{lb}$}‚त‚त्प्र‚तिपाद‚नाय प‚र्या ‚{\tiny $_{8}$}‚ \leavevmode\ledsidenote{\textenglish{69b/msK}} य‚श‚ब्दान्त‚र‚मुपादीय‚मान‚म‚न‚र्थ‚क‚न्त‚था साम‚र्थ्य‚ग‚म्येप्य‚र्थ इति ‚{\tiny $_{lb}$}‚अर्थ‚पुन‚रुक्तेनैवास्य स‚ङ्ग्र‚ह इति स‚मुदायार्थः । क्व पुन‚रेत‚त्प्र‚ति ‚{\tiny $_{1}$}‚ ज्ञादिव‚च‚न‚म‚र्था‚{\tiny $_{lb}$}‚प‚त्तिल‚भ्यं पुन‚रुक्तं स‚न्निग्र‚ह‚स्थान‚म्भ‚व‚तीति प्र‚श्ने \quotelemma{निय‚त‚प्र‚योगे साध‚न‚वाक्ये} \cite[15b9]{vn-msN} ‚{\tiny $_{lb}$}‚ इत्येत‚देव‚म्प‚क्षेण विवृणोति । अय‚म‚पि दोषो ग‚म्य‚मानार्थ‚पु ‚{\tiny $_{2}$}‚ न‚र्व‚च‚न‚कृतः साध‚न‚वाक्य ‚{\tiny $_{lb}$}‚एव निय‚त‚प‚द‚प्र‚योग इति व‚र्त‚ते । इद‚मुक्त‚म्भ‚व‚ति । य‚दा प्राश्निकाः श‚ब्दार्थ‚प्र‚माण‚{\tiny $_{lb}$}‚प्र‚विच‚य‚निपुना\edtext{}{\lemma{निपुना}\Bfootnote{? णा}}ः प्रेक्षाव‚न्तोत्यं ‚{\tiny $_{3}$}‚ त‚म‚व‚हित‚म‚न‚स‚श्च भ‚व‚न्ति । प्र‚तिवाद्य‚पि ‚{\tiny $_{lb}$}‚त‚थाभूत एवेति व‚द‚न्ति य‚द‚न्त‚रेण न साध्य‚सिद्धिः त‚देव प्र‚योक्त‚व्यं । नाभ्य‚धिक‚{\tiny $_{lb}$}‚मिति त‚दाय‚न्दोषो नान्य‚था ‚{\tiny $_{4}$}‚ य‚स्मात्क‚रुणाप‚र‚त‚न्त्र‚चेत‚सोऽनिब‚न्ध‚न‚व‚त्स‚लाः प्र‚तिवादि‚{\tiny $_{lb}$}‚न‚म‚तिद\edtext{}{\lemma{तिद}\Bfootnote{? दु}}र्ल्ल‚भ‚मिव शिष्यं न्याय‚व‚र्त्माव‚तार‚यितुं य‚त‚न्ते त‚त्र पुन‚र्व‚च‚न‚म‚पि न ‚{\tiny $_{lb}$}‚दोषाय । ए ‚{\tiny $_{5}$}‚ त‚देवाह । \quotelemma{व्याच‚क्षाणो हि वादी साक्षीप्र‚भृतीनाम‚स‚म्य‚क्श्र‚व‚ण} प्र‚तिप‚त्ति‚{\tiny $_{lb}$}‚श‚ङ्क‚या क‚र‚ण‚भूत‚या \quotelemma{स‚म्य‚क्श्र‚व‚ण‚प्र‚तिप‚त्य‚र्थ‚म्पुनः पुन‚र्ब्रूयाद‚पीति} \cite[15b10]{vn-msN} । ना‚{\tiny $_{lb}$}‚\quotelemma{विष‚य ‚{\tiny $_{6}$}‚ त्वादिति} प‚रः । इद‚मेव व्याच‚ष्टे \quotelemma{नाय‚म्वादी गुरुः} \cite[15b11]{vn-msN} प्र‚तिवादिनः । ‚{\tiny $_{lb}$}‚न \quotelemma{शिष्यः} प्र‚तिवाद्य‚पि वादिनः । द्व‚योर‚पि प‚र‚स्प‚र‚जिगीष‚या व्य‚व‚स्थानादिति । ‚{\tiny $_{lb}$}‚त‚स्मात् ‚{\tiny $_{7}$}‚ न वादिना प्र‚तिवादी य‚त्न‚तः प्र‚तिपाद‚नीयः । \quotelemma{ने} \cite[15b11]{vn-msN} त्याद्याचार्यः । ‚{\tiny $_{lb}$}‚य‚दि नाम प्र‚तिवादी न प्र‚तिपाद्य‚ते य‚त्नेन । साक्षिण‚स्त्व‚व‚श्यं य‚त्नेन प्र‚तिपाद्यास्त‚{\tiny $_{lb}$}‚द्बोध‚ना ‚{\tiny $_{8}$}‚ \leavevmode\ledsidenote{\textenglish{70a/msK}} देव हि वादिनो ज‚योन्य‚था च प‚राज‚य इति क‚थं साक्षिण एव न प्र‚ति‚{\tiny $_{lb}$}‚पाद‚येत\edtext{}{\lemma{येत}\Bfootnote{? त्}}किञ्चाव‚श्यं साक्षिव‚त्प्र‚तिवाद्य‚पि प्र‚तिपाद्यः । क‚स्मात्त‚द‚प्र‚तिपाद‚ने ‚{\tiny $_{lb}$}‚दोषाभि ‚{\tiny $_{1}$}‚ धानात् । त‚च्छ[ले]न साक्षिप्र‚भृत‚यः प्र‚त्य‚व‚मृश्य‚न्ते य‚दि साक्षिप्र‚भृत‚यो न ‚{\tiny $_{lb}$}‚प्र‚तिपाद्या भ‚वेयुस्त‚तो य‚द् भ‚व‚द्भिः प‚रिष‚त्प्र‚तिवादिभ्यान्त्रिर‚भिहित‚म‚विज्ञात‚{\tiny $_{2}$}‚म‚{\tiny $_{lb}$}‚विज्ञातार्थ निग्र‚ह‚स्थान‚मुक्तं \href{http://sarit.indology.info/?cref=ns\%C5\%AB.5.2.16}{न्या० सू० ५।२।१६ } । त‚द्विरुद्ध्य‚त इत्य‚र्थः । ‚{\tiny $_{lb}$}‚ \leavevmode\ledsidenote{\textenglish{113/s}} य‚च्चोच्य‚ते नायं शिष्य इति त‚द‚सिद्धं । \quotelemma{प्र‚तिपाद्य‚स्य शिष्य‚त्वात्} \cite[16a1]{vn-msN} । त‚त्व‚ज्ञा‚{\tiny $_{lb}$}‚नार्थ‚त‚या प्र‚तिपाद्य एव ‚{\tiny $_{3}$}‚ शिष्योन्य‚स्य त‚ल्ल‚क्ष‚ण‚स्याभावात् । प्र‚तिवादी च त‚थाभूतः ‚{\tiny $_{lb}$}‚क‚थं न शिष्यः । किमुच्य‚ते नैवासौ प्र‚तिवादी त‚त्व‚ज्ञानार्थास्प‚र्ध‚या व्युत्थित‚त्वा‚{\tiny $_{lb}$}‚दिति । ‚{\tiny $_{4}$}‚ त‚द‚युक्तं । पूर्व्व‚ञ्जिगीषुवाद‚प्र‚तिषेधात् \cite[16a1]{vn-msN} । एव‚म‚पि नैवासौ य‚त्न‚{\tiny $_{lb}$}‚प्र‚तिपाद्य‚स्त्रिर‚भिधान‚निय‚म‚स्य \quotelemma{म‚ह‚र्षिणा} कृत‚त्वादित्य‚त आह । \quotelemma{त्रि ‚{\tiny $_{5}$}‚ र‚भिधान} ‚{\tiny $_{lb}$}‚ \cite[16a1]{vn-msN} व‚च‚नादित्यादि । अनेनैत‚द्द‚र्श‚य‚ति । य‚द्व‚क्ष्य‚ति । य‚दि ताव‚त्प‚र‚प्र‚तिपाद‚नार्था ‚{\tiny $_{lb}$}‚प्र‚वृत्तिः किन्त्रिर‚भिधीय‚ते त‚था त‚था स ग्राहिणीयो ‚{\tiny $_{6}$}‚ य‚थास्य प्र‚तिप‚त्तिर्भ‚व‚ति । ‚{\tiny $_{lb}$}‚अथ प‚रोप‚ताप‚नार्था त‚थापि किं त्रिर‚भिधीय‚ते । साक्षिणाङ्क‚र्णे निवेद्य प्र‚तिवादी ‚{\tiny $_{lb}$}‚क‚ष्टाप्र‚तीत‚द्रुत‚संक्षिप्तादिभिरुप‚द्रो ‚{\tiny $_{7}$}‚ त‚व्यो य‚थोत्त‚र‚प्र‚तिप‚त्तिविमूढ‚स्तूष्णीम्भ‚व‚तीति । ‚{\tiny $_{lb}$}‚ \quotelemma{न चेद} \cite[16a1]{vn-msN} मिति श‚ब्दार्थ‚योः पुन‚र्व‚च‚नं । ग‚म्य‚मानार्थ‚पुन‚र्व‚च‚नं च । अभेद‚मेव ‚{\tiny $_{lb}$}‚साध‚य‚ति । \quotelemma{विनिय ‚{\tiny $_{8}$}‚ ते} \cite[16a1]{vn-msN} त्यादिना । \quotelemma{आधिक्यं} \cite[16a2]{vn-msN} हेतूदाह‚र‚ण‚योर्दोषः । ‚{\tiny $_{lb}$}‚एकेन कृत‚त्वादित‚र‚स्यान‚र्थ‚क्य‚मिति व‚च‚नात् । पुन‚र्व‚च‚नेपि ग‚तो ज्ञातः पूर्वेणैव ‚{\tiny $_{lb}$}‚श‚ब्देनार्थो य‚स्यो ‚{\tiny $_{9}$}‚ \leavevmode\ledsidenote{\textenglish{70b/msK}} त्त‚र‚स्य प‚द‚स्य त‚देव‚मुक्तं ।  त‚स्याधिक्य‚मेव दोष इत्य‚धिकृतं । ‚{\tiny $_{lb}$}‚किम्पुन‚र्निय‚त‚प‚द‚प्र‚योगेऽय‚न्दोष इत्युक्त‚मिति चेदाह । \quotelemma{प्र‚प‚ञ्च‚क‚थायाम‚दोष} ‚{\tiny $_{lb}$}‚ \cite[16a2]{vn-msN} इत्य‚भिस‚म्ब‚न्धः । क ‚{\tiny $_{1}$}‚ स्य \quotelemma{हेत्वादिबाहुल्य‚स्य} \cite[16a2]{vn-msN} । पुन‚र्व‚च‚न‚स्य ‚{\tiny $_{lb}$}‚च । आदिश‚ब्देनोदाह‚र‚ण‚बाहुल्य‚ग्र‚ह‚णं । कीदृश्याम‚निरूपितैकार्थ‚साध‚नाधिक‚र‚{\tiny $_{lb}$}‚णायां अर्थः साध्यः । अर्थ्य‚त इ ‚{\tiny $_{2}$}‚ ति कृत्वा साध‚नं । हेतुर‚धिक‚र‚ण‚न्ध‚र्मी । अर्थ‚स‚हितं ‚{\tiny $_{lb}$}‚साध‚न‚म‚र्थ‚साध‚नं । म‚ध्य‚प‚द‚लोपात् । एक‚ञ्च त‚द‚र्थ‚साध‚न‚ञ्च त‚थोक्त‚म् । त‚स्याधि‚{\tiny $_{lb}$}‚क‚र‚ण‚न्त‚द‚निरूपि ‚{\tiny $_{3}$}‚ त‚मेकार्थ‚साध‚नाधिक‚र‚णं य‚स्यां प्र‚प‚ञ्च‚क‚थायां प्र‚तिवादिना ध‚र्मिणो ‚{\tiny $_{lb}$}‚जीव‚श‚रीरादेर्नैको ध‚र्मो नैरात्म्यादिषु प्र‚मातुमिष्टोऽपित्व‚नेकः क्ष‚णि ‚{\tiny $_{4}$}‚ क‚त्वानात्म‚त्वा‚{\tiny $_{lb}$}‚नीश्व‚र‚क‚र्त्तृत्वादिस्त‚था नैकेनैव हेतुना किन्त्व‚नेकेनापि त‚स्यामित्य‚र्थः । एत‚देव ‚{\tiny $_{lb}$}‚य‚थाक्र‚मं ब्रूते । नानार्थ‚साध‚नेप्सायां \quotelemma{नाना ‚{\tiny $_{5}$}‚ साध‚नेप्सायां वा श्रोतुरिति} \cite[16a2]{vn-msN} ‚{\tiny $_{lb}$}‚ पूर्व्व‚कः साध‚न‚श‚ब्दो भाव‚साध‚न‚त्वात्सिद्धिव‚च‚नः । उत्त‚र‚स्तु क‚र‚ण‚साध‚न‚त्वाद्धेतुव‚{\tiny $_{lb}$}‚च‚नः । त‚स्माद्धेत्वादिबाहु ‚{\tiny $_{6}$}‚ ल्यं व‚च‚न‚बाहुल्यं साध‚नेन विनिय‚त‚प‚दे दोषः । क‚स्मा[त्] ‚{\tiny $_{lb}$}‚ \leavevmode\ledsidenote{\textenglish{114/s}} प्र‚तीताय्याभावात् । प्र‚त्य तुल्यो दोष इति कृत्वा स‚ङ्ग्र‚ह एव न्याय्यः । अधिक‚मेव वा ‚{\tiny $_{lb}$}‚व‚क्त‚व्यं पुन‚रुक्त ‚{\tiny $_{7}$}‚ मेव चेत्य‚र्थः । अन‚योरेक‚स्मिन्द्वितीय‚स्यान्त‚र्भावात् । क‚थं पुनः श‚ब्द‚{\tiny $_{lb}$}‚पुन‚रुक्ते ऽधिक‚स्यान्त‚र्भाव इत्याह । \quotelemma{प‚र्य्याय‚श‚ब्द‚क‚ल्पो} \cite[16a4]{vn-msN} ह्य‚प‚रो द्वितीयो ‚{\tiny $_{lb}$}‚हेतुरेक‚हेतु \quotelemma{प्र ‚{\tiny $_{8}$}‚ तिपादिते विष‚ये प्र‚व‚र्त्त‚मानः} [।] किं कार‚णं [।] प्र‚तिपाद्य‚स्य विशेषाभा‚{\tiny $_{lb}$}‚वात् । अर्थ‚स्य पुन‚रुक्त‚न्त‚र्हि क‚थ‚म‚धिकेन्त‚र्भ‚व‚तीत्याह \quotelemma{अर्थः पुनः प्र‚तिपाद‚ना ‚{\tiny $_{9}$}‚ \leavevmode\ledsidenote{\textenglish{71a/msK}} न्न भिद्य‚त} ‚{\tiny $_{lb}$}‚ इति ॥ अर्थ‚श‚ब्देनार्थ‚पुन‚रुक्त‚मुप‚ल‚क्ष‚य‚ति । पुनः प्र‚तिपाद्य‚ते अनेनेति पुनः प्र‚तिपाद‚नं ‚{\tiny $_{lb}$}‚हेतूदाह‚र‚णाधिक‚मेव । इद‚मुक्त‚म्भ‚व‚ति [।] स्फुट‚मेवास्य उदा ‚{\tiny $_{1}$}‚ ह‚र‚णाधिकेन्त‚र्भावः । ‚{\tiny $_{lb}$}‚त‚थाहि साध‚र्म्य‚व‚ति प्र‚योगे वैध‚र्म्योदाह‚र‚ण‚स्याप्र‚योगोऽर्थ‚पुन‚रुक्त‚स्योदाह‚र‚ण‚मुक्तं । ‚{\tiny $_{lb}$}‚य‚त्पुन‚रुक्त‚मेवाद्य‚प‚वाद‚प्र‚तिषेधः सुज्ञानः ॥ ० ॥ 
	{\color{gray}{\rmlatinfont\textsuperscript{§~\theparCount}}}
	\pend% ending standard par
      ‚{\tiny $_{lb}$}‚

	  
	  \pstart \leavevmode% starting standard par
	विज्ञातो वाक्यार्थो य‚स्य त्रिर‚भिहित‚स्य त‚त्त‚था । विशेष‚ण‚स‚मासो वा विज्ञात‚{\tiny $_{lb}$}‚श्चासौ वाक्यार्थ‚श्चेति त्रिर‚भिहित‚स्य वादिनेति प्र‚तिप‚त्त‚व्यं ॥ प्र‚तिवादिना प ‚{\tiny $_{3}$}‚ द‚{\tiny $_{lb}$}‚प्र‚त्युच्चार‚ण‚मिति स‚म्ब‚न्ध‚नीयं । त्रिव‚च‚नं स‚कृद‚भिहित‚स्यान‚नुभाष‚णेपि न निग्र‚ह ‚{\tiny $_{lb}$}‚इति ज्ञाप‚नार्थं । अप्र‚त्युच्चार‚ण‚ञ्च श‚ब्द‚द्वारेणार्थ‚द्वारेण ‚{\tiny $_{4}$}‚ वा । निग्र‚ह‚स्थान‚त्वे कार‚ण‚{\tiny $_{lb}$}‚माह । \quotelemma{अप्र‚त्युच्चार‚य‚न् किमाश्र‚य‚म्प‚र‚प‚क्षे प्र‚तिषेधं ब्रूयादि} \cite[16a7]{vn-msN} ति न विष‚य‚न्दू‚{\tiny $_{lb}$}‚ष‚णाभिधान‚न्नास्तीत्य‚र्थः ॥ इद‚न्त्व‚योक्तं ‚{\tiny $_{5}$}‚ म‚ति ङ्कृत्त्वा दूष‚ण‚म्वाच्यं । एव‚न्दूष‚ण‚वाक्य ‚{\tiny $_{lb}$}‚ \leavevmode\ledsidenote{\textenglish{115/s}} म‚पि साध‚न‚वादिना प्र‚त्य‚नुभाष्य प‚रिह‚र्त‚व्यं । अतो द्व‚योर‚पीदं निग्र‚ह‚स्थानं । अत्र \quotelemma{भार‚{\tiny $_{lb}$}‚द्वाजो} न्य‚क्षेणा ‚{\tiny $_{6}$}‚ क्षेप‚न्ताव‚त्क‚रोति \quotelemma{उत्त‚रेणाव‚सानात्प‚रिज्ञानान्नेद‚न्निग्र‚ह‚स्थान‚मि} ति चेदि‚{\tiny $_{lb}$}‚ति । इद‚म्वाक्य‚म्व्याच‚ष्टे [।] \quotelemma{स्वादेत‚दि} \cite[16a7]{vn-msN} त्यादिना । नोत्त‚र‚विष‚य‚प‚रिज्ञानादिति ‚{\tiny $_{lb}$}‚स ए ‚{\tiny $_{7}$}‚ व प्र‚तिविध‚त्ते । \quotelemma{य‚द्य‚य‚मि} \cite[16a8]{vn-msN} त्याद्य‚स्यैव विभागः । अप्र‚तिज्ञानाच्चेति स ‚{\tiny $_{lb}$}‚एव । \quotelemma{उत्त‚र‚ञ्चाश्र‚याभावे प‚र‚प‚क्षोप‚क्षेपाभावे स‚त्य‚युक्त‚मिति युक्त‚म‚प्र‚त्युच्चार‚णे ‚{\tiny $_{8}$}‚ ‚{\tiny $_{lb}$}‚निग्र‚ह‚स्थान‚मित्ये} तावान् प‚र‚कीयो ग्र‚न्थः । अत्राचार्यो दूष‚ण‚म्व‚क्तुमार‚भ‚ते । \quotelemma{य‚दि नाम ‚{\tiny $_{lb}$}‚वादीस्व‚साध‚नार्थ‚स्य विव‚र‚ण‚व्याजेन प्र‚स‚ङ्गाद‚प‚राप‚रं घोष‚ये ‚{\tiny $_{9}$}‚ \leavevmode\ledsidenote{\textenglish{71b/msK}} त् । य‚थोदा- ‚{\tiny $_{lb}$}‚हृत‚म्प्राक्त‚त्र क‚र‚ण‚भुव‚नानि बुद्धिस‚त्कार‚ण‚पूर्व्व‚काणीति प्र‚तिज्ञाश‚रीरादिव्याख्यान‚{\tiny $_{lb}$}‚च्छ‚द्म‚ना स‚क‚लं वैशेषिक‚त‚न्त्रं घोष‚येदिति} \cite[16a10]{vn-msN} । त‚था जिज्ञासित‚म ‚{\tiny $_{1}$}‚ र्थ‚मात्र‚{\tiny $_{lb}$}‚मुक्त्वा क‚थां विस्तार‚येद्य‚दि नाम वादीति व‚र्त‚ते । किङ्कृत्वा विस्तार‚येदित्याह । ‚{\tiny $_{lb}$}‚ \quotelemma{प्र‚तिज्ञादिष्व‚र्थ‚विशेष‚ण‚प‚र‚म्प‚र‚याऽप‚रान् सिध्य‚नुप‚योगिनोर्थानुप‚क्षिप्य} \cite[16b10]{vn-msN} । ‚{\tiny $_{2}$}‚ ‚{\tiny $_{lb}$}‚ य‚था निद‚र्शितं पूर्व्व‚न्नित्यः श‚ब्दोऽनित्यः श‚ब्द इति विवादे \quotelemma{जैमिनीयः} प्र‚माण‚य‚ति । ‚{\tiny $_{lb}$}‚ \quotelemma{द्वाद‚श‚ल‚क्ष‚णे} त्यादिना । व्याच‚ष्टे च द्वाद‚श‚ल‚क्ष‚णानि । य‚था वा \quotelemma{ऽक्ष‚पादा} ए ‚{\tiny $_{3}$}‚ व‚ङ्कुर्व‚न्ति । ‚{\tiny $_{lb}$}‚किम‚मी स‚र्वे संस्काराः क्ष‚णिका नो वेति विवादे रूप‚त्वादिसामान्याश्र‚य[त्त्वा]त्त‚दा‚{\tiny $_{lb}$}‚श्र‚यास्त‚द्विष‚याश्च प्र‚त्य‚क्षाद‚यः प्र‚त्य‚याः स्वात्म‚लाभान‚न्त ‚{\tiny $_{4}$}‚ र‚प्र‚ध्वंसिनो न भ‚व‚न्ति । ‚{\tiny $_{lb}$}‚स‚मानानाम‚स‚मान‚जातीय‚द्र‚व्य‚संयोग‚विभाग‚ज‚नित‚श‚ब्द‚कार्य‚श‚ब्दाभिधेय‚त्वात्प्राग‚भा‚{\tiny $_{lb}$}‚वादिव‚दिति । न‚नु च प्र‚तिज्ञादीष्वित्य‚त्रादिश‚ब्देन किं गृह्य‚ते । न ताव‚द्धेतूदाह‚र‚णे ‚{\tiny $_{lb}$}‚त‚न्मात्र‚मुक्त्वेति व‚च‚नात् । न चाप‚रः क‚श्चित्प्र‚स्तुतोऽत्रेति [।] नैष दोषो य‚तो ‚{\tiny $_{lb}$}‚हेत्वादिमात्र‚म‚प्युक्त्वेति द्र‚ष्ट ‚{\tiny $_{6}$}‚ व्यं । तेन हेत्वादीनामेवादिग्र‚ह‚णेन आक्षेप इति केचित् । ‚{\tiny $_{lb}$}‚अप‚रे पुन‚राहुर‚स्थान‚मेवेद‚माश‚ङ्कितं । क्त्वाप्र‚त्य‚य‚निर्देशेत्र । य‚स्माद‚य‚म‚त्रार्थो य‚त्र ‚{\tiny $_{lb}$}‚प्र‚तिवा ‚{\tiny $_{7}$}‚ दिना जिज्ञासित‚म‚र्थ‚मात्र‚म‚न्य‚विशेष‚ण‚र‚हित‚म‚क्ष‚णिक‚त्वादिकं त‚दुक्त्वा वाद्य‚{\tiny $_{lb}$}‚\leavevmode\ledsidenote{\textenglish{116/s}} ह‚मेत‚त्साध‚यामीत्युत्त‚र‚काल‚प्र‚माण‚मार‚च‚य‚न्प्र‚तिज्ञादिष्व‚र्थ‚विशेष‚ण‚प ‚{\tiny $_{8}$}‚ र‚म्प‚र‚याप‚रान‚{\tiny $_{lb}$}‚र्थानुप‚क्षिप्य क‚थाम्विस्तार‚येदिति । \quotelemma{त‚च्चे} \cite[16a11]{vn-msN} ति प्र‚तिज्ञादिविशेष‚ण‚प‚र‚म्प‚र‚या ‚{\tiny $_{lb}$}‚य‚द‚प्र‚स्तुत‚मेव नाट‚काख्यायिकाघोष‚ण‚क‚ल्पं वादिनोद्ग्रा ‚{\tiny $_{9}$}‚ \leavevmode\ledsidenote{\textenglish{72a/msK}} हितं । त‚दा क‚स्त‚स्य विवा‚{\tiny $_{lb}$}‚दाश्र‚य‚श्चासाव‚र्थ‚मात्र‚श्चाक्ष‚णिक‚त्वादिक‚न्त‚स्योत्त‚र‚व‚च‚ने साम‚र्थ्य‚विघातो नैवेत्य‚र्थः । ‚{\tiny $_{lb}$}‚त‚स्मान्न वादिक‚थाम‚न‚नुभाष‚माणः प्र‚ति ‚{\tiny $_{1}$}‚ वाद्युत्त‚र‚वाद्येन \edtext{}{\lemma{वाद्येन}\Bfootnote{?}} स‚म‚र्थः । किन्तु य‚द्व‚च‚न‚ना‚{\tiny $_{lb}$}‚न्त‚रीयिका जिज्ञासितार्थ‚सिद्धिस्त‚द‚व‚श्य‚मुप‚द‚र्श्य‚त एवेति स‚म्ब‚न्धः । क‚स्मात्सा‚{\tiny $_{lb}$}‚ध‚नाङ्ग‚विष‚य‚त्वाद् दूष‚ण‚स्य । प‚रो ‚{\tiny $_{2}$}‚ प‚नीते हि साध‚ने दूष‚ण‚म्प्र‚व‚र्त‚त इति स‚म्ब‚न्धः ‚{\tiny $_{lb}$}‚किन्नान्त‚रीयिका पुन‚र्जिज्ञासितार्थ‚सिद्धिरित्याह । \quotelemma{य‚था प‚क्ष‚ध‚र्म‚ता व्याप्तिप्र‚साध‚न‚{\tiny $_{lb}$}‚मात्र} मि \cite[16b1]{vn-msN} ति व्याप्तिः प्र‚सा ‚{\tiny $_{3}$}‚ ध्य‚तेनेनेति व्याप्तिप्र‚साध‚नं बाध‚क‚प्र‚माणोप‚{\tiny $_{lb}$}‚द‚र्श‚नं किमिय‚मित्य‚पि साध‚न‚प्र‚योगेऽर्थान्त‚रोप‚क्षेपः क‚र्त‚व्यो नेत्याह । न । \quotelemma{त‚त्रापि ‚{\tiny $_{lb}$}‚प्र‚स‚ङ्गान्त‚रो ‚{\tiny $_{4}$}‚ प} क्षेप \cite[16b1]{vn-msN} इति नैर‚र्थ‚क्यादिति म‚तिः [।] ताव‚त् मात्र‚मुप‚द‚र्श्य‚ते ‚{\tiny $_{lb}$}‚किं प्राग‚नुक्र‚मेण । प‚श्चात्तु दूष्य‚ते नेत्याह । \quotelemma{त‚त्रापि} दूष‚ण‚विष‚योप‚द‚र्श‚नार्थेऽनु‚{\tiny $_{lb}$}‚भा ‚{\tiny $_{5}$}‚ ष‚णे न स‚र्व्वं याव‚दुप‚न्य‚स्तं वादिना त‚द्दूष‚णाभिधानात् । \quotelemma{प्राग‚नुक्र‚मेणो‚{\tiny $_{lb}$}‚च्चार‚यित‚व्यं} । क‚स्मात् त्रिरुच्चार‚ण‚प्र‚स‚ङ्गात् । द्विरुच्चार‚ण‚प्र‚स‚ङ्ग‚मेव ‚{\tiny $_{6}$}‚ प्र‚ति‚{\tiny $_{lb}$}‚पाद‚यितुमादिप्र‚स्थान‚माच‚र‚ति । \quotelemma{दूष‚णेत्या} \cite[16b2]{vn-msN} दिना । य‚दि व‚च‚नानुक्र‚म‚घोष‚णं ‚{\tiny $_{lb}$}‚न क‚रोति निर्विष‚य‚मिदानीं दूष‚ण‚म्प्र‚स‚क्त‚मित्याह । \quotelemma{नान्त‚रीय ‚{\tiny $_{7}$}‚ क‚त्वा}[द्]\cite[16b3]{vn-msN} ‚{\tiny $_{lb}$}‚दूष‚य‚ता विष‚योप‚द‚र्श‚नं क्रिय‚त एव । क‚थ‚म्प्र‚तिदोष‚व‚च‚नं दोष‚व‚च‚नं दोष‚व‚च‚न‚म्प्र‚ति । ‚{\tiny $_{lb}$}‚यो यो दोषो भ‚ण्य‚ते त‚स्य त‚स्य विष‚यः क‚थ्य‚त इत्य ‚{\tiny $_{8}$}‚ र्थः । इद‚मेवाह [।] अस्य वाद्युक्त‚{\tiny $_{lb}$}‚स्याय‚न्दोष इति । किम्पुनः कार‚णं स‚र्व्व‚प्र‚त्युच्चार्य‚युग‚प‚द्दूष‚ण‚न्नोच्य‚त इति चेदाह । ‚{\tiny $_{lb}$}‚ \quotelemma{न‚ही} \cite[16b3]{vn-msN} त्यादि । कुतः \quotelemma{प्र‚त्य‚र्थं दो ‚{\tiny $_{9}$}‚ \leavevmode\ledsidenote{\textenglish{72b/msK}} ष‚भेदात्} । विष‚य‚व‚द् भिद्य‚ते दोष इति ‚{\tiny $_{lb}$}‚ \leavevmode\ledsidenote{\textenglish{117/s}} याव‚त् । एव‚मार‚चितादिप्र‚स्थानो द्विरुच्चार‚ण‚प्र‚स‚ङ्ग‚म्प्र‚द‚र्श‚य‚ति स‚कृदेवाप्र‚घुष्टो ‚{\tiny $_{lb}$}‚न । स‚र्व्वानुभाष‚णेस्य प्र‚द‚र्शिते वि ‚{\tiny $_{1}$}‚ ष‚य‚दोष‚स्य व‚क्तुम‚श‚क्य‚त्वात् । केव‚ल‚मिदं ‚{\tiny $_{lb}$}‚निःप्र‚योज‚न‚प‚राज‚याधिक‚र‚णं चेत्याह । \quotelemma{दूष‚ण‚वादिना दूष‚णे व‚क्त‚व्ये} \cite[16b6]{vn-msN} स‚ति ‚{\tiny $_{lb}$}‚य‚त्र स‚र्व्वानुक्र‚म‚भाष‚णं त‚त्र ‚{\tiny $_{2}$}‚ प‚र‚प‚क्ष‚क्षोभेनोप‚युज्य‚ते त‚स्याभिधान‚मिद‚न्द्विरुक्त‚प‚दो‚{\tiny $_{lb}$}‚द्भाव‚न‚ञ्चेत्येवं व्य‚त्याशेन\edtext{}{\lemma{त्याशेन}\Bfootnote{? सेन}}प‚द‚विन्यासः कार्य इति त‚स्मात्स‚र्व्वानुक्र‚म‚{\tiny $_{lb}$}‚भाष‚ण‚म्प‚राज‚या ‚{\tiny $_{3}$}‚ धिक‚र‚णं वाच्यं अत्रेदानी \quotelemma{माक्ष‚पादः} स‚र्व्व‚मिदं दूष‚ण‚म‚न‚भ्युप‚ग‚मेनैव ‚{\tiny $_{lb}$}‚पूर्व्व‚प‚क्ष‚स्यास्माभिः प्र‚तिव्यूढ‚मिति म‚न्य‚मानोऽभ्य‚नुजानाति । \quotelemma{त‚थास्त्वि ‚{\tiny $_{4}$}‚ \cite[16b7]{vn-msN} ‚{\tiny $_{lb}$}‚ति} । स्यादेत‚दित्येत‚देव व्याच‚ष्टे । \quotelemma{य‚तः कुत‚श्चिदि \cite[16b7]{vn-msN} ति} साध‚नार्थ‚{\tiny $_{lb}$}‚विव‚र‚ण‚स्य व्याजेन प्र‚तिज्ञादिष्व‚र्थ‚विशेष‚ण‚प‚र‚म्प‚रोप‚क्षेपेण चाप्र‚स‚ङ्गात् अनं ‚{\tiny $_{5}$}‚ ‚{\tiny $_{lb}$}‚त‚रीय‚काभिधानं \cite[16b7]{vn-msN} रूप‚सिद्धिनामादिव्याख्यान‚क‚ल्प‚त्त्वाद्वादिनोऽर्थान्त‚र‚{\tiny $_{lb}$}‚ग‚म‚न‚मेवेति स तेन निग्र‚हार्हः । प्रास‚ङ्गिकं ब्रुवाणः किमिति निगृ ‚{\tiny $_{6}$}‚ ह्य‚त ‚{\tiny $_{lb}$}‚इति चेदाह । न‚हि क‚श्चित् क्व‚चित् क्रिय‚माण‚प्र‚स‚ङ्गो न प्र‚युज्य‚ते । ‚{\tiny $_{lb}$}‚य‚थोक्त‚म्प्राक् नैरात्म्य‚वादिनः । त‚त्साध‚ने नृत्य‚गीतादेर‚पि प्र‚स‚ङ्ग इति । नापि ‚{\tiny $_{lb}$}‚त‚द्य‚द्वादि ‚{\tiny $_{7}$}‚ ना प्र‚स‚ङ्ग‚त्वेन आहितं । त‚स्य प्र‚तिवादिनो ऽनुभाष‚णीयं \cite[16b8]{vn-msN} । ‚{\tiny $_{lb}$}‚अनुप‚युज्य‚माण\edtext{}{\lemma{माण}\Bfootnote{? न}}त्व‚न्नै \uline{वानु} भाष‚ण‚म‚र्ह‚तीत्य‚र्थः । त‚देतेन य‚त्पूर्व‚मुक्तं \quotelemma{य‚दि नाम ‚{\tiny $_{lb}$}‚वादी ‚{\tiny $_{8}$}‚ स्व‚साध‚नार्थ‚विव‚र‚ण‚व्याजेने} \cite[15b10]{vn-msN} त्यादि त‚द‚भ्य‚नुज्ञातं । संप्र‚ति य‚दुक्तं ‚{\tiny $_{lb}$}‚त‚त्रापि न स‚र्व्वं क्र‚मेणोच्चार‚यित‚व्यं । प‚श्चाद् दूष‚ण‚म्वाच्यं द्विरुच्चार‚ण‚प्र‚स‚ङ्गादिति ‚{\tiny $_{lb}$}‚त‚द‚नु ‚{\tiny $_{9}$}‚\leavevmode\ledsidenote{\textenglish{73a/msK}} जानाति । \quotelemma{न चेद‚म‚प्य‚स्माभिरि} त्या \cite[16b8]{vn-msN} दिना ।
	{\color{gray}{\rmlatinfont\textsuperscript{§~\theparCount}}}
	\pend% ending standard par
      ‚{\tiny $_{lb}$}‚\textsuperscript{\textenglish{118/s}}

	  
	  \pstart \leavevmode% starting standard par
	\hphantom{.}\quotelemma{आचार्य आह । य‚दि भ‚व‚द्भिर‚प्ये\edtext{}{\lemma{प्ये}\Bfootnote{? पी}}}द \quotelemma{मेवेष्ट‚मेव‚न्त‚र्हि नान‚नुभाष‚ण‚म्पृथ‚{\tiny $_{lb}$}‚ग्वाच्यं} \cite[16b9]{vn-msN} । क‚स्मात् प्र‚तिभ‚याग‚त्वात् । ग‚त‚त्त्व‚मेव प्र ‚{\tiny $_{1}$}‚ तिपाद‚य‚ति । ‚{\tiny $_{lb}$}‚ उत्त‚र‚स्य ह्य‚प्र‚तिप‚त्तेर‚प्र‚तिभा \href{http://sarit.indology.info/?cref=ns\%C5\%AB.5.2.18}{न्या० सू० ५।२।१८ } । त‚तः क‚थ‚ङ्ग‚त‚मित्याह । ‚{\tiny $_{lb}$}‚न \quotelemma{चोत्त‚र‚विष‚य‚म‚प्र‚द‚र्श‚य‚न्} प्र‚तिवाद्युत्त‚रं प्र‚तिप‚त्तुं ज्ञातुम‚भिधातुं वा स‚म‚र्थः । ‚{\tiny $_{2}$}‚ ‚{\tiny $_{lb}$}‚किमिति न श‚क्त इत्याह । न \quotelemma{हीत्यादि} । चोत्त‚र‚प्र‚तिप‚त्तिरुत्त‚रा प्र‚तिप‚त्तिरि‚{\tiny $_{lb}$}‚त्य‚र्थः । सा नाक्षिप्ता येनान‚नुभाष‚णेन त‚त्त‚था । प्र‚तिषेध‚द्व‚याद्विध्य‚व‚सीय ‚{\tiny $_{3}$}‚ आक्षि‚{\tiny $_{lb}$}‚प्तोत्त‚राप्र‚तिप‚त्तिक‚मेवेति । एत‚दुक्त‚म्भ‚व‚ति । यो हि नामोत्त‚र‚म्प्र‚तिप‚द्य‚तेऽतोव‚श्यं ‚{\tiny $_{lb}$}‚त‚द्विष‚य‚म‚प्य‚वेत्य‚स्येदं दूष‚ण‚मिति प‚रिज्ञानात् ‚{\tiny $_{4}$}‚ प‚रिज्ञात‚विष‚य‚श्च क‚थं स‚चेत‚नो ‚{\tiny $_{lb}$}‚न त‚म‚नुभाष‚ते । त‚स्माद्य‚त्रान‚नुभाष‚ण‚न्त‚त्राप्र‚तिभ‚याभाव्य‚मिति सा त‚स्य व्यापिका ‚{\tiny $_{lb}$}‚त‚रूरिव‚ख‚दिर‚स्य ‚{\tiny $_{5}$}‚ त‚स्याश्च विहितं निग्र‚ह‚स्थान‚त्वं व्याप्येऽन‚नुभाष‚णे त‚दा ‚{\tiny $_{lb}$}‚क्षिप‚तीत्येत‚न्निग‚म‚न‚व्याजेनाह । \quotelemma{तेनेत्यादि} । अत्रैव दृष्टान्तं ब्रूते ग‚व्य‚प‚रामृष्ट ‚{\tiny $_{6}$}‚ ‚{\tiny $_{lb}$}‚त‚द्भेदायां सामान्य‚भूतायाम्विहित‚मिव‚सा\edtext{}{\lemma{सा}\Bfootnote{? सा}}स्नादिम‚त्त्व‚त्त‚द्व्याप्त‚बाहुलेयेऽपि‚{\tiny $_{lb}$}‚ल‚ब्ध‚मिति व‚र्त्त‚ते । प्र‚योगः पुन‚र्य‚देक‚विधान‚साम‚र्थ्याद‚नुक्त‚म‚पि ल‚भ्य ‚{\tiny $_{7}$}‚ ते । न‚नु‚{\tiny $_{lb}$}‚भूयः प्रेक्षापूर्व‚कारिणा विधात‚व्यं । त‚द्य‚था गोजातौशा\edtext{}{\lemma{गोजातौशा}\Bfootnote{? सा}}स्नादिम‚त्व‚विधान‚साम‚{\tiny $_{lb}$}‚र्थ्यात् प्र‚तिल‚ब्धं त‚र्ह्येषुशा\edtext{}{\lemma{र्ह्येषुशा}\Bfootnote{? सा}}स्नादिम‚त्वं । अप्र‚तिभानिग्र‚ह‚स्थान ‚{\tiny $_{8}$}‚ त्व‚विधान‚{\tiny $_{lb}$}‚साम‚र्थ्यात् प्र‚तिब‚द्ध‚श्चान‚नुभाष‚ण‚निग्र‚ह‚स्थान‚त्व‚मिति व्याप‚क‚विरुद्धोप‚ल‚ब्धिः [।] ‚{\tiny $_{lb}$}‚न‚नु च विष‚यं विष‚य‚श्च प्र‚प‚ञ्चोत्त‚रं प्र‚तिप‚द्य‚मानोप्य‚ति ‚{\tiny $_{9}$}‚ \leavevmode\ledsidenote{\textenglish{73b/msK}} भ‚य‚क‚म्पादिभिर्व्याकुलीकृत‚{\tiny $_{lb}$}‚चेताः प्र‚तिवादीनानुभाष‚ते स विष‚योऽन‚नुभाष‚ण‚स्याप्र‚तिभ‚यानालीढ‚स्त‚त्क‚थं सा त‚स्य ‚{\tiny $_{lb}$}‚व्यापिका य‚तोऽयं हेतुः सिद्धो भ‚विष्य‚ति ‚{\tiny $_{1}$}‚ नैव स‚म्भ‚वात् । न‚हि विष‚यं विष‚य‚विष‚य‚{\tiny $_{lb}$}‚ञ्चोत्त‚रं प्र‚तिप‚द्य‚मानः कुत‚श्चिद्विभेति वेप‚ते वा त‚द‚ज्ञान‚कृत‚त्वाद् भ‚य‚वेप‚थुस्वेदा‚{\tiny $_{lb}$}‚दीनां । अथ त‚थाभूतोऽपि भ ‚{\tiny $_{2}$}‚ यादिभिराकुलीक्रिय‚ते स त‚र्ह्यादावेव त‚थाभूतो वाद‚{\tiny $_{lb}$}‚म‚पि क‚र्तुन्नैव धाव‚ति । अपि च । य‚दि प‚रं बाला एवैवं भूता भ‚व‚न्ति । न च बाल‚{\tiny $_{lb}$}‚व्य‚व‚हारान‚धिकृत्य ‚{\tiny $_{3}$}‚ न्याय‚शास्त्राणि प्र‚णीय‚न्ते । य‚द्वैव‚म‚प्य‚प्र‚तिभायाम‚न्त‚र्भावो ‚{\tiny $_{lb}$}‚ \leavevmode\ledsidenote{\textenglish{119/s}} नैव व्याह‚न्य‚ते । य‚स्माद्विविधोत्त‚रा प्र‚तिप‚त्तिरुत्त‚राज्ञान‚रूपोत्त‚रान‚भिधान‚रूपा ‚{\tiny $_{4}$}‚ च । ‚{\tiny $_{lb}$}‚त‚स्माद्य‚त्किञ्चिदेत‚त् । अथ प‚रोप‚ताप‚नार्था त‚थापि किन्त्रिर‚भिधीय‚ते । किन्त‚र्हि ‚{\tiny $_{lb}$}‚कार्य्य‚मित्याह । \quotelemma{साक्षिणामुत्को} चोप‚स‚ङ्क्र‚मं क‚र्णें निवेद्या ‚{\tiny $_{5}$}‚ य‚म‚त्रार्थो म‚या विव‚{\tiny $_{lb}$}‚क्षित इत्युत्त‚र‚कालं प्र‚तिवाद्य‚नाथो व‚राकः क‚ष्टाऽप्र‚तीत‚द्रुत‚संक्षिप्तादिभिः ‚{\tiny $_{lb}$}‚श‚ब्दैरिति शेषः । उप‚द्रोत‚व्यः । क‚स्माद्दु ‚{\tiny $_{6}$}‚ र्भ‚णाः । अप्र‚तीताः सिंह‚ल‚भाषादिव‚द‚सं ‚{\tiny $_{lb}$}‚केतिकाः । द्रुताः शीध्र‚मुच्चारिताः । संक्षिप्ता सूत्र‚वाण्टादिव‚द्व‚र्त्तुलीकृतार्थाः । आदि‚{\tiny $_{lb}$}‚ग्र‚ह‚णेन गोपिता ‚{\tiny $_{7}$}‚ र्थानाङ्ग्र‚ह‚णं । य‚था । स‚त्वादुर्वायुस्ते दिश्यायोतायाञ्चारात्य‚{\tiny $_{lb}$}‚स्व‚न्तं प‚क्षे नोलंव‚म्विज्ञायैःवेष्टातोयास्पृष्टेशः श‚मिति । य‚था । संप्र‚ति वादी ‚{\tiny $_{lb}$}‚उत्त‚र‚प्र‚तिप‚त्तौ ‚{\tiny $_{8}$}‚ विमूढ[ः] तूष्णीं भ‚व‚ति । प‚र्ष‚त्प्र‚तिवादिभ्यां त्रिर‚भिहित‚म‚{\tiny $_{lb}$}‚विज्ञात‚म‚विज्ञातार्थ \href{http://sarit.indology.info/?cref=ns\%C5\%AB.5.2.9}{न्या० सू० ५।२।९ } मित्य‚त्र श्लिष्ट‚क‚ष्टादिश‚ब्द‚प्र‚योग‚स्य \quotelemma{मुनि} ‚{\tiny $_{lb}$}‚नानिवारित‚त्वात् नैव‚म‚न्यायं क‚तुं ल‚भ ‚{\tiny $_{9}$}‚ \leavevmode\ledsidenote{\textenglish{74a/msK}} त इति चेदाह । \quotelemma{न‚हि प‚रोप‚ताप‚क्र‚म इत्यादि} ‚{\tiny $_{lb}$}‚किञ्च न प‚रोप‚ताप‚नाय स‚न्तः प्र‚व‚र्त‚न्ते शास्त्राणि वा प्र‚णीय‚न्ते तैरित्युक्तं दुर्ज‚न‚{\tiny $_{lb}$}‚विप्र‚तिप‚त्य ‚{\tiny $_{1}$}‚ धिक‚र‚णे स‚तांसा\edtext{}{\lemma{तांसा}\Bfootnote{? शा}}स्त्राप्र‚वृतेरित्य‚त्र । य‚त‚श्च प‚रानुप‚ताप‚यितुं ‚{\tiny $_{lb}$}‚न स‚न्तः प्र‚व‚र्त‚न्ते त‚स्मात्ताव‚द् व‚क्त‚व्यं याव‚द‚नेन न गृहीतं न त्रिरेव व‚क्त‚व्य‚मित्य‚धि‚{\tiny $_{lb}$}‚कृतं । अथ ‚{\tiny $_{2}$}‚ वादिना श‚त‚धापुनः पुन र‚भिधाने प्र‚तिवाद्य‚तिज‚ड‚त्वाद् गृहीतुं न श‚क्नो‚{\tiny $_{lb}$}‚तीति निश्चित‚न्त‚दाऽग्र‚ह‚ण‚साम‚र्थ्ये क‚थ‚ञ्चिन्निश्चिते साध‚न‚प्र‚योगात्प्रागेव ‚{\tiny $_{3}$}‚ प‚रि‚{\tiny $_{lb}$}‚ह‚र्त‚व्यो नानेन स‚होद्ग्राह‚यामीति प‚रिच्छिंन\edtext{}{\lemma{रिच्छिंन}\Bfootnote{? च्छिन्न}}म‚साम‚र्थं\edtext{}{\lemma{र्थं}\Bfootnote{? र्थ्यं}}। ग्र‚ह‚णेऽति‚{\tiny $_{lb}$}‚जाड्याप‚र‚नाम‚क‚प्र‚तिवादिस‚म्ब‚न्धियेन वादिना स त‚था । क‚थं ‚{\tiny $_{4}$}‚ त‚था प‚रिह‚र‚न्ना‚{\tiny $_{lb}$}‚श‚क्तः शंक्य‚त इत्याह । प‚राण‚न\edtext{}{\lemma{न}\Bfootnote{? न्}}साक्षिणः प्र‚बोध्य नायं श‚क्तो वाक्यार्थं बोद्धुं ‚{\tiny $_{lb}$}‚व‚स्तु त्वेवं व्य‚व‚स्थित‚मिति ॥ ४ ॥
	{\color{gray}{\rmlatinfont\textsuperscript{§~\theparCount}}}
	\pend% ending standard par
      ‚{\tiny $_{lb}$}‚\textsuperscript{\textenglish{120/s}}

	  
	  \pstart \leavevmode% starting standard par
	अविज्ञा ‚{\tiny $_{5}$}‚ त‚ञ्चाज्ञान \href{http://sarit.indology.info/?cref=ns\%C5\%AB.5.2.17}{न्या० सू० ५।२।१७ } मिति भावे निष्ठाविधानात् ‚{\tiny $_{lb}$}‚साध‚न‚वाक्यार्थाप‚रिज्ञानं निग्र‚ह‚स्थानं । त‚त एव भाष्ये टीकाकृतो विवृण्व‚न्ति ‚{\tiny $_{lb}$}‚वाक्यार्थ‚विष‚य‚स्य विज्ञान‚स्यानुत्प ‚{\tiny $_{6}$}‚ त्तिर‚ज्ञान‚मिति । अस्तु वा क‚र्म‚ण्येव निष्ठाविधानं ‚{\tiny $_{lb}$}‚त‚थापि वाक्यार्थ‚विष‚य‚ज्ञानानुत्प‚त्या विशेषि तं वादिप्र‚युक्तं वाक्य‚मेव प्र‚तिवादिनो ‚{\tiny $_{lb}$}‚निग्र‚ह‚स्थान ‚{\tiny $_{7}$}‚ मिति न किञ्चिद्व्याह‚न्य‚ते । अन्ये पुन‚र्विव‚र‚णेर्थ‚ग्र‚ह‚णं प‚श्य‚न्तः सूत्रे‚{\tiny $_{lb}$}‚प्य‚र्थ‚ग्र‚ह‚णं भ्रान्त्या प‚ठ‚न्ति । अविज्ञातार्थ‚ञ्चाज्ञान‚मिति सोऽन्येषां पाठः । विज्ञा‚{\tiny $_{lb}$}‚तार्थं ‚{\tiny $_{8}$}‚ साध‚न‚वाक्यं प‚रिष‚दा त‚स्य प्र‚तिवादिना य‚द‚विज्ञात‚म‚न‚व‚बोध‚स्त‚द‚ज्ञान‚{\tiny $_{lb}$}‚मित्येवं भाव‚प‚क्षेऽक्ष‚र‚विन्यासः । क‚र्म‚प‚क्षे तु त‚स्येति नाध्याह‚र्त‚व्य‚मेक‚वा ‚{\tiny $_{9}$}‚ \leavevmode\ledsidenote{\textenglish{74b/msK}} क्य‚त‚यैव ‚{\tiny $_{lb}$}‚तु व्याख्यातं । विज्ञातं प‚र्ष‚देति किम‚र्थं प‚र्ष‚दाप्य‚विज्ञाते वादिन एवाविज्ञातार्थ ‚{\tiny $_{lb}$}‚निग्र‚ह‚स्थानं भ‚व‚तीति ज्ञाप‚नाय । निग्र‚ह‚स्थान‚त्वे कार‚ण‚माह । अर्थे ख‚ल्व ‚{\tiny $_{1}$}‚ विज्ञाते ‚{\tiny $_{lb}$}‚प्र‚तिवादी न त‚स्य प्र‚तिषेधं ब्रूयादिति । अप‚रे तूत्त‚रेण दूष‚ण‚ग्र‚न्थेन स‚हैत‚त् स‚म्ब‚ध्न‚न्ति ‚{\tiny $_{lb}$}‚त‚च्चायुक्तं भाष्य‚वार्तिक‚ग्र‚न्थ‚त्वाद‚स्य । ग‚म्य‚त्व‚मेव साध ‚{\tiny $_{2}$}‚ य‚ति य‚थाऽन‚नुभाष‚णेऽनु‚{\tiny $_{lb}$}‚त्त‚र‚प्र‚तिप‚त्यैव निग्र‚ह‚स्थान‚त्वं क‚थ‚मुत्त‚राप्र‚तिप‚त्तिरित्याह । अप्र‚द‚र्शित‚विष‚य‚त्वा‚{\tiny $_{lb}$}‚त्प्र‚तिवादिनोत्त‚र‚प्र‚तिप‚त्तिर‚श‚क्येति ‚{\tiny $_{3}$}‚ कृत्वाऽप्र‚द‚र्शितो विष‚यो येनेति विज्ञेयं विशे‚{\tiny $_{lb}$}‚ष‚ण‚स‚मासो वा । त‚थाहि दूष‚ण‚स्य विष‚य‚ख्याप‚नार्थ‚म‚नुभाष‚ते त‚ञ्च प‚रित्य‚ज्य ‚{\tiny $_{lb}$}‚य‚द्य‚देव वा ‚{\tiny $_{4}$}‚ दिनाऽनुद्ग्राहित‚माल‚जाल‚म‚नुभाष‚ते । त‚दानीमुत्त‚र‚विष‚य‚प्र‚द‚र्श‚न‚प्र‚स‚ङ्ग‚{\tiny $_{lb}$}‚म‚न्त‚रेण त‚थाभूत‚स्यानुभाष‚ण‚स्य वैय‚र्थ्याद‚श‚क्येतिव‚र्त‚ते । अ ‚{\tiny $_{5}$}‚ नुग्र‚ह‚प्र‚तिप‚त्यैव निग्र‚ह‚{\tiny $_{lb}$}‚स्थान‚त्व‚मिति वा । दार्ष्टान्तिक‚मुप‚संह‚र‚ति । त‚था ज्ञानेऽप्युत‚राप्र‚तिप‚त्यैव निग्र‚ह‚{\tiny $_{lb}$}‚स्थान‚त्व‚मिति । य‚स्माद‚जान‚न् प्र‚ति ‚{\tiny $_{6}$}‚ वादिदूष‚ण‚त‚द्विष‚यौ क‚थ‚मुत्त‚र‚विष‚य‚ञ्च ‚{\tiny $_{lb}$}‚ब्रूयात् । उत्त‚र‚विष‚यो दूष्यः । क्व‚चित्तु पाठः । क‚थ‚मुत्त‚र‚मुत्त‚र‚विष‚य‚ञ्चोत‚र‚मिति । ‚{\tiny $_{lb}$}‚अत्रैवं य‚द‚स‚म्ब ‚{\tiny $_{7}$}‚ न्धः । अजान‚न्नुत्त‚र‚विष‚य‚ञ्च क‚थ‚मुत्त‚र‚म्ब्रूयादिति । त‚स्माद्विष‚या‚{\tiny $_{lb}$}‚ज्ञान‚मुत‚राज्ञान‚ञ्च निग्र‚ह‚स्थान‚म‚प्र‚तिभ‚यैव ग‚म्य‚त्वात् । अवाच्य‚मिति व‚र्त‚ते किं ‚{\tiny $_{8}$}‚ ‚{\tiny $_{lb}$}‚कार‚ण‚म‚न्य‚थैव‚म‚निष्प्र‚माणे स‚त्य‚प्र‚तिभाया निर्विष‚य‚त्वात् । क‚थं निर्व्विष‚य‚त्व‚मित्याह । ‚{\tiny $_{lb}$}‚ \leavevmode\ledsidenote{\textenglish{121/s}} \quotelemma{अन‚व‚धारितार्थो हीत्यादि} । अन‚व‚धारि ‚{\tiny $_{9}$}‚ \leavevmode\ledsidenote{\textenglish{75a/msK}} तोर्थ‚पूर्व‚प‚क्ष‚स्योत्त‚र‚स्य च येन प्र‚तिवादिना ‚{\tiny $_{lb}$}‚स नानु ‚{\tiny $_{1}$}‚ भाषेत् । अन‚नुभाष्य‚माण‚श्चासौ विष‚य‚म‚प्र‚द‚र्श्योत्त‚रं प्र‚तिप‚त्तुन्न श‚क्नुयादित्य‚त ‚{\tiny $_{lb}$}‚उत‚र‚न्न प्र‚तिप‚द्येत । न जानीयान्नाभिद‚ध्याद्वा । क‚स्मादुत्त‚र‚विष‚य‚योर‚ज्ञाने स‚त्युत्त‚रा‚{\tiny $_{lb}$}‚प्र‚तिप‚त्तिरित्य‚त आह \quotelemma{ज्ञानोत्त‚र‚त‚द्विष‚य‚स्योत्त‚राप्र‚तिप‚त्तेर‚संभ‚वादिति} । ज्ञाता ‚{\tiny $_{lb}$}‚उत्त ‚{\tiny $_{2}$}‚ र‚त‚द्विष‚यो येनेति वृत्तः । त‚स्मादुभ‚य‚मेत‚दुत्त‚राज्ञानं । विष‚याज्ञान‚ञ्च प्र‚ति‚{\tiny $_{lb}$}‚प‚त्तेर‚प्र‚तिभाप‚र‚प‚र्य्यायायाः कार‚णं । न‚नु चोत्त‚राज्ञान‚मेवाप्र‚तिभा त ‚{\tiny $_{3}$}‚ त्क‚थं सैवा‚{\tiny $_{lb}$}‚त्म‚नः कार‚ण‚त्वेनोप‚दिश्य‚ते । नोत्त‚रान‚भिधान‚ल‚क्ष‚णाया अप्र‚तिभासाया विव‚क्षित‚{\tiny $_{lb}$}‚त्वात् । त‚द‚भाव इति त‚योरुत्त‚र‚विष‚याज्ञान‚योर‚भा ‚{\tiny $_{4}$}‚ वे स‚ति प्र‚तिप‚त्तिर‚भिधान‚{\tiny $_{lb}$}‚मुत्त‚र‚स्य भ‚व‚त्येव । इति त‚स्मात्त‚योर्विष‚याज्ञानोत्त‚र‚ज्ञान‚योर‚ज्ञान‚संज्ञितेन निग्र‚ह‚{\tiny $_{lb}$}‚स्थानेनाप्र‚तिभा निग्र‚ह‚स्थाना ‚{\tiny $_{5}$}‚ त् पृथ‚क्व‚च‚ने स‚त्य‚प्र‚तिभायाः को विष‚य इति ‚{\tiny $_{lb}$}‚व‚क्त‚व्यं । न चेद्विष‚यो भ‚ण्य‚ते । त‚दा निर्विष‚य‚त्वाद‚वाच्यैव स्याद‚प्र‚तिभा । त‚योर‚{\tiny $_{lb}$}‚ज्ञानानुभाष‚ण‚यो ‚{\tiny $_{6}$}‚ ः पृथ‚ग्व‚च‚न इत्य‚न्ये व्याच‚क्ष‚ते । अज्ञानाप्र‚तिभ‚योर्विष‚य‚भेद‚व्य‚व‚{\tiny $_{lb}$}‚स्थाप‚नाय प‚रः प्राह । नोत्त‚र‚ज्ञान‚म‚ज्ञान‚मुच्य‚ते । य‚तोऽप्र‚तिभा निर्विष‚य‚त्वाद‚वा ‚{\tiny $_{7}$}‚ ‚{\tiny $_{lb}$}‚च्य‚म्भ‚वेत् । किन्त‚र्ह्य‚ज्ञान‚मित्याह । विष‚याज्ञानं । एव‚म‚पि क‚थ‚म‚प्र‚तिभा विष‚य‚{\tiny $_{lb}$}‚व‚ती भ‚व‚तीत्याह । \quotelemma{ज्ञाते विष‚ये} स‚त्युत्त‚र‚काल‚मुत्त‚राज्ञानात् । प्र‚तिवादी त ‚{\tiny $_{8}$}‚ दुत‚र‚न्न ‚{\tiny $_{lb}$}‚प्र‚तिप‚द्येत न ब्रूयात् । अतोऽस्ति विष‚योऽप्र‚तिभायाः । अज्ञानाक्रान्तः । एव‚म‚प्य‚{\tiny $_{lb}$}‚वाच्य ‚{\tiny $_{9}$}‚ \leavevmode\ledsidenote{\textenglish{75b/msK}} त्वान्न‚मुच्य‚स इत्याह । \quotelemma{एव‚न्त‚र्हीति} । अज्ञानेनानुभाष‚ण‚स्याक्षेप‚मेव साध‚य‚ति । ‚{\tiny $_{lb}$}‚न‚हि विष‚यं स‚म्य‚क् प्र‚तिप‚द्य‚मानः क‚श्चित् स‚चेत‚नो नानुभाषेतेति नानुभाष‚ण‚म ‚{\tiny $_{1}$}‚ ‚{\tiny $_{lb}$}‚ज्ञानात्पृथ‚ग्वाच्यं । अपिचैव‚म‚प्र‚तिभाप्य‚न‚नुभाष‚ण‚व‚द‚ज्ञानात्पृथ‚ग्न‚वाच्येत्याह । \quotelemma{उत्त‚{\tiny $_{lb}$}‚राज्ञान‚स्य चाक्षेपादिति} । इदं व्याच‚ष्टे \quotelemma{विष‚ये} त्यादिना । ज्ञाते विष‚य ‚{\tiny $_{2}$}‚ इत्यादि प‚रः । ‚{\tiny $_{lb}$}‚  \leavevmode\ledsidenote{\textenglish{122/s}} इद‚मुक्त‚म्भ‚व‚ति द्विधोत्त‚राज्ञान‚विष‚याज्ञान‚स‚ह‚च‚र‚ञ्च विष‚य‚ज्ञान‚स‚ह‚च‚र‚ञ्च । त‚त्रा‚{\tiny $_{lb}$}‚द्य‚स्य विष‚याज्ञाने नाक्षेपेऽप्युत्त‚र‚म‚नाक्षिप्त‚मेव ‚{\tiny $_{3}$}‚ त‚तो द्वितीयापेक्ष‚याऽप्र‚तिभायाः ‚{\tiny $_{lb}$}‚पृथ‚गुपादान‚मिति । अन‚व‚स्थैव निग्र‚ह‚स्थानानां प्र‚स‚ज्य‚त इत्याह । \quotelemma{एव‚न्त‚र्ही} ‚{\tiny $_{lb}$}‚त्यादि । य‚थेत्याद्य‚स्यैव वि ‚{\tiny $_{4}$}‚ भागः । त‚था ज्ञान‚योर‚पीति विष‚योत्त‚राज्ञान‚योर‚पि । ‚{\tiny $_{lb}$}‚स‚र्व्व‚स्योत्त‚र‚स्य विष‚य‚स्य चाज्ञानं । आदिग्र‚ह‚णेन द्वित्रिच‚तुर्भागाद्य‚व‚रोधः । वि ‚{\tiny $_{5}$}‚ ष‚{\tiny $_{lb}$}‚योत्त‚राज्ञान‚योः स‚ङ्ग्र‚ह‚व‚च‚ने दोष इति चेदाह । न चेति । य‚था न दोष‚स्त‚थागुणोपि ‚{\tiny $_{lb}$}‚नास्तीति चेदाह । गु \quotelemma{ण‚श्च लाघ‚व‚संज्ञः} स्यादिति सं ‚{\tiny $_{5}$}‚ ग्र‚ह‚व‚च‚नं न्याय्यं । अप्र‚तिभा‚{\tiny $_{lb}$}‚विष‚य‚त्वान्न पृथ‚ग्व‚च‚नं । अप्र‚तिभाव‚च‚नेनैवान‚योः स‚ङ्ग्र‚ह इत्य‚र्थः । न केव‚ल‚म‚न‚{\tiny $_{lb}$}‚योरेवापृथ‚ग्व‚च‚नं । न्याय्य‚म‚पि त्व‚न्ये ‚{\tiny $_{7}$}‚ षाम‚पीत्याह । \quotelemma{अपि चेत्यादि} । त‚दुभ‚य‚व‚च‚{\tiny $_{lb}$}‚नेनैवेति । हेत्वाभासाऽप्र‚तिभ‚योरेव व‚च‚नेन स‚र्व‚प्र‚तिज्ञाहान्य‚न‚नुभाष‚णाद्युक्तं । न‚हि ‚{\tiny $_{lb}$}‚क‚श्चिद्ध्य‚न्य‚स्साध‚न‚वादी ‚{\tiny $_{8}$}‚ प्र‚तिप‚क्ष‚ध‚र्म‚म‚भ्युनुजानाति प्र‚तिज्ञाम्वा प्र‚तिज्ञासाध‚{\tiny $_{lb}$}‚नायोपाद‚त्त इत्यादि वाच्यं । त‚दा न क‚ञ्चि[द्]दूष‚णं व्य‚क्त‚मेव य‚न्नानुभाष‚ते । ‚{\tiny $_{lb}$}‚क‚थां विक्षिप‚ति । प‚र‚म ‚{\tiny $_{9}$}‚ \leavevmode\ledsidenote{\textenglish{76a/msK}} त‚ञ्चानुजानातीदि\edtext{}{\lemma{ञ्चानुजानातीदि}\Bfootnote{? ति}}व‚क्त‚व्यं । त‚दुभ‚याक्षेपेपि प्र‚प‚ञ्चो ‚{\tiny $_{lb}$}‚गुण‚वान‚त‚स्त‚द‚व‚च‚नाद‚रोमुनेरिति चेदाह । त‚दुभ‚याक्षिप्तेषु प्र‚भेदेषु गुणातिश‚य‚म ‚{\tiny $_{1}$}‚ ‚{\tiny $_{lb}$}‚न्त‚रेण । अनुप‚ल‚भ्य‚मान‚त्वाद् गुण‚स्य प्र‚प‚ञ्च‚व‚च‚नाद‚रेऽतिप्र‚स‚ङ्गात् । क‚क्ष‚पिट्टिता‚{\tiny $_{lb}$}‚ \leavevmode\ledsidenote{\textenglish{123/s}} दीनाम‚भिधान‚प्र‚स‚ङ्गात् । अतो व्य‚र्थः प्र‚प‚ञ्चो म‚हामुनिनाक्रिय‚त ‚{\tiny $_{2}$}‚ ॥ ० ॥
	{\color{gray}{\rmlatinfont\textsuperscript{§~\theparCount}}}
	\pend% ending standard par
      ‚{\tiny $_{lb}$}‚

	  
	  \pstart \leavevmode% starting standard par
	प‚र‚प‚क्ष‚प्र‚तिषेधे क‚र्त‚व्ये उत्त‚रं दूष‚णं य‚दा न प्र‚तिप‚द्य‚ते न वेत्ति नाभिद‚धाति ‚{\tiny $_{lb}$}‚त‚दा निगृहीतो वेदित‚व्य इतीयान् प‚र‚ग्र‚न्थः । साध्वेत‚न्निग्र‚ह‚स्थानं । अतएवास्मा‚{\tiny $_{lb}$}‚भिर‚पीद‚म‚दोषोद्भाव‚न‚मित्य‚त्रोक्त‚मित्येत‚त् म‚त्वाऽभ्य‚नुजानाति । \quotelemma{साध‚नेत्यादि} । ‚{\tiny $_{lb}$}‚साध‚न व‚च‚नान‚न्त‚रं प्र‚तिवादिना दूष‚ण‚म्व‚क्त‚व्यं । स य ‚{\tiny $_{4}$}‚ दा स‚र्व्वं त‚द‚कृत्वा स‚र्व्वा‚{\tiny $_{lb}$}‚नुक्र‚मानुभाष‚णेन श्लोक‚पाठेन स‚भास‚म्व‚र्ण्ण‚नेनान्येन काल‚न्न‚य‚ति त‚दासौ व्य‚र्थं ‚{\tiny $_{lb}$}‚निष्प्र‚योज‚र्न काल‚ङ्ग‚म‚य‚न्क‚र्त्त‚व्य‚स्य दूष‚णाभि ‚{\tiny $_{5}$}‚ धान‚स्य प्र‚तिप‚त्त्याऽन‚नुष्ठानेन निगृ‚{\tiny $_{lb}$}‚ह्य‚ते । व्य‚र्थ‚स्येदं क्रियायाः काल‚स्य विशेष‚णं ॥ ० ॥
	{\color{gray}{\rmlatinfont\textsuperscript{§~\theparCount}}}
	\pend% ending standard par
      ‚{\tiny $_{lb}$}‚

	  
	  \pstart \leavevmode% starting standard par
	कार्य‚व्यास‚ङ्गः क‚र्णीयोप‚न्यासः क‚थाविच्छेदः क ‚{\tiny $_{6}$}‚ थानिवृत्तिः । य‚था जीर्ण्ण‚{\tiny $_{lb}$}‚क‚ला मे बाध‚ते । स‚म्प्र‚ति व‚क्तुँ न श‚क्नोमि प‚श्चात् क‚थ‚यिष्यामीति एव‚मादिना ‚{\tiny $_{lb}$}‚प्र‚कारेण क‚थामुद्ग्राह‚णे काचिच्छिन‚त्तिः । निग्र‚ह‚स्थाने ‚{\tiny $_{7}$}‚ कार‚ण‚माह । एक‚त‚र‚स्य ‚{\tiny $_{lb}$}‚वादिनः प्र‚तिवादिनो वाऽसाध‚नाङ्ग‚व‚च‚नेनादोषोद्भाव‚नेन च निगृह‚ण‚न्ती निग्र‚ह‚{\tiny $_{lb}$}‚प‚र्य‚व‚साना क‚था । त‚स्याञ्च त‚थाभूतायां प्र‚स्तुता ‚{\tiny $_{8}$}‚ यां स स्व‚य‚मेव क‚थांतं क‚था ‚{\tiny $_{lb}$}‚प‚र्य‚व‚सानं प्र‚तिप‚द्य‚त इति निग्र‚ह‚स्थान‚मेत‚त् । अत्राचार्योब्रूत [।] इद‚म‚पि कार्य‚व्यास‚{\tiny $_{lb}$}‚ञ्ज‚नं य‚दि ताव‚त् पूर्व‚प‚क्ष‚वादी कुर्यात् ‚{\tiny $_{9}$}‚ \leavevmode\ledsidenote{\textenglish{76b/msK}} साध‚नाभिधान‚श‚क्तिविक‚ल‚त्वाद् व्याजो- ‚{\tiny $_{lb}$}‚प‚क्षेप‚मात्रेण येन केन‚चिच्छ‚लेनेत्य‚र्थः । न पुन‚र्भूत‚स्य त‚थाविध‚क‚थोप‚रोधिनः ‚{\tiny $_{lb}$}‚कार्य‚स्य भावे स‚ति कुर्यादिति व‚र्त्त‚ते । त‚था वि ‚{\tiny $_{1}$}‚ धामुद्ग्राह‚णिकारूपाङ्क‚थामुप‚रोद्धुं ‚{\tiny $_{lb}$}‚शीलं य‚स्य कार्य‚स्याजीर्ण्ण‚क‚ला कुक्षिशूल‚गेहु \edtext{}{\lemma{गेहु}\Bfootnote{?}} दाहाद्यैस्त‚त्त‚था । य‚दि स‚द्भावे‚{\tiny $_{lb}$}‚नैव त‚स्य त‚स्याम्वेलायां कुक्षिग‚ल‚शूल‚गेहु \edtext{}{\lemma{गेहु}\Bfootnote{?}} दा ‚{\tiny $_{2}$}‚ हाद‚यो भ‚व‚न्ति त‚था स‚ति नैव ‚{\tiny $_{lb}$}‚निग्र‚ह‚स्थान‚मित्य‚र्थः । य‚दा पुन‚र्व्याज‚मात्रेणैव क‚रोति त‚दा त‚स्य पूर्व्व‚प‚क्ष‚वादिनः ‚{\tiny $_{lb}$}‚ \leavevmode\ledsidenote{\textenglish{124/s}} स्व‚साध‚नासाम‚र्थ्य‚प‚रिच्छेदादेव विक्षेप ‚{\tiny $_{3}$}‚ ः स्यात्त‚तः किमित्याह । त‚था चेदं विक्षेप‚{\tiny $_{lb}$}‚स‚ञ्ज्ञित‚न्निग्र‚ह‚स्थान‚म‚र्थान्त‚र एवान्त‚र्भ‚वेत् । रूप‚सिद्धिनामादिव्याख्यान‚स‚मान‚{\tiny $_{lb}$}‚त्त्वात् क‚र‚णीयोप‚न्यास ‚{\tiny $_{4}$}‚ स्य । हेत्वाभासेष्वेवान्त‚र्भ‚वेदित्य‚धिकृतं । क‚स्माद‚स‚म‚र्थ‚ञ्च ‚{\tiny $_{lb}$}‚त‚त्साध‚न‚ञ्च त‚स्याभिधानात् । किञ्चेदं निर‚र्थ‚कापार्थ‚काभ्यां स‚काशान्न भिद्य‚ते । ‚{\tiny $_{lb}$}‚किं ‚{\tiny $_{5}$}‚ कार‚णं प्र‚कृत‚ञ्च त‚त्साध‚न‚ञ्च तेनास‚म्ब‚द्धा च सा प्र‚तिप‚त्तिश्च त‚तः साध‚न‚{\tiny $_{lb}$}‚वाक्येन स‚हास्य द‚श‚दाडिमादिव‚च‚न‚स्येव ज‚ब‚ग‚डादिव‚र्ण्ण‚क्र‚म‚स्येव च ‚{\tiny $_{6}$}‚ स‚म्ब‚द्धानुप‚{\tiny $_{lb}$}‚ल‚म्भादित्य‚र्थः । किञ्चिन्मात्र‚भेदान्निमित्त‚लेशेन पृथ‚गुक्त‚मिति चेदाह । \quotelemma{अति ‚{\tiny $_{lb}$}‚प्र‚स‚ङ्ग‚श्चे} त्यादि । अस‚म्ब‚द्धासाध‚न‚वाक्येन प्र‚तिप‚त्तिर्येषां ‚{\tiny $_{7}$}‚ प्र‚तिभेदानान्तेऽस‚म्ब‚द्ध‚{\tiny $_{lb}$}‚साध‚न‚वाक्य‚प्र‚तिप‚त्त‚यः । ते च ते प्र‚भेदाश्च तेषामिति कार्यं एव‚न्ताव‚त् पूर्व‚प‚क्ष‚वादि‚{\tiny $_{lb}$}‚स‚म्ब‚न्धेन विक्षेप‚स्य पृथ‚ग‚न‚भिधान ‚{\tiny $_{8}$}‚ मुक्तं ॥ अधुना प्र‚तिवादिस‚म्ब‚न्धेनाप्याह । ‚{\tiny $_{lb}$}‚ \quotelemma{अथोत्त‚र‚प‚क्ष} वाद्येवं ब‚लास‚क‚लात्म‚क‚ण्ठं क्षिणोतीत्यादिना प्र‚क्र‚मेण क‚थां विक्षिपेत् ‚{\tiny $_{lb}$}‚त‚दानीन्त‚स्याप्यु ‚{\tiny $_{9}$}‚ \leavevmode\ledsidenote{\textenglish{77a/msK}} त्त‚र‚प‚क्ष‚वादिनः साध‚नान‚न्त‚र‚मुत्त‚रे प्र‚तिप‚त्त‚व्ये स‚ति त‚द‚प्र‚तिप‚त्त्या ‚{\tiny $_{lb}$}‚त‚स्योत्त‚र‚स्यान‚भिधानेन विक्षेप‚प्र‚तिप‚त्तिर्यासाऽप्र‚तिभायाम‚र्थान्त‚रे वान्त‚र्भ‚व‚ती ‚{\tiny $_{1}$}‚ ति ‚{\tiny $_{lb}$}‚प‚र‚स्तु य‚थोक्त‚म‚न्त‚र्भाव‚म‚स‚ह‚मान‚श्चोद‚य‚ति ।
	{\color{gray}{\rmlatinfont\textsuperscript{§~\theparCount}}}
	\pend% ending standard par
      ‚{\tiny $_{lb}$}‚

	  
	  \pstart \leavevmode% starting standard par
	\hphantom{.}\quotelemma{न‚नु नाव‚श्य} मिति त‚देव द्र‚ढ‚य‚ति [।] भ‚व‚ति ह्य‚निब‚द्धेनापि साध‚न‚वाक्येनास‚{\tiny $_{lb}$}‚म्ब‚द्धे नापि क‚थाप्र‚ब‚न्धेन ‚{\tiny $_{2}$}‚ प‚र‚प्र‚तिभाह‚र‚णायान्त‚शो ज‚न‚नीव्य‚भिचार‚चोद‚नेनापि ‚{\tiny $_{lb}$}‚विवाद इति । आचार्य आह । नास‚म्भ‚वादेवंविध‚स्य विवाद‚स्य । य‚स्मादेक‚त्र ‚{\tiny $_{lb}$}‚श‚ब्दादाव‚धिक‚र ‚{\tiny $_{3}$}‚ णे नित्य‚त्वानित्य‚त्वादिप्र‚तिज्ञानाविरुद्धाव‚भ्युप‚ग‚मौ य‚योर्वादिप्र‚ति ‚{\tiny $_{lb}$}‚वादिनोस्त‚योर्विवादः स्यात् । कुत एत‚दित्याह । अविरुद्धाव‚भ्युप‚ग ‚{\tiny $_{4}$}‚ मौ य‚योस्तौ ‚{\tiny $_{lb}$}‚ \leavevmode\ledsidenote{\textenglish{125/s}} त‚था न विद्येते विरुद्धाविरुद्ध‚योर‚भ्युप‚ग‚मौ य‚योः पुरुष‚योस्ताव‚भ्युप‚ग‚मौ । त‚यो‚{\tiny $_{lb}$}‚र्विवादाभावात् । त‚त्रैत‚स्मिन्व्य‚व‚स्थिते न्याय ‚{\tiny $_{5}$}‚ निर्धार‚णे वा त‚त्र श‚ब्दः । ‚{\tiny $_{lb}$}‚एक‚स्य वादिनः प्र‚तिवादिनोव‚श्यं प्राग्व‚च‚न‚प्र‚वृत्तिः । यौग‚प‚द्येन किन्न ब्रूत ‚{\tiny $_{lb}$}‚इत्याह । \quotelemma{युग‚प‚त्प्र‚वृत्तौ} स्व‚स्थात्म‚ना ‚{\tiny $_{6}$}‚ म‚प्र‚वृत्तेरिति स‚म्ब‚न्धः । एत‚देव कुत ‚{\tiny $_{lb}$}‚इत्याह । प‚र‚स्प‚र व‚च‚न श्र‚व‚णाव‚धार‚णोत्त‚राणाम‚स‚म्भ‚वेन क‚र‚ण‚भूतेन प्र‚वृत्ति‚{\tiny $_{lb}$}‚वैफ‚ल्यात् । य‚दि हि प‚र‚स्प‚र‚व‚च ‚{\tiny $_{7}$}‚ न‚स्यास‚ङ्क‚रेण श्र‚व‚ण‚म्भ‚वेत्त‚त‚स्त‚द‚र्थ‚म‚व‚धार‚{\tiny $_{lb}$}‚य‚त्युत्त‚र‚ञ्च । युग‚प‚त्प्र‚वृत्तौ च दिग‚म्ब‚र‚पाठ‚क‚ल‚क‚ल इव स‚र्व्व‚मेत‚न्न संभ‚व‚ति त‚स्मा‚{\tiny $_{lb}$}‚द‚व‚श्य‚मेक‚स्य प्राग्व‚च‚न ‚{\tiny $_{8}$}‚ प्र‚वृत्तिः । अत‚स्तेन च स्व‚स्थात्म‚ना पूर्व्व‚प‚क्ष‚वादिनाऽनित्यं ‚{\tiny $_{lb}$}‚श‚ब्दं साध‚यामीत्यादिना स्वोप‚ग‚मोप‚न्यासे कृते स‚त्य‚व‚श्यं साध‚नं व‚क्त‚व्यं । अन्य‚थेति ‚{\tiny $_{lb}$}‚हेत्व‚न‚भिधा ‚{\tiny $_{9}$}‚ \leavevmode\ledsidenote{\textenglish{77b/msK}} ने प‚रेसां\edtext{}{\lemma{रेसां}\Bfootnote{? षां}}साक्षिप्र‚भृतीनाम‚प्र‚तिप‚त्तेः । अप‚रेण चेत्युत्त‚र‚प‚क्ष- ‚{\tiny $_{lb}$}‚वादिना त‚त्स‚म्ब‚न्धिवादिप्रोक्त‚साध‚न‚स‚म्ब‚न्धि दूष‚णं व‚क्त‚व्य‚मिति व‚र्त्त‚ते [।] ‚{\tiny $_{lb}$}‚त‚स्मादुभ‚योर्वादिप्र ‚{\tiny $_{1}$}‚ [ति] वादिनोर‚स‚म्य‚क् प्र‚वृत्तौ स‚त्यां हेत्वाभासाप्र‚तिभ‚योः ‚{\tiny $_{lb}$}‚संग्र‚ह इति स‚र्व्वो न्याय‚प्र‚वृत्तः पूर्व्वोत्त‚र‚प‚क्षोप‚न्यासो द्व‚यं हेत्वाभासाप्र‚तिभाञ्च ‚{\tiny $_{lb}$}‚नातिप‚त‚ति ।
	{\color{gray}{\rmlatinfont\textsuperscript{§~\theparCount}}}
	\pend% ending standard par
      ‚{\tiny $_{lb}$}‚

	  
	  \pstart \leavevmode% starting standard par
	न‚नु च य‚दि न्यायः प्र‚वृत्तः क‚थ‚न्त‚त्रास्य द्व‚य‚स्याधिकारः । क‚थ‚ञ्चैक‚त्र ‚{\tiny $_{lb}$}‚ध‚र्मिणि विरुद्धावुप‚न्यासौ न्याय‚प्र‚वृत्ताव‚व‚श्यं हि त‚त्रैकेनोप‚न्यासेन न्यायं प्र‚वृत्तेन ‚{\tiny $_{lb}$}‚भाव्यं । अन्य ‚{\tiny $_{3}$}‚ था ध‚र्मीद्व्यात्म‚को भ‚वेत् । नाभिप्रायाप‚रिज्ञानात् । नेद‚म्भ‚व‚ता ‚{\tiny $_{lb}$}‚न्याय‚प्र‚वृत्त‚त्व‚माचार्येण विव‚क्षितं प‚र्य‚ज्ञायि । न्याय‚प्र‚वृत्तौ\edtext{}{\lemma{वृत्तौ}\Bfootnote{?}} हि पूर्व्वोत्त‚र‚प‚क्षोप ‚{\tiny $_{4}$}‚ ‚{\tiny $_{lb}$}‚न्यास‚स्य युग‚प‚त्प्र‚वृत्य‚भावेन ज‚न‚नी व्य‚भिचार‚वेद‚नाद्य‚भावेन चाभिप्रेतं । एतेनैक‚त्र ‚{\tiny $_{lb}$}‚ह्य‚धिक‚र‚णे विरुद्धाभ्युप‚ग‚म‚योर्विवादः स्यादित्यादि ‚{\tiny $_{5}$}‚ ना वित‚ण्डा प्र‚त्युक्ता । क‚थं ‚{\tiny $_{lb}$}‚प्र‚त्युक्तेत्याह । \quotelemma{अभ्युप‚ग‚माभावे विवादाभावात्} । इद‚मुक्त‚म्भ‚व‚ति । स्व‚प‚क्ष‚स्थाप‚ना ‚{\tiny $_{lb}$}‚हीनो वाक्य‚स‚मूहो वित‚ण्डे त्युच्य ‚{\tiny $_{6}$}‚ ते \href{http://sarit.indology.info/?cref=ns\%C5\%AB.1.2.3}{न्या० सू० १।२।३ } । य‚दि च‚वैत‚ण्डिक्र‚स्य ‚{\tiny $_{lb}$}‚ \leavevmode\ledsidenote{\textenglish{126/s}} स्व‚प‚क्षो नास्ति विवाद‚स्त‚र्हि क‚थ‚मिति व‚क्त‚व्यं । प‚र‚प‚क्ष‚प्र‚तिषेधार्थ‚म्वैत‚ण्डिकः ‚{\tiny $_{lb}$}‚प्र‚व‚र्त्त‚त इति चेत् । प‚र‚प‚क्ष‚प्र‚तिषेध एव त‚र्ह्य‚स्य ‚{\tiny $_{7}$}‚ स्व‚प‚क्ष‚स्थाप‚नेति वित‚ण्डाल‚क्ष‚णं ‚{\tiny $_{lb}$}‚विशीर्य‚ते । त‚था हि यो येनाभ्युप‚ग‚तः स त‚स्य स्व‚प‚क्षः । प‚र‚प‚क्ष‚प्र‚तिषेध‚श्च तेनाभ्युप‚{\tiny $_{lb}$}‚ग‚तः स्व‚प‚क्ष‚तां नातिव‚र्त‚त इति । ‚{\tiny $_{8}$}‚ य‚दा त‚र्ह्युप‚ग‚म्य वादं प्र‚तिज्ञामात्रेण विफ‚ल‚त‚या ‚{\tiny $_{lb}$}‚प‚रिष‚च्छार‚द्येन व्याकुलीकृत‚त्त्वादित्य‚र्थः । न किञ्चित् साध‚नं त‚दायासं वा व‚क्ति । ‚{\tiny $_{lb}$}‚अन्य‚द्वा किञ्चित् प्र‚ल‚प ‚{\tiny $_{9}$}‚ \leavevmode\ledsidenote{\textenglish{78a/msK}} ति । साध‚न‚त‚दाभास‚व्य‚तिरिक्तं काको विरूपं विरौति ‚{\tiny $_{lb}$}‚नून‚म‚यं मे गेहे विप‚दं सूच‚य‚ति त‚द‚ल‚म‚नेन विवादेन । यामि ताव‚द् गेहे किन्नु मे पितु‚{\tiny $_{lb}$}‚र्म‚र‚ण‚म‚न्य‚द्वा व‚र्त्त‚त इत्या ‚{\tiny $_{1}$}‚ दि । त‚था क‚थं हेत्वाभासान्त‚र्भावः । साध‚नाभावाद्धेत्वा‚{\tiny $_{lb}$}‚भासास‚म्भ‚वं म‚न्य‚ते । उत्त‚राप्र‚तिप‚त्तिर‚पि नास्त्येव पूर्व्व‚प‚क्ष‚वादित्वादित्य‚भिप्रायः । ‚{\tiny $_{lb}$}‚त‚द‚नेन द्व‚य‚न्नातिप‚त‚ती ‚{\tiny $_{2}$}‚ त्येत‚द्विघ‚ट‚यितुमिच्छ‚ति प‚रः । आचार्य आह । \quotelemma{अस‚म‚र्थित‚{\tiny $_{lb}$}‚साघ‚नाभिधान एव‚मुक्तं द्व‚यं नातिप‚त‚तीति} प्रोक्त‚साध‚न एत‚दुक्त‚मिति याव‚त् । ‚{\tiny $_{lb}$}‚अप्रोक्ते तु क‚थं प्र‚तिप‚त्त ‚{\tiny $_{3}$}‚ व्य‚मित्याह । अन‚भिधानान्य‚भिधान‚योर‚पि स‚तोः प‚राज‚यः ‚{\tiny $_{lb}$}‚एवेत्युक्तं प्र‚क‚र‚णाव‚तार एव । त‚देव स्म‚र‚य‚ति । अभ्युप‚ग‚म्य‚वाद‚म‚साध‚नाङ्ग‚व‚च‚{\tiny $_{lb}$}‚नादिति । ‚{\tiny $_{4}$}‚ त‚थाहि त‚त्र व्याख्यातं । साध‚नाङ्ग‚स्यानुच्चार‚णं । साध‚नाङ्गाद्वा य‚{\tiny $_{lb}$}‚द‚न्य‚स्याभिधानं त‚त्स‚र्व्व‚म‚साध‚नाङ्ग‚व‚च‚न‚मिति । एतेनेत्य‚न्याभिधानेन प‚राज‚य‚{\tiny $_{lb}$}‚व‚च‚ने ‚{\tiny $_{5}$}‚ नाधिक‚स्य पुन‚रुक्त‚स्य च प्र‚तिज्ञादेर्व‚च‚न‚स्य च निग्र‚ह‚स्थान‚त्वं व्याख्यातं । ‚{\tiny $_{lb}$}‚क‚थ‚मित्याह । \quotelemma{त‚द‚पि हीत्यादि} । अनेनैत‚दाह । य‚द्युक्तियुक्त \quotelemma{म‚क्ष‚पादेन} किञ्चि ‚{\tiny $_{6}$}‚ न्नि‚{\tiny $_{lb}$}‚ग्र‚ह‚स्थान‚मुक्त‚न्त‚द‚स्माभिर‚साध‚नाङ्ग‚व‚च‚न‚प‚देनैव संगृहीत‚मिति य‚द्येवं प्र‚तिज्ञा‚{\tiny $_{lb}$}‚देर्व‚च‚न‚स्य चेति किम‚र्थ‚युक्त‚न्न‚हि प्र‚तिज्ञोप‚न‚य‚निग‚म‚नानां व‚च‚न‚न्नि ‚{\tiny $_{7}$}‚ ग्र‚ह‚स्थान \quotelemma{म‚क्ष‚{\tiny $_{lb}$}‚पादे} नोक्तं । प्र‚त्युत त‚द‚व‚च‚न‚मेव निग्र‚ह‚स्थान‚त‚या । य‚दिष्टं हीन‚म‚न्य‚त‚मेनाप्य‚व‚{\tiny $_{lb}$}‚य‚वेन न्यून \href{http://sarit.indology.info/?cref=ns\%C5\%AB.5.2.12}{न्या० सू० ५।२।१२ } मिति । एवं त‚र्हि दृष्टान्तार्थ‚मेत‚द्य‚था त‚स्याप्र‚तीत ‚{\tiny $_{8}$}‚ ‚{\tiny $_{lb}$}‚प्र‚त्याय‚न‚श‚क्तिविक‚ल‚त्वाद‚साध‚नाङ्ग‚व‚च‚न‚प‚देनाभिधानं । त‚थाधिक‚पुन‚र्व‚च‚न‚यो‚{\tiny $_{lb}$}‚र‚पीति । त‚त एव च द्वितीय‚श्च‚कार इव श‚ब्दार्थे व‚र्त्त‚ते । अन्य‚था पुन‚रु ‚{\tiny $_{9}$}‚ \leavevmode\ledsidenote{\textenglish{78b/msK}} क्त‚स्य चेत्य‚यं ‚{\tiny $_{lb}$}‚बोध्य‚र्थः स्यात् । केचित्तूत्त‚र‚ञ्च‚कार‚न्न प‚ठ‚न्ति । तैः पुन‚रुक्त‚व्याख्यान‚मेव ‚{\tiny $_{lb}$}‚प्र‚तिज्ञादेर्व‚च‚न‚स्य चेत्येत‚द् व्याख्येयं । एव‚म‚पि न युक्त‚म‚क्ष‚पादेनैव‚म्विध‚स्य पुन ‚{\tiny $_{1}$}‚ रुक्त‚{\tiny $_{lb}$}‚स्यानिष्ट‚त्वान्नास्ति दोषः । पूर्व्व‚तुल्य‚ध‚र्म‚त‚याऽस्यापि पुन‚रुक्तेऽन्त‚र्भावित‚त्वात् ॥ ० ॥
	{\color{gray}{\rmlatinfont\textsuperscript{§~\theparCount}}}
	\pend% ending standard par
      ‚{\tiny $_{lb}$}‚\textsuperscript{\textenglish{127/s}}

	  
	  \pstart \leavevmode% starting standard par
	\hphantom{.}स्व‚प‚क्ष‚दोषाभ्युप‚ग‚मात् प‚र‚प‚क्ष‚दोष‚प्र‚स‚ङ्गो म‚तानुज्ञा \href{http://sarit.indology.info/?cref=ns\%C5\%AB.5.2.20}{न्या० सू० ५।२।२० } ‚{\tiny $_{lb}$}‚ \cite[18b3]{vn-msN} दोष‚प‚रि ‚{\tiny $_{2}$}‚ हारे व‚क्त‚व्ये दोष‚स्याप‚रिज्ञानात् प‚र‚म‚त‚म‚नुजानात्य‚तो नि‚{\tiny $_{lb}$}‚गृह्य‚ते । त‚दाह प‚रेण वादिना चोदितं प‚र्य‚नुयुक्तं दोष‚म‚नुवृत्या प‚रिहृत्य भ‚व‚तोप्य‚यं ‚{\tiny $_{lb}$}‚दोष इति ब्र ‚{\tiny $_{3}$}‚ वीति । \quotelemma{य‚था भ‚वांश्चौरः पुरुष‚त्वा} \cite[16b4]{vn-msN} च्छ‚व‚रादि [व‚दि] त्युक्ते ‚{\tiny $_{lb}$}‚वादिना स प्र‚तिवादी तं वादिनं प्र‚तिब्रूयात् । भ‚वान‚पि चौर इति सोपि ‚{\tiny $_{lb}$}‚श‚ब्द‚प्र‚योगादात्म‚न‚श्चौर ‚{\tiny $_{4}$}‚ त्व‚म‚भ्युप‚ग‚म्य प‚र‚प‚क्षे त‚न्दोष‚मास‚ञ्ज‚य‚न्नापाद‚य‚त्य‚प‚रेण ‚{\tiny $_{lb}$}‚वादिना य‚न्म‚तं प्र‚तिवादिन‚श्चौर‚त्त्वं त‚द‚नुजानाति । त‚था हि ते न मुक्त‚संस\edtext{}{\lemma{संस}\Bfootnote{? श}} ‚{\tiny $_{lb}$}‚ य‚न्ताव‚दात्म‚न ‚{\tiny $_{5}$}‚ श्चौर‚त्त्वं प्र‚तिप‚त्तुम‚न्य‚था नापि त‚म‚भिद‚ध्यात् । वादिनि तु ‚{\tiny $_{lb}$}‚त‚द‚स्तिनास्तीति चिन्त्य‚म‚तो म‚तानुज्ञा निग्र‚ह‚स्थानं । इद‚माचार्यो निराक‚रोति । ‚{\tiny $_{lb}$}‚अत्रापि ‚{\tiny $_{6}$}‚ \cite[18b5]{vn-msN} य‚द्य‚य‚म‚भिप्राय उत्त‚र‚वादिनः पुरुष‚त्वाच्चौंरो भ‚वान‚पि स्याद‚ह‚{\tiny $_{lb}$}‚मिव । न च भ‚व‚तात्मैवं चौर‚त्वेनेष्ट‚स्त‚न्नायं पुरुष‚त्वादिति चौर्ये साध्ये हेतुर‚चौ‚{\tiny $_{lb}$}‚रेपि भ‚व ‚{\tiny $_{7}$}‚ ति विप‚क्ष‚भूते वृत्तेर‚नैकान्तिक‚दोष‚दुष्ट‚त्वादिति । त‚द‚स्मिन्प्र‚तिवादिनोऽभि‚{\tiny $_{lb}$}‚प्राये न क‚श्चित्त‚स्य दोषो म‚तानुज्ञाल‚क्ष‚णोऽन्यो वा । क‚स्माद‚न‚भिम‚ते चौर‚त्त्वे न ‚{\tiny $_{lb}$}‚रू ‚{\tiny $_{8}$}‚ पेण त‚स्य वादिन आत्म‚नि विप‚क्ष‚भूते हेतोः स‚त्व‚प्र‚द‚र्श‚नेन प्र‚कारेण दूष‚णात् । ‚{\tiny $_{lb}$}‚विद‚ग्ध‚भ‚ङ्गाव्य‚भिचारोद्भाव‚नादिति याव‚त् । औद्योत‚क‚रं \edtext{\textsuperscript{*}}{\lemma{*}\Bfootnote{\href{http://sarit.indology.info/?cref=nv.5.2.21}{न्याय‚वार्त्तिके ५।२।२१ पृष्ठ ५५९}}}चोद्य‚माश‚ङ्क‚ते ‚{\tiny $_{9}$}‚ \leavevmode\ledsidenote{\textenglish{79a/msK}} ‚{\tiny $_{lb}$}‚प्र‚स‚ङ्ग‚म‚न्त‚रेण भ‚वान‚पि स्यादित्येव‚माञ्ज‚सेनैव मृजुनैव क्र‚मेण किन्न ‚{\tiny $_{lb}$}‚व्य‚भिचारितो हेतुस्त्व‚य्य‚पि अचौरे व‚र्त्त‚ते पुरुष‚त्व‚म‚तोऽनैकान्तिक‚त्व‚मिति । ‚{\tiny $_{lb}$}‚त‚स्माद्य‚त ‚{\tiny $_{1}$}‚ एवासाव‚कौटिल्ये क‚र्त‚व्ये कौटिल्य‚माच‚र‚ति त‚त एव निगृह्य‚त इति । ‚{\tiny $_{lb}$}‚  ‚{\tiny $_{lb}$}‚ \leavevmode\ledsidenote{\textenglish{128/s}} आचार्य आह । य‚त्किञ्चिदेत‚द \cite[18b6]{vn-msN} \quotelemma{औद्योत‚क‚रं} व‚चो य‚स्मात् स‚न्ति ह्येवं ‚{\tiny $_{lb}$}‚प्र‚कारा वैद‚ग्ध्य‚प्र‚व‚र्तिता व्य ‚{\tiny $_{2}$}‚ व‚हारा लोके । त‚था हि मात‚रो भाव‚त्क्यो ब‚न्ध‚क्यः ‚{\tiny $_{lb}$}‚स्त्रीत्वादित‚र‚ब‚न्ध‚कीव‚दित्युक्ताः प‚शुपालाद‚योपि ज‚ड‚ज‚न‚ङ्ग‚मादिज‚न‚साधार‚णं ‚{\tiny $_{lb}$}‚वैद‚ग्ध्य‚म‚नुस‚रं ‚{\tiny $_{3}$}‚ तः प्र‚त्य‚व‚तिष्ठ‚न्ते । ताव‚कीनापि माता त‚था स्यादिति न च तेऽनेन ‚{\tiny $_{lb}$}‚प्र‚कारेण स्व‚स्याः स्व‚स्या मातुर्ब‚न्ध‚कीत्वं प्र‚तिप‚द्य‚न्ते । अपि तु भ‚ङ्ग्या हेतुव्य‚भि ‚{\tiny $_{4}$}‚ चार‚{\tiny $_{lb}$}‚चोद‚न‚या प‚रं प्र‚तिव‚द‚न्ति । त‚स्मादेवं बाल‚हालिकादिलोक‚प्र‚क‚ट‚म‚पि व्य‚व‚हारालोक‚{\tiny $_{lb}$}‚म‚प‚सार‚य‚ता य‚दि प‚र‚मुद्योत‚क‚र‚त्व‚मेवो \quotelemma{द्योत ‚{\tiny $_{5}$}‚ क‚रेण} उद्योतित‚मात्म‚नः । अथोच्य‚ते ‚{\tiny $_{lb}$}‚नैवासौ भंग्या व्य‚भिचार‚माद‚र्श‚य‚त्य‚पि तु त‚स्य साध‚न‚स्य स‚म्य‚क्त्व‚म‚भ्युप‚ग‚म्यैव तेन ‚{\tiny $_{lb}$}‚दोषेण प‚र‚म‚पि ‚{\tiny $_{6}$}‚ क‚ल‚ङ्क‚य‚तीत्य‚त आह । \quotelemma{अथ त‚दुप‚क्षेप[ः] पुरुष‚त्वाद् भ‚वांश्चौर} इत्ये‚{\tiny $_{lb}$}‚न‚म‚भ्युप‚ग‚च्छ‚त्येव त‚दाप्य‚सौ त‚त्साध‚न उत्त‚राप्र‚तिप‚त्यैव निग्र‚हार्हो नाप‚र‚त्र ‚{\tiny $_{7}$}‚ वादिनि‚{\tiny $_{lb}$}‚स्व‚दोष‚स्य चौर‚त्व‚स्योप‚क्षेपात् । निग्र‚हार्ह इति व‚र्त्त‚ते । इद‚मेवोपोद्ब‚ल‚य‚ति । ‚{\tiny $_{lb}$}‚त‚त्साध‚न‚निर्दोष‚तायां \cite[18b8]{vn-msN} ह्यंगीकृतायामिति शेषः । त‚स्योप‚क्षेप ‚{\tiny $_{8}$}‚ स्याभ्युप‚ग‚म ‚{\tiny $_{lb}$}‚एव यः स एवोत्त‚राप्र‚तिप‚त्तिरिति ताव‚तैवोत्त‚राप्र‚तिप‚त्तिमात्रेणैवाप‚र‚त्र दोष‚प्र‚स‚ञ्ज‚{\tiny $_{lb}$}‚नात् । पूर्व‚साध‚न‚निग्र‚ह‚स्य स‚तः प्र‚तिवादि ‚{\tiny $_{9}$}‚ \leavevmode\ledsidenote{\textenglish{79b/msK}} नः आप‚न्नः प्राप्तो निग्र‚हो येन त‚स्येति ‚{\tiny $_{lb}$}‚चेति विग्र‚हः । प‚र‚दोषोप‚क्षेप‚स्य म‚तानुज्ञाल‚क्ष‚ण‚स्यान‚पेक्ष‚णीय‚त्वात्प‚राजित‚प‚राज‚या‚{\tiny $_{lb}$}‚भावादिति भावः ॥ ० ॥
	{\color{gray}{\rmlatinfont\textsuperscript{§~\theparCount}}}
	\pend% ending standard par
      ‚{\tiny $_{lb}$}‚

	  
	  \pstart \leavevmode% starting standard par
	\hphantom{.}\quotelemma{निग्र ‚{\tiny $_{1}$}‚ ह‚प्राप्त‚स्यानिग्र‚हः प‚र्य‚नुयोज्योपेक्ष‚णं} \cite[18b9]{vn-msN} प‚र्य‚नुयोज्यो नाम निग्र‚ह‚{\tiny $_{lb}$}‚प्राप्त‚स्यो[पे]क्ष‚ण‚न्निग्र‚ह‚प्राप्तोसीत्य‚न‚भिधानं । क[ः] पुन‚रिदं प‚र्य‚नुयोज्योपेक्ष‚णं ‚{\tiny $_{lb}$}‚निग्र‚ह‚स्थानं ‚{\tiny $_{2}$}‚ चोद‚य‚ति । न ताव‚त् प‚र्य‚नुयोज्य इति युक्तं । अस‚म्भ‚वात् । न ह्य‚स्ति ‚{\tiny $_{lb}$}‚स‚म्भ‚वो य‚त् प‚र‚दोष‚प्र‚तिपाद‚नार्थ‚मात्म‚नो दोष‚व‚त्व‚म‚साव‚भ्युपेति । निग्र‚ह‚प्राप्तः स‚न्न ‚{\tiny $_{lb}$}‚ \leavevmode\ledsidenote{\textenglish{129/s}} ‚{\tiny $_{3}$}‚ ह‚म‚नेनोपेक्षितो निग्र‚ह‚स्थान‚स्याप‚रिज्ञानात् । त‚स्माद‚य‚न्दोष‚वानिति नाप्युपेक्ष इति ‚{\tiny $_{lb}$}‚युक्तं । य‚स्माद‚सौनि जानात्येवायं निग्र‚ह‚प्राप्त इति । त‚था ‚{\tiny $_{4}$}‚ ह्य‚प‚रिज्ञानादेवासौ नानु‚{\tiny $_{lb}$}‚युंक्ते निग्र‚हं प्राप्तोसीति । प‚रिज्ञाने वा क‚थ‚मुपेक्षेत । उपेक्ष‚णे वा स‚म‚चित्तः क‚थ‚मेवं ‚{\tiny $_{lb}$}‚प्र‚क‚ट‚येद‚यं म‚योपेक्षितः ‚{\tiny $_{5}$}‚ स दोष‚स्त‚तो म‚म प‚र्य‚नुयोज्योपेक्ष‚णं निग्र‚ह‚स्थान‚मिति । ‚{\tiny $_{lb}$}‚न चान्य‚स्तृतीयः क‚श्चिदिहानुष‚ङ्गी त‚त्केनेदं चोद‚नीय‚मित्येत‚त् स‚र्व्व‚माश‚ङ्क्य ‚{\tiny $_{lb}$}‚ \quotelemma{प‚क्षिल ‚{\tiny $_{6}$}‚} स्वामी ब्रूते । एत‚च्च प‚र्य‚नुयोज्योपेक्ष‚णं व‚क्त‚व्य‚ञ्चोद‚नीय‚ङ्क‚स्य प‚राज‚य ‚{\tiny $_{lb}$}‚इत्येवं वादिप्र‚तिवादिभ्यां प्र‚गुणा त‚द‚न्यैर्वा प‚र्य‚नुयुक्त‚या पृष्ट‚या स‚त्या प‚रिष‚दा ‚{\tiny $_{7}$}‚ ‚{\tiny $_{lb}$}‚प्राश्निकैर्व‚क्त‚व्य‚मित्य‚र्थः । च श‚ब्दोऽव‚धार‚णार्थः । एत‚देव\edtext{}{\lemma{देव}\Bfootnote{? एव‚मेव}}अन्यानि निग्र‚ह‚{\tiny $_{lb}$}‚स्थानानि वादिप्र[ति]वादिभ्यामेवोद्भाव्य‚न्ते । एत‚त्पुनः प्राश्निकैरेव । किं पुनः कार‚{\tiny $_{lb}$}‚ ‚{\tiny $_{8}$}‚ \leavevmode\ledsidenote{\textenglish{80a/msK}} णं ताभ्यामेव नोच्य‚त इत्याह । न \quotelemma{ख‚लु निग्र‚ह‚प्राप्तः स्व‚कौपीनं} स्व‚दोषं \quotelemma{विवृणुयात्} ‚{\tiny $_{lb}$}‚ \cite[18b10]{vn-msN} प्र‚काश‚येत् । अत्रापीत्याद्याचार्यः । य‚दि तु न्याय‚श्चिन्त्य‚ते त‚दानैक‚स्यापि ‚{\tiny $_{1}$}‚ ‚{\tiny $_{lb}$}‚ज‚य‚प‚राज‚यौ न्याय्यौ । क‚थं वादिनो ज‚य इत्याह साध‚नाभासेन जिज्ञासित‚स्यार्थ‚स्या‚{\tiny $_{lb}$}‚प्र‚तिपाद‚नात् । अत एव न प्र‚तिवादिनोपि प‚राज‚यो वादिविव‚क्षि ‚{\tiny $_{2}$}‚ तार्थ‚सिद्ध्य‚पेक्ष‚या ‚{\tiny $_{lb}$}‚प्र‚तिवादिनः प‚राज‚य‚व्य‚व‚स्थाप‚नात् । प्र‚तिवादिन‚स्त‚र्हि किं ज‚य इत्याह [।] भूते ‚{\tiny $_{lb}$}‚ \quotelemma{दोषान‚भिधानाच्च} \cite[19a1]{vn-msN} । अतएव च न वादिनः प‚राज‚य‚स्त ‚{\tiny $_{3}$}‚ द्दूष‚णापेक्ष‚या ‚{\tiny $_{lb}$}‚त‚द्व्य‚व‚स्थितेः । अथोत्त‚र‚प‚क्ष‚वाद्य‚नेक‚दोष‚स‚द्भावेपि वादिप्रोक्त‚स्य साध‚न‚स्य ‚{\tiny $_{lb}$}‚क‚ञ्चिद्दोष‚मुद्भाव‚य‚ति क‚ञ्चिन्न । न त‚दासौ ‚{\tiny $_{4}$}‚ निग्र‚ह‚म‚र्ह‚ति । किङ्कार‚ण‚मुत्त‚र‚स्य ‚{\tiny $_{lb}$}‚प्र‚तिप‚त्तेर‚भिधानादित्य‚र्थः । प‚र आह । अर्ह‚त्येव निग्र‚हं स‚र्व्वेषान्दोषाणाम‚नु‚{\tiny $_{lb}$}‚द्भाव‚नात् । आचार्य आह । \quotelemma{न ख‚लु ‚{\tiny $_{5}$}‚ भोः स‚न्त इति कृत्वा स‚र्व्वे दोषा अव‚श्यं ‚{\tiny $_{lb}$}‚व‚क्त‚व्याः प्र‚तिवादिना} । अव‚च‚ने वा दोषान्त‚र‚स्य निग्र‚हो भ‚व‚ति नेति व‚र्त्त‚ते । ‚{\tiny $_{lb}$}‚क‚स्मात् स‚र्व्वे दोषा नोद्भाव्यं ‚{\tiny $_{6}$}‚ त इत्याह । \quotelemma{एकेनापि} \cite[19a2]{vn-msN} दोषेणासिद्ध‚{\tiny $_{lb}$}‚त्वादिनोद्भावितेन न त‚स्य वादिप्र‚युक्त‚स्य साध‚न‚स्य विघातात् । साध्य‚सिद्धिं ‚{\tiny $_{lb}$}‚प्र‚त्य‚स‚म‚र्थ‚त्व‚प्र‚तिपाद‚नादित्य‚र्थः । भाव ‚{\tiny $_{7}$}‚ साध‚नो वा साध‚न‚श‚ब्दः । अत्रैव दृष्टान्त‚{\tiny $_{lb}$}‚ \leavevmode\ledsidenote{\textenglish{130/s}} माह । \quotelemma{एक‚साध‚न‚व‚च‚न‚व‚दिति} \cite[19a3]{vn-msN} । य‚थेत्याद्य‚स्यैव विभागः । एक‚स्यार्थ‚स्य ‚{\tiny $_{lb}$}‚क्ष‚णिक‚त्वादेः प्र‚तिपाद‚नायानेक‚स्य ‚{\tiny $_{8}$}‚ साध‚न‚स्य स‚त्व‚कार्य‚त्व‚प्र‚य‚त्नोत्थ‚त्वादेः स‚द्भावेपि ‚{\tiny $_{lb}$}‚स‚त्येकेनैव स‚त्वादीनाम‚न्य‚त‚मेनोपात्तेन त‚स्य क्ष‚णिक‚त्वादेर‚र्थ‚स्य सिद्धेर्निश्च‚यान्न ‚{\tiny $_{lb}$}‚स‚र्व्वेषां साध‚ना ‚{\tiny $_{9}$}‚ \leavevmode\ledsidenote{\textenglish{81a/msK}} नामुपादानं । त‚थैकेनापि दोषेण त‚त्साध‚न‚विघातान्न स‚र्व्वोपादान‚{\tiny $_{lb}$}‚मितीद‚न्दृष्टान्तेन साम्यं । इति त‚स्मान्नोत्त‚र‚प‚क्ष‚वादी पूर्व्व‚मेकं दोष‚मुद्भाव‚य‚न्नेवाप‚र ‚{\tiny $_{1}$}‚ ‚{\tiny $_{lb}$}‚स्य दोषान्त‚र‚स्यानुद्भाव‚नान्निग्र‚हार्हः । पूर्व्व‚व‚दिति साध‚नाभासेनाप्र‚तिपाद‚नात् । ‚{\tiny $_{lb}$}‚भूत‚दोषान‚भिधानाच्च ।
	{\color{gray}{\rmlatinfont\textsuperscript{§~\theparCount}}}
	\pend% ending standard par
      ‚{\tiny $_{lb}$}‚

	  
	  \pstart \leavevmode% starting standard par
	न‚नु च क‚थ‚न्न वादिनो ज‚यो याव‚ता न तेन साध ‚{\tiny $_{2}$}‚ नाभासः प्र‚युक्तः । प्र‚तिवादी ‚{\tiny $_{lb}$}‚त्व‚स‚न्तं दोष‚मुद्भाव‚य‚तीत्य‚त आह । \quotelemma{दोषाभासं} ब्रुवाण‚मुत्त‚र‚प‚क्ष‚वादिनं \cite[18a4]{vn-msN} ‚{\tiny $_{lb}$}‚स्व‚साध‚नात्स‚काशाद‚नुसार‚य‚तोऽनिव‚र्त्त‚य‚त‚स्त ‚{\tiny $_{3}$}‚ दुक्त‚दूष‚णाभास‚त्वेनाप्र‚तिपाद‚य‚त इति ‚{\tiny $_{lb}$}‚याव‚त् । वादिनो न ज‚यः क‚स्माद‚स‚म‚न्वित‚साध‚नाङ्ग‚त्वात् । अस‚म‚न्वित‚साध‚नाङ्गं ‚{\tiny $_{lb}$}‚येन त‚स्य भाव‚स्त ‚{\tiny $_{4}$}‚ त्वं । एत‚देव कुत इत्याह [।] \quotelemma{स‚र्व्व‚दोषाभाव} प्र‚द‚र्श‚नेन साध‚नाङ्ग‚स‚म‚{\tiny $_{lb}$}‚र्थ‚नात् \cite[19a5]{vn-msN} । इत्त्थ‚म्भूत‚ल‚क्ष‚णे क‚र‚णे वा तृतीया \href{http://sarit.indology.info/?cref=P\%C4\%81.2.3.29}{पाणिनि २।३।२९ } ‚{\tiny $_{lb}$}‚अस‚म‚र्थित‚त्त्वात् साध‚ना ‚{\tiny $_{5}$}‚ भास एव तेन प्र‚युक्त इति संक्षेपार्थः । नाप्युत्त‚र‚प‚क्ष‚वादिनो ‚{\tiny $_{lb}$}‚ज‚य इति व‚र्त्त‚ते । त‚स्मादेव‚म‚पीति य‚दि पूर्व‚प‚क्ष‚वाद्युत्त‚र‚प‚क्ष‚वादिनं ‚{\tiny $_{6}$}‚ न निग्र‚ह‚प्राप्तं ‚{\tiny $_{lb}$}‚निगृह्णाति न केव‚ल‚मुत्त‚र‚वादिस‚म्ब‚न्धेनेत्य‚पि श‚ब्दः ॥ ० ॥
	{\color{gray}{\rmlatinfont\textsuperscript{§~\theparCount}}}
	\pend% ending standard par
      ‚{\tiny $_{lb}$}‚

	  
	  \pstart \leavevmode% starting standard par
	\hphantom{.}\quotelemma{निर‚नुयोज्य‚स्यानुयोगः} \cite[19a6]{vn-msN} । अनिग्र‚ह‚प्राप्ते निगृहीतोसीत्य‚भिधानं । किं ‚{\tiny $_{7}$}‚ ‚{\tiny $_{lb}$}‚पुन‚रेवं ब्रूत इत्याह [।] \quotelemma{निग्र‚ह‚स्थान‚ल‚क्ष‚ण‚स्य} मिथ्याव्य‚व‚सायाद्य \cite[19a6]{vn-msN} थोक्त‚स्य ‚{\tiny $_{lb}$}‚निग्र‚ह‚स्थान‚ल‚क्ष‚ण‚स्य स‚म्य‚ग‚प‚रिज्ञानादित्य‚र्थः । एव‚ञ्चाप्र‚तिप‚त्तितो निगृ ‚{\tiny $_{8}$}‚ ह्य‚ते । ‚{\tiny $_{lb}$}‚ \leavevmode\ledsidenote{\textenglish{131/s}} अत्रापीत्याद्याचार्यः । य‚दि त‚स्य साध‚न‚स्य वादिन‚म‚भूतैर‚लीकैर्दोषैः स‚व्य‚भिचारा‚{\tiny $_{lb}$}‚दिदोष‚दुष्टं त्व‚या साध‚नं प्र‚युक्तं त‚तो निगृहीतोसीत्येव‚म ‚{\tiny $_{9}$}‚ भियुञ्जीत । त‚दा सोऽस्था‚{\tiny $_{lb}$}‚नेऽस्य व्याख्यानं निर्दोष‚निग्र‚ह‚स्थान‚स्य अस्य विभागादेवास्येति । अभियोक्तेत्य‚स्य ‚{\tiny $_{lb}$}‚विवृतिरुद्भाव‚यितेति । त‚था चालीक‚दोष ‚{\tiny $_{1}$}‚ स्याभिधायित्वे स‚ति दोषोद्भाव‚ल‚क्ष‚ण‚{\tiny $_{lb}$}‚स्योत्त‚र‚स्याप्र‚तिप‚त्तेर‚भिधानाद‚प्र‚तिभ‚यैव क‚र‚ण‚भूत‚योत्त‚र‚वादी निगृहीत इति कृत्वा ‚{\tiny $_{lb}$}‚नेद‚न्निर‚नुयो ‚{\tiny $_{2}$}‚ ज्यानुयोगाभिधान‚न्निग्र‚ह‚स्थान‚म‚तोऽप्र‚तिभानिग्र‚ह‚स्थानात्स‚काशान्न नि‚{\tiny $_{lb}$}‚ग्र‚ह‚स्थानान्त‚रं । क‚दा चाय‚म‚प्र‚तिभ‚या निगृह्य‚त इत्याह [।] \quotelemma{इत‚रेण} \cite[19a8]{vn-msN} ‚{\tiny $_{lb}$}‚वादि ‚{\tiny $_{3}$}‚ ना त‚दुक्त‚स्योत्त‚राभास‚त्वे प्र‚तिपादिते अन्य‚था न द्व‚योरेक‚स्यापि पूर्व‚व‚ज्ज‚य‚प‚रा‚{\tiny $_{lb}$}‚ज‚यावित्याकूतं । एवं प्र‚तिवादिस‚म्ब‚न्धेनास्यापृथ‚ग्व‚च ‚{\tiny $_{4}$}‚ नं प्र‚तिपाद्य वादिस‚म्ब‚न्धेना‚{\tiny $_{lb}$}‚प्याह [।] अथोत्त‚र‚वादिनं भूतं स‚त्यं साध‚न‚दोषं स‚व्य‚भिचा[रा]दिक‚मुद्भाव‚य‚न्त‚{\tiny $_{lb}$}‚म‚प‚र इति पूर्व‚प‚क्ष‚वादी दोषाभा ‚{\tiny $_{5}$}‚ स‚व‚च‚नेनाभियुञ्जीत । जात्युत्त‚र‚म‚नैकान्तिकाद्या‚{\tiny $_{lb}$}‚भासं त्व‚या प्र‚युक्तं । त‚स्मान्निगृ[ही]तोसीत्येवं य‚द्य‚भियुञ्जीतेत्य‚र्थः । त‚दा त‚स्योद्‚{\tiny $_{lb}$}‚भावित‚स्य दो ‚{\tiny $_{6}$}‚ ष‚स्य व्य‚भिचारादेस्तेनोत्त‚र‚वादिना भूत‚दोष‚त्वे प्र‚तिपादिते जात्यु‚{\tiny $_{lb}$}‚ [त्त]र‚व‚त्वे प‚रिहृत इति याव‚त् । साध‚नाभास‚व‚च‚नेनैव वादी निगृह्य‚ते इति ॥ ‚{\tiny $_{lb}$}‚त‚स्मादेव‚म‚पि प्र ‚{\tiny $_{7}$}‚ तिवादिस‚म्ब‚न्धेनापि नेदं हेत्वाभासेभ्यो भिद्य‚त इति पृथ‚ग्वाच्यं । ‚{\tiny $_{lb}$}‚अस्यैवोपोद्व‚ल‚न‚म‚व‚श्यं हि द्वाविंश‚तिनिग्र‚ह‚स्थान‚वादिना हेत्वाभासाः पृथ‚ग् निग्र‚ह‚{\tiny $_{lb}$}‚स्था ‚{\tiny $_{8}$}‚ न‚त्वेन व‚क्त‚व्याः । किम‚र्थ‚मित्याह । विष‚यान्त‚र‚प्राप्त्य‚र्थं \cite[19b1]{vn-msN} निर‚नुयो‚{\tiny $_{lb}$}‚ज्यानुयोगादिभिर्न्निग्र‚ह‚स्थानैर‚नाक्रान्त‚स‚ङ्ग्र‚ह‚म‚पीति अन्य‚था द्वाविंश‚तित्वं निग्र‚{\tiny $_{lb}$}‚ह‚स्था ‚{\tiny $_{9}$}‚ \leavevmode\ledsidenote{\textenglish{81b/msK}} नानाम‚भ्युप‚ग‚म‚म्विरुद्ध्य‚त इत्य‚भिप्रायः । त‚था च त‚दुक्तौ तेषां हेत्वाभासानां ‚{\tiny $_{lb}$}‚निग्र‚ह‚स्थानेनोक्तौ स‚त्याम‚प‚रोक्तिः । अप‚र‚स्य निर‚नुयोज्यानुयोग‚स्योक्तिर्निर[ि] ‚{\tiny $_{lb}$}‚र्थ‚का ‚{\tiny $_{1}$}‚ हेत्वाभास‚व‚च‚नेनैव संगृहीत‚त्वात् ॥ ० ॥
	{\color{gray}{\rmlatinfont\textsuperscript{§~\theparCount}}}
	\pend% ending standard par
      ‚{\tiny $_{lb}$}‚\textsuperscript{\textenglish{132/s}}

	  
	  \pstart \leavevmode% starting standard par
	\hphantom{.}\quotelemma{सिद्धान्त‚म‚भ्युपेत्यानिय‚मात् क‚थाप्र‚स‚ङ्गोऽप‚सिद्धान्त} इति \cite[19b1]{vn-msN} सूत्रं ‚{\tiny $_{lb}$}‚सिद्धान्त‚म‚भ्युपेत्य प‚क्ष‚प‚रिग्र‚हं कृत्वाऽनिय‚मात् पूर्व‚प्र‚कृतार्थोप‚रोधेन शास्त्र‚व्य‚व‚स्था‚{\tiny $_{lb}$}‚म‚नादृत्येति याव‚त् । क‚थाप्र‚स‚ङ्गोऽर्थान्त‚रोप‚व‚र्ण‚नं । क‚स्य‚चिद‚र्थ‚स्येति ध‚र्मिणो ध‚र्मा‚{\tiny $_{lb}$}‚न्त‚रं प्र‚तिज्ञाय प्र‚तिज्ञातार्थ ‚{\tiny $_{3}$}‚ विप‚र्य‚यो विरोधः । इदं उदाह‚र‚णेन स्प‚ष्ट‚य‚ति । \quotelemma{य‚था न ‚{\tiny $_{lb}$}‚स‚तो व‚स्तुनो विनाशो} \cite[19b2]{vn-msN} निर‚न्व‚यः केव‚लं तिरोभाव‚मात्रं भ‚व‚ति नास‚त् ‚{\tiny $_{lb}$}‚ख‚र‚विषाण‚तुल्य‚मु ‚{\tiny $_{4}$}‚ त्प‚द्य‚ते । किन्त‚र्ह्याविर्भाव‚तः । स‚देवोत्प‚द्य‚त इत्येवं \quotelemma{कापिलः} सिद्धा‚{\tiny $_{lb}$}‚न्त‚व्य‚व‚स्थामाद‚र्श्य प‚क्ष‚ङ्क‚रोति । एका प्र‚कृतिर्व्य‚क्त‚स्याव्य‚क्त‚ल‚क्ष‚णा । व्य‚क्त‚स्येति ‚{\tiny $_{lb}$}‚म ‚{\tiny $_{5}$}‚ ह‚दादेः । अत्र हेतुमाह विकाराणां श‚ब्दादीनाम‚न्व‚य‚द‚र्श‚नात् । मृद‚न्व‚यानामि‚{\tiny $_{lb}$}‚त्यादिदृष्टान्तः । त‚था चाय‚मित्युप‚न‚य‚नः\edtext{}{\lemma{नः}\Bfootnote{? प‚न‚यः}}। \quotelemma{सुख‚दुःख‚मोह‚स‚म‚न्वित} \cite[19b3]{vn-msN} ‚{\tiny $_{lb}$}‚इति ‚{\tiny $_{6}$}‚ सुखादिम‚य‚त्वं द‚र्श‚य‚ति । द‚र्शित‚ञ्च सुखादिम‚य‚त्वं व्य‚क्त‚स्य पूर्वं य‚थासांख्येना‚{\tiny $_{lb}$}‚भिम‚तं । त‚त्त‚स्मात् सुखादिभिरेक‚प्र‚कृतिरित्य‚यं व्य‚क्त‚भेदः । इति निग‚म‚नं ‚{\tiny $_{7}$}‚ सुखादि‚{\tiny $_{lb}$}‚भिरितीत्त्थंभूत‚ल‚क्ष‚णे तृतीया । सुखादिप्र‚कारा सुखादिल‚क्ष‚णा । एका प्र‚कृतिर‚स्ये‚{\tiny $_{lb}$}‚त्य‚र्थः । अन्ये प‚ठ‚न्ति । एका \quotelemma{प्र‚कृतिर्व्य‚क्ता} व्य‚क्त‚विकाराणाम‚न्व‚य‚द ‚{\tiny $_{8}$}‚ \leavevmode\ledsidenote{\textenglish{82a/msK}} र्श‚नादिति । ‚{\tiny $_{lb}$}‚एव‚ञ्च व्याच‚क्ष‚ते । एका प्र‚कृतिर‚भिन्ना स‚र्व्वात्म‚स्व‚भावा व्य‚क्ताव्य‚क्त‚विकारा‚{\tiny $_{lb}$}‚णाम‚न्व‚य‚द‚र्श‚नात् । ये व्य‚क्ता विकारा म‚ह‚दाद‚यो ये चाव्य‚क्ताः प्र‚धानात्म‚नि व्य‚व‚{\tiny $_{lb}$}‚स्थितास्तेषा ‚{\tiny $_{1}$}‚ म‚प्य‚न्व‚य‚द‚र्श‚नादिति । अप‚रे तु प‚ठ‚न्ति । \quotelemma{एका प्र‚कृतिर‚व्य‚क्ता} । व्य‚क्त‚{\tiny $_{lb}$}‚विकाराणामिति व्य‚क्त‚रूपाणां विकाराणामिति चाहुः । प्र‚कृतार्थ‚विप‚र्य[ये]णेयं ‚{\tiny $_{lb}$}‚य‚था प्र‚वृत्तेति प्र‚द‚र्श‚नाऽ ‚{\tiny $_{2}$}‚ र्थ‚माह [।] \quotelemma{स कापिल एव‚मुक्त‚वान्प‚र्य‚नुयुज्य‚ते} ‚{\tiny $_{lb}$}‚ \cite[19b4]{vn-msN} । अथ प्र‚कृतिर्विकार इत्येत‚दुभ‚य‚ङ्क‚थं ल‚क्ष‚यित‚व्यं । प्र‚तिप‚त्त‚व्य‚मिति । ‚{\tiny $_{lb}$}‚स एव‚म‚नुयुक्तः प्राह । य‚स्याव‚स्थित‚स्य ध‚र्मान्त‚र‚नि ‚{\tiny $_{3}$}‚ वृत्तौ ध‚र्मान्त‚र‚म्प्र‚व‚र्त्त‚ते सा ‚{\tiny $_{lb}$}‚प्र‚कृतिर‚व‚स्थित‚रूपा । य‚त्त‚त्प्र‚वृत्तिनिवृत्तिस‚द्ध‚र्मान्त‚रं स विकार इति ल‚क्ष‚यित‚व्यं । ‚{\tiny $_{lb}$}‚प‚र‚मुक्त‚वान् \quotelemma{साङ्ख्यः} प्र‚कृतार्थ‚प‚रित्याग‚दो ‚{\tiny $_{4}$}‚ षेणोप‚पाद्य‚ते । सोय‚म्वादी प्र‚कृतार्थ‚{\tiny $_{lb}$}‚विप‚र्य‚याद‚निय‚मात् क‚थाम्प्र‚स‚ञ्ज‚य‚ति । पूर्व्व‚प्र‚कृतं प‚रित्य‚ज‚तीत्य‚र्थः । क‚थ‚मित्याह ‚{\tiny $_{lb}$}‚ \leavevmode\ledsidenote{\textenglish{133/s}} [।] प्र‚तिज्ञातं ख‚ल्व‚नेनेति \cite[19b5]{vn-msN} पूर्वोक्तं स्म‚र ‚{\tiny $_{5}$}‚ य‚ति । य‚द्येव‚ङ्को दोष इत्याह । ‚{\tiny $_{lb}$}‚ \quotelemma{स‚द‚स‚तो} रित्यादि । स‚त‚स्तिरोभाव‚मेकान्तेन विनाश‚म‚न्त‚रेण न क‚स्य‚चिद्ध‚र्म‚स्य ‚{\tiny $_{lb}$}‚प्र‚वृत्युप‚र‚मः सिध्य‚ति । केन चिद्विरूपे ‚{\tiny $_{6}$}‚ णाव‚स्थाने स‚ति स तिरोहितोऽङ्ग‚स्त‚स्या‚{\tiny $_{lb}$}‚व‚स्थित‚स्यात्म‚भूतः प‚र‚भूतो वा भ‚वेत् । आत्म‚भूत‚त्वे तिरोहिताद‚व्य‚तिरेकात् ‚{\tiny $_{lb}$}‚तिरोहित‚व‚द‚व‚स्थित‚स्याप्य‚न‚व‚स्थानं ‚{\tiny $_{7}$}‚ अव‚स्थित‚व‚च्च त‚द‚व्य‚तिरेक‚त‚स्तिरोहित‚स्या‚{\tiny $_{lb}$}‚प्य‚व‚स्थान‚मासं\edtext{}{\lemma{मासं}\Bfootnote{? शं}}क्य‚ते । प‚र‚भूत‚त्वेपि क‚थ‚म‚न‚न्व‚यो न विनाशो न ह्य‚न्य‚स्या‚{\tiny $_{lb}$}‚व‚स्थानेऽन्य‚द‚व‚तिष्ठ‚ते । अन्यो वान्य‚स्यान्व‚य‚श्चैत‚न्य‚स्याऽ ‚{\tiny $_{8}$}‚ पि घ‚टान्व‚य‚प्र‚स‚ङ्गात् । त‚था ‚{\tiny $_{lb}$}‚नास‚त आविर्भाव‚मुत्पाद‚म‚न्त‚रेण क‚स्य‚चिद्ध‚र्म‚स्य प्र‚वृत्तिर्वा सिध्य‚ति ।
	{\color{gray}{\rmlatinfont\textsuperscript{§~\theparCount}}}
	\pend% ending standard par
      ‚{\tiny $_{lb}$}‚

	  
	  \pstart \leavevmode% starting standard par
	न‚नु च विद्य‚मान‚मेव ध‚र्मान्त‚र‚माविर्भाव्य‚ते । ग्र‚ह‚ण‚विष‚य‚भा ‚{\tiny $_{9}$}‚ \leavevmode\ledsidenote{\textenglish{82b/msK}} व‚मापाद्य‚ते । ‚{\tiny $_{lb}$}‚न विद्य‚मान‚स्य क्रियास्त्युपादान‚मिति उप‚ल‚ब्धिर्वा विद्य‚मान‚त्वात् । न कार‚क‚{\tiny $_{lb}$}‚ज‚न्य‚त्व‚मित्येवं प्र‚त्य‚व‚स्थितः प्र‚तिषिद्धः \quotelemma{साङ्ख्यः} । क्व‚चित् स‚प्त‚म्याप‚द्य‚ते । त‚त्र ‚{\tiny $_{lb}$}‚प्र‚त्य ‚{\tiny $_{1}$}‚ व‚स्थिते प्र‚तिवादिनि स‚तीति व्याख्येयं । य‚दि स \quotelemma{कापिलः} स‚तो ध‚र्म‚स्यात्म‚{\tiny $_{lb}$}‚हान‚म‚स‚त‚श्चात्म‚लाभ‚म‚भ्युपैति त‚दानीम‚प‚सिद्धान्तो भ‚व‚ति । अभ्युप‚ग‚म‚विरुद्ध‚स्य ‚{\tiny $_{lb}$}‚प्र‚तिज्ञा ‚{\tiny $_{2}$}‚ नाद‚प‚सिद्धान्त‚संज्ञ‚कं निग्र‚ह‚स्थान‚म‚स्य भ‚व‚तीत्य‚र्थः । अथ स‚त आत्म‚{\tiny $_{lb}$}‚हान‚म‚स‚त‚श्चात्म‚लाभ‚न्नाभ्युपैति । एव‚म‚प्येक‚प्र‚कृतिर्विकाराणामिति योयं प‚क्षः ‚{\tiny $_{lb}$}‚पूर्व्व‚प्र‚तिज्ञातः सोस्य ‚{\tiny $_{3}$}‚ न सिध्य‚ति प्र‚कृतिविकार‚ल‚क्ष‚ण‚स्यान‚व‚स्थित‚त्वात् । त‚था हि ‚{\tiny $_{lb}$}‚त‚योर्ल‚क्ष‚णं य‚स्याव‚स्थित‚स्येत्यादिनोक्तं । त‚स्य चायोगः । स‚द‚स‚तोश्चे \cite[19b6]{vn-msN} ‚{\tiny $_{lb}$}‚त्यादिना प्र‚तिपा[ि]द‚त इत्येतावा ‚{\tiny $_{4}$}‚ न्प‚र‚ग्र‚न्थः । अत्र स‚म्प्र‚त्याचार्यः प्र‚तिविध‚त्ते । ‚{\tiny $_{lb}$}‚इतोपि प्र‚तिविद‚ध्म‚ह इति शेषः । न क‚श्चिद‚निय‚मात् सिद्धान्त‚नीतिविरोधात् ‚{\tiny $_{lb}$}‚ \quotelemma{साङ्ख्य} स्य प्र‚स‚ङ्गः । त‚स्माद्य‚त्तेनोप‚ग ‚{\tiny $_{5}$}‚ तं नास‚दुत्प‚द्य‚ते न स[त्]तिरोभ‚व‚तीति ‚{\tiny $_{lb}$}‚त‚स्य स‚म‚र्थ‚नायेद‚मुक्तं । किमुक्त‚मित्याह [।] \quotelemma{एक‚प्र‚कृतिक‚मिदं व्य‚क्त‚म‚न्व‚य‚{\tiny $_{lb}$}‚द‚र्श‚नादिति} \cite[19b8]{vn-msN} । त‚त्रैकेत्येत‚देव विभ‚ज‚ति । त‚द‚वि ‚{\tiny $_{6}$}‚ भ‚क्त‚योनिक‚मिदं ‚{\tiny $_{lb}$}‚ \leavevmode\ledsidenote{\textenglish{134/s}} व्य‚क्तं । ते सुखाद‚योऽविभ‚क्ताः अपृथ‚ग्भूता योनिः स्थान‚म‚धिक‚र‚णं य‚स्य ‚{\tiny $_{lb}$}‚व्य‚व‚त‚स्य त‚त्त‚द‚विभ‚क्त‚योनिकं किङ्कार‚णं त‚द‚न्व‚य‚द‚र्श‚नात् । तै[ः] सुखादिभि‚{\tiny $_{lb}$}‚ ‚{\tiny $_{7}$}‚ र‚न्व‚य‚द‚र्श‚नात् तादात्म्योप‚ल‚म्भात् । त‚तः किं सिद्ध‚मित्याह व्य‚क्त‚स्य त‚त्स्व‚{\tiny $_{lb}$}‚भाव‚ता सुखादिस्व‚भाव‚ता । त‚त्र स्व‚भाव‚तैव क‚थ‚न्निश्चितेत्याह । \quotelemma{अभेदोप‚ल‚ब्धे} ‚{\tiny $_{lb}$}‚ \cite[19b8]{vn-msN} रिति । सुखादिभिः ‚{\tiny $_{8}$}‚ श‚ब्दादीनाम‚नानात्त्व‚द‚र्श‚नादिति याव‚त् । एव‚म‚पि ‚{\tiny $_{lb}$}‚किं सिद्ध‚म्भ‚व‚तीत्याह । \quotelemma{स‚र्व्व‚स्य} \cite[19b9]{vn-msN} श‚ब्दादेर्विकार‚ग्राम‚स्य सुखाद्या‚{\tiny $_{lb}$}‚त्म‚क‚स्य नोत्प‚त्तिविनाशाविति स‚म्भ‚व‚ति । क‚स्मादि ‚{\tiny $_{9}$}‚ \leavevmode\ledsidenote{\textenglish{83a/msK}} त्याह [।] \quotelemma{सुखादीनामुत्प‚त्ति} ‚{\tiny $_{lb}$}‚विनाशाभावात् । सुखाद्य‚व्य‚तिरेकात्त‚दात्म‚व‚च्छ‚ब्दाद‚योपि नित्याः सिद्धा भ‚व‚न्ति । ‚{\tiny $_{lb}$}‚त‚था च य‚त्पूर्व्व‚म‚भ्युप‚ग‚तं न‚श\edtext{}{\lemma{श}\Bfootnote{? स}}तो विनाशो नास‚दुत्प‚द्य‚त इति । त‚त्स‚म‚र्थि ‚{\tiny $_{1}$}‚ तं ‚{\tiny $_{lb}$}‚भ‚व‚ति । अत्रैवं \quotelemma{कापिलेन} स्वोप‚ग‚मे स‚म‚र्थिते स‚ति त‚दुक्त‚स्य तेन साङ्ख्येनोक्त‚स्य ‚{\tiny $_{lb}$}‚हेतोर‚न्व‚य‚द‚र्श‚न‚स्य दोष‚म‚सिद्ध‚तादिक‚म‚नुद्भाव्य स एव‚मुक्त‚वान् प‚र्य‚नुमुज्य‚ते । ‚{\tiny $_{lb}$}‚अथ प्र ‚{\tiny $_{2}$}‚ कृतिर्निर्विकार इति क‚थं ल‚क्ष‚यित‚व्य‚मित्येवं विकार‚प्र‚कृत्योर्ल‚क्ष‚णं पृच्छ‚न् ‚{\tiny $_{lb}$}‚स्व‚य‚म‚य‚म‚क्ष‚पादः प्र‚कृतास‚म्ब‚न्धेनानिय‚मात् प्र‚कृतार्थोप‚रोधात् क‚थाम्प्र‚व‚र्त्त ‚{\tiny $_{3}$}‚ य‚ति । ‚{\tiny $_{lb}$}‚य‚स्मात् प्र‚कृतिविकार‚योरिह ल‚क्ष‚णं न प्र‚कृत‚मेव त‚त्किन्त‚द‚भिधाय प‚र्य‚नुयुज्य‚ते ‚{\tiny $_{lb}$}‚त‚स्माद‚प्र‚स्तुत‚प‚र्य‚नुयोक्तृत्त्वाद‚क्ष‚पाद एव निग्र‚हार्ह इति ‚{\tiny $_{4}$}‚ भावः । किन्त‚र्ह्य‚त्रोत्त‚र‚स‚म्ब‚द्ध ‚{\tiny $_{lb}$}‚वाक्य‚मित्याह [।] त‚त्रान्व‚य‚द‚र्श‚न‚हेताविदं स्याद् वाच्यं \cite[20a1]{vn-msN} । व्य‚क्त‚न्नाम प्र‚वृत्ति‚{\tiny $_{lb}$}‚निवृत्तिध‚र्म‚कं न त‚था व्य‚क्त‚व‚त् सुखाद‚यः प्र‚वृत्तिनि ‚{\tiny $_{5}$}‚ वृत्तिध‚र्म‚का इति लिङ्ग‚व‚च‚न‚{\tiny $_{lb}$}‚प‚रिणामेन स‚म्ब‚न्धः । त‚था च व्य‚क्त‚स्य सुखाद्य‚न्व‚येऽस्य व्याख्यानं सुखादिस्व‚भाव‚{\tiny $_{lb}$}‚तायां स‚त्यां प्र‚वृत्तिनिवृत्तिध‚र्म‚ताल‚क्ष ‚{\tiny $_{6}$}‚ णं व्य‚क्त‚स्याव‚हीय‚ते । त‚द‚व्य‚तिरेकेण तेषा‚{\tiny $_{lb}$}‚म‚पि स‚दाव‚स्थानात् । इति त‚स्मान्न त‚द्र‚हित‚सुखादिस्व‚भाव‚ता व्य‚क्त‚स्य । ताभ्या‚{\tiny $_{lb}$}‚म्प्र‚वृत्तिनिवृत्तिभ्यां र‚हितास्त‚थोक्तास्ते च सु ‚{\tiny $_{7}$}‚ खाद‚य‚श्च । ते स्व‚भावो य‚स्य त‚स्य ‚{\tiny $_{lb}$}‚भाव‚स्त‚द्र‚हित‚सुखादिस्व‚भाव‚ता । क‚स्मादित्याह [।] व्य‚क्त‚ल‚क्ष‚ण‚विरोधादि\cite[20a2]{vn-msN}‚{\tiny $_{lb}$}‚\leavevmode\ledsidenote{\textenglish{135/s}}ति व्य‚क्त‚स्य ल‚क्ष‚णं प्र‚वृत्तिनिवृत्तिध‚र्म‚क‚त्वं त‚स्य त‚द्विप‚रीतः । ‚{\tiny $_{8}$}‚ सुखादिभिः प‚र‚स्प‚र‚{\tiny $_{lb}$}‚प‚रिहार‚स्थितिल‚क्ष‚णो विरोधः । त‚था च साध‚न‚न्न स‚दाव‚स्थित‚रूप‚सुखादिस्व‚भाव‚{\tiny $_{lb}$}‚मिदं ते व्य‚क्तं प्राप्नोति त‚द्विप‚रीत‚ध‚र्म‚त्वात् । क्षेत्र‚ज्ञ‚व‚त् ‚{\tiny $_{9}$}‚ \leavevmode\ledsidenote{\textenglish{83b/msK}} न च सुखादिव्य‚क्त‚योरेक- ‚{\tiny $_{lb}$}‚स्व‚भाव‚ता । प‚र‚स्प‚र‚विरुद्ध‚ध‚र्माध्यासित‚त्वात् । स‚त्त्व‚र‚ज‚स्त‚म‚सामिव चैत‚न्यानामिव ‚{\tiny $_{lb}$}‚वा । एव‚ञ्च व्य‚क्त‚स्य सुखादिस्व‚भाव‚तायोगे सुखाद्य‚न्व‚य‚द‚र्श‚न ‚{\tiny $_{1}$}‚ सिद्धो हेतुः । क‚स्मा‚{\tiny $_{lb}$}‚दिदं स‚म्ब‚द्धं दूष‚ण‚मित्याह । \quotelemma{एवं हि त‚स्य साङ्ख्य‚स्य साध‚न‚दोषोद्भाव‚नेन} हेत्व‚सि‚{\tiny $_{lb}$}‚द्ध‚ताचोद‚नेनैक‚प्र‚कृतीदं व्य‚क्त‚मित्य‚यं \quotelemma{प‚क्षो दूषितो भ‚व‚नि} \cite[20a3]{vn-msN} । स पु ‚{\tiny $_{2}$}‚ न‚र्नैयायिकः ‚{\tiny $_{lb}$}‚साध‚ने दोष‚म‚सिद्ध‚ताख्य‚म‚नुप‚संहृत्याप‚द‚र्श्य अप्र‚कृत‚प्र‚कृतिविकार‚ल‚क्ष‚ण‚प‚र्य‚नुयोगेन ‚{\tiny $_{lb}$}‚क‚थां प्र‚तान‚य‚त्य‚विमुञ्चं स्व‚दोष‚म‚न्य‚मात्क‚था ‚{\tiny $_{3}$}‚ प्र‚स‚ङ्गं प‚र‚त्र \quotelemma{साङ्ख्ये} त‚प‚स्विन्यु‚{\tiny $_{lb}$}‚प‚क्षिप‚ति । प‚र आहाय‚मेवासिद्ध‚ताख्यो दोषोनेन प्र‚कारेण प्र‚कृतिविकार‚ल‚क्ष‚ण‚{\tiny $_{lb}$}‚प‚र्य‚नुयोग‚द्वारेणास्माभिर‚प्युच्य ‚{\tiny $_{4}$}‚ त इति । आचार्य आह । \quotelemma{एष नैमित्तिकानां} \cite[20a3]{vn-msN} ‚{\tiny $_{lb}$}‚ज्योतिर्ज्ञान‚विदां विष‚यः । नायं त्व‚दुक्त‚स्य वाक्य‚स्यार्थ इति याव‚त् । य‚तो न लोकः ‚{\tiny $_{lb}$}‚श‚ब्दैर‚प्र‚तिपादित‚म‚र्थं प्र‚ति ‚{\tiny $_{5}$}‚ प‚त्तुं स‚म‚र्थः अर्थ‚प्र‚क‚र‚णादिभिर्विनेत्य‚ध्याहारः । त‚स्मात् स ‚{\tiny $_{lb}$}‚एवायं \cite[20a4]{vn-msN} प्र‚तिज्ञाविरोध‚प्र‚स्तावे निर्द्दिष्टो \quotelemma{भ‚ण्डालेख्य‚न्यायोत्राप्य‚प‚सिद्धान्तो} ‚{\tiny $_{lb}$}‚न केव‚लं त‚त्रेत्यऽ ‚{\tiny $_{6}$}‚ पि श‚ब्दः । य‚था हि \quotelemma{भ‚ण्डाः} प्राकृतान् विस्माप‚य‚न्तः शीघूम‚र्द्ध‚च‚न्द्रा‚{\tiny $_{lb}$}‚काराम‚ल्पीय‚सीं रेखामालिख्य भ‚ण‚न्ति प‚श्य‚त ताल‚मात्रेण ह‚स्ती विलिखितोस्माभि‚{\tiny $_{lb}$}‚रिति त‚त्र केचि ‚{\tiny $_{7}$}‚ त् म‚न्द‚म‚त‚य‚स्त‚थैव प्र‚तिप‚द्य‚न्ते । केचिद् दुर्विद‚ग्ध‚धियः प‚र्य‚नुयुञ्ज‚ते । ‚{\tiny $_{lb}$}‚न‚नु नोस्य क‚र्ण्ण‚पाद‚द‚न्ताद‚यः प्र‚तीय‚न्ते त‚त्क‚थ‚म‚य‚न्त‚द्विक‚लो ह‚स्ती भ‚व‚तीति । ते पु ‚{\tiny $_{lb}$}‚न‚राहु ‚{\tiny $_{8}$}‚ ः । स‚त्यं न प्र‚तीय‚न्ते । अस्माभिस्तु स‚माप्त‚स‚क‚ल‚क‚लः क‚रेणुर‚यं लिखितः । ‚{\tiny $_{lb}$}‚तास्त‚स्य स‚क‚लाः क‚लाः स‚लिल इव म‚ग्न‚त्वान्नोप‚ल‚भ\edtext{}{\lemma{भ}\Bfootnote{? भ्य}}न्ते कुम्भ‚क‚देश‚मात्र‚{\tiny $_{lb}$}‚न्त्विद‚म‚स्योप ‚{\tiny $_{9}$}‚ \leavevmode\ledsidenote{\textenglish{84a/msK}} ल‚भ्य‚त इति त‚थाजातीय‚क‚मेत‚त् प‚र‚स्यापि धार्ष्ट्य‚विजृम्भितं । य‚दि ‚{\tiny $_{lb}$}‚नाम नाय‚म‚र्थोस्माद् बाह्यात् प्र‚तीय‚ते त‚थाप्य‚नेन प्र‚कारेणोच्य‚त इति । अपि चोच्य‚{\tiny $_{lb}$}‚ताम ‚{\tiny $_{1}$}‚ य‚मेवार्थोनेन प्र‚कारेण त‚थाप्य‚सिद्ध‚स्य हेत्वाभासेष्व‚न्त‚र्भावात् त‚द्व‚च‚नेनैवा‚{\tiny $_{lb}$}‚भिधान‚मिति नाप‚सिद्धान्तः प‚थ‚गुपादेयो भ‚वेदित्येत‚दुप‚संहार‚व्याजेनाह । \quotelemma{य‚थो ‚{\tiny $_{2}$}‚ क्तेन} ‚{\tiny $_{lb}$}‚न्यायेने \cite[20a4]{vn-msN} त्यादि ॥ ० ॥
	{\color{gray}{\rmlatinfont\textsuperscript{§~\theparCount}}}
	\pend% ending standard par
      ‚{\tiny $_{lb}$}‚\textsuperscript{\textenglish{136/s}}

	  
	  \pstart \leavevmode% starting standard par
	\hphantom{.}\quotelemma{हेत्वाभासाश्च य‚थोक्ता} इति सूत्रं । इद‚माक्षेप‚पूर्व‚कं \quotelemma{वात्स्याय‚नो} व्याच‚ष्टे । \quotelemma{किं ‚{\tiny $_{lb}$}‚पुन‚रिति} \cite[20a6]{vn-msN} हेत्वाभास‚ल‚क्ष‚णाद्य‚द‚न्य‚ल्ल‚क्ष‚णं तेन स‚म्ब‚न्धा ‚{\tiny $_{3}$}‚ न्निग्र‚ह‚स्थान‚त्व‚मा‚{\tiny $_{lb}$}‚प‚द्य‚न्ते । किमिवेत्याह । \quotelemma{य‚था प्र‚माणानि प्र‚मेय‚त्त्वं ल‚क्ष‚णान्त‚र‚व‚सा\edtext{}{\lemma{सा}\Bfootnote{? व‚सा}}दाप‚द्य‚न्त} ‚{\tiny $_{lb}$}‚ \cite[20a6]{vn-msN} इति व‚र्त्त‚ते तानि हिप्र‚मिति क्रियाया[ः] कार‚ण‚त्वात् । प्र‚मा ‚{\tiny $_{4}$}‚ णानि प्र‚माणा‚{\tiny $_{lb}$}‚न्त‚रेण तु य‚दा प्र‚मीय‚न्ते त‚दा क‚र्म‚त्वात् प्र‚मेयानि । त‚त एव प‚दार्थ‚त्वात् प्राप्तः संश‚यः । ‚{\tiny $_{lb}$}‚अत्राह मुनिना य‚थोक्त इति । अस्यैव विव‚र‚णं । ‚{\tiny $_{5}$}‚ \quotelemma{य‚थोक्त‚हेत्वाभास‚ल‚क्ष‚णेनैव निग्र‚ह‚{\tiny $_{lb}$}‚स्थान} भाव इति । इद‚मुक्त‚म्भ‚व‚ति । स‚व्य‚भिचार‚विरुद्ध‚प्र‚क‚र‚ण‚स[म]साध्य‚स‚मातीत‚{\tiny $_{lb}$}‚काला\href{http://sarit.indology.info/?cref=ns\%C5\%AB.1.2.4}{न्या० सू० १।२।४ } इति हेत्वाभासा इति प्र ‚{\tiny $_{6}$}‚ भेद‚मुप‚क्र‚म्य य‚त्प्र‚त्येकं ल‚क्ष‚ण‚मुक्तं । ‚{\tiny $_{lb}$}‚ अनैकान्तिकः स‚व्य‚भिचारः\href{http://sarit.indology.info/?cref=ns\%C5\%AB.1.2.5}{न्या० सू० १।२।५ } सिद्धान्त‚म‚भ्युपेत्य त‚द्विरोधाद्विरुद्धं ‚{\tiny $_{lb}$}‚ \href{http://sarit.indology.info/?cref=ns\%C5\%AB.1.2.6}{न्या० सू० १।२।६ } । य‚स्मात् प्र‚क‚र‚ण‚चिन्ता स निर्ण्ण‚यार्थ‚म‚प‚दिष्टः प्र‚क‚र‚ण ‚{\tiny $_{7}$}‚ स‚मः ‚{\tiny $_{lb}$}‚ \href{http://sarit.indology.info/?cref=ns\%C5\%AB.1.2.7}{न्या० सू० १।२।७ } साध्याविशिष्टः साध्य‚त्वात् साध्य‚स‚मः \href{http://sarit.indology.info/?cref=ns\%C5\%AB.1.2.8}{न्या० सू० १।२।८ } ‚{\tiny $_{lb}$}‚ कालात्य‚याप‚दिष्टः कालातीत \href{http://sarit.indology.info/?cref=ns\%C5\%AB.1.2.9}{न्या० सू० १।२।९ } इति तेनैव ल‚क्ष‚णेनैषान्निग्र‚ह‚{\tiny $_{lb}$}‚स्थान‚त्त्वं न पुन‚स्त‚त ल‚क्ष‚णान्त‚र‚म‚पेक्ष्य‚त इति । अत्रापी ‚{\tiny $_{8}$}‚ \cite[20a7]{vn-msN} त्याचार्यः । ‚{\tiny $_{lb}$}‚क‚थ‚ञ्चिन्त्य‚मित्याह । \quotelemma{किन्ते य‚था भ‚व‚द्भिर्ल‚क्षित‚प्र‚भेदास्त‚थैव ते भ‚व‚न्त्याहो ‚{\tiny $_{lb}$}‚स्विद‚न्य‚थे} ति \cite[20a7]{vn-msN} । ल‚क्षितः प्र‚भेदो येषामिति विग्र‚हः । त‚त्त‚र्हि किन्त[द्] ‚{\tiny $_{lb}$}‚ चिन्त्य‚त इत्याह ‚{\tiny $_{9}$}‚ \leavevmode\ledsidenote{\textenglish{84b/msK}} त‚त्तु चिन्त्य‚मान‚मिहातिप्र‚स‚ज्य‚त इति न प्र‚त‚न्य‚ते । इद‚मेवागूरितं । ‚{\tiny $_{lb}$}‚विद‚न्त्येव केचिद‚त्र हेत्वाभासा एव न युज्यंते केचित्तु हेत्वाभासा अपि न स‚ङ्ग्र‚हीता ‚{\tiny $_{lb}$}‚इत्य‚स्मिंश्च विचारे हेत्वाभास ‚{\tiny $_{1}$}‚ वार्त्तिकं स‚क‚ल‚म‚व‚तार‚यित‚व्य‚मिति शास्त्रान्त‚र‚मेव ‚{\tiny $_{lb}$}‚भ‚वेत् । अव‚दात‚म‚त‚य‚स्त्व‚स्म‚द्विहित‚हेत्वाभास‚ल‚क्ष‚ण‚विप‚र्य‚येण दूरान्त‚र‚त्वात्त‚द् ‚{\tiny $_{lb}$}‚वैस‚शं\edtext{}{\lemma{शं}\Bfootnote{?}} । त‚स्मादुपेक्षै ‚{\tiny $_{2}$}‚ व युज्य‚त इति । त‚थापि म‚न्द‚म‚तिविबोध‚नायापि शास्त्र‚{\tiny $_{lb}$}‚मुच्य‚त इति । कालातीत‚प्र‚क‚र‚ण‚स‚म‚योस्ताव‚द्धेत्वाभास‚त्वं य‚था नोप‚प‚द्य‚ते त‚था व‚र्ण्य‚ते । ‚{\tiny $_{lb}$}‚त‚त्र \quotelemma{कालात्य‚याप‚दि ‚{\tiny $_{3}$}‚ ष्टः क‚लातीतः} त‚दिह [।] बृद्ध‚नैयायिकानाम‚पास्य म‚त‚{\tiny $_{lb}$}‚माचार्य \quotelemma{दिङ्नाग} पादैर्भाषित‚त्वादिदानीन्त‚ना \quotelemma{वात्स्याय‚ना} द‚योमुमेव स्थिप‚क्ष‚माहुः । ‚{\tiny $_{lb}$}‚त‚त्रैव‚म्ब्रूमः । ‚{\tiny $_{4}$}‚ कालात्य‚येन युक्तो य‚स्यार्थैक‚देशोऽप‚दिश्य‚मान‚स्य स कालात्य‚याप‚दिष्टः ‚{\tiny $_{lb}$}‚ \leavevmode\ledsidenote{\textenglish{137/s}} कालातीत इत्युच्य‚ते । निद‚र्श‚नं [।] नित्यः श‚ब्दः संयोग‚व्यंग्य‚त्वाद्रूप‚व‚त् । प्रागूद्र्ध्वं ‚{\tiny $_{5}$}‚ ‚{\tiny $_{lb}$}‚ च व्य‚क्तेर‚व‚स्थितं रूपं प्र‚दीप‚घ‚ट‚संयोगेन व्य‚ज्य‚ते । त‚था श‚ब्दो व्य‚व‚स्थितो ‚{\tiny $_{lb}$}‚भेरीक‚र्ण्ण‚संयोगेन दारुप‚र्ण‚योगेन वा व्य‚ज्य‚ते । त‚स्मात्संयोग‚व्यंग्य‚त्वान्नित्यः ‚{\tiny $_{6}$}‚ श‚ब्द ‚{\tiny $_{lb}$}‚इति । अय‚म‚हेतुः कालात्य‚याप‚देशात् व्यंज‚क‚स्य संयोग‚स्य कालं न व्यंग्य‚स्य रूप‚स्य ‚{\tiny $_{lb}$}‚व्य‚क्तिर‚त्येति स‚ति प्र‚दीप‚संयोगे रूप‚स्य ग्र‚ह‚णं भ‚व‚ति । न निवृत्त‚संयोगे ‚{\tiny $_{7}$}‚ रूप‚ङ्गृ‚{\tiny $_{lb}$}‚ह्य‚ते । निवृत्ते तु दारुप‚र्ण‚संयोगे दूर‚स्थेन श‚ब्दः श्रूय‚ते विभाग‚काले नेयं श‚ब्द‚स्य ‚{\tiny $_{lb}$}‚व्य‚क्तिः संयोग‚काल‚म‚त्येतीति संयोग‚निमित्ता भ‚व‚ति । कार‚णाभावाद्धि कार्या ‚{\tiny $_{8}$}‚ भाव ‚{\tiny $_{lb}$}‚इति । न‚न्व‚य‚म‚नैकान्तिक एव । संयोग‚व्यंग्य‚त्वादिति । अनित्य‚म‚पि संयोगेन व्य‚ज्य‚{\tiny $_{lb}$}‚मानं दृष्टं य‚था घ‚ट इति । न । संयोग व्यंग्य‚त्वेनाव‚स्थान‚स्य साध्य‚त्वान्न ब्रूमो ‚{\tiny $_{9}$}‚ \leavevmode\ledsidenote{\textenglish{85a/msK}} ‚{\tiny $_{lb}$}‚ नित्यः श‚ब्द इति । अपि त्व‚व‚तिष्ठ‚ते श‚ब्द इत्य‚यं प्र‚तिज्ञार्थ‚स्त‚दा च संयोग‚व्यंग्य‚त्वा‚{\tiny $_{lb}$}‚दित्य‚यं हेतुर‚नैकान्तिको न ह्य‚न‚व‚स्थितं किञ्चित्संयोगेनाभिव्य‚ज्य‚मान[ः] ‚{\tiny $_{lb}$}‚ क‚थ‚मिति ‚{\tiny $_{1}$}‚ त‚द‚नेन प्र‚कारेण संयोग‚व्य‚ङ्ग्य‚त्व‚मेव श‚ब्द‚स्य प्र‚तिषिद्ध्य‚त इति नाय‚म‚{\tiny $_{lb}$}‚सिद्धाद् व्याव‚र्त‚ते । अन्य‚थानेयं श‚ब्द‚स्य व्य‚क्तिः संयोगिकाल‚म‚त्येतीति न संयोग‚{\tiny $_{lb}$}‚निमित्ता ‚{\tiny $_{2}$}‚ भ‚व‚तीति । व‚च‚न‚स्य कोर्थ इति व‚क्त‚व्यं । स्याद् बुद्धिः स‚र्व‚दाध‚र्मिण्य‚विद्य‚{\tiny $_{lb}$}‚मान‚स्यासिद्ध‚त्वं । अय‚न्तु न स‚र्व‚था ध‚र्मिण्य‚सिद्धो येनोत्प‚त्तिकाले संयोग‚व्य‚ङ्ग्य‚त्व‚म‚स्ति । ‚{\tiny $_{lb}$}‚न ‚{\tiny $_{3}$}‚ तूप‚ल‚ब्धिकाल इति । त‚दुक्तं । एक‚देशासिद्ध‚स्यापि असिद्ध‚त्त्द‚प‚रिज्ञानात् । य‚था ‚{\tiny $_{lb}$}‚नित्याः प‚र‚माण‚वो ग‚न्ध‚व‚त्वात् । श्वेत‚नाश्च त‚र‚वः स्वापादिति । य‚श्चा ‚{\tiny $_{4}$}‚ नित्यः श‚ब्द ‚{\tiny $_{lb}$}‚इति प्र‚तिजानीते स कुठार‚दारुसंयोगादेः श‚ब्द‚स्योत्प‚त्तिमेव प्र‚तिप‚द्य‚ते । न पुन‚{\tiny $_{lb}$}‚र‚व‚स्थित‚स्याभिव्य‚क्त‚मिति व्य‚क्त‚म‚स्यान्य‚त‚रासिद्ध ‚{\tiny $_{5}$}‚ त्वं । अथ संयोगे स‚त्युप‚ल‚ब्धे‚{\tiny $_{lb}$}‚रिति हेत्व‚र्थाभ्युप‚ग‚मान् नाय‚म‚सिद्धो हेतुरिति स‚माधीय‚ते । त‚थापि तैल‚तेजो‚{\tiny $_{lb}$}‚व‚र्त्तिसंयोगे कुलाल‚मृत्पिण्ड‚द‚ण्ड‚सं ‚{\tiny $_{6}$}‚ योगे च स‚ति दीप‚घ‚टाद‚यः स‚मुप‚ल‚भ्य‚न्ते । न च ‚{\tiny $_{lb}$}‚तेषान्त‚त्र संयोगाप्राप्त्य‚व‚स्थान‚मित्य‚नेनानैकान्तिक एव प्राप्नोतीति न कालातीतः । ‚{\tiny $_{lb}$}‚त‚दुत्त‚र‚काल‚म‚प्य‚व‚स्थाने साध्ये ‚{\tiny $_{7}$}‚ स‚मुदायान्त‚र‚व्य‚य \edtext{}{\lemma{य}\Bfootnote{?}} वादिनो विरुद्धः । स‚प‚क्षा‚{\tiny $_{lb}$}‚भावादेव त‚त्र वृत्तेर‚भावात् । क्ष‚ण‚स्थितिध‚र्म‚व‚ति च ध‚र्मिणि । रूपादिके विद्य‚मान‚{\tiny $_{lb}$}‚त्वात् । स्थिर‚भाव‚वादिन‚न्तु प्र‚ति ‚{\tiny $_{8}$}‚ प्र‚तिब‚न्ध‚वैक‚ल्यं साध‚न‚वैफ‚ल्य‚ञ्च । अन‚ङ्गीकृत‚{\tiny $_{lb}$}‚सिद्धान्ते तु न्याय‚वादिनि प्र‚तिवादिनि पूर्व‚प‚क्ष‚प्र‚तिपादितो दोष इति । एव‚मुदा‚{\tiny $_{lb}$}‚ह‚र‚णान्त ‚{\tiny $_{9}$}‚ \leavevmode\ledsidenote{\textenglish{85b/msK}} रेपि दूष‚ण‚मुत्प्रेक्ष्य व‚क्त‚व्य‚मिति ॥ ० ॥
	{\color{gray}{\rmlatinfont\textsuperscript{§~\theparCount}}}
	\pend% ending standard par
      ‚{\tiny $_{lb}$}‚

	  
	  \pstart \leavevmode% starting standard par
	\hphantom{.}य‚स्मात्प्र‚क‚र‚ण‚चिन्ता स निर्ण्ण‚यार्थ‚म‚प‚दिष्टः प्र‚क‚र‚ण‚स‚मः । \href{http://sarit.indology.info/?cref=ns\%C5\%AB.1.2.7}{न्या० सू० १।२।७ } ‚{\tiny $_{lb}$}‚ विम‚र्शाधिष्ठानौ प‚क्ष‚प्र‚तिप‚क्षाव‚न‚व‚सितौ प्र‚क‚र‚ण‚न्त‚स्य चिन्ताम‚विम‚र्शात् प्र‚भृति ‚{\tiny $_{lb}$}‚प्राङ्निर्ण्ण‚यात् प‚रीक्ष‚णं सा य‚त्र कृता स निर्ण्ण‚यार्थं प्र‚युक्तोभ‚य‚प‚क्ष‚साम्यात् प्र‚क‚र‚ण ‚{\tiny $_{2}$}‚‚{\tiny $_{lb}$}‚म‚न‚तिव‚र्त्त‚मानः प्र‚क‚र‚ण‚स‚मो न निर्ण्ण‚याय क‚ल्प्य‚ते । क‚स्मात्पुनः प्र‚क‚र‚ण‚चिन्ता त‚त्त्वा‚{\tiny $_{lb}$}‚\{??\}प‚ल‚ब्धेः । य‚स्मादुप‚ल‚ब्धे त‚त्वेर्थे निव‚र्त्त‚ते चिन्ता त‚स्मात्सामान्येनाधिग ‚{\tiny $_{3}$}‚ त‚स्य या ‚{\tiny $_{lb}$}‚ \leavevmode\ledsidenote{\textenglish{138/s}} विशेष‚तोऽनुप‚ल‚ब्धिः सा प्र‚क‚र‚ण‚चिन्तां प्र‚योज‚य‚तीति । उदाह‚र‚ण‚म‚नित्यः श‚ब्दो ‚{\tiny $_{lb}$}‚नित्य‚ध‚र्मानुप‚ल‚ब्धेः । अनुप‚ल‚भ्य‚मान‚नित्य‚ध‚र्म‚क‚म‚नित्य‚न्दृ ‚{\tiny $_{4}$}‚ ष्टं स्थाल्यादि । य‚त्र ‚{\tiny $_{lb}$}‚स‚मानो ध‚र्मः संश‚य‚कार‚ण‚हेतुत्वेनोपादीय‚ते संश‚य‚स‚मः स‚व्य‚भिचार एव । या तु ‚{\tiny $_{lb}$}‚विम‚र्श‚स्य विशेषापेक्ष‚तोभ‚य‚प‚क्ष‚विशेषानु ‚{\tiny $_{5}$}‚ प‚ल‚ब्धौ सा प्र‚क‚र‚ण‚म्प्र‚व‚र्त‚य‚ति । य‚था ‚{\tiny $_{lb}$}‚च श‚ब्दे नित्य‚ध‚र्मो नोप‚ल‚भ्य‚ते त‚थानित्य‚ध‚र्मोपि । सेय‚मुभ‚य‚प‚क्ष‚विशेषानुप‚ल‚ब्धिः ‚{\tiny $_{lb}$}‚प्र‚क‚र‚ण‚चिन्ताम्प्र‚यो ‚{\tiny $_{6}$}‚ ज‚य‚ति क‚थ‚म्विप‚र्य‚ये प्र‚क‚र‚ण‚निवृत्तेः । य‚दि नित्य‚ध‚र्मः श‚ब्दे ‚{\tiny $_{lb}$}‚गृह्येत न स्यात्प्र‚क‚र‚णं । य‚दि [न] नित्य‚ध‚र्मो गृह्येत एव‚म‚पि निव‚र्त्त‚ते प्र‚क‚र‚णं । सोयं ‚{\tiny $_{lb}$}‚हेतु ‚{\tiny $_{7}$}‚ रुभौ प‚क्षौ प्र‚व‚र्त्त‚य‚न्नान्य‚त‚र‚स्य निर्ण्ण‚याय क‚ल्प्य‚त इति । न त्व‚यं साध्या‚{\tiny $_{lb}$}‚विशिष्ट एव । नाविशिष्टः । त‚स्यैव प्र‚क‚र‚ण‚प्र‚वृत्तिहेतोर्द्ध‚र्म‚स्य हेतुत्वेनोपादानात् । ‚{\tiny $_{lb}$}‚य‚त्र ‚{\tiny $_{8}$}‚ साध्येन स‚मानो ध‚र्मो हेतुत्वेनोपादीय‚ते स साध्याविशिष्टः । य‚त्र पुनः प्र‚क‚र‚ण‚{\tiny $_{lb}$}‚प्र‚वृत्तिहेतुरेव स प्र‚क‚र‚ण‚स‚म इति । अत्रापि नित्यानित्य‚ध‚र्मानुप‚ल‚म्भ‚द्व ‚{\tiny $_{9}$}‚ \leavevmode\ledsidenote{\textenglish{86a/msK}} यादेव ‚{\tiny $_{lb}$}‚प्र‚क‚र‚ण‚चिन्ता । न त्वेक‚स्मात् । विप‚र्य‚ये प्र‚क‚र‚ण‚निवृत्तिरिति व‚च‚नात् । त‚द्य‚दि ‚{\tiny $_{lb}$}‚नित्यानित्य‚ध‚र्मानुप‚ल‚ब्धेरिति हेतुः स्यात् । स्यात् प्र‚क‚र‚ण‚स‚मः । त‚देक‚ध‚र्मानु ‚{\tiny $_{1}$}‚ प‚{\tiny $_{lb}$}‚ल‚ब्धेस्तूपादाने क‚थं प्र‚क‚र‚ण‚स‚म इत्य‚भिधानीयं । उभ‚य‚ध‚र्मानुप‚ल‚म्भोपादानेपि ‚{\tiny $_{lb}$}‚स‚प‚क्ष‚विप‚क्ष‚योर‚नुवृत्तिव्यावृत्योर‚निश्च‚याद‚साध‚र‚णानैकान्तिको ‚{\tiny $_{2}$}‚ भ‚व‚तीति क‚थ‚{\tiny $_{lb}$}‚म‚स्य हेत्वा[भा]सान्त‚र‚त्वं । भ‚व‚तु नामैक‚ध‚र्मानुप‚ल‚ब्धिरेव हेतुः प्र‚क‚र‚ण‚स‚मः । ‚{\tiny $_{lb}$}‚त‚थापि नित्य‚श‚ब्द‚वाद्य‚व‚श्य‚मेव व्यामोहान्नित्य‚ध‚र्मान् प्र‚तिप‚द्य‚त ‚{\tiny $_{3}$}‚ इति प्र‚तिवाद्य‚{\tiny $_{lb}$}‚सिद्धीयं भ‚व‚ति । अथ प्र‚माणेन नित्य‚ध‚र्म्म‚प्र‚तिक्षेपान्नित्य‚ध‚र्मानुप‚ल‚ब्धिः प्र‚ति‚{\tiny $_{lb}$}‚पाद्य‚ते । त‚दापि निश्चाय‚क‚त्वात् स‚म्य‚ग्ज्ञान‚हेतुरेवायं ‚{\tiny $_{4}$}‚ इति क‚थं हेत्वाभासः प्र‚क‚{\tiny $_{lb}$}‚र‚ण‚स‚मः । त‚दा हि विशेषोप‚ल‚ब्धिरेव हेत्व‚र्थो व्य‚व‚तिष्ठ‚ते । विशेषाश्च नित्य‚स्य ‚{\tiny $_{lb}$}‚कृत‚क‚त्वाद‚यः । न च त‚त्कृता प्र‚क‚र‚ण‚चिन्ता ‚{\tiny $_{5}$}‚ विप‚र्य‚ये प्र‚क‚र‚ण‚निवृत्तेरिति व‚च‚नात् । ‚{\tiny $_{lb}$}‚अपि च नित्य‚ध‚र्मानुप‚ल‚ब्धेरिति किम‚यं प्र‚स‚ज्य‚प्र‚तिषेधः किम्वा प्र‚तियोगिविधानं [।] ‚{\tiny $_{lb}$}‚ य‚दि प्र‚स‚ज्य‚प्र‚तिषे ‚{\tiny $_{6}$}‚ ध‚स्त‚दा प्र‚मेय‚त्वादिव‚त् साधार‚णानैकान्तिकोयं नित्य‚ध‚र्मोप‚{\tiny $_{lb}$}‚ल‚ब्धिः प्र‚तिषेध‚मात्र‚स्यानित्य‚त्व‚र‚हितेष्व‚स‚त्स्व‚पि स‚म्भ‚वात् । अथ प्र‚तियोगि‚{\tiny $_{lb}$}‚विधान‚न्त‚दाप्य‚न‚न्त‚रो ‚{\tiny $_{7}$}‚ दित‚या युक्त्या हेतुप्र‚तिरूप‚त्वायोगः । अन्य‚स्त्व‚न्य‚थेदं सूत्र‚{\tiny $_{lb}$}‚द्व‚यं व्याच‚ष्टे । यो हेतुर्हेतुकालेऽप‚दिष्टोऽत्येत्य‚पैति । क‚स्माद‚पैति । प्र‚त्य‚क्षेणाग‚{\tiny $_{lb}$}‚मेन उभ‚येन वा ‚{\tiny $_{8}$}‚ पीड्य‚मानः स काल‚म‚तीत इति कालातीत इत्युच्य‚ते । कुतः पुनः ‚{\tiny $_{lb}$}‚प्र‚त्य‚क्षाग‚म‚विरोधो ल‚भ्य‚त इति चेत् । च‚तुर्ल‚क्ष‚णो हेतुरिति व‚च‚नात् । त‚थाहि ‚{\tiny $_{lb}$}‚ पूर्व‚व‚च्छेष ‚{\tiny $_{9}$}‚ \leavevmode\ledsidenote{\textenglish{86b/msK}} व‚त्सामान्य‚तो दृष्ट \href{http://sarit.indology.info/?cref=ns\%C5\%AB.1.2.5}{न्या० सू० १।२।५ } ञ्चेत्य‚त्र च‚तूरूपो हेतु‚{\tiny $_{lb}$}‚रिष्टः । पूर्व‚व‚न्नाम साध्ये व्याप‚कं । शेष‚व‚दिति त‚त्स‚मानेस्ति । सामान्य‚त‚श्च दृष्ट‚{\tiny $_{lb}$}‚ञ्च श‚ब्दाद‚विरुद्ध‚ञ्चेति । त‚था भाष्य‚व‚च‚न‚म‚प्य‚स्ति । ‚{\tiny $_{1}$}‚ य‚त्पुन‚र‚नुमानं प्र‚त्य‚क्षा‚{\tiny $_{lb}$}‚ग‚म‚विरुद्धं न्यायाभासः स इति । त‚देवं त्रैरूप्ये स‚ति प्र‚त्य‚क्षाग‚माभ्यां यो वाध्य‚ते ‚{\tiny $_{lb}$}‚ \leavevmode\ledsidenote{\textenglish{139/s}} स कालात्य‚याप‚दिष्टः । स च त्रिधा भिद्य‚ते प्र‚त्य‚क्ष‚विरुद्ध आ ‚{\tiny $_{2}$}‚ ग‚म‚विरुद्ध उभ‚य‚विरुद्ध‚{\tiny $_{lb}$}‚श्चेति [।] प्र‚त्य‚क्ष‚विरुद्धो य‚था अनुष्णोग्निर्द्र‚व्य‚त्वादुद‚क‚व‚त् । आग‚म‚विरुद्धो य‚था ‚{\tiny $_{lb}$}‚ब्राह्म‚णेन सुरा पात‚व्या द्र‚व‚त्वात् क्षीर‚व‚त् । उभ‚य‚विरुद्धो य‚थाऽ ‚{\tiny $_{3}$}‚ र‚श्मिव‚च्च‚क्षुरिन्द्रिय‚{\tiny $_{lb}$}‚त्वाद् घ्राणादिव‚दिति । न चायं किल प‚क्ष‚विरोधः प‚क्ष‚विरोध‚स्य प्र‚तिक्षेपादिति । ‚{\tiny $_{lb}$}‚त‚देत‚त् त्रैरूप्य‚ल‚क्ष‚णान‚व‚वोध‚वैश‚द्यं [।] त्रैरूप्यं हि य‚दा ‚{\tiny $_{4}$}‚ स्वं प्र‚माणैः प‚रिनिश्चितं ‚{\tiny $_{lb}$}‚प‚क्ष‚ध‚र्म‚त्वादिकं त्र‚यं च य‚त्र बाधा त‚त्र प्र‚तिब‚न्धोस्ति । बाधाविनाभाव‚योर्विरो‚{\tiny $_{lb}$}‚घात् । अविनाभावो हि स‚त्येव साध्य‚ध‚र्मे हेतोर्भावः ‚{\tiny $_{5}$}‚ क‚थ‚ञ्चासौ त‚ल्ल‚क्ष‚णो ध‚र्मिणि ‚{\tiny $_{lb}$}‚हेतुः स्यान्न चात्र साध्य‚ध‚र्म इत्यादिक‚म‚त्राबाधित‚विष‚य‚त्व‚दूष‚णानुसारेण व‚क्त‚व्यं । ‚{\tiny $_{lb}$}‚य‚त्र पुन‚रियं बाधोदाहृता न ‚{\tiny $_{6}$}‚ तेषां त्रैल‚क्ष‚ण्यं म‚नाग‚प्य‚स्ति प्र‚तिब‚न्ध‚वैक‚ल्यात् । ‚{\tiny $_{lb}$}‚अभ्युप‚ग‚त‚प‚क्ष‚प्र‚योग‚स्य च प‚क्ष‚दोष एवायं युक्तः । य‚त्पुनः प‚क्ष‚दोष‚त्व‚प‚रिहाराय ‚{\tiny $_{lb}$}‚ब‚ह्व‚स‚म्ब‚द्ध‚मुद्ग्राहि ‚{\tiny $_{7}$}‚ तं त‚द‚त्य‚न्त‚म‚सार‚मिति नेहाव‚सीय‚ते ॥ ० ॥
	{\color{gray}{\rmlatinfont\textsuperscript{§~\theparCount}}}
	\pend% ending standard par
      ‚{\tiny $_{lb}$}‚

	  
	  \pstart \leavevmode% starting standard par
	य‚स्मात्प्र‚क‚र‚ण‚चिन्तेति प्र‚क‚र‚णं भाष्ये निरूपितं । त‚स्योदाह‚र‚णं । अणुर‚ण्व‚{\tiny $_{lb}$}‚न्त‚र‚कार्य‚त्वं प्र‚तिप‚द्य‚ते ‚{\tiny $_{8}$}‚ न‚वेति चिन्तायाङ्क‚श्चिद‚भिध‚त्ते । अणुर‚ण्व‚न्त‚र‚कार्यो ‚{\tiny $_{lb}$}‚रूपादिम‚त्वात् त‚द्द्व्य‚णुकादिव‚दिति । योसाव‚णोर‚णुः कार‚ण‚त्वेनोपादीय‚ते त‚त्रापि ‚{\tiny $_{lb}$}‚रूपादिम‚त्व‚म‚स्तीति ‚{\tiny $_{9}$}‚ \leavevmode\ledsidenote{\textenglish{87a/msK}} चिन्ता किमियं रूपादिम‚त्वाद‚ण्व‚न्त‚र‚कार्यो न वेति चिन्ता- ‚{\tiny $_{lb}$}‚याञ्च य‚दि त‚स्याप्य‚ण्व‚न्त‚र‚कार्य‚त्वं रूपादिम‚त्वादिति व‚क्ति त‚दा त‚स्यापि रूपादि‚{\tiny $_{lb}$}‚म‚त्व‚म‚स्तीति पुन‚र‚पि चिन्ता ‚{\tiny $_{1}$}‚ त‚देव‚म‚न‚व‚स्थारूपं प्र‚क‚र‚णं प्र‚व‚र्त्त‚य‚तीति प्र‚क‚र‚ण‚स‚म ‚{\tiny $_{lb}$}‚इत्युच्य‚ते । अथाणुर‚ण्व‚न्त‚र‚कार्य‚त्वं न प्र‚तिप‚द्य‚ते रूपादिम‚त्वे स‚ति त‚दानैकान्तिको ‚{\tiny $_{lb}$}‚हेतुरिति त‚स्माद् भि ‚{\tiny $_{2}$}‚ द्य‚तेऽनैकान्तिकात् प्र‚क‚र‚ण‚स‚मः । न चाय‚म्विरुद्धोऽविप‚र्य‚य‚{\tiny $_{lb}$}‚साध‚क‚त्वात् । नासिद्धः प‚क्ष‚ध‚र्म‚त्व‚द‚र्श‚नात् । न कालात्य‚याप‚दिष्टः प्र‚त्य‚क्षाग‚माभ्या‚{\tiny $_{lb}$}‚म‚बाध्य‚मान‚त्वात् । ‚{\tiny $_{3}$}‚ अतोऽर्थान्त‚र‚मिति । अथ‚वा प्र‚देशे क‚र‚णं प्र‚क‚र‚ण‚ञ्चेति कार‚ण‚{\tiny $_{lb}$}‚सिद्धिरित्य‚र्थः । प्र‚देशे सिद्धिरितीयं चिन्ता य‚स्माद्धेतोर‚प‚दिष्टा भ‚व‚ति स प्र‚क‚र‚ण‚{\tiny $_{lb}$}‚स‚मः ‚{\tiny $_{4}$}‚ स प्र‚देश‚साध‚क‚त्वात् स‚मः । य‚थैक‚देशेऽसाध‚क‚त्व‚न्त‚थेत‚र‚त्रापीत्य‚साध‚क‚त्व‚{\tiny $_{lb}$}‚सामान्यात् स‚मः । त‚स्मादेक‚देश‚व‚र्त्ती ध‚र्मः प्र‚क‚र‚ण‚स‚मः । त‚द्य‚था पृथिव्य‚प्तेजोवा ‚{\tiny $_{5}$}‚ ‚{\tiny $_{lb}$}‚ य्वाकाशान्य‚नित्यानि स‚त्ताव‚त्वादिति । अत्रापि य‚द्य‚क्ष \quotelemma{पाद} म‚तानुसारी ताव‚देवं ‚{\tiny $_{lb}$}‚प्र‚माण‚य‚ति । प‚र‚माणुः प‚र‚माण्व‚न्त‚र‚पूर्व्व‚को रूपादिम‚त्वाद् द्व्य‚णुकादिव‚दि ‚{\tiny $_{6}$}‚ ति त‚दा ‚{\tiny $_{lb}$}‚त‚स्याभ्युपेत‚विरोध इति नाय‚म‚तीत‚कालाद् भिद्य‚ते । अथ \quotelemma{बौद्धः} क‚रोति । त‚दापि ‚{\tiny $_{lb}$}‚हेतोर‚सिद्धिः प‚र‚माणूनां रूपादिव्य‚तिरेकेणान‚भ्युप‚ग‚मात् । अयोगाच्च ‚{\tiny $_{7}$}‚ द्व्य‚दीनाञ्चा‚{\tiny $_{lb}$}‚भावादुभ‚य‚विक‚लो दृष्टान्तः । य‚दाप्य‚न‚पेक्षित‚सिद्धान्तो न्याय‚वादी ब्रूते त‚दापि ‚{\tiny $_{lb}$}‚ \leavevmode\ledsidenote{\textenglish{140/s}} द्वितीय‚प‚क्षोदित‚दोषानिवृत्तिरिति नाय‚म‚सिद्धाद्व्याव‚र्त्त‚ते । योऽ ‚{\tiny $_{8}$}‚ प्य‚नुमेयैक‚देश‚व‚र्त्ती ‚{\tiny $_{lb}$}‚ध‚र्मः पृथिव्यादीन्य‚नित्यानि ग‚न्ध‚व‚त्वादिति अय‚म‚प्य‚प‚सिद्धान्तान्त‚र्भूत एवेति न पृथ‚ग्वा‚{\tiny $_{lb}$}‚च्यः । नासिद्धः प‚क्षैक‚देश‚ध‚र्म‚त्वात् । स‚प‚क्षैक‚देश‚व‚र्त्तिव ‚{\tiny $_{9}$}‚ \leavevmode\ledsidenote{\textenglish{87b/msK}} दिति चेत् विष‚मोय‚मुप‚न्यासः । ‚{\tiny $_{lb}$}‚स‚प‚क्ष एव च स‚त्व‚मित्य‚त्र हि स‚मुच्चीय‚मानाव‚धार‚ण‚मेव न स‚क‚ल‚स‚प‚क्ष‚ध‚र्म‚तां ‚{\tiny $_{lb}$}‚साध‚न‚स्य प्र‚तिपाद‚य‚ति । अनुमेये स‚त्व‚व‚च‚नं पुन‚र‚योग‚व्य ‚{\tiny $_{1}$}‚ व‚च्छेदेन निय‚न्तृभूत‚म‚{\tiny $_{lb}$}‚शेष‚साध्य‚ध‚र्मिध‚र्म‚तायाः प्र‚तिपाद‚क‚मित्य‚नेनैव प‚क्षैक‚देशासिद्ध‚भेदानाम‚पोहः कृत ‚{\tiny $_{lb}$}‚इत्य‚पार्थ‚कं य‚त्नान्त‚र‚मिति य‚त्किञ्चिदेत‚त् ॥ ० ॥
	{\color{gray}{\rmlatinfont\textsuperscript{§~\theparCount}}}
	\pend% ending standard par
      ‚{\tiny $_{lb}$}‚

	  
	  \pstart \leavevmode% starting standard par
	\hphantom{.}\quotelemma{भावि ‚{\tiny $_{2}$}‚ विक्तो} प्य‚त्रैव ख‚र‚र‚वे प‚तितः । प्र‚क‚र‚ण‚स‚म‚म‚न्य‚था स‚म‚र्थ‚य‚ति । य‚स्मा‚{\tiny $_{lb}$}‚द्धेतो[ः] प्र‚क‚र‚ण‚चिन्ता विप‚क्ष‚स्यापि विचारः प‚श्चाद् भ‚व‚ति स एवं ल‚क्ष‚णो ‚{\tiny $_{lb}$}‚हेतुनिर्ण‚याय योप‚दिश्य‚मानः प्र ‚{\tiny $_{3}$}‚ क‚र‚ण‚स‚मो भ‚व‚ति । प्र‚क‚र‚णे साध्ये स‚म‚स्तुल्यः ‚{\tiny $_{lb}$}‚स‚त्त्वे ऽस‚त्त्वे वा य‚था स‚त्स‚र्व‚ज्ञ‚मित‚र‚त‚द्विप‚रीत‚विनिर्मुक्त‚त्वाद् रूपादिव‚दिति । ‚{\tiny $_{lb}$}‚य‚स्माद‚यं हेतुरुभ‚य‚त्र स‚मानो योप्य‚स ‚{\tiny $_{4}$}‚ त्वं साध‚य‚ति त‚स्यापि स‚मानः । क‚थ‚म‚स‚त्स‚{\tiny $_{lb}$}‚र्व‚ज्ञ‚त्व‚मित‚र‚त‚द्विप‚रीत‚विनिर्मुक्त‚त्वात् ख‚र‚विषाण‚व‚दिति । न चायं किलोभ‚य‚ध‚र्म‚{\tiny $_{lb}$}‚त्वेप्य‚नैकान्तिको विप‚क्ष‚वृत्तिर्वैक‚ल्या ‚{\tiny $_{5}$}‚ त् । त‚दिद‚माचार्येण स्व‚यं \quotelemma{प्र‚माण‚विनिश्च‚ये} \edtext{\textsuperscript{*}}{\lemma{*}\Bfootnote{आचार्य‚ध‚र्म‚कीर्तिप्र‚णीतेषु स‚प्त‚सु न्याय‚प्र‚ब‚न्धेष्व‚न्य‚त‚मो ग्र‚थः\begin{english}\textenglish{See →} bStan-ḥ gyur, mdo.XCV. 11\end{english}}} ‚{\tiny $_{lb}$}‚ प्र‚तिसि\edtext{}{\lemma{तिसि}\Bfootnote{? षि}}द्धं । क‚म्पुन‚र‚त्र भ‚वान् विप‚क्षं प्र‚त्येति साध्याभावं । क‚थ‚मिदानीं ‚{\tiny $_{lb}$}‚हेतुं विप‚क्ष‚वृत्तिरुभ‚य‚ध‚र्मेणैवेत्यादिना । अर्थ‚ग्र‚ह‚ण ‚{\tiny $_{6}$}‚ व्याख्याने च य‚दुक्तं ‚{\tiny $_{lb}$}‚त‚द‚त्रापि व‚क्त‚व्यं । विशेषेण दूष‚ण‚ञ्चास्य प्र‚प‚ञ्चेनोक्त‚मेव । एवं प्र‚क‚र‚ण‚स‚मातीत‚{\tiny $_{lb}$}‚काल‚योर‚नुप‚प‚त्तिः । साध्य‚व्य‚भिचार‚स्य तु युज्य‚ते हेत्वाभास‚त्वं ‚{\tiny $_{7}$}‚ न तु य‚था भ‚व‚ता‚{\tiny $_{lb}$}‚म‚भ्युप‚ग‚मः । त‚था हि भ‚व‚न्तः स‚न्दिग्ध‚विप‚क्ष‚व्यावृत्तिक‚स्यानैकान्तिक‚त्वं न प्र‚ति‚{\tiny $_{lb}$}‚प‚द्य‚न्ते । अद‚र्श‚न‚मात्रेणैव व्य‚तिरेक‚सिद्ध्य‚भ्युप‚ग‚मात् । अत एव च भ‚व‚द्भि ‚{\tiny $_{8}$}‚ र‚प‚{\tiny $_{lb}$}‚ग‚त‚स\edtext{}{\lemma{स}\Bfootnote{? श}}ङ्कैरेवं प्र‚युज्य‚ते प्राप्य‚कारिणी च‚क्षुःश्रोत्रे बाह्येन्द्रिय‚त्वात् घ्राणादिव‚त् । ‚{\tiny $_{lb}$}‚स‚विक‚ल्पं प्र‚त्य‚क्षं प्र‚माण‚त्वाद‚नुमान‚व‚दित्यादि । न चाद‚र्श‚न‚मात्रेणैव विना प्र‚ति‚{\tiny $_{lb}$}‚ब‚न्धे ‚{\tiny $_{9}$}‚ \leavevmode\ledsidenote{\textenglish{88a/msK}} न व्य‚तिरेक‚सिद्धिरिति प्र‚तानित‚म‚न्य‚त्र । ‚{\tiny $_{lb}$}‚ 
	    \pend% close preceding par
	  
	    
	    \stanza[\smallbreak]
	  \flagstanza{\tiny\textenglish{...45}}{\normalfontlatin\large ``\qquad}आत्म‚मृच्चेत‚नादीनां यो भाव‚स्याप्र‚साध‚कः ।&‚{\tiny $_{lb}$}‚स एवानुप‚ल‚म्भः किं हेत्व‚भाव‚स्य साध‚क [४५]{\normalfontlatin\large\qquad{}"}\&[\smallbreak]
	  
	  
	  
	    \pstart  \leavevmode% new par for following
	    \hphantom{.}
	   इत्यादिना । ‚{\tiny $_{lb}$}‚त‚था स‚प‚क्ष‚विप‚क्ष‚योः स‚न्दि ‚{\tiny $_{1}$}‚ ग्धः स‚द‚स‚त्व‚स्यापि स‚म्य‚ग्ज्ञान‚हेतुत्व‚मेव युष्माभि‚{\tiny $_{lb}$}‚रिष्य‚ते नानैकान्तिक‚त्वं य‚था सात्म‚कं जीव‚च्छ‚रीर‚म्प्राणादिम‚त्त्वादिति । अस्य च ‚{\tiny $_{lb}$}‚त‚द्भावः प्र‚तिपादितः \quotelemma{प्र‚माण‚विनिश्च‚या ‚{\tiny $_{2}$}‚} दौ विरुद्ध‚प्र‚भेद‚स्तु \quotelemma{भार‚द्वाज} विहितः प्र‚ति‚{\tiny $_{lb}$}‚ \leavevmode\ledsidenote{\textenglish{141/s}} ज्ञाविरोध‚प्र‚स्ताव एव निराकृतः । साध्य‚स‚मेपि योय‚म‚न्य‚थासिद्धो व‚र्ण्य‚ते य‚था‚{\tiny $_{lb}$}‚ऽनित्याः प‚र‚माण‚वः क्रियाव‚त्वाद् घ‚टादिव‚दि ‚{\tiny $_{3}$}‚ ति अय‚म‚पि किल साध्य‚स‚मो य‚स्मात् ‚{\tiny $_{lb}$}‚मूर्तिक्रियारूपादिम‚त्वाद‚णूनां क्रियाव‚त्वं नानित्य‚त्वादिति । स नोप‚प‚द्य‚ते । ध‚र्मिणि ‚{\tiny $_{lb}$}‚सिद्ध‚त्वान्न‚हि ध‚र्मिणि विद्य‚मान एवासिद्धो ‚{\tiny $_{4}$}‚ हेतुर्युज्य‚ते । स‚र्व‚हेतूनाम‚सिद्ध‚ताप्र‚स‚ङ्‚{\tiny $_{lb}$}‚गात् । नैत‚देव‚न्न‚हि प‚क्षेस्तीत्येताव‚ता प‚क्ष‚ध‚र्म‚त्वं । साध्य‚व‚शेन हि ध‚र्मिणः प‚क्ष‚ध‚र्म‚त्व‚{\tiny $_{lb}$}‚मिष्य‚ते । केव‚ल‚स्य साध्य‚त्वा ‚{\tiny $_{5}$}‚ त् [।] न च साध्यो ध‚र्मो य‚दि ध‚र्मिणि तेन साध‚नेन ‚{\tiny $_{lb}$}‚विना न स‚म्भ‚व‚ति त‚स्य च साध‚न‚स्य साध्य‚ध‚र्माभावे ध‚र्मिणा स‚म्भ‚व‚स्त‚तो हेतोः ‚{\tiny $_{lb}$}‚प‚क्ष‚ध‚र्म‚त्वं । य‚दा पुन‚र‚न्य‚थापि ‚{\tiny $_{6}$}‚ साध‚नायोपात्ते ध‚र्मिणि ध‚र्म उप‚प‚द्य‚ते त‚दा हेतुत्त्व‚मे‚{\tiny $_{lb}$}‚व‚ञ्च विशिष्ट‚मेव स‚त्वं प‚क्ष‚ध‚र्म‚त्वेन विव‚क्षितं । अन्य‚थासिद्ध‚त्वं युक्त‚मेव साध्य‚स‚मं । ‚{\tiny $_{lb}$}‚त‚दिद‚म‚त्र प्र‚तिविधानं ‚{\tiny $_{7}$}‚ य‚दि ख‚लु साध्य‚ध‚र्माभावे ध‚र्मिणि अस‚म्भ‚वो हेतोरेवं विध‚{\tiny $_{lb}$}‚मेव स‚त्वं प‚क्ष‚ध‚र्म‚त्वेन विव‚क्षितं न तु भाव‚मात्रं त‚दा किन्त‚दित्त्थंभूतं प‚क्ष‚ध‚र्म‚त्व‚म‚{\tiny $_{lb}$}‚विज्ञात‚मेवानुमेय‚प्र‚काश ‚{\tiny $_{8}$}‚ क‚माहोस्वित् प‚रिनिश्चित‚मेवेति प्र‚कार‚द्व‚ये य‚द्य‚विज्ञातं ‚{\tiny $_{lb}$}‚प्र‚काश‚कं त‚दाज्ञाप‚क‚हेतुन्याय‚म‚तिव‚र्त्त‚ते [।] ज्ञाप‚को हि हेतुः स्वात्म‚नि ज्ञानापेक्षो ‚{\tiny $_{lb}$}‚ज्ञाप्य‚म‚र्थं प्र‚काश‚य‚ति । स‚त्ता ‚{\tiny $_{9}$}‚ \leavevmode\ledsidenote{\textenglish{88b/msK}} मात्रेण च हेत‚वो विप्र‚तिप‚त्तिनिराकार‚ण‚प‚ट‚वः स‚न्तीति ‚{\tiny $_{lb}$}‚प्र‚तिवादिनां प‚र‚स्प‚र‚प‚राह‚तं प्र‚व‚च‚न‚नानात्वं न भ‚वेत् । विज्ञात‚स्यापि ग‚म‚क‚त्वे ‚{\tiny $_{lb}$}‚प्र‚माणाद्वा त‚स्य प‚रिनिश्च‚यः प्र ‚{\tiny $_{1}$}‚ माणाद्वा । न ताव‚द‚प्र‚माण‚स्य भूतार्थ‚निश्च‚य‚हेतुत्वा‚{\tiny $_{lb}$}‚भावाद‚प्र‚माणाद् ग‚तिर‚न्य‚था प्रामाण्य‚मेवाव‚हीय‚ते । य‚स्मादिद‚मेव प्र‚माण‚स्य प्र‚मा‚{\tiny $_{lb}$}‚ण‚त्वं य‚द्य‚थाव‚स्थित‚व‚स्तुप्र‚का ‚{\tiny $_{2}$}‚ श‚क‚त्वं । त‚च्चेद‚म‚प्र‚माण‚स्याप्य‚स्ति त‚दा क‚थं त‚द ‚{\tiny $_{lb}$}‚प्र‚माणात्साध्ये ध‚र्मिणि विना साध्य‚ध‚र्मेणास‚द्भूष्णोर्हेतोः स‚त्त्व‚म्प‚क्ष‚ध‚र्म‚त्वेनाधि‚{\tiny $_{lb}$}‚ग‚म्य‚ते ‚{\tiny $_{3}$}‚ त‚दापि य‚त एव प्र‚माणाद्धेतोः सिद्धिस्त‚त एव साध्य‚ध‚र्म‚स्यापीयं जाय‚त इति ‚{\tiny $_{lb}$}‚किम‚र्थ‚म‚य‚म‚किञ्चित्क‚रो हेतुरूपादीय‚ते । न च हेतोरेव केव‚ल‚स्य त‚तः सिद्धि ‚{\tiny $_{4}$}‚ [ः] ‚{\tiny $_{lb}$}‚साध्य‚ध‚र्म‚स्य तु नेति युक्त‚म्व‚क्तुं । हेतोर‚पि त‚तोऽसिद्धिप्र‚स‚ङ्गात् त‚था ह्येव‚म‚यं ‚{\tiny $_{lb}$}‚हेतु[ः] त‚त्र ध‚र्मिणि सिध्य‚ति य‚द्य‚नेन साध्य‚ध‚र्मेण विनेह नोप‚प‚द्य‚ते स‚हैव तू ‚{\tiny $_{5}$}‚ प‚{\tiny $_{lb}$}‚प‚द्य‚त इति सिध्येत् । त‚था च क‚थ‚न्त‚त एव प्र‚माणात्साध्य‚ध‚र्म‚स्यापि न सिद्धिः ‚{\tiny $_{lb}$}‚स‚ञ्जातेति चिन्त‚नीय‚मेत‚त् । एव‚ञ्च तेनैव प्र‚माणेन स‚हास्य साध्य‚ध‚र्म‚स्य ‚{\tiny $_{6}$}‚ । ग‚म्य‚{\tiny $_{lb}$}‚ग‚म‚क‚भावो न त्व‚नेन हेतुनेति म‚ह‚द‚निष्ट‚माप‚द्य‚ते । एव‚म्विध‚प‚क्ष‚ध‚र्म‚त्व‚स‚माश्र‚य‚णे ‚{\tiny $_{lb}$}‚च याव‚त् साध्य‚स्यासिद्धिस्ताव‚द्धेतोर‚पि याव‚च्च हेतोर‚सिद्धिस्ताव‚त्सा ‚{\tiny $_{7}$}‚ ध्य‚स्यापीति ‚{\tiny $_{lb}$}‚प‚र‚स्प‚राश्र‚य‚प्र‚स‚ङ्गः । प‚क्ष‚ध‚र्म‚त्व‚निश्च‚य‚वेलायाञ्च साध्य‚ध‚र्म‚सिद्धिः स‚म्प‚द्य‚त इति ‚{\tiny $_{lb}$}‚व्य‚र्थ‚मुत्त‚र‚लिङ्ग‚रूपानुस‚र‚ण‚मित्त्थ‚ञ्च न द्विल‚क्ष‚ण‚श्च‚तुर्ल‚क्ष‚णः प‚ञ्च ‚{\tiny $_{8}$}‚ ल‚क्ष‚ण‚श्च ‚{\tiny $_{lb}$}‚हेतुर्व‚क्त‚व्यः । अस्म‚न्म‚ते तु ध‚र्मिणि स‚त्व‚मात्रं विज्ञाय च त‚दुत्त‚र‚काल‚म‚न्व‚य‚व्य‚ति‚{\tiny $_{lb}$}‚रेक‚योर्विज्ञान‚म‚न्व‚य‚व्य‚तिरेकौ वा स‚र्व्वोप‚संहारेण विज्ञाय‚त उत्त‚र‚कालं ‚{\tiny $_{9}$}‚ \leavevmode\ledsidenote{\textenglish{89a/msK}} ध‚र्मिणि ‚{\tiny $_{lb}$}‚स‚त्व‚मात्रं विज्ञात‚म‚त‚श्चान‚न्त‚र्येणैव त‚त्साम‚र्थ्यात्साध्य‚ध‚र्म‚स्य त‚त्र प्र‚तीतिरुप‚{\tiny $_{lb}$}‚ \leavevmode\ledsidenote{\textenglish{142/s}} प‚द्य‚ते [।] तेनेद‚म‚त्र स‚क‚लं दोष‚जालं न‚भ‚सीवाम‚ले ज‚ले नाव‚स्थान‚म‚लं ल‚भ‚त ‚{\tiny $_{lb}$}‚इत्य‚ल‚म‚प्र‚तिष्ठित ‚{\tiny $_{1}$}‚ मिथ्याप्र‚लापैरिति विर‚म्य‚ते ।
	{\color{gray}{\rmlatinfont\textsuperscript{§~\theparCount}}}
	\pend% ending standard par
      ‚{\tiny $_{lb}$}‚

	  
	  \pstart \leavevmode% starting standard par
	\hphantom{.}य‚द्येवं किं पुन‚र‚त्रेष्ट‚मिष्ट‚मित्याह । \quotelemma{हेत्वाभासास्तु य‚थान्याय‚मित्यादि} \cite[2a7]{vn-msN} ‚{\tiny $_{lb}$}‚ये न्याया हेत्वाभासास्त‚दुक्तिर्न्निग्र‚ह‚स्थान‚म्भ‚व‚ति । ते च येस्माभिरुक्ताः ॥ ‚{\tiny $_{lb}$}‚ 
	    \pend% close preceding par
	  
	    
	    \stanza[\smallbreak]
	  \flagstanza{\tiny\textenglish{...46}}{\normalfontlatin\large ``\qquad}एकाप्र‚सिद्धिसंदेहेऽप्र‚सिद्ध‚व्य‚भिचार‚भाक् ।&‚{\tiny $_{lb}$}‚द्व‚योर्व्विरुद्धोसिद्धौ च संदेह‚व्य‚भिचार‚भागि [४६]{\normalfontlatin\large\qquad{}"}\&[\smallbreak]
	  
	  
	  
	    \pstart  \leavevmode% new par for following
	    \hphantom{.}
	   ति ।
	{\color{gray}{\rmlatinfont\textsuperscript{§~\theparCount}}}
	\pend% ending standard par
      ‚{\tiny $_{lb}$}‚

	  
	  \pstart \leavevmode% starting standard par
	\hphantom{.}न‚नु चायं वाद‚न्याय‚मार्गः स‚क‚ल‚लोकानिब‚न्ध‚न‚ब‚न्धुना \quotelemma{वाद‚विधानादावार्य ‚{\tiny $_{3}$}‚ ‚{\tiny $_{lb}$}‚व‚सुब‚न्धुना} म‚हाराज‚प‚थीकृतः [।] क्षुण्ण‚श्च त‚द‚नु म‚ह‚त्यां \quotelemma{न्याय‚प‚रीक्षा} यां कुम‚ति‚{\tiny $_{lb}$}‚म‚त‚म‚न्त\edtext{}{\lemma{न्त}\Bfootnote{? म‚त्त}}मात‚ङ्ग‚शिरःपीठ‚पाट‚न‚प‚टुभिराचार्य \quotelemma{दिग्नाग} पादैस्त‚त्किमिदं पुन‚श्च‚{\tiny $_{lb}$}‚र्व्वित‚च ‚{\tiny $_{4}$}‚ र्व्व‚ण‚मास्थितं त्व‚येति । एत‚च्चोद्य‚प‚रिहार‚प‚र‚मिमं श्लोक‚मुप‚न्य‚स्य‚तिप‚न्य‚स्य‚ते ‚{\tiny $_{lb}$}‚ \quotelemma{लोक} \cite[20a8]{vn-msN} इत्यादि । तिमिर‚ञ्च प‚ट‚ल‚ञ्चेति तिमिर‚प‚ट‚लं अविद्यैव तिमिर‚प‚ट‚ल‚म‚{\tiny $_{lb}$}‚विद्यातिमिर ‚{\tiny $_{5}$}‚ प‚ट‚लं भूतार्थ‚द‚र्श‚न‚विब‚न्ध‚क‚त्वात् । त‚स्योल्लेख‚नो वाद‚न्याय इति ‚{\tiny $_{lb}$}‚स‚म्ब‚न्धः [।] उल्लेख‚न‚श‚ब्दः क‚र्तृसाध‚नः । क‚स्य पुन‚र‚विद्यातिमिर‚प‚ट‚ल‚मित्याह । त‚त्त्व ‚{\tiny $_{lb}$}‚दृष्टेस्त ‚{\tiny $_{6}$}‚ त्व‚द‚र्श‚न‚स्य । प्र‚ज्ञालोच‚न‚स्येत्य‚र्थः । य एष स‚म‚न‚न्त‚र‚मावेदितो वाद‚न्यायः । ‚{\tiny $_{lb}$}‚ \quotelemma{स‚द्} भिः पूर्वाचार्यैः प‚र‚हित‚र‚तैः क‚रुणापार‚त‚न्त्र्याल्लोकान् स‚म्य‚ग्व‚र्त्त्म‚नि व्य‚व‚स्थाप‚{\tiny $_{lb}$}‚यितुं प्र ‚{\tiny $_{7}$}‚ णीतः प‚रां प्र‚सिद्धिं नीतो लोके सुष्टु स्फुटीकृत इत्य‚र्थः । न तु प‚र‚स्प‚र्द्ध‚या ‚{\tiny $_{lb}$}‚नापि य‚शःकाम‚तादिभिः ।त\edtext{}{\lemma{त}\Bfootnote{? य}}द्येव‚न्त‚र्हि त‚द‚व‚स्थितं चोद्य‚मित्य‚त आह । ‚{\tiny $_{lb}$}‚त‚त्व‚स्यालोक‚मुद्योत‚म्वा ‚{\tiny $_{8}$}‚ द‚न्याय‚माचाचार्यैरूप‚दिष्टं [।] तिमिर‚य‚त्य‚न्ध‚कारीक‚रोति ‚{\tiny $_{lb}$}‚कुदूष‚ण‚त‚म‚सा प्र‚च्छाद‚य‚तीति याव‚त् । कः पुन‚र‚साव‚तिसाह‚सिको यो म‚हानागैः क्षुण्णं ‚{\tiny $_{lb}$}‚प‚न्थानं रोद्धुमीह‚त इत्याह ‚{\tiny $_{8}$}‚ \leavevmode\ledsidenote{\textenglish{89b/msK}} \quotelemma{दुर्विद‚ग्धः} स‚म्य‚ग्विवेक‚र‚हित‚त‚या ज‚नोय \quotelemma{मुद्योत‚क‚र‚{\tiny $_{lb}$}‚प्रीतिच‚न्द्र\edtext{}{\lemma{न्द्र}\Bfootnote{?}}भाविविक्त} प्र‚भृतिः । य‚त‚श्च एवं त‚स्माद्य‚त्नः कृत इह \quotelemma{वाद‚न्याय} प्र‚क‚र‚णे ‚{\tiny $_{lb}$}‚म‚या त‚स्य म‚ह‚द्भिरुद्भावित‚स्या ‚{\tiny $_{1}$}‚ न्त‚राज‚तैर‚व‚धूत‚स्य स‚मुज्वाल‚नाय । कुदूष‚ण‚प‚रि‚{\tiny $_{lb}$}‚हारेण त‚न्नीत्युद्योत‚नेन म‚म व्यापृत‚त्वान्न म‚या पिष्टं पिष्ट‚मिति संक्षेपार्थः ॥
	{\color{gray}{\rmlatinfont\textsuperscript{§~\theparCount}}}
	\pend% ending standard par
      ‚{\tiny $_{lb}$}‚\textsuperscript{\textenglish{143/s}}
	  \bigskip
	  \begingroup
	
	    
	    \stanza[\smallbreak]
	  \flagstanza{\tiny\textenglish{...47}}{\normalfontlatin\large ``\qquad}अन‚र्घ \edtext{}{\lemma{र्घ}\Bfootnote{?}} व‚निताव‚गाह‚न‚म‚न‚ल्प‚धीश‚क्तिना &‚{\tiny $_{lb}$}‚प्य‚दृष्ट‚प‚र‚मार्थ‚सार‚म‚धिकाभियोगैर‚पि ।&‚{\tiny $_{lb}$}‚म‚तं म‚तित‚मः स्फुट‚म्प्र‚तिविभ‚ज्य स‚म्य‚ग्म‚या&‚{\tiny $_{lb}$}‚य‚दाप्त‚म‚कृशं शुभ‚म्भ‚व‚तु तेन शान्तो ज‚नः ॥ [४७]{\normalfontlatin\large\qquad{}"}\&[\smallbreak]
	  
	  
	  
	  \endgroup
	‚{\tiny $_{lb}$}‚

	  
	  \pstart \leavevmode% starting standard par
	अह‚ञ्च
	{\color{gray}{\rmlatinfont\textsuperscript{§~\theparCount}}}
	\pend% ending standard par
      ‚{\tiny $_{lb}$}‚
	  \bigskip
	  \begingroup
	
	    
	    \stanza[\smallbreak]
	  \flagstanza{\tiny\textenglish{...48}}{\normalfontlatin\large ``\qquad}नैरात्म्य‚बोध‚प‚रिपाटि ‚{\tiny $_{3}$}‚ त‚दोष‚शैल‚स‚म्बुद्ध‚भार‚व‚ह‚न‚क्ष‚म‚भूरिश‚क्त-&‚{\tiny $_{lb}$}‚म‚ञ्जुश्रियः श्रिय‚म‚वाप्य स‚म‚स्त‚स‚त्त्व‚स‚र्वावृतिक्ष‚य‚विधान‚प‚टुर्भ‚वेयं ।&‚{\tiny $_{lb}$}‚म‚हार‚येनैव न किञ्चिद‚त्र त्य‚क्त‚म्वि ‚{\tiny $_{4}$}‚ विक्तं न विभ‚ज्य‚मेव ।&‚{\tiny $_{lb}$}‚त‚थापि म‚न्द‚प्र‚तिबोध‚नार्थ‚मालोक एष ज्व‚लितः प्र‚दीपः ॥ [४८]{\normalfontlatin\large\qquad{}"}\&[\smallbreak]
	  
	  
	  
	  \endgroup
	‚{\tiny $_{lb}$}‚
	  \bigskip
	  \begingroup
	
	    
	    \stanza[\smallbreak]
	  \flagstanza{\tiny\textenglish{...49}}{\normalfontlatin\large ``\qquad}लोकेऽविद्यातिमिर‚प‚ट‚लोल्लेख‚न‚स्त‚त्त्व‚दृष्टे&‚{\tiny $_{lb}$}‚वाद‚न्यायः प‚र‚हित‚र‚तैरेष स‚म्य ‚{\tiny $_{5}$}‚ [क्] प्र‚णीतः ।&‚{\tiny $_{lb}$}‚त‚त्त्वालोकं तिमिर‚य‚ति तं दुर्विंद‚ग्धो ज‚नोय‚न्&‚{\tiny $_{lb}$}‚त‚स्माद् य‚त्नः कृत इह म‚या त‚त्स‚मुज्वाल‚नायेति ॥ [४९]{\normalfontlatin\large\qquad{}"}\&[\smallbreak]
	  
	  
	  
	  \endgroup
	‚{\tiny $_{lb}$}‚

	  
	  \pstart \leavevmode% starting standard par
	विप‚ञ्चितार्था नाम वाद‚न्याय‚टीका स‚माप्ता ॥ ॥
	{\color{gray}{\rmlatinfont\textsuperscript{§~\theparCount}}}
	\pend% ending standard par
      ‚{\tiny $_{lb}$}‚

	  
	  \pstart \leavevmode% starting standard par
	कृतिरियंसान्त‚रिक्ष\edtext{}{\lemma{रिक्ष}\Bfootnote{? शान्त‚र‚क्षित}}पादानामिति ॥ 
	{\color{gray}{\rmlatinfont\textsuperscript{§~\theparCount}}}
	\pend% ending standard par
      

	  
	  \pstart \leavevmode% starting standard par
	स‚म्व‚त आचू२ \begin{english}\textit{(272 N.E.–1152 A.D.)}\end{english} श्राव‚ण‚कृष्ण एकाद‚श्यांलिखितं ‚{\tiny $_{lb}$}‚म‚या । राजाधिराज‚प‚र‚मेश्व‚र‚प‚र‚म‚भ‚ट्टार‚कः श्रीम‚दान‚न्द‚दे ‚{\tiny $_{7}$}‚ व‚पादीय‚विज‚य‚राज्ये ‚{\tiny $_{lb}$}‚शुभ‚दिने ॥
	{\color{gray}{\rmlatinfont\textsuperscript{§~\theparCount}}}
	\pend% ending standard par
      ‚{\tiny $_{lb}$}‚
	  \bigskip
	  \begingroup
	
	    
	    \stanza[\smallbreak]
	  \flagstanza{\tiny\textenglish{...50}}{\normalfontlatin\large ``\qquad}ग्र‚न्थ‚स्यास्य प्र‚माण‚ञ्च निपुणैर्न्न‚व‚श‚ताऽधिकं ।&‚{\tiny $_{lb}$}‚स‚ह‚स्र‚द्वित‚यं स‚म्प‚त्\edtext{}{\lemma{त्}\Bfootnote{? म्य‚क्}}संख्यात‚म्पूर्व्व‚शूरिभिः\edtext{}{\lemma{शूरिभिः}\Bfootnote{? सूरिभिः}}॥ ० ॥{\normalfontlatin\large\qquad{}"}\&[\smallbreak]
	  
	  
	  
	  \endgroup
	‚{\tiny $_{lb}$}‚

	  
	  \pstart \leavevmode% starting standard par
	शुभ‚म‚स्तु स‚र्व्व‚ज‚ग‚तां इ ‚{\tiny $_{8}$}‚ ... ‚{\tiny $_{lb}$}‚...स‚र्वैः र‚क्षित‚व्य‚म्प्र‚य‚त्न‚त इति ॥ ‚{\tiny $_{lb}$}‚न‚मः स‚र्व‚ज्ञाय ॥ 
	{\color{gray}{\rmlatinfont\textsuperscript{§~\theparCount}}}
	\pend% ending standard par
      
	    
	    \endnumbering% ending numbering from div
	    
	  % running endDocumentHook
     \backmatter 
	 \chapter{The TEI Header}
	 \begin{minted}[fontfamily=rmfamily,fontsize=\footnotesize,breaklines=true]{xml}
       <teiHeader xmlns="http://www.tei-c.org/ns/1.0" xml:lang="en">
   <fileDesc>
      <titleStmt>
         <title type="main">Vādanyāyaṭīkā Vipañcitārthā</title>
         <author role="commentator">Śāntarakṣita</author>
         <funder>Deutsche Forschungsgemeinschaft</funder>
         <funder>The National Endowment for the Humanities</funder>
         <principal>
	           <persName>Birgit Kellner</persName>
	        </principal>
         <respStmt>
            <resp>data entry by</resp>
            <name key="name aurorachana">Aurorachana, Auroville</name>
         </respStmt>
         <respStmt>
            <resp>prepared for SARIT by</resp>
            <persName key="name person lo">Liudmila Olalde</persName>
         </respStmt>
      </titleStmt>
      <editionStmt>
         <p>
	</p>
      </editionStmt>
      <publicationStmt>
         <publisher>SARIT: Search and Retrieval of Indic Texts. DFG/NEH Project (NEH-No. HG5004113), 2013-2016 </publisher>
         <idno>2014-10-17</idno>
         <availability status="restricted">
            <p>Copyright Notice:</p>
            <p>Copyright 2014-2016 SARIT</p>
            <licence>
	              <p> Distributed under a <ref target="https://creativecommons.org/licenses/by-sa/4.0/">Creative Commons Attribution-ShareAlike 4.0 International licence. </ref> Under this licence, you are free to: </p>
	              <list>
                  <item>Share — copy and redistribute the material in any medium or format. </item>
                  <item>Adapt — remix, transform, and build upon the material for any purpose, even commercially. </item>
               </list>
	              <p>The licensor cannot revoke these freedoms as long as you follow the license terms. </p>
	              <p>Under the following terms:</p>
	              <list>
                  <item>Attribution — You must give appropriate credit, provide a link to the license, and indicate if changes were made. You may do so in any reasonable manner, but not in any way that suggests the licensor endorses you or your use. </item>
                  <item>ShareAlike — If you remix, transform, or build upon the material, you must distribute your contributions under the same license as the original. </item>
               </list>
	              <p>More information and fuller details of this license are given on the Creative Commons website. </p>
	           </licence>
            <p>SARIT assumes no responsibility for unauthorised use that infringes the rights of any copyright owners, known or unknown. </p>
         </availability>
         <date>2014</date>
      </publicationStmt>
      <sourceDesc>
         <bibl xml:id="vnt-sankrtyayana-book">
	           <author>Dharmakīrti</author>
	           <author>Śāntarakṣita</author>
	           <title type="main">Dharmakīrti's Vādanyāya</title>
	           <title type="sub">With the Commentary of Śāntarakṣita</title>
	           <editor key="name person rs">Rāhula Sāṅkṛtyāyana</editor>
	           <publisher>Bihar and Orissa Research Society</publisher>
	           <pubPlace>Patna</pubPlace>
	           <date>1935-1936</date>
	           <note>Appendix to the Journal of the Bihar and Orissa Research Society, vols. 21/4 and 22/1</note>
	           <note>The manuscript consulted by Sāṅkṛtyāyana is described below. </note>
	        </bibl>
         <msDesc xml:id="vn-msK">
            <msIdentifier>
               <idno/>
               <altIdentifier>
                  <idno>Kun-de-ling-Manuscript.</idno>
                  <!-- is there a standard identifier? --></altIdentifier>
            </msIdentifier>
            <msContents>
               <msItem>
                  <author>Śāntarakṣita</author>
                  <title>Vipañcitārthā</title>
               </msItem>
            </msContents>
            <physDesc>
               <objectDesc>
                  <p>Palm-leaf manuscript. 89 leaves in Kuṭilā script. Apparently written in 1152 A.C. </p>
               </objectDesc>
            </physDesc>
            <history>
               <p>In June 1934, Sāṅkṛtyāyana found this manuscript in the monastery of Kun-de-ling (Lhasa). </p>
            </history>
         </msDesc>
         <msDesc xml:id="vn-msN">
            <msIdentifier>
               <idno/>
               <altIdentifier>
                  <idno>Nagor-Manuscript.</idno>
                  <!-- is there a standard identifier? --></altIdentifier>
            </msIdentifier>
            <msContents>
               <msItem>
                  <author>Dharmakīrti</author>
                  <title>Vādanyāya</title>
               </msItem>
            </msContents>
            <physDesc>
               <objectDesc>
                  <p>Palm-leaf manuscript. 20 leaves in Kuṭilā script. Each page contains 9 to 11 lines. 12th century CE. </p>
               </objectDesc>
            </physDesc>
            <history>
               <p>In June 1934, Sāṅkṛtyāyana found this manuscript in the monastery of Nagor, Tibet. </p>
            </history>
         </msDesc>
      </sourceDesc>
   </fileDesc>
   <encodingDesc>
      <p>Line brakes, page breaks and folio numbers: <list>
            <item>The line breaks and page breaks of <ref sameAs="#vadanyayatika-book">Sāṅkṛtyāyana's edition</ref> were given the ed-attribute "s". In the source file, there were two types of line breaks: returns (and possible surrounding space) and hyphens+returns. These were replaced with lb-elements. I didn't check whether the source was consequent in this respect.</item>
            <item>The folio numbers of the Vipañcitārthā manuscript were encoded as pb-elements with the attribute ed="msK", wich refers to the <ref target="#vn-msK">manuscript</ref> used by Sāṅkṛtyāyana. The line numbers of the manuscript were encoded as lb-elements with the attribute ed="msK".</item>
            <item>After word(s) quoted from the Vādanyāya, Sāṅkṛtyāyana added the corresponding folio and line numbers in the <ref target="#vn-msN">Vādanyāya manuscript</ref>. These were encoded as follows: &lt;ref target="#vn-msN" corresp="1b1"/&gt;.</item>
         </list>
      </p>
      <p>Words quoted from the Vādanyāya (pratīkā) (printed in a thiner script in <ref sameAs="#vadanyayatika-book">Sāṅkṛtyāyana's edition</ref>) were encoded as &lt;q type="lemma"&gt;.</p>
      <p>Quotations from other texts (printed in a thiner script in <ref sameAs="#vadanyayatika-book">Sāṅkṛtyāyana's edition</ref>) were encoded as quote-elements. Some quotations are also enclosed in quotation marks; in these cases the attribute "quote" has been added.</p>
      <p>Footnotes were encoded as &lt;note rend="footnote"&gt;</p>
      <p>Round and square brackets were were replaced with the following TEI-elements:
      <list>
            <item>Bracketed references to other works were enclosed in &lt;ref cRef=""&gt;.</item>
            <item>Bracketed text was enclosed in &lt;supplied resp="#rs"&gt;.</item>
            <item>Bracketed text preceded or followed by a question mark was enclosed in &lt;note type="correction" resp="#rs"&gt;. The question mark was kept.</item>
            <item>Bracketed suspension points ("...") were enclosed in &lt;supplied reason="gap" resp="#rs"&gt;.</item>
            <item>Question marks surrounded by brackets were enclosed in: &lt;note type="uncertain" resp="#rs"&gt;.</item>
            <item>All other punctuation marks that were surrounded by brackets were enclosed in &lt;supplied resp="#rs"&gt;.</item>
         </list>
      </p>
      <p>The text is structured in 2 chapters, encoded as: div n=".." type="chapter"<list>
            <item>1. Nigrahasthānalakṣaṇa, pp. 1-73</item>
            <item>2. Nyāyamatakhaṃḍana, pp. 75-143</item>
         </list>
      </p>
      <p>Abbreviations used in the attributes ed, cRef and xml:id's in this file: <!-- this is a provisory list and has to be replaced by a refsDecl -->
      <list ana="abbreviations">
            <item>ak = Abhidharmakośa</item>
            <item>MaBhā = Mahābhāṣya </item>
            <item>MīSū = Mīmāṃsāsūtra </item>
            <item>msK = <ref target="#vn-msK">Kun-de-ling-Manuscript of the Vipañcitārthā</ref>
            </item>
            <item>msN = <ref target="#vn-msN">Nagor-Manuscript of the Vādanyāya</ref>
            </item>
            <item>nb = Nyāyabindu</item>
            <item>nbh = Nyāyabhāṣya </item>
            <item>nv = Nyāyavarttika </item>
            <item>nsū = Nyāyasūtra</item>
            <item>Pā = Pāṇini</item>
            <item>pv = Pramāṇavartika </item>
            <item>s = <ref target="#vadanyayatika-book">Sāṅkṛtyāyana's edition of the Vipāñcitārtha</ref>
            </item>
         </list>
      </p>
   </encodingDesc>
   <profileDesc><!-- ... --></profileDesc>
   <revisionDesc>
      <change who="lo" when="2014-10-29">
	        <list>
            <item>I corrected folio number 46b to 49b on p. 73.</item>
            <item>I added folio number 53b, which was missing in the printed edition.</item>
         </list>
      </change>
      <change who="lo" when="2015-12-30">Added @xml-lang to the front-element. </change>
      <change who="lo" when="2016-04-25">Added @type to note- and add-elements.</change>
      <change who="lo" when="2016-05-19">Removed front.</change>
      <change who="lo" when="2016-05-24">Added content to ref-elements, e.g.: <tag>ref target="#vn-msN" corresp="1b1"</tag>1b1<tag type="end">ref</tag>
      </change>
      <change who="lo" when="2016-05-31">Replaced <tag>add</tag> with <tag>supplied</tag>.</change>
      <change who="lo" when="2016-05-31">Wrapped question marks in <tag>note type="uncertain</tag>.</change>
   </revisionDesc>
</teiHeader>
	 \end{minted}
       
      \clearpage
      \begin{english}
      \printshorthands
      \printbibliography
      \end{english}
    
\end{document}
