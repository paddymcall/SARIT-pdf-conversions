%% require snapshot package to record versions to log files
    \RequirePackage[log]{snapshot}
    \documentclass[article,12pt,a4paper]{memoir}%
    
      %% useful for debugging
      %% \usepackage{syntonly}%
      %%\syntaxonly%
    
	  \usepackage[normalem]{ulem}
	  \usepackage{eulervm}
	  \usepackage{xltxtra}
  \usepackage{polyglossia}
  \PolyglossiaSetup{sanskrit}{
  hyphenmins={2,3},% default is {1,3}
  }
  \setdefaultlanguage{sanskrit}
  % english etc. should also be available, notes and bib
  \setotherlanguages{english,german,italian,french}
  \setotherlanguage[numerals=arabic]{tibetan}
  \usepackage{fontspec}
  %% redefine some chars (either changed by parsing, or not commonly in font)
  \catcode`⃥=\active \def⃥{\textbackslash}
  \catcode`‿=\active \def‿{\textunderscore}
  \catcode`❴=\active \def❴{\{}
  \catcode`❵=\active \def❵{\}}
  \catcode`〔=\active \def〔{{[}}% translate 〔OPENING TORTOISE SHELL BRACKET
  \catcode`〕=\active \def〕{{]}}% translate 〕CLOSING TORTOISE SHELL BRACKET
  \catcode` =\active \def {\,}
  \catcode`·=\active \def·{\textbullet}
  %% BREAK PERMITTED HERE: \discretionary{-}{}{}\nobreak\hspace{0pt}
  \catcode`‚=\active \def‚{\-}
  \catcode`ꣵ=\active \defꣵ{%
  म्\textsuperscript{cb}%for candrabindu
  }
  %% show a lot of tolerance
  \tolerance=9000
  \def\textJapanese{\fontspec{Kochi Mincho}}
  \def\textChinese{\fontspec{HAN NOM A}}
  \def\textKorean{\fontspec{Baekmuk Gulim} }
  % make sure English font is there
  \newfontfamily\englishfont[Mapping=tex-text]{TeX Gyre Schola}
    % set up a devanagari font
  \newfontfamily\devanagarifont[Script=Devanagari,Mapping=devanagarinumerals,AutoFakeBold=1.5,AutoFakeSlant=0.3]{Chandas}
	\newfontfamily\rmlatinfont[Mapping=tex-text]{TeX Gyre Pagella}
	\newfontfamily\tibetanfont[Script=Tibetan,Scale=1.2]{Tibetan Machine Uni}
  \newcommand\bo\tibetanfont
  
    \defaultfontfeatures{Scale=MatchLowercase,Mapping=tex-text}
	\setmainfont{Chandas}
    \setsansfont{TeX Gyre Bonum}
  
  \setmonofont{DejaVu Sans Mono}
	  %% page layout start: fit to a4 and US letterpaper (example in memoir.pdf)
	  %% page layout start
	  % stocksize (actual size of paper in the printer) is a4 as per class
	  % options;
	  
	  % trimming, i.e., which part should be cut out of the stock (this also
	  % sets \paperheight and \paperwidth):
	  % \settrimmedsize{0.9\stockheight}{0.9\stockwidth}{*}
	  % \settrimmedsize{225mm}{150mm}{*}
	  % % say where you want to trim
	  \setlength{\trimtop}{\stockheight}    % \trimtop = \stockheight
	  \addtolength{\trimtop}{-\paperheight} %           - \paperheight
	  \setlength{\trimedge}{\stockwidth}    % \trimedge = \stockwidth
	  \addtolength{\trimedge}{-\paperwidth} %           - \paperwidth
	  % % this makes trims equal on top and bottom (which means you must cut
	  % % twice). if in doubt, cut on top, so that dust won't settle when book
	  % % is in shelf
	  \settrims{0.5\trimtop}{0.5\trimedge}

	  % figure out which font you're using
	  \setxlvchars
	  \setlxvchars
	  % \typeout{LENGTH: lxvchars: \the\lxvchars}
	  % \typeout{LENGTH: xlvchars: \the\xlvchars}

	  % set the size of the text block next:
	  % this sets \textheight and \textwidth (not the whole page including
	  % headers and footers)
	  \settypeblocksize{230mm}{130mm}{*}

	  % left and right margins:
	  % this way spine and edge margins are the same
	  % \setlrmargins{*}{*}{*}
	  \setlrmargins{*}{*}{1.5}

	  % upper and lower, same logic as before
	  % \setulmargins{*}{*}{*}% upper = lower margin
	  % \uppermargin = \topmargin + \headheight + \headsep
	  %\setulmargins{*}{*}{1.5}% 1.5*upper = lower margin
	  \setulmargins{*}{*}{1.5}% 

	  % header and footer spacings
	  \setheadfoot{2\baselineskip}{2\baselineskip}

	  % \setheaderspaces{ headdrop }{ headsep }{ ratio }
	  \setheaderspaces{*}{*}{1.5}

	  % see memman p. 51 for this solution to widows/orphans 
	  \setlength{\topskip}{1.6\topskip}
	  % fix up layout
	  \checkandfixthelayout
	  %% page layout end
	
	  \sloppybottom
	
	    % numbering depth
	    \maxtocdepth{section}
	    % set up layout of toc
	    \setpnumwidth{4em}
	    \setrmarg{5em}
	    \setsecnumdepth{all}
	    \newenvironment{docImprint}{\vskip 6pt}{\ifvmode\par\fi }
	    \newenvironment{docDate}{}{\ifvmode\par\fi }
	    \newenvironment{docAuthor}{\ifvmode\vskip4pt\fontsize{16pt}{18pt}\selectfont\fi\itshape}{\ifvmode\par\fi }
	    % \newenvironment{docTitle}{\vskip6pt\bfseries\fontsize{18pt}{22pt}\selectfont}{\par }
	    \newcommand{\docTitle}[1]{#1}
	    \newenvironment{titlePart}{ }{ }
	    \newenvironment{byline}{\vskip6pt\itshape\fontsize{16pt}{18pt}\selectfont}{\par }
	    % setup title page; see CTAN /info/latex-samples/TitlePages/, and memoir
	  \newcommand*{\plogo}{\fbox{$\mathcal{SARIT}$}}
	  \newcommand*{\makeCustomTitle}{\begin{english}\begingroup% from example titleTH, T&H Typography
	  \thispagestyle{empty}
	  \raggedleft
	  \vspace*{\baselineskip}
	  
	      % author(s)
	    {\Large Śāntarakṣita}\\[0.167\textheight]
	    % maintitle
	    {\Huge Vādanyāyaṭīkā Vipañcitārthā}\\[\baselineskip]
	    {\Large SARIT}\\\vspace*{\baselineskip}\plogo\par
	  \vspace*{3\baselineskip}
	  \endgroup
	  \end{english}}
	  \newcommand{\gap}[1]{}
	  \newcommand{\corr}[1]{($^{x}$#1)}
	  \newcommand{\sic}[1]{($^{!}$#1)}
	  \newcommand{\reg}[1]{#1}
	  \newcommand{\orig}[1]{#1}
	  \newcommand{\abbr}[1]{#1}
	  \newcommand{\expan}[1]{#1}
	  \newcommand{\unclear}[1]{($^{?}$#1)}
	  \newcommand{\add}[1]{($^{+}$#1)}
	  \newcommand{\deletion}[1]{($^{-}$#1)}
	  \newcommand{\quotelemma}[1]{\textcolor{cyan}{#1}}
	  \newcommand{\name}[1]{#1}
	  \newcommand{\persName}[1]{#1}
	  \newcommand{\placeName}[1]{#1}
	  % running latexPackages template
     \usepackage[x11names]{xcolor}
     \definecolor{shadecolor}{gray}{0.95}
     \usepackage{longtable}
     \usepackage{ctable}
     \usepackage{rotating}
     \usepackage{lscape}
     \usepackage{ragged2e}
     
	 \usepackage{titling}
	 \usepackage{marginnote}
	 \renewcommand*{\marginfont}{\color{black}\rmlatinfont\scriptsize}
	 \setlength\marginparwidth{.75in}
	 \usepackage{graphicx}
	 \graphicspath{{images/}}
	 \usepackage{csquotes}
       
	 \def\Gin@extensions{.pdf,.png,.jpg,.mps,.tif}
       
      \usepackage[noend,series={A,B}]{reledmac}
       % simplify what ledmac does with fonts, because it breaks. From the documentation of ledmac:
       % The notes are actually given seven parameters: the page, line, and sub-line num-
       % ber for the start of the lemma; the same three numbers for the end of the lemma;
       % and the font specifier for the lemma. 
       \makeatletter
       \def\select@lemmafont#1|#2|#3|#4|#5|#6|#7|%
       {}
       \makeatother
       \AtEveryPstart{\refstepcounter{parCount}}
       \setlength{\stanzaindentbase}{20pt}
     \setstanzaindents{3,2,2,2,2,2,2,2,2,}
     % \setstanzapenalties{1,5000,10500}
     \lineation{page}
     % \linenummargin{inner}
     \linenumberstyle{arabic}
     \firstlinenum{5}
    \linenumincrement{5}
    \renewcommand*{\numlabfont}{\normalfont\scriptsize\color{black}}
    \addtolength{\skip\Afootins}{1.5mm}
    \Xnotenumfont{\bfseries\footnotesize}
    \sidenotemargin{outer}
    \linenummargin{inner}
    \Xarrangement{twocol}
    \arrangementX{twocol}
    %% biblatex stuff start
	 \usepackage[backend=biber,%
	 citestyle=authoryear,%
	 bibstyle=authoryear,%
	 language=english,%
	 sortlocale=en_US,%
	 ]{biblatex}
	 
		 \addbibresource{/home/beta/webstuff/indology.info/SARIT-pdf-conversions/Stylesheets/profiles/sarit/latex/bib/sarit.bib}
	 \renewcommand*{\citesetup}{%
	 \rmlatinfont
	 \biburlsetup
	 \frenchspacing}
	 \renewcommand{\bibfont}{\rmlatinfont}
	 \DeclareFieldFormat{postnote}{:#1}
	 \renewcommand{\postnotedelim}{}
	 %% biblatex stuff end
	 
	 \setcounter{errorcontextlines}{400}
       
	 \usepackage{lscape}
	 \usepackage{minted}
       
	   % pagestyles
	   \pagestyle{ruled}
	   
	   \makeoddfoot{ruled}{{\tiny\rmlatinfont \textit{Compiled: \today}}}{%
	   {\tiny\rmlatinfont \textit{Revision: \href{https://github.com/paddymcall/SARIT-pdf-conversions/commit/HEAD}{HEAD}}}%
	   }{\rmlatinfont\thepage}
	   \makeevenfoot{ruled}{\rmlatinfont\thepage}{%
	   {\tiny\rmlatinfont \textit{Revision: \href{https://github.com/paddymcall/SARIT-pdf-conversions/commit/HEAD}{HEAD}}}%
	   }{{\tiny\rmlatinfont \textit{Compiled: \today}}}
	   
	 
	   \usepackage{perpage}
           \MakePerPage{footnote}
	 
       \usepackage[destlabel=true,% use labels as destination names; ; see dvipdfmx.cfg, option 0x0010, if using xelatex
       pdftitle={Vādanyāyaṭīkā Vipañcitārthā // Śāntarakṣita},
       pdfauthor={SARIT: Search and Retrieval of Indic Texts. DFG/NEH Project (NEH-No. HG5004113), 2013-2016 },
       unicode=true]{hyperref}
       
       \renewcommand\UrlFont{\rmlatinfont}
       \newcounter{parCount}
       \setcounter{parCount}{0}
       % cleveref should come last; note: also consider zref, this could become more useful than cleveref?
       \usepackage[english]{cleveref}% clashes with eledmac < 1.10.1 standard
       \crefname{parCount}{§}{§§}
     
\begin{document}
    
     \makeCustomTitle
     \let\tabcellsep&
	\frontmatter
	\tableofcontents
	% \listoffigures
	% \listoftables
	\cleardoublepage
         \mainmatter 
	  
	% new div opening: depth here is 0
	
	    
	    \beginnumbering% beginning numbering from div depth=0
	    
	  
\chapter[{१. निग्रहस्थानलक्षणम्}][{१. निग्रहस्थानलक्षणम्}]{१. निग्रहस्थानलक्षणम्}\textsuperscript{\textenglish{1/s}}

	  
	  \pstart \leavevmode% starting standard par
	नमो विघ्घ्नप्रमथनाय ॥ \leavevmode\ledsidenote{\textenglish{1b/msK}}
	{\color{gray}{\rmlatinfont\textsuperscript{§~\theparCount}}}
	\pend% ending standard par
      ‚{\tiny $_{lb}$}‚
	  \bigskip
	  \begingroup
	
	    
	    \stanza[\smallbreak]
	  \flagstanza{\tiny\textenglish{...1}}{\normalfontlatin\large ``\qquad}नानासद्गुणरत्नराशिकिरणध्वस्तान्धकारस्सदा \add{,}&‚{\tiny $_{lb}$}‚यो नानाविधसत्त्ववांछितफलप्राप्त्यर्थमत्त्युद्यतः ।&‚{\tiny $_{lb}$}‚तन्निःशेषजगद्धितोदयपरन्नत्वार्यमञ्जुश्रियं । &‚{\tiny $_{lb}$}‚वादन्यायविभाग एष विमलः सङ्क्षिप्त आरभ्यते ॥ \add{१}{\normalfontlatin\large\qquad{}"}\&[\smallbreak]
	  
	  
	  
	  \endgroup
	‚{\tiny $_{lb}$}‚

	  
	  \pstart \leavevmode% starting standard par
	यत्प्रयोजनरहितं तत्प्रेक्षापूर्व्वकारिभिर्न्नारभ्यते । यथा बलित्वग्दर्शन‚{\tiny $_{lb}$}‚विनिश्चयादिकं । अप्रयोजनञ्चेदं प्रकरण ‚{\tiny $_{2}$}‚ मित्याशंकावतस्तदाशंकापरिजिहीर्षया ‚{\tiny $_{lb}$}‚प्रयोजनप्रदर्शनाय ‚{\color{DodgerBlue3}‚न्यायवादिन} ‚{\tiny $_{1b1}$}‚ मित्यादिवाक्यमुपन्यस्तवान् । कथम्पु‚{\tiny $_{lb}$}‚नरनेन वाक्येनास्य प्रयोजनमुपदर्श्यत इत्यास्तां ‚{\tiny $_{3}$}‚ तावदेतद् । अर्थस्तु ‚{\tiny $_{lb}$}‚व्याख्यायते ॥ न्यायस्त्रिरूपलिङ्गलक्षणा युक्तिः । नीयते प्राप्यते विवक्षितार्थ‚{\tiny $_{lb}$}‚सिद्धिरनेनेति कृत्त्वा अत एव ‚{\color{DodgerBlue3}‚त्रिविधं लिङ्गमि}‚{\tiny $_{1b2}$}‚ त्यादिना त्रि ‚{\tiny $_{4}$}‚ रूपमेव ‚{\tiny $_{lb}$}‚लिङ्गमनन्तरम्वक्ष्यति । तदभिधायि वचनमित्यपरे । तम्वदितुं शीलं यस्य स ‚{\tiny $_{lb}$}‚तथोक्तः । तमपि निगृह्णन्ति पराजयन्त इत्यर्थः । निगृह्णन्तु नामान्याय ‚{\tiny $_{5}$}‚ ‚{\tiny $_{lb}$}‚वादिनम्पराजयाधिकरणत्त्वादेव । न्यायवादिनन्त्वनिग्रहार्हमपि यन्निगृह्णन्त्ये‚{\tiny $_{lb}$}‚तन्न सम्भाव्यते । परोत्कर्षव्यारोपधियस्तु तमपि पराजयन्त इति सम्भावना ‚{\tiny $_{6}$}‚ ‚{\tiny $_{lb}$}‚यामपि शब्दः समुच्चयार्थोऽतिशयद्योतनार्थो वा केषु \add{।} निगृह्णन्तीत्याह । ‚{\tiny $_{lb}$}‚ ‚{\color{DodgerBlue3}‚वादेषु} ‚{\tiny $_{1b1}$}‚ साधनदूषणसंशब्दितेषु विचारेष्विति यावत् । ‚{\color{DodgerBlue3}‚विवादेष्विति} ‚{\tiny $_{lb}$}‚क्वचित्पाठः । ‚{\tiny $_{7}$}‚ तत्र विरुद्धा वादा ‚{\color{DodgerBlue3}‚विवादा} स्तेष्विति व्याख्येयं । ‚{\tiny $_{lb}$}‚विरुद्वाश्च कथं \add{।} साधनदूषणसंशब्दितानाम्विचाराणान्तद्विरुद्धार्थसाधन‚{\tiny $_{lb}$}‚प्रवृत्तत्त्वात् ॥ कथम्पुनर्न्यायवादिनमसत्स्वसि ‚{\tiny $_{8}$}‚ द्ध्यादिषु हेतुदोषेषु निगृह्णन्ती ‚{\tiny $_{lb}$}‚त्याह ॥ ‚{\color{DodgerBlue3}‚असद्व्ययवस्थोपन्यासैः} । ‚{\tiny $_{1b1}$}‚ असतामसाधूनाम्व्यवस्थाः । असत्यो ‚{\tiny $_{lb}$}‚ \leavevmode\ledsidenote{\textenglish{2/s}} वा व्यवस्थाः सद्भिः कुत्सितत्त्वात् । ताश्च फलजात्यसन्निग्रहस्थान‚{\tiny $_{lb}$}‚लक्ष ‚{\tiny $_{9}$}‚\leavevmode\ledsidenote{\textenglish{2a/msK}} णास्तासामुपन्यासाः । प्रयोगस्तैरिति विग्रहः । के पुनरह्रीकास्त एवं ‚{\tiny $_{lb}$}‚विधा इत्याह ॥ शठा धूर्ता मायाविनः परसम्पत्तावीर्ष्यालव इति यावत् । ‚{\tiny $_{lb}$}‚यस्मात्तं तथा निगृह्णन्ति ‚{\tiny $_{1}$}‚ इति तस्मात्तन्निषेधार्थं तेषां शठानान्तेषाम्वा ‚{\tiny $_{lb}$}‚ऽसद्व्यवस्थोपन्यासानान्तस्य ‚{\color{DodgerBlue3}‚वा} निग्रहस्य त्रयाणां ‚{\color{DodgerBlue3}‚वा} निषेधो निरासस्तदर्थन्तन्नि‚{\tiny $_{lb}$}‚मित्तमिदम्प्रकरणमारभ्यते ॥ ‚{\color{DodgerBlue3}‚स ए} वार्थोऽस्ये ‚{\tiny $_{2}$}‚ ति विग्रहीतव्यं । तन्निषेधे च ‚{\tiny $_{lb}$}‚कृते सम्यग्विचारः प्रवर्तते तत्पूर्वकश्च सर्वः पुरुषार्थ इत्यभिप्रायः । आसन्न‚{\tiny $_{lb}$}‚विषयिणा त्वन्तर्विपरिवर्त्ति प्रकरणमिदमापरा ‚{\tiny $_{3}$}‚ मृषति । अन्तस्तन्त्वात्मना ‚{\tiny $_{lb}$}‚परिनिष्पन्नत्वात् । अन्यथाऽपरिनिष्पन्नात्मतयाऽसन्नत्वाभावादिदम् शब्द‚{\tiny $_{lb}$}‚प्रयोगो न स्यात् । ‚{\color{DodgerBlue3}‚आरभ्यत} ‚{\tiny $_{1b1}$}‚ इति वर्तमानकालनिर्देशः \add{।} ‚{\color{DodgerBlue3}‚क ‚{\tiny $_{4}$}‚} ‚{\tiny $_{lb}$}‚थमिति चेत् । वर्तमानसामीप्ये वर्तमानवद्वे \href{http://sarit.indology.info/?cref=P\%C4\%81.3.3.31}{पाणिनिः ३।३।३१} ति वचनात् ‚{\tiny $_{lb}$}‚सम्बन्धोऽप्यभिधानीय एवान्यथा बालोन्मत्तप्रलापवदग्राह्यमिदं प्रेक्षापूर्वकारिणा‚{\tiny $_{lb}$}‚म्भवेदिति चेत् ‚{\tiny $_{5}$}‚ सत्त्यमेतत् । प्रयोजनान्तर्गतत्त्वात् पृथ ‚{\color{DodgerBlue3}‚ग} सौ नाभिहितः । तथाहि ‚{\tiny $_{lb}$}‚ ‚{\color{DodgerBlue3}‚तन्निषेधार्थमिदमारभ्यते} ‚{\tiny $_{1b1}$}‚ ततश्चैतत्प्रयोजनमनेन प्रकरणेन साध्यते । ‚{\tiny $_{lb}$}‚तथा च प्रयोजनप्रकरण ‚{\tiny $_{6}$}‚ योः साध्यसाधनलक्षणः सम्बन्ध इति सूचितं । ये त्वन्ये ‚{\tiny $_{lb}$}‚क्रियानन्तर्यादिलक्षणाः सम्बन्धास्ते न वाच्या एव प्रकरणक्रियायामनङ्गभू‚{\tiny $_{lb}$}‚तत्त्वात् । तथाहि तेषु सत्स्वपि प्रयो ‚{\tiny $_{7}$}‚ जनाभावे नारभ्यत एव प्रकरणं । ‚{\tiny $_{lb}$}‚असत्स्वपि च तेषु सति प्रयोजने प्रारभ्यत एव । तस्मात्प्रकरणारम्भस्य ‚{\tiny $_{lb}$}‚प्रयोजनान्वयव्यतिरेकानुविधानात्प्रयोजनमेवाभिधानी ‚{\tiny $_{8}$}‚ यम्प्रेक्षापूर्वकारिणा । तस्मिं‚{\tiny $_{lb}$}‚श्चाभिहिते सम्बन्धोप्युक्त एव भवतीति मन्यते । नानवधारितप्रकरण ‚{\tiny $_{lb}$}‚शरीराः प्रवर्तन्ते प्रेक्षावन्तस्तस्मात्प्रयोजनवत्प्रवृत्त्यङ्गत्वात्प्रकरण ‚{\tiny $_{9}$}‚\leavevmode\ledsidenote{\textenglish{2b/msK}} शरीरमपि ‚{\tiny $_{lb}$}‚वक्त्व्यमेवेति चेत् । एवमेतत्प्रयोजनवाक्येन त्वभिहितत्वान्नैव तदपि सम्बन्धव‚{\tiny $_{lb}$}‚त्पृथगभिधानमर्हति । यतस्तन्निषेधार्थमिदमारभ्यत इत्युक्तमतश्च तत् ‚{\tiny $_{1}$}‚ निषे ‚{\tiny $_{lb}$}‚धोऽस्य शरीरमित्युक्तम्भवति । यदा चैतन्निषेधार्थमिदमारभ्यते तदा पूर्वको हेतु‚{\tiny $_{lb}$}‚रसिद्धः । प्रतिप्रमाणद्वयञ्चानेन सूचितं प्रारब्धव्यमिदम्प्रकरणम्प्रेक्षावता ‚{\tiny $_{2}$}‚ सति ‚{\tiny $_{lb}$}‚सामर्थ्ये । ग्राह्यम्वा प्रयोजनवत्वात् । सम्बन्धवत्वाच्च तदन्यशास्त्रवदिति ‚{\tiny $_{lb}$}‚स्वभावहेतुः ।
	{\color{gray}{\rmlatinfont\textsuperscript{§~\theparCount}}}
	\pend% ending standard par
      ‚{\tiny $_{lb}$}‚\textsuperscript{\textenglish{3/s}}

	  
	  \pstart \leavevmode% starting standard par
	एवमभिधाय प्रयोजनं सकलप्रकरणार्थसंग्राहकं श्लोकमाह ‚{\tiny $_{3}$}‚ ‚{\color{DodgerBlue3}‚असाधनाङ्वचन} ‚{\tiny $_{lb}$}‚ ‚{\tiny $_{1b1}$}‚ मित्यादिना । असाधनाङ्गवचनमदोषोद्भावनञ्च द्वयोर्वादिप्रतिवादि ‚{\tiny $_{lb}$}‚नोर्यथाक्रमं निग्रहस्थानं पराजयाधिकरणं । अन्यत्त्वित्येत ‚{\tiny $_{4}$}‚ द्धेयव्यतिरिक्तम‚{\tiny $_{lb}$}‚ ‚{\color{DodgerBlue3}‚क्षपादप} रिकल्पितं प्रतिज्ञासंन्यासादिकं वक्ष्यमाणं निग्रहस्थानं न ‚{\color{DodgerBlue3}‚युक्तमिति} ‚{\tiny $_{lb}$}‚कृत्वा ‚{\color{DodgerBlue3}‚नेष्यते} \add{।} निग्रहस्थानमिति वर्त्तते । अयं तावत् स ‚{\tiny $_{5}$}‚ मासेन श्लोकार्थः । ‚{\tiny $_{lb}$}‚ ‚{\color{DodgerBlue3}‚इष्टस्ये} ‚{\tiny $_{1b1}$}‚ त्यादिना विभागमारभते । इष्टोऽर्थोऽनित्त्यः शब्द इत्यादि ‚{\tiny $_{lb}$}‚साध्यत्त्वेनेप्सितः । तस्य ‚{\color{DodgerBlue3}‚सिद्धिः} प्रतिपत्तिः साधनं । तदनेन भावस्य ‚{\tiny $_{6}$}‚ ‚{\tiny $_{lb}$}‚साधनोयं साधनशब्दस्तावदस्मिन्व्याख्यानेऽभिप्रेतो न तु करणसाधन इति ‚{\tiny $_{lb}$}‚दर्शयति ॥ तस्य साधनस्येष्टार्थसिद्धिलक्षणमस्याङ्गं किन्तदित्याह । ‚{\color{DodgerBlue3}‚निर्वर्त्तकं} ‚{\tiny $_{lb}$}‚जनकं । अने ‚{\tiny $_{7}$}‚ नाङ्गशब्दं व्याचष्टे । कारणपर्यायोयमत्राङ्गशब्दो नावयवपर्याय ‚{\tiny $_{lb}$}‚इत्यर्थः । तच्च साधनाङ्गमिह निश्चितत्रैरूप्यं लिङ्गमुच्यते । अस्य साधनाङ्गस्य ‚{\tiny $_{lb}$}‚वचनं त्रिरूपलिङ्गा ‚{\tiny $_{8}$}‚ ख्यानं । तस्य ‚{\color{DodgerBlue3}‚साधनाङ्गस्यावचनमनुच्चारणम} ‚{\tiny $_{1b2}$}‚ ‚{\tiny $_{lb}$}‚नभिधानं यत्तदसाधनाङ्गवचनं । अनेनैतत्कथयति असाधनाङ्गस्य पक्षोपनयना‚{\tiny $_{lb}$}‚देर्वचनमसाधनाङ्गवचनमिति नै ‚{\tiny $_{9}$}‚\leavevmode\ledsidenote{\textenglish{3a/msK}} व प्रतिपत्तव्यमस्मिन्व्याख्याने । किन्तु साधनाङ्ग‚{\tiny $_{lb}$}‚स्यैवावचनमसाधनाङ्गवचनमिति । तदसाधनाङ्गवचनं वादिनो निग्रहस्थानं । ‚{\tiny $_{lb}$}‚तदेतेन श्लोकस्य पूर्व्वभागम्विवृणोति । कथम्पुनः साधनाङ्गस्यानुच्चारणम्भवति ‚{\tiny $_{lb}$}‚ ‚{\color{DodgerBlue3}‚निग्रहस्थानं} चेत्याह । ‚{\color{DodgerBlue3}‚तदभ्युपगम्येति} ‚{\tiny $_{1b2}$}‚ तदितीष्टं साध्यमभ्युपगम्याह‚{\tiny $_{lb}$}‚मेतत्साधयामीति प्रतिज्ञयाप्रतिभया करण ‚{\tiny $_{2}$}‚ भूतया तूष्णीम्भावात् । अप्रतिभाऽत्र ‚{\tiny $_{lb}$}‚पूर्व्वाधिगतार्थविस्मरणं स्तम्भितत्त्वञ्च गृह्यते ।
	{\color{gray}{\rmlatinfont\textsuperscript{§~\theparCount}}}
	\pend% ending standard par
      ‚{\tiny $_{lb}$}‚

	  
	  \pstart \leavevmode% starting standard par
	अनेन सर्व्वथा साधनाङ्गस्यावचनमाह अभिधाने वा यदि न समर्थितं तदोक्त‚{\tiny $_{lb}$}‚मप्यऽ ‚{\tiny $_{3}$}‚ नूक्तमेव स्वकार्याकरणात् । इत्यभिप्रायवानाह \add{—} ‚{\color{DodgerBlue3}‚साधनाङ्गस्या‚{\tiny $_{lb}$}‚समर्थनाद्वेति} ‚{\tiny $_{1b2}$}‚ । तदभ्युपगम्येतिवर्तते । वा शब्दः पुर्व्वापेक्षया ‚{\tiny $_{lb}$}‚विकल्पार्थः । साधना ‚{\tiny $_{4}$}‚ ङ्गस्यासमर्थनं त्रिष्वपि रूपेषु निश्चयाप्रदर्शनं । तस्मात्तू‚{\tiny $_{lb}$}‚ष्णीम्भावादसमर्थनाच्च साधनाङ्गस्यानुच्चारणं । ततश्च प्रतिज्ञातार्थाकारणात् ‚{\tiny $_{lb}$}‚वादिनो ‚{\tiny $_{5}$}‚ ‚{\color{DodgerBlue3}‚निग्रहाधिकरण} मिति ‚{\tiny $_{1b2}$}‚ प्रकृतेन सम्बन्धः । क्वचित्तु ‚{\color{DodgerBlue3}‚वादिन} ‚{\tiny $_{lb}$}‚इति पाठः । तत्र तच्छब्देन प्रकृतमनुच्चारणं संबध्यते ।
	{\color{gray}{\rmlatinfont\textsuperscript{§~\theparCount}}}
	\pend% ending standard par
      ‚{\tiny $_{lb}$}‚

	  
	  \pstart \leavevmode% starting standard par
	कथम्पुनः साधनाङ्गासमर्थनम्भ ‚{\tiny $_{6}$}‚ वति । येन तद्विपर्ययेनासमर्थनात्प्रति‚{\tiny $_{lb}$}‚ज्ञातार्थाकारणाद्वादिनो निग्रहाधिकरणत्त्वमिति कदाचित्कश्चिद् ब्रूयादित्येतत्परि‚{\tiny $_{lb}$}‚जिहीर्षुरादिप्रस्थानमारचयति ‚{\tiny $_{7}$}‚ ‚{\color{DodgerBlue3}‚त्रिविधमेवे} ‚{\tiny $_{1b2}$}‚ त्यादिना त्रिप्रकारमेव लिङ्ग‚{\tiny $_{lb}$}‚ \leavevmode\ledsidenote{\textenglish{4/s}} मङ्गङ्करणं कस्यासिद्धेः प्रतिपत्तिरूपायाः । कस्य सिद्धेरित्याह । ‚{\color{DodgerBlue3}‚अप्रत्य‚{\tiny $_{lb}$}‚क्षस्या} परोक्षस्यानुमेयभूतस्य वस्तुन इति ‚{\tiny $_{8}$}‚ यावत् । अवधारणञ्चतुरादि ‚{\tiny $_{lb}$}‚व्यवच्छेदमाचष्टे । ‚{\color{DodgerBlue3}‚त्रिविधमेवेति} नियमः कथमयमिति चेत् । यस्माद्विधि ‚{\tiny $_{lb}$}‚प्रतिषेधरूपतया द्विधा साध्यं व्यवस्थितं । विधिरूपञ्च सद्भा ‚{\tiny $_{9}$}‚ \leavevmode\ledsidenote{\textenglish{3b/msK}} वरूपङ्कारणरू ‚{\tiny $_{lb}$}‚पम्वा भवद् भवेत् । नान्यत् । तत्र हेतोः प्रतिबन्धायोगात् । नह्यर्थान्तरस्याका‚{\tiny $_{lb}$}‚र्यस्य सद्भावेऽपरस्य सद्भावो युक्तः । पटसद्भाव इवोष्ट्रस्य । नाप्यर्थां ‚{\tiny $_{1}$}‚ ‚{\tiny $_{lb}$}‚तरस्याकारणस्य निवृत्तावकार्यस्यान्यस्य च निवृत्तिर्युक्तिमती । उष्ट्रे निवृत्ताविव ‚{\tiny $_{lb}$}‚पटस्य । न चान्वयव्यतिरेकविकलस्यागमकत्त्वं युक्तं । पटस्याप्युष्ट्रगमकत्त्व ‚{\tiny $_{2}$}‚ ‚{\tiny $_{lb}$}‚प्रसङ्गात् । स्वभावभूतधर्मसद्भावे तु स्वभावभूतस्यान्यस्य सद्भावो युक्तो नहि ‚{\tiny $_{lb}$}‚स्वभावः स्वम्भावम्परित्यज्य वर्त्ते । तत्स्वभावत्त्वाभावप्रसङ्गात् । तन्निवृत्तौ‚{\tiny $_{lb}$}‚च निवृ ‚{\tiny $_{3}$}‚ त्तिः ॥ कार्यस्यापि भावे कारणस्य भावो युक्तः । तन्निवृत्तौ च ‚{\tiny $_{lb}$}‚निवृत्तिरन्यथा तेन विनापि भावात्कार्यमेव तत्तस्य न स्यात्तदन्यवत् । ‚{\tiny $_{lb}$}‚तस्मात्स्वभाव ‚{\tiny $_{4}$}‚ कारणभूतसाध्यभेदात् । द्विधैव विधिरूपं साध्यं तत्रैव हेतोः ‚{\tiny $_{lb}$}‚प्रतिबन्धात् । स्वभावभूतञ्च साध्यं स्वभावहेतुः साधयति । कारणभूतञ्च ‚{\tiny $_{lb}$}‚कार्यहेतु ‚{\tiny $_{5}$}‚ रित्यभ्युपेयं । प्रतिषेधमप्युपलब्धिलक्षणप्राप्तानुपलब्धिरेव साधयति । ‚{\tiny $_{lb}$}‚नान्या यथा तथा विस्तरेण प्रतिपादयिष्यति । तदिदमिह संक्षिप्तमर्थ ‚{\tiny $_{6}$}‚ तत्त्वं ‚{\tiny $_{lb}$}‚विषयव्यपेक्षया विषयिणो लिंगस्य व्यवस्था । विषयश्च विधिः प्रतिषेधो वा ‚{\tiny $_{lb}$}‚भवेत् । विधावप्यर्थान्तरम्वा विधीये तानर्थान्तरं वा अर्थान्तरविधावपि का ‚{\tiny $_{7}$}‚ ‚{\tiny $_{lb}$}‚र्यकारणमनुभयम्वा साध्यते । कार्यप्रतिपादनेपि कारणसामान्यम्वाऽमेघादि‚{\tiny $_{lb}$}‚व्यावृत्तं वस्तुमात्रं लिङ्गत्वेनोच्यते । कारणविशेषो वा योऽप्रति‚{\tiny $_{lb}$}‚बद्धसामर्थ्ये मे ‚{\tiny $_{8}$}‚ घादिः । प्रतिषेधोपि निषेध्याभिमतस्यानुपलम्भेनोपलम्भेन ‚{\tiny $_{lb}$}‚वा प्रतिपाद्येत । अनुपलम्भेपि उपलम्भनिवृत्तिमात्रलक्षणो वा भवेत् । ‚{\tiny $_{lb}$}‚तत्तुल्ययोगावस्थ\add{ः} केवला ‚{\tiny $_{9}$}‚\leavevmode\ledsidenote{\textenglish{4a/msK}} परपदार्थोपलम्भरूपो वेति विकल्पाः । तत्र‚{\tiny $_{lb}$}‚नावश्यं गम्भीरध्वानादियुक्तमपि मेघादिकारणमात्रं वृष्ट्यादिकार्याविर्भावक‚{\tiny $_{lb}$}‚मन्तरा प्रतिबन्धसम्भवेन व्य ‚{\tiny $_{1}$}‚ भिचारात् । अतो न कारणमात्रं गमकं । कारण‚{\tiny $_{lb}$}‚विशेषादप्यप्रतिहतशक्तेरनन्तरं सम्बन्धस्मृतिव्यवहितादनुमेयविज्ञानात्प्राक् ‚{\tiny $_{lb}$}‚कार्यमेवोद्भूतमक्षज्ञानग्राह्यं ‚{\tiny $_{2}$}‚ भवतीति न तस्यापि लिङ्गत्वं । कार्यन्तु कारण‚{\tiny $_{lb}$}‚लिङ्गं युक्तं । तदविनाभावात् । तदतिक्रमे वा हेतुमत्ताम्विलंघयेत् । अनुभयमप्य‚{\tiny $_{lb}$}‚सम्बन्धानुगमकमतिप्रसङ्गो ‚{\tiny $_{3}$}‚ वा । अनर्थान्तरमप्यव्यभिचाराङ्गम्यत इति युक्ति‚{\tiny $_{lb}$}‚मत् । निषेधोप्युपलम्भेन न युज्यते विरोधात् । कथं हि नामोपलभ्यते च नास्ति ‚{\tiny $_{lb}$}‚ \leavevmode\ledsidenote{\textenglish{5/s}} चेत्युपपद्यते । तज्ज्ञान ‚{\tiny $_{4}$}‚ निवृत्तिमात्रमपि व्यभिचारि । एकज्ञानसंसर्गिणस्तु कैवल्य‚{\tiny $_{lb}$}‚दृष्टेरसत्ताव्यवहारो युक्तो यदि हि स्यादुपलभ्य सत्त्व एव भवेत् । प्रमाणञ्च यद‚{\tiny $_{lb}$}‚नुपपद्य ‚{\tiny $_{5}$}‚ मानविषयं न तद्विषयि युक्तं । यथा वाजिविषाणं । अनुपपद्यमानविषय‚{\tiny $_{lb}$}‚ञ्चोक्तप्रकारेण स्वभावादि स्यादन्यसंयोग्यादिपरपरिकल्पितं लिं ‚{\tiny $_{6}$}‚ गमिति व्याप‚{\tiny $_{lb}$}‚कानुपलम्भः । वैधर्म्येण नीलादिज्ञानं । तदेव वा कार्यादिलिङ्गं । नैमित्तिकशब्दा‚{\tiny $_{lb}$}‚र्थानुपपत्तिर्बाधिका । किमेवं सिद्धमगमकत्त्वमन्येषां । तथाहि विष ‚{\tiny $_{7}$}‚ यित्वङ्ग‚{\tiny $_{lb}$}‚मकत्त्वं प्रकाशकत्त्वमित्यादयः पर्यायाः । तस्मात्साध्यस्य त्रिविधत्त्वात्तङ्गमको ‚{\tiny $_{lb}$}‚हेतुरपि त्रिविध एवेति स्थितमेतत् ।
	{\color{gray}{\rmlatinfont\textsuperscript{§~\theparCount}}}
	\pend% ending standard par
      ‚{\tiny $_{lb}$}‚

	  
	  \pstart \leavevmode% starting standard par
	\hphantom{.}कथन्त्रिविधमित्याह । ‚{\color{DodgerBlue3}‚स्वभाव} ‚{\tiny $_{1b2}$}‚ इत्यादि ‚{\tiny $_{8}$}‚ \add{।} च शब्दः ‚{\tiny $_{lb}$}‚स्वभावकार्यापेक्षया समुच्चयार्थः । एतच्चाभिमतहेतुप्रदर्शनं स्वभावकार्यानुप‚{\tiny $_{lb}$}‚लम्भभेदात्त्रिविधमप्रत्यक्षस्य सिद्धेरङ्गं । नतु कारणैकार्थसमवायिविरो ‚{\tiny $_{9}$}‚\leavevmode\ledsidenote{\textenglish{4b/msK}} ध्यादि ‚{\tiny $_{lb}$}‚भदादिति दर्शनार्थं । ननु चोपलब्धिलक्षणप्राप्तानुपलम्भश्चेति वक्तव्यं । तस्यैवा ‚{\tiny $_{lb}$}‚प्रत्यक्षस्य सिद्धेर्वक्ष्यमाणेन न्यायेनाङ्गत्त्वान्नानुपलम्भमात्रस्य तत्कथं सामान्ये‚{\tiny $_{lb}$}‚तो ‚{\tiny $_{1}$}‚ क्तमिति चेत् । एवमेतत्समर्थितसाधनाङ्गाधिकारात्तु सामान्यशब्दोप्यय‚{\tiny $_{lb}$}‚मनुपलम्भशब्दौपलब्धिलक्षणप्राप्तानुपलम्भ एव वर्त्तते । तथाहि सामान्यशब्दा ‚{\tiny $_{lb}$}‚अपि शब्दा ‚{\tiny $_{2}$}‚ न्तरसन्निधानात्प्रकरणसामर्थ्याच्च विशेषेष्ववतिष्ठन्ते । यदाह । ‚{\color{DodgerBlue3}‚नहि ‚{\tiny $_{lb}$}‚विशेषशब्दसन्निधेरेव शब्दानां विशेषावस्थितिहेतुरपि तु प्रकरणसामर्थ्यादि‚{\tiny $_{lb}$}‚कमपीति} ‚{\tiny $_{3}$}‚ यथा चानुपलम्भमात्रस्य समर्थितसाधनाङ्गत्त्वं न सम्भवति तथोत्तरत्र ‚{\tiny $_{lb}$}‚प्रतिपादयिष्यति । अप्रत्यक्षसिद्धिहेतुलिङ्गाधिकाराद्वाऽचोद्यमेवैतत् । ‚{\tiny $_{4}$}‚ किम्पुन‚{\tiny $_{lb}$}‚रस्य त्रिविधस्य साधनाङ्गस्य समर्थनं । यद्विपर्ययादसमर्थनम्भविष्यतीत्याह । ‚{\tiny $_{lb}$}‚ ‚{\color{DodgerBlue3}‚तस्ये} ‚{\tiny $_{1b3}$}‚ त्यादि । साध्यशब्दोत्रानित्यत्त्वादिधर्ममात्रस्य वाचकः । ‚{\tiny $_{5}$}‚ अवयव‚{\tiny $_{lb}$}‚समुदायोपचारात् न तु साध्यधर्मिधर्मसमुदायस्य व्याप्तेरेवाभावप्रसङ्गात । ‚{\tiny $_{lb}$}‚शब्दादिधर्म्मिविशिष्टस्यानित्त्यत्त्वादेर्दृष्टान्तधर्मिण्यभावात् । ‚{\tiny $_{6}$}‚ व्याप्तिं प्रसाध्या‚{\tiny $_{lb}$}‚न्वयव्यतिरेकसाधकेन प्रमाणेन अनेनान्वयव्यतिरेकनिश्चयावुक्तौ । ‚{\color{DodgerBlue3}‚धर्मिणि} ‚{\tiny $_{lb}$}‚ ‚{\tiny $_{1b3}$}‚ जिज्ञासितविशेषे शब्दादौ भावसाधनं ‚{\tiny $_{1b2}$}‚ । पक्षधर्मत्त्वसाधकेन ‚{\tiny $_{lb}$}‚प्रमा ‚{\tiny $_{7}$}‚ णेन सत्त्वकथनमित्यर्थः । अनेनापि पक्षधर्मत्त्वनिश्चय उक्तः । कथम्पुनः ‚{\tiny $_{lb}$}‚सर्व्वोपसंहारेण व्याप्तिं प्रसाध्य धर्मिणि भावः कथ्यत इत्यत्रोदाहरणमाह । ‚{\tiny $_{lb}$}‚ ‚{\color{DodgerBlue3}‚यथेत्या} ‚{\tiny $_{1b3}$}‚ ‚{\tiny $_{8}$}‚ दि । ‚{\color{DodgerBlue3}‚तत्सर्व्वमि} त्यनेन सर्व्वोपसंहारेण व्याप्तिप्रदर्शनङ्क‚{\tiny $_{lb}$}‚थयति । किमर्थं \add{।} विप्रतिपत्तिनिरासार्थं \add{।} तथाहि पक्षसपक्षा‚{\tiny $_{lb}$}‚ \leavevmode\ledsidenote{\textenglish{6/s}} पेक्षयान्तर्व्याप्तिर्ब्बहिर्व्याप्तिश्च प्रदर्श्य ‚{\tiny $_{9}$}‚\leavevmode\ledsidenote{\textenglish{5a/msK}} त इत्येके विप्रतिपन्नाः । तच्च ‚{\tiny $_{lb}$}‚न युक्तं वस्तुबलायातत्त्वाद्व्याप्तेः । पूर्व्व साध्येन व्यार्प्तिं प्रसाध्य पश्चा‚{\tiny $_{lb}$}‚द् धर्म्मिणि सत्त्वं कथयितव्यमित्ययमीदृशः क्रमनियमः किमत्रास्ति न वेत्या ‚{\tiny $_{1}$}‚ ह । ‚{\tiny $_{lb}$}‚ ‚{\color{DodgerBlue3}‚अत्रापी} ‚{\tiny $_{1b3}$}‚ त्यादि । अत्रेति मन्मते साधनाङ्गसमर्थने वा ऽपिशब्दोऽवधारणे‚{\tiny $_{lb}$}‚प्रतिषेधे न च सम्बन्धनीयः । नैव कश्चिदयमीदृशः क्रमनियमः ‚{\tiny $_{1b3}$}‚ परिपाटि‚{\tiny $_{lb}$}‚नियम इति या ‚{\tiny $_{2}$}‚ वत् । किंकारणमित्याह । इष्टार्थसिद्धेरुभयत्राविशेषा ‚{\tiny $_{1b3}$}‚ ‚{\tiny $_{lb}$}‚दिति । व्याप्तिसाधनाभिधानपूर्व्वकधर्मिभावसाधनाभिधाने धर्मिभावसाधनाभिधाने ‚{\tiny $_{lb}$}‚धर्मिभावसाधनाभिधानपूर्व्वके वा व्याप्तिसा ‚{\tiny $_{3}$}‚ धनाभिधाने साध्यार्थसिद्धेर्विशेषाभावा‚{\tiny $_{lb}$}‚दित्यर्थः । एवम कूतं क्रमनियमो हि किमर्थमाश्रीयते \add{।} साध्यसिध्यर्थं । ‚{\tiny $_{lb}$}‚यथा साधर्म\edtext{}{\lemma{साधर्म}\Bfootnote{? र्म्य}}वति दृष्टान्तप्रयोगे सा ‚{\tiny $_{4}$}‚ ध्येनैव हेतोरविनाभावः प्रदर्श्यते । ‚{\tiny $_{lb}$}‚न हेतुना साध्यस्य । तथा वैधर्म्यवति साध्याभाव एव हेतोरभावः कथ्यते । न तु ‚{\tiny $_{lb}$}‚हेत्वभावे साध्यस्य । किमर्थं । माभूद्धेतोः ‚{\tiny $_{5}$}‚ साध्येनाविनाभावित्त्वाप्रदर्शनेनेष्टार्थ‚{\tiny $_{lb}$}‚सिद्धेरसिद्धिर्विपर्ययसिद्धिश्चेति । यदाह । एवं हि हेतोः सपक्ष एव सत्त्वं । साध्या‚{\tiny $_{lb}$}‚भावे चासत्त्वमेव शक्यं द ‚{\tiny $_{6}$}‚ र्शयितुं । न विपर्ययादिति । यथा नित्यताऽकृतकत्त्वेन ‚{\tiny $_{lb}$}‚नाशित्त्वाद्वाऽत्र कार्यता । स्यादनुक्ता कृता व्यापित्त्वनिष्ठश्च समन्वय इति । ‚{\tiny $_{lb}$}‚तदत्र युक्तं क्रमसमाश्रयणं ‚{\tiny $_{7}$}‚ इह तु विनाप्यनेनाभिमतार्थसिद्धिः सम्पद्यत इति ‚{\tiny $_{lb}$}‚सूवतन्न कश्चित्क्रमनियम इति । इष्टार्थसिद्धेरुभ\add{य}त्राविशेषादित्येतदेवात्र कुत ‚{\tiny $_{lb}$}‚इति चेदाह । यस्माद् धर्मिणी ‚{\tiny $_{8}$}‚ त्यादि । साध्येन व्याप्तिं प्रसाध्येत्युक्तं प्राक् । किम्पु‚{\tiny $_{lb}$}‚नस्तद्व्याप्तिसाधनमित्याह । ‚{\color{DodgerBlue3}‚अत्रेत्यादि} ‚{\tiny $_{1b3}$}‚ । अत्रेति स्यभावहेतौ । कार्या‚{\tiny $_{lb}$}‚नुपलम्भयोस्तु पश्चाद्व्याप्तिसाधनमभिधा ‚{\tiny $_{9}$}‚ \leavevmode\ledsidenote{\textenglish{5b/msK}} स्यात् । विपर्यये साध्यस्य हेतोर्व ‚{\tiny $_{lb}$}‚र्त्तमानस्य सत इति शेषः । बाधकं प्रमाणं येन साध्यविपर्यये वर्त्तमानो हेतुर्बाध्यते ‚{\tiny $_{lb}$}‚तस्य कथनं यत्तद्व्याप्तिसाधनमित्यर्थः ।
	{\color{gray}{\rmlatinfont\textsuperscript{§~\theparCount}}}
	\pend% ending standard par
      ‚{\tiny $_{lb}$}‚

	  
	  \pstart \leavevmode% starting standard par
	किं ‚{\tiny $_{1}$}‚ पुनस्तद्वाधकप्रमाणोपदर्शनमित्याह \add{—} यदि न सर्व्वं वस्तु सत्कृतकं ‚{\tiny $_{lb}$}‚ \leavevmode\ledsidenote{\textenglish{7/s}} चेति ‚{\tiny $_{1b5}$}‚ पूर्व्वोक्तिहेतुद्वयं परामृषति । प्रतिक्षणविनाशि ‚{\tiny $_{1b5}$}‚ स्यात्तदा‚{\tiny $_{lb}$}‚ऽसदेव स्यादिति सम्बन्धः । कुतः ‚{\tiny $_{2}$}‚ अक्षणिकस्य पदार्थस्य क्रमयौगपद्याभ्यामर्थ‚{\tiny $_{lb}$}‚क्रिया ‚{\tiny $_{1b5}$}‚ ऽयोगात् । तथाह \add{—} क्षणिकत्त्वेनाभिमतस्य भावस्य क्रमेण ‚{\tiny $_{lb}$}‚तावदर्थक्रिया न युज्यते । कार्यनिर्वर्त्तनयोग्यस्य ‚{\tiny $_{3}$}‚ स्वभावस्य सदा सत्त्वात् । अन्यथा ‚{\tiny $_{lb}$}‚पश्चादपि न कुर्यात् । पूर्व्वस्वभावाप्रच्युतः पुरावत् । सहकारिणमासाद्य करोतीति ‚{\tiny $_{lb}$}‚चेत् । न अनाधेयात्माति ‚{\tiny $_{4}$}‚ शयस्य पूर्व्वस्वभावापरित्यागिनः सहकारिष्वपेक्षायो‚{\tiny $_{lb}$}‚गात् । आदधत्त्येव सहकारिणस्तस्यात्मातिशयमिति चेत् । न सहकारिभिराहि‚{\tiny $_{lb}$}‚तस्यातिश ‚{\tiny $_{5}$}‚ यस्य तत्त्वान्यत्त्वायोगात् । तथाहि न तावदयमात्मातिशयस्तस्यात्म‚{\tiny $_{lb}$}‚भूतः । तस्यैव तदव्यतिरेकादात्मातिशयवत्सहकारिबलादुत्पत्तिप्रसङ्गात् ‚{\tiny $_{6}$}‚ \add{।} ‚{\tiny $_{lb}$}‚एवञ्चाभ्युपेतमस्याक्षणिकत्त्वमवहीयते व्यतिरिक्त एव स तस्मादिति चेत् । ‚{\tiny $_{lb}$}‚भवतु किन्तु तस्मादेवात्मातिशयात्कार्योत्पत्तेस्तदवस्थमस्यार्थक्रियास्वसामर्थ\edtext{}{\lemma{तस्मादेवात्मातिशयात्कार्योत्पत्तेस्तदवस्थमस्यार्थक्रियास्वसामर्थ}\Bfootnote{‚{\tiny $_{lb}$}‚? र्थ्य}}मि ‚{\tiny $_{7}$}‚ ति दुर्न्निवारः प्रसङ्गः समापतति । सम्बन्धोप्यनेन कथमिति वश्चिन्ता ‚{\tiny $_{lb}$}‚विषयमवतरत्येव । अतिशयबलात्करोतीत्यत्रापि सहकार्यपेक्षापक्षोदितो दोषः ॥ ‚{\tiny $_{8}$}‚ ‚{\tiny $_{lb}$}‚समर्थस्वभावत्त्वादनाधेयातिशयत्त्वेपि कुविन्दादिवत् किञ्चिदपेक्ष्य कार्य‚{\tiny $_{lb}$}‚जनक इति चेत् । न तत् सारं । नहि सहकारिणः प्रत्ययास्तस्य तावदतिशयमा‚{\tiny $_{lb}$}‚धातुं ‚{\tiny $_{9}$}‚ \leavevmode\ledsidenote{\textenglish{6a/msK}} क्षमाः । न चाप्यनुपकारके भावेऽपेक्षा युक्तिमनुपतत्यतिप्रसा\edtext{}{\lemma{युक्तिमनुपतत्यतिप्रसा}\Bfootnote{? स}}ङ्गात् । ‚{\tiny $_{lb}$}‚एवञ्च सर्व्वकालमस्याकार्यजनकत्त्वप्रसङ्गः । कुविन्दादीनामपि तत्स्वभावस्य ‚{\tiny $_{lb}$}‚करणादका ‚{\tiny $_{1}$}‚ रकस्य वा \add{।} ‚{\color{DodgerBlue3}‚तत्स्वभावत्त्वादित्यादि हेतुविन्दा}\edtext{\textsuperscript{*}}{\lemma{*}\Bfootnote{न्यायविन्दावपि - तत्स्वभावत्त्वात्तत्स्वभावस्य च हेतुत्त्वात् \href{http://sarit.indology.info/?cref=nb.3.69}{तृ० परि० पृ० ६९} \begin{english}\textit{Kashi Sanskrit Series.}\end{english}}} ‚{\tiny $_{1}$}‚ वुक्तमिति ‚{\tiny $_{lb}$}‚नेहोच्यते । एवन्तावत्क्रमेणास्यार्थक्रिया न युज्यते नापि युगपत् । तथाहि ‚{\tiny $_{lb}$}‚अर्थक्रियानिवर्त्तनयोग्यस्व ‚{\tiny $_{2}$}‚ भावाध्यासितमूर्तिः सहैवासावतश्च पश्चादपि तद्रूप‚{\tiny $_{lb}$}‚वियोगात्कार्यमुत्पादयेदन्यथोदयानन्तरमेवास्य क्षयः स्यात् । न चाप्यक्षणि‚{\tiny $_{lb}$}‚कत्त्वेनोपगतस्य सकृ ‚{\tiny $_{3}$}‚ त् कार्यमुत्पद्यमानमुपलभ्यते क्रमसम्भवदर्शनात् । तदे‚{\tiny $_{lb}$}‚वमयमक्षणिकः पदार्थः क्रमेण युगपद्वा न काञ्चिदप्यदर्थक्रियामात्रामंशतोपि क्ष ‚{\tiny $_{4}$}‚ मो ‚{\tiny $_{lb}$}‚निर्वर्तयितुमित्त्यसत्त्वमेवास्य । यदि नामार्थक्रिया सा न युक्ता तक्तिमित्यसत्त्वमे‚{\tiny $_{lb}$}‚वास्य स्यादित्याह । ‚{\color{DodgerBlue3}‚अर्थक्रियेत्यादि} ‚{\tiny $_{1b5}$}‚ । अर्थक्रियायाः साम ‚{\tiny $_{5}$}‚ र्थ्यन्तदेव लक्षणं ‚{\tiny $_{lb}$}‚यस्य सत्त्वस्येति विग्रहः । अतोऽर्थक्रियासामर्थ्यलक्षणात्सत्त्वाद्व्यावृत्तम्व्यवच्छिन्नं । ‚{\tiny $_{lb}$}‚ \leavevmode\ledsidenote{\textenglish{8/s}} ‚{\color{DodgerBlue3}‚इति} श्रुतेर्हेतौ । तस्मादर्थे वा \add{।} कथम्पुनरि ‚{\tiny $_{6}$}‚ दमवसीयते \add{।} अर्थक्रियासामर्थं\edtext{}{\lemma{अर्थक्रियासामर्थं}\Bfootnote{‚{\tiny $_{lb}$}‚? र्थ्यं}}सत्त्वलक्षणमित्यत आह । ‚{\color{DodgerBlue3}‚सर्व्वे} ‚{\tiny $_{1b5}$}‚ त्यादि । सर्व्वेषां सामर्थ्याना‚{\tiny $_{lb}$}‚मुपाख्या श्रुतिः । उपाख्यायते अनयेति कृत्वा तस्याविरहोऽभावः । ‚{\tiny $_{7}$}‚ स एव लक्षणं ‚{\tiny $_{lb}$}‚यस्य निरुपाख्यस्य स तथा ख्यायते । अत्रापीति श्रुतिहेतौ । सर्व्वग्रहणं घटादीना‚{\tiny $_{lb}$}‚मपि क्षणिकत्त्वेनाभिमतानाम्विषयभेदेनार्थक्रियासामर्थ्य ‚{\tiny $_{8}$}‚ निर्वृत्तेरस्तीति तेषामसत्त्व‚{\tiny $_{lb}$}‚व्यवच्छेदाय । नैवार्थक्रियासामर्थ्यं सत्त्वलक्षणमपि तु सत्तायोग इति चेत् । न । ‚{\tiny $_{lb}$}‚सत्ताया अभा ‚{\tiny $_{9}$}‚ \leavevmode\ledsidenote{\textenglish{6b/msK}} वात् । तदभावश्चान्यत्र प्रतिपादित इति नेहोच्यते । सत्तायाश्च ‚{\tiny $_{lb}$}‚नैव सत्त्वं प्राप्नोति सत्तायोगाभावात् । निःसामान्या ‚{\tiny $_{1}$}‚ नि सामान्यानीति समयात् । ‚{\tiny $_{lb}$}‚न च स्वयमतद्रूपाः पदार्थात्मानः स्वभावान्तरसम्पर्क्कमासादयन्तोपि ताद्रूप्यम्प्रति‚{\tiny $_{lb}$}‚पद्यन्ते । स्फटिकाभ्रपटलादय इव ‚{\tiny $_{2}$}‚ जवाकुसुमादिरूपमिति यक्तिञ्चिदेतत् । ‚{\tiny $_{lb}$}‚तत्वेतावदेवाभिधानीयं । ‚{\color{DodgerBlue3}‚सर्व्वसामर्थ्यविरहलक्षणं निरुपाख्यमि} ‚{\tiny $_{1b6}$}‚ ति ‚{\tiny $_{lb}$}‚तक्तिमनेनोपाख्याग्रहणेनेति चेत् । सूक्त ‚{\tiny $_{3}$}‚ मेतत्सर्व्वसामर्थ्यरहितस्य तु सामर्थ्य‚{\tiny $_{lb}$}‚निबन्धनस्य कस्यचिदपि शब्दस्यावृत्तेरसद्व्यवहारविषयत्त्वख्यापनाय । संज्ञा‚{\tiny $_{lb}$}‚याश्चानुगतार्थत्त्वसिद्ध्यर्थ ‚{\tiny $_{4}$}‚ मिदमुक्तमिति गम्यते । यदि त्वेवं साध्यविपर्यये हेतो‚{\tiny $_{lb}$}‚त्बाधकप्रमाणोपदर्शनं न क्रियते ततः किं स्यादितिचेदाह । एवं ‚{\color{DodgerBlue3}‚साधनस्य साध्य‚{\tiny $_{lb}$}‚विप ‚{\tiny $_{5}$}‚ र्यये बाधकप्रमाणानुपदर्शने} ‚{\tiny $_{1b6}$}‚ सत्यनिवृत्तिरेवाशंकाया इति सम्बन्धः । ‚{\tiny $_{lb}$}‚का पुनः सा शंका । स कृतको वा स्यान्नित्यश्चैवं प्रकारा ।
	{\color{gray}{\rmlatinfont\textsuperscript{§~\theparCount}}}
	\pend% ending standard par
      ‚{\tiny $_{lb}$}‚

	  
	  \pstart \leavevmode% starting standard par
	ननु विप ‚{\tiny $_{6}$}‚ क्षहेतोर्वृत्तिर्नोपलभ्यते तत्कथमनिवृत्तिरित्याह । ‚{\color{DodgerBlue3}‚अदर्शने ‚{\tiny $_{1b6}$}‚ पि} ‚{\tiny $_{lb}$}‚वृत्तेरस्य साधनस्याक्षविपर्यये साध्यस्य हयुपस्कारः । किमित्येवमप्यनिवृत्तिरित्याह । ‚{\tiny $_{lb}$}‚ ‚{\color{DodgerBlue3}‚वि ‚{\tiny $_{7}$}‚ रोधाभावादि} ‚{\tiny $_{1b6}$}‚ ति । साधनसाध्यविपक्षयोरिति शेषः । अयमस्याभिप्रायो यदि ‚{\tiny $_{lb}$}‚साधनस्य साध्यविपर्यस्य च परस्परम्विरोधः सिद्धः स्यात् । भवेददर्शनमात्रे ‚{\tiny $_{8}$}‚ \leavevmode\ledsidenote{\textenglish{7a/msK}} ते अन्यथा ‚{\tiny $_{lb}$}‚बाधकासिद्धौ संशयो दुर्न्निवारः स्यादित्याशङ्क्याह । न च नैव सर्व्वानुपलब्धिर्बा‚{\tiny $_{lb}$}‚धिका प्रतिषेधिका युक्तेत्युपस्कारः । कस्य भावस्य सत्त्वस्य साध्यविपर्यये ‚{\tiny $_{1}$}‚ हेतोरिति ‚{\tiny $_{lb}$}‚शेषः । प्रकरणाद्वैतद् गम्यत एव । इदमुक्तम्भवति । व्यापकानुपलब्धिरेव सहभाव ‚{\tiny $_{lb}$}‚ \leavevmode\ledsidenote{\textenglish{9/s}} बाधते हेतोः साध्याभावेन । यथा प्रतिपादितम्प्राक् । ‚{\color{DodgerBlue3}‚नाप्यदर्श ‚{\tiny $_{2}$}‚ नमात्राद्व्यावृत्तिर्वि‚{\tiny $_{lb}$}‚प्रकृष्टसर्व्वदर्शिनोऽदर्शनस्याभावासाधनादि} ‚{\tiny $_{1b7}$}‚ त्यादिना । तस्मात्सूक्तमन्यथा ‚{\tiny $_{lb}$}‚बाधकासिद्धौ संशयो दुर्न्निवारः स्यादिति । अनेन पूर्व्वोक्त ‚{\tiny $_{3}$}‚ मेव स्मारयति । एवञ्चै ‚{\tiny $_{lb}$}‚तदर्थवत् । अभावसाधनस्यादर्शनस्याप्रतिषेधादित्युक्तं । तत्र भवेत्कस्यचिदाशंका ‚{\tiny $_{lb}$}‚किमनेनाभावसाधनस्येति विशे ‚{\tiny $_{4}$}‚ षणेन यावता सर्व्वमेवादर्शनमभावसाधनमित्य‚{\tiny $_{lb}$}‚तस्तदाशङ्काविनिवृत्त्यर्थमिदमाह । ‚{\color{DodgerBlue3}‚न चेत्यादि} ‚{\tiny $_{1b11}$}‚ । अत्र च शब्दो हि ‚{\tiny $_{lb}$}‚शब्दार्थे प्रतिपत्तव्यः । तस्मिं ‚{\tiny $_{5}$}‚ नैव वाऽवधारणे व्याख्यानन्तु पूर्व्ववत् ।
	{\color{gray}{\rmlatinfont\textsuperscript{§~\theparCount}}}
	\pend% ending standard par
      ‚{\tiny $_{lb}$}‚

	  
	  \pstart \leavevmode% starting standard par
	ननु च बाधकस्यैव प्रमाणस्य क्रमाक्रमायोगस्यासामर्थ्येन व्याप्तिर्न सिद्धा तत्कथं ‚{\tiny $_{lb}$}‚तत्स्वयं असिद्धव्याप्तिकं स ‚{\tiny $_{6}$}‚ दपरस्य सत्त्वादेर्हेतोर्व्याप्तिसाधने पर्याप्तं । प्रमा‚{\tiny $_{lb}$}‚णान्तरेण तत्र व्याप्तिः साध्यत इति चेत् । सैव तर्हीयमनवस्थादोषादिवानुबध्ना‚{\tiny $_{lb}$}‚तीति कदाचित्परो ब्रूयादित्या ‚{\tiny $_{7}$}‚ शङ्क्य सर्व्वमिदमाह । ‚{\color{DodgerBlue3}‚तत्रेत्यादि} ‚{\tiny $_{1b11}$}‚ । तत्र ‚{\tiny $_{lb}$}‚शब्दो वाक्योपन्यासार्थः । सामर्थ्यं यद्वस्तुलक्षणं ‚{\color{DodgerBlue3}‚तत्क्रमाक्रमयोगेन व्याप्तं सिद्धं} । ‚{\tiny $_{lb}$}‚यत्र सामर्थ्यं तत्र क्रमाक्रमाभ्यामर्थक्रियया ‚{\tiny $_{8}$}‚ भवितव्यमित्यनेनाकारेण तदेव कुत ‚{\tiny $_{lb}$}‚इति चेदाह । ‚{\color{DodgerBlue3}‚प्रकारान्तरासग्भवात्} ‚{\tiny $_{1b11}$}‚ । यस्मादन्यत् क्रमाक्रमव्यतिरिक्तं ‚{\tiny $_{lb}$}‚प्रकारान्तरं नास्ति । तस्माद्यत्रेदं सत्त्वलक्षणमर्थक्रिया ‚{\tiny $_{9}$}‚ \leavevmode\ledsidenote{\textenglish{7b/msK}} सामर्थ्यं तत्रावश्यं च ‚{\tiny $_{lb}$}‚क्रमाक्रमाभ्यामर्थक्रियया भवितव्यं \add{।} ननु च क्रमाक्रमाभ्यामन्यो रासि ‚{\tiny $_{lb}$}‚\leavevmode\ledsidenote{\textenglish{10/s}}\edtext{}{\lemma{रासि}\Bfootnote{? शि}}र्नास्तीत्येतदेव कथं सिद्धं । क्रमाक्रमयोरन्योन्यपरिहारस्थितलक्षणत्त्वेन ‚{\tiny $_{lb}$}‚तृती ‚{\tiny $_{1}$}‚ यप्रकारव्यतिरेकत्वात् । भावाभाववदिति ब्रूमः । भवत्वेवं स तु क्रमयौगपद्य‚{\tiny $_{lb}$}‚रूप एवेति कुत इति चेत् । कस्तहर्यन्यो भवतु । कश्चिद् भवेदिति चेत् । किमर्थ‚{\tiny $_{lb}$}‚न्तर्हि मह ‚{\tiny $_{2}$}‚ त्यनर्थसङ्कटे पतितोसि यदिदन्तया तद्रूपाभिधानेप्यसमर्थोसि । यदि ‚{\tiny $_{lb}$}‚नाम सामर्थ्यं क्रमाक्रमयोगेन व्याप्तं सिद्धं तथापिकिं सिद्धमिति चेदाह । तेन ‚{\tiny $_{lb}$}‚सामर्थ्य ‚{\tiny $_{3}$}‚ स्य क्रमाक्रमयोगेन व्याप्तत्वेन ‚{\color{DodgerBlue3}‚व्यापकस्य धर्मस्य} क्रमाक्रमयोगस्यानुप‚{\tiny $_{lb}$}‚लब्धिः ‚{\tiny $_{2a1}$}‚ । ‚{\color{DodgerBlue3}‚अक्षणिके} पदार्थेऽभ्युपगते ‚{\color{DodgerBlue3}‚सामर्थ्यं बाधते} निराकरोतीत्यर्थः ॥ ‚{\tiny $_{4}$}‚ ‚{\tiny $_{lb}$}‚तथा ह्ययं क्रमाक्रमयोगस्तस्य सामर्थ्यस्य व्यापकः । ततश्चास्य निवृत्तावश्य‚{\tiny $_{lb}$}‚मेव सामर्थ्यस्यापि निवृत्तिरन्यथाऽयमस्य व्यापक एव न प्राप्नोति । यस्मा ‚{\tiny $_{5}$}‚ त्तद्वा‚{\tiny $_{lb}$}‚धते इति तस्मा ‚{\color{DodgerBlue3}‚त्क्रमयौगपद्यायोगस्य} व्यापकाभावस्य कर्मभूतस्य ‚{\color{DodgerBlue3}‚सामर्थ्याभावेन} ‚{\tiny $_{lb}$}‚व्याप्याभावेन कर्त्तृभूतेन व्याप्तिसिद्धेः कारणान्नानवस्थाप्र ‚{\tiny $_{6}$}‚ सङ्गः ।
	{\color{gray}{\rmlatinfont\textsuperscript{§~\theparCount}}}
	\pend% ending standard par
      ‚{\tiny $_{lb}$}‚

	  
	  \pstart \leavevmode% starting standard par
	\hphantom{.}एवं स्वभावहेतोः साधनाङ्गसमर्थनमभिधायाधुना निगमयति । ‚{\color{DodgerBlue3}‚एवमि} ‚{\tiny $_{2a1}$}‚ ‚{\tiny $_{lb}$}‚त्यादिना । एवञ्च यदि न समर्थनं क्रियते तदेव तद्वादिनः पराजयमावहती ‚{\tiny $_{7}$}‚ ति प्राक्‚{\tiny $_{lb}$}‚प्रतिज्ञातमेवायोजयति । ‚{\color{DodgerBlue3}‚तस्यासमर्थन} ‚{\tiny $_{2a2}$}‚ मित्यादिना । कस्मादेवं प्रारब्धार्था‚{\tiny $_{lb}$}‚साधनादिति । न ह्यसमर्थितं साधनमारब्धमर्थं साधयितुं समर्थं । विवादाभा ‚{\tiny $_{8}$}‚ \leavevmode\ledsidenote{\textenglish{8a/msK}} व‚{\tiny $_{lb}$}‚प्रसङ्गात् । तथाहि सार्व्वज्ञज्ञानसाधने संस्कारोत्कर्षभेदेन सम्भवे प्रकर्षपर्यन्तवृ‚{\tiny $_{lb}$}‚त्तयः प्रज्ञादयो गुणाः स्थिराश्रयवर्त्ति सकृद्यथाकथंचिदाहितविशेषं विना ‚{\tiny $_{9}$}‚ हेतु‚{\tiny $_{lb}$}‚रात्मनीति । यदि तर्हि बाधकप्रमाणोपदर्शनेन हेतोर्व्याप्तिः प्रसाध्यते । तथा सत्त्य‚{\tiny $_{lb}$}‚नवस्था भवतः प्राप्नोतीत्याशंकापनोदनाय पूर्व्वपक्षमारच\add{य}न्नाह । ‚{\color{DodgerBlue3}‚अत्रापी} ‚{\tiny $_{1}$}‚ ‚{\tiny $_{1b9}$}‚ ‚{\tiny $_{lb}$}‚त्यादि । अत्रेति बाधके प्रमाणे । अदर्शनमप्रमाणयतस्तव नादर्शनमात्राद्धेतोर्व्यति‚{\tiny $_{lb}$}‚रेकनिश्चय इत्यनेनाकारेण न केवलं मौले हेतावित्यपि शब्दः किं पूर्व्व ‚{\tiny $_{2}$}‚ स्यापि ‚{\tiny $_{lb}$}‚मौलस्य हेतोरव्याप्तिः प्राप्नोतीति क्रियापदं । किङ्कारणं । क्रमयौगपद्यायोगस्य वा ‚{\tiny $_{lb}$}‚सामर्थ्येन व्याप्त्यसिद्धः । तथाहि यद्यदर्शनमात्रेण न व्यतिरेकनिश्चय ‚{\tiny $_{3}$}‚ स्तथा सति ‚{\tiny $_{lb}$}‚क्रमयौगपद्यायोगश्च भविष्यति सल्लक्षणं सामर्थ्यञ्च भविष्यति । कोऽनयोर्वि‚{\tiny $_{lb}$}‚रोध इत्यत्रैव बाधके प्रमाणे व्याप्तिर्न सिद्धा । यदि नामात्र ‚{\tiny $_{4}$}‚ न सिद्धा मौलहेता‚{\tiny $_{lb}$}‚वेतद्वलेन व्याप्तिः सेत्स्यतीति चेदाह । पूर्व्वस्यापि मौलस्यापि हेतोरव्याप्तिः प्राप्नो‚{\tiny $_{lb}$}‚तीत्यध्याहारः । तथाहि न स्वयमप्रमाणकं बा ‚{\tiny $_{5}$}‚ धकं प्रमाणमन्यस्य प्रमाणमुपकल्पयि‚{\tiny $_{lb}$}‚ \leavevmode\ledsidenote{\textenglish{11/s}} तुमलं प्रामाण्यप्रसङ्गात् ।
	{\color{gray}{\rmlatinfont\textsuperscript{§~\theparCount}}}
	\pend% ending standard par
      ‚{\tiny $_{lb}$}‚

	  
	  \pstart \leavevmode% starting standard par
	यद्येवमत्रापि तर्हि व्यापके प्रमाणेऽन्येन व्यापकेन प्रमाणेन व्याप्तिर्निश्ची ‚{\tiny $_{6}$}‚ यत ‚{\tiny $_{lb}$}‚इत्याह । ‚{\color{DodgerBlue3}‚इहापि} ‚{\tiny $_{1b9}$}‚ न केवलं मौलहेतावित्यपिनाह । पुनः साधनोपक्रमे ‚{\tiny $_{lb}$}‚सत्यनवस्था भवेत् । तथाहि यत्तद्वाधके प्रमाणे व्याप्तिप्रसाधनार्थं बाधकं प्रमा ‚{\tiny $_{7}$}‚ ण‚{\tiny $_{lb}$}‚मुच्यते । तत्रापि व्याप्तिरन्येन बाधकेन प्रमाणेन साध्या । यस्मान्न तदपि स्वयम‚{\tiny $_{lb}$}‚प्रमाणमितरस्य प्रामाण्यं कर्त्तु समर्थमित्येतत्तत्रापि शक्यम्वक्तुं । तस्याप्यन्येन ‚{\tiny $_{8}$}‚ ‚{\tiny $_{lb}$}‚व्याप्तिः साध्यत इति चेत् । यद्येवं तत्रापीयमेव वार्त्तेंत्यनवस्था भवतस्तथा ‚{\tiny $_{lb}$}‚सति प्रसजति । एवं समारचितपूर्व्वपक्षः साम्प्रतमत्र प्रतिविधानमाह । ‚{\color{DodgerBlue3}‚नाभा‚{\tiny $_{lb}$}‚वसा ‚{\tiny $_{9}$}‚ \leavevmode\ledsidenote{\textenglish{8b/msK}} धनस्ये} ‚{\tiny $_{1b10}$}‚ त्यादि । व्यतिरेकसाधनत्त्वेनेत्युपस्कार\add{ः ।} इदमुक्तम्भवति । ‚{\tiny $_{lb}$}‚न सर्व्वमत्रादर्शनं प्रतिक्षिप्यते । व्यतिरेकनिश्चायकस्य व्यापकानुपलब्धिसंज्ञ‚{\tiny $_{lb}$}‚कस्यानिषेधात् । किन्त्व ‚{\tiny $_{1}$}‚ दर्शनमात्रमिति । यदाह । ‚{\color{DodgerBlue3}‚यददर्शनम्विपर्ययमभावं साध‚{\tiny $_{lb}$}‚यती} ‚{\tiny $_{1b10}$}‚ ‚{\color{DodgerBlue3}‚ति} । कस्य हेतोः कुत्र साध्यविपर्यये तददर्शनमस्य हेतोर्बाधकम्प्रमाण‚{\tiny $_{lb}$}‚मुच्यते । कस्माद्विरुद्धप्रत्यु ‚{\tiny $_{2}$}‚ पस्थापनात् अस्येति वर्त्तते । तथाहि यस्य क्रमयौगपद्या‚{\tiny $_{lb}$}‚भ्यामर्थक्रियायोगस्तस्य सामर्थ्यलक्षणं सत्त्वं नास्ति । यथा बन्ध्यातनयादीनान्तथा ‚{\tiny $_{lb}$}‚वा क्षणिकानामपि क्रमयौ ‚{\tiny $_{3}$}‚ गपद्याभ्यामर्थक्रियाऽयोग इति । क्रमाक्रमाभ्यामर्थ‚{\tiny $_{lb}$}‚क्रियाऽयोगादित्ययं व्यापकानुपलम्भः सत्त्वादित्यस्य हेतोर्विरुद्धमसत्त्वं साध्यवि‚{\tiny $_{lb}$}‚पर्यये प्रत्यु ‚{\tiny $_{4}$}‚ पस्थापयद्वाधकं प्रमाणमुच्यते । एवञ्च कृतकत्वादावपि यथायोग्य‚{\tiny $_{lb}$}‚म्वाच्यं ।
	{\color{gray}{\rmlatinfont\textsuperscript{§~\theparCount}}}
	\pend% ending standard par
      ‚{\tiny $_{lb}$}‚

	  
	  \pstart \leavevmode% starting standard par
	\hphantom{.}कस्माद्विरुद्धप्रत्युपस्थापनादस्य तद्वाधकम्प्रमाणमुच्यत इत्याह । ‚{\color{DodgerBlue3}‚एवं ‚{\tiny $_{5}$}‚ हि ‚{\tiny $_{lb}$}‚स} हेतुः सत्त्वादिलक्षणः साध्याभावे तस्मिन्नसन्निति सिध्येत् यदि तत्र साध्याभावे ‚{\tiny $_{lb}$}‚बाध्यते निराक्रियता केन स्वविरुद्धेन स्वरूपविरुद्धेनासत्त्वादि ‚{\tiny $_{6}$}‚ नेति यावत् । ‚{\tiny $_{lb}$}‚किम्भूतेन प्रमाणवता प्रमाणयुक्तेन । कस्मादेवमसौ तत्रासस्तिध्यतीत्याह । ‚{\color{DodgerBlue3}‚अन्यथा ‚{\tiny $_{lb}$}‚तत्र ‚{\tiny $_{1b11}$}‚ साध्यविपर्ययेऽस्य हेतोर्बाधकस्यासिद्धौ सत्त्यां संशयः} । सं ‚{\tiny $_{7}$}‚ श्च ‚{\tiny $_{lb}$}‚स्यान्नित्यश्चेत्यादि दुर्निवारः स्यादिति शेषः । दुःखेन निवार्यत इति दुर्निवारो ‚{\tiny $_{lb}$}‚दुर्न्निषेध इत्यर्थः । बाधकग्रहणेनात्र विरुद्धस्य प्रत्युपस्थापकम्प्रमाणं ‚{\tiny $_{8}$}‚ गृह्यते । ‚{\tiny $_{lb}$}‚वाधकप्रमाणप्रत्युपस्थापितम्वा हेतुविरुद्धं । ननु चानुपलब्धिमात्रादेव साध्यविपर्यये ‚{\tiny $_{lb}$}‚हेतोर्व्यावृत्तिनिश्चयादसन्दिग्धो व्यतिरेको भविष्यति । तक्तिमुच्य ‚{\tiny $_{9}$}‚ \leavevmode\ledsidenote{\textenglish{9a/msK}} से शङ्काया ‚{\tiny $_{lb}$}‚व्यावृत्तिः । बाधकप्रमाणानुपदर्शने तु स एव न सिध्यति तत्कथमियं निवर्त्ते‚{\tiny $_{lb}$}‚तेति । यदि नामेयमाशङ्का न व्यावर्त्तते । ततः किमित्याह । ततः आशंकाया ‚{\tiny $_{1}$}‚ ‚{\tiny $_{lb}$}‚ \leavevmode\ledsidenote{\textenglish{12/s}} अनिवृत्ते ‚{\color{DodgerBlue3}‚रनेकान्तिकः स्याद्धेत्त्वाभासः} ‚{\tiny $_{1b7}$}‚ । कस्माद्व्यतिरेकस्यस्या\edtext{}{\lemma{कस्माद्व्यतिरेकस्यस्या}\Bfootnote{? सा}} ‚{\tiny $_{lb}$}‚ ध्याभावे हेतोरभावलक्षणस्य सन्देहात्कारणात्सन्दिग्धविपक्षव्यावृत्तिकः स्याद्धे‚{\tiny $_{lb}$}‚त्वाभास इत्यर्थः ।
	{\color{gray}{\rmlatinfont\textsuperscript{§~\theparCount}}}
	\pend% ending standard par
      ‚{\tiny $_{lb}$}‚

	  
	  \pstart \leavevmode% starting standard par
	किम्पुन ‚{\tiny $_{2}$}‚ रदर्शनेप्यनिवृत्तिराशंकाया यावता तददर्शनमभावं साधयतीत्याह । ‚{\tiny $_{lb}$}‚ ‚{\color{DodgerBlue3}‚नाप्यदर्शनमात्राद्व्यावृत्तिः} ‚{\tiny $_{1b7}$}‚ साध्याभावे हेतोः सिध्यतीति वाक्याध्याहारः । ‚{\tiny $_{lb}$}‚अपिशब्दो य ‚{\tiny $_{3}$}‚ स्मादर्थे । मात्रग्रहणमुपलब्धिलक्षणप्राप्तादर्शनस्य व्यवच्छेदार्थं । ‚{\tiny $_{lb}$}‚कुत एतत् । विप्रकृष्टेषु देशकालस्वभावविप्रकर्षैः पदार्थेषु चीनदा ‚{\tiny $_{4}$}‚ शरथिपिशाच ‚{\tiny $_{lb}$}‚प्रभृतिषु यददर्शनं तस्याभावासाधनात् । कस्याभावं साधयतीति चेत् । प्रकृतत्त्वा‚{\tiny $_{lb}$}‚द्विप्रकृष्टानामिति गम्यते ।
	{\color{gray}{\rmlatinfont\textsuperscript{§~\theparCount}}}
	\pend% ending standard par
      ‚{\tiny $_{lb}$}‚

	  
	  \pstart \leavevmode% starting standard par
	ननु समासादि ‚{\tiny $_{5}$}‚ तसकलपदार्थव्यापि जानाति स यस्यादर्शनमभावम्विप्रकृष्टाना‚{\tiny $_{lb}$}‚मपि साधयति तत्कथमिदमुक्तमित्याह । ‚{\color{DodgerBlue3}‚असर्व्वदर्शिन} ‚{\tiny $_{1b7}$}‚ इति । सर्व्वन्द्रष्टुं ‚{\tiny $_{lb}$}‚शी ‚{\tiny $_{6}$}‚ लमस्य ततो नञा समासः । कस्मात्तस्याप्यदर्शनमभावन्न साधयतीति । अर्व्वा‚{\tiny $_{lb}$}‚ग्दर्शनेन पुंसा सतामपि केषाञ्चिदर्थानाम्विप्रकृष्टानामदर्शनात् । इदमागूरितं । ‚{\tiny $_{7}$}‚ नेह ‚{\tiny $_{lb}$}‚सर्व्वदर्शिदर्शनं समस्तवस्तुसत्तां प्राप्नोति । येन तन्निवर्त्तमानमर्थसत्ताम्वृक्षवच्छिं‚{\tiny $_{lb}$}‚सपां निवर्त्तयेद् भेदात् । नापि तत्तस्याः कारणं येन वह्निवद्धूमं निवर्त्तमानं नि ‚{\tiny $_{8}$}‚ व‚{\tiny $_{lb}$}‚र्त्तयेत् । तदभावेपि भावादिति । बाधकं पुनः प्रमाणमित्यादि । अत्र केचिदेवं ‚{\tiny $_{lb}$}‚पूर्व्वपक्षयन्ति । किम्पुनर्बाधकं प्रमाणं यस्योपदर्शनेन मौलस्य हेतोर्व्याप्तिप्र ‚{\tiny $_{9}$}‚ \leavevmode\ledsidenote{\textenglish{9b/msK}} ती‚{\tiny $_{lb}$}‚तिर्भवतीत्याह । ‚{\color{DodgerBlue3}‚बाधक} म्पुन ‚{\tiny $_{1b7}$}‚ रित्यादि । तेषाङ्कथमत्र व्याप्तिसाधनम्विपर्यये ‚{\tiny $_{lb}$}‚बाधकप्रमाणोपदर्शनं यदि न सर्व्वं सत्कृतकं वा प्रतिक्षणम्विनाशि स्यादित्यादि‚{\tiny $_{lb}$}‚नाऽत्रैव ‚{\tiny $_{1}$}‚ प्रागर्थस्याभिहितत्त्वात्पुनरुक्तदोषप्रसक्तिर्न भवतीति चिन्त्यमेतत् ‚{\tiny $_{lb}$}‚तैरेवेत्यलं परदोषसंकीर्त्तनेन । तस्मादन्यथा पूर्व्वपक्ष्यते । यस्यापि तर्हि बाधकम्प्र‚{\tiny $_{lb}$}‚माणमस्ति त ‚{\tiny $_{2}$}‚ स्य कथमयमदोष इत्यत आह । ‚{\color{DodgerBlue3}‚बाधकं पुनः प्रमाणं} ‚{\tiny $_{1b7}$}‚ प्रवर्त्त‚{\tiny $_{lb}$}‚मानमसामर्थ्यमाकर्षतीति क्रियापदं । कीदृशमसल्लक्षणं । कथं प्रमाणं यस्य पदार्थस्य ‚{\tiny $_{lb}$}‚क्रमयौग ‚{\tiny $_{3}$}‚ पद्यायोगः । अर्थक्रियाया इत्यध्याहार्यं । न तस्य क्वचित्कार्ये सामर्थ्यं ‚{\tiny $_{lb}$}‚यथा नभस्तलारविन्दस्येत्यध्याहार्यो दृष्टान्तः । अस्ति चाक्षणिके भावे स क्र ‚{\tiny $_{4}$}‚ म‚{\tiny $_{lb}$}‚यौगपद्याभ्यामर्थक्रियाया अयोग इत्येवम्प्रवर्त्तमानं । ततः किञ्जातमिति चेदाह । ‚{\tiny $_{lb}$}‚ ‚{\color{DodgerBlue3}‚तेन} ‚{\tiny $_{1b8}$}‚ कारणेन येन तत्प्रवर्त्तमानमसल्लक्षणमसाम ‚{\tiny $_{5}$}‚ र्थ्यमाकर्षयति । यत्स‚{\tiny $_{lb}$}‚त्कृतकम्वा तदनित्यमेवेति सिध्यति ।
	{\color{gray}{\rmlatinfont\textsuperscript{§~\theparCount}}}
	\pend% ending standard par
      ‚{\tiny $_{lb}$}‚

	  
	  \pstart \leavevmode% starting standard par
	\hphantom{.}एवमपि किं सिद्धम्भवतीत्याह । ‚{\color{DodgerBlue3}‚तावता} च वाधकप्रमाणोपदर्शनमात्रेण ‚{\tiny $_{6}$}‚ ‚{\color{DodgerBlue3}‚साधन-} ‚{\tiny $_{lb}$}‚ \leavevmode\ledsidenote{\textenglish{13/s}} ‚{\color{DodgerBlue3}‚धर्ममात्रान्वयः} ‚{\tiny $_{1b8}$}‚ सिध्यतीति वर्त्तते । केनेत्याह । साध्यधर्मस्य कर्त्तरि चेयं ‚{\tiny $_{lb}$}‚षष्ठी प्रतिपत्तव्या । तेन साध्यधर्मेण साधनधर्ममात्रस्यानपेक्षितहेत्वन्तरव्यापा ‚{\tiny $_{7}$}‚ रस्या‚{\tiny $_{lb}$}‚न्वयः सिध्यतीति वाक्यार्थः सन्तिष्ठते \add{।} नत्वेवङ्करणीयं साधनधर्ममात्रेणान्वयः ‚{\tiny $_{lb}$}‚साध्यधर्मेति । एवं हि साध्यमेव हेतुना ऽविनाभूतं जातं न हेतुरिति हेतो ‚{\tiny $_{8}$}‚ रगम‚{\tiny $_{lb}$}‚कत्त्वम्भवेत् । ततश्च को गुणो लभ्यत इत्याह । ‚{\color{DodgerBlue3}‚स्वभावहेतुलक्षणञ्च सिद्धम्भ‚{\tiny $_{lb}$}‚वति} ‚{\tiny $_{1b9}$}‚ तावता चेति वर्त्तते । स्वभावहेतुलक्षणञ्च तद्भावमात्राद्धर्मिनिस्वभा‚{\tiny $_{lb}$}‚वो ‚{\tiny $_{9}$}‚ \leavevmode\ledsidenote{\textenglish{10a/msK}} पि प्रत्ययाभावेऽपुनर्यत्नापेक्षितत्त्वात् । कलधौतमलविशुद्धिवदित्येवमादयो हेतवः ‚{\tiny $_{lb}$}‚प्रतिलब्धसामर्थ्यातिशयाः सन्त्येव ते च यद्यसमर्थिता एव ज्ञाप्यसमर्थं ज्ञाप ‚{\tiny $_{1}$}‚ येयुस्तदा ‚{\tiny $_{lb}$}‚जैमिनिप्रभृतीनां विवादाभाव एव भवेत् । नन्वेतदेव न सम्भाव्यते । यत्परमार्थतः ‚{\tiny $_{lb}$}‚समर्थस्यापि हेतोरभिधाने निग्रहार्होऽसावित्याशङ्कायां स्वाभिप्रा ‚{\tiny $_{2}$}‚ यं प्रकटयति । ‚{\tiny $_{lb}$}‚वस्तुतः समर्थस्य हेतोरुपादानेपि सामर्थ्यप्रतिपादनादिति । अयमस्य भावो यदि ‚{\tiny $_{lb}$}‚नामानेन परमार्थतः समर्थो हेतुरुपात्तस्तथापि तस्य सा ‚{\tiny $_{3}$}‚ मर्थ्यं ‚{\color{DodgerBlue3}‚साधनाङ्गासमर्थनान्न} ‚{\tiny $_{lb}$}‚प्रतिपादितमनेनेति असमर्थकल्प एवासौ । न ह्यर्थस्य परार्थानुमाने गुणदोषा‚{\tiny $_{lb}$}‚वधिक्रियेते । किन्तर्हि वचनस्य वक्तुरय ‚{\tiny $_{4}$}‚ थार्थाभिधानेनोपालम्भात् । अत एव ‚{\tiny $_{lb}$}‚यत्राप्यर्थस्य गुणदोषावधिक्रियेते तत्रापि वचनद्वारेणैव । एवमेतदभ्युपगन्तव्य‚{\tiny $_{lb}$}‚मन्यथा क्षणिकः शब्द इ ‚{\tiny $_{5}$}‚ त्येतावन्मात्रमेव प्रतिज्ञावचनमभिधाय स्थातव्यं । तथा‚{\tiny $_{lb}$}‚भिधानादेवाभिमतार्थसिद्धेरिति ॥ ० ॥
	{\color{gray}{\rmlatinfont\textsuperscript{§~\theparCount}}}
	\pend% ending standard par
      ‚{\tiny $_{lb}$}‚

	  
	  \pstart \leavevmode% starting standard par
	एवं स्वभावहेतावुपदर्श्य साधनाङ्गसमर्थनमिदा ‚{\tiny $_{6}$}‚ नीङ्कार्यहेतावाह \add{।} ‚{\tiny $_{lb}$}‚ ‚{\color{DodgerBlue3}‚कार्यहेतावपी} ‚{\tiny $_{2a2}$}‚ त्यादिना । किम्पुनस्तदित्याह \add{।} ‚{\color{DodgerBlue3}‚यत्कार्यं लिङ्गं} धूमादि‚{\tiny $_{lb}$}‚संज्ञकं ‚{\color{DodgerBlue3}‚कारणस्य} दहनादेः ‚{\color{DodgerBlue3}‚साधनायोपादीयते} ‚{\tiny $_{2a3}$}‚ । तस्य धूमादेस्तेन दहना‚{\tiny $_{lb}$}‚दिना ‚{\tiny $_{7}$}‚ सह कार्यकारणभावप्रसाधनलिङ्गिलिङ्गयोर्हेतुफलभावसाधनमेव यत्तदेव ‚{\tiny $_{lb}$}‚कार्यहेतौ साधनाङ्गसमर्थनमित्यर्थः । केन पुनस्तयोः कार्यकारणभावः प्रसा ‚{\tiny $_{8}$}‚ ‚{\tiny $_{lb}$}‚ध्यत इत्याह । ‚{\color{DodgerBlue3}‚भावाभावसाधनप्रमाणाभ्या} ‚{\tiny $_{2a3}$}‚ मिति । भावाभावौ कार्य‚{\tiny $_{lb}$}‚कारणयोः सदसत्ते तयोः साधने ते च ते प्रमाणे चेति व्यु ‚{\tiny $_{9}$}‚ \leavevmode\ledsidenote{\textenglish{10b/msK}} त्पत्तिक्रमः \add{।} साधन- ‚{\tiny $_{lb}$}‚शब्दश्च करणसाधनः भावाभावसाधनप्रमाणे च प्रत्यक्षानुपलम्भो यथाक्रमं \add{।} ‚{\tiny $_{lb}$}‚कथम्पुनः प्रत्यक्षानुपलम्भाभ्यां कार्यकारणभावः प्रसाध्यत इत्याह ‚{\tiny $_{1}$}‚ \add{।} यथे‚{\tiny $_{lb}$}‚ \leavevmode\ledsidenote{\textenglish{14/s}} ‚{\tiny $_{2a3}$}‚ त्यादि । इदन्धूमादिसंज्ञकं कार्यमस्मिन्दहने सति भवति । उपलब्धिलक्षण‚{\tiny $_{lb}$}‚प्राप्तं सदनुपलब्धं प्रागिति वाक्यशेषः कार्यः । अन्यथा दहनस्य तत्र धूमे व्यापार ‚{\tiny $_{lb}$}‚एव ‚{\tiny $_{2}$}‚ न कथितः स्यात् दहनसन्निधानात्प्रागप्येतदासीदित्याशङ्कासम्भवात् । यथो‚{\tiny $_{lb}$}‚क्तानुपलम्भग्रहणे तु नैवेयमवतरति । अत एव च ‚{\color{DodgerBlue3}‚प्रमाणविनिश्चि\edtext{}{\lemma{प्रमाणविनिश्चि}\Bfootnote{? श्च}}या} दा‚{\tiny $_{lb}$}‚वप्येवमेवा ‚{\tiny $_{3}$}‚ भिहितमिति । अनेन च प्रत्यक्षप्रमाणव्यापार उक्तः । तथाहि अस्मि‚{\tiny $_{lb}$}‚न्सतीदं भवतीति प्रत्यक्षेणैतद् गम्यते । सम्प्रत्यनुपलम्भस्य व्यापारं निर्दिदिक्षु ‚{\tiny $_{4}$}‚ ‚{\tiny $_{lb}$}‚राह । ‚{\color{DodgerBlue3}‚सत्स्वपी} ‚{\tiny $_{2a3}$}‚ त्यादि । तस्माद्दहनादन्येषु विद्यमानेष्वपि । तथा तस्य ‚{\tiny $_{lb}$}‚धूमस्य हेतुष्विन्धनानिलादिषु समर्थेषु सत्स्वपीति वर्त्तते । चकारश्चात्र लुप्त ‚{\tiny $_{5}$}‚ ‚{\tiny $_{lb}$}‚निर्दिष्टः प्रतिपत्तव्यः समर्थेषु चेति । तस्य दहनस्याभावे न भवति । इदमित्यधि‚{\tiny $_{lb}$}‚कृतं । इति एवमित्यस्यार्थे वर्त्तते व्यवच्छेदे वा । तदनेन गवाश्वादी ‚{\tiny $_{6}$}‚ नां ‚{\tiny $_{lb}$}‚तद्देशकालस\add{न्}निहितानामपि धूमजननं प्रति कारणत्त्वन्निसि\edtext{}{\lemma{कारणत्त्वन्निसि}\Bfootnote{? षि}}द्धं । यतो ‚{\tiny $_{lb}$}‚यदि ते गवाश्वादयस्तस्य कारणम्भवेयुस्तदा व्यतीतेप्यग्नौ तेषां सन्निहितत्त्वाद् ‚{\tiny $_{lb}$}‚धूमो ‚{\tiny $_{7}$}‚ त्पत्तिप्रसङ्गः । इन्धनादिकारणान्तरापेक्षास्ते तस्य जनका भवन्ति ततोयम‚{\tiny $_{lb}$}‚प्रसङ्ग इति चेत् । यद्येवं तेपि तत्र सन्निहिता एवेति न व्यावर्त्तते प्रसङ्ग इति ‚{\tiny $_{lb}$}‚द ‚{\tiny $_{8}$}‚ हनमपि सहकारिणमपेक्ष्य तं जनयन्ति ततो न युक्तमिदमिति चेत् । नन्वेवं ‚{\tiny $_{lb}$}‚सत्यायातन्दहनस्य धूमोर्त्पत्तिं प्रति कारणत्त्वं । तक्तिमिदमुच्यते पूर्व्वापरव्याह‚{\tiny $_{lb}$}‚तङ्ग ‚{\tiny $_{9}$}‚ \leavevmode\ledsidenote{\textenglish{11a/msK}} वाश्वादय एव तस्य कारणन्न दहन इति । अस्तु तर्हि तस्यापि दहनस्य ‚{\tiny $_{lb}$}‚कारणत्त्वङ्गवाश्वादीनामपीति चेत् । न युक्तमेतत् । तत्र व्यतिरेकगतेर्दुर्घटत्वात् । ‚{\tiny $_{lb}$}‚तथा ह्यप ‚{\tiny $_{1}$}‚ गतेष्वपि गवादिषु सति च दहने तत्रेन्धनादिकलापे भवत्येव हुतभुग्धे‚{\tiny $_{lb}$}‚तोरुत्पत्तिः । यतः कारणप्रबन्धङ्कार्यप्रबन्धञ्चाश्रित्य हेतुफलभावश्चिन्त्यते ‚{\tiny $_{lb}$}‚भावानाङ्का ‚{\tiny $_{2}$}‚ रणप्रबन्धपूर्व्वः कार्यप्रबन्ध इति । तत्तु क्षणभेदं । नहि समासादित‚{\tiny $_{lb}$}‚ज्ञानातिशयानामयं पूर्व्वः क्षणोऽयमुत्तर इति विशेषावलम्बि ज्ञानमुदेति \add{।} ‚{\tiny $_{lb}$}‚अर्व्वाग्दर्शि ‚{\tiny $_{3}$}‚ भिश्चाधिकृत्य प्रमाणलक्षणं प्रणयितं कृपावद्भिः । यथोक्तं ‚{\tiny $_{lb}$}‚सांव्यवहारिकस्यैतत् प्रमाणस्य रूपमुक्तमत्रापि परे विमूढा विसम्वादयन्ति लो ‚{\tiny $_{4}$}‚ क‚{\tiny $_{lb}$}‚मिति । वासगृहादिषु तर्हि दहनाभावेपि धूमसद्भावाद्व्यभिचार इति चेत् । ‚{\tiny $_{lb}$}‚भूतस्यापि दहनप्रबन्धपूर्व्वकत्त्वमस्त्येव । साक्षात्पारम्प ‚{\tiny $_{5}$}‚ र्यकृतस्तु विशेषः । ‚{\tiny $_{lb}$}‚अवयन्ति च विच्छिन्नाविच्छिन्नदर्शनप्रबन्धयोर्द्धूमप्रबन्धयोर्वासगृहादिरसवतीप्रदे‚{\tiny $_{lb}$}‚शादिभाविनोः स्फुटमेव भेदम्विचित्र ‚{\tiny $_{6}$}‚ भावस्वभावविवेकाभ्यासबलोपजातविदग्ध‚{\tiny $_{lb}$}‚ \leavevmode\ledsidenote{\textenglish{15/s}} बुद्धय इति भेदेनाप्यनुमानमविरुद्धमत एव देशकालाद्यपेक्षमनुमानं कार्यहेतौ ‚{\tiny $_{lb}$}‚विरुद्धकार्योपलम्भे चोक्तं ।‚{\tiny $_{7}$}‚ ‚{\tiny $_{lb}$}‚ 
	    \pend% close preceding par
	  
	    
	    \stanza[\smallbreak]
	  \flagstanza{\tiny\textenglish{...2}}{\normalfontlatin\large ``\qquad}इष्टम्विरुद्धकार्येपि देशकालाद्यपेक्षणं ।&‚{\tiny $_{lb}$}‚अन्यथा व्यभिचारि स्याद् गत्येवासीत \add{?} साधनं \add{॥ २}{\normalfontlatin\large\qquad{}"}\&[\smallbreak]
	  
	  
	  
	    \pstart  \leavevmode% new par for following
	    \hphantom{.}
	   इति ।
	{\color{gray}{\rmlatinfont\textsuperscript{§~\theparCount}}}
	\pend% ending standard par
      ‚{\tiny $_{lb}$}‚

	  
	  \pstart \leavevmode% starting standard par
	अथापि कश्चिद्विविधभावभेदप्रविचयचातुर्यातिशयशलाकोन्मी ‚{\tiny $_{8}$}‚ लितप्रज्ञाचक्षु‚{\tiny $_{lb}$}‚ष्ट्वादयं ज्वलनजनितो धूमोऽयं धूमजनित इति विवेचयति । तथापि न सुतरां ‚{\tiny $_{lb}$}‚व्यभिचारः । तथाहि नाग्निजन्यो धूमो धूमाद् भवति निर्हेतुकत्त्वप्रसङ्गात् । ‚{\tiny $_{9}$}‚ \leavevmode\ledsidenote{\textenglish{11b/msK}} ‚{\tiny $_{lb}$}‚तथा च यदुक्तं \add{।} ‚{\tiny $_{lb}$}‚ 
	    \pend% close preceding par
	  
	    
	    \stanza[\smallbreak]
	  \flagstanza{\tiny\textenglish{...3}}{\normalfontlatin\large ``\qquad}अतश्चानग्नितो धूमो यदि धूमस्य सम्भवः ।&‚{\tiny $_{lb}$}‚शक्रमूर्ध्नस्तथास्त\edtext{}{\lemma{शक्रमूर्ध्नस्तथास्त}\Bfootnote{? त}}स्य केन वार्येत सम्भवः \add{॥ ३}{\normalfontlatin\large\qquad{}"}\&[\smallbreak]
	  
	  
	  
	    \pstart  \leavevmode% new par for following
	    \hphantom{.}
	   इत्यादि
	{\color{gray}{\rmlatinfont\textsuperscript{§~\theparCount}}}
	\pend% ending standard par
      ‚{\tiny $_{lb}$}‚

	  
	  \pstart \leavevmode% starting standard par
	तदसारमित्यप्युपेक्षते \add{।} तस्मान्न तेषां गवाश्वादीनां तत्र ‚{\tiny $_{1}$}‚ कारणत्त्वमस्तीति ‚{\tiny $_{lb}$}‚निश्चयः समाधीयतां । अतश्च दहन एव तस्य कारणं नाश्वादय इति स्थितमेतत् । ‚{\tiny $_{lb}$}‚तथा च दहनस्य कारणत्त्वं योजितमन्वयव्यतिरेकाभ्यां यथो ‚{\tiny $_{2}$}‚ क्तप्रकाराभ्यामेव‚{\tiny $_{lb}$}‚मिंधनादिसामग्र्याः सर्वस्याः कारणं योजयितव्यं । यदि वैकवाक्यतयैव व्याख्या‚{\tiny $_{lb}$}‚यते । सत्स्वपि तस्माद्दहनादन्येषु समर्थेषु तद्धेतुष्विं ‚{\tiny $_{3}$}‚ धनादिष्वस्याभावे न भव‚{\tiny $_{lb}$}‚तीति । गवाश्वादीनां त्वहेतुत्त्वम्व्यतिरेकाभावतया यथोक्तेन विधिना बोद्धव्यं ।
	{\color{gray}{\rmlatinfont\textsuperscript{§~\theparCount}}}
	\pend% ending standard par
      ‚{\tiny $_{lb}$}‚

	  
	  \pstart \leavevmode% starting standard par
	ननु चैतदेव युक्तम्वक्तुं तदभावेन भवती ‚{\tiny $_{4}$}‚ त्यथ किमर्थं सत्स्वपि तदन्येषु ‚{\tiny $_{lb}$}‚समर्थेषु तद्धेतुष्वित्युच्यत इति चेदाह । ‚{\color{DodgerBlue3}‚एवं ही} ‚{\tiny $_{2a4}$}‚ त्यादि । यस्मादेवं सत्स्व‚{\tiny $_{lb}$}‚पीत्यादिनाऽभिधीयमानोऽस्य धूमस्य तत्कार्य ‚{\tiny $_{5}$}‚ त्वमग्निकार्यत्त्वं समर्थितं निश्चि‚{\tiny $_{lb}$}‚तमसन्दिग्धम्भवति । अन्यथा यद्येवं नोपदर्श्यते । केवलं तदभावेन भवतीत्युप‚{\tiny $_{lb}$}‚दर्श्यते तदा तदभावेन भवतीत्युप ‚{\tiny $_{6}$}‚ दर्शने क्रियमाणेन्यस्यापि तु गवाश्वादेरिन्ध‚{\tiny $_{lb}$}‚नादेश्च तत्राग्निशून्यभूभागेऽभावे सति सन्दिग्धमस्याग्नेः सामर्थ्यम्भवतीति कुतः ‚{\tiny $_{lb}$}‚कारणभावनिश्चय इति समुदाया ‚{\tiny $_{7}$}‚ र्थः । तत एवाह \add{।} ‚{\color{DodgerBlue3}‚सत्सु हि समर्थेषु तद्धे‚{\tiny $_{lb}$}‚तुषु ‚{\tiny $_{2a3}$}‚ कार्यानुत्पत्तिः कारणान्तरविकल्पं सूचयती} ति सन्दिग्धमस्यान्यथा ‚{\tiny $_{lb}$}‚सामर्थ्यमित्येतदेवान्य ‚{\color{DodgerBlue3}‚त्तत्रेत्या} ‚{\tiny $_{2a4}$}‚ दिना सूचयति । तत्र धू ‚{\tiny $_{8}$}‚ माश्वकार्येऽन्य‚{\tiny $_{lb}$}‚देवाश्वादि । यदिन्धनादिसमर्थन्तदभावात्तन्न भूतन्दहनशून्यदेशे । अस्य स्वभावा‚{\tiny $_{lb}$}‚त्तन्न जातमिति कुतोयं निश्चय इत्यर्थः ।
	{\color{gray}{\rmlatinfont\textsuperscript{§~\theparCount}}}
	\pend% ending standard par
      ‚{\tiny $_{lb}$}‚\textsuperscript{\textenglish{16/s}}

	  
	  \pstart \leavevmode% starting standard par
	 \leavevmode\ledsidenote{\textenglish{12a/msK}}यद्यन्यत्तत्र समर्थन्तद भावात्तन्न जातमेतन्निवृत्तौ निवृत्तिस्तर्ह्यस्य कथमिति ‚{\tiny $_{lb}$}‚चेदाह \add{।} ‚{\color{DodgerBlue3}‚एतन्निवृत्तावित्या ‚{\tiny $_{2a5}$}‚ दि} । एतस्याग्नेर्न्निवृत्तौ धूमनिवृत्तिर्येयं ‚{\tiny $_{lb}$}‚धूमस्य सा यदृच्छासम्वादः । ‚{\color{DodgerBlue3}‚काकता ‚{\tiny $_{1}$}‚ ली} यन्यायेनेत्यर्थः । यदा तु सत्स्वपीति क्रियते ‚{\tiny $_{lb}$}‚तदा सर्व्वेषां तत्र सन्निपातादेतस्यैव निवृत्तावस्य निवृत्तिरिति निश्चयान्न यदृच्छा‚{\tiny $_{lb}$}‚सम्वाद इत्यभिप्रायः ।
	{\color{gray}{\rmlatinfont\textsuperscript{§~\theparCount}}}
	\pend% ending standard par
      ‚{\tiny $_{lb}$}‚

	  
	  \pstart \leavevmode% starting standard par
	किम्वदे ‚{\tiny $_{2}$}‚ तन्निवृत्तौ निवृत्तिर्यदृच्छासम्वाद इत्याह \add{।} ‚{\color{DodgerBlue3}‚मातृविवाह} ‚{\tiny $_{2a5}$}‚ ‚{\tiny $_{lb}$}‚इत्यादि । मातुर्विवाह उचित आचरितो यस्मिन्देशे स तथा । ततो देश‚{\tiny $_{lb}$}‚शब्देन सह विशेषणसमास ‚{\tiny $_{3}$}‚ \add{ः ।} तत्र स च जन्माश्रयत्त्वादुपचारात् । जन्म ‚{\tiny $_{lb}$}‚उत्पत्तिर्यस्य तस्य ‚{\color{DodgerBlue3}‚पारसीक} देशभावि न यावत् । देशान्तरे ‚{\color{DodgerBlue3}‚मालवका} दिदेशे यथा‚{\tiny $_{lb}$}‚ऽभावो मातृविवा ‚{\tiny $_{4}$}‚ हाभावे यदृच्छासम्वादस्तद्वदत्रापि । तथाहि मृद्विशेषाभावा‚{\tiny $_{lb}$}‚द्देशान्तरे तस्याभावो न तु मातृविवाहाभावादिति काकतालीयस्तदभावे तस्या‚{\tiny $_{lb}$}‚भा ‚{\tiny $_{2}$}‚ व इति । एवञ्चैतत् ।
	{\color{gray}{\rmlatinfont\textsuperscript{§~\theparCount}}}
	\pend% ending standard par
      ‚{\tiny $_{lb}$}‚

	  
	  \pstart \leavevmode% starting standard par
	अथवा अन्यथा व्याख्यायते । यथेदं धूमादिकार्यमस्मिन्नग्नीन्धनादिकारण‚{\tiny $_{lb}$}‚कलापे सति भवति । वाक्याध्याहारस्तु पूर्व्व ‚{\tiny $_{6}$}‚ वत्कार्यः । तस्य प्रयोजनं तदेवाव‚{\tiny $_{lb}$}‚गन्तव्यं । इदम्प्रत्यक्षव्यापारसङ्कीर्त्तनं । सत्स्वपीत्यादिनाऽनुपलम्भस्य तदन्येषु ‚{\tiny $_{lb}$}‚पुनस्तस्मादग्न्यादिकारणकलापात् । अन्ये ‚{\tiny $_{7}$}‚ षु गवाश्वादिषु समर्थेषु तद्धेतुष्वस्या‚{\tiny $_{lb}$}‚ग्न्यादिकारणकलापस्याभावे न भवति । एतच्च परमतापेक्षमुक्तं । न तु तेषान्तद्धे‚{\tiny $_{lb}$}‚तुत्त्वमस्ति । यदि पुनस्ते तस्य हेतवः स्यु ‚{\tiny $_{8}$}‚ स्तदा तत्कलापसन्निधेः प्रागपि पश्चा‚{\tiny $_{lb}$}‚दिव धूमोत्पादप्रसङ्गः । तत्सापेक्षतया तत्कृतकत्त्वं तेषामिति चेत् । आयातं तर्हि ‚{\tiny $_{lb}$}‚तस्य कलापस्य कारणत्त्वं ।
	{\color{gray}{\rmlatinfont\textsuperscript{§~\theparCount}}}
	\pend% ending standard par
      ‚{\tiny $_{lb}$}‚

	  
	  \pstart \leavevmode% starting standard par
	भवतु ‚{\tiny $_{9}$}‚\leavevmode\ledsidenote{\textenglish{12b/msK}} तर्ह्युभयोरपि न नः काचित् क्षतिरिति चेत् । न । व्यतिरेकगतस्तत्र ‚{\tiny $_{lb}$}‚दुर्घटत्वादित्युक्तं । यथाऽपगतेष्वपि सर्व्वेषु तेषु तस्मिं कलापे सति भवत्येव तस्य ‚{\tiny $_{lb}$}‚सम्भव इति तद ‚{\tiny $_{1}$}‚ भावे न भवतीति वाच्यं । तत्किमर्थं सत्स्वपीत्याद्युक्तमिति ‚{\tiny $_{lb}$}‚चेदाह । ‚{\color{DodgerBlue3}‚एवं ही ‚{\tiny $_{2a4}$}‚ त्यादि} । अन्यथा तस्य कलापस्याभावे न भवतीत्युपदर्शने ‚{\tiny $_{lb}$}‚तस्यापि गवाश्वादेस्तत्राभावे सति ‚{\tiny $_{2}$}‚ सन्दिग्धमस्य कलापस्य सामर्थ्यम्भवेत् । ‚{\tiny $_{lb}$}‚यतोऽन्यद् गवाश्वादि तत्र शक्तं तदभावात्तन्न भूतमेतस्य कलापस्य निवृत्तौ निवृ‚{\tiny $_{lb}$}‚त्तिर्यदृच्छासम्वादः शेषं पूर्व्वयत् ।
	{\color{gray}{\rmlatinfont\textsuperscript{§~\theparCount}}}
	\pend% ending standard par
      ‚{\tiny $_{lb}$}‚\textsuperscript{\textenglish{17/s}}

	  
	  \pstart \leavevmode% starting standard par
	एवाङ्कार्य ‚{\tiny $_{3}$}‚ कारणभावनिश्चयोपायविधिमुक्त्वा प्रकृतमुपसंहरति । ‚{\color{DodgerBlue3}‚एव‚{\tiny $_{lb}$}‚मि} ‚{\tiny $_{2a5}$}‚ त्यादिना । एवं यथोक्तेन विधिना तद्धूमादि तस्य वह्न्यादेः ‚{\color{DodgerBlue3}‚कार्यं ‚{\tiny $_{lb}$}‚समर्थितं} निश्चितं ‚{\tiny $_{4}$}‚ सिध्यति भवति । यदेवासमर्थितमसन्दिग्धं सिध्यति निश्ची‚{\tiny $_{lb}$}‚यते । अथवा एवं प्रत्यक्षानुपलम्भाभ्यां सम र्थित\edtext{}{\lemma{र्थित}\Bfootnote{? तं}}सत्तत्तस्य कार्यं सिध्यति ।
	{\color{gray}{\rmlatinfont\textsuperscript{§~\theparCount}}}
	\pend% ending standard par
      ‚{\tiny $_{lb}$}‚

	  
	  \pstart \leavevmode% starting standard par
	यदि नाम सिध्यति त ‚{\tiny $_{5}$}‚ त् किभित्याह । ‚{\color{DodgerBlue3}‚सिद्धं सत् स्वसम्भवेन} आत्मसन्निधानेन ‚{\tiny $_{lb}$}‚ ‚{\color{DodgerBlue3}‚तत्सम्भवं} तस्य कारणस्य सन्निधानं साधयति ‚{\tiny $_{2a6}$}‚ । देशकालाद्यपेक्षयेत्या‚{\tiny $_{lb}$}‚ध्याहर्त्तव्यं । एतदुक्तम्भ ‚{\tiny $_{6}$}‚ वति \add{।} कार्यकारणभावनिश्चयास्तिद्धं तदुत्पत्तिलक्षण ‚{\tiny $_{lb}$}‚प्रतिवस्तु यत्रैवमतादृश उपलभ्यते तत्रैव स्वसत्तामात्रेण देशकालाद्यपेक्षया ‚{\tiny $_{lb}$}‚तत् स्वकारणङ्गमयतीति । किं ‚{\tiny $_{7}$}‚ कारणमित्याह \add{।} ‚{\color{DodgerBlue3}‚कार्यस्य कारणाव्यभि‚{\tiny $_{lb}$}‚चारादि} ‚{\tiny $_{2a6}$}‚ ति । अन्यथा हि तत्तस्य कार्यमेव न स्यात् । तद्व्यभिचारात् । ‚{\tiny $_{lb}$}‚नहि यद्व्यतिरेकेन\edtext{}{\lemma{यद्व्यतिरेकेन}\Bfootnote{? ण}}यद् भवति तत्तस्य कार्यं युक्तं । कुण्डलमिव केयूर ‚{\tiny $_{8}$}‚ स्येत्यभि‚{\tiny $_{lb}$}‚सन्धिः ।
	{\color{gray}{\rmlatinfont\textsuperscript{§~\theparCount}}}
	\pend% ending standard par
      ‚{\tiny $_{lb}$}‚

	  
	  \pstart \leavevmode% starting standard par
	ननु यदि नाम धूमोऽग्निकार्यत्त्वन्न व्यभिचरति । अन्यस्य त्वङ्कुरादेः ‚{\tiny $_{lb}$}‚स्वकारणैरेव बीजादिभिरव्यभिचार इति कुत एतदित्याह । ‚{\color{DodgerBlue3}‚अव्यभिचारे चे} ‚{\tiny $_{lb}$}‚ ‚{\tiny $_{2a6}$}‚ त्यादि । इ ‚{\tiny $_{9}$}‚ \leavevmode\ledsidenote{\textenglish{13a/msK}} दमुक्तम्भवति । यदा धूमस्य स्वकारणाव्यभिचारस्त- ‚{\tiny $_{lb}$}‚दुत्पत्तेः सिद्धस्तदा बीजादिभिरात्मीयैः कारणैः सहाङ्कुरादीनां सर्व्वकार्याणां ‚{\tiny $_{lb}$}‚सदृशोऽव्यभिचारन्यायः । तथाहि ‚{\tiny $_{1}$}‚ तेपि यथोक्तप्रकाराभ्यां प्रत्यक्षानुपलम्भाभ्यां ‚{\tiny $_{lb}$}‚तत्कार्यतया सिद्धाः सन्तस्तदव्यभिचारिणः सिद्ध्यन्ति । एतत्साधनाङ्गस‚{\tiny $_{lb}$}‚मर्थनं कार्यहेतौ । एतद्विपरीतञ्चासमर्थ ‚{\tiny $_{2}$}‚ नं । तद्वादिनः पराजयाधिकरणमिति दर्श‚{\tiny $_{lb}$}‚यन्नाह । ‚{\color{DodgerBlue3}‚एवमि} ‚{\tiny $_{2a6}$}‚ त्यादि । एवमिति प्रत्यक्षानुपलम्भाभ्यां यथोक्तप्रकारा‚{\tiny $_{lb}$}‚भ्याङ्कार्यहेतावपि न केवलं पूर्व्वोक्तेन प्रकारे ‚{\tiny $_{3}$}‚ ण स्वभावहेतावसमर्थनम्वादिनः ‚{\tiny $_{lb}$}‚पराजयस्थानमित्यपि शब्दः । कस्मादेवमित्याह । ‚{\color{DodgerBlue3}‚असमर्थित} ‚{\tiny $_{2a7}$}‚ मित्यादि । ‚{\tiny $_{lb}$}‚असमर्थिते तस्मिन्कार्यकारणभावे ‚{\tiny $_{4}$}‚ लिङ्गलिङ्गिनो\add{र्}लिङ्गस्य वा तत्का‚{\tiny $_{lb}$}‚र्यत्त्वे आरब्धार्थासिद्धेरिति क्रियापदं ॥ आरब्धोऽर्थः कारणस्य सत्तासाधनं । ‚{\tiny $_{lb}$}‚तस्यासिद्धेस्तत्पराजयस्थानमिति ‚{\tiny $_{5}$}‚ प्रकृतेन सम्बन्धः ।
	{\color{gray}{\rmlatinfont\textsuperscript{§~\theparCount}}}
	\pend% ending standard par
      ‚{\tiny $_{lb}$}‚

	  
	  \pstart \leavevmode% starting standard par
	\hphantom{.}एतदेव कुत इत्याह । ‚{\color{DodgerBlue3}‚अर्थान्तरस्य} ‚{\tiny $_{2a7}$}‚ धूमादेर्भावे सत्त्वे तस्या‚{\tiny $_{lb}$}‚ \leavevmode\ledsidenote{\textenglish{18/s}} न्यादेर्भावनियमाभावात् । नियमग्रहणं यदृच्छासम्वादनिरा ‚{\tiny $_{6}$}‚ सार्थं । अर्थान्तर‚{\tiny $_{lb}$}‚स्यापि तद्भावे प्रतिबद्धस्वभावस्य भावे भवत्येव तद्भावनियम इत्यत आह । ‚{\tiny $_{lb}$}‚ ‚{\color{DodgerBlue3}‚तद्भावाप्रतिबद्धस्वभावस्ये} ‚{\tiny $_{2a7}$}‚ ति । तद्भावेऽग्न्यादिभावेऽप्रतिबद्धोऽना‚{\tiny $_{lb}$}‚य ‚{\tiny $_{7}$}‚ त्तः स्वभावोऽस्येति विग्रहः । एतच्चार्थान्तरस्य भावे तद्भावनियमाभावादि‚{\tiny $_{lb}$}‚त्येतस्य कारणमवगन्तव्यं । प्रयोगः पुनर्योऽर्थान्तरभूतो यस्मिन्न प्रतिबद्धस्वभाव ‚{\tiny $_{8}$}‚ ‚{\tiny $_{lb}$}‚स्तस्य भावे न तद्भावनियमः । तद्यथा नूपुरस्य भावे मुकुटस्य । अर्थान्तरभूत‚{\tiny $_{lb}$}‚श्चायं धूमादिरप्रतिबद्धस्वभावस्तस्मिन्नग्न्यादाविति व्यापकानुपलब्धिः । व्यापक‚{\tiny $_{lb}$}‚विरु ‚{\tiny $_{9}$}‚ \leavevmode\ledsidenote{\textenglish{13b/msK}} द्धोपलब्धिविधिना वा हेत्वर्थकल्पनातद्भावाप्रतिबद्धस्वभावत्त्वमेव कुत ‚{\tiny $_{lb}$}‚इत्याह \add{।} ‚{\color{DodgerBlue3}‚कार्यत्त्वासिद्धे} ‚{\tiny $_{2a7}$}‚ रिति । एतत्पुनरसमर्थिते तस्मि‚{\tiny $_{lb}$}‚नि\edtext{}{\lemma{नि}\Bfootnote{? न्नि}}ति बोद्धव्यं । तत् च परमार्थतस्ते ‚{\tiny $_{1}$}‚ न कार्यहेतुरेवोपात्तस्त्तद्यदि नाम त‚{\tiny $_{lb}$}‚त्कार्यन्तेन न समर्थितं तथाहि कथमसौ निगृहीत इत्याह । ‚{\color{DodgerBlue3}‚वस्तुतः कार्यस्याप्युपादान‚{\tiny $_{lb}$}‚प्रतिपादनादि} ‚{\tiny $_{2a8}$}‚ ति तत्कार्यत्वस्येति वि ‚{\tiny $_{2}$}‚ शेषः ॥ ० ॥
	{\color{gray}{\rmlatinfont\textsuperscript{§~\theparCount}}}
	\pend% ending standard par
      ‚{\tiny $_{lb}$}‚

	  
	  \pstart \leavevmode% starting standard par
	\hphantom{.}एवं कार्यहेतावपि साधनाङ्गसमर्थनमभिधायानुपलब्धावाह । ‚{\color{DodgerBlue3}‚अनुपलब्धा‚{\tiny $_{lb}$}‚वपि समर्थन} ‚{\tiny $_{2a8}$}‚ मिति सम्बन्धः । साधनाङ्स्येत्यध्याहार्य । किं पुनस्त‚{\tiny $_{lb}$}‚दित्या ‚{\tiny $_{3}$}‚ ह । अनुपलब्धिसाधनं । किं यस्य कस्यचिन्नेत्याह । ‚{\color{DodgerBlue3}‚उपलब्धिलक्षणप्रा‚{\tiny $_{lb}$}‚प्तस्येति} ‚{\tiny $_{2a8}$}‚ । दृश्यस्वभावस्य नान्यस्येति यावत् । उपलब्धिर्ज्ञानं । उपल‚{\tiny $_{lb}$}‚ब्धिशब्दस्य भा ‚{\tiny $_{4}$}‚ वकरणसाधनतया ज्ञानपर्यायत्त्वात्तस्या लक्षणङ्कारणं । लक्षण‚{\tiny $_{lb}$}‚शब्दस्य करणसाधनत्वेन कृतकाभिधायित्त्वात् । तच्च प्रत्ययान्तरसाकल्यं स्वभाव‚{\tiny $_{lb}$}‚वि ‚{\tiny $_{5}$}‚ शेषश्च । तद्व्याप्तस्यानुपलब्धिः । तस्याः साधनं प्रतिपादनमिति व्युत्पत्ति‚{\tiny $_{lb}$}‚क्रमः ।
	{\color{gray}{\rmlatinfont\textsuperscript{§~\theparCount}}}
	\pend% ending standard par
      ‚{\tiny $_{lb}$}‚

	  
	  \pstart \leavevmode% starting standard par
	\hphantom{.}कथमेवंविधस्यानुपलब्धिरिति चेत् । ‚{\color{DodgerBlue3}‚नोद्यते तस्य तत्रैवेत्} यपित्वन्य ‚{\tiny $_{6}$}‚ त्र ‚{\tiny $_{lb}$}‚तज्जातीयस्य । कस्य पुनरेवं विधस्यानुपलब्धिः प्रसाध्यत इत्याह । ‚{\color{DodgerBlue3}‚प्रतिपत्तुः} ‚{\tiny $_{lb}$}‚ ‚{\tiny $_{2a9}$}‚ प्रतिवादिनः । यदि चोपलब्धिलक्षणप्राप्ताः पिशाचादयोपि भवन्ति ‚{\tiny $_{lb}$}‚ \leavevmode\ledsidenote{\textenglish{19/s}} तज्जातीयानां ‚{\tiny $_{7}$}‚ अन्येषाम्प्रभाववता वा । तक्तिन्तेषामप्यनुपलब्धिसाधनं साध‚{\tiny $_{lb}$}‚नाङ्गसमर्थनम्भवति । नेत्याह । प्रतिपत्तुः । एतदुक्तम्भवति \add{।} य एवासौ प्रति‚{\tiny $_{lb}$}‚पाद्यस्तस्यैव यदुपलब्धिलक्षणञ्चा ‚{\tiny $_{8}$}‚ याति तस्यैवानुपलब्धिसाधनं नान्यस्येति । किं ‚{\tiny $_{lb}$}‚पुनः कारणमेवम्प्रकारस्यैवानुपलब्धिसाधनं । नान्यस्येत्याह । ‚{\color{DodgerBlue3}‚तादृश्या एवा- ‚{\tiny $_{lb}$}‚नुपलब्धेरसद्व्यवहारसिद्धेरिति} ‚{\tiny $_{2a8}$}‚ । अ ‚{\tiny $_{9}$}‚ \leavevmode\ledsidenote{\textenglish{14a/msK}} नुपलब्धिलक्षणप्राप्तानुपलब्धेः ‚{\tiny $_{lb}$}‚संशयहेतुतया अगमकत्त्वादिति भावः । असद्व्यवहारसिद्धेरिति वचनमसद्व्यव‚{\tiny $_{lb}$}‚हार एव तया साध्यते न त्वभावः स्वभावानु ‚{\tiny $_{1}$}‚ पलब्धेः स्वयमभावरूपत्वादिति ‚{\tiny $_{lb}$}‚प्रदर्शनार्थं । असद्व्यवहारग्रहणञ्चोपलक्षणार्थं । तेनासज्ज्ञानशब्दावपि ग्राह्यौ । ‚{\tiny $_{lb}$}‚एतत्पुनः कुतोऽवसीयते इति चेदाह ॥ ‚{\tiny $_{2}$}‚ ‚{\color{DodgerBlue3}‚अनुपलब्धिलक्षणप्राप्तस्यार्थस्य ‚{\tiny $_{lb}$}‚प्रतिपत्तुः प्रत्यक्षं तदेवोपलब्धिस्तस्यानिवृत्तावपि सत्यामभावासिद्धेः} ‚{\tiny $_{2a9}$}‚ । ‚{\tiny $_{lb}$}‚अभावग्रहणमभावव्यवहारशब्दज्ञानोपलक्ष ‚{\tiny $_{3}$}‚ णं । उपलब्धिलक्षणञ्च ज्ञानपरिग्रहेण ‚{\tiny $_{lb}$}‚तत्प्रमितवस्तुव्युदासाय । का पुनरियमुपलब्धिलक्षणप्राप्तिर्यद्योगादुपलब्धि‚{\tiny $_{lb}$}‚लक्षणप्राप्त इत्युच्यत इ ‚{\tiny $_{4}$}‚ त्याह । ‚{\color{DodgerBlue3}‚तत्रेत्यादि} ‚{\tiny $_{2a9}$}‚ । तत्र श्रुतिवचनोपन्या‚{\tiny $_{lb}$}‚सार्था । स्वभावविशेषः । किमियदेव । नेत्याह । कारणान्तरसाकल्यञ्च । तस्मा‚{\tiny $_{lb}$}‚त्स्वभावविशेषाद्यान्यन्यानि ‚{\tiny $_{5}$}‚ कारणानीन्द्रियमनस्कारादीनि तानि कारणान्त‚{\tiny $_{lb}$}‚राणि तेषां । साकल्यं सामग्र्यं । स्वभावविशेषापेक्षया समुच्चयार्थश्चकारः । ‚{\tiny $_{lb}$}‚कः पुनरयं स्वभाव ‚{\tiny $_{6}$}‚ विशेष इत्याह । ‚{\color{DodgerBlue3}‚स्वभाव इत्यादि} ‚{\tiny $_{2a9}$}‚ । यद्ययं त्रिविधेन ‚{\tiny $_{lb}$}‚विप्रकर्षेण व्यवधानेन देशकालस्वभावलक्षणेन न विप्रकृष्टं मेरुरामसुरादिरूपवत् ‚{\tiny $_{lb}$}‚स्वभाववि ‚{\tiny $_{7}$}‚ शेष उच्यते । तमेव स्पष्टयति । ‚{\color{DodgerBlue3}‚यदि} ‚{\tiny $_{2a10}$}‚ त्यादिना । न आत्म‚{\tiny $_{lb}$}‚रूपोऽनात्मरूपः पररूप इत्यर्थः । स चासौ प्रतिभासश्च तस्य विवेकोऽभावस्तेना‚{\tiny $_{lb}$}‚कारेण यत्प्रतिप ‚{\tiny $_{8}$}‚ त्तुः प्रत्यक्षन्तत्राप्रतिभासितुं शीलं यस्य रूपस्य स्वभावस्य ‚{\tiny $_{lb}$}‚तद्रूपन्तथोक्तं । अथवा रूपशब्देण\edtext{}{\lemma{रूपशब्देण}\Bfootnote{? न}}सह विशेषणसमासः कार्यः । सत्युप‚{\tiny $_{lb}$}‚लम्भप्रत्ययान्तरसा ‚{\tiny $_{9}$}‚ \leavevmode\ledsidenote{\textenglish{14b/msK}} कल्य इत्युपस्क्रियते । यः सजातीयविजातीयरहितेनात्मना ‚{\tiny $_{lb}$}‚प्रतिभासते स्वज्ञाने तदन्यकारणसमवधाने सति स स्वभाव इति यावत् । तादृश ‚{\tiny $_{lb}$}‚इति त्रिविध ‚{\tiny $_{1}$}‚ विप्रकर्षाविप्रकृष्टरूपः पदार्थस्तथाऽनात्मरूपप्रतिभासविवेकेन प्रति‚{\tiny $_{lb}$}‚पत्तृप्रत्यक्षप्रतिभासेनाशयेनानुपलब्धः स न असद्व्यवहारस्य विषयो भवति । ‚{\tiny $_{2}$}‚ ‚{\tiny $_{lb}$}‚असद्व्यवहारप्रतिपत्तियोग्यो भवतीत्यर्थः । विद्यमानोपीन्द्रियस्यालोकस्य मनस्का‚{\tiny $_{lb}$}‚ \leavevmode\ledsidenote{\textenglish{20/s}} रस्य वाऽभावान्नोपलभ्यते तादृशस्तत्कथमसद्व्यवहारविषयो भवतीति ‚{\tiny $_{3}$}‚ चेदाह । ‚{\tiny $_{lb}$}‚ ‚{\color{DodgerBlue3}‚सत्स्वन्येषूपलम्भ} ‚{\tiny $_{2a10}$}‚ कारणेष्विति ।
	{\color{gray}{\rmlatinfont\textsuperscript{§~\theparCount}}}
	\pend% ending standard par
      ‚{\tiny $_{lb}$}‚

	  
	  \pstart \leavevmode% starting standard par
	नन्वविप्रकृष्टोपि घटादिरुपलम्भकारणान्तरसमवधानेपि च सन्तानविप‚{\tiny $_{lb}$}‚रिणामापेक्षत्वान्नोपलभ्यते । ‚{\tiny $_{4}$}‚ नहि हेत्वन्तरसन्निधानमिति स्वफलोत्पादनानु‚{\tiny $_{lb}$}‚गुणः परिणामो भवति कारणस्य \add{।} तथाहि सत्यामपि पृथिवीबीजजलादि‚{\tiny $_{lb}$}‚सामग्र्यामतिबहुनैव ‚{\tiny $_{5}$}‚ कालेन तालबीजस्य स्वकार्योदयानुकूला परिणतिर्भवति । ‚{\tiny $_{lb}$}‚शणादिबीजस्य त्वनन्तरमेव तथात्रापि भविष्यतीति । किञ्चान्यत्प्रभाववता‚{\tiny $_{lb}$}‚योगे पि ‚{\tiny $_{6}$}‚ शाचमायाकारादिनाऽधिष्ठितो भवति यदायं भावस्तदा विद्यमानोपि नोप‚{\tiny $_{lb}$}‚लभ्यते तत्कथमुक्तं तादृशः सत्स्वन्येषूपलम्भकारणेष्वनुपलब्धोऽसद्व्यवहारविष ‚{\tiny $_{7}$}‚ य ‚{\tiny $_{lb}$}‚इति । नूनम्भवा ‚{\color{DodgerBlue3}‚न्न्यायविन्दा} \add{व}\edtext{\textsuperscript{*}}{\lemma{*}\Bfootnote{न्यायविन्दौ द्वितीयपरिच्छेदे लिङ्गस्य त्रिषु भेदेष्वेकः ।}} ‚{\tiny $_{1}$}‚ प्यकृतपरिश्रमः । तथाहि अत्रोक्तं ‚{\color{DodgerBlue3}‚स्वभावो यः ‚{\tiny $_{lb}$}‚सत्स्वन्येषूपलम्भकारणेषु सन्प्रत्यक्ष एव भवती} ति । यश्चायं सन्तानपरिणामम ‚{\tiny $_{8}$}‚ पेक्षते ‚{\tiny $_{lb}$}‚यश्च प्रभाववताधिष्ठितः स स्वभावविशेष एव न भवति । सकलतदन्योपलम्भ‚{\tiny $_{lb}$}‚प्रत्ययसमवधानेपि स्वरूपविषयोपलम्भजनकत्वात्तथैवंविधस्य पि ‚{\tiny $_{9}$}‚ \leavevmode\ledsidenote{\textenglish{15a/msK}} शाचादि‚{\tiny $_{lb}$}‚स्वभावाविशिष्टरूपस्याभावव्यवहारविषयता साध्यते । किन्तर्हीन्द्रियाण्युपलम्भ‚{\tiny $_{lb}$}‚प्रत्ययान्तरसन्निधाने यः सन्प्रत्यक्ष एव भवति तस्य । नैवन्तर्हि सर्वथाऽ ‚{\tiny $_{1}$}‚ भावः ‚{\tiny $_{lb}$}‚साधितो भवतीति चेत् सुष्ट्वनुकूलमाचरसि । यतोऽनन्तरमेवोक्तं । ‚{\color{DodgerBlue3}‚य एवायमनु‚{\tiny $_{lb}$}‚पहतेन्द्रियादिसाकल्ये दर्शनपथमुपयाति । तस्य} च तत्साकल्येऽनुप ‚{\tiny $_{2}$}‚ लम्भेस्य च ‚{\tiny $_{lb}$}‚व्यवहारविषयता साध्यते न तु पिशाचादिस्वभावाविशिष्टरूपस्येति । न च तथा‚{\tiny $_{lb}$}‚विधस्यापि सकलतदन्योपलम्भप्रत्ययसमवधानेऽनुपलब्धस्या ‚{\tiny $_{3}$}‚ स्तित्वं युक्त‚{\tiny $_{lb}$}‚मनुपलब्धेरेवायोगात् । उपलम्भजनने कस्यचिदपेक्षणीयस्याभावात् । ‚{\color{DodgerBlue3}‚प्रमाण‚{\tiny $_{lb}$}‚विनिश्चये} तु स्पष्टीकृतमेवेदं । ‚{\color{DodgerBlue3}‚न कार्यकालेऽभावप्रति ‚{\tiny $_{4}$}‚ पत्ते} रित्यादिना । एतेनैव ‚{\tiny $_{lb}$}‚यदेकेनावश्यं सामग्रिसाकल्येपि परिणामस्तालगणबीजवदित्यादिना स तमति‚{\tiny $_{lb}$}‚शयवत् मतिमतो मनागप्यनव ‚{\tiny $_{5}$}‚ गच्छन्तश्चोद्यचुञ्चवश्चोचुदुस्तत्र सर्वंमयं ‚{\tiny $_{lb}$}‚दुःस्थितं वेदितव्यमित्यलमप्रतिष्ठितबालप्रलापैरिति विरम्यते । तस्मादुपलब्धि‚{\tiny $_{lb}$}‚लक्षणप्राप्तानुपल ‚{\tiny $_{6}$}‚ ब्धिरेवाभावव्यवहारसाधनीति स्थितमेतत् । यतश्चैतदेवं ‚{\tiny $_{lb}$}‚ततस्तस्मात्कारणादन्यथा सति लिङ्गे समवाय उपलब्धिलक्षणप्राप्तानुपलब्धि‚{\tiny $_{lb}$}‚ \leavevmode\ledsidenote{\textenglish{21/s}} मुक्त्वा यदन्यदसद्व्य ‚{\tiny $_{7}$}‚ वहारसाधनमनुपलब्धिमात्रं लिङ्गमुपादीयते । तदा ‚{\tiny $_{lb}$}‚तस्मिन्सति संशयो भवति नास्त्यसद्व्यवहारनिश्चयः उपलब्धिनिवृत्तावप्यर्था‚{\tiny $_{lb}$}‚भावासिद्धेरिति समुदायार्थ ‚{\tiny $_{8}$}‚ \add{ः ।} यदि वा ततो दृश्यानुपलम्भाल्लिङ्गात् ‚{\tiny $_{lb}$}‚सकाशादन्यथा सति लिङ्गे संशय इति व्याख्यातव्यं । तस्माच्छब्दस्तु पूर्वमध्या ‚{\tiny $_{lb}$}‚हर्त्तव्यं । अथवा तत उपलब्धिलक्षणप्राप्ताद ‚{\tiny $_{9}$}‚ \leavevmode\ledsidenote{\textenglish{15b/msK}} न्यथा तद्विहीने संशये सति तल्लिङ्ग ‚{\tiny $_{lb}$}‚इति व्याख्येयं । का पुनरत्रानुपलब्धौ व्याप्तिरित्याह । ‚{\color{DodgerBlue3}‚अत्रापीत्यादि} ‚{\tiny $_{2b11}$}‚ । ‚{\tiny $_{lb}$}‚एवं विधमिति दृश्यं सदनुपलब्धं \add{।} सर्व्वग्रहणं सर्व्वोपसं ‚{\tiny $_{1}$}‚ हारेण व्याप्तिप्रदर्शनार्थं ॥
	{\color{gray}{\rmlatinfont\textsuperscript{§~\theparCount}}}
	\pend% ending standard par
      ‚{\tiny $_{lb}$}‚

	  
	  \pstart \leavevmode% starting standard par
	ननु यदि नाम कस्यचिद्विषाणादेः शशमस्तकादावुपलब्धिलक्षणप्राप्तानु‚{\tiny $_{lb}$}‚पलब्धस्यासद्व्यवहारविषयता । अन्येनापि सामान्ये वि ‚{\tiny $_{2}$}‚ शेष्येवयविद्रव्यसंयोग‚{\tiny $_{lb}$}‚विभागादिना तथाविधेन तथा भवितव्यमिति कुतोऽयं नियम इत्यत आह । ‚{\color{DodgerBlue3}‚कस्य} ‚{\tiny $_{lb}$}‚चिदि ‚{\tiny $_{2b1}$}‚ त्यादि । कस्यचिदुपलब्धिलक्षणप्राप्तस्या ‚{\tiny $_{3}$}‚ नुपलब्धस्य शशविषा‚{\tiny $_{lb}$}‚णादेरसतोऽसद्व्यवहारविषयेस्त्यभ्युपगमेऽसद्व्यवहारादिविषयोऽसन्नित्युक्तः । ‚{\tiny $_{lb}$}‚ ‚{\color{DodgerBlue3}‚तल्लक्षणाविशेषादि ‚{\tiny $_{2b1}$}‚ ति} । तस्या ‚{\tiny $_{4}$}‚ सतो लक्षणं निमित्तं यथोक्तानुपलब्धि‚{\tiny $_{lb}$}‚र्लक्षणशब्दश्च करणसाधनस्तस्याविशिष्टत्वात् सामान्यविशेषावयविद्रव्यादा‚{\tiny $_{lb}$}‚विति वाक्यशेषः । एत ‚{\tiny $_{5}$}‚ दुक्तम्भवति शशविषाणादेरप्यसद्व्यवहारविष\add{य} ‚{\tiny $_{lb}$}‚त्त्वं कस्मादिष्यते । यथोक्तानुपलम्भस्य तन्निमित्तस्य सद्भावादिति चेत् । यद्येवं ‚{\tiny $_{lb}$}‚सामान्यविशे ‚{\tiny $_{6}$}‚ षता तस्यास्तीति कस्मात्तथा सद्व्यवहारविषयत्वन्नाभ्युपगम्यते \add{।} ‚{\tiny $_{lb}$}‚अन्यथा तत्रापि तत्स्याच्चेत् । नहि पुरुषेच्छावशाद्धेतौ विषयप्रविभागो युक्त ‚{\tiny $_{lb}$}‚इति । ‚{\color{DodgerBlue3}‚नही} ‚{\tiny $_{2b1}$}‚ त्या ‚{\tiny $_{7}$}‚ दिनैतदेव व्यनक्ति । एवंविधस्य दृश्यस्य सत्त्वेऽनुपलब्धस्या ‚{\tiny $_{lb}$}‚ ‚{\color{DodgerBlue3}‚सत्वानभ्युपगम} ‚{\tiny $_{2b1}$}‚ इति । असद्व्यवहारादिविषयत्वान्नाभ्युपगम इत्यर्थः ‚{\tiny $_{lb}$}‚असत्वशब्देना ‚{\tiny $_{8}$}‚ सद्व्यवहारो विनिश्चयस्तस्योपलक्षणम् । युक्तोपलम्भस्य तस्यैवा ‚{\tiny $_{lb}$}‚नुपलम्भनं प्रतिषेधहेतुरित्यादि चेत् । अन्यत्र शशशृङ्गाभावे दण्डेन पुरुषस्य \add{?} ‚{\tiny $_{lb}$}‚योगः स ‚{\tiny $_{9}$}‚ \leavevmode\ledsidenote{\textenglish{16a/msK}} एव इत्यर्थः \add{।} नह्येवंविधस्य दृश्यस्य चक्षुरादिशून्येषूपलम्भकारणेषु ‚{\tiny $_{lb}$}‚स अनुपलब्धिर्भवति ‚{\tiny $_{1}$}‚ ॥
	{\color{gray}{\rmlatinfont\textsuperscript{§~\theparCount}}}
	\pend% ending standard par
      ‚{\tiny $_{lb}$}‚

	  
	  \pstart \leavevmode% starting standard par
	किन्तर्ह्युपलब्धिरेव भवतीति प्रतिषेधद्वयेनाह । अन्यस्योपलब्धिप्रत्ययस्य ‚{\tiny $_{lb}$}‚कस्यचिवपेक्षणीयस्याभावादिति भावः । तदनेन प्रकृतमेव स्पष्टयति । अनुपल ‚{\tiny $_{2}$}‚ ‚{\tiny $_{lb}$}‚भ्यमानं त्वीदृशमित्युपलब्धिलक्षणप्राप्तन्नास्ति तस्मादेतावत्सात्र उपलब्धिलक्षण‚{\tiny $_{lb}$}‚प्राप्तानुपलब्धिमात्रन्निमित्तं यस्यासद्व्यवहारस्य स तथा ख्यातः तदनेनासत् ‚{\tiny $_{3}$}‚ ‚{\tiny $_{lb}$}‚ \leavevmode\ledsidenote{\textenglish{22/s}} व्यवहारस्यानन्यनिमित्ततामाह । एतदेव कुत इत्याह । ‚{\color{DodgerBlue3}‚अन्यस्ये} ‚{\tiny $_{2b2}$}‚ त्यादि । ‚{\tiny $_{lb}$}‚यथोक्तानुपलब्धिमपास्यान्यस्य नास्तित्वव्यवहारनिमित्तस्याभावादिति स‚{\tiny $_{4}$}‚म\edtext{}{\lemma{म}\Bfootnote{? मु}} ‚{\tiny $_{lb}$}‚च्चयार्थः ।
	{\color{gray}{\rmlatinfont\textsuperscript{§~\theparCount}}}
	\pend% ending standard par
      ‚{\tiny $_{lb}$}‚

	  
	  \pstart \leavevmode% starting standard par
	ननु च यस्य यत्र न किञ्चित्सामर्थ्यमस्ति तदसद्व्यवहारविषयो यथा नभस्तले ‚{\tiny $_{lb}$}‚कमलं । तथाभिमतेपि देशादावभिमतस्य भावस्य ‚{\tiny $_{5}$}‚ न किञ्चित्सामर्थ्यमस्तीति ‚{\tiny $_{lb}$}‚सर्व्वसामर्थ्यविवेक एव नास्तित्वव्यवहारस्य निमित्तं भविष्यति । एतक्तिमसं‚{\tiny $_{lb}$}‚बद्धमेवोद्घाटितशिरोभिरभिधीयते । अ ‚{\tiny $_{6}$}‚ न्यस्य तन्निमित्तस्याभावादिति कदाचित् ‚{\tiny $_{lb}$}‚कश्चित् ब्रूयादिति तन्मतमाशंकते । ‚{\color{DodgerBlue3}‚सर्व्वसामर्थ्यविवेको निमित्तमिति} ‚{\tiny $_{2b2}$}‚ चेदि ‚{\tiny $_{lb}$}‚ति । अत्र समाधिमाह । ‚{\color{DodgerBlue3}‚एवमि} ‚{\tiny $_{2b2}$}‚ त्यादिना । एवं ‚{\tiny $_{7}$}‚ मन्यते सूक्तमेतत् सर्व्व‚{\tiny $_{lb}$}‚सामर्थ्यविवेको निमित्तमिति किन्तु स एव सर्व्वसामर्थ्यविवेकोयं पदार्थः कथमव‚{\tiny $_{lb}$}‚गतो यथोक्तामनुपलब्धिमपास्य येन सर्व्वसामर्थ्यवि ‚{\tiny $_{8}$}‚ वेकोऽस्यासद्व्यवहारस्य ‚{\tiny $_{lb}$}‚निमित्तम्भविष्यति । न चासावज्ञात एव तस्य निमित्तम्भवितुमर्हति । ज्ञापकहेत्व‚{\tiny $_{lb}$}‚धिकारात् । कस्मात्सर्व्वसामर्थ्यविवेकिनो यथोक्तानु ‚{\tiny $_{9}$}‚ \leavevmode\ledsidenote{\textenglish{16b/msK}} पलम्भेने\edtext{}{\lemma{पलम्भेने}\Bfootnote{नै ?}}व प्रतीति‚{\tiny $_{lb}$}‚रित्याह । ‚{\color{DodgerBlue3}‚अन्यस्य तत्प्रतिपत्युपायस्याभावा} ‚{\tiny $_{2b3}$}‚ दिति । यदा तु यथोक्तानुप‚{\tiny $_{lb}$}‚लब्ध्या तस्य सर्व्वसामर्थ्यविवेकिनः प्रतीतिर्भवति । तदा तत्प्रतिपत्तौ ‚{\tiny $_{1}$}‚ सत्याम‚{\tiny $_{lb}$}‚सद्व्यवहारो भवति । इति तस्मादिदं यथोक्तानुपलम्भनं तस्यासद्व्यवहारस्या‚{\tiny $_{lb}$}‚निमित्तमुच्यते ।
	{\color{gray}{\rmlatinfont\textsuperscript{§~\theparCount}}}
	\pend% ending standard par
      ‚{\tiny $_{lb}$}‚

	  
	  \pstart \leavevmode% starting standard par
	पुनरपि परोन्यस्य तन्निमित्तस्याभावादित्यस्य कदाचि ‚{\tiny $_{2}$}‚ दयुक्तताम्ब्रूयादि‚{\tiny $_{lb}$}‚त्याशङ्कते ‚{\color{DodgerBlue3}‚बुद्धिव्यपदेश} ‚{\tiny $_{2b3}$}‚ इत्यादिना । अयमस्याभिसन्धिर्बुद्धिव्यपदे‚{\tiny $_{lb}$}‚शार्थक्रियाभ्यः सकासा\edtext{}{\lemma{सकासा}\Bfootnote{? शा}}त्तद्व्यवहारो भवति । तथा हि ताः प्रवर्त्तमाना ‚{\tiny $_{3}$}‚ ‚{\tiny $_{lb}$}‚वस्तुसत्तां साधयन्ति । तद्भेदाभेदौ च वस्तुभेदाभेदावित्याशयः । ‚{\tiny $_{lb}$}‚ताश्च निवर्त्तमानाः स्वनिमित्तं सद्व्यवहारं निवर्तयन्त्यग्निरिव धूमं । तन्निवृत्तौ ‚{\tiny $_{4}$}‚ ‚{\tiny $_{lb}$}‚चासद्व्यवहारः । सद्व्यवहारासद्व्यवहारयोरन्योन्यव्यवच्छेदस्थितरूपत्वेन ‚{\tiny $_{lb}$}‚एकत्यागस्यापरोपादानेनान्तरीयकत्वात् । ततश्च बुद्ध्यादिनिवृत्तौ ‚{\tiny $_{5}$}‚ चासद्व्यव‚{\tiny $_{lb}$}‚हारनिमित्तमिति नेदं युक्तं वक्तुमन्यस्येत्यादि । अत्रापि प्रतिविधानमाह । ‚{\color{DodgerBlue3}‚भव}‚{\tiny $_{lb}$}‚ \leavevmode\ledsidenote{\textenglish{23/s}} ती ‚{\tiny $_{2b4}$}‚ त्यादि । यथोक्तप्रतिभाषा\edtext{}{\lemma{यथोक्तप्रतिभाषा}\Bfootnote{? सा}}बुद्धिः प्रतिपत्तृप्रत्यक्षप्र ‚{\tiny $_{6}$}‚ तिभासि‚{\tiny $_{lb}$}‚रूपनिर्भासा यथोक्तः प्रतिभासो यस्या इति विग्रहः । उक्तञ्च यदनात्मेत्यादिना । ‚{\tiny $_{lb}$}‚तस्याः सकासा\edtext{}{\lemma{सकासा}\Bfootnote{? शा}}त्सद्व्यवहारो भवति साक्षा\add{द्} वस्तुग्रहणात् । तस्या ‚{\tiny $_{7}$}‚ ञ्च ‚{\tiny $_{lb}$}‚विपर्ययोऽभावस्तस्मिन् सत्यसद्व्यवहारो भवति । सत्स्वन्येषूपलम्भकारणेष्विति ‚{\tiny $_{lb}$}‚वाक्यपरिसमाप्तिः कार्याः\edtext{}{\lemma{कार्याः}\Bfootnote{? र्या}}। अन्यथा संस\edtext{}{\lemma{संस}\Bfootnote{? श}}योत्पत्तेः । नहि वस्तु‚{\tiny $_{lb}$}‚सत्व उपलं ‚{\tiny $_{8}$}‚ भप्रत्ययान्तरसाकल्ये च सा निवर्त्तत इति ‚{\tiny $_{9}$}‚ \leavevmode\ledsidenote{\textenglish{17a/msK}} निवेदितमेतत् पुरोऽस्मा- ‚{\tiny $_{lb}$}‚भिरितिभावः । तवनेन यद्येवंविधा बुद्धिरभिमता त्वया तदावयोरैकमत्यमेव ‚{\tiny $_{lb}$}‚तथापि न नः किञ्चिदनिष्टमुक्तं स्यात । अथान्या\edtext{}{\lemma{अथान्या}\Bfootnote{? न्यः}}तदा व्य ‚{\tiny $_{1}$}‚ भिचार ‚{\tiny $_{lb}$}‚इति दर्शयति । तमेव व्यभिचारन्दर्शयन्नाह । ‚{\color{DodgerBlue3}‚प्रत्यक्षाविषये} ‚{\tiny $_{2b4}$}‚ त्यादिना । ‚{\tiny $_{lb}$}‚लिङ्गाज्जाता लिङ्गजाः । अनुमानमित्यर्थः । तस्याः सकासा\edtext{}{\lemma{सकासा}\Bfootnote{? शा}}त्सद्व्यवहारः ‚{\tiny $_{lb}$}‚स्यात्परो ‚{\tiny $_{2}$}‚ क्षेऽर्थेन केवलमनन्तरोदितरूपायाः स्वग्राह्य इत्यपिनाह । किं ‚{\tiny $_{lb}$}‚लिङ्गजायाः सर्व्वस्याः सम्भवति नेत्याह । ‚{\color{DodgerBlue3}‚कुतश्चिदि} ‚{\tiny $_{2b4}$}‚ ति \add{।} स्वभाव‚{\tiny $_{lb}$}‚कार्यलिङ्गद्वयबलोपजाता ‚{\tiny $_{3}$}‚ या इत्यर्थः । अनुपलम्भस्यासत्ताऽसद्व्यवहार‚{\tiny $_{lb}$}‚साधकत्वादिति भावः । यदि नामे\edtext{}{\lemma{नामे}\Bfootnote{? मै}}वं ततः कथम्व्यभिचार इति चेदाह । ‚{\tiny $_{lb}$}‚ ‚{\color{DodgerBlue3}‚असद्व्यवहारस्त्वि} ‚{\tiny $_{2b4}$}‚ त्या ‚{\tiny $_{4}$}‚ दि । तद्विपर्य इति तस्या यथोक्तलिङ्गजाया ‚{\tiny $_{lb}$}‚बुद्धेर्विनिवृत्तावनैकान्तिकः सन्दिग्ध इत्यर्थः । किं कारणं विप्रकृष्टेर्थे देशादि‚{\tiny $_{lb}$}‚विप्रकर्षैः ‚{\tiny $_{5}$}‚ प्रतिपत्तृप्रत्यक्षस्य प्रमाणस्य निवृत्तावपि संशयात्कारणात् । अर्था‚{\tiny $_{lb}$}‚भाव इति शेषः \add{।} प्रतिपत्तुः प्रत्यक्षमिति षष्ठीसमासः । इदञ्च प्रही ‚{\tiny $_{6}$}‚ णसकलज्ञे‚{\tiny $_{lb}$}‚यावरणस्य प्रत्यक्षनिवृत्तौ त्वसंदेह एवेति कथनायोपात्तं । अन्यस्य चेत्यनुमानस्या‚{\tiny $_{lb}$}‚गमस्य च । एतच्चागमस्य प्रामाण्यमभ्युपगम्याभिधीय ‚{\tiny $_{7}$}‚ ते । न तु तस्य प्रामाण्य‚{\tiny $_{lb}$}‚मस्ति \add{।}
	{\color{gray}{\rmlatinfont\textsuperscript{§~\theparCount}}}
	\pend% ending standard par
      ‚{\tiny $_{lb}$}‚
	  \bigskip
	  \begingroup
	
	    
	    \stanza[\smallbreak]
	  \flagstanza{\tiny\textenglish{...4}}{\normalfontlatin\large ``\qquad}नान्तरीयकताभावाच्छब्दानाम्वस्तुभिः सह \add{।}&‚{\tiny $_{lb}$}‚नार्थसिद्धिस्ततस्ते हि वक्त्रभिप्रायसूचका \add{ः ॥ ४}{\normalfontlatin\large\qquad{}"}\&[\smallbreak]
	  
	  
	  
	  \endgroup
	‚{\tiny $_{lb}$}‚

	  
	  \pstart \leavevmode% starting standard par
	इत्यादिवचनात् ।
	{\color{gray}{\rmlatinfont\textsuperscript{§~\theparCount}}}
	\pend% ending standard par
      ‚{\tiny $_{lb}$}‚

	  
	  \pstart \leavevmode% starting standard par
	अयमस्याभिप्रायो यदि ‚{\tiny $_{8}$}‚ नाम प्रमाणत्रयन्निवृत्तमप्रत्यक्षवस्तुनि तथापि ‚{\tiny $_{lb}$}‚तन्नास्तीति कुतोयन्निश्चयः । तथाहि मलयनगनिकुञ्जवर्त्तिदूर्व्वाप्रवालपत्र‚{\tiny $_{lb}$}‚प्रभृतयः प्रमाणत्रयगोच ‚{\tiny $_{9}$}‚\leavevmode\ledsidenote{\textenglish{17b/msK}} रभावातिक्रान्तां मूर्तिमुद्वहन्तस्तिष्ठन्ति । न च ते न ‚{\tiny $_{lb}$}‚ \leavevmode\ledsidenote{\textenglish{24/s}} सन्तीति शक्यमभिधातुं । प्रमाणभावस्य सकलविषयसत्वव्यापकत्वकारणत्वा‚{\tiny $_{lb}$}‚भावात् । न च तद्रूपविकलपदार्थनि ‚{\tiny $_{1}$}‚ वृत्तावन्यनिवृत्तिनियमेनातिप्रसङ्गदोषो‚{\tiny $_{lb}$}‚पनिपातादित्यावेदितमेतः\edtext{}{\lemma{पनिपातादित्यावेदितमेतः}\Bfootnote{? तत्}}पुरस्तात् । यदाह । ‚{\tiny $_{lb}$}‚ 
	    \pend% close preceding par
	  
	    
	    \stanza[\smallbreak]
	  \flagstanza{\tiny\textenglish{...5}}{\normalfontlatin\large ``\qquad}शास्त्राधिकारासम्बद्धा बहवोऽर्थाकृतेन्द्रियाः ।&‚{\tiny $_{lb}$}‚अलिङ्गाश्च कथन्तेषामभावोऽनुपलम्भता \add{॥ ५}{\normalfontlatin\large\qquad{}"}\&[\smallbreak]
	  
	  
	  
	    \pstart  \leavevmode% new par for following
	    \hphantom{.}
	   इति ।
	{\color{gray}{\rmlatinfont\textsuperscript{§~\theparCount}}}
	\pend% ending standard par
      ‚{\tiny $_{lb}$}‚

	  
	  \pstart \leavevmode% starting standard par
	ननु चात्र लिङ्गजाया मतेरसद्व्यवहारहेतुत्त्वं निषेद्धुमारब्धन्तद्विपर्य्यय ‚{\tiny $_{lb}$}‚इत्याद्यभिधानात् । तत्कस्मादप्रस्तुतस्यैव प्रत्यक्षस्य चागम ‚{\tiny $_{3}$}‚ स्य चोपक्षेपः कृत इति‚{\tiny $_{lb}$}‚चेत् । युक्तमेतत् । सर्वप्रमाणनिवृत्तेस्त्वगमकत्वप्रदर्शनेनैतदेकान्तासम्भवदर्श‚{\tiny $_{lb}$}‚नायोक्तमिति लक्ष्यतेऽस्य सुधियोऽ ‚{\tiny $_{4}$}‚ भिसन्धिः ।
	{\color{gray}{\rmlatinfont\textsuperscript{§~\theparCount}}}
	\pend% ending standard par
      ‚{\tiny $_{lb}$}‚

	  
	  \pstart \leavevmode% starting standard par
	\hphantom{.}नन्विदमुक्तं ‚{\color{DodgerBlue3}‚सद्व्यवहारासद्व्यवहारयोरन्योन्यव्यवच्छेद} स्थितलक्षणत्वे ‚{\tiny $_{lb}$}‚नैकाभावस्यापरभावनान्तरीयकत्वात् विप्रकृष्टे तन्निमित्ता ‚{\tiny $_{5}$}‚ भावात् सद्व्यव‚{\tiny $_{lb}$}‚हारनिवृत्त्याऽसद्व्यवहार इति \add{।} सत्यमुक्तमेवैतन्न पुनर्युक्तं । तथाहि परोक्षेऽर्थे ‚{\tiny $_{lb}$}‚सद्व्यवहारनिवृत्तिः कथन्तन्निमित्ताभावेपि ‚{\tiny $_{6}$}‚ द्वयोरप्यनयोरनुपलब्ध्योः स्वविपर्यय‚{\tiny $_{lb}$}‚हेत्वभावभावाभ्यां सद्व्यवहारप्रतिषेधफलत्वन्तुल्यमेकत्र संशयादपरत्र विप‚{\tiny $_{lb}$}‚र्ययादिति वचनात् संशयेनेति चेत् । ‚{\tiny $_{7}$}‚ यद्येवं कथन्तर्ह्यसद्व्यवहारनिश्चयस्तत्र ‚{\tiny $_{lb}$}‚युक्तिमनुपतति । यत्र तु निश्चयेन सद्व्यवहारनिवृत्तिस्तत्रासद्व्यवहारोपि ‚{\tiny $_{lb}$}‚पुक्त एवान्यः प्रवर्तनफलोसीत्युक्तेः ‚{\tiny $_{8}$}‚ न चाभावरूपव्यवच्छेदे भावानुषङ्गोस्ति ‚{\tiny $_{lb}$}‚नियमेन । नहि बन्ध्यातनयनभःपङ्कजादिष्वसदवस्थता भवति प्रतिषेधात् ‚{\tiny $_{lb}$}‚सदवस्थता भवति प्रतिषेधमात्रन्तु स्यात् । तयो ‚{\tiny $_{9}$}‚\leavevmode\ledsidenote{\textenglish{18a/msK}} रन्योन्यव्यवच्छेदेनावस्थानात् । ‚{\tiny $_{lb}$}‚तथात्राप्यप्रत्यक्षे सद्व्यवहारप्रतिषेधान्न विधिभूता सद्व्यवहारानुस\edtext{}{\lemma{सद्व्यवहारानुस}\Bfootnote{? ष}}ङ्ग‚{\tiny $_{lb}$}‚स्तद्व्यवच्छेदमात्रन्तु स्यात् । तद्भावस्य तद्भावस्यान्योन्यपरिहारेण ‚{\tiny $_{1}$}‚ अवस्थि‚{\tiny $_{lb}$}‚तत्वात् । उक्तञ्चैतदमलन्यायतत्वप्रबोधोद्गतप्रज्ञालोकतिरस्कृताशेषपरतीर्थ्य ‚{\tiny $_{lb}$}‚प्रवादध्वान्तेन ‚{\color{DodgerBlue3}‚धर्मकीर्तिनैवानित्यनिरात्मतादिव्यवच्छेदेपि तच्च स्यादित्यादि} - ‚{\tiny $_{2}$}‚ ‚{\tiny $_{lb}$}‚नेत्यास्तान्तावत् ।
	{\color{gray}{\rmlatinfont\textsuperscript{§~\theparCount}}}
	\pend% ending standard par
      ‚{\tiny $_{lb}$}‚

	  
	  \pstart \leavevmode% starting standard par
	अधुना सामान्यभूतानां बुद्धिव्यपदेशानां सद्व्यवहारहेतुत्वमपि नास्तीति ‚{\tiny $_{lb}$}‚कथयन्नाह । ‚{\color{DodgerBlue3}‚न चे} ‚{\tiny $_{2b5}$}‚ त्यादि । तत्र च यथाक्रममभिसम्बन्धः । ते सर्व्वे बुद्धि‚{\tiny $_{lb}$}‚व्यपदे ‚{\tiny $_{3}$}‚ शा न वस्तुसत्तां साधयन्ति । तेषाम्वा भेदाभेदौ न वस्तुभेदाभेदयोः सत्ता‚{\tiny $_{lb}$}‚ \leavevmode\ledsidenote{\textenglish{25/s}} सद्भावमिति । सर्व्वग्रहणं केचित्तु साधयन्त्येवेति प्रदर्शनाय । कुत एतदित्याह । ‚{\tiny $_{lb}$}‚ ‚{\color{DodgerBlue3}‚अस ‚{\tiny $_{4}$}‚ त्स्वप्यतीतानागतादिषु} ‚{\tiny $_{2b5}$}‚ वृत्तेरिति क्रियापदं । आदिशब्देन व्योमो‚{\tiny $_{lb}$}‚त्पलादयः परिगृह्यन्ते । कथम्पुनर्विषयमन्तरेण तेषु तेषां वृत्तिर्युक्तेति चेदाह । ‚{\tiny $_{lb}$}‚ ‚{\color{DodgerBlue3}‚कथं ‚{\tiny $_{5}$}‚ चित्त} ‚{\tiny $_{2b6}$}‚ द्रूपोनुभवाहितवासनापरिपाकप्रभावादित्यर्थः । तेषाञ्च ‚{\tiny $_{lb}$}‚वस्तुप्रतिबन्धाभावादिति भावः । ‚{\color{DodgerBlue3}‚शङ्खचक्रवर्ती} ‚{\tiny $_{2b7}$}‚ त्यादिना प्रकारेण ‚{\tiny $_{lb}$}‚तदनेन वस्तुसत्तां साधयन्तीत्येतस्य कारणमाह । ‚{\color{DodgerBlue3}‚नानैकामर्थक्रिया} ‚{\tiny $_{2b6}$}‚ ‚{\tiny $_{lb}$}‚ङ्कर्तुंसी\edtext{}{\lemma{ङ्कर्तुंसी}\Bfootnote{? शी}}लं येषां ते तथोक्ताः । तेष्वपि च वृत्तेः कारणात् । किमर्थं तेषु ‚{\tiny $_{lb}$}‚तेषाम्वृत्तिस्तद्भावख्यापनाय । तेषान्नाऽ ‚{\tiny $_{7}$}‚ नैकार्थक्रियाकारिणाम्भावस्तस्य ख्या‚{\tiny $_{lb}$}‚पनाय । नानार्थक्रियाकारित्वस्यैकार्थक्रियाकारित्वस्य च कथनार्थमिति यावत् । ‚{\tiny $_{lb}$}‚अस्त्येव तर्हि तस्य वस्तुनस्तत्वमित्यत आह \add{।} ‚{\tiny $_{8}$}‚ ‚{\color{DodgerBlue3}‚नानैकात्मताया} ‚{\tiny $_{2b6}$}‚ ‚{\tiny $_{lb}$}‚अभावेपि तस्य वस्तुन इत्यधा\edtext{}{\lemma{इत्यधा}\Bfootnote{? ध्या}}हर्तव्यं । नानैकरूपाणाम्बुद्धिव्यपदेशानां ‚{\tiny $_{lb}$}‚तदनेन न वस्तुभेदाभेदौ साधयन्तीति साधयति । इदमेव निदर्शन ‚{\tiny $_{9}$}‚ \leavevmode\ledsidenote{\textenglish{18b/msK}} प्रदर्शनेन सफली- ‚{\tiny $_{lb}$}‚करोति । ‚{\color{DodgerBlue3}‚राजा महासम्मत} ‚{\tiny $_{2b6}$}‚ इदमतीतवृत्तेरुदाहरणं । यथेति चाध्याहार्यं । ‚{\tiny $_{lb}$}‚ ‚{\color{DodgerBlue3}‚शङ्ख} चक्रवर्तीत्याद्यनुत्पन्नवृत्तेः शब्दैर्विषाणमित्यादि यदोपात्तस्य रूपं सनिदर्शन‚{\tiny $_{lb}$}‚ञ्चक्षुर्विज्ञानजनकत्वात् । सप्रतिघश्च स्वदेशे परोत्पत्तिप्रतिबन्धात् । एतन्ना‚{\tiny $_{lb}$}‚नार्थक्रियाकारिषु वृत्तेरित्येतस्य निदर्शनं । यस्मात्तच्चक्षुर्विज्ञानादि ‚{\tiny $_{2}$}‚ कार्य‚{\tiny $_{lb}$}‚जनकत्वादेकरूपमपि नानारूपैः सनिदर्शनादि शब्दैर्विषयीक्रियते । तस्मान्न ते ‚{\tiny $_{lb}$}‚वस्तुभेदसाधनायालं । घटश्चेत्येतदेकार्थक्रियाकारिष्वित्येतस्योदा ‚{\tiny $_{3}$}‚ हरणन्तथाहि ‚{\tiny $_{lb}$}‚बहवो रूपगन्धरसस्पर्शा उदकधारणविशेषादिकार्यनिर्वर्त्तनसमर्थत्वादभिन्न‚{\tiny $_{lb}$}‚समैस्तै\add{ः} विषयत्वेनात्मसाक्त्रियन्ते । ततस्ते नाभेदं सा ‚{\tiny $_{4}$}‚ धयितुं क्षमाः । ‚{\tiny $_{lb}$}‚तच्च\add{ा}तीतानागतशशविषाणादिषु तत्प्रतिपत्तिर्न वस्तु साधयतीत्यतिप्रतीत‚{\tiny $_{lb}$}‚मेतत् । अथ कथमिदङ्गम्यते सनिदर्शनादिबुद्धिशब्दा ‚{\tiny $_{5}$}‚ न वस्तुभेदं साधयन्तीत्यतः ‚{\tiny $_{lb}$}‚प्राह । ‚{\color{DodgerBlue3}‚नहीत्या} ‚{\tiny $_{2b7}$}‚ दि । कस्मादेवात्र वस्तुनि रूपादावुपसंहारात्सनिदर्शनं ‚{\tiny $_{lb}$}‚सप्रतिघं रूपमित्येव समानाधिकरणत्वादिति ‚{\tiny $_{6}$}‚ यावत् । अन्यथा भिन्नाधिकरण‚{\tiny $_{lb}$}‚त्वाद्वकुलोत्पलकमलमालतीमल्लिकादिशब्दानामेव सामानाधिकरण्यमेव न ‚{\tiny $_{lb}$}‚ \leavevmode\ledsidenote{\textenglish{26/s}} भवेदिति भावः । ‚{\color{DodgerBlue3}‚कणभक्षाक्षपाद} मतानुसारिण ‚{\tiny $_{7}$}‚ स्तु मिथ्यादर्शनानुरागजनिता‚{\tiny $_{lb}$}‚सद्विकल्पमलोपलिप्तान्तर्लोचनाः सञ्चक्षते \add{।} नानाविषयत्वेप्यभ्युपगम्यमाने ‚{\tiny $_{lb}$}‚तेषामेकत्रोपसंहारोऽविरुद्ध एव । तन्निमित्तानां ‚{\tiny $_{8}$}‚ सनिदर्शनादीनान्तत्र रूपादौ ‚{\tiny $_{lb}$}‚समवायादिति ।
	{\color{gray}{\rmlatinfont\textsuperscript{§~\theparCount}}}
	\pend% ending standard par
      ‚{\tiny $_{lb}$}‚

	  
	  \pstart \leavevmode% starting standard par
	तदेतत्सर्वमेषामविचारितरमणीयतया विचारविमर्दीक्षमत्वात् पण्डितजन‚{\tiny $_{lb}$}‚हासकारि दर्शनमित्यभिप्रायवा ‚{\tiny $_{9}$}‚ \leavevmode\ledsidenote{\textenglish{19a/msK}} नाह । ‚{\color{DodgerBlue3}‚आयासे वताय} मित्या ‚{\tiny $_{2b8}$}‚दि \add{।} ‚{\tiny $_{lb}$}‚वतशब्दोऽनुकंपायाङ् कासावित्याह । ‚{\color{DodgerBlue3}‚अनेकं सम्बन्धिनं} सनिदर्शनत्वादिकमुप‚{\tiny $_{lb}$}‚कृत्यानुपकारे तेन तेषान्तत्र सम्बन्धित्वायोगादित्यभि ‚{\tiny $_{1}$}‚ प्रायः । अनेकं सनिदर्शनादि‚{\tiny $_{lb}$}‚ ‚{\color{DodgerBlue3}‚शब्दं} ‚{\tiny $_{2b9}$}‚ तेभ्यः सम्बन्धिभ्यःशकासा\edtext{}{\lemma{सम्बन्धिभ्यःशकासा}\Bfootnote{? सकाशा}}दात्मनि संमार्गयन्निद‚{\tiny $_{lb}$}‚मायासपतने कारणं ।
	{\color{gray}{\rmlatinfont\textsuperscript{§~\theparCount}}}
	\pend% ending standard par
      ‚{\tiny $_{lb}$}‚

	  
	  \pstart \leavevmode% starting standard par
	ननु सरूपादिभावो यैः शक्तिभेदैरनेकसम्बन्धिन ‚{\tiny $_{2}$}‚ मुपकरोति । तैरेव शक्ति‚{\tiny $_{lb}$}‚भेदैरनेकं बुद्ध्यादिशब्दं किन्नोत्थापयति । यदि पुनरेवं भवेत्तदाको गुणो लभ्यत ‚{\tiny $_{lb}$}‚इत्याह । ‚{\color{DodgerBlue3}‚एवं हयनेन परम्परानुसारश्रमः परिहृ ‚{\tiny $_{3}$}‚ तो भवती} ‚{\tiny $_{2b9}$}‚ ति । शक्ति‚{\tiny $_{lb}$}‚भेदैः सम्बन्धिनमुपकरोति तेभ्यश्च शब्दाः प्रवर्तन्त इत्ययम्परम्परानुसरणाया‚{\tiny $_{lb}$}‚सोऽनेनतपश्विना\edtext{}{\lemma{सोऽनेनतपश्विना}\Bfootnote{? तपस्विना}}रूपादिना त्यक्तो भवतीत्यर्थः । 
	{\color{gray}{\rmlatinfont\textsuperscript{§~\theparCount}}}
	\pend% ending standard par
      ‚{\tiny $_{lb}$}‚

	  
	  \pstart \leavevmode% starting standard par
	ननु च प्रतिनियतोपि कार्यशक्तिमन्तः सर्व्व एव भावास्त्वयाप्येतदवस्य\edtext{}{\lemma{भावास्त्वयाप्येतदवस्य}\Bfootnote{‚{\tiny $_{lb}$}‚? श्य}}मेवाभ्युपेयमन्यथा कस्माच्छालिबीजंसा\edtext{}{\lemma{कस्माच्छालिबीजंसा}\Bfootnote{? शा}}ल्यङ्कुरमेवोत्पादयति न ‚{\tiny $_{lb}$}‚यवाङ्कुरमिति ‚{\tiny $_{5}$}‚ परेणाभियुक्तेन किमभिधानीयं भावप्रकृतिं मुक्त्वा \add{।} तस्मात्तव ‚{\tiny $_{lb}$}‚पदार्थप्रकृतिसमाश्रयणमेव शरणमन्यथास्य दोषस्य परिहर्त्तुमशक्यत्वात् ‚{\tiny $_{6}$}‚ \add{।} ‚{\tiny $_{lb}$}‚एतच्च न ममापि राजकुलनिवारितं । तथाहि शक्यमे\add{त}त्मयाप्यभिधातुमनेक‚{\tiny $_{lb}$}‚सम्बन्ध्युपकार एव तस्य सामर्थ्यं नानैकशब्दोत्थापनमिति चेत् । सत्यमेवमेत ‚{\tiny $_{7}$}‚ त् । ‚{\tiny $_{lb}$}‚एवन्तु मन्यते । न तावत् सनिदर्शनत्वादयः सन्ति । क्रमयौगपद्याभ्यामर्थक्रिया त्वनु‚{\tiny $_{lb}$}‚प्रयोगात् । उपलब्धिलक्षणप्राप्तानाञ्चानुपलम्भात् न चोपलब्धिलक्ष ‚{\tiny $_{8}$}‚ णप्राप्तं ‚{\tiny $_{lb}$}‚सदनुपलभ्यमानमस्तीति शक्यते वक्तुमतिप्रसङ्गात् । अनुपलब्धिलक्षणप्राप्ततायां ‚{\tiny $_{lb}$}‚ \leavevmode\ledsidenote{\textenglish{27/s}} वा कथं तन्निबन्धनाः प्रत्ययव्यपदेशाः प्रवर्त्तन्ते गोगवयादिषु \add{।} ‚{\tiny $_{9}$}‚ \leavevmode\ledsidenote{\textenglish{19b/msK}} एतेनोपलब्धा- ‚{\tiny $_{lb}$}‚नामपि क्षणिकत्वादिवत् व्यक्तिव्यतिरिक्तेणा\edtext{}{\lemma{व्यक्तिव्यतिरिक्तेणा}\Bfootnote{? ना}}नुपलक्षणं प्रत्यक्षं यस्मात्सा‚{\tiny $_{lb}$}‚मान्यं यदि दृष्टमप्यविकलं भिन्नं न संलक्षते । भावे तद्बलभाविनी भवति ‚{\tiny $_{lb}$}‚सा ‚{\tiny $_{1}$}‚ या शब्दवृत्तिः कथं । दण्ड्यादौ न निबन्धनस्य न गतौ धीशब्दयोरस्ति सा ‚{\tiny $_{lb}$}‚तस्मादस्य कथञ्चिदेव तदपि ते युक्त्या न सङ्गच्छते ‚{\tiny $_{1}$}‚ ।
	{\color{gray}{\rmlatinfont\textsuperscript{§~\theparCount}}}
	\pend% ending standard par
      ‚{\tiny $_{lb}$}‚

	  
	  \pstart \leavevmode% starting standard par
	किञ्च ॥
	{\color{gray}{\rmlatinfont\textsuperscript{§~\theparCount}}}
	\pend% ending standard par
      ‚{\tiny $_{lb}$}‚
	  \bigskip
	  \begingroup
	
	    
	    \stanza[\smallbreak]
	  \flagstanza{\tiny\textenglish{...6}}{\normalfontlatin\large ``\qquad}भावानामैकदेश्यं प्रसजति भवतो ‚{\tiny $_{2}$}‚ दर्शने सर्वथैषां&‚{\tiny $_{lb}$}‚सत्तादेसा\edtext{}{\lemma{सत्तादेसा}\Bfootnote{? शा}}दभेदात्सकृदिदमथवा भिन्नदेशे निवृत्तं ।&‚{\tiny $_{lb}$}‚वृत्तौ वानेकमेतन्नहि भवति सकृत्सर्व्वथा वृत्तिभाजां&‚{\tiny $_{lb}$}‚तालादीनां फलानां बहुषु बहुवि‚{\tiny $_{3}$}‚धेष्वाश्रयेष्वेकभावः ॥ \add{६}{\normalfontlatin\large\qquad{}"}\&[\smallbreak]
	  
	  
	  
	  \endgroup
	‚{\tiny $_{lb}$}‚

	  
	  \pstart \leavevmode% starting standard par
	सत्वे वा तेषान्न बुद्धिशब्दोत्थापनसामर्थ्यमन्वयव्यतिरेकाभ्यामवधार्यते । ‚{\tiny $_{lb}$}‚अत एवानुपलभ्यमानत्वात्पाचकादिष्वपि च ‚{\tiny $_{4}$}‚ तद्व्यतिरेकेणापि तेषाम्भावात् । ‚{\tiny $_{lb}$}‚उपपादितञ्चैतद् \href{http://sarit.indology.info/?cref=pv.1.160}{प्रमाण–वर्तिके १।१६० ।}
	{\color{gray}{\rmlatinfont\textsuperscript{§~\theparCount}}}
	\pend% ending standard par
      ‚{\tiny $_{lb}$}‚
	  \bigskip
	  \begingroup
	
	    
	    \stanza[\smallbreak]
	  \flagstanza{\tiny\textenglish{...7}}{\normalfontlatin\large ``\qquad}पाचकादिष्वभिन्नेन विनाप्यर्थेन वाचकः ।&‚{\tiny $_{lb}$}‚भेदान्न हेतुः कर्मास्य न जातिः कर्मसंश्र‚{\tiny $_{5}$}‚या \add{॥ ७}{\normalfontlatin\large\qquad{}"}\&[\smallbreak]
	  
	  
	  
	  \endgroup
	‚{\tiny $_{lb}$}‚

	  
	  \pstart \leavevmode% starting standard par
	इत्यादिनेति नेहोच्यते । न च तेषामुपकार्यत्वमस्ति नित्यतयाऽनाधेयातिशयत्वात् । ‚{\tiny $_{lb}$}‚न च पुरुषाभिप्रायानपेक्षो व्यक्त्युपकृतसप्रतिघत्वा ‚{\tiny $_{6}$}‚ दिसामान्यसामर्थ्यभावे‚{\tiny $_{lb}$}‚नोपनेयो विवक्षायान्तु संमुखीभावेपि प्रवृत्तिप्राप्तः । पुरुषाभिप्रायानुरूपे वा स ए‚{\tiny $_{lb}$}‚वास्तु नियामकः किमन्तर्गंतृभिः सामान्यैः ‚{\tiny $_{7}$}‚ \add{।} तथाहि तद्गतान्वयव्यतिरेकानुविधान ‚{\tiny $_{lb}$}‚मेव लक्ष्यते सामान्यानां न नित्यानामव्यतिरेकत्वात् ।
	{\color{gray}{\rmlatinfont\textsuperscript{§~\theparCount}}}
	\pend% ending standard par
      ‚{\tiny $_{lb}$}‚

	  
	  \pstart \leavevmode% starting standard par
	तस्मात् । सर्व्वमेतत् कुदर्शनसमाश्रयेण कल्पनामात्रं । कल्पना च ‚{\tiny $_{8}$}‚ सैव ‚{\tiny $_{lb}$}‚कर्तव्या या पुनर्न्न पर्यनुयोगमर्हति तत्रत्यां कल्पनायां वरमेव कल्पनादोषाभावात् । ‚{\tiny $_{lb}$}‚गुणसद्भावाच्चेति ।सात्मत\edtext{}{\lemma{सात्मत}\Bfootnote{? स्यान्मतं}}युक्तैवेयं कल्पना । नहि एकस्य नाना ‚{\tiny $_{lb}$}‚सु\edtext{}{\lemma{सु}\Bfootnote{? र्थ}}क ‚{\tiny $_{9}$}‚ \leavevmode\ledsidenote{\textenglish{20a/msK}} शब्दोत्थापने सामर्थ्यमस्तीति । यद्येवमत्रापि ब्रूम इत्याह । नानाशब्दो- ‚{\tiny $_{lb}$}‚त्थापनासामर्थ्ये ‚{\color{DodgerBlue3}‚नानासम्बन्ध्युपकारोपि माभूदि} ‚{\tiny $_{2b10}$}‚ ति । एकस्यानेकोपकार‚{\tiny $_{lb}$}‚कत्वविरोधाभ्युपगमादित्यभिसन्धिः । नित्यत्वात्सम्बन्धिनामनुपकारोऽभ्युपेत एवेति ‚{\tiny $_{lb}$}‚चेदाह । ‚{\color{DodgerBlue3}‚अनुपकारे हि तेषां} ‚{\tiny $_{2b10}$}‚ सनिदर्शन ‚{\tiny $_{2}$}‚ त्वादीनान्तेनाङ्गी ‚{\color{DodgerBlue3}‚क्रियमाणे ‚{\tiny $_{lb}$}‚तत्सम्बन्धिता न सिध्यति} । नहि यो येन नोपक्रियते हेदुः स तस्य सम्वन्धियुक्तो ‚{\tiny $_{lb}$}‚ \leavevmode\ledsidenote{\textenglish{28/s}} ‚{\color{DodgerBlue3}‚हिमवा} निव ‚{\color{DodgerBlue3}‚मलयगिरेरिति} भावः । एवन्तावत्पर ‚{\tiny $_{3}$}‚ पक्षनिराकरणेन सनिदर्शनादि‚{\tiny $_{lb}$}‚शब्दानामभिन्नविषयत्वं साधितं । त एव जडिम्नः पदमुद्वहन्तः पुनरपि पर्य‚{\tiny $_{lb}$}‚नुयुञ्जते ।
	{\color{gray}{\rmlatinfont\textsuperscript{§~\theparCount}}}
	\pend% ending standard par
      ‚{\tiny $_{lb}$}‚

	  
	  \pstart \leavevmode% starting standard par
	ननु भवतु नाम ‚{\tiny $_{4}$}‚ सनिदर्शनादिशब्दानामभिन्नविषयत्वं । अथ कथमवसीयते । ‚{\tiny $_{lb}$}‚घटपटादिशब्दानामनेकार्थविषयत्वमिति यावता रूपादिव्यतिरिक्त ‚{\tiny $_{5}$}‚ मन्यदे‚{\tiny $_{lb}$}‚वावयवि द्रव्यमस्ति । तदेव च घटपटादिशब्दैर्विषयीक्रियते । तथाहि विचार‚{\tiny $_{lb}$}‚विषयापन्नः पटस्तन्तुभ्यो व्यतिरिच्यते भिन्नकर्तृकत्वात् ‚{\tiny $_{6}$}‚ घटादिवत् । तथा ‚{\tiny $_{lb}$}‚समस्तव्यस्तप्रत्ययाविषयत्वाद् गवादिवत् । नहि तन्तवः तन्तुसमुदाय इति वा पटे ‚{\tiny $_{lb}$}‚प्रत्ययो दुष्टः । उपायान्तरसाध्यत्त्वाच्च घटादिवत् । भिन्नदेशावस्थितैश्च क्रिय‚{\tiny $_{lb}$}‚माणत्त्वात् । घटादिवदेव भिन्नपरिमाणत्त्वाच्च । वकुलामलकबिम्बादिवत् । अतश्च ‚{\tiny $_{lb}$}‚ \leavevmode\ledsidenote{\textenglish{20b/msK}} पूर्व्वोत्तर कालभावित्त्वाद् बीजाङ्कुरादिवत् । अथवा पटादन्ये तन्तवस्तत्कारणत्त्वात् ‚{\tiny $_{lb}$}‚तुर्यादिवत् । तन्तुपटयोर्वाऽन्यत्त्वं भिन्नशक्तिमत्त्वात् जलानलादिवत् । तथेदम‚{\tiny $_{lb}$}‚परम्विचारविषयापन्नमिन्दीवरङ्गन्धादि ‚{\tiny $_{1}$}‚ भ्योऽत्यन्तभिन्नन्तेषाम्व्यवच्छेदकत्वात् । ‚{\tiny $_{lb}$}‚चैत्रादिवत् । इह यद्यस्य व्यवच्छेदकन्तत्तस्मादन्यत्तद्यथा गोपिण्डाच्चैत्र ‚{\tiny $_{lb}$}‚इत्येतानि तद्व्यतिरेकसाधनप्रमाणानि सन्ति । तत्क ‚{\tiny $_{1}$}‚ थन्तेषाम्भिन्नविषयत्वम्भवि‚{\tiny $_{lb}$}‚ष्यतीति ॥ ० ॥
	{\color{gray}{\rmlatinfont\textsuperscript{§~\theparCount}}}
	\pend% ending standard par
      ‚{\tiny $_{lb}$}‚

	  
	  \pstart \leavevmode% starting standard par
	तदेतदप्येषामसद्दर्शनाभिनिवेशपटलप्रच्छादितान्तःकरणानां नाल्पीयसस्तमसो ‚{\tiny $_{lb}$}‚दुर्विलसितमित्यागूर्या ‚{\tiny $_{2}$}‚ ह । ‚{\color{DodgerBlue3}‚घट इत्यपि च रूपादय एवैकार्थक्रियाकारिण ‚{\tiny $_{lb}$}‚एकशब्दवाच्या भवन्तु किमर्थान्तरकल्पनये} ‚{\tiny $_{2b10}$}‚ ति कार्यमित्युपस्क्रियते । नैव ‚{\tiny $_{lb}$}‚किञ्चित्तस्य तादृशस्य नीलादि ‚{\tiny $_{3}$}‚ व्यतिरेकेणानुपलक्षणादित्याकूतमस्य । यानि त्वे‚{\tiny $_{lb}$}‚तानि तत्प्रतिपादनाय प्रमाणान्युक्तानि तान्यसिद्धतादिदोषदुष्टत्वान्नालं तद्भेद‚{\tiny $_{lb}$}‚साधनायेति भो ‚{\tiny $_{4}$}‚ ता \add{?} नामेव पुरतः छायान्दधतीति मन्यते । तथाहि नेदं ताव‚{\tiny $_{lb}$}‚दाद्यं प्रमाणं परीक्ष्यमाणं पूर्व्वामपि परीक्षां क्षमते । यतोत्र विकल्पद्वयमाविर्भवति । ‚{\tiny $_{lb}$}‚अ ‚{\tiny $_{5}$}‚ न्यावस्थावस्थितेभ्यो वा तन्तुभ्यः पटस्यान्यत्वं साध्यते विशिष्टसंस्थानावस्थिते‚{\tiny $_{lb}$}‚भ्यो वेति । तत्र न तावदयमाद्यः प्रकारः सहते विचारभारगौ ‚{\tiny $_{6}$}‚ रवं । सिद्धसाधनता‚{\tiny $_{lb}$}‚दोषानुषङ्गात् । यस्मात्समधिगतसमस्तवस्तुयाथातथ्यसुगतसमयनयसमाश्रय‚{\tiny $_{lb}$}‚प्रवृत्तिबलासादितावदातमतयः प्रसवानन्तरनि ‚{\tiny $_{7}$}‚ रोधभाजः सर्व्वभावाइति प्रकल्प‚{\tiny $_{lb}$}‚यन्ति । ततश्च तेभ्योन्यत्त्वमिष्टमेवेति सिद्धसाध्यताप्रसङ्गोपनिपातपिशाचः ‚{\tiny $_{lb}$}‚कथमिव भवन्तं न गृहणाति । द्वितीयोपि विक ‚{\tiny $_{8}$}‚ ल्पः तीव्रानलोपतप्त इवोपलतले ‚{\tiny $_{lb}$}‚तलानि पादानां न प्रतिष्ठां समासादप\edtext{}{\lemma{समासादप}\Bfootnote{? यं}}ति । हेतोः परं प्रत्यसिद्धत्वात् । ‚{\tiny $_{lb}$}‚नहि विशिष्टस्थानावस्थितेभ्यः पटस्य भिन्नकर्तृकत्वं परं ‚{\tiny $_{9}$}‚ \leavevmode\ledsidenote{\textenglish{21a/msK}} प्रति सिद्धपद्धतिमवतरति ‚{\tiny $_{lb}$}‚ \leavevmode\ledsidenote{\textenglish{29/s}} योहि तादृक्प्रकारेभ्योन्यत्वमभावादेव नाभ्युपैति स कथमिव भिन्नकर्तृकत्व‚{\tiny $_{lb}$}‚मभ्युपगमिष्यतीति । तदनन्तराभिहितमपि प्रमा ‚{\tiny $_{1}$}‚ णेन समभिलसित\edtext{}{\lemma{समभिलसित}\Bfootnote{? षित}}‚{\tiny $_{lb}$}‚मनोरथपरिपूरणायालं हेत्वसिद्धेः । यतः पट इति तन्तुष्वेव सन्निवेशविशेषेणा‚{\tiny $_{lb}$}‚वस्थितेषु प्रत्ययो वर्तते । तद्विविक्तरूपस्यात्यन्तमु ‚{\tiny $_{2}$}‚ न्मिषितचक्षुषाप्यदर्शनात्स्फ‚{\tiny $_{lb}$}‚टिकादौ दृष्टमिति चेत् । एतदुत्तरत्र निषेत्स्यामः । यत्त्विदमुपायान्तरसाध्यत्वाद् ‚{\tiny $_{lb}$}‚भिन्नदेशावस्थितैः क्रियमाणत्वात् भिन्नपरिमा ‚{\tiny $_{3}$}‚ णत्वात्पूर्वोत्तरकालभावित्त्वाद् ‚{\tiny $_{lb}$}‚भिन्नशक्तिमत्वाच्चेति ॥
	{\color{gray}{\rmlatinfont\textsuperscript{§~\theparCount}}}
	\pend% ending standard par
      ‚{\tiny $_{lb}$}‚

	  
	  \pstart \leavevmode% starting standard par
	अत्र प्रथमसाधनाभिहितविकल्पदोषस्तीव्रामर्शविरक्तलोचन इवारातिस्त‚{\tiny $_{lb}$}‚त्सम्पदन्न स ‚{\tiny $_{4}$}‚ हते ॥
	{\color{gray}{\rmlatinfont\textsuperscript{§~\theparCount}}}
	\pend% ending standard par
      ‚{\tiny $_{lb}$}‚
	  \bigskip
	  \begingroup
	
	    
	    \stanza[\smallbreak]
	  \flagstanza{\tiny\textenglish{...8}}{\normalfontlatin\large ``\qquad}प्रथमे सिद्धसाध्यत्वं द्वितीये हेत्वसिद्धता ।&‚{\tiny $_{lb}$}‚क्षणिकत्वाद्विशिष्टानामुत्पादोभिमतो यत \add{ः ॥ ८}{\normalfontlatin\large\qquad{}"}\&[\smallbreak]
	  
	  
	  
	  \endgroup
	‚{\tiny $_{lb}$}‚

	  
	  \pstart \leavevmode% starting standard par
	इति सङ्ग्रहश्लोकः । ‚{\tiny $_{lb}$}‚यच्चेदमुक्तं विचारविषया पं ‚{\tiny $_{5}$}‚ न \edtext{}{\lemma{न}\Bfootnote{? पन्न}}मिन्दीवरमित्यादि । तदपि न सङ्ग‚{\tiny $_{lb}$}‚च्छते । यस्मादिन्दीवरस्य गन्धादयइतीन्दिवर\edtext{}{\lemma{गन्धादयइतीन्दिवर}\Bfootnote{? इतीन्दीवर}}स्वभावा गन्धादयो ‚{\tiny $_{lb}$}‚मधुभावनाविशेषादिकार्यनिवर्त्तन\edtext{}{\lemma{मधुभावनाविशेषादिकार्यनिवर्त्तन}\Bfootnote{? निर्वर्त्तन}}स ‚{\tiny $_{6}$}‚ मर्था इति यावत् । अवि‚{\tiny $_{lb}$}‚शिष्टकार्यसाधनात्मना सामान्यभूतगन्धादिशब्दैः प्रसिद्धाविशिष्टकार्यसाधना‚{\tiny $_{lb}$}‚ख्येन विशेषेण ये विशिष्टास्त एवमुच्यन्ते । न पुनरत्रान्यत् ‚{\tiny $_{7}$}‚ किञ्चिदित्यर्था‚{\tiny $_{lb}$}‚वर्णितलक्षणं द्रव्यमस्ति तस्य तादृशोऽनुपलब्धेरित्युक्तप्रायं । तथा चानेन प्रकारेण ‚{\tiny $_{lb}$}‚तेषान्तद्व्यवच्छेदकं भवतीति । तेषान्तद्व्यच्छेदकं च न चात्यन्तं ‚{\tiny $_{8}$}‚ भिन्नमिति ‚{\tiny $_{lb}$}‚कोऽनयोर्विरोध इति । सन्दिग्धविपक्षव्यावृत्तिको हेतुः ‚{\color{DodgerBlue3}‚प्रतिबन्धासिद्धः} । नहि दृष्टान्त‚{\tiny $_{lb}$}‚मात्रास्तिद्धिरस्ति । सर्व्वसिद्धिप्रसङ्गात् । अपि च शिलापु ‚{\tiny $_{9}$}‚ \leavevmode\ledsidenote{\textenglish{21b/msK}} त्रकस्य शरीरं । ‚{\tiny $_{lb}$}‚रूपस्य स्वभाव इत्यत्रापि शिलापुत्रकरूपयोः शरीरस्वभावव्यवच्छेदकत्वमस्तीति ‚{\tiny $_{lb}$}‚भेदस्तयोरपि ततः प्रसजते न च भवति । नैःस्वाभाव्यप्रस ‚{\tiny $_{1}$}‚ ङ्गात् । तस्मा‚{\tiny $_{lb}$}‚दयमेतेनानैकान्तिकः स्फुटमेव भवद्भिरभिधानीयः । किञ्चेदमतिविकलैर्मि‚{\tiny $_{lb}$}‚थ्यादर्शनसंरागपिशाचाविष्टबुद्धिभिः किमिन्दीवरस्य गन्धादय इत्येते श ‚{\tiny $_{2}$}‚ ब्दा\add{ः} ‚{\tiny $_{lb}$}‚पुरुषाभिप्रायव्यापारनिरपेक्षा एव वस्तुतत्वनिबन्धनाः प्रवर्त्तन्ते । किम्वा यथैव तैः ‚{\tiny $_{lb}$}‚प्रयुज्यन्ते तथैव वस्तुतत्वमनपेक्ष्य तमर्थमसत्कारेण प्रतिपादयन्तीति \add{।} य ‚{\tiny $_{3}$}‚ द्याद्यः ‚{\tiny $_{lb}$}‚पक्षस्तदा सदाध्वननप्रसङ्गोऽतीतादिष्वन्यत्र च पुरुषे च्छावसा\edtext{}{\lemma{च्छावसा}\Bfootnote{? वशा}}न्नियो‚{\tiny $_{lb}$}‚जनन्नभवेत् । न च प्रवचनान्तरभेदेष्वर्थेषु प्रवृत्तिः प्राप्नोति । न च क ‚{\tiny $_{4}$}‚ स्याश्चि‚{\tiny $_{lb}$}‚द्वाचोऽसत्यार्थता स्यात् । अथोत्तरस्तदा । ‚{\tiny $_{lb}$}‚ 
	    \pend% close preceding par
	  
	    
	    \stanza[\smallbreak]
	  \flagstanza{\tiny\textenglish{...9}}{\normalfontlatin\large ``\qquad}येषाम्वस्तुवसा\edtext{}{\lemma{येषाम्वस्तुवसा}\Bfootnote{? वशा}}वाचो न विवक्षापराश्रया\add{ः} ।&‚{\tiny $_{lb}$}‚\leavevmode\ledsidenote{\textenglish{30/s}}षष्ठिवचनभेदादि चोद्यन्तान्प्रति युक्तिमत् ॥ \add{९}{\normalfontlatin\large\qquad{}"}\&[\smallbreak]
	  
	  
	  
	    \pstart  \leavevmode% new par for following
	    \hphantom{.}
	   ‚{\tiny $_{lb}$}‚यदाहुः ॥
	{\color{gray}{\rmlatinfont\textsuperscript{§~\theparCount}}}
	\pend% ending standard par
      ‚{\tiny $_{lb}$}‚
	  \bigskip
	  \begingroup
	
	    
	    \stanza[\smallbreak]
	  \flagstanza{\tiny\textenglish{...10}}{\normalfontlatin\large ``\qquad}यद्य ‚{\tiny $_{5}$}‚ था वाचकत्वेन वक्तृभिर्विनियम्यते ।&‚{\tiny $_{lb}$}‚अनपेक्षितवाहयार्थन्तत्तथावाचकम्वचः । \add{१०}{\normalfontlatin\large\qquad{}"}\&[\smallbreak]
	  
	  
	  
	  \endgroup
	‚{\tiny $_{lb}$}‚

	  
	  \pstart \leavevmode% starting standard par
	तदा न पुरुषेच्छाबलप्रवृत्तशब्दवसा\edtext{}{\lemma{पुरुषेच्छाबलप्रवृत्तशब्दवसा}\Bfootnote{? वशा}}दर्थतत्वं व्यवतिष्ठत इति तद ‚{\tiny $_{lb}$}‚वस्थं ‚{\tiny $_{6}$}‚ सन्दिग्धविपक्षव्यतिरेकत्वं हेतोरिति । एतेनैतदपि प्रत्युक्तं विप्रतिपत्ति‚{\tiny $_{lb}$}‚विषयापन्नाच्चन्दनादन्ये रूपरसगन्धस्पर्शा हेयत्वादयश्च\edtext{}{\lemma{हेयत्वादयश्च}\Bfootnote{? यश्चे}}ति प्रतिजानी‚{\tiny $_{lb}$}‚महे न ‚{\tiny $_{7}$}‚ व्यपदिश्यमाण\edtext{}{\lemma{व्यपदिश्यमाण}\Bfootnote{? न}}त्वात् । शिलातुलाढकप्रसेविकावदिति । तस्मात्तद्भाव‚{\tiny $_{lb}$}‚प्रतिपादनाय न किञ्चित्प्रमाणमस्तीति स्थितमेतत् । अस्माकन्तु तदभावप्रमाण‚{\tiny $_{lb}$}‚साधकं प्रमा ‚{\tiny $_{8}$}‚ \leavevmode\ledsidenote{\textenglish{22a/msK}} णमेतत् । ये परस्परव्यावर्त्तमानस्वभावावस्थितिसमालिङ्गित ‚{\tiny $_{lb}$}‚सरीरा\edtext{}{\lemma{सरीरा}\Bfootnote{? शरीरा}}स्ते व्यतिरिक्तावयविद्रव्यानुगतमूर्त्तिमात्मातिशयं नात्म‚{\tiny $_{lb}$}‚सात्\edtext{}{\lemma{सात्}\Bfootnote{? शात्}}कुर्वन्ति । यथा बहवो भस्माधा ‚{\tiny $_{1}$}‚ रन \add{?} लालाथूकादयस्तथा ‚{\tiny $_{lb}$}‚च यथोपदिष्टधर्मवन्तस्तन्त्वादय इति स्वभावहेतुः । वैधर्म्येण नभःपङ्कजादयस्ते‚{\tiny $_{lb}$}‚षान्नि\add{ः}स्वभावत्वात् । परस्परव्यावर्त्तमानानामपि ‚{\tiny $_{1}$}‚ यद्येकस्वभावानभ्युपगमे तस्य ‚{\tiny $_{lb}$}‚तेषु सर्व्वात्मनाऽन्यथा वा वृत्त्ययोगो बाधकम्प्रमाणं । कुतस्तद्धि युगपदनेकत्र ‚{\tiny $_{lb}$}‚सर्व्वात्मना वर्त्तमानमनेकाधारस्थिताधेयव ‚{\tiny $_{2}$}‚ दनेकत्त्वमात्मनोऽनुमापयतीति ‚{\tiny $_{lb}$}‚कथमस्याभिन्नस्वभावता योज्यते । एकावयवोपलम्भवेलायाञ्च सकलस्य तत्र ‚{\tiny $_{lb}$}‚परिसमाप्तत्वादुपलब्धिप्रसङ्गः । अनेकाव ‚{\tiny $_{3}$}‚ यवोपलब्धिद्वारेणोपलम्भकति‚{\tiny $_{lb}$}‚पयावयवदर्शनेपि स्यात् समस्तावयवोपलम्भद्वारेण उपलब्धौ सर्व्वकालमदर्शन‚{\tiny $_{lb}$}‚प्रसङ्गः । तस्याम्भास्वरमध्य ‚{\tiny $_{4}$}‚ भागानां सकृदनुपलम्भात् । एकावयवकम्पे च ‚{\tiny $_{lb}$}‚सर्व्वकम्पादिप्रसङ्गश्च वाच्यः । नाप्येकदेशेन सावयवत्वप्रसङ्गात् । एकदेशा‚{\tiny $_{lb}$}‚नाञ्चानवस्थाप्रसङ् ‚{\tiny $_{5}$}‚ गात् । तेपि हि तस्यावयवा इति पाण्यवयववृत्तेष्वपि अन्येन ‚{\tiny $_{lb}$}‚वर्त्तितव्यमित्यादिना तदन्यैकदेशाभाववानेकः कश्चिदवयवी विद्यते । तथा चा ‚{\tiny $_{6}$}‚ ‚{\tiny $_{lb}$}‚ण्वादिसमुदाय एवास्तु कोनुरोधः स्वात्मभूतेष्ववयवेष्विति । न वा क्वचिदप्यसौ ‚{\tiny $_{lb}$}‚वृत्तो न ह्येकदेशाः प्रत्येकमवयवीत्यलं प्रतिष्ठितमिथ्याप्रलापैरिति विरम्य ‚{\tiny $_{7}$}‚ ते ।
	{\color{gray}{\rmlatinfont\textsuperscript{§~\theparCount}}}
	\pend% ending standard par
      ‚{\tiny $_{lb}$}‚

	  
	  \pstart \leavevmode% starting standard par
	तदेवमेतत् परमतमलमालोच्यमानतीव्रतरार्क्करश्मिसंपातयोगिहिमशैलशिला‚{\tiny $_{lb}$}‚शकलवद्विलयमुपयातीति मन्यमानः प्राह । ‚{\color{DodgerBlue3}‚किमर्थान्तरकल्पन} ‚{\tiny $_{3a1}$}‚ ‚{\tiny $_{8}$}‚ येति । ‚{\tiny $_{lb}$}‚स्यादियत्तराशापरस्य नैवानेकस्यैकार्थक्रियाकारित्वमस्तीत्यत आह । ‚{\color{DodgerBlue3}‚बहवो} पि ही‚{\tiny $_{lb}$}‚ ‚{\tiny $_{3a1}$}‚ ‚{\color{DodgerBlue3}‚त्यादि} । किंवत् । चक्षुरादिवत् । यथा रूपालोकमनस्कारचक्षुराद‚{\tiny $_{9}$}‚‚{\tiny $_{lb}$}‚\leavevmode\ledsidenote{\textenglish{22b/msK}} \leavevmode\ledsidenote{\textenglish{31/s}}यश्चक्षुरादिविज्ञानमेकंर्व कुव्न्ति तथा रूपा\add{द}योप्युदकधारणविशेषादिकामे- ‚{\tiny $_{lb}$}‚कामर्थक्रियाङ्करिष्यन्तीत्यर्थः । यतश्चैतदेवं तस्मात्तस्यैकार्थक्रियासामर्थ्यस्य ‚{\tiny $_{lb}$}‚ख्यापनाय ‚{\tiny $_{1}$}‚ तत्र रूपादावेकस्य पटादिशब्दस्य नियोगोपि स्यादिति एतद्युक्तं पश्यामः । ‚{\tiny $_{lb}$}‚न केवलमेकार्थक्रियाकारित्वं तेषामित्यपिशब्देनाह । कथं युक्तमिति चेत् । एवं ‚{\tiny $_{lb}$}‚मन्यते ‚{\tiny $_{2}$}‚ केनचित्प्रयोजनेन केचिच्छब्दाः क्वचिन्निवेश्यन्ते तत्र यदनेकमेकत्रोप‚{\tiny $_{lb}$}‚युज्यते तदवश्यन्तत्र चोदनीयं । तस्य च पृथक्कथञ्चोदनेऽतिगौरवं स्यात् । न चा‚{\tiny $_{lb}$}‚स्यानन्यसाधार ‚{\tiny $_{3}$}‚ णं रूपं शक्यं चोदयितुं । नाप्यस्यायासस्य किञ्चित्साफल्यं ‚{\tiny $_{lb}$}‚केवलमनेन योग्यास्तत्र तेर्थाश्चोदनीयास्त एकेन वा शब्देन चोद्येरन् बहुभिर्वेति ‚{\tiny $_{lb}$}‚स्वातन्त्र्यमत्र ‚{\tiny $_{4}$}‚ वक्तुः । तदियमेका श्रुतिर्बहुषु वक्त्रभिप्रायवसा\edtext{}{\lemma{वक्त्रभिप्रायवसा}\Bfootnote{? वशा}}त्प्रवर्त्त‚{\tiny $_{lb}$}‚माना नोपालम्भमर्हति । न चेयमशक्यप्रवर्त्तमाना इच्छाधीनत्वात् । यदि हि न ‚{\tiny $_{lb}$}‚प्रयोक्तु ‚{\tiny $_{5}$}‚ रिच्छा कथमियमेकत्रापि वर्त्तेत । इच्छायाम्वा क एनां बहुष्वपि ‚{\tiny $_{lb}$}‚प्रतिबद्धुं समर्थः । प्रयोजनाभावादप्रवर्त्तनमित्यपि नाशङ्कनीयं । भिन्नेष्व ‚{\tiny $_{6}$}‚ प्ये‚{\tiny $_{lb}$}‚कस्माच्छब्दात्प्रतीतिरतत्प्रयोजनभेदेन यथा स्यादित्युक्तत्वात् प्रयोजनस्य तस्मा‚{\tiny $_{lb}$}‚त्सूक्तमस्माभिर्युक्तं पश्याम इति । यथा कथञ्चिद्विनैव प्रयोजनेन लोकः शब्दं ‚{\tiny $_{7}$}‚ ‚{\tiny $_{lb}$}‚प्रयुंक्ते । ततो न युक्तमेतदिति चेदाह । न च ‚{\color{DodgerBlue3}‚निःप्रयोजना लोकस्यार्थेषु} शब्दयोजने‚{\tiny $_{lb}$}‚ ‚{\tiny $_{3a1}$}‚ ति । न हि व्यसनमेवैतल्लोकस्य यदयमसङ्गतं यन्न प्रयुज्जानो वा शब्दान्तः ‚{\tiny $_{lb}$}‚खल्वा ‚{\tiny $_{8}$}‚ सीत् । किन्तर्हि सर्व्व एवास्यारम्भः प्रयोजनसापेक्षः प्रयोजनञ्चेतदुक्त ‚{\tiny $_{lb}$}‚मिति मन्यते तत्र प्रयोजनवत्वेनेति । यथा रूपगन्धरसादयः सहैकप्रयोजनाः ‚{\tiny $_{lb}$}‚सङ्कलि ‚{\tiny $_{9}$}‚ \leavevmode\ledsidenote{\textenglish{23a/msK}} ता एककार्यकारिण इत्यर्थः । समवहितानामपि कदाचित्कस्यचिदेव ‚{\tiny $_{lb}$}‚कार्ये व्यापारोन्यस्य त्वौदासीन्यमिति ‚{\tiny $_{1}$}‚ स्यादाशङ्कासंभवस्तत आह । ‚{\color{DodgerBlue3}‚पृथग्वेति} ‚{\tiny $_{lb}$}‚ ‚{\tiny $_{3a2}$}‚ \add{।} वा शब्दश्चार्थे । सर्व्व एव व्यापारवन्त इत्यर्थः । अन्ये त्वन्यथा ‚{\tiny $_{lb}$}‚व्याख्यानयन्ति । व्याख्यानञ्चादो दूषयन्ति । तत्रैतस्मिन् शब्दैर ‚{\tiny $_{2}$}‚ र्थ प्रत्यायन‚{\tiny $_{lb}$}‚क्रमे येर्था रूपादयः सह पृथग्वैकप्रयोजनास्तेषां रूपादीनां संहितानां पृथुबुध्नोद‚{\tiny $_{lb}$}‚राकारसंस्थापितानामेकं प्रयोजनं । यदुत मधूदकाद्याहरणं पृथग्वा ‚{\tiny $_{3}$}‚ तेषामेव ‚{\tiny $_{lb}$}‚प्रत्येकं स्वाकारज्ञानजननं ॥ एकञ्च तत्प्रयोजनमेकत्र दृष्टं यत्तदन्यत्र नास्तीति ‚{\tiny $_{lb}$}‚सहभूतानामपि कदाचिदौदासीन्यदर्शनात्सर्वेषां सव्यापा ‚{\tiny $_{4}$}‚ रतामादर्शयितुं पृथ‚{\tiny $_{lb}$}‚ग्वेत्यभिहितं \add{।} वा शब्दश्च समुच्चय इत्यन्ये । केवलमत्रैकप्रयोजना इत्यभिधाना‚{\tiny $_{lb}$}‚ \leavevmode\ledsidenote{\textenglish{32/s}} त्सर्वेषान्तथाभावप्रतीतिरस्त्येवेति व्यर्थम्पृ ‚{\tiny $_{5}$}‚ थग्वेति स्यात् न चायं शब्दार्थ इति ‚{\tiny $_{lb}$}‚यक्तिञ्चिदेतत् । तैः प्रकरणं न लक्षितं तथा ह्यत्र समुदायशब्दस्यैकवचनप्रवृ‚{\tiny $_{lb}$}‚त्यविरोधः कथयितुमारब्धः । तत्र कः प्रस्तावः पृथग्वेत्यभिधानस्य ॥
	{\color{gray}{\rmlatinfont\textsuperscript{§~\theparCount}}}
	\pend% ending standard par
      ‚{\tiny $_{lb}$}‚

	  
	  \pstart \leavevmode% starting standard par
	केवलं रूपादिशब्दश्चायञ्जातिशब्दः । तत्र चान्यादृश्येव प्रक्रिया भविष्यति । ‚{\tiny $_{lb}$}‚यत्त्विदमुक्तं केवलमत्रैकप्रयोजना इत्यभिधानात्सर्व्वे ‚{\tiny $_{7}$}‚ षां तथाभावप्रतीतिरस्त्येवेति ‚{\tiny $_{lb}$}‚तदपि न युक्तिसङ्गतं । तथाहि परबलपराजयोद्यतानामेकप्रयोजनवत्वेपि न तत्र सर्व्वे ‚{\tiny $_{lb}$}‚व्यापारवन्तो भवन्ति । तद्वदत्रापि भ ‚{\tiny $_{8}$}‚ वेत् । अत एव च स्यादाशङ्कासम्भव इति ‚{\tiny $_{lb}$}‚व्याख्यातं । यदा तु सर्व्वेषामेव सव्यापारताख्यापनाय पृथग्वेत्येतदुच्यते तदाऽपह्नु ‚{\tiny $_{lb}$}‚र\add{?}तमुत्सार्यते । तदेतेनैवाशब्दार्थ ‚{\tiny $_{9}$}‚ \leavevmode\ledsidenote{\textenglish{23b/msK}} त्वमपि प्रत्युक्तमिति यक्तिञ्चिदेतदेव ।
	{\color{gray}{\rmlatinfont\textsuperscript{§~\theparCount}}}
	\pend% ending standard par
      ‚{\tiny $_{lb}$}‚

	  
	  \pstart \leavevmode% starting standard par
	\hphantom{.}अस्तु वैतदपि व्याख्यानं यदि कथञ्चिद्व्यवस्थापितुं पार्यते । ‚{\color{DodgerBlue3}‚तेषा} मेवं विधा‚{\tiny $_{lb}$}‚नामर्थानान्तस्यैकार्थक्रियाकारिणो भावस्य ख्यापना ‚{\tiny $_{1}$}‚ यैकोघटादिशब्दो यदि ‚{\tiny $_{lb}$}‚नियुज्येत तदा किं स्यान्न कश्चिद्दोषः स्यात् । गुण एव तु केवलो लभ्यत इत्याह \add{।} ‚{\tiny $_{lb}$}‚ ‚{\color{DodgerBlue3}‚तदर्थक्रियास\edtext{}{\lemma{तदर्थक्रियास}\Bfootnote{? श}}} ‚{\color{DodgerBlue3}‚क्ते} रभिन्नाया\add{ः} ख्यापनाय ‚{\color{DodgerBlue3}‚नियुक्तस्य समुदायश ‚{\tiny $_{2}$}‚ ब्दस्यैकवचन‚{\tiny $_{lb}$}‚विरोधोपि नास्त्येव} ‚{\tiny $_{3a2}$}‚ । कुतः \add{।} यस्मात्स ‚{\color{DodgerBlue3}‚हितानां सा शक्तिरेका} ‚{\tiny $_{3a3}$}‚ ‚{\tiny $_{lb}$}‚ऽभिन्ना न प्रत्येकं न तु पृथग्भूतानामित्यर्थः । इति तस्मात्समुदायशब्दे तस्मिन्नै‚{\tiny $_{lb}$}‚कस्मिन्घटादौ ‚{\tiny $_{3}$}‚ समुदाये वाच्ये एकवचनं घट इति भवतीति शेषः । स्यादिति वा ‚{\tiny $_{lb}$}‚वक्ष्यमाणं क्रियापदं । नन्वयङ्घटादिशब्दो गवादिशब्दवज्जातिशब्दस्तत्कथमे‚{\tiny $_{lb}$}‚तदु ‚{\tiny $_{4}$}‚ क्तमिति चेत् । सत्यं समुदा\add{या}न्तरवृत्यपेक्षया जातिशब्दोयं रूपादिसमु‚{\tiny $_{lb}$}‚दाय्यपेक्षया तु समुदायशब्दोपीत्यभिसन्धेरदोषः । तथाहि त्रय्येवगतिः श ‚{\tiny $_{5}$}‚ ब्दा‚{\tiny $_{lb}$}‚नाङ्केचिज्जातिशब्दा एव । यथा सुखादिशब्दाः सुखादेरनवयवत्वात् । केचित्तु ‚{\tiny $_{lb}$}‚समुदायशब्दा एव यथा ‚{\color{DodgerBlue3}‚विन्ध्यहिमवत्सुमेर्वादि} शब्दाः । तज्जातीयस ‚{\tiny $_{6}$}‚ मुदाया‚{\tiny $_{lb}$}‚न्तराभावात् । अपरे पुनर्जातिसमुदायशब्दाः । यथैत एव घटादिशब्दाः समुदाया‚{\tiny $_{lb}$}‚न्तरसमुदाय्यपेक्षयेति । एवन्तावत्समुदायशब्देषु वचनप्रवत्यवि ‚{\tiny $_{7}$}‚ रोध उक्तः । ‚{\tiny $_{lb}$}‚अथ कथञ्जातिशब्देष्वित्याह । ‚{\color{DodgerBlue3}‚जातिशब्देष्वित्यादि} ‚{\tiny $_{3a3}$}‚ । अर्थानां घटा‚{\tiny $_{lb}$}‚वीनां प्रत्येकं सहितानाञ्च शक्तेः कारणात् नानाशक्तिरेका च । एतदुक्तं भवति ‚{\tiny $_{lb}$}‚यस्मादेको ‚{\tiny $_{8}$}‚ \leavevmode\ledsidenote{\textenglish{24a/msK}} पि वृक्षो गृहकरणाद्यर्थक्रियानिवर्त्तने\edtext{}{\lemma{गृहकरणाद्यर्थक्रियानिवर्त्तने}\Bfootnote{? निर्वर्त्तने}}पि योग्यो बह‚{\tiny $_{lb}$}‚ \leavevmode\ledsidenote{\textenglish{33/s}} वोपि वृक्षाः । ततश्च तेषाङ्केवलानामपि योग्यत्वादनेका शक्तिः समवहिता‚{\tiny $_{lb}$}‚नामपि योग्यत्वादेका शक्तिरेकप्रत्यवमर्शप्रत्ययनिबन्धनत्वेनैकत्वोपचारात् । ‚{\tiny $_{lb}$}‚यतश्चैव ‚{\tiny $_{1}$}‚ मिति तस्माद्यथाक्रमं नानाशक्तिविवक्षायां सत्यां बहुवचनमनेक‚{\tiny $_{lb}$}‚त्वाच्छक्तेर्वृक्षा इति भवति । एकशक्तिविवक्षायान्तु एकत्वाच्छब्द एकवचन‚{\tiny $_{lb}$}‚मुक्त इति स्यात् । त ‚{\tiny $_{2}$}‚ दा यद्येष नियमो भवद्भिरसद्ग्रहग्रहावेशव्याकुलितचेतोभि‚{\tiny $_{lb}$}‚रिष्यते । बहुष्वेव वाच्येषु बहुवचनं भवति । एकस्मिन्नेव चैकवचनमिति । तदनेन ‚{\tiny $_{lb}$}‚यदाप्येतद्द ‚{\tiny $_{3}$}‚ र्शनमाश्रीयते बहुषु \add{वहु}वचनम्भवति । ‚{\color{DodgerBlue3}‚द्व्येकयोर्द्विवचनैकवचने} ‚{\tiny $_{lb}$}‚ \href{http://sarit.indology.info/?cref=P\%C4\%81.1.4.22}{पाणिनिः १।४।२२} इति तदापि न कश्चिद्दोष इति दर्शयति । भवतान्तु कथ‚{\tiny $_{lb}$}‚मित्याह । ‚{\color{DodgerBlue3}‚अस्माकमि} ‚{\tiny $_{3a4}$}‚ त्यादि । ‚{\tiny $_{4}$}‚ संकेतबसा\edtext{}{\lemma{संकेतबसा}\Bfootnote{? बशा}}च्छब्दानाम्बहुवच‚{\tiny $_{lb}$}‚नान्तानान्दाराः सिकताः पादाः । गुरव इत्यादिनाऽसत्यपि बहुत्वेऽभिधेयस्य वृत्तिः ॥
	{\color{gray}{\rmlatinfont\textsuperscript{§~\theparCount}}}
	\pend% ending standard par
      ‚{\tiny $_{lb}$}‚

	  
	  \pstart \leavevmode% starting standard par
	तथासत्यप्यनेकत्वेषण्णगरी ‚{\tiny $_{5}$}‚ षट्पू\add{?}लीवनमित्यादिनैकवचनान्तानाम्वृत्तिरि‚{\tiny $_{lb}$}‚त्यनभिनिवेश एव । को हि नाम सचेतनः पुरुषाभिप्रायमात्राधीनवृत्तिषु शब्देष्वभिनि‚{\tiny $_{lb}$}‚वेशं कर्त्तुमु ‚{\tiny $_{6}$}‚ त्सहत इति भावः । परपक्षं पूर्वपक्षयति । ‚{\color{DodgerBlue3}‚नानेको रूपादिरेकशब्दो‚{\tiny $_{lb}$}‚त्थापने समर्थ इति चे} ‚{\tiny $_{3a5}$}‚ दिति । नहि अनेकस्यैकेन सम्बन्धो युज्यत इति । ‚{\tiny $_{lb}$}‚किमि ‚{\tiny $_{3a5}$}‚ त्यादिना परि ‚{\tiny $_{7}$}‚ हारः । पुरुषाणाम्वृत्तिरिच्छा तत्रानपेक्षाः सन्तोऽर्थाः ‚{\tiny $_{lb}$}‚किं स्वयं शब्दानुत्थापयन्ति । आहोस्वित्पुरुषैस्ते व्यवहारार्थमर्थेषु यथा कथञ्चि‚{\tiny $_{lb}$}‚न्नियुज्यन्त इति विकल्पद्व ‚{\tiny $_{8}$}‚ यं । तत्र पुरुषैरेव ते यथेष्टं नियुज्यन्तेऽन्यथाऽतीता ‚{\tiny $_{lb}$}‚जातयोर्दर्शनान्तरभेदिष्वन्यत्र वा नियोजनन्न भ ‚{\tiny $_{9}$}‚ \leavevmode\ledsidenote{\textenglish{24b/msK}} वेदिति भावः । ततश्च ‚{\color{DodgerBlue3}‚स्वयं} ‚{\tiny $_{lb}$}‚ ‚{\tiny $_{3a6}$}‚ पुरुषेच्छाऽनपेक्षाणामर्थानां शब्दोत्थापने सति भावस्य शक्तिरसक्ति\edtext{}{\lemma{शक्तिरसक्ति}\Bfootnote{? शक्ति}} ‚{\tiny $_{lb}$}‚ र्वा चिन्त्येत ‚{\color{DodgerBlue3}‚नामैक} इत्यादिना । अस्त्येव तर्हि स्वयमुत्त्थापनमिति चेदा ‚{\tiny $_{1}$}‚ ह । ‚{\color{DodgerBlue3}‚न च ‚{\tiny $_{lb}$}‚तद्युक्तं} ‚{\tiny $_{3a6}$}‚ । अनन्तरोक्तात् कारणत्रयादित्यभिप्रायः । तस्मात्पुरुषैस्तेषां ‚{\tiny $_{lb}$}‚शब्दानां नियोगोर्थेषु विनाप्येकत्वादिना ते पुरुषाः यथेष्टमेकत्रापि वहुवचनान्तम ‚{\tiny $_{2}$}‚ ‚{\tiny $_{lb}$}‚नेकत्राप्येकवचनान्तं शब्दं नियुञ्जीरन्निति कस्तत्र तेषु शब्देषूपालम्भो नानेको ‚{\tiny $_{lb}$}‚ \leavevmode\ledsidenote{\textenglish{34/s}} रूपादिरेकशब्दोत्थापने समर्थ इत्ययं नैव कश्चित् केवलमतिबहुलव्यामोहविजृ‚{\tiny $_{lb}$}‚म्भितमिति मन्यते । स्यान्मतङ्किमित्येकं शब्दमनेकत्र नियुञ्जत इत्याह । ‚{\color{DodgerBlue3}‚निमि‚{\tiny $_{lb}$}‚तञ्च नियोगस्योक्तमेवेति} ‚{\tiny $_{3a6}$}‚ तत्सामर्थ्यख्यापनाय तत्रैक ‚{\tiny $_{4}$}‚ शब्दनियोगोऽपि ‚{\tiny $_{lb}$}‚स्यादित्यत्रावसरे । उपचयमाह । ‚{\color{DodgerBlue3}‚अपि चे} ‚{\tiny $_{3a7}$}‚ त्यादिना । आश्रयाभिमतेने‚{\tiny $_{lb}$}‚त्यवयविद्रव्येण । तेषान्तत्र समवायसम्बन्धेन सम्बन्धात् । क ‚{\tiny $_{5}$}‚ थं सम्बन्धो नैवाने‚{\tiny $_{lb}$}‚कस्य एकेन सह सम्बन्धो विरोधाभ्युपगमात् । अन्यथैकेन शब्देनापि सह प्राप्नो‚{\tiny $_{lb}$}‚तीत्यभिसन्धिः । परः प्राह । ‚{\color{DodgerBlue3}‚न चे} ‚{\tiny $_{3a7}$}‚ दयमेकेन स ‚{\tiny $_{6}$}‚ ह ‚{\color{DodgerBlue3}‚सम्बन्धविरोधात्} ‚{\tiny $_{lb}$}‚कारणा ‚{\color{DodgerBlue3}‚देकशब्दं} रूपादिषु ‚{\color{DodgerBlue3}‚नेच्छामः} । किन्त्वभिन्नानामविशिष्टानां रूपादीनां‚{\tiny $_{lb}$}‚घटकम्बलपर्यङ्कादिषु । नानाविधा येयमर्थक्रिया जलधा ‚{\tiny $_{7}$}‚ रणप्रावरणादिस्तस्या ‚{\tiny $_{lb}$}‚विरोधः । तथा च तत्सामर्थ्यख्यापनाय शब्दस्य विरोधात् । तेषाञ्चाभेदस्तदाश्रय‚{\tiny $_{lb}$}‚द्रव्यभेदाभावात् । एतदेव स्फुटयति । ते रूपादय ए ‚{\tiny $_{8}$}‚ \leavevmode\ledsidenote{\textenglish{25a/msK}} कस्वभावाः सन्तः ‚{\tiny $_{lb}$}‚समुदायान्तरे कम्वलादौ येयमसम्भाविनी उदक धारण विशेषाद्यर्थक्रिया ‚{\tiny $_{lb}$}‚तामेव कुर्युस्तेन कारणेन तस्या अर्थक्रियायाः प्रकाशना ‚{\tiny $_{1}$}‚ येमामेतेऽर्थक्रियां न ते ‚{\tiny $_{lb}$}‚तदसम्भाविनीमर्थक्रियाङ्कुर्वन्ति \add{।} यथा त एव कम्बलग ‚{\tiny $_{2}$}‚ ता रूपादयः ‚{\tiny $_{lb}$}‚सजातीयेभ्यः । तथा च घटगता अपि रूपादयः कम्बलगतेभ्यो रूपादि‚{\tiny $_{lb}$}‚भ्योऽविशिष्टस्वभावा इति व्यापकानुपलब्धिः । एवमन्यत्रापि ‚{\tiny $_{3}$}‚ योज्यमितीयं पूर्व्व‚{\tiny $_{lb}$}‚पक्षरचना । अत्रोत्तरमाह । ‚{\color{DodgerBlue3}‚भवतु नामेत्यादिना} ‚{\tiny $_{3a9}$}‚ । तदनेन हेतोरसिद्धि‚{\tiny $_{lb}$}‚मुद्भावयति । अयमत्रार्थो नहि रूपादीनाङ् कम्बला ‚{\tiny $_{4}$}‚ दिष्वभेदोस्ति । परस्पर‚{\tiny $_{lb}$}‚रूपविविक्तानामेव प्रत्यक्षप्रमाणपरिच्छेद्यत्त्वात् । एवञ्च सतीदं प्रत्यक्षं किमे‚{\tiny $_{lb}$}‚नाम्वाञ्छामुपेक्षते । किम्प्रश्ने क्षेपे वा ‚{\tiny $_{5}$}‚ नैव क्षन्तुमर्हत्यपाकरोतीति । ‚{\tiny $_{lb}$}‚किंञ्चानिष्टञ्चेद ‚{\tiny $_{3a10}$}‚ मस्माभिर्घटकम्बलादिष्वभिन्ना रूपादय इति कुतो ‚{\tiny $_{lb}$}‚ \leavevmode\ledsidenote{\textenglish{35/s}} रूपादीनाम्प्रतिसमुदायत्त्वे हेतुबलादनपेक्षितद्रव्याणां स्वभेदाभ्युपगमात् । तद‚{\tiny $_{lb}$}‚नेनाभ्युपगमद्वारेणाप्यसिद्धताञ्चोदयति । पूर्व्वेण प्रत्यक्षद्वारेणेति विशेषः । ‚{\tiny $_{lb}$}‚पुनरपीर्ष्यालुः परः प्रा ‚{\tiny $_{7}$}‚ ह । यद्यन्य एव रूपादिभ्यो घटः स्यात् किं ‚{\tiny $_{lb}$}‚स्यादिति \add{।} न कश्चिद्दोषः स्यादित्याकूतं । न वयं मात्सर्यात्तं नेच्छामः । ‚{\tiny $_{lb}$}‚किन्तु भवत्येतावत्त्वत्र वक्तव्यस्तीत्याह । तस्यावय ‚{\tiny $_{8}$}‚ विनः ‚{\color{DodgerBlue3}‚प्रत्यक्षस्य सतः} ‚{\tiny $_{lb}$}‚ ‚{\tiny $_{3a10}$}‚ चक्षुः स्पर्शनेन्द्रियग्राह्यतयाभ्युपगतत्त्वात् । ‚{\color{DodgerBlue3}‚अरूपादिरूपस्य} ‚{\tiny $_{lb}$}‚ ‚{\tiny $_{3a10}$}‚ रूपगन्धादिस्वभावरहितस्येत्यर्थः । गुणद्रव्ययोर्भेदाभ्युपगमात् ॥ ‚{\tiny $_{9}$}‚ \leavevmode\ledsidenote{\textenglish{25b/msK}} ‚{\tiny $_{lb}$}‚तद्विवेकेन रूपादिविवेकेन बुद्धौ चक्षुः स्प\add{र्श}नेन्द्रियजायां प्रतिभासने किमा‚{\tiny $_{lb}$}‚वरणन्न कश्चित्प्रतिबन्ध इत्यर्थः । न च कश्चिदत्यादरेणाप्रतिहतकरणोपि नि ‚{\tiny $_{1}$}‚ ‚{\tiny $_{lb}$}‚रूपयन्नीलमधुरसुरभिकर्क्कशादिव्यतिरेकेण तद्रूपम्विविक्तरूपं घटादिद्रव्य‚{\tiny $_{lb}$}‚मुपलब्धुमीश इति मन्यते । ‚{\color{DodgerBlue3}‚अबिद्धकर्ण्णस्त्वाह} \add{।} रूपाद्यग्रहेपि द्रव्यग्र ‚{\tiny $_{2}$}‚ हणमस्त्येव ‚{\tiny $_{lb}$}‚यतो मन्दमन्दप्रकाशेऽनुपलभ्यमानरूपादिकं द्रव्यमुपलभ्यतेऽनिश्चितरूपं गौरश्वो ‚{\tiny $_{lb}$}‚वेति । ननु च तत्रापि संस्थानमात्रमुपलभ्यते । सत्यमुप ‚{\tiny $_{3}$}‚ लभ्यते न तु तद्रूपा‚{\tiny $_{lb}$}‚द्यात्मकं । रूपाद्यात्मकत्वे वा नीलपीतादिविशेषग्रहणप्रसङ्गः । तथायस्कञ्चु‚{\tiny $_{lb}$}‚कान्तर्गते पुरुषे पुरुषरूपाद्यग्रहे ‚{\tiny $_{4}$}‚ पि पुरुषप्रत्ययो दृष्टः । रात्रौ च वलाकाव्यामुक्त ‚{\tiny $_{lb}$}‚रूपाद्यग्रहेपि पक्षिप्रत्ययो दृष्टः । तथानीलाद्युपधानभेदानुविधायिनः स्फटिकम ‚{\tiny $_{5}$}‚ णेः ‚{\tiny $_{lb}$}‚स्फटिकरूपाद्यग्रहेपि स्फटिकप्रत्ययः । तथा कषायरूपेण पटरूपाभिभवे पटरूपाद्य‚{\tiny $_{lb}$}‚ग्रहेऽपि पटप्रत्ययो दृष्ट इति । तदेतत्स ‚{\tiny $_{6}$}‚ र्व्वमस्यानल्पकालोपचितकुदर्शनाभ्या‚{\tiny $_{lb}$}‚सोपजातबुद्धिमान्द्यविजृम्भितमेव प्रकटयति वचः । तथाहि यत्तावदिदमुक्तं मन्द‚{\tiny $_{lb}$}‚मन्दालोके रात्रौ च नीलाद्युपधान ‚{\tiny $_{7}$}‚ सद्भावे च तद्रूपाद्यग्रहेपि द्रव्यमुपलभ्यत इति ‚{\tiny $_{lb}$}‚तत्र वक्तव्यं कीदृशं तत्र द्रव्यमुपलभ्यत इति । दृश्यत एव तद्यादृशमुपलभ्यत इति ‚{\tiny $_{lb}$}‚चेत् । ननु श्यामरूपं ‚{\tiny $_{8}$}‚ मन्दमन्दालोके रात्रौ च तत्र तदपुलभ्यते उपधानं रूपञ्च । ‚{\tiny $_{lb}$}‚न च तद्रूपन्तत् । ताद्रूप्येऽनन्तरोदितपक्षक्षयप्रसङ्गात् । तत्समीपपार्श्ववर्त्तिभिश्च ‚{\tiny $_{lb}$}‚तथानुप ‚{\tiny $_{9}$}‚ \leavevmode\ledsidenote{\textenglish{26a/msK}} लम्भात् । न चाप्यण्या\edtext{}{\lemma{चाप्यण्या}\Bfootnote{? न्या}}कारेण बोधेन वस्तुनोऽवगतिः यस्य कस्य- ‚{\tiny $_{lb}$}‚चिज्ज्ञानस्य सर्व्ववस्तुपरिच्छेदकत्त्वप्रसङ्गात् । तस्माद् भ्रान्तमेतत् ज्ञानम्भ्रान्ति‚{\tiny $_{lb}$}‚वीजात्स्वोपादानादनादि ‚{\tiny $_{1}$}‚ कालीनान्निर्विषयमेव तथा प्रतिभासि द्विचन्द्रादिप्रत्यय‚{\tiny $_{lb}$}‚वदुपजायते \add{।} निर्विषयत्त्वेपि प्रतिनियतदेशकालभावि भवति । स्वोपादान‚{\tiny $_{lb}$}‚वासनाप्रबोधकबाह्या ‚{\tiny $_{2}$}‚ धिपतिप्रत्ययापेक्षना\edtext{}{\lemma{धिपतिप्रत्ययापेक्षना}\Bfootnote{? णा}}त् । द्विचन्द्रादिज्ञानवदेव । ‚{\tiny $_{lb}$}‚भ्रान्तत्त्वेप्यर्थाविष\edtext{}{\lemma{भ्रान्तत्त्वेप्यर्थाविष}\Bfootnote{? स}}म्वादो विशिष्टाधिपतिप्रत्ययसद्भावात् । मणि‚{\tiny $_{lb}$}‚ \leavevmode\ledsidenote{\textenglish{36/s}} प्रभायां मणिभ्रान्तिरिव । न चार्थाविसम्वाद ‚{\tiny $_{3}$}‚ नादेवास्य सविषयत्वं युक्तमनु ‚{\tiny $_{lb}$}‚मानेन व्यभिचारात् । मणिभ्रान्त्या च । तदेव द्रव्यन्तथा गृहणाति ततोन्यस्य निर्वि‚{\tiny $_{lb}$}‚षयत्त्वमिति चेत् । ननु न तद् द्रव्यन्त ‚{\tiny $_{4}$}‚ द्रूपन्न वान्याकारानुस्यूतः प्रत्ययोऽन्य‚{\tiny $_{lb}$}‚स्य परिच्छेदक इत्युक्तं । एवञ्च सति सद्विषयत्त्वे सत्यपेक्षेपि सद्विषयत्त्वमस्त्येव । ‚{\tiny $_{lb}$}‚तथा हि समाप्येतच्छक्यम्वक्तुं ‚{\tiny $_{5}$}‚ त एव नीलादयस्तथा प्रतिभासन्त इति । असति ‚{\tiny $_{lb}$}‚भ्रांतिसन्देहकारणे सालोकावस्थायां योग्यदेशावस्थाने च निरुपधानावस्थायाञ्च ‚{\tiny $_{lb}$}‚नोपलभ्यते ‚{\tiny $_{6}$}‚ \add{।} तत् द्रव्यमनात्मरूपप्रतिभासि विवेकेनान्यदा तु सति भ्रान्ति‚{\tiny $_{lb}$}‚सन्देहकारणे निशान्धकारावच्छादितलोचनावस्थायां दूरदेशावस्थाने सोपधा‚{\tiny $_{lb}$}‚नावस्थायाञ्च तद ‚{\tiny $_{7}$}‚ न्याकारविवेकेन प्रतिभासत इति कोन्यो भौतिकाद्वक्तु‚{\tiny $_{lb}$}‚मर्हति । अयस्कञ्चुकान्तर्गते पुरुषप्रत्ययो न प्रत्यक्षः । किन्तर्हि \add{।} लैगिकः । ‚{\tiny $_{lb}$}‚तथा हि पुरुषश ‚{\tiny $_{8}$}‚ रीरावयवसमाश्रयबलोद्भूतविशिष्टसंस्थानावस्थितकञ्चुक‚{\tiny $_{lb}$}‚दर्शनात्कार्यलिंगज्ञानात् सम्बन्धस्मरणापेक्षिणः कारणभूते तथाविधे पुंसि ‚{\tiny $_{lb}$}‚पुरुषोयमित्यनन्तर ‚{\tiny $_{9}$}‚ \leavevmode\ledsidenote{\textenglish{26b/msK}} मेव प्रत्ययः समुद्भूतिमासादयति । अत एव चास्पष्टाकारा सा ‚{\tiny $_{lb}$}‚प्रतीतिः कश्चायमिति संशयश्च भवति । तथाविधसंस्थानस्य च कञ्चुकस्योत्त्पत्तेः ‚{\tiny $_{lb}$}‚पुरुषरूपादय ए ‚{\tiny $_{1}$}‚ व हेतवो भवन्ति नत्वन्यदवयवि द्रव्यं । तस्यासिद्धेरसिद्धस्य च ‚{\tiny $_{lb}$}‚कारणत्वाभ्युपगमायोगात् । रूपादिभिस्तु प्रत्यक्षानुपलम्भाभ्याङ्कार्यकारणभाव‚{\tiny $_{lb}$}‚सिद्धेः । तेषा ‚{\tiny $_{2}$}‚ मेव केवलानान्तथा सन्निविष्टानामुपलम्भात् । पटे तु कषायमञ्जिष्ठा‚{\tiny $_{lb}$}‚दिसम्पर्कादर्थान्तरमेव केवलन्तत्तथा जातमीक्ष्यते । नतु नानारूपयोर्द्रव्ययोः सं ‚{\tiny $_{3}$}‚ सर्गा‚{\tiny $_{lb}$}‚दविभागात् तथोपलम्भः । पुनस्तद्द्रव्यसंस्थानस्थितिकारणविच्छेदात्तन्निवृत्तिः । ‚{\tiny $_{lb}$}‚तदुपादानकारणापेक्षिणश्च जलपावकादेरपरोत्त्पत्तिरिति ‚{\tiny $_{4}$}‚ । एतेनायोगोलक‚{\tiny $_{lb}$}‚तद्रूपग्रहणेपि तत्प्रत्ययो दृष्ट इत्येतदपि प्रतिस्फुटं । तदेवं द्रव्यस्य प्रत्य‚{\tiny $_{lb}$}‚क्षत्वासिद्धेर्यदुक्तङ्गवाश्वमहिषवराहमातङ्गा ‚{\tiny $_{5}$}‚ विमत्यधिकरणभावापन्ना रूपादि‚{\tiny $_{lb}$}‚व्यतिरिक्ता इत्येव घोषणा । ऐन्द्रि\add{यक}त्वे सति समस्तरूपादिग्राहकवाक्येन्द्रियान‚{\tiny $_{lb}$}‚वच्छेद्यत्वात्प्रीत्यादिवदिति त ‚{\tiny $_{6}$}‚ दपहस्तितं । प्रयोगाः पुनः । यद् दृश्यं सत्सद्व्यतिरेकेण ‚{\tiny $_{lb}$}‚नोपलभ्यते तत्ततो भिन्नन्नाभ्युपेयन्नास्तीति वाभ्युपगन्तव्यं । यथा नरशिरसि ‚{\tiny $_{lb}$}‚विषाणन्नोपलभ्यते च दृश्यं ‚{\tiny $_{7}$}‚ सन्नीलादिषु तद्व्यतिरेकेण सामान्यविशेषसंयोग‚{\tiny $_{lb}$}‚विभागपरत्वापरत्वादिकमिति स्वभावानुपलब्धिः । नास्य सिद्धिः । दृश्यत्वेन ‚{\tiny $_{lb}$}‚स्वयमभ्युपगमात् । तथा ‚{\tiny $_{8}$}‚ रूपाद्यर्थपञ्चकव्यतिरिक्तत्वेनोपगतं द्रव्यन्नचक्षुःप्रत्यया‚{\tiny $_{lb}$}‚वसेयमुपलब्धिलक्षणप्राप्तत्वेनोपगतत्वे सति नीलादिवस्तुरूपविरहात् । शब्द ‚{\tiny $_{lb}$}‚ \leavevmode\ledsidenote{\textenglish{37/s}} गन्धरसवत् । न च ‚{\tiny $_{9}$}‚ \leavevmode\ledsidenote{\textenglish{27a/msK}} तस्मादव्यतिरिक्त एवायं तत्र चायन्दोष इत्यागूर्याह । सोतिशयो ‚{\tiny $_{lb}$}‚व्यवच्छेदलक्षणस्तस्यातिशयवतोऽवस्थातुरात्मभूतोऽनन्वय इत्येकान्तेन निवर्त्त्त‚{\tiny $_{lb}$}‚मानः । व्या ‚{\tiny $_{1}$}‚ पकत्वाभावात्प्रवर्तमानोऽसन्नेव कथन्न स्वभावनानात्वं सुखदुःख‚{\tiny $_{lb}$}‚धीरिवाकर्षति । अन्वाकर्षत्येवेत्यर्थः ।प्रयोगो\edtext{}{\lemma{प्रयोगो}\Bfootnote{? प्रयोगः}}पुनर्यो यस्यात्मभूतः ‚{\tiny $_{lb}$}‚स तन्निवृ ‚{\tiny $_{2}$}‚ त्तावेकान्तेन निवर्त्तते प्रवृत्तौ चासन्नेव प्रवर्त्तते यथा तस्यैवातिशयस्यात्मा । ‚{\tiny $_{lb}$}‚आत्मभूतश्चातिशयस्यातिशयवानिति स्वभावहेतुः । ततश्च तयोरवस्थ ‚{\tiny $_{3}$}‚ योरवस्था‚{\tiny $_{lb}$}‚तुर्न्नानात्वम्परस्परविरोधिपर्य्याध्यासितत्वात् सुखदुःखवदितिस्वभावहेतुरेव । न वा ‚{\tiny $_{lb}$}‚सावतिशयोऽनन्वयः प्रवर्त्तते निव ‚{\tiny $_{4}$}‚ र्त्तते वाऽतः पुर्वस्मिन् प्रमाणे साध्यविकल‚{\tiny $_{lb}$}‚त्वन्दृष्टान्तस्य । उत्तरत्र त्वसिद्धिर्हेतोरिति चेदाह । ‚{\color{DodgerBlue3}‚सान्वयत्वे} चातिशयस्य ‚{\tiny $_{lb}$}‚निवृत्तिप्रवृत्योरं ‚{\tiny $_{5}$}‚ गीक्रियमाणे ‚{\color{DodgerBlue3}‚का कस्य निवृत्तिः प्रवृत्तिर्वेति} ‚{\tiny $_{4b4}$}‚ । नैव ‚{\tiny $_{lb}$}‚काचित् कस्यचिन्निवृत्तिः प्रवृत्तिर्वा । सर्वस्य सर्वदा सत्वात् । तथा च सर्वं सर्वत्र ‚{\tiny $_{lb}$}‚समुपयु ‚{\tiny $_{6}$}‚ ज्येतेत्यादिना पुरोनुक्रान्तो दोषोनुपयुज्यत इत्यभिप्रायः । उपचयमाह । ‚{\tiny $_{lb}$}‚यदि च कस्यचित् स्वभावस्यातिशयाख्यस्य प्रवृत्तिर्निवृत्तिर्वेति स्वयमभ्यनुज्ञा ‚{\tiny $_{7}$}‚ ‚{\tiny $_{lb}$}‚यते त्वया । एकातिशयनिवृत्याऽपरातिशयोत्त्पत्या व्यवहारभेदोपगमादित्यभि‚{\tiny $_{lb}$}‚धानात् तदेतदेव परस्तथागतवचोऽभ्यासोपजातावदातमति ‚{\tiny $_{8}$}‚ र्ब्रुवाणः । नानु‚{\tiny $_{lb}$}‚मन्यते भद्रमुखेण\edtext{}{\lemma{भद्रमुखेण}\Bfootnote{? मुखेन}}भवेदेवं यदि यथामया प्रवृत्तिनिवृत्ती अभ्यनुज्ञायेते तथा ‚{\tiny $_{lb}$}‚तेनापि । यावतास्य निरन्वयोपजननविनाशोपगमसम ‚{\tiny $_{9}$}‚ \leavevmode\ledsidenote{\textenglish{27b/msK}} त्वाद् द्रव्यस्यालोकनीलादि- ‚{\tiny $_{lb}$}‚वत्तद्व्यतिरेकेणाप्रतिभासनमभिन्नेंद्रियग्राह्यत्वाद्वा तद्वदेवेत्यतआह । ‚{\color{DodgerBlue3}‚प्रतिभास‚{\tiny $_{lb}$}‚मानाश्च विवेकेने} ‚{\tiny $_{3b1}$}‚ दं नीलमिदं सुरभि मधुरं कर्कशमिद ‚{\tiny $_{1}$}‚ मिति चेति प्रत्यक्षा ‚{\tiny $_{lb}$}‚अर्था दृश्यन्तेऽपृथग्देशत्वेपि सति के ते रूपादयः । लोकप्रसिद्ध्या चेदमुक्तन्न तु ‚{\tiny $_{lb}$}‚तेषामभिन्नदेशत्वमस्ति । सप्रतिघा दश रूपिण \href{http://sarit.indology.info/?cref=ak.1.29}{अभिधर्मकोशे १।२९ } इति वचनात् । ‚{\tiny $_{lb}$}‚त ‚{\tiny $_{2}$}‚ था ऽभिन्नेन्द्रिय ‚{\color{DodgerBlue3}‚ग्राहयत्वेपि वातातपस्पर्शादय} ‚{\tiny $_{3ba}$}‚ इति यथाक्रमं चैतदुत्तरं । ‚{\tiny $_{lb}$}‚अनेनैकान्तिकत्वं हेतोरुद्भावयति । उपेत्य च धर्मिसम्बन्धं अभिन्नेन्द्रियग्राह्यत्वस्य ‚{\tiny $_{lb}$}‚व्यभि ‚{\tiny $_{3}$}‚ चार उक्तो नत्वसावस्यास्त्येकदेशासिद्धेः । कथं । यतो न सुरभिमधुरादयो ‚{\tiny $_{lb}$}‚द्रव्यग्राहकेन्द्रियग्राह्याः । सार्वेन्द्रियत्वप्रसङ्गाद् द्रव्यस्य । आलोकनीलादीनां ‚{\tiny $_{4}$}‚ त्वभेद ‚{\tiny $_{lb}$}‚एव यतः प्रदीपादिसन्निधानात् प्रकाशरूपा एव तथाविधस्वभावाध्यासित‚{\tiny $_{lb}$}‚वपुषस्ते समुद्भवन्ति न तु तेषां भेदोऽस्तीति साधनविकल्पताऽपि दृ ‚{\tiny $_{5}$}‚ ष्टान्तस्येति ‚{\tiny $_{lb}$}‚मन्यते । तस्मादस्य प्रत्यक्षत्वमभ्युपगच्छद्भिर्न बहिरवश्यं रूपादिविवेकेन प्रति‚{\tiny $_{lb}$}‚ \leavevmode\ledsidenote{\textenglish{38/s}} भासनमभ्युपगन्तव्यमन्यथा प्रत्यक्षत्वासिद्धेः । कुतो य ‚{\tiny $_{6}$}‚ स्मादिदमेवेत्यादि सुबोधं । ‚{\tiny $_{lb}$}‚अयम्पुनर्घटादिर्भवद्भीरूपादिव्यतिरेकेणाभ्युपगतः । को सावमूल्यदानक्रयीयः । ‚{\tiny $_{lb}$}‚तदेतेन नायमीदृशो लोकव्यवहारपद्धति ‚{\tiny $_{7}$}‚ मवतरतीत्याचष्टे । स स्वरूपञ्च ‚{\tiny $_{lb}$}‚नोत्कर्षेण दर्शयत्यप्रतिभासमानत्वाप्रत्यक्षताञ्च स्वीकर्त्तुमिच्छति दार्शनं स्पार्श‚{\tiny $_{lb}$}‚नञ्च द्रव्यमिति सिद्धान्ते पाठात् । इत्येतदात्मनि ‚{\tiny $_{8}$}‚ रन्तरप्रेमाणः सुहृदः प्रत्येष्यन्ती‚{\tiny $_{lb}$}‚त्त्यध्याहर्त्तव्यं । मूल्यदानक्रया विद्यन्तेस्येति मूल्यदानक्रयी न तथेति वृत्तिः । अथवा ‚{\tiny $_{lb}$}‚क्रेतुं शीलं यस्यासौ क्रयी मूल्यदानेन क्र ‚{\tiny $_{9}$}‚ \leavevmode\ledsidenote{\textenglish{28a/msK}} यी न तथा । कथमेतदित्याह । ‚{\color{DodgerBlue3}‚यः प्रत्यक्षता} ‚{\tiny $_{lb}$}‚ ‚{\tiny $_{3b2}$}‚ मित्यादि । मूल्यदानञ्चात्र स्वरूपार्पणमित्युपहसति ।
	{\color{gray}{\rmlatinfont\textsuperscript{§~\theparCount}}}
	\pend% ending standard par
      ‚{\tiny $_{lb}$}‚

	  
	  \pstart \leavevmode% starting standard par
	ननु चैको घट इति प्रत्ययव्यपदेशसद्भावाद्रूपादिवत् तदस्त्येव । तत्कथ‚{\tiny $_{lb}$}‚मस्यासत्वमिति चेदाह ‚{\color{DodgerBlue3}‚बुद्धिशब्दादयोपि व्याख्या} ता न च सर्व्व इत्यादिना । आदि‚{\tiny $_{lb}$}‚शब्देन तद्भेदाभेदोपादानं । यदि तैर्वुद्धिव्यपदेशादिभिस्तस्य साधनसिद्धि ‚{\tiny $_{2}$}‚ ‚{\tiny $_{lb}$}‚रिष्यते । स्यादेतत्प्रतिभासमानमपि द्रव्यं लवणरसाभिभवे खण्डरसवन्नोपलक्ष्यते । ‚{\tiny $_{lb}$}‚ततस्तत्प्रसाधनाय लिङ्गमुच्यत इत्यत आह । ‚{\color{DodgerBlue3}‚न च प्रत्यक्षस्यार्थस्य रूपानुप ‚{\tiny $_{3}$}‚ ल‚{\tiny $_{lb}$}‚क्षणं} ‚{\tiny $_{3b3}$}‚ युक्तं । द्रव्यान्तरेणानभिभवे सति । अभिभवे तु युक्तमेव । यथा ‚{\tiny $_{lb}$}‚खण्डादिरसस्य । न चात्र केनचिदभिभवोस्ति । नीलादिभिरस्तीति चेति । न ‚{\tiny $_{lb}$}‚महत्यने ‚{\tiny $_{4}$}‚ कद्रव्यवत्ताद्रूपाच्चोपलद्धिः । तथा रूपसंस्काराभावाद्या वानुपलब्धि‚{\tiny $_{lb}$}‚रित्युक्तं । तस्य चानुपलक्षणे तेषामप्यनु\add{प}लक्षणप्रसङ्गः । ततश्च स ‚{\tiny $_{5}$}‚ र्व्व‚{\tiny $_{lb}$}‚पदार्थानामनुपलक्षणाल्लोकव्यवहारोच्छेद एव भवेदिति मन्यते । येनानुपलक्षणेन ‚{\tiny $_{lb}$}‚तस्यावयविनः साधनाय लिङ्गमुच्यते । तद्भावसाध ‚{\tiny $_{6}$}‚ नञ्च लिङ्गमभ्युपगम्येत ‚{\tiny $_{lb}$}‚तद्भाष्यते न तु तद्गमकं लिङ्गं किञ्चिदप्यस्ति । यथोक्तम्प्राक् । तत्प्रतिपादक‚{\tiny $_{lb}$}‚प्रमाणाभावेपि तदस्त्येवेति चेदाह \add{।} ‚{\color{DodgerBlue3}‚अप्रत्यक्षत्वेप्यप्रमाण ‚{\tiny $_{7}$}‚ स्य सत्वोपगमो‚{\tiny $_{lb}$}‚ऽयुक्त} ‚{\tiny $_{3b3}$}‚ इति । अप्रमाणस्येत्यनेन प्रत्यक्षव्यतिरिक्ततत्प्रसाधकप्रमाणा‚{\tiny $_{lb}$}‚भावमाह । यस्य ‚{\color{DodgerBlue3}‚सद्भावसाधकं प्रमाणं नास्ति न तदस्ती} त्यङ्गीकर्त्त ‚{\tiny $_{8}$}‚ व्यं । यथा‚{\tiny $_{lb}$}‚नभस्तले कमलं नास्ति चावयविनोऽस्तित्वसाधकं प्रमाणमिति सद्व्यवहारप्रति‚{\tiny $_{lb}$}‚षेधफलामनुपलब्धिं मन्यते । एवं विस्तरेणैकद्रव्याभावं प्र ‚{\tiny $_{9}$}‚ \leavevmode\ledsidenote{\textenglish{28b/msK}} तिपाद्य प्रकृतमुप‚{\tiny $_{lb}$}‚ \leavevmode\ledsidenote{\textenglish{39/s}} संहरति । ‚{\color{DodgerBlue3}‚तदि} त्यादिना । मूलप्रकरणमपि निगमयति । ‚{\color{DodgerBlue3}‚एवन्ताव} दि ‚{\tiny $_{3b3}$}‚ ‚{\tiny $_{lb}$}‚त्यादिना । अत एव न तेषाम्बुद्ध्यादीनाम्विपर्ययात्तेषां सत्तादीनाम्वि ‚{\tiny $_{1}$}‚ पर्ययोऽभावः । ‚{\tiny $_{lb}$}‚योहि यस्य भावमेव न साधयति स कथमिव वर्त्तमानस्तदभावं साधयतीत्याकूतं । ‚{\tiny $_{lb}$}‚यदि नाम बुद्ध्यादयः सत्ताम्भेदाभेदौ वा न साधयन्त्यर्थक्रि ‚{\tiny $_{2}$}‚ या तु तान्साधयिष्यतीत्यत ‚{\tiny $_{lb}$}‚आह । ‚{\color{DodgerBlue3}‚अर्थंक्रियातस्तु सत्ताव्यवहारः स्यादि} ‚{\tiny $_{3b4}$}‚ ति तल्लक्षणत्वात् सत्त्वस्येति ‚{\tiny $_{lb}$}‚भावः । अनेनावयोरत्र साम्यमेवेति दर्शयति । अन्य ‚{\tiny $_{3}$}‚ त्र तु विवाद इत्याह । ‚{\color{DodgerBlue3}‚न सत्ता‚{\tiny $_{lb}$}‚भेदाभेदव्यवहार} ‚{\tiny $_{3b4}$}‚ इति । कुत एकस्याप्यनेकार्थक्रियादर्शनात् । तत्र नैक‚{\tiny $_{lb}$}‚प्रत्ययजनितं किञ्चिदस्ति तत्कथ ‚{\tiny $_{4}$}‚ मेवमुच्यते । सत्यमेतदेकं तु बहवीषु सामग्रीषु ‚{\tiny $_{lb}$}‚वर्त्तत इत्यनेकार्थकृदित्युच्यते । यदाह ॥
	{\color{gray}{\rmlatinfont\textsuperscript{§~\theparCount}}}
	\pend% ending standard par
      ‚{\tiny $_{lb}$}‚
	  \bigskip
	  \begingroup
	
	    
	    \stanza[\smallbreak]
	  \flagstanza{\tiny\textenglish{...11}}{\normalfontlatin\large ``\qquad}न किञ्चदेकमेकस्मात् सामग्र्या सर्व्वसम्भवः ।&‚{\tiny $_{lb}$}‚एकं‚{\tiny $_{5}$}‚स्यादपि सामग्र्योरित्युक्तं तदनेककृदिति ॥ \add{११}{\normalfontlatin\large\qquad{}"}\&[\smallbreak]
	  
	  
	  
	  \endgroup
	‚{\tiny $_{lb}$}‚

	  
	  \pstart \leavevmode% starting standard par
	किम्वत् । यथा प्रदीपस्य विज्ञानस्य वर्त्तिविकारस्य ज्वालान्तरस्य च स्वपर‚{\tiny $_{lb}$}‚सन्तानसम्बन्धिक्षणान्तरस्यो ‚{\tiny $_{6}$}‚ त्पादनानि तदेवं सत्ताभेदव्यवहाराभावे कारणं । ‚{\tiny $_{lb}$}‚कथमेकमनेकं कार्यमुत्पादयतीति चेति । एकस्यैव ईदृशस्यानेककार्यजननात्तुर्या‚{\tiny $_{lb}$}‚तिशयक्रोडी ‚{\tiny $_{7}$}‚ कृतं रूपवतः स्वहेतुभ्यः संजातत्त्वादिति भावो न्यायतत्त्वविदः । ‚{\tiny $_{lb}$}‚तथानेकस्यापि चक्षुरादेरेकविज्ञानक्रियादर्शनात् । अभेदव्यवहाराभावे कार ‚{\tiny $_{8}$}‚ ‚{\tiny $_{lb}$}‚णमेतत् । ‚{\color{DodgerBlue3}‚कणभु} ग्मतविपर्यासितधियस्त्वाहुः । ‚{\color{DodgerBlue3}‚न ब्रूम} इत्यादि । किन्तर्ह्यदृष्टार्थ ‚{\tiny $_{lb}$}‚क्रियाभेदेन सत्ताभेद इति वर्त्तते । तदेव व्यनक्ति । यार्थक्रिया ‚{\tiny $_{3b6}$}‚ मध्वाद्या ‚{\tiny $_{9}$}‚ \leavevmode\ledsidenote{\textenglish{29a/msK}} ‚{\tiny $_{lb}$}‚हरणादिलक्षणा तस्मिन्घटादावदृष्टा सती पुनर्दृश्यते । अन्यत्र घटादौ । ‚{\tiny $_{lb}$}‚सैवम्विधा सत्ताभेदं साधयति । तेषां घटादीनामनेन व्याप्तिः कथिता । किमिव । ‚{\tiny $_{1}$}‚ ‚{\tiny $_{lb}$}‚ ‚{\color{DodgerBlue3}‚यथा पटेऽदृष्टा सत्युदकधारणाद्यर्थक्रिया घटे दृश्यमाना} ‚{\tiny $_{3b7}$}‚ सत्ताभेदं साधय‚{\tiny $_{lb}$}‚तीति प्रकृतेनाभिसम्बन्धः । दृष्टान्तकथनं चैतत् । ‚{\color{DodgerBlue3}‚अदृष्टा च तन्तुषु प्रावरणद्य ‚{\tiny $_{2}$}‚ र्थ‚{\tiny $_{lb}$}‚\leavevmode\ledsidenote{\textenglish{40/s}} क्रिया पटे दृश्यत इति} ‚{\tiny $_{3b7}$}‚ पक्षधर्मोपदर्शनमिति । तस्मात् सत्ताभेदस्तन्तु‚{\tiny $_{lb}$}‚पटयोः सिद्ध इति शेषः । तदनेन साधनफलं सङ्कीर्तितं । स्वभावहेतुश्चायं ‚{\tiny $_{3}$}‚ यस्मात्तद‚{\tiny $_{lb}$}‚दृष्टार्थक्रियाकरणमात्रानुबन्धी सत्ताभेद इति । तदेतेन तन्तुभ्यः पटस्यान्यत्वं ‚{\tiny $_{lb}$}‚साधयन्नर्थान्तरभूतावयविसिद्धिं मन्यते ॥ ‚{\color{DodgerBlue3}‚आचार्यस्त्} वाह ॥ ‚{\color{DodgerBlue3}‚सिध्यत्येवं तन्तु‚{\tiny $_{lb}$}‚पटयोः सत्ताभेद} इति प्रकृतं ॥ वांछितार्थसिद्धिस्तु भवतो नैवास्तीत्यभिप्रायवा ‚{\tiny $_{lb}$}‚नाह । ‚{\color{DodgerBlue3}‚अर्थान्तरन्तथाप्यवयवी न ‚{\tiny $_{5}$}‚ सिध्यतीति} ‚{\tiny $_{3b7}$}‚ कुत एतद्यतो यथाप्रत्यय‚{\tiny $_{lb}$}‚मस्यां संस्कारसंततौ स्वभावभेदोत्पत्तेः कारणादर्थक्रियाभेदः प्रावरणादिलक्षणो ‚{\tiny $_{lb}$}‚भवति । ‚{\tiny $_{6}$}‚ एतदुक्तं भवति पिण्डीकृतेभ्यस्तन्तुभ्य उपादानकारणभूतेभ्यः ‚{\tiny $_{lb}$}‚कुविन्दादिसहकारिप्रत्ययसन्निधानाच्च विशिष्टसन्निवेशावच्छिन्ना एव ‚{\tiny $_{lb}$}‚तन्तवो जायं ‚{\tiny $_{7}$}‚ ते । ये प्रावरणाद्यर्थक्रिया\add{या}मुपयुज्यन्ते । तेभ्यश्च पूर्व्वेभ्यः पटस्या‚{\tiny $_{lb}$}‚न्यत्वमिष्टमेवास्माभिरपि । न तु विशिष्टसंस्थानावच्छिन्नेभ्य इति त्यज्यतामियमर्था‚{\tiny $_{lb}$}‚न्तरा ‚{\tiny $_{8}$}‚ \leavevmode\ledsidenote{\textenglish{29b/msK}} वयविसिद्धिप्रत्याशेति । तदेतेनैवा ‚{\color{DodgerBlue3}‚विद्धकर्ण्णो} क्तं पूर्व्वोत्तरकालभावित्वादित्यादि ‚{\tiny $_{lb}$}‚तत्साधनमपहस्तितं वेदितव्यं । अस्माभिस्तु विस्तरेण प्राक् प्रयुक्तमेवेति । ‚{\tiny $_{lb}$}‚न पुनर्योज्यते । किम्वत् प्रत्ययवसात्\edtext{}{\lemma{प्रत्ययवसात्}\Bfootnote{? वशात्}}स्वभावविशोषोत्पत्ते ‚{\color{DodgerBlue3}‚रर्थकिया‚{\tiny $_{lb}$}‚भेद} ‚{\tiny $_{3b7}$}‚ इत्याह । ‚{\color{DodgerBlue3}‚अरणिनिर्मथना} ‚{\tiny $_{3b7}$}‚ दित्यादि सुज्ञानं \add{।} दृष्टान्तं प्रदर्श्य‚{\tiny $_{lb}$}‚दार्ष्टान्तिकमाह । ‚{\color{DodgerBlue3}‚तथा यथे} ‚{\tiny $_{3b8}$}‚ त्यादि ‚{\tiny $_{1}$}‚ । अनेनैव यथाप्रत्ययस्वभावभेदेन‚{\tiny $_{lb}$}‚यदेके चोदयन्ति । ननु च तन्तवः पट इति बुद्धिव्यपदेशभेदात् । कथमस्यान्यत्वं ‚{\tiny $_{lb}$}‚नास्तीति तत्प्रतिक्षिप्तमित्याकूतवानाह ‚{\tiny $_{2}$}‚ । ‚{\color{DodgerBlue3}‚एतेन बुद्धिव्यपदेशभेदौ व्याख्याता‚{\tiny $_{lb}$}‚विति} ‚{\tiny $_{3b8}$}‚ । तत्रैवं स्थिते यदुक्तं प्राक्त्वयाऽर्थक्रियातः सद्व्यवहारसिद्धिर्भवति ‚{\tiny $_{lb}$}‚विपर्ययाच्चार्थक्रिया निवृत्ते विपर्ययोऽसद्व्यवहार ‚{\tiny $_{3}$}‚ इति सत्यमेतत् । एतावत्तु‚{\tiny $_{lb}$}‚ब्रूमः । स एव विपर्ययोऽर्थक्रियाया अनुपलब्धिलक्षणप्राप्तेषु न सिध्यति । तथाहि ‚{\tiny $_{lb}$}‚यद्ययमुपलब्धिलक्षणप्राप्तानुप ‚{\tiny $_{4}$}‚ लम्भो नेष्यते तदास्यार्थक्रियासामर्थ्यं नास्तीति ‚{\tiny $_{lb}$}‚कथमधिगतं भवता \add{।} न चानुपलब्धिमात्रादिति युक्तम्वक्तुँ । तस्य व्यभिचारात् । ‚{\tiny $_{lb}$}‚ \leavevmode\ledsidenote{\textenglish{41/s}} तत्र पुनरनिच्छतो ‚{\tiny $_{5}$}‚ प्यायातं तव । यस्येदमर्थक्रियासामर्थ्यमुपलब्धिलक्षणप्राप्तं ‚{\tiny $_{lb}$}‚सन्नोपलभ्यते सोऽसद्व्यवहारविषय इति । कुतः सामर्थ्यलक्षणत्वात् सत्वस्य ‚{\tiny $_{6}$}‚ । ‚{\tiny $_{lb}$}‚भवत्वेवङ्को दोष इति चेदाह । ‚{\color{DodgerBlue3}‚तथापि कोतिशयः} ‚{\tiny $_{3b10}$}‚ पूर्व्वकादस्मादुप‚{\tiny $_{lb}$}‚वर्णितादुपलभ्यानुपलम्भात् । अस्य सामर्थ्यानुपलम्भस्य भवत्परिकल्पितस्य । ‚{\tiny $_{lb}$}‚स्यात् म ‚{\tiny $_{7}$}‚ तं स स्वभावस्यैवानुपलम्भोऽयं तु पुनः सामर्थ्यस्येत्यत आह । ‚{\color{DodgerBlue3}‚न ही}‚{\tiny $_{3b10}$}‚ ‚{\tiny $_{lb}$}‚त्यादि । तथा च तस्य सामर्थ्यस्य योनुपलम्भः स स्वभावस्यैव इति तस्मात् ‚{\tiny $_{lb}$}‚पूर्वकैव स्वभावा ‚{\tiny $_{8}$}‚ नुपलब्धिरेवेयं सामर्थ्यानुपलब्धिः । तस्माद् दृश्यानुपलब्धिरेवा‚{\tiny $_{lb}$}‚सद्व्यवहारसाधनेति स्थितमेतत् । यस्माच्चैवन्तस्मादनेन वादिना क्वचिच्छश‚{\tiny $_{lb}$}‚विषाणादावसद्व्यव ‚{\tiny $_{9}$}‚ \leavevmode\ledsidenote{\textenglish{30a/msK}} वहारमभ्युपगच्छता दृश्यानुपलम्भादभ्युपगन्तव्यो- ‚{\tiny $_{lb}$}‚ऽन्यस्य तत्प्रतिपत्त्युपायस्याभावादिति भावः । ततः सोनुपलम्भोऽन्यत्रापि सामान्य‚{\tiny $_{lb}$}‚विशेषसंयोगावय ‚{\tiny $_{1}$}‚ विद्रव्यादौ तथाविधे उपलब्धिलक्षणप्राप्ते अविशेष इति ‚{\tiny $_{lb}$}‚सोपि सामान्यविशेषादिस्तथास्त्वसद्व्यवहारविषयत्त्वेनास्तीत्यर्थः । स्यादेतन्नैव ‚{\tiny $_{lb}$}‚सामान्यविशे ‚{\tiny $_{2}$}‚ षादिस्तथाविधोनुपलब्धोस्य सद्व्यवहारविषय इत्यत आह । न वा ‚{\tiny $_{lb}$}‚क्वचिच्छशविषाणादावसद्व्यवहारो ऽभ्युपगन्तव्यः । कुतो विशेषाभावादनुप ‚{\tiny $_{lb}$}‚लम्भस्य ‚{\tiny $_{3}$}‚ । अयमत्रार्थो द्व्योरनुपलम्भे तन्निमित्ते तुल्येपि यद्यसद्व्यवहारः ‚{\tiny $_{lb}$}‚सामान्यादौ नोत्पद्यते । अन्यत्रापि तर्हि स नाभ्युपेयो विशेषहेत्वभावादि ‚{\tiny $_{4}$}‚ त्यनेन च ‚{\tiny $_{lb}$}‚पूर्व्वोक्तमेव कस्य चिदसतोभ्युपगमे तल्लक्षणाविशेषादिति स्मारयति । तस्मा ‚{\tiny $_{lb}$}‚ ‚{\color{DodgerBlue3}‚त्सर्व्व एवंविधो दृश्यानुपलम्भोऽसद्व्यवहारस्य वि ‚{\tiny $_{5}$}‚ षय} ‚{\tiny $_{4a2}$}‚ इति व्याप्तिः । ‚{\tiny $_{lb}$}‚अनुपलब्धौ सिद्धेतिशेषः ॥ ० ॥
	{\color{gray}{\rmlatinfont\textsuperscript{§~\theparCount}}}
	\pend% ending standard par
      ‚{\tiny $_{lb}$}‚\textsuperscript{\textenglish{42/s}}

	  
	  \pstart \leavevmode% starting standard par
	\hphantom{.}‚{\color{DodgerBlue3}‚कापिला} स्त्वाहुः । सर्व्वमेव सर्व्वात्मकं अन्यथा यदि मृत्पिण्डदुग्धबीजादिषु ‚{\tiny $_{lb}$}‚घटदध्यङ्कुराद ‚{\tiny $_{6}$}‚ यो न विद्यन्त एव शक्त्यात्मना तदा कथं पुनस्तेभ्यस्तेषामुत्पत्तिः । ‚{\tiny $_{lb}$}‚नहि शशविषाणमविद्यमानन्तत्रोदेति । एवञ्च सति दृश्यः सन्ननुपलब्धोपि कश्चिद् ‚{\tiny $_{lb}$}‚घटा ‚{\tiny $_{7}$}‚ दिः क्वचिद्देशादौ कथञ्चिच्च संस्थानविशेषादिना नैवाभावव्यवहारविषयो ‚{\tiny $_{lb}$}‚भवतीति तेषां मतमासङ्क\edtext{}{\lemma{मतमासङ्क}\Bfootnote{? शङ्क}}ते । ‚{\color{DodgerBlue3}‚नैवे} ‚{\tiny $_{4a3}$}‚ त्यादिना ।
	{\color{gray}{\rmlatinfont\textsuperscript{§~\theparCount}}}
	\pend% ending standard par
      ‚{\tiny $_{lb}$}‚
	  \bigskip
	  \begingroup
	
	    
	    \stanza[\smallbreak]
	  \flagstanza{\tiny\textenglish{...12}}{\normalfontlatin\large ``\qquad}एतत् साङ्ख्यपशोः कोन्यः सलज्जो वक्तुमीहते ।&‚{\tiny $_{lb}$}‚अदृष्टपूर्व्वमस्तीति तृणाग्रे करिणां शतम् \add{। १२} \href{http://sarit.indology.info/?cref=pv.2.0}{प्रमाणवार्त्तिके २ परि०}{\normalfontlatin\large\qquad{}"}\&[\smallbreak]
	  
	  
	  
	  \endgroup
	‚{\tiny $_{lb}$}‚

	  
	  \pstart \leavevmode% starting standard par
	\hphantom{.}इत्यभिप्रायवानाह । ‚{\color{DodgerBlue3}‚सर्व्वस्ये} ‚{\tiny $_{4a3}$}‚ त्यादि । यद्यदृष्टमपि तत्रास्ति तदा‚{\tiny $_{lb}$}‚सर्व्व एव क्षीरादयः सर्व्वे र्घटादिरूपै ‚{\tiny $_{9}$}‚ \leavevmode\ledsidenote{\textenglish{30b/msK}} रनुमतत्वात् तत्साध्यामर्थक्रियाङ्कुर्युरित्यर्थः । ‚{\tiny $_{lb}$}‚किञ्चेदमपरं न स्यादिदन्दध्यादिकार्यमतः क्षीरादेर्भवति नान्यतो जलादेः । ‚{\tiny $_{lb}$}‚यदि वा नातः क्षीरादेरिदं मध्वादि ‚{\tiny $_{1}$}‚ कं तथेदङ्कुङ्कुमादिकमिह ‚{\color{DodgerBlue3}‚कस्मीरा\edtext{}{\lemma{कस्मीरा}\Bfootnote{? ‚{\tiny $_{lb}$}‚कश्मीरा}}} दिदेशे नेदमिह ‚{\color{DodgerBlue3}‚मालवकादि} देशे यदि वा नेदं चन्दनादिकमिह । तथेदङ्कु‚{\tiny $_{lb}$}‚न्दादिकमिदानीं शिशिरसमये नत्विदमि ‚{\tiny $_{2}$}‚ दानीन्निदाघकाले । अथवा नेदं कमलादिक‚{\tiny $_{lb}$}‚मिदानीन्तथेदं खण्डादिकमेवं माधुर्यादिगुणविशिष्टं । नेदमेवङ्कटुकादिरूपं । ‚{\tiny $_{lb}$}‚यद्वा नेदं निम्वादिकमे ‚{\tiny $_{3}$}‚ वमिति व्याख्यातव्यं \add{।} किङ्कारणमेवमेतदित्यत आह । ‚{\tiny $_{lb}$}‚ ‚{\color{DodgerBlue3}‚कस्यचिदपी} ‚{\tiny $_{4a4}$}‚ त्यादि । इदमेवम्बिधमधुनास्य रूपन्नास्तीति योयम्विवेको‚{\tiny $_{lb}$}‚ऽभावस्त ‚{\tiny $_{4}$}‚ स्य हेतोरभावात् । सर्व्वस्य सर्व्वरूपाणां सर्व्वदानुवृत्तेरिति मतिः । नन्विद‚{\tiny $_{lb}$}‚मनन्तरमेव वस्तुतोऽभिहितमेव सर्व्व सर्व्वत्र सर्व्वदा समुप ‚{\tiny $_{5}$}‚ युज्येतेत्यत्र तक्तिमिदं ‚{\tiny $_{lb}$}‚पुनश्चर्वितचर्व्वणमास्थीयत इति चेत् । सत्यं पूर्वं कारणगतो व्यापारः कथितो‚{\tiny $_{lb}$}‚ऽधुना तु कार्यगत इति विशेषा ‚{\tiny $_{6}$}‚ ददोषः । इदञ्चान्यतरमुखेण\edtext{}{\lemma{इदञ्चान्यतरमुखेण}\Bfootnote{? मुखेन}}दूषण‚{\tiny $_{lb}$}‚वचनं शिष्यव्युत्पादनाय । ततश्च भेदाभावान्न विद्येते अन्वयव्यतिरेकौ यस्मिं ‚{\tiny $_{lb}$}‚निति विग्रहः । इदमत्रास्तीत्या ‚{\tiny $_{7}$}‚ द्यन्वयो नास्तीति व्यतिरेकः । परः प्राह \add{।} ‚{\tiny $_{lb}$}‚ ‚{\color{DodgerBlue3}‚अवस्थेत्या} दिना । एतदुक्तम्भवति । यत्र यद् व्यक्तन्तत्तत्रास्तीत्यादि व्यवहिरयते ‚{\tiny $_{lb}$}‚ \leavevmode\ledsidenote{\textenglish{43/s}} यत्र तु यन्नैव व्यक्तन्तत्र तन्नास्तीत्यतो ‚{\tiny $_{8}$}‚ ऽयमदोष इति । ‚{\color{DodgerBlue3}‚नत्वि} ‚{\tiny $_{4a5}$}‚ त्याद्याचार्यः । ‚{\tiny $_{lb}$}‚‘त एवावस्थानिवृत्तिप्रवृत्तिभेदा न सम्भवन्ति तावकीने दर्शने । कुतः सर्व्वविषयस्या‚{\tiny $_{lb}$}‚सद्व्यवहारस्याभावात् । अथापि क्वचि ‚{\tiny $_{9}$}‚ \leavevmode\ledsidenote{\textenglish{31a/msK}} द्विषयेऽसदव्यवहार इष्यते तदा तस्य कारणं ‚{\tiny $_{lb}$}‚भवद्भिर्वक्तव्यमित्याह । क् ‚{\color{DodgerBlue3}‚वचिदि} ‚{\tiny $_{4a5}$}‚ त्यादि । यदि वावश्यमनेन क्वचित्परि‚{\tiny $_{lb}$}‚कल्पिते व्यतिरिक्तावयव्यादावसद्व्यवहारों ‚{\tiny $_{1}$}‚ गीकर्त्तव्यः । स चास्य न युक्तो‚{\tiny $_{lb}$}‚हेत्वभावादित्याह । ‚{\color{DodgerBlue3}‚क्वचिदि} त्यादि । कुतः । यस्मान्नहि अनुपलम्भादन्यो व्यवच्छे‚{\tiny $_{lb}$}‚दस्याभावस्य हेतुरस्ति प्रसाधक इति शेषः । स च त्वया ‚{\tiny $_{2}$}‚ नेष्टक इति भावः । कस्मा‚{\tiny $_{lb}$}‚देवं यतो विधिना स्वभावविरुद्धोपलम्भादौ प्रतिषेधेन व्यापकानुपलम्भादौ व्यवच्छेदे ‚{\tiny $_{lb}$}‚साध्येऽनुपलम्भस्यैव सर्व्वदा साधकत्वात् । अथा ‚{\tiny $_{3}$}‚ हमप्यस्मादेवानुपलम्भाद् व्यव‚{\tiny $_{lb}$}‚च्छेदं साधयामीति ब्रूषे तदत्रापि ब्रूम इत्याह । सोनुपलम्भो ‚{\color{DodgerBlue3}‚यत्रैवास्ति स सर्व्वो ‚{\tiny $_{lb}$}‚ऽसद्व्यवहारविषय} ‚{\tiny $_{4a6}$}‚ इति वक्तव्यं ॥ ‚{\tiny $_{4}$}‚ किमिति विशेषाभावात् । तथा ‚{\tiny $_{lb}$}‚चघटादि\edtext{}{\lemma{चघटादि}\Bfootnote{? घटादे}}रपि क्वचित्प्रदेशविशेषादावसद्व्यवहारविषयत्वं सिद्धं । ‚{\tiny $_{lb}$}‚तथाविधस्यानुपलम्भस्यात्रापि भावात् तत्किं ‚{\tiny $_{5}$}‚ ब्रूषे । नैव क्वचित् कश्चिद् दृष्टो‚{\tiny $_{lb}$}‚प्यसद्व्यवहारविषय इति अभिप्रायः । अन्यथा अत्रापि व्यतिरिक्तावयव्यादौ ‚{\tiny $_{lb}$}‚मा भूदसद्व्यवहार इति यावत् ।
	{\color{gray}{\rmlatinfont\textsuperscript{§~\theparCount}}}
	\pend% ending standard par
      ‚{\tiny $_{lb}$}‚

	  
	  \pstart \leavevmode% starting standard par
	\hphantom{.}‚{\color{DodgerBlue3}‚सर्व्व ‚{\tiny $_{6}$}‚ प्रमाणनिवृत्तिरि} ‚{\tiny $_{4a7}$}‚ त्यादि परः । तथा चायुक्तं उक्तं । नहि ‚{\tiny $_{lb}$}‚अनुपलम्भादन्यो व्यवच्छेदहेतुरस्तीति । एवञ्च सति न घटस्यापि क्वचिदसद्व्यव ‚{\tiny $_{lb}$}‚हारविषयत्वमा ‚{\tiny $_{7}$}‚ गमानुमानभावेन सर्व्वप्रमाणनिवृत्तेरेवाभावादिति मन्यते । ‚{\tiny $_{lb}$}‚कुतः पुनरिदमतिप्रज्ञाकौशलमासादितं भवतेत्यागूर्योपहसति । ‚{\color{DodgerBlue3}‚सुकुमारप्रज्ञ} ‚{\tiny $_{4a7}$}‚ ‚{\tiny $_{lb}$}‚इत्या ‚{\tiny $_{8}$}‚ दिना । ‚{\color{DodgerBlue3}‚न प्रसहते प्रमाणचिन्तापरिक्लेश} ‚{\tiny $_{4a7}$}‚ मिति सुकुमारप्रज्ञत्वे ‚{\tiny $_{lb}$}‚कारणं ।
	{\color{gray}{\rmlatinfont\textsuperscript{§~\theparCount}}}
	\pend% ending standard par
      ‚{\tiny $_{lb}$}‚

	  
	  \pstart \leavevmode% starting standard par
	\hphantom{.}ननु च किमत्रायुक्तमुक्तमस्माभिर्येनोपहससीत्याह । ‚{\color{DodgerBlue3}‚नही} ‚{\tiny $_{4a8}$}‚ त्यादि । ‚{\tiny $_{lb}$}‚अदिश ‚{\tiny $_{9}$}‚ \leavevmode\ledsidenote{\textenglish{31b/msK}} ब्देनागमपरिग्रहः । व्यभिचारश्च पूर्वमेव प्रतिपादितः । सर्व्वप्राणि ‚{\tiny $_{lb}$}‚प्रत्यक्षनिवृत्तिस्तर्हि गमयिष्यतीत्याह । ‚{\color{DodgerBlue3}‚न सर्व्वप्रत्यक्षनिवृत्तिरि} ‚{\tiny $_{4a8}$}‚ ति । ‚{\tiny $_{lb}$}‚ \leavevmode\ledsidenote{\textenglish{44/s}} कुतोऽसिद्धेः । आत्मपर ‚{\tiny $_{1}$}‚ योरप्रतिपत्तेरित्यर्थः । न ह्यत्र सर्वेषाम्प्रत्यक्षन्निवृत्तमिति ‚{\tiny $_{lb}$}‚निश्चये प्रमाणमस्ति किञ्चित् । आत्मप्रत्यक्षनिवृत्तिरेव तर्हि गमयतीति चेदाह । ‚{\tiny $_{lb}$}‚ ‚{\color{DodgerBlue3}‚नात्मप्रत्यक्षा विशे ‚{\tiny $_{2}$}‚ षनिवृत्तिरपी} ‚{\tiny $_{4a8}$}‚ ति \add{।} न केवलं पूर्वोक्तेत्यपि शब्दः । ‚{\tiny $_{lb}$}‚अविशेषेण निवृत्तिरविशेषनिवृत्तिः । आत्मप्रत्यक्षस्याविशेषनिवृत्तिरात्मप्रत्यक्षा ‚{\tiny $_{lb}$}‚विशेषनिवृत्तरिति ‚{\tiny $_{3}$}‚ व्युत्पत्तिक्रमः । सन्निहितसकलतदन्यकारणस्य त्वात्म‚{\tiny $_{lb}$}‚प्रत्यक्षस्य निवृत्तिस्त्रिविधविप्रकर्षाविप्रकृष्टेऽभावङ्गमयत्येवेति कथनीयाविशेष ‚{\tiny $_{lb}$}‚वि ‚{\tiny $_{4}$}‚ प्रकृष्टवचनं । यस्मात् सर्व्वप्रमाणनिवृत्तिर्न्नासद्व्यवहारहेतुस्तस्मात् स ‚{\tiny $_{lb}$}‚स्वभावविशेषस्त्रिविधविप्रकर्षाविप्रकृष्टरूपो भावो यतः प्रमाणात्सं ‚{\tiny $_{5}$}‚ निहित‚{\tiny $_{lb}$}‚समस्ततदन्यत्क्रियादिकारणात् प्रत्यक्षान्नियमेन सद्व्यवहारम्प्रतिपद्यते समासाद‚{\tiny $_{lb}$}‚यति । तस्यैव यथोक्तस्य प्रमाणस्य निवृत्तिस्तस्य स्व ‚{\tiny $_{6}$}‚ भावविशेषस्यासद्व्यवहारं ‚{\tiny $_{lb}$}‚प्रसाधयति । अवधारणमनुमानावगमादिनिवृत्तेर्व्यवच्छेदाय । किङ्कारणन्तस्य ‚{\tiny $_{lb}$}‚स्वभावविशेषस्य स्वभावसत्तायास्तस्य यथो ‚{\tiny $_{7}$}‚ क्तस्य प्रमाणस्य येयं सत्ता तया व्याप्तः ‚{\tiny $_{lb}$}‚कारणात् । तथाहि यत्र स तादृग्विधः पदार्थस्तत्रावश्यं तेनापि प्रमाणेन भवितव्यं । ‚{\tiny $_{lb}$}‚समर्थस्य कारणस्य कार्या ‚{\tiny $_{8}$}‚ व्यभिचारात् । एवञ्चैतत्प्रमाणं तद्व्यापकत्वान्निवर्त्त‚{\tiny $_{lb}$}‚मानं तामपि वृक्षवच्छिंशपां निवर्त्तयति । अनेन च यथोक्तादनुपलम्भादित्याद्युक्त‚{\tiny $_{lb}$}‚मुपसंहरति स्यादेत ‚{\tiny $_{9}$}‚ \leavevmode\ledsidenote{\textenglish{32a/msK}} दुपलब्धिलक्षणप्राप्तमपि क्षीरादिषु दध्यादिकं न प्रत्यक्षे ‚{\tiny $_{1}$}‚ ‚{\tiny $_{lb}$}‚णोपलभ्यतेऽपि त्वनुमानेनाशक्तादनुत्पत्तिरिति । अतो न तन्निवृत्याप्यसद्व्यवहार‚{\tiny $_{lb}$}‚विषयत्वन्तस्येत्यत आह । ‚{\color{DodgerBlue3}‚न चे} ‚{\tiny $_{4a9}$}‚ त्यादि । येनान्योपलब्धित्वेनानुमाना‚{\tiny $_{lb}$}‚दस्यो ‚{\tiny $_{2}$}‚ पलब्धिः स्यात् । किम्पुनरन्योपलब्धिर्न युज्यत इत्याह । ‚{\color{DodgerBlue3}‚न चेत्या} दि । यस्मादर्थे‚{\tiny $_{lb}$}‚चकारः । तस्य रूपस्योपलब्धिलक्षणप्राप्तस्यान्यथाभावप्रच्युतिर्न्न तमन्त ‚{\tiny $_{3}$}‚ रेणा‚{\tiny $_{lb}$}‚प्रत्यक्षः स भावो युक्त इति शेषः । तदेतेन प्रत्यक्षमेव तस्योपलब्धिरपेक्षणीयस्य ‚{\tiny $_{lb}$}‚कस्यचिद्धेतोरभावादिति प्रसाधयति । प्रयो ‚{\tiny $_{4}$}‚ गः पुनः । यद्यदासन्निहितसकला‚{\tiny $_{lb}$}‚प्रतिबद्धसामर्थ्यकारणन्तत्तदा भवत्येव न चाक्षेपकारि । यथा समग्राप्रतिहत‚{\tiny $_{lb}$}‚सामर्थ्यकारणसा ‚{\tiny $_{5}$}‚ मग्रीकोङ्कुरः । तथा च क्षीराद्यवस्थासु यथोपदिष्टपक्षधर्मवद् ‚{\tiny $_{lb}$}‚दध्यादिविषयं ज्ञानमिति स्वभावहेतुः । द्वितीयसाध्यापेक्षया व्यापकवि ‚{\tiny $_{6}$}‚ रुद्धोप‚{\tiny $_{lb}$}‚ \leavevmode\ledsidenote{\textenglish{45/s}} लब्धिः । अन्यथा त्वन्तस्य भवत्येवातोयं हेतुरसिद्ध इति चेदाह \add{।} ‚{\color{DodgerBlue3}‚अन्यथा भावे} ‚{\tiny $_{lb}$}‚चेष्यमाणे ‚{\color{DodgerBlue3}‚तदेवो} पलब्धिलक्षणप्राप्तं दध्यादि ‚{\color{DodgerBlue3}‚न स्यात्} ‚{\tiny $_{4b1}$}‚ प्राच्यरूपात् प्रच्यु ‚{\tiny $_{7}$}‚ तेः । ‚{\tiny $_{lb}$}‚तथा च तद्रूपतायां निरन्वयविनास\edtext{}{\lemma{निरन्वयविनास}\Bfootnote{? विनाश}}प्रसङ्ग इति भावः । ‚{\color{DodgerBlue3}‚अपि चे}‚{\tiny $_{lb}$}‚‚{\tiny $_{4b1}$}‚त्यादिनोपचयमाह । यदयम्भावः । अजातोऽनष्टश्च रूपातिशयो‚{\tiny $_{lb}$}‚स्येति विग्रहः ‚{\tiny $_{8}$}‚ \add{।} नित्यमेकत्वरूपत्वाद् द्रव्यान्तरेण व्यवधाने दूरदेशस्थितौ च भवे‚{\tiny $_{lb}$}‚युरपि प्रत्यक्षाप्रत्यक्षत्वादय इत्यत आह । ‚{\color{DodgerBlue3}‚अव्यवधानदूरस्थान} ‚{\tiny $_{4b1}$}‚ इति । ‚{\tiny $_{lb}$}‚न विद्यते व्यवधानदू ‚{\tiny $_{9}$}‚ \leavevmode\ledsidenote{\textenglish{32b/msK}} रस्थाने चास्येति विग्रहः । क्वचिदव्यवधानादूरस्थान ‚{\tiny $_{lb}$}‚इति पठ्यते । तत्र व्यवधानदूरस्थानशब्दयोः प्रत्येकं नञा समासं कृत्वा पश्चा\add{द्} ‚{\tiny $_{lb}$}‚विशेषणसमासः कार्यः । कञ्चित्पुरुषमपेक्ष्य कोपि प्रत्यक्षोऽन्यञ्चापेक्ष्य प्रत्यक्ष ‚{\tiny $_{lb}$}‚इति न विरोध इत्याह । ‚{\color{DodgerBlue3}‚तस्यैव} ‚{\tiny $_{4b2}$}‚ । तस्याप्युन्मीलितलोचनाद्यवस्थायां ‚{\tiny $_{lb}$}‚प्रत्यक्षोऽन्यदा चाप्रत्यक्ष इति न ‚{\tiny $_{2}$}‚ काचित् क्षतिरित्याह । ‚{\color{DodgerBlue3}‚तदवस्थेन्द्रियादेरेव} ‚{\tiny $_{lb}$}‚ ‚{\tiny $_{4b2}$}‚ तदवस्थमविकृतमिन्द्रियमस्येति विग्रहः । आदिग्रहणं मनस्काराद्याक्षेपाय । ‚{\tiny $_{lb}$}‚कदाचिदभिव्यक्तवेलायां ‚{\tiny $_{3}$}‚ प्रत्यक्षो भवति कदाचिच्चानभिव्यक्तक्षीरादिवेला‚{\tiny $_{lb}$}‚यामप्रत्यक्षश्चेति येन प्रत्यक्षाप्रत्यक्षत्वेन कदाचिदनुमानस्योपलब्धिरशक्तादनुत्प‚{\tiny $_{lb}$}‚त्ते ‚{\tiny $_{4}$}‚ रिति । कदाचित्तु व्यक्तावस्थायां प्रत्यक्षं । किं पुनरत्रायुक्तं । येनैवं ब्रूष ‚{\tiny $_{lb}$}‚इति चेदाह । एकस्मिन्नेवानतिशये दध्यादावमीषां प्रकाराणाम्प्र ‚{\tiny $_{5}$}‚ त्यक्षाप्रत्यक्षत्त्वा‚{\tiny $_{lb}$}‚दीनां ‚{\color{DodgerBlue3}‚विरोधादिति} ‚{\tiny $_{4b2}$}‚ । ये परस्परविरुद्धरूपा न तेषामेकत्रानतिशये सम्भवः । ‚{\tiny $_{lb}$}‚तद्यथा शीतोष्णस्पर्शादीनां । परस्परविरु ‚{\tiny $_{6}$}‚ द्धाश्च प्रत्यक्षाप्रत्यक्षत्वादयः । इति ‚{\tiny $_{lb}$}‚व्यापकविरुद्धोपलब्धिम्मन्यते ॥
	{\color{gray}{\rmlatinfont\textsuperscript{§~\theparCount}}}
	\pend% ending standard par
      ‚{\tiny $_{lb}$}‚

	  
	  \pstart \leavevmode% starting standard par
	\hphantom{.}परः प्राह ॥ ‚{\color{DodgerBlue3}‚नानतिशय} ‚{\tiny $_{4b3}$}‚ इति । एकस्याव्यक्तावस्थालक्षणस्या‚{\tiny $_{lb}$}‚तिशयस्य निवृत्याऽपर ‚{\tiny $_{7}$}‚ स्य व्यक्तावस्थालक्षणस्योत्पत्या च क्षीरं दधीति ‚{\tiny $_{lb}$}‚व्यवहारस्योपगमात् । अनेनाजातानष्टरूपातिशय इत्यादेरसिद्धत्वमाह । न ‚{\tiny $_{lb}$}‚तावदयमतिशयो भवद्भिरति ‚{\tiny $_{8}$}‚ शयवद् भावव्यतिरिक्तोऽभ्युपगतोऽभ्युपगमे ‚{\tiny $_{lb}$}‚वा तदवस्थोऽनन्तराभिहितो दोषः स्यात् ‚{\tiny $_{9}$}‚ \leavevmode\ledsidenote{\textenglish{33a/msK}} \add{।} तस्मादव्यतिरिक्त एवायं ‚{\tiny $_{lb}$}‚तत्र चायन्दोष इत्यागूर्याह । सोतिशयो व्यवस्थालक्षणस्तस्यातिशयवतोऽवस्थातु‚{\tiny $_{lb}$}‚ \leavevmode\ledsidenote{\textenglish{46/s}} रात्मभूतोऽनन्वय इत्येकान्तेन निवर्त्तमानः । व्या ‚{\tiny $_{1}$}‚ पकस्वभावात्प्रवर्त्तमानोऽसन्नेव ‚{\tiny $_{lb}$}‚कथन्न स्वभावनानात्वं । सुखदुःखयोरिवाकर्षति । अन्वाकर्षत्येवेत्यर्थः ।प्रयोगो\edtext{}{\lemma{प्रयोगो}\Bfootnote{‚{\tiny $_{lb}$}‚? प्रयोगः}}पुनर्यो यस्या ‚{\color{DodgerBlue3}‚त्मभूतः} ‚{\tiny $_{4b3}$}‚ स तन्निवृ ‚{\tiny $_{2}$}‚ त्तावेकान्तेन निवर्त्तते प्रवृ‚{\tiny $_{lb}$}‚त्तौ चासन्नेव प्रवर्त्तते यथा तस्यैवातिशयस्यात्मा । आत्मभूतश्चातिशयस्यातिशय‚{\tiny $_{lb}$}‚वानिति स्वभावहेतुः । ततश्च तयोरवस्थ ‚{\tiny $_{3}$}‚ योरवस्थातुर्न्नानात्वं परस्परविरोधि‚{\tiny $_{lb}$}‚धर्माध्यासितत्वात् सुखदुःखवदिति स्वभावहेतुरेव । नैवासावतिशयोऽनन्वयः ‚{\tiny $_{lb}$}‚प्रवर्त्तते निव ‚{\tiny $_{4}$}‚ र्त्तते वाऽतः पूर्वस्मिन्प्रमाणे साध्यविकलत्वन्दृष्टान्तस्य । उत्तरत्र त्व‚{\tiny $_{lb}$}‚सिद्धिर्हेतोरिति चेदाह । ‚{\color{DodgerBlue3}‚सान्वयत्वेचा} तिशयस्य निवृत्तिप्रवृत्योरं ‚{\tiny $_{5}$}‚ गीक्रियमाणे ‚{\tiny $_{lb}$}‚का कस्य ‚{\color{DodgerBlue3}‚निवृत्तिः प्रवृत्तिर्वेति} ‚{\tiny $_{4b3}$}‚ । नैव काचित्कस्यचिन्निवृत्तिः प्रवृतिर्वा । ‚{\tiny $_{lb}$}‚सर्व्वस्य सर्व्वदा सत्त्वात् । तथा च सर्व्वं सर्व्वत्र समुपयु ‚{\tiny $_{6}$}‚जे\edtext{}{\lemma{जे}\Bfootnote{? ज्ये}}तेत्यादिना ‚{\tiny $_{lb}$}‚पुरोनुक्रान्तो दोषोनुपयुज्यत इत्यभिप्रायः । उपचयमाह । यदि च कस्यचित् ‚{\tiny $_{lb}$}‚स्वभावस्यातिशयाख्यस्य प्रवृत्तिर्निवृत्तिर्वेति स्वयमभ्यनुज्ञायते ‚{\tiny $_{7}$}‚ त्वया । एकाति‚{\tiny $_{lb}$}‚शयनिवृत्याऽपरातिशयोत्पत्या व्यवहारभेदोपगमादित्यविधानात् । तदेतदेव ‚{\tiny $_{lb}$}‚परस्तथागतवचोऽभ्यासोपजातावदातमति ‚{\tiny $_{8}$}‚ र्ब्रुवाणः । नानुमन्यते भद्रमुखेण\edtext{}{\lemma{भद्रमुखेण}\Bfootnote{‚{\tiny $_{lb}$}‚? मुखेन}}भवेदेवं यदि यथा मया प्रवृत्तिनिवृत्ती अभ्यनुज्ञायेते तथा तेनापि । ‚{\tiny $_{lb}$}‚यावतास्य निरन्वयोपजननविनाशोपगमो मम ‚{\tiny $_{9}$}‚ \leavevmode\ledsidenote{\textenglish{33b/msK}} त्वाविर्भावतिरोभावमात्रन्तत्कथमिवा‚{\tiny $_{lb}$}‚नुमन्यत इति कदाचिद् ब्रूयात्पर इति तन्मतमाशङ्कते । ‚{\color{DodgerBlue3}‚तस्ये} ‚{\tiny $_{4b4}$}‚ त्यादिना । ‚{\tiny $_{1}$}‚ ‚{\tiny $_{lb}$}‚सदैव भवद्भिः शून्यहृदयैरयमन्वयो घोष्यते । तत्र वक्तव्यङ्कोयमन्वयो नाम ‚{\tiny $_{lb}$}‚भावस्य जन्मविनाशयोरिति सत्त्यं एषा । परः प्राह \add{।} ‚{\color{DodgerBlue3}‚किमत्राभिधानीयं} याव‚{\tiny $_{lb}$}‚ता शक्तिरन्वयो भावस्य जन्मविनाशयोरिति व ‚{\tiny $_{2}$}‚ र्त्तते । कथम्पुनः सान्वय इत्याह । ‚{\tiny $_{lb}$}‚यतोस्त्येव ‚{\color{DodgerBlue3}‚प्रागपि जन्मनो निरोधादप्यूर्ध्वं} ‚{\tiny $_{4b5}$}‚ सा शक्तिरवस्थातृलक्षणा ‚{\tiny $_{lb}$}‚येनैतदेवन्तेनायम्भावो नापूर्वः सन् सर्वथा जायते ‚{\tiny $_{3}$}‚ अपि तु शक्तिरूपेण पूर्व्वं व्यवस्थित ‚{\tiny $_{lb}$}‚एव केवलमाविर्भवतीति सर्व्वथाग्रहणेन ज्ञापयति । तथा न पूर्व्वो विनश्यत्येकान्ते‚{\tiny $_{lb}$}‚नापि तु तिरोभवति । ‚{\tiny $_{4}$}‚ असतो नास्त्युदयः सतश्च नास्ति विनाश इति यावत् । ‚{\tiny $_{lb}$}‚ \leavevmode\ledsidenote{\textenglish{47/s}} आचार्य आह । ‚{\color{DodgerBlue3}‚यदि सा शक्तिः सर्व्वदा} ‚{\tiny $_{4b6}$}‚ तिरोभावाविर्भावकालेऽ ‚{\color{DodgerBlue3}‚नतिशयाति‚{\tiny $_{lb}$}‚शय ‚{\tiny $_{5}$}‚ रहिता एकरूपे} ति यावत् । तदा ‚{\color{DodgerBlue3}‚किमिदानीं} अतिशयवद्विद्यते । यत\add{ः} ‚{\tiny $_{lb}$}‚कुतोयं व्यवहारविभागः क्षीरन्दधितक्रमित्यादि । ‚{\color{DodgerBlue3}‚साङ्ख्य} आह । ‚{\tiny $_{6}$}‚ अवस्था ‚{\color{DodgerBlue3}‚अति‚{\tiny $_{lb}$}‚शयवत्य} ‚{\tiny $_{4b6}$}‚ इति । ‚{\color{DodgerBlue3}‚ता} ‚{\tiny $_{4b6}$}‚ इत्याद्या ‚{\color{DodgerBlue3}‚चार्यः} । विकल्पद्वयञ्च प्रकारान्तरा‚{\tiny $_{lb}$}‚सम्भवात्कृतं । न वाहरीकवादो युज्यते तत्स्वान्यत्वयोः परस्परपरिहारस्थि ‚{\tiny $_{7}$}‚ ति‚{\tiny $_{lb}$}‚लक्षणतया तृतीयराशिव्यतिरेचकत्वात् । एकत्वे को दोष इति चेदाह \add{।} ‚{\color{DodgerBlue3}‚एकश्चे‚{\tiny $_{lb}$}‚त्तदा कथमिदमेकत्राविभक्ता} त्मन्यविभक्तस्वरूपे ‚{\color{DodgerBlue3}‚योक्ष्यते} ‚{\tiny $_{4b7}$}‚ । व्यपेक्षया ‚{\tiny $_{lb}$}‚भ ‚{\tiny $_{8}$}‚ वेदपीत्याह । ‚{\color{DodgerBlue3}‚निष्पर्यायं} किम्पुनस्तत्परस्परव्याहत इत्याह ‚{\color{DodgerBlue3}‚जन्म अवस्थानाम‚{\tiny $_{lb}$}‚जन्मशक्तेः} । तथार्थक्रियायामुपयोगोऽवस्थानां शक्तस्त्वनुपयोग ‚{\tiny $_{4b7}$}‚ इ ‚{\tiny $_{9}$}‚ \leavevmode\ledsidenote{\textenglish{34a/msK}} ति । ‚{\tiny $_{lb}$}‚प्रयोगाः पुनः \add{।} शक्तेरपि जन्मास्ति । अवस्थाभ्योऽव्यतिरेकात् । अवस्था‚{\tiny $_{lb}$}‚स्वरूपवत् । अवस्थानाम्वा न जन्म शक्तेरव्यतिरेकात् । शक्तिस्वरूपवत् । ‚{\tiny $_{lb}$}‚स्वभावहेतुविरुद्धव्याप्तो ‚{\tiny $_{1}$}‚ पलब्धि\add{ः} । एतेनैव प्रकारेणार्थक्रियोपयोगानुपयोगनिवृत्य‚{\tiny $_{lb}$}‚निवृत्यादिषु स्वभावहेतुविरुद्धव्याप्तोपलब्धयो योज्याः । आदिग्रहणेन पतनापतन‚{\tiny $_{lb}$}‚पोरपि परिग्रहः ।
	{\color{gray}{\rmlatinfont\textsuperscript{§~\theparCount}}}
	\pend% ending standard par
      ‚{\tiny $_{lb}$}‚

	  
	  \pstart \leavevmode% starting standard par
	पु ‚{\tiny $_{2}$}‚ नरपि ‚{\color{DodgerBlue3}‚साङ्खीय} म्मतमाशङ्कते । ‚{\color{DodgerBlue3}‚अस्ती} त्या ‚{\tiny $_{4b8}$}‚ दिना । केनचित्पर्या‚{\tiny $_{lb}$}‚येण अवस्थाशक्त्योरनन्यत्वम्परमार्थतस्तु भेद एव तेन जन्मादीनामविरोध इति । ‚{\tiny $_{lb}$}‚नूनम्भवतः स्वपक्षरक्षणा ‚{\tiny $_{3}$}‚ कुलबुद्धेरात्मापि विस्मृतः । इत्याकूतवानाह । ‚{\color{DodgerBlue3}‚विस्मरण‚{\tiny $_{lb}$}‚शील} ‚{\tiny $_{4b8}$}‚ इत्यादि । यतोऽनन्यत्वपक्षेऽयन्दोषोस्माभिरुक्तोऽन्यत्वपक्षेत्वन्य ‚{\tiny $_{lb}$}‚एव भविष्यति । कः पुनरसावन्य ‚{\tiny $_{4}$}‚ इति तमेवदर्शयितुमुपक्रमते । अथाप्यनयोः ‚{\tiny $_{lb}$}‚शक्त्यवस्थयोर्विभागोऽन्यत्वन्तदा न कश्चिद्विरोधः । केवलं सान्वयो भावस्य जन्म ‚{\tiny $_{lb}$}‚विनासा\edtext{}{\lemma{विनासा}\Bfootnote{? विनाशा}}दिति न स्यात् । किं ‚{\tiny $_{5}$}‚ कारणं । यस्मात् यस्यान्वयः शक्ति‚{\tiny $_{lb}$}‚त्वेनाभिमतस्य न तस्य जन्मविनाशौ नित्यमेकस्मिन्नेव स्वभावे व्यवस्थानात् । यस्य ‚{\tiny $_{lb}$}‚ \leavevmode\ledsidenote{\textenglish{48/s}} वा ता उत्त्पादविनाशाववस्थात्वेनाभी ‚{\tiny $_{6}$}‚ ष्टस्य न तस्यान्वयः । अपरापरावस्थो‚{\tiny $_{lb}$}‚दयास्तमयेनावस्थितरूपाभावात् । तयोः शक्तिव्यक्तयोरभेदाददोष इति ‚{\color{DodgerBlue3}‚कापिलः । ‚{\tiny $_{lb}$}‚अनुत्तर} ‚{\tiny $_{4b10}$}‚ ‚{\color{DodgerBlue3}‚मित्याद्याचार्यः} । किमत्रायुज्यमा ‚{\tiny $_{7}$}‚ नकं येनैवं वदसीत्याह । ‚{\color{DodgerBlue3}‚अभे‚{\tiny $_{lb}$}‚दो हि नामैक्यमुच्यते} ‚{\tiny $_{4b10}$}‚ । तौ शक्तिव्यक्तिभेदा ‚{\color{DodgerBlue3}‚वित्ययञ्च} भेदाधिष्ठानो‚{\tiny $_{lb}$}‚ ‚{\color{DodgerBlue3}‚न्यत्वनिबन्धनो व्यवहारो भाविक} इति कल्पनाविरचितस्याप्रतिक्षे ‚{\tiny $_{8}$}‚ पात् । किञ्च ‚{\tiny $_{lb}$}‚निवृत्तिप्रादुर्भावयोः सतोरनिवृत्तिप्रादुर्भावौ तथा स्थितौ सत्यामस्थितिः । आदि‚{\tiny $_{lb}$}‚ग्रहणाद् गतावगतिरित्यादि योज्यं । एतद् भेदलक्षणङ्कथं योज्यते भ ‚{\tiny $_{9}$}‚ \leavevmode\ledsidenote{\textenglish{34b/msK}} वता । तथा‚{\tiny $_{lb}$}‚ह्यवस्थानिवृत्तिप्रादुर्भावाभ्यामनिवृत्तिप्रादुर्भाववत्याः शक्तेरभेदो नेष्यते त्वया । ‚{\tiny $_{lb}$}‚तथा शक्तेरवस्थानेपि नावस्थानामवस्थानं । न च शक्तेस्तासामन्यत्वमिष्टं ‚{\tiny $_{1}$}‚ । ‚{\tiny $_{lb}$}‚तस्मादेवं रूपं नानात्वमित्याह । एष हि ‚{\color{DodgerBlue3}‚निवृत्तिप्रादुर्भावयोरनिवृत्तिप्रादुर्भाव} ‚{\tiny $_{5a1}$}‚ ‚{\tiny $_{lb}$}‚इत्यादिभेदः । तथा हि यन्निवृत्यादिना न यस्य निवृत्यादयस्तत्तस्माद् भिन्नं यथा ‚{\tiny $_{lb}$}‚ता ‚{\tiny $_{2}$}‚ लतरुस्तमालादित्यतिप्रतीतमेतत् । एतद्विरहश्चाभेद इति यन्निवृत्या यस्य ‚{\tiny $_{lb}$}‚निवृत्तिरित्यादि । ननु च भूतभौतिकचित्तचैत्तादीनाम्प्रतिनियतसहोत्पादनिरो‚{\tiny $_{lb}$}‚ध ‚{\tiny $_{3}$}‚ स्थितीनामेतद्विद्यते । न च तेषामभेदस्तत् कथमुक्तमेतद्विरहश्चाभेद इति चेत् । ‚{\tiny $_{lb}$}‚न तेषाम्भिन्नोत्पादादिमत्वात् । यथाक्रममुदाहरणद्वयमाह । ‚{\color{DodgerBlue3}‚यथे} ‚{\tiny $_{4}$}‚ त्यादि । अन्य‚{\tiny $_{lb}$}‚थे ‚{\tiny $_{5a1}$}‚ ति । यद्यनन्तरोक्तम्भेदाभेदलक्षणन्नाश्रीयते तदा भेदयोर्लक्षणा‚{\tiny $_{lb}$}‚भावात्कारणाद् भेदाभेदयोरव्यवस्था स्यात् । सर्व्वत्रेति सुखादीनाम्परस्परं ‚{\tiny $_{5}$}‚ चैतन्या‚{\tiny $_{lb}$}‚नाञ्च । सुखादिभ्यश्चैतन्यानां अभेदः । सुखादीनाम्प्रत्येकम्भेदो न भवेदिति ‚{\tiny $_{lb}$}‚यावत् । तदात्मनीत्यादिना परः स्वसमयप्रतीतम्भेदाभेदयोर्लक्षण ‚{\tiny $_{6}$}‚ माह । ‚{\color{DodgerBlue3}‚तेना‚{\tiny $_{lb}$}‚\leavevmode\ledsidenote{\textenglish{49/s}} विरोध} ‚{\tiny $_{5a3}$}‚ इति जन्माजन्मादीनां । अनन्तरोक्तस्य वा । न वै मृदात्मनीत्या‚{\tiny $_{lb}$}‚दिना मृप्तिण्डघटयोराधाराधेयभावं प्रतिक्षिपति’ । किन्तर्हि मृदात्मैव कश्चित् ‚{\tiny $_{7}$}‚ ‚{\tiny $_{lb}$}‚विशिष्टग्रीवादिसन्निवेशावच्छिन्नो घट इत्यभीधीयते \add{।} नन्वेकमेव मृद्द्रव्यं सर्व्वत्र ‚{\tiny $_{lb}$}‚तत्कथमिदमभिहितमित्यत आह । ‚{\color{DodgerBlue3}‚नहि एकस्त्रैलोक्यमृदात्मे} ‚{\tiny $_{5a3}$}‚ ति । कुतः ‚{\tiny $_{lb}$}‚प्रतिविज्ञ ‚{\tiny $_{8}$}‚ प्तिप्रतिभासभेदन्द्रव्यस्वभावभेदादिति सम्बन्धः । अन्योन्यभिन्नानामेव ‚{\tiny $_{lb}$}‚द्रव्याणाम्विज्ञाने प्रतिभासनादित्यर्थः । तथा प्रत्यवस्थाभेदभिन्नावस्थत्वात् । ‚{\tiny $_{lb}$}‚प्रत्यर्थ ‚{\tiny $_{9}$}‚ \leavevmode\ledsidenote{\textenglish{35a/msK}} क्रियाभेदं चाश्रित्य द्रव्यस्वभावभेदात् । परस्परासम्भविकार्यकारणा- ‚{\tiny $_{lb}$}‚दिति यावत् । परस्यापि सत्वरजस्तमश्चैतन्येषु भेदाभ्युपगम इदमेव कारणं ‚{\tiny $_{lb}$}‚युक्तमिति कथय ‚{\tiny $_{1}$}‚ \add{न्} नाह । ‚{\color{DodgerBlue3}‚एवं हीति} ‚{\tiny $_{5a4}$}‚ । यदि प्रतिविज्ञप्तिप्रतिभास‚{\tiny $_{lb}$}‚भेदादिना भेद इष्यते । चैतन्येषु चेति बहुवचनं बहवः पुमांस इति सिद्धान्तात् । यद्येव‚{\tiny $_{lb}$}‚मिति प्रतिविज्ञप्ति प्र ‚{\tiny $_{2}$}‚ तिभासभेदादिना । पुनरप्याह । ‚{\color{DodgerBlue3}‚सत्यप्येतस्मिन्} प्रतिविज्ञप्ति‚{\tiny $_{lb}$}‚प्रतिभासभेदादौ कस्यचि ‚{\color{DodgerBlue3}‚दात्मन} ‚{\tiny $_{5a4}$}‚ इति शक्तेरनुगमादैक्यमवस्थानामिति । ‚{\tiny $_{lb}$}‚ ‚{\color{DodgerBlue3}‚आचार्य} आह । ‚{\color{DodgerBlue3}‚यद्ये ‚{\tiny $_{3}$}‚ वं सुखादिष्वप्ययमेवाभेदप्रसङ्} गश्चैतन्येषु च । सुखादिष्वपि ‚{\tiny $_{lb}$}‚हि गुणत्वाद् भोक्तृत्वकर्त्तृत्वादीनामनुगमाच्चैतन्येषु च भोक्तृत्वाकर्त्तृत्वागुण‚{\tiny $_{lb}$}‚त्वादी ‚{\tiny $_{4}$}‚ नान्तथा सुखादि चैतन्येषु सत्वज्ञेयत्वादीनामन्वयादित्यभिप्रायः । प्रयोगो\edtext{}{\lemma{प्रयोगो}\Bfootnote{? गः}} ‚{\tiny $_{lb}$}‚ पुनरभिन्नाः पुरुषसुखादयः परस्परमन्वयान्वयभाक्त्वात् । श ‚{\tiny $_{5}$}‚ क्तिव्यक्तिवत् । शक्ति‚{\tiny $_{lb}$}‚व्यक्ती वा भिन्नेऽन्वयोनन्वयभाक्त्वादेव । सुखादिचैतन्यवदिति स्वभावहेतु\add{ः} । ‚{\tiny $_{lb}$}‚अथापि स्याद्यत्र सर्व्वात्मनैवान्वयस्तत्राभेदो न ‚{\tiny $_{6}$}‚ तु यत्र केनचिद्रूपेण । घटादिषु च ‚{\tiny $_{lb}$}‚सर्व्वात्मनान्वयस्ततोयमदोष इत्यत आह । ‚{\color{DodgerBlue3}‚न च घटादिष्वपि सर्व्वात्मनान्वयो ‚{\tiny $_{lb}$}‚ ‚{\tiny $_{5a5}$}‚ पि तु केनचिद्रूपेणेति} न केवलं सुखादि ‚{\tiny $_{7}$}‚ ष्वित्यपि शब्दः । कुतोऽवैश्वरूप्य‚{\tiny $_{lb}$}‚सहोत्पादादिप्रसङ्गात् । तथाहि सर्व्वासामवस्थानां सर्व्वप्रकारेणान्वये सत्यैक्य‚{\tiny $_{lb}$}‚म्प्राप्नोति । ततश्च विशिष्टरूपरसगन्धश ‚{\tiny $_{8}$}‚ वीर्यविपाकाभावात् । ‚{\tiny $_{lb}$}‚वैचित्र्यन्न भवेत् । एवञ्च पञ्चभूताभावप्रसङ्गोऽध्यक्षादिवाधाप्रसं ‚{\tiny $_{9}$}‚ \leavevmode\ledsidenote{\textenglish{35b/msK}} गश्चेति ‚{\tiny $_{lb}$}‚भावः । सहोत्पत्तिश्च सर्व्वासामवस्थानाम्प्रसज्यते । आदिशब्देन ह्यनिरो‚{\tiny $_{lb}$}‚धार्थक्रियाव्यापारविकारादय ‚{\tiny $_{1}$}‚ उपादीयन्ते । प्रयोगाः पुनर्यद्विशिष्टरूपरसगन्ध‚{\tiny $_{lb}$}‚ \leavevmode\ledsidenote{\textenglish{50/s}} शब्दादिभिरनेकप्रकारं न भवति । न तस्य वैश्वरूप्यमस्ति । यथैकस्य सुखाद्या‚{\tiny $_{lb}$}‚त्मनः । तथा सति \add{?} मताना ‚{\tiny $_{2}$}‚ मप्यवस्थानामनन्तरोक्तो धर्मो नास्ति न चासिद्धो ‚{\tiny $_{lb}$}‚हेतुर्यतो यद्यस्मान्न व्यतिरिच्यते न तद्विशिष्टरूपादिभिरनेकप्रकारं यथा तस्यै‚{\tiny $_{lb}$}‚वात्मा । न व्यतिरिच्यन्ते चा ‚{\tiny $_{3}$}‚ वस्था अभीष्टा इति व्यापकविरुद्धोपलब्धी । तथा ‚{\tiny $_{lb}$}‚यद्यस्मादपृथग्भूतं तत्तदुत्पादादिभिरुत्पादादिमत् । यथा तस्यैव स्वरूपं । अपृथग्भू‚{\tiny $_{lb}$}‚ताश्चा ‚{\tiny $_{4}$}‚ भिमता अवस्थास्ताभ्योऽन्यस्या इति स्वभावहेतुः । अन्यथा घटोयमित्यन‚{\tiny $_{lb}$}‚न्यत्वमेवायुक्तं । नामान्तरम्वा अर्थभेदमभ्युपगम्य तथाभिधाना ‚{\tiny $_{5}$}‚ त् । उपचयमाह । ‚{\tiny $_{lb}$}‚ ‚{\color{DodgerBlue3}‚न च घटं मृदात्मानञ्च कश्चिद ‚{\tiny $_{5a5}$}‚ त्यर्थ} मुन्मीलितलोचनोप्ययं घटोयं च ‚{\tiny $_{lb}$}‚मृदात्मेति विवेकेनोपलक्षयति । येनैवं स्यादिदमि ‚{\tiny $_{6}$}‚ ह प्रादुर्भूतमिति तदनेनाभेदलक्षण‚{\tiny $_{lb}$}‚मत्यन्तासम्बद्धमेवेत्याह । ननु च पिण्डरूपात्मृदात्मनो घटस्य विवेकेनोपलक्षणमस्त्येव ‚{\tiny $_{lb}$}‚तक्तिमेवमुक्तमिति चे ‚{\tiny $_{7}$}‚ त् । सत्यमस्ति । न तु घटाद् भिन्नन्तं परोभिमन्यत इत्यभि‚{\tiny $_{lb}$}‚प्रायाददोषः । यदि नाम भेदेनानुपलक्षणन्तयोस्तथापि कस्मादेवं न स्यादिति चेदाह ॥ ‚{\tiny $_{lb}$}‚ ‚{\color{DodgerBlue3}‚नह्यधि ‚{\tiny $_{8}$}‚ ष्ठानाधिष्ठानिनोराधाराधेययोः कुण्डेवदरयोर्विवेकेनानुनपलक्षणे सत्ये‚{\tiny $_{lb}$}‚वम्भवतीदमिह प्रादूर्भूतमिति} ‚{\tiny $_{4a6}$}‚ । तदनेन घटमृदात्मनोराधाराधेयभावो ‚{\tiny $_{9}$}‚ ‚{\tiny $_{lb}$}‚ \leavevmode\ledsidenote{\textenglish{36a/msK}} नास्ति विवेकेनानुपलक्षणात्सत्वादितत्स्वभावयोरेवेति व्यापकानुपलब्धिं मन्यते । ‚{\tiny $_{lb}$}‚अधुना यद्यस्मिन्प्रादुर्भवति तत्ततोऽभिन्नमित्यस्याभेदलक्षणस्याव्या ‚{\tiny $_{1}$}‚ पितासा\edtext{}{\lemma{पितासा}\Bfootnote{? शा}} ‚{\tiny $_{lb}$}‚ चिख्यासुराह । न च शक्त्यात्मनि प्रादुर्भावस्तस्या नित्यमवस्थानाभ्युपगमात् ॥ ‚{\tiny $_{lb}$}‚अन्यथावस्थैव सा स्यात् । तथा च तस्याः स्वात्मनःसकासा\edtext{}{\lemma{स्वात्मनःसकासा}\Bfootnote{? सकाशा}}दभ्यु ‚{\tiny $_{2}$}‚ पेतो‚{\tiny $_{lb}$}‚ऽभेदो न स्यात् । अभेदलक्षणाभावात् उपलक्षणञ्चैतद्व्यक्तौ सुखादिषु पुरुषेषु च ‚{\tiny $_{lb}$}‚तुल्यदोषत्वात् । अन्ये तु स्वदर्शनापराधमलीमसधिय\add{ः} ‚{\tiny $_{3}$}‚ केचित् ‚{\color{DodgerBlue3}‚सांख्या} एवमाहुः । ‚{\tiny $_{lb}$}‚यो यस्य परिणामस्स तस्मादभिन्नः । तद्यथा हेम्नः कुण्डलाद्यवस्थाविशेष इति ‚{\tiny $_{lb}$}‚तेप्यनेनैव पूर्व्वस्याभेदलक्षण ‚{\tiny $_{4}$}‚ स्याव्यापिताप्रदर्शनेनापहस्तिता इति चेतस्या‚{\tiny $_{lb}$}‚रोप्याह । ‚{\color{DodgerBlue3}‚एतेनैवे} ‚{\tiny $_{5a6}$}‚ त्यादि । युष्मद्दर्शनपरिणामोपि न युक्त इत्यभिप्रायवा‚{\tiny $_{lb}$}‚नपक्षेप ‚{\tiny $_{5}$}‚ ङ्करोति । ‚{\color{DodgerBlue3}‚किञ्चेद} ‚{\tiny $_{5a7}$}‚ मित्यादिना । परेणापि किमत्र वक्तव्यं यावता ‚{\tiny $_{lb}$}‚ \leavevmode\ledsidenote{\textenglish{51/s}} भगवता ‚{\color{DodgerBlue3}‚कपिलेन} स्पष्टमिदमुक्तमित्यभिसन्धायाह । अवस्थितस्य ‚{\tiny $_{6}$}‚ द्रव्यस्य यथा ‚{\tiny $_{lb}$}‚काञ्चनस्य धर्मान्तरस्य केयूरस्य निवृत्तिः । धर्मान्तरस्य च कुण्डलादेः प्रादुर्भावः ‚{\tiny $_{lb}$}‚परिणाम इति । ‚{\color{DodgerBlue3}‚आचार्यस्तस्यैव} तावदिदमीदृशं ‚{\tiny $_{7}$}‚ प्रज्ञास्खलितङ्कथं वृत्तमिति सवि‚{\tiny $_{lb}$}‚स्मयानुकंपन्नश्चेतः । तदपरेप्यनुवदन्तीति निर्दयाक्रान्तभुवनं दिग्व्यापकन्तमः । कः ‚{\tiny $_{lb}$}‚प्राणिनो हितेच्छा विपुलत्व ‚{\tiny $_{8}$}‚ स्यापराध इति मन्यमान\add{ः} प्राह ।
	{\color{gray}{\rmlatinfont\textsuperscript{§~\theparCount}}}
	\pend% ending standard par
      ‚{\tiny $_{lb}$}‚

	  
	  \pstart \leavevmode% starting standard par
	\hphantom{.}‚{\color{DodgerBlue3}‚ननु यदि नाम तेनैवमुक्तं} । भवद्भिस्तु निभालनीयमेतत् यत्तद्धर्मान्तरं ‚{\tiny $_{lb}$}‚कुण्डादिकं निवर्त्तते प्रादुर्भवति च किं त ‚{\tiny $_{9}$}‚ \leavevmode\ledsidenote{\textenglish{36b/msK}} देवावस्थितं काञ्चनद्रव्यं स्यात्ततोर्था- ‚{\tiny $_{lb}$}‚न्तरम्वेति । कस्माद्विकल्पद्वयमेव कृतमितिचेदाह \add{।} ‚{\color{DodgerBlue3}‚अन्यविकल्पाभावात्} ‚{\tiny $_{lb}$}‚ ‚{\tiny $_{5a8}$}‚ । ‚{\color{DodgerBlue3}‚निर्ग्रन्थवाद} स्यायोगादित्यभिसन्धिः । यद्याद्यो ‚{\tiny $_{1}$}‚ विकल्पस्तदा को‚{\tiny $_{lb}$}‚दोष इति चेदाह । यदि तद्धर्मान्तरन्तदेवावस्थितं द्रव्यं ‚{\tiny $_{5a8}$}‚ । तदा तस्याव ‚{\tiny $_{lb}$}‚स्थानान्न निवृत्तिप्रादुर्भावावाविर्भावतिरोभावलक्षणाविति त ‚{\tiny $_{2}$}‚ स्मात्कस्य ‚{\tiny $_{lb}$}‚ताविति वक्तव्यम्भवद्भिः । ‚{\color{DodgerBlue3}‚प्रयोगो\edtext{}{\lemma{प्रयोगो}\Bfootnote{? प्रयोगः}}} पुनः यस्यावस्थानं न तस्य ‚{\tiny $_{lb}$}‚निवृत्तिप्रादुर्भावौ । यथावस्थातुर्द्रव्यस्य । तथा चावस्थानन्तस्य धर्मा ‚{\tiny $_{3}$}‚ न्तर‚{\tiny $_{lb}$}‚स्येति व्यापकविरुद्धोपलब्धिः । न चासिद्धो हेतुर्यतो यदवस्थातुरनन्यत्तस्या‚{\tiny $_{lb}$}‚वस्थानं यथा तत्स्वरूपस्यैव । अनन्यच्चैतद्धर्मान्तरं तस्मा ‚{\tiny $_{4}$}‚ दिति स्वभावहेतुः ‚{\tiny $_{lb}$}‚किञ्च यद्यवस्थितमेव द्रव्यं तद्धर्मान्तरं तदावस्थितस्य द्रव्यस्य धर्मान्तरमिति । ‚{\tiny $_{lb}$}‚वचनं सिध्यति ॥ किङ्कारणमित्याह । ‚{\color{DodgerBlue3}‚न ‚{\tiny $_{5}$}‚ हि तदेव तस्य धर्मान्तरम्भवती} ‚{\tiny $_{5a9}$}‚ ‚{\tiny $_{lb}$}‚ति भवत्येव तदेव तस्य धर्मान्तरं यथा कृतकत्वं शब्दस्याव्यतिरिक्तमपि तस्मादिति ‚{\tiny $_{lb}$}‚चेदाह । ‚{\color{DodgerBlue3}‚अनपाश्रितव्य ‚{\tiny $_{6}$}‚ पेक्षाभेदं} ‚{\tiny $_{5a9}$}‚ । एतदुक्तम्भवति । अत्र हि स एव ‚{\tiny $_{lb}$}‚सब्दो\edtext{}{\lemma{सब्दो}\Bfootnote{? शब्दो}}ऽकृतकादिभ्योव्यावृत्तत्त्वात् तद्व्यावृत्त्यपेक्षया तन्मात्रजिज्ञासायां ‚{\tiny $_{lb}$}‚प्रतिक्षिप्तभेदान्तरेण शब्देन धर्मत्वे ‚{\tiny $_{7}$}‚ न व्यपदिस्यते\edtext{}{\lemma{व्यपदिस्यते}\Bfootnote{? व्यपदिश्यते}}इह तु पुनर्व्य‚{\tiny $_{lb}$}‚पेक्षाभेदोपि नास्ति तत्कथन्तदेव तस्य धर्मान्तरम्भविष्यतीति ।
	{\color{gray}{\rmlatinfont\textsuperscript{§~\theparCount}}}
	\pend% ending standard par
      ‚{\tiny $_{lb}$}‚

	  
	  \pstart \leavevmode% starting standard par
	\hphantom{.}धर्मस्य द्रव्यादर्थान्तरपक्षे तर्हि को दोष इत्याह ॥ ‚{\color{DodgerBlue3}‚अथेत्या} ‚{\tiny $_{5a9}$}‚ ‚{\tiny $_{8}$}‚ दि ॥ कस्मा‚{\tiny $_{lb}$}‚द्धर्मनिवृत्तिप्रादुर्भावाभ्यां न द्रव्यस्य परिणामो यस्मा\add{न्} न ह्यर्थान्तरगताभ्यां निवृ‚{\tiny $_{lb}$}‚ \leavevmode\ledsidenote{\textenglish{52/s}} त्तिप्रादुर्भावाभ्यामर्थान्तरस्य परिणतिर्भवति । तदेव कुतश्चैत ‚{\tiny $_{9}$}‚ \leavevmode\ledsidenote{\textenglish{37a/msK}} न्येपि परिणतेः प्रसङ्‚{\tiny $_{lb}$}‚गात् । न च चैतन्यस्य परिणतिरिष्यते । प्रधानपुरुषयोरैक्यापत्तेरकर्तृता चेति वच‚{\tiny $_{lb}$}‚नात् । प्रयोगः पुनः । यद्यतोर्थान्तरन्न तद्गताभ्यां निवृ ‚{\tiny $_{1}$}‚ त्तिप्रादुर्भावाभ्यां तस्य परि‚{\tiny $_{lb}$}‚णतिः । तद्यथा चैतन्यभिन्नस्वभावस्याङ्कुरस्य निवृत्तिप्रादुर्भावाभ्यां चैतन्यस्य धर्मान्त‚{\tiny $_{lb}$}‚रञ्च द्रव्यादिति व्यापकविरुद्धोपलब्धिः । भवे ‚{\tiny $_{2}$}‚ देतन्न यस्य कस्यचिदर्थान्तरस्या‚{\tiny $_{lb}$}‚सम्बद्धस्यापि निवृत्तिप्रादुर्भावाभ्यामन्यस्य परिणतिरपि कुत\add{ः} चासम्बद्धस्यैव । ‚{\tiny $_{lb}$}‚यथा तस्यैवाङ्कुरस्य बीजसम्बद्धस्य निवृत्तिप्रादुर्भावाभ्याम्बी ‚{\tiny $_{3}$}‚ जस्य तेन सामान्येन ‚{\tiny $_{lb}$}‚साधने सिद्धसाधनं । धर्मस्य द्रव्यसम्बन्धात् \add{त}द्विशेषेण तु साधनविकलता ‚{\tiny $_{lb}$}‚निदर्शनस्य । अङ्कुरस्य चैतन्येन सह सम्बन्धाभावादिति ‚{\tiny $_{4}$}‚ चेदाह । ‚{\color{DodgerBlue3}‚द्रव्यस्य धर्म ‚{\tiny $_{lb}$}‚इति ‚{\tiny $_{5a10}$}‚ व्यपदेशो न सिध्यति} ‚{\tiny $_{5a10}$}‚ । कुतः सम्बन्धाभावात् । एवं ‚{\tiny $_{lb}$}‚मन्यते द्रव्यसम्बन्धोयं धर्म इत्येतदेव न विद्यते । तत् कुतो व्यावृत्ति ‚{\tiny $_{5}$}‚ प्रसङ्गस्येति । ‚{\tiny $_{lb}$}‚अस्त्येव तर्हि द्रव्यधर्मयोराधाराधेयभावलक्षणस्सम्बन्धस्ततश्च सविशेषणेपि हेतौ ‚{\tiny $_{lb}$}‚असिद्धिरित्यत आह । ‚{\color{DodgerBlue3}‚नहि कार्यकारणभा ‚{\tiny $_{6}$}‚ वादन्यो वस्तुभूतः सम्बन्धोस्ती} ‚{\tiny $_{5b1}$}‚ ‚{\tiny $_{lb}$}‚ति । आधाराधेयभावोऽपि कार्यकारणभावविशेषादेव व्यवस्थाप्यते । यथा निर्णीतमा‚{\tiny $_{lb}$}‚धारतोभिनिवृत्तेरात्मनस्तादृ ‚{\tiny $_{7}$}‚ शो नु\add{?}...यः कार्यन्तस्येत्यत्र प्रकरणे ‚{\color{DodgerBlue3}‚प्रमाणविनिश्चय} ‚{\tiny $_{lb}$}‚इत्यभिप्रायः । अस्तु तर्हि कार्यकारणभावस्तयोरिति चेदाह । ‚{\color{DodgerBlue3}‚न चानयोर्द्रव्यधर्मयोः ‚{\tiny $_{lb}$}‚कार्यकारणभाव ‚{\tiny $_{8}$}‚} ‚{\tiny $_{5b1}$}‚ इति । कुतः स्वयमतदात्मनोऽतत्कारणत्वात् । यद्धि यत्स्व‚{\tiny $_{lb}$}‚भावं न भवति न तत्तत्कारणतया भवद्भिरभ्युपेयं यथा रजस्तमसः । तथा चेदमपि ‚{\tiny $_{lb}$}‚द्रव्यधर्मस्स्वभावो ‚{\tiny $_{9}$}‚ \leavevmode\ledsidenote{\textenglish{37b/msK}} भवति तत्कथमिव तस्य कारणत्वमुपेयादिति व्यापकानुपलब्धि‚{\tiny $_{lb}$}‚प्रसङ्गं मन्यते । न चायमसिद्धो हेतुरिति मन्तव्यं । अर्थाभावपक्षं समाश्रित्य दोषा‚{\tiny $_{lb}$}‚भिधानस्य प्रकृत ‚{\tiny $_{1}$}‚ त्त्वात् । यदाह । ‚{\color{DodgerBlue3}‚धर्मस्य द्रव्यादर्थान्तरत्वं ‚{\tiny $_{5b1}$}‚ स्यादिति} । अथा‚{\tiny $_{lb}$}‚प्यस्मद्वै\add{फ}ल्ये स्यात् पूर्व्वकान् कापिलानतिपत्य साङ्ख्यानां शकमाधववत् । ‚{\tiny $_{lb}$}‚द्रव्यस्य व्यतिरेकेपि धर्मकार ‚{\tiny $_{2}$}‚ णत्वमिष्यते तदापि ब्रूम इत्याह । ‚{\color{DodgerBlue3}‚अर्थान्तरत्त्वेपि द्रव्यस्य ‚{\tiny $_{lb}$}‚धर्मकारणत्वे} ऽङ्गीक्रिय ‚{\color{DodgerBlue3}‚माणेऽर्थान्तरस्य कार्यस्योत्पादनात्} ‚{\tiny $_{5b2}$}‚ कारणात् । द्रव्यस्य ‚{\tiny $_{lb}$}‚ \leavevmode\ledsidenote{\textenglish{53/s}} ‚{\color{DodgerBlue3}‚परिणाम इतीष्टं स्या} ‚{\tiny $_{5b2}$}‚ द् भवता ततः किं स्यात् इत्याह \add{।} ‚{\color{DodgerBlue3}‚तद्विरुद्धस्यापि तथा‚{\tiny $_{lb}$}‚गता} नुसारिणः \add{।} किङ्कारणन्तेनापि ‚{\color{DodgerBlue3}‚हेतुफलसंतानं} मृद्द्रव्याख्ये पूर्व्वकात् ‚{\color{DodgerBlue3}‚मृप्तिण्डा} ‚{\tiny $_{lb}$}‚त्कारणभूता ‚{\tiny $_{4}$}‚ दुत्तरस्य ‚{\color{DodgerBlue3}‚घटद्रव्यस्य कार्यस्योत्पत्तौ} सत्यां ‚{\color{DodgerBlue3}‚मृद्द्रव्यं परिणतमिति व्यवहार‚{\tiny $_{lb}$}‚भेदस्योपगमात्} ‚{\tiny $_{5b3}$}‚ कारणात् । स्यात् मतं । यदि नाम प्रकारद्वयेनापि परि‚{\tiny $_{lb}$}‚णामो ‚{\tiny $_{5}$}‚ न युज्यते प्रकारान्तरेण तु भविष्यतीत्येतदाह । ‚{\color{DodgerBlue3}‚न चे} ‚{\tiny $_{5b3}$}‚ त्यादि । ‚{\tiny $_{lb}$}‚तस्मादुभ\add{य}थापि न परिणाम इत्युपसंहारः । ‚{\color{DodgerBlue3}‚न निर्विवेकं} निर्विशेषं ‚{\color{DodgerBlue3}‚द्रव्यमेव} ‚{\tiny $_{lb}$}‚ ‚{\tiny $_{5b3}$}‚ परो ना ‚{\tiny $_{6}$}‚ पि ‚{\color{DodgerBlue3}‚द्रव्यादर्थान्तर} मेकान्तेनैव ‚{\color{DodgerBlue3}‚किन्तर्हि द्रव्यसन्निवेशोऽवस्था‚{\tiny $_{lb}$}‚न्त} रन्नान्यः ‚{\color{DodgerBlue3}‚यथाङ्गुलीनां} सन्निवेशोऽवस्थान्तरम्मुष्टिः । यथाङ्गुलीनां सन्निवेशो‚{\tiny $_{lb}$}‚ऽवस्थान्तरन्तत्वान्यत्वाभ्यामनिर्वचनीयम्मुष्टिः कस्माद्धेतोर्न ह्यङ्गुल्य ‚{\tiny $_{7}$}‚ एव ‚{\tiny $_{lb}$}‚निर्विवेका मुष्टिः । कुतः ‚{\color{DodgerBlue3}‚प्रसारितानाममुष्टित्वात्} ‚{\tiny $_{5b4}$}‚ । अन्यथा प्रसारिता‚{\tiny $_{lb}$}‚नामपि विशेषाभावात् मुष्ट्यवस्थायामिव मुष्टित्त्वप्रसङ्ग इति । अभावहेतुकाले ‚{\tiny $_{lb}$}‚रूप ‚{\tiny $_{8}$}‚ कः । ‚{\color{DodgerBlue3}‚नाप्यर्थान्तरं} मुष्टिरङ्गुलिव्यतिरेकेणाप्रतिहतकारणेन प्रयत्नवतापि ‚{\tiny $_{lb}$}‚मुष्टेरनुपलब्धेरिति । कदाचि ‚{\color{DodgerBlue3}‚त्कापिला} एवं ब्रूयुरिति तन्मतं शङ्कते \add{।} ‚{\color{DodgerBlue3}‚न निर्वि‚{\tiny $_{lb}$}‚वेक} ‚{\tiny $_{9}$}‚ \leavevmode\ledsidenote{\textenglish{38a/msK}} मित्यादिना । गतार्थमेतद् । ‚{\color{DodgerBlue3}‚नहि मुष्टेरङ्गुलिविशेषत्वादि} ‚{\tiny $_{5b5}$}‚ ति परिह- ‚{\tiny $_{lb}$}‚रति । असक्ताङ्गुल्य एव च निर्विवेका मुष्टिरिति कथयन् दृष्टान्तायोगमाह । ‚{\color{DodgerBlue3}‚अतोपि ‚{\tiny $_{lb}$}‚यदुत । ‚{\tiny $_{1}$}‚ प्रसारितानाममुष्टित्वादिति} तदप्ययुक्तमेव । किङ्कारणं \add{।} यतोङ्गु ‚{\color{DodgerBlue3}‚ल्य ‚{\tiny $_{lb}$}‚एव हि} विशिष्टहेतुप्रत्ययबलेन तथोत्पन्ना ‚{\color{DodgerBlue3}‚काश्चन मुष्टिर्न्न तु सर्वाः} । तदेव कुत ‚{\tiny $_{lb}$}‚इत्याह ‚{\tiny $_{2}$}‚ । ‚{\color{DodgerBlue3}‚न प्रसारिता} अङ्गुल्यो ‚{\color{DodgerBlue3}‚निर्विवेकस्वभावा मुष्ट्यङ्गुल्यश्चेति} \add{।} च शब्दोत्र ‚{\tiny $_{lb}$}‚लुप्तनिर्दिष्टो ज्ञेयः । अथवा मुष्ट्यात्मिका अङ्गुल्यः प्रसारिताः सत्यो नहि निर्विशिष्ट‚{\tiny $_{lb}$}‚रूपा इ ‚{\tiny $_{3}$}‚ ति व्याख्येयं । कस्मात् । अवस्थादूयेपि प्रसारिताप्रसारितरूपे । उभयोर‚{\tiny $_{lb}$}‚प्रसारितप्रसारितावस्थयोर्यथाक्रमं ‚{\color{DodgerBlue3}‚प्रतिपत्तिप्रसङ्गात्} ‚{\tiny $_{5b5}$}‚ । प्रयोगः \add{पुनः ।} ‚{\tiny $_{lb}$}‚प्रसारितावस्थायामप्रसारितावस्थायाः प्रतिपत्तिर्भवेत् अङ्गुलीनां विवेकाभा‚{\tiny $_{lb}$}‚ \leavevmode\ledsidenote{\textenglish{54/s}} वात् । अप्रसारितावस्थायामिव स्वभावहे ‚{\tiny $_{5}$}‚ तुः । एवमप्रसारितावस्थायां प्रसारिता‚{\tiny $_{lb}$}‚वस्थायां तत्प्रति ‚{\tiny $_{7}$}‚ पत्तिः स्यादित्यपरो योज्यः । यत्तूभयस्येति व्यापकानुपलब्धिः ‚{\tiny $_{lb}$}‚योज्या । अथाऽ ‚{\tiny $_{6}$}‚ पि कथञ्चित्कश्चिद्विवेको स्थितयोर\add{व}स्थयोस्तदा स विवेकश्चा‚{\tiny $_{lb}$}‚सामङ्गुलीनां स्वभावभूतो वा भवेन्नवेति विकल्पद्वयं \add{।} प्रथमे तावद् दोषमाह । ‚{\color{DodgerBlue3}‚य ‚{\tiny $_{lb}$}‚एव} खलु ‚{\color{DodgerBlue3}‚विवेकः स्व ‚{\tiny $_{7}$}‚ भावभूतः । स एव स्वभेदलक्षणं सुखदुःखवदिति} ‚{\tiny $_{5b6}$}‚ । ‚{\tiny $_{lb}$}‚द्वितीयेप्याह । ‚{\color{DodgerBlue3}‚परभूते च विवेकोत्पादेऽङ्गुल्यः प्रसारिता एवोपलभ्येरन्} ‚{\tiny $_{5b6}$}‚ ‚{\tiny $_{lb}$}‚मुष्ट्यवस्थायामपीति शेषः ‚{\tiny $_{8}$}‚ \add{।} किमिति । यतो नहि स्वभावादप्रच्युतस्यार्थान्तरो‚{\tiny $_{lb}$}‚त्पादे सत्यन्यथोपलब्धिर्भवत्यतिप्रसङ्गात् । उष्ट्रस्याप्यर्थान्तरस्य कलभस्योत्पादे‚{\tiny $_{lb}$}‚ऽत्यथोपलब्धिः स्या ‚{\tiny $_{9}$}‚ \leavevmode\ledsidenote{\textenglish{38b/msK}} दित्यतिप्रसङ्गो वक्तव्यः । प्रयोगः पुनः । यत्रस्वस्यात्म‚{\tiny $_{lb}$}‚भावादप्रच्युतं न तस्यार्थान्तरोत्पादेपि अन्यथोपलब्धिः । यथोष्ट्रस्य कलभ‚{\tiny $_{lb}$}‚प्रादुर्भावे । अप्रच्युताश्च स्व ‚{\tiny $_{1}$}‚ स्मात्स्वभावादङ्गुल्यो विवेकोत्पादेपीति विधि‚{\tiny $_{lb}$}‚प्रतिषेधाभ्यां हेत्ववकल्पनायां कारणविरुद्धकारणानुपलब्धी । नन्वित्यादि ‚{\tiny $_{lb}$}‚परः । तत्वान्यत्वाभ्यामनिर्वचनीयं तदुक्तमिति वाक्यार्थः । उक्तमेतन्न पुनर्युक्त‚{\tiny $_{lb}$}‚मित्या ‚{\color{DodgerBlue3}‚चार्यः} । कथमयुक्तमित्याह । ‚{\color{DodgerBlue3}‚नहि सतो वस्तुनस्तत्त्वान्यत्वे मुक्त्वान्य\add{ः}प्रकारः ‚{\tiny $_{lb}$}‚सम्भवती} ‚{\tiny $_{5b7}$}‚ ति सद्वस्तुग्रहणं कल्प ‚{\tiny $_{3}$}‚ नाशिल्पोपरिचतस्यान्यापोहादेः सम्भव‚{\tiny $_{lb}$}‚तीति प्रतिपादनाय । कुत इत्याह । ‚{\color{DodgerBlue3}‚तयोरि} ‚{\tiny $_{5b8}$}‚ त्यादि । प्रयोगः पुनः । यौ परस्पर‚{\tiny $_{lb}$}‚परिहारस्थितलक्षणौ ‚{\tiny $_{4}$}‚ तयोरेकत्यागोऽपरोपादाननान्तरीयकः । एकोपादानञ्चा‚{\tiny $_{lb}$}‚परत्यागनान्तरीयकं तद्यथा भावाभावौ । यथोक्तधर्मवन्तौ च तत्वान्यत्व‚{\tiny $_{lb}$}‚प्रकारावि ‚{\tiny $_{5}$}‚ ति स्वभावहेतुः । नन्वङ्गुलीभ्यो मुष्टेस्तत्वान्यत्वप्रकारौ मुक्त्त्वाप्य‚{\tiny $_{lb}$}‚न्यः प्रकारः संभवत्येव । न ह्यङ्गुल्य एव मुष्टिः प्रसारितानाममुष्टित्वात् । नाप्य‚{\tiny $_{lb}$}‚ ‚{\tiny $_{6}$}‚ र्थान्तरं पृथक्स्वभावानुपलब्धेरिति चेदाह । ‚{\color{DodgerBlue3}‚अङ्गुलीषु पुनरि} ‚{\tiny $_{5b8}$}‚ ‚{\tiny $_{lb}$}‚त्यादि । प्रतिक्षणं विनाशो विद्यते यासां इति विग्रहः । ता एव क्षणिकत्वात् ‚{\tiny $_{lb}$}‚तथाविधा जाय ‚{\tiny $_{7}$}‚ न्ते येन मुष्ट्यादिवाच्या भवन्तीत्यर्थः ॥ तदेतच्च वस्तुतो न ‚{\tiny $_{lb}$}‚ \leavevmode\ledsidenote{\textenglish{55/s}} मुष्टेरङ्गुलिविशेषादित्यत्रोक्तमपि प्रसङ्गात् युक्तमुक्तमित्यवसेयं । अन्यथा कि‚{\tiny $_{lb}$}‚मनेन यद्ये ‚{\tiny $_{8}$}‚ वं कथन्तर्हि मुष्टिरङ्गुलीति च व्यपदेशभेद इत्यत आह । ‚{\color{DodgerBlue3}‚तत्र मुष्ट्यादि‚{\tiny $_{lb}$}‚शब्दा विशिष्टविषया} ‚{\tiny $_{5b9}$}‚ विशिष्टावस्थानामेवाङ्गुलीनां वाचकत्वात् । ‚{\tiny $_{lb}$}‚अङ्गुलीशब्दस्तु सामा ‚{\tiny $_{9}$}‚ \leavevmode\ledsidenote{\textenglish{39a/msK}} न्यशब्दः सर्व्वावस्थानां तासामभिधायकत्वात् । यथाक्रम- ‚{\tiny $_{lb}$}‚मुदाहरणद्वयमाह । ‚{\color{DodgerBlue3}‚वीजाङ्कुरादिशब्दवद् ब्रीह्यादिशब्दवच्चे} ‚{\tiny $_{5b9}$}‚ ति । एवं ‚{\tiny $_{lb}$}‚शकलीकृतसकलपरपक्षः कु ‚{\tiny $_{1}$}‚ चोद्यशेषं परोपन्यस्तं परिजिहीषुः । परमुखेन चोद्य‚{\tiny $_{lb}$}‚मुपस्थापयति \add{।} ‚{\color{DodgerBlue3}‚तद्यदीत्या} ‚{\tiny $_{5b9}$}‚ दिना । इदमस्याकूतं यथा हि तिलेष्वविद्यमानं ‚{\tiny $_{lb}$}‚घृतं । तथा तैलमपि ‚{\tiny $_{2}$}‚ । तद्यदि प्रागसदेव कारणे कार्यमुत्पद्यते तथा घृतस्यापि तिलेभ्य ‚{\tiny $_{lb}$}‚उत्पत्तिः स्यात् । असत्वात् तैलवत् । न वा तैलस्यापि तत एव घृतवत् । नहि असत्ये ‚{\tiny $_{lb}$}‚कश्चिद् वि ‚{\tiny $_{3}$}‚ शेष इति स्वभावहेतुव्यापकानुपलब्धित्वेनाभिमतयोर्व्याप्यव्यापकभाव‚{\tiny $_{lb}$}‚प्रसाधनप्रकार एषः ।
	{\color{gray}{\rmlatinfont\textsuperscript{§~\theparCount}}}
	\pend% ending standard par
      ‚{\tiny $_{lb}$}‚

	  
	  \pstart \leavevmode% starting standard par
	तदेतत् सर्वं\add{म}भ्यवधाय कृत्योत्थापनम्भवत इ ‚{\tiny $_{4}$}‚ ति मन्यमानः प्राह । ‚{\color{DodgerBlue3}‚ननु सर्व्वत्र ‚{\tiny $_{lb}$}‚सर्व्वस्यासत्वेप्ययन्तुल्यो दोषः} ‚{\tiny $_{5b10}$}‚ । नहि सत्वे कश्चिद्विशेष इतिप्रयोगो\edtext{}{\lemma{इतिप्रयोगो}\Bfootnote{‚{\tiny $_{lb}$}‚? प्रयोगः}}पुनस्तावेव सत्वादिति हेतुविपर्ययं ‚{\tiny $_{5}$}‚ कृत्वा कार्यौ । अथापि कश्चिद्वि‚{\tiny $_{lb}$}‚शेषोस्ति तेन सत्वेपि न सर्व्वं सर्वस्मात् जायते तेन संदिग्धविपक्षव्यावृत्तिकत्वं ‚{\tiny $_{lb}$}‚प्रमाणयोरिति चेदाह । विशेषे चाऽ ‚{\tiny $_{6}$}‚ ‚{\color{DodgerBlue3}‚भ्युपगम्य} माने सविशेषस्त्रैगुण्यात् सत्वर‚{\tiny $_{lb}$}‚जस्तमोरूपाद् भिन्नः स्यात् । कस्मात्तस्य त्रैगुण्यस्य भावेपि विशेषस्याननुवृत्तेः ‚{\tiny $_{lb}$}‚कारणात् । प्रयोगः पुनः । यद्भावेपि ‚{\tiny $_{7}$}‚ यन्नानुवर्त्तते तत्तस्मादत्यन्तं भिन्नं । यथा ‚{\tiny $_{lb}$}‚शब्दस्पर्शरूपरसगन्धेभ्यश्चैतन्यन्नानुवर्त्तते च विशेषस्त्रैगुण्यभावेपीति स्वभावहेतुः । ‚{\tiny $_{lb}$}‚एतच्चाभ्युपगम्योद्ग्राहितं ‚{\tiny $_{8}$}‚ । अधुना सत्कार्यवादे जन्मार्थ एव न युक्त इत्याह । ‚{\tiny $_{lb}$}‚ ‚{\color{DodgerBlue3}‚सतश्चे} ‚{\tiny $_{5b10}$}‚ त्यादि । नैव तस्य चासत्वेनाभिमतस्य जन्मास्ति । सत्वात् । ‚{\tiny $_{lb}$}‚निष्पन्नावस्थायामिवेति विरुद्धव्याप्तोपलब्धिरस्य ‚{\tiny $_{9}$}‚ \leavevmode\ledsidenote{\textenglish{39b/msK}} मनसि वर्त्तते । अन्यथा ‚{\tiny $_{lb}$}‚पुनर्जातस्यापि पुनर्जातिः प्रसज्यत इत्यनवस्था स्यात् । यदाह । ‚{\tiny $_{lb}$}‚ 
	    \pend% close preceding par
	  
	    
	    \stanza[\smallbreak]
	  \flagstanza{\tiny\textenglish{...13}}{\normalfontlatin\large ``\qquad}सतो यदि भवेज्जन्म जातस्यापि भवेद् भव \add{१३}{\normalfontlatin\large\qquad{}"}\&[\smallbreak]
	  
	  
	  
	    \pstart  \leavevmode% new par for following
	    \hphantom{.}
	   इति ।
	{\color{gray}{\rmlatinfont\textsuperscript{§~\theparCount}}}
	\pend% ending standard par
      ‚{\tiny $_{lb}$}‚\textsuperscript{\textenglish{56/s}}

	  
	  \pstart \leavevmode% starting standard par
	किञ्च साधनानां कारणानाम्बीजतेजोजलादी ‚{\tiny $_{1}$}‚ नां वैफल्यं प्रसज्येत साध्यस्य ‚{\tiny $_{lb}$}‚कर्त्तव्यस्य कस्यचिद्रूपस्याभावादिति प्रयोगः । यत्र साध्यन्न किञ्चिदप्यस्ति तत्र ‚{\tiny $_{lb}$}‚साधनसाफल्यं विद्यते यथा नभस्यनाधेयातिशये । न च साध्यङ्कि ‚{\tiny $_{2}$}‚ ञ्चि‚{\tiny $_{lb}$}‚दप्यस्ति कारणे व्यवस्थिते सति कार्यं इति व्यापकानुपलब्धिः । न चायमसिद्धो ‚{\tiny $_{lb}$}‚हेतुरिति मन्तव्यं । यस्माद् यस्य कस्यचिदतिशयस्य तत्र कारणे स्थिते कार्ये कथञ्चि‚{\tiny $_{lb}$}‚दुत्पत्ता ‚{\tiny $_{3}$}‚ विष्यमाणायां सोतिशयस्तत्रासन् कथञ्जायेत नैव जायेतासत्वात् । ‚{\tiny $_{lb}$}‚व्योमोत्पलमिव दुग्ध इति व्यापकानुपलब्धिरस्य चेतसि स्थिता । अथासन्नप्यति‚{\tiny $_{lb}$}‚शयो जा ‚{\tiny $_{4}$}‚ यते । तदा जातौ वा तस्यासतोपि सर्व्वोतिशयः सर्व्वस्माज्जायेतेति तुल्यः ‚{\tiny $_{lb}$}‚पर्यनुयोग इति । भवतोपि घृतातिशयोपि तिलेभ्य उत्पद्येतासत्वात् । तेनाति ‚{\tiny $_{5}$}‚ ‚{\tiny $_{lb}$}‚शयवदित्यर्थः । स्वभावहेतुप्रसङ्गः । परमतमाशङ्कते । ‚{\color{DodgerBlue3}‚नातिशयस्तत्रे} ‚{\tiny $_{6a2}$}‚ ‚{\tiny $_{lb}$}‚त्यादिना । यथा नास्ति स कथन्तत्रासन् प्रकारो जायेतेति प्रक्षिपति । जातो ‚{\tiny $_{6}$}‚ वा ‚{\tiny $_{lb}$}‚सर्व्वः सर्व्वस्माज्जायेतेति तुल्यः पर्यनुयोग इति पूर्वोक्तो दोषो न युज्यत ‚{\tiny $_{lb}$}‚इत्यभिप्रायः । सर्वप्रकारेणैव तर्हि निष्पन्नरूपातिशयोस्तीति चेदाह । ‚{\tiny $_{7}$}‚ ‚{\color{DodgerBlue3}‚न चेत्यादि} \add{।} ‚{\tiny $_{lb}$}‚एवन्तावत्सदसत्कार्यवादिनोः सर्व्वस्मात्सर्व्वस्योत्पत्तिदोषस्तुल्य इति प्रतिपादितं ।
	{\color{gray}{\rmlatinfont\textsuperscript{§~\theparCount}}}
	\pend% ending standard par
      ‚{\tiny $_{lb}$}‚

	  
	  \pstart \leavevmode% starting standard par
	न च तयोरपि तुल्यञ्चोद्यन्न तदेको वक्तुमर्हति । सत्कार्यवादे च न ‚{\tiny $_{8}$}‚ कश्चिज्ज‚{\tiny $_{lb}$}‚न्मार्थ इति प्रसाधितं तेनायमस्तीत्यधिको दोषः । तदेवङ्कदाचित्परोऽभिदध्यान्ननु भो ‚{\tiny $_{lb}$}‚यदि नाम मयैतन्न परिहृतं भवता त्ववस्यं\edtext{}{\lemma{त्ववस्यं}\Bfootnote{? श्यं}}स्थितेः किञ्चित् स्वपक्षस्य रक्षणाय ‚{\tiny $_{9}$}‚ \leavevmode\ledsidenote{\textenglish{40a/msK}} ‚{\tiny $_{lb}$}‚वाच्यं । नहि परस्य पक्षं दूषयता स्वपक्षस्थितिरनवद्या लभ्यते । न भवति नित्यः ‚{\tiny $_{lb}$}‚शब्दो मूर्त्तत्वात् । सुखादिभिर्व्यभिचारेणेत्यादावनित्यत्वा ‚{\tiny $_{1}$}‚ सिद्धवदित्यत आह । ‚{\tiny $_{lb}$}‚ ‚{\color{DodgerBlue3}‚असतोपि कार्यस्य कारणादुत्पादे यो यज्जननस्वभावस्तत एव तस्य जन्म जन्म नान्य‚{\tiny $_{lb}$}‚स्मादिति नियम} ‚{\tiny $_{6a3}$}‚ इति । अपि शब्दः सम्भावनायां । इदं अत्रा ‚{\tiny $_{2}$}‚ र्थतत्व‚{\tiny $_{lb}$}‚मविद्यमानमपि तैलं तिलेभ्य एवोत्पद्यते । तदुत्पादनशक्तियुक्तत्वात् तिलानां ‚{\tiny $_{lb}$}‚नान्यस्मात् तज्जननशक्तिविकलत्वात्तस्य । शक्तिप्रतिनियम एव च कथमि ‚{\tiny $_{3}$}‚ ति च ‚{\tiny $_{lb}$}‚पर्यनुयोगे वस्तुस्वभावैरुत्तरं वाच्यं । य एवम्भवन्ति यथा वा तथैव प्रधाना‚{\tiny $_{lb}$}‚ \leavevmode\ledsidenote{\textenglish{57/s}} न्महान् एव जायते नाहङ्कारो महतो ऽहङ्कारो न पञ्चतन्मात्राणीत्या ‚{\tiny $_{4}$}‚ दि प्रक्रिया । ‚{\tiny $_{lb}$}‚तत्र च भवतः शक्तिप्रतिनियमावलम्बनमेवसरणं\edtext{}{\lemma{शक्तिप्रतिनियमावलम्बनमेवसरणं}\Bfootnote{? शरणं}}। अन्यस्य ‚{\tiny $_{lb}$}‚परिहारोपायस्याभावात् \add{।} तदेतच्च न ममापि काकेन भक्षितं । तेन ‚{\tiny $_{5}$}‚ यदुक्तन्न‚{\tiny $_{lb}$}‚ह्यसत्वे कश्चिद् विशेष इति तदयुक्तिमत् । कारणसामर्थ्यासामर्थ्यकृतत्वात् ‚{\tiny $_{lb}$}‚कार्योत्पत्यनुत्पत्योः । तस्मात् पुरोनुक्रान्तयोः प्रमाणयोः सन्दिग्धवि ‚{\tiny $_{6}$}‚ पक्षव्यावृत्ति‚{\tiny $_{lb}$}‚कत्वसाधनकलङ्काङ्कितो हेतुरिति । भवेदेतत्तस्यापि हेतो \add{ः} तज्जनन‚{\tiny $_{lb}$}‚स्वभावनियमः । कुतो जात इत्यत आह । ‚{\color{DodgerBlue3}‚तस्यापि स स्वभावनियमः । स्वहेतो ‚{\tiny $_{7}$}‚} ‚{\tiny $_{lb}$}‚रिति ‚{\tiny $_{6a4}$}‚ । तस्यापि स कुत इति चेदाह । इत्येवमनादिभावस्वभावनियम ‚{\tiny $_{lb}$}‚इति । न विद्यते आदिरस्येति विग्रहः अनादित्वाभ्युपगमाद्धेतुफलप्रकृति ‚{\tiny $_{lb}$}‚परं ‚{\tiny $_{8}$}‚ पराया नानवस्थादोषो लघीयसीमपि क्षतिमावहत्यन्यथाऽदौ कल्प्यमाने ‚{\tiny $_{lb}$}‚तस्याहेतुकत्वप्रसङ्गस्तेनास्थान एवेयमाशङ्का भवत इति भावः । अथवान्यथा‚{\tiny $_{lb}$}‚ऽ ‚{\tiny $_{9}$}‚ \leavevmode\ledsidenote{\textenglish{40b/msK}} यङ्ग्रन्थो व्याख्यायते \add{।} निष्पर्यायेणासन्नेव तर्ह्यतिशयो जायते । न च ‚{\tiny $_{lb}$}‚सर्व्वं सर्व्वस्माज्जायेतेति पर्यनुयोज्यं । यो यज्जननस्वभावस्तत एव तस्यातिशयो‚{\tiny $_{lb}$}‚त्पत्तिरिति शक्तिनियमसमाश्रयादिति कदाचित्स्वसिद्धान्तमनादृत्यापि परोभि‚{\tiny $_{lb}$}‚दधात्याशङ्कायां न ममाप्येतच्छक्तिप्रतिनियमावलम्वनङ्केनचिद्दण्डेन निवारित‚{\tiny $_{lb}$}‚मित्यागूर्याह । ‚{\color{DodgerBlue3}‚असतोपी} ‚{\tiny $_{6a3}$}‚ त्यादि । पदवि ‚{\tiny $_{2}$}‚ भागस्तु पूर्ववत् ।प्रयोगो\edtext{}{\lemma{प्रयोगो}\Bfootnote{? ‚{\tiny $_{lb}$}‚प्रयोगः}}पुनर्यस्य यज्जननाय समर्थं कारणमस्ति सोसन्नपि जायत एव यथातिशय‚{\tiny $_{lb}$}‚विशेषः । तज्जननाय समर्थङ्कारणमस्ति च कार्यविशेषस्येति ‚{\tiny $_{3}$}‚ स्वभावहेतुः । ‚{\tiny $_{lb}$}‚तथा यो यत्राविद्यमानतज्जननसमर्थकारणः स तत्रासत्वेपि नोदेति । यथा ‚{\tiny $_{lb}$}‚तिलेषु घृतातिशयस्तथा चाविद्यमानतज्जननसमर्थकारणः ‚{\tiny $_{4}$}‚ कार्यविशेषः कारण‚{\tiny $_{lb}$}‚विशेष इति व्यापकानुपलब्धिः । अपरः पर्य्यायः । साधनस्य लिङ्गस्य ।
	{\color{gray}{\rmlatinfont\textsuperscript{§~\theparCount}}}
	\pend% ending standard par
      ‚{\tiny $_{lb}$}‚
	  \bigskip
	  \begingroup
	
	    
	    \stanza[\smallbreak]
	  \flagstanza{\tiny\textenglish{...14}}{\normalfontlatin\large ``\qquad}सदकारणादुपादानग्रहणात् सर्व्वसम्भवाभावात् ।&‚{\tiny $_{lb}$}‚शक्यस्य श ‚{\tiny $_{5}$}‚ क्यकरणात् कारणभावाच्च सत्का\edtext{}{\lemma{सत्का}\Bfootnote{सांख्यकारिका ।}} र्यं \add{॥ १४}{\normalfontlatin\large\qquad{}"}\&[\smallbreak]
	  
	  
	  
	  \endgroup
	‚{\tiny $_{lb}$}‚

	  
	  \pstart \leavevmode% starting standard par
	इत्येवमादेर्वैफल्यं । साध्यस्य कर्तव्यस्य कस्यचित्संशयविपर्यासव्यवच्छेदस्य ‚{\tiny $_{lb}$}‚निश्चयप्रत्ययजन्मनश्चाभावात् । सर्व्वं हि ‚{\tiny $_{6}$}‚ साधनं विवादपदे वस्तुनि संशय‚{\tiny $_{lb}$}‚विपर्यासावपनयन्तद्विषयन्निश्चयप्रत्ययमुत्पादयद्विभर्त्ति नामानुरूपं न द्वयमप्येतत् ‚{\tiny $_{lb}$}‚ ‚{\color{DodgerBlue3}‚कापिलमते} सम्भवति । सदावस्थितस्य का ‚{\tiny $_{7}$}‚ र्यस्य हान्युपजननायोगात् । अथ ‚{\tiny $_{lb}$}‚ \leavevmode\ledsidenote{\textenglish{58/s}} सन्नप्ययं निश्चयः साधनवचनादनभिव्यक्तं । पूर्व्वमभिव्यक्तिमुपयात्यतो न ‚{\tiny $_{lb}$}‚वैफल्यमिति मतमत आह । ‚{\color{DodgerBlue3}‚यस्य कस्यचिदति}‚सय\edtext{}{\lemma{सय}\Bfootnote{? शय}}‚{\tiny $_{8}$}‚ स्या ‚{\tiny $_{6a1}$}‚ भि‚{\tiny $_{lb}$}‚व्यक्तिलक्षणस्य तत्र साध्ये निश्चयरूपे कथञ्चिदसत उत्पत्तौ प्राप्तात्साधनात् ‚{\tiny $_{lb}$}‚सोऽतिशयस्तत्रासन् ‚{\tiny $_{9}$}‚ \leavevmode\ledsidenote{\textenglish{41a/msK}} कथञ्जायते । जातो वा सर्व्वातिशयः । समस्तसाध्य‚{\tiny $_{lb}$}‚निश्चयाभिव्यक्तिलक्षणः सर्व्वस्मादन्यसाधनात् साधनाभासात् वोत्पद्येतेति तुल्यः ‚{\tiny $_{lb}$}‚प्रसङ्गः । पावकादिप्रतिपत्तिहेतवो ‚{\tiny $_{1}$}‚ धूमादयः सत्कार्यविनिश्चयाद्यभिव्यक्तिङ्कुर्यु‚{\tiny $_{lb}$}‚रित्यर्थः । उत्पत्त्य चाभिव्यक्तिमेतदुच्यते । नत्वियमविकृतरूपेषु कृतास्पदा सा ‚{\tiny $_{lb}$}‚हि तत्स्वरूपलक्षणा तद्विषयज्ञानलक्ष ‚{\tiny $_{2}$}‚ णा । रूपान्तरप्रादुर्भावलक्षणाभावा भवेत्स्वरूपं ‚{\tiny $_{lb}$}‚तावत् अविकार्यमिति न साधनैरन्यैर्वा कर्तुं शक्यते । विकारे वा पूर्व्वस्वभाववानिव ‚{\tiny $_{lb}$}‚पूर्व्वरूपप्रादुर्भावश्चेत्यसत्कार्य ‚{\tiny $_{3}$}‚ वाद एव समर्थितः पूर्वापररूपत्यागावाप्तिलक्षण‚{\tiny $_{lb}$}‚त्वात् विकारस्य । चैतन्यस्यैकत्वादपरस्तद्विषयः प्रत्ययो न भवति परस्येति ‚{\tiny $_{lb}$}‚तद्रूपाभि ‚{\tiny $_{4}$}‚ व्यक्तिरनुपपन्ना । रूपान्तरप्रादुर्भावे च नान्यस्य किंचिदप्युपजायते ‚{\tiny $_{lb}$}‚विलक्षणत्वादिति तृतीयापि व्यक्तिरसम्भविनी द्वितीयायामप्ययमनिवारितो ‚{\tiny $_{5}$}‚ दोषः । ‚{\tiny $_{lb}$}‚तद्विषयप्रत्ययोदयेप्यर्थान्तरस्याभूतभाववैपरीत्यस्य व्यक्तेरयोगात् । न चानुप‚{\tiny $_{lb}$}‚कारकः प्रत्ययस्य विषयः सम्भवी । तदुपकारकत्वे वा त ‚{\tiny $_{6}$}‚ स्मादेवास्योत्पत्तिरिति ‚{\tiny $_{lb}$}‚लिङ्गानपेक्षा । स्वत एव साध्यनिश्चयोस्याभिव्यक्तिरिति प्राप्तं । साधनापेक्षादेव ‚{\tiny $_{lb}$}‚साध्यनिश्चयात् स्वविषयज्ञानोत्पादेनैवापेक्षातिश ‚{\tiny $_{7}$}‚ योत्पत्तिलक्षणास्थिरेषु लब्धा‚{\tiny $_{lb}$}‚स्पदेति प्रतिपादितं सर्व्वदा वा भवेत् । लिङ्गस्यापि सदा सन्निहितरूपत्वात् । ‚{\tiny $_{lb}$}‚लिङ्गज्ञानापेक्षायामपि तुल्यः । तस्यापि सत्वे वादिनः स ‚{\tiny $_{8}$}‚ र्व्वकालास्तित्वादिति ।
	{\color{gray}{\rmlatinfont\textsuperscript{§~\theparCount}}}
	\pend% ending standard par
      ‚{\tiny $_{lb}$}‚

	  
	  \pstart \leavevmode% starting standard par
	\hphantom{.}अपि चेत्यादिना सत्कार्यवादनिराकरणे कारणान्तरमाह । ‚{\color{DodgerBlue3}‚तदवस्थाया} मिति ‚{\tiny $_{lb}$}‚ ‚{\tiny $_{6a4}$}‚ \add{।} मृत्पिण्डावस्थायां पश्चाद्वदभिव्यक्तावस्थायामिव तदर्थक्रियेति ‚{\tiny $_{9}$}‚ \leavevmode\ledsidenote{\textenglish{41b/msK}} घट‚{\tiny $_{lb}$}‚साध्योदकधारणविशेषाद्यर्थक्रिया । व्यक्तेरविशिष्टसंस्थानाया अप्रादुर्भावादिति ‚{\tiny $_{lb}$}‚चेत् । परमतासङ्का\edtext{}{\lemma{परमतासङ्का}\Bfootnote{? शङ्का}}तस्या एवेत्यादि प्रतिविधानं । एतदुक्तम्भवति । ‚{\tiny $_{lb}$}‚ग्रीवादिसन्निवे ‚{\tiny $_{1}$}‚ शविशेषावच्छिन्न एषोर्थक्रियाविशेषकारी कश्चित् मृद्विकारो ‚{\tiny $_{lb}$}‚घट इत्युच्यते नान्यः । स चेत् प्रागपि मृप्तिण्डावस्थायामपि तदाव्यक्तावस्थायामिव ‚{\tiny $_{lb}$}‚तदर्थक्रियोपल्ब्धौ स्या ‚{\tiny $_{2}$}‚ तां । न च भवतस्तस्मान्नास्त्येवासाविति निश्चयः समाधी‚{\tiny $_{lb}$}‚ \leavevmode\ledsidenote{\textenglish{59/s}} यतां किमलीकनिर्बन्धेनेति । अवस्थातुर्भावादसावप्यस्तीति चेदाह । ‚{\color{DodgerBlue3}‚नहि रूपान्त‚{\tiny $_{lb}$}‚रस्य भावे रूपान्तरमस्ति} ‚{\tiny $_{6a5}$}‚ । ‚{\tiny $_{3}$}‚ पीत इव नीलमिति विरुद्धव्याप्तोपलब्धिरा‚{\tiny $_{lb}$}‚कूता । न चावस्थावस्थात्रोरभेदादसिद्धो हेतुरिति गर्जितव्यं । यस्मान्न च रूपप्रति‚{\tiny $_{lb}$}‚भासभेदेपि वस्तुभेदो युक्त ‚{\tiny $_{4}$}‚ \add{ः,} अतिप्रसङ्गात् । रूपप्रतिभासभेदग्रहणमुपलक्षणार्थं । ‚{\tiny $_{lb}$}‚तेनार्थक्रियाभेदोप्यभ्युपगन्तव्यः । एवं मन्यते । यदि भिन्नप्रतिभासि ज्ञानं भेदं ‚{\tiny $_{lb}$}‚साधयति तदा ‚{\tiny $_{5}$}‚ सुखदुःखमोहानां असङ्कीर्णा भेदव्यवस्था भवेत् । नान्यथा तथा ‚{\tiny $_{lb}$}‚च मृप्तिण्डघटयोरपि परस्परमत्यंतम्भेद इति प्रतिजानीमहे । भिन्नाकारज्ञानपरि ‚{\tiny $_{6}$}‚ ‚{\tiny $_{lb}$}‚च्छेद्यत्वात् । परस्परासम्भविकार्यकारित्वाच्चसुखादिवदिति स्वभावहेतू । अन्यथा ‚{\tiny $_{lb}$}‚सुखादीनामपि परस्परमभेदप्रसङ्गः शक्तिव्यक्तिवत् । विशेषो वा वाच्य इति । ‚{\tiny $_{7}$}‚ ‚{\tiny $_{lb}$}‚तस्मादित्युपसंहारः । नहि तस्य घटादेस्तस्मिन्नुपलब्धिलक्षणप्राप्ते स्वभावे स्थितौ ‚{\tiny $_{lb}$}‚सत्यामनुपलब्धिर्युज्यते । अथापि भवति तदाऽस्थितिश्च तस्मिन्स्वभावेऽतत्वमत ‚{\tiny $_{8}$}‚ ‚{\tiny $_{lb}$}‚त्स्वभावत्वमुपलब्धिलक्षणप्राप्तात्स्वभावादेकान्तेन भेद इति यावत् ।
	{\color{gray}{\rmlatinfont\textsuperscript{§~\theparCount}}}
	\pend% ending standard par
      ‚{\tiny $_{lb}$}‚

	  
	  \pstart \leavevmode% starting standard par
	एवं स्वभाव\add{ा}नुपलब्धौ साधनाङ्गसमर्थनं प्रपञ्चेनाभिधाय परिशिष्टास्वनुपल‚{\tiny $_{lb}$}‚ब्धिष्वाचिख्यासुराह ‚{\tiny $_{9}$}‚ \leavevmode\ledsidenote{\textenglish{42a/msK}} । ‚{\color{DodgerBlue3}‚व्यापकानुपलब्धावि} ‚{\tiny $_{6a8}$}‚ त्यादि । धर्मयोर्यथा शिंशपा- ‚{\tiny $_{lb}$}‚त्ववृक्षत्वयोर्व्याप्यव्यापकभावं केनचित्प्रमाणेन प्रसाध्य व्यापकस्य वृक्षत्वादेर्न्निवृत्ति‚{\tiny $_{lb}$}‚प्रसाधनं समर्थनं साधनाङ्गस्येत्यध्याहारः । यथा नास्त्यत्र शिंशपा वृक्षाभावादिति । ‚{\tiny $_{lb}$}‚ननु तत्र स्वभावानुपलब्ध्यैव तदभावः सिध्यति तक्तिमनया । नहि निष्पादितक्रिये ‚{\tiny $_{lb}$}‚कर्मणां ‚{\tiny $_{2}$}‚ विशेषाधायि साधु साधनम्भवति । ‚{\color{DodgerBlue3}‚साधकतमङ्करणमिति} \href{http://sarit.indology.info/?cref=P\%C4\%81.1.4}{पाणिनिः १।४। ४२} वचनात् अनधिगतार्थाधिगमरूपञ्च प्रमाणमुक्तमज्ञातार्थप्रकाशो वेति । ‚{\tiny $_{lb}$}‚सत्यमेत ‚{\tiny $_{3}$}‚ त् । तथाहि नेयं सर्व्वत्र प्रयुज्यते । किन्तर्हि व्योमगतत्रपादिमात्रे यत्र ‚{\tiny $_{lb}$}‚सालसरलपलार्शाशंशपादिपादपभेदावधारणन्नास्ति तत्र । सर्व्वथा यत्रैव ‚{\tiny $_{4}$}‚ व्याप्या‚{\tiny $_{lb}$}‚ \leavevmode\ledsidenote{\textenglish{60/s}} भावो न निश्चीयते क्वचित् कुतश्चिद् भ्रान्तिनिमित्तात् तत्रैवेयं प्रयुज्यते । कारणा‚{\tiny $_{lb}$}‚नुपलब्धिरपि यत्र कार्याभावो न निश्चीयते तत्रैव प्रयोक्तव्या ना ‚{\tiny $_{5}$}‚ न्यत्र वैयर्थ्यात् । ‚{\tiny $_{lb}$}‚यथा सन्तमसे धूमाभावानिश्चये नास्त्यत्र धूमोऽग्न्यभावादिति । कार्याभावे संशयात् । ‚{\tiny $_{lb}$}‚कारणाभावे च निश्चयात् । स्वभावविरुद्धोपलब्धिरपि सं ‚{\tiny $_{6}$}‚ गविषयभावावस्थित‚{\tiny $_{lb}$}‚गात्रस्पर्शवालाकलापाकुलानलालीढ एव व्योमादिमात्रवर्तिनिर्देशे प्रयोक्तव्या । ‚{\tiny $_{lb}$}‚कारणविरुद्धोपलब्धिश्चाप्यदृश्यमानकमारोमोद्ग ‚{\tiny $_{7}$}‚ \leavevmode\ledsidenote{\textenglish{42b/msK}}मदन्तवीणादिभेदभावाभा वाक्य‚{\tiny $_{lb}$}‚शक्यगानुसमीपावस्थितपुरुषसमाक्रान्तभूतल एव प्रकृतेनान्यत्र वैफल्यात् । अनया‚{\tiny $_{lb}$}‚दिसा\edtext{}{\lemma{दिसा}\Bfootnote{? दिशा}}ऽन्यासामप्यनुपलब्धीनाम्प्रयो ‚{\tiny $_{8}$}‚ गविषयोऽनुसर्त्तव्य इति । तेषां स्वभाव‚{\tiny $_{lb}$}‚व्यापककारणानां । विरुद्धास्तेषामुपलब्धयस्तास्विति विग्रहः । द्वयोर्विरोधयोर्म‚{\tiny $_{lb}$}‚ध्ये एकस्योपदर्शनं । द्वौ पुनर्विरोधावविकलका ‚{\tiny $_{1}$}‚ रणस्य भवतोन्यभावे भावः । ‚{\tiny $_{lb}$}‚परस्परपरिहारस्थितलक्षणश्च । अनया दिशा स्वभावविरुद्धकार्योपलब्ध्यादिष्वपि ‚{\tiny $_{lb}$}‚साधनाङ्गसमर्थनं सुज्ञानमेवेति नोक्तं । तथा ‚{\tiny $_{2}$}‚ पि किञ्चिन्मात्रप्रयोगभेदादेका‚{\tiny $_{lb}$}‚दशानुपलब्धिव्यतिरिक्तास्वपि कारणविरुद्धव्याप्तोपलब्धिकार्यविरुद्धव्याप्तोप‚{\tiny $_{lb}$}‚लब्धिव्यापकविरुद्धकार्योपलब्धिकार्यविरुद्धका ‚{\tiny $_{3}$}‚ र्योपलब्ध्यादिषु साधनाङ्गसम‚{\tiny $_{lb}$}‚र्थनमुक्तम्वेदितव्यं । तासां पुनरुदाहरणानि यथाक्रमं । नात्र धूमस्तुषारस्पर्शात् । ‚{\tiny $_{lb}$}‚नेहाप्रतिवद्धसामर्थ्यान्यग्निकार ‚{\tiny $_{4}$}‚ णानि सन्ति तुषारस्पर्शात् । न तुषारस्पर्शोऽत्र ‚{\tiny $_{lb}$}‚धूमात् । नेहाप्रतिबद्धसामर्थ्यानि शीतकारणानि सन्ति धूमादिति ।
	{\color{gray}{\rmlatinfont\textsuperscript{§~\theparCount}}}
	\pend% ending standard par
      ‚{\tiny $_{lb}$}‚
	  \bigskip
	  \begingroup
	
	    
	    \stanza[\smallbreak]
	  \flagstanza{\tiny\textenglish{...15}}{\normalfontlatin\large ``\qquad}हेतुकार्यविरुद्धाप्तभावो व्यापक ‚{\tiny $_{5}$}‚ कार्ययोः ।&‚{\tiny $_{lb}$}‚विरुद्धकार्ययोरन्यः प्रतिषेधस्य साधकः ॥ \add{१५}{\normalfontlatin\large\qquad{}"}\&[\smallbreak]
	  
	  
	  
	  \endgroup
	
	  \bigskip
	  \begingroup
	
	    
	    \stanza[\smallbreak]
	  \flagstanza{\tiny\textenglish{...16}}{\normalfontlatin\large ``\qquad}नेह धूमो हिमस्पर्शात् समर्थन्नाग्निकारणं ।&‚{\tiny $_{lb}$}‚नेह धूमाद्धिमस्पर्शो न शक्यं शीतकारण \add{१६} मिति{\normalfontlatin\large\qquad{}"}\&[\smallbreak]
	  
	  
	  
	  \endgroup
	‚{\tiny $_{lb}$}‚

	  
	  \pstart \leavevmode% starting standard par
	\add{—} ‚{\tiny $_{6}$}‚ सङ्ग्रहश्लोकौ ।
	{\color{gray}{\rmlatinfont\textsuperscript{§~\theparCount}}}
	\pend% ending standard par
      ‚{\tiny $_{lb}$}‚

	  
	  \pstart \leavevmode% starting standard par
	एवं तावदेकेन प्रकारेणासाधनाङ्गवचनत्वादिनो निग्रहस्थानमिति प्रतिपादितं । ‚{\tiny $_{lb}$}‚प्रकारान्तरेणापि तदेवोपपादयति । ‚{\color{DodgerBlue3}‚अथवेत्यादि ‚{\tiny $_{7}$}‚} ‚{\tiny $_{6b1}$}‚ न चेति समुदायश्चायमत्रा‚{\tiny $_{lb}$}‚वृत्या पूर्वोदितार्थपरित्यागेनार्थान्तरसमुच्चये वर्त्तते ॥ नतु धवस्थित्याथवा खदिर‚{\tiny $_{lb}$}‚मित्यादाविव पूर्व्वार्थपरित्यागेन विकल्पादिवि ‚{\tiny $_{8}$}‚ \leavevmode\ledsidenote{\textenglish{43a/msK}} धस्याप्यर्थस्य विवक्षितत्वात् । इह ‚{\tiny $_{lb}$}‚ \leavevmode\ledsidenote{\textenglish{61/s}} च पर्याये साधनशब्दः करणसाधनः । इहाङ्गशब्दोऽवयववचनः पूर्वस्मिन्कारण‚{\tiny $_{lb}$}‚वचन इति विशेषः । त्रिरूपहेतुवचन ‚{\tiny $_{1}$}‚ समुदायग्रहणेन तेषु लक्षणादिवचनानां ‚{\tiny $_{lb}$}‚साधनत्वं तिरयति । स्याद् बुद्धिः साधर्म्यवति प्रयोगेनासपक्षे हेतोरसत्वमुच्यते । ‚{\tiny $_{lb}$}‚वैधर्म्यवति च न सपक्षसत्वमनन्तर ‚{\tiny $_{2}$}‚ मेव निषेध्यमानत्वात् । तत्कथन्तस्यैकस्याप्य‚{\tiny $_{lb}$}‚वचनमसाधनाङ्गवचनमित्येतन्न वक्ष्यमाणे व्याहतमिति । एतच्च नैवमेव हि ‚{\tiny $_{lb}$}‚व्याख्यायते । त्रिरूपो हेतुरर्थात्मकः । ‚{\tiny $_{3}$}‚ परमार्थतोवस्थितस्तस्य वचने ये प्रकाशके ‚{\tiny $_{lb}$}‚पक्षधर्मवचनं सपक्षसत्ववचने पक्षधर्मवचनं विपक्षसत्ववचने वा तयोस्समुदायः तस्य ‚{\tiny $_{lb}$}‚व ‚{\tiny $_{4}$}‚ चनद्वयसमुदायस्याङ्गम्पक्षधर्मादिवचनमिति पक्षधर्मवदनन्तावदविचलमितरयोः ‚{\tiny $_{lb}$}‚त्वन्यतरान्यतरत् कादाचित्कं । तेन वचनद्वयसमु ‚{\tiny $_{5}$}‚ दायरूपस्याङ्गिनोङ्गं द्विविधमेव ‚{\tiny $_{lb}$}‚सदा तस्येदानीमङ्गस्यैकस्याप्यवचनमसाधनाङ्गम्वचनं । न केवलं द्वयोः प्रथम‚{\tiny $_{lb}$}‚व्याख्यानुसारेणेत्यपि श ‚{\tiny $_{6}$}‚ ब्दात् । द्वयोर्ह्यवचनं तूष्णीम्भावः । स चोक्तोऽप्रतिभया ‚{\tiny $_{lb}$}‚तूष्णींभावादिति पर्यायान्तरमप्याह ।
	{\color{gray}{\rmlatinfont\textsuperscript{§~\theparCount}}}
	\pend% ending standard par
      ‚{\tiny $_{lb}$}‚

	  
	  \pstart \leavevmode% starting standard par
	\hphantom{.}‚{\color{DodgerBlue3}‚अथवे} ‚{\tiny $_{6b1}$}‚ त्यादि । तस्यैवेति त्रिरूपवचनसमुदायस्य यन्नाङ्गं ना ‚{\tiny $_{7}$}‚ वयवः । ‚{\tiny $_{lb}$}‚कथं पुनः प्रतिज्ञादीनामसाधनाङ्गत्वमिति चेत् । उच्यते । प्रतिज्ञावचन‚{\tiny $_{lb}$}‚साधनं । साक्षात् पारंपर्येण वा तस्याः सिद्धेरनुत्पत्तेः तथाह्यर्थ ए ‚{\tiny $_{8}$}‚ व प्रतिबन्धा‚{\tiny $_{lb}$}‚र्थङ्गमयति । नाभिधानमर्थप्रतिबन्धविकलत्वात् तस्मात् प्रतिज्ञावचनं हेतु‚{\tiny $_{lb}$}‚वचनं वा न साक्षात्साधनमर्थसिद्धौ । संशयश्च पक्षवचनादर्यें दृष्टो ‚{\tiny $_{9}$}‚ \leavevmode\ledsidenote{\textenglish{43b/msK}} न ‚{\tiny $_{lb}$}‚निश्चयस्तदतोपि न साक्षात् साधनं । स्यान्मतं संशय एवासिद्धः पक्षवचनाद्वादि‚{\tiny $_{lb}$}‚प्रतिवादिनोर्निश्चितत्वादथान्येषां भवति । एवं सति कृतकत्वादिवचनेप्यव्युत्पं\edtext{}{\lemma{कृतकत्वादिवचनेप्यव्युत्पं}\Bfootnote{‚{\tiny $_{lb}$}‚? व्युत्पन्}}‚{\tiny $_{1}$}‚ नानां संशयो भवतीत्यनेकान्तः । तदेतदसम्बद्धं । वादिप्रतिवादिनो ‚{\tiny $_{lb}$}‚र्हि निश्चितत्वमेकस्मिन् वा धर्मेऽनित्यत्वादिके प्रत्याययितुमारब्धे भवेत् प्रत्यनीक‚{\tiny $_{lb}$}‚धर्मद्वये वा ॥ ‚{\tiny $_{2}$}‚ न तावदेकस्मिन् विवादाभावतः । साधनप्रयोगानर्थक्यप्रसङ्गात् \add{।} ‚{\tiny $_{lb}$}‚नापि प्रत्यनीकधर्मद्वये वस्तुनो विरुद्धधर्मद्वयाध्यासप्रसङ्गात् । यदाह्येकस्मिन्व‚{\tiny $_{lb}$}‚स्तुनि प्रमा ‚{\tiny $_{3}$}‚ णबलेन विरुद्धौ धर्मौ वादिप्रतिवादिभ्यां निश्चितौ भवतस्तदा तद्वस्तु ‚{\tiny $_{lb}$}‚ \leavevmode\ledsidenote{\textenglish{62/s}} द्व्यात्मकं प्राप्तं । अथ न प्रमाणसामर्थ्यात् तौ निश्चितावपि तु स्वस्मात्स्व‚{\tiny $_{lb}$}‚स्मादाग ‚{\tiny $_{4}$}‚ मात् । एवमपि तु धर्मयोः प्रमाणेन निश्चयात् कथन्न पक्षवचनात् संशयो ‚{\tiny $_{lb}$}‚भवतीति वाच्यं । तस्मात् पक्षवचनं न साक्षात् साधनं । नापि पारम्पर्य्येण सा ‚{\tiny $_{5}$}‚ ध्या‚{\tiny $_{lb}$}‚भिधायकत्वेनासिद्धे हेतुदृष्टान्ताभासोक्तिवदशक्यसूचकत्वात् । हेतुवचनन्तु शक्य‚{\tiny $_{lb}$}‚सूचकत्वात् शक्तितः साधननिष्टं सदोच्यते साधनाङ्गम्प्रतिज्ञाव ‚{\tiny $_{6}$}‚ चनत्वे सति ‚{\tiny $_{lb}$}‚साधनोपकारकत्वाद्धेतुवचनवत् । साधनविषयप्रकाशनद्वारेण च प्रतिज्ञासाधनम‚{\tiny $_{lb}$}‚नुगृह्णाति । अन्यथाह्यविषयं तत्साधनं प्रवर्त्तते । ज्ञा ‚{\tiny $_{7}$}‚ नात्ममनःसन्निकर्षादीनामपि ‚{\tiny $_{lb}$}‚साधनोपकारकत्वमतो वचनत्वे सतीति विशेषणं । इतश्च साधनाङ्गसाध्यसाधन‚{\tiny $_{lb}$}‚विषयप्रकाशनात् दृष्टान्तवचनवदिति । ‚{\tiny $_{8}$}‚ इदमप्यत्यर्थमसारं । यस्मादनित्यं शब्दं ‚{\tiny $_{lb}$}‚साधयेत्यभ्यर्थंना वाह्यं वचनत्वे सति साधनोपकारकं साध्यसाधकविपर्ययप्रकाश‚{\tiny $_{lb}$}‚कञ्च न च तदन्तरङ्गं साध्यसिद्धा ‚{\tiny $_{9}$}‚ \leavevmode\ledsidenote{\textenglish{44a/msK}} वाङ्गं । को वा विषयोपदर्शनस्योपयोगो यदि ‚{\tiny $_{lb}$}‚ह्यनेन विना न साध्यसिद्धिः स्यात् । ‚{\color{DodgerBlue3}‚सर्व्व}‚सोभेत\edtext{}{\lemma{सोभेत}\Bfootnote{? शोभेत}}यावता विनाप्यनेन ‚{\tiny $_{lb}$}‚यावत् । \add{यः} कश्चित्कृतकः स सर्वोऽनित्यो यथा कुम्भादिः ‚{\tiny $_{1}$}‚ शब्दश्च कृतक इत्यनु‚{\tiny $_{lb}$}‚क्तेपि पक्षशब्दोऽनित्य इत्यर्थाङ्गमात्र एव । तस्मादस्य निर्देशो निरर्थक एव । स्यादयं ‚{\tiny $_{lb}$}‚विपर्यासो यदि हेतुव्यापारविषयोपदर्शनाय पक्षवद् ‚{\tiny $_{2}$}‚ वचनन्नैव प्रयुज्यते तदा कथम्पक्ष‚{\tiny $_{lb}$}‚समाश्रयलब्धव्यपदेशा । पक्षधर्मत्वादयः सम्पद्यन्ते । तेषु वा निश्रितात्मसुसम्भूत‚{\tiny $_{lb}$}‚सामर्थ्यात् पक्षगतिरसम्भाव्यैव । सामर्थ्य ‚{\tiny $_{3}$}‚ लभ्यपक्षबलेन पक्षधर्मत्वादयः सम्पद्यन्त ‚{\tiny $_{lb}$}‚इत्यप्ययुक्तं । तेष्वसत्सु सामर्थ्यलभ्यस्यैव पक्षस्यासम्भवात् । अन्योन्याश्रयं चेदम्पक्ष‚{\tiny $_{lb}$}‚धर्मत्वादिसामर्थ्या ‚{\tiny $_{4}$}‚ यातपक्षवशेन पक्षधर्मत्वादयः सम्पद्यन्ते । पक्षधर्मत्वादिबलेन ‚{\tiny $_{lb}$}‚च पक्ष इति । तदत्रोच्यते । न खलु साधनकाले पक्षधर्मत्वादिविकल्पोऽस्ति के ‚{\tiny $_{5}$}‚ वलं ‚{\tiny $_{lb}$}‚यत्रैव जिज्ञासितविशेषे धर्मिणि शब्दादौ तु च करीशादिस्थगिततेजसि वा कुण्डादौ ‚{\tiny $_{lb}$}‚यो धर्मः कृतकत्वधूमत्वादिलक्षणोनुमानतः प्रत्यक्षतो ‚{\tiny $_{6}$}‚ वा प्रतीयते । प्रत्याय्यते ‚{\tiny $_{lb}$}‚वा । तद्विशेषयोगितया वा निश्चितेऽपरस्मिन्घटमहानसादावस्थितत्वेन स्मर्यते ‚{\tiny $_{lb}$}‚तद्विशेषविरहिणि वा गगनसागरादौ नास्तित्वेनै ‚{\tiny $_{7}$}‚ व स्मर्यते । स वस्तुधर्मतयैव ‚{\tiny $_{lb}$}‚विनापि पक्षधर्मत्वादिव्यपदेशेन तत् धर्मिणं जिज्ञासितधर्मविशिष्टं सामर्थ्यादेव ‚{\tiny $_{lb}$}‚प्रतिपादयति । स चास्य सामर्थ्यविषयः पक्ष ‚{\tiny $_{8}$}‚ इति गीयते । ततः पश्चात् तत्समाश्रय‚{\tiny $_{lb}$}‚भाविन्यो यथेष्टपक्षधर्मत्वादिसंज्ञाः शास्त्रेषु संव्यवहारार्थम्प्रतन्यन्ते । यदि वा प्रत्या‚{\tiny $_{lb}$}‚लोचनप्रकरणबलात् साधनका ‚{\tiny $_{9}$}‚ \leavevmode\ledsidenote{\textenglish{44b/msK}} लेपि भवन्तु पक्षधर्मत्वादिविकल्पाः । कथं योहि ‚{\tiny $_{lb}$}‚वस्तुनो धर्मो वादिना विवा\add{दा}स्पदीभूतधर्मिविशिष्टतया साधयितुमिष्टः स पक्षस्तस्य ‚{\tiny $_{lb}$}‚योन्यो धर्मः स पक्षधर्मः । ‚{\tiny $_{1}$}‚ प्रकृतसाध्यधर्मसामान्येन च समानोर्थः सपक्षः । तद्विरही ‚{\tiny $_{lb}$}‚ \leavevmode\ledsidenote{\textenglish{63/s}} वासपक्ष इति । यस्यापि हि साधनकाले पक्षप्रयोगोस्ति तस्यापि न वाद्यकाण्डमेव ‚{\tiny $_{lb}$}‚पक्षं जातेऽनि ‚{\tiny $_{2}$}‚ त्यः शब्द इति । कस्तु प्रस्तावान्तरेना\edtext{}{\lemma{प्रस्तावान्तरेना}\Bfootnote{? रेणा}}पि प्रकरणबलेनैव ‚{\tiny $_{lb}$}‚पक्षधर्मत्वादयोपि वक्तव्या \add{ः} । तच्च पक्षप्रयोगदूषकस्यापि समानं । तस्मा‚{\tiny $_{lb}$}‚त्प्रतिज्ञावचनं न साधनां ‚{\tiny $_{3}$}‚ गं ।
	{\color{gray}{\rmlatinfont\textsuperscript{§~\theparCount}}}
	\pend% ending standard par
      ‚{\tiny $_{lb}$}‚

	  
	  \pstart \leavevmode% starting standard par
	\hphantom{.}उपनयनिगमनवचनन्तु यथा न साधनाङ्गन्तथोच्यते ॥ तत्रतावदु दाहरणा‚{\tiny $_{lb}$}‚पेक्षस्तथेत्युपसंहारो न तथेति वा साधनस्योपनयः \href{http://sarit.indology.info/?cref=ns\%C5\%AB.1.2.38}{न्या० सू०  १।२।३८ } । यथा ‚{\tiny $_{lb}$}‚त ‚{\tiny $_{4}$}‚ थेतिप्रतिबिम्बनार्थं । किम्पुनरत्र प्रतिबिम्बनन्दृष्टान्तगतस्य धर्मस्याव्यभिचारत्त्वे ‚{\tiny $_{lb}$}‚सिद्धे । तेन साध्यगतस्य तुल्यधर्मता । एवञ्चायङ्कृतक इति सा ‚{\tiny $_{5}$}‚ ध्येन सह ‚{\tiny $_{lb}$}‚सम्भव उपनयार्थः । ननु च कृतकत्वादित्यनेन सम्भव उक्तः । नोक्तः । साध्यसाधन‚{\tiny $_{lb}$}‚धर्ममात्रनिर्देशात् । साध्यसाधनधर्ममात्रनिर्देशः कृ ‚{\tiny $_{6}$}‚ तकत्वादनित्यः शब्दो भवति । ‚{\tiny $_{lb}$}‚तत्पुनः शब्दे कृतकत्त्वमस्ति । नास्तीत्युपनयेन सम्भवो गम्यते । अस्ति च शब्दे कृत‚{\tiny $_{lb}$}‚कत्वमिति । तथा च हेतुवचनाद् भिन्नार्थप्रतिपाद ‚{\tiny $_{7}$}‚ कत्वमुपनयस्याभिन्नरूपत्वे प्रसिद्ध‚{\tiny $_{lb}$}‚पर्यायव्यतिरिक्तत्वे च सति हेतुवचनोत्तरकालमुपादीयमानत्वात् दृष्टान्तवचन‚{\tiny $_{lb}$}‚वदिति शक्येत अनुमातुं ।
	{\color{gray}{\rmlatinfont\textsuperscript{§~\theparCount}}}
	\pend% ending standard par
      ‚{\tiny $_{lb}$}‚

	  
	  \pstart \leavevmode% starting standard par
	\hphantom{.}हेत्वपदेशात् ‚{\tiny $_{8}$}‚ प्रतिज्ञायाः पूनर्वचनन्निगमनं \href{http://sarit.indology.info/?cref=ns\%C5\%AB.1.1.39}{न्या० सू० १।१।३९ } । ‚{\tiny $_{lb}$}‚प्रतिज्ञायाः पुनर्वचनमिति प्रतिज्ञाविषयस्यार्थस्याशेषप्रमाणोपपत्तौ विपरीत‚{\tiny $_{lb}$}‚प्रसङ्गप्रतिषेधार्थं यत्पुनरभिधानं ‚{\tiny $_{9}$}‚ \leavevmode\ledsidenote{\textenglish{45a/msK}} तन्निगमनं । न पुनः प्रतिज्ञाया एव ‚{\tiny $_{lb}$}‚पुनर्वचनं । किङ्कारणं यस्मात्प्रतिज्ञासाध्यनिर्देशः सिद्धनिर्देशो निगमनमिति । ‚{\tiny $_{lb}$}‚पुनः शब्दश्च नानात्वे दृष्टः \add{।} पुनरियमचिरप्रभा नि ‚{\tiny $_{1}$}‚ श्चरति । पुनरिदङ्गन्ध‚{\tiny $_{lb}$}‚र्व्वनगरं दृश्यत इति । अत्र च सामर्थ्यादुपनयानन्तरभावी हेत्वपदेशो गृह्यते । न ‚{\tiny $_{lb}$}‚प्रतिज्ञानन्तरभावी । असम्भवात् । नहि कश्चित्प्रतिज्ञा ‚{\tiny $_{2}$}‚ नन्तरं हेत्वपदेशान्निगमनं ‚{\tiny $_{lb}$}‚प्रयुंक्ते । अनित्यः शब्दः कृतकत्वात् । तस्मादनित्यः शब्द इति । अतश्च प्रतिज्ञार्थ‚{\tiny $_{lb}$}‚वाक्याद् भिन्नार्थं निगमनवचनं । प्रतिज्ञावाक्याद् भिन्न ‚{\tiny $_{3}$}‚ रूपत्वे सति हेतुवचनोत्तर‚{\tiny $_{lb}$}‚कालमभिधीयमानत्वात् ‚{\color{DodgerBlue3}‚दृष्टान्तवचनवत्} । न च साध्यार्थप्रतिपादकन्निगमनं । ‚{\tiny $_{lb}$}‚शब्दान्तरोपात्तस्यावधारणरूपेण प्रवृत्तत्वात् । योय ‚{\tiny $_{4}$}‚ मागच्छत्ययं विषाणीति ‚{\tiny $_{lb}$}‚केनचिदुक्ते तस्मादनश्व इत्यादिवचनवत् । तस्माच्छब्दसहितं वाक्यम्विचार‚{\tiny $_{lb}$}‚विषयाय प्रसाध्यार्थप्रतिपादकन्न भवति । का ‚{\tiny $_{5}$}‚ रणोपदेशोत्तरकालमुपात्तत्त्वात् । ‚{\tiny $_{lb}$}‚दृष्टान्तः पूर्ववत् । तदेतत् प्रतिषिध्यते न खल्वेवं प्रयोगः क्रियते । अनित्यः शब्दः ‚{\tiny $_{lb}$}‚कृतकत्वात् । प्रतिज्ञाप्रयोग ‚{\tiny $_{6}$}‚ स्यानन्तरं निराकृतत्वात् । अपि तु कृतकः शब्दः । ‚{\tiny $_{lb}$}‚पश्चैवं स सर्वोऽनित्यो यथा कलशादिः । यो वा कृतकः स सर्व्वोऽनित्यो यथा घटादिः । ‚{\tiny $_{lb}$}‚ \leavevmode\ledsidenote{\textenglish{64/s}} तथा च कृतकः शब्द इ ‚{\tiny $_{7}$}‚ त्येवमुभयथा यथेष्टं प्रयोग\add{ः} क्रियते । साध्यसिद्धेरुभ‚{\tiny $_{lb}$}‚यथापि भावात् । तत्र यदि कृतकः शब्दो यश्चैवं स सर्व्वोऽनित्यो यथा घटादि‚{\tiny $_{lb}$}‚रित्यभिधाय तथा कृत ‚{\tiny $_{8}$}‚ कः शब्द इति प्रतिबिंबनार्थ ‚{\tiny $_{9}$}‚ \leavevmode\ledsidenote{\textenglish{45b/msK}} मुपनयवचनमुच्यते । तदे\edtext{}{\lemma{तदे}\Bfootnote{? दि}}‚{\tiny $_{lb}$}‚दमनर्थकं ‚{\tiny $_{1}$}‚ विनाप्यनेन प्रतिबिंबनेनानन्तरोक्तप्रयोगमात्रात् प्रतीतिभावात् । ‚{\tiny $_{lb}$}‚साधनञ्च यदनर्थकं न तत्साधनवाक्ये विद्वद्भिरुपादेयं । तद्यथा दशदाडिमादि ‚{\tiny $_{lb}$}‚वाक्यं तथा ‚{\tiny $_{2}$}‚ चानर्थकं प्रतिबिंबनार्थमुपनयवचनमिति व्यापकविरुद्धोपलब्धेः । ‚{\tiny $_{lb}$}‚स्वार्थानुमितावप्ययमेव न्यायो दृष्टो नहि कश्चित्सचेतनः कृतकत्त्वस्य भावं शब्दे ‚{\tiny $_{lb}$}‚गृही ‚{\tiny $_{3}$}‚ त्वा तस्य चाविनाभावित्वमनुस्मृत्य तथा च कृतकः शब्द इति प्रतिबिम्बनार्थ‚{\tiny $_{lb}$}‚करोति । अथापि यः कृतकः स सर्व्वोऽनित्यो यथा घटः । तथा च कृतकः शब्द ‚{\tiny $_{lb}$}‚इ ‚{\tiny $_{4}$}‚ ति सम्भवप्रदर्शनार्थमुपनयवचनमुच्यते । तदेतद् द्वयमप्यङ्गीकुर्मः । प्रतिज्ञा‚{\tiny $_{lb}$}‚नन्तरभाविनस्तु साधनमात्रनिर्देशमनित्यः शब्दः कृतकत्वादित्येव ‚{\tiny $_{5}$}‚ न प्रतिपद्यामहे । ‚{\tiny $_{lb}$}‚प्रतिज्ञायाः प्रयोगाभावात् । ततश्चोपनयस्यावयवान्तरत्वप्रतिपादनायोक्तो ‚{\tiny $_{lb}$}‚हेतुरसिद्धतोरगदष्टत्वाङ्गतावशक्त एव । यत् ‚{\tiny $_{6}$}‚ पुनरिदं सिद्धार्थनिर्देशलक्षणं निग‚{\tiny $_{lb}$}‚मनं पौनरुक्त्यपरिहाराय वर्ण्यते तन्नैवोपपद्यते विना निगमनेनार्थसिद्धेरेव पञ्चा‚{\tiny $_{lb}$}‚वयवसाधनवादिनोऽनुपपत्तेः ‚{\tiny $_{7}$}‚ अन्यथा निगमनात् प्रागेवार्थस्य सिद्धत्वात् व्यर्थतया ‚{\tiny $_{lb}$}‚न साधनाङ्गन्निगमनम्प्राप्नोति । ततश्च नेदमुपादेयं साधनवाक्ये सिद्धमित्य‚{\tiny $_{lb}$}‚प्रतिज्ञा । भवेद्व्यामोहो विप्र ‚{\tiny $_{8}$}‚ तिपन्नस्य प्रमाणान्तरव्यपेक्षा नास्तीति सिद्धमनित्यत्व‚{\tiny $_{lb}$}‚मुच्यते । निगमनं तु प्रतिविषयस्यार्थस्याशेषप्रमाणोपपत्तावशेषावयवपरामर्शेनाव‚{\tiny $_{lb}$}‚धारणार्थम ‚{\tiny $_{9}$}‚ \leavevmode\ledsidenote{\textenglish{46a/msK}} नित्य एवेति प्रवर्त्तत इति । यदि तर्हि प्रमाणान्तरव्यपेक्षा नास्ति तत्तर्हि ‚{\tiny $_{lb}$}‚\add{साध्यं} सामर्थ्यादवधार्यत एव । तथाहि यदकृतकन्तदनित्यमेव । यथा कुण्डादि‚{\tiny $_{lb}$}‚शब्दश्च कृतक ‚{\tiny $_{1}$}‚ इत्येवमनि\add{त्य}त्वाविनाभाविनः कृतकत्वस्य शब्दे भावख्यातौ ‚{\tiny $_{lb}$}‚तत्सामर्थ्यादेवानित्यः शब्द इति निश्चयो भवति \add{।} तदस्य वचनं सामर्थ्यं प्रतीता‚{\tiny $_{lb}$}‚र्थप्रत्यायकत्वात् पुनरुक्तमनु ‚{\tiny $_{2}$}‚ पादानार्हञ्च । न चात्र विपर्ययप्रसङ्गस्य लेशोप्या‚{\tiny $_{lb}$}‚शङ्क्यते । येन तद्व्यवच्छेदाय सफलमेतस्योपादानं स्यात् । अनित्यत्वेनैव कृतकत्वस्य ‚{\tiny $_{lb}$}‚व्याप्तिप्रसाधनात् । प्रयोगस्तु ‚{\tiny $_{3}$}‚ \add{।} यत्सामर्थ्यात् प्रतीयते न तस्य वचनम्प्रेक्षावता ‚{\tiny $_{lb}$}‚कर्त्तव्यं । तद्वचनम्पुनरुक्तम्वा तद्यथा गेहे नास्ति कुमारो जीवति चेत्येतत्सामर्थ्यात् ‚{\tiny $_{lb}$}‚प्रतीयंमानस्य तद्वहि ‚{\tiny $_{4}$}‚ र्भावस्य वचनं । पक्षधर्मान्वयव्यतिरेकतद्वचनसामर्थ्याच्च ‚{\tiny $_{lb}$}‚प्रतीयते तस्मादनित्य एवेत्येवमादिना पुनः सिसाधयिषितोर्थः प्रथमसाध्यापेक्षया ‚{\tiny $_{5}$}‚ ‚{\tiny $_{lb}$}‚व्यापकविरुद्धोपलब्धिर्द्वितीयसाध्यापेक्षया च स्वभावहेतुः । अत एव निगमनस्या‚{\tiny $_{lb}$}‚वयवान्तरत्वप्रतिपादनायोक्ता हेतवोऽसिद्धाः । तदप्येतेनैव प्र ‚{\tiny $_{6}$}‚ त्युक्तं । यदाह ।
	{\color{gray}{\rmlatinfont\textsuperscript{§~\theparCount}}}
	\pend% ending standard par
      ‚{\tiny $_{lb}$}‚
	  \bigskip
	  \begingroup
	
	    
	    \stanza[\smallbreak]
	  \flagstanza{\tiny\textenglish{...17}}{\normalfontlatin\large ``\qquad}प्रत्ययेक्ष \add{?} प्रतिज्ञादीन्वाक्यार्थप्रतिपत्तये ।&‚{\tiny $_{lb}$}‚\leavevmode\ledsidenote{\textenglish{65/s}}प्रोच्यमानन्निगमनं पुनरुक्तन्न जायते ॥ \add{१७}{\normalfontlatin\large\qquad{}"}\&[\smallbreak]
	  
	  
	  
	  \endgroup
	
	  \bigskip
	  \begingroup
	
	    
	    \stanza[\smallbreak]
	  \flagstanza{\tiny\textenglish{...18}}{\normalfontlatin\large ``\qquad}विप्रकीर्णौश्च वचनैर्नैकोर्थः प्रतिपाद्यते ।&‚{\tiny $_{lb}$}‚तेन सम्बन्धसि ‚{\tiny $_{7}$}‚ ध्यर्थम्वाच्यन्निगमनं पृथग् \add{॥ १८}{\normalfontlatin\large\qquad{}"}\&[\smallbreak]
	  
	  
	  
	  \endgroup
	‚{\tiny $_{lb}$}‚

	  
	  \pstart \leavevmode% starting standard par
	इत्यलमतिप्रसारिण्या कथया ॥ ० ॥
	{\color{gray}{\rmlatinfont\textsuperscript{§~\theparCount}}}
	\pend% ending standard par
      ‚{\tiny $_{lb}$}‚

	  
	  \pstart \leavevmode% starting standard par
	अन्वयव्यतिरेकयोर्वेति पर्य्यायान्तरकथनमुपादानमिति \href{http://sarit.indology.info/?cref=ns\%C5\%AB.2.1.12}{न्या० सू० २।१।१२ } ‚{\tiny $_{lb}$}‚वर्तते द्वितीयस्यासामर्थ्यं जातायाः ‚{\tiny $_{8}$}‚ सिद्धेः पुनरजन्यत्वात् । \add{प्रमाण-} समुच्चय ‚{\tiny $_{lb}$}‚ टीकाकारास्त्वाहुः नन्वि ‚{\tiny $_{6b4}$}‚ त्यादि । ने ‚{\tiny $_{6b4}$}‚ त्याद्युत्तरः । यदि चेत्युपचय‚{\tiny $_{lb}$}‚हेतुः । साधनावयवः प्रतिज्ञां प्राप्नोति नियमेन साध्यप्रतीतिनि ‚{\tiny $_{9}$}‚ \leavevmode\ledsidenote{\textenglish{46b/msK}} मित्तत्वात् ‚{\tiny $_{lb}$}‚पक्षधर्मादिवचनवत् । सन्दिग्धव्यतिरेको हेतुरिति चेदाह । ‚{\color{DodgerBlue3}‚नहि पक्षधर्मवचनस्या‚{\tiny $_{lb}$}‚पीति} ‚{\tiny $_{6b6}$}‚ । तत्तुल्यमिति विरुद्धानैकान्तिकयोः पक्षधर्मसद्भावेप्यगमकत्वात् ‚{\tiny $_{1}$}‚ । ‚{\tiny $_{lb}$}‚तत एवसंसयो\edtext{}{\lemma{एवसंसयो}\Bfootnote{? संशयो}}त्पत्तेः पक्षधर्मवचनन्न साधनं साधारणादिवचनवदिति ‚{\tiny $_{lb}$}‚चेदाह । ‚{\color{DodgerBlue3}‚ऐतेन} ‚{\tiny $_{6b7}$}‚ तत्तुल्यमित्यादिना संशयोत्पत्तिः प्रत्युक्तेति । एतदेव व्यनक्ति ‚{\tiny $_{lb}$}‚पक्षधर्म ‚{\tiny $_{2}$}‚ वचनादपीति । तदनेनानन्तरस्य हेतोर्व्यभिचारङ्कथयति ।
	{\color{gray}{\rmlatinfont\textsuperscript{§~\theparCount}}}
	\pend% ending standard par
      ‚{\tiny $_{lb}$}‚

	  
	  \pstart \leavevmode% starting standard par
	ननु च पक्षधर्मस्य श्रावणत्वादेरप्रदर्शिते सम्बन्धेनैव साधनावयवत्वमतो विपक्ष‚{\tiny $_{lb}$}‚त्वाभावान्न व्यभि ‚{\tiny $_{3}$}‚ चारः । प्रदर्शिते तु सम्बन्धे साधनावयवत्वं तदा च तस्मात् संशयो ‚{\tiny $_{lb}$}‚नास्तीति सुतरान्नानेकान्त इति ॥ एवं मन्यते । ‚{\color{DodgerBlue3}‚पक्षवचनेपि तुल्यमे} ‚{\tiny $_{6b6}$}‚ तदिति ‚{\tiny $_{lb}$}‚तदपि सा ‚{\tiny $_{4}$}‚ धनं स्यात् । अथ प्रतिपद्येथा सत्यं स्याद्यदि साध्यं स्यान्न चास्त्यन्यतः ‚{\tiny $_{lb}$}‚ \leavevmode\ledsidenote{\textenglish{66/s}} साध्यसिद्धः । न च निष्पादितक्रिये दारुणि दात्रादयः कञ्चनार्थं पुष्यन्ति । अप्रदर्शिते ‚{\tiny $_{lb}$}‚तु संब ‚{\tiny $_{5}$}‚ न्धे संशयोत्पत्तिहेतुत्वादिदमुक्तन्तत एवसंसयो\edtext{}{\lemma{एवसंसयो}\Bfootnote{? संशयो}}त्पत्तेरिति । ‚{\tiny $_{lb}$}‚यद्येवं न तर्हि तत्प्रयोगमन्तरेण साध्यसिद्धेरभाव इति व्यर्थ एव तत्प्रयोगः स्यात् ‚{\tiny $_{6}$}‚ ‚{\tiny $_{lb}$}‚अन्यथा कः पक्षवचनं साधनादपाकर्त्तुं समर्थः । ततश्च ‚{\color{DodgerBlue3}‚त्रिरूपलिङ्गा} ख्यानं ‚{\color{DodgerBlue3}‚परार्थ} ‚{\tiny $_{lb}$}‚मनुमानमित्याद्याचार्यवचो व्याहन्येत । कथं तर्ह्युक्तं ।
	{\color{gray}{\rmlatinfont\textsuperscript{§~\theparCount}}}
	\pend% ending standard par
      ‚{\tiny $_{lb}$}‚

	  
	  \pstart \leavevmode% starting standard par
	पक्षधर्मत्वसम्बन्धसाध्योक्तेरन्यवर्जनमिति नास्ति विरोधः । पक्षधर्मत्वसंबन्धा‚{\tiny $_{lb}$}‚भ्यां साध्यस्योक्तिप्रकास\edtext{}{\lemma{साध्यस्योक्तिप्रकास}\Bfootnote{? प्रकाश}}नमाक्षेपस्तस्मादन्येषां पक्षोपनयवचनादीना‚{\tiny $_{lb}$}‚मुपादेयत्वेन साधनवाक्यवर्जनमि ‚{\tiny $_{8}$}‚ \leavevmode\ledsidenote{\textenglish{47a/msK}} ति व्याख्यानात् । विवरणेप्ययमर्थो यस्मात् ‚{\tiny $_{lb}$}‚पक्षधर्मत्वसम्बन्धवचनमेवान्वयव्यतिरेकाभ्याम्विवक्षितार्थसिद्धिकारणं युक्तं नान्यत् । ‚{\tiny $_{lb}$}‚तस्मादनुमेयस्योपदर्शनार्थ ‚{\tiny $_{9}$}‚ सिद्ध्यर्थं पक्षवचनमुपादेयं नान्यदित्युपस्कारः । पक्ष ‚{\tiny $_{lb}$}‚उच्यते आक्षिप्यते प्रकाश्यते अनेनेति पक्षवचनन्त्रिरूपं लिङ्गं । आक्षेपो ह्यभि‚{\tiny $_{lb}$}‚धानतुल्य इति वचनमित्युक्तं वचेर ‚{\tiny $_{1}$}‚ नेकार्थत्वाद्वा । अस्माकं तु \add{।}
	{\color{gray}{\rmlatinfont\textsuperscript{§~\theparCount}}}
	\pend% ending standard par
      ‚{\tiny $_{lb}$}‚
	  \bigskip
	  \begingroup
	
	    
	    \stanza[\smallbreak]
	  \flagstanza{\tiny\textenglish{...19}}{\normalfontlatin\large ``\qquad}तत्रानुमेयनिर्देशो हेत्वर्थविषयो मत \add{१९}{\normalfontlatin\large\qquad{}"}\&[\smallbreak]
	  
	  
	  
	  \endgroup
	‚{\tiny $_{lb}$}‚

	  
	  \pstart \leavevmode% starting standard par
	इत्यपि वचनं विरुध्यते । यस्मा
	{\color{gray}{\rmlatinfont\textsuperscript{§~\theparCount}}}
	\pend% ending standard par
      ‚{\tiny $_{lb}$}‚
	  \bigskip
	  \begingroup
	
	    
	    \stanza[\smallbreak]
	  \flagstanza{\tiny\textenglish{...20}}{\normalfontlatin\large ``\qquad}तत्रेति तर्कशास्त्रस्य सम्बन्धोत्राभिधीयते ।&‚{\tiny $_{lb}$}‚प्रयोगस्य तु सम्बन्धे बहु स्यादसमं ‚{\tiny $_{2}$}‚ जसं ॥ \add{२०}{\normalfontlatin\large\qquad{}"}\&[\smallbreak]
	  
	  
	  
	  \endgroup
	‚{\tiny $_{lb}$}‚
	  \bigskip
	  \begingroup
	
	    
	    \stanza[\smallbreak]
	  \flagstanza{\tiny\textenglish{...21}}{\normalfontlatin\large ``\qquad}तस्यैव प्रकृतेरुक्तमेतच्चास्यैव लक्षणे ।&‚{\tiny $_{lb}$}‚परविप्रतिपत्तीनान्निषेधाय विशेषत \add{॥ २१}{\normalfontlatin\large\qquad{}"}\&[\smallbreak]
	  
	  
	  
	  \endgroup
	‚{\tiny $_{lb}$}‚

	  
	  \pstart \leavevmode% starting standard par
	इत्यलं प्रसङ्गेन ॥ ० ॥
	{\color{gray}{\rmlatinfont\textsuperscript{§~\theparCount}}}
	\pend% ending standard par
      ‚{\tiny $_{lb}$}‚

	  
	  \pstart \leavevmode% starting standard par
	तद्भावरूपं साधनमङ्गन्धर्मो विषयित्वेन । यस्यार्थस्य ‚{\tiny $_{3}$}‚ प्रस्तुतस्य स साध‚{\tiny $_{lb}$}‚ \leavevmode\ledsidenote{\textenglish{67/s}} नाङ्गस्तस्यैवाभिव्यक्तिरुत्तरेण पदद्वयेन ॥ अजिज्ञासितं प्रतिवादिनाऽशास्त्राश्रय‚{\tiny $_{lb}$}‚व्याजादिभिरित्यादिपदेनासम्बद्धप्रसङ्गपरिग्रहः । प्रक्षे ‚{\tiny $_{4}$}‚ पो नाममात्रेण घोषणं ‚{\tiny $_{lb}$}‚विस्तरेण । यथा बुद्धीन्द्रियदेहकलापव्यतिरेकात्मास्ति नास्तीत्येतावत् मात्रे ‚{\tiny $_{lb}$}‚वुभुस्तिते नैयायिकाः प्रमाणयन्ति । सदाद्यविशेषवि ‚{\tiny $_{5}$}‚ षया विषयज्ञेयविषया मदीयाः ‚{\tiny $_{lb}$}‚प्रत्यक्षादयः प्रत्यया मदीयशरीरादिव्यतिरिक्तसम्वेदकसम्वेद्याः स्वकारणायत्त‚{\tiny $_{lb}$}‚जन्मवत्वादिभ्यः पुरुषान्तर ‚{\tiny $_{6}$}‚ प्रत्ययवदिति ततः सदनित्यन्द्रव्यवत् कार्यकारणं ‚{\tiny $_{lb}$}‚सामान्यविशेषवदिति द्रव्यगुणकर्मणामविशेष इति महता व्यासेन सदाद्यविशेषाद् ‚{\tiny $_{lb}$}‚व्याचक्षते । नह्यत्र स ‚{\tiny $_{7}$}‚ दाद्यविशेषविषया विषयज्ञेयविषयत्वन्धर्मविशेषणं कथं‚{\tiny $_{lb}$}‚चिदपि प्रकृतसाध्यसिध्युपकारि । परव्यामोहनानुभाषणशक्तिविघातादिहेतो‚{\tiny $_{lb}$}‚रित्यत्रादिश ‚{\tiny $_{8}$}‚ ब्देनोत्तरप्रतिपत्तिशक्तिविघातहेतोः परिग्रहः क्रियमाणः प्रसङ्गो‚{\tiny $_{lb}$}‚ \leavevmode\ledsidenote{\textenglish{68/s}} यस्येति विग्रहः । नैरात्म्यवाद्युदाहरणेन किं ज्ञापयति । ‚{\color{DodgerBlue3}‚यत्र नाम} विहितप्रतिसिद्धो\edtext{}{\lemma{विहितप्रतिसिद्धो}\Bfootnote{? प्रतिषिद्धो }}वा ‚{\tiny $_{9}$}‚ \leavevmode\ledsidenote{\textenglish{47b/msK}} दिदोषगुणसौगतधर्मविनयस्याप्यहङ्कारनिमित्तसकलोद्ध ‚{\tiny $_{lb}$}‚वादिमलक्षालनायोद्यतमत्रैव नात्मवादिनस्तत्साधने नृत्यगीतादेः प्रसङ्गः । ‚{\tiny $_{lb}$}‚तत्रान्येषामन्यस्य च का ग ‚{\tiny $_{1}$}‚ णना । ननु च वयं ‚{\color{DodgerBlue3}‚बौद्धा} ब्रूम इति कथं यावता ‚{\tiny $_{lb}$}‚सविशेषणस्य प्रतिषेधाभिधानात् । अहम्बौद्धो ब्रवीमीति भवितव्यं । यथाहंगार्गो\edtext{}{\lemma{यथाहंगार्गो}\Bfootnote{‚{\tiny $_{lb}$}‚? गार्ग्यो}}ब्रवीम्यहं पटु ब्रवीमि इति न च बहु ‚{\tiny $_{2}$}‚ ष्वेवेतद्वहुवचनमिति\href{http://sarit.indology.info/?cref=P\%C4\%81.2.4.21}{पाणिनिः  २।४।२१} शक्यमभिधातुँ कश्चिदिति वचनात् । नैव यस्मादसावात्मनि परान् ‚{\tiny $_{lb}$}‚स्वयूथ्यानप्यन्यान्बहूनपेक्ष्य तथा प्रयुक्तवान् । ईदृश्यामेव च वादिनो विवक्षा ‚{\tiny $_{3}$}‚ यामिद‚{\tiny $_{lb}$}‚मुक्तमुदाहरणं नान्यस्यामिति प्रतिपत्तव्यं \add{।} अथवा जडशाब्दिकाभिनिवेशनिवा‚{\tiny $_{lb}$}‚रणायेदमेवमुक्तं तथा च व्यर्थता शब्दानुसासन\edtext{}{\lemma{शब्दानुसासन}\Bfootnote{? नुशासन}}स्य प्रतिपादयि ‚{\tiny $_{4}$}‚ ष्यति । ‚{\tiny $_{lb}$}‚अत एवान्येन महारथेनापीदं प्रयुक्तं ॥ ‚{\tiny $_{lb}$}‚ 
	    \pend% close preceding par
	  
	    
	    \stanza[\smallbreak]
	  \flagstanza{\tiny\textenglish{...22}}{\normalfontlatin\large ``\qquad}त्वं राजा वयमप्युपासितगुरुप्रज्ञाभिमानोन्नता । \add{२२}{\normalfontlatin\large\qquad{}"}\&[\smallbreak]
	  
	  
	  
	    \pstart  \leavevmode% new par for following
	    \hphantom{.}
	  इति ।
	{\color{gray}{\rmlatinfont\textsuperscript{§~\theparCount}}}
	\pend% ending standard par
      ‚{\tiny $_{lb}$}‚

	  
	  \pstart \leavevmode% starting standard par
	\hphantom{.}सभ्यः साधुसंमतानामित्युपहसति । ‚{\color{DodgerBlue3}‚अहो} ‚{\tiny $_{5}$}‚ शब्दश्चेहाध्याह्रियते । द्वादशा‚{\tiny $_{lb}$}‚नाम्प्रमाणादिलक्षणानां यः प्रपञ्चो विस्तरस्तस्य प्रकाशनाय यच्छास्त्रं मीमांसा‚{\tiny $_{lb}$}‚ख्यं तस्य प्रणेता स चासौ जैमिनिश्च तेन प्र ‚{\tiny $_{6}$}‚ तिज्ञातं यत्तत्वं नित्यताभिधानं । ‚{\tiny $_{lb}$}‚ \leavevmode\ledsidenote{\textenglish{69/s}} तस्याधिकरणं यः शब्दः स च घटश्च तयोरन्यतरस्तेन स द्वितीयो घट इतीत्थं ‚{\tiny $_{lb}$}‚प्रतिज्ञामुपरचय्य द्वादशलक्षणादिव्याख्यानङ्करो ‚{\tiny $_{7}$}‚ ति । प्रमाणलक्षणमेव तावदेकं ‚{\tiny $_{lb}$}‚महता कालेन व्याचष्टे । चोदनालक्षणो धर्म \href{http://sarit.indology.info/?cref=M\%C4\%ABS\%C5\%AB.1.1.2}{मीमांसा सू० १।१।२} श्चोद‚{\tiny $_{lb}$}‚नेति क्रियायाः प्रवर्तकम्वचनमाहुश् ‚{\color{DodgerBlue3}‚चोदना हि भूतं भवन्तं भविष्यन्तं सूक्ष्मं व्यव ‚{\tiny $_{lb}$}‚हि ‚{\tiny $_{8}$}‚ \leavevmode\ledsidenote{\textenglish{48a/msK}} तं विप्रकृष्ट\add{...} मर्थं शक्नोत्यवगमयितुं नान्यत् किञ्चनेन्द्रियं}\edtext{}{\lemma{प्रवर्तकम्वचनमाहुश्}\Bfootnote{मीमांसाशबरभाष्ये १।१।२}} \add{।} तथाहि ‚{\tiny $_{lb}$}‚ \add{सत्} संप्रयोगे पुरुषस्येन्द्रियाणाम्बुद्धिजन्म तत्प्रत्यक्षं । अनिमित्तं विद्यमानोप‚{\tiny $_{lb}$}‚लम्भनत्वादि \href{http://sarit.indology.info/?cref=M\%C4\%ABS\%C5\%AB.1.1.4}{मीमांसा सू० १।१।४ } त्यादिना । संस्कारदुःखतासिद्धिमन्तरेण ‚{\tiny $_{lb}$}‚नानित्यतासिद्धिरप्रतीत्य समुत्पन्नस्य क्षणिकत्वायोगात् । स तर्हि तादृशो धर्मः ‚{\tiny $_{lb}$}‚पृथग्वाच्यो नेत्याह । ‚{\color{DodgerBlue3}‚तथाविधस्त्वित्या} ‚{\tiny $_{1}$}‚ दि । एवंविधस्यापि प्रस्तुतसाध्यधर्म‚{\tiny $_{lb}$}‚नान्तरीयकस्य प्रतिवादिनाऽजिज्ञासितस्य तद्व्यतिरेकेण प्रतिज्ञायामन्यत्र चाहेतु‚{\tiny $_{lb}$}‚दृष्टान्तयोः कदा पुनरेतदसाधना ‚{\tiny $_{2}$}‚ ङ्गवचनं यथोक्तं निग्रहस्थानमित्याह । ‚{\color{DodgerBlue3}‚प्रतिवादिना} ‚{\tiny $_{lb}$}‚तथाभावेऽसाधनाङ्गत्वे प्रतिपादिते सति । यदा तु न प्रतिपादयति तदा द्वयोरेकस्यापि ‚{\tiny $_{lb}$}‚न जयपराजयौ भवत\add{ः} ‚{\tiny $_{3}$}‚ कुतः साधनानभिधानान्न वादिनो जयः । प्रतिवादिना ‚{\tiny $_{lb}$}‚तथाभावस्याप्रतिपादितत्वाच्च पराजयोपि नास्त्येव । तस्य प्रतिपन्नापेक्षत्वात् । ‚{\tiny $_{lb}$}‚अत एव प्र ‚{\tiny $_{4}$}‚ तिवादिन्यपि तयोरभावः ॥ ४ ॥
	{\color{gray}{\rmlatinfont\textsuperscript{§~\theparCount}}}
	\pend% ending standard par
      ‚{\tiny $_{lb}$}‚

	  
	  \pstart \leavevmode% starting standard par
	\hphantom{.}सम्प्रति प्रतिवादिनो निग्रहस्थानमधिकृत्याह । ‚{\color{DodgerBlue3}‚अदोषोदूभावनमित्यादि} ‚{\tiny $_{7b6}$}‚ । ‚{\tiny $_{lb}$}‚यत्र विषये जिज्ञासिते अजिज्ञासिते ‚{\tiny $_{5}$}‚ पुनर्दोषस्यानुद्भावनेपि नापराध इत्यभिप्रायः । ‚{\tiny $_{lb}$}‚के पुनस्ते साधनस्य दोषा इत्याह । न्यूनत्वं षट्प्रकारमेकैकद्विद्विरूपानुक्तौ \add{।} ‚{\tiny $_{lb}$}‚ स्यान्मतिः सपक्ष ‚{\tiny $_{6}$}‚ विपक्षयोः सदसत्त्वयोर्यौगपद्येनाप्रयोगे कथञ्च प्रकारात् न्यूनतोच्यते ‚{\tiny $_{lb}$}‚ \add{।} यदा सर्व्वोपसंहारेण व्याप्तिव्यतिरेकाभ्यान्तदाक्षेपोपि नास्ति तदेयं व्यवस्थाप्यते । ‚{\tiny $_{lb}$}‚ \leavevmode\ledsidenote{\textenglish{70/s}} अ ‚{\tiny $_{7}$}‚ थोच्यते तदाप्यप्रदर्शितान्वयव्यतिरेकादिदृष्टान्तदोषो भवति । भवत्व‚{\tiny $_{lb}$}‚यमपरोस्यापराधो न ह्येकदोषालीढान्येव साधनानि भवन्ति त्रयो हेत्वाभासा दृष्टा‚{\tiny $_{lb}$}‚न्ता ‚{\tiny $_{8}$}‚ \leavevmode\ledsidenote{\textenglish{48b/msK}} भासाश्चाष्टादश ‚{\color{DodgerBlue3}‚न्यायविन्दौ} \add{तृतीये परिच्छेदे} सोदाहरणा\add{ः} प्रपञ्चेन ‚{\tiny $_{lb}$}‚द्रष्टव्याः । तेषामनुद्भावनं पर्यायशब्दद्वयेन व्याचष्टे । तच्चानुद्भावनं त्रिभिः ‚{\tiny $_{lb}$}‚कारणैरित्याह । ततः पुनः ‚{\color{DodgerBlue3}‚‚{\tiny $_{9}$}‚ साधनस्य निर्दोषत्वादित्यादि} ।
	{\color{gray}{\rmlatinfont\textsuperscript{§~\theparCount}}}
	\pend% ending standard par
      ‚{\tiny $_{lb}$}‚

	  
	  \pstart \leavevmode% starting standard par
	ननु च युक्तो निर्दोषे साधने प्रतिवादिनो दोषानुद्भावनान्निग्रहः । ‚{\tiny $_{lb}$}‚सदोषे त्वज्ञानासामर्थ्याभ्यामनुद्भावनेपि दोषस्य दुष्टसाधनप्रयो ‚{\tiny $_{1}$}‚ गाद्वादिन एव ‚{\tiny $_{lb}$}‚पराजयो युक्तो न प्रतिवादिन इति । अत्राह । ‚{\color{DodgerBlue3}‚न हि दुष्टसाधनाभिधानेपीति} ‚{\tiny $_{7b8}$}‚ । ‚{\tiny $_{lb}$}‚यद्येवं दुष्टेनापि साधनेन वादिना प्रतिवादिनस्तिरस्कृतत्वात् कस्माज्ज ‚{\tiny $_{2}$}‚ यो ‚{\tiny $_{lb}$}‚न भवति तस्येत्याह । ‚{\color{DodgerBlue3}‚केवलमित्यादि} ‚{\tiny $_{7b9}$}‚ । यद्येवं किन्न पराजयः । तत्वसिद्धि‚{\tiny $_{lb}$}‚भ्रंशादिति चोद्यं । नानिराकरणादित्याद्युत्तरं । दुर्जनानाम्विप्रतिपत्तिरशोभनो ‚{\tiny $_{lb}$}‚व्य ‚{\tiny $_{3}$}‚ वहारः तस्मान्न योगविहितो न्याय्यः कश्चिद्विजगीषुवादो नाम यच्छला‚{\tiny $_{lb}$}‚दिभिः क्रियत इत्यध्याहारः । उक्ते सति न्याये तत्वार्थी चेत् प्रतिवादी प्र ‚{\tiny $_{4}$}‚ तिपद्येत ‚{\tiny $_{lb}$}‚तमर्थं न्यायोपेतं । अथ स्वपक्षरागस्य वलीयस्त्वादुक्तेपि न्याये न प्रतिपद्येत । तदा ‚{\tiny $_{lb}$}‚तेन प्रतिवादिना तस्य न्यायस्यार्थस्याप्रतिपत्तावन्य ‚{\tiny $_{5}$}‚ समीपवर्त्त्यात्मज्ञो जनकायो न ‚{\tiny $_{lb}$}‚विप्रतिपद्येतेति कृत्वा न्यायानुसरणमेव सतां वाद इति वर्त्तते । तत्वरक्षणार्थमिति‚{\tiny $_{lb}$}‚परः । यथोक्तं तत्वा\add{ध्य}वसाय ‚{\tiny $_{6}$}‚ संरक्षणार्थञ्जल्पवितण्डे बीजप्ररोहसंरक्षणार्थं ‚{\tiny $_{lb}$}‚कण्टकशाखावरणवदिति \href{http://sarit.indology.info/?cref=ns\%C5\%AB.4.2.50}{न्या० सू० ४।२।५०} । नेत्याद्याचार्यः । एवन्तत्वं ‚{\tiny $_{lb}$}‚सुरक्षितम्भवति । एकान्तेन प्रतिद्वन्द्युन्मूलनादिति भाव\add{ः ।} ‚{\tiny $_{7}$}‚ तदभाव इति साधन‚{\tiny $_{lb}$}‚प्रख्यापनसाधनाभासदूषणयोरभावे । अन्यथापीति मिथ्याप्रलापाद्यभावेपि ॥ ० ॥
	{\color{gray}{\rmlatinfont\textsuperscript{§~\theparCount}}}
	\pend% ending standard par
      ‚{\tiny $_{lb}$}‚

	  
	  \pstart \leavevmode% starting standard par
	\hphantom{.}कथमसौ न दोषः साधनस्येत्याह । ‚{\color{DodgerBlue3}‚तस्य दोषत्वे} ‚{\tiny $_{8}$}‚ नाभिमतस्य भावेपि सिद्धेर्वि‚{\tiny $_{lb}$}‚\leavevmode\ledsidenote{\textenglish{71/s}} धाताभावात् । साधयितुमनिष्टोप्याकाशगुणत्वादिकार्यत्वेनानित्यत्वमात्रसाधने ‚{\tiny $_{lb}$}‚ध्वनौ विवक्षिते सति ‚{\color{DodgerBlue3}‚काणादाः} केचिच्चोदयन्ति ‚{\tiny $_{9}$}‚ \leavevmode\ledsidenote{\textenglish{49a/msK}} न्यायानभिज्ञाः । शास्त्रोपगमात् ‚{\tiny $_{lb}$}‚सर्व्वस्तदिष्टः साध्यः । तत्प्रधाने च हेतुप्रतिज्ञयोर्दोष इति तच्चायुज्यमानं शास्त्रा‚{\tiny $_{lb}$}‚श्रयेप्यस्त्यपगतमात्रस्यैव साध्यत्वात् । अन्यथा गन्धे भूगुणताविपर्ययसाधनादय‚{\tiny $_{lb}$}‚मेव हेतुरस्यामेव प्रतिज्ञायां विरुद्धः प्राप्नोति \add{।} तथेदमपरमदोषोद्भावनं । ‚{\tiny $_{lb}$}‚यथाह ‚{\color{DodgerBlue3}‚भारद्वाजो नास्त्यात्मेति प्रतिज्ञापदयोः} प ‚{\tiny $_{2}$}‚ रस्परविरोध इति । यस्मादात्मेति ‚{\tiny $_{lb}$}‚वस्त्वभिधीयते नास्तीति तस्य प्रतिषेधः । इदमप्ययुक्तमनादिवासनोद्भूतात्मविक‚{\tiny $_{lb}$}‚ल्पपरिनिष्ठितप्रतिभासभेदस्य श ‚{\tiny $_{3}$}‚ ब्दार्थस्य परेष्टानित्यचित्तत्वादिविशेषणात्म‚{\tiny $_{lb}$}‚\leavevmode\ledsidenote{\textenglish{72/s}} लक्षणभावोपादानत्वस्य निराचिकीर्षितत्वात् । अत्रैव हि धर्मिणि व्यवस्थिताः ‚{\tiny $_{lb}$}‚सदसत्वञ्चिन्तयन्ति सन्त ‚{\tiny $_{4}$}‚ \add{ः} किमयमात्मविकल्पप्रतिभास्यर्थो यथाभिमत‚{\tiny $_{lb}$}‚भावोपादानो न वेति । न तु पुनरत्रायमेव विकल्पप्रतिभास्येवार्थोऽपह्नूयते तस्यैव ‚{\tiny $_{lb}$}‚बुद्धावुपस्थाप ‚{\tiny $_{5}$}‚ नाय शब्दप्रयोगात् प्रत्यात्मवेद्यत्वाच्च । विकल्पप्रतिबिम्बव्यतिरिक्तं ‚{\tiny $_{lb}$}‚तु बाह्यं स्वलक्षणं नैव शब्दार्थ इति न तस्य विधिर्नापि प्रतिषेधणं\edtext{}{\lemma{प्रतिषेधणं}\Bfootnote{? नं}}। ‚{\tiny $_{lb}$}‚अन्यथा
	{\color{gray}{\rmlatinfont\textsuperscript{§~\theparCount}}}
	\pend% ending standard par
      ‚{\tiny $_{lb}$}‚
	  \bigskip
	  \begingroup
	
	    
	    \stanza[\smallbreak]
	  \flagstanza{\tiny\textenglish{...23}}{\normalfontlatin\large ``\qquad}परमा चै ‚{\tiny $_{6}$}‚ कतानत्वे शब्दानामनिबन्धना \add{।}&‚{\tiny $_{lb}$}‚न स्यात् प्रवृतिरथेषु दर्शनान्तरभेदिषु ॥ \add{२३}{\normalfontlatin\large\qquad{}"}\&[\smallbreak]
	  
	  
	  
	  \endgroup
	‚{\tiny $_{lb}$}‚
	  \bigskip
	  \begingroup
	
	    
	    \stanza[\smallbreak]
	  \flagstanza{\tiny\textenglish{...24}}{\normalfontlatin\large ``\qquad}अतीताजातयोर्वापि न च स्याद नृतार्थता ।&‚{\tiny $_{lb}$}‚वाचः कस्याश्चिदित्येषा बौद्धार्थविषया मता ‚{\tiny $_{7}$}‚ ॥ \add{२४}{\normalfontlatin\large\qquad{}"}\&[\smallbreak]
	  
	  
	  
	  \endgroup
	‚{\tiny $_{lb}$}‚\textsuperscript{\textenglish{73/s}}

	  
	  \pstart \leavevmode% starting standard par
	स चायम्विकल्पो भावोपादानत्वेन निराचीकीर्षितो देशकालप्रतिनियतिमन‚{\tiny $_{lb}$}‚पेक्ष्य विकल्पप्रतिबिंबविषयत्वादेव चात्मशब्दस्य न निर्विषयत्वमस्ति । ततश्च ‚{\tiny $_{lb}$}‚यदु ‚{\tiny $_{8}$}‚ क्तं यच्च यत्र प्रतिषिद्ध्यते तत् तस्मादन्यत्रास्ति । यथा नास्ति नासमाना‚{\tiny $_{lb}$}‚धिकरणो घटशब्दो न घटाभावं प्रतिपादयितुँ शक्नोति । अपि तु देशकालविशेषात् ‚{\tiny $_{lb}$}‚प्रतिषेधाग ‚{\tiny $_{9}$}‚ \leavevmode\ledsidenote{\textenglish{49b/msK}} ति\add{ः} । नास्ति घट इति देशविशेषे प्रतिषेधो गेहे नास्ति इति । कालविशेषे ‚{\tiny $_{lb}$}‚वा प्रतिषेधः । इदानीं नास्ति । प्राग्नास्ति । ऊर्ध्वं नास्ति । सर्व्वस्यायं प्रतिषेधो ‚{\tiny $_{lb}$}‚नाननुभूतघटसत्व ‚{\tiny $_{1}$}‚ स्य युक्तः । तथा नास्त्यात्मेति किमयन्देशविशेषः प्रतिषिध्यते । ‚{\tiny $_{lb}$}‚उत्तरकालविशेष इति । यदि तावद्देशविशेषप्रतिषेधः । स आत्मनि न युक्तोऽदेश‚{\tiny $_{lb}$}‚त्वादात्मनः । न च देशविशेषप्रतिषेधादात्मा प्रतिषिद्धो भवति । न चायम्भवता‚{\tiny $_{lb}$}‚मभिप्रायः । शरीरमात्मा न भवतीति चेत् । कस्य वा शरीरमात्मा यं प्रति प्रतिषेधः । ‚{\tiny $_{lb}$}‚शरीरे नास्त्या ‚{\tiny $_{3}$}‚ त्मेत्येवं प्रतिषेध इति चेत् । कस्य शरीरे आत्मा यं प्रति प्रतिषेधः । ‚{\tiny $_{lb}$}‚क्व तर्ह्यात्मा । न क्वचिदात्मा । किमयं नास्त्येव । न नास्ति विशेषप्रतिषेधात् । ‚{\tiny $_{lb}$}‚केयं वाचो युक्ति ‚{\tiny $_{4}$}‚ र्न्न शरीरे नान्यत्र । न च नास्ति । एषैवेषा वाचो युक्तिः । यद्यथा ‚{\tiny $_{lb}$}‚भूतन्तत्तथा निर्दिश्यत इति न चायमात्मा क्वचिदपीति । तस्मात्तथैव निर्देशः । न च ‚{\tiny $_{lb}$}‚कालविशेषे ‚{\tiny $_{5}$}‚ प्रतिषेधो युक्तः । आत्मनि त्रैकाल्यस्यानभिव्यक्तेरात्मप्रतिषेधञ्च ‚{\tiny $_{lb}$}‚कुर्वाणेनात्मशब्दस्य विषयो वक्तव्यः । न ह्येकं पदं निरर्थकं पश्यामः ॥ अथापि ‚{\tiny $_{lb}$}‚शरी ‚{\tiny $_{6}$}‚ रादिषु आत्मशब्दं प्रतिपद्येथाः । एवमप्यनिवृत्तौ व्याघातः कथमिति । नास्त्या‚{\tiny $_{lb}$}‚त्मेत्यस्य वाक्यस्य तदानीमयमर्थो भवति शरीरादयो न सन्तीति । एवमादि बह्व‚{\tiny $_{lb}$}‚सं ‚{\tiny $_{7}$}‚ बद्धं तदपहस्तितम्भवति । प्रतिज्ञार्थैकदेश इत्येतदप्यसत् सामान्यविशेषस्याभा‚{\tiny $_{lb}$}‚वात् । यद्वा न प्रयत्नानन्तरीयकत्वस्य प्रतिज्ञार्थैकदेशता धर्मिणमुपलक्ष्य निवृत्त‚{\tiny $_{lb}$}‚त्वात् ‚{\tiny $_{8}$}‚ \leavevmode\ledsidenote{\textenglish{50a/msK}} \add{।} यस्य हि यदुपलक्षकं न तस्य तदेकदेशत्वं यथा न काकस्य गृद्धैकदेश- ‚{\tiny $_{lb}$}‚त्वमिति ॥
	{\color{gray}{\rmlatinfont\textsuperscript{§~\theparCount}}}
	\pend% ending standard par
      
	    
	    \endnumbering% ending numbering from div
	    
	  \textsuperscript{\textenglish{74/s}}
	  
	% new div opening: depth here is 0
	
	    
	    \beginnumbering% beginning numbering from div depth=0
	    
	  
\chapter[{२. न्यायमतखंडनम्}][{२. न्यायमतखंडनम्}]{२. न्यायमतखंडनम्}\textsuperscript{\textenglish{75/s}}

	  
	  \pstart \leavevmode% starting standard par
	अन्यत्तु न युक्तमिति ‚{\tiny $_{8b5}$}‚ यदुक्तमक्षपादेन द्वाविंशतिविधं निग्रहस्थानं । ‚{\tiny $_{lb}$}‚ प्रतिज्ञाहानिः । प्रतिज्ञान्तरं । प्रतिज्ञाविरोधः । प्रतिज्ञासंन्यासो हेत्वन्तरमर्थान्तर‚{\tiny $_{lb}$}‚न्निरर्थकमविज्ञातार्थमपार्थकमप्राप्तकालं न्यून मधिकं पुनरुक्तमननुभाषणमज्ञा‚{\tiny $_{lb}$}‚नमप्रतिभा विक्षेपो मतानुज्ञा पर्यनुयोज्योपेक्षणन्निरनुयोज्यानुयोगोपसिद्धान्तो ‚{\tiny $_{lb}$}‚हेत्वाभासाश्च निग्रहस्थानानि \href{http://sarit.indology.info/?cref=ns\%C5\%AB.5.2.1}{न्या० सू० ५।२।१} । तानीमानि द्वाविंशति‚{\tiny $_{lb}$}‚विधानि विभज्य वक्ष्यन्ते\edtext{\textsuperscript{*}}{\lemma{*}\Bfootnote{\href{http://sarit.indology.info/?cref=nbh.5.2.1}{न्यायवात्स्यायनभाष्ये ५।२।१} ।}}। प्रतिदृ‚{\tiny $_{2}$}‚ष्टान्तधर्मानुज्ञा स्वदृष्टान्ते प्रतिज्ञाहानिः । ‚{\tiny $_{lb}$}‚ \href{http://sarit.indology.info/?cref=ns\%C5\%AB.5.2.}{न्या० सू० ५।२। } तत्र भाष्यकारमतं दूषयित्वा वार्त्तिककारोयं स्थितपक्षमाह । ‚{\tiny $_{lb}$}‚तमेव ब्रूम इति । भाष्यकारमतस्य ‚{\color{DodgerBlue3}‚भारद्वाजे}‚नैव दूषि‚{\tiny $_{3}$}‚ तत्वादस्माकमर्द्धन्तावदवसितं ‚{\tiny $_{lb}$}‚भारस्येति भावः । तत्रेदम्भाष्यकारस्य मतं । साध्यधर्मप्रत्यनीकेनधर्मेण ‚{\tiny $_{lb}$}‚प्रत्यवस्थितः प्रतिदृष्टान्तधर्म स्वदृष्टां ‚{\tiny $_{4}$}‚ तेनुजानन् प्रतिज्ञां जहातीति प्रतिज्ञाहानिः । ‚{\tiny $_{lb}$}‚निदर्शनमनित्यः शब्द ऐन्द्रियकत्वात् घटवदिति कृते पर आह । दृष्टमैन्द्रियकं ‚{\tiny $_{lb}$}‚सामान्यं नित्यङ्कस्मात् ‚{\tiny $_{5}$}‚ न तथा शब्द इति प्रत्यवस्थित इदमाह यद्यैन्द्रियकं ‚{\tiny $_{lb}$}‚सामान्यं कामं घटोपि नित्योस्त्विति । स खल्वयं साधनस्य दृष्टान्तस्य नित्यत्वं ‚{\tiny $_{lb}$}‚प्रसञ्जयन्निगमना ‚{\tiny $_{6}$}‚ नन्त\add{र}मेव पक्षञ्जहाति पक्षञ्च जहतः प्रतिज्ञाहानिरित्युच्यते । ‚{\tiny $_{lb}$}‚प्रतिज्ञाश्रयत्वात् पक्षस्येति ।\edtext{\textsuperscript{*}}{\lemma{*}\Bfootnote{तत्रैव \href{http://sarit.indology.info/?cref=nbh.5.2.12}{५।२।२} ।}} ‚{\color{DodgerBlue3}‚वार्तिककारेण} चैवमेतद् दूषितं । \edtext{\textsuperscript{*}}{\lemma{*}\Bfootnote{\href{http://sarit.indology.info/?cref=nv.5.2.2}{न्यायवात्तिंके ५।२।२} ।}}एतत्तु न बुद्ध्या‚{\tiny $_{lb}$}‚महे कथमत्र प्रति ‚{\tiny $_{7}$}‚ ज्ञा हीयत इति हेतोरनैकान्तिकत्वं सामान्यदृष्टान्तेन परेण चोद्यते । ‚{\tiny $_{lb}$}‚ \leavevmode\ledsidenote{\textenglish{76/s}} तस्यानैकान्तिकदोषोद्धारमनुक्त्वा स्वदृष्टान्ते नित्यतां प्रतिपद्यते । नित्यताप्रतिपत्ते‚{\tiny $_{lb}$}‚श्चा ‚{\tiny $_{8}$}‚ सिद्धतादृष्टान्तदोषो भवतीति सोयं दृष्टान्तदोषेण वा हेतुदोषेण वा निग्रहो ‚{\tiny $_{lb}$}‚न प्रतिज्ञाहानिरिति । दृष्टान्तञ्च जहत् प्रतिज्ञाञ्जहातीति उपचारेण निग्रहस्थानं । ‚{\tiny $_{9}$}‚ \leavevmode\ledsidenote{\textenglish{50b/msK}} ‚{\tiny $_{lb}$}‚न च प्रधानासम्भवे उपचारो लभ्यत इति प्रतिज्ञाहानेर्मुख्यो विषयो वक्तव्य इति । ‚{\tiny $_{lb}$}‚इदानीम्वार्तिककारमतं स्वयमेवोपन्यस्यति । ‚{\color{DodgerBlue3}‚प्रतिदृष्टान्तस्येत्यादिना} ‚{\tiny $_{8b6}$}‚ कः ‚{\tiny $_{lb}$}‚पुनर ‚{\tiny $_{1}$}‚ त्र दृष्टान्तोऽभिमतो यदि तावत् यत्र ‚{\tiny $_{1}$}‚ लौकिकपरीक्षकानो\edtext{}{\lemma{लौकिकपरीक्षकानो}\Bfootnote{? णां}}बुद्धि‚{\tiny $_{lb}$}‚साम्यं स दुष्टान्त \href{http://sarit.indology.info/?cref=ns\%C5\%AB.1.1.25}{न्या० सू० १।१।२५} इति पारिभाषिकस्तदा भाष्यकारमताद‚{\tiny $_{lb}$}‚विषेशस्तत्र च प्रतिविहितं । अथान्यः स न ग ‚{\tiny $_{2}$}‚ म्यत इत्याह । तत्र दृष्टश्चासौ पञ्चाव‚{\tiny $_{lb}$}‚यवेन साधनेनान्ते च निगमनस्य व्यवस्थित इति दृष्टान्तः पक्षः । ततः स्वशब्देन सह ‚{\tiny $_{lb}$}‚विशेषणसमासः । तद्विपरीतः प्रतिदृष्टान्त ‚{\tiny $_{3}$}‚ \add{ः} । यथाऽनित्यः शब्दः ऐन्द्रियकत्वादिति ‚{\tiny $_{lb}$}‚ब्रुवन्वादी प्रतिपक्षवादिनि सामान्यादिकमैन्द्रियकं नित्यं च । ततोविपक्षेपि वृत्तेर्व्य‚{\tiny $_{lb}$}‚भिचार्ययं हेतुरित्येवं सामा ‚{\tiny $_{4}$}‚ न्येन प्रत्यवस्थिते सत्याह यद्येवं शब्दोप्येवमस्त्विति ‚{\tiny $_{lb}$}‚एषा प्रतिज्ञाहानिर्नाम निग्रहस्थानं । कस्मात् । प्राग्प्रतिज्ञातस्य शब्दानित्यत्वस्य ‚{\tiny $_{lb}$}‚त्यागात् । प्रतिज्ञाश ‚{\tiny $_{5}$}‚ ब्देन धर्मिविशेषणभूतो धर्म उच्यते समुदायावयवत्वात् । एतत् ‚{\tiny $_{lb}$}‚प्रतिक्षिपति । अत्र ‚{\color{DodgerBlue3}‚भारद्वाज} मते उपगतायाः प्रतिज्ञायास्त्यागात् कारणात् । येयं ‚{\tiny $_{lb}$}‚प्रति ‚{\tiny $_{6}$}‚ ज्ञाहानिर्व्यवस्थापिता तस्यां विशेषनियमः किङ्कृतः । कोसावनेन प्रकारेण ‚{\tiny $_{lb}$}‚स्वपक्षे प्रतिपक्षधर्मानुज्ञास्वरूपेण प्रतिज्ञाहानिरिति । स्यात् मतमयमेव प्रति ‚{\tiny $_{7}$}‚ ज्ञा‚{\tiny $_{lb}$}‚हानिः प्रकारो नान्योस्ति ततो नियमार्थमुच्यत इति । ‚{\color{DodgerBlue3}‚सम्भवति ह्यन्येनापी} ति । अथ ‚{\tiny $_{lb}$}‚ \leavevmode\ledsidenote{\textenglish{77/s}} मतिः प्रधानमेतन्निमित्तं तस्यास्ततोस्मिन् प्रदर्शितेऽन्योपि प्रकाशित एव भवती ‚{\tiny $_{8}$}‚ ति । ‚{\tiny $_{lb}$}‚तदत्राप्याह । इदमेव च हेतुदोषोद्भावनादिकङ्कारणं यस्मादेवं हेतुदोषोद्भावना‚{\tiny $_{lb}$}‚दिना प्रतिपादितेन प्रतिवादिना प्रतिज्ञा हातव्या सम्यग्दूषणाभिधानात् । ‚{\tiny $_{9}$}‚ \leavevmode\ledsidenote{\textenglish{51a/msK}} यच्चेद- ‚{\tiny $_{lb}$}‚मभ्यधायि सामान्यं नित्यमैन्द्र्यिकमित्युक्ते शब्दोप्येवमस्त्वित्यत्र प्रतिविधत्ते । ‚{\color{DodgerBlue3}‚इद‚{\tiny $_{lb}$}‚म्पुनरसम्बद्धमेव} ‚{\tiny $_{9a1}$}‚ । यस्मात् कः स्वस्थात्मा सामान्योपदर्शनमात्रेण सामान्य‚{\tiny $_{lb}$}‚मस्ति ‚{\tiny $_{1}$}‚ न चैन्द्रियकन्नित्यञ्चेत्येतदविचार्य शब्दं नित्यं प्रतिपद्येत । एतावत्तु भवेत् ‚{\tiny $_{lb}$}‚सामन्यस्यापि नित्यस्यैन्द्रियकत्वे तस्य ऐन्द्रियकत्वस्यानित्येपि घटे दर्शनात् संशयितः ‚{\tiny $_{lb}$}‚स्यात् ‚{\tiny $_{2}$}‚ \add{।} अपि च प्रतिदृष्टान्तधर्मानुज्ञैवात्र न युक्तेत्याह । न च तद्धमं तस्य सामा‚{\tiny $_{lb}$}‚न्यस्य धर्मन्नित्यत्वं यतोऽनित्यः शब्द इति वदता कस्यचिन्नित्यः शब्द इत्ययमञ्जशो\edtext{}{\lemma{इत्ययमञ्जशो}\Bfootnote{‚{\tiny $_{lb}$}‚? से}}‚{\tiny $_{3}$}‚ ति प्रत्यासन्नः प्रतिपक्षः स्यान्न सामान्यन्तस्य धर्म्यन्तरत्वात् । तथा ह्येका‚{\tiny $_{lb}$}‚धिकरणयोरेव नित्यत्वानित्यत्वयोर्विरोधो न नानाधिकरणयोः । आञ्जस ‚{\tiny $_{4}$}‚ ग्रहणमय‚{\tiny $_{lb}$}‚मपि विरुद्धधर्माधिकरणत्वात् प्रतिपक्षो न त्वतिनिकटो यथा नित्यः शब्द इत्ययमिति ‚{\tiny $_{lb}$}‚परिदीपनार्थं । नानेन प्रकारेण प्रतिज्ञाहाने ‚{\tiny $_{5}$}‚ र्निग्रहार्ह इति वर्तते । केनानेनेत्याह । ‚{\tiny $_{lb}$}‚प्रतिपक्षधर्मानुज्ञया । अथवा अनेनेत्यसाधनाङ्गवचनेन । यथोक्तमिदमेव प्रधानं ‚{\tiny $_{lb}$}‚निमित्तमिति ॥
	{\color{gray}{\rmlatinfont\textsuperscript{§~\theparCount}}}
	\pend% ending standard par
      ‚{\tiny $_{lb}$}‚\textsuperscript{\textenglish{78/s}}

	  
	  \pstart \leavevmode% starting standard par
	\hphantom{.}प्रतिज्ञा ‚{\tiny $_{6}$}‚ तार्थप्रतिषेधे धर्मविकल्पात्तदर्थनिर्देशः प्रतिज्ञान्तरमिति ‚{\tiny $_{9a5}$}‚ \href{http://sarit.indology.info/?cref=ns\%C5\%AB.5.2.3}{न्या० सू० ५।२।३ } द्वितीयलक्षणसूत्रं \add{।} निग्रहस्थानमिति सर्व्वत्रानुवर्तते । अस्यार्थः ‚{\tiny $_{lb}$}‚प्रतिषेधो विपक्षे हेतुसद्भाव ‚{\tiny $_{7}$}‚ कथनं तस्मिन्सति सपक्षविपक्षयोर्द्धर्मभेदेन करणभू‚{\tiny $_{lb}$}‚तेन पूर्व्वप्रतिज्ञार्थप्रतिपत्यर्थं प्रतिज्ञान्तरङ्करोति । यथा घटोऽसर्वगत एवं शब्दोप्य‚{\tiny $_{lb}$}‚सर्व्वगतो घटवदेवा ‚{\tiny $_{8}$}‚ नित्यः शब्द इति शेषः सुज्ञानः । इदं निराकरोति ‚{\color{DodgerBlue3}‚अत्रापी} ‚{\tiny $_{9a8}$}‚ ‚{\tiny $_{lb}$}‚त्यादिना । ‚{\color{DodgerBlue3}‚अबिद्धकर्ण} स्तु भाष्यटीकायामिदमाशङ्क्यपरिजिहीर्षति \add{।} ‚{\color{DodgerBlue3}‚ननु चासर्व्व‚{\tiny $_{lb}$}‚गतत्वे} सतीति । हे ‚{\tiny $_{9}$}‚ \leavevmode\ledsidenote{\textenglish{51b/msK}} तुविशेषणमुक्तं । सविशेषणश्च हेतुर्विपक्षे नास्तीति न प्रतिज्ञान्तरं ‚{\tiny $_{lb}$}‚निग्रहस्थानं । नहि तदेवमसर्व्वगतः शब्द इति प्रतिज्ञान्तरोपादानात् । हेतुविशेषणो‚{\tiny $_{lb}$}‚पादाने हेत्वन्तरं निग्रहस्थानमिति । एतच्चातिस्थूलं । स ह्येवं पक्षधर्ममेव विदग्ध‚{\tiny $_{lb}$}‚बुद्धिर्विशिनष्टि न तु प्रतिज्ञान्तरमुपादत्ते सिद्धत्वात् । यदपि हेतुविशेषणोपादाने ‚{\tiny $_{lb}$}‚हेत्व ‚{\tiny $_{2}$}‚ न्तरन्निग्रहस्थानमित्यभ्यधायि तदप्यतिपेलवं । यस्मादेवं तदेव नामास्तु प्रतिज्ञा‚{\tiny $_{lb}$}‚न्तरत्वसम्बद्धं । उदाहरणसाधर्म्यादेश्चेति । उदाहरणसाधर्म्यात्साध्यसाधनं हेतु ‚{\tiny $_{lb}$}‚ \href{http://sarit.indology.info/?cref=ns\%C5\%AB.1.1.34}{न्या० सू० १।१।३४} रित्येतस्य प्रतिज्ञालक्षणस्य साध्यनिर्देशः प्रतिज्ञेत्येतस्याभावात् । ‚{\tiny $_{lb}$}‚उपादवता चानेन प्रतिज्ञां प्रतिज्ञासाधनाय प्रतिज्ञामात्रेण युक्तिरहिते ‚{\tiny $_{4}$}‚ न सिद्धिरिष्टा ‚{\tiny $_{lb}$}‚ \leavevmode\ledsidenote{\textenglish{79/s}} भवति । ततश्च प्रागपि प्रथमप्रतिज्ञानन्तरमपि हेतुमैन्द्रियकत्वन्न ब्रूयात् । तस्मादेवं ‚{\tiny $_{lb}$}‚प्रकाराणाम्बालप्रलापानां प्रतिज्ञासाधनाय प्रति ‚{\tiny $_{5}$}‚ ज्ञान्तरमुच्यत इत्येवं रूपाणां ‚{\tiny $_{lb}$}‚प रिस ङ्ख्यातुमस\edtext{}{\lemma{ङ्ख्यातुमस}\Bfootnote{? श}}क्यत्वात् लक्षणनियमोप्यसम्बद्ध एव । कोसौ । प्रतिज्ञान्तरा‚{\tiny $_{lb}$}‚भिधाने प्रतिज्ञान्तरं नाम निग्रहस्थानमिति ‚{\tiny $_{6}$}‚ ।
	{\color{gray}{\rmlatinfont\textsuperscript{§~\theparCount}}}
	\pend% ending standard par
      ‚{\tiny $_{lb}$}‚

	  
	  \pstart \leavevmode% starting standard par
	ननु नायमीदृशो लक्षणनियमः प्रतिज्ञातार्थप्रतिषेधे धर्मविकल्पात्तदर्थनिर्देश ‚{\tiny $_{lb}$}‚इत्येवं कृतत्वात् । नास्ति दोषस्तस्यैव पर्य्यायान्तरेण कथनात् । अथोच्यते य ‚{\tiny $_{7}$}‚ था ‚{\tiny $_{lb}$}‚विद्वांसो न प्रतिज्ञां प्रतिज्ञासाधानयाहुस्तथा साध्यसिध्यर्थमसिद्धविरुद्धानैकान्तकादी‚{\tiny $_{lb}$}‚नपि प्रयुञ्जते ततश्चासाधनाङ्गवचनमित्यादि त्वयापि न वाच्यं ‚{\tiny $_{8}$}‚ भवेदतः प्राह \add{।} ‚{\tiny $_{lb}$}‚विदुषामपी ‚{\tiny $_{9b6}$}‚ ति । अनुद्दिश्याप्रमाणकं शास्त्रोपगममिति मामकीने तन्त्रे ‚{\tiny $_{lb}$}‚ \leavevmode\ledsidenote{\textenglish{80/s}} सामान्यं यथा भूतं सिद्धमित्येव न प्रदर्श्यत\add{इ}त्यर्थः । तथाहि व्युत्थित ‚{\tiny $_{9}$}‚ \leavevmode\ledsidenote{\textenglish{52a/msK}} चेतसो ‚{\tiny $_{lb}$}‚न परसमयव्यवस्थोपरोधमाद्रियन्ते तत्वदर्शनाध्यवसायशूराः शू\edtext{}{\lemma{शू}\Bfootnote{? सू}}रयः । ‚{\tiny $_{lb}$}‚अप्रमाणकम्वचनं प्रमाणोपेतस्याभ्युपगमस्य विद्वद्भिरलङ्घनीयत्वात् । एतच्च ‚{\tiny $_{lb}$}‚स्या ‚{\tiny $_{1}$}‚ त् प्रमाणैरसमर्थितसाधनाभिधानाद्वाद्यपि जेता न भवति प्रतिपक्षस्य ‚{\tiny $_{lb}$}‚निराकरणात् ॥ ४ ॥
	{\color{gray}{\rmlatinfont\textsuperscript{§~\theparCount}}}
	\pend% ending standard par
      ‚{\tiny $_{lb}$}‚

	  
	  \pstart \leavevmode% starting standard par
	\hphantom{.}प्रतिज्ञाहेत्वोर्विरोधः प्रतिज्ञाविरोधो \href{http://sarit.indology.info/?cref=ns\%C5\%AB.5.2.4}{न्या० सू० ५।२।४ }‚{\tiny $_{9b10}$}‚ नाम ‚{\tiny $_{lb}$}‚निग्रहस्थानं । गुणव्यति ‚{\tiny $_{2}$}‚ रिक्तं द्रव्यमिति प्रतिज्ञा । रूपादिभ्योर्थान्तरस्यानुप‚{\tiny $_{lb}$}‚लब्धेरिति हेतुः । सोयम्प्रतिज्ञाहेत्वोर्विरोधः । यदि गुणव्यतिरिक्तं द्रव्यं रूपादिभ्ये‚{\tiny $_{lb}$}‚ऽर्थान्तरस्यानुपलब्धिर्नोपपद्यते । अथ रूपादिभ्योर्थान्तरस्यानुपलब्धिर्गुणव्यतिरिक्तं ‚{\tiny $_{lb}$}‚द्रव्यमिति नोपपद्यते । एतेनैव प्रतिज्ञाहेत्वोर्विरोधेन प्रतिज्ञाविरोधः स्ववचनेन ‚{\tiny $_{lb}$}‚व्याख्यात \add{।} ‚{\tiny $_{4}$}‚ सूत्रकारेणास्योपलक्षणार्थमुक्तमेतत् । श्रमणा ‚{\tiny $_{10a1}$}‚ प्रतिविरत‚{\tiny $_{lb}$}‚पुरुषसम्भोगा गर्भश्च नान्तरेण पुरुषसम्भोगमिति स्ववचनव्याहतिः । हेतुविरोध ‚{\tiny $_{lb}$}‚एतेन ‚{\tiny $_{5}$}‚ चोक्त इति वर्तते । सर्व्व पृथग् नाना नास्त्येको भाव इति यावत् । समूहे ‚{\tiny $_{lb}$}‚भावसब्द\edtext{}{\lemma{भावसब्द}\Bfootnote{? शब्द}}प्रयोगात् समूहवाचकघटादिभावशब्दवाच्यत्वादित्यर्थः । यस्मात् ‚{\tiny $_{lb}$}‚समूह इ ‚{\tiny $_{6}$}‚ ति ब्रुवाणेन एकोभ्युपगतो भवति । एकसमुच्चयो हि समूह इति । तथा हि ‚{\tiny $_{lb}$}‚गवादिद्रव्याणि समुदितानि प्रतिपद्यमानेन समूहोभ्युपेयः । स चायं समूहयन्ति ‚{\tiny $_{lb}$}‚ \leavevmode\ledsidenote{\textenglish{81/s}} द्र ‚{\tiny $_{7}$}‚ व्याण्येतानि गवादिभावेन व्यवस्थितानीति न व्यवतिष्ठते । भेदोप्यल्पतरतमत्वेन ‚{\tiny $_{lb}$}‚यत्तत्र परमाल्पं यदभेद्यं ततो निवर्त्तते यतश्चायं भेदो निवर्त्तते तदेकं । अथ म ‚{\tiny $_{8}$}‚ न्यसे ‚{\tiny $_{lb}$}‚यं तमभेद्यं परमाणुं मन्यसे सोपि रूपादीनां समुदाय इति । एतस्मिन्वै दर्शने ये ‚{\tiny $_{lb}$}‚रूपादयः समुदितास्ते परमाणुरिति परमाणौ रूपं स कस्य समुदाय इ ‚{\tiny $_{9}$}‚ \leavevmode\ledsidenote{\textenglish{52b/msK}} ति वक्तव्यं । ‚{\tiny $_{lb}$}‚एवं शेषेषु गुणेषु । अथ न तं समुदायम्प्रतिपद्यसे । अष्टौ द्रव्याणि समुदितानि ‚{\tiny $_{lb}$}‚परमाणुरिति शास्त्रं व्याहतं । ‚{\color{DodgerBlue3}‚कामेऽष्टद्रव्यकोऽशब्दः परमाणुरिति \href{http://sarit.indology.info/?cref=ak.2.22}{अभिधर्मकोशे २।२२}} । तस्मा ‚{\tiny $_{1}$}‚ दनुपपत्तावनेकोपपत्तिरित्यतिमौढ्यं । असिद्धश्चायं हेतुः । ‚{\tiny $_{lb}$}‚यस्मादनेकविधलक्षणैर्गन्धादिभिर्गुणैर्बुध्नादिभिश्चावयवैः सम्बद्ध एको भाव ‚{\tiny $_{lb}$}‚उपपद्यते । अतः ‚{\tiny $_{2}$}‚ शब्दादेकार्थाधिगतौ शेषोनुसक्तो\edtext{}{\lemma{शेषोनुसक्तो}\Bfootnote{? षक्तो}}र्थो गम्यत इति ।
	{\color{gray}{\rmlatinfont\textsuperscript{§~\theparCount}}}
	\pend% ending standard par
      ‚{\tiny $_{lb}$}‚

	  
	  \pstart \leavevmode% starting standard par
	ननु चायमपि प्रतिज्ञाहेत्वोर्विरोध इति प्रथमादस्याविशेषः । मैवमुभयाश्रित ‚{\tiny $_{lb}$}‚त्वात् विरोधस्य । विवक्षातो ‚{\tiny $_{3}$}‚ ऽन्यतरनिर्देश इति ‚{\color{DodgerBlue3}‚भारद्वाजे} नैवोक्तत्त्वात् । प्रति‚{\tiny $_{lb}$}‚ज्ञाया दृष्टान्तविरोधो यथा व्यक्तमेकप्रकृतिकं परिमितत्वात् शरावादिवदिति ‚{\tiny $_{lb}$}‚शरावादिर्दृष्टान्त ए ‚{\tiny $_{4}$}‚ कप्रकृतित्वं बाधते । दृष्टान्तभूतायाः प्रकृतेः प्रकृत्यंतरत्वात् । ‚{\tiny $_{lb}$}‚एकप्रकृतित्वे वा शरावादिर्दृष्टान्तोऽयुक्तः । हेतोश्च दृष्टन्तादिभिर्विरोधो यथा ‚{\tiny $_{lb}$}‚गुण ‚{\tiny $_{5}$}‚ व्यतिरिक्तं द्रव्यमर्थान्तरत्वेनानुपलभ्यमानत्त्वात् । घटादिवदिति । घटादी‚{\tiny $_{lb}$}‚नाम्भेदेन ग्रहणाद्धेतुं बाधते दृष्टान्तः । आदिग्रहणेन हेतोरुपनयनिगम\add{न}आभ्यां ‚{\tiny $_{lb}$}‚विरोधो गृह्यते । अनयोरुदाहरणमनित्यः शब्दः कृतकत्वात् । यत्कृतकन्तदनित्यं ‚{\tiny $_{lb}$}‚यथाकाशन्तथा च कृतकः शब्द इत्युपनयेन हेतोर्विरोधः । तथा ह्युदा ‚{\tiny $_{7}$}‚ हरणा ‚{\tiny $_{lb}$}‚पेक्षस्तथेत्युयुपसंहारो न तथेति चेति \href{http://sarit.indology.info/?cref=ns\%C5\%AB.1.1.38}{न्या० सू० १।१।३८} साध्यस्योपनय उक्तः । ‚{\tiny $_{lb}$}‚इह च विपरीतमुदाहरणमित्येतदपेक्षोपनयेन हेतोर्विरोधः । ईदृशे च प्रयोगे ‚{\tiny $_{lb}$}‚त ‚{\tiny $_{8}$}‚ स्मादनित्य इत्युपसंहारे निगमनेन । प्रमाणविरोधश्च प्रतिज्ञाहेतोर्यथाऽनुष्णो‚{\tiny $_{lb}$}‚ग्निर्द्रव्यत्वाज्जलवदिति प्रत्यक्षम्बाधते । ‚{\color{DodgerBlue3}‚परपक्ष} ‚{\tiny $_{10a2}$}‚ इत्यादि । एतच्च ‚{\tiny $_{lb}$}‚यच्च ‚{\color{DodgerBlue3}‚स्वपक्षा ‚{\tiny $_{9}$}‚\leavevmode\ledsidenote{\textenglish{53a/msK}} नपेक्षञ्} चेत्यादि ‚{\tiny $_{10a3}$}‚ । एतदप्युभयम्प्रतिज्ञाहेतोर्विरोध ‚{\tiny $_{lb}$}‚इत्यनेनैव सङ्गृहीतत्वात् पृथग् निग्रहस्थानत्वेन नैव वक्तव्यमिति दर्शयति । ‚{\tiny $_{1}$}‚ ‚{\tiny $_{lb}$}‚परपक्ष इत्यत्र परेणप्र माणे कृते ‚{\color{DodgerBlue3}‚कणा} दोऽनैकान्तिकमुद्भावयति । स्वपक्षानपेक्षञ्चेत्यत्र ‚{\tiny $_{lb}$}‚ \leavevmode\ledsidenote{\textenglish{82/s}} तु ‚{\color{DodgerBlue3}‚वैशेषिक} एव प्रमाणङ्करोति । परस्तं व्यभिचारयतीति भेदः । यदि तर्हि ‚{\tiny $_{lb}$}‚गो ‚{\tiny $_{2}$}‚ त्वादिना व्यभिचारे कृते विरुद्धमुत्तरं तथा सत्यनैकान्तिको निर्विषय इत्याह । ‚{\tiny $_{lb}$}‚ ‚{\color{DodgerBlue3}‚उभयेत्या ‚{\tiny $_{10a4}$}‚ दि} । वादिप्रतिवादिप्रसिद्ध उभयपक्षसंप्रतिपन्नः सोऽनैकान्तिकस्त‚{\tiny $_{lb}$}‚द्विषयत्वादुप ‚{\tiny $_{3}$}‚ चारेण तथा च वृत्तिस्तेनानैकान्तिकचोदनेति । ‚{\color{DodgerBlue3}‚अत्रापी} ‚{\tiny $_{10a3}$}‚ त्यादि । ‚{\tiny $_{lb}$}‚नैतदपि प्रतिक्षिपति तदाश्रयः सा प्रतिज्ञाऽश्रयो यस्य विरोधस्य स तथा । तत्कृतो ‚{\tiny $_{lb}$}‚वेति त ‚{\tiny $_{4}$}‚ या प्रतिज्ञया कृतः । परिशिष्टमतिस्फुटं । व्यतिरिक्तानामपि कुतश्चित् पर्व‚{\tiny $_{lb}$}‚तादेः सकाशाद्विप्रकर्षिणाम्पिसा\edtext{}{\lemma{सकाशाद्विप्रकर्षिणाम्पिसा}\Bfootnote{? शा}}चादीनां तत्रेदमेव निग्रहाधिकरणं । ‚{\tiny $_{5}$}‚ यदुत ‚{\tiny $_{lb}$}‚प्रतिज्ञायाः प्रयोगः । न विरोधः प्रतिज्ञायाः निग्रहाधिकरणमिति वर्तते । किमिति । ‚{\tiny $_{lb}$}‚तदधिकरणत्वात् । प्रतिज्ञाश्रयत्वात् इत्यर्थः । यदि पु ‚{\tiny $_{6}$}‚ नस्तदधिकरणो न भवेद् ‚{\tiny $_{lb}$}‚ \leavevmode\ledsidenote{\textenglish{83/s}} भवेन्निग्रहाधिकरणमित्याह । ‚{\color{DodgerBlue3}‚यदी} ‚{\tiny $_{10b1}$}‚ त्यादि । प्रस्तावस्य वादस्योपसंहारः परि‚{\tiny $_{lb}$}‚समाप्तिस्तस्यावसानन्निमितं प्रतिज्ञाप्रयोगः । तन्मात्रेणै ‚{\tiny $_{7}$}‚ वासाधनाङ्गाभिधानात् ‚{\tiny $_{lb}$}‚वादिनोभङ्गात् । क्वचित्प्रस्तावोपसंहारावसरत्वादिति पठ्यते । तत्रापि ‚{\tiny $_{lb}$}‚वादपरिसमाप्तेः प्रतिज्ञापदप्रयोगे सत्यवसरोऽधिकार इत्यर्थः । अथ बुद्धिर्यथा ‚{\tiny $_{lb}$}‚भवद्भिः कस्यचिदर्थस्य क्षणिकत्वादिकमेकमेव साध्यं बहुभिः सत्वोत्पत्तिमत्व‚{\tiny $_{lb}$}‚प्रत्ययभेदभेदित्वादिभिर्हेतुभिः प्रतिपाद्यते तथैक ‚{\tiny $_{9}$}‚‚{\tiny $_{53b}$}‚ मपि दूष्यम्परोपन्यस्तं साधन‚{\tiny $_{lb}$}‚वाक्यं प्रतिज्ञोपादानद्वारेण तद्विरोधद्वारेणान्यथा वा दूष्यते । तथा च नायन्दोषः ‚{\tiny $_{lb}$}‚पराजितपराजयाभावादिति । तदत्राह । ये तु हेतवः ‚{\tiny $_{1}$}‚ उच्यन्ते ‚{\tiny $_{10b2}$}‚ तेषाम्विकल्पेन ‚{\tiny $_{lb}$}‚पूर्व्वहेत्वनपेक्षया । एवं वैतत् । अथवान्यथा साधयामीत्येतत् साध्यसाधनाय वृत्तेः ‚{\tiny $_{lb}$}‚कारणात्सामर्थ्यमस्ति \add{।} किं पुनः कारणं न समुच्चये नैव प्रयोग ‚{\tiny $_{2}$}‚ इत्याह । ‚{\color{DodgerBlue3}‚अन्यथा ‚{\tiny $_{lb}$}‚यदि} ‚{\tiny $_{10b3}$}‚ समुच्चये नैवापरहेत्वन्तरप्रयोगोभीष्टस्तदा द्वितीयस्य वैयर्थ्यात् विकल्पेन ‚{\tiny $_{lb}$}‚सामान्यमिति वर्तते । वैयर्थ्यमेव प्रतिपादयति । यदि हि तत्रा ‚{\tiny $_{3}$}‚ प्येकप्रयोगमन्तरेणा‚{\tiny $_{lb}$}‚परस्य प्रयोगो न सम्भवेत् । उभयप्रतिषेधेन विध्यवसायात् । यद्येकस्य प्रयोगे‚{\tiny $_{lb}$}‚ऽपरस्य समुच्चयेन प्रयोगः सम्भवेदि ‚{\tiny $_{4}$}‚ त्यर्थः । तदा न द्वितीयस्य कश्चित् साधानार्थो ‚{\tiny $_{lb}$}‚प्रतीतप्रतिपादनाभावात् । प्रथमहेतुप्रतिपादित एवार्थे व्यापृतत्वान्निष्पादित‚{\tiny $_{lb}$}‚क्रिये दारुणि प्रवृत्त ‚{\tiny $_{5}$}‚ स्यैव दात्रादेर्न कश्चित्साधकतमत्वार्थ इति यावत् । ननु च ‚{\tiny $_{lb}$}‚साधनवद्विकल्पेनैव दूषणमपि भविष्यति । एवं मन्यते । नैवं परोभ्युपगन्तुर्महति । ‚{\tiny $_{lb}$}‚ए ‚{\tiny $_{6}$}‚ वं हि तेन स्वयमेव प्रतिज्ञाया असाधनाङ्गत्वम्प्रतिपन्नम्भवेत् । ततश्चैतद् व्याह‚{\tiny $_{lb}$}‚न्यते । प्रतिज्ञाहेतूदाहरणोपनयनिगमनान्यवयवा \href{http://sarit.indology.info/?cref=ns\%C5\%AB.1.1.32}{न्या० सू० १।१।३२ } इति । ‚{\tiny $_{lb}$}‚अन्यैरेव हेतुभिरित्यव ‚{\tiny $_{7}$}‚ यविद्रव्यनिषेधकैः पूर्व्वोक्तप्रकारैः कुम्भादिशब्दस्यैक‚{\tiny $_{lb}$}‚घटाद्यवयविद्रव्यलक्षणविशेषानभिधानमनेकस्य चार्थस्य रूपादेर्यत्सामान्यमेकार्थ‚{\tiny $_{lb}$}‚ \leavevmode\ledsidenote{\textenglish{84/s}} क्रियासामर्थ्यात्म ‚{\tiny $_{8}$}‚ कन्तदभिधानञ्च प्रतिपाद्य सर्व्वस्य शब्दार्थस्य रूपादेरेकार्थ‚{\tiny $_{lb}$}‚क्रियासमर्थस्य नानार्थरूपतया करणभूतया । एकश्चासौ वस्तुविशेषस्वभावश्चा‚{\tiny $_{lb}$}‚वयविद्रव्यरूपस्त ‚{\tiny $_{9}$}‚ \leavevmode\ledsidenote{\textenglish{54a/msK}} स्य भाव एकवस्तुविशेषस्वभावता तस्या अभावमुपदर्शयन्नास्त्येको ‚{\tiny $_{lb}$}‚भाव इत्यभिद ‚{\color{DodgerBlue3}‚ध्याद्} बौद्धो न तु रूपाणीन्द्रियार्थान् प्रतिक्षिपन् । स्यात् मती ‚{\tiny $_{lb}$}‚रूपाद्यव्यतिरेकात् सामर्थ्यमप्यनेकं तत्कथन्तदेकमित्युच्यते कथं वा तस्य शब्दार्थत्वं । ‚{\tiny $_{lb}$}‚नहि स्वलक्षणं शब्दार्थ इत्युच्यते । नानाभूतमपि सामर्थ्यभिन्नवत्स्वव्यतिरेकादेकार्थ‚{\tiny $_{lb}$}‚क्रियाकारितयैकप्रत्यवम ‚{\tiny $_{2}$}‚ र्षहेतुत्वात् परम्परयैकमित्याख्यायते । यथोक्तम् ।
	{\color{gray}{\rmlatinfont\textsuperscript{§~\theparCount}}}
	\pend% ending standard par
      ‚{\tiny $_{lb}$}‚
	  \bigskip
	  \begingroup
	
	    
	    \stanza[\smallbreak]
	  \flagstanza{\tiny\textenglish{...25}}{\normalfontlatin\large ``\qquad}एकप्रत्यवमर्षस्य हेतुत्वाद्धीरभेदिनी \add{।}&‚{\tiny $_{lb}$}‚एकधा हेतुभावेन व्यक्तीनामप्यभिन्नतेति ॥ \add{२५}{\normalfontlatin\large\qquad{}"}\&[\smallbreak]
	  
	  
	  
	  \endgroup
	‚{\tiny $_{lb}$}‚

	  
	  \pstart \leavevmode% starting standard par
	पुरुषाध्यवसायानिरोधे ‚{\tiny $_{3}$}‚ न शब्दार्थत्वं तस्य व्यवस्थाप्यते । पुरुषोह्यनादिमिथ्या‚{\tiny $_{lb}$}‚भ्यासवासनापरिपाकप्रभावादन्तर्मात्राविपरिवर्तिनमाकारं बाहयेष्वेवारोप्य दृश्य‚{\tiny $_{lb}$}‚विकल्पयो ‚{\tiny $_{4}$}‚ रेकत्वम्प्रतिपन्नः परमार्थतस्तु निर्विषया एव ध्वनयः । व्यक्तीनाम्विज्ञाना‚{\tiny $_{lb}$}‚कारस्य चार्थान्तरानुगमाभावेनाभिलापागोचरत्वात् । यथाध्यवसायञ्चाका ‚{\tiny $_{5}$}‚ रस्य ‚{\tiny $_{lb}$}‚सत्वात् । यथोक्तं ‚{\color{DodgerBlue3}‚सूत्रे} ॥ ‚{\tiny $_{lb}$}‚ 
	    \pend% close preceding par
	  
	    
	    \stanza[\smallbreak]
	  \flagstanza{\tiny\textenglish{...26}}{\normalfontlatin\large ``\qquad}येन येन हि नाम्ना वै यो यो धर्मोभिलप्यते ।&‚{\tiny $_{lb}$}‚न स सम्विद्यते तत्र धर्माणां सा हि धर्मतेति ॥ \add{२६}{\normalfontlatin\large\qquad{}"}\&[\smallbreak]
	  
	  
	  
	    \pstart  \leavevmode% new par for following
	    \hphantom{.}
	   ‚{\tiny $_{lb}$}‚तदयमत्र समदायार्थो रूपादी ‚{\tiny $_{6}$}‚ नाङ्घटस्य च यथा क्रममनेकत्वमेकत्वञ्च वहुवचनैक‚{\tiny $_{lb}$}‚वचनाभिधयत्वात् \add{।} तद्यथा नक्षत्राणि शशीत्येवमादिभिरनुमानाभासैः परेण ‚{\tiny $_{lb}$}‚घटादिशब्दस्य विषयो ‚{\tiny $_{7}$}‚ योयमेकार्थोऽवयव्यभिधानोभ्युपगतः स एव प्रतिक्षिप्यते । नतु ‚{\tiny $_{lb}$}‚रूपरसादयः परमाणुस्वभावास्तथा हि तेषाम्प्रत्येकमेकैकात्मकत्वमिष्टमेव । केवला ‚{\tiny $_{8}$}‚ ‚{\tiny $_{lb}$}‚स्तदातिसफलबीजवन्न समुदायमासादयन्तीति नियतसहोत्पादत्वपरिदीपनायोक्तं ॥ ‚{\tiny $_{lb}$}‚ 
	    \pend% close preceding par
	  
	    
	    \stanza[\smallbreak]
	  \flagstanza{\tiny\textenglish{...27}}{\normalfontlatin\large ``\qquad}कामेष्टद्रव्यकोऽशब्दः परमाणुरतीन्द्रियः \add{।}&‚{\tiny $_{lb}$}‚ \leavevmode\ledsidenote{\textenglish{54b/msK}}कायेन्द्रियो नवद्रव्यो दशद्रव्योऽपरेन्द्रिय इति । \add{२७}{\normalfontlatin\large\qquad{}"}\&[\smallbreak]
	  
	  
	  
	    \pstart  \leavevmode% new par for following
	    \hphantom{.}
	  \href{http://sarit.indology.info/?cref=ak.2.22}{अभिधर्मकोशे २।२२}
	{\color{gray}{\rmlatinfont\textsuperscript{§~\theparCount}}}
	\pend% ending standard par
      ‚{\tiny $_{lb}$}‚

	  
	  \pstart \leavevmode% starting standard par
	यथा तु परमाणूनामैन्द्रियकत्वमनित्यत्वञ्च तद्विस्तरेणोक्तमन्यत्रास्माभिः । ‚{\tiny $_{lb}$}‚यत्पुनरेतद्वहुवचनैकवचनाभिधेयत्वादिति तद्व्यभिचा ‚{\tiny $_{1}$}‚ रि । तथाहि यदैकस्यामपि ‚{\tiny $_{lb}$}‚ \leavevmode\ledsidenote{\textenglish{85/s}} योषिति जले सिकताद्रव्ये वा दारा आपः सिकता इति व्यवहारः । तदा किन्तत्र ‚{\tiny $_{lb}$}‚बाहुल्यं येनैवं भवति शक्तिभेद इति चेत् । सर्व्वत्रोच्छि ‚{\tiny $_{2}$}‚ न्नमिदानीमेकवचनमेक‚{\tiny $_{lb}$}‚शक्तेरभावात् । वस्त्वभेदादन्यत्रैकवचनमिति चेत् । इहाप्यस्तु । तदयन्निर्वस्तुको ‚{\tiny $_{lb}$}‚नियमः क्रियमाणः स्वातन्त्र्यमिच्छायाः शब्दप्रयो ‚{\tiny $_{3}$}‚ गे ख्यापयति । एतेन तदपि ‚{\tiny $_{lb}$}‚प्रत्युक्तं यदाह ‚{\color{DodgerBlue3}‚कुमारिलः} \add{।} ‚{\tiny $_{lb}$}‚ 
	    \pend% close preceding par
	  
	    
	    \stanza[\smallbreak]
	  \flagstanza{\tiny\textenglish{...28}}{\normalfontlatin\large ``\qquad}तत्र व्यक्तौ च जातौ च दारादिश्चेत्प्रयुज्यते ।&‚{\tiny $_{lb}$}‚व्यक्तेरवयवानाम्वा संख्यामादाय वर्तत \add{२८}{\normalfontlatin\large\qquad{}"}\&[\smallbreak]
	  
	  
	  
	    \pstart  \leavevmode% new par for following
	    \hphantom{.}
	  इति ॥ ‚{\tiny $_{lb}$}‚षण्णगरीति च कथम्वहुष्वेकवचनं । नहि नगराण्येव किञ्चित् कुतस्तेषां समाहारः । ‚{\tiny $_{lb}$}‚प्रासादपुरुषादीनां विजातीयानामनारम्भात् कुतस्तत्समु ‚{\tiny $_{5}$}‚ दायो द्रव्यं असंयोगाच्च ‚{\tiny $_{lb}$}‚नापि संयोगः । प्रासादादीनां परस्परसंयोगात् । प्रासादस्य स्वयं संयोगात्मकस्य ‚{\tiny $_{lb}$}‚निर्गुणतयापरेणासंयोगाच्च । तत ‚{\tiny $_{6}$}‚ एव च संख्याभावः । तत्संयोगपुरुषविशिष्टा ‚{\tiny $_{lb}$}‚सत्ता नगरमिति चेत् । किमस्यानिरतिस\edtext{}{\lemma{किमस्यानिरतिस}\Bfootnote{? श}}याया विशेषणं सत्तायाश्चैकत्वात् ‚{\tiny $_{lb}$}‚नगरबहुत्वेपि नगराणीति बहु ‚{\tiny $_{7}$}‚ वचनं स्यात् \add{।} द्वयस्य परस्परसहिततेति चेत् । ‚{\tiny $_{lb}$}‚अनुपकारकयोः कः सहायीभावः । पुरुषसंयोगसत्तानां च वहुत्वान्नगरमिति कथमेक‚{\tiny $_{lb}$}‚वचनं । तथा भू ‚{\tiny $_{8}$}‚ तानां क्वचिदभिन्ना शक्तिः सा निमित्तमिति चेन्न । शक्तेर्वस्तु‚{\tiny $_{lb}$}‚रूपाव्यतिरेकात् । व्यतिरेके चानुपकार्यस्य पारतन्त्र्यायोगात् । उपकारे वा श‚{\tiny $_{lb}$}‚क्त्युपकारिण्या अपि श ‚{\tiny $_{9}$}‚ \leavevmode\ledsidenote{\textenglish{55a/msK}} क्तेर्व्यतिरेक इत्यवस्थितेरप्रतिपत्तिः । तदव्यतिरेके अन्यासा- ‚{\tiny $_{lb}$}‚मपि प्रसंग इति यत्किञ्चिदेतत् । प्रकारान्तरमप्याह । ‚{\color{DodgerBlue3}‚दृष्टोपदर्शन} \uline{श्चै} तदिति । ‚{\tiny $_{lb}$}‚किं पुनः पञ्चम्यन्त ‚{\tiny $_{1}$}‚ निर्देशेपि दृष्टान्तो भवतीत्याह । ‚{\color{DodgerBlue3}‚कृतकानित्यत्वादि ‚{\tiny $_{10b7}$}‚ ‚{\tiny $_{lb}$}‚ति} यथा येनोक्तं । ‚{\tiny $_{lb}$}‚ 
	    \pend% close preceding par
	  
	    
	    \stanza[\smallbreak]
	  \flagstanza{\tiny\textenglish{...29}}{\normalfontlatin\large ``\qquad}हेतोः साध्यान्वयो यत्राभावेभावश्चकथ्यते ।&‚{\tiny $_{lb}$}‚पञ्चम्या तत्र दृष्टान्तो हेतुस्तूपनयाऽ ‚{\tiny $_{2}$}‚ त्मक \add{२९}{\normalfontlatin\large\qquad{}"}\&[\smallbreak]
	  
	  
	  
	    \pstart  \leavevmode% new par for following
	    \hphantom{.}
	   इति ॥ ‚{\tiny $_{lb}$}‚क्वचिदर्थे घटादिद्रव्ये विप्रतिपत्तौ सत्यां रूपादिव्यतिरिक्तमस्ति नास्तीत्यने‚{\tiny $_{lb}$}‚कस्यार्थस्य परस्परव्यावृत्तस्य नगरादेः सामान्यं षण्णगरीत्यादि ‚{\tiny $_{3}$}‚ यद् बुध्यारोपितं ‚{\tiny $_{lb}$}‚ \leavevmode\ledsidenote{\textenglish{86/s}} तत्र प्रसिद्धं शब्दप्रयोगमादर्श्य परस्परव्यावृत्तानामेकार्थाननुगतानां बुद्धिसमाकृते ‚{\tiny $_{lb}$}‚समूहे भावशब्दप्रयोगादित्यनेन पश्चा ‚{\tiny $_{4}$}‚ दुपनयेन पक्षधर्मोपसंहारमागूर्य प्रतिपादित‚{\tiny $_{lb}$}‚विप्रतिपत्तिस्थानः सन्सामान्येनोपसंहरति । ‚{\color{DodgerBlue3}‚सर्व्वं पृथगि ‚{\tiny $_{10b8}$}‚ ति} । प्रतिपादितं ‚{\tiny $_{lb}$}‚प्रतिपत्तिस्थानम ‚{\tiny $_{5}$}‚ नेनेति विग्रहः । एतदुक्तम्भवति । कपालादिव्यतिरेकेना\edtext{}{\lemma{कपालादिव्यतिरेकेना}\Bfootnote{? णा}}‚{\tiny $_{lb}$}‚वयव्यस्ति नास्तीति विवादे सत्ययं त्रिलक्षणहेतुसूचनपरो दृष्टान्त उपन्यस्तो न ‚{\tiny $_{lb}$}‚हेतुः ‚{\tiny $_{6}$}‚ । प्रयोगस्त्वत्रैवं क्रियते । ये परस्परव्यावृत्ता न ते व्यतिरिक्तैकावयविद्रव्यानु‚{\tiny $_{lb}$}‚गतमूर्त्तयः । तद्यथा षण्णगर्यादयः । तथा च परस्परव्यावृत्ताः कपालादय इ ‚{\tiny $_{7}$}‚ ति ॥
	{\color{gray}{\rmlatinfont\textsuperscript{§~\theparCount}}}
	\pend% ending standard par
      ‚{\tiny $_{lb}$}‚

	  
	  \pstart \leavevmode% starting standard par
	ननु च यद्ययं दृष्टान्तप्रयोगस्तक्तिमृजुनैव तत्प्रयोगक्रमेण न प्रयुक्तो यथा ‚{\tiny $_{lb}$}‚यत्सत् तत्क्षणिकं यथा घट इत्यादौ । किम्पञ्चम्यन्तनिर्देशेन । विप्रतिपत्ति ‚{\tiny $_{8}$}‚ ‚{\tiny $_{lb}$}‚विषयश्च किन्न दर्शितः कपालादेरवयविप्रतिषेधविशिष्टः । यथान्यत्रानित्यः शब्दः ‚{\tiny $_{lb}$}‚कृतकानित्यत्वादिति । चकारात् स्पष्टश्च कस्मात् हेतुः साध्यानुगतो न प्रद ‚{\tiny $_{9}$}‚ \leavevmode\ledsidenote{\textenglish{55b/msK}} र्शितः । ‚{\tiny $_{lb}$}‚तथाह्यत्र परस्परव्यावृत्तानामेकार्थाननुगतानां बुद्ध्या समाहिते समूहभावशब्द‚{\tiny $_{lb}$}‚प्रयोगादित्यभ्यूह्य वाक्यपरिसमाप्तिः क्रियते । अत्रोत्तरं न समासनिर्दे ‚{\tiny $_{1}$}‚ शात् ‚{\tiny $_{lb}$}‚संक्षेपाभिधानादित्यर्थः । एवमपि प्रयोगदर्शनात् कृतकानित्यत्वादित्यादौ । ‚{\tiny $_{lb}$}‚असाधनं वाक्यत्वाच्च साधनप्रयोगोत्प्रेक्षासूचकं वाक्यमेतत् । नत्विदं साधन‚{\tiny $_{lb}$}‚वा ‚{\tiny $_{2}$}‚ क्यमित्यर्थः । अत एवेति दृष्टान्तवाक्यत्वादेवेति । यश्चायं हेतुस्तन्तुपटरूपे ‚{\tiny $_{lb}$}‚भिन्नकारणे विशेषवत्वाद्रूपस्पर्शवदिति ॥ अयमपि तन्तुपटयोर्भेदासिद्धौ तदा ‚{\tiny $_{3}$}‚ ‚{\tiny $_{lb}$}‚श्रितस्यापि गुणस्य विभागासिद्धेरसिद्धाश्रय इति नालमिष्टसिद्धये । तथा हि सूक्ष्मस्‚{\tiny $_{lb}$}‚थूलद्रव्यसमवायो विशेषवत्वं भिन्नकालोत्पन्नद्रव्यसंवाया ‚{\tiny $_{4}$}‚ वेति व्याचक्षते ।
	{\color{gray}{\rmlatinfont\textsuperscript{§~\theparCount}}}
	\pend% ending standard par
      ‚{\tiny $_{lb}$}‚

	  
	  \pstart \leavevmode% starting standard par
	परे । ननु विचित्राभिसन्धयः योक्तारः । तत्र ये केचिद्धेत्वभिप्रायेनैव\edtext{}{\lemma{केचिद्धेत्वभिप्रायेनैव}\Bfootnote{? णैव}} ‚{\tiny $_{lb}$}‚ वाचः प्रयुञ्जते तान्प्रत्यस्माभिः प्रतिज्ञया हेतोर्बाधनमु ‚{\tiny $_{5}$}‚ च्यते न तु ये दृष्टान्ताभि‚{\tiny $_{lb}$}‚मानिन इत्यत्राह \add{।} ‚{\color{DodgerBlue3}‚नचे ‚{\tiny $_{10b9}$}‚ त्यादि} । भगव ‚{\color{DodgerBlue3}‚त्तथागत} मतावलम्बिनामुपर्ययमु‚{\tiny $_{lb}$}‚पक्षिप्तो विरोधो भवद्भि ‚{\color{DodgerBlue3}‚राक्षपादै} र्न च नः स्वप्न ‚{\tiny $_{6}$}‚ व्ये तादृशोस्तीति पिण्डार्थः । ‚{\tiny $_{lb}$}‚स्यात् मतमस्त्येव ‚{\color{DodgerBlue3}‚योगाचारो यः} \add{—}
	{\color{gray}{\rmlatinfont\textsuperscript{§~\theparCount}}}
	\pend% ending standard par
      ‚{\tiny $_{lb}$}‚

	  
	  \pstart \leavevmode% starting standard par
	\leavevmode\ledsidenote{\textenglish{87/s}} 
	    \pend% close preceding par
	  
	    
	    \stanza[\smallbreak]
	  \flagstanza{\tiny\textenglish{...30}}{\normalfontlatin\large ``\qquad}पङ्केन युगपद्योगात् परमाणोः पतङ्गतां ।&‚{\tiny $_{lb}$}‚षण्णां समानदेशत्वात् पिण्डः स्यादणुमात्रक \add{ः ॥ ३०}{\normalfontlatin\large\qquad{}"}\&[\smallbreak]
	  
	  
	  
	    \pstart  \leavevmode% new par for following
	    \hphantom{.}
	   ‚{\tiny $_{lb}$}‚इ‚{\tiny $_{7}$}‚ त्यादिना परमाणोरेकत्वमनभ्युपगच्छन्नपि पिण्डं समूहापरपर्यायमिच्छती‚{\tiny $_{lb}$}‚त्येतदुच्यते । ‚{\color{DodgerBlue3}‚योपी ‚{\tiny $_{11a1}$}‚ त्यादि} किन्तर्ह्यभाव एवाणोरनेन प्रकारेण साधयितु ‚{\tiny $_{8}$}‚ ‚{\tiny $_{lb}$}‚मिष्टः । कथं । एकानेकप्रतिषेधात् । पङ्कायोगादिना तावदेकत्वं प्रतिसि\edtext{}{\lemma{प्रतिसि}\Bfootnote{? षि}}द्धं । ‚{\tiny $_{lb}$}‚तत्समुदायरूपमनेकत्वमपि तदभावादेव न विद्यते । यथोक्तन् ननु  \add{।} ‚{\tiny $_{lb}$}‚ 
	    \pend% close preceding par
	  
	    
	    \stanza[\smallbreak]
	  \flagstanza{\tiny\textenglish{...31}}{\normalfontlatin\large ``\qquad}तस्य तस्यै ‚{\tiny $_{9}$}‚\leavevmode\ledsidenote{\textenglish{56a/msK}} कता नास्ति यो यो भावः परीक्ष्यते ।&‚{\tiny $_{lb}$}‚न सन्ति तेनानेकेपि येनैकोपि न विद्यत \add{३१}{\normalfontlatin\large\qquad{}"}\&[\smallbreak]
	  
	  
	  
	    \pstart  \leavevmode% new par for following
	    \hphantom{.}
	   इति ॥ ‚{\tiny $_{lb}$}‚ननु ‚{\tiny $_{1}$}‚ पङ्कयोगादिना कथमेकत्वमपोदितं । यावता तत्र तस्य सावयव‚{\tiny $_{lb}$}‚त्वमापादितं ॥ त एव चावयवास्तस्याल्पीयांसः परमाणवो वि ‚{\tiny $_{2}$}‚ भागपर्यवसानलक्ष‚{\tiny $_{lb}$}‚णत्वात् परमाणूनां । अथ तेषामप्यङ्गानामनेनैव विधानेन सावयवत्वमापाद्यते । तथा ‚{\tiny $_{lb}$}‚सति तत्राप्येतदेवोत्तरमित्यनेनैव प्रकारे ‚{\tiny $_{3}$}‚ ण न शक्यते परमाणोरेकत्वनिषेधं कर्त्तु । ‚{\tiny $_{lb}$}‚विभागस्य विभज्यमाण\edtext{}{\lemma{विभज्यमाण}\Bfootnote{? न}}तन्त्रत्त्वात् । कथञ्चानभ्युपगताणुस्तस्य पङ्कयोगा‚{\tiny $_{lb}$}‚दिकमभ्युपगच्छतीति त ‚{\tiny $_{4}$}‚ दसत्वप्रतिपादने सर्वे हेतवः स्वत एवाश्रयासिद्धा इति । ‚{\tiny $_{lb}$}‚एतच्च नैवं यस्मात्समर्था वादिनोऽपगतावयवविभागमासादितापकर्षयन्तं ‚{\tiny $_{lb}$}‚भाव ‚{\tiny $_{5}$}‚ मणुरित्याचक्षते तस्य तेन पङ्कायोगादिनैकत्वमपाक्रियते । ते च यद्येवं ‚{\tiny $_{lb}$}‚निराकृताः सन्तो यथोपगतस्य सावयवत्वं प्रतिपद्यन्ते तदा स्व ‚{\tiny $_{6}$}‚ प्रतिज्ञायाश्च्यवे‚{\tiny $_{lb}$}‚रन् । न हि अनङ्गीकृतसावयवत्वास्तथा प्रत्यवस्थानमर्हन्ति । त एवावयवाः सन्तु ‚{\tiny $_{lb}$}‚परमाणव इति । तैरेव च तल्लक्षणम्व्यवस्थापनीयं ‚{\color{DodgerBlue3}‚योगाचा ‚{\tiny $_{7}$}‚ रे} ण च निषेध्यमिति ‚{\tiny $_{lb}$}‚निगृह्यन्ते । अत एव नानवस्था । प्रसङ्गसाधनत्वाच्चासिद्धतादोषोपि नास्तीत्यल‚{\tiny $_{lb}$}‚मेतेन । अथोच्यते न वयं भवन्तं प्रतीदं ब्रूमो यस्तु कश्चिदधौ ‚{\tiny $_{8}$}‚ तपादो वाद्येवं ‚{\tiny $_{lb}$}‚ \leavevmode\ledsidenote{\textenglish{88/s}} प्राह तम्प्रतीति । तच्चासत्सर्व्वं पृथग्भावलक्षणपृथग्त्वात् नानेकलक्षणे‚{\tiny $_{lb}$}‚नैकभावनिष्पत्तेरित्यत्र प्रस्तावे ‚{\color{DodgerBlue3}‚भारद्वाजे} नास्मान्प्रत्येव ‚{\color{DodgerBlue3}‚कामेष्टद्रव्य ‚{\tiny $_{9}$}‚ \leavevmode\ledsidenote{\textenglish{56b/msK}} क} इत्यादिना ‚{\tiny $_{lb}$}‚सिद्धान्तमस्माकीनमुपक्षिप्याप्यभिधानात् । तथाप्यभ्युपगम्य दोषान्तरमाह । न ‚{\tiny $_{lb}$}‚चायम्पूर्वकाद् गुणव्यतिरिक्तमित्यादिपदसूचितात् परस्परार्थमा ‚{\tiny $_{1}$}‚ धाय भिद्यते । ‚{\tiny $_{lb}$}‚हेतुप्रतिज्ञयोः सम्बन्धिन्योः बाधयोरुदाहरणोपेतयोः पृथग्बाधोदाहरणयोर्न कश्चि‚{\tiny $_{lb}$}‚दर्थभेदः शब्दभेदस्तु केवलः । तथाविधस्य च पृथगुदा ‚{\tiny $_{2}$}‚ हरणेऽतिप्रसङ्गोऽकृतकः ‚{\tiny $_{lb}$}‚शब्दः कृतकत्वादित्याद्यप्युदाहर्त्तव्यम्भवेत । सह पृथग्वेति क्वचित्पाठः । तत्राय‚{\tiny $_{lb}$}‚मर्थः सह यौगपद्येन यथा प्रथमे पृथक् प्रत्येकं । य ‚{\tiny $_{3}$}‚ थेह अथवा विरोधचिन्ताप्यत्रा‚{\tiny $_{lb}$}‚युक्तेत्याह \add{।} ‚{\color{DodgerBlue3}‚अपिचे} ‚{\tiny $_{11a5}$}‚ त्यादि । सर्व्वं पृथक् समूहे भावशब्दप्रयो\add{गा} ‚{\tiny $_{lb}$}‚दित्ययं हेतुः । सर्व्वस्य धर्मिणो धर्म एव न भवति शब्दधर्म ‚{\tiny $_{4}$}‚ त्वादित्यसिद्धः । ‚{\tiny $_{lb}$}‚तथा च व्यधिकरणत्वादसिद्धतैव दोषो गुडो मधुरः काकस्य कार्ष्ण्यादिति यथा । ‚{\tiny $_{lb}$}‚तत्र न विरोधो भिन्नाधिकरणत्वाद्धेतुप्रति ‚{\tiny $_{5}$}‚ ज्ञार्थयोः । स्याद् बुद्धिः समूहवाचक‚{\tiny $_{lb}$}‚शब्दवाच्यत्वादित्येवं ‚{\color{DodgerBlue3}‚भाविविक्तेन} भाष्यटीकायां प्रयोगाद् व्यधिकरणत्वं नास्ति । ‚{\tiny $_{lb}$}‚एवं मन्यते न तावदय ‚{\color{DodgerBlue3}‚मु ‚{\tiny $_{6}$}‚ द्योतकरे} णैवं प्रयुक्तस्य वायमस्माभिर्दोषोभिधातुमारब्धो ‚{\tiny $_{lb}$}‚येपि सम्प्रत्यन्यथा प्रयुञ्जते तेषामपि यद्ययं दोषो न भवति । भवतु अनन्तरोक्तस्तु ‚{\tiny $_{lb}$}‚दोषो वक्ष्य ‚{\tiny $_{7}$}‚ माणश्च ब्रह्मणाऽपि न शक्यते परिहर्तुमिति । प्रतिज्ञाहेत्वोर्विरोधस्य च ‚{\tiny $_{lb}$}‚निग्रहस्थानान्तरत्वमङ्गीकृत्य मयेदमभ्यधायि । न त्वस्य तद्युक्तं । हेत्वाभासा‚{\tiny $_{lb}$}‚श्च निग्रह ‚{\tiny $_{8}$}‚ स्थानानी \href{http://sarit.indology.info/?cref=ns\%C5\%AB.5.2.24}{न्या० सू० ५।२।२४ } त्यनेनैव सङ्गृहीतत्वादित्येतद्विभणि‚{\tiny $_{lb}$}‚षुराह । ‚{\color{DodgerBlue3}‚अपिचे} ‚{\tiny $_{11a5}$}‚ त्यादि । द्वाववयवौ यस्या दोषजातेर्दो ‚{\tiny $_{9}$}‚ \leavevmode\ledsidenote{\textenglish{57a/msK}} षप्रकारस्य सा द्वयी । ‚{\tiny $_{lb}$}‚कामित्याह । विरुद्धतामसिद्धताञ्च । कथम्पुनर्विरुद्धतेत्याह । विरुद्धतेत्यादि । अयमत्र ‚{\tiny $_{lb}$}‚संक्षेपार्थः । प्रतिज्ञाहेत्वोर्यत्र प्रयोगेंविरोधश्चोद्यते तत्रा ‚{\tiny $_{1}$}‚ वश्यं सिद्धेन धर्मिणा भाव्यं । ‚{\tiny $_{lb}$}‚सिद्धे च तस्मिन्धर्मणि\edtext{}{\lemma{तस्मिन्धर्मणि}\Bfootnote{? धर्मिणि}}हेतोर्वा सत्वम्भवेत् साध्यधर्मस्य । द्वयोर्वा । तत्र न ‚{\tiny $_{lb}$}‚तावत् द्वयोरपि सत्वं प ‚{\tiny $_{2}$}‚ रस्परविरोधित्वेन शीतोष्णयोरिव एकाधिकरणत्वाभावात् । ‚{\tiny $_{lb}$}‚अन्यथा सहैकत्रावस्थानाद्रसरूपवदविरोध एव भवेदिति प्रतिज्ञाहेत्वोर्विरोधो दूरतर ‚{\tiny $_{lb}$}‚एव ‚{\tiny $_{3}$}‚ प्रसज्यते । तद्वक्ष्यति । विरुद्धयोः स्वभावयोरेकत्रासम्भवान्न चान्यथा विरोध ‚{\tiny $_{lb}$}‚इति । अथ हेतोस्तत्र सत्वं । एवमपि यत्र हेतुस्तत्र न साध्यधर्मस्तद्विप ‚{\tiny $_{4}$}‚ र्ययस्तु विद्यत ‚{\tiny $_{lb}$}‚ \leavevmode\ledsidenote{\textenglish{89/s}} इति व्यक्तमस्य विरुद्धत्वं । नित्यः शब्दः कृतकत्वादिवत् । तदाह \add{।} ‚{\color{DodgerBlue3}‚विरुद्धता ‚{\tiny $_{lb}$}‚सिद्धेर्हेत्वोर्धर्मिणि भाव} ‚{\tiny $_{11a6}$}‚ इति । यदा पुनस्तस्मिन्धर्मिणि प्रमा ‚{\tiny $_{5}$}‚ णान्तरेण ‚{\tiny $_{lb}$}‚साध्यधर्मस्य सत्वं निश्चितं तदा तत्र हेतोरवृत्तिर्विरोधिना क्रोडीकृतत्त्वात् । ‚{\tiny $_{lb}$}‚अतश्चासिद्धत्वं हेतोः । कृतकः शब्दोऽकार्यत्वादिति यथा । तज्जा ‚{\tiny $_{6}$}‚ ते असिद्धता ‚{\tiny $_{lb}$}‚पुनर्द्धर्मिणीत्यादि । अथमन्यसे । प्रमाणेन सिद्ध एव गुणव्यतिरिक्ते द्रव्यादौ ‚{\tiny $_{lb}$}‚धर्मिणि प्रतिज्ञाहेतोर्विरोधो व्यवस्थाप्यते ततो नायं दोष इत्य ‚{\tiny $_{7}$}‚ त इदमासङ्कते\edtext{}{\lemma{इदमासङ्कते}\Bfootnote{‚{\tiny $_{lb}$}‚? शङ्कते}}‚{\color{DodgerBlue3}‚असिद्ध ‚{\tiny $_{11b8}$}‚ इत्यादिना} । एवमपि यदि नाम धर्म्यभावेन ‚{\tiny $_{lb}$}‚पक्षधर्मस्यासम्भवात् विरुद्धत्वं परिहृतं । असिद्धत्वं पुनस्तदवस्थमेवेति मनस्या ‚{\tiny $_{8}$}‚ ‚{\tiny $_{lb}$}‚धायाह । प्रमाणयोगे तूभयोर्वादिप्रतिवादिनोर्धर्मिणि हेतोर्वृत्तिसंशयः । प्रमाण ‚{\tiny $_{lb}$}‚निवृत्तावप्यर्थाभावासिद्धेः । अतश्चासिद्धतैव सन्दिग्धाश्रयत्वात् । इह नि ‚{\tiny $_{9}$}‚ \leavevmode\ledsidenote{\textenglish{57b/msK}} कुञ्जे ‚{\tiny $_{lb}$}‚मयूरः केकायितत्त्वादित्यादिवत् । तु शब्दः प्रतिपादकप्रमाणायोगे धर्मिणः ‚{\tiny $_{lb}$}‚सन्दिग्धाश्रयताहेतोर्धर्मिबाधकप्रमाणवृत्तौ स्फुटमेवाश्रयासिद्धतत्वं । सर्व्वगत आ ‚{\tiny $_{1}$}‚ त्मनि ‚{\tiny $_{lb}$}‚साध्ये सर्वत्रोपलभ्यमाण\edtext{}{\lemma{सर्वत्रोपलभ्यमाण}\Bfootnote{? न}}गुणत्ववदित्यस्य समुच्चयार्थः । तथा ह्यसिद्धेः ‚{\tiny $_{lb}$}‚धर्मिस्वभाव इत्यत्र प्रतिपादकप्रमाणावृत्तेरसिद्धो धर्मी विवक्षितः स्यात् । ‚{\tiny $_{lb}$}‚ \leavevmode\ledsidenote{\textenglish{90/s}} बाधक ‚{\tiny $_{2}$}‚ प्रमाणवृत्तेर्वा । पूर्वस्मिन्पक्षे कण्ठेनैवोक्तो दोष उत्तरत्र शब्देन समुच्चितः । ‚{\tiny $_{lb}$}‚अत्रौद्योतकरमुत्तरमाशङ्कते । ‚{\color{DodgerBlue3}‚उभयाश्रयत्वा ‚{\tiny $_{11a9}$}‚ दित्यादिना} । गतार्थत्वात् ‚{\tiny $_{lb}$}‚सुज्ञानं ‚{\tiny $_{3}$}‚ सर्व्वमेतत् । न सर्व्वत्रेत्यादिना निराकरोति । यथोक्तं प्राग् न द्वयीं दोष‚{\tiny $_{lb}$}‚जातिमित्यत्र । अथ प्रतिज्ञामात्रभाव्येव हेत्वनपेक्षः प्रतिज्ञाविरोधो व्यवस्था ‚{\tiny $_{4}$}‚ प्यते ‚{\tiny $_{lb}$}‚यथा नास्त्यात्मा श्रमणा गर्भिणीत्यत्रेत्यत आह \add{।} ‚{\color{DodgerBlue3}‚अनपेक्षे च हेतुग्रहणमसम्वद्धं} ‚{\tiny $_{lb}$}‚ ‚{\tiny $_{11b3}$}‚ । अनुपकारकत्वात् । यदपीदं हेतुविरोधस्योदाहर ‚{\tiny $_{5}$}‚ णं दत्तं नित्यः शब्द ‚{\tiny $_{lb}$}‚इत्यादिना तत्प्रतिज्ञाविरोधस्य हेतुनायुक्तमिति कथनायाह । न ‚{\color{DodgerBlue3}‚चेदि ‚{\tiny $_{11b3}$}‚ ‚{\tiny $_{lb}$}‚ त्यादि} । स्यात् मतमुभयाश्रयत्वाद्विरोधस्यैवमपि न हेतुत ‚{\tiny $_{6}$}‚ एवेत्यत उच्यते ‚{\tiny $_{lb}$}‚ ‚{\color{DodgerBlue3}‚उभयाश्रयेपी ‚{\tiny $_{11b4}$}‚ त्यादि} । एवमुपदर्शितान्युदाहरणानि प्रक्षिप्यातिदिष्ट‚{\tiny $_{lb}$}‚दूषणायाह । यच्चोक्तमेतेन प्रतिज्ञायाः दृष्टान्तविरोधादयोपि ‚{\tiny $_{7}$}‚ वक्तव्या ‚{\color{DodgerBlue3}‚भण्डा‚{\tiny $_{lb}$}‚लेख्यन्या} येने ‚{\tiny $_{11b5}$}‚ ति । इति शब्दो वक्तव्य इत्यत्र प्रतिपत्तव्योऽन्यथापरे‚{\tiny $_{lb}$}‚णोत्तरस्याप्रयुक्तत्त्वात् दुःश्लिष्टो भवेत । भण्डग्रहणन्नित्यपुरुषोपलक्षणार्थं । ‚{\tiny $_{lb}$}‚यथा हि भण्डा प्राकृतान् विस्मापयन्तश्चित्रलक्षणोपेतकपिशालभञ्जिकादिप्रति‚{\tiny $_{lb}$}‚च्छन्दकमालिख्य विचित्रशिल्पकलाकौशलसादि\edtext{}{\lemma{विचित्रशिल्पकलाकौशलसादि}\Bfootnote{? शालि}}नोऽतिदि ‚{\tiny $_{9}$}‚ \leavevmode\ledsidenote{\textenglish{58a/msK}} शंत्येवं प्रका‚{\tiny $_{lb}$}‚राण्यप्यस्मत्कौशलनिर्मितान्येकतालमात्रेण हस्त्यादिरूपकस्थानानि प्रतिपत्तव्या‚{\tiny $_{lb}$}‚नीति तथा जातीयकमेतदु ‚{\color{DodgerBlue3}‚द्योतकरस्य} । तथा ह्येतदेव भाव उप ‚{\tiny $_{1}$}‚ दर्शितहेतुविरोधा‚{\tiny $_{lb}$}‚दिकं हेत्वाभासव्यतिरिक्तलक्षणोपेतं । तदतिदिष्टे पुनः कैव चिन्ता । तामेव चाति‚{\tiny $_{lb}$}‚दिष्टस्य दृष्टान्तविरोधादेर्हेत्वाभासव्यतिरिक्त ‚{\tiny $_{2}$}‚ लक्षणापेततामभिधातुमुपक्रमते । ‚{\tiny $_{lb}$}‚ ‚{\color{DodgerBlue3}‚तत्रापी} ‚{\tiny $_{11b5}$}‚ त्यादिना । यत्र प्रतिज्ञायाः दृष्टान्तविरोधस्तत्रापि पक्षीकृतधर्मवि‚{\tiny $_{lb}$}‚पर्ययवति दृष्टान्ते सति विरोधः स्यात् प्र ‚{\tiny $_{3}$}‚ तिज्ञायाः दृष्टान्तेनेति शेषः । पक्षीकृतश्चा‚{\tiny $_{lb}$}‚ \leavevmode\ledsidenote{\textenglish{91/s}} सौ धर्मश्च तस्य विपर्ययः स विद्यते यस्मिन्निति विग्रहः । दृष्टान्त इति च साधर्म्यदृ‚{\tiny $_{lb}$}‚ष्टान्तो । अभिप्रेतः । यस्मा ‚{\tiny $_{4}$}‚ द्वैधर्म्यदृष्टान्तः साध्यधर्मविपर्ययवानेव तत्र को विरोधः । ‚{\tiny $_{lb}$}‚तत्रोदाहरणं । नित्यः शब्दो घटवदिति । विरुद्धे च दृष्टान्ते सति यदि पक्षधर्मस्य वृत्ति ‚{\tiny $_{5}$}‚ ‚{\tiny $_{lb}$}‚रनन्यसाधारणा प्रसाध्यते प्रमाणेन विरुद्धस्तदा हेत्वाभासः । नान्यसाधारणेत्यनन्य‚{\tiny $_{lb}$}‚साधारणा । अन्यशब्देन पक्षीकृतधर्मविपर्ययवतः ‚{\tiny $_{6}$}‚ पृथग्भूतः पक्षीकृतधर्मवानभि‚{\tiny $_{lb}$}‚प्रेतः पक्षीकृतधर्मविपर्ययवत्येववर्तते इत्येवं यदि साध्यत इत्यर्थः । यथानयोरेव ‚{\tiny $_{lb}$}‚साध्यदृष्टान्तयोः कार्यत्वादि ‚{\tiny $_{7}$}‚ ति तद्विपक्षीकृतधर्मबहिर्व्योमादौ न वर्तते तद्विपरीते ‚{\tiny $_{lb}$}‚पुनर्घटे वर्तत इति । साधारणायाम्वृत्तौ साधितायां सपक्षविपक्षयोरिति शेषः । अनै‚{\tiny $_{lb}$}‚कान्तिकः ‚{\tiny $_{8}$}‚ साधारणाख्यः । यथानयोरेव साध्योदाहरणयोः प्रमेयत्वादिति । अप्रसा‚{\tiny $_{lb}$}‚धिते चातद्वृत्तिनियमे तयोः सपक्षविपक्षयोर्वृत्तिनियमे सपक्ष एव वर्त्तते वि ‚{\tiny $_{9}$}‚ \leavevmode\ledsidenote{\textenglish{58b/msK}} पक्ष ‚{\tiny $_{lb}$}‚एवेति अनैकान्तिक एव सन्दिग्धान्वयः सन्दिग्धव्यतिरेको वा । यथा सर्व्वविद्वीत‚{\tiny $_{lb}$}‚रागो वा विवक्षितः पुरुषो न वा तथा वक्तृत्वाद्रथ्यानरवदिति । तयोरेव सपक्ष ‚{\tiny $_{1}$}‚ ‚{\tiny $_{lb}$}‚विपक्षयोरवृत्तौ वा सत्यामसाधारणः । नित्यः शब्दः श्रावणत्वादिति यथा । परः प्राह ‚{\tiny $_{lb}$}‚विरुद्धदृष्टान्तावृत्तौ हेतोर्विपर्ययवृत्तौ च सत्यान्न कश्चिद्धेतुदोषः ‚{\tiny $_{2}$}‚ तद्यथाऽनित्यः ‚{\tiny $_{lb}$}‚शब्दः प्रत्ययभेदभेदित्वात् नभोवदिति साधर्म्येण । वैधर्म्येण च घटवदिति । अत्र ‚{\tiny $_{lb}$}‚नासिद्धत्वं ध\add{ि}र्मणि हेतोः सद्भावात् । नाप्यनैकान्तिकत्वमुभ ‚{\tiny $_{3}$}‚ यत्रावृत्तेः । ‚{\tiny $_{lb}$}‚प्रतिबन्धसद्भावाच्च । न च विरुद्धत्वं सपक्षविपक्षयोर्वैपरीत्येन वृत्यभावात् । ‚{\tiny $_{lb}$}‚दृष्टान्तेन तु विरोधः प्रतिज्ञायाः इत्ययं हेतुदोषानति ‚{\tiny $_{4}$}‚ क्रान्तो विषयः प्रतिज्ञायाः ‚{\tiny $_{lb}$}‚दृष्टान्तेन च विरोधस्येति । इदमपनुदति । न । तदापि संशयहेतुत्वानतिवृत्तेः । ‚{\tiny $_{lb}$}‚यस्माद् दृष्टान्ते न प्रतिज्ञाया विरोधः सा ‚{\tiny $_{5}$}‚ धर्म्ये दृष्टान्ते दोषो न वैधर्म्ये । कस्माद‚{\tiny $_{lb}$}‚भिमतत्वाद् विरोधस्य । पक्षीकृतधर्मविपर्ययवानेव हि वैधर्म्यंदृष्टान्त उच्यत इत्य‚{\tiny $_{lb}$}‚भिप्रायः । यदि ना ‚{\tiny $_{6}$}‚ मैवं तथापि कथं हेत्वाभासानतिवृत्तिरित्याह साधर्म्यदृष्टान्ते ‚{\tiny $_{lb}$}‚च विपरीतधर्मवति नभसि नाऽव्यभिचारधर्मता शक्या दर्शयितुँ । तदर्थश्च दृष्टान्तः ‚{\tiny $_{lb}$}‚प्रदर्शते ॥ ‚{\tiny $_{7}$}‚ यदाह ‚{\tiny $_{lb}$}‚ \leavevmode\ledsidenote{\textenglish{92/s}} 
	    \pend% close preceding par
	  
	    
	    \stanza[\smallbreak]
	  \flagstanza{\tiny\textenglish{...32}}{\normalfontlatin\large ``\qquad}त्रिरूपो हेतुरित्युक्तं पक्षधर्मे च संस्थितः ।&‚{\tiny $_{lb}$}‚रूढे रूपद्वयं शेषं दृष्टान्तेन प्रदर्श्यत \add{३२}{\normalfontlatin\large\qquad{}"}\&[\smallbreak]
	  
	  
	  
	    \pstart  \leavevmode% new par for following
	    \hphantom{.}
	  इति । ‚{\tiny $_{lb}$}‚ननु च कथमशक्या यावता प्रत्ययभेदभेदित्वमनित्यत्वाव्यभि ‚{\tiny $_{8}$}‚ चार्येव तत्वत ‚{\tiny $_{lb}$}‚इत्यत आह । ‚{\color{DodgerBlue3}‚वस्तुतः साध्याव्यभिचारेपी ‚{\tiny $_{11b7}$}‚ ति} । विद्यमानोप्यव्यभिचारः ‚{\tiny $_{lb}$}‚प्रमाणेनाप्रतिपादितत्वादसत्कल्प इति भावः । तदेतन्नाप्रदर्शितावि ‚{\tiny $_{9}$}‚ \leavevmode\ledsidenote{\textenglish{59a/msK}} नाभावसम्बद्धा‚{\tiny $_{lb}$}‚द्धेतोः साध्यनिश्चयः । तत्तस्मान्न प्रतिज्ञाया दृष्टान्तविरोधोपि हेत्वाभासानतिवर्तते । ‚{\tiny $_{lb}$}‚अस्यापि तदानीं संदिग्धविपक्षव्यावृत्तिकत्वादित्यागूरितं ‚{\tiny $_{1}$}‚ । न केवलहेतुविरोध ‚{\tiny $_{lb}$}‚इत्यपि शब्दः परमतमास\edtext{}{\lemma{परमतमास}\Bfootnote{? श}}ङ्कते । उभयथापि हेतुद्वारेण दृष्टान्तद्वारेण च । न ‚{\tiny $_{lb}$}‚हेतुद्वारेण प्राग्दृष्टान्तदोषात् प्रसङ्गेन पराजितस्य वादि ‚{\tiny $_{2}$}‚ नो दोषान्तरस्य दृष्टान्त‚{\tiny $_{lb}$}‚विरोधस्य वाच्यस्य वानपेक्षणात् पराजितपराजयाभावादित्याकूतं । विशेषेण साध‚{\tiny $_{lb}$}‚नावयवानुक्रमवादिनो नैयायिकस्य स हि ‚{\tiny $_{3}$}‚ प्रतिज्ञाहेतूदाहरणोपनयनिगमनानामानु‚{\tiny $_{lb}$}‚पूर्वीं प्रतिपन्नः । कः पुनः तस्यातिशय इत्याह । ‚{\color{DodgerBlue3}‚उदाहरणसाधर्म्यमि} ‚{\tiny $_{11b9}$}‚‚{\tiny $_{lb}$}‚त्यादि । अङ्गीकृत्य चेदमवादि ‚{\tiny $_{4}$}‚ न तु दृष्टान्तविरोधो हेत्वाभासरूपासंस्पर्श्यस्ति । ‚{\tiny $_{lb}$}‚यथोक्तमनन्तरमिति । एतेन विकल्पतो दोषविधानं प्रत्युक्तं । एवन्तावद्व्यवस्थित‚{\tiny $_{lb}$}‚मेतद्यथा प्र ‚{\tiny $_{5}$}‚ तिज्ञाया दृष्टान्तविरोधो हेत्वाभासान्नातिवर्तत इति । यत्पुनरुदाहृत‚{\tiny $_{lb}$}‚‚{\color{DodgerBlue3}‚मविद्धकरणेन} भाष्यटीकायां व्यक्तमेकप्रकृतिकं परिमितत्वाच्छरावादि ‚{\tiny $_{6}$}‚ वदिति । ‚{\tiny $_{lb}$}‚तत्रापि विरुद्धो हेतुः परिमितत्त्वस्य हेतोः सपक्षेऽभावे वा वृत्तेः । विपक्षे चानेकप्रकृति‚{\tiny $_{lb}$}‚के शरावादौ वृत्तेः । मृदः प्रतिक्षणं प्रत्यवयवञ्च भिद्य ‚{\tiny $_{7}$}‚ मानत्वात् । संप्रति हेतोरपि ‚{\tiny $_{lb}$}‚दृष्टान्तेन विरोधो हेत्वाभासान्तर्गत इति कथयति । हेतोरपि दृष्टान्तविरोधे ‚{\tiny $_{lb}$}‚सत्यसा\add{धा}रणत्वमुभयत्रावृत्तेः । विरुद्धत्वम्वा । कदा ‚{\tiny $_{8}$}‚ विरुद्धत्त्वमित्याह । वैधर्म्ये ‚{\tiny $_{lb}$}‚यदि स्यादप्यत्रोदाहरणमुक्तं तेनैव गुणव्यतिरिक्तं द्रव्यमर्थान्तरत्वेनानुपलभ्यमा‚{\tiny $_{lb}$}‚नत्वाद् घटवदिति अत्रापि दृश्यत्वे सतीति ‚{\tiny $_{9}$}‚ \leavevmode\ledsidenote{\textenglish{59b/msK}} हेतुविशेषणे विरुद्धः सपक्षे अवर्तमान‚{\tiny $_{lb}$}‚त्वात् । विपक्षे च रूपादीनां स्वरूपे वर्तमागत्त्वात् । विशेषणानुपादाने तु व्यभिचारो‚{\tiny $_{lb}$}‚\leavevmode\ledsidenote{\textenglish{93/s}} र्थान्तरत्त्वेनानुपलब्धानामपि पि ‚{\tiny $_{1}$}‚ शाचादीनां परस्परव्यतिरेकित्वात् । न चात्र घट‚{\tiny $_{lb}$}‚वदिति दृष्टान्तो युक्तस्तस्यैव द्रव्यान्तरत्वेन पक्षीकृतत्वात् । तस्य रूपादिभ्यो भेदेन ‚{\tiny $_{lb}$}‚ग्रहणं पूर्व्वमेव प्रतिसि\edtext{}{\lemma{प्रतिसि}\Bfootnote{? षि}}द्धं ‚{\tiny $_{2}$}‚ ग्रहणे चासिद्धो हेत्वाभास इत्यस्मन्मतमेव ‚{\tiny $_{lb}$}‚स्थितं । अथ हेतोः प्रमाणविरोधे को हेत्वाभास इत्याह । असिद्धोग्नेः शैत्यस्या‚{\tiny $_{lb}$}‚विद्यमानत्वात् । यत्पुनरत्रो ‚{\tiny $_{3}$}‚ दाहरणमन्यदनुष्णोग्निर्द्रव्यत्वाज्जलवदिति तदयुक्तं । ‚{\tiny $_{lb}$}‚नहि प्रत्यक्षं द्रव्यं हेतुं बाधते । तस्य धर्मिणि सिद्धत्वात् । किन्तु प्रतिज्ञार्थमनुष्णत्वं ॥ ‚{\tiny $_{4}$}‚ ‚{\tiny $_{lb}$}‚ अथ प्रतिज्ञार्थस्य प्रत्यक्षेण बाधितत्वाद्धेतोस्तेन व्याप्तिर्न्नास्तीति हेतोः प्रमाणविरोध ‚{\tiny $_{lb}$}‚उच्यते । एवन्तर्हि विरुद्धेन साध्यधर्मेणाव्याप्तेः सन्दि ‚{\tiny $_{5}$}‚ ग्धव्यतिरेको हेत्वाभास ‚{\tiny $_{lb}$}‚इत्यस्मत्पक्ष एव समर्थितः ।
	{\color{gray}{\rmlatinfont\textsuperscript{§~\theparCount}}}
	\pend% ending standard par
      ‚{\tiny $_{lb}$}‚
	  \bigskip
	  \begingroup
	
	    
	    \stanza[\smallbreak]
	  \flagstanza{\tiny\textenglish{...33}}{\normalfontlatin\large ``\qquad}हेतोःप्रमाण\edtext{}{\lemma{हेतोःप्रमाण}\Bfootnote{? मान}}विरोघस्य हेत्वाभासानतिक्रमात् ॥ \add{३३}{\normalfontlatin\large\qquad{}"}\&[\smallbreak]
	  
	  
	  
	  \endgroup
	‚{\tiny $_{lb}$}‚

	  
	  \pstart \leavevmode% starting standard par
	तदुक्तम्
	{\color{gray}{\rmlatinfont\textsuperscript{§~\theparCount}}}
	\pend% ending standard par
      ‚{\tiny $_{lb}$}‚
	  \bigskip
	  \begingroup
	
	    
	    \stanza[\smallbreak]
	  \flagstanza{\tiny\textenglish{...34}}{\normalfontlatin\large ``\qquad}प्रत्यक्षादि\add{वि}रोधा ये व्याप्तकालो ‚{\tiny $_{6}$}‚ पपातिनः ।&‚{\tiny $_{lb}$}‚ते सर्वे न विरुद्धेन व्याप्तिधर्मेण युञ्जत \add{३४} इति ॥{\normalfontlatin\large\qquad{}"}\&[\smallbreak]
	  
	  
	  
	  \endgroup
	‚{\tiny $_{lb}$}‚

	  
	  \pstart \leavevmode% starting standard par
	स्यान्मतम्प्रतिज्ञायाः प्रमाणविरोधस्तन्मात्रभावित्वाद्धेत्वाभासेऽन्तर्गमयितुं न ‚{\tiny $_{lb}$}‚पार्यत इत्यत आह । ‚{\tiny $_{7}$}‚ ‚{\color{DodgerBlue3}‚प्रतिज्ञायाः प्रमाणविरोधः स्ववचनविरोधेन व्याख्यातः} ‚{\tiny $_{12a2}$}‚ ‚{\tiny $_{lb}$}‚ कृतप्रतिक्रियस्तत्रेदमेव निग्रहाधिकरणमसाधनाङ्गभूतायाः प्रतिज्ञायाः साधनवाक्ये ‚{\tiny $_{lb}$}‚प्र ‚{\tiny $_{8}$}‚ योग इत्यादिना । इति तस्मात् सर्व्व एवेत्युपसंहरति । यत्तु विरुद्धमुत्तरमिति ‚{\tiny $_{lb}$}‚पूर्व्वपक्षोक्तमपरमुपक्षिपति तदसम्बद्धमेव । ‚{\color{DodgerBlue3}‚यस्माद्यदि ही ‚{\tiny $_{12a3}$}‚ त्यदि} । अनि ‚{\tiny $_{9}$}‚ \leavevmode\ledsidenote{\textenglish{60a/msK}} त्यः ‚{\tiny $_{lb}$}‚शब्द ऐन्द्रियकत्वाद् घटवदित्येकं बौद्धेनान्येन वा कृते ‚{\color{DodgerBlue3}‚मीमांसकः काणादोन्यो} वा ‚{\tiny $_{lb}$}‚स्वपक्षसिद्धेन गोत्वादिना सामान्ये ‚{\tiny $_{1}$}‚ न परस्य साधनवादिनो बौद्धस्य हेतोर्व्यभिचार‚{\tiny $_{lb}$}‚सिद्धिमाकांक्षेत गोत्वमप्यैन्द्रियकं तदपि भवतोऽनित्यं प्रसज्यत इत्येव यदि परं प्रत्ये‚{\tiny $_{lb}$}‚\leavevmode\ledsidenote{\textenglish{94/s}} वाध्यारोप्याभिदध्याद् व्यभि ‚{\tiny $_{2}$}‚ चारं तदा तस्य ‚{\color{DodgerBlue3}‚बौद्ध} स्य तत्सामान्यमैन्द्रियकं नित्यञ्च ‚{\tiny $_{lb}$}‚स्वपक्षविरुद्धं नित्यपदार्थानभ्युपगामान्नाभिमतमतश्च कथं व्यभिचार इति ‚{\tiny $_{lb}$}‚विरोधो व्याहतिरयुक्तत्त्वमिति ‚{\tiny $_{3}$}‚ यावत् युज्यत उत्तरस्येत्यध्याहर्तव्यं । न तु पुनरेव‚{\tiny $_{lb}$}‚मसौ परस्येवोपरि भारमुपक्षिप्य व्यभिचारमुद्भावयति तत्कथमुत्तरस्य विरोधः ‚{\tiny $_{lb}$}‚यतः स ‚{\tiny $_{4}$}‚ ह्युत्तरवादी स्वयं प्रतिपन्ने नित्यत्वेन गोत्वे हेतोरैन्द्रियकत्वस्य वृत्तेः संश‚{\tiny $_{lb}$}‚यानः सन् किङ्घटवदैन्द्रियकत्वादनित्यः शब्दो भवतु किम्वा गोत्वा ‚{\tiny $_{5}$}‚ दिवन्नित्य इत्य‚{\tiny $_{lb}$}‚प्रतिपत्तिमनिश्चयमात्मनस्तथा ब्रुवाणः ख्यापयति सत्पक्षे खल्वेन्द्रियकमपि गोत्वं ‚{\tiny $_{lb}$}‚नित्यं तस्मादयं सांप्रत्यनैकान्तिक इती ‚{\tiny $_{6}$}‚ त्थमात्मीयमेवाभ्युपगमं पुरस्कृत्यानेकान्त‚{\tiny $_{lb}$}‚ञ्चोदयति । ततः साध्विवोत्तरमिति समुदायार्थः । स्यात् मतम्बौद्धस्य नास्त्येव‚{\tiny $_{lb}$}‚गोत्वं नित्यं ततो व्याहतमेवोत्तर ‚{\tiny $_{7}$}‚ मित्यत आह । स च हेतु ‚{\tiny $_{12a5}$}‚ रैन्द्रियकत्वादिति ‚{\tiny $_{lb}$}‚सत्यसति वा गोत्वे परमार्थतः । अप्रसाधितसाधनसामर्थ्यः सन् विपर्यये बाधक‚{\tiny $_{lb}$}‚प्रमाणावृत्या संशयहेतुत्वाद ‚{\tiny $_{8}$}‚ नैकान्तिक एव । अप्रसाधितं साधनाय सामर्थ्यं साध्या ‚{\tiny $_{lb}$}‚विनाभावित्वलक्षणमस्येति विग्रहः । साधनशब्दो भावसाधनः । यदा तु बाधक‚{\tiny $_{lb}$}‚प्रमाणबलेन हेतोरवि ‚{\tiny $_{9}$}‚ \leavevmode\ledsidenote{\textenglish{60b/msK}} नाभावं सर्व्वोपसंहारेण साधयति यत्किञ्चिदिन्द्रियज्ञानग्राह्यं ‚{\tiny $_{lb}$}‚स्वनिर्भासज्ञानजनकत्वात्तत्र सर्व्वमनित्यं नित्यत्वे सर्व्वदा तद्विषयं ज्ञानं प्रसञ्जते न ‚{\tiny $_{lb}$}‚वा कदाचिदपि ‚{\tiny $_{1}$}‚ तथाहि । ‚{\tiny $_{lb}$}‚ 
	    \pend% close preceding par
	  
	    
	    \stanza[\smallbreak]
	  \flagstanza{\tiny\textenglish{...35}}{\normalfontlatin\large ``\qquad}स्वात्मनि ज्ञानजनने यच्छक्तं शक्तमेव तत् ।&‚{\tiny $_{lb}$}‚अथवाऽशक्तं कदाचिच्चेदशक्तं सर्वदैव तत् ॥ \add{३५}{\normalfontlatin\large\qquad{}"}\&[\smallbreak]
	  
	  
	  
	    \pstart  \leavevmode% new par for following
	    \hphantom{.}
	   ‚{\tiny $_{lb}$}‚ 
	    \pend% close preceding par
	  
	    
	    \stanza[\smallbreak]
	  \flagstanza{\tiny\textenglish{...36}}{\normalfontlatin\large ``\qquad}तस्य शक्तिरशक्तिर्वा या स्वभावेन संस्थिता ।&‚{\tiny $_{lb}$}‚नित्यत्वादचिकित्स्य‚{\tiny $_{2}$}‚स्य कस्तां क्षपयितुं क्षम \add{३६}{\normalfontlatin\large\qquad{}"}\&[\smallbreak]
	  
	  
	  
	    \pstart  \leavevmode% new par for following
	    \hphantom{.}
	   इति ॥
	{\color{gray}{\rmlatinfont\textsuperscript{§~\theparCount}}}
	\pend% ending standard par
      ‚{\tiny $_{lb}$}‚

	  
	  \pstart \leavevmode% starting standard par
	तदानीं गोत्वादीनामपि नित्यानामेकप्रघटेन इव पाटितत्वात् गोत्वे हेतोर‚{\tiny $_{lb}$}‚वृत्तेर्न संशय एव भवति । ‚{\color{DodgerBlue3}‚एतेने ‚{\tiny $_{12a6}$}‚ त्यादि} सुज्ञानं । तत्रा ‚{\tiny $_{3}$}‚ प्यनैकान्तिकहेत्वा‚{\tiny $_{lb}$}‚भासत्वानतिवृत्तिरिति संक्षेपः । तत्संशयहेतुत्वमुखेनानैकान्तिकत्वमसमर्थिते सति ‚{\tiny $_{lb}$}‚हेतौ । ‚{\color{DodgerBlue3}‚अन्यत्रापी} त्येकपक्षप्रतिपन्नेपि ‚{\tiny $_{4}$}‚ वस्तुनि तुल्यमिति नोभयसिद्धेतरयोर्वस्तुनोरनै‚{\tiny $_{lb}$}‚कान्तिकत्वविशेषः । यथा कथितमनन्तरमेव । स च हेतुः सत्यसति वेत्यादिना । इतर‚{\tiny $_{lb}$}‚\leavevmode\ledsidenote{\textenglish{95/s}} देकपक्ष ‚{\tiny $_{5}$}‚ प्रतिपन्नमनैकान्तिकविषयत्वाच्चानैकान्तिकमिति व्याख्यातं । स्याच्चित्तन्ना‚{\tiny $_{lb}$}‚निष्टेर्दूषणं सर्व्वप्रसिद्धस्तु द्वयोरपि साधनं । दूषणम्वेत्येतत्कथमेवन्न ‚{\tiny $_{6}$}‚ व्याहन्यत इति ‚{\tiny $_{lb}$}‚तच्च नैवं । निश्चितदूषणाभिसन्धिवचनात् । तत एव तदनन्तरमाहान्यः पुनः साध्य‚{\tiny $_{lb}$}‚त्वमीक्षत इति । एतत्तु स्यात् । तदा द्वयोरेकस्यापि न जय ‚{\tiny $_{7}$}‚ पराजयौ । यदप्युक्त ‚{\color{DodgerBlue3}‚मुद्योत ‚{\tiny $_{lb}$}‚करेण} प्रतिज्ञाविरोधसूत्रमेव विवृण्वता दृष्टान्ताभासा इत्यादि तदप्यवयवान्तरवा‚{\tiny $_{lb}$}‚दिनो नैयायिकस्यायुक्तं । बौद्ध एवैवं ब्रुवा ‚{\tiny $_{8}$}‚ णः शोभत इत्यभिप्रेतं \add{।} तद्वचनेन हेत्वा‚{\tiny $_{lb}$}‚भासवचनेन गम्यमानस्य दृष्टान्ताभासस्य तस्माद्धेतोः सकाशात् सा ‚{\tiny $_{9}$}‚ \leavevmode\ledsidenote{\textenglish{61a/msK}} धनान्तरत्वा- ‚{\tiny $_{lb}$}‚भावप्रसङ्गात् । दृष्टान्तस्येति शेषः ।
	{\color{gray}{\rmlatinfont\textsuperscript{§~\theparCount}}}
	\pend% ending standard par
      ‚{\tiny $_{lb}$}‚

	  
	  \pstart \leavevmode% starting standard par
	ननु च दृष्टान्ताभासानां हेत्वाभासेष्वन्तर्भावेऽतिदिष्टे हेतोर्दृष्टान्तेऽवयवान्तरं ‚{\tiny $_{lb}$}‚न प्राप्नोतीति वचन ‚{\tiny $_{1}$}‚ मसम्बद्धमेवेत्यत आह । ‚{\color{DodgerBlue3}‚दृष्टान्ताभासाना} मि ‚{\tiny $_{12a9}$}‚ त्यादि । ‚{\tiny $_{lb}$}‚अयमस्य प्रयोगो मनसि विजृम्भते । यद्यतोऽर्थान्तरभूतं न तदाभासवचनेन तदाभास‚{\tiny $_{lb}$}‚वचनं न्याय्यं ‚{\tiny $_{2}$}‚ न च तदाभासेषु तदाभासानामन्तर्भावः । तद्यथा प्रत्यक्षाभासानाम‚{\tiny $_{lb}$}‚नुमानाभासेषु । तथा च भवतो हेतोर्दृष्टान्तोर्थान्तरभूत इति व्यापकविरुद्धोप‚{\tiny $_{lb}$}‚लब्धिः ‚{\tiny $_{3}$}‚ अतोऽवश्यं दृष्टान्तस्य हेतावन्तर्भाव एष्टव्यः । तत्र च न दृष्टान्तः पृथक् ‚{\tiny $_{lb}$}‚साधनावयवः स्यात् । अपृथग्वृत्तेः एकव्यापारत्त्वादित्यर्थः । एतदेव व्या ‚{\tiny $_{4}$}‚ चष्टे ‚{\color{DodgerBlue3}‚यो ‚{\tiny $_{lb}$}‚दृष्टान्त} ‚{\tiny $_{12b1}$}‚ इत्यादिना । एवं प्रतिज्ञाहेत्वोर्विरोधस्य प्रपञ्चस्य हेत्वाभासैः ‚{\tiny $_{lb}$}‚सङ्गृहीतत्वान्न पृथग्वचनं कर्त्तव्यमित्यभिधायाधुना प्रतिघहा ‚{\tiny $_{5}$}‚ न्यादीनामपीयमेव ‚{\tiny $_{lb}$}‚गतिरित्यावेदनायाह । ‚{\color{DodgerBlue3}‚अपि चे} ‚{\tiny $_{12b1}$}‚ त्यादि । पूर्व्वपक्षवादिग्रहणमुत्तरपक्षवादि‚{\tiny $_{lb}$}‚\leavevmode\ledsidenote{\textenglish{96/s}} नोऽज्ञानादीनि हेत्वाभासस्पर्शानि संतीति ‚{\tiny $_{6}$}‚ कथनार्थं । तत्सम्बन्धीनीति हेत्वाभास‚{\tiny $_{lb}$}‚पूर्व्वपक्षवादिसम्बन्धीनि वा । अथोच्यते । अर्थान्तरगमनादीनां हेत्वाभासासंस्पर्शित्त्वा ‚{\tiny $_{lb}$}‚न्नतेस्व\edtext{}{\lemma{न्नतेस्व}\Bfootnote{? ष्व}}न्तर्भाव इ ‚{\tiny $_{7}$}‚ ति । तच्चासत् । अर्थान्तरगमनादेरपि हेतोरस‚{\tiny $_{lb}$}‚मर्थ एवमतिसम्भवात् । कुतः असमर्थस्य न्यायबलेन साध्यप्रतिपादने वादिन इति ‚{\tiny $_{lb}$}‚शेषः । मिथ्याप्रवृ ‚{\tiny $_{8}$}‚ त्तेरर्थान्तरगमनादिनेत्यभिप्रायः ॥ ४ ॥
	{\color{gray}{\rmlatinfont\textsuperscript{§~\theparCount}}}
	\pend% ending standard par
      ‚{\tiny $_{lb}$}‚

	  
	  \pstart \leavevmode% starting standard par
	उत्तरः पश्चाद् फलभावी स चासौ प्रतिज्ञासन्यासश्च तस्यापेक्षया किन्न किञ्चि‚{\tiny $_{lb}$}‚दित्यर्थः । अशक्तः परिच्छेदः सं ‚{\tiny $_{9}$}‚ \leavevmode\ledsidenote{\textenglish{61b/msK}} ख्ये येषां क्लीवप्रलापचेष्टितानां तानि तथा क्लीवा‚{\tiny $_{lb}$}‚दीनां प्रलापा येषां वादिनान्तेषां चेष्टितानि प्रतिज्ञासंन्यासादीनि वै किमुपन्यस्तैः \add{।} ‚{\tiny $_{lb}$}‚ क\add{ः} पुनरेवं सति दोष इत्याह । ‚{\color{DodgerBlue3}‚एवं ह्यतिप्रसङ्गः} ‚{\tiny $_{11b6}$}‚ स्यात् । एवमाद्यपीति ‚{\tiny $_{lb}$}‚मूर्च्छावेपथुत्रसत्त्वादीनामादिशब्देनावरोधः । तस्मादेतदप्यसम्बद्धं विद्वत्सदस्येवं ‚{\tiny $_{lb}$}‚प्रकारस्य स्थूलत्वादित्त्याभिप्रायः तदत्र ‚{\color{DodgerBlue3}‚भाविविक्तः} स्वयमाशंक्य किल प्रतिविधत्ते ‚{\tiny $_{lb}$}‚स्थूलत्वेनेदं निग्रहस्थानमिति चेत् । प्राश्निकप्रतिवादिसन्निधौ प्रतिज्ञातार्थापह्नव‚{\tiny $_{lb}$}‚\leavevmode\ledsidenote{\textenglish{97/s}} ङ्करोतीति । असम्बद्ध ‚{\tiny $_{3}$}‚ मुच्यते तन्नाभिप्रायापरिज्ञानात् । न ब्रूमो ध्वंसी शब्द इति ‚{\tiny $_{lb}$}‚किन्तु संयोगविभागाभ्यां न व्यक्त इत्ययं प्रतिज्ञातार्थ इत्याह सामान्यस्य च स्वाश्रय‚{\tiny $_{lb}$}‚व्यङ्ग्य ‚{\tiny $_{4}$}‚ त्वात् विवादाभाव इति । निग्रहस्थानन्तु पूर्वमप्रतिज्ञातार्थत्वात् । अनैकान्ति‚{\tiny $_{lb}$}‚कदोषेण प्रतिषेधे हेतौ प्रतिज्ञातार्थापह्नवङ्करोतीति निगृह्यत इति ‚{\tiny $_{5}$}‚ तत्रवाच्यं ‚{\tiny $_{lb}$}‚यदि वादी साकांक्ष एवान्तराले केनचिद् दुर्व्विदग्धेनानैकान्तिकदोषेण चोदितः ‚{\tiny $_{lb}$}‚सन्प्रतिज्ञातार्थफलीकरणेन स्वाभिप्रायमाविष्करोति । तदा ‚{\tiny $_{6}$}‚ न्योपि न कश्चि‚{\tiny $_{lb}$}‚द्दोषः । किमङ्ग पुनः प्रतिज्ञासंन्यासः । अथ निराकांक्षः सन् पश्चाच्चोदितः ‚{\tiny $_{lb}$}‚प्रतिज्ञां विशिनष्टि । तदप्यनैकान्तिकदोषेणैव निगृह्यत इति कि\edtext{}{\lemma{कि}\Bfootnote{\href{http://sarit.indology.info/?cref=nbh.259-60}{न्यायभाष्ये २५९-६०} अल्पभेदेन ।}}मुत्तरप्रतिज्ञा‚{\tiny $_{lb}$}‚संन्यासापेक्षयेति न किञ्चित्परिहृतं किञ्च स्फुटमिदं प्रतिज्ञान्तरेन्तर्भवतीति नः ‚{\tiny $_{lb}$}‚पृथग्वाच्यमिति ॥ ४ ॥
	{\color{gray}{\rmlatinfont\textsuperscript{§~\theparCount}}}
	\pend% ending standard par
      ‚{\tiny $_{lb}$}‚

	  
	  \pstart \leavevmode% starting standard par
	\hphantom{.}‚{\color{DodgerBlue3}‚अविशेषोक्ते हेतावि} त्यादि सूत्रं ‚{\tiny $_{8}$}‚ अत्र निदर्शनमुदाहरणमित्यर्थः । ‚{\color{DodgerBlue3}‚कापिलः} ‚{\tiny $_{lb}$}‚ प्रमाणयति प्रधानसिद्धिप्रत्याशया । एकप्रकृतीदं व्यक्तं व्यक्तादिपरिमितत्वाद् घटश‚{\tiny $_{lb}$}‚रावादिवदिति । एका प्र ‚{\tiny $_{9}$}‚ \leavevmode\ledsidenote{\textenglish{62a/msK}} कृतिरस्येति विग्रहः । प्रकृतिरुपादानकारणं । या च ‚{\tiny $_{lb}$}‚किल सा प्रकृतिर्विकारग्रामस्य तत्प्रधानमितीयमलीकप्रत्यासा\edtext{}{\lemma{तत्प्रधानमितीयमलीकप्रत्यासा}\Bfootnote{? शा}}साङ्ख्यस्या‚{\tiny $_{lb}$}‚\leavevmode\ledsidenote{\textenglish{98/s}} परिमाणञ्चतुरस्रम्परिमण्डलमित्यादि । मृत्पूर्व्वकाणामित्यन्वयमाह । अस्य हेतो‚{\tiny $_{lb}$}‚र्व्यभिचारेण प्रत्यवस्थानं प्रतिवादिना क्रियते । नानाप्रकृतीनाङ्गवाश्वादीनामेक‚{\tiny $_{lb}$}‚प्रकृतीनाञ्च कुम्भोदञ्चना ‚{\tiny $_{2}$}‚ दीनान्दृष्टम्परिमाणमित्येवं प्रत्यवस्थिते सति प्रति‚{\tiny $_{lb}$}‚वादिनि । यदि वा प्रत्यवस्थितः प्रतिषिद्धः प्रधानवाद्याह । एकप्रकृतिसमन्वये सति ‚{\tiny $_{lb}$}‚परिमाणदर्शना ‚{\tiny $_{3}$}‚ दिति सविशेषणत्वाद्धेतोर्व्यभिचाराभाव इति मतिः । कथं पुनरेकप्र‚{\tiny $_{lb}$}‚कृतिसमन्वय इत्याह । सुखदुःख ‚{\color{DodgerBlue3}‚मोहसमन्वितं} हीदं व्यक्तं परिमितं गृह्य ‚{\tiny $_{4}$}‚ ते । सर्व्वत्र ‚{\tiny $_{lb}$}‚तत्कार्यदर्शनादित्याकूतं । तथाहि सुखबहुलानाम्प्रसादलाघवप्रसवाभिष्वङ्गाद्धर्ष ‚{\tiny $_{lb}$}‚प्रीतयः कार्यं । रजोबहुलानां शोषतापभेदस्तं ‚{\tiny $_{5}$}‚ भोद्वेगापद्वेषाः । तमोबहुलानां साव‚{\tiny $_{lb}$}‚रणमादनायध्वंसवीभत्सदैन्यगौरवाणि । एतानि च सर्व्वाणि सर्व्वत्रैव यथोत्कर्षाप‚{\tiny $_{lb}$}‚कर्षभेदमुपलभ्यन्ते । त ‚{\tiny $_{6}$}‚ स्मात्त्रैगुण्यप्रकृतीदं विश्वं । तदिदमित्यादिना निग्रहस्थानत्वे ‚{\tiny $_{lb}$}‚कारणमाह । ‚{\color{DodgerBlue3}‚अत्रापी} त्याद्यस्य प्रतिषेधः सुज्ञानः । अविरामादच्छेदादपरिसमाप्त‚{\tiny $_{lb}$}‚त्वात् ‚{\tiny $_{7}$}‚ साधनाभिधानस्येत्यर्थः ॥ ० ॥
	{\color{gray}{\rmlatinfont\textsuperscript{§~\theparCount}}}
	\pend% ending standard par
      ‚{\tiny $_{lb}$}‚

	  
	  \pstart \leavevmode% starting standard par
	यथोक्तलक्षण इत्येकाधिकरणौ विरुद्धौ धर्माविति पक्षप्रतिपक्षलक्षणं स्म‚{\tiny $_{lb}$}‚रयति । परिग्रहे वादिप्रतिवादिभ्यां कृते सति हेतुतः ‚{\tiny $_{8}$}‚ साध्यसिद्धौ प्रकृतायां ‚{\tiny $_{lb}$}‚हेतुवसा\edtext{}{\lemma{हेतुवसा}\Bfootnote{? शा}}त्साध्यसिद्धिरित्येतस्मिन्प्रकरणे सति प्रकृतोर्थः शब्दनित्यत्वं । तेना‚{\tiny $_{lb}$}‚सङ्गतत्वात् । तदसम्बद्धत्त्वात्तदनुपकारकत्वादित्यर्थः \add{।} तथा ‚{\tiny $_{9}$}‚ \leavevmode\ledsidenote{\textenglish{62b/msK}} हि विनापिरूपसि‚{\tiny $_{lb}$}‚द्ध्या प्रातिपदिकादिव्याख्यानं यथा कथञ्चित्प्रतिपादितादर्थादेवार्थः सिध्यति । ‚{\tiny $_{lb}$}‚न्याय्यमेतदिति स्वमतेनाविरुद्धत्वादभ्यनुजानाति । कदा च पूर्वो ‚{\tiny $_{1}$}‚ त्तरपक्षवादिनो‚{\tiny $_{lb}$}‚र्न्याय्यं निग्रहस्थानमित्याह । प्रतिपादिते दोषे सति वादिप्र\add{ति}वादिभ्यामन्यो‚{\tiny $_{lb}$}‚न्यमसाधनाङ्गवचनमेतददोषद्भावनञ्च भवेदिति अन्यथा न ह्य ‚{\tiny $_{2}$}‚ \add{न}‚{\tiny $_{lb}$}‚\leavevmode\ledsidenote{\textenglish{99/s}} योरेकस्यापि जयपराजयावित्युक्तं । प्रकृतं परित्यज्येति न्याय्यतामेवास्य ‚{\tiny $_{lb}$}‚प्रतिपादयति । प्रकृतमत्र साध्यसाधनहेत्वभिधानं तदकृत्वेति उपन्यस्ते दोषे ‚{\tiny $_{3}$}‚ न ‚{\tiny $_{lb}$}‚समर्थनं । ‚{\color{DodgerBlue3}‚अपरस्य} ‚{\tiny $_{13a7}$}‚ रूपसिध्यादेः । अतन्नान्तरीयकस्यापीति । उपन्यस्त‚{\tiny $_{lb}$}‚सावनसमर्थनाङ्गस्येत्यर्थः । अपरस्य नामादिव्याख्यानादेरुपक्षे ‚{\tiny $_{4}$}‚ पः पराजयस्थान‚{\tiny $_{lb}$}‚मिति वर्त्तते ॥ ४ ॥
	{\color{gray}{\rmlatinfont\textsuperscript{§~\theparCount}}}
	\pend% ending standard par
      ‚{\tiny $_{lb}$}‚

	  
	  \pstart \leavevmode% starting standard par
	\hphantom{.}वर्णक्रमनिदेश\add{व}न्निरर्थकं \href{http://sarit.indology.info/?cref=ns\%C5\%AB.2.1.8}{न्या० सू० २।१।८ } यत्र वर्णा एव केवलं क्रमेण ‚{\tiny $_{lb}$}‚निर्दिश्यन्ते । न पदन्नापि वाक्यं । अर्थान्तरे किलाप्र ‚{\tiny $_{5}$}‚ कृतार्थकथनमिह वर्णमात्रोच्चा‚{\tiny $_{lb}$}‚रणमिति शेषः ॥ असम्वद्धतामेवाह । ‚{\color{DodgerBlue3}‚नहि वर्ण्णक्रमनिर्देशादेव} ‚{\tiny $_{13a9}$}‚ केवलादानर्थ‚{\tiny $_{lb}$}‚क्यमपि तु यदेव किञ्चिदसाध ‚{\tiny $_{6}$}‚ नाङ्गस्यासिद्धविरुद्धादेः शब्दरूपसिध्यादेश्च वचन‚{\tiny $_{lb}$}‚न्तदेवानर्थकं । किं कारणं । साध्यसिद्ध्युपयोगिनोऽभिधेयस्याभावात् । साध्यसिद्ध्यु‚{\tiny $_{lb}$}‚पयोगिनोऽभावे ‚{\tiny $_{7}$}‚ पि कस्यान्यत्प्रयोजनमस्तीत्यपि न मन्तव्यं इति कथयति । ‚{\tiny $_{lb}$}‚निष्प्रयोजनत्वाच्चेति । साध्यसिद्धेरेव प्रस्तुतत्वादन्यप्रयोजनवत्वेपि आनर्थक्यमेव ‚{\tiny $_{lb}$}‚तत्र प्रस्ता ‚{\tiny $_{8}$}‚ व इत्यभिप्रायः । तस्मात्प्रकारविशेषोपादानवर्णक्रमनिर्देशवदित्यसम्बद्धं । ‚{\tiny $_{lb}$}‚परः प्राह । न साध्यसिद्धौ यदनर्थकमनङ्गन्तन्निरर्थकमभिप्रेतमपि ‚{\tiny $_{9}$}‚ \leavevmode\ledsidenote{\textenglish{63a/msK}} तु यस्य वचनस्य ‚{\tiny $_{lb}$}‚काकवासितादेरिव नैव कश्चिदर्थः । तथा च नार्थान्तरापार्थकादीनामनेनैव संग्रहस्तत्र‚{\tiny $_{lb}$}‚\leavevmode\ledsidenote{\textenglish{100/s}} कस्यचिदर्थलेशस्य सद्भावात् । ‚{\color{DodgerBlue3}‚आचार्य} आह ‚{\tiny $_{1}$}‚ \add{।} ‚{\color{DodgerBlue3}‚यस्य कस्य चिद} ‚{\tiny $_{13b3}$}‚ प्यादिनोपि ‚{\tiny $_{lb}$}‚निरर्थकाभिधाने वाहित इव किन्न निग्रहो भवति । कथं स्यादित्याह \add{।} ‚{\color{DodgerBlue3}‚निग्रहनिमित्त ‚{\tiny $_{lb}$}‚त्तस्य} निरर्थकाभिधानस्य वाद्यवादिनोरवि ‚{\tiny $_{2}$}‚ शेषात् । नेति परन्तस्य वादिन इह ‚{\tiny $_{lb}$}‚वादप्रकरणे । आयातमित्याचार्यः । तस्य तेनैव निरर्थकाभिधानेन । तत्रैवं स्थिते ‚{\tiny $_{lb}$}‚वादे तुल्यं । सर्वस्यासाधनाङ्ग ‚{\tiny $_{3}$}‚ वादिनो निरर्थकाभिधायित्वमित्यध्याहर्त्तव्यं । क्व‚{\tiny $_{lb}$}‚चित्तवेतिपाठः । तत्र नोपस्कारेण किञ्चित् । अनेनैव निरर्थकाभिधानेन । प्रत्यु‚{\tiny $_{lb}$}‚च्यते । यस्य नैव ‚{\tiny $_{4}$}‚ कश्चिदर्थ इति । एतदप्यसम्बद्धं । यस्मान्न च वर्णक्रमनिर्देशोपि ‚{\tiny $_{lb}$}‚निरर्थकः क्वचित्प्रकरणे प्रत्याहारादावर्थवत्वाच्च । तस्मादत्रैव वादेस्य वर्ण्णक्र ‚{\tiny $_{5}$}‚ म‚{\tiny $_{lb}$}‚स्यानर्थक्यं । तच्चार्थान्तरादेरपि तुल्यमिति चित्तं कक्कङ्पिङ्गितमित्यत्रादिशब्देन ‚{\tiny $_{lb}$}‚उत्प्लुत्य गमनं तालदाननृत्त\edtext{}{\lemma{तालदाननृत्त}\Bfootnote{? नृत्य}}आदीनाङ्गहणं ॥ ४ ॥
	{\color{gray}{\rmlatinfont\textsuperscript{§~\theparCount}}}
	\pend% ending standard par
      ‚{\tiny $_{lb}$}‚

	  
	  \pstart \leavevmode% starting standard par
	त्रिर ‚{\tiny $_{6}$}‚ भिहितमिति त्रिवचनङ्कार्यमिति न्यायत्वं दर्शयति । सकृदुक्तं स्पष्टार्थ‚{\tiny $_{lb}$}‚मपि कदाचिन्न ज्ञायत इति त्रिरुच्चारणङ्कार्यं । कस्मात्पुनः पदवाक्यप्रमाणवि ‚{\tiny $_{7}$}‚ द्‚{\tiny $_{lb}$}‚\leavevmode\ledsidenote{\textenglish{101/s}} भिर्वाक्यन्न ज्ञायत इत्याह । ‚{\color{DodgerBlue3}‚क्लिष्टशब्दमित्यादि} । क्लिष्टशब्दं मनागुच्चारितत्वात् । ‚{\tiny $_{lb}$}‚अपशब्दत्वादित्यपरे । कस्मादेवं प्रयुक्तमित्याह । ‚{\color{DodgerBlue3}‚असामर्थ्यसम्वरणा ‚{\tiny $_{8}$}‚ ये ‚{\tiny $_{13b9}$}‚ ‚{\tiny $_{lb}$}‚ति} । स्पष्टार्थस्य प्रयोगे दूषणम्वक्ष्यतीति भयात्प्रयुंक्ते । इदञ्च साधनदूषण‚{\tiny $_{lb}$}‚वादिनोः समानं । दूषणवाक्यमपि ह्येवंभूतनिग्रहप्राप्तिकारणं । नेदं ‚{\tiny $_{9}$}‚ \leavevmode\ledsidenote{\textenglish{63b/msK}} निरर्थकाद्- ‚{\tiny $_{lb}$}‚भिद्यते । तथा हि श्लिष्टशब्दादिभिः प्रकृतार्थसम्बद्धङ्गमकमेव ब्रूयात् । एतद्विप‚{\tiny $_{lb}$}‚रीतम्वा । प्राक्तने प्रकारे नास्यासामर्थ्यन्तत्र तु परिषदादयो जाड्या ‚{\tiny $_{1}$}‚ त्तदुक्तन्न ‚{\tiny $_{lb}$}‚प्रतिपद्यंत इति नेयता विद्वान्वादी निग्रहमर्हति ।
	{\color{gray}{\rmlatinfont\textsuperscript{§~\theparCount}}}
	\pend% ending standard par
      ‚{\tiny $_{lb}$}‚
	  \bigskip
	  \begingroup
	
	    
	    \stanza[\smallbreak]
	  \flagstanza{\tiny\textenglish{...37}}{\normalfontlatin\large ``\qquad}वक्तुरेव हि तज्जाड्यं यच्छ्रोत्रा नावबुद्ध्यतें । \add{३७}{\normalfontlatin\large\qquad{}"}\&[\smallbreak]
	  
	  
	  
	  \endgroup
	‚{\tiny $_{lb}$}‚

	  
	  \pstart \leavevmode% starting standard par
	ततोसौ निग्रहार्ह एवेत्याकूतवानाह परः । परिषत् ‚{\tiny $_{2}$}‚ प्रज्ञामिति । ‚{\color{DodgerBlue3}‚न्यायवादिन} ‚{\tiny $_{lb}$}‚ ‚{\tiny $_{13b9}$}‚ इति परिहरति । न्यायवादिनः उक्तमिति सम्बन्धः । वादी तु जाड्या‚{\tiny $_{lb}$}‚त्परिषदादेरविज्ञातसाधनसामर्थ्य इति कृत्वा विजेता न स्यात् । परिषत्प्रतिवादि‚{\tiny $_{lb}$}‚प्रत्यायनेन जयव्यवस्थापनात् । अविज्ञातं प्रतिपादनसामर्थ्यं परिषत्प्रतिवादिभ्यां ‚{\tiny $_{lb}$}‚यस्येति कार्यं । द्वितीयन्तु विकल्पमधि ‚{\tiny $_{4}$}‚ कृत्याह । ‚{\color{DodgerBlue3}‚असम्बद्धाभिधाने निरर्थकमेवे} ‚{\tiny $_{lb}$}‚ ‚{\tiny $_{14a1}$}‚ ‚{\color{DodgerBlue3}‚ति} ॥ ४ ॥
	{\color{gray}{\rmlatinfont\textsuperscript{§~\theparCount}}}
	\pend% ending standard par
      ‚{\tiny $_{lb}$}‚

	  
	  \pstart \leavevmode% starting standard par
	\hphantom{.}‚{\color{DodgerBlue3}‚अनेकस्य पदस्येति} । यदानीमसम्बद्धार्थप्रतिपादकत्त्वे वाक्यार्थप्रतिपादकत्वं ‚{\tiny $_{lb}$}‚निराकरोति ‚{\tiny $_{5}$}‚ वाक्यस्यासम्बद्धार्थप्रतिपादकत्वे प्रकरणाध्यायप्रतिपत्यभावः । ‚{\tiny $_{lb}$}‚समुदायप्रतिपत्यभावाच्च निग्रहस्थानं । उदाहरणं दश डा\edtext{}{\lemma{डा}\Bfootnote{? दा}}डिमाः षडपू ‚{\tiny $_{6}$}‚ पाः ‚{\tiny $_{lb}$}‚कुण्डमजाजिनं पललपिण्डं । अथ रौरुकमेतत् कुमार्यः स्फैयकृतस्य पिता प्रतिशीन ‚{\tiny $_{lb}$}‚इति अत्र च ‚{\color{DodgerBlue3}‚भारद्वाजेन} निरर्थकापार्थकयोरभेद इत्याशङ्कय ‚{\tiny $_{7}$}‚ प्रतिविहितं तत्र हि ‚{\tiny $_{lb}$}‚वर्ण्णमात्रमिह यदान्यसम्बद्धानीति । तदेवाचार्योप्युपक्षिपति । ‚{\color{DodgerBlue3}‚इदं किले} ‚{\tiny $_{14a2}$}‚ त्या‚{\tiny $_{lb}$}‚दिना । असम्बद्धा वर्ण्णा यस्मिन्निरर्थक इति विग्रहः । कि ‚{\tiny $_{8}$}‚ ल शब्दोऽनभिमतत्व‚{\tiny $_{lb}$}‚\leavevmode\ledsidenote{\textenglish{102/s}} प्रदर्शनार्थः । अनभिमतत्वमेवाह । नन्वयं पदानामसम्बन्धादपार्थकवदसम्बन्ध‚{\tiny $_{lb}$}‚वाक्यमपि निरर्थकात् पृथग् वाच्यं स्यात् । स्यात्मतमपार्थकं ‚{\tiny $_{9}$}‚ \leavevmode\ledsidenote{\textenglish{64a/msK}} नैवासम्बद्धपदार्था‚{\tiny $_{lb}$}‚सम्बद्धवाक्यार्थयोः सङ्गृहीतत्वात् पृथग् न वाच्यमित्यत उच्यते । ‚{\color{DodgerBlue3}‚नोभय‚{\tiny $_{lb}$}‚सङ्ग्रहाद} ‚{\tiny $_{14a3}$}‚ पार्थकं युक्तं । कस्मादसम्बद्धपदार्थेनापार्थकेनैवासम्बद्धवा ‚{\tiny $_{1}$}‚ क्यस्येव ‚{\tiny $_{lb}$}‚निरर्थकस्यापि वर्ण्णक्रममात्रलक्षणस्य सङ्ग्रहप्रसङ्गात् । अथोच्यते । निरर्थकं किमु‚{\tiny $_{lb}$}‚च्यते । यस्यार्थ एव नास्ति केवलं वर्ण्णक्रममात्रं । असम्बद्धपद ‚{\tiny $_{2}$}‚ वाक्ययोस्तु साध्य‚{\tiny $_{lb}$}‚सिद्ध्यनुपयोगेपि न सर्वथा नैरर्थक्यमतोऽर्थतत्वे साम्यात् द्वयोरेवैकीकरणमित्यत ‚{\tiny $_{lb}$}‚आह । एवं विधाच्चेत्यादि । कपोलवादितादीनामपि ‚{\tiny $_{3}$}‚ पृथगभिधानप्रसङ्ग ‚{\tiny $_{lb}$}‚इत्यत्रातिप्रसङ्ग उक्तः । नहि किञ्चित्मात्रेण विशेषो न शक्यते क्वचित्प्रदर्शयितु‚{\tiny $_{lb}$}‚मित्यभिसन्धिः अथ निरर्थकापार्थकयोः ‚{\tiny $_{4}$}‚ सङ्ग्रहनिर्देशदोषं भेदनिर्देशे च गुणम्प‚{\tiny $_{lb}$}‚श्यताऽ ‚{\color{DodgerBlue3}‚क्षपादेन} न सङ्ग्रहनिर्देशः कृत इति मन्यसे । न साधु मन्यस इत्याह । ‚{\color{DodgerBlue3}‚न च ‚{\tiny $_{lb}$}‚सङ्ग्रह} ‚{\tiny $_{14a4}$}‚ इत्यादि ॥ ४ ॥ 
	{\color{gray}{\rmlatinfont\textsuperscript{§~\theparCount}}}
	\pend% ending standard par
      ‚{\tiny $_{lb}$}‚

	  
	  \pstart \leavevmode% starting standard par
	यथा लक्षणमर्थवसा\edtext{}{\lemma{लक्षणमर्थवसा}\Bfootnote{? शा}}दित्यर्थः सामर्थ्यं । अनुपदर्शिते हि विषये निर्विषया ‚{\tiny $_{lb}$}‚साधनप्रवृत्तिर्मा भूदिति साध्यनिर्देशलक्षणा प्रतिज्ञा पूर्व्वमुच्य ‚{\tiny $_{6}$}‚ ते । तदनन्तर मुदाहरण‚{\tiny $_{lb}$}‚साधर्म्यांत्साध्यसाधनं हेतु रित्येवं लक्षणो हेतुस्तत्साधनायोच्यते । ततो हेतोर्वहिर्व्या‚{\tiny $_{lb}$}‚प्तिप्रदर्शनार्थंसाध्यसाधर्म्यात्तद्धर्मभाविदृ ‚{\tiny $_{7}$}‚ ष्टान्त उदाहरणमि \href{http://sarit.indology.info/?cref=ns\%C5\%AB.1.1.36}{न्या० सू० १।१।३६} ‚{\tiny $_{lb}$}‚ त्येवं लक्षणमुदाहरणं । ततः प्रतिबिंबनार्थं साध्यधर्मिणि सम्भवप्रदर्शनार्थम्वा‚{\tiny $_{lb}$}‚उदाहरणापेक्षस्तथेत्युपसंहारो न तथेति वेति ‚{\tiny $_{8}$}‚ साधनस्योपनय\href{http://sarit.indology.info/?cref=ns\%C5\%AB.1.1.38}{न्या० सू० १।१।३८ } ‚{\tiny $_{lb}$}‚ इत्येवंलक्षण उपनयः । तत उत्तरकालं सर्व्वावयवपरामर्षेण\edtext{}{\lemma{सर्व्वावयवपरामर्षेण}\Bfootnote{? र्शेन}}विपरीतप्रसङ्ग‚{\tiny $_{lb}$}‚निवृत्यर्थं हेत्वपदेशात् प्रतिज्ञायाः पुनर्वचनं निग ‚{\tiny $_{9}$}‚ \leavevmode\ledsidenote{\textenglish{64b/msK}} मनमि \href{http://sarit.indology.info/?cref=ns\%C5\%AB.1.1.39}{न्या० सू० १।१।३९ } त्येवं ‚{\tiny $_{lb}$}‚लक्षणं निगमनमिति । अयमसौ यथालक्षणमर्थवसा\edtext{}{\lemma{यथालक्षणमर्थवसा}\Bfootnote{? शा}}त्क्रमः । तथाहि लोकेपि ‚{\tiny $_{lb}$}‚पूर्व्वङ्कार्यं मृत्पिण्डाद्युपादीयते पश्चात्तु करणञ्चक्रदण्डादिकमिति ‚{\tiny $_{1}$}‚ न्यायः । ‚{\tiny $_{lb}$}‚तत्रैतस्मिनक्रम\edtext{}{\lemma{तत्रैतस्मिनक्रम}\Bfootnote{? न्क्रमे}}न्यायतः । स्थितेऽवयवानां प्रतिज्ञादीनां विपर्ययेणाभिधानं ‚{\tiny $_{lb}$}‚ \leavevmode\ledsidenote{\textenglish{103/s}} निग्रहस्थानं । यथा घटवत्कृतकत्वादनित्य इति । नैवमपि सिद्धेरिति ‚{\color{DodgerBlue3}‚भार ‚{\tiny $_{2}$}‚ द्वाजः} ‚{\tiny $_{lb}$}‚ स्वयमेवाशङ्क्य परिहरति । ‚{\color{DodgerBlue3}‚न प्रयोगापेतशब्दवदेतत्स्यादिति} अनेनेति गोणीपदेन । ‚{\tiny $_{lb}$}‚यथा \add{।}
	{\color{gray}{\rmlatinfont\textsuperscript{§~\theparCount}}}
	\pend% ending standard par
      ‚{\tiny $_{lb}$}‚
	  \bigskip
	  \begingroup
	
	    
	    \stanza[\smallbreak]
	  \flagstanza{\tiny\textenglish{...38}}{\normalfontlatin\large ``\qquad}अम्बम्बिति यथा वालः शिक्ष्यमाणः प्रभासते\edtext{}{\lemma{प्रभासते}\Bfootnote{? षते ।}}&‚{\tiny $_{lb}$}‚अव्यक्तं ‚{\tiny $_{3}$}‚ तद्विदान्तेन व्यक्ते भवति निश्चयः । \add{३८}{\normalfontlatin\large\qquad{}"}\&[\smallbreak]
	  
	  
	  
	  \endgroup
	‚{\tiny $_{lb}$}‚

	  
	  \pstart \leavevmode% starting standard par
	तथा किल गोण्यादयः शब्दाः ते साधुष्वनुमाणे\edtext{}{\lemma{साधुष्वनुमाणे}\Bfootnote{? ने}}न प्रत्ययोत्पत्तिहेतव ‚{\tiny $_{lb}$}‚इति । तदेतदुन्मत्तकस्य वैया ‚{\tiny $_{4}$}‚ करणस्योन्मत्तकसंवर्ण्णनमुन्मत्तकेनो ‚{\color{DodgerBlue3}‚द्योतकरेण} संवर्ण्णनं ‚{\tiny $_{lb}$}‚यथा ह्येक उन्मत्तो द्वितीयमुन्मत्तकं सम्वर्ण्णयति तथा भूतमेतदपीति या ‚{\tiny $_{5}$}‚ वत् । यदि ‚{\tiny $_{lb}$}‚चोन्मत्तकस्यो ‚{\color{DodgerBlue3}‚द्योतकर} स्योन्मत्तकस्य वैयाकरणस्य सम्वर्ण्णनं । तथा हि शाब्दिक एव ‚{\tiny $_{lb}$}‚तावदुन्मत्तः प्रमाणविरुद्धवत्त्वाभिधायित्वात् । तत\add{ः} कु ‚{\tiny $_{6}$}‚ तस्तत्प्रक्रियायाः प्रमाण‚{\tiny $_{lb}$}‚चिन्ताया ज्ञापकत्वमित्यभिप्रेतं । कथम्पुनः शाब्दिकस्यायुक्ताभिधायित्वमित्याह । ‚{\tiny $_{lb}$}‚ ‚{\color{DodgerBlue3}‚यदि ‚{\tiny $_{14a4}$}‚ त्यादि} सुबोधं । स्त्रीशूद्रशब्दो मूर्खवचनः । ‚{\tiny $_{7}$}‚ यस्तु ‚{\color{DodgerBlue3}‚नक्क} शब्दं ‚{\color{DodgerBlue3}‚मुक्क} शब्द‚{\tiny $_{lb}$}‚मेव नासापर्यायम्वेत्ति । स कथमपशब्दाच्छब्दं साधुं प्रतिपद्यातः साधोः शब्दा‚{\tiny $_{lb}$}‚दर्थम्प्रतिपद्येत । किमुच्यते नैवासौ तथा विवोधम्प्रतिप ‚{\tiny $_{8}$}‚ द्यत इत्याह । ‚{\color{DodgerBlue3}‚दृष्टाचानुभय‚{\tiny $_{lb}$}‚वेदिनोपि ‚{\tiny $_{14b3}$}‚ सनका} देः प्रतीतिरिति तस्मान्न परम्परया प्रतीतिरर्थस्य । अयमत्र ‚{\tiny $_{lb}$}‚संक्षेपः । स्यादेवमसाधूनां साध्वनुमापकत्वम् । य ‚{\tiny $_{9}$}‚ \leavevmode\ledsidenote{\textenglish{65a/msK}} द्येषान्धूमादीनामिव त्रैरूप्यम्भवे- ‚{\tiny $_{lb}$}‚ \leavevmode\ledsidenote{\textenglish{104/s}} न्निश्चितं । तच्च न सम्भवति । यस्मादेतावदनुभयवेदिनः ‚{\color{DodgerBlue3}‚सनका} दयस्ते सन्तमपि ‚{\tiny $_{lb}$}‚व्याप्यव्यापकभावन्न प्रतिपद्यन्ते । न चासाव ‚{\tiny $_{1}$}‚ ज्ञातो गमको ज्ञापकत्वात् । येपि ‚{\tiny $_{lb}$}‚शब्दापशब्दप्रविभागकुशलास्तेप्यविद्यमानत्वादेव भावयन्ति । तथाह्यसाधूनां साधुभिः ‚{\tiny $_{lb}$}‚सम्बन्धस्तादात्म्यं कार्यकारणभावो वा ‚{\tiny $_{2}$}‚ भवेत । तदुभयविकलस्याव्यभिचारनियमा‚{\tiny $_{lb}$}‚भावात् । तत्र च तावन्न तादात्म्यमभ्युपेयं पारमार्थिकस्यैव भेदस्य स्फुटं प्रत्यक्षतः ‚{\tiny $_{lb}$}‚प्रतीतेः । शब्दवद ‚{\tiny $_{3}$}‚ साधोरप्यव्यतिरेकतो वाचकत्वप्रसङ्गाच्च । तदुत्पत्तिरपि दूरो‚{\tiny $_{lb}$}‚त्सारितेव । यतो नासाधवः साधुभ्यो जायन्ते । क\add{ा}रणगुणवक्तुकामतामात्रहेतुत्वा ‚{\tiny $_{4}$}‚‚{\tiny $_{lb}$}‚त्तेषां । न च तेषान्नित्यत्वङ्कादाचित्कोपलम्भतः । तत्वे वा सुतरान्तदुत्पत्तेरभावः ‚{\tiny $_{lb}$}‚सत्यपि वा व्याप्यव्यापकभावे तत्परिज्ञाने च पक्षधर्मत्ववैक ‚{\tiny $_{5}$}‚ ल्याच्चाक्षुषत्वादे‚{\tiny $_{lb}$}‚रिवासाधुभ्यो नानुमानं । नह्यत्र धर्मे विद्यते । यतः पक्षधर्मत्वं निष्पद्यते । नहि ‚{\tiny $_{lb}$}‚साधूनामेव धर्मित्वन्तेषामेवानुमीयमानत्वात् । न च धर्मिसा ‚{\tiny $_{6}$}‚ धनं युक्तिमतः । भावा‚{\tiny $_{lb}$}‚भावोभयधर्मस्यासिद्धविरुद्धानैकान्तिकदोषदुष्टत्वतः । कथं वा साधूनां तत्धर्मत्वं । ‚{\tiny $_{lb}$}‚नहि तत्काले ते सन्ति । असताञ्च धर्मित्वं वाचकत्वं ‚{\tiny $_{7}$}‚ चेति सुभाषितं । किमुच्यते ‚{\tiny $_{lb}$}‚पुरुषो धर्मी साधुशब्दविवक्षा साध्यधर्मः पक्षधर्मश्चासाधुरिति तदप्यसम्बद्धं । ‚{\tiny $_{lb}$}‚व्याप्यव्यापकभावाभावादेव । यस्मान्न च ‚{\tiny $_{8}$}‚ \leavevmode\ledsidenote{\textenglish{65b/msK}} गोणीशब्दप्रयोगकाले गोशब्दविवक्षामु‚{\tiny $_{lb}$}‚पलभामहे । अथ प्रत्यवस्थीयते । यथा पक्षधर्मत्वादिवैकल्येप्यव्यक्तं । बालवचोव्यक्त‚{\tiny $_{lb}$}‚मनुमापयति । तथैवासाधवोपि ‚{\tiny $_{1}$}‚ साधूनिति \add{।} तदयुक्तं तत्रापि तुल्यपर्यनुयोगत्वा‚{\tiny $_{lb}$}‚त् । वयन्तु प्रतिपद्यामहे साक्षादेव तस्मादप्यव्यक्तान्मात्राद्यर्थः प्रतीयत इति । ‚{\tiny $_{lb}$}‚तत्र संज्ञासंज्ञिसम्बन्धस्याननुभूत ‚{\tiny $_{2}$}‚ त्वादयुक्ताप्रतीतावित्यपि न मन्तव्यं । अनादि‚{\tiny $_{lb}$}‚मति संसारे व्यवहारपरम्परायास्तथाभूतायाः सम्बन्धस्योल्लिङ्गितत्वात् । तथाहि ‚{\tiny $_{lb}$}‚न गवादिशब्दानामपि प्रा ‚{\tiny $_{3}$}‚ यः शृङ्गङ्ग्राहिकयार्थनियमः सङ्केत्यतेपि तु व्यवहार‚{\tiny $_{lb}$}‚पारम्पर्यतो विदग्धा निश्चिन्वन्ति । तच्चेहापि समानमेव । तस्मादेतदरण्यरुदितं । ‚{\tiny $_{lb}$}‚ 
	    \pend% close preceding par
	  
	    
	    \stanza[\smallbreak]
	  \flagstanza{\tiny\textenglish{...39}}{\normalfontlatin\large ``\qquad}अम्बम्विति यथा बालः शिक्ष्यमाणः प्रभासते ।&‚{\tiny $_{lb}$}‚अव्यक्तन्तद्विदान्तेन व्यक्तेन भवति निश्चयः ॥ \add{३९}{\normalfontlatin\large\qquad{}"}\&[\smallbreak]
	  
	  
	  
	    \pstart  \leavevmode% new par for following
	    \hphantom{.}
	   
	    \pend% close preceding par
	  
	    
	    \stanza[\smallbreak]
	  \flagstanza{\tiny\textenglish{...40}}{\normalfontlatin\large ``\qquad}एवं साधौ प्रयोक्तव्ये यो यद्भ्रंशः प्रयुज्यते ।&‚{\tiny $_{lb}$}‚तेन साधु व्यवहितः कश्चिद ‚{\tiny $_{5}$}‚ र्थोवसीयत \add{४०}{\normalfontlatin\large\qquad{}"}\&[\smallbreak]
	  
	  
	  
	    \pstart  \leavevmode% new par for following
	    \hphantom{.}
	   इति ॥
	{\color{gray}{\rmlatinfont\textsuperscript{§~\theparCount}}}
	\pend% ending standard par
      ‚{\tiny $_{lb}$}‚

	  
	  \pstart \leavevmode% starting standard par
	\hphantom{.}यदप्यभ्यधायि‚{\color{DodgerBlue3}‚कुमारिलेन ।} ‚{\tiny $_{lb}$}‚ \leavevmode\ledsidenote{\textenglish{105/s}} 
	    \pend% close preceding par
	  
	    
	    \stanza[\smallbreak]
	  \flagstanza{\tiny\textenglish{...41}}{\normalfontlatin\large ``\qquad}गोशब्देऽवस्थितेस्माकन्तदशक्तिजकारिता ।&‚{\tiny $_{lb}$}‚गाव्यादेरपि गोबुद्धिर्मूलशब्दानुसारिणी \add{४१}{\normalfontlatin\large\qquad{}"}\&[\smallbreak]
	  
	  
	  
	    \pstart  \leavevmode% new par for following
	    \hphantom{.}
	   ति ॥
	{\color{gray}{\rmlatinfont\textsuperscript{§~\theparCount}}}
	\pend% ending standard par
      ‚{\tiny $_{lb}$}‚

	  
	  \pstart \leavevmode% starting standard par
	तस्यापीद ‚{\tiny $_{6}$}‚ मेव प्रतिविधानमिदञ्च सर्वमागूर्य्य निगमयति । न ‚{\color{DodgerBlue3}‚परम्परया ‚{\tiny $_{lb}$}‚प्रतीति} रिति । अत्रैवोपचयमाह । अर्थे प्रतिपादनायासमर्थस्यासाधोः शब्देपि ‚{\tiny $_{lb}$}‚साधौ प्रतीतिज ‚{\tiny $_{7}$}‚ ननासामर्थ्याच्च । तत्रैतत्स्यान्न वयमसाधूनामर्थेषु प्रतीति‚{\tiny $_{lb}$}‚जनकत्वं निराकुर्मः । किन्तु वाचकत्वं । शब्दे चासाधुः प्रतीतिजनक एव ‚{\tiny $_{lb}$}‚न वाचक एव इत्यत आह । न ‚{\color{DodgerBlue3}‚ह्य ‚{\tiny $_{8}$}‚ \leavevmode\ledsidenote{\textenglish{66a/msK}} र्थेपि शब्दस्य वाचकत्वमन्यदेवे} ‚{\tiny $_{14b4}$}‚- ‚{\tiny $_{lb}$}‚त्यादि । यद्यसाधोरर्थे प्रतीतिजनकत्वमिष्यते । तदैतावता वयमाहितपरितोषाः । ‚{\tiny $_{lb}$}‚किमस्माकमभिधानान्तरकल्पितेन वाचक ‚{\tiny $_{1}$}‚ त्वेनेत्याकूतं । नैव तर्ह्यसावर्थप्रतीतिं ‚{\tiny $_{lb}$}‚जनयितुँ क्षमोऽपि तु शब्द एवेति चेदाह । ‚{\color{DodgerBlue3}‚अपशब्दश्चेदि ‚{\tiny $_{14b4}$}‚ ति} । अथोच्यते ‚{\tiny $_{lb}$}‚शब्देन तस्य स्वाभाविकः सम्बन्धो नार्थेन ततस्तमे ‚{\tiny $_{2}$}‚ व प्रतिपादयति नार्थन्तद्यथा ‚{\tiny $_{lb}$}‚स्वभावतश्चक्षूरूपं प्रकाश\add{य}ति न शब्दादीनत आह । ‚{\color{DodgerBlue3}‚अकृतसमयस्ये ‚{\tiny $_{14b5}$}‚ ‚{\tiny $_{lb}$}‚ त्यादि} । अदर्शनादिति । न ह्यप्रतीतसम्बन्धाः ‚{\color{DodgerBlue3}‚सिंहल} शब्दा आर्य ‚{\tiny $_{3}$}‚ जनव्यवहा‚{\tiny $_{lb}$}‚राय वर्त्तन्ते । समय एव तु जनयेत् प्रतीतिं । सामयिके च तत्र सम्बन्धे सोर्थेप्य‚{\tiny $_{lb}$}‚निवार्यः । समयवसा\edtext{}{\lemma{समयवसा}\Bfootnote{? शा}}दसाधुः साधौ वर्त्तमानोर्थ एव गवादौ ‚{\tiny $_{4}$}‚ किन्न ‚{\tiny $_{lb}$}‚प्रवर्तते । नहि किञ्चित्तथा दोषो गुणस्तु केवल इत्याह । ‚{\color{DodgerBlue3}‚एवं ही} त्या ‚{\tiny $_{14b6}$}‚ ‚{\tiny $_{lb}$}‚दि । एतदुक्तम्भवति । ये स्वभावतः प्रकाशका न ते समयमपेक्षन्ते । यथा चक्षुर्दी‚{\tiny $_{lb}$}‚पा ‚{\tiny $_{5}$}‚ दयो रूपादीनां । स्वभावतश्चापशब्दो यदि शब्दस्य प्रकाशको भवेत् । ततस्ते‚{\tiny $_{lb}$}‚नापि सम्बन्धोनापेक्षः स्यात् । अपेक्ष्यते च ततो नास्य शब्दे स्वा\add{भा}विकं सामर्थ्यं । ‚{\tiny $_{6}$}‚ ‚{\tiny $_{lb}$}‚तथा चेदमपि शक्यमनुमातुं । ये समयाक्षेक्ष\edtext{}{\lemma{समयाक्षेक्ष}\Bfootnote{? पेक्ष}}प्रवृत्तयस्ते सर्वत्र यथासमय‚{\tiny $_{lb}$}‚मनिवारितप्रसराः साक्षादेव प्रतिपादका भवन्ति । यथाकायविज्ञप्त्यादयः । त ‚{\tiny $_{7}$}‚ था ‚{\tiny $_{lb}$}‚चापशब्दा अपि समयापेक्षप्रवृत्तय इति सिद्धमेषामव्यवधानत एवार्थप्रति‚{\tiny $_{lb}$}‚ \leavevmode\ledsidenote{\textenglish{106/s}} पादकत्वमिति । विपर्ययदर्शनाच्चेत्युपचयान्तरं । तथाहि वृक्षोग्निरुत्पलमित्युक्ते ‚{\tiny $_{8}$}‚ ‚{\tiny $_{lb}$}‚ऽव्युत्पन्नधियो वालाः प्रश्नोपक्रमं सन्तिष्ठन्ते । कोयं वृक्ष इत्यादिना । ते चान्यस्य ‚{\tiny $_{lb}$}‚व्युत्पादनोपायस्याभावादपशब्दैरेव व्युत्पाद्यन्ते रुक्ख अग्गी उप्पलमिति ॥ त ‚{\tiny $_{9}$}‚ \leavevmode\ledsidenote{\textenglish{66b/msK}} देव‚{\tiny $_{lb}$}‚मत्रासाधव एव वाचका न साधवः सन्तोपीति विपर्ययो दृश्यते \add{।} अथ प्रतिपद्यसे ‚{\tiny $_{lb}$}‚धर्मसाधनता शब्दसंस्कारो यथोवतं ।
	{\color{gray}{\rmlatinfont\textsuperscript{§~\theparCount}}}
	\pend% ending standard par
      ‚{\tiny $_{lb}$}‚

	  
	  \pstart \leavevmode% starting standard par
	\hphantom{.}
	    \pend% close preceding par
	  
	    
	    \stanza[\smallbreak]
	  \flagstanza{\tiny\textenglish{...42}}{\normalfontlatin\large ``\qquad}शिष्टेभ्य आगमात् सिद्धं साधनो धर्मसाधनं &‚{\tiny $_{lb}$}‚अर्थप्रत्यायनाभेदे विपरीतास्त्वसाधव \add{४२}{\normalfontlatin\large\qquad{}"}\&[\smallbreak]
	  
	  
	  
	    \pstart  \leavevmode% new par for following
	    \hphantom{.}
	   इति । तथा
	{\color{gray}{\rmlatinfont\textsuperscript{§~\theparCount}}}
	\pend% ending standard par
      ‚{\tiny $_{lb}$}‚
	  \bigskip
	  \begingroup
	

	  
	  \pstart \leavevmode% starting standard par
	\hphantom{.}
	    \pend% close preceding par
	  
	    
	    \stanza[\smallbreak]
	  \flagstanza{\tiny\textenglish{...43}}मन्त्र\add{ो} हीनः स्वरतो वर्णतो\add{वा}मिथ्याप्रयुक्तो न तमर्थमाह ।&‚{\tiny $_{lb}$}‚स वाग्वज्रो यजमानं हिनस्ति यथेन्द्रशत्रुः स्वरतोपराधात्\edtext{}{\lemma{स्वरतोपराधात्}\Bfootnote{व्याकरणमहाभाष्ये पस्पशाह्निके ।}}। ‚{\tiny $_{2}$}‚ \add{४३}\&[\smallbreak]
	  
	  
	  
	    \pstart  \leavevmode% new par for following
	    \hphantom{.}
	   ‚{\tiny $_{lb}$}‚ते सुरा हेऽलयो\add{हे}ऽलय इत्युक्तवंतः परावभूवुः । एकोपि शब्दः सम्यक्प्रयुक्तः सुकृ‚{\tiny $_{lb}$}‚तिनां लोकङ्गमयति । आहिताग्निरपशब्दमभिधाय प्रायश्चित्तीयामिष्टिं नि ‚{\tiny $_{3}$}‚ र्वपे‚{\tiny $_{lb}$}‚ \add{दि}त्यादि ।
	{\color{gray}{\rmlatinfont\textsuperscript{§~\theparCount}}}
	\pend% ending standard par
      
	  \endgroup
	

	  
	  \pstart \leavevmode% starting standard par
	\href{http://sarit.indology.info/?cref=MaBh\%C4\%81.1}{महाभाष्ये आह्निके १} इदमपसारयति ‚{\color{DodgerBlue3}‚न धर्मसाधनता} ‚{\tiny $_{14b9}$}‚ ‚{\tiny $_{lb}$}‚शब्दानां संस्कार इति वर्तते किङ्कारणमित्याह । ‚{\color{DodgerBlue3}‚मिथ्यावृत्तिचोदनेभ्य} ‚{\tiny $_{14b9}$}‚ ‚{\tiny $_{lb}$}‚इत्यादि । मिथ्यावृत्तिश्चोद्यते यैरि ‚{\tiny $_{4}$}‚ ति कार्यं । यथा ह्यस्याभिनवविद्रुमाङ्कुरप्रकरा‚{\tiny $_{lb}$}‚भिरामकिशलयमञ्जुमञ्जरीराजीविराजिततरोरशोकवनस्पतेरधः शयितस्य द्विज‚{\tiny $_{lb}$}‚न्मनो नीलनीरज ‚{\tiny $_{5}$}‚ नीलतातिशायिना मण्डलाग्रेण शिरश्छित्त्वेत्युक्तेपि भवत्येव ब्रह्मह‚{\tiny $_{lb}$}‚त्यया सम्बन्धः प्रयोजकस्य । अन्येभ्य इत्यसम्भूतेभ्यो विपर्ययेण सम्यक्त्ववृत्तिचोद‚{\tiny $_{lb}$}‚ने ‚{\tiny $_{6}$}‚ न । यथा अस्स बम्भणस्स गावी दीअदि । सर्वञ्चेदमप्रमाणत्वाद्वचनमात्रं ‚{\tiny $_{lb}$}‚भवताविमत्याकूतवानुपचयमाह । ‚{\color{DodgerBlue3}‚शब्दस्य सुप्रयोगादेवेत्यादि} ‚{\tiny $_{14b9}$}‚ । एवं ‚{\tiny $_{lb}$}‚विधानित्यप्रमा ‚{\tiny $_{7}$}‚ णकान् ।
	{\color{gray}{\rmlatinfont\textsuperscript{§~\theparCount}}}
	\pend% ending standard par
      ‚{\tiny $_{lb}$}‚

	  
	  \pstart \leavevmode% starting standard par
	\hphantom{.}ननु च प्रतिष्ठिते भूप्रदेशे चैत्यङ्कारयति ब्राह्म्यं पुण्यं प्रसवति कल्पं स्वगषु ‚{\tiny $_{lb}$}‚मोदत इत्यादावपि प्रमाणाभावादयं तुल्यः प्रसङ्गो भवतामपि । न तुल्यो ‚{\tiny $_{8}$}‚ यस्मादत्र ‚{\tiny $_{lb}$}‚ \leavevmode\ledsidenote{\textenglish{107/s}} विषयद्वयपरिशुद्धिः परिनार्था \add{?} विसम्वादश्चास्तीति तृतीयेपि राशावाहितपरि‚{\tiny $_{lb}$}‚तोषाः प्रेक्षावन्तः प्रवर्त्तन्ते । नत्वेवं भवन्मतेऽनन्तरोदितद्वयमपि ‚{\tiny $_{8}$}‚ \leavevmode\ledsidenote{\textenglish{67a/msK}} प्रमाणव्याहतत्वात् ‚{\tiny $_{lb}$}‚प्रमाणव्याहतिश्चानन्तरमेवावेदिता । तस्माद्दरिद्रेश्वरस्पर्धासमानमेतत् \add{।}
	{\color{gray}{\rmlatinfont\textsuperscript{§~\theparCount}}}
	\pend% ending standard par
      ‚{\tiny $_{lb}$}‚

	  
	  \pstart \leavevmode% starting standard par
	विदितवेद्यादिगुणप्रयुक्ता ‚{\tiny $_{15a2}$}‚ इत्यन्तर्भावितभावप्रत्यययो\edtext{}{\lemma{इत्यन्तर्भावितभावप्रत्यययो}\Bfootnote{? य}} ‚{\tiny $_{lb}$}‚ ‚{\color{DodgerBlue3}‚निर्देश ‚{\tiny $_{1}$}‚ ः} । विदितं वेद्यं हेयोपादेयत्वं यैस्ते तथोक्ताः । आदिग्रहणात् करुणादिप‚{\tiny $_{lb}$}‚रिग्रहः । अमूने ‚{\tiny $_{5a2}$}‚ व संस्कृतानपरानसंस्कृतान् । एतदुक्तम्भवति \add{।} ‚{\tiny $_{lb}$}‚शब्दो हि व्यवहारोर्थप्र ‚{\tiny $_{2}$}‚ त्यायनफलः । तच्च यथा संस्कृतेभ्यः सङ्केतवसा\edtext{}{\lemma{सङ्केतवसा}\Bfootnote{? शा}} ‚{\tiny $_{lb}$}‚ त्सम्पद्यते तथाऽसंस्कृतेभ्योपीति किमस्थानेभिनिविष्टाः शिष्टाः । अत एव च ‚{\tiny $_{lb}$}‚मन्ये प्रेक्षावद्भ्योन्यत्वादनुगता ‚{\tiny $_{3}$}‚ र्थमेवं नामामीषामिति । अथवा किमस्माक‚{\tiny $_{lb}$}‚मर्हषितिः\add{?} प्रत्याख्यानैः ॥ ते अमून्नैव प्रयुञ्जते नापरानित्यत्रैव निश्चया‚{\tiny $_{lb}$}‚भावात् । यदाह \add{।} न चात्र श ‚{\tiny $_{4}$}‚ ब्दे परोक्षः साक्षी यतः साक्षिण इदमेवामूनेव ‚{\tiny $_{lb}$}‚प्रयुञ्जते नापरानिति निश्चिनुमः ‚{\tiny $_{15a3}$}‚ ।
	{\color{gray}{\rmlatinfont\textsuperscript{§~\theparCount}}}
	\pend% ending standard par
      ‚{\tiny $_{lb}$}‚

	  
	  \pstart \leavevmode% starting standard par
	\hphantom{.}ननु चोक्तन्तदन्वाख्यानस्य प्रयोजनं ‚{\color{DodgerBlue3}‚रक्षोहागमलघ्वसन्दे ‚{\tiny $_{5}$}‚ हा} \add{महाभाष्येआह्‚{\tiny $_{lb}$}‚निके १} इति । तत्कथं गुणातिशयाभावादित्युच्यत इत्याह । ‚{\color{DodgerBlue3}‚वेदरक्षादिकञ्चाप्रयो‚{\tiny $_{lb}$}‚जनमेवातत्समयस्थायिन} ‚{\tiny $_{15a4}$}‚ स्ताथागतस्य । न्यायानुपायित्वात् । तत्स्वभा ‚{\tiny $_{6}$}‚ ‚{\tiny $_{lb}$}‚वस्य ‚{\tiny $_{15a4}$}‚ साधुशब्दरूपस्य । अन्यतोपीति ‚{\tiny $_{15a4}$}‚ बृद्धप्रवादपारम्पर्यात् । ‚{\tiny $_{lb}$}‚एतदेव दृष्टान्तोप्रक्रमं व्यनक्ति । ‚{\color{DodgerBlue3}‚प्राकृतेत्यादिना} ‚{\tiny $_{15a4}$}‚ ।
	{\color{gray}{\rmlatinfont\textsuperscript{§~\theparCount}}}
	\pend% ending standard par
      ‚{\tiny $_{lb}$}‚\textsuperscript{\textenglish{108/s}}

	  
	  \pstart \leavevmode% starting standard par
	इ\add{त्}थं शा\add{ि}ब्दकस्योन्मत्तकतामुपदर्श्याधुना ‚{\color{DodgerBlue3}‚भा ‚{\tiny $_{7}$}‚ रद्वाज} स्याह ।
	{\color{gray}{\rmlatinfont\textsuperscript{§~\theparCount}}}
	\pend% ending standard par
      ‚{\tiny $_{lb}$}‚

	  
	  \pstart \leavevmode% starting standard par
	\hphantom{.}‚{\color{DodgerBlue3}‚अवयवविपर्ययेपीत्यादि} ‚{\tiny $_{15a6}$}‚ । सम्वन्धप्रतीतिरिति सम्बन्धः परस्परमुप‚{\tiny $_{lb}$}‚कार्योपकारकभावः । सामर्थ्याद्विवक्षितप्रतिपादन इति शेषः । ‚{\tiny $_{8}$}‚ अथ स्या ‚{\color{DodgerBlue3}‚दक्षपाद} ‚{\tiny $_{lb}$}‚सिद्धान्तनीतिपालनाय न प्रतिज्ञादीनां क्रमव्यत्ययः क्रियत इत्यत्राह । ‚{\color{DodgerBlue3}‚नह्यत्र ‚{\tiny $_{lb}$}‚कश्चित्स} मयः ‚{\tiny $_{15a6}$}‚ सिद्धान्तो नियमो वा प्रमाणोपेत इत्यप्याह । न ‚{\tiny $_{9}$}‚ \leavevmode\ledsidenote{\textenglish{67b/msK}} पर आह । ‚{\tiny $_{lb}$}‚ ‚{\color{DodgerBlue3}‚न विपर्ययात्प्रतीतिः} ‚{\tiny $_{15a7}$}‚ साध्यस्य । किन्तु ततो विपर्ययादानुपूर्व्या प्रतीति‚{\tiny $_{lb}$}‚रिति । अस्य प्रतिषेधः । ‚{\color{DodgerBlue3}‚नाप्रतीयमानसम्बन्धेभ्य आनुपूर्वी प्रतीतिरिति ‚{\tiny $_{15a8}$}‚ । ‚{\tiny $_{1}$}‚ ‚{\tiny $_{lb}$}‚ येषामित्यादि} नै ‚{\tiny $_{15a8}$}‚ तदेव व्याचष्टे ॥ अपि च प्रतिज्ञोपनयनिगमनानां ‚{\tiny $_{lb}$}‚पूर्वमेवास्माभिः साधनवाक्ये प्रयोगः प्रतिक्षिप्तः । तत्कुतस्तत्कृतो विपर्यय इत्येत‚{\tiny $_{lb}$}‚त्कथयति ‚{\tiny $_{2}$}‚ \add{।} ‚{\color{DodgerBlue3}‚प्रतिपादित} ‚{\tiny $_{15a10}$}‚ मित्यादिना । प्रतिज्ञाग्रहणमुपलक्षणार्थं । ‚{\tiny $_{lb}$}‚अथ सामर्थ्यलभ्यापि प्रयुज्यते तदातिप्रसङ्ग इत्येतदाह । ‚{\color{DodgerBlue3}‚प्रतीयमानार्थस्य च ‚{\tiny $_{lb}$}‚प्रयोगेति} ‚{\tiny $_{15b1}$}‚ प्रसङ्गः सा ‚{\tiny $_{3}$}‚ धर्म्यवति प्रयोगे वैधर्म\edtext{}{\lemma{वैधर्म}\Bfootnote{? र्म्य}}स्यापि प्रयोग‚{\tiny $_{lb}$}‚प्रसङ्गः । न चेष्यते । अर्थादापन्नस्य स्वशब्देन पुनर्वचनञ्चेति \href{http://sarit.indology.info/?cref=ns\%C5\%AB.5.2.15}{न्या० सू० ५।२।१५ } ‚{\tiny $_{lb}$}‚निग्रहस्थानवचनात् । पक्षधर्मान्वयव्यतिरे ‚{\tiny $_{4}$}‚ केषु तर्हि प्रतिज्ञाद्यभावेपि क्रमनियमो‚{\tiny $_{lb}$}‚ \leavevmode\ledsidenote{\textenglish{109/s}} भविष्यतीत्यत आह । ‚{\color{DodgerBlue3}‚परिशिष्टे} ‚{\tiny $_{15b1}$}‚ ष्वित्यादि । अप्रतीयमानसम्बन्धपक्षे ‚{\tiny $_{lb}$}‚दोषान्तरं ब्रूते \add{।} ‚{\color{DodgerBlue3}‚नेदमपार्थकाद् भि ‚{\tiny $_{5}$}‚ द्यत} ‚{\tiny $_{15b2}$}‚ इति न पृथग्वाच्यं स्यादिति ॥ ४ ॥
	{\color{gray}{\rmlatinfont\textsuperscript{§~\theparCount}}}
	\pend% ending standard par
      ‚{\tiny $_{lb}$}‚

	  
	  \pstart \leavevmode% starting standard par
	\hphantom{.}‚{\color{DodgerBlue3}‚यस्मिन्वाक्ये प्रतिज्ञादीनान्निगमनपर्यन्तानामन्यतमोऽवयवो न भवति । तद्वा‚{\tiny $_{lb}$}‚क्यं हीनं} ‚{\tiny $_{15b3}$}‚ निग्रहस्था ‚{\tiny $_{6}$}‚ नत्वे कारणमाह । ‚{\color{DodgerBlue3}‚साधनाभावे साध्यासिद्धे‚{\tiny $_{lb}$}‚रिति} ‚{\tiny $_{15b3}$}‚ । इदन्निराकरोति \add{।} न ‚{\color{DodgerBlue3}‚प्रतिज्ञादीनामित्यादिना} ‚{\tiny $_{15b3}$}‚ । ‚{\tiny $_{lb}$}‚प्रतिज्ञाग्रहणमुपलक्षणार्थं तेनोपनयनिगमनयोरपि ‚{\tiny $_{7}$}‚ परिग्रहः । ‚{\color{DodgerBlue3}‚उद्योतकरस्य} मतमुप‚{\tiny $_{lb}$}‚न्यस्यति । ‚{\color{DodgerBlue3}‚हीनमेव तत्} ‚{\tiny $_{15b3}$}‚ । प्रतिज्ञान्यूनं । तस्याः प्रतिज्ञायाः न्यूनतायामपि ‚{\tiny $_{lb}$}‚निग्रहादिति अस्यायुक्ततामाह । ‚{\color{DodgerBlue3}‚यः} साधनसा ‚{\tiny $_{8}$}‚ मर्थ्या ‚{\color{DodgerBlue3}‚त्प्रतीयमानार्थमनर्थकं शब्दं} ‚{\tiny $_{lb}$}‚साध्याभिधायिनं साधने ‚{\color{DodgerBlue3}‚प्रयुङ्क्ते स निग्रहमर्हेत्} ‚{\tiny $_{15b4}$}‚ । तथा हि शब्दस्या‚{\tiny $_{lb}$}‚नित्यत्त्वविचारे प्रस्तुते यदा ब्रवीति । कृतकानामनित्यत्वं ‚{\tiny $_{9}$}‚ \leavevmode\ledsidenote{\textenglish{68a/msK}} दृष्टङ्कृतकश्च शब्द ‚{\tiny $_{lb}$}‚इति । तदा वचनद्वया देवसाध्यार्थः प्रतीयत इति निरर्थकम्प्रतिज्ञावचनं । नार्थोपसं ‚{\tiny $_{1}$}‚ ‚{\tiny $_{lb}$}‚हितस्य युक्तियुक्तस्य पक्षधर्मसम्बन्धमात्रस्याभिधानेत्य ‚{\color{DodgerBlue3}‚समीक्षिताभिधानमेत} ‚{\tiny $_{15b4}$}‚ ‚{\tiny $_{lb}$}‚द्वार्तिककारस्य । अत एव चेति ‚{\tiny $_{15b4}$}‚ यतः प्रतीयमानार्थें शब्दे ‚{\tiny $_{lb}$}‚प्रयुक्ते ‚{\tiny $_{2}$}‚ निग्रहमर्हति । तद ‚{\color{DodgerBlue3}‚त्राबिद्धकर्णः} प्रतिबन्धकन्यायेन प्रत्यवतिष्ठते । ‚{\tiny $_{lb}$}‚यद्येवङ्कृतकश्च शब्द इत्येतदपि न वक्तव्यं किंकारणनी\edtext{}{\lemma{किंकारणनी}\Bfootnote{? निमित्त}}मनित्यत्व‚{\tiny $_{lb}$}‚मित्येतेनैव शब्दे ‚{\tiny $_{3}$}‚ पि कृतकत्वमनित्यत्वञ्चोभयं प्रतिपद्यते । यस्मात्पूर्वमपि शब्दे ‚{\tiny $_{lb}$}‚कृतकत्वम्परेण प्रतिपन्नमेव करणाच्छब्दोपि बुद्धौ व्यवस्थितः । अतोन्वय‚{\tiny $_{lb}$}‚वा ‚{\tiny $_{4}$}‚ क्येन स्मृतिमात्रकमुत्पाद्यते । अप्रतिपन्नकृतकत्त्वस्य पुनः कृतकश्च शब्द इत्ये‚{\tiny $_{lb}$}‚ \leavevmode\ledsidenote{\textenglish{110/s}} तस्मादपि नैव भवति । यद्वा कृतकः शब्द इत्येतावद्वक्तव्यं । कृतक ‚{\tiny $_{5}$}‚ त्वस्य त्वनित्यत्वे‚{\tiny $_{lb}$}‚नाविनाभावित्वं परस्य प्रसिद्धमिति शब्देप्यनित्यत्वं प्रतिपद्यत इति तेनानुकूल‚{\tiny $_{lb}$}‚मेवाचरितं । तथा हि यदि वादिना कथञ्चिन्निश्चि ‚{\tiny $_{6}$}‚ तम्भवति प्रतिपन्नमनेन वादिना ‚{\tiny $_{lb}$}‚कृतकत्वं शब्द इति तदा नैव तेन पक्षधर्मोपसंहारः कर्तव्यो निष्फलत्वात् । प्रति‚{\tiny $_{lb}$}‚बन्धमात्रन्तु प्रदर्शनीयं । अथ तथा न ‚{\tiny $_{7}$}‚ निश्चितं । तथापि यद्ययं परः पक्षधर्मोपसंहारे ‚{\tiny $_{lb}$}‚मया कृते तस्यासिद्धिञ्चोदयिष्यति । तदाहन्तां प्रत्ययभेदभेदित्वादिभिरुपायैः ‚{\tiny $_{lb}$}‚प्रतिनिवारयिष्यामि स्व ‚{\tiny $_{8}$}‚ यमेव वा ऽचोदित एवाशङ्क्यैतच्चेतस्याधाय पक्षधर्मत्वमुप‚{\tiny $_{lb}$}‚संहर्तव्यमेव कृतकश्च शब्द इति । यदाप्येवं वादी निश्चितवान् कृतकत्वस्यानित्यत्वे‚{\tiny $_{lb}$}‚नाविनाभा ‚{\tiny $_{9}$}‚ \leavevmode\ledsidenote{\textenglish{68b/msK}} वित्वं परस्य प्रसिद्धमिति तस्यामप्यवस्थायां कृतकः शब्द इत्येतावेदव ‚{\tiny $_{lb}$}‚वक्तव्यं । यथोक्तन् ‚{\tiny $_{lb}$}‚ 
	    \pend% close preceding par
	  
	    
	    \stanza[\smallbreak]
	  \flagstanza{\tiny\textenglish{...44}}{\normalfontlatin\large ``\qquad}तद्भानहेतुभावौ हि दृष्टान्ते तदवेदिनः&‚{\tiny $_{lb}$}‚ख्याप्यते विदुषाम्वाच्यो हेतुरे ‚{\tiny $_{1}$}‚ व हि केवल इति ॥{\normalfontlatin\large\qquad{}"}\&[\smallbreak]
	  
	  
	  
	    \pstart  \leavevmode% new par for following
	    \hphantom{.}
	   \add{४४} ॥
	{\color{gray}{\rmlatinfont\textsuperscript{§~\theparCount}}}
	\pend% ending standard par
      ‚{\tiny $_{lb}$}‚

	  
	  \pstart \leavevmode% starting standard par
	तदेतन्नियमाभ्युपगम इत्यधिकं निग्रहस्थानं । विशिष्टे विषये स्थापयति तञ्च ‚{\tiny $_{lb}$}‚विशिष्टं विषयमाह । ‚{\color{DodgerBlue3}‚यत्रेत्यादिना} ‚{\tiny $_{15b5}$}‚ । ननु चेदं निय ‚{\tiny $_{2}$}‚ माभ्युपगमे वेदि‚{\tiny $_{lb}$}‚तव्यमिति भाष्यकारेणैवोक्तं । तत्किमत्र दूषणमाचार्येणोक्तं सत्यन्न किञ्चिदुक्तं । ‚{\tiny $_{lb}$}‚आ\add{चा}र्येण तु ‚{\color{DodgerBlue3}‚पक्षिलो} क्तमेवनूद्यतेऽभ्यनुज्ञानार्थम् ॥ ४ ॥
	{\color{gray}{\rmlatinfont\textsuperscript{§~\theparCount}}}
	\pend% ending standard par
      ‚{\tiny $_{lb}$}‚

	  
	  \pstart \leavevmode% starting standard par
	\hphantom{.}‚{\color{DodgerBlue3}‚शब्दार्थयोः पुनर्वचनं पुनरुक्त} मित्यस्यापवादमाह ‚{\color{DodgerBlue3}‚अन्यत्रानुवादादिति} । अनु‚{\tiny $_{lb}$}‚वादो निगमनं । अनुवादो हि न पुनरुक्तव्यपदे ‚{\tiny $_{4}$}‚ शं लभते । शब्दाभ्यामर्थविशेषोत्पत्तिः । ‚{\tiny $_{lb}$}‚यस्मात्साध्यनिर्देशः । प्रतिज्ञासिद्धनिर्देशो निगमनमित्युक्तं । पुनः शब्दश्च नानात्वे ‚{\tiny $_{lb}$}‚दृष्टः । पुनरि ‚{\tiny $_{5}$}‚ यमचिरप्रभा निश्चरतीत्यप्यावेदितमेव । यद्येवन्तत्र तर्हि पुन‚{\tiny $_{lb}$}‚\leavevmode\ledsidenote{\textenglish{111/s}} रुक्ततायाः प्राप्तिरेव नास्तीति किमर्थमयमपवादः प्रारभ्यते । सत्यमेवमे ‚{\tiny $_{6}$}‚ तत् । ‚{\tiny $_{lb}$}‚त एव तु प्रकृष्टतार्किकाः प्रष्टव्याः । कथमेतदिति । अस्माकन्तु किं परकीयाभिर्गृह‚{\tiny $_{lb}$}‚चिन्ताभिश्चिन्तिताभिरित्यलम्प्रसङ्गेन । अत्र चेदमपि द्वितीयसूत्रम ‚{\tiny $_{7}$}‚ स्ति अर्था‚{\tiny $_{lb}$}‚दापन्नस्य स्वशब्देन पुनर्वचनमिति \href{http://sarit.indology.info/?cref=ns\%C5\%AB.5.2.15}{न्या० सू० ५।२।१५ } तदाचार्येण नोपन्यस्त‚{\tiny $_{lb}$}‚मुपलक्षणार्थत्वात् । तद्भाष्य\add{म}पक्षिप्य निराकरिष्यति । ‚{\color{DodgerBlue3}‚गम्यमानार्थं} पुनर्वचनम‚{\tiny $_{lb}$}‚पी ‚{\tiny $_{8}$}‚ त्यादिना ‚{\tiny $_{15b9}$}‚ । ‚{\color{DodgerBlue3}‚अत्रेत्यादिना} ‚{\tiny $_{15b7}$}‚ दूषणमारभते । एतदुक्तम्भवति । ‚{\tiny $_{lb}$}‚यत्र शब्दसाम्येप्यर्थो न भिद्यते तत्रार्थपुनरुक्ते ‚{\tiny $_{9}$}‚ \leavevmode\ledsidenote{\textenglish{69a/msK}} न गतं यत्र तु शब्दसाम्येप्यर्थभेदस्तत्र ‚{\tiny $_{lb}$}‚शब्दपुनरुक्ततायामपि न किञ्चित्कृतं । किमस्त्ययमीदृशः सम्भवो यच्छब्दपुनरुक्त‚{\tiny $_{lb}$}‚तायामप्यर्थभेदोस्तीत्यत आह । ‚{\color{DodgerBlue3}‚यथा हस ‚{\tiny $_{1}$}‚ ति हसतीत्यादि} ‚{\tiny $_{15b7}$}‚ । अत्र हि ‚{\tiny $_{lb}$}‚पूर्वो हसतिशब्दः सप्तम्यन्तो द्वितीयश्च तिङ्न्त इत्यर्थभेदः । एवमुत्तरत्रापि । काव्य ‚{\tiny $_{lb}$}‚ईदृशः सम्भवो न तु वाद इत्याशङ्कायामुदाहर ‚{\tiny $_{2}$}‚ ति । यथा चेत्यादि ‚{\tiny $_{15b8}$}‚ । ‚{\tiny $_{lb}$}‚ ‚{\color{DodgerBlue3}‚ननु} चेहाप्यर्थभेदवच्छब्दो\edtext{}{\lemma{चेहाप्यर्थभेदवच्छब्दो}\Bfootnote{? ब्द}}भेदोप्यस्ति सुबन्ततिङन्ततया । सत्त्यन्न केवलम‚{\tiny $_{lb}$}‚त्रापि । अत्राप्यनित्त्यः शब्दोऽनित्यःशब्द इत्यत्रास्त्येव शब्दभेदः स्वलक्ष ‚{\tiny $_{3}$}‚ णभेदात् । ‚{\tiny $_{lb}$}‚अन्यथा न क्रमभावि श्रवणं स्यात् । समानश्रुतिसमाश्रयमिह पौनरुक्त्यं यदि ‚{\tiny $_{lb}$}‚व्यवस्थाप्यते तदत्रापि तुल्यमेव । अर्थभेद एवायं । क्रि ‚{\tiny $_{4}$}‚ याभेदादिवाच्यभेदात् । ‚{\tiny $_{lb}$}‚तद्वलकल्पित एव हि पदभेदः । ‚{\color{DodgerBlue3}‚गम्यमानार्थं पुनर्वचनमपि पुनरुक्तमिति} ‚{\tiny $_{15b9}$}‚ ‚{\tiny $_{lb}$}‚ द्वितीयम्पुनरुक्तलक्षणसूत्रमुपलक्षयति । अस्य ‚{\tiny $_{5}$}‚ चोदाहरणं ‚{\color{DodgerBlue3}‚वात्स्यायनेन} न्यायभाष्य ‚{\tiny $_{lb}$}‚उक्तं । साधर्म्यवति प्रयोगे वैधर्म्यस्य । आचार्यस्तु प्रतिज्ञायामप्येतत्समानमित्यागूर्य ‚{\tiny $_{lb}$}‚प्रतिज्ञायाः साधनवा ‚{\tiny $_{6}$}‚ क्येऽनुपन्यासं प्रतिपादयितुकामो वक्रोक्त्या प्रतिज्ञावचनमेवो‚{\tiny $_{lb}$}‚\leavevmode\ledsidenote{\textenglish{112/s}} दाहरणत्वेनोपन्यस्यति । ‚{\color{DodgerBlue3}‚नियतपदप्रयोगे साधनवाक्ये यथा प्रतिज्ञावचनमिति} ‚{\tiny $_{lb}$}‚ ‚{\tiny $_{15b9}$}‚ नियतानां ‚{\tiny $_{7}$}‚ पदानां प्रयोगो यस्मिन्निति कार्यं । इदम्प्रतिक्षिपति ‚{\tiny $_{lb}$}‚ ‚{\color{DodgerBlue3}‚अर्थपुनरुक्तेनैव गत\add{ार्थ}त्वान्न पृथग्वाच्यमिति} । यथा ह्येकशब्दप्रतिपादितेर्थे ‚{\tiny $_{lb}$}‚तत्प्रतिपादनाय पर्या ‚{\tiny $_{8}$}‚ \leavevmode\ledsidenote{\textenglish{69b/msK}} यशब्दान्तरमुपादीयमानमनर्थकन्तथा सामर्थ्यगम्येप्यर्थ इति ‚{\tiny $_{lb}$}‚अर्थपुनरुक्तेनैवास्य सङ्ग्रह इति समुदायार्थः । क्व पुनरेतत्प्रति ‚{\tiny $_{1}$}‚ ज्ञादिवचनमर्था‚{\tiny $_{lb}$}‚पत्तिलभ्यं पुनरुक्तं सन्निग्रहस्थानम्भवतीति प्रश्ने ‚{\color{DodgerBlue3}‚नियतप्रयोगे साधनवाक्ये} ‚{\tiny $_{15b9}$}‚ ‚{\tiny $_{lb}$}‚ इत्येतदेवम्पक्षेण विवृणोति । अयमपि दोषो गम्यमानार्थपु ‚{\tiny $_{2}$}‚ नर्वचनकृतः साधनवाक्य ‚{\tiny $_{lb}$}‚एव नियतपदप्रयोग इति वर्तते । इदमुक्तम्भवति । यदा प्राश्निकाः शब्दार्थप्रमाण‚{\tiny $_{lb}$}‚प्रविचयनिपुना\edtext{}{\lemma{प्रविचयनिपुना}\Bfootnote{? णा}}ः प्रेक्षावन्तोत्यं ‚{\tiny $_{3}$}‚ तमवहितमनसश्च भवन्ति । प्रतिवाद्यपि ‚{\tiny $_{lb}$}‚तथाभूत एवेति वदन्ति यदन्तरेण न साध्यसिद्धिः तदेव प्रयोक्तव्यं । नाभ्यधिक‚{\tiny $_{lb}$}‚मिति तदायन्दोषो नान्यथा ‚{\tiny $_{4}$}‚ यस्मात्करुणापरतन्त्रचेतसोऽनिबन्धनवत्सलाः प्रतिवादि‚{\tiny $_{lb}$}‚नमतिद\edtext{}{\lemma{नमतिद}\Bfootnote{? दु}}र्ल्लभमिव शिष्यं न्यायवर्त्मावतारयितुं यतन्ते तत्र पुनर्वचनमपि न ‚{\tiny $_{lb}$}‚दोषाय । ए ‚{\tiny $_{5}$}‚ तदेवाह । ‚{\color{DodgerBlue3}‚व्याचक्षाणो हि वादी साक्षीप्रभृतीनामसम्यक्श्रवण} प्रतिपत्ति‚{\tiny $_{lb}$}‚शङ्कया करणभूतया ‚{\color{DodgerBlue3}‚सम्यक्श्रवणप्रतिपत्यर्थम्पुनः पुनर्ब्रूयादपीति} ‚{\tiny $_{15b10}$}‚ । ना‚{\tiny $_{lb}$}‚‚{\color{DodgerBlue3}‚विषय ‚{\tiny $_{6}$}‚ त्वादिति} परः । इदमेव व्याचष्टे ‚{\color{DodgerBlue3}‚नायम्वादी गुरुः} ‚{\tiny $_{15b11}$}‚ प्रतिवादिनः । ‚{\tiny $_{lb}$}‚न ‚{\color{DodgerBlue3}‚शिष्यः} प्रतिवाद्यपि वादिनः । द्वयोरपि परस्परजिगीषया व्यवस्थानादिति । ‚{\tiny $_{lb}$}‚तस्मात् ‚{\tiny $_{7}$}‚ न वादिना प्रतिवादी यत्नतः प्रतिपादनीयः । ‚{\color{DodgerBlue3}‚ने} ‚{\tiny $_{15b11}$}‚ त्याद्याचार्यः । ‚{\tiny $_{lb}$}‚यदि नाम प्रतिवादी न प्रतिपाद्यते यत्नेन । साक्षिणस्त्ववश्यं यत्नेन प्रतिपाद्यास्त‚{\tiny $_{lb}$}‚द्बोधना ‚{\tiny $_{8}$}‚ \leavevmode\ledsidenote{\textenglish{70a/msK}} देव हि वादिनो जयोन्यथा च पराजय इति कथं साक्षिण एव न प्रति‚{\tiny $_{lb}$}‚पादयेत\edtext{}{\lemma{पादयेत}\Bfootnote{? त्}}किञ्चावश्यं साक्षिवत्प्रतिवाद्यपि प्रतिपाद्यः । कस्मात्तदप्रतिपादने ‚{\tiny $_{lb}$}‚दोषाभि ‚{\tiny $_{1}$}‚ धानात् । तच्छ\add{ले}न साक्षिप्रभृतयः प्रत्यवमृश्यन्ते यदि साक्षिप्रभृतयो न ‚{\tiny $_{lb}$}‚प्रतिपाद्या भवेयुस्ततो यद् भवद्भिः परिषत्प्रतिवादिभ्यान्त्रिरभिहितमविज्ञात‚{\tiny $_{2}$}‚म‚{\tiny $_{lb}$}‚विज्ञातार्थ निग्रहस्थानमुक्तं \href{http://sarit.indology.info/?cref=ns\%C5\%AB.5.2.16}{न्या० सू० ५।२।१६ } । तद्विरुद्ध्यत इत्यर्थः । ‚{\tiny $_{lb}$}‚ \leavevmode\ledsidenote{\textenglish{113/s}} यच्चोच्यते नायं शिष्य इति तदसिद्धं । ‚{\color{DodgerBlue3}‚प्रतिपाद्यस्य शिष्यत्वात्} ‚{\tiny $_{16a1}$}‚ । तत्वज्ञा‚{\tiny $_{lb}$}‚नार्थतया प्रतिपाद्य एव ‚{\tiny $_{3}$}‚ शिष्योन्यस्य तल्लक्षणस्याभावात् । प्रतिवादी च तथाभूतः ‚{\tiny $_{lb}$}‚कथं न शिष्यः । किमुच्यते नैवासौ प्रतिवादी तत्वज्ञानार्थास्पर्धया व्युत्थितत्वा‚{\tiny $_{lb}$}‚दिति । ‚{\tiny $_{4}$}‚ तदयुक्तं । पूर्व्वञ्जिगीषुवादप्रतिषेधात् ‚{\tiny $_{16a1}$}‚ । एवमपि नैवासौ यत्न‚{\tiny $_{lb}$}‚प्रतिपाद्यस्त्रिरभिधाननियमस्य ‚{\color{DodgerBlue3}‚महर्षिणा} कृतत्वादित्यत आह । ‚{\color{DodgerBlue3}‚त्रि ‚{\tiny $_{5}$}‚ रभिधान} ‚{\tiny $_{lb}$}‚ ‚{\tiny $_{16a1}$}‚ वचनादित्यादि । अनेनैतद्दर्शयति । यद्वक्ष्यति । यदि तावत्परप्रतिपादनार्था ‚{\tiny $_{lb}$}‚प्रवृत्तिः किन्त्रिरभिधीयते तथा तथा स ग्राहिणीयो ‚{\tiny $_{6}$}‚ यथास्य प्रतिपत्तिर्भवति । ‚{\tiny $_{lb}$}‚अथ परोपतापनार्था तथापि किं त्रिरभिधीयते । साक्षिणाङ्कर्णे निवेद्य प्रतिवादी ‚{\tiny $_{lb}$}‚कष्टाप्रतीतद्रुतसंक्षिप्तादिभिरुपद्रो ‚{\tiny $_{7}$}‚ तव्यो यथोत्तरप्रतिपत्तिविमूढस्तूष्णीम्भवतीति । ‚{\tiny $_{lb}$}‚ ‚{\color{DodgerBlue3}‚न चेद} ‚{\tiny $_{16a1}$}‚ मिति शब्दार्थयोः पुनर्वचनं । गम्यमानार्थपुनर्वचनं च । अभेदमेव ‚{\tiny $_{lb}$}‚साधयति । ‚{\color{DodgerBlue3}‚विनिय ‚{\tiny $_{8}$}‚ ते} ‚{\tiny $_{16a1}$}‚ त्यादिना । ‚{\color{DodgerBlue3}‚आधिक्यं} ‚{\tiny $_{16a2}$}‚ हेतूदाहरणयोर्दोषः । ‚{\tiny $_{lb}$}‚एकेन कृतत्वादितरस्यानर्थक्यमिति वचनात् । पुनर्वचनेपि गतो ज्ञातः पूर्वेणैव ‚{\tiny $_{lb}$}‚शब्देनार्थो यस्यो ‚{\tiny $_{9}$}‚ \leavevmode\ledsidenote{\textenglish{70b/msK}} त्तरस्य पदस्य तदेवमुक्तं ।  तस्याधिक्यमेव दोष इत्यधिकृतं । ‚{\tiny $_{lb}$}‚किम्पुनर्नियतपदप्रयोगेऽयन्दोष इत्युक्तमिति चेदाह । ‚{\color{DodgerBlue3}‚प्रपञ्चकथायामदोष} ‚{\tiny $_{lb}$}‚ ‚{\tiny $_{16a2}$}‚ इत्यभिसम्बन्धः । क ‚{\tiny $_{1}$}‚ स्य ‚{\color{DodgerBlue3}‚हेत्वादिबाहुल्यस्य} ‚{\tiny $_{16a2}$}‚ । पुनर्वचनस्य ‚{\tiny $_{lb}$}‚च । आदिशब्देनोदाहरणबाहुल्यग्रहणं । कीदृश्यामनिरूपितैकार्थसाधनाधिकर‚{\tiny $_{lb}$}‚णायां अर्थः साध्यः । अर्थ्यत इ ‚{\tiny $_{2}$}‚ ति कृत्वा साधनं । हेतुरधिकरणन्धर्मी । अर्थसहितं ‚{\tiny $_{lb}$}‚साधनमर्थसाधनं । मध्यपदलोपात् । एकञ्च तदर्थसाधनञ्च तथोक्तम् । तस्याधि‚{\tiny $_{lb}$}‚करणन्तदनिरूपि ‚{\tiny $_{3}$}‚ तमेकार्थसाधनाधिकरणं यस्यां प्रपञ्चकथायां प्रतिवादिना धर्मिणो ‚{\tiny $_{lb}$}‚जीवशरीरादेर्नैको धर्मो नैरात्म्यादिषु प्रमातुमिष्टोऽपित्वनेकः क्षणि ‚{\tiny $_{4}$}‚ कत्वानात्मत्वा‚{\tiny $_{lb}$}‚नीश्वरकर्त्तृत्वादिस्तथा नैकेनैव हेतुना किन्त्वनेकेनापि तस्यामित्यर्थः । एतदेव ‚{\tiny $_{lb}$}‚यथाक्रमं ब्रूते । नानार्थसाधनेप्सायां ‚{\color{DodgerBlue3}‚नाना ‚{\tiny $_{5}$}‚ साधनेप्सायां वा श्रोतुरिति} ‚{\tiny $_{16a2}$}‚ ‚{\tiny $_{lb}$}‚ पूर्व्वकः साधनशब्दो भावसाधनत्वात्सिद्धिवचनः । उत्तरस्तु करणसाधनत्वाद्धेतुव‚{\tiny $_{lb}$}‚चनः । तस्माद्धेत्वादिबाहु ‚{\tiny $_{6}$}‚ ल्यं वचनबाहुल्यं साधनेन विनियतपदे दोषः । कस्मा\add{त्} ‚{\tiny $_{lb}$}‚ \leavevmode\ledsidenote{\textenglish{114/s}} प्रतीताय्याभावात् । प्रत्य तुल्यो दोष इति कृत्वा सङ्ग्रह एव न्याय्यः । अधिकमेव वा ‚{\tiny $_{lb}$}‚वक्तव्यं पुनरुक्त ‚{\tiny $_{7}$}‚ मेव चेत्यर्थः । अनयोरेकस्मिन्द्वितीयस्यान्तर्भावात् । कथं पुनः शब्द‚{\tiny $_{lb}$}‚पुनरुक्ते ऽधिकस्यान्तर्भाव इत्याह । ‚{\color{DodgerBlue3}‚पर्य्यायशब्दकल्पो} ‚{\tiny $_{16a4}$}‚ ह्यपरो द्वितीयो ‚{\tiny $_{lb}$}‚हेतुरेकहेतु ‚{\color{DodgerBlue3}‚प्र ‚{\tiny $_{8}$}‚ तिपादिते विषये प्रवर्त्तमानः} \add{।} किं कारणं \add{।} प्रतिपाद्यस्य विशेषाभा‚{\tiny $_{lb}$}‚वात् । अर्थस्य पुनरुक्तन्तर्हि कथमधिकेन्तर्भवतीत्याह ‚{\color{DodgerBlue3}‚अर्थः पुनः प्रतिपादना ‚{\tiny $_{9}$}‚ \leavevmode\ledsidenote{\textenglish{71a/msK}} न्न भिद्यत} ‚{\tiny $_{lb}$}‚ इति ॥ अर्थशब्देनार्थपुनरुक्तमुपलक्षयति । पुनः प्रतिपाद्यते अनेनेति पुनः प्रतिपादनं ‚{\tiny $_{lb}$}‚हेतूदाहरणाधिकमेव । इदमुक्तम्भवति \add{।} स्फुटमेवास्य उदा ‚{\tiny $_{1}$}‚ हरणाधिकेन्तर्भावः । ‚{\tiny $_{lb}$}‚तथाहि साधर्म्यवति प्रयोगे वैधर्म्योदाहरणस्याप्रयोगोऽर्थपुनरुक्तस्योदाहरणमुक्तं । ‚{\tiny $_{lb}$}‚यत्पुनरुक्तमेवाद्यपवादप्रतिषेधः सुज्ञानः ॥ ० ॥ 
	{\color{gray}{\rmlatinfont\textsuperscript{§~\theparCount}}}
	\pend% ending standard par
      ‚{\tiny $_{lb}$}‚

	  
	  \pstart \leavevmode% starting standard par
	विज्ञातो वाक्यार्थो यस्य त्रिरभिहितस्य तत्तथा । विशेषणसमासो वा विज्ञात‚{\tiny $_{lb}$}‚श्चासौ वाक्यार्थश्चेति त्रिरभिहितस्य वादिनेति प्रतिपत्तव्यं ॥ प्रतिवादिना प ‚{\tiny $_{3}$}‚ द‚{\tiny $_{lb}$}‚प्रत्युच्चारणमिति सम्बन्धनीयं । त्रिवचनं सकृदभिहितस्याननुभाषणेपि न निग्रह ‚{\tiny $_{lb}$}‚इति ज्ञापनार्थं । अप्रत्युच्चारणञ्च शब्दद्वारेणार्थद्वारेण ‚{\tiny $_{4}$}‚ वा । निग्रहस्थानत्वे कारण‚{\tiny $_{lb}$}‚माह । ‚{\color{DodgerBlue3}‚अप्रत्युच्चारयन् किमाश्रयम्परपक्षे प्रतिषेधं ब्रूयादि} ‚{\tiny $_{16a7}$}‚ ति न विषयन्दू‚{\tiny $_{lb}$}‚षणाभिधानन्नास्तीत्यर्थः ॥ इदन्त्वयोक्तं ‚{\tiny $_{5}$}‚ मति ङ्कृत्त्वा दूषणम्वाच्यं । एवन्दूषणवाक्य ‚{\tiny $_{lb}$}‚ \leavevmode\ledsidenote{\textenglish{115/s}} मपि साधनवादिना प्रत्यनुभाष्य परिहर्तव्यं । अतो द्वयोरपीदं निग्रहस्थानं । अत्र ‚{\color{DodgerBlue3}‚भार‚{\tiny $_{lb}$}‚द्वाजो} न्यक्षेणा ‚{\tiny $_{6}$}‚ क्षेपन्तावत्करोति ‚{\color{DodgerBlue3}‚उत्तरेणावसानात्परिज्ञानान्नेदन्निग्रहस्थानमि} ति चेदि‚{\tiny $_{lb}$}‚ति । इदम्वाक्यम्व्याचष्टे \add{।} ‚{\color{DodgerBlue3}‚स्वादेतदि} ‚{\tiny $_{16a7}$}‚ त्यादिना । नोत्तरविषयपरिज्ञानादिति ‚{\tiny $_{lb}$}‚स ए ‚{\tiny $_{7}$}‚ व प्रतिविधत्ते । ‚{\color{DodgerBlue3}‚यद्ययमि} ‚{\tiny $_{16a8}$}‚ त्याद्यस्यैव विभागः । अप्रतिज्ञानाच्चेति स ‚{\tiny $_{lb}$}‚एव । ‚{\color{DodgerBlue3}‚उत्तरञ्चाश्रयाभावे परपक्षोपक्षेपाभावे सत्ययुक्तमिति युक्तमप्रत्युच्चारणे ‚{\tiny $_{8}$}‚ ‚{\tiny $_{lb}$}‚निग्रहस्थानमित्ये} तावान् परकीयो ग्रन्थः । अत्राचार्यो दूषणम्वक्तुमारभते । ‚{\color{DodgerBlue3}‚यदि नाम ‚{\tiny $_{lb}$}‚वादीस्वसाधनार्थस्य विवरणव्याजेन प्रसङ्गादपरापरं घोषये ‚{\tiny $_{9}$}‚ \leavevmode\ledsidenote{\textenglish{71b/msK}} त् । यथोदा- ‚{\tiny $_{lb}$}‚हृतम्प्राक्तत्र करणभुवनानि बुद्धिसत्कारणपूर्व्वकाणीति प्रतिज्ञाशरीरादिव्याख्यान‚{\tiny $_{lb}$}‚च्छद्मना सकलं वैशेषिकतन्त्रं घोषयेदिति} ‚{\tiny $_{16a10}$}‚ । तथा जिज्ञासितम ‚{\tiny $_{1}$}‚ र्थमात्र‚{\tiny $_{lb}$}‚मुक्त्वा कथां विस्तारयेद्यदि नाम वादीति वर्तते । किङ्कृत्वा विस्तारयेदित्याह । ‚{\tiny $_{lb}$}‚ ‚{\color{DodgerBlue3}‚प्रतिज्ञादिष्वर्थविशेषणपरम्परयाऽपरान् सिध्यनुपयोगिनोर्थानुपक्षिप्य} ‚{\tiny $_{16b10}$}‚ । ‚{\tiny $_{2}$}‚ ‚{\tiny $_{lb}$}‚ यथा निदर्शितं पूर्व्वन्नित्यः शब्दोऽनित्यः शब्द इति विवादे ‚{\color{DodgerBlue3}‚जैमिनीयः} प्रमाणयति । ‚{\tiny $_{lb}$}‚ ‚{\color{DodgerBlue3}‚द्वादशलक्षणे} त्यादिना । व्याचष्टे च द्वादशलक्षणानि । यथा वा ‚{\color{DodgerBlue3}‚ऽक्षपादा} ए ‚{\tiny $_{3}$}‚ वङ्कुर्वन्ति । ‚{\tiny $_{lb}$}‚किममी सर्वे संस्काराः क्षणिका नो वेति विवादे रूपत्वादिसामान्याश्रय\add{त्त्वा}त्तदा‚{\tiny $_{lb}$}‚श्रयास्तद्विषयाश्च प्रत्यक्षादयः प्रत्ययाः स्वात्मलाभानन्त ‚{\tiny $_{4}$}‚ रप्रध्वंसिनो न भवन्ति । ‚{\tiny $_{lb}$}‚समानानामसमानजातीयद्रव्यसंयोगविभागजनितशब्दकार्यशब्दाभिधेयत्वात्प्रागभा‚{\tiny $_{lb}$}‚वादिवदिति । ननु च प्रतिज्ञादीष्वित्यत्रादिशब्देन किं गृह्यते । न तावद्धेतूदाहरणे ‚{\tiny $_{lb}$}‚तन्मात्रमुक्त्वेति वचनात् । न चापरः कश्चित्प्रस्तुतोऽत्रेति \add{।} नैष दोषो यतो ‚{\tiny $_{lb}$}‚हेत्वादिमात्रमप्युक्त्वेति द्रष्ट ‚{\tiny $_{6}$}‚ व्यं । तेन हेत्वादीनामेवादिग्रहणेन आक्षेप इति केचित् । ‚{\tiny $_{lb}$}‚अपरे पुनराहुरस्थानमेवेदमाशङ्कितं । क्त्वाप्रत्ययनिर्देशेत्र । यस्मादयमत्रार्थो यत्र ‚{\tiny $_{lb}$}‚प्रतिवा ‚{\tiny $_{7}$}‚ दिना जिज्ञासितमर्थमात्रमन्यविशेषणरहितमक्षणिकत्वादिकं तदुक्त्वा वाद्य‚{\tiny $_{lb}$}‚\leavevmode\ledsidenote{\textenglish{116/s}} हमेतत्साधयामीत्युत्तरकालप्रमाणमारचयन्प्रतिज्ञादिष्वर्थविशेषणप ‚{\tiny $_{8}$}‚ रम्परयापरान‚{\tiny $_{lb}$}‚र्थानुपक्षिप्य कथाम्विस्तारयेदिति । ‚{\color{DodgerBlue3}‚तच्चे} ‚{\tiny $_{16a11}$}‚ ति प्रतिज्ञादिविशेषणपरम्परया ‚{\tiny $_{lb}$}‚यदप्रस्तुतमेव नाटकाख्यायिकाघोषणकल्पं वादिनोद्ग्रा ‚{\tiny $_{9}$}‚ \leavevmode\ledsidenote{\textenglish{72a/msK}} हितं । तदा कस्तस्य विवा‚{\tiny $_{lb}$}‚दाश्रयश्चासावर्थमात्रश्चाक्षणिकत्वादिकन्तस्योत्तरवचने सामर्थ्यविघातो नैवेत्यर्थः । ‚{\tiny $_{lb}$}‚तस्मान्न वादिकथामननुभाषमाणः प्रति ‚{\tiny $_{1}$}‚ वाद्युत्तरवाद्येन \add{?} समर्थः । किन्तु यद्वचनना‚{\tiny $_{lb}$}‚न्तरीयिका जिज्ञासितार्थसिद्धिस्तदवश्यमुपदर्श्यत एवेति सम्बन्धः । कस्मात्सा‚{\tiny $_{lb}$}‚धनाङ्गविषयत्वाद् दूषणस्य । परो ‚{\tiny $_{2}$}‚ पनीते हि साधने दूषणम्प्रवर्तत इति सम्बन्धः ‚{\tiny $_{lb}$}‚किन्नान्तरीयिका पुनर्जिज्ञासितार्थसिद्धिरित्याह । ‚{\color{DodgerBlue3}‚यथा पक्षधर्मता व्याप्तिप्रसाधन‚{\tiny $_{lb}$}‚मात्र} मि ‚{\tiny $_{16b1}$}‚ ति व्याप्तिः प्रसा ‚{\tiny $_{3}$}‚ ध्यतेनेनेति व्याप्तिप्रसाधनं बाधकप्रमाणोप‚{\tiny $_{lb}$}‚दर्शनं किमियमित्यपि साधनप्रयोगेऽर्थान्तरोपक्षेपः कर्तव्यो नेत्याह । न । ‚{\color{DodgerBlue3}‚तत्रापि ‚{\tiny $_{lb}$}‚प्रसङ्गान्तरो ‚{\tiny $_{4}$}‚ प} क्षेप ‚{\tiny $_{16b1}$}‚ इति नैरर्थक्यादिति मतिः \add{।} तावत् मात्रमुपदर्श्यते ‚{\tiny $_{lb}$}‚किं प्रागनुक्रमेण । पश्चात्तु दूष्यते नेत्याह । ‚{\color{DodgerBlue3}‚तत्रापि} दूषणविषयोपदर्शनार्थेऽनु‚{\tiny $_{lb}$}‚भा ‚{\tiny $_{5}$}‚ षणे न सर्व्वं यावदुपन्यस्तं वादिना तद्दूषणाभिधानात् । ‚{\color{DodgerBlue3}‚प्रागनुक्रमेणो‚{\tiny $_{lb}$}‚च्चारयितव्यं} । कस्मात् त्रिरुच्चारणप्रसङ्गात् । द्विरुच्चारणप्रसङ्गमेव ‚{\tiny $_{6}$}‚ प्रति‚{\tiny $_{lb}$}‚पादयितुमादिप्रस्थानमाचरति । ‚{\color{DodgerBlue3}‚दूषणेत्या} ‚{\tiny $_{16b2}$}‚ दिना । यदि वचनानुक्रमघोषणं ‚{\tiny $_{lb}$}‚न करोति निर्विषयमिदानीं दूषणम्प्रसक्तमित्याह । ‚{\color{DodgerBlue3}‚नान्तरीय ‚{\tiny $_{7}$}‚ कत्वा}\add{द्}‚{\tiny $_{16b3}$}‚ ‚{\tiny $_{lb}$}‚दूषयता विषयोपदर्शनं क्रियत एव । कथम्प्रतिदोषवचनं दोषवचनं दोषवचनम्प्रति । ‚{\tiny $_{lb}$}‚यो यो दोषो भण्यते तस्य तस्य विषयः कथ्यत इत्य ‚{\tiny $_{8}$}‚ र्थः । इदमेवाह \add{।} अस्य वाद्युक्त‚{\tiny $_{lb}$}‚स्यायन्दोष इति । किम्पुनः कारणं सर्व्वप्रत्युच्चार्ययुगपद्दूषणन्नोच्यत इति चेदाह । ‚{\tiny $_{lb}$}‚ ‚{\color{DodgerBlue3}‚नही} ‚{\tiny $_{16b3}$}‚ त्यादि । कुतः ‚{\color{DodgerBlue3}‚प्रत्यर्थं दो ‚{\tiny $_{9}$}‚ \leavevmode\ledsidenote{\textenglish{72b/msK}} षभेदात्} । विषयवद् भिद्यते दोष इति ‚{\tiny $_{lb}$}‚ \leavevmode\ledsidenote{\textenglish{117/s}} यावत् । एवमारचितादिप्रस्थानो द्विरुच्चारणप्रसङ्गम्प्रदर्शयति सकृदेवाप्रघुष्टो ‚{\tiny $_{lb}$}‚न । सर्व्वानुभाषणेस्य प्रदर्शिते वि ‚{\tiny $_{1}$}‚ षयदोषस्य वक्तुमशक्यत्वात् । केवलमिदं ‚{\tiny $_{lb}$}‚निःप्रयोजनपराजयाधिकरणं चेत्याह । ‚{\color{DodgerBlue3}‚दूषणवादिना दूषणे वक्तव्ये} ‚{\tiny $_{16b6}$}‚ सति ‚{\tiny $_{lb}$}‚यत्र सर्व्वानुक्रमभाषणं तत्र ‚{\tiny $_{2}$}‚ परपक्षक्षोभेनोपयुज्यते तस्याभिधानमिदन्द्विरुक्तपदो‚{\tiny $_{lb}$}‚द्भावनञ्चेत्येवं व्यत्याशेन\edtext{}{\lemma{व्यत्याशेन}\Bfootnote{? सेन}}पदविन्यासः कार्य इति तस्मात्सर्व्वानुक्रम‚{\tiny $_{lb}$}‚भाषणम्पराजया ‚{\tiny $_{3}$}‚ धिकरणं वाच्यं अत्रेदानी ‚{\color{DodgerBlue3}‚माक्षपादः} सर्व्वमिदं दूषणमनभ्युपगमेनैव ‚{\tiny $_{lb}$}‚पूर्व्वपक्षस्यास्माभिः प्रतिव्यूढमिति मन्यमानोऽभ्यनुजानाति । ‚{\color{DodgerBlue3}‚तथास्त्वि ‚{\tiny $_{4}$}‚ ‚{\tiny $_{16b7}$}‚ ‚{\tiny $_{lb}$}‚ति} । स्यादेतदित्येतदेव व्याचष्टे । ‚{\color{DodgerBlue3}‚यतः कुतश्चिदि ‚{\tiny $_{16b7}$}‚ ति} साधनार्थ‚{\tiny $_{lb}$}‚विवरणस्य व्याजेन प्रतिज्ञादिष्वर्थविशेषणपरम्परोपक्षेपेण चाप्रसङ्गात् अनं ‚{\tiny $_{5}$}‚ ‚{\tiny $_{lb}$}‚तरीयकाभिधानं ‚{\tiny $_{16b7}$}‚ रूपसिद्धिनामादिव्याख्यानकल्पत्त्वाद्वादिनोऽर्थान्तर‚{\tiny $_{lb}$}‚गमनमेवेति स तेन निग्रहार्हः । प्रासङ्गिकं ब्रुवाणः किमिति निगृ ‚{\tiny $_{6}$}‚ ह्यत ‚{\tiny $_{lb}$}‚इति चेदाह । नहि कश्चित् क्वचित् क्रियमाणप्रसङ्गो न प्रयुज्यते । ‚{\tiny $_{lb}$}‚यथोक्तम्प्राक् नैरात्म्यवादिनः । तत्साधने नृत्यगीतादेरपि प्रसङ्ग इति । नापि ‚{\tiny $_{lb}$}‚तद्यद्वादि ‚{\tiny $_{7}$}‚ ना प्रसङ्गत्वेन आहितं । तस्य प्रतिवादिनो ऽनुभाषणीयं ‚{\tiny $_{16b8}$}‚ । ‚{\tiny $_{lb}$}‚अनुपयुज्यमाण\edtext{}{\lemma{अनुपयुज्यमाण}\Bfootnote{? न}}त्वन्नै \uline{वानु} भाषणमर्हतीत्यर्थः । तदेतेन यत्पूर्वमुक्तं ‚{\color{DodgerBlue3}‚यदि नाम ‚{\tiny $_{lb}$}‚वादी ‚{\tiny $_{8}$}‚ स्वसाधनार्थविवरणव्याजेने} ‚{\tiny $_{15b10}$}‚ त्यादि तदभ्यनुज्ञातं । संप्रति यदुक्तं ‚{\tiny $_{lb}$}‚तत्रापि न सर्व्वं क्रमेणोच्चारयितव्यं । पश्चाद् दूषणम्वाच्यं द्विरुच्चारणप्रसङ्गादिति ‚{\tiny $_{lb}$}‚तदनु ‚{\tiny $_{9}$}‚\leavevmode\ledsidenote{\textenglish{73a/msK}} जानाति । ‚{\color{DodgerBlue3}‚न चेदमप्यस्माभिरि} त्या ‚{\tiny $_{16b8}$}‚ दिना ।
	{\color{gray}{\rmlatinfont\textsuperscript{§~\theparCount}}}
	\pend% ending standard par
      ‚{\tiny $_{lb}$}‚\textsuperscript{\textenglish{118/s}}

	  
	  \pstart \leavevmode% starting standard par
	\hphantom{.}‚{\color{DodgerBlue3}‚आचार्य आह । यदि भवद्भिरप्ये\edtext{}{\lemma{भवद्भिरप्ये}\Bfootnote{? पी}}}‚द ‚{\color{DodgerBlue3}‚मेवेष्टमेवन्तर्हि नाननुभाषणम्पृथ‚{\tiny $_{lb}$}‚ग्वाच्यं} ‚{\tiny $_{16b9}$}‚ । कस्मात् प्रतिभयागत्वात् । गतत्त्वमेव प्र ‚{\tiny $_{1}$}‚ तिपादयति । ‚{\tiny $_{lb}$}‚ उत्तरस्य ह्यप्रतिपत्तेरप्रतिभा \href{http://sarit.indology.info/?cref=ns\%C5\%AB.5.2.18}{न्या० सू० ५।२।१८ } । ततः कथङ्गतमित्याह । ‚{\tiny $_{lb}$}‚न ‚{\color{DodgerBlue3}‚चोत्तरविषयमप्रदर्शयन्} प्रतिवाद्युत्तरं प्रतिपत्तुं ज्ञातुमभिधातुं वा समर्थः । ‚{\tiny $_{2}$}‚ ‚{\tiny $_{lb}$}‚किमिति न शक्त इत्याह । न ‚{\color{DodgerBlue3}‚हीत्यादि} । चोत्तरप्रतिपत्तिरुत्तरा प्रतिपत्तिरि‚{\tiny $_{lb}$}‚त्यर्थः । सा नाक्षिप्ता येनाननुभाषणेन तत्तथा । प्रतिषेधद्वयाद्विध्यवसीय ‚{\tiny $_{3}$}‚ आक्षि‚{\tiny $_{lb}$}‚प्तोत्तराप्रतिपत्तिकमेवेति । एतदुक्तम्भवति । यो हि नामोत्तरम्प्रतिपद्यतेऽतोवश्यं ‚{\tiny $_{lb}$}‚तद्विषयमप्यवेत्यस्येदं दूषणमिति परिज्ञानात् ‚{\tiny $_{4}$}‚ परिज्ञातविषयश्च कथं सचेतनो ‚{\tiny $_{lb}$}‚न तमनुभाषते । तस्माद्यत्राननुभाषणन्तत्राप्रतिभयाभाव्यमिति सा तस्य व्यापिका ‚{\tiny $_{lb}$}‚तरूरिवखदिरस्य ‚{\tiny $_{5}$}‚ तस्याश्च विहितं निग्रहस्थानत्वं व्याप्येऽननुभाषणे तदा ‚{\tiny $_{lb}$}‚क्षिपतीत्येतन्निगमनव्याजेनाह । ‚{\color{DodgerBlue3}‚तेनेत्यादि} । अत्रैव दृष्टान्तं ब्रूते गव्यपरामृष्ट ‚{\tiny $_{6}$}‚ ‚{\tiny $_{lb}$}‚तद्भेदायां सामान्यभूतायाम्विहितमिवसा\edtext{}{\lemma{सामान्यभूतायाम्विहितमिवसा}\Bfootnote{? सा}}स्नादिमत्त्वत्तद्व्याप्तबाहुलेयेऽपि‚{\tiny $_{lb}$}‚लब्धमिति वर्त्तते । प्रयोगः पुनर्यदेकविधानसामर्थ्यादनुक्तमपि लभ्य ‚{\tiny $_{7}$}‚ ते । ननु‚{\tiny $_{lb}$}‚भूयः प्रेक्षापूर्वकारिणा विधातव्यं । तद्यथा गोजातौशा\edtext{}{\lemma{गोजातौशा}\Bfootnote{? सा}}स्नादिमत्वविधानसाम‚{\tiny $_{lb}$}‚र्थ्यात् प्रतिलब्धं तर्ह्येषुशा\edtext{}{\lemma{तर्ह्येषुशा}\Bfootnote{? सा}}स्नादिमत्वं । अप्रतिभानिग्रहस्थान ‚{\tiny $_{8}$}‚ त्वविधान‚{\tiny $_{lb}$}‚सामर्थ्यात् प्रतिबद्धश्चाननुभाषणनिग्रहस्थानत्वमिति व्यापकविरुद्धोपलब्धिः \add{।} ‚{\tiny $_{lb}$}‚ननु च विषयं विषयश्च प्रपञ्चोत्तरं प्रतिपद्यमानोप्यति ‚{\tiny $_{9}$}‚ \leavevmode\ledsidenote{\textenglish{73b/msK}} भयकम्पादिभिर्व्याकुलीकृत‚{\tiny $_{lb}$}‚चेताः प्रतिवादीनानुभाषते स विषयोऽननुभाषणस्याप्रतिभयानालीढस्तत्कथं सा तस्य ‚{\tiny $_{lb}$}‚व्यापिका यतोऽयं हेतुः सिद्धो भविष्यति ‚{\tiny $_{1}$}‚ नैव सम्भवात् । नहि विषयं विषयविषय‚{\tiny $_{lb}$}‚ञ्चोत्तरं प्रतिपद्यमानः कुतश्चिद्विभेति वेपते वा तदज्ञानकृतत्वाद् भयवेपथुस्वेदा‚{\tiny $_{lb}$}‚दीनां । अथ तथाभूतोऽपि भ ‚{\tiny $_{2}$}‚ यादिभिराकुलीक्रियते स तर्ह्यादावेव तथाभूतो वाद‚{\tiny $_{lb}$}‚मपि कर्तुन्नैव धावति । अपि च । यदि परं बाला एवैवं भूता भवन्ति । न च बाल‚{\tiny $_{lb}$}‚व्यवहारानधिकृत्य ‚{\tiny $_{3}$}‚ न्यायशास्त्राणि प्रणीयन्ते । यद्वैवमप्यप्रतिभायामन्तर्भावो ‚{\tiny $_{lb}$}‚ \leavevmode\ledsidenote{\textenglish{119/s}} नैव व्याहन्यते । यस्माद्विविधोत्तरा प्रतिपत्तिरुत्तराज्ञानरूपोत्तरानभिधानरूपा ‚{\tiny $_{4}$}‚ च । ‚{\tiny $_{lb}$}‚तस्माद्यत्किञ्चिदेतत् । अथ परोपतापनार्था तथापि किन्त्रिरभिधीयते । किन्तर्हि ‚{\tiny $_{lb}$}‚कार्य्यमित्याह । ‚{\color{DodgerBlue3}‚साक्षिणामुत्को} चोपसङ्क्रमं कर्णें निवेद्या ‚{\tiny $_{5}$}‚ यमत्रार्थो मया विव‚{\tiny $_{lb}$}‚क्षित इत्युत्तरकालं प्रतिवाद्यनाथो वराकः कष्टाऽप्रतीतद्रुतसंक्षिप्तादिभिः ‚{\tiny $_{lb}$}‚शब्दैरिति शेषः । उपद्रोतव्यः । कस्माद्दु ‚{\tiny $_{6}$}‚ र्भणाः । अप्रतीताः सिंहलभाषादिवदसं ‚{\tiny $_{lb}$}‚केतिकाः । द्रुताः शीध्रमुच्चारिताः । संक्षिप्ता सूत्रवाण्टादिवद्वर्त्तुलीकृतार्थाः । आदि‚{\tiny $_{lb}$}‚ग्रहणेन गोपिता ‚{\tiny $_{7}$}‚ र्थानाङ्ग्रहणं । यथा । सत्वादुर्वायुस्ते दिश्यायोतायाञ्चारात्य‚{\tiny $_{lb}$}‚स्वन्तं पक्षे नोलंवम्विज्ञायैःवेष्टातोयास्पृष्टेशः शमिति । यथा । संप्रति वादी ‚{\tiny $_{lb}$}‚उत्तरप्रतिपत्तौ ‚{\tiny $_{8}$}‚ विमूढ\add{ः} तूष्णीं भवति । पर्षत्प्रतिवादिभ्यां त्रिरभिहितम‚{\tiny $_{lb}$}‚विज्ञातमविज्ञातार्थ \href{http://sarit.indology.info/?cref=ns\%C5\%AB.5.2.9}{न्या० सू० ५।२।९ } मित्यत्र श्लिष्टकष्टादिशब्दप्रयोगस्य ‚{\color{DodgerBlue3}‚मुनि} ‚{\tiny $_{lb}$}‚नानिवारितत्वात् नैवमन्यायं कतुं लभ ‚{\tiny $_{9}$}‚ \leavevmode\ledsidenote{\textenglish{74a/msK}} त इति चेदाह । ‚{\color{DodgerBlue3}‚नहि परोपतापक्रम इत्यादि} ‚{\tiny $_{lb}$}‚किञ्च न परोपतापनाय सन्तः प्रवर्तन्ते शास्त्राणि वा प्रणीयन्ते तैरित्युक्तं दुर्जन‚{\tiny $_{lb}$}‚विप्रतिपत्य ‚{\tiny $_{1}$}‚ धिकरणे सतांसा\edtext{}{\lemma{सतांसा}\Bfootnote{? शा}}स्त्राप्रवृतेरित्यत्र । यतश्च परानुपतापयितुं ‚{\tiny $_{lb}$}‚न सन्तः प्रवर्तन्ते तस्मात्तावद् वक्तव्यं यावदनेन न गृहीतं न त्रिरेव वक्तव्यमित्यधि‚{\tiny $_{lb}$}‚कृतं । अथ ‚{\tiny $_{2}$}‚ वादिना शतधापुनः पुन रभिधाने प्रतिवाद्यतिजडत्वाद् गृहीतुं न शक्नो‚{\tiny $_{lb}$}‚तीति निश्चितन्तदाऽग्रहणसामर्थ्ये कथञ्चिन्निश्चिते साधनप्रयोगात्प्रागेव ‚{\tiny $_{3}$}‚ परि‚{\tiny $_{lb}$}‚हर्तव्यो नानेन सहोद्ग्राहयामीति परिच्छिंन\edtext{}{\lemma{परिच्छिंन}\Bfootnote{? च्छिन्न}}मसामर्थं\edtext{}{\lemma{मसामर्थं}\Bfootnote{? र्थ्यं}}। ग्रहणेऽति‚{\tiny $_{lb}$}‚जाड्यापरनामकप्रतिवादिसम्बन्धियेन वादिना स तथा । कथं ‚{\tiny $_{4}$}‚ तथा परिहरन्ना‚{\tiny $_{lb}$}‚शक्तः शंक्यत इत्याह । पराणन\edtext{}{\lemma{पराणन}\Bfootnote{? न्}}साक्षिणः प्रबोध्य नायं शक्तो वाक्यार्थं बोद्धुं ‚{\tiny $_{lb}$}‚वस्तु त्वेवं व्यवस्थितमिति ॥ ४ ॥
	{\color{gray}{\rmlatinfont\textsuperscript{§~\theparCount}}}
	\pend% ending standard par
      ‚{\tiny $_{lb}$}‚\textsuperscript{\textenglish{120/s}}

	  
	  \pstart \leavevmode% starting standard par
	अविज्ञा ‚{\tiny $_{5}$}‚ तञ्चाज्ञान \href{http://sarit.indology.info/?cref=ns\%C5\%AB.5.2.17}{न्या० सू० ५।२।१७ } मिति भावे निष्ठाविधानात् ‚{\tiny $_{lb}$}‚साधनवाक्यार्थापरिज्ञानं निग्रहस्थानं । तत एव भाष्ये टीकाकृतो विवृण्वन्ति ‚{\tiny $_{lb}$}‚वाक्यार्थविषयस्य विज्ञानस्यानुत्प ‚{\tiny $_{6}$}‚ त्तिरज्ञानमिति । अस्तु वा कर्मण्येव निष्ठाविधानं ‚{\tiny $_{lb}$}‚तथापि वाक्यार्थविषयज्ञानानुत्पत्या विशेषि तं वादिप्रयुक्तं वाक्यमेव प्रतिवादिनो ‚{\tiny $_{lb}$}‚निग्रहस्थान ‚{\tiny $_{7}$}‚ मिति न किञ्चिद्व्याहन्यते । अन्ये पुनर्विवरणेर्थग्रहणं पश्यन्तः सूत्रे‚{\tiny $_{lb}$}‚प्यर्थग्रहणं भ्रान्त्या पठन्ति । अविज्ञातार्थञ्चाज्ञानमिति सोऽन्येषां पाठः । विज्ञा‚{\tiny $_{lb}$}‚तार्थं ‚{\tiny $_{8}$}‚ साधनवाक्यं परिषदा तस्य प्रतिवादिना यदविज्ञातमनवबोधस्तदज्ञान‚{\tiny $_{lb}$}‚मित्येवं भावपक्षेऽक्षरविन्यासः । कर्मपक्षे तु तस्येति नाध्याहर्तव्यमेकवा ‚{\tiny $_{9}$}‚ \leavevmode\ledsidenote{\textenglish{74b/msK}} क्यतयैव ‚{\tiny $_{lb}$}‚तु व्याख्यातं । विज्ञातं पर्षदेति किमर्थं पर्षदाप्यविज्ञाते वादिन एवाविज्ञातार्थ ‚{\tiny $_{lb}$}‚निग्रहस्थानं भवतीति ज्ञापनाय । निग्रहस्थानत्वे कारणमाह । अर्थे खल्व ‚{\tiny $_{1}$}‚ विज्ञाते ‚{\tiny $_{lb}$}‚प्रतिवादी न तस्य प्रतिषेधं ब्रूयादिति । अपरे तूत्तरेण दूषणग्रन्थेन सहैतत् सम्बध्नन्ति ‚{\tiny $_{lb}$}‚तच्चायुक्तं भाष्यवार्तिकग्रन्थत्वादस्य । गम्यत्वमेव साध ‚{\tiny $_{2}$}‚ यति यथाऽननुभाषणेऽनु‚{\tiny $_{lb}$}‚त्तरप्रतिपत्यैव निग्रहस्थानत्वं कथमुत्तराप्रतिपत्तिरित्याह । अप्रदर्शितविषयत्वा‚{\tiny $_{lb}$}‚त्प्रतिवादिनोत्तरप्रतिपत्तिरशक्येति ‚{\tiny $_{3}$}‚ कृत्वाऽप्रदर्शितो विषयो येनेति विज्ञेयं विशे‚{\tiny $_{lb}$}‚षणसमासो वा । तथाहि दूषणस्य विषयख्यापनार्थमनुभाषते तञ्च परित्यज्य ‚{\tiny $_{lb}$}‚यद्यदेव वा ‚{\tiny $_{4}$}‚ दिनाऽनुद्ग्राहितमालजालमनुभाषते । तदानीमुत्तरविषयप्रदर्शनप्रसङ्ग‚{\tiny $_{lb}$}‚मन्तरेण तथाभूतस्यानुभाषणस्य वैयर्थ्यादशक्येतिवर्तते । अ ‚{\tiny $_{5}$}‚ नुग्रहप्रतिपत्यैव निग्रह‚{\tiny $_{lb}$}‚स्थानत्वमिति वा । दार्ष्टान्तिकमुपसंहरति । तथा ज्ञानेऽप्युतराप्रतिपत्यैव निग्रह‚{\tiny $_{lb}$}‚स्थानत्वमिति । यस्मादजानन् प्रति ‚{\tiny $_{6}$}‚ वादिदूषणतद्विषयौ कथमुत्तरविषयञ्च ‚{\tiny $_{lb}$}‚ब्रूयात् । उत्तरविषयो दूष्यः । क्वचित्तु पाठः । कथमुत्तरमुत्तरविषयञ्चोतरमिति । ‚{\tiny $_{lb}$}‚अत्रैवं यदसम्ब ‚{\tiny $_{7}$}‚ न्धः । अजानन्नुत्तरविषयञ्च कथमुत्तरम्ब्रूयादिति । तस्माद्विषया‚{\tiny $_{lb}$}‚ज्ञानमुतराज्ञानञ्च निग्रहस्थानमप्रतिभयैव गम्यत्वात् । अवाच्यमिति वर्तते किं ‚{\tiny $_{8}$}‚ ‚{\tiny $_{lb}$}‚कारणमन्यथैवमनिष्प्रमाणे सत्यप्रतिभाया निर्विषयत्वात् । कथं निर्व्विषयत्वमित्याह । ‚{\tiny $_{lb}$}‚ \leavevmode\ledsidenote{\textenglish{121/s}} ‚{\color{DodgerBlue3}‚अनवधारितार्थो हीत्यादि} । अनवधारि ‚{\tiny $_{9}$}‚ \leavevmode\ledsidenote{\textenglish{75a/msK}} तोर्थपूर्वपक्षस्योत्तरस्य च येन प्रतिवादिना ‚{\tiny $_{lb}$}‚स नानु ‚{\tiny $_{1}$}‚ भाषेत् । अननुभाष्यमाणश्चासौ विषयमप्रदर्श्योत्तरं प्रतिपत्तुन्न शक्नुयादित्यत ‚{\tiny $_{lb}$}‚उतरन्न प्रतिपद्येत । न जानीयान्नाभिदध्याद्वा । कस्मादुत्तरविषययोरज्ञाने सत्युत्तरा‚{\tiny $_{lb}$}‚प्रतिपत्तिरित्यत आह ‚{\color{DodgerBlue3}‚ज्ञानोत्तरतद्विषयस्योत्तराप्रतिपत्तेरसंभवादिति} । ज्ञाता ‚{\tiny $_{lb}$}‚उत्त ‚{\tiny $_{2}$}‚ रतद्विषयो येनेति वृत्तः । तस्मादुभयमेतदुत्तराज्ञानं । विषयाज्ञानञ्च प्रति‚{\tiny $_{lb}$}‚पत्तेरप्रतिभापरपर्य्यायायाः कारणं । ननु चोत्तराज्ञानमेवाप्रतिभा त ‚{\tiny $_{3}$}‚ त्कथं सैवा‚{\tiny $_{lb}$}‚त्मनः कारणत्वेनोपदिश्यते । नोत्तरानभिधानलक्षणाया अप्रतिभासाया विवक्षित‚{\tiny $_{lb}$}‚त्वात् । तदभाव इति तयोरुत्तरविषयाज्ञानयोरभा ‚{\tiny $_{4}$}‚ वे सति प्रतिपत्तिरभिधान‚{\tiny $_{lb}$}‚मुत्तरस्य भवत्येव । इति तस्मात्तयोर्विषयाज्ञानोत्तरज्ञानयोरज्ञानसंज्ञितेन निग्रह‚{\tiny $_{lb}$}‚स्थानेनाप्रतिभा निग्रहस्थाना ‚{\tiny $_{5}$}‚ त् पृथक्वचने सत्यप्रतिभायाः को विषय इति ‚{\tiny $_{lb}$}‚वक्तव्यं । न चेद्विषयो भण्यते । तदा निर्विषयत्वादवाच्यैव स्यादप्रतिभा । तयोर‚{\tiny $_{lb}$}‚ज्ञानानुभाषणयो ‚{\tiny $_{6}$}‚ ः पृथग्वचन इत्यन्ये व्याचक्षते । अज्ञानाप्रतिभयोर्विषयभेदव्यव‚{\tiny $_{lb}$}‚स्थापनाय परः प्राह । नोत्तरज्ञानमज्ञानमुच्यते । यतोऽप्रतिभा निर्विषयत्वादवा ‚{\tiny $_{7}$}‚ ‚{\tiny $_{lb}$}‚च्यम्भवेत् । किन्तर्ह्यज्ञानमित्याह । विषयाज्ञानं । एवमपि कथमप्रतिभा विषय‚{\tiny $_{lb}$}‚वती भवतीत्याह । ‚{\color{DodgerBlue3}‚ज्ञाते विषये} सत्युत्तरकालमुत्तराज्ञानात् । प्रतिवादी त ‚{\tiny $_{8}$}‚ दुतरन्न ‚{\tiny $_{lb}$}‚प्रतिपद्येत न ब्रूयात् । अतोऽस्ति विषयोऽप्रतिभायाः । अज्ञानाक्रान्तः । एवमप्य‚{\tiny $_{lb}$}‚वाच्य ‚{\tiny $_{9}$}‚ \leavevmode\ledsidenote{\textenglish{75b/msK}} त्वान्नमुच्यस इत्याह । ‚{\color{DodgerBlue3}‚एवन्तर्हीति} । अज्ञानेनानुभाषणस्याक्षेपमेव साधयति । ‚{\tiny $_{lb}$}‚नहि विषयं सम्यक् प्रतिपद्यमानः कश्चित् सचेतनो नानुभाषेतेति नानुभाषणम ‚{\tiny $_{1}$}‚ ‚{\tiny $_{lb}$}‚ज्ञानात्पृथग्वाच्यं । अपिचैवमप्रतिभाप्यननुभाषणवदज्ञानात्पृथग्नवाच्येत्याह । ‚{\color{DodgerBlue3}‚उत्त‚{\tiny $_{lb}$}‚राज्ञानस्य चाक्षेपादिति} । इदं व्याचष्टे ‚{\color{DodgerBlue3}‚विषये} त्यादिना । ज्ञाते विषय ‚{\tiny $_{2}$}‚ इत्यादि परः । ‚{\tiny $_{lb}$}‚  \leavevmode\ledsidenote{\textenglish{122/s}} इदमुक्तम्भवति द्विधोत्तराज्ञानविषयाज्ञानसहचरञ्च विषयज्ञानसहचरञ्च । तत्रा‚{\tiny $_{lb}$}‚द्यस्य विषयाज्ञाने नाक्षेपेऽप्युत्तरमनाक्षिप्तमेव ‚{\tiny $_{3}$}‚ ततो द्वितीयापेक्षयाऽप्रतिभायाः ‚{\tiny $_{lb}$}‚पृथगुपादानमिति । अनवस्थैव निग्रहस्थानानां प्रसज्यत इत्याह । ‚{\color{DodgerBlue3}‚एवन्तर्ही} ‚{\tiny $_{lb}$}‚त्यादि । यथेत्याद्यस्यैव वि ‚{\tiny $_{4}$}‚ भागः । तथा ज्ञानयोरपीति विषयोत्तराज्ञानयोरपि । ‚{\tiny $_{lb}$}‚सर्व्वस्योत्तरस्य विषयस्य चाज्ञानं । आदिग्रहणेन द्वित्रिचतुर्भागाद्यवरोधः । वि ‚{\tiny $_{5}$}‚ ष‚{\tiny $_{lb}$}‚योत्तराज्ञानयोः सङ्ग्रहवचने दोष इति चेदाह । न चेति । यथा न दोषस्तथागुणोपि ‚{\tiny $_{lb}$}‚नास्तीति चेदाह । गु ‚{\color{DodgerBlue3}‚णश्च लाघवसंज्ञः} स्यादिति सं ‚{\tiny $_{5}$}‚ ग्रहवचनं न्याय्यं । अप्रतिभा‚{\tiny $_{lb}$}‚विषयत्वान्न पृथग्वचनं । अप्रतिभावचनेनैवानयोः सङ्ग्रह इत्यर्थः । न केवलमन‚{\tiny $_{lb}$}‚योरेवापृथग्वचनं । न्याय्यमपि त्वन्ये ‚{\tiny $_{7}$}‚ षामपीत्याह । ‚{\color{DodgerBlue3}‚अपि चेत्यादि} । तदुभयवच‚{\tiny $_{lb}$}‚नेनैवेति । हेत्वाभासाऽप्रतिभयोरेव वचनेन सर्वप्रतिज्ञाहान्यननुभाषणाद्युक्तं । नहि ‚{\tiny $_{lb}$}‚कश्चिद्ध्यन्यस्साधनवादी ‚{\tiny $_{8}$}‚ प्रतिपक्षधर्ममभ्युनुजानाति प्रतिज्ञाम्वा प्रतिज्ञासाध‚{\tiny $_{lb}$}‚नायोपादत्त इत्यादि वाच्यं । तदा न कञ्चि\add{द्}दूषणं व्यक्तमेव यन्नानुभाषते । ‚{\tiny $_{lb}$}‚कथां विक्षिपति । परम ‚{\tiny $_{9}$}‚ \leavevmode\ledsidenote{\textenglish{76a/msK}} तञ्चानुजानातीदि\edtext{}{\lemma{तञ्चानुजानातीदि}\Bfootnote{? ति}}वक्तव्यं । तदुभयाक्षेपेपि प्रपञ्चो ‚{\tiny $_{lb}$}‚गुणवानतस्तदवचनादरोमुनेरिति चेदाह । तदुभयाक्षिप्तेषु प्रभेदेषु गुणातिशयम ‚{\tiny $_{1}$}‚ ‚{\tiny $_{lb}$}‚न्तरेण । अनुपलभ्यमानत्वाद् गुणस्य प्रपञ्चवचनादरेऽतिप्रसङ्गात् । कक्षपिट्टिता‚{\tiny $_{lb}$}‚ \leavevmode\ledsidenote{\textenglish{123/s}} दीनामभिधानप्रसङ्गात् । अतो व्यर्थः प्रपञ्चो महामुनिनाक्रियत ‚{\tiny $_{2}$}‚ ॥ ० ॥
	{\color{gray}{\rmlatinfont\textsuperscript{§~\theparCount}}}
	\pend% ending standard par
      ‚{\tiny $_{lb}$}‚

	  
	  \pstart \leavevmode% starting standard par
	परपक्षप्रतिषेधे कर्तव्ये उत्तरं दूषणं यदा न प्रतिपद्यते न वेत्ति नाभिदधाति ‚{\tiny $_{lb}$}‚तदा निगृहीतो वेदितव्य इतीयान् परग्रन्थः । साध्वेतन्निग्रहस्थानं । अतएवास्मा‚{\tiny $_{lb}$}‚भिरपीदमदोषोद्भावनमित्यत्रोक्तमित्येतत् मत्वाऽभ्यनुजानाति । ‚{\color{DodgerBlue3}‚साधनेत्यादि} । ‚{\tiny $_{lb}$}‚साधन वचनानन्तरं प्रतिवादिना दूषणम्वक्तव्यं । स य ‚{\tiny $_{4}$}‚ दा सर्व्वं तदकृत्वा सर्व्वा‚{\tiny $_{lb}$}‚नुक्रमानुभाषणेन श्लोकपाठेन सभासम्वर्ण्णनेनान्येन कालन्नयति तदासौ व्यर्थं ‚{\tiny $_{lb}$}‚निष्प्रयोजर्न कालङ्गमयन्कर्त्तव्यस्य दूषणाभि ‚{\tiny $_{5}$}‚ धानस्य प्रतिपत्त्याऽननुष्ठानेन निगृ‚{\tiny $_{lb}$}‚ह्यते । व्यर्थस्येदं क्रियायाः कालस्य विशेषणं ॥ ० ॥
	{\color{gray}{\rmlatinfont\textsuperscript{§~\theparCount}}}
	\pend% ending standard par
      ‚{\tiny $_{lb}$}‚

	  
	  \pstart \leavevmode% starting standard par
	कार्यव्यासङ्गः कर्णीयोपन्यासः कथाविच्छेदः क ‚{\tiny $_{6}$}‚ थानिवृत्तिः । यथा जीर्ण्ण‚{\tiny $_{lb}$}‚कला मे बाधते । सम्प्रति वक्तुँ न शक्नोमि पश्चात् कथयिष्यामीति एवमादिना ‚{\tiny $_{lb}$}‚प्रकारेण कथामुद्ग्राहणे काचिच्छिनत्तिः । निग्रहस्थाने ‚{\tiny $_{7}$}‚ कारणमाह । एकतरस्य ‚{\tiny $_{lb}$}‚वादिनः प्रतिवादिनो वाऽसाधनाङ्गवचनेनादोषोद्भावनेन च निगृहणन्ती निग्रह‚{\tiny $_{lb}$}‚पर्यवसाना कथा । तस्याञ्च तथाभूतायां प्रस्तुता ‚{\tiny $_{8}$}‚ यां स स्वयमेव कथांतं कथा ‚{\tiny $_{lb}$}‚पर्यवसानं प्रतिपद्यत इति निग्रहस्थानमेतत् । अत्राचार्योब्रूत \add{।} इदमपि कार्यव्यास‚{\tiny $_{lb}$}‚ञ्जनं यदि तावत् पूर्वपक्षवादी कुर्यात् ‚{\tiny $_{9}$}‚ \leavevmode\ledsidenote{\textenglish{76b/msK}} साधनाभिधानशक्तिविकलत्वाद् व्याजो- ‚{\tiny $_{lb}$}‚पक्षेपमात्रेण येन केनचिच्छलेनेत्यर्थः । न पुनर्भूतस्य तथाविधकथोपरोधिनः ‚{\tiny $_{lb}$}‚कार्यस्य भावे सति कुर्यादिति वर्त्तते । तथा वि ‚{\tiny $_{1}$}‚ धामुद्ग्राहणिकारूपाङ्कथामुपरोद्धुं ‚{\tiny $_{lb}$}‚शीलं यस्य कार्यस्याजीर्ण्णकला कुक्षिशूलगेहु \add{?} दाहाद्यैस्तत्तथा । यदि सद्भावे‚{\tiny $_{lb}$}‚नैव तस्य तस्याम्वेलायां कुक्षिगलशूलगेहु \add{?} दा ‚{\tiny $_{2}$}‚ हादयो भवन्ति तथा सति नैव ‚{\tiny $_{lb}$}‚निग्रहस्थानमित्यर्थः । यदा पुनर्व्याजमात्रेणैव करोति तदा तस्य पूर्व्वपक्षवादिनः ‚{\tiny $_{lb}$}‚ \leavevmode\ledsidenote{\textenglish{124/s}} स्वसाधनासामर्थ्यपरिच्छेदादेव विक्षेप ‚{\tiny $_{3}$}‚ ः स्यात्ततः किमित्याह । तथा चेदं विक्षेप‚{\tiny $_{lb}$}‚सञ्ज्ञितन्निग्रहस्थानमर्थान्तर एवान्तर्भवेत् । रूपसिद्धिनामादिव्याख्यानसमान‚{\tiny $_{lb}$}‚त्त्वात् करणीयोपन्यास ‚{\tiny $_{4}$}‚ स्य । हेत्वाभासेष्वेवान्तर्भवेदित्यधिकृतं । कस्मादसमर्थञ्च ‚{\tiny $_{lb}$}‚तत्साधनञ्च तस्याभिधानात् । किञ्चेदं निरर्थकापार्थकाभ्यां सकाशान्न भिद्यते । ‚{\tiny $_{lb}$}‚किं ‚{\tiny $_{5}$}‚ कारणं प्रकृतञ्च तत्साधनञ्च तेनासम्बद्धा च सा प्रतिपत्तिश्च ततः साधन‚{\tiny $_{lb}$}‚वाक्येन सहास्य दशदाडिमादिवचनस्येव जबगडादिवर्ण्णक्रमस्येव च ‚{\tiny $_{6}$}‚ सम्बद्धानुप‚{\tiny $_{lb}$}‚लम्भादित्यर्थः । किञ्चिन्मात्रभेदान्निमित्तलेशेन पृथगुक्तमिति चेदाह । ‚{\color{DodgerBlue3}‚अति ‚{\tiny $_{lb}$}‚प्रसङ्गश्चे} त्यादि । असम्बद्धासाधनवाक्येन प्रतिपत्तिर्येषां ‚{\tiny $_{7}$}‚ प्रतिभेदानान्तेऽसम्बद्ध‚{\tiny $_{lb}$}‚साधनवाक्यप्रतिपत्तयः । ते च ते प्रभेदाश्च तेषामिति कार्यं एवन्तावत् पूर्वपक्षवादि‚{\tiny $_{lb}$}‚सम्बन्धेन विक्षेपस्य पृथगनभिधान ‚{\tiny $_{8}$}‚ मुक्तं ॥ अधुना प्रतिवादिसम्बन्धेनाप्याह । ‚{\tiny $_{lb}$}‚ ‚{\color{DodgerBlue3}‚अथोत्तरपक्ष} वाद्येवं बलासकलात्मकण्ठं क्षिणोतीत्यादिना प्रक्रमेण कथां विक्षिपेत् ‚{\tiny $_{lb}$}‚तदानीन्तस्याप्यु ‚{\tiny $_{9}$}‚ \leavevmode\ledsidenote{\textenglish{77a/msK}} त्तरपक्षवादिनः साधनानन्तरमुत्तरे प्रतिपत्तव्ये सति तदप्रतिपत्त्या ‚{\tiny $_{lb}$}‚तस्योत्तरस्यानभिधानेन विक्षेपप्रतिपत्तिर्यासाऽप्रतिभायामर्थान्तरे वान्तर्भवती ‚{\tiny $_{1}$}‚ ति ‚{\tiny $_{lb}$}‚परस्तु यथोक्तमन्तर्भावमसहमानश्चोदयति ।
	{\color{gray}{\rmlatinfont\textsuperscript{§~\theparCount}}}
	\pend% ending standard par
      ‚{\tiny $_{lb}$}‚

	  
	  \pstart \leavevmode% starting standard par
	\hphantom{.}‚{\color{DodgerBlue3}‚ननु नावश्य} मिति तदेव द्रढयति \add{।} भवति ह्यनिबद्धेनापि साधनवाक्येनास‚{\tiny $_{lb}$}‚म्बद्धे नापि कथाप्रबन्धेन ‚{\tiny $_{2}$}‚ परप्रतिभाहरणायान्तशो जननीव्यभिचारचोदनेनापि ‚{\tiny $_{lb}$}‚विवाद इति । आचार्य आह । नासम्भवादेवंविधस्य विवादस्य । यस्मादेकत्र ‚{\tiny $_{lb}$}‚शब्दादावधिकर ‚{\tiny $_{3}$}‚ णे नित्यत्वानित्यत्वादिप्रतिज्ञानाविरुद्धावभ्युपगमौ ययोर्वादिप्रति ‚{\tiny $_{lb}$}‚वादिनोस्तयोर्विवादः स्यात् । कुत एतदित्याह । अविरुद्धावभ्युपग ‚{\tiny $_{4}$}‚ मौ ययोस्तौ ‚{\tiny $_{lb}$}‚ \leavevmode\ledsidenote{\textenglish{125/s}} तथा न विद्येते विरुद्धाविरुद्धयोरभ्युपगमौ ययोः पुरुषयोस्तावभ्युपगमौ । तयो‚{\tiny $_{lb}$}‚र्विवादाभावात् । तत्रैतस्मिन्व्यवस्थिते न्याय ‚{\tiny $_{5}$}‚ निर्धारणे वा तत्र शब्दः । ‚{\tiny $_{lb}$}‚एकस्य वादिनः प्रतिवादिनोवश्यं प्राग्वचनप्रवृत्तिः । यौगपद्येन किन्न ब्रूत ‚{\tiny $_{lb}$}‚इत्याह । ‚{\color{DodgerBlue3}‚युगपत्प्रवृत्तौ} स्वस्थात्मना ‚{\tiny $_{6}$}‚ मप्रवृत्तेरिति सम्बन्धः । एतदेव कुत ‚{\tiny $_{lb}$}‚इत्याह । परस्पर वचन श्रवणावधारणोत्तराणामसम्भवेन करणभूतेन प्रवृत्ति‚{\tiny $_{lb}$}‚वैफल्यात् । यदि हि परस्परवच ‚{\tiny $_{7}$}‚ नस्यासङ्करेण श्रवणम्भवेत्ततस्तदर्थमवधार‚{\tiny $_{lb}$}‚यत्युत्तरञ्च । युगपत्प्रवृत्तौ च दिगम्बरपाठकलकल इव सर्व्वमेतन्न संभवति तस्मा‚{\tiny $_{lb}$}‚दवश्यमेकस्य प्राग्वचन ‚{\tiny $_{8}$}‚ प्रवृत्तिः । अतस्तेन च स्वस्थात्मना पूर्व्वपक्षवादिनाऽनित्यं ‚{\tiny $_{lb}$}‚शब्दं साधयामीत्यादिना स्वोपगमोपन्यासे कृते सत्यवश्यं साधनं वक्तव्यं । अन्यथेति ‚{\tiny $_{lb}$}‚हेत्वनभिधा ‚{\tiny $_{9}$}‚ \leavevmode\ledsidenote{\textenglish{77b/msK}} ने परेसां\edtext{}{\lemma{परेसां}\Bfootnote{? षां}}साक्षिप्रभृतीनामप्रतिपत्तेः । अपरेण चेत्युत्तरपक्ष- ‚{\tiny $_{lb}$}‚वादिना तत्सम्बन्धिवादिप्रोक्तसाधनसम्बन्धि दूषणं वक्तव्यमिति वर्त्तते \add{।} ‚{\tiny $_{lb}$}‚तस्मादुभयोर्वादिप्र ‚{\tiny $_{1}$}‚ \add{ति} वादिनोरसम्यक् प्रवृत्तौ सत्यां हेत्वाभासाप्रतिभयोः ‚{\tiny $_{lb}$}‚संग्रह इति सर्व्वो न्यायप्रवृत्तः पूर्व्वोत्तरपक्षोपन्यासो द्वयं हेत्वाभासाप्रतिभाञ्च ‚{\tiny $_{lb}$}‚नातिपतति ।
	{\color{gray}{\rmlatinfont\textsuperscript{§~\theparCount}}}
	\pend% ending standard par
      ‚{\tiny $_{lb}$}‚

	  
	  \pstart \leavevmode% starting standard par
	ननु च यदि न्यायः प्रवृत्तः कथन्तत्रास्य द्वयस्याधिकारः । कथञ्चैकत्र ‚{\tiny $_{lb}$}‚धर्मिणि विरुद्धावुपन्यासौ न्यायप्रवृत्ताववश्यं हि तत्रैकेनोपन्यासेन न्यायं प्रवृत्तेन ‚{\tiny $_{lb}$}‚भाव्यं । अन्य ‚{\tiny $_{3}$}‚ था धर्मीद्व्यात्मको भवेत् । नाभिप्रायापरिज्ञानात् । नेदम्भवता ‚{\tiny $_{lb}$}‚न्यायप्रवृत्तत्वमाचार्येण विवक्षितं पर्यज्ञायि । न्यायप्रवृत्तौ\add{?} हि पूर्व्वोत्तरपक्षोप ‚{\tiny $_{4}$}‚ ‚{\tiny $_{lb}$}‚न्यासस्य युगपत्प्रवृत्यभावेन जननी व्यभिचारवेदनाद्यभावेन चाभिप्रेतं । एतेनैकत्र ‚{\tiny $_{lb}$}‚ह्यधिकरणे विरुद्धाभ्युपगमयोर्विवादः स्यादित्यादि ‚{\tiny $_{5}$}‚ ना वितण्डा प्रत्युक्ता । कथं ‚{\tiny $_{lb}$}‚प्रत्युक्तेत्याह । ‚{\color{DodgerBlue3}‚अभ्युपगमाभावे विवादाभावात्} । इदमुक्तम्भवति । स्वपक्षस्थापना ‚{\tiny $_{lb}$}‚हीनो वाक्यसमूहो वितण्डे त्युच्य ‚{\tiny $_{6}$}‚ ते \href{http://sarit.indology.info/?cref=ns\%C5\%AB.1.2.3}{न्या० सू० १।२।३ } । यदि चवैतण्डिक्रस्य ‚{\tiny $_{lb}$}‚ \leavevmode\ledsidenote{\textenglish{126/s}} स्वपक्षो नास्ति विवादस्तर्हि कथमिति वक्तव्यं । परपक्षप्रतिषेधार्थम्वैतण्डिकः ‚{\tiny $_{lb}$}‚प्रवर्त्तत इति चेत् । परपक्षप्रतिषेध एव तर्ह्यस्य ‚{\tiny $_{7}$}‚ स्वपक्षस्थापनेति वितण्डालक्षणं ‚{\tiny $_{lb}$}‚विशीर्यते । तथा हि यो येनाभ्युपगतः स तस्य स्वपक्षः । परपक्षप्रतिषेधश्च तेनाभ्युप‚{\tiny $_{lb}$}‚गतः स्वपक्षतां नातिवर्तत इति । ‚{\tiny $_{8}$}‚ यदा तर्ह्युपगम्य वादं प्रतिज्ञामात्रेण विफलतया ‚{\tiny $_{lb}$}‚परिषच्छारद्येन व्याकुलीकृतत्त्वादित्यर्थः । न किञ्चित् साधनं तदायासं वा वक्ति । ‚{\tiny $_{lb}$}‚अन्यद्वा किञ्चित् प्रलप ‚{\tiny $_{9}$}‚ \leavevmode\ledsidenote{\textenglish{78a/msK}} ति । साधनतदाभासव्यतिरिक्तं काको विरूपं विरौति ‚{\tiny $_{lb}$}‚नूनमयं मे गेहे विपदं सूचयति तदलमनेन विवादेन । यामि तावद् गेहे किन्नु मे पितु‚{\tiny $_{lb}$}‚र्मरणमन्यद्वा वर्त्तत इत्या ‚{\tiny $_{1}$}‚ दि । तथा कथं हेत्वाभासान्तर्भावः । साधनाभावाद्धेत्वा‚{\tiny $_{lb}$}‚भासासम्भवं मन्यते । उत्तराप्रतिपत्तिरपि नास्त्येव पूर्व्वपक्षवादित्वादित्यभिप्रायः । ‚{\tiny $_{lb}$}‚तदनेन द्वयन्नातिपतती ‚{\tiny $_{2}$}‚ त्येतद्विघटयितुमिच्छति परः । आचार्य आह । ‚{\color{DodgerBlue3}‚असमर्थित‚{\tiny $_{lb}$}‚साघनाभिधान एवमुक्तं द्वयं नातिपततीति} प्रोक्तसाधन एतदुक्तमिति यावत् । ‚{\tiny $_{lb}$}‚अप्रोक्ते तु कथं प्रतिपत्त ‚{\tiny $_{3}$}‚ व्यमित्याह । अनभिधानान्यभिधानयोरपि सतोः पराजयः ‚{\tiny $_{lb}$}‚एवेत्युक्तं प्रकरणावतार एव । तदेव स्मरयति । अभ्युपगम्यवादमसाधनाङ्गवच‚{\tiny $_{lb}$}‚नादिति । ‚{\tiny $_{4}$}‚ तथाहि तत्र व्याख्यातं । साधनाङ्गस्यानुच्चारणं । साधनाङ्गाद्वा य‚{\tiny $_{lb}$}‚दन्यस्याभिधानं तत्सर्व्वमसाधनाङ्गवचनमिति । एतेनेत्यन्याभिधानेन पराजय‚{\tiny $_{lb}$}‚वचने ‚{\tiny $_{5}$}‚ नाधिकस्य पुनरुक्तस्य च प्रतिज्ञादेर्वचनस्य च निग्रहस्थानत्वं व्याख्यातं । ‚{\tiny $_{lb}$}‚कथमित्याह । ‚{\color{DodgerBlue3}‚तदपि हीत्यादि} । अनेनैतदाह । यद्युक्तियुक्त ‚{\color{DodgerBlue3}‚मक्षपादेन} किञ्चि ‚{\tiny $_{6}$}‚ न्नि‚{\tiny $_{lb}$}‚ग्रहस्थानमुक्तन्तदस्माभिरसाधनाङ्गवचनपदेनैव संगृहीतमिति यद्येवं प्रतिज्ञा‚{\tiny $_{lb}$}‚देर्वचनस्य चेति किमर्थयुक्तन्नहि प्रतिज्ञोपनयनिगमनानां वचनन्नि ‚{\tiny $_{7}$}‚ ग्रहस्थान ‚{\color{DodgerBlue3}‚मक्ष‚{\tiny $_{lb}$}‚पादे} नोक्तं । प्रत्युत तदवचनमेव निग्रहस्थानतया । यदिष्टं हीनमन्यतमेनाप्यव‚{\tiny $_{lb}$}‚यवेन न्यून \href{http://sarit.indology.info/?cref=ns\%C5\%AB.5.2.12}{न्या० सू० ५।२।१२ } मिति । एवं तर्हि दृष्टान्तार्थमेतद्यथा तस्याप्रतीत ‚{\tiny $_{8}$}‚ ‚{\tiny $_{lb}$}‚प्रत्यायनशक्तिविकलत्वादसाधनाङ्गवचनपदेनाभिधानं । तथाधिकपुनर्वचनयो‚{\tiny $_{lb}$}‚रपीति । तत एव च द्वितीयश्चकार इव शब्दार्थे वर्त्तते । अन्यथा पुनरु ‚{\tiny $_{9}$}‚ \leavevmode\ledsidenote{\textenglish{78b/msK}} क्तस्य चेत्ययं ‚{\tiny $_{lb}$}‚बोध्यर्थः स्यात् । केचित्तूत्तरञ्चकारन्न पठन्ति । तैः पुनरुक्तव्याख्यानमेव ‚{\tiny $_{lb}$}‚प्रतिज्ञादेर्वचनस्य चेत्येतद् व्याख्येयं । एवमपि न युक्तमक्षपादेनैवम्विधस्य पुन ‚{\tiny $_{1}$}‚ रुक्त‚{\tiny $_{lb}$}‚स्यानिष्टत्वान्नास्ति दोषः । पूर्व्वतुल्यधर्मतयाऽस्यापि पुनरुक्तेऽन्तर्भावितत्वात् ॥ ० ॥
	{\color{gray}{\rmlatinfont\textsuperscript{§~\theparCount}}}
	\pend% ending standard par
      ‚{\tiny $_{lb}$}‚\textsuperscript{\textenglish{127/s}}

	  
	  \pstart \leavevmode% starting standard par
	\hphantom{.}स्वपक्षदोषाभ्युपगमात् परपक्षदोषप्रसङ्गो मतानुज्ञा \href{http://sarit.indology.info/?cref=ns\%C5\%AB.5.2.20}{न्या० सू० ५।२।२० } ‚{\tiny $_{lb}$}‚ ‚{\tiny $_{18b3}$}‚ दोषपरि ‚{\tiny $_{2}$}‚ हारे वक्तव्ये दोषस्यापरिज्ञानात् परमतमनुजानात्यतो नि‚{\tiny $_{lb}$}‚गृह्यते । तदाह परेण वादिना चोदितं पर्यनुयुक्तं दोषमनुवृत्या परिहृत्य भवतोप्ययं ‚{\tiny $_{lb}$}‚दोष इति ब्र ‚{\tiny $_{3}$}‚ वीति । ‚{\color{DodgerBlue3}‚यथा भवांश्चौरः पुरुषत्वा} ‚{\tiny $_{16b4}$}‚ च्छवरादि \add{वदि} त्युक्ते ‚{\tiny $_{lb}$}‚वादिना स प्रतिवादी तं वादिनं प्रतिब्रूयात् । भवानपि चौर इति सोपि ‚{\tiny $_{lb}$}‚शब्दप्रयोगादात्मनश्चौर ‚{\tiny $_{4}$}‚ त्वमभ्युपगम्य परपक्षे तन्दोषमासञ्जयन्नापादयत्यपरेण ‚{\tiny $_{lb}$}‚वादिना यन्मतं प्रतिवादिनश्चौरत्त्वं तदनुजानाति । तथा हि ते न मुक्तसंस\edtext{}{\lemma{मुक्तसंस}\Bfootnote{? श}} ‚{\tiny $_{lb}$}‚ यन्तावदात्मन ‚{\tiny $_{5}$}‚ श्चौरत्त्वं प्रतिपत्तुमन्यथा नापि तमभिदध्यात् । वादिनि तु ‚{\tiny $_{lb}$}‚तदस्तिनास्तीति चिन्त्यमतो मतानुज्ञा निग्रहस्थानं । इदमाचार्यो निराकरोति । ‚{\tiny $_{lb}$}‚अत्रापि ‚{\tiny $_{6}$}‚ ‚{\tiny $_{18b5}$}‚ यद्ययमभिप्राय उत्तरवादिनः पुरुषत्वाच्चौंरो भवानपि स्यादह‚{\tiny $_{lb}$}‚मिव । न च भवतात्मैवं चौरत्वेनेष्टस्तन्नायं पुरुषत्वादिति चौर्ये साध्ये हेतुरचौ‚{\tiny $_{lb}$}‚रेपि भव ‚{\tiny $_{7}$}‚ ति विपक्षभूते वृत्तेरनैकान्तिकदोषदुष्टत्वादिति । तदस्मिन्प्रतिवादिनोऽभि‚{\tiny $_{lb}$}‚प्राये न कश्चित्तस्य दोषो मतानुज्ञालक्षणोऽन्यो वा । कस्मादनभिमते चौरत्त्वे न ‚{\tiny $_{lb}$}‚रू ‚{\tiny $_{8}$}‚ पेण तस्य वादिन आत्मनि विपक्षभूते हेतोः सत्वप्रदर्शनेन प्रकारेण दूषणात् । ‚{\tiny $_{lb}$}‚विदग्धभङ्गाव्यभिचारोद्भावनादिति यावत् । औद्योतकरं \edtext{\textsuperscript{*}}{\lemma{*}\Bfootnote{\href{http://sarit.indology.info/?cref=nv.5.2.21}{न्यायवार्त्तिके ५।२।२१ पृष्ठ ५५९}}}चोद्यमाशङ्कते ‚{\tiny $_{9}$}‚ \leavevmode\ledsidenote{\textenglish{79a/msK}} ‚{\tiny $_{lb}$}‚प्रसङ्गमन्तरेण भवानपि स्यादित्येवमाञ्जसेनैव मृजुनैव क्रमेण किन्न ‚{\tiny $_{lb}$}‚व्यभिचारितो हेतुस्त्वय्यपि अचौरे वर्त्तते पुरुषत्वमतोऽनैकान्तिकत्वमिति । ‚{\tiny $_{lb}$}‚तस्माद्यत ‚{\tiny $_{1}$}‚ एवासावकौटिल्ये कर्तव्ये कौटिल्यमाचरति तत एव निगृह्यत इति । ‚{\tiny $_{lb}$}‚  ‚{\tiny $_{lb}$}‚ \leavevmode\ledsidenote{\textenglish{128/s}} आचार्य आह । यत्किञ्चिदेतद ‚{\tiny $_{18b6}$}‚ ‚{\color{DodgerBlue3}‚औद्योतकरं} वचो यस्मात् सन्ति ह्येवं ‚{\tiny $_{lb}$}‚प्रकारा वैदग्ध्यप्रवर्तिता व्य ‚{\tiny $_{2}$}‚ वहारा लोके । तथा हि मातरो भावत्क्यो बन्धक्यः ‚{\tiny $_{lb}$}‚स्त्रीत्वादितरबन्धकीवदित्युक्ताः पशुपालादयोपि जडजनङ्गमादिजनसाधारणं ‚{\tiny $_{lb}$}‚वैदग्ध्यमनुसरं ‚{\tiny $_{3}$}‚ तः प्रत्यवतिष्ठन्ते । तावकीनापि माता तथा स्यादिति न च तेऽनेन ‚{\tiny $_{lb}$}‚प्रकारेण स्वस्याः स्वस्या मातुर्बन्धकीत्वं प्रतिपद्यन्ते । अपि तु भङ्ग्या हेतुव्यभि ‚{\tiny $_{4}$}‚ चार‚{\tiny $_{lb}$}‚चोदनया परं प्रतिवदन्ति । तस्मादेवं बालहालिकादिलोकप्रकटमपि व्यवहारालोक‚{\tiny $_{lb}$}‚मपसारयता यदि परमुद्योतकरत्वमेवो ‚{\color{DodgerBlue3}‚द्योत ‚{\tiny $_{5}$}‚ करेण} उद्योतितमात्मनः । अथोच्यते ‚{\tiny $_{lb}$}‚नैवासौ भंग्या व्यभिचारमादर्शयत्यपि तु तस्य साधनस्य सम्यक्त्वमभ्युपगम्यैव तेन ‚{\tiny $_{lb}$}‚दोषेण परमपि ‚{\tiny $_{6}$}‚ कलङ्कयतीत्यत आह । ‚{\color{DodgerBlue3}‚अथ तदुपक्षेप\add{ः} पुरुषत्वाद् भवांश्चौर} इत्ये‚{\tiny $_{lb}$}‚नमभ्युपगच्छत्येव तदाप्यसौ तत्साधन उत्तराप्रतिपत्यैव निग्रहार्हो नापरत्र ‚{\tiny $_{7}$}‚ वादिनि‚{\tiny $_{lb}$}‚स्वदोषस्य चौरत्वस्योपक्षेपात् । निग्रहार्ह इति वर्त्तते । इदमेवोपोद्बलयति । ‚{\tiny $_{lb}$}‚तत्साधननिर्दोषतायां ‚{\tiny $_{18b8}$}‚ ह्यंगीकृतायामिति शेषः । तस्योपक्षेप ‚{\tiny $_{8}$}‚ स्याभ्युपगम ‚{\tiny $_{lb}$}‚एव यः स एवोत्तराप्रतिपत्तिरिति तावतैवोत्तराप्रतिपत्तिमात्रेणैवापरत्र दोषप्रसञ्ज‚{\tiny $_{lb}$}‚नात् । पूर्वसाधननिग्रहस्य सतः प्रतिवादि ‚{\tiny $_{9}$}‚ \leavevmode\ledsidenote{\textenglish{79b/msK}} नः आपन्नः प्राप्तो निग्रहो येन तस्येति ‚{\tiny $_{lb}$}‚चेति विग्रहः । परदोषोपक्षेपस्य मतानुज्ञालक्षणस्यानपेक्षणीयत्वात्पराजितपराजया‚{\tiny $_{lb}$}‚भावादिति भावः ॥ ० ॥
	{\color{gray}{\rmlatinfont\textsuperscript{§~\theparCount}}}
	\pend% ending standard par
      ‚{\tiny $_{lb}$}‚

	  
	  \pstart \leavevmode% starting standard par
	\hphantom{.}‚{\color{DodgerBlue3}‚निग्र ‚{\tiny $_{1}$}‚ हप्राप्तस्यानिग्रहः पर्यनुयोज्योपेक्षणं} ‚{\tiny $_{18b9}$}‚ पर्यनुयोज्यो नाम निग्रह‚{\tiny $_{lb}$}‚प्राप्तस्यो\add{पे}क्षणन्निग्रहप्राप्तोसीत्यनभिधानं । क\add{ः} पुनरिदं पर्यनुयोज्योपेक्षणं ‚{\tiny $_{lb}$}‚निग्रहस्थानं ‚{\tiny $_{2}$}‚ चोदयति । न तावत् पर्यनुयोज्य इति युक्तं । असम्भवात् । न ह्यस्ति ‚{\tiny $_{lb}$}‚सम्भवो यत् परदोषप्रतिपादनार्थमात्मनो दोषवत्वमसावभ्युपेति । निग्रहप्राप्तः सन्न ‚{\tiny $_{lb}$}‚ \leavevmode\ledsidenote{\textenglish{129/s}} ‚{\tiny $_{3}$}‚ हमनेनोपेक्षितो निग्रहस्थानस्यापरिज्ञानात् । तस्मादयन्दोषवानिति नाप्युपेक्ष इति ‚{\tiny $_{lb}$}‚युक्तं । यस्मादसौनि जानात्येवायं निग्रहप्राप्त इति । तथा ‚{\tiny $_{4}$}‚ ह्यपरिज्ञानादेवासौ नानु‚{\tiny $_{lb}$}‚युंक्ते निग्रहं प्राप्तोसीति । परिज्ञाने वा कथमुपेक्षेत । उपेक्षणे वा समचित्तः कथमेवं ‚{\tiny $_{lb}$}‚प्रकटयेदयं मयोपेक्षितः ‚{\tiny $_{5}$}‚ स दोषस्ततो मम पर्यनुयोज्योपेक्षणं निग्रहस्थानमिति । ‚{\tiny $_{lb}$}‚न चान्यस्तृतीयः कश्चिदिहानुषङ्गी तत्केनेदं चोदनीयमित्येतत् सर्व्वमाशङ्क्य ‚{\tiny $_{lb}$}‚ ‚{\color{DodgerBlue3}‚पक्षिल ‚{\tiny $_{6}$}‚} स्वामी ब्रूते । एतच्च पर्यनुयोज्योपेक्षणं वक्तव्यञ्चोदनीयङ्कस्य पराजय ‚{\tiny $_{lb}$}‚इत्येवं वादिप्रतिवादिभ्यां प्रगुणा तदन्यैर्वा पर्यनुयुक्तया पृष्टया सत्या परिषदा ‚{\tiny $_{7}$}‚ ‚{\tiny $_{lb}$}‚प्राश्निकैर्वक्तव्यमित्यर्थः । च शब्दोऽवधारणार्थः । एतदेव\edtext{}{\lemma{एतदेव}\Bfootnote{? एवमेव}}अन्यानि निग्रह‚{\tiny $_{lb}$}‚स्थानानि वादिप्र\add{ति}वादिभ्यामेवोद्भाव्यन्ते । एतत्पुनः प्राश्निकैरेव । किं पुनः कार‚{\tiny $_{lb}$}‚ ‚{\tiny $_{8}$}‚ \leavevmode\ledsidenote{\textenglish{80a/msK}} णं ताभ्यामेव नोच्यत इत्याह । न ‚{\color{DodgerBlue3}‚खलु निग्रहप्राप्तः स्वकौपीनं} स्वदोषं ‚{\color{DodgerBlue3}‚विवृणुयात्} ‚{\tiny $_{lb}$}‚ ‚{\tiny $_{18b10}$}‚ प्रकाशयेत् । अत्रापीत्याद्याचार्यः । यदि तु न्यायश्चिन्त्यते तदानैकस्यापि ‚{\tiny $_{1}$}‚ ‚{\tiny $_{lb}$}‚जयपराजयौ न्याय्यौ । कथं वादिनो जय इत्याह साधनाभासेन जिज्ञासितस्यार्थस्या‚{\tiny $_{lb}$}‚प्रतिपादनात् । अत एव न प्रतिवादिनोपि पराजयो वादिविवक्षि ‚{\tiny $_{2}$}‚ तार्थसिद्ध्यपेक्षया ‚{\tiny $_{lb}$}‚प्रतिवादिनः पराजयव्यवस्थापनात् । प्रतिवादिनस्तर्हि किं जय इत्याह \add{।} भूते ‚{\tiny $_{lb}$}‚ ‚{\color{DodgerBlue3}‚दोषानभिधानाच्च} ‚{\tiny $_{19a1}$}‚ । अतएव च न वादिनः पराजयस्त ‚{\tiny $_{3}$}‚ द्दूषणापेक्षया ‚{\tiny $_{lb}$}‚तद्व्यवस्थितेः । अथोत्तरपक्षवाद्यनेकदोषसद्भावेपि वादिप्रोक्तस्य साधनस्य ‚{\tiny $_{lb}$}‚कञ्चिद्दोषमुद्भावयति कञ्चिन्न । न तदासौ ‚{\tiny $_{4}$}‚ निग्रहमर्हति । किङ्कारणमुत्तरस्य ‚{\tiny $_{lb}$}‚प्रतिपत्तेरभिधानादित्यर्थः । पर आह । अर्हत्येव निग्रहं सर्व्वेषान्दोषाणामनु‚{\tiny $_{lb}$}‚द्भावनात् । आचार्य आह । ‚{\color{DodgerBlue3}‚न खलु ‚{\tiny $_{5}$}‚ भोः सन्त इति कृत्वा सर्व्वे दोषा अवश्यं ‚{\tiny $_{lb}$}‚वक्तव्याः प्रतिवादिना} । अवचने वा दोषान्तरस्य निग्रहो भवति नेति वर्त्तते । ‚{\tiny $_{lb}$}‚कस्मात् सर्व्वे दोषा नोद्भाव्यं ‚{\tiny $_{6}$}‚ त इत्याह । ‚{\color{DodgerBlue3}‚एकेनापि} ‚{\tiny $_{19a2}$}‚ दोषेणासिद्ध‚{\tiny $_{lb}$}‚त्वादिनोद्भावितेन न तस्य वादिप्रयुक्तस्य साधनस्य विघातात् । साध्यसिद्धिं ‚{\tiny $_{lb}$}‚प्रत्यसमर्थत्वप्रतिपादनादित्यर्थः । भाव ‚{\tiny $_{7}$}‚ साधनो वा साधनशब्दः । अत्रैव दृष्टान्त‚{\tiny $_{lb}$}‚ \leavevmode\ledsidenote{\textenglish{130/s}} माह । ‚{\color{DodgerBlue3}‚एकसाधनवचनवदिति} ‚{\tiny $_{19a3}$}‚ । यथेत्याद्यस्यैव विभागः । एकस्यार्थस्य ‚{\tiny $_{lb}$}‚क्षणिकत्वादेः प्रतिपादनायानेकस्य ‚{\tiny $_{8}$}‚ साधनस्य सत्वकार्यत्वप्रयत्नोत्थत्वादेः सद्भावेपि ‚{\tiny $_{lb}$}‚सत्येकेनैव सत्वादीनामन्यतमेनोपात्तेन तस्य क्षणिकत्वादेरर्थस्य सिद्धेर्निश्चयान्न ‚{\tiny $_{lb}$}‚सर्व्वेषां साधना ‚{\tiny $_{9}$}‚ \leavevmode\ledsidenote{\textenglish{81a/msK}} नामुपादानं । तथैकेनापि दोषेण तत्साधनविघातान्न सर्व्वोपादान‚{\tiny $_{lb}$}‚मितीदन्दृष्टान्तेन साम्यं । इति तस्मान्नोत्तरपक्षवादी पूर्व्वमेकं दोषमुद्भावयन्नेवापर ‚{\tiny $_{1}$}‚ ‚{\tiny $_{lb}$}‚स्य दोषान्तरस्यानुद्भावनान्निग्रहार्हः । पूर्व्ववदिति साधनाभासेनाप्रतिपादनात् । ‚{\tiny $_{lb}$}‚भूतदोषानभिधानाच्च ।
	{\color{gray}{\rmlatinfont\textsuperscript{§~\theparCount}}}
	\pend% ending standard par
      ‚{\tiny $_{lb}$}‚

	  
	  \pstart \leavevmode% starting standard par
	ननु च कथन्न वादिनो जयो यावता न तेन साध ‚{\tiny $_{2}$}‚ नाभासः प्रयुक्तः । प्रतिवादी ‚{\tiny $_{lb}$}‚त्वसन्तं दोषमुद्भावयतीत्यत आह । ‚{\color{DodgerBlue3}‚दोषाभासं} ब्रुवाणमुत्तरपक्षवादिनं ‚{\tiny $_{18a4}$}‚ ‚{\tiny $_{lb}$}‚स्वसाधनात्सकाशादनुसारयतोऽनिवर्त्तयतस्त ‚{\tiny $_{3}$}‚ दुक्तदूषणाभासत्वेनाप्रतिपादयत इति ‚{\tiny $_{lb}$}‚यावत् । वादिनो न जयः कस्मादसमन्वितसाधनाङ्गत्वात् । असमन्वितसाधनाङ्गं ‚{\tiny $_{lb}$}‚येन तस्य भावस्त ‚{\tiny $_{4}$}‚ त्वं । एतदेव कुत इत्याह \add{।} ‚{\color{DodgerBlue3}‚सर्व्वदोषाभाव} प्रदर्शनेन साधनाङ्गसम‚{\tiny $_{lb}$}‚र्थनात् ‚{\tiny $_{19a5}$}‚ । इत्त्थम्भूतलक्षणे करणे वा तृतीया \href{http://sarit.indology.info/?cref=P\%C4\%81.2.3.29}{पाणिनि २।३।२९ } ‚{\tiny $_{lb}$}‚असमर्थितत्त्वात् साधना ‚{\tiny $_{5}$}‚ भास एव तेन प्रयुक्त इति संक्षेपार्थः । नाप्युत्तरपक्षवादिनो ‚{\tiny $_{lb}$}‚जय इति वर्त्तते । तस्मादेवमपीति यदि पूर्वपक्षवाद्युत्तरपक्षवादिनं ‚{\tiny $_{6}$}‚ न निग्रहप्राप्तं ‚{\tiny $_{lb}$}‚निगृह्णाति न केवलमुत्तरवादिसम्बन्धेनेत्यपि शब्दः ॥ ० ॥
	{\color{gray}{\rmlatinfont\textsuperscript{§~\theparCount}}}
	\pend% ending standard par
      ‚{\tiny $_{lb}$}‚

	  
	  \pstart \leavevmode% starting standard par
	\hphantom{.}‚{\color{DodgerBlue3}‚निरनुयोज्यस्यानुयोगः} ‚{\tiny $_{19a6}$}‚ । अनिग्रहप्राप्ते निगृहीतोसीत्यभिधानं । किं ‚{\tiny $_{7}$}‚ ‚{\tiny $_{lb}$}‚पुनरेवं ब्रूत इत्याह \add{।} ‚{\color{DodgerBlue3}‚निग्रहस्थानलक्षणस्य} मिथ्याव्यवसायाद्य ‚{\tiny $_{19a6}$}‚ थोक्तस्य ‚{\tiny $_{lb}$}‚निग्रहस्थानलक्षणस्य सम्यगपरिज्ञानादित्यर्थः । एवञ्चाप्रतिपत्तितो निगृ ‚{\tiny $_{8}$}‚ ह्यते । ‚{\tiny $_{lb}$}‚ \leavevmode\ledsidenote{\textenglish{131/s}} अत्रापीत्याद्याचार्यः । यदि तस्य साधनस्य वादिनमभूतैरलीकैर्दोषैः सव्यभिचारा‚{\tiny $_{lb}$}‚दिदोषदुष्टं त्वया साधनं प्रयुक्तं ततो निगृहीतोसीत्येवम ‚{\tiny $_{9}$}‚ भियुञ्जीत । तदा सोऽस्था‚{\tiny $_{lb}$}‚नेऽस्य व्याख्यानं निर्दोषनिग्रहस्थानस्य अस्य विभागादेवास्येति । अभियोक्तेत्यस्य ‚{\tiny $_{lb}$}‚विवृतिरुद्भावयितेति । तथा चालीकदोष ‚{\tiny $_{1}$}‚ स्याभिधायित्वे सति दोषोद्भावलक्षण‚{\tiny $_{lb}$}‚स्योत्तरस्याप्रतिपत्तेरभिधानादप्रतिभयैव करणभूतयोत्तरवादी निगृहीत इति कृत्वा ‚{\tiny $_{lb}$}‚नेदन्निरनुयो ‚{\tiny $_{2}$}‚ ज्यानुयोगाभिधानन्निग्रहस्थानमतोऽप्रतिभानिग्रहस्थानात्सकाशान्न नि‚{\tiny $_{lb}$}‚ग्रहस्थानान्तरं । कदा चायमप्रतिभया निगृह्यत इत्याह \add{।} ‚{\color{DodgerBlue3}‚इतरेण} ‚{\tiny $_{19a8}$}‚ ‚{\tiny $_{lb}$}‚वादि ‚{\tiny $_{3}$}‚ ना तदुक्तस्योत्तराभासत्वे प्रतिपादिते अन्यथा न द्वयोरेकस्यापि पूर्ववज्जयपरा‚{\tiny $_{lb}$}‚जयावित्याकूतं । एवं प्रतिवादिसम्बन्धेनास्यापृथग्वच ‚{\tiny $_{4}$}‚ नं प्रतिपाद्य वादिसम्बन्धेना‚{\tiny $_{lb}$}‚प्याह \add{।} अथोत्तरवादिनं भूतं सत्यं साधनदोषं सव्यभिचा\add{रा}दिकमुद्भावयन्त‚{\tiny $_{lb}$}‚मपर इति पूर्वपक्षवादी दोषाभा ‚{\tiny $_{5}$}‚ सवचनेनाभियुञ्जीत । जात्युत्तरमनैकान्तिकाद्या‚{\tiny $_{lb}$}‚भासं त्वया प्रयुक्तं । तस्मान्निगृ\add{ही}तोसीत्येवं यद्यभियुञ्जीतेत्यर्थः । तदा तस्योद्‚{\tiny $_{lb}$}‚भावितस्य दो ‚{\tiny $_{6}$}‚ षस्य व्यभिचारादेस्तेनोत्तरवादिना भूतदोषत्वे प्रतिपादिते जात्यु‚{\tiny $_{lb}$}‚ \add{त्त}रवत्वे परिहृत इति यावत् । साधनाभासवचनेनैव वादी निगृह्यते इति ॥ ‚{\tiny $_{lb}$}‚तस्मादेवमपि प्र ‚{\tiny $_{7}$}‚ तिवादिसम्बन्धेनापि नेदं हेत्वाभासेभ्यो भिद्यत इति पृथग्वाच्यं । ‚{\tiny $_{lb}$}‚अस्यैवोपोद्वलनमवश्यं हि द्वाविंशतिनिग्रहस्थानवादिना हेत्वाभासाः पृथग् निग्रह‚{\tiny $_{lb}$}‚स्था ‚{\tiny $_{8}$}‚ नत्वेन वक्तव्याः । किमर्थमित्याह । विषयान्तरप्राप्त्यर्थं ‚{\tiny $_{19b1}$}‚ निरनुयो‚{\tiny $_{lb}$}‚ज्यानुयोगादिभिर्न्निग्रहस्थानैरनाक्रान्तसङ्ग्रहमपीति अन्यथा द्वाविंशतित्वं निग्र‚{\tiny $_{lb}$}‚हस्था ‚{\tiny $_{9}$}‚ \leavevmode\ledsidenote{\textenglish{81b/msK}} नानामभ्युपगमम्विरुद्ध्यत इत्यभिप्रायः । तथा च तदुक्तौ तेषां हेत्वाभासानां ‚{\tiny $_{lb}$}‚निग्रहस्थानेनोक्तौ सत्यामपरोक्तिः । अपरस्य निरनुयोज्यानुयोगस्योक्तिर्निर\add{ि} ‚{\tiny $_{lb}$}‚र्थका ‚{\tiny $_{1}$}‚ हेत्वाभासवचनेनैव संगृहीतत्वात् ॥ ० ॥
	{\color{gray}{\rmlatinfont\textsuperscript{§~\theparCount}}}
	\pend% ending standard par
      ‚{\tiny $_{lb}$}‚\textsuperscript{\textenglish{132/s}}

	  
	  \pstart \leavevmode% starting standard par
	\hphantom{.}‚{\color{DodgerBlue3}‚सिद्धान्तमभ्युपेत्यानियमात् कथाप्रसङ्गोऽपसिद्धान्त} इति ‚{\tiny $_{19b1}$}‚ सूत्रं ‚{\tiny $_{lb}$}‚सिद्धान्तमभ्युपेत्य पक्षपरिग्रहं कृत्वाऽनियमात् पूर्वप्रकृतार्थोपरोधेन शास्त्रव्यवस्था‚{\tiny $_{lb}$}‚मनादृत्येति यावत् । कथाप्रसङ्गोऽर्थान्तरोपवर्णनं । कस्यचिदर्थस्येति धर्मिणो धर्मा‚{\tiny $_{lb}$}‚न्तरं प्रतिज्ञाय प्रतिज्ञातार्थ ‚{\tiny $_{3}$}‚ विपर्ययो विरोधः । इदं उदाहरणेन स्पष्टयति । ‚{\color{DodgerBlue3}‚यथा न ‚{\tiny $_{lb}$}‚सतो वस्तुनो विनाशो} ‚{\tiny $_{19b2}$}‚ निरन्वयः केवलं तिरोभावमात्रं भवति नासत् ‚{\tiny $_{lb}$}‚खरविषाणतुल्यमु ‚{\tiny $_{4}$}‚ त्पद्यते । किन्तर्ह्याविर्भावतः । सदेवोत्पद्यत इत्येवं ‚{\color{DodgerBlue3}‚कापिलः} सिद्धा‚{\tiny $_{lb}$}‚न्तव्यवस्थामादर्श्य पक्षङ्करोति । एका प्रकृतिर्व्यक्तस्याव्यक्तलक्षणा । व्यक्तस्येति ‚{\tiny $_{lb}$}‚म ‚{\tiny $_{5}$}‚ हदादेः । अत्र हेतुमाह विकाराणां शब्दादीनामन्वयदर्शनात् । मृदन्वयानामि‚{\tiny $_{lb}$}‚त्यादिदृष्टान्तः । तथा चायमित्युपनयनः\edtext{}{\lemma{चायमित्युपनयनः}\Bfootnote{? पनयः}}। ‚{\color{DodgerBlue3}‚सुखदुःखमोहसमन्वित} ‚{\tiny $_{19b3}$}‚ ‚{\tiny $_{lb}$}‚इति ‚{\tiny $_{6}$}‚ सुखादिमयत्वं दर्शयति । दर्शितञ्च सुखादिमयत्वं व्यक्तस्य पूर्वं यथासांख्येना‚{\tiny $_{lb}$}‚भिमतं । तत्तस्मात् सुखादिभिरेकप्रकृतिरित्ययं व्यक्तभेदः । इति निगमनं ‚{\tiny $_{7}$}‚ सुखादि‚{\tiny $_{lb}$}‚भिरितीत्त्थंभूतलक्षणे तृतीया । सुखादिप्रकारा सुखादिलक्षणा । एका प्रकृतिरस्ये‚{\tiny $_{lb}$}‚त्यर्थः । अन्ये पठन्ति । एका ‚{\color{DodgerBlue3}‚प्रकृतिर्व्यक्ता} व्यक्तविकाराणामन्वयद ‚{\tiny $_{8}$}‚ \leavevmode\ledsidenote{\textenglish{82a/msK}} र्शनादिति । ‚{\tiny $_{lb}$}‚एवञ्च व्याचक्षते । एका प्रकृतिरभिन्ना सर्व्वात्मस्वभावा व्यक्ताव्यक्तविकारा‚{\tiny $_{lb}$}‚णामन्वयदर्शनात् । ये व्यक्ता विकारा महदादयो ये चाव्यक्ताः प्रधानात्मनि व्यव‚{\tiny $_{lb}$}‚स्थितास्तेषा ‚{\tiny $_{1}$}‚ मप्यन्वयदर्शनादिति । अपरे तु पठन्ति । ‚{\color{DodgerBlue3}‚एका प्रकृतिरव्यक्ता} । व्यक्त‚{\tiny $_{lb}$}‚विकाराणामिति व्यक्तरूपाणां विकाराणामिति चाहुः । प्रकृतार्थविपर्य\add{ये}णेयं ‚{\tiny $_{lb}$}‚यथा प्रवृत्तेति प्रदर्शनाऽ ‚{\tiny $_{2}$}‚ र्थमाह \add{।} ‚{\color{DodgerBlue3}‚स कापिल एवमुक्तवान्पर्यनुयुज्यते} ‚{\tiny $_{lb}$}‚ ‚{\tiny $_{19b4}$}‚ । अथ प्रकृतिर्विकार इत्येतदुभयङ्कथं लक्षयितव्यं । प्रतिपत्तव्यमिति । ‚{\tiny $_{lb}$}‚स एवमनुयुक्तः प्राह । यस्यावस्थितस्य धर्मान्तरनि ‚{\tiny $_{3}$}‚ वृत्तौ धर्मान्तरम्प्रवर्त्तते सा ‚{\tiny $_{lb}$}‚प्रकृतिरवस्थितरूपा । यत्तत्प्रवृत्तिनिवृत्तिसद्धर्मान्तरं स विकार इति लक्षयितव्यं । ‚{\tiny $_{lb}$}‚परमुक्तवान् ‚{\color{DodgerBlue3}‚साङ्ख्यः} प्रकृतार्थपरित्यागदो ‚{\tiny $_{4}$}‚ षेणोपपाद्यते । सोयम्वादी प्रकृतार्थ‚{\tiny $_{lb}$}‚विपर्ययादनियमात् कथाम्प्रसञ्जयति । पूर्व्वप्रकृतं परित्यजतीत्यर्थः । कथमित्याह ‚{\tiny $_{lb}$}‚ \leavevmode\ledsidenote{\textenglish{133/s}} \add{।} प्रतिज्ञातं खल्वनेनेति ‚{\tiny $_{19b5}$}‚ पूर्वोक्तं स्मर ‚{\tiny $_{5}$}‚ यति । यद्येवङ्को दोष इत्याह । ‚{\tiny $_{lb}$}‚ ‚{\color{DodgerBlue3}‚सदसतो} रित्यादि । सतस्तिरोभावमेकान्तेन विनाशमन्तरेण न कस्यचिद्धर्मस्य ‚{\tiny $_{lb}$}‚प्रवृत्युपरमः सिध्यति । केन चिद्विरूपे ‚{\tiny $_{6}$}‚ णावस्थाने सति स तिरोहितोऽङ्गस्तस्या‚{\tiny $_{lb}$}‚वस्थितस्यात्मभूतः परभूतो वा भवेत् । आत्मभूतत्वे तिरोहितादव्यतिरेकात् ‚{\tiny $_{lb}$}‚तिरोहितवदवस्थितस्याप्यनवस्थानं ‚{\tiny $_{7}$}‚ अवस्थितवच्च तदव्यतिरेकतस्तिरोहितस्या‚{\tiny $_{lb}$}‚प्यवस्थानमासं\edtext{}{\lemma{प्यवस्थानमासं}\Bfootnote{? शं}}क्यते । परभूतत्वेपि कथमनन्वयो न विनाशो न ह्यन्यस्या‚{\tiny $_{lb}$}‚वस्थानेऽन्यदवतिष्ठते । अन्यो वान्यस्यान्वयश्चैतन्यस्याऽ ‚{\tiny $_{8}$}‚ पि घटान्वयप्रसङ्गात् । तथा ‚{\tiny $_{lb}$}‚नासत आविर्भावमुत्पादमन्तरेण कस्यचिद्धर्मस्य प्रवृत्तिर्वा सिध्यति ।
	{\color{gray}{\rmlatinfont\textsuperscript{§~\theparCount}}}
	\pend% ending standard par
      ‚{\tiny $_{lb}$}‚

	  
	  \pstart \leavevmode% starting standard par
	ननु च विद्यमानमेव धर्मान्तरमाविर्भाव्यते । ग्रहणविषयभा ‚{\tiny $_{9}$}‚ \leavevmode\ledsidenote{\textenglish{82b/msK}} वमापाद्यते । ‚{\tiny $_{lb}$}‚न विद्यमानस्य क्रियास्त्युपादानमिति उपलब्धिर्वा विद्यमानत्वात् । न कारक‚{\tiny $_{lb}$}‚जन्यत्वमित्येवं प्रत्यवस्थितः प्रतिषिद्धः ‚{\color{DodgerBlue3}‚साङ्ख्यः} । क्वचित् सप्तम्यापद्यते । तत्र ‚{\tiny $_{lb}$}‚प्रत्य ‚{\tiny $_{1}$}‚ वस्थिते प्रतिवादिनि सतीति व्याख्येयं । यदि स ‚{\color{DodgerBlue3}‚कापिलः} सतो धर्मस्यात्म‚{\tiny $_{lb}$}‚हानमसतश्चात्मलाभमभ्युपैति तदानीमपसिद्धान्तो भवति । अभ्युपगमविरुद्धस्य ‚{\tiny $_{lb}$}‚प्रतिज्ञा ‚{\tiny $_{2}$}‚ नादपसिद्धान्तसंज्ञकं निग्रहस्थानमस्य भवतीत्यर्थः । अथ सत आत्म‚{\tiny $_{lb}$}‚हानमसतश्चात्मलाभन्नाभ्युपैति । एवमप्येकप्रकृतिर्विकाराणामिति योयं पक्षः ‚{\tiny $_{lb}$}‚पूर्व्वप्रतिज्ञातः सोस्य ‚{\tiny $_{3}$}‚ न सिध्यति प्रकृतिविकारलक्षणस्यानवस्थितत्वात् । तथा हि ‚{\tiny $_{lb}$}‚तयोर्लक्षणं यस्यावस्थितस्येत्यादिनोक्तं । तस्य चायोगः । सदसतोश्चे ‚{\tiny $_{19b6}$}‚ ‚{\tiny $_{lb}$}‚त्यादिना प्रतिपा\add{ि}दत इत्येतावा ‚{\tiny $_{4}$}‚ न्परग्रन्थः । अत्र सम्प्रत्याचार्यः प्रतिविधत्ते । ‚{\tiny $_{lb}$}‚इतोपि प्रतिविदध्मह इति शेषः । न कश्चिदनियमात् सिद्धान्तनीतिविरोधात् ‚{\tiny $_{lb}$}‚ ‚{\color{DodgerBlue3}‚साङ्ख्य} स्य प्रसङ्गः । तस्माद्यत्तेनोपग ‚{\tiny $_{5}$}‚ तं नासदुत्पद्यते न स\add{त्}तिरोभवतीति ‚{\tiny $_{lb}$}‚तस्य समर्थनायेदमुक्तं । किमुक्तमित्याह \add{।} ‚{\color{DodgerBlue3}‚एकप्रकृतिकमिदं व्यक्तमन्वय‚{\tiny $_{lb}$}‚दर्शनादिति} ‚{\tiny $_{19b8}$}‚ । तत्रैकेत्येतदेव विभजति । तदवि ‚{\tiny $_{6}$}‚ भक्तयोनिकमिदं ‚{\tiny $_{lb}$}‚ \leavevmode\ledsidenote{\textenglish{134/s}} व्यक्तं । ते सुखादयोऽविभक्ताः अपृथग्भूता योनिः स्थानमधिकरणं यस्य ‚{\tiny $_{lb}$}‚व्यवतस्य तत्तदविभक्तयोनिकं किङ्कारणं तदन्वयदर्शनात् । तै\add{ः} सुखादिभि‚{\tiny $_{lb}$}‚ ‚{\tiny $_{7}$}‚ रन्वयदर्शनात् तादात्म्योपलम्भात् । ततः किं सिद्धमित्याह व्यक्तस्य तत्स्व‚{\tiny $_{lb}$}‚भावता सुखादिस्वभावता । तत्र स्वभावतैव कथन्निश्चितेत्याह । ‚{\color{DodgerBlue3}‚अभेदोपलब्धे} ‚{\tiny $_{lb}$}‚ ‚{\tiny $_{19b8}$}‚ रिति । सुखादिभिः ‚{\tiny $_{8}$}‚ शब्दादीनामनानात्त्वदर्शनादिति यावत् । एवमपि ‚{\tiny $_{lb}$}‚किं सिद्धम्भवतीत्याह । ‚{\color{DodgerBlue3}‚सर्व्वस्य} ‚{\tiny $_{19b9}$}‚ शब्दादेर्विकारग्रामस्य सुखाद्या‚{\tiny $_{lb}$}‚त्मकस्य नोत्पत्तिविनाशाविति सम्भवति । कस्मादि ‚{\tiny $_{9}$}‚ \leavevmode\ledsidenote{\textenglish{83a/msK}} त्याह \add{।} ‚{\color{DodgerBlue3}‚सुखादीनामुत्पत्ति} ‚{\tiny $_{lb}$}‚विनाशाभावात् । सुखाद्यव्यतिरेकात्तदात्मवच्छब्दादयोपि नित्याः सिद्धा भवन्ति । ‚{\tiny $_{lb}$}‚तथा च यत्पूर्व्वमभ्युपगतं नश\edtext{}{\lemma{नश}\Bfootnote{? स}}तो विनाशो नासदुत्पद्यत इति । तत्समर्थि ‚{\tiny $_{1}$}‚ तं ‚{\tiny $_{lb}$}‚भवति । अत्रैवं ‚{\color{DodgerBlue3}‚कापिलेन} स्वोपगमे समर्थिते सति तदुक्तस्य तेन साङ्ख्येनोक्तस्य ‚{\tiny $_{lb}$}‚हेतोरन्वयदर्शनस्य दोषमसिद्धतादिकमनुद्भाव्य स एवमुक्तवान् पर्यनुमुज्यते । ‚{\tiny $_{lb}$}‚अथ प्र ‚{\tiny $_{2}$}‚ कृतिर्निर्विकार इति कथं लक्षयितव्यमित्येवं विकारप्रकृत्योर्लक्षणं पृच्छन् ‚{\tiny $_{lb}$}‚स्वयमयमक्षपादः प्रकृतासम्बन्धेनानियमात् प्रकृतार्थोपरोधात् कथाम्प्रवर्त्त ‚{\tiny $_{3}$}‚ यति । ‚{\tiny $_{lb}$}‚यस्मात् प्रकृतिविकारयोरिह लक्षणं न प्रकृतमेव तत्किन्तदभिधाय पर्यनुयुज्यते ‚{\tiny $_{lb}$}‚तस्मादप्रस्तुतपर्यनुयोक्तृत्त्वादक्षपाद एव निग्रहार्ह इति ‚{\tiny $_{4}$}‚ भावः । किन्तर्ह्यत्रोत्तरसम्बद्ध ‚{\tiny $_{lb}$}‚वाक्यमित्याह \add{।} तत्रान्वयदर्शनहेताविदं स्याद् वाच्यं ‚{\tiny $_{20a1}$}‚ । व्यक्तन्नाम प्रवृत्ति‚{\tiny $_{lb}$}‚निवृत्तिधर्मकं न तथा व्यक्तवत् सुखादयः प्रवृत्तिनि ‚{\tiny $_{5}$}‚ वृत्तिधर्मका इति लिङ्गवचन‚{\tiny $_{lb}$}‚परिणामेन सम्बन्धः । तथा च व्यक्तस्य सुखाद्यन्वयेऽस्य व्याख्यानं सुखादिस्वभाव‚{\tiny $_{lb}$}‚तायां सत्यां प्रवृत्तिनिवृत्तिधर्मतालक्ष ‚{\tiny $_{6}$}‚ णं व्यक्तस्यावहीयते । तदव्यतिरेकेण तेषा‚{\tiny $_{lb}$}‚मपि सदावस्थानात् । इति तस्मान्न तद्रहितसुखादिस्वभावता व्यक्तस्य । ताभ्या‚{\tiny $_{lb}$}‚म्प्रवृत्तिनिवृत्तिभ्यां रहितास्तथोक्तास्ते च सु ‚{\tiny $_{7}$}‚ खादयश्च । ते स्वभावो यस्य तस्य ‚{\tiny $_{lb}$}‚भावस्तद्रहितसुखादिस्वभावता । कस्मादित्याह \add{।} व्यक्तलक्षणविरोधादि‚{\tiny $_{lb}$}‚‚{\tiny $_{lb}$}‚\leavevmode\ledsidenote{\textenglish{135/s}}ति व्यक्तस्य लक्षणं प्रवृत्तिनिवृत्तिधर्मकत्वं तस्य तद्विपरीतः । ‚{\tiny $_{8}$}‚ सुखादिभिः परस्पर‚{\tiny $_{lb}$}‚परिहारस्थितिलक्षणो विरोधः । तथा च साधनन्न सदावस्थितरूपसुखादिस्वभाव‚{\tiny $_{lb}$}‚मिदं ते व्यक्तं प्राप्नोति तद्विपरीतधर्मत्वात् । क्षेत्रज्ञवत् ‚{\tiny $_{9}$}‚ \leavevmode\ledsidenote{\textenglish{83b/msK}} न च सुखादिव्यक्तयोरेक- ‚{\tiny $_{lb}$}‚स्वभावता । परस्परविरुद्धधर्माध्यासितत्वात् । सत्त्वरजस्तमसामिव चैतन्यानामिव ‚{\tiny $_{lb}$}‚वा । एवञ्च व्यक्तस्य सुखादिस्वभावतायोगे सुखाद्यन्वयदर्शन ‚{\tiny $_{1}$}‚ सिद्धो हेतुः । कस्मा‚{\tiny $_{lb}$}‚दिदं सम्बद्धं दूषणमित्याह । ‚{\color{DodgerBlue3}‚एवं हि तस्य साङ्ख्यस्य साधनदोषोद्भावनेन} हेत्वसि‚{\tiny $_{lb}$}‚द्धताचोदनेनैकप्रकृतीदं व्यक्तमित्ययं ‚{\color{DodgerBlue3}‚पक्षो दूषितो भवनि} ‚{\tiny $_{20a3}$}‚ । स पु ‚{\tiny $_{2}$}‚ नर्नैयायिकः ‚{\tiny $_{lb}$}‚साधने दोषमसिद्धताख्यमनुपसंहृत्यापदर्श्य अप्रकृतप्रकृतिविकारलक्षणपर्यनुयोगेन ‚{\tiny $_{lb}$}‚कथां प्रतानयत्यविमुञ्चं स्वदोषमन्यमात्कथा ‚{\tiny $_{3}$}‚ प्रसङ्गं परत्र ‚{\color{DodgerBlue3}‚साङ्ख्ये} तपस्विन्यु‚{\tiny $_{lb}$}‚पक्षिपति । पर आहायमेवासिद्धताख्यो दोषोनेन प्रकारेण प्रकृतिविकारलक्षण‚{\tiny $_{lb}$}‚पर्यनुयोगद्वारेणास्माभिरप्युच्य ‚{\tiny $_{4}$}‚ त इति । आचार्य आह । ‚{\color{DodgerBlue3}‚एष नैमित्तिकानां} ‚{\tiny $_{20a3}$}‚ ‚{\tiny $_{lb}$}‚ज्योतिर्ज्ञानविदां विषयः । नायं त्वदुक्तस्य वाक्यस्यार्थ इति यावत् । यतो न लोकः ‚{\tiny $_{lb}$}‚शब्दैरप्रतिपादितमर्थं प्रति ‚{\tiny $_{5}$}‚ पत्तुं समर्थः अर्थप्रकरणादिभिर्विनेत्यध्याहारः । तस्मात् स ‚{\tiny $_{lb}$}‚एवायं ‚{\tiny $_{20a4}$}‚ प्रतिज्ञाविरोधप्रस्तावे निर्द्दिष्टो ‚{\color{DodgerBlue3}‚भण्डालेख्यन्यायोत्राप्यपसिद्धान्तो} ‚{\tiny $_{lb}$}‚न केवलं तत्रेत्यऽ ‚{\tiny $_{6}$}‚ पि शब्दः । यथा हि ‚{\color{DodgerBlue3}‚भण्डाः} प्राकृतान् विस्मापयन्तः शीघूमर्द्धचन्द्रा‚{\tiny $_{lb}$}‚कारामल्पीयसीं रेखामालिख्य भणन्ति पश्यत तालमात्रेण हस्ती विलिखितोस्माभि‚{\tiny $_{lb}$}‚रिति तत्र केचि ‚{\tiny $_{7}$}‚ त् मन्दमतयस्तथैव प्रतिपद्यन्ते । केचिद् दुर्विदग्धधियः पर्यनुयुञ्जते । ‚{\tiny $_{lb}$}‚ननु नोस्य कर्ण्णपाददन्तादयः प्रतीयन्ते तत्कथमयन्तद्विकलो हस्ती भवतीति । ते पु ‚{\tiny $_{lb}$}‚नराहु ‚{\tiny $_{8}$}‚ ः । सत्यं न प्रतीयन्ते । अस्माभिस्तु समाप्तसकलकलः करेणुरयं लिखितः । ‚{\tiny $_{lb}$}‚तास्तस्य सकलाः कलाः सलिल इव मग्नत्वान्नोपलभ\edtext{}{\lemma{मग्नत्वान्नोपलभ}\Bfootnote{? भ्य}}न्ते कुम्भकदेशमात्र‚{\tiny $_{lb}$}‚न्त्विदमस्योप ‚{\tiny $_{9}$}‚ \leavevmode\ledsidenote{\textenglish{84a/msK}} लभ्यत इति तथाजातीयकमेतत् परस्यापि धार्ष्ट्यविजृम्भितं । यदि ‚{\tiny $_{lb}$}‚नाम नायमर्थोस्माद् बाह्यात् प्रतीयते तथाप्यनेन प्रकारेणोच्यत इति । अपि चोच्य‚{\tiny $_{lb}$}‚ताम ‚{\tiny $_{1}$}‚ यमेवार्थोनेन प्रकारेण तथाप्यसिद्धस्य हेत्वाभासेष्वन्तर्भावात् तद्वचनेनैवा‚{\tiny $_{lb}$}‚भिधानमिति नापसिद्धान्तः पथगुपादेयो भवेदित्येतदुपसंहारव्याजेनाह । ‚{\color{DodgerBlue3}‚यथो ‚{\tiny $_{2}$}‚ क्तेन} ‚{\tiny $_{lb}$}‚न्यायेने ‚{\tiny $_{20a4}$}‚ त्यादि ॥ ० ॥
	{\color{gray}{\rmlatinfont\textsuperscript{§~\theparCount}}}
	\pend% ending standard par
      ‚{\tiny $_{lb}$}‚\textsuperscript{\textenglish{136/s}}

	  
	  \pstart \leavevmode% starting standard par
	\hphantom{.}‚{\color{DodgerBlue3}‚हेत्वाभासाश्च यथोक्ता} इति सूत्रं । इदमाक्षेपपूर्वकं ‚{\color{DodgerBlue3}‚वात्स्यायनो} व्याचष्टे । ‚{\color{DodgerBlue3}‚किं ‚{\tiny $_{lb}$}‚पुनरिति} ‚{\tiny $_{20a6}$}‚ हेत्वाभासलक्षणाद्यदन्यल्लक्षणं तेन सम्बन्धा ‚{\tiny $_{3}$}‚ न्निग्रहस्थानत्वमा‚{\tiny $_{lb}$}‚पद्यन्ते । किमिवेत्याह । ‚{\color{DodgerBlue3}‚यथा प्रमाणानि प्रमेयत्त्वं लक्षणान्तरवसा\edtext{}{\lemma{लक्षणान्तरवसा}\Bfootnote{? वसा}}दापद्यन्त} ‚{\tiny $_{lb}$}‚ ‚{\tiny $_{20a6}$}‚ इति वर्त्तते तानि हिप्रमिति क्रियाया\add{ः} कारणत्वात् । प्रमा ‚{\tiny $_{4}$}‚ णानि प्रमाणा‚{\tiny $_{lb}$}‚न्तरेण तु यदा प्रमीयन्ते तदा कर्मत्वात् प्रमेयानि । तत एव पदार्थत्वात् प्राप्तः संशयः । ‚{\tiny $_{lb}$}‚अत्राह मुनिना यथोक्त इति । अस्यैव विवरणं । ‚{\tiny $_{5}$}‚ ‚{\color{DodgerBlue3}‚यथोक्तहेत्वाभासलक्षणेनैव निग्रह‚{\tiny $_{lb}$}‚स्थान} भाव इति । इदमुक्तम्भवति । सव्यभिचारविरुद्धप्रकरणस\add{म}साध्यसमातीत‚{\tiny $_{lb}$}‚काला\href{http://sarit.indology.info/?cref=ns\%C5\%AB.1.2.4}{न्या० सू० १।२।४ } इति हेत्वाभासा इति प्र ‚{\tiny $_{6}$}‚ भेदमुपक्रम्य यत्प्रत्येकं लक्षणमुक्तं । ‚{\tiny $_{lb}$}‚ अनैकान्तिकः सव्यभिचारः\href{http://sarit.indology.info/?cref=ns\%C5\%AB.1.2.5}{न्या० सू० १।२।५ } सिद्धान्तमभ्युपेत्य तद्विरोधाद्विरुद्धं ‚{\tiny $_{lb}$}‚ \href{http://sarit.indology.info/?cref=ns\%C5\%AB.1.2.6}{न्या० सू० १।२।६ } । यस्मात् प्रकरणचिन्ता स निर्ण्णयार्थमपदिष्टः प्रकरण ‚{\tiny $_{7}$}‚ समः ‚{\tiny $_{lb}$}‚ \href{http://sarit.indology.info/?cref=ns\%C5\%AB.1.2.7}{न्या० सू० १।२।७ } साध्याविशिष्टः साध्यत्वात् साध्यसमः \href{http://sarit.indology.info/?cref=ns\%C5\%AB.1.2.8}{न्या० सू० १।२।८ } ‚{\tiny $_{lb}$}‚ कालात्ययापदिष्टः कालातीत \href{http://sarit.indology.info/?cref=ns\%C5\%AB.1.2.9}{न्या० सू० १।२।९ } इति तेनैव लक्षणेनैषान्निग्रह‚{\tiny $_{lb}$}‚स्थानत्त्वं न पुनस्तत लक्षणान्तरमपेक्ष्यत इति । अत्रापी ‚{\tiny $_{8}$}‚ ‚{\tiny $_{20a7}$}‚ त्याचार्यः । ‚{\tiny $_{lb}$}‚कथञ्चिन्त्यमित्याह । ‚{\color{DodgerBlue3}‚किन्ते यथा भवद्भिर्लक्षितप्रभेदास्तथैव ते भवन्त्याहो ‚{\tiny $_{lb}$}‚स्विदन्यथे} ति ‚{\tiny $_{20a7}$}‚ । लक्षितः प्रभेदो येषामिति विग्रहः । तत्तर्हि किन्त\add{द्} ‚{\tiny $_{lb}$}‚ चिन्त्यत इत्याह ‚{\tiny $_{9}$}‚ \leavevmode\ledsidenote{\textenglish{84b/msK}} तत्तु चिन्त्यमानमिहातिप्रसज्यत इति न प्रतन्यते । इदमेवागूरितं । ‚{\tiny $_{lb}$}‚विदन्त्येव केचिदत्र हेत्वाभासा एव न युज्यंते केचित्तु हेत्वाभासा अपि न सङ्ग्रहीता ‚{\tiny $_{lb}$}‚इत्यस्मिंश्च विचारे हेत्वाभास ‚{\tiny $_{1}$}‚ वार्त्तिकं सकलमवतारयितव्यमिति शास्त्रान्तरमेव ‚{\tiny $_{lb}$}‚भवेत् । अवदातमतयस्त्वस्मद्विहितहेत्वाभासलक्षणविपर्ययेण दूरान्तरत्वात्तद् ‚{\tiny $_{lb}$}‚वैसशं\add{?} । तस्मादुपेक्षै ‚{\tiny $_{2}$}‚ व युज्यत इति । तथापि मन्दमतिविबोधनायापि शास्त्र‚{\tiny $_{lb}$}‚मुच्यत इति । कालातीतप्रकरणसमयोस्तावद्धेत्वाभासत्वं यथा नोपपद्यते तथा वर्ण्यते । ‚{\tiny $_{lb}$}‚तत्र ‚{\color{DodgerBlue3}‚कालात्ययापदि ‚{\tiny $_{3}$}‚ ष्टः कलातीतः} तदिह \add{।} बृद्धनैयायिकानामपास्य मत‚{\tiny $_{lb}$}‚माचार्य ‚{\color{DodgerBlue3}‚दिङ्नाग} पादैर्भाषितत्वादिदानीन्तना ‚{\color{DodgerBlue3}‚वात्स्यायना} दयोमुमेव स्थिपक्षमाहुः । ‚{\tiny $_{lb}$}‚तत्रैवम्ब्रूमः । ‚{\tiny $_{4}$}‚ कालात्ययेन युक्तो यस्यार्थैकदेशोऽपदिश्यमानस्य स कालात्ययापदिष्टः ‚{\tiny $_{lb}$}‚ \leavevmode\ledsidenote{\textenglish{137/s}} कालातीत इत्युच्यते । निदर्शनं \add{।} नित्यः शब्दः संयोगव्यंग्यत्वाद्रूपवत् । प्रागूद्र्ध्वं ‚{\tiny $_{5}$}‚ ‚{\tiny $_{lb}$}‚ च व्यक्तेरवस्थितं रूपं प्रदीपघटसंयोगेन व्यज्यते । तथा शब्दो व्यवस्थितो ‚{\tiny $_{lb}$}‚भेरीकर्ण्णसंयोगेन दारुपर्णयोगेन वा व्यज्यते । तस्मात्संयोगव्यंग्यत्वान्नित्यः ‚{\tiny $_{6}$}‚ शब्द ‚{\tiny $_{lb}$}‚इति । अयमहेतुः कालात्ययापदेशात् व्यंजकस्य संयोगस्य कालं न व्यंग्यस्य रूपस्य ‚{\tiny $_{lb}$}‚व्यक्तिरत्येति सति प्रदीपसंयोगे रूपस्य ग्रहणं भवति । न निवृत्तसंयोगे ‚{\tiny $_{7}$}‚ रूपङ्गृ‚{\tiny $_{lb}$}‚ह्यते । निवृत्ते तु दारुपर्णसंयोगे दूरस्थेन शब्दः श्रूयते विभागकाले नेयं शब्दस्य ‚{\tiny $_{lb}$}‚व्यक्तिः संयोगकालमत्येतीति संयोगनिमित्ता भवति । कारणाभावाद्धि कार्या ‚{\tiny $_{8}$}‚ भाव ‚{\tiny $_{lb}$}‚इति । नन्वयमनैकान्तिक एव । संयोगव्यंग्यत्वादिति । अनित्यमपि संयोगेन व्यज्य‚{\tiny $_{lb}$}‚मानं दृष्टं यथा घट इति । न । संयोग व्यंग्यत्वेनावस्थानस्य साध्यत्वान्न ब्रूमो ‚{\tiny $_{9}$}‚ \leavevmode\ledsidenote{\textenglish{85a/msK}} ‚{\tiny $_{lb}$}‚ नित्यः शब्द इति । अपि त्ववतिष्ठते शब्द इत्ययं प्रतिज्ञार्थस्तदा च संयोगव्यंग्यत्वा‚{\tiny $_{lb}$}‚दित्ययं हेतुरनैकान्तिको न ह्यनवस्थितं किञ्चित्संयोगेनाभिव्यज्यमान\add{ः} ‚{\tiny $_{lb}$}‚ कथमिति ‚{\tiny $_{1}$}‚ तदनेन प्रकारेण संयोगव्यङ्ग्यत्वमेव शब्दस्य प्रतिषिद्ध्यत इति नायम‚{\tiny $_{lb}$}‚सिद्धाद् व्यावर्तते । अन्यथानेयं शब्दस्य व्यक्तिः संयोगिकालमत्येतीति न संयोग‚{\tiny $_{lb}$}‚निमित्ता ‚{\tiny $_{2}$}‚ भवतीति । वचनस्य कोर्थ इति वक्तव्यं । स्याद् बुद्धिः सर्वदाधर्मिण्यविद्य‚{\tiny $_{lb}$}‚मानस्यासिद्धत्वं । अयन्तु न सर्वथा धर्मिण्यसिद्धो येनोत्पत्तिकाले संयोगव्यङ्ग्यत्वमस्ति । ‚{\tiny $_{lb}$}‚न ‚{\tiny $_{3}$}‚ तूपलब्धिकाल इति । तदुक्तं । एकदेशासिद्धस्यापि असिद्धत्त्दपरिज्ञानात् । यथा ‚{\tiny $_{lb}$}‚नित्याः परमाणवो गन्धवत्वात् । श्वेतनाश्च तरवः स्वापादिति । यश्चा ‚{\tiny $_{4}$}‚ नित्यः शब्द ‚{\tiny $_{lb}$}‚इति प्रतिजानीते स कुठारदारुसंयोगादेः शब्दस्योत्पत्तिमेव प्रतिपद्यते । न पुन‚{\tiny $_{lb}$}‚रवस्थितस्याभिव्यक्तमिति व्यक्तमस्यान्यतरासिद्ध ‚{\tiny $_{5}$}‚ त्वं । अथ संयोगे सत्युपलब्धे‚{\tiny $_{lb}$}‚रिति हेत्वर्थाभ्युपगमान् नायमसिद्धो हेतुरिति समाधीयते । तथापि तैलतेजो‚{\tiny $_{lb}$}‚वर्त्तिसंयोगे कुलालमृत्पिण्डदण्डसं ‚{\tiny $_{6}$}‚ योगे च सति दीपघटादयः समुपलभ्यन्ते । न च ‚{\tiny $_{lb}$}‚तेषान्तत्र संयोगाप्राप्त्यवस्थानमित्यनेनानैकान्तिक एव प्राप्नोतीति न कालातीतः । ‚{\tiny $_{lb}$}‚तदुत्तरकालमप्यवस्थाने साध्ये ‚{\tiny $_{7}$}‚ समुदायान्तरव्यय \add{?} वादिनो विरुद्धः । सपक्षा‚{\tiny $_{lb}$}‚भावादेव तत्र वृत्तेरभावात् । क्षणस्थितिधर्मवति च धर्मिणि । रूपादिके विद्यमान‚{\tiny $_{lb}$}‚त्वात् । स्थिरभाववादिनन्तु प्रति ‚{\tiny $_{8}$}‚ प्रतिबन्धवैकल्यं साधनवैफल्यञ्च । अनङ्गीकृत‚{\tiny $_{lb}$}‚सिद्धान्ते तु न्यायवादिनि प्रतिवादिनि पूर्वपक्षप्रतिपादितो दोष इति । एवमुदा‚{\tiny $_{lb}$}‚हरणान्त ‚{\tiny $_{9}$}‚ \leavevmode\ledsidenote{\textenglish{85b/msK}} रेपि दूषणमुत्प्रेक्ष्य वक्तव्यमिति ॥ ० ॥
	{\color{gray}{\rmlatinfont\textsuperscript{§~\theparCount}}}
	\pend% ending standard par
      ‚{\tiny $_{lb}$}‚

	  
	  \pstart \leavevmode% starting standard par
	\hphantom{.}यस्मात्प्रकरणचिन्ता स निर्ण्णयार्थमपदिष्टः प्रकरणसमः । \href{http://sarit.indology.info/?cref=ns\%C5\%AB.1.2.7}{न्या० सू० १।२।७ } ‚{\tiny $_{lb}$}‚ विमर्शाधिष्ठानौ पक्षप्रतिपक्षावनवसितौ प्रकरणन्तस्य चिन्तामविमर्शात् प्रभृति ‚{\tiny $_{lb}$}‚प्राङ्निर्ण्णयात् परीक्षणं सा यत्र कृता स निर्ण्णयार्थं प्रयुक्तोभयपक्षसाम्यात् प्रकरण ‚{\tiny $_{2}$}‚‚{\tiny $_{lb}$}‚मनतिवर्त्तमानः प्रकरणसमो न निर्ण्णयाय कल्प्यते । कस्मात्पुनः प्रकरणचिन्ता तत्त्वा‚{\tiny $_{lb}$}‚\{??\}पलब्धेः । यस्मादुपलब्धे तत्वेर्थे निवर्त्तते चिन्ता तस्मात्सामान्येनाधिग ‚{\tiny $_{3}$}‚ तस्य या ‚{\tiny $_{lb}$}‚ \leavevmode\ledsidenote{\textenglish{138/s}} विशेषतोऽनुपलब्धिः सा प्रकरणचिन्तां प्रयोजयतीति । उदाहरणमनित्यः शब्दो ‚{\tiny $_{lb}$}‚नित्यधर्मानुपलब्धेः । अनुपलभ्यमाननित्यधर्मकमनित्यन्दृ ‚{\tiny $_{4}$}‚ ष्टं स्थाल्यादि । यत्र ‚{\tiny $_{lb}$}‚समानो धर्मः संशयकारणहेतुत्वेनोपादीयते संशयसमः सव्यभिचार एव । या तु ‚{\tiny $_{lb}$}‚विमर्शस्य विशेषापेक्षतोभयपक्षविशेषानु ‚{\tiny $_{5}$}‚ पलब्धौ सा प्रकरणम्प्रवर्तयति । यथा ‚{\tiny $_{lb}$}‚च शब्दे नित्यधर्मो नोपलभ्यते तथानित्यधर्मोपि । सेयमुभयपक्षविशेषानुपलब्धिः ‚{\tiny $_{lb}$}‚प्रकरणचिन्ताम्प्रयो ‚{\tiny $_{6}$}‚ जयति कथम्विपर्यये प्रकरणनिवृत्तेः । यदि नित्यधर्मः शब्दे ‚{\tiny $_{lb}$}‚गृह्येत न स्यात्प्रकरणं । यदि \add{न} नित्यधर्मो गृह्येत एवमपि निवर्त्तते प्रकरणं । सोयं ‚{\tiny $_{lb}$}‚हेतु ‚{\tiny $_{7}$}‚ रुभौ पक्षौ प्रवर्त्तयन्नान्यतरस्य निर्ण्णयाय कल्प्यत इति । न त्वयं साध्या‚{\tiny $_{lb}$}‚विशिष्ट एव । नाविशिष्टः । तस्यैव प्रकरणप्रवृत्तिहेतोर्द्धर्मस्य हेतुत्वेनोपादानात् । ‚{\tiny $_{lb}$}‚यत्र ‚{\tiny $_{8}$}‚ साध्येन समानो धर्मो हेतुत्वेनोपादीयते स साध्याविशिष्टः । यत्र पुनः प्रकरण‚{\tiny $_{lb}$}‚प्रवृत्तिहेतुरेव स प्रकरणसम इति । अत्रापि नित्यानित्यधर्मानुपलम्भद्व ‚{\tiny $_{9}$}‚ \leavevmode\ledsidenote{\textenglish{86a/msK}} यादेव ‚{\tiny $_{lb}$}‚प्रकरणचिन्ता । न त्वेकस्मात् । विपर्यये प्रकरणनिवृत्तिरिति वचनात् । तद्यदि ‚{\tiny $_{lb}$}‚नित्यानित्यधर्मानुपलब्धेरिति हेतुः स्यात् । स्यात् प्रकरणसमः । तदेकधर्मानु ‚{\tiny $_{1}$}‚ प‚{\tiny $_{lb}$}‚लब्धेस्तूपादाने कथं प्रकरणसम इत्यभिधानीयं । उभयधर्मानुपलम्भोपादानेपि ‚{\tiny $_{lb}$}‚सपक्षविपक्षयोरनुवृत्तिव्यावृत्योरनिश्चयादसाधरणानैकान्तिको ‚{\tiny $_{2}$}‚ भवतीति कथ‚{\tiny $_{lb}$}‚मस्य हेत्वा\add{भा}सान्तरत्वं । भवतु नामैकधर्मानुपलब्धिरेव हेतुः प्रकरणसमः । ‚{\tiny $_{lb}$}‚तथापि नित्यशब्दवाद्यवश्यमेव व्यामोहान्नित्यधर्मान् प्रतिपद्यत ‚{\tiny $_{3}$}‚ इति प्रतिवाद्य‚{\tiny $_{lb}$}‚सिद्धीयं भवति । अथ प्रमाणेन नित्यधर्म्मप्रतिक्षेपान्नित्यधर्मानुपलब्धिः प्रति‚{\tiny $_{lb}$}‚पाद्यते । तदापि निश्चायकत्वात् सम्यग्ज्ञानहेतुरेवायं ‚{\tiny $_{4}$}‚ इति कथं हेत्वाभासः प्रक‚{\tiny $_{lb}$}‚रणसमः । तदा हि विशेषोपलब्धिरेव हेत्वर्थो व्यवतिष्ठते । विशेषाश्च नित्यस्य ‚{\tiny $_{lb}$}‚कृतकत्वादयः । न च तत्कृता प्रकरणचिन्ता ‚{\tiny $_{5}$}‚ विपर्यये प्रकरणनिवृत्तेरिति वचनात् । ‚{\tiny $_{lb}$}‚अपि च नित्यधर्मानुपलब्धेरिति किमयं प्रसज्यप्रतिषेधः किम्वा प्रतियोगिविधानं \add{।} ‚{\tiny $_{lb}$}‚ यदि प्रसज्यप्रतिषे ‚{\tiny $_{6}$}‚ धस्तदा प्रमेयत्वादिवत् साधारणानैकान्तिकोयं नित्यधर्मोप‚{\tiny $_{lb}$}‚लब्धिः प्रतिषेधमात्रस्यानित्यत्वरहितेष्वसत्स्वपि सम्भवात् । अथ प्रतियोगि‚{\tiny $_{lb}$}‚विधानन्तदाप्यनन्तरो ‚{\tiny $_{7}$}‚ दितया युक्त्या हेतुप्रतिरूपत्वायोगः । अन्यस्त्वन्यथेदं सूत्र‚{\tiny $_{lb}$}‚द्वयं व्याचष्टे । यो हेतुर्हेतुकालेऽपदिष्टोऽत्येत्यपैति । कस्मादपैति । प्रत्यक्षेणाग‚{\tiny $_{lb}$}‚मेन उभयेन वा ‚{\tiny $_{8}$}‚ पीड्यमानः स कालमतीत इति कालातीत इत्युच्यते । कुतः पुनः ‚{\tiny $_{lb}$}‚प्रत्यक्षागमविरोधो लभ्यत इति चेत् । चतुर्लक्षणो हेतुरिति वचनात् । तथाहि ‚{\tiny $_{lb}$}‚ पूर्ववच्छेष ‚{\tiny $_{9}$}‚ \leavevmode\ledsidenote{\textenglish{86b/msK}} वत्सामान्यतो दृष्ट \href{http://sarit.indology.info/?cref=ns\%C5\%AB.1.2.5}{न्या० सू० १।२।५ } ञ्चेत्यत्र चतूरूपो हेतु‚{\tiny $_{lb}$}‚रिष्टः । पूर्ववन्नाम साध्ये व्यापकं । शेषवदिति तत्समानेस्ति । सामान्यतश्च दृष्ट‚{\tiny $_{lb}$}‚ञ्च शब्दादविरुद्धञ्चेति । तथा भाष्यवचनमप्यस्ति । ‚{\tiny $_{1}$}‚ यत्पुनरनुमानं प्रत्यक्षा‚{\tiny $_{lb}$}‚गमविरुद्धं न्यायाभासः स इति । तदेवं त्रैरूप्ये सति प्रत्यक्षागमाभ्यां यो वाध्यते ‚{\tiny $_{lb}$}‚ \leavevmode\ledsidenote{\textenglish{139/s}} स कालात्ययापदिष्टः । स च त्रिधा भिद्यते प्रत्यक्षविरुद्ध आ ‚{\tiny $_{2}$}‚ गमविरुद्ध उभयविरुद्ध‚{\tiny $_{lb}$}‚श्चेति \add{।} प्रत्यक्षविरुद्धो यथा अनुष्णोग्निर्द्रव्यत्वादुदकवत् । आगमविरुद्धो यथा ‚{\tiny $_{lb}$}‚ब्राह्मणेन सुरा पातव्या द्रवत्वात् क्षीरवत् । उभयविरुद्धो यथाऽ ‚{\tiny $_{3}$}‚ रश्मिवच्चक्षुरिन्द्रिय‚{\tiny $_{lb}$}‚त्वाद् घ्राणादिवदिति । न चायं किल पक्षविरोधः पक्षविरोधस्य प्रतिक्षेपादिति । ‚{\tiny $_{lb}$}‚तदेतत् त्रैरूप्यलक्षणानववोधवैशद्यं \add{।} त्रैरूप्यं हि यदा ‚{\tiny $_{4}$}‚ स्वं प्रमाणैः परिनिश्चितं ‚{\tiny $_{lb}$}‚पक्षधर्मत्वादिकं त्रयं च यत्र बाधा तत्र प्रतिबन्धोस्ति । बाधाविनाभावयोर्विरो‚{\tiny $_{lb}$}‚घात् । अविनाभावो हि सत्येव साध्यधर्मे हेतोर्भावः ‚{\tiny $_{5}$}‚ कथञ्चासौ तल्लक्षणो धर्मिणि ‚{\tiny $_{lb}$}‚हेतुः स्यान्न चात्र साध्यधर्म इत्यादिकमत्राबाधितविषयत्वदूषणानुसारेण वक्तव्यं । ‚{\tiny $_{lb}$}‚यत्र पुनरियं बाधोदाहृता न ‚{\tiny $_{6}$}‚ तेषां त्रैलक्षण्यं मनागप्यस्ति प्रतिबन्धवैकल्यात् । ‚{\tiny $_{lb}$}‚अभ्युपगतपक्षप्रयोगस्य च पक्षदोष एवायं युक्तः । यत्पुनः पक्षदोषत्वपरिहाराय ‚{\tiny $_{lb}$}‚बह्वसम्बद्धमुद्ग्राहि ‚{\tiny $_{7}$}‚ तं तदत्यन्तमसारमिति नेहावसीयते ॥ ० ॥
	{\color{gray}{\rmlatinfont\textsuperscript{§~\theparCount}}}
	\pend% ending standard par
      ‚{\tiny $_{lb}$}‚

	  
	  \pstart \leavevmode% starting standard par
	यस्मात्प्रकरणचिन्तेति प्रकरणं भाष्ये निरूपितं । तस्योदाहरणं । अणुरण्व‚{\tiny $_{lb}$}‚न्तरकार्यत्वं प्रतिपद्यते ‚{\tiny $_{8}$}‚ नवेति चिन्तायाङ्कश्चिदभिधत्ते । अणुरण्वन्तरकार्यो ‚{\tiny $_{lb}$}‚रूपादिमत्वात् तद्द्व्यणुकादिवदिति । योसावणोरणुः कारणत्वेनोपादीयते तत्रापि ‚{\tiny $_{lb}$}‚रूपादिमत्वमस्तीति ‚{\tiny $_{9}$}‚ \leavevmode\ledsidenote{\textenglish{87a/msK}} चिन्ता किमियं रूपादिमत्वादण्वन्तरकार्यो न वेति चिन्ता- ‚{\tiny $_{lb}$}‚याञ्च यदि तस्याप्यण्वन्तरकार्यत्वं रूपादिमत्वादिति वक्ति तदा तस्यापि रूपादि‚{\tiny $_{lb}$}‚मत्वमस्तीति पुनरपि चिन्ता ‚{\tiny $_{1}$}‚ तदेवमनवस्थारूपं प्रकरणं प्रवर्त्तयतीति प्रकरणसम ‚{\tiny $_{lb}$}‚इत्युच्यते । अथाणुरण्वन्तरकार्यत्वं न प्रतिपद्यते रूपादिमत्वे सति तदानैकान्तिको ‚{\tiny $_{lb}$}‚हेतुरिति तस्माद् भि ‚{\tiny $_{2}$}‚ द्यतेऽनैकान्तिकात् प्रकरणसमः । न चायम्विरुद्धोऽविपर्यय‚{\tiny $_{lb}$}‚साधकत्वात् । नासिद्धः पक्षधर्मत्वदर्शनात् । न कालात्ययापदिष्टः प्रत्यक्षागमाभ्या‚{\tiny $_{lb}$}‚मबाध्यमानत्वात् । ‚{\tiny $_{3}$}‚ अतोऽर्थान्तरमिति । अथवा प्रदेशे करणं प्रकरणञ्चेति कारण‚{\tiny $_{lb}$}‚सिद्धिरित्यर्थः । प्रदेशे सिद्धिरितीयं चिन्ता यस्माद्धेतोरपदिष्टा भवति स प्रकरण‚{\tiny $_{lb}$}‚समः ‚{\tiny $_{4}$}‚ स प्रदेशसाधकत्वात् समः । यथैकदेशेऽसाधकत्वन्तथेतरत्रापीत्यसाधकत्व‚{\tiny $_{lb}$}‚सामान्यात् समः । तस्मादेकदेशवर्त्ती धर्मः प्रकरणसमः । तद्यथा पृथिव्यप्तेजोवा ‚{\tiny $_{5}$}‚ ‚{\tiny $_{lb}$}‚ य्वाकाशान्यनित्यानि सत्तावत्वादिति । अत्रापि यद्यक्ष ‚{\color{DodgerBlue3}‚पाद} मतानुसारी तावदेवं ‚{\tiny $_{lb}$}‚प्रमाणयति । परमाणुः परमाण्वन्तरपूर्व्वको रूपादिमत्वाद् द्व्यणुकादिवदि ‚{\tiny $_{6}$}‚ ति तदा ‚{\tiny $_{lb}$}‚तस्याभ्युपेतविरोध इति नायमतीतकालाद् भिद्यते । अथ ‚{\color{DodgerBlue3}‚बौद्धः} करोति । तदापि ‚{\tiny $_{lb}$}‚हेतोरसिद्धिः परमाणूनां रूपादिव्यतिरेकेणानभ्युपगमात् । अयोगाच्च ‚{\tiny $_{7}$}‚ द्व्यदीनाञ्चा‚{\tiny $_{lb}$}‚भावादुभयविकलो दृष्टान्तः । यदाप्यनपेक्षितसिद्धान्तो न्यायवादी ब्रूते तदापि ‚{\tiny $_{lb}$}‚ \leavevmode\ledsidenote{\textenglish{140/s}} द्वितीयपक्षोदितदोषानिवृत्तिरिति नायमसिद्धाद्व्यावर्त्तते । योऽ ‚{\tiny $_{8}$}‚ प्यनुमेयैकदेशवर्त्ती ‚{\tiny $_{lb}$}‚धर्मः पृथिव्यादीन्यनित्यानि गन्धवत्वादिति अयमप्यपसिद्धान्तान्तर्भूत एवेति न पृथग्वा‚{\tiny $_{lb}$}‚च्यः । नासिद्धः पक्षैकदेशधर्मत्वात् । सपक्षैकदेशवर्त्तिव ‚{\tiny $_{9}$}‚ \leavevmode\ledsidenote{\textenglish{87b/msK}} दिति चेत् विषमोयमुपन्यासः । ‚{\tiny $_{lb}$}‚सपक्ष एव च सत्वमित्यत्र हि समुच्चीयमानावधारणमेव न सकलसपक्षधर्मतां ‚{\tiny $_{lb}$}‚साधनस्य प्रतिपादयति । अनुमेये सत्ववचनं पुनरयोगव्य ‚{\tiny $_{1}$}‚ वच्छेदेन नियन्तृभूतम‚{\tiny $_{lb}$}‚शेषसाध्यधर्मिधर्मतायाः प्रतिपादकमित्यनेनैव पक्षैकदेशासिद्धभेदानामपोहः कृत ‚{\tiny $_{lb}$}‚इत्यपार्थकं यत्नान्तरमिति यत्किञ्चिदेतत् ॥ ० ॥
	{\color{gray}{\rmlatinfont\textsuperscript{§~\theparCount}}}
	\pend% ending standard par
      ‚{\tiny $_{lb}$}‚

	  
	  \pstart \leavevmode% starting standard par
	\hphantom{.}‚{\color{DodgerBlue3}‚भावि ‚{\tiny $_{2}$}‚ विक्तो} प्यत्रैव खररवे पतितः । प्रकरणसममन्यथा समर्थयति । यस्मा‚{\tiny $_{lb}$}‚द्धेतो\add{ः} प्रकरणचिन्ता विपक्षस्यापि विचारः पश्चाद् भवति स एवं लक्षणो ‚{\tiny $_{lb}$}‚हेतुनिर्णयाय योपदिश्यमानः प्र ‚{\tiny $_{3}$}‚ करणसमो भवति । प्रकरणे साध्ये समस्तुल्यः ‚{\tiny $_{lb}$}‚सत्त्वे ऽसत्त्वे वा यथा सत्सर्वज्ञमितरतद्विपरीतविनिर्मुक्तत्वाद् रूपादिवदिति । ‚{\tiny $_{lb}$}‚यस्मादयं हेतुरुभयत्र समानो योप्यस ‚{\tiny $_{4}$}‚ त्वं साधयति तस्यापि समानः । कथमसत्स‚{\tiny $_{lb}$}‚र्वज्ञत्वमितरतद्विपरीतविनिर्मुक्तत्वात् खरविषाणवदिति । न चायं किलोभयधर्म‚{\tiny $_{lb}$}‚त्वेप्यनैकान्तिको विपक्षवृत्तिर्वैकल्या ‚{\tiny $_{5}$}‚ त् । तदिदमाचार्येण स्वयं ‚{\color{DodgerBlue3}‚प्रमाणविनिश्चये} \edtext{\textsuperscript{*}}{\lemma{*}\Bfootnote{आचार्यधर्मकीर्तिप्रणीतेषु सप्तसु न्यायप्रबन्धेष्वन्यतमो ग्रथः\begin{english}\textenglish{See →} bStan-ḥ gyur, mdo.XCV. 11\end{english}}} ‚{\tiny $_{lb}$}‚ प्रतिसि\edtext{}{\lemma{प्रतिसि}\Bfootnote{? षि}}द्धं । कम्पुनरत्र भवान् विपक्षं प्रत्येति साध्याभावं । कथमिदानीं ‚{\tiny $_{lb}$}‚हेतुं विपक्षवृत्तिरुभयधर्मेणैवेत्यादिना । अर्थग्रहण ‚{\tiny $_{6}$}‚ व्याख्याने च यदुक्तं ‚{\tiny $_{lb}$}‚तदत्रापि वक्तव्यं । विशेषेण दूषणञ्चास्य प्रपञ्चेनोक्तमेव । एवं प्रकरणसमातीत‚{\tiny $_{lb}$}‚कालयोरनुपपत्तिः । साध्यव्यभिचारस्य तु युज्यते हेत्वाभासत्वं ‚{\tiny $_{7}$}‚ न तु यथा भवता‚{\tiny $_{lb}$}‚मभ्युपगमः । तथा हि भवन्तः सन्दिग्धविपक्षव्यावृत्तिकस्यानैकान्तिकत्वं न प्रति‚{\tiny $_{lb}$}‚पद्यन्ते । अदर्शनमात्रेणैव व्यतिरेकसिद्ध्यभ्युपगमात् । अत एव च भवद्भि ‚{\tiny $_{8}$}‚ रप‚{\tiny $_{lb}$}‚गतस\edtext{}{\lemma{गतस}\Bfootnote{? श}}ङ्कैरेवं प्रयुज्यते प्राप्यकारिणी चक्षुःश्रोत्रे बाह्येन्द्रियत्वात् घ्राणादिवत् । ‚{\tiny $_{lb}$}‚सविकल्पं प्रत्यक्षं प्रमाणत्वादनुमानवदित्यादि । न चादर्शनमात्रेणैव विना प्रति‚{\tiny $_{lb}$}‚बन्धे ‚{\tiny $_{9}$}‚ \leavevmode\ledsidenote{\textenglish{88a/msK}} न व्यतिरेकसिद्धिरिति प्रतानितमन्यत्र । ‚{\tiny $_{lb}$}‚ 
	    \pend% close preceding par
	  
	    
	    \stanza[\smallbreak]
	  \flagstanza{\tiny\textenglish{...45}}{\normalfontlatin\large ``\qquad}आत्ममृच्चेतनादीनां यो भावस्याप्रसाधकः ।&‚{\tiny $_{lb}$}‚स एवानुपलम्भः किं हेत्वभावस्य साधक \add{४५}{\normalfontlatin\large\qquad{}"}\&[\smallbreak]
	  
	  
	  
	    \pstart  \leavevmode% new par for following
	    \hphantom{.}
	   इत्यादिना । ‚{\tiny $_{lb}$}‚तथा सपक्षविपक्षयोः सन्दि ‚{\tiny $_{1}$}‚ ग्धः सदसत्वस्यापि सम्यग्ज्ञानहेतुत्वमेव युष्माभि‚{\tiny $_{lb}$}‚रिष्यते नानैकान्तिकत्वं यथा सात्मकं जीवच्छरीरम्प्राणादिमत्त्वादिति । अस्य च ‚{\tiny $_{lb}$}‚तद्भावः प्रतिपादितः ‚{\color{DodgerBlue3}‚प्रमाणविनिश्चया ‚{\tiny $_{2}$}‚} दौ विरुद्धप्रभेदस्तु ‚{\color{DodgerBlue3}‚भारद्वाज} विहितः प्रति‚{\tiny $_{lb}$}‚ \leavevmode\ledsidenote{\textenglish{141/s}} ज्ञाविरोधप्रस्ताव एव निराकृतः । साध्यसमेपि योयमन्यथासिद्धो वर्ण्यते यथा‚{\tiny $_{lb}$}‚ऽनित्याः परमाणवः क्रियावत्वाद् घटादिवदि ‚{\tiny $_{3}$}‚ ति अयमपि किल साध्यसमो यस्मात् ‚{\tiny $_{lb}$}‚मूर्तिक्रियारूपादिमत्वादणूनां क्रियावत्वं नानित्यत्वादिति । स नोपपद्यते । धर्मिणि ‚{\tiny $_{lb}$}‚सिद्धत्वान्नहि धर्मिणि विद्यमान एवासिद्धो ‚{\tiny $_{4}$}‚ हेतुर्युज्यते । सर्वहेतूनामसिद्धताप्रसङ्‚{\tiny $_{lb}$}‚गात् । नैतदेवन्नहि पक्षेस्तीत्येतावता पक्षधर्मत्वं । साध्यवशेन हि धर्मिणः पक्षधर्मत्व‚{\tiny $_{lb}$}‚मिष्यते । केवलस्य साध्यत्वा ‚{\tiny $_{5}$}‚ त् \add{।} न च साध्यो धर्मो यदि धर्मिणि तेन साधनेन ‚{\tiny $_{lb}$}‚विना न सम्भवति तस्य च साधनस्य साध्यधर्माभावे धर्मिणा सम्भवस्ततो हेतोः ‚{\tiny $_{lb}$}‚पक्षधर्मत्वं । यदा पुनरन्यथापि ‚{\tiny $_{6}$}‚ साधनायोपात्ते धर्मिणि धर्म उपपद्यते तदा हेतुत्त्वमे‚{\tiny $_{lb}$}‚वञ्च विशिष्टमेव सत्वं पक्षधर्मत्वेन विवक्षितं । अन्यथासिद्धत्वं युक्तमेव साध्यसमं । ‚{\tiny $_{lb}$}‚तदिदमत्र प्रतिविधानं ‚{\tiny $_{7}$}‚ यदि खलु साध्यधर्माभावे धर्मिणि असम्भवो हेतोरेवं विध‚{\tiny $_{lb}$}‚मेव सत्वं पक्षधर्मत्वेन विवक्षितं न तु भावमात्रं तदा किन्तदित्त्थंभूतं पक्षधर्मत्वम‚{\tiny $_{lb}$}‚विज्ञातमेवानुमेयप्रकाश ‚{\tiny $_{8}$}‚ कमाहोस्वित् परिनिश्चितमेवेति प्रकारद्वये यद्यविज्ञातं ‚{\tiny $_{lb}$}‚प्रकाशकं तदाज्ञापकहेतुन्यायमतिवर्त्तते \add{।} ज्ञापको हि हेतुः स्वात्मनि ज्ञानापेक्षो ‚{\tiny $_{lb}$}‚ज्ञाप्यमर्थं प्रकाशयति । सत्ता ‚{\tiny $_{9}$}‚ \leavevmode\ledsidenote{\textenglish{88b/msK}} मात्रेण च हेतवो विप्रतिपत्तिनिराकारणपटवः सन्तीति ‚{\tiny $_{lb}$}‚प्रतिवादिनां परस्परपराहतं प्रवचननानात्वं न भवेत् । विज्ञातस्यापि गमकत्वे ‚{\tiny $_{lb}$}‚प्रमाणाद्वा तस्य परिनिश्चयः प्र ‚{\tiny $_{1}$}‚ माणाद्वा । न तावदप्रमाणस्य भूतार्थनिश्चयहेतुत्वा‚{\tiny $_{lb}$}‚भावादप्रमाणाद् गतिरन्यथा प्रामाण्यमेवावहीयते । यस्मादिदमेव प्रमाणस्य प्रमा‚{\tiny $_{lb}$}‚णत्वं यद्यथावस्थितवस्तुप्रका ‚{\tiny $_{2}$}‚ शकत्वं । तच्चेदमप्रमाणस्याप्यस्ति तदा कथं तद ‚{\tiny $_{lb}$}‚प्रमाणात्साध्ये धर्मिणि विना साध्यधर्मेणासद्भूष्णोर्हेतोः सत्त्वम्पक्षधर्मत्वेनाधि‚{\tiny $_{lb}$}‚गम्यते ‚{\tiny $_{3}$}‚ तदापि यत एव प्रमाणाद्धेतोः सिद्धिस्तत एव साध्यधर्मस्यापीयं जायत इति ‚{\tiny $_{lb}$}‚किमर्थमयमकिञ्चित्करो हेतुरूपादीयते । न च हेतोरेव केवलस्य ततः सिद्धि ‚{\tiny $_{4}$}‚ \add{ः} ‚{\tiny $_{lb}$}‚साध्यधर्मस्य तु नेति युक्तम्वक्तुं । हेतोरपि ततोऽसिद्धिप्रसङ्गात् तथा ह्येवमयं ‚{\tiny $_{lb}$}‚हेतु\add{ः} तत्र धर्मिणि सिध्यति यद्यनेन साध्यधर्मेण विनेह नोपपद्यते सहैव तू ‚{\tiny $_{5}$}‚ प‚{\tiny $_{lb}$}‚पद्यत इति सिध्येत् । तथा च कथन्तत एव प्रमाणात्साध्यधर्मस्यापि न सिद्धिः ‚{\tiny $_{lb}$}‚सञ्जातेति चिन्तनीयमेतत् । एवञ्च तेनैव प्रमाणेन सहास्य साध्यधर्मस्य ‚{\tiny $_{6}$}‚ । गम्य‚{\tiny $_{lb}$}‚गमकभावो न त्वनेन हेतुनेति महदनिष्टमापद्यते । एवम्विधपक्षधर्मत्वसमाश्रयणे ‚{\tiny $_{lb}$}‚च यावत् साध्यस्यासिद्धिस्तावद्धेतोरपि यावच्च हेतोरसिद्धिस्तावत्सा ‚{\tiny $_{7}$}‚ ध्यस्यापीति ‚{\tiny $_{lb}$}‚परस्पराश्रयप्रसङ्गः । पक्षधर्मत्वनिश्चयवेलायाञ्च साध्यधर्मसिद्धिः सम्पद्यत इति ‚{\tiny $_{lb}$}‚व्यर्थमुत्तरलिङ्गरूपानुसरणमित्त्थञ्च न द्विलक्षणश्चतुर्लक्षणः पञ्च ‚{\tiny $_{8}$}‚ लक्षणश्च ‚{\tiny $_{lb}$}‚हेतुर्वक्तव्यः । अस्मन्मते तु धर्मिणि सत्वमात्रं विज्ञाय च तदुत्तरकालमन्वयव्यति‚{\tiny $_{lb}$}‚रेकयोर्विज्ञानमन्वयव्यतिरेकौ वा सर्व्वोपसंहारेण विज्ञायत उत्तरकालं ‚{\tiny $_{9}$}‚ \leavevmode\ledsidenote{\textenglish{89a/msK}} धर्मिणि ‚{\tiny $_{lb}$}‚सत्वमात्रं विज्ञातमतश्चानन्तर्येणैव तत्सामर्थ्यात्साध्यधर्मस्य तत्र प्रतीतिरुप‚{\tiny $_{lb}$}‚ \leavevmode\ledsidenote{\textenglish{142/s}} पद्यते \add{।} तेनेदमत्र सकलं दोषजालं नभसीवामले जले नावस्थानमलं लभत ‚{\tiny $_{lb}$}‚इत्यलमप्रतिष्ठित ‚{\tiny $_{1}$}‚ मिथ्याप्रलापैरिति विरम्यते ।
	{\color{gray}{\rmlatinfont\textsuperscript{§~\theparCount}}}
	\pend% ending standard par
      ‚{\tiny $_{lb}$}‚

	  
	  \pstart \leavevmode% starting standard par
	\hphantom{.}यद्येवं किं पुनरत्रेष्टमिष्टमित्याह । ‚{\color{DodgerBlue3}‚हेत्वाभासास्तु यथान्यायमित्यादि} ‚{\tiny $_{2a7}$}‚ ‚{\tiny $_{lb}$}‚ये न्याया हेत्वाभासास्तदुक्तिर्न्निग्रहस्थानम्भवति । ते च येस्माभिरुक्ताः ॥ ‚{\tiny $_{lb}$}‚ 
	    \pend% close preceding par
	  
	    
	    \stanza[\smallbreak]
	  \flagstanza{\tiny\textenglish{...46}}{\normalfontlatin\large ``\qquad}एकाप्रसिद्धिसंदेहेऽप्रसिद्धव्यभिचारभाक् ।&‚{\tiny $_{lb}$}‚द्वयोर्व्विरुद्धोसिद्धौ च संदेहव्यभिचारभागि \add{४६}{\normalfontlatin\large\qquad{}"}\&[\smallbreak]
	  
	  
	  
	    \pstart  \leavevmode% new par for following
	    \hphantom{.}
	   ति ।
	{\color{gray}{\rmlatinfont\textsuperscript{§~\theparCount}}}
	\pend% ending standard par
      ‚{\tiny $_{lb}$}‚

	  
	  \pstart \leavevmode% starting standard par
	\hphantom{.}ननु चायं वादन्यायमार्गः सकललोकानिबन्धनबन्धुना ‚{\color{DodgerBlue3}‚वादविधानादावार्य ‚{\tiny $_{3}$}‚ ‚{\tiny $_{lb}$}‚वसुबन्धुना} महाराजपथीकृतः \add{।} क्षुण्णश्च तदनु महत्यां ‚{\color{DodgerBlue3}‚न्यायपरीक्षा} यां कुमति‚{\tiny $_{lb}$}‚मतमन्त\edtext{}{\lemma{मतमन्त}\Bfootnote{? मत्त}}मातङ्गशिरःपीठपाटनपटुभिराचार्य ‚{\color{DodgerBlue3}‚दिग्नाग} पादैस्तत्किमिदं पुनश्च‚{\tiny $_{lb}$}‚र्व्वितच ‚{\tiny $_{4}$}‚ र्व्वणमास्थितं त्वयेति । एतच्चोद्यपरिहारपरमिमं श्लोकमुपन्यस्यतिपन्यस्यते ‚{\tiny $_{lb}$}‚ ‚{\color{DodgerBlue3}‚लोक} ‚{\tiny $_{20a8}$}‚ इत्यादि । तिमिरञ्च पटलञ्चेति तिमिरपटलं अविद्यैव तिमिरपटलम‚{\tiny $_{lb}$}‚विद्यातिमिर ‚{\tiny $_{5}$}‚ पटलं भूतार्थदर्शनविबन्धकत्वात् । तस्योल्लेखनो वादन्याय इति ‚{\tiny $_{lb}$}‚सम्बन्धः \add{।} उल्लेखनशब्दः कर्तृसाधनः । कस्य पुनरविद्यातिमिरपटलमित्याह । तत्त्व ‚{\tiny $_{lb}$}‚दृष्टेस्त ‚{\tiny $_{6}$}‚ त्वदर्शनस्य । प्रज्ञालोचनस्येत्यर्थः । य एष समनन्तरमावेदितो वादन्यायः । ‚{\tiny $_{lb}$}‚ ‚{\color{DodgerBlue3}‚सद्} भिः पूर्वाचार्यैः परहितरतैः करुणापारतन्त्र्याल्लोकान् सम्यग्वर्त्त्मनि व्यवस्थाप‚{\tiny $_{lb}$}‚यितुं प्र ‚{\tiny $_{7}$}‚ णीतः परां प्रसिद्धिं नीतो लोके सुष्टु स्फुटीकृत इत्यर्थः । न तु परस्पर्द्धया ‚{\tiny $_{lb}$}‚नापि यशःकामतादिभिः ।त\edtext{}{\lemma{त}\Bfootnote{? य}}द्येवन्तर्हि तदवस्थितं चोद्यमित्यत आह । ‚{\tiny $_{lb}$}‚तत्वस्यालोकमुद्योतम्वा ‚{\tiny $_{8}$}‚ दन्यायमाचाचार्यैरूपदिष्टं \add{।} तिमिरयत्यन्धकारीकरोति ‚{\tiny $_{lb}$}‚कुदूषणतमसा प्रच्छादयतीति यावत् । कः पुनरसावतिसाहसिको यो महानागैः क्षुण्णं ‚{\tiny $_{lb}$}‚पन्थानं रोद्धुमीहत इत्याह ‚{\tiny $_{8}$}‚ \leavevmode\ledsidenote{\textenglish{89b/msK}} ‚{\color{DodgerBlue3}‚दुर्विदग्धः} सम्यग्विवेकरहिततया जनोय ‚{\color{DodgerBlue3}‚मुद्योतकर‚{\tiny $_{lb}$}‚प्रीतिचन्द्र\add{?}भाविविक्त} प्रभृतिः । यतश्च एवं तस्माद्यत्नः कृत इह ‚{\color{DodgerBlue3}‚वादन्याय} प्रकरणे ‚{\tiny $_{lb}$}‚मया तस्य महद्भिरुद्भावितस्या ‚{\tiny $_{1}$}‚ न्तराजतैरवधूतस्य समुज्वालनाय । कुदूषणपरि‚{\tiny $_{lb}$}‚हारेण तन्नीत्युद्योतनेन मम व्यापृतत्वान्न मया पिष्टं पिष्टमिति संक्षेपार्थः ॥
	{\color{gray}{\rmlatinfont\textsuperscript{§~\theparCount}}}
	\pend% ending standard par
      ‚{\tiny $_{lb}$}‚\textsuperscript{\textenglish{143/s}}
	  \bigskip
	  \begingroup
	
	    
	    \stanza[\smallbreak]
	  \flagstanza{\tiny\textenglish{...47}}{\normalfontlatin\large ``\qquad}अनर्घ \add{?} वनितावगाहनमनल्पधीशक्तिना &‚{\tiny $_{lb}$}‚प्यदृष्टपरमार्थसारमधिकाभियोगैरपि ।&‚{\tiny $_{lb}$}‚मतं मतितमः स्फुटम्प्रतिविभज्य सम्यग्मया&‚{\tiny $_{lb}$}‚यदाप्तमकृशं शुभम्भवतु तेन शान्तो जनः ॥ \add{४७}{\normalfontlatin\large\qquad{}"}\&[\smallbreak]
	  
	  
	  
	  \endgroup
	‚{\tiny $_{lb}$}‚

	  
	  \pstart \leavevmode% starting standard par
	अहञ्च
	{\color{gray}{\rmlatinfont\textsuperscript{§~\theparCount}}}
	\pend% ending standard par
      ‚{\tiny $_{lb}$}‚
	  \bigskip
	  \begingroup
	
	    
	    \stanza[\smallbreak]
	  \flagstanza{\tiny\textenglish{...48}}{\normalfontlatin\large ``\qquad}नैरात्म्यबोधपरिपाटि ‚{\tiny $_{3}$}‚ तदोषशैलसम्बुद्धभारवहनक्षमभूरिशक्त-&‚{\tiny $_{lb}$}‚मञ्जुश्रियः श्रियमवाप्य समस्तसत्त्वसर्वावृतिक्षयविधानपटुर्भवेयं ।&‚{\tiny $_{lb}$}‚महारयेनैव न किञ्चिदत्र त्यक्तम्वि ‚{\tiny $_{4}$}‚ विक्तं न विभज्यमेव ।&‚{\tiny $_{lb}$}‚तथापि मन्दप्रतिबोधनार्थमालोक एष ज्वलितः प्रदीपः ॥ \add{४८}{\normalfontlatin\large\qquad{}"}\&[\smallbreak]
	  
	  
	  
	  \endgroup
	‚{\tiny $_{lb}$}‚
	  \bigskip
	  \begingroup
	
	    
	    \stanza[\smallbreak]
	  \flagstanza{\tiny\textenglish{...49}}{\normalfontlatin\large ``\qquad}लोकेऽविद्यातिमिरपटलोल्लेखनस्तत्त्वदृष्टे&‚{\tiny $_{lb}$}‚वादन्यायः परहितरतैरेष सम्य ‚{\tiny $_{5}$}‚ \add{क्} प्रणीतः ।&‚{\tiny $_{lb}$}‚तत्त्वालोकं तिमिरयति तं दुर्विंदग्धो जनोयन्&‚{\tiny $_{lb}$}‚तस्माद् यत्नः कृत इह मया तत्समुज्वालनायेति ॥ \add{४९}{\normalfontlatin\large\qquad{}"}\&[\smallbreak]
	  
	  
	  
	  \endgroup
	‚{\tiny $_{lb}$}‚

	  
	  \pstart \leavevmode% starting standard par
	विपञ्चितार्था नाम वादन्यायटीका समाप्ता ॥ ॥
	{\color{gray}{\rmlatinfont\textsuperscript{§~\theparCount}}}
	\pend% ending standard par
      ‚{\tiny $_{lb}$}‚

	  
	  \pstart \leavevmode% starting standard par
	कृतिरियंसान्तरिक्ष\edtext{}{\lemma{कृतिरियंसान्तरिक्ष}\Bfootnote{? शान्तरक्षित}}पादानामिति ॥ 
	{\color{gray}{\rmlatinfont\textsuperscript{§~\theparCount}}}
	\pend% ending standard par
      

	  
	  \pstart \leavevmode% starting standard par
	सम्वत आचू२ \begin{english}\textit{(272 N.E.–1152 A.D.)}\end{english} श्रावणकृष्ण एकादश्यांलिखितं ‚{\tiny $_{lb}$}‚मया । राजाधिराजपरमेश्वरपरमभट्टारकः श्रीमदानन्ददे ‚{\tiny $_{7}$}‚ वपादीयविजयराज्ये ‚{\tiny $_{lb}$}‚शुभदिने ॥
	{\color{gray}{\rmlatinfont\textsuperscript{§~\theparCount}}}
	\pend% ending standard par
      ‚{\tiny $_{lb}$}‚
	  \bigskip
	  \begingroup
	
	    
	    \stanza[\smallbreak]
	  \flagstanza{\tiny\textenglish{...50}}{\normalfontlatin\large ``\qquad}ग्रन्थस्यास्य प्रमाणञ्च निपुणैर्न्नवशताऽधिकं ।&‚{\tiny $_{lb}$}‚सहस्रद्वितयं सम्पत्\edtext{}{\lemma{सम्पत्}\Bfootnote{? म्यक्}}संख्यातम्पूर्व्वशूरिभिः\edtext{}{\lemma{संख्यातम्पूर्व्वशूरिभिः}\Bfootnote{? सूरिभिः}}॥ ० ॥{\normalfontlatin\large\qquad{}"}\&[\smallbreak]
	  
	  
	  
	  \endgroup
	‚{\tiny $_{lb}$}‚

	  
	  \pstart \leavevmode% starting standard par
	शुभमस्तु सर्व्वजगतां इ ‚{\tiny $_{8}$}‚ ... ‚{\tiny $_{lb}$}‚...सर्वैः रक्षितव्यम्प्रयत्नत इति ॥ ‚{\tiny $_{lb}$}‚नमः सर्वज्ञाय ॥ 
	{\color{gray}{\rmlatinfont\textsuperscript{§~\theparCount}}}
	\pend% ending standard par
      
	    
	    \endnumbering% ending numbering from div
	    
	  % running endDocumentHook
     \backmatter 
	 \chapter{The TEI Header}
	 \begin{minted}[fontfamily=rmfamily,fontsize=\footnotesize,breaklines=true]{xml}
       <teiHeader xmlns="http://www.tei-c.org/ns/1.0" xml:lang="en">
   <fileDesc>
      <titleStmt>
         <title type="main">Vādanyāyaṭīkā Vipañcitārthā</title>
         <author role="commentator">Śāntarakṣita</author>
         <funder>Deutsche Forschungsgemeinschaft</funder>
         <funder>The National Endowment for the Humanities</funder>
         <principal>
	           <persName>Birgit Kellner</persName>
	        </principal>
         <respStmt>
            <resp>data entry by</resp>
            <name key="name aurorachana">Aurorachana, Auroville</name>
         </respStmt>
         <respStmt>
            <resp>prepared for SARIT by</resp>
            <persName key="name person lo">Liudmila Olalde</persName>
         </respStmt>
      </titleStmt>
      <editionStmt>
         <p>
	</p>
      </editionStmt>
      <publicationStmt>
         <publisher>SARIT: Search and Retrieval of Indic Texts. DFG/NEH Project (NEH-No. HG5004113), 2013-2016 </publisher>
         <idno>2014-10-17</idno>
         <availability status="restricted">
            <p>Copyright Notice:</p>
            <p>Copyright 2014-2016 SARIT</p>
            <licence>
	              <p> Distributed under a <ref target="https://creativecommons.org/licenses/by-sa/4.0/">Creative Commons Attribution-ShareAlike 4.0 International licence. </ref> Under this licence, you are free to: </p>
	              <list>
                  <item>Share — copy and redistribute the material in any medium or format. </item>
                  <item>Adapt — remix, transform, and build upon the material for any purpose, even commercially. </item>
               </list>
	              <p>The licensor cannot revoke these freedoms as long as you follow the license terms. </p>
	              <p>Under the following terms:</p>
	              <list>
                  <item>Attribution — You must give appropriate credit, provide a link to the license, and indicate if changes were made. You may do so in any reasonable manner, but not in any way that suggests the licensor endorses you or your use. </item>
                  <item>ShareAlike — If you remix, transform, or build upon the material, you must distribute your contributions under the same license as the original. </item>
               </list>
	              <p>More information and fuller details of this license are given on the Creative Commons website. </p>
	           </licence>
            <p>SARIT assumes no responsibility for unauthorised use that infringes the rights of any copyright owners, known or unknown. </p>
         </availability>
         <date>2014</date>
      </publicationStmt>
      <sourceDesc>
         <bibl xml:id="vnt-sankrtyayana-book">
	           <author>Dharmakīrti</author>
	           <author>Śāntarakṣita</author>
	           <title type="main">Dharmakīrti's Vādanyāya</title>
	           <title type="sub">With the Commentary of Śāntarakṣita</title>
	           <editor key="name person rs">Rāhula Sāṅkṛtyāyana</editor>
	           <publisher>Bihar and Orissa Research Society</publisher>
	           <pubPlace>Patna</pubPlace>
	           <date>1935-1936</date>
	           <note>Appendix to the Journal of the Bihar and Orissa Research Society, vols. 21/4 and 22/1</note>
	           <note>The manuscript consulted by Sāṅkṛtyāyana is described below. </note>
	        </bibl>
         <msDesc>
            <msIdentifier xml:id="msK">
               <idno/>
               <altIdentifier>
                  <idno>Kun-de-ling-Manuscript.</idno>
                  <!-- is there a standard identifier? --></altIdentifier>
            </msIdentifier>
            <msContents>
               <msItem>
                  <author>Śāntarakṣita</author>
                  <title>Vipañcitārthā</title>
               </msItem>
            </msContents>
            <physDesc>
               <objectDesc>
                  <p>Palm-leaf manuscript. 89 leaves in Kuṭilā script. Apparently written in 1152 A.C. </p>
               </objectDesc>
            </physDesc>
            <history>
               <p>In June 1934, Sāṅkṛtyāyana found this manuscript in the monastery of Kun-de-ling (Lhasa). </p>
            </history>
         </msDesc>
         <msDesc xml:id="vn-msN">
            <msIdentifier>
               <idno/>
               <altIdentifier>
                  <idno>Nagor-Manuscript.</idno>
                  <!-- is there a standard identifier? --></altIdentifier>
            </msIdentifier>
            <msContents>
               <msItem>
                  <author>Dharmakīrti</author>
                  <title>Vādanyāya</title>
               </msItem>
            </msContents>
            <physDesc>
               <objectDesc>
                  <p>Palm-leaf manuscript. 20 leaves in Kuṭilā script. Each page contains 9 to 11 lines. 12th century CE. </p>
               </objectDesc>
            </physDesc>
            <history>
               <p>In June 1934, Sāṅkṛtyāyana found this manuscript in the monastery of Nagor, Tibet. </p>
            </history>
         </msDesc>
      </sourceDesc>
   </fileDesc>
   <encodingDesc>
      <p>Line brakes, page breaks and folio numbers: <list>
            <item>The line breaks and page breaks of <ref sameAs="#vadanyayatika-book">Sāṅkṛtyāyana's edition</ref> were given the ed-attribute "s". In the source file, there were two types of line breaks: returns (and possible surrounding space) and hyphens+returns. These were replaced with lb-elements. I didn't check whether the source was consequent in this respect.</item>
            <item>The folio numbers of the Vipañcitārthā manuscript were encoded as pb-elements with the attribute ed="msK", wich refers to the <ref target="#msK">manuscript</ref> used by Sāṅkṛtyāyana. The line numbers of the manuscript were encoded as lb-elements with the attribute ed="msK".</item>
            <item>After word(s) quoted from the Vādanyāya, Sāṅkṛtyāyana added the corresponding folio and line numbers in the <ref target="#msN">Vādanyāya manuscript</ref>. These were encoded as follows: &lt;ref target="#msN" corresp="1b1"/&gt;.</item>
         </list>
      </p>
      <p>Words quoted from the Vādanyāya (pratīkā) (printed in a thiner script in <ref sameAs="#vadanyayatika-book">Sāṅkṛtyāyana's edition</ref>) were encoded as &lt;q type="lemma"&gt;.</p>
      <p>Quotations from other texts (printed in a thiner script in <ref sameAs="#vadanyayatika-book">Sāṅkṛtyāyana's edition</ref>) were encoded as quote-elements. Some quotations are also enclosed in quotation marks; in these cases the attribute "quote" has been added.</p>
      <p>Footnotes were encoded as &lt;note rend="footnote"&gt;</p>
      <p>Round and square brackets were were replaced with the following TEI-elements:
      <list>
            <item>Bracketed references to other works were enclosed in &lt;ref cRef=""&gt;.</item>
            <item>Bracketed text was enclosed in &lt;add rend="brackets" resp="#rs"&gt;.</item>
            <item>Bracketed text preceded or followed by a question mark was enclosed in &lt;note type="correction" resp="#rs"&gt;. The question mark was kept.</item>
            <item>Bracketed suspension points ("...") were enclosed in &lt;add type="gap" resp="#rs"&gt;.</item>
            <item>Question marks surrounded by brackets were enclosed in: &lt;add rend="brackets" resp="#rs"&gt;.</item>
            <item>All other punctuation marks that were surrounded by brackets were enclosed in &lt;add resp="#rs" type="punctuation"&gt;.</item>
         </list>
      </p>
      <p>The text is structured in 2 chapters, encoded as: div n=".." type="chapter"<list>
            <item>1. Nigrahasthānalakṣaṇa, pp. 1-73</item>
            <item>2. Nyāyamatakhaṃḍana, pp. 75-143</item>
         </list>
      </p>
      <p>Abbreviations used in the attributes ed, cRef and xml:id's in this file: <!-- this is a provisory list and has to be replaced by a refsDecl -->
      <list ana="abbreviations">
            <item>ak = Abhidharmakośa</item>
            <item>MaBhā = Mahābhāṣya </item>
            <item>MīSū = Mīmāṃsāsūtra </item>
            <item>msK = <ref target="#msK">Kun-de-ling-Manuscript of the Vipañcitārthā</ref>
            </item>
            <item>msN = <ref target="#msN">Nagor-Manuscript of the Vādanyāya</ref>
            </item>
            <item>nb = Nyāyabindu</item>
            <item>nbh = Nyāyabhāṣya </item>
            <item>nv = Nyāyavarttika </item>
            <item>nsū = Nyāyasūtra</item>
            <item>Pā = Pāṇini</item>
            <item>pv = Pramāṇavartika </item>
            <item>s = <ref target="#vadanyayatika-book">Sāṅkṛtyāyana's edition of the Vipāñcitārtha</ref>
            </item>
         </list>
      </p>
   </encodingDesc>
   <profileDesc><!-- ... --></profileDesc>
   <revisionDesc>
      <change who="lo" when="2014-10-29">
	        <list>
            <item>I corrected folio number 46b to 49b on p. 73.</item>
            <item>I added folio number 53b, which was missing in the printed edition.</item>
         </list>
      </change>
      <change who="lo" when="2015-12-30">Added @xml-lang to the front-element. </change>
      <change who="lo" when="2016-04-25">Added @type to note- and add-elements.</change>
      <change who="lo" when="2016-05-19">Removed front.</change>
      <change who="lo" when="2016-05-24">Added content to ref-elements, e.g.: <tag>ref target="#msN" corresp="1b1"</tag>1b1<tag type="end">ref</tag>
      </change>
   </revisionDesc>
</teiHeader>
	 \end{minted}
       
      \clearpage
      \begin{english}
      \printshorthands
      \printbibliography
      \end{english}
    
\end{document}
